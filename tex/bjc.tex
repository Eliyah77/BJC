%& -output-directory="../pdf"
% type document & taille police
\documentclass[11pt]{book}
% package format document
\usepackage[paperwidth=6.5in, paperheight=9.05in, top=0in, bottom=0in, left=0in, right=0in]{geometry}
% formatage marges, etc.
\setlength{\voffset}{-0.7in} % offset haut
%\setlength{\hoffset}{-0.3in} % offset gauche
\setlength{\topmargin}{0in} % marge en tête
\setlength{\headsep}{0.2in} % marge header/body
\setlength{\oddsidemargin}{-0.5in} % marge texte gauche
\setlength{\evensidemargin}{-0.5in} % marge texte droite
\setlength{\textheight}{8in} % hauteur du texte
\setlength{\textwidth}{5.5in} % largeur du texte
\setlength{\columnseprule}{0.4pt} % épaisseur séparateur colonne
\setlength{\parskip}{0pt} % espace entre paragraphes
% package pour afficher les cadres
%\usepackage{showframe}
% package langue
\usepackage[francais]{babel}
% package polices système
\usepackage{fontspec}
% définition police
\setmainfont[Ligatures=TeX,Scale=0.95]{Liberation Serif}
\setsansfont{Liberation Sans}
\setmonofont{Liberation Mono}
% package titlesec
\usepackage{titlesec}
% package multicolonne
\usepackage{multicol}
% package utiles à la concordance
\usepackage{enumitem}
\usepackage{calc}
% package liens cliquables
\usepackage[xetex]{hyperref}
% package inclusion copyright (dépandant de hyperref)
\usepackage{hyperxmp}
% copyright
\hypersetup{
    pdfauthor = {ANJC Productions},
    pdftitle = {Bible de Jésus-Christ},
    pdfkeywords = {BJC, Bible, Jesus},
    pdfcopyright = {ANJC Productions. Distribution et Diffusion Libre - Pas d'Utilisation Commerciale - Pas de Dénaturation de l'Œuvre - International},
    pdflicenseurl = {http://www.bibledejesuschrist.org/}
}
% ???
\setcounter{collectmore}{-1}
% style
\pagestyle{myheadings}
% ???
\sloppy\hyphenpenalty=2000
% titres de livres
\newcommand{\ShortTitle}[1]{\def\webbook{#1}\par\goodbreak\bigskip\setcounter{footnote}{0}}
\newcommand{\BookTitle}[1]{\par\goodbreak\bigskip{\parindent=0mm\begin{center}{\small\bfseries{\LARGE #1\nopagebreak}}\end{center}}\addcontentsline{toc}{subsection}{#1}\nopagebreak\par\nobreak}
% chapitres
\newcommand{\Chap}[1]{\def\webchap{#1:}\def\webvs{1}\def\vchap{#1}\ssubsection{\centerline{\textbf{{CHAPITRE\ #1}}}}}
% versets
\newcommand{\VerseOne}{\def\webvs{1}{\up{\footnotesize 1}}\markboth{\webbook\ \webchap 1}{\webbook\ \webchap 1}}
\newcommand{\VS}[1]{\def\webvs{#1}{\up{\footnotesize #1}}\markboth{\webbook\ \webchap #1}{\webbook\ \webchap #1}}
\newcommand{\vref}[1]{\NoAutoSpaceBeforeFDP{#1}}
% commentaires
%\interfootnotelinepenalty=10000 % longueur max commentaires
\renewcommand{\thefootnote}{\alph{footnote}} % repères alphabetiques
\renewcommand{\footnoterule}{\hrule width \textwidth} % longueur ligne
\newcommand{\FTNT}[1]{\ifnum\value{footnote}>25\setcounter{footnote}{0}\fi\footnote{[\NoAutoSpaceBeforeFDP{\webchap\webvs}]\ #1}}
% commentaire sur les titres
\newcounter{webvst}
\newcommand{\FTNTT}[1]{
    % intialisation de l'indice de note
    \ifnum \value{footnote}>25 \setcounter{footnote}{0} \fi
    % initialisation de la référence du numéro de verset
    \setcounter{webvst}{\webvs}
    % si le titre est sur le premier verset, incrémenter de 1
    \ifnum \value{webvst}>1 \addtocounter{webvst}{1} \fi
    % écriture note
    \footnote{[\NoAutoSpaceBeforeFDP{\webchap\thewebvst}]\ #1}
}
% titres de paragraphes
\titlespacing*{\subsection}{0pt}{5pt plus 0pt minus 0pt}{5pt plus 0pt minus 0pt}
\titlespacing*{\subsubsection}{0pt}{5pt plus 0pt minus 0pt}{5pt plus 0pt minus 0pt}
\newcommand{\ssubsection}[1]{\subsection*{\centering\footnotesize\normalfont #1}\PP}
\newcommand{\ssubsubsection}[1]{\subsubsection*{\centering\footnotesize\normalfont #1}\PP}
\newcommand{\TextTitle}[1]{\ssubsubsection{[\textit{#1}]}}
\newcommand{\TextDial}[1]{{\scriptsize[\textit{#1}]}}
% dictionnaire
\newcommand{\DicoEntry}[1]{\smallskip\parindent=0mm{\textbf{#1}}\markboth{#1}{#1}}
% concordance
\newcommand{\ConcordanceEntry}[1]{\smallskip\parindent=0mm{\textbf{#1}\newline}\markboth{#1}{#1}}
\newlist{listverse}{description}{1}
\setlist[listverse]{nosep, noitemsep, topsep=0pt,leftmargin=!,labelwidth=\widthof{1 Ch 33:10},font=\normalfont}
\newlist{legend}{itemize}{1}
\setlist[legend]{label=.,font=\small,nosep, noitemsep, topsep=0pt, leftmargin=*,align=left}
% commandes diverses
\newcommand{\BFont}{\normalfont\small}
\newcommand{\PP}{\par\parindent=0mm}
\newcommand{\PPE}{\par\parindent=4mm}
% debut document
\begin{document}
% en-tête vide
\makeatletter
\def\@evenhead{}
\def\@oddhead{}
\makeatother
% inclusion intro
\clearpage\begin{center}{\LARGE Introduction}\end{center}
\begin{small}
\subsection*{Pourquoi cette Bible révisée~?}

En novembre 2013, alors que j'étais en prière, je demandais au Seigneur ce qu'il attendait de moi. Ce dernier m'a répondu à travers plusieurs songes dans lesquels il me disait de réviser la Bible. Je dois dire que j'ai eu du mal à croire que Dieu puisse me demander une telle chose. De plus, je me sentais incapable d'assumer un si grand projet, aussi je lui ai demandé à plusieurs reprises de me confirmer que c'était bien sa volonté, chose qu'il a faite. J'ai ensuite parlé de ce que j'avais reçu à des frères et sœurs qui travaillent avec moi et ces derniers m'ont confirmé que cette vision venait bien du Seigneur. Une dynamique s'est créée aussitôt et bien qu'aucun d'entre nous ne se sentît à la hauteur de la tâche qui nous était confiée, nous nous sommes rapidement organisés pour concrétiser cette vision, comptant sur le Seigneur pour qu'il nous donne les capacités et la sagesse dont nous avions besoin.\bigskip

Deux constats majeurs nous ont amenés à la conclusion qu'une révision de la Bible était plus que nécessaire. Tout d'abord, la plupart des bibles modernes les plus diffusées sont basées sur le texte minoritaire comportant une quantité importante de fautes de traduction, d'omissions et de rajouts qui altèrent la compréhension du message et induisent par conséquent le lecteur en erreur. Or il est du devoir de tout chrétien de mettre en pratique la Parole, notamment en veillant sur son authenticité.\bigskip

«~\emph{Car, je vous le dis en vérité, tant que le ciel et la terre ne passeront point, il ne disparaîtra pas de la loi un seul iota ou un seul trait de lettre jusqu'à ce que tout soit arrivé. Celui donc qui aura violé l'un de ces petits commandements, et qui aura enseigné les hommes à faire de même, sera appelé le plus petit au Royaume des cieux~; mais celui qui les observera, et qui enseignera à les observer, celui-là sera appelé grand au Royaume des cieux.}~» Matthieu 5:18-19.\bigskip

«~\emph{Je le déclare à quiconque entend les paroles de la prophétie de ce livre~: Si quelqu'un y ajoute quelque chose, Dieu le frappera des fléaux décrits dans ce livre. Et si quelqu'un retranche quelque chose des paroles du livre de cette prophétie, Dieu retranchera sa part de l'arbre de vie, et de la ville sainte, décrits dans ce livre.}~» Apocalypse 22:18-19.\bigskip

Nous ne devons pas oublier que la Bible a été initialement écrite en trois langues, à savoir l'hébreu, le grec et quelques versets en araméen. En réalisant cette révision, notre but est de restituer le sens des mots d'origine et d'expurger toute l'influence de l'ennemi. Ce travail a permis de mettre en lumière une évidence~: la personne de Jésus-Christ occupe une place centrale de Genèse à Apocalypse, ce qui ne fait que confirmer et attester sa divinité.\bigskip

«~\emph{Puis il leur dit~: C'est là ce que je disais lorsque j'étais encore avec vous, qu'il fallait que s'accomplisse tout ce qui est écrit de moi dans la loi de Moïse, dans les prophètes, et dans les psaumes.}~» Luc 24:44.\bigskip

Ensuite, nous déplorons le fait que la majorité des bibles en circulation soient vendues alors que Jésus-Christ a dit «~\emph{Vous l'avez reçu gratuitement, donnez-le gratuitement}~» (Mt. 10:8). Il est donc impensable que celui qui a chassé du temple vendeurs et changeurs puisse approuver un seul instant le commerce qui est fait avec sa Parole (Jn. 2:14-16).\bigskip

«~\emph{Vous tous qui avez soif, venez aux eaux, et vous qui n'avez pas d'argent, venez, achetez et mangez~; venez, dis-je, achetez du vin et du lait sans argent et sans rien payer~!}~» Esaïe 55:1.\bigskip

«~\emph{Il me dit aussi~: Tout est accompli. Je suis l'Alpha et l'Oméga, le commencement et la fin. A celui qui a soif, je lui donnerai de la source d'eau vive, gratuitement.}~» Apocalypse 21:6.\bigskip

«~\emph{Et l'Esprit et l'épouse disent~: Viens. Et que celui qui entend dise~: Viens. Et que celui qui a soif vienne~; que celui qui veut prenne gratuitement de l'eau de la vie.}~» Apocalypse 22:17.\bigskip

Les apôtres ont scrupuleusement respecté l'ordre du Seigneur en Matthieu 10:8. Pierre a dénoncé avec la plus grande sévérité Simon, le magicien qui avait eu la folie de croire que le don de Dieu pouvait être monnayé. Et durant tout son service, Paul a enseigné l'Evangile gratuitement.\bigskip

«~\emph{Puis ils leur imposèrent les mains, et ils reçurent le Saint-Esprit. Lorsque Simon vit que le Saint-Esprit était donné par l’imposition des mains des apôtres, il leur présenta de l’argent, en leur disant : Donnez-moi aussi ce pouvoir, afin que tous ceux à qui j’imposerai les mains reçoivent le Saint-Esprit. Mais Pierre lui dit~: Que ton argent périsse avec toi, puisque tu as estimé que le don de Dieu s’acquérait avec de l’argent. Tu n’as point de part ni d’héritage en cette affaire ; car ton coeur n’est point droit devant Dieu. Repens-toi donc de cette méchanceté, et prie Dieu, afin que, s’il est possible, la pensée de ton coeur te soit pardonnée. Car je vois que tu es dans un fiel très amer et dans un lien d’iniquité.}~» Actes 8:17-23.\bigskip

«~\emph{Je n'ai désiré ni l'argent, ni l'or, ni les vêtements de personne.}~» Actes 20:33.\bigskip

«~\emph{Quelle récompense en ai-je donc~? C’est qu’en prêchant l’Evangile, je prêche l’Evangile de Christ sans qu’il en coûte rien, afin que je n’abuse pas de mon pouvoir dans l’Evangile.}~» 1 Corinthiens 9:18.\bigskip

Nous pensons qu'il est juste et honnête que la Bible porte le nom de son véritable auteur et qu'elle soit gratuitement diffusée selon sa volonté et l'ordre clair qu'il a donné. Cette Bible s'appelle donc La Bible de Jésus-Christ et est gratuitement mise à la disposition de ceux qui souhaitent se la procurer.\bigskip

\subsection*{Comment a été réalisée cette révision~?}

Pour réaliser cette révision, nous nous sommes appuyés sur le texte majoritaire (originaux et traductions). Ainsi, tout en essayant de conserver un vocabulaire qui soit à la portée de tous, certains mots et expressions ont été changés pour restituer pleinement leur signification initiale. A titre d'exemple, vous constaterez régulièrement que certains mots sont répétés deux fois de suite. Cela n'est pas une erreur mais la restitution littérale de certaines expressions qui insistent sur une vérité (voir commentaire en Gn. 2:15-17). En effet, Dieu parle une fois et une seconde fois pour avertir les hommes (Job. 33:14). «~Et quant à ce que le songe a été réitéré à Pharaon pour la seconde fois, c'est que la chose est arrêtée de la part de Dieu, et que Dieu se hâtera de l'exécuter~» (Genèse 41:32).\bigskip

Les Ecrits ont été classés dans l'ordre de la tradition juive pour le Tanakh et dans l'ordre chronologique de leur rédaction pour les épîtres afin de permettre au lecteur de mieux comprendre le contexte et le déroulement de la prophétie biblique. L'appellation «~Ancien Testament~» a été remplacée par l'acronyme hébreu Tanakh (voir sommaire). Quant à ce qu'on appelle communément le «~Nouveau Testament~», il sera désormais question du Testament de Jésus. En effet, l'Ancienne Alliance n'étant pas un testament, on ne peut donc pas parler de «~Nouveau Testament~» mais plutôt d'une Nouvelle Alliance (voir commentaires en Ex. 19:5~; Mt. 27:51~; Jn. 19:30).\bigskip

Je remercie tout d'abord le Seigneur pour son aide précieuse qu'il m'a apportée pour la révision de cette Bible, ainsi qu'à celles et ceux qui m’ont assisté dans ce travail.\newline

\begin{flushright}
Shora KUETU
\end{flushright}
\end{small}

% formatage sommaire
\makeatletter
\renewcommand\tableofcontents{
    \begin{center}{\LARGE Sommaire}\end{center}
    \setlength{\columnseprule}{0pt} % désactivation séparateur colonne temp
    \begin{multicols}{2}\medskip\footnotesize{\@starttoc{toc}}\end{multicols}
    \setlength{\columnseprule}{0.4pt} % réactivation séparateur colonne
}
\makeatother
% inclusion table des matières
\clearpage\pagenumbering{arabic}\tableofcontents\clearpage
% en-tête pages
\makeatletter
\def\@evenhead{{\NoAutoSpaceBeforeFDP{\small{\rightmark\hfil\thepage\hfil\leftmark}}}}
\def\@oddhead{{\NoAutoSpaceBeforeFDP{\small{\rightmark\hfil\thepage\hfil\leftmark}}}}
\makeatother
% inclusion des livres
\addcontentsline{toc}{chapter}{Tanakh}\clearpage
\addcontentsline{toc}{section}{Torah (Loi)}\clearpage
\clearpage\makeatletter\def\@evenhead{}\def\@oddhead{}\makeatother

\vspace*{\fill}
\begin{center}
{\Huge Tanakh : Torah}
\end{center}
\vspace*{\fill}

\clearpage

\makeatletter\def\@evenhead{{\NoAutoSpaceBeforeFDP{\small{\rightmark\hfil\thepage\hfil\leftmark}}}}\def\@oddhead{{\NoAutoSpaceBeforeFDP{\small{\rightmark\hfil\thepage\hfil\leftmark}}}}\makeatother

\clearpage\ShortTitle{Genèse}\BookTitle{Genèse}\BFont
\noindent\hrulefill
{\footnotesize
\textit{
\bigskip
{\centering{}
\\Auteur : Probablement Moïse
\\(Heb. : Bereshit)
\\Signification : Au commencement
\\Thème : Le Messie d'Israël
\\Date de rédaction : Env. 1450-1410 av. J.-C.\\}
}
%\bigskip
\textit{
\\Premier livre du Tanakh, la Genèse est le livre des commencements.
Elle relate l’histoire des origines de l’humanité, la création des cieux, de la terre et de tout ce qui s’y trouve par Yahweh, le Dieu créateur.
%\bigskip
\\Il y est décrit le péché de l’homme et sa séparation d’avec Dieu, ainsi que la décadence de l’univers qui en résulta. En réponse à la méchanceté du cœur de l’homme, Yahweh  exerça sa justice en détruisant la terre par le déluge.
Dans sa prescience, Yahweh avait cependant résolu de se réconcilier avec l’homme. Il se révéla donc comme sauveur en accordant sa grâce à Noé et à sa famille. Après cet événement, les hommes se tournèrent une fois de plus vers le mal en tentant Dieu par la construction de la tour de Babel, œuvre à l’origine de la dispersion des nations.
%\bigskip
\\Ce livre présente aussi l’élection d’Abraham, originaire d’Ur en Chaldée - actuelle Mésopotamie - qui reçut la promesse divine de devenir une grande nation, en qui toutes les familles de la terre seraient bénies. Le récit se poursuit par l’histoire de ses descendants Isaac, Jacob et ses douze fils,  qui formèrent par la suite la nation d’Israël.\bigskip
}
}
\par\nobreak\noindent\hrulefill
\begin{multicols}{2}
\Chap{1}
\VerseOne{}Au commencement, Dieu créa les cieux et la terre.
\TextTitle{La terre devient informe et vide}
\VS{2}Et la terre devint informe et vide\FTNT{Les termes «~informe~» et «~vide~» viennent des mots hébreux «~tohuw~» et «~bohuw~»  qui désignent la confusion, le chaos, la vanité.}, les ténèbres étaient à la surface de l'abîme ; et l'Esprit de Dieu se mouvait au-dessus des eaux.
\TextTitle{Jour «~un~» : Apparition de la lumière}
\VS{3}Dieu dit : Que la lumière apparaisse\FTNT{«~Que la lumière apparaisse !~» (Es. 9:1 ; Mt. 4:16 ; Jn. 1:1-5). Cette lumière n’est autre que Yahweh lui-même qui va s’incarner en la personne de Jésus-Christ pour chasser les ténèbres (2 S. 22:9-12 ; Es. 60:1 ; 60:19-20 ; Jn. 1:1-2 ; 8:12-14 ; 2 Co. 4:6).} ; et la lumière apparut.
\VS{4}Et Dieu vit que la lumière était bonne ; et Dieu sépara la lumière des ténèbres.
\VS{5}Dieu appela la lumière jour, et il appela les ténèbres nuit\FTNT{La lumière et les ténèbres, ainsi que leurs champs lexicaux respectifs, personnifient  souvent Jésus et Satan.  Ainsi, Jésus est la Lumière du monde (Jn. 9:5), l’Etoile brillante du matin (Ap. 22:16), le Soleil levant ou le Soleil de la justice (Mal. 4:2 ; Ps. 19:6 ; Lu. 1:78). Il est associé au jour (Jn. 9:4) d’où les expressions  «~Jour du Seigneur~» (1 Th. 5:2) ou «~Jour de Yahweh~» (Joë. 1:15). A l’inverse, la Bible associe Satan aux ténèbres (Es. 8:23 ; Ps. 143:3 ; Ep. 6:12 ; Col. 1:13) et à la nuit (Jn. 9:4 ; Ro. 13:12).}. Ainsi fut le soir, ainsi fut le matin\FTNT{Contrairement au calendrier grégorien où le jour commence à minuit, selon Dieu et le calendrier hébraïque, le jour commence le soir à 18 heures pour se terminer le lendemain à la même heure. Voir commentaire en Mc. 16:9.} ; ce fut le  jour un\FTNT{L’hébreu utilise le terme «~ehad~» qui signifie «~un~», au sens de l’indivisible, pour qualifier le premier jour. Ce jour nous parle de Yahweh tel qu’il s’est présenté à son peuple sur le mont Sinaï en De. 6:4:«~Shema Yisrael Yahweh elohénou Yahweh ehad~» («~écoute Israël, Yahweh [est] notre Dieu, Yahweh [est] UN~»). Un n’est pas divisible sinon on obtient un zéro ce qui équivaut au néant. Dieu est tout sauf le néant, il remplit tout (Ep. 1:23), il est partout (Ps. 139:7-13), les cieux des cieux ne peuvent le contenir (1 R. 8:27).}.
\TextTitle{Second jour : Une étendue entre les eaux}
\VS{6}Puis Dieu dit : Qu'il y ait une étendue entre les eaux, et qu'elle sépare les eaux d'avec les eaux.
\VS{7}Dieu donc fit l'étendue, et il sépara les eaux qui sont au-dessous de l'étendue d'avec celles qui sont au-dessus de l'étendue, et il fut ainsi.
\VS{8}Et Dieu appela l'étendue cieux. Ainsi fut le soir, ainsi fut le matin ; ce fut le second jour.
\TextTitle{Troisième jour : Les mers, la terre et la végétation}
\VS{9}Puis Dieu dit : Que les eaux qui sont au-dessous des cieux soient rassemblées en un lieu, et que le sec paraisse ; et il fut ainsi.
\VS{10}Et Dieu appela le sec terre ; et il appela l'amas des eaux mers ; et Dieu vit que cela était bon.
\VS{11}Puis Dieu dit : Que la terre produise de la verdure, de l'herbe portant de la semence, et des arbres fruitiers portant du fruit selon leur espèce, qui aient leur semence en eux-mêmes sur la terre ; et il fut ainsi.
\VS{12}La terre donc produisit de la verdure, de l'herbe portant de la semence selon son espèce ; et des arbres portant du fruit qui avaient leur semence en eux-mêmes selon leur espèce ; et Dieu vit que cela était bon.
\VS{13}Ainsi fut le soir, ainsi fut le matin ; ce fut le troisième jour.
\TextTitle{Quatrième jour : Les luminaires du ciel}
\VS{14}Puis Dieu dit : Qu'il y ait des luminaires dans l'étendue du ciel pour séparer la nuit d'avec le jour, et qui servent de signes pour les saisons, pour les jours, et pour les années ;
\VS{15}et qu’ils servent de luminaires dans l'étendue du ciel afin d'éclairer la terre ; et il fut ainsi.
\VS{16}Dieu donc fit deux grands luminaires, le plus grand luminaire pour présider au jour, et le plus petit luminaire pour présider à la nuit ; il fit aussi les étoiles.
\VS{17}Dieu les plaça dans l'étendue du ciel pour éclairer la terre,
\VS{18}pour présider au jour et à la nuit, et pour séparer la lumière d’avec les ténèbres ; et Dieu vit que cela était bon.
\VS{19}Ainsi fut le soir, ainsi fut le matin ; ce fut le quatrième jour.
\TextTitle{Cinquième jour : Les animaux vivant dans les eaux et les airs\FTNTT{ Ge. 2:19}}
\VS{20}Puis Dieu dit : Que les eaux produisent en toute abondance des reptiles vivants ; et qu'il y ait des oiseaux qui volent sur la terre vers l'étendue du ciel.
\VS{21}Dieu créa les grands poissons et tous les animaux vivants qui se meuvent et que les eaux produisirent en toute abondance selon leur espèce ; il créa aussi tout oiseau ayant des ailes selon son espèce ; et Dieu vit que cela était bon.
\VS{22}Dieu les bénit en disant : Soyez féconds, multipliez, et remplissez les eaux des mers ; et que les oiseaux multiplient sur la terre.
\VS{23}Ainsi fut le soir, ainsi fut le matin ; ce fut le cinquième jour.
\TextTitle{Sixième jour : Les animaux terrestres}
\VS{24}Puis Dieu dit : Que la terre produise des animaux selon leur espèce, le bétail, les reptiles, et les bêtes de la terre selon leur espèce ; et il fut ainsi.
\VS{25}Dieu donc fit les animaux de la terre selon leur espèce, et le bétail selon son espèce, et les reptiles de la terre selon leur espèce ; et Dieu vit que cela était bon.
\TextTitle{Mission confiée à l’homme ; son autorité sur la création}
\VS{26}Puis Dieu dit : Faisons l'homme à notre image, selon notre ressemblance\FTNT{L’image de Dieu n’est autre que Jésus-Christ lui-même (Col. 1:15). Adam, qui signifie terrien, a été créé à  l’image du dernier Adam (1 Co. 15:40-49) qui est venu comme Fils, afin de nous montrer le modèle de fils et de filles que Dieu souhaite (Ro. 8:29). Nous avons ici une autre image de l’incarnation de Dieu en la personne de Jésus-Christ. Ainsi, avant que l’homme ne pèche, le projet de la rédemption était déjà là (1 Pi. 1:19-21).}, et qu'il domine sur les poissons de la mer, sur les oiseaux du ciel, sur le bétail, sur toute la terre, et sur tout reptile qui rampe sur la terre.
\VS{27}Dieu créa l'homme à son image, il le créa à l'image de Dieu, il les créa mâle et femelle.
\TextTitle{Autorité de l'homme sur la création}
\VS{28}Dieu les bénit et leur dit : Soyez féconds, multipliez, remplissez la terre, et assujettissez-la ; et dominez sur les poissons de la mer, sur les oiseaux du ciel, et sur toute bête qui se meut sur la terre.
\VS{29}Et Dieu dit : Voici, je vous donne toute herbe portant de la semence qui est sur toute la terre, et tout arbre ayant en lui du fruit d'arbre et portant de la semence, ce sera votre nourriture.
\VS{30}Et à tout animal de la terre, à tout oiseau du ciel, et à tout ce qui se meut sur la terre, ayant en soi un souffle de vie, je donne toute herbe verte pour nourriture. Et cela fut ainsi.
\VS{31}Dieu vit tout ce qu'il avait fait, et voici cela était très bon. Ainsi fut le soir, ainsi fut le matin ; ce fut le sixième jour.
\Chap{2}
\TextTitle{Septième jour : Le sabbat}
\VerseOne{}Les cieux donc et la terre furent achevés, avec toute leur armée.
\VS{2}Dieu acheva au septième jour son œuvre qu'il avait faite, et il se reposa au septième jour de toute son œuvre qu'il avait faite.
\VS{3}Dieu bénit le septième jour, et le sanctifia, parce qu'en ce jour-là il s'était reposé de toute son œuvre qu'il avait créée en la faisant.
\VS{4}Telles sont les origines des cieux et de la terre, lorsqu'ils furent créés.
\VS{5}Lorsque Yahweh Dieu fit la terre et les cieux, aucun arbuste des champs n’était encore sur la terre, et aucune herbe des champs ne germait encore ; car Yahweh Dieu n'avait pas fait pleuvoir sur la terre, et il n'y avait point d'homme pour cultiver la terre.
\TextTitle{Yahweh forme l’homme et le place en Eden\FTNTT{Job 10:8-9 ; Ps. 119:73}}
\VS{6}Et il monta une vapeur de la terre qui arrosa toute la surface de la terre.
\VS{7}Yahweh Dieu forma l'homme de la poussière de la terre, et il souffla dans ses narines un souffle de vie ; et l'homme devint une âme vivante.
\VS{8}Aussi Yahweh Dieu planta un jardin en Eden, du côté de l’orient, et il y mit l'homme qu'il avait formé.
\TextTitle{Description du jardin en Eden\FTNTT{Ge. 1:28-3:6}}
\VS{9}Yahweh Dieu fit germer de la terre des arbres de toute espèce, agréables à voir et bons à manger, et l'arbre de la vie au milieu du jardin, et l'arbre de la connaissance du bien et du mal.
\VS{10}Un fleuve sortait d'Eden pour arroser le jardin ; et de là il se divisait en quatre bras.
\VS{11}Le nom du premier est Pischon ; c'est le fleuve qui coule en entourant tout le pays de Havila où se trouve l'or.
\VS{12}L'or de ce pays est bon ; c'est là aussi que se trouvent le bdellium et la pierre d'onyx.
\VS{13}Le nom du second fleuve est Guihon ; c'est celui qui coule en entourant tout le pays de Cusch.
\VS{14}Le nom du troisième fleuve est Hiddékel, qui coule vers l'Assyrie ; et le quatrième fleuve est l'Euphrate.
\TextTitle{Commandement donné par Yahweh à l’homme\FTNTT{Ge. 1:28}}
\VS{15}Yahweh Dieu prit donc l'homme et le mit dans le jardin d'Eden pour le cultiver et pour le garder.
\VS{16}Puis Yahweh Dieu donna cet ordre à l'homme, en disant : Tu mangeras, tu mangeras\FTNT{En hebreu, le mot «akal» signifie «~manger~», «~se nourrir~», «~goûter~», «~jouir~», «~dévorer~», «~consumer~», et il a été utilisé deux fois de suite dans ce passage.} de tout arbre du jardin.
\VS{17}Mais quant à l'arbre de la connaissance du bien et du mal, tu n'en mangeras point, car le jour où tu en mangeras, tu mourras, tu mourras\FTNT{Dans la plupart des versions ce passage est mal traduit par «~tu mourras certainement~», alors que le terme mort en hébreu «~muwth~» est utilisé deux fois dans ce passage, et les écritures nous parlent de la mort physique et de la seconde mort, qui est le lac de feu (Ap. 2:11 ; Ap. 20:6,14). La mort physique précède la seconde physique.}.
\TextTitle{Yahweh forme une femme pour l'homme\FTNTT{Ge. 1:27}}
\VS{18}Yahweh Dieu dit : Il n'est pas bon que l'homme soit seul ; je lui ferai une aide semblable à lui.
\VS{19}Car Yahweh Dieu forma de la terre tous les animaux des champs et tous les oiseaux du ciel, puis il les fit venir vers Adam pour voir comment il les nommerait, et afin que le nom qu'Adam donnerait à tout animal fût, son nom.
\VS{20}Et Adam donna des noms à tout le bétail, et aux oiseaux du ciel, et à tous les animaux des champs ; mais pour Adam, il ne trouva point d'aide semblable à lui.
\VS{21}Et Yahweh Dieu fit tomber un profond sommeil sur Adam, qui s'endormit ; et Dieu prit une de ses côtes, et referma la chair à la place de cette côte.
\VS{22}Yahweh Dieu forma une femme de la côte qu'il avait prise d'Adam, et il l’amena vers Adam\FTNT{1 Co. 11:8.}.
\TextTitle{Union d’Adam et Eve}
\VS{23}Alors Adam dit : Voici cette fois celle qui est os de mes os et chair de ma chair ; on l’appellera femme, parce qu'elle a été prise de l'homme.
\VS{24}C'est pourquoi l'homme quittera son père et sa mère et s’attachera à sa femme, et ils deviendront une seule chair\FTNT{Ep. 5:30-31 ; Mt. 19:5 ; Mc. 10:7 ; 1 Co. 6:16.}.
\VS{25}Adam et sa femme étaient tous deux nus, et ils n’en avaient pas honte.
\Chap{3}
\TextTitle{Séduction du serpent et chute de l’homme}
\VerseOne{}Or le serpent\FTNT{Satan ou le serpent ancien (Ap. 12:9 ; Ap. 20:2).} était le plus prudent\FTNT{La prudence, la ruse, la subtilité du serpent, sont marquées dans l'Ecriture comme des qualités qui le distinguent des autres animaux (Mt. 10:16).} de tous les animaux des champs que Yahweh Dieu avait faits ; et il dit à la femme : Quoi ! Dieu a dit : Vous ne mangerez pas de tous les arbres du jardin ?
\VS{2}La femme répondit au serpent : Nous mangeons du fruit des arbres du jardin ;
\VS{3}mais quant au fruit de l'arbre qui est au milieu du jardin, Dieu a dit : Vous n'en mangerez point, et vous ne le toucherez point, de peur que vous ne mouriez.
\VS{4}Alors le serpent dit à la femme : Vous ne mourrez nullement ;
\VS{5}mais Dieu sait que le jour où vous en mangerez, vos yeux seront ouverts, et vous serez comme des dieux, connaissant le bien et le mal.
\VS{6}La femme donc voyant que le fruit de l'arbre était bon à manger et agréable à la vue, et que cet arbre était désirable pour donner de la science ; elle prit de son fruit, et en mangea, et elle en donna aussi à son mari qui était auprès d’elle, et il en mangea.
\TextTitle{La connaissance du bien et du mal}
\VS{7}Les yeux de tous les deux s’ouvrirent, ils connurent qu'ils étaient nus, et ils cousirent ensemble des feuilles de figuier, et s'en firent des ceintures.
\VS{8}Alors ils entendirent au vent du jour la voix de Yahweh Dieu qui se promenait par le jardin ; et Adam et sa femme se cachèrent loin de la face de Yahweh Dieu, au milieu des arbres du jardin.
\VS{9}Mais Yahweh Dieu appela Adam et lui dit : Où es-tu ?
\VS{10}Il répondit : J'ai entendu ta voix dans le jardin, et j'ai eu peur parce que je suis nu, et je me suis caché.
\VS{11}Et Dieu dit : Qui t'a appris que tu es nu ? Est-ce que tu as mangé du fruit de l'arbre dont je t'avais défendu de manger ?
\VS{12}Adam répondit : La femme que tu m'as donnée pour être avec moi m'a donné du fruit de l'arbre, et j'en ai mangé.
\VS{13}Et Yahweh Dieu dit à la femme : Pourquoi as-tu fait cela ? Et la femme répondit : Le serpent m'a séduite, et j'en ai mangé.
\TextTitle{La création soumise à la vanité\FTNTT{Ro. 8:20-22}}
\VS{14}Alors Yahweh Dieu dit au serpent : Parce que tu as fait cela, tu seras maudit entre tout le bétail et entre tous les animaux des champs ; tu marcheras sur ton ventre, et tu mangeras la poussière\FTNT{La poussière dont il est question n’est autre que l’homme pécheur (Ge. 3:19). Satan ne peut rien contre les véritables enfants de Dieu (Mt. 16:18 ; Lu. 10:19).} tous les jours de ta vie.
\VS{15}Je mettrai inimitié entre toi et la femme\FTNT{La femme représente en premier lieu Eve, la mère de tous les hommes. Ici, elle représente aussi Israël, l’épouse de Yahweh selon Ge. 37:5-11 et Ap. 12:1.}, et entre ta postérité\FTNT{La postérité du serpent regroupe l’homme impie (2 Th. 2:3-4 ; 1 Jn. 2:18-22), et tous ceux qui n’ont pas reçu Jésus-Christ comme Seigneur et Sauveur. En effet, seuls ceux qui ont reçu Jésus dans leur vie sont appelés enfants de Dieu (Jn. 1:12 ; 1 Jn. 3:8-10 ; 1 Jn. 5:19).} et sa postérité\FTNT{La postérité de la femme regroupe Jésus-Christ homme (Es. 7:14 ; Lu. 2:4-7), et l’Église, le Corps de Christ (Col. 1:24).} ; celle-ci te brisera la tête, et tu lui blesseras le talon.
\VS{16}Et il dit à la femme : J'augmenterai beaucoup la souffrance de tes grossesses ; tu enfanteras dans la douleur tes enfants ; tes désirs se porteront vers ton mari, et il dominera sur toi.
\VS{17}Puis il dit à Adam : Parce que tu as obéi à la parole de ta femme, et que tu as mangé le fruit de l'arbre au sujet duquel je t'avais donné cet ordre, en disant : Tu n'en mangeras point, la terre sera maudite à cause de toi ; tu en mangeras les fruits dans la peine, tous les jours de ta vie.
\VS{18}Et elle te produira des épines et des chardons ; et tu mangeras l'herbe des champs.
\VS{19}C’est à la sueur de ton visage que tu mangeras du pain, jusqu'à ce que tu retournes dans la terre, d’où tu as été pris ; car tu es poussière, et tu retourneras dans la poussière.
\TextTitle{L’homme et la femme revêtu de tuniques de peaux}
\VS{20}Et Adam appela sa femme Eve, parce qu'elle a été la mère de tous les vivants.
\VS{21}Yahweh Dieu fit à Adam et à sa femme des tuniques de peaux, et il les en revêtit.
\VS{22}Yahweh Dieu dit : Voici, l'homme est devenu comme l'un de nous, connaissant le bien et le mal. Mais maintenant il faut prendre garde, qu’il n’avance sa main, et aussi qu’il ne prenne de l'arbre de vie, et qu’il n’en mange, et ne vive éternellement.
\TextTitle{L’homme chassé du jardin}
\VS{23}Et Yahweh Dieu le chassa du jardin d’Eden pour qu’il cultive la terre d’où il avait été pris.
\VS{24}C’est ainsi qu’il chassa l'homme, et il mit à l’orient du jardin d’Eden des chérubins qui tournent ça et là une épée flamboyante pour garder le chemin de l'arbre de vie.
\Chap{4}
\TextTitle{La jalousie de Caïn contre son frère Abel}
\VerseOne{}Adam connut Eve sa femme ; elle conçut, et enfanta Caïn ; et elle dit : J'ai acquis un homme de par Yahweh.
\VS{2}Elle enfanta encore Abel, son frère ; et Abel fut berger, et Caïn laboureur.
\VS{3}Or, au bout de quelque temps, Caïn offrit à Yahweh une offrande des fruits de la terre\FTNT{Caïn était du diable, il est l’archétype du religieux qui pense pouvoir être sauvé par les œuvres (Lu. 11:51 ; 1 Jn. 3:12). Son offrande fut rejetée car il avait apporté devant Dieu le fruit de la terre qui avait été maudite (Ge. 3:17).  Cela revenait à offrir à Dieu le péché, la malédiction.} ;
\VS{4}et Abel, de son côté, offrit des premiers-nés de son troupeau, et de leur graisse\FTNT{Abel était juste et pieux, aussi il sut instinctivement apporter une offrande agréable à Dieu (Mt. 23:35 ; Lu. 11:51 ; Hé. 11:4). En l’occurrence, son offrande préfigurait le sacrifice du Seigneur.}. Yahweh eut égard à Abel, et à son offrande.
\VS{5}Mais il n'eut point d'égard à Caïn ni à son offrande ; et Caïn fut fort irrité, et son visage fut abattu.
\TextTitle{Yahweh avertit Caïn}
\VS{6}Et Yahweh dit à Caïn : Pourquoi es-tu irrité, et pourquoi ton visage est-il abattu ?
\VS{7}Si tu agis bien, tu relèveras ton visage, et si tu agis mal, le péché est couché à la porte, et ses désirs se portent vers toi ;  mais toi, domine sur lui.
\TextTitle{Caïn tue son frère Abel\FTNTT{Ge. 4:23}}
\VS{8}Et Caïn parla avec Abel son frère, et comme ils étaient dans les champs, Caïn se jeta sur Abel, son frère, et le tua.
\VS{9}Yahweh dit à Caïn : Où est Abel ton frère ? Et il lui répondit : Je ne sais, suis-je le gardien de mon frère, moi ?
\VS{10}Et Dieu dit : Qu'as-tu fait ? La voix du sang de ton frère crie de la terre à moi.
\VS{11}Maintenant donc tu seras maudit de la terre, qui a ouvert sa bouche pour recevoir de ta main le sang de ton frère.
\VS{12}Quand tu cultiveras la terre, elle ne te donnera plus son fruit, et tu seras vagabond et fugitif sur la terre.
\VS{13}Caïn dit à Yahweh : Mon châtiment est trop grand pour être supporté.
\VS{14}Voici, tu me chasses aujourd'hui de cette terre ; je serai caché loin de ta face, je serai vagabond et fugitif sur la terre, et quiconque me trouvera me tuera.
\VS{15}Yahweh lui dit : Si quelqu’un tuait Caïn, Caïn serait vengé sept fois. Ainsi Yahweh mit une marque sur Caïn afin que quiconque le trouverait ne le tue point.
\TextTitle{Caïn bâtit une cité loin de Yahweh}
\VS{16}Alors, Caïn s’éloigna de la face de Yahweh, et habita dans la terre de Nod, à l'orient d’Eden.
\VS{17}Puis Caïn connut sa femme ; elle conçut et enfanta Hénoc. Il bâtit une ville, et il donna à cette ville le nom de son fils Hénoc.
\VS{18}Hénoc engendra Irad, Irad engendra Mehujaël, Mehujaël engendra Metuschaël, et Métuschaël engendra Lémec.
\VS{19}Lémec prit deux femmes ; le nom de l'une était Ada, et le nom de l'autre Tsilla.
\VS{20}Ada enfanta Jabal : Il fut le père de ceux qui habitent dans les tentes et près des troupeaux.
\VS{21}Le nom de son frère était Jubal : Il fut le père de tous ceux qui jouent de la harpe et du chalumeau.
\VS{22}Tsilla aussi enfanta Tubal-Caïn, qui forgeait toutes sortes d'instruments d'airain et de fer. La soeur de Tubal-Caïn était Naama.
\VS{23}Lémec dit à Ada et à Tsilla ses femmes : Ecoutez ma voix femmes de Lémec, écoutez ma parole ! J’ai tué un homme pour ma blessure et un jeune homme pour ma meurtrissure.
\VS{24}Car si Caïn est vengé sept fois, Lémec le sera soixante-dix-sept fois.
\TextTitle{Naissance de Seth}
\VS{25}Adam connut encore sa femme ; elle enfanta un fils, et il l’appela du nom de Seth, car, dit-il, Dieu m'a donné un autre fils à la place d'Abel, que Caïn a tué.
\VS{26}Il naquit aussi un fils à Seth, et il l'appela du nom d’Enosch. C’est alors que l’on commença à proclamer le nom de Yahweh.
\Chap{5}
\TextTitle{La postérité d'Adam soumise à la mort\FTNTT{Ro. 5:12}}
\VerseOne{}Voici le livre de la postérité d'Adam, depuis le jour où Dieu créa l'homme, il le fit à la ressemblance de Dieu.
\VS{2}Il les créa mâle et femelle, et les bénit, et il leur donna le nom d'homme, le jour où ils furent créés.
\VS{3}Adam vécut cent trente ans, et engendra un fils à sa ressemblance, selon son image\FTNT{Désormais les hommes naissent à la ressemblance d’Adam, c’est-à-dire pécheurs (Ro. 3:23 ; Ro. 5:14-17).}, et il lui donna le nom de Seth.
\VS{4}Les jours d'Adam, après qu'il eut engendré Seth, furent de huit cents ans, et il engendra des fils et des filles.
\VS{5}Tous les jours qu'Adam vécut furent de neuf cent trente ans ; puis il mourut.
\TextTitle{De Seth aux fils de Noé\FTNTT{Ro. 5:12}}
\VS{6}Seth aussi vécut cent cinq ans, et engendra Enosch.
\VS{7}Seth, après qu'il eut engendré Enosch, vécut huit cent sept ans ; et il engendra des fils et des filles.
\VS{8}Tous les jours que Seth vécut furent de neuf cent douze ans ; puis il mourut.
\VS{9}Enosch, ayant vécu quatre-vingt-dix ans, engendra Kénan.
\VS{10}Enosch, après qu'il eut engendré Kénan, vécut huit cent quinze ans, et il engendra des fils et des filles.
\VS{11}Tous les jours qu'Enosch vécut furent de neuf cent cinq ans ; puis il mourut.
\VS{12}Kénan, ayant vécu soixante-dix ans, engendra Mahalaleel.
\VS{13}Kénan, après qu'il eut engendré Mahalaleel, vécut huit cent quarante ans ; et il engendra des fils et des filles.
\VS{14}Tous les jours que Kénan vécut furent de neuf cent dix ans ; puis il mourut.
\VS{15}Mahalaleel vécut soixante-cinq ans ; et il engendra Jéred.
\VS{16}Et Mahalaleel, après qu'il eut engendré Jéred, vécut huit cent trente ans, et il engendra des fils et des filles.
\VS{17}Tous les jours donc que Mahalaleel vécut furent de huit cent quatre-vingt-quinze ans ; puis il mourut.
\VS{18}Jéred, ayant vécu cent soixante-deux ans, engendra Hénoc.
\VS{19}Jéred, après avoir engendré Hénoc, vécut huit cents ans, et il engendra des fils et des filles.
\VS{20}Tous les jours que Jéred vécut furent de neuf cent soixante-deux ans ; puis il mourut.
\VS{21}Hénoc vécut soixante-cinq ans, et engendra Metuschélah.
\VS{22}Hénoc, après qu'il eut engendré Metuschélah, marcha avec Dieu trois cents ans ; et il engendra des fils et des filles.
\VS{23}Tous les jours qu'Hénoc vécut furent de trois cent soixante-cinq ans.
\VS{24}Hénoc marcha avec Dieu ; mais il ne parut plus parce que Dieu le prit.
\VS{25}Metuschélah, ayant vécu cent quatre-vingt-sept ans, engendra Lémec.
\VS{26}Metuschélah, après qu'il eut engendré Lémec, vécut sept cent quatre-vingt-deux ans ; et il engendra des fils et des filles.
\VS{27}Tous les jours que Metuschélah vécut furent de neuf cent soixante-neuf ans ; puis il mourut.
\VS{28}Lémec aussi vécut cent quatre-vingt-deux ans, et il engendra un fils.
\VS{29}Il l'appela Noé, en disant : Celui-ci nous consolera de notre oeuvre, et du travail pénible de nos mains, sur la terre que Yahweh a maudite.
\VS{30}Lémec, après qu'il eut engendré Noé, vécut cinq cent quatre-vingt-quinze ans ; et il engendra des fils et des filles.
\VS{31}Tous les jours que Lémec vécut furent de sept cent soixante-dix-sept ans ; puis il mourut.
\VS{32}Noé, âgé de cinq cents ans, engendra Sem, Cham, et Japhet.
\Chap{6}
\TextTitle{Le mal dans le cœur de l'homme\FTNTT{Ro. 5:12}}
\VerseOne{}Lorsque les hommes eurent commencé à se multiplier sur la face de la terre, et qu'ils eurent engendré des filles,
\VS{2}les fils de Dieu\FTNT{Ici, les fils de Dieu sont des anges qui ont quitté leur demeure (Jud. 1:5-7).} virent que les filles des hommes étaient belles, et ils en prirent pour femmes parmi toutes celles qu'ils choisirent.
\TextTitle{Yahweh ne conteste plus avec les hommes}
\VS{3}Yahweh dit : Mon Esprit ne contestera point à toujours avec les hommes\FTNT{C’est le Saint-Esprit qui nous convainc de péché, de jugement et de justice (Jn. 16:8). Lorsqu’il constate que le cœur d’une personne est définitivement endurci au point de refuser la repentance, il renonce à la convaincre de péché et il se retire. La génération antédiluvienne avait définitivement rejeté Dieu en choisissant de faire du mal son idole (Ge. 6:5). Elle était allée si loin dans l’abomination au point de s’accoupler avec des anges déchus (Ge. 6:4), ce qui laisse supposer un culte volontaire aux démons. Lorsque le Saint-Esprit est retiré d’une personne, il est remplacé par l’esprit d’égarement qui enferme le pécheur dans l’erreur et l’entraîne ainsi à sa condamnation éternelle (Mt. 12:31 ; 2 Th. 2:11 )}, car les hommes ne sont que chair, et leurs jours seront de cent vingt ans.
\TextTitle{Le monde avant le déluge\FTNTT{Lu. 17:27}}
\VS{4}Les géants étaient sur la terre en ce temps-là. Il en fut de même après que les fils de Dieu furent venus vers les filles des hommes, et qu’elles leur eurent donné des enfants. Ce sont ces hommes vaillants qui furent des gens de renom dans l’antiquité.
\TextTitle{Yahweh prépare un jugement}
\VS{5}Yahweh vit que la méchanceté des hommes était très grande sur la terre, et que toute l'imagination des pensées de leur cœur n'était que mal en tout temps.
\VS{6}Yahweh se repentit d'avoir fait l'homme sur la terre, et il fut affligé en son cœur.
\VS{7}Et Yahweh dit : J'exterminerai de la face de la terre les hommes que j'ai créés, depuis les hommes jusqu'au bétail, jusqu'aux reptiles, et même jusqu'aux oiseaux du ciel ; car je me repens de les avoir faits.
\TextTitle{La grâce de Yahweh sur Noé : Construction de l'arche}
\VS{8}Mais Noé trouva grâce aux yeux de Yahweh.
\VS{9}Voici la postérité de Noé. Noé était un homme juste et intègre en son temps ; Noé marchait avec Dieu.
\VS{10}Noé engendra trois fils : Sem, Cham, et Japhet.
\VS{11}Et la terre était corrompue devant Dieu, et remplie de violence.
\VS{12}Dieu donc regarda la terre, et voici elle était corrompue ; car toute chair avait corrompu sa voie sur la terre.
\VS{13}Et Dieu dit à Noé : La fin de toute chair est venue devant moi ; car ils ont rempli la terre de violence, et voici, je les détruirai avec la terre.
\VS{14}Fais-toi une arche\FTNT{L’arche  est un type de Christ et du salut en lui et par lui.  On peut voir plusieurs aspects de Jésus-Christ et de la rédemption dans la structure de l’arche :
- L’arche a été une révélation de Jésus-Christ donnée à Noé.  C’est en Jésus-Christ que nous avons le salut et la protection (Col. 1:12-13 ; Col. 3:3). 
-L’arche était faite de bois de gopher, probablement du cèdre.  Ce bois est un bois qui ne pourrit pas en condition normale. Ce bois préfigurait l’incorruptibilité de Jésus-Christ homme (Es. 53:9 ;  Hé. 4:15 ; 1 Pi. 2:22). 
-Au verset 14, on lit que Dieu demande à Noé d’enduire l’arche en dedans et en dehors avec de la  poix, c’est-à-dire du bitume.  Le mot «~poix~» vient de l’hébreu «~kaphar~», qui signifie «~expiation~».  Ce mot est traduit près de 70 fois dans le Tanakh par expiation.  Il est également traduit par «~réconciliation~», «~pardon~», «~miséricordieux~» et «~apaiser~».  L’allusion à l’expiation des péchés faite par Jésus-Christ est claire. Par son sacrifice, nous sommes rendus parfaits à jamais (Hé. 10:14-15).} de bois de gopher ; tu feras cette arche en cellules, et tu l’enduiras de poix en dedans et en dehors.
\VS{15}Et voici comment tu la feras : La longueur de l'arche sera de trois cents coudées ; sa largeur de cinquante coudées, et sa hauteur de trente coudées.
\VS{16}Tu feras une fenêtre à l'arche, et feras son comble d'une coudée de hauteur, et tu mettras la porte de l'arche à son côté, et tu la feras avec un bas, un second, et un troisième étage.
\VS{17}Et voici, je ferai venir un déluge d'eau sur la terre, pour détruire toute chair dans laquelle il y a souffle de vie sous les cieux ; et tout ce qui est sur la terre expirera.
\VS{18}Mais j'établirai mon alliance avec toi ; et tu entreras dans l'arche toi et tes fils, et ta femme, et les femmes de tes fils avec toi.
\VS{19}Et de tout ce qui a vie d'entre toute chair, tu en feras entrer deux de chaque espèce dans l'arche, pour les conserver en vie avec toi, à savoir le mâle et la femelle.
\VS{20}Des oiseaux, selon leur espèce, des bêtes à quatre pattes, selon leur espèce, et de tous les reptiles, selon leur espèce. Ils y entreront tous par paires avec toi, afin que tu les conserves en vie.
\VS{21}Prends aussi avec toi de tous les aliments que l’on mange, et rassemble-les auprès de toi, afin qu'ils servent pour ta nourriture et pour celle des animaux.
\VS{22}Et Noé fit selon tout ce que Dieu lui avait ordonné ; il le fit ainsi.
\Chap{7}
\TextTitle{Le jugement par le déluge}
\VerseOne{}Yahweh dit à Noé : Entre dans l’arche, toi et toute ta maison ; car je t'ai vu juste devant moi parmi cette génération. 
\VS{2} Tu prendras de toutes les bêtes pures sept de chaque espèce, le mâle et sa femelle ; mais des bêtes qui ne sont point pures, un couple, le mâle et la femelle.
\VS{3}Tu prendras aussi des oiseaux du ciel sept de chaque espèce, le mâle et sa femelle ; afin d'en conserver la race sur toute la terre.
\VS{4}Car dans sept jours, je ferai pleuvoir sur la terre pendant quarante jours et quarante nuits ; et j'exterminerai de la surface de la terre tous les êtres qui subsistent que j'ai faits.
\VS{5}Noé fit selon tout ce que Yahweh lui avait ordonné.
\VS{6}Noé était âgé de six cents ans quand le déluge des eaux vint sur la terre.
\VS{7}Noé donc entra dans l’arche avec ses fils, sa femme, et les femmes de ses fils, pour échapper aux eaux du déluge.
\VS{8}Des bêtes pures, des bêtes qui ne sont point pures, des oiseaux, et tout ce qui se meut sur la terre.
\VS{9}Elles entrèrent deux à deux vers Noé dans l'arche, le mâle et la femelle, comme Dieu l’avait ordonné à Noé.
\VS{10}Sept jours après, les eaux du déluge furent sur la terre.
\VS{11}En l'an six cent de la vie de Noé, au second mois, le dix-septième jour du mois, en ce jour-là toutes les sources du grand abîme furent rompues, et les écluses des cieux furent ouvertes.
\VS{12}La pluie tomba sur la terre pendant quarante jours et quarante nuits.
\VS{13}Ce même jour entrèrent dans l’arche Noé, Sem, Cham, et Japhet, fils de Noé, avec la femme de Noé, et les trois femmes de ses fils avec eux.
\VS{14}Eux, et tous les animaux selon leur espèce, et tout le bétail selon son espèce, et tous les reptiles qui se meuvent sur la terre selon leur espèce, et tous les oiseaux selon leur espèce ; et tout petit oiseau ayant des ailes, de quelque sorte que ce soit.
\VS{15}Ils entrèrent dans l'arche auprès de Noé, deux à deux, de toute chair ayant souffle de vie.
\VS{16}Il en entra mâle et femelle de toute chair comme Dieu l’avait ordonné à Noé, puis Yahweh ferma l'arche sur lui.
\VS{17}Le déluge fut pendant quarante jours sur la terre ; et les eaux crurent et élevèrent l'arche, et elle fut élevée au-dessus de la terre.
\VS{18}Les eaux  grossirent et s'accrurent beaucoup sur la terre, et l'arche flottait au-dessus des eaux.
\VS{19}Les eaux grossirent de plus en plus sur la terre, et toutes les hautes montagnes qui sont sous le ciel entier en furent couvertes.
\VS{20}Les eaux s’élevèrent de quinze coudées au-dessus des montagnes  qui furent couvertes.
\VS{21}Toute chair qui se mouvait sur la terre périt, tant les oiseaux que le bétail et les animaux, tous les reptiles qui rampaient sur la terre, et tous les hommes.
\VS{22}Tout ce qui avait respiration, souffle de vie dans ses narines, et qui était sur la terre sèche mourut.
\VS{23}Tous les êtres qui étaient sur la face de la terre furent donc exterminés, depuis les hommes jusqu’au bétail, aux reptiles et aux oiseaux du ciel ; ils furent exterminés de la face de la terre ; il ne resta seulement que Noé, et ce qui était avec lui dans l'arche.
\VS{24}Les eaux furent grosses sur la terre pendant cent cinquante jours.
\Chap{8}
\TextTitle{Fin du déluge}
\VerseOne{}Dieu se souvint de Noé, de tous les animaux et de tout le bétail qui étaient avec lui dans l'arche ; et Dieu fit passer un vent sur la terre, et les eaux s’apaisèrent.
\VS{2}Les sources de l'abîme et les écluses des cieux furent fermées et la pluie ne tomba plus du ciel.
\VS{3}Au bout de cent cinquante jours, les eaux se retirèrent sans interruption de dessus la terre, et diminuèrent.
\VS{4}Le dix-septième jour du septième mois, l'arche s'arrêta sur les montagnes d'Ararat.
\VS{5}Les eaux allèrent en diminuant de plus en plus jusqu'au dixième mois ; et au premier jour du dixième mois, les sommets des montagnes apparurent.
\VS{6}Au bout de quarante jours, Noé ouvrit la fenêtre qu’il avait faite à l'arche.
\VS{7}Il lâcha le corbeau, qui sortit, allant et revenant, jusqu'à ce que les eaux aient séché sur la terre.
\VS{8}Il lâcha aussi une colombe pour voir si les eaux avaient diminué à la surface de la terre.
\VS{9}Mais la colombe ne trouvant aucun lieu pour poser la plante de son pied, retourna à lui dans l'arche, car les eaux étaient sur toute la terre ; et Noé avançant sa main la reprit et la fit entrer dans l'arche.
\VS{10}Il attendit encore sept autres jours, il lâcha de nouveau la colombe hors de l'arche.
\VS{11}Sur le soir, la colombe revint à lui ; et voici, elle avait dans son bec une feuille d'olivier qu'elle avait arrachée ; et Noé connut que les eaux avaient diminué sur la terre.
\VS{12}Il attendit encore sept autres jours, puis il lâcha la colombe qui ne retourna plus à lui.
\VS{13}L’an six cent un de l'âge de Noé, le premier jour du premier mois, les eaux avaient diminué sur la terre. Noé ôta la couverture de l'arche, regarda, et voici, la surface de la terre avait séché.
\VS{14}Le vingt-septième jour du second mois la terre fut sèche.
\TextTitle{Noé sort de l'arche: Le règne des hommes\FTNTT{Ge. 8:11-15:32}}
\VS{15}Puis Dieu parla à Noé, en disant :
\VS{16}Sors de l'arche, toi et ta femme, tes fils, et les femmes de tes fils avec toi.
\VS{17}Fais sortir avec toi tous les animaux qui sont avec toi, de toute chair, tant les oiseaux que le bétail, et tous les reptiles qui rampent sur la terre ; qu'ils se répandent sur la terre, et qu'ils soient féconds et multiplient sur la terre.
\VS{18}Noé donc sortit, et avec lui ses fils, sa femme, et les femmes de ses fils.
\VS{19}Tous les animaux, tous les reptiles, tous les oiseaux, tout ce qui se meut sur la terre, selon leurs espèces, sortirent de l'arche.
\VS{20}Noé bâtit un autel à Yahweh, il prit de toutes les bêtes pures, et de tout oiseau pur, et il offrit des holocaustes sur l'autel.
\VS{21}Yahweh respira une odeur agréable, et dit en son cœur : Je ne maudirai plus la terre à cause des hommes, quoique les dispositions du coeur des hommes soient mauvaises dès leur jeunesse ; et je ne frapperai plus tout ce qui est vivant, comme je l’ai fait.
\VS{22}Tant que la terre subsistera, les semailles et les moissons, le froid et la chaleur, l'été et l'hiver, le jour et la nuit ne cesseront point.
\Chap{9}
\TextTitle{Yahweh établit une alliance avec Noé\FTNTT{Ge. 9:16}}
\VerseOne{}Dieu bénit Noé et ses fils, et leur dit : Soyez féconds, multipliez, et remplissez la terre.
\VS{2}Vous serez un sujet de crainte et d’effroi pour tout animal de la terre, pour tout oiseau du ciel, pour tout ce qui se meut sur la terre, et pour tous les poissons de la mer : Ils sont livrés entre vos mains.
\VS{3}Tout ce qui se meut et qui a vie sera votre nourriture ; je vous donne tout cela comme l'herbe verte.
\VS{4}Seulement, vous ne mangerez point de chair avec son âme, c'est-à-dire, son sang.
\VS{5}Sachez-le aussi, je redemanderai votre sang, le sang de vos âmes, je le redemanderai à tout animal ; et je redemanderai l’âme de l’homme de la main de l’homme, de la main de son frère.
\VS{6}Celui qui aura versé le sang de l'homme, par l'homme son sang sera versé ; car Dieu a fait l'homme à son image.
\VS{7}Vous donc, soyez féconds et multipliez, répandez-vous sur la terre et multipliez sur elle.
\VS{8}Dieu parla aussi à Noé et à ses fils qui étaient avec lui, en disant :
\VS{9}Et quant à moi, voici, j'établis mon alliance avec vous, et avec votre postérité après vous ;
\VS{10}avec tous les êtres vivants qui sont avec vous, tant les oiseaux que le bétail, et tous les animaux de la terre qui sont avec vous, tous ceux qui sont sortis de l'arche jusqu'à tous les animaux de la terre.
\VS{11}J'établis donc mon alliance avec vous ; aucune chair ne sera plus exterminée par les eaux du déluge, et il n'y aura plus de déluge pour détruire la terre.
\VS{12}Puis Dieu dit : C'est ici le signe de l'alliance que j’établis entre moi et vous, et tous les êtres vivants qui sont avec vous, pour les générations à toujours :
\VS{13}J’ai placé mon arc dans la nuée, et il servira de signe de l'alliance entre moi et la terre.
\VS{14}Quand j’aurai rassemblé des nuages au-dessus de la terre, l’arc paraîtra dans la nuée ;
\VS{15}et je me souviendrai de mon alliance entre moi et vous, et tous les êtres vivants de toute chair, et les eaux ne deviendront plus un déluge pour détruire toute chair.
\VS{16}L'arc donc sera dans la nuée, et je le regarderai, et je me souviendrai de l'alliance perpétuelle entre Dieu et tous les êtres vivants  de toute chair qui est sur la terre.
\VS{17}Dieu donc dit à Noé : C'est là le signe de l'alliance que j'ai établie entre moi et toute chair qui est sur la terre.
\VS{18}Les fils de Noé qui sortirent de l'arche étaient Sem, Cham, et Japhet. Cham fut père de Canaan.
\VS{19}Ce sont là les trois fils de Noé, et c’est leur postérité qui peupla toute la terre.
\TextTitle{Le péché de Noé}
\VS{20}Or, Noé commença à cultiver la terre, et planta de la vigne.
\VS{21}Il but du vin, s'enivra, et se découvrit au milieu de sa tente.
\VS{22}Cham, père de Canaan, vit la nudité de son père\FTNT{Lé. 18:6-19 ; Lé. 20:11-21.}, et il le rapporta dehors à ses deux frères.
\VS{23}Alors Sem et Japhet prirent un manteau qu'ils mirent sur leurs deux épaules, et marchant à reculons, ils couvrirent la nudité de leur père ; et leurs visages étaient tournés en arrière, de sorte qu'ils ne virent point la nudité de leur père.
\VS{24}Et quand Noé se réveilla de son vin, il apprit ce que lui avait fait son fils cadet.
\TextTitle{Noé prononce une malédiction contre Canaan}
\VS{25}C'est pourquoi il dit : Maudit soit Canaan\FTNT{Une idée erronée selon laquelle les noirs auraient été maudits par Dieu au travers de la malédiction de Canaan s’est répandue pendant des siècles. On a ainsi légitimé la domination des peuples africains par les puissances occidentales blanches, et par la même occasion l’esclavage. Il faut préciser que les descendants de Cham furent Cush (Ethiopie), Mitsraïm (Egypte), Puth (les Celtes) et Canaan (Palestine, pays que Dieu a donné aux descendants de Sem, selon Ge. 15). Cham est le fils cadet, c’est-à-dire le coupable aux yeux de Noé. Mais c’est à Canaan (Palestine), le fils de Cham, donc petit-fils de Noé, que s’adresse la malédiction. Selon la Bible, les peuples africains sont des descendants de Cham, mais par son fils Cush et non par Canaan. La prétendue malédiction des noirs n’a donc aucun fondement.} ; il sera serviteur des serviteurs de ses frères.
\VS{26}Il dit aussi : Béni soit Yahweh, Dieu de Sem ; et que Canaan soit leur serviteur.
\VS{27}Que Dieu étende en douceur Japhet, et qu’il habite dans les tentes de Sem ; et que Canaan soit leur serviteur.
\VS{28}Noé vécut après le déluge trois cent cinquante ans.
\VS{29}Tout le temps donc que Noé vécut fut de neuf cent cinquante ans ; puis il mourut.
\Chap{10}
\TextTitle{La postérité de Noé}
\VerseOne{}Voici la postérité des enfants de Noé, Sem, Cham et Japhet ; il leur naquit des fils après le déluge.
\VS{2}Les fils de Japhet furent : Gomer, Magog, Madaï, Javan, Tubal, Mésech, et Tiras.
\VS{3}Les fils de Gomer : Aschkenaz, Riphat, et Togarma.
\VS{4}Les fils de Javan : Elischa, Tarsis, Kittim, et Dodanim.
\VS{5}C’est par eux qu’ont été peuplées les îles des nations selon leurs terres, chacun selon sa langue, selon leurs familles, entre leurs nations.
\VS{6}Les fils de Cham furent : Cusch, Mitsraïm, Puth, et Canaan.
\VS{7}Les fils de Cusch : Saba, Havila, Sabta, Raema, et Sabteca. Les fils de Raema : Séba et Dedan.
\VS{8}Cusch engendra aussi Nimrod\FTNT{Nimrod ou Nemrod, dont le nom signifie «~rebelle~», fut le premier roi de l’histoire biblique. Fils de Cusch (Ethiopie), lui-même premier-né de Cham, fils de Noé (Ge. 10:8-10), il fut à la tête du premier empire après le déluge. Il se distingua en qualité de puissant chasseur  «~devant Yahweh~» ou «~contre Yahweh~». Le contexte du chapitre 10 laisse entendre que Nimrod était un puissant chasseur qui provoquait Dieu. Fondateur de Ninive, il est surtout connu pour avoir été à l’origine du projet de la tour de Babel.}, c’est lui qui commença à être puissant sur la terre.
\VS{9}Il fut un puissant chasseur devant Yahweh, c'est pourquoi l'on a dit : Comme Nimrod, le puissant chasseur devant Yahweh.
\VS{10}Il régna d’abord sur Babel\FTNT{Le nom Babel signifie confusion par le mélange.}, Erec, Accad, et Calné au pays de Schinear.
\VS{11}De ce pays-là sortit Assur, et il bâtit Ninive et les rues de la ville, Rehoboth-Hir et Calach,
\VS{12}et Résen, entre Ninive et Calach, qui est une grande ville.
\VS{13}Mitsraïm engendra les Ludim, les Anamim, les Lehabim, les Naphtuhim,
\VS{14}les Patrusim, les Casluhim, d’où sont sortis les Philistins, et les Caphtorim.
\VS{15}Canaan engendra Sidon, son premier-né, et Heth ;
\VS{16}et les Jébusiens, les Amoréens, les Guirgasiens,
\VS{17}les Héviens, les Arkiens, les Siniens,
\VS{18}les Arvadiens, les Tsemariens, les Hamathiens. Ensuite, les familles des Cananéens se sont dispersées.
\VS{19}Les limites des Cananéens furent depuis Sidon, quand on vient vers Guérar, jusqu'à Gaza, en allant vers Sodome et Gomorrhe, Adma, et Tseboïm, jusqu'à Léscha.
\VS{20}Ce sont là les fils de Cham selon leurs familles et leurs langues, selon leurs pays, et selon leurs nations.
\VS{21}Il naquit aussi des fils à Sem, père de tous les fils d'Héber, et frère aîné de Japhet.
\VS{22}Les fils de Sem furent : Elam, Assur, Arpacschad, Lud et Aram.
\VS{23}Les fils d'Aram : Uts, Hul, Guéter et Masch.
\VS{24}Arpacschad engendra Schélach ; et Schélach engendra Héber.
\VS{25}Il naquit à Héber deux fils : Le nom de l'un était Péleg, parce que de son temps la terre fut partagée ; et le nom de son frère était Jokthan.
\VS{26}Jokthan engendra Almodad, Schéleph, Hatsarmaveth, Jérach,
\VS{27}Hadoram, Uzal, Dikla,
\VS{28}Obal, Abimaël, Séba,
\VS{29}Ophir, Havila, et Jobab. Tous ceux-là sont les enfants de Jokthan.
\VS{30}Ils habitèrent depuis Méscha, du côté de Sephar jusqu’à la montagne de l’orient.
\VS{31}Ce sont là les fils de Sem, selon leurs familles, selon leurs langues, selon leurs pays, et selon leurs nations.
\VS{32}Telles sont les familles des fils de Noé, selon leurs lignées, selon leurs nations. Et c’est d’eux que sont sorties les nations qui se sont répandues sur la terre après le déluge.
\Chap{11}
\TextTitle{Un projet humain : La tour de Babel}
\VerseOne{}Alors toute la terre avait un même langage et les mêmes paroles.
\VS{2}Mais il arriva qu'étant partis d'orient, ils trouvèrent une vallée au pays de Schinear où ils habitèrent.
\VS{3}Et ils se dirent l'un à l'autre : Allons ! Faisons des briques, et cuisons-les très bien au feu. Et la brique leur servit de pierre, et le bitume leur servit d’argile.
\VS{4}Puis ils dirent : Allons ! Bâtissons-nous une ville, et une tour dont le sommet soit jusqu’aux cieux ; et faisons-nous un nom, de peur que nous ne soyons dispersés sur toute la terre.
\VS{5}Alors Yahweh descendit pour voir la ville et la tour que les fils des hommes bâtissaient.
\VS{6}Et Yahweh dit : Voici, ce n'est qu'un seul et même peuple, ils ont un même langage, et ils commencent à travailler ; et maintenant rien ne les empêchera d'exécuter ce qu'ils ont projeté.
\TextTitle{Yahweh confond le langage humain}
\VS{7}Allons ! Descendons, et là confondons leur langage afin qu'ils n'entendent point le langage les uns des autres.
\VS{8}Ainsi Yahweh les dispersa de là par toute la terre, et ils cessèrent de bâtir la ville.
\VS{9}C'est pourquoi on l’appela du nom de Babel, car c’est là que Yahweh confondit le langage de toute la terre, et c’est de là que Yahweh les dispersa sur toute la terre.
\TextTitle{La postérité de Sem, ancêtre d’Abram}
\VS{10}Voici la postérité de Sem : Sem, âgé de cent ans, engendra Arpacschad, deux ans après le déluge.
\VS{11}Sem, après qu'il eut engendré Arpacschad, vécut cinq cents ans, et engendra des fils et des filles.
\VS{12}Arpacschad vécut trente-cinq ans, et engendra Schélach.
\VS{13}Arpacschad, après qu'il eut engendré Schélach, vécut quatre cent trois ans, et engendra des fils et des filles.
\VS{14}Schélach, ayant vécu trente ans, engendra Héber.
\VS{15}Schélach, après qu'il eut engendré Héber, vécut quatre cent trois ans, et engendra des fils et des filles.
\VS{16}Héber, ayant vécu trente-quatre ans, engendra Péleg.
\VS{17}Héber, après qu'il eut engendré Péleg, vécut quatre cent trente ans, et engendra des fils et des filles.
\VS{18}Péleg, ayant vécu trente ans, engendra Rehu.
\VS{19}Péleg, après qu'il eut engendré Rehu, vécut deux cent neuf ans, et engendra des fils et des filles.
\VS{20}Rehu, ayant vécu trente-deux ans, engendra Serug.
\VS{21}Rehu, après qu'il eut engendré Serug, vécut deux cent sept ans, et engendra des fils et des filles.
\VS{22}Serug, ayant vécu trente ans, engendra Nachor.
\VS{23}Serug, après qu'il eut engendré Nachor, vécut deux cents ans, et engendra des fils et des filles.
\VS{24}Nachor, ayant vécu vingt-neuf ans, engendra Térach.
\VS{25}Nachor, après qu'il eut engendré Térach, vécut cent dix-neuf ans, et engendra des fils et des filles.
\VS{26}Térach, ayant vécu soixante-dix ans, engendra Abram, Nachor, et Haran.
\VS{27}Voici la postérité de Térach : Térach engendra Abram, Nachor, et Haran ; et Haran engendra Lot.
\VS{28}Et Haran mourut en présence de son père, au pays de sa naissance, à Ur en Chaldée.
\VS{29}Abram et Nachor prirent chacun une femme. Le nom de la femme d'Abram était Saraï ; et le nom de la femme de Nachor était Milca, fille de Haran, père de Milca et de Jisca.
\VS{30}Saraï était stérile, et n'avait point d'enfants.
\TextTitle{Séjour à Charan}
\VS{31}Térach prit son fils Abram, et Lot fils de son fils, qui était fils de Haran, et Saraï, sa belle-fille, femme d'Abram, son fils, et ils sortirent ensemble d'Ur en Chaldée pour aller au pays de Canaan, et ils vinrent jusqu'à Charan, et ils y habitèrent.
\VS{32}Les jours de Térach furent de deux cent cinq ans ; puis il mourut à Charan.
\Chap{12}
\TextTitle{Appel d'Abram : La promesse de Yahweh\FTNTT{Ge. 12:2 ; 13:14-18 ; 15:1-21 ; 17:4-8 ; 22:15-24 ; 26:1-5 ; 28:10-15}}
\VerseOne{}Yahweh dit à Abram : Va pour toi, hors de ta terre, de ta patrie, et de la maison de ton père, vers la terre que je te montrerai\FTNT{Ac. 7:3 ; Hé. 11:8.}.
\VS{2}Je te ferai devenir une grande nation, et je te bénirai, je rendrai ton nom grand, et tu seras béni.
\VS{3}Je bénirai ceux qui te béniront, et je maudirai ceux qui te maudiront ; et toutes les familles de la terre seront bénies en toi\FTNT{Ac. 3:25 ; Ga. 3:8.}.
\TextTitle{Abram sur la terre de Canaan}
\VS{4}Abram donc partit, comme Yahweh le lui avait dit, et Lot alla avec lui. Abram était âgé de soixante-quinze ans quand il sortit de Charan.
\VS{5}Abram prit aussi Saraï, sa femme, et Lot, fils de son frère, avec tous les biens qu'ils avaient acquis, et les personnes qu'ils avaient eues à Charan ; et ils partirent pour aller dans le pays de Canaan, et ils arrivèrent au pays de Canaan\FTNT{Ac. 7:4.}.
\VS{6}Abram parcourut le pays jusqu'au lieu nommé Sichem, et jusqu'aux chênes de Moré ; et les Cananéens étaient alors dans le pays.
\VS{7}Yahweh apparut à Abram, et lui dit : Je donnerai ce pays à ta postérité. Et Abram bâtit là un autel à Yahweh qui lui était apparu.
\VS{8}Il se transporta de là vers la montagne, à l'orient de Béthel, et il dressa ses tentes, ayant Béthel à l'occident, et Aï à l'orient ; et il bâtit là un autel à Yahweh, et invoqua le nom de Yahweh.
\VS{9}Puis Abram partit de là, marchant et s'avançant vers le midi.
\TextTitle{Abram en Egypte}
\VS{10}Mais la famine étant survenue dans le pays, Abram descendit en Egypte pour s'y retirer, car la famine était grande dans le pays.
\VS{11}Comme il était près d'entrer en Egypte, il dit à Saraï, sa femme : Voici, je sais que tu es une fort belle femme ;
\VS{12}c'est pourquoi, quand les Egyptiens te verront, ils diront : C'est la femme de cet homme, et ils me tueront, mais ils te laisseront vivre.
\VS{13}Dis donc, je te prie, que tu es ma sœur, afin que je sois bien traité à cause de toi, et que par ton moyen, ma vie soit préservée.
\VS{14}Il arriva donc qu'aussitôt qu'Abram fut arrivé en Egypte, les Egyptiens virent que cette femme était fort belle.
\VS{15}Les principaux de la cour de Pharaon la virent aussi et la vantèrent à Pharaon, et elle fut enlevée pour être menée dans la maison de Pharaon.
\VS{16}Il traita bien Abram à cause d'elle, de sorte qu'il en eut des brebis, des bœufs, des ânes, des serviteurs, des servantes, des ânesses, et des chameaux.
\VS{17}Mais Yahweh frappa de grandes plaies Pharaon et sa maison, à cause de Saraï, femme d'Abram.
\VS{18}Alors Pharaon appela Abram, et lui dit : Qu'est-ce que tu m'as fait ? Pourquoi ne m'as-tu pas déclaré que c'était ta femme ?
\VS{19}Pourquoi as-tu dit : C'est ma sœur ? Car je l'avais prise pour ma femme ; mais maintenant, voici ta femme, prends-la, et va-t'en.
\VS{20}Et Pharaon ayant donné ordre à ses gens, ils le renvoyèrent, lui, sa femme, et tout ce qui était à lui.
\Chap{13}
\TextTitle{Retour d'Abram à Canaan}
\VerseOne{}Abram donc monta d'Egypte vers le midi, lui, sa femme, et tout ce qui lui appartenait, et Lot avec lui.
\VS{2}Et Abram était très riche en bétail, en argent, et en or.
\VS{3}Et il s'en retourna en suivant la route qu'il avait suivie du midi à Béthel, jusqu'au lieu où il avait dressé ses tentes au commencement, entre Béthel et Aï,
\VS{4}au même lieu où était l'autel qu'il y avait bâti au commencement, et Abram invoqua là le nom de Yahweh.
\TextTitle{Abram se sépare de Lot\FTNTT{Ge. 13:12}}
\VS{5}Lot aussi, qui marchait avec Abram, avait des brebis, des boeufs, et des tentes.
\VS{6}Et le pays ne pouvait les porter pour demeurer ensemble ; car leurs biens étaient si grand qu'ils ne pouvaient demeurer ensemble.
\VS{7}Il y eut querelle entre les bergers du bétail d'Abram et les bergers du bétail de Lot ; or en ce temps-là, les Cananéens et les Phérésiens habitaient dans le pays.
\VS{8}Et Abram dit à Lot : Je te prie qu'il n'y ait point de dispute entre moi et toi, ni entre mes bergers et les tiens, car nous sommes frères.
\VS{9}Tout le pays n'est-il pas devant toi ? Sépare-toi je te prie d'avec moi. Si tu vas à gauche, j’irai à droite ; et si tu vas à droite, j’irai à gauche.
\TextTitle{Lot s'établit à Sodome\FTNTT{Ge. 13:10}}
\VS{10}Lot, levant les yeux, vit que toute la plaine du Jourdain était entièrement arrosée. Avant que Yahweh ait détruit Sodome et Gomorrhe, c’était, jusqu'à Tsoar, comme le jardin de Yahweh, et comme le pays d'Egypte.
\VS{11}Lot choisit pour lui toute la plaine du Jourdain, et alla du côté de l’orient ; ainsi ils se séparèrent l'un de l'autre.
\VS{12}Abram habita dans le pays de Canaan, et Lot habita dans les villes de la plaine, et dressa ses tentes jusqu'à Sodome.
\VS{13}Les habitants de Sodome étaient méchants et de grands pécheurs contre Yahweh.
\TextTitle{Yahweh confirme son alliance avec Abram}
\VS{14}Yahweh dit à Abram, après que Lot se fut séparé de lui : Lève maintenant tes yeux, et regarde du lieu où tu es vers le nord, le midi, l'orient, et l'occident.
\VS{15}Car je te donnerai, à toi et à ta postérité pour toujours, tout le pays que tu vois.
\VS{16}Je rendrai ta postérité comme la poussière de la terre ; en sorte que si quelqu'un peut compter la poussière de la terre, il comptera aussi ta postérité\FTNT{Ro. 4:18 ; Hé. 11:12.}.
\VS{17}Lève-toi donc et promène-toi dans le pays, dans sa longueur et dans sa largeur, car je te le donnerai.
\VS{18}Abram ayant transporté ses tentes, alla habiter dans les plaines de Mamré, qui sont près d’Hébron et là, il bâtit un autel à Yahweh.
\Chap{14}
\TextTitle{Abram va au secours de Lot}
\VerseOne{}Dans le temps d'Amraphel, roi de Schinear, d'Arjoc, roi d'Ellasar, de Kedorlaomer, roi d'Elam, et de Tideal, roi de Gojim,
\VS{2}il arriva qu’ils firent la guerre contre Béra, roi de Sodome, et contre Birscha, roi de Gomorrhe, et contre Schineab, roi d'Adma, et contre Schémeéber, roi de Tseboïm, et contre le roi de Béla, qui est Tsoar.
\VS{3}Tous ceux-ci se joignirent dans la vallée de Siddim, qui est la mer salée.
\VS{4}Ils avaient été asservis douze années à Kedorlaomer, et la treizième année, ils s'étaient révoltés.
\VS{5}A la quatorzième année, Kedorlaomer et les rois qui étaient avec lui vinrent et ils battirent les Rephaïm à Aschteroth-Karnaïm, les Zuzim à Ham, et les Emin à la plaine de Schavé-Kirjathaïm,
\VS{6}et les Horiens dans leur montagne de Séir, jusqu'au chêne de Paran, qui est près du désert.
\VS{7}Puis ils s’en retournèrent et vinrent à En-Mischpath, qui est Kadès ; et ils frappèrent tout le pays des Amalécites et des Amoréens qui habitaient dans Hatsatson-Thamar.
\VS{8}Alors le roi de Sodome, le roi de Gomorrhe, le roi d'Adma, le roi de Tseboïm, et le roi de Béla qui est Tsoar, sortirent et rangèrent leurs troupes contre eux dans la vallée de Siddim.
\VS{9}C'est-à-dire contre Kedorlaomer, roi d'Elam, et contre Tideal, roi de Gojim, et contre Amraphel, roi de Schinear, et contre Arjoc, roi d'Ellasar : Quatre rois contre cinq.
\VS{10}La vallée de Siddim était pleine de puits de bitume ; les rois de Sodome et de Gomorrhe s'enfuirent et y tombèrent, et le reste s'enfuit dans la montagne.
\VS{11}Ils prirent donc toutes les richesses de Sodome et de Gomorrhe, et tous leurs vivres ; puis ils se retirèrent.
\VS{12}Ils prirent aussi Lot, fils du frère d'Abram, qui habitait dans Sodome, et tous ses biens ; puis ils s'en allèrent.
\VS{13}Un fuyard vint avertir Abram, l’Hébreu, qui demeurait dans les plaines de Mamré, l’Amoréen, frère d'Eschcol, et frère d’Aner, qui avaient fait alliance avec Abram.
\VS{14}Dès qu’Abram eut appris que son frère avait été emmené prisonnier, il arma trois cent dix-huit de ses plus braves serviteurs, nés dans sa maison, et il poursuivit ces rois jusqu'à Dan.
\VS{15}Il divisa sa troupe, il se jeta sur eux de nuit, lui et ses serviteurs ; il les battit et les poursuivit jusqu'à Choba, qui est à la gauche de Damas.
\VS{16}Il ramena tous les biens qu'ils avaient pris ; il ramena aussi Lot, son frère, ses biens, les femmes et le peuple.
\TextTitle{Melchisédek, sacrificateur d'El Elyon (Dieu Très-Haut)}
\VS{17}Le roi de Sodome sortit à la rencontre d’Abram qui revenait vainqueur de Kedorlaomer, et des rois qui étaient avec lui, dans la vallée de la plaine, qui est la vallée royale.
\VS{18}Melchisédek\FTNT{Melchisédek est un type de Christ (Ps. 110:4 ; Hé. 5:5-6 ; Hé. 6:20 ; Hé. 7:1-2). Ce personnage nous montre l’aspect de Christ en tant que roi de Salem, ce qui signifie «~paix~», et Souverain Sacrificateur possédant un sacerdoce non transmissible (Hé. 7:24).}, roi de Salem, fit apporter du pain et du vin, or il était Sacrificateur du Dieu Très-Haut.
\VS{19}Il bénit Abram en disant : Béni soit Abram par le Dieu Très-Haut, Maître du ciel et de la terre.
\VS{20}Béni soit le Dieu Très-Haut qui a livré tes ennemis entre tes mains. Et Abram lui donna la dîme\FTNT{Voir commentaire sur la dîme en No. 18:21 et Mal. 3:10.} de tout.
\VS{21}Le roi de Sodome dit à Abram : Donne-moi les personnes, et prends pour toi les richesses.
\VS{22}Abram répondit au roi de Sodome : Je lève ma main vers Yahweh, le Dieu Très-Haut, Maître du ciel et de la terre :
\VS{23}Je ne prendrai rien de tout ce qui est à toi, pas même un fil, ni un cordon de soulier, afin que tu ne dises point : J'ai enrichi Abram.
\VS{24}Seulement, ce que les jeunes gens ont mangé, et la part des hommes qui sont venus avec moi, Aner, Eschcol, et Mamré, qui prendront leur part.
\Chap{15}
\TextTitle{Yahweh promet un enfant à Abram}
\VerseOne{}Après ces choses, la parole de Yahweh fut adressée à Abram dans une vision, en disant : Abram, ne crains point, je suis ton bouclier, et ta récompense sera très grande.
\VS{2}Abram répondit : Seigneur Yahweh, que me donneras-tu ? Je m'en vais sans laisser d'enfants après moi, et l’héritier de ma maison c'est Eliézer de Damas.
\VS{3}Abram dit aussi : Voici, tu ne m'as point donné d'enfants ; et voilà, le serviteur né dans ma maison sera mon héritier.
\VS{4}Alors la parole de Yahweh lui fut adressée ainsi : Ce n’est pas lui qui sera ton héritier, mais c’est celui qui sortira de tes entrailles qui sera ton héritier.
\VS{5}Puis l'ayant fait sortir dehors, il lui dit : Lève maintenant les yeux au ciel et compte les étoiles si tu peux les compter. Et il lui dit : Ainsi sera ta postérité.
\VS{6}Abram crut à Yahweh qui lui imputa cela à justice\FTNT{Ga. 3:6 ; Ja. 2:23 ; Ro. 4:3.}.
\TextTitle{Yahweh annonce l'esclavage de la postérité d'Abram}
\VS{7}Et il lui dit : Je suis Yahweh qui t'ai fait sortir d'Ur en Chaldée, afin de te donner ce pays-ci pour le posséder.
\VS{8}Abram répondit : Seigneur Yahweh, à quoi connaîtrai-je que je le posséderai ?
\VS{9}Et Yahweh lui répondit : Prends une génisse de trois ans,  une chèvre de trois ans, un bélier de trois ans, une tourterelle, et un pigeon.
\VS{10}Abram prit tous ces animaux, les coupa par le milieu, et mit chaque morceau l’un vis-à-vis de l’autre, mais il ne partagea point les oiseaux.
\VS{11}Les oiseaux de proie descendirent sur les cadavres, mais Abram les chassa.
\VS{12}Au coucher du soleil, un profond sommeil tomba sur Abram, et voici, une frayeur d'une grande obscurité tomba sur lui.
\VS{13}Et Yahweh dit à Abram : Sache comme une chose certaine que tes descendants habiteront quatre cents ans comme étrangers dans un pays qui ne leur appartiendra point, et qu’ils seront asservis aux habitants du pays qui les opprimera\FTNT{Ac. 7:6 ; Ga. 3:17.}.
\VS{14}Mais je jugerai la nation à laquelle ils seront asservis, et après cela ils sortiront avec de grands biens\FTNT{Ex. 3:22.}.
\VS{15}Et toi tu iras vers tes pères en paix, et tu seras enterré après une heureuse vieillesse.
\VS{16}A la quatrième génération, ils reviendront ici ; car l'iniquité des Amoréens n'est pas encore à son comble.
\VS{17}Quand le soleil fut couché, il y eut une obscurité profonde, et voici, ce fut une fournaise fumante, et des flammes passèrent entre les animaux qui avaient été partagés.
\VS{18}En ce jour-là, Yahweh traita alliance avec Abram, en disant : Je donne ce pays à ta postérité, depuis le fleuve d'Egypte jusqu'au grand fleuve, le fleuve d'Euphrate ;
\VS{19}le pays des Kéniens, des Keniziens, des Kadmoniens,
\VS{20}des Héthiens, des Phéréziens, des Rephaïm,
\VS{21}des Amoréens, des Cananéens, des Guirgasiens, et des Jébusiens.
\Chap{16}
\TextTitle{Saraï pousse Abram dans les bras de sa servante}
\VerseOne{}Saraï, femme d'Abram, ne lui avait enfanté aucun enfant, mais elle avait une servante égyptienne nommée Agar.
\VS{2}Et Saraï dit à Abram : Voici, Yahweh m'a rendue stérile ; viens je te prie vers ma servante, peut-être aurai-je des enfants par elle. Et Abram écouta la voix de Saraï.
\VS{3}Alors Saraï, femme d'Abram, prit Agar, sa servante égyptienne, et la donna pour femme à Abram, son mari, après qu’Abram eut habité dix ans dans le pays de Canaan.
\VS{4}Il alla donc vers Agar, et elle conçut. Quand Agar se vit enceinte, elle regarda sa maîtresse avec mépris.
\VS{5}Et Saraï dit à Abram : L'outrage qui m'est fait retombe sur toi. J’ai mis ma servante dans ton sein, mais quand elle a vu qu'elle avait conçu, elle m'a regardée avec mépris. Que Yahweh soit juge entre moi et toi !
\VS{6}Alors Abram répondit à Saraï : Voici, ta servante est entre tes mains, traite-la comme il te plaira. Saraï donc la maltraita, et Agar s'enfuit de devant elle.
\VS{7}Mais l'Ange de Yahweh la trouva auprès d'une fontaine d'eau dans le désert, près de la fontaine qui est sur le chemin de Schur.
\VS{8}Il lui dit : Agar, servante de Saraï, d'où viens-tu ? Et où vas-tu ? Et elle répondit : Je m'enfuis de devant Saraï, ma maîtresse.
\VS{9}L'Ange de Yahweh lui dit : Retourne vers ta maîtresse et humilie-toi sous sa main.
\VS{10}L'Ange de Yahweh lui dit : Je multiplierai beaucoup ta postérité, elle sera si nombreuse qu'on ne pourra la compter.
\VS{11}L'Ange de Yahweh lui dit aussi : Voici, tu as conçu, et tu enfanteras un fils que tu appelleras Ismaël, car Yahweh a entendu ton affliction.
\VS{12}Et ce sera un homme farouche comme un âne sauvage ; sa main sera contre tous, et la main de tous contre lui ; et il habitera en face de tous ses frères.
\VS{13}Alors elle appela Atta-El-roï (tu es le Dieu qui me voit) le nom de Yahweh qui lui avait parlé ; car elle dit : N'ai-je pas même, ici, vu celui qui me voyait ?
\VS{14}C'est pourquoi on a appelé ce puits le puits du vivant qui me voit ; lequel est entre Kadès et Bared.
\TextTitle{Naissance d'Ismaël}
\VS{15}Agar donc enfanta un fils à Abram ; et Abram donna le nom d’Ismaël  au fils qu'Agar lui avait enfanté\FTNT{Ga. 4:22.}.
\VS{16}Abram était âgé de quatre-vingt-six ans quand Agar enfanta Ismaël à Abram.
\Chap{17}
\TextTitle{El Schaddaï (Dieu Tout-Puissant) confirme sa promesse}
\VerseOne{}Lorsqu’Abram fut âgé de quatre-vingt-dix-neuf ans, Yahweh lui apparut et lui dit : Je suis le Dieu Tout-Puissant\FTNT{Dieu se révèle ici à Abraham comme le Dieu Tout-Puissant. Or Christ s’est présenté à l’apôtre Jean comme le Dieu Tout Puissant (Ap. 1:8).  Plus loin en Ap. 5:6, le Seigneur apparaît au milieu du trône céleste sous la forme d’un Agneau ayant sept cornes qui représentent sa toute-puissance. Jésus est bien le Dieu Tout-Puissant qui s’était révélé à Abraham (Da. 8:20-22).}. Marche devant ma face, et sois intègre.
\VS{2}J’établirai mon alliance entre moi et toi, et je te multiplierai très abondamment.
\VS{3}Alors Abram tomba sur sa face, et Dieu lui parla et lui dit :
\TextTitle{Abram devient Abraham}
\VS{4}Quant à moi, voici, mon alliance est avec toi, et tu deviendras père d'une multitude de nations\FTNT{Ro. 4:17.}.
\VS{5}On ne t’appellera plus Abram\FTNT{Né. 9:7.}, mais ton nom sera Abraham ; car je t'ai établi père d'une multitude de nations.
\TextTitle{Promesse d’une alliance éternelle}
\VS{6}Je te rendrai fécond  à l’extrême, et je te ferai devenir des nations ; même des rois sortiront de toi\FTNT{Mt. 1:6.}.
\VS{7}J'établirai donc mon alliance entre moi et toi, et entre ta postérité après toi, selon leurs générations, ce sera une alliance éternelle en vertu de laquelle je serai ton Dieu et celui de ta postérité après toi.
\VS{8}Je te donnerai, et à ta postérité après toi, le pays où tu demeures comme étranger, à savoir tout le pays de Canaan, en possession perpétuelle, et je serai leur Dieu.
\TextTitle{La circoncision, signe de l'alliance}
\VS{9}Dieu dit encore à Abraham : Tu garderas donc mon alliance, toi et ta postérité après toi, selon leurs générations.
\VS{10}C’est ici mon alliance entre moi et vous, et entre ta postérité après toi, que vous garderez : Tout mâle parmi vous sera circoncis.
\VS{11}Vous circoncirez la chair de votre prépuce ; et cela sera le signe de l'alliance entre moi et vous\FTNT{Ac. 7:8 ; Ro. 4:11.}.
\VS{12}Tout enfant mâle de huit jours sera circoncis parmi vous dans vos générations, tant celui qui est né dans la maison que l'esclave acquis à prix d’argent de tout étranger qui n'est point de ta race\FTNT{Lu. 2:21 ; Lé. 12:3.}.
\VS{13}On ne manquera donc point de circoncire celui qui est né dans ta maison, et celui qui est acquis à prix d’argent, et mon alliance sera dans votre chair pour être une alliance perpétuelle.
\VS{14}Et le mâle incirconcis qui n’aura pas été circoncis dans sa  chair sera retranché du milieu de son peuple parce qu'il aura violé mon alliance.
\TextTitle{Saraï devient Sara ; promesse de la naissance d’Isaac}
\VS{15}Dieu dit aussi à Abraham : Quant à Saraï, ta femme, tu n'appelleras plus son nom Saraï, mais son nom sera Sara.
\VS{16}Je la bénirai, et même je te donnerai un fils d'elle. Je la bénirai et elle deviendra des nations ; des rois, chefs de peuples sortiront d'elle.
\VS{17}Alors Abraham se prosterna la face contre terre, et sourit en disant en son cœur : Naîtrait-il un fils à un homme âgé de cent ans ? Et Sara, âgée de quatre-vingt-dix ans, aurait-elle un enfant ?
\VS{18}Et Abraham dit à Dieu : Je te prie, qu'Ismaël vive devant toi.
\VS{19}Et Dieu dit : Certainement Sara, ta femme, t'enfantera un fils, et tu appelleras son nom Isaac ; et j'établirai mon alliance avec lui pour être une alliance perpétuelle pour sa postérité après lui.
\TextTitle{Une nation sortira d'Ismaël}
\VS{20}Je t'ai aussi exaucé touchant Ismaël : Voici, je le bénirai, et je le ferai croître et multiplier très abondamment. Il engendrera douze princes, et je le ferai devenir une grande nation.
\VS{21}Mais j'établirai mon alliance avec Isaac, que Sara t'enfantera l'année qui vient, en cette même saison.
\VS{22}Et Dieu ayant achevé de parler, s’éleva au-dessus d'Abraham.
\VS{23}Et Abraham prit son fils Ismaël, avec tous ceux qui étaient nés dans sa maison, et tous ceux qu'il avait acquis à prix d’argent, tous les mâles qui étaient des gens de sa maison, et il circoncit la chair de leur prépuce en ce même jour-là, comme Dieu le lui avait dit.
\VS{24}Abraham était âgé de quatre-vingt-dix-neuf ans quand il circoncit la chair de son prépuce ;
\VS{25}et Ismaël, son fils, était âgé de treize ans lorsqu'il fut circoncis.
\VS{26}En ce même jour, Abraham fut circoncis, et son fils Ismaël aussi.
\VS{27}Et tous les gens de sa maison, tant ceux qui étaient nés dans sa maison que ceux qui avaient été acquis à prix d’argent des étrangers, furent circoncis avec lui.
\Chap{18}
\TextTitle{Abraham, ami de Yahweh\FTNTT{Jn 3:29 ; 15:13-15}}
\VerseOne{}Puis Yahweh lui apparut dans les plaines de Mamré, comme il était assis à la porte de sa tente, pendant la chaleur du jour.
\VS{2}Levant ses yeux, il regarda : Et voici, trois hommes parurent devant lui. Quand il les vit, il courut au-devant d'eux depuis la porte de sa tente, et se prosterna à terre\FTNT{Hé. 13:2.} ;
\VS{3}Et il dit : Mon Seigneur, je te prie, si j'ai trouvé grâce devant tes yeux, ne passe point outre, je te prie, et arrête-toi chez ton serviteur.
\VS{4}Qu'on prenne, je vous prie, un peu d'eau, et lavez vos pieds, et reposez-vous sous un arbre.
\VS{5}J’apporterai un morceau de pain pour fortifier votre cœur, après quoi vous passerez outre ; car c'est pour cela que vous êtes venus vers votre serviteur. Et ils dirent : Fais ce que tu as dit.
\VS{6}Abraham donc s'en alla en hâte dans la tente vers Sara, et lui dit : Hâte-toi, prends trois mesures de fleur de farine, pétris-les, et fais des gâteaux.
\VS{7}Puis Abraham courut au troupeau et prit un veau tendre et bon, et le donna à un serviteur qui se hâta de l'apprêter.
\VS{8}Ensuite, il prit du beurre et du lait, et le veau qu'on avait apprêté, et le mit devant eux ; et il se tint auprès d'eux sous l'arbre, et ils mangèrent.
\VS{9}Et ils lui dirent : Où est Sara ta femme ? Et il répondit : La voilà dans la tente.
\VS{10}Et l'un d'entre eux dit : Je ne manquerai pas de revenir vers toi en ce même temps où nous sommes, et voici, Sara, ta femme, aura un fils. Et Sara écoutait à la porte de la tente qui était derrière lui\FTNT{Ro. 9:9.}.
\VS{11}Or Abraham et Sara étaient vieux, fort avancés en âge ; et Sara n'avait plus ce que les femmes sont accoutumées d'avoir\FTNT{Ro. 4:19 ; Hé. 11:11.}.
\VS{12}Et Sara rit en elle-même et dit : Etant vieille, et mon Seigneur étant fort âgé, aurai-je encore des désirs ?
\VS{13}Et Yahweh dit à Abraham : Pourquoi Sara a-t-elle ri en disant : Serait-il vrai que j'aurais un enfant, étant vieille comme je suis ?
\VS{14}Y a-t-il quelque chose qui soit difficile à Yahweh ? Je reviendrai vers toi à cette époque, en ce même temps où nous sommes et Sara aura un fils\FTNT{Mt. 19:26 ; Lu. 1:37.}.
\VS{15}Et Sara le nia en disant : Je n'ai point ri ; car elle avait peur. Mais il dit : Cela n'est pas, car tu as ri.
\VS{16}Et ces hommes se levèrent de là, et regardèrent vers Sodome ; et Abraham alla  avec eux pour les accompagner.
\VS{17}Et Yahweh dit : Cacherai-je à Abraham ce que je vais faire ?
\VS{18}Abraham deviendra certainement une nation grande et puissante, et toutes les nations de la terre seront bénies en lui\FTNT{Ac. 3:25 ; Ga. 3:8.}.
\VS{19}Car je le connais, et je sais qu'il ordonnera à ses enfants, et à sa maison après lui, de garder la voie de Yahweh, pour faire ce qui est juste et droit ; afin que Yahweh fasse venir sur Abraham tout ce qu'il lui a dit.
\VS{20}Et Yahweh dit : Le cri contre Sodome et Gomorrhe s’est accru, et leur péché s’est fort aggravé.
\VS{21}Je descendrai maintenant, et je verrai s'ils ont fait entièrement selon le cri qui est venu jusqu'à moi ; et si cela n'est pas, je le saurai.
\VS{22}Ces hommes donc partant de là allèrent vers Sodome ; mais Abraham se tint encore devant Yahweh.
\TextTitle{Intercession d'Abraham}
\VS{23}Et Abraham s'approcha et dit : Feras-tu périr le juste avec le méchant ?
\VS{24}Peut-être y a-t-il cinquante justes dans la ville, les feras-tu périr aussi ? Ne pardonneras-tu point à la ville à cause des cinquante justes qui sont au milieu d’elle ?
\VS{25}Non, il n'arrivera pas que tu fasses une telle chose, que tu fasses mourir le juste avec le méchant, et que le juste soit traité comme le méchant ! Non, tu ne le feras point. Celui qui juge toute la terre ne fera-t-il point justice\FTNT{Ro. 3:5-6.} ?
\VS{26}Et Yahweh dit : Si je trouve dans Sodome cinquante justes au milieu de la ville, je pardonnerai à toute la ville à cause d'eux.
\VS{27}Et Abraham répondit en disant : Voici, j'ai pris maintenant la hardiesse de parler au Seigneur, moi qui ne suis que poussière et cendres.
\VS{28}Peut-être en manquera-t-il cinq des cinquante justes ; détruiras-tu toute la ville pour ces cinq-là ? Et Yahweh lui répondit : Je ne la détruirai point si j'y trouve quarante-cinq justes.
\VS{29}Abraham continua de lui parler en disant : Peut-être s'y trouvera-t-il quarante ? Et il dit : Je ne la détruirai point pour l'amour des quarante.
\VS{30}Abraham dit : Je prie le Seigneur de ne pas s'irriter si je parle encore. Peut-être s'en trouvera-t-il trente ? Et il dit : Je ne la détruirai point si j'y trouve trente.
\VS{31}Abraham dit : Voici, maintenant j'ai pris la hardiesse de parler au Seigneur : Peut-être s'en trouvera-t-il vingt ? Et il dit : Je ne la détruirai point pour l'amour des vingt.
\VS{32}Abraham dit : Je prie le Seigneur de ne pas s'irriter, je parlerai encore une seule fois : Peut-être s'y trouvera-t-il dix. Et Yahweh dit : Je ne la détruirai point pour l'amour des dix.
\VS{33}Yahweh s'en alla quand il eut achevé de parler avec Abraham. Et Abraham retourna dans sa demeure.
\Chap{19}
\TextTitle{Des anges chez Lot\FTNTT{Ge. 13:10, 12 ; 19:33}}
\VerseOne{}Sur le soir, les deux anges arrivèrent à Sodome, et Lot était assis à la porte de Sodome. Quand Lot les vit, il se leva pour aller au-devant d'eux, et se prosterna la face contre terre.
\VS{2}Et il leur dit : Voici, je vous prie, mes seigneurs, entrez maintenant dans la maison de votre serviteur, et passez-y la nuit ; lavez-vous les pieds ; puis vous vous lèverez dès le matin et continuerez votre chemin ; et ils dirent : Non, mais nous passerons la nuit dans la rue.
\VS{3}Mais il les pressa tellement qu'ils se retirèrent chez lui ; et quand ils furent entrés dans sa maison, il leur fit un festin, et fit cuire des pains sans levain, et ils mangèrent.
\VS{4}Ils n’étaient pas encore couchés que les hommes de la ville, les hommes de Sodome, environnèrent la maison, depuis les plus jeunes jusqu'aux vieillards, tout le peuple était ensemble.
\VS{5}Ils appelèrent Lot et ils lui dirent : Où sont les hommes qui sont venus cette nuit chez toi ? Fais-les sortir afin que nous les connaissions.
\VS{6}Mais Lot sortit de sa maison pour leur parler à la porte, et ayant fermé la porte après lui,
\VS{7}il leur dit : Je vous prie, mes frères, ne leur faites point de mal.
\VS{8}Voici, j'ai deux filles qui n'ont point encore connu d'homme ; je vous les amènerai et vous les traiterez comme il vous plaira. Seulement, ne faites pas de mal à ces hommes, car ils sont venus à l'ombre de mon toit.
\VS{9}Ils lui dirent : Retire-toi de là. Ils dirent aussi : Cet homme seul est venu pour habiter ici comme étranger, et il veut nous gouverner ? Maintenant nous te ferons pis qu'à eux. Et faisant violence à Lot,  ils s'approchèrent pour briser la porte\FTNT{2 Pi. 2:7-8.}.
\VS{10}Mais les hommes étendirent leurs mains, firent rentrer Lot vers eux dans la maison, et fermèrent la porte.
\VS{11}Et ils frappèrent d’aveuglement les hommes qui étaient à la porte de la maison, depuis le plus petit jusqu'au plus grand, de sorte qu'ils se lassèrent à chercher la porte.
\VS{12}Alors ces hommes dirent à Lot : Qui as-tu encore ici qui t'appartienne ? Gendres, fils et filles, et tout ce qui t'appartient dans la ville, fais-les sortir de ce lieu.
\VS{13}Car nous allons détruire ce lieu parce que le cri contre ses habitants est grand devant Yahweh. Yahweh nous a envoyés pour le détruire.
\VS{14}Lot sortit donc et parla à ses gendres, qui devaient prendre ses filles, et leur dit : Levez-vous, sortez de ce lieu, car Yahweh va détruire la ville. Mais aux yeux de ses gendres, il parut plaisanter.
\TextTitle{Jugement sur Sodome}
\VS{15}Dès l’aube du jour, les anges pressèrent Lot en disant : Lève-toi, prends ta femme et tes deux filles qui se trouvent ici, de peur que tu ne périsses dans le châtiment de la ville.
\VS{16}Et comme il tardait, ces hommes le prirent par la main, et ils prirent aussi par la main sa femme et ses deux filles, parce que Yahweh voulait l'épargner ; et ils l'emmenèrent et le mirent hors de la ville.
\VS{17}Après les avoir fait sortir, l'un d’eux dit : Sauve ta vie, ne regarde point derrière toi, et ne t'arrête en aucun endroit de la plaine ; sauve-toi sur la montagne, de peur que tu ne périsses.
\VS{18}Lot leur répondit : Non, Seigneur, je te prie.
\VS{19}Voici, ton serviteur a maintenant trouvé grâce devant toi, et tu as montré la grandeur de ta bonté à mon égard en préservant ma vie, mais je ne pourrai pas me sauver vers la montagne avant que le mal ne m'atteigne, et je mourrai.
\VS{20}Voici, je te prie, cette ville-là est proche ; je puis m'y enfuir, et elle est petite. Je te prie, que je m'y sauve ; n'est-elle pas petite ? Et mon âme vivra.
\VS{21}Et il lui dit : Voici, je t'ai exaucé encore en cela, de ne point détruire la ville dont tu as parlé.
\VS{22}Hâte-toi, sauve-toi là, car je ne pourrai rien faire jusqu'à ce que tu y sois entré ; c'est pourquoi cette ville fut appelée Tsoar.
\VS{23}Comme le soleil se levait sur la terre, Lot entra dans Tsoar.
\VS{24}Alors Yahweh fit pleuvoir du ciel, sur Sodome et sur Gomorrhe, du soufre et du feu, de la part de Yahweh\FTNT{De. 29:23 ; Lu. 17:29 ; Jud. 1:7.} ;
\VS{25}et il détruisit ces villes-là, et toute la plaine, et tous les habitants des villes, et les herbes de la terre.
\VS{26}Mais la femme de Lot regarda en arrière, et elle devint une statue de sel\FTNT{Lu. 17:31-33.}.
\VS{27}Abraham se leva de bon matin et vint au lieu où il s'était tenu devant Yahweh ;
\VS{28}et regardant vers Sodome et Gomorrhe, et vers toute la terre de cette plaine-là, il vit monter de la terre une fumée comme la fumée d'une fournaise.
\VS{29}Lorsque Dieu détruisit les villes de la plaine, il se souvint d'Abraham, et laissa Lot s’en aller  du milieu du désastre par lequel il détruisit les villes où Lot avait établi sa demeure.
\TextTitle{Une abomination commise dans la famille de Lot\FTNTT{Ge. 13:10,12 ; 19:1 ; Lu. 22:31-62}}
\VS{30}Lot quitta Tsoar et habita sur la montagne avec ses deux filles, car il craignait de demeurer dans Tsoar, et il se retira dans une caverne avec ses deux filles.
\VS{31}L'aînée dit à la plus jeune : Notre père est vieux, et il n'y a personne sur la terre pour venir vers nous, selon la coutume de tous les pays.
\VS{32}Viens, donnons du vin à notre père, et couchons avec lui  afin que nous conservions la race de notre père.
\VS{33}Elles donnèrent donc du vin à boire à leur père cette nuit-là ; et l'aînée vint, et coucha avec son père, mais il ne s'aperçut point ni quand elle se coucha ni quand elle se leva.
\VS{34}Le lendemain, l'aînée dit à la plus jeune : Voici, j'ai couché la nuit dernière avec mon père, donnons-lui encore du vin à boire cette nuit, puis va et couche avec lui, et nous conserverons la race de notre père.
\VS{35}Elles firent boire du vin à leur père encore cette nuit-là ; et la plus jeune se leva et coucha avec lui ; mais il ne s'aperçut point ni quand elle se coucha ni quand elle se leva.
\VS{36}Ainsi, les deux filles de Lot conçurent de leur père.
\VS{37}L’aînée enfanta un fils qu’elle appela du nom de Moab ; c'est le père des Moabites jusqu'à ce jour.
\VS{38}La plus jeune aussi enfanta un fils qu’elle appela du nom de Ben-Ammi ; c'est le père des Ammonites jusqu'à ce jour.
\Chap{20}
\TextTitle{Faute d'Abraham à Guérar\FTNTT{Ge. 26:6-32}}
\VerseOne{}Abraham s'en alla de là pour le pays du midi ; il demeura entre Kadès et Schur, et il habita comme étranger à Guérar.
\VS{2}Abraham disait de Sara sa femme : C'est ma sœur. Et Abimélec, roi de Guérar, envoya des gens prendre Sara.
\VS{3}Mais Dieu apparut la nuit dans un songe à Abimélec, et lui dit : Voici, tu vas mourir, à cause de la femme que tu as prise, car elle a un mari.
\VS{4}Abimélec, qui ne s'était point approché d'elle, répondit : Seigneur, feras-tu donc mourir une nation juste ?
\VS{5}Ne m'a-t-il pas dit : C'est ma sœur? Et elle-même aussi n'a-t-elle pas dit : C'est mon frère ? J'ai fait ceci dans l'intégrité de mon cœur et dans la pureté de mes mains.
\VS{6}Dieu lui dit en songe : Je sais que tu l'as fait dans l'intégrité de ton cœur, aussi ai-je empêché que tu ne pèches contre moi ; c'est pourquoi je n'ai pas permis que tu la touches.
\VS{7}Maintenant donc rends la femme de cet homme, car il est prophète ; et il priera pour toi et tu vivras. Mais si tu ne la rends pas, sache que tu mourras toi et tout ce qui est à toi.
\VS{8}Abimélec se leva de bon matin, appela tous ses serviteurs, et rapporta à leurs oreilles  toutes ces choses, et ils furent saisis de crainte.
\VS{9}Puis Abimélec appela Abraham et lui dit : Que nous as-tu fait ? Et en quoi t'ai-je offensé que tu aies fait venir sur moi et sur mon royaume un grand péché ? Tu m'as fait des choses qui ne doivent point se faire.
\VS{10}Abimélec dit aussi à Abraham : Qu'as-tu vu qui t'aie obligé de faire cela ?
\VS{11}Abraham répondit : C'est parce que je disais : Assurément, il n'y a point de crainte de Dieu dans ce pays, et ils me tueront à cause de ma femme.
\VS{12}De plus, il est vrai qu’elle est ma soeur, fille de mon père ; mais elle n'est pas fille de ma mère ; et elle m'a été donnée pour femme.
\VS{13}Lorsque Dieu me fit errer loin de la maison de mon père, je dis à Sara : Voici la grâce que tu me feras, dis de moi dans tous les lieux où nous irons : C'est mon frère.
\VS{14}Alors Abimélec prit des brebis, des bœufs, des serviteurs et des servantes, et les donna à Abraham, et lui rendit Sara, sa femme.
\VS{15}Abimélec lui dit : Voici, mon pays est à ta disposition, demeure où il te plaira.
\VS{16}Et il dit à Sara : Voici, je donne à ton frère mille pièces d'argent ; cela te sera un voile sur les yeux  pour tous ceux qui sont avec toi, et envers tous les autres ; et ainsi elle fut reprise.
\VS{17}Abraham pria Dieu, et Dieu guérit Abimélec, sa femme, et ses servantes ; et elles eurent des enfants.
\VS{18}Car Yahweh avait frappé de stérilité en  fermant toute matrice de la maison d'Abimélec, à cause de Sara, femme d'Abraham.
\Chap{21}
\TextTitle{Naissance d'Isaac}
\VerseOne{}Et Yahweh visita Sara, comme il avait dit ; et il agit selon ses paroles.
\VS{2}Sara donc conçut, et enfanta un fils à Abraham dans sa vieillesse, au temps précis que Dieu lui avait dit.
\VS{3}Abraham donna le nom d’Isaac au fils qui lui était né, que Sara lui avait enfanté.
\VS{4}Abraham circoncit son fils Isaac âgé de huit jours, comme Dieu le lui avait ordonné.
\VS{5}Abraham était âgé de cent ans quand Isaac, son fils, lui naquit.
\VS{6}Et Sara dit : Dieu m'a donné de quoi rire ; tous ceux qui l'apprendront riront avec moi.
\VS{7}Elle dit aussi : Qui aurait dit à Abraham que Sara allaiterait des enfants ? Car je lui ai enfanté un fils dans sa vieillesse.
\VS{8}L'enfant grandit et fut sevré ; et Abraham fit un grand festin le jour où Isaac fut sevré.
\TextTitle{Abraham chasse Agar avec Ismaël\FTNTT{Ga. 4:21-31}}
\VS{9}Sara vit rire le fils qu’Agar, l’Egyptienne, avait enfanté à Abraham ;
\VS{10}et elle dit à Abraham : Chasse cette servante et son fils, car le fils de cette servante n'héritera point avec mon fils, avec Isaac\FTNT{Ga. 4:30.}.
\VS{11}Cette parole déplut fort à Abraham à cause de son fils.
\VS{12}Mais Dieu dit à Abraham : N'aie point de chagrin au sujet de l'enfant ni de ta servante ;  écoute la parole de Sara dans toutes les choses qu’elle te dira, car en Isaac te sera donnée une postérité.
\VS{13}Je ferai aussi devenir le fils de la servante une nation, parce qu'il est ta semence.
\VS{14}Puis Abraham se leva de bon matin et prit du pain et une outre d'eau, et il les donna à Agar en les mettant sur son épaule. Il lui donna aussi l'enfant et la renvoya. Elle se mit en chemin et fut errante au désert de Beer-Schéba.
\VS{15}Quand l'eau de l’outre fut épuisée, elle jeta l'enfant sous un arbrisseau,
\VS{16}et elle alla s’asseoir vis-à-vis, à une portée d’arc, car elle dit : Que je ne voie pas mourir mon enfant. Elle s’assit donc vis-à-vis de lui, éleva la voix et pleura.
\VS{17}Dieu entendit la voix de l'enfant, et l'Ange de Dieu appela des cieux Agar et lui dit : Qu'as-tu Agar ? Ne crains point, car Dieu a entendu la voix de l'enfant du lieu où il est.
\VS{18}Lève-toi, lève l'enfant, et prends-le par la main, car je le ferai devenir une grande nation.
\VS{19}Et Dieu lui ouvrit les yeux et elle vit un puits d'eau ; elle alla remplir d'eau l’outre, et donna à boire à l'enfant.
\VS{20}Dieu fut avec l'enfant, qui devint grand, et demeura dans le désert ; et il fut tireur d'arc.
\VS{21}Il habita dans le désert de Paran ; et sa mère lui prit une femme du pays d'Egypte.
\TextTitle{Abraham à Beer-Schéba}
\VS{22}Et il arriva en ce temps-là qu'Abimélec, et Picol, chef de son armée, parla à Abraham en disant : Dieu est avec toi dans toutes les choses que tu fais.
\VS{23}Maintenant donc jure-moi ici par le nom de Dieu que tu ne me mentiras point, ni à mes enfants ni aux enfants de mes enfants, et que selon la faveur que je t'ai faite, tu agiras envers moi et envers le pays où tu séjournes comme étranger.
\VS{24}Abraham répondit : Je te le jurerai.
\VS{25}Mais Abraham fit des reproches à Abimélec au sujet d'un puits d'eau, dont les serviteurs d'Abimélec s'étaient emparés de force.
\VS{26}Abimélec répondit : J’ignore qui a fait cela, et aussi tu ne m'en as point informé, et moi, je ne l’apprends qu’aujourd’hui.
\VS{27}Alors Abraham prit des brebis et des bœufs, et les donna à Abimélec, et ils firent alliance ensemble.
\VS{28}Abraham mit à part sept jeunes brebis de son troupeau.
\VS{29}Et Abimélec dit à Abraham : Que veulent dire ces sept jeunes brebis que tu as mises à part ?
\VS{30}Il répondit : C'est que tu prendras ces sept jeunes brebis de ma main pour me servir de témoignage que j'ai creusé ce puits.
\VS{31}C'est pourquoi on appela ce lieu-là Beer-Schéba, car tous deux y jurèrent.
\VS{32}Ils traitèrent donc alliance à Beer-Schéba, puis Abimélec se leva avec Picol, chef de son armée, et ils retournèrent au pays des Philistins.
\VS{33}Abraham planta des tamaris à Beer-Schéba ; et là il invoqua le nom de Yahweh, le Dieu de l’éternité.
\VS{34}Abraham séjourna beaucoup de jours comme étranger dans le pays des Philistins.
\Chap{22}
\TextTitle{Abraham présente Isaac en sacrifice\FTNTT{Hé. 11:17-19}}
\VerseOne{}Or, il arriva après ces choses, que Dieu éprouva Abraham et lui dit : Abraham ! Et il répondit : Me voici.
\VS{2}Et Dieu lui dit : Prends maintenant ton fils, ton unique, celui que tu aimes, Isaac, et va-t'en au pays de Morija, et là offre-le en holocauste sur l'une des montagnes que je te dirai.
\VS{3}Abraham donc s'étant levé de bon matin, sella son âne, et prit deux de ses serviteurs avec lui, et Isaac son fils ; et ayant fendu le bois pour l'holocauste, il se mit en chemin et s'en alla au lieu que Dieu lui avait dit.
\VS{4}Le troisième jour, Abraham levant ses yeux, vit le lieu de loin.
\VS{5}Et Abraham dit à ses serviteurs : Restez ici avec l'âne ; moi et l'enfant nous irons jusque-là pour adorer, après quoi nous reviendrons auprès de vous.
\VS{6}Abraham prit le bois de l'holocauste et le mit sur Isaac, son fils, et prit le feu dans sa main, et un couteau ; et ils s'en allèrent tous deux ensemble.
\VS{7}Alors Isaac parla à Abraham, son père, et dit : Mon père ! Abraham répondit : Me voici mon fils. Et il dit : Voici le feu et le bois, mais où est l’agneau pour l'holocauste\FTNT{Isaac est un autre type de Christ qui s’offre en sacrifice pour l’expiation de nos péchés. La réponse à sa question au v. 7:«~Voici le feu et le bois, mais où est l’agneau pour l’holocauste ?~», a été apportée bien des siècles plus tard par Jean-Baptiste:«~Voici l’agneau de Dieu, qui ôte le péché du monde~». (Jn. 1:29).} ?
\VS{8}Abraham répondit : Mon fils, Dieu se pourvoira lui-même de l’agneau pour l'holocauste. Et ils marchèrent tous deux ensemble.
\VS{9}Et étant arrivés au lieu que Dieu lui avait dit, Abraham bâtit là un autel, et rangea le bois, et ensuite il lia Isaac, son fils, et le mit sur l'autel, par-dessus le bois\FTNT{Ja. 2:21.}.
\VS{10}Puis Abraham étendit sa main et prit le couteau pour égorger son fils.
\VS{11}Mais l'Ange de Yahweh l’appela des cieux et dit : Abraham, Abraham ! Il répondit : Me voici.
\VS{12}L’Ange lui dit : Ne porte pas ta main sur l'enfant, et ne lui fais rien ; car maintenant je sais que tu crains Dieu, puisque tu ne m’as point refusé ton fils, ton unique.
\VS{13}Abraham leva les yeux et regarda ; et voici,  il vit derrière lui un bélier qui était retenu à un buisson par ses cornes ; et Abraham alla prendre le bélier et l'offrit en holocauste à la place de son fils.
\VS{14}Abraham donna à ce lieu le nom de Yahweh-Jiré (Yahweh pourvoira) ; c'est pourquoi on dit aujourd'hui : Dans la montagne de Yahweh il y sera pourvu.
\VS{15}L'Ange de Yahweh appela des cieux Abraham pour la seconde fois,
\VS{16}et dit : Je le jure par moi-même\FTNT{Hé. 6:13-15.}, parole de Yahweh ! Parce que tu as fait cela, et que tu n'as point refusé ton fils, ton unique,
\VS{17}certainement je te bénirai, et je multiplierai très abondamment ta postérité, comme les étoiles du ciel et comme le sable qui est sur le bord de la mer ; et ta postérité possédera la porte de ses ennemis.
\VS{18}Toutes les nations de la terre seront bénies en ta postérité, parce que tu as obéi à ma voix.
\VS{19}Ainsi Abraham retourna vers ses serviteurs, et ils se levèrent et s'en allèrent ensemble à Beer-Schéba ; car Abraham demeurait à Beer-Schéba.
\VS{20}Après ces choses, quelqu'un apporta des nouvelles à Abraham, en disant : Voici, Milca a aussi enfanté des fils à Nachor, ton frère.
\VS{21}Uts, son premier-né, et Buz, son frère, Kemuel, père d'Aram,
\VS{22}Késed, Hazo, Pildasch, Jidlaph et Bethuel.
\VS{23}Bethuel a engendré Rebecca. Milca enfanta ces huit fils à Nachor, frère d'Abraham.
\VS{24}Sa concubine, nommée Réuma, enfanta aussi Thébach, Gaham, Tahasch, et Maaca.
\Chap{23}
\TextTitle{Mort de Sara}
\VerseOne{}Or, Sara vécut cent vingt-sept ans ; ce sont là les années de la vie de Sara.
\VS{2}Sara mourut à Kirjath-Arba, qui est Hébron, dans le pays de Canaan ; et Abraham vint pour mener deuil sur Sara et pour la pleurer.
\VS{3}Et Abraham se leva de devant son mort, il parla aux fils de Heth, en disant :
\VS{4}Je suis étranger et habitant parmi vous ; donnez-moi une possession de sépulcre parmi vous, afin que j'enterre mon mort et que je l'ôte de devant moi\FTNT{Ac. 7:5.}.
\VS{5}Les fils de Heth répondirent à Abraham et lui dirent :
\VS{6}Mon seigneur, écoute-nous ! Tu es un prince de Dieu parmi nous, enterre ton mort dans le plus distingué de nos sépulcres ; nul de nous ne te refusera son sépulcre afin que tu y enterres ton mort.
\VS{7}Alors Abraham se leva et se prosterna devant le peuple du pays, devant les Héthiens.
\VS{8}Et il leur parla et dit : S'il vous plaît que j'enterre mon mort et que je l'ôte de devant moi ; écoutez-moi, et intercédez pour moi envers Ephron, fils de Tsochar,
\VS{9}afin qu'il me cède sa caverne de Macpéla, qui est à l’extrémité de son champ ; qu'il me la cède contre sa valeur en argent, afin qu’elle me serve de possession sépulcrale au milieu de vous.
\VS{10}Ephron était assis parmi les fils de Heth. Et Ephron, l’Héthien, répondit à Abraham, en présence des fils de Heth qui l'écoutaient, devant tous ceux qui entraient par la porte de sa ville, et dit :
\VS{11}Non, mon seigneur, écoute-moi ! Je te donne le champ, je te donne aussi la caverne qui y est, je te la donne en présence des enfants de mon peuple ; enterres-y ton mort.
\VS{12}Abraham se prosterna devant le peuple du pays.
\VS{13}Et il parla ainsi à Ephron, en présence de tout le peuple du pays qui écoutait et dit : S'il te plaît, je te prie, écoute-moi ! Je donnerai l'argent du champ ; reçois-le de moi, et j'y enterrerai mon mort.
\VS{14}Et Ephron répondit à Abraham, en disant :
\VS{15}Mon seigneur, écoute-moi ! La terre vaut quatre cents sicles d'argent, qu’est-ce que cela entre moi et toi ? Enterre donc ton mort.
\VS{16}Abraham ayant entendu Ephron, lui paya l'argent dont il avait parlé, en présence des fils de Heth, à savoir quatre cents sicles d'argent ayant cours chez les marchands\FTNT{Ac. 7:16.}.
\VS{17}Le champ d'Ephron, qui était à Macpéla, vis-à-vis de Mamré, le champ et la caverne qui y est, et tous les arbres qui sont dans le champ et dans toutes ses limites alentour,
\VS{18}tout fut acquis comme propriété d’Abraham, en présence des fils de Heth, et de tous ceux qui entraient par la porte de la ville.
\VS{19}Après cela, Abraham enterra Sara, sa femme, dans la caverne du champ de Macpéla, vis-à-vis de Mamré, qui est Hébron, dans le pays de Canaan.
\VS{20}Le champ et la caverne qui y est demeurèrent à Abraham comme possession sépulcrale, acquise des fils de Heth.
\Chap{24}
\TextTitle{Abraham recherche une épouse pour Isaac}
\VerseOne{}Or, Abraham devint vieux et fort avancé en âge ; et Yahweh avait béni Abraham en toute chose.
\VS{2}Abraham dit à son serviteur, le plus ancien des serviteurs de sa maison, l’intendant de tout ce qui lui appartenait : Mets, je te prie, ta main sous ma cuisse ;
\VS{3}et je te ferai jurer par Yahweh, le Dieu du ciel et le Dieu de la terre, que tu ne prendras point de femme pour mon fils parmi les filles des Cananéens, au milieu desquels j'habite.
\VS{4}Mais tu iras dans mon pays et vers mes parents, et tu y prendras une femme pour mon fils Isaac.
\VS{5}Le serviteur lui répondit : Peut-être que la femme ne voudra-t-elle pas me suivre dans ce pays ; me faudra-t-il nécessairement ramener ton fils dans le pays d'où tu es sorti ?
\VS{6}Abraham lui dit : Garde-toi bien d'y ramener mon fils !
\VS{7}Yahweh, le Dieu du ciel, qui m'a fait sortir de la maison de mon père et de ma patrie, qui m'a parlé et qui m’a juré en disant : Je donnerai ce pays à ta postérité, enverra lui-même son ange devant toi ; et c’est là que tu prendras une femme pour mon fils.
\VS{8}Si la femme ne veut pas te suivre, tu seras quitte de ce serment que je te fais faire. Quoi qu'il en soit, tu n’y ramèneras point mon fils.
\VS{9}Le serviteur mit la main sous la cuisse d'Abraham, son Seigneur, et lui jura d’observer ces choses.
\VS{10}Alors le serviteur prit dix chameaux parmi les chameaux de son maître, et s'en alla, ayant à sa disposition tous les biens. Il partit donc et s'en alla en Mésopotamie, à la ville de Nachor.
\VS{11}Il fit reposer les chameaux sur leurs genoux hors de la ville, près d'un puits d'eau, sur le soir, au temps où sortent celles qui vont puiser de l'eau.
\VS{12}Et il dit : Ô Yahweh, Dieu de mon seigneur Abraham, fais que j'aie une heureuse rencontre aujourd'hui, et sois favorable à mon seigneur Abraham.
\VS{13}Voici, je me tiens près de la source d'eau, et les filles des gens de la ville vont sortir pour puiser de l'eau.
\VS{14}Fais donc que la jeune fille à laquelle je dirai : Penche ta cruche, je te prie, afin que je boive, et qui me répondra : Bois, et je donnerai aussi à boire à tes chameaux, soit celle que tu as destinée à ton serviteur Isaac, et par là je connaîtrai que tu es favorable à mon seigneur.
\VS{15}Il n’avait pas encore fini de parler que sortit sa cruche sur l’épaule, Rebecca, fille de Bethuel, fils de Milca, femme de Nachor, frère d'Abraham.
\VS{16}Et la jeune fille était très belle de figure ; elle était vierge, et aucun homme ne l'avait connue. Elle descendit donc à la source, et comme elle remontait après avoir rempli sa cruche,
\VS{17}le serviteur courut au-devant d'elle et lui dit : Laisse-moi boire, je te prie, un peu d’eau de ta cruche.
\VS{18}Elle répondit : Mon seigneur, bois. Elle s’empressa d’abaisser sa cruche sur sa main,  et elle lui donna à boire.
\VS{19}Quand elle eut achevé de lui donner à boire, elle dit : Je puiserai aussi pour tes chameaux jusqu'à ce qu'ils aient achevé de boire.
\VS{20}Et elle s’empressa de vider sa cruche dans l’abreuvoir ; elle courut encore au puits pour puiser de l'eau, et elle puisa pour tous ses chameaux.
\VS{21}L’homme la regardait avec étonnement et sans rien dire, pour voir si Yahweh faisait réussir son voyage ou non.
\VS{22}Quand les chameaux eurent fini de boire, l’homme prit un anneau d'or, du poids d'un demi-sicle, et deux bracelets, pour les mettre sur les mains de cette fille, pesant dix sicles d'or.
\VS{23}Et il lui dit : De qui es-tu fille ? Je te prie, fais-le-moi savoir. Y a-t-il dans la maison de ton père de la place pour nous loger ?
\VS{24}Elle lui répondit : Je suis fille de Bethuel, fils de Milca et de Nachor.
\VS{25}Elle lui dit encore : Il y a chez nous de la paille et du fourrage en abondance, et de la place pour loger.
\VS{26}Alors l’homme s'inclina et adora Yahweh,
\VS{27}et dit : Béni soit Yahweh, le Dieu de mon seigneur Abraham, qui n'a point cessé d'exercer sa bonté et sa fidélité envers mon Seigneur ! Lorsque j'étais en chemin, Yahweh m'a conduit dans la maison des frères de mon seigneur.
\VS{28}La jeune fille courut et rapporta toutes ces choses à la maison de sa mère.
\VS{29}Rebecca avait un frère nommé Laban, qui courut dehors vers l’homme près de la source.
\VS{30}Il avait vu l’anneau et les bracelets aux mains de sa sœur, et il avait entendu les paroles de Rebecca sa sœur, disant : Ainsi m’a parlé l’homme. Il vint donc à cet homme qui se tenait auprès des chameaux, près de la source,
\VS{31}et il lui dit : Entre, béni de Yahweh ! Pourquoi te tiens-tu dehors ? J'ai préparé la maison et une place pour tes chameaux.
\VS{32}L'homme donc entra dans la maison. Laban fit décharger les chameaux, et il donna de la paille et du fourrage  aux chameaux ; et il apporta de l'eau pour laver les pieds de l’homme et les pieds de ceux qui étaient avec lui.
\VS{33}Et il lui présenta à manger. Mais il dit : Je ne mangerai point avant d’avoir dit  ce que j'ai à dire. Parle ! dit Laban.
\VS{34}Alors il dit : Je suis serviteur d'Abraham.
\VS{35}Yahweh a comblé de bénédictions mon seigneur qui est devenu puissant. Il lui a donné des brebis, des bœufs, de l'argent, de l'or, des serviteurs, des servantes, des chameaux, et des ânes.
\VS{36}Sara, la femme de mon seigneur, a enfanté dans sa vieillesse un fils à mon seigneur ; et il lui a donné tout ce qu'il possède.
\VS{37}Mon seigneur m'a fait jurer en disant : Tu ne prendras point de femme pour mon fils parmi les filles des Cananéens dans le pays desquels j’habite ;
\VS{38}mais tu iras dans la maison de mon père et de ma famille prendre une femme pour mon fils.
\VS{39}J’ai dit à mon seigneur : Peut-être que la femme ne voudra-t-elle pas me suivre.
\VS{40}Et il m’a répondu : Yahweh, devant la face de qui j'ai marché, enverra son ange avec toi, et fera réussir ton voyage ; et tu prendras pour mon fils une femme de ma famille et de la maison de mon père.
\VS{41}Quand tu auras été vers ma famille, tu seras alors dégagé de la punition du serment que je te fais faire ; et si on ne te la donne pas, tu seras dégagé de la punition du serment que je te fais faire.
\VS{42}Je suis arrivé aujourd'hui à la source et j'ai dit : Ô Yahweh ! Dieu de mon seigneur Abraham, si tu daignes faire réussir le voyage que j'ai entrepris,
\VS{43}voici, je me tiendrai près de la source d'eau, et la jeune fille qui sortira pour puiser à qui je dirai : Laisse-moi boire, je te prie, un peu d’eau de ta cruche ; et qui me répondra :
\VS{44}Bois toi-même, et je puiserai aussi pour tes chameaux, que cette jeune fille soit la femme que Yahweh a destinée au fils de mon seigneur.
\VS{45}Avant que j’ai fini de parler en mon cœur, voici, Rebecca est sortie, ayant sa cruche sur son épaule ; elle est descendue à la source et a puisé de l'eau ; et je lui ai dit : Donne-moi, à boire, je te prie.
\VS{46}Elle s’est empressée d’abaisser sa cruche de dessus son épaule et m'a dit : Bois, et même je donnerai à boire à tes chameaux. J'ai donc bu, et elle a aussi donné à boire aux chameaux.
\VS{47}Puis je l'ai interrogée en disant : De qui es-tu fille ? Elle a répondu : Je suis fille de Bethuel, fils de Nachor et de Milca. Alors je lui ai mis un anneau à son nez et les bracelets à ses mains.
\VS{48}Puis je me suis incliné, j’ai adoré Yahweh, et j'ai béni Yahweh, le Dieu de mon seigneur Abraham, qui m'a conduit fidèlement, afin que je prenne la fille du frère de mon seigneur pour son fils.
\VS{49}Maintenant donc, si vous voulez user de bonté et de fidélité envers mon seigneur, déclarez-le-moi ; sinon, déclarez-le-moi aussi ; et je me tournerai à droite ou à gauche.
\VS{50}Laban et Bethuel répondirent et dirent : Cette affaire vient de Yahweh, nous ne pouvons te parler ni en bien ni en mal.
\VS{51}Voici Rebecca est devant toi, prends-la et va, et qu'elle soit la femme du fils de ton seigneur, comme Yahweh l’a dit.
\VS{52}Lorsque le serviteur d'Abraham eut entendu leurs paroles, il se prosterna à terre devant Yahweh.
\VS{53}Et le serviteur sortit des objets d'argent et d'or, et des vêtements, et les donna à Rebecca. Il donna aussi de riches présents à son frère et à sa mère.
\VS{54}Puis ils mangèrent et burent, lui et les gens qui étaient avec lui, et ils passèrent la nuit.  Le matin,  quand ils furent levés, le serviteur dit : Laissez-moi retourner vers mon seigneur.
\VS{55}Le frère et la mère lui dirent : Que la jeune fille reste avec nous quelques jours encore, une dizaine de jours ;  après quoi, elle s'en ira.
\VS{56}Il leur répondit : Ne me retardez pas puisque Yahweh a fait réussir mon voyage ; laissez-moi partir  afin que je m'en aille vers mon seigneur.
\VS{57}Alors ils dirent : Appelons la jeune fille et demandons-lui son avis.
\VS{58}Ils appelèrent donc Rebecca et lui dirent : Veux-tu aller avec cet homme ? Et elle répondit : J'irai.
\VS{59}Ainsi ils laissèrent partir Rebecca, leur sœur, et sa nourrice, avec le serviteur d'Abraham et ses gens.
\VS{60}Ils bénirent Rebecca et lui dirent : Tu es notre sœur, puisses-tu devenir des milliers de myriades, et que ta postérité possède la porte de ses ennemis !
\VS{61}Alors Rebecca se leva avec ses servantes, et elles montèrent sur les chameaux et suivirent l’homme. Et le serviteur prit Rebecca et s'en alla.
\VS{62}Or Isaac revenait du puits de Lachaï-roï, et il habitait dans le pays du midi.
\VS{63}Un soir qu’Isaac était sorti dans les champs pour prier, il leva les yeux et regarda, et voici, des chameaux arrivaient.
\VS{64}Rebecca leva aussi les yeux, vit Isaac, et descendit de son chameau ;
\VS{65}car elle avait dit au serviteur : Qui est cet homme qui marche dans les champs à notre rencontre ? Et le serviteur avait répondu : C'est mon seigneur ; et elle prit son voile et se couvrit.
\VS{66}Le serviteur raconta à Isaac toutes les choses qu'il avait faites.
\VS{67}Alors Isaac conduisit Rebecca dans la tente de Sara, sa mère ; il prit Rebecca pour sa femme\FTNT{Pr. 18:22 ; Pr. 31:10-31.} et l'aima. Ainsi Isaac fut consolé après la mort de sa mère.
\Chap{25}
\TextTitle{Ketura, femme d'Abraham}
\VerseOne{}Or, Abraham prit une autre femme nommée Ketura.
\VS{2}Elle lui enfanta Zimram, Jokschan, Medan, Madian, Jischbak, et Schuach.
\VS{3}Jokschan engendra Séba et Dedan. Les fils de Dedan furent Aschurim, Letuschim et Leummim.
\VS{4}Les fils de Madian furent Epha, Epher, Hénoc, Abida, Eldaa. Ce sont là tous les fils de Ketura.
\TextTitle{Isaac hérite d'Abraham\FTNTT{Hé. 1:2}}
\VS{5}Abraham donna tout ce qui lui appartenait à Isaac.
\VS{6}Mais il fit des dons aux fils de ses concubines, et tandis qu’il vivait encore, il les envoya loin de son fils Isaac, du côté de l'orient, dans le pays d’orient.
\TextTitle{Mort d'Abraham}
\VS{7}Voici les jours des années de la vie d’Abraham : Il vécut cent soixante-quinze ans.
\VS{8}Abraham expira et mourut après une heureuse vieillesse, fort âgé et rassasié de jours, et il fut recueilli auprès de son peuple.
\VS{9}Isaac et Ismaël, ses fils, l'enterrèrent dans la caverne de Macpéla, dans le champ d'Ephron, fils de Tschoar, le Héthien, qui est vis-à-vis de Mamré.
\VS{10}C’est le champ qu'Abraham avait acheté des fils de Heth. Là furent enterrés Abraham et Sara, sa femme.
\VS{11}Après la mort d'Abraham, Dieu bénit Isaac son fils.  Isaac habitait près du puits de Lachaï-roï.
\TextTitle{Postérité d'Ismaël}
\VS{12}Voici la postérité d'Ismaël, fils d'Abraham, qu'Agar l’Egyptienne, servante de Sara, avait enfanté à Abraham.
\VS{13}Voici les noms des fils d'Ismaël, par leurs noms, selon leurs générations. Le premier-né d'Ismaël fut Nebajoth, puis Kédar, Adbeel, Mibsam,
\VS{14}Mischma, Duma, Massa,
\VS{15}Hadad, Théma, Jethur, Naphisch, et Kedma.
\VS{16}Ce sont là les fils d'Ismaël, et ce sont là leurs noms, selon leurs parcs, et selon leurs enclos ; douze princes de leurs peuples.
\VS{17}Et voici les années de la vie d'Ismaël : Cent trente-sept ans. Il expira et mourut, et il fut recueilli auprès de son peuple.
\VS{18}Ses descendants habitèrent depuis Havila jusqu'à Schur, qui est vis-à-vis de l'Egypte, en allant vers l'Assyrie. Et le pays qui était échu à Ismaël était à la vue de tous ses frères.
\TextTitle{Postérité d'Isaac}
\VS{19}Voici la postérité d'Isaac, fils d'Abraham.
\VS{20}Abraham engendra Isaac. Isaac était âgé de quarante ans quand il épousa Rebecca, fille de Bethuel, le Syrien, de Paddan-Aram, sœur de Laban, le Syrien.
\VS{21}Isaac pria instamment Yahweh au sujet de sa femme parce qu'elle était stérile ; et Yahweh exauça ses prières ; et Rebecca, sa femme, conçut.
\VS{22}Mais les enfants se heurtaient dans son ventre, et elle dit : S'il en est ainsi, pourquoi suis-je enceinte ? Et elle alla consulter Yahweh.
\VS{23}Et Yahweh lui dit : Deux nations sont dans ton ventre, et deux peuples se sépareront au sortir de tes entrailles ; un de ces peuples sera plus fort que l'autre, et le plus grand sera asservi au plus petit\FTNT{Ro. 9:12.}.
\TextTitle{Naissance des jumeaux : Esaü et Jacob}
\VS{24}Les jours où elle devait accoucher s’accomplirent ; et voici, il y avait deux jumeaux dans son ventre.
\VS{25}Celui qui sortit le premier était roux et tout velu, comme un manteau de poil ; et on lui donna le nom d’Esaü.
\VS{26}Ensuite sortit son frère, tenant de sa main le talon d'Esaü ; c'est pourquoi il fut appelé Jacob\FTNT{Jacob:«~celui qui prend par le talon~» ou «~qui supplante~».}. Isaac était âgé de soixante ans quand ils naquirent.
\TextTitle{Esaü méprise son droit d'aînesse}
\VS{27}Depuis, les enfants devinrent grands. Esaü devint un habile chasseur, et un homme des champs ; mais Jacob fut un homme intègre, se tenant dans les tentes.
\VS{28}Isaac aimait Esaü ; car le gibier était sa nourriture. Mais Rebecca aimait Jacob.
\VS{29}Comme Jacob faisait cuire du potage, Esaü arriva des champs, et il était fatigué.
\VS{30}Et Esaü dit à Jacob : Donne-moi, je te prie, à manger de ce roux, de ce roux-là\FTNT{Probablement un plat de lentilles.} ; car je suis fatigué. C'est pourquoi on appela son nom, Edom\FTNT{Edom:«~rouge, de couleur rousse~».}.
\VS{31}Mais Jacob lui dit : Vends-moi aujourd'hui ton droit d'aînesse.
\VS{32}Et Esaü répondit : Voici, je m'en vais mourir ; et de quoi me servira le droit d'aînesse ?
\VS{33}Et Jacob dit : Jure-moi aujourd'hui ; et il lui jura ; ainsi il vendit son droit d'aînesse à Jacob\FTNT{Hé. 12:16.}.
\VS{34}Et Jacob donna à Esaü du pain et du potage de lentilles ; et il mangea et but ; puis il se leva et s'en alla ; ainsi Esaü méprisa son droit d'aînesse.
\Chap{26}
\TextTitle{Yahweh confirme son alliance à Isaac}
\VerseOne{}Or, il y eut une famine dans le pays, outre la première famine qui eut lieu du temps d'Abraham ; et Isaac s'en alla vers Abimélec, roi des Philistins, à Guérar.
\VS{2}Yahweh lui apparut et lui dit : Ne descends pas en Egypte ; demeure dans le pays que je te dirai.
\VS{3}Demeure dans ce pays-ci, et je serai avec toi, et je te bénirai ; car je donnerai toutes ces contrées à toi et à ta postérité, et j’accomplirai le serment que j'ai fait à ton père Abraham.
\VS{4}Je multiplierai ta postérité comme les étoiles du ciel ; et je donnerai ces contrées à ta postérité ; et toutes les nations de la terre seront bénies en ta postérité,
\VS{5}parce qu'Abraham a obéi à ma voix, et qu'il a gardé mon ordonnance, mes commandements, mes statuts et mes lois.
\TextTitle{Faute d'Isaac à Guérar\FTNTT{Ge. 20}}
\VS{6}Isaac donc demeura à Guérar.
\VS{7}Et quand les gens du lieu posaient des questions sur sa femme, il disait : C'est ma sœur ; car il craignait de dire : C'est ma femme ; de peur, disait-il, que les habitants du lieu ne me tuent à cause de Rebecca, car elle est belle de figure.
\VS{8}Comme son séjour se prolongeait, il arriva qu'Abimélec, roi des Philistins, regardant par la fenêtre, vit Isaac qui plaisantait avec Rebecca, sa femme\FTNT{Ge. 20.}.
\VS{9}Alors Abimélec appela Isaac et lui dit : Voici, c'est véritablement ta femme. Comment as-tu pu dire : C'est ma soeur ? Et Isaac lui répondit : C'est parce que j'ai dit : Il ne faut pas que je meure à cause d'elle.
\VS{10}Et Abimélec dit : Que nous as-tu fait ? Il s'en est peu fallu que quelqu'un du peuple n'ait couché avec ta femme, et tu nous aurais rendus coupables.
\VS{11}Abimélec donc fit une ordonnance à tout le peuple en disant : Celui qui touchera cet homme, ou à sa femme, sera certainement puni de mort.
\VS{12}Isaac sema dans cette terre-là et il recueillit cette année-là le centuple ; car Yahweh le bénit.
\VS{13}Cet homme devint riche, et il alla s’enrichissant de plus en plus, jusqu'à ce qu'il devint fort riche.
\VS{14}Il avait des troupeaux de menu bétail et des troupeaux de gros bétail, et un grand nombre de serviteurs ; et les Philistins lui portèrent envie ; 
\VS{15}Et tous les puits que les serviteurs de son père avaient creusés, du temps de son père Abraham, les Philistins les bouchèrent et les remplirent de terre.
\VS{16}Abimélec aussi dit à Isaac : Va-t’en de chez nous, car tu es devenu beaucoup plus puissant que nous.
\TextTitle{Les puits d'Isaac}
\VS{17}Isaac donc partit de là, et campa dans la vallée de Guérar, où il s’établit.
\VS{18}Isaac creusa de nouveau les puits d'eau qu'on avait creusés du temps d'Abraham, son père, et que les Philistins avaient bouchés après la mort d'Abraham, et il leur donna les mêmes noms que son père leur avait donnés.
\VS{19}Les serviteurs d'Isaac creusèrent dans cette vallée et y trouvèrent un puits d'eau vive.
\VS{20}Mais les bergers de Guérar eurent une querelle avec les bergers d'Isaac, disant : L'eau est à nous. Et il appela le nom du puits Esek parce qu'ils avaient contesté avec lui.
\VS{21}Ensuite, ils creusèrent un autre puits, pour lequel ils contestèrent aussi ; et il appela son nom Sitna.
\VS{22}Alors il se transporta de là et creusa un autre puits pour lequel ils ne contestèrent point, et il le nomma Rehoboth, en disant : C'est parce que Yahweh nous a maintenant mis au large, et nous fructifierons dans le pays.
\VS{23}Et de là il remonta à Beer-Schéba.
\VS{24}Yahweh lui apparut cette nuit-là et lui dit : Je suis le Dieu d'Abraham, ton père ; ne crains point, car je suis avec toi, je te bénirai et je multiplierai ta postérité à cause d'Abraham, mon serviteur.
\VS{25}Alors il bâtit là un autel, et invoqua le nom de Yahweh, et il y dressa ses tentes. Et les serviteurs d'Isaac y creusèrent un puits.
\VS{26}Abimélec vint à lui de Guérar avec Ahuzath, son ami, et Picol, chef de son armée.
\VS{27}Mais Isaac leur dit : Pourquoi venez-vous vers moi, puisque vous me haïssez et que vous m'avez renvoyé de chez vous ?
\VS{28}Ils répondirent : Nous avons vu clairement que Yahweh est avec toi ; et nous avons dit : Qu'il y ait maintenant un serment solennel entre nous, c'est-à-dire entre nous et toi ; et traitons alliance avec toi.
\VS{29}Jure que tu ne nous feras aucun mal, de même que nous ne t'avons point maltraité, que nous t'avons fait seulement du bien, et que nous t’avons laissé partir en paix. Toi qui es maintenant béni de Yahweh.
\VS{30}Alors il leur fit un festin, et ils mangèrent et burent.
\VS{31}Ils se levèrent de bon matin, et jurèrent l'un à l'autre. Puis Isaac les renvoya, et ils s'en allèrent en paix.
\VS{32}Ce même jour, les serviteurs d'Isaac vinrent lui parler du puits qu'ils avaient creusé, et lui dirent : Nous avons trouvé de l'eau.
\VS{33}Et il l'appela Schiba. C'est pourquoi le nom de la ville a été Beer-Schéba jusqu'à aujourd'hui.
\VS{34}Esaü, âgé de quarante ans, prit pour femmes Judith, fille de Beéri, le Héthien, et Basmath, fille d'Elon, le Héthien.
\VS{35}Elles furent un sujet d’amertume pour l’esprit d’Isaac et de Rebecca.
\Chap{27}
\TextTitle{Jacob prend la bénédiction d'Isaac à la place d'Esaü}
\VerseOne{}Et il arriva que quand Isaac fut devenu vieux, et que ses yeux furent si affaiblis qu'il ne pouvait plus voir, il appela Esaü, son fils aîné, et lui dit : Mon fils ! Et il lui répondit : Me voici.
\VS{2}Isaac lui dit : Voici, maintenant je suis devenu vieux, et je ne connais pas le jour de ma mort.
\VS{3}Maintenant donc, je te prie, prends tes armes, ton carquois et ton arc, va dans les champs, et chasse-moi du gibier.
\VS{4}Apprête-moi un mets comme j’aime, et apporte-le-moi, afin que je mange, et que mon âme te bénisse avant que je meure.
\VS{5}Or Rebecca écoutait pendant qu'Isaac parlait à Esaü, son fils. Esaü donc s'en alla dans les champs pour chasser du gibier et pour le rapporter.
\VS{6}Et Rebecca parla à Jacob, son fils, et lui dit : Voici, j'ai entendu parler ton père à Esaü, ton frère, disant :
\VS{7}Apporte-moi du gibier, et fais-moi un mets, afin que je le mange et je te bénirai devant Yahweh avant de mourir.
\VS{8}Maintenant donc, mon fils, obéis à ma parole, et fais ce que je vais te commander.
\VS{9}Va maintenant à la bergerie, et prends-moi là deux bons chevreaux parmi les chèvres, et j'en ferai un mets pour ton père comme il aime.
\VS{10}Et tu le porteras à ton père, afin qu'il le mange et qu'il te bénisse avant sa mort.
\VS{11}Jacob répondit à Rebecca sa mère : Voici, Esaü, mon frère, est un homme velu, et je suis un homme sans poil.
\VS{12}Peut-être que mon père me touchera-t-il, et il me regardera comme un homme qui a voulu le tromper, et j'attirerai sur moi sa malédiction et non pas sa bénédiction.
\VS{13}Sa mère lui dit : Mon fils, que la malédiction que tu crains retombe sur moi ! Obéis seulement à ma parole, et va me prendre ce que je t'ai dit.
\TextTitle{Déception d'Esaü\FTNTT{Hé. 12:16-17}}
\VS{14}Jacob alla les prendre et les apporta à sa mère ; et sa mère fit un mets comme son père aimait.
\VS{15}Puis Rebecca prit les plus précieux habits d'Esaü, son fils aîné, qu'elle avait dans la maison, et elle les fit mettre à Jacob, son fils cadet.
\VS{16}Elle couvrit ses mains et son cou, qui étaient sans poil, des peaux des chevreaux.
\VS{17}Puis elle mit entre les mains de son fils Jacob le mets et le pain qu'elle avait apprêtés.
\VS{18}Il vint vers son père, et lui dit : Mon père ! Il répondit : Me voici ; qui es-tu, mon fils ?
\VS{19}Jacob répondit à son père : Je suis Esaü, ton fils aîné ; j'ai fait ce que tu m’as dit. Lève-toi, je te prie, assieds-toi et mange de mon gibier, afin que ton âme me bénisse.
\VS{20}Isaac dit à son fils : Eh quoi ! Tu en as déjà trouvé, mon fils ! Et il dit : Yahweh ton Dieu l'a fait venir devant moi.
\VS{21}Isaac dit à Jacob : Approche-toi, je te prie, mon fils, et que je te touche, afin que je sache si tu es mon fils Esaü ou non.
\VS{22}Jacob donc s'approcha de son père Isaac, qui le toucha et dit : Cette voix est la voix de Jacob, mais ces mains sont les mains d'Esaü.
\VS{23}Et il ne le reconnut pas, car ses mains étaient velues comme les mains de son frère Esaü ; et il le bénit.
\VS{24}Il dit : C’est toi, mon fils Esaü ? Il répondit : Je le suis.
\VS{25}Isaac lui dit : Apporte-moi donc la viande, et que je mange du gibier de mon fils, afin que mon âme te bénisse. Jacob l'apporta, et Isaac mangea ; il lui apporta aussi du vin, et il but.
\VS{26}Puis Isaac, son père, lui dit : Approche-toi, je te prie, et embrasse-moi mon fils.
\VS{27}Jacob s'approcha et l’embrassa. Isaac sentit l'odeur de ses habits, et le bénit en disant : Voici l'odeur de mon fils, comme l'odeur d'un champ que Yahweh a béni.
\VS{28}Que Dieu te donne de la rosée du ciel, et de la graisse de la terre, du blé et du vin en abondance\FTNT{Hé. 11:20.} !
\VS{29}Que des peuples te servent, et que des nations se prosternent devant toi ! Sois le maître de tes frères, et que les fils de ta mère se prosternent devant toi ! Maudit soit quiconque te maudira, et béni soit quiconque te bénira.
\VS{30}Isaac avait fini de bénir Jacob, et Jacob avait à peine quitté son père Isaac, qu’Esaü, son frère, revint de la chasse.
\VS{31}Il apprêta aussi un mets, l’apporta à son père, et lui dit : Que mon père se lève et mange du gibier de son fils, afin que ton âme me bénisse.
\VS{32}Isaac, son père, lui dit : Qui es-tu ? Et il dit : Je suis ton fils, ton fils aîné, Esaü.
\VS{33}Isaac fut saisi d'une grande, d’une violente émotion, et dit : Qui est donc celui qui a chassé du gibier et me l’a apporté ? J'ai mangé de tout avant que tu ne viennes, et je l'ai béni. Aussi sera-t-il béni !
\VS{34}Dès qu'Esaü entendit les paroles de son père, il poussa de forts cris, pleins d’amertume, et il dit à son père : Bénis-moi aussi, bénis-moi, mon père !
\VS{35}Mais il dit : Ton frère est venu avec tromperie, et il a enlevé ta bénédiction.
\VS{36}Esaü dit : N'est-ce pas avec raison qu'on a appelé son nom Jacob ? Car il m'a déjà supplanté deux fois ; il m'a enlevé mon droit d'aînesse, et voici, maintenant il a enlevé ma bénédiction. Puis il dit : Ne m'as-tu point réservé de bénédiction ?
\VS{37}Isaac répondit à Esaü en disant : Voici, je l'ai établi ton maître, et lui ai donné tous ses frères pour serviteurs, et je l'ai pourvu de blé et de vin ; et que ferai-je maintenant pour toi, mon fils ?
\VS{38}Esaü dit à son père : N'as-tu qu'une bénédiction, mon père ? Bénis-moi aussi, bénis-moi, mon père ! Et Esaü éleva la voix et pleura\FTNT{Hé. 12:17.}.
\VS{39}Isaac, son père, répondit, et dit : Voici, ta demeure sera privée de la graisse de la terre, et de la rosée du ciel, d'en haut.
\VS{40}Tu vivras par ton épée, et tu seras asservi à ton frère ; mais il arrivera qu'étant devenu maître, tu briseras son joug de dessus ton cou.
\TextTitle{Fuite de Jacob chez Laban}
\VS{41}Esaü conçut de la haine contre Jacob, à cause de la bénédiction dont son père l'avait béni ; et Esaü dit en son cœur : Les jours du deuil de mon père approchent, et je tuerai Jacob, mon frère.
\VS{42}On rapporta à Rebecca les paroles d'Esaü, son fils aîné ; et elle fit alors appeler Jacob, son fils cadet, et lui dit : Voici, Esaü, ton frère, se console dans l'espérance qu'il a de te tuer.
\VS{43}Maintenant donc, mon fils, obéis à ma parole ! Lève-toi, et enfuis-toi à Charan, vers Laban, mon frère.
\VS{44}Et reste avec lui quelque temps, jusqu'à ce que la fureur de ton frère soit passée ;
\VS{45}jusqu’à ce que la colère de ton frère se détourne de toi, et qu'il oublie ce que tu lui as fait. Pourquoi serais-je privée de vous deux en un même jour ?
\VS{46}Rebecca dit à Isaac : Je suis dégoûtée de la vie, à cause de filles de Heth. Si Jacob prend une femme, comme celles-ci, parmi les filles de Heth, parmi les filles du pays, à quoi me sert la vie ?
\Chap{28}
\TextTitle{A Béthel, Yahweh confirme son alliance à Jacob}
\VerseOne{}Isaac donc appela Jacob, et le bénit, et lui donna cet ordre : Tu ne prendras point de femme parmi les filles de Canaan.
\VS{2}Lève-toi, va à Paddan-Aram, à la maison de Bethuel, père de ta mère, et prends-toi une femme de là, parmi les filles de Laban, frère de ta mère.
\VS{3}Que le Dieu Tout-Puissant te bénisse, te rende fécond et te multiplie, afin que tu deviennes une assemblée de peuples.
\VS{4}Qu’il te donne la bénédiction d'Abraham, à toi et à ta postérité avec toi, afin que tu obtiennes en héritage le pays où tu as été étranger, que Dieu a donné à Abraham.
\VS{5}Isaac donc fit partir Jacob, qui s'en alla à Paddan-Aram, vers Laban, fils de Bethuel, le Syrien, frère de Rebecca, mère de Jacob et d'Esaü.
\VS{6}Esaü vit qu'Isaac avait béni Jacob, et qu'il l'avait envoyé à Paddan-Aram afin qu'il prenne une femme de ce pays-là pour lui, et qu'il lui avait donné cet ordre, quand il le bénissait, disant : Ne prends point de femme parmi les filles de Canaan ;
\VS{7}il vit que Jacob avait obéi à son père et à sa mère, et qu’il était parti à Paddan-Aram.
\VS{8}Esaü comprit ainsi que les filles de Canaan déplaisaient à Isaac, son père.
\VS{9}Et Esaü s’en alla vers Ismaël. Il prit pour femme, outre ses autres femmes, Mahalath, fille d'Ismaël, fils d'Abraham, sœur de Nebajoth.
\VS{10}Jacob partit de Beer-Schéba et s'en alla à Charan.
\VS{11}Il arriva dans un lieu où il passa la nuit, parce que le soleil était couché. Il y prit donc une pierre\FTNT{1 Pi. 2:4. Voir  commentaire en Es. 8:13-15.}, et en fit son chevet, et il se coucha dans ce lieu-là.
\VS{12}Il eut un songe ; et voici, une échelle dressée sur la terre, dont le sommet touchait le ciel. Et voici, les anges de Dieu montaient et descendaient par cette échelle\FTNT{Jn. 1:51.}.
\VS{13}Et voici, Yahweh se tenait sur l'échelle, et il lui dit : Je suis Yahweh, le Dieu d'Abraham, ton père, et le Dieu d'Isaac ; je te donnerai à toi et à ta postérité, la terre sur laquelle tu es couché.
\VS{14}Ta postérité sera comme la poussière de la terre, et tu t'étendras à l'occident et à l'orient, au nord et au midi, et toutes les familles de la terre seront bénies en toi et en ta postérité.
\VS{15}Voici, je suis avec toi ; et je te garderai partout où tu iras ; et je te ramènerai dans ce pays ; car je ne t'abandonnerai point que je n'aie exécuté ce que je t'ai dit.
\VS{16}Et quand Jacob fut réveillé de son sommeil, il dit : Certainement, Yahweh est en ce lieu-ci, et moi, je ne le savais pas !
\VS{17}Il eut peur et dit : Que ce lieu-ci est effrayant ! C'est ici la maison de Dieu, et c'est ici la porte des cieux !
\VS{18}Et Jacob se leva de bon matin, prit la pierre dont il avait fait son chevet, il la dressa pour monument, et versa de l'huile sur son sommet.
\VS{19}Il donna à ce lieu le nom de Béthel ; mais auparavant la ville s'appelait Luz.
\VS{20}Jacob fit un vœu en disant : Si Dieu est avec moi, et s'il me garde pendant le voyage que je fais, s'il me donne du pain à manger, et des habits pour me vêtir,
\VS{21}et si je retourne en paix à la maison de mon père, certainement Yahweh sera mon Dieu.
\VS{22}Cette pierre que j'ai dressée pour monument sera la maison de Dieu ; et de tout ce que tu m'auras donné, je t'en donnerai entièrement la dîme\FTNT{Voir commentaire sur la dîme en No. 18:21 et Mal. 3:10.}.
\Chap{29}
\TextTitle{Jacob épouse Léa et Rachel chez Laban}
\VerseOne{}Jacob donc se mit en chemin, et s'en alla au pays des fils de l’orient.
\VS{2}Il regarda. Et voici, il y avait un puits dans un champ ; et voici il y avait à côté trois troupeaux de brebis couchées près du puits, car c’était à ce puits qu’on abreuvait les troupeaux.  Et il y avait une grosse pierre sur l'ouverture du puits.
\VS{3}Tous les troupeaux se rassemblaient là ; on roulait la pierre de dessus l'ouverture du puits, et on abreuvait les troupeaux ; et ensuite on remettait la pierre à sa place, sur l'ouverture du puits.
\VS{4}Jacob leur dit : Mes frères, d'où êtes-vous ? Ils répondirent : Nous sommes de Charan.
\VS{5}Il leur dit : Connaissez-vous Laban, fils de Nachor ? Ils répondirent : Nous le connaissons.
\VS{6}Il leur dit : Se porte-t-il bien ? Ils lui répondirent : Il se porte bien ; et voici Rachel, sa fille, qui vient avec le troupeau.
\VS{7}Il dit : Voici, il est encore grand jour, et il n'est pas temps de rassembler les troupeaux ; abreuvez les brebis, puis allez et faites-les paître.
\VS{8}Ils répondirent : Nous ne le pouvons pas, jusqu'à ce que tous les troupeaux soient rassemblés et qu'on ait ôté la pierre de dessus l'ouverture du puits, afin d'abreuver les troupeaux.
\VS{9}Comme il parlait encore avec eux, Rachel arriva avec le troupeau de son père ; car elle était bergère.
\VS{10}Lorsque Jacob vit Rachel, fille de Laban, frère de sa mère, et le troupeau de Laban, frère de sa mère, il s'approcha et roula la pierre de dessus l'ouverture du puits, et abreuva le troupeau de Laban, frère de sa mère.
\VS{11}Et Jacob embrassa Rachel, et il éleva sa voix et pleura.
\VS{12}Jacob apprit à Rachel qu'il était frère de son père, et qu'il était fils de Rebecca ; et elle courut le rapporter à son père.
\VS{13}Dès que Laban eut entendu parler de Jacob, fils de sa soeur, il courut au-devant de lui, il le prit dans ses bras et l’embrassa, et il le fit venir dans sa maison ; et Jacob raconta à Laban tout ce qui lui était arrivé.
\VS{14}Et Laban lui dit : Certainement, tu es mon os et ma chair. Jacob demeura un mois entier chez Laban.
\VS{15}Puis Laban dit à Jacob : Me serviras-tu pour rien parce que tu es mon frère ? Dis-moi quel sera ton salaire ?
\VS{16}Or Laban avait deux filles : L'aînée s'appelait Léa, et la cadette Rachel.
\VS{17}Léa avait les yeux délicats, mais Rachel était belle de taille et belle de figure.
\VS{18}Jacob aimait Rachel, et il dit : Je te servirai sept ans pour Rachel, ta cadette.
\VS{19}Et Laban répondit : Il vaut mieux que je te la donne que de la donner à un autre homme ; demeure avec moi.
\VS{20}Ainsi Jacob servit sept années pour Rachel ; et elles furent à ses yeux comme quelques jours, parce qu'il l'aimait.
\VS{21}Et Jacob dit à Laban : Donne-moi ma femme, car mon temps est accompli, et j’irai vers elle.
\VS{22}Laban réunit tous les gens du lieu et fit un festin.
\VS{23}Mais quand le soir fut venu, il prit Léa, sa fille, et l'amena vers Jacob qui s’approcha d’elle.
\VS{24}Et Laban donna Zilpa, sa servante, à Léa, sa fille, pour servante.
\VS{25}Le lendemain matin, voilà que c'était Léa. Alors Jacob dit à Laban : Qu'est-ce que tu m'as fait ? N'ai-je pas servi chez toi pour Rachel ? Et pourquoi m'as-tu trompé ?
\VS{26}Laban répondit : On ne fait pas ainsi dans ce lieu de donner la plus jeune avant l'aînée.
\VS{27}Achève la semaine avec celle-ci, et nous te donnerons aussi l'autre, pour le service que tu feras encore chez moi sept autres années.
\VS{28}Jacob donc fit ainsi, et il acheva la semaine avec Léa ; et Laban lui donna aussi pour femme Rachel, sa fille.
\VS{29}Et Laban donna Bilha, sa servante, à Rachel, sa fille, pour servante.
\VS{30}Jacob alla aussi vers Rachel, et il aima Rachel plus que Léa ; et il servit encore chez Laban sept autres années.
\VS{31}Yahweh vit que Léa était haïe, et il ouvrit sa matrice, tandis que Rachel était stérile.
\TextTitle{Les enfants de Jacob}
\VS{32}Léa conçut et enfanta un fils à qui elle donna le nom de Ruben, car elle dit : C'est parce que Yahweh a vu mon affliction, et maintenant mon mari m'aimera.
\VS{33}Elle conçut encore et enfanta un fils, et elle dit : Parce que Yahweh a entendu que j'étais haïe, il m'a aussi donné celui-ci. Et elle lui donna le nom de Siméon.
\VS{34}Elle conçut encore et enfanta un fils, et elle dit : Maintenant mon mari s'attachera à moi, car je lui ai enfanté trois fils. C'est pourquoi on lui donna le nom de Lévi.
\VS{35}Elle conçut encore et enfanta un fils, et elle dit : Cette fois je louerai Yahweh. C'est pourquoi elle lui donna le nom de Juda. Et elle cessa d'avoir des enfants.
\Chap{30}
\TextTitle{Les enfants de Jacob (suite)}
\VerseOne{}Alors Rachel, voyant qu’elle ne donnait point d'enfants à Jacob, fut jalouse de Léa, sa sœur, et elle dit à Jacob : Donne-moi des enfants, autrement je meurs !
\VS{2}La colère de Jacob s’enflamma contre Rachel, et il dit : Suis-je à la place de Dieu pour t’empêcher d'avoir des enfants ?
\VS{3}Elle dit : Voici ma servante Bilha ; va vers elle ; qu’elle enfante sur mes genoux, et que j’aie des fils par elle.
\VS{4}Et elle lui donna pour femme Bilha, sa servante, et Jacob alla vers elle.
\VS{5}Bilha conçut et enfanta un fils à Jacob.
\VS{6}Rachel dit : Dieu a jugé en ma faveur, et il a aussi exaucé ma voix, et m'a donné un fils ; c'est pourquoi elle l’appela du nom de Dan.
\VS{7}Bilha, servante de Rachel, conçut encore et enfanta un second fils à Jacob.
\VS{8}Rachel dit : J'ai fortement lutté contre ma sœur, aussi j'ai eu la victoire ; c'est pourquoi elle l’appela du nom de Nephthali.
\VS{9}Alors Léa, voyant qu'elle avait cessé de faire des enfants, prit Zilpa, sa servante, et la donna pour femme à Jacob.
\VS{10}Zilpa, servante de Léa, enfanta un fils à Jacob.
\VS{11}Léa dit : Le bonheur est arrivé, c'est pourquoi elle l’appela du nom de Gad.
\VS{12}Zilpa, servante de Léa, enfanta un second fils à Jacob.
\VS{13}Léa dit : C'est pour me rendre heureuse, car les filles me diront bienheureuse ; c'est pourquoi elle l’appela du nom d’Aser.
\VS{14}Ruben sortit au temps de la moisson des blés, trouva des mandragores\FTNT{La mandragore, appelée pomme d'amour, était utilisée comme excitant du désir sexuel ainsi que pour favoriser la procréation. On attribuait à cette plante aux propriétés hallucinogènes des vertus magiques.} aux champs, et les apporta à Léa, sa mère ; et Rachel dit à Léa : Donne-moi, je te prie, des mandragores de ton fils.
\VS{15}Elle lui répondit : Est-ce peu que tu aies pris mon mari, pour que tu prennes aussi les mandragores de mon fils ? Et Rachel dit : Qu'il couche donc cette nuit avec toi pour les mandragores de ton fils.
\VS{16}Le soir, comme Jacob revenait des champs, Léa sortit au-devant de lui et lui dit : Tu viendras vers moi, car je t'ai acheté pour les mandragores de mon fils ; et il coucha avec elle cette nuit-là.
\VS{17}Dieu exauça Léa, et elle conçut et enfanta à Jacob un cinquième fils.
\VS{18}Léa dit : Dieu m'a récompensée, parce que j'ai donné ma servante à mon mari ; c'est pourquoi elle l’appela du nom d’Issacar.
\VS{19}Léa conçut encore et enfanta un sixième fils à Jacob.
\VS{20}Léa dit : Dieu m'a donné un beau don ; maintenant mon mari habitera avec moi, car je lui ai enfanté six fils ; c'est pourquoi elle l’appela du nom de Zabulon.
\VS{21}Puis elle enfanta une fille et la nomma Dina.
\VS{22}Dieu se souvint de Rachel, il l’exauça et il ouvrit sa matrice.
\VS{23}Alors elle conçut et enfanta un fils, et elle dit : Dieu a ôté mon opprobre.
\VS{24}Et elle lui donna le nom de Joseph, en disant : Que Yahweh m'ajoute un autre fils !
\TextTitle{Jacob devient de plus en plus riche}
\VS{25}Lorsque Rachel eut enfanté Joseph, Jacob dit à Laban : Laisse-moi partir, pour que je m’en aille chez moi, dans mon pays.
\VS{26}Donne-moi mes femmes et mes enfants, pour lesquels je t'ai servi, et je m'en irai ; car tu sais de quelle manière je t'ai servi.
\VS{27}Laban lui répondit : Ecoute, je te prie, si j'ai trouvé grâce à tes yeux ; j’ai deviné que Yahweh m'a béni à cause de toi.
\VS{28}Il lui dit aussi : Fixe-moi le salaire que tu veux, et je te le donnerai.
\VS{29}Jacob lui répondit : Tu sais comment je t'ai servi et ce qu'est devenu ton bétail avec moi.
\VS{30}Car le peu que tu avais avant que je vienne s’est beaucoup accru, et Yahweh t'a béni depuis que j’ai mis mes pieds chez toi. Et maintenant, quand ferai-je aussi quelque chose pour ma maison ?
\VS{31}Laban lui dit : Que te donnerai-je ? Et Jacob répondit : Tu ne me donneras rien ; mais je ferai paître encore tes troupeaux, et je les garderai, si tu consens à ce que je vais te dire.
\VS{32}Je parcourrai aujourd'hui tes troupeaux, mets à part parmi toutes les brebis tachetées et marquetées, et tous les agneaux noirs, et les chèvres marquetées et tachetées. Ce sera mon salaire.
\VS{33}Ma justice me rendra témoignage à l’avenir devant toi ; quand tu viendras reconnaître mon salaire, en ta présence ; et tout ce qui ne sera pas marqueté ou tacheté parmi les chèvres, et noirs parmi les agneaux, sera considéré comme un vol s'il est trouvé chez moi.
\VS{34}Laban dit : Voici, qu'il te soit fait comme tu l'as dit.
\VS{35}Ce même jour, il sépara les boucs rayés et marquetés, et toutes les chèvres tachetées et marquetées, toutes celles où il y avait du blanc, et tous les agneaux noirs. Il les remit entre les mains de ses fils.
\VS{36}Puis il mit l'espace de trois journées de chemin entre lui et Jacob ; et Jacob fit paître le reste des troupeaux de Laban.
\VS{37}Mais Jacob prit des branches vertes de peuplier, d’amandier et de platane ; il y pela des bandes blanches,  mettant à nu le blanc qui était sur les branches.
\VS{38}Puis il plaça les branches qu’il avait pelées dans les auges, dans les abreuvoirs, sous les yeux des brebis qui venaient boire, et elles entraient en chaleur quand elles venaient boire.
\VS{39}Les brebis entraient en chaleur près des branches, et elles faisaient des brebis rayées, tachetées et marquetées.
\VS{40}Jacob séparait les agneaux, et il mettait ensemble ce qui était rayé et tout ce qui était noir dans les troupeaux de Laban. Il se fit ainsi des troupeaux à part, qu’il ne réunit point aux troupeaux de Laban.
\VS{41}Toutes les fois que les brebis vigoureuses entraient en chaleur, Jacob mettait les branches dans les auges sous les yeux des brebis, afin qu'elles entrent en chaleur près des  branches.
\VS{42}Mais pour les brebis chétives, il ne les mettait point ; de sorte que les chétives appartenaient à Laban et les vigoureuses à Jacob.
\VS{43}Ainsi cet homme devint de plus en plus riche ; il eut du menu bétail en abondance, des servantes et des serviteurs, des chameaux et des ânes.
\Chap{31}
\TextTitle{Yahweh demande à Jacob de rentrer dans la terre de se pères}
\VerseOne{}Or Jacob entendit les discours des fils de Laban qui disaient : Jacob a pris tout ce qui appartenait à notre père, et c’est de ce qui était à notre père qu’il s’est acquis toute cette richesse.
\VS{2}Jacob regarda le visage de Laban, et voici, il n'était plus à son égard comme auparavant.
\VS{3}Alors Yahweh dit à Jacob : Retourne au pays de tes pères et vers ta parenté, et je serai avec toi.
\VS{4}Jacob fit appeler Rachel et Léa qui étaient aux champs vers son troupeau,
\VS{5}et leur dit : Je vois au visage de votre père qu'il n'est plus envers moi comme il était auparavant ; toutefois le Dieu de mon père a été avec moi.
\VS{6}Vous savez que j'ai servi votre père de tout mon pouvoir.
\VS{7}Mais votre père s'est moqué de moi et a changé dix fois mon salaire ; mais Dieu ne lui a pas permis de me faire du mal.
\VS{8}Quand il disait : Les tachetées seront ton salaire, alors toutes les brebis faisaient des agneaux tachetés ; et quand il disait : Les marquetées seront ton salaire, alors toutes les brebis faisaient des agneaux marquetés.
\VS{9}Ainsi Dieu a ôté à votre père son bétail et me l'a donné.
\VS{10}Au temps où les brebis entraient en chaleur, je levai mes yeux et vis en songe que les boucs qui couvraient les brebis étaient rayés, tachetés, et marquetés.
\VS{11}Et l'Ange de Dieu\FTNT{Gn. 7:7} me dit en songe : Jacob ! Et je répondis : Me voici.
\VS{12}Il dit : Lève maintenant tes yeux et regarde : Tous les boucs qui couvrent les brebis sont rayés, tachetés et marquetés, car j'ai vu tout ce que te fait Laban.
\VS{13}Je suis le Dieu de Béthel, où tu oignis la pierre que tu dressas pour monument, où tu me fis un vœu.  Maintenant lève-toi, sors de ce pays, et retourne au pays de ta naissance.
\TextTitle{Jacob fuit de chez Laban avec sa famille}
\VS{14}Alors Rachel et Léa lui répondirent et dirent : Avons-nous encore quelque portion et quelque héritage dans la maison de notre père ?
\VS{15}Ne nous a-t-il pas traitées comme des étrangères ? Car il nous a vendues, et même il a entièrement mangé notre argent.
\VS{16}Car toutes les richesses que Dieu a ôtées à notre père nous appartenaient ainsi qu’à nos enfants. Maintenant donc fais tout ce que Dieu t'a dit.
\VS{17}Ainsi Jacob se leva, et fit monter ses enfants et ses femmes sur des chameaux.
\VS{18}Il emmena tout son bétail et tous les biens qu'il avait acquis, et tout ce qu'il possédait et qu'il avait acquis à Paddan-Aram, pour aller vers Isaac, son père, au pays de Canaan.
\VS{19}Or comme Laban était allé tondre ses brebis, Rachel déroba les théraphim de son père\FTNT{Les théraphim étaient des idoles utilisées dans un sanctuaire de maison ou dans un lieu de culte. Voir Jg. 18:14 ; 2 R. 23:24.}.
\VS{20}Et Jacob trompa Laban, le Syrien, en ne l’avertissant pas de son dessein, parce qu'il s'enfuyait.
\VS{21}Il s'enfuit avec tout ce qui lui appartenait ; il se leva, passa le fleuve, et se dirigea vers la montagne de Galaad.
\VS{22}Le troisième jour, on rapporta à Laban que Jacob s’était enfui.
\VS{23}Alors il prit avec lui ses frères, et il le poursuivit sept journées de marche, et l'atteignit à la montagne de Galaad.
\VS{24}Mais Dieu apparut à Laban, le Syrien, en songe la nuit, et lui dit : Garde-toi de parler à Jacob ni en bien ni en mal.
\VS{25}Laban donc atteignit Jacob. Jacob avait dressé ses tentes sur la montagne ; et Laban dressa aussi les siennes avec ses frères sur la montagne de Galaad.
\VS{26}Et Laban dit à Jacob : Qu'as-tu fait ? Tu m’as trompé, tu as emmené mes filles comme des prisonnières de guerre.
\VS{27}Pourquoi as-tu pris la fuite secrètement, m’as-tu trompé et ne m’as-tu pas averti ? Car je t'aurais laissé partir avec joie et avec des chansons, au son des tambours et des violons.
\VS{28}Tu ne m'as pas laissé embrasser mes fils et mes filles ! C’est en insensé que tu as agi.
\VS{29}J'ai en main le pouvoir de vous faire du mal, mais le Dieu de votre père m'a parlé la nuit passée et m'a dit : Garde-toi de ne parler à Jacob ni en bien ni en mal.
\VS{30}Maintenant que tu es parti, parce que tu  languissais après la maison de ton père, pourquoi as-tu dérobé mes dieux ?
\VS{31}Jacob répondit et dit à Laban : Je me suis enfui parce que je craignais ; car je me disais qu'il fallait prendre garde que tu ne me ravisses tes filles.
\VS{32}Mais celui chez qui tu trouveras tes dieux ne vivra point. En présence de nos frères, examine  s'il y a chez moi quelque chose qui t'appartienne, et prends-le ; car Jacob ignorait que Rachel les avait dérobés.
\VS{33}Alors Laban entra dans la tente de Jacob, et dans celle de Léa, et dans la tente des deux servantes, et il ne les trouva point ; et étant sorti de la tente de Léa, il entra dans la tente de Rachel.
\VS{34}Mais Rachel avait prit les théraphim et les avait mis dans le bât d'un chameau, et s’était assise dessus ; et Laban fouilla toute la tente et ne les trouva point.
\VS{35}Elle dit à son père : Que mon seigneur ne se fâche point de ce que je ne puis me lever devant lui, car j'ai ce que les femmes ont coutume d'avoir ; et il fouilla, mais il ne trouva point les théraphim.
\VS{36}Jacob se mit en colère et querella Laban. Il reprit la parole et lui dit : Quel est mon crime ? Quel est mon péché, pour que tu me poursuives avec tant d’ardeur ?
\VS{37}Car tu as fouillé tous mes effets, qu'as-tu trouvé des effets de ta maison ? Mets-les ici devant mes frères et les tiens, et qu'ils soient juges entre nous deux.
\VS{38}Voilà vingt ans que j’ai passés chez toi ; tes brebis et tes chèvres n'ont point avorté, je n'ai point mangé les moutons de tes troupeaux.
\VS{39}Je ne t'ai point rapporté de bêtes déchirées par les bêtes sauvages, j'en ai moi-même subi la perte ; et tu redemandais de ma main ce qui avait été dérobé de jour et ce qui avait été dérobé de nuit.
\VS{40}Le jour la chaleur me consumait, et la nuit le froid ; et le sommeil fuyait de mes yeux.
\VS{41}Voilà vingt ans que j’ai passés dans ta maison, quatorze ans pour tes deux filles, et six ans pour tes troupeaux, et tu m'as changé dix fois mon salaire.
\VS{42}Si je n’avais pas eu pour moi le Dieu de mon père, le Dieu d'Abraham, et celui que craint Isaac, certes tu m’aurais maintenant renvoyé à vide. Mais Dieu a regardé mon affliction et le travail de mes mains, et il t'a repris la nuit passée.
\VS{43}Laban répondit à Jacob et dit : Ces filles sont mes filles, et ces enfants sont mes enfants, et ces troupeaux sont mes troupeaux, et tout ce que tu vois est à moi ; et que ferais-je aujourd'hui à mes filles et aux enfants qu'elles ont enfantés ?
\VS{44}Maintenant donc, viens, faisons ensemble une alliance, et qu’elle serve de témoignage entre moi et toi.
\VS{45}Jacob prit une pierre et il la dressa pour monument.
\VS{46}Jacob dit à ses frères : Ramassez des pierres. Et ils prirent des pierres et ils en firent un monceau, et ils mangèrent là sur ce monceau.
\VS{47}Laban l'appela Jegar-Sahadutha, et Jacob l'appela Galed.
\VS{48}Et Laban dit : Ce monceau sera aujourd'hui témoin entre moi et toi ; c'est pourquoi il fut nommé Galed (poste d’observation).
\VS{49}Il fut aussi appelé Mitspa ; parce que Laban dit : Que Yahweh veille sur moi et sur toi, quand nous nous serons l'un et l'autre perdus de vue.
\VS{50}Si tu maltraites mes filles et si tu prends une autre femme que mes filles, ce n’est pas un homme qui sera témoin entre nous, prends-y garde ; c'est Dieu qui est témoin entre moi et toi.
\VS{51}Laban dit encore à Jacob : Regarde ce monceau, et considère le monument que j'ai dressé entre moi et toi.
\VS{52}Que ce monceau soit témoin et que ce monument soit témoin que je n’irai pas vers toi au-delà de ce monceau, et que tu ne viendras pas vers moi au-delà de ce monceau et de ce monument pour me faire du mal.
\VS{53}Que le Dieu d'Abraham et le Dieu de Nachor, le Dieu de leur père, juge entre nous ; mais Jacob jura par celui que craignait Isaac, son père.
\VS{54}Jacob offrit un sacrifice sur la montagne et invita ses frères pour manger du pain ; ils mangèrent donc du pain et passèrent la nuit sur la montagne.
\VS{55}Laban se leva de bon matin, embrassa ses fils et ses filles, et les bénit. Ensuite il s'en alla. Ainsi Laban retourna chez lui.
\Chap{32}
\TextTitle{Jacob devient Israël}
\VerseOne{}Et Jacob continua son chemin, et des anges de Dieu le rencontrèrent.
\VS{2}En les voyant, Jacob dit : C'est ici le camp de Dieu ! Et il donna à ce lieu le nom de Mahanaïm.
\VS{3}Jacob envoya devant lui des messagers vers Esaü, son frère, au pays de Séir, dans le territoire d'Edom.
\VS{4}Il leur donna cet ordre : Vous parlerez de cette manière à mon seigneur Esaü : Ainsi a dit ton serviteur Jacob : J'ai séjourné comme étranger chez Laban, et j’y ai habité jusqu'à présent ;
\VS{5}j’ai des bœufs, des ânes, des brebis, des serviteurs, et des servantes ; et j'envoie l’annoncer à mon seigneur, afin de trouver grâce à  tes yeux.
\VS{6}Et les messagers revinrent auprès de Jacob et lui dirent : Nous sommes allés vers ton frère Esaü, et il marche aussi à ta rencontre avec quatre cents hommes.
\VS{7}Alors Jacob fut très effrayé et rempli d’angoisse ; et il partagea le peuple qui était avec lui, et les brebis, et les boeufs, et les chameaux, en deux camps ; et  il dit :
\VS{8}Si Esaü attaque l'un des camps et le frappe, le camp qui restera pourra s’échapper.
\VS{9}Jacob dit aussi : Ô Dieu de mon père Abraham, Dieu de mon père Isaac, ô Yahweh qui m'as dit : Retourne dans ton pays, et vers ta parenté, et je te ferai du bien.
\VS{10}Je suis trop petit pour toutes les faveurs et pour toute la fidélité dont tu as usé envers ton serviteur ; car j'ai passé ce Jourdain avec mon bâton, et maintenant je forme deux camps.
\VS{11}Je te prie, délivre-moi de la main de mon frère Esaü ; car je crains qu'il ne vienne, et qu'il ne me frappe, et qu'il ne tue la mère avec les enfants.
\VS{12}Et toi, tu as dit : Certes, je te ferai du bien, et je rendrai ta postérité comme le sable de la mer, si abondant qu’on ne saurait le compter.
\VS{13}C’est dans ce lieu-là que Jacob passa la nuit.  Il prit de ce qu’il avait sous la main pour faire un présent à Esaü, son frère :
\VS{14}à savoir deux cents chèvres, vingt boucs, deux cents brebis et vingt béliers.
\VS{15}Trente femelles de chameaux qui allaitaient, et leurs petits ; quarante jeunes vaches, dix jeunes taureaux, vingt ânesses et dix ânes.
\VS{16}Il les mit entre les mains de ses serviteurs, chaque troupeau à part, et leur dit : Passez devant moi, et faites qu'il y ait un intervalle entre chaque troupeau.
\VS{17}Il donna cet ordre au premier, disant : Quand Esaü, mon frère, te rencontrera et te demandera, disant : A qui es-tu ? Et où vas-tu ? Et à qui sont ces choses qui sont devant toi ?
\VS{18}Alors tu diras : Je suis à ton serviteur Jacob ; c'est un présent qu'il envoie à mon seigneur Esaü ; et voici, il vient lui-même derrière nous.
\VS{19}Il donna le même ordre au deuxième, au troisième, et à tous ceux qui suivaient les troupeaux, disant : C’est ainsi que vous parlerez à mon seigneur Esaü, quand vous le rencontrerez.
\VS{20}Vous lui direz : Voici, ton serviteur Jacob vient aussi derrière nous. Car il se disait : J'apaiserai sa colère par ce présent qui va devant moi, et après cela, je verrai sa face ; peut-être qu'il me regardera favorablement.
\VS{21}Le présent passa devant lui ; mais il resta cette nuit-là dans le camp.
\VS{22}Il se leva cette nuit, et prit ses deux femmes, ses deux servantes, et ses onze enfants, et passa le gué de Jabbok.
\VS{23}Il les prit donc, et leur fit passer le torrent ; il fit aussi passer tout ce qu'il avait.
\VS{24}Jacob demeura seul. Alors un homme lutta avec lui jusqu'au lever de l’aurore.
\VS{25}Et quand cet homme vit qu'il ne pouvait pas le vaincre, il frappa à l'emboîture de la hanche de Jacob ; ainsi l'emboîture de l'os de la hanche de Jacob se démit pendant qu’il luttait avec lui.
\VS{26}Et cet homme lui dit : Laisse-moi, car l'aube du jour est levée. Mais il dit : Je ne te laisserai point que tu ne m'aies béni.
\VS{27}Cet homme lui dit : Quel est ton nom ? Il répondit : Jacob.
\VS{28}Alors il dit : Ton nom ne sera plus Jacob, mais tu seras appelé Israël ; car tu as été le vainqueur en luttant avec Dieu et avec les hommes, et tu as été le plus fort.
\VS{29}Jacob l’interrogea en disant : Je te prie, déclare-moi ton nom. Et il répondit : Pourquoi demandes-tu mon nom ? Et il le bénit là\FTNT{Jg. 13:18.}.
\VS{30}Jacob appela ce lieu du nom de Peniel ; car, dit-il, j’ai vu Dieu face à face, et mon âme a été délivrée.
\VS{31}Le soleil se levait lorsqu’il passa Peniel. Jacob boitait de la hanche.
\VS{32}C'est pourquoi, jusqu'à ce jour, les enfants d'Israël ne mangent point le tendon qui est à l’emboîture de la hanche ; parce que Dieu frappa Jacob à l'emboîture de la hanche, au tendon.
\Chap{33}
\TextTitle{Jacob demande pardon à son frère Esaü}
\VerseOne{}Et Jacob leva ses yeux et regarda ; et voici, Esaü arrivait avec quatre cents hommes. Et Jacob répartit les enfants entre Léa, Rachel, et les deux servantes.
\VS{2}Il plaça en tête les servantes avec leurs enfants ; Léa et ses enfants ensuite ; et Rachel et Joseph au dernier rang.
\VS{3}Quant à lui,  il passa devant eux et se prosterna à terre sept fois, jusqu'à ce qu'il soit près de son frère.
\VS{4}Esaü courut à sa rencontre ; il le prit dans ses bras, se jeta sur son cou, et l’embrassa. Et ils pleurèrent.
\VS{5}Esaü leva ses yeux, vit les femmes et les enfants, et dit : Qui sont ceux-là ? Sont-ils à toi ? Jacob lui répondit : Ce sont les enfants que Dieu, par sa grâce, a donnés à ton serviteur.
\VS{6}Les servantes s'approchèrent, elles et leurs enfants, et se prosternèrent.
\VS{7}Puis Léa aussi s'approcha avec ses enfants, et ils se prosternèrent, et ensuite Joseph et Rachel s'approchèrent et se prosternèrent aussi.
\VS{8}Esaü dit : Que veux-tu faire avec tout ce camp que j'ai rencontré ? Et Jacob répondit : C'est pour trouver grâce aux yeux de mon seigneur.
\VS{9}Esaü dit : Je suis dans l’abondance, mon frère ; garde ce qui est à toi.
\VS{10}Et Jacob répondit : Non, je te prie, si j'ai maintenant trouvé grâce à tes yeux, reçois ce présent de ma main ; parce que j'ai vu ta face comme si j'avais vu la face de Dieu, et parce que tu m’as accueilli favorablement.
\VS{11}Accepte, je te prie, mon présent qui t'a été offert ; car Dieu m’a comblé de grâce, et je ne manque de rien. Il le pressa tant qu'il le prit.
\VS{12}Esaü dit : Partons et marchons, et je marcherai devant toi.
\VS{13}Mais Jacob lui dit : Mon seigneur sait que ces enfants sont jeunes et que j’ai des brebis et des vaches qui allaitent ; si l’on forçait leur marche un seul jour, tout le troupeau mourra.
\VS{14}Je te prie que mon seigneur passe devant son serviteur, et je m’avancerai tout doucement, au pas de ce bétail qui est devant moi, et au pas de ces enfants, jusqu'à ce que j'arrive chez mon seigneur à Séir.
\VS{15}Esaü dit : Je te prie, je vais au moins laisser avec toi une partie de ce peuple qui est avec moi ; et il répondit : Pourquoi cela ? Je te prie que je trouve grâce aux yeux de mon seigneur.
\VS{16}Ainsi Esaü  retourna ce jour-là par son chemin à Séir.
\TextTitle{Jacob dresse un autel à El-Elohé-Israël (Dieu Fort, Dieu d'Israël)}
\VS{17}Jacob partit pour Succoth. Il bâtit une maison pour lui, et il fit des cabanes pour son bétail. C’est pourquoi il appela ce lieu du nom de Succoth.
\VS{18}A son retour de  Paddan-Aram, Jacob arriva sain et sauf à la ville de Sichem, dans le pays de Canaan, et il campa devant la ville.
\VS{19}Il acheta une portion du champ où il avait dressé sa tente de la main des fils de Hamor, père de Sichem, pour cent pièces d'argent.
\VS{20}Et là, il dressa un autel qu'il appela El-Elohé-Israël (le Dieu Fort, le Dieu d'Israël).
\Chap{34}
\TextTitle{Déshonneur de Dina et vengeance de ses frères}
\VerseOne{}Or Dina, la fille que Léa avait enfantée à Jacob, sortit pour voir les filles du pays.
\VS{2}Elle fut aperçue de Sichem, fils de Hamor, le Hévien, prince du pays. Il l'enleva et coucha avec elle, et la déshonora.
\VS{3}Son cœur fut attaché à Dina, fille de Jacob ; il aima la jeune fille et sut parler au cœur de la jeune fille.
\VS{4}Et Sichem parla à Hamor, son père, en disant : Prends-moi cette fille pour femme.
\VS{5}Jacob apprit qu'il avait déshonoré Dina, sa fille. Or ses fils étaient avec son bétail aux champs ; Jacob garda le silence jusqu’à leur retour.
\VS{6}Hamor, père de Sichem, sortit vers Jacob pour lui  parler.
\VS{7}Et les fils de Jacob revinrent des champs dès qu’ils apprirent ce qui était arrivé ; ces hommes furent dans  une grande douleur, et furent fort irrités de l'infamie que Sichem avait commise contre Israël, en couchant avec la fille de Jacob, ce qui ne devait point se faire.
\VS{8}Hamor leur parla en disant : L’âme de Sichem, mon fils, s’est attachée à votre fille ; donnez-la-lui je vous prie pour femme.
\VS{9}Alliez-vous avec nous, vous nous donnerez vos filles, et vous prendrez pour vous les nôtres.
\VS{10}Vous habiterez avec nous, et le pays sera à votre disposition ; restez pour y trafiquer et y acquérir des possessions.
\VS{11}Sichem dit aussi au père et aux frères de la fille : Que je trouve grâce à vos yeux, et je donnerai tout ce que vous me direz.
\VS{12}Exigez de moi une forte dot, et beaucoup de présents que vous voudrez, et je les donnerai comme vous me direz ; et donnez-moi la jeune fille pour femme.
\VS{13}Alors les fils de Jacob répondirent avec ruse à Sichem et à Hamor, son père ; ils parlèrent ainsi parce que Sichem avait déshonoré Dina, leur sœur.
\VS{14}Ils leur dirent : C’est une chose que nous ne pouvons pas faire, que de donner notre sœur à un homme incirconcis, car ce serait un opprobre pour nous.
\VS{15}Mais nous ne consentirons à ce que vous demandez que si vous deveniez semblables à nous en circoncisant tous les mâles qui sont parmi vous.
\VS{16}Alors nous vous donnerons nos filles, et nous prendrons vos filles pour nous, et nous habiterons avec vous, et nous ne serons qu'un seul peuple.
\VS{17}Mais si vous ne voulez pas nous écouter et vous circoncire, nous prendrons notre fille et nous nous en irons.
\VS{18}Leurs discours plurent à Hamor et à Sichem, fils d'Hamor.
\VS{19}Le jeune homme ne tarda point à faire ce qu'on lui avait proposé, car la fille de Jacob lui plaisait beaucoup ; et il était le plus considéré de tous ceux de la maison de son père.
\VS{20}Hamor et Sichem, son fils, se rendirent à la porte de leur ville et parlèrent aux gens de leur ville en leur disant :
\VS{21}Ces hommes sont paisibles à notre égard ; qu'ils habitent dans le pays et qu'ils y trafiquent ; car voici, le pays est assez vaste pour eux. Nous prendrons pour femmes leurs filles, et nous leur donnerons nos filles.
\VS{22}Mais ces hommes ne consentiront à habiter avec nous, pour former un seul peuple, que si tout mâle qui est parmi nous est circoncis, comme ils sont eux-mêmes circoncis.
\VS{23}Leur bétail, et leurs biens, et toutes leurs bêtes, ne seront-ils pas à nous ? Accordons-leur seulement cela, et qu'ils demeurent avec nous.
\VS{24}Tous ceux qui sortaient par la porte de leur ville obéirent à Hamor et à Sichem, son fils ; et tout mâle d'entre tous ceux qui sortaient par la porte de leur ville fut circoncis.
\VS{25}Le troisième jour, pendant qu’ils étaient souffrants, deux des fils de Jacob, Siméon et Lévi, frères de Dina, prirent leurs épées, entrèrent hardiment dans la ville et tuèrent tous les mâles.
\VS{26}Ils passèrent aussi au tranchant de l'épée Hamor et Sichem, son fils ; ils enlevèrent Dina de la maison de Sichem, et sortirent.
\VS{27}Les fils de Jacob se jetèrent sur les morts et pillèrent la ville, parce qu'on avait déshonoré leur sœur.
\VS{28}Ils prirent leurs troupeaux, leurs bœufs, leurs ânes, et ce qui était dans la ville et dans les champs ;
\VS{29}et toutes leurs richesses, leurs petits enfants, et ils emmenèrent prisonnières leurs femmes ; et ils les pillèrent avec tout ce qui était dans les maisons.
\VS{30}Alors Jacob dit à Siméon et Lévi : Vous m'avez troublé en me rendant odieux aux habitants du pays, aux Cananéens et aux Phérésiens, et je n'ai qu’un petit nombre d’hommes ; ils s'assembleront contre moi, et me frapperont, et me détruiront, moi et ma maison.
\VS{31}Ils répondirent : Doit-on traiter notre sœur comme une prostituée ?
\Chap{35}
\TextTitle{Jacob revient à Béthel pour adorer Yahweh}
\VerseOne{}Or Dieu dit à Jacob : Lève-toi, monte à Béthel, et demeures-y ; là, tu y dresseras un autel au Dieu qui t'apparut lorsque tu fuyais Esaü, ton frère.
\VS{2}Jacob dit à sa famille, et à tous ceux qui étaient avec lui : Otez les dieux des étrangers qui sont au milieu de vous, purifiez-vous, et changez de vêtements\FTNT{Jos. 24:23.}.
\VS{3}Levons-nous et montons à Béthel ; là je dresserai un autel au Dieu qui m'a exaucé dans le jour de ma détresse, et qui a été avec moi dans le chemin où j'ai marché.
\VS{4}Alors ils donnèrent à Jacob tous les dieux des étrangers qui étaient entre leurs mains, et les anneaux qui étaient à leurs oreilles, et il les cacha sous un térébinthe qui est près de Sichem.
\VS{5}Puis ils partirent. Et Dieu frappa de terreur les villes qui les entouraient, et l’on ne poursuivit point les fils de Jacob.
\VS{6}Ainsi Jacob, et tout le peuple qui était avec lui, arrivèrent à Luz, qui est Béthel, dans le pays de Canaan.
\VS{7}Il bâtit là un autel, et il appela ce lieu El-Béthel (le Dieu Puissant de Béthel) ; car c’est là que Dieu s’était révélé à lui lorsqu’il fuyait son frère.
\VS{8}Débora,  nourrice de Rebecca, mourut ; et elle fut ensevelie au-dessous de Béthel sous un chêne, auquel on donna le nom d’Allon-Bacuth (chêne des pleurs).
\VS{9}Dieu apparut encore à Jacob, après son retour de Paddan-Aram, et il le bénit\FTNT{Os. 12:5.}.
\VS{10}Dieu lui dit : Ton nom est Jacob, mais tu ne seras plus appelé Jacob, car ton nom sera Israël. Et il lui donna le nom d’Israël.
\VS{11}Dieu lui dit aussi : Je suis le Dieu Fort, Tout-Puissant. Sois fécond et multiplie : Une nation et une multitude de nations naîtront de toi, et des rois sortiront de tes reins.
\VS{12}Je te donnerai le pays que j'ai donné à Abraham et à Isaac, et je le donnerai à ta postérité après toi.
\VS{13}Dieu s’éleva au-dessus de lui dans le lieu où il lui avait parlé.
\VS{14}Et Jacob dressa un monument dans le lieu où Dieu lui avait parlé, à savoir un monument de pierre, et il fit dessus une aspersion et y versa de l'huile.
\VS{15}Jacob donna le nom de Béthel au lieu où Dieu lui avait parlé.
\VS{16}Puis ils partirent de Béthel, et il y avait encore une certaine distance jusqu’à Ephrata\FTNT{Ephrata : «~lieu de la fécondité~».} lorsque Rachel accoucha. Elle eut un accouchement difficile ;
\VS{17}et comme elle avait beaucoup de peine à accoucher, la sage-femme lui dit : Ne crains point, car tu as encore un fils.
\VS{18}Et comme elle rendait l'âme, car elle était mourante, elle lui donna le nom de Ben-Oni\FTNT{Ben-Oni : «~fils de ma douleur~».}, mais son père l’appela Benjamin\FTNT{Benjamin : «~fils de ma main droite~», «~fils de félicité~».}.
\VS{19}C'est ainsi que mourut Rachel, et elle fut ensevelie sur le chemin d'Ephrata, qui est Bethléhem.
\VS{20}Jacob dressa un monument sur son sépulcre. C'est le monument du sépulcre de Rachel qui subsiste encore aujourd'hui.
\VS{21}Puis Israël partit et dressa ses tentes au-delà de Migdal-Eder.
\VS{22}Pendant qu’Israël habitait dans ce pays, Ruben alla coucher avec Bilha, concubine de son père.  Et Israël l'apprit. Or Jacob avait douze fils.
\VS{23}Les fils de Léa étaient Ruben, premier-né de Jacob, Siméon, Lévi, Juda, Issacar, et Zabulon.
\VS{24}Les fils de Rachel : Joseph et Benjamin.
\VS{25}Les fils de Bilha, servante de Rachel : Dan et Nephthali.
\VS{26}Les fils de Zilpa, servante de Léa : Gad et Aser. Ce sont là les enfants de Jacob qui lui naquirent à Paddan-Aram.
\TextTitle{Jacob voit vers son père Isaac avant sa mort}
\VS{27}Jacob arriva auprès d’Isaac, son père, à la plaine de Mamré, à Kirjath-Arba, qui est Hébron, où Abraham et Isaac avaient séjourné comme étrangers.
\VS{28}Les jours d’Isaac furent de cent quatre-vingts ans.
\VS{29}Isaac expira et mourut, et fut recueilli auprès de son peuple, âgé et rassasié de jours ; et Esaü et Jacob ses fils l'ensevelirent.
\Chap{36}
\TextTitle{Postérité d'Esaü (Edom)}
\VerseOne{}Et voici la postérité d'Esaü, qui est Edom.
\VS{2}Esaü prit ses femmes parmi les filles de Canaan, à savoir Ada, fille d'Elon, le Héthien, Oholibama, fille d’Ana, petite-fille de Tsibeon, le Hévien.
\VS{3}Il prit aussi Basmath, fille d'Ismaël, sœur de Nebajoth.
\VS{4}Ada enfanta à Esaü Eliphaz ; et Basmath enfanta Réuel.
\VS{5}Et Oholibama enfanta Jéusch, Jaelam et Koré. Ce sont là les enfants d'Esaü qui lui naquirent dans le  pays de Canaan.
\VS{6}Esaü prit ses femmes, ses fils et ses filles, et toutes les personnes de sa maison, tous ses troupeaux, ses bêtes, et tout le bien qu'il avait acquis dans le pays de Canaan, et il s'en alla dans un autre pays, loin de Jacob, son frère.
\VS{7}Car leurs richesses étaient si grandes qu'ils n'auraient pas pu demeurer ensemble ; et le pays où ils séjournaient comme étrangers ne pouvait plus les contenir à cause de leurs troupeaux.
\VS{8}Ainsi Esaü habita dans la montagne de Séir ; Esaü est Edom.
\VS{9}Voici la postérité d'Esaü, père d'Edom, dans la montagne de Séir.
\VS{10}Voici les noms des fils d'Esaü : Eliphaz fils d’Ada, femme d'Esaü ; Réuel, fils de Basmath, femme d'Esaü.
\VS{11}Les fils d'Eliphaz furent : Théman, Omar, Tsepho, Gaetham et Kenaz.
\VS{12}Et Timna était la concubine d'Eliphaz, fils d'Esaü, et elle enfanta à Eliphaz Amalek. Ce sont là les fils d’Ada, femme d'Esaü.
\VS{13}Voici les fils de Réuel : Nahath, Zérach, Schamma et Mizza. Ce sont là les fils de Basmath, femme d'Esaü.
\VS{14}Voici les fils d'Oholibama, fille d’Ada, petite fille de Tsibeon, femme d'Esaü ; elle enfanta à Esaü Jéusch, Jaelam et Koré.
\VS{15}Voici les chefs des fils d'Esaü. Voici les fils d'Eliphaz, premier-né d'Esaü, le chef Théman, le chef Omar, le chef Tsepho, le chef Kenaz,
\VS{16}le chef Koré, le chef Gaetham, le chef Amalek. Ce sont là les chefs d'Eliphaz dans le  pays d'Edom. Ce sont les fils d’Ada.
\VS{17}Voici les fils de Réuel, fils d'Esaü : le chef Nahath, le chef Zérach, le chef Schamma, et le chef Mizza. Ce sont là les chefs sortis de Réuel, dans le pays d'Edom.  Ce sont là les fils de Basmath, femme d'Esaü.
\VS{18}Voici les fils d'Oholibama, femme d'Esaü : Le chef Jéusch, le chef Jaelam, le chef Koré. Ce sont là les chefs sortis d'Oholibama, fille d’Ana, femme d'Esaü.
\VS{19}Ce sont là les fils d'Esaü, qui est Edom, et ce sont là leurs chefs.
\VS{20}Voici les fils de Séir, le Horien, qui avaient habité dans le pays : Lothan, Schobal, Tsibeon, Ana,
\VS{21}Dischon, Etser, et Dischan. Ce sont là les chefs des Horiens, fils de Séir, dans le pays d'Edom.
\VS{22}Les fils de Lothan furent Hori et Héman.  Et Thimna était sœur de Lothan.
\VS{23}Voici les fils de Schobal : Alvan, Manahath, Ebal, Schepho et Onam.
\VS{24}Voici les fils de Tsibeon : Ajja et Ana. C’est cet Ana qui trouva les sources chaudes dans le désert, quand il faisait paître les ânes de Tsibeon, son père.
\VS{25}Voici les fils d’Ana : Dischon, et Oholibama, fille d’Ana.
\VS{26}Voici les fils de Dischon : Hemdan, Eschban, Jithran et Keran.
\VS{27}Voici les fils d'Etser : Bilhan, Zaavan et Akan.
\VS{28}Voci les fils de Dischan : Huts et Aran.
\VS{29}Voici les chefs des Horiens : Le chef Lothan, le chef Schobal, le chef Tsibeon, le chef Ana.
\VS{30}Le chef Dischon, le chef Etser, le chef Dischan. Ce sont là les chefs des Horiens, les chefs qu’ils établirent dans le pays de Séir.
\VS{31}Voici les rois qui ont régné dans le pays d'Edom, avant qu’un roi règne sur les enfants d'Israël.
\VS{32}Béla, fils de Béor, régna sur Edom, et le nom de sa ville était Dinhaba.
\VS{33}Béla mourut, et Jobab, fils de Zérach de Botsra, régna à sa place.
\VS{34}Jobab mourut, et Huscham, du pays des Thémanites, régna à sa place.
\VS{35}Huscham mourut, et Hadad, fils de Bédad, régna à sa place. C’est lui qui frappa Madian dans le territoire de Moab ; et le nom de sa ville était Avith.
\VS{36}Hadad mourut, et Samla, de Masréka, régna à sa place.
\VS{37}Samla mourut, et Saül de Réhoboth sur le fleuve, régna à sa place.
\VS{38}Saül mourut, et Baal-Hanan, fils d’Acbor, régna à sa place.
\VS{39}Baal-Hanan, fils de Hacbor mourut, et Hadar régna à sa place. Le nom de sa ville était Pau ; et le nom de sa femme Mehéthabeel, fille de Mathred, petite-fille de Mézahab.
\VS{40}Voici les noms des chefs d'Esaü selon leurs familles, selon leurs territoires, et d’après leurs noms : Le chef Thimna, le chef Alva, le chef Jétheth,
\VS{41}le chef Oholibama, le chef Ela, le chef Pinon,
\VS{42}Le chef Kenaz, le chef Théman, le chef Mibtsar,
\VS{43}le chef Magdiel, et le chef Iram. Ce sont là les chefs d'Edom, selon leurs habitations dans le pays qu’ils possédaient. C'est Esaü le père d'Edom.
\Chap{37}
\TextTitle{Jacob aime Joseph plus que ses autres fils}
\VerseOne{}Or Jacob demeura dans le pays de Canaan, pays où avait séjourné son père comme étranger.
\VS{2}Voici la postérité de Jacob. Joseph, âgé de dix-sept ans, faisait paître le troupeau avec ses frères ; et il était jeune garçon auprès des fils de Bilha et des fils de Zilpa, femmes de son père. Et Joseph rapportait à leur père leurs mauvais propos.
\VS{3}Or Israël aimait Joseph plus que tous ses autres fils, parce qu'il l'avait eu dans sa vieillesse, et il lui fit une tunique de plusieurs couleurs.
\VS{4}Ses frères voyant que leur père l'aimait plus qu'eux tous, le haïssaient et ne pouvaient lui parler paisiblement.
\VS{5}Joseph eut un songe et il raconta à ses frères ; et ils le haïrent encore davantage.
\VS{6}Il leur dit donc : Ecoutez, je vous prie, le songe que j'ai eu.
\VS{7}Voici, nous étions à lier des gerbes au milieu d'un champ ; et voici, ma gerbe se leva et se tint droite ; et voici, vos gerbes l’entourèrent et se prosternèrent devant elle.
\TextTitle{Joseph haï par ses frères}
\VS{8}Alors ses frères lui dirent : Régnerais-tu sur nous ? Et dominerais-tu sur nous ? Et ils le haïrent encore plus pour ses songes et pour ses paroles.
\VS{9}Il eut encore un autre songe, et il le raconta à ses frères, en disant : Voici, j'ai eu encore un songe ; et voici, le soleil, la lune et onze étoiles se prosternaient devant moi\FTNT{Ap. 12:1.}.
\VS{10}Il le raconta à son père et à ses frères. Son père le réprimanda et lui dit : Que veut dire ce songe que tu as eu ? Faut-il que nous venions moi, ta mère, et tes frères, nous prosterner à terre devant toi ?
\VS{11}Ses frères eurent de l'envie contre lui, mais son père garda ses discours\FTNT{Ac. 7:9.}.
\VS{12}Les frères de Joseph s'en allèrent paître les troupeaux de leur père à Sichem.
\VS{13}Israël dit à Joseph : Tes frères ne font-ils pas paître le troupeau à Sichem ? Viens, que je t'envoie vers eux ; et il lui répondit : Me voici.
\VS{14}Israël lui dit : Va maintenant, vois si tes frères se portent bien, et si le troupeau est en bon état, et rapporte-le-moi. Ainsi il l'envoya de la vallée d’Hébron, et il alla jusqu'à Sichem.
\VS{15}Un homme le rencontra, comme il errait dans les champs ; et cet homme le questionna et lui dit : Que cherches-tu ?
\VS{16}Joseph répondit : Je cherche mes frères ; je te prie, dis-moi où ils font paître leur troupeau.
\VS{17}Et l'homme dit : Ils sont partis d'ici, et je les ai entendus dire : Allons à Dothan. Joseph alla après ses frères et les trouva à Dothan.
\VS{18}Ils le virent de loin ; et avant qu'il soit près d’eux, ils complotèrent contre lui pour le tuer.
\VS{19}Ils se dirent l'un à l'autre : Voici ce maître songeur qui arrive.
\TextTitle{Joseph dans la citerne}
\VS{20}Venez maintenant, tuons-le, et jetons-le dans l’une de ces citernes ; et nous dirons qu'une bête féroce l'a dévoré, et nous verrons ce que deviendront ses songes.
\VS{21}Mais Ruben entendit cela et le délivra de leurs mains en disant : Ne lui ôtons point la vie.
\VS{22}Ruben leur dit encore : Ne répandez point le sang ; jetez-le dans cette citerne qui est au désert, mais ne mettez point la main sur lui. C'était pour le délivrer de leurs mains et le renvoyer à son père.
\VS{23}Lorsque Joseph fut arrivé auprès de ses frères, ils le dépouillèrent de sa tunique, de cette tunique de plusieurs couleurs qui était sur lui.
\VS{24}Ils le prirent et le jetèrent dans la citerne.  Cette citerne était vide, il n'y avait point d'eau.
\VS{25}Ensuite, ils s'assirent pour manger du pain ; et levant les yeux, ils virent une caravane d'Ismaélites qui passait et qui venait de Galaad ; et leurs chameaux étaient chargés d’aromates, du baume et de la myrrhe, qu’ils transportaient en Egypte.
\VS{26}Et Juda dit à ses frères : Que gagnerons-nous à tuer notre frère et à cacher son sang ?
\VS{27}Venez, vendons-le à ces Ismaélites, et ne mettons point notre main sur lui, car il est notre frère, notre chair ; et ses frères lui obéirent.
\TextTitle{Joseph vendu à des marchands et emmené en Egypte}
\VS{28}Et comme les marchands Madianites passaient, ils tirèrent et firent remonter Joseph de la citerne, et le vendirent pour vingt pièces d'argent aux Ismaélites, qui emmenèrent Joseph en Egypte\FTNT{Ps. 105:17.}.
\VS{29}Puis Ruben revint à la citerne, et voici, Joseph n'était plus dans la citerne. Alors il déchira ses vêtements.
\VS{30}Il retourna vers ses frères et leur dit : L'enfant n’y est plus ! Et moi ! Moi ! Où irai-je ?
\VS{31}Ils prirent la tunique de Joseph et tuèrent un bouc d'entre les chèvres, ils plongèrent la tunique dans le sang.
\VS{32}Puis ils envoyèrent et firent porter à leur père la tunique de plusieurs couleurs, en lui disant : Voici ce que nous avons trouvé ! Reconnais maintenant si c'est la tunique de ton fils ou non.
\VS{33}Jacob la reconnut, et dit : C'est la tunique de mon fils ! Une bête féroce l'a dévoré ! Certainement Joseph a été déchiré !
\VS{34}Et Jacob déchira ses vêtements, il mit un sac sur ses reins, et il porta le deuil de son fils durant plusieurs jours.
\VS{35}Tous ses fils et toutes ses filles vinrent pour le consoler, mais il rejeta toute consolation. Il disait : C’est en pleurant que je descendrai vers mon fils dans le scheol ! C'est ainsi que son père le pleurait.
\VS{36}Les Madianites le vendirent en Egypte à Potiphar, eunuque de Pharaon, chef des gardes.
\Chap{38}
\TextTitle{Péché de Juda}
\VerseOne{}Il arriva qu’en ce temps-là, Juda s’éloigna de ses frères et se retira vers un homme d’Adullam, nommé Hira.
\VS{2}Là, Juda vit la fille d'un Cananéen, nommé Schua, il la prit pour femme et alla vers elle.
\VS{3}Elle conçut et enfanta un fils qu’elle appela Er.
\VS{4}Elle conçut encore et enfanta un fils qu’elle appela Onan.
\VS{5}Elle enfanta de nouveau un fils qu’elle appela Schéla. Juda était à Czib quand elle l’enfanta.
\VS{6}Juda prit une femme pour Er, son premier-né, une femme nommée Tamar.
\VS{7}Mais Er le premier-né de Juda était méchant devant Yahweh, et Yahweh le fit mourir\FTNT{No. 26:19.}.
\VS{8}Alors Juda dit à Onan : Va vers la femme de ton frère, et prends-la pour femme, comme tu es son beau-frère, et suscite des enfants à ton frère\FTNT{Lé. 25:25  ; Lé. 25:48. Voir commentaire en Ru. 2:20.}.
\VS{9}Mais Onan, sachant que les enfants ne seraient pas à lui, se souillait à terre lorsqu’il allait vers la femme de son frère, afin de ne pas donner de postérité à son frère.
\VS{10}Ce qu'il faisait déplut à Yahweh, c'est pourquoi il le fit aussi mourir.
\VS{11}Et Juda dit à Tamar, sa belle-fille : Demeure veuve dans la maison de ton père, jusqu'à ce que Schéla, mon fils, soit grand ; car il dit : Il faut prendre garde qu'il ne meure comme ses frères. Ainsi Tamar s'en alla et demeura dans la maison de son père.
\VS{12}Et après plusieurs jours, la fille de Schua, femme de Juda, mourut ; lorsque Juda fut consolé, il monta vers ceux qui tondaient ses brebis à Thimna, avec Hira, l’Adullamite, son ami intime.
\VS{13}On en informa Tamar et on lui dit : Voici, ton beau-père monte à Thimna pour tondre ses brebis.
\VS{14}Alors elle ôta ses habits de veuve, se couvrit d'un voile, et s'enveloppa, et elle s’assit à l’entrée d’Enaïm, sur le chemin de Thimna ; car elle voyait que Schéla était devenu grand et qu’elle ne lui était point donnée pour femme.
\VS{15}Et quand Juda la vit, il s'imagina que c'était une prostituée, car elle avait couvert son visage.
\VS{16}Il l’aborda sur le chemin et lui dit : Permets, je te prie, que je vienne vers toi ; car il ne savait pas que c’était sa belle-fille. Et elle répondit : Que me donneras-tu pour venir vers moi ?
\VS{17}Il répondit : Je t'enverrai un chevreau d'entre les chèvres du troupeau. Elle répondit : Me donneras-tu un gage jusqu'à ce que tu l'envoies ?
\VS{18}Il répondit : Quel gage te donnerai-je ? Et elle répondit : Ton cachet, ton cordon, et ton bâton que tu as à la main. Et il les lui donna. Il alla vers elle, et elle devint enceinte de lui.
\VS{19}Puis elle se leva et s'en alla ; elle ôta son voile et remit ses habits de veuve.
\VS{20}Juda envoya un chevreau d'entre ses chèvres par son ami intime l’Adullamite, pour qu'il retire le gage de la main de la femme, mais il ne la trouva point.
\VS{21}Il interrogea les hommes du lieu où elle avait été, en disant : Où est cette prostituée qui était à Enaïm, sur le chemin ? Ils répondirent : Il n'y a point eu ici de prostituée.
\VS{22}Il retourna auprès de Juda et lui dit : Je ne l'ai point trouvée ; et même les gens du lieu m'ont dit : Il n'y a point eu ici de prostituée.
\VS{23}Juda dit : Qu'elle garde le gage, il ne faut pas nous faire mépriser. Voici, j'ai envoyé ce chevreau, mais tu ne l'as point trouvée.
\VS{24}Environ trois mois après, on fit un rapport à Juda, en disant : Tamar, ta belle-fille, a commis un adultère, et voici elle est même enceinte. Et Juda dit : Faites-la sortir, et qu'elle soit brûlée.
\VS{25}Comme on la faisait sortir, elle envoya dire à son beau-père : Je suis enceinte de l'homme à qui ces choses appartiennent. Elle dit aussi : Reconnais, je te prie, à qui est ce cachet, ce cordon, et ce bâton.
\VS{26}Alors Juda les reconnut et il dit : Elle est plus juste que moi, parce que je ne l'ai point donnée à Schéla, mon fils ; et il ne la connut plus.
\VS{27}Quand elle fut au moment d'accoucher, voici, des jumeaux étaient dans son ventre.
\VS{28}Et pendant qu’elle accouchait, il y en eut un qui présenta la main ; la sage-femme la prit et y attacha un fil cramoisi, en disant : Celui-ci sort le premier.
\VS{29}Mais il retira la main, et son frère sortit. Alors la sage-femme dit : Quelle brèche tu as faite ! Et elle lui donna le nom de Pérets.
\VS{30}Ensuite sortit son frère, qui avait à la main le fil cramoisi ; et on lui donna le nom de Zérach.
\Chap{39}
\TextTitle{Joseph fidèle à Yahweh devant la tentation}
\VerseOne{}Or, quand on fit descendre Joseph en Egypte, Potiphar, eunuque de Pharaon, chef des gardes, Egyptien, l'acheta de la main des Ismaélites qui l'y avaient amené.
\VS{2}Yahweh était avec Joseph ; et il prospéra, et demeura dans la maison de son maître,  l’Egyptien.
\VS{3}Son maître vit que Yahweh était avec lui, et que Yahweh faisait prospérer entre ses mains tout ce qu'il faisait.
\VS{4}C'est pourquoi Joseph trouva grâce aux yeux de son maître, qui l’employa à son service. Et son maître l'établit sur sa maison, et lui remit entre les mains tout ce qui lui appartenait.
\VS{5}Dès que Potiphar l’eut établi sur sa maison et sur tout ce qu’il possédait, Yahweh bénit la maison de l’Egyptien, à cause de Joseph ; et la bénédiction de Yahweh fut sur tout ce qui lui appartenait, soit à la maison, soit aux champs.
\VS{6}Il abandonna aux mains de Joseph tout ce qui lui appartenait, et il n’avait avec lui d’autre soin que celui de prendre sa nourriture. Or Joseph était beau de  taille et beau de figure.
\VS{7}Après ces choses, il arriva que la femme de son maître porta les yeux sur Joseph, et elle lui dit : Couche avec moi\FTNT{Pr. 7:9-13.} !
\VS{8}Mais il le refusa, et dit à la femme de son maître : Voici, mon maître ne prend avec moi connaissance de rien dans la maison, et il a remis entre mes mains tout ce qui lui appartient.
\VS{9}Il n'y a personne dans cette maison qui soit plus grand que moi, et il ne m'a rien interdit excepté toi, parce que tu es sa femme ; et comment ferais-je un si grand mal et pécherais-je contre Dieu ?
\VS{10}Quoiqu’elle parlât tous les jours à Joseph, il refusa de coucher auprès d’elle, d’être avec elle.
\VS{11}Un jour qu'il était entré dans la maison pour faire son ouvrage, et qu'il n'y avait là aucun des gens dans la maison,
\VS{12}elle le saisit par son vêtement et lui dit : Couche avec moi ! Mais il laissa son vêtement entre ses mains, s'enfuit, et sortit dehors\FTNT{1 Co. 6:18.}.
\TextTitle{Fausse accusation contre Joseph}
\VS{13}Et lorsqu'elle vit qu'il lui avait laissé son vêtement entre les mains, et qu'il s'était enfui dehors,
\VS{14}elle appela les gens de sa maison, et leur parla en disant : Voyez, on nous a amené un Hébreu pour se moquer de nous.  Cet homme est venu vers moi pour coucher avec moi ; mais j'ai crié à haute voix.
\VS{15}Et dès qu’il a entendu que j’élevais la voix et que je  criais, il a laissé son vêtement à côté de moi, et s’est enfui dehors.
\VS{16}Et elle garda le vêtement de Joseph jusqu'à ce que son maître rentre à la maison.
\VS{17}Alors elle lui parla en ces mêmes termes et dit : Le serviteur Hébreu que tu nous as amené est venu vers moi pour se moquer de moi.
\VS{18}Mais comme j'ai élevé ma voix et que j'ai crié, il a laissé son vêtement à côté de moi et s'est enfui.
\VS{19}Et dès que le maître de Joseph eut entendu les paroles de sa femme qui lui disait : Ton serviteur m'a fait ce que je t'ai dit, sa colère s'enflamma.
\VS{20}Et le maître de Joseph le prit et le mit dans une étroite prison ; dans l'endroit où les prisonniers du roi étaient enfermés, et il fut là en prison.
\VS{21}Mais Yahweh fut avec Joseph ; il étendit sa bonté sur lui et lui fit trouver grâce auprès du chef de la prison.
\VS{22}Et le chef de la prison mit entre les mains de Joseph tous les prisonniers qui étaient dans la prison, et tout ce qu'il y avait à faire, il le faisait.
\VS{23}Le chef de la prison ne prenait aucune connaissance de ce que Joseph avait en main, parce que Yahweh était avec lui. Et Yahweh faisait prospérer tout ce qu'il faisait.
\Chap{40}
\TextTitle{Joseph demeure en prison}
\VerseOne{}Après ces choses, il arriva que l'échanson et le panetier du roi d'Egypte offensèrent leur maître, le roi d'Egypte.
\VS{2}Pharaon fut fort irrité contre ces deux eunuques, contre le chef des échansons, et contre le chef des panetiers.
\VS{3}Et il les fit mettre dans la maison du chef des gardes, dans la prison étroite, dans le même lieu où Joseph était enfermé.
\VS{4}Le chef des gardes les mit entre les mains de Joseph qui les servait ; et ils furent quelques jours en prison.
\VS{5}Pendant une même nuit, l’échanson et le panetier du roi d’Egypte, qui étaient enfermés dans la prison, eurent tous les deux un songe, chacun le sien, pouvant recevoir une explication distincte.
\VS{6}Joseph, étant venu le matin vers eux, les regarda ; et voici, ils étaient fort tristes.
\VS{7}Et il interrogea ces deux eunuques de Pharaon, qui étaient avec lui dans la prison de son maître, et leur dit : Pourquoi avez-vous mauvais visage aujourd'hui ?
\VS{8}Ils lui répondirent : Nous avons eu des songes, et il n'y a personne qui les interprète. Et Joseph leur dit : Les interprétations n’appartiennent-elles pas à Dieu ? Je vous prie, racontez-moi vos songes\FTNT{1 Co. 12:8-10 ; Job. 33:15.}.
\VS{9}Le chef des échansons raconta son songe à Joseph et lui dit : Dans mon songe, voici, il y avait un cep devant moi.
\VS{10}Ce cep avait trois sarments. Quand il eut poussé, sa fleur se développa et ses grappes donnèrent des raisins mûrs.
\VS{11}La coupe de Pharaon était dans ma main. Je pris les raisins, je les pressai dans la coupe de Pharaon, et je mis la coupe dans la main de Pharaon.
\VS{12}Joseph lui dit : Voici son interprétation : Les trois sarments sont trois jours.
\VS{13}Dans trois jours Pharaon élèvera ta tête et te rétablira dans ta charge, et tu mettras la coupe dans sa main, comme tu le faisais auparavant, lorsque tu étais son échanson.
\VS{14}Mais souviens-toi de moi quand tu  seras heureux, et use de bonté envers moi je te prie ; fais mention de moi à Pharaon, afin  qu’il me fasse sortir de cette maison.
\VS{15}Car certainement j'ai été enlevé du pays des Hébreux ; ici non plus je n’ai rien fait  pour  être mis en prison.
\VS{16}Le chef des panetiers, voyant que Joseph avait interprété favorablement ce songe, lui dit : Voici, il y avait aussi dans mon songe trois corbeilles de pain blanc sur ma tête.
\VS{17}Dans la corbeille la plus élevée, il y avait pour Pharaon des mets de toute espèce, cuits au four ; et les oiseaux les mangeaient dans la corbeille au-dessus de ma tête.
\VS{18}Joseph répondit et dit : Voici son interprétation : Les trois corbeilles sont trois jours.
\VS{19}Dans trois jours Pharaon enlèvera ta tête de dessus toi et te fera pendre à un bois, et les oiseaux mangeront ta chair sur toi.
\VS{20}Le troisième jour, jour de la naissance de Pharaon, il fit un festin à tous ses serviteurs ; et il éleva la tête du chef des échansons et la tête du chef des panetiers, au milieu de ses serviteurs.
\VS{21}Il rétablit le chef des échansons dans sa charge d’échanson, pour qu’il mette la coupe dans la main de Pharaon.
\VS{22}Mais il fit pendre le chef des panetiers, selon l’explication que Joseph leur avait donnée.
\VS{23}Cependant, le chef des échansons ne pensa plus à Joseph. Il l’oublia.
\Chap{41}
\TextTitle{Les songes de Pharaon}
\VerseOne{}Mais il arriva qu’au bout de deux ans entiers, Pharaon eut un songe. Et il lui semblait qu'il était près du fleuve.
\VS{2}Et voici, sept jeunes vaches belles à voir, grasses de chair, montèrent hors du fleuve et se mirent à paître dans les  prairies.
\VS{3}Et voici sept autres jeunes vaches, laides à voir, et maigres de chair, montèrent hors du fleuve derrière les autres et se tinrent auprès des autres jeunes vaches sur le bord du fleuve.
\VS{4}Les jeunes vaches laides à voir, et maigres, mangèrent les sept jeunes vaches belles à voir, et grasses. Alors Pharaon s'éveilla.
\VS{5}Il se rendormit et il eut un second songe. Voici, sept épis gras et beaux montèrent sur une même tige.
\VS{6}Et sept épis maigres et brûlés par le vent d’orient poussèrent après eux.
\VS{7}Les épis maigres engloutirent les sept épis gras et pleins. Et Pharaon s'éveilla ; et voilà le songe.
\VS{8}Le matin, Pharaon eut l’esprit troublé, et il envoya appeler tous les magiciens et tous les sages d'Egypte, et leur raconta ses songes.  Mais personne ne put les interpréter à Pharaon.
\VS{9}Alors le chef des échansons parla à Pharaon en disant : Je rappellerai aujourd'hui le souvenir de mes fautes.
\VS{10}Lorsque Pharaon fut irrité contre ses serviteurs, et nous fit mettre, le chef des panetiers et moi, en prison, dans la maison du chef des gardes,
\VS{11}nous eûmes l’un et l’autre un songe dans une même nuit ; et chacun de nous reçut une interprétation en rapport avec le songe qu’il avait eu.
\VS{12}Il y avait là avec nous un garçon Hébreu, esclave du chef des gardes. Nous lui racontâmes nos songes, et il nous les expliqua.
\VS{13}Les choses sont arrivées comme il nous les avait interprétées ; car le roi me rétablit dans ma charge et fit pendre le chef des panetiers.
\TextTitle{Joseph sort de prison et est établi sur l'Egypte par Pharaon}
\VS{14}Alors Pharaon envoya appeler Joseph.  On le fit sortir en hâte de la prison ; on le rasa, et on lui fit changer de vêtements ; puis il se rendit vers Pharaon.
\VS{15}Pharaon dit à Joseph : J'ai eu un songe, et personne ne peut  l'expliquer ; or j'ai appris que tu sais expliquer les songes.
\VS{16}Joseph répondit à Pharaon en disant : Ce n’est pas moi !  C’est Dieu qui donnera une réponse concernant la paix de Pharaon.
\VS{17}Pharaon dit alors à Joseph : Dans mon songe, voici, je me tenais sur le bord du fleuve.
\VS{18}Et voici, sept vaches grasses de chair et belles d’apparence montèrent hors du fleuve et se mirent à paître dans la prairie.
\VS{19}Sept autres vaches montèrent derrière elles, maigres, fort laides d’apparence, et décharnées ; je n’en ai point vu d’aussi laides dans tout le pays d’Egypte.
\VS{20}Les vaches décharnées et laides mangèrent les sept premières vaches qui étaient grasses ;
\VS{21}elles les engloutirent dans leur ventre, sans qu’on s’aperçoive qu’elles y étaient entrées ; et leur apparence était laide comme auparavant. Et je m’éveillai.
\VS{22}Je vis encore en songe sept épis pleins et beaux, qui montèrent sur une même tige.
\VS{23}Et sept épis vides, maigres, brûlés par le vent d’orient, poussèrent après eux.
\VS{24}Les épis maigres engloutirent les sept beaux épis. Je l’ai dit aux magiciens, mais personne ne m’a donné l’explication. 25 Joseph dit à Pharaon : Ce qu’a rêvé Pharaon est une seule chose ; Dieu a fait connaître à Pharaon ce qu’il va faire.
\VS{26}Les sept vaches belles sont sept années ; et les sept épis beaux sont sept années ; c’est un seul songe.
\VS{27}Les sept vaches décharnées et laides, qui montaient derrière les premières, sont sept années ; et les sept épis vides, brûlés par le vent d’orient, seront sept années de famine.
\VS{28}Ainsi, comme je viens de le dire à Pharaon, Dieu a fait connaître à Pharaon ce qu’il va faire.
\VS{29}Voici, il y aura sept années de grande abondance dans tout le pays d’Egypte.
\VS{30}Sept années de famine viendront après elles ; et l’on oubliera toute cette abondance au pays d’Egypte, et la famine consumera le pays.
\VS{31}Cette famine qui suivra sera si forte qu’on ne s’apercevra plus de l’abondance dans le pays.
\VS{32}Si Pharaon a vu le songe se répéter une seconde fois, c’est que la chose est arrêtée de la part de Dieu, et que Dieu se hâtera de l’exécuter.
\VS{33}Maintenant que Pharaon choisisse un homme intelligent et sage, et qu'il l'établisse sur le pays d'Egypte.
\VS{34}Que Pharaon établisse et institue des commissaires sur le pays, et qu'ils prennent la cinquième partie du revenu du pays d'Egypte durant les sept années d'abondance.
\VS{35}Qu’ils rassemblent tous les produits de ces bonnes années  qui viennent ; qu’ils fassent sous l’autorité de Pharaon des amas de blé, des approvisionnements dans les villes, et qu’ils en aient la garde.
\VS{36}Ces provisions seront en réserve pour le pays durant les sept années de famine qui seront dans le  pays d'Egypte, afin que le pays ne soit pas consumé par la famine.
\VS{37}Ces paroles plurent à Pharaon et à tous ses serviteurs\FTNT{Ac. 7:10.}.
\VS{38}Et Pharaon dit à ses serviteurs : Trouverions-nous un homme semblable à celui-ci, qui a l'Esprit de Dieu ?
\VS{39}Et Pharaon dit à Joseph : Puisque Dieu t'a fait connaître toutes ces choses, il n'y a personne qui soit aussi intelligent et aussi sage que toi.
\VS{40}C’est toi qui seras sur ma maison, et tout mon peuple obéira à tes ordres ; je serai seulement plus grand que toi par le trône.
\VS{41}Pharaon dit encore à Joseph : Regarde, je t’établis sur tout le pays d'Egypte.
\VS{42}Alors Pharaon ôta son anneau de sa main et le mit à la main de Joseph ; il le fit revêtir d'habits de fin lin et lui mit un collier d'or au cou.
\VS{43}Il le fit monter sur le char qui suivait le sien, et on criait devant lui : A genoux ! Et il l'établit sur tout le pays d'Egypte.
\VS{44}Et Pharaon dit à Joseph : Je suis Pharaon ! Et sans toi nul ne lèvera la main ni le pied dans tout le pays d'Egypte.
\TextTitle{Joseph épouse une égyptienne}
\VS{45}Pharaon appela Joseph du nom de Tsaphnath-Paenéach ; et il lui donna pour femme Asnath, fille de Poti-Phéra, prêtre d'On. Et Joseph alla visiter le pays d'Egypte.
\VS{46}Joseph était âgé de trente ans lorsqu’il se présenta devant Pharaon, roi d'Egypte ; et il quitta Pharaon et parcourut tout le pays d'Egypte.
\VS{47}Et la terre rapporta très abondamment pendant les sept années de fertilité.
\VS{48}Joseph rassembla tous les produits de ces sept années dans le pays d’Egypte ; il fit des approvisionnements dans les villes, mettant dans l’intérieur de chaque ville les productions des champs d’alentour.
\VS{49}Ainsi Joseph amassa une grande quantité de blé, comme le sable de la mer ; tellement qu'on cessa de le compter, parce qu’il n’y avait plus de nombre.
\VS{50}Avant les années de famine, il naquit à Joseph deux fils, que lui enfanta Asnath, fille de Poti-Phéra, prêtre  d'On.
\VS{51}Joseph donna au premier-né le nom de Manassé, parce que, dit-il, Dieu m'a fait oublier toute ma peine et toute la maison de mon père.
\VS{52}Et il donna au second le nom d’Ephraïm, parce que, dit-il, Dieu m'a fait fructifier dans le  pays de mon affliction.
\VS{53}Alors finirent les sept années de l'abondance qui avaient été dans le pays d'Egypte.
\VS{54}Et les sept années de la famine commencèrent à venir comme Joseph l'avait prédit. Et la famine fut dans tous les pays ; mais il y avait du pain dans tout le pays d'Egypte.
\VS{55}Ensuite tout le pays d'Egypte fut affamé, et le peuple cria à Pharaon pour avoir du pain. Et Pharaon répondit à tous les Egyptiens : Allez vers Joseph, et faites ce qu'il vous dira.
\VS{56}La famine régnait dans tout le pays. Joseph ouvrit tous les lieux d’approvisionnements et vendit du blé aux Egyptiens. La famine augmentait dans le pays d’Egypte.
\VS{57}On venait de tous les pays jusqu’en Egypte, pour acheter du blé  auprès de Joseph ; car la famine était fort grande sur toute la terre.
\Chap{42}
\TextTitle{Les frères de Joseph viennent acheter des vivres en Egypte}
\VerseOne{}Et Jacob, voyant qu'il y avait du blé à vendre en Egypte, dit à ses fils : Pourquoi vous regardez-vous les uns les autres ?
\VS{2}Il leur dit aussi : Voici, j'ai appris qu'il y a du blé à vendre en Egypte, descendez-y pour nous en acheter là, afin que nous vivions, et que nous ne mourions point.
\VS{3}Alors les frères de Joseph descendirent pour acheter du blé en Egypte.
\VS{4}Mais Jacob n'envoya point Benjamin, frère de Joseph, avec ses frères ; car il disait : Il faut prendre garde qu’un malheur ne lui arrive.
\VS{5}Ainsi les fils d'Israël allèrent en Egypte pour acheter du blé avec ceux qui y allaient, car la famine était dans le pays de Canaan.
\TextTitle{Joseph met ses frères à l'épreuve}
\VS{6}Joseph commandait dans le pays, et c’était lui qui vendait le blé à tous les peuples de la terre. Les frères de Joseph vinrent et se prosternèrent devant lui la face contre terre.
\VS{7}Joseph vit ses frères et les reconnut ; mais il feignit d’être un étranger pour eux, et il leur parla rudement, en leur disant : D'où venez-vous ? Et ils répondirent : Du pays de Canaan, pour acheter des vivres.
\VS{8}Joseph reconnut ses frères, mais eux ne le connurent point.
\VS{9}Alors Joseph se souvint des songes qu'il avait eus à leur sujet et leur dit : Vous êtes des espions, vous êtes venus pour observer les lieux faibles du pays.
\VS{10}Et ils lui répondirent : Non, mon seigneur, mais tes serviteurs sont venus pour acheter des vivres.
\VS{11}Nous sommes tous enfants d'un même homme, nous sommes des gens de bien ; tes serviteurs ne sont pas des espions.
\VS{12}Et il leur dit : Nullement ; vous êtes venus pour observer les lieux faibles du pays.
\VS{13}Et ils répondirent : Nous, tes serviteurs, étions douze frères, fils d'un même homme, dans le pays de Canaan. Et voici, le plus jeune est aujourd'hui avec notre père, et l'un n'est plus.
\VS{14}Joseph leur dit : C'est ce que je vous disais, vous êtes des espions.
\VS{15}Voici comment vous serez éprouvés : Par la vie de Pharaon ! Vous ne sortirez pas d'ici que votre jeune frère ne soit venu ici.
\VS{16}Envoyez l’un de vous et qu’il amène votre frère ; et  vous, restez prisonniers. Vos paroles seront éprouvées et je saurai si vous avez dit la vérité. Autrement, par la vie de Pharaon ! Vous êtes des espions.
\VS{17}Et il les mit tous ensemble en prison pendant trois jours.
\VS{18}Le troisième jour, Joseph leur dit : Faites ceci, et vous vivrez. Je crains Dieu !
\VS{19}Si vous êtes sincères, que l'un de vos frères reste enfermé dans votre prison ; et vous, partez et emportez du blé pour nourrir vos familles.
\VS{20}Puis amenez-moi votre jeune frère afin que vos paroles soient éprouvées, et vous ne mourrez point ; et ils firent ainsi.
\TextTitle{Siméon gardé en Egypte en attendant que Benjamin soit présenté à Joseph}
\VS{21}Et ils se dirent alors l'un à l'autre : Nous sommes certainement coupables à l'égard de notre frère ; car nous avons vu l'angoisse de son âme quand il nous demandait grâce, et nous ne l'avons point écouté ; c'est pour cela que cette détresse nous est arrivée.
\VS{22}Ruben leur répondit en disant : Ne vous disais-je pas : Ne commettez point ce péché contre l'enfant ? Et vous ne m’avez point écouté ; et voici que son sang vous est redemandé.
\VS{23}Ils ne savaient pas que Joseph les comprenait, parce qu'il se servait d’un interprète pour leur parler.
\VS{24}Il s’éloigna d’eux pour pleurer. Et il revint, leur parla ; puis il prit parmi eux Siméon, et le fit enchaîner sous leurs yeux.
\VS{25}Et Joseph ordonna qu'on remplisse leurs sacs de blé, et qu'on remette l'argent de chacun d’eux dans son sac, et qu'on leur donne de la provision pour la route ; et cela fut fait ainsi.
\VS{26}Ils chargèrent donc leur blé sur leurs ânes, et s'en allèrent.
\VS{27}L’un d'eux ouvrit son sac pour donner du fourrage à son âne dans l'hôtellerie ; et il vit son argent qui était à l’entrée de son sac.
\VS{28}Il dit à ses frères : Mon argent m'a été rendu ; et le voici dans mon sac. Alors leur cœur fut en défaillance ; et ils furent saisis de peur, et se dirent l'un à l'autre : Qu'est-ce que Dieu nous a fait ?
\VS{29}Et étant arrivés dans le pays de Canaan, vers Jacob leur père, ils lui racontèrent toutes les choses qui leur étaient arrivées, en disant :
\VS{30}L'homme qui est le seigneur du pays, nous a parlé rudement et nous a pris pour des espions du pays.
\VS{31}Mais nous lui avons répondu : Nous sommes sincères, nous ne sommes point des espions.
\VS{32}Nous étions douze frères, fils de notre père ; l'un n'est plus, et le plus jeune est aujourd'hui avec notre père dans le pays de Canaan.
\VS{33}Et cet homme, qui est le seigneur  du pays, nous a dit : A ceci je connaîtrai que vous êtes sincères : Laissez-moi l'un de vos frères, et prenez de quoi nourrir vos familles et partez.
\VS{34}Puis amenez-moi votre jeune frère, et je saurai que vous n'êtes point des espions, que vous êtes sincères ; je vous rendrai votre frère, et vous pourrez librement trafiquer dans le  pays.
\VS{35}Lorsqu’ils vidèrent leurs sacs, voici, le paquet d’argent de chacun était dans son sac. Ils virent, eux et leur père, leurs paquets d’argent, et ils eurent peur.
\VS{36}Jacob leur père leur dit : Vous me privez de mes enfants ! Joseph n'est plus, et Siméon n'est plus, et vous prendriez Benjamin ! C’est sur moi que tout cela retombe.
\VS{37}Ruben parla à son père et lui dit : Fais mourir deux de mes fils si je ne te ramène pas Benjamin ! Remets-le entre mes mains et je te le ramènerai.
\VS{38}Jacob répondit : Mon fils ne descendra point avec vous, car son frère est mort, et il reste seul ; s’il lui arrivait un malheur dans le voyage que vous allez faire, vous feriez descendre mes cheveux blancs avec douleur dans le scheol.
\Chap{43}
\TextTitle{Jacob renvoie ses fils en Egypte\FTNTT{Ge. 37:26-28}}
\VerseOne{}Or la famine devint fort grande dans le pays.
\VS{2}Et quand  ils eurent achevé de manger le blé qu'ils avaient apporté d'Egypte, leur père leur dit : Retournez, achetez-nous un peu de vivres.
\VS{3}Juda lui répondit et lui dit : Cet homme nous a expressément déclaré, disant : Vous ne verrez point ma face, à moins que votre frère ne soit avec vous.
\VS{4}Si donc tu envoies notre frère avec nous, nous descendrons en Egypte et nous t'achèterons des vivres.
\VS{5}Mais si tu ne l'envoies pas, nous n'y descendrons point ; car cet homme nous a dit : Vous ne verrez point ma face, à moins que votre frère ne soit avec vous.
\VS{6}Et Israël dit : Pourquoi avez-vous mal agi à mon égard, en disant à cet homme que vous aviez encore un frère ?
\VS{7}Ils répondirent : Cet homme nous a interrogés sur nous et sur notre famille, en disant : Votre père vit-il encore ? N'avez-vous point de frère ? Et nous lui avons déclaré selon ce qu'il nous avait demandé ; pouvions-nous savoir qu'il dirait : Faites descendre votre frère ?
\VS{8}Juda dit à Israël, son père : Laisse venir l'enfant avec moi, afin que nous nous levions et que nous partions ; et nous vivrons et nous ne mourons point, nous, toi et nos enfants.
\VS{9}Je réponds de lui, tu le redemanderas de ma main. Si je ne te le ramène pas auprès de toi et si je ne le remets pas devant ta face, je serai coupable toute ma vie envers toi.
\VS{10}Car si nous n’avions pas tardé, certainement nous serions déjà de retour deux fois.
\VS{11}Alors Israël leur père leur dit : Si cela est ainsi, faites ceci, prenez dans vos bagages les meilleures productions du pays, pour en  porter un présent à cet homme, un peu de baume, et un peu de miel, des épices, de la myrrhe, des dattes, et des amandes.
\VS{12}Prenez avec vous de l'argent au double dans vos mains, et rapportez l’argent qu’on avait mis à l’entrée de vos sacs ; peut-être était-ce une erreur.
\VS{13}Prenez votre frère, et levez-vous, retournez vers cet homme.
\VS{14}Que le Dieu Tout-Puissant vous fasse trouver grâce devant cet homme, afin qu'il relâche votre autre frère et Benjamin ; et s'il faut que je sois privé de ces deux fils, que j'en sois privé.
\VS{15}Alors ils prirent le présent, et ayant pris de l'argent au double dans leurs mains, et Benjamin, ils se levèrent et descendirent en Egypte ; puis ils se présentèrent devant Joseph.
\VS{16}Dès que Joseph vit Benjamin avec eux, il dit à l’intendant de sa maison : Fais entrer ces gens dans la maison, tue et apprête quelques bêtes, car ils mangeront à midi avec moi.
\VS{17}Cet homme fit ce que Joseph lui avait dit ; et il conduit ces gens dans la maison de Joseph.
\VS{18}Ils eurent peur lorsqu’ils furent conduits dans la maison de Joseph, et ils dirent : Nous sommes emmenés à cause de l'argent remis l’autre fois dans nos sacs ; c’est pour se jeter sur nous, se précipiter sur nous ; c’est pour nous prendre comme esclaves et s’emparer de nos ânes.
\VS{19}Ils s’approchèrent de l’intendant de la maison de Joseph, et lui adressèrent la parole, à l’entrée de la maison.
\VS{20}Ils dirent : Pardon ! Mon seigneur, nous sommes déjà descendus une fois pour acheter des vivres.
\VS{21}Puis, quand nous arrivâmes, au lieu où nous devions passer la nuit, nous avons ouvert nos sacs ; et voici, l’argent de chacun était à l’entrée de son sac, notre argent selon son poids ;  nous le rapportons avec nous.
\VS{22}Nous avons aussi apporté d'autre argent dans nos mains pour acheter des vivres ; et nous ne savons point qui a remis notre argent dans nos sacs.
\VS{23}L’intendant leur dit : Tout va bien pour vous, ne craignez point. C’est votre Dieu, le Dieu de votre père vous a donné un trésor dans vos sacs ; votre argent est parvenu jusqu'à moi ; et il leur amena Siméon.
\VS{24}Cet homme les fit entrer dans la maison de Joseph, et leur donna de l'eau, et ils lavèrent leurs pieds ; il donna aussi à manger à leurs ânes.
\VS{25}Ils préparèrent leur présent en attendant que Joseph revienne à midi ; car ils avaient appris qu'ils mangeraient du pain chez lui.
\VS{26}Quand  Joseph fut arrivé à la maison, ils lui offrirent le présent qu'ils avaient dans leurs mains, et se prosternèrent à terre devant lui dans la maison.
\VS{27}Il leur demanda comment ils se portaient et leur dit : Votre vieux père, dont vous m'avez parlé, se porte-t-il bien ? Vit-il encore ?
\VS{28}Ils répondirent : Ton serviteur, notre père, se porte bien, il vit encore. Et ils s’inclinèrent et se prosternèrent.
\VS{29}Joseph leva les yeux, il vit Benjamin, son frère, fils de sa mère, et il dit : Est-ce là votre jeune frère dont vous m'avez parlé ? Et il ajouta : Mon fils, Dieu te fasse grâce !
\VS{30}Et Joseph se retira promptement, car ses entrailles étaient émues à la vue de son frère, et il cherchait un lieu pour pleurer ; il entra dans sa chambre et il y pleura.
\VS{31}Après s’être lavé le visage, il sortit de là, et faisant des efforts pour se contenir, il dit : Servez le pain.
\VS{32}On servit Joseph à part, et ses frères à part, et les Egyptiens qui mangeaient avec lui furent aussi servis à part, car les Egyptiens ne pouvaient manger du pain avec les Hébreux,  parce que c’est à leurs yeux une abomination.
\VS{33}Les frères de Joseph s’assirent en sa présence, le premier-né selon son droit d’aînesse, et le plus jeune selon son âge ; et ils se regardaient les uns les autres avec étonnement.
\VS{34}Joseph leur fit porter des mets qui étaient devant lui, et Benjamin en eut cinq fois plus que les autres. Ils burent et s’enivrèrent  avec lui.
\Chap{44}
\TextTitle{Juda se rend esclave de Joseph à la place de Benjamin\FTNTT{Ge. 43:9}}
\VerseOne{}Et Joseph donna un ordre à son intendant, en disant : Remplis de vivres les sacs de ces gens, autant qu'ils en pourront porter, et remets l'argent de chacun à l’entrée de son sac.
\VS{2}Tu mettras aussi ma coupe, la coupe d'argent, à l’entrée du sac du plus petit avec l'argent de son blé ; et il fit comme Joseph lui avait dit.
\VS{3}Le matin, dès qu'il fit jour, on renvoya ces hommes avec leurs ânes.
\VS{4}Ils étaient sortis de la ville, ils n’en étaient guère éloignés, lorsque Joseph dit à son intendant : Va, poursuis ces hommes, et quand tu les auras atteints, tu leur diras : Pourquoi avez-vous rendu le mal pour le bien ?
\VS{5}N'est-ce pas la coupe dont se sert mon seigneur pour boire et pour deviner ? Vous avez mal fait d’agir ainsi.
\VS{6}L’intendant les atteignit, et leur dit ces paroles.
\VS{7}Ils lui répondirent : Pourquoi mon seigneur parle-t-il ainsi ? Loin de tes serviteurs la pensée de faire pareille chose !
\VS{8}Voici, nous t'avons rapporté du pays de Canaan l'argent que nous avions trouvé à l’entrée de nos sacs, et comment aurions-nous dérobé de l'argent ou de l'or de la maison de ton maître ?
\VS{9}Que celui de tes serviteurs sur qui se trouvera la coupe meure ; et nous serons aussi esclaves de mon seigneur !
\VS{10}Il leur dit : Qu'il soit fait maintenant selon vos paroles ! Qu’il en soit ainsi ! Que celui sur qui se trouvera la coupe soit mon esclave, et vous, vous serez innocents.
\VS{11}Et ils se hâtèrent de déposer chacun son sac à terre ; et chacun ouvrit son sac.
\VS{12}L’intendant les fouilla, en commençant par le plus âgé, et finissant par le plus jeune ; et la coupe fut trouvée dans le sac de Benjamin.
\VS{13}Alors ils déchirèrent leurs vêtements, et chacun rechargea son âne, et ils retournèrent à la ville.
\VS{14}Juda et ses frères arrivèrent à la maison de Joseph, qui était encore là, et ils se jetèrent à terre devant lui.
\VS{15}Joseph leur dit : Quelle action avez-vous faite ? Ne savez-vous pas qu'un homme tel que moi ne manque pas de deviner ?
\VS{16}Juda lui répondit : Que dirons-nous à mon seigneur ? Comment parlerons-nous ? Et comment nous justifierons-nous ? Dieu a trouvé l'iniquité de tes serviteurs ; voici, nous sommes esclaves de mon seigneur, nous, et celui entre les mains de qui la coupe a été trouvée.
\VS{17}Mais il dit : Loin de moi la pensée d’agir ainsi ! L’homme dans la main duquel la coupe a été trouvée sera mon esclave ; mais vous, remontez en paix vers votre père.
\VS{18}Alors Juda s'approcha de lui en disant : Pardon mon seigneur ! Je te prie, que ton serviteur dise un mot, je te prie aux oreilles de mon seigneur, et que ta colère ne s'enflamme point contre ton serviteur, car tu es comme Pharaon.
\VS{19}Mon seigneur interrogea ses serviteurs en disant : Avez-vous un père ou un frère ?
\VS{20}Nous avons répondu à mon seigneur : Nous avons notre père qui est âgé, et un enfant de sa vieillesse, et qui est le plus jeune d'entre nous ; son frère est mort, et celui-ci est resté le seul enfant  de sa mère ; et son père l'aime.
\VS{21}Tu as dis à tes serviteurs : Faites-le descendre vers moi, et que je le voie de mes yeux.
\VS{22}Nous avons répondu  à mon seigneur : Cet enfant ne peut quitter son père, car s'il le quitte, son père mourra.
\VS{23}Alors tu dis à tes serviteurs : Si votre petit frère ne descend avec vous, vous ne verrez plus ma face.
\VS{24}Lorsque nous sommes remontés auprès de ton serviteur, mon père, nous lui avons rapporté les paroles de mon seigneur.
\VS{25}Notre père nous a dit : Retournez, et achetez-nous un peu de vivres.
\VS{26}Nous lui avons répondu : Nous ne pouvons pas descendre ; mais si notre petit frère est avec nous, nous descendrons, car nous ne pouvons pas voir la face de cet homme, à moins que notre jeune frère ne soit avec nous.
\VS{27}Ton serviteur, mon père, nous répondit : Vous savez que ma femme m'a enfanté deux fils.
\VS{28}L’un étant sorti de chez moi, je pense qu’il a été sans doute déchiré, car je ne l’ai pas revu jusqu’à présent.
\VS{29}Si vous me prenez encore celui-ci, et qu’il lui arrive un malheur, vous ferez descendre mes cheveux blancs avec douleur dans le scheol.
\VS{30}Maintenant, si je retourne auprès de ton serviteur, mon père, sans avoir avec nous l’enfant à l’âme duquel son âme est attachée,
\VS{31}il mourra, en voyant que l’enfant n’y est pas ; et tes serviteurs feront descendre avec douleur dans le scheol les cheveux blancs de ton serviteur, notre père.
\VS{32}De plus, ton serviteur a répondu pour l'enfant, en le prenant à mon père, en disant : Si je ne te le ramène pas, je serai pour toujours coupable envers mon père.
\VS{33}Permets donc, je te prie, à ton serviteur de rester à la place de l’enfant, comme esclave de mon seigneur ; et que l’enfant remonte avec ses frères.
\VS{34}Car comment pourrai-je remonter vers mon père, si l'enfant n'est pas avec moi ? Que je ne voie point l'affliction qu'en aurait mon père !
\Chap{45}
\TextTitle{Joseph révèle son identité à ses frères}
\VerseOne{}Alors Joseph, ne pouvant plus se contenir devant tous ceux qui étaient là présents, cria : Faites sortir tout le monde ! Et il ne resta personne quand il se fit connaître à ses frères.
\VS{2}Et en pleurant, il éleva sa voix, et les Egyptiens l'entendirent, et la maison de Pharaon l'entendit aussi.
\VS{3}Et Joseph dit à ses frères : Je suis Joseph ! Mon père vit-il encore ? Mais ses frères ne pouvaient lui répondre, car ils étaient tout troublés en sa présence.
\VS{4}Joseph dit encore à ses frères : Je vous prie, approchez-vous de moi ; et ils s'approchèrent, et il leur dit : Je suis Joseph, votre frère, que vous avez vendu pour être mené en Egypte\FTNT{Ac. 7:13.}.
\VS{5}Mais maintenant ne soyez pas en peine, et n'ayez point de regret de ce que vous m'avez vendu pour être mené ici, car Dieu m'a envoyé devant vous pour la conservation de votre vie.
\VS{6}Car voici, il y a déjà deux ans que la famine est sur la terre, et il y aura encore cinq ans pendant lesquels il n'y aura ni labour ni moisson.
\VS{7}Mais Dieu m'a envoyé devant vous, pour vous faire subsister sur la terre, et vous faire vivre par une grande délivrance.
\VS{8}Maintenant donc ce n'est pas vous qui m'avez envoyé ici, mais c'est Dieu ; il m'a établi père de Pharaon, et seigneur sur toute sa maison, et gouverneur de tout le pays d'Egypte.
\VS{9}Hâtez-vous d'aller vers mon père, et dites-lui : Ainsi a dit ton fils, Joseph : Dieu m'a établi seigneur sur toute l'Egypte, descends vers moi, ne t'arrête point.
\VS{10}Et tu habiteras dans la contrée de Gosen, et tu seras près de moi, toi, tes fils, et les fils de tes fils, tes brebis, et tes bœufs, et tout ce qui est à toi.
\VS{11}Là, je te nourrirai, car il y aura encore cinq années de famine ; et ainsi tu ne périras point, toi et ta maison, et tout ce qui est à toi.
\VS{12}Et voici, vous voyez de vos yeux, et Benjamin mon frère voit aussi de ses yeux, que c'est moi qui vous parle de ma propre bouche.
\VS{13}Rapportez donc à mon père quelle est ma gloire en Egypte, et tout ce que vous avez vu ; hâtez-vous, et faites descendre ici mon père.
\VS{14}Alors il se jeta sur le cou de Benjamin, son frère, et pleura. Benjamin pleura aussi sur son cou.
\VS{15}Puis il embrassa tous ses frères et pleura sur eux ; après cela ses frères parlèrent avec lui.
\TextTitle{Jacob pardonne ses frères et fait venir son père Jacob\FTNTT{Ge. 43:9}}
\VS{16}Et le bruit se répandit dans la maison de Pharaon que les frères de Joseph étaient venus, ce qui plut fort à Pharaon et à ses serviteurs.
\VS{17}Alors Pharaon dit à Joseph : Dis à tes frères : Faites ceci : Chargez vos bêtes, et allez, retournez dans le pays de Canaan ;
\VS{18}et prenez votre père et vos familles, et revenez vers moi, et je vous donnerai le meilleur du pays d'Egypte ; et vous mangerez la graisse de la terre.
\VS{19}Tu as ordre de leur dire: Faites ceci : Prenez dans le pays d’Egypte des chars pour vos enfants et pour vos femmes; amenez votre père, et venez.
\VS{20}Ne regrettez point ce que vous laisserez, car ce qu’il y a de meilleur dans tout le pays d’Egypte sera pour vous.
\VS{21}Et les fils d'Israël firent ainsi. Et Joseph leur donna des chars selon l'ordre de Pharaon ; il leur donna aussi de la provision pour la route.
\VS{22}Il leur donna à chacun des vêtements de rechange ; et il donna à Benjamin trois cents pièces d'argent et cinq vêtements de rechange.
\VS{23}Il envoya aussi à son père dix ânes chargés des plus excellentes choses qu'il y avait en Egypte, et dix ânesses portant du blé, du pain, et des vivres à son père pour la route.
\VS{24}Il renvoya donc ses frères, et ils partirent ; et il leur dit : Ne vous querellez point en chemin.
\VS{25}Ainsi ils remontèrent d'Egypte, et vinrent dans le  pays de Canaan auprès de Jacob, leur père.
\VS{26}Et ils lui rapportèrent et lui dirent : Joseph vit encore, et même c’est lui qui gouverne tout le pays d'Egypte ; mais le cœur de Jacob resta froid, parce qu’il ne les croyait pas.
\VS{27}Et ils lui dirent toutes les paroles que Joseph leur avait dites ; puis il vit les chars que Joseph avait envoyés pour le porter ; et l'esprit de Jacob, leur père, se ranima.
\VS{28}Alors Israël dit : C'est assez ! Joseph, mon fils, vit encore ! J'irai, et je le verrai avant que je meure.
\Chap{46}
\TextTitle{Jacob en Egypte}
\VerseOne{}Israël donc partit avec tout ce qui lui appartenait, et vint à Beer-Schéba, et il offrit des sacrifices au Dieu de son père Isaac.
\VS{2}Et Dieu parla à Israël dans une vision pendant la nuit et lui dit : Jacob, Jacob ! Et il répondit : Me voici.
\VS{3}Et Dieu lui dit : Je suis le Dieu, le Dieu de ton père. Ne crains point de descendre en Egypte, car là je te ferai devenir une grande nation.
\VS{4}Je descendrai avec toi en Egypte, et je t'en ferai aussi très certainement remonter ; et Joseph te fermera les yeux avec sa main.
\VS{5}Ainsi Jacob partit de Beer-Schéba, et les fils d'Israël mirent Jacob, leur père, et leurs petits enfants, et leurs femmes, sur les chars que Pharaon avait envoyés pour le porter.
\VS{6}Ils emmenèrent aussi leur bétail et leur bien qu'ils avaient acquis dans le pays de Canaan ; et Jacob et toute sa famille avec lui vinrent en Egypte.
\VS{7}Il amena avec lui en Egypte ses fils, et les fils de ses fils, ses filles, et les filles de ses fils, et toute sa famille.
\TextTitle{Les fils de Jacob en Egypte}
\VS{8}Voici les noms des fils d'Israël qui vinrent en Egypte : Jacob et ses fils. Le premier-né de Jacob fut Ruben.
\VS{9}Et les fils de Ruben : Hénoc, Pallu, Hetsron, et Carmi.
\VS{10}Et les fils de Siméon : Jemuel, Jamin, Ohad, Jakin, Tsochar, et Saül, fils d'une Cananéenne.
\VS{11}Et les fils de Lévi : Guerschon, Kehath, et Merari.
\VS{12}Et les fils de Juda : Er, Onan, Schéla, Pérets et Zérach ; mais Er et Onan moururent au pays de Canaan. Les fils de Pérets furent Hetsron et Hamul.
\VS{13}Et les fils d'Issacar : Thola, Puva, Job et Schimron.
\VS{14}Et les fils de Zabulon : Séred, Elon et Jahleel.
\VS{15}Ce sont là les fils de Léa, qu'elle enfanta à Jacob à Paddan-Aram, avec Dina, sa fille. Ses fils et ses filles formaient en tout trente-trois personnes.
\VS{16}Et les fils de Gad : Tsiphjon, Haggi, Schuni, Etsbon, Eri, Arodi et Areéli.
\VS{17}Et les fils d'Aser : Jimna, Jischva, Jischvi, Beria et Sérach, leur sœur. Les fils de Beria : Héber et Malkiel.
\VS{18}Ce sont là les fils de Zilpa que Laban donna à Léa, sa fille ; et elle les enfanta à Jacob. En tout seize personnes.
\VS{19}Les fils de Rachel, femme de Jacob, furent Joseph et Benjamin.
\VS{20}Et il naquit à Joseph dans le  pays d'Egypte, Manassé et Ephraïm, qu'Asnath, fille de Poti-Phéra, prêtre d'On, lui enfanta.
\VS{21}Et les fils de Benjamin étaient Béla, Béker, Aschbel, Guéra, Naaman, Ehi, Rosch, Muppim, Huppim et Ard.
\VS{22}Ce sont là les fils de Rachel, qu'elle enfanta à Jacob. En tout quatorze personnes.
\VS{23}Et les fils de Dan : Huschim.
\VS{24}Et les enfants de Nephthali : Jahtseel, Guni, Jetser, et Schillem.
\VS{25}Ce sont là les fils de Bilha, que Laban donna à Rachel, sa fille, et elle les enfanta à Jacob. En tout sept personnes.
\VS{26}Toutes les personnes appartenant à Jacob qui vinrent en Egypte, et qui étaient issues de lui, sans les femmes des fils de Jacob, furent en tout soixante-dix.
\VS{27}Et les fils de Joseph qui lui étaient nés en Egypte furent deux personnes. Toutes les personnes de la maison de Jacob qui vinrent en Egypte furent soixante-dix.
\VS{28}Jacob envoya Juda devant lui vers Joseph, pour l’informer qu’il se rendait en Gosen. Ils vinrent donc dans la contrée de Gosen.
\VS{29}Et Joseph fit atteler son char, et y monta pour aller à la rencontre d'Israël, son père, en Gosen. Dès qu’il le vit, il se jeta à son cou, et pleura longtemps sur son cou.
\VS{30}Et Israël dit à Joseph : Que je meure à présent, puisque j'ai vu ton visage, et que tu vis encore.
\VS{31}Puis Joseph dit à ses frères et à la famille de son père : Je monterai pour informer Pharaon, et je lui dirai : Mes frères et la famille de mon père, qui étaient au pays de Canaan, sont arrivés auprès de moi.
\VS{32}Et ces hommes sont bergers, ils se sont toujours occupés du bétail, et ils ont amené leurs brebis et leurs bœufs, et tout ce qui était à eux.
\VS{33}Et quand Pharaon vous fera appeler et vous dira : Quel est votre métier ?
\VS{34}Vous direz : Tes serviteurs se sont toujours occupés de bétail dès leur jeunesse jusqu'à maintenant, nous, et nos pères. De cette manière, vous habiterez dans le pays de Gosen, car les Egyptiens ont en abomination les bergers.
\Chap{47}
\TextTitle{La famille de Jacob honorée en Egypte}
\VerseOne{}Joseph alla avertir Pharaon, et lui dit : Mon père  et mes frères sont arrivés du pays de Canaan avec leurs troupeaux et leurs bœufs, et tout ce qui est à eux ; et voici, ils sont dans le pays de Gosen.
\VS{2}Et il prit une partie de ses frères, à savoir cinq, et il les présenta à Pharaon.
\VS{3}Et Pharaon dit aux frères de Joseph : Quel est votre métier ? Ils répondirent à Pharaon : Tes serviteurs sont bergers, comme l'ont été nos pères.
\VS{4}Ils dirent aussi à Pharaon : Nous sommes venus séjourner comme étrangers dans ce pays, parce qu'il n'y a plus de pâturages pour les troupeaux de tes serviteurs, et il y a une grande famine au pays de Canaan ; maintenant nous te prions que tes serviteurs demeurent dans le pays de Gosen.
\VS{5}Et Pharaon parla à Joseph et lui dit : Ton père et tes frères sont arrivés auprès de toi.
\VS{6}Le pays d'Egypte est à ta disposition ; fais habiter ton père et tes frères dans le meilleur endroit du pays ; qu'ils demeurent dans la terre de Gosen ; et si tu connais parmi eux des hommes habiles tu les établiras chefs de tous mes troupeaux.
\VS{7}Alors Joseph amena Jacob, son père, et le présenta à Pharaon ; et Jacob bénit Pharaon.
\VS{8}Et Pharaon dit à Jacob : Quel est le nombre de jours de tes années ?
\VS{9}Jacob répondit à Pharaon : Les jours des années de mes pèlerinages sont de cent trente ans ; les jours des années de ma vie ont été courts et mauvais et n'ont point atteint les jours des années de la vie de mes pères, du temps de leurs pèlerinages.
\VS{10}Jacob donc bénit Pharaon, et sortit de devant lui.
\VS{11}Et Joseph assigna une demeure à son père et à ses frères, et leur donna une possession au pays d'Egypte, au meilleur endroit du pays, dans le pays d'Egypte, comme Pharaon l'avait ordonné.
\VS{12}Et Joseph fournit du pain à son père et à ses frères, et à toute la maison de son père, selon le nombre de leurs familles.
\VS{13}Or il n'y avait point de pain sur toute la terre, car la famine était très grande ; et le pays d'Egypte et le pays de Canaan étaient épuisés par la famine.
\VS{14}Et Joseph amassa tout l'argent qui se trouva dans le pays d'Egypte, et dans le pays de Canaan, contre le blé qu'on achetait ; et il apporta l'argent à la maison de Pharaon.
\VS{15}Quand l'argent du pays d'Egypte et du pays de Canaan fut épuisé, tous les Egyptiens vinrent à Joseph en disant : Donne-nous du pain ; et pourquoi mourrions-nous en ta présence, parce que l'argent manque ?
\VS{16}Joseph répondit : Donnez votre bétail, et je vous en donnerai pour votre bétail, puisque l'argent manque.
\VS{17}Alors ils amenèrent à Joseph leur bétail, et Joseph leur donna du pain pour des chevaux, pour des troupeaux de brebis, pour des troupeaux de boeufs, et pour des ânes ; ainsi il leur fournit du pain en échange de leurs troupeaux cette année-là.
\VS{18}Lorsque cette année fut écoulée, ils revinrent à Joseph l'année suivante et lui dirent : Nous ne cacherons point à mon seigneur que l'argent est épuisé et les troupeaux de bétail ont été amenés à mon seigneur, il ne nous reste plus rien devant mon seigneur que nos corps et nos terres.
\VS{19}Pourquoi mourrions-nous sous tes yeux ? Achète-nous avec nos terres, pour du pain ; et nous serons esclaves de Pharaon, et nos terres seront à lui ; donne-nous aussi de quoi semer, afin que nous vivions et ne mourions point, et que nos terres ne soient point désolées.
\VS{20}Ainsi, Joseph acheta toutes les terres de l’Egypte pour Pharaon ; car les Egyptiens vendirent chacun son champ, parce que la famine les pressait. Et le pays devint la propriété de Pharaon.
\VS{21}Et il fit passer le peuple dans les villes, d’un bout à l’autre des frontières de l’Egypte.
\VS{22}Seulement, il n’acheta point les terres des prêtres, parce qu’il y avait une loi de Pharaon en faveur des prêtres, qui vivaient du revenu que leur assurait Pharaon, c’est pourquoi ils ne vendirent point leurs terres.
\VS{23}Et Joseph dit au peuple : Voici, je vous ai achetés aujourd'hui, vous et vos terres pour Pharaon, voilà de la semence pour ensemencer la terre.
\VS{24}Et quand le temps de la récolte viendra, vous donnerez la cinquième partie à Pharaon, et les quatre autres seront à vous, pour ensemencer les champs, et pour votre nourriture, et pour celle de ceux qui sont dans vos maisons, et pour la nourriture de vos petits enfants.
\VS{25}Et ils dirent : Tu nous sauves la vie ! Que nous trouvions grâce aux yeux de mon seigneur, et nous serons esclaves de Pharaon.
\VS{26}Et Joseph fit de cela une loi qui a subsisté jusqu’à ce jour, et d’après laquelle un cinquième du revenu des terres de l’Egypte appartient à Pharaon ; il n’y a que les terres des prêtres qui ne soient point à Pharaon.
\TextTitle{Jacob demande à être enterré à Canaan}
\VS{27}Israël habita dans le pays d’Egypte, dans le pays de Gosen. Ils eurent des possessions, ils furent féconds et multiplièrent beaucoup.
\VS{28}Jacob vécut dix-sept ans dans le pays d’Egypte ; et les jours des années de la vie de Jacob furent de cent quarante-sept ans.
\VS{29}Et quand le jour de la mort d'Israël approcha, il appela Joseph, son fils, et lui dit : Je te prie, si j'ai trouvé grâce à tes yeux, mets présentement ta main sous ma cuisse, et jure-moi que tu useras envers moi de bonté et de fidélité : Je te prie, ne m'enterre point en Egypte !
\VS{30}Quand  je serai couché avec mes pères, tu me transporteras hors de l'Egypte, et m'enterreras dans leur sépulcre. Et il répondit : Je le ferai selon ta parole.
\VS{31}Et Jacob lui dit : Jure-le-moi ; et il le lui jura. Et Israël se prosterna sur le chevet du lit.
\Chap{48}
\TextTitle{Bénédiction de Jacob sur les fils de Joseph}
\VerseOne{}Or il arriva après ces choses que l'on vint dire à Joseph : Voici, ton père est malade. Et il prit avec lui ses deux fils, Manassé et Ephraïm.
\VS{2}On avertit Jacob et on lui dit : Voici Joseph, ton fils, qui vient vers toi. Alors Israël rassembla ses forces et s’assit sur son lit.
\VS{3}Puis Jacob dit à Joseph : Le Dieu Tout-Puissant m’est apparu  à Luz, au pays de Canaan, et m’a béni.
\VS{4}Et il m’a dit : Voici, je te ferai croître et multiplier, et je te ferai devenir une assemblée de peuples, et je donnerai ce pays en possession perpétuelle à ta postérité après toi.
\VS{5}Et maintenant tes deux fils, qui te sont nés au pays d'Egypte, avant mon arrivée vers toi, seront à moi : Ephraïm et Manassé seront à moi comme Ruben et Siméon.
\VS{6}Mais les enfants que tu auras engendrés après eux, seront à toi, et ils seront appelés selon le nom de leurs frères dans leur héritage.
\VS{7}A mon retour de Paddan, Rachel mourut en route auprès de moi, dans le pays de Canaan, à quelque distance d’Ephrata ; et c’est là que je l’ai enterrée, sur le chemin d’Ephrata, qui est Bethléhem.
\VS{8}Puis Israël vit les fils de Joseph, et il dit : Qui sont ceux-ci ?
\VS{9}Et Joseph répondit à son père : Ce sont mes fils que Dieu m'a donnés ici ; et il dit : Amène-les-moi, je te prie, afin que je les bénisse.
\VS{10}Or les yeux d'Israël étaient appesantis par la vieillesse, et il ne pouvait plus voir ; et il les fit approcher de lui, les embrassa et les prit dans ses bras.
\VS{11}Et Israël dit à Joseph : Je ne pensais pas revoir ton visage ; et voici, Dieu m'a fait voir et toi et ta postérité.
\VS{12}Et Joseph les retira des genoux de son père, et se prosterna le visage contre terre.
\VS{13}Puis Joseph les prit tous deux, Ephraïm de sa main droite à la gauche d’Israël, et Manassé de sa main gauche à la droite d’Israël, et il les fit approcher de lui.
\VS{14}Israël étendit sa main droite et la posa sur la tête d’Ephraïm qui était le plus jeune, et il posa sa main gauche sur la tête de Manassé ; ce fut avec intention qu’il posa ses mains ainsi, car Manassé était le premier-né.
\VS{15}Il bénit Joseph et dit : Que le Dieu en présence duquel ont marché mes pères, Abraham et Isaac, que le Dieu qui m’a conduit depuis que j’existe jusqu’à ce jour\FTNT{Hé. 11:21.},
\VS{16}que l’Ange qui m’a délivré de tout mal, bénisse ces enfants ! Qu’ils soient appelés de mon nom et du nom de mes pères, Abraham et Isaac, et qu’ils multiplient en abondance comme les poissons au milieu du pays.
\VS{17}Joseph vit avec déplaisir que son père posait sa main droite sur la tête d’Ephraïm ; il saisit la main de son père, pour la détourner de dessus la tête d’Ephraïm, et la diriger sur celle de Manassé.
\VS{18}Et Joseph dit à son père : Ce n'est pas ainsi mon père ! Car celui-ci est l'aîné ; mets ta main droite sur sa tête.
\VS{19}Mais son père le refusa en disant : Je le sais, mon fils, je le sais. Celui-ci deviendra aussi un peuple, et même il sera grand ; mais toutefois son frère, qui est plus jeune, sera plus grand que lui, et sa postérité sera une multitude de nations.
\VS{20}Il les bénit ce jour-là et dit : C’est par toi qu’Israël bénira en disant : Que Dieu te traite comme Ephraïm et comme Manassé ! Et il mit Ephraïm avant Manassé.
\VS{21}Puis Israël dit à Joseph : Voici, je  vais mourir, mais Dieu sera avec vous, et vous fera retourner au pays de vos pères.
\VS{22}Et je te donne une portion  de plus qu'à tes frères, celle que j'ai prise avec mon épée et mon arc sur les Amoréens.
\Chap{49}
\TextTitle{Prophétie de Jacob qui bénit ses fils}
\VerseOne{}Puis Jacob appela ses fils et leur dit : Assemblez-vous, et je vous annoncerai ce qui vous arrivera dans les derniers jours\FTNT{L’expression «~dans les derniers jours~» vient de l’hébreu «~achariyth~» qui veut dire «~dernier~». Son équivalent grec est «~eschatos~» : «~dernier~», «~extrémité~» etc. Jacob est le premier homme à avoir utilisé  cette expression. Cette promesse de Jacob devait arriver à Israël dans les derniers jours, selon leurs tribus. Ainsi, les promesses du droit d'aînesse de Ge. 49 étaient pour l'âge messianique, lequel est associé aux derniers jours, et a commencé à la Fête de la Pentecôte (Ac. 2:14-21). 
Ces jours impliquent :
- L’effusion de l’Esprit, le réveil de l’Eglise de Christ (Mt. 25:1-13 ; Ac. 2)
- Le réveil des faux prophètes ou l’apostasie (2 Pi. 3:3 ; 1 Jn. 2)
- La dégradation de la moralité (2 Ti. 3)
- L’enrichissement des hommes de ce monde (Ja. 5:3 ; Ap. 3:14-22)
- Le fait que Dieu nous parle par le Fils (Hé. 1:2)
- La future résurrection des saints lors du retour du Messie (Jn. 6:39-54 ; 1 Th. 4:12-17).Le temps des nations (fin des temps) s’achèvera lors du retour visible de Jésus-Christ pour établir son règne sur toute la terre. Le temps des nations a commencé lorsque, à la suite de l’infidélité d’Israël, la gloire de Dieu a quitté le temple et la ville de Jérusalem (Ez. 11), la puissance fut confiée aux nations en la personne de Nebucadnetsar qui s’empara de Jérusalem (2 R. 24 et 25 ; 2 Ch. 36:6-21 ; Da. 1 ; Jé 39). Ces temps dureront jusqu’à la destruction finale du dernier empire des nations représenté par la Bête romaine ressuscitée (Ap. 13:3). Cette destruction n’aura lieu que lorsque Jésus-Christ, la pierre détachée sans le secours d’aucune main, deviendra une grande montagne qui remplira toute la terre (Da. 2:34 ; Mi. 4). Jérusalem ne sera délivrée du joug des nations qu’à ce moment-là. Les temps des nations ne seront accomplis que lorsque le trône de Dieu sera de nouveau établi à Jérusalem.}.
\VS{2}Rassemblez-vous, et écoutez, fils de Jacob ; écoutez Israël\FTNT{«~Ecoutez Israël~» : Le «~shema~» Israël est le texte principal de la liturgie juive. Composé de trois extraits de la Torah, on le récite matin et soir accompagné de bénédictions. Voir De. 6:4-9.}, votre père.
\VS{3}Ruben, tu es mon premier-né, ma force et le commencement de ma vigueur, qui excelle en dignité et qui excelle aussi en force ;
\VS{4}impétueux comme les eaux ; tu n'auras pas la prééminence, car tu es monté sur la couche de ton père, et tu as souillé mon lit en y montant.
\VS{5}Siméon et Lévi, sont frères, leurs glaives sont des instruments de violence dans leurs demeures.
\VS{6}Que mon âme n'entre point dans leur conseil secret, que ma gloire ne soit point jointe à leur compagnie, car ils ont tué les gens dans leur colère, et ont enlevé les bœufs pour leur plaisir.
\VS{7}Maudite soit leur colère, car elle a été violente ; et leur fureur, car elle a été cruelle ; je les diviserai dans Jacob, et les disperserai dans Israël.
\VS{8}Juda, quant à toi, tes frères te loueront ; ta main sera sur la nuque de tes ennemis ; les fils de ton père se prosterneront devant toi.
\VS{9}Juda est un jeune lion. Mon fils, tu reviens du carnage, mon fils ! Il ploie les genoux, il se couche comme un lion, comme une lionne : Qui le fera lever ?
\VS{10}Le sceptre ne s’éloignera point de Juda, ni le bâton de législateur d'entre ses pieds, jusqu'à ce que le Schilo vienne, et que  les peuples lui obéissent.
\VS{11}Il attache à la vigne son ânon, et au cep excellent le petit de son ânesse ; il lavera son vêtement dans le vin, et son vêtement dans le sang des raisins.
\VS{12}Il a les yeux rouges de vin, et les dents blanches de lait.
\VS{13}Zabulon habitera sur la côte des mers, il sera un port des navires ; et ses côtés s'étendront vers Sidon.
\VS{14}Issacar est un âne robuste, couché entre les barres des étables.
\VS{15}Il voit que le lieu où il repose est agréable et que la contrée est magnifique ; et il courbe son épaule sous le fardeau, il s’assujettit à un tribut.
\VS{16}Dan jugera son peuple, comme l’une des tribus d'Israël.
\VS{17}Dan sera un serpent sur le chemin, une vipère sur le sentier, mordant les talons du cheval, pour que le cavalier tombe à la renverse.
\VS{18}Ô Yahweh ! J’espère en ton salut\FTNT{Le mot secours se dit «~yeshuw`ah~» en hébreu et veut littéralement dire  «~salut~», «~délivrance~». Avant de mourir, Jacob a  donc  placé son espérance en Jésus-Christ qui est la résurrection et la vie (Jn. 11:25). Voir commentaire en Es. 26:1.} !
\VS{19}Quant à Gad, des troupes viendront l’attaquer, mais il ravagera leur arrière-garde.
\VS{20}Le pain excellent viendra d'Aser, et il fournira les mets délicats des rois.
\VS{21}Nephtali est une biche en liberté ; il profère des belles paroles.
\VS{22}Joseph est un fils fertile, un rameau fertile près d'une fontaine ; ses branches se sont étendues sur la muraille.
\VS{23}Des archers l’ont provoqué, ils ont lancé des traits ; les archers l’ont poursuivi de leur haine.
\VS{24}Mais son arc est demeuré ferme, et ses mains ont été fortifiées par les mains du Puissant de Jacob : Il est ainsi devenu le pasteur, le rocher d’Israël.
\VS{25}C’est l’œuvre du Dieu de ton père qui t’aidera ; c’est l’œuvre du Tout-Puissant qui te bénira des bénédictions des cieux en haut, des bénédictions des eaux en bas, des bénédictions des mamelles et du sein maternel.
\VS{26}Les bénédictions de ton père ont surpassé les bénédictions de ceux qui m'ont engendré, jusqu'à la cime des antiques collines ; elles seront sur la tête de Joseph, et sur le sommet de la tête du Nazaréen d'entre ses frères.
\VS{27}Benjamin est un loup qui déchirera ; le matin il dévorera la proie, et sur le soir il partagera le butin.
\VS{28}Ce sont là tous ceux qui forment les douze tribus d'Israël.  Et c’est là ce que leur père leur dit en les bénissant. Il bénit chacun d'eux selon la bénédiction qui lui était propre.
\VS{29}Il leur donna aussi cet ordre : Je vais être recueilli auprès de mon peuple, enterrez-moi avec mes pères dans la caverne qui est au champ d'Ephron, le Héthien,
\VS{30}dans la caverne du champ de Macpéla, vis-à-vis de Mamré, dans le pays de Canaan. C’est le champ qu’Abraham a acheté d’Ephron, le Héthien, comme propriété sépulcrale.
\VS{31}C'est là qu'on a enterré Abraham avec Sara, sa femme ; c'est là qu'on a enterré Isaac et Rebecca, sa femme ; et c'est là que j'ai enterré Léa.
\VS{32}Le champ a été acquis des fils de Heth avec la caverne qui s’y trouve.
\VS{33}Lorsque Jacob eut achevé de donner ses ordres à ses fils, il retira ses pieds dans le lit, il expira, et fut recueilli auprès de son peuple.
\Chap{50}
\TextTitle{Mort de Jacob}
\VerseOne{}Alors Joseph se jeta sur le visage de son père, pleura sur lui et l’embrassa.
\VS{2}Et Joseph ordonna à ceux de ses serviteurs qui étaient médecins d'embaumer son père ; et les médecins embaumèrent Israël.
\VS{3}Et on employa quarante jours à l'embaumer, car c'était la coutume d'embaumer les corps pendant quarante jours ; et les Egyptiens le pleurèrent soixante-dix jours.
\VS{4}Quand les jours du deuil furent passés, Joseph s’adressa aux gens de la maison de Pharaon, et leur dit : Si j’ai trouvé grâce à vos yeux, rapportez, je vous prie, à Pharaon ce que je vous dis.
\VS{5}Mon père m’a fait jurer en disant : Voici, je vais mourir ! Tu m’enterreras dans le sépulcre que je me suis acheté au pays de Canaan. Je voudrais donc y monter, pour enterrer mon père ; et je reviendrai.
\VS{6}Et Pharaon répondit : Monte, et enterre ton père comme il t'a fait jurer.
\VS{7}Alors Joseph monta pour enterrer son père, et les serviteurs de Pharaon, les anciens de la maison de Pharaon, et tous les anciens du pays d'Egypte montèrent avec lui.
\VS{8}Et toute la maison de Joseph, et ses frères, et la maison de son père y montèrent aussi, laissant seulement leurs familles, et leurs troupeaux, et leurs bœufs dans le pays de Gosen.
\VS{9}Il y avait encore avec Joseph des chars et des cavaliers, en sorte que le cortège était très nombreux.
\VS{10}Arrivés à l’aire d’Athad, qui est au-delà du Jourdain, ils firent entendre de grandes et profondes lamentations ; et Joseph fit en l’honneur de son père un deuil de sept jours.
\VS{11}Et les Cananéens, habitants du pays, voyant ce deuil dans l'aire d'Athad, dirent : Ce deuil est grand pour les Egyptiens ; c'est pourquoi cette aire, qui est au-delà du Jourdain, fut nommée Abel-Mitsraïm\FTNT{Abel-Mitsraïm : «~Pré du deuil de l’Egypte~».}.
\VS{12}Les fils de Jacob firent à l'égard de son corps ce qu'il leur avait ordonné.
\VS{13}Ils le transportèrent au pays de Canaan, et l’enterrèrent dans la caverne du champ de Macpéla, qu’Abraham avait achetée d’Ephron, le Héthien, comme propriété sépulcrale, et qui est vis-à-vis de Mamré.
\VS{14}Et après que Joseph eut enseveli son père, il retourna en Egypte avec ses frères et tous ceux qui étaient montés avec lui pour ensevelir son père.
\VS{15}Et les frères de Joseph, voyant que leur père était mort, se dirent entre eux : Peut-être que Joseph nous aura en haine, et ne manquera pas de nous rendre tout le mal que nous lui avons fait.
\VS{16}Et ils firent dire à Joseph : Ton père a donné cet ordre avant de mourir en disant :
\VS{17}Vous parlerez ainsi à Joseph : Je te prie, pardonne maintenant l'iniquité de tes frères, et leur péché, car ils t'ont fait du mal. Maintenant, je te supplie, pardonne le crime des serviteurs du Dieu de ton père. Et Joseph pleura quand on lui parla.
\VS{18}Ses frères vinrent eux-mêmes se prosterner devant lui, et ils dirent : Nous sommes tes serviteurs.
\VS{19}Et Joseph leur dit : Ne craignez point, car suis-je à la place de Dieu ?
\VS{20}Vous aviez médité de me faire du mal : Dieu l’a changé en bien, pour accomplir ce qui arrive aujourd’hui, pour sauver la vie à un peuple nombreux.
\VS{21}Soyez donc sans crainte ; je vous entretiendrai, vous et vos familles ; et il les consola en parlant à leur cœur.
\VS{22}Joseph demeura donc en Egypte, lui et la maison de son père, et vécut cent dix ans.
\VS{23}Et Joseph vit les fils d'Ephraïm jusqu'à la troisième génération. Makir aussi, fils de Manassé, eut des fils qui furent élevés sur les genoux de Joseph.
\VS{24}Et Joseph dit à ses frères : Je vais mourir ! Mais Dieu ne manquera pas de vous visiter, et il vous fera remonter de ce pays au pays qu’il a juré  de donner à Abraham, Isaac et à Jacob.
\VS{25}Et Joseph fit jurer les enfants d'Israël et leur dit : Dieu ne manquera pas de vous visiter, et alors vous transporterez mes os d'ici\FTNT{Hé. 11:22 ; Ex. 13:19.}.
\VS{26}Puis Joseph mourut, âgé de cent dix ans. On l'embauma, et on le mit dans un cercueil en Egypte.
\PPE{}
\end{multicols}

\clearpage\ShortTitle{Ex.}\BookTitle{Exode}\BFont
\noindent\hrulefill
{\footnotesize
\textit{
\bigskip
{\centering{}
\\Auteur~: Probablement Moïse
\\(Heb.~: Shemot)
\\Signification~: Noms
\\Thème~: La délivrance
\\Date de rédaction~: Env. 1450-1410 av. J.-C.\\}
}
%\bigskip
\textit{
\\Les fils de Jacob s'étaient retrouvés en Egypte pour survivre à une famine qui avait frappé la terre entière pendant plusieurs années. Grâce à leur frère Joseph, alors gouverneur d'Egypte, ils bénéficièrent d'un bon traitement. Mais la mort de ce dernier et la montée au pouvoir d'un nouveau Pharaon (probablement Ramsès II) inaugurèrent une période de quatre siècles de souffrances pour le peuple élu.
%\bigskip
\\En effet, les Hébreux avaient été réduits en esclavage. En réponse aux cris de douleur de son peuple, Dieu suscita Moïse, dont le nom signifie «~tiré de~». Ce descendant de Lévi fut élevé dans le palais de Pharaon, mais dut s'enfuir parce qu'il avait tué un Egyptien. Après quarante ans passés dans le pays de Madian, le Dieu qui s'appelle «~Je suis~» se révéla à Moïse sur la montagne d'Horeb et lui confia la mission d'aller délivrer son peuple du joug égyptien.
%\bigskip
\\Ce livre retrace la sortie d'Egypte et le début de la traversée du désert, jalonnée de prodiges exceptionnels.\bigskip
}
}
\par\nobreak\noindent\hrulefill
\begin{multicols}{2}
\Chap{1}
\TextTitle{Après la mort de Joseph}
\VerseOne{}Et ce sont ici les noms des fils d'Israël qui entrèrent en Egypte avec Jacob. Ils y entrèrent chacun avec sa famille~: 
\VS{2}Ruben, Siméon, Lévi, et Juda,
\VS{3}Issacar, Zabulon, et Benjamin,
\VS{4}Dan, et Nephthali, Gad, et Aser.
\VS{5}Toutes les personnes issues des reins de Jacob étaient soixante-dix âmes. Joseph était alors en Egypte.
\VS{6}Joseph mourut ainsi que tous ses frères et toute cette génération-là.
\VS{7}Les enfants d'Israël fructifièrent et s'accrurent abondamment, et se multiplièrent et devinrent extrêmement puissants, de sorte que le pays en fut rempli\FTNT{De. 26:5~; Ac. 7:17.}.
\TextTitle{Israël esclave en Egypte}
\VS{8}Depuis, il s'éleva un nouveau roi sur l'Egypte, qui n'avait point connu Joseph.
\VS{9}Et il dit à son peuple~: Voici, le peuple des enfants d'Israël est plus grand et plus puissant que nous.
\VS{10}Agissons donc prudemment avec lui, de peur qu'il ne se multiplie, et que s'il survenait une guerre, il ne se joigne à nos ennemis, ne fasse la guerre contre nous, et qu'il ne s'en aille du pays.
\VS{11}Ils établirent donc sur le peuple des commissaires d'impôts, pour l'affliger en le surchargeant~; car le peuple bâtit des villes à greniers pour Pharaon~; à savoir Pithom et Ramsès.
\VS{12}Mais plus ils l'affligeaient et plus il multipliait et croissait en toute abondance~; c'est pourquoi ils haïssaient les enfants d'Israël\FTNT{Ps. 105:24.}.
\VS{13}Et les Egyptiens assujettirent les enfants d'Israël à une rude servitude\FTNT{Ge. 15:13.}.
\VS{14}Tellement qu'ils leur rendirent la vie amère par un rude travail, en les employant à faire du mortier, des briques, et toute sorte d'ouvrage qui se fait aux champs~; c'était avec cruauté qu'ils leur imposaient toutes ces charges.
\VS{15}Le roi d'Egypte parla aussi aux sages-femmes des Hébreux, nommées l'une Schiphra et l'autre Pua.
\VS{16}Il leur dit~: Quand vous accoucherez les femmes des Hébreux, et que vous les verrez sur les sièges, si c'est un fils, mettez-le à mort~; mais si c'est une fille, qu'elle vive.
\VS{17}Mais les sages-femmes craignirent Dieu et ne firent pas ce que le roi d'Egypte leur avait dit~; car elles laissèrent vivre les fils.
\VS{18}Alors le roi d'Egypte appela les sages-femmes et leur dit~: Pourquoi avez-vous fait cela, et avez-vous laissé vivre les fils~?
\VS{19}Les sages-femmes répondirent à Pharaon~: Parce que les femmes des Hébreux ne sont point comme les femmes Egyptiennes~; car elles sont vigoureuses, elles ont accouché avant que la sage-femme ne soit arrivée chez elles.
\VS{20}Dieu fit du bien aux sages-femmes~; et le peuple multiplia et devint très puissant.
\VS{21}Parce que les sages-femmes craignirent Dieu, il leur édifia des maisons.
\VS{22}Alors Pharaon donna cet ordre à tout son peuple~: Jetez dans le fleuve tous les fils qui naîtront, mais laissez vivre toutes les filles.
\Chap{2}
\TextTitle{Naissance de Moïse\FTNTT{Hé. 11:23-27.}}
\VerseOne{}Un homme de la maison de Lévi s'en alla et prit une fille de Lévi\FTNT{No. 26:59.}.
\VS{2}Cette femme conçut et enfanta un fils. Voyant qu'il était beau, elle le cacha pendant trois mois\FTNT{Hé. 11:23}.
\VS{3}Mais ne pouvant le tenir caché plus longtemps, elle prit une arche de jonc, et l'enduisit de bitume et de poix, mit l'enfant dedans, et le posa parmi des roseaux sur le bord du fleuve.
\VS{4}Et la sœur de cet enfant se tenait loin pour savoir ce qu'il en arriverait.
\VS{5}La fille de Pharaon descendit à la rivière pour se baigner, et ses servantes se promenaient sur le bord de la rivière, et ayant vu le coffret au milieu des roseaux, elle envoya une de ses servantes pour le prendre.
\VS{6}Et l'ayant ouvert, elle vit l'enfant et voici l'enfant pleurait. Elle en fut touchée de compassion et dit~: C'est un des enfants de ces Hébreux~!
\VS{7}Alors la sœur de l'enfant dit à la fille de Pharaon~: Irai-je appeler une femme d'entre les Hébreux, qui allaite~? Et elle t'allaitera cet enfant.
\VS{8}La fille de Pharaon lui répondit~: Va~! Et la jeune fille s'en alla et appela la mère de l'enfant.
\VS{9}Et La fille de Pharaon lui dit~: Emporte cet enfant, et allaite-le moi, je te donnerai ton salaire~; et la femme prit l'enfant et l'allaita.
\VS{10}Quand l'enfant fut devenu grand, elle l'amena à la fille de Pharaon~; il fut pour elle comme un fils. Elle lui donna le nom de Moïse parce que, dit-elle, je l'ai tiré des eaux.
\TextTitle{Moïse prend à cœur le sort d'Israël~; fuite à Madian}
\VS{11}Or il arriva, en ce temps-là, que Moïse, étant devenu grand, sortit vers ses frères et vit leurs travaux~; il vit aussi un Egyptien qui frappait un Hébreu d'entre ses frères\FTNT{Hé. 11:24-25.}.
\VS{12}Et ayant regardé çà et là, et voyant qu'il n'y avait personne, il tua l'Egyptien et le cacha dans le sable.
\VS{13}Il sortit encore le second jour~; et voici, deux hommes Hébreux se querellaient. Il dit à celui qui avait tort~: Pourquoi frappes-tu ton prochain~?
\VS{14}Lequel répondit~: Qui t'a établi prince et juge sur nous~? Veux-tu me tuer comme tu as tué l'Egyptien~? Et Moïse craignit, et dit~: Certainement le fait est connu.
\VS{15}Or Pharaon ayant appris ce fait-là, chercha à faire mourir Moïse~; mais Moïse s'enfuit de devant Pharaon, s'arrêta au pays de Madian et s'assit près d'un puits.
\VS{16}Or le prêtre de Madian avait sept filles qui vinrent puiser de l'eau, et elles remplirent les auges pour abreuver le troupeau de leur père.
\VS{17}Mais des bergers survinrent et les chassèrent~; et Moïse se leva et les secourut, et abreuva leur troupeau.
\VS{18}Et quand elles furent revenues chez Réuel, leur père, il leur dit~: Comment êtes-vous revenues si tôt aujourd'hui~?
\VS{19}Elles répondirent~: Un homme Egyptien nous a délivrées de la main des bergers~; et même il nous a puisé abondamment de l'eau et a abreuvé le troupeau.
\VS{20}Il dit à ses filles~: Où est-il~? Pourquoi avez-vous ainsi laissé cet homme~? Appelez-le, et qu'il mange du pain.
\VS{21}Et Moïse s'accorda de demeurer avec cet homme-là, qui donna Séphora, sa fille, à Moïse.
\VS{22}Et elle enfanta un fils, et il le nomma Guerschom~: Car, dit-il, je séjourne dans un pays étranger.
\TextTitle{Yahweh entend les cris de son peuple}
\VS{23}Or il arriva longtemps après que le roi d'Egypte mourut, et les enfants d'Israël soupirèrent à cause de la servitude, et ils crièrent~; et leur cri monta jusqu'à Dieu, à cause de la servitude\FTNT{No. 20:15-16.}.
\VS{24}Dieu entendit leurs gémissements, et Dieu se souvint de l'alliance qu'il avait traitée avec Abraham, Isaac et Jacob.
\VS{25}Ainsi Dieu regarda les enfants d'Israël et il fit attention à leur état.
\Chap{3}
\TextTitle{Yahweh se révèle à Moïse dans le buisson ardent}
\VerseOne{}Or Moïse fut berger du troupeau de Jéthro, son beau-père, prêtre de Madian~; il mena le troupeau derrière le désert, et vint à la montagne de Dieu à Horeb.
\VS{2}Et l'Ange de Yahweh lui apparut dans une flamme de feu, du milieu d'un buisson. Il regarda, et voici, le buisson était tout en feu, et le buisson ne se consumait point.
\VS{3}Alors Moïse dit~: Je me détournerai maintenant, et je regarderai cette grande vision, pourquoi le buisson ne se consume point.
\VS{4}Et Yahweh vit que Moïse s'était détourné pour regarder~; et Dieu l'appela du milieu du buisson, en disant~: Moïse~! Moïse~! Et il répondit~: Me voici~!
\VS{5}Et Dieu dit~: N'approche point d'ici~; déchausse tes souliers de tes pieds, car le lieu où tu es arrêté est une terre sainte.
\VS{6}Il dit aussi~: Je suis le Dieu de ton père, le Dieu d'Abraham, le Dieu d'Isaac et le Dieu de Jacob\FTNT{Mt. 22:32~; Mc. 12:26~; Lu. 20:37~; Ac. 7:32.}~; Moïse cacha son visage, parce qu'il craignait de regarder vers Dieu.
\VS{7} Et Yahweh dit~: J'ai très bien vu l'affliction de mon peuple qui est en Egypte et j'ai entendu le cri qu'ils ont jeté à cause de leurs oppresseurs, car je connais leurs douleurs.
\VS{8}C'est pourquoi je suis descendu pour le délivrer de la main des Egyptiens et pour le faire remonter de ce pays-là, dans un pays bon et vaste, dans un pays découlant de lait et de miel~; au lieu où sont les Cananéens, les Héthiens, les Amoréens, les Phéréziens, les Héviens et les Jébusiens.
\VS{9}Et maintenant, voici le cri des enfants d'Israël est parvenu à moi, et j'ai vu aussi l'oppression dont les Egyptiens les oppriment.
\VS{10}Maintenant donc viens, et je t'enverrai vers Pharaon~; et tu retireras mon peuple, les enfants d'Israël, hors d'Egypte\FTNT{Os. 12:14~; Mi. 6:4.}.
\VS{11}Et Moïse répondit à Dieu~: Qui suis-je, moi, pour aller vers Pharaon, et pour retirer de l'Egypte les enfants d'Israël~?
\VS{12}Et Dieu lui dit~: Va, car je serai avec toi. Et tu auras ce signe que c'est moi qui t'envoie~: C'est que quand tu auras retiré mon peuple d'Egypte, vous servirez Dieu près de cette montagne.
\TextTitle{Yahweh révèle son Nom à Moïse}
\VS{13}Et Moïse dit à Dieu~: Voici, quand je serai venu vers les enfants d'Israël, et que je leur aurai dit~: Le Dieu de vos pères m'a envoyé vers vous, s'ils me disent alors~: Quel est son Nom~? Que leur dirai-je~?
\VS{14} Et Dieu dit à Moïse~: JE SUIS CELUI QUI SUIS. Il dit aussi~: Tu diras ainsi aux enfants d'Israël~: Celui qui s'appelle JE SUIS\FTNT{Je suis («~Ehyeh~» en hébreu), c'est de là que vient le Nom de Yahweh. Or le Nom de Jésus signifie «~Yahweh est Salut~». Dieu révèle son Nom à Moïse~: «~Je suis celui qui suis~». Or Jésus-Christ s'est ouvertement attribué ce Nom en Jn. 8:58. N'ayant compris ni le plan de Dieu ni qui était celui qui les visitait, les religieux Juifs ont voulu le lapider car ils estimaient qu'il blasphémait. Car en déclarant être «~Je suis~», Jésus-Christ proclamait ouvertement sa divinité (Ro. 9:5), chose que les Juifs ne pouvaient concevoir. Dans l'évangile de Jean, Jésus déclare clairement qu'il est le «~JE SUIS~» d'Ex. 3:14. «~Je suis le pain de vie~» (Jn. 6:35), «~Je suis la lumière du monde~» (Jn. 8:12), «~Je suis le bon berger~» (Jn. 10:11), «~Je suis la porte~» (Jn. 10:7), «~Je suis la résurrection~» (Jn. 11:25), «~Je suis le chemin, la vérité et la vie~» (Jn. 14:6), «~Je suis la vraie vigne~» (Jn. 15:1).}, m'a envoyé vers vous.
\VS{15}Dieu dit encore à Moïse~: Tu diras ainsi aux enfants d'Israël~: Yahweh, le Dieu de vos pères, le Dieu d'Abraham, le Dieu d'Isaac et le Dieu de Jacob m'a envoyé vers vous. C'est ici mon Nom éternellement, et c'est ici le souvenir que vous aurez de moi de génération en génération.
\VS{16}Va, et rassemble les anciens d'Israël, et dis leur~: Yahweh, le Dieu de vos pères, le Dieu d'Abraham, d'Isaac et de Jacob, m'est apparu, en disant~: Certainement je vous ai visités, et j'ai vu ce qu'on vous fait en Egypte.
\VS{17}Et j'ai dit~: Je vous ferai remonter de l'Egypte où vous êtes affligés, dans le pays des Cananéens, des Héthiens, des Amoréens, des Phéréziens, des Héviens et des Jébusiens, qui est un pays découlant de lait et de miel.
\VS{18}Et ils obéiront à ta parole~; et tu iras, toi et les anciens d'Israël, vers le roi d'Egypte, et vous lui direz~: Yahweh, le Dieu des Hébreux, est venu nous rencontrer. Et maintenant donc, laisse-nous aller, nous te prions, à trois jours de marche dans le désert, afin que nous puissions sacrifier à Yahweh, notre Dieu.
\VS{19}Or je sais que le roi d'Egypte ne vous permettra point de vous en aller, si ce n'est par une main forte.
\VS{20}Mais j'étendrai ma main et je frapperai l'Egypte par toutes les merveilles que je ferai au milieu d'elle~; et après cela, il vous laissera aller.
\VS{21}Je ferai que ce peuple trouve grâce envers les Egyptiens, et il arrivera que, quand vous partirez, vous ne vous en irez point à vide.
\VS{22}Mais chacune demandera à sa voisine, et à l'hôtesse de sa maison, des vases d'argent, des vases d'or, et des vêtements, que vous mettrez sur vos fils et sur vos filles~: Ainsi vous dépouillerez les Egyptiens.
\Chap{4}
\TextTitle{Moïse résiste en évoquant l'incrédulité du peuple}
\VerseOne{}Et Moïse répondit, et dit~: Mais voici, ils ne me croiront pas et n'obéiront pas à ma parole~; car ils diront~: Yahweh ne t'est point apparu.
\VS{2}Et Yahweh lui dit~: Qu'est-ce que tu as dans ta main~? Il répondit~: Une verge.
\VS{3}Et Dieu lui dit~: Jette-la par terre~; il la jeta par terre et elle devint un serpent. Et Moïse s'enfuyait de devant lui.
\VS{4}Et Yahweh dit à Moïse~: Etends ta main et saisis sa queue~; et il étendit sa main et l'empoigna~; et il redevint une verge dans sa main.
\VS{5}C'est là ce que tu feras, afin qu'ils croient que Yahweh, le Dieu de leurs pères, le Dieu d'Abraham, le Dieu d'Isaac et le Dieu de Jacob, t'est apparu.
\VS{6}Yahweh lui dit encore~: Mets maintenant ta main dans ton sein, et il mit sa main dans son sein~; puis il la tira~; et voici, sa main était blanche de lèpre comme la neige.
\VS{7}Et Dieu lui dit~: Remets ta main dans ton sein~; et il remit sa main dans son sein~; puis il la retira hors de son sein~; et voici, elle était redevenue comme son autre chair.
\VS{8}Mais s'il arrive qu'ils ne te croient point, et qu'ils n'obéissent point à la voix du premier signe, ils croiront à la voix du second signe.
\VS{9}S'il arrive qu'ils ne croient point à ces deux signes et qu'ils n'obéissent point à ta parole, tu prendras de l'eau du fleuve et tu la répandras sur la terre, et les eaux que tu auras prises du fleuve deviendront du sang sur la terre.
\TextTitle{Moïse résiste en évoquant son incapacité à parler}
\VS{10}Et Moïse répondit à Yahweh~: Hélas~! Seigneur~! Je ne suis point un homme qui ait, ni d'hier ni d'avant-hier, la parole aisée, ni même depuis que tu parles à ton serviteur~; car j'ai la bouche et la langue empêchées.
\VS{11}Et Yahweh lui dit~: Qui a fait la bouche de l'homme~? Ou qui a fait le muet, ou le sourd, ou le voyant, ou l'aveugle~? N'est-ce pas moi Yahweh\FTNT{Ps. 94:9.}~?
\VS{12}Va donc maintenant, je serai avec ta bouche et je t'enseignerai ce que tu auras à dire\FTNT{Lu. 12:12~; Mt. 10:19~; Mc. 13:11.}.
\VS{13}Et Moïse répondit~: Hélas~! Seigneur~! Envoie, je te prie, celui que tu dois envoyer.
\VS{14}Et la colère de Yahweh s'enflamma contre Moïse, et il lui dit~: Aaron, le Lévite, n'est-il pas ton frère~? Je sais qu'il parlera très bien, et même le voilà qui sort à ta rencontre, et quand il te verra, il se réjouira dans son cœur.
\VS{15}Tu lui parleras donc et tu mettras ces paroles dans sa bouche~; je serai avec ta bouche et avec la sienne, et je vous enseignerai ce que vous aurez à faire.
\VS{16}Et il parlera pour toi au peuple, et ainsi il te sera pour bouche, et tu lui seras pour Dieu.
\VS{17}Tu prendras aussi dans ta main cette verge, avec laquelle tu feras ces signes-là.
\TextTitle{Moïse accepte sa mission et part en Egypte}
\VS{18}Ainsi Moïse s'en alla, et retourna vers Jéthro, son beau-père, et lui dit~: Je te prie, que je m'en aille, et que je retourne vers mes frères qui sont en Egypte, pour voir s'ils vivent encore. Et Jéthro lui dit~: Va en paix~!
\VS{19}Or Yahweh dit à Moïse au pays de Madian~: Va, et retourne en Egypte~; car tous ceux qui cherchaient ta vie sont morts.
\VS{20}Moïse prit sa femme et ses fils, les mit sur un âne, et retourna au pays d'Egypte. Moïse prit aussi la verge de Dieu dans sa main.
\VS{21}Et Yahweh avait dit à Moïse~: Quand tu t'en iras pour retourner en Egypte, tu prendras garde à tous les miracles que j'ai mis dans ta main~; et tu les feras devant Pharaon~; mais j'endurcirai son cœur et il ne laissera point aller le peuple.
\VS{22}Tu diras donc à Pharaon, ainsi parle Yahweh~: Israël est mon fils, mon premier-né\FTNT{Os. 11:1.}.
\VS{23}Et je t'ai dit~: Laisse aller mon fils, afin qu'il me serve. Mais tu as refusé de le laisser aller~: Voici, je m'en vais tuer ton fils, ton premier-né.
\VS{24}Or il arriva que, comme Moïse était en chemin dans l'hotellerie, Yahweh le rencontra et chercha à le faire mourir.
\VS{25}Et Séphora prit un couteau tranchant, coupa le prépuce de son fils et le jeta à ses pieds, et dit~: Certes, tu es pour moi un époux de sang~!
\VS{26}Alors Yahweh se retira de lui~; et Séphora dit~: Epoux de sang~; à cause de la circoncision.
\TextTitle{Yahweh envoie Aaron vers Moïse}
\VS{27}Et Yahweh dit à Aaron~: Va dans le désert, au-devant de Moïse. Il y alla donc, et le rencontra sur la montagne de Dieu et l'embrassa.
\VS{28}Et Moïse raconta à Aaron toutes les paroles de Yahweh qui l'avait envoyé, et tous les signes qu'il lui avait ordonné de faire.
\VS{29}Moïse donc poursuivit son chemin avec Aaron~; et ils assemblèrent tous les anciens des enfants d'Israël.
\VS{30}Et Aaron rapporta toutes les paroles que Yahweh avait dites à Moïse, et il exécuta les signes aux yeux du peuple.
\VS{31}Et le peuple crut. Ils apprirent que Yahweh avait visité les enfants d'Israël, qu'il avait vu leur affliction~; et ils s'inclinèrent et se prosternèrent.
\Chap{5}
\TextTitle{Pharaon s'oppose à Moïse\FTNTT{Ex. 5-14.}}
\VerseOne{}Après cela, Moïse et Aaron se rendirent ensuite auprès de Pharaon et lui dirent~: Ainsi parle Yahweh, le Dieu d'Israël~: Laisse aller mon peuple, afin qu'il me célèbre une fête solennelle dans le désert.
\VS{2}Mais Pharaon dit~: Qui est Yahweh pour que j'obéisse à sa voix et que je laisse aller Israël~? Je ne connais point Yahweh et je ne laisserai point aller Israël.
\VS{3}Et ils dirent~: Le Dieu des Hébreux est venu au-devant de nous. Permets-nous de faire trois journées de marche dans le désert, et que nous sacrifions à Yahweh, notre Dieu~; de peur qu'il ne se jette sur nous par la peste ou par l'épée.
\VS{4}Et le roi d'Egypte leur dit~: Moïse et Aaron, pourquoi détournez-vous le peuple de son ouvrage~? Allez maintenant à vos charges.
\VS{5}Pharaon dit aussi~: Voici, le peuple de ce pays est maintenant en grand nombre, et vous lui feriez cesser leur travail~!
\VS{6}Et ce jour-là, Pharaon donna cet ordre aux oppresseurs établis sur le peuple et à ses commissaires, en disant~:
\VS{7}Vous ne donnerez plus de paille à ce peuple pour faire des briques comme auparavant, mais qu'ils aillent s'amasser de la paille.
\VS{8}Néanmoins, vous leur imposerez la quantité de briques qu'ils faisaient auparavant, sans en rien diminuer~; car ils sont paresseux, et c'est pour cela qu'ils crient, en disant~: Allons et sacrifions à notre Dieu~!
\VS{9}Que la servitude soit aggravée sur ces gens-là, et qu'ils s'occupent, et ne s'amusent plus à des paroles de mensonge.
\VS{10}Alors les oppresseurs du peuple et ses commissaires sortirent et dirent au peuple~: Ainsi parle Pharaon~: Je ne vous donnerai plus de paille.
\VS{11}Allez vous-mêmes et prenez de la paille où vous en trouverez~; mais il ne sera rien diminué de votre travail.
\VS{12}Alors le peuple se répandit par tout le pays d'Egypte, pour ramasser du chaume au lieu de paille.
\VS{13}Et les oppresseurs les pressaient en disant~: Achevez vos ouvrages, chaque jour sa tâche, comme lorsque la paille vous était fournie.
\VS{14}Même les commissaires des enfants d'Israël, que les oppresseurs de Pharaon avaient établis sur eux, furent battus, et on leur dit~: Pourquoi n'avez-vous point achevé votre tâche en faisant des briques hier et aujourd'hui, comme auparavant~?
\VS{15}Alors les commissaires des enfants d'Israël vinrent crier à Pharaon, en disant~: Pourquoi fais-tu ainsi à tes serviteurs~?
\VS{16}On ne donne point de paille à tes serviteurs, et toutefois on nous dit~: Faites des briques. Et voici, tes serviteurs sont battus, et ton peuple est traité comme coupable.
\VS{17}Et il répondit~: Vous êtes des paresseux, des paresseux~! C'est pourquoi vous dites~: Allons, sacrifions à Yahweh~!
\VS{18}Maintenant donc allez, travaillez~; car on ne vous donnera point de paille, et vous rendrez la même quantité de briques.
\VS{19}Les commissaires des enfants d'Israël virent qu'ils souffraient, puisqu'on disait~: Vous ne diminuerez rien de vos briques sur la tâche de chaque jour.
\VS{20}Et en sortant de chez Pharaon, ils rencontrèrent Moïse et Aaron, qui se trouvèrent au-devant d'eux~;
\VS{21}et ils leur dirent~: Que Yahweh vous regarde, et en juge, vu que vous nous avez mis en mauvaise odeur devant Pharaon et devant ses serviteurs, leur mettant l'épée à la main pour nous tuer.
\VS{22}Alors Moïse retourna vers Yahweh, et dit~: Seigneur~! Pourquoi as-tu fait maltraiter ce peuple~? Pourquoi m'as-tu envoyé~?
\VS{23}Car depuis que je suis allé vers Pharaon pour parler en ton Nom, il a maltraité ce peuple, et tu n'as point délivré ton peuple.
\Chap{6}
\TextTitle{Yahweh fortifie Moïse et rappelle son alliance avec Israël}
\VerseOne{}Et Yahweh dit à Moïse~: Tu verras maintenant ce que je ferai à Pharaon~; car il les laissera aller y étant contraint par une main puissante, étant, dis-je contraint par ma main puissante, il les chassera de son pays.
\VS{2}Dieu parla encore à Moïse et lui dit~: Je suis Yahweh.
\VS{3}Je suis apparu à Abraham, à Isaac et à Jacob, comme le Dieu Tout-Puissant, mais je n'ai point été connu d'eux par mon Nom YAHWEH.
\VS{4}J'ai aussi fait cette alliance avec eux, que je leur donnerai le pays de Canaan, le pays de leurs pèlerinages, dans lequel ils ont demeuré comme étrangers.
\VS{5}Et j'ai entendu les sanglots des enfants d'Israël, que les Egyptiens tiennent esclaves, et je me suis souvenu de mon alliance~;
\VS{6}c'est pourquoi dis aux enfants d'Israël~: Je suis Yahweh, et je vous retirerai de dessous les charges des Egyptiens, et je vous délivrerai de leur servitude, je vous rachèterai à bras étendu, et par de grands jugements.
\VS{7}Et je vous prendrai pour être mon peuple, je vous serai Dieu~; et vous connaîtrez que je suis Yahweh, votre Dieu, qui vous retire de dessous les charges des Egyptiens.
\VS{8}Et je vous ferai entrer dans le pays au sujet duquel j'ai levé ma main que je le donnerai à Abraham, à Isaac et à Jacob, et je vous le donnerai en héritage~; je suis Yahweh.
\VS{9}Moïse donc parla de cette manière aux enfants d'Israël. Mais ils n'écoutèrent point Moïse, à cause de l'angoisse de leur esprit, et à cause de leur dure servitude.
\VS{10}Et Yahweh parla à Moïse, en disant~:
\VS{11}Va, et dis à Pharaon, roi d'Egypte, qu'il laisse sortir les enfants d'Israël de son pays.
\VS{12}Alors Moïse parla devant Yahweh, en disant~: Voici, les enfants d'Israël ne m'ont point écouté, et comment Pharaon m'écoutera-t-il, moi, qui suis incirconcis des lèvres~?
\VS{13} Mais Yahweh parla à Moïse et à Aaron, et leur ordonna d'aller trouver les enfants d'Israël, et Pharaon, roi d'Egypte, pour retirer les fils d'Israël du pays d'Egypte.
\TextTitle{Les chefs d'Israël}
\VS{14}Voici les chefs des pères~: Les fils de Ruben, premier-né d'Israël~: Hénoc et Pallu, Hetsron et Carmi~; ce sont là les familles de Ruben\FTNT{Ge. 46:9~; No. 26:5~; 1 Ch. 5:3.}.
\VS{15}Les fils de Siméon~: Jemuel, Jamin, Ohad, Jakin et Tsochar, et Saül, fils d'une Cananéenne~; ce sont là les familles de Siméon.
\VS{16}Voici les noms des fils de Lévi selon leur naissance~: Guerschon, Kehath et Merari. Les années de la vie de Lévi furent de cent trente-sept ans.
\VS{17}Les fils de Guerschon~: Libni et Schimeï, selon leurs familles.
\VS{18}Les fils de Kehath~: Amram, Jitsehar, Hébron et Uziel. Et les années de la vie de Kehath furent de cent trente-trois ans.
\VS{19}Les fils de Merari~: Machli et Muschi~; ce sont là les familles de Lévi selon leurs générations.
\VS{20}Or Amram prit Jokébed, sa tante, pour femme, qui lui enfanta Aaron et Moïse~; les années de la vie d'Amram furent de cent trente-sept ans.
\VS{21}Et les fils de Jitsehar~: Koré, Népheg et Zicri.
\VS{22}Et les fils d'Uziel~: Mischaël, Eltsaphan, et Sithri.
\VS{23}Aaron prit pour femme Elischéba, fille d'Amminadab, sœur de Nachschon, qui lui enfanta Nadab, Abihu, Eléazar et Ithamar.
\VS{24}Et les fils de Koré~: Assir, Elkana, et Abiasaph. Ce sont là les familles des Korites.
\VS{25}Eléazar, fils d'Aaron, prit pour femme une des filles de Puthiel, qui lui enfanta Phinées. Ce sont là les chefs des pères des Lévites selon leurs familles.
\VS{26}Or c'est là cet Aaron et ce Moïse à qui Yahweh dit~: Retirez les enfants d'Israël du pays d'Egypte selon leurs armées.
\VS{27}Ce sont eux qui parlèrent à Pharaon, roi d'Egypte, pour retirer d'Egypte les enfants d'Israël. C'est ce Moïse et c'est cet Aaron.
\VS{28}Le jour où Yahweh parla à Moïse dans le pays d'Egypte,
\VS{29}Yahweh parla à Moïse et dit~: Je suis Yahweh~; dis à Pharaon, roi d'Egypte, toutes les paroles que je t'ai dites.
\VS{30}Et Moïse dit en présence de Yahweh~: Voici, je suis incirconcis des lèvres, comment Pharaon m'écoutera-t-il~?
\Chap{7}
\TextTitle{L'appel de Moïse confirmé}
\VerseOne{}Et Yahweh dit à Moïse~: Voici, je t'ai établi pour être Dieu à Pharaon, et Aaron, ton frère, sera ton prophète.
\VS{2}Tu diras tout ce que je t'ordonnerai, et Aaron, ton frère, parlera à Pharaon pour qu'il laisse aller les enfants d'Israël hors de son pays.
\VS{3}J'endurcirai le cœur de Pharaon, et je multiplierai mes signes et mes miracles dans le pays d'Egypte.
\VS{4}Pharaon ne vous écoutera point~; je mettrai ma main sur l'Egypte, et je sortirai mes armées, mon peuple, les enfants d'Israël, du pays d'Egypte, par de grands jugements.
\VS{5}Les Egyptiens connaîtront\FTNT{Les nations reconnaîtront que Jésus-Christ est le Dieu d'Israël lorsqu'il reviendra en Sion pour délivrer et restaurer son peuple (Za. 14).} que je suis Yahweh quand j'aurai étendu ma main sur l'Egypte, et que j'aurai retiré du milieu d'eux les enfants d'Israël.
\VS{6}Et Moïse et Aaron firent comme Yahweh leur avait ordonné~; ils firent ainsi.
\VS{7}Or Moïse était âgé de quatre-vingts ans, et Aaron de quatre-vingt-trois ans quand ils parlèrent à Pharaon.
\TextTitle{La verge d'Aaron devient un serpent}
\VS{8}Yahweh parla à Moïse et à Aaron, en disant~:
\VS{9}Quand Pharaon vous parlera, en disant~: Faites un miracle~; tu diras alors à Aaron~: Prends ta verge, jette-la devant Pharaon et elle deviendra un serpent.
\VS{10}Moïse donc et Aaron allèrent auprès de Pharaon, et firent comme Yahweh avait ordonné~; Aaron jeta sa verge devant Pharaon et devant ses serviteurs, et elle devint un serpent.
\VS{11}Mais Pharaon fit venir aussi les sages et les enchanteurs~; et les magiciens d'Egypte, et eux aussi firent autant par leurs enchantements.
\VS{12}Ils jetèrent donc chacun leurs verges et elles devinrent des serpents~; mais la verge d'Aaron engloutit leurs verges.
\VS{13}Le cœur de Pharaon s'endurcit et il ne les écouta point~; selon ce que Yahweh avait dit.
\TextTitle{Les eaux du fleuve changées en sang}
\VS{14}Yahweh dit à Moïse~: Le cœur de Pharaon est endurci, il a refusé de laisser aller le peuple.
\VS{15}Va-t'en dès le matin vers Pharaon~; voici, il sortira pour aller près de l'eau~; tu te présenteras donc devant lui sur le bord du fleuve, et tu prendras dans ta main la verge qui a été changée en serpent.
\VS{16}Et tu lui diras~: Yahweh, le Dieu des Hébreux, m'avait envoyé vers toi pour te dire~: Laisse aller mon peuple, afin qu'il me serve au désert~; mais voici, tu ne m'as point écouté jusqu'ici.
\VS{17}Ainsi parle Yahweh~: A ceci tu sauras que je suis Yahweh~; je m'en vais frapper de la verge qui est dans ma main les eaux du fleuve, et elles seront changées en sang.
\VS{18}Et le poisson qui est dans le fleuve mourra, le fleuve deviendra puant, et les Egyptiens éprouveront du dégoût à boire des eaux du fleuve.
\VS{19}Yahweh parla aussi à Moïse~: Dis à Aaron~: Prends ta verge et étends ta main sur les eaux des Egyptiens, sur leurs rivières, sur leurs ruisseaux, et sur leurs marais, et sur tous les amas de leurs eaux, et elles deviendront du sang~; il y aura du sang par tout le pays d'Egypte, dans les vases de bois et de pierre.
\VS{20}Moïse donc et Aaron firent ce que Yahweh avait ordonné. Aaron, ayant levé la verge, en frappa les eaux du fleuve, sous les yeux de Pharaon et de ses serviteurs~; et toutes les eaux du fleuve furent changées en sang.
\VS{21}Et le poisson qui était dans le fleuve mourut, et le fleuve en devint puant, tellement que les Egyptiens ne pouvaient point boire les eaux du fleuve~; il y eut du sang dans tout le pays d'Egypte.
\VS{22}Et les magiciens d'Egypte en firent de même par leurs enchantements. Et le cœur de Pharaon s'endurcit tellement, qu'il ne les écouta point, selon ce que Yahweh avait dit.
\VS{23}Et Pharaon leur ayant tourné le dos, alla dans sa maison, et ne prit même pas à cœur ces choses qu'il avait vues.
\VS{24}Or tous les Egyptiens creusèrent autour du fleuve pour trouver de l'eau à boire, parce qu'ils ne pouvaient pas boire de l'eau du fleuve.
\VS{25}Il se passa sept jours depuis que Yahweh eut frappé le fleuve.
\TextTitle{Invasion de grenouilles}
\VS{26}Après cela, Yahweh dit à Moïse~: Va vers Pharaon et dis-lui~: Ainsi parle Yahweh~: Laisse aller mon peuple, afin qu'il me serve.
\VS{27}Si tu refuses de le laisser aller, voici, je m'en vais frapper de grenouilles toutes tes contrées~;
\VS{28}et le fleuve fourmillera de grenouilles, qui monteront et entreront dans ta maison, et dans la chambre où tu couches, et sur ton lit, et dans les maisons de tes serviteurs, et parmi tout ton peuple, dans tes fours et dans tes maies.
\VS{29}Ainsi, les grenouilles monteront sur toi, sur ton peuple et sur tous tes serviteurs.
\Chap{8}
\VerseOne{}Yahweh donc dit à Moïse~: Dis à Aaron~: Etends ta main avec ta verge sur les fleuves, sur les rivières, et sur les marais, et fais monter les grenouilles sur le pays d'Egypte.
\VS{2}Et Aaron étendit sa main sur les eaux de l'Egypte, et les grenouilles montèrent et couvrirent le pays d'Egypte.
\VS{3}Mais les magiciens firent de même par leurs enchantements et firent monter des grenouilles sur le pays d'Egypte.
\VS{4}Alors Pharaon appela Moïse et Aaron, et leur dit~: Fléchissez Yahweh par vos prières, afin qu'il retire les grenouilles de dessus moi et de dessus mon peuple~; et je laisserai aller le peuple, afin qu'ils sacrifient à Yahweh.
\VS{5}Et Moïse dit à Pharaon~: Glorifie-toi sur moi~! Pour quel temps fléchirai-je par mes prières Yahweh pour toi, et pour tes serviteurs et pour ton peuple, afin que Yahweh retire les grenouilles loin de toi et de tes maisons~? Il en demeurera seulement dans le fleuve.
\VS{6}Alors il répondit~: Pour demain. Et Moïse dit~: Il sera fait selon ta parole, afin que tu saches qu'il n'y a nul Dieu tel que Yahweh, notre Dieu.
\VS{7}Les grenouilles donc se retireront de toi, de tes maisons, de tes serviteurs et de ton peuple~; il en demeurera seulement dans le fleuve.
\VS{8}Alors Moïse et Aaron sortirent de chez Pharaon~; Moïse cria à Yahweh au sujet des grenouilles qu'il avait fait venir sur Pharaon.
\VS{9}Et Yahweh fit selon la parole de Moïse. Ainsi les grenouilles moururent dans les maisons, dans les villages et dans les champs.
\VS{10}On les amassa par monceaux, et la terre en fut infectée.
\VS{11}Mais Pharaon, voyant qu'il y avait du relâche, endurcit son cœur et ne les écouta point, selon ce que Yahweh avait dit.
\TextTitle{Invasion de poux}
\VS{12}Et Yahweh dit à Moïse~: Dis à Aaron~: Etends ta verge et frappe la poussière de la terre, et elle deviendra des poux dans tout le pays d'Egypte.
\VS{13}Et ils firent ainsi~; et Aaron étendit sa main avec sa verge, et frappa la poussière de la terre~; et elle fut changée en poux, sur les hommes et sur les bêtes~; toute la poussière du pays fut changée en poux dans tout le pays d'Egypte.
\VS{14}Et les magiciens voulurent faire de même par leurs enchantements, pour produire des poux, mais ils ne purent pas. Les poux furent donc tant sur les hommes que sur les bêtes.
\VS{15}Alors les magiciens dirent à Pharaon~: C'est ici le doigt de Dieu\FTNT{Lu. 11:20.}~! Toutefois, le cœur de Pharaon s'endurcit et il ne les écouta point, selon ce que Yahweh avait dit.
\TextTitle{Invasion de mouches}
\VS{16}Puis Yahweh dit à Moïse~: Lève-toi de bon matin, et présente-toi devant Pharaon~; voici, il sortira près de l'eau, et tu lui diras~: Ainsi parle Yahweh~: Laisse aller mon peuple, afin qu'il me serve.
\VS{17}Car si tu ne laisses pas aller mon peuple, voici, je m'en vais envoyer contre toi, contre tes serviteurs, contre ton peuple et contre tes maisons, un mélange d'insectes~; et les maisons des Egyptiens seront remplies de ce mélange, et la terre aussi sur laquelle ils seront \FTNT{Ps. 105:31~; Ps. 78:43.}.
\VS{18}Mais je distinguerai ce jour-là le pays de Gosen, où se tient mon peuple, tellement qu'il n'y aura nul mélange d'insectes~; afin que tu saches que je suis Yahweh au milieu de la terre.
\VS{19}Et je ferai la différence entre ton peuple et mon peuple~; demain, ce signe-là se fera.
\VS{20}Et Yahweh le fit ainsi~; et un grand mélange d'insectes entra dans la maison de Pharaon et dans chaque maison de ses serviteurs, et dans tout le pays d'Egypte, de sorte que la terre fut gâtée par ce mélange.
\TextTitle{Pharaon tente de compromettre Moïse}
\VS{21} Et Pharaon appela Moïse et Aaron, et leur dit~: Allez, sacrifiez à votre Dieu dans ce pays.
\VS{22}Mais Moïse dit~: Il n'est pas convenable de faire ainsi~; car nous sacrifierions à Yahweh, notre Dieu, l'abomination des Egyptiens. Voici, si nous sacrifions l'abomination des Egyptiens devant leurs yeux, ne nous lapideraient-ils pas~?
\VS{23}Nous irons le chemin de trois jours au désert, et nous sacrifierons à Yahweh, notre Dieu, comme il nous dira.
\VS{24}Alors Pharaon dit~: Je vous laisserai aller pour sacrifier dans le désert à Yahweh, votre Dieu~; toutefois, vous ne vous éloignerez pas en y allant. Fléchissez Yahweh pour moi, par vos prières.
\VS{25}Moïse dit~: Voici, je sors de chez toi et je supplierai Yahweh, afin que le mélange d'insectes se retire demain de Pharaon, de ses serviteurs, et de son peuple. Mais que Pharaon ne continue point à se moquer en ne laissant point aller le peuple pour sacrifier à Yahweh.
\VS{26}Alors Moïse sortit de chez Pharaon et fléchit Yahweh par la prière.
\VS{27}Et Yahweh fit selon la parole de Moïse~; et le mélange d'insectes se retira de Pharaon, de ses serviteurs et de son peuple~; il n'en resta pas un seul insecte.
\VS{28}Mais Pharaon endurcit son cœur cette fois encore et ne laissa point aller le peuple.
\Chap{9}
\TextTitle{La mort des troupeaux}
\VerseOne{}Alors Yahweh dit à Moïse~: Va vers Pharaon et dis-lui~: Ainsi parle Yahweh, le Dieu des Hébreux~: Laisse aller mon peuple, afin qu'il me serve.
\VS{2}Car si tu refuses de les laisser aller et si tu le retiens encore,
\VS{3}voici, la main de Yahweh sera sur ton bétail qui est dans les champs, tant sur les chevaux que sur les ânes, sur les chameaux, sur les bœufs, et sur les brebis, et il y aura une très grande mortalité.
\VS{4}Et Yahweh distinguera le bétail des Israélites du bétail des Egyptiens, afin que rien de ce qui est aux enfants d'Israël ne meure.
\VS{5}Et Yahweh fixa un temps, en disant~: Demain, Yahweh fera ceci dans le pays.
\VS{6}Yahweh donc fit cela dès le lendemain~; et tout le bétail des Egyptiens mourut~; mais du bétail des enfants d'Israël, il ne mourut pas une seule bête.
\VS{7}Et Pharaon envoya examiner, et voici, il n'y avait pas une seule bête morte du bétail des enfants d'Israël. Toutefois, le cœur de Pharaon s'endurcit, et il ne laissa point aller le peuple.
\TextTitle{Des ulcères sur les Egyptiens et les bêtes}
\VS{8}Alors Yahweh dit à Moïse et à Aaron~: Remplissez vos mains de cendre de fournaise~; et que Moïse les répande vers les cieux en la présence de Pharaon.
\VS{9}Et elles deviendront de la poussière sur tout le pays d'Egypte, et il s'en fera des ulcères bourgeonnant en pustules tant sur les hommes que sur les bêtes, dans tout le pays d'Egypte.
\VS{10}Ils prirent donc de la cendre de fournaise et se tinrent devant Pharaon~; Moïse la répandit vers les cieux et il se forma des ulcères bourgeonnant en pustules tant sur les hommes que sur les bêtes.
\VS{11} Et les magiciens ne purent se tenir devant Moïse, à cause des ulcères~; car les magiciens avaient des ulcères, comme tous les Egyptiens.
\VS{12} Et Yahweh endurcit le cœur de Pharaon, et il ne les écouta point selon ce que Yahweh avait dit à Moïse.
\TextTitle{L'Egypte frappée par la grêle et le feu}
\VS{13}Puis Yahweh parla à Moïse~: Lève-toi de bon matin, et présente-toi devant Pharaon, et dis-lui~: Ainsi parle Yahweh, le Dieu des Hébreux~: Laisse aller mon peuple, afin qu'il me serve.
\VS{14}Car cette fois, je vais faire venir toutes mes plaies contre ton cœur, sur tes serviteurs et sur ton peuple, afin que tu saches qu'il n'y a nul Dieu semblable à moi sur toute la terre.
\VS{15}Car maintenant si j'avais étendu ma main, je t'aurais frappé de la peste, toi et ton peuple, et tu serais effacé de la terre.
\VS{16}Mais certainement, je t'ai fait subsister pour te faire voir ma puissance, afin que mon Nom soit célébré sur toute la terre\FTNT{Ro. 9:17.}.
\VS{17}T'élèves-tu encore contre mon peuple, pour ne point le laisser aller~?
\VS{18}Voici, je m'en vais faire pleuvoir demain à cette même heure, une grêle tellement forte qu'il n'y en a point eu de semblable en Egypte, depuis le jour où elle fut fondée jusqu'à maintenant.
\VS{19}Maintenant envoie rassembler ton bétail et tout ce que tu as à la campagne~; car la grêle tombera sur tous les hommes, sur le bétail qui se trouvera à la campagne, et qu'on n'aura pas renfermé, et ils mourront.
\VS{20}Celui d'entre les serviteurs de Pharaon, qui craignit la parole de Yahweh, fit promptement retirer dans les maisons ses serviteurs et ses bêtes.
\VS{21}Mais celui qui n'appliqua point son cœur à la parole de Yahweh, laissa ses serviteurs et ses bêtes à la campagne.
\VS{22}Et Yahweh dit à Moïse~: Etends ta main vers les cieux, et il y aura de la grêle sur tout le pays d'Egypte, sur les hommes et sur les bêtes, et sur toutes les herbes des champs au pays d'Egypte.
\VS{23}Moïse donc étendit sa verge vers les cieux, et Yahweh envoya des tonnerres et de la grêle, et le feu se promenait sur la terre. Yahweh fit pleuvoir de la grêle sur le pays d'Egypte.
\VS{24}Il y eut donc de la grêle et du feu entremêlé avec la grêle, laquelle était si grosse qu'il n'y en avait point eu de semblable sur toute la terre d'Egypte, depuis qu'elle a été habitée.
\VS{25}La grêle frappa dans tout le pays d'Egypte tout ce qui était aux champs, depuis les hommes jusqu'aux bêtes. La grêle frappa aussi toutes les herbes des champs et brisa tous les arbres des champs.
\VS{26}Il n'y eut que la contrée de Gosen, dans laquelle étaient les enfants d'Israël, où il n'y eut point de grêle.
\TextTitle{Pharaon continue d'endurcir son cœur}
\VS{27}Alors Pharaon envoya appeler Moïse et Aaron, et leur dit~: J'ai péché cette fois~; Yahweh est juste, mais moi et mon peuple sommes méchants.
\VS{28}Fléchissez par des prières Yahweh~: Que ce soit assez, et que Dieu ne fasse plus tonner ni grêler, car je vous laisserai aller, et on ne vous arrêtera plus.
\VS{29}Alors Moïse dit~: Aussitôt que je sortirai de la ville, j'étendrai mes mains vers Yahweh et les tonnerres cesseront. Il n'y aura plus de grêle, afin que tu saches que la terre est à Yahweh\FTNT{Ps. 24:1.}.
\VS{30}Mais quant à toi et tes serviteurs, je sais que vous ne craindrez pas encore Yahweh Dieu.
\VS{31}Or le lin et l'orge avaient été frappés, car l'orge était en épis et c'était la floraison du lin.
\VS{32} Mais le blé et l'épeautre ne furent point frappés, parce qu'ils sont tardifs.
\VS{33}Moïse donc sortit de chez Pharaon pour aller hors de la ville. Il étendit ses mains vers Yahweh, et les tonnerres cessèrent, et la grêle et la pluie ne tombèrent plus sur la terre.
\VS{34}Pharaon, voyant que la pluie, la grêle, et les tonnerres avaient cessé, continua encore à pécher, et il endurcit son cœur, lui et ses serviteurs.
\VS{35}Le cœur donc de Pharaon s'endurcit et il ne laissa point aller les enfants d'Israël, selon ce que Yahweh avait dit par l'intermédiaire de Moïse.
\Chap{10}
\TextTitle{Invasion de sauterelles}
\VerseOne{} Et Yahweh dit à Moïse~: Va vers Pharaon, car j'ai endurci son cœur et le cœur de ses serviteurs, afin que je mette au-dedans de lui les signes que je m'en vais faire~;
\VS{2}et afin que tu racontes à ton fils et au fils de ton fils, les signes que j'accomplirai sur les Egyptiens et les prodiges que je ferai au milieu d'eux, et que vous sachiez que je suis Yahweh.
\VS{3}Moïse donc et Aaron vinrent vers Pharaon, et lui dirent~: Ainsi parle Yahweh, le Dieu des Hébreux~: Jusqu'à quand refuseras-tu de t'humilier devant moi~? Laisse aller mon peuple, afin qu'il me serve.
\VS{4}Car si tu refuses de laisser aller mon peuple, voici, je ferai venir demain des sauterelles dans ton territoire.
\VS{5}Elles couvriront la face de la terre, et l'on ne pourra plus voir la terre~; elles dévoreront le reste de ce qui a échappé, ce que la grêle vous a laissé~; et elles dévoreront tous les arbres qui poussent dans vos champs.
\VS{6}Et elles rempliront tes maisons, et les maisons de tous tes serviteurs, et les maisons de tous les Egyptiens~; ce que tes pères n'ont point vu ni les pères de tes pères, depuis qu'ils existent sur la terre jusqu'à ce jour. Puis, ayant tourné le dos à Pharaon, il sortit d'auprès de lui.
\VS{7}Et les serviteurs de Pharaon lui dirent~: Jusqu'à quand celui-ci nous sera-t-il un piège~? Laisse aller ces gens et qu'ils servent Yahweh, leur Dieu. Attendras-tu de savoir avant cela que l'Egypte est perdue~?
\VS{8}Alors on fit revenir Moïse et Aaron vers Pharaon, il leur dit~: Allez, servez Yahweh, votre Dieu. Qui sont tous ceux qui iront~?
\VS{9} Et Moïse répondit~: Nous irons avec nos jeunes gens et nos vieillards, avec nos fils et nos filles~; nous irons avec nos brebis et nos bœufs~; car nous avons à célébrer une fête solennelle à Yahweh.
\VS{10}Alors il leur dit~: Que Yahweh soit avec vous, comme je laisserai aller vos petits enfants~! Prenez garde, car le mal est devant vous.
\VS{11}Il n'en sera pas ainsi que vous l'avez demandé~; mais vous, hommes, allez maitenant et servez Yahweh~; car c'est ce que vous demandiez. Et on les chassa de la présence de Pharaon.
\VS{12}Alors Yahweh dit à Moïse~: Etends ta main sur le pays d'Egypte, pour faire venir les sauterelles, afin qu'elles montent sur le pays d'Egypte, qu'elles dévorent toute l'herbe de la terre, tout ce que la grêle a laissé.
\VS{13}Moïse étendit donc sa verge sur le pays d'Egypte~; et Yahweh amena sur le pays, tout ce jour-là et toute la nuit, un vent d'orient~; le matin vint, et le vent d'orient enleva les sauterelles.
\VS{14}Et il fit monter les sauterelles sur tout le pays d'Egypte, et les mit dans toutes les contrées d'Egypte~; elles étaient fort grosses et il y n'en avait point eu avant elles de semblables, et il n'y en aura point de semblables après elles.
\VS{15}Et elles couvrirent la face de tout le pays, tellement que le pays en fut obscurci~; elles dévorèrent toute l'herbe de la terre, tout le fruit des arbres que la grêle avait laissé~; il ne resta aucune verdure aux arbres ni aux herbes des champs, dans tout le pays d'Egypte.
\VS{16}Aussitôt Pharaon se hâta d'appeler Moïse et Aaron, et dit~: J'ai péché contre Yahweh, votre Dieu, et contre vous.
\VS{17}Mais pardonne, je te prie, mon péché, pour cette fois seulement~; et suppliez Yahweh, votre Dieu, par vos prières, afin qu'il retire de moi cette mort-ci seulement.
\VS{18}Il sortit donc de chez Pharaon, et fléchit Yahweh par ses prières.
\VS{19}Et Yahweh fit lever un vent d'occident très fort qui enleva les sauterelles et les précipita dans la Mer Rouge. Il ne resta pas une seule sauterelle dans tout le territoire de l'Egypte.
\VS{20}Mais Yahweh endurcit le cœur de Pharaon et il ne laissa point aller les enfants d'Israël.
\TextTitle{Les ténèbres sur les Egyptiens}
\VS{21}Puis Yahweh dit à Moïse~: Etends ta main vers les cieux, qu'il y ait sur le pays d'Egypte des ténèbres si épaisses, qu'on puisse les toucher à la main.
\VS{22}Moïse étendit donc sa main vers les cieux, et il y eut d'épaisses ténèbres dans tout le pays d'Egypte, pendant trois jours\FTNT{Ps. 105:28.}.
\VS{23}On ne se voyait pas l'un l'autre, et nul ne se leva de sa place pendant trois jours. Mais pour tous les enfants d'Israël, il y eut de la lumière dans le lieu de leurs demeures.
\TextTitle{Pharaon tente encore de compromettre Moïse}
\VS{24}Alors Pharaon appela Moïse et dit~: Allez, servez Yahweh~; que vos brebis et vos bœufs seuls demeurent~; vos petits enfants iront aussi avec vous.
\VS{25}Moïse répondit~: Tu mettras toi-même entre nos mains de quoi faire des sacrifices et des holocaustes, que nous ferons à Yahweh, notre Dieu.
\VS{26}Et même, nos troupeaux viendront aussi avec nous, il n'en restera pas un sabot. Car nous en prendrons pour servir Yahweh, notre Dieu~; car nous ne savons pas ce que nous choisirons pour offrir à Yahweh, jusqu'à ce que nous soyons arrivés en ce lieu là.
\VS{27}Mais Yahweh endurcit le cœur de Pharaon et il ne voulut point les laisser aller.
\VS{28}Et Pharaon lui dit~:Va-t-en~!Arrière de moi~! Garde-toi de revoir ma face, car le jour où tu verras ma face, tu mourras.
\VS{29}Alors Moïse répondit~: Tu as bien dit, je ne reverrai plus ta face\FTNT{Hé. 11:27.}.
\Chap{11}
\TextTitle{Pharaon méprise l'avertissement sur la mort des premiers-nés}
\VerseOne{}Or Yahweh dit à Moïse~: Je ferai venir encore une plaie sur Pharaon, et sur l'Egypte, et après cela il vous laissera aller d'ici~; il vous laissera entièrement aller, et vous chassera tout à fait d'ici.
\VS{2}Parle maintenant aux oreilles du peuple, et dis leur~: Que chacun demande à son voisin, et chacune à sa voisine, des vases d'argent et des vases d'or.
\VS{3}Or Yahweh fit trouver grâce au peuple devant les Egyptiens~; et même Moïse passait pour un grand homme dans le pays d'Egypte, tant parmi les serviteurs de Pharaon que parmi le peuple.
\VS{4}Et Moïse dit~: Ainsi parle Yahweh~: Vers le milieu de la nuit, je passerai au travers de l'Egypte~;
\VS{5}et tout premier-né mourra dans le pays d'Egypte, depuis le premier-né de Pharaon, qui devait être assis sur son trône, jusqu'au premier-né de la servante qui est derrière la meule, et jusqu'à tous les premiers-nés des bêtes.
\VS{6}Et il y aura un grand cri dans tout le pays d'Egypte, tel qu'il n'y en a jamais eu et qu'il n'y en aura jamais de semblable.
\VS{7}Mais contre tous les enfants d'Israël, un chien même ne remuera point sa langue, depuis l'homme jusqu'aux bêtes~; afin que vous sachiez que Dieu fera la différence entre les Egyptiens et les Israélites.
\VS{8}Et tous tes serviteurs viendront vers moi, et se prosterneront devant moi, en disant~: Sors, toi, et tout le peuple qui est avec toi. Après cela, je sortirai. Ainsi, Moïse sortit de chez Pharaon dans une ardente colère.
\VS{9}Yahweh donc dit à Moïse~: Pharaon ne vous écoutera point, afin que mes miracles soient multipliés dans le pays d'Egypte.
\VS{10}Et Moïse et Aaron firent tous ces miracles-là devant Pharaon. Et Yahweh endurcit le cœur de Pharaon, tellement qu'il ne laissa point aller les enfants d'Israël hors de son pays.
\Chap{12}
\TextTitle{La première Pâque}
\VerseOne{}Or Yahweh dit à Moïse et à Aaron dans le pays d'Egypte~:
\VS{2}Ce mois-ci sera pour vous le premier des mois, il sera pour vous le premier des mois de l'année.
\VS{3}Parlez à toute l'assemblée d'Israël, en disant~: Jusqu'au dixième jour de ce mois, que chacun prenne un petit d'entre les brebis ou d'entre les chèvres, selon les familles des pères~; un petit, dis-je, d'entre les brebis ou d'entre les chèvres, par famille.
\VS{4}Mais si la famille est moindre qu'il ne faut pour manger un petit d'entre les brebis ou d'entre les chèvres, qu'elle le prenne avec son voisin qui est près de sa maison, selon le nombre de personnes~; vous compterez combien il en faudra pour manger d'entre les brebis ou d'entre les chèvres, ayant égard à ce que chacun de vous peut manger.
\VS{5}Or le petit d'entre les brebis ou d'entre les chèvres sera sans défaut, et sera un mâle ayant un an\FTNT{La Pâque juive était célébrée le 14ème jour du premier mois de l'année juive soit, le 14 du mois de Nissan (Ex. 12:2~; No. 9:1-5). L'agneau pascal était une préfiguration de Jésus-Christ~: L'agneau de Dieu qui ôte le péché du monde (Jn. 1:29). Ses caractéristiques sont les suivantes~: \\- L'agneau devait nécessairement être un mâle sans défaut (Ex. 12:5). Jésus est l'enfant mâle mis au monde par une vierge, il n'a pas été affecté par le sang corrompu d'Adam, il est donc sans défaut (Es. 7:14~; Mt. 1:20-21). Pour être certains de la perfection de l'animal, les hébreux devaient l'examiner pendant quatre jours avant de l'immoler (Ex. 12:3-6). Il est à noter que la loi juive exigeait que deux ou trois témoins soient présents pour constater un crime ou un péché (De. 17:6~; De. 19:15), ces quatre jours font donc office de quatre témoins pour attester de la pureté de l'animal. De même, les quatre auteurs de l'évangile attestent la sainteté du Seigneur. De plus, avant sa mise à mort, le Seigneur a été examiné par deux législations~: juive (le sanhédrin) et romaine (Ponce Pilate). Ces deux législations attestèrent, malgré elles, son innocence (Mt. 25:60~; Mt. 27:24~; Mc. 14:55-56~; Mc. 15:14~; Lu. 23:4~; Jn. 18:31~; Jn. 19:6) et confirmèrent qu'il était sans défaut et donc digne d'être offert en sacrifice.\\- Yahweh avait prescrit aux hébreux d'immoler l'agneau entre les deux soirs (Ex. 12:6), c'est-à-dire avant le crépuscule, entre la neuvième et la onzième heure. Jésus fut arrêté la nuit de Pâque (Mc. 14:12-41). Sa crucifixion eut lieu le lendemain, à la troisième heure (Mc. 15:25), et sa mort survint à la neuvième heure (Mt. 27:45). L'agneau devait être rôti au feu puis consommé avec du pain sans levain et des herbes amères (Ex. 12:8). Le feu symbolise le jugement que le Seigneur a pris sur lui à cause de nos péchés (Es. 53:5~; Ro. 4:25~; 1 Pi. 1:18-20). Le pain sans levain est une autre image de Jésus, le pain de vie (Jn. 6:35) sans aucun péché (1 Co. 5:8). Les herbes amères préfigurent, quant à elles, l'affliction et la souffrance du Seigneur (Hé. 2:10).}~; vous le prendrez d'entre les brebis ou d'entre les chèvres~;
\VS{6}et vous le garderez jusqu'au quatorzième jour de ce mois~; et toute la congrégation de l'assemblée d'Israël l'égorgera entre les deux soirs.
\VS{7} Et ils prendront de son sang, et le mettront sur les deux poteaux et sur le linteau de la porte des maisons où ils le mangeront.
\VS{8} Et ils en mangeront la chair rôtie au feu cette nuit-là~; et ils la mangeront avec des pains sans levain, et avec des herbes amères.
\VS{9}N'en mangez rien à demi cuit, ni qui ait été bouilli dans l'eau~; mais qu'il soit rôti au feu, sa tête, ses jambes et ses entrailles.
\VS{10} Et ne laissez aucun reste jusqu'au matin, mais s'il en reste quelque chose le matin, vous le brûlerez au feu.
\VS{11}Et vous le mangerez ainsi~: Vos reins seront ceints, vous aurez vos souliers à vos pieds, et votre bâton à la main, et vous le mangerez à la hâte. C'est la Pâque de Yahweh.
\TextTitle{Le sang qui sauve~; l'instauration de la fête de la Pâque}
\VS{12}Car je passerai cette nuit-là par le pays d'Egypte, et je frapperai tout premier-né au pays d'Egypte, depuis les hommes jusqu'aux bêtes~; et j'exercerai des jugements sur tous les dieux de l'Egypte. Je suis Yahweh.
\VS{13}Et le sang sera pour vous un signe sur les maisons où vous serez~; car je verrai le sang et je passerai par-dessus vous, et il n'y aura point de plaie à destruction quand je frapperai le pays d'Egypte.
\VS{14}Et ce jour là, vous conserverez le souvenir de ce jour, et vous le célébrerez comme une fête solennelle à Yahweh~; vous le célébrerez comme une fête solennelle par une ordonnance perpétuelle de génération en génération.
\VS{15}Vous mangerez pendant sept jours des pains sans levain, et dès le premier jour, vous ôterez le levain de vos maisons~; car quiconque mangera du pain levé, depuis le premier jour jusqu'au septième, cette personne-là sera retranchée d'Israël.
\VS{16}Au premier jour il y aura une sainte convocation, et il y aura de même au septième jour une sainte convocation~; il ne se fera aucune œuvre dans ces jours-là~; seulement, on vous apprêtera à manger ce qu'il faudra pour chaque personne.
\VS{17}Vous prendrez donc garde aux pains sans levain, parce qu'en ce même jour, j'aurai retiré vos armées du pays d'Egypte~; vous observerez donc ce jour-là de génération en génération par une ordonance perpétuelle.
\VS{18}Au premier mois, le quatorzième jour du mois, au soir, vous mangerez des pains sans levain jusqu'au vingt et unième jour du mois, au soir.
\VS{19}Il ne se trouvera point de levain dans vos maisons pendant sept jours, car quiconque mangera du pain levé, cette personne-là sera retranchée de l'assemblée d'Israël, tant celui qui habite comme étranger que celui qui est né au pays.
\VS{20}Vous ne mangerez point de pain levé~; mais vous mangerez dans tous les lieux où vous demeurerez des pains sans levain.
\VS{21}Moïse donc appela tous les anciens d'Israël et leur dit~: Choisissez et prenez un petit d'entre les brebis ou d'entre les chèvres selon vos familles, et égorgez la Pâque.
\VS{22}Puis vous prendrez un bouquet d'hysope et le tremperez dans le sang qui sera dans un bassin, et vous arroserez du sang qui sera dans le bassin, le linteau et les deux poteaux~; et nul de vous ne sortira de la porte de sa maison jusqu'au matin.
\VS{23}Car Yahweh passera pour frapper l'Egypte et il verra le sang sur le linteau et sur les deux poteaux, et Yahweh passera par-dessus la porte, et ne permettra point que le destructeur entre dans vos maisons pour frapper.
\VS{24}Vous garderez ceci comme une ordonnance perpétuelle pour toi et pour tes fils.
\VS{25}Quand donc vous serez entrés dans le pays que Yahweh vous donnera, selon qu'il en a parlé, vous observerez ce service.
\VS{26}Et quand vos fils vous diront~: Que signifie pour vous ce service~?
\VS{27}Alors vous répondrez~: C'est le sacrifice de la Pâque à Yahweh, qui passa en Egypte par-dessus les maisons des enfants d'Israël, quand il frappa l'Egypte, et qu'il préserva nos maisons. Alors le peuple s'inclina et se prosterna.
\VS{28}Ainsi les enfants d'Israël s'en allèrent et firent comme Yahweh l'ordonna à Moïse et à Aaron, ils le firent ainsi.
\TextTitle{Les premiers-nés d'Egypte frappés}
\VS{29}Et il arriva qu'à minuit Yahweh frappa tous les premiers-nés du pays d'Egypte, depuis le premier-né de Pharaon, qui devait être assis sur son trône, jusqu'aux premiers-nés des captifs qui étaient dans la prison, et tous les premiers-nés des bêtes.
\VS{30}Et Pharaon se leva de nuit, lui et ses serviteurs, et tous les Egyptiens~; et il y eut un grand cri en Egypte, parce qu'il n'y avait point de maison où il n'y ait eu un mort\FTNT{Hé. 11:28~; No. 8:17~; Ps. 78:51~; Ps. 105:36.}.
\TextTitle{Israël sort d'Egypte}
\VS{31}Il appela donc Moïse et Aaron de nuit, et leur dit~: Levez-vous, sortez du milieu de mon peuple, tant vous que les enfants d'Israël, allez et servez Yahweh, comme vous en avez parlé.
\VS{32}Prenez aussi votre menu et gros bétail, comme vous en avez parlé, et allez-vous-en et bénissez-moi.
\VS{33}Et les Egyptiens pressaient le peuple et se hâtaient de les faire sortir du pays, car ils disaient~: Nous sommes tous morts.
\VS{34}Le peuple donc prit sa pâte avant qu'elle fût levée, ayant leurs mains liées avec leurs vêtements, sur leurs épaules.
\VS{35}Or les enfants d'Israël firent selon la parole de Moïse, et demandèrent aux Egyptiens des vases d'argent et d'or, et des vêtements.
\VS{36}Et Yahweh fit trouver grâce au peuple auprès des Egyptiens, qui les leur prêtèrent~; de sorte qu'ils dépouillèrent les Egyptiens.
\VS{37}Ainsi, les fils d'Israël étant partis de Ramsès, vinrent à Succoth, environ six cent mille hommes de pied, sans les enfants.
\VS{38}Il s'en alla aussi avec eux un grand nombre de toutes sortes de gens~; et du menu et du gros bétail, en fort grands troupeaux.
\VS{39} Or parce qu'ils avaient été chassés d'Egypte, et qu'ils n'avaient pas pu tarder plus longtemps, et que même ils n'avaient fait aucune provision, ils cuisirent par gâteaux sans levain, la pâte qu'ils avaient emportée d'Egypte~; car ils ne l'avaient point fait lever. 
\VS{40}Or le séjour des enfants d'Israël en Egypte fut de quatre cent trente ans\FTNT{Ge. 15:13~; Ac. 7:6~; Ga. 3:17.}.
\VS{41}Il arriva donc au bout de quatre cent trente ans, il arriva dis-je, en ce propre jour-là, que toutes les armées de Yahweh sortirent du pays d'Egypte.
\VS{42}C'est la nuit qui doit être soigneusement observée en l'honneur de Yahweh, parce qu'alors il les retira du pays d'Egypte~; cette nuit-là est à observer en l'honneur de Yahweh, par tous les enfants d'Israël de génération en génération\FTNT{De. 16:1-6.}.
\VS{43}Yahweh dit aussi à Moïse et à Aaron~: C'est ici l'ordonnance de la Pâque~: Aucun étranger n'en mangera~;
\VS{44}mais tout esclave qu'on aura acheté par argent sera circoncis, et alors il en mangera.
\VS{45}L'étranger et le mercenaire n'en mangeront point.
\VS{46}On la mangera dans une même maison, et vous n'emporterez point de sa chair hors de la maison, et vous n'en casserez point les os.
\VS{47}Toute l'assemblée d'Israël la fera.
\VS{48}Et si quelque étranger qui habite chez toi veut faire la Pâque à Yahweh, que tout mâle qui lui appartient soit circoncis~; et alors il s'approchera pour la faire, et il sera comme celui qui est né dans le pays~; mais aucun incirconcis n'en mangera.
\VS{49}Il y aura une même loi pour celui qui est né dans le pays et pour l'étranger qui habite parmi vous.
\VS{50}Tous les enfants d'Israël firent ce que Yahweh avait ordonné à Moïse et à Aaron~; ils le firent ainsi.
\VS{51}Il arriva donc en ce même jour que Yahweh retira les enfants d'Israël du pays d'Egypte, selon leurs armées.
\Chap{13}
\TextTitle{Consécration des premiers-nés à Yahweh}
\VerseOne{}Et Yahweh parla à Moïse, et dit~:
\VS{2}Sanctifie-moi tout premier-né, tout premier-né issu du sein maternel parmi les fils d'Israël, tant des hommes que des bêtes, car il est à moi\FTNT{Lé. 27:26-27~; No. 3:13~; No. 8:17~; Lu. 2:22-23.}.
\VS{3}Moïse donc dit au peuple~: Souvenez-vous de ce jour où vous êtes sortis d'Egypte, de la maison de servitude~; car Yahweh vous en a retirés par sa main puissante~; on ne mangera donc point de pain levé.
\VS{4}Vous sortez aujourd'hui dans le mois où les épis mûrissent.
\VS{5}Quand donc Yahweh t'aura introduit dans le pays des Cananéens, des Héthiens, des Amoréens, des Héviens et des Jébusiens, qu'il a juré à tes pères de te donner, et qui est un pays découlant de lait et de miel, alors tu feras ce service durant ce mois-ci.
\VS{6}Pendant sept jours tu mangeras des pains sans levain, et au septième jour il y aura une fête solennelle à Yahweh.
\VS{7}On mangera durant sept jours des pains sans levain~; il ne sera point vu chez toi de pain levé et même il ne sera point vu de levain dans toutes tes contrées.
\VS{8}Et ce jour-là, tu feras entendre ces choses à tes enfants, en disant~: C'est à cause de ce que Yahweh m'a fait en me retirant d'Egypte.
\VS{9}Et ceci te sera pour signe sur ta main, et comme un rappel entre tes yeux, afin que la loi de Yahweh soit dans ta bouche, car Yahweh t'aura retiré d'Egypte par sa main puissante\FTNT{De. 6:8~; De. 11:18.}.
\VS{10}Tu observeras cette ordonnance au jour fixé d'année en année.
\VS{11}Aussi, quand Yahweh t'aura introduit dans le pays des Cananéens, selon qu'il a juré à toi et à tes pères, et qu'il te l'aura donné,
\VS{12}tu consacreras à Yahweh tout premier-né issu du sein de sa mère, même tout premier-né des animaux que tu auras~; les mâles appartiendront à Yahweh.
\VS{13}Et tu rachèteras avec un petit d'entre les brebis ou d'entre les chèvres, tout premier-né de l'ânesse, et si tu ne le rachètes point, tu lui briseras la nuque. Tu rachèteras aussi tout premier-né des hommes parmi tes fils.
\VS{14}Et quand ton fils t'interrogera à l'avenir, en disant~: Que veut dire ceci~? Alors tu lui diras~: Yahweh nous a retirés par main forte hors d'Egypte, de la maison de servitude.
\VS{15}Car il arriva que, quand Pharaon s'obstinait à ne point nous laisser aller, Yahweh tua tous les premiers-nés au pays d'Egypte, depuis les premiers-nés des hommes jusqu'aux premiers-nés des bêtes. Voilà pourquoi je sacrifie à Yahweh tout premier-né mâle issu du sein de sa mère, et je rachète tout premier-né de mes fils.
\VS{16}Ceci te sera donc pour signe sur ta main, et pour fronteaux entre tes yeux, que Yahweh nous a retirés d'Egypte par sa main puissante.
\TextTitle{Début du voyage, Yahweh dirige son peuple}
\VS{17}Or lorsque Pharaon laissa aller le peuple, Dieu ne les conduisit point par le chemin du pays des Philistins, bien qu'il fût le plus court~; car Dieu dit~: C'est afin qu'il n'arrive que le peuple se repente quand il verra la guerre, et qu'il ne retourne en Egypte.
\VS{18}Mais Dieu fit tourner le peuple par le chemin du désert, vers la Mer Rouge. Ainsi, les enfants d'Israël montèrent en armes hors du pays d'Egypte.
\VS{19}Et Moïse avait pris avec lui les ossements de Joseph, parce que Joseph avait expressément fait jurer les enfants d'Israël, en leur disant~: Dieu vous visitera très certainement, et vous transporterez donc avec vous mes ossements d'ici\FTNT{Ge. 50:25~; Jos. 24:32.}.
\VS{20}Et ils partirent de Succoth, et campèrent à Etham, qui est à l'extrémité du désert.
\VS{21}Et Yahweh allait devant eux, de jour dans une colonne de nuée pour les conduire par le chemin~; et de nuit dans une colonne de feu pour les éclairer, afin qu'ils marchent jour et nuit\FTNT{No. 9:13-23~; No. 10:43~; De. 1:33~; Né. 9:12-19~; 1 Co. 10:1.}.
\VS{22}Et il ne retira point la colonne de nuée le jour, ni la colonne de feu la nuit de devant le peuple.
\Chap{14}
\TextTitle{Pharaon et son armée à la poursuite d'Israël}
\VerseOne{}Et Yahweh parla à Moïse et dit~:
\VS{2}Parle aux enfants d'Israël et dis-leur: Qu'ils se détournent, et qu'ils campent devant Pi-Hahiroth, entre Migdol et la mer, vis-à-vis de Baal-Tsephon. Vous camperez vis-à-vis de ce lieu-là près de la mer\FTNT{No. 33:7.}.
\VS{3}Pharaon dira des enfants d'Israël~: Ils sont confus dans le pays, le désert les a enfermés.
\VS{4}Et j'endurcirai le cœur de Pharaon, et il vous poursuivra. Ainsi je serai glorifié en Pharaon et en toute son armée et les Egyptiens sauront que je suis Yahweh~; et ils firent ainsi.
\VS{5}Or on avait rapporté au roi d'Egypte que le peuple s'enfuyait, et le cœur de Pharaon et de ses serviteurs fut changé à l'égard du peuple, et ils dirent~: Qu'est-ce que nous avons fait en laissant aller Israël, de sorte qu'il ne nous servira plus~?
\VS{6}Alors il fit atteler son char et il prit son peuple avec lui.
\VS{7}Il prit donc six cents chars d'élite et tous les chars de l'Egypte~; et il y avait des capitaines sur tout cela.
\VS{8}Et Yahweh endurcit le cœur de Pharaon, roi d'Egypte, qui poursuivit les enfants d'Israël. Or les fils d'Israël étaient sortis à main levée\FTNT{Lé. 26:13~; No. 33:3.}.
\VS{9}Les Egyptiens donc les poursuivirent~; et tous les chevaux des chars de Pharaon, ses cavaliers et son armée les atteignirent comme ils étaient campés près de la mer, vers Pi-Hahiroth vis-à-vis de Baal-Tsephon.
\VS{10}Et Pharaon approchait. Les enfants d'Israël levèrent leurs yeux, et voici, les Egyptiens marchaient après eux. Et les fils d'Israël eurent une grande frayeur et crièrent à Yahweh.
\VS{11}Ils dirent aussi à Moïse~: Est-ce qu'il n'y avait pas des sépulcres en Egypte pour que tu nous aies emmenés pour mourir au désert~? Que nous as-tu fait en nous faisant sortir d'Egypte~?
\VS{12}N'est-ce pas ce que nous te disions en Egypte, en disant~: Retire-toi de nous et que nous servions les Egyptiens~? Car nous aimons mieux les servir que de mourir au désert.
\TextTitle{Délivrance miraculeuse par Yahweh}
\VS{13}Et Moïse dit au peuple~: Ne craignez point, arrêtez-vous et voyez la délivrance que Yahweh vous donnera aujourd'hui~; car les Egyptiens que vous voyez aujourd'hui, vous ne les verrez plus.
\VS{14}Yahweh combattra pour vous et vous demeurerez tranquilles.
\VS{15}Or Yahweh avait dit à Moïse~: Que cries-tu à moi~? Parle aux enfants d'Israël, qu'ils marchent.
\VS{16}Et toi, élève ta verge, étends ta main sur la mer, et fends-la~; et que les enfants d'Israël entrent au milieu de la mer à sec.
\VS{17} Et quant à moi, voici, je m'en vais endurcir le cœur des Egyptiens, afin qu'ils entrent après eux~; et je serai glorifié en Pharaon, et en toute son armée, en ses chars et en ses cavaliers.
\VS{18}Et les Egyptiens sauront que je suis Yahweh, quand j'aurai été glorifié en Pharaon, avec ses chars et ses cavaliers.
\VS{19}Et l'Ange de Dieu qui allait devant le camp d'Israël partit, et s'en alla derrière eux~; et la colonne de nuée partit de devant eux et se tint derrière eux.
\VS{20}Et elle vint entre le camp des Egyptiens et le camp d'Israël. Elle était aux uns une nuée et une obscurité~; et pour les autres, elle les éclairait la nuit. L'un des camps n'approcha point de l'autre durant toute la nuit.
\VS{21}Or Moïse avait étendu sa main sur la mer, et Yahweh fit reculer la mer toute la nuit par un vent d'orient qui souffla avec puissance~; il mit la mer à sec, et les eaux se fendirent\FTNT{Jos. 4:23~; Ps. 66:6~; Ps. 106:9~; Hé. 11:29.}.
\VS{22}Et les enfants d'Israël entrèrent au milieu de la mer à sec, et les eaux leur servaient de mur à droite et à gauche.
\VS{23}Et les Egyptiens les poursuivirent~; et ils entrèrent après eux au milieu de la mer, à savoir tous les chevaux de Pharaon, ses chars et ses cavaliers.
\VS{24}Mais il arriva que sur la veille du matin, Yahweh étant dans la colonne de feu et dans la nuée, regarda le camp des Egyptiens et le mit en déroute.
\VS{25}Il ôta les roues de leurs chars et alourdit leur marche. Alors les Egyptiens dirent~: Fuyons de devant les Israëlites, car Yahweh combat pour eux contre les Egyptiens.
\VS{26}Et Yahweh dit à Moïse~: Etends ta main sur la mer, et les eaux retourneront sur les Egyptiens, sur leurs chars et sur leurs cavaliers.
\VS{27}Moïse donc étendit sa main sur la mer, et la mer reprit son impétuosité vers le matin. Et les Egyptiens s'enfuyant rencontrèrent la mer qui s'était rejointe~; et ainsi Yahweh jeta les Egyptiens au milieu de la mer.
\VS{28}Car les eaux retournèrent et couvrirent les chars et les cavaliers de toute l'armée de Pharaon, qui étaient entrés après les Israélites dans la mer, et il n'en resta pas un seul.
\VS{29}Mais les enfants d'Israël marchèrent au milieu de la mer à sec, et les eaux leur servaient de mur à droite et à gauche.
\VS{30}Ainsi, Yahweh délivra, en ce jour-là, Israël de la main des Egyptiens~; et Israël vit sur le bord de la mer les Egyptiens morts.
\VS{31}Israël vit donc la grande puissance que Yahweh avait déployée contre les Egyptiens~; et le peuple craignit Yahweh, ils crurent en Yahweh, et en Moïse, son serviteur.
\Chap{15}
\TextTitle{Cantique de délivrance}
\VerseOne{}Alors Moïse et les enfants d'Israël chantèrent ce cantique à Yahweh, et dirent~: Je chanterai à Yahweh, car il est hautement élevé~; il a jeté dans la mer le cheval et celui qui le montait.
\VS{2}Yahweh est ma force et ma louange, et il a été mon Sauveur, mon Dieu. Je lui dresserai un tabernacle, c'est le Dieu de mon père, je l'exalterai.
\VS{3}Yahweh est un vaillant guerrier, son Nom est Yahweh.
\VS{4}Il a jeté dans la mer les chars de Pharaon et son armée~; l'élite de ses capitaines a été submergée dans la Mer Rouge.
\VS{5}Les gouffres les ont couverts, ils sont descendus au fond des eaux comme une pierre\FTNT{Né. 9:11.}.
\VS{6}Ta droite, ô Yahweh, s'est montrée magnifique en force~! Ta droite, ô Yahweh, a brisé l'ennemi\FTNT{Ps. 118:15-16~; Ps. 77:16.}~!
\VS{7}Tu as ruiné par la grandeur de ta majesté ceux qui s'élevaient contre toi~; tu as lâché ta colère et elle les a consumés comme du chaume.
\VS{8}Par le souffle de tes narines, les eaux ont été amoncelées~; les eaux courantes se sont arrêtés comme un monceau~; les gouffres ont été gelés au milieu de la mer.
\VS{9}L'ennemi disait~: Je poursuivrai, j'atteindrai, je partagerai le butin~; mon âme sera assouvie d'eux, je tirerai mon épée, ma main les détruira.
\VS{10}Tu as soufflé de ton vent, la mer les a couverts~; ils ont été enfoncés comme du plomb au plus profond des eaux.
\VS{11}Qui est comme toi parmi les dieux, ô Yahweh~! Qui est comme toi, magnifique en sainteté, digne d'être révéré et célébré, faisant des choses merveilleuses~?
\VS{12}Tu as étendu ta droite, la terre les a engloutis.
\VS{13}Tu as conduit par ta miséricorde ce peuple que tu as racheté~; tu l'as conduit par ta force à la demeure de ta sainteté.
\VS{14}Les peuples l'ont entendu, et ils en ont tremblé~; la douleur a saisi les habitants du pays des Philistins.
\VS{15}Alors les princes d'Edom seront troublés, et le tremblement saisira les puissants de Moab, tous les habitants de Canaan se fondront.
\VS{16}La frayeur et l'épouvante tomberont sur eux~; ils seront rendus muets comme une pierre par la grandeur de ton bras, jusqu'à ce que ton peuple soit passé, ô Yahweh~! Jusqu'à ce que ce peuple que tu as acquis soit passé\FTNT{De. 2:25~; De. 11:25~; Jos. 2:9.}.
\VS{17}Tu les introduiras et les planteras sur la montagne de ton héritage, au lieu que tu as préparé pour ta demeure, ô Yahweh~! Au lieu saint, ô Seigneur, que tes mains ont établi~!
\VS{18}Yahweh régnera à jamais et à perpétuité.
\VS{19}Car les chevaux de Pharaon, ses chars et ses cavaliers sont entrés dans la mer, et Yahweh a fait retourner sur eux les eaux de la mer~; mais les enfants d'Israël ont marché à sec au milieu de la mer.
\VS{20}Et Marie, la prophétesse, sœur d'Aaron, prit un tambour dans sa main, et toutes les femmes sortirent après elle, avec des tambours et des flûtes.
\VS{21}Et Marie leur répondait~: Chantez à Yahweh, car il est hautement élevé~; il a jeté dans la mer le cheval et celui qui le montait.
\TextTitle{Yahweh pourvoit pour son peuple}
\VS{22}Après cela, Moïse fit partir les Israélites de la Mer Rouge, et ils partirent vers le désert de Schur~; et ayant marché trois jours dans le désert, ils ne trouvèrent point d'eau.
\VS{23}De là, ils vinrent à Mara, mais ils ne purent boire les eaux de Mara, parce qu'elles étaient amères~; c'est pourquoi ce lieu fut appelé Mara.
\VS{24}Et le peuple murmura contre Moïse en disant~: Que boirons-nous~?
\VS{25}Et Moïse cria à Yahweh, et Yahweh lui montra\FTNT{«~Montra~» de l'hébreu «~yarah~» qui veut également dire «~enseigner~», «~signaler~», «~lancer~», «~instruire~», «~informer~», «~montrer~», «~jeter~» etc.} un certain bois qu'il jeta dans les eaux~; et les eaux devinrent douces. Il lui proposa là une ordonnance et une loi, et il l'éprouva là,
\VS{26}et lui dit~: Si tu écoutes attentivement la voix de Yahweh, ton Dieu, si tu fais ce qui est droit devant lui, si tu prêtes l'oreille à ses commandements, si tu gardes toutes ses ordonnances, je ne ferai venir sur toi aucune des infirmités que j'ai fait venir sur l'Egypte, car je suis Yahweh qui te guérit\FTNT{De. 7:12-15.}.
\VS{27}Puis ils vinrent à Elim, où il y avait douze fontaines d'eau, et soixante-dix palmiers. Et ils campèrent là, près des eaux.
\Chap{16}
\TextTitle{Yahweh envoie la manne}
\VerseOne{}Et toute l'assemblée des enfants d'Israël étant partie d'Elim, vint au désert de Sin, qui est entre Elim et Sinaï, le quinzième jour du second mois après qu'ils furent sortis du pays d'Egypte.
\VS{2}Et toute l'assemblée des enfants d'Israël murmura dans ce désert contre Moïse et Aaron.
\VS{3}Et les enfants d'Israël leur dirent~: Ah~! Pourquoi ne sommes-nous point morts par la main de Yahweh dans le pays d'Egypte, quand nous étions assis près des pots de viande, et que nous mangions du pain à satiété~? Car vous nous avez amenés dans ce désert pour faire mourir de faim toute cette assemblée\FTNT{1 Co. 10:10~; No. 11:4.}.
\VS{4}Et Yahweh dit à Moïse~: Voici, je vais vous faire pleuvoir des cieux du pain, et le peuple sortira et en recueillera chaque jour la provision d'un jour, afin que je l'éprouve, pour voir s'il observera ma loi ou non.
\VS{5}Mais qu'ils apprêtent au sixième jour ce qu'ils auront apporté, et qu'il y ait le double de ce qu'ils recueilleront chaque jour.
\VS{6}Moïse donc et Aaron dirent à tous les enfants d'Israël~: Ce soir vous saurez que Yahweh vous a tirés du pays d'Egypte.
\VS{7}Et au matin vous verrez la gloire de Yahweh, parce qu'il a entendu vos murmures, qui sont contre Yahweh~; car que sommes-nous pour que vous murmuriez contre nous~?
\VS{8}Moïse dit donc~: Ce sera quand Yahweh vous aura donné ce soir de la chair à manger, et qu'au matin, il vous aura rassasiés de pain, parce qu'il a entendu vos murmures, par lesquels vous avez murmuré contre lui. Car que sommes-nous~? Vos murmures ne sont pas contre nous, mais contre Yahweh.
\VS{9}Et Moïse dit à Aaron~: Dis à toute l'assemblée des enfants d'Israël~: Approchez-vous de la présence de Yahweh, car il a entendu vos murmures.
\VS{10}Or il arriva qu'aussitôt qu'Aaron eut parlé à toute l'assemblée des enfants d'Israël, ils regardèrent vers le désert, et voici, la gloire de Yahweh se montra dans la nuée.
\VS{11} Et Yahweh parla à Moïse, en disant~:
\VS{12}J'ai entendu les murmures des enfants d'Israël. Parle-leur et dis-leur~: Entre les deux soirs, vous mangerez de la chair, et au matin vous serez rassasiés de pain~; et vous saurez que je suis Yahweh, votre Dieu.
\VS{13}Sur le soir donc, il monta des cailles qui couvrirent le camp, et au matin il y eut une couche de rosée autour du camp.
\VS{14}Et cette couche de rosée étant évanouie, voici, sur la surface du désert, quelque chose de menu et de rond, comme du grain sur la terre.
\VS{15}Ce que les enfants d'Israël ayant vu, ils se dirent l'un à l'autre~: Qu'est-ce~? Car ils ne savaient ce que c'était. Et Moïse leur dit~: C'est le pain que Yahweh vous donne à manger\FTNT{Ps. 105:40.}.
\TextTitle{Récolte de la manne}
\VS{16}Or ce que Yahweh a ordonné, c'est que chacun en recueille autant qu'il lui en faut pour sa nourriture, un homer par tête, selon le nombre de vos personnes~; chacun en prendra pour ceux qui sont dans sa tente.
\VS{17}Les enfants d'Israël firent donc ainsi~; et les uns en recueillirent plus, les autres moins.
\VS{18}Et ils le mesuraient par homer~; et celui qui en avait recueilli beaucoup n'en avait pas plus qu'il ne lui en fallait~; ni celui qui en avait recueilli peu, n'en avait pas moins~; mais chacun en recueillait selon ce qu'il en pouvait manger.
\VS{19}Et Moïse leur avait dit~: Que personne n'en laisse rien de reste jusqu'au matin.
\VS{20}Mais il y en eut qui n'obéirent point à Moïse, car quelques-uns en réservèrent jusqu'au matin~; et il s'y engendra des vers, et cela puait. Et Moïse se mit en grande colère contre eux.
\VS{21}Ainsi, chacun en recueillait tous les matins autant qu'il lui en fallait pour se nourrir, et lorsque la chaleur du soleil était venue, elle se fondait.
\VS{22}Mais le sixième jour, ils recueillirent du pain en double, deux homers pour chacun~; et les principaux de l'assemblée vinrent pour le rapporter à Moïse.
\TextTitle{Le sabbat\FTNTT{Né. 9:13-14~; Mt. 12:1.}}
\VS{23} Et il leur dit~: C'est ce que Yahweh a dit~: Demain est le repos, le sabbat consacré à Yahweh~; faites cuire ce que vous avez à cuire, et faites bouillir ce que vous avez à bouillir, et serrez tout ce qui sera de surplus, pour le garder jusqu'au matin.
\VS{24}Ils le serrèrent donc jusqu'au matin, comme Moïse l'avait ordonné, et il ne pua point, et il n'y eut point de vers dedans.
\VS{25}Alors Moïse dit~: Mangez-le aujourd'hui, car c'est aujourd'hui le repos de Yahweh~; aujourd'hui vous n'en trouverez point dans les champs.
\VS{26}Durant six jours vous le recueillerez, mais le septième est le sabbat, il n'y en aura point ce jour-là.
\VS{27}Et au septième jour, quelques-uns du peuple sortirent pour en recueillir, mais ils n'en trouvèrent point.
\VS{28}Et Yahweh dit à Moïse~: Jusqu'à quand refuserez-vous de garder mes commandements et mes lois~?
\VS{29}Considérez que Yahweh vous a ordonné le sabbat, c'est pourquoi il vous donne au sixième jour du pain pour deux jours~; que chacun demeure au lieu où il sera, et qu'aucun ne sorte du lieu où il est le septième jour.
\VS{30}Le peuple donc se reposa le septième jour.
\VS{31}Et la maison d'Israël nomma ce pain manne\FTNT{Le mot «~manne~» vient de l'hébreu «~man~» et veut dire «~Qu'est-ce que cela~?~». La manne est une image de Jésus, le Pain de vie descendu du ciel (Jn. 6:32-52). La consommation quotidienne du Pain de vie, qui est aussi la Parole de Dieu, apporte la vie éternelle.}. Elle était comme de la semence de coriandre blanche, et ayant le goût d'un gâteau au miel.
\VS{32}Et Moïse dit~: Voici ce que Yahweh a ordonné~: Qu'on en remplisse un homer pour le garder pour vos générations, afin qu'on voie le pain que je vous ai fait manger au désert, après vous avoir retirés du pays d'Egypte.
\VS{33}Moïse dit à Aaron~: Prends un vase, et mets-y un plein d'homer de manne, et pose-le devant Yahweh, afin qu'il soit conservé pour vos générations.
\VS{34}Et Aaron le posa devant le témoignage pour y être gardé, selon que le Seigneur l'avait ordonné à Moïse.
\VS{35}Et les enfants d'Israël mangèrent la manne durant quarante ans, jusqu'à leur arrivée dans un pays habité~; ils mangèrent, dis-je, la manne, jusqu'à leur arrivée aux frontières du pays de Canaan.
\VS{36}Or un homer est la dixième partie d'un épha.
\Chap{17}
\TextTitle{Miracle de l'eau qui sort du rocher}
\VerseOne{}Et toute l'assemblée des enfants d'Israël partit du désert de Sin, selon l'ordre de marche que Yahweh leur avait ordonné, et ils campèrent à Rephidim, où il n'y avait point d'eau à boire pour le peuple.
\VS{2}Et le peuple se souleva contre Moïse et ils lui dirent~: Donnez-nous de l'eau à boire. Et Moïse leur dit~: Pourquoi vous soulevez-vous contre moi~? Pourquoi tentez-vous Yahweh\FTNT{No. 20:2-5.}~?
\VS{3}Le peuple donc eut soif en ce lieu-là, par faute d'eau~; et ainsi le peuple murmura contre Moïse, en disant~: Pourquoi nous as-tu fait monter hors d'Egypte, pour nous faire mourir de soif, nous, nos enfants, et nos troupeaux~?
\VS{4}Et Moïse cria à Yahweh, en disant~: Que ferai-je à ce peuple~? Encore un peu, et ils me lapideront.
\VS{5}Et Yahweh répondit à Moïse: Passe devant le peuple, et prends avec toi des anciens d'Israël, prends aussi dans ta main la verge avec laquelle tu as frappé le fleuve, et viens~!
\VS{6}Voici, je vais me tenir là devant toi sur le rocher d'Horeb~; et tu frapperas le rocher, et il en sortira des eaux, et le peuple en boira. Moïse donc fit ainsi aux yeux des anciens d'Israël\FTNT{De. 9:8~; Ps. 78:15~; 1 Co. 10:4.}.
\VS{7}Et il nomma le lieu Massa et Meriba, à cause de la querelle des enfants d'Israël, et parce qu'ils avaient tenté Yahweh, en disant~: Yahweh est-il au milieu de nous ou non~?
\TextTitle{Bataille et victoire contre Amalek}
\VS{8}Alors Amalek vint et livra bataille contre Israël à Rephidim\FTNT{De. 25:17-18.}.
\VS{9}Et Moïse dit à Josué~: Choisis-nous des hommes, et sors pour combattre contre Amalek~; et je me tiendrai demain sur le sommet de la colline, et la verge de Dieu sera dans ma main.
\VS{10}Et Josué fit comme Moïse lui avait ordonné en combattant contre Amalek. Mais Moïse, Aaron et Hur montèrent au sommet de la colline.
\VS{11}Et il arrivait que lorsque Moïse élevait sa main, Israël était alors le plus fort, mais quand il reposait sa main, alors Amalek était le plus fort.
\VS{12}Et les mains de Moïse étant devenues pesantes, ils prirent une pierre et la mirent sous lui, et il s'assit dessus~; Aaron et Hur soutenaient ses mains, l'un d'un côté, et l'autre de l'autre côté~; et ainsi ses mains furent fermes jusqu'au soleil couchant.
\VS{13}Josué donc défit Amalek et son peuple au tranchant de l'épée.
\VS{14}Et Yahweh dit à Moïse~: Ecris ceci pour mémoire dans un livre, et fais entendre à Josué que j'effacerai entièrement la mémoire d'Amalek de dessous les cieux.
\VS{15}Et Moïse bâtit un autel et le nomma Yahweh, ma bannière.
\VS{16}Il dit aussi~: Parce que la main a été levée contre le trône de Yahweh, Yahweh aura toujours la guerre contre Amalek.
\Chap{18}
\TextTitle{Jéthro conseille Moïse}
\VerseOne{}Or Jéthro, prêtre de Madian, beau-père de Moïse, apprit toutes les choses que Yahweh avait faites à Moïse, et à Israël, son peuple, à savoir comment Yahweh avait retiré Israël de l'Egypte.
\VS{2}Jéthro, beau-père de Moïse, prit Séphora la femme de Moïse, après que Moïse l'eut renvoyée,
\VS{3}et les deux fils de cette femme, dont l'un s'appelait Guerschom, car il avait dit~: J'habite un pays étranger~;
\VS{4}et l'autre Eliézer, car il avait dit~: Le Dieu de mon père m'a secouru et m'a délivré de l'épée de Pharaon.
\VS{5}Jéthro donc, beau-père de Moïse, vint vers Moïse avec ses fils et sa femme au désert, où il était campé, à la montagne de Dieu.
\VS{6}Il fit dire à Moïse~: Jéthro, ton beau-père, vient vers toi, et ta femme et ses deux fils avec elle.
\VS{7}Et Moïse sortit au-devant de son beau-père, et s'étant prosterné, il l'embrassa~; et ils s'enquirent l'un de l'autre, de leur santé, puis ils entrèrent dans la tente.
\VS{8}Et Moïse raconta à son beau-père toutes les choses que Yahweh avait faites à Pharaon et aux Egyptiens en faveur d'Israël, et toute la fatigue qu'ils avaient soufferte en chemin, et comment Yahweh les avait délivrés.
\VS{9}Et Jéthro se réjouit de tout le bien que Yahweh avait fait à Israël, parce qu'il les avait délivrés de la main des Egyptiens.
\VS{10}Puis Jéthro dit~: Béni soit Yahweh qui vous a délivrés de la main des Egyptiens et de la main de Pharaon, qui a, dis-je, délivré le peuple de la main des Egyptiens~!
\VS{11}Je sais maintenant que le Seigneur est plus grand que tous les dieux, car la chose même en laquelle ils se sont enorgueillis, il a eu le dessus sur eux.
\VS{12}Jéthro, beau-père de Moïse, apporta aussi un holocauste et des sacrifices pour les offrir à Dieu. Et Aaron et tous les anciens d'Israël vinrent pour manger du pain avec le beau-père de Moïse dans la présence de Dieu.
\VS{13}Et il arriva, le lendemain, comme Moïse siégeait pour juger le peuple, et que le peuple se tenait devant Moïse depuis le matin jusqu'au soir,
\VS{14}que le beau-père de Moïse vit tout ce qu'il faisait au peuple, et il lui dit~: Qu'est-ce que tu fais à l'égard de ce peuple~? Pourquoi es-tu assis seul, et tout le peuple se tient devant toi depuis le matin jusqu'au soir~?
\VS{15} Et Moïse répondit à son beau-père~: C'est que le peuple vient à moi pour s'enquérir de Dieu.
\VS{16}Quand ils ont quelque affaire, ils viennent à moi, et je juge entre l'un et l'autre, et je leur fais entendre les ordonnances de Dieu et ses lois.
\VS{17}Mais le beau-père de Moïse lui dit~: Ce que tu fais n'est pas bien.
\VS{18}Certainement, tu succomberas, toi et ce peuple qui est avec toi~; car cela est trop pesant pour toi, tu ne saurais faire cela toi seul.
\VS{19}Ecoute donc mon conseil~; je te conseillerai et Dieu sera avec toi: Sois pour ce peuple auprès de Dieu, et rapporte les causes à Dieu.
\VS{20}Et instruis-les des ordonnances et des lois~; et fais-leur connaître la voie par laquelle ils auront à marcher et ce qu'ils auront à faire.
\VS{21}Et choisis-toi d'entre tout le peuple des hommes vertueux, craignant Dieu~; des hommes véritables, haïssant le gain déshonnête, et établis-les chefs de milliers, chefs de centaines, chefs de cinquantaines et chef des dizaines.
\VS{22}Et qu'ils jugent le peuple en tout temps, mais qu'ils te rapportent toutes les grandes affaires, et qu'ils jugent toutes les petites causes~; ainsi ils te soulageront et porteront une partie de la charge avec toi.
\VS{23}Si tu fais cela, et que Dieu te l'ordonne, tu pourras subsister, et tout le peuple parviendra en paix à destination.
\VS{24}Moïse donc obéit à la parole de son beau-père, et fit tout ce qu'il lui avait dit.
\VS{25}Ainsi, Moïse choisit de tout Israël des hommes vertueux, et les établit chefs sur le peuple, chefs de milliers, chefs de centaines, chefs de cinquantaines, et chefs de dizaines,
\VS{26}lesquels devaient juger le peuple en tout temps, mais ils devaient rapporter à Moïse les choses difficiles, et juger de toutes les petites affaires.
\VS{27}Puis Moïse laissa partir son beau-père, qui s'en alla dans son pays.
\Chap{19}
\TextTitle{DEBUT DE LA PÉRIODE DE LA LOI MOSAÏQUE OU DE LA PREMIÈRE ALLIANCE}
\VerseOne{}Au premier jour du troisième mois, après que les enfants d'Israël furent sortis du pays d'Egypte, en ce même jour-là, ils vinrent au désert de Sinaï.
\VS{2}Etant donc partis de Rephidim, ils vinrent au désert de Sinaï et campèrent au désert. Et Israël campa vis-à-vis de la montagne.
\VS{3}Et Moïse monta vers Dieu, car Yahweh l'avait appelé de la montagne pour lui dire~: Tu parleras ainsi à la maison de Jacob, et tu annonceras ceci aux enfants d'Israël~:
\VS{4}Vous avez vu ce que j'ai fait aux Egyptiens~; comment je vous ai portés comme sur des ailes d'aigle et vous ai amenés à moi.
\VS{5}Maintenant donc, si vous obéissez exactement à ma voix, et si vous gardez mon alliance, vous serez aussi d'entre tous les peuples mon plus précieux joyau, car toute la terre m'appartient\FTNT{C'est ici que débute la période de la Loi ou Première Alliance. Le fait d'avoir réuni les textes de Genèse à Malachie sous l'appellation «~Ancien Testament~» a induit beaucoup de personnes en erreur quant à leur compréhension du plan de Dieu pour nos vies. Tout d'abord, l'emploi du mot «~testament~» est inapproprié puisqu'on ne peut parler de testament sans qu'il y ait au préalable la mort du testateur (Hé. 9:16-17). Certes, des animaux étaient tués sous la Loi pour couvrir les péchés. Toutefois, ces sacrifices étaient imparfaits et par conséquent prévus pour ne durer qu'un temps, en attendant le sacrifice parfait de Jésus-Christ (Hé. 10:1-14). De plus, il est évident que les animaux sacrifiés ne nous ont rien légué.\\ Ensuite, il est à noter que tous les textes classés dans ce que l'on appelle à tort «~Ancien Testament~» ne se rapportent pas exclusivement et nécessairement à la Loi. Ainsi, des prophètes, en commençant par Moïse en personne, ayant vécu sous la Loi, ont prophétisé et écrit sur d'autres sujets que la Loi, notamment sur la grâce et la fin des temps. N'oublions pas non plus que Jésus-Christ est né et a vécu sous la Loi (Ga. 4:4). En tant que juif, il l'a scrupuleusement respectée de telle sorte qu'elle fut totalement accomplie en Lui (Mt. 5:17-18~; Jn. 19:30). En conséquence, la fin de la Loi mosaïque eut lieu après la mort du Seigneur, précisément au moment où le Seigneur a dit «~Tout est accompli~», et lorsque le voile du temple s'est déchiré de haut en bas (Mt. 27:50-51~; Jn. 19:30). La Nouvelle Alliance, ou le Testament de Jésus débuta avec l'effusion de l'Esprit (Ac. 2). De la mort du Seigneur à la Pentecôte, une période de transition de cinquante jours s'est écoulée. Jésus-Christ s'est présenté pendant ce temps dans le sanctuaire céleste pour présenter son sang dans le Saint des saints. Une fois son sacrifice examiné et accepté, le Saint-Esprit qui avait été retiré de l'homme (Ge. 6:3) put de nouveau revenir habiter le cœurs des croyants.\\Mais qu'est-ce que la loi exactement~? Beaucoup de chrétiens sont dans la confusion à ce sujet. En réalité, il n'y avait pas qu'une loi mais trois sortes de lois~: Les lois morales et les lois cérémonielles qui préexistaient depuis l'éternité~; et les lois civiles qui ont débuté avec Moïse car elles ne concernaient que son peuple.\\- Les lois civiles régissaient le fonctionnement de la vie en communauté des Hébreux. Elles étaient exclusivement réservées au peuple d'Israël dans le camp puis dans le pays de Canaan (Ex. 21:1-2~; De. 23).\\- Les lois morales font référence à la nature de Dieu~: Son amour, sa justice, sa sainteté, etc. Les dix commandements, à l'exception du sabbat tel que prescrit par Moïse (Ex. 16:28-29~; Lé. 15:32), font partie des lois morales (Ex. 20:1-17). Les dix paroles ne constituent qu'une base, un résumé. Ainsi, d'autres règles morales sont énoncées tout au long des Ecritures notamment sur la sexualité (Lé. 18:1-22), l'interdiction des sacrifices humains et de l'occultisme (De. 18:10-13), le respect d'autrui et l'entraide (Lé. 19:10-18~; Lé. 19:29-36). Comme il est impossible de consigner dans un livre tous les péchés moraux, le Seigneur a inscrit les lois morales dans le cœur de l'homme afin qu'il sache instinctivement faire la différence entre le bien et mal (Ro. 2:14-15). Jésus les a résumées en ces quelques mots~: «~Tu aimeras le Seigneur ton Dieu de tout ton cœur, et de toute ton âme, et de toute ta pensée. Celui-ci est le premier et le plus grand commandement. Et le second semblable à celui-là, est~: Tu aimeras ton prochain comme toi-même.~» (Mt. 22:37-39). Ces lois sont encore en vigueur aujourd'hui et le resteront pour toujours.\\- Les lois cérémonielles étaient relatives au culte et au sanctuaire terrestre, c'est-à-dire le tabernacle puis le temple de Jérusalem (Hé. 9:1-10). Elles regroupent toutes les ordonnances concernant les sacrifices, les ablutions, les sabbats, les fêtes de Yahweh, la dîme des Lévites et des prêtres (voir commentaire en No. 18:21 et Mal. 3:10). Les livres du Lévitique et des Nombres exposent en détail toutes les ordonnances reçues par Moïse d'après le modèle céleste que Yahweh lui avait montré sur le Mont Sinaï (Ex. 26:30). Les lois cérémonielles préexistaient donc depuis l'éternité.\\Les lois cérémonielles représentent la Première Alliance qui avait pour fondement la loi morale. Or cette alliance a vieilli puis disparu car elle n'était que l'ombre des choses à venir (Hé. 8:13). En effet, elle était basée sur quatre points principaux~: Le temple, le culte centralisé, le sacrifice et les prêtres. En Christ, nous n'avons plus besoin d'un temple physique puisque nous sommes devenus les temples vivants de Yahweh (1 Co. 6:19~; Ep. 2:22). Nous pouvons désormais adorer le Seigneur en Esprit et en vérité, à tout moment et en tout lieu (Jn. 4:23). Le sacerdoce lévitique ayant été aboli, chaque enfant de Dieu est devenu un prêtre (Ap. 5:10) qui offre en sacrifice sa propre vie consacrée au Seigneur (Ro. 12:1).\\Les lois cérémonielles ont donc trouvé leur parfait accomplissement en Jésus-Christ~: Tous les sacrifices sanglants le préfiguraient, toutes les solennités ont été réalisées en Lui (voir note en Lé. 23). Christ est donc la fin de la Loi, non pas morale, mais cérémonielle (Ro. 10:4).\\Un lien étroit existe entre les lois morales et les lois cérémonielles. La loi morale est comme un diagnostic qui révèle une pathologie incurable comme le sida~: Le péché (Ro. 5:13-20~; Ro. 7:7-14). En la découvrant, l'homme se sent condamné car il réalise qu'il ne peut pas répondre aux exigences de la justice divine. La loi cérémonielle (le sang des animaux - Hé. 9:1-13~; Hé. 10:11) a donné aux hommes une sorte de trithérapie pour les soulager provisoirement de leurs péchés mais sans pour autant les ôter (guérir, délivrer, nettoyer, laver) définitivement. Seul le sang de la Nouvelle Alliance, c'est-à-dire le sang de Jésus-Christ, a pu nous délivrer une fois pour toutes (Jn. 1:29~; Hé. 9:11-26~; Hé. 10:1-23~; Ap. 1:6).}.
\VS{6}Et vous me serez un royaume de prêtres, et une nation sainte~; ce sont là les discours que tu tiendras aux enfants d'Israël.
\VS{7}Puis Moïse vint et appela les anciens du peuple, et proposa devant eux toutes ces choses-là que Yahweh lui avait ordonné.
\VS{8}Et tout le peuple répondit d'un commun accord, en disant~: Nous ferons tout ce que Yahweh a dit. Et Moïse rapporta à Yahweh toutes les paroles du peuple.
\TextTitle{Moïse doit sanctifier le peuple pour qu'il rencontre Yahweh}
\VS{9}Et Yahweh dit à Moïse~: Voici, je viendrai à toi dans une nuée épaisse, afin que le peuple entende quand je parlerai avec toi, et qu'il te croie aussi toujours~; car Moïse avait rapporté à Yahweh les paroles du peuple.
\VS{10}Yahweh dit aussi à Moïse~: Va-t'en vers le peuple, et sanctifie-les aujourd'hui et demain, et qu'ils lavent leurs vêtements.
\VS{11}Et qu'ils soient tous prêts pour le troisième jour, car au troisième jour, Yahweh descendra sur la montagne de Sinaï, à la vue de tout le peuple.
\VS{12}Or tu mettras des bornes pour le peuple tout autour, et tu diras~: Gardez-vous de monter sur la montagne et de toucher aucune de ses extrémités. Quiconque touchera la montagne sera puni de mort.
\VS{13}Aucune main ne la touchera, et certainement il sera lapidé, ou percé de flèches~; soit bête, soit homme, il ne vivra point. Quand la trompette sonnera longuement, ils monteront vers la montagne.
\VS{14}Et Moïse descendit de la montagne vers le peuple, et sanctifia le peuple, et ils lavèrent leurs vêtements.
\VS{15}Et il dit au peuple~: Soyez tous prêts pour le troisième jour, et ne vous approchez point de vos femmes.
\VS{16}Et le troisième jour au matin, il y eut des tonnerres, et des éclairs, et une grosse nuée sur la montagne, avec un très fort son de shofar, et tout le peuple dans le camp fut effrayé.
\VS{17}Alors Moïse fit sortir le peuple du camp pour aller au-devant de Dieu~; et ils s'arrêtèrent au pied de la montagne.
\VS{18}Or le mont Sinaï était tout couvert de fumée, parce que Yahweh y était descendu en feu~; et sa fumée montait comme la fumée d'une fournaise, et toute la montagne tremblait fort.
\VS{19}Et comme le son du shofar se renforçait de plus en plus, Moïse parla, et Dieu lui répondit par une voix.
\VS{20}Yahweh donc étant descendu sur la montagne de Sinaï, au sommet de la montagne, Yahweh appela Moïse au sommet de la montagne~; et Moïse y monta.
\VS{21}Et Yahweh dit à Moïse~: Descends. Somme le peuple qu'il ne rompe point les barrières pour monter vers Yahweh afin de regarder~; de peur qu'un grand nombre d'entre eux ne périsse.
\VS{22}Et même, que les prêtres qui s'approchent de Yahweh se sanctifient aussi, de peur qu'il n'arrive que Yahweh se jette sur eux.
\VS{23} Et Moïse dit à Yahweh~: Le peuple ne pourra pas monter sur la montagne de Sinaï, parce que tu nous as sommés en me disant~: Mets des bornes sur la montagne, et sanctifie-la.
\VS{24}Et Yahweh lui dit~: Va, descends~; puis tu monteras, toi, et Aaron avec toi~; mais que les prêtres et le peuple ne rompent point les bornes pour monter vers Yahweh, de peur qu'il n'arrive qu'il se jette sur eux.
\VS{25}Moïse descendit donc vers le peuple, et lui dit ces choses.
\Chap{20}
\TextTitle{Les dix paroles}
\VerseOne{}Alors Dieu prononça toutes ces paroles, disant~:
\VS{2}Je suis Yahweh, ton Dieu, qui t'ai retiré du pays d'Egypte, de la maison de servitude.
\VS{3}Tu n'auras point d'autres dieux devant ma face.
\VS{4}Tu ne te feras point d'image taillée, ni aucune ressemblance des choses qui sont là-haut aux cieux, ni ici-bas sur la terre, ni dans les eaux sous la terre\FTNT{Lé. 26:1.}.
\VS{5}Tu ne te prosterneras point devant elles, et ne les serviras point~; car je suis Yahweh, ton Dieu~; le Dieu qui est jaloux, punissant l'iniquité des pères sur les fils, jusqu'à la troisième et à la quatrième génération de ceux qui me haïssent~;
\VS{6}et faisant miséricorde en mille générations à ceux qui m'aiment et qui gardent mes commandements.
\VS{7}Tu ne prendras point le Nom de Yahweh, ton Dieu, en vain~; car Yahweh ne tiendra point pour innocent celui qui aura pris son Nom en vain\FTNT{Lé. 19:12~; Mt. 5:33.}.
\VS{8}Souviens-toi du jour du repos pour le sanctifier.
\VS{9}Tu travailleras six jours, et tu feras toute ton œuvre.
\VS{10}Mais le septième jour est le repos de Yahweh ton Dieu. Tu ne feras aucune œuvre en ce jour-là, ni toi, ni ton fils, ni ta fille, ni ton serviteur, ni ta servante, ni ton bétail, ni ton étranger qui est dans tes portes.
\VS{11}Car Yahweh a fait en six jours les cieux, la terre, la mer, et tout ce qui est en eux, et s'est reposé le septième jour~; c'est pourquoi Yahweh a béni le jour du repos et l'a sanctifié\FTNT{Ge. 2:3~; Ex. 31:14~; Ez. 20:12.}.
\VS{12}Honore ton père et ta mère, afin que tes jours soient prolongés sur la terre que Yahweh, ton Dieu, te donne\FTNT{Lé. 19:3~; De. 5:16~; Mt. 15:4~; Ep. 6:2.}.
\VS{13}Tu ne commettras pas de meurtre\FTNT{Mt. 5:21.}.
\VS{14}Tu ne commettras pas d'adultère\FTNT{Lé. 20:10~; De. 5:18~; Pr. 6:32~; Mt. 5:32~; Ro. 7:3.}.
\VS{15}Tu ne déroberas pas.
\VS{16}Tu ne diras pas de faux témoignage contre ton prochain.
\VS{17}Tu ne convoiteras pas la maison de ton prochain~; tu ne convoiteras pas la femme de ton prochain, ni son serviteur, ni sa servante, ni son bœuf, ni son âne, ni aucune chose qui soit à ton prochain.
\TextTitle{Le peuple tout tremblant devant Yahweh}
\VS{18}Or tout le peuple apercevait les tonnerres, les éclairs, le son du shofar, et la montagne fumante. Et le peuple voyant cela tremblait et se tenait loin.
\VS{19}Et ils dirent à Moïse~: Parle, toi, avec nous, et nous écouterons~; mais que Dieu ne parle point avec nous, de peur que nous ne mourrions\FTNT{De. 5:23-24~; Hé. 12:18-19.}.
\VS{20}Et Moïse dit au peuple~: Ne craignez point car Dieu est venu pour vous éprouver, et afin que sa crainte soit devant vous, et que vous ne péchiez point.
\VS{21}Le peuple donc se tint loin, mais Moïse s'approcha de l'obscurité dans laquelle Dieu était.
\VS{22}Et Yahweh dit à Moïse~: Tu diras ainsi aux enfants d'Israël~: Vous avez vu que je vous ai parlé des cieux.
\VS{23}Vous ne vous ferez point avec moi de dieux d'argent ni de dieux d'or.
\VS{24}Tu me feras un autel de terre, sur lequel tu sacrifieras tes holocaustes, et tes offrandes de paix\FTNT{Voir commentaire en Lé. 3:1.}, ton menu et ton gros bétail. En quelque lieu que ce soit où je mettrai la mémoire de mon Nom, je viendrai là à toi, et je te bénirai.
\VS{25}Si tu me fais un autel de pierres, ne les taille point~; car si tu fais passer le fer dessus, tu le souillerais.
\VS{26}Et tu ne monteras point à mon autel par des marches, de peur que ta nudité ne soit découverte en y montant.
\Chap{21}
\TextTitle{Lois sur les maîtres et leurs esclaves}
\VerseOne{}Ce sont ici les lois que tu leur proposeras.
\VS{2}Si tu achètes un esclave Hébreu, il te servira six ans, et au septième il sortira pour être libre, sans rien payer\FTNT{Lé. 25:39-43~; De. 15:12~; Jé. 34:14.}.
\VS{3}S'il est venu avec son corps seulement, il sortira avec son corps~; s'il avait une femme, sa femme sortira aussi avec lui.
\VS{4}Si son maître lui a donné une femme qui lui ait enfanté des fils ou des filles, sa femme et les enfants qu'il en aura seront à son maître, mais il sortira avec son corps.
\VS{5}Si l'esclave dit positivement: J'aime mon maître, ma femme, et mes fils, je ne sortirai point pour être libre.
\VS{6}Alors son maître le fera venir devant les juges, et le fera approcher de la porte ou du poteau, et son maître lui percera l'oreille avec un poinçon~; et il le servira pour toujours.
\VS{7}Si quelqu'un vend sa fille pour être esclave, elle ne sortira point comme les esclaves sortent.
\VS{8}Si elle déplaît à son maître, qui ne l'aura point fiancée, il la fera acheter~; mais il n'aura pas le pouvoir de la vendre à un peuple étranger, après lui avoir été infidèle.
\VS{9}Mais s'il l'a fiancée à son fils, il fera pour elle selon le droit des filles.
\VS{10}S'il en prend une pour lui, il ne retranchera rien de sa nourriture, de ses vêtements et du droit conjugal.
\VS{11}S'il ne fait pas pour elle ces trois choses-là, elle sortira sans payer aucun argent.
\TextTitle{Lois sur les dommages corporels}
\VS{12}Si quelqu'un frappe un homme et qu'il en meure,on le fera mourir de mort \FTNT{Lé. 24:17~; No. 35:11-16~; De. 19:2-11~; Jos. 20:2.}.
\VS{13}S'il ne lui a point dressé d'embûches, mais que Dieu l'ait fait tomber entre ses mains, je t'établirai un lieu où il s'enfuira.
\VS{14}Mais si quelqu'un s'élève de propos délibéré contre son prochain, pour le tuer par ruse, tu le tireras de mon autel, afin qu'il meure.
\VS{15}Celui qui aura frappé son père ou sa mère sera puni de mort\FTNT{Lé. 20:9~; De. 27:16~; Mt. 15:4.}.
\VS{16}Si quelqu'un dérobe un homme et le vend, ou s'il est trouvé entre ses mains, on le fera mourir de mort.
\VS{17}Celui qui aura maudit son père ou sa mère sera puni de mort.
\VS{18}Si quelques uns ont une querelle, et que l'un ait frappé l'autre d'une pierre ou du poing, sans causer sa mort, mais qu'il soit obligé de se mettre au lit,
\VS{19}s'il se lève et marche dehors en s'appuyant sur son bâton, celui qui l'aura frappé sera absous~; toutefois, il le dédommagera de ce qu'il a chômé et le fera guérir entièrement.
\VS{20}Si quelqu'un a frappé du bâton son serviteur ou sa servante, et qu'il soit mort sous sa main, on ne manquera point de le venger.
\VS{21}Mais s'il survit un jour ou deux, il ne sera point vengé, car c'est son argent.
\VS{22}Si des hommes se querellent, et que l'un d'eux frappe une femme enceinte, et qu'elle en accouche, s'il n'y a pas cas de mort, il sera condamné à l'amende que le mari de la femme lui imposera, et il la donnera selon que les juges en ordonneront.
\VS{23}Mais s'il y a cas de mort, tu donneras vie pour vie,
\VS{24}œil pour œil, dent pour dent, main pour main, pied pour pied\FTNT{Lé. 24:20~; De. 19:21~; Mt. 5:38.},
\VS{25}brûlure pour brûlure, plaie pour plaie, meurtrissure pour meurtrissure.
\VS{26}Si quelqu'un frappe l'œil de son serviteur, ou l'œil de sa servante, et lui gâte l'œil, il le laissera aller libre pour son œil~;
\VS{27}et s'il fait tomber une dent à son serviteur, ou à sa servante, il le laissera aller libre pour sa dent.
\VS{28}Si un bœuf heurte de sa corne un homme ou une femme, et que la personne en meure, le bœuf sera lapidé sans nulle exception, et on ne mangera point de sa chair, mais le maître du bœuf sera absous.
\VS{29}Si le bœuf était auparavant sujet à frapper de sa corne, et que son maître en ait été averti avec protestation, et qu'il ne l'ait point surveillé, s'il tue un homme ou une femme, le bœuf sera lapidé, et on fera aussi mourir son maître.
\VS{30}Si on lui impose un prix pour se racheter, il donnera la rançon de sa vie, selon tout ce qui lui sera imposé.
\VS{31}Si le bœuf heurte de sa corne un fils ou une fille, il lui sera fait selon cette même loi.
\VS{32}Si le bœuf heurte de sa corne un esclave, soit homme, soit femme, celui à qui est le bœuf donnera trente sicles d'argent au maître de l'esclave, et le bœuf sera lapidé.
\VS{33}Si quelqu'un découvre une fosse, ou si quelqu'un creuse une fosse, et ne la couvre point, et qu'il y tombe un bœuf ou un âne,
\VS{34}le maître de la fosse donnera satisfaction, et rendra l'argent au maître du bœuf, mais la bête morte lui appartiendra.
\VS{35}Et si le bœuf de quelqu'un blesse le bœuf de son prochain, et qu'il en meure, ils vendront le bœuf vivant, et en partageront l'argent par moitié, ils partageront aussi par moitié le bœuf mort.
\VS{36}Mais s'il est connu que le bœuf avait auparavant l'habitude de heurter avec sa corne, et que le maître ne l'ait point gardé, il restituera bœuf pour bœuf~; mais le bœuf mort sera pour lui.
\Chap{22}
\TextTitle{Lois sur les torts causés à autrui}
\VerseOne{}Si quelqu'un dérobe un bœuf, ou un chevreau, ou un agneau, et qu'il le tue, ou le vende, il restituera cinq bœufs pour le bœuf, et quatre agneaux ou chevreaux pour l'agneau ou pour le chevreau.
\VS{2}Si le voleur est trouvé dérobant avec effraction, et est frappé de sorte qu'il en meure, celui qui l'aura frappé ne sera point coupable de meurtre.
\VS{3}Mais si le soleil est levé sur lui, il sera coupable de meurtre. Il fera donc une entière restitution~; et s'il n'a pas de quoi, il sera vendu pour son vol.
\VS{4}Si ce qui a été dérobé est trouvé vivant entre ses mains, soit bœuf, soit âne, soit brebis ou chèvre, il rendra le double.
\VS{5}Si quelqu'un fait brouter dans un champ ou dans une vigne, en lâchant son bétail qui aille paître dans le champ d'autrui, il rendra le meilleur de son champ et le meilleur de sa vigne.
\VS{6}Si un feu éclate et rencontre des épines, et que le blé qui est en tas, ou sur pied, ou le champ, soit consumé, celui qui aura allumé le feu rendra entièrement ce qui en aura été brûlé.
\VS{7}Si quelqu'un donne à son prochain de l'argent ou des vases à garder, et qu'on le dérobe de sa maison, et si l'on trouve le voleur, il rendra le double\FTNT{Lé. 5:20-26.}.
\VS{8}Mais si on ne trouve point le voleur, on fera venir le maître de la maison devant les juges pour jurer s'il n'a point mis sa main sur le bien de son prochain.
\VS{9}Dans toute affaire d'infidélité concernant un bœuf, un âne, une brebis, une chèvre, un vêtement ou tout objet perdu, dont quelqu'un dira qu'il lui appartient, la cause des deux parties viendra devant les juges~; et celui que les juges auront condamné, rendra le double à son prochain.
\VS{10}Si quelqu'un donne à garder à son prochain un âne, un bœuf, quelque menue ou grosse bête, et qu'elle meure, ou qu'elle se soit cassée quelque membre, ou qu'on l'ait emmenée sans que personne l'ait vue,
\VS{11}le serment de Yahweh interviendra entre les deux parties\FTNT{Hé. 6:16.}, pour savoir s'il n'a point mis sa main sur le bien de son prochain, et le maître de la bête se contentera du serment, et l'autre ne la rendra point.
\VS{12}Mais s'il est vrai qu'elle lui a été dérobée, il la rendra à son maître.
\VS{13}S'il est vrai qu'elle ait été déchirée par les bêtes sauvages, il la produira en témoignage, et il ne rendra point ce qui a été déchiré.
\VS{14}Si quelqu'un a emprunté de son prochain quelque bête, et qu'elle se casse quelque membre, ou qu'elle meure, son maître n'étant point présent, il ne manquera pas de la rendre.
\VS{15}Mais si son maître est avec lui, il ne la rendra point~; si elle a été louée, on payera seulement son louage.
\TextTitle{Lois diverses}
\VS{16}Si un homme séduit une vierge non fiancée, et couche avec elle, il faut qu'il la dote, et qu'il la prenne pour femme\FTNT{De. 22:28.}.
\VS{17}Mais si le père de la fille refuse absolument de la lui donner, il lui comptera autant d'argent qu'on en donne pour la dot des vierges.
\VS{18}Tu ne laisseras point vivre la sorcière\FTNT{De. 18:10-11~; Lé. 20:27.}.
\VS{19}Celui qui couche avec une bête sera puni de mort\FTNT{Lé. 18:23~; Lé. 20:15~; De. 27:21.}.
\VS{20}Celui qui sacrifie à d'autres dieux qu'à Yahweh seul sera dévoué à la façon de l'interdit\FTNT{Lé. 17:7~; De. 13:6-16~; De. 17:2-5.}.
\VS{21}Tu ne fouleras ni n'opprimeras point l'étranger~; car vous avez été étrangers au pays d'Egypte\FTNT{Lé. 19:34.}.
\VS{22}Vous n'affligerez point la veuve ni l'orphelin\FTNT{De. 24:17-18~; Za. 7:10.}.
\VS{23}Si vous les affligez en quoi que ce soit, et qu'ils crient à moi, certainement j'entendrai leur cri.
\VS{24}Et ma colère s'embrasera, et je vous ferai mourir par l'épée~; et vos femmes seront veuves, et vos fils orphelins.
\VS{25}Si tu prêtes de l'argent à mon peuple, au pauvre qui est avec toi, tu ne te comporteras point avec lui en créancier, vous ne lui exigerez point d'intérêt.
\VS{26}Si tu prends en gage le vêtement de ton prochain, tu le lui rendras avant que le soleil soit couché\FTNT{De. 24:10-13.}.
\VS{27}Car c'est sa seule couverture, c'est son vêtement pour couvrir sa peau~; où coucherait-il~? S'il arrive donc qu'il crie à moi, je l'entendrai~; car je suis miséricordieux.
\VS{28}Tu ne maudiras point les juges, et tu ne maudiras point le prince de ton peuple\FTNT{Lé. 24:15-16.}.
\VS{29}Tu ne différeras point de m'offrir de ton abondance et de tes liqueurs~; tu me donneras le premier-né de tes fils\FTNT{Ex. 13:12-15~; De. 26:2-11.}.
\VS{30}Tu feras la même chose de ta vache, de ta brebis, et de ta chèvre. Il sera sept jours avec sa mère, et le huitième jour tu me le donneras.
\VS{31}Vous me serez saints, et vous ne mangerez point de la chair déchirée dans les champs, mais vous la jetterez aux chiens.
\Chap{23}
\TextTitle{Lois diverses (suite)}
\VerseOne{}Tu ne léveras point de faux bruit, et tu ne te joindras point au méchant pour être un faux témoin, afin que violence soit faite\FTNT{Ex. 20:16~; De. 19:16-21.}.
\VS{2}Tu ne suivras point la multitude pour faire le mal~; et tu ne témoigneras point dans un procès en sorte que tu te détournes après un grand nombre pour pervertir le droit.
\VS{3}Tu n'honoreras point le pauvre dans son procès\FTNT{De. 1:17.}.
\VS{4}Si tu rencontres le bœuf de ton ennemi, ou son âne égaré, tu ne manqueras point de le lui ramener.
\VS{5}Si tu vois l'âne de celui qui te hait, abattu sous sa charge, tu t'arrêteras pour le secourir, et tu ne manqueras pas de l'aider.
\VS{6}Tu ne pervertiras point le droit de l'indigent, qui est au milieu de toi, dans son procès.
\VS{7}Tu t'éloigneras de toute parole fausse, et tu ne feras point mourir l'innocent et le juste~; car je ne justifierai point le méchant.
\VS{8}Tu ne prendras point de présent~; car le présent aveugle les plus éclairés, et pervertit les paroles des justes.
\VS{9}Tu n'opprimeras point l'étranger~; car vous savez ce que c'est que d'être étrangers, parce que vous avez été étrangers au pays d'Egypte.
\TextTitle{Le sabbat, le repos de la terre}
\VS{10}Pendant six ans tu ensemenceras ta terre, et en recueilleras le revenu.
\VS{11}Mais la septième année, tu lui donneras du relâche, et la laisseras reposer, afin que les pauvres de ton peuple en mangent, et que les bêtes des champs mangent ce qui restera. Tu en feras de même de ta vigne et de tes oliviers.
\VS{12}Tu travailleras six jours, mais tu te reposeras au septième jour, afin que ton bœuf et ton âne se reposent, et que le fils de ta servante et l'étranger reprennent courage.
\VS{13}Vous prendrez garde à toutes les choses que je vous ai ordonnées. Vous ne ferez point mention du nom des dieux étrangers, on ne l'entendra point de ta bouche\FTNT{Jos. 23:7~; Ps. 16:4.}.
\TextTitle{Les fêtes solennelles}
\VS{14}Trois fois l'an, tu me célébreras une fête solennelle\FTNT{Lé. 23:4-44.}.
\VS{15}Tu garderas la fête solennelle des pains sans levain\FTNT{Ex. 29:2.}~; tu mangeras des pains sans levain pendant sept jours, comme je t'ai ordonné, en la saison et au mois où les épis mûrissent~; car c'est en ce mois-là que tu es sorti d'Egypte~; et nul ne se présentera devant ma face à vide.
\VS{16}Et la fête solennelle de la moisson des premiers fruits de ton travail, de ce que tu auras semé au champ~; et la fête de la récolte, après la fin de l'année, quand tu auras recueilli du champ les fruits de ton travail\FTNT{Ex. 34:22.}.
\VS{17}Trois fois l'an, tous les mâles d'entre vous se présenteront devant le Seigneur Yahweh.
\VS{18}Tu ne sacrifieras point le sang de mon sacrifice avec du pain levé~; et la graisse de ma fête solennelle ne passera point la nuit jusqu'au matin\FTNT{Ex. 34:25-26.}.
\VS{19}Tu apporteras dans la maison de Yahweh, ton Dieu, les prémices des premiers fruits de ta terre. Tu ne feras point cuire le chevreau dans le lait de sa mère.
\TextTitle{Mises en garde et promesses de Yahweh}
\VS{20}Voici, j'envoie un Ange devant toi, afin qu'il te garde dans le chemin, et qu'il t'introduise dans le lieu que je t'ai préparé.
\VS{21}Garde-toi de provoquer sa colère, et écoute sa voix, et ne l'irrite point, car il ne pardonnera point votre péché~; car mon Nom est en lui.
\VS{22}Mais si tu écoutes attentivement sa voix, et si tu fais tout ce que je te dirai, je serai l'ennemi de tes ennemis, et j'affligerai ceux qui t'affligeront.
\VS{23}Car mon Ange marchera devant toi, et t'introduira au pays des Amoréens, des Héthiens, des Phéréziens, des Cananéens, des Héviens, et des Jébusiens, et je les exterminerai.
\VS{24}Tu ne te prosterneras point devant leurs dieux, et tu ne les serviras point, et tu ne feras point selon leurs œuvres, mais tu les détruiras entièrement, et tu briseras entièrement leurs statues\FTNT{Ex. 20:5~; Ex. 34:13~; No. 33:52.}.
\VS{25}Vous servirez Yahweh, votre Dieu. Et il bénira ton pain et tes eaux~; et j'ôterai les maladies du milieu de toi\FTNT{Ex. 15:26~; De. 6:13~; De. 7:15-16~; Mt. 4:10.}.
\VS{26}Il n'y aura point dans ton pays de femme qui avorte, ou qui soit stérile~; j'accomplirai le nombre de tes jours.
\VS{27}J'enverrai la terreur de mon Nom devant toi, et j'effrayerai tout peuple vers lequel tu arriveras, et je ferai que tous tes ennemis tourneront le dos devant toi\FTNT{De. 7:23.}.
\VS{28}Et j'enverrai des frelons devant toi, qui chasseront les Héviens, les Cananéens, et les Héthiens, de devant ta face\FTNT{De. 7:20~; Jos. 24:12.}.
\VS{29}Je ne les chasserai point loin de devant ta face en une année, de peur que le pays ne devienne un désert, et que les bêtes des champs ne se multiplient contre toi.
\VS{30}Mais je les chasserai peu à peu loin de devant toi, jusqu'à ce que tu te sois accru, et que tu possèdes le pays.
\VS{31}Et je mettrai des bornes depuis la Mer Rouge jusqu'à la mer des Philistins, et depuis le désert jusqu'au fleuve~; car je livrerai entre tes mains les habitants du pays et je les chasserai de devant toi.
\VS{32}Tu ne traiteras point d'alliance avec eux ni avec leurs dieux.
\VS{33}Ils n'habiteront point dans ton pays, de peur qu'ils ne te fassent pécher contre moi~; car tu servirais leurs dieux, et ce serait un piège pour toi.
\Chap{24}
\TextTitle{La loi lue au peuple~; le sang de l'Alliance}
\VerseOne{}Puis il dit à Moïse~: Monte vers Yahweh, toi et Aaron, Nadab et Abihu, et soixante-dix des anciens d'Israël, et vous vous prosternerez de loin.
\VS{2}Et Moïse s'approchera seul de Yahweh, mais eux ne s'en approcheront point, et le peuple ne montera point avec lui.
\VS{3}Alors Moïse vint, et récita au peuple toutes les paroles de Yahweh, et toutes ses lois, et tout le peuple répondit d'une voix, et dit~: Nous ferons toutes les choses que Yahweh a dites.
\VS{4}Or Moïse écrivit toutes les paroles de Yahweh, et s'étant levé de bon matin, il bâtit un autel au bas de la montagne, et dressa pour monument douze pierres pour les douze tribus d'Israël.
\VS{5}Et il envoya des jeunes hommes, des enfants d'Israël, qui offrirent des holocaustes et qui sacrifièrent des veaux à Yahweh en sacrifice d'offrande de paix.
\VS{6}Et Moïse prit la moitié du sang, et le mit dans des bassins, et répandit l'autre moitié sur l'autel.
\VS{7}Ensuite, il prit le livre de l'Alliance et le lut, et le peuple qui l'écoutait dit~: Nous ferons tout ce que Yahweh a dit, et nous obéirons.
\VS{8}Moïse donc prit le sang, et le répandit sur le peuple, en disant~: Voici le sang de l'Alliance que Yahweh a traitée avec vous, selon toutes ces paroles\FTNT{Mt. 26:28~; Mc. 14:24~; Lu. 22:20~; 1 Co. 11:25~; Hé. 9:20.}.
\TextTitle{Yahweh fait monter Moïse sur la montagne}
\VS{9}Puis Moïse, Aaron, Nadab, Abihu, et les soixante-dix anciens d'Israël montèrent.
\VS{10}Et ils virent le Dieu d'Israël, et sous ses pieds comme un ouvrage de saphir transparent, comme le ciel dans toute sa pureté.
\VS{11}Et il ne mit point sa main sur ceux qui avaient été choisis d'entre les enfants d'Israël~; ainsi, ils virent Dieu, et ils mangèrent et burent.
\VS{12}Et Yahweh dit à Moïse~: Monte vers moi sur la montagne, et demeure là~; et je te donnerai des tables de pierre, la loi et les commandements que j'ai écrits pour les enseigner.
\VS{13}Alors Moïse se leva avec Josué qui le servait~; et Moïse monta sur la montagne de Dieu.
\VS{14}Et il dit aux anciens d'Israël~: Demeurez ici en nous attendant jusqu'à ce que nous retournions vers vous. Et voici, Aaron et Hur seront avec vous~; quiconque aura quelque affaire, qu'il s'adresse à eux.
\VS{15}Moïse donc monta sur la montagne, et une nuée couvrit la montagne\FTNT{Ex. 19:9-16.}.
\VS{16}Et la gloire de Yahweh demeura sur la montagne de Sinaï, et la nuée la couvrit pendant six jours. Et au septième jour, il appela Moïse du milieu de la nuée.
\VS{17}Et ce qu'on voyait de la gloire de Yahweh au sommet de la montagne, était comme un feu dévorant aux yeux des enfants d'Israël\FTNT{De. 4:24~; De. 9:3~; Hé. 12:29.}.
\VS{18}Et Moïse entra dans la nuée et monta sur la montagne. Moïse fut sur la montagne quarante jours et quarante nuits.
\Chap{25}
\TextTitle{Des offrandes volontaires pour les matériaux du tabernacle}
\VerseOne{}Et Yahweh parla à Moïse, en disant~:
\VS{2}Parle aux enfants d'Israël, et qu'on prenne une offrande pour moi. Vous prendrez mon offrande de tout homme dont le cœur me l'offrira volontairement.
\VS{3}Et c'est ici l'offrande que vous prendrez d'eux~: De l'or, de l'argent, de l'airain,
\VS{4}de la pourpre, de l'écarlate, du cramoisi\FTNT{La couleur cramoisi s'obtient grâce à la femelle cochenille aptère qui contient dans son corps et dans ses œufs un pigment rouge à base d'acide carminique qui permet à l'insecte et à ses larves de se protéger des prédateurs. Au moment de la ponte, cette dernière fixe fermement son corps au tronc d'un arbre puis libère ses œufs qui demeurent ainsi protégés en dessous d'elle jusqu'à leur éclosion. Ensuite, l'insecte meurt en libérant cette substance rouge qui se propage sur tout son corps et sur le bois hôte. C'est ce fluide que l'homme récupère pour en faire un colorant à la couleur caractéristique. Une subtile analogie peut être faire entre la cochenille et le Seigneur qui a versé son sang à la croix pour nous donner la vie. «~Et moi, je suis un ver, et non un homme, l'opprobre des hommes et le méprisé du peuple~» (Ps. 22~:7).}, du fin lin, du poil de chèvre,
\VS{5}des peaux de béliers teintes en rouge, des peaux de taissons\FTNT{Le mot hébreu employé ici est «~tachash~», il désigne le matériau servant à fabriquer la couverture extérieure de la tente d'assignation. Si tout le monde s'accorde pour dire qu'il s'agissait d'une fourrure ou d'une peau d'animal, un doute subsiste sur la race exacte de l'animal. On hésite entre le marsouin, le dauphin, le blaireau (taisson) ou peut-être le mouton. Dans de nombreuses bibles, on a pris le parti de traduire par «~peaux de dauphins~». Cette hypothèse est cependant très peu probable. D'une part parce que le dauphin n'a pas de fourrure~; d'autre part parce que sa peau n'est absolument pas adaptée à la vie terrestre. Elle est donc impossible à conserver et à transformer, en particulier dans le contexte d'un climat propre au désert. Certains pensent qu'il s'agit tout simplement de peaux de béliers. Dans ce cas, comment expliquer qu'on n'ait pas employé le terme «~'ayil~» comme cela est mentionné pour les peaux teintes en rouge~? Il reste donc les peaux de taissons, c'est-à-dire de blaireaux, dont la fourrure est utilisée depuis des siècles. On peut objecter qu'il est impossible que le Seigneur puisse accepter la peau d'un animal impur pour construire son sanctuaire. Si tel était le cas, la peau du dauphin, qui est également un animal impur, n'aurait pas non-plus été autorisée. Toutefois, d'un point de vue prophétique, le symbole est important~: La présence de cet animal impur préfigurait Christ qui a pris une chair semblable à celle du péché (Ro. 8:3) mais aussi le levain que l'on peut trouver dans la pate nouvelle (1 Co. 5:6-7), l'ivraie qui se glisse parmi le blé (Mt. 13:25-43).}, du bois d'acacia,
\VS{6}de l'huile pour le luminaire, des aromates pour l'huile d'onction et pour le parfum odoriférant,
\VS{7}des pierres d'onyx, et d'autres pierres pour la garniture de l'éphod et pour le pectoral.
\VS{8}Et ils me feront un sanctuaire, et j'habiterai au milieu d'eux\FTNT{Ex. 29:45-46.}.
\VS{9}Ils le feront conformément à tout ce que je vais te montrer, selon le modèle du tabernacle et selon le modèle de tous ses ustensiles~; vous le ferez donc ainsi.
\TextTitle{L'arche de l'alliance}
\VS{10}Et ils feront une arche de bois d'acacia~; et sa longueur sera de deux coudées et demie, et sa largeur d'une coudée et demie, et sa hauteur d'une coudée et demie.
\VS{11}Et tu la couvriras d'or pur, tu l'en couvriras en dehors et en dedans~; et tu feras sur elle un couronnement d'or tout autour\FTNT{Ex. 37:1-9.}.
\VS{12}Et tu fondras pour elle quatre anneaux d'or, que tu mettras à ses quatre coins, deux anneaux à l'un de ses côtés, et deux autres de l'autre côté.
\VS{13}Tu feras aussi des barres de bois d'acacia, et tu les couvriras d'or.
\VS{14}Puis tu feras entrer les barres dans les anneaux aux côtés de l'arche, pour porter l'arche avec elles.
\VS{15}Les barres seront dans les anneaux de l'arche, et on ne les en tirera point.
\VS{16}Et tu mettras dans l'arche le témoignage que je te donnerai\FTNT{Hé. 9:4.}.
\VS{17}Tu feras aussi un propitiatoire d'or pur, dont la longueur sera de deux coudées et demie, et la largeur d'une coudée et demie.
\VS{18}Et tu feras deux chérubins d'or~; tu les feras d'ouvrage étendu au marteau, tirés des deux extrémités du propitiatoire.
\VS{19}Fais donc un chérubin tiré des extrémités et un chérubin tiré de l'autre extrémité~; vous ferez les chérubins tirés du propitiatoire à ses deux extrémités.
\VS{20}Et les chérubins étendront les ailes en haut, couvrant de leurs ailes le propitiatoire, et leurs faces seront vis-à-vis l'une de l'autre~; et le regard des chérubins sera vers le propitiatoire\FTNT{1 R. 8:6-7~; Hé. 9:5.}.
\VS{21}Et tu poseras le propitiatoire au-dessus de l'arche, et tu mettras dans l'arche le témoignage que je te donnerai.
\VS{22}Et je me rencontrerai là avec toi, et je te dirai de dessus le propitiatoire, d'entre les deux chérubins qui seront sur l'arche du témoignage, toutes les choses que je t'ordonnerai pour les enfants d'Israël\FTNT{Ex. 29:42-43~; No. 7:89.}.
\TextTitle{La table des pains de proposition}
\VS{23}Tu feras aussi une table de bois d'acacia. Sa longueur sera de deux coudées, et sa largeur d'une coudée, et sa hauteur d'une coudée et demie.
\VS{24}Tu la couvriras d'or pur, et tu lui feras un couronnement d'or tout autour.
\VS{25}Tu lui feras aussi à l'entour une clôture d'une largeur de main, et tout autour de sa clôture tu feras un couronnement d'or.
\VS{26}Tu lui feras aussi quatre anneaux d'or que tu mettras aux quatre coins qui seront à ses quatre pieds.
\VS{27}Les anneaux seront à l'endroit de la clôture, afin d'y mettre les barres pour porter la table.
\VS{28}Tu feras les barres de bois d'acacia, et tu les couvriras d'or, et on portera la table avec elles.
\VS{29}Tu feras aussi ses plats, ses tasses, ses gobelets, et ses bassins, avec lesquels on fera les aspersions~; tu les feras d'or pur\FTNT{Ex. 37:10-16.}.
\VS{30}Et tu mettras sur cette table le pain de proposition continuellement devant moi\FTNT{Lé. 24:5-9.}.
\TextTitle{Le chandelier d'or pur}
\VS{31}Tu feras aussi un chandelier d'or pur\FTNT{Le chandelier avait une double symbolique. D'une part, il préfigurait Jésus-Christ, notre Lumière (Jn. 1:4-5~; Jn. 8:12). Les sept lampes évoquaient l'omniscience de l'Esprit de Jésus-Christ (Za. 3:9~; Jn. 16:29-30~; Ap. 1:4~; Ap. 3:1~; Ap. 4:5~; Ap. 5:6). Il est à noter que ce chandelier comportait des calices en forme de fleurs, de pommes et d'amandes (Ex. 25:33) qui symbolisaient les fruits de l'Esprit que nous devons nécessairement porter (Ga. 5:22). D'autre part, il est une image de l'Eglise (Ap. 1:20). Voir commentaire Ex. 37:17-24~; Es. 8:13-17.}. Le chandelier sera étendu au marteau~; son pied, sa tige et ses branches, ses plats, ses pommeaux et ses fleurs seront tirés de lui.
\VS{32}Six branches sortiront de ses côtés~: Trois branches d'un côté du chandelier, et trois autres de l'autre côté du chandelier.
\VS{33}Il y aura sur l'une des branches trois petits plats en forme d'amande, un pommeau et une fleur~; sur l'autre branche trois petits plats en forme d'amande, un pommeau et une fleur~; il en sera de même des six branches sortant du chandelier.
\VS{34}ll y aura aussi au chandelier quatre petits plats en forme d'amande, ses pommeaux et ses fleurs.
\VS{35}Un pommeau sous deux branches tirées du chandelier, un pommeau sous deux autres branches tirées de lui, et un pommeau sous deux autres branches tirées de lui~; il en sera de même des six branches sortant du chandelier.
\VS{36}Leurs pommeaux et leurs branches seront tirés de lui, et tout le chandelier sera un seul ouvrage étendu au marteau, et d'or pur.
\VS{37}Tu feras aussi ses sept lampes, et on les allumera afin qu'elles éclairent vis-à-vis du chandelier.
\VS{38}Et ses mouchettes et ses encensoirs destinés à recevoir ce qui tombe des lampes seront d'or pur.
\VS{39}On le fera avec tous ses ustensiles d'un talent d'or pur.
\VS{40}Regarde donc, et fais selon le modèle qui t'est montré sur la montagne.
\Chap{26}
\TextTitle{Les tapis de fin lin}
\VerseOne{}Tu feras aussi le tabernacle de dix tapis de fin lin retors, de pourpre, d'écarlate, et de cramoisi~; et tu les feras semés de chérubins d'un ouvrage exquis\FTNT{Ex. 36:8-38.}.
\VS{2}La longueur d'un tapis sera de vingt-huit coudées, et la largeur du même tapis de quatre coudées~; tous les tapis auront une même mesure.
\VS{3}Cinq de ces tapis seront joints l'un à l'autre, et les cinq autres seront aussi joints l'un à l'autre.
\VS{4}Fais aussi des lacets de pourpre sur le bord d'un tapis, au bord du premier assemblage~; et tu feras la même chose au bord du dernier tapis dans l'autre assemblage.
\VS{5}Tu feras donc cinquante lacets au premier tapis, et tu feras cinquante lacets au bord du tapis qui est dans le second assemblage. Les lacets seront vis-à-vis l'un de l'autre.
\VS{6}Tu feras aussi cinquante crochets d'or, et tu attacheras les tapis l'un à l'autre avec les crochets~; ainsi le tabernacle ne fera qu'un.
\TextTitle{Les tapis de poils de chèvre}
\VS{7}Tu feras aussi des tapis de poils de chèvres pour servir de tente sur le tabernacle~; tu feras onze de ces tapis.
\VS{8}La longueur d'un tapis sera de trente coudées, et la largeur du même tapis sera de quatre coudées~; les onze tapis auront une même mesure.
\VS{9}Puis tu joindras séparément cinq de ces tapis, et les six tapis à part~; mais tu redoubleras le sixième tapis sur le devant du tabernacle.
\VS{10}Tu feras aussi cinquante lacets sur le bord de l'un des tapis, à savoir au dernier qui est assemblé, et cinquante lacets au bord du tapis du second assemblage.
\VS{11}Tu feras aussi cinquante crochets d'airain, et tu feras entrer les crochets dans les lacets~; et tu assembleras ainsi la tente qui fera un tout.
\VS{12}Mais ce qu'il y aura en surplus dans les tapis de la tente, à savoir la moitié du tapis de reste, retombera sur le derrière du tabernacle.
\VS{13}La coudée d'une part, et la coudée d'autre part, qui seront de reste sur la longueur des tapis de la tente, retomberont sur les deux côtés du tabernacle, pour le couvrir.
\TextTitle{Les couvertures de peaux de béliers}
\VS{14}Tu feras aussi pour ce tabernacle une couverture de peaux de béliers teintes en rouge, et une couverture de peaux de taissons par-dessus\FTNT{Ex. 35:7~; Ex. 35:23~; Ex. 36:19~; Ex. 39:34.}.
\TextTitle{Les planches et leurs bases}
\VS{15}Et tu feras pour le tabernacle, des planches de bois d'acacia, qu'on fera tenir debout\FTNT{Ex. 36:20-34.}.
\VS{16}La longueur d'une planche sera de dix coudées, et la largeur d'une même planche d'une coudée et demie.
\VS{17}Il y aura à chaque planche deux tenons joints l'un à l'autre~; et tu feras de même pour toutes les planches du tabernacle.
\VS{18}Tu feras donc les planches du tabernacle, à savoir vingt planches qui regardent vers le midi.
\VS{19}Et au-dessous des vingt planches, tu feras quarante bases d'argent~; deux bases sous une planche pour ses deux tenons, et deux bases sous l'autre planche pour ses deux tenons.
\VS{20}Et vingt planches de l'autre côté du tabernacle, du coté nord.
\VS{21}Et leurs quarante bases seront d'argent, deux bases sous une planche, et deux bases sous l'autre planche.
\VS{22}Et pour le fond du tabernacle, vers l'occident, tu feras six planches.
\VS{23}Tu feras aussi deux planches pour les angles du tabernacle, aux deux cotés du fond.
\VS{24}Et elles seront égales par le bas, et elles seront jointes et unies par le haut avec un anneau~; il en sera de même des deux planches qui seront aux deux angles.
\VS{25}Il y aura donc huit planches, et seize bases d'argent~; deux bases sous une planche et deux bases sous une autre planche.
\VS{26}Après cela, tu feras cinq barres de bois d'acacia, pour les planches d'un des côtés du tabernacle.
\VS{27}Pareillement, tu feras cinq barres pour les planches de l'autre côté du tabernacle~; et cinq barres pour les planches du côté du tabernacle, pour le fond, vers le côté de l'occident.
\VS{28}Et la barre du milieu sera au milieu des planches d'une extrémité à l'autre.
\VS{29}Tu couvriras aussi d'or les planches, et tu feras d'or leurs anneaux pour mettre les barres, et tu couvriras d'or les barres.
\VS{30}Tu dresseras le tabernacle selon le modèle qui t'est montré sur la montagne.
\TextTitle{Les voiles intérieurs et extérieurs}
\VS{31}Et tu feras un voile\FTNT{Le voile intérieur symbolisait la chair de Jésus-Christ qui a été brisée à cause de nos péchés (Es. 53:5~; Hé. 10:20). Ex. 36:35-38~; Mt. 27:51~; Hé. 9:3.} de pourpre, d'écarlate, de cramoisi, et de fin lin retors~; on le fera d'ouvrage exquis, avec des chérubins.
\VS{32}Et tu le mettras sur quatre piliers de bois d'acacia couverts d'or, ayant leurs crochets d'or. Et ils seront sur quatre bases d'argent.
\VS{33}Puis tu mettras le voile sous les crochets, et tu feras entrer là-dedans, c'est-à-dire au-dedans du voile, l'arche du témoignage~; et ce voile vous fera la séparation entre le lieu saint et le Saint des saints.
\VS{34}Et tu poseras le propitiatoire sur l'arche du témoignage, dans le Saint des saints.
\VS{35}Et tu mettras la table au dehors de ce voile, et le chandelier vis-à-vis de la table, au côté du tabernacle, vers le sud~; et tu placeras la table côté nord.
\VS{36}Et à l'entrée du tabernacle, tu feras un rideau de pourpre, d'écarlate, de cramoisi et de fin lin retors, d'ouvrage de broderie.
\VS{37}Tu feras aussi pour ce rideau cinq piliers de bois d'acacia, que tu couvriras d'or, et leurs crochets seront d'or~; et tu fondras pour eux cinq bases d'airain.
\Chap{27}
\TextTitle{L'autel d'airain}
\VerseOne{}Tu feras aussi un autel de bois d'acacia, ayant cinq coudées de long, et cinq coudées de large~; l'autel sera carré, et sa hauteur sera de trois coudées.
\VS{2}Tu feras ses cornes à ses quatre coins~; ses cornes seront tirées de lui, et tu le couvriras d'airain\FTNT{C'est sur l'autel d'airain que les animaux étaient sacrifiés. Il préfigurait la croix et le jugement que Jésus-Christ a pris sur lui à notre place (Es. 53:5~; 2 Co. 13:4~; Ph. 2:8).}.
\VS{3}Tu feras ses chaudrons pour recevoir ses cendres, et ses racloirs, ses bassins, ses fourchettes, et ses encensoirs~; tu feras tous ses ustensiles d'airain.
\VS{4}Tu lui feras une grille d'airain en forme de treillis, et tu feras au treillis quatre anneaux d'airain à ses quatre coins.
\VS{5}Et tu le mettras au-dessous de l'enceinte de l'autel en bas, et le treillis s'étendra jusqu'au milieu de l'autel.
\VS{6}Tu feras aussi des barres pour l'autel, des barres de bois d'acacia, et tu les couvriras d'airain.
\VS{7}Et on fera passer ses barres dans les anneaux~; les barres seront aux deux côtés de l'autel pour le porter.
\VS{8}Tu le feras creux avec des planches~; ils le feront ainsi qu'il t'a été montré sur la montagne\FTNT{Ex. 38:1-7.}.
\TextTitle{Le parvis}
\VS{9}Tu feras aussi le parvis du tabernacle, du côté qui regarde vers le sud~; il y aura pour former le parvis, des courtines de fin lin retors~; la longueur de l'un des côtés sera de cent coudées.
\VS{10}Il y aura vingt piliers avec leurs vingt bases d'airain, mais les crochets des piliers et leurs filets seront d'argent.
\VS{11}Ainsi du côté nord, il y aura également des courtines sur une longueur de cent coudées, avec vingt piliers avec leurs vingt bases d'airain~; mais les crochets des piliers avec leurs filets seront d'argent.
\VS{12}La largeur du parvis du côté de l'occident sera de cinquante coudées de courtines, qui auront dix piliers, avec leurs dix bases.
\VS{13}Et la largeur du parvis du côté de l'orient, directement vers le levant, sera de cinquante coudées.
\VS{14}A l'un des côtés, il y aura quinze coudées de courtines, avec leurs trois piliers et leurs trois bases.
\VS{15}Et de l'autre côté, quinze coudées de courtines, avec leurs trois piliers et leurs trois bases.
\TextTitle{La porte du parvis}
\VS{16}Il y aura aussi pour la porte du parvis un rideau de vingt coudées, fait de pourpre, d'écarlate, de cramoisi, et de fin lin retors, ouvrage de broderie, avec quatre piliers et quatre bases.
\VS{17}Tous les piliers du parvis seront ceints d'un filet d'argent, et leurs crochets seront d'argent, mais leurs bases seront d'airain.
\VS{18}La longueur du parvis sera de cent coudées, et la largeur de cinquante, de chaque côté~; et la hauteur de cinq coudées. Il sera de fin lin retors, et les bases des piliers seront d'airain.
\VS{19}Que tous les ustensiles du tabernacle, pour tout son service, et tous ses pieux, avec les pieux du parvis, soient d'airain\FTNT{Ex. 38:9-20.}.
\TextTitle{L'huile d'olive vierge pour les lampes}
\VS{20}Tu ordonneras aux fils d'Israël qu'ils t'apportent de l'huile d'olive vierge pour le luminaire, afin de faire luire les lampes continuellement\FTNT{Ex. 35:8-28~; Lé. 24:1-4.}.
\VS{21}Aaron avec ses fils les prépareront dans la présence de Yahweh, depuis le soir jusqu'au matin, dans la tente d'assignation, hors du voile qui est devant le témoignage~; ce sera une ordonnance perpétuelle pour les enfants d'Israël.
\Chap{28}
\TextTitle{La prêtrise}
\VerseOne{}Et toi, fais approcher de toi Aaron, ton frère, et ses fils avec lui, d'entre les enfants d'Israël, pour m'exercer la prêtrise, à savoir Aaron, Nadab et Abihu, Eléazar et Ithamar, fils d'Aaron.
\VS{2}Et tu feras à Aaron, ton frère, de saints vêtements pour gloire et pour ornement.
\TextTitle{Les vêtements sacrés des prêtres}
\VS{3}Et tu parleras à tous les hommes d'esprit, à chacun de ceux que j'ai remplis de l'esprit de science, afin qu'ils fassent des vêtements à Aaron pour le sanctifier, afin qu'il m'exerce la prêtrise.
\VS{4}Et ce sont ici les vêtements qu'ils feront~: Le pectoral, l'éphod, la robe, la tunique brodée, la tiare, et la ceinture. Ils feront donc les saints vêtements à Aaron, ton frère, et à ses fils, pour m'exercer la prêtrise.
\VS{5}Et ils prendront de l'or, de la pourpre, de l'écarlate, du cramoisi, et du fin lin.
\TextTitle{L'éphod}
\VS{6}Et ils feront l'éphod d'or, de pourpre, d'écarlate et de cramoisi, et de fin lin retors~; d'un ouvrage exquis.
\VS{7}Il aura deux épaulettes qui se joindront par les deux bouts~; et c'est ainsi qu'il sera joint.
\VS{8}La ceinture exquise dont il sera ceint, et qui sera par-dessus, sera de même ouvrage, et tirée de lui, étant d'or, de pourpre, d'écarlate, de cramoisi, et de fin lin retors.
\VS{9}Et tu prendras deux pierres d'onyx, et tu graveras sur elles les noms des enfants d'Israël~:
\VS{10}Six de leurs noms sur une pierre et les six noms des autres sur l'autre pierre, selon leur naissance.
\VS{11}Tu graveras sur les deux pierres les noms des enfants d'Israël, comme on grave les pierres et les cachets, tu les entoureras de montures d'or.
\VS{12}Et tu mettras les deux pierres sur les épaulettes de l'éphod, afin qu'elles soient des pierres de souvenir pour les enfants d'Israël~; car Aaron portera leurs noms sur ses deux épaules devant Yahweh, pour souvenir.
\VS{13}Tu feras aussi des montures d'or,
\VS{14}et deux chaînettes d'or pur que tu tresseras en forme de cordons, et tu fixeras aux montures les chaînettes ainsi tressées.
\TextTitle{Le pectoral}
\VS{15}Tu feras aussi le pectoral du jugement d'un ouvrage exquis, comme l'ouvrage de l'éphod, d'or, de pourpre, d'écarlate, de cramoisi, et de fin lin retors.
\VS{16}Il sera carré et double~; et sa longueur sera d'un empan, et sa largeur d'un empan.
\VS{17}Et tu le rempliras de garniture de pierres, à quatre rangées de pierres précieuses. A la première rangée, on mettra une sardoine, une topaze, et une émeraude.
\VS{18}Et à la seconde rangée, une escarboucle, un saphir, et un jaspe.
\VS{19}Et à la troisième rangée, une opale, une agate, et une améthyste.
\VS{20}Et à la quatrième rangée, un chrysolithe, un onyx et un béryl, qui seront enchâssés dans de l'or, selon leur garniture.
\VS{21}Et ces pierres-là seront selon les noms des enfants d'Israël, douze selon leurs noms, chacune d'elles gravées comme des cachets, selon le nom qu'elle doit porter, et elles seront pour les douze tribus.
\VS{22}Tu feras donc pour le pectoral des chaînettes d'or pur, tressées en forme de cordon.
\VS{23}Et tu feras sur le pectoral deux anneaux d'or, et tu mettras les deux anneaux aux deux bouts du pectoral.
\VS{24}Et tu mettras les deux chaînettes d'or, faites en cordon, dans les deux anneaux à l'extrémité du pectoral.
\VS{25}Et tu mettras les deux autres bouts des deux chaînettes en cordon sur les deux montures, et tu les mettras sur les épaulettes de l'éphod, sur le devant de l'éphod.
\VS{26}Tu feras aussi deux autres anneaux d'or, que tu mettras aux deux autres bouts du pectoral, sur le bord qui sera du côté de l'éphod à l'intérieur.
\VS{27}Et tu feras deux autres anneaux d'or, que tu mettras aux deux épaulettes de l'éphod par le bas, sur le devant, à l'endroit où il se joint, au-dessus de la ceinture exquise de l'éphod.
\VS{28}Et ils joindront le pectoral élevé par ses anneaux, aux anneaux de l'éphod, avec un cordon de pourpre, afin qu'il tienne au-dessus de la ceinture exquise de l'éphod, et que le pectoral ne puisse pas se séparer de l'éphod.
\VS{29}Ainsi, Aaron portera sur son cœur les noms des enfants d'Israël gravés sur le pectoral du jugement, quand il entrera dans le lieu saint, pour souvenir devant Yahweh continuellement.
\TextTitle{L'urim et le thummim}
\VS{30}Et tu mettras sur le pectoral de jugement l'urim et le thummim\FTNT{L'urim («~lumières~») et le thummim («~perfections~») étaient deux pierres du pectoral que l'on utilisait ensemble pour déterminer la décision de Dieu sur certaines questions.}, qui seront sur le cœur d'Aaron, quand il viendra devant Yahweh~; et Aaron portera le jugement des enfants d'Israël sur son cœur devant Yahweh continuellement.
\TextTitle{La robe de l'éphod}
\VS{31}Tu feras aussi la robe de l'éphod entièrement de pourpre.
\VS{32}Il y aura, au milieu, une ouverture pour la tête, et cette ouverture aura tout autour un bord tissé, comme l'ouverture d'une cotte de mailles, afin que la robe ne se déchire pas.
\VS{33}Tu feras à ses bords des grenades de pourpre, d'écarlate, et de cramoisi tout autour, et des clochettes d'or entre elles tout autour.
\VS{34}Une clochette d'or, puis une grenade, une clochette d'or, puis une grenade, aux bords de la robe tout autour.
\VS{35}Et Aaron en sera revêtu quand il fera le service, et on en entendra le son lorsqu'il entrera dans le lieu saint devant Yahweh, et quand il en sortira, afin qu'il ne meure pas.
\TextTitle{La lame d'or gravée~: La sainteté à Yahweh}
\VS{36}Et tu feras une lame d'or pur, sur laquelle tu graveras ces mots, comme on grave un cachet~: La sainteté à Yahweh.
\VS{37}Tu l'attacheras avec un cordon de pourpre sur la tiare, sur le devant de la tiare.
\VS{38}Et elle sera sur le front d'Aaron~; et Aaron portera l'iniquité commise par les enfants d'Israël, en faisant leurs saintes offrandes, elle sera continuellement sur son front devant Yahweh, pour qu'il leur soit favorable.
\TextTitle{Les vêtements de service d'Aaron et ses fils}
\VS{39}Tu feras aussi une tunique de fin lin qui s'appliquera sur le corps, et tu feras aussi la tiare de fin lin~; mais tu feras la ceinture d'ouvrage de broderie\FTNT{Ex. 39:1-32.}.
\VS{40}Tu feras aussi aux fils d'Aaron des tuniques, des ceintures, et des bonnets, pour leur gloire et leur ornement.
\VS{41}Et tu en revêtiras Aaron, ton frère, et ses fils avec lui~; tu les oindras, tu les consacreras et tu les sanctifieras~; puis ils exerceront la prêtrise pour moi\FTNT{Lé. 8:12~; Lé. 16:32~; No. 3:3.}.
\VS{42}Et tu leur feras des caleçons de lin, pour couvrir leur nudité, qui tiendront depuis les reins jusqu'au bas des cuisses.
\VS{43}Et Aaron et ses fils seront ainsi habillés quand ils entreront dans la tente d'assignation, ou quand ils approcheront de l'autel pour faire le service dans le lieu saint~; et ils ne porteront point la peine d'aucune iniquité, et ne mourront point. Ce sera une ordonnance perpétuelle pour lui et pour sa postérité après lui.
\Chap{29}
\TextTitle{Les prêtres consacrés au service de Yahweh}
\VerseOne{}Or c'est ici ce que tu leur feras, quand tu les sanctifieras pour exercer la prêtrise pour moi~: Prends un veau du troupeau, et deux béliers sans tare\FTNT{Lé. 8:2~; Lé. 9:2~; Hé. 7:26-28.}~;
\VS{2}et des pains sans levain, et des gâteaux sans levain pétris à l'huile, et des beignets sans levain, oints d'huile~; et tu les feras de fine farine de froment\FTNT{Lé. 6:13.}.
\VS{3}Tu les mettras dans une corbeille, et tu les présenteras dans la corbeille~; tu présenteras aussi le veau et les deux moutons.
\VS{4}Puis, tu feras approcher Aaron et ses fils à l'entrée de la tente d'assignation, et tu les laveras avec de l'eau\FTNT{Ex. 40:12.}.
\VS{5}Ensuite, tu prendras les vêtements, et tu feras vêtir à Aaron la tunique et la robe de l'éphod, l'éphod et pectoral, et tu le ceindras par-dessus avec la ceinture exquise de l'éphod.
\VS{6}Puis, tu mettras sur sa tête la tiare et la couronne de sainteté sur la tiare.
\VS{7}Et tu prendras l'huile d'onction et la répandras sur sa tête~; et tu l'oindras ainsi.
\VS{8}Puis, tu feras approcher ses fils, et tu leur feras vêtir les tuniques.
\VS{9}Et tu les ceindras des ceintures, Aaron, dis-je, et ses fils\FTNT{Es. 11:5~; Ep. 6:14.}, et tu leur attacheras des bonnets, et ils posséderont la prêtrise par ordonnance perpétuelle. Et tu consacreras ainsi Aaron et ses fils.
\VS{10}Et tu feras approcher le veau devant la tente d'assignation, et Aaron et ses fils poseront leurs mains sur la tête du veau.
\VS{11}Et tu égorgeras le veau devant Yahweh, à l'entrée de la tente d'assignation.
\VS{12}Puis, tu prendras du sang du veau, et le mettras avec ton doigt sur les cornes de l'autel, et tu répandras tout le reste du sang au pied de l'autel.
\VS{13}Tu prendras aussi toute la graisse qui couvre les entrailles, et le lobe du foie, les deux rognons, la graisse qui les entoure, et tu les feras fumer sur l'autel.
\VS{14}Mais tu brûleras au feu la chair du veau, sa peau, et ses excréments, hors du camp. C'est un sacrifice pour le péché\FTNT{Lé. 1:3-13~; Hé. 9:11~; Hé. 13:11.}.
\VS{15}Puis, tu prendras l'un des béliers, et Aaron et ses fils poseront leurs mains sur la tête du bélier.
\VS{16}Puis, tu égorgeras le bélier, et prenant son sang, tu le répandras sur l'autel tout autour.
\VS{17}Après, tu couperas le bélier par pièces, et ayant lavé ses entrailles et ses jambes, tu les mettras sur ses pièces et sur sa tête.
\VS{18} Et tu feras fumer tout le bélier sur l'autel~; c'est un holocauste à Yahweh, c'est un sacrifice consumé par le feu d'une agréable odeur à Yahweh.
\VS{19}Puis, tu prendras l'autre bélier, et Aaron et ses fils mettront leurs mains sur sa tête.
\VS{20}Et tu égorgeras le bélier, et prenant de son sang, tu le mettras sur le lobe de l'oreille droite d'Aaron, et sur le lobe de l'oreille droite de ses fils, sur le pouce de leur main droite, sur le gros orteil de leur pied droit, et tu répandras le reste du sang sur l'autel tout autour.
\VS{21}Et tu prendras du sang qui sera sur l'autel, de l'huile d'onction, et tu en feras l'aspersion sur Aaron, sur ses vêtements, sur ses fils, et sur les vêtements de ses fils avec lui. Ainsi, lui, ses vêtements, ses fils, et les vêtements de ses fils, seront sanctifiés avec lui.
\VS{22}Tu prendras aussi la graisse du bélier, la queue, et la graisse qui couvre les entrailles, le grand lobe du foie, les deux rognons, la graisse qui est dessus, et l'épaule droite~; car c'est le bélier de consécration.
\VS{23}Tu prendras aussi un pain, un gâteau à l'huile, et un beignet dans la corbeille où seront ces choses sans levain, laquelle sera devant Yahweh.
\VS{24}Et tu mettras toutes ces choses sur les mains d'Aaron et sur les mains de ses fils, et tu les agiteras de côté et d'autre devant Yahweh\FTNT{No. 6:19.}.
\VS{25}Puis, les recevant de leurs mains, tu les feras fumer sur l'autel, sur l'holocauste, pour être une odeur agréable devant Yahweh~; c'est un sacrifice consumé par le feu à Yahweh.
\TextTitle{La part des prêtres}
\VS{26}Tu prendras aussi la poitrine du bélier des consécrations, qui est pour Aaron, et tu l'agiteras de côté et d'autre en offrande agitée devant Yahweh. Ce sera ta part.
\VS{27}Tu sanctifieras donc la poitrine de l'offrande agitée, et l'épaule de l'offrande élevée, tant ce qui aura été agité que ce qui aura été élevé du bélier de consécration, de ce qui est pour Aaron et de ce qui est pour ses fils. \FTNT{Lé. 10:14~; No. 18:18.}.
\VS{28}Et ceci sera une ordonnance perpétuelle pour Aaron et pour ses fils, de ce qui sera offert par les enfants d'Israël~; car c'est une offrande élevée. Quand il y aura une offrande élevée de celles qui sont faites par les enfants d'Israël, de leurs offrandes de paix, leur offrande élevée sera à Yahweh.
\VS{29}Et les saints vêtements qui seront pour Aaron, seront pour ses fils après lui, afin qu'ils soient oints et consacrés dans ces vêtements.
\VS{30}Le prêtre qui succédera à sa place d'entre ses fils, et qui viendra à la tente d'assignation, pour faire le service dans le lieu saint, en sera revêtu durant sept jours.
\VS{31}Or tu prendras le bélier des consécrations, et tu feras bouillir sa chair dans un lieu saint~;
\VS{32}et Aaron et ses fils mangeront à l'entrée de la tente d'assignation la chair du bélier et le pain qui sera dans la corbeille.
\VS{33}Ils mangeront donc ces choses, par lesquelles la propitiation aura été faite, pour les consacrer et les sanctifier~; mais l'étranger n'en mangera point, parce qu'elles sont saintes.
\VS{34}S'il y a des restes de la chair des consécrations et du pain jusqu'au matin, tu brûleras ces restes-là au feu~; on n'en mangera point, parce que c'est une chose sainte.
\VS{35}Tu feras donc ainsi à Aaron et à ses fils, selon toutes les choses que je t'ai ordonnées~; tu les consacreras durant sept jours\FTNT{Lé. 8:31-35.}.
\VS{36}Et tu offriras comme sacrifice pour l'expiation tous les jours un veau pour faire l'expiation, et tu purifieras l'autel par cette propitiation, et tu l'oindras pour le sanctifier\FTNT{Ez. 43:19-20.}.
\VS{37}Pendant sept jours, tu feras propitiation pour l'autel, et tu le sanctifieras~; l'autel sera une chose très sainte~; tout ce qui touchera l'autel sera saint\FTNT{No. 28:3.}.
\TextTitle{L'holocauste perpétuel}
\VS{38}Or c'est ici ce que tu feras sur l'autel~: Tu offriras chaque jour continuellement deux agneaux d'un an.
\VS{39}Tu sacrifieras l'un des agneaux au matin, et l'autre agneau entre les deux soirs~;
\VS{40}avec un dixième de fine farine pétrie dans la quatrième partie d'un hin d'huile vierge, et avec une libation de vin de la quatrième partie d'un hin pour chaque agneau,
\VS{41}et tu sacrifieras l'autre agneau entre les deux soirs, avec un gâteau comme au matin, et tu lui feras la même libation, en bonne odeur~; c'est un sacrifice consumé par le feu à Yahweh.
\VS{42}Ce sera l'holocauste perpétuel qui sera offert en vos générations, à l'entrée de la tente d'assignation, devant Yahweh, où je me trouverai avec vous pour te parler.
\VS{43}Je me trouverai là pour les enfants d'Israël, et la tente sera sanctifiée par ma gloire.
\VS{44}Je sanctifierai donc la tente d'assignation et l'autel. Je sanctifierai aussi Aaron et ses fils, afin qu'ils exercent la prêtrise pour moi.
\VS{45} Et j'habiterai au milieu des enfants d'Israël, et je serai leur Dieu.
\VS{46}Et ils sauront que je suis Yahweh, leur Dieu, qui les ai tirés du pays d'Egypte, pour habiter au milieu d'eux. Je suis Yahweh leur Dieu.
\Chap{30}
\TextTitle{L'autel des parfums}
\VerseOne{}Tu feras aussi un autel pour les parfums, et tu le feras de bois d'acacia.
\VS{2}Sa longueur sera d'une coudée, et sa largeur d'une coudée~; il sera carré~; mais sa hauteur sera de deux coudées, et ses cornes seront tirées de lui.
\VS{3}Tu le couvriras d'or pur, tant le dessus, que ses côtés tout autour, et ses cornes. Et tu lui feras un couronnement d'or tout autour.
\VS{4}Tu lui feras aussi deux anneaux d'or au-dessous de son couronnement, à ses deux côtés, lesquels tu mettras aux deux coins, pour y faire passer les barres qui serviront à le porter.
\VS{5}Tu feras les barres de bois d'acacia, et tu les couvriras d'or.
\VS{6}Et tu les mettras devant le voile, qui est au-devant de l'arche du témoignage, à l'endroit du propitiatoire qui est sur le témoignage, où je me trouverai avec toi.
\VS{7}Et Aaron fera sur cet autel un parfum de choses aromatiques~; il y fera un parfum chaque matin, quand il préparera les lampes.
\VS{8}Et quand Aaron allumera les lampes entre les deux soirs, il y fera aussi le parfum, à savoir le parfum perpétuel devant Yahweh dans vos générations\FTNT{Ex. 37:25-29~; 2 Ch. 13:11.}.
\VS{9}Vous n'offrirez point sur cet autel aucun parfum étranger, ni d'holocauste, ni d'offrande, et vous n'y répandrez aucune libation.
\VS{10}Mais Aaron fera une fois l'an la propitiation sur les cornes de cet autel~; il fera, dis-je, la propitiation une fois l'an sur cet autel dans vos générations, avec le sang de l'offrande pour l'expiation faite pour les propitiations. C'est une chose très sainte à Yahweh.
\TextTitle{L'offrande du rachat\FTNTT{Ex. 15:1-21~; Ps. 107:1-2.}}
\VS{11}Yahweh parla aussi à Moïse, et lui dit~:
\VS{12}Quand tu feras le dénombrement des fils d'Israël, selon leur nombre, ils donneront chacun à Yahweh le rachat de sa personne, quand tu en feras le dénombrement, et il n'y aura point de plaie sur eux quand tu en feras le dénombrement\FTNT{No. 1:2.}.
\VS{13}Tous ceux qui passeront par le dénombrement donneront un demi-sicle, selon le sicle du sanctuaire, qui est de vingt guéras~; le demi-sicle donc sera l'offrande que l'on donnera à Yahweh\FTNT{Lé. 27:25~; No. 3:47~; Ez. 45:12.}.
\VS{14}Tous ceux qui passeront par le dénombrement, depuis l'âge de vingt ans et au-dessus, donneront cette offrande à Yahweh.
\VS{15}Le riche n'augmentera rien, et le pauvre ne diminuera rien du demi-sicle, quand ils donneront à Yahweh l'offrande pour faire le rachat de vos personnes.
\VS{16}Tu prendras donc des enfants d'Israël l'argent des expiations, et tu l'appliqueras à l'œuvre de la tente d'assignation. Ce sera pour les fils d'Israël, un souvenir devant Yahweh pour faire le rachat de vos personnes.
\TextTitle{Purification par l'eau de la cuve d'airain\FTNTT{Jn. 13:3-10~; Hé. 10:22~; 1 Jn. 1:9.}}
\VS{17}Yahweh parla encore à Moïse, en disant~:
\VS{18}Fais aussi une cuve d'airain, avec sa base d'airain, pour laver. Et tu la mettras entre la tente d'assignation et l'autel, et tu mettras de l'eau dedans~;
\VS{19}Aaron et ses fils y laveront leurs mains et leurs pieds.
\VS{20}Quand ils entreront dans la tente d'assignation ils se laveront avec de l'eau, afin qu'ils ne meurent point, et quand ils approcheront de l'autel pour faire le service, afin de faire fumer l'offrande consumée par le feu à Yahweh.
\VS{21}Ils laveront donc leurs pieds et leurs mains, afin qu'ils ne meurent point~; ce leur sera une ordonnance perpétuelle, tant pour Aaron que pour ses fils, en leur génération.
\TextTitle{L'huile pour l'onction sainte\FTNTT{Jn. 4:23~; Ep. 2:18, 5:18-19.}}
\VS{22}Yahweh parla aussi à Moïse, en disant~:
\VS{23}Prends des choses aromatiques les plus exquises~; de la myrrhe franche le poids de cinq cent sicles, la moitié de cinnamome odoriférant, c'est-à-dire, le poids de deux cent cinquante sicles, et du roseau aromatique deux cent cinquante sicles.
\VS{24}De la casse le poids de cinq cent sicles, selon le sicle du sanctuaire, et un hin d'huile d'olive.
\VS{25}Et tu en feras de l'huile pour l'onction sainte, un onguent composé selon l'art du parfumeur, ce sera l'huile de l'onction sainte.
\VS{26}Puis tu en oindras la tente d'assignation, et l'arche du témoignage.
\VS{27}La table et tous ses ustensiles, le chandelier et ses ustensiles, et l'autel du parfum,
\VS{28} et l'autel des holocaustes et tous ses ustensiles, la cuve et sa base.
\VS{29}Ainsi, tu les sanctifieras, et ils seront une chose très-sainte~; tout ce qui les touchera sera saint.
\VS{30}Tu oindras aussi Aaron et ses fils, et les sanctifieras pour m'exercer la prêtrise.
\VS{31}Tu parleras aussi aux enfants d'Israël, en disant~: Ce me sera une huile d'onction sainte dans toutes vos générations.
\VS{32}On n'en oindra point la chair d'aucun homme, et vous n'en ferez point d'autre de même composition~; elle est sainte, elle vous sera sainte.
\VS{33}Quiconque composera un onguent semblable, et qui en mettra sur un autre, sera retranché de ses peuples.
\TextTitle{L'encens pur parfumé}
\VS{34}Yahweh dit aussi à Moïse~: Prends des aromates, à savoir de la gomme, de l'ongle odorant, du galbanum, le tout préparé, et de l'encens pur, le tout à poids égal.
\VS{35}Et tu en feras un parfum aromatique selon l'art du parfumeur, et tu y mettras du sel~; vous le ferez pur, et ce sera pour vous une chose sainte.
\VS{36}Et quand tu l'auras pilé bien menu, tu en mettras dans la tente d'assignation devant le témoignage, où je me trouverai avec toi. Ce sera pour vous une chose très sainte.
\VS{37}Quant au parfum que tu feras, vous ne ferez point pour vous de semblable composition~; ce sera une chose sainte pour Yahweh.
\VS{38}Quiconque en fera un semblable pour le sentir sera retranché de ses peuples.
\Chap{31}
\TextTitle{Yahweh suscite des artisans}
\VerseOne{}Yahweh parla aussi à Moïse, en disant~:
\VS{2}Regarde, j'ai appelé par son nom Betsaleel, fils d'Uri, fils de Hur, de la tribu de Juda.
\VS{3}Et je l'ai rempli de l'Esprit de Dieu, de sagesse, d'intelligence et de science pour toutes sortes d'ouvrages,
\VS{4}afin d'inventer des dessins pour travailler  l'or, l'argent et l'airain~;
\VS{5}dans la sculpture des pierres précieuses, pour les mettre en œuvre, et dans la menuiserie pour travailler dans toutes sortes d'ouvrages.
\VS{6}Et voici, je lui ai donné pour compagnon Oholiab, fils d'Ahisamac, de la tribu de Dan~; et au cœur de tout homme sage, j'ai mis de l'intelligence, afin qu'ils fassent toutes les choses que je t'ai ordonnées,
\VS{7}à savoir la tente d'assignation, l'arche du témoignage, et le propitiatoire qui doit être au-dessus, et tous les ustensiles du tabernacle~;
\VS{8}et la table avec tous ses ustensiles~; et le chandelier pur avec tous ses ustensiles~; et l'autel du parfum~;
\VS{9}et l'autel de l'holocauste avec tous ses ustensiles, la cuve et sa base~;
\VS{10}et les vêtements du service~; les saints vêtements pour le prêtre Aaron, et les vêtements de ses fils pour exercer la prêtrise~;
\VS{11}et l'huile d'onction, et le parfum des choses aromatiques pour le sanctuaire, et ils feront toutes les choses que je t'ai ordonnées.
\TextTitle{Le sabbat comme signe entre Yahweh et Israël}
\VS{12}Yahweh parla encore à Moïse, en disant~:
\VS{13}Toi aussi parle aux enfants d'Israël, en disant~: Certes, vous garderez mes sabbats, car c'est un signe entre moi et vous, et parmi vos générations, afin que vous sachiez que je suis Yahweh qui vous sanctifie.
\VS{14}Gardez donc le sabbat, car il doit vous être saint. Quiconque le violera sera puni de mort~; quiconque,dis-je, fera une œuvre en ce jour-là sera retranché du milieu de son peuple.
\VS{15}On travaillera six jours, mais le septième jour est le sabbat du repos, consacré à Yahweh~; quiconque fera une œuvre le jour du repos sera puni de mort.
\VS{16}Ainsi, les enfants d'Israël garderont le sabbat pour célébrer le jour du repos, en leur génération, par une alliance perpétuelle.
\VS{17}C'est un signe entre moi et les enfants d'Israël à perpétuité~; car Yahweh a fait en six jours les cieux et la terre, et il a cessé au septième, et s'est reposé\FTNT{Ge. 2:2~; Ez. 20:12.}.
\VS{18}Et Dieu donna à Moïse, après qu'il eut achevé de parler avec lui sur la montagne de Sinaï, les deux tables du témoignage~; tables de pierre, écrites du doigt de Dieu\FTNT{De. 9:10.}.
\Chap{32}
\TextTitle{Le culte du veau d'or}
\VerseOne{}Mais le peuple, voyant que Moïse tardait tant à descendre de la montagne, s'assembla autour d'Aaron, et lui dit~: Lève-toi, fais-nous des dieux qui marchent devant nous, car quant à ce Moïse, cet homme qui nous a fait monter du pays d'Egypte, nous ne savons ce qui lui est arrivé\FTNT{Ac. 7:40.}.
\VS{2}Et Aaron leur répondit~: Mettez en pièces les anneaux d'or qui sont aux oreilles de vos femmes, de vos fils, et de vos filles, et apportez-les-moi\FTNT{Ex. 35:22.}.
\VS{3} Et aussitôt, tout le peuple mit en pièces les anneaux d'or qui étaient à leurs oreilles, et ils les apportèrent à Aaron, 
\VS{4}qui les ayant reçu de leurs mains, forma l'or avec un burin, et en fit un veau\FTNT{Selon toute vraisemblance, les Israélites s'étaient inspirés d'une idole égyptienne, le taureau sacré Apis, pour faire le veau d'or. Dieu de la puissance sexuelle, de la fertilité et de la force, il était souvent représenté sous la forme d'un homme avec une tête de taureau, puis avec un disque solaire entre les cornes à partir du Nouvel Empire.} en métal fondu. Et ils dirent~: Ce sont ici tes dieux, ô Israël, qui t'ont fait monter du pays d'Egypte.
\VS{5}Ce qu'Aaron ayant vu, il bâtit un autel devant le veau, et cria en disant~: Demain, il y aura une fête solennelle à Yahweh.
\VS{6}Ainsi, ils se levèrent le lendemain, dès le matin, et ils offrirent des holocaustes, et présentèrent des offrandes de paix. Et le peuple s'assit pour manger et pour boire, puis ils se levèrent pour jouer\FTNT{1 Co. 10:7.}.
\TextTitle{Yahweh condamne l'idolâtrie d'Israël}
\VS{7}Alors Yahweh dit à Moïse~: Va, descends, car ton peuple que tu as fait monter du pays d'Egypte s'est corrompu\FTNT{De. 32:5.}.
\VS{8}Ils se sont promptement détournés de la voie que je leur avais ordonnée, et ils se sont fait un veau en métal fondu, et se sont prosternés devant lui, ils lui ont offert des sacrifices, puis ont dit~: Ce sont ici tes dieux, ô Israël, qui t'ont fait monter du pays d'Egypte\FTNT{1 R. 12:28.}.
\VS{9}Yahweh dit encore à Moïse~: J'ai regardé ce peuple, et voici, c'est un peuple au cou raide.
\VS{10}Maintenant laisse-moi, et ma colère s'embrasera contre eux, et je les consumerai~; mais je te ferai devenir une grande nation.
\TextTitle{Moïse implore Yahweh pour le peuple}
\VS{11}Alors Moïse supplia Yahweh, son Dieu, et dit~: Ô Yahweh, pourquoi ta colère s'embraserait-elle contre ton peuple, que tu as retiré du pays d'Egypte par une grande puissance et par une main forte\FTNT{Ps. 106:23.}~?
\VS{12}Pourquoi les Egyptiens diraient~: Il les a retirés dans de mauvaises vues, pour les tuer sur les montagnes, et pour les consumer de dessus la terre~? Reviens de l'ardeur de ta colère, et repens-toi de ce mal que tu veux faire à ton peuple\FTNT{No. 14:11-15~; De. 9:28.}.
\VS{13}Souviens-toi d'Abraham, d'Isaac et d'Israël, tes serviteurs, auxquels tu as juré par toi-même en leur disant~: Je multiplierai votre postérité comme les étoiles des cieux, et je donnerai à votre postérité tout ce pays, dont j'ai parlé, et ils l'hériteront à jamais\FTNT{De. 34:4.}.
\VS{14}Et Yahweh se repentit du mal qu'il avait dit qu'il ferait à son peuple.
\TextTitle{Jugement sur le peuple}
\VS{15}Moïse regarda, et descendit de la montagne, ayant dans sa main les deux tables du témoignage, et les tables étaient écrites des deux côtés, écrites de l'un et de l'autre côté.
\VS{16}Et les tables étaient l'ouvrage de Dieu, et l'écriture était l'écriture de Dieu, gravée sur les tables.
\VS{17}Et Josué, entendant la voix du peuple qui faisait un grand bruit, dit à Moïse~: Il y a un bruit de bataille au camp.
\VS{18}Et Moïse lui répondit~: Ce n'est pas une voix ni un cri de gens qui soient les plus forts, ni une voix ni un cri de gens qui soient les plus faibles~; mais j'entends une voix de gens qui chantent.
\VS{19}Et il arriva que lorsque Moïse fut approché du camp, il vit le veau et les danses, et la colère de Moïse s'embrasa~; et il jeta de ses mains les tables, et les rompit au pied de la montagne.
\VS{20}Il prit ensuite le veau qu'ils avaient fait, et le brûla au feu, et le moulut jusqu'à ce qu'il fut en poudre, puis il répandit cette poudre dans de l'eau, et il en fit boire aux enfants d'Israël\FTNT{De. 9:17-21.}.
\VS{21}Et Moïse dit à Aaron~: Que t'a fait ce peuple pour que tu aies fait venir sur lui un si grand péché~?
\VS{22}Et Aaron lui répondit~: Que la colère de mon seigneur ne s'embrase point, tu sais que ce peuple est porté au mal.
\VS{23}Ils m'ont dit~: Fais-nous un dieu qui marche devant nous, car ce Moïse, cet homme qui nous a fait monter du pays d'Egypte, nous ne savons ce qui lui est arrivé.
\VS{24}Alors je leur ai dit~: Que celui qui a de l'or, le mette en pièces~! Et ils me l'ont donné~; et je l'ai jeté au feu, et ce veau en est sorti.
\VS{25}Or Moïse vit que le peuple était dénudé, car Aaron l'avait dénudé pour être en opprobre parmi leurs ennemis.
\VS{26}Et Moïse se tenant à la porte du camp, dit~: Qui est pour Yahweh~? Qu'il vienne vers moi~! Et tous les fils de Lévi s'assemblèrent vers lui.
\VS{27}Et il leur dit~: Ainsi parle Yahweh, le Dieu d'Israël~: Que chacun mette son épée à son côté, passez et repassez de porte en porte par le camp, et que chacun de vous tue son frère, son ami, et son voisin.
\VS{28}Et les fils de Lévi firent selon la parole de Moïse~; et ce jour-là il tomba parmi le peuple environ trois mille hommes.
\VS{29}Car Moïse avait dit~: Consacrez aujourd'hui vos mains à Yahweh, chacun même contre son fils, et contre son frère, afin que vous attiriez aujourd'hui sur vous la bénédiction.
\TextTitle{Moïse intercède pour Israël}
\VS{30}Et le lendemain, Moïse dit au peuple~: Vous avez commis un grand péché~; mais je monterai vers Yahweh, et peut-être je ferais propitiation pour votre péché.
\VS{31}Moïse donc retourna vers Yahweh et dit~: Hélas~! Je te prie, ce peuple a commis un grand péché, en se faisant des dieux d'or.
\VS{32}Maintenant pardonne leur péché~! Sinon, efface-moi maintenant de ton livre que tu as écrit.
\VS{33}Et Yahweh répondit à Moïse~: C'est celui qui aura péché contre moi que j'effacerai de mon livre\FTNT{Ap. 3:5~; Ap. 20:15~; Ap. 21:27.}.
\VS{34}Va maintenant, conduis le peuple au lieu duquel je t'ai parlé. Voici, mon Ange ira devant toi~; et le jour où je ferai punition, je punirai sur eux leur péché.
\VS{35} Ainsi, Yahweh frappa le peuple, parce qu'ils avaient été les auteurs du veau qu'Aaron avait fait.
\Chap{33}
\TextTitle{Yahweh ne veut plus marcher avec Israël}
\VerseOne{}Yahweh donc dit à Moïse~: Va, monte d'ici, toi et le peuple que tu as fait monter du pays d'Egypte, au pays que j'ai juré de donner à Abraham, à Isaac, et à Jacob, en disant~: Je le donnerai à ta postérité.
\VS{2}Et j'enverrai un Ange devant toi, et je chasserai les Cananéens, les Amoréens, les Héthiens, les Phéréziens, les Héviens et les Jébusiens,
\VS{3}pour vous conduire au pays découlant de lait et de miel, mais je ne monterai point au milieu de toi, parce que tu es un peuple au cou raide, de peur que je ne te consume en chemin.
\VS{4}Et le peuple entendit ces tristes nouvelles, et en mena le deuil, et aucun d'eux ne mit ses ornements sur soi.
\VS{5}Car Yahweh avait dit à Moïse~: Dis aux enfants d'Israël~: Vous êtes un peuple au cou raide~; je monterai en un moment au milieu de toi, et je te consumerai. Maintenant donc ôte tes ornements de dessus toi, et je saurai ce que je te ferai.
\VS{6}Ainsi, les enfants d'Israël se dépouillèrent de leurs ornements vers la montagne d'Horeb.
\TextTitle{Moïse dresse la tente d'assignation hors du camp}
\VS{7}Et Moïse prit une tente, et la tendit pour soi hors du camp, l'éloignant du camp~; et il l'appela la tente d'assignation~; et tous ceux qui cherchaient Yahweh sortaient vers la tente d'assignation qui était hors du camp.
\VS{8}Et il arrivait qu'aussitôt que Moïse sortait vers la tente, tout le peuple se levait, et chacun se tenait à l'entrée de sa tente, et regardait Moïse par-derrière, jusqu'à ce qu'il soit entré dans la tente.
\VS{9}Et sitôt que Moïse était entré dans la tente, la colonne de nuée descendait et s'arrêtait à la porte de la tente, et Yahweh parlait avec Moïse.
\VS{10}Et tout le peuple voyant la colonne de nuée s'arrêtant à la porte de la tente se levait, et chacun se prosternait à la porte de sa tente.
\VS{11}Et Yahweh parlait à Moïse face à face, comme un homme parle avec son intime ami. Puis Moïse retournait au camp, mais son serviteur Josué, fils de Nun, jeune homme, ne bougeait point de la tente\FTNT{No. 12:8~; De. 34:10~; Jn. 15:14-15.}.
\TextTitle{Moïse demande que Yahweh marche avec Israël}
\VS{12}Moïse donc dit à Yahweh~: Regarde, tu m'as dit~: Fais monter ce peuple, et tu ne m'as point fait connaître celui que tu dois envoyer avec moi~; tu as même dit~: Je te connais par ton nom, et aussi, tu as trouvé grâce devant mes yeux.
\VS{13}Or maintenant, je te prie, si j'ai trouvé grâce devant tes yeux, fais-moi connaître ton chemin, et je te connaîtrai, afin que je trouve grâce devant tes yeux~; considère aussi que cette nation est ton peuple\FTNT{Ps. 25:4.}.
\VS{14}Et Yahweh dit~: Ma face ira, et je te donnerai du repos.
\VS{15}Et Moïse lui dit~: Si ta face ne vient, ne nous fais point monter d'ici.
\VS{16}Car en quoi connaîtra-t-on que nous avons trouvé grace devant tes yeux, moi et ton peuple~? Ne sera-ce pas quand tu marcheras avec nous~? Et alors, moi et ton peuple serons en admiration plus que tous les peuples qui sont sur la terre~?
\VS{17}Et Yahweh dit à Moïse~: Je ferai aussi ce que tu dis~; car tu as trouvé grâce devant mes yeux, et je te connais par ton nom.
\TextTitle{Moïse veut voir la gloire de Yahweh}
\VS{18}Moïse dit aussi~: Je te prie, fais-moi voir ta gloire~!
\VS{19}Et Dieu dit~: Je ferai passer toute ma bonté devant ta face, et je crierai le Nom de Yahweh devant toi~; et je ferai grâce à qui je ferai grâce, et j'aurai compassion de celui de qui j'aurai compassion\FTNT{Ro. 9:15.}.
\VS{20}Puis il dit~: Tu ne pourras pas voir ma face, car nul homme ne peut me voir et vivre\FTNT{Jn. 1:18~; Jn. 14:8-11.}.
\VS{21}Yahweh dit aussi~: Voici, il y a un lieu près de moi, et tu t'arrêteras sur le rocher\FTNT{Le rocher préfigurait Jésus-Christ, le Roc sur lequel nous devons bâtir nos vies et le fondement de l'Eglise (Ps. 18:32~; Mt. 7:24-25~; Mt. 16:18~; 1 Co. 3:11). Voir commentaire Es. 8:13-17.}.
\VS{22}Et quand ma gloire passera, je te mettrai dans un creux du rocher, et te couvrirai de ma main, jusqu'à ce que je sois passé.
\VS{23}Puis je retirerai ma main, et tu me verras par-derrière, mais ma face ne se verra point.
\Chap{34}
\TextTitle{De nouvelles tables~; la gloire de Yahweh\FTNTT{Ex. 33:18-23.}}
\VerseOne{}Et Yahweh dit à Moïse~: Aplanis-toi deux tables de pierre comme les premières, et j'écrirai sur elles les paroles qui étaient sur les premières tables que tu as rompues\FTNT{De. 10:1.}.
\VS{2}Et sois prêt au matin, et monte au matin sur la montagne de Sinaï, et présente-toi là devant moi sur le haut de la montagne.
\VS{3}Mais que personne ne monte avec toi, et même que personne ne paraisse sur toute la montagne~; et que ni menu ni gros bétail ne paisse sur cette montagne\FTNT{Ex. 19:12-13.}.
\VS{4}Moïse donc aplanit deux tables de pierre comme les premières, et se leva de bon matin, et monta sur la montagne de Sinaï, comme Yahweh le lui avait ordonné, et il prit dans sa main les deux tables de pierre.
\VS{5}Et Yahweh descendit dans la nuée, et s'arrêta là avec lui, et cria le Nom de Yahweh.
\VS{6}Comme donc Yahweh passait par devant lui, il cria~: Yahweh, Yahweh~! Le Dieu compatissant, miséricordieux, lent à la colère, abondant en bonté et en fidélité\FTNT{No. 14:18~; 2 Ch. 30:9~; Né. 9:17~; Ps. 103:8.}.
\VS{7}Qui conserve sa bonté jusqu'à mille générations, ôtant l'iniquité, le crime, et le péché, qui ne tient point le coupable pour innocent, et qui punit l'iniquité des pères sur les fils, et sur les fils des fils, jusqu'à la troisième et à la quatrième génération\FTNT{Ex. 20:6~; De. 5:10~; Jé. 32:18.}.
\VS{8}Et Moïse se hatant, baissa la tête contre terre et se prosterna. 
\VS{9}Et il dit~: Ô Seigneur~! Je te prie, si j'ai trouvé grâce à tes yeux, que le Seigneur marche maintenant au milieu de nous, car c'est un peuple au cou raide. Pardonne donc nos iniquités et notre péché, et prends-nous pour ta possession.
\TextTitle{Yahweh renouvelle ses promesses\FTNTT{Ex. 33:18-23.}}
\VS{10}Et il répondit~: Voici, je traite alliance devant tout ton peuple, je ferai des merveilles qui n'ont point été faites sur toute la terre ni dans aucune nation. Et tout le peuple au milieu duquel tu es, verra l'œuvre de Yahweh, car ce que je m'en vais faire avec toi sera une chose redoutable.
\VS{11}Garde soigneusement ce que je t'ordonne aujourd'hui. Voici, je m'en vais chasser devant toi les Amoréens, les Cananéens, les Héthiens, les Phéréziens, les Héviens, et les Jébusiens.
\VS{12}Garde-toi de traiter alliance avec les habitants du pays où tu dois entrer, de peur que peut-être ils ne soient un piège pour toi\FTNT{De. 7:2~; Jos. 23:12-13~; 2 Co. 6:14.}.
\VS{13}Mais vous démolirez leurs autels, vous briserez leurs statues, et vous couperez leurs emblèmes d'Asherah\FTNT{Cité au moins quarante fois dans le Tanakh, le terme hébreu «~Asherah~» fait référence à un «~arbre sacré, un pieu près d'un autel~» ou encore «~une idole~», terme par lequel il est majoritairement traduit. Il s'agit également de l'objet en bois utilisé dans le culte de la parèdre de Baal. Manassé, roi de Juda, introduisit l'emblème d'Asherah dans le temple (2 R. 21:1-7) en dépit de l'interdiction formelle de Yahweh (De. 16:21). Il n'en fut enlevé que lors des réformes de Josias et d'Ezéchias (2 R. 18:3-4~; 2 R. 23:6-14). Pourtant, Yahweh a toujours exigé la destruction de celle qu'il a nommé «~l'abomination des Sidoniens~», de peur que son peuple y trouve une occasion de chute (Jg. 2:13~; Jg. 10:6~; 1 S. 31:10~; 1 R. 11:5-33~; 2 R. 23:13). Bien qu'étant majoritairement citée dans les Ecritures en tant qu'objet de culte, Asherah est également associée à la divinité Astarté, connue pour être la Diane des Ephésiens (Ac. 19:23-40), la reine du ciel (Jé. 7:18~; Jé. 44:15-30), l'Isis des Egyptiens et l'épouse de Baal (voir commentaire en Jg. 2:13).}.
\VS{14}Car tu ne te prosterneras point devant un autre dieu, parce que Yahweh se nomme le Dieu jaloux~; c'est le Dieu qui est jaloux.
\VS{15}Afin qu'il n'arrive que tu traites alliance avec les habitants du pays, et que quand ils viendront à se prostituer après leurs dieux et à sacrifier à leurs dieux, quelqu'un ne t'invite et que tu ne manges de leurs sacrifices~;
\VS{16}et que tu ne prennes de leurs filles pour tes fils, lesquelles se prostituant après leurs dieux, n'entraînent tes fils à se prostituer après leurs dieux.
\VS{17}Tu ne te feras aucun dieu de métal fondu.
\TextTitle{Les fêtes et le sabbat\FTNTT{Lé. 23:4-44.}}
\VS{18}Tu garderas la fête solennelle des pains sans levain~; tu mangeras les pains sans levain pendant sept jours, comme je te l'ai ordonné, dans la saison où les épis mûrissent~; car c'est dans le mois des épis que tu es sorti du pays d'Egypte.
\VS{19}Tout premier-né sera à moi~; même le premier mâle qui naîtra de toutes les bêtes, tant du gros que du menu bétail.
\VS{20}Mais tu rachèteras avec un agneau ou un chevreau le premier-né d'un âne. Si tu ne le rachètes pas, tu lui couperas le cou. Tu rachèteras tout premier-né de tes fils~; et nul ne se présentera devant ma face à vide.
\VS{21}Tu travailleras six jours, mais au septième tu te reposeras~; tu te reposeras au temps du labourage et de la moisson.
\VS{22}Tu feras la fête solennelle des semaines au temps des premiers fruits de la moisson du froment~; et la fête solennelle de la récolte à la fin de l'année.
\VS{23}Trois fois l'an, tout mâle d'entre vous comparaîtra devant le Seigneur Yahweh, le Dieu d'Israël.
\VS{24}Car je déposséderai les nations de devant toi, et j'étendrai tes limites, et nul ne convoitera ton pays lorsque tu monteras pour comparaître trois fois l'an devant Yahweh, ton Dieu.
\VS{25}Tu n'offriras point le sang de mon sacrifice avec du pain levé~; on ne gardera rien du sacrifice de la fête solennelle de la Pâque jusqu'au matin.
\VS{26}Tu apporteras les prémices des premiers fruits de la terre dans la maison de Yahweh, ton Dieu. Tu ne feras point cuire le chevreau dans le lait de sa mère.
\VS{27}Yahweh dit aussi à Moïse~: Ecris ces paroles, car suivant la teneur de ces paroles, j'ai traité alliance avec toi et avec Israël.
\VS{28}Et Moïse demeura là avec Yahweh quarante jours et quarante nuits, sans manger de pain et sans boire d'eau~; et Yahweh écrivit sur les tables les paroles de l'alliance, c'est-à-dire les dix paroles.
\TextTitle{La gloire de Yahweh sur le visage de Moïse}
\VS{29}Or il arriva que lorsque Moïse descendait de la montagne de Sinaï, tenant dans sa main les deux tables du témoignage, lorsque, dis-je, il descendait de la montagne, il ne s'aperçut point que la peau de son visage était devenue rayonnante pendant qu'il parlait avec Dieu.
\VS{30}Mais Aaron et tous les enfants d'Israël ayant vu Moïse, et s'étant aperçus que la peau de son visage était rayonnante, ils craignirent de s'approcher de lui.
\VS{31}Mais Moïse les appela, et Aaron et tous les principaux de l'assemblée retournèrent vers lui~; et Moïse parla avec eux.
\VS{32}Après quoi, tous les enfants d'Israël s'approchèrent, et il leur ordonna toutes les choses que Yahweh lui avait dites sur la montagne de Sinaï.
\VS{33}Ainsi, Moïse acheva de leur parler~; or il avait mis un voile sur son visage.
\VS{34}Et quand Moïse entrait vers Yahweh pour parler avec lui, il ôtait le voile jusqu'à ce qu'il sorte~; et étant sorti, il disait aux enfants d'Israël ce qui lui avait été ordonné.
\VS{35}Or les enfants d'Israël avaient vu que le visage de Moïse, la peau, dis-je, de son visage rayonnait. C'est pourquoi Moïse remettait le voile sur son visage, jusqu'à ce qu'il entre pour parler avec Yahweh.
\Chap{35}
\TextTitle{Rappels sur le sabbat}
\VerseOne{}Moïse donc assembla toute la congrégation des enfants d'Israël, et leur dit~: Ce sont ici les choses que Yahweh a ordonnées de faire.
\VS{2}On travaillera six jours, mais le septième jour il y aura sainteté pour vous, car c'est le sabbat du repos consacré à Yahweh~; quiconque travaillera en ce jour-là sera puni de mort.
\VS{3}Vous n'allumerez point de feu dans aucune de vos demeures le jour du repos.
\TextTitle{Les offrandes pour le tabernacle\FTNTT{Ex. 25:1-8.}}
\VS{4}Puis Moïse parla à toute l'assemblée des enfants d'Israël, et leur dit~: C'est ici ce que Yahweh vous a ordonné, en disant~:
\VS{5}Prenez des choses qui sont chez vous une offrande pour Yahweh. Quiconque sera de bonne volonté, apportera cette offrande pour Yahweh, à savoir de l'or, de l'argent, de l'airain\FTNT{Ex. 25:2~; 2 Co. 8:12.},
\VS{6}de la pourpre, de l'écarlate, du cramoisi, du fin lin, du poil de chèvre,
\VS{7}des peaux de béliers teintes en rouge, des peaux de taissons, du bois d'acacia,
\VS{8}de l'huile pour le chandelier, des aromates pour l'huile d'onction et pour le parfum odoriférant,
\VS{9}des pierres d'onyx, et des pierres pour la garniture de l'éphod et pour le pectoral.
\VS{10}Et tous les hommes d'esprit d'entre vous viendront et feront tout ce que Yahweh a ordonné, 
\VS{11}à savoir le tabernacle, sa tente, et sa couverture, ses agrafes, ses planches, ses barres, ses colonnes, et ses bases~;
\VS{12}l'arche et ses barres, le propitiatoire et le voile qui sert de rideau~;
\VS{13}la table et ses barres, et tous ses ustensiles, et le pain de proposition~;
\VS{14}et le chandelier du luminaire, ses ustensiles, ses lampes et l'huile du luminaire~;
\VS{15}l'autel du parfum et ses barres~; l'huile d'onction, le parfum odoriférant, le rideau de la porte pour l'entrée du tabernacle~;
\VS{16}l'autel de l'holocauste, sa grille d'airain, ses barres et tous ses ustensiles~; la cuve avec sa base~;
\VS{17}les courtines du parvis, ses colonnes, ses bases, et le rideau de la porte du parvis~;
\VS{18}les pieux du tabernacle, les pieux du parvis et leur cordage~;
\VS{19}les vêtements du service pour faire le service dans le sanctuaire, les saints vêtements d'Aaron, le prêtre, et les vêtements de ses fils pour exercer la prêtrise.
\VS{20}Alors toute l'assemblée des enfants d'Israël sortit de la présence de Moïse.
\VS{21}Et quiconque fut ému en son cœur, quiconque, dis-je, se sentit porté à la libéralité, apporta l'offrande de Yahweh pour l'ouvrage de la tente d'assignation et pour tout son service et pour les saints vêtements.
\VS{22}Et les hommes vinrent avec les femmes~; quiconque fut de cœur volontaire, apporta des boucles, des bagues, des anneaux, des bracelets, et des joyaux d'or~; et quiconque offrit quelque offrande d'or à Yahweh.
\VS{23}Tout homme aussi chez qui se trouvait de la pourpre, de l'écarlate, du cramoisi, du fin lin, du poil de chèvre, des peaux de béliers teintes en rouge et des peaux de taissons, les apportèrent.
\VS{24}Tout homme qui avait de quoi faire une offrande d'argent et d'airain, l'apporta pour l'offrande de Yahweh~; tout homme aussi chez qui fut trouvé du bois d'acacia pour tout l'ouvrage du service, l'apporta.
\VS{25}Toute femme adroite fila de sa main et apporta ce qu'elle avait filé~: De la pourpre, de l'écarlate, du cramoisi, et du fin lin\FTNT{Pr. 31:19.}.
\VS{26}Toutes les femmes aussi dont le cœur les y porta en sagesse, filèrent du poil de chèvre.
\VS{27}Les principaux aussi de l'assemblée apportèrent des pierres d'onyx, et d'autres pierres pour la garniture de l'éphod et du pectoral~;
\VS{28}et des aromates, et de l'huile tant pour le chandelier que pour l'huile d'onction, et pour le parfum odoriférant.
\VS{29}Tout homme donc et toute femme que le cœur incita à la libéralité pour apporter de quoi faire l'ouvrage que Yahweh avait ordonné par le moyen de Moïse, tous les enfants, dis-je, d'Israël apportèrent volontairement des présents à Yahweh.
\TextTitle{Betsaleel et Oholiab oints pour l'œuvre du tabernacle}
\VS{30}Alors Moïse dit aux fils d'Israël~: Voyez, Yahweh a appelé par son nom Betsaleel, fils d'Uri, fils de Hur, de la tribu de Juda.
\VS{31}Et il l'a rempli de l'Esprit de Dieu, de sagesse, d'intelligence, de science, pour toutes sortes d'ouvrages.
\VS{32}Même afin d'inventer des dessins, pour travailler l'or, l'argent et l'airain~;
\VS{33}dans la sculpture des pierres précieuses pour les mettre en œuvre, et dans la menuiserie pour travailler en tout ouvrage exquis.
\VS{34}Et il lui a mis aussi au cœur, tant à lui qu'à Oholiab, fils d'Ahisamac, de la tribu de Dan, de l'enseigner.
\VS{35}Et il les a remplis de sagesse pour faire toutes sortes d'ouvrages d'ouvrier, même d'ouvrier en ouvrage exquis, et en broderie, en pourpre, en écarlate, en cramoisi, et en fin lin, et d'ouvrage de tisserand, faisant toutes sortes d'ouvrages et inventant toutes sortes de dessins\FTNT{Es. 28:26.}.
\Chap{36}
\TextTitle{Construction du tabernacle d'après le modèle donné par Yahweh\FTNTT{Ex. 36-39.}}
\VerseOne{}Et Betsaleel et Oholiab, et tous les hommes au coeur sage auxquels Yahweh avait donné de la sagesse et de l'intelligence pour savoir faire tout l'ouvrage du service du sanctuaire, firent selon toutes les choses que Yahweh avait ordonnées.
\VS{2}Moïse donc appela Betsaleel et Oholiab, et tous les hommes d'esprit, dans le cœur desquels Yahweh avait mis de la sagesse, et tous ceux qui furent émus en leur cœur de se présenter pour faire cet ouvrage.
\VS{3}Lesquels emportèrent de devant Moïse toute l'offrande que les enfants d'Israël avaient apportée pour faire l'ouvrage du service du sanctuaire. Or on apportait encore chaque matin quelques offrandes volontaires.
\VS{4}C'est pourquoi, tous les hommes sages qui faisaient tout l'ouvrage du sanctuaire, vinrent chacun de l'ouvrage qu'ils faisaient,
\VS{5}et parlèrent à Moïse, en disant~: Le peuple ne cesse d'apporter plus qu'il ne faut pour le service et pour l'ouvrage que Yahweh a ordonné de faire.
\VS{6}Alors, par l'ordre de Moïse, on fit crier dans le camp que ni homme ni femme ne fasse plus d'ouvrage pour l'offrande du sanctuaire~; et ainsi, on empêcha le peuple d'offrir.
\VS{7}Car ils avaient du travail suffisant pour tout l'ouvrage à faire, et il y en avait même de reste.
\TextTitle{Les tapis de fin lin}
\VS{8}Tous les hommes donc au coeur sage d'entre ceux qui faisaient l'ouvrage, firent le tabernacle, à savoir dix tapis de fin lin retors, de pourpre, d'écarlate, et de cramoisi~; et ils les firent semés de chérubins d'un ouvrage exquis.
\VS{9}La longueur d'un tapis était de vingt-huit coudées, et la largeur du même tapis de quatre coudées~; tous les tapis avaient une même mesure\FTNT{Ex. 26:1-6.}.
\VS{10}Et ils joignirent cinq tapis l'un à l'autre, et cinq autres tapis l'un à l'autre.
\VS{11}Et ils firent des lacets de pourpre sur le bord d'un tapis, à savoir au bord de celui qui était attaché~; ils en firent ainsi au bord du dernier tapis dans l'assemblage de l'autre.
\VS{12}Ils firent cinquante lacets à un tapis, et cinquante lacets au bord du tapis qui était dans l'assemblage de l'autre~; les lacets étant vis-à-vis l'un de l'autre.
\VS{13}Puis on fit cinquante agrafes d'or, et on attacha les tapis l'un à l'autre avec les agrafes~; ainsi fut fait le tabernacle.
\TextTitle{Les tapis de poils de chèvres}
\VS{14}Puis on fit des tapis de poils de chèvres, pour servir de tente au-dessus du tabernacle~; on fit onze de ces tapis.
\VS{15}La longueur d'un tapis était de trente coudées, et la largeur du même tapis de quatre coudées~; et les onze tapis étaient d'une même mesure.
\VS{16}Et on assembla cinq de ces tapis à part, et six tapis à part.
\VS{17}On fit aussi cinquante lacets sur le bord de l'un des tapis, à savoir au dernier qui était attaché, et cinquante lacets sur le bord de l'autre tapis, qui était attaché.
\VS{18}On fit aussi cinquante agrafes d'airain pour assembler la tente, afin qu'il n'y en eut qu'une.
\TextTitle{Les couvertures de peaux de béliers et de taissons}
\VS{19}Puis, on fit pour la tente une couverture de peaux de béliers teintes en rouge, et une couverture de peaux de taissons par-dessus.
\TextTitle{Les planches et leurs bases}
\VS{20}Et on fit pour le tabernacle des planches de bois d'acacia, qu'on fit tenir debout.
\VS{21}La longueur d'une planche était de dix coudées, et la largeur de la même planche d'une coudée et demie.
\VS{22}Il y avait deux tenons à chaque planche en façon d'échelons l'un après l'autre~; on fit la même chose pour toutes les planches du tabernacle.
\VS{23}On fit donc les planches pour le tabernacle~; à savoir vingt planches au côté qui regardaient directement vers le sud.
\VS{24}Et au-dessous des vingt planches, on fit quarante bases d'argent, deux bases sous une planche, pour ses deux tenons, et deux bases sous l'autre planche, pour ses deux tenons.
\VS{25}On fit aussi vingt planches pour l'autre côté du tabernacle, du côté nord,
\VS{26}et leurs quarante bases d'argent~: Deux bases sous une planche, et deux bases sous l'autre planche.
\VS{27}Et pour le fond du tabernacle, vers l'occident, on fit six planches.
\VS{28}Et on fit deux planches pour les angles du tabernacle aux deux cotés du fond~;
\VS{29}qui étaient égales par le bas, et qui étaient jointes et unies par le haut avec un anneau~; on fit la même chose aux deux planches qui étaient aux deux angles.
\VS{30}Il y avait donc huit planches et seize bases d'argent, à savoir deux bases sous chaque planche.
\VS{31}Puis on fit cinq barres de bois d'acacia, pour les planches de l'un des côtés du tabernacle~;
\VS{32}et cinq barres pour les planches de l'autre côté du tabernacle~; et cinq barres pour les planches du tabernacle pour le fond, vers le côté de l'occident.
\VS{33}Et on fit que la barre du milieu passait par le milieu des planches d'une extrémité à l'autre.
\VS{34}Et on couvrit d'or les planches, et on fit leurs anneaux d'or pour y faire passer les barres, et on couvrit d'or les barres.
\TextTitle{Le voile et le rideau extérieur}
\VS{35}On fit aussi le voile de pourpre, d'écarlate, de cramoisi, et de fin lin retors~; on le fit d'ouvrage exquis, avec des chérubins.
\VS{36}Et on lui fit quatre piliers de bois d'acacia, qu'on couvrit d'or, ayant leurs crochets d'or~; et on fondit pour eux quatre bases d'argent.
\VS{37}On fit aussi à l'entrée de la tente un rideau de pourpre, d'écarlate, de cramoisi, et de fin lin retors~; d'ouvrage de broderie~;
\VS{38}et ses cinq piliers avec leurs crochets~; et on couvrit d'or leurs chapiteaux et leurs filets~; mais leurs cinq bases étaient d'airain.
\Chap{37}
\TextTitle{L'arche de l'alliance}
\VerseOne{}Puis Betsaleel fit l'arche de bois d'acacia. Sa longueur était de deux coudées et demie, et sa largeur d'une coudée et demie, et sa hauteur d'une coudée et demie\FTNT{Ex. 23:10-31.}.
\VS{2}Et il la couvrit par dedans et par dehors de pur or, et lui fit un couronnement d'or tout autour.
\VS{3}Et il lui fondit pour elle quatre anneaux d'or pour les mettre sur ses quatre coins, à savoir deux anneaux à l'un de ses côtés, et deux autres à l'autre côté.
\VS{4}Et il fit aussi des barres de bois d'acacia, et les couvrit d'or.
\VS{5}Et il fit entrer les barres dans les anneaux aux côtés de l'arche, pour porter l'arche.
\TextTitle{Le propitiatoire}
\VS{6}Il fit aussi le propitiatoire d'or pur~; sa longueur était de deux coudées et demie, et sa largeur d'une coudée et demie.
\VS{7}Et il fit deux chérubins d'or~; il les fit d'ouvrage étendu au marteau, tirés des deux extrémités du propitiatoire~;
\VS{8}à savoir un chérubin tiré de l'une des extrémités et un chérubin tiré de l'autre extrémité~; il fit, dis-je, les chérubins tirés du propitiatoire~; à savoir de ses deux extrémités. 
\VS{9}Et les chérubins étendaient leurs ailes en haut, couvrant de leurs ailes le propitiatoire~; et leurs faces étaient vis-à-vis l'une de l'autre, et les chérubins regardaient vers le propitiatoire.
\TextTitle{La table des pains de proposition}
\VS{10}Il fit aussi la table de bois d'acacia~; sa longueur était de deux coudées, et sa largeur d'une coudée, et sa hauteur d'une coudée et demie.
\VS{11}Et il la couvrit d'or pur, et lui fit un couronnement d'or tout autour.
\VS{12}Il lui fit aussi à l'entour un rebord d'une largeur d'une paume, et à l'entour de sa bordure un couronnement d'or.
\VS{13}Et il lui fondit quatre anneaux d'or, et il mit les anneaux aux quatre coins, qui étaient à ses quatre pieds.
\VS{14}Les anneaux étaient à coté du rebord, pour y mettre les barres afin de porter la table avec elles.
\VS{15}Et il fit les barres de bois d'acacia, et les couvrit d'or pour porter la table.
\VS{16}Il fit aussi d'or pur des ustensiles pour poser sur la table, ses plats, ses tasses, ses bassins et ses gobelets avec lesquels on devait faire les aspersions.
\TextTitle{Le chandelier}
\VS{17}Il fit aussi le chandelier d'or pur~; il le fit d'ouvrage façonné au marteau~; sa tige, ses branches, ses plats, ses pommeaux et ses fleurs étaient tirés de lui.
\VS{18}Et six branches sortaient de ses côtés, trois branches d'un côté du chandelier, et trois de l'autre côté du chandelier.
\VS{19}Il y avait sur l'une des branches trois plats en forme d'amande, un pommeau et une fleur~; et sur l'autre branche trois plats en forme d'amande, un pommeau et une fleur~; il fit la même chose aux six branches qui sortaient du chandelier.
\VS{20}Et il y avait sur le chandelier quatre plats en forme d'amande, ses pommeaux et ses fleurs.
\VS{21}Et un pommeau sous deux branches tirées du chandelier, et un pommeau sous deux autres branches, tirées de lui, et un pommeau sous deux autres branches, tirées de lui, à savoir des six branches qui procédaient du chandelier.
\VS{22}Leurs pommeaux et leurs branches étaient tirés de lui, et tout le chandelier était un ouvrage d'une seule pièce étendue au marteau et d'or pur.
\VS{23}Il fit aussi ses sept lampes, ses mouchettes, et ses encensoirs d'or pur.
\VS{24}Et il le fit avec toute sa garniture d'un talent d'or pur.
\TextTitle{L'autel des parfums}
\VS{25}Il fit aussi de bois d'acacia l'autel des parfums. Sa longueur était d'une coudée, et sa largeur d'une coudée. Il était carré, et sa hauteur était de deux coudées, et ses cornes procédaient de lui\FTNT{Ex. 30:1-10.}.
\VS{26} Et il couvrit d'or pur le dessus de l'autel, ses côtés tout à l'entour, et ses cornes~; et il lui fit tout à l'entour un couronnement d'or.
\VS{27}Il fit aussi au-dessous de son couronnement deux anneaux d'or à ses deux côtés, lesquels il mit aux deux coins, pour y faire passer les barres afin de le porter avec elles.
\VS{28}Et il fit les barres de bois d'acacia, et les couvrit d'or.
\TextTitle{L'huile d'onction et le parfum}
\VS{29}Il composa aussi l'huile pour l'onction, qui était une chose sainte, et le parfum pur odoriférant, d'ouvrage de parfumeur.
\Chap{38}
\TextTitle{L'autel des holocaustes}
\VerseOne{}Il fit aussi de bois d'acacia l'autel des holocaustes. Et sa longueur était de cinq coudées, et sa largeur de cinq coudées. Il était carré, et sa hauteur était de trois coudées\FTNT{Ex. 27:1-8.}.
\VS{2}Et il fit ses cornes à ses quatre coins. Ses cornes sortaient de lui, et il le couvrit d'airain.
\VS{3}Il fit aussi tous les ustensiles de l'autel~: Les chaudrons, les racloirs, les bassins, les fourchettes et les encensoirs~; il fit tous ses ustensiles d'airain.
\VS{4}Et il fit pour l'autel une grille d'airain en forme de treillis, au-dessous de l'enceinte de l'autel, depuis le bas jusqu'au milieu.
\VS{5}Et il fondit quatre anneaux aux quatre coins de la grille d'airain pour mettre les barres.
\VS{6} Et il fit les barres de bois d'acacia, et les couvrit d'airain.
\VS{7}Et il fit passer les barres dans les anneaux, au cotés de l'autel, pour le porter avec elles. Il le fit creux, avec des planches.
\TextTitle{La cuve d'airain}
\VS{8}Il fit aussi la cuve d'airain et sa base d'airain avec les miroirs des femmes qui s'assemblaient à l'entrée de la tente d'assignation\FTNT{Ex. 30:14-18.}.
\TextTitle{Le parvis}
\VS{9}Il fit aussi un parvis, pour le côté qui regarde vers le sud, et des courtines de fin lin retors, de cent coudées, pour le parvis.
\VS{10}Il fit d'airain leurs vingt colonnes avec leurs vingt bases, mais les crochets des colonnes et leurs filets étaient d'argent.
\VS{11}Et pour le côté nord, il fit des courtines de cent coudées, leurs vingt colonnes et leurs vingt bases étaient d'airain, mais les crochets des colonnes et leurs filets étaient d'argent.
\VS{12}Pour le côté de l'occident, des courtines de cinquante coudées, leurs dix colonnes, et leurs dix bases. Les crochets des colonnes et leurs filets étaient d'argent.
\VS{13}Pour le côté de l'orient droit vers le levant, des courtines de cinquante coudées.
\VS{14}Il fit pour l'un des côtés quinze coudées de courtines, et leurs trois colonnes avec leurs trois bases.
\VS{15}Et pour l'autre côté, quinze coudées de courtines, afin qu'il y en ait autant de part et d'autre de la porte du parvis, et leurs trois colonnes avec leurs trois bases.
\VS{16}Il fit donc toutes les courtines du parvis qui étaient tout autour de fin lin retors.
\VS{17}Il fit aussi d'airain les bases des colonnes, mais il fit d'argent les crochets des colonnes et les filets, et leurs chapiteaux furent couverts d'argent~; et toutes les colonnes du parvis furent ceintes tout autour d'un filet d'argent.
\TextTitle{La porte du parvis}
\VS{18} Et il fit le rideau de la porte du parvis de pourpre, d'écarlate, et de cramoisi et de fin lin retors, d'ouvrage de broderie, de la longueur de vingt coudées, et de la hauteur qui était comme la largeur de cinq coudées, à la correspondance des courtines du parvis~;
\VS{19}et ses quatre colonnes avec leurs bases d'airain, et leurs crochets d'argent, la couverture aussi de leur chapiteaux et leurs filets d'argent~; 
\VS{20} et tous les pieux du tabernacle et du parvis tout autour d'airain.
\TextTitle{Les comptes du tabernacle}
\VS{21}C'est ici le compte des choses qui furent employées au tabernacle, à savoir à la tente d'assignation, selon que le compte en fut fait par l'ordre de Moïse, à quoi furent employés les Lévites, sous la conduite d'Ithamar, fils du prêtre Aaron.
\VS{22}Et Betsaleel, fils d'Uri, fils de Hur, de la tribu de Juda, fit toutes les choses que Yahweh avait ordonnées à Moïse~;
\VS{23}et avec lui Oholiab, fils d'Ahisamac, de la tribu de Dan, les ouvriers, et ceux qui travaillaient en ouvrage exquis, et les brodeurs en pourpre, en écarlate, en cramoisi, et en fin lin.
\VS{24}Tout l'or qui fut employé pour l'ouvrage, à savoir pour tout l'ouvrage du sanctuaire, qui était l'or des offrandes, fut de vingt-neuf talents et sept cent trente sicles, selon le sicle du sanctuaire.
\VS{25} Et l'argent de ceux de l'assemblée qui furent dénombrés fut de cent talents et mille sept cent soixante-quinze sicles, selon le sicle du sanctuaire.
\VS{26}Un demi-sicle par tête, la moitié d'un sicle selon le sicle du sanctuaire. Tous ceux qui passèrent par le dénombrement depuis l'âge de vingt ans et au-dessus, furent six cent trois mille cinq cent cinquante.
\VS{27}Il y eut donc cent talents d'argent pour fondre les bases du sanctuaire, et les bases du voile, à savoir cent bases de cent talents, un talent pour chaque base.
\VS{28}Mais des mille sept cent soixante-quinze sicles, il fit les crochets pour les colonnes, et il couvrit leurs chapiteaux et en fit des filets tout autour.
\VS{29}L'airain des offrandes fut de soixante-dix talents et deux mille quatre cents sicles~;
\VS{30}dont on fit les bases de la porte de la tente d'assignation, et l'autel d'airain avec sa grille d'airain, et tous les ustensiles de l'autel~;
\VS{31}et les bases tout autour du parvis, les bases de la porte du parvis, et tous les pieux du tabernacle, et tous les pieux du parvis tout autour.
\Chap{39}
\TextTitle{Les vêtements sacrés d'Aaron}
\VerseOne{}Ils firent aussi de pourpre, d'écarlate, et de cramoisi les vêtements du service, pour faire le service du sanctuaire. Et ils firent les saints vêtements sacrés pour Aaron, comme Yahweh l'avait ordonné à Moïse\FTNT{Ex. 28.}.
\VS{2}On fit donc l'éphod d'or, de pourpre, d'écarlate, de cramoisi, et de fin lin retors.
\VS{3}Or on étendit des lames d'or, et on les coupa par filets pour les brocher parmi la pourpre, l'écarlate, le cramoisi, et le fin lin d'ouvrage exquis.
\VS{4}On fit à l'éphod des épaulettes\FTNT{Voir annexe «~Les habits du grand-prêtre~».} qui s'attachaient, en sorte qu'il était joint par ses deux extrémités.
\VS{5}Et la ceinture exquise de laquelle il était ceint, était tirée de lui, et de même ouvrage, d'or, de pourpre, d'écarlate, de cramoisi, et de fin lin retors, comme Yahweh l'avait ordonné à Moïse.
\VS{6}On enchassa aussi les pierres d'onyx dans leurs montures d'or, ayant les noms des enfants d'Israël gravés comme on grave les cachets.
\VS{7}Et on les mit sur les épaulettes de l'éphod, afin qu'elles soient des pierres de souvenir pour les enfants d'Israël, comme Yahweh l'avait ordonné à Moïse.
\VS{8}On fit aussi le pectoral\FTNT{Voir annexe «~Les habits du grand-prêtre~».} d'ouvrage exquis, comme l'ouvrage de l'éphod, d'or, de pourpre, d'écarlate, de cramoisi, et de fin lin retors.
\VS{9}On fit le pectoral carré et double~; sa longueur était d'une paume, et sa largeur d'une paume de part et d'autre.
\VS{10}Et on le garnit de quatre rangs de pierres~: A la première rangée on mit une sardoine, une topaze et une émeraude.
\VS{11}A la seconde rangée une escarboucle, un saphir, et un jaspe.
\VS{12}A la troisième rangée, une opale, une agate, et une améthyste.
\VS{13}A la quatrième rangée, un chrysolithe, un onyx, et un béryl\FTNT{Ap. 21:18-19.}, enchâssés dans leur monture d'or.
\VS{14}Ainsi, il y avait autant de pierres qu'il y avait de noms des enfants d'Israël, douze selon leurs noms, chacune d'elles gravées comme des cachets, selon le nom qu'elle devait porter, et elles étaient pour les douze tribus.
\VS{15}Et on fit sur le pectoral des chaînettes à bouts en façon de cordon, d'or pur.
\VS{16}On fit aussi deux montures d'or et deux anneaux d'or, on mit les deux anneaux aux deux extrémités du pectoral.
\VS{17}Et on mit les deux chaînettes d'or faites à cordon dans les deux anneaux à l'extrémité du pectoral.
\VS{18}Et on mit les deux autres bouts des deux chaînettes faites à cordon aux deux montures, sur les épaulettes de l'éphod, sur le devant de l'éphod.
\VS{19}On fit aussi deux autres anneaux d'or, et on les mit aux deux autres extrémités du pectoral sur son bord, qui était du côté de l'éphod à l'intérieur.
\VS{20}On fit aussi deux autres anneaux d'or, et on les mit aux deux épaulettes de l'éphod par le bas, répondant sur le devant de l'éphod, à l'endroit où il se joignait au-dessus de la ceinture exquise de l'éphod.
\VS{21}Et on joignit le pectoral élevé par ses anneaux aux anneaux de l'éphod, avec un cordon de pourpre, afin qu'il tienne au-dessus de la ceinture exquise de l'éphod, et que le pectoral ne bouge de dessus l'éphod, comme Yahweh l'avait ordonné à Moïse.
\VS{22}On fit aussi la robe de l'éphod d'ouvrage tissé et entièrement de pourpre.
\VS{23} Et l'ouverture pour passer la tête était au milieu de la robe, comme l'ouverture d'une cotte de mailles~; et il y avait un ourlet à l'ouverture de la robe tout autour, afin qu'elle ne se déchire pas.
\VS{24}Aux bordures de la robe, on fit des grenades de pourpre, d'écarlate et de cramoisi, à fil retors.
\VS{25}On fit aussi des clochettes d'or pur~; et on mit les clochettes entre les grenades aux bordures de la robe tout autour, parmi les grenades~;
\VS{26}à savoir une clochette puis une grenade, une clochette puis une grenade, sur la bordure de la robe tout autour, pour faire le service, comme Yahweh l'avait ordonné à Moïse.
\VS{27}On fit aussi à Aaron et à ses fils des tuniques de fin lin d'ouvrage tissé.
\VS{28}Et la tiare de fin lin, et les ornements des calottes de fin lin, et les caleçons de lin, de fin lin retors.
\VS{29}Et la ceinture de fin lin retors, de pourpre, d'écarlate et de cramoisi, d'ouvrage de broderie~; comme Yahweh l'avait ordonné à Moïse~;
\VS{30}et la lame du saint diadème d'or pur, sur laquelle on écrivit comme on grave un cachet~: La sainteté à Yahweh.
\VS{31}Et on mit sur elle un cordon de pourpre, pour l'appliquer à la tiare par-dessus, comme Yahweh l'avait ordonné à Moïse.
\TextTitle{Le matériel pour excercer la prêtrise est prêt}
\VS{32}Ainsi fut achevé tout l'ouvrage du tabernacle, de la tente d'assignation. Les enfants d'Israël firent selon toutes les choses que Yahweh avait ordonnées à Moïse~; ils les firent ainsi.
\VS{33}Et ils apportèrent à Moïse le tabernacle, la tente, et tous ses ustensiles, ses crochets, ses planches, ses barres, ses colonnes, et ses bases~;
\VS{34}la couverture de peaux de béliers teintes en rouge, la couverture de peaux de taissons, et le voile qui sert de rideau devant le Saint des saints~;
\VS{35}l'arche du témoignage et ses barres, et le propitiatoire~;
\VS{36}la table avec tous ses ustensiles, et les pains de proposition\FTNT{Ex. 31:8-10.}~;
\VS{37}et le chandelier d'or pur avec toutes ses lampes arrangées, et tous ses ustensiles, et l'huile du chandelier~;
\VS{38}et l'autel d'or, l'huile d'onction, le parfum odoriférant, et le rideau de l'entrée de la tente~;
\VS{39}l'autel d'airain, avec sa grille d'airain, ses barres et tous ses ustensiles~; la cuve et sa base~;
\VS{40}et les courtines du parvis, ses colonnes, ses bases, le rideau pour la porte du parvis, son cordage, ses pieux, et tous les ustensiles pour le service du tabernacle, pour la tente d'assignation~;
\VS{41}les vêtements du service pour faire le service du sanctuaire, les saints vêtements pour le prêtre Aaron, et les vêtements de ses fils pour exercer la prêtrise.
\VS{42}Les enfants d'Israël donc firent tout l'ouvrage, comme Yahweh l'avait ordonné à Moïse.
\VS{43}Et Moïse vit tout l'ouvrage, et voici, on l'avait fait ainsi que Yahweh l'avait ordonné, on l'avait, dis-je, fait ainsi. Et Moïse les bénit.
\Chap{40}
\TextTitle{Moïse dresse le tabernacle}
\VerseOne{}Et Yahweh parla à Moïse, en disant~:
\VS{2}Au premier jour du premier mois, tu dresseras le tabernacle de la tente d'assignation.
\VS{3}Et tu y mettras l'arche du témoignage, au-devant de laquelle tu tendras le voile.
\VS{4}Puis tu apporteras la table et y arrangeras ce qui doit y être arrangé. Tu apporteras aussi le chandelier et allumeras ses lampes.
\VS{5}Tu mettras aussi l'autel d'or pour le parfum au-devant de l'arche du témoignage, et tu mettras le rideau de l'entrée au tabernacle.
\VS{6}Tu mettras aussi l'autel de l'holocauste vis-à-vis de l'entrée du tabernacle de la tente d'assignation.
\VS{7}Tu mettras aussi la cuve entre la tente d'assignation et l'autel, et y mettras de l'eau.
\VS{8}Tu mettras aussi le parvis tout autour, et tu mettras le rideau à la porte du parvis.
\VS{9}Tu prendras aussi l'huile de l'onction, et tu en oindras le tabernacle, et tout ce qui y est, et tu le sanctifieras avec tous ses ustensiles~; et il sera saint.
\VS{10}Tu oindras aussi l'autel de l'holocauste, et tous ses ustensiles, et tu sanctifieras l'autel, et l'autel sera très saint.
\VS{11}Tu oindras aussi la cuve et sa base, et la sanctifieras.
\VS{12}Tu feras aussi approcher Aaron et ses fils à l'entrée de la tente d'assignation, et les laveras avec de l'eau.
\VS{13}Et tu feras vêtir à Aaron les saints vêtements, et tu l'oindras et le sanctifieras~; et il exercera la prêtrise pour moi.
\VS{14}Tu feras aussi approcher ses fils que tu revêtiras de tuniques.
\VS{15} Et tu les oindras comme tu auras oint leur père~; et ils m'exerceront la prêtrise, et leur onction leur sera pour exercer la prêtrise à toujours parmi leur génération.
\VS{16} Ce que Moïse fit selon toutes les choses que Yahweh lui avait ordonnées~; il le fit ainsi.
\VS{17}Car au premier jour du premier mois de la seconde année, le tabernacle fut dressé.
\VS{18}Moïse donc dressa le tabernacle, mit ses bases, posa ses planches, mit ses barres et dressa ses colonnes.
\VS{19}Et il étendit la tente sur le tabernacle, et mit la couverture de la tente au-dessus du tabernacle par le haut, comme Yahweh l'avait ordonné à Moïse.
\VS{20}Puis il prit et posa le témoignage dans l'arche et mit les barres à l'arche~; il mit aussi le propitiatoire au-dessus de l'arche.
\VS{21}Et il apporta l'arche dans le tabernacle, et posa le voile qui sert de rideau, et le mit au-devant de l'arche du témoignage, comme Yahweh l'avait ordonné à Moïse.
\VS{22}Il mit aussi la table dans la tente d'assignation, au côté du tabernacle vers le nord, en dehors du voile.
\VS{23}Et il arrangea sur elle les rangées de pains devant Yahweh, comme Yahweh l'avait ordonné à Moïse.
\VS{24}Il mit aussi le chandelier dans la tente d'assignation, vis-à-vis de la table, du côté du tabernacle, vers le sud.
\VS{25}Et il alluma les lampes devant Yahweh, comme Yahweh l'avait ordonné à Moïse.
\VS{26}Il posa aussi l'autel d'or dans la tente d'assignation, devant le voile.
\VS{27}Et il fit fumer sur lui le parfum odoriférant, comme Yahweh l'avait ordonné à Moïse.
\VS{28}Il mit aussi le rideau de l'entrée du tabernacle.
\VS{29}Et il mit l'autel de l'holocauste à l'entrée du tabernacle de la tente d'assignation~; et offrit sur lui l'holocauste et l'offrande, comme Yahweh l'avait ordonné à Moïse.
\VS{30}Et il plaça la cuve entre la tente d'assignation et l'autel, et y mit de l'eau pour se laver.
\VS{31}Et Moïse et Aaron avec ses fils en lavèrent leurs mains et leurs pieds.
\VS{32}Et quand ils entraient dans la tente d'assignation, et qu'ils approchaient de l'autel, ils se lavaient, selon que Yahweh l'avait ordonné à Moïse.
\VS{33}Il dressa aussi le parvis tout autour du tabernacle et de l'autel, et tendit le rideau de la porte du parvis. Ainsi Moïse acheva l'ouvrage.
\TextTitle{La gloire de Yahweh sur le tabernacle}
\VS{34}Et la nuée couvrit la tente d'assignation, et la gloire de Yahweh remplit le tabernacle\FTNT{No. 9:15~; 1 R. 8:10.},
\VS{35}tellement que Moïse ne put entrer dans la tente d'assignation, car la nuée se tenait dessus et la gloire de Yahweh remplissait le tabernacle.
\VS{36} Or quand la nuée se levait de dessus le tabernacle, les enfants d'Israël partaient dans toutes leurs marches.
\VS{37}Mais si la nuée ne se levait point, ils ne partaient point, jusqu'au jour où elle se levait.
\VS{38}Car la nuée de Yahweh était le jour sur le tabernacle, et le feu y était la nuit, devant les yeux de toute la maison d'Israël, dans toutes leurs marches.
\PPE
\end{multicols}

\clearpage\ShortTitle{Lé.}\BookTitle{Lévitique}\BFont
\noindent\hrulefill
{\footnotesize
\textit{
\bigskip
{\centering{}
\\Auteur : Probablement Moïse
\\(Heb. : Vayiqra)
\\Signification : Et Il (Yahweh) appela
\\Thème : La sainteté
\\Date de rédaction : Env. 1450-1410 av. J.-C.\\}
}
%\bigskip
\textit{
\\Après avoir construit et dressé le tabernacle selon le modèle que Yahweh avait donné à Moïse, les fils d'Israël reçurent le détail des prescriptions relatives aux offrandes, aux sacrifices et aux fêtes en l'honneur de Yahweh. 
%\bigskip
\\Ce livre, dont le nom tire son origine de Lévi, explique la manière dont Aaron et ses fils devaient exercer la sacrificature et amener le peuple à s'approcher de Dieu dans le respect de ses ordonnances.
%\bigskip
\\Les lois que Moïse avait recueillies présentent la voie du pardon, laquelle est impossible sans effusion de sang. Bien que les mêmes sacrifices furent réitérés tous les ans, ces préceptes mettaient en évidence l'impuissance de l'homme à atteindre la justice de Dieu par ses propres moyens.\bigskip
}
}
\par\nobreak\noindent\hrulefill
\begin{multicols}{2}
\Chap{1}
\TextTitle{L'holocauste\FTNTT{voir Lé. 6:1-6.}}
\VerseOne{}Et Yahweh appela Moïse, et lui parla de la tente d'assignation, en disant :
\VS{2}Parle aux enfants d'Israël, et dis-leur : Quand quelqu'un d'entre vous offrira à Yahweh une offrande d'une bête à quatre pattes, il fera son offrande de gros ou de menu bétail.
\VS{3}Si son offrande pour un holocauste est de gros bétail, il offrira un mâle sans défaut\FTNT{L'holocauste était le sacrifice pour l'expiation par excellence. Contrairement aux autres sacrifices, l'holocauste était entièrement consumé sur l'autel. Il symbolisait d'une part le sacrifice parfait de Christ et d'autre part notre vie, volontairement offerte à Dieu (Ro. 12:1). Les animaux aptes à être offerts en holocauste devaient être des mâles sans défaut :
\\- Le veau (Lé. 1:5), image de Christ, l'humble serviteur, soumis et obéissant (Mt. 20:28 ; Ph. 2:5-8).
\\- L'agneau ou le chevreau, image de Christ qui livre sa vie à la croix sans résistance ni contestation, et qui prend sur lui nos péchés (Es. 53:7 ; Mt. 26:63 ; Ac. 8:32). 
\\- Les tourterelles ou les jeunes pigeons, image de la simplicité de Christ (Mt. 10:16).
\\Toutes les étapes de la réalisation de ce sacrifice enseignent le disciple sur la mort à soi-même et le dépouillement des œuvres de la chair (Ga. 5:19-21).
\\Le sang de l'animal égorgé devait être répandu sur l'autel (Lé. 1:5), image de la croix. L'âme (contenue dans le sang selon Lé. 17:14), liée à la chair et ses désirs, doit être crucifiée (Ga. 2:20 ; Ga. 5:24). L'objet de la mise à mort était certainement un couteau tranchant comme une épée, image de la Parole de Dieu (Hé. 4:12). La mise en pratique de la Parole nous amène nécessairement à nous séparer du monde et à renoncer à soi-même.} ; il l'offrira de son bon gré à l'entrée de la tente d'assignation ; devant Yahweh\FTNT{Ex. 29:10-11.}.
\VS{4}Et il posera sa main sur la tête de l'holocauste, et il sera agréé pour lui, afin de faire la propitiation pour lui.
\VS{5}Puis, on égorgera le jeune taureau devant Yahweh ; et les fils d'Aaron, les prêtres, en offriront le sang et ils répandront le sang sur l'autel tout autour, qui est à l'entrée de la tente d'assignation.
\VS{6}Et on égorgera l'holocauste et le coupera en morceaux.
\VS{7}Les fils du prêtre Aaron mettront le feu sur l'autel, et disposeront le bois sur le feu.
\VS{8}Et les fils d'Aaron, les prêtres, poseront les morceaux, la tête et la graisse sur le bois qui sera au feu sur l'autel.
\VS{9}Mais il lavera avec de l'eau les entrailles et les jambes ; et le prêtre brûlera toutes ces choses sur l'autel. C'est un holocauste, un sacrifice consumé par le feu, d'une bonne odeur à Yahweh.
\VS{10}Si son offrande est un holocauste de menu bétail, d'entre les agneaux ou d'entre les chèvres, il offrira un mâle sans défaut.
\VS{11}Et on l'égorgera à côté de l'autel, vers le nord, devant Yahweh ; et les prêtres, fils d'Aaron, en répandront le sang sur l'autel tout autour.
\VS{12}Puis on le coupera en morceaux, avec sa tête et sa graisse ; et le prêtre les posera sur le bois qui sera au feu sur l'autel.
\VS{13}Mais il lavera avec de l'eau les entrailles et les jambes. Puis le prêtre offrira toutes ces choses, et les brûlera sur l'autel. C'est un holocauste, un sacrifice consumé par le feu, d'une agréable odeur à Yahweh\FTNT{Ez. 40:38.}.
\VS{14}Si son offrande à Yahweh est un holocauste d'oiseaux, il offrira son offrande de tourterelles, ou de jeunes pigeons.
\VS{15}Le prêtre l'apportera sur l'autel, lui ouvrira la tête avec l'ongle, la brûlera sur l'autel, et il en exprimera le sang contre un côté de l'autel.
\VS{16}Il ôtera son jabot avec ses plumes, et le jettera près de l'autel, vers l'orient, dans le lieu où seront les cendres.
\VS{17}Il le déchirera avec ses ailes, sans le séparer ; et le prêtre le brûlera sur l'autel, sur le bois qui sera au feu. C'est un holocauste, un sacrifice consumé par le feu, d'une agréable odeur à Yahweh.
\Chap{2}
\TextTitle{L'offrande de gâteau\FTNTT{Lé. 6:7-16.}}
\VerseOne{}Lorsque quelqu'un offrira l'offrande de gâteau\FTNT{L'offrande de farine ou de gâteau correspond aux perfections de la vie du Seigneur Jésus-Christ en tant qu'homme. Ce sacrifice ne comporte ni victime ni sang, mais seulement de la farine, de l'huile, de l'encens et du sel. Jésus, le grain de blé (Jn. 12:24), a été complètement broyé, pétri et oint d'huile, éprouvé par toutes sortes de douleurs. Sa vie sainte était pour le Père un parfum de bonne odeur. Son amour pour les âmes, sa dépendance totale au Père, sa persévérance, sa douceur, sa sagesse et sa bonté, n'ont pas varié malgré toutes les souffrances par lesquelles il est passé. Voilà quelques-uns des fruits admirables qui correspondent à l'offrande de gâteau saupoudrée d'encens. Le levain, image du péché (1 Co. 5:6-8), n'y entrait pas, ni le miel, symbole des affections humaines (Pr. 5:3). Quant au sel, il préserve de la corruption des aliments, il est comparé à la saveur des disciples de Christ (Mt. 5:13).} à Yahweh, son offrande sera de fine farine ; il versera de l'huile dessus, et mettra de l'encens.
\VS{2}Il l'apportera aux fils d'Aaron, les prêtres, et le prêtre prendra une pleine poignée de cette fine farine, et d'huile, avec tout l'encens, et il brûlera son souvenir\FTNT{En hébreu « azkarah », offrande de souvenir, la portion de nourriture offerte et qui est consumée.} sur l'autel. C'est une offrande d'une bonne odeur à Yahweh.
\VS{3}Ce qui restera du gâteau sera pour Aaron et ses fils ; c'est une chose très sainte parmi les offrandes consumées par le feu à Yahweh.
\VS{4}Et quand tu offriras une offrande de gâteaux cuits au four, ce sera de fine farine, des gâteaux sans levain, pétris avec de l'huile, et des galettes sans levain, ointes d'huile.
\VS{5}Si ton offrande est un gâteau cuit sur la plaque, elle sera de fine farine pétrie à l'huile, sans levain.
\VS{6}Tu la rompras en morceaux, et tu verseras de l'huile sur elle ; c'est une offrande de gâteau.
\VS{7}Si ton offrande est un gâteau cuit sur le gril, elle sera faite de fine farine avec de l'huile.
\VS{8}Puis tu apporteras à Yahweh l'offrande de gâteaux qui sera faite de ces choses, et on la présentera au prêtre, qui l'apportera sur l'autel.
\VS{9}Le prêtre lèvera de l'offrande de gâteaux, son souvenir, et le brûlera sur l'autel. C'est une offrande consumée par le feu de bonne odeur à Yahweh.
\VS{10}Ce qui restera de l'offrande de gâteau sera pour Aaron et ses fils ; c'est une chose très sainte parmi les offrandes consumées par le feu devant Yahweh.
\VS{11}Aucune offrande de gâteau que vous offrirez à Yahweh ne sera faite avec du levain ; car vous ne brûlerez point de levain ni de miel, parmi l'offrande consumée par le feu devant Yahweh.
\VS{12}Vous pourrez bien les offrir à Yahweh dans l'offrande des prémices, mais ils ne seront point mis sur l'autel comme offrande d'une bonne odeur.
\VS{13}Tu mettras du sel\FTNT{Voir No. 18:19 ; 2 Ch. 13:5. Le sel est un agent purificateur (2 R. 2:19-22). Le sel préserve de la corruption et conserve les aliments. Les chrétiens sont le sel de la terre (Mt. 5:13). Nos paroles doivent être assaisonnées de sel (Col. 4:6).} sur toutes tes offrandes de gâteaux, et tu ne laisseras point ton offrande de gâteau manquer de sel, signe de l'alliance de ton Dieu ; mais sur toutes tes offrandes, tu offriras du sel.
\VS{14}Si tu offres à Yahweh une offrande de gâteau des premiers fruits, tu offriras, pour l'offrande de gâteau des premiers fruits, des épis qui commencent à mûrir, rôtis au feu, les grains de quelques épis bien grenés, broyés entre les mains.
\VS{15}Puis tu mettras de l'huile sur le gâteau, et tu mettras aussi de l'encens dessus : C'est une offrande de gâteaux.
\VS{16}Et le prêtre brûlera son souvenir, pris de ses grains broyés, et de son huile avec tout l'encens. C'est une offrande consumée par le feu à Yahweh.
\Chap{3}
\TextTitle{Le sacrifice d'offrande de paix\FTNTT{Lé. 7:11-21.}}
\VerseOne{}Si son offrande est un sacrifice d'offrande de paix\FTNT{La plupart des traducteurs ont traduit par « sacrifice d'actions de grâces », or l'étymologie hébraïque du mot grâce est « shelem », ce qui signifie d'abord « paix ». Ce terme peut aussi vouloir dire « remerciement » ou « reconnaissance ». La racine de « shelem » est « shalam » : « être dans une alliance de paix », « être en paix ».Il est donc question ici d'une offrande de paix qui préfigure l'ensemble de l'œuvre de la croix accomplie par le Messie, et grâce à laquelle nous sommes réconciliés avec le Père (Col. 1:20 ; Ep. 2:14-17). Cette offrande préfigure aussi la Pâque incarnée par le Messie (1 Co. 5:7) ainsi que le repas du Seigneur. En effet, sur cette offrande, Dieu prenait pour lui la graisse et la queue entière (Lé. 3:3 ; Lé. 3:9-17), le prêtre prenait la poitrine et l'épaule droite (Lé. 7:31-34), et celui qui offrait l'animal pouvait consommer le reste avec d'autres personnes pures (Lé. 7:20). Ainsi, comme pour le repas du Seigneur, tous ceux qui étaient saints pouvaient participer au repas (1 Co. 11:27-34).}, et qu'il offre du gros bétail, soit mâle, soit femelle, il l'offrira sans défaut devant Yahweh.
\VS{2}Il posera sa main sur la tête de son offrande, et l'égorgera à l'entrée de la tente d'assignation, et les fils d'Aaron, les prêtres, répandront le sang sur l'autel tout autour.
\VS{3}Puis on offrira de cette offrande de paix, un sacrifice consumé par le feu à Yahweh, à savoir la graisse qui couvre les entrailles et toute la graisse qui est sur les entrailles ;
\VS{4}les deux rognons avec la graisse qui est dessus et qui est sur les flancs ; et on ôtera le grand lobe qui est sur le foie pour le mettre avec les rognons.
\VS{5}Les fils d'Aaron brûleront tout cela sur l'autel, sur l'holocauste, qui sera sur le bois mis au feu. C'est une offrande consumée par le feu d'agréable odeur à Yahweh\FTNT{Ex. 29:13-25.}.
\VS{6}Si son offrande pour le sacrifice d'offrande de paix à Yahweh est de menu bétail, soit mâle, soit femelle, il l'offrira sans défaut.
\VS{7}S'il offre un agneau pour son offrande, il l'offrira devant Yahweh.
\VS{8}Il posera sa main sur la tête de son offrande, et l'égorgera devant la tente d'assignation, et les fils d'Aaron répandront son sang sur l'autel tout autour.
\VS{9}De ce sacrifice d'offrande de paix, il offrira en offrande consumée par le feu à Yahweh, sa graisse et sa queue entière, séparée jusqu'à l'échine, avec la graisse qui couvre les entrailles et toute la graisse qui est sur les entrailles,
\VS{10}les deux rognons avec la graisse qui est dessus, sur les flancs, et il ôtera le grand lobe qui est sur le foie, jusqu'aux rognons.
\VS{11}Le prêtre brûlera tout cela sur l'autel. C'est un aliment d'offrande consumée par le feu à Yahweh\FTNT{No. 28:2.}.
\VS{12}Si son offrande est une chèvre, il l'offrira devant Yahweh.
\VS{13}Il posera sa main sur sa tête, et l'égorgera devant la tente d'assignation ; et les fils d'Aaron répandront son sang sur l'autel tout autour.
\VS{14}Puis il offrira son offrande en sacrifice consumé par le feu à Yahweh, la graisse qui couvre les entrailles et toute la graisse qui est sur les entrailles,
\VS{15}les deux rognons, et la graisse qui est dessus, sur les flancs, et il ôtera le grand lobe qui est sur le foie, jusqu'aux rognons.
\VS{16}Puis le prêtre brûlera toutes ces choses sur l'autel. C'est un aliment d'offrande consumée par le feu de bonne odeur. Toute graisse appartient à Yahweh.
\VS{17}C'est une loi perpétuelle pour vos descendants, dans toutes vos demeures : Vous ne mangerez ni graisse ni sang\FTNT{Ge. 9:4 ; 1 S. 14:33.}.
\Chap{4}
\TextTitle{Le sacrifice pour l'expiation\FTNTT{Lé. 6:17-23.}}
\VerseOne{}Yahweh parla encore à Moïse en disant :
\VS{2}Parle aux enfants d'Israël, et dis-leur : Quand une personne aura péché involontairement\FTNT{Avant la promulgation de la loi, certains hommes péchaient par ignorance (Ro. 5:13). Néanmoins, ces péchés étaient tout de même punis et nécessitaient un sacrifice (Lé. 4:13-14. No. 15:22-36 ; Job. 1). Sous la grâce, l'excuse du péché par ignorance ne peut être invoquée puisque nous sommes scellés du Saint-Esprit qui nous enseigne toutes choses (1 Jn. 2:20 et 27).} contre l'un des commandements de Yahweh, en commettant des choses qui ne doivent point se faire, et qu'il aura fait une de ces choses ;
\VS{3} si c'est le prêtre oint qui ait commis un péché, semblable à quelque faute du peuple, il offrira à Yahweh pour son péché qu'il aura fait, un jeune taureau sans défaut, pris du troupeau en sacrifice pour l'expiation.
\VS{4}Il amènera le taureau à l'entrée de la tente d'assignation, devant Yahweh, il posera sa main sur la tête du taureau, et l'égorgera devant Yahweh.
\VS{5}Et le prêtre oint prendra du sang du taureau, et l'apportera dans la tente d'assignation.
\VS{6}Le prêtre trempera son doigt dans le sang, et fera sept fois l'aspersion du sang devant Yahweh, en face du voile du lieu saint\FTNT{No. 19:4.}.
\VS{7}Le prêtre mettra aussi devant Yahweh du sang sur les cornes de l'autel des parfums odoriférants, qui est dans la tente d'assignation ; et il répandra tout le reste du sang du taureau au pied de l'autel de l'holocauste, qui est à l'entrée de la tente d'assignation.
\VS{8}Il enlèvera toute la graisse du taureau du sacrifice pour l'expiation, à savoir, la graisse qui couvre les entrailles, et toute la graisse qui est sur les entrailles,
\VS{9}et les deux rognons avec la graisse qui les entoure, qui couvre les flancs, et il ôtera le grand lobe qui est sur le foie, pour le mettre sur les rognons.
\VS{10}Comme on les enlève du taureau du sacrifice d'offrande de paix\FTNT{Voir commentaire en Lé. 3:1.}, et le prêtre brûlera toutes ces choses-là sur l'autel de l'holocauste.
\VS{11}Mais quant à la peau du taureau et toute sa chair, avec sa tête, ses jambes, ses entrailles, et ses excréments,
\VS{12}et même tout le taureau, il l'emportera hors du camp, dans un lieu pur, où l'on répand les cendres, et il le brûlera au feu sur du bois : Il sera brûlé au lieu où l'on répand les cendres.
\VS{13}Et si toute l'assemblée d'Israël a péché involontairement, et que la chose soit restée cachée aux yeux de l'assemblée, et qu'ils aient violé l'un des commandements de Yahweh, en commettant des choses qui ne doivent pas se faire, et s'en soit rendu coupable,
\VS{14}et que le péché qu'ils ont fait vienne en évidence, l'assemblée offrira en sacrifice pour l'expiation un jeune taureau pris du troupeau, et on l'amènera devant la tente d'assignation.
\VS{15}Les anciens de l'assemblée poseront leurs mains sur la tête du taureau devant Yahweh, et on égorgera le taureau devant Yahweh.
\VS{16}Et le prêtre oint, apportera du sang du taureau dans la tente d'assignation ;
\VS{17}ensuite le prêtre trempera son doigt dans le sang, et en fera aspersion devant Yahweh en face du voile, par sept fois.
\VS{18}Et il mettra du sang sur les cornes de l'autel, qui est devant Yahweh dans la tente d'assignation ; et il répandra tout le reste du sang au pied de l'autel de l'holocauste, qui est à l'entrée de la tente d'assignation.
\VS{19}Il enlèvera toute sa graisse et la brûlera sur l'autel.
\VS{20}Et il fera de ce taureau comme il l'a fait du taureau pour le sacrifice d'expiation. Le prêtre fera ainsi ; il fera propitiation pour eux, et il leur sera pardonné.
\VS{21}Puis il emportera le taureau hors du camp, et le brûlera comme il a brûlé le premier taureau. Car c'est le sacrifice pour l'expiation de l'assemblée.
\VS{22}Que si un chef a péché involontairement, en violant l'un des commandements de Yahweh son Dieu, ce qui ne doit point se faire, et s'en soit rendu coupable,
\VS{23}et qu'on vienne à connaître le péché qu'il a commis, il amènera pour sacrifice un jeune bouc, mâle, sans défaut ;
\VS{24}et il posera sa main sur la tête du bouc, et l'égorgera au lieu où l'on égorge l'holocauste devant Yahweh. C'est un sacrifice pour expiation.
\VS{25}Puis le prêtre prendra avec son doigt du sang de l'offrande pour l'expiation, et le mettra sur les cornes de l'autel de l'holocauste, et il répandra le reste de son sang au pied de l'autel de l'holocauste.
\VS{26}Et il brûlera toute sa graisse sur l'autel, comme la graisse du sacrifice d'offrande de paix. Ainsi le prêtre fera propitiation pour lui de son péché, et il lui sera pardonné.
\VS{27}Que si quelqu'un du peuple du pays a péché involontairement, en violant l'un des commandements de Yahweh, et en commettant des choses qui ne doivent point se faire, et s'en soit rendu coupable,
\VS{28}et qu'on vienne à connaître le péché qu'il a commis, il amènera pour offrande une jeune chèvre, femelle, sans défaut, pour le péché qu'il a commis.
\VS{29}Et il posera sa main sur la tête de l'offrande pour le péché, et égorgera l'offrande pour l'expiation au lieu où l'on égorge l'holocauste.
\VS{30}Puis le prêtre prendra du sang de la chèvre avec son doigt, et le mettra sur les cornes de l'autel de l'holocauste, et il répandra tout le reste de son sang au pied de l'autel.
\VS{31}Et il ôtera toute sa graisse, comme on ôte la graisse de dessus le sacrifice d'offrande de paix, et le prêtre la brûlera sur l'autel, en bonne odeur à Yahweh. Il fera propitiation pour lui, et il lui sera pardonné.
\VS{32}Que s'il amène un agneau comme offrande, pour le sacrifice d'expiation, il amènera une femelle sans défaut.
\VS{33}Et il posera sa main sur la tête de l'offrande d'expiation, et on l'égorgera en sacrifice pour l'expiation au lieu où l'on égorge l'holocauste.
\VS{34}Puis le prêtre prendra avec son doigt du sang de l'offrande pour l'expiation, et le mettra sur les cornes de l'autel de l'holocauste, et il répandra tout le reste de son sang au pied de l'autel.
\VS{35}Et il ôtera toute sa graisse, comme on ôte la graisse de l'agneau du sacrifice d'offrande de paix, et le prêtre la brûlera sur l'autel, par-dessus les sacrifices de Yahweh consumés par le feu, et il fera propitiation pour lui, pour son péché qu'il aura commis, et il lui sera pardonné.
\Chap{5}
\TextTitle{Le sacrifice de culpabilité\FTNTT{Lé. 7:1-7.}}
\VerseOne{}Et quand quelqu'un, étant témoin, après avoir entendu la parole du serment, aura péché en ne déclarant pas ce qu'il a vu ou ce qu'il sait, il portera son iniquité\FTNT{Pr. 29:24.}.
\VS{2}Et quand quelqu'un, à son insu, aura touché une chose souillée, soit le cadavre d'un animal impur, soit le cadavre d'une bête sauvage impure, soit le cadavre d'un reptile impur, il sera souillé et coupable\FTNT{Ag. 2:14 ; 2 Co. 6:17.}.
\VS{3}Ou quand il aura touché à l'impureté d'un homme, quelle que soit son impureté par laquelle il se rend impur, et que cela lui soit resté caché, quand il le sait, alors il est coupable.
\VS{4}Ou quand quelqu'un, parlant légèrement de ses lèvres, a juré de faire du mal ou du bien, selon tout ce que l'homme profère légèrement en jurant, et que cela lui soit resté caché, quand il le sait, alors il est coupable dans l'un de ces points-là.
\VS{5}Quand donc quelqu'un sera coupable sur l'un de ces points là, il confessera ce en quoi il aura péché.
\VS{6}Et il amènera son sacrifice de culpabilité à Yahweh pour le péché qu'il a commis, à savoir, une femelle du menu bétail, soit une brebis, soit une chèvre, pour l'offrande d'expiation. Et le prêtre fera pour lui propitiation de son péché.
\VS{7}Et s'il n'a pas le moyen de trouver une brebis ou une chèvre, il apportera en offrande pour le péché à Yahweh, pour sa culpabilité, deux tourterelles ou deux jeunes pigeons, l'un comme sacrifice pour l'expiation, l'autre pour l'holocauste\FTNT{Lu. 2:24.}.
\VS{8}Il les apportera au prêtre, qui offrira premièrement celui qui est pour l'offrande d'expiation. Il leur ouvrira la tête avec l'ongle, près du cou, sans la séparer ;
\VS{9}puis il fera l'aspersion du sang du sacrifice d'expiation sur un côté de l'autel, et ce qui restera du sang sera exprimé au pied de l'autel : C'est un sacrifice pour l'expiation.
\VS{10}Et il fera de l'autre un holocauste, selon l'ordonnance. Et le prêtre fera pour lui la propitiation pour son péché qu'il aura commis, et il lui sera pardonné.
\VS{11}Si celui qui aura péché n'a pas le moyen de trouver deux tourterelles ou deux jeunes pigeons, il apportera pour son offrande un dixième d'épha de fine farine en offrande pour le sacrifice d'expiation ; il ne mettra ni huile ni encens, car c'est un sacrifice d'expiation.
\VS{12}Il l'apportera au prêtre, et le prêtre qui en prendra une pleine poignée pour souvenir\FTNT{Lé. 2:2.}, la brûlera sur l'autel, comme offrande consumée par le feu à Yahweh : C'est un sacrifice d'expiation.
\VS{13}Ainsi le prêtre fera propitiation pour lui, pour le péché qu'il a commis dans l'une de ces choses, et il lui sera pardonné. Le reste sera pour le prêtre, comme étant une offrande de gâteau.
\VS{14}Yahweh parla aussi à Moïse, en disant :
\VS{15}Quand quelqu'un aura commis une transgression et péchera involontairement, en retenant des choses consacrées à Yahweh, il amènera en sacrifice de culpabilité à Yahweh, à savoir un bélier sans défaut, pris du troupeau, avec l'estimation que tu feras de la chose sainte, la faisant en sicles d'argent, selon le sicle du sanctuaire, à cause de son péché.
\VS{16}Il restituera donc ce en quoi il aura péché en retenant de la chose sainte et il y ajoutera un cinquième par dessus, et le donnera au prêtre ; et le prêtre fera propitiation pour lui, par le bélier du sacrifice de culpabilité, et il lui sera pardonné.
\VS{17}Lorsque quelqu'un aura péché, en violant, sans le savoir, l'un des commandements de Yahweh, des choses qu'on ne doit point faire, il sera coupable et portera son iniquité.
\VS{18} Il amènera donc en sacrifice de culpabilité au prêtre un bélier sans tâche, pris du troupeau, avec l'estimation que tu feras du péché involontaire ; et le prêtre fera propitiation pour lui du péché involontaire qu'il a commis et dont il ne se sera point aperçu ; et ainsi il lui sera pardonné.
\VS{19}C'est un sacrifice de culpabilité. Il s'est rendu coupable contre Yahweh.
\TextTitle{La restitution au jour du sacrifice de culpabilité\FTNTT{Lé. 7:1-7.}}
\VS{20}Yahweh parla aussi à Moïse, en disant :
\VS{21}Quand quelqu'un aura péché et aura commis une transgression contre Yahweh, en mentant à son prochain pour un dépôt, pour une chose qu'on aura mise entre ses mains, un vol, ou qu'il ait extorqué son prochain,
\VS{22}ou s'il a trouvé quelque chose perdue, et qu'il mente à ce sujet, ou s'il jure faussement concernant l'une des choses qu'un homme fait en péchant ;
\VS{23}quand il péchera et se rendra coupable, il rendra la chose qu'il a volée ou extorquée, ou le dépôt qui lui a été donné en garde, ou la chose perdue qu'il a trouvée,
\VS{24}ou tout ce dont il aura juré faussement. Il le restituera totalement, et il y ajoutera un cinquième ; il le donnera à celui à qui il appartenait, le jour de son sacrifice de culpabilité.
\VS{25}Et il amènera pour Yahweh, au prêtre le sacrifice de culpabilité, à savoir un bélier sans défaut, pris du troupeau, avec l'estimation que tu feras de la culpabilité.
\VS{26}Et le prêtre fera propitiation pour lui devant Yahweh, et il lui sera pardonné, quelle que soit la faute dont il se sera rendu coupable. 
\Chap{6}
\TextTitle{Loi de l'holocauste\FTNTT{Lé. 1:1-17.}}
\VerseOne{}Yahweh parla aussi à Moïse, en disant :
\VS{2}Ordonne à Aaron et à ses fils, et dis-leur : C'est ici la loi de l'holocauste. L'holocauste demeurera sur le foyer de l'autel toute la nuit jusqu'au matin, et le feu brûlera sur l'autel.
\VS{3}Et le prêtre revêtira sa tunique de lin, mettra ses caleçons de lin sur son corps, et il enlèvera la cendre de l'holocauste que le feu aura consumé sur l'autel, puis il la mettra près de l'autel.
\VS{4}Alors il ôtera ses vêtements et portera d'autres vêtements pour transporter les cendres hors du camp, dans un lieu pur.
\VS{5}Et quant au feu qui brûle sur l'autel, il continuera de brûler, on ne l'éteindra point ; le prêtre y brûlera du bois tous les matins, il préparera l'holocauste sur le bois, et y brûlera les graisses des offrandes de paix.
\VS{6}Le feu brûlera continuellement sur l'autel, on ne le laissera point s'éteindre.
\TextTitle{Loi de l'offrande de gâteau\FTNTT{Lé. 2:1-16.}}
\VS{7}Et c'est ici la loi de l'offrande de gâteau. Les fils d'Aaron l'offriront devant Yahweh sur l'autel\FTNT{No. 15:4.}.
\VS{8}Et on lèvera une poignée de la fine farine du gâteau et de son huile, avec tout l'encens qui est sur le gâteau, et on le brûlera sur l'autel, en bonne odeur, en mémorial à Yahweh.
\VS{9}Aaron et ses fils mangeront ce qui en restera ; ils le mangeront sans levain dans un lieu saint, ils le mangeront dans le parvis de la tente d'assignation\FTNT{Ex. 29:26-37.}.
\VS{10}On ne le cuira point avec du levain. Je leur ai donné cela pour leur portion d'entre mes offrandes consumées par le feu. C'est une chose très sainte, comme le sacrifice d'expiation et le sacrifice de culpabilité.
\VS{11}Tout mâle d'entre les fils d'Aaron en mangera. C'est une ordonnance perpétuelle pour vos descendants concernant les offrandes consumées par le feu à Yahweh : Quiconque les touchera sera sanctifié.
\VS{12}Yahweh parla aussi à Moïse, en disant :
\VS{13}C'est ici l'offrande d'Aaron et de ses fils, qu'ils offriront à Yahweh le jour où il sera oint : Un dixième d'épha de fine farine, comme offrande de gâteau perpétuelle, une moitié le matin et une moitié le soir.
\VS{14}Elle sera apprêtée sur une plaque avec de l'huile, tu l'apporteras mélangée, et tu offriras les morceaux cuits du gâteau en bonne odeur à Yahweh.
\VS{15}Et le prêtre, d'entre ses fils, qui sera oint à sa place, fera cela. C'est une ordonnance perpétuelle devant Yahweh : On le brûlera tout entier.
\VS{16}Tout le gâteau du prêtre sera entièrement consumé ; on n'en mangera pas.
\TextTitle{Loi de l'offrande pour le péché\FTNTT{Lé. 4:1-35.}}
\VS{17}Yahweh parla aussi à Moïse, en disant :
\VS{18}Parle à Aaron et à ses fils, et dis-leur : C'est ici la loi du sacrifice d'expiation. L'offrande pour l'expiation sera égorgée devant Yahweh, dans le même lieu où l'on égorge l'holocauste : C'est une chose très sainte.
\VS{19}Le prêtre qui offrira l'offrande pour l'expiation la mangera ; elle se mangera dans un lieu saint, dans le parvis de la tente d'assignation\FTNT{No. 18:10.}.
\VS{20}Quiconque touchera sa chair sera saint. Et s'il en jaillit du sang sur le vêtement, ce sur quoi il aura jailli sera lavé dans un lieu saint.
\VS{21}Et le vase de terre dans lequel on l'aura fait cuire sera brisé ; mais si on l'a fait cuire dans un vase d'airain, il sera nettoyé et lavé dans l'eau.
\VS{22}Tout mâle d'entre les prêtres en mangera ; car c'est une chose très sainte.
\VS{23}Aucune offrande pour le sacrifice d'expiation, dont on portera le sang dans la tente d'assignation pour faire la propitiation dans le sanctuaire, ne sera mangée, mais elle sera brûlée au feu\FTNT{Hé. 13:11.}.
\Chap{7}
\TextTitle{Loi du sacrifice de culpabilité\FTNTT{Lé. 5:1-26.}}
\VerseOne{}Or c'est ici la loi du sacrifice de culpabilité : C'est une chose très sainte.
\VS{2}Au même lieu où l'on égorgera l'holocauste, on égorgera le sacrifice de culpabilité. On en répandra le sang sur l'autel tout autour.
\VS{3}Puis on en offrira toute la graisse, avec la queue, et toute la graisse qui couvre les entrailles,
\VS{4}les deux rognons, la graisse qui est dessus sur les flancs, et le grand lobe qui est sur le foie, qu'on ôtera jusqu'aux rognons.
\VS{5}Le prêtre brûlera toutes ces choses sur l'autel comme offrande consumée par le feu à Yahweh : C'est un sacrifice pour la culpabilité.
\VS{6}Tout mâle d'entre les prêtres en mangera ; il sera mangé dans un lieu saint ; car c'est une chose très sainte.
\VS{7}Le sacrifice pour l'expiation sera semblable au sacrifice de culpabilité, il y aura une même loi pour les deux ; et la victime appartiendra au prêtre qui aura fait propitiation par elle.
\VS{8}Et le prêtre qui offrira l'holocauste de quelqu'un aura la peau de l'holocauste qu'il aura offert.
\VS{9}Et toute offrande de gâteau cuit au four, apprêtée sur le gril ou sur la plaque, appartiendra au prêtre qui l'offre.
\VS{10}Et toute offrande pétrie à l'huile, ou sèche, sera pour tous les fils d'Aaron, pour l'un comme pour l'autre.
\TextTitle{Loi du sacrifice d'offrande de paix\FTNTT{Lé. 3:1-17.}} 
\VS{11}Et c'est ici, la loi du sacrifice d'offrande de paix\FTNT{Voir commentaire en Lé. 3:1.} qu'on offrira à Yahweh.
\VS{12}Si quelqu'un l'offre pour un sacrifice de reconnaissance, il offrira avec le sacrifice de reconnaissance, des gâteaux sans levain pétris à l'huile, des galettes sans levain ointes d'huile, et des gâteaux de fine farine mêlés et pétris à l'huile.
\VS{13}En plus des gâteaux, il offrira pour son offrande du pain levé avec le sacrifice de reconnaissance de ses offrandes de paix.
\VS{14}Il présentera une part de chaque offrande, qu'il offrira comme offrande élevée à Yahweh ; elle sera pour le prêtre qui a répandu le sang du sacrifice d'offrande de paix.
\VS{15}Mais la chair du sacrifice de reconnaissance de ses offrandes de paix sera mangée le jour où elle sera offerte ; on n'en laissera rien jusqu'au matin.
\VS{16}Que si le sacrifice de son offrande est un vœu ou une offrande volontaire, son sacrifice sera mangé le jour où il l'aura offert ; ce qui en restera sera mangé le lendemain.
\VS{17}Mais ce qui restera de la chair du sacrifice sera brûlé au feu le troisième jour.
\VS{18}Que si on mange de la chair du sacrifice d'offrande de paix le troisième jour, celui qui l'aura offert ne sera point agréé, il ne lui sera point imputé, ce sera une chose infâme, et la personne qui en mangera portera son iniquité\FTNT{Ez. 4:14.}.
\VS{19}Et la chair de ce sacrifice qui a touché quelque chose d'impure ne sera point mangée, elle sera brûlée au feu. Mais quiconque sera pur, mangera de cette chair.
\VS{20}Car une personne qui mangera de la chair du sacrifice d'offrande de paix, laquelle appartient à Yahweh, et qui aura sur elle son impureté, cette personne-là sera retranchée de son peuple.
\VS{21}Si une personne touche quelque chose d'impure, soit une impureté d'homme, soit une bête impure, ou quelque autre chose impure, et qu'il mange de la chair du sacrifice d'offrande de paix qui appartient à Yahweh, cette personne-là sera retranchée d'entre son peuple.
\VS{22}Yahweh parla à Moïse, en disant :
\VS{23}Parle aux enfants d'Israël, et dis-leur : Vous ne mangerez aucune graisse de bœuf, ni d'agneau, ni de chèvre.
\VS{24}On pourra se servir pour un usage quelconque de la graisse d'une bête morte ou de la graisse d'une bête déchirée ; mais vous n'en mangerez point.
\VS{25}Car quiconque mangera de la graisse d'une bête que l'on offre comme offrande consumée par le feu à Yahweh, la personne qui en mangera, sera retranché de son peuple.
\VS{26}Vous ne mangerez point de sang, ni d'oiseaux, ni d'autres bêtes, dans aucune de vos demeures.
\VS{27}Toute personne, qui aura mangé de quelque sang que ce soit, sera retranchée de son peuple.
\VS{28}Yahweh parla à Moïse, en disant :
\VS{29}Parle aux enfants d'Israël, et dis-leur : Celui qui offrira son sacrifice d'offrande de paix à Yahweh, apportera son offrande à Yahweh, prise sur son sacrifice d'offrande de paix.
\VS{30}Il apportera de ses mains les offrandes consumées par le feu devant Yahweh. Il apportera la graisse avec la poitrine, la poitrine pour l'agiter d'un côté et de l'autre devant Yahweh.
\VS{31}Puis le prêtre brûlera la graisse sur l'autel, mais la poitrine sera pour Aaron et ses fils.
\VS{32}Vous donnerez aussi au prêtre pour offrande élevée, l'épaule droite de vos sacrifices d'offrande de paix\FTNT{No. 18:18.}.
\VS{33}Celui des fils d'Aaron qui offrira le sang et la graisse de l'offrande de paix, aura pour sa part l'épaule droite.
\VS{34}Car je prends sur les enfants d'Israël, la poitrine qu'on agite d'un côté et de l'autre, et l'épaule qu'on présente par élévation, de tous les sacrifices d'offrande de paix, et je les donne à Aaron le prêtre et à ses fils, par une ordonnance perpétuelle, de la part des fils d'Israël.
\VS{35}C'est là, le droit de l'onction d'Aaron et de l'onction de ses fils sur ces offrandes consumées par le feu devant Yahweh, depuis le jour où on les aura présentés pour exercer la sacrificature à Yahweh.
\VS{36}Et c'est ce que Yahweh ordonne aux enfants d'Israël de leur donner, depuis le jour où on les aura oints ; par une loi perpétuelle parmi leurs descendants\FTNT{Ex. 40:15.}.
\VS{37}Telle est donc la loi de l'holocauste, du gâteau, du sacrifice pour l'expiation, du sacrifice pour la culpabilité, de la consécration et du sacrifice d'offrande de paix.
\VS{38}Yahweh l'ordonna à Moïse sur la montagne de Sinaï, le jour où il ordonna aux enfants d'Israël d'offrir leurs offrandes à Yahweh dans le désert de Sinaï.
\Chap{8}
\TextTitle{Consécration d'Aaron et ses fils}
\VerseOne{}Yahweh parla aussi à Moïse en disant :
\VS{2}Prends Aaron et ses fils avec lui, les vêtements, l'huile d'onction, un jeune taureau pour le sacrifice d'expiation, deux béliers et une corbeille de pains sans levain\FTNT{Ex. 29:1-2 ; Ex. 30:25.} ;
\VS{3}et convoque toute l'assemblée à l'entrée de la tente d'assignation.
\VS{4}Et Moïse fit comme Yahweh lui avait ordonné ; et l'assemblée se rassembla à l'entrée de la tente d'assignation.
\VS{5}Moïse dit à l'assemblée : Voici ce que Yahweh a ordonné de faire.
\TextTitle{La purification avec l'eau} 
\VS{6}Et Moïse fit approcher Aaron et ses fils, et les lava avec de l'eau.
\TextTitle{Les vêtements d'Aaron} 
\VS{7}Et il mit sur Aaron la tunique, il le ceignit de la ceinture, le revêtit de la robe, mit sur lui l'éphod, et le ceignit avec la ceinture de l'éphod dont il le lia.
\VS{8}Puis il mit sur lui le pectoral, après avoir mis au pectoral l'urim et le thummim.
\VS{9}Il lui mit aussi la tiare sur la tête, et il mit sur le devant de la tiare la lame d'or, la couronne de sainteté, comme Yahweh l'avait ordonné à Moïse\FTNT{Ex. 28.}.
\TextTitle{L'onction d'huile} 
\VS{10}Puis Moïse prit l'huile d'onction, et oignit le tabernacle et toutes les choses qui y étaient, et les sanctifia.
\VS{11}Et il en fit l'aspersion sur l'autel par sept fois, et il oignit l'autel, tous ses ustensiles, et la cuve avec sa base, pour les sanctifier.
\VS{12}Il versa aussi de l'huile d'onction sur la tête d'Aaron, et l'oignit pour le sanctifier\FTNT{Ps. 133:2.}.
\TextTitle{Les vêtements des fils d'Aaron}
\VS{13}Puis Moïse fit approcher les fils d'Aaron, les revêtit des tuniques, les ceignit des ceintures et leur attacha des turbans, comme Yahweh l'avait ordonné à Moïse.
\TextTitle{Les offrandes et les sacrifices}
\VS{14}Alors il fit approcher le jeune taureau pour le sacrifice d'expiation, et Aaron et ses fils posèrent leurs mains sur la tête du taureau pour le sacrifice d'expiation.
\VS{15}Et Moïse l'égorgea, prit de son sang, et en mit avec son doigt sur les cornes de l'autel tout autour, et purifia l'autel ; et il répandit le reste du sang au pied de l'autel, ainsi il le sanctifia pour faire la propitiation sur lui.
\VS{16}Puis il prit toute la graisse qui était sur les entrailles, le grand lobe du foie, les deux rognons avec leur graisse, et Moïse les brûla sur l'autel.
\VS{17}Mais il brûla au feu, hors du camp, le jeune taureau avec sa peau, sa chair, et ses excréments, comme Yahweh l'avait ordonné à Moïse.
\VS{18}Il fit aussi approcher le bélier de l'holocauste, et Aaron et ses fils posèrent leurs mains sur la tête du bélier.
\VS{19}Et Moïse l'égorgea et répandit le sang sur l'autel tout autour.
\VS{20}Puis il coupa le bélier en morceaux, et Moïse brûla la tête, les morceaux, et la graisse.
\VS{21}Et il lava dans l'eau les entrailles et les jambes, et brûla tout le bélier sur l'autel : Ce fut un holocauste d'une agréable odeur, c'était une offrande consumée par le feu à Yahweh, comme Yahweh l'avait ordonné à Moïse.
\VS{22}Il fit aussi approcher l'autre bélier, le bélier de consécration, et Aaron et ses fils posèrent les mains sur la tête du bélier.
\VS{23}Et Moïse l'égorgea, prit de son sang, et le mit sur le lobe de l'oreille droite d'Aaron, et sur le pouce de sa main droite et sur le gros orteil de son pied droit.
\VS{24}Il fit aussi approcher les fils d'Aaron, et mit du même sang sur le lobe de leur oreille droite, et sur le pouce de leur main droite, et sur le gros orteil de leur pied droit, et Moïse répandit le reste du sang sur l'autel tout autour.
\VS{25}Après, il prit la graisse, la queue, toute la graisse qui est sur les entrailles, et le grand lobe du foie, et les deux rognons avec leur graisse, et l'épaule droite.
\VS{26}Il prit aussi de la corbeille des pains sans levain, qui étaient devant Yahweh, un gâteau sans levain, et un gâteau de pain fait à l'huile et une galette, et il les mit sur les graisses, et sur l'épaule droite.
\VS{27}Puis il mit toutes ces choses sur les paumes des mains d'Aaron et sur les paumes des mains de ses fils, et les agita d'un côté et de l'autre devant Yahweh.
\VS{28}Puis Moïse les prit de leurs mains et les brûla sur l'autel, sur l'holocauste : Ce fut l'offrande de consécration de bonne odeur, c'est une offrande consumée par le feu devant Yahweh.
\VS{29}Moïse prit aussi la poitrine du bélier de consécration, et l'agita d'un côté et de l'autre devant Yahweh : Ce fut la part de Moïse, comme Yahweh l'avait ordonné à Moïse.
\TextTitle{L'aspersion d'huile et de sang}
\VS{30}Moïse prit de l'huile d'onction et du sang qui était sur l'autel, et il en fit l'aspersion sur Aaron et sur ses vêtements, sur ses fils et sur les vêtements de ses fils ; ainsi il sanctifia Aaron et ses vêtements, les fils d'Aaron et les vêtements de ses fils.
\TextTitle{La nourriture de consécration\FTNTT{Ex. 29:26 ; Lé. 7:31-34 ; 8:29.}}
\VS{31}Après cela, Moïse dit à Aaron et à ses fils : Faites cuire la chair à l'entrée de la tente d'assignation, et vous la mangerez là, avec le pain qui est dans la corbeille de consécration, comme je l'ai ordonné, en disant : Aaron et ses fils la mangeront.
\VS{32}Mais vous brûlerez au feu ce qui restera de la chair et du pain.
\TextTitle{Les prêtres mis à part}
\VS{33}Et vous ne sortirez point pendant sept jours, de l'entrée de la tente d'assignation, jusqu'à ce que vos jours de consécration soient accomplis ; car on emploiera sept jours à vous consacrer.
\VS{34}Yahweh a ordonné de faire en ces autres jours comme on a fait en celui-ci, pour faire la propitiation en votre faveur.
\VS{35}Vous resterez donc pendant sept jours à l'entrée de la tente d'assignation, jour et nuit, et vous observerez ce que Yahweh vous a ordonné d'observer, afin que vous ne mouriez pas ; car il m'a été ainsi ordonné.
\VS{36}Ainsi Aaron et ses fils firent toutes les choses que Yahweh avait ordonnées par Moïse.
\Chap{9}
\TextTitle{Aaron et ses fils commencent leur service dans le tabernacle}
\VerseOne{}Et il arriva au huitième jour, que Moïse appela Aaron et ses fils, et les anciens d'Israël.
\VS{2}Et il dit à Aaron : Prends un jeune taureau du troupeau pour l'offrande d'expiation, et un bélier pour l'holocauste, tous deux sans défaut, et offre-les devant Yahweh.
\VS{3}Et tu parleras aux enfants d'Israël, en disant : Prenez un bouc pour l'offrande d'expiation, un jeune taureau et un agneau, tous deux d'un an et sans défaut, pour l'holocauste ;
\VS{4}un bœuf et un bélier pour l'offrande de paix\FTNT{Voir commentaire en Lé. 3:1.}, pour les sacrifier devant Yahweh ; et un gâteau pétri à l'huile. Car aujourd'hui Yahweh vous apparaîtra.
\VS{5}Ils prirent donc les choses que Moïse avait ordonné et les amenèrent devant la tente d'assignation, et toute l'assemblée s'approcha, et se tint devant Yahweh.
\VS{6}Et Moïse dit : Faites ce que Yahweh vous a ordonné, et la gloire de Yahweh vous apparaîtra.
\VS{7}Moïse dit à Aaron : Approche-toi de l'autel, fais ton sacrifice pour l'expiation et ton holocauste, et fais propitiation pour toi et pour le peuple ; présente l'offrande pour le peuple, et fais propitiation pour eux, comme Yahweh l'a ordonné\FTNT{Hé. 7:26-27.}.
\VS{8}Alors Aaron s'approcha de l'autel et égorgea le veau de son sacrifice d'expiation.
\VS{9}Et les fils d'Aaron lui présentèrent le sang, et il trempa son doigt dans le sang et le mit sur les cornes de l'autel ; puis il répandit le reste du sang au pied de l'autel.
\VS{10}Mais il brûla sur l'autel la graisse, et les rognons, et le grand lobe du foie de l'offrande pour le péché, comme Yahweh l'avait ordonné à Moïse.
\VS{11}Et il brûla au feu la chair et la peau hors du camp.
\VS{12}Il égorgea aussi l'holocauste. Les fils d'Aaron lui présentèrent le sang, lequel il répandit sur l'autel tout autour.
\VS{13}Puis ils lui présentèrent l'holocauste coupé en morceaux, avec la tête, et il les brûla sur l'autel.
\VS{14}Et il lava les entrailles et les jambes, qu'il brûla sur l'holocauste, sur l'autel.
\VS{15}Il offrit l'offrande du peuple. Il prit le bouc pour le sacrifice d'expiation du peuple, il l'égorgea et l'offrit pour le péché, comme la première offrande.
\VS{16}Il l'offrit en holocauste, faisant selon l'ordonnance.
\VS{17}Ensuite, il offrit l'offrande du gâteau, et il en remplit la paume de sa main, et la brûla sur l'autel, outre l'holocauste du matin.
\VS{18}Il égorgea aussi le bœuf et le bélier pour le sacrifice d'offrande de paix, qui était pour le peuple. Les fils d'Aaron lui présentèrent le sang, lequel il répandit sur l'autel tout autour.
\VS{19}Ils présentèrent la graisse du bœuf et du bélier, la queue, ce qui couvre les entrailles, les rognons, et le grand lobe du foie ;
\VS{20}ils mirent les graisses sur les poitrines, et il brûla les graisses sur l'autel.
\VS{21}Et Aaron agita d'un côté et de l'autre devant Yahweh les poitrines et l'épaule droite, comme Yahweh l'avait ordonné à Moïse.
\VS{22}Aaron éleva aussi ses mains vers le peuple, et le bénit. Puis il descendit, après avoir offert le sacrifice pour l'expiation, l'holocauste et l'offrande de paix.
\VS{23}Moïse donc et Aaron entrèrent dans la tente d'assignation, puis ils sortirent et ils bénirent le peuple. Et la gloire de Yahweh apparut à tout le peuple.
\VS{24}Car le feu sortit de devant Yahweh, et consuma sur l'autel l'holocauste et les graisses. Tout le peuple le vit et ils poussèrent des cris de joie et tombèrent sur leur face\FTNT{1 R. 18:38 ; 2 Ch. 7:1.}.
\Chap{10}
\TextTitle{Un feu étranger présenté à Yahweh}
\VerseOne{}Or les fils d'Aaron, Nadab et Abihu, prirent chacun leur encensoir, mirent du feu, et ils posèrent dessus du parfum ; ils offrirent devant Yahweh un feu étranger\FTNT{Ce passage nous avertit du danger auquel s'exposent ceux qui apportent un feu étranger dans le temple. Les feux étrangers sont les fausses doctrines, le péché, les conceptions cartésiennes, pernicieuses, mercantiles, destinés à remplacer la Parole de Dieu et à conduire le chrétien dans les ténèbres.}, ce qu'il ne leur avait point été ordonné.
\VS{2}Et le feu sortit de devant Yahweh, et les dévora ; ils moururent devant Yahweh\FTNT{No. 3:4.}.
\VS{3}Moïse dit à Aaron : C'est ce dont Yahweh avait parlé, en disant : Je serai sanctifié par ceux qui s'approchent de moi, et je serai glorifié en présence de tout le peuple. Et Aaron se tut.
\VS{4}Et Moïse appela Mischaël et Eltsaphan, les fils d'Uziel, oncle d'Aaron, et leur dit : Approchez-vous, emportez vos frères de devant le sanctuaire, hors du camp.
\VS{5}Alors ils s'approchèrent et les emportèrent avec leurs tuniques hors du camp, comme Moïse l'avait dit.
\TextTitle{Instructions données par Moïse} 
\VS{6}Puis Moïse dit à Aaron, à Eléazar et à Ithamar, ses fils : Ne découvrez point vos têtes, et ne déchirez point vos vêtements, de peur que vous ne mouriez, et que Yahweh ne se mette en colère contre toute l'assemblée. Mais que vos frères, toute la maison d'Israël, pleurent à cause de l'embrasement que Yahweh a allumé\FTNT{Ez. 24:17.}.
\VS{7}Et ne sortez point de l'entrée de la tente d'assignation, de peur que vous ne mouriez, car l'huile de l'onction de Yahweh est sur vous. Et ils firent selon la parole de Moïse.
\VS{8}Et Yahweh parla à Aaron, en disant :
\VS{9}Vous ne boirez point de vin, ni de boisson forte, ni toi ni tes fils avec toi, quand vous entrerez dans la tente d'assignation, de peur que vous ne mouriez ; c'est une ordonnance perpétuelle pour vos descendants\FTNT{No. 6:3 ; Jg. 13:7.},
\VS{10}afin que vous puissiez discerner entre ce qui est saint et ce qui est profane, entre ce qui est impur et ce qui est pur,
\VS{11}afin que vous enseigniez aux enfants d'Israël toutes les ordonnances que Yahweh leur a prononcées par Moïse.
\VS{12}Puis Moïse parla à Aaron, à Eléazar et à Ithamar, ses fils qui lui restaient : Prenez l'offrande de gâteau, leur dit-il, ce qui reste des offrandes de Yahweh consumées par le feu, et mangez-la avec des pains sans levain auprès de l'autel, car c'est une chose très sainte.
\VS{13}Vous la mangerez dans un lieu saint, parce que c'est la portion qui est assignée à toi et à tes fils sur les offrandes consumées par le feu à Yahweh ; car il m'a été ainsi ordonné.
\VS{14}Vous mangerez aussi la poitrine offerte par agitation et l'épaule présentée par élévation dans un lieu pur, toi, tes fils et tes filles avec toi ; car ces choses-là t'ont été données, dans les sacrifices d'offrande de paix\FTNT{Voir commentaire en Lé. 3:1.} des enfants d'Israël, comme ton droit et le droit de tes fils.
\VS{15}Ils apporteront l'épaule présentée par élévation et la poitrine offerte par agitation, avec les offrandes consumées par le feu, qui sont les graisses, pour les agiter en offrande çà et là devant Yahweh : Cela t'appartiendra, et à tes fils avec toi, par une ordonnance perpétuelle, comme Yahweh l'a ordonné.
\VS{16}Or Moïse cherchait soigneusement le bouc de l'offrande pour l'expiation mais voici, il avait été brûlé. Et Moïse se mit en grande colère contre Eléazar et Ithamar, les fils d'Aaron qui lui restaient, et leur dit :
\VS{17}Pourquoi n'avez-vous point mangé l'offrande pour l'expiation dans un lieu saint ? Car c'est une chose très sainte ; vu qu'elle vous a été donnée pour porter l'iniquité de l'assemblée, afin de faire propitiation pour eux devant Yahweh.
\VS{18}Voici, son sang n'a point été porté dans l'intérieur du sanctuaire ; ne manquez donc plus à la manger dans le lieu saint, comme je l'avais ordonné.
\VS{19}Alors Aaron répondit à Moise : Voici, ils ont offert aujourd'hui leur offrande pour l'expiation et leur holocauste devant Yahweh, et ces choses-ci me sont arrivées. Si j'avais mangé aujourd'hui l'offrande pour le péché, cela aurait-il plu à Yahweh ?
\VS{20}Et Moïse l'entendit, et cela fut bon à ses yeux.
\Chap{11}
\TextTitle{Lois de purification : les bêtes pures et impures}
\VerseOne{}Et Yahweh parla à Moïse et à Aaron, et leur dit :
\VS{2}Parlez aux enfants d'Israël, et dites-leur : Ce sont ici les bêtes dont vous mangerez d'entre toutes les bêtes qui sont sur la terre\FTNT{De. 14:4 ; Ac. 10:11-14.}.
\VS{3}Vous mangerez d'entre les bêtes de tous ceux qui ont le sabot fendu, qui ont le pied fourchu, et qui ruminent.
\VS{4}Mais vous ne mangerez point de celles qui ruminent uniquement, ou qui ont uniquement le sabot fendu : Comme le chameau, car il rumine mais il n'a point le sabot fendu : Il vous sera impur.
\VS{5}Et le lapin, car il rumine mais il n'a point le sabot fendu : Il vous sera impur.
\VS{6}Le lièvre, car il rumine mais il n'a point le sabot fendu : Il vous sera impur.
\VS{7}Le porc, car il a bien le sabot fendu et le pied fourchu, mais il ne rumine pas : Il vous sera impur.
\VS{8}Vous ne mangerez point de leur chair, même vous ne toucherez point leur cadavre : Ils vous seront impurs.
\VS{9}Vous mangerez de ceci d'entre tout ce qui est dans les eaux. Vous mangerez de tout ce qui a des nageoires et des écailles dans les eaux, soit dans la mer, soit dans les fleuves.
\VS{10}Mais vous ne mangerez rien de ce qui n'a point de nageoires et d'écailles, soit dans la mer, soit dans les fleuves, tant des reptiles des eaux, que de toute chose vivante qui est dans les eaux, cela vous sera en abomination.
\VS{11}Elles vous seront donc en abomination, vous ne mangerez point de leur chair, et vous tiendrez pour une chose abominable leur cadavre.
\VS{12}Tout ce donc qui vit dans les eaux et qui n'a point de nageoires et d'écailles, vous sera en abomination.
\VS{13}Et d'entre les oiseaux vous tiendrez ceux-ci pour abominables, on n'en mangera point, ils vous seront en abomination : L'aigle, l'orfraie, l'aigle de mer ;
\VS{14}le vautour, et le milan, selon leur espèce ;
\VS{15}tout corbeau, selon son espèce ;
\VS{16}l'autruche, le hibou, la mouette, et l'épervier selon leur espèce ;
\VS{17}le chat-huant, le plongeon, la chouette ;
\VS{18}le cygne, le cormoran, le pélican ;
\VS{19}la cigogne, le héron selon leur espèce, la huppe et la chauve-souris,
\VS{20}et tout reptile volant qui marche sur quatre pattes vous sera en abomination.
\VS{21}Mais, vous pourrez manger de toute chose rampante qui vole et qui va sur quatre pattes qui ont des jambes au-dessus de leurs pieds, pour sauter avec celles-ci sur la terre.
\VS{22}Ce sont donc ici ceux dont vous mangerez : La sauterelle selon son espèce, le solam\FTNT{« Solam », « hargol » et « hagab » sont diverses espèces de sauterelles.} selon son espèce, le hargol, selon son espèce et le hagab, selon son espèce.
\VS{23}Mais tout autre reptile volant qui a quatre pattes vous sera en abomination.
\VS{24}Vous serez donc impurs par ces bêtes ; quiconque touchera leur cadavre sera impur jusqu'au soir,
\VS{25}et quiconque aussi portera leur cadavre lavera ses vêtements et sera impur jusqu'au soir.
\VS{26}Toute bête qui a le sabot fendu, et qui n'a point le pied fourchu et ne rumine point, vous sera impur : Quiconque les touchera sera impur.
\VS{27}Tout ce qui marche sur ses pattes, entre tous les animaux qui marchent à quatre pieds, vous sera impur : Quiconque touchera leur cadavre sera impur jusqu'au soir,
\VS{28}et celui qui portera leur cadavre lavera ses vêtements et sera impur jusqu'au soir. Ils vous seront impurs.
\VS{29}Ceci aussi vous sera impur entre les reptiles, qui rampent sur la terre : La taupe, la souris et la tortue, selon leur espèce ;
\VS{30}le hérisson, la grenouille, le lézard, la limace et le caméléon.
\VS{31}Ces choses vous seront impures entre les reptiles : Quiconque les touchera mortes sera impur jusqu'au soir.
\VS{32}Aussi, tout ce sur quoi il en tombera quelque chose quand elles seront mortes sera impur, soit ustensile de bois, soit vêtement, soit peau, ou sac, quelque objet que ce soit dont on se sert pour faire quelque chose ; il sera mis dans l'eau, et sera impur jusqu'au soir ; puis il sera pur.
\VS{33}Mais s'il en tombe quelque chose dans quelque vase de terre que ce soit, tout ce qui est dedans sera impur, et vous casserez le vase.
\VS{34}Et tout aliment qu'on mange, sur lequel il y aura eu de cette eau, sera impur ; tout breuvage qu'on boit dans quelque vase que ce soit, en sera impur.
\VS{35}Et s'il tombe quelque chose de leur cadavre sur quoi que ce soit, cela sera impur ; le four et le foyer seront détruits : Ils seront impurs, et ils vous seront impurs.
\VS{36}Toutefois, la source, le puits ou tel autre amas d'eaux resteront purs ; mais celui donc qui touchera leur cadavre sera impur.
\VS{37}Et s'il est tombé de leur cadavre sur quelque semence qui se sème, elle restera pure.
\VS{38}Mais si on avait mis de l'eau sur la semence, et que quelque chose de leur cadavre tombe sur elle, elle vous sera impure.
\VS{39}Et si une des bêtes qui vous servent pour nourriture meurt, celui qui en touchera le cadavre sera impur jusqu'au soir ;
\VS{40}celui qui mangera de son cadavre lavera ses vêtements et sera impur jusqu'au soir, et celui aussi qui portera le cadavre de cette bête, lavera ses vêtements et sera impur jusqu'au soir.
\VS{41}Tout reptile donc qui rampe sur la terre vous sera en abomination ; et on n'en mangera point\FTNT{cp. Ge. 3:14.}.
\VS{42}Vous ne mangerez point de tout ce qui rampe sur la poitrine, ni de tout ce qui marche sur les quatre pieds, ni de tout ce qui a plusieurs pieds entre tous les reptiles qui se traînent sur la terre ; car ils seront en abomination.
\VS{43}Ne rendez point vos personnes abominables par aucun reptile qui se traîne ; ne vous rendez point impurs par eux, ne vous souillez point par eux.
\VS{44}Car je suis Yahweh, votre Dieu ; vous vous sanctifierez donc et vous serez saints, car je suis saint\FTNT{1 Pi. 1:16.} ! Ainsi, vous ne rendrez point vos personnes impures par aucun reptile qui se traîne sur la terre.
\VS{45}Car je suis Yahweh, qui vous ai fait monter du pays d'Egypte, afin que je sois votre Dieu, et que vous soyez saints ; car je suis saint !
\VS{46}Telle est la loi touchant les animaux, les oiseaux, tout être vivant, qui se meut dans les eaux, et toute être vivant, qui se rampe sur la terre,
\VS{47}afin de discerner entre la chose impure et la chose pure, entre les animaux qu'on peut manger et les animaux dont on ne doit point manger.
\Chap{12}
\TextTitle{Lois de purification : Le flux de sang\FTNTT{Ps. 51:7.}}
\VerseOne{}Yahweh parla aussi à Moïse, en disant :
\VS{2}Parle aux enfants d'Israël, et dis-leur : Si la femme après avoir conçu, enfante un mâle, elle sera impure pendant sept jours ; elle sera impure comme au temps de son indisposition menstruelle.
\VS{3}Et au huitième jour, on circoncira la chair du prépuce de l'enfant\FTNT{Les parents de Jésus ont observé cette loi (Lu. 2:21-24). Jn. 7:22.}.
\VS{4}Et elle demeurera trente-trois jours à se purifier de son sang ; elle ne touchera aucune chose sainte, et ne viendra point au sanctuaire, jusqu'à ce que les jours de sa purification soient accomplis.
\VS{5}Si elle enfante une fille, elle sera impure deux semaines, comme au temps de son indisposition menstruelle, et elle restera soixante-six jours à se purifier de son sang.
\VS{6}Après que le temps de sa purification sera accompli, soit pour un fils ou pour une fille, elle présentera au prêtre un agneau d'un an en holocauste, et un jeune pigeon ou une tourterelle en sacrifice d'expiation, à l'entrée de la tente d'assignation\FTNT{No. 6:10.}.
\VS{7}Et le prêtre offrira ces choses devant Yahweh, et fera propitiation pour elle ; et elle sera purifiée du flux de son sang. Telle est la loi pour celle qui enfante un fils ou une fille.
\VS{8}Et que, si elle n'a pas le moyen de trouver un agneau, alors elle prendra deux tourterelles ou deux jeunes pigeons, l'un pour l'holocauste, et l'autre pour le sacrifice d'expiation. Le prêtre fera propitiation pour elle, et elle sera pure.
\Chap{13}
\TextTitle{Lois de purification : La lèpre}
\VerseOne{}Yahweh parla aussi à Moïse et à Aaron, en disant :
\VS{2}L'homme qui aura sur la peau de son corps une tumeur, une dartre, ou une tache blanche, et que cela paraîtra sur la peau de son corps comme une plaie de lèpre, on l'amènera à Aaron, le prêtre, ou à l'un de ses fils prêtres.
\VS{3}Et le prêtre regardera la plaie qui est sur la peau du corps. Si le poil de la plaie est devenu blanc, et si la plaie, à la voir, est plus profonde que la peau du corps, c'est une plaie de lèpre : Le prêtre donc le regardera et le jugera impur.
\VS{4}Mais si la tache est blanche sur la peau du corps, et qu'à la voir, elle n'est point plus profonde que la peau, et si son poil n'est pas devenu blanc, le prêtre fera enfermer pendant sept jours celui qui a la plaie.
\VS{5}Et le prêtre la regardera le septième jour. Si à ses yeux la plaie s'est arrêtée, et qu'elle ne s'est point étendue sur la peau, le prêtre le fera renfermer pendant sept autres jours.
\VS{6}Et le prêtre la regardera une seconde fois le septième jour suivant. Si la plaie est devenue pâle, et qu'elle ne s'est point étendue sur la peau, le prêtre le jugera pur : C'est de la dartre ; il lavera ses vêtements, et sera pur.
\VS{7}Mais si la dartre s'est étendue sur la peau, après avoir été vu par le prêtre pour être jugé pur, il se fera examiner pour la seconde fois par le prêtre.
\VS{8}Le prêtre le regardera encore. S'il aperçoit que la dartre s'est étendue sur la peau, le prêtre le jugera impur : C'est de la lèpre.
\VS{9}Quand il y aura une plaie de lèpre sur un homme, on l'amènera au prêtre.
\VS{10}Le prêtre le regardera. Et s'il aperçoit qu'il y a une tumeur blanche sur la peau, que le poil est devenu blanc, et qu'il y a une trace de chair vive dans la tumeur,
\VS{11}c'est une lèpre invétérée dans la peau du corps : Le prêtre le jugera impur ; il ne le fera point enfermer, car il est jugé impur.
\VS{12}Si la lèpre fait une éruption sur la peau, et qu'elle couvre toute la peau de celui qui a la plaie, depuis la tête de cet homme jusqu'à ses pieds, partout où pourra voir le prêtre, le prêtre le regardera,
\VS{13}et si le prêtre voit que la lèpre couvre tout le corps de cet homme, alors il jugera pur celui qui a la plaie : La plaie est devenue toute blanche, il est pur.
\VS{14}Mais le jour où l'on apercevra de la chair vive, il sera impur ;
\VS{15}alors le prêtre regardera la chair vive, et le jugera impur : La chair vive est impure, c'est de la lèpre.
\VS{16}Si la chair vive se change et devient blanche, alors il viendra vers le prêtre ;
\VS{17}et le prêtre le regardera, et s'il aperçoit que la plaie est devenue blanche, le prêtre jugera pur celui qui a la plaie : Il est pur.
\VS{18}Si le corps a eu sur la peau un ulcère qui soit guéri,
\VS{19}et qu'à l'endroit où était l'ulcère il y ait une tumeur blanche, ou une tache blanche rougeâtre, il sera regardé par le prêtre.
\VS{20}Le prêtre donc la regardera. Et s'il aperçoit, qu'à la voir, elle paraît plus enfoncée que la peau, et que son poil est devenu blanc, alors le prêtre le jugera impur : C'est une plaie de lèpre qui a fait éruption dans l'ulcère.
\VS{21}Si le prêtre la regardant, voit que le poil n'est point blanc, et qu'elle n'est point plus enfoncée que la peau, mais qu'elle est devenue pâle, le prêtre le fera enfermer pendant sept jours.
\VS{22}Si elle s'est étendue sur la peau en quelque sorte que ce soit, le prêtre le jugera impur : C'est une plaie.
\VS{23}Mais si la tache est restée à la même place et ne s'est pas étendue, c'est une cicatrice d'ulcère : Ainsi le prêtre le jugera pur.
\VS{24}Si le corps a sur la peau une brûlure par le feu, et que la chair vive de la partie brûlée soit une tache blanche rougeâtre ou blanc seulement, le prêtre la regardera,
\VS{25}et si le poil est devenu blanc dans la tache, et qu'à la voir, elle est plus profonde que la peau, c'est de la lèpre, elle a fait éruption dans la brûlure ; le prêtre donc le jugera impur : C'est une plaie de lèpre.
\VS{26}Mais si le prêtre la regardant aperçoit qu'il n'y a point de poil blanc dans la tache, et qu'elle n'est point plus basse que la peau, qu'elle est devenue pâle, le prêtre le fera enfermer pendant sept jours.
\VS{27}Puis le prêtre la regardera le septième jour. Si la tache s'est étendue sur la peau, le prêtre le jugera impur : C'est une plaie de lèpre.
\VS{28}Si la tache est restée à la même place, ne s'est pas étendue, et est devenue pâle, c'est la tumeur de la brûlure ; et le prêtre le jugera pur ; c'est la cicatrice de la brûlure.
\VS{29}Si l'homme ou la femme a une plaie à la tête, ou l'homme à la barbe,
\VS{30}le prêtre regardera la plaie, et si à la voir, elle est plus profonde que la peau, et qu'il y ait en elle du poil jaunâtre et fin, le prêtre le jugera impur : C'est de la teigne, c'est une lèpre de la tête ou de la barbe.
\VS{31}Si le prêtre regardant la plaie de la teigne, voit qu'elle n'est point plus profonde que la peau, et n'a en elle aucun poil noir, le prêtre fera enfermer pendant sept jours celui qui a la plaie de la teigne.
\VS{32}Et le septième jour le prêtre regardera la plaie. Si la teigne ne s'est point étendue, qu'elle n'a aucun poil jaunâtre, et, qu'à voir la teigne, elle n'est pas plus profonde que la peau,
\VS{33}celui qui a la plaie de la teigne se rasera, mais il ne se rasera point à l'endroit de la teigne, et le prêtre fera enfermer pendant sept autres jours celui qui a la teigne.
\VS{34}Puis le prêtre regardera la teigne au septième jour. Si la teigne ne s'est point étendue sur la peau et, qu'à la voir, elle n'est point plus profonde que la peau, le prêtre le jugera pur, et cet homme lavera ses vêtements, et il sera pur.
\VS{35}Mais si la teigne s'est étendue sur la peau, après sa purification, le prêtre la regardera,
\VS{36}et si la teigne s'est étendue sur la peau, le prêtre ne cherchera point de poil jaunâtre : Il est impur.
\VS{37}Mais si la teigne s'est arrêtée, et qu'il y ait poussé du poil noir, la teigne est guérie : Il est pur, et le prêtre le jugera pur.
\VS{38}Si l'homme ou la femme ont sur la peau de leur corps des taches, des taches qui sont blanches,
\VS{39}le prêtre les regardera. Si sur la peau de leur corps il y a des taches d'un blanc pâle, c'est une tache blanche qui a fait éruption sur la peau : Il est donc pur.
\VS{40}Si l'homme a la tête dépouillée de cheveux, c'est un chauve : Il est pur.
\VS{41}Et si sa tête est dépouillée de cheveux du côté de son visage, c'est un front chauve : Il est pur.
\VS{42}Et si dans la partie chauve de devant ou de derrière, il y a une plaie d'un blanc rougeâtre, c'est une lèpre qui a fait éruption dans sa partie chauve de derrière ou de devant.
\VS{43}Et le prêtre le regardera. S'il aperçoit que la tumeur de la plaie est d'un blanc rougeâtre dans sa partie chauve de derrière ou de devant, semblable à la lèpre de la peau du corps,
\VS{44}l'homme est lépreux, il est impur : Le prêtre ne manquera pas de le juger impur ; sa plaie est à la tête.
\VS{45}Or le lépreux en qui sera la plaie aura ses vêtements déchirés, et sa tête nue ; et il se couvrira sur la lèvre de dessus et il criera : Impur ! Impur !
\VS{46}Pendant tout le temps qu'il aura cette plaie, il sera jugé impur : Il est impur. Il demeurera seul ; sa demeure sera hors du camp\FTNT{2 R. 7:3 ; La. 4:15 ; Lu. 17:12-13.}.
\VS{47}Et si le vêtement est infecté de la plaie de la lèpre, soit sur un vêtement de laine, soit sur un vêtement de lin,
\VS{48}à la chaîne ou à la trame du lin, ou de laine, sur la peau ou sur quelque ouvrage de peau,
\VS{49}et si cette plaie est verdâtre ou rougeâtre sur le vêtement ou sur la peau, à la chaîne ou à la trame, ou sur un objet quelconque de peau, ce sera une plaie de lèpre, et elle sera montrée au prêtre.
\VS{50}Et le prêtre regardera la plaie, et fera enfermer pendant sept jours celui qui a la plaie.
\VS{51}Et au septième jour, il regardera la plaie. Si la plaie s'est étendue sur le vêtement, à la chaîne ou à la trame, sur la peau ou sur quelque ouvrage de peau, la plaie est une lèpre invétérée : La chose est impure.
\VS{52}Il brûlera le vêtement, la chaîne ou la trame de laine ou de lin, et toutes les choses de peau, qui auront cette plaie, car c'est une lèpre rongeuse : Cela sera brûlé au feu.
\VS{53}Mais si le prêtre regarde, et que la plaie ne s'est point étendue sur le vêtement, sur la chaîne ou sur la trame, ou sur quelque objet de peau,
\VS{54}le prêtre ordonnera qu'on lave la chose où est la plaie, et il le fera enfermer pendant sept autres jours.
\VS{55}Si le prêtre, après qu'on aura fait laver la plaie, la regarde, et s'il aperçoit que la plaie n'a point changé sa couleur, et qu'elle ne s'est point étendue, c'est une chose impure : Tu la brûleras au feu ; c'est une partie de l'endroit ou de l'envers qui a été rongée.
\VS{56}Si le prêtre regarde, et aperçoit que la plaie est devenue pâle, après qu'on l'ait fait laver, il la déchirera du vêtement ou de la peau, de la chaîne ou de la trame.
\VS{57}Si elle paraît encore sur le vêtement, à la chaîne ou à la trame, ou sur quelque chose de peau, c'est une lèpre qui a fait éruption : Vous brûlerez au feu la chose où est la plaie.
\VS{58}Mais si tu as lavé le vêtement, la chaîne ou la trame, ou quelque chose de peau, et que la plaie s'en est allée, il sera lavé une seconde fois, puis il sera pur.
\VS{59}Telle est la loi sur la plaie de la lèpre sur un vêtement de laine ou de lin, la chaîne ou la trame, ou quelque chose de peau, pour la juger pure ou impure.
\Chap{14}
\TextTitle{Loi du lépreux pour le jour de sa purification}
\VerseOne{}Yahweh parla aussi à Moïse, en disant :
\VS{2}C'est ici la loi du lépreux pour le jour de sa purification. Il sera amené au prêtre\FTNT{Mt. 8:2-4 ; Mc. 1:42-44 ; Lu. 5:12-14.}.
\VS{3}Le prêtre sortira hors du camp et le regardera. Si la plaie de la lèpre du lépreux est guérie,
\VS{4}le prêtre ordonnera qu'on prenne pour celui qui doit être purifié, deux oiseaux vivants et purs, avec du bois de cèdre, du cramoisi et de l'hysope\FTNT{Ex. 12:22.}.
\VS{5}Et le prêtre ordonnera qu'on égorge l'un des oiseaux sur un vase de terre, sur de l'eau vive.
\VS{6}Puis il prendra l'oiseau vivant, le bois de cèdre, le cramoisi et l'hysope ; et il trempera toutes ces choses avec l'oiseau vivant, dans le sang de l'autre oiseau qui aura été égorgé sur de l'eau vive.
\VS{7}Il en fera sept fois l'aspersion sur celui qui doit être purifié de la lèpre. Il le déclarera pur, et il laissera aller par les champs, l'oiseau vivant.
\VS{8}Et celui qui doit être purifié lavera ses vêtements, rasera tout son poil, et se lavera dans l'eau ; et il sera pur. Ensuite il entrera dans le camp, mais il demeurera sept jours hors de sa tente.
\VS{9}Au septième jour, il rasera tout son poil, sa tête, sa barbe, les sourcils de ses yeux, tout son poil ; il rasera tout son poil ; puis il lavera ses vêtements et son corps, et il sera pur.
\VS{10}Et au huitième jour, il prendra deux agneaux sans défaut, une brebis d'un an sans défaut, et trois dixièmes de fine farine en offrande de gâteau, pétrie à l'huile, et un log d'huile.
\VS{11}Le prêtre qui fait la purification présentera celui qui doit être purifié et ces choses-là devant Yahweh, à l'entrée de la tente d'assignation.
\VS{12}Puis le prêtre prendra l'un des agneaux et l'offrira en sacrifice pour la culpabilité avec un log d'huile ; il agitera ces choses devant Yahweh, en offrande agitée.
\VS{13}Et il égorgera l'agneau au lieu où l'on égorge l'offrande pour l'expiation et l'holocauste, dans le lieu saint ; car le sacrifice pour la culpabilité appartient au prêtre, comme le sacrifice pour l'expiation ; c'est une chose très sainte.
\VS{14}Le prêtre prendra du sang de l'offrande pour la culpabilité ; il le mettra sur le lobe de l'oreille droite de celui qui doit être purifié, sur le pouce de sa main droite et sur le gros orteil de son pied droit.
\VS{15}Puis le prêtre prendra du log d'huile et en versera dans la paume de sa main gauche.
\VS{16}Et le prêtre trempera le doigt de sa main droite dans l'huile qui est dans sa paume gauche, et fera l'aspersion de l'huile avec son doigt sept fois devant Yahweh.
\VS{17}Et du reste de l'huile qui sera dans sa paume, le prêtre en mettra sur le lobe de l'oreille droite de celui qui doit être purifié, sur le pouce de sa main droite et sur le gros orteil de son pied droit, sur le sang pris de l'offrande pour la culpabilité.
\VS{18}Mais ce qui restera de l'huile sur la paume du prêtre, il le mettra sur la tête de celui qui doit être purifié ; et ainsi le prêtre fera propitiation pour lui devant Yahweh.
\VS{19}Ensuite le prêtre offrira le sacrifice pour l'expiation et fera propitiation pour celui qui doit être purifié de sa souillure. Puis il égorgera l'holocauste.
\VS{20}Le prêtre offrira l'holocauste et le gâteau sur l'autel, et fera propitiation pour celui qui doit être purifié, et il sera pur.
\VS{21}Mais s'il est pauvre et s'il n'a pas le moyen de fournir ces choses, il prendra un agneau en offrande agitée pour la culpabilité, afin de faire propitiation pour lui. Et un dixième de fine farine pétrie à l'huile pour le gâteau, avec un log d'huile.
\VS{22}Et deux tourterelles ou deux jeunes pigeons, selon ce qu'il pourra fournir, dont l'un sera pour le péché et l'autre pour l'holocauste.
\VS{23}Et le huitième jour de sa purification, il les apportera au prêtre, à l'entrée de la tente d'assignation, devant Yahweh.
\VS{24}Et le prêtre recevra l'agneau du sacrifice pour la culpabilité et le log d'huile, et les agitera devant Yahweh en offrande agitée.
\VS{25}Et il égorgera l'agneau du sacrifice pour la culpabilité. Puis le prêtre prendra du sang de l'offrande pour la culpabilité, il le mettra sur le lobe de l'oreille droite de celui qui doit être purifié, sur le pouce de sa main droite et sur le gros orteil de son pied droit.
\VS{26}Puis le prêtre versera de l'huile dans la paume de sa main gauche.
\VS{27}Et avec le doigt de sa main droite, il fera l'aspersion de l'huile qui est dans sa main gauche sept fois devant Yahweh.
\VS{28}Il mettra de cette huile qui est dans sa paume, sur le lobe de l'oreille droite de celui qui doit être purifié et sur le pouce de sa main droite et sur le gros orteil de son pied droit, sur le lieu du sang pris de l'offrande pour la culpabilité.
\VS{29}Après il mettra le reste de l'huile qui est dans sa paume sur la tête de celui qui doit être purifié, afin de faire propitiation pour lui devant Yahweh.
\VS{30}Puis il sacrifiera l'une des tourterelles ou l'un des jeunes pigeons, selon ce qu'il aura pu fournir.
\VS{31}De ce donc qu'il aura pu fournir, l'un sera pour le sacrifice d'expiation et l'autre pour l'holocauste, avec le gâteau ; ainsi le prêtre fera propitiation devant Yahweh pour celui qui doit être purifié.
\VS{32}Telle est la loi de celui qui a une plaie de lèpre, et dont les ressources sont insuffisantes à sa purification.
\TextTitle{Lois de purification d'une maison lépreuse}
\VS{33}Puis Yahweh parla à Moïse et à Aaron, en disant :
\VS{34}Quand vous serez entrés dans le pays de Canaan, que je vous donne en possession, si j'envoie une plaie de lèpre sur une maison du pays que vous posséderez,
\VS{35}celui à qui la maison appartiendra viendra et le fera savoir au prêtre, en disant : Il me semble que j'aperçois comme une plaie dans ma maison.
\VS{36}Alors le prêtre ordonnera qu'on vide la maison avant qu'il y entre pour regarder la plaie, afin que rien de ce qui est dans la maison ne soit impur, puis le prêtre entrera pour voir la maison.
\VS{37}Et il regardera la plaie. Si la plaie qui est sur les murs de la maison a des creux verdâtres ou rougeâtres, qui soient, à les voir, plus enfoncés que le mur ;
\VS{38}le prêtre sortira de la maison, à l'entrée, et fera fermer la maison pendant sept jours.
\VS{39}Au septième jour, le prêtre retournera et la regardera. Si la plaie s'est étendue sur les murs de la maison,
\VS{40}alors il ordonnera de retirer les pierres sur lesquelles est la plaie, et de les jeter hors de la ville, dans un lieu impur.
\VS{41}Il fera aussi racler l'enduit de la maison à l'intérieur, tout autour ; et l'enduit qu'on aura raclé, on le jettera hors de la ville, dans un lieu impur.
\VS{42}Puis on prendra d'autres pierres, et on les mettra à la place des premières pierres ; et on prendra d'autres mortiers pour recrépir la maison.
\VS{43}Mais si la plaie revient et fait éruption dans la maison, après avoir retiré les pierres, après avoir raclé et recrépi la maison,
\VS{44}le prêtre y entrera et la regardera. Si la plaie s'est étendue dans la maison, c'est une lèpre invétérée dans la maison : Elle est impure.
\VS{45}On démolira la maison, ses pierres, son bois, et tout le mortier de la maison ; et on les transportera hors de la ville, dans un lieu impur.
\VS{46}Si quelqu'un est entré dans la maison pendant tout le temps que le prêtre l'avait faite fermer, il sera impur jusqu'au soir.
\VS{47}Celui qui dormira dans cette maison lavera ses vêtements. Celui aussi qui mangera dans cette maison lavera ses vêtements.
\VS{48}Mais quand le prêtre y sera entré, et qu'il aura aperçu que la plaie ne s'est point étendue dans cette maison, après l'avoir recrépie, il jugera la maison pure, car sa plaie est guérie.
\VS{49}Alors il prendra pour purifier la maison deux oiseaux, du bois de cèdre, du cramoisi et de l'hysope.
\VS{50}Il égorgera l'un des oiseaux sur un vase de terre, sur de l'eau vive.
\VS{51}Il prendra le bois de cèdre, l'hysope, le cramoisi et l'oiseau vivant ; il trempera le tout dans le sang de l'oiseau qu'on aura égorgé et dans l'eau vive, puis il fera sept fois l'aspersion sur la maison.
\VS{52}Il purifiera la maison avec le sang de l'oiseau, avec l'eau vive, avec l'oiseau vivant, le bois de cèdre, l'hysope et le cramoisi.
\VS{53}Puis il laissera aller hors de la ville par les champs l'oiseau vivant. C'est ainsi qu'il fera propitiation pour la maison, et elle sera pure.
\VS{54}Telle est la loi pour toute plaie de lèpre et de teigne,
\VS{55}de lèpre de vêtement et de maison,
\VS{56}de tumeur, de dartre, et de tache ;
\VS{57}pour enseigner quand une chose est impure et quand elle est pure. Telle est la loi sur la lèpre.
\Chap{15}
\TextTitle{Lois de purification : Gonorrhée et flux menstruel\FTNTT{Jn. 13:3-10 ; Ep. 5:25-27 ; 1 Jn. 1:9.}}
\VerseOne{}Yahweh parla aussi à Moïse et à Aaron, en disant :
\VS{2}Parlez aux enfants d'Israël et dites-leur : Tout homme qui a une gonorrhée\FTNT{Gonorrhée : infection des organes génito-urinaires.} sera impur à cause de son flux.
\VS{3}Et telle sera l'impureté de son flux : Quand sa chair laissera aller son flux, ou que sa chair retiendra son flux, c'est son impureté.
\VS{4}Tout lit sur lequel se couchera celui qui est atteint d'un flux sera impur ; et toute chose sur laquelle il se sera assis sera impure.
\VS{5}L'homme aussi qui touchera son lit lavera ses vêtements et se lavera avec de l'eau ; et il sera impur jusqu'au soir.
\VS{6}Et celui qui s'assiéra sur quelque chose sur laquelle celui qui a ce flux s'est assis, lavera ses vêtements et se lavera dans l'eau, et il sera impur jusqu'au soir.
\VS{7}Et celui qui touchera la chair de celui qui a ce flux lavera ses vêtements et se lavera dans l'eau, et il sera impur jusqu'au soir.
\VS{8}Si celui qui a ce flux crache sur celui qui est pur, celui qui était pur lavera ses vêtements et se lavera dans l'eau, et il sera impur jusqu'au soir.
\VS{9}Toute monture que celui qui a ce flux aura montée sera impure.
\VS{10}Quiconque touchera quelque chose qui aura été sous lui sera impur jusqu'au soir ; et quiconque portera une telle chose lavera ses vêtements, et se lavera dans l'eau ; il sera impur jusqu'au soir.
\VS{11}Quiconque aura été touché par celui qui a ce flux, sans qu'il ait lavé ses mains dans l'eau, lavera ses vêtements et il se lavera dans l'eau, et il sera impur jusqu'au soir.
\VS{12}Et le vase de terre que celui qui a ce flux aura touché sera cassé, mais tout vase de bois sera lavé dans l'eau.
\VS{13}Or quand celui qui a ce flux sera purifié de son flux, il comptera sept jours pour sa purification ; il lavera ses vêtements et sa chair avec de l'eau vive, et ainsi il sera pur.
\VS{14}Au huitième jour, il prendra pour lui deux tourterelles ou deux jeunes pigeons, et il viendra devant Yahweh à l'entrée de la tente d'assignation, et les donnera au prêtre.
\VS{15}Et le prêtre les sacrifiera, l'un en sacrifice pour l'expiation et l'autre en holocauste ; ainsi le prêtre fera propitiation pour lui devant Yahweh à cause de son flux.
\VS{16}L'homme aussi duquel sera sortie de la semence lavera dans l'eau tout son corps, et il sera impur jusqu'au soir.
\VS{17}Et tout vêtement et toute peau sur lequel il y aura de la semence seront lavés dans l'eau, et seront impurs jusqu'au soir.
\VS{18}Même la femme qui couchera avec un tel homme se lavera dans l'eau avec son mari, et ils seront impurs jusqu'au soir.
\VS{19}Et quand la femme aura un flux, un flux de sang en sa chair, elle sera séparée sept jours. Quiconque la touchera sera impur jusqu'au soir\FTNT{Mt. 9:18-22 ; Mc. 5:21-34 ; Lu. 8:41-48.}.
\VS{20}Toute chose sur laquelle elle aura couché durant sa séparation sera impure, toute chose aussi sur laquelle elle aura été assise sera impure.
\VS{21}Quiconque aussi touchera le lit de cette femme lavera ses vêtements et se lavera dans l'eau, et il sera impur jusqu'au soir.
\VS{22}Et quiconque touchera quelque chose sur laquelle elle se sera assise lavera ses vêtements et se lavera dans l'eau, et il sera impur jusqu'au soir.
\VS{23}Même si la chose que quelqu'un aura touchée était sur le lit ou sur quelque chose sur laquelle elle était assise, quand quelqu'un aura touché cette chose-là, il sera impur jusqu'au soir.
\VS{24}Et si un homme a couché avec elle et que son impureté soit sur lui, il sera impur sept jours, et toute couche sur laquelle il dormira sera impure.
\VS{25}La femme qui aura un flux de sang pendant plusieurs jours, hors de l'époque de ses menstruations, ou dont le flux durera plus longtemps que l'époque de ses menstruations, sera impure tout le temps du flux de son impureté, comme au temps de sa séparation.
\VS{26}Toute couche sur laquelle elle couchera tous les jours de son flux lui sera comme la couche de sa séparation, et toute chose sur laquelle elle s'assiéra sera impure comme pour l'impureté de sa séparation.
\VS{27}Et quiconque aura touché ces choses-là sera impur ; il lavera ses vêtements et se lavera dans l'eau, et il sera impur jusqu'au soir.
\VS{28}Mais si elle est purifiée de son flux, elle comptera sept jours, et après elle sera pure.
\VS{29}Au huitième jour, elle prendra deux tourterelles ou deux jeunes pigeons, et les apportera au prêtre à l'entrée de la tente d'assignation.
\VS{30}Et le prêtre en sacrifiera l'un en sacrifice pour l'expiation et l'autre en holocauste ; ainsi le prêtre fera propitiation pour elle devant Yahweh, à cause du flux de son impureté.
\VS{31}Ainsi, vous séparerez les enfants d'Israël de leurs impuretés, et ils ne mourront point à cause de leurs impuretés, en rendant impur mon tabernacle, qui est au milieu d'eux.
\VS{32}Telle est la loi pour celui qui a une gonorrhée ou de celui duquel sort la semence qui le rend impur.
\VS{33}Telle est aussi la loi pour celle qui a son indisposition menstruelle ou de toute personne qui découle et qui a son flux, soit mâle, soit femelle, et de celui qui couche avec celle qui est impure.
\Chap{16}
\TextTitle{Expiation pour le prêtre, sa maison et le peuple.\FTNTT{Hé. 9:1-14.}}
\VerseOne{}Or Yahweh parla à Moïse après la mort des deux fils d'Aaron, qui moururent lorsqu'ils s'étaient approchés de la présence de Yahweh.
\VS{2}Yahweh donc dit à Moïse : Parle à Aaron, ton frère, et dis-lui qu'il n'entre point en tout temps dans le sanctuaire, au-dedans du voile, devant le propitiatoire qui est sur l'arche, afin qu'il ne meure point ; car j'apparaîtrai dans une nuée sur le propitiatoire.
\VS{3}Aaron entrera dans le sanctuaire de cette manière, après avoir offert un jeune taureau du troupeau pour le péché, et un bélier pour l'holocauste.
\VS{4}Il se revêtira de la sainte tunique de lin, et portera les caleçons de lin sur son corps ; il se ceindra de la ceinture de lin\FTNT{La ceinture de vérité (Ep. 6:14).}, et se couvrira la tête de la tiare\FTNT{La tiare, le casque du salut (Ep. 6:17).} de lin, qui sont les saints vêtements, et il s'en vêtira après avoir lavé son corps avec de l'eau\FTNT{Le lavement préfigure ici la régénération (Tit. 3:5).}.
\VS{5}Et il prendra de l'assemblée des enfants d'Israël deux jeunes boucs en offrande pour le péché et un bélier pour l'holocauste.
\VS{6}Puis Aaron offrira son veau en sacrifice pour l'expiation, et fera propitiation tant pour lui que pour sa maison.
\TextTitle{Les deux boucs expiatoires\FTNTT{2 Co. 5:21.}}
\VS{7}Et il prendra les deux boucs, et les présentera devant Yahweh, à l'entrée de la tente d'assignation.
\VS{8}Puis Aaron jettera le sort sur les deux boucs, un sort pour Yahweh et un sort pour le bouc qui doit être Azazel.
\VS{9}Et Aaron offrira le bouc sur lequel le sort sera échu pour Yahweh, et l'offrira en sacrifice pour l'expiation.
\VS{10}Mais le bouc sur lequel le sort sera tombé pour être Azazel, sera présenté vivant devant Yahweh pour faire propitiation par lui, et on l'enverra dans le désert pour être Azazel.
\VS{11}Aaron donc, présentera le veau en sacrifice pour l'expiation, et fera propitiation pour lui et pour sa maison. Et il égorgera, dis-je, son veau qui est le sacrifice pour l'expiation.
\VS{12}Puis il prendra un encensoir plein de charbons ardents, de dessus l'autel devant Yahweh, et deux poignées de parfum odoriférant en poudre ; et il les apportera au-dedans du voile ;
\VS{13}et il mettra le parfum sur le feu devant Yahweh, afin que la nuée du parfum couvre le propitiatoire qui est sur le témoignage, ainsi il ne mourra point.
\VS{14}Il prendra aussi du sang du veau, et il en fera l'aspersion avec son doigt au-devant du propitiatoire vers l'orient ; il fera l'aspersion de ce sang-là sept fois avec son doigt devant le propitiatoire.
\VS{15}Il égorgera aussi le bouc du peuple, qui est l'offrande pour l'expiation, et il apportera son sang au-dedans du voile. Il fera de son sang comme il a fait du sang du veau, en faisant l'aspersion sur le propitiatoire et sur le devant du propitiatoire.
\VS{16}Et il fera propitiation pour le sanctuaire, le purifiant des impuretés des enfants d'Israël, et de leurs transgressions, selon tous leurs péchés. Il fera la même chose pour la tente d'assignation, qui demeure avec eux au milieu de leurs impuretés.
\VS{17}Et personne ne sera dans la tente d'assignation quand le prêtre y entrera pour faire propitiation dans le sanctuaire, jusqu'à ce qu'il en sorte, lorsqu'il fera propitiation pour lui et pour sa maison, et pour toute l'assemblée d'Israël.
\VS{18}Puis il sortira vers l'autel qui est devant Yahweh, et fera propitiation pour lui ; il prendra du sang du veau et du sang du bouc, il le mettra sur les cornes de l'autel tout autour.
\VS{19}Et il fera par sept fois l'aspersion du sang avec son doigt sur l'autel, et le purifiera et le sanctifiera des impuretés des enfants d'Israël.
\VS{20}Et quand il achèvera de faire propitiation pour le sanctuaire, pour la tente d'assignation et pour l'autel, alors il offrira le bouc vivant.
\VS{21}Et Aaron posera ses deux mains sur la tête du bouc vivant, et il confessera sur lui toutes les iniquités des enfants d'Israël et toutes leurs transgressions, selon tous leurs péchés ; et il les mettra sur la tête du bouc, et l'enverra au désert par un homme prêt pour cela.
\VS{22}Et le bouc portera sur lui toutes leurs iniquités dans une terre inhabitable, puis cet homme laissera aller le bouc par le désert.
\VS{23}Et Aaron reviendra dans la tente d'assignation ; il quittera les vêtements de lin dont il s'était vêtu quand il était entré dans le sanctuaire, et les posera là.
\VS{24}Il lavera aussi son corps avec de l'eau dans le lieu saint, et se revêtira de ses vêtements. Puis il sortira, il offrira son holocauste et l'holocauste du peuple, et fera propitiation pour lui et pour le peuple.
\VS{25}Il brûlera aussi sur l'autel la graisse de l'offrande pour le péché.
\VS{26}Et celui qui aura conduit le bouc pour être Azazel lavera ses vêtements et son corps avec de l'eau ; après cela, il rentrera dans le camp.
\VS{27}Mais on tirera hors du camp le veau et le bouc qui auront été offerts en sacrifice pour l'expiation, et desquels le sang aura été porté dans le sanctuaire pour y faire propitiation, et on brûlera au feu leurs peaux, leur chair et leurs excréments\FTNT{Hé. 13:11.}.
\VS{28}Et celui qui les aura brûlés lavera ses vêtements et son corps avec de l'eau ; après cela, il rentrera dans le camp.
\VS{29}Et ceci sera pour vous une ordonnance perpétuelle : Le dixième jour du septième mois, vous affligerez vos âmes, et vous ne ferez aucune œuvre, tant celui qui est du pays que l'étranger qui fait son séjour parmi vous\FTNT{La fête des expiations (ou yom kippour) avait lieu une fois par an, le dixième jour du septième mois (Ex. 30:10 ; Lé. 16:29). A cette occasion, le grand prêtre jetait le sort sur deux boucs : un sort pour Yahweh et un sort pour Azazel (Lé. 16:8-10). Le bouc pour Yahweh était sacrifié, il préfigurait la mort expiatoire de Christ. Le bouc émissaire, pour Azazel, n'avait lui-même rien fait de mal, mais il était choisi par Dieu pour porter le péché du peuple afin qu'il soit dégagé de toute accusation. Ce que l'on faisait de ce bouc, préfigurait l'œuvre de Jésus-Christ. Il symbolisait le Seigneur qui s'est chargé de nos péchés pour les emporter loin de nous (Es. 53 ; Ps. 103:12 ; Hé. 10:17 ; Hé. 13:12-14). Christ est mort et ressuscité hors du camp et c'est là qu'il nous appelle à le rejoindre : hors du monde et des systèmes religieux (Hé. 13:10-14).}.
\VS{30}Car en ce jour-là le prêtre fera propitiation pour vous, afin de vous purifier : Ainsi vous serez purifiés de tous vos péchés devant Yahweh.
\VS{31}Ce sera pour vous donc un sabbat, un jour de repos, et vous affligerez vos âmes. C'est une ordonnance perpétuelle.
\VS{32}Et le prêtre qu'on aura oint, et qu'on aura consacré pour exercer la sacrificature à la place de son père, fera propitiation, s'étant revêtu des vêtements de lin, qui sont les saints vêtements.
\VS{33}Et il fera propitiation pour le saint sanctuaire et il fera propitiation pour la tente d'assignation et pour l'autel, et pour les prêtres et pour tout le peuple de l'assemblée.
\VS{34}Ceci donc sera pour vous une ordonnance perpétuelle, afin de faire propitiation pour les enfants d'Israël de tous leurs péchés une fois par an. On fit comme Yahweh l'avait ordonné à Moïse.
\Chap{17}
\TextTitle{Les sacrifices apportés à l'entrée de la tente d'assignation}
\VerseOne{}Yahweh parla aussi à Moïse, en disant :
\VS{2}Parle à Aaron et à ses fils, et à tous les enfants d'Israël, et dis-leur : C'est ici ce que Yahweh a ordonné, en disant :
\VS{3}Quiconque de la maison d'Israël aura égorgé un bœuf, un agneau ou une chèvre dans le camp, ou qui l'aura égorgé hors du camp\FTNT{De. 12:6.},
\VS{4}et ne l'aura point amené à l'entrée de la tente d'assignation, pour en faire une offrande à Yahweh, devant le tabernacle de Yahweh, le sang sera imputé à cet homme-là ; il a répandu du sang, c'est pourquoi cet homme-là sera retranché du milieu de son peuple.
\VS{5}C'est afin que les enfants d'Israël amènent leurs sacrifices, qu'ils sacrifient dans les champs, qu'ils les amènent à Yahweh, à l'entrée de la tente d'assignation, vers le prêtre, et qu'ils les sacrifient en sacrifices d'offrande de paix\FTNT{Voir commentaire en Lé. 3:1.} à Yahweh ;
\VS{6}et que le prêtre en répande le sang sur l'autel de Yahweh, à l'entrée de la tente d'assignation, et en brûle la graisse en bonne odeur à Yahweh.
\VS{7}Et qu'ils n'offrent plus leurs sacrifices aux démons, avec lesquels ils se sont prostitués. Ceci leur sera une ordonnance perpétuelle pour eux et leurs descendants\FTNT{De. 32:17 ; Ps. 106:37.}.
\VS{8}Tu leur diras donc : Si un homme de la maison d'Israël, ou des étrangers qui font leur séjour parmi eux, aura offert un holocauste ou un sacrifice,
\VS{9}et qui ne l'aura point amené à l'entrée de la tente d'assignation, pour le sacrifier à Yahweh, cet homme-là sera retranché d'entre ses peuples.
\TextTitle{Importance du sang}
\VS{10}Quiconque de la maison d'Israël ou des étrangers qui font leur séjour parmi eux, aura mangé de quelque sang que ce soit, je mettrai ma face contre cette personne qui aura mangé du sang, et je la retrancherai du milieu de son peuple\FTNT{Ge. 9:4 ; De. 12:16-23 ; 1 S. 14:33.}.
\VS{11}Car l'âme de la chair est dans le sang. C'est pourquoi je vous ai ordonné qu'il soit mis sur l'autel, afin de faire propitiation pour vos âmes, car c'est le sang qui fera propitiation pour l'âme.
\VS{12}C'est pourquoi j'ai dit aux enfants d'Israël : Que personne d'entre vous ne mange du sang, que même l'étranger qui fait son séjour parmi vous ne mange point de sang.
\VS{13}Et quiconque des enfants d'Israël, et des étrangers qui font leur séjour parmi eux, aura pris à la chasse une bête sauvage ou un oiseau que l'on mange, il répandra leur sang et le couvrira de poussière.
\VS{14}Car l'âme de toute chair est dans son sang, c'est son âme. C'est pourquoi j'ai dit aux enfants d'Israël : Vous ne mangerez point le sang d'aucune chair ; car l'âme de toute chair est son sang : Quiconque en mangera sera retranché.
\VS{15}Et toute personne qui aura mangé de la chair de quelque bête morte d'elle-même ou déchirée par les bêtes sauvages, tant celui qui est né dans le pays que l'étranger, lavera ses vêtements et se lavera avec de l'eau, et il sera impur jusqu'au soir ; puis il sera pur.
\VS{16}S'il ne lave pas ses vêtements et son corps, il portera son iniquité.
\Chap{18}
\TextTitle{Condamnation des incestes}
\VerseOne{}Yahweh parla encore à Moïse, en disant :
\VS{2}Parle aux enfants d'Israël et dis-leur : Je suis Yahweh, votre Dieu.
\VS{3}Vous ne ferez point ce qui se fait dans le pays d'Egypte où vous avez habité, ni ce qui se fait dans le pays de Canaan, auquel je vous amène : Vous ne vivrez point selon leurs statuts\FTNT{Jé. 10:2.}.
\VS{4}Mais vous ferez selon mes statuts, et vous garderez mes ordonnances pour marcher en elles. Je suis Yahweh, votre Dieu.
\VS{5}Vous garderez donc mes statuts et mes ordonnances, l'homme qui les pratiquera vivra par elles. Je suis Yahweh\FTNT{Ez. 20:11-13 ; Ga. 3:12 ; Ro. 10:5.}.
\VS{6}Que nul ne s'approche de celle qui est sa proche parente pour découvrir sa nudité. Je suis Yahweh.
\VS{7}Tu ne découvriras point la nudité de ton père, ni la nudité de ta mère. C'est ta mère ; tu ne découvriras point sa nudité.
\VS{8}Tu ne découvriras point la nudité de la femme de ton père. C'est la nudité de ton père\FTNT{De. 22:30 ; 1 Co. 5:1.}.
\VS{9}Tu ne découvriras point la nudité de ta sœur, fille de ton père ou fille de ta mère, née dans la maison ou hors de la maison. Tu ne découvriras point leur nudité.
\VS{10}Quant à la nudité de la fille de ton fils ou de la fille de ta fille, tu ne découvriras point leur nudité. Car elles sont ta nudité.
\VS{11}Tu ne découvriras point la nudité de la fille de la femme de ton père, née de ton père. C'est ta sœur.
\VS{12}Tu ne découvriras point la nudité de la sœur de ton père. Elle est la proche parente de ton père.
\VS{13}Tu ne découvriras point la nudité de la sœur de ta mère ; car elle est la proche parente de ta mère.
\VS{14}Tu ne découvriras point la nudité du frère de ton père. Et tu ne t'approcheras point de sa femme. Elle est ta tante.
\VS{15}Tu ne découvriras point la nudité de ta belle-fille. Elle est la femme de ton fils ; tu ne découvriras point sa nudité.
\VS{16}Tu ne découvriras point la nudité de la femme de ton frère. C'est la nudité de ton frère.
\VS{17}Tu ne découvriras point la nudité d'une femme et de sa fille. Et tu ne prendras point la fille de son fils, ni la fille de sa fille pour découvrir leur nudité. Elles sont tes proches parentes : C'est un crime.
\VS{18}Tu ne prendras point aussi une femme avec sa sœur pour exciter une rivalité en découvrant sa nudité à côté d'elle pendant sa vie.
\TextTitle{Condamnation des abominations}
\VS{19}Tu ne t'approcheras point d'une femme durant son impureté menstruelle, pour découvrir sa nudité.
\VS{20}Tu ne coucheras point avec la femme de ton prochain pour te souiller avec elle\FTNT{Ex. 20:17 ; De. 5:21 ; Mt. 5:28.}.
\VS{21}Tu ne donneras point tes enfants pour les faire passer par le feu devant Moloc\FTNT{Moloc est le nom du dieu auquel les Ammonites, peuple issu de la relation incestueuse de Loth et sa fille, sacrifiaient leurs premiers-nés en les jetant dans un brasier. De. 18:9-10 ; 1 R. 11:5-7 ; 2 R. 23:10 ; Jé. 32:35.}, et tu ne profaneras point le nom de ton Dieu. Je suis Yahweh.
\VS{22}Tu ne coucheras pas aussi avec un homme, comme on couche avec une femme. C'est une abomination\FTNT{1 Co. 6:9-10 ; Ge. 13:13 ; Ro. 1:26-27.}.
\VS{23}Tu ne coucheras point aussi avec une bête pour te souiller avec elle ; et la femme ne se prostituera point à une bête ; c'est une confusion\FTNT{1 Co. 6:9-10 ; Ro. 1:26-27.}.
\VS{24}Ne vous rendez point impurs par aucune de ces choses, car les nations que je vais chasser de devant vous se sont rendues impures par toutes ces choses.
\VS{25}Le pays a été rendu impur ; et je punirai sur lui son iniquité, et le pays vomira ses habitants.
\VS{26}Mais quant à vous, vous garderez mes ordonnances et mes jugements, et vous ne ferez aucune de ces abominations, tant celui qui est né dans le pays que l'étranger qui fait son séjour parmi vous.
\VS{27}Car les gens de ce pays-là qui ont été avant vous, ont fait toutes ces abominations, et le pays en a été rendu impur.
\VS{28}Prenez garde que le pays ne vous vomisse, si vous le rendez impur, comme il aura vomi les nations qui y étaient avant vous.
\VS{29}Car tous ceux qui feront l'une de toutes ces abominations, seront retranchés du milieu de leur peuple.
\VS{30}Vous garderez donc ce que j'ai ordonné de garder, et vous ne pratiquerez aucune de ces coutumes abominables qui ont été pratiquées avant vous, et vous ne vous rendrez point impurs par elles. Je suis Yahweh, votre Dieu.
\Chap{19}
\TextTitle{Mise en garde contre l'idolâtrie}
\VerseOne{}Yahweh parla aussi à Moïse, en disant :
\VS{2}Parle à toute l'assemblée des enfants d'Israël, et dis-leur : Soyez saints, car je suis saint, moi, Yahweh, votre Dieu.
\VS{3}Chacun de vous craindra sa mère et son père, et vous garderez mes sabbats. Je suis Yahweh, votre Dieu\FTNT{Ex. 20:12 ; De. 5:16 ; Mt. 15:4.}.
\VS{4}Vous ne vous tournerez point vers les idoles, et vous ne vous ferez aucun dieu de fonte. Je suis Yahweh, votre Dieu\FTNT{Ex. 20:3-5.}.
\TextTitle{Recommandation pour les sacrifices}
\VS{5}Si vous offrez un sacrifice d'offrande de paix\FTNT{Voir commentaire en Lé. 3:1.} à Yahweh, vous le sacrifierez de votre bon gré.
\VS{6}II se mangera le jour où vous l'aurez sacrifié, et le lendemain, mais ce qui restera jusqu'au troisième jour sera brûlé au feu.
\VS{7}Si on en mange au troisième jour, ce sera une abomination : Il ne sera point agréé.
\VS{8}Quiconque aussi en mangera portera son iniquité ; car il aura profané la chose sainte de Yahweh : Cette personne-là sera retranchée d'entre ses peuples.
\TextTitle{La justice de Yahweh, l'amour pour son prochain}
\VS{9}Quand vous ferez la moisson de votre pays, tu n'achèveras point de moissonner le bout de ton champ, et tu ne glaneras point ce qui restera à cueillir de ta moisson.
\VS{10}Tu ne grappilleras point ta vigne, ni ne recueilleras point les grains tombés de ta vigne, mais tu les laisseras au pauvre et à l'étranger\FTNT{De. 24:19.}. Je suis Yahweh, votre Dieu.
\VS{11}Vous ne déroberez point, et vous ne vous tromperez point les uns les autres ; et aucun de vous ne mentira à son prochain\FTNT{Ex. 20:15 ; Ep. 4:25 ; Col. 3:9.}.
\VS{12}Vous ne jurerez point par mon Nom en mentant, car tu profanerais le Nom de ton Dieu\FTNT{Ex. 20:7 ; De. 5:11.}. Je suis Yahweh.
\VS{13}Tu n'opprimeras point ton prochain, et tu ne le pilleras point\FTNT{De. 24:14-15 ; Ja. 5:4.}. Le salaire de ton mercenaire ne demeurera point chez toi jusqu'au lendemain.
\VS{14}Tu ne maudiras point le sourd, et tu ne mettras point d'achoppement devant l'aveugle, mais tu craindras ton Dieu. Je suis Yahweh.
\VS{15}Vous ne ferez point d'iniquité dans vos jugements : Tu n'auras point d'égard à la personne du pauvre, et tu n'honoreras point la personne du grand, mais tu jugeras ton prochain selon la justice.
\VS{16}Tu ne répandras point de calomnies parmi ton peuple. Tu ne t'élèveras point contre le sang de ton prochain. Je suis Yahweh.
\VS{17}Tu ne haïras point ton frère dans ton cœur ; tu reprendras soigneusement ton prochain\FTNT{Ge. 4:8 ; Mt. 18:15 ; 1 Jn. 2:9-11.}, et tu ne te chargeras point d'un péché à cause de lui.
\VS{18}Tu n'useras point de vengeance, et tu ne la garderas point aux enfants de ton peuple ; mais tu aimeras ton prochain comme toi-même\FTNT{Mt. 7:12 ; Mc. 12:28-34.}. Je suis Yahweh.
\VS{19}Vous garderez mes ordonnances. Tu n'accoupleras point tes bêtes de deux espèces différentes ; tu ne sèmeras point ton champ de diverses sortes de grains ; et tu ne mettras point sur toi de vêtements de diverses espèces, comme de la laine et du lin.
\VS{20}Si un homme couche et a commerce avec une femme, si c'est une esclave, fiancée à un homme, qui n'a pas été rachetée, et que la liberté ne lui a pas été donnée, ils auront le fouet, mais on ne les fera point mourir, parce qu'elle n'a pas été affranchie.
\VS{21}L'homme amènera son sacrifice pour la culpabilité à Yahweh à l'entrée de la tente d'assignation, à savoir un bélier pour la culpabilité.
\VS{22}Et le prêtre fera propitiation pour lui devant Yahweh par le bélier du sacrifice pour la culpabilité, à cause de son péché qu'il aura commis, et son péché qu'il aura commis lui sera pardonné.
\TextTitle{Ordonnances diverses}
\VS{23}Et quand vous serez entrés dans le pays, et que vous y aurez planté quelque arbre fruitier, vous considérerez son fruit comme incirconcis ; il vous sera incirconcis pendant trois ans, on n'en mangera point.
\VS{24}Mais à la quatrième année, tout son fruit sera une chose sainte à la louange de Yahweh.
\VS{25}Et à la cinquième année, vous mangerez son fruit, afin qu'il vous multiplie son produit. Je suis Yahweh, votre Dieu.
\VS{26}Vous ne mangerez rien avec le sang. Vous n'userez point de divinations, et vous ne pronostiquerez point le temps\FTNT{De. 12:23.}.
\VS{27}Vous ne couperez point en rond les coins de votre chevelure, et vous ne raserez point les coins de votre barbe.
\VS{28}Vous ne ferez point d'incisions dans votre chair pour un mort, et vous n'imprimerez point de caractères sur vous. Je suis Yahweh.
\VS{29}Tu ne profaneras point ta fille en la prostituant ; afin que le pays ne se prostitue point et ne se remplisse point de crimes.
\VS{30}Vous garderez mes sabbats et vous aurez en révérence mon sanctuaire. Je suis Yahweh.
\VS{31}Ne vous tournez point vers ceux qui évoquent les morts, ni vers les devins\FTNT{Ac. 16:16.} ; ne cherchez point à vous rendre impurs avec eux. Je suis Yahweh, votre Dieu.
\VS{32}Lève-toi devant les cheveux blancs, et tu honoreras la personne du vieillard. Tu craindras ton Dieu. Je suis Yahweh.
\VS{33}Si quelque étranger séjourne dans votre pays, vous ne lui ferez point de tort.
\VS{34}L'étranger qui séjourne parmi vous, vous sera comme celui qui est né parmi vous, et vous l'aimerez comme vous-mêmes, car vous avez été étrangers dans le pays d'Egypte. Je suis Yahweh, votre Dieu.
\VS{35}Vous ne ferez point d'iniquité dans les jugements, ni dans les mesures de dimension, ni dans les poids, ni dans les mesures de capacité.
\VS{36}Vous aurez les balances justes, les pierres à peser justes, l'épha juste et le hin juste. Je suis Yahweh, votre Dieu, qui vous ai fait sortir du pays d'Egypte.
\VS{37}Gardez donc toutes mes ordonnances et mes jugements, et pratiquez-les. Je suis Yahweh.
\Chap{20}
\TextTitle{Abominations diverses et leurs châtiments}
\VerseOne{}Yahweh parla aussi à Moïse, en disant :
\VS{2}Tu diras aux enfants d'Israël : Quiconque des enfants d'Israël ou des étrangers qui demeurent en Israël, qui donnera de sa postérité à Moloc, sera puni de mort : Le peuple du pays le lapidera.
\VS{3}Et je mettrai ma face contre un tel homme, et je le retrancherai du milieu de son peuple, parce qu'il a donné de sa postérité à Moloc, pour rendre impur mon sanctuaire et profaner le Nom de ma sainteté.
\VS{4}Si le peuple du pays ferme les yeux en quelque manière que ce soit sur cet homme-là, qui donne de sa postérité à Moloc, et s'il ne le fait pas mourir,
\VS{5}je mettrai ma face contre cet homme-là, contre sa famille, et je le retrancherai du milieu de mon peuple, avec tous ceux qui se prostituent comme lui, en se prostituant après Moloc.
\VS{6}Quant à la personne qui se tournera vers ceux qui évoquent les morts, vers les devins, en se prostituant après eux, je mettrai ma face contre cette personne-là, et je la retrancherai du milieu de son peuple.
\VS{7}Sanctifiez-vous donc, et soyez saints, car je suis Yahweh, votre Dieu.
\VS{8}Gardez aussi mes lois et pratiquez-les. Je suis Yahweh, qui vous sanctifie.
\VS{9}Un homme qui maudit son père ou sa mère sera puni de mort ; il a maudit son père ou sa mère : Son sang retombera sur lui.
\VS{10}Quant à l'homme qui commet un adultère avec la femme d'un autre, parce qu'il a commis un adultère avec la femme de son prochain, l'homme et la femme adultères seront mis à mort.
\VS{11}L'homme qui couche avec la femme de son père, découvre la nudité de son père, les deux seront mis à mort, leur sang est sur eux.
\VS{12}Quant un homme couche avec sa belle-fille, ils seront mis à mort, tous deux ; ils ont fait une confusion : Leur sang est sur eux.
\VS{13}Quant un homme couche avec un homme comme on couche avec une femme, ils ont tous deux fait une chose abominable ; ils seront mis à mort : Leur sang est sur eux.
\VS{14}Et si un homme prend pour femmes la fille et la mère, c'est un crime : Il sera brûlé au feu avec elles, afin que ce crime n'existe pas au milieu de vous.
\VS{15}Si un homme couche avec une bête, il sera puni de mort ; et vous tuerez aussi la bête.
\VS{16}Et si une femme s'approche d'une bête, tu tueras cette femme et la bête ; ils seront mis à mort : Leur sang sera sur eux.
\VS{17}Si un homme prend sa sœur, fille de son père ou fille de sa mère, et voit sa nudité, et qu'elle voit la nudité de cet homme, c'est une chose infâme ; ils seront donc retranchés sous les yeux des fils de leur peuple : Il a découvert la nudité de sa sœur, il portera son iniquité.
\VS{18}Si un homme couche avec une femme qui a son indisposition menstruelle, et qu'il découvre la nudité de cette femme, en découvrant son flux, et qu'elle découvre le flux de son sang, ils seront tous deux retranchés du milieu de leur peuple.
\VS{19}Tu ne découvriras point la nudité de la sœur de ta mère, ni de la sœur de ton père, car c'est découvrir sa proche parente, ils porteront tous deux leur iniquité.
\VS{20}Si un homme couche avec sa tante, il a découvert la nudité de son oncle ; ils porteront leur péché, et ils mourront privés d'enfants.
\VS{21}Si un homme prend la femme de son frère, c'est une impureté ; il a découvert la nudité de son frère, ils seront privés d'enfants.
\VS{22}Vous garderez toutes mes ordonnances et mes jugements et vous les pratiquerez, afin que le pays où je vous fais entrer pour y habiter ne vous vomisse point.
\VS{23}Vous ne suivrez point les statuts des nations que je vais chasser devant vous ; car elles ont fait toutes ces choses-là, et je les ai eues en abomination.
\VS{24}Et je vous ai dit : Vous posséderez leur pays, je vous le donnerai en possession : C'est un pays où coulent le lait et le miel. Je suis Yahweh, votre Dieu, qui vous ai séparés des autres peuples.
\VS{25}C'est pourquoi séparez les bêtes pures de celles qui sont impures, les oiseaux purs de ceux qui sont impurs, et ne rendez point abominables vos personnes en mangeant des bêtes et des oiseaux impurs, ni rien qui rampe sur la terre, rien de ce que je vous ai défendu comme une chose impure.
\VS{26}Vous me serez donc saints, car je suis saint, moi, Yahweh ; je vous ai séparés des autres peuples afin que vous soyez à moi.
\VS{27}Si un homme ou une femme évoquent les morts ou se livrent à la divination, on les mettra à mort ; on les lapidera : Leur sang sera sur eux.
\Chap{21}
\TextTitle{Recommandations aux prêtres}
\VerseOne{}Yahweh dit aussi à Moïse : Parle aux prêtres, fils d'Aaron, et dis-leur : Aucun d'eux ne se rendra impur parmi son peuple pour un mort,
\VS{2}excepté pour son proche parent, pour sa mère, pour son père, pour son fils, pour sa fille, et pour son frère,
\VS{3}et aussi pour sa sœur vierge, qui lui est proche, et qui n'aura point eu de mari, il se rendra impur pour elle.
\VS{4}Chef parmi son peuple, il ne se rendra point impur en se profanant.
\VS{5}Ils ne se feront point de place chauve sur la tête, ils ne raseront point les coins de leur barbe, ni ne feront point d'incisions dans leur chair.
\VS{6}Ils seront consacrés à leur Dieu, et ils ne profaneront point le Nom de leur Dieu ; car ils offrent à Yahweh les sacrifices consumés par le feu, qui sont la nourriture de leur Dieu : C'est pourquoi ils seront très saints.
\VS{7}Ils ne prendront point une femme prostituée ou déshonorée ; ils ne prendront point une femme répudiée par son mari, car ils sont saints pour leur Dieu.
\VS{8}Tu regarderas chacun d'eux comme saint, parce qu'ils offrent la nourriture de ton Dieu ; ils seront saints, car je suis saint, moi, Yahweh, qui vous sanctifie.
\VS{9}Si la fille du prêtre se profane en se prostituant, elle déshonore son père : Qu'elle soit brûlée au feu.
\VS{10}Le grand prêtre d'entre ses frères, sur la tête duquel l'huile d'onction a été répandue, et qui se sera consacré pour vêtir les saints vêtements, ne découvrira point sa tête et ne déchirera point ses vêtements.
\VS{11}Il n'ira vers aucune personne morte, il ne se rendra point impur pour son père ni pour sa mère.
\VS{12}Il ne sortira point du sanctuaire, et ne profanera point le sanctuaire de son Dieu ; car l'huile d'onction de son Dieu est une couronne sur lui. Je suis Yahweh.
\VS{13}Il prendra pour femme une vierge.
\VS{14}Il ne prendra point une veuve, ni une répudiée, ni une femme déshonorée ou prostituée ; mais il prendra pour femme une vierge parmi son peuple.
\VS{15}Il ne profanera point sa postérité parmi son peuple ; car je suis Yahweh qui le sanctifie.
\VS{16}Yahweh parla aussi à Moïse, en disant :
\VS{17}Parle à Aaron, et dis-lui : Si quelqu'un de ta postérité, parmi tes descendants, qui a quelque défaut corporel, il ne s'approchera point pour offrir la nourriture de son Dieu.
\VS{18}Car tout homme en qui il y aura un défaut n'en approchera point ; l'homme aveugle, boiteux, ayant le nez camus ou qui aura un membre allongé ;
\VS{19}ou l'homme qui aura une fracture aux pieds ou aux mains ;
\VS{20}ou qui sera bossu ou grêle, qui aura une tache à l'œil, qui aura une gale sèche, une dartre, ou qui aura les testicules écrasés.
\VS{21}Nul homme de la postérité d'Aaron, le prêtre, en qui il y aura un défaut corporel, ne s'approchera pour offrir les offrandes consumées par le feu à Yahweh ; il y a un défaut en lui, il ne s'approchera donc point pour offrir la nourriture de son Dieu.
\VS{22}Il pourra manger la nourriture de son Dieu, des choses très saintes et des choses saintes.
\VS{23}Mais il n'entrera point vers le voile, ni ne s'approchera point de l'autel, car il a un défaut corporel, et il ne profanera point mes sanctuaires, car je suis Yahweh, qui les sanctifie.
\VS{24}Moïse parla ainsi à Aaron et à ses fils, et à tous les enfants d'Israël.
\Chap{22}
\TextTitle{Consécration d'Aaron et de ses fils}
\VerseOne{}Puis Yahweh parla à Moïse, en disant :
\VS{2}Parle à Aaron et à ses fils, afin qu'ils s'abstiennent des choses saintes des enfants d'Israël, et qu'ils ne profanent point le Nom de ma sainteté dans les choses qu'ils me consacrent. Je suis Yahweh.
\VS{3}Dis-leur donc : Tout homme parmi votre génération et de vos descendants qui, étant impur, s'approchera des choses saintes que les enfants d'Israël auront sanctifiées à Yahweh, cette personne-là sera retranchée de devant moi. Je suis Yahweh.
\VS{4}Tout homme de la postérité d'Aaron, qui aura la lèpre ou une gonorrhée, ne mangera point des choses saintes jusqu'à ce qu'il soit pur. Il en sera de même pour celui qui touchera quelqu'un s'étant rendu impur en touchant un mort, ou celui qui aura une perte séminale,
\VS{5}et celui qui touchera un reptile et qui en aura été impur, ou un homme atteint d'une impureté quelconque, il en sera rendu impur.
\VS{6}La personne qui touchera ces choses sera rendu impur jusqu'au soir ; il ne mangera point des choses saintes s'il n'a point lavé son corps dans l'eau ;
\VS{7}Ensuite il sera pur après le coucher du soleil, et il mangera des choses saintes, car c'est sa nourriture.
\VS{8}Il ne mangera de la chair d'aucune bête morte d'elle-même ou déchirée par les bêtes sauvages, pour se rendre impur par elle. Je suis Yahweh.
\VS{9}Ils garderont ce que j'ai ordonné de garder, et ils ne commettront point de péché au sujet de la nourriture sainte, afin qu'ils ne meurent point, pour l'avoir profanée. Je suis Yahweh, qui les sanctifie.
\VS{10}Aucun étranger ne mangera des choses saintes ; l'étranger logé chez le prêtre et le mercenaire ne mangeront point des choses saintes.
\VS{11}Mais si le prêtre achète une personne avec son argent, elle en mangera, de même pour celui qui sera né dans sa maison ; ils mangeront de sa nourriture.
\VS{12}Si la fille du prêtre est mariée à un homme étranger, elle ne mangera point des choses saintes présentées en offrande par élévation.
\VS{13}Mais si la fille du prêtre est veuve ou répudiée, et si elle n'a point d'enfants, et est retournée dans la maison de son père, comme dans sa jeunesse, elle mangera de la nourriture de son père. Mais aucun étranger n'en mangera.
\VS{14}Si quelqu'un, pèche involontairement en mangeant d'une chose sainte, il y ajoutera un cinquième et le donnera au prêtre avec la chose sainte.
\VS{15}Et ils ne profaneront point les choses sanctifiées des enfants d'Israël, qu'ils auront offertes à Yahweh.
\VS{16}Mais on leur fera porter la peine du péché, parce qu'ils auront mangé de leurs choses saintes : Car je suis Yahweh, qui les sanctifie.
\TextTitle{Des animaux sans défaut pour les sacrifices\FTNTT{Hé. 9:14.}}
\VS{17}Yahweh parla encore à Moïse, en disant :
\VS{18}Parle à Aaron, à ses fils, et à tous les enfants d'Israël, et dis-leur : Quiconque de la maison d'Israël ou des étrangers qui sont en Israël, offrira son offrande, selon tous ses vœux, ou toutes ses offrandes volontaires, qu'on offre en holocauste à Yahweh,
\VS{19}il offrira de son bon gré, un mâle sans défaut, parmi les bœufs, les agneaux ou les chèvres.
\VS{20}Vous n'offrirez aucune chose qui ait un défaut, car elle ne serait point agréée pour vous.
\VS{21}Si un homme offre à Yahweh un sacrifice d'offrande de paix\FTNT{Voir commentaire en Lé. 3:1.} en s'acquittant d'un vœu, ou en faisant une offrande volontaire, soit de gros ou de menu bétail, elle sera sans défaut pour être agréée ; il ne doit y avoir aucun défaut.
\VS{22}Vous n'offrirez point à Yahweh ce qui sera aveugle, estropié, ou mutilé, qui ait un ulcère, une gale sèche ou une dartre ; et vous n'en ferez point sur l'autel un sacrifice consumé par le feu pour Yahweh.
\VS{23}Tu pourras bien faire une offrande volontaire d'un bœuf, ou d'une brebis, ou d'une chèvre ayant quelques membres allongés, ou quelque défaut dans ses membres, mais ils ne seront point agréés pour le vœu.
\VS{24}Vous n'offrirez point à Yahweh, et ne sacrifierez point dans votre pays un animal qui ait les testicules froissés, cassés, arrachés ou taillés.
\VS{25}Vous ne prendrez point de la main de l'étranger aucune de toutes ces choses pour les offrir comme nourriture à votre Dieu ; car la corruption qui est en eux est un défaut en elles : Elles ne seront point agréées pour vous.
\VS{26}Yahweh parla encore à Moïse, en disant :
\VS{27}Quand un veau, un agneau ou une chèvre seront nés, et qu'ils auront été sept jours sous leur mère, depuis le huitième jour et les suivants, ils seront agréables pour l'offrande du sacrifice consumé par le feu à Yahweh.
\VS{28}Vous n'égorgerez point aussi en un même jour la vache, ou la brebis, ou la chèvre, avec son petit.
\VS{29}Quand vous offrirez un sacrifice de remerciement à Yahweh, vous le sacrifierez de votre bon gré.
\VS{30}Il sera mangé le jour même ; vous n'en laisserez rien jusqu'au matin. Je suis Yahweh.
\VS{31}Gardez mes commandements et pratiquez-les. Je suis Yahweh.
\VS{32}Ne profanez point le nom de ma sainteté, car je serai sanctifié entre les enfants d'Israël. Je suis Yahweh, qui vous sanctifie,
\VS{33}et qui vous ai fait sortir du pays d'Egypte, pour être votre Dieu. Je suis Yahweh.
\Chap{23}
\TextTitle{Les fêtes de Yahweh}
\VerseOne{}Yahweh parla aussi à Moïse en disant :
\VS{2}Parle aux enfants d'Israël et dis-leur : Les fêtes\FTNT{Les fêtes de Yahweh étaient des jours solennels, c'est-à-dire des temps fixés pour s'approcher de Dieu et présenter des sacrifices (Voir le tableau en annexe « Les 7 fêtes de Yahweh » et également le dictionnaire).} solennelles de Yahweh, que vous publierez, seront de saintes convocations. Ce sont ici mes fêtes solennelles.
\VS{3}On travaillera six jours ; mais au septième jour, qui est le sabbat, le jour du repos, il y aura une sainte convocation. Vous ne ferez aucune œuvre, car c'est le sabbat à Yahweh, dans toutes vos demeures.
\TextTitle{La Pâque}
\VS{4}Ce sont ici les fêtes solennelles de Yahweh, qui seront de saintes convocations, que vous publierez en leur saison.
\VS{5}Au premier mois, le quatorzième jour du mois, entre les deux soirs, sera la Pâque\FTNT{La pâque était une fête qui commémorait la sortie d'Egypte (Ex. 12:1-14). Elle préfigurait la rédemption en Jésus-Christ, notre Pâque (1 Co. 5:7). Elle était fixée au 14ème jour du mois de Nisan, le premier mois.} à Yahweh.
\TextTitle{La fête des pains sans levain\FTNTT{Ex. 12:18 ; 13:6-8 ; 1 Co. 11:23-26.}}
\VS{6}Et le quinzième jour de ce même mois, sera la fête solennelle des pains sans levain\FTNT{La fête des pains sans levain commençait le 15ème jour du même mois (Nisan) et durait sept jours. Elle annonçait Christ, notre Pain descendu du ciel (Jn. 6:32-35). Seul le Seigneur Jésus a été sans levain, c'est-à-dire sans aucun péché. Le croyant est sauvé à la Pâque de Christ et doit vivre une vie sans péché (la fête des pains sans levain).} à Yahweh ; vous mangerez des pains sans levain pendant sept jours.
\VS{7}Le premier jour, vous aurez une sainte convocation : Vous ne ferez aucune œuvre servile.
\VS{8}Mais vous offrirez à Yahweh pendant sept jours des offrandes consumées par le feu. Et au septième jour, il y aura une sainte convocation : Vous ne ferez aucune œuvre servile.
\TextTitle{La fête des prémices\FTNTT{1 Co. 15:23.}}
\VS{9}Yahweh parla aussi à Moïse, en disant :
\VS{10}Parle aux enfants d'Israël et dis-leur : Quand vous serez entrés dans le pays que je vous donne, et que vous en aurez fait la moisson, vous apporterez alors au prêtre une gerbe des premiers fruits\FTNT{La fête des prémices annonce d'abord la résurrection du Seigneur Jésus-Christ, ensuite celle de tous ceux qui lui appartiennent (1 Th. 4:13-18 ; 1 Co. 15:23). Elle commençait le premier jour de la semaine suivant le sabbat de la Pâque, au mois de Nisan.} de votre moisson.
\VS{11}Et il agitera cette gerbe-là devant Yahweh, afin qu'elle soit agréée pour vous : Le prêtre l'agitera le lendemain du sabbat.
\VS{12}Et le jour où vous agiterez cette gerbe, vous sacrifierez un agneau sans défaut et d'un an, en holocauste à Yahweh ;
\VS{13}et le gâteau de cet holocauste sera de deux dixièmes de fine farine, pétrie à l'huile, pour offrande consumée par le feu, en bonne odeur à Yahweh ; et sa libation de vin sera le quart d'un hin.
\VS{14}Vous ne mangerez ni pain, ni grain rôti, ni grain en épi, jusqu'à ce jour-là, même jusqu'à ce que vous ayez apporté l'offrande à votre Dieu. C'est une loi perpétuelle pour vos descendants, dans toutes vos demeures.
\TextTitle{La Pentecôte ou la fête des semaines}
\VS{15}Vous compterez aussi dès le lendemain du sabbat, à savoir dès le jour où vous aurez apporté la gerbe qu'on doit agiter, sept semaines entières.
\VS{16}Vous compterez donc cinquante jours\FTNT{La fête des semaines ou fête de la moisson est désignée également comme la Pentecôte. Elle avait lieu au mois de Sivan et préfigurait l'effusion du Saint-Esprit et l'inauguration de la Nouvelle Alliance (Ac. 2:1-4). Le levain autorisé lors de cette fête évoquait par avance la présence de l'ivraie, symbole du péché et des fils du malin, parmi le blé, c'est-à-dire les enfants de Dieu (Mt. 13:24-41). Cinquante jours séparent la Pâque de la Pentecôte. Cet intervalle correspond exactement à la période séparant la résurrection du Seigneur Jésus-Christ de la naissance de l'Eglise (Ac. 2:1-4).} jusqu'au lendemain du septième sabbat ; et vous offrirez à Yahweh un gâteau nouveau.
\VS{17}Vous apporterez de vos demeures deux pains pour en faire une offrande agitée, ils seront de deux dixièmes, et de fine farine, pétris avec du levain : Ce sont les premiers fruits à Yahweh.
\VS{18}Vous offrirez aussi avec ce pain-là sept agneaux sans défaut et d'un an, un jeune taureau pris du troupeau et deux béliers, qui seront un holocauste à Yahweh, avec leurs gâteaux et leurs libations, des sacrifices consumés par le feu, en bonne odeur à Yahweh.
\VS{19}Vous sacrifierez aussi un jeune bouc en sacrifice pour l'expiation, et deux agneaux d'un an pour le sacrifice d'offrande de paix\FTNT{Voir commentaire en Lé. 3:1.}.
\VS{20}Et le prêtre les agitera avec le pain des premiers fruits, et avec les deux agneaux, en offrande agitée devant Yahweh : Ils seront saints à Yahweh, pour le prêtre.
\VS{21}Vous publierez donc, en ce même jour-là, une sainte convocation : Vous ne ferez aucune œuvre servile. C'est une ordonnance perpétuelle dans toutes vos demeures, pour vos descendants.
\VS{22}Et quand vous ferez la moisson de votre pays, tu n'achèveras point de moissonner le bout de ton champ, et tu ne glaneras point les épis qui resteront de ta moisson. Mais tu les laisseras pour le pauvre et pour l'étranger. Je suis Yahweh, votre Dieu.
\TextTitle{La fête des trompettes}
\VS{23}Yahweh parla aussi à Moïse, en disant :
\VS{24}Parle aux enfants d'Israël et dis-leur : Au septième mois, le premier jour du mois, il y aura un jour de repos pour vous, un mémorial de jubilation\FTNT{La fête des trompettes préfigure le rassemblement futur du peuple d'Israël après sa longue dispersion et l'enlèvement de l'Eglise. Cette fête était fixée au premier jour du septième mois (Tishri).}, et une sainte convocation.
\VS{25}Vous ne ferez aucune œuvre servile, et vous offrirez à Yahweh des offrandes consumées par le feu.
\TextTitle{Le jour des expiations\FTNTT{Hé. 9:1-16.}}
\VS{26}Yahweh parla aussi à Moïse, en disant :
\VS{27}Pareillement en ce même mois, qui est le septième, le dixième jour sera le jour des expiations\FTNT{Le jour des expiations ou du grand pardon (Voir Lé. 16) était célébré le dixième jour du septième mois (Tishri). Le Seigneur Jésus-Christ a fait l'expiation de nos péchés afin de nous amener à Dieu. Le propitiatoire au lieu d'être le trône du jugement, devenait ainsi le lieu de rencontre de Dieu avec le croyant (Ex. 25:22). Christ est la propitiation pour nos péchés (1 Jn. 2:2), mais il est aussi lui-même le propitiatoire (Ro. 3:25). Le péché ôté, les fautes confessées, le pardon acquis, l'holocauste offert, le chemin est ouvert pour la joie de la fête des tabernacles.} : Vous aurez une sainte convocation, vous humilierez vos âmes, et vous offrirez à Yahweh des sacrifices consumés par le feu.
\VS{28}En ce jour-là, vous ne ferez aucune œuvre, car c'est le jour des expiations, afin de faire propitiation pour vous devant Yahweh, votre Dieu.
\VS{29}Toute personne qui ne s'humiliera point en ce jour-là sera retranchée d'entre son peuple.
\VS{30}Et toute personne qui aura fait quelque œuvre en ce jour-là, je ferai périr cette personne-là du milieu de son peuple.
\VS{31}Vous ne ferez donc aucune œuvre. C'est une ordonnance perpétuelle pour vos descendants dans toutes vos demeures.
\VS{32}Ce sera pour vous un sabbat, un jour de repos, et vous humilierez vos âmes. Le neuvième jour du mois, au soir, depuis le soir jusqu'à l'autre soir, vous célébrerez votre sabbat.
\TextTitle{La fête des tabernacles\FTNTT{Esd. 3:4.}}
\VS{33}Yahweh parla aussi à Moïse, en disant :
\VS{34}Parle aux enfants d'Israël, et dis-leur : Au quinzième jour de ce septième mois sera la fête solennelle des tabernacles\FTNT{La fête des tabernacles ou des récoltes, était la fête du souvenir et de la joie. Célébrée au mois de Tishri, elle était aussi celle du repos, dans l'accomplissement des promesses. Elle préfigure le Royaume millénaire (Za. 14).} pendant sept jours, à Yahweh.
\VS{35}Au premier jour, il y aura une sainte convocation : Vous ne ferez aucune œuvre servile.
\VS{36}Pendant sept jours, vous offrirez à Yahweh des offrandes consumées par le feu. Et au huitième jour, vous aurez une sainte convocation, et vous offrirez à Yahweh des offrandes consumées par le feu ; ce sera une assemblée solennelle : Vous ne ferez aucune œuvre servile.
\VS{37}Ce sont là les fêtes solennelles de Yahweh, que vous publierez pour être des convocations saintes, afin d'offrir à Yahweh des offrandes consumées par le feu ; à savoir un holocauste, un gâteau, un sacrifice et une libation, chacune de ces choses en son jour ;
\VS{38}outre les sabbats de Yahweh, et outre vos dons, outre tous vos vœux, outre toutes les offrandes volontaires que vous présenterez à Yahweh.
\VS{39}Et aussi au quinzième jour du septième mois, quand vous aurez recueilli le produit du pays, vous célébrerez la fête solennelle de Yahweh pendant sept jours : Le premier jour sera un jour de repos, le huitième aussi sera un jour de repos.
\VS{40}Et au premier jour, vous prendrez du fruit d'un bel arbre, des branches de palmier, des rameaux d'arbres touffus et des saules de rivière ; et vous vous réjouirez pendant sept jours, devant Yahweh, votre Dieu.
\VS{41}Et vous célébrerez à Yahweh cette fête solennelle pendant sept jours dans l'année. C'est une loi perpétuelle pour vos descendants. Vous la célébrerez le septième mois.
\VS{42}Vous demeurerez sept jours sous des tentes ; tous ceux qui seront nés entre les Israélites demeureront sous des tentes,
\VS{43}afin que votre postérité sache que j'ai fait habiter les enfants d'Israël sous des tentes, quand je les ai fait sortir du pays d'Egypte. Je suis Yahweh, votre Dieu.
\VS{44}Moïse déclara ainsi aux enfants d'Israël les fêtes solennelles de Yahweh.
\Chap{24}
\TextTitle{L'huile du chandelier\FTNTT{Ex. 25:6.}}
\VerseOne{}Yahweh parla aussi à Moïse, en disant :
\VS{2}Ordonne aux enfants d'Israël de t'apporter de l'huile pure d'olives pressées pour le chandelier, afin de faire brûler les lampes continuellement.
\VS{3}Aaron les arrangera devant Yahweh continuellement, depuis le soir jusqu'au matin, en dehors du voile du témoignage dans la tente d'assignation. C'est une ordonnance perpétuelle pour vos descendants.
\VS{4}Il arrangera, dis-je, continuellement les lampes sur le chandelier pur, devant Yahweh.
\TextTitle{Les pains de proposition\FTNTT{Ex. 25:23-30.}}
\VS{5}Tu prendras aussi de la fine farine\FTNT{La fine farine est une farine de blé très pure, la première qui passe à travers les tamis de bluterie.}, et tu en feras cuire douze gâteaux\FTNT{Les pains de proposition étaient au nombre de douze et ne pouvaient être consommés que par les prêtres (Lé. 24:9). Ils préfiguraient Christ, le véritable pain de vie descendu du ciel (Jn. 6:48-51). Sous la Nouvelle Alliance, chaque enfant de Dieu est également un prêtre (Ap. 1:6), et est invité par conséquent à manger ce pain. Le nombre douze nous parle du fondement sur lequel nous devons êtres bâtis, à savoir Jésus-Christ lui-même et l'enseignement des apôtres et des prophètes (1 Co. 3:11 ; Ep. 2:20).}, chaque gâteau sera de deux dixièmes.
\VS{6}Et tu les exposeras devant Yahweh en deux rangées sur la table d'or pur, six à chaque rangée.
\VS{7}Et tu mettras de l'encens pur sur chaque rangée, qui sera comme un souvenir\FTNT{Voir commentaire en Lé. 2:2.} pour le pain, c'est une offrande consumée par le feu à Yahweh.
\VS{8}On les arrangera chaque jour de sabbat continuellement devant Yahweh, de la part des enfants d'Israël : C'est une alliance perpétuelle.
\VS{9}Et ils appartiendront à Aaron et à ses fils, qui les mangeront dans un lieu saint ; car ce sera pour eux une chose très sainte d'entre les offrandes de Yahweh consumées par le feu. C'est une ordonnance perpétuelle.
\TextTitle{Le blasphème contre le Nom de Yahweh\FTNTT{Jn. 8:59 ; 10:31.}}
\VS{10}Or le fils d'une femme israélite, qui était aussi fils d'un homme égyptien, sortit parmi les fils d'Israël, et ce fils de la femme israélite se querella dans le camp avec un homme israélite.
\VS{11}Et le fils de la femme israélite blasphéma et maudit le Nom de Yahweh. On l'amena à Moïse. Or sa mère s'appelait Schelomith, fille de Dibri, de la tribu de Dan.
\VS{12}Et on le mit en prison, jusqu'à ce que Moïse ait déclaré ce qu'il devrait faire selon la parole de Yahweh.
\VS{13}Et Yahweh parla à Moïse, en disant :
\VS{14}Tire hors du camp celui qui a maudit ; et que tous ceux qui l'ont entendu mettent les mains sur sa tête, et que toute l'assemblée le lapide.
\VS{15}Tu parleras aux enfants d'Israël, et tu leur diras : Quiconque aura maudit son Dieu, portera la peine de son péché.
\VS{16}Et celui qui aura blasphémé le Nom de Yahweh sera puni de mort : Toute l'assemblée ne manquera pas de le lapider, on fera mourir tant l'étranger que celui qui est né au pays, lequel aura blasphémé le Nom de Yahweh.
\TextTitle{La violence punie}
\VS{17}On punira aussi de mort celui qui aura frappé à mort quelque personne que ce soit.
\VS{18}Celui qui aura frappé une bête à mort, la remplacera : Vie pour vie.
\VS{19}Et quand quelque homme aura fait une blessure à son prochain, on lui fera comme il a fait :
\VS{20}Fracture pour fracture, œil pour œil, dent pour dent, selon le mal qu'il aura fait à un homme, il lui sera fait de même.
\VS{21}Celui qui frappera une bête à mort, la remplacera ; mais on fera mourir celui qui aura frappé un homme à mort.
\VS{22}Vous rendrez un même jugement. Vous traiterez l'étranger comme celui qui est né au pays ; car je suis Yahweh, votre Dieu.
\VS{23}Moïse parla aux enfants d'Israël, qui firent sortir hors du camp celui qui avait maudit, et le lapidèrent. Ainsi les fils d'Israël firent comme Yahweh l'avait ordonné à Moïse.
\Chap{25}
\TextTitle{L'année sabbatique}
\VerseOne{}Yahweh parla aussi à Moïse sur la montagne de Sinaï, en disant :
\VS{2}Parle aux enfants d'Israël, et dis-leur : Quand vous serez entrés dans le pays que je vous donne, la terre se reposera : Ce sera un sabbat à Yahweh.
\VS{3}Pendant six ans tu sèmeras ton champ, et pendant six ans tu tailleras ta vigne ; et tu en recueilleras le produit.
\VS{4}Mais la septième année il y aura un sabbat, un temps de repos pour la terre, ce sera un sabbat à Yahweh : Tu ne sèmeras point ton champ, et tu ne tailleras point ta vigne.
\VS{5}Tu ne moissonneras point ce qui proviendra des grains tombés dans ta moisson, et tu ne vendangeras point les raisins de ta vigne non taillée : Ce sera une année de repos total pour la terre.
\VS{6}Mais ce qui proviendra de la terre l'année du sabbat vous servira de nourriture, à toi, à ton serviteur et à ta servante, à ton mercenaire et à l'étranger qui demeurent avec toi,
\VS{7}à ton bétail et aux animaux qui sont dans ton pays ; tout son produit servira de nourriture.
\TextTitle{L'année du jubilé}
\VS{8}Tu compteras aussi sept sabbats d'années, à savoir sept fois sept ans, et les jours de sept sabbats feront quarante-neuf ans.
\VS{9}Puis tu feras sonner le shofar de jubilation le dixième jour du septième mois ; le jour, dis-je, des expiations, vous ferez sonner le shofar dans tout votre pays.
\VS{10}Et vous sanctifierez la cinquantième année, et publierez la liberté dans le pays à tous ses habitants : Ce sera pour vous l'année du jubilé ; et vous retournerez chacun dans sa possession, et chacun dans sa famille.
\VS{11}Cette cinquantième année vous sera l'année du jubilé : Vous ne sèmerez point et vous ne moissonnerez point ce que la terre rapportera d'elle-même, et vous ne vendangerez point les fruits de la vigne non taillée.
\VS{12}Car c'est l'année du jubilé, elle vous sera sainte. Vous mangerez ce que les champs rapporteront cette année-là.
\VS{13}En cette année du jubilé chacun de vous retournera dans sa possession.
\VS{14}Et si tu fais une vente à ton prochain, ou si tu achètes quelque chose de ton prochain, que nul de vous ne trompe son frère.
\VS{15}Mais tu achèteras de ton prochain selon le nombre des années après le jubilé. Pareillement on te fera les ventes selon le nombre des années de rapport.
\VS{16}Selon qu'il y aura plus d'années, tu augmenteras le prix de ce que tu achètes ; et selon qu'il y aura moins d'années, tu le diminueras ; car on te vend le nombre des récoltes.
\VS{17}Que nul de vous ne trompe son prochain, mais craignez votre Dieu ; car je suis Yahweh, votre Dieu.
\VS{18}Pratiquez mes ordonnances, gardez mes jugements et observez-les, et vous habiterez en sécurité dans le pays.
\VS{19}Et le pays vous donnera ses fruits, vous en mangerez, vous en serez rassasiés, et vous y habiterez en sécurité.
\VS{20}Et si vous dites : Que mangerons-nous la septième année si nous ne semons point, et si nous ne recueillons point notre récolte ?
\VS{21}J'ordonnerai à ma bénédiction de se répandre sur vous dans la sixième année, et la terre rapportera pour trois ans.
\VS{22}Puis vous sèmerez la huitième année, et vous mangerez de l'ancienne récolte jusqu'à la neuvième année ; jusqu'à ce que sa récolte soit venue, vous mangerez de l'ancienne.
\VS{23}La terre ne sera point vendue à perpétuité ; car le pays est à moi, et vous êtes étrangers et forains\FTNT{Forain : Quelqu'un d'extérieur, d'étranger à un lieu.} chez moi.
\VS{24}C'est pourquoi dans tout le pays dont vous aurez la possession, vous donnerez le droit de rachat\FTNT{Pour voir un exemple de ce droit de rachat, voir Ru. 4:1-13.} pour la terre.
\TextTitle{Le droit de rachat}
\VS{25}Si ton frère est devenu pauvre et vend quelque chose de ce qu'il possède, celui qui a le droit de rachat, à savoir son plus proche parent, viendra et rachètera la chose vendue par son frère.
\VS{26}Si cet homme n'a personne qui ait le droit de rachat, et qu'il ait trouvé de lui-même suffisamment de quoi faire le rachat de ce qu'il a vendu,
\VS{27}il comptera les années du temps qu'il a fait la vente, et il restituera le surplus à l'homme auquel il l'avait faite, et ainsi il retournera dans sa possession.
\VS{28}Mais s'il n'a pas trouvé suffisamment de quoi lui rendre, la chose qu'il aura vendue sera dans les mains de celui qui l'aura acheté, jusqu'à l'année du jubilé ; puis l'acheteur en sortira au jubilé, et le vendeur retournera dans sa possession.
\VS{29}Et si quelqu'un a vendu une maison d'habitation dans quelque ville entourée de murs, il aura le droit de rachat jusqu'à la fin de l'année de sa vente ; son droit de rachat sera d'une année.
\VS{30}Mais si elle n'est point rachetée dans l'année accomplie, la maison qui est dans la ville entourée de murs, demeurera à l'acheteur absolument et à ses descendants ; il n'en sortira point au jubilé.
\VS{31}Mais les maisons des villages, qui ne sont point entourés de murs, seront comptées comme des fonds de terre ; le vendeur aura droit de rachat, et l'acheteur sortira au jubilé.
\VS{32}Et quant aux villes des Lévites, les Lévites auront un droit de rachat perpétuel des maisons des villes de leur possession.
\VS{33}Et celui qui achètera une maison des Lévites, sortira au jubilé de la maison vendue, qui est dans la ville de sa possession ; car les maisons des villes des Lévites sont leur possession parmi les enfants d'Israël.
\VS{34}Mais les champs situés autour des villes des Lévites ne seront point vendus ; car c'est leur possession perpétuelle.
\TextTitle{Les traitements du frère pauvre}
\VS{35}Quand ton frère sera devenu pauvre, et qu'il tendra vers toi ses mains tremblantes, tu le soutiendras, tu soutiendras aussi l'étranger, et le forain, afin qu'il vive avec toi.
\VS{36}Tu ne prendras point de lui d'usure ni d'intérêt, mais tu craindras ton Dieu, et ton frère vivra avec toi.
\VS{37}Tu ne lui prêteras point ton argent à intérêt ni ne lui prêteras de tes vivres pour en tirer du profit.
\VS{38}Je suis Yahweh, votre Dieu qui vous ai fait sortir du pays d'Egypte, pour vous donner le pays de Canaan, afin d'être votre Dieu.
\VS{39}Pareillement, quand ton frère sera devenu pauvre auprès de toi, et qu'il se sera vendu à toi, tu ne te serviras point de lui comme on se sert des esclaves.
\VS{40}Mais il sera chez toi comme serait le mercenaire et l'étranger, et il te servira jusqu'à l'année du jubilé.
\VS{41}Alors il sortira de chez toi avec ses fils, il s'en retournera dans sa famille, et rentrera dans la possession de ses pères.
\VS{42}Car ils sont mes serviteurs, parce que je les ai retirés du pays d'Egypte ; c'est pourquoi ils ne seront point vendus comme on vend les esclaves.
\VS{43}Tu ne domineras point sur lui avec dureté, et tu craindras ton Dieu.
\VS{44}C'est des nations qui vous entourent que tu prendras ton esclave et ta servante qui t'appartiendront ; c'est d'elles que vous achèterez l'esclave et la servante.
\VS{45}Vous pourrez aussi en acheter des fils des étrangers qui demeureront chez toi, et même de leurs familles qui seront parmi vous, qui auront engendré dans votre pays, et vous les posséderez.
\VS{46}Vous les aurez comme un héritage pour les laisser à vos enfants après vous, afin qu'ils en héritent la possession, et vous vous servirez d'eux à perpétuité. Mais quant à vos frères, les fils d'Israël, nul ne dominera avec dureté sur son frère.
\VS{47}Et lorsque l'étranger ou le forain qui est avec toi se sera enrichi, et que ton frère qui est avec lui sera devenu si pauvre qu'il se soit vendu à l'étranger, ou au forain qui est avec toi, ou à quelqu'un de la postérité de la famille de l'étranger,
\VS{48}après s'être vendu, il y aura droit de rachat pour lui : Un de ses frères le rachètera.
\VS{49}Son oncle, ou le fils de son oncle, ou quelque autre proche parent de son sang d'entre ceux de sa famille, le rachètera ; ou lui-même, s'il en trouve le moyen, se rachètera.
\VS{50}Et il comptera avec son acheteur depuis l'année qu'il s'est vendu à lui, jusqu'à l'année du jubilé ; de sorte que l'argent du prix pour lequel il s'est vendu, se comptera à raison du nombre des années, le temps qu'il aura servi lui sera compté comme les journées d'un mercenaire.
\VS{51}S'il y a encore plusieurs années, il restituera le prix de son achat à raison de ces années, selon le prix pour lequel il a été acheté ;
\VS{52}et s'il reste peu d'années jusqu'à l'année du jubilé, il comptera avec lui, et restituera le prix de son achat à raison des années qu'il a servi.
\VS{53}Il aura été avec lui comme un mercenaire qui se loue d'année en année, et cet étranger ne dominera point sur lui avec dureté en ta présence.
\VS{54}S'il n'est pas racheté par quelqu'un de ces moyens, il sortira l'année du jubilé, lui et ses fils avec lui.
\VS{55}Car c'est de moi que les enfants d'Israël sont esclaves ; ce sont mes esclaves que j'ai fait sortir du pays d'Egypte. Je suis Yahweh, votre Dieu.
\Chap{26}
\TextTitle{Mise en garde contre le péché}
\VerseOne{}Vous ne vous ferez point d'idoles, vous ne vous dresserez point d'image taillée, ni de statue, et vous ne mettrez point de pierre sculptée dans votre pays, pour vous prosterner devant elles ; car je suis Yahweh, votre Dieu.
\VS{2}Vous garderez mes sabbats et vous révérerez mon sanctuaire. Je suis Yahweh.
\TextTitle{La bénédiction conditionnelle à l'obéissance à Yahweh}
\VS{3}Si vous marchez dans mes ordonnances et si vous gardez mes commandements et les pratiquez,
\VS{4}je vous donnerai les pluies en leur temps, la terre donnera ses produits, et les arbres des champs donneront leurs fruits.
\VS{5}Le foulage des grains atteindra la vendange chez vous, et la vendange atteindra les semailles ; vous mangerez votre pain à satiété et vous habiterez en sécurité dans votre pays.
\VS{6}Je donnerai la paix au pays, vous dormirez sans que personne ne vous trouble ; je ferai disparaître les bêtes méchantes du pays, et l'épée ne passera point par votre pays.
\VS{7}Vous poursuivrez vos ennemis, et ils tomberont par l'épée devant vous.
\VS{8}Cinq d'entre vous en poursuivront cent, et cent en poursuivront dix mille, et vos ennemis tomberont par l'épée devant vous.
\VS{9}Je me tournerai vers vous, je vous ferai fructifier et multiplier, et j'établirai mon alliance avec vous.
\VS{10}Vous mangerez de vieilles provisions, et vous sortirez le vieux pour y loger le nouveau.
\VS{11}Même, je mettrai mon tabernacle au milieu de vous, et mon âme ne vous aura point en horreur.
\VS{12}Mais je marcherai au milieu de vous, je serai votre Dieu, et vous serez mon peuple.
\VS{13}Je suis Yahweh, votre Dieu, qui vous ai fait sortir du pays d'Egypte, afin que vous ne soyez point leurs esclaves ; j'ai brisé les liens de votre joug, et je vous ai fait marcher la tête levée.
\TextTitle{Les châtiments en cas de désobéissance à Yahweh}
\VS{14}Mais si vous ne m'écoutez point et que vous ne pratiquez pas tous ces commandements,
\VS{15}et si vous rejetez mes ordonnances, et que votre âme a en horreur mes jugements, afin de ne point pratiquer tous mes commandements, et que vous rompiez mon alliance,
\TextTitle{La domination par les ennemis}
\VS{16}aussi je vous ferai ceci : Je répandrai sur vous la frayeur, la langueur et l'ardeur, qui vous consumerons les yeux et feront languir votre âme ; et vous sèmerez en vain votre semence car vos ennemis la mangeront.
\VS{17}Je tournerai ma face contre vous, vous serez battus devant vos ennemis ; ceux qui vous haïssent domineront sur vous ; et vous fuirez sans que personne ne vous poursuive.
\TextTitle{Le manque de fertilité de la terre}
\VS{18}Si après ces choses vous ne m'écoutez point, je vous châtierai sept fois plus à cause de vos péchés.
\VS{19}Je briserai l'orgueil de votre force et je ferai que votre ciel soit pour vous comme du fer, et votre terre comme de l'airain.
\VS{20}Votre force se consumera en vain, votre terre ne donnera point ses produits, et les arbres de la terre ne donneront point leurs fruits.
\TextTitle{Les attaques des bêtes des champs}
\VS{21}Si vous marchez en opposition avec moi et que vous ne voulez point m'écouter, je vous frapperai sept fois plus, selon vos péchés.
\VS{22}J'enverrai contre vous les bêtes des champs, qui vous priveront de vos enfants, qui détruiront votre bétail, et vous réduiront à un petit nombre, et vos chemins seront déserts.
\TextTitle{La peste}
\VS{23}Si après ces choses, vous ne recevez pas ma correction, et que vous marchiez en opposition avec moi,
\VS{24}je marcherai aussi en opposition avec vous, et je vous frapperai sept fois plus, selon vos péchés.
\VS{25}Et je ferai venir sur vous l'épée qui fera la vengeance de mon alliance ; et quand vous vous rassemblerez dans vos villes, j'enverrai la peste au milieu de vous, et vous serez livrés entre les mains de l'ennemi.
\TextTitle{Le manque de nourriture}
\VS{26}Lorsque je vous briserai le bâton du pain, dix femmes cuiront votre pain dans un seul four, et vous rendront votre pain au poids ; vous en mangerez, et vous n'en serez point rassasiés.
\VS{27}Si avec cela vous ne m'écoutez point, et que vous marchiez en opposition avec moi,
\VS{28}je marcherai aussi en opposition avec vous, avec fureur, et je vous châtierai aussi sept fois plus, selon vos péchés ;
\VS{29}vous mangerez la chair de vos fils, et vous mangerez aussi la chair de vos filles\FTNT{La. 4:10.}.
\VS{30}Je détruirai vos hauts lieux, j'abattrai vos statues consacrées au soleil, je mettrai vos cadavres sur les cadavres de vos idoles, et mon âme vous aura en horreur.
\VS{31}Je réduirai vos villes en désert, je dévasterai vos sanctuaires, et je ne respirerai plus l'agréable odeur de vos parfums.
\TextTitle{La dispersion dans les nations\FTNTT{De. 28:58-67.}}
\VS{32}Je dévasterai le pays, et vos ennemis qui l'habiteront, en seront étonnés.
\VS{33}Je vous disperserai parmi les nations, et je tirerai l'épée après vous, et votre pays sera dévasté, et vos villes désertes.
\VS{34}Alors la terre prendra plaisir à ses sabbats\FTNT{2 Ch. 36:21.}, tout le temps qu'elle sera dévastée, et lorsque vous serez dans le pays de vos ennemis, la terre se reposera et prendra plaisir à ses sabbats.
\VS{35}Tout le temps qu'elle sera dévastée, elle se reposera parce qu'elle ne s'était point reposée dans vos sabbats, lorsque vous y habitiez.
\VS{36}Et quant à ceux d'entre vous qui survivront dans le pays de leurs ennemis, je rendrai leur cœur lâche, de sorte que le bruit d'une feuille agitée les poursuivra, ils fuiront comme on fuit devant l'épée, et ils tomberont sans que personne ne les poursuive.
\VS{37}Et ils trébucheront les uns sur les autres comme devant l'épée, sans que personne ne les poursuive ; et vous ne tiendrez point devant vos ennemis ;
\VS{38}vous périrez parmi les nations, et le pays de vos ennemis vous consumera.
\VS{39}Et ceux d'entre vous qui survivront, se fondront à cause de leurs iniquités, dans les pays de vos ennemis ; ils se fondront aussi à cause des iniquités de leurs pères.
\TextTitle{Repentance et restauration de l'alliance d'Abraham, d'Isaac et de Jacob}
\VS{40}Alors, ils confesseront leurs iniquités et les iniquités de leurs pères, selon les transgressions qu'ils auront commises contre moi ; et aussi parce qu'ils auront marché en opposition avec moi.
\VS{41}Moi aussi, je marcherai en opposition avec eux, je les amènerai dans le pays de leurs ennemis. Et alors, leur cœur incirconcis s'humiliera, et ils recevront la peine de leur iniquité.
\VS{42}Et alors je me souviendrai de mon alliance avec Jacob, et de mon alliance avec Isaac, et je me souviendrai aussi de mon alliance avec Abraham, et je me souviendrai de la terre.
\VS{43}Quand donc la terre sera abandonnée par eux, et prendra plaisir à ses sabbats, ayant été désolée à cause d'eux ; et qu'ils recevront la peine de leur iniquité, parce qu'ils ont rejeté mes ordonnances, et que leur âme a eu mes ordonnances en horreur.
\VS{44}Je m'en souviendrai, dis-je, lorsqu'ils seront dans le pays de leurs ennemis, je ne les rejetterai point, et je ne les aurai point en horreur pour les consumer entièrement jusqu'à rompre mon alliance avec eux ; car je suis Yahweh, leur Dieu.
\VS{45}Je me souviendrai en leur faveur de la Première Alliance, par laquelle je les ai fait sortir du pays d'Egypte, aux yeux des nations, pour être leur Dieu. Je suis Yahweh.
\VS{46}Ce sont là les statuts, les ordonnances, et les lois que Yahweh établit entre lui et les enfants d'Israël sur la montagne de Sinaï, par Moïse.
\Chap{27}
\TextTitle{Lois des personnes et des biens voués à Yahweh}
\VerseOne{}Yahweh parla aussi à Moïse, en disant :
\VS{2}Parle aux enfants d'Israël, et dis-leur : quand quelqu'un aura fait un vœu important, les personnes vouées à Yahweh seront mises à ton estimation.
\VS{3}Et l'estimation que tu feras d'un homme, depuis l'âge de vingt ans jusqu'à l'âge de soixante ans, sera du prix de cinquante sicles d'argent, selon le sicle du sanctuaire.
\VS{4}Mais si c'est une femme, alors ton estimation sera de trente sicles.
\VS{5}Si c'est un homme de cinq ans jusqu'à vingt ans, alors ton estimation sera de vingt sicles ; et quant à la femme, de dix sicles.
\VS{6}Et si c'est un homme d'un mois jusqu'à cinq ans, ton estimation sera de cinq sicles d'argent ; et l'estimation d'une femme sera de trois sicles d'argent.
\VS{7}Et lorsque c'est un homme de soixante ans et au-dessus, ton estimation sera de quinze sicles ; et si c'est une femme, de dix sicles.
\VS{8}Et si celui qui a fait le vœu est plus pauvre que ton estimation, on le présentera devant le prêtre, qui en fera l'estimation, et le prêtre fera l'estimation selon les ressources de celui qui a fait le vœu.
\VS{9}Si c'est d'une des bêtes que l'on présente en offrande à Yahweh, tout ce qu'on donnera à Yahweh de la sorte sera saint.
\VS{10}On ne la changera point, et on n'en mettra point une autre à la place, d'une bonne pour une mauvaise, ou une mauvaise pour une bonne ; si l'on remplace une bête par une autre bête, elles seront l'une et l'autre chose sainte.
\VS{11}Si c'est d'une bête impure, qu'on ne peut présenter en offrande à Yahweh, on présentera la bête devant le prêtre,
\VS{12}qui en fera l'évaluation selon qu'elle sera bonne ou mauvaise, et il en sera fait ainsi, selon l'estimation du prêtre.
\VS{13}Mais si on veut la racheter, on ajoutera un cinquième à ton estimation.
\VS{14}Et quand quelqu'un sanctifiera sa maison pour être sainte à Yahweh, le prêtre l'estimera selon qu'elle sera bonne ou mauvaise, et on se tiendra à l'estimation que le prêtre en aura faite.
\VS{15}Mais si celui qui l'a sanctifiée veut racheter sa maison, il ajoutera par-dessus un cinquième de l'argent de ton estimation, et elle lui appartiendra.
\VS{16}Et si l'homme sanctifie à Yahweh une partie du champ de sa possession, ton estimation sera selon ce qu'on y sème, le homer de semence d'orge à cinquante sicles d'argent.
\VS{17}S'il a sanctifié son champ dès l'année du jubilé, on s'en tiendra à ton estimation ;
\VS{18}mais s'il sanctifie son champ après le jubilé, le prêtre estimera l'argent selon le nombre des années qui restent jusqu'à l'année du jubilé, et il sera fait une réduction sur ton estimation.
\VS{19}Et si celui qui a sanctifié le champ veut le racheter en quelque sorte que ce soit, il ajoutera par-dessus un cinquième de l'argent de ton estimation, et il lui restera.
\VS{20}Mais s'il ne rachète point le champ, et que le champ se vende à un autre homme, il ne se rachètera plus.
\VS{21}Et ce champ-là ayant passé le jubilé sera consacré à Yahweh, comme un champ d'interdit, la possession en sera au prêtre.
\VS{22}Et s'il sanctifie à Yahweh un champ qu'il ait acheté, qui ne soit point des champs de sa possession,
\VS{23}le prêtre lui comptera la somme de ton estimation jusqu'à l'année du jubilé, et il donnera en ce jour-là ton estimation, afin que ce soit une chose consacrée à Yahweh.
\VS{24}Mais l'année du jubilé, le champ retournera à celui de qui il avait été acheté, et auquel était la possession de la terre.
\VS{25}Et toute estimation que tu auras faite, sera selon le sicle du sanctuaire : Le sicle est de vingt guéras.
\TextTitle{Consécration des premiers-nés du bétail}
\VS{26}Toutefois, nul ne pourra consacrer le premier-né d'entre les bêtes, car il appartient à Yahweh par droit de primogéniture, soit de bœuf, soit d'agneau, il est à Yahweh.
\VS{27}Mais s'il s'agit d'une bête impure, il le rachètera selon ton estimation, et il ajoutera à ton estimation un cinquième ; et s'il n'est point racheté, il sera vendu selon ton estimation.
\TextTitle{Consécration des choses et personnes dévouées par interdit à Yahweh}
\VS{28}Or toute chose dévouée que quelqu'un dévouera à la façon de l'interdit à Yahweh, de tout ce qui est sien, soit homme, ou bête, ou champ de sa possession, ne se revendra ni ne se rachètera ; toute chose dévouée sera entièrement consacrée à Yahweh.
\VS{29}Nul interdit dévoué par interdit d'entre les hommes ne pourra être racheté, mais on le fera mettre à mort.
\TextTitle{Consécration de la dîme de la terre et du bétail}
\VS{30}Toute dîme de la terre, tant du grain de la terre que du fruit des arbres, est à Yahweh ; c'est une chose consacrée à Yahweh.
\VS{31}Mais si quelqu'un veut racheter en quelque sorte que ce soit quelque chose de sa dîme, il y ajoutera un cinquième par-dessus.
\VS{32}Mais toute dîme de bœufs, de brebis et de chèvres, à savoir tout ce qui passe sous la verge, le dixième en sera consacrée à Yahweh.
\VS{33}On ne choisira point le bon ou le mauvais, et l'on ne fera point d'échange ; si on l'échange, la bête changée et l'autre seront consacrées, et ne seront point rachetées.
\VS{34}Ce sont là les commandements que Yahweh donna à Moïse sur la montagne de Sinaï, pour les enfants d'Israël.
\PPE{}
\end{multicols}

\clearpage\ShortTitle{No.}\BookTitle{Nombres}\BFont
\noindent\hrulefill
{\footnotesize
\textit{
\bigskip
{\centering{}
\\Auteur~: Probablement Moïse
\\(Heb.~: Bamidbar)
\\Signification~: Dans le désert
\\Thème~: Pérégrination dans le désert
\\Date de rédaction~: Env. 1450-1410 av. J.-C.\\}
}
\textit{
\\Ce livre commence par le recensement des fils d'Israël et relate trente-huit des quarante années qu'ils passèrent dans le désert du Sinaï. Il couvre une période qui s'étend de la deuxième année après la sortie d'Egypte à la veille de l'entrée en Canaan, terre que Dieu avait promis de donner à la descendance d'Abraham. Ce pays où coulaient le lait et le miel s'étendait de Sidon jusqu'à Lesha, en passant par Gaza et Sodome. En plus des Cananéens, il accueillait en son sein des enfants d'Anak, les Amalécites, les Hétiens, les Jébusiens et les Amoréens.
\\Ces écrits retracent les premières victoires d'Israël et regroupent diverses lois et instructions sur le partage de la terre promise. Ils témoignent également de la révolte et de l'incrédulité de la génération sortie d'Egypte dont la quasi-totalité périt dans le désert.\bigskip
}
}
\par\nobreak\noindent\hrulefill
\begin{multicols}{2}
\Chap{1}
\TextTitle{Dénombrement des hommes de guerre}
\VerseOne{}Or Yahweh parla à Moïse dans le désert de Sinaï, dans la tente d'assignation, le premier jour du second mois, la seconde année, après qu'ils furent sortis du pays d'Egypte, en disant~:
\VS{2}Faites le dénombrement de toute l'assemblée des fils d'Israël, selon leurs familles, selon les maisons de leurs pères, en comptant nom par nom, savoir tous les mâles\FTNT{Ex. 30:12~; Ex. 38:26.}, chacun par tête~;
\VS{3}depuis l'âge de vingt ans et au-dessus, tous ceux d'Israël qui peuvent aller à la guerre, vous les compterez selon leurs armées, toi et Aaron.
\VS{4}Il y aura avec vous un homme par tribu, celui qui est le chef de la maison de ses pères.
\VS{5}Voici les noms des hommes qui vous assisteront. Pour la tribu de Ruben~: Elitsur, fils de Schedéur~;
\VS{6}pour celle de Siméon~: Schelumiel, fils de Tsurischaddaï~;
\VS{7}pour celle de Juda~: Nachschon, fils d'Amminadab~;
\VS{8}pour celle d'Issacar~: Nethaneel, fils de Tsuar~;
\VS{9}pour celle de Zabulon~: Eliab, fils de Hélon~;
\VS{10}pour les fils de Joseph, pour la tribu d'Ephraïm~: Elischama, fils d'Ammihud~; pour celle de Manassé~: Gamliel, fils de Pedahtsur~;
\VS{11}pour la tribu de Benjamin~: Abidan, fils de Guideoni~;
\VS{12}pour celle de Dan~: Ahiézer, fils d'Ammischaddaï~;
\VS{13}pour celle d'Aser~: Paguiel, fils d'Ocran~;
\VS{14}pour celle de Gad~: Eliasaph, fils de Déuel~;
\VS{15}pour celle de Nephthali~: Ahira, fils d'Enan.
\VS{16}C'étaient là ceux qu'on appelait pour tenir l'assemblée~; ils étaient les princes des tribus de leurs pères, chefs des milliers d'Israël.
\VS{17}Alors Moïse et Aaron prirent ces hommes qui avaient été désignés par leurs noms,
\VS{18}et ils convoquèrent toute l'assemblée, le premier jour du second mois. On les enregistra selon leurs familles et selon la maison de leurs pères, en comptant les noms depuis l'âge de vingt ans et au-dessus, chacun par tête.
\VS{19}Comme Yahweh l'avait commandé à Moïse, il les dénombra au désert de Sinaï.
\VS{20}Les fils donc de Ruben, premier-né d'Israël, selon leurs générations, leurs familles, et les maisons de leurs pères, dont on fit le dénombrement par leur nom, et par tête, savoir tous les mâles de l'âge de vingt ans, et au dessus, tous ceux qui pouvaient aller à la guerre.
\VS{21}Ceux, dis-je, de la tribu de Ruben, qui furent dénombrés, furent quarante-six mille cinq cents.
\VS{22}Des enfants de Siméon, selon leurs générations, leurs familles, et les maisons de leurs pères, ceux qui furent dénombrés par leur nom et par tête, savoir tous les mâles de l'âge de vingt ans, et au dessus, tous ceux qui pouvaient aller à la guerre~;
\VS{23}ceux, dis-je, de la tribu de Siméon, qui furent dénombrés, furent cinquante-neuf mille trois cents.
\VS{24}Des fils de Gad, selon leurs générations, leurs familles, et les maisons de leurs pères, dénombrés chacun par leur nom, depuis l'âge de vingt ans, et au dessus, tous ceux qui pouvaient aller à la guerre~;
\VS{25}ceux, dis-je, de la tribu de Gad, qui furent dénombrés, furent quarante-cinq mille six cent cinquante.
\VS{26}Des enfants de Juda, selon leurs générations, leurs familles, et les maisons de leurs pères, dénombrés chacun par leur nom, depuis l'âge de vingt ans, et au dessus, tous ceux qui pouvaient aller à la guerre~;
\VS{27}ceux, dis-je, de la tribu de Juda, qui furent dénombrés, furent soixante-quatorze mille six cents.
\VS{28}Des fils d'Issacar, selon leurs générations, leurs familles, et les maisons de leurs pères, dénombrés chacun par leur nom, depuis l'âge de vingt ans, et au dessus, tous ceux qui pouvaient aller à la guerre~;
\VS{29}ceux, dis-je, de la tribu d'Issacar, qui furent dénombrés, furent cinquante-quatre mille quatre cents.
\VS{30}Des enfants de Zabulon, selon leurs générations, leurs familles, et les maisons de leurs pères, dénombrés chacun par leur nom, depuis l'âge de vingt ans, et au dessus, tous ceux qui pouvaient aller à la guerre~;
\VS{31}ceux, dis-je, de la tribu de Zabulon, qui furent dénombrés, furent cinquante-sept mille quatre cents.
\VS{32}Quant aux fils de Joseph~; les fils d'Ephraïm, selon leurs générations, leurs familles, et les maisons de leurs pères, dénombrés chacun par leur nom, depuis l'âge de vingt ans, et au dessus, tous ceux qui pouvaient aller à la guerre~;
\VS{33}ceux, dis-je, de la tribu d'Ephraïm, qui furent dénombrés, furent quarante mille cinq cents.
\VS{34}Des fils de Manassé, selon leurs générations, leurs familles, et les maisons de leurs pères, dénombrés chacun par leur nom, depuis l'âge de vingt ans, et au dessus, tous ceux qui pouvaient aller à la guerre~;
\VS{35}ceux, dis-je, de la tribu de Manassé, qui furent dénombrés, furent trente-deux mille deux cents.
\VS{36}Des fils de Benjamin, selon leurs générations, leurs familles, et les maisons de leurs pères, dénombrés chacun par leur nom, depuis l'âge de vingt ans, et au dessus, tous ceux qui pouvaient aller à la guerre~;
\VS{37}ceux, dis-je, de la tribu de Benjamin, qui furent dénombrés, furent trente-cinq mille quatre cents.
\VS{38}Des fils de Dan, selon leurs générations, leurs familles, et les maisons de leurs pères, dénombrés chacun par leur nom, depuis l'âge de vingt ans, et au dessus, tous ceux qui pouvaient aller à la guerre~;
\VS{39}ceux, dis-je, de la tribu de Dan qui furent dénombrés, furent soixante-deux mille sept cents.
\VS{40}Des fils d'Aser, selon leurs générations, leurs familles, et les maisons de leurs pères, dénombrés chacun par leur nom, depuis l'âge de vingt ans, et au dessus, tous ceux qui pouvaient aller à la guerre~;
\VS{41}ceux, dis-je, de la tribu d'Aser, qui furent dénombrés, furent quarante et un mille cinq cents.
\VS{42}Des fils de Nephthali, selon leurs générations, leurs familles, et les maisons de leurs pères, dénombrés chacun par leur nom, depuis l'âge de vingt ans, et au dessus, tous ceux qui pouvaient aller à la guerre~;
\VS{43}ceux, dis-je, de la tribu de Nephthali, qui furent dénombrés, furent cinquante-trois mille quatre cents.
\VS{44}Ce sont là ceux dont Moïse et Aaron firent le dénombrement, les douze princes d'entre les enfants d'Israël y étant, un pour chaque maison de leurs pères.
\VS{45}Ainsi tous ceux des enfants d'Israël, dont on fit le dénombrement, selon les maisons de leurs pères, depuis l'âge de vingt ans, et au dessus, tous ceux d'entre les Israélites, qui pouvaient aller à la guerre~;
\VS{46}tous ceux, dis-je, dont on fit le dénombrement, furent six cent trois mille cinq cent cinquante.
\VS{47}Mais les Lévites ne furent point dénombrés avec eux, selon la tribu de leurs pères.
\VS{48}Car Yahweh avait parlé à Moïse, en disant~:
\VS{49}Tu ne feras aucun dénombrement de la tribu de Lévi, et tu n'en lèveras point la somme avec les autres enfants d'Israël.
\VS{50}Mais tu donneras aux Lévites la charge du tabernacle du témoignage, et de tous ses ustensiles, et de tout ce qui lui appartient~; ils porteront le tabernacle, et tous ses ustensiles~; ils y serviront, et camperont autour du tabernacle.
\VS{51}Et quand le tabernacle partira, les Lévites le démonteront, et quand le tabernacle campera, les Lévites le dresseront. Que si quelque étranger en approche, on le fera mourir\FTNT{Ez. 44:8-9.}.
\VS{52}Or les enfants d'Israël camperont chacun dans son camp, et chacun sous sa bannière, selon leurs armées.
\VS{53}Mais les Lévites camperont autour du tabernacle du témoignage, afin qu'il n'y ait point d'indignation sur l'assemblée des enfants d'Israël, et ils prendront en leur charge le tabernacle du Témoignage.
\VS{54}Et les enfants d'Israël firent selon toutes les choses que Yahweh avait commandées à Moïse~; ils le firent ainsi.
\Chap{2}
\TextTitle{Disposition du camp d'Israël par tribu}
\VerseOne{}Et Yahweh parla à Moïse et à Aaron, en disant~:
\VS{2}Les enfants d'Israël camperont chacun sous sa bannière, avec les enseignes des maisons de leurs pères, tout autour de la tente d'assignation, vis-à-vis de lui.
\VS{3}Ceux de la bannière du camp de Juda camperont droit vers l'est, selon ses armées~; et Nachschon, fils d'Amminadab, sera le chef des fils de Juda~;
\VS{4}et son armée, et ses dénombrés, soixante-quatorze mille six cents.
\VS{5}Près de lui campera la tribu d'Issacar, et Nethanaël, fils de Tsuar, sera le chef des enfants d'Issacar~;
\VS{6}et son armée, et ses dénombrés, cinquante-quatre mille quatre cents.
\VS{7}Puis la tribu de Zabulon, et Eliab, fils de Hélon, sera le chef des enfants de Zabulon~;
\VS{8}et son armée, et ses dénombrés, cinquante-sept mille quatre cents.
\VS{9}Tous les dénombrés du camp de Juda, cent quatre-vingt-six mille quatre cents, selon leurs armées, partiront les premiers.
\VS{10}La bannière du camp de Ruben, selon ses armées, sera vers le sud, et Elitsur, fils de Schedéur, sera le chef des enfants de Ruben~;
\VS{11}et son armée, et ses dénombrés, quarante-six mille cinq cents.
\VS{12}Près de lui campera la tribu de Siméon, et Schelumiel, fils de Tsurischaddaï, sera le chef des enfants de Siméon~;
\VS{13}et son armée, et ses dénombrés, cinquante-neuf mille trois cents.
\VS{14}Puis la tribu de Gad, et Eliasaph, fils de Déuel, sera le chef des enfants de Gad~;
\VS{15}et son armée, et ses dénombrés, quarante-cinq mille six cent cinquante.
\VS{16}Tous les dénombrés du camp de Ruben, cent cinquante et un mille quatre cent cinquante, selon leurs armées, partiront les seconds.
\VS{17}Ensuite la tente d'assignation partira avec le camp des Lévites, au milieu des camps qui partiront comme ils auront campés, chacune en sa place, selon leurs bannières.
\VS{18}La bannière du camp d'Ephraïm, selon ses armées, sera vers l'occident~; et Elischama, fils de Ammihud, sera le chef des enfants d'Ephraïm~;
\VS{19}et son armée, et ses dénombrés, quarante mille cinq cents.
\VS{20}Près de lui campera la tribu de Manassé, et Gamliel, fils de Pedahtsur, sera le chef des fils de Manassé~;
\VS{21}et son armée, et ses dénombrés, trente-deux mille deux cents.
\VS{22}Puis la tribu de Benjamin, et Abidan, fils de Guideoni, sera le chef des fils de Benjamin~;
\VS{23}et son armée, et ses dénombrés, trente-cinq mille et quatre cents.
\VS{24}Tous les dénombrés pour le camp d'Ephraïm, cent huit mille et cent, selon leurs armées, partiront les troisièmes.
\VS{25}La bannière du camp de Dan, selon ses armées, sera vers le nord, et Ahiézer, fils de Ammischaddaaï, sera le chef des fils de Dan~;
\VS{26}et son armée, et ses dénombrés, soixante-deux mille sept cents.
\VS{27}Près de lui campera la tribu d'Aser, et Paguiel, fils de Ocran, sera le chef des fils d'Aser~;
\VS{28}et son armée, et ses dénombrés, quarante et un mille cinq cents.
\VS{29}Puis la tribu de Nephthali, et Ahira, fils d'Enan, sera le chef des fils de Nephthali~;
\VS{30}et son armée, et ses dénombrés, cinquante-trois mille quatre cents.
\VS{31}Tous les dénombrés du camp de Dan, cent cinquante-sept mille six cents, partiront les derniers des bannières.
\VS{32}Ce sont là ceux des enfants d'Israël dont on fit le dénombrement selon les maisons de leurs pères. Tous les dénombrés des camps selon leurs armées furent six cent trois mille cinq cent cinquante.
\VS{33}Mais les Lévites ne furent point dénombrés avec les autres enfants d'Israël, comme Yahweh l'avait commandé à Moïse.
\VS{34}Et les enfants d'Israël firent selon toutes les choses que Yahweh avait commandées à Moïse, et campèrent ainsi selon leurs bannières, et partirent ainsi, chacun selon leurs familles, et selon la maison de leurs pères.
\Chap{3}
\TextTitle{Organisation des prêtres et des Lévites}
\VerseOne{}Or ce sont ici les générations d'Aaron et de Moïse, au temps que Yahweh parla à Moïse sur la montagne de Sinaï.
\VS{2}Et ce sont ici les noms des fils d'Aaron~; Nadab, qui était l'aîné, Abihu, Eléazar, et Ithamar.
\VS{3}Ce sont là les noms des fils d'Aaron, les prêtres, qui furent oints et consacrés pour exercer la prêtrise\FTNT{Ex. 40:15~; Lé. 8:30.}.
\VS{4}Mais Nadab et Abihu moururent en la présence de Yahweh, quand ils offrirent un feu étranger devant Yahweh au désert de Sinaï, et ils n'eurent point d'enfants~; mais Eléazar et Ithamar exercèrent la prêtrise en la présence d'Aaron leur père\FTNT{Lé. 10:1-2~; 1 Ch. 24:2.}.
\VS{5}Yahweh parla à Moïse, en disant~:
\VS{6}Fais approcher la tribu de Lévi, et fais qu'elle se tienne devant Aaron, le prêtre, afin qu'ils le servent.
\VS{7}Et qu'ils aient la charge de ce qu'il leur ordonnera de garder, et de ce que toute l'assemblée leur ordonnera de garder, devant la tente d'assignation, en faisant le service du tabernacle.
\VS{8}Et qu'ils gardent tous les ustensiles de la tente d'assignation, et ce qui leur sera donné en charge par les enfants d'Israël, pour faire le service du tabernacle.
\VS{9}Ainsi tu donneras les Lévites à Aaron et à ses fils~; ils lui sont complètement donnés d'entre les enfants d'Israël.
\VS{10}Tu établiras donc Aaron et ses fils, et ils exerceront leur prêtrise. Que si quelque étranger en approche, on le fera mourir.
\VS{11}Et Yahweh parla à Moïse, en disant~:
\VS{12}Voici, j'ai pris les Lévites d'entre les enfants d'Israël, à la place de tout premier-né qui ouvre la matrice parmi les enfants d'Israël~; c'est pourquoi les Lévites seront à moi.
\VS{13}Car tout premier-né m'appartient, depuis le jour où je frappai tout premier-né au pays d'Egypte~; je me suis sanctifié tout premier-né en Israël, depuis les hommes jusqu'aux bêtes~; ils seront à moi, je suis Yahweh\FTNT{Ex. 13:2~; Ex. 22:29~; Ex. 34:19~; Lé. 27:26.}.
\TextTitle{Les familles des Lévites}
\VS{14}Yahweh parla aussi à Moïse au désert de Sinaï, en disant~:
\VS{15}Dénombre les enfants de Lévi, par les maisons de leurs pères, et par leurs familles, en comptant tout mâle depuis l'âge d'un mois, et au dessus.
\VS{16}Et Moïse les dénombra, selon le commandement de Yahweh, ainsi qu'il lui avait été ordonné.
\VS{17}Or ce sont ici les fils de Lévi selon leurs noms~: Guerschon, Kehath, et Merari.
\VS{18}Et ce sont ici les noms des fils de Guerschon, selon leurs familles, Libni, et Schimeï.
\VS{19}Et les fils de Kehath selon leurs familles, Amram, Jitsehar, Hébron et Uziel~;
\VS{20}et les fils de Merari, selon leurs familles, Machli et Muschi~; ce sont là les familles de Lévi, selon les maisons de leurs pères.
\VS{21}De Guerschon est sortie la famille de Libni, et la famille de Schimeï~; ce sont les familles des Guerschonites.
\VS{22}Ceux dont on fit le dénombrement, en comptant de tous les mâles depuis l'âge d'un mois et au dessus, furent au nombre de sept mille cinq cents.
\VS{23}Les familles des Guerschonites camperont derrière le tabernacle à l'occident.
\VS{24}Et Eliasaph, fils de Laël, sera le chef de la maison des pères des Guerschonites.
\TextTitle{Les fonctions des Lévites}
\VS{25}Et les fils de Guerschon auront en charge à la tente d'assignation, la tente, le tabernacle, sa couverture, le rideau de l'entrée de la tente d'assignation.
\VS{26}Et les courtines du parvis avec le rideau de l'entrée du parvis, qui servent pour tabernacle et pour l'autel, tout autour, et son cordage, pour tout son service.
\VS{27}Et de Kehath est sortie la famille des Amramites, la famille des Jitseharites, la famille des Hébronites, et la famille des Uziélites~; ce furent là les familles des Kehathites,
\VS{28}dont tous les mâles depuis l'âge d'un mois, et au dessus, furent au nombre de huit mille six cents, ayant la charge du sanctuaire.
\VS{29}Les familles des fils de Kehath camperont du côté du tabernacle vers le sud.
\VS{30}Et Elitsaphan, fils d'Uziel, sera le chef de la maison des pères des familles des Kehathites.
\VS{31}Et ils auront en charge l'arche, la table, le chandelier, les autels, et les ustensiles du sanctuaire avec lesquels on fait le service, et le rideau, avec tout ce qui y sert.
\VS{32}Et le chef des chefs des Lévites sera Eléazar, fils d'Aaron, le prêtre~; qui aura la surveillance sur ceux qui auront la charge du sanctuaire.
\VS{33}Et de Merari est sortie la famille des Machlites, et la famille des Muschi~; ce furent là les familles de Merari~;
\VS{34}ceux dont on fit le dénombrement, après le compte qui fut fait de tous les mâles, depuis l'âge d'un mois et au dessus, furent six mille deux cents.
\VS{35}Et Esuriel, fils d'Abihaïl, sera le chef de la maison des pères des familles des Merarites~; ils camperont du côté du tabernacle vers le nord.
\VS{36}Et on donnera aux enfants de Merari la surveillance des planches du tabernacle, de ses barres, de ses piliers, de ses bases, et de tous ses ustensiles, avec tout ce qui y sera~;
\VS{37}et des piliers du parvis tout autour, avec leurs bases, leurs pieux, et leurs cordes.
\VS{38}Et Moïse, et Aaron et ses fils campaient devant le tabernacle, à l'orient, devant la tente d'assignation, vers l'orient~; ils avaient la garde et le soin du sanctuaire, remis à la garde des enfants d'Israël~; et si quelque étranger en approche, on le fera mourir.
\VS{39}Tous ceux des Lévites dont on fit le dénombrement, lesquels Moïse et Aaron comptèrent par leurs familles, suivant le commandement de Yahweh, tous les mâles de l'âge d'un mois et au dessus, furent de vingt-deux mille.
\TextTitle{Le rachat des premiers-nés}
\VS{40}Yahweh dit à Moïse~: Fais le dénombrement de tous les premiers-nés mâles des enfants d'Israël, depuis l'âge d'un mois, et au dessus, et relève le nombre de leurs noms.
\VS{41}Et tu prendras pour moi, je suis Yahweh, les Lévites, à la place de tous les premiers-nés qui sont entre les enfants d'Israël~; tu prendras aussi les bêtes des Lévites, à la place de tous les premiers-nés des bêtes des enfants d'Israël.
\VS{42}Moïse fit le dénombrement, comme Yahweh lui avait commandé, de tous les premiers-nés qui étaient parmi les enfants d'Israël.
\VS{43}Et tous les premiers-nés des mâles, selon le nombre des noms, depuis l'âge d'un mois et au dessus, selon leur dénombrement, furent vingt-deux mille deux cent soixante-treize.
\VS{44}Et Yahweh parla à Moïse, en disant~:
\VS{45}Prends les Lévites à la place de tous les premiers-nés qui sont parmi les enfants d'Israël, et les bêtes des Lévites, à la place de leurs bêtes~; et les Lévites seront à moi~; je suis Yahweh.
\VS{46}Et quant à ceux qu'il faut racheter, les deux cent soixante-treize parmi les premiers-nés des fils d'Israël, qui sont de plus que les Lévites,
\VS{47}tu prendras cinq sicles par tête, tu les prendras selon le sicle du sanctuaire~; le sicle est de vingt guéras\FTNT{Ex. 30:13~; Lé. 27:6~; Lé. 27:25~; Ez. 45:12.}.
\VS{48}Et tu donneras à Aaron et à ses fils l'argent de ceux qui auront été rachetés, dépassant le nombre des Lévites.
\VS{49}Moïse donc prit l'argent du rachat de ceux qui étaient de plus, outre ceux qui avaient été rachetés par l'échange des Lévites.
\VS{50}Et il reçut l'argent des premiers-nés des enfants d'Israël, qui fut mille trois cent soixante-cinq sicles, selon le sicle du sanctuaire.
\VS{51}Et Moïse donna l'argent des rachetés à Aaron, et à ses fils, selon le commandement de Yahweh, ainsi que Yahweh le lui avait commandé.
\Chap{4}
\TextTitle{Les fonctions des fils de Kehath}
\VerseOne{}Et Yahweh parla à Moïse et à Aaron, en disant~:
\VS{2}Faites le dénombrement des fils de Kehath d'entre les enfants de Lévi par leurs familles, et par les maisons de leurs pères,
\VS{3}depuis l'âge de trente ans et au dessus, jusqu'à l'âge de cinquante ans, tous ceux qui entrent en rang, pour s'employer à la tente d'assignation.
\VS{4}C'est ici le service des fils de Kehath à la tente d'assignation, c'est-à-dire, le Saint des saints.
\VS{5}Quand le camp partira, Aaron et ses fils viendront démonter le voile\FTNT{Le voile intérieur est l'image du corps humain de Christ (Mt. 26:26). Ce voile fut déchiré de haut en bas lorsque le Seigneur est mort sur la croix (Mt. 27:50-51). Désormais, le croyant peut pénétrer dans la présence du Père (Hé. 10:19-20).} qui sert de rideau, et en couvriront l'arche du témoignage~;
\VS{6}puis ils mettront au dessus une couverture de peaux de taissons, ils étendront par dessus un drap de pourpre, et ils y mettront ses barres.
\VS{7}Et ils étendront un drap de pourpre sur la table des pains de proposition, et mettront sur elle les plats, les tasses, les bassins, et les calices de libations. Le pain continuel sera sur elle.
\VS{8}Ils étendront au dessus un drap teint de cramoisi, ils le couvriront d'une couverture de peaux de taissons, et ils y mettront ses barres.
\VS{9}Et ils prendront un drap de pourpre, en couvriront le chandelier du luminaire avec ses lampes, ses mouchettes, ses vases à cendre, et tous ses vases à huile, dont on fait usage pour son service\FTNT{Ex. 25:30-38.}~;
\VS{10}ils le mettront avec tous ses ustensiles, dans une couverture de peaux de taissons, et le mettront sur une perche.
\VS{11}Ils étendront sur l'autel d'or un drap de pourpre, ils le couvriront d'une couverture de peaux de taissons, et ils y mettront ses barres.
\VS{12}Ils prendront aussi tous les ustensiles du service dont on se sert dans le lieu saint, ils les mettront dans un drap de pourpre, et ils les couvriront d'une couverture de peaux de taissons, et les mettront sur des perches.
\VS{13}Ils ôteront les cendres de l'autel, et étendront dessus un drap de pourpre.
\VS{14}Et ils mettront dessus les ustensiles dont on se sert pour l'autel, les brasiers, les fourchettes, les pelles, les bassins, et tous les ustensiles de l'autel~; ils étendront dessus une couverture de peaux de taissons, et ils y mettront ses barres.
\VS{15}Le camp partira après qu'Aaron et ses fils auront achevé de couvrir le lieu saint et tous ses ustensiles, et après cela les fils de Kehath viendront pour le porter, et ils ne toucheront point les choses saintes, de peur qu'ils ne meurent~; c'est là ce que les fils de Kehath porteront de la tente d'assignation.
\TextTitle{Les fonctions d'Eléazar}
\VS{16}Et Eléazar fils d'Aaron, le prêtre, aura la surveillance de l'huile du luminaire, du parfum odoriférant, de l'offrande continuelle, et de l'huile de l'onction~; la charge de tout le tabernacle, et de toutes les choses qui sont dans le lieu saint, et de ses ustensiles\FTNT{Ex. 30:23-35.}.
\VS{17}Yahweh parla à Moïse et à Aaron, en disant~:
\VS{18}Ne retranchez pas la tribu des familles des Kehathites d'entre les Lévites.
\VS{19}Mais faites ceci pour eux, afin qu'ils vivent et ne meurent point~; c'est que quand ils approcheront du Saint des saints, Aaron et ses fils viendront, qui les placeront chacun à son service, et à sa charge.
\VS{20}Et ils n'entreront point pour regarder quand on enveloppera les choses saintes, afin qu'ils ne meurent point.
\TextTitle{Les fonctions des fils de Guerschon}
\VS{21}Yahweh parla à Moïse, en disant~:
\VS{22}Fais aussi le dénombrement des fils de Guerschon selon les maisons de leurs pères, et selon leurs familles~;
\VS{23}depuis l'âge de trente ans, et au dessus, jusqu'à l'âge de cinquante ans, dénombrant tous ceux qui entrent pour tenir leur rang, afin de s'employer à servir à la tente d'assignation.
\VS{24}C'est ici le service des familles des Guerschonites, ce à quoi, ils doivent servir et en ce qu'ils doivent porter.
\VS{25}Ils porteront donc les tapis du tabernacle, et la tente d'assignation, sa couverture, la couverture de peaux de taissons qui est sur lui par dessus, et le rideau de l'entrée de la tente d'assignation~;
\VS{26}les courtines du parvis, et le rideau de l'entrée de la porte du parvis, qui servent pour le tabernacle et pour l'autel tout autour, leurs cordages, et tous les ustensiles de leur service, et tout ce qui est fait pour eux~; c'est ce en quoi ils serviront.
\VS{27}Tout le service des fils de Guerschonites en tout ce qu'ils doivent porter, et en tout ce à quoi ils doivent servir, sera réglé par les ordres d'Aaron et de ses fils, et vous les chargerez d'observer tout ce qu'ils doivent porter.
\VS{28}C'est là le service des familles des fils des Guerschonites dans la tente d'assignation~; et leur charge sera sous la conduite d'Ithamar, fils d'Aaron, le prêtre.
\TextTitle{Les fonctions des fils de Merari}
\VS{29}Tu dénombreras aussi les fils de Mérari selon leurs familles et selon les maisons de leurs pères.
\VS{30}Tu les dénombreras depuis l'âge de trente ans et au dessus, jusqu'à l'âge de cinquante ans, tous ceux qui entrent en rang pour s'employer au service dans la tente d'assignation.
\VS{31}Or c'est ici la charge de ce qu'ils auront à porter, selon tout le service qu'ils auront à faire à la tente d'assignation, savoir les planches du tabernacle, ses barres, et ses piliers, avec ses bases\FTNT{Ex. 26:15.},
\VS{32}et les piliers du parvis tout autour, et leurs bases, leurs pieux, leurs cordages, tous leurs ustensiles, et tout ce dont on se sert en ces choses-là, et vous leur compterez, en les désignant par nom, tous les ustensiles, qu'ils auront en charge de porter, pièce par pièce.
\VS{33}C'est là le service des familles des fils de Merari, pour tout leur service à la tente d'assignation, sous la conduite d'Ithamar, fils d'Aaron, le prêtre.
\VS{34}Moïse, Aaron et les princes de l'assemblée dénombrèrent les fils des Kéhathites, selon leurs familles, et selon les maisons de leurs pères.
\VS{35}Depuis l'âge de trente ans, et au dessus, jusqu'à l'âge de cinquante ans, tous ceux qui entraient en rang pour servir à la tente d'assignation.
\VS{36}Et ceux dont on fit le dénombrement selon leurs familles, étaient deux mille sept cent cinquante.
\VS{37}Ce sont là les dénombrés des familles des Kéhathites, tous servant à la tente d'assignation, que Moïse et Aaron dénombrèrent selon le commandement que Yahweh avait fait par le moyen de Moïse.
\VS{38}Or quant aux dénombrés des fils de Guerschon selon leurs familles, et selon les maisons de leurs pères,
\VS{39}depuis l'âge de trente ans, et au dessus, jusqu'à l'âge de cinquante ans, tous ceux qui entraient en rang pour servir à la tente d'assignation,
\VS{40}ceux, dis-je, qui en furent dénombrés selon leurs familles, et selon les maisons de leurs pères, étaient deux mille six cent trente.
\VS{41}Ce sont là, les dénombrés des familles des fils de Guerschon, tous servant dans la tente d'assignation, que Moïse et Aaron dénombrèrent selon le commandement de Yahweh.
\VS{42}Et quant aux dénombrés des familles des fils de Merari, selon leurs familles, et selon les maisons de leurs pères,
\VS{43}depuis l'âge de trente ans, et au dessus, jusqu'à l'âge de cinquante ans, tous ceux qui entraient en rang, pour servir à la tente d'assignation~;
\VS{44}ceux, dis-je, qui en furent dénombrés selon leurs familles, étaient trois mille deux cents.
\VS{45}Ce sont là, les dénombrés des familles des fils de Merari, que Moïse et Aaron dénombrèrent selon le commandement que Yahweh avait fait par le moyen de Moïse.
\VS{46}Ainsi tous ces dénombrés, que Moïse, Aaron et les princes d'Israël dénombrèrent d'entre les Lévites, selon leurs familles, et selon les maisons de leurs pères~;
\VS{47}depuis l'âge de trente ans, et au dessus, jusqu'à l'âge de cinquante ans, tous ceux qui entraient en service pour s'employer en ce à quoi il fallait servir, et à ce qu'il fallait porter de la tente d'assignation.
\VS{48}Tous ceux, dis-je, qui en furent dénombrés, étaient huit mille cinq cent quatre-vingts.
\VS{49}On les dénombra selon le commandement que Yahweh en avait fait par le moyen de Moïse, chacun selon ce en quoi il avait à servir, et ce qu'il avait à porter, et la charge de chacun fut telle que Yahweh l'avait commandé à Moïse.
\Chap{5}
\TextTitle{Mise en garde contre toute souillure~; lois diverses}
\VerseOne{}Et Yahweh parla à Moïse, en disant~:
\VS{2}Ordonne aux enfants d'Israël qu'ils mettent hors du camp tout lépreux, tout homme ayant une gonorrhée, et tout homme souillé pour un mort\FTNT{Lé. 13~; Lé. 15.}.
\VS{3}Vous les mettrez dehors, tant l'homme que la femme, vous les mettrez, dis-je, hors du camp, afin qu'ils ne souillent point le camp au milieu duquel j'habite.
\VS{4}Et les enfants d'Israël firent ainsi, et les envoyèrent hors du camp, comme Yahweh l'avait dit à Moïse~; les enfants d'Israël firent ainsi.
\VS{5}Et Yahweh parla à Moïse, en disant~:
\VS{6}Parle aux enfants d'Israël~; quand un homme ou une femme aura commis un des péchés que l'homme commet en faisant un crime contre Yahweh, et qu'une telle personne en sera trouvée coupable~;
\VS{7}alors ils confesseront leur péché, qu'ils auront commis~; et le coupable restituera la somme totale de ce en quoi il aura été trouvé coupable, et il y ajoutera un cinquième par-dessus, et le donnera à celui contre qui il aura commis le délit.
\VS{8}Que si cet homme n'a personne à qui appartienne le droit de restituer pour retirer ce en quoi aura été commis le délit, cette chose-là sera restituée à Yahweh, et elle appartiendra au prêtre, outre le bélier expiatoire avec lequel on fera propitiation pour lui.
\VS{9}De même, toute offrande élevée d'entre toutes les choses sanctifiées des enfants d'Israël, qu'ils présenteront\FTNT{Ez. 44:30.} au prêtre, lui appartiendra.
\VS{10} Les choses donc que quelqu'un aura sanctifiées appartiendront au prêtre~; ce que chacun lui aura donné, lui appartiendra\FTNT{Lé. 10:12-13.}.
\VS{11}Yahweh parla à Moïse, en disant~:
\VS{12}Parle aux enfants d'Israël, et dis leur~: Si la femme de quelqu'un se détourne et lui devienne infidèle~;
\VS{13}et que quelqu'un aura couché avec elle, et l'aura connue, sans que son mari en ait rien su, mais qu'elle se soit cachée, et qu'elle se soit souillée, et qu'il n'y ait point de témoin contre elle, et qu'elle n'ait point été surprise~;
\VS{14}et que l'esprit de jalousie saisisse son mari, tellement qu'il soit jaloux de sa femme, parce qu'elle s'est souillée~; ou que l'esprit de jalousie le saisisse tellement, qu'il soit jaloux de sa femme, encore qu'elle ne se soit point souillée~;
\VS{15}cet homme-là fera venir sa femme devant le prêtre, et il apportera l'offrande de cette femme pour elle, savoir la dixième partie d'un epha de farine d'orge~; mais il ne répandra point d'huile dessus~; et il n'y mettra point d'encens~; car c'est un gâteau de jalousie, un gâteau de souvenir, pour remettre en mémoire l'iniquité\FTNT{Lé. 5:11.}.
\VS{16}Le prêtre la fera approcher et la fera tenir debout devant Yahweh.
\VS{17}Puis le prêtre prendra de l'eau sainte dans un vase de terre, et il prendra de la poussière qui sera sur le sol du tabernacle, et la mettra dans l'eau.
\VS{18}Ensuite le prêtre fera tenir debout la femme devant Yahweh, il découvrira la tête de cette femme, et lui posera sur les paumes des mains le gâteau de souvenir, le gâteau de jalousie~; le prêtre tiendra dans sa main les eaux amères, qui apportent la malédiction.
\VS{19}Et le prêtre fera jurer la femme et lui dira~: Si aucun homme n'a couché avec toi, et si étant sous la puissance de ton mari tu ne t'es point détournée et souillée, sois exempte du mal de ces eaux amères qui apportent la malédiction.
\VS{20}Mais si, étant sous la puissance de ton mari, tu t'es détournée et souillée, et si un autre homme que ton mari a couché avec toi,
\VS{21}alors le prêtre fera jurer la femme avec un serment d'imprécation et lui dira~: Que Yahweh te livre à la malédiction et à l'exécration au milieu de ton peuple, en faisant flétrir ta cuisse et enfler ton ventre,
\VS{22}et que ces eaux qui apportent la malédiction, entrent dans tes entrailles pour te faire enfler le ventre et flétrir ta cuisse~! Alors la femme répondra~: Amen~! Amen~!
\VS{23}Ensuite le prêtre écrira dans un livre ces imprécations, et les effacera avec les eaux amères.
\VS{24}Et il fera boire à la femme les eaux amères qui apportent la malédiction, et les eaux qui apportent la malédiction entreront en elle pour être amères.
\VS{25}Le prêtre donc prendra des mains de la femme le gâteau de jalousie, et l'agitera de côté et d'autre devant Yahweh, et l'offrira sur l'autel~;
\VS{26}le prêtre prendra une poignée de cette offrande comme souvenir\FTNT{Voir commentaire en Lé. 2:2.}, et il la brûlera sur l'autel. C'est après cela qu'il fera boire les eaux à la femme.
\VS{27}Et après qu'il lui aura fait boire les eaux, s'il est vrai qu'elle se soit souillée et qu'elle a été infidèle à son mari, les eaux qui apportent la malédiction entreront en elle et lui seront amères, et son ventre enflera, sa cuisse se flétrira, et cette femme sera assujettie à l'exécration du serment au milieu de son peuple.
\VS{28}Mais si la femme ne s'est point souillée, mais qu'elle soit pure, elle sera reconnue innocente et aura des enfants.
\VS{29}Telle est la loi sur la jalousie, quand la femme qui est sous la puissance de son mari se détourne et se souille,
\VS{30}ou quand un mari saisi d'un esprit de jalousie a des soupçons sur sa femme~: Le prêtre la fera tenir debout devant Yahweh et fera à l'égard de cette femme tout ce qui est ordonné par cette loi.
\VS{31}Le mari sera exempt de faute, mais cette femme portera son iniquité.
\Chap{6}
\TextTitle{Le vœu de naziréat}
\VerseOne{}Yahweh parla à Moïse, en disant~:
\VS{2}Parle aux enfants d'Israël, et dis-leur~: Lorsqu'un homme ou une femme se consacrera en faisant un vœu de naziréat pour se consacrer à Yahweh,
\VS{3}il s'abstiendra de vin et de boisson forte, il ne boira ni vinaigre fait de vin, ni vinaigre fait avec une boisson forte~; il ne boira d'aucune liqueur de raisins, et il ne mangera point de raisins, frais ou secs.
\VS{4}Durant tous les jours de son naziréat il ne mangera d'aucun fruit de la vigne, depuis les pépins jusqu'à la peau du raisin\FTNT{Jg. 13:7~; Lu. 1:15.}.
\VS{5}Le rasoir ne passera point sur sa tête durant tous les jours de son naziréat. Il sera saint jusqu'à ce que les jours pour lesquels il s'est consacré à Yahweh soient accomplis, et il laissera croître les cheveux de sa tête\FTNT{Jg. 13:5~; 1 S. 1:11.}.
\VS{6}Durant tous les jours pour lesquels il s'est consacré à Yahweh il ne s'approchera d'aucune personne morte\FTNT{Lé. 21:1-4.}~;
\VS{7}il ne se souillera point à la mort de son père, ni de sa mère, ni de son frère, ni de sa sœur, car il porte sur sa tête la consécration de son Dieu.
\VS{8}Durant tous les jours de son naziréat, il sera consacré à Yahweh.
\VS{9}Que si quelqu'un vient à mourir subitement près de lui, la tête de son naziréat sera souillée, et il rasera sa tête au jour de sa purification, il la rasera le septième jour.
\VS{10}Le huitième jour, il apportera au prêtre deux tourterelles ou deux pigeonneaux, à l'entrée de la tente d'assignation\FTNT{Lé. 1~; Lé. 12:6.}.
\VS{11}Et le prêtre en sacrifiera l'un pour le sacrifice d'expiation et l'autre en holocauste, et il fera propitiation pour lui de ce qu'il a péché à l'occasion du mort. Il sanctifiera donc ainsi sa tête en ce jour-là.
\VS{12}Et il séparera à Yahweh les jours de son naziréat, offrant un agneau d'un an pour le délit, et les premiers jours seront comptés pour rien, car son naziréat a été souillé.
\VS{13}Or c'est ici la loi du naziréen. Lorsque les jours de son naziréat seront accomplis, on le fera venir à la porte de la tente d'assignation.
\VS{14}Il présentera son offrande à Yahweh~: Un agneau d'un an et sans défaut pour l'holocauste, une brebis d'un an et sans défaut pour le sacrifice d'expiation, et un bélier sans défaut pour le sacrifice d'offrande de paix\FTNT{Voir commentaire en Lé. 3:1.}.
\VS{15}Une corbeille de pains sans levain, de gâteaux de fine farine, pétrie à l'huile, et de galettes sans levain, oints d'huile, avec leur gâteau, et leurs libations~;
\VS{16}lesquels, le prêtre offrira devant Yahweh~; il sacrifiera aussi son offrande pour le péché, et son holocauste.
\VS{17}Et il offrira le bélier en sacrifice d'offrande de paix à Yahweh, avec la corbeille des pains sans levain~; le prêtre offrira aussi son gâteau, et sa libation.
\VS{18}Et le naziréen rasera la tête de son naziréat à l'entrée de la tente d'assignation, et prendra les cheveux de la tête de son naziréat, et les mettra sur le feu qui est sous le sacrifice d'offrande de paix.
\VS{19}Et le prêtre prendra l'épaule cuite du bélier, et un gâteau sans levain de la corbeille, et une galette sans levain, et les mettra sur les paumes des mains du naziréen, après qu'il se sera fait raser son naziréat.
\VS{20}Et le prêtre les agitera de côté et d'autre devant Yahweh~: C'est une chose sainte qui appartient au prêtre, avec la poitrine agitée et l'épaule offerte par élévation. Et après cela le naziréen boira du vin\FTNT{Lé. 7:32-34~; Ex. 29:24-27.}.
\VS{21}Telle est la loi du naziréen qui aura voué à Yahweh son offrande pour son naziréat, outre ce qu'il aura encore moyen d'offrir~; il fera selon son vœu qu'il aura voué, suivant la loi de son naziréat.
\TextTitle{Aaron et ses fils bénissent Israël}
\VS{22}Yahweh parla à Moïse, en disant~:
\VS{23}Parle à Aaron et à ses fils, et dis-leur~: Vous bénirez ainsi les enfants d'Israël, en leur disant~:
\VS{24}Yahweh te bénisse, et te garde~!
\VS{25}Yahweh fasse luire sa face sur toi, et te fasse grâce\FTNT{Ps. 67:2~; Ps. 119:135.}~!
\VS{26}Yahweh tourne sa face vers toi, et te donne la paix~!
\VS{27}Ils mettront donc mon Nom sur les enfants d'Israël, et je les bénirai.
\Chap{7}
\TextTitle{Les offrandes des princes}
\VerseOne{}Or il arriva le jour que Moïse eut achevé de dresser le tabernacle, et qu'il l'eut oint et sanctifié avec tous ses ustensiles, de même que l'autel avec tous ses ustensiles, il arriva, dis-je, après qu'il les eut oints et sanctifiés~;
\VS{2}que les princes d'Israël, et les chefs des maisons de leurs pères, qui sont les princes des tribus, et qui avaient assisté à faire les dénombrements, firent leur offrande.
\VS{3}Et ils amenèrent leur offrande devant Yahweh~: Six chars couverts et douze bœufs~; chaque char pour deux des princes, et chaque bœuf pour chacun d'eux~; ils les offrirent devant le tabernacle.
\VS{4}Alors Yahweh parla à Moïse, en disant~:
\VS{5}Prends d'eux ces choses, et elles seront employées pour le service de la tente d'assignation~; et tu les donneras aux Lévites, à chacun selon ses fonctions.
\VS{6}Moïse prit donc les chars et les bœufs, et il les remit aux Lévites.
\VS{7}Il donna aux fils de Guerschon deux chars et quatre bœufs, selon leurs fonctions.
\VS{8}Mais il donna aux fils de Merari quatre chars et huit bœufs, selon leurs fonctions, sous la conduite d'Ithamar, fils d'Aaron, le prêtre.
\VS{9}Or il n'en donna point aux fils de Kehath, parce que le service du sanctuaire était de leur charge~; ils portaient ces choses saintes sur les épaules.
\VS{10}Et les princes présentèrent leur offrande pour la dédicace de l'autel, le jour où on l'oignit~; les princes, dis-je, présentèrent leur offrande devant l'autel.
\VS{11}Et Yahweh dit à Moïse~: Un des princes offrira un jour, et un autre l'autre jour, son offrande pour la dédicace de l'autel.
\VS{12}Le premier jour donc, Nachschon, fils d'Amminadab, présenta son offrande pour la tribu de Juda.
\VS{13}Il offrit un plat d'argent du poids de cent trente sicles, un bassin d'argent de soixante-dix sicles, selon le sicle du sanctuaire, tous deux pleins de fine farine pétrie à l'huile, pour l'offrande~;
\VS{14}une coupe d'or de dix sicles pleine de parfum~;
\VS{15}un jeune taureau, un bélier, un agneau d'un an, pour l'holocauste~;
\VS{16}un jeune bouc pour le sacrifice d'expiation~;
\VS{17}et pour le sacrifice d'offrande de paix, deux bœufs, cinq béliers, cinq boucs, et cinq agneaux d'un an. Telle fut l'offrande de Nachschon, fils d'Amminadab.
\VS{18}Le second jour, Nethaneel, fils de Tsuar, chef de la tribu d'Issacar, présenta son offrande.
\VS{19}Et il offrit pour son offrande un plat d'argent du poids de cent trente sicles, un bassin d'argent de soixante-dix sicles, selon le sicle du sanctuaire, tous deux pleins de fine farine pétrie à l'huile, pour l'offrande~;
\VS{20}une coupe d'or de dix sicles pleine de parfum~;
\VS{21}un jeune taureau, un bélier, un agneau d'un an, pour l'holocauste~;
\VS{22}un jeune bouc pour le sacrifice d'expiation~;
\VS{23}et pour le sacrifice d'offrande de paix, deux bœufs, cinq béliers, cinq boucs, et cinq agneaux d'un an. Telle fut l'offrande de Nethaneel, fils de Tsuar.
\VS{24}Le troisième jour, Eliab, fils de Hélon, chef des fils de Zabulon, présenta son offrande.
\VS{25}Il offrit un plat d'argent du poids de cent trente sicles, un bassin d'argent de soixante-dix sicles, selon le sicle du sanctuaire, tous deux pleins de fine farine pétrie à l'huile, pour l'offrande~;
\VS{26}une coupe d'or de dix sicles pleine de parfum~;
\VS{27}un jeune taureau, un bélier, un agneau d'un an, pour l'holocauste~;
\VS{28}un jeune bouc pour le sacrifice d'expiation~;
\VS{29}et pour le sacrifice d'offrande de paix, deux bœufs, cinq béliers, cinq boucs, et cinq agneaux d'un an. Telle fut l'offrande d'Eliab, fils de Hélon.
\VS{30}Le quatrième jour, Elitsur, fils de Schedéur, prince des fils de Ruben, présenta son offrande.
\VS{31}Il offrit un plat d'argent du poids de cent trente sicles, un bassin d'argent de soixante-dix sicles, selon le sicle du sanctuaire, tous deux pleins de fine farine pétrie à l'huile, pour l'offrande~;
\VS{32}une coupe d'or de dix sicles pleine de parfum~;
\VS{33}un jeune taureau, un bélier, un agneau d'un an, pour l'holocauste~;
\VS{34}un jeune bouc pour le sacrifice d'expiation~;
\VS{35}et pour le sacrifice d'offrande de paix, deux bœufs, cinq béliers, cinq boucs, et cinq agneaux d'un an. Telle fut l'offrande d'Elitsur, fils de Schedéur.
\VS{36}Le cinquième jour, Schelumiel, fils de Tsurischaddaï, prince des fils de Siméon, présenta son offrande.
\VS{37}Il offrit un plat d'argent du poids de cent trente sicles, un bassin d'argent de soixante-dix sicles, selon le sicle du sanctuaire, tous deux pleins de fine farine pétrie à l'huile pour l'offrande~;
\VS{38}une coupe d'or de dix sicles pleine de parfum~;
\VS{39}un jeune taureau, un bélier, un agneau d'un an, pour l'holocauste~;
\VS{40}un jeune bouc pour le sacrifice d'expiation~;
\VS{41}et pour le sacrifice d'offrande de paix, deux bœufs, cinq béliers, cinq boucs, et cinq agneaux d'un an. Telle fut l'offrande de Schelumiel, fils de Tsurischaddaï.
\VS{42}Le sixième jour, Eliasaph, fils de Déuel, prince des fils de Gad, présenta son offrande.
\VS{43}Il offrit un plat d'argent du poids de cent trente sicles, un bassin d'argent de soixante-dix sicles, selon le sicle du sanctuaire, tous deux pleins de fine farine pétrie à l'huile pour l'offrande~;
\VS{44}une coupe d'or de dix sicles pleine de parfum~;
\VS{45}un jeune taureau, un bélier, un agneau d'un an, pour l'holocauste~;
\VS{46}un jeune bouc pour le sacrifice d'expiation~;
\VS{47}et pour le sacrifice d'offrande de paix, deux bœufs, cinq béliers, cinq boucs, et cinq agneaux d'un an. Telle fut l'offrande d'Eliasaph, fils de Déuel.
\VS{48}Le septième jour, Elischama, fils d'Ammihud, prince des fils d'Ephraïm, présenta son offrande.
\VS{49}Il offrit un plat d'argent, du poids de cent trente sicles, un bassin d'argent de soixante-dix sicles, selon le sicle du sanctuaire, tous deux pleins de fine farine pétrie à l'huile pour l'offrande~;
\VS{50}une coupe d'or de dix sicles pleine de parfum~;
\VS{51}un jeune taureau, un bélier, un agneau d'un an, pour l'holocauste~;
\VS{52}un jeune bouc pour le sacrifice d'expiation~;
\VS{53}et pour le sacrifice d'offrande de paix, deux bœufs, cinq béliers, cinq boucs, et cinq agneaux d'un an. Telle fut l'offrande d'Elischama, fils d'Ammihud.
\VS{54}Le huitième jour, Gamliel, fils de Pedahtsur, prince des fils de Manassé, présenta son offrande.
\VS{55}Il offrit un plat d'argent, du poids de cent trente sicles, un bassin d'argent de soixante-dix sicles, selon le sicle du sanctuaire, tous deux pleins de fine farine pétrie à l'huile pour l'offrande~;
\VS{56}une coupe d'or de dix sicles pleine de parfum~;
\VS{57}un jeune taureau, un bélier, un agneau d'un an, pour l'holocauste~;
\VS{58}un jeune bouc pour le sacrifice d'expiation~;
\VS{59}et pour le sacrifice d'offrande de paix, deux bœufs, cinq béliers, cinq boucs, et cinq agneaux d'un an. Telle fut l'offrande de Gamliel, fils de Pedahtsur.
\VS{60}Le neuvième jour, Abidan, fils de Guideoni, prince des fils de Benjamin, présenta son offrande.
\VS{61}Il offrit un plat d'argent, du poids de cent trente sicles, un bassin d'argent de soixante-dix sicles, selon le sicle du sanctuaire, tous deux pleins de fine farine pétrie à l'huile pour l'offrande~;
\VS{62}une coupe d'or de dix sicles pleine de parfum~;
\VS{63}un jeune taureau, un bélier, un agneau d'un an, pour l'holocauste~;
\VS{64}un jeune bouc pour le sacrifice d'expiation~;
\VS{65}et pour le sacrifice d'offrande de paix, deux bœufs, cinq béliers, cinq boucs, et cinq agneaux d'un an. Telle fut l'offrande d'Abidan, fils de Guideoni.
\VS{66}Le dixième jour, Ahiézer, fils d'Ammischaddaï, prince des fils de Dan, présenta son offrande.
\VS{67}Il offrit un plat d'argent du poids de cent trente sicles, un bassin d'argent de soixante-dix sicles, selon le sicle du sanctuaire, tous deux pleins de fine farine pétrie à l'huile pour l'offrande~;
\VS{68}une coupe d'or de dix sicles pleine de parfum~;
\VS{69}Un jeune taureau, un bélier, un agneau d'un an, pour l'holocauste~;
\VS{70}un jeune bouc pour le sacrifice d'expiation~;
\VS{71}et pour le sacrifice d'offrande de paix, deux bœufs, cinq béliers, cinq boucs, et cinq agneaux d'un an. Telle fut l'offrande d'Ahiézer, fils d'Ammischaddaï.
\VS{72}Le onzième jour, Paguiel, fils d'Ocran, prince des fils d'Aser, présenta son offrande.
\VS{73}Il offrit un plat d'argent, du poids de cent trente sicles, un bassin d'argent de soixante-dix sicles, selon le sicle du sanctuaire, tous deux pleins de fine farine pétrie à l'huile pour l'offrande~;
\VS{74}une coupe d'or de dix sicles pleine de parfum~;
\VS{75}un jeune taureau, un bélier, un agneau d'un an, pour l'holocauste~;
\VS{76}un jeune bouc pour le sacrifice d'expiation~;
\VS{77}et pour le sacrifice d'offrande de paix, deux bœufs, cinq béliers, cinq boucs, et cinq agneaux d'un an. Telle fut l'offrande de Paguiel, fils d'Ocran.
\VS{78}Le douzième jour, Ahira, fils d'Enan, prince des fils de Nephthali, présenta son offrande.
\VS{79}Il offrit un plat d'argent du poids de cent trente sicles, un bassin d'argent de soixante-dix sicles, selon le sicle du sanctuaire, tous deux pleins de fine farine pétrie à l'huile pour l'offrande~;
\VS{80}une coupe d'or de dix sicles pleine de parfum~;
\VS{81}un jeune taureau, un bélier, un agneau d'un an, pour l'holocauste~;
\VS{82}un jeune bouc pour le sacrifice d'expiation~;
\VS{83}et pour le sacrifice d'offrande de paix, deux bœufs, cinq béliers, cinq boucs, et cinq agneaux d'un an. Telle fut l'offrande d'Ahira, fils d'Enan.
\TextTitle{Les dons des princes}
\VS{84}Telle fut la dédicace de l'autel, qui fut faite par les princes d'Israël, lorsqu'il fut oint. Douze plats d'argent, douze bassins d'argent, douze tasses d'or~;
\VS{85}chaque plat d'argent était de cent trente sicles, et chaque bassin de soixante-dix~; tout l'argent de ces ustensiles montait à deux mille quatre cents sicles, selon le sicle du sanctuaire~;
\VS{86}douze coupes d'or pleines de parfum, chacune de dix sicles, selon le sicle du sanctuaire~; tout l'or des tasses montait à cent-vingt sicles.
\VS{87}Tous les animaux pour l'holocauste étaient douze veaux, douze béliers, et douze agneaux d'un an, avec leurs offrandes, et douze jeunes boucs pour le sacrifice d'expiation.
\VS{88}Tous les animaux du sacrifice d'offrande de paix étaient vingt-quatre veaux, avec soixante béliers, soixante boucs, et soixante agneaux d'un an. Telle fut donc la dédicace de l'autel, après qu'on l'eut oint.
\VS{89}Et quand Moïse entrait dans la tente d'assignation pour parler avec Yahweh, il entendait une voix qui lui parlait du haut du propitiatoire placé sur l'arche du témoignage, entre les deux chérubins. Et il lui parlait\FTNT{Ex. 25:22.}.
\Chap{8}
\TextTitle{Les lampes sur le chandelier}
\VerseOne{}Yahweh parla à Moïse, en disant~:
\VS{2}Parle à Aaron, et tu lui diras~: Quand tu allumeras les lampes, les sept lampes éclaireront sur le devant du chandelier\FTNT{Ex. 25:37.}.
\VS{3}Et Aaron fit ainsi~; il plaça les lampes pour éclairer sur le devant du chandelier, comme Yahweh l'avait commandé à Moïse.
\VS{4}Or le chandelier était fait de telle manière, qu'il était d'or battu au marteau, d'ouvrage fait au marteau, sa tige aussi, et ses fleurs. On fit ainsi le chandelier selon le modèle que Yahweh en avait fait voir à Moïse\FTNT{Ex. 25:31-40.}.
\TextTitle{Purification des Lévites}
\VS{5}Puis Yahweh parla à Moïse, en disant~:
\VS{6}Prends les Lévites du milieu des enfants d'Israël, et purifie-les.
\VS{7}Tu leur feras ainsi pour les purifier. Tu feras aspersion sur eux de l'eau de purification~; ils feront passer le rasoir sur toute leur chair, ils laveront leurs vêtements, et ils se purifieront.
\VS{8}Puis ils prendront un jeune taureau avec son offrande de gâteau de fine farine pétrie à l'huile~; et tu prendras un autre jeune taureau pour le sacrifice d'expiation.
\VS{9}Alors tu feras approcher les Lévites devant la tente d'assignation, et tu convoqueras toute l'assemblée des enfants d'Israël.
\VS{10}Tu feras, dis-je, approcher les Lévites devant Yahweh, et les enfants d'Israël poseront leurs mains sur les Lévites.
\VS{11}Et Aaron fera tourner de côté et d'autre les Lévites devant Yahweh, comme offrande de la part des enfants d'Israël, et ils seront employés au service de Yahweh.
\VS{12}Et les Lévites poseront leurs mains sur la tête des veaux~; puis tu offriras l'un en sacrifice pour l'expiation, et l'autre en holocauste à Yahweh, afin de faire propitiation pour les Lévites.
\VS{13}Après tu feras tenir les Lévites devant Aaron et devant ses fils, et tu les présenteras en offrande à Yahweh.
\VS{14}Ainsi tu sépareras les Lévites du milieu des enfants d'Israël, et les Lévites m'appartiendront.
\VS{15}Après cela, les Lévites viendront pour servir dans la tente d'assignation quand tu les auras purifiés et présentés en offrande.
\VS{16}Car ils me sont entièrement donnés du milieu des enfants d'Israël~; je les ai pris pour moi à la place des premiers-nés~; de tous les premiers-nés des fils d'Israël.
\VS{17}Car tout premier-né des enfants d'Israël est à moi, tant des hommes que des animaux~; je me les suis consacrés le jour où j'ai frappé tous les premiers-nés dans le pays d'Egypte.
\VS{18}Or j'ai pris les Lévites au lieu de tous les premiers-nés d'entre les enfants d'Israël.
\VS{19}Et j'ai entièrement donné, d'entre les enfants d'Israël, les Lévites à Aaron et à ses fils, pour faire le service des enfants d'Israël dans la tente d'assignation, et pour faire propitiation pour les enfants d'Israël~; afin qu'il n'y ait point de plaie sur les enfants d'Israël, comme il y aurait si les enfants d'Israël s'approchaient du sanctuaire.
\VS{20}Moïse, Aaron et toute l'assemblée des enfants d'Israël firent à l'égard des Lévites tout ce que Yahweh avait ordonné à Moïse touchant les Lévites~; ainsi firent les enfants d'Israël.
\VS{21}Les Lévites donc se purifièrent, et lavèrent leurs vêtements, et Aaron les fit tourner de côté et d'autre comme une offrande devant Yahweh, et il fit propitiation pour eux afin de les purifier.
\VS{22}Cela étant fait, les Lévites vinrent faire leur service dans la tente d'assignation devant Aaron, et devant ses fils, selon ce que Yahweh avait commandé à Moïse touchant les Lévites~; ainsi fut-il fait à leur égard.
\VS{23}Puis Yahweh parla à Moïse, en disant~:
\VS{24}Voici ce qui concerne les Lévites. Depuis l'âge de vingt-cinq ans et au-dessus, tout Lévite entrera en fonction dans la tente d'assignation~;
\VS{25}dès l'âge de cinquante ans, il sortira du service et ne servira plus.
\VS{26}Cependant il servira ses frères dans la tente d'assignation, pour garder ce qui leur a été commis, mais il ne fera plus de service. Tu agiras ainsi à l'égard des Lévites pour ce qui concerne leurs fonctions.
\Chap{9}
\TextTitle{La Pâque}
\VerseOne{}Yahweh avait aussi parlé à Moïse dans le désert de Sinaï, le premier mois de la seconde année, après qu'ils furent sortis du pays d'Egypte, en disant~:
\VS{2}Que les enfants d'Israël célèbrent la Pâque\FTNT{Ex. 12~; 1 Co. 5:7.} au temps fixé.
\VS{3}Vous la ferez en sa saison, le quatorzième jour de ce mois entre les deux soirs, selon toutes ses ordonnances et selon tout ce qu'il faut y faire.
\VS{4}Moïse donc, parla aux enfants d'Israël afin qu'ils célèbrent la Pâque.
\VS{5}Et ils firent la Pâque le quatorzième jour du premier mois, entre les deux soirs, dans le désert de Sinaï~; selon tout ce que Yahweh avait commandé à Moïse, les enfants d'Israël le firent ainsi.
\VS{6}Or il y eut quelques-uns qui étaient impurs à cause d'un mort et qui ne purent célébrer la Pâque ce jour-là. Ils se présentèrent ce même jour devant Moïse et devant Aaron,
\VS{7}et ces hommes leur dirent~: Nous sommes impurs à cause d'un mort, pourquoi serions-nous privés de présenter l'offrande à Yahweh dans sa saison au milieu des enfants d'Israël~?
\VS{8}Et Moïse leur dit~: Arrêtez-vous, et j'entendrai ce que Yahweh commandera sur votre sujet.
\VS{9}Alors Yahweh parla à Moïse, en disant~:
\VS{10}Parle aux enfants d'Israël, et dis-leur~: Si quelqu'un d'entre vous, ou de votre postérité, est impur à cause d'un mort, ou est en voyage dans un lieu éloigné, il célébrera cependant la Pâque en l'honneur de Yahweh.
\VS{11}Ils la feront le quatorzième jour du second mois, entre les deux soirs~; et ils la mangeront avec du pain sans levain et des herbes amères\FTNT{Ex. 12:10~; Ex. 23:18~; Ex. 34:25~; De. 16:4~; Jn. 19:33-36.}.
\VS{12}Ils n'en laisseront rien jusqu'au matin, et n'en briseront point les os. Ils la feront selon toutes les ordonnances de la Pâque.
\VS{13}Mais si celui qui est pur et qui n'est pas en voyage s'abstient de célébrer la Pâque, il sera retranché d'entre ses peuples parce qu'il n'a pas présenté l'offrande de Yahweh en sa saison.
\VS{14}Et si un étranger en séjour chez vous célèbre la Pâque de Yahweh, il la fera selon l'ordonnance de la Pâque. II y aura une même ordonnance entre vous, pour l'étranger comme pour celui qui est né au pays\FTNT{Ex. 12:49.}.
\TextTitle{La nuée conduit Israël}
\VS{15}Or le jour où le tabernacle fut dressé, la nuée couvrit le tabernacle de la tente d'assignation~; et le soir jusqu'au matin, elle parut sur le tabernacle avec l'apparence d'un feu\FTNT{Ex. 13:21-22~; Ex. 40:34-38~; De. 1:33.}.
\VS{16}Il en fut ainsi continuellement~; la nuée le couvrait, mais elle paraissait la nuit comme du feu.
\VS{17}Et selon que la nuée se levait de dessus le tabernacle, les enfants d'Israël partaient~; et au lieu où la nuée s'arrêtait, les enfants d'Israël y campaient.
\VS{18}Les enfants d'Israël marchaient sur le commandement de Yahweh, et ils campaient sur le commandement de Yahweh~; ils campaient aussi longtemps que la nuée se tenait sur le tabernacle.
\VS{19}Et quand la nuée restait plusieurs jours sur le tabernacle, les enfants d'Israël observaient l'ordre de Yahweh, et ne partaient point.
\VS{20}Et pour peu de jours que la nuée fût sur le tabernacle, ils campaient sur le commandement de Yahweh, et ils partaient sur le commandement de Yahweh.
\VS{21}Et quand la nuée y était depuis le soir jusqu'au matin, et que la nuée se levait au matin, ils partaient~; fût-ce de jour ou de nuit, quand la nuée se levait, ils partaient.
\VS{22}Si la nuée s'arrêtait sur le tabernacle deux jours, ou un mois, ou plus longtemps, les enfants d'Israël restaient campés, et ne partaient point~; mais quand elle se levait, ils partaient.
\VS{23}Ils campaient donc au commandement de Yahweh, et ils partaient au commandement de Yahweh~; et ils prenaient garde à Yahweh, suivant le commandement de Yahweh, qu'il leur faisait savoir par Moïse.
\Chap{10}
\TextTitle{Les trompettes d'argent}
\VerseOne{}Puis Yahweh parla à Moïse, en disant~:
\VS{2}Fais-toi deux trompettes d'argent, battues au marteau. Elles te serviront pour convoquer l'assemblée, et pour le départ des camps.
\VS{3}Quand on en sonnera, toute l'assemblée s'assemblera auprès de toi à l'entrée de la tente d'assignation.
\VS{4}Et quand on sonnera d'une seule, les princes, qui sont les chefs des milliers d'Israël, s'assembleront vers toi.
\VS{5}Mais quand vous sonnerez avec un retentissement bruyant, ceux qui campent à l'orient partiront.
\VS{6}Et quand vous sonnerez la seconde fois avec un retentissement bruyant, ceux qui campent au midi partiront, on sonnera avec un retentissement bruyant, pour leur départ.
\VS{7}Lorsque vous convoquerez l'assemblée, vous ne sonnerez pas avec un retentissement bruyant.
\VS{8}Or les fils d'Aaron, les prêtres, sonneront des trompettes. Ce sera une loi perpétuelle pour vous et pour vos descendants.
\VS{9}Et lorsque, dans votre pays, vous irez à la guerre contre l'ennemi qui vous combattra, vous sonnerez des trompettes avec un retentissement bruyant, et Yahweh votre Dieu, se souviendra de vous, et vous serez délivrés de vos ennemis.
\VS{10}Aussi dans vos jours de joie, dans vos fêtes solennelles, et au commencement de vos mois, vous sonnerez des trompettes en offrant vos holocaustes et vos sacrifices d'offrande de paix, et elles vous serviront de souvenir devant votre Dieu. Je suis Yahweh, votre Dieu.
\TextTitle{La nuée se lève, reprise de la marche dans le désert}
\VS{11}Or il arriva le vingtième jour du second mois de la seconde année, que la nuée se leva de dessus le tabernacle du témoignage.
\VS{12}Et les enfants d'Israël partirent du désert de Sinaï, selon l'ordre fixé pour leur marche. La nuée se posa dans le désert de Paran.
\VS{13}Ils partirent donc pour la première fois, suivant le commandement de Yahweh, déclaré par Moïse.
\VS{14}Et la bannière du camp des fils de Juda partit la première, selon leurs armées. Nachschon, fils d'Amminadab, commandait l'armée de Juda~;
\VS{15}et Nethaneel, fils de Tsuar, commandait l'armée de la tribu des fils d'Issacar~;
\VS{16}et Eliab, fils de Hélon, commandait l'armée de la tribu des fils de Zabulon.
\VS{17}Et le tabernacle fut démonté~; et les fils de Guerschon, et les fils de Merari, qui portaient le tabernacle, partirent.
\VS{18}Puis la bannière du camp de Ruben partit, selon leurs armées. Et Elitsur, fils de Schedéur, commandait l'armée de Ruben~;
\VS{19}et Schelumiel, fils de Tsurischaddaï, commandait l'armée de la tribu des fils de Siméon~;
\VS{20}et Eliasaph, fils de Déuel, commandait l'armée des fils de Gad.
\VS{21}Alors les Kehathites, qui portaient le sanctuaire, partirent~; cependant on dressait le tabernacle, en attendant leur arrivée.
\VS{22}Puis la bannière du camp des fils d'Ephraïm partit, selon leurs armées. Elischama, fils d'Ammihud, commandait l'armée d'Ephraïm~;
\VS{23}et Gamliel, fils de Pedahtsur, commandait l'armée de la tribu des fils de Manassé~;
\VS{24}et Abidan, fils de Guideoni, commandait l'armée de la tribu des fils de Benjamin.
\VS{25}Enfin la bannière des camps des fils de Dan, qui faisait l'arrière-garde, partit, selon leurs armées~; et Ahiézer, fils d'Ammischaddaï, commandait l'armée de Dan.
\VS{26}Et Paguiel, fils d'Ocran, commandait l'armée de la tribu des fils d'Aser~;
\VS{27}et Ahira, fils d'Enan, commandait l'armée de la tribu des fils de Nephthali.
\VS{28}Tel fut l'ordre d'après lequel les enfants d'Israël se mirent en marche selon leurs armées, c'est ainsi qu'ils partirent.
\VS{29}Or Moïse dit à Hobab, fils de Réuel, le Madianite, beau-père de Moïse~: Nous allons au lieu dont Yahweh a dit~: Je vous le donnerai. Viens avec nous, et nous te ferons du bien, car Yahweh a promis de faire du bien à Israël.
\VS{30}Et Hobab lui répondit~: Je n'irai point, mais je m'en irai dans mon pays, et vers ma parenté.
\VS{31}Et Moïse lui dit~: Je te prie, ne nous quitte pas~; car tu nous serviras de guide, parce que tu connais les lieux où nous aurons à camper dans le désert.
\VS{32}Et il arrivera que, quand tu seras venu avec nous, et que le bien que Yahweh doit nous faire sera arrivé, nous te ferons aussi du bien.
\VS{33}Ainsi ils partirent de la montagne de Yahweh et ils marchèrent trois jours~; et l'arche de l'alliance de Yahweh alla devant eux, et fit une marche de trois jours pour leur chercher un lieu de repos.
\VS{34}Et la nuée de Yahweh était sur eux le jour, quand ils partaient du camp.
\VS{35}Or il arrivait qu'au départ de l'arche, Moïse disait~: Lève-toi, ô Yahweh, et tes ennemis seront dispersés, et ceux qui te haïssent s'enfuiront de devant toi\FTNT{Ps. 68:2.}~!
\VS{36}Et quand on la posait, il disait~: Reviens Yahweh, aux dix mille milliers d'Israël~!
\Chap{11}
\TextTitle{Jugement contre les murmures du peuple}
\VerseOne{}Après, il arriva que le peuple murmura et cela déplut aux oreilles de Yahweh. Lorsque Yahweh l'entendit, sa colère s'enflamma, et le feu de Yahweh s'alluma parmi eux et en consuma l'extrémité du camp.
\VS{2}Alors le peuple cria à Moïse. Moïse pria Yahweh, et le feu s'éteignit.
\VS{3}Et on nomma ce lieu-là Tabeéra, parce que le feu de Yahweh s'était allumé parmi eux.
\TextTitle{Le peuple regrette l'Egypte}
\VS{4}Et le peuple nombreux qui se trouvaient au milieu d'Israël fut épris de convoitise~; et même, les enfants d'Israël se mirent à pleurer disant~: Qui nous donnera de la viande à manger\FTNT{Ex. 16:3~; Ps. 106:14~; 1 Co. 10:6.}~?
\VS{5}Nous nous souvenons des poissons que nous mangions en Egypte, et qui ne nous coûtaient rien, des concombres, des melons, des poireaux, des oignons, et de l'ail.
\VS{6}Et maintenant nos âmes sont asséchées~; nos yeux ne voient que de la manne\FTNT{Ps. 78:24.}.
\VS{7}Or la manne était comme la graine de coriandre, et avait l'apparence du bdellium\FTNT{Ex. 16:14-31~; Jn. 6:31-58.}.
\VS{8}Le peuple se dispersait et la ramassait, il la moulait aux meules, ou la pilait dans un mortier, il la cuisait au pot et en faisait des gâteaux. Elle avait le goût d'une liqueur d'huile fraîche.
\VS{9}Et quand la rosée descendait la nuit sur le camp, la manne y descendait aussi.
\TextTitle{Moïse dans l'affliction}
\VS{10}Moïse donc entendit le peuple qui pleurait, chacun dans sa famille et à l'entrée de sa tente. La colère de Yahweh s'enflamma fortement et Moïse en fut attristé.
\VS{11}Et Moïse dit à Yahweh~: Pourquoi affliges-tu ton serviteur et pourquoi n'ai-je pas trouvé grâce à tes yeux, que tu aies mis sur moi la charge de tout ce peuple~?
\VS{12}Est-ce moi qui ai conçu tout ce peuple ou l'ai-je engendré pour que tu me dises~: Porte-le dans ton sein comme le nourricier porte un enfant qui tète, porte-le jusqu'au pays que tu as juré à ses pères~?
\VS{13}D'où aurais-je de la viande pour en donner à tout ce peuple~? Car il pleure auprès de moi, en disant~: Donne-nous de la viande à manger~!
\VS{14}Je ne puis, à moi seul, porter tout ce peuple, car il est trop pesant pour moi\FTNT{De. 1:9-12.}.
\VS{15}Si tu agis ainsi à mon égard, tue-moi, je te prie donc, si j'ai trouvé grâce à tes yeux, et que je ne voie pas mon malheur.
\TextTitle{Yahweh établit soixante-dix anciens autour de Moïse\FTNTT{Ex. 18:19.}}
\VS{16}Alors Yahweh dit à Moïse~: Assemble-moi soixante-dix hommes des anciens d'Israël, que tu connais être les anciens du peuple et ses officiers, et amène-les à la tente d'assignation, et qu'ils s'y présentent avec toi.
\VS{17}Puis je descendrai, et je parlerai là avec toi, je mettrai de l'Esprit qui est sur toi sur eux~; afin qu'ils portent avec toi la charge du peuple, et que tu ne la portes pas toi seul.
\VS{18}Et tu diras au peuple~: Sanctifiez-vous pour demain, et vous mangerez de la viande~; puisque vous avez pleuré aux oreilles de Yahweh, en disant~: Qui nous fera manger de la viande~? Car nous étions bien en Egypte. Ainsi Yahweh vous donnera de la viande, et vous en mangerez.
\VS{19}Vous n'en mangerez pas un jour, ni deux jours, ni cinq jours, ni dix jours, ni vingt jours,
\VS{20}mais jusqu'à un mois entier, jusqu'à ce qu'elle vous sorte par les narines, et que vous en ayez du dégoût, parce que vous avez rejeté Yahweh qui est au milieu de vous~; vous avez pleuré devant lui, en disant~: Pourquoi sommes-nous sortis d'Egypte~?
\VS{21}Moïse dit~: Six cent mille hommes de pied forment ce peuple au milieu duquel je suis, et tu as dit~: Je leur donnerai de la viande afin qu'ils en mangent un mois entier~!
\VS{22}Leur tuera-t-on des brebis ou des bœufs, en sorte qu'il y en ait assez pour eux~? Ou leur assemblera-t-on tous les poissons de la mer, en sorte qu'ils en aient assez~?
\VS{23}Yahweh répondit à Moïse: La main de Yahweh serait-elle trop courte~? Tu verras maintenant si ce que je t'ai dit arrivera ou non\FTNT{Es. 50:2~; Es. 59:1-2.}.
\VS{24}Moïse donc sortit et rapporta au peuple les paroles de Yahweh. Il assembla soixante-dix hommes des anciens du peuple, et les plaça autour de la tente.
\VS{25}Yahweh descendit dans la nuée et parla à Moïse~; il prit de l'Esprit qui était sur lui et le mit sur les soixante-dix hommes anciens. Et dès que l'Esprit reposa sur eux, ils prophétisèrent~; mais ils ne continuèrent pas.
\TextTitle{Prophétie d'Eldad et de Médad}
\VS{26}Or il y eut deux hommes restés au camp, l'un s'appelait Eldad, et l'autre Médad, sur lesquels l'Esprit reposa. Ils étaient de ceux qui avaient été inscrits, mais ils n'étaient pas allés à la tente, et ils prophétisaient dans le camp.
\VS{27}Alors un garçon courut le rapporter à Moïse, en disant~: Eldad et Médad prophétisent dans le camp.
\VS{28}Et Josué, fils de Nun, qui servait Moïse, l'un de ses jeunes gens, répondit, en disant~: Mon seigneur Moïse, empêche-les.
\VS{29}Et Moïse lui répondit~: Es-tu jaloux pour moi~? Plût à Dieu que tout le peuple de Yahweh fût prophète, et que Yahweh mît son Esprit sur eux~!
\VS{30}Puis Moïse se retira au camp, lui et les anciens d'Israël.
\TextTitle{Les cailles et le jugement de Yahweh}
\VS{31}Alors Yahweh fit lever un vent de la mer qui amena des cailles et les répandit sur le camp environ le chemin d'une journée, de çà et de là, tout autour du camp~; et il y en avait presque la hauteur de deux coudées sur la terre\FTNT{Ex. 16:13-15~; Ps. 78:26-29~; Ps. 105:40.}.
\VS{32}Et le peuple se leva tout ce jour-là, et toute la nuit, et tout le jour suivant, et amassa des cailles~; celui qui en avait amassé le moins en avait dix homers~; et ils les étendirent soigneusement pour eux tout autour du camp.
\VS{33}Mais la chair était encore entre leurs dents, avant qu'elle fût mâchée, la colère de Yahweh s'embrasa contre le peuple, et il frappa le peuple d'une très grande plaie\FTNT{Ps. 78:30-31.}.
\VS{34}Et on nomma ce lieu-là Kibroth-Hattaava~; car on ensevelit là le peuple qui avait convoité.
\VS{35}Et de Kibroth-Hattaava le peuple s'en alla pour Hatséroth, et il s'arrêta à Hatséroth.
\Chap{12}
\TextTitle{Marie et Aaron murmurent contre Moïse}
\VerseOne{}Alors Marie et Aaron parlèrent contre Moïse au sujet de la femme éthiopienne\FTNT{Voir commentaire en Ge. 2:13.} qu'il avait prise, car il avait pris une femme éthiopienne.
\VS{2}Et ils dirent~: Est-ce seulement par Moïse que Yahweh parle~? N'est-ce pas aussi par nous qu'il parle~? Et Yahweh entendit cela. 
\VS{3}Or cet homme Moïse était un homme fort doux, plus que tous les hommes qui étaient sur la terre.
\VS{4}Et soudain Yahweh dit à Moïse, à Aaron, et à Marie~: Venez vous trois à la tente d'assignation~; et ils y allèrent eux trois.
\VS{5}Alors Yahweh descendit dans la colonne de nuée et se tint à l'entrée de la tente. Puis il appela Aaron et Marie, qui s'avancèrent tous les deux.
\VS{6}Et il dit~: Ecoutez maintenant mes paroles~! Lorsqu'il y aura parmi vous un prophète, moi qui suis Yahweh je me ferai bien connaître à lui en vision, et je lui parlerai en songe.
\VS{7}Il n'en est pas ainsi de mon serviteur Moïse, qui est fidèle dans toute ma maison\FTNT{Hé. 3:2.}.
\VS{8}Je parle avec lui bouche à bouche, et il me voit en effet, et non point en obscurité, ni dans aucune représentation de Yahweh. Pourquoi donc n'avez-vous pas craint de parler contre mon serviteur, contre Moïse~?
\FTNT{Ex. 33:11~; De. 34:10.} 
\VS{9}Ainsi la colère de Yahweh s'embrasa contre eux. Et il s'en alla.
\VS{10}Car la nuée se retira de dessus la tente. Et voici, Marie était frappée d'une lèpre blanche comme la neige~; et Aaron se tourna vers Marie et la vit lépreuse.
\VS{11}Alors Aaron dit à Moïse~: Hélas, de grâce, mon seigneur~! Je te prie ne mets point sur nous ce péché, car nous avons fait follement, et nous avons péché.
\VS{12}Je te prie qu'elle ne soit pas comme un enfant mort-né, dont la moitié de la chair est déjà consumée quand il sort du ventre de sa mère~!
\VS{13}Alors Moïse cria à Yahweh, en disant~: Ô Dieu, je te prie, guéris-la, je t'en prie.
\VS{14}Et Yahweh répondit à Moïse~: Si son père lui avait craché au visage, ne serait-elle pas dans l'ignominie pendant sept jours~? Qu'elle soit enfermée sept jours en dehors du camp, après quoi, elle y sera reçue\FTNT{Lé. 13:46.}.
\VS{15}Ainsi Marie fut enfermée hors du camp sept jours~; et le peuple ne partit pas de là jusqu'à ce que Marie fût rentrée.
\VS{16}Après cela le peuple partit de Hatséroth, et il campa dans le désert de Paran.
\Chap{13}
\TextTitle{Douze espions envoyés pour explorer Canaan}
\VerseOne{}Et Yahweh parla à Moïse, en disant~:
\VS{2}Envoie des hommes pour explorer le pays de Canaan, que je donne aux enfants d'Israël. Tu enverras un homme de chaque tribu de leurs pères, tous seront des principaux d'entre eux.
\VS{3}Moïse donc les envoya du désert de Paran, d'après l'ordre de Yahweh~; et tous ces hommes étaient chefs des enfants d'Israël.
\VS{4}Et ce sont ici leurs noms~: De la tribu de Ruben~: Schammua, fils de Zaccur~;
\VS{5}de la tribu de Siméon~: Schaphath, fils de Hori~;
\VS{6}de la tribu de Juda~: Caleb, fils de Jephunné~;
\VS{7}de la tribu d'Issacar~: Jigual, fils de Joseph~;
\VS{8}de la tribu d'Ephraïm~: Hosée, fils de Nun~;
\VS{9}de la tribu de Benjamin~: Palthi, fils de Raphu~;
\VS{10}de la tribu de Zabulon~: Gaddiel, fils de Sodi~;
\VS{11}de l'autre tribu de Joseph~: la tribu de Manassé, Gaddi, fils de Susi~;
\VS{12}de la tribu de Dan~: Ammiel, fils de Guemalli~;
\VS{13}de la tribu d'Aser~: Sethur, fils de Micaël~;
\VS{14}de la tribu de Nephthali~: Nachbi, fils de Vophsi~;
\VS{15}de la tribu de Gad~: Guéuel, fils de Maki.
\VS{16}Ce sont là les noms des hommes que Moïse envoya pour explorer le pays. Moïse donna à Hosée, fils de Nun, le nom de Josué\FTNT{Moïse changea le nom d'Hosée en y ajoutant le Nom de Yahweh. Hosée signifie «~sauveur~» et Josué (ou Jésus) «~Yahweh est salut~». Josué préfigurait Jésus-Christ qui nous a délivrés et transportés dans le Royaume des cieux (Col. 1:12-14). Moïse avait compris prophétiquement que seul Jésus peut nous faire rentrer dans notre héritage.}.
\VS{17}Moïse les envoya pour explorer le pays de Canaan, et il leur dit~: Montez de ce côté par le sud~; et vous monterez sur la montagne.
\VS{18}Et vous verrez quel est ce pays-là, et quel est le peuple qui l'habite, s'il est fort ou faible~; s'il est en petit ou en grand nombre.
\VS{19}Et quel est le pays où il habite, s'il est bon ou mauvais~; et quelles sont les villes dans lesquelles il habite, si c'est dans des camps, ou dans des villes fortifiées.
\VS{20}Et quelle est la terre, si elle est grasse ou maigre, s'il y a des arbres, ou non. Ayez bon courage, et prenez du fruit du pays. Or c'était alors le temps des premiers raisins.
\VS{21}Etant donc partis, ils examinèrent le pays, depuis le désert de Tsin jusqu'à Rehob, à l'entrée de Hamath.
\VS{22}Ils montèrent par le sud, et ils allèrent jusqu'à Hébron, où étaient Ahiman, Schéschaï, et Talmaï, enfants d'Anak. Hébron avait été bâtie sept ans avant Tsoan en Egypte.
\VS{23}Et ils vinrent jusqu'au torrent d'Eschcol, et coupèrent de là un sarment de vigne, avec une grappe de raisins~; ils étaient deux à le porter avec une perche. Ils apportèrent aussi des grenades et des figues.
\VS{24}Et on donna à ce lieu le nom de vallée d'Eschcol~; à cause de la grappe que les fils d'Israël y coupèrent.
\VS{25}Et au bout de quarante jours, ils furent de retour du pays qu'ils étaient allés explorer.
\TextTitle{Comptes rendus des envoyés}
\VS{26}Et à leur arrivée, ils se rendirent auprès de Moïse et d'Aaron, et de toute l'assemblée des enfants d'Israël, dans le désert de Padan à Kadès. Ils leur firent le rapport, ainsi qu'à toute l'assemblée, ils leur montrèrent les fruits du pays.
\VS{27}Ils firent donc leur rapport à Moïse, et lui dirent~: Nous avons été dans le pays où tu nous as envoyés. Véritablement, c'est un pays où coulent le lait et le miel, et en voici les fruits.
\VS{28}Seulement, le peuple qui habite ce pays est puissant, les villes sont fortifiées, très grandes~; nous y avons vu des enfants d'Anak\FTNT{De. 1:24-28.}.
\VS{29}Les Amalécites habitent la contrée du midi~; les Héthiens, les Jébusiens et les Amoréens habitent la montagne~; les Cananéens habitent le long de la mer, et vers le rivage du Jourdain.
\VS{30}Caleb fit taire le peuple devant Moïse, et il dit~: Montons, possédons ce pays, car nous y serons vainqueurs~!
\VS{31}Mais les hommes qui y étaient montés avec lui dirent~: Nous ne pouvons pas monter contre ce peuple-là, car il est plus fort que nous.
\VS{32}Et ils décrièrent devant les enfants d'Israël le pays qu'ils avaient exploré, en disant~: Le pays que nous avons parcouru pour l'explorer est un pays qui dévore ses habitants et tous ceux que nous y avons vus sont des gens de grande taille.
\VS{33}Et nous y avons vu aussi des géants, des enfants d'Anak, de la race des géants et nous étions à nos yeux et à leurs yeux comme des sauterelles.
\Chap{14}
\TextTitle{Rébellion et incrédulité d'Israël\FTNTT{1 Co. 10:1-5~; Hé. 3:7-19}}
\VerseOne{}Alors toute l'assemblée éleva la voix et se mit à pousser des cris, et le peuple pleura cette nuit-là.
\VS{2}Et tous les enfants d'Israël murmurèrent contre Moïse et Aaron, et toute l'assemblée leur dit~: Oh~! Si nous étions morts dans le pays d'Egypte~! Ou si nous étions morts dans ce désert\FTNT{De. 1:26-27.}~!
\VS{3}Et pourquoi Yahweh nous fait-il aller dans ce pays, où nous tomberons par l'épée, où nos femmes et nos petits enfants deviendront une proie~? Ne vaut-il pas mieux retourner en Egypte~?
\VS{4}Et ils se dirent l'un à l'autre~: Etablissons-nous un chef, et retournons en Egypte.
\VS{5}Alors Moïse et Aaron tombèrent sur leurs visages devant toute l'assemblée des enfants d'Israël.
\VS{6}Et Josué, fils de Nun, et Caleb, fils de Jephunné, qui étaient parmi ceux qui avaient exploré le pays, déchirèrent leurs vêtements,
\VS{7}et parlèrent à toute l'assemblée des enfants d'Israël, en disant~: Le pays que nous avons exploré est un très bon pays.
\VS{8}Si nous sommes agréables à Yahweh, il nous fera entrer dans ce pays, et il nous le donnera. C'est un pays où coulent le lait et le miel.
\VS{9}Seulement, ne soyez point rebelles contre Yahweh, et ne craignez point le peuple de ce pays-là, car ils seront notre pain, leur protection s'est retirée de dessus eux. Yahweh est avec nous, ne les craignez point\FTNT{De. 20:3-4.}~!
\VS{10}Alors toute l'assemblée parlait de les lapider~; mais la gloire de Yahweh apparut à tous les enfants d'Israël, devant la tente d'assignation.
\TextTitle{Moïse intercède pour le pardon d'Israël}
\VS{11}Et Yahweh dit à Moïse~: Jusqu'à quand ce peuple-ci m'irritera-t-il par mépris et jusqu'à quand ne croira-t-il point en moi, malgré tous les signes que j'ai faits au milieu de lui~?
\VS{12}Je le frapperai par la peste et je le détruirai, mais je ferai de toi une nation plus grande et plus puissante que lui.
\VS{13}Et Moïse dit à Yahweh~: Mais les Egyptiens l'entendront, car tu as fait monter par ta puissance ce peuple-ci du milieu d'eux\FTNT{Ex. 32:10-12.},
\VS{14}et ils diront avec les habitants de ce pays qui auront entendu que tu étais, ô Yahweh, au milieu de ce peuple, et que tu apparaissais, ô Yahweh à vue d'œil, que ta nuée s'arrêtait sur eux, et que tu marchais devant eux le jour dans la colonne de nuée, et la nuit dans la colonne de feu~;
\VS{15}si tu fais mourir ce peuple comme un seul homme, les nations qui ont entendu parler de toi diront~:
\VS{16}Yahweh n'avait pas le pouvoir de faire entrer ce peuple dans le pays qu'il avait juré de leur donner, il l'a égorgé dans le désert.
\VS{17}Maintenant, je te prie, que la puissance du Seigneur se montre dans sa grandeur, comme tu l'as déclaré en disant~:
\VS{18}Yahweh est lent à la colère et riche en bonté, il ôte l'iniquité et pardonne la rébellion, mais il ne tient point le coupable pour innocent, et il punit l'iniquité des pères sur les fils, jusqu'à la troisième et à la quatrième génération\FTNT{Ex. 20:5~; Ex. 34:6~; Ex. 34:7~; Ps. 86:15~; Ps. 103:8~; Ps. 145:8~; Jon. 4:2~; De. 5:9.}.
\VS{19}Pardonne, je te prie, l'iniquité de ce peuple, selon la grandeur de ta miséricorde, comme tu as pardonné à ce peuple depuis l'Egypte jusqu'ici.
\TextTitle{Réponse de Yahweh à Moïse}
\VS{20}Et Yahweh dit~: Je pardonne selon ta parole.
\VS{21}Mais certainement je suis vivant, et la gloire de Yahweh remplira toute la terre.
\VS{22}Car tous ceux qui ont vu ma gloire, et les prodiges que j'ai faits en Egypte et dans le désert, qui m'ont déjà tenté par dix fois, et qui n'ont point écouté ma voix,
\VS{23}tous ceux-là ne verront point le pays que j'ai juré à leurs pères de leur donner, tous ceux, dis-je, qui m'ont irrité par mépris, ne le verront pas\FTNT{De. 1:35-38.}.
\VS{24}Mais parce que mon serviteur Caleb a été animé d'un autre esprit, et qu'il a persévéré à me suivre, je le ferai entrer dans le pays où il a été, et ses descendants le posséderont en héritage.
\VS{25}Or les Amalécites et les Cananéens habitent la vallée. Demain, tournez-vous et partez pour le désert, dans la direction de la Mer Rouge.
\VS{26}Yahweh parla à Moïse et à Aaron, en disant~:
\VS{27}Jusqu'à quand laisserai-je cette méchante assemblée murmurer contre moi~? J'ai entendu les murmures des enfants d'Israël, qui murmuraient contre moi\FTNT{Ps. 106:25.}.
\VS{28}Dis-leur~: Je suis vivant, dit Yahweh, je vous ferai ainsi que vous avez parlé à mes oreilles.
\VS{29}Vos cadavres tomberont dans ce désert, et tous ceux d'entre vous qui ont été dénombrés, selon tout le compte que vous en avez fait, depuis l'âge de vingt ans, et au dessus, vous tous qui avez murmuré contre moi~;
\VS{30}vous n'entrerez pas dans le pays que j'avais juré de vous faire habiter, excepté Caleb, fils de Jephunné, et Josué, fils de Nun.
\VS{31}Et quant à vos petits enfants, dont vous avez dit~: Ils deviendront une proie~! Je les y ferai entrer, et ils connaîtront le pays que vous avez méprisé.
\VS{32}Mais quant à vous, vos cadavres tomberont dans ce désert~;
\VS{33}mais vos enfants paîtront dans ce désert quarante ans et ils porteront la peine de vos prostitutions, jusqu'à ce que vos cadavres soient tous consumés dans le désert.
\VS{34}Selon le nombre des jours que vous avez mis à reconnaître le pays, qui ont été quarante jours, un jour pour une année, vous porterez la peine de vos iniquités quarante ans, et vous connaîtrez ma rupture de promesse.
\VS{35}Je suis Yahweh, j'ai parlé~! C'est ainsi que je traiterai cette méchante assemblée, qui s'est assemblée contre moi~; ils seront consumés dans ce désert, et ils y mourront.
\VS{36}Les hommes donc que Moïse avait envoyés pour épier le pays, et qui étant de retour avaient fait murmurer contre lui toute l'assemblée, en diffamant le pays~;
\VS{37}ces hommes-là, qui avaient décrié le pays, moururent frappés d'une plaie devant Yahweh.
\VS{38}Mais Josué, fils de Nun, et Caleb, fils de Jephunné, restèrent seuls vivants parmi ceux qui étaient allés pour explorer le pays.
\TextTitle{Israël battu par les Amalécites et les Cananéens}
\VS{39}Or Moïse dit ces choses à tous les enfants d'Israël, et le peuple fut dans un grand deuil.
\VS{40}Puis ils se levèrent de bon matin et montèrent au sommet de la montagne, en disant~: Nous voici, et nous monterons au lieu dont Yahweh a parlé car nous avons péché.
\VS{41}Mais Moïse leur dit~: Pourquoi transgressez-vous le commandement de Yahweh~? Cela ne réussira point.
\VS{42}Ne montez pas~; car Yahweh n'est pas au milieu de vous~; afin que vous ne soyez pas battus devant vos ennemis\FTNT{De. 1:41-42.}.
\VS{43}Car les Amalécites et les Cananéens sont là devant vous, et vous tomberez par l'épée~; parce que vous vous êtes détournés de Yahweh, Yahweh ne sera point avec vous.
\VS{44}Toutefois ils s'obstinèrent à monter au sommet de la montagne~; mais l'arche de l'alliance de Yahweh et Moïse ne sortirent point du milieu du camp.
\VS{45}Alors les Amalécites et les Cananéens qui habitaient sur cette montagne descendirent, les battirent, et les taillèrent en pièces jusqu'à Horma.
\Chap{15}
\TextTitle{Consignes pour le pays de Canaan}
\VerseOne{}Puis Yahweh parla à Moïse, en disant~:
\VS{2}Parle aux enfants d'Israël, et dis-leur~: Quand vous serez entrés au pays que je vous donne, où vous devez demeurer,
\VS{3}et que vous voudrez faire un sacrifice consumé par le feu à Yahweh, un holocauste, ou un sacrifice en accompagnement d'un vœu, ou en offrande volontaire, ou bien dans vos fêtes, pour produire avec votre gros ou votre menu bétail une agréable odeur à Yahweh\FTNT{Ex. 29:18~; Lé. 22:21.},
\VS{4}celui qui offrira son offrande à Yahweh présentera en offrande un dixième de fleur de farine, pétrie dans un quart de hin d'huile\FTNT{Lé. 2:1-2.},
\VS{5}et un quart de hin de vin pour la libation que tu feras sur l'holocauste, ou sur un autre sacrifice pour chaque agneau.
\VS{6}Si c'est pour un bélier, tu feras en offrande deux dixièmes de fleur de farine, pétrie dans un tiers de hin d'huile,
\VS{7}et un tiers de hin de vin pour la libation, comme offrande d'une bonne odeur à Yahweh.
\VS{8}Et si tu sacrifies un veau, soit comme holocauste, soit comme sacrifice en accompagnement d'un vœu, ou comme sacrifice d'offrande de paix à Yahweh,
\VS{9}on présentera en offrande avec le veau trois dixièmes de fleur de farine, pétrie dans un demi-hin d'huile.
\VS{10}Et tu offriras la moitié d'un hin de vin pour la libation, en offrande consumée par le feu d'une bonne odeur à Yahweh.
\VS{11}On fera de même pour chaque bœuf, chaque bélier, et chaque petit des brebis ou des chèvres.
\VS{12}Selon le nombre que vous en sacrifierez, vous ferez ainsi à chacun, d'après leur nombre.
\VS{13}Tous ceux qui sont nés au pays feront ces choses de cette manière, en offrant un sacrifice consumé par le feu, d'une bonne odeur à Yahweh.
\TextTitle{Loi sur l'étranger vivant au milieu d'Israël}
\VS{14}Si un étranger séjournant chez vous, ou se trouvant au milieu de vous en vos générations, offre un sacrifice consumé par le feu d'une bonne odeur à Yahweh, il l'offrira de la même manière que vous.
\VS{15}Ô assemblée~! Il y aura une même ordonnance pour vous et pour l'étranger qui fait son séjour parmi vous, il y aura une même ordonnance perpétuelle en vos âges~; il en sera de l'étranger comme de vous en la présence de Yahweh.
\VS{16}Il y aura une même loi et une seule ordonnance pour vous et pour l'étranger qui séjourne au milieu de vous.
\TextTitle{Lois diverses}
\VS{17}Yahweh parla à Moïse, en disant~:
\VS{18}Parle aux enfants d'Israël, et dis-leur~: Quand vous serez arrivés dans le pays où je vous ferai entrer,
\VS{19}et que vous mangerez du pain de ce pays, vous en offrirez à Yahweh une offrande élevée.
\VS{20}Vous offrirez en offrande élevée un gâteau, les prémices de votre pâte~; vous l'offrirez comme ce qu'on prélève de l'aire.
\VS{21}Vous donnerez pour Yahweh une offrande des prémices de votre pâte, dans les temps à venir.
\VS{22}Et lorsque vous aurez péché involontairement\FTNT{Voir commentaire en Lé. 4:2.}, et que vous n'aurez pas fait tous ces commandements que Yahweh a fait connaître à Moïse,
\VS{23}tout ce que Yahweh vous a commandé par Moïse, depuis le jour où Yahweh a commencé de donner ses commandements, et dans la suite dans vos générations,
\VS{24}s'il arrive que la chose ait été faite involontairement, sans que l'assemblée s'en soit aperçue, toute l'assemblée sacrifiera un jeune taureau en holocauste d'une bonne odeur à Yahweh, avec l'offrande et la libation, d'après les règles établies~; elle offrira encore un jeune bouc en sacrifice pour l'expiation.
\VS{25}Ainsi le prêtre fera propitiation pour toute l'assemblée des enfants d'Israël, et il leur sera pardonné parce que c'est une chose arrivée involontairement, et ils ont apporté leur offrande, un sacrifice consumé par le feu à Yahweh et l'offrande pour l'expiation devant Yahweh, à cause de leur péché involontaire.
\VS{26}Alors il sera pardonné à toute l'assemblée des enfants d'Israël, et à l'étranger qui séjourne au milieu d'eux, car c'est involontairement que tout le peuple a péché.
\VS{27}Si c'est une seule personne qui a péché involontairement, elle offrira une chèvre d'un an en offrande pour le péché\FTNT{Lé. 4:27-28.}.
\VS{28}Et le prêtre fera propitiation pour la personne qui aura péché involontairement, de ce qu'elle aura péché involontairement devant Yahweh, et faisant propitiation pour elle, il lui sera pardonné.
\VS{29}Il y aura une même loi pour celui qui aura fait quelque chose involontairement, tant pour celui qui est né au pays des enfants d'Israël, que pour l'étranger qui fait son séjour parmi eux.
\VS{30}Mais quant à celui qui aura péché par fierté, tant celui qui est né au pays, que l'étranger, il a outragé Yahweh, cette personne-là sera retranchée du milieu de son peuple.
\VS{31}Parce qu'il a méprisé la parole de Yahweh, et qu'il a enfreint son commandement. Cette personne donc sera certainement retranchée~; son iniquité est sur elle.
\TextTitle{Un homme lapidé selon la loi\FTNTT{Ro. 3:19~; 7:7-11~; 2 Co. 3:7-9~; Ga. 3:10.}}
\VS{32}Or comme les enfants d'Israël étaient dans le désert, on trouva un homme qui ramassait du bois le jour du sabbat.
\VS{33}Et ceux qui l'avaient trouvé ramassant du bois, l'amenèrent à Moïse, à Aaron, et à toute l'assemblée.
\VS{34}Et on le mit sous garde, car ce qu'on devait lui faire n'avait pas été déclaré.
\VS{35}Alors Yahweh dit à Moïse~: On punira de mort cet homme, et toute l'assemblée le lapidera hors du camp.
\VS{36}Toute l'assemblée donc le mena hors du camp et le lapida, et il mourut, comme Yahweh l'avait ordonné à Moïse.
\VS{37}Et Yahweh parla à Moïse, en disant~:
\VS{38}Parle aux enfants d'Israël, et dis-leur~: Qu'ils se fassent de génération en génération des franges aux bords de leurs vêtements, et qu'ils mettent sur les franges au bords de leurs vêtements un cordon de couleur pourpre\FTNT{De. 22:12~; Mt. 23:5.}.
\VS{39}Quand vous aurez cette frange, vous la regarderez et vous vous souviendrez de tous les commandements de Yahweh, pour les mettre en pratique, et vous ne suivrez pas les désirs de vos cœurs et de vos yeux, pour vous laisser entraîner à la prostitution.
\VS{40}Afin que vous vous souveniez de tous mes commandements, et que vous les fassiez, et que vous soyez saints à votre Dieu.
\VS{41}Je suis Yahweh, votre Dieu, qui vous ai retiré du pays d'Egypte, pour être votre Dieu. Je suis Yahweh, votre Dieu.
\Chap{16}
\TextTitle{La révolte de Koré\FTNTT{Jud. 11.}}
\VerseOne{}Or Koré\FTNT{Koré, Dathan et Abiram, s'étaient révoltés contre Aaron et Moïse, car ils voulaient s'attribuer l'honneur d'offrir à Dieu des sacrifices. Ils voulaient exercer la prêtrise (sacerdoce) alors que Yahweh ne les avait pas établis pour le service du culte. Vouloir servir Dieu sans avoir reçu un appel divin est dangereux.}, fils de Jitsehar, fils de Kehath, fils de Lévi, se révolta avec Dathan et Abiram, fils d'Eliab, et On, fils de Péleth, tous trois fils de Ruben.
\VS{2}Et ils s'élevèrent contre Moïse, avec deux cent cinquante hommes des fils d'Israël, qui étaient des principaux de l'assemblée, de ceux que l'on convoquait pour tenir le conseil, et qui étaient des gens de renom.
\VS{3}Et ils s'assemblèrent contre Moïse et contre Aaron, et leur dirent~: C'en est assez~! Puisque tous ceux de l'assemblée sont saints, et que Yahweh est au milieu d'eux, pourquoi vous élevez-vous au-dessus de l'assemblée de Yahweh~?
\VS{4}Quand Moïse eut entendu cela, il se jeta sur son visage.
\VS{5}Et il parla à Koré et à tous ceux qui étaient assemblés avec lui, et leur dit~: Demain au matin, Yahweh fera connaître celui qui lui appartient, et celui qui est saint, et il le fera approcher de lui~; il fera, dis-je, approcher de lui celui qu'il aura choisi.
\VS{6}Faites ceci, prenez des encensoirs, Koré et toute son assemblée.
\VS{7}Et demain, mettez-y du feu, et mettez-y du parfum devant Yahweh~; et celui que Yahweh choisira, c'est celui-là qui sera saint. C'en est assez, fils de Lévi~!
\VS{8}Moïse dit aussi à Koré~: Ecoutez maintenant, fils de Lévi~:
\VS{9}Est-ce trop peu de chose pour vous, que le Dieu d'Israël vous ait séparés de l'assemblée d'Israël, pour vous faire approcher de lui, afin de faire le service du tabernacle de Yahweh, et pour vous tenir devant l'assemblée, afin de la servir~?
\VS{10}Et qu'il t'ait fait approcher de lui, toi et tous tes frères, les fils de Lévi, et vous recherchez encore la prêtrise~!
\VS{11}C'est pourquoi toi et toute ton assemblée, vous vous êtes rassemblés contre Yahweh~! Car qui est Aaron pour que vous murmuriez contre lui~?
\VS{12}Et Moïse envoya appeler Dathan et Abiram, fils d'Eliab, qui répondirent~: Nous n'y monterons point.
\VS{13}Est-ce peu de chose que tu nous aies fait monter hors d'un pays où coulent le lait et le miel, pour nous faire mourir dans le désert, que tu veuilles aussi dominer sur nous~?
\VS{14}Certes, tu ne nous as pas fait venir dans un pays où coulent le lait et le miel~! Et tu ne nous as pas donné un héritage de champs ni de vignes~! Veux-tu crever les yeux de ces gens~? Nous ne monterons pas.
\VS{15}Alors Moïse fut très irrité, et il dit à Yahweh~: N'aie point égard à leur offrande. Je n'ai point pris d'eux un seul âne, et je n'ai fait de mal à aucun d'eux.
\VS{16}Puis Moïse dit à Koré~: Toi et tous ceux qui sont assemblés avec toi, trouvez-vous demain devant Yahweh, toi et eux avec Aaron.
\VS{17}Et prenez chacun vos encensoirs, et mettez-y du parfum~; et que chacun présente devant Yahweh son encensoir~: Il y aura deux cent cinquante encensoirs~; toi et Aaron aussi, chacun avec son encensoir.
\VS{18}Ils prirent donc chacun son encensoir, et y mirent du feu, et ensuite y posèrent du parfum, et ils se tinrent à l'entrée de la tente d'assignation, avec Moïse et Aaron.
\VS{19}Et Koré fit assembler contre eux toute l'assemblée à l'entrée de la tente d'assignation~; et la gloire de Yahweh apparut à toute l'assemblée.
\VS{20}Puis Yahweh parla à Moïse et à Aaron, en disant~:
\VS{21}Séparez-vous du milieu de cette assemblée, et je les consumerai en un seul instant\FTNT{Ex. 32:10.}.
\VS{22}Mais ils tombèrent sur leur visage et dirent~: Ô Dieu~! Dieu des esprits de toute chair~! Un seul homme a péché, et tu te mettrais en colère contre toute l'assemblée\FTNT{Hé. 12:9.}~?
\VS{23}Et Yahweh parla à Moïse, en disant~:
\VS{24}Parle à l'assemblée, et dis lui~: Retirez-vous d'auprès de la demeure de Koré, de Dathan, et d'Abiram.
\VS{25}Moïse donc se leva, et alla vers Dathan et Abiram~; et les anciens d'Israël le suivirent.
\VS{26}Et il parla à l'assemblée, en disant~: Eloignez-vous, je vous prie, d'auprès des tentes de ces méchants hommes, et ne touchez à rien qui leur appartienne, de peur que vous ne périssiez punis pour tous leurs péchés.
\VS{27}Ils se retirèrent donc d'auprès des demeures de Koré, de Dathan et d'Abiram. Et Dathan et Abiram sortirent et se tinrent debout à l'entrée de leurs tentes, avec leurs femmes, leurs fils, et leurs petits-enfants.
\VS{28}Et Moïse dit~: Vous connaîtrez à ceci que Yahweh m'a envoyé pour faire toutes ces choses, et que je n'agis pas de moi-même.
\VS{29}Si ces gens meurent comme tous les hommes meurent, et s'ils subissent le sort commun à tous les hommes, Yahweh ne m'a point envoyé~;
\VS{30}mais si Yahweh fait une chose nouvelle, et si la terre ouvre sa bouche pour les engloutir avec tout ce qui leur appartient, et qu'ils descendent vivants dans le scheol, vous saurez alors que ces hommes-là ont irrité par mépris Yahweh.
\VS{31}Et il arriva qu'aussitôt qu'il eut achevé de dire toutes ces paroles, la terre qui était sous eux se fendit.
\VS{32}Et la terre ouvrit sa bouche et les engloutit, avec leurs tentes et tous les hommes qui étaient à Koré, et tous leurs biens\FTNT{De. 11:6~; Ps. 106:17.}.
\VS{33}Ils descendirent donc vivants dans le scheol, eux et tout ceux qui leur appartenait~; la terre les recouvrit, et ils disparurent au milieu de l'assemblée.
\VS{34}Et tout Israël qui était autour d'eux s'enfuit à leurs cris~; car ils disaient~: Prenons garde que la terre ne nous engloutisse~!
\VS{35}Un feu sortit de part Yahweh et consuma les deux cent cinquante hommes qui offraient le parfum.
\VS{36}Puis Yahweh parla à Moïse, en disant~:
\VS{37}Dis à Eléazar, fils d'Aaron, le prêtre, qu'il ramasse les encensoirs du milieu de l'embrasement, et d'en répandre au loin le feu, car ils sont sanctifiés.
\VS{38}Avec les encensoirs de ceux qui ont péché contre leurs âmes, que l'on fasse des lames étendues dont on couvrira l'autel. Puisqu'ils ont été offerts devant Yahweh et qu'ils sont sanctifiés, ils serviront de signe aux enfants d'Israël.
\VS{39}Ainsi Eléazar, le prêtre, prit les encensoirs d'airain, que ces hommes qui furent brûlés avaient présentés, et on en fit des lames pour couvrir l'autel.
\VS{40}C'est un souvenir pour les enfants d'Israël, afin qu'aucun étranger qui n'est pas de la race d'Aaron, ne s'approche pour offrir du parfum devant Yahweh, et ne soit comme Koré, et comme ceux qui ont été assemblés avec lui~; selon ce que Yahweh avait déclaré par Moïse.
\TextTitle{Le peuple frappé à cause des murmures}
\VS{41}Or dès le lendemain, toute l'assemblée des enfants d'Israël murmura contre Moïse et contre Aaron, en disant~: Vous avez fait mourir le peuple de Yahweh.
\VS{42}Et il arriva comme l'assemblée s'amassait contre Moïse et contre Aaron, et comme ils tournaient les regards vers la tente d'assignation, voici la nuée la couvrit, et la gloire de Yahweh apparut.
\VS{43}Moïse donc et Aaron vinrent donc devant la tente d'assignation.
\VS{44}Et Yahweh parla à Moïse, en disant~:
\VS{45}Retirez-vous du milieu de cette assemblée, et je les consumerai en un instant. Alors ils se prosternèrent le visage contre terre~;
\VS{46}puis Moïse dit à Aaron~: Prends l'encensoir, et mets-y du feu de dessus l'autel, mets-y aussi du parfum, et va promptement à l'assemblée, et fais propitiation pour eux~; car une grande colère est sortie de devant Yahweh, la plaie a commencé.
\VS{47}Et Aaron prit l'encensoir, comme Moïse lui avait dit, et il courut au milieu de l'assemblée, et voici la plaie avait déjà commencé sur le peuple. Alors il mit du parfum et fit propitiation pour le peuple.
\VS{48}Et comme il se tenait entre les morts et les vivants, la plaie fut arrêtée.
\VS{49}Et il y en eut quatorze mille sept cents qui moururent de cette plaie, outre ceux qui étaient morts à cause de Koré.
\VS{50}Et Aaron retourna auprès de Moïse, à l'entrée de la tente d'assignation, et la plaie s'arrêta.
\Chap{17}
\TextTitle{Yahweh confirme l'appel d'Aaron, sa verge fleurit}
\VerseOne{}Après cela Yahweh parla à Moïse, en disant~:
\VS{2}Parle aux enfants d'Israël, et prends une verge de chacun d'eux selon la maison de leur père, de tous ceux qui sont les princes, selon la maison de leurs pères, douze verges, puis tu écriras le nom de chacun sur sa verge,
\VS{3}mais tu écriras le nom d'Aaron sur la verge de Lévi\FTNT{La verge d'Aaron est une image du Messie ressuscité. Elle avait produit la vie tandis que celles des autres princes n'avaient produit aucun fruit. Cette histoire nous parle également de la confirmation de l'appel d'Aaron face aux critiques dont il était l'objet. On reconnaît l'arbre par ses fruits (Mt. 7:16-20~; Lu. 7:17-22).}~; car il y aura une verge pour chaque chef des maisons de leurs pères.
\VS{4}Et tu les déposeras dans la tente d'assignation, devant le témoignage, où je me rencontre avec vous.
\VS{5}Et la verge de l'homme que j'aurai choisi fleurira~; et je ferai cesser de devant moi les murmures des enfants d'Israël, par lesquels ils murmurent contre vous.
\VS{6}Quand Moïse parla aux enfants d'Israël, tous leurs princes lui donnèrent une verge, chaque prince une verge, selon les maisons de leurs pères, soit douze verges~; or la verge d'Aaron était au milieu des leurs.
\VS{7}Et Moïse mit les verges devant Yahweh, dans la tente du témoignage.
\VS{8}Et le lendemain, lorsque Moïse entra dans la tente du témoignage, voici, la verge d'Aaron, avait fleuri, pour la maison de Lévi, et elle avait poussé des boutons, produit des fleurs et mûri des amandes.
\VS{9}Alors Moïse ôta de devant Yahweh toutes les verges et les porta à tous les fils d'Israël, afin qu'ils les voient et qu'ils prennent chacun leurs verges.
\VS{10}Et Yahweh dit à Moïse~: Reporte la verge d'Aaron devant le témoignage, pour être conservée comme un signe pour les fils de rébellion, afin que tu fasses cesser de devant moi leurs murmures et qu'ils ne meurent point\FTNT{Hé. 9:3-5.}.
\VS{11}Et Moïse fit ainsi~; il se conforma à l'ordre que Yahweh lui avait donné.
\VS{12}Les enfants d'Israël parlèrent à Moïse, en disant~: Voici, nous expirons, nous périssons, nous périssons tous~!
\VS{13}Quiconque s'approche du tabernacle de Yahweh, meurt. Serons-nous tous entièrement expirés~?
\Chap{18}
\TextTitle{Droits et devoirs des prêtres et des Lévites}
\VerseOne{}Alors Yahweh dit à Aaron~: Toi et tes fils, et la maison de ton père avec toi, vous porterez l'iniquité du sanctuaire~; et toi, et tes fils avec toi, vous porterez l'iniquité de votre prêtrise.
\VS{2}Fais aussi approcher de toi tes frères, la tribu de Lévi, qui est la tribu de ton père, afin qu'ils te soient attachés et qu'ils te servent, mais toi et tes fils avec toi, vous servirez devant la tente du témoignage.
\VS{3}Ils garderont ce que tu leur ordonneras de garder, et ce qu'il faut garder de toute la tente, mais ils n'approcheront point des ustensiles du sanctuaire, ni de l'autel de peur qu'ils ne meurent, et que vous ne mouriez avec eux.
\VS{4}Ils te seront donc attachés, et ils garderont tout ce qu'il faut garder dans la tente d'assignation, selon tout le service du tabernacle et aucun étranger n'approchera de vous.
\VS{5}Mais vous prendrez garde à ce qu'il faut faire dans le sanctuaire, et à ce qu'il faut faire à l'autel, afin qu'il n'y ait plus d'indignation sur les enfants d'Israël.
\VS{6}Car quant à moi voici, j'ai pris vos frères, les Lévites, du milieu des enfants d'Israël, qui sont donnés en pur don pour Yahweh, afin qu'ils soient employés au service de la tente d'assignation.
\VS{7}Mais toi et tes fils avec toi, vous observerez la fonction de votre prêtrise en tout ce qui concerne l'autel et ce qui est au dedans du voile, et vous y ferez le service. J'établis votre prêtrise en office de pur don~; c'est pourquoi si un étranger en approche, on le fera mourir.
\VS{8}Yahweh dit encore à Aaron~: Voici, je t'ai donné la garde de mes offrandes élevées sur toutes les choses consacrées par les enfants d'Israël~; je te les ai données, et à tes enfants, par ordonnance perpétuelle, à cause de l'onction.
\VS{9}Ceci t'appartiendra d'entre les choses très saintes qui ne sont pas brûlées, savoir toutes leurs offrandes, soit de tous leurs gâteaux, soit de tous leurs sacrifices pour l'expiation, et tous leurs sacrifices pour la culpabilité qu'ils m'apporteront~; ce sont des choses très saintes pour toi et pour tes enfants.
\VS{10}Vous les mangerez dans un lieu très saint~; tout mâle en mangera~; vous les regarderez comme saintes\FTNT{Lé. 6:17-22~; Lé. 7:6~; Lé. 10:13.}.
\VS{11}Voici encore ce qui t'appartiendra~: Tous les dons que les enfants d'Israël présenteront par élévation et en les agitant de côté et d'autre, je te les donne à toi, à tes fils, et à tes filles avec toi, par une loi perpétuelle~; quiconque sera pur dans ta maison en mangera\FTNT{Lé. 7:34~; Lé. 10:14.}.
\VS{12}Je te donne aussi leurs prémices qu'ils offriront à Yahweh~: Tout ce qu'il y aura de meilleur en huile, et tout le meilleur du moût et du blé.
\VS{13}Les premiers fruits de toutes les choses que leur terre produira, et qu'ils apporteront à Yahweh t'appartiendront~; quiconque sera pur dans ta maison, en mangera.
\VS{14}Tout ce qui sera dévoué en Israël t'appartiendra\FTNT{Lé. 27:28~; Ez. 44:29.}.
\VS{15}Tout premier-né de toute chair, qu'ils offriront à Yahweh, tant des hommes que des animaux t'appartiendront. Mais, tu feras racheter le premier-né de l'homme, et tu feras racheter le premier-né d'un animal impur.
\VS{16}Et ceux qui doivent être rachetés, depuis l'âge d'un mois, tu les rachèteras selon ton estimation que tu en feras, au prix de cinq sicles d'argent, selon le sicle du sanctuaire, qui est de vingt guéras.
\VS{17}Mais tu ne feras point racheter le premier-né du bœuf, ni le premier-né de la brebis, ni le premier-né de la chèvre~: Ce sont des choses saintes. Tu répandras leur sang sur l'autel, et tu brûleras leur graisse~: Ce sera un sacrifice consumé par le feu d'une bonne odeur à Yahweh.
\VS{18}Mais leur chair t'appartiendra, comme la poitrine qu'on agite de côté et d'autre, et comme l'épaule droite.
\VS{19}Je t'ai donné, à toi et à tes fils, et à tes filles avec toi, par une loi perpétuelle, toutes les offrandes présentées par élévation des choses sanctifiées, que les enfants d'Israël offriront à Yahweh. C'est une alliance de sel\FTNT{Le sel est un aliment pratiquement impérissable et incorruptible. Dans l'Antiquité, il symbolisait l'incorruptibilité (Lé. 2:13).} et à perpétuité devant Yahweh, pour toi et pour ta postérité avec toi.
\VS{20}Puis Yahweh dit à Aaron~: Tu ne posséderas rien dans leur pays, et il n'y aura point de part pour toi au milieu d'eux~; c'est moi qui suis ta part et ta possession, au milieu des enfants d'Israël\FTNT{De. 10:9~; De. 18:2~; Ez. 44:28.}.
\TextTitle{Lois sur les dîmes (De. 14:22-29)}
\VS{21}Et je donne comme possession aux fils de Lévi, toutes les dîmes\FTNT{Il y avait plusieurs sortes de dîmes dans la loi Mosaïque~:
\\- La 1ère dîme~: Le peuple devait payer une dîme générale au bénéfice des Lévites (No. 18:21).
\\Toutes les tribus d'Israël, à l'exception des Lévites, eurent une possession géographique qu'ils reçurent comme héritage après leur entrée en Canaan. Mais les Lévites devaient accomplir une tâche particulière au sein de la nation. Ils devaient s'occuper du service dans la tente d'assignation. En compensation de ce service, ils devaient percevoir un impôt de 10\% des revenus de tous les Israélites.
\\- La 2ème dîme~: Les Lévites devaient payer la «~dîme de la dîme~», au bénéfice des prêtres (No. 18:25-31).
\\Tous les prêtres étaient des Lévites, mais tous les Lévites n'étaient pas des prêtres. Les prêtres descendaient d'Aaron et ils exerçaient des responsabilités particulières dans la tente d'assignation, puis dans le temple. Cette seconde dîme permettait aux prêtres d'être nourris et assurait donc le bon fonctionnement du service du temple.
\\- La 3ème dîme~: Tous les Israélites devaient conserver une dîme de toute leur production en prévision de leurs pèlerinages annuels à Jérusalem (De. 14:22-26).
\\Trois fois par an, tout le peuple devait s'assembler à Jérusalem, l'endroit choisi par le Seigneur, à l'occasion des principales fêtes. Dieu avait prévu que chacun puisse disposer de ressources suffisantes pour leur permettre de se réjouir pleinement à ces occasions. C'est pour cela qu'ils devaient mettre de côté 10\% de leurs productions agricoles annuelles. Il est intéressant de noter que la dîme n'était jamais payée en argent, mais toujours en nature.
\\- La 4ème dîme~: Il fallait payer une dîme spéciale à l'intention des pauvres, des orphelins et des veuves (De. 14:28-29). 
\\Certains affirment que la dîme existait bien avant la loi. Mais ils ignorent que la Bible parle de plusieurs sortes de lois.
\\- Les lois cérémonielles (Hé. 9:1)
\\Ces lois étaient relatives au culte et concernaient le tabernacle puis le temple, les sacrifices, les ablutions (Lé. 16~; Hé. 9:1-10). Les dîmes (la dîme des prêtres) devaient être amenées dans le temple (Mal. 3:10), elles faisaient donc partie des lois cérémonielles. Or les Lévites et les prêtres de la Première Alliance n'existent plus sous la Nouvelle Alliance car les enfants de Dieu sont un royaume de rois et de prêtres (Ap. 1:6~; Ap. 5:10).
\\- Les lois morales (Ex. 20:1-17). Dieu est saint et il veut un peuple saint qui marche dans sa crainte, dans la sainteté et dans l'obéissance. Lé. 18 nous parle des lois morales~; elles n'ont pas été abolies, elles existent toujours. Elles sont inscrites dans la conscience de l'homme, elles sont gravées dans notre cœur (Hé. 8:10).
\\- Les lois sociales (Ex. 21:1-24). Ce sont des lois civiles régissant la vie sociale d'Israël, comme nous pouvons le lire dans Ex. 21 par exemple. Ces lois n'ont rien à voir avec les croyants de la Nouvelle Alliance. Les lois morales témoignent de la nature de Dieu, ce sont des lois éternelles qui existaient bien avant Abraham. Les lois cérémonielles ont commencé dès la fondation du monde (Ap. 13:8) car l'Agneau de Dieu était immolé avant la fondation du monde (1 Pi. 1:19-20). Seules les lois sociales ont débuté avec Moïse car elles concernaient exclusivement les Israélites. Ces trois sortes de lois ont été institutionnalisées par Moïse, mais les deux premières (morales et cérémonielles) existaient avant ce dernier. Les quatre sortes de dîmes faisaient bel et bien partie des lois sociales et cérémonielles. Or ces lois ne sont plus d'actualité sous la Nouvelle Alliance. En conclusion, nous pouvons dire que Jésus nous a rachetés en accomplissant les lois cérémonielles afin que nous pratiquions les lois morales (Ep. 2:10). Voir également commentaire en Mal 3~: 10.} d'Israël, pour le service auquel ils sont employés, le service de la tente d'assignation.
\VS{22}Et les enfants d'Israël n'approcheront plus de la tente d'assignation, afin qu'ils ne se chargent d'un péché et qu'ils ne meurent point.
\VS{23}Mais les Lévites s'emploieront au service de la tente d'assignation, et ils resteront chargés de leurs iniquités. Cette loi sera perpétuelle parmi vos descendants, et ils ne posséderont point d'héritage parmi les enfants d'Israël.
\VS{24}Car je donne comme possession aux Lévites les dîmes que les enfants d'Israël présenteront à Yahweh en offrande élevée~; c'est pourquoi je dis d'eux qu'ils n'auront point d'héritage parmi les fils d'Israël.
\VS{25}Puis Yahweh parla à Moïse, en disant~:
\VS{26}Tu parleras aussi aux Lévites, et tu leur diras~: Quand vous recevrez des enfants d'Israël les dîmes que je vous donne de leur part comme possession, vous en offrirez l'offrande élevée à Yahweh, la dîme de la dîme~;
\VS{27}et votre offrande élevée vous sera comptée comme le blé qu'on prélève de l'aire, et comme l'abondance qu'on prélève de la cuve.
\VS{28}C'est ainsi que vous prélèverez une offrande pour Yahweh de toutes les dîmes que vous recevrez des enfants d'Israël, et vous donnerez au prêtre Aaron l'offrande que vous en aurez prélevée pour Yahweh.
\VS{29}Sur tous les dons qui vous seront faits, vous prélèverez toute l'offrande élevée pour Yahweh~; sur tout ce qu'il y aura de meilleur, vous prélèverez la portion consacrée.
\VS{30}Et tu leur diras~: Quand vous aurez offert en offrande élevée le meilleur de la dîme, pris de la dîme même, il sera imputé aux Lévites comme le revenu de l'aire, et comme le revenu de la cuve.
\VS{31}Et vous la mangerez en tout lieu, vous et votre maison~; car c'est votre salaire pour le service auquel vous êtes employés dans la tente d'assignation.
\VS{32}Vous ne serez point coupables de péché au sujet de la dîme, quand vous en aurez offert en offrande élevée sur ce qu'il y aura de meilleur et vous ne souillerez point les choses saintes des enfants d'Israël et vous ne mourrez point.
\Chap{19}
\TextTitle{La jeune vache rousse~; l'eau de purification}
\VerseOne{}Yahweh parla à Moïse et à Aaron, en disant~:
\VS{2}Voici ce qui est ordonné par la loi que Yahweh a commandé, en disant~: Parle aux enfants d'Israël, et dis-leur qu'ils t'amènent une jeune vache rousse, entière, sans défaut, et qui n'ait point porté le joug.
\VS{3}Puis vous la donnerez à Eléazar, le prêtre, qui la mènera hors du camp, et on l'égorgera en sa présence\FTNT{Lé. 4:12~; Hé. 13:11-12.}.
\VS{4}Ensuite, Eléazar, le prêtre, prendra de son sang avec son doigt, et fera sept fois l'aspersion du sang vers le devant de la tente d'assignation.
\VS{5}Et on brûlera la jeune vache en sa présence~; on brûlera sa peau, sa chair, son sang et ses excréments\FTNT{Ex. 29:14.}.
\VS{6}Le prêtre prendra du bois de cèdre, de l'hysope, et du cramoisi, et les jettera dans le feu où sera brûlée la jeune vache.
\VS{7}Puis le prêtre lavera ses vêtements et son corps avec de l'eau~; après cela, il rentrera au camp, et le prêtre sera impur jusqu'au soir.
\VS{8}Celui qui l'aura brûlé, lavera ses vêtements dans l'eau, il lavera aussi dans l'eau son corps~; et il sera impur jusqu'au soir.
\VS{9}Et un homme pur ramassera les cendres de la jeune vache, et les mettra hors du camp, dans un lieu pur~; elles seront gardées pour l'assemblée des enfants d'Israël~; afin d'en faire l'eau de purification. C'est une purification pour le péché.
\VS{10}Celui qui aura ramassé les cendres de la jeune vache, lavera ses vêtements, et sera impur jusqu'au soir~; ce sera une loi perpétuelle pour les enfants d'Israël, et pour l'étranger en séjour au milieu d'eux.
\VS{11}Celui qui touchera un mort, un corps humain quel qu'il soit, sera impur pendant sept jours\FTNT{Ag. 2:13.}.
\VS{12}Il se purifiera avec cette eau le troisième jour et le septième jour, et il sera pur~; mais s'il ne se purifie pas le troisième jour, il ne sera pas pur le septième jour.
\VS{13}Alors celui qui touchera un mort, le corps d'un homme qui sera mort et qui ne se purifiera pas, souille le tabernacle de Yahweh~; celui-là sera retranché d'Israël. Il est impur, car l'eau de purification n'a pas été répandue sur lui, son impureté demeure encore sur lui.
\VS{14}Voici la loi. Lorsqu'un homme mourra dans une tente, quiconque entrera dans la tente, et quiconque se trouvera dans la tente sera impur pendant sept jours.
\VS{15}Aussi tout vase découvert, sur lequel il n'y aura point de couvercle attaché, sera impur.
\VS{16}Et quiconque touchera, dans les champs, un homme qui aura été tué par l'épée, ou un mort, ou des ossements humains, ou un sépulcre, sera impur durant sept jours.
\VS{17}Et on prendra, pour celui qui est impur, de la poudre de la jeune vache brûlée pour faire la purification, et on la mettra dans un vase, avec de l'eau vive par-dessus.
\VS{18}Puis un homme pur prendra de l'hysope, et la trempera dans l'eau~; il en fera aspersion sur la tente, et sur tous les ustensiles, et sur toutes les personnes qui auront été là, et sur celui qui a touché des ossements ou un homme tué, ou un mort, ou un sépulcre.
\VS{19}Celui qui est pur fera l'aspersion sur celui qui est impur, le troisième jour et le septième jour, et il le purifiera le septième jour~; puis il lavera ses vêtements, et se lavera dans l'eau, et il sera pur le soir.
\VS{20}Mais l'homme qui sera impur, et qui ne se purifiera point, sera retranché du milieu de l'assemblée, parce qu'il a souillé le sanctuaire de Yahweh~; comme l'eau de purification n'a pas été répandue sur lui, il est impur.
\VS{21}Et ce sera pour eux une loi perpétuelle, et celui qui fera l'aspersion de l'eau de purification lavera ses vêtements~; et quiconque touchera l'eau de purification sera impur jusqu'au soir.
\VS{22}Et tout ce que l'homme impur touchera sera souillé, et la personne qui le touchera sera impure jusqu'au soir.
\Chap{20}
\TextTitle{Mort de Marie}
\VerseOne{}Or toute l'assemblée des enfants d'Israël arriva dans le désert de Tsin au premier mois, et le peuple s'arrêta à Kadès. Marie mourut là, et y fut ensevelie.
\TextTitle{Murmures du peuple à cause du manque d'eau\FTNTT{De. 32:51~; cp. Ex. 17:1-7.}}
\VS{2}Et il n'y avait point d'eau pour l'assemblée~; et ils se soulevèrent contre Moïse et contre Aaron.
\VS{3}Et le peuple contesta contre Moïse et ils lui dirent~: Pourquoi ne sommes-nous pas morts quand nos frères moururent devant Yahweh~?
\VS{4}Et pourquoi avez-vous fait venir l'assemblée de Yahweh dans ce désert, pour que nous y mourions, nous et notre bétail\FTNT{Ex. 17:3.}~?
\VS{5}Et pourquoi nous avez-vous fait monter hors d'Egypte, pour nous amener dans ce méchant lieu qui n'est pas un lieu où l'on puisse semer, ni un lieu pour des figuiers, ni pour des vignes, ni pour des grenadiers, et sans eau pour boire~?
\VS{6}Alors Moïse et Aaron se retirèrent de devant l'assemblée à l'entrée de la tente d'assignation et ils tombèrent sur leurs faces~; et la gloire de Yahweh apparut.
\TextTitle{Incrédulité de Moïse et d'Aaron à Meriba}
\VS{7}Yahweh parla à Moïse, en disant~:
\VS{8}Prends la verge, et convoque l'assemblée, toi et Aaron, ton frère. Vous parlerez en leur présence au rocher\FTNT{Christ, le rocher des âges ( Es. 8:13-17~; 1 Co. 10:1-4).}, et il donnera son eau~; ainsi tu leur feras sortir de l'eau du rocher, et tu donneras à boire à l'assemblée et à leur bétail.
\VS{9}Moïse prit la verge qui était devant Yahweh, comme il lui avait ordonné.
\VS{10}Moïse et Aaron convoquèrent l'assemblée devant le rocher. Et il leur dit~: Ecoutez donc, rebelles~! Est-ce de ce rocher que nous vous ferons sortir de l'eau~?
\VS{11}Puis Moïse leva sa main, et frappa deux fois le rocher avec sa verge et il en sortit des eaux en abondance. L'assemblée but, et leur bétail aussi.
\VS{12}Alors Yahweh dit à Moïse et à Aaron~: Parce que vous n'avez pas cru en moi, pour me sanctifier aux yeux des enfants d'Israël, ainsi vous ne ferez point entrer cette assemblée dans le pays que je lui donne.
\VS{13}Ce sont là, les eaux de Meriba, où les enfants d'Israël contestèrent avec Yahweh, qui fut sanctifié en eux.
\TextTitle{La méchanceté d'Edom\FTNTT{Ge. 25:30~; Ab. 10.}}
\VS{14}Puis Moïse envoya des ambassadeurs de Kadès au roi d'Edom, pour lui dire~: Ainsi parle ton frère Israël~: Tu sais toutes les souffrances que nous avons eu.
\VS{15}Comment nos pères descendirent en Egypte, où nous avons demeuré longtemps~; et comment les Egyptiens nous ont maltraités, nous et nos pères.
\VS{16}Et nous avons crié à Yahweh, et il a entendu nos cris. Il a envoyé l'Ange et nous a retirés d'Egypte. Et voici, nous sommes à Kadès, ville qui est à l'extrémité de ton territoire\FTNT{Ex. 2:23~; Ex. 23:20~; Ac. 7:30-38.}.
\VS{17}Je te prie, laisse-nous passer par ton pays~; nous ne traverserons ni les champs ni les vignes, et nous ne boirons l'eau d'aucun puits~; nous marcherons par le chemin royal~; nous ne nous détournerons ni à droite ni à gauche, jusqu'à ce que nous ayons passé ton territoire.
\VS{18}Et Edom lui dit~: Tu ne passeras point par mon pays, de peur que je ne sorte en armes à ta rencontre.
\VS{19}Les enfants d'Israël lui répondirent~: Nous monterons par le grand chemin, et si nous buvons de tes eaux, moi et mes bêtes, je t'en payerai le prix~; je veux seulement passer à pied.
\VS{20}Mais il lui répondit~: Tu ne passeras pas~! Et sur cela, Edom sortit à sa rencontre avec une grande multitude, et à main armée.
\VS{21}Ainsi Edom ne voulut point permettre à Israël de passer par ses frontières~; c'est pourquoi Israël se détourna de lui.
\VS{22}Et toute l'assemblée des enfants d'Israël partit de Kadès et arriva à la montagne de Hor.
\TextTitle{Mort d'Aaron}
\VS{23}Et Yahweh parla à Moïse et à Aaron à la montagne de Hor, près des frontières du pays d'Edom, en disant~:
\VS{24}Aaron sera recueilli auprès de son peuple, car il n'entrera pas dans le pays que je donne aux enfants d'Israël, parce que vous avez été rebelles à mon commandement aux eaux de la dispute\FTNT{«~Meriba~»}.
\VS{25}Prends donc Aaron et Eléazar, son fils, et fais-les monter sur la montagne de Hor.
\VS{26}Puis fais dépouiller Aaron de ses vêtements, et fais-les revêtir à Eléazar, son fils. C'est là qu'Aaron sera recueilli et qu'il mourra.
\VS{27}Moïse fit ce que Yahweh avait ordonné~; et ils montèrent sur la montagne de Hor, aux yeux de toute l'assemblée.
\VS{28}Et Moïse dépouilla Aaron de ses vêtements et en fit revêtir Eléazar, son fils. Aaron mourut là, au sommet de la montagne. Moïse et Eléazar descendirent de la montagne\FTNT{De. 10:6.}.
\VS{29}Toute l'assemblée, toute la maison d'Israël, voyant qu'Aaron était mort, le pleurèrent trente jours.
\Chap{21}
\TextTitle{Les Cananéens livrés à Israël}
\VerseOne{}Quand le roi d'Arad, Cananéen, qui habitait le midi, eut appris qu'Israël venait par le chemin d'Atharim, il combattit Israël et emmena des prisonniers.
\VS{2}Alors Israël fit un vœu à Yahweh, en disant~: Si tu livres ce peuple entre mes mains, je dévouerai ses villes par le moyen de l'interdit.
\VS{3}Et Yahweh exauça la voix d'Israël et livra entre ses mains les Cananéens. On les dévoua par interdit, avec leurs villes~; et on donna à ce lieu le nom de Horma.
\TextTitle{Le serpent d'airain\FTNTT{Jn. 3:14-15~; 2 Co. 5:20.}}
\VS{4}Puis ils partirent de la montagne de Hor, par le chemin de la Mer Rouge, pour faire le tour du pays d'Edom. Le cœur du peuple s'impatienta en route,
\VS{5}et parla contre Dieu, et contre Moïse, en disant~: Pourquoi nous as-tu fait monter hors d'Egypte, pour mourir dans ce désert~? Car il n'y a point de pain ni d'eau, et notre âme est dégoûtée de cette nourriture misérable.
\VS{6}Et Yahweh envoya contre le peuple des serpents brûlants qui mordaient le peuple~; tellement qu'il en mourut un grand nombre en Israël\FTNT{1 Co. 10:9.}.
\VS{7}Alors le peuple vint vers Moïse, et dit~: Nous avons péché, car nous avons parlé contre Yahweh et contre toi. Invoque Yahweh afin qu'il éloigne de nous les serpents et Moïse pria pour le peuple.
\VS{8}Et Yahweh dit à Moïse~: Fais-toi un serpent brûlant, et mets-le sur une perche~; quiconque aura été mordu et le regardera conservera la vie.
\VS{9}Moïse fit un serpent d'airain\FTNT{Voir Jn. 3:14-16. Ceux qui regardent à Jésus-Christ, et non aux hommes, obtiennent la délivrance. L'airain nous parle du jugement (Job 20:24), le serpent de la malédiction (Ge. 3:14), et la perche parle de la croix (1 Co. 1:18). Jésus a pris nos malédictions sur la croix de Golgotha (Ga. 3:13).}, et le mit sur une perche~; quiconque avait été mordu par un serpent et regardait le serpent d'airain conservait la vie.
\VS{10}Les enfants d'Israël partirent et campèrent à Oboth.
\VS{11}Et ils partirent d'Oboth et ils campèrent en Ijjé-Abarim, dans le désert qui est vis-à-vis de Moab, vers le soleil levant.
\VS{12}Puis ils partirent de là et campèrent vers le torrent de Zéred.
\VS{13}Et ils partirent de là et campèrent de l'autre côté de l'Arnon, qui est dans le désert, en sortant du territoire des Amoréens~; car l'Arnon est la frontière de Moab, entre les Moabites et les Amoréens\FTNT{Jg. 11:18.}.
\VS{14}C'est pourquoi il est dit dans le livre des batailles de Yahweh~: Vaheb en Supha, et les torrents de l'Arnon,
\VS{15}et le cours des torrents qui s'étend du côté d'Ar et touche à la frontière de Moab.
\VS{16}De là ils allèrent à Beer. C'est là le puit où Yahweh dit à Moïse~: Rassemble le peuple, et je leur donnerai de l'eau.
\VS{17}Alors Israël chanta ce cantique~: Monte, puits~! Chantez-lui en vous répondant les uns aux autres.
\VS{18}Puits que des princes ont creusés. Que les grands du peuple ont creusé, avec le législateur, avec leurs bâtons~! Du désert ils vinrent à Matthana~;
\VS{19}de Matthana à Nahaliel~; et de Nahaliel à Bamoth~;
\VS{20}de Bamoth à la vallée qui est dans le territoire de Moab, au sommet de Pisga, et qui regarde vers Jeshimon.
\TextTitle{Israël bat le roi des Amoréens et le roi de Basan}
\VS{21}Puis Israël envoya des messagers à Sihon, roi des Amoréens, pour lui dire~:
\VS{22}Laisse-moi passer par ton pays~; nous ne nous détournerons ni dans les champs, ni dans les vignes, et nous ne boirons l'eau d'aucun des puits~; mais nous marcherons par la route royale, jusqu'à ce que nous ayons passé ton territoire.
\VS{23}Mais Sihon ne permit pas à Israël de passer sur son territoire~; il rassembla tout son peuple et sortit à la rencontre d'Israël, dans le désert~; il vint à Jahats, et combattit Israël\FTNT{De. 2:26-30~; Jg. 11:29-30.}.
\VS{24}Israël le fit passer au fil de l'épée et conquit son pays, depuis l'Arnon jusqu'à Jabbok, et jusqu'à la frontière des fils d'Ammon~; car la frontière des fils d'Ammon était forte\FTNT{De. 2:30~; De. 29:7~; Ps. 135:11-12.}.
\VS{25}Et Israël prit toutes les villes qui étaient là, et habitat dans toutes les villes des Amoréens, à Hesbon, et dans toutes les villes de son ressort.
\VS{26}Or Hesbon était la ville de Sihon, roi des Amoréens, qui avait le premier fait la guerre au roi de Moab, et pris sur lui tout son pays jusqu'à l'Arnon.
\VS{27}C'est pourquoi les poètes disent~: Venez à Hesbon~! Que la ville de Sihon soit rebâtie et fortifiée~!
\VS{28}Car le feu est sorti de Hesbon, et la flamme de la cité de Sihon~; elle a consumé Ar-Moab, les habitants des hauteurs de l'Arnon.
\VS{29}Malheur à toi, Moab~! Peuple de Kemosch, tu es perdu~! Il a livré ses fils qui se sauvaient et ses filles en captivité à Sihon, roi des Amoréens\FTNT{Jé. 48:46.}.
\VS{30}Nous les avons défaits à coups de flèches~: De Hesbon à Dibon tout est détruit~; nous les avons mis en déroute jusqu'à Nophach, jusqu'à Médeba.
\VS{31}Israël s'établit dans le pays des Amoréens.
\VS{32}Puis Moïse envoya des gens pour reconnaître Jaezer, ils prirent les villes de son ressort, et chassèrent les Amoréens qui y étaient.
\VS{33}Ensuite, ils se tournèrent et montèrent par le chemin de Basan. Og, roi de Basan, sortit à leur rencontre, avec tout son peuple pour les combattre à Edréï.
\VS{34}Et Yahweh dit à Moïse~: Ne le crains point, car je le livre entre tes mains, lui et tout son peuple, et son pays~; tu le traiteras comme tu as traité Sihon, roi des Amoréens, qui habitait à Hesbon\FTNT{De. 3:1-2.}.
\VS{35}Ils le battirent donc, lui et ses fils, et tout son peuple, sans en laisser échapper un seul, et ils s'emparèrent de son pays.
\Chap{22}
\TextTitle{Balak cherche à maudire Israël~; Balaam\FTNTT{2 Pi. 2:15~; Jud. 11~; Ap. 2:14} séduit par les honneurs}
\VerseOne{}Puis les enfants d'Israël partirent, et ils campèrent dans les plaines de Moab, au-delà du Jourdain, vis-à-vis de Jéricho.
\VS{2}Balak, fils de Tsippor, vit tout ce qu'Israël avait fait aux Amoréens.
\VS{3}Et Moab eut une grande frayeur du peuple, parce qu'il était en grand nombre, il fut saisi de terreur en face des enfants d'Israël.
\VS{4}Et Moab dit aux anciens de Madian~: Maintenant cette multitude va brouter tout ce qui nous entoure, comme le bœuf broute l'herbe des champs. Balak, fils de Tsippor, était alors roi de Moab.
\VS{5}Il envoya des messagers auprès de Balaam, fils de Beor, à Pethor, située sur le fleuve, dans le pays des fils de son peuple, afin de l'appeler et de lui dire~: Voici, un peuple est sorti d'Egypte, il couvre la surface de la terre, et il habite vis-à-vis de moi.
\VS{6}Viens donc maintenant, je te prie, maudis-moi ce peuple, car il est plus puissant que moi~; peut-être que je serai le plus fort, et que nous le battrons, et que je le chasserai du pays~; car je sais que celui que tu bénis est béni, et que celui que tu maudis est maudit.
\VS{7}Les anciens de Moab s'en allèrent avec les anciens de Madian, ayant dans leurs mains de quoi payer le devin. Ils arrivèrent auprès de Balaam, et lui rapportèrent les paroles de Balak.
\VS{8}Il leur répondit~: Demeurez ici cette nuit, et je vous répondrai d'après ce que Yahweh me dira. Et les chefs des Moabites restèrent chez Balaam.
\VS{9}Et Dieu vint à Balaam et dit~: Qui sont ces hommes que tu as chez toi~?
\VS{10}Et Balaam répondit à Dieu~: Balak, fils de Tsippor, roi de Moab, les a envoyés pour me dire~:
\VS{11}Voici, un peuple qui est sorti d'Egypte, et qui couvre la face de la terre~; viens donc, maudis-le-moi~; peut-être qu'ainsi je pourrai le combattre, et je le chasserai.
\VS{12}Et Dieu dit à Balaam~: Tu n'iras point avec eux, et tu ne maudiras point ce peuple, car il est béni.
\VS{13}Et Balaam se leva le matin, et il dit aux chefs qui avaient été envoyés par Balak~: Retournez dans votre pays, car Yahweh refuse de me laisser venir avec vous.
\VS{14}Ainsi les chefs des Moabites se levèrent et retournèrent auprès de Balak, et dirent~: Balaam a refusé de venir avec nous.
\VS{15}Et Balak envoya encore des chefs en plus grand nombre, et plus considérés que les premiers.
\VS{16}Ils arrivèrent auprès de Balaam, et lui dirent~: Ainsi parle Balak, fils de Tsippor~: Que l'on ne t'empêche donc pas de venir vers moi~;
\VS{17}car je te rendrai beaucoup d'honneur, et je ferai tout ce que tu me diras~; je te prie donc viens, maudis-moi ce peuple.
\VS{18}Et Balaam répondit et dit aux serviteurs de Balak~: Quand Balak me donnerait sa maison pleine d'or et d'argent, je ne pourrais point transgresser l'ordre de Yahweh, mon Dieu~; je ne pourrais faire aucune chose, ni petite ni grande.
\VS{19}Toutefois, je vous prie, demeurez maintenant ici encore cette nuit, et je saurai ce que Yahweh aura de plus à me dire.
\VS{20}Dieu vint, la nuit à Balaam, et lui dit~: Puisque ces hommes sont venus t'appeler, lève-toi, et va avec eux~; mais quoi qu'il en soit, tu feras ce que je te dirai.
\VS{21}Ainsi Balaam se leva le matin, et sella son ânesse, et partit avec les chefs de Moab.
\VS{22}Mais la colère de Dieu s'enflamma parce qu'il était parti~; et l'Ange de Yahweh se plaça sur le chemin pour lui résister. Balaam était monté sur son ânesse, et ses deux serviteurs étaient avec lui.
\VS{23}L'ânesse vit l'Ange de Yahweh qui se tenait sur le chemin, son épée nue dans la main~; elle se détourna du chemin et alla dans les champs. Balaam frappa l'ânesse pour la ramener dans le chemin\FTNT{2 Pi. 2:16~; Jud. 1:11.}.
\VS{24}L'Ange de Yahweh se plaça dans un sentier entre les vignes~; il y avait un mur de chaque côté.
\VS{25}L'ânesse vit l'Ange de Yahweh~; elle se serra contre le mur, et elle serra le pied de Balaam contre le mur. Balaam la frappa de nouveau.
\VS{26}Et l'Ange de Yahweh passa plus loin et s'arrêta dans un lieu étroit où il n'y avait point d'espace pour se détourner à droite ou à gauche.
\VS{27}Et l'ânesse vit l'Ange de Yahweh, et elle s'abattit sous Balaam. Balaam se mit en grande colère, et il frappa l'ânesse avec son bâton.
\VS{28}Alors Yahweh fit parler l'ânesse, et elle dit à Balaam~: Que t'ai-je fait, pour que tu m'aies déjà frappée trois fois~?
\VS{29}Et Balaam répondit à l'ânesse~: C'est parce que tu t'es moquée de moi~; si j'avais une épée dans la main, je te tuerai sur le champ !
\VS{30}Et l'ânesse dit à Balaam: Ne suis-je pas ton ânesse, sur laquelle tu montes depuis que je suis à toi, jusqu'à aujourd'hui~? Ai-je l'habitude de te faire ainsi~? Et il répondit~: Non.
\VS{31}Alors Yahweh ouvrit les yeux de Balaam, et il vit l'Ange de Yahweh qui se tenait sur le chemin, et qui avait dans sa main son épée nue~; et il s'inclina et se prosterna sur son visage.
\VS{32}Et l'Ange de Yahweh lui dit~: Pourquoi as-tu frappé ton ânesse déjà trois fois~? Voici je suis sorti pour m'opposer à toi~; car ta voie est devant moi une voie de perdition.
\VS{33}Mais l'ânesse m'a vu et elle s'est détournée de devant moi déjà trois fois~; autrement, si elle ne s'était détournée de moi, je t'aurais même déjà tué, et je lui aurais laissé la vie.
\VS{34}Alors Balaam dit à l'Ange de Yahweh~: J'ai péché, car je ne savais point que tu t'étais placé au-devant de moi sur le chemin~; et maintenant, si cela te déplaît, je m'en retournerai.
\VS{35}L'Ange de Yahweh dit à Balaam~: Va avec ces hommes~; mais tu ne feras que répéter les paroles que je te dirai. Et Balaam alla avec les chefs envoyés par Balak.
\VS{36}Et quand Balak apprit que Balaam arrivait, il sortit à sa rencontre jusqu'à la ville de Moab, qui est sur la limite de l'Arnon, à l'extrême frontière.
\VS{37}Et Balak dit à Balaam~: N'ai-je pas auparavant envoyé vers toi pour t'appeler~? Pourquoi n'es-tu pas venu vers moi~? Ne puis-je donc pas te traiter avec honneur~?
\VS{38}Et Balaam répondit à Balak~: Je suis venu vers toi~; mais pourrais-je maintenant dire quelque chose~? Je ne dirai que les paroles que Dieu m'aura mis dans la bouche.
\VS{39}Et Balaam alla avec Balak, et ils arrivèrent dans la cité de Kirjath-Hutsoth.
\VS{40}Et Balak sacrifia des bœufs et des brebis, et il en envoya à Balaam et aux chefs qui étaient venus avec lui.
\VS{41}Quand le matin fut venu, il prit Balaam et le fit monter à Bamoth-Baal, et de là il vit une partie du peuple.
\Chap{23}
\TextTitle{Balaam ne maudit pas mais bénit Israël des hauts lieux de Baal}
\VerseOne{}Et Balaam dit à Balak~: Bâtis-moi ici sept autels, et prépare-moi ici sept veaux et sept béliers.
\VS{2}Et Balak fit ce que Balaam avait dit~; et Balak offrit avec Balaam un veau et un bélier sur chaque autel.
\VS{3}Balaam dit à Balak~: Tiens-toi près de ton holocauste, et je m'éloignerai~; peut-être que Yahweh viendra à ma rencontre, et je te rapporterai tout ce qu'il me révélera. Ainsi il se retira à l'écart.
\VS{4}Et Dieu vint au-devant de Balaam, et Balaam lui dit~: J'ai dressé sept autels, et j'ai sacrifié un veau et un bélier sur chaque autel.
\VS{5}Et Yahweh mit des paroles dans la bouche de Balaam et lui dit~: Retourne vers Balak, et tu parleras ainsi.
\VS{6}Il s'en retourna donc vers lui~; et voici, Balak se tenait près de son holocauste, tant lui que tous les chefs de Moab.
\VS{7}Alors Balaam prononça son discours sentencieux et dit~: Balak, roi de Moab, m'a fait descendre d'Aram\FTNT{De l'hébreu «~Aram~» traduit par «~Aram~» ou «~Syrie~» (1 R. 11:25).}, des montagnes d'orient, en me disant~: Viens, maudis-moi Jacob~! Viens, dis-je, déteste Israël~!
\VS{8}Mais comment le maudirai-je~? Dieu ne l'a point maudit. Et comment le détesterai-je~? Yahweh ne l'a point détesté.
\VS{9}Car je le regarderai du sommet des rochers, et je le contemplerai du haut des collines~: Voici, ce peuple habitera à part, et il ne sera pas compté parmi les nations\FTNT{De. 33:28.}.
\VS{10}Qui comptera la poussière de Jacob, et dira le nombre du quart d'Israël~? Que je meure de la mort des justes, et que ma fin soit semblable à la leur~!
\VS{11}Alors Balak dit à Balaam~: Que m'as-tu fait~? Je t'ai pris pour maudire mes ennemis, et voici, tu les a bénis, tu les a bénis\FTNT{Le verbe bénir vient de l'hébreu «~Barak~», il est utilisé deux fois de suite dans ce passage. Voir commentaire en Ge. 2:16-17.}.~!
\VS{12}Et il répondit, et dit~: Ne prendrais-je pas garde de dire les paroles que Yahweh aura mis dans ma bouche~?
\TextTitle{Balaam bénit Israël au sommet de Pisga}
\VS{13}Alors Balak lui dit~: Viens, je te prie, avec moi dans un autre lieu, d'où tu pourras le voir, car tu en voyais seulement une extrémité, et tu ne le voyais pas tout entier~; maudis-le moi de là.
\VS{14}Puis l'ayant conduit au territoire de Tsophim, sur le sommet de Pisga~; il bâtit sept autels, et offrit un taureau et un bélier sur chaque autel.
\VS{15}Alors Balaam dit à Balak~: Tiens-toi ici près de ton holocauste, et je m'en irai à la rencontre de Dieu, comme j'ai déjà fait.
\VS{16}Yahweh donc vint au-devant de Balaam, il mit des paroles dans sa bouche et lui dit~: Retourne vers Balak, et tu parleras ainsi.
\VS{17}Il retourna vers Barak~; et voici, il se tenait près de son holocauste, et les chefs de Moab avec lui. Et Balak lui dit~: Qu'est-ce que Yahweh a dit~?
\VS{18}Alors il prononça son discours sentencieux et dit~: Lève-toi, Balak, écoute~! Fils de Tsippor, prête-moi l'oreille~!
\VS{19}Dieu n'est point un homme pour mentir ni fils d'un homme pour se repentir. Ce qu'il a dit, ne le fera-t-il pas~? Ce qu'il a déclaré, ne l'exécutera-t-il pas\FTNT{Ja. 1:17.}~?
\VS{20}Voici, j'ai reçu la parole pour bénir~: Puisqu'il a béni, je ne le révoquerai point.
\VS{21}Il n'a point aperçu d'iniquité en Jacob, il ne voit point de perversité en Israël~; Yahweh, son Dieu, est avec lui, et il y a en lui un chant de triomphe royal\FTNT{Jé. 50:20~; Ro. 4:7.}.
\VS{22}Dieu les a tirés d'Egypte, il est pour eux comme la vigueur du buffle.
\VS{23} Car il n'y a pas d'enchantement contre Jacob, ni la divination contre Israël. Au temps marqué, il sera dit à Jacob et à Israël~: Qu'est-ce que Dieu a fait~?
\VS{24}Voici, ce peuple se lèvera comme un vieux lion, et se dressera comme un lion qui est dans sa force~; il ne se couchera pas jusqu'à ce qu'il ait dévoré la proie, et bu le sang des blessés à mort.
\VS{25}Balak dit à Balaam~: Et bien~! Ne le maudis pas, mais du moins ne le bénis pas.
\VS{26}Et Balaam répondit à Balak~: Ne t'ai-je pas dit que tout ce que Yahweh dira, je le ferai~? 
\TextTitle{Balaam bénit Israël de Peor}
\VS{27}Balak dit encore à Balaam~: Viens maintenant, je te conduirai dans un autre lieu~; peut-être que Dieu trouvera bon que tu me le maudisses de là.
\VS{28}Balak conduisit donc Balaam sur le sommet de Peor, qui regarde du côté de Jeshimon.
\VS{29}Et Balaam lui dit~: Bâtis-moi ici sept autels, et apprête-moi ici sept veaux et sept béliers.
\VS{30}Et Balak fit donc comme Balaam lui avait dit~; puis il offrit un taureau et un bélier sur chaque autel.
\Chap{24}
\VerseOne{}Or Balaam, voyant que Yahweh voulait bénir Israël, n'alla plus comme les autres fois chercher des enchantements~; mais il tourna son visage du côté du désert.
\VS{2}Et Balaam leva les yeux, il vit Israël qui se tenait rangé selon ses tribus. Alors l'Esprit de Dieu fut sur lui.
\VS{3}Et il prononça à haute voix son discours sentencieux et dit~: Balaam, fils de Beor, dit, et l'homme qui a l'œil ouvert dit:
\VS{4}Celui qui entend les paroles de Dieu, qui voit la vision du Tout-Puissant, qui tombe à terre, et qui a les yeux ouverts dit~:
\VS{5}Que tes tentes sont belles, ô Jacob~! Et tes tabernacles, ô Israël~!
\VS{6}Ils sont étendus comme des torrents, comme des jardins près d'un fleuve, comme des arbres d'aloès que Yahweh a plantés, comme des cèdres auprès des eaux.
\VS{7}L'eau coule de ses seaux, et sa semence est parmi d'abondantes eaux. Et son roi s'élève au-dessus d'Agag, et son royaume sera haut élevé.
\VS{8}Dieu, qui l'a tiré d'Egypte, est pour lui comme la vigueur du buffle~; il consumera les nations qui sont ses ennemies~; il brisera leurs os, et les percera de ses flèches.
\VS{9}Il s'est courbé, il s'est couché comme un lion qui est dans sa force, et comme un vieux lion~; qui le réveillera~? Quiconque te bénit, sera béni, et quiconque te maudit, sera maudit.
\VS{10}Alors Balak se mit très en colère contre Balaam, il frappa des mains et Balak parla ainsi à Balaam~: Je t'ai appelé pour maudire mes ennemis, et voici, tu les as bénis, tu les a bénis\FTNT{voir le commentaire en Ge. 2:16-17.} trois fois déjà.
\VS{11}Et maintenant, fuis dans ton pays~! J'avais dit que je t'honorerais, je t'honorerais\FTNT{Le mot hébreu est utilisé deux fois dans ce passage. Voir le commentaire en Ge. 2:17.}, mais Yahweh t'empêche d'être honoré.
\VS{12}Et Balaam répondit à Balak~: N'ai-je pas dit à tes messagers que tu m'as envoyés~:
\VS{13}Quand Balak me donnerait sa maison pleine d'argent et d'or, je ne pourrais transgresser l'ordre de Yahweh pour faire de moi-même du bien ou du mal~; mais ce que Yahweh dira, je le dirai.
\VS{14}Maintenant donc je m'en vais vers mon peuple. Viens, je te donnerai un conseil, et je te dirai ce que ce peuple fera à ton peuple, dans les derniers jours.
\TextTitle{Prophétie sur le Roi qui sort de Jacob, le Messie}
\VS{15}Alors il prononça son discours sentencieux et dit~: Balaam, fils de Beor dit, et l'homme qui a l'œil ouvert dit~:
\VS{16}Celui qui entend les paroles de Dieu, qui connaît la science du Très-Haut, qui voit la vision du Tout-Puissant, qui tombe à terre, et qui a les yeux ouverts.
\VS{17}Je le vois, mais non pas maintenant~; je le regarde, mais non pas de près~; une Etoile est sortie de Jacob\FTNT{L'Etoile en question est Jésus-Christ, qui se révéla à Jean comme l'Etoile brillante du matin (Ap. 22:16).}, et un Sceptre s'est élevé d'Israël. Il transpercera les cotés de Moab, il détruira tous les enfants de Seth.
\VS{18}Edom sera sa possession, Séir sera possédé par ses ennemis, et Israël se portera vaillamment.
\VS{19}Et il y en aura un de Jacob qui dominera, il fera périr le reste de la ville.
\VS{20}Il vit aussi Amalek, il prononça son discours sentencieux et dit~: Amalek est le premier des nations, mais à la fin il sera détruit.
\VS{21}Il vit aussi les Kéniens\FTNT{Il y a plusieurs sens à ce mot~:
\\Caïn = «~possession~», «~artisan, forgeron~», fils d'Adam.
\\Kéniens = «~forgerons~» tribu du beau-père de Moïse qui vivait dans la région du sud de la Palestine.}. Il prononça à haute voix son discours sentencieux et dit~: Ta demeure est dans un lieu solide, et tu as mis ton nid dans le rocher~;
\VS{22}toutefois, le Kénien sera consumé, jusqu'à ce que l'Assyrien t'emmène en captivité.
\VS{23}Il continua à prononcer à haute voix son discours sentencieux, et il dit~: Malheur à celui qui vivra quand Dieu fera ces choses.
\VS{24}Et des navires viendront de Kittim, et ils humilieront l'Assyrien et l'Hébreu~; et lui aussi sera détruit.
\VS{25}Puis Balaam se leva, et s'en alla pour retourner chez lui. Balak aussi s'en alla son chemin.
\Chap{25}
\TextTitle{Prostitution d'Israël à Baal-Peor\FTNTT{No. 31:16~; Ja. 4:4~; Ap. 2:14.}}
\VerseOne{}Alors Israël demeurait à Sittim~; et le peuple commença à commettre la fornication avec les filles de Moab.
\VS{2}Car elles convièrent le peuple aux sacrifices de leurs dieux~; et le peuple mangea et se prosterna devant leurs dieux.
\VS{3}Et Israël s'accoupla à Baal-Peor, c'est pourquoi la colère de Yahweh s'enflamma contre Israël\FTNT{Ps. 106:28~; Os. 9:10.}.
\VS{4}Et Yahweh dit à Moïse~: Prends tous les chefs du peuple, et fais-les pendre devant Yahweh en face du soleil, afin que la colère de Yahweh se détourne d'Israël\FTNT{De. 4:3~; Jos. 22:17.}.
\VS{5}Moïse donc dit aux juges d'Israël~: Que chacun de vous fasse mourir les hommes qui sont à sa charge, et qui se sont joints à Baal-Peor.
\VS{6}Et voici, un homme des enfants d'Israël vint, et amena à ses frères une Madianite, devant Moïse et devant toute l'assemblée des fils d'Israël, tandis qu'ils pleuraient à l'entrée de la tente d'assignation.
\VS{7}Ce que Phinées, fils d'Eléazar, fils d'Aaron le prêtre, ayant vu, se leva du milieu de l'assemblée et prit une lance dans sa main.
\VS{8}Et il entra dans la tente de l'homme Israélite et les transperça tous deux, l'homme Israélite puis la femme, par le ventre. Et la plaie s'arrêta parmi les enfants d'Israël\FTNT{Ps. 106:30.}.
\VS{9}Or il y en eut vingt-quatre mille qui moururent de cette plaie.
\VS{10}Et Yahweh parla à Moïse, en disant~:
\VS{11}Phinées, fils d'Eléazar, fils d'Aaron, le prêtre, a détourné ma colère de dessus les enfants d'Israël, parce qu'il a été animé de mon zèle au milieu d'eux~; et je n'ai point, dans mon ardeur, consumé les fils d'Israël.
\VS{12}C'est pourquoi, dis-lui~: Voici, je lui donne mon alliance de paix.
\VS{13}Et l'alliance de prêtrise perpétuelle sera tant pour lui que pour sa postérité après lui, parce qu'il a été animé de zèle pour son Dieu, et qu'il a fait propitiation pour les enfants d'Israël.
\VS{14}Et le nom de l'homme Israélite tué, lequel fut tué avec la Madianite, était Zimri, fils de Salu, chef d'une maison de père des Siméonites.
\VS{15}Et le nom de la femme Madianite qui fut tuée était Cozbi, fille de Tsur, chef du peuple, et d'une maison de père en Madian.
\VS{16}Yahweh parla à Moïse, en disant~:
\VS{17}Mettez en détresse les Madianites, tuez-les~;
\VS{18}car ils vous ont serrés les premiers par leurs ruses, par lequelles ils vous surpris dans l'affaire de Peor, et dans l'affaire de Cozbi, fille d'un chef d'entre les Madianites, leur sœur, qui a été tuée le jour de la plaie, causée par l'affaire de Peor.
\Chap{26}
\TextTitle{Nouveau dénombrement des hommes de guerre}
\VerseOne{} Or il arriva qu'après cette plaie-là, que Yahweh parla à Moïse, et à Eléazar, fils d'Aaron, le prêtre, en disant~:
\VS{2}Faites le dénombrement de toute l'assemblée des enfants d'Israël, depuis l'âge de vingt ans et au-dessus, selon les maisons de leurs pères, à savoir de tous ceux d'Israël qui peuvent aller à la guerre.
\VS{3}Moïse donc et Eléazar, le prêtre, leur parlèrent donc dans les plaines de Moab, près du Jourdain de Jéricho, en disant~:
\VS{4}Qu'on fasse le dénombrement depuis l'âge de vingt ans et au-dessus, comme Yahweh l'avait ordonné à Moïse et aux enfants d'Israël, quand ils furent sortis du pays d'Egypte.
\VS{5}Ruben, premier-né d'Israël. Fils de Ruben~: Hénoc, de qui descend la famille des Hénokites~; Pallu, de qui descend la famille des Palluites~;
\VS{6}Hetsron, de qui descend la famille des Hetsronites~; Carmi, de qui descend la famille des Carmites.
\VS{7}Ce sont là les familles des Rubénites~: Ceux qui furent dénombrés étaient quarante-trois mille sept cent trente.
\VS{8}Et les fils de Pallu~: Eliab.
\VS{9}Fils d'Eliab~: Nemuel, Dathan et Abiram. Ce Dathan et cet Abiram, qui étaient de ceux qu'on appelait pour tenir l'assemblée, et qui se révoltèrent contre Moïse et contre Aaron dans l'assemblée de Koré, lors de leur révolte contre Yahweh.
\VS{10}Et lorsque la terre ouvrit sa bouche et les engloutit, ainsi que Koré, ceux qui s'étaient assemblés avec lui moururent. Et le feu dévora les deux cent cinquante hommes qui servirent d'avertissement.
\VS{11}Mais les fils de Koré ne moururent pas.
\VS{12}Les fils de Siméon selon leurs familles~: De Nemuel descend la famille des Némuélites~; de Jamin, la famille des Jaminites~; de Jakin, la famille des Jakinites~;
\VS{13}de Zérach, la famille des Zérachites~; de Saül, la famille des Saülites.
\VS{14}Ce sont là les familles des Siméonites, qui furent vingt-deux mille deux cents.
\VS{15}Fils de Gad selon leurs familles. De Tsephon, descend la famille des Tsephonites~; de Haggi, la famille des Haggites~; de Schuni, la famille des Schunites~;
\VS{16}d'Ozni, la famille des Oznites~; d'Eri, la famille des Erites~;
\VS{17}d'Arod, la famille des Arodites~; d'Areéli, la famille des Areélites.
\VS{18}Ce sont là les familles des fils de Gad, d'après leur dénombrement~: Quarante mille cinq cents.
\VS{19}Fils de Juda, Er, et Onan~; mais Er et Onan moururent au pays de Canaan\FTNT{Ge. 38:7-10~; Ge. 46:12.}.
\VS{20}Voici les fils de Juda selon leurs familles~: De Schéla descend la famille des Schélanites~; de Pérets, la famille des Péretsites~; de Zérach, la famille des Zérachites.
\VS{21}Les fils de Pérets furent~: Hetsron, de qui descend la famille des Hetsronites~; Hamul, de qui descend la famille des Hamulites.
\VS{22}Ce sont là les familles de Juda, selon leur dénombrement: Soixante-seize mille cinq cents.
\VS{23}Fils d'Issacar, selon leurs familles~: De Thola descend la famille des Tholaïtes~; de Puva, la famille des Puvites~;
\VS{24}de Jaschub, la famille des Jaschubites~; de Schimron, la famille des Schimronites.
\VS{25}Ce sont là les familles d'Issacar, d'après leur dénombrement~: Soixante-quatre mille trois cents.
\VS{26}Fils de Zabulon, selon leurs familles~: De Séred, descend la famille des Sardites~; d'Elon, la famille des Elonites~; de Jahleel, la famille des Jahleélites.
\VS{27}Ce sont là les familles des Zabulonites, d'après leur dénombrement~: Soixante mille cinq cents.
\VS{28}Fils de Joseph, selon leurs familles~: Manassé et Ephraïm.
\VS{29}Fils de Manassé. De Makir descend la famille des Makirites. Makir engendra Galaad. De Galaad descend la famille des Galaadites.
\VS{30}Voici les fils de Galaad~: Jézer, de qui descend la famille des Jézerites~; Hélek, la famille des Hélekites.
\VS{31}Asriel, la famille des Asriélites~; Sichem, la famille des Sichémites~;
\VS{32}Schemida, la famille des Schemidaïtes~; Hépher, la famille des Héphrites.
\VS{33}Tselophchad, fils de Hépher, n'eut point de fils, mais des filles. Voici les noms des filles de Tselophchad~: Machla, Noa, Hogla, Milca, et Thirtsa.
\VS{34}Ce sont là les familles de Manassé, d'après leur dénombrement~: Cinquante-deux mille sept cents.
\VS{35}Voici les fils d'Ephraïm, selon leurs familles~: De Schutélach descend la famille des Schutalchites~; de Béker, la famille des Bakrites~; de Thachan, la famille des Thachanites.
\VS{36}Voici les fils de Schutélach~: D'Eran est descendue la famille des Eranites.
\VS{37}Ce sont là les familles des fils d'Ephraïm, d'après leur dénombrement~: Trente-deux mille cinq cents. Ce sont là les fils de Joseph, selon leurs familles.
\VS{38}Fils de Benjamin, selon leurs familles~: De Béla descend la famille des Balites~; d'Aschbel, la famille des Aschbélites~; d'Achiram, la famille des Achiramites~;
\VS{39}De Schupham, la famille des Schuphamites~; de Hupham, la famille des Huphamites.
\VS{40}Les fils de Béla furent Ard et Naaman. D'Ard descend la famille des Ardites~; et de Naaman la famille des Naamanites.
\VS{41}Ce sont là les fils de Benjamin, d'après leurs familles~; et leur dénombrement~: Quarante-cinq mille six cents.
\VS{42}Voici les fils de Dan, selon leurs familles~: De Schucham descend la famille des Schuchamites. Ce sont là les familles de Dan, selon leurs familles.
\VS{43}Toutes les familles des Schuchamites, selon leur dénombrement~: Soixante-quatre mille quatre cents.
\VS{44}Fils d'Aser, selon leurs familles~: De Jimna descend la famille des Jimnites~; de Jischvi, la famille des Jischvites~; de Beria la famille des Beriites.
\VS{45}Des fils de Beria descendent~: De Héber, la famille des Hébrites~; de Malkiel, la famille des Malkiélites.
\VS{46}Et le nom de la fille d'Aser était Sérach.
\VS{47}Ce sont là les familles des fils d'Aser, d'après leur dénombrement~: Cinquante-trois mille quatre cents.
\VS{48}Fils de Nephthali, selon leurs familles~: De Jahtseel descend la famille des Jahtseélites~; de Guni, la famille des Gunites~;
\VS{49}de Jetser la famille des Jitsrites~; de Schillem, la famille des Schillémites.
\VS{50}Ce sont là les familles de Nephthali, selon leurs familles, et leur dénombrement~: Quarante-cinq mille quatre cents.
\VS{51}Voici les dénombrés des fils d'Israël, qui furent six cent un mille sept cent trente.
\VS{52}Yahweh parla à Moïse, en disant~:
\VS{53}Le pays sera partagé entre ceux-ci en héritage, selon le nombre des noms.
\VS{54}A ceux qui sont en plus grand nombre, tu donneras plus d'héritage, et à ceux qui sont en plus petit nombre tu donneras moins d'héritage~; on donnera à chacun son héritage selon le nombre de ses dénombrés.
\VS{55}Toutefois, que le pays soit partagé par le sort~; et qu'ils prennent leur héritage selon les noms des tribus de leurs pères\FTNT{Jos. 11:23~; Jos. 14:2~; Jos. 18:6-8.}.
\VS{56}L'héritage de chacun sera selon que le sort le montrera, et on aura égard au plus grand et au plus petit nombre.
\VS{57}Et ce sont ici les dénombrés de Lévi selon leurs familles~; de Guerschon, la famille des Guerschonites~; de Kehath, la famille des Kehathites~; de Merari, la famille des Merarites.
\VS{58}Ce sont ici les familles de Lévi~; la famille des Libnites, la famille des Hébronites, la famille des Machlites, la famille des Muschites, la famille des Korites. Kehath engendra Amram.
\VS{59}Et le nom de la femme d'Amram était Jokébed, fille de Lévi, qui naquit à Lévi en Egypte~; et elle enfanta à Amram: Aaron, Moïse, et Marie, leur sœur.
\VS{60}Et il naquit à Aaron~: Nadab et Abihu, Eléazar et Ithamar.
\VS{61}Nadab et Abihu moururent lorsqu'ils apportèrent du feu étranger devant Yahweh\FTNT{Lé. 10:1-2~; 1 Ch. 24:2.}.
\VS{62}Et tous les dénombrés des Lévites furent vingt-trois mille, tous mâles, depuis l'âge d'un mois, et au dessus, qui ne furent point dénombrés avec les autres enfants d'Israël, car on ne leur donna point d'héritage entre les enfants d'Israël.
\VS{63}Ce sont là ceux qui furent dénombrés par Moïse et Eléazar, le prêtre, qui firent le dénombrement des fils d'Israël dans les plaines de Moab, près du Jourdain de Jéricho.
\VS{64}Entre lesquels il ne s'en trouva aucun de ceux qui avaient été dénombrés par Moïse et Aaron le prêtre, quand ils firent le dénombrement des enfants d'Israël au désert de Sinaï.
\VS{65}Car Yahweh avait dit d'eux~: ils mourront certainement dans le désert, et qu'ainsi il n'en restera pas un, excepté Caleb, fils de Jephunné, et Josué, fils de Nun\FTNT{1 Co. 10:5.}.
\Chap{27}
\TextTitle{Loi sur les héritages\FTNTT{No. 36.}}
\VerseOne{}Or les filles de Tselophchad, fils de Hépher, fils de Galaad, fils de Makir, fils de Manassé, d'entre les familles de Manassé, fils de Joseph, s'approchèrent~; et ce sont ici les noms de ses filles~: Machla, Noa, Hogla, Milca, et Thirtsa.
\VS{2}Elles se présentèrent devant Moïse, devant Eléazar, le prêtre, et devant les princes et toute l'assemblée, à l'entrée de la tente d'assignation. Elles dirent~:
\VS{3}Notre père est mort dans le désert~; il n'était toutefois pas dans la troupe de ceux qui s'assemblèrent contre Yahweh, dans l'assemblée de Koré, mais il est mort dans son péché, et il n'avait point de fils.
\VS{4}Pourquoi le nom de notre père serait-il retranché de sa famille, parce qu'il n'a point eu de fils~? Donne-nous une possession parmi les frères de notre père.
\VS{5}Moïse rapporta leur cause devant Yahweh.
\VS{6}Et Yahweh parla à Moïse, en disant~:
\VS{7}Les filles de Tselophchad ont parlé droitement. Tu ne manqueras pas de leur donner un héritage à posséder parmi les frères de leur père, et tu leur feras passer l'héritage de leur père.
\VS{8}Tu parleras aussi aux enfants d'Israël, et tu leur diras~: Lorsqu'un homme mourra sans avoir de fils, vous ferez passer son héritage à sa fille.
\VS{9}S'il n'a pas de fille, vous donnerez son héritage à ses frères.
\VS{10}S'il n'a pas de frères, vous donnerez son héritage aux frères de son père.
\VS{11}Et si son père n'a pas de frère, vous donnerez son héritage à son parent le plus proche de sa famille, et il le possédera. Et ce sera pour les enfants d'Israël une ordonnance de droit, comme Yahweh l'a ordonné à Moïse.
\TextTitle{Moïse voit de loin le pays promis aux fils d'Israël}
\VS{12}Yahweh dit aussi à Moïse~: Monte sur cette montagne d'Abarim, et regarde le pays que je donne aux enfants d'Israël\FTNT{De. 32:48-49.}.
\VS{13}Tu le regarderas donc~; et puis tu seras toi aussi recueilli auprès de ton peuple, comme Aaron ton frère y a été recueilli~;
\VS{14}parce que vous avez été rebelles à mon ordre dans le désert de Tsin, lors de la contestation de l'assemblée, vous ne m'avez point sanctifié au sujet des eaux devant eux~; ce sont les eaux de Meriba, à Kadès, dans le désert de Tsin.
\TextTitle{Yahweh désigne Josué comme successeur de Moïse}
\VS{15}Moïse parla à Yahweh, en disant~:
\VS{16}Que Yahweh, le Dieu des esprits de toute chair, établisse sur l'assemblée un homme\FTNT{Hé. 12:9.},
\VS{17}qui sorte devant eux et qui entre devant eux, et qui les fasse sortir et qui les fasse entrer, afin que l'assemblée de Yahweh ne soit pas comme des brebis qui n'ont point de berger\FTNT{1 R. 22:17~; Mt. 9:36~; Mc. 6:34.}.
\VS{18}Alors Yahweh dit à Moïse~: Prends Josué, fils de Nun, un homme en qui est l'Esprit, et tu poseras ta main sur lui\FTNT{De. 34:9.}.
\VS{19}Tu le présenteras devant Eléazar, le prêtre, et devant toute l'assemblée~; et tu lui donneras des instructions sous leurs yeux.
\VS{20}Et tu lui feras part de ton autorité, afin que toute l'assemblée des enfants d'Israël l'écoute.
\VS{21}Et il se présentera devant Eléazar, le prêtre, qui consultera pour lui les jugements de l'urim\FTNT{Lé. 8:8.} devant Yahweh~; et à sa parole ils sortiront, et à sa parole ils entreront, lui, les enfants d'Israël, avec lui, et toute l'assemblée.
\VS{22}Moïse donc fit comme Yahweh lui avait ordonné. Il prit Josué et le présenta devant Eléazar, le prêtre, et devant toute l'assemblée.
\VS{23}Puis il posa ses mains sur lui, et lui donna des instructions, comme Yahweh l'avait dit par Moïse.
\Chap{28}
\TextTitle{Consignes relatives au temps des sacrifices}
\VerseOne{}Yahweh parla à Moïse, en disant~:
\VS{2}Donne cet ordre aux enfants d'Israël, et dis-leur~: Vous aurez soin de m'offrir en leur temps, mon offrande, ma nourriture, pour mes sacrifices consumés par le feu, qui me sont d'une bonne odeur\FTNT{Lé. 3:11~; Lé. 21:6.}.
\VS{3}Tu leur diras~: Voici le sacrifice consumé par le feu que vous offrirez à Yahweh~: Deux agneaux d'un an sans défaut, chaque jour, en holocauste perpétuel\FTNT{Ex. 29:38.}.
\VS{4}Tu sacrifieras l'un des agneaux le matin, et l'autre agneau entre les deux soirs,
\VS{5}et la dixième partie d'épha de fine farine pour le gâteau pétrie avec le quart d'un hin d'huile vierge\FTNT{Lé. 2:1~; Ex. 29:40~; Ex. 16:36.}.
\VS{6}C'est l'holocauste perpétuel, qui a été offert à la montagne de Sinaï, c'est un sacrifice consumé par le feu, d'une bonne odeur à Yahweh.
\VS{7}Et sa libation sera d'un quart de hin pour chaque agneau~: Et tu verseras dans le lieu saint la libation de boisson forte à Yahweh.
\VS{8}Et tu sacrifieras l'autre agneau entre les deux soirs, tu feras le même gâteau qu'au matin, et la même libation, en sacrifice consumé par le feu d'une bonne odeur à Yahweh.
\VS{9}Mais le jour du sabbat vous offrirez deux agneaux d'un an sans défaut, et deux dixièmes de fine farine pétrie à l'huile pour le gâteau, avec sa libation.
\VS{10}C'est l'holocauste du sabbat, pour chaque sabbat, outre l'holocauste perpétuel avec sa libation.
\VS{11}Et au commencement de vos mois, vous offrirez en holocauste à Yahweh deux jeunes taureaux, un bélier, et sept agneaux d'un an sans défaut~;
\VS{12}et trois dixièmes de fine farine pétrie à l'huile, pour le gâteau de chaque taureau, et deux dixièmes de fine farine pétrie à l'huile pour le gâteau du bélier~;
\VS{13}et un dixième de fine farine pétrie à l'huile, comme gâteau pour chaque agneau, en holocauste, d'une bonne odeur, et en sacrifice consumé par le feu à Yahweh.
\VS{14}Et leurs libations seront d'un demi-hin de vin pour chaque veau, d'un tiers de hin pour un bélier, et d'un quart de hin pour chaque agneau, c'est l'holocauste du commencement de chaque mois, selon tous les mois de l'année.
\VS{15}On sacrifiera aussi à Yahweh un jeune bouc en sacrifice d'expiation, outre l'holocauste perpétuel, et sa libation.
\VS{16}Au quatorzième jour du premier mois, ce sera la Pâque à Yahweh.
\VS{17}Et au quinzième jour du même mois sera un jour de fête. On mangera pendant sept jours des pains sans levain\FTNT{Ex. 12~; Lé. 23:5-6.}.
\VS{18}Au premier jour, il y aura une sainte convocation~: Vous ne ferez aucune œuvre servile.
\VS{19}Et vous offrirez un sacrifice consumé par le feu en holocauste à Yahweh~: Deux jeunes taureaux, un bélier, et sept agneaux d'un an, sans défaut.
\VS{20}Leur gâteau sera de fine farine pétrie à l'huile, vous en offrirez trois dixièmes pour chaque jeune taureau, et deux dixièmes pour un bélier~;
\VS{21}tu en offriras aussi un dixième pour chacun des sept agneaux,
\VS{22}et un bouc en sacrifice pour l'expiation, afin de faire propitiation pour vous.
\VS{23}Vous offrirez ces choses là, outre l'holocauste du matin, qui est l'holocauste perpétuel.
\VS{24}Vous offrirez ces choses-là chaque jour, pendant sept jours, comme l'aliment d'un sacrifice consumé par le feu, d'une bonne odeur à Yahweh. On offrira cela outre l'holocauste perpétuel, et sa libation.
\VS{25}Et au septième jour, vous aurez une sainte convocation~: Vous ne ferez aucune œuvre servile.
\VS{26}Et au jour des prémices, quand vous offrirez à Yahweh une offrande nouvelle de gâteau à votre fête des semaines, vous aurez une sainte convocation~: Vous ne ferez aucune œuvre servile.
\VS{27}Et vous offrirez en holocauste d'une bonne odeur à Yahweh, deux jeunes taureaux, un bélier, et sept agneaux d'un an.
\VS{28}Et leur gâteau sera de fine farine pétrie à l'huile, de trois dixièmes pour chaque jeune taureau, et de deux dixièmes pour le bélier,
\VS{29}et d'un dixième pour chacun des sept agneaux~;
\VS{30}et un jeune bouc, afin de faire propitiation pour vous.
\VS{31}Vous les offrirez, outre l'holocauste perpétuel et son offrande, lesquels seront sans défaut, avec leurs libations.
\Chap{29}
\TextTitle{Consignes relatives au temps des sacrifices - suite}
\VerseOne{}Et le premier jour du septième mois, vous aurez une sainte convocation~: Vous ne ferez aucune œuvre servile. Ce jour sera publié parmi vous au son des trompettes\FTNT{Lé. 23:24-25.}.
\VS{2}Et vous offrirez en holocauste de bonne odeur à Yahweh, un jeune taureau, un bélier, et sept agneaux d'un an, sans défaut.
\VS{3}Et leur gâteau sera de fine farine pétrie à l'huile, de trois dixièmes pour le jeune taureau, de deux dixièmes pour le bélier,
\VS{4}et un dixième pour chacun des sept agneaux.
\VS{5}Et un jeune bouc en sacrifice pour l'expiation, afin de faire propitiation pour vous,
\VS{6}outre l'holocauste du commencement du mois et son gâteau, et l'holocauste perpétuel et son gâteau, et leurs libations selon leur ordonnance. Ce sont des sacrifices consumés par le feu en bonne odeur à Yahweh.
\VS{7}Et au dixième jour de ce septième mois, vous aurez une sainte convocation, et vous affligerez vos âmes~: Vous ne ferez aucune œuvre\FTNT{Lé. 16:29-31~; Lé. 23:27.}.
\VS{8}Et vous offrirez en holocauste, de bonne odeur à Yahweh, un jeune taureau, un bélier, et sept agneaux d'un an, qui seront sans défaut.
\VS{9}Et leur gâteau sera de fine farine pétrie à l'huile, de trois dixièmes pour le taureau, et de deux dixièmes pour le bélier,
\VS{10}et d'un dixième pour chacun des sept agneaux.
\VS{11}Un jeune bouc aussi en sacrifice d'expiation, outre le sacrifice des expiations, l'holocauste perpétuel et son gâteau, avec leurs libations.
\VS{12}Et au quinzième jour du septième mois, vous aurez une sainte convocation~: Vous ne ferez aucune œuvre servile. Vous célébrerez une fête à Yahweh, pendant sept jours\FTNT{Lé. 23:34-43.}.
\VS{13}Et vous offrirez en holocauste un sacrifice consumé par le feu, d'une agréable odeur à Yahweh, treize jeunes taureaux, deux béliers, et quatorze agneaux d'un an, sans défaut.
\VS{14}Et leur gâteau sera de fine farine pétrie à l'huile, de trois dixièmes pour chacun des treize jeunes taureaux, de deux dixièmes pour chacun des deux béliers,
\VS{15}et d'un dixième pour chacun des quatorze agneaux.
\VS{16}Et un jeune bouc en sacrifice d'expiation, outre l'holocauste perpétuel, son gâteau, et sa libation.
\VS{17}Et au second jour, vous offrirez douze jeunes taureaux, deux béliers, et quatorze agneaux d'un an, sans défaut,
\VS{18}avec les gâteaux et les libations pour les jeunes taureaux, pour les béliers, et pour les agneaux, selon leur nombre, d'après les ordonnances.
\VS{19}Vous offrirez un jeune bouc en sacrifice d'expiation, outre l'holocauste perpétuel, et son offrande, avec leurs libations.
\VS{20}Et au troisième jour, vous offrirez onze taureaux, deux béliers, et quatorze agneaux d'un an, sans défaut~;
\VS{21}et les gâteaux et les libations pour les jeunes taureaux, les béliers et les agneaux, selon leur nombre, selon leur ordonnance.
\VS{22}Et un bouc en sacrifice d'expiation, outre l'holocauste continuel, son gâteau et sa libation.
\VS{23}Et au quatrième jour, vous offrirez dix jeunes taureaux, deux béliers, et quatorze agneaux d'un an, sans défaut,
\VS{24}les gâteaux et les libations pour les taureaux, les béliers, et les agneaux, selon leur nombre et leur ordonnance.
\VS{25}Et un jeune bouc en sacrifice d'expiation, outre l'holocauste perpétuel, son offrande, et sa libation.
\VS{26}Et au cinquième jour, vous offrirez neuf jeunes taureaux, deux béliers, et quatorze agneaux d'un an, sans défaut,
\VS{27}avec les gâteaux et les libations pour les taureaux, les béliers, et les agneaux, selon leur nombre et leur ordonnance.
\VS{28}Et un bouc en sacrifice d'expiation, outre l'holocauste continuel, son gâteau, et sa libation.
\VS{29}Et le sixième jour, vous offrirez huit jeunes taureaux, deux béliers et quatorze agneaux d'un an, sans défaut,
\VS{30}et les gâteaux, les libations pour les taureaux, les béliers, et les agneaux selon leur nombre leur ordonnance.
\VS{31}Et un bouc en sacrifice d'expiation, outre l'holocauste continuel, son offrande, et sa libation.
\VS{32}Et au septième jour, vous offrirez sept jeunes taureaux, deux béliers, et quatorze agneaux d'un an, sans défaut,
\VS{33}avec les gâteaux et les libations pour les jeunes taureaux, les béliers, et les agneaux, selon leur nombre et leur ordonnance.
\VS{34}Et un bouc en sacrifice d'expiation, outre l'holocauste continuel, son gâteau, et sa libation.
\VS{35}Et au huitième jour, vous aurez une assemblée solennelle~: Vous ne ferez aucune œuvre servile.
\VS{36}Et vous offrirez en holocauste un sacrifice consumé par le feu, d'une agréable odeur à Yahweh~: Un jeune taureau, un bélier, et sept agneaux d'un an, sans défaut,
\VS{37}avec les gâteaux et les libations pour le jeune taureau, le bélier, et les agneaux, selon leur nombre et leur ordonnance.
\VS{38}Et un bouc en sacrifice d'expiation, outre l'holocauste perpétuel, son offrande, et sa libation.
\VS{39}Vous offrirez ces choses à Yahweh dans vos fêtes solennelles, outre vos vœux, et vos offrandes volontaires, selon vos holocaustes, vos gâteaux, vos libations, et vos sacrifices d'offrande de paix.
\Chap{30}
\TextTitle{Les vœux}
\VerseOne{}Et Moïse parla aux enfants d'Israël selon toutes les choses que Yahweh lui avait ordonné.
\VS{2}Moïse parla aussi aux chefs des tribus des enfants d'Israël, en disant~: Voici ce que Yahweh ordonne.
\VS{3}Quand un homme fera un vœu à Yahweh, ou aura juré par serment, pour lier son âme par un vœu, il ne violera pas sa parole~; il fera selon toutes les choses qui sont sorties de sa bouche\FTNT{De. 23:21.}.
\VS{4}Mais quand une femme fera un vœu à Yahweh, et qu'elle se liera par un serment, dans sa jeunesse, étant encore dans la maison de son père,
\VS{5}et que son père aura entendu son vœu et le serment par lequel elle a lié son âme, si son père ne lui dit rien, tous ses vœux seront valables, et tout serment par lequel elle aura lié son âme sera valable~;
\VS{6}mais si son père la désapprouve le jour où il l'a entendue, aucun de ses vœux ou de ses serments par lesquels elle a lié son âme ne sera valable, et Yahweh lui pardonnera~; parce que son père l'a désapprouvée.
\VS{7}Et si elle a un mari, et qu'elle s'est engagée par quelque vœu ou par une parole échappée de ses lèvres par laquelle elle aura lié son âme,
\VS{8}et que son mari l'aura entendue, et que le jour même où il l'a entendue, il ne lui a rien dit, ses vœux alors seront valables, et ses serments par lesquels elle aura lié son âme seront valables~;
\VS{9}mais si son mari la désapprouve le jour où il l'a entendue, alors il annulera le vœu par lequel elle s'est engagée et la parole échappée de ses lèvres, par laquelle elle avait lié son âme~; et Yahweh lui pardonnera.
\VS{10}Mais le vœu de la veuve ou de la répudiée, tout ce par quoi elle aura lié son âme, sera valable pour elle.
\VS{11}Que si étant encore dans la maison de son mari elle a fait un vœu, ou si elle a lié son âme par serment,
\VS{12}et que son mari l'ait entendue, et ne lui en ait rien dit, et ne l'ait pas désapprouvée, alors tous ses vœux seront valables, et tout serment par lequel elle a lié son âme sera valable.
\VS{13}Mais si son mari les a entièrement annulés le jour où il les a entendus, alors rien de ce qui est sorti de ses lèvres, soit ses vœux, soit le serment par lequel elle a lié son âme ne seront valables~; parce que son mari les a annulés, et Yahweh lui pardonnera.
\VS{14}Son mari ratifiera ou son mari annulera tout vœu et toute obligation faite par serment, pour affliger l'âme.
\VS{15}Mais si son mari ne lui en a absolument rien dit, d'un jour à l'autre, il aura ratifié tous ses vœux ou toutes ses obligations dont elle était tenue~; il les aura, dis-je, ratifiés, parce qu'il ne lui en a rien dit le jour où il les a entendus.
\VS{16}Mais s'il les a expressément annulés après les avoir entendus, alors il portera l'iniquité de sa femme.
\VS{17}Telles sont les ordonnances que Yahweh ordonna à Moïse, entre un mari et sa femme~; entre un père et sa fille, étant encore dans la maison de son père, dans sa jeunesse.
\Chap{31}
\TextTitle{Jugements sur Madian\FTNTT{No. 25:6-18.}}
\VerseOne{}Yahweh parla à Moïse, en disant~:
\VS{2}Fais la vengeance des enfants d'Israël sur les Madianites, puis tu seras recueilli auprès de ton peuple.
\VS{3}Moïse donc parla au peuple, en disant~: Que quelques-uns d'entre vous s'équipent pour aller à la guerre, et qu'ils aillent contre Madian, pour exécuter la vengeance de Yahweh sur Madian.
\VS{4}Vous enverrez à la guerre mille hommes de chaque tribu, de toutes les tribus d'Israël.
\VS{5}On donna d'entre les milliers d'Israël mille hommes de chaque tribu, qui furent douze mille hommes équipés pour la guerre.
\VS{6}Moïse les envoya à la guerre, savoir mille de chaque tribu, et avec eux Phinées, fils d'Eléazar, le prêtre, qui portait les instruments sacrés et les trompettes retentissantes.
\VS{7}Ils s'avancèrent donc contre Madian, comme Yahweh l'avait ordonné à Moïse, et ils en tuèrent tous les mâles.
\VS{8}Ils tuèrent aussi les rois de Madian, outre les autres qui y furent tués, Evi, Rékem, Tsur, Hur, et Réba, cinq rois de Madian~; ils firent aussi passer au fil de l'épée Balaam, fils de Beor\FTNT{Jos. 13:21-22.}.
\VS{9}Et les fils d'Israël emmenèrent prisonniers les femmes de Madian, avec leurs petits enfants, et pillèrent tout leur gros et menu bétail, et tous leurs biens.
\VS{10}Ils brûlèrent par le feu toutes leurs villes, leurs demeures, et tous leurs châteaux.
\VS{11}Ils prirent tout le butin et tout le pillage, tant des hommes que du bétail\FTNT{De. 20:14.}~;
\VS{12}puis ils amenèrent les captifs, le pillage, et le butin, à Moïse, à Eléazar le prêtre, et à l'assemblée des enfants d'Israël, au camp, dans les plaines de Moab, qui sont près du Jourdain, vis-à-vis de Jéricho.
\VS{13}Moïse, Eléazar, le prêtre, et tous les princes de l'assemblée sortirent au-devant d'eux, hors du camp.
\VS{14}Et Moïse se mit en grande colère contre les officiers de l'armée, les chefs des milliers, et les chefs des centaines, qui revenaient de cet exploit de guerre.
\VS{15}Et Moïse leur dit~: N'avez-vous pas gardé en vie toutes les femmes~?
\VS{16}Voici ce sont elles qui, à la parole de Balaam, ont donné l'occasion aux fils d'Israël de pécher contre Yahweh dans l'affaire de Peor~; ce qui attira la plaie sur l'assemblée de Yahweh\FTNT{2 Pi. 2:15~; Ap. 2:14.}.
\VS{17}Or maintenant, tuez tous les mâles d'entre les petits enfants, et tuez toute femme qui a connu un homme en couchant avec lui\FTNT{Jg. 21:11.}~;
\VS{18}mais vous garderez en vie toutes les jeunes filles qui n'ont point connu la couche d'un homme.
\VS{19}Au reste, demeurez sept jours hors du camp~; quiconque aura tué quelqu'un, et quiconque aura touché quelqu'un qui aura été tué, se purifiera le troisième et le septième jour, tant vous que vos prisonniers.
\VS{20}Vous purifierez aussi tous vos vêtements, et tout ce qui sera fait de peau, et tout ouvrage de poil de chèvre, et toute vaisselle de bois.
\VS{21}Eléazar, le prêtre, dit aux hommes de guerre qui étaient allés au combat~: Voici l'ordonnance et la loi que Yahweh a ordonné à Moïse.
\VS{22}En général l'or, l'argent, l'airain, le fer, l'étain, le plomb~;
\VS{23}tout ce qui peut passer par le feu, vous le ferez passer par le feu pour le rendre pur. Seulement on purifiera avec l'eau de purification toutes les choses qui ne peuvent aller au feu, vous les ferez passer dans l'eau.
\VS{24}Vous laverez aussi vos vêtements le septième jour, ensuite vous serez purs~; puis vous entrerez au camp.
\TextTitle{Partage du butin}
\VS{25}Et Yahweh parla à Moïse, en disant~:
\VS{26}Fais le compte du butin et de tout ce qu'on a emmené, tant des personnes que des bêtes, toi et Eléazar, le prêtre, et les chefs des pères de l'assemblée.
\VS{27}Et partage par moitié le butin entre les combattants qui sont allés à la guerre et toute l'assemblée\FTNT{1 S. 30:24.}.
\VS{28}Tu prélèveras aussi pour Yahweh un tribut sur les hommes de guerre qui sont allés à la bataille, savoir un sur cinq cents, tant des personnes, que des bœufs, des ânes et des brebis.
\VS{29}On le prendra sur leur moitié, et tu le donneras à Eléazar, le prêtre, en offrande présentée par élévation à Yahweh.
\VS{30}Et sur la moitié qui appartient aux enfants d'Israël, tu prendras un sur cinquante, tant des personnes que des bœufs, des ânes, des brebis et de tous les autres animaux, et tu le donneras aux Lévites qui ont la charge de garder le tabernacle de Yahweh.
\VS{31}Moïse et Eléazar, le prêtre, firent comme Yahweh l'avait ordonné à Moïse.
\VS{32}Or le butin qui était resté du pillage du peuple qui était allé à la guerre, était de six cent soixante-quinze mille brebis~;
\VS{33}de soixante-douze mille bœufs~;
\VS{34}de soixante et un mille ânes,
\VS{35}quant aux femmes qui n'avaient point connu la couche d' un homme, elles étaient en tout trente-deux mille âmes.
\VS{36}Et la moitié du butin, à savoir la part de ceux qui étaient allés à la guerre, montait à trois cent trente-sept mille cinq cents brebis~;
\VS{37}dont le tribut pour Yahweh, quant aux brebis, était de six cent soixante-quinze.
\VS{38}Trente-six mille bœufs~; dont le tribut pour Yahweh, quant aux bœufs, était de soixante-douze bœufs,
\VS{39}trente mille cinq cents ânes~; dont le tribut pour Yahweh, quant aux ânes, était de soixante et un ânes~;
\VS{40}et de seize mille personnes, dont le tribut pour Yahweh était de trente-deux personnes.
\VS{41}Et Moïse donna à Eléazar, le prêtre, le tribut de l'offrande présentée par élévation à Yahweh, comme Yahweh le lui avait ordonné.
\VS{42}Et de l'autre moitié qui appartenait aux enfants d'Israël, que Moïse avait tiré des hommes qui étaient allés à la guerre~;
\VS{43}or de cette moitié qui fut pour l'assemblée, et qui montait à trois cent trente-sept mille cinq cents brebis,
\VS{44}trente-six mille bœufs,
\VS{45}trente mille cinq cents ânes,
\VS{46}et à seize mille personnes~;
\VS{47}de cette moitié, dis-je, qui appartenait aux enfants d'Israël, Moïse prit un sur cinquante, tant des personnes que des bêtes, et les donna aux Lévites qui avaient la charge de garder le tabernacle de Yahweh, comme Yahweh le lui avait ordonné.
\VS{48}Les commandants des milliers de l'armée, tant les chefs des milliers que les chefs des centaines, s'approchèrent de Moïse,
\VS{49}et lui dirent~: Tes serviteurs ont fait le compte des hommes de guerre qui étaient sous nos ordres, il ne manque pas un homme d'entre nous.
\VS{50}C'est pourquoi, nous offrons l'offrande de Yahweh, chacun les objets que nous avons trouvés~: Des joyaux d'or, des chaînes de cheville, des bracelets, des anneaux, des pendants d'oreilles et des colliers, afin de faire propitiation pour nos personnes devant Yahweh.
\VS{51}Moïse et Eléazar, le prêtre, reçurent d'eux cet or, tous ces objets travaillés.
\VS{52}Et tout l'or de l'offrande présentée par élévation à Yahweh, de la part des chefs de milliers et des chefs de centaines, montait à seize mille sept cent cinquante sicles.
\VS{53}Or les hommes de guerre gardèrent chacun pour soi ce qu'ils avaient pillé.
\VS{54}Moïse donc et Eléazar, le prêtre, prirent l'or des chefs des milliers et des chefs de centaines, et l'apportèrent à la tente d'assignation, comme souvenir pour les enfants d'Israël, devant Yahweh.
\Chap{32}
\TextTitle{Ruben et Gad en Galaad}
\VerseOne{}Les fils de Ruben et les fils de Gad avaient beaucoup de bétail, en très grande quantité, et ils virent que le pays de Jaezer et le pays de Galaad étaient un lieu propre pour du bétail.
\VS{2}Ainsi les fils de Gad et les fils de Ruben vinrent, et parlèrent à Moïse et à Eléazar, le prêtre, et aux princes de l'assemblée, en disant~:
\VS{3}Atharoth, Dibon, Jaezer, Nimra, Hesbon, et Elealé, Sebam, Nebo, et Beon,
\VS{4}ce pays-là que Yahweh a frappé devant l'assemblée d'Israël, est un pays propre pour le bétail, et tes serviteurs ont des troupeaux.
\VS{5}Ils dirent donc: Si nous avons trouvé grâce à tes yeux, que ce pays soit donné en possession à tes serviteurs~; et ne nous fais point passer le Jourdain.
\VS{6}Mais Moïse répondit aux fils de Gad, et aux fils de Ruben~: Vos frères iront-ils à la guerre, et vous, demeurerez-vous ici~?
\VS{7}Pourquoi voulez-vous décourager les enfants d'Israël de passer dans le pays que Yahweh leur a donné~?
\VS{8}C'est ainsi que firent vos pères quand je les envoyai de Kadès-Barnéa pour examiner le pays.
\VS{9}Car ils montèrent jusqu'à la vallée d'Eschcol, virent le pays, puis découragèrent les enfants d'Israël, afin qu'ils n'entrent point dans le pays que Yahweh leur avait donné.
\VS{10}C'est pourquoi la colère de Yahweh s'enflamma ce jour-là, et il jura en disant~:
\VS{11}Les hommes qui sont montés hors d'Egypte, depuis l'âge de vingt ans et au-dessus, ne verront point le pays que j'ai juré de donner à Abraham, Isaac, et à Jacob~; car ils n'ont point persévéré à me suivre\FTNT{De. 1:35.},
\VS{12}excepté Caleb, fils de Jephunné, le Kénizien, et Josué, fils de Nun, car ils ont persévéré à suivre Yahweh.
\VS{13}Ainsi la colère de Yahweh s'enflamma contre Israël et il les fit errer dans le désert pendant quarante ans, jusqu'à ce que toute la génération qui avait fait le mal aux yeux de Yahweh, ait été consumée.
\VS{14}Et voici, vous vous êtes levés à la place de vos pères, comme une race d'hommes pécheurs, pour augmenter encore l'ardeur de la colère de Yahweh contre Israël.
\VS{15}Si vous vous détournez de lui, il continuera encore à vous laisser au désert, et vous ferez détruire tout ce peuple.
\VS{16}Mais ils s'approchèrent de lui et lui dirent~: Nous bâtirons ici des cloisons pour nos troupeaux, et des villes pour nos petits enfants~;
\VS{17}et nous nous équiperons pour marcher promptement devant les enfants d'Israël, jusqu'à ce que nous les ayons introduits en leur lieu~; mais nos petits enfants demeureront dans les villes fortes, à cause des habitants du pays.
\VS{18}Nous ne retournerons point dans nos maisons avant que chacun des enfants d'Israël n'ait pris possession de son héritage~;
\VS{19}et nous ne posséderons rien en héritage avec eux au-delà du Jourdain, ni plus avant~; parce que nous aurons notre héritage de ce côté-ci du Jourdain, à l'orient.
\VS{20}Et Moïse leur dit~: Si vous faites cela, si vous vous équipez devant Yahweh pour aller à la guerre,
\VS{21}si chacun de vous étant équipé passe le Jourdain devant Yahweh, jusqu'à ce qu'il ait chassé ses ennemis loin de devant lui,
\VS{22}et que le pays soit assujetti devant Yahweh, et qu'ensuite vous vous en retournez, alors vous serez innocents envers Yahweh, et envers Israël~; et ce pays-ci vous appartiendra pour le posséder devant Yahweh.
\VS{23}Mais si vous ne faites point cela, vous péchez contre Yahweh~; et sachez que votre péché vous atteindra.
\VS{24}Bâtissez donc des villes pour vos petits enfants, et des cloisons pour vos troupeaux, et faites ce que vous avez dit.
\VS{25}Alors les fils de Gad et les fils de Ruben parlèrent à Moïse, en disant~: Tes serviteurs feront ce que mon seigneur a ordonné.
\VS{26}Nos petits enfants, nos femmes, nos troupeaux, et tout notre bétail demeureront ici dans les villes de Galaad~;
\VS{27}et tes serviteurs passeront chacun armés pour aller à la guerre devant Yahweh, prêts à combattre, comme mon seigneur a parlé.
\VS{28}Alors Moïse donna des ordres à leur sujet à Eléazar, le prêtre, à Josué, fils de Nun, et aux chefs des pères des tribus des fils d'Israël.
\VS{29}Il leur dit~: Si les fils de Gad et les fils de Ruben passent avec vous le Jourdain tous armés, prêts à combattre devant Yahweh, et que le pays vous soit assujetti, vous leur donnerez le pays de Galaad en possession.
\VS{30}Mais s'ils ne marchent point en armes avec vous, qu'ils s'établissent au milieu de vous dans le pays de Canaan.
\VS{31}Les fils de Gad et les fils de Ruben répondirent, en disant~: Nous ferons ce que Yahweh a dit à tes serviteurs.
\VS{32}Nous passerons en armes devant Yahweh au pays de Canaan, afin que nous possédions pour notre héritage ce qui est de ce côté-ci du Jourdain.
\VS{33}Ainsi Moïse donna aux fils de Gad et aux fils de Ruben, et à la demi-tribu de Manassé, fils de Joseph, le royaume de Sihon, roi des Amoréens~; et le royaume de Og, roi de Basan, le pays avec ses villes, selon les bornes des villes du pays tout autour.
\VS{34}Alors les fils de Gad rebâtirent Dibon, Atharoth, Aroër,
\VS{35}Athroth-Schophan, Jaezer, Jogbeha,
\VS{36}Beth-Nimra et Beth-Haran, villes fortifiées. Ils firent aussi des cloisons pour les troupeaux.
\VS{37}Et les fils de Ruben rebâtirent Hesbon, Elealé, Kirjathaim,
\VS{38}Nébo, Baal-Meon, et Sibma, dont ils changèrent les noms, et ils donnèrent des noms aux villes qu'ils rebâtirent.
\VS{39}Or les fils de Makir, fils de Manassé, allèrent en Galaad, le prirent et dépossédèrent les Amoréens qui y étaient.
\VS{40}Moïse donc donna Galaad à Makir, fils de Manassé, qui y habita\FTNT{De. 3:15.}.
\VS{41}Jaïr, fils de Manassé, se mit en marche, prit leurs villages, et les appela villages de Jaïr\FTNT{De. 3:14~; 1 Ch. 2:22.}.
\VS{42}Et Nobach se mit en marche, prit Kenath avec les villes de son ressort, et l'appela Nobach d'après son nom.
\Chap{33}
\TextTitle{Les stations de l'Egypte jusqu'au Jourdain}
\VerseOne{}Ce sont ici les étapes des enfants d'Israël, qui sortirent du pays d'Egypte, selon leurs armées, sous la main de Moïse et d'Aaron.
\VS{2}Moïse écrivit leurs départs, et leurs étapes, d'après l'ordre de Yahweh~! Et voici leurs étapes selon leurs départs.
\VS{3}Les enfants d'Israël donc partirent de Ramsès le quinzième jour du premier mois, dès le lendemain de la Pâque, et ils sortirent à main levée, à la vue de tous les Egyptiens\FTNT{Ex. 14:8.}.
\VS{4}Et les Egyptiens ensevelissaient ceux que Yahweh avait frappés parmi eux, à savoir tous les premiers-nés~; même Yahweh exerçait aussi ses jugements contre leurs dieux\FTNT{Ex. 12:12~; Ex. 18:11.}.
\VS{5}Et les enfants d'Israël partirent de Ramsès, et campèrent à Succoth\FTNT{Ex. 12:37.}.
\VS{6}Et ils partirent de Succoth et campèrent à Etham, qui est au bout du désert\FTNT{Ex. 13:20.}.
\VS{7}Et ils partirent d'Etham et se détournèrent vers Pi-Hahiroth, qui est vis-à-vis de Baal-Tsephon, et campèrent devant Migdol\FTNT{Ex. 14:2.}.
\VS{8}Et ils partirent de devant Pi-Hahiroth et passèrent au travers de la mer vers le désert, et firent trois journées de marche par le désert d'Etham et campèrent à Mara.
\VS{9}Puis ils partirent de Mara et vinrent à Elim où il y avait douze fontaines d'eaux et soixante-dix palmiers, et ils y campèrent\FTNT{Ex. 15:27.}.
\VS{10}Et ils partirent d'Elim et campèrent près de la Mer Rouge.
\VS{11}Puis ils partirent de la Mer Rouge et campèrent au désert de Sin\FTNT{Ex. 16:1.}.
\VS{12}Ils partirent du désert de Sin et campèrent à Dophka.
\VS{13}Puis ils partirent de Dophka et campèrent à Alusch.
\VS{14}Et ils partirent d'Alusch et campèrent à Rephidim où il n'y avait point d'eau à boire pour le peuple\FTNT{Ex. 17:1.}.
\VS{15}Puis ils partirent de Rephidim et campèrent dans le désert de Sinaï\FTNT{Ex. 17:1.}.
\VS{16}Ils partirent du désert de Sinaï et campèrent à Kibroth-Hattaava.
\VS{17}Et ils partirent de Kibroth-Hattaava et campèrent à Hatséroth.
\VS{18}Puis ils partirent de Hatséroth et campèrent à Rithma.
\VS{19}Et ils partirent de Rithma et campèrent à Rimmon-Pérets.
\VS{20}Ils partirent de Rimmon-Pérets et campèrent à Libna.
\VS{21}Et ils partirent de Libna et campèrent à Rissa.
\VS{22}Puis ils partirent de Rissa et campèrent vers Kehélatha.
\VS{23}Et ils partirent de Kehélatha et campèrent à la montagne de Schapher.
\VS{24}Ils partirent de la montagne de Schapher et campèrent à Harada.
\VS{25}Et ils partirent de Harada et campèrent à Makhéloth.
\VS{26}Puis ils partirent de Makhéloth et campèrent à Tahath.
\VS{27}Ils partirent de Tahath et campèrent à Tarach.
\VS{28}Et ils partirent de Tarach et campèrent à Mithka.
\VS{29}Puis ils partirent de Mithka et campèrent à Haschmona.
\VS{30}Ils partirent de Haschmona et campèrent à Moséroth.
\VS{31}Et ils partirent de Moséroth et campèrent à Bené-Jaakan.
\VS{32}Ils partirent de Bené-Jaakan et campèrent à Hor-Guidgad.
\VS{33}Puis ils partirent de Hor-Guidgad et campèrent vers Jothbatha.
\VS{34}Ils partirent de Jothbatha et campèrent à Abrona.
\VS{35}Et ils partirent d'Abrona et campèrent à Etsjon-Guéber.
\VS{36}Ils partirent d'Etsjon-Guéber et campèrent dans le désert de Tsin, qui est Kadès.
\VS{37}Puis ils partirent de Kadès et campèrent à la montagne de Hor, qui est au bout du pays d'Edom.
\VS{38}Et Aaron le prêtre, monta sur la montagne de Hor, suivant l'ordre de Yahweh, et mourut là, la quarantième année après que les enfants d'Israël furent sortis du pays d'Egypte, le premier jour du cinquième mois.
\VS{39}Et Aaron était âgé de cent vingt-trois ans quand il mourut sur la montagne de Hor.
\VS{40}Alors le Cananéen, roi d'Arad, qui habitait vers le midi au pays de Canaan, apprit que les enfants d'Israël venaient.
\VS{41}Et ils partirent de la montagne de Hor et campèrent à Tsalmona.
\VS{42}Puis ils partirent de Tsalmona et campèrent à Punon.
\VS{43}Et Ils partirent de Punon et campèrent à Oboth.
\VS{44}Ils partirent d'Oboth et campèrent à Ijjé-Abarim, sur les frontières de Moab.
\VS{45}Puis ils partirent d'Ijjé-Abarim et campèrent à Dibon-Gad.
\VS{46}Et ils partirent de Dibon-Gad, et campèrent à Almon-Diblathaïm.
\VS{47}Ils partirent d'Almon-Diblathaïm et campèrent aux montagnes de Abarim devant Nébo.
\VS{48}Et ils partirent des montagnes d'Abarim et campèrent aux plaines de Moab, près du Jourdain de Jéricho.
\VS{49}Pui ils campèrent près du Jourdain, depuis Beth-Jeschimoth jusqu'à Abel-Sittim, dans les plaines de Moab.
\TextTitle{Consignes pour les possessions attribuées à Israël}
\VS{50}Et Yahweh parla à Moïse dans les plaines de Moab, près du Jourdain de Jéricho, en disant~:
\VS{51}Parle aux enfants d'Israël, et dis-leur~: Puisque vous allez passer le Jourdain pour entrer au pays de Canaan,
\VS{52}vous chasserez de devant vous tous les habitants du pays, vous détruirez toutes leurs peintures, et vous ruinerez toutes leurs images de fonte, et vous démolirez tous leurs hauts lieux\FTNT{De. 7:5~; De. 12:2.}.
\VS{53}Et vous prendrez possession du pays, et vous y habiterez. Car je vous ai donné le pays pour le posséder.
\VS{54}Or vous recevrez le pays en héritage par le sort, selon vos familles. A ceux qui sont en plus grand nombre, vous donnerez plus d'héritage, et à ceux qui sont en plus petit nombre, vous donnerez moins d'héritage. Chacun aura selon ce qui lui sera échu par le sort, et vous hériterez selon les tribus de vos pères.
\VS{55}Mais si vous ne chassez pas de devant vous les habitants du pays, il arrivera que ceux d'entre eux que vous aurez laissés comme reste, seront comme des épines à vos yeux, et comme des pointes à vos côtés, et ils vous serreront de près dans le pays auquel vous habiterez\FTNT{Jos. 23:13.}.
\VS{56}Et il arrivera que je vous ferai tout comme j'ai eu dessein de leur faire.
\Chap{34}
\TextTitle{Consignes sur les limites de chaque tribu}
\VerseOne{}Yahweh parla aussi à Moïse, en disant~:
\VS{2}Donne l'ordre aux enfants d'Israël, et dis-leur~: Parce que vous allez entrer au pays de Canaan, ce pays deviendra votre héritage, le pays de Canaan selon ses limites.
\VS{3}Votre frontière du côté du sud sera depuis le désert de Tsin, le long d'Edom, et votre frontière du côté du sud commencera au bout de la mer salée, vers l'orient~;
\VS{4}et cette frontière tournera du sud vers la montée d'Akrabbim, et passera jusqu'à Tsin~; et elle aboutira du côté du sud de Kadès-Barnéa~; et sortira aussi par Hatsar-Addar, et passera jusqu'à Atsmon.
\VS{5}Et cette frontière tournera depuis Atsmon jusqu'au torrent d'Egypte~; et elle aboutira à la mer.
\VS{6}Quant à la frontière d'occident, vous aurez la grande mer et ses limites~; ce sera votre frontière occidentale.
\VS{7}Et ce sera ici votre frontière au nord~; depuis la grande mer, vous marquerez pour vos limites la montagne de Hor~;
\VS{8}et depuis la montagne de Hor, vous marquerez pour vos limites l'entrée de Hamath, et cette frontière aboutira vers Tsedad~;
\VS{9}cette frontière passera jusqu'à Ziphron, et elle aboutira à Hatsar-Enan~; telle sera votre frontière au nord.
\VS{10}Puis vous marquerez pour vos limites vers l'orient de Hatsar-Enan à Schepham.
\VS{11}Et cette frontière descendra de Schepham à Ribla, du côté de l'orient d'Aïn~; et cette frontière descendra et s'étendra le long de la Mer de Kinnéreth vers l'orient.
\VS{12}Cette frontière descendra au Jourdain pour aboutir à la Mer Salée~; tel sera le pays que vous aurez avec ses limites tout autour.
\VS{13}Et Moïse donna l'ordre aux enfants d'Israël, en disant~: C'est là le pays que vous hériterez par le sort, et que Yahweh a ordonné de donner à neuf tribus, et à la demi-tribu.
\VS{14}Car la tribu des fils de Ruben selon les familles de leurs pères, et la tribu des fils de Gad, selon les familles de leurs pères, ont pris leur héritage~; et la demi-tribu de Manassé a pris aussi son héritage.
\VS{15}Deux tribus, dis-je, et la demi-tribu ont pris leur héritage de l'autre côté du Jourdain, vis-à-vis de Jéricho, du côté du levant.
\VS{16}Et Yahweh parla à Moïse, en disant~:
\VS{17}Ce sont ici les noms des hommes qui vous partageront le pays~: Eléazar le prêtre, et Josué fils de Nun.
\VS{18}Vous prendrez aussi un prince de chaque tribu pour faire le partage du pays.
\VS{19}Et voici les noms de ces hommes. Pour la tribu de Juda~: Caleb, fils de Jephunné~;
\VS{20}pour la tribu des fils de Siméon~: Samuel, fils d'Ammihud~;
\VS{21}pour la tribu de Benjamin~: Elidad, fils de Kislon~;
\VS{22}pour la tribu des fils de Dan~: Celui qui en est le chef, Buki, fils de Jogli~;
\VS{23}pour les fils de Joseph, pour la tribu des fils de Manassé~: Celui qui en est le chef, Hanniel, fils d'Ephod~;
\VS{24}et pour la tribu des fils d'Ephraïm~: Celui qui en est le chef, Kemuel, fils de Schiphtan~;
\VS{25}pour la tribu des fils de Zabulon~: Celui qui en est le chef, Elitsaphan, fils de Parnac~;
\VS{26}pour la tribu des fils d'Issacar~: Celui qui en est le chef, Paltiel, fils d'Azzan~;
\VS{27}pour la tribu des fils d'Aser~: Celui qui en est le chef, Ahihud, fils de Schelomi~;
\VS{28}pour la tribu des fils de Nephthali~: Celui qui en est le chef, Pedahel, fils d'Ammihud.
\VS{29}Ce sont là, ceux à qui Yahweh donna l'ordre de partager l'héritage aux enfants d'Israël dans le pays de Canaan.
\Chap{35}
\TextTitle{Quarante-huit villes pour les Lévites dont six villes de refuge}
\VerseOne{}Yahweh parla à Moïse dans les plaines de Moab, près du Jourdain, vis-à-vis de Jéricho, en disant~:
\VS{2}Donne l'ordre aux enfants d'Israël qu'ils donnent aux Lévites, sur l'héritage qu'ils posséderont, des villes pour y habiter. Vous leur donnerez aussi les faubourgs qui sont autour de ces villes\FTNT{Jos. 21:2.}.
\VS{3}Ils auront donc les villes pour y habiter~; et les faubourgs de ces villes seront pour leurs bétails, pour leurs biens, et pour tous leurs animaux.
\VS{4}Les faubourgs des villes que vous donnerez aux Lévites, seront de mille coudées tout autour depuis la muraille de la ville en dehors.
\VS{5}Et vous mesurerez depuis le dehors de la ville du côté de l'orient, deux mille coudées~; et du côté du sud, deux mille coudées~; et du côté de l'occident, deux mille coudées~; et du côté du nord, deux mille coudées~; et la ville sera au milieu~; tels seront les faubourgs de leurs villes.
\VS{6}Et des villes que vous donnerez aux Lévites, il y aura six villes de refuge que vous donnerez pour que le meurtrier s'y enfuie, et outre celles-là, vous leur donnerez quarante-deux villes.
\VS{7}Toutes les villes que vous donnerez aux Lévites seront quarante-huit villes, elles et leurs faubourgs.
\VS{8}Et quant aux villes que vous leur donnerez sur la possession des enfants d'Israël, de ceux qui en auront plus vous en prendrez plus, et de ceux qui en auront moins vous en prendrez moins~; chacun donnera de ses villes aux Lévites, en proportion de l'héritage qu'il possédera.
\VS{9}Puis Yahweh parla à Moïse, en disant~:
\VS{10}Parle aux enfants d'Israël, et dis-leur~: Quand vous aurez passé le Jourdain, pour entrer au pays de Canaan~;
\VS{11}établissez-vous des villes qui vous soient des villes de refuge, afin que le meurtrier qui aura frappé à mort quelqu'un involontairement, s'y enfuie\FTNT{Jos. 20:2-3~; Ex. 21:13.}.
\VS{12}Et ces villes seront pour vous des villes de refuge contre le vengeur, afin que le meurtrier ne meure pas, jusqu'à ce qu'il ait comparu en jugement devant l'assemblée.
\VS{13}De ces villes que vous donnerez, il y en aura six de refuge pour vous.
\VS{14}Vous donnerez trois de ces villes au-delà du Jourdain, et les trois autres dans le pays de Canaan, qui seront des villes de refuge\FTNT{De. 19:2~; De. 4:41-42.}.
\VS{15}Ces six villes serviront de refuge aux enfants d'Israël, à l'étranger et à celui qui séjourne au milieu de vous, afin que quiconque aura frappé à mort quelqu'un involontairement, s'y enfuie.
\VS{16}Mais si un homme en frappe un autre avec un instrument de fer, et qu'il en meure, il est meurtrier~; on punira de mort le meurtrier.
\VS{17}Et s'il le frappe avec une pierre qu'il tenait à la main, dont on puisse mourir, et qu'il en meure, c'est un meurtrier~; le meurtrier sera puni de mort.
\VS{18}De même s'il le frappe d'un instrument de bois qu'il tenait à la main, dont on puisse mourir, et qu'il en meure, il est un meurtrier~; on punira de mort le meurtrier.
\VS{19}Et le vengeur du sang fera mourir le meurtrier quand il le rencontrera, il pourra le faire mourir.
\VS{20}Et s'il le pousse par haine, ou s'il jette quelque chose sur lui avec préméditation, et qu'il en meure~;
\VS{21}ou si par inimitié il le frappe de sa main, et qu'il en meure, on punira de mort celui qui l'a frappé, car il est meurtrier~; le vengeur du sang pourra le faire mourir quand il le rencontrera\FTNT{De. 19:11-12.}.
\VS{22}Mais s'il le pousse subitement, sans inimitié, ou s'il jette quelque chose sur lui, sans préméditation,
\VS{23}ou s'il fait tomber sur lui quelque pierre sans l'avoir vu, et qu'il en meure, n'étant pas son ennemi et ne lui cherchant pas du mal,
\VS{24}alors l'assemblée jugera entre celui qui a frappé et le vengeur du sang, selon ces ordonnances~;
\VS{25}l'assemblée délivrera le meurtrier de la main du vengeur de sang, et le fera retourner dans la ville de refuge où il s'était enfui. Il y demeurera jusqu'à la mort du grand-prêtre, qui aura été oint de la sainte huile.
\VS{26}Mais si le meurtrier sort de quelque manière que ce soit hors des bornes de la ville de son refuge, où il s'est enfui,
\VS{27}et si le vengeur du sang le rencontre hors des bornes de la ville de son refuge, et qu'il tue le meurtrier, il ne sera point coupable de meurtre.
\VS{28}Car il doit demeurer dans la ville de son refuge jusqu'à la mort du grand-prêtre~; et après la mort du grand-prêtre, le meurtrier pourra retourner dans sa possession.
\VS{29}Et ces choses-ci seront des ordonnances de jugement pour vous et pour vos générations, dans toutes vos demeures.
\VS{30}Celui qui fera mourir le meurtrier, le fera mourir sur la parole de deux témoins~; mais un seul témoin ne sera point reçu en témoignage contre quelqu'un, pour le faire mourir\FTNT{De. 17:6~; De. 19:15.}.
\VS{31}Et vous ne prendrez point de rançon pour la vie du meurtrier, qui est coupable et digne de mort~; mais il doit être puni de mort.
\VS{32}Vous ne prendrez point de rançon pour le laisser s'enfuir de sa ville de refuge, pour qu'il retourne habiter dans le pays, jusqu'à la mort du prêtre.
\VS{33}Et vous ne souillerez point le pays où vous serez, car le sang souille le pays~; et il ne se fera point de propitiation pour le pays, du sang qui y sera répandu que par le sang de celui qui l'aura répandu.
\VS{34}Vous ne souillerez donc point le pays où vous allez demeurer, et au milieu duquel j'habiterai~; car je suis Yahweh qui habite au milieu des enfants d'Israël.
\Chap{36}
\TextTitle{Loi sur les héritages\FTNTT{No. 27:1-11.}}
\VerseOne{}Or les chefs des pères de la famille des fils de Galaad, fils de Makir, fils de Manassé, d'entre les familles des fils de Joseph, s'approchèrent et parlèrent devant Moïse, et devant les princes, les chefs des pères des enfants d'Israël,
\VS{2}et ils dirent~: Yahweh a donné l'ordre à mon seigneur de donner aux enfants d'Israël le pays en héritage par le sort~; et mon seigneur a reçu l'ordre de Yahweh de donner l'héritage de Tselophchad, notre frère, à ses filles.
\VS{3}Si elles se marient à l'un des fils des autres tribus d'Israël, leur héritage sera retranché de l'héritage de nos pères et sera ajouté à l'héritage de la tribu de laquelle elles seront~; ainsi sera diminué l'héritage qui nous est échu par le sort.
\VS{4}Même quand viendra le jubilé pour les enfants d'Israël, on ajoutera leur héritage à l'héritage de la tribu à laquelle elles appartiendront, ainsi leur héritage sera retranché de l'héritage de la tribu de nos pères\FTNT{Lé. 25:10-13.}.
\VS{5}Et Moïse ordonna aux enfants d'Israël, suivant l'ordre de la bouche de Yahweh, en disant~: Ce que la tribu des fils de Joseph dit est juste.
\VS{6}C'est ici ce que Yahweh ordonne au sujet des filles de Tselophchad~: Elles se marieront à qui bon leur semblera, toutefois elles se marieront dans l'une des familles de la tribu de leurs pères.
\VS{7}Ainsi l'héritage ne sera point transporté entre les enfants d'Israël de tribu en tribu~; car chacun des enfants d'Israël se tiendra à l'héritage de la tribu de ses pères.
\VS{8}Et toute fille, qui possédera un héritage d'entre les tribus des enfants d'Israël, se mariera à quelqu'un de la famille de la tribu de son père, afin que chacun des enfants d'Israël possède l'héritage de ses pères.
\VS{9}L'héritage donc ne sera point transporté d'une tribu à une autre, mais chacune des tribus des enfants d'Israël se tiendra à son héritage.
\VS{10}Les filles de Tselophchad firent comme Yahweh avait donné à Moïse.
\VS{11}Machla, Thirtsa, Hogla, Milca, et Noa, filles de Tselophchad, se marièrent aux fils de leurs oncles.
\VS{12}Ainsi elles se marièrent à ceux qui étaient des familles des fils de Manassé, fils de Joseph~; et leur héritage demeura dans la tribu de la famille de leur père.
\VS{13}Ce sont là les ordonnances et les jugements que Yahweh ordonna par Moïse aux enfants d'Israël, dans les plaines de Moab, près du Jourdain, vis-à-vis de Jéricho.
\PPE{}
\end{multicols}

\clearpage\ShortTitle{De.}\BookTitle{Deutéronome}\BFont
\noindent\hrulefill
{\footnotesize
\textit{
\bigskip
{\centering{}
\\Auteur~: Probablement Moïse
\\(Heb.~: Devarim)
\\Signification~: Paroles
\\Thème~: Rappel de la loi
\\Date de rédaction~: 1450-1410 av. J.-C.\\}
}
\textit{
\\Ce livre est un rappel de la loi de Yahweh. Après quarante années d'errance dans le désert, Moïse s'adresse à la nouvelle génération par des discours et des exhortations, depuis les plaines de Moab. Au travers de son serviteur, Dieu rappelle ainsi la loi donnée sur le mont Sinaï, les expériences vécues par la génération passée et par conséquent, l'importance de la soumission à Dieu. De leur obéissance dépendraient les bénédictions ou les malédictions contenues dans ce livre.\bigskip
}
}
\par\nobreak\noindent\hrulefill
\begin{multicols}{2}
\Chap{1}
\TextTitle{Rappel de l'infidélité d'Israël\FTNTT{No. 14.}}
\VerseOne{}Ce sont ici les paroles que Moïse déclara à tout Israël de l'autre côté du Jourdain, dans le désert, dans la plaine, qui est vis-à-vis de Suph, entre Paran, Tophel, Laban, Hatséroth, et Di-Zahab.
\VS{2}Il y a onze journées depuis Horeb, par le chemin de la montagne de Séir, jusqu'à Kadès-Barnéa.
\VS{3}Or il arriva dans la quarantième année, au onzième mois, le premier jour du mois, que Moïse parla aux enfants d'Israël selon tout ce que Yahweh lui avait ordonné de leur dire,
\VS{4}après qu'il eut battu Sihon, roi des Amoréens, qui habitait à Hesbon, et Og, roi de Basan, qui demeurait à Aschtaroth et à Edréi\FTNT{No. 21:23-24.}.
\VS{5}Moïse donc commença à expliquer cette loi, de l'autre côté du Jourdain, dans le pays de Moab, en disant~:
\VS{6}Yahweh, notre Dieu, nous a parlé à Horeb, en disant~: Vous avez assez demeuré dans cette montagne.
\VS{7}Tournez-vous, partez, et allez à la montagne des Amoréens et dans tous les lieux voisins, dans la plaine, dans la montagne, dans la vallée, vers le sud, sur le rivage de la mer, au pays des Cananéens et au Liban, jusqu'au grand fleuve, le fleuve d'Euphrate.
\VS{8}Regardez, j'ai mis devant vous le pays~; entrez et prenez possession du pays que Yahweh a juré de donner à vos pères, Abraham, Isaac et Jacob, et à leur postérité après eux.
\VS{9}Et je vous parlai en ce temps-là, et je vous dis~: Je ne puis pas, à moi seul, vous porter.
\VS{10}Yahweh, votre Dieu, vous a multipliés, et vous voici aujourd'hui comme les étoiles du ciel par le nombre.
\VS{11}Que Yahweh, le Dieu de vos pères, vous fasse croître mille fois au delà de ce que vous êtes et vous bénisse, comme il vous l'a dit~!
\VS{12}Comment porterais-je moi seul vos chagrins, vos charges, et vos procès~?
\VS{13}Prenez dans vos tribus des hommes sages, intelligents et connus, et je les établirai chefs sur vous.
\VS{14}Et vous me répondîtes, et dîtes~: Il est bon de faire ce que tu as dit.
\VS{15}Alors je pris les chefs de vos tribus, des hommes sages et connus, et je les établis chefs sur vous, chefs de milliers, chefs de centaines, chefs de cinquantaines, chefs de dizaines et officiers selon vos tribus. 
\VS{16}Puis j'ordonnai en ce temps-là à vos juges, en disant~: Ecoutez les différends qui seront entre vos frères, et jugez droitement entre l'homme et son frère, et entre l'étranger qui est avec lui\FTNT{Lé. 19:15~; De. 16:19~; Pr. 24:23.}.
\VS{17}Vous n'aurez point d'égard à l'apparence de la personne en jugement~; vous entendrez autant le petit que le grand~; vous ne craindrez personne, car le jugement est à Dieu~; et vous ferez venir devant moi la cause qui sera trop difficile pour vous, et je l'entendrai. 
\VS{18}Et en ce temps-là, je vous ordonnai toutes les choses que vous deviez faire.
\VS{19}Puis nous partîmes d'Horeb, et nous marchâmes dans tout ce grand et affreux désert que vous avez vu~; par le chemin de la montagne des Amoréens, ainsi que Yahweh, notre Dieu, nous l'avait ordonné, et nous vînmes jusqu'à Kadès-Barnéa.
\VS{20}Alors je vous dis~: Vous êtes arrivés jusqu'à la montagne des Amoréens, que Yahweh, notre Dieu, nous donne.
\VS{21}Regarde, Yahweh, ton Dieu, met le pays devant toi~; monte et prends-en possession, comme Yahweh, le Dieu de tes pères, te l'a dit~; ne crains point et ne t'effraie point.
\VS{22}Et vous vous approchâtes tous de moi, et dîtes~: Envoyons devant nous des hommes pour explorer le pays, et qui nous rapportent des nouvelles du chemin par lequel nous devrons monter, et des villes où nous devrons aller\FTNT{No. 13:2.}.
\VS{23}Et ce discours sembla bon à mes yeux~; et je pris douze hommes parmi vous, un homme par tribu.
\VS{24}Et ils se mirent en chemin et montèrent dans la montagne, et vinrent jusqu'au torrent d'Eschcol et explorèrent le pays.
\VS{25}Et ils prirent dans leurs mains des fruits du pays, et ils nous les apportèrent~; ils nous donnèrent des nouvelles, et nous dirent~: Le pays que Yahweh, notre Dieu, nous donne est bon. 
\VS{26}Mais vous refusâtes d'y monter, et vous fûtes rebelles à l'ordre de Yahweh, votre Dieu.
\VS{27}Et vous murmurâtes dans vos tentes, en disant~: C'est parce que Yahweh nous hait qu'il nous a fait sortir du pays d'Egypte, afin de nous livrer entre les mains des Amoréens pour nous exterminer.
\VS{28}Où monterions-nous~? Nos frères nous ont fait fondre le cœur, en disant~: Le peuple est plus grand que nous, et de plus haute taille~; les villes sont grandes et closes jusqu'au ciel~; et même nous avons vu là les fils des Anakim.
\VS{29}Mais je vous dis~: Ne tremblez point et ne les craignez point.
\VS{30}Yahweh, votre Dieu, qui marche devant vous, lui-même combattra pour vous, selon tout ce que vous avez vu qu'il a fait pour vous en Egypte~;
\VS{31}et au désert, où tu as vu de quelle manière Yahweh, ton Dieu, t'a porté comme un homme porterait son fils, sur tout le chemin où vous avez marché, jusqu'à ce que vous soyez arrivés dans ce lieu-ci.
\VS{32}Mais malgré cela, vous ne crûtes point encore en Yahweh, votre Dieu,
\VS{33}qui marchait devant vous sur le chemin afin de vous chercher un lieu pour camper, marchant de nuit dans la colonne de feu pour vous éclairer dans le chemin par lequel vous deviez marcher et de jour dans la nuée.
\VS{34}Et Yahweh entendit la voix de vos paroles et se mit en grande colère et jura, disant~:
\VS{35}Aucun des hommes de cette méchante génération ne verra ce bon pays que j'ai juré de donner à vos pères,
\VS{36}à l'exception de Caleb, fils de Jephunné~; lui le verra, et je donnerai à lui et à ses fils le pays sur lequel il a marché, parce qu'il a persévéré à suivre Yahweh\FTNT{No. 14:22-24.}.
\VS{37}Même Yahweh s'est mis en colère contre moi à cause de vous, disant~: Et toi aussi tu n'y entreras pas.
\VS{38}Josué, fils de Nun, qui te sert, y entrera~; fortifie-le, car c'est lui qui mettra les enfants d'Israël en possession de ce pays\FTNT{De. 34:4.}.
\VS{39}Et vos petits-enfants, dont vous avez dit qu'ils seront en proie, vos enfants, dis-je, qui aujourd'hui ne savent pas ce que c'est le bien ou le mal, eux y entreront, et je leur donnerai ce pays et ils le posséderont.
\VS{40}Mais vous, retournez vous-en en arrière, et allez dans le désert par le chemin de la Mer Rouge.
\VS{41}Et vous répondîtes et me dîtes~: Nous avons péché contre Yahweh, nous monterons et nous combattrons, comme Yahweh, notre Dieu, nous l'a ordonné. Et vous ceignîtes chacun vos armes de guerre, et vous entreprîtes hardiment de monter à la montagne.
\VS{42}Et Yahweh me dit~: Dis-leur~: Ne montez point et ne combattez point, car je ne suis point au milieu de vous~; afin que vous ne soyez point battus par vos ennemis.
\VS{43}Je vous parlai, mais vous ne m'écoutâtes point et vous vous rebellâtes contre l'ordre de Yahweh, et vous fûtes orgueilleux et vous montâtes sur la montagne.
\VS{44}Et les Amoréens, qui demeuraient sur cette montagne, sortirent contre vous et vous poursuivirent comme font les abeilles~; et ils vous battirent depuis Séir jusqu'à Horma.
\VS{45}Et étant retournés vous pleurâtes devant Yahweh~; mais Yahweh n'écouta point votre voix, et ne vous prêta point l'oreille.
\VS{46}Ainsi, vous demeurâtes à Kadès plusieurs jours, autant de temps que vous y aviez demeuré.
\Chap{2}
\TextTitle{Périple du peuple dans le désert}
\VerseOne{}Alors nous retournâmes en arrière, et nous partîmes pour le désert, par le chemin de la Mer Rouge, comme Yahweh me l'avait dit, et nous tournâmes autour de la montagne de Séir plusieurs jours.
\VS{2}Et Yahweh me parla, en disant~:
\VS{3}Vous avez assez tourné autour de cette montagne. Tournez-vous vers le nord.
\VS{4}Ordonne au peuple, en disant~: Vous allez passer la frontière de vos frères, les fils d'Esaü, qui demeurent en Séir. Ils auront peur de vous~; mais soyez bien sur vos gardes.
\VS{5}N'ayez pas de démêlé avec eux~; car je ne vous donnerai rien dans leur pays, pas même de quoi poser la plante du pied~: J'ai donné à Esaü la montagne de Séir en héritage.
\VS{6}Vous achèterez d'eux la nourriture à prix d'argent et vous en mangerez, et vous achèterez d'eux l'eau à prix d'argent et vous en boirez.
\VS{7}Car Yahweh, ton Dieu, t'a béni dans tout le travail de tes mains, il a connu ta marche dans ce grand désert. Yahweh, ton Dieu, a été avec toi pendant ces quarante années, et tu n'as manqué de rien.
\VS{8}Nous passâmes à distance de nos frères, les fils d'Esaü, qui demeuraient en Séir, à distance du chemin de la plaine, d'Elath et d'Etsjon-Guéber, et nous nous tournâmes, et nous passâmes par le chemin du désert de Moab.
\VS{9}Yahweh me dit~: N'assiège point Moab, et ne t'engage pas dans un combat avec lui~; car je ne te donnerai rien en héritage dans son pays~: J'ai donné Ar en héritage aux fils de Lot\FTNT{Ge. 19:36-38.}.
\VS{10}Les Emim y habitaient auparavant~; c'était un peuple grand, nombreux et de haute taille comme les Anakim.
\VS{11}Ils étaient considérés comme des Rephaïm, de même que les Anakim~; mais les Moabites les appelaient Emim.
\VS{12}Séir était habité autrefois par les Horiens~; mais les fils d'Esaü les en dépossédèrent, les détruisirent devant eux, et y habitèrent à leur place, comme l'a fait Israël dans le pays de son héritage que Yahweh lui a donné.
\VS{13}Mais maintenant, levez-vous, et passez le torrent de Zéred. Et nous passâmes le torrent de Zéred.
\VS{14}Or le temps que nous avons marché de Kadès-Barnéa, jusqu'à ce que nous ayons passé le torrent de Zéred, fut de trente-huit ans, jusqu'à ce que toute la génération des hommes de guerre eût été consumée du milieu du camp, comme Yahweh le leur avait juré.
\VS{15}La main de Yahweh fut aussi sur eux pour les détruire du milieu du camp, jusqu'à ce qu'ils eussent été consumés.
\VS{16}Or il est arrivé qu'après que tous les hommes de guerre eurent été consumés par la mort du milieu du peuple,
\VS{17}Yahweh me parla, et dit~:
\VS{18}Tu vas passer aujourd'hui la frontière de Moab, à savoir Har.
\VS{19}Tu t'approcheras en face des fils d'Ammon, mais ne les assiège point, et ne t'engage point dans un combat avec eux~; car je ne te donnerai rien en possession dans le pays des fils d'Ammon~: Je l'ai donné en héritage aux fils de Lot.
\VS{20} Ce pays était aussi considéré comme un pays de Rephaïm~; car les Rephaïm y habitaient auparavant, et les Ammonites les appelaient Zamzummim~;
\VS{21}c'était un peuple grand, nombreux, et de haute taille, comme les Anakim, Yahweh les détruisit devant eux, et ils les dépossédèrent, et habitèrent à leur place.
\VS{22}Comme il fit pour les fils d'Esaü qui demeurent en Séir, quand il détruisit les Horiens devant eux~; ils les dépossédèrent et habitèrent à leur place jusqu'à ce jour.
\VS{23}Or quant aux Avviens, qui demeuraient en Hatserim jusqu'à Gaza, ils furent détruits par les Caphtorim, sortis de Caphtor, qui demeurèrent à leur place.
\TextTitle{Yawheh livre Sihon, roi de Hesbon, entre les mains d'Israël}
\VS{24}Levez-vous, partez et passez le torrent de l'Arnon. Regarde, j'ai livré entre tes mains Sihon, roi de Hesbon, l'Amoréen, et son pays. Commence à en prendre possession, et fais-lui la guerre~!
\VS{25}Aujourd'hui, je vais commencer à mettre la frayeur et la crainte de toi sur les peuples qui sont sous les cieux~; et ayant entendu parler de toi, ils trembleront et seront dans l'angoisse à cause de ta présence.
\VS{26}J'envoyai, du désert de Kedémoth, des messagers à Sihon, roi de Hesbon, avec des paroles de paix, disant\FTNT{No. 21:21.}~:
\VS{27}Permets que je passe par ton pays~; et j'irai par le grand chemin, sans me détourner ni à droite ni à gauche.
\VS{28}Tu me vendras de la nourriture à prix d'argent, afin que je mange, et tu me donneras de l'eau à prix d'argent, afin que je boive~; seulement que j'y passe de mes pieds.
\VS{29}C'est ce qu'ont fait les fils d'Esaü qui demeurent en Séir, et les Moabites qui demeurent à Ar, jusqu'à ce que je passe le Jourdain pour entrer au pays que Yahweh, notre Dieu, nous donne.
\VS{30}Mais Sihon, roi de Hesbon, ne voulut point nous laisser passer par son pays~; car Yahweh, ton Dieu, avait endurci son esprit, et raidit son cœur afin de le livrer entre tes mains, comme tu le vois aujourd'hui.
\VS{31}Yahweh me dit~: Regarde, j'ai commencé à te livrer Sihon et son pays~; commence à posséder son pays, pour le tenir en héritage.
\VS{32}Sihon donc, sortit nous rencontrer avec tout son peuple pour nous combattre à Jahats.
\VS{33}Mais Yahweh, notre Dieu, nous le livra en face, et nous le battîmes, lui, ses fils, et tout son peuple.
\VS{34}Et en ce temps-là, nous prîmes toutes ses villes, et nous détruisîmes par le moyen de l'interdit les villes, les hommes, les femmes, et les petits enfants, sans laisser de survivants.
\VS{35}Seulement nous pillâmes les bêtes pour nous, et le butin des villes que nous avions prises.
\VS{36}Depuis Aroër, qui est sur le bord du torrent de l'Arnon, et la ville qui est dans la vallée, jusqu'à Galaad, il n'y eut pas une ville qui fût trop haute pour nous~: Yahweh, notre Dieu, nous livra tout.
\VS{37}Seulement tu n'approchas point du pays des fils d'Ammon, de tous les bords du torrent de Jabbok, des villes de la montagne, ni d'aucun lieu que Yahweh, notre Dieu, t'avait ordonné de ne point attaquer.
\Chap{3}
\TextTitle{Yawheh livre Og, roi de Basan, entre les mains d'Israël}
\VerseOne{}Alors nous nous tournâmes, et nous montâmes par le chemin de Basan. Et Og, roi de Basan, sortit nous rencontrer, avec tout son peuple, pour nous combattre à Edréi.
\VS{2}Et Yahweh me dit~: Ne le crains point~; car je le livre entre tes mains, lui, tout son peuple, et son pays~; et tu lui feras comme tu as fait à Sihon, roi des Amoréens, qui demeurait à Hesbon.
\VS{3}Ainsi Yahweh, notre Dieu, livra aussi entre nos mains Og, roi de Basan, avec tout son peuple~; nous le battîmes sans laisser de survivants.
\VS{4}En ce même temps, nous prîmes aussi toutes ses villes, et il n'y eut point de ville que nous ne lui prîmes pas~: Soixante villes, toute la contrée d'Argob, le royaume d'Og en Basan.
\VS{5}Toutes ces villes-là étaient fortifiées, avec de hautes murailles, des portes et des barres. Il y avait aussi des villes sans murailles en fort grand nombre.
\VS{6}Et nous les détruisîmes par le moyen de l'interdit, comme nous l'avions fait à Sihon, roi de Hesbon~; nous dévouâmes par le moyen de l'interdit toutes les villes, les hommes, les femmes, et les petits enfants.
\VS{7}Mais nous pillâmes pour nous toutes les bêtes et le butin des villes.
\VS{8}Nous prîmes donc, en ce temps-là, le pays de la main des deux rois des Amoréens, qui étaient de l'autre côté du Jourdain, depuis le torrent de l'Arnon jusqu'à la montagne de l'Hermon~;
\VS{9}or les Sidoniens donnent à l'Hermon le nom de Sirion, mais les Amoréens le nomment Senir~;
\VS{10}toutes les villes de la plaine, tout Galaad, et tout Basan jusqu'à Salca et Edréi, les villes du royaume d'Og en Basan.
\VS{11}Og, roi de Basan, avait survécu seul du reste des Rephaïm. Voici, son lit, un lit de fer, n'est-il pas dans Rabbath, ville des fils d'Ammon~? Sa longueur est de neuf coudées, et sa largeur de quatre coudées, en coudées d'homme.
\TextTitle{Premières terres attribuées à Ruben, Gad et à la demi-tribu de Manassé}
\VS{12}En ce temps-là donc, nous prîmes possession de ce pays. Je donnai aux Rubénites et aux Gadites le territoire à partir d'Aroër, sur le torrent de l'Arnon, et la moitié de la montagne de Galaad, avec ses villes\FTNT{Jos. 13:23-32.}.
\VS{13}Je donnai à la demi-tribu de Manassé le reste de Galaad et tout le royaume d'Og, en Basan~: Toute la contrée d'Argob avec tout le Basan, c'est ce qu'on appelait le pays des Réphaïm.
\VS{14}Jaïr, fils de Manassé, prit toute la contrée d'Argob jusqu'à la frontière des Gueschuriens et des Maacathiens, et il donna son nom à Basan, appelé villages de Jaïr jusqu'à aujourd'hui.
\VS{15}Je donnai aussi Galaad à Makir.
\VS{16}Mais aux Rubénites et aux Gadites, je donnai de Galaad jusqu'au torrent de l'Arnon, dont le milieu du torrent sert de frontière, et jusqu'au torrent de Jabbok, frontière des fils d'Ammon~;
\VS{17}la plaine, et le Jourdain, de la frontière de Kinnéreth jusqu'à la mer de la plaine, la Mer Salée, aux pieds de Pisga vers l'orient.
\VS{18}Or en ce temps-là, je vous ordonnai, en disant~: Yahweh, votre Dieu, vous donne ce pays pour le posséder. Vous tous, qui êtes vaillants, vous passerez armés devant vos frères, les fils d'Israël.
\VS{19}Seulement vos femmes, vos petits-enfants, et vos troupeaux, car je sais que vous avez beaucoup de troupeaux, resteront dans les villes que je vous ai données,
\VS{20}jusqu'à ce que Yahweh ait accordé du repos à vos frères comme à vous, et qu'eux aussi possèdent le pays que Yahweh, votre Dieu, leur donne de l'autre côté du Jourdain. Puis vous retournerez chacun dans l'héritage que je vous ai donné.
\VS{21}En ce temps-là, j'ordonnai à Josué, en disant~: Tes yeux ont vu tout ce que Yahweh, votre Dieu, a fait à ces deux rois~: Yahweh en fera de même à tous les royaumes vers lesquels tu vas passer.
\VS{22}Ne les craignez point~; car Yahweh, votre Dieu, combattra lui-même pour vous.
\TextTitle{Moïse n'entrera pas dans la terre promise}
\VS{23}En ce même temps, j'implorai la grâce de Yahweh, en disant~:
\VS{24}Seigneur Yahweh, tu as commencé à montrer à ton serviteur ta grandeur et ta main puissante~; car quel est le dieu dans le ciel et sur la terre qui puisse faire selon tes œuvres et selon ta puissance~?
\VS{25}Que je passe, je te prie, et que je voie ce bon pays de l'autre côté du Jourdain, ces bonnes montagnes et le Liban.
\VS{26}Mais Yahweh s'irrita contre moi, à cause de vous, et ne m'écouta point. Yahweh me dit~: C'est assez, ne me parle plus de cette affaire.
\VS{27}Monte au sommet du Pisga, et lève tes yeux à l'occident, au nord, au sud, et à l'orient, et regarde de tes yeux~; car tu ne passeras point ce Jourdain.
\VS{28}Donnes-en la charge à Josué, fortifie-le et affermis-le~; car c'est lui qui passera devant ce peuple et qui le mettra en possession du pays que tu verras.
\VS{29}Ainsi nous demeurâmes dans la vallée, vis-à-vis de Beth-Peor.
\Chap{4}
\TextTitle{Encouragement à garder la loi de Yahweh}
\VerseOne{}Et maintenant Israël, écoute les lois et les ordonnances que je vous enseigne, pour les pratiquer afin que vous viviez, que vous entriez et possédiez le pays que Yahweh, le Dieu de vos pères, vous donne.
\VS{2}Vous n'ajouterez\FTNT{De. 12:32~; Pr. 30:6~; Ap. 22:18-19.} rien à la parole que je vous ordonne, et vous n'en retrancherez rien~; afin de garder les commandements de Yahweh, votre Dieu, que je vous ordonne.
\VS{3}Vos yeux ont vu ce que Yahweh a fait à cause de Baal-Peor~: Yahweh, ton Dieu, a détruit du milieu de toi tout homme qui était allé après Baal-Peor\FTNT{No. 25:4-9.}.
\VS{4}Mais vous, qui vous êtes attachés à Yahweh, votre Dieu, vous êtes tous vivants aujourd'hui.
\VS{5}Regardez, je vous ai enseigné des lois et des ordonnances, comme Yahweh, mon Dieu, me l'a ordonné, afin que vous les pratiquiez au milieu du pays où vous allez pour le posséder.
\VS{6}Vous les garderez et vous les pratiquerez, car c'est là votre sagesse et votre intelligence aux yeux de tous les peuples, qui entendront ces lois, et qui diront~: Cette grande nation est un peuple sage et intelligent~!
\TextTitle{Israël, privilégié parmi tous les peuples}
\VS{7}Car quelle est la grande nation qui ait ses dieux proches d'elle, comme nous avons Yahweh, notre Dieu, toutes les fois que nous l'invoquons~?
\VS{8}Et quelle est la grande nation qui ait des lois et des ordonnances justes, comme toute cette loi que je mets aujourd'hui devant vous~?
\VS{9}Seulement, prends garde à toi et garde soigneusement ton âme, afin que tous les jours de ta vie tu n'oublies point les choses que tes yeux ont vues, et qu'elles ne sortent de ton cœur\FTNT{Pr. 4:23.}~; enseigne-les à tes fils, et aux fils de tes fils.
\VS{10}Rappelle-toi du jour où tu te tins face à Yahweh, ton Dieu, à Horeb, après que Yahweh me dit~: Convoque le peuple~! Je veux leur faire entendre mes paroles, pour qu'ils apprennent à me craindre tout le temps qu'ils seront vivants sur la terre~; et pour qu'ils les enseignent à leurs fils.
\VS{11}Et que vous vous approchâtes, et vous vous tîntes au pied de la montagne. Or la montagne était embrasée de feu jusqu'au milieu du ciel. Il y avait des ténèbres, une nuée, et une obscurité.
\VS{12}Et Yahweh vous parla du milieu du feu~; vous entendîtes le son de ses paroles, mais vous ne vîtes aucune image, vous entendîtes seulement la voix\FTNT{Ex. 19:17-19.}.
\VS{13}Et il déclara son alliance, qu'il vous ordonna d'observer, les dix paroles, qu'il écrivit sur deux tables de pierre.
\VS{14}Yahweh m'ordonna aussi, en ce temps-là, de vous enseigner les lois et les ordonnances, afin que vous les pratiquiez dans le pays que vous allez posséder.
\VS{15}Prenez bien garde à vos âmes, puisque vous n'avez vu aucune image le jour où Yahweh, votre Dieu, vous parla du milieu du feu à Horeb,
\VS{16}de peur que vous ne vous corrompiez et que vous ne vous fassiez une image taillée, une représentation d'idole ayant la forme d'un mâle ou d'une femelle,
\VS{17}ou la forme d'un animal qui soit sur la terre, ou la forme d'un oiseau ailé qui vole dans les cieux,
\VS{18}ou la forme d'un animal qui rampe sur la terre, ou la forme d'un poisson qui soit dans les eaux au-dessous de la terre.
\VS{19}De peur aussi qu'élevant tes yeux vers les cieux, et voyant le soleil, la lune, et les étoiles, toute l'armée des cieux, tu ne sois poussé à te prosterner devant elles, et que tu ne les serves~: C'est ce que Yahweh, ton Dieu, a donné en partage à tous les peuples, sous tous les cieux.
\VS{20}Mais vous, Yahweh vous a pris, et vous a fait sortir d'Egypte, du fourneau de fer, afin que vous fussiez un peuple de son héritage, comme vous l'êtes aujourd'hui.
\TextTitle{Conséquences de la désobéissance et de l'idolâtrie}
\VS{21}Or Yahweh s'irrita contre moi, à cause de vos paroles, et il jura que je ne passerais point le Jourdain, et que je n'entrerais point dans ce bon pays que Yahweh, ton Dieu, te donne en héritage.
\VS{22}Je mourrai dans ce pays-ci, je ne passerai point le Jourdain~; mais vous le passerez, et vous posséderez ce bon pays.
\VS{23}Gardez-vous d'oublier l'alliance de Yahweh, votre Dieu, qu'il a traitée avec vous, et que vous ne vous fassiez d'image taillée, de représentation quelconque, que Yahweh, votre Dieu, vous a défendues.
\VS{24}Car Yahweh, ton Dieu, est un feu dévorant\FTNT{Hé. 12:29.}, un Dieu jaloux.
\VS{25}Quand tu auras engendré des fils, et des fils de tes fils, et que vous serez depuis longtemps dans le pays, si vous vous corrompez, et que vous faites des images taillées, ou des représentations de quelque chose que ce soit, si vous faites ce qui est mal aux yeux de Yahweh, votre Dieu, afin de l'irriter
\VS{26}j'appelle aujourd'hui à témoin les cieux et la terre contre vous, certainement vous périrez promptement dans ce pays que vous allez posséder au-delà du Jourdain, vous n'y prolongerez point vos jours, car vous serez entièrement détruits.
\VS{27}Yahweh vous dispersera parmi les peuples, et vous ne resterez qu'un petit nombre parmi les nations, chez lesquelles Yahweh vous emmènera.
\VS{28}Et là, vous servirez des dieux, œuvres de main d'homme, du bois et de la pierre, qui ne peuvent voir, ni entendre, ni manger, ni sentir\FTNT{Es. 44:9~; Es. 46:7~; Ps. 115:4-7}.
\TextTitle{Yahweh, puissant, miséricordieux et fidèle à son alliance}
\VS{29}Mais de là, tu chercheras Yahweh, ton Dieu, et tu le trouveras, si tu le cherches de tout ton cœur et de toute ton âme.
\VS{30}Quand tu seras dans la détresse, et que toutes ces choses te seront arrivées, alors, dans les derniers jours, tu retourneras à Yahweh, ton Dieu, et tu obéiras à sa voix~;
\VS{31}parce que Yahweh, ton Dieu, est le Dieu puissant et miséricordieux, il ne t'abandonnera point et ne te détruira point, il n'oubliera point l'alliance de tes pères qu'il leur a jurée.
\VS{32}Interroge les premiers temps, qui ont été avant toi, depuis le jour que Dieu créa l'homme sur la terre, et d'une extrémité des cieux à l'autre, s'il a jamais été rien fait de semblable à cette grande chose, et s'il a été jamais rien entendu de semblable.
\VS{33}Est-ce qu'un peuple a entendu la voix de Dieu parlant du milieu du feu, comme tu l'as entendue, et qui soit demeuré en vie~?
\VS{34}Où Dieu a-t-il essayé de venir prendre pour lui une nation du milieu d'une nation, par des épreuves, des signes, des miracles, et des batailles, à main forte, et à bras étendu, et par des choses grandes et terribles, comme tout ce que Yahweh, notre Dieu, a fait pour vous en Egypte, sous vos yeux~?
\VS{35}Cela t'a été montré afin que tu reconnaisses que Yahweh est Dieu et qu'il n'y en a point d'autre.
\VS{36}Il t'a fait entendre sa voix des cieux pour t'instruire~; et il t'a montré son grand feu sur la terre, et tu as entendu ses paroles du milieu du feu.
\VS{37}Et parce qu'il a aimé tes pères, il a choisi leur postérité après eux et il t'a retiré d'Egypte en sa présence, par sa grande puissance~;
\VS{38}pour chasser de devant toi des nations plus grandes et plus puissantes que toi, pour te faire entrer dans leur pays, et pour te le donner en héritage, comme tu le vois aujourd'hui.
\VS{39}Sache donc aujourd'hui, et rappelle dans ton cœur que Yahweh est Dieu, en haut dans les cieux et sur la terre, et qu'il n'y en a point d'autre.
\VS{40}Garde donc ses lois et ses commandements que je t'ordonne aujourd'hui, afin que tu sois heureux, toi et tes fils après toi, et que tu prolonges tes jours sur la terre que Yahweh, ton Dieu, te donne\FTNT{Ex 20.}.
\TextTitle{Trois villes de refuge à l'est du Jourdain}
\VS{41}Alors Moïse sépara trois villes de l'autre côté du Jourdain vers le soleil levant,
\VS{42}afin que le meurtrier qui aurait tué son prochain involontairement, sans l'avoir haï auparavant, s'y enfuie~; et qu'en s'enfuyant dans l'une de ces villes-là, il eût sa vie sauve.
\VS{43}C'étaient~: Betser dans le désert, dans la plaine du pays, chez les Rubénites~; Ramoth en Galaad, chez les Gadites~; Golan en Basan, chez les Manassites.
\VS{44}C'est ici la loi que Moïse plaça face aux enfants d'Israël.
\VS{45}Voici les témoignages, les lois, et les ordonnances que Moïse déclara aux enfants d'Israël, après qu'ils furent sortis d'Egypte.
\VS{46}C'était de l'autre côté du Jourdain, dans la vallée, vis-à-vis de Beth-Peor, au pays de Sihon, roi des Amoréens, qui demeurait à Hesbon, et qui fut battu par Moïse et les enfants d'Israël après être sortis d'Egypte.
\VS{47}Et ils s'emparèrent de son pays avec le pays d'Og, roi de Basan, deux rois des Amoréens qui étaient de l'autre côté du Jourdain, vers le soleil levant.
\VS{48}Depuis Aroër, sur le bord du torrent de l'Arnon, jusqu'à la montagne de Sion, qui est l'Hermon,
\VS{49}et toute la plaine de l'autre côté du Jourdain vers l'orient, jusqu'à la mer de la plaine, au pied du Pisga.
\Chap{5}
\TextTitle{L'alliance établie à Horeb rappelée à la nouvelle génération}
\VerseOne{}Moïse appela tout Israël, et leur dit~: Ecoute, Israël, les lois et les ordonnances que je prononce aujourd'hui à vos oreilles, apprenez-les, et veillez à les mettre en pratique.
\VS{2}Yahweh, notre Dieu, a traité avec nous une alliance en Horeb\FTNT{Ex. 19:5.}.
\VS{3}Dieu n'a point traité cette alliance avec nos pères, mais avec nous, qui sommes ici aujourd'hui tous vivants.
\VS{4}Yahweh vous parla face à face sur la montagne du milieu du feu.
\VS{5}Je me tenais en ce temps-là entre Yahweh et vous, pour vous rapporter la parole de Yahweh~; parce que vous aviez peur face à ce feu, et vous ne montâtes point sur la montagne. Il dit\FTNT{Les dix paroles (Ex. 20).}~:
\VS{6}Je suis Yahweh, ton Dieu, qui t'ai fait sortir du pays d'Egypte, de la maison de servitude.
\VS{7}Tu n'auras point d'autres dieux devant ma face.
\VS{8}Tu ne te feras point d'image taillée, ni de représentation des choses qui sont en haut dans les cieux, ni sur la terre, ni dans les eaux sous la terre.
\VS{9}Tu ne te prosterneras point devant elles, et tu ne les serviras point~; car je suis Yahweh, ton Dieu, un Dieu jaloux, qui punis l'iniquité des pères sur les enfants jusqu'à la troisième et à la quatrième génération de ceux qui me haïssent,
\VS{10}et qui fais miséricorde jusqu'à mille générations à ceux qui m'aiment et qui gardent mes commandements.
\VS{11}Tu ne prendras point le Nom de Yahweh, ton Dieu, en vain~; car Yahweh ne tiendra pas pour innocent celui qui prendra son Nom en vain.
\VS{12}Garde le jour du sabbat pour le sanctifier, comme Yahweh, ton Dieu, te l'a ordonné.
\VS{13}Tu travailleras six jours, et tu feras toute ton œuvre,
\VS{14}mais le septième jour est le sabbat de Yahweh, ton Dieu~: Tu ne feras aucune œuvre, ni ton fils, ni ta fille, ni ton serviteur, ni ta servante, ni ton bœuf, ni ton âne, ni aucune de tes bêtes, ni l'étranger qui est dans tes portes, afin que ton serviteur et ta servante se reposent comme toi.
\VS{15}Et tu te souviendras que tu as été esclave au pays d'Egypte, et que Yahweh, ton Dieu, t'en a fait sortir à main forte et à bras étendu~: C'est pourquoi Yahweh, ton Dieu, t'a ordonné d'observer le jour du sabbat.
\VS{16}Honore ton père et ta mère, comme Yahweh, ton Dieu, te l'a ordonné, afin que tes jours se prolongent et que tu sois heureux sur la terre que Yahweh, ton Dieu, te donne.
\VS{17}Tu ne tueras point.
\VS{18}Tu ne commettras point d'adultère.
\VS{19}Tu ne déroberas point.
\VS{20}Tu ne diras point de faux témoignage contre ton prochain.
\VS{21}Tu ne convoiteras point la femme de ton prochain~; tu ne désireras point la maison de ton prochain, ni son champ, ni son serviteur, ni sa servante, ni son bœuf, ni son âne, ni aucune chose qui soit à ton prochain.
\VS{22}Yahweh déclara ces paroles à toute votre assemblée sur la montagne, du milieu du feu, des nuées et de l'obscurité, à voix forte sans rien ajouter. Il les écrivit sur deux tables de pierre qu'il me donna.
\TextTitle{Moïse, intermédiaire entre Yahweh et le peuple}
\VS{23}Or il arriva qu'aussitôt que vous eûtes entendu la voix du milieu de l'obscurité, parce que la montagne était embrasée par le feu, vos chefs de tribus et vos anciens s'approchèrent de moi,
\VS{24}et vous dîtes~: Voici, Yahweh, notre Dieu, nous a fait voir sa gloire et sa grandeur, et nous avons entendu sa voix du milieu du feu~; aujourd'hui, nous avons vu que Dieu a parlé avec l'homme, et qu'il est resté en vie.
\VS{25}Et maintenant pourquoi mourrions-nous~? Car ce grand feu là nous dévorera~; si nous entendons encore la voix de Yahweh, notre Dieu, nous mourrons.
\VS{26}Car qui, de toute chair, a entendu comme nous la voix du Dieu vivant parlant du milieu du feu, et qui soit resté en vie~?
\VS{27}Approche-toi et écoute tout ce que Yahweh, notre Dieu, dira~; puis tu nous diras tout ce que Yahweh, notre Dieu, t'aura dit~; nous l'entendrons, et nous le ferons.
\VS{28}Yahweh entendit la voix de vos paroles pendant que vous me parliez. Et Yahweh me dit~: J'ai entendu les paroles que ce peuple t'ont adressées~: Tout ce qu'ils ont dit est bien.
\VS{29}Ô~! S'ils avaient toujours ce même cœur pour me craindre et pour garder tous mes commandements, afin qu'ils fussent heureux, eux et leurs enfants, pour toujours~!
\VS{30}Va, dis-leur~: Retournez dans vos tentes.
\VS{31}Mais toi, reste ici avec moi, et je te dirai tous les commandements, les lois, et les ordonnances que tu leur enseigneras, afin qu'ils les pratiquent dans le pays que je leur donne en possession.
\VS{32}Vous prendrez donc garde de faire ce que Yahweh, votre Dieu, vous a ordonné~; vous ne vous en détournerez ni à droite ni à gauche.
\VS{33}Vous marcherez dans toute la voie que Yahweh, votre Dieu, vous a ordonnée, afin que vous viviez et que vous soyez heureux, et que vous prolongiez vos jours sur la terre que vous posséderez.
\Chap{6}
\TextTitle{Obéissance à la loi, source de bénédictions}
\VerseOne{}Voici les commandements, les lois et les ordonnances que Yahweh, votre Dieu, m'a ordonné de vous enseigner, afin que vous les pratiquiez dans le pays dans lequel vous allez passer pour le posséder~;
\VS{2}afin que tu craignes Yahweh, ton Dieu, en gardant durant tous les jours de ta vie, toi, ton fils, et le fils de ton fils, toutes ses lois et ses commandements que je t'ordonne, pour que tes jours soient prolongés.
\VS{3}Tu les écouteras donc, ô Israël, et tu auras soin de les mettre en pratique, afin que tu sois heureux, et que vous vous multipliiez sur la terre où coulent le lait et le miel, comme Yahweh, le Dieu de tes pères, l'a dit\FTNT{Ex. 3:8.}.
\VS{4}Ecoute Israël~! Yahweh, notre Dieu, Yahweh est Un\FTNT{Jacob fut le premier à faire cette prière qui affirme l'unicité de Dieu. Dieu est UN (en hébreu «~Echad~» ou «~Ehad~»). Loin de l'infirmer ou de la contredire, Jésus a confirmé cette prière et l'enseignement capital qu'elle contient (Mc. 12:29). Dieu n'est pas trois personnes en une, mais UN. Cette parole annonce un monothéisme absolu. Elle s'oppose catégoriquement au polythéisme des Cananéens qui adoraient de multiples dieux, les étoiles, la lune, le soleil, les arbres, les rois etc. Aussi les Hébreux avaient reçu l'ordre de la part de Yahweh de détruire toutes les idoles qu'ils trouveraient en la terre promise (De. 16:21). Voir également commentaire en Ge. 1:5.}.
\VS{5}Tu aimeras donc Yahweh, ton Dieu, de tout ton cœur, de toute ton âme, et de toute ta force\FTNT{Mt. 22:37~; Mc. 12:30.}.
\TextTitle{La loi de Yahweh doit être enseignée aux enfants}
\VS{6}Et ces paroles, que je t'ordonne aujourd'hui, seront dans ton cœur.
\VS{7}Tu les enseigneras soigneusement à tes enfants, et tu en parleras quand tu te tiendras dans ta maison, quand tu iras en voyage, quand tu te coucheras et quand tu te lèveras.
\VS{8}Et tu les lieras comme un signe sur tes mains, et elles seront comme des fronteaux entre tes yeux.
\VS{9}Tu les écriras aussi sur les poteaux de ta maison et sur tes portes.
\VS{10}Yahweh, ton Dieu, te fera entrer dans le pays qu'il a juré à tes pères, Abraham, Isaac, et Jacob, de te donner. Tu posséderas de grandes et bonnes villes que tu n'as point bâties,
\VS{11}des maisons pleines de toutes sortes de biens que tu n'as point remplies, des puits creusés que tu n'as point creusés, des vignes et des oliviers que tu n'as point plantés, tu mangeras, et tu te rassasieras.
\VS{12}Prends garde à toi, de peur que tu n'oublies Yahweh, qui t'a fait sortir du pays d'Egypte, de la maison de servitude.
\VS{13}Tu craindras Yahweh, ton Dieu, tu le serviras et tu jureras par son Nom.
\VS{14}Vous n'irez point après d'autres dieux, d'entre les dieux des peuples qui sont autour de vous~;
\VS{15}car Yahweh, ton Dieu, est un Dieu jaloux au milieu de toi~; de peur que la colère de Yahweh, ton Dieu, ne s'enflamme contre toi, et qu'il ne t'extermine de dessus la terre.
\VS{16}Vous ne tenterez point Yahweh, votre Dieu, comme vous l'avez tenté à Massa.
\VS{17}Vous garderez soigneusement les commandements de Yahweh, votre Dieu, ses ordonnances et ses lois qu'il vous a ordonnées.
\VS{18}Tu feras ce qui est droit et bon aux yeux de Yahweh, afin que tu sois heureux, que tu entres et que tu possèdes le bon pays que Yahweh a juré à tes pères,
\VS{19}après qu'il aura chassé tous tes ennemis de devant toi, comme Yahweh l'a dit.
\VS{20}Quand ton enfant t'interrogera à l'avenir, en disant~: Que veulent dire ces préceptes, ces lois, et ces ordonnances que Yahweh, notre Dieu, vous a ordonnés~?
\VS{21}Tu diras à ton enfant~: Nous étions esclaves de Pharaon en Egypte, et Yahweh nous a fait sortir de l'Egypte par sa main puissante.
\VS{22}Yahweh a fait sous nos yeux des signes et des miracles, grands et désastreux contre l'Egypte, contre Pharaon et contre toute sa maison~;
\VS{23}et il nous a fait sortir de là pour nous conduire dans le pays qu'il avait juré à nos pères de nous donner.
\VS{24}Yahweh nous a ordonné de pratiquer toutes ces lois, et de craindre Yahweh, notre Dieu, afin que nous soyons toujours heureux, et qu'il préserve notre vie, comme aujourd'hui.
\VS{25}Et ceci sera notre justice, que nous prenions garde de pratiquer tous ces commandements devant Yahweh, notre Dieu, comme il nous l'a ordonné.
\Chap{7}
\TextTitle{Yahweh interdit les alliances avec les peuples païens}
\VerseOne{}Quand Yahweh, ton Dieu, t'aura fait entrer dans le pays où tu vas entrer pour le posséder, et qu'il aura chassé de devant toi beaucoup de nations~: Les Héthiens, les Guirgasiens, les Amoréens, les Cananéens, les Phéréziens, les Héviens, et les Jébusiens, sept nations plus grandes et plus puissantes que toi~;
\VS{2}et que Yahweh, ton Dieu, te les aura livrées en face et que tu les auras battues, tu les dévoueras complètement à la façon de l'interdit, tu ne traiteras point d'alliance avec elles, et tu ne leur feras point de grâce.
\VS{3}Tu ne t'allieras point par mariage avec elles, tu ne donneras point tes filles à leurs fils, et tu ne prendras point leurs filles pour tes fils\FTNT{Jos. 23:12-13.}~;
\VS{4}car elles détourneraient de moi tes fils, et ils serviraient d'autres dieux, et la colère de Yahweh s'enflammerait contre vous~: Il te détruirait promptement.
\VS{5}Mais vous les traiterez de cette manière~: Vous renverserez leurs autels, vous briserez leurs statues, vous abattrez leurs Asherah\FTNT{Le mot idole vient de l'hébreu «~Asherah~». Il est cité au moins quarante fois dans le Tanakh. Il fait référence à un objet en bois utilisé dans le culte d'Astarté, l'épouse de Baal. Voir De. 19:21.}, et vous brûlerez au feu leurs images taillées.
\VS{6}Car tu es un peuple saint pour Yahweh, ton Dieu. Yahweh, ton Dieu, t'a choisi pour que tu sois pour lui un peuple précieux entre tous les peuples qui sont sur la face de la terre.
\VS{7}Ce n'est pas parce que vous êtes plus nombreux que tous les peuples que Yahweh vous a aimé et qu'il vous a choisis~; car vous êtes le plus petit de tous les peuples.
\VS{8}Mais c'est parce que Yahweh vous aime, et qu'il garde le serment qu'il a juré à vos pères, Yahweh vous a fait sortir par sa main puissante, et vous a rachetés de la maison de servitude, de la main de Pharaon, roi d'Egypte.
\VS{9}Sache que c'est Yahweh, ton Dieu, qui est Dieu. Ce Dieu fidèle garde son alliance et sa miséricorde jusqu'à mille générations envers ceux qui l'aiment et qui gardent ses commandements,
\VS{10}et qui rend la pareille en face à ceux qui le haïssent, et les fait périr~; il ne diffère point envers celui qui le hait, il lui rend la pareille en face.
\VS{11}Garde les commandements, les lois, et les ordonnances que je t'ordonne aujourd'hui, et mets-les en pratique.
\TextTitle{L'obéissance à Yahweh, source de bénédictions et de victoires}
\VS{12}Et il arrivera que si vous écoutez ces ordonnances, si vous les gardez et les mettez en pratique, Yahweh, ton Dieu, gardera l'alliance et la bonté qu'il a jurées à tes pères.
\VS{13}Et il t'aimera, te bénira, et te multipliera~; il bénira le fruit de tes entrailles, et le fruit de ta terre, ton blé, ton vin, et ton huile, les portées de ton gros et de ton menu bétail, sur la terre qu'il a juré de donner à tes pères.
\VS{14}Tu seras béni plus que tous les peuples~; il n'y aura chez toi et parmi tes bêtes, ni mâle ni femelle stérile\FTNT{Ex. 23:26.}.
\VS{15}Yahweh détournera de toi toute maladie~; il ne t'enverra aucun de ces mauvais maux d'Egypte qui te sont connus, mais il les fera venir sur tous ceux qui te haïssent.
\VS{16}Tu détruiras donc tous les peuples que Yahweh, ton Dieu, va te livrer, ton œil n'aura point de pitié, et tu ne serviras point leurs dieux, car cela te serait un piège.
\VS{17}Si tu dis dans ton cœur~: Ces nations sont plus nombreuses que moi, comment pourrai-je les déposséder~?
\VS{18}Ne les crains point. Rappelle-toi bien ce que Yahweh, ton Dieu, a fait à Pharaon, et à tous les Egyptiens,
\VS{19}de ces grandes épreuves que tes yeux ont vues, les signes et les miracles, la main forte et le bras étendu par lesquels Yahweh, ton Dieu, t'a fait sortir~; ainsi fera Yahweh, ton Dieu, à tous ces peuples que tu crains.
\VS{20}Yahweh, ton Dieu, enverra contre eux les frelons, jusqu'à ce que périssent ceux qui resteront, et ceux qui se seront cachés de devant toi.
\VS{21}Ne t'effraie point devant eux, car Yahweh, ton Dieu, le Dieu grand et terrible est au milieu de toi.
\VS{22}Or Yahweh, ton Dieu, chassera peu à peu ces nations de devant toi~; tu ne pourras pas les exterminer promptement, de peur que les bêtes des champs ne se multiplient contre toi.
\VS{23}Mais Yahweh, ton Dieu, les livrera devant toi~; et il les troublera par de grandes confusions, jusqu'à ce qu'elles soient détruites.
\VS{24}Et il livrera leurs rois entre tes mains, et tu feras disparaître leurs noms de dessous les cieux~; aucun homme ne tiendra face à toi, jusqu'à ce que tu les aies détruits.
\VS{25}Tu brûleras au feu les images taillées de leurs dieux. Tu ne convoiteras point et tu ne prendras point pour toi l'argent et l'or qui seront sur elles, de peur que tu en sois pris au piège~; car c'est une abomination pour Yahweh, ton Dieu.
\VS{26}Ainsi tu n'introduiras point de choses abominables dans ta maison, afin que tu ne sois pas, comme cette chose, dévoué par interdit~; tu la détesteras fortement, et tu l'auras en abomination, car c'est une chose dévouée par interdit.
\Chap{8}
\TextTitle{Le désert, lieu de formation, d'humiliation et d’épreuve}
\VerseOne{}Vous observerez et vous mettrez en pratique tous les commandements que je vous ordonne aujourd'hui, afin que vous viviez, que vous multipliiez, et que vous entriez en possession du pays que Yahweh a juré de donner à vos pères.
\VS{2}Et souviens-toi de tout le chemin par lequel Yahweh, ton Dieu, t'a fait marcher pendant ces quarante ans dans ce désert, afin de t'humilier et de t'éprouver, pour connaître ce qui était dans ton cœur, et si tu garderais ses commandements ou non.
\VS{3}Il t'a donc humilié, il t'a laissé avoir faim, mais il t'a nourri de la manne, que tu ne connaissais pas et que tes pères n'avaient pas connue, afin de te faire connaître que l'homme ne vivra pas de pain seulement, mais que l'homme vivra de tout ce qui sort de la bouche de Yahweh\FTNT{Mt. 4:4~; Lu. 4:4.}.
\VS{4}Ton vêtement ne s'est point usé sur toi, et ton pied ne s'est point enflé durant ces quarante années\FTNT{Né. 9:21.}.
\VS{5}Reconnais dans ton cœur que Yahweh, ton Dieu, te châtie comme un homme châtie son enfant\FTNT{Hé. 12:5-12.}.
\TextTitle{Se garder d'oublier Yahweh}
\VS{6}Et garde les commandements de Yahweh, ton Dieu, pour marcher dans ses voies, et pour le craindre.
\VS{7}Car Yahweh, ton Dieu, va te faire entrer dans un bon pays, un pays de torrents d'eaux, de fontaines et d'abîmes, qui jaillissent des vallées et des montagnes~;
\VS{8}un pays de blé, d'orge, de vignes, de figuiers, et de grenadiers~; un pays d'oliviers donnant de l'huile et du miel~;
\VS{9}un pays où tu ne mangeras point le pain avec disette, où tu ne manqueras de rien~; un pays dont les pierres sont du fer, et des montagnes desquelles tu tailleras l'airain.
\VS{10}Tu mangeras et tu te rassasieras, tu béniras Yahweh, ton Dieu, pour le bon pays qu'il t'a donné.
\VS{11}Prends garde à toi de peur que tu n'oublies Yahweh, ton Dieu, en ne gardant point ses commandements, ses ordonnances, et ses lois que je t'ordonne aujourd'hui~;
\VS{12}de peur que quand tu mangeras et que tu seras rassasié~; que tu bâtiras et habiteras de belles maisons~;
\VS{13}que ton gros et menu bétail se multipliera~; que ton argent et ton or augmentera, et que tout ce qui est à toi se multipliera,
\VS{14}que ton cœur ne s'élève point et que tu n'oublies point Yahweh, ton Dieu, qui t'a fait sortir du pays d'Egypte, de la maison de servitude,
\VS{15}qui t'a fait marcher dans ce grand et affreux désert de serpents brûlants et de scorpions, dans des lieux arides et sans eau, et qui a fait jaillir pour toi de l'eau du rocher le plus dur,
\VS{16}qui t'as fait manger dans ce désert la manne que tes pères n'avaient point connue, afin de t'humilier et de t'éprouver, pour te faire ensuite du bien,
\VS{17}et que tu ne dises dans ton cœur~: Ma force et la puissance de ma main m'ont acquis ces richesses.
\VS{18}Mais tu te souviendra de Yahweh, ton Dieu, car c'est lui qui te donne de la force pour acquérir ces richesses, afin de confirmer son alliance, qu'il a jurée à tes pères, comme tu le vois aujourd'hui.
\VS{19}Mais si tu oublies Yahweh, ton Dieu, et que tu vas après d'autres dieux, si tu les sers, et que tu te prosternes devant eux, je vous avertis aujourd'hui que vous périrez certainement.
\VS{20}Vous périrez comme les nations que Yahweh fait périr devant vous, parce que vous n'aurez pas obéi à la voix de Yahweh, votre Dieu.
\Chap{9}
\TextTitle{Yahweh, fidèle à son alliance malgré la rébellion du peuple}
\VerseOne{}Ecoute, Israël~! Tu vas passer aujourd'hui le Jourdain, pour aller posséder des nations plus grandes et plus puissantes que toi, des villes grandes et fortifiées jusqu'au ciel,
\VS{2}un peuple grand et de haute taille, les fils d'Anak, que tu connais, et dont tu as entendu dire~: Qui tiendra face aux fils d'Anak~?
\VS{3}Sache donc aujourd'hui que Yahweh, ton Dieu, passera devant toi, comme un feu dévorant, c'est lui qui les détruira, qui les humiliera devant toi~; tu les chasseras, et tu les feras périr promptement, comme Yahweh te l'a dit.
\VS{4}Ne parle pas en ton cœur, quand Yahweh, ton Dieu, les chassera de devant toi, en disant~: C'est à cause de ma justice que Yahweh me fait entrer en possession de ce pays. Car c'est à cause de la méchanceté de ces nations-là que Yahweh les chasse devant toi.
\VS{5}Ce n'est point pour ta justice ni pour la droiture de ton cœur que tu entres en possession de leur pays, mais c'est pour la méchanceté de ces nations-là que Yahweh, ton Dieu, les chasse de devant toi, et pour confirmer la parole que Yahweh a jurée à tes pères, Abraham, Isaac, et Jacob.
\VS{6}Sache donc que ce n'est point pour ta justice que Yahweh, ton Dieu, te donne ce bon pays pour que tu le possèdes~; car tu es un peuple au cou raide.
\VS{7}Souviens-toi, n'oublie pas que tu as excité la colère de Yahweh, ton Dieu, dans le désert. Depuis le jour où tu es sorti du pays d'Egypte jusqu'à ce que vous arriviez dans ce lieu, vous avez été rebelles contre Yahweh.
\VS{8}Même à Horeb, vous avez excité la colère de Yahweh~; et Yahweh s'irrita contre vous, pour vous détruire.
\VS{9}Quand je montai sur la montagne, pour prendre les tables de pierre, les tables de l'alliance que Yahweh a traitée avec vous, je demeurai sur la montagne quarante jours et quarante nuits, sans manger de pain et sans boire d'eau~;
\VS{10}et Yahweh me donna les deux tables de pierre écrites du doigt de Dieu, et contenant toutes les paroles que Yahweh avait déclarées sur la montagne, du milieu du feu, le jour de l'assemblée.
\VS{11}Et il arriva qu'au bout de quarante jours et quarante nuits, Yahweh me donna les deux tables de pierre, qui sont les tables de l'alliance.
\VS{12}Puis Yahweh me dit~: Lève-toi, descends promptement d'ici~; car ton peuple, que tu as fait sortir d'Egypte, s'est corrompu. Ils se sont détournés promptement de la voie que je leur avais ordonnée, ils se sont fait une image en métal fondu.
\VS{13}Yahweh me parla, en disant~: Je vois que ce peuple est un peuple au cou raide.
\VS{14}Laisse-moi les détruire et effacer leur nom de dessous les cieux~; et je te ferai devenir une nation plus puissante et plus grande que celle-ci.
\VS{15}Je retournai et je descendis de la montagne~; or la montagne était toute en feu, et j'avais les deux tables de l'alliance dans mes deux mains.
\VS{16}Puis je regardai, et voici, vous aviez péché contre Yahweh, votre Dieu, vous vous étiez fait un veau en métal fondu, vous vous étiez détournés promptement de la voie que vous avait ordonnée Yahweh.
\VS{17}Alors je saisis les deux tables, je les jetai de mes deux mains, et je les brisai devant vos yeux.
\TextTitle{Moïse, intercède pour Israël devant Yahweh}
\VS{18}Puis je me prosternai devant Yahweh, comme auparavant, quarante jours et quarante nuits, sans manger de pain et sans boire d'eau, à cause de tout votre péché, que vous aviez commis en faisant ce qui est mal aux yeux de Yahweh, afin de l'irriter.
\VS{19}Car je craignais face à la colère et à la fureur dont Yahweh était enflammé contre vous, pour vous détruire. Et Yahweh m'exauça encore cette fois.
\VS{20}Yahweh était très irrité contre Aaron, voulant le faire périr, mais j'intercédai pour Aaron en ce temps-là.
\VS{21}Puis je pris le veau\FTNT{Le veau d'or (Ex. 32).} que vous aviez fait, votre péché, et je le brûlai au feu, je le brisai en le broyant, jusqu'à ce qu'il soit réduit en poudre, et je jetai cette poudre dans le torrent qui descend de la montagne.
\VS{22}Vous avez fort irrité la colère de Yahweh à Tabeéra, à Massa, et à Kibroth-Hattaava.
\VS{23}Et quand Yahweh vous envoya à Kadès-Barnéa, en disant~: Montez, et prenez possession du pays que je vous donne~! Vous fûtes rebelles à la parole de Yahweh, votre Dieu, vous n'eûtes point confiance, et vous n'obéîtes point à sa voix.
\VS{24}Vous avez été rebelles à Yahweh depuis le jour où je vous ai connu.
\VS{25}Je me prosternai donc devant Yahweh, je me prosternai quarante jours et quarante nuits, parce que Yahweh avait dit qu'il vous détruirait.
\VS{26}Et je priai Yahweh, et je dis~: Ô Seigneur, Yahweh, ne détruis point ton peuple, ton héritage que tu as racheté par ta grandeur, et que tu as fait sortir d'Egypte par ta main puissante.
\VS{27}Souviens-toi de tes serviteurs Abraham, Isaac, et Jacob. Ne regarde point à l'obstination de ce peuple, ni à sa méchanceté, ni à son péché,
\VS{28}de peur que le pays d'où tu nous as fait sortir ne dise~: Parce que Yahweh n'était pas capable de les conduire dans le pays qu'il leur avait promis, et parce qu'il les haïssait, il les a fait sortir pour les faire mourir dans le désert.
\VS{29}Cependant ils sont ton peuple et ton héritage, que tu as fait sortir par ta grande puissance et par ton bras étendu.
\Chap{10}
\TextTitle{Rappel du remplacement des tables de la loi}
\VerseOne{}En ce temps-là Yahweh me dit~: Taille deux tables de pierre comme les premières, et monte vers moi sur la montagne~; tu feras une arche de bois\FTNT{Ex. 25:10~; 34:1-4.}.
\VS{2}Et j'écrirai sur ces tables les paroles qui étaient sur les premières tables que tu as brisées, et tu les mettras dans l'arche.
\VS{3}Ainsi je fis une arche de bois d'acacia, je taillai deux tables de pierre comme les premières, et je montai sur la montagne, les deux tables dans ma main\FTNT{Ex. 34:4.}.
\VS{4}Et Yahweh écrivit sur ces tables ce qui avait été écrit sur les premières, les dix paroles qu'il avait dites sur la montagne, du milieu du feu, le jour de l'assemblée~; puis Yahweh me les donna.
\VS{5}Je me retournai et je descendis de la montagne~; je mis les tables dans l'arche que j'avais faite, et elles y sont demeurées, comme Yahweh me l'avait ordonné.
\VS{6}Or les enfants d'Israël partirent de Beéroth-Bené-Jaakan pour Moséra. Là mourut Aaron, et il fut enseveli~; Eléazar, son fils, exerça la prêtrise à sa place.
\VS{7}De là ils partirent pour Gudgoda, et de Gudgoda pour Jothbatha, qui est un pays de torrents d'eau.
\VS{8}Or en ce temps-là, Yahweh sépara la tribu de Lévi afin de porter l'arche de l'alliance de Yahweh, de se tenir devant Yahweh, de le servir, et de bénir en son Nom, jusqu'à ce jour.
\VS{9}C'est pourquoi Lévi n'a ni portion ni d'héritage avec ses frères~: Yahweh est son héritage, comme Yahweh, ton Dieu, lui a dit.
\VS{10}Je restai sur la montagne, comme la première fois, quarante jours et quarante nuits. Yahweh m'exauça encore cette fois~; Yahweh ne voulut point te détruire.
\VS{11}Mais Yahweh me dit~: Lève-toi, va, marche devant ce peuple. Qu'ils aillent prendre possession du pays que j'ai juré à leurs pères de leur donner.
\TextTitle{Une alliance basée sur l'amour de Yahweh}
\VS{12}Maintenant donc, ô Israël, que demande de toi Yahweh, ton Dieu, sinon que tu craignes Yahweh, ton Dieu, afin de marcher dans toutes ses voies, d'aimer et de servir Yahweh, ton Dieu, de tout ton cœur, et de toute ton âme~;
\VS{13}de garder les commandements de Yahweh et ses lois que je t'ordonne aujourd'hui, afin que tu sois heureux~?
\VS{14}Voici, les cieux, et les cieux des cieux appartiennent à Yahweh, ton Dieu, la terre et tout ce qu'elle renferme.
\VS{15}Et Yahweh s'est attaché à tes pères, pour les aimer~; et après eux, il vous a choisis, vous, leur postérité, entre tous les peuples, comme vous le voyez aujourd'hui.
\VS{16}Circoncisez donc le prépuce de votre cœur, et vous ne raidirez plus votre cou.
\VS{17}Car Yahweh, votre Dieu, est le Dieu des dieux, le Seigneur des seigneurs\FTNT{Yahweh, le Dieu des dieux et le Seigneur des seigneurs n'est autre que Jésus-Christ, notre Seigneur qui s'est révélé à Jean comme le Seigneur des seigneurs et le Roi des rois (Ap. 19:16).}, le Fort, le Grand, le Puissant et le Redoutable, qui n'a point d'égard à l'apparence des personnes, et qui ne prend point de présents~;
\VS{18}qui fait justice à l'orphelin et à la veuve, qui aime l'étranger et lui donne le pain et le vêtement.
\VS{19}Vous aimerez donc l'étranger~; car vous avez été étrangers dans le pays d'Egypte.
\VS{20}Tu craindras Yahweh, ton Dieu, tu le serviras, tu t'attacheras à lui, et tu jureras par son Nom.
\VS{21}Il est ta louange, il est ton Dieu, qui a fait pour toi des choses grandes et redoutables que tes yeux ont vues.
\VS{22}Tes pères descendirent en Egypte au nombre de soixante-dix âmes~; et maintenant Yahweh, ton Dieu, t'a fait devenir comme les étoiles des cieux, tant tu es en grand nombre.
\Chap{11}
\TextTitle{Exhortation à la reconnaissance et à l'obéissance}
\VerseOne{}Tu aimeras donc Yahweh, ton Dieu, et tu garderas toujours ses lois, ses ordonnances, et ses commandements.
\VS{2}Et reconnaissez aujourd'hui, ce que n'ont point connu ni vu vos fils, le châtiment de Yahweh, votre Dieu, sa grandeur, sa main puissante, et son bras étendu,
\VS{3}ses signes, et les œuvres qu'il a accomplies au milieu de l'Egypte contre Pharaon, roi d'Egypte, et contre tout son pays~;
\VS{4}et ce que Yahweh a fait à l'armée d'Egypte, à ses chevaux et à ses chars, quand il a fait déborder sur eux les eaux de la Mer Rouge, car Yahweh les a détruits jusqu'à ce jour\FTNT{Ex. 14:28.}~;
\VS{5}ce qu'il a fait dans le désert, jusqu'à votre arrivée en ce lieu-ci~;
\VS{6}ce qu'il a fait à Dathan et à Abiram, fils d'Eliab, fils de Ruben, comment la terre ouvrit sa bouche et les engloutit, avec leurs maisons et leurs tentes, et tous les êtres qui les suivaient, au milieu de tout Israël\FTNT{No. 16:1-33.}.
\VS{7}Car ce sont vos yeux qui ont vu toutes les grandes œuvres que Yahweh a faites.
\TextTitle{Les bienfaits de la terre promise sont pour un peuple fidèle}
\VS{8}Vous garderez donc tous les commandements que je vous ordonne aujourd'hui, afin que vous ayez la force d'entrer et de vous emparer du pays où vous allez passer pour en prendre possession,
\VS{9}et afin que vous prolongiez vos jours sur la terre que Yahweh a juré à vos pères de leur donner, ainsi qu'à leur postérité, pays où coulent le lait et le miel.
\VS{10}Car le pays où tu vas entrer afin de le posséder n'est pas comme le pays d'Egypte, d'où vous êtes sortis, où tu semais ta semence, et l'arrosais avec ton pied, comme un jardin potager.
\VS{11}Mais le pays où vous allez passer pour le posséder est un pays de montagnes et de vallées, qui boit les eaux de la pluie du ciel~;
\VS{12}c'est un pays dont Yahweh, ton Dieu, prend soin, et sur lequel Yahweh, ton Dieu, a continuellement ses yeux, du commencement de l'année jusqu'à la fin de l'année.
\VS{13}Il arrivera donc que, si vous obéissez attentivement à mes commandements que je vous ordonne aujourd'hui, si vous aimez Yahweh, votre Dieu, et que vous le servez de tout votre cœur et de toute votre âme,
\VS{14}alors je donnerai à votre pays la pluie en son temps, la pluie de la première et de l'arrière-saison, et tu recueilleras ton blé, ton vin, et ton huile.
\VS{15}Je mettrai aussi dans ton champ de l'herbe pour ton bétail, tu mangeras et tu seras rassasié.
\VS{16}Prenez garde à vous, de peur que votre cœur ne soit trompé, et que vous ne vous détourniez, et ne serviez d'autres dieux, et ne vous prosterniez devant eux.
\VS{17}Et que la colère de Yahweh s'enflamme contre vous, et qu'il ne ferme les cieux, tellement qu'il n'y aurait point de pluie, la terre ne donnerait plus son produit, et vous péririez promptement dans ce bon pays que Yahweh vous donne.
\VS{18}Mettez donc dans votre cœur et dans votre âme ces paroles. Liez-les comme un signe sur vos mains, et qu'elles soient comme des fronteaux entre vos yeux.
\VS{19}Et enseignez-les à vos enfants, en leur en parlant, quand tu seras dans ta maison, quand tu partiras en voyage, quand tu te coucheras et quand tu te lèveras.
\VS{20}Tu les écriras aussi sur les poteaux de ta maison, et sur tes portes.
\VS{21}Afin que vos jours et les jours de vos fils, sur la terre que Yahweh a juré à vos pères de leur donner, soient aussi nombreux que les jours des cieux sur la terre.
\VS{22}Car si vous gardez et si vous pratiquez tous ces commandements que je vous ordonne de faire, aimant Yahweh, votre Dieu, marchant dans toutes ses voies, et vous attachant à lui,
\VS{23}alors Yahweh chassera devant vous toutes ces nations et vous prendrez possession de nations plus grandes et plus puissantes que vous.
\VS{24}Tout lieu que foulera la plante de votre pied sera à vous\FTNT{Jos. 1:3~; 14:9.}~: Votre territoire s'étendra du désert au Liban, et du fleuve, le fleuve de l'Euphrate, jusqu'à la Mer Occidentale.
\VS{25}Aucun homme ne tiendra face à vous. Yahweh, votre Dieu, mettra, comme il vous l'a dit, la frayeur et la crainte de vous sur tout le pays où vous marcherez.
\TextTitle{La malédiction et la bénédiction}
\VS{26}Regardez, je mets aujourd'hui devant vous la bénédiction et la malédiction~:
\VS{27}La bénédiction, si vous obéissez aux commandements de Yahweh, votre Dieu, que je vous ordonne aujourd'hui~;
\VS{28}la malédiction, si vous n'obéissez point aux commandements de Yahweh, votre Dieu, et si vous vous détournez du chemin que je vous ordonne aujourd'hui, pour aller après d'autres dieux que vous ne connaissez point.
\VS{29}Et quand Yahweh, ton Dieu, t'aura fait entrer dans le pays dont tu vas prendre possession, tu prononceras alors les bénédictions, étant sur la montagne de Garizim, et les malédictions, étant sur la montagne d'Ebal.
\VS{30}Ces montagnes ne sont-elles pas de l'autre côté du Jourdain, derrière le chemin du soleil couchant, au pays des Cananéens qui demeurent dans la plaine, vis-à-vis de Guilgal, près des chênes de Moré~?
\VS{31}Car vous allez passer le Jourdain, pour entrer et prendre possession du pays que Yahweh, votre Dieu, vous donne~; vous le posséderez, et vous y habiterez.
\VS{32}Vous garderez et pratiquerez toutes les lois et les ordonnances que je mets aujourd'hui devant vous.
\Chap{12}
\TextTitle{Lois sur les sacrifices offerts au lieu où résidera le Nom de Yahweh}
\VerseOne{}Ce sont ici les lois et les ordonnances que vous garderez et pratiquerez dans le pays que Yahweh, le Dieu de vos pères, vous a donné à posséder, tout le temps que vous vivrez sur cette terre.
\VS{2}Vous détruirez, vous détruirez tous les lieux où les nations que vous allez déposséder servent leurs dieux, sur les hautes montagnes et sur les collines, et sous tout arbre verdoyant.
\VS{3}Vous démolirez aussi leurs autels, vous briserez leurs statues, vous brûlerez au feu leurs asheras, vous mettrez en pièces les images taillées de leurs dieux, et vous ferez périr leur nom de ce lieu-là.
\VS{4}Vous ne ferez pas ainsi à Yahweh, votre Dieu.
\VS{5}Mais vous le chercherez dans sa demeure, et vous irez au lieu que Yahweh, votre Dieu, aura choisi d'entre toutes vos tribus, pour y mettre son Nom.
\VS{6}Et vous y apporterez vos holocaustes, vos sacrifices, vos dîmes, vos offrandes élevées, vos vœux, vos offrandes volontaires de vos mains, et les premiers-nés de votre gros et de votre menu bétail\FTNT{Lé. 17:3-4.}.
\VS{7}Et là, vous mangerez devant Yahweh, votre Dieu, et vous vous réjouirez, vous et vos familles, de toutes les choses auxquelles vous aurez mis la main, et dans lesquelles Yahweh, votre Dieu, vous aura bénis.
\VS{8}Vous ne ferez pas comme nous faisons ici aujourd'hui, où chacun fait ce qui lui semble juste à ses yeux,
\VS{9}car vous n'êtes point encore entrés dans le lieu de repos, et dans l'héritage que Yahweh, votre Dieu, vous donne.
\VS{10}Vous passerez le Jourdain, et vous habiterez dans le pays que Yahweh, votre Dieu, vous donne en héritage~; il vous donnera du repos de tous vos ennemis qui vous entourent, et vous y habiterez en sécurité.
\VS{11}Et il y aura un lieu que Yahweh, votre Dieu, choisira pour y faire habiter son Nom. Vous y apporterez tout ce que je vous ordonne, vos holocaustes, vos sacrifices, vos dîmes, vos offrandes élevées de vos mains, et toutes offrandes de choix pour les vœux que vous aurez voués à Yahweh.
\VS{12}Et là, vous vous réjouirez devant Yahweh, votre Dieu, vous, vos fils et vos filles, vos serviteurs et vos servantes, et le Lévite qui sera dans vos portes~; car il n'a ni part ni héritage avec vous.
\VS{13}Garde-toi d'offrir tes holocaustes dans tous les lieux que tu verras~;
\VS{14}mais tu offriras tes holocaustes dans le lieu que Yahweh choisira dans l'une de tes tribus, et tu y feras tout ce que je t'ordonne.
\VS{15}Toutefois, selon le désir de ton âme, tu pourras tuer et manger de la viande dans toutes tes portes, selon la bénédiction que t'accordera Yahweh, ton Dieu~; celui qui sera impur et celui qui sera pur en mangeront, comme on mange de la gazelle et du cerf.
\VS{16}Seulement, vous ne mangerez point de sang. Tu le répandras sur la terre, comme de l'eau.
\VS{17}Tu ne pourras pas manger dans tes portes la dîme de ton blé, de ton vin, de ton huile, ni les premiers-nés de ton gros et menu bétail, ni aucune de tes offrandes en accomplissement d'un vœu, ni tes offrandes volontaires, ni les offrandes élevées de tes mains.
\VS{18}Mais tu les mangeras devant Yahweh, ton Dieu, au lieu que Yahweh, ton Dieu, choisira~; toi, ton fils, ta fille, ton serviteur et ta servante, et le Lévite qui sera dans tes portes~; et tu te réjouiras devant Yahweh, ton Dieu, de tout ce à quoi tu auras mis la main.
\VS{19}Garde-toi, tout le temps que tu vivras sur la terre, d'abandonner le Lévite.
\VS{20}Quand Yahweh, ton Dieu, aura élargi tes frontières, comme il te l'a promis, et que tu diras~: Je mangerai de la chair, parce que ton âme désirera manger de la chair, tu en mangeras selon tous les désirs de ton âme.
\VS{21}Si le lieu que Yahweh, ton Dieu, aura choisi pour y mettre son Nom, est loin de toi, alors tu tueras de ton gros et menu bétail, comme je te l'ai ordonné, et tu en mangeras dans tes portes selon tous les désirs de ton âme.
\VS{22}Tu en mangeras comme on mange de la gazelle et du cerf~; celui qui sera impur et celui qui sera pur en mangeront également.
\VS{23}Seulement, garde-toi de manger le sang, car le sang c'est l'âme~; et tu ne mangeras point l'âme avec la chair\FTNT{Lé. 7:26.}.
\VS{24}Tu n'en mangeras point~: Tu le répandras sur la terre comme de l'eau.
\VS{25}Tu n'en mangeras point, afin que tu sois heureux, toi et tes enfants après toi, parce que tu auras fait ce qui est droit aux yeux de Yahweh.
\VS{26}Mais tu prendras les choses que tu auras consacrées, qui seront à toi, et ce que tu auras voué, tu les prendras et tu viendras au lieu que Yahweh aura choisi.
\VS{27}Et tu offriras tes holocaustes, la chair et le sang, sur l'autel de Yahweh, ton Dieu~; mais le sang de tes autres sacrifices sera versé sur l'autel de Yahweh, ton Dieu, et tu en mangeras la chair.
\VS{28}Garde et écoute toutes ces paroles que je t'ordonne, afin que tu sois heureux, toi et tes enfants après toi, à jamais, en faisant ce qui est bon et droit aux yeux de Yahweh, ton Dieu.
\TextTitle{Mise en garde contre la séduction et les dieux étrangers}
\VS{29}Quand Yahweh, ton Dieu, aura exterminé de devant toi les nations que tu vas prendre en possession, que tu les auras possédées, et que tu habiteras dans leur pays,
\VS{30} prends garde à toi, de peur que tu ne sois pris au piège après elles, quand elles auront été détruites de devant toi~; et que tu ne recherches leurs dieux, en disant~: Comme ces nations-là servaient leurs dieux, je le ferai aussi tout de même.
\VS{31}Tu ne feras point ainsi à Yahweh, ton Dieu~; car elles ont fait à leurs dieux tout ce qui est en abomination et qui est odieux à Yahweh, et même ils brûlaient au feu leurs fils et leurs filles à leurs dieux.
\VS{32}Vous prendrez garde de faire tout ce que je vous commande. Vous n'y ajouterez rien, et vous n'en retrancherez rien.
\Chap{13}
\TextTitle{Eprouver les faux prophètes, ôter le méchant du milieu de l'assemblée}
\VerseOne{}S'il s'élève au milieu de toi un prophète ou un songeur de songes, qui te donne un signe ou miracle,
\VS{2}et que ce signe ou ce miracle dont il t'a parlé, arrive, et qu'il te dise~: Allons après d'autres dieux que tu ne connais point, et servons-les~!
\VS{3}Tu n'écouteras point les paroles de ce prophète ni de ce songeur de songes, car Yahweh, votre Dieu, vous met à l'épreuve pour savoir si vous aimez Yahweh, votre Dieu, de tout votre cœur et de toute votre âme.
\VS{4}Vous marcherez après Yahweh, votre Dieu, vous le craindrez~; vous garderez ses commandements, vous obéirez à sa voix, vous le servirez, et vous vous attacherez à lui.
\VS{5}Mais on fera mourir ce prophète-là ou ce songeur de songes, parce qu'il a parlé de révolte\FTNT{Le mot «~révolte~» utilisé ici, traduit le terme hébreu «~carah~» et signifie «~apostasie~». L'apostasie est une déviation progressive. C'est tout d'abord l'abandon d'une vérité reçue. Paul, l'apôtre enseigne que deux événements doivent avoir lieu avant le retour du Seigneur sur la terre~: L'apostasie et la révélation de l'homme du péché, le fils de perdition, c'est-à-dire l'Antichrist (2 Th. 2:1-3~; 2 Ti. 4:1).} contre Yahweh, votre Dieu, qui vous a fait sortir du pays d'Egypte et vous a délivrés de la maison de servitude, pour vous conduire loin de la voie que Yahweh, votre Dieu, vous a ordonné de marcher. Tu ôteras le méchant du milieu de toi.
\VS{6}Quand ton frère, fils de ta mère, ou ton fils, ou ta fille, ou ta femme bien-aimée, ou ton intime ami, qui est comme ton âme, t'incitera, en te disant en secret~: Allons, et servons d'autres dieux, que tu n'as point connus, ni tes pères,
\VS{7}d'entre les dieux des peuples qui sont autour de vous, près ou loin de toi, d'une extrémité de la terre jusqu'à l'autre,
\VS{8}tu ne t'accorderas pas avec lui, et tu ne l'écouteras point. Ton œil ne le regardera pas avec pitié, tu ne l'épargneras point, et tu ne le cacheras point.
\VS{9}Mais tu le feras mourir, tu le feras mourir\FTNT{Répétition du mot «~mourir~», voir commentaire en Ge. 2:17}~; ta main sera la première sur lui pour le mettre à mort, et ensuite la main de tout le peuple.
\VS{10}Tu le lapideras avec des pierres, et il mourra, parce qu'il a cherché à t'éloigner loin de Yahweh, ton Dieu, qui t'a fait sortir du pays d'Egypte, de la maison de servitude.
\VS{11}Afin que tout Israël entende et craigne, et que l'on ne fasse plus une action aussi méchante au milieu de toi.
\TextTitle{Jugement des villes idolâtres}
\VS{12}Si tu entends dire dans l'une des villes que Yahweh, ton Dieu, t'a données pour y habiter~:
\VS{13}Des hommes, fils de Bélial, sont sortis du milieu de toi, et ont chassé les habitants de leur ville, en disant~: Allons et servons d'autres dieux, des dieux que tu ne connais point~!
\VS{14}Tu chercheras, tu examineras, tu t'enquerras bien. Et si c'est la vérité, si la chose est établie, si cette abomination a été faite au milieu de toi,
\VS{15}tu frapperas, tu frapperas\FTNT{Répétition du mot «~frapperas~». Dans les écrits hébraïque, la répétition de mots est utilisée afin d'accentuer une action, pour appuyer un fait précis et le renforcer.} du tranchant de l'épée les habitants de cette ville, tu la dévoueras par interdit, et tu passeras le bétail au fil de l'épée.
\VS{16}Tu assembleras tout son butin au milieu de la place, et tu brûleras entièrement au feu cette ville et tout son butin, devant Yahweh, ton Dieu~: Elle sera pour toujours un monceau de ruines, sans être jamais rebâtie.
\VS{17}Rien de ce qui sera dévoué ne s'attachera à ta main, afin que Yahweh revienne de l'ardeur de sa colère, qu'il te fasse miséricorde et grâce, et qu'il te multiplie, comme il a juré à tes pères,
\VS{18}si tu obéis à la voix de Yahweh, ton Dieu, en gardant tous ses commandements que je t'ordonne aujourd'hui, et en faisant ce qui est droit aux yeux de Yahweh, ton Dieu.
\Chap{14}
\TextTitle{Israël, peuple mis en part}
\VerseOne{}Vous êtes les enfants de Yahweh, votre Dieu. Vous ne vous ferez aucune incision, et vous ne vous ferez point de place chauve entre les yeux pour aucun mort.
\VS{2}Car tu es un peuple saint pour Yahweh, ton Dieu~; et Yahweh t'a choisi pour que tu lui sois un peuple qui lui appartienne entre tous les peuples qui sont sur la face de la terre.
\TextTitle{Lois sur l'alimentation}
\VS{3}Tu ne mangeras d'aucune chose abominable.
\VS{4}Ce sont ici les bêtes que vous mangerez~: Le bœuf, la brebis et la chèvre~;
\VS{5}le cerf, la gazelle et le daim~; le bouquetin, le chevreuil, la chèvre sauvage, et le mouflon.
\VS{6}Vous mangerez donc toute bête qui a le sabot divisé, le pied fendu, et qui rumine.
\VS{7}Mais vous ne mangerez point de ceux qui ruminent seulement, ou qui ont le sabot divisé et le pied fendu seulement, comme le chameau, le lièvre et le lapin, car ils ruminent bien, mais ils n'ont pas le sabot qui est fendu~: Ils vous seront impurs.
\VS{8}Le porc aussi, car il a le sabot fendu, mais il ne rumine point~: Il vous sera impur. Vous ne mangerez point de leur chair, et vous ne toucherez point à leur cadavre.
\VS{9}Voici ce que vous mangerez de tout ce qui est dans les eaux~: Vous mangerez de tout ce qui a des nageoires et des écailles.
\VS{10}Mais vous ne mangerez point de ce qui n'a ni nageoires ni écailles~: Cela vous sera impur.
\VS{11}Vous mangerez tout oiseau pur.
\VS{12}Mais voici ceux dont vous ne mangerez point~: L'aigle, l'orfraie, l'aigle de mer~;
\VS{13}le vautour, le milan, et l'autour, selon leur espèce~;
\VS{14}le corbeau, selon son espèce~;
\VS{15}l'autruche, le hibou, la mouette, l'épervier, selon son espèce~;
\VS{16}le chat-huant, la chouette et le cygne~;
\VS{17}le cormoran, le pélican, le plongeon~;
\VS{18}la cigogne, le héron, selon leur espèce, la huppe et la chauve-souris.
\VS{19}Et tout reptile qui vole sera impur pour vous~; on n'en mangera point.
\VS{20}Mais vous mangerez de tout ce qui vole et qui est pur.
\VS{21}Vous ne mangerez aucun cadavre~; tu le donneras à l'étranger qui sera dans tes portes, et il le mangera, ou tu le vendras à un étranger~; car tu es un peuple saint pour Yahweh, ton Dieu. Tu ne feras point cuire le chevreau dans le lait de sa mère.
\TextTitle{Lois sur les dîmes\FTNT{No. 18:21-32.}}
\VS{22}Tu prendras la dîme, tu prendras la dîme\FTNT{Il est question ici de la dîme que les Hébreux consommaient chaque année.} de tout le produit de ta semence, de ce qui sortira de ton champ, chaque année.
\VS{23}Et tu mangeras devant Yahweh, ton Dieu, au lieu qu'il aura choisi pour y faire habiter son Nom, la dîme de ton blé, de ton vin et de ton huile, et les premiers-nés de ton gros et menu bétail, afin que tu apprennes à toujours craindre Yahweh, ton Dieu.
\VS{24}Mais quand le chemin sera trop long pour que tu puisses les transporter, parce que le lieu que Yahweh, ton Dieu, aura choisi pour y mettre son Nom, sera trop loin de toi, lorsque Yahweh, ton Dieu, t'aura béni,
\VS{25}alors tu l'échangeras contre de l'argent, tu serreras l'argent dans ta main, et tu iras au lieu que Yahweh, ton Dieu, aura choisi.
\VS{26}Et tu donneras l'argent contre tout ce que ton âme désirera, des bœufs, des brebis, du vin et des liqueurs fortes, tout ce que ton âme demandera, tu le mangeras devant Yahweh, ton Dieu, et tu te réjouiras, toi et ta famille.
\VS{27}Tu n'abandonneras point le Lévite qui sera dans tes portes, parce qu'il n'a ni portion ni héritage avec toi\FTNT{Ce verset fait référence à la première dîme qui devait être donnée aux Lévites (Voir commentaire en No. 18:21 et Mal. 3:1).}.
\VS{28}Au bout de trois ans, tu feras sortir toutes les dîmes de tes produits de cette année-là, et tu les déposeras dans tes portes.
\VS{29}Alors le Lévite, qui n'a ni portion ni héritage avec toi, l'étranger, l'orphelin, et la veuve qui seront dans tes portes, viendront, mangeront et se rassasieront, afin que Yahweh, ton Dieu, te bénisse dans toute l'œuvre que tu feras de tes mains.
\Chap{15}
\TextTitle{Lois sur l'année de relâche~: la justice et la bonté de Yahweh}
\VerseOne{}Tous les sept ans, tu célébreras l'année de relâche\FTNT{Ex. 21:2, Jé. 34:14.}.
\VS{2}Et c'est ici la manière de célébrer l'année de relâche. Que tout homme ayant droit d'exiger quelque chose que ce soit, qu'il puisse exiger de son prochain, donnera relâche, et ne l'exigera point de son prochain ni de son frère, quand on aura proclamé le relâche, en l'honneur de Yahweh.
\VS{3}Tu l'exigeras de l'étranger~; mais ta main relâchera tout ce qui t'appartiendra chez ton frère,
\VS{4}afin qu'il n'y ait point d'indigent chez toi, car Yahweh te bénira, te bénira\FTNT{Voir commentaire en Ge. 2:16} abondamment dans le pays que Yahweh, ton Dieu, te donnera à posséder pour héritage~;
\VS{5}pourvu que tu obéisses, que tu obéisses\FTNT{Voir commentaire en Ge. 2:16} bien à la voix de Yahweh, ton Dieu, en prenant garde de pratiquer tous ces commandements que je t'ordonne aujourd'hui.
\VS{6}Parce que Yahweh, ton Dieu, te bénira comme il te l'a promis, tu prêteras sur gage à beaucoup de nations, et tu n'emprunteras point sur gage~; tu domineras sur beaucoup de nations, et elles ne domineront point sur toi.
\VS{7}Quand un de tes frères sera indigent au milieu de toi, dans l'une de tes portes, dans le pays que Yahweh, ton Dieu, te donne, tu n'endurciras point ton cœur, et tu ne fermeras point ta main à ton frère indigent.
\VS{8}Mais tu lui ouvriras, tu lui ouvriras\FTNT{Voir commentaire en Ge. 2:16} ta main, et tu lui prêteras, lui prêteras\FTNT{Voir commentaire en Ge. 2:16} sur gage autant qu'il en aura besoin pour son indigence, dans laquelle il se trouvera.
\VS{9}Prends garde à toi, de peur que tu n'aies dans ton cœur quelque chose de Bélial, et que tu ne dises~: La septième année, l'année du relâche approche~! Et que ton œil soit méchant envers ton frère indigent, afin de ne rien lui donner et qu'il ne crie à Yahweh contre toi, et qu'il n'y ait du péché en toi.
\VS{10}Tu lui donneras, lui donneras\FTNT{Voir commentaire en Ge. 2:16} et que ton cœur ne lui donne point à regret~; car à cause de cela, Yahweh, ton Dieu, te bénira dans toutes tes œuvres, et dans tout ce à quoi tu mettras tes mains.
\VS{11}Car il y aura toujours des indigents dans le pays~; c'est pourquoi je t'ordonne, et je te dis~: Tu ouvriras, tu ouvriras\FTNT{Voir commentaire en Ge. 2:16} ta main à ton frère, à l'affligé, et à l'indigent dans ton pays.
\TextTitle{Loi sur les esclaves}
\VS{12}Quand l'un de tes frères Hébreux, homme ou femme, te sera vendu, il te servira six ans~; mais la septième année, tu le renverras libre de chez toi.
\VS{13}Et quand tu le renverras libre de chez toi, tu ne le renverras point à vide.
\VS{14}Tu chargeras, chargeras\FTNT{Voir commentaire en Ge. 2:16} de quelque chose de ton menu bétail, de ton aire, de ton pressoir, et tu lui donneras de ce que Yahweh, ton Dieu, t'aura béni.
\VS{15}Et tu te souviendras que tu as été esclave au pays d'Egypte, et que Yahweh, ton Dieu, t'en a racheté~; et c'est pour cela que je t'ordonne ceci aujourd'hui.
\VS{16}Mais s'il arrive qu'il te dise~: Je ne sortirai point de chez toi~; parce qu'il t'aime, toi et ta maison, et qu'il se trouve bien chez toi,
\VS{17}alors tu prendras un poinçon\FTNT{Ex. 21:6} et tu lui perceras l'oreille contre la porte, et il sera ton serviteur pour toujours. Tu en feras de même à ta servante.
\VS{18}Ce ne sera point, à tes yeux, dur de le renvoyer libre de chez toi, car il t'a servi six ans, ce qui est le double salaire d'un mercenaire~; et Yahweh, ton Dieu, te bénira en tout ce que tu feras.
\TextTitle{Loi sur les premiers-nés des animaux}
\VS{19}Tu consacreras à Yahweh, ton Dieu, tout premier-né mâle qui naîtra parmi ton gros et ton menu bétail. Tu ne travailleras point avec le premier-né de ton bœuf, et tu ne tondras point le premier-né de tes brebis\FTNT{Ex. 13:2.}.
\VS{20}Tu le mangeras, toi et ta famille, chaque année devant Yahweh, ton Dieu, dans le lieu que Yahweh aura choisi.
\VS{21}Mais s'il a quelque défaut, boiteux ou aveugle, ou qu'il ait quelque autre mauvais défaut, tu ne le sacrifieras point à Yahweh, ton Dieu.
\VS{22}Mais tu le mangeras dans tes portes~; celui qui sera impur et celui qui sera pur en mangeront également, comme on mange de la gazelle et du cerf.
\VS{23}Seulement, tu n'en mangeras point le sang~; mais tu le répandras sur la terre comme de l'eau.
\Chap{16}
\TextTitle{La Pâque et la fête des pains sans levain}
\VerseOne{}Observe le mois des épis, et fais la Pâque à Yahweh, ton Dieu~; car c'est au mois des épis que Yahweh, ton Dieu, t'a fait sortir, de nuit, d'Egypte\FTNT{Ex. 12:2-29.}.
\VS{2}Et tu sacrifieras la Pâque à Yahweh, ton Dieu, du gros et du menu bétail, au lieu que Yahweh choisira pour y faire habiter son Nom.
\VS{3}Tu ne mangeras point de pain levé, mais tu mangeras sept jours des pains sans levain, du pain d'affliction, parce que tu es sorti précipitamment du pays d'Egypte, afin que tous les jours de ta vie tu te souviennes du jour où tu es sorti du pays d'Egypte.
\VS{4}Et il ne se verra point de levain chez toi, sur tout le territoire de ton pays pendant sept jours\FTNT{1 Co. 5:7.}~; et aucune chair que tu sacrifieras le soir du premier jour ne restera jusqu'au matin.
\VS{5}Tu ne pourras point sacrifier la Pâque dans l'une de tes portes que Yahweh, ton Dieu, te donne~;
\VS{6}mais c'est au lieu que Yahweh, ton Dieu, choisira pour y faire habiter son Nom, que tu sacrifieras la Pâque, le soir, au coucher du soleil, moment où tu es sorti d'Egypte.
\VS{7}Tu la cuiras et tu la mangeras dans le lieu que Yahweh, ton Dieu, aura choisi. Et le matin, tu t'en retourneras et tu t'en iras dans tes tentes.
\VS{8}Pendant six jours, tu mangeras des pains sans levain~; et le septième jour, il y aura une assemblée solennelle à Yahweh, ton Dieu~: Tu ne feras aucune œuvre.
\TextTitle{La fête des semaines}
\VS{9}Tu te compteras sept semaines~; tu commenceras à compter ces sept semaines dès que la faucille sera mise dans les blés.
\VS{10}Puis tu feras la fête des semaines à Yahweh, ton Dieu, en présentant l'offrande volontaire de ta main, que tu donneras, selon que Yahweh, ton Dieu, t'aura béni.
\VS{11}Et tu te réjouiras devant Yahweh, ton Dieu, toi, ton fils et ta fille, ton serviteur et ta servante, le Lévite qui sera dans tes portes, l'étranger, l'orphelin et la veuve qui seront au milieu de toi, dans le lieu que Yahweh, ton Dieu, aura choisi pour y faire habiter son Nom.
\VS{12}Et tu te souviendras que tu as été esclave en Egypte, et tu garderas et pratiqueras ces lois.
\TextTitle{La fête des tabernacles}
\VS{13}Tu feras la fête des tabernacles pendant sept jours, après que tu auras recueilli le produit de ton aire et de ton pressoir.
\VS{14}Et tu te réjouiras à cette fête, toi, ton fils et ta fille, ton serviteur et ta servante, le Lévite, l'étranger, l'orphelin, et la veuve qui seront dans tes portes.
\VS{15}Tu célébreras la fête pendant sept jours à Yahweh, ton Dieu, dans le lieu que Yahweh aura choisi~; car Yahweh, ton Dieu, te bénira dans toute ta récolte, et dans tout le travail de tes mains, et tu vivras dans la joie.
\TextTitle{Offrandes à Yahweh selon ses moyens}
\VS{16}Trois fois l'an, tout mâle d'entre vous se présentera devant Yahweh, ton Dieu, dans le lieu qu'il aura choisi, à la fête des pains sans levain, à la fête des semaines, et à la fête des tabernacles. On ne se présentera point devant Yahweh à vide.
\VS{17}Mais chacun donnera à proportion de ce qu'il aura, selon la bénédiction de Yahweh,ton Dieu, qu'il t'aura donnée. 
\TextTitle{Des juges établis pour faire respecter la justice de Yahweh}
\VS{18}Tu t'établiras des juges et des officiers dans toutes les villes que Yahweh, ton Dieu, te donne, selon tes tribus~; et ils jugeront le peuple d'un juste jugement.
\VS{19}Tu ne te détourneras point de la justice, tu ne prêteras point attention à l'apparence des personnes, et tu ne recevras point de présents, car les présents aveuglent les yeux des sages et corrompent les paroles des justes.
\VS{20}Tu suivras fermement la justice, afin que tu vives et que tu possèdes le pays que Yahweh, ton Dieu, te donne.
\TextTitle{Prescriptions sur les cultes}
\VS{21}Tu ne planteras point d'arbre d'Asherah\FTNT{Bois ou arbre d'Asherah~: Il est question d'un objet en bois, pieu sacré ou arbre utilisé dans le culte d'Astarté, l'épouse de Baal (Ex. 34:13~; De. 7:5~; De. 12:3~; Jg. 3:7~; Jg. 6:25-30~; 1 R. 14:15-23).}, près de l'autel que tu feras à Yahweh, ton Dieu.
\VS{22}Tu ne dresseras point non plus de statue~; Yahweh, ton Dieu, hait ces choses.
\Chap{17}
\VerseOne{}Tu ne sacrifieras à Yahweh, ton Dieu, ni bœuf, ni agneau qui ait quelque défaut ou quelque chose de mauvais~; car c'est une abomination à Yahweh, ton Dieu.
\TextTitle{Punition de l'idolâtrie}
\VS{2}S'il se trouve au milieu de toi dans l'une des villes que Yahweh, ton Dieu, te donne, un homme ou une femme faisant ce qui est mal aux yeux de Yahweh, ton Dieu, en transgressant son alliance,
\VS{3}et allant servir d'autres dieux et se prosterner devant eux, devant le soleil, devant la lune, ou devant toute l'armée des cieux, ce que je n'ai pas ordonné~;
\VS{4}et que cela t'aura été rapporté, et que tu l'auras entendu, alors tu feras des recherches avec soin. Si la chose est vraie, que le fait est établi, et que cette abomination a été commise en Israël,
\VS{5}alors tu feras sortir vers tes portes cet homme ou cette femme, qui aura fait cette mauvaise action, cet homme, dis-je, ou cette femme, et tu les lapideras avec des pierres, et ils mourront.
\VS{6}On fera mourir sur la parole de deux témoins ou de trois témoins\FTNT{Mt. 18:15-17.}, celui qui doit être mis à mort~; il ne sera pas mis à mort sur la parole d'un seul témoin.
\VS{7}La main des témoins sera la première sur lui pour le faire mourir, et ensuite la main de tout le peuple. Et ainsi tu ôteras le mal du milieu de toi.
\TextTitle{Soumission aux autorités}
\VS{8}Quand une affaire te paraîtra trop difficile à juger entre meurtre et meurtre, entre cause et cause, entre plaie et plaie, qui sont des affaires de procès dans tes portes, alors tu te lèveras et tu monteras au lieu que Yahweh, ton Dieu, aura choisi.
\VS{9}Et tu iras vers les prêtres, les Lévites, et vers le juge qu'il y aura en ce temps-là, tu les consulteras, et ils te feront connaître et te déclareront la sentence du jugement.
\VS{10}Tu feras conformément à la sentence qu'ils t'auront déclarée de leur bouche dans le lieu que Yahweh aura choisi, et tu prendras garde de faire tout ce qu'ils t'enseigneront.
\VS{11}Tu feras conformément à la loi qu'ils t'auront enseignée de leur bouche et selon la sentence qu'ils t'auront prononcée~; tu ne te détourneras ni à droite ni à gauche de ce qu'ils t'auront déclaré.
\VS{12}Mais l'homme qui agira par orgueil et n'obéira pas au prêtre qui se tient là pour servir Yahweh, ton Dieu, ou au juge, cet homme mourra. Tu ôteras le mal d'Israël,
\VS{13}et tout le peuple l'entendra et craindra, et n'agira plus par orgueil.
\TextTitle{Instructions sur la royauté}
\VS{14}Quand tu seras entré dans le pays que Yahweh, ton Dieu, te donne, que tu le posséderas, que tu y demeureras, et que tu diras~: J'établirai un roi sur moi, comme toutes les nations qui sont autour de moi,
\VS{15}tu ne manqueras pas de t'établir pour roi celui que Yahweh, ton Dieu, aura choisi, tu établiras un roi du milieu de tes frères, tu ne pourras point désigner un homme étranger qui ne soit pas ton frère\FTNT{Dans sa prescience, Yahweh savait que le peuple se détournerait de ses voies et réclamerait un roi, à l'identique des nations alentour. (1 S. 8). Or depuis leur sortie d'Egypte, seul Yahweh était leur Dieu et leur Roi.}.
\VS{16}Seulement, il n'aura pas de nombreux chevaux, et il ne ramènera point le peuple en Egypte pour augmenter le nombre de chevaux~; car Yahweh vous a dit~: Vous ne retournerez plus par ce chemin.
\VS{17}Il n'aura point un grand nombre de femmes, afin que son cœur ne se détourne point~; et qu'il n'accumule point beaucoup d'argent et d'or.
\VS{18}Et dès qu'il sera assis sur le trône de son royaume, il écrira pour lui, dans un livre, une copie de cette loi, qu'il prendra des prêtres, les Lévites.
\VS{19}Il l'aura auprès de lui et la lira tous les jours de sa vie, afin qu'il apprenne à craindre Yahweh, son Dieu, à prendre garde à toutes les paroles de cette loi, et à ces ordonnances, afin de les pratiquer~;
\VS{20}afin que son cœur ne s'élève point au-dessus de ses frères, et qu'il ne se détourne point de ce commandement ni à droite ni à gauche~; afin qu'il prolonge ses jours dans son royaume, lui et ses fils, au milieu d'Israël.
\Chap{18}
\TextTitle{Héritage des Lévites et des prêtres}
\VerseOne{}Les prêtres, les Lévites, et même toute la tribu de Lévi, n'auront ni part ni héritage avec Israël~; ils mangeront les sacrifices consumés par le feu de Yahweh, et de son héritage.
\VS{2}Ils n'auront point d'héritage parmi leurs frères~: Yahweh sera leur héritage, comme il leur a dit.
\VS{3}Or c'est ici le droit que les prêtres prendront du peuple, sur ceux qui offriront un sacrifice, un bœuf ou un agneau~: On donnera au prêtre l'épaule, les mâchoires et l'estomac.
\VS{4}Tu lui donneras les prémices de ton blé, de ton vin et de ton huile, et les prémices de la toison de tes brebis.
\VS{5}Car Yahweh, ton Dieu, l'a choisi d'entre toutes les tribus, afin qu'il se tienne devant lui, et qu'il fasse le service au Nom de Yahweh, lui et ses fils, à toujours.
\VS{6}Or quand le Lévite viendra de l'une de tes portes, de tout lieu où il habite en Israël, et qu'il viendra selon tout le désir de son âme, au lieu que Yahweh aura choisi,
\VS{7}et qu'il fera le service au Nom de Yahweh, son Dieu, comme tous ses frères Lévites qui se tiennent là devant Yahweh,
\VS{8}il mangera une portion égale à la leur, outre ce qu'il aura vendu de son patrimoine.
\TextTitle{Les abominations des nations interdites en Israël}
\VS{9}Quand tu seras entré dans le pays que Yahweh, ton Dieu, te donne, tu n'apprendras point à faire les abominations de ces nations-là.
\VS{10}Qu'on ne trouve au milieu de toi personne qui fasse passer par le feu son fils ou sa fille, personne qui pratique la divination, l'astrologie, l'augure, la sorcellerie,
\VS{11}ni d'enchanteur qui use d'enchantements, personne qui consulte les médiums ou disent la bonne aventure, personne qui interroge les morts\FTNT{Yahweh interdit tout contact avec le monde des esprits et des démons. Le croyant qui accepte l'Evangile comprendra sans peine et simplement en obéissant à la Parole que ce domaine est interdit. Voir Ex. 22:18~; Lé. 19:26~; Lé. 19:31~; Lé. 20:6~; Lé. 20:27~; Es. 8:19~; 2 Ch. 33:6~; Ac. 19:13-20.}.
\VS{12}Car quiconque fait ces choses est en abomination à Yahweh~; et à cause de ces abominations, Yahweh, ton Dieu, va chasser ces nations-là devant toi.
\VS{13}Tu seras intègre avec Yahweh, ton Dieu.
\VS{14}Car ces nations, que tu vas déposséder, écoutent les pronostiqueurs et les devins~; mais à toi, Yahweh, ton Dieu, ne le permet point.
\TextTitle{Annonce sur la venue du Messie}
\VS{15}Yahweh, ton Dieu, te suscitera du milieu de toi, d'entre tes frères, un prophète comme moi\FTNT{Moïse a annoncé la venue d'un prophète comme lui, c'est-à-dire un prophète de la délivrance et de l'exode. Ce prophète n'est autre que Jésus-Christ qui nous délivre de l'emprise de Satan et nous sort du monde pour nous amener dans la Nouvelle Jérusalem (Jn. 14:2~; Col. 1:13). Notons qu'au moment de la transfiguration, Elie et Moïse parlaient avec Jésus de son départ («~exodus~» en grec~; Lu. 9:31).}: Vous l'écouterez.
\VS{16}Selon tout ce que tu as demandé à Yahweh, ton Dieu, à Horeb, le jour de l'assemblée, quand tu disais~: Que je n'entende plus la voix de Yahweh, mon Dieu, et que je ne voie plus ce grand feu, de peur de mourir.
\VS{17}Alors Yahweh me dit~: Ce qu'ils ont dit est bien.
\VS{18}Je leur susciterai un prophète comme toi du milieu de leurs frères, je mettrai mes paroles dans sa bouche, et il leur dira tout ce que je lui ordonnerai.
\VS{19}Et il arrivera que si un homme n'écoute pas mes paroles qu'il dira en mon Nom, je lui en demanderai compte.
\TextTitle{Comment éprouver les prophètes~?}
\VS{20}Mais le prophète qui agira de manière orgueilleuse pour dire en mon Nom une parole que je ne lui aurai point ordonnée de dire, ou qui parlera au nom des autres dieux, ce prophète-là mourra.
\VS{21}Et si tu dis dans ton cœur~: Comment connaîtrons-nous la parole que Yahweh n'aura point dite~?
\VS{22}Quand le prophète parlera au Nom de Yahweh, et que ce qu'il aura dit n'arrivera pas, ce sera une parole que Yahweh ne lui aura point dite. C'est par orgueil que le prophète l'a dite~: N'aie point peur de lui.
\Chap{19}
\TextTitle{Les villes de refuge\FTNTT{No. 35:1-34.}}
\VerseOne{}Quand Yahweh, ton Dieu, aura exterminé les nations dont Yahweh, ton Dieu, te donne le pays, et que tu les auras dépossédées et que tu demeureras dans leurs villes, et dans leurs maisons,
\VS{2}alors tu sépareras trois villes au milieu du pays que Yahweh, ton Dieu, te donne à posséder.
\VS{3}Tu établiras des chemins, et tu diviseras en trois le territoire de ton pays, que Yahweh, ton Dieu, te donnera en héritage. Ce sera afin que tout meurtrier s'y enfuie.
\VS{4}Or voici comment on procédera envers le meurtrier qui s'enfuira pour sauver sa vie. Celui qui aura frappé son prochain involontairement, et sans l'avoir haï dans le passé~;
\VS{5}ainsi, si quelqu'un va couper du bois dans la forêt avec une autre personne, la hache à la main pour couper du bois, si le fer glisse du manche, trouve son compagnon, et s'il en meurt~; il s'enfuira alors dans une de ces villes, afin qu'il vive.
\VS{6}De peur que celui qui venge le sang ne poursuive le meurtrier, parce que son cœur est échauffé, et qu'il ne le rattrape, si le chemin est trop long, et ne le frappe à mort, alors qu'il ne mérite pas la mort, parce qu'il ne le haïssait pas auparavant\FTNT{No. 35:1-34.}.
\VS{7}C'est pourquoi je t'ordonne, en disant~: Sépare-toi trois villes.
\VS{8}Lorsque Yahweh, ton Dieu, aura élargi tes frontières, comme il l'a juré à tes pères, et qu'il t'aura donné tout le pays qu'il a promis à tes pères de te donner,
\VS{9}parce que tu auras gardé et mis en pratique tous ces commandements que je t'ordonne aujourd'hui, en aimant Yahweh, ton Dieu, et en marchant toujours dans ses voies, alors tu ajouteras encore trois villes à ces trois-là,
\VS{10}afin que le sang innocent ne soit versé au milieu du pays que Yahweh, ton Dieu, te donne en héritage, et que tu ne sois pas coupable de meurtre.
\VS{11}Mais si un homme hait son prochain, lui dresse un piège, se lève contre lui et frappe cette personne, de sorte qu'il meure, et qu'il s'enfuit dans l'une de ces villes,
\VS{12}alors les anciens de sa ville l'enverront saisir, et le livreront entre les mains du vengeur de sang, afin qu'il meure.
\VS{13}Ton œil ne l'épargnera point, mais tu feras disparaître d'Israël le sang innocent, et tu seras heureux.
\VS{14}Tu ne déplaceras point les bornes de ton prochain, fixées par tes ancêtres, dans l'héritage que tu posséderas, dans le pays que Yahweh, ton Dieu, te donne à posséder.
\TextTitle{Résoudre des différends}
\VS{15}Un seul témoin ne sera point valable contre un homme pour constater un crime ou un péché, quel que soit le péché~; mais sur la parole de deux témoins ou de trois témoins la chose sera valable.
\VS{16}Quand un faux témoin s'élèvera contre un homme pour témoigner contre lui d'un crime,
\VS{17}ces deux hommes en contestation comparaîtront devant Yahweh, en présence des prêtres et des juges qui seront là en ce temps-là.
\VS{18}Et les juges feront des recherches avec soin. Si le témoin est un faux témoin, s'il a donné un faux témoignage contre son frère,
\VS{19}tu lui feras comme il avait pensé faire à son frère. Tu ôteras ainsi le mal du milieu de toi.
\VS{20}Et les autres entendront et craindront, et ne feront plus une chose aussi méchante au milieu de toi.
\VS{21}Ton œil ne l'épargnera point~: Vie pour vie, œil pour œil, dent pour dent, main pour main, pied pour pied.
\Chap{20}
\TextTitle{Instructions diverses pour la guerre}
\VerseOne{}Quand tu iras à la guerre contre tes ennemis, et que tu verras des chevaux et des chars, et un peuple plus grand que toi, tu ne les craindras point, car Yahweh, ton Dieu, qui t'a fait monter du pays d'Egypte, est avec toi.
\VS{2}Et quand vous vous approcherez du combat, le prêtre s'avancera et parlera au peuple.
\VS{3}Et leur dira~: Ecoute Israël~: Vous vous approchez aujourd'hui pour combattre vos ennemis. Que votre cœur ne faiblisse pas~; ne craignez point, ne soyez point effrayés et ne soyez point terrifiés face à eux.
\VS{4}Car Yahweh, votre Dieu, marche avec vous, pour combattre vos ennemis, pour vous sauver.
\VS{5}Les officiers parleront au peuple, en disant~: Qui est l'homme qui a bâti une maison neuve et ne l'a pas inaugurée~? Qu'il s'en aille et retourne dans sa maison, de peur qu'il ne meure dans la bataille et qu'un autre homme ne l'inaugure.
\VS{6}Qui est celui qui a planté une vigne et n'en a point encore cueilli le fruit~? Qu'il s'en aille et retourne dans sa maison, de peur qu'il ne meure dans la bataille et qu'un autre homme n'en cueille le fruit.
\VS{7}Qui est celui qui a fiancé une femme et ne l'a point prise en mariage~? Qu'il s'en aille et retourne dans sa maison, de peur qu'il ne meure dans la bataille et qu'un autre homme ne la prenne en mariage.
\VS{8}Et les officiers continueront à parler au peuple, et diront~: Si un homme a peur et est timide, qu'il s'en aille et retourne dans sa maison, de peur que le cœur de ses frères ne devienne craintif comme le sien.
\VS{9}Quand les officiers auront fini de parler au peuple, ils désigneront les chefs des armées à la tête du peuple.
\VS{10}Quand tu t'approcheras d'une ville pour lui faire la guerre, tu l'inviteras à la paix.
\VS{11}Et si elle te donne une réponse de paix et s'ouvre à toi, tout le peuple qui s'y trouvera te sera tributaire et te servira.
\VS{12}Si elle ne fait pas la paix avec toi et qu'elle te fait la guerre, alors tu l'assiègeras.
\VS{13}Et quand Yahweh, ton Dieu, l'aura livrée entre tes mains, tu frapperas tous les mâles au fil de l'épée.
\VS{14}Mais les femmes, les enfants, le bétail, tout ce qui sera dans la ville, et tout son butin, tu le prendras pour toi et tu mangeras le butin de tes ennemis, que Yahweh, ton Dieu, t'aura donné.
\VS{15}Tu feras ainsi à toutes les villes qui sont très éloignées de toi, et qui ne sont point des villes de ces nations.
\VS{16}Mais dans les villes de ces peuples que Yahweh, ton Dieu, te donne en héritage, tu ne laisseras vivre personne qui respire.
\VS{17}Car tu ne manqueras point de les dévouer par interdit~: Héthiens, Amoréens, Cananéens, Phéréziens, Héviens, et Jébusiens, comme Yahweh, ton Dieu, te l'a ordonné.
\VS{18}Afin qu'ils ne vous enseignent point à faire toutes les abominations qu'ils font pour leurs dieux, et que vous ne péchiez point contre Yahweh, votre Dieu.
\VS{19}Quand tu assiégeras une ville durant plusieurs jours, en lui faisant la guerre pour la saisir, tu ne détruiras point les arbres à coups de hache, tu t'en nourriras et tu ne les couperas point, car l'arbre des champs est-il un homme pour être assiégé par toi~?
\VS{20}Mais seulement tu détruiras et tu couperas les arbres que tu sauras ne point être des arbres fruitiers, et tu construiras des retranchements contre la ville qui te fait la guerre, jusqu'à ce qu'elle tombe.
\Chap{21}
\TextTitle{Lois sur le meurtre anonyme}
\VerseOne{}S'il se trouve sur la terre que Yahweh, ton Dieu, te donne à posséder, un homme tué, étendu dans un champ, sans que l'on sache qui l'a frappé,
\VS{2}tes anciens et tes juges sortiront, et ils mesureront de l'homme tué jusqu'aux villes qui sont autour.
\VS{3}Puis les anciens de la ville la plus proche de l'homme tué prendront une génisse du troupeau qui n'a pas travaillé et qui n'a point tiré au joug.
\VS{4}Et les anciens de cette ville feront descendre cette génisse vers un torrent intarissable, où on ne travaille ni ne sème~; et là, ils briseront la nuque à la génisse dans le torrent.
\VS{5}Et les prêtres, fils de Lévi, s'approcheront~; car Yahweh, ton Dieu, les a choisis pour qu'ils le servent, et qu'ils bénissent au Nom de Yahweh~; et leur bouche doit décider de toute contestation et toute blessure.
\VS{6}Et tous les anciens de cette ville, qui seront les plus proches de l'homme qui aura été tué, laveront leurs mains sur la génisse à laquelle on aura brisé la nuque dans le torrent.
\VS{7}Et prenant la parole, ils diront~: Nos mains n'ont point répandu ce sang et nos yeux ne l'ont point vu.
\VS{8}Ô Yahweh~! Sois propice à ton peuple d'Israël que tu as racheté~; ne lui impute point le sang innocent qui a été répandu au milieu de ton peuple d'Israël~; et le meurtre sera expié pour eux.
\VS{9}Et tu ôteras le sang innocent du milieu de toi, en faisant ce qui est droit aux yeux de Yahweh.
\TextTitle{Lois sur le mariage et l'héritage}
\VS{10}Quand tu iras en guerre contre tes ennemis, que Yahweh, ton Dieu, les aura livrés entre tes mains, et que tu en auras emmené des captifs,
\VS{11}si tu vois parmi les captifs une femme belle de figure, et que tu désires la prendre pour femme,
\VS{12}alors tu la conduiras à l'intérieur de ta maison, et elle rasera sa tête et fera ses ongles,
\VS{13}elle ôtera les vêtements de sa captivité, elle demeurera dans ta maison, et pleurera son père et sa mère durant un mois. Puis tu iras vers elle, tu l'épouseras, et elle sera ta femme.
\VS{14}Si il arrive qu'elle ne te plaise plus, tu la renverras où elle voudra, mais tu ne la vendras certainement pas pour de l'argent ni la traiteras en esclave, parce que tu l'auras humiliée.
\VS{15}Quand un homme, qui a deux femmes, aime l'une et hait l'autre, si celle qu'il aime et celle qu'il hait enfantent des fils, et que le fils aîné est de celle qui est haïe,
\VS{16}alors, le jour où il laissera en héritage ce qu'il aura, il ne pourra pas reconnaître comme premier-né le fils de celle qu'il aime, à la place du fils de celle qui est haïe, et qui est le premier-né.
\VS{17}Mais il reconnaîtra pour premier-né le fils de celle qui est haïe, et il lui donnera la double portion de tout ce qui s'y trouvera être à lui~; car il est le commencement de sa vigueur, le droit d'aînesse lui appartient.
\TextTitle{Le fils indocile sous la loi\FTNTT{cp. Lu. 15:11-23.}}
\VS{18}Si un homme a un fils indocile et rebelle, n'obéissant point à la voix de son père, ni à la voix de sa mère, et qui, bien qu'ils l'aient châtié, ne les écoute point,
\VS{19}alors le père et la mère le prendront et le mèneront aux anciens de sa ville, et à la porte du lieu de sa demeure.
\VS{20}Et ils diront aux anciens de sa ville~: Voici notre fils qui est indocile et rebelle, qui n'obéit point à notre voix, et qui se livre à l'excès et à l'ivrognerie.
\VS{21}Et tous les gens de la ville le lapideront avec des pierres, et il mourra. Tu ôteras le mal du milieu de toi, afin que tout Israël entende et craigne.
\VS{22}Si un homme a commis un péché digne de mort, et qu'on le fait mourir, et que tu l'aies pendu à un bois,
\VS{23}son cadavre ne passera point la nuit sur le bois~; mais tu ne manqueras point de l'ensevelir le même jour, car celui qui est pendu est malédiction de Dieu\FTNT{Ga. 3:13.}, et tu ne souilleras point la terre que Yahweh, ton Dieu, te donne en héritage.
\Chap{22}
\TextTitle{Lois sur la vie en société}
\VerseOne{}Si tu vois le bœuf ou la brebis de ton frère s'égarer, tu ne t'en cacheras point, tu ne manqueras point de les ramener à ton frère.
\VS{2}Si ton frère ne demeure point près de toi, et que tu ne le connais point, tu les recueilleras dans ta maison et il sera chez toi jusqu'à ce que ton frère les cherche~; et alors tu les lui rendras.
\VS{3}Tu feras de même pour son âne, tu feras de même pour son vêtement, et tu feras de même pour tout ce que ton frère aura perdu et que tu trouveras~; tu ne devras point t'en détourner.
\VS{4}Si tu vois l'âne de ton frère ou son bœuf tombé dans le chemin, tu ne t'en détourneras point, et tu ne manqueras point de le relever.
\VS{5}La femme ne portera point l'habit d'un homme ni l'homme ne se vêtira point d'un habit de femme~; car celui qui fait ces choses est en abomination à Yahweh, ton Dieu\FTNT{Dans ce passage, Yahweh condamne le travestisme. Cette pratique était répandue chez les Cananéens. Le travestisme consiste à adopter le comportement, les habitudes sociales et la tenue vestimentaire du sexe opposé dans le but de lui ressembler.}.
\VS{6}Si tu rencontres sur le chemin, sur un arbre ou sur la terre, un nid d'oiseaux, ayant des petits ou des œufs, et la mère couchée sur les petits ou les œufs, tu ne prendras point la mère et les petits,
\VS{7}mais tu ne manqueras point de laisser aller la mère et tu ne prendras que les petits, afin que tu sois heureux et que tu prolonges tes jours.
\VS{8}Si tu bâtis une maison neuve, tu feras un parapet tout autour de ton toit, afin que tu ne mettes point de sang sur ta maison, si quelqu'un tombait de là.
\TextTitle{Lois sur les mélanges}
\VS{9}Tu ne sèmeras point dans ta vigne diverses sortes de grains~; de peur que le tout, à savoir les grains, que tu auras semés, et le rapport de ta vigne, ne soit souillé. 
\VS{10}Tu ne laboureras point avec un âne et un bœuf ensemble.
\VS{11}Tu ne te vêtiras point d'un tissu mélangé de laine et de lin ensemble.
\VS{12}Tu te feras des franges aux quatre pans du vêtement dont tu te couvriras.
\TextTitle{Lois sur la virginité, l'adultère et la fidélité}
\VS{13}Si un homme a pris une femme et est allé vers elle, et qu'il la haïsse,
\VS{14}et qu'il lui impute des choses qui donnent l'occasion de parler d'elle et de la diffamer, en disant~: J'ai pris cette femme, et quand je me suis approché d'elle, je ne l'ai point trouvé vierge,
\VS{15}alors le père et la mère de la jeune femme prendront et produiront les signes de la virginité de la jeune femme devant les anciens de la ville, à la porte.
\VS{16}Et le père de la jeune femme dira aux anciens~: J'ai donné ma fille à cet homme pour femme, et il l'a haïe~;
\VS{17}et voici, il lui impute des choses qui lui donnent l'occasion de parler d'elle, disant~: Je n'ai point trouvé ta fille vierge. Cependant, voici les signes de la virginité de ma fille. Et ils étendront le drap devant les anciens de la ville.
\VS{18}Alors les anciens de la ville prendront le mari, et le châtieront~;
\VS{19}et parce qu'il aura répandu une mauvaise réputation sur une vierge d'Israël, ils le condamneront à une amende de cent sicles d'argent, qu'ils donneront au père de la jeune femme. Elle sera sa femme, et il ne pourra pas la répudier, tant qu'il vivra.
\VS{20}Mais si la chose est vraie, si la jeune femme ne s'est point trouvée vierge,
\VS{21}alors ils feront sortir la jeune femme à l'entrée de la maison de son père~; les gens de sa ville la lapideront de pierres et elle mourra, car elle a commis une infamie en Israël, en se prostituant dans la maison de son père. Tu ôteras le mal du milieu de toi.
\VS{22}Si l'on trouve un homme couché avec une femme mariée, ils mourront tous les deux, l'homme qui a couché avec la femme, et la femme aussi. Tu ôteras ainsi le mal d'Israël.
\VS{23}Si une jeune fille vierge est fiancée à un homme, et qu'un homme la rencontre dans la ville, et couche avec elle,
\VS{24}vous les conduirez tous deux à la porte de la ville, vous les lapiderez de pierres, et ils mourront~; la jeune fille, parce qu'elle n'a point crié étant dans la ville, et l'homme parce qu'il a humilié la femme de son prochain. Tu ôteras le mal du milieu de toi.
\VS{25}Si l'homme rencontre dans les champs la jeune fille fiancée, et que l'homme lui fait violence et couche avec elle, alors l'homme qui aura couché avec elle mourra lui seul.
\VS{26}Mais tu ne feras rien à la jeune fille~; la jeune fille n'a point commis de péché digne de mort, car c'est comme si un homme s'élevait contre son prochain et lui ôtait la vie.
\VS{27}Parce que l'ayant trouvée dans les champs, la jeune fille fiancée a pu crier, sans que personne ne l'ait délivrée.
\VS{28}Si un homme rencontre une jeune fille vierge non fiancée, lui fait violence et couche avec elle, et qu'ils soient découverts,
\VS{29}l'homme qui aura couché avec elle donnera au père de la jeune fille cinquante sicles d'argent~; et il la prendra pour femme, parce qu'il l'a humiliée, et il ne pourra point la répudier, tant qu'il vivra.
\VS{30}Un homme ne prendra point la femme de son père ni ne découvrira le pan de la robe de son père.
\Chap{23}
\TextTitle{Lois sur l'accès à l'assemblée de Yahweh}
\VerseOne{}Celui dont les testicules ont été écrasés ou l'urètre coupé n'entrera point dans l'assemblée de Yahweh.
\VS{2}Le bâtard\FTNT{Le mot bâtard, «~mamzer~» en hébreu, désigne l'enfant illégitime, celui issu de l'inceste, celui né d'une population mélangée ou d'un père Juif et d'une mère païenne, et inversement.} n'entrera point dans l'assemblée de Yahweh~; même sa dixième génération n'entrera point dans l'assemblée de Yahweh.
\VS{3}L'Ammonite et le Moabite n'entreront point dans l'assemblée de Yahweh, même leur dixième génération, à jamais,
\VS{4}parce qu'ils ne sont point venus à votre rencontre avec du pain et de l'eau, sur le chemin, lorsque vous sortiez d'Egypte, et parce qu'ils ont engagé à prix d'argent contre vous Balaam, fils de Beor, de Pethor en Mésopotamie, pour qu'il vous maudisse.
\VS{5}Mais Yahweh, ton Dieu, n'a point voulu écouter Balaam~; et Yahweh, ton Dieu, a changé la malédiction en bénédiction, parce que Yahweh, ton Dieu, t'aime.
\VS{6}Tu ne chercheras jamais, tant que tu vivras, leur paix ni leur bien.
\VS{7}Tu n'auras point en abomination l'Edomite, car il est ton frère~; tu n'auras point en abomination l'Egyptien, car tu as été étranger dans son pays~:
\VS{8}Les enfants qui leur naîtront à la troisième génération entreront dans l'assemblée de Yahweh.
\TextTitle{La sainteté et la justice dans le camp de Yahweh}
\VS{9}Quand le camp sortira contre tes ennemis, garde-toi de toute chose mauvaise.
\VS{10}S'il y a parmi vous un homme qui ne soit point pur, par suite d'un accident nocturne, il sortira hors du camp, et n'entrera point dans le camp.
\VS{11}Et sur le soir, il se lavera dans l'eau, et dès que le soleil sera couché, il rentrera dans le camp.
\VS{12}Tu auras un endroit hors du camp, et tu sortiras là dehors.
\VS{13}Tu auras un pieu parmi tes bagages, et quand tu voudras aller dehors, tu creuseras, puis tu recouvriras tes excréments.
\VS{14}Car Yahweh, ton Dieu, marche au milieu de ton camp pour te délivrer et pour livrer tes ennemis devant toi~; que tout ton camp soit saint, afin qu'il ne voie chez toi aucune chose honteuse, et qu'il ne se détourne point de toi.
\VS{15}Tu ne livreras point à son maître l'esclave qui se sera sauvé chez toi d'auprès de son maître.
\VS{16}Il demeurera avec toi, au milieu de toi, dans le lieu qu'il choisira, dans l'une de tes villes, là où bon lui semblera~: Tu ne l'opprimeras point.
\VS{17}Il n'y aura, parmi les filles d'Israël, aucune prostituée, et il n'y aura, parmi les fils d'Israël, aucun qui se prostitue.
\VS{18}Tu n'apporteras point dans la maison de Yahweh, ton Dieu, le salaire d'une prostituée, ni le prix d'un chien, pour quelque vœu que ce soit~; car tous les deux sont en abomination devant Yahweh, ton Dieu.
\VS{19}Tu n'exigeras aucun intérêt à ton frère, ni intérêt pour de l'argent, ni intérêt pour des vivres, ni intérêt pour quelque chose que ce soit que l'on prête avec intérêt.
\VS{20}Tu prêteras avec intérêt à l'étranger, mais tu ne prêteras point avec intérêt à ton frère, afin que Yahweh, ton Dieu, te bénisse dans tout ce que ta main entreprendra dans le pays où tu vas entrer en possession.
\TextTitle{Vœux faits à Yahweh}
\VS{21}Si tu fais un vœu à Yahweh, ton Dieu, tu ne tarderas point à l'accomplir, car Yahweh, ton Dieu, ne manquerait point de te le redemander, ainsi il y aurait du péché en toi.
\VS{22}Mais si tu t'abstiens de faire un vœu, il n'y aura pas de péché en toi.
\VS{23}Mais tu prendras garde de faire ce qui sortira de tes lèvres, l'offrande volontaire que tu auras vouée à Yahweh, ton Dieu, et que ta bouche aura prononcée.
\TextTitle{Lois diverses}
\VS{24}Si tu entres dans la vigne de ton prochain, tu pourras manger des raisins selon ton appétit, jusqu'à en être rassasié~; mais tu n'en mettras point dans ton vase.
\VS{25}Si tu entres dans les blés de ton prochain, tu pourras arracher des épis avec ta main~; mais tu n'agiteras point la faucille sur les blés de ton prochain.
\Chap{24}
\TextTitle{Loi sur le divorce}
\VerseOne{}Quand un homme aura pris et épousé une femme, s'il arrive qu'elle ne trouve pas grâce à ses yeux, parce qu'il aura trouvé en elle quelque chose de honteux, il lui écrira une lettre de divorce, et après la lui avoir remise en main, il la renverra de sa maison.
\VS{2}Elle sortira de sa maison, s'en ira, et elle pourra devenir la femme d'un autre homme.
\VS{3}Si ce dernier homme la hait, écrit une lettre de divorce, la lui donne dans sa main, et la renvoie de sa maison, ou que ce dernier homme qui l'a prise pour femme, meure,
\VS{4}alors son premier mari qui l'avait renvoyée ne pourra pas la reprendre pour femme après avoir été souillée, car c'est une abomination devant Yahweh, ainsi tu ne feras point pécher le pays que Yahweh, ton Dieu, te donne en héritage.
\TextTitle{Lois diverses sur l'organisation de la société}
\VS{5}Quand un homme aura nouvellement épousé une femme, il n'ira point à la guerre, et on ne lui imposera aucune charge~; il en sera libre pour sa maison pendant un an, et il réjouira la femme qu'il a prise.
\VS{6}On ne prendra point pour gage les deux meules, pas même la meule de dessus~; parce qu'on prendrait pour gage la vie.
\VS{7}Si l'on trouve un homme qui ait dérobé l'un de ses frères, l'un des enfants d'Israël, qui en ait fait son esclave ou qui l'ait vendu, ce voleur mourra. Tu ôteras le mal du milieu de toi.
\VS{8}Prends garde à la plaie de la lèpre, afin de bien observer et de faire tout ce que les prêtres, les Lévites, vous enseigneront~; vous prendrez garde de faire selon ce que je leur ai ordonné.
\VS{9}Souviens-toi de ce que Yahweh, ton Dieu, fit à Marie, en chemin, après votre sortie d'Egypte.
\TextTitle{Lois en faveur des nécessiteux}
\VS{10}Lorsque tu feras à ton prochain un prêt quelconque, tu n'entreras point dans sa maison pour prendre son gage~;
\VS{11}mais tu te tiendras dehors, et l'homme à qui tu feras le prêt t'apportera le gage dehors.
\VS{12}Si cet homme est pauvre, tu ne te coucheras point ayant encore son gage~;
\VS{13}tu ne manqueras point de lui rendre le gage dès que le soleil sera couché, afin qu'il se couche dans son vêtement et qu'il te bénisse~; et cela te sera imputé à justice devant Yahweh, ton Dieu.
\VS{14}Tu n'opprimeras point le mercenaire, le pauvre et l'indigent, d'entre tes frères, ou d'entre les étrangers qui demeurent dans ton pays, dans tes portes.
\VS{15}Tu lui donneras son salaire le jour même avant que le soleil se couche~; car il est pauvre, et son désir s'y porte. Afin qu'il ne crie point contre toi à Yahweh, et que tu ne pèches point.
\VS{16}On ne fera point mourir les pères pour les fils, et on ne fera point mourir les fils pour les pères~; mais on fera mourir chacun pour son péché.
\VS{17}Tu ne feras pas d'injustice à l'étranger ni à l'orphelin, et tu ne prendras point en gage le vêtement de la veuve.
\VS{18}Et tu te souviendras que tu as été esclave en Egypte, et que Yahweh, ton Dieu, t'a racheté de là~; c'est pourquoi je t'ordonne de faire ces choses.
\VS{19}Quand tu moissonneras dans ton champ, et que tu auras oublié une gerbe dans ton champ, tu ne retourneras point la prendre~: Elle sera pour l'étranger, pour l'orphelin et pour la veuve, afin que Yahweh, ton Dieu, te bénisse dans toute l'œuvre de tes mains.
\VS{20}Quand tu secoueras tes oliviers, tu n'y retourneras point pour cueillir ce qui reste aux branches~: Ce sera pour l'étranger, pour l'orphelin et pour la veuve.
\VS{21}Quand tu vendangeras ta vigne, tu ne grappilleras point après~: Ce sera pour l'étranger, pour l'orphelin et pour la veuve.
\VS{22}Et tu te souviendras que tu as été esclave dans le pays d'Egypte~; c'est pourquoi je t'ordonne de faire ces choses.
\Chap{25}
\TextTitle{Le juste justifié et le méchant condamné}
\VerseOne{}Quand il y aura un différend entre des hommes et qu'ils viendront en jugement afin qu'on les juge, on justifiera le juste, et on condamnera le méchant.
\VS{2}Si le méchant mérite d'être battu, le juge le fera jeter par terre et frapper en sa présence par un certain nombre de coups, selon l'exigence de son crime.
\VS{3}Il le fera battre de quarante coups, pas plus, de peur que si l'on continuait à le frapper avec plus de coups, ton frère ne soit méprisé à tes yeux.
\VS{4}Tu n'emmuselleras point ton bœuf lorsqu'il foulera le grain.
\TextTitle{Loi sur la continuité de la postérité}
\VS{5}Quand des frères demeureront ensemble, et que l'un d'entre eux mourra sans fils, alors la femme du défunt ne se mariera point dehors avec un homme qui est étranger, mais son beau-frère viendra vers elle, la prendra pour femme, et l'épousera comme son beau-frère.
\VS{6}Et le premier-né qu'elle enfantera succédera au frère mort et portera son nom, afin que son nom ne soit point effacé d'Israël.
\VS{7}Et s'il ne plaît pas à cet homme-là de prendre sa belle-sœur, alors sa belle-sœur montera à la porte vers les anciens\FTNT{Ru. 4:1-10}, et dira~: Mon beau-frère refuse de relever le nom de son frère en Israël, et ne veut point m'épouser par droit de beau-frère.
\VS{8}Alors les anciens de la ville l'appelleront et lui parleront. S'il demeure ferme, et qu'il dit~: Il ne me plaît point de la prendre,
\VS{9}alors sa belle-sœur s'approchera de lui à la vue des anciens, lui ôtera son soulier du pied, et lui crachera au visage. Et prenant la parole, elle dira~: C'est ainsi qu'on fera à l'homme qui ne bâtit point la maison de son frère.
\VS{10}Et son nom sera appelé en Israël la maison du déchaussé.
\TextTitle{L'abomination sévèrement et justement punie}
\VS{11}Quand des hommes se querelleront ensemble, l'un contre l'autre, si la femme de l'un s'approche pour délivrer son mari de la main de celui qui le frappe, et qu'étendant sa main elle saisisse ses parties intimes,
\VS{12}tu lui couperas la main, et ton œil ne l'épargnera point.
\VS{13}Tu n'auras point dans ton sac deux poids différents, un grand et un petit.
\VS{14}Il n'y aura point dans ta maison deux épha différents, un grand et un petit\FTNT{Lé. 19:35-37.}.
\VS{15}Mais tu auras un poids exact et juste, tu auras un épha exact et juste, afin que tes jours se prolongent sur la terre que Yahweh, ton Dieu, te donne.
\VS{16}Car celui qui fait ces choses, celui qui commet une injustice, est en abomination à Yahweh, ton Dieu.
\TextTitle{Yahweh confirme le sort d'Amalek}
\VS{17}Souviens-toi ce que te fit Amalek en chemin, quand vous sortiez d'Egypte\FTNT{Ex. 17:8.},
\VS{18}comment il est venu te rencontrer sur le chemin, et, sans aucune crainte de Dieu, attaqua par derrière ceux qui étaient fatigués, quand toi-même tu étais épuisé.
\VS{19}Quand Yahweh, ton Dieu, t'aura accordé du repos de tous tes ennemis qui t'entourent, dans le pays que Yahweh, ton Dieu, te donne en héritage afin que tu le possèdes, alors tu effaceras la mémoire d'Amalek de dessous les cieux~: Ne l'oublie point.
\Chap{26}
\TextTitle{La loi des prémices\FTNTT{cp. Ex. 23:16-19.}}
\VerseOne{}Quand tu seras entré dans le pays que Yahweh, ton Dieu, te donne en héritage, et quand tu le posséderas et y habiteras,
\VS{2}alors tu prendras des prémices de tous les fruits que tu retireras de la terre dans le pays que Yahweh, ton Dieu, te donne~; tu les mettras dans une corbeille, et tu iras au lieu que Yahweh, ton Dieu, choisira pour y faire habiter son Nom\FTNT{Ex. 23:16-19.}.
\VS{3}Et tu viendras vers le prêtre qui sera en ce temps-là, et tu lui diras~: Je déclare aujourd'hui à Yahweh, ton Dieu, que je suis entré dans le pays que Yahweh a juré à nos pères de nous donner.
\VS{4}Et le prêtre prendra la corbeille de ta main, et la posera devant l'autel de Yahweh, ton Dieu.
\VS{5}Puis tu prendras la parole, et tu diras devant Yahweh, ton Dieu~: Mon père était un Araméen qui périssait, il descendit en Egypte avec un petit nombre de gens, il y séjourna et il y devint une nation grande, puissante, et nombreuse.
\VS{6}Puis les Egyptiens nous maltraitèrent, nous humilièrent, et nous imposèrent une dure servitude.
\VS{7}Nous criâmes à Yahweh, le Dieu de nos pères. Yahweh entendit notre voix, et il vit notre souffrance, notre travail, et notre oppression.
\VS{8}Et Yahweh nous fit sortir d'Egypte, à main forte et à bras étendu, avec une grande frayeur, avec des signes et des miracles.
\VS{9}Et il nous a conduits dans ce lieu, et nous a donné ce pays où coulent le lait et le miel.
\VS{10}Maintenant donc voici, j'apporte les prémices des fruits de la terre que tu m'as donnée, ô Yahweh~! Tu les poseras devant Yahweh, ton Dieu, et tu te prosterneras devant Yahweh, ton Dieu.
\VS{11}Et tu te réjouiras de tout le bien que Yahweh, ton Dieu, t'aura donné, et à ta maison, toi et le Lévite, et l'étranger qui sera au milieu de toi.
\VS{12}Quand tu auras achevé de lever toute la dîme de ta récolte, la troisième année, l'année de la dîme, tu la donneras au Lévite, à l'étranger, à l'orphelin, et à la veuve~; ils en mangeront dans tes portes, et ils en seront rassasiés.
\VS{13}Tu diras en la présence de Yahweh, ton Dieu~: J'ai fait disparaître de ma maison ce qui est consacré, et je l'ai donné au Lévite, à l'étranger, à l'orphelin, et à la veuve, selon tous tes commandements que tu m'as ordonnés~; je n'ai transgressé ni oublié aucun de tes commandements.
\VS{14}Je n'en ai point mangé dans mon affliction, et je n'en ai rien fait disparaître pour un usage impur, et je n'en ai point donné pour un mort~; j'ai obéi à la voix de Yahweh, mon Dieu~; j'ai fait selon tout ce que tu m'avais ordonné.
\VS{15}Regarde de ta sainte demeure, des cieux, et bénis ton peuple d'Israël et la terre que tu nous as donnée, comme tu l'avais juré à nos pères, pays où coulent le lait et le miel.
\VS{16}Aujourd'hui, Yahweh, ton Dieu, t'ordonne de mettre en pratique ces lois et ces ordonnances~; prends garde de les faire de tout ton cœur et de toute ton âme.
\VS{17}Tu as fait promettre aujourd'hui à Yahweh qu'il sera ton Dieu, pour que tu marches dans ses voies, que tu observes ses lois, ses commandements et ses ordonnances, et que tu obéisses à sa voix.
\VS{18}Et aujourd'hui, Yahweh t'a fait promettre que tu seras un peuple précieux, comme il te l'a dit, et que tu observeras tous ses commandements,
\VS{19}pour qu'il te donne sur toutes les nations qu'il a créées la supériorité en louange, en renom, et en beauté, et pour que tu sois un peuple saint à Yahweh, ton Dieu, comme il te l'a dit.
\Chap{27}
\TextTitle{La loi gravée sur des pierres au mont Ebal}
\VerseOne{}Or Moïse et les anciens d'Israël ordonnèrent au peuple, en disant~: Gardez tous les commandements que je vous ordonne aujourd'hui.
\VS{2}Le jour où vous aurez traversé le Jourdain, pour entrer dans le pays que Yahweh, ton Dieu, te donne, tu dresseras de grandes pierres, et tu les enduiras de chaux.
\VS{3}Puis tu écriras sur elles toutes les paroles de cette loi, quand tu auras traversé le Jourdain, pour entrer dans le pays que Yahweh, ton Dieu, te donne, pays où coulent le lait et le miel, comme te l'a dit Yahweh, le Dieu de tes pères.
\VS{4}Quand donc vous aurez traversé le Jourdain, vous dresserez ces pierres-là sur le mont Ebal, selon ce que je vous ordonne aujourd'hui, et tu les enduiras de chaux.
\VS{5}Tu bâtiras aussi là un autel à Yahweh, ton Dieu~; un autel, dis-je, de pierres, sur lesquelles tu ne lèveras point le fer.
\VS{6}Tu bâtiras l'autel de Yahweh, ton Dieu, de pierres entières. Tu y offriras des holocaustes à Yahweh, ton Dieu~;
\VS{7}tu y offriras aussi des offrandes de paix\FTNT{Voir commentaire en Lé. 3:1.}, et tu mangeras là et te réjouiras devant Yahweh, ton Dieu.
\VS{8}Et tu écriras sur ces pierres toutes les paroles de cette loi, en les gravant bien distinctement.
\TextTitle{Les malédictions prononcées sur le mont Ebal}
\VS{9}Et Moïse et les prêtres, les Lévites, parlèrent à tout Israël, en disant~: Ecoute et garde le silence, Israël~! Aujourd'hui, tu es devenu le peuple de Yahweh, ton Dieu.
\VS{10}Tu obéiras à la voix de Yahweh, ton Dieu, et tu feras ses commandements et ses lois que je t'ordonne aujourd'hui.
\VS{11}Moïse ordonna au peuple ce jour-là, disant~:
\VS{12}Quand vous aurez traversé le Jourdain, Siméon, Lévi, Juda, Issacar, Joseph, et Benjamin, se tiendront sur le mont Garizim, pour bénir le peuple~;
\VS{13}et Ruben, Gad, Aser, Zabulon, Dan et Nephthali, se tiendront sur le mont Ebal, pour maudire.
\VS{14}Et les Lévites prendront la parole, et diront à haute voix à tous les hommes d'Israël~:
\VS{15}Maudit soit l'homme qui fait une image taillée ou une image en métal fondu, car c'est une abomination à Yahweh, œuvre des mains d'un artisan, et qui la met dans un lieu secret~! Et tout le peuple répondra, et dira~: Amen~!
\VS{16}Maudit soit celui qui méprise son père et sa mère~! Et tout le peuple dira~: Amen~!
\VS{17}Maudit soit celui qui déplace les bornes de son prochain~! Et tout le peuple dira~: Amen~!
\VS{18}Maudit soit celui qui égare un aveugle dans le chemin~! Et tout le peuple dira~: Amen~!
\VS{19}Maudit soit celui qui fait injustice à l'étranger, à l'orphelin, et à la veuve~! Et tout le peuple dira~: Amen~!
\VS{20}Maudit soit celui qui couche avec la femme de son père, car il découvre le pan de la robe de son père~! Et tout le peuple dira~: Amen~!
\VS{21}Maudit soit celui qui couche avec une bête~! Et tout le peuple dira~: Amen~!
\VS{22}Maudit soit celui qui couche avec sa sœur, fille de son père, ou fille de sa mère~! Et tout le peuple dira~: Amen~!
\VS{23}Maudit soit celui qui couche avec sa belle-mère~! Et tout le peuple dira~: Amen~!
\VS{24}Maudit soit celui qui frappe son prochain en secret~! Et tout le peuple dira~: Amen~!
\VS{25}Maudit soit celui qui reçoit un présent pour mettre à mort un homme, en versant le sang innocent~! Et tout le peuple dira~: Amen~!
\VS{26}Maudit soit celui qui n'accomplit point les paroles de cette loi et ne les met pas en pratique~! Et tout le peuple dira~: Amen~!
\Chap{28}
\TextTitle{Les bénédictions accompagnent l'obéissance}
\VerseOne{}Or il arrivera que si tu écoutes attentivement la voix de Yahweh, ton Dieu, et que tu prennes garde de pratiquer tous ses commandements que je t'ordonne aujourd'hui, Yahweh, ton Dieu, te donnera la supériorité sur toutes les nations de la terre.
\VS{2}Voici toutes les bénédictions qui viendront sur toi, et qui t'atteindront, quand tu obéiras à la voix de Yahweh, ton Dieu~:
\VS{3}Tu seras béni dans la ville, et tu seras aussi béni aux champs.
\VS{4}Le fruit de tes entrailles, le fruit de ta terre, le fruit de tes troupeaux, les portées de ton gros et de ton menu bétail seront bénis.
\VS{5}Ta corbeille et ta huche seront bénies.
\VS{6}Tu seras béni en entrant, et tu seras béni en sortant.
\VS{7}Yahweh fera que tes ennemis qui s'élèveront contre toi seront battus devant toi, ils sortiront contre toi par un chemin, et ils s'enfuiront devant toi par sept chemins.
\VS{8}Yahweh ordonnera à la bénédiction d'être avec toi dans tes greniers et dans tout ce à quoi tu mettras ta main~; il te bénira dans le pays que Yahweh, ton Dieu, te donne.
\VS{9}Yahweh t'établira pour lui être un peuple saint, comme il te l'a juré, quand tu garderas les commandements de Yahweh, ton Dieu, et que tu marcheras dans ses voies.
\VS{10}Et tous les peuples de la terre verront que tu es appelé du Nom de Yahweh, et ils te craindront.
\VS{11}Yahweh te fera abonder de biens dans le fruit de tes entrailles, le fruit de tes troupeaux, et le fruit de ton sol, sur la terre que Yahweh a juré à tes pères de te donner.
\VS{12}Yahweh t'ouvrira son bon trésor, les cieux, pour donner à ton pays la pluie en sa saison et pour bénir tout le travail de tes mains~; tu prêteras à beaucoup de nations, et tu n'emprunteras point.
\VS{13}Yahweh te mettra à la tête et non à la queue, tu seras toujours en haut et jamais en bas, lorsque tu obéiras aux commandements de Yahweh, ton Dieu, que je t'ordonne aujourd'hui, afin que tu prennes garde de les faire,
\VS{14}et que tu ne te détournes ni à droite ni à gauche de toutes les paroles que je t'ordonne aujourd'hui, pour aller après d'autres dieux et pour les servir.
\TextTitle{Les malédictions accompagnent la désobéissance}
\VS{15}Mais si tu n'obéis point à la voix de Yahweh, ton Dieu, pour prendre garde de pratiquer tous ses commandements et ses lois que je t'ordonne aujourd'hui, voici toutes les malédictions qui viendront sur toi, et qui t'atteindront~:
\VS{16}Tu seras maudit dans la ville, et tu seras maudit dans les champs.
\VS{17}Ta corbeille et ta huche seront maudites.
\VS{18}Le fruit de tes entrailles, le fruit de ta terre, les portées de ton gros et de ton menu bétail seront maudits.
\VS{19}Tu seras maudit à ton entrée, et tu seras maudit à ta sortie.
\VS{20}Yahweh enverra sur toi la malédiction, la confusion, et la ruine dans tout ce à quoi tu mettras ta main et que tu feras, jusqu'à ce que tu sois détruit, et que tu périsses promptement, à cause de la méchanceté de tes pratiques, par lesquelles tu m'auras abandonné.
\VS{21}Yahweh fera que la peste s'attachera à toi, jusqu'à ce qu'elle te consume sur la terre où tu vas entrer pour en prendre possession.
\VS{22}Yahweh te frappera de tuberculose, de fièvre, d'inflammation, de chaleur brûlante, de l'épée, de sécheresse et de rouille, qui te poursuivront jusqu'à ce que tu périsses.
\VS{23}Les cieux sur ta tête seront d'airain, et la terre sous toi sera de fer.
\VS{24}Yahweh te donnera pour pluie à ton pays de la poussière et de la poudre, qui descendra des cieux sur toi jusqu'à ce que tu sois détruit.
\VS{25}Yahweh fera que tu seras battu devant tes ennemis~; tu sortiras par un chemin contre eux, et tu t'enfuiras devant eux par sept chemins~; et tu seras tremblant face à tous les royaumes de la terre.
\VS{26}Ton cadavre sera la viande de tous les oiseaux des cieux et des bêtes de la terre~; et il n'y aura personne qui les effraye.
\VS{27}Yahweh te frappera de l'ulcère d'Egypte, d'hémorroïdes, de gale, et de teigne, dont tu ne pourras guérir.
\VS{28}Yahweh te frappera de folie, d'aveuglement, et d'égarement d'esprit~;
\VS{29}et tu tâtonneras en plein midi comme tâtonne un aveugle dans l'obscurité, tu ne prospéreras pas dans tes voies, et tu seras opprimé et dépouillé tous les jours, et il n'y aura personne pour venir te sauver.
\VS{30}Tu fianceras une femme, mais un autre homme couchera avec elle et la violera~; tu bâtiras une maison, mais tu ne l'habiteras point~; tu planteras une vigne, mais tu n'en jouiras point.
\VS{31}Ton bœuf sera tué sous tes yeux, et tu n'en mangeras point~; ton âne sera enlevé devant toi, et on ne te le rendra point~; tes brebis seront livrées à tes ennemis, et il n'y aura personne pour te sauver.
\VS{32}Tes fils et tes filles seront livrés à un autre peuple, tes yeux le verront, et languiront tout le jour après eux, et tu n'auras aucun pouvoir en ta main.
\VS{33}Un peuple que tu n'auras point connu mangera le fruit de ta terre et tout ton travail, et tu seras opprimé et écrasé tous les jours.
\VS{34}Tu deviendras fou à cause de ce que tu verras de tes yeux.
\VS{35}Yahweh te frappera d'un ulcère malin sur les genoux et sur les cuisses dont tu ne pourras guérir, il t'en frappera depuis la plante du pied jusqu'au sommet de ta tête.
\VS{36}Yahweh te fera marcher, toi et ton roi que tu auras établi sur toi, vers une nation que tu n'auras point connue, ni toi ni tes pères. Et là, tu serviras d'autres dieux, du bois et de la pierre.
\VS{37}Et tu seras un sujet d'étonnement, de proverbes, de railleries, parmi tous les peuples vers lesquels Yahweh t'aura emmené.
\VS{38}Tu jetteras beaucoup de semence dans ton champ, et tu recueilleras peu, car les sauterelles la consumeront.
\VS{39}Tu planteras des vignes et tu les cultiveras~; mais tu n'en boiras point le vin et tu n'en recueilleras rien, car les vers la mangeront.
\VS{40}Tu auras des oliviers sur tout le territoire~; mais tu ne t'oindras point d'huile, car tes olives perdront leurs fruits.
\VS{41}Tu engendreras des fils et des filles~; mais ils ne seront pas à toi, car ils iront en captivité.
\VS{42}Les insectes posséderont tous tes arbres et le fruit de ta terre.
\VS{43}L'étranger qui sera au milieu de toi montera toujours plus au-dessus de toi, et toi, tu descendras toujours plus bas.
\VS{44}Il te prêtera, et tu ne lui prêteras point~; il sera à la tête, et tu seras à la queue.
\VS{45}Toutes ces malédictions viendront sur toi, elles te poursuivront et t'atteindront jusqu'à ce que tu sois détruit, parce que tu n'auras pas obéi à la voix de Yahweh, ton Dieu, pour garder ses commandements et ses lois qu'il t'a ordonnés.
\VS{46}Et ces choses seront à jamais pour toi et ta postérité comme des signes et des prodiges.
\VS{47}Et parce que tu n'auras pas servi Yahweh, ton Dieu, avec joie, et de bon cœur, malgré l'abondance de toutes choses,
\VS{48}tu serviras, dans la faim, dans la soif, dans la nudité, et dans la disette de toutes choses, ton ennemi que Yahweh enverra contre toi. Il mettra un joug de fer sur ton cou, jusqu'à ce qu'il t'ait détruit.
\TextTitle{Prophétie sur l'invasion babylonienne et la dispersion d'Israël}
\VS{49}Yahweh fera lever de loin, des extrémités de la terre, une nation qui volera comme l'aigle, une nation dont tu ne comprendras pas la langue,
\VS{50}une nation au visage féroce, et qui ne soutiendra point le vieillard et n'aura point pitié pour l'enfant\FTNT{Cette prophétie s'est accomplie en 587 av. J.-C. Voir 2 R. 24-25.}.
\VS{51}Elle mangera le fruit de tes troupeaux et les fruits de ta terre, jusqu'à ce que tu sois détruit~; elle n'épargnera ni blé, ni vin, ni huile, ni portée de ton gros et de ton menu bétail, jusqu'à ce qu'elle t'ait fait périr.
\VS{52}Et elle t'assiégera dans toutes tes portes, jusqu'à ce que tombent ces hautes et fortes murailles dans lesquelles tu auras mis ta confiance dans tout ton pays~; elle t'assiégera, dis-je, dans toutes tes portes, dans tout le pays que Yahweh, ton Dieu, te donne.
\VS{53}Tu mangeras le fruit de tes entrailles, la chair de tes fils et de tes filles que Yahweh, ton Dieu, t'aura donnés, durant le siège et la détresse dont ton ennemi te serrera.
\VS{54}L'homme le plus tendre et le plus délicat d'entre vous regardera d'un œil malin son frère, sa femme bien-aimée, et le reste de ses fils qu'il a épargnés~;
\VS{55}pour ne donner à aucun d'eux de la chair de ses fils, qu'il mangera, parce qu'il ne lui restera rien du tout, à cause du siège et de la détresse dont ton ennemi te serrera dans toutes tes portes.
\VS{56}La femme la plus tendre et la plus délicate d'entre vous, qui n'a point osé mettre la plante de son pied sur la terre, par délicatesse et par mollesse, regardera d'un œil malin son mari bien-aimé, son fils, et sa fille~;
\VS{57}et le placenta qui sortira d'entre ses jambes, et les fils qu'elle enfantera~; car manquant de tout, elle les mangera secrètement, à cause du siège et de la détresse, dont ton ennemi te serrera dans toutes les villes.
\VS{58}Si tu ne prends pas garde d'observer toutes les paroles de cette loi, qui sont écrites dans ce livre, en craignant le Nom glorieux et redoutable de Yahweh, ton Dieu,
\VS{59}alors Yahweh rendra difficile tes plaies et les plaies de ta postérité, par des plaies grandes et persistantes, des maladies malignes et persistantes. 
\VS{60}Et il fera retourner sur toi toutes les maladies d'Egypte, devant lesquelles tu avais peur~; et elles s'attacheront à toi.
\VS{61}Même Yahweh fera venir sur toi toutes maladies et toutes plaies, qui ne sont point écrites dans le livre de cette loi, jusqu'à ce que tu sois détruit.
\VS{62}Et vous resterez en petit nombre, après avoir été aussi nombreux que les étoiles des cieux, parce que tu n'auras point obéi à la voix de Yahweh, ton Dieu.
\VS{63}Et il arrivera que comme Yahweh s'est réjoui sur vous, en vous faisant du bien et en vous multipliant, de même Yahweh se réjouira sur vous en vous faisant périr et en vous détruisant~; et vous serez arrachés de la terre dans laquelle vous allez entrer en possession.
\VS{64}Et Yahweh te dispersera parmi tous les peuples, d'un bout de la terre jusqu'à l'autre~; et là, tu serviras d'autres dieux que ni toi ni tes pères n'avez connus, le bois et la pierre.
\VS{65}Tu n'auras aucun repos parmi ces nations, même la plante de ton pied n'aura aucun repos. Car Yahweh te donnera un cœur tremblant, des yeux languissants, et une âme souffrante.
\VS{66}Et ta vie sera en suspens devant toi, tu trembleras la nuit et le jour, et tu ne seras point sûr de ta vie.
\VS{67}Tu diras le matin~: Qui me fera voir le soir~? Et le soir tu diras~: Qui me fera voir le matin~? A cause de l'effroi dont ton cœur sera effrayé, et à cause des choses que tu verras de tes yeux.
\VS{68}Et Yahweh te fera retourner en Egypte sur des navires, pour faire le chemin dont je t'ai dit~: Tu ne le verras plus~; et là, vous vous vendrez à vos ennemis, comme esclaves et servantes~; et il n'y aura personne pour vous acheter.
\Chap{29}
\TextTitle{Yahweh rappelle sa fidélité à Israël}
\VerseOne{}Voici les paroles de l'alliance que Yahweh ordonna à Moïse de traiter avec les enfants d'Israël au pays de Moab, outre l'alliance qu'il avait traitée avec eux à Horeb.
\VS{2}Moïse donc appela tout Israël, et leur dit~: Vous avez vu tout ce que Yahweh a fait sous vos yeux, dans le pays d'Egypte, à Pharaon, à tous ses serviteurs, et à tout son pays,
\VS{3}les grandes épreuves que tes yeux ont vues, ces signes et ces grands miracles.
\VS{4}Mais, jusqu'à ce jour, Yahweh ne vous a point donné un cœur pour connaître, ni des yeux pour voir, ni des oreilles pour entendre.
\VS{5}Je t'ai conduit pendant quarante ans par le désert~; tes vêtements ne se sont point usés, et ton soulier ne s'est point usé à ton pied.
\VS{6}Vous n'avez point mangé de pain, ni bu de vin ni de liqueur forte, afin que vous connaissiez que je suis Yahweh, votre Dieu.
\VS{7}Et vous êtes parvenus dans ce lieu~; Sihon, roi de Hesbon, et Og, roi de Basan, sont sortis à notre rencontre, pour nous combattre, et nous les avons battus.
\VS{8}Et nous avons pris leur pays, et nous l'avons donné en héritage aux Rubénites, aux Gadites, et à la demi-tribu des Manassites.
\TextTitle{Béni celui qui reste fidèle à l'alliance}
\VS{9}Vous garderez donc les paroles de cette alliance, et vous les pratiquerez, afin de réussir dans tout ce que vous ferez.
\VS{10}Vous vous tiendrez aujourd'hui devant Yahweh, votre Dieu, vos chefs de tribus, vos anciens, vos officiers, tous les hommes d'Israël,
\VS{11}vos enfants, vos femmes, et l'étranger qui est au milieu de ton camp, depuis celui qui coupe ton bois jusqu'à celui qui puise ton eau~;
\VS{12}afin que tu entres dans l'alliance de Yahweh, ton Dieu, dans ce serment, que Yahweh, ton Dieu, traite aujourd'hui avec toi,
\VS{13}afin qu'il t'établisse aujourd'hui pour son peuple et qu'il soit ton Dieu, comme il te l'a dit, et comme il l'a juré à tes pères, Abraham, Isaac et Jacob.
\VS{14}Et ce n'est pas seulement avec vous que je traite cette alliance, ce serment.
\VS{15}Mais c'est avec ceux qui sont ici, avec nous aujourd'hui devant Yahweh, notre Dieu, et avec ceux qui ne sont point ici, avec nous aujourd'hui.
\TextTitle{Mise en garde contre celui qui abandonne l'alliance}
\VS{16}Car vous savez comment nous avons habité dans le pays d'Egypte, et comment nous sommes passés au milieu des nations, que vous avez traversées.
\VS{17}Et vous avez vu leurs abominations et leurs idoles, le bois et la pierre, l'argent et l'or qui sont parmi eux.
\VS{18}Qu'il n'y ait parmi vous ni homme, ni femme, ni famille, ni tribu qui détourne son cœur aujourd'hui de Yahweh, notre Dieu, pour aller servir les dieux de ces nations. Qu'il n'y ait parmi vous de racine qui produise du poison et de l'absinthe.
\VS{19}Et qu'il n'arrive que quelqu'un en entendant les paroles de cette malédiction, ne se bénisse dans son cœur, en disant~: J'aurai la paix, même si je marche dans les penchants de mon cœur, et que j'ajoute l'ivresse à la soif.
\VS{20}Yahweh ne voudra point lui pardonner. Mais la colère de Yahweh et la jalousie s'enflammeront contre cet homme, et toutes les malédictions écrites dans ce livre reposeront sur lui, et Yahweh effacera son nom de dessous les cieux.
\VS{21}Et Yahweh le séparera de toutes les tribus d'Israël, pour son malheur, selon toutes les malédictions de l'alliance écrite dans ce livre de la loi.
\VS{22}Et la génération à venir, vos fils qui se lèveront après vous, et l'étranger qui viendra d'un pays lointain, quand ils verront les plaies et les maladies, dont Yahweh aura frappé ce pays~;
\VS{23}et que toute la terre de ce pays-là ne sera que soufre, que sel, et qu'embrasement, qu'elle ne sera point semée, et qu'elle ne fera rien germer, et que nulle herbe n'en sortira, ainsi qu'en la subversion de Sodome, et de Gomorrhe, et d'Adma, et de Tseboïm, que Yahweh détruisit dans sa colère et dans sa fureur.
\VS{24}Mais toutes les nations diront~: Pourquoi Yahweh a-t-il traité ainsi ce pays~? D'où vient l'ardeur de cette grande colère~?
\VS{25}Et on répondra~: C'est parce qu'ils ont abandonné l'alliance de Yahweh, le Dieu de leurs pères, qu'il a traitée avec eux quand il les fit sortir du pays d'Egypte~;
\VS{26}car ils sont allés servir d'autres dieux et se sont prosternés devant eux~; des dieux qu'ils ne connaissaient point et qu'il ne leur avait point donnés en partage.
\VS{27}A cause de cela, la colère de Yahweh s'est enflammée contre ce pays, et il a fait venir sur lui toutes les malédictions écrites dans ce livre.
\VS{28}Et Yahweh les a arrachés de leur terre avec colère, avec fureur, avec une grande indignation, et il les a chassés sur un autre pays, comme on le voit aujourd'hui.
\VS{29}Les choses cachées sont à Yahweh, notre Dieu~; les choses révélées sont à nous et à nos fils, à jamais, afin que nous pratiquions toutes les paroles de cette loi.
\Chap{30}
\TextTitle{Yahweh bénira et restaurera le peuple repentant}
\VerseOne{}Or il arrivera que lorsque toutes ces choses seront venues sur toi, la bénédiction et la malédiction, que je mets devant toi, si tu les rappelles dans ton cœur, parmi toutes les nations vers lesquelles Yahweh, ton Dieu, t'aura chassé~;
\VS{2}si tu reviens à Yahweh, ton Dieu, et si tu obéis à sa voix de tout ton cœur, de toute ton âme, toi et tes fils, selon tout ce que je t'ordonne aujourd'hui,
\VS{3}Yahweh, ton Dieu, ramènera tes captifs et aura compassion de toi~; il te rassemblera encore du milieu de tous les peuples parmi lesquels Yahweh, ton Dieu, t'aura dispersé.
\VS{4}Quand tu seras dispersé à l'extrémité des cieux, Yahweh, ton Dieu, te rassemblera de là, et de là, il te prendra.
\VS{5}Yahweh, ton Dieu, dis-je, te ramènera dans le pays que tes pères possédaient, et tu le posséderas~; il te fera du bien, et te rendra plus nombreux que tes pères.
\VS{6}Yahweh, ton Dieu, circoncira ton cœur, et le cœur de ta postérité, pour que tu aimes Yahweh, ton Dieu, de tout ton cœur, et de toute ton âme, afin que tu vives\FTNT{Ro. 2:29.}.
\VS{7}Et Yahweh, ton Dieu, mettra toutes ces malédictions sur tes ennemis, et sur ceux qui te haïront et te persécuteront.
\VS{8}Ainsi tu retourneras à Yahweh, tu obéiras à sa voix, et tu feras tous ses commandements que je t'ordonne aujourd'hui.
\TextTitle{Faire connaître la loi aux futures générations}
\VS{9}Et Yahweh, ton Dieu, te fera abonder en bien dans toute l'œuvre de ta main, dans le fruit de tes entrailles, dans le fruit de tes troupeaux et dans le fruit de ta terre~; car Yahweh se réjouira de nouveau de ton bonheur, comme il s'est réjoui de celui de tes pères,
\VS{10}quand tu obéiras à la voix de Yahweh, ton Dieu, en gardant ses commandements et ses ordonnances écrites dans ce livre de la loi, quand tu reviendras à Yahweh, ton Dieu, de tout ton cœur et de toute ton âme.
\TextTitle{Le peuple devant un choix}
\VS{11}Car ce commandement que je t'ordonne aujourd'hui n'est pas trop difficile pour toi et hors de ta portée.
\VS{12}Il n'est pas aux cieux, pour dire~: Qui montera pour nous aux cieux, nous l'apportera et nous le fera entendre, pour que nous le fassions~?
\VS{13}Il n'est point aussi de l'autre côté de la mer pour dire~: Qui passera de l'autre côté de la mer pour nous, et nous l'apportera, et nous le fera entendre pour que nous le fassions~?
\VS{14}Car cette parole est fort près de toi, dans ta bouche et dans ton cœur, afin que tu la pratiques\FTNT{Ro. 10:6.}.
\VS{15}Regarde, je mets aujourd'hui devant toi la vie et le bien, la mort et le mal.
\VS{16}Car je t'ordonne aujourd'hui d'aimer Yahweh, ton Dieu, de marcher dans ses voies, de garder ses commandements, ses lois, et ses ordonnances, afin que tu vives, que tu multiplies, et que Yahweh, ton Dieu, te bénisse dans le pays où tu vas entrer en possession.
\VS{17}Mais si ton cœur se détourne, si tu n'obéis point, et si tu te laisses entraîner à te prosterner devant d'autres dieux et à les servir,
\VS{18}je vous déclare aujourd'hui que vous périrez certainement, et que vous ne prolongerez point vos jours sur la terre dont vous allez entrer en possession, après avoir passé le Jourdain.
\VS{19}J'en prends aujourd'hui à témoin les cieux et la terre contre vous~: J'ai mis devant toi la vie et la mort, la bénédiction et la malédiction. Choisis donc la vie\FTNT{cp. Mt. 7:13-14.}, afin que tu vives, toi et ta postérité.
\VS{20}en aimant Yahweh, ton Dieu, en obéissant à sa voix, et en t'attachant à lui~: Car c'est lui qui est ta vie et la longueur de tes jours, afin que tu demeures sur la terre que Yahweh a juré à tes pères, Abraham, Isaac, et Jacob, de leur donner.
\Chap{31}
\TextTitle{Moïse encourage et affermit le peuple}
\VerseOne{}Puis Moïse s'en alla, et dit ces paroles à tout Israël~:
\VS{2}Aujourd'hui, leur dit-il, je suis âgé de cent vingt ans, je ne pourrai plus sortir ni entrer, et Yahweh m'a dit~: Tu ne passeras point ce Jourdain.
\VS{3}Yahweh, ton Dieu, passera lui-même devant toi, il détruira ces nations devant toi, et tu les posséderas. Josué passera aussi devant toi, comme Yahweh l'a dit.
\VS{4}Et Yahweh leur fera comme il a fait à Sihon et à Og, rois des Amoréens, qu'il a détruits avec leurs pays.
\VS{5}Et Yahweh les livrera devant vous, et vous leur ferez selon tout le commandement que je vous ai ordonné.
\VS{6}Fortifiez-vous donc et prenez courage~! Ne craignez point et ne soyez point effrayés devant eux~; car Yahweh, ton Dieu, marchera avec toi, il ne te délaissera point et ne t'abandonnera point.
\VS{7}Et Moïse appela Josué, et lui dit en présence de tout Israël~: Fortifie-toi et prends courage, car tu entreras avec ce peuple dans le pays que Yahweh a juré à leurs pères de leur donner, et c'est toi qui les en mettras en possession.
\VS{8}Yahweh est celui qui marchera devant toi, il sera lui-même avec toi, il ne te délaissera point, il ne t'abandonnera point~; ne crains point, et ne t'effraie point.
\VS{9}Or Moïse écrivit cette loi, et il la donna aux prêtres, fils de Lévi, qui portaient l'arche de l'alliance de Yahweh, et à tous les anciens d'Israël.
\VS{10}Moïse leur ordonna, en disant~: Tous les sept ans, au temps fixé de l'année du relâche, à la fête des tabernacles,
\VS{11}quand tout Israël viendra se présenter devant Yahweh, ton Dieu, dans le lieu qu'il aura choisi, tu liras alors cette loi devant tout Israël, à leurs oreilles.
\VS{12}Tu rassembleras le peuple, les hommes, les femmes, les enfants et l'étranger qui sera dans tes portes, pour qu'ils t'entendent, et qu'ils apprennent à craindre Yahweh, votre Dieu, et qu'ils prennent garde de faire toutes les paroles de cette loi.
\VS{13}Et leurs fils qui ne la connaîtront point l'entendront, et ils apprendront à craindre Yahweh, votre Dieu, tous les jours que vous vivrez sur cette terre que vous allez posséder après avoir passé le Jourdain.
\TextTitle{Yahweh annonce les événements à venir}
\VS{14}Alors Yahweh dit à Moïse~: Voici, le jour où tu vas mourir est proche. Appelle Josué, et tenez-vous dans la tente d'assignation. Je lui donnerai mes ordres. Moïse et Josué allèrent et se présentèrent dans la tente d'assignation.
\VS{15}Et Yahweh apparut dans la tente, dans une colonne de nuée~; et la colonne de nuée s'arrêta à l'entrée de la tente.
\VS{16}Yahweh dit à Moïse~: Voici, tu vas te coucher avec tes pères. Et ce peuple se lèvera et se prostituera après les dieux étrangers du pays au milieu duquel il va entrer. Il m'abandonnera et violera mon alliance que j'ai traitée avec lui.
\VS{17}En ce jour-là ma colère s'enflammera contre lui. Je les abandonnerai, et je leur cacherai ma face. Il sera dévoré, une multitude de maux et d'angoisses l'atteindront, et il dira en ce jour-là~: N'est-ce pas parce que mon Dieu n'est point au milieu de moi, que ces maux m'ont atteint~?
\VS{18}En ce jour-là, dis-je, je cacherai entièrement ma face, à cause de tout le mal qu'il aura fait, parce qu'il se sera tourné vers d'autres dieux.
\VS{19}Maintenant donc, écrivez ce cantique. Enseigne-le aux enfants d'Israël, mets-le dans leur bouche, afin que ce cantique me serve de témoignage contre les fils d'Israël.
\VS{20}Car je le conduirai sur la terre que j'ai juré à ses pères, où coulent le lait et le miel~; il mangera, se rassasiera, et s'engraissera~; puis il se tournera vers d'autres dieux, et il les servira, il m'irritera par mépris et violera mon alliance~;
\VS{21}et il arrivera qu'il sera atteint par une multitude de maux et d'angoisses, ce cantique, qui ne sera point oublié et qui sera dans la bouche de la postérité, répondra comme témoin contre eux. Je connais ses desseins, qu'il a déjà préparés aujourd'hui, avant même que je l'aie fait entrer dans le pays que j'ai juré.
\VS{22}Ainsi Moïse écrivit ce cantique en ce jour-là, et l'enseigna aux enfants d'Israël.
\VS{23}Et Yahweh commanda à Josué, fils de Nun, en disant~: Fortifie-toi et prends courage, car c'est toi qui feras entrer les enfants d'Israël dans le pays que je leur ai juré~; et je serai avec toi.
\VS{24}Et il arrivera que quand Moïse eut achevé d'écrire dans un livre les paroles de cette loi jusqu'à ce qu'elle soit complète,
\VS{25}Moïse ordonna aux Lévites qui portaient l'arche de l'alliance de Yahweh, en disant~:
\VS{26}Prenez ce livre de la loi, et mettez-le à côté de l'arche de l'alliance de Yahweh, votre Dieu, et il sera là comme témoin contre toi.
\VS{27}Car je connais ta rébellion et ton cou raide. Voici, déjà aujourd'hui étant en vie avec vous, vous avez été rebelles contre Yahweh, combien plus le serez-vous après ma mort~?
\VS{28}Faites assembler devant moi tous les anciens de vos tribus, et vos officiers, et je dirai ces paroles en leur présence, et j'appellerai à témoin contre eux les cieux et la terre.
\VS{29}Car je sais qu'après ma mort vous vous corromprez, et que vous vous détournerez de la voie que je vous ai ordonnée~; mais à la fin, le malheur vous atteindra, parce que vous aurez fait ce qui déplaît aux yeux de Yahweh, en l'irritant par les œuvres de vos mains.
\VS{30}Ainsi Moïse prononça entièrement les paroles de ce cantique-ci, en présence de toute l'assemblée d'Israël.
\Chap{32}
\TextTitle{Cantique de Moïse}
\VerseOne{}Cieux~! Prêtez l'oreille, et je parlerai. Terre~! écoute les paroles de ma bouche.
\VS{2}Que mon enseignement tombe comme la pluie, que ma parole se répande comme la rosée, comme une pluie fine sur l'herbe naissante, et comme une averse sur la verdure~!
\VS{3}Car j'invoquerai le Nom de Yahweh~; attribuez la grandeur à notre Dieu.
\VS{4}L'œuvre du rocher\FTNT{Voir commentaire en Es. 8:13-17.} est parfaite, car toutes ses voies sont justes. C'est un Dieu fidèle et sans iniquité, il est juste et droit.
\VS{5}Ils se sont corrompus, à lui n'est point la faute~; la faute est à ses fils, c'est une génération fausse et tortueuse.
\TextTitle{Israël, le choix de Yahweh}
\VS{6}Est-ce ainsi que tu récompenses Yahweh, peuple insensé et dépourvu de sagesse~? N'est-il pas ton père, celui qui t'a acquis~? Il t'a fait et t'a façonné.
\VS{7}Souviens-toi des anciens jours, considère les années, de génération en génération, interroge ton père, et il te l'apprendra, et tes anciens, et ils te le diront.
\VS{8}Quand le Très-Haut laissa un héritage aux nations, quand il sépara les enfants des hommes, il fixa les limites des peuples selon le nombre des fils d'Israël~;
\VS{9}car la portion de Yahweh, c'est son peuple, Jacob est le lot de son héritage.
\VS{10}Il l'a trouvé dans un pays désert, dans la désolation des hurlements d'une solitude, il l'a entouré, il l'a dirigé, il l'a gardé comme la prunelle de son œil,
\VS{11}comme l'aigle éveille sa nichée, couve ses petits, étend ses ailes, les prend, les porte sur ses ailes.
\VS{12}Yahweh seul l'a conduit, et il n'y a point eu avec lui de dieu étranger.
\VS{13}Il l'a fait monter à cheval sur les hauteurs du pays, et il a mangé les fruits des champs~; il lui a donné à sucer le miel du rocher, l'huile du rocher le plus dur,
\VS{14}la crème des vaches, le lait des brebis, et la graisse des agneaux, des béliers de Basan, et des boucs, et la fleur du froment~; et tu as bu le vin qui était le sang de la grappe.
\TextTitle{Condamnation de l'apostasie d'Israël}
\VS{15}Jeshurun\FTNT{Littéralement «~Jeshurun~» en hébreu~: «~celui qui est droit~». Nom symbolique donné à Israël pour décrire son caractère idéal.} s'est engraissé, et a regimbé~; tu es devenu gras, gros et épais~! Et il a abandonné Dieu qui l'a fait, et il a méprisé le rocher de son salut.
\VS{16}Ils ont provoqué sa jalousie par des dieux étrangers, ils l'ont irrité par des abominations.
\VS{17}Ils ont sacrifié à des démons, qui ne sont point Dieu~; aux dieux qu'ils ne connaissaient point, dieux nouveaux, venus depuis peu, et que vos pères n'ont point redoutés.
\VS{18}Tu as oublié le rocher qui t'a engendré, et tu as oublié le Dieu qui t'a fait naître.
\VS{19}Yahweh l'a vu, et a été irrité, parce que ses fils et ses filles l'ont provoqué à la colère.
\VS{20}Et il a dit~: Je cacherai ma face, je verrai quelle sera leur fin~; car ils sont une génération perverse, des fils infidèles.
\VS{21}Ils ont excité ma jalousie par ce qui n'est point Dieu, ils m'ont irrité par leurs vanités~; ainsi je provoquerai leur jalousie par ce qui n'est point un peuple, et je les offenserai par une nation insensée.
\VS{22}Car le feu de ma colère s'est allumé, et brûlera jusqu'au fond du scheol, dévorera la terre et son fruit, et embrasera les fondements des montagnes.
\VS{23}Je rassemblerai sur eux des maux, et je détruirai toutes mes flèches sur eux.
\VS{24}Ils seront consumés par la famine, rongés par des charbons ardents, et par une destruction amère~; j'enverrai contre eux la dent des bêtes et le venin des serpents qui rampent sur la poussière.
\VS{25}L'épée venant de dehors les privera les uns des autres~; et au-dedans, la terreur les privera d'enfants. Il en sera du jeune homme comme de la vierge, de l'enfant à la mamelle comme de l'homme aux cheveux blancs.
\TextTitle{A Yahweh la vengeance et la rétribution}
\VS{26}Je dirais~: Je les détruirai, et je ferai disparaître leur mémoire d'entre les hommes~!
\VS{27}Si je ne craignais la colère de l'ennemi, de peur que leurs adversaires ne se méprennent, et ne disent~: Notre main est élevée, et ce n'est pas Yahweh qui a fait tout ceci.
\VS{28}Car c'est une nation qui se perd par ses conseils, et il n'y a en eux aucune intelligence.
\VS{29}Ô s'ils étaient sages, ils comprendraient ceci, et ils considéreraient leur fin.
\VS{30}Comment un seul en poursuivrait-il mille, et deux en mettraient-ils dix mille en fuite, si ce n'était que leur Rocher les avait vendus, et que Yahweh ne les avait enserrés~?
\VS{31}Car leur rocher n'est pas comme notre Rocher, nos ennemis en sont juges.
\VS{32}Car leur vigne est du plant de Sodome, et du terroir de Gomorrhe~; leurs raisins sont des raisins empoisonnés, leurs grappes sont amères.
\VS{33}Leur vin est un venin de dragon, et du poison cruel d'aspic.
\VS{34}Cela n'est-il pas caché près de moi, scellé dans mes trésors~?
\VS{35}A moi la vengeance et la rétribution, le temps où leur pied glissera~! Car le jour de leur calamité est près, et les choses qui doivent leur arriver se hâtent.
\VS{36}Mais Yahweh jugera son peuple~; et il se repentira en faveur de ses serviteurs, quand il verra que leur force a disparu, et qu'il n'y a personne de retenu ni d'abandonné.
\VS{37}Et il dira~: Où sont leurs dieux, le rocher en qui ils se confiaient,
\VS{38}qui mangeaient la graisse de leurs sacrifices, qui buvaient le vin de leurs libations~? Qu'ils se lèvent, qu'ils vous aident, et qu'ils vous servent de refuge~!
\VS{39}Voyez maintenant que moi,JE SUIS\FTNT{«~JE SUIS. Il s'agit ici du Nom que Dieu a révélé à Moïse et à Esaïe (Ex. 3:14) et sous lequel Jésus s'est présenté (Jn. 18:5-8).}, et il n'y a point de dieu avec moi\FTNT{Ce verset confirme que Dieu est un puisqu'il n'y a pas d'autres dieux à ses côtés.}~; je fais mourir et je fais vivre, je blesse et je guéris~; et il n'y a personne qui puisse délivrer de ma main.
\VS{40}Car je lève ma main au ciel, et je dis~: Je vis éternellement.
\VS{41}Si j'aiguise l'éclair de mon épée, et si ma main saisit la justice, je rendrai la vengeance à mes adversaires et je rétribuerai ceux qui me haïssent.
\VS{42} J'enivrerai mes flèches de sang et mon épée dévorera la chair, j'enivrerai, dis-je, mes flèches du sang des tués et des captifs, de la tête des chefs de l'ennemi.
\VS{43}Nations, réjouissez-vous avec son peuple~! Car il venge le sang de ses serviteurs, il tire vengeance de ses ennemis, et fait propitiation pour sa terre et pour son peuple.
\TextTitle{Fin du cantique, invitation à demeurer fidèle}
\VS{44}Moïse donc vint et prononça toutes les paroles de ce cantique, à l'oreille du peuple, lui et Josué, fils de Nun.
\VS{45}Et quand Moïse eut achevé de prononcer toutes ces paroles à tout Israël,
\VS{46}il leur dit~: Appliquez votre cœur à toutes ces paroles que je vous conjure aujourd'hui d'ordonner à vos fils, afin qu'ils prennent garde de faire toutes les paroles de cette loi.
\VS{47}Car ce n'est pas une parole vaine pour vous, mais c'est votre vie~; et par cette parole vous prolongerez vos jours sur la terre que vous posséderez, après avoir passé le Jourdain.
\TextTitle{Moïse, invité à monter sur le mont Nebo}
\VS{48}En ce même jour-là, Yahweh parla à Moïse, en disant~:
\VS{49}Monte sur cette montagne d'Abarim, sur le mont Nebo, au pays de Moab, vis-à-vis de Jéricho~; et regarde le pays de Canaan, que je donne en possession aux enfants d'Israël.
\VS{50}Tu mourras sur la montagne où tu vas monter, et tu seras recueilli vers ton peuple, comme Aaron, ton frère, est mort sur la montagne d'Hor, et a été recueilli vers son peuple,
\VS{51}parce que vous avez péché contre moi au milieu des fils d'Israël, aux eaux de Meriba, à Kadès, dans le désert de Tsin~; car vous ne m'avez point sanctifié au milieu des enfants d'Israël.
\VS{52}Tu verras le pays devant toi, mais tu n'entreras point dans le pays, que je donne aux enfants d'Israël.
\Chap{33}
\TextTitle{Moïse bénit les tribus d'Israël}
\VerseOne{}Or c'est ici la bénédiction dont Moïse, homme de Dieu, bénit les enfants d'Israël avant sa mort.
\VS{2}Il dit donc~: Yahweh est venu de Sinaï, il s'est levé sur eux de Séir, il a resplendi de la montagne de Paran, et il est sorti d'entre les dix milliers des saints, et de sa droite le feu de la loi est sorti vers eux.
\VS{3}En effet, il aime les peuples~; tous ses saints sont dans ta main. Ils se sont mis à tes pieds pour recevoir tes paroles.
\VS{4}Moïse nous a donné la loi, héritage de l'assemblée de Jacob.
\VS{5}Il était roi de Jeshurun\FTNT{Voir commentaire en De. 32:15.}, quand les chefs du peuple s'assemblaient ensemble, avec les tribus d'Israël.
\VS{6}Que Ruben vive et qu'il ne meure point, encore que ses hommes soient en petit nombre.
\VS{7}Et voici ce qu'il dit pour Juda~: Ô Yahweh~! Ecoute la voix de Juda, et ramène-le vers son peuple. Que ses mains soient puissantes, et sois-lui en aide contre ses ennemis.
\VS{8}Il dit aussi touchant Lévi~: Tes thummim et tes urim sont à l'homme fidèle que tu as éprouvé à Massa, et avec qui tu as contesté aux eaux de Meriba.
\VS{9}Il dit de son père et de sa mère~: Je ne les ai point vus~! Il ne reconnait point ses frères, et ne connait point ses fils. Car ils gardent tes paroles, et ils gardent ton alliance.
\VS{10}Ils enseignent tes ordonnances à Jacob, et ta loi à Israël~; ils mettent l'encens sous tes narines, et l'holocauste sur ton autel.
\VS{11}Ô Yahweh, bénis sa force~! Agrée l'œuvre de ses mains~! Brise les reins de ceux qui s'élèvent contre lui, et que ceux qui le haïssent ne se relèvent plus~!
\VS{12}Il dit de Benjamin~: Le bien-aimé de Yahweh habitera en sécurité avec lui~; il le protégera toujours, et demeurera entre ses épaules.
\VS{13}Il dit de Joseph~: Son pays est béni par Yahweh, de ce qu'il y a de plus précieux au ciel, de la rosée, et de l'abîme qui est en bas,
\VS{14}et du plus précieux des produits du soleil, et du plus précieux des produits de la lune, 
\VS{15}et de ce qui croît sur le sommet des montagnes d'ancienneté, du plus précieux des collines éternelles,
\VS{16}et du plus précieux de la terre et de sa plénitude. Que la grâce de celui qui demeura dans le buisson vienne sur la tête de Joseph, sur le sommet, sur le sommet de la tête de celui qui est consacré d'entre ses frères~!
\VS{17}Sa majesté est comme le premier-né de son taureau~; et ses cornes comme les cornes du buffle~; il poussera tous les peuples ensemble jusqu'aux extrémités de la terre~: Ce sont les dix milliers d'Ephraïm, et ce sont les milliers de Manassé.
\VS{18}Il dit de Zabulon~: Réjouis-toi, Zabulon, dans ta sortie, et toi, Issacar, dans tes tentes.
\VS{19}Ils appelleront les peuples sur la montagne, ils y offriront des sacrifices de justice, car ils suceront l'abondance des mers, et les trésors cachés dans le sable.
\VS{20}Il dit aussi de Gad~: Béni soit celui qui élargit Gad~! Il habite comme un lion, et il déchire le bras et la tête.
\VS{21}Il a choisi les prémices, parce que c'était là qu'était cachée la portion du législateur, et il est venu en tête du peuple~; il a exécuté la justice de Yahweh et ses jugements envers Israël.
\VS{22}Et il dit de Dan~: Dan est un jeune lion, il s'élance de Basan.
\VS{23}Il dit de Nephthali~: Nephthali, rassasié de faveur, et rempli de la bénédiction de Yahweh, possède l'occident et le sud.
\VS{24}Il dit aussi d'Aser~: Aser sera béni entre les fils~; il sera agréable à ses frères, et il trempera son pied dans l'huile.
\VS{25}Tes verrous seront de fer et d'airain, et ta force durera autant que tes jours.
\VS{26}Nul n'est comme le Dieu de Jeshurun\FTNT{Voir commentaire en De. 32:15.}, porté sur les cieux pour te venir en aide, et sur les nuées dans sa majesté.
\VS{27}Le Dieu d'éternité est un refuge, et au-dessous de toi sont ses bras éternels~; car il a chassé de devant toi tes ennemis, et il a dit~: Extermine.
\VS{28}Israël donc habitera en sécurité, la source de Jacob est à part dans un pays de blé et de vin, et ses cieux distilleront la rosée.
\VS{29}Ô que tu es heureux, Israël~! Qui est le peuple semblable à toi, qui ait été sauvé par Yahweh, le bouclier de ton secours et l'épée de ta majesté~? Tes ennemis dissimuleront devant toi, et tu fouleras de tes pieds leurs lieux élevés.
\Chap{34}
\TextTitle{Moïse voit le pays mais n'y entre pas}
\VerseOne{}Alors Moïse monta des plaines de Moab sur le mont Nebo, au sommet du Pisga, vis-à-vis de Jéricho. Et Yahweh lui fit voir tout le pays~: De Galaad jusqu'à Dan,
\VS{2}tout Nephthali, le pays d'Ephraïm et de Manassé, tout le pays de Juda, jusqu'à la Mer Occidentale,
\VS{3}le sud, les environs du Jourdain, la plaine de Jéricho, la ville des palmiers, jusqu'à Tsoar.
\VS{4}Yahweh lui dit~: C'est ici le pays que j'ai juré à Abraham, à Isaac, et à Jacob, en disant~: Je le donnerai à ta postérité. Je te l'ai fait voir de tes yeux~; mais tu n'y entreras point.
\TextTitle{Mort de Moïse}
\VS{5}Ainsi Moïse, serviteur de Yahweh, mourut là, dans le pays de Moab, selon la parole de Yahweh.
\VS{6}Et il l'ensevelit dans la vallée, au pays de Moab, vis-à-vis de Beth-Peor. Personne n'a connu son sépulcre jusqu'à aujourd'hui\FTNT{Jud. 1:9.}.
\VS{7}Or Moïse était âgé de cent vingt ans quand il mourut~; sa vue n'était point affaiblie, et sa vigueur n'était point passée.
\VS{8}Les enfants d'Israël pleurèrent Moïse trente jours dans les plaines de Moab~; et ces jours de pleurs et de deuil sur Moïse furent accomplis.
\TextTitle{Josué, successeur de Moïse}
\VS{9}Et Josué, fils de Nun, fut rempli de l'Esprit de sagesse, parce que Moïse lui avait imposé les mains\FTNT{Jos. 1:9.}. Les enfants d'Israël lui obéirent, et firent ce que Yahweh avait ordonné à Moïse.
\VS{10}Et il ne s'est plus levé en Israël de prophète comme Moïse, que Yahweh connaissait face à face.
\VS{11}Selon tous les signes et les miracles que Yahweh l'envoya faire au pays d'Egypte, devant Pharaon, et tous ses serviteurs, et tout son pays,
\VS{12}et selon toute cette main forte, et tous ces terribles prodiges, que Moïse fit sous les yeux de tout Israël.
\PPE{}
\end{multicols}
\clearpage
\addcontentsline{toc}{section}{Nevi'im (Prophètes)}\clearpage
\clearpage\makeatletter\def\@evenhead{}\def\@oddhead{}\makeatother

\vspace*{\fill}
\begin{center}
{\Huge Tanakh : Nevi'im}
\end{center}
\vspace*{\fill}

\clearpage

\makeatletter\def\@evenhead{{\NoAutoSpaceBeforeFDP{\small{\rightmark\hfil\thepage\hfil\leftmark}}}}\def\@oddhead{{\NoAutoSpaceBeforeFDP{\small{\rightmark\hfil\thepage\hfil\leftmark}}}}\makeatother

\clearpage\ShortTitle{Josué}\BookTitle{Josué}\BFont
\noindent\hrulefill
{\footnotesize
\textit{
\bigskip
{\centering{}
\\Auteur : Probablement Josué
\\(Heb. : Yehowshuwa)
\\Signification : Yahweh est salut
\\Thème : La conquête de Canaan
\\Date de rédaction : 14\up{ème} siècle av. J.-C.\\}
}
%\bigskip
\textit{
\\Né en Egypte, Josué, fils de Nun, originaire de la tribu d'Ephraïm, servit Moïse de la sortie d'Egypte jusqu'à sa mort. Choisi par Dieu pour succéder au prophète, il fut le seul de l'ancienne génération, avec Caleb, à avoir survécu à la longue épreuve du désert. Ce livre relate les étapes du voyage du peuple et sa conquête de la terre promise. Il présente par ailleurs les victoires acquises par la puissance de Yahweh sous la conduite de Josué. C'est l'histoire de la prise de Canaan et de son partage aux douze tribus d'Israël.\bigskip
}
}
\par\nobreak\noindent\hrulefill
\begin{multicols}{2}
\Chap{1}
\TextTitle{Josué succède à Moïse à sa mort\FTNTT{De. 34:9.}}
\VerseOne{}Or, il arriva après la mort de Moïse, serviteur de Yahweh, que Yahweh parla à Josué, fils de Nun, qui avait servi Moïse, en disant :
\VS{2}Moïse, mon serviteur est mort ; maintenant donc, lève-toi, passe ce Jourdain, toi et tout ce peuple, pour entrer dans le pays que je donne aux enfants d'Israël.
\VS{3}Tout lieu que foulera la plante de votre pied, je vous l'ai donné, comme je l'ai déclaré à Moïse\FTNT{De. 11:24.}.
\VS{4}Vos frontières seront depuis ce désert et le Liban, jusqu'au grand fleuve, le fleuve de l'Euphrate, tout le pays des Héthiens jusqu'à la grande mer, vers le soleil couchant.
\VS{5}Nul ne tiendra devant toi, tous les jours de ta vie. Je serai avec toi comme j'ai été avec Moïse ; je ne te délaisserai point, et je ne t'abandonnerai point\FTNT{De. 31:6 ; Hé. 13:5-6.}.
\VS{6}Fortifie-toi et prends courage, car c'est toi qui mettras ce peuple en possession du pays dont j'ai juré à leurs pères de leur donner.
\VS{7}Seulement fortifie-toi et renforce-toi de plus en plus, afin que tu prennes garde de faire selon toute la loi que Moïse mon serviteur t'a ordonnée ; ne t'en détourne point ni à droite ni à gauche, afin que tu prospères partout où tu iras.
\VS{8}Que ce livre de la loi ne s'éloigne point de ta bouche, mais médite-le jour et nuit, pour agir fidèlement selon tout ce qui y est écrit\FTNT{La clé d'une vie chrétienne épanouie est la Parole de Dieu. Méditer signifie : 
\\- Murmurer la Parole de Dieu : Partout où nous sommes, nous pouvons dans nos cœurs murmurer les promesses du Seigneur (Ps. 63:5-8 ; Ps. 119:11).
\\- Proclamer à haute voix : Il est intéressant de noter que le mot hébreu traduit dans Jos. 1:8 par méditer est traduit par « proclamer » ou « dire » dans Pr. 8:7 ; Ps. 35:8 ; Ps. 77:13. 
\\- Réfléchir profondément : Il faut être dans le lieu secret (Mt. 6:5-6). En Israël il est de coutume d'aller étudier la Torah à l'ombre d'un figuier. Voir Jn. 1:43-51.} ; car c'est alors que tu auras du succès dans tes entreprises, c'est alors que tu réussiras.
\VS{9}Ne t'ai-je pas donné cet ordre, fortifie-toi et prends courage ? Ne t'épouvante point et ne t'effraie point ; car Yahweh ton Dieu est avec toi partout où tu iras.
\TextTitle{Josué prend la direction du peuple}
\VS{10}Après cela, Josué donna cet ordre aux officiers du peuple, en disant :
\VS{11}Passez par le camp, ordonnez au peuple et dites-lui : Préparez-vous des provisions, car dans trois jours vous passerez ce Jourdain pour aller prendre possession du pays que Yahweh, votre Dieu, vous donne afin que vous le possédiez.
\VS{12}Josué parla aussi aux Rubénites, aux Gadites et à la demi-tribu de Manassé, en disant :
\VS{13}Souvenez-vous de la parole que Moïse, serviteur de Yahweh, vous a prescrite, en disant : Yahweh votre Dieu vous a accordé du repos, et vous a donné ce pays.
\VS{14}Vos femmes, vos petits-enfants, et vos bêtes resteront dans le pays que Moïse vous a donné de l'autre côté du Jourdain ; mais vous tous, hommes vaillants, vous passerez en armes devant vos frères, et vous les aiderez\FTNT{Ex. 13:18.} ;
\VS{15}jusqu'à ce que Yahweh ait accordé du repos à vos frères comme à vous, et qu'ils soient aussi en possession du pays que Yahweh, votre Dieu, leur donne. Puis vous reviendrez prendre possession du pays qui est votre propriété, et que vous a donné Moïse, serviteur de Yahweh, de l'autre côté du Jourdain, vers l'orient.
\VS{16}Ils répondirent à Josué, en disant : Nous ferons tout ce que tu nous as ordonné, et nous irons partout où tu nous enverras.
\VS{17}Nous t'obéirons comme nous avons obéi à Moïse ; seulement que Yahweh ton Dieu soit avec toi, comme il a été avec Moïse.
\VS{18}Tout homme qui sera rebelle à ton ordre, et qui n'obéira point à tes paroles dans tout ce que tu lui commanderas, sera mis à mort ; seulement, fortifie-toi, et sois courageux !
\Chap{2}
\TextTitle{Josué envoie deux espions à Jéricho ; ils sont reçus par Rahab\FTNTT{Ja. 2:25.}}
\VerseOne{}Or, Josué fils de Nun, envoya secrètement de Sittim deux hommes, pour épier secrètement le pays, et il leur dit : Allez, examinez le pays, et Jéricho. Ils partirent donc et entrèrent dans la maison d'une femme prostituée, nommée Rahab\FTNT{Rahab avait entendu parler du Dieu des Hébreux et avait placé son espérance de salut en lui (Ro. 10 :11). Par cet acte de foi, sa destinée a changé. Cette femme qui était vouée à une double condamnation du fait de sa condition de prostituée (De.23:17) et de son appartenance à une nation païenne qui devait être dévouée à la façon de l'interdit (Jos. 6), a été sauvée avec sa famille (Ac. 2:21 ; Ac. 16:31 ). Ainsi, bien des siècles plus tard, on ne la mentionnera plus comme une prostituée, mais comme une ancêtre du Sauveur et une héroïne de la foi (Mt. 1:5 ; Hé. 11 :21). Rahab est donc l'archétype des païens qui sont rentrés dans l'alliance de Dieu par la foi.}, et ils y couchèrent.
\VS{2}Alors on dit au roi de Jéricho : Voici, des hommes sont venus ici cette nuit de la part des enfants d'Israël pour explorer le pays.
\VS{3}Et le roi de Jéricho envoya dire à Rahab : Fais sortir les hommes qui sont venus chez toi et qui sont entrés dans ta maison ; car ils sont venus pour explorer tout le pays.
\VS{4}Or la femme prit les deux hommes et les cacha ; et elle dit : Il est vrai que des hommes sont venus chez moi, mais je ne savais pas d'où ils étaient ;
\VS{5}et comme on fermait la porte sur le soir, ces hommes sont sortis ; je ne sais pas où ces hommes sont allés ; poursuivez-les bien vite car vous les atteindrez.
\VS{6}Or elle les avait fait monter sur le toit et les avait cachés sous des tiges de lin qu'elle avait arrangées sur le toit.
\VS{7}Et quelques gens les poursuivirent par le chemin du Jourdain jusqu'aux passages ; et on ferma la porte après que ceux qui les poursuivaient furent sortis.
\VS{8}Or, avant qu'ils se couchent, elle monta vers eux sur le toit ;
\VS{9}et leur dit : Je sais que Yahweh vous a donné ce pays, et que la terreur de votre nom nous a saisis, et que tous les habitants du pays perdent courage à cause de vous\FTNT{Ex. 23:27.}.
\VS{10}Car nous avons entendu que Yahweh a mis à sec devant vous les eaux de la Mer Rouge à votre sortie du pays d'Egypte ; et ce que vous avez fait aux deux rois des Amoréens qui étaient de l'autre côté du Jourdain, à Sihon et à Og, que vous avez détruits complètement en les dévouant par le moyen de l'interdit.
\VS{11}Nous l'avons entendu, et notre cœur a fondu, et depuis aucun homme n'a eu le courage à cause de vous. Car Yahweh, votre Dieu, est le Dieu des cieux en haut et de la terre\FTNT{De. 4:39.} en bas.
\VS{12}Maintenant donc, je vous prie, jurez-moi par Yahweh, que puisque j'ai usé de bonté envers vous, vous userez aussi de bonté envers la maison de mon père, 
\VS{13}et que vous me donnerez un signe de votre fidélité\FTNT{La couleur cramoisi s'obtient grâce à la femelle cochenille aptère qui contient dans son corps et dans ses œufs un pigment rouge à base d'acide carminique qui permet à l'insecte et à ses larves de se protéger des prédateurs. Au moment de la ponte, cette dernière fixe fermement son corps au tronc d'un arbre puis libère ses œufs qui demeurent ainsi protégés en dessous d'elle jusqu'à leur éclosion. Ensuite, l'insecte meurt en libérant cette substance rouge qui se propage sur tout son corps et sur le bois hôte. C'est ce fluide que l'homme récupère pour en faire un colorant à la couleur caractéristique. Une subtile analogie peut être faire entre la cochenille et le Seigneur qui a versé son sang à la croix pour nous donner la vie. « Et moi, je suis un ver, et non un homme, l'opprobre des hommes et le méprisé du peuple » (Ps. 22 :7).} d'une ferme assurance que vous laisserez vivre mon père, ma mère, mes frères, mes sœurs, et tous ceux qui leur appartiennent, et que vous sauverez nos âmes de la mort.
\VS{14}Et ces hommes lui répondirent : Nos personnes répondront pour vous jusqu'à la mort, pourvu que vous ne divulguez pas cette affaire ; et quand Yahweh nous aura donné le pays nous userons envers toi de bonté et de vérité. 
\TextTitle{Les espions s'enfuient aidés par Rahab}
\VS{15}Elle les fit donc descendre avec une corde par la fenêtre ; car sa maison était sur la muraille de la ville, et elle habitait sur la muraille de la ville. 
\VS{16}Et elle leur dit : Allez à la montagne, de peur que ceux qui vous poursuivent ne vous rencontrent, et cachez-vous là pendant trois jours jusqu'à ce qu'ils soient de retour. Après cela vous suivrez votre chemin.
\VS{17}Et ces hommes lui dirent : Voici comment nous serons quittes de ce serment que tu nous as fait faire.
\VS{18}Voici, quand nous entrerons dans le pays, tu lieras ce cordon de fil d'écarlate à la fenêtre par laquelle tu nous auras fait descendre, et tu recueilleras chez toi, dans cette maison, ton père et ta mère, tes frères, et toute la famille de ton père.
\VS{19}Et quiconque sortira hors de la porte de ta maison, son sang sera sur sa tête, et nous en serons quittes ; mais quiconque sera avec toi, dans la maison, son sang sera sur notre tête si quelqu'un met la main sur lui.
\VS{20}Et si tu divulgues cette affaire, nous serons quittes du serment que tu nous as fait faire.
\VS{21}Et elle répondit : Que cela soit ainsi que vous l'avez dit. Alors elle les laissa aller. Ils s'en allèrent et elle lia le cordon de fil d'écarlate à la fenêtre.
\VS{22}Et ils marchèrent et arrivèrent à la montagne, où ils restèrent trois jours, jusqu'à ce que ceux qui les poursuivaient soient de retour. Ceux qui les poursuivaient les cherchèrent par tout le chemin, mais ils ne les trouvèrent pas.
\VS{23}Ainsi ces deux hommes s'en retournèrent, descendirent de la montagne, passèrent le Jourdain. Ils vinrent auprès de Josué, fils de Nun. Ils lui racontèrent toutes les choses qui leur étaient arrivées.
\VS{24}Et ils dirent à Josué : Certainement, Yahweh a livré tout le pays entre nos mains, et même tous les habitants ont perdu le courage à notre vue.
\Chap{3}
\TextTitle{Israël traverse le Jourdain à sec}
\VerseOne{}Or Josué se leva de bon matin, lui et tous les enfants d'Israël partirent de Sittim, ils vinrent jusqu'au Jourdain, et ils logèrent là cette nuit, avant de le traverser.
\VS{2}Et au bout de trois jours les officiers traversèrent le milieu du camp,
\VS{3}et donnèrent cet ordre au peuple en disant : Dès que vous verrez l'arche de l'alliance de Yahweh, votre Dieu, portée par les prêtres, les Lévites, vous partirez de votre quartier, et vous marcherez après elle.
\VS{4}Et afin que vous n'approchez pas d'elle, il y aura entre vous et elle une distance de la mesure d'environ deux mille coudées. Elle vous fera connaître le chemin par lequel vous devez marcher ; car vous n'avez pas encore passé par ce chemin.
\VS{5}Josué dit au peuple : Sanctifiez-vous, car Yahweh fera demain des choses merveilleuses au milieu de vous\FTNT{Ex. 19:10-11.}.
\VS{6}Josué parla aussi aux prêtres, en disant : Portez l'arche de l'alliance, et passez devant le peuple. Ainsi ils portèrent l'arche de l'alliance, et marchèrent devant le peuple.
\VS{7}Or Yahweh dit à Josué : Aujourd'hui je commencerai à t'élever aux yeux de tout Israël, afin qu'ils sachent que je serai aussi avec toi, comme j'ai été avec Moïse.
\VS{8}Tu donneras cet ordre aux prêtres qui portent l'arche de l'alliance, en leur disant : Dès que vous arriverez au bord des eaux du Jourdain, vous vous arrêterez dans le Jourdain.
\VS{9}Et Josué dit aux enfants d'Israël : Approchez-vous d'ici, et écoutez les paroles de Yahweh, votre Dieu.
\VS{10}Puis Josué dit : Vous reconnaîtrez à ceci que le Dieu vivant est au milieu de vous et qu'il chassera et déshéritera devant vous les Cananéens, les Héthiens, les Héviens, les Phéréziens, les Guirgasiens, les Amoréens et les Jébusiens.
\VS{11}Voici, l'arche de l'alliance du Seigneur de toute la terre va passer devant vous dans le Jourdain.
\VS{12}Maintenant, prenez douze hommes des tribus d'Israël, un homme de chaque tribu.
\VS{13}Et il arrivera qu'aussitôt que les plantes des pieds des prêtres qui portent l'arche de Yahweh, le Seigneur de toute la terre, seront posés dans les eaux du Jourdain, les eaux du Jourdain seront coupées, les eaux, dis-je, qui descendent d'en haut, et elles s'arrêteront en un monceau\FTNT{Ps. 114:3.}.
\VS{14}Et il arriva que le peuple étant parti de ses tentes pour passer le Jourdain, et les prêtres qui portaient l'arche de l'alliance, étaient devant le peuple.
\VS{15}Aussitôt que ceux qui portaient l'arche furent arrivés au Jourdain, et que les pieds des prêtres qui portaient l'arche furent mouillés au bord de l'eau. Le Jourdain regorge par-dessus toutes ses rives durant tout le temps de la moisson\FTNT{1 Ch. 12:15}.
\VS{16}Les eaux qui descendent d'en haut, s'arrêtèrent, et s'élevèrent en un monceau, à une très grande distance, depuis la ville d'Adam, qui est à côté de Tsarthan ; et celles d'en bas, qui descendaient vers la mer de la plaine, qui est la mer salée, furent totalement coupées. Le peuple passa vis-à-vis de Jéricho.
\VS{17}Mais les prêtres qui portaient l'arche de l'alliance de Yahweh, s'arrêtèrent de pied ferme sur le sec, au milieu du Jourdain, pendant que tout Israël passait à sec, jusqu'à ce que tout le peuple ait achevé de passer le Jourdain.
\Chap{4}
\TextTitle{Josué dresse un monument de pierres en souvenir de la traversée}
\VerseOne{}Or il arriva que quand tout le peuple eut achevé de passer le Jourdain, que Yahweh parla à Josué et dit :
\VS{2}Prenez douze hommes parmi le peuple, un homme de chaque tribu.
\VS{3}et donnez-leur cet ordre, en disant : Prenez ici, du milieu du Jourdain, de la place où les prêtres se sont arrêtés de pied ferme, douze pierres, que vous emporterez avec vous, et vous les poserez au lieu où vous passerez cette nuit.
\VS{4}Josué appela les douze hommes qu'il choisit parmi les enfants d'Israël, un homme de chaque tribu.
\VS{5}Et il leur dit : Passez devant l'arche de Yahweh, votre Dieu, au milieu du Jourdain, et que chacun de vous charge une pierre sur son épaule, selon le nombre des tribus des enfants d'Israël ;
\VS{6}afin que cela soit un signe au milieu de vous. Et quand vos fils interrogeront à l'avenir leurs pères, en disant : Que signifient ces pierres-ci ?
\VS{7}Alors vous leur répondrez : Les eaux du Jourdain ont été coupées devant l'arche de l'alliance de Yahweh ; lorsqu'elle passa le Jourdain, les eaux du Jourdain ont été arrêtées ; c'est pourquoi ces pierres-là seront à jamais un souvenir pour les enfants d'Israël.
\VS{8}Les enfants d'Israël firent donc comme Josué leur avait ordonné. Ils prirent douze pierres du milieu du Jourdain, comme Yahweh l'avait ordonné à Josué, selon le nombre des tribus des enfants d'Israël. Ils les emportèrent avec eux et les posèrent au lieu où ils devaient passer la nuit.
\VS{9}Josué dressa aussi douze pierres au milieu du Jourdain, à l'endroit où les pieds des prêtres qui portaient l'arche de l'alliance s'étaient arrêtés ; et elles y sont restées jusqu'à ce jour.
\VS{10}Les prêtres donc qui portaient l'arche se tinrent debout au milieu du Jourdain, jusqu'à ce que tout ce que Yahweh avait ordonné à Josué de dire au peuple soit accompli, selon tout ce que Moïse avait prescrit à Josué. Et le peuple se hâta de passer.
\VS{11}Et quand tout le peuple eut achevé de passer, alors l'arche de Yahweh et les prêtres passèrent devant le peuple.
\VS{12}Et les fils de Ruben, les fils de Gad, et la demi-tribu de Manassé passèrent en armes devant les enfants d'Israël, comme Moïse le leur avait dit\FTNT{No. 32:20-29.}.
\VS{13}Ils passèrent, dis-je, dans les plaines de Jérico environ quarante mille hommes en équipage de guerre, devant Yahweh, pour combattre. 
\VS{14}Ce jour-là, Yahweh éleva Josué à la vue de tout Israël, et ils le craignirent, comme ils avaient craint Moïse, tous les jours de sa vie.
\VS{15}Yahweh parla à Josué, et dit :
\VS{16}Ordonne aux prêtres qui portent l'arche du témoignage qu'ils montent hors du Jourdain.
\VS{17}Et Josué donna cet ordre aux prêtres, en disant : Montez hors du Jourdain.
\VS{18}Or sitôt que les prêtres, qui portaient l'arche de l'alliance de Yahweh furent montés hors du milieu du Jourdain, et qu'ils eurent mis la plante de leurs pieds sur le sec, les eaux du Jourdain retournèrent à leur place, et coulèrent comme auparavant sur tous les rivages.
\VS{19}Le peuple donc monta hors du Jourdain le dixième jour du premier mois, et il campa à Guilgal, à l'orient de Jéricho.
\VS{20}Josué aussi dressa à Guilgal les douze pierres qu'ils avaient prises du Jourdain.
\VS{21}Et il parla aux enfants d'Israël et leur dit : Quand vos enfants interrogeront à l'avenir leurs pères, et leur diront : Que signifient ces pierres-ci ?
\VS{22}Vous l'apprendrez à vos enfants, en leur disant : Israël a passé ce Jourdain à sec.
\VS{23}Car Yahweh, votre Dieu, a fait tarir les eaux du Jourdain devant vous jusqu'à ce que vous eussiez passé, comme Yahweh, votre Dieu, l'avait fait à la Mer Rouge, qu'il mit à sec devant nous, jusqu'à ce que nous eussions passé,
\VS{24}afin que tous les peuples de la terre sachent que la main de Yahweh est puissante, et afin que vous ayez toujours la crainte de Yahweh, votre Dieu.
\Chap{5}
\TextTitle{La crainte s'empare des Amoréens}
\VerseOne{}Or il arriva qu'aussitôt que tous les rois des Amoréens qui étaient au-delà du Jourdain, vers l'occident, et tous les rois des Cananéens qui étaient près de la mer, apprirent que Yahweh avait mis à sec les eaux du Jourdain devant les enfants d'Israël, jusqu'à ce que nous eussions passé, leur cœur fut fondu, et il n'y avait plus de courage en eux à cause des enfants d'Israël.
\TextTitle{Israël circoncis à nouveau ; la fin de la manne}
\VS{2}En ce temps-là, Yahweh dit à Josué : Fais-toi des couteaux de pierre tranchants, et circoncis de nouveau les enfants d'Israël, une seconde fois.
\VS{3}Et Josué se fit des couteaux de pierre tranchants, et circoncit les enfants d'Israël sur la colline d'Araloth.
\VS{4}Or la raison pour laquelle Josué les circoncit, c'est que tout le peuple sorti d'Egypte, tous les mâles, dis-je, hommes de guerre étaient morts en chemin dans le désert, après leur sortie d'Egypte.
\VS{5}Et tout le peuple sorti d'Egypte était circoncis, mais aucun du peuple né dans le désert en chemin n'avait été circoncis, après leur sortie d'Egypte.
\VS{6}Car les enfants d'Israël avaient marché dans le désert quarante ans jusqu'à ce que soit consummée toute la nation des hommes de guerre qui étaient sortis d'Egypte, et qui n'avaient point écouté la voix de Yahweh ; auxquels Yahweh avait juré qu'il ne leur laisserait point voir le pays qu'il avait juré à leurs pères de nous donner, pays où coulent le lait et le miel\FTNT{No. 14:32-33.}.
\VS{7}Et il a suscité à leur place leurs enfants que Josué circoncit, parce qu'ils étaient incirconcis ; car on ne les avait pas circoncis pendant le voyage.
\VS{8}Et quand on eut achevé de circoncire tout le peuple, ils restèrent dans leur camp, jusqu'à ce qu'ils soient guéris.
\VS{9}Et Yahweh dit à Josué : Aujourd'hui j'ai roulé de dessus vous l'opprobre de l'Egypte. Et ce lieu-là fut appelé Guilgal jusqu'à ce jour.
\VS{10}Ainsi les enfants d'Israël campèrent à Guilgal, et célébrèrent la Pâque le quatorzième jour du mois, sur le soir, dans les plaines de Jéricho\FTNT{Ex. 12:6.}.
\VS{11}Et dès le lendemain de la Pâque, ils mangèrent du blé du pays, savoir, des pains sans levain et du grain rôti, en ce même jour\FTNT{Ex. 12:39 ; Lé. 2:14}.
\VS{12}Et la manne cessa dès le lendemain de la Pâque, après qu'ils eurent manger du blé du pays ; les enfants d'Israël n'eurent plus de manne, mais ils mangèrent les récoltes de la terre de Canaan cette année-là\FTNT{Ex. 16:35.}.
\TextTitle{Rencontre avec le chef de l'armée de Yahweh}
\VS{13}Or il arriva, comme Josué était près de Jéricho, qu'il leva les yeux et regarda. Voici, un homme qui avait son épée nue à la main, se tenait debout devant lui. Josué alla vers lui et lui dit : Es-tu des nôtres ou de nos ennemis ?
\VS{14}Et il répondit : Non, mais je suis le Chef de l'armée de Yahweh, je viens maintenant. Josué tomba à terre sur son visage, l'adora, et lui dit : Qu'est-ce que mon Seigneur dit à son serviteur ?
\VS{15}Et le Chef de l'armée de Yahweh dit à Josué : Délie tes souliers de tes pieds ; car le lieu sur lequel tu te tiens est saint\FTNT{Ex. 3:5.}. Et Josué fit ainsi.
\Chap{6}
\TextTitle{Jéricho miraculeusement livré à Israël ; Rahab sauvée}
\VerseOne{}Or Jéricho était barricadée et fermée soigneusement, à cause des enfants d'Israël. Personne ne sortait, et personne n'entrait.
\VS{2}Et Yahweh dit à Josué : Regarde, j'ai livré entre tes mains Jéricho et son roi, ses hommes vaillants.
\VS{3}Vous tous donc, hommes de guerre, vous ferez le tour de la ville, en tournant une fois autour d'elle. Tu feras ainsi durant six jours.
\VS{4}Et Sept prêtres porteront sept shofars retentissants devant l'arche. Mais au septième jour, vous ferez sept fois le tour de la ville et les prêtres sonneront des shofars.
\VS{5}Et quand ils sonneront avec la corne de bélier, aussitôt que vous entendrez le son du shofar retentissant, tout le peuple poussera un grand cri de joie et la muraille de la ville tombera sur elle. Et le peuple montera, les hommes devant lui.
\VS{6}Josué donc, fils de Nun, appela les prêtres et leur dit : Portez l'arche de l'alliance et que sept prêtres portent sept shofars devant l'arche de Yahweh.
\VS{7}Il dit aussi au peuple : Passez et faites le tour de la ville, que tous ceux qui seront armés passent devant l'arche de Yahweh.
\VS{8}Et quand Josué eut parlé au peuple, les sept prêtres qui portaient les sept cornes de béliers devant Yahweh passèrent et sonnèrent des cornes. Et l'arche de l'alliance de Yahweh les suivait.
\VS{9}Et les hommes qui étaient armés marchaient devant les prêtres qui sonnaient des shofars ; mais l'arrière-garde suivait derrière l'arche ; on sonnait des shofars en marchant.
\VS{10}Or Josué avait donné cet ordre au peuple, en disant : Vous ne pousserez point de cris de joie et vous ne ferez point entendre votre voix. Et il ne sortira point un seul mot de votre bouche, jusqu'au jour où je vous dirai : Poussez des cris de joie ! Alors vous crierez.
\VS{11}L'arche de Yahweh fit ainsi le tour de la ville, en tournant tout autour une fois, puis on revint au camp, et on y passa la nuit.
\VS{12}Ensuite Josué se leva de bon matin, et les prêtres portèrent l'arche de Yahweh.
\VS{13}Et les sept prêtres qui portaient les sept cornes de bélier devant l'arche de Yahweh se mirent en marche et sonnèrent du shofar. Et les hommes armés allaient devant eux ; puis l'arrière-garde suivait l'arche de Yahweh ; on sonnait des shofars en marchant.
\VS{14}Ainsi ils firent une fois le tour de la ville le deuxième jour, et ils retournèrent au camp. Ils firent de même durant six jours.
\VS{15}Mais quand le septième jour fut venu, ils se levèrent dès le matin à l'aube du jour, et ils firent sept fois le tour de la ville de la même manière ; ce fut le seul jour où ils firent sept fois le tour de la ville.
\VS{16}Et à la septième fois, comme les prêtres sonnaient des shofars, Josué dit au peuple : Poussez des cris de joie, car Yahweh vous a donné la ville !
\VS{17}La ville sera dévouée par le moyen de l'interdit à Yahweh, elle et toutes les choses qui y sont ; seulement Rahab, la prostituée\FTNT{Rahab sauva sa famille par sa foi en Dieu (Ac. 16:31). Voir Josué 2.}, vivra, elle et tous ceux qui seront avec elle dans la maison, parce qu'elle a caché soigneusement les messagers que nous avions envoyés.
\VS{18}Mais quoi qu'il en soit gardez-vous de l'interdit, de peur que vous ne vous mettiez en interdit, et que vous ne mettriez le camp d'Israël en interdit et que vous le troubliez\FTNT{De. 7:26.}.
\VS{19}Mais tout l'argent et tout l'or, tous les objets d'airain et de fer seront consacrés à Yahweh, ils entreront dans le trésor de Yahweh\FTNT{No. 31:54.}.
\VS{20}Le peuple donc poussa de cris de joie et on sonna des shofars. Et quand le peuple entendit le son des shofars, il poussa de grands cris de joie et la muraille tomba sur elle-même\FTNT{Hé. 11:30.}. Alors le peuple monta dans la ville, les hommes devant le peuple. Et ils prirent la ville. 
\VS{21}Et ils la dévouèrent entièrement par le moyen de l'interdit, et passèrent au fil de l'épée tout ce qui était dans la ville, depuis l'homme jusqu'à la femme, depuis l'enfant jusqu'au vieillard, même jusqu'aux bœufs, aux brebis et aux ânes.
\VS{22}Mais Josué dit aux deux hommes qui avaient espionné le pays : Entrez dans la maison de cette femme prostituée, et faites-la sortir de là, avec tous ceux qui lui appartiennent, selon que vous lui avez juré.
\VS{23}Les jeunes hommes donc qui avaient espionné le pays, entrèrent et firent sortir Rahab, et son père, et sa mère et ses frères, avec tous ceux qui lui appartenaient ; ils firent aussi sortir toutes les familles qui lui appartenaient, et les mirent hors du camp d'Israël.
\VS{24}Puis ils allumèrent le feu et brûlèrent la ville et tout ce qui s'y trouvait ; seulement ils mirent l'argent et l'or, les objets d'airain et de fer dans le trésor de la maison de Yahweh.
\VS{25}Ainsi Josué sauva la vie à Rahab la prostituée, la maison de son père, et tous ceux qui lui appartenaient ; et elle a habité au milieu d'Israël jusqu'à ce jour, parce qu'elle avait caché les messagers que Josué avait envoyés pour explorer Jéricho.
\VS{26}Et en ce temps-là Josué jura, en disant : Maudit soit devant Yahweh l'homme qui se mettra à rebâtir cette ville de Jéricho ! Il la fondera sur son premier-né, et il posera ses portes sur son plus jeune fils\FTNT{Cette parole s'est accomplie en 1 R. 16:34.}.
\VS{27}Yahweh fut avec Josué, et sa renommée se répandit dans tout le pays.
\Chap{7}
\TextTitle{Israël battu à Aï suite au péché d'Acan}
\VerseOne{}Mais les enfants d'Israël se rendirent coupables au sujet de l'interdit. Car Acan, fils de Carmi, fils de Zabdi, fils de Zérach, de la tribu de Juda, prit de l'interdit, et la colère de Yahweh s'enflamma contre les enfants d'Israël.
\VS{2}Car Josué envoya de Jéricho des hommes vers Aï, qui est près de Beth-Aven, à l'orient de Béthel. Il leur parla, et dit : Montez, et reconnaissez le pays. Ces hommes donc montèrent et reconnurent Aï.
\VS{3}Et étant retournés vers Josué, ils lui dirent : Que tout le peuple n'y monte point mais qu'environ deux mille ou trois mille hommes y montent, et ils battront Aï. Ne fatigue pas tout le peuple en l'envoyant là, car ils sont en petit nombre.
\VS{4}Ainsi, environ trois mille hommes du peuple y montèrent, mais ils s'enfuirent devant les gens d'Aï.
\VS{5}Et les gens d'Aï leur tuèrent environ trente-six hommes ; car ils les poursuivirent depuis la porte jusqu'à Schebarim, et les battirent à la descente. Le cœur du peuple se fondit et devint comme de l'eau.
\VS{6}Alors Josué déchira ses vêtements, et se jeta sur le visage contre terre, devant l'arche de Yahweh jusqu'au soir, lui et les anciens d'Israël, et ils jettèrent de la poussière sur leur tête.
\VS{7}Et Josué dit : Helas ! Seigneur Yahweh, pourquoi as-tu fait si magnifiquement passer le Jourdain à ce peuple, pour nous livrer entre les mains des Amoréens, et nous faire périr ? Oh ! Que n'avons-nous eu dans l'esprit de demeurer de l'autre côté du Jourdain !
\VS{8}Hélas ! Seigneur, que dirai-je, puisqu'Israël a tourné le dos devant ses ennemis ?
\VS{9}Les Cananéens et tous les habitants du pays l'entendront ; ils nous envelepperont, et ils retrancheront notre nom de dessus la terre. Et que feras-tu à ton grand Nom ?
\VS{10}Alors Yahweh dit à Josué : Lève-toi ! Pourquoi te jettes-tu ainsi le visage contre terre ?
\VS{11}Israël a péché ; ils ont transgressé mon alliance que je leur avais prescrite, même ils ont pris de l'interdit, même ils en ont dérobé, même ils ont menti, et même ils l'ont caché parmi leurs objets\FTNT{Il est impossible de remporter une victoire contre Satan en ayant avec soi des choses qui lui appartiennent (Jn. 14:30). Celui qui pèche est du diable nous dit la Parole de Dieu (1 Jn. 3:4-10). Les grandes victoires sont remportées par ceux qui se sanctifient et invoquent le Nom de Jésus-Christ.}.
\VS{12}C'est pourquoi les enfants d'Israël ne pourront subsister devant leurs ennemis ; ils tourneront le dos devant leurs ennemis ; car ils sont devenus un interdit. Je ne serai plus avec vous si vous ne détruisez pas l'interdit du milieu de vous.
\VS{13}Lève-toi, sanctifie le peuple, et dis : Sanctifiez-vous pour demain ; car ainsi parle Yahweh, le Dieu d'Israël : Il y a de l'interdit au milieu de toi, Israël ! Tu ne pourras subsister et faire face à tes ennemis jusqu'à ce que vous ayez ôté l'interdit du milieu de vous.
\VS{14}Vous vous approcherez donc le matin selon vos tribus ; et la tribu que Yahweh aura saisi s'approchera selon les familles, et la famille que Yahweh aura saisie s'approchera selon les maisons, et la maison que Yahweh aura saisie s'approchera selon les hommes.
\VS{15}Alors celui qui aura été saisi avec l'interdit sera brûlé au feu, lui et tout ce qui lui appartient parce qu'il a transgressé l'alliance de Yahweh, et qu'il a commis une infamie en Israël.
\VS{16}Josué donc se leva de bon matin, et fit approcher Israël selon ses tribus, et la tribu de Juda fut saisie.
\VS{17}Puis il fit approcher les familles de Juda, et la famille de Zérach fut saisie. Puis il fit approcher les hommes de la famille de ceux qui étaient descendants de Zérach, et Zabdi fut saisie.
\VS{18}Et quand il fit approcher la maison de Zabdi par hommes, Acan fils de Carmi, fils de Zabdi, fils de Zérach, de la tribu de Juda, fut saisi.
\VS{19}Josué dit à Acan : Mon fils, je te prie donne gloire à Yahweh, le Dieu d'Israël, et fais-lui confession. Déclare-moi je te prie ce que tu as fait, ne me le cache point.
\VS{20}Et Acan répondit à Josué, et dit : J'ai péché il est vrai, contre Yahweh, le Dieu d'Israël, et voici ce que j'ai fait.
\VS{21}J'ai vu parmi le butin un beau manteau de Schinear\FTNT{Ge. 10:6-12.}, deux cents sicles d'argent et un lingot d'or du poids de cinquante sicles ; je les ai convoités, je les ai pris et voilà, ces choses sont cachées dans la terre au milieu de ma tente, et l'argent est sous le manteau.
\VS{22}Alors Josué envoya des messagers qui coururent à cette tente ; et voici, le manteau était caché dans la tente d'Acan, et l'argent sous le manteau.
\VS{23}Ils les tirèrent donc du milieu de la tente et les apportèrent à Josué et à tous les enfants d'Israël, et ils les déposèrent devant Yahweh.
\VS{24}Alors Josué et tout Israël avec lui, prirent Acan, fils de Zérach, l'argent, le manteau, le lingot d'or, ses fils et ses filles, ses bœufs, ses ânes et ses brebis, sa tente et tout ce qui lui appartenait, et ils les firent monter dans la vallée d'Acor.
\VS{25}Et Josué dit : Pourquoi nous as-tu troublés ? Yahweh te troublera aujourd'hui. Et tout Israël le lapida avec des pierres, et les brûlèrent au feu, après les avoir lapidés avec des pierres.
\VS{26}Et ils dressèrent sur lui un grand monceau de pierres, qui dure jusqu'à ce jour. Et Yahweh apaisa l'ardeur de sa colère. C'est pourquoi ce lieu-là a été appelé jusqu'à aujourd'hui, la vallée d'Acor\FTNT{2 S. 18:17.}.
\Chap{8}
\TextTitle{Victoire d'Israël à Aï}
\VerseOne{}Puis Yahweh dit à Josué : Ne crains point, et ne t'effraie de rien\FTNT{De. 1:21 ; De. 7:18.} ! Prends avec toi tout le peuple propre à la guerre et lève-toi, et monte contre Aï. Regarde, j'ai livré entre tes mains le roi d'Aï et son peuple, sa ville et son pays.
\VS{2}Et tu traiteras Aï et son roi, comme tu as fais Jéricho et son roi : Seulement vous pillerez pour vous le butin et les bêtes. Place des gens en embuscade derrière la ville.
\VS{3}Josué donc se leva avec tout le peuple propre à la guerre, pour monter contre Aï. Josué choisit trente mille vaillants hommes armés, et les envoya de nuit.
\VS{4}Et il leur donna cet ordre en disant : Voyez, vous qui serez en embuscade derrière la ville ; ne vous éloignez pas beaucoup de la ville, mais tenez-vous prêts.
\VS{5}Et moi et tout le peuple qui est avec moi, nous nous approcherons de la ville. Et quand ils sortiront à notre rencontre, comme ils ont fait la première fois, nous nous enfuirons devant eux.
\VS{6}Ainsi ils sortiront après nous, jusqu'à ce que nous les ayons attirés hors de la ville ; car ils diront : Ils fuient devant nous comme la première fois ; parce que nous fuirons devant eux.
\VS{7}Alors vous vous lèverez de l'embuscade, et vous vous saisirez de la ville ; car Yahweh, votre Dieu, la livrera entre vos mains.
\VS{8}Et quand vous aurez pris la ville, vous y mettrez le feu ; vous agirez selon la parole de Yahweh. Regardez, je vous l'ai ordonné.
\VS{9}Josué donc les envoya, et ils allèrent se mettre en embuscade, et se tinrent entre Béthel et Aï, à l'occident d'Aï. Mais Josué resta cette nuit-là au milieu du peuple.
\VS{10}Puis Josué se leva de bon matin, et dénombra le peuple ; et il monta lui et les anciens d'Israël, devant le peuple vers Aï.
\VS{11}Et tout le peuple propre à la guerre qui étaient avec lui, monta et s'approcha ; et ils vinrent en face de la ville et campèrent du côté du nord d'Aï ; et la vallée était entre lui et Aï.
\VS{12}Il prit aussi environ cinq mille hommes, et les mit en embuscade entre Béthel et Aï, à l'occident de la ville.
\VS{13}Après que tout le camp eut pris position au nord de la ville, et l'embuscade à l'occident de la ville, cette nuit-là, Josué s'avança au milieu de la vallée.
\VS{14}Or il arriva qu'aussitôt que le roi de Aï l'eut vu, les hommes de la ville se hâtèrent, et se levèrent de bon matin, et au temps marqué, le Roi et tout son peuple sortirent à la campagne contre Israël pour le combattre. Or il ne savait pas qu'il y eût des gens en embuscade contre lui derrière la ville.
\VS{15}Alors Josué et tout Israël feignirent d'être battus devant eux, et ils s'enfuirent par le chemin du désert.
\VS{16}Alors tout le peuple qui était dans la ville d'Aï, fut assemblé à grand cri pour les poursuivre. Ils poursuivirent Josué, et ils furent ainsi attirés loin de la ville.
\VS{17}Il ne resta pas un seul homme dans Aï ni dans Béthel qui ne sortit contre Israël. Ils laissèrent la ville ouverte, et ils poursuivirent Israël.
\VS{18}Alors Yahweh dit à Josué : Etends vers Aï l'étandard qui est dans ta main, car je la livrerai entre tes mains. Et Josué étendit vers la ville l'étandard qui était dans sa main.
\VS{19}Aussitôt qu'il eut étendu sa main, les hommes qui étaient en embuscade se levèrent précipitamment du lieu où ils étaient ; ils pénétrèrent dans la ville, la prirent, et se hâtèrent de mettre le feu dans la ville.
\VS{20}Et les gens d'Aï, se tournant derrière eux, regardèrent ; et voici, la fumée de la ville montait vers le ciel, et ils n'y eut en eux aucune force pour fuir ça ou là. Et le peuple qui fuyait vers le désert se tourna contre ceux qui le poursuivaient ;
\VS{21}Et Josué et tout Israël, voyant que ceux qui étaient en embuscade avaient pris la ville, et que la fumée de la ville montait, se retournèrent, et frappèrent les gens d'Aï.
\VS{22}Les autres aussi sortirent de la ville contre eux, et ils furent enveloppés par les Israélites ayant les uns d'un côté et les autres de l'autre. Ils furent tellement battus qu'il n'en laissa aucun qui resta en vie ou qui échappât\FTNT{De 7:2.} ;
\VS{23}ils prirent aussi vivant le roi d'Aï, et le présentèrent à Josué.
\VS{24}Et quand les Israélites eurent achevé de tuer tous les habitants d'Aï dans la campagne, dans le désert, où ils les avaient poursuivis, et que tous furent tombés sous le tranchant de l'épée, jusqu'à être entièrement défaits, tous les Israélites revinrent vers Aï, et la frappèrent au tranchant de l'épée.
\VS{25}Et tous ceux qui tombèrent ce jour-là, tant des hommes que des femmes, furent au nombre de douze mille, tous gens d'Aï.
\VS{26}Et Josué ne retira point sa main qu'il tenait étendue avec l'étandard, jusqu'à ce que tous les habitants d'Aï aient été entièrement dévoués par le moyen de l'interdit.
\VS{27}Seulement les Israélites pillèrent pour eux les bêtes et le butin de cette ville-là, suivant ce que Yahweh avait prescrit à Josué\FTNT{No. 31:22-26.}.
\VS{28}Josué donc brûla Aï, et en fit un monceau perpétuel de ruines, jusqu'à aujourd'hui.
\VS{29}Puis il fit pendre le roi d'Aï à un arbre jusqu'au temps du soir. Et comme le soleil se couchait, Josué ordonna qu'on descende de l'arbre son cadavre ; on le jeta à l'entrée de la porte de la ville, puis on dressa sur lui un grand amas de pierres, qui subsiste encore aujourd'hui.
\TextTitle{Sacrifices offerts à Yahweh et lecture de la loi de Moïse}
\VS{30}Alors Josué bâtit un autel à Yahweh, le Dieu d'Israël, sur la montagne d'Ebal,
\VS{31}comme Moïse, serviteur de Yahweh, l'avait ordonné aux enfants d'Israël, ainsi qu'il est écrit dans le livre de la loi de Moïse : Il fit cet autel de pierres brutes sur lesquelles personne ne porta le fer\FTNT{L'autel devait être construit avec des pierres taillées par Dieu lui-même dans la nature (Ex. 20:25). L'Eglise du Seigneur est construite avec des pierres vivantes, taillées par Dieu et non par les hommes (Mt. 16:18). Babylone est construite avec des briques, œuvre des hommes (Ge. 11:1-3).} ; et ils offrirent dessus des holocaustes à Yahweh, et sacrifièrent des sacrifices d'offrande de paix\FTNT{Voir commentaire en Lé. 3:1.}.
\VS{32}Il écrivit aussi là, sur les pierres une copie de la loi que Moïse avait mise par écrit devant les enfants d'Israël.
\VS{33}Et tout Israël, ses anciens, ses officiers et ses juges étaient des deux côtés de l'arche, en face des prêtres qui sont de la race de Lévi, qui portaient l'arche de l'alliance de Yahweh, les étrangers comme les Hébreux naturels, une moitié du côté du mont Garizim\FTNT{Voir Jn. 4:19-24.}, et l'autre moitié du côté du mont Ebal, selon l'ordre qu'avait précédemment donné Moïse, serviteur de Yahweh, de bénir le peuple d'Israël.
\VS{34}Et après cela, il lut tout haut toutes les paroles de la loi, tant les bénédictions que les malédictions, selon tout ce qui est écrit dans le livre de la loi.
\VS{35}Il n'y eut rien de tout ce que Moïse avait prescrit, que Josué ne lise tout haut devant toute l'assemblée d'Israël, des femmes et des petits-enfants, et des étrangers qui marchaient au milieu d'eux.
\Chap{9}
\TextTitle{Josué tombe dans la ruse des Gabaonites}
\VerseOne{}Or, dès que tous les rois qui étaient au-delà du Jourdain, dans la montagne et dans la plaine, et sur toute la côte de la grande mer, jusque près du Liban, les Héthiens, les Amoréens, les Cananéens, les Phéréziens, les Héviens et les Jébusiens, eurent appris ces choses,
\VS{2}ils s'assemblèrent tous d'un commun accord pour faire la guerre à Josué et à Israël.
\VS{3}Mais les habitants de Gabaon\FTNT{Les Gabaonites étaient rusés. Ils poussèrent les Hébreux à faire alliance avec eux, comme le font les faux chrétiens aujourd'hui (Esd. 4 ; Es. 30:1). Il n'y a pas de rapport entre la lumière et les ténèbres (2 Co. 6:14-18). Combien de chrétiens ne se font-ils pas avoir par des loups ravisseurs dans le domaine du mariage ?}, ayant entendu ce que Josué avait fait à Jéricho et à Aï,
\VS{4}usèrent de ruse, car ils se mirent en chemin et contrefirent les ambassadeurs et prirent de vieux sacs pour leurs ânes, et de vieilles outres de vin déchirées et recousues,
\VS{5}Et ils avaient à leurs pieds de vieux souliers raccommodés et de vieux habits sur eux ; et tout le pain qu'ils avaient pour nourriture était sec et moisi.
\VS{6}Et ils arrivèrent auprès de Josué au camp de Guilgal, et lui dirent, ainsi qu'à tous les hommes d'Israël : Nous sommes venus d'un pays éloigné, maintenant donc traitez alliance avec nous.
\VS{7}Et les hommes d'Israël répondirent à ces Héviens : Peut-être que vous habitez au milieu de nous, et comment traiterions-nous alliance avec vous ?
\VS{8}Mais ils dirent à Josué : Nous sommes tes serviteurs. Alors Josué leur dit : Qui êtes-vous ? Et d'où venez-vous ?
\VS{9}Ils lui répondirent : Tes serviteurs sont venus d'un pays très éloigné, sur la renommée de Yahweh, ton Dieu ; car nous avons entendu sa renommée, et toutes les choses qu'il a faites en Egypte,
\VS{10}et tout ce qu'il a fait aux deux rois des Amoréens, qui étaient au-delà du Jourdain, Sihon, roi de Hesbon, et Og, roi de Basan, qui demeurait à Aschtaroth.
\VS{11}Et nos anciens et tous les habitants de notre pays nous ont dit : Prenez avec vous des provisions pour le chemin, et allez au-devant d'eux, et dites-leur : Nous sommes vos serviteurs, et maintenant traitez alliance avec nous.
\VS{12}Voci notre pain : Nous l'avons pris dans nos maisons tout chaud pour notre provision, le jour où nous sommes partis pour venir vers vous, mais maintenant voici, il est devenu sec et moisi.
\VS{13}Et voici aussi les outres de vin neuves que nous avons remplies, elles se sont déchirées ; nos habits et nos souliers sont usés à cause de la longueur de la marche.
\VS{14}Les hommes d'Israël prirent de leur provision, et aucun d'eux ne consulta la bouche de Yahweh\FTNT{Josué et les chefs ne consultèrent pas Yahweh avant de traiter alliance avec les Gabaonites. Prenez le temps dans la prière afin de connaître le cœur de la personne avec laquelle vous voulez marcher.}.
\VS{15}Car Josué fit la paix avec eux, et traita avec eux une alliance par laquelle il devait leur laisser la vie, et les chefs de l'assemblée le leur jurèrent.
\TextTitle{Les Gabaonites démasqués}
\VS{16}Mais il arriva, trois jours après l'alliance traitée avec eux, qu'ils apprirent que c'étaient leurs voisins et qu'ils habitaient parmi eux.
\VS{17}Car les enfants d'Israël partirent, et arrivèrent à leurs villes le troisième jour. Leurs villes étaient Gabaon, Kephira, Beéroth, et Kirjath-Jearim.
\VS{18}Et les enfants d'Israël ne les frappèrent point, parce que les chefs de l'assemblée leur avaient juré par Yahweh, le Dieu d'Israël. Mais toute l'assemblée murmura contre les chefs.
\VS{19}Alors tous les chefs dirent à toute l'assemblée : Nous leur avons juré par Yahweh, le Dieu d'Israël, c'est pourquoi maintenant nous ne pouvons pas les frapper.
\VS{20}Faisons-leur ceci, et qu'on les laisse vivre afin qu'il n'y ait pas de colère contre nous, à cause du serment que nous leur avons fait.
\VS{21}Ils vivront, leur dirent les chefs. Mais ils furent employés à couper le bois et à puiser l'eau pour toute l'assemblée, comme les chefs le leur avaient dit\FTNT{2 S. 21:1-14. La présence des Gabaonites en plein centre de Canaan tendait à isoler les tribus du nord de celles du sud, favorisant ainsi le schisme des deux royaumes (1 R. 12).}.
\VS{22}Car Josué les fit appeler, et leur parla, en disant : Pourquoi nous avez-vous trompés, en nous disant : Nous sommes très éloignés de vous, alors que vous habitez au milieu de nous ?
\VS{23}Maintenant vous êtes maudits ; il y aura toujours des esclaves parmi vous, des coupeurs de bois et des puiseurs d'eau pour la maison de mon Dieu.
\VS{24}Et ils répondirent à Josué, et dirent : Après qu'il ait été exactement rapporté à tes serviteurs les ordres que Yahweh, ton Dieu, avait ordonnés à Moïse, son serviteur, pour vous donner tout le pays et pour en exterminer tous les habitants devant vous ; nous avons extrêmement crains pour nos personnes à cause de vous et nous avons fait ceci. 
\VS{25}Et maintenant nous voici entre tes mains ; fais-nous comme il te semblera bon et juste de nous faire.
\VS{26}Il leur fit donc ainsi et il les délivra de la main des enfants d'Israël, de sorte qu'il ne les tuèrent point.
\VS{27}Et en ce jour-là, Josué les établit coupeurs de bois et puiseurs d'eau pour l'assemblée, et pour l'autel de Yahweh, jusqu'à aujourd'hui, dans le lieu qu'il choisirait.
\Chap{10}
\TextTitle{Josué secoure Gabaon des cinq rois des Amoréens}
\VerseOne{}Or quand Adoni-Tsédek, roi de Jérusalem, entendit que Josué avait pris Aï, et qu'il l'avait entièrement détruite par le moyen de l'interdit, ayant fait à Aï et à son roi, comme il avait fait à Jéricho et à son roi, et que les habitants de Gabaon avaient fait la paix avec Israël, et étaient au milieu d'eux.
\VS{2}Il eut une grande frayeur, parce que Gabaon était une grande ville, comme une ville royale, et elle était plus grande qu'Aï, et parce que tous ses hommes étaient vaillants.
\VS{3}C'est pourquoi Adoni-Tsédek, roi de Jérusalem, envoya dire à Hoham, roi d'Hébron, et à Piream, roi de Jarmuth, et à Japhia, roi de Lakis, et à Debir, roi d'Eglon :
\VS{4}Montez vers moi, et aidez-moi afin que nous frappions Gabaon, car elle a fait la paix avec Josué et avec les enfants d'Israël.
\VS{5}Ainsi cinq rois des Amoréens, savoir, le roi de Jérusalem, le roi d'Hébron, le roi de Jarmuth, le roi de Lakis, et le roi d'Eglon, s'assemblèrent et montèrent avec toutes leurs armées ; et ils campèrent près de Gabaon, et lui firent la guerre.
\VS{6}Alors les gens de Gabaon dirent à Josué au camp de Guilgal : Ne retire point tes mains de tes serviteurs, monte rapidement vers nous, délivre-nous, et donne-nous du secours ; car tous les rois des Amoréens qui habitent aux montagnes se sont rassemblés contre nous.
\VS{7}Josué donc monta de Guilgal, et avec lui tout le peuple qui était propre à la guerre, et tous les hommes forts et vaillants.
\TextTitle{Yahweh accorde à Israël une grande victoire à Makkéda}
\VS{8}Et Yahweh dit à Josué : Ne les crains point, car je les ai livré entre tes mains, et aucun d'eux ne tiendra devant toi.
\VS{9}Josué arriva subitement sur eux, après avoir marché toute la nuit depuis Guilgal.
\VS{10}Yahweh les mit en déroute devant Israël, qui en fit un grand carnage près de Gabaon, et les poursuivit par le chemin de la montagne de Beth-Horon, les battit jusqu'à Azéka, et jusqu'à Makkéda.
\VS{11}Et comme ils s'enfuyaient devant Israël, et qu'ils étaient à la descente de Beth-Horon, Yahweh fit tomber du ciel sur eux de grosses pierres jusqu'à Azéka, et ils périrent ; ceux qui moururent des pierres de grêle furent plus nombreux que ceux qui furent tués avec l'épée par les enfants d'Israël.
\VS{12}Alors Josué parla à Yahweh, le jour où Yahweh livra les Amoréens aux enfants d'Israël, et dit en présence d'Israël : Soleil, arrête-toi sur Gabaon, et toi lune, sur la vallée d'Ajalon !
\VS{13}Et le soleil s'arrêta, et la lune aussi s'arrêta, jusqu'à ce que le peuple ait tiré vengeance de ses ennemis. Cela n'est-il pas écrit dans le livre du Juste ? Le soleil s'arrêta au milieu du ciel et ne se hâta point de se coucher environ un jour entier\FTNT{Ha. 3:11.}.
\VS{14}Et il n'y a point eu de jour semblable à celui-là, ni avant ni après, où Yahweh exauça la voix d'un homme ; car Yahweh combattait pour Israël.
\VS{15}Et Josué, et tout Israël avec lui, retourna au camp à Guilgal.
\VS{16}Au reste, ces cinq rois restants s'enfuirent, et se cachèrent dans une caverne à Makkéda.
\VS{17}Et on le rapporta à Josué, en disant : On a trouvé les cinq rois cachés dans une caverne à Makkéda.
\VS{18}Et Josué dit : Roulez de grosses pierres à l'entrée de la caverne et mettez près d'elle quelques hommes pour les garder.
\VS{19}Mais vous, ne vous arrêtez pas, poursuivez vos ennemis, attaquez-les par-derrière jusqu'au dernier, ne les laissez pas entrer dans leurs villes, car Yahweh, votre Dieu, les a livrés entre vos mains.
\VS{20}Et quand Josué et les enfants d'Israël eurent achevé d'en faire une très grande boucherie, jusqu'à les détruire entièrement, ceux d'entre eux qui s'étaient échappés se retirèrent dans les villes fortifiées,
\VS{21}tout le peuple revint en paix au camp vers Josué à Makkéda, et personne ne remua sa langue contre les enfants d'Israël.
\VS{22}Alors Josué dit : Ouvrez l'entrée de la caverne, et amenez-moi ces cinq rois hors de la caverne.
\VS{23}Et ils firent ainsi, et ils lui amenèrent hors de la caverne ces cinq rois : Le roi de Jérusalem, le roi d'Hébron, le roi de Jarmuth, le roi de Lakis et le roi d'Eglon.
\VS{24}Et après qu'ils eurent amené à Josué ces cinq rois hors de la caverne, Josué appela tous les hommes d'Israël, et dit aux chefs des gens de guerre qui étaient allés avec lui : Approchez-vous, mettez vos pieds sur les cous de ces rois. Ils s'approchèrent, et mirent leurs pieds sur leurs cous\FTNT{Ps. 110:1.}.
\VS{25}Alors Josué leur dit : Ne craignez point, et ne soyez point effrayés, fortifiez-vous, et ayez du courage, car Yahweh traitera ainsi tous vos ennemis contre lesquels vous combattez.
\VS{26}Et après cela, Josué les frappa et les fit mourir, il les fit pendre à cinq arbres, et ils restèrent pendus à ces arbres jusqu'au soir.
\VS{27}Et comme le soleil se couchait, Josué ordonna qu'on les descende de ces arbres, et on les jeta dans la caverne où ils s'étaient cachés, et on mit à l'entrée de la caverne de grosses pierres qui y sont demeurées jusqu'à ce jour\FTNT{De. 21:23.}.
\VS{28}Josué prit aussi Makkéda le même jour, la frappa du tranchant de l'épée, et dévoua à la façon de l'interdit son roi et ses habitants, et ne laissa échapper personne qui était dans cette ville. Et il fit au roi de Makkéda comme il fait au roi de Jéricho.
\TextTitle{Conquête des territoires du sud}
\VS{29}Après cela, Josué, et tout Israël avec lui, passa de Makkéda à Libna, et fit la guerre à Libna.
\VS{30}Et Yahweh la livra aussi entre les mains d'Israël, avec son roi, et il la frappa du tranchant de l'épée, elle et tous ceux qui s'y trouvaient ; il n'en laissa échapper aucune personne qui était dans cette ville ; et il fit à son roi comme il avait fait au roi de Jéricho.
\VS{31}Ensuite Josué, et tout Israël avec lui, passa de Libna à Lakis, campa devant elle, et lui fit la guerre.
\VS{32}Et Yahweh livra Lakis entre les mains d'Israël, qui la prit le deuxième jour, et la frappa du tranchant de l'épée, et toutes les personnes qui s'y trouvaient, comme il avait fait à Libna.
\VS{33}Alors Horam, roi de Guézer, monta pour secourir Lakis. Josué le frappa, lui et son peuple, de sorte qu'il n'en laissa pas échapper un seul homme.
\VS{34}Après cela Josué, et tout Israël avec lui, passa de Lakis à Eglon ; ils campèrent devant elle, et lui firent la guerre.
\VS{35}Ils la prirent le jour même, la frappèrent du tranchant de l'épée ; et Josué dévoua à la façon de l'interdit ce jour-là toutes les personnes qui y étaient, comme il avait fait à Lakis.
\VS{36}Puis Josué, et tout Israël avec lui, monta d'Eglon à Hébron, et ils lui firent la guerre.
\VS{37}Et ils la prirent, et la frappèrent du tranchant de l'épée, avec son roi, toutes ses villes, et toutes les personnes qui y étaient ; il n'en laissa échapper aucune, comme il avait fait à Eglon ; et il dévoua à la façon de l'interdit, toutes les personnes qui y étaient.
\VS{38}Ensuite Josué, et tout Israël avec lui, retourna vers Debir, et ils lui firent la guerre.
\VS{39}Et il la prit, avec son roi et toutes ses villes ; et ils les frappèrent du tranchant de l'épée, et dévouèrent à la façon de l'interdit toutes les personnes qui y étaient ; il n'en laissa échapper aucune. Il fait à Debir et son roi comme il avait fait à Hébron, et comme il avait fait à Libna et à son roi.
\VS{40}Josué donc frappa tout ce pays, la montagne et le midi, la plaine et les coteaux, et tous leurs rois ; il n'en laissa échapper aucun, et il dévoua par le moyen de l'interdit toutes les personnes qui y respiraient, comme Yahweh, le Dieu d'Israël, l'avait ordonné\FTNT{De. 20:16-17.}.
\VS{41}Ainsi Josué les battit depuis Kadès-Barnéa jusqu'à Gaza, et tout le pays de Gosen jusqu'à Gabaon.
\VS{42}Josué prit tous ces rois en même temps et leur pays, parce que Yahweh, le Dieu d'Israël, combattait pour Israël.
\VS{43}Après quoi Josué, et tout Israël avec lui, retourna au camp à Guilgal.
\Chap{11}
\TextTitle{Conquête des territoires du nord}
\VerseOne{}Et aussitôt que Jabin, roi de Hatsor, eut appris ces choses, il envoya des messagers à Jobab, roi de Madon, au roi de Schimron, et au roi d'Acschaph,
\VS{2}et aux rois qui habitaient vers le nord, aux montagnes et dans la plaine, vers le midi de Kinnéreth, dans la vallée, et sur les hauteurs de Dor vers l'occident,
\VS{3}aux Cananéens qui étaient à l'orient et à l'occident, aux Amoréens, aux Héthiens, aux Phéréziens, aux Jébusiens dans les montagnes, et aux Héviens au pied de la montagne de l'Hermon, dans le pays de Mitspa.
\VS{4}Ils sortirent donc avec toutes leurs armées, un grand peuple par leur grand nombre, comme le sable qui est sur le bord de la mer, il y avait aussi des chevaux et des chars en très grand nombre.
\VS{5}Tous ces rois se réunirent, et campèrent ensemble près des eaux de Mérom, pour combattre contre Israël.
\VS{6}Et Yahweh dit à Josué : Ne les crains point, car demain, à cette même heure, je les livrerai tous, blessés à mort, devant Israël. Tu couperas les jarrets à leurs chevaux, et brûleras au feu leurs chars\FTNT{2 S. 8:4.}.
\VS{7}Josué donc, et tous les gens de guerre avec lui vinrent subitement sur eux près des eaux de Mérom, et ils se précipitèrent au milieu d'eux.
\VS{8}Et Yahweh les livra entre les mains d'Israël ; ils les battirent, et les poursuivirent jusqu'à Sidon la grande, jusqu'aux eaux de Misrephoth-Maïm, et jusqu'à la vallée de Mitspa vers l'orient, et ils les battirent tellement qu'ils ne laissèrent aucun survivant.
\VS{9}Et Josué leur fit comme Yahweh lui avait dit ; il coupa les jarrets de leurs chevaux, et brûla au feu leurs chars.
\VS{10}A son retour, et dans le même temps, Josué prit Hatsor, et frappa son roi avec l'épée ; car Hatsor avait été auparavant la capitale de tous ces royaumes.
\VS{11}On frappa aussi du tranchant de l'épée et l'on dévoua à la façon de l'interdit tous ceux qui s'y trouvaient, il ne resta rien de ce qui respirait, et l'on brûla au feu Hatsor.
\VS{12}Josué prit aussi toutes les villes de ces rois, et tous leurs rois, et les frappa du tranchant de l'épée, et il les dévoua à la façon de l'interdit, comme Moïse, serviteur de Yahweh, l'avait ordonné.
\VS{13}Mais Israël ne brûla aucune des villes situées sur des collines, excepté de Hatsor seule, que Josué brûla.
\VS{14}Et les enfants d'Israël pillèrent pour eux tout le butin de ces villes et le bétail ; mais ils frappèrent du tranchant de l'épée tous les hommes, jusqu'à ce qu'ils les aient exterminés, ils n'y laissèrent aucun qui respirait.
\VS{15}Comme Yahweh l'avait ordonné à Moïse son serviteur, ainsi Moïse l'avait ordonné à Josué ; et Josué le fit ainsi ; de sorte qu'il n'omit rien de tout ce que Yahweh avait ordonné à Moïse. 
\TextTitle{Josué s'empare de tout le pays}
\VS{16}Josué donc prit tout ce pays-là, la montagne et tout le pays du midi, avec tout le pays de Gosen, la vallée et la plaine, la montagne d'Israël et ses vallées.
\VS{17}Depuis la montagne de Halak, qui s'élève vers Séir, jusqu'à Baal-Gad dans la vallée du Liban, au pied de la montagne d'Hermon. Il prit aussi tous leurs rois, les battit et les fit mourir.
\VS{18} Josué fit la guerre plusieurs jours contre tous ces rois.
\VS{19}Il n'y eut aucune ville qui fit la paix avec les enfants d'Israël, excepté les Héviens qui habitaient à Gabaon ; ils les prirent toutes par la guerre.
\VS{20}Car cela venait de Yahweh, qu'ils endurcissent leur cœur pour qu'ils sortent en bataille contre Israël, afin qu'il les dévoue à la façon de l'interdit, sans qu'il y ait pour eux de miséricorde, et qu'il les extermine, comme Yahweh l'avait ordonné à Moïse\FTNT{Ex. 4:21 ; De. 2:30 ; 1 R. 12:15.}.
\VS{21}En ce même temps-là aussi, Josué se mit en marche, et il extermina les Anakim des montagnes d'Hébron, de Debir, d'Anab, et de toute la montagne de Juda, et de toute la montagne d'Israël ; Josué, dis-je, les dévoua à la façon de l'interdit avec leurs villes.
\VS{22}Il ne resta aucun Anakim dans le pays des enfants d'Israël ; il n'en resta seulement qu'à Gaza, à Gath et à Asdod\FTNT{2 S. 21:20.}.
\VS{23}Josué donc prit tout le pays, suivant tout ce que Yahweh avait dit à Moïse. Et Josué le donna en héritage à Israël, selon leurs portions, et leurs tribus. Et le pays fut en repos et sans avoir guerre.
\Chap{12}
\TextTitle{Liste des rois vaincus par Moïse et Josué}
\VerseOne{}Voici les rois du pays que les enfants d'Israël frappèrent, et dont ils possédèrent le pays de l'autre côté du Jourdain, vers l'orient, depuis le torrent de l'Arnon jusqu'à la montagne de l'Hermon, et toute la plaine vers l'orient.
\VS{2}Savoir, Sihon, roi des Amoréens, qui habitait à Hesbon, et qui dominait depuis Aroër, qui est sur le bord du torrent de l'Arnon, et depuis le milieu du torrent, sur la moitié de Galaad, jusqu'au torrent de Jabbok, qui est la frontière des enfants d'Ammon\FTNT{De. 3:8-16.} ;
\VS{3}et depuis la plaine jusqu'à la mer de Kinnéreth vers l'orient, et jusqu'à la mer de la plaine, qui est la mer salée, vers l'orient, au chemin de Beth-Jeschimoth ; et depuis le midi sur le pied du Pisga.
\VS{4}Et les contrées d'Og, roi de Basan, qui était seul reste des Rephaïm, et qui habitait à Aschtaroth et à Edréï.
\VS{5}Et sa domination s'étendait sur la montagne de l'Hermon, sur Salca, et sur tout Basan, jusqu'à la frontière des Gueschuriens et des Maacathiens, et sur la moitié de Galaad, frontière de Sihon, roi de Hesbon.
\VS{6}Moïse, serviteur de Yahweh, et les enfants d'Israël, les battirent ; et Moïse, serviteur de Yahweh, en donna la possession aux Rubénites, aux Gadites, et à la demi-tribu de Manassé\FTNT{No. 32:33.}.
\VS{7}Voici les rois du pays que Josué et les enfants d'Israël frappèrent de ce côté-ci du Jourdain vers l'occident, depuis Baal-Gad, dans la vallée du Liban, jusqu'à la montagne de Halak qui monte vers Séir, et que Josué donna aux tribus d'Israël en possession, selon leurs portions,
\VS{8}pays consistant en montagnes et en vallées, en plaines et en collines, en pays de désert et de midi: Les Héthiens, les Amoréens, les Cananéens, les Phéréziens, les Héviens et les Jébusiens.
\VS{9}Le roi de Jéricho, un ; le roi d'Aï, près de Béthel, un ;
\VS{10}le roi de Jérusalem, un ; le roi d'Hébron, un ;
\VS{11}le roi de Jarmuth, un ; le roi de Lakis, un ;
\VS{12}le roi d'Eglon, un ; le roi de Guézer, un ;
\VS{13}le roi de Debir, un ; le roi de Guéder, un ;
\VS{14}le roi de Horma, un ; le roi d'Arad, un ;
\VS{15}le roi de Libna, un ; le roi d'Adullam, un ;
\VS{16}le roi de Makkéda, un ; le roi de Béthel, un ;
\VS{17}le roi de Tappuach, un ; le roi de Hépher, un ;
\VS{18}le roi d'Aphek, un ; le roi de Lascharon, un ;
\VS{19}le roi de Madon, un ; le roi de Hatsor, un ;
\VS{20}le roi de Schimron-Meron, un ; le roi d'Acschaph, un ;
\VS{21}le roi de Taanac, un ; le roi de Meguiddo, un ;
\VS{22}le roi de Kédesch, un ; le roi de Jokneam, au Carmel, un ;
\VS{23}le roi de Dor, sur les hauteurs de Dor, un ; le roi de Gojim, près de Guilgal, un ;
\VS{24}le roi de Thirtsa, un ; en tout trente et un rois.
\Chap{13}
\TextTitle{Les territoires de Ruben, de Gad et de la demi-tribu de Manassé}
\VerseOne{}Or, quand Josué fut devenu vieux, fort avancé en âge, Yahweh lui dit : Tu es devenu vieux, fort avancé en âge, et il te reste encore un très grand pays à posséder.
\VS{2}Voici le pays qui reste, toutes les contrées des Philistins, et des Gueschuriens,
\VS{3}depuis le Schichor, qui coule devant l'Egypte, jusqu'à la frontière d'Ekron au nord, contrée qui doit être tenue pour Cananéenne, et qui est occupée par les cinq princes des Philistins, celui de Gaza, celui d'Asdod, celui d'Askalon, celui de Gath, celui d'Ekron, et par les Avviens ;
\VS{4}du côté du midi, tout le pays des Cananéens, et Meara qui est aux Sidoniens, jusqu'à Aphek, jusqu'à la frontière des Amoréens ;
\VS{5}le pays qui appartient aux Guibliens, et tout le Liban, vers l'orient, depuis Baal-Gad, au pied de la montagne d'Hermon, jusqu'à l'entrée de Hamath ;
\VS{6}tous les habitants de la montagne, depuis le Liban jusqu'aux eaux de Misrephoth-Maïm, tous les Sidoniens. Je les chasserai moi-même devant les fils d'Israël. Donne seulement ce pays en héritage par le sort à Israël, comme je te l'ai prescrit.
\VS{7}Maintenant donc divise ce pays en héritage aux neuf tribus, et à la demi-tribu de Manassé.
\VS{8}Avec l'autre moitié de laquelle les Rubénites et les Gadites ont pris leur héritage, lequel Moïse leur a donné au delà du Jourdain, vers l'orient, selon que Moïse, serviteur de Yahweh, le leur a donné ;
\VS{9}depuis Aroër, qui est sur le bord du torrent de l'Arnon, et la ville qui est au milieu de la vallée, et toute la plaine de Médeba, jusqu'à Dibon ;
\VS{10}et toutes les villes de Sihon, roi des Amoréens, qui régnait à Hesbon, jusqu'à la frontière des enfants d'Ammon ;
\VS{11}et Galaad, et les territoires des Gueschuriens et des Maacathiens, toute la montagne de l'Hermon, et tout Basan jusqu'à Salca ;
\VS{12}tout le royaume d'Og en Basan, qui régnait à Aschtaroth, et à Edréï, et qui était resté le seul reste des Rephaïm ; Moïse battit ces rois, et les chassa.
\VS{13}Or les fils d'Israël ne chassèrent point les Gueschuriens et les Maacathiens, mais les Gueschuriens et les Maacathiens ont habité au milieu d'Israël jusqu'à ce jour.
\VS{14}Seulement il ne donna point d'héritage à la tribu de Lévi ; les sacrifices consumés par le feu devant Yahweh, le Dieu d'Israël, tel fut son héritage, comme il le lui avait dit\FTNT{No. 18:20-24 ; De. 10:9 ; De. 18:2 ; Ez. 44:28.}.
\VS{15}Moïse donc donna un héritage à la tribu des fils de Ruben selon leurs familles.
\VS{16}Et leurs frontières furent depuis Aroër qui est sur le bord du torrent d'Arnon, et de la ville qui est au milieu du torrent, et toute la plaine qui est près de Médeba.
\VS{17}Hesbon et toutes ses villes, qui étaient dans la plaine, Dibon, Bamoth-Baal, Beth-Baal-Meon,
\VS{18}Jahats, Kedémoth et Méphaath,
\VS{19}Kirjathaïm, Sibma, Tséreth-Haschachar sur la montagne de la vallée,
\VS{20}Beth-Peor, les coteaux du Pisga et Beth-Jeschimoth,
\VS{21}et toutes les villes de la plaine, et tout le royaume de Sihon, roi des Amoréens qui régnait à Hesbon ; Moïse l'avait battu, lui et les princes de Madian, Evi, Rékem, Tsur, Hur, et Réba, princes qui relevaient de Sihon, et qui habitaient dans le pays.
\VS{22}Les enfants d'Israël firent passer aussi par l'épée Balaam\FTNT{Voir No. 22. Balaam était l'exemple type du prophète corrompu, soucieux de tirer profit de son service.}, fils de Beor, le devin, avec les autres qui y furent tués.
\VS{23}Et les frontières des enfants d'Israël fut le Jourdain et sa frontière. Tel fut l'héritage des fils de Ruben, selon leurs familles ; savoir, ces villes-là et leurs villages\FTNT{No. 34:14-15.}.
\VS{24}Moïse donna aussi un héritage à la tribu de Gad, pour les fils de Gad, selon leurs familles.
\VS{25}Et leur pays fut Jaezer, et toutes les villes de Galaad et la moitié du pays des enfants d'Ammon, jusqu'à Aroër, qui est vis-à-vis de Rabba,
\VS{26}et depuis Hesbon jusqu'à Ramath-Mitspé, et Bethonim, et depuis Mahanaïm jusqu'à la frontière de Debir,
\VS{27}et, dans la vallée, Beth-Haram, Beth-Nimra, Succoth et Tsaphon, reste du royaume de Sihon, roi de Hesbon, ayant le Jourdain pour frontière jusqu'à l'extrémité de la mer de Kinnéreth, de l'autre côté du Jourdain, vers l'orient.
\VS{28}Tel fut l'héritage des fils de Gad, selon leurs familles ; savoir, les villes et leurs villages.
\VS{29}Moïse donna aussi à la demi-tribu de Manassé un héritage, qui est resté à la demi-tribu des fils de Manassé, selon leurs familles.
\VS{30}Leur pays fut depuis Mahanaïm, tout Basan, et tout le royaume d'Og, roi de Basan, et tous les villages de Jaïr qui sont en Basan, soixante villes.
\VS{31}Et la moitié de Galaad, Aschtaroth et Edréï, villes du royaume d'Og en Basan, furent aux fils de Makir, fils de Manassé, à la moitié des enfants de Makir, selon leurs familles.
\VS{32}Ce sont là les pays que Moïse avait donnés en héritage, lorsqu'il était dans les plaines de Moab, de l'autre côté du Jourdain, vis-à-vis de Jéricho, à l'orient.
\VS{33}Mais Moïse ne donna point d'héritage à la tribu de Lévi ; car Yahweh, le Dieu d'Israël, fut leur héritage, comme il le lui avait dit.
\Chap{14}
\TextTitle{Caleb reçoit Hébron}
\VerseOne{}Voici les terres que les enfants d'Israël eurent pour héritage dans le pays de Canaan, ce que partagèrent entre eux le prêtre Eléazar, Josué, fils de Nun, et les chefs des familles des tribus des enfants d'Israël.
\VS{2}Selon le sort de leur héritage ; comme Yahweh l'avait ordonné par le moyen de Moïse ; savoir, à neuf tribus et à la demi-tribu\FTNT{No. 26:55.}.
\VS{3}Car Moïse avait donné un héritage aux deux tribus et à la demi-tribu de l'autre côté du Jourdain, mais il n'avait point donné de part aux Lévites parmi eux.
\VS{4}Parce que les fils de Joseph, savoir, Manassé et Ephraïm, formaient deux tribus ; et l'on ne donna point de part aux Lévites dans le pays, excepté des villes pour habitation, et les faubourgs pour leurs troupeaux, et pour le reste de leurs biens.
\VS{5}Les enfants d'Israël firent comme Yahweh l'avait ordonné à Moïse, et ils partagèrent le pays.
\VS{6}Or les fils de Juda s'approchèrent de Josué à Guilgal ; et Caleb, fils de Jephunné, le Kenizien, lui dit : Tu sais la parole que Yahweh a déclarée à Moïse, homme de Dieu, à mon sujet et au tien à Kadès-Barnéa\FTNT{No. 14:24 ; No. 32:12 ; De. 1:36.}.
\VS{7}J'étais âgé de quarante ans quand Moïse, serviteur de Yahweh, m'envoya à Kadès-Barnéa pour espionner le pays, et je lui fis un rapport avec droiture de cœur.
\VS{8}Et mes frères qui étaient montés avec moi découragèrent le cœur du peuple, mais moi je persévérai à suivre Yahweh, mon Dieu.
\VS{9}Et ce jour-là Moïse jura, en disant : La terre que ton pied a foulée sera ton héritage à perpétuité, pour toi et pour tes fils, parce que tu as persévéré à suivre Yahweh, mon Dieu.
\VS{10}Or maintenant voici, Yahweh m'a fait vivre comme il l'a dit. Il y a déjà quarante-cinq ans que Yahweh déclarait cette parole à Moïse, lorsqu'Israël marchait dans le désert. Et maintenant voici, je suis aujourd'hui âgé de quatre-vingt-cinq ans.
\VS{11}Et je suis encore aujourd'hui aussi vigoureux que j'étais le jour où Moïse m'envoya ; et j'ai maintenant la même force que j'avais alors pour le combat, soit pour sortir et pour entrer.
\VS{12}Maintenant, donne-moi donc cette montagne, dont Yahweh a parlé ce jour-là ; car tu as appris en ce jour qu'il s'y trouve des Anakim, et qu'il y a de grandes villes fortifiées. Yahweh sera peut-être avec moi, et je les chasserai, comme Yahweh a dit.
\VS{13}Josué donc bénit Caleb, fils de Jephunné, et lui donna Hébron pour héritage.
\VS{14}C'est ainsi que Caleb, fils de Jephunné, le Kenizien, a eu jusqu'à ce jour Hébron pour héritage, parce qu'il avait persévéré à suivre Yahweh, le Dieu d'Israël.
\VS{15}Or Hébron s'appelait autrefois Kirjath-Arba ; et Arba avait été le plus grand homme parmi les Anakim. Le pays fut en repos et sans guerre.
\Chap{15}
\TextTitle{Le territoire de Juda}
\VerseOne{} Ce sont ici la part échue par le sort à la tribu des enfants de Juda, selon leurs familles ; à la frontière d'Edom, au désert de Tsin, vers le midi, fut la dernière extrémité de leurs pays vers le midi ;
\VS{2}tellement que leur frontière, du côté du midi, fut la dernière extrémité de la mer Salée, depuis le bras qui regarde vers le midi. 
\VS{3}Et elle devait sortir vers le midi de la montée d'Akrabbim, et passer vers Tsin ; et montant du midi de Kadès-Barnéa, passer à Hetsron ; puis montant vers Addar, se tourner vers Karkaa ; 
\VS{4}puis, passant vers Atsmon, sortir au torrent d'Egypte ; tellement que les extrémités de cette frontière devaient se rendre à la mer. Ce sera là, dit Josué, votre frontière, du côté du midi.
\VS{5}Et la frontière vers l'orient était la mer salée jusqu'à l'embouchure du Jourdain. La frontière du côté du nord sera depuis la langue de mer, qui est à l'embouchure du Jourdain.
\VS{6}Et cette frontière montera jusqu'à Beth-Hogla, et passera du côté du nord de Beth-Araba ; et cette frontière montera jusqu'à la pierre de Bohan, fils de Ruben.
\VS{7}Puis cette frontière montera vers Debir, depuis la vallée d'Acor, et même vers le nord, du côté de Guilgal, qui est vis-à-vis de la montée d'Adummim, au sud du torrent. Puis cette frontière passera près des eaux d'En-Schémesch, et ses extrémités se prolongeront à En-Roguel.
\VS{8}Puis cette frontière montera de là par la vallée de Ben-Hinnom, au côté du midi de Jebus, qui est Jérusalem, puis cette frontière montera jusqu'au sommet de la montagne, qui est vis-à-vis de la vallée de Hinnom, à l'occident, et à l'extrémité de la vallée des Rephaïm, au nord.
\VS{9}Et cette frontière s'alignera, depuis le sommet de la montagne jusqu'à la source des eaux de Nephthoach, et continuera vers les villes de la montagne d'Ephron, puis cette frontière s'alignera à Baala, qui est Kirjath-Jearim.
\VS{10}Et cette frontière se tournera depuis Baala, vers l'occident, jusqu'à la montagne de Séir, puis elle traversera le côté nord de la montagne de Jearim, à Kesalon, puis descendait à Beth-Schémesch, et passera par Thimna.
\VS{11}Et cette frontière sortira jusqu'au côté d'Ekron, vers le nord et cette frontière s'alignera vers Schicron, puis ayant passé la montagne de Baala, elle se sortira jusqu'à Jabneel ; tellement que les extrémités de cette frontière se rendront à la mer. 
\VS{12}Or la frontière du côté de l'occident sera ce qui est vers la grande mer et ses limites. Telles furent de tous les côtés les frontières des fils de Juda, selon leurs familles.
\VS{13}Au reste, on donna à Caleb, fils de Jephunné, une part au milieu des fils de Juda, comme Yahweh l'avait ordonné à Josué ;savoir, Kirjath-Arba, or Arba était père d'Anak ; et Kirjath-Arba c'est Hébron.
\VS{14}Et Caleb chassa de là les trois fils d'Anak : Schéschaï, Ahiman, et Talmaï, fils d'Anak.
\VS{15}Et de là il monta contre les habitants de Debir ; Debir s'appelait autrefois Kirjath-Sépher.
\VS{16}Et Caleb dit : Je donnerai ma fille Acsa pour femme à celui qui battra Kirjath-Sépher, et la prendra\FTNT{Jg. 1:12-14.}.
\VS{17}Et Othniel, fils de Kenaz, frère de Caleb, la prit ; et Caleb lui donna sa fille Acsa pour femme.
\VS{18}Et il arriva que comme elle s'en allait, elle l'incita à demander à son père un champ ; puis elle descendit impétueusement de dessus son âne, et Caleb lui dit : Qu'as-tu ? 
\VS{19}Elle répondit : Donne-moi un présent, puisque tu m'as donné une terre du sud, donne-moi aussi des sources d'eau. Et il lui donna les sources supérieures et les sources inférieures.
\VS{20}Tel fut l'héritage de la tribu des fils de Juda, selon leurs familles.
\VS{21}Les villes situées dans la contrée du midi, à l'extrémité de la tribu des fils de Juda, près de la frontière d'Edom, étaient : Kabtseel, Eder, Jagur,
\VS{22}Kina, Dimona, Adada,
\VS{23}Kédesch, Hatsor, Ithnan,
\VS{24}Ziph, Thélem, Bealoth,
\VS{25}Hatsor-Hadattha, Kerijoth-Hetsron qui est Hatsor,
\VS{26}Amam, Schema, Molada,
\VS{27}Hatsar-Gadda, Heschmon, Beth-Paleth,
\VS{28}Hatsar-Schual, Beer-Schéba, Bizjothja,
\VS{29}Baala, Ijjim, Atsem,
\VS{30}Eltholad, Kesil, Horma,
\VS{31}Tsiklag, Madmanna, Sansanna,
\VS{32}Lebaoth, Schilhim, Aïn et Rimmon. Total des villes : Vingt-neuf villes, et leurs villages.
\VS{33}Dans la plaine : Eschthaol, Tsorea, Aschna,
\VS{34}Zanoach, En-Gannim, Tappuach, Enam,
\VS{35}Jarmuth, Adullam, Soco, Azéka,
\VS{36}Schaaraïm, Adithaïm, Guedéra et Guedérothaïm ; quatorze villes, et leurs villages.
\VS{37}Tsenan, Hadascha, Migdal-Gad,
\VS{38}Dilean, Mitspé, Joktheel,
\VS{39}Lakis, Botskath, Eglon,
\VS{40}Cabbon, Lachmas, Kithlisch,
\VS{41}Guedéroth, Beth-Dagon, Naama, et Makkéda ; seize villes, et leurs villages.
\VS{42}Libna, Ether, Aschan,
\VS{43}Jiphtach, Aschna, Netsib,
\VS{44}Keïla, Aczib et Maréscha ; neuf villes, et leurs villages.
\VS{45}Ekron, et les villes de son ressort, et ses villages.
\VS{46}Depuis Ekron et à l'occident, toutes les villes près d'Asdod, et leurs villages.
\VS{47}Asdod, les villes de son ressort, et ses villages, Gaza, les villes de son ressort, et ses villages, jusqu'au torrent d'Egypte, et à la grande mer, qui sert de limite.
\VS{48}Dans la montagne : Schamir, Jatthir, Soco,
\VS{49}Danna, Kirjath-Sanna, qui est Debir,
\VS{50}Anab, Eschthemo, Anim,
\VS{51}Gosen, Holon, et Guilo ; onze villes et leurs villages.
\VS{52}Arab, Duma, Eschean,
\VS{53}Janum, Beth-Tappuach, Aphéka,
\VS{54}Humta, Kirjath-Arba, qui est Hébron, et Tsior ; neuf villes, et leurs villages.
\VS{55}Maon, Carmel, Ziph, Juta,
\VS{56}Jizreel, Jokdeam, Zanoach,
\VS{57}Kaïn, Guibea, et Thimna ; dix villes, et leurs villages.
\VS{58}Halhul, Beth-Tsur, Guedor,
\VS{59}Maarath, Beth-Anoth, et Elthekon ; six villes, et leurs villages.
\VS{60}Kirjath-Baal, qui est Kirjath-Jearim, et Rabba ; deux villes, et leurs villages.
\VS{61}Au désert : Beth-Araba, Middin, Secaca,
\VS{62}Nibschan, Ir-Hammélach, et En-Guédi : Six villes et leurs villages.
\VS{63}Au reste, les fils de Juda ne purent pas chasser les Jébusiens qui habitaient à Jérusalem, c'est pourquoi les Jébusiens ont habité avec les fils de Juda à Jérusalem jusqu'à ce jour.
\Chap{16}
\TextTitle{Le territoire d'Ephraïm}
\VerseOne{}La part échue par le sort aux fils de Joseph depuis le Jourdain près de Jéricho, aux eaux de Jéricho, vers l'orient qui est le désert ; montant de Jéricho par la montagne jusqu'à Béthel.
\VS{2}Et cette frontière devait sortir de Béthel à Luz, puis passer vers la frontière des Arkiens jusqu'à Atharoth.
\VS{3}Et elle devait descendre tirant vers l'occident, vers la frontière des Japhléthiens, jusqu'à celle de Beth-Horon la basse et jusqu'à Guézer, de sorte que ses extrémités aboutissent à la mer.
\VS{4}Ainsi les fils de Joseph, savoir, Manassé et Ephraïm, reçurent leur héritage.
\VS{5}Or la frontière des fils d'Ephraïm, selon leurs familles, la frontière de leur héritage était à l'orient, Atharoth-Addar, jusqu'à Beth-Horon la haute.
\VS{6}Et cette frontière devait sortir vers la mer à Micmethath, du côté du nord ; et cette frontière devait se tourner vers l'orient jusqu'à Thaanath-Silo, et passant du côté d'orient, se rendre à Janoach.
\VS{7}Puis descendre de Janoach à Atharoth et à Naaratha, se rencontrer à Jéricho, et sortir au Jourdain.
\VS{8}Et cette frontière devait aller de Tappuach, vers l'occident, jusqu'au torrent de Kana, tellement que ses extrémités devaient se rendre à la mer. Ce fut là l'héritage de la tribu des fils d'Ephraïm, selon leurs familles.
\VS{9}Les fils d'Ephraïm avaient aussi des villes séparées au milieu de l'héritage des fils de Manassé, toutes ces villes, avec leurs villages.
\VS{10}Or ils ne chassèrent point les Cananéens qui habitaient à Guézer, c'est pourquoi les Cananéens ont habité parmi Ephraïm jusqu'à ce jour, mais ils furent réduits à la servitude et assujettis à un tribut\FTNT{Jg. 1:29 ; 1 R. 9:16.}.
\Chap{17}
\TextTitle{Le territoire de Manassé}
\VerseOne{}Il y eut aussi une part échut par le sort à la tribu de Manassé qui était le premier-né de Joseph. Quant à Makir, premier-né de Manassé, et père de Galaad, il avait eu Galaad et Basan parce qu'il était un homme de guerre.
\VS{2}Puis on jeta donc le sort pour les autres enfants de Manassé, selon ses familles ; aux fils d'Abiézer, aux fils de Hélek, aux fils d'Asriel, aux fils de Sichem, aux fils de Hépher, et aux fils de Schemida. Ce sont là les enfants mâles de Manassé fils de Joseph, selon leurs familles.
\VS{3}Or Tselophchad, fils de Hépher, fils de Galaad, fils de Makir, fils de Manassé, n'eut point de fils, mais il eut des filles dont voici les noms : Machla, Noa, Hogla, Milca et Thirtsa.
\VS{4}Elles vinrent se présenter devant le prêtre Eléazar, devant Josué, fils de Nun, et devant les princes, en disant : Yahweh a ordonné à Moïse de nous donner un héritage parmi nos frères. C'est pourquoi on leur donna un héritage parmi les frères de leur père, selon l'ordre de Yahweh\FTNT{No. 27:7 ; No. 36:2.}.
\VS{5}Et dix portions échurent à Manassé, outre le pays de Galaad et de Basan, qui est de l'autre côté du Jourdain.
\VS{6}Car les filles de Manassé eurent un héritage parmi ses fils, et le pays de Galaad fut pour les autres des fils de Manassé.
\VS{7}Or la frontière de Manassé fut du côté d'Aser, venant à Micmethath, qui est près de Sichem ; puis cette frontière devait aller à main droite vers les habitants d'En-Tappuach.
\VS{8}Or le pays de Tappuach appartenait à Manassé, mais Tappuach qui était près de la frontière de Manassé, appartenait aux fils d'Ephraïm.
\VS{9}De là, cette frontière devait descendre au torrent de Kana, au midi du torrent. Ces villes étaient à Ephraïm parmi les villes de Manassé. La frontière de Manassé était au côté du nord du torrent, et ses extrémités devaient se rendre à la mer.
\VS{10}Ce qui était vers le midi était à Ephraïm, et celui qui était vers le nord était à Manassé, et la mer leur servait de frontière ; et du côté du nord, les frontières se rencontraient à Aser, à Issacar, vers l'orient.
\VS{11}Car Manassé possédait dans Issacar et dans Aser : Beth-Schean et les villes de son ressort, Jibleam et les villes de son ressort, les habitants de Dor et les villes de son ressort, les habitants d'En-Dor, et les villes de son ressort, les habitants de Thaanac et les villes de son ressort, les habitants de Meguiddo et les villes de son ressort, qui sont trois contrées.
\VS{12}Au reste, les fils de Manassé ne purent pas chasser les habitants de ces villes, et les Cananéens voulurent rester dans le même pays.
\VS{13}Mais lorsque les fils d'Israël furent assez forts, ils assujettirent les Cananéens à un tribut, mais ils ne les chassèrent pas entièrement.
\VS{14}Or les fils de Joseph parlèrent à Josué, et dirent : Pourquoi nous as-tu donné en héritage un seul lot, et une seule part, vu que nous sommes un peuple nombreux, et que Yahweh nous a bénis jusqu'à présent ?
\VS{15}Et Josué leur dit : Si vous êtes un peuple nombreux, montez à la forêt, et vous l'abattrez, pour vous y faire de la place dans le pays des Phéréziens et des Rephaïm, si la montagne d'Ephraïm est trop étroite pour vous.
\VS{16}Et les fils de Joseph répondirent : Cette montagne ne sera pas suffisante pour nous, et tous les Cananéens qui habitent la vallée ont des chars de fer, et ceux qui sont à Beth-Schean, et dans les villes de son ressort, et ceux qui habitent dans la vallée de Jizreel\FTNT{Jg. 1:19 ; Jg. 4:3.}.
\VS{17}Donc Josué parla à la maison de Joseph, à Ephraïm et à Manassé, et dit : Vous êtes un peuple nombreux, et vous avez de grandes forces, vous n'aurez pas qu'une seule part.
\VS{18}Mais vous aurez la montagne, car c'est une forêt que vous abattrez et dont les extrémités vous appartiendront, et vous chasserez les Cananéens, quoiqu'ils aient des chars de fer, et qu'ils soient puissants.
\Chap{18}
\TextTitle{La tente d'assignation à Silo}
\VerseOne{}Or toute l'assemblée des enfants d'Israël s'assembla à Silo\FTNT{Silo fut pendant la période des Juges le centre religieux d'Israël car c'est dans cette ville que l'on avait déposé l'arche jusqu'à ce que le roi David l'amène à Jérusalem (Jos. 18:1 ; 2 S. 6 ; 1 Ch. 15:3). Durant le schisme, Silo, située en Samarie, fit office de capitale du royaume de sud. La ville fut finalement détruite par les Philistins aux alentours de 1050 av. J.-C.}, et ils y posèrent la tente d'assignation, après que le pays leur ait été assujetti. 
\VS{2}Mais il restait sept tribus des enfants d'Israël qui n'avaient pas encore reçu leur héritage.
\VS{3}Josué dit aux enfants d'Israël : Jusqu'à quand négligerez-vous de prendre possession du pays que Yahweh, le Dieu de vos pères, vous a donné ?
\VS{4}Prenez trois hommes de chaque tribu, que j'enverrai. Ils se lèveront, traverseront le pays, traceront un plan en vue de l'héritage, puis ils reviendront auprès de moi.
\VS{5}Ils le diviseront en sept parts ; Juda restera dans ses limites au midi, et la maison de Joseph restera dans ses limites au nord.
\VS{6}Vous donc faites-vous un plan du pays en sept parts, et apportez-le-moi ici. Puis je jetterai pour vous le sort devant Yahweh, notre Dieu.
\VS{7}Et il n'y aura point de part pour les Lévites au milieu de vous, parce que le sacerdoce de Yahweh est leur héritage. Quant à Gad et à Ruben, et à la demi-tribu de Manassé, ils ont reçu leur héritage de l'autre côté du Jourdain, vers l'orient, que Moïse, serviteur de Yahweh, leur a donné.
\VS{8}Ces hommes-là donc se levèrent et s'en allèrent pour tracer un plan du pays, Josué leur donna cet ordre en disant : Allez et traversez le pays, et tracez-en un plan, puis revenez auprès de moi, et je jetterai ici le sort pour vous devant Yahweh, à Silo.
\VS{9}Ces hommes-là donc s'en allèrent, parcoururent le pays, et en tracèrent un plan dans un livre en sept parts selon les villes ; puis ils revinrent auprès de Josué dans le camp à Silo.
\VS{10}Et Josué jeta le sort pour eux à Silo devant Yahweh, et Josué fit le partage du pays entre les enfants d'Israël, selon leurs parts.
\TextTitle{Le territoire de Benjamin}
\VS{11}Et le sort tomba sur la tribu des fils de Benjamin selon leurs familles, et la part qui leur échut par le sort avait ses frontières entre les fils de Juda et les fils de Joseph.
\VS{12}Et leur frontière du côté du nord fut depuis le Jourdain ; et cette frontière devait monter à côté de Jéricho vers le nord, puis monter en la montagne tirant vers l'Occident ; de sorte que ses extrémités devaient se rendre au désert de Beth-aven. 
\VS{13}Puis cette frontière devait passer de là vers Luz, à côté de Luz, qui est Béthel tirant vers le midi ; et cette frontière devait descendre à Hatroth-addar, près de la montagne qui est du côté du midi de Beth-horon la basse. 
\VS{14}Et cette frontière devait s'aligner et tourner du côté occidental qui regarde vers le midi, depuis la montagne qui est vis-à-vis de Beth-horon, vers le midi ; tellement que ses extrémités devaient se rendre à Kirjath Baal, qui est Kirjath Jearim, ville des enfants de Juda. C'est là le côté d'occident. 
\VS{15}Mais le côté méridional est l'extrémité de Kirjath Jearim ; et cette frontière devait sortir vers l'Occident, puis elle devait sortir à la fontaine des eaux de Nephtoah. 
\VS{16}Et cette frontière devait descendre à l'extrémité de la montagne qui est vis-à-vis de la vallée de Ben-Hinnom, dans la vallée des Rephaïm, vers le nord, et descendre par la vallée de Hinnom, sur le côté méridional des Jébusiens, puis descendre jusqu'à En-Roguel.
\VS{17}Et elle devait s'aligner vers le nord, et sortir à En-Schémesch, de là à Gueliloth, qui est vis-à-vis de la montée d'Adummim, et descendre à la pierre de Bohan, fils de Ruben,
\VS{18}et passer sur le côté nord en face d'Araba, et descendre à Araba,
\VS{19}puis cette frontière devait passer à côté de Beth-Hogla vers le nord ; de sorte que les extrémités de cette frontière aboutissent à la langue de la mer salée vers le nord, à l'embouchure du Jourdain vers le midi. C'était la frontière du midi.
\VS{20}Et le Jourdain devait borner du côté de l'orient. Ce fut là l'héritage des fils de Benjamin avec ses frontières tout autour, selon leurs familles.
\VS{21}Les villes de la tribu des fils de Benjamin, selon leurs familles, étaient : Jéricho, Beth-Hogla, Emek-Ketsits,
\VS{22}Beth-Araba, Tsemaraïm, Béthel,
\VS{23}Avvim, Para, Ophra,
\VS{24}Kephar-Ammonaï, Ophni et Guéba ; douze villes et leurs villages.
\VS{25}Gabaon, Rama, Beéroth,
\VS{26}Mitspé, Kephira, Motsa,
\VS{27}Rékem, Jirpeel, Thareala,
\VS{28}Tséla, Eleph, Jebus, qui est Jérusalem, Guibeath et Kirjath ; quatorze villes et leurs villages. Tel fut l'héritage des fils de Benjamin selon leurs familles.
\Chap{19}
\TextTitle{Le territoire de Siméon}
\VerseOne{}La deuxième part échut par le sort à Siméon, pour la tribu des fils de Siméon, selon leurs familles. Leur héritage était parmi l'héritage des fils de Juda\FTNT{Ge. 49:5-7.}.
\VS{2}Ils eurent dans leur héritage Beer-Schéba, Schéba, Molada,
\VS{3}Hatsar-Schual, Bala, Atsem,
\VS{4}Eltholad, Bethul, Horma,
\VS{5}Tsiklag, Beth-Marcaboth, Hatsar-Susa,
\VS{6}Beth-Lebaoth et Scharuchen ; treize villes et leurs villages.
\VS{7}Aïn, Rimmon, Ether, et Aschan ; quatre villes et leurs villages ;
\VS{8}et tous les villages qui étaient autour de ces villes-là jusqu'à Baalath-Beer, qui est Ramath du midi. Tel fut l'héritage de la tribu des fils de Siméon, selon leurs familles.
\VS{9}L'héritage des fils de Siméon fut pris sur la portion des fils de Juda ; car la portion des fils de Juda était trop grande pour eux ; c'est pourquoi les fils de Siméon reçurent leur héritage parmi le leur.
\TextTitle{Le territoire de Zabulon}
\VS{10}La troisième part échut par le sort aux fils de Zabulon, selon leurs familles.
\VS{11}Et leur frontière devait monter vers le quartier devers la mer, même jusqu'à Mareala, puis se rencontrer à Dabbéscheth, et de là au torrent qui est vis-à-vis de Jokneam.
\VS{12}Or cette frontière devait retourner vers Sarid à l'orient, vers le soleil levant, jusqu'à la frontière de Kisloth-Thabor, puis continuer à Dabrath, et monter à Japhia.
\VS{13}De là passer à l'orient, par Guittha-Hépher, par Ittha-Katsin, puis continuer à Rimmon, jusqu'à Néa.
\VS{14}Puis cette frontière devait tourner du côté du nord vers Hannathon, et ses extrémités devaient se rendre à la vallée de Jiphthach-El.
\VS{15}Avec Katthath, Nahalal, Schimron, Jideala, et Bethléhem ; il y avait douze villes et leurs villages.
\VS{16}Tel fut l'héritage des fils de Zabulon selon leurs familles, ces villes-là, et leurs villages.
\TextTitle{Le territoire d'Issacar}
\VS{17}La quatrième part échut par le sort à Issacar, aux fils d'Issacar, selon leurs familles.
\VS{18}Et leur frontière devaient passer par Jizreel, Kesulloth, Sunem,
\VS{19}Hapharaïm, Schion, Anacharath,
\VS{20}Rabbith, Kischjon, Abets,
\VS{21}Rémeth, En-Gannim, En-Hadda et Beth-Patsets ;
\VS{22}elle devait se rencontrer à Thabor, et vers Schachatsima et Beth-Schémesch, et les extrémités de leur frontière devaient se rendre au Jourdain. Seize villes et leurs villages.
\VS{23}Tel fut l'héritage de la tribu des fils d'Issacar, selon leurs familles, ces villes-là et leurs villages.
\TextTitle{Le territoire d'Aser}
\VS{24}La cinquième part échut par le sort à la tribu des fils d'Aser, selon leurs familles.
\VS{25}Et leur frontière fut Helkath, Hali, Béthen, Acschaph,
\VS{26}Allammélec, Amead et Mischeal ; et elle devait se rencontrer à Carmel, au quartier vers la mer, et à Schichor-Libnath.
\VS{27}Puis elle devait retourner vers l'orient, à Beth-Dagon, et se rencontrer à Zabulon, et à la vallée de Jiphthach-El, vers le nord de Beth-Emek et de Neïel, puis sortir vers Cabul, à gauche,
\VS{28}et vers Ebron, Rehob, Hammon et Kana, jusqu'à Sidon la grande.
\VS{29}Puis la frontière devait retourner à Rama, jusqu'à la ville forte de Tyr, et cette frontière devait retourner à Hosa ; de sorte que ses extrémités se rencontrent au quartier qui est vers la mer, par la contrée d'Aczib.
\VS{30}Avec Umma, Aphek et Rehob ; vingt-deux villes et leurs villages.
\VS{31}Tel fut l'héritage de la tribu des fils d'Aser, selon leurs familles ; ces villes-là et leurs villages.
\TextTitle{Le territoire de Nephthali}
\VS{32}La sixième part échut par le sort aux fils de Nephthali, selon leurs familles.
\VS{33}Leur frontière fut depuis Héleph, depuis Allon par Tsaanannim, Adami-Nékeb et Jabneel, jusqu'à Lakkum, et ses extrémités devaient se rendre au Jourdain.
\VS{34}Puis cette frontière devait retourner du côté d'occident, vers Aznoth-Thabor, et sortir de là à Hukkok ; de sorte que du côté du midi elle devait se rencontrer à Zabulon, et du côté d'occident elle devait se rencontrer à Aser et à Juda ; le Jourdain était du côté au soleil levant.
\VS{35}Au reste, les villes fortifiées étaient : Tsiddim, Tser, Hammath, Rakkath, Kinnéreth,
\VS{36}Adama, Rama, Hatsor,
\VS{37}Kédesch, Edréï, En-Hatsor,
\VS{38}Jireon, Migdal-El, Horem, Beth-Anath et Beth-Schémesch ; dix-neuf villes et leurs villages.
\VS{39}Tel fut l'héritage de la tribu des fils de Nephthali, selon leurs familles ; ces villes-là, et leurs villages.
\TextTitle{Le territoire de Dan}
\VS{40}La septième part échut par le sort à la tribu des fils de Dan selon leurs familles.
\VS{41}La limite de leur héritage fut, Tsorea, Eschthaol, Ir-Schémesch,
\VS{42}Schaalabbin, Ajalon, Jithla,
\VS{43}Elon, Thimnatha, Ekron,
\VS{44}Eltheké, Guibbethon, Baalath,
\VS{45}Jehud, Bené-Berak, Gath-Rimmon,
\VS{46}Mé-Jarkon et Rakkon, avec le territoire qui est vis-à-vis de Japho.
\VS{47}Le territoire échu aux fils de Dan était trop petit pour eux. C'est pourquoi les fils de Dan montèrent, et combattirent contre Léschem ; ils s'en emparèrent et la frappèrent du tranchant de l'épée ; ils en prirent possession, s'y établirent, et l'appelèrent Léschem, Dan, du nom de Dan leur père.
\VS{48}Tel fut l'héritage de la tribu des fils de Dan selon leurs familles ; ces villes-là et leurs villages.
\TextTitle{Josué reçoit Thimnath-Sérach}
\VS{49}Après qu'on eut achevé de partager le pays selon ses frontières, les enfants d'Israël donnèrent à Josué, fils de Nun, une possession au milieu d'eux.
\VS{50}Selon l'ordre de Yahweh, ils lui donnèrent la ville qu'il demanda, Thimnath-Sérach, dans la montagne d'Ephraïm. Il rebâtit la ville, et y habita.
\VS{51}Ce sont là les héritages que le prêtre Eléazar, Josué, fils de Nun, et les chefs de pères des tribus des enfants d'Israël partagèrent par le sort à Silo, devant Yahweh, à l'entrée de la tente d'assignation, et ils achevèrent ainsi le partage du pays.
\Chap{20}
\TextTitle{Les six villes de refuge\FTNTT{No. 35.}}
\VerseOne{}Puis Yahweh parla à Josué et dit :
\VS{2}Parle aux enfants d'Israël et dis : Etablissez-vous des villes de refuge comme je vous l'ai ordonné par le moyen de Moïse,
\VS{3}où pourra s'enfuir le meurtrier qui aura tué quelqu'un involontairement, sans intention, et elles vous serviront de refuge devant celui qui a le droit de venger le sang.
\VS{4}Et le meurtrier s'enfuira dans l'une de ces villes, s'arrêtera à l'entrée de la porte de la ville, et il exposera son affaire aux anciens de cette ville-là, ils l'écouteront, et le recevront chez eux dans la ville, et lui donneront une demeure, afin qu'il habite avec eux.
\VS{5}Et quand celui qui a le droit de venger le sang le poursuivra, ils ne livreront pas le meurtrier entre ses mains ; puisque c'est involontairement qu'il a tué son prochain, et qu'il ne le haïssait point auparavant.
\VS{6}Mais il demeurera dans cette ville-là, jusqu'à ce qu'il comparaisse devant l'assemblée pour être jugé, jusqu'à la mort du grand prêtre qui sera en fonction en ce temps-là. Alors le meurtrier s'en retournera, et reviendra dans sa ville et dans sa maison, dans la ville d'où il s'était enfui\FTNT{Ex. 21:13 ; No. 35:9-34 ; De. 19.}.
\VS{7}Ils consacrèrent donc Kédesch, en Galilée, dans la montagne de Nephthali ; Sichem dans la montagne d'Ephraïm ; et Kirjath-Arba, qui est Hébron, dans la montagne de Juda.
\VS{8}Et de l'autre côté du Jourdain, à l'orient de Jéricho, ils choisirent Betser, dans la tribu de Ruben, dans le désert, dans la plaine ; Ramoth en Galaad, dans la tribu de Gad ; et Golan en Basan, dans la tribu de Manassé\FTNT{De. 4:43.}.
\VS{9}Telles furent les villes désignées pour tous les enfants d'Israël et pour l'étranger en séjour au milieu d'eux, afin que quiconque aurait tué quelqu'un involontairement puisse s'y réfugier, et qu'il ne meure pas de la main de celui qui a le droit de venger le sang, avant d'avoir comparu devant l'assemblée.
\Chap{21}
\TextTitle{Les quarante-huit villes des Lévites}
\VerseOne{}Or les chefs des pères de famille des Lévites s'approchèrent d'Eléazar, le prêtre, de Josué, fils de Nun, et des chefs des pères de famille des tribus des enfants d'Israël.
\VS{2}Et leur parlèrent à Silo, dans le pays de Canaan, et dirent : Yahweh a ordonné par Moïse qu'on nous donne des villes pour habiter, et leurs faubourgs pour nos bêtes\FTNT{No. 35:2-3.}.
\VS{3}Alors les enfants d'Israël donnèrent aux Lévites, sur leur héritage, les villes suivantes et leurs faubourgs, d'après l'ordre de Yahweh.
\VS{4}Et on tira au sort pour les familles des Kehathites ; et les Lévites, fils d'Aaron, le prêtre eurent par le sort treize villes de la tribu de Juda, de la tribu de Siméon, et de la tribu de Benjamin.
\VS{5}Les autres fils de Kehath eurent par le sort dix villes des familles de la tribu d'Ephraïm, de la tribu de Dan, et de la demi-tribu de Manassé.
\VS{6}Et les fils de Guerschon eurent par le sort treize villes, des familles de la tribu d'Issacar, de la tribu d'Aser, de la tribu de Nephthali, et de la demi-tribu de Manassé en Basan.
\VS{7}Et les fils de Merari selon leurs familles, eurent douze villes, de la tribu de Ruben, de la tribu de Gad, et de la tribu de Zabulon.
\VS{8}Les enfants d'Israël donnèrent donc par le sort aux Lévites ces villes-là avec leurs faubourgs, comme Yahweh l'avait ordonné par Moïse.
\VS{9}Ils donnèrent donc de la tribu des fils de Juda et de la tribu des fils de Siméon, ces villes, qui vont être nommées par leurs noms,
\VS{10}et qui furent pour les fils d'Aaron, qui étaient des familles des Kehathites, et des fils de Lévi, car le sort les avait indiqués les premiers.
\VS{11}Ils leur donnèrent Kirjath-Arba, qui est Hébron, dans la montagne de Juda, avec ses faubourgs tout autour : Arba était le père d'Anak.
\VS{12}Mais quant au territoire de la ville, et à ses villages, on les donna à Caleb, fils de Jephunné, pour sa possession.
\VS{13}Ils donnèrent donc aux fils d'Aaron, le prêtre, les villes de refuge pour les meurtriers, Hébron, avec ses faubourgs, et Libna avec ses faubourgs.
\VS{14}Jatthir, avec ses faubourgs, Eschthemoa, avec ses faubourgs,
\VS{15}Holon, avec ses faubourgs, Debir, avec ses faubourgs,
\VS{16}Aïn, avec ses faubourgs, Jutta, avec ses faubourgs ; et Beth-Schémesch, avec ses faubourgs ; neuf villes de ces deux tribus-là ;
\VS{17}et de la tribu de Benjamin, Gabaon, avec ses faubourgs, et Guéba, avec ses faubourgs,
\VS{18}Anathoth, avec ses faubourgs, et Almon, avec ses faubourgs ; quatre villes.
\VS{19}Toutes les villes des prêtres, fils d'Aaron, furent treize villes, avec leurs faubourgs.
\VS{20}Quant aux Lévites, appartenant aux familles des autres fils de Kehath, ils eurent par le sort des villes de la tribu d'Ephraïm.
\VS{21}On leur donna donc les villes de refuge pour les meurtriers, Sichem, avec ses faubourgs, dans la montagne d'Ephraïm, et Guézer avec ses faubourgs ;
\VS{22}Kibtsaïm, avec ses faubourgs, et Beth-Horon, avec ses faubourgs ; quatre villes ;
\VS{23}et de la tribu de Dan, Eltheké, avec ses faubourgs ; Guibbethon, avec ses faubourgs,
\VS{24}Ajalon, avec ses faubourgs, Gath-Rimmon, avec ses faubourgs ; quatre villes.
\VS{25}Et de la demi-tribu de Manassé, Thaanac, avec ses faubourgs ; et Gath-Rimmon, avec ses faubourgs, deux villes.
\VS{26}Total des villes : Dix villes avec leurs faubourgs, pour les familles des autres fils de Kehath.
\VS{27}On donna aussi aux fils de Guerschon, d'entre les familles des Lévites : De la demi-tribu de Manassé les villes de refuge pour les meurtriers, Golan en Basan, avec ses faubourgs, et Beeschthra, avec ses faubourgs ; deux villes ;
\VS{28}et de la tribu d'Issacar, Kischjon, avec ses faubourgs, Dabrath, avec ses faubourgs,
\VS{29}Jarmuth, avec ses faubourgs, En-Gannim, avec ses faubourgs ; quatre villes ;
\VS{30}et de la tribu d'Aser, Mischeal, avec ses faubourgs, Abdon, avec ses faubourgs,
\VS{31}Helkath, avec ses faubourgs, et Rehob, avec ses faubourgs ; quatre villes ;
\VS{32}et de la tribu de Nephthali, les villes de refuge pour les meurtriers, Kédesch en Galilée avec ses faubourgs, Hammoth-Dor, avec ses faubourgs, et Karthan, avec ses faubourgs ; trois villes.
\VS{33}Total des villes des Guerschonites, selon leurs familles : Treize villes, et leurs faubourgs.
\VS{34}On donna aussi au reste des Lévites, qui appartenaient aux familles des fils de Merari : De la tribu de Zabulon, Jokneam, avec ses faubourgs, Kartha, avec ses faubourgs,
\VS{35}Dimna, avec ses faubourgs, et Nahalal, avec ses faubourgs ; quatre villes ;
\VS{36}et de la tribu de Ruben, Betser, avec ses faubourgs, et Jahtsa, avec ses faubourgs ;
\VS{37}Kedémoth, avec ses faubourgs, et Méphaath, avec ses faubourgs ; quatre villes ;
\VS{38}et de la tribu de Gad, les villes de refuge pour les meurtriers, Ramoth en Galaad, avec ses faubourgs, et Mahanaïm, avec ses faubourgs,
\VS{39}Hesbon, avec ses faubourgs, et Jaezer, avec ses faubourgs ; en tout quatre villes.
\VS{40}Total des villes qui échurent par le sort aux fils de Merari, selon leurs familles, formant le reste des familles des Lévites : Douze villes.
\VS{41}Total des villes des Lévites qui étaient parmi la possession des enfants d'Israël : Quarante-huit villes, et leurs faubourgs.
\VS{42}Chacune de ces villes avait ses faubourgs autour d'elle ; il en était ainsi de toutes ces villes-là.
\TextTitle{Yahweh accomplit sa promesse}
\VS{43}Yahweh donna donc à Israël tout le pays qu'il avait juré de donner à leurs pères ; ils le possédèrent, et y habitèrent\FTNT{Dieu accomplit toujours ses promesses (Jé. 1:12).}.
\VS{44}Yahweh leur accorda un parfait repos tout autour, selon tout ce qu'il avait juré à leurs pères ; aucun de leurs ennemis ne put leur résister, car Yahweh les livra entre leurs mains.
\VS{45}Il ne tomba pas un seul mot de toutes les bonnes paroles que Yahweh avait dites à la maison d'Israël : Toutes s'accomplirent.
\Chap{22}
\TextTitle{Ruben, Gad et la demi-tribu de Manassé retournent sur leurs terres}
\VerseOne{}Alors Josué appela les Rubénites, les Gadites et la demi-tribu de Manassé.
\VS{2}Et il leur dit : Vous avez gardé tout ce que Moïse, serviteur de Yahweh, vous a prescrit, et vous avez obéi à ma voix dans tout ce que je vous ai ordonné.
\VS{3}Vous n'avez pas abandonné vos frères, depuis une très longue période jusqu'à ce jour ; et vous avez gardé les ordres, les commandements de Yahweh votre Dieu.
\VS{4}Maintenant que Yahweh, votre Dieu, a donné du repos à vos frères, comme il le leur avait dit, retournez et allez dans vos tentes, dans le pays qui vous appartient, et que Moïse, serviteur de Yahweh, vous a donné de l'autre côté du Jourdain\FTNT{No. 32:33 ; De. 3:13 ; De. 29:8.}.
\VS{5}Prenez seulement bien garde d'observer les ordonnances et les lois que Moïse, serviteur de Yahweh, vous a prescrites : Aimez Yahweh votre Dieu, marchez dans toutes ses voies, gardez ses commandements, attachez-vous à lui, et servez-le de tout votre cœur et de toute votre âme\FTNT{De. 10:12.}.
\VS{6}Puis Josué les bénit et les renvoya ; et ils s'en allèrent vers leurs tentes.
\VS{7}Moïse avait donné à la moitié de la tribu de Manassé son héritage en Basan ; et Josué donna à l'autre moitié son héritage avec leurs frères de l'autre côté du Jourdain vers l'occident. Josué les renvoya dans leurs tentes, et les bénit.
\VS{8}Et il leur parla et dit : Vous retournez à vos tentes avec de grandes richesses, une très nombreuse quantité de bétail, avec une quantité considérable d'argent, d'or, d'airain, de fer, et de vêtements. Partagez avec vos frères le butin de vos ennemis.
\VS{9}Ainsi donc les fils de Ruben, les fils de Gad, et la demi-tribu de Manassé s'en retournèrent, et partirent de Silo, dans le pays de Canaan, après avoir quitté les enfants d'Israël, pour s'en aller dans le pays de Galaad, sur la terre de leur possession et où ils s'établirent, suivant ce que Yahweh avait ordonné par Moïse.
\TextTitle{L'autel Ed, sujet d'incompréhension}
\VS{10}Quand ils furent arrivés aux frontières du Jourdain, qui appartiennent au pays de Canaan, les fils de Ruben, les fils de Gad, et la demi-tribu de Manassé y bâtirent un autel, près du Jourdain, un autel dont la grandeur frappait les regards.
\VS{11}Les enfants d'Israël apprirent que l'on disait : Voici, les fils de Ruben, les fils de Gad, et la demi-tribu de Manassé ont bâti un autel en face du pays de Canaan, sur les frontières du Jourdain, du côté des enfants d'Israël.
\VS{12}Lorsque les enfants d'Israël entendirent cela, toute l'assemblée des enfants d'Israël se réunit à Silo, pour monter en guerre contre eux.
\VS{13}Cependant les enfants d'Israël envoyèrent vers les fils de Ruben, vers les fils de Gad, et vers la demi-tribu de Manassé, au pays de Galaad, Phinées, fils du prêtre Eléazar,
\VS{14}et avec lui dix princes, un prince par maison paternelle pour chacune des tribus d'Israël ; tous étaient chefs de maison paternelle parmi les milliers d'Israël.
\VS{15}Ils se rendirent auprès des fils de Ruben, des fils de Gad et de la demi-tribu de Manassé au pays de Galaad, et leur parlèrent, en disant :
\VS{16}Ainsi parle toute l'assemblée de Yahweh : Quelle est cette infidélité que vous avez commise contre le Dieu d'Israël, et pourquoi vous détournez-vous aujourd'hui de Yahweh, en vous bâtissant un autel, pour vous rebeller aujourd'hui contre Yahweh ?
\VS{17}Regardons-nous comme peu de chose l'iniquité de Peor\FTNT{Peor : No. 25:1-9.}, dont nous ne nous sommes pas encore bien purifiés jusqu'à présent, malgré la plaie qu'il attira sur l'assemblée de Yahweh ?
\VS{18}Et vous vous détournez aujourd'hui de Yahweh ! Si vous vous rebellez aujourd'hui contre Yahweh, demain il s'irritera contre toute l'assemblée d'Israël.
\VS{19}Si vous tenez pour impure la terre qui est votre propriété, passez sur la terre qui est la possession de Yahweh, où est fixé le tabernacle de Yahweh, ayez votre possession parmi nous, mais ne vous révoltez point contre Yahweh, et ne soyez point rebelles contre nous, en vous bâtissant un autel, outre l'autel de Yahweh notre Dieu.
\VS{20}Acan\FTNT{Acan : Jos. 7:1-26.}, fils de Zérach, ne commit-il pas une infidélité en prenant des choses dévouées par le moyen de l'interdit, et la colère de Yahweh ne s'enflamma-t-elle pas contre toute l'assemblée d'Israël ? Cependant, cet homme ne fut pas le seul qui périt à cause de son iniquité.
\VS{21}Mais les fils de Ruben, les fils de Gad, et la demi-tribu de Manassé répondirent, et dirent aux chefs des milliers d'Israël :
\VS{22}Dieu\FTNT{Dieu : de l'hébreu « El » : puissant, etc.}, Dieu\FTNT{Dieu : de l'hébreu « elohim » : juge, ange.} Yahweh, Dieu\FTNT{Dieu : de l'hébreu « El » : puissant, etc.}, Dieu\FTNT{Dieu : de l'hébreu « elohim » : juge, ange.}Yahweh, le sait, et Israël lui-même le saura ! Si c'est par rébellion et par infidélité envers Yahweh, alors qu'il ne nous vienne point en aide aujourd'hui.
\VS{23}Si nous nous sommes bâti un autel pour nous détourner de Yahweh, si c'est pour y offrir des holocaustes, ou des offrandes, ou si c'est pour y faire des sacrifices d'offrande de paix, que Yahweh lui-même nous en demande compte !
\VS{24}C'est bien plutôt par une sorte d'inquiétude que nous avons fait cela, en pensant que vos fils pourraient un jour parler à nos fils et leur dire : Qu'y a-t-il de commun entre vous et Yahweh, le Dieu d'Israël ?
\VS{25}Puisque Yahweh a mis le Jourdain pour frontière entre nous et vous, fils de Ruben, et fils de Gad ; vous n'avez point de part à Yahweh ! Et ainsi vos fils feraient qu'un jour nos fils cesseraient de craindre Yahweh\FTNT{Né. 2:20 ; Ac. 8:21.}.
\VS{26}C'est pourquoi nous avons dit : Mettons-nous maintenant à bâtir un autel, non pour des holocaustes ni pour des sacrifices ;
\VS{27}mais afin qu'il serve de témoignage entre nous et vous, et entre nos descendants et les vôtres, que nous voulons servir Yahweh devant sa face par nos holocaustes et nos sacrifices d'expiation et d'offrande de paix, afin que vos fils ne disent pas un jour à nos fils : Vous n'avez point de part à Yahweh\FTNT{Ge. 31:48.} !
\VS{28}C'est pourquoi nous avons dit : Lorsqu'ils nous tiendront ce discours, ou à nos descendants, nous leur dirons : Voyez la forme de l'autel de Yahweh qu'ont fait nos pères, non pour des holocaustes, ni pour des sacrifices, mais afin qu'il soit témoin entre nous et vous.
\VS{29}A Dieu ne plaise que nous nous révoltions contre Yahweh et que nous nous détournions aujourd'hui de Yahweh, en bâtissant un autel pour des holocaustes, pour des offrandes, et pour des sacrifices, outre l'autel de Yahweh notre Dieu, qui est devant son tabernacle !
\VS{30}Or, après que le prêtre Phinées, et les princes de l'assemblée, les chefs des milliers d'Israël qui étaient avec lui, eurent entendu les paroles que les fils de Ruben, les fils de Gad, et les fils de Manassé leur dirent, ils furent satisfaits.
\VS{31}Et Phinées, fils du prêtre Eléazar, dit aux fils de Ruben, aux fils de Gad, et aux fils de Manassé : Nous reconnaissons aujourd'hui que Yahweh est au milieu de nous, puisque vous n'avez point commis cette infidélité contre Yahweh ; vous avez ainsi délivré les enfants d'Israël de la main de Yahweh.
\VS{32}Ainsi Phinées, fils du prêtre Eléazar, et les princes, quittèrent les fils de Ruben, les fils de Gad, et revinrent du pays de Galaad dans le pays de Canaan, auprès des enfants d'Israël, auxquels ils firent un rapport.
\VS{33}Et la chose plut aux enfants d'Israël ; ils bénirent Dieu, et ne parlèrent plus de monter en armes contre eux pour détruire le pays où habitaient les fils de Ruben, et les fils de Gad.
\VS{34}Les fils de Ruben, et les fils de Gad appelèrent l'autel Ed ; car, dirent-ils, il est témoin entre nous que Yahweh est Dieu.
\Chap{23}
\TextTitle{Avertissements de Josué}
\VerseOne{}Or il arriva, plusieurs jours après, que Yahweh ayant donné du repos à Israël de tous les ennemis qui l'entouraient, Josué était vieux, fort avancé en âge.
\VS{2}Et Josué convoqua tout Israël, ses anciens, ses chefs, ses juges, ses officiers, et leur dit : Je suis devenu vieux, fort avancé en âge.
\VS{3}Vous avez vu tout ce que Yahweh, votre Dieu, a fait à toutes ces nations devant vous ; car Yahweh, votre Dieu, est celui qui combat pour vous.
\VS{4}Voyez, je vous ai donné en héritage par le sort, selon vos tribus, ces nations qui sont restées, depuis le Jourdain, et toutes les nations que j'ai exterminées, jusqu'à la grande mer vers le soleil couchant.
\VS{5}Yahweh, votre Dieu, les repoussera devant vous et les chassera ; et vous posséderez leur pays en héritage, comme Yahweh, votre Dieu, vous l'a dit\FTNT{Ex. 14:14 ; Ex. 23:27 ; No. 33:53 ; De. 6:18-19.}.
\VS{6}Appliquez-vous avec force à observer et à mettre en pratique tout ce qui est écrit dans le livre de la loi de Moïse, sans vous en détourner ni à droite ni à gauche\FTNT{De. 5:32 ; De. 28:14.}.
\VS{7}Ne vous mêlez point avec ces nations qui sont restées parmi vous ; et ne faites point mention du nom de leurs dieux, et ne faites jurer personne par eux, ne les servez point, et ne vous prosternez point devant eux\FTNT{Ex. 23:13 ; De. 12:3 ; Jé. 5:7 ; De. 6:14.}.
\VS{8}Mais attachez-vous à Yahweh, votre Dieu, comme vous l'avez fait jusqu'à ce jour\FTNT{De. 11:22.}.
\VS{9}C'est pour cela que Yahweh a chassé devant vous des nations grandes et puissantes ; nul n'a pu vous résister jusqu'à ce jour.
\VS{10}Un seul homme d'entre vous en poursuivait mille ; car Yahweh votre Dieu est celui qui combat pour vous, comme il vous l'a dit\FTNT{Lé. 26:8 ; De. 32:30.}.
\VS{11}Veillez donc attentivement sur vos âmes, afin d'aimer Yahweh, votre Dieu.
\VS{12}Autrement, si vous vous détournez et que vous vous attachez au reste de ces nations qui sont demeurées parmi vous, si vous faites alliance par des mariages avec elles, et si vous formez ensemble des relations,
\VS{13}sachez certainenement que Yahweh, votre Dieu, ne continuera pas à chasser ces nations devant vous ; mais elles seront pour vous un piège et un filet, un fouet dans vos côtés et des épines dans vos yeux, jusqu'à ce que vous ayez péri de dessus cette bonne terre que Yahweh, votre Dieu, vous a donnée\FTNT{Ex. 23:33 ; De. 7:16 ; Jg. 2:3.}.
\VS{14}Voici, je m'en vais aujourd'hui par le chemin de toute la terre. Reconnaissez de tout votre cœur et de toute votre âme qu'aucune de toutes les bonnes paroles prononcées sur vous par Yahweh, votre Dieu, n'est restée sans effet ; toutes se sont accomplies pour vous, aucune n'est restée sans effet\FTNT{Jos. 21:45 ; 2 R. 10:10.}.
\VS{15}Et il arrivera que comme toutes les bonnes paroles que Yahweh, votre Dieu, vous a dites vous sont arrivées ; ainsi Yahweh fera venir sur vous toutes les paroles mauvaises, jusqu'à ce qu'il vous ait exterminés de dessus cette bonne terre que Yahweh, votre Dieu, vous a donnée.
\VS{16}Si vous transgressez l'alliance que Yahweh, votre Dieu, vous a prescrite, et si vous allez servir d'autres dieux et vous prosterner devant eux, la colère de Yahweh s'enflammera contre vous, et vous périrez promptement de dessus cette bonne terre qu'il vous a donnée.
\Chap{24}
\TextTitle{Josué rappelle à Israël son histoire}
\VerseOne{}Josué assembla toutes les tribus d'Israël à Sichem, et il convoqua les anciens d'Israël, ses chefs, ses juges, et ses officiers, qui se présentèrent devant Dieu.
\VS{2}Et Josué dit à tout le peuple : Ainsi parle Yahweh, le Dieu d'Israël : Vos pères, Térach père d'Abraham, et père de Nachor, ont anciennement habité de l'autre côté du fleuve, où ils servaient d'autres dieux.
\VS{3}Mais j'ai pris votre père Abraham de l'autre côté du fleuve, je lui fis parcourir tout le pays de Canaan, je multipliai sa postérité, et lui donnai Isaac\FTNT{Ge. 12 ; Ge. 21:2.}.
\VS{4}Je donnai à Isaac, Jacob et Esaü ; et je donnai à Esaü le mont de Séir, pour le posséder ; mais Jacob et ses fils descendirent en Egypte\FTNT{Ge. 25:24 ; Ge. 36:6.}.
\VS{5}Puis j'envoyai Moïse et Aaron, et je frappai l'Egypte, par les prodiges que j'opérai au milieu d'elle ; puis je vous en fis sortir\FTNT{Ex. 3:10.}.
\VS{6}Je fis donc sortir vos pères hors de l'Egypte, et vous arrivâtes à la mer. Les Egyptiens poursuivirent vos pères avec des chars et des cavaliers, jusqu'à la Mer Rouge\FTNT{Ex. 14:9.}.
\VS{7}Alors ils crièrent à Yahweh. Et il mit des ténèbres entre vous et les Egyptiens, et ramena sur eux la mer, qui les couvrit. Vos yeux ont vu ce que j'ai fait aux Egyptiens. Puis vous restâtes longtemps dans le désert.
\VS{8}Ensuite je vous conduisis dans le pays des Amoréens, qui habitaient de l'autre côté du Jourdain, et ils combattirent contre vous. Mais je les livrai entre vos mains ; vous prîtes possession de leur pays, et je les détruisis devant vous.
\VS{9}Balak\FTNT{Balaak : Voir No. 22:2-14.} aussi, fils de Tsippor, roi de Moab, se leva, et fit la guerre à Israël. Il fit appeler Balaam\FTNT{Balaam : Voir No. 22.}, fils de Beor, pour qu'il vous maudisse.
\VS{10}Mais je ne voulus point écouter Balaam ; il s'agenouilla et vous bénit, et je vous délivrai de la main de Balak.
\VS{11}Et vous passâtes le Jourdain, et arrivâtes près de Jéricho. Les habitants de Jéricho, les Amoréens, les Phéréziens, les Cananéens, les Héthiens, les Guirgasiens, les Héviens et les Jébusiens vous firent la guerre. Je les livrai entre vos mains,
\VS{12}et j'envoyai devant vous des frelons qui les chassèrent loin de votre face, comme les deux rois des Amoréens : Ce ne fut ni par ton épée, ni par ton arc\FTNT{Ex. 23:28 ; De. 7:20.}.
\VS{13}Je vous donnai une terre que vous n'aviez point cultivée, des villes que vous n'aviez point bâties, et que vous habitez, et vous mangez les fruits des vignes et des oliviers que vous n'avez point plantés\FTNT{De. 6:10 ; Ps. 105:44 ; Né. 9:25.}.
\TextTitle{Le peuple choisit de servir Yahweh}
\VS{14}Maintenant, craignez Yahweh, et servez-le avec intégrité et avec fidélité. Ôtez les dieux que vos pères ont servis de l'autre côté du fleuve et en Egypte, et servez Yahweh\FTNT{1 S. 12:23-24 ; Ez. 20:7-44.}.
\VS{15}Et s'il vous déplaît de servir Yahweh, choisissez aujourd'hui qui vous voulez servir, ou les dieux que servaient vos pères au-delà du fleuve, ou les dieux des Amoréens dans le pays desquels vous habitez. Mais moi et ma maison, nous servirons Yahweh.
\VS{16}Alors le peuple répondit, et dit : Que Dieu nous garde d'abandonner Yahweh pour servir d'autres dieux !
\VS{17}Car Yahweh, notre Dieu, est celui qui nous a fait monter, nous et nos pères, hors du pays d'Egypte, de la maison de servitude, qui a fait devant nos yeux ces grands signes, qui nous a gardés dans tout le chemin par lequel nous avons marché, et entre tous les peuples parmi lesquels nous avons passé.
\VS{18}Yahweh a chassé devant nous tous les peuples, et même les Amoréens qui habitaient ce pays. Nous servirons aussi Yahweh, car il est notre Dieu.
\VS{19}Josué dit au peuple : Vous ne pourrez pas servir Yahweh, car c'est un Dieu Saint, qui est jaloux, il ne pardonnera point votre rébellion et vos péchés.
\VS{20}Lorsque vous abandonnerez Yahweh et que vous servirez les dieux des étrangers, il reviendra vous faire du mal, et il vous consumera après vous avoir fait du bien.
\VS{21}Le peuple dit à Josué : Non ! Car nous servirons Yahweh.
\VS{22}Et Josué dit au peuple : Vous êtes témoins contre vous-mêmes que c'est vous qui avez choisi Yahweh pour le servir. Et ils répondirent : Nous en sommes témoins.
\VS{23}Maintenant donc ôtez les dieux étrangers qui sont au milieu de vous, et tournez votre cœur vers Yahweh, le Dieu d'Israël.
\VS{24}Et le peuple répondit à Josué : Nous servirons Yahweh notre Dieu et nous obéirons à sa voix.
\VS{25}Ce jour-là, Josué traita alliance avec le peuple, et lui donna des lois et des ordonnances à Sichem.
\VS{26}Josué écrivit ces paroles dans le livre de la loi de Dieu. Il prit aussi une grande pierre\FTNT{Cette Pierre entend selon Josué, elle est également appelée « témoin ». Jésus-Christ, la Pierre angulaire (Es. 8:13-16) est le témoin fidèle (Ap. 19:11). Cette Pierre suivait les Hébreux dans le désert (1 Co. 10:1-3).}, qu'il dressa là sous le chêne qui était dans le lieu consacré à Yahweh.
\VS{27}Josué dit à tout le peuple : Voici, cette pierre servira de témoin contre nous, car elle a entendu toutes les paroles que Yahweh nous a déclarées ; elle servira de témoin contre vous, afin que vous ne reniiez pas votre Dieu.
\VS{28}Puis Josué renvoya le peuple, chacun dans son héritage.
\TextTitle{Mort de Josué et d'Eléazar ; ensevelissement des os de Joseph (Ge. 50 :26)}
\VS{29}Or il arriva, après ces choses, que Josué, fils de Nun, serviteur de Yahweh, mourut, âgé de cent dix ans.
\VS{30}Et on l'ensevelit dans le territoire de son héritage, à Thimnath-Sérach, dans la montagne d'Ephraïm, du côté du nord de la montagne de Gaasch.
\VS{31}Et Israël servit Yahweh tout le temps de Josué, et tout le temps des anciens qui survécurent à Josué, qui avaient connu toutes les œuvres que Yahweh avait faites pour Israël.
\VS{32}Les os de Joseph\FTNT{(Ge. 50:25 ; Ex. 13:19 ; Hé. 11:22).}, que les enfants d'Israël avaient rapportés d'Egypte, furent ensevelis à Sichem, dans la portion du champ que Jacob avait achetée des fils de Hamor, père de Sichem, pour cent kesita, et qui appartint à l'héritage des fils de Joseph.
\VS{33}Et Eléazar, fils d'Aaron, mourut, on l'enterra à Guibeath-Phinées, qui avait été donnée à son fils Phinées, dans la montagne d'Ephraïm.
\PPE{}
\end{multicols}

\clearpage\ShortTitle{Juges}\BookTitle{Juges}\BFont
\noindent\hrulefill
{\footnotesize
\textit{
\bigskip
{\centering{}
\\Auteur : Inconnu
\\(Heb. : Shoftim)
\\Signification : Être juge, prononcer, punir
\\Thème : Défaites et délivrances
\\Date de rédaction : Environ 1100 av. J.-C.\\}
}
%\bigskip
\textit{
\\A la mort de Josué et des anciens, il s’éleva en Israël une nouvelle génération qui n'avait pas connu l’expérience du désert. Elle fit ce qui est mal aux yeux de Dieu, l’abandonna et tomba dans l’idolâtrie. Ainsi, la colère de Yahweh s’abattit sur Israël et il livra le peuple entre les mains de ses ennemis. Dans ces temps de troubles, Dieu suscita des juges - douze hommes et une femme - pour délivrer Israël de ses oppresseurs. Aussi longtemps que le juge était en vie, Israël était en paix. Mais dès qu’il venait à mourir, le peuple se corrompait de nouveau et ses oppressions recommençaient.\bigskip
}
}
\par\nobreak\noindent\hrulefill
\begin{multicols}{2}
\Chap{1}
\TextTitle{Poursuite de la conquête de Canaan}
\VerseOne{}Or il arriva qu'après la mort de Josué, les enfants d'Israël consultèrent Yahweh, en disant : Qui de nous montera le premier contre les Cananéens pour leur faire la guerre ?
\VS{2}Et Yahweh répondit : Juda montera ; voici, j'ai livré le pays entre ses mains.
\VS{3}Juda dit à Siméon son frère : Monte avec moi dans mon lot et nous ferons la guerre aux Cananéens ; et j'irai aussi avec toi dans ton lot. Ainsi Siméon alla avec lui.
\TextTitle{Victoires de Juda ; Caleb prend possession d’Hébron}
\VS{4}Juda monta, et Yahweh livra les Cananéens et les Phéréziens entre leurs mains ; ils battirent dix mille hommes à Bézek.
\VS{5}Et ils trouvèrent Adoni-Bézek à Bézek ; ils l'attaquèrent et frappèrent les Cananéens et les Phéréziens.
\VS{6}Adoni-Bézek s'enfuit mais ils le poursuivirent ; et l'ayant pris, ils lui coupèrent les pouces des mains et des pieds.
\VS{7}Alors Adoni-Bézek dit : Soixante-dix rois, dont les pouces des mains et des pieds avaient été coupés, ramassaient du pain sous ma table ; Dieu me rend ce que j’ai fait. On l’amena à Jérusalem et il y mourut\FTNT{Es. 33:1}.
\VS{8}Les fils de Juda firent la guerre contre Jérusalem et la prirent, ils frappèrent ses habitants du tranchant de l'épée et mirent le feu à la ville.
\VS{9}Puis les fils de Juda descendirent pour faire la guerre aux Cananéens, qui habitaient la montagne, la contrée du midi et la plaine.
\VS{10}Juda marcha contre les Cananéens qui habitaient à Hébron ; or le nom d'Hébron était auparavant Kirjath-Arba ; et il battit Schéschaï, Ahiman et Talmaï\FTNT{Jos. 15:14.}.
\VS{11}De là, il marcha contre les habitants de Debir ; Debir s’appelait auparavant Kirjath-Sépher\FTNT{Jos. 15:15.}.
\VS{12}Caleb dit : Je donnerai ma fille Acsa pour femme à celui qui frappera Kirjath-Sépher et qui la prendra\FTNT{Jos. 15:16.}.
\VS{13}Othniel, fils de Kenaz, frère cadet de Caleb, s’en empara ; et Caleb lui donna sa fille Acsa pour femme.
\VS{14}Et il arriva que comme elle s'en allait, elle l'incita à demander à son père un champ. Puis elle descendit  impétueusement de dessus son âne ; et Caleb lui dit : Qu'as-tu ?\FTNT{Jos. 15:18.}
\VS{15}Elle lui répondit : Donne-moi un présent, puisque tu m'as donné une terre du midi ; donne-moi aussi des sources d'eau. Et Caleb lui donna les sources supérieures et les sources inférieures.
\VS{16}Les fils du Kénien, beau-père de Moïse, montèrent de la ville des palmiers avec les fils de Juda, dans le désert de Juda, qui est au midi d'Arad, et ils allèrent et demeurèrent avec le peuple\FTNT{Jg. 4:11}.
\VS{17}Puis Juda se mit en marche avec Siméon son frère et ils frappèrent les Cananéens qui habitaient à Tsephath ; et ils détruisirent la ville par le moyen de l'interdit, c'est pourquoi on appela la ville du nom de Horma.
\VS{18}Juda prit aussi Gaza avec ses territoires ; Askalon avec ses territoires ; et Ekron avec ses territoires.
\TextTitle{Des victoires en demi-teintes}
\VS{19}Yahweh fut avec Juda et il se rendit maître de la montagne, mais il ne pût chasser les habitants de la vallée, parce qu'ils avaient des chars de fer.
\VS{20}On donna Hébron à Caleb, comme Moïse l'avait dit ; et il en chassa les trois fils d'Anak\FTNT{No. 14:24}.
\VS{21}Quant aux fils de Benjamin, ils ne chassèrent pas les Jébusiens qui habitaient à Jérusalem ; c'est pourquoi les Jébusiens ont habité avec les fils de Benjamin à Jérusalem jusqu'à ce jour.
\VS{22}Ceux de la maison de Joseph montèrent aussi contre Béthel, et Yahweh fut avec eux.
\VS{23}Ceux de la maison de Joseph firent explorer Béthel, dont le nom était auparavant Luz.
\VS{24}Les espions virent un homme qui sortait de la ville, et ils dirent : Nous te prions de nous montrer un endroit par où l’on puisse entrer dans la ville, et nous te ferons grâce.
\VS{25}Il leur montra par où ils pourraient entrer dans la ville. Et ils frappèrent la ville du tranchant de l'épée ; mais ils laissèrent aller cet homme et toute sa famille.
\VS{26}Puis cet homme se rendit dans le pays des Héthiens ; il bâtit une ville et lui donna le nom de Luz, nom qu’elle a porté jusqu'à ce jour.
\VS{27}Manassé ne chassa pas les habitants de Beth-Schean et des villes de son ressort, de Thaanac et des villes de son ressort, de Dor et des villes de son ressort, les habitants de Jibleam et des villes de son ressort, les habitants de Meguiddo et des villes de son ressort ; et les Cananéens persistèrent à habiter dans ce pays-là.
\VS{28}Il est vrai qu’il arriva que quand Israël fut devenu plus fort, il assujettit les Cananéens à un tribut mais il ne les chassa pas entièrement.
\VS{29}Ephraïm ne chassa pas les Cananéens qui habitaient à Guézer, et les Cananéens habitèrent avec lui à Guézer.
\VS{30}Zabulon ne chassa pas les habitants de Kitron, ni les habitants de Nahalol ; et les Cananéens habitèrent avec lui et lui furent assujettis à un tribut.
\VS{31}Aser ne chassa pas les habitants d’Acco, ni les habitants de Sidon, ni ceux d’Achlal, ni d'Aczib, ni d'Helba, ni d'Aphik, ni de Rehob ;
\VS{32}Mais ceux d'Aser habitèrent parmi les Cananéens, habitants du pays ; car ils ne les chassèrent pas.
\VS{33}Nephthali ne chassa pas les habitants de Beth-Schémesch, ni les habitants de Beth-Anath, mais il habita parmi les Cananéens habitants du pays ; et les habitants de Beth-Schémesch, et de Beth-Anath lui furent assujettis au tribut.
\VS{34}Les Amoréens repoussèrent les enfants de Dan dans la montagne et ne les laissèrent pas descendre dans la vallée.
\VS{35}Les Amoréens voulurent encore habiter à Har-Hérès, à Ajalon et à Schaalbim ; mais la main de la maison de Joseph étant devenue plus forte, ils furent assujettis au tribut.
\VS{36}Le territoire des Amoréens s'étendait depuis la montée d’Akrabbim, depuis Séla et en dessus.
\Chap{2}
\TextTitle{Le peuple repris pour sa désobéissance}
\VerseOne{}Or l'Ange de Yahweh monta de Guilgal à Bokim, et dit : Je vous ai fait monter hors d'Egypte, et je vous ai fait entrer dans le pays que j’avais juré à vos pères, et j’ai dit : Je n’enfreindrai jamais mon alliance que j’ai traitée avec vous\FTNT{Ge. 17:7.} ;
\VS{2}Et vous aussi vous ne traiterez pas alliance avec les habitants de ce pays, vous démolirez leurs autels. Mais vous n'avez pas obéi à ma voix. Pourquoi avez-vous fait cela\FTNT{Ex. 23:32 ; De. 7:2 ; De. 12:3.} ?
\VS{3}J’ai dit alors : Je ne les chasserai pas devant vous, mais ils seront à vos côtés, et leurs dieux vous seront un piège\FTNT{Ex. 23:33 ; Jos. 23:13.}.
\VS{4}Et il arriva que, comme l'Ange de Yahweh disait ces paroles à tous les enfants d'Israël, le peuple éleva la voix et pleura.
\VS{5}C'est pourquoi ils appelèrent ce lieu Bokim et ils y offrirent des sacrifices à Yahweh.
\VS{6}Josué renvoya le peuple, et les enfants d'Israël allèrent chacun dans son héritage pour prendre possession du pays\FTNT{Jos. 24:28–32.}.
\VS{7}Le peuple servit Yahweh tout le temps de Josué, et tout le temps des anciens qui survécurent à Josué et qui avaient vu toutes les grandes œuvres que Yahweh avait faites en faveur d’Israël\FTNT{Jos. 24:31.}.
\VS{8}Puis Josué, fils de Nun, serviteur de Yahweh, mourut, âgé de cent dix ans\FTNT{Jos. 24:29.}.
\VS{9}On l’ensevelit dans le territoire qu’il avait eu en partage à Thimnath-Hérès, dans la montagne d'Ephraïm, au nord de la montagne de Gaasch\FTNT{Jos. 24:30.}.
\TextTitle{La nouvelle génération abandonne Yahweh}
\VS{10}Toute cette génération fut recueillie auprès de ses pères, puis il s’éleva après elle une autre génération, qui ne connaissait pas Yahweh ni les œuvres qu'il avait faites en faveur d’Israël.
\VS{11}Les enfants d'Israël firent alors ce qui est mal aux yeux de Yahweh et ils servirent les Baals\FTNT{Baal est un dieu phénicien qui, sous les Ramessides, était assimilé dans la mythologie égyptienne à Seth et à Montou. Baal est un dieu d’origine sémite. Il est le dieu de la pluie. Son nom – «~le maître~» ou «~l’époux~»- se retrouve partout dans le Moyen-Orient, depuis les zones peuplées par les sémites jusqu’aux colonies phéniciennes, dont Carthage. Il était invariablement accompagné d’une divinité féminine (Astarté, Ishtar, Tanit...). Voir Jg. 3:7 ; Jg. 8:33 ; Jg. 10:6.}.
\VS{12}Et ils abandonnèrent Yahweh, le Dieu de leurs pères, qui les avait fait sortir du pays d’Égypte, ils allèrent après d'autres dieux, d'entre les dieux des peuples qui les entouraient ; et ils se prosternèrent devant eux, irritant ainsi Yahweh.
\VS{13}Ils abandonnèrent donc Yahweh, et servirent Baal et les Astartés\FTNT{Astarté ou Ashtart en punico-phénicien, ou Ishtar, dérivé de la déesse de Babylone, était  généralement assimilée à la déesse Mésopotamienne Innana. Déesse phénicienne présentant un caractère belliqueux, elle était souvent représentée à califourchon sur son cheval, accompagnant et protégeant le souverain. Elément féminin du couple suprême qu’elle formait avec Baal, celle-ci assumait des fonctions variées : protectrice du souverain et de sa dynastie ou encore des marins.  Comme pour la plupart des divinités féminines primordiales de l’antiquité (et de la proto-histoire), son culte était  lié à la fertilité et à la fécondité. Parfois vénérée sous le nom de Tanit, elle sera assimilée à Vénus par les Romains sous le nom officiel de Venere Ericina.}.
\VS{14}La colère de Yahweh s'enflamma contre Israël. Il les livra entre les mains de pillards\FTNT{Lorsqu’un enfant de Dieu ouvre la porte au péché, il s’expose aux pillards, c’est-à-dire à Satan et ses démons (Jn. 10:10).} qui les pillèrent, il les vendit entre les mains de leurs ennemis d'alentour, de sorte qu'ils ne purent plus résister face à leurs ennemis\FTNT{Ps. 44:12-13 ; Es. 50:1.}.
\VS{15}Partout où ils allaient, la main de Yahweh était contre eux pour leur faire du mal, comme Yahweh l’avait dit et leur avait juré. Ils furent dans une grande détresse\FTNT{Lé. 26:25 ; De. 28:25.}.
\TextTitle{Yahweh suscite des libérateurs : Les juges}
\VS{16}Yahweh leur suscita des juges\FTNT{Les Juges étaient principalement des libérateurs de l’oppression des ennemis d’Israël.} et ils les délivrèrent de la main de ceux qui les pillaient.
\VS{17}Mais ils ne voulurent pas écouter leurs juges, ils se prostituèrent auprès d'autres dieux, se prosternèrent devant eux. Ils se détournèrent promptement du chemin qu’avaient suivi leurs pères et ils n’obéirent pas comme eux aux commandements de Yahweh.
\VS{18}Quand Yahweh leur suscitait des juges, Yahweh était avec le juge, et il les délivrait de la main de leurs ennemis pendant tout le temps de la vie du juge ; car Yahweh se repentait à cause de leurs gémissements contre ceux qui les opprimaient et les tourmentaient.
\VS{19}Puis il arrivait que quand le juge mourrait, ils se corrompaient de nouveau plus que leurs pères en allant après d'autres dieux pour les servir et se prosterner devant eux, et ils persévéraient dans la même conduite et dans la même voie obstinée\FTNT{Jg. 3:12.}.
\TextTitle{IYahweh éprouve Israël et ne chasse pas ses ennemis}
\VS{20}C'est pourquoi la colère de Yahweh s'enflamma contre Israël, et il dit : Puisque cette nation a transgressé mon alliance que j'avais prescrite à leurs pères et puisqu’ils n'ont pas obéi à ma voix,
\VS{21}aussi je ne chasserai plus devant eux aucune des nations que Josué laissa quand il mourut\FTNT{Jos. 23:13.},
\VS{22}afin d'éprouver par elles Israël, pour savoir s'ils prendront garde ou non de suivre la voie de Yahweh, comme leurs pères y ont pris garde.
\VS{23}Yahweh laissa en repos ces nations qu'il n'avait pas livrées entre les mains de Josué et il ne se hâta pas de les chasser\FTNT{Jg. 3:1-3.}.
\Chap{3}
\VerseOne{}Voici les nations que Yahweh laissa pour éprouver par elles Israël, tous ceux qui n'avaient pas connu toutes les guerres de Canaan\FTNT{Jg. 2:21-23.} ; 
\VS{2}afin qu’au moins les générations des enfants d'Israël connaissent et apprennent la guerre, ceux qui ne l’avaient pas connue auparavant.
\VS{3}Ces nations étaient : Les cinq princes des Philistins, tous les Cananéens, les Sidoniens et les Héviens qui habitaient la montagne du Liban depuis la montagne de Baal-Hermon, jusqu'à l'entrée de Hamath\FTNT{No. 13:22.}.
\VS{4}Ces nations, dis-je, servirent à éprouver Israël pour voir s'ils obéiraient aux commandements que Yahweh avait donnés à leurs pères par le moyen de Moïse.
\TextTitle{Israël se mélange aux nations païennes}
\VS{5}Ainsi les enfants d'Israël habitèrent parmi les Cananéens, les Héthiens, les Amoréens, les Phéréziens, les Héviens et les Jébusiens.
\VS{6}Ils prirent leurs filles pour femmes, ils donnèrent leurs filles à leurs fils et servirent leurs dieux.
\VS{7}Les enfants d'Israël firent ce qui est mal aux yeux de Yahweh, ils oublièrent Yahweh et servirent les Baals et les Astartés\FTNT{Jg. 2:11.}.
\TextTitle{Othniel, premier juge suscité par Yahweh}
\VS{8}C'est pourquoi la colère de Yahweh s'enflamma contre Israël, et il les vendit entre la main de Cuschan-Rischeathaïm, roi de Mésopotamie. Et les enfants d'Israël furent asservis à Cuschan-Rischeathaïm durant huit ans.
\VS{9}Puis les enfants d'Israël crièrent à Yahweh, et Yahweh leur suscita un libérateur qui les délivra, Othniel, fils de Kenaz, frère cadet de Caleb.
\VS{10}L’Esprit de Yahweh fut sur lui. Il devint juge en Israël, et il sortit pour la guerre. Yahweh livra entre ses mains Cuschan-Rischeathaïm, roi de Mésopotamie ; et sa main fut puissante contre Cuschan-Rischeathaïm.
\VS{11}Le pays fut en repos pendant quarante ans. Puis Othniel, fils de Kenaz, mourut.
\TextTitle{Ehud, juge en Israël}
\VS{12}Les enfants d'Israël firent encore ce qui est mal aux yeux de Yahweh ; et Yahweh fortifia Eglon, roi de Moab, contre Israël, parce qu'ils avaient fait ce qui est mauvais aux yeux de Yahweh.
\VS{13}Eglon réunit auprès de lui les fils d'Ammon et les Amalécites et il se mit en marche. Il battit Israël et ils s'emparèrent de la ville des palmiers\FTNT{Palmiers: un autre nom de Jéricho.}.
\VS{14}Et les enfants d'Israël furent asservis à Eglon, roi de Moab, durant dix-huit ans.
\VS{15}Puis les enfants d'Israël crièrent à Yahweh, et Yahweh leur suscita un libérateur, Ehud, fils de Guéra, Benjamite, qui ne se servait pas de sa main droite. Les enfants d'Israël envoyèrent par lui un présent à Eglon, roi de Moab.
\VS{16}Ehud se fit une épée à deux tranchants, de la longueur d'une coudée\FTNT{Une coudée correspond environ à 45 cm.} et il la ceignit sous ses vêtements, sur sa cuisse droite.
\VS{17}Il offrit le présent à Eglon, roi de Moab ; et Eglon était un homme fort gras.
\VS{18}Or il arriva que lorsqu’il eut achevé d’offrir le présent, il renvoya le peuple qui avait apporté le présent.
\VS{19}Mais Ehud revint depuis les idoles de pierre, qui étaient près de Guilgal et il dit : Ô roi ! J’ai quelque chose de secret à te dire. Et il lui répondit : Tais-toi ! Et tous ceux qui étaient auprès de lui sortirent de là.
\VS{20}Ehud s'approcha de lui, comme il était assis seul dans sa chambre d'été, et il dit : J'ai un mot à te dire de la part de Dieu, alors le roi se leva du trône.
\VS{21}Et Ehud avança sa main gauche, tira l'épée de son côté droit et la lui enfonça dans le ventre.
\VS{22}Et la poignée entra après la lame, et la graisse serra tellement la lame, qu’il ne pouvait retirer l’épée du ventre, et il en sortit de l’excrément.
\VS{23}Après cela, Ehud sortit par le portique, ferma après lui les portes de la chambre et tira le verrou.
\VS{24}Quand il fut sorti, les serviteurs d'Eglon vinrent et regardèrent ; et voici, les portes de la chambre étaient fermées au verrou. Ils dirent : Sans doute il se couvre les pieds dans sa chambre d’été.
\VS{25}Et ils attendirent tant qu'ils en furent déconcertés ; et voyant qu'il n'ouvrait pas les portes de la chambre, ils prirent la clef et ouvrirent ; et voici, leur maître était mort, étendu à terre.
\VS{26}Mais Ehud s'échappa pendant qu’ils hésitaient ; et il dépassa les carrières de pierre et se sauva à Seïra.
\VS{27}Dès qu’il fut arrivé, il sonna du shofar dans la montagne d'Ephraïm. Les enfants d'Israël descendirent avec lui de la montagne et il marchait à leur tête.
\VS{28}Il leur dit : Suivez-moi, car Yahweh a livré entre vos mains les Moabites, vos ennemis. Ainsi ils descendirent après lui, s’emparèrent des passages du Jourdain vis-à-vis de Moab et ne laissèrent passer personne.
\VS{29}Ils battirent dans ce temps-là environ dix mille hommes de Moab, tous robustes, tous vaillants et il n'en échappa aucun.
\VS{30}En ce jour, Moab fut humilié sous la main d'Israël. Et le pays fut en repos pendant quatre-vingts ans.
\TextTitle{Schamgar, juge en Israël}
\VS{31}Après lui, il y eut Schamgar, fils d'Anath. Il battit six cents Philistins avec un aiguillon à bœufs et délivra Israël.
\Chap{4}
\TextTitle{Débora et Barak, juges en Israël}
\VerseOne{}Mais les enfants d'Israël firent encore ce qui est mal aux yeux de Yahweh après qu'Ehud fut mort.
\VS{2}C'est pourquoi Yahweh les vendit entre la main de Jabin, roi de Canaan, qui régnait à Hatsor. Le chef de son armée était Sisera, qui habitait à Haroscheth-Goïm\FTNT{Jg. 3:8-16 ; Jos. 11:11-13 ; 1 S. 12:9.}.
\VS{3}Les enfants d'Israël crièrent à Yahweh car Jabin avait neuf cents chars de fer, et il avait violemment opprimé les enfants d'Israël durant vingt ans\FTNT{Jg. 1:19.}.
\VS{4}Dans ce temps-là, Débora, prophétesse, femme de Lappidoth, était juge en Israël.
\VS{5}Débora se tenait sous un palmier, entre Rama et Béthel, dans la montagne d'Ephraïm ; et les enfants d'Israël montaient vers elle pour être jugés.
\VS{6}Elle envoya appeler Barak, fils d'Abinoam, de Kédesch-Nephthali et elle lui dit : Yahweh, le Dieu d'Israël, n'a-t-il pas donné cet ordre ? En disant : Va, et dirige-toi sur la montagne de Thabor et prends avec toi dix mille hommes des enfants de Nephthali, et des enfants de Zabulon\FTNT{Hé. 11:32.} ;
\VS{7}J’attirerai vers toi, au torrent de Kison, Sisera, chef de l'armée de Jabin, avec ses chars et ses troupes et je le livrerai entre tes mains\FTNT{Ps. 83:9-10.}.
\VS{8}Barak lui dit : Si tu viens avec moi, j'irai ; mais si tu ne viens pas avec moi, je n’irai pas.
\VS{9}Elle répondit : J'irai, j'irai avec toi, mais tu n'auras pas d'honneur sur le chemin où tu marches ; car Yahweh livrera Sisera entre les mains d'une femme. Débora se leva et elle alla avec Barak à Kédesch.
\VS{10}Barak convoqua Zabulon et Nephthali à Kédesch ; dix mille hommes marchèrent à sa suite ; et Débora monta avec lui.
\VS{11}Héber, le Kénien, s’était séparé des fils de Hobab, beau-père de Moïse et il avait dressé ses tentes jusqu'au chêne de Tsaannaïm, près de Kédesch\FTNT{No. 10:29.}.
\TextTitle{Yahweh accorde la victoire à Israël}
\VS{12}On rapporta à Sisera que Barak, fils d'Abinoam, s’était dirigé sur la montagne de Thabor.
\VS{13}Et Sisera rassembla tous ses chars, neuf cents chars de fer, et tout le peuple qui était avec lui, depuis Haroscheth-Goïm, jusqu'au torrent de Kison.
\VS{14}Alors Débora dit à Barak : Lève-toi, car voici le jour où Yahweh livre Sisera entre tes mains. Yahweh ne marche-t-il pas devant toi ? Barak descendit de la montagne de Thabor, ayant dix mille hommes à sa suite.
\VS{15}Yahweh mit en déroute devant Barak, Sisera, tous ses chars et toute l'armée, par le tranchant de l'épée. Sisera descendit du char et s'enfuit à pied\FTNT{Ps. 83:9-10.}.
\VS{16}Barak poursuivit les chars et l'armée jusqu'à Haroscheth-Goïm ; et toute l'armée de Sisera fut passée au fil de l'épée ; il n'en resta pas un seul.
\VS{17}Sisera se sauva à pied dans la tente de Jaël, femme de Héber, le Kénien ; car il y avait paix entre Jabin, roi de Hatsor et la maison de Héber, le Kénien.
\VS{18}Jaël étant sortie au-devant de Sisera, lui dit : Entre, mon seigneur, entre chez moi, ne crains pas. Il entra donc chez elle dans la tente et elle le cacha sous une couverture.
\VS{19}Puis il lui dit : Je te prie, donne-moi un peu d'eau à boire, car j'ai soif. Et elle ouvrit une outre de lait, lui donna à boire et le couvrit\FTNT{Jg. 5:25.}.
\VS{20}Il lui dit encore : Tiens-toi à l'entrée de la tente et si l’on vient t’interroger, en disant : Y a-t-il ici quelqu'un ? Alors tu répondras : Non.
\VS{21}Jaël, femme de Héber, saisit un pieu de la tente, prit en sa main un marteau, s’approcha de lui doucement, et lui enfonça dans la tempe le pieu, qui pénétra en terre, pendant qu'il dormait profondément, car il était accablé de fatigue. Et ainsi il mourut.
\VS{22}Et voici, Barak poursuivait Sisera, Jaël sortit au-devant de lui et lui dit : Viens, et je te montrerai l'homme que tu cherches. Barak entra chez elle, et voici, Sisera était étendu mort, et le pieu était dans sa tempe.
\VS{23}En ce jour-là, Dieu humilia Jabin, roi de Canaan, devant les enfants d'Israël.
\VS{24}Et la main des enfants d'Israël s’appesantit et se renforça de plus en plus sur Jabin, roi de Canaan, jusqu'à ce qu'ils aient exterminé Jabin, roi de Canaan.
\Chap{5}
\TextTitle{Cantique à la gloire de Yahweh, le Dieu qui délivre}
\VerseOne{}En ce jour-là, Débora chanta ce cantique avec Barak, fils d'Abinoam, en disant :
\VS{2}Bénissez Yahweh de ce qu’il a fait de telles vengeances en Israël et de ce que le peuple s’est offert volontairement.
\VS{3}Vous, rois, écoutez ! Vous, princes, prêtez l'oreille ! Moi, je chanterai à Yahweh, je chanterai un hymne à Yahweh, le Dieu d'Israël.
\VS{4}Ô Yahweh ! Quand tu sortis de Séir, quand tu t’avanças des champs d'Edom, la terre trembla, les cieux se fondirent, les nuées fondirent en eaux ;
\VS{5}Les montagnes s'ébranlèrent devant Yahweh, ce Sinaï devant Yahweh, le Dieu d'Israël\FTNT{Ps. 68:8-9}.
\VS{6}Aux jours de Schamgar, fils d’Anath, aux jours de Jaël, les grandes routes étaient délaissées, et ceux qui voyageaient prenaient des chemins détournés.
\VS{7}Les villes non murées n’étaient plus habitées en Israël, elles n’étaient point habitées, jusqu’à ce que je me suis levée, moi Débora, jusqu’à ce que je me suis levée pour être mère en Israël.
\VS{8}Israël choisissait-il des dieux nouveaux aussitôt la guerre était aux portes. On ne voyait ni bouclier ni lance chez quarante milliers en Israël.
\VS{9}J’ai mon cœur vers les chefs d'Israël, qui se sont portés volontairement d’entre le peuple. Bénissez Yahweh !
\VS{10}Vous qui montez sur les ânesses blanches, vous qui avez pour sièges des tapis et vous qui marchez sur le chemin, méditez !
\VS{11}Le bruit des archers ayant cessé dans les abreuvoirs, qu’on s’y entretienne des justices de Yahweh et des justices de ses villes non murées en Israël ; alors le peuple de Dieu descendra aux portes.
\VS{12}Réveille-toi, réveille-toi, Débora !  Réveille-toi, réveille-toi, dit le cantique, lève-toi Barak et emmène en captivité ceux que tu as faits captifs, toi fils d'Abinoam\FTNT{Jg. 4:6.}.
\VS{13}Yahweh a fait dominer un reste du peuple sur les puissants ; Yahweh m'a fait dominer sur les héros.
\VS{14}Leur racine est depuis Ephraïm jusqu’à Amalek. A ta suite marcha Benjamin parmi ta troupe. De Makir descendirent les chefs, et de Zabulon ceux qui manient la plume du scribe.
\VS{15}Et les chefs d’Issacar ont été avec Débora, et Issacar ainsi que Barak ; il a été envoyé avec sa suite dans la vallée ; il y a eu aux ruisseaux de Ruben, de grandes considérations dans leur cœur.
\VS{16}Pourquoi es-tu resté entre les barres des étables, à écouter le bêlement des troupeaux ? Aux ruisseaux de Ruben, grandes furent les résolutions du cœur !
\VS{17}Galaad est resté au-delà du Jourdain ; et pourquoi Dan est-il resté sur ses navires ? Aser s'est tenu sur le rivage de la mer, et s’est reposé dans ses ports.
\VS{18}Mais pour Zabulon, c'est un peuple qui a exposé son âme à la mort ; et Nephthali de même, sur les hauteurs des champs.
\VS{19}Les rois vinrent, ils combattirent. Alors combattirent les rois de Canaan, à Thaanac, près des eaux de Meguiddo ; mais ils ne remportèrent nul butin, nul argent.
\VS{20}On a combattu des cieux, les étoiles, dis-je, ont combattu du lieu de leur cours contre Sisera\FTNT{Jg. 4:7.}.
\VS{21}Le torrent de Kison les a emportés, le torrent des anciens temps, le torrent de Kison. Mon âme tu as foulé aux pieds les héros.
\VS{22}Alors les talons des chevaux battirent le sol à cause de la course rapide, de la course rapide de ses puissants chevaux.
\VS{23}Maudissez Méroz, dit l'Ange de Yahweh ; maudissez, maudissez ses habitants, car ils ne sont pas venus au secours de Yahweh, au secours de Yahweh, avec les héros.
\VS{24}Bénie soit par-dessus toutes les femmes Jaël, femme de Héber, le Kénien ! Qu'elle soit bénie entre les femmes qui habitent sous les tentes !
\VS{25}Il demanda de l'eau, elle lui a donné du lait ; elle lui a présenté de la crème dans la coupe des chefs.
\VS{26}Elle a saisi de sa main gauche le pieu et de sa main droite le marteau des ouvriers ; elle a frappé Sisera et lui a fendu la tête ; elle a fracassé et transpercé ses tempes.
\VS{27}Il s'est affaissé aux pieds de Jaël, il est tombé, il s’est couché aux pieds de Jaël ; il s'est affaissé, il est tombé ; là où il s'est affaissé, il est tombé là tout défiguré.
\VS{28}La mère de Sisera regardait par la fenêtre et s'écriait en regardant par les treillis : Pourquoi son char tarde-t-il à venir ? Pourquoi ses chars vont-ils si lentement ?
\VS{29}Les plus sages de ses dames lui répondent, et elle se répond à elle-même :
\VS{30}N’ont-ils pas trouvé ? ils partagent le butin ; une fille, deux filles à chacun par tête. Le butin des vêtements de couleurs est à Sisera, le butin de couleurs de broderie ; couleur de broderie à deux endroits, autour du cou de ceux du butin.
\VS{31}Périssent ainsi, tous tes ennemis ô Yahweh ! Et que ceux qui t'aiment soient comme le soleil quand il sort dans sa force. Et le pays fut en repos pendant quarante ans.
\Chap{6}
\TextTitle{Israël assujetti par Madian}
\VerseOne{}Or, les enfants d'Israël firent ce qui est mal aux yeux de Yahweh ; et Yahweh les livra entre les mains de Madian pendant sept ans.
\VS{2}La main de Madian fut puissante contre Israël. Pour échapper aux Madianites, les enfants d'Israël se retiraient dans les ravins des montagnes, dans des cavernes et sur les rochers fortifiés.
\VS{3}Car il arrivait que quand Israël avait semé, Madian montait avec Amalek et les fils de l’orient, et ils montaient contre lui.
\VS{4}Ils faisaient un camp contre lui, ravageaient les fruits du pays jusqu'à Gaza et ne laissaient en Israël ni vivres, ni brebis, ni bœufs, ni ânes.
\VS{5}Car ils montaient avec leurs troupeaux et leurs tentes, ils arrivaient comme une multitude de sauterelles, ils étaient innombrables, eux et leurs chameaux et ils venaient dans le pays pour le ravager.
\VS{6}Israël fut très appauvri par Madian, et les enfants d'Israël crièrent à Yahweh.
\VS{7}Lorsque les enfants d'Israël crièrent à Yahweh au sujet de Madian,
\VS{8}Yahweh envoya un prophète aux enfants d'Israël, qui leur dit : Ainsi parle Yahweh, le Dieu d'Israël : Je vous ai fait monter hors d’Égypte et je vous ai retirés de la maison de servitude.
\VS{9}Je vous ai délivrés de la main des Egyptiens et de la main de tous ceux qui vous opprimaient ; je les ai chassés devant vous et je vous ai donné leur pays.
\VS{10}Je vous ai dit : Je suis Yahweh, votre Dieu ; vous ne craindrez pas les dieux des Amoréens, dans le pays desquels vous habitez. Mais vous n'avez pas obéi à ma voix.
\TextTitle{Gédéon rencontre l’Ange de Yahweh}
\VS{11}Puis l'Ange de Yahweh vint et s'assit sous le térébinthe d’Ophra, qui appartenait à Joas, de la famille d'Abiézer. Gédéon, son fils, battait du froment au pressoir pour le mettre à l'abri de Madian.
\VS{12}Alors l'Ange de Yahweh lui apparut et lui dit : Très fort et vaillant héros, Yahweh est avec toi !
\VS{13}Gédéon lui répondit : Hélas mon Seigneur ! Est-il possible que Yahweh soit avec nous ? Pourquoi donc toutes ces choses nous sont-elles arrivées ? Et où sont tous ces prodiges que nos pères nous ont racontés, en disant : Yahweh ne nous a-t-il pas fait monter hors d'Egypte ? Car maintenant Yahweh nous a abandonnés et nous a livrés entre les mains des Madianites.
\VS{14}Yahweh le regarda et lui dit : Va avec cette force que tu as et tu délivreras Israël de la main des Madianites ; ne t'ai-je pas envoyé\FTNT{Hé.11:32}?
\VS{15}Et il lui répondit : Hélas, mon Seigneur ! Avec quoi délivrerai-je Israël ? Voici, mon millier de bétail est le plus pauvre en Manassé et je suis le plus petit de la maison de mon père\FTNT{1 S. 9:21 ; 1 S. 16:11.}.
\VS{16}Yahweh lui dit : Parce que je serai avec toi, tu frapperas les Madianites comme s'ils n'étaient qu'un seul homme.
\VS{17}Et il lui répondit : Je te prie, si j'ai trouvé grâce à tes yeux, donne-moi un signe pour montrer que c'est toi qui me parles.
\VS{18}Je te prie, ne t’éloigne pas d’ici jusqu'à ce que je revienne auprès de toi, que j'apporte mon offrande et que je la dépose devant toi. Yahweh dit : Je resterai jusqu'à ce que tu reviennes.
\VS{19}Alors Gédéon rentra et apprêta un chevreau de lait, et fit avec un épha de farine des pains sans levain. Il mit la chair dans un panier, le jus dans un pot et il les lui apporta sous le térébinthe, et les présenta.
\VS{20}L'Ange de Dieu lui dit : Prends la chair et les pains sans levain et pose-les sur ce rocher\FTNT{Voir commentaire en  Es. 8:13-17} et répands le jus. Et il fit ainsi.
\VS{21}Alors l'Ange de Yahweh avança l’extrémité du bâton qu'il avait à la main, et toucha la chair et les pains sans levain. Le feu monta du rocher, et consuma la chair et les pains sans levain. Puis l'Ange de Yahweh disparut à ses yeux.
\VS{22}Gédéon, voyant que c'était l'Ange de Yahweh, dit : Ah, malheur à moi, Seigneur Yahweh ! Car j'ai vu l’Ange de Yahweh face à face.
\VS{23}Et Yahweh lui dit : Sois en paix, ne crains pas, tu ne mourras pas.
\VS{24}Gédéon bâtit là un autel à Yahweh, et lui donna pour nom Yahweh-Shalom. Cet autel, qui appartenait à la famille d'Abiézer, existe encore aujourd'hui à Ophra.
\TextTitle{Gédéon détruit les idole ; Yahweh lui confirme sa mission}
\VS{25}Or il arriva dans cette nuit-là que Yahweh lui dit : Prends un jeune taureau d'entre les bœufs qui sont à ton père et un deuxième taureau de sept ans ; et démolis l'autel de Baal qui est à ton père, et abats l’idole d'Astarté qui est dessus.
\VS{26}Tu bâtiras ensuite et tu disposeras, sur le haut de ce rocher, un autel à Yahweh, ton Dieu. Tu prendras ce deuxième taureau, et tu l'offriras en holocauste avec le bois de l’emblème d’Astarté que tu auras démoli.
\VS{27}Gédéon ayant pris dix hommes parmi ses serviteurs, fit comme Yahweh lui avait dit ; et parce qu'il craignait la maison de son père et les gens de la ville, il l’exécuta de nuit et non de jour.
\VS{28}Lorsque les gens de la ville se levèrent de bon matin, voici, l'autel de Baal avait été démoli, et l'idole d'Astarté qui est dessus était abattue, et le deuxième taureau était offert en holocauste sur l'autel qui avait été bâti.
\VS{29}Ils se dirent les uns aux autres : Qui a fait cela ? Et ils s’informèrent et firent des recherches. On leur dit : C’est Gédéon, fils de Joas, qui a fait cela.
\VS{30}Puis les gens de la ville dirent à Joas : Fais sortir ton fils et qu'il meure ; car il a démoli l'autel de Baal et abattu l'idole d'Astarté qui est dessus.
\VS{31}Joas répondit à tous ceux qui s'adressèrent à lui : Est-ce à vous de prendre parti pour Baal, est-ce à vous de venir à son secours ? Quiconque prendra parti pour Baal sera mis à mort avant le matin. Si Baal est un dieu, qu'il défende lui-même sa cause puisqu'on a démoli son autel.
\VS{32}Et en ce jour on donna à Gédéon le nom de Jerubbaal, en disant : Que Baal défende sa cause, puisque Gédéon a démoli son autel.
\VS{33}Tout Madian, Amalek, et les fils de l’orient se rassemblèrent ;  ils passèrent le Jourdain et campèrent dans la vallée de Jizréel.
\VS{34}Gédéon fut revêtu de l'Esprit de Yahweh ; il sonna du shofar et Abiézer fut convoqué pour marcher à sa suite\FTNT{Jg. 11:29 ; Jg. 13:25.}.
\VS{35}Il envoya des messagers dans tout Manassé qui fut aussi convoqué pour marcher à sa suite. Puis il envoya des messagers dans Aser, dans Zabulon et dans Nephthali, qui montèrent à leur rencontre.
\VS{36}Gédéon dit à Dieu : Si tu veux délivrer Israël par ma main, comme tu l'as dit,
\VS{37}voici, je vais mettre une toison de laine dans l'aire de battage ; si la toison seule se couvre de rosée et que tout le terrain reste sec, je connaîtrai que tu délivreras Israël par ma main, comme tu l’as dit.
\VS{38}Et il arriva ainsi. Le jour suivant, il se leva de bon matin, pressa la toison et en fit sortir la rosée qui donna de l’eau plein une coupe.
\VS{39}Gédéon dit encore à Dieu : Que ta colère ne s'enflamme pas contre moi, et je ne parlerai plus que cette fois : Je te prie, je voudrais seulement faire encore une épreuve avec la toison : Que la toison seule reste sèche et que tout le terrain se couvre de rosée.
\VS{40}Et Dieu fit ainsi cette nuit-là. La toison seule resta sèche, et tout le terrain se couvrit de rosée.
\Chap{7}
\TextTitle{Yahweh sélectionne un petit nombre pour le combat}
\VerseOne{}Jerubbaal qui est Gédéon, et tout le peuple qui était avec lui, se levèrent de bon matin et campèrent près de la source de Harod. Le camp de Madian était au nord, vers la colline de Moré, dans la vallée.
\VS{2}Yahweh dit à Gédéon : Le peuple qui est avec toi est trop nombreux pour que je livre Madian entre ses mains, de peur qu'Israël ne se glorifie contre moi, en disant : C’est ma main qui m'a délivré.
\VS{3}Maintenant donc fais plublier ceci aux oreilles du peuple, et qu'on dise : Que celui qui est craintif et qui a peur s’en retourne et s’éloigne de la montagne de Galaad. Vingt-deux mille hommes parmi le peuple s'en retournèrent et il en resta dix mille\FTNT{De. 20:8.}.
\VS{4}Yahweh dit à Gédéon : Le peuple est encore trop nombreux. Fais-les descendre vers l'eau et là je les épurerai\FTNT{C’est Dieu qui qualifie ses ouvriers, il les éprouve et les épure pour les rendre inébranlables. Voir le test de l’épreuve des Hébreux dans le désert de Sinaï (De. 8).} ; et celui dont je te dirai : Que celui-ci aille avec toi, ira avec toi ; et celui dont je te dirai : Que celui-ci n’aille pas avec toi, n’ira pas avec toi.
\VS{5}Il fit donc descendre le peuple vers l'eau ; et Yahweh dit à Gédéon : Tous ceux qui laperont l'eau avec la langue comme lape le chien, tu les sépareras de tous ceux qui se mettront à genoux pour boire\FTNT{Ps. 110:7.}.
\VS{6}Ceux qui lapèrent l’eau en la portant à la bouche avec leur main furent au nombre de trois cents hommes et tout le reste du peuple se mit à genoux pour boire.
\VS{7}Alors Yahweh dit à Gédéon : C’est par les trois cents hommes qui ont lapé, que je vous délivrerai et que je livrerai Madian entre tes mains. Que tout le reste du peuple s'en aille donc chacun chez soi.
\VS{8}Ainsi le peuple prit entre ses mains des provisions et ses shofars. Gédéon renvoya tous les hommes d'Israël chacun dans sa tente et il retint les trois cents hommes. Or le camp de Madian était au-dessous de lui, dans la vallée.
\TextTitle{Victoire de Gédéon sur Madian}
\VS{9}Et il arriva cette nuit-là que Yahweh lui dit : Lève-toi, descends au camp car je l'ai livré entre tes mains.
\VS{10}Si tu crains de descendre, descends-y avec Pura, ton serviteur.
\VS{11}Tu écouteras ce qu'ils diront et après cela, tes mains seront fortifiées ; descends donc au camp. Il descendit avec Pura, son serviteur, jusqu'aux avant-postes du camp.
\VS{12}Or Madian, Amalek et tous les fils de l'orient étaient répandus dans la vallée comme des sauterelles, tant il y en avait, et leurs chameaux étaient sans nombre, comme le sable qui est sur le bord de la mer, tant il y en avait\FTNT{Jg. 6:3-33.}.
\VS{13}Gédéon arriva ; et voici, un homme racontait à son compagnon un songe. Il lui disait : Voici, j'ai eu un songe ; il me semblait qu'un gâteau de pain d'orge roulait dans le camp de Madian ; et il est venu heurter jusqu’à la tente et elle est tombée ; il l’a retournée sens dessus dessous et elle a été renversée.
\VS{14}Alors son compagnon répondit et dit : Ce n'est pas autre chose que l'épée de Gédéon, fils de Joas, homme d'Israël ; Dieu a livré Madian et tout le camp entre ses mains.
\VS{15}Lorsque Gédéon eut entendu le récit du songe et son interprétation, il se prosterna, revint au camp d'Israël et dit : Levez-vous car Yahweh a livré le camp de Madian entre vos mains.
\VS{16}Puis il divisa les trois cents hommes en trois corps et il leur donna à chacun des shofars à la main et des cruches vides, avec des flambeaux dans les cruches.
\VS{17}Il leur dit : Regardez-moi et faites comme je ferai. Dès que je serai arrivé à l’extrémité du camp, vous ferez comme je ferai.
\VS{18}Quand je sonnerai du shofar, moi et tous ceux qui sont avec moi, alors vous sonnerez aussi du shofar tout autour du camp et vous direz : Pour Yahweh et pour Gédéon !
\VS{19}Gédéon et les cent hommes qui étaient avec lui arrivèrent à l’extrémité du camp, au commencement de la veille de la nuit, comme on venait de placer les gardes. Ils sonnèrent du shofar et  brisèrent les cruches qu'ils avaient à la main.
\VS{20}Ainsi les trois corps sonnèrent du shofar, et brisèrent les cruches ; ils saisirent de la main gauche les flambeaux et de la main droite les shofars pour sonner et ils s’écrièrent : L'épée de Yahweh et de Gédéon !
\VS{21}Ils restèrent chacun à sa place autour du camp, et tout le camp se mit à courir ça et là, à pousser des cris et à prendre la fuite.
\VS{22}Car comme les trois cents hommes sonnèrent encore du shofar, Yahweh leur fit tourner l'épée les uns contre les autres. Le camp s'enfuit jusqu'à Beth-Schitta, vers Tseréra, jusqu'au bord d'Abel-Mehola, près de Tabbath\FTNT{1 S. 14:20 ; Ez. 38:21.}.
\VS{23}Les hommes d'Israël, à savoir ceux de Nephthali, d'Aser et de tout Manassé, se rassemblèrent, et ils poursuivirent Madian.
\VS{24}Alors Gédéon envoya des messagers dans toute la montagne d'Ephraïm, pour leur dire : Descendez pour aller à la rencontre de Madian, et coupez-leur les premiers le passage des eaux jusqu'à Beth-Bara et celui du Jourdain. Tous les hommes d'Ephraïm se rassemblèrent, et ils s’emparèrent du passage des eaux jusqu’à Beth-Bara et de celui du Jourdain.
\VS{25}Ils saisirent deux des chefs de Madian, Oreb et Zeeb ; ils tuèrent Oreb au rocher d’Oreb, et ils tuèrent Zeeb au pressoir de Zeeb. Ils poursuivirent Madian, et ils apportèrent les têtes de Oreb et de Zeeb à Gédéon, de l’autre côté du Jourdain\FTNT{Ps. 83:11 ; Es. 10:26.}.
\Chap{8}
\TextTitle{Poursuite de Zébach et Tsalmunna ; exécution des rois de Madian}
\VerseOne{}Alors les hommes d'Ephraïm dirent à Gédéon : Que signifie cette manière d’agir envers nous ? Pourquoi ne pas nous avoir appelés quand tu es allé à la guerre contre Madian ? Et ils s'emportèrent fortement contre lui\FTNT{Jg. 12:1.}.
\VS{2}Et il leur répondit : Qu'ai-je fait maintenant au prix de ce que vous avez fait ? Les grappillages d'Ephraïm ne sont-ils pas meilleurs que la vendange d'Abiézer ?
\VS{3}Dieu a livré entre vos mains les chefs de Madian, Oreb et Zeeb. Qu'ai-je pu faire au prix de ce que vous avez fait ? Et leur esprit fut apaisé envers lui lorsqu’il eut ainsi parlé.
\VS{4}Gédéon arriva au Jourdain, et il le passa, lui et les trois cents hommes qui étaient avec lui, fatigués, mais poursuivant toujours l'ennemi.
\VS{5}C'est pourquoi il dit aux gens de Succoth : Donnez, je vous prie, quelques pains aux hommes qui m’accompagnent, car ils sont fatigués, et ainsi je poursuivrai Zébach et Tsalmunna, rois de Madian.
\VS{6}Mais les chefs de Succoth répondirent : La main de Zébach et celle de Tsalmunna sont-elles déjà en ton pouvoir, pour que nous donnions du pain à ton armée ?
\VS{7}Et Gédéon dit : Eh bien ! Quand Yahweh aura livré Zébach et Tsalmunna entre mes mains, je foulerai au pied votre chair avec des épines du désert et avec des chardons.
\VS{8}Puis de là il monta à Penuel, et il fit la même demande aux gens de Penuel. Les gens de Penuel lui répondirent comme avaient répondu ceux de Succoth.
\VS{9}Et il dit aussi aux gens de Penuel : Quand je reviendrai en paix, je démolirai cette tour.
\VS{10}Zébach et Tsalmunna étaient à Karkor et leurs armées avec eux, environ quinze mille hommes, tous ceux qui étaient restés de l'armée entière des fils de l’orient ; cent vingt mille hommes tirant l’épée avaient été tués.
\VS{11}Gédéon monta par le chemin de ceux qui habitent sous les tentes, à l’orient de Nobach et de Jogbeha, et il battit l'armée, qui se croyait en sûreté.
\VS{12}Et comme Zébach et Tsalmunna s'enfuyaient, il les poursuivit, et prit les deux rois de Madian, Zébach et Tsalmunna, et mit en déroute toute l'armée\FTNT{Ps. 83:11.}.
\TextTitle{Vengeance sur Succoth et Penuel ; exécution de Zébach et Tsalmunna}
\VS{13}Puis Gédéon, fils de Joas, revint de la bataille par la montée de Hérès.
\VS{14}Il saisit un garçon d’entre les hommes de Succoth, il l'interrogea, et ce garçon lui donna par écrit le nom des chefs et des anciens de Succoth, au nombre de soixante-dix-sept hommes.
\VS{15}Et il vint auprès de gens de Succoth, et leur dit : Voici Zébach et Tsalmunna, au sujet desquels vous m'avez insulté, en disant : La main de Zébach et celle de Tsalmunna sont-elles déjà en ton pouvoir, pour que nous donnions du pain à tes hommes fatigués ?
\VS{16}Il prit donc les anciens de la ville et châtia les hommes de Succoth avec des épines du désert et des chardons.
\VS{17}Il démolit la tour de Penuel, et tua les gens de la ville.
\VS{18}Puis il dit à Zébach et à Tsalmunna : Comment étaient les hommes que vous avez tués à Thabor ? Ils répondirent : Ils étaient entièrement comme toi, chacun d'eux avait l'air d'un fils de roi.
\VS{19}Il leur dit : C'étaient mes frères, fils de ma mère. Yahweh est vivant, si vous les aviez laissés vivre, je ne vous tuerais pas.
\VS{20}Puis il dit à Jéther, son premier-né : Lève-toi, tue-les ! Mais le jeune garçon ne tira pas son épée, car il avait peur, car il était encore un enfant.
\VS{21}Et Zébach et Tsalmunna dirent : Lève-toi toi-même, et jette-toi sur nous ! Car tel est l'homme, telle est sa force. Et Gédéon se leva, et tua Zébach et Tsalmunna. Il prit ensuite les croissants qui étaient aux cous de leurs chameaux.
\TextTitle{Gédéon recommande au peuple le règne de Yahweh}
\VS{22}Les hommes d'Israël dirent tous d'un commun accord à Gédéon : Domine sur nous, tant toi que ton fils, et le fils de ton fils, car tu nous as délivrés de la main de Madian.
\VS{23}Gédéon leur répondit : Je ne dominerai pas sur vous, et mon fils ne dominera pas sur vous ; c'est Yahweh qui dominera sur vous\FTNT{De. 17:15.}.
\TextTitle{Gédéon introduit une occasion de chute en Israël}
\VS{24}Mais Gédéon leur dit : J’ai une demande à vous faire : Donnez-moi chacun les anneaux que vous avez eus pour butin. Les ennemis avaient des anneaux d'or, car ils étaient Ismaélites.
\VS{25}Ils répondirent : Nous les donnerons volontiers. Et ils étendirent un manteau sur lequel chacun jeta les anneaux de son butin.
\VS{26}Le poids des anneaux d'or que Gédéon demanda fut de mille sept cents sicles d'or, sans les croissants, les pendants d'oreilles, et les vêtements de pourpre que portaient les rois de Madian, et sans les colliers qui étaient aux cous de leurs chameaux.
\VS{27}Puis Gédéon en fit un  éphod\FTNT{Sous Moïse, il y avait deux sortes d'éphods, le premier était de simple lin pour les sacrificateurs, et le deuxième de broderie pour le souverain sacrificateur. Comme celui des simples sacrificateurs n'avait rien de particulier, Moïse ne s'est pas arrêté à le décrire. Mais il décrit longuement celui du souverain sacrificateur. (Ex. 28:6-9). Il était composé d'or, d'hyacinthe, de pourpre, de cramoisi, de coton retors ; c'était un tissu de différentes couleurs. Il y avait à l'endroit de l'éphod qui venait sur les deux épaules du souverain sacrificateur, deux grosses pierres précieuses, qui étaient chargées du nom des douze tribus d’Israël, six noms sur chaque pierre. A l'endroit où l'éphod se croisait sur la poitrine du grand prêtre, il y avait un ornement carré, nommé le rational, en hébreu «~choschen~», dans lequel étaient enchâssées douze pierres précieuses, où l'on avait gravé les noms des douze tribus d'Israël ; un sur chacune des pierres.}, et le mit dans sa ville, à Ophra, où il devint un objet de prostitution pour tout Israël ; il fut un piège pour Gédéon et pour sa maison.
\TextTitle{Fin de la vie de Gédéon ; rechute d’Israël après sa mort}
\VS{28}Ainsi Madian fut humilié devant les enfants d'Israël, et il ne leva plus la tête. Le pays fut en repos pendant quarante ans, durant les jours de Gédéon.
\VS{29}Jerubbaal, fils de Joas s’en retourna dans sa ville, et demeura dans sa maison.
\VS{30}Gédéon eut soixante-dix fils, issus de ses reins, car il eut plusieurs femmes.
\VS{31}Sa concubine, qui était à Sichem, lui enfanta aussi un fils, et il lui donna le nom d’Abimélec.
\VS{32}Puis Gédéon, fils de Joas, mourut après une heureuse vieillesse ; et il fut enseveli dans le sépulcre de Joas, son père, à Ophra, qui appartenait à la famille d’Abiézer.
\TextTitle{Rechute dans l'idolâtrie}
\VS{33}Et il arriva après que Gédéon fut mort, que les enfants d'Israël se détournèrent et se prostituèrent aux Baals, et ils établirent Baal-Berith pour leur dieu\FTNT{Jg. 2:11-17 ; 10:6.}.
\VS{34}Ainsi les enfants d'Israël ne se souvinrent pas de Yahweh, leur Dieu, qui les avait délivrés de la main de tous leurs ennemis qui les entouraient.
\VS{35}Et ils n'usèrent d'aucune loyauté envers la maison de Jerubbaal, de Gédéon, après tout le bien qu'il avait fait à Israël.
\Chap{9}
\TextTitle{Conspiration d’Abimélec pour régner sur Israël}
\VerseOne{}Et Abimélec, fils de Jerubbaal, s'en alla à Sichem vers les frères de sa mère, et leur parla, ainsi qu'à toute la maison du père de sa mère :
\VS{2}Je vous prie, faites entendre ces paroles à tous les seigneurs de Sichem : Lequel vous semble le meilleur, que soixante-dix hommes, tous fils de Jerubbaal, dominent sur vous, ou qu'un seul homme domine sur vous ? Et souvenez-vous que je suis votre os et votre chair\FTNT{Ge. 29:14.}.
\VS{3}Les frères de sa mère dirent de sa part toutes ces paroles aux oreilles de tous les seigneurs de Sichem, et leur cœur se tourna après Abimélec, car ils disaient : C'est notre frère.
\VS{4}Ils lui donnèrent soixante-dix sicles d'argent de la maison de Baal-Berith. Abimélec s'en servit pour acheter des hommes misérables et turbulents, qui allèrent après lui.
\VS{5}Et il vint dans la maison de son père à Ophra, et tua sur une seule pierre ses frères, fils de Jerubbaal, qui étaient soixante-dix hommes. Il ne resta que Jotham, le plus jeune fils de Jerubbaal, parce qu'il s'était caché.
\VS{6}Et tous les seigneurs de Sichem s'assemblèrent avec toute la maison de Millo ; ils vinrent, et firent d'Abimélec leur roi près du chêne à Sichem.
\VS{7}On le rapporta à Jotham, qui alla se tenir au sommet de la montagne de Garizim, et les appelant, il dit en élevant la voix : Écoutez-moi, seigneurs de Sichem, et que Dieu vous entende !
\VS{8}Les arbres allèrent pour oindre un roi, et ils dirent à l'olivier : Règne sur nous.
\VS{9}Mais l'olivier leur répondit : Renoncerai-je à mon huile, par laquelle Dieu et les hommes sont honorés, pour aller m'agiter sur les arbres\FTNT{Ps. 104:15.} ?
\VS{10}Puis les arbres dirent au figuier : Viens, toi, règne sur nous.
\VS{11}Mais le figuier leur répondit : Renoncerai-je à ma douceur, et à mon bon fruit, pour aller m'agiter sur les arbres ?
\VS{12}Puis les arbres dirent à la vigne : Viens, toi, et règne sur nous.
\VS{13}Mais la vigne répondit : Renoncerai-je à mon vin, qui réjouit Dieu et les hommes, pour aller m'agiter sur les arbres ?
\VS{14}Alors tous les arbres dirent à l'épine : Viens, toi, et règne sur nous.
\VS{15}Et l'épine répondit aux arbres : Si c'est en vérité que vous m'oignez pour roi, venez, et réfugiez-vous sous mon ombrage ; sinon, que le feu sorte de l'épine, et qu'il dévore les cèdres du Liban.
\VS{16}Maintenant donc, est-ce en vérité et avec intégrité que vous avez agi en établissant Abimélec pour roi ? Avez-vous bien fait envers Jerubbaal et sa maison ? L'avez-vous fait selon les bienfaits qu'il a rendus de sa main ?
\VS{17}Car mon père a combattu pour vous, il a exposé sa vie devant vous, et vous a délivrés de la main de Madian ;
\VS{18}Mais vous vous êtes levés aujourd'hui contre la maison de mon père, et avez tué sur une pierre ses fils, soixante-dix hommes, et avez établi pour roi Abimélec, fils de sa servante, sur les habitants de Sichem, parce qu'il est votre frère.
\VS{19}Si, dis-je, vous avez agi aujourd'hui en vérité et avec intégrité envers Jerubbaal, et sa maison, réjouissez-vous d'Abimélec, et qu'il se réjouisse aussi de vous !
\VS{20}Sinon, que le feu sorte d'Abimélec et qu'il dévore les seigneurs de Sichem, et la maison de Millo ; et que le feu sorte des seigneurs de Sichem, et de la maison de Millo, et qu'il dévore Abimélec !
\VS{21}Puis Jotham s'enfuit rapidement ; il s'en alla à Beer, où il demeura loin d'Abimélec, son frère.
\TextTitle{Sichem se retourne contre Abimélec}
\VS{22}Abimélec gouverna sur Israël durant trois ans.
\VS{23}Alors Dieu envoya un mauvais esprit entre Abimélec et les seigneurs de Sichem, et les seigneurs de Sichem furent infidèles à Abimélec.
\VS{24}Afin que la violence faite aux soixante-dix fils de Jerubbaal vienne et que leur sang se tourne contre Abimélec, leur frère, qui les avait tués, et sur les seigneurs de Sichem, qui l'avaient aidé par leur main à tuer ses frères.
\VS{25}Les seigneurs de Sichem mirent des embûches sur le sommet des montagnes, des gens pillaient tous ceux qui passaient près d'eux sur le chemin. Cela fut rapporté à Abimélec.
\VS{26}Alors Gaal, fils d'Ebed, vint avec ses frères, et ils passèrent à Sichem. Les seigneurs de Sichem eurent confiance en lui.
\VS{27}Puis étant sortis aux champs, ils vendangèrent leurs vignes, foulèrent les raisins, et se donnèrent à des réjouissances ; ils entrèrent dans la maison de leur dieu, ils mangèrent et burent, et ils maudirent Abimélec.
\VS{28}Alors Gaal, fils d'Ebed, dit : Qui est Abimélec, et qui est Sichem pour que nous servions Abimélec ? N'est-il pas le fils de Jerubbaal et Zebul n'est-il pas son commissaire ? Servez plutôt les hommes de Hamor, père de Sichem ; mais pour quelle raison servirions-nous Abimélec ?
\VS{29}Plaise à Dieu ! Qu'on mette ce peuple sous mon pouvoir, et je chasserais Abimélec. Et il disait d'Abimélec : Multiplie ton armée, et sors !
\VS{30}Zebul, gouverneur de la ville, entendit les paroles de Gaal, fils d'Ebed, et sa colère s'enflamma.
\VS{31}Puis il envoya astucieusement des messagers vers Abimélec, pour lui dire : Voici, Gaal fils d'Ebed, et ses frères, sont entrés dans Sichem, et voici, ils assiègent la ville contre toi.
\VS{32}Maintenant donc, lève-toi de nuit, toi et le peuple qui est avec toi, et mets-toi en embuscade dans les champs.
\VS{33}Et le matin, au lever du soleil, tu te lèveras et tu te jetteras sur la ville. Gaal et le peuple qui est avec lui sortiront contre toi, ta main lui fera selon les forces que tu trouveras.
\VS{34}Abimélec et tout le peuple qui était avec lui se levèrent de nuit, et ils se mirent en embuscade contre Sichem, divisés en quatre bandes.
\VS{35}Alors Gaal, fils d'Ebed, sortit, et il se tint à l'entrée de la porte de la ville. Abimélec et tout le peuple qui était avec lui se levèrent de l'embuscade.
\VS{36}Gaal voyant le peuple, dit à Zebul : Voici un peuple qui descend du sommet des montagnes. Zebul lui dit : Tu vois l'ombre des montagnes comme des hommes.
\VS{37}Gaal, parla encore, et dit : C'est bien un peuple qui descend des hauteurs du pays, et une bande vient du chemin du chêne des devins.
\VS{38}Et Zebul lui dit : Où est donc ta bouche, toi qui disais : Qui est Abimélec, pour que nous le servions ? N'est-ce pas ici ce peuple que tu méprisais ? Sors maintenant, je te prie, et combats !
\VS{39}Alors, Gaal sortit conduisant les seigneurs de Sichem, et combattit contre Abimélec.
\VS{40}Abimélec le poursuivit, et il s'enfuit de devant lui, et plusieurs tombèrent morts jusqu'à l'entrée de la porte.
\VS{41}Abimélec s'arrêta à Aruma. Zebul repoussa Gaal et ses frères, afin qu'ils ne restent plus à Sichem.
\VS{42}Et il arriva, dès le lendemain, que le peuple sortit aux champs. Cela fut rapporté à Abimélec,
\VS{43}qui prit son peuple, et le divisa en trois bandes, et les mit en embuscade dans les champs. Ayant vu que le peuple sortait de la ville, il se leva contre eux, et les battit.
\VS{44}Abimélec et la bande qui était avec lui se répandirent, et se tinrent à l'entrée de la porte de la ville ; mais les deux autres bandes se jetèrent sur tous ceux qui étaient aux champs, et les battirent.
\VS{45}Ainsi Abimélec combattit contre la ville toute la journée ; il prit la ville, et tua le peuple qui y était. Il la rasa, et y sema du sel.
\VS{46}Ayant appris cela, tous les seigneurs de la tour de Sichem entrèrent dans la forteresse de la maison du dieu Baal-Berith\FTNT{Jg. 9:4 ; 8:33.}.
\VS{47}On rapporta à Abimélec que tous les seigneurs de la tour de Sichem s'étaient assemblés dans la forteresse.
\VS{48}Alors Abimélec monta sur la montagne de Tsalmon, lui et tout le peuple qui était avec lui. Il prit en main une hache, coupa une branche d'arbre, et l'ayant mise sur son épaule, la porta, et dit au peuple qui était avec lui : Avez-vous vu ce que j'ai fait ? Hâtez-vous de faire comme moi.
\VS{49}Chacun donc de tout le peuple coupa une branche, et ils marchèrent derrière Abimélec ; ils mirent ces branches tout autour de la forteresse, et y mirent le feu. Il brûlèrent la forteresse, et toutes les personnes de la tour de Sichem moururent ; au nombre d'environ mille, tant hommes que femmes.
\TextTitle{Abimélec meurt}
\VS{50}Puis Abimélec marcha contre Thébets, y mit son camp, et la prit.
\VS{51}Il y avait au milieu de la ville une forte tour, où s'enfuirent tous les hommes et toutes les femmes, et tous les seigneurs de la ville, et ayant fermé les portes après eux, ils montèrent sur le toit de la Tour.
\VS{52}Alors Abimélec alla jusqu'à la tour, l'attaqua, et s'approcha jusqu'à la porte pour la brûler par le feu.
\VS{53}Mais une femme jeta une pièce de meule de moulin sur la tête d'Abimélec, et lui brisa le crâne\FTNT{2 S. 11:21.}.
\VS{54}Rapidement, il appela le garçon qui portait ses armes, et lui dit : Tire ton épée, et tue-moi, de peur qu'on ne dise de moi : C'est une femme qui l'a tué. Le garçon le transperça, et il mourut\FTNT{1 S. 31:4.}.
\VS{55}Quand les hommes d'Israël virent qu'Abimélec était mort, ils s'en allèrent chacun en son lieu.
\VS{56}Ainsi Dieu rendit à Abimélec le mal qu'il avait fait contre son père, en tuant ses soixante-dix frères,
\VS{57}Et toute la méchanceté des hommes de Sichem ; Dieu, dis-je, la fit retourner sur leurs têtes ; et ainsi la malédiction de Jotham, fils de Jérubbaal, vint sur eux.
\Chap{10}
\TextTitle{Thola, juge en Israël}
\VerseOne{}Après Abimélec, Thola fils de Pua, fils de Dodo, homme d'Issacar, se leva pour délivrer Israël ; il habitait à Schamir, dans la montagne d'Ephraïm.
\VS{2}Il fut juge en Israël pendant vingt-trois ans ; puis il mourut, et fut enterré à Schamir.
\TextTitle{Jaïr, juge en Israël}
\VS{3}Après lui se leva Jaïr, le Galaadite, qui fut juge en Israël pendant vingt-deux ans.
\VS{4}Il avait trente fils, qui montaient sur trente ânons, et qui avaient trente villes, qu'on appelle jusqu'à ce jour bourgs de Jaïr, lesquelles sont situées au pays de Galaad\FTNT{Jg. 5:10.}.
\VS{5}Et Jaïr mourut, et fut enterré à Kamon.
\TextTitle{Idolâtrie d’Israël et oppression par ses ennemis}
\VS{6}Puis les enfants d'Israël firent encore ce qui est mal aux yeux de Yahweh ; et servirent les Baals et les Astartés, les dieux de Syrie, les dieux de Sidon, les dieux de Moab, les dieux des fils d'Ammon, et les dieux des Philistins, et ils abandonnèrent Yahweh, et ne le servirent plus\FTNT{Jg. 2:11 ; 3:7 ; 8:33.}.
\VS{7}Alors la colère de Yahweh s'enflamma contre Israël, et il les vendit entre les mains des Philistins, et des fils d'Ammon.
\VS{8}Ils opprimèrent et écrasèrent les enfants d'Israël cette année-là, et pendant dix-huit ans tous les enfants d'Israël qui étaient au-delà du Jourdain, au pays des Amoréens en Galaad.
\VS{9}Même les fils d'Ammon passèrent le Jourdain pour combattre contre Juda, contre Benjamin, et contre la maison d'Ephraïm. Israël fut dans une grande détresse.
\VS{10}Alors les enfants d'Israël crièrent à Yahweh, en disant : Nous avons péché contre toi, et certes, nous avons abandonné notre Dieu et nous avons servi les Baals.
\VS{11}Mais Yahweh répondit aux enfants d'Israël : N'avez-vous pas été opprimés par les Egyptiens, les Amoréens, les fils d'Ammon et les Philistins ?
\VS{12}Et lorsque les Sidoniens, Amalek et Maon, vous opprimèrent, et que vous criâtes à moi, ne vous ai-je pas délivrés de leurs mains ?
\VS{13}Mais vous, vous m'avez abandonné, et vous avez servi d'autres dieux. C'est pourquoi je ne vous délivrerai plus.
\VS{14}Allez et criez vers les dieux que vous avez choisis ; qu'ils vous délivrent au temps de votre détresse !
\VS{15}Mais les enfants d'Israël répondirent à Yahweh : Nous avons péché ; traite-nous comme tu le trouveras bon. Nous te prions seulement que tu nous délivres aujourd'hui !
\VS{16}Alors ils ôtèrent du milieu d'eux les dieux des étrangers, et servirent Yahweh, qui fut affligé des souffrances d'Israël.
\VS{17}Les fils d'Ammon se rassemblèrent et campèrent en Galaad, et les enfants d'Israël se rassemblèrent et campèrent à Mitspa.
\VS{18}Le peuple, les chefs de Galaad se dirent l'un à l'autre : Qui sera l'homme qui commencera à combattre contre les fils d'Ammon ? Il sera chef de tous les habitants de Galaad.
\Chap{11}
\TextTitle{Jephté, juge en Israël}
\VerseOne{}Or Jephthé, le Galaadite, était un fort et vaillant homme. Il était le fils d'une femme prostituée ; et c'est Galaad qui l'avait engendré.
\VS{2}La femme de Galaad lui enfanta des fils ; et quand les fils de cette femme furent grands, ils chassèrent Jephthé, en lui disant : Tu n'auras pas d'héritage dans la maison de notre père, car tu es fils d'une autre femme.
\VS{3}Jephthé s'enfuit donc de devant ses frères, et habita au pays de Tob. Des misérables se rassemblèrent auprès de Jephthé, et ils sortirent dehors avec lui\FTNT{Jg. 9:4 ; 1 S. 22:2 ; 1 S 10:6-8.}.
\VS{4}Et il arriva, quelque temps après, les fils d'Ammon firent la guerre à Israël.
\VS{5}Et comme les fils d'Ammon faisaient la guerre à Israël, les anciens de Galaad s'en allèrent pour emmener Jephthé du pays de Tob.
\VS{6}Ils dirent à Jephthé : Viens, et sois notre chef, afin que nous combattions contre les fils d'Ammon.
\VS{7}Jephthé répondit aux anciens de Galaad : N'est-ce pas vous qui m'avez haï et chassé de la maison de mon père ? Pourquoi êtes-vous venus à moi maintenant que vous êtes dans la détresse ?
\VS{8}Alors les anciens de Galaad dirent à Jephthé : La raison pour laquelle nous retournons à toi maintenant, c'est afin que tu viennes avec nous, que tu combattes contre les fils d'Ammon, et que tu sois notre chef, celui de tous les habitants de Galaad.
\VS{9}Jephthé répondit aux anciens de Galaad : Si vous me ramenez pour combattre contre les fils d'Ammon, et que Yahweh les livre devant moi, je serai votre chef.
\VS{10}Les anciens de Galaad dirent à Jephthé : Que Yahweh nous entende, et qu'il juge, si nous ne faisons pas ce que tu dis.
\VS{11}Jephthé donc s'en alla avec les anciens de Galaad. Le peuple le mit à sa tête et l'établit pour chef, et Jephthé déclara devant Yahweh, à Mitspa, toutes les paroles qu'il avait dites.
\VS{12}Puis Jephthé envoya des messagers au roi des fils d'Ammon, pour lui dire : Qu'y a-t-il entre toi et moi, que tu viennes contre moi pour faire la guerre à mon pays ?
\VS{13}Le roi des fils d'Ammon répondit aux messagers de Jephthé : C'est parce qu'Israël a pris mon pays quand il est monté d'Egypte, depuis l'Arnon jusqu'à Jabbok, et même jusqu'au Jourdain. Maintenant rends-le de bon gré.
\VS{14}Mais Jephthé envoya encore des messagers au roi des fils d'Ammon,
\VS{15}qui lui dirent : Ainsi parle Jephthé : Israël n'a rien pris du pays de Moab, ni du pays des fils d'Ammon.
\VS{16}Mais lorsqu’Israël est monté d'Egypte, il est venu par le désert jusqu'à la Mer Rouge et il a atteint Kadès.
\VS{17}Alors Israël envoya des messagers au roi d'Edom, pour lui dire : Que je passe, je te prie, par ton pays. Le roi d'Edom ne voulut pas l'entendre. Il en envoya aussi au roi de Moab, qui ne voulut pas non plus l'entendre. Et Israël demeura à Kadès.
\VS{18}Puis il marcha par le désert, tourna le pays d'Edom et le pays de Moab, et vint à l'orient du pays de Moab ; il campa au-delà de l'Arnon, et n'entra pas sur les frontières de Moab, car l'Arnon est la frontière de Moab.
\VS{19}Mais Israël envoya des messagers à Sihon, roi des Amoréens, roi de Hesbon, auquel Israël dit : Laisse-nous passer par ton pays jusqu'au lieu où nous allons.
\VS{20}Mais Sihon n'eut pas assez confiance en Israël pour le laisser passer sur son territoire ; il rassembla tout son peuple, ils campèrent vers Jahats, et combattirent contre Israël.
\VS{21}Et Yahweh, le Dieu d'Israël, livra Sihon et tout son peuple entre les mains d'Israël, qui les battit. Israël prit possession de tout le pays des Amoréens qui habitaient cette terre.
\VS{22}Ils conquirent donc tout le pays des Amoréens, depuis l'Arnon jusqu'à Jabbok, et depuis le désert jusqu'au Jourdain.
\VS{23}Et maintenant que Yahweh, le Dieu d'Israël, a dépossédé les Amoréens de devant son peuple d'Israël, aurais-tu la possession de leur pays ?
\VS{24}Ce que ton dieu Kemosch te donne à posséder, ne le posséderais-tu pas ? Et tout ce que Yahweh, notre Dieu, a mis en notre possession devant nous, nous ne le posséderions pas !
\VS{25}Or maintenant vaux-tu mieux en quelque sorte que ce soit que Balak, fils de Tsippor, roi de Moab ? A-t-il contesté et combattu contre Israël ?
\VS{26}Voilà trois cents ans qu'Israël demeure à Hesbon, et dans les villes de son ressort, à Aroër, et dans les villes de son ressort, et dans toutes les villes qui sont le long de l'Arnon : Pourquoi ne les avez-vous pas saisies pendant ce temps-là ?
\VS{27}Je ne t'ai pas offensé, mais tu fais mal de me faire la guerre. Que Yahweh, qui est le juge, juge aujourd'hui entre les enfants d'Israël et les fils d'Ammon !
\VS{28}Le roi des fils d'Ammon n'écouta pas les paroles que Jephthé lui fit dire.
\VS{29}L'Esprit de Yahweh fut sur Jephthé. Il passa au travers de Galaad et de Manassé ; il passa jusqu'à Mitspé de Galaad, et de Mitspé de Galaad, il passa jusqu'aux fils d'Ammon.
\TextTitle{Jephté fait un vœu ; Ammon livré entre ses mains}
\VS{30}Jephthé fit un vœu à Yahweh, et dit : Si tu livres les fils d'Ammon entre mes mains,
\VS{31}alors tout ce qui sortira des portes de ma maison au-devant de moi, quand je retournerai en paix chez les fils d'Ammon, sera consacré à Yahweh, et je l'offrirai en holocauste.
\VS{32}Jephthé passa jusqu'où étaient les fils d'Ammon, et Yahweh les livra entre ses mains.
\VS{33}Il les battit par une grande défaite, depuis Aroër jusqu'à Minnith, espace qui renfermait vingt villes, et jusqu'à Abel-Keramim. Et les fils d'Ammon furent humiliés devant les fils d'Israël.
\VS{34}Puis comme Jephthé retourna à Mitspa dans sa maison, voici, sa fille, qui était seule et unique, sans qu'il eût d'autres fils ou filles, sortit au-devant de lui avec des tambourins et des danses.
\VS{35}Et il arriva qu'aussitôt qu'il l'eut aperçue, il déchira ses vêtements, et dit : Ha ! Ma fille ! Tu m'as entièrement abaissé, tu es du nombre de ceux qui me troublent ! J'ai ouvert ma bouche à Yahweh, et je ne puis le révoquer.
\VS{36}Elle répondit : Mon père, si tu as ouvert ta bouche à Yahweh, fais-moi selon ce qui est sorti de ta bouche, puisque Yahweh t'a fait vengeance de tes ennemis, des fils d'Ammon.
\VS{37}Toutefois, elle dit à son père : Que ceci me soit fait : Laisse-moi pendant deux mois ! Je m'en irai, je descendrai par les montagnes, et je pleurerai ma virginité, avec mes compagnes.
\VS{38}Il répondit : Va ! Et il la laissa aller pour deux mois. Elle s'en alla donc avec ses compagnes, et pleura sa virginité dans les montagnes.
\VS{39}Et au bout de deux mois, elle retourna vers son père ; et il lui fit selon le vœu qu'il avait fait\FTNT{Yahweh interdit les sacrifices humains (Lé. 20:2-5 ; Lé. 21 ; De. 12:31 ; De. 18:10).}. Elle n'avait pas connu d'homme. Dès lors, ce fut une coutume en Israël,
\VS{40}tous les ans les filles d'Israël allaient pour célébrer la fille de Jephthé, le Galaadite, quatre jours par an.
\Chap{12}
\TextTitle{Querelle entre Jephté et Ephraïm}
\VerseOne{}Or les hommes d'Ephraïm se rassemblèrent, passèrent par le nord, et dirent à Jephthé : Pourquoi es-tu passé pour combattre contre les enfants d'Ammon, sans nous avoir appelés pour aller avec toi ? Nous brûlerons ta maison, et toi aussi\FTNT{Jg. 8:1.}.
\VS{2}Et Jephthé leur dit : J’ai eu un grand différend avec les enfants de Ammon, moi et mon peuple, et quand je vous ai appelés, vous ne m’avez point délivré de leurs mains.
\VS{3}Voyant que vous ne me délivriez pas, j'ai exposé ma vie, et je suis passé jusqu'où étaient les fils d'Ammon. Yahweh les a livrés entre mes mains. Pourquoi donc aujourd'hui montez-vous vers moi pour me faire la guerre ?
\VS{4}Puis Jephthé assembla tous les hommes de Galaad, et combattit contre Ephraïm. Les hommes de Galaad battirent Ephraïm, parce qu'ils disaient : Vous êtes des fugitifs d'Ephraïm ! Galaad est au milieu d'Ephraïm, au milieu de Manassé !
\VS{5}Les Galaadites se saisirent des gués du Jourdain du côté d'Ephraïm. Et quand l'un des fuyards d'Ephraïm disait : Que je passe ! Les hommes de Galaad lui disaient : Es-tu Ephraïmite ? Il répondait : Non.
\VS{6}Alors ils lui disaient : dis un peu « Schibboleth ». Et il disait « Sibboleth », car il ne pouvait pas le prononcer. Sur quoi, se saisissant de lui, ils le tuaient aux gués du Jourdain. En ce temps-là quarante-deux mille hommes d'Ephraïm périrent.
\VS{7}Jephthé fut juge en Israël pendant six ans ; puis Jephthé le Galaadite mourut, et fut enterré dans l'une des villes de Galaad.
\TextTitle{Ibstan, juge en Israël}
\VS{8}Après lui, Ibtsan de Bethléhem fut juge en Israël.
\VS{9}Il eut trente fils, il envoya trente filles au-dehors, et il fit venir du dehors trente filles pour ses fils. Il fut juge en Israël pendant sept ans.
\VS{10}Puis Ibtsan mourut, et fut enterré à Bethléhem.
\TextTitle{Le juge Elon}
\VS{11}Après lui, Elon de Zabulon fut juge en Israël pendant dix ans.
\VS{12}Puis Elon de Zabulon mourut, et fut enterré à Ajalon, dans le pays de Zabulon.
\TextTitle{Le juge Abdon}
\VS{13}Après lui, Abdon, fils d'Hillel, le Pirathonite, fut juge en Israël.
\VS{14}Il eut quarante fils et trente petits-fils, qui montaient sur soixante-dix ânons. Il fut juge en  Israël pendant huit ans\FTNT{Jg. 10:4. }.
\VS{15}Puis Abdon, fils d'Hillel, le Pirathonite, mourut, et fut enterré à Pirathon, dans le pays d'Ephraïm, sur la montagne des Amalécites.
\Chap{13}
\TextTitle{Israël à nouveau asservi pas les Philistins}
\VerseOne{}Et les enfants d’Israël recommencèrent à faire ce qui est mauvais aux yeux de Yahweh ; et Yahweh les livra entre les mains des Philistins, pendant quarante ans.
\TextTitle{Naissance du juge Samson}
\VS{2}Or il y avait un homme de Tsorea, de la famille des Danites, dont le nom était Manoach. Sa femme était stérile, et n'enfantait pas.
\VS{3}L’Ange de Yahweh apparut à la femme, et lui dit : Voici, tu es stérile, et tu n'as jamais eu d'enfants ; mais tu concevras, et tu enfanteras un fils.
\VS{4}Prends donc bien garde dès maintenant de ne boire ni vin ni liqueur forte, et de ne manger aucune chose impure.
\VS{5}Car voici tu vas être enceinte et tu enfanteras un fils. Le rasoir ne s'élèvera pas sur sa tête, parce que l'enfant sera Naziréen\FTNT{Naziréen vient du mot  «~nazir~» qui signifie «~consacré~» ou «~séparé~». Voir No. 6.} pour Dieu dès le ventre de sa mère ; et ce sera lui qui commencera à délivrer Israël de la main des Philistins.
\VS{6}Et La femme vint, et parla à son mari en disant : Un homme de Dieu est venu vers moi et il avait l'aspect d'un ange de Dieu, un aspect fort redoutable. Je ne lui ai pas demandé d'où il était, et il ne m'a pas déclaré son nom.
\VS{7}Mais il m'a dit : Tu vas être enceinte, et tu enfanteras un fils ; maintenant donc ne bois ni vin ni liqueur forte, et ne mange aucune chose impure, car cet enfant sera Naziréen pour Dieu dès le ventre de sa mère jusqu'au jour de sa mort.
\TextTitle{Prière de Manoach}
\VS{8}Et Manoach pria instamment Yahweh, et dit : Ah ! Seigneur, que l'homme de Dieu que tu as envoyé vienne encore vers nous, et qu'il nous enseigne ce que nous devons faire à l'enfant quand il naîtra !
\VS{9}Et Dieu exauça la prière de Manoach, et l'Ange de Dieu vint encore vers la femme lorsqu'elle était assise dans un champ ; mais Manoach, son mari, n'était pas avec elle.
\VS{10}Et la femme courut vite le rapporter à son mari, en lui disant : Voici, l'homme qui était venu vers moi l'autre jour m'est apparu.
\VS{11}Manoach se leva, suivit sa femme, et venant vers l'homme, il lui dit : Es-tu cet homme qui a parlé à cette femme ? Il répondit : C'est moi.
\VS{12}Manoach dit : Tout ce que tu as dit arrivera, quelle conduite faudra-t-il tenir envers l'enfant, et que lui faudra-t-il faire ?
\VS{13}L'Ange de Yahweh répondit à Manoach : La femme se gardera de tout ce que je lui ai dit.
\VS{14}Elle ne mangera rien qui sorte de la vigne, elle ne boira ni vin ni liqueur forte, et ne mangera aucune chose impure ; elle prendra garde à tout ce que je lui ai ordonné.
\VS{15}Alors Manoach dit à l'Ange de Yahweh : Permets que nous te retenions, et que nous apprêtions un chevreau en ta présence.
\VS{16}Et l'Ange de Yahweh répondit à Manoach : Quand tu me retiendrais, je ne mangerai pas de ton mets ; mais si tu fais un holocauste, tu l'offriras à Yahweh. Manoach ne savait pas que ce fût un Ange de Yahweh.
\VS{17}Et Manoach dit à l'Ange de Yahweh : Quel est ton nom, afin que nous te rendions les honneurs lorsque ta parole viendra ?
\VS{18}Et l'Ange de Yahweh lui répondit : Pourquoi demandes-tu mon nom ? Il est merveilleux.
\VS{19}Alors Manoach prit un chevreau, et une offrande, et les offrit à Yahweh sur le rocher. Il se produisit une chose merveilleuse à la vue de Manoach et de sa femme.
\VS{20}Comme la flamme montait de dessus l'autel vers les cieux, l'Ange de Yahweh monta aussi avec la flamme de l'autel. A cette vue, Manoach et sa femme tombèrent la face contre terre.
\VS{21}L'Ange de Yahweh n'apparut plus à Manoach ni à sa femme. Alors Manoach sut que c'était l'Ange de Yahweh.
\VS{22}Et Manoach dit à sa femme : Certainement nous mourrons, car nous avons vu Dieu.
\VS{23}Mais sa femme lui répondit : Si Yahweh avait voulu nous faire mourir, il n'aurait pas pris de nos mains l'holocauste ni l'offrande, il ne nous aurait pas fait voir toutes ces choses ni fait entendre les choses que nous avons entendues.
\VS{24}Puis cette femme enfanta un fils, et elle l'appela du nom de Samson. L'enfant devint grand, et Yahweh le bénit.
\VS{25}Et l'Esprit de Yahweh commença à l'agiter à Machané-Dan, entre Tsorea et Eschthaol.
\Chap{14}
\TextTitle{Yahweh, le maître des évènements}
\VerseOne{}Samson descendit à Thimna, et il y vit une femme d'entre les filles des Philistins.
\VS{2}Etant remonté dans sa maison, il le déclara à son père et à sa mère, en disant : J'ai vu une femme à Thimna d'entre les filles des Philistins ; prenez-la maintenant, afin qu'elle soit ma femme.
\VS{3}Son père et sa mère lui dirent : N'y a-t-il pas de femme parmi les filles de tes frères et parmi tout notre peuple, pour que tu ailles prendre une femme d'entre les Philistins, ces incirconcis ? Et Samson dit à son père : Prenez-la pour moi, car elle est droite à mes yeux.
\VS{4}Mais son père et sa mère ne savaient pas que cela venait de Yahweh : Car Samson cherchait une occasion de dispute de la part des Philistins. Or en ce temps-là, les Philistins dominaient sur Israël.
\TextTitle{L'énigme de Samson}
\VS{5}Samson descendit avec son père et sa mère à Thimna. Ils allèrent jusqu'aux vignes de Thimna, et voici, un jeune lion rugissant vint à sa rencontre.
\VS{6}Et l'Esprit de Yahweh saisit Samson ; sans avoir rien en sa main, il déchira le lion comme on déchire un chevreau. Il ne déclara pas à son père ni à sa mère ce qu'il avait fait\FTNT{1 S. 17:34-35.}.
\VS{7}Il descendit et parla à la femme, et elle fut trouvée droite à ses yeux.
\VS{8}Puis quelque temps après, il retourna à Thimna pour la prendre, et se détourna pour voir la carcasse du lion. Et voici, il y avait dans la carcasse du lion un essaim d'abeilles et du miel.
\VS{9}Il en prit entre ses mains, et s'en alla en mangeant ; et lorsqu'il fut arrivé vers son père et sa mère, il leur en donna, et ils en mangèrent. Mais il ne leur déclara pas qu'il avait pris ce miel dans la carcasse du lion.
\VS{10}Son père descendit chez la femme. Samson fit là un festin ; car c'est ainsi que les jeunes gens faisaient.
\VS{11}Dès qu'on le vit, on prit trente compagnons qui furent avec lui.
\VS{12}Samson leur dit : Je vous propose une énigme. Si vous me l'expliquez au cours des sept jours du festin, et si vous la trouvez, je vous donnerai trente chemises et trente vêtements de rechange.
\VS{13}Mais si vous ne pouvez pas me l'expliquer, vous me donnerez trente chemises et trente vêtements de rechange. Ils lui répondirent : Propose ton énigme, et nous l'écouterons.
\VS{14}Et il leur dit : De celui qui mange est sorti ce qui se mange, et du fort est sorti le doux. Pendant trois jours, ils ne purent pas expliquer l'énigme.
\VS{15}Et au septième jour, ils dirent à la femme de Samson : Persuade ton mari de nous expliquer l'énigme ; de peur que nous ne te brûlions au feu, toi et la maison de ton père. C'est pour nous déposséder que vous nous avez appelés ici, n'est-ce pas ?
\VS{16}La femme de Samson pleurait auprès de lui, et disait : Certainement tu me hais, et tu ne m'aimes pas ; tu as proposé une énigme aux enfants de mon peuple, et tu ne me l'as pas expliquée ! Et il lui répondait : Je ne l'ai expliquée ni à mon père ni à ma mère ; est-ce à toi que je l'expliquerais ?
\VS{17}Elle pleura ainsi auprès de lui durant les sept jours du festin ; mais au septième jour, il la lui expliqua, parce qu'elle le tourmentait. Puis elle l'expliqua aux enfants de son peuple.
\VS{18}Les gens de la ville lui dirent au septième jour, avant le coucher du soleil : Qu'y a-t-il de plus doux que le miel, et qu'y a-t-il de plus fort que le lion ? Et il leur dit : Si vous n'aviez pas labouré avec ma génisse vous n'auriez pas trouvé mon énigme.
\VS{19}L'Esprit de Yahweh le saisit, et il descendit à Askalon. Il tua trente hommes, il prit leurs dépouilles, et donna les vêtements de rechange à ceux qui avaient expliqué l'énigme. Sa colère s'enflamma, et il monta à la maison de son père.
\VS{20}Et la femme de Samson fut donnée à son compagnon, avec lequel il était lié.
\Chap{15}
\TextTitle{Samson utilisé pour le jugement des Philistins}
\VerseOne{}Et il arriva quelque jours après, au jour de la moisson des blés, que Samson alla visiter sa femme, et lui porta un chevreau. Il dit : J'entrerai vers ma femme dans sa chambre. Mais le père de sa femme ne lui permit pas d'y entrer.
\VS{2}Car il lui dit : J'ai cru que tu avais de la haine pour elle, c'est pourquoi je l'ai donnée à ton compagnon. Sa jeune sœur n'est-elle pas plus belle qu'elle ? Prends-la donc à sa place.
\VS{3}Samson leur dit : Cette fois je serai innocent à l'égard des Philistins si je leur fais du mal.
\VS{4}Samson s'en alla donc. Il prit trois cents renards, il prit aussi des torches ; puis il tourna les renards queue contre queue, et mit une torche entre les deux queues, au milieu.
\VS{5}Puis il mit le feu aux torches, et lâcha les renards dans les blés des Philistins, et brûla le tas de gerbes, le blé sur pied, jusqu'aux plantations d'oliviers.
\VS{6}Les Philistins dirent : Qui a fait cela ? On répondit : Samson, le gendre du Thimnien, parce qu'il lui a pris sa femme et l'a donnée à son compagnon. Les Philistins montèrent, et ils la brûlèrent au feu, elle avec son père.
\VS{7}Alors Samson leur dit : Est-ce donc ainsi que vous faites ? Je ne cesserai qu'après m'être vengé de vous.
\VS{8}Il les battit par une grande défaite, dos et ventre ; puis il descendit, et demeura dans une caverne du rocher d'Etam.
\VS{9}Alors les Philistins montèrent, campèrent en Juda, et s'étendirent jusqu'à Léchi.
\VS{10}Les hommes de Juda dirent : Pourquoi êtes-vous montés contre nous ? Ils répondirent : Nous sommes montés pour lier Samson, afin que nous lui fassions comme il nous a fait.
\VS{11}Alors trois mille hommes de Juda descendirent à la caverne du rocher d'Etam, et dirent à Samson : Ne sais-tu pas que les Philistins dominent sur nous ? Que nous as-tu donc fait ? Il leur répondit : Je leur ai fait comme ils m'ont fait.
\VS{12}Ils lui dirent : Nous sommes descendus pour te lier, afin de te livrer entre les mains des Philistins. Samson leur dit : Jurez-moi que vous ne me tuerez pas.
\VS{13}Ils lui répondirent, en disant : Non ; mais nous te lierons, afin de te livrer entre leurs mains, mais nous ne te tuerons pas. Ils le lièrent avec deux cordes neuves, et le firent monter hors du rocher.
\VS{14}Lorsqu'il entra à Léchi, les Philistins poussèrent des cris de joie à sa rencontre. Alors l'Esprit de Yahweh le saisit. Les cordes qui étaient sur ses bras devinrent comme du lin brûlé par le feu, et les liens tombèrent de ses mains.
\VS{15}Il trouva une mâchoire d'âne fraîche, il étendit sa main, la prit, et il en tua mille hommes.
\VS{16}Puis Samson dit : Avec une mâchoire d'âne, un monceau, deux monceaux ; avec une mâchoire d'âne, j'ai tué mille hommes.
\VS{17}Quand il cessa de parler, il jeta de sa main la mâchoire. On appela ce lieu Ramath-Léchi.
\VS{18}Il eut extrêmement soif, et invoqua Yahweh en disant : Tu as accordé par la main de ton serviteur cette grande délivrance ; et maintenant mourrais-je de soif, et tomberais-je entre les mains des incirconcis\FTNT{1 S. 17:26.} ?
\VS{19}Alors Dieu fendit la cavité du rocher qui est à Léchi, et il en sortit de l'eau. Samson but, l'Esprit lui revint, et il reprit vie. C'est pourquoi on a appelé cette source du nom d'En-Hakkoré ; elle existe encore aujourd'hui  à Léchi.
\VS{20}Samson fut juge en Israël, au temps des Philistins, pendant vingt ans\FTNT{Jg. 16:31.}.
\Chap{16}
\TextTitle{Faiblesse de Samson}
\VerseOne{}Or Samson s'en alla à Gaza ; il y vit une femme prostituée, et il entra chez elle.
\VS{2}On dit aux gens de Gaza : Samson est venu ici. Ils l'entourèrent, et se tinrent en embuscade toute la nuit à la porte de la ville. Ils restèrent tranquilles toute la nuit, en disant : Au point du jour, nous le tuerons.
\VS{3}Samson demeura couché jusqu'à minuit. Au milieu de la nuit, il se leva ; et il saisit les battants des portes de la ville et les deux poteaux, les retira avec la barre, les mit sur ses épaules, et les porta sur le sommet de la montagne qui est en face d'Hébron.
\VS{4}Après cela, il aima une femme dans la vallée de Sorek. Elle se nommait Delila.
\VS{5}Les princes des Philistins montèrent vers elle, et lui dirent : Séduis-le, jusqu'à ce que tu saches de lui en quoi consiste sa grande force, et comment pourrions-nous le vaincre ; afin que nous le lions pour l'abattre, et nous te donnerons chacun mille cent sicles d'argent.
\VS{6}Delila dit à Samson : Dis-moi, je te prie, en quoi consiste ta grande force, et avec quoi il faudrait te lier pour t'abattre.
\VS{7}Samson lui répondit : Si on me liait avec sept cordes fraîches, qui ne soient pas encore sèches, je deviendrais faible et je serais comme un autre homme.
\VS{8}Les princes des Philistins emmenèrent à Delila sept cordes fraîches, qui n'étaient pas encore sèches. Et elle le lia.
\VS{9}Or il y avait chez elle, dans une chambre, des gens qui se tenaient en embuscade. Elle lui dit : Les Philistins sont sur toi, Samson ! Alors il rompit les cordes comme se romprait un cordon d'étoupe dès qu'il sent le feu. Et l'on ne connut pas d'où lui venait sa force.
\VS{10}Puis Delila dit à Samson : Voici, tu t'es moqué de moi, car tu m'as dit des mensonges. Je te prie, déclare-moi maintenant avec quoi il faut te lier.
\VS{11}Il lui répondit : Si on me liait avec des cordes neuves, dont on ne se serait jamais servi pour un quelconque ouvrage, je deviendrais faible, et je serais comme un autre homme.
\VS{12}Delila prit des cordes neuves avec lesquelles elle le lia. Puis elle lui dit : Les Philistins sont sur toi, Samson ! Or il y avait des gens en embuscade dans une chambre. Et il rompit les cordes comme un fil.
\VS{13}Puis Delila dit à Samson : Tu t'es moqué de moi, jusqu'ici tu m'as dit des mensonges. Déclare-moi avec quoi il faut te lier. Il lui dit : Tu n'as qu'à tresser les sept tresses de ma tête avec la chaîne du tissu.
\VS{14}Et elle les fixa par la cheville. Puis elle dit : Les Philistins sont sur toi, Samson ! Alors il se réveilla de son sommeil, et il retira la chaîne du tissu.
\TextTitle{Samson révèle son secret}
\VS{15}Alors elle lui dit : Comment peux-tu dire : Je t'aime ! Puisque ton cœur n'est pas avec moi ? Tu t'es moqué de moi par trois fois, et tu ne m'as pas déclaré en quoi consiste ta grande force.
\VS{16}Comme elle le tourmentait et l'importunait tous les jours par ses paroles, son âme en fut affligée jusqu'à la mort,
\VS{17}alors il lui ouvrit tout son cœur, et lui dit : Le rasoir n'est jamais passé sur ma tête, car je suis Naziréen de Dieu dès le ventre de ma mère. Si j'étais rasé, ma force partirait, je me trouverais faible, et je serais comme tous les autres hommes.
\VS{18}Delila, voyant qu'il lui avait ouvert tout son cœur, envoya appeler les princes des Philistins, et leur fit dire : Montez cette fois, car il m'a ouvert tout son cœur. Les princes des Philistins montèrent vers elle, et emmenèrent l'argent dans leurs mains.
\VS{19}Elle l'endormit sur ses genoux. Et ayant appelé un homme, elle rasa les sept tresses de la tête de Samson, et commença à le dompter. Sa force partit.
\VS{20}Alors elle dit : Les Philistins sont sur toi, Samson ! Et il se réveilla de son sommeil, et dit : Je m'en sortirai comme les autres fois, et je me dégagerai. Mais il ne savait pas que Yahweh s'était retiré de lui\FTNT{L’immoralité sexuelle de Samson et sa désobéissance à Yahweh, dues à son manque de caractère, ont ruiné à jamais son ministère et compromis l’avenir du peuple d’Israël qu’il devait diriger (Jg. 16). Cet homme avait reçu un appel puissant dès le sein de sa mère, mais il ne vivait pas dans la crainte de Dieu. Le manque de discernement de Samson lui coûta ainsi toutes les grâces que le Seigneur lui avait accordées : La sainteté symbolisée par ses sept tresses, la force ou l’onction,  la vision, la liberté (Jg. 16:21).}.
\VS{21}Les Philistins donc le saisirent, et lui crevèrent les yeux ; ils le descendirent à Gaza, et le lièrent de deux chaînes d'airain. Il tournait la meule dans la prison\FTNT{2 S. 3:34.}.
\VS{22}Les cheveux de sa tête commencèrent à repousser, depuis qu'il avait été rasé.
\TextTitle{Samson achève le jugement des Philistins}
\VS{23}Or les princes des Philistins s'assemblèrent pour offrir un grand sacrifice à Dagon, leur dieu, et pour se réjouir. Ils disaient : Notre dieu a livré en nos mains Samson, notre ennemi.
\VS{24}Et quand le peuple le vit, il loua son dieu, en disant : Notre dieu a livré entre nos mains notre ennemi, celui qui ravageait notre pays, et qui multipliait nos morts.
\VS{25}Comme ils avaient le cœur joyeux, ils dirent : Qu'on appelle Samson, afin qu'il nous fasse rire ! Ils appelèrent Samson et le tirèrent de la prison ; et il joua devant eux. Ils le firent tenir entre les colonnes.
\VS{26}Alors Samson dit au garçon qui le tenait par la main : Laisse-moi afin que je puisse toucher les colonnes sur lesquelles repose la maison pour que je m'y appuie.
\VS{27}Or la maison était remplie d'hommes et de femmes ; tous les princes des Philistins y étaient, et il y avait même sur le toit près de trois mille personnes, hommes et femmes, qui regardaient Samson jouer.
\VS{28}Alors Samson invoqua Yahweh, et dit : Seigneur Yahweh ! Je te prie, souviens-toi de moi ; ô Dieu ! Fortifie-moi seulement cette fois, et que par un coup je me venge des Philistins pour mes deux yeux\FTNT{Hé. 11:32.} !
\VS{29}Samson embrassa les deux colonnes du milieu sur lesquelles reposait la maison, et il s'appuya contre elles ; l'une à sa droite, et l'autre à sa gauche.
\VS{30}Et il dit : Que mon âme meure avec les Philistins ! Il se pencha donc de toute sa force, et la maison tomba sur les princes et sur tout le peuple qui y était. Et il fit mourir beaucoup plus de gens à sa mort, qu'il n'en avait fait mourir pendant sa vie.
\VS{31}Ensuite ses frères et toute la maison de son père descendirent, et le transportèrent. Lorsqu'ils furent montés, ils l'enterrèrent entre Tsorea et Eschthaol dans le sépulcre de Manoach, son père. Il avait été juge en Israël pendant vingt ans\FTNT{Jg. 13:2.}.
\Chap{17}
\TextTitle{Confusion en Israël}
\VerseOne{}Il y avait un homme de la montagne d'Ephraïm, nommé Mica.
\VS{2}Il dit à sa mère : Les mille cent sicles d'argent qu'on t'a pris, et pour lesquels tu as fait des imprécations même à mes oreilles, voici, j'ai cet argent, c'est moi qui l'avais pris. Alors sa mère dit : Béni soit mon fils par Yahweh !
\VS{3}Et il rendit à sa mère les mille cent sicles d'argent ; sa mère dit : Je consacre de ma main cet argent à Yahweh, afin d'en faire pour mon fils une image taillée, et une image en métal fondu ; et c'est ainsi que je te le rendrai.
\VS{4}Et il rendit l'argent à sa mère. Elle prit deux cents sicles d'argent et les donna au fondeur, qui en fit une image taillée, et une image en métal fondu.  On les plaça dans la maison de Mica.
\VS{5}Ainsi cet homme, savoir Mica, avait une maison de Dieu ; il fit un éphod et des téraphim, et il consacra par sa main l'un de ses fils, qui lui servit de sacrificateur.
\VS{6}En ce temps-là, il n'y avait pas de roi en Israël. Chacun faisait ce qui lui semblait être droit à ses yeux\FTNT{Jg. 18:1}.
\VS{7}Or il y avait un jeune homme de Bethléhem de Juda, de la famille de la tribu de Juda ; il était Lévite, et il séjournait là.
\VS{8}Cet homme partit de la ville de Bethléhem de Juda, pour trouver une demeure qui lui convienne.  En chemin, il entra dans la montagne d'Ephraïm jusqu'à la maison de Mica.
\VS{9}Mica lui dit : D'où viens-tu ? Il lui répondit : Je suis Lévite, de Bethléhem de Juda, et je voyage pour trouver une demeure qui me convienne.
\VS{10}Mica lui dit : Demeure avec moi ; tu me serviras de père et de sacrificateur, et je te donnerai dix sicles d'argent par an, les vêtements d'ordre dont tu auras besoin, et ton entretien. Et le Lévite vint\FTNT{Jg. 18:19.}.
\VS{11}Ainsi le Lévite convint de demeurer avec cet homme, qui regarda le jeune homme comme l'un de ses fils.
\VS{12}Mica consacra\FTNT{"Consacrer" signifie litteralement "remplir la main".} le Lévite, qui lui servit de sacrificateur, et qui demeura dans sa maison.
\VS{13}Mica dit : Maintenant je sais que Yahweh me fera du bien, parce que j'ai un Lévite pour sacrificateur.
\Chap{18}
\TextTitle{Dan recherche un territoire}
\VerseOne{}En ce temps-là, il n'y avait pas de roi en Israël ; et en ce même temps la tribu des Danites cherchait un héritage afin de pouvoir s'établir, car jusqu'à ce jour il ne lui était pas échu d'héritage au milieu des tribus d'Israël\FTNT{Jg. 17:6.}.
\VS{2}C'est pourquoi les fils de Dan envoyèrent de leur famille cinq hommes vaillants, de Tsorea et d'Eschthaol, pour explorer le pays et l'examiner. Ils leur dirent : Allez examiner le pays. Ils entrèrent dans la montagne d'Ephraïm jusqu'à la maison de Mica, et ils y passèrent la nuit.
\VS{3}Comme ils étaient près de la maison de Mica, ils reconnurent la voix du jeune homme Lévite et lui dirent : Qui t'a amené ici ? Qu'y fais-tu ? Que fais-tu ici ?
\VS{4}Il leur répondit : Mica fait pour moi telle et telle chose, il me donne un salaire, et je lui sers de sacrificateur.
\VS{5}Ils lui dirent : Nous te prions consulte Dieu, afin que nous sachions si le voyage que nous entreprenons prospérera.
\VS{6}Et le sacrificateur leur répondit : Allez en paix ; Yahweh a sous ses yeux le voyage que vous mènerez.
\VS{7}Ces cinq hommes s'en allèrent, et entrèrent à Laïs. Ils virent le peuple qui y habitait en sécurité selon les coutumes des Sidoniens, tranquille et en confiance ; il n'y avait personne au pays qui les humiliait en quelque chose en dominant sur eux ; ils étaient éloignés des Sidoniens, et ils n'avaient aucune affaire avec d'autres hommes.
\VS{8}Puis ils vinrent auprès de leurs frères à Tsorea et à Eschthaol, et leurs frères leur dirent : Quelle nouvelle rapportez-vous ?
\VS{9}Et ils répondirent : Allons ! Montons contre eux ; car nous avons vu le pays, et nous l'avons trouvé très bon. Quoi ! Vous restez sans rien faire ? Ne soyez pas paresseux pour aller posséder ce pays.
\VS{10}Quand vous y entrerez, vous irez vers un peuple en sécurité. Le pays est vaste, Dieu l'a livré entre vos mains ; c'est un lieu où il ne manque rien de tout ce qui est sur la terre.
\VS{11}Il partit de Tsorea et d'Eschthaol, six cents hommes de la famille de Dan, munis de leurs armes de guerre.
\VS{12}Ils montèrent, et campèrent à Kirjath-Jearim en Juda ; c'est pourquoi on a appelé ce lieu qui est derrière Kirjath-Jearim jusqu'à ce jour, Machané-Dan.
\VS{13}Puis ils passèrent par la montagne d'Ephraïm, et ils entrèrent dans la maison de Mica.
\TextTitle{Campagnes de la tribu de Dan}
\VS{14}Alors les cinq hommes qui étaient allés explorer le pays de Laïs prirent la parole et dirent à leurs frères : Savez-vous qu'il y a dans ces maisons-là un éphod, des théraphim, une image taillée et une image en métal fondu ? Voyez maintenant ce que vous avez à faire.
\VS{15}Alors ils se détournèrent de ce lieu, et entrèrent dans la maison où était le jeune homme Lévite, dans la maison de Mica, et lui demandèrent comment il se portait.
\VS{16}Et les six cents hommes d'entre les fils de Dan, qui étaient munis de leurs armes de guerre, se tenaient à l'entrée de la porte.
\VS{17}Mais les cinq hommes, qui étaient allés explorer le pays, montèrent et entrèrent dans la maison ; ils prirent l'image taillée, l'éphod, les théraphim, et l'image en métal fondu, pendant que le sacrificateur était à l'entrée de la porte avec les six cents hommes munis de leurs armes de guerre.
\VS{18}Etant entrés dans la maison de Mica, ils prirent l'image taillée, l'éphod, les théraphim, et l'image en métal fondu. Le sacrificateur leur dit : Que faites-vous ?
\VS{19}Ils lui répondirent : Tais-toi, mets ta main sur ta bouche, et viens avec nous ; sois pour nous un père et un sacrificateur. Vaut-il mieux que tu serves de sacrificateur à la maison d'un homme seul, ou que tu serves de sacrificateur à une tribu et à une famille en Israël\FTNT{Jg. 17:10.} ?
\VS{20}Le sacrificateur eut de la joie dans son cœur ; il prit l'éphod, les théraphim, et l'image taillée, et vint au milieu du peuple.
\VS{21}Après quoi ils se retournèrent et marchèrent, en mettant devant eux les petits enfants, le bétail, et les bagages.
\VS{22}Comme ils étaient loin de la maison de Mica, les gens qui  habitaient les maisons voisines de celle de Mica furent assemblés à grand cri ; et poursuivirent les fils de Dan.
\VS{23}Et ils crièrent aux fils de Dan, qui se tournèrent de face et dirent à Mica : Qu’as-tu, que tu te sois ainsi écrié pour rassembler ces gens ?
\VS{24}Il répondit : Vous avez enlevé mes dieux que j'avais faits, vous avez pris le sacrificateur, et vous vous en êtes allés : Que me reste-t-il ? Comment pouvez-vous me dire : Qu'as-tu\FTNT{Ge. 31:30.} ?
\VS{25}Les fils de Dan lui dirent : Ne fais pas entendre ta voix après nous, de peur que des hommes exaspérés ne se jettent sur vous, et que vous n’y laissiez la vie, toi, et tous ceux de ta famille.
\VS{26}Les fils de Dan firent leur chemin. Mica, voyant qu'ils étaient plus forts que lui, s'en retourna et revint dans sa maison.
\VS{27}Ainsi ils prirent les choses que Mica avait faites, et le sacrificateur qu'il avait, et ils entrèrent à Laïs, vers un peuple tranquille et en sécurité ; ils les firent passer au fil de l'épée, et ils brûlèrent la ville.
\VS{28}Et il n'y eut personne qui la délivrât, car elle était éloignée de Sidon, et ses habitants n'avaient pas d'affaires avec les autres hommes : Elle était située dans la vallée qui appartenait au pays de Beth-Rehob. Les fils de Dan rebâtirent la ville, et y demeurèrent.
\VS{29}Ils appelèrent la ville Dan, selon le nom de Dan, leur père qui était né à Israël ; mais la ville s'appelait auparavant Laïs\FTNT{Jos. 19:47.}.
\VS{30}Et les fils de Dan dressèrent l'image taillée ; et Jonathan, fils de Guerschom, fils de Manassé, lui et ses fils, furent sacrificateurs pour la tribu des Danites, jusqu'au jour de la captivité du pays.
\VS{31}Ils y dressèrent donc l'image taillée que Mica avait faite, pendant tout le temps que la maison de Dieu fut à Silo.
\Chap{19}
\TextTitle{Dégradation morale}
\VerseOne{}Il arriva aussi en ce temps-là, où il n'y avait pas de roi en Israël, qu'un Lévite qui habitait aux côtés de la montagne d'Ephraïm, prit pour concubine une femme de Bethléhem de Juda\FTNT{Jg. 17:6 ; 21:25.}.
\VS{2}Mais sa concubine se prostitua chez lui, et elle s'en alla pour aller dans la maison de son père à Bethléhem de Juda, où elle resta pendant quatre mois.
\VS{3}Puis son mari se leva et alla après elle, pour parler à son cœur, et la ramener. Il avait avec lui son serviteur et deux ânes. Elle le fit entrer dans la maison de son père ; et quand le père de la jeune femme le vit, il s'approcha avec joie.
\VS{4}Son beau-père, le père de la jeune femme, le retint avec grande instance, de sorte qu'il demeura trois jours avec lui. Ils mangèrent et burent, et logèrent là.
\VS{5}Le quatrième jour, ils se levèrent de bon matin, et le Lévite se levait pour s'en aller. Mais le père de la jeune femme dit à son gendre : Fortifie ton cœur avec un morceau de pain, et vous partirez ensuite.
\VS{6}Ils s'assirent, et ils mangèrent et burent eux deux ensemble. Puis le père de la jeune femme dit au mari : Je te prie consens à passer encore ici cette nuit, et que ton cœur se réjouisse.
\VS{7}Le mari se levait pour s'en aller ; mais son beau-père le pressa tellement, qu'il s'en retourna, et y passa encore la nuit.
\VS{8}Le cinquième jour, il se leva de bon matin pour s'en aller. Alors le père de la jeune femme dit : Fortifie ton cœur ; et attendez le déclin du jour. Et ils mangèrent eux deux.
\VS{9}Puis le mari se levait pour s'en aller, avec sa concubine et son serviteur ; mais son beau-père, le père de la jeune femme, lui dit : Voici, maintenant le jour baisse, il se fait tard, je vous prie passez ici la nuit ; voici le jour est sur son déclin, passe ici la nuit, et que ton cœur se réjouisse ; demain matin vous vous mettrez en route, et tu t'en iras à ta tente.
\VS{10}Mais le mari ne voulut pas y passer la nuit, il se leva, et s'en alla.  Il vint jusque vis-à-vis de Jébus, qui est Jérusalem, avec les deux ânes bâtés et sa concubine.
\VS{11}Comme ils étaient près de Jébus, le jour avait beaucoup baissé. Le serviteur dit à son maître : Allons, détournons-nous vers cette ville des Jébusiens, afin que nous y passions la nuit.
\VS{12}Son maître lui répondit : Nous ne nous détournerons pas vers une ville d'étrangers, où il n'y a pas d'enfants d'Israël, mais nous passerons par Guibea.
\VS{13}Il dit aussi à son serviteur : Allons, approchons-nous de l'un de ces lieux, Guibea ou Rama, et passons-y la nuit.
\VS{14}Ils continuèrent à marcher, et le soleil se coucha quand ils furent près de Guibea, qui appartient à Benjamin.
\VS{15}Alors ils se détournèrent vers Guibea, et y entrèrent pour passer la nuit. Le Lévite entra, et il s'assit sur la place de la ville. Il n'y eut aucun homme qui les reçut dans sa maison afin qu'ils y passent la nuit.
\VS{16}Et voici, sur le soir, un vieil homme venait de travailler dans les champs ; cet homme était de la montagne d'Ephraïm, il séjournait à Guibea, et les gens du lieu étaient Benjamites.
\VS{17}Et levant ses yeux, il vit le voyageur sur la place de la ville. Le vieil homme lui dit : Où vas-tu, et d'où viens-tu ?
\VS{18}Il lui répondit : Nous passons de Bethléhem de Juda vers les côtés de la montagne d'Ephraïm, d'où je suis. J'étais allé jusqu'à Bethléhem de Juda, mais maintenant je m'en vais à la maison de Yahweh. Mais il n'y a aucun homme qui me reçoive dans sa maison.
\VS{19}Nous avons pourtant de la paille et du fourrage pour nos ânes ; du pain et du vin pour moi,  pour ta servante, et pour le garçon qui est avec tes serviteurs. Nous n'avons besoin d'aucune chose.
\VS{20}Le vieil homme dit : Pourvu que la paix soit ! Quoi qu'il en soit, je me charge de tous tes besoins, je te prie seulement de ne pas passer la nuit sur la place.
\VS{21}Alors il les fit entrer dans sa maison, et il donna du fourrage aux ânes. Les voyageurs se lavèrent les pieds ; puis ils mangèrent et burent\FTNT{Ge. 43:24.}.
\VS{22}Comme ils se réjouissaient, voici, les hommes de la ville, fils d'hommes pervers, environnèrent la maison, frappèrent à la porte, et dirent au vieil homme, maître de la maison : Fais sortir l'homme qui est entré dans ta maison, afin que nous le connaissions\FTNT{Jg. 20:13 ; Os. 9:9 ; 10:9 ; Ge. 19:4.}.
\VS{23}Mais cet homme, savoir le maître de la maison, sortit vers eux, et leur dit : Non, mes frères, ne lui faites pas de mal, je vous prie ; puisque cet homme est entré dans ma maison, ne faites pas une telle infamie.
\VS{24}Voici, j'ai une fille vierge, et cet homme a une concubine ; je vous les amènerai dehors ; vous les déshonorerez, et vous ferez d'elles comme il semblera bon à vos yeux. Mais ne faites pas cette action infâme à l'égard de cet homme.
\VS{25}Mais ces gens ne voulurent pas l'écouter. C'est pourquoi l'homme saisit sa concubine, et la leur amena dehors. Ils la connurent, et abusèrent d'elle toute la nuit jusqu'au matin ; puis ils la renvoyèrent au lever de l'aurore.
\VS{26}Vers le matin, cette femme alla tomber à la porte de la maison de l'homme où était son mari, et elle y demeura jusqu'au jour.
\VS{27}Et le matin, son mari se leva, et ayant ouvert la porte de la maison, il sortit pour poursuivre son chemin. Mais voici, la femme concubine était tombée à la porte de la maison, et avait les mains sur le seuil.
\VS{28}Il lui dit : Lève-toi, et allons-nous-en. Mais elle ne répondit pas. Alors il l'emmena sur un âne, se mit en chemin, et s'en alla dans sa demeure.
\VS{29}En entrant en sa maison, il prit un couteau, et saisissant sa concubine, il la coupa avec ses os en douze morceaux, qu'il envoya dans tout le territoire d'Israël.
\VS{30}Et il arriva que tous ceux qui virent cela dirent : Une telle chose n'a été faite ni vue depuis le jour où les enfants d'Israël sont montés hors du pays d'Egypte, jusqu'à ce jour ; prenez la chose à cœur, consultez-vous, et parlez !
\Chap{20}
\TextTitle{Israël devant Yahweh à Mitspa}
\VerseOne{}Alors tous les fils d'Israël sortirent, et toute l'assemblée se réunit comme un seul homme, depuis Dan jusqu'à Beer-Schéba et jusqu'au pays de Galaad, devant Yahweh, à Mitspa.
\VS{2}Les chefs de tout le peuple, toutes les tribus d'Israël, se présentèrent à l'assemblée du peuple de Dieu, au nombre de quatre cent mille hommes de pied, tirant l'épée.
\VS{3}Les fils de Benjamin entendirent que les fils d'Israël étaient montés à Mitspa. Les fils d'Israël dirent : Parlez, comment ce mal est arrivé ?
\VS{4}Alors le Lévite, mari de la femme tuée, répondit, et dit : J'étais venu à Guibea de Benjamin, avec ma concubine, pour y passer la nuit.
\VS{5}Les seigneurs de Guibea se sont élevés contre moi, et ont encerclé de nuit la maison où j'étais. Ils avaient l'intention de me tuer, et ils ont tellement violé ma concubine qu'elle en est morte.
\VS{6}C'est pourquoi j'ai saisi ma concubine, je l'ai coupée en morceaux, et je les ai envoyés dans tout le territoire de l'héritage d'Israël ; car ils ont fait un crime et une infamie en Israël.
\VS{7}Vous voici tous, fils d'Israël ; consultez-vous sur la question, et prenez ici une décision !
\VS{8}Tout le peuple se leva comme un seul homme, et ils dirent : Aucun homme n'ira dans sa tente, et aucun homme ne se retirera dans sa maison.
\VS{9}Et maintenant voici ce que nous ferons à Guibea : Nous marcherons contre elle d'après le sort.
\VS{10}Nous prendrons dans toutes les tribus d'Israël dix hommes sur cent, cent sur mille, et mille sur dix mille ; nous prendrons des provisions pour le peuple, afin qu'en entrant à Guibea de Benjamin, on leur fasse selon toute l'infamie qu'elle a commise en Israël.
\VS{11}Ainsi tous les hommes d'Israël s'assemblèrent contre la ville, unis comme un seul homme.
\VS{12}Alors les tribus d'Israël envoyèrent des hommes vers la maison de Benjamin, pour dire : Quelle méchanceté a été faite parmi vous ?
\VS{13}Maintenant donc livrez-nous les fils des hommes pervers qui sont à Guibea, afin que nous les fassions mourir et que nous ôtions le mal du milieu d'Israël. Mais les fils de Benjamin ne voulurent pas écouter la voix de leurs frères, les enfants d'Israël.
\VS{14}Et les fils de Benjamin s'assemblèrent à Guibea pour sortir en guerre contre les fils d'Israël.
\VS{15}En ce jour-là, on fit le dénombrement des fils de Benjamin qui étaient dans ces villes, et il se trouva vingt-six mille hommes, tirant l'épée, sans compter les habitants de Guibea formant sept cents hommes d'élite.
\VS{16}De tout ce peuple, il y avait sept cents hommes d'élite qui ne se servaient pas de la main droite ; tous tirant la pierre avec la fronde,  à un cheveu près,  ils n'y manquaient pas.
\VS{17}On fit aussi le dénombrement des hommes d'Israël, excepté ceux de Benjamin, et l'on en trouva quatre cent mille hommes tirant l'épée, tous gens de guerre.
\TextTitle{Coalition pour monter contre Benjamin}
\VS{18}Et les fils d'Israël se levèrent, montèrent vers Dieu à Béthel pour le consulter, en disant : Qui d'entre nous montera le premier pour faire la guerre aux fils de Benjamin ? Yahweh répondit : Juda montera le premier.
\VS{19}Puis les fils d'Israël se levèrent de bon matin, et campèrent près de Guibea.
\VS{20}Et les hommes d'Israël sortirent pour combattre ceux de Benjamin, et se rangèrent en bataille près de Guibea.
\VS{21}Les fils de Benjamin sortirent de Guibea, et ils tuèrent ce jour-là vingt-deux mille hommes d'Israël.
\VS{22}Toutefois le peuple, les hommes d'Israël, se fortifièrent et se rangèrent de nouveau en bataille au lieu où ils s'étaient rangés le premier jour.
\VS{23}Et les fils d'Israël montèrent, et ils pleurèrent devant Yahweh jusqu'au soir ; ils consultèrent Yahweh en disant : M'approcherai-je encore pour combattre contre les fils de Benjamin, mon frère ? Yahweh dit : Montez contre lui.
\VS{24}Le second jour, les fils d'Israël s'approchèrent des fils de Benjamin.
\VS{25}Ce même jour, les Benjamites sortirent de Guibea à leur rencontre, et ils tuèrent encore dix-huit mille hommes des fils d'Israël, tous tirant l'épée.
\VS{26}Alors tous les fils d'Israël et tout le peuple montèrent et vinrent vers Dieu à Béthel ; ils pleurèrent, et restèrent là devant Yahweh. Ce jour-là ils jeûnèrent jusqu'au soir, et ils offrirent des holocaustes, et des sacrifices de paix devant Yahweh.
\VS{27}Ensuite les fils d'Israël consultèrent Yahweh, c'était là que se trouvait l'arche de l'alliance de Dieu ;
\VS{28}et Phinées, fils d'Eléazar, fils d'Aaron, se tenait devant Yahweh en ce temps-là en disant : Sortirai-je encore en guerre contre les fils de Benjamin, mon frère, ou dois-je m'en abstenir ? Yahweh répondit : Montez, car demain je les livrerai entre vos mains.
\VS{29}Alors Israël mit une embuscade autour de Guibea.
\VS{30}Le troisième jour, les fils d'Israël montèrent contre les fils de Benjamin, et ils se rangèrent en bataille contre Guibea, comme les autres fois.
\VS{31}Alors les fils de Benjamin sortirent à la rencontre du peuple, et ils furent attirés hors de la ville. Ils commencèrent à frapper à mort quelques-uns du peuple comme les autres fois, environ trente hommes d'Israël, sur les routes dont l'une monte à Béthel et l'autre à Guibea, par les champs.
\VS{32}Les fils de Benjamin disaient : Ils tombent battus devant nous, comme la première fois ! Mais les fils d'Israël disaient : Fuyons, et attirons-les hors de la ville dans les chemins.
\VS{33}Tous les hommes d'Israël se levant de leur lieu, se rangèrent à Baal-Thamar ; et l'embuscade sortit du lieu où ils étaient, de Maaré-Guibea.
\VS{34}Dix mille hommes choisis sur tout Israël vinrent contre Guibea. La bataille fut rude, et les Benjamites ne surent pas que le mal les atteindrait.
\VS{35}Yahweh battit Benjamin devant Israël, et les fils d'Israël tuèrent ce jour-là vingt-cinq mille cent hommes de Benjamin, tous tirant l'épée.
\VS{36}Les fils de Benjamin regardaient comme battus les hommes d'Israël, qui cédaient du terrain à Benjamin et se reposaient sur l'embuscade qu'ils avaient mise près de Guibea.
\VS{37}Ceux qui étaient en embuscade se jetèrent promptement sur Guibea, ils se portèrent en avant et frappèrent toute la ville au tranchant de l'épée.
\VS{38}Et le signal convenu entre les hommes d’Israël et l’embuscade était qu’ils fassent monter beaucoup de fumée de la ville.
\VS{39}Les hommes d’Israël avaient donc tourné le dos dans la bataille, et les Benjamites avaient commencé de frapper et de blesser à mort environ trente hommes de ceux d’Israël ; et ils disaient : Certainement ils tombent devant nous comme à la première bataille !
\VS{40}Mais quand l'épaisse colonne de fumée commençait à monter de la ville, les Benjamites se tournèrent ; et voici, derrière eux toute la ville disparaissait montant en feu vers le ciel.
\VS{41}Les hommes d'Israël tournèrent le visage ; et ceux de Benjamin furent épouvantés en voyant le mal qui allait les  atteindre.
\VS{42}Ils tournèrent le dos devant les hommes d'Israël par le chemin du désert. Mais les assaillants s'attachaient à leurs pas, et ils détruisirent ceux qui étaient sortis des villes.
\VS{43}Ils environnèrent Benjamin, le poursuivirent, l'écrasèrent dès qu'il voulut se reposer jusqu'en face de Guibea, du côté du soleil levant.
\VS{44}Il tomba dix-huit mille hommes de Benjamin, tous des vaillants hommes.
\VS{45}Et parmi ceux de Benjamin qui tournèrent le dos pour s'enfuir vers le désert au rocher de Rimmon, les hommes d'Israël en firent périr cinq mille hommes sur les routes ; et les poursuivant de près jusqu'à Guideom, ils frappèrent deux mille hommes.
\TextTitle{La tribu de Benjamin décimée}
\VS{46}En ce jour-là, le nombre de Benjamites qui tombèrent fut de vingt-cinq mille hommes tirant l'épée, et tous étaient des vaillants hommes.
\VS{47}Et il y eut six cents hommes de ceux qui avaient tourné le dos, qui s'échappèrent vers le désert au rocher de Rimmon, et qui demeurèrent au rocher de Rimmon pendant quatre mois.
\VS{48}Les hommes d'Israël retournèrent vers les fils de Benjamin, et ils les frappèrent du tranchant de l'épée, depuis les hommes des villes jusqu'aux bêtes, et tout ce qui s'y trouva. Ils brûlèrent toutes les villes qu'ils trouvaient.
\Chap{21}
\TextTitle{Deuil national}
\VerseOne{}Les hommes d'Israël avaient juré à Mitspa, en disant : Aucun homme ne donnera sa fille pour femme à un Benjamite.
\VS{2}Puis le peuple vint vers Dieu à Béthel, jusqu'au soir. Ils élevèrent leurs voix, et pleurèrent grandement,
\VS{3}Et ils dirent : Ô Yahweh, Dieu d'Israël, pourquoi est-il arrivé en Israël qu'une tribu d'Israël ait été aujourd'hui punie ?
\VS{4}Le lendemain, le peuple se leva de bon matin ; ils bâtirent là un autel, et ils offrirent des holocaustes et des sacrifices d'offrande de paix.
\VS{5}Alors les fils d'Israël dirent : Quel est celui d'entre toutes les tribus d'Israël qui n'est pas monté à l'assemblée vers Yahweh ? Car on avait fait un grand serment contre tout homme qui ne monterait pas vers Yahweh à Mitspa, en disant : Il sera puni de mort.
\VS{6}Les fils d'Israël se repentaient de ce qui était arrivé à Benjamin, leur frère, et ils disaient : Aujourd'hui une tribu a été retranchée d'Israël.
\VS{7}Comment ferons-nous pour donner des femmes à ceux qui ont survécu, puisque nous avons juré par Yahweh que nous ne leur donnerions pas nos filles pour femmes ?
\TextTitle{Avenir de la tribu de Benjamin}
\VS{8}Ils dirent donc : Y a-t-il quelqu'un d'entre les tribus d'Israël qui ne soit pas monté vers Yahweh à Mitspa ? Et voici, aucun homme de Jabès en Galaad n'était venu au camp, à l'assemblée.
\VS{9}Quand on fit le dénombrement du peuple, il n'y avait aucun des hommes habitant à Jabès en Galaad.
\VS{10}C'est pourquoi l'assemblée envoya contre eux douze mille hommes des fils vaillants, en leur donnant cet ordre : Allez, et frappez du tranchant de l'épée les habitants de Jabès en Galaad, tant les femmes que les enfants.
\VS{11}Voici les choses que vous ferez : Vous détruirez par le moyen de l'interdit tout mâle et toute femme qui a connu la couche d'un homme.
\VS{12}Ils trouvèrent parmi les habitants de Jabès en Galaad quatre cents filles vierges, qui n'avaient pas connu d'homme en couchant avec lui, et ils les amenèrent au camp de Silo, qui est sur la terre de Canaan.
\VS{13}Alors toute l'assemblée envoya parler aux fils de Benjamin qui étaient au rocher de Rimmon,  pour leur proclamer la paix.
\VS{14}En ce temps-là, les Benjamites revinrent, et on leur donna pour femmes celles qui avaient été conservées en vie d'entre les femmes de Jabès en Galaad. Mais ils n’en trouvèrent pas assez pour eux.
\VS{15}Le peuple se repentit de ce qui avait été fait à Benjamin, car Yahweh avait fait une brèche dans les tribus d'Israël.
\VS{16}Les anciens de l'assemblée dirent : Comment ferons-nous pour donner des femmes à ceux qui restent, car les femmes de Benjamin ont été détruites ?
\VS{17}Et ils dirent : Que ceux qui sont réchappés de Benjamin possèdent leur héritage, afin qu'une tribu d'Israël ne soit pas effacée.
\VS{18}Cependant, nous ne pouvons pas leur donner des femmes d'entre nos filles, car les fils d'Israël ont juré, en disant : Maudit soit celui qui donnera une femme à un Benjamite !
\VS{19}Et ils dirent : Voici, il y a chaque année une fête de Yahweh à Silo, qui est au nord de Béthel, à l'orient qui monte à Béthel, à Sichem, et au midi de Lebona.
\VS{20}Puis ils ordonnèrent aux fils de Benjamin : Allez, et placez-vous en embuscade dans les vignes.
\VS{21}Vous verrez, et voici, lorsque les filles de Silo sortiront pour danser, alors vous sortirez des vignes, vous enlèverez chacun une des filles de Silo pour en faire votre femme, et vous vous en irez dans le pays de Benjamin.
\VS{22}Si leurs pères ou leurs frères viennent se plaindre auprès de nous, nous leur dirons : Accordez-nous cette faveur, puisque nous n'avons pas pris de femmes pour chaque homme dans cette guerre.  Ce n'est pas vous qui les leur avez données ; sinon vous en seriez coupables en ce temps.
\VS{23}Les fils de Benjamin firent ainsi ; ils prirent des femmes selon leur nombre, parmi les danseuses qu'ils saisirent, puis ils s'en allèrent et retournèrent dans leur héritage ; ils rebâtirent les villes, et y habitèrent.
\VS{24}Ainsi en ce temps-là chacun des enfants d’Israël s’en alla de là dans sa tribu, et dans sa famille, et ils se retirèrent de là chacun dans son héritage.
\VS{25}En ce temps-là, il n'y avait pas de roi en Israël. L'homme faisait ce qui lui semblait être droit à ses yeux.
\PPE{}
\end{multicols}

\clearpage\ShortTitle{1 S.}\BookTitle{1 Samuel}\BFont
\noindent\hrulefill
{\footnotesize
\textit{
\bigskip
{\centering{}
\\Auteur~: Inconnu
\\(Heb.~: Shemuw'el)
\\Signification~: Entendu, exaucé de Dieu
\\Thème~: Histoire de Samuel, Saül et David
\\Date de rédaction~: 10\up{ème} siècle av. J.-C.\\}
}
\textit{
\\Samuel était le fils d'Elkana, de la montagne d'Ephraïm. Anne, sa mère, avait longtemps désiré un enfant. Elle fit donc une alliance avec Dieu en lui promettant de lui consacrer son premier fils s'il la rendait féconde. Ainsi, dès son plus jeune âge, Samuel fut amené à la maison de Dieu où il grandit aux côtés d'Eli, le prêtre. A la mort de ce dernier, Samuel exerça les fonctions de juge, prêtre et prophète. C'est en son temps qu'Israël exprima le désir d'avoir un roi, marquant ainsi la fin de l'ère des juges et le début de la monarchie en Israël.
\\Ce livre relate l'histoire de Saül, premier roi d'Israël, à qui Yahweh accorda de puissantes victoires notamment sur les Philistins, grands ennemis du peuple de Dieu. Mais très vite, Saül s'écarta de la volonté de Dieu, aussi Yahweh le disqualifia et choisit pour lui succéder au trône un homme de la tribu de Juda~: David, fils d'Isaï. Son accession à la royauté ne fut pas immédiate. David dut faire preuve de patience, de courage et de foi en son Dieu au milieu de nombreuses persécutions. L'expérience des deux premiers rois d'Israël est une exhortation à l'obéissance à Dieu.\bigskip
}
}
\par\nobreak\noindent\hrulefill
\begin{multicols}{2}
\Chap{1}
\TextTitle{Stérilité d'Anne}
\VerseOne{}Il y avait un homme de Ramathaïm-Tsophim, de la montagne d'Ephraïm, nommé Elkana, fils de Jeroham, fils d'Elihu, fils de Thohu, fils de Tsuph, Ephratien.
\VS{2}Il avait deux femmes, dont l'une s'appelait Anne, et l'autre Peninna. Peninna avait des enfants, mais Anne n'en avait pas.
\VS{3}Or cet homme-là montait tous les ans, de sa ville à Silo\FTNT{Jos. 18:1.}, pour adorer Yahweh des armées, et lui offrir des sacrifices. Là étaient les deux fils d'Eli, Hophni et Phinées, prêtres de Yahweh.
\VS{4}Le jour où Elkana offrait son sacrifice, il donnait des portions à Peninna, sa femme, à tous les fils et à toutes les filles qu'il avait d'elle.
\VS{5}Mais il donnait à Anne une portion double~; car il aimait Anne, mais Yahweh avait fermé sa matrice\FTNT{Dieu est celui qui ferme et ouvre les portes des bénédictions.}.
\VS{6}Sa rivale lui portait envie et la provoquait fort aigrement afin de l'irriter, car Yahweh avait fermé sa matrice.
\VS{7}Et Elkana faisait donc ainsi tous les ans. Mais quand Anne montait à la maison de Yahweh, Peninna la provoquait de la même manière, et Anne pleurait et ne mangeait pas.
\VS{8}Elkana, son mari, lui disait~: Anne, pourquoi pleures-tu et pourquoi ne manges-tu pas~? Pourquoi ton cœur est-il triste~? Est-ce que je ne vaux pas pour toi mieux que dix fils~?
\TextTitle{Prière et vœu d'Anne à Yahweh}
\VS{9}Anne se leva après avoir mangé et bu à Silo. Et le prêtre Eli était assis sur un siège, près de l'un des poteaux du temple de Yahweh.
\VS{10}Elle donc, ayant le cœur rempli d'amertume, pria Yahweh en pleurant abondamment.
\VS{11}Et elle fit un vœu, en disant~: Yahweh des armées~! Si tu regardes attentivement l'affliction de ta servante, et si tu te souviens de moi, et n'oublies pas ta servante, et que tu donnes à ta servante un enfant mâle, je le donnerai à Yahweh pour tous les jours de sa vie, et aucun rasoir ne passera sur sa tête.
\VS{12}Il arriva, comme elle continuait à prier devant Yahweh, qu'Eli observait sa bouche.
\VS{13}Or Anne parlait dans son cœur, elle ne faisait que remuer ses lèvres et on n'entendait pas sa voix. C'est pourquoi Eli estima qu'elle était ivre,
\VS{14}et Eli lui dit~: Jusqu'à quand seras-tu ivre~? Eloigne-toi du vin.
\VS{15}Mais Anne répondit, et dit~: Je ne suis pas ivre, mon seigneur, je suis une femme affligée en son esprit, je n'ai bu ni vin ni boisson forte~; mais je répandais mon âme devant Yahweh.
\VS{16}Ne met pas ta servante au rang d'une fille de Bélial\FTNT{Voir commentaire en 1 S. 2:12.}, car c'est l'excès de ma douleur et de mon affliction qui m'a fait parler jusqu'à présent.
\VS{17}Alors Eli répondit, et dit~: Va en paix, et que le Dieu d'Israël veuille t'accorder la demande que tu lui as faite~!
\VS{18}Et elle dit~: Que ta servante trouve grâce à tes yeux~! Puis cette femme poursuivit son voyage. Elle mangea, et son visage ne fut plus le même.
\TextTitle{Naissance de Samuel}
\VS{19}Après cela, ils se levèrent de bon matin, et se prosternèrent devant Yahweh, puis ils s'en retournèrent et revinrent dans leur maison à Rama. Elkana connut Anne, sa femme, et Yahweh se souvint d'elle.
\VS{20}Il arriva donc, quelque temps après, qu'Anne conçut et enfanta un fils, elle le nomma Samuel, parce que, dit-elle, je l'ai demandé à Yahweh.
\VS{21}Puis Elkana son mari monta avec toute sa maison pour offrir à Yahweh le sacrifice annuel et son vœu.
\VS{22}Mais Anne n'y monta pas, car elle dit à son mari~: Je n'irai pas jusqu'à ce que le petit enfant soit sevré, et alors je le mènerai, afin qu'il soit présenté devant Yahweh et qu'il demeure toujours-là.
\VS{23}Elkana son mari lui dit~: Fais ce qui te semblera bon, reste jusqu'à ce que tu l'aies sevré. Seulement que Yahweh accomplisse sa parole~! Ainsi cette femme resta et allaita son fils, jusqu'à ce qu'elle l'ait sevré.
\TextTitle{Samuel chez Eli~; Anne accomplit son vœu}
\VS{24}Et dès qu'elle l'eut sevré, elle le fit monter avec elle, et ayant pris trois veaux, un épha de farine et une outre de vin, elle le mena dans la maison de Yahweh à Silo~; l'enfant était très jeune.
\VS{25}Puis ils égorgèrent un veau, et ils amenèrent l'enfant à Eli.
\VS{26}Elle dit~: Pardon, mon seigneur~! Aussi vrai que ton âme vit, mon seigneur, je suis cette femme qui me tenais en ta présence pour prier Yahweh.
\VS{27}J'ai prié pour avoir cet enfant, et Yahweh m'a accordé la demande que je lui ai faite.
\VS{28}C'est pourquoi je le prête à Yahweh~: Il sera prêté à Yahweh pour tous les jours de sa vie. Et ils se prosternèrent là devant Yahweh.
\Chap{2}
\TextTitle{Prière d'Anne}
\VerseOne{}Alors Anne pria, et dit~: Mon cœur se réjouit en Yahweh, ma force\FTNT{Littéralement «~corne~».} a été relevée par Yahweh~; ma bouche s'est ouverte contre mes ennemis, parce que je me suis réjouie de ton salut\FTNT{Le mot «~salut~» vient de l'hébreu «~yeshuw'ah~» c'est-à-dire «~Jésus~». Voir commentaire en Es. 26:1.}.
\VS{2}Nul n'est saint comme Yahweh~; car il n'y en a pas d'autres que toi~; et il n'y a pas de rocher\FTNT{Voir commentaire en Es. 8:13-17.} tel que notre Dieu.
\VS{3}Ne proférez pas tant de paroles hautaines~; qu'il ne sorte pas de votre bouche des paroles arrogantes~; car Yahweh est le Dieu qui sait tout, c'est lui qui pèse toutes les actions.
\VS{4}L'arc des puissants est brisé, mais ceux qui chancellent ont la force pour ceinture.
\VS{5}Ceux qui étaient rassasiés se louent pour du pain, mais les affamés ont cessé de l'être~; même la stérile en a enfanté sept, et celle qui avait beaucoup de fils est devenue languissante.
\VS{6}Yahweh est celui qui fait mourir et qui fait vivre, qui fait descendre au scheol et qui en fait remonter.
\VS{7}Yahweh appauvrit et il enrichit, il abaisse et il élève.
\VS{8}Il élève le pauvre de la poussière, et il tire le misérable de dessus le fumier, pour le faire asseoir avec les nobles. Et il leur donne en héritage un trône de gloire~; car les fondements de la terre sont à Yahweh, et il a posé sur eux la terre habitable.
\VS{9}Il gardera les pieds de ses bien-aimés, et les méchants se tairont dans les ténèbres~; car l'homme ne triomphera pas par sa force.
\VS{10}Ceux qui contestent contre Yahweh seront effrayés~; des cieux il lancera son tonnerre sur chacun d'eux~; Yahweh jugera les extrémités de la terre. Et il donnera la force à son Roi\FTNT{Le Roi dont il est question ici est le Seigneur Jésus-Christ, le Roi des rois (Za. 14:9~; Ap. 19:16).}, et élèvera la force de son Messie\FTNT{Anne a annoncé la glorification ou la résurrection du Seigneur Jésus, le Messie (Jn. 3:14).}.
\VS{11}Puis Elkana s'en alla à Rama dans sa maison, et le jeune garçon vaquait au service de Yahweh, en présence du prêtre Eli.
\TextTitle{Corruption des fils d'Eli}
\VS{12}Or les fils d'Eli\FTNT{Les fils d'Eli, Hophni et Phinées étaient corrompus. Ils volaient les offrandes de Dieu, couchaient avec les femmes qui venaient servir et adorer Dieu. L'esprit qui animait ces prêtres n'a pas disparu après leur mort, mais il opère encore dans beaucoup d'institutions religieuses actuelles. Beaucoup de dirigeants d'églises continuent à s'approprier ce qui appartient à Dieu (l'adoration, les âmes… ). Ils ne craignent pas Yahweh. Ils abusent de leur position et de leur autorité pour contraindre leurs fidèles à leur donner la dîme et toutes sortes d'offrandes. Ils font payer les entretiens, les prières, et les divers dons qu'ils peuvent avoir. Non seulement l'esprit qui animait les fils d'Eli existe encore, mais il s'est accru en ces temps actuels.} étaient des fils de Bélial\FTNT{Les fils d'Eli étaient qualifiés de «~fils de Bélial~». Ce mot vient de l'hébreu «~beliya'al~» qui signifie «~indigne~», «~bon à rien~», «~méchant~», «~ruine~», «~destruction~». Il est à noter que Bélial est aussi un des noms de Satan. (2 Co. 6:15). Les fils d'Eli servaient Dieu sans le connaître. En fait, ils étaient au service de Satan. Ce terme est également utilisé au sujet des méchants qui incitèrent les Israélites à servir les dieux étrangers (De. 13:12-14), les hommes iniques de Guibea (Jg. 19:22~; 20:13), les deux vauriens qui accusèrent Naboth (1 R. 21:10-13) et les individus qui s'opposèrent à la monarchie (1 S. 10:27~; 2 S. 20:1~; 2 Ch. 13:7). Voir aussi De. 13:13~; 15:9~; Job 34:18~; Ps. 18:4~; 34:17~; Pr. 6:12~; 16:27~; 19:28~; Na. 1:11.} et ils ne connaissaient pas Yahweh.
\VS{13}Et voici la coutume de ces prêtres envers le peuple~: Lorsque quelqu'un faisait quelque sacrifice, le serviteur du prêtre venait lorsqu'on faisait bouillir la chair, ayant à la main une fourchette à trois dents,
\VS{14}avec laquelle il piquait dans la chaudière, dans le chaudron, dans la marmite, dans le pot~; et le prêtre prenait pour lui tout ce que la fourchette enlevait. C'est ainsi qu'ils agissaient envers tous ceux d'Israël qui venaient à Silo.
\VS{15}Même avant qu'on fasse brûler la graisse, le serviteur du prêtre venait, et disait à l'homme qui sacrifiait~: Donne-moi de la chair à rôtir pour le prêtre~; car il ne prendra pas de toi de chair bouillie, mais de la chair crue.
\VS{16}Et si l'homme lui répondait~: On va d'abord faire brûler la graisse, et après cela tu prendras ce que ton âme souhaitera, alors le serviteur lui disait~: Quoi qu'il en soit, tu en donneras maintenant, sinon j'en prendrai de force.
\VS{17}Et le péché de ces jeunes hommes fut très grand devant Yahweh, car les hommes méprisaient l'offrande de Yahweh.
\TextTitle{Samuel au service de Yahweh}
\VS{18}Samuel faisait le service en présence de Yahweh, étant jeune garçon, vêtu d'un éphod de lin.
\VS{19}Sa mère lui faisait une petite tunique, qu'elle lui apportait tous les ans, quand elle montait avec son mari pour offrir le sacrifice annuel.
\VS{20}Eli bénit Elkana, et sa femme, et dit~: Que Yahweh te donne des enfants de cette femme, pour le prêt qu'elle a fait à Yahweh~! Et ils s'en retournèrent chez eux.
\VS{21}Et Yahweh visita Anne, elle conçut, et enfanta trois fils et deux filles. Et le jeune garçon Samuel grandissait en présence de Yahweh.
\TextTitle{Eli averti des péchés commis par ses fils}
\VS{22}Or Eli était très vieux, il apprit tout ce que faisaient ses fils à tout Israël, et qu'ils couchaient avec les femmes qui s'assemblaient à la porte de la tente d'assignation.
\VS{23}Et il leur dit~: Pourquoi commettez-vous de telles choses~? Car j'apprends vos méchantes actions de tout le peuple.
\VS{24}Ne faites pas ainsi, mes fils, car ce que j'entends dire de vous n'est pas bon~; vous faites pécher le peuple de Yahweh.
\VS{25}Si un homme a péché contre un autre homme, le Juge\FTNT{Ici, le mot «~juge~» signifie aussi «~Dieu~» («~elohim~» en hébreux). Dieu est le juste juge (Ec. 3:17~; Ac. 10:42).} interviendra~; mais si quelqu'un pèche contre Yahweh, qui interviendra pour lui~? Mais ils n'obéirent pas à la voix de leur père parce que Yahweh voulait les faire mourir.
\VS{26}Cependant le jeune garçon Samuel croissait et il était agréable à Yahweh et aux hommes.
\VS{27}Or un homme de Dieu vint auprès d'Eli, et lui dit~: Ainsi parle Yahweh~: Ne me suis-je pas clairement manifesté à la maison de ton père, quand ils étaient en Egypte, dans la maison de Pharaon~?
\VS{28}Je l'ai choisi parmi toutes les tribus d'Israël pour être mon prêtre, afin d'offrir sur mon autel, et faire brûler les parfums, et porter l'éphod devant moi, et j'ai donné à la maison de ton père tous les holocaustes des enfants d'Israël.
\VS{29}Pourquoi avez-vous foulé aux pieds mes sacrifices et mes offrandes, que j'ai ordonné de faire dans ma demeure~? Et pourquoi as-tu honoré tes fils plus que moi, afin de vous engraisser du meilleur de toutes les offrandes d'Israël, mon peuple~?
\VS{30}C'est pourquoi voici ce que dit Yahweh, le Dieu d'Israël~: J'avais dit et promis que ta maison et la maison de ton père marcheraient devant moi éternellement. Et Yahweh dit~: Il n'en sera pas ainsi~! Car j'honorerai ceux qui m'honorent, mais ceux qui me méprisent seront méprisés.
\VS{31}Voici, les jours viennent où je couperai ton bras, et le bras de la maison de ton père, de telle sorte qu'il n'y ait plus de vieillards dans ta maison.
\VS{32}Et tu verras un adversaire dans ma demeure, au temps où Dieu enverra toutes sortes de biens à Israël~; et il n'y aura plus jamais de vieillards dans ta maison.
\VS{33}Celui de tes descendants que je n'aurai pas retranché d'auprès de mon autel subsistera pour consumer tes yeux et affliger ton âme~; et tous les enfants de ta maison mourront dans la fleur de l'âge.
\VS{34}Et ceci sera pour toi un signe, à savoir ce qui arrivera à tes deux fils, Hophni et Phinées~; ils mourront tous les deux le même jour.
\VS{35}Et je m'établirai un prêtre fidèle\FTNT{Hé. 2:17~; 7:26-28.}, qui agira selon mon cœur, et selon mon âme~; et je lui édifierai une maison stable\FTNT{La maison stable fait premièrement allusion à Israël (Mi. 4) et ensuite à l'Eglise (Mt. 16:18). Cette prophétie sera pleinement réalisée lors du millénium (Za. 14).}, et il marchera à toujours devant mon Messie.
\VS{36}Et il arrivera que quiconque sera resté de ta maison viendra se prosterner devant lui pour avoir une pièce d'argent et un morceau de pain, et dira~: Place-moi, je te prie, dans une des charges de la prêtrise pour manger un morceau de pain.
\Chap{3}
\TextTitle{Yahweh appelle Samuel}
\VerseOne{}Or le jeune garçon Samuel servait Yahweh en présence d'Eli. La parole de Yahweh était rare en ce temps-là, et les visions n'étaient pas fréquentes.
\VS{2}Il arriva en ce temps qu'Eli était couché à sa place, ses yeux commençaient à se ternir et il ne pouvait plus voir.
\VS{3}Et avant que les lampes\FTNT{Le chandelier d'or à sept branches du tabernacle et du temple de Jérusalem a été décrit avec une extrême minutie dans plusieurs passages de la Bible. Il a été réalisé selon le modèle imposé par Dieu à Moïse au Sinaï (Ex. 25:31-40~; Ex. 37:17-24~; No. 8:4).} de Dieu soient éteintes, Samuel était aussi couché dans le temple de Yahweh, où était l'arche de Dieu.
\VS{4}Yahweh appela Samuel. Et il répondit~: Me voici~!
\VS{5}Et il courut vers Eli, et lui dit~: Me voici, car tu m'as appelé. Mais Eli dit~: Je ne t'ai pas appelé~; retourne te coucher. Et il s'en alla, et se coucha.
\VS{6}Yahweh appela encore Samuel. Et Samuel se leva, et s'en alla vers Eli, et lui dit~: Me voici, car tu m'as appelé~! Et Eli dit~: Mon fils, je ne t'ai pas appelé, retourne, et couche-toi.
\VS{7}Or Samuel ne connaissait pas encore Yahweh, et la parole de Yahweh ne lui avait pas encore été révélée.
\VS{8}Et Yahweh appela encore Samuel pour la troisième fois. Et Samuel se leva, et s'en alla vers Eli, et dit~: Me voici, car tu m'as appelé. Eli reconnut que Yahweh appelait ce jeune garçon,
\VS{9}alors Eli dit à Samuel~: Va et couche-toi~; et si on t'appelle, tu diras~: Parle, Yahweh, car ton serviteur écoute. Samuel donc s'en alla, et se coucha à sa place.
\VS{10}Yahweh donc vint, et se tint là, et appela comme les autres fois~: Samuel, Samuel~! Et Samuel dit~: Parle, car ton serviteur écoute.
\TextTitle{Jugement de Yahweh sur la maison d'Eli}
\VS{11}Alors Yahweh dit à Samuel~: Voici, je vais faire une chose en Israël, qui étourdira les oreilles de quiconque l'entendra.
\VS{12}En ce jour-là, j'accomplirai sur Eli tout ce que j'ai déclaré contre sa maison~; je commencerai et j'achèverai.
\VS{13}Car je l'ai averti que je vais punir sa maison à perpétuité, à cause de l'iniquité dont il a connaissance, par laquelle ses fils se sont rendus infâmes, sans qu'ils les ait réprimés.
\VS{14}C'est pourquoi j'ai juré contre la maison d'Eli~; si jamais il se fait propitiation pour l'iniquité de la maison d'Eli, par sacrifice ou par offrande.
\VS{15}Et Samuel resta couché jusqu'au matin, puis il ouvrit les portes de la maison de Yahweh. Or Samuel craignait de rapporter cette vision à Eli.
\VS{16}Mais Eli appela Samuel, et lui dit~: Samuel, mon fils~! Il répondit~: Me voici~!
\VS{17}Et Eli dit~: Quelle est la parole qui t'a été adressée~? Je te prie ne me la cache pas. Ainsi Dieu te fasse, et ainsi il y ajoute, si tu me caches un seul mot de tout ce qui t'a été dit.
\VS{18}Samuel lui déclara donc toutes ces paroles, et ne lui en cacha rien. Et Eli répondit~: C'est Yahweh, qu'il fasse ce qui lui semblera bon~!
\VS{19}Samuel grandissait. Et Yahweh était avec lui, il ne laissa pas tomber à terre une seule de ses paroles.
\VS{20}Tout Israël, depuis Dan jusqu'à Beer-Schéba, reconnut que Samuel était établi prophète de Yahweh.
\VS{21}Yahweh continuait de se manifester dans Silo~; car Yahweh se manifestait à Samuel, dans Silo, par la parole de Yahweh.
\Chap{4}
\TextTitle{Les Philistins s'emparent de l'arche~; Yahweh juge la maison d'Eli}
\VerseOne{}Or la parole de Samuel arriva à tout Israël. Car Israël sortit en bataille pour aller à la rencontre des Philistins. Ils campèrent près d'Eben-Ezer, et les Philistins campaient à Aphek.
\VS{2}Les Philistins se rangèrent en bataille contre Israël, et le combat s'engagea. Israël fut battu par les Philistins, qui en tuèrent environ quatre mille hommes sur le champ de bataille.
\VS{3}Quand le peuple rentra au camp, les anciens d'Israël dirent~: Pourquoi Yahweh nous a-t-il battus aujourd'hui par les Philistins~? Ramenons de Silo l'arche de l'alliance de Yahweh, et qu'elle vienne au milieu de nous, et nous délivre de la main de nos ennemis.
\VS{4}Le peuple envoya donc à Silo, d'où l'on apporta l'arche de l'alliance de Yahweh des armées qui habite entre les chérubins. Les deux fils d'Eli, Hophni et Phinées, étaient là, avec l'arche de l'alliance de Dieu.
\VS{5}Et il arriva que, comme l'arche de Yahweh entrait dans le camp, tout Israël poussa de grands cris de joie et la terre en fut ébranlée.
\VS{6}Les Philistins entendirent le bruit de ces cris de joie, et ils dirent~: Que veut dire ce bruit, et que signifient ces grands cris de joie dans le camp de ces Hébreux~? Et ils apprirent que l'arche de Yahweh était arrivée dans le camp.
\VS{7}Les Philistins eurent peur, car ils disaient~: Dieu est entré dans le camp. Et ils dirent~: Malheur à nous~! Car il n'en a pas été ainsi auparavant.
\VS{8}Malheur à nous~! Qui nous délivrera de la main de ces Dieux puissants\FTNT{Le terme hébreu «~elohim~», généralement traduit par «~dieu~» ou «~dieux~», signifie également «~dirigeants~», «~juges~» ou encore «~anges~». Dans les textes bibliques, «~Elohim~» est employé pour désigner Moïse, qui a été fait «~dieu~» («~elohim~») pour Pharaon (Ex. 7:1), ainsi que pour les dieux païens Baal, Kemosh et Dagon (Jg. 6:31~; Jg. 11:24~; 1 S. 5:7). Les Philistins avaient une vision polythéiste de la divinité et n'avaient pas la révélation du Dieu des Hébreux qui est Un (De. 6:4).}~? Ce sont ces Dieux qui ont frappé les Egyptiens de toutes sortes de plaies dans le désert.
\VS{9}Philistins, prenez courage et agissez en hommes, de peur que vous ne soyez esclaves des Hébreux, comme ils vous ont été asservis~; agissez en hommes, et combattez~!
\VS{10}Les Philistins donc combattirent, et Israël fut battu. Et chacun s'enfuit dans sa tente. La défaite fut très grande, trente mille hommes de pied d'Israël périrent.
\VS{11}L'arche de Dieu fut prise, et les deux fils d'Eli, Hophni et Phinées, moururent.
\VS{12}Un homme de Benjamin s'enfuit de la bataille, et arriva à Silo ce même jour, ayant ses vêtements déchirés et la tête recouverte de terre.
\VS{13}Et comme il arrivait, voici, Eli était assis sur un siège au bord du chemin, étant attentif, car son cœur tremblait à cause de l'arche de Dieu. Cet homme entra donc dans la ville, et donna les nouvelles, et toute la ville se mit à crier.
\VS{14}Eli, entendant les cris, dit~: Que veut dire ce grand tumulte~? Et aussitôt cet homme vint à Eli, et lui raconta tout.
\VS{15}Or Eli était âgé de quatre-vingt-dix-huit ans, ses yeux étaient fixes, il ne pouvait plus voir.
\VS{16}L'homme dit à Eli~: Je viens de la bataille, car je me suis enfui aujourd'hui de la bataille. Et Eli dit~: Qu'est-il arrivé, mon fils~?
\VS{17}Celui qui apportait les nouvelles répondit, et dit~: Israël a fui devant les Philistins, et il y a eu une grande défaite du peuple~; tes deux fils, Hophni et Phinées, sont morts et l'arche de Dieu a été prise.
\VS{18}Et il arriva qu'aussitôt qu'il eut fait mention de l'arche de Dieu, Eli tomba à la renverse, de dessus son siège, à côté de la porte, se rompit le cou et mourut~; car cet homme était vieux et pesant. Il avait été juge en Israël pendant quarante ans.
\VS{19}Sa belle-fille, femme de Phinées, était enceinte, et sur le point d'accoucher. Lorsqu'elle apprit la nouvelle de la prise de l'arche de Dieu, de la mort de son beau-père et de son mari, elle se coucha et enfanta, car les douleurs la surprirent.
\VS{20}Comme elle mourait, celles qui l'assistaient lui dirent~: Ne crains pas, car tu as enfanté un fils~! Mais elle ne répondit rien et n'en tint pas compte.
\VS{21}Mais elle appela l'enfant I-Kabod, en disant~: La gloire s'en est allée d'Israël~! Parce que l'arche de Yahweh était prise à cause de son beau-père et de son mari.
\VS{22}Elle dit donc~: La gloire s'en est allée d'Israël, car l'arche de Dieu est prise~!
\Chap{5}
\TextTitle{Jugements de Yahweh sur les Philistins}
\VerseOne{}Les Philistins prirent l'arche de Dieu, et l'emmenèrent d'Eben-Ezer à Asdod.
\VS{2}Les Philistins donc prirent l'arche de Dieu, et l'emmenèrent dans la maison de Dagon\FTNT{L'étymologie du nom «~Dagon~» avait justifié la représentation qu'on faisait de ce dieu~: une sorte de sirène mâle ou un homme avec une queue de poisson. En effet, «~dâg~», en hébreu signifie «~poisson~». Il était le dieu des semences et de l'agriculture chez les peuples d'origine sémite, mais également l'un des principaux dieux des Philistins.}, et la posèrent auprès de Dagon.
\VS{3}Le lendemain les Asdodiens, s'étant levés de bon matin, trouvèrent Dagon le visage contre terre, devant l'arche de Yahweh. Mais ils le prirent et le remirent à sa place.
\VS{4}Ils se levèrent encore le lendemain de bon matin, et voici, Dagon était tombé le visage contre terre, devant l'arche de Yahweh~; la tête de Dagon et les deux paumes de ses mains découpées étaient sur le seuil, et il ne lui restait que le tronc.
\VS{5}C'est pour cela que les prêtres de Dagon, et tous ceux qui entrent dans la maison de Dagon, à Asdod, ne marchent pas sur le seuil jusqu'à aujourd'hui.
\VS{6}Puis la main de Yahweh s'appesantit sur les Asdodiens et les dévasta~; et il les frappa d'hémorroïdes à Asdod et dans tout son territoire.
\VS{7}Ceux donc d'Asdod, voyant qu'il en allait ainsi, dirent~: L'arche du Dieu d'Israël ne demeurera pas chez nous, car sa main s'est appesantie sur nous, et sur Dagon, notre dieu.
\VS{8}Et ils firent appeler et assemblèrent auprès d'eux tous les princes des Philistins, et dirent~: Que ferons-nous de l'arche du Dieu d'Israël~? Et ils répondirent~: Qu'on transporte à Gath l'arche du Dieu d'Israël. Ainsi on transporta l'arche du Dieu d'Israël.
\VS{9}Mais il arriva après qu'on l'eut transportée, la main de Yahweh fut sur la ville, et il y eut une très grande terreur~; et il frappa les gens de la ville, depuis le plus petit jusqu'au plus grand, par une éruption d'hémorroïdes.
\VS{10}Ils envoyèrent donc l'arche de Dieu à Ekron. Or comme l'arche de Dieu entrait à Ekron, ceux d'Ekron s'écrièrent, en disant~: Ils ont transporté vers nous l'arche du Dieu d'Israël, pour nous faire mourir, nous et notre peuple~!
\VS{11}C'est pourquoi ils firent appeler, et assemblèrent tous les princes des Philistins, en disant~: Renvoyez l'arche du Dieu d'Israël, et qu'elle retourne en son lieu, afin qu'elle ne nous fasse pas mourir, nous et notre peuple. Car il régnait une terreur mortelle dans toute la ville~; et la main de Dieu s'y appesantissait fortement.
\VS{12}Les hommes qui n'en mouraient pas étaient frappés d'hémorroïdes, de sorte que le cri de la ville montait jusqu'au ciel.
\Chap{6}
\TextTitle{L'arche de Yahweh revient en Israël}
\VerseOne{}L'arche de Yahweh ayant été pendant sept mois dans le pays des Philistins,
\VS{2}les Philistins appelèrent les prêtres et les devins, et leur dirent~: Que ferons-nous de l'arche de Yahweh~? Dites-nous comment nous devons la renvoyer en son lieu.
\VS{3}Ils répondirent~: Si vous renvoyez l'arche du Dieu d'Israël, ne la renvoyez pas à vide, et n'oubliez pas de lui payer le sacrifice pour la culpabilité~; alors vous serez guéris, et vous saurez pourquoi sa main ne s'est pas retirée de dessus vous.
\VS{4}Et ils dirent~: Quelle offrande lui payerons-nous pour le délit\FTNT{Le mot «~délit~» vient de l'hébreu «~asham~» qui signifie «~culpabilité~», «~offense~», «~ce qui est acquis par un délit, mal acquis~». Il est question ici de l'arche qui avait été volée par les Philistins. Aux yeux de Dieu, cet acte était un délit.}~? Et ils répondirent~: Selon le nombre des princes des Philistins, vous donnerez cinq hémorroïdes d'or, et cinq souris d'or, car une même plaie a été sur vous tous, et sur vos princes.
\VS{5}Vous ferez donc des figures de vos hémorroïdes, et des figures des souris qui ravagent le pays, et vous donnerez gloire au Dieu d'Israël~: Peut-être retirera-t-il sa main de dessus vous, et de dessus vos dieux, et de dessus votre pays.
\VS{6}Et pourquoi endurciriez-vous votre cœur, comme l'Egypte et Pharaon ont endurci leur cœur~? Après qu'il eut fait de merveilleux exploits parmi eux, ne les laissèrent-ils pas partir et s'en aller~?
\VS{7}Maintenant, donc prenez de quoi faire un char tout neuf, et deux jeunes vaches qui allaitent leurs veaux et qui n'aient point porté le joug~; et attelez au char les deux jeunes vaches, et ramenez leurs petits à la maison.
\VS{8}Puis prenez l'arche de Yahweh et mettez-la sur le char~; mettez les ouvrages d'or, que vous lui aurez payés pour le délit, dans un petit coffre à côté de l'arche, puis renvoyez-la, et elle s'en ira.
\VS{9}Et vous observerez~: Si l'arche monte vers Beth-Schémesch, par le chemin de sa frontière, c'est Yahweh qui nous a fait tout ce grand mal~; si elle n'y va pas, nous saurons alors que sa main ne nous a pas touchés, mais que ceci nous est arrivé par hasard.
\VS{10}Ces gens firent ainsi. Ils prirent donc deux jeunes vaches qui allaitaient, ils les attelèrent au char, et ils enfermèrent leurs petits dans l'étable.
\VS{11}Ils mirent sur le char l'arche de Yahweh, et le coffre avec les souris d'or, et les figures de leurs hémorroïdes.
\VS{12}Alors les jeunes vaches prirent tout droit le chemin de Beth-Schémesch, elles suivirent toujours le même chemin, en marchant et en mugissant, et elles ne se détournèrent ni à droite ni à gauche. Les princes des Philistins allèrent après elles jusqu'à la frontière de Beth-Schémesch.
\VS{13}Or ceux de Beth-Schémesch moissonnaient les blés dans la vallée~; et ayant élevé leurs yeux, ils virent l'arche, et se réjouirent en la voyant.
\VS{14}Le char arriva dans le champ de Josué de Beth-Schémesch, et s'arrêta là. Or il y avait là une grande pierre, et on fendit le bois du char, et on offrit les jeunes vaches en holocauste à Yahweh.
\VS{15}Les Lévites descendirent l'arche de Yahweh, et le coffre dans lequel étaient les objets d'or~; et ils les mirent sur cette grande pierre. En ce même jour, ceux de Beth-Schémesch offrirent des holocaustes et des sacrifices à Yahweh.
\VS{16}Les cinq princes des Philistins, après avoir vu cela, retournèrent le même jour à Ekron.
\VS{17}Voici les hémorroïdes d'or que les Philistins donnèrent à Yahweh en offrande pour le délit~: Un pour Asdod, un pour Gaza, un pour Askalon, un pour Gath, un pour Ekron.
\VS{18}Les souris d'or, selon le nombre de toutes les villes des Philistins, appartenant aux cinq princes, tant des villes fortifiées que des villages sans murailles. Et ils les amenèrent jusqu'à la grande pierre sur laquelle on posa l'arche de Yahweh, et qui jusqu'à ce jour est dans le champ de Josué de Beth-Schémesch.
\VS{19}Yahweh frappa des gens de Beth-Schémesch parce qu'ils avaient regardé dans l'arche de Yahweh~; il frappa (cinquante mille) et soixante-dix hommes\FTNT{Ce nombre est généralement considéré comme une erreur des copistes.}, et le peuple mena le deuil parce que Yahweh l'avait frappé d'une grande plaie.
\VS{20}Alors ceux de Beth-Schémesch dirent~: Qui pourrait subsister en présence de Yahweh, ce Dieu Saint~? Et vers qui montera-t-il en s'éloignant de nous~?
\VS{21}Et ils envoyèrent des messagers aux habitants de Kirjath-Jearim, en disant~: Les Philistins ont ramené l'arche de Yahweh~; descendez, et faites-la monter vers vous.
\Chap{7}
\TextTitle{Un réveil après l'apostasie}
\VerseOne{}Ceux donc de Kirjath-Jearim vinrent, et firent monter l'arche de Yahweh, et la mirent dans la maison d'Abinadab, sur la colline, et ils consacrèrent Eléazar, son fils, pour garder l'arche de Yahweh.
\VS{2}Il s'écoula un long moment depuis le jour où l'arche de Yahweh fut déposée à Kirjath-Jearim. Vingt années s'étaient écoulées. Toute la maison d'Israël soupira après Yahweh.
\VS{3}Et Samuel parla à toute la maison d'Israël, en disant~: Si vous revenez à Yahweh de tout votre cœur, ôtez du milieu de vous les dieux étrangers, et les Astartés, dirigez votre cœur vers Yahweh, et servez-le, lui seul~; et il vous délivrera de la main des Philistins.
\VS{4}Alors les enfants d'Israël ôtèrent les Baals et les Astartés, et ils servirent Yahweh seul.
\VS{5}Samuel dit~: Assemblez tout Israël à Mitspa, et je prierai Yahweh pour vous.
\VS{6}Ils s'assemblèrent donc à Mitspa. Ils puisèrent de l'eau qu'ils répandirent devant Yahweh et ils jeûnèrent ce jour-là, en disant~: Nous avons péché contre Yahweh~! Et Samuel jugea les enfants d'Israël à Mitspa.
\VS{7}Or quand les Philistins eurent appris que les enfants d'Israël étaient assemblés à Mitspa, les princes des Philistins montèrent contre Israël. Les enfants d'Israël l'apprirent et ils eurent peur des Philistins.
\VS{8}Les enfants d'Israël dirent à Samuel~: Ne cesse pas de crier pour nous à Yahweh, notre Dieu, afin qu'il nous délivre de la main des Philistins.
\TextTitle{Victoire d'Israël contre les Philistins}
\VS{9}Alors Samuel prit un agneau de lait, et l'offrit tout entier à Yahweh en holocauste. Et Samuel cria à Yahweh pour Israël, et Yahweh l'exauça.
\VS{10}Il arriva donc, comme Samuel offrait l'holocauste, les Philistins s'approchèrent pour combattre contre Israël, mais Yahweh fit gronder, en ce jour-là, un grand tonnerre sur les Philistins, et les mit en déroute, et ils furent battus devant Israël.
\VS{11}Les hommes d'Israël sortirent de Mitspa, et poursuivirent les Philistins, et les frappèrent jusqu'au-dessous de Beth-Car.
\VS{12}Alors Samuel prit une pierre, et la mit entre Mitspa et Schen, et il appela ce lieu Eben-Ezer, en disant~: Yahweh nous a secourus jusqu'en ce lieu-ci.
\VS{13}Les Philistins furent humiliés, et ils ne vinrent plus sur le territoire d'Israël. La main de Yahweh fut contre les Philistins durant la vie de Samuel.
\VS{14}Les villes que les Philistins avaient prises sur Israël retournèrent à Israël, depuis Ekron jusqu'à Gath, avec leurs territoires~; Israël les délivra donc de la main des Philistins. Et il y eut paix entre Israël et les Amoréens.
\TextTitle{Samuel, juge en Israël}
\VS{15}Samuel fut juge en Israël tous les jours de sa vie.
\VS{16}Il allait tous les ans faire le tour de Béthel, de Guilgal et de Mitspa, et il jugeait Israël dans tous ces lieux.
\VS{17}Puis il revenait à Rama, où était sa maison~; et là il jugeait Israël, et il y bâtit un autel à Yahweh.
\Chap{8}
\TextTitle{Les fils de Samuel corrompus~; Israël demande un roi}
\VerseOne{}Et il arriva que quand Samuel était devenu vieux, il établit ses fils pour juges sur Israël.
\VS{2}Son fils premier-né s'appelait Joël, et le second Abija~; ils jugeaient à Beer-Schéba.
\VS{3}Mais ses fils ne marchèrent pas dans ses voies, ils s'en détournèrent pour les profits acquis par la violence~; ils recevaient des présents et violaient la justice.
\VS{4}C'est pourquoi tous les anciens d'Israël s'assemblèrent, et vinrent auprès de Samuel à Rama.
\VS{5}Ils lui dirent~: Voici, tu es devenu vieux, et tes fils ne suivent pas tes voies~; maintenant, établis sur nous un roi pour nous juger, comme il y en a chez toutes les nations.
\TextTitle{Yahweh accepte la requête du peuple}
\VS{6}Samuel fut affligé de ce qu'ils lui avaient dit~: Etablis sur nous un roi pour nous juger. Et Samuel pria Yahweh.
\VS{7}Yahweh dit à Samuel~: Obéis à la voix du peuple dans tout ce qu'il te dira~; car ce n'est pas toi qu'ils ont rejeté, mais c'est moi qu'ils ont rejeté, afin que je ne règne plus sur eux.
\VS{8}Ils agissent à ton égard comme ils ont agi depuis le jour où je les ai fait monter hors d'Egypte jusqu'à ce jour~; ils m'ont abandonné, pour servir d'autres dieux.
\VS{9}Maintenant donc, obéis à leur voix~; mais les avertis-les, avertis-les en leur déclarant comment le roi qui régnera sur eux les traitera.
\TextTitle{Avertissement~: Le roi sera un joug pour le peuple}
\VS{10}Ainsi Samuel dit toutes les paroles de Yahweh au peuple qui lui avait demandé un roi.
\VS{11}Il leur dit donc~: Voici comment vous traitera le roi qui régnera sur vous. Il prendra vos fils et les mettra sur ses chars et parmi ses cavaliers, afin qu'ils courent devant son char~;
\VS{12}il en établira des chefs de mille, et des chefs de cinquante, pour labourer ses terres, pour récolter ses moissons, et pour fabriquer ses armes de guerre et l'équipement de ses chars.
\VS{13}Il prendra aussi vos filles pour en faire des parfumeuses, des cuisinières, et des boulangères.
\VS{14}Il prendra ce qu'il y a de meilleur parmi vos champs, vos vignes et vos oliviers, et il les donnera à ses serviteurs.
\VS{15}Il prélèvera la dîme de ce que vous aurez semé et de ce que vous aurez vendangé, et il la donnera à ses eunuques, et à ses serviteurs.
\VS{16}Il prendra vos serviteurs et vos servantes, l'élite de vos jeunes gens, vos ânes, et les emploiera à ses ouvrages.
\VS{17}Il prélèvera la dîme de vos troupeaux, et vous serez ses esclaves.
\VS{18}En ce jour-là, vous crierez à cause du roi que vous vous serez choisi, mais Yahweh ne vous exaucera pas.
\VS{19}Mais le peuple refusa d'écouter la voix de Samuel, et ils dirent~: Non~! Mais il y aura un roi sur nous.
\VS{20}Nous serons aussi comme toutes les nations~; et notre roi nous jugera, il sortira devant nous, et il conduira nos guerres.
\VS{21}Samuel entendit donc toutes les paroles du peuple, et les rapporta à Yahweh.
\VS{22}Et Yahweh dit à Samuel~: Obéis à leur voix, et établis un roi sur eux. Et Samuel dit aux hommes d'Israël~: Allez-vous-en chacun dans sa ville.
\Chap{9}
\TextTitle{Saül choisi pour devenir le premier roi d'Israël}
\VerseOne{}Or il y avait un homme de Benjamin, nommé Kis, fort et vaillant, fils d'Abiel, fils de Tseror, fils de Becorath, fils d'Aphiach, fils d'un Benjamite.
\VS{2}Il avait un fils nommé Saül, jeune et beau, et aucun des enfants d'Israël n'était plus beau que lui, des épaules en haut, il dépassait tout le peuple.
\VS{3}Les ânesses de Kis, père de Saül, s'égarèrent~; et Kis dit à Saül, son fils~: Prends maintenant avec toi un des serviteurs et lève-toi, et va chercher les ânesses.
\VS{4}Il passa donc par la montagne d'Ephraïm et traversa le pays de Schalischa, mais ils ne les trouvèrent pas~; puis ils passèrent par le pays de Schaalim, mais elles n'y étaient pas~; ils passèrent ensuite par le pays de Benjamin, mais ils ne les trouvèrent pas.
\VS{5}Quand ils furent arrivés dans le pays de Tsuph, Saül dit à son serviteur qui était avec lui~: Viens, et retournons, de peur que mon père oublie les ânesses, et s'inquiète pour nous.
\VS{6}Le serviteur lui dit~: Voici, je te prie, il y a dans cette ville un homme de Dieu, qui est un homme très honoré~; tout ce qu'il déclare ne manque pas d'arriver. Allons y maintenant, peut-être nous renseignera-t-il sur le chemin que nous devons prendre.
\VS{7}Et Saül dit à son serviteur~: Mais si nous y allons, que porterons-nous à l'homme de Dieu~? Nous n'avons plus de provisions, et nous n'avons aucun présent pour l'homme de Dieu. Qu'est-ce que nous avons~?
\VS{8}Le serviteur reprit la parole et dit à Saül~: Voici, j'ai encore entre mes mains le quart d'un sicle d'argent, et je le donnerai à l'homme de Dieu, et il nous indiquera notre chemin.
\VS{9}Autrefois en Israël, quand on allait consulter Dieu, on se disait l'un à l'autre~: Venez, allons vers le voyant~! Car le prophète s'appelait autrefois le voyant.
\VS{10}Saül dit à son serviteur~: Tu as bien dit~: Viens, allons~! Et ils s'en allèrent dans la ville où était l'homme de Dieu.
\VS{11}Et comme ils montaient à la ville, ils trouvèrent des jeunes filles qui sortaient pour puiser de l'eau, et ils leur dirent~: Le voyant n'est-il pas ici~?
\VS{12}Elles leur répondirent, et dirent~: Il y est, le voilà devant toi~; hâte-toi maintenant, car il est venu aujourd'hui à la ville, parce qu'il y a aujourd'hui un sacrifice pour le peuple sur le haut lieu.
\VS{13}Quand vous entrerez dans la ville, vous le trouverez avant qu'il monte au haut lieu pour manger~; car le peuple ne mangera pas jusqu'à ce qu'il soit venu, parce qu'il doit bénir le sacrifice~; après quoi, les conviés mangeront. Montez donc maintenant, car vous le trouverez aujourd'hui.
\VS{14}Ils montèrent donc à la ville. Comme ils entraient dans la ville, Samuel, qui sortait pour monter au haut lieu, les rencontra.
\VS{15}Or un jour avant l'arrivée de Saül, Yahweh avait fait une révélation à Samuel, en disant~:
\VS{16}Demain, à cette même heure, je t'enverrai un homme du pays de Benjamin, et tu l'oindras pour être le conducteur de mon peuple d'Israël. Il délivrera mon peuple de la main des Philistins~; car j'ai regardé mon peuple, parce que son cri est venu jusqu'à moi.
\VS{17}Et dès que Samuel eut aperçu Saül, Yahweh lui dit~: Voici l'homme dont je t'ai parlé~; c'est lui qui dominera sur mon peuple.
\VS{18}Et Saül s'approcha de Samuel au milieu de la porte, et dit~: Indique-moi, je te prie, où est la maison du voyant.
\VS{19}Et Samuel répondit à Saül, et dit~: Je suis le voyant. Monte devant moi au haut lieu, et vous mangerez aujourd'hui avec moi. Je te laisserai partir demain, et je te dirai tout ce que tu as sur le cœur.
\VS{20}Mais quant aux ânesses que tu as perdues il y a trois jours, ne t'en inquiète pas, parce qu'elles ont été retrouvées. Et vers qui tend tout le désir d'Israël~? N'est-ce pas vers toi, et vers toute la maison de ton père~?
\VS{21}Saül répondit~: Ne suis-je pas de Benjamin, l'une des moindres tribus d'Israël, et ma famille n'est-elle pas la plus petite de toutes les familles de la tribu de Benjamin~? Pourquoi m'as-tu tenu de tels discours~?
\VS{22}Samuel prit Saül et son serviteur, et les fit entrer dans la salle, et les plaça à la tête des conviés, qui étaient environ trente hommes.
\VS{23}Et Samuel dit au cuisinier~: Apporte la portion que je t'ai donnée, en te disant~: Mets-la à part.
\VS{24}Le cuisinier prit l'épaule, et ce qui l'entoure, et il la servit à Saül. Et Samuel dit~: Voici ce qui a été réservé~; mets-le devant toi, et mange, car il t'a été gardé expressément pour cette heure, lorsque j'ai résolu de convier le peuple. Et Saül mangea avec Samuel ce jour-là.
\VS{25}Puis ils descendirent du haut lieu dans la ville, et Samuel parla avec Saül sur le toit.
\VS{26}Puis ils se levèrent de bon matin~; et, dès l'aurore, Samuel appela Saül sur le toit, et lui dit~: Lève-toi, et je te laisserai aller. Saül donc se leva, et ils sortirent tous deux dehors, lui et Samuel.
\VS{27}Et comme ils descendaient à l'extrémité de la ville, Samuel dit à Saül~: Dis au serviteur de passer devant nous. Et le serviteur passa devant. Arrête-toi maintenant, afin que je te fasse entendre la parole de Dieu.
\Chap{10}
\TextTitle{Samuel oint Saül comme roi}
\VerseOne{}Or Samuel prit une fiole d'huile, qu'il répandit sur la tête de Saül. Il l'embrassa, et lui dit~: Yahweh ne t'a-t-il pas oint pour être le conducteur de son héritage~?
\VS{2}Aujourd'hui, après m'avoir quitté, tu trouveras deux hommes près du sépulcre de Rachel, sur la frontière de Benjamin à Tseltsach, qui te diront~: Les ânesses que tu es allé chercher sont retrouvées~; et voici, ton père ne pense plus aux ânesses, mais il s'inquiète pour vous, disant~: Que dois-je faire à propos de mon fils~?
\VS{3}En allant plus loin, tu arriveras au chêne de Thabor, où tu seras rencontré par trois hommes qui montent vers Dieu, à Béthel, et l'un porte trois chevreaux, l'autre trois pains, et l'autre une outre de vin.
\VS{4}Ils te demanderont comment tu te portes, et ils te donneront deux pains, que tu recevras de leurs mains.
\VS{5}Après cela tu arriveras à Guibea-Elohim, où se trouve une garnison de Philistins. Et il arrivera qu'en entrant dans la ville, tu rencontreras une troupe de prophètes descendant du haut lieu, précédés du luth, du tambourin, de la flûte, et de la harpe, et qui prophétisent.
\VS{6}Alors l'Esprit de Yahweh te saisira, et tu prophétiseras avec eux, et tu seras changé en un autre homme.
\VS{7}Et quand ces signes te seront arrivés, fais avec force ce que tu trouveras, car Dieu est avec toi.
\VS{8}Puis tu descendras devant moi à Guilgal~; et voici, je descendrai vers toi pour offrir des holocaustes, et des sacrifices d'offrande de paix\FTNT{Voir commentaire en Lé. 3:1.}. Tu m'attendras là sept jours, jusqu'à ce que je vienne, et que je te déclare ce que tu devras faire.
\VS{9}Il arriva donc qu'aussitôt que Saül eut tourné le dos pour se séparer de Samuel, Dieu changea son cœur, et tous ces signes s'accomplirent le même jour.
\VS{10}Quand ils arrivèrent à Guibea, voici, une troupe de prophètes vint à sa rencontre. L'Esprit de Dieu le saisit, et il prophétisa au milieu d'eux.
\VS{11}Et il arriva que, quand tous ceux qui l'avaient connu auparavant virent qu'il était avec les prophètes, et qu'il prophétisait, ceux du peuple se dirent l'un à l'autre~: Qu'est-il arrivé au fils de Kis~? Saül est-il aussi parmi les prophètes~?
\VS{12}Un homme répondit~: Et qui est leur père~? De là le proverbe~: Saül est-il aussi parmi les prophètes~?
\VS{13}Lorsqu'il eut cessé de prophétiser, il se rendit au haut lieu.
\VS{14}L'oncle de Saül dit à Saül et à son serviteur~: Où êtes-vous allés~? Et il répondit~: Chercher les ânesses, mais ne les trouvant pas, nous sommes allés vers Samuel.
\VS{15}Et l'oncle de Saül dit~: Déclare-moi, je te prie, ce que vous a dit Samuel.
\VS{16}Saül répondit à son oncle~: Il nous a assuré que les ânesses étaient retrouvées. Mais il ne lui déclara rien concernant la royauté dont Samuel lui avait parlé.
\VS{17}Samuel convoqua le peuple devant Yahweh à Mitspa.
\VS{18}Et il dit aux enfants d'Israël~: Ainsi parle Yahweh, le Dieu d'Israël~: J'ai fait monter Israël hors d'Egypte, et je vous ai délivrés de la main des Egyptiens, et de la main de tous les royaumes qui vous opprimaient.
\VS{19}Mais aujourd'hui, vous avez rejeté votre Dieu, celui qui vous a délivrés de tous vos malheurs, et de vos afflictions, et vous avez dit~: Non, établis-nous un roi~! Présentez-vous donc maintenant, devant Yahweh, par tribus et par familles.
\VS{20}Ainsi Samuel fit approcher toutes les tribus d'Israël, et la tribu de Benjamin fut désignée.
\VS{21}Après, il fit approcher la tribu de Benjamin selon ses familles, et la famille de Matri fut désignée. Puis Saül\FTNT{Voir en annexe le tableau~: «~Chronologie~: Les rois et les prophètes~».}, fils de Kis, fut désigné. On le chercha, mais on ne le trouva pas.
\VS{22}On consulta de nouveau Yahweh~: Est-il encore venu quelqu'un ici~? Yahweh répondit~: Il est caché parmi les bagages.
\VS{23}Ils coururent donc le chercher, et il se présenta au milieu du peuple, et il était plus grand que tout le peuple, depuis les épaules en haut.
\VS{24}Et Samuel dit à tout le peuple~: Voyez-vous celui que Yahweh a choisi~? Il n'y a personne dans tout le peuple qui soit semblable à lui. Et le peuple poussa des cris de joie, et dit~: Vive le roi~!
\VS{25}Alors Samuel fit connaître au peuple les règles de la royauté, et les écrivit dans un livre, qu'il déposa devant Yahweh. Puis Samuel renvoya le peuple, chacun dans sa maison.
\VS{26}Saül aussi s'en alla chez lui à Guibea. Il fut accompagné par des vaillants hommes dont Dieu avait touché le cœur.
\VS{27}Mais il y eut des fils de Bélial\FTNT{1 S. 2:12.} qui dirent~: Comment celui-ci nous délivrerait-il~? Et ils le méprisèrent, et ne lui apportèrent pas de présent. Mais Saül fit le sourd.
\Chap{11}
\TextTitle{Saül bat les Ammonites}
\VerseOne{}Nachasch, l'Ammonite, vint et assiégea Jabès en Galaad. Les habitants de Jabès dirent à Nachasch~: Traite alliance avec nous et nous te servirons.
\VS{2}Mais Nachasch, l'Ammonite, leur répondit~: Je traiterai avec vous à la condition que je vous crève à tous l'œil droit, et que je mette cet opprobre sur tout Israël.
\VS{3}Les anciens de Jabès lui dirent~: Donne-nous sept jours de trêve, et nous enverrons des messagers dans tout le territoire d'Israël~; et s'il n'y a personne qui nous délivre, nous nous rendrons à toi.
\VS{4}Les messagers arrivèrent à Guibea de Saül, et dirent ces paroles devant le peuple. Tout le peuple éleva sa voix, et pleura.
\VS{5}Et voici, Saül revenait des champs derrière ses bœufs, et il dit~: Qu'est-ce qu'a ce peuple pour pleurer ainsi~? Et on lui raconta ce qu'avaient dit ceux de Jabès.
\VS{6}Et l'Esprit de Dieu saisit Saül, lorsqu'il entendit ces paroles, et sa colère s'enflamma fortement.
\VS{7}Il prit une paire de bœufs, et les coupa en morceaux, qu'il envoya dans tout le territoire d'Israël, par des messagers, en disant~: Les bœufs de tous ceux qui ne sortiront pas pour suivre Saül et Samuel, seront traités de la même manière. Et la frayeur de Yahweh tomba sur le peuple, et ils sortirent comme un seul homme.
\VS{8}Saül en fit la revue à Bézek~; les fils d'Israël étaient trois cents mille et ceux de Juda trente mille.
\VS{9}Puis, ils dirent aux messagers qui étaient venus~: Vous parlerez ainsi à ceux de Jabès en Galaad~: Vous serez délivrés demain, quand le soleil sera dans sa force. Les messagers rapportèrent donc cela à ceux de Jabès, qui s'en réjouirent~;
\VS{10}et ils dirent aux Ammonites~: Demain nous nous rendrons à vous, et vous nous traiterez selon votre bon plaisir.
\VS{11}Le lendemain, Saül disposa le peuple en trois corps. Ils entrèrent dans le camp des Ammonites à la veille du matin, et ils les battirent jusqu'à la chaleur du jour. Ceux qui échappèrent furent dispersés si bien qu'il n'en resta pas deux ensemble.
\TextTitle{Le peuple reconnaît Saül comme roi}
\VS{12}Le peuple dit à Samuel~: Qui est-ce qui dit~: Saül régnera-t-il sur nous~? Donnez-nous ces hommes-là, et nous les ferons mourir.
\VS{13}Saül répondit~: Personne ne sera mis à mort en ce jour, car Yahweh a délivré Israël aujourd'hui.
\VS{14}Et Samuel dit au peuple~: Venez, allons à Guilgal, et nous y renouvellerons la royauté.
\VS{15}Et tout le peuple se rendit à Guilgal, et là, ils établirent Saül pour roi devant Yahweh, à Guilgal. Et ils offrirent des sacrifices d'offrande de paix devant Yahweh~; Saül et tous ceux d'Israël se réjouirent beaucoup.
\Chap{12}
\TextTitle{Le peuple rend un bon témoignage de Samuel}
\VerseOne{}Alors Samuel dit à tout Israël~: Voici, j'ai obéi à votre voix dans tout ce que vous m'avez dit, et j'ai établi un roi sur vous.
\VS{2}Et maintenant, voici le roi qui marchera devant vous. Car moi, je suis vieux et tout blanc, et voici, mes fils aussi sont avec vous~; pour moi j'ai marché devant vous, depuis ma jeunesse jusqu'à ce jour.
\VS{3}Me voici~! Témoignez contre moi, devant Yahweh, et devant son oint. De qui ai-je pris le bœuf~? Et de qui ai-je pris l'âne~? Qui ai-je opprimé~? Qui ai-je traité durement~? Et de la main de qui ai-je reçu des présents, afin de fermer les yeux sur lui~? Et je vous le rendrai.
\VS{4}Et ils répondirent~: Tu ne nous as pas opprimés, tu ne nous as pas traités durement et tu n'as rien reçu de la main de personne.
\VS{5}Il leur dit encore~: Yahweh est témoin contre vous, et son oint aussi est témoin aujourd'hui, que vous n'avez rien trouvé entre mes mains. Et ils répondirent~: Il en est témoin.
\TextTitle{Rappel des péchés du peuple~; exhortation à craindre Yahweh}
\VS{6}Alors Samuel dit au peuple~: Yahweh est celui qui a établi Moïse et Aaron, et qui a fait monter vos pères hors du pays d'Egypte.
\VS{7}Maintenant donc, présentez-vous, et je vous jugerai devant Yahweh sur tous les bienfaits que Yahweh vous a accordés, à vous et à vos pères.
\VS{8}Après que Jacob fut entré en Egypte, vos pères crièrent à Yahweh, et Yahweh envoya Moïse et Aaron qui firent sortir vos pères hors d'Egypte, et les firent habiter en ce lieu.
\VS{9}Mais ils oublièrent Yahweh, leur Dieu, et il les livra entre les mains de Sisera, chef de l'armée de Hatsor, et entre les mains des Philistins, et entre les mains du roi de Moab, qui leur firent la guerre.
\VS{10}Ils crièrent encore à Yahweh, et dirent~: Nous avons péché, car nous avons abandonné Yahweh, et nous avons servi les Baals et les Astartés. Maintenant donc, délivre-nous de la main de nos ennemis, et nous te servirons.
\VS{11}Et Yahweh envoya Jerubbaal, Bedan, Jephthé et Samuel, et il vous délivra de la main de tous vos ennemis d'alentour, et vous demeurâtes en sécurité.
\VS{12}Mais voyant que Nachasch, roi des fils d'Ammon, marchait contre vous, vous m'avez dit~: Non~! Mais un roi régnera sur nous. Alors que Yahweh, votre Dieu, était votre Roi.
\VS{13}Maintenant donc, voici le roi que vous avez choisi, que vous avez demandé~; et voici Yahweh l'a établi roi sur vous.
\VS{14}Si vous craignez Yahweh, si vous le servez, et obéissez à sa voix, et que vous n'êtes pas rebelles au commandement de Yahweh, alors vous et votre roi qui règne sur vous, vous serez sous la conduite de Yahweh, votre Dieu.
\VS{15}Mais si vous n'obéissez pas à la voix de Yahweh, et si vous êtes rebelles au commandement de Yahweh, la main de Yahweh sera aussi contre vous, comme elle a été contre vos pères.
\VS{16}Maintenant, préparez-vous, et voyez cette grande chose que Yahweh va opérer sous vos yeux.
\VS{17}N'est-ce pas aujourd'hui la moisson des blés~? Je crierai à Yahweh, et il enverra des tonnerres et de la pluie. Sachez alors et voyez combien vous avez mal agi aux yeux de Yahweh en demandant un roi.
\VS{18}Alors Samuel cria à Yahweh, et Yahweh envoya des tonnerres et de la pluie ce même jour. Tout le peuple eut une grande crainte de Yahweh, et de Samuel.
\VS{19}Et tout le peuple dit à Samuel~: Prie Yahweh, ton Dieu, pour tes serviteurs, afin que nous ne mourions pas~; car nous avons ajouté à nos péchés, celui d'avoir demandé un roi.
\VS{20}Alors Samuel dit au peuple~: Ne craignez pas~! Vous avez fait tout ce mal, néanmoins ne vous détournez pas de Yahweh, mais servez Yahweh de tout votre cœur.
\VS{21}Ne vous en détournez pas, car vous iriez après des choses de néant, qui ne vous apportent ni profit ni délivrance, puisque ce sont des choses de néant.
\VS{22}Car Yahweh n'abandonne pas son peuple, pour l'amour de son grand Nom, car Yahweh a résolu de faire de vous son peuple.
\VS{23}Et pour moi, Dieu me garde de pécher contre Yahweh, et de cesser de prier pour vous~! Je vous enseignerai le bon et le droit chemin.
\VS{24}Craignez seulement Yahweh, et servez-le en vérité, de tout votre cœur~; car vous avez vu les choses magnifiques qu'il a faites pour vous.
\VS{25}Mais si vous persévérez à faire le mal, vous serez détruits vous et votre roi.
\Chap{13}
\TextTitle{Impatience et désobéissance de Saül~; la royauté lui sera enlevée}
\VerseOne{}Saül régna un an sur Israël et après deux années,
\VS{2}Saül choisit trois mille hommes d'Israël, deux milles avec lui à Micmasch, et sur la montagne de Béthel, et mille étaient avec Jonathan à Guibea de Benjamin. Il renvoya le reste du peuple, chacun à sa tente.
\VS{3}Et Jonathan battit le poste des Philistins qui était à Guéba, et les Philistins en furent informés. Et Saül fit sonner le shofar dans tout le pays, en disant~: Que les Hébreux écoutent~!
\VS{4}Tout Israël apprit donc que Saül avait battu le poste des Philistins, et Israël se rendit odieux aux Philistins. Et le peuple fut convoqué auprès de Saül, à Guilgal.
\VS{5}Les Philistins s'assemblèrent pour combattre Israël, ayant trente mille chars et six mille cavaliers, et le peuple était aussi nombreux que le sable au bord de la mer, tant il était en grand nombre. Ils allèrent prendre position à Micmasch, à l'orient de Beth-Aven.
\VS{6}Les hommes d'Israël furent pris d'une grande angoisse, car ils étaient oppressés, c'est pourquoi le peuple se cacha dans les cavernes, dans les buissons, dans les rochers, dans les tours et dans des citernes.
\VS{7}Les Hébreux passèrent le Jourdain pour aller au pays de Gad et de Galaad. Saül était encore à Guilgal, aussi tout le peuple effrayé le rejoignit.
\VS{8}Il attendit sept jours selon le terme fixé par Samuel. Mais Samuel ne venait pas à Guilgal et le peuple se dispersait.
\VS{9}Et Saül dit~: Amenez-moi un holocauste et des sacrifices d'offrande de paix. Et il offrit l'holocauste.
\VS{10}Comme il achevait d'offrir l'holocauste, Samuel arriva, et Saül sortit au-devant de lui pour le saluer.
\VS{11}Et Samuel lui dit~: Qu'as-tu fait~? Saül répondit~: Lorsque j'ai vu que le peuple se dispersait, que tu ne venais pas au jour fixé, et que les Philistins étaient assemblés à Micmasch,
\VS{12}j'ai dit~: Les Philistins descendront maintenant contre moi à Guilgal, et je n'ai pas supplié Yahweh~! Je me suis maîtrisé un temps, mais j'ai fini par offrir l'holocauste.
\VS{13}Samuel répondit à Saül~: C'est en insensé que tu as agi, car tu n'as pas gardé le commandement que Yahweh, ton Dieu, t'avait donné~; car Yahweh aurait maintenu à jamais ta royauté sur Israël.
\VS{14}Et maintenant ta royauté ne subsistera pas. Yahweh s'est choisi un homme selon son cœur, et Yahweh l'a destiné à être le chef de son peuple parce que tu n'as pas respecté le commandement de Yahweh.
\TextTitle{Saül et ses hommes à Guibea de Benjamin}
\VS{15}Puis Samuel se leva, et monta de Guilgal à Guibea de Benjamin. Et Saül passa en revue le peuple qui se trouvait avec lui, qui fut d'environ six cents hommes.
\VS{16}Or Saül vint s'établir avec son fils Jonathan, et le peuple qui était sous ses ordres à Guéba de Benjamin, et les Philistins étaient campés à Micmasch.
\VS{17}Les Philistins sortirent du camp en trois divisions pour ravager~: L'une de ces divisions prit le chemin d'Ophra, vers le pays de Schual~;
\VS{18}l'autre division prit le chemin de Beth-Horon~; et la troisième prit le chemin de la frontière qui regarde vers la vallée de Tseboïm, du côté du désert.
\VS{19}Or dans tout le pays d'Israël, il ne se trouvait aucun forgeron~; car les Philistins avaient dit~: Empêchons les Hébreux de faire des épées ou des lances.
\VS{20}C'est pourquoi chaque homme descendait vers les Philistins pour aiguiser son soc, son hoyau, sa hache, et sa bêche,
\VS{21}lorsque le tranchant des bêches, des hoyaux, des tridents, et des haches était émoussé, même pour redresser un aiguillon.
\VS{22}C'est pourquoi il arriva qu'au jour du combat, nul n'avait d'épée ni de lance dans toute l'armée qui était avec Saül et Jonathan~; si ce n'est Saül lui-même et Jonathan, son fils.
\VS{23}Un poste de Philistins s'établit au passage de Micmasch.
\Chap{14}
\TextTitle{Courage de Jonathan}
\VerseOne{}Or il arriva un jour que Jonathan, fils de Saül, dit au garçon qui portait ses armes~: Viens et allons jusqu'au poste de garde des Philistins qui est au-delà de ce lieu-là. Mais il ne dit rien à son père.
\VS{2}Saül se tenait à l'extrémité de Guibea sous un grenadier, à Migron, entouré d'environ six cents hommes.
\VS{3}Achija, fils d'Achithub, frère d'I-Kabod, fils de Phinées, fils d'Eli, prêtre de Yahweh à Silo, portait l'éphod. Et le peuple ignorait que Jonathan s'en était allé.
\VS{4}Or entre les passages par lesquels Jonathan voulait arriver au poste de garde des Philistins, il y avait une dent de rocher d'un côté et une dent de rocher de l'autre, l'une s'appelait Botsets et l'autre Séné.
\VS{5}L'une de ces dents était située du côté nord vis-à-vis de Micmasch, et l'autre, du côté sud vis-à-vis de Guéba.
\VS{6}Jonathan dit au garçon qui portait ses armes~: Viens, poursuivons jusqu'au poste de garde de ces incirconcis. Peut-être Yahweh agira-t-il pour nous, car on ne saurait empêcher Yahweh de délivrer avec peu ou beaucoup de gens.
\VS{7}Et celui qui portait ses armes lui dit~: Fais tout ce que tu as dans le cœur, vas-y, voici je serai avec toi où tu voudras.
\VS{8}Et Jonathan lui dit~: Allons vers ces hommes, et montrons-nous à eux.
\VS{9}S'ils nous disent~: Attendez jusqu'à ce que nous venions à vous~! Alors nous resterons sur place, et nous ne monterons pas vers eux.
\VS{10}Mais s'ils disent~: Montez vers nous~! Nous irons, car Yahweh les aura livrés entre nos mains. Que cela soit pour nous un signe.
\VS{11}Ils se montrèrent donc tous deux au poste de garde des Philistins, et les Philistins dirent~: Voici, les Hébreux sortent des trous où ils s'étaient cachés.
\VS{12}Et ceux du poste de garde dirent à Jonathan, et à celui qui portait ses armes~: Montez vers nous, nous avons quelque chose à vous apprendre. Alors Jonathan dit à celui qui portait ses armes~: Monte avec moi, car Yahweh les a livrés entre les mains d'Israël.
\VS{13}Et Jonathan monta en s'aidant des mains et des pieds~; celui qui portait ses armes le suivit. Puis ceux du poste de garde tombèrent sous les coups de Jonathan, et celui qui portait ses armes les tuait à sa suite.
\VS{14}Dans cette première victoire, Jonathan et celui qui portait ses armes, tuèrent environ vingt hommes, dans un espace d'environ une moitié d'un arpent de terre.
\VS{15}Et il y eut un grand effroi au camp, à la campagne, et parmi tout le peuple~; le poste de garde aussi, et ceux qui avaient ravagé furent effrayés et le pays fut tellement troublé que cela fut comme une frayeur de Dieu.
\TextTitle{Victoire d'Israël}
\VS{16}Et les sentinelles de Saül, qui étaient à Guibea de Benjamin, regardèrent, et voici, la multitude était en un si grand désordre qu'elle s'écroulait et s'en allait en s'entre-tuant.
\VS{17}Alors Saül dit au peuple qui était avec lui~: Faites donc la revue et voyez qui s'en est allé du milieu de nous. Ils firent donc la revue, et voici Jonathan n'y était pas, ni celui qui portait ses armes.
\VS{18}Et Saül dit à Achija~: Fais approcher l'arche de Dieu~! Car l'arche de Dieu était, en ce jour-là, avec les enfants d'Israël.
\VS{19}Mais il arriva que pendant que Saül parlait au prêtre, le tumulte venant du camp des Philistins augmentait de plus en plus~; et Saül dit au prêtre~: Retire ta main~!
\VS{20}Saül et tout le peuple se rassemblèrent, et vinrent au champ de bataille~; les Philistins tournaient les épées les uns contre les autres, la confusion était extrême.
\VS{21}Les Hébreux, qui étaient montés auparavant dans le camp des Philistins et qui étaient dispersés, se joignirent aux Israélites qui étaient avec Saül et Jonathan.
\VS{22}Et tous les Israélites qui s'étaient cachés dans la montagne d'Ephraïm, ayant appris que les Philistins s'enfuyaient, les poursuivirent aussi pour les combattre.
\VS{23}Ce jour-là, Yahweh délivra Israël, et le combat s'étendit jusqu'à Beth-Aven.
\TextTitle{Jonathan épargné des conséquences du vœu de Saül}
\VS{24}Les hommes d'Israël furent épuisés cette journée-là. Mais Saül avait fait jurer le peuple, en disant~: Maudit soit l'homme qui prendra de la nourriture avant le soir, avant que je me sois vengé de mes ennemis~! Et le peuple ne goûta pas de nourriture.
\VS{25}Tout le peuple arriva dans une forêt, où il y avait du miel à la surface du sol.
\VS{26}Lorsque le peuple entra dans la forêt, il vit le miel qui coulait~; mais nul ne porta la main à sa bouche, car le peuple craignait le serment.
\VS{27}Or Jonathan n'avait pas entendu son père lorsqu'il avait fait jurer le peuple~; il étendit le bout du bâton qu'il avait à la main, le trempa dans un rayon de miel et porta sa main à sa bouche~; et ses yeux furent éclaircis.
\VS{28}Alors quelqu'un du peuple lui dit~: Ton père a fait jurer expressément le peuple en disant~: Maudit soit l'homme qui mangera aujourd'hui quelque chose~! Quoique le peuple soit très fatigué.
\VS{29}Et Jonathan dit~: Mon père trouble le peuple~; voyez comment mes yeux sont éclaircis après avoir goûté un peu de ce miel.
\VS{30}Combien plus si le peuple s'était aujourd'hui restauré du butin de ses ennemis, la défaite des Philistins n'aurait-elle pas été plus grande~?
\VS{31}En ce jour-là donc ils frappèrent les Philistins de Micmasch à Ajalon. Le peuple était très fatigué.
\VS{32}Puis il se jeta sur le butin, il prit des brebis, des bœufs, et des veaux, et les égorgea sur la terre, et le peuple les mangeait avec le sang.
\VS{33}On le rapporta à Saül, en disant~: Voici, le peuple pèche contre Yahweh, en mangeant avec le sang. Et il dit~: Vous avez péché~; roulez-moi ici une grosse pierre.
\VS{34}Et Saül dit: Allez parmi le peuple, et dites à chacun d'amener son bœuf et ses brebis, vous les égorgerez ici. Vous les mangerez, et vous ne pécherez plus contre Yahweh, en mangeant avec le sang. Et chacun amena, cette nuit-là, son bœuf à la main, et ils les égorgèrent.
\VS{35}Saül bâtit un autel à Yahweh~; ce fut le premier autel qu'il bâtit à Yahweh.
\VS{36}Puis Saül dit~: Descendons et poursuivons de nuit les Philistins, afin de les piller jusqu'au matin, et n'en laissons pas un de reste. Ils lui répondirent~: Fais tout ce qui te semble bon. Mais le prêtre dit~: Approchons-nous d'abord de Dieu.
\VS{37}Alors Saül consulta donc Dieu en disant~: Descendrai-je à la poursuite des Philistins~? Les livreras-tu entre les mains d'Israël~? Mais il ne lui répondit pas ce jour-là.
\VS{38}Et Saül dit~: Approchez ici, vous tous les chefs du peuple, recherchez et voyez par qui ce péché est arrivé aujourd'hui.
\VS{39}Car Yahweh est vivant, lui qui délivre Israël, quand il s'agirait de mon fils, Jonathan, il en mourrait. Mais dans tout le peuple, personne ne lui répondit.
\VS{40}Puis il dit à tout Israël~: Mettez-vous d'un côté, et nous serons de l'autre, moi et mon fils, Jonathan. Le peuple répondit à Saül~: Fais ce qui te semble bon.
\VS{41}Et Saül dit à Yahweh, le Dieu d'Israël~: Fais connaître la vérité. Jonathan et Saül furent désignés, et le peuple fut écarté.
\VS{42}Et Saül dit~: Jetez le sort entre moi et Jonathan, mon fils. Et Jonathan fut désigné.
\VS{43}Alors Saül dit à Jonathan~: Déclare-moi ce que tu as fait. Et Jonathan lui déclara, et dit~: Il est vrai que j'ai goûté un peu de miel avec le bout de mon bâton que j'avais à la main~: Me voici, je mourrai.
\VS{44}Et Saül dit~: Que Dieu agisse à mon égard comme il le veut, tu mourras, tu mourras\FTNT{Répétition des mots «~tu mourras, tu mourras~», voir commentaire en Ge. 2:17.}, Jonathan.
\VS{45}Mais le peuple dit à Saül~: Jonathan qui a accompli cette grande délivrance en Israël mourrait-il~? Garde-toi bien~! Yahweh est vivant~! Il ne tombera pas à terre un seul des cheveux de sa tête, car c'est avec Dieu qu'il a agi en ce jour. Le peuple délivra Jonathan de la mort.
\VS{46}Saül renonça à poursuivre les Philistins, qui regagnèrent leur pays.
\TextTitle{Les guerres sous le règne de Saül}
\VS{47}Après que Saül eut pris possession de la royauté sur Israël, il fit la guerre de tous côtés contre ses ennemis, Moab, les fils d'Ammon, Edom, les rois de Tsoba et les Philistins~; partout où il se tournait, il était vainqueur.
\VS{48}Il manifesta sa puissance en frappant Amalek et délivra Israël de la main de ceux qui le pillaient.
\VS{49}Les fils de Saül étaient Jonathan, Jischvi et Malkischua~; et quant aux noms de ses deux filles, le nom de l'aînée était Mérab, et la plus jeune, Mical.
\VS{50}Et le nom de la femme de Saül était Achinoam, fille d'Achimaats~; et le nom du chef de son armée était Abner, fils de Ner, oncle de Saül.
\VS{51}Kis, père de Saül, et Ner, père d'Abner, étaient fils d'Abiel.
\VS{52}La guerre contre les Philistins fut violente durant toute la vie de Saül~; et chaque fois que Saül remarquait un homme fort et vaillant, il le prenait auprès de lui.
\Chap{15}
\TextTitle{Saül désobéit une fois de plus}
\VerseOne{}Et Samuel dit à Saül~: Yahweh m'a envoyé pour t'oindre afin que tu sois roi sur son peuple, sur Israël~; maintenant donc, écoute les paroles de Yahweh.
\VS{2}Ainsi parle Yahweh des armées~: Je me rappelle de ce qu'Amalek a fait à Israël, comment il s'opposa à lui sur le chemin, à sa sortie d'Egypte.
\VS{3}Va maintenant, et frappe Amalek, et détruisez à la façon de l'interdit tout ce qui lui appartient~; ne l'épargne pas, mais fais mourir hommes et femmes, enfants et nourrissons, bœufs et menu bétail, chameaux et ânes.
\VS{4}Saül donc convoqua le peuple, et en fit la revue à Thelaïm~: Il y avait deux cent mille hommes de pied, et dix mille hommes de Juda.
\VS{5}Et Saül marcha jusqu'à la ville d'Amalek, et mit une embuscade dans la vallée.
\VS{6}Et Saül dit aux Kéniens~: Allez, retirez-vous, séparez-vous des Amalécites, de peur que je ne vous détruise avec eux~; car vous avez agi avec bonté envers tous les enfants d'Israël, quand ils montèrent d'Egypte. Et les Kéniens se séparèrent des Amalécites.
\VS{7}Et Saül frappa les Amalécites depuis Havila jusqu'à Schur, qui est face à l'Egypte.
\VS{8}Il fit passer tout le peuple au fil de l'épée, le dévouant par interdit~; mais il épargna Agag, roi d'Amalek.
\VS{9}Saül et le peuple épargnèrent Agag, les meilleures brebis, les meilleurs bœufs, les bêtes grasses, les agneaux, ce qu'il y avait de meilleur~; ils ne voulurent pas les dévouer par interdit, détruisant seulement tout ce qui était chétif et méprisable.
\VS{10}Alors la parole de Yahweh vint à Samuel en disant~:
\VS{11}Je me repens d'avoir établi Saül pour roi, car il s'est détourné de moi et n'a pas exécuté mes paroles. Samuel fut très irrité, et il cria à Yahweh toute la nuit.
\TextTitle{Yahweh rejette Saül}
\VS{12}Puis Samuel se leva de bon matin pour aller rencontrer Saül. On lui rapporta que Saül, venu à Carmel, s'est érigé un monument, puis s'en est retourné, pour enfin descendre à Guilgal.
\VS{13}Samuel se rendit auprès de Saül, et Saül lui dit~: Sois béni de Yahweh~! J'ai exécuté la parole de Yahweh.
\VS{14}Samuel dit~: Quel est donc ce bêlement de brebis qui parvient à mes oreilles, et ce mugissement de bœufs que j'entends~?
\VS{15}Et Saül répondit~: Ils les ont amenés de chez les Amalécites, car le peuple a épargné les meilleures brebis et les meilleurs bœufs, pour les sacrifier à Yahweh, ton Dieu~; et nous avons détruit le reste, nous l'avons dévoué par interdit.
\VS{16}Samuel dit à Saül~: Laisse-moi et je te déclarerai ce que Yahweh m'a dit cette nuit. Et il lui répondit~: Parle~!
\VS{17}Samuel dit~: N'est-il pas vrai que, quand tu étais petit à tes yeux, tu as été fait chef des tribus d'Israël, et Yahweh t'a oint pour roi sur Israël~?
\VS{18}Yahweh t'avait envoyé dans cette expédition, et t'avait dit~: Va, et détruis ces pécheurs, les Amalécites, et fais-leur la guerre jusqu'à ce qu'ils soient exterminés.
\VS{19}Pourquoi n'as-tu pas obéi à la voix de Yahweh, t'es-tu jeté sur le butin, et as-tu fait ce qui déplaît à Yahweh~?
\VS{20}Et Saül répondit à Samuel~: J'ai pourtant obéi à la voix de Yahweh, et je suis allé par le chemin par lequel Yahweh m'a envoyé. Et j'ai amené Agag, roi des Amalécites, et j'ai dévoué les Amalécites, par interdit~;
\VS{21}mais le peuple a pris des brebis, des bœufs, du butin, comme prémices de ce qui devait être dévoué, pour le sacrifier à Yahweh, ton Dieu, à Guilgal.
\VS{22}Samuel répondit~: Yahweh prend-il plaisir aux holocaustes et aux sacrifices, autant qu'à l'obéissance à sa voix~? Voici, l'obéissance vaut mieux que les sacrifices, et l'observation de sa parole vaut mieux que la graisse des béliers.
\VS{23}Car la rébellion est un péché autant que la divination, et la résistance ne l'est pas moins que l'idolâtrie et les théraphim. Puisque tu as rejeté la parole de Yahweh, il te rejette aussi afin que tu ne sois plus roi.
\VS{24}Et Saül répondit à Samuel~: J'ai péché parce que j'ai transgressé le commandement de Yahweh, ainsi que tes paroles~; car je craignais le peuple et j'ai obéi à sa voix.
\VS{25}Mais maintenant, je te prie, pardonne-moi mon péché, et reviens avec moi, que je me prosterne devant Yahweh.
\VS{26}Et Samuel dit à Saül~: Je n'irai pas avec toi~; parce que tu as rejeté la parole de Yahweh, Yahweh te rejette afin que tu ne sois plus roi d'Israël.
\VS{27}Comme Samuel se détournait pour s'en aller, Saül le saisit par le pan de son manteau qui se déchira.
\VS{28}Alors Samuel lui dit~: Yahweh déchire aujourd'hui le royaume d'Israël de dessus toi, et le donne à un autre, qui est meilleur que toi.
\VS{29}En effet, le Puissant d'Israël ne ment pas, il ne se repent pas, car il n'est pas un homme pour se repentir.
\VS{30}Et Saül répondit~: J'ai péché~! Mais honore-moi maintenant, je te prie, en présence des anciens de mon peuple, et en présence d'Israël, et reviens avec moi, et je me prosternerai devant Yahweh, ton Dieu.
\VS{31}Samuel retourna et suivit Saül, et Saül se prosterna devant Yahweh.
\VS{32}Puis Samuel dit~: Amenez-moi Agag, roi d'Amalek. Et Agag s'avança vers lui faisant le gracieux~; car Agag disait~: Certainement l'amertume de la mort est passée.
\VS{33}Mais Samuel dit~: Comme ton épée a privé les femmes de leurs enfants, ainsi ta mère entre les femmes sera privée d'enfants. Et Samuel mit Agag en pièces devant Yahweh à Guilgal.
\VS{34}Puis il s'en alla à Rama, et Saül monta dans sa maison à Guibea de Saül.
\VS{35}Et Samuel n'alla plus voir Saül jusqu'au jour de sa mort~; car Samuel pleurait sur Saül, de ce que Yahweh s'était repenti d'avoir établi Saül, roi sur Israël.
\Chap{16}
\TextTitle{Samuel envoyé à Bethléhem pour oindre David}
\VerseOne{}Yahweh dit à Samuel~: Jusqu'à quand mèneras-tu deuil sur Saül, vu que je l'ai rejeté, afin qu'il ne règne plus sur Israël~? Remplis ta corne d'huile, et viens~; je t'enverrai chez Isaï, Bethléhémite, car je me suis pourvu d'un de ses fils pour roi.
\VS{2}Et Samuel dit~: Comment irai-je~? Car Saül l'apprendra et il me tuera. Et Yahweh répondit~: Tu emmèneras avec toi une jeune vache du troupeau, et tu diras~: Je suis venu pour sacrifier à Yahweh.
\VS{3}Et tu inviteras Isaï au sacrifice, et je te ferai savoir ce que tu auras à faire, et tu m'oindras celui que je te dirai.
\VS{4}Samuel fit donc comme Yahweh lui avait dit, et il alla à Bethléhem. Les anciens de la ville tout effrayés accoururent au-devant de lui et lui dirent~: Ton arrivée annonce-t-elle la paix~?
\VS{5}Et il répondit~: Soyez en paix~; je suis venu pour sacrifier à Yahweh. Sanctifiez-vous, et venez avec moi au sacrifice. Il fit sanctifier aussi Isaï et ses fils, et les invita au sacrifice.
\VS{6}A son entrée, il remarqua Eliab, et se dit~: L'oint de Yahweh est certainement devant lui.
\VS{7}Mais Yahweh dit à Samuel~: Ne prête pas attention à son apparence ni à la hauteur de sa taille, car je l'ai rejeté. Yahweh ne considère pas ce que l'homme considère~; car l'homme considère ce que voient ses yeux, mais Yahweh regarde au cœur.
\VS{8}Puis Isaï appela Abinadab, et le fit passer devant Samuel~; et il dit~: Yahweh n'a pas non plus choisi celui-ci.
\VS{9}Isaï fit passer Schamma~; et Samuel dit~: Yahweh n'a pas non plus choisi celui-ci.
\VS{10}Ainsi Isaï fit passer ses sept fils devant Samuel~; et Samuel dit à Isaï~: Yahweh n'a pas choisi ceux-ci.
\VS{11}Puis Samuel dit à Isaï~: Sont-ce là tous tes garçons~? Et il dit~: Il reste encore le plus jeune, seulement, il fait paître les brebis. Alors Samuel dit à Isaï~: Envoie-le chercher, car nous ne retournerons pas avant qu'il ne soit venu ici.
\VS{12}Il le fit donc venir. Il était roux, avec de beaux yeux et une belle apparence. Et Yahweh dit à Samuel~: Lève-toi, et oins-le, car c'est celui que j'ai choisi~!
\VS{13}Alors Samuel prit la corne d'huile, et l'oignit au milieu de ses frères. Et depuis ce jour-là, l'Esprit de Yahweh saisit David. Et Samuel se leva, et s'en alla à Rama.
\TextTitle{David entre au service de Saül}
\VS{14}L'Esprit de Yahweh se retira de Saül, et un mauvais esprit\FTNT{Saül a été frappé d'un esprit d'égarement (2 Th. 2:9-12). Voir commentaires en Ge. 6:3 et Mt. 12:31.} envoyé par Yahweh le troublait.
\VS{15}Les serviteurs de Saül lui dirent~: Voici, un mauvais esprit envoyé de Dieu te tourmente.
\VS{16}Que le roi, notre seigneur, parle~! Tes serviteurs sont devant toi. Ils chercheront un homme qui sache jouer de la harpe~; et quand le mauvais esprit envoyé par Dieu sera sur toi, il jouera de sa main, et tu seras soulagé.
\VS{17}Saül répondit à ses serviteurs~: Trouvez-moi un homme qui sache bien jouer et amenez-le-moi.
\VS{18}L'un des serviteurs répondit~: Voici, j'ai vu l'un des fils d'Isaï, le Bethléhémite, qui sait jouer des instruments, il est fort et vaillant, c'est un guerrier qui parle bien, bel homme, et Yahweh est avec lui.
\VS{19}Alors Saül envoya des messagers à Isaï, pour lui dire~: Envoie-moi David, ton fils, qui est avec les brebis.
\VS{20}Isaï prit un âne, qu'il chargea de pain, et une outre de vin, et un jeune chevreau, et les envoya par David, son fils, à Saül.
\VS{21}David arrivé chez Saül, se présenta devant lui~; et Saül l'aima beaucoup, et il lui servit à porter ses armes.
\VS{22}Saül fit dire à Isaï~: Je te prie que David demeure à mon service, car il a trouvé grâce devant moi.
\VS{23}Il arrivait donc que quand le mauvais esprit envoyé de Dieu était sur Saül, David prenait la harpe, et en jouait de sa main~; et Saül en était soulagé, parce que le mauvais esprit se retirait de lui.
\Chap{17}
\TextTitle{Goliath sème la terreur dans le camp d'Israël}
\VerseOne{}Or les Philistins réunirent leurs armées pour faire la guerre, et ils se rassemblèrent à Soco, qui est de Juda~; et ils campèrent entre Soco et Azéka, à Ephès-Dammim.
\VS{2}Saül et ceux d'Israël se rassemblèrent aussi~; et ils campèrent dans la vallée du chêne, et ils se mirent en ordre de bataille contre les Philistins.
\VS{3}Les Philistins étaient sur une montagne d'un côté, et les Israélites sur une montagne de l'autre côté~; de sorte que la vallée les séparait.
\VS{4}Il sortit du camp des Philistins un homme qui se présentait entre les deux armées, il s'appelait Goliath, de la ville de Gath, haut de six coudées et d'un empan.
\VS{5}Il avait un casque d'airain sur sa tête et était armé d'une cuirasse à écailles pesant cinq mille sicles d'airain.
\VS{6}Il avait aussi des jambières d'airain, et un javelot d'airain entre ses épaules.
\VS{7}Le bois de sa lance était comme une ensouple d'un tisserand, et le fer de sa lance pesait six cents sicles de fer. Celui qui portait son bouclier marchait devant lui.
\VS{8}Il se présenta donc, et cria aux troupes d'Israël rangées en bataille, et leur disait~: Pourquoi sortez-vous pour vous ranger en bataille~? Ne suis-je pas Philistin, et n'êtes-vous pas esclaves de Saül~? Choisissez l'un d'entre vous, et qu'il descende contre moi.
\VS{9}S'il peut me battre et qu'il me tue, nous serons vos esclaves~; mais si j'ai l'avantage sur lui, et que je le tue, vous serez nos esclaves, et vous nous serez asservis.
\VS{10}Le Philistin disait~: Je jette un défi en ce jour aux troupes rangées d'Israël~: Donnez-moi un homme, et nous combattrons ensemble.
\VS{11}Mais Saül et tous les Israélites ayant entendu les paroles du Philistin furent épouvantés et saisis d'une grande frayeur.
\VS{12}Or David était le fils d'un homme Ephratien, de Bethléhem de Juda, nommé Isaï, qui avait huit fils, et qui du temps de Saül, était vieux et mis au rang des personnes de qualité.
\VS{13}Et les trois fils aînés d'Isaï avaient suivi Saül à la guerre. Les noms de ses trois fils qui s'en étaient allés à la guerre étaient Eliab, le premier-né, Abinadab, le second, et Schamma, le troisième.
\VS{14}David était le plus jeune, et les trois plus grands suivaient Saül.
\VS{15}David allait et revenait d'auprès de Saül pour paître les brebis de son père à Bethléhem.
\VS{16}Et le Philistin, s'approchant le matin et le soir, se présenta pendant quarante jours.
\TextTitle{David prêt à affronter Goliath}
\VS{17}Isaï dit à David, son fils~: Prends maintenant pour tes frères un épha de ce blé rôti, et ces dix pains, et porte-les promptement au camp, à tes frères.
\VS{18}Tu porteras aussi ces dix fromages au chef de leur millier, tu t'informeras du bien-être de tes frères et tu m'en apporteras des nouvelles sûres.
\VS{19}Or Saül, et eux, et tous ceux d'Israël étaient dans la vallée du chêne, combattant contre les Philistins.
\VS{20}Et David se leva de bon matin, et laissa les brebis aux soins d'un gardien~; puis ayant pris sa charge, s'en alla, comme son père Isaï le lui avait ordonné, et il arriva au retranchement où l'armée sortait pour se ranger en bataille, et on poussait des cris de guerre.
\VS{21}Car les Israélites et les Philistins se rangèrent armée contre armée.
\VS{22}Alors David se déchargea de son bagage, le laissant entre les mains de celui qui gardait le bagage, et courut vers les rangs de l'armée. Aussitôt arrivé, il demanda à ses frères s'ils se portaient bien.
\VS{23}Et comme il parlait avec eux, le Philistin de Gath, nommé Goliath, sortit des rangs de l'armée des Philistins, se présenta entre les deux armées et proféra les mêmes paroles qu'il avait proférées auparavant, et David les entendit.
\VS{24}A la vue de cet homme, tous ceux d'Israël s'enfuirent devant lui, saisis d'une grande frayeur.
\VS{25}Et les Israélites disaient~: Avez-vous vu s'avancer cet homme~? Il est monté pour jeter un défi à Israël~! Mais si quelqu'un le tue, le roi le comblera de richesses, et lui donnera sa fille, et affranchira la maison de son père en Israël.
\VS{26}Alors David parla aux personnes qui étaient là avec lui, en disant~: Quel bien fera-t-on à l'homme qui frappera ce Philistin, et qui ôtera l'opprobre de dessus Israël~? Car qui est ce Philistin, cet incirconcis, pour insulter l'armée du Dieu vivant~?
\VS{27}Et le peuple lui répéta ces mêmes paroles, et lui dit~: C'est le bien qu'on fera à l'homme qui l'aura tué.
\VS{28}Et quand Eliab, son frère aîné, entendit qu'il parlait à ces personnes, sa colère s'enflamma contre David, et il lui dit~: Pourquoi es-tu descendu, et à qui as-tu laissé ce peu de brebis au désert~? Je connais ton orgueil et la malice de ton cœur, car tu es descendu pour voir la bataille.
\VS{29}Et David répondit~: Qu'ai-je donc fait~? Ne puis-je pas parler ainsi~?
\VS{30}Puis il se détourna de lui vers un autre, et lui posa les mêmes questions. Et le peuple lui répondit comme la première fois.
\VS{31}Les paroles que David avait dites furent entendues et rapportées devant Saül qui le fit venir.
\VS{32}David dit à Saül~: Que personne ne perde courage à cause de ce Philistin~! Ton serviteur ira et se battra contre lui.
\VS{33}Mais Saül dit à David~: Tu ne peux aller te battre contre ce Philistin, car tu n'es qu'un enfant, et il est un homme de guerre depuis sa jeunesse.
\VS{34}David répondit à Saül~: Ton serviteur faisait paître les brebis de son père, quand un lion ou un ours venait emporter une brebis du troupeau,
\VS{35}je le poursuivais, je le frappais, et j'arrachais la brebis de sa gueule. S'il se jetait sur moi, je le saisissais par la mâchoire, je le frappais, et je le tuais.
\VS{36}Ton serviteur a tué et le lion, et l'ours, et ce Philistin, cet incirconcis, sera comme l'un d'eux, car il a déshonoré l'armée du Dieu vivant.
\VS{37}David dit encore~: Yahweh qui m'a délivré de la griffe du lion et de la patte de l'ours me délivrera de la main de ce Philistin. Alors Saül dit à David~: Va, et que Yahweh soit avec toi~!
\TextTitle{David tue Goliath~; les Philistins sont battus}
\VS{38}Saül fit revêtir David de ses vêtements, et lui mit son casque d'airain sur sa tête, et lui fit endosser une cuirasse.
\VS{39}Puis David ceignit l'épée par-dessus ses vêtements, et voulut marcher, car il n'avait pas encore essayé. Et David dit à Saül~: Je ne saurais marcher ainsi, je ne l'ai jamais essayé. Et il s'en débarrassa.
\VS{40}Alors il prit en main son bâton, et se choisit dans le torrent cinq pierres bien polies, et les mit dans le sac de berger et dans sa poche. Puis sa fronde en main, il s'approcha du Philistin.
\VS{41}Le Philistin aussi s'avança et s'approcha lentement de David, précédé de l'homme qui portait son bouclier.
\VS{42}Le Philistin regarda, et lorsqu'il vit David, il le méprisa, car ce n'était qu'un jeune garçon, roux et beau de figure.
\VS{43}Le Philistin dit à David~: Suis-je un chien, pour que tu viennes contre moi avec des bâtons~? Et le Philistin maudit David par ses dieux.
\VS{44}Le Philistin ajouta~: Viens vers moi et je donnerai ta chair aux oiseaux du ciel et aux bêtes des champs.
\VS{45}Et David dit au Philistin~: Tu marches contre moi avec l'épée, la lance, et le javelot~; mais moi, je marche contre toi au Nom de Yahweh des armées, le Dieu de l'armée d'Israël, que tu as blasphémé.
\VS{46}Aujourd'hui Yahweh te livrera entre mes mains, je t'abattrai, je te couperai la tête~; aujourd'hui je donnerai les cadavres du camp des Philistins aux oiseaux du ciel, et aux animaux de la terre. Et toute la terre saura qu'Israël a un Dieu.
\VS{47}Et toute cette assemblée saura que Yahweh ne délivre pas par l'épée ni par la lance. Car la victoire est à Yahweh, qui vous livrera entre nos mains.
\VS{48}Et il arriva que comme le Philistin se fut levé, et qu'il s'approchait pour rencontrer David, David se hâta, et courut au lieu du combat pour rencontrer le Philistin.
\VS{49}Alors David mit la main à son sac, prit une pierre, et la lança avec sa fronde~; il frappa tellement le Philistin au front que la pierre s'enfonça dans son front, il tomba le visage contre terre.
\VS{50}Ainsi avec une fronde et une pierre, David fut plus fort que le Philistin~; il le frappa, et le tua, sans avoir une épée à la main.
\VS{51}Alors David courut, se jeta sur le Philistin, prit son épée, la tira de son fourreau, le tua, et lui coupa la tête. Les Philistins, voyant que leur héros était mort, prirent la fuite.
\VS{52}Alors les hommes d'Israël et de Juda se levèrent, et poussèrent des cris de guerre, et poursuivirent les Philistins jusqu'à la vallée, et jusqu'aux portes d'Ekron. Les Philistins blessés à mort tombèrent dans le chemin de Schaaraïm, jusqu'à Gath, et jusqu'à Ekron.
\VS{53}Et les fils d'Israël revinrent de la poursuite des Philistins, et pillèrent leurs camps.
\VS{54}David prit la tête du Philistin et la porta à Jérusalem, et il mit aussi dans sa tente les armes du Philistin.
\VS{55}Quand Saül vit David sortant à la rencontre du Philistin, il dit à Abner, chef de l'armée~: Abner, de qui ce jeune homme est-il le fils~? Abner répondit~: Que ton âme vive, ô roi~! Je n'en sais rien.
\VS{56}Le roi lui dit~: Informe-toi de qui ce jeune garçon est fils.
\VS{57}Et quand David fut de retour après avoir tué le Philistin, Abner le prit, et le mena devant Saül. David avait la tête du Philistin à la main.
\VS{58}Et Saül lui dit~: Jeune garçon, de qui es-tu fils~? David répondit~: Je suis fils d'Isaï, Bethléhémite, ton serviteur.
\Chap{18}
\TextTitle{Alliance entre Jonathan et David}
\VerseOne{}Or il arriva qu'aussitôt que David eut achevé de parler à Saül, l'âme de Jonathan fut attachée à l'âme de David, et Jonathan l'aima comme son âme.
\VS{2}Ce jour-là donc Saül le retint, et ne lui permit plus de retourner à la maison de son père.
\VS{3}Alors Jonathan fit alliance avec David, parce qu'il l'aimait comme son âme.
\VS{4}Jonathan se dépouilla du manteau qu'il portait, et le donna à David, avec ses habits, jusqu'à son épée, son arc, et sa ceinture.
\TextTitle{Saül est jaloux de David et cherche à le tuer}
\VS{5}David, envoyé par Saül, réussissait partout où il allait, de sorte que Saül l'établit sur son armée, et il plaisait à tout le peuple, même aux serviteurs de Saül.
\VS{6}Or il arriva que, comme ils revenaient, lors du retour de David après qu'il eut tué le Philistin, des femmes sortirent de toutes les villes d'Israël, en chantant et dansant devant le roi Saül, avec des tambourins, des triangles et en poussant des cris de joie.
\VS{7}Les femmes chantaient, se répondant les unes aux autres, en disant~: Saül a frappé ses mille, et David ses dix mille.
\VS{8}Saül fut très irrité, car cette parole lui déplut. Il dit~: Elles en ont donné dix mille à David, et à moi mille~! Il ne lui manque plus que le royaume.
\VS{9}Depuis ce jour-là, Saül regardait David d'un mauvais œil.
\VS{10}Et il arriva, dès le lendemain que l'esprit malin envoyé de Dieu saisit Saül, et il faisait le prophète au milieu de la maison, et David joua de sa main, comme les autres jours, et Saül avait une lance dans sa main.
\VS{11}Saül jeta sa lance, se disant~: Je frapperai David, contre le mur. Mais David l'évita deux fois.
\VS{12}Saül avait peur de la présence de David, parce que Yahweh était avec David, et qu'il s'était retiré de Saül.
\VS{13}C'est pourquoi Saül éloigna David de lui, et l'établit chef de mille. Et David allait et venait devant le peuple.
\VS{14}David réussissait dans tout ce qu'il entreprenait, car Yahweh était avec lui.
\VS{15}Saül, voyant que David réussissait beaucoup, avait peur de sa présence.
\VS{16}Mais tout Israël et Juda aimaient David, parce qu'il allait et venait devant eux.
\TextTitle{David épouse Mical, fille de Saül}
\VS{17}Saül dit à David~: Voici, je te donnerai Mérab, ma fille aînée, pour femme~; sois pour moi un fils vaillant, et conduis les guerres de Yahweh. Car Saül disait~: Que ma main ne le touche pas, mais que ce soit celle des Philistins.
\VS{18}David répondit à Saül~: Qui suis-je, et quelle est ma vie, et la famille de mon père en Israël, pour que je devienne gendre du roi~?
\VS{19}Or il arriva qu'au temps où l'on devait donner Mérab, fille de Saül, à David, elle fut donnée pour femme à Adriel de Mehola.
\VS{20}Mais Mical, fille de Saül, aima David~; ce qu'on rapporta à Saül, et la chose lui plut.
\VS{21}Et Saül dit~: Je la lui donnerai, afin qu'elle soit pour lui un piège, et que par ce moyen la main des Philistins l'atteigne. Saül donc dit à David pour la seconde fois~: Tu seras aujourd'hui mon gendre.
\VS{22}Et Saül ordonna à ses serviteurs de parler à David en secret, et de lui dire~: Voici, le roi prend plaisir en toi, et tous ses serviteurs t'aiment~; sois donc maintenant gendre du roi.
\VS{23}Les serviteurs de Saül répétèrent toutes ces paroles à David, et David répondit~: Pensez-vous qu'il soit facile de devenir le gendre du roi, moi qui suis un homme pauvre, et peu important~?
\VS{24}Et les serviteurs de Saül le lui rapportèrent, et lui dirent~: David a tenu tel discours.
\VS{25}Saül dit~: Vous parlerez ainsi à David~: Le roi ne désire pas de dot, mais cent prépuces de Philistins, afin d'être vengé de ses ennemis. Or Saül avait pour but de faire tomber David aux mains des Philistins.
\VS{26}Les serviteurs de Saül rapportèrent tous ces discours à David, à qui il plut de devenir gendre du roi. Le temps n'était pas encore écoulé,
\VS{27}que David se leva, et s'en alla, lui et ses gens, et tua deux cents hommes parmi les Philistins~; il apporta leurs prépuces, et on les livra au complet au roi, afin qu'il devienne gendre du roi. Alors Saül lui donna pour femme Mical, sa fille.
\VS{28}Saül vit et comprit que Yahweh était avec David~; et Mical, fille de Saül, l'aimait.
\VS{29}Saül craignait David de plus en plus, et devint son ennemi toute sa vie durant.
\VS{30}Les chefs des Philistins firent des incursions, mais chaque fois qu'ils sortaient, David remportait du succès mieux que tous les serviteurs de Saül et son nom devint fort estimé.
\Chap{19}
\TextTitle{David échappe aux assauts de Saül}
\VerseOne{}Saül parla à Jonathan, son fils, et à tous ses serviteurs de faire mourir David.
\VS{2}Mais Jonathan, fils de Saül, avait une grande affection pour David. C'est pourquoi Jonathan le fit savoir à David, et lui dit~: Saül, mon père, cherche à te faire mourir. Maintenant donc, tiens-toi sur tes gardes jusqu'au matin, demeure dans un lieu secret, et cache-toi.
\VS{3}Je me tiendrai auprès de mon père, je sortirai dans le champ où tu seras~; car je parlerai de toi à mon père, je verrai ce qu'il en sera, et je te le rapporterai.
\VS{4}Jonathan parla favorablement de David à Saül, son père, et lui dit~: Que le roi ne pèche pas contre son serviteur David, car il n'a pas péché contre toi. Au contraire, il a agi pour ton bien~;
\VS{5}car il a exposé sa vie, il a tué le Philistin, et Yahweh a opéré une grande délivrance pour tout Israël. Tu l'as vu, et tu t'en es réjoui. Pourquoi donc pécherais-tu contre le sang innocent en faisant mourir David sans cause~?
\VS{6}Saül écouta la voix de Jonathan et jura~: Yahweh est vivant~! Il ne mourra pas.
\VS{7}Alors Jonathan appela David, et lui répéta toutes ces choses. Jonathan l'introduisit auprès de Saül, et il fut à son service comme auparavant.
\VS{8}La guerre ayant recommencé, David se mit en campagne et frappa les Philistins, et leur infligea une grande défaite, de sorte qu'ils prirent la fuite.
\VS{9}Le mauvais esprit envoyé de Yahweh fut sur Saül, comme il était assis dans sa maison, ayant sa lance à la main. David jouait de sa main,
\VS{10}Saül voulut frapper David avec sa lance contre le mur~; mais il se glissa de devant Saül, qui frappa le mur de la lance. David s'enfuit et s'échappa cette nuit-là.
\VS{11}Saül envoya des messagers à la maison de David pour le garder, et le faire mourir au matin. Mical, femme de David, l'en informa, en disant~: Si tu ne te sauves pas, demain on te fera mourir.
\VS{12}Mical fit descendre David par une fenêtre, et ainsi il s'en alla et s'enfuit.
\VS{13}Ensuite Mical prit un théraphim, qu'elle plaça dans le lit~; elle mit une peau de chèvre à son chevet et l'enveloppa d'une couverture.
\VS{14}Lorsque Saül envoya des gens pour prendre David, elle dit~: Il est malade.
\VS{15}Saül envoya encore des gens pour prendre David, en leur disant~: Apportez-le-moi dans son lit, afin que je le fasse mourir.
\VS{16}Ces gens donc vinrent, et voici, un théraphim était au lit, et la peau de chèvre à son chevet.
\VS{17}Saül dit à Mical~: Pourquoi m'as-tu trompé de la sorte, et as-tu laissé aller mon ennemi, de sorte qu'il s'est échappé~? Et Mical répondit à Saül~: Il m'a dit~: Laisse-moi aller~; pourquoi te tuerais-je~?
\VS{18}C'est ainsi que David prit la fuite et qu'il s'échappa. Il se rendit auprès de Samuel à Rama, et lui raconta tout ce que Saül lui avait fait. Puis il s'en alla avec Samuel, et ils demeurèrent à Najoth.
\VS{19}On le rapporta à Saül, en lui disant~: Voici, David est à Najoth, en Rama.
\VS{20}Alors Saül envoya des gens pour s'emparer de David. Ils virent une assemblée de prophètes qui prophétisaient, et Samuel, à leur tête, se tenait là. L'Esprit de Dieu saisit les envoyés de Saül, qui prophétisèrent aussi.
\VS{21}On le rapporta à Saül, qui envoya d'autres gens, et eux aussi prophétisèrent. Saül en envoya encore pour la troisième fois et ils prophétisèrent également.
\VS{22}Alors il alla lui-même à Rama. Arrivé à la grande citerne qui est à Sécou, il s'informa disant~: Où sont Samuel et David~? Et on lui répondit~: Ils sont à Najoth, en Rama.
\VS{23}Il se dirigea vers Najoth, en Rama. Et l'Esprit de Dieu le saisit à son tour, et il continua son chemin en prophétisant, jusqu'à son arrivée à Najoth, en Rama.
\VS{24}Il se dépouilla lui aussi de ses vêtements et prophétisa devant Samuel~; et il se jeta à terre nu, tout ce jour-là et toute la nuit. C'est pourquoi on dit~: Saül est-il aussi parmi les prophètes~?
\Chap{20}
\TextTitle{Renouvellement de l'alliance entre David et Jonathan}
\VerseOne{}David s'enfuit de Najoth, qui est en Rama. Il alla voir Jonathan et lui dit~: Qu'ai-je fait~? Quelle est mon iniquité, et quel est mon péché devant ton père, pour qu'il en veuille à ma vie~?
\VS{2}Jonathan lui dit~: A Dieu ne plaise~! Tu ne mourras pas. Voici, mon père ne fait aucune chose, ni grande ni petite, qu'il ne m'en informe~; pourquoi mon père me cacherait-il cette chose-là~? Il n'en est rien.
\VS{3}Alors David jurant, dit encore~: Ton père sait certainement que j'ai trouvé grâce à tes yeux, et il aura dit~: Que Jonathan ne sache rien de ceci, de peur qu'il n'en soit attristé. Mais Yahweh est vivant, et ton âme vit, il n'y a qu'un pas entre moi et la mort.
\VS{4}Alors Jonathan dit à David~: Que désires-tu que je fasse~? Et je le ferai pour toi.
\VS{5}Et David dit à Jonathan~: Voici, c'est demain la nouvelle lune, et je devrais m'asseoir auprès du roi pour manger, laisse-moi donc aller et je me cacherai aux champs, jusqu'au troisième soir.
\VS{6}Si ton père me cherche, tu lui répondras~: David m'a demandé la permission de courir à Bethléhem, sa ville, parce que toute sa famille fait un sacrifice annuel.
\VS{7}S'il dit ainsi~: C'est bien~! Ton serviteur n'a rien à craindre. Mais s'il se met en colère, sache qu'il a résolu mon malheur.
\VS{8}Use donc de bonté envers ton serviteur, puisque tu as conclu une alliance avec ton serviteur devant Yahweh. S'il y a de l'iniquité en moi, tue-moi toi-même, car pourquoi me mènerais-tu jusqu'à ton père~?
\VS{9}Jonathan lui dit~: A Dieu ne plaise que cela t'arrive~! Si je savais ta perte arrêtée dans la pensée de mon père, ne t'en informerais-je pas~?
\VS{10}David répondit à Jonathan~: Qui m'avertira si la réponse que t'aura faite ton père est sévère~?
\VS{11}Et Jonathan dit à David~: Viens et sortons dans les champs. Ils sortirent donc eux deux dans les champs.
\VS{12}Alors Jonathan dit à David~: Par Yahweh, le Dieu d'Israël~! Je sonderai mon père demain, environ à cette heure, ou après-demain, et s'il est favorable envers David, et que je n'envoie personne vers toi pour t'en informer,
\VS{13}que Yahweh traite Jonathan dans toute sa rigueur~! Si mon père a résolu de te faire du mal, je t'en informerai, et je te laisserai aller, et tu t'en iras en paix, de sorte que Yahweh sera avec toi comme il a été avec mon père.
\VS{14}Si je vis encore, tu useras de la bonté de Yahweh envers moi, en sorte que je ne meure pas.
\VS{15}Ne retire jamais ta bonté de ma maison, pas même quand Yahweh retranchera tous les ennemis de David de dessus la surface de la terre.
\VS{16}Ainsi Jonathan traita alliance avec la maison de David, en disant~: Que Yahweh tire vengeance des ennemis de David~!
\VS{17}Jonathan se lia encore par serment à David pour l'amour qu'il lui portait, car il l'aimait comme son âme.
\VS{18}Puis Jonathan lui dit~: C'est demain la nouvelle lune, et on s'informera sur toi, car ta place sera vide.
\VS{19}Le troisième jour, au soir, tu descendras en hâte, jusqu'au fond du lieu où tu t'étais caché le jour de l'affaire et tu resteras près de la pierre d'Ezel.
\VS{20}Je tirerai trois flèches à côté de cette pierre, comme si je visais un but.
\VS{21}Et voici, j'enverrai un jeune homme, et je lui dirai~: Va, trouve les flèches. Si je dis au jeune homme~: Voici, les flèches sont au deçà de toi, prends-les~! Alors viens, car la paix est avec toi et tu n'as rien à craindre, Yahweh est vivant~!
\VS{22}Mais si je dis ainsi au jeune homme~: Voici, les flèches sont au-delà de toi~! Va-t'en, car Yahweh te renvoie.
\VS{23}Et quant à la parole que nous nous sommes donnée, toi et moi~; voici, Yahweh est entre moi et toi, à jamais.
\TextTitle{Saül en colère contre Jonathan}
\VS{24}David donc se cacha dans le champ. La nouvelle lune étant venue, le roi s'assit pour prendre son repas.
\VS{25}Et le roi s'assit à sa place, comme à l'ordinaire, sur son siège près du mur, Jonathan se leva, et Abner s'assit à côté de Saül~; mais la place de David resta vide.
\VS{26}Saül ne dit rien ce jour-là, car il se disait~: Il lui est arrivé quelque chose, il n'est pas pur, certainement il n'est pas pur.
\VS{27}Mais le lendemain, le second jour de la nouvelle lune, la place de David était encore vide. Et Saül dit à Jonathan, son fils~: Pourquoi le fils d'Isaï n'a-t-il été ni hier ni aujourd'hui au repas~?
\VS{28}Et Jonathan répondit à Saül~: David m'a instamment demandé la permission d'aller à Bethléhem.
\VS{29}Même il m'a dit~: Je te prie, laisse-moi aller, car notre famille fait un sacrifice dans la ville, et mon frère m'a ordonné de m'y trouver~; maintenant donc si j'ai trouvé grâce à tes yeux, je te prie que j'y aille, afin que je voie mes frères. C'est pour cela qu'il n'est pas venu à la table du roi.
\VS{30}Alors la colère de Saül s'enflamma contre Jonathan et il lui dit~: Fils de la perfide et rebelle, ne sais-je pas que tu as choisi le fils d'Isaï à ta honte et à la honte de ta mère~?
\VS{31}Car aussi longtemps que le fils d'Isaï sera vivant sur la terre, tu ne seras pas stable, ni toi, ni ta royauté. C'est pourquoi, maintenant, amène-le-moi, car il est digne de mort.
\VS{32}Et Jonathan répondit à Saül, son père, et lui dit~: Pourquoi le ferait-on mourir~? Qu'a-t-il fait~?
\VS{33}Et Saül lança sa lance contre lui pour le frapper. Alors Jonathan reconnut que son père avait résolu la mort de David.
\VS{34}Jonathan se leva de table dans une ardente colère, et ne mangea pas le pain le deuxième jour de la nouvelle lune~; car il était affligé à cause de David, parce que son père l'avait insulté.
\VS{35}Le matin venu, Jonathan sortit dans les champs, au lieu convenu avec David, et il amena avec lui un petit garçon.
\VS{36}Et il dit à son garçon~: Cours, trouve maintenant les flèches que je m'en vais tirer. Et le garçon courut, et Jonathan tira une flèche qui le dépassa.
\VS{37}Lorsque le garçon arriva au lieu où était la flèche que Jonathan avait tirée, Jonathan cria après lui, et lui dit~: La flèche n'est-elle pas plus loin de toi~?
\VS{38}Jonathan cria encore après le garçon~: Hâte-toi, ne t'arrête pas~! Et le garçon ramassa les flèches, et revint vers son maître.
\VS{39}Le garçon ne savait rien de cette affaire~; seuls David et Jonathan le savaient.
\VS{40}Jonathan remit ses armes au garçon et lui dit~: Va, porte-les à la ville.
\VS{41}Le garçon parti, David se leva du côté du sud, se jeta le visage contre terre et se prosterna à trois reprises. Ils s'embrassèrent et pleurèrent ensemble, David versa d'abondantes larmes.
\VS{42}Jonathan dit à David~: Va en paix, comme nous l'avons juré au Nom de Yahweh, en disant~: Que Yahweh soit entre moi et toi, entre ma postérité et ta postérité.
\VS{43}David donc se leva, s'en alla, et Jonathan rentra dans la ville.
\Chap{21}
\TextTitle{David et Achimélec le prêtre}
\VerseOne{}David se rendit à Nob, vers Achimélec, le prêtre, qui tout effrayé courut au-devant de David, et lui dit~: Pourquoi es-tu seul et n'y a-t-il personne avec toi~?
\VS{2}David répondit à Achimélec, le prêtre~: Le roi m'a donné un ordre et m'a dit~: Que personne ne sache rien de l'affaire pour laquelle je t'envoie, ni de l'ordre que je t'ai donné. J'ai donné rendez-vous à mes hommes en un certain lieu.
\VS{3}Maintenant donc qu'as-tu sous la main~? Donne-moi cinq pains ou ce qui s'y trouvera.
\VS{4}Le prêtre répondit à David, et dit~: Je n'ai pas de pain ordinaire sous la main, mais du pain sacré\FTNT{Mt. 12:4.}~; pourvu que tes gens se soient abstenus de femmes~!
\VS{5}David répondit au prêtre~: Il est vrai que depuis que je suis parti, il y a trois jours, les femmes ont été éloignées de nous, et les vases des serviteurs sont restés purs~; et ce pain est tenu pour commun, vu qu'aujourd'hui on en consacre de nouveau pour le mettre dans les vases.
\VS{6}Alors le prêtre lui donna du pain sacré, car il n'y avait pas là d'autres pains que les pains de proposition qui avaient été ôtés de devant Yahweh, pour le remplacer par du pain chaud le jour où on l'avait pris.
\VS{7}Or il y avait là un homme d'entre les serviteurs de Saül, retenu ce jour-là devant Yahweh~; il s'appelait Doëg, un Edomite, le plus puissant de tous les pasteurs de Saül.
\VS{8}David dit à Achimélec~: Mais n'as-tu pas ici sous la main quelque lance, ou quelque épée~? Car je n'ai pas pris mon épée ni mes armes sur moi, parce que l'ordre du roi était pressant.
\VS{9}Et le prêtre dit~: Voici l'épée de Goliath, le Philistin, que tu as tué dans la vallée du chêne~; elle est enveloppée d'un drap, derrière l'éphod~; si tu veux la prendre pour toi, prends-la, car il n'y en a pas ici d'autres que celle-là. Et David dit~: Il n'y en a pas de pareille~; donne-la-moi.
\TextTitle{David s'enfuit à Gath}
\VS{10}Alors David se leva, et s'enfuit ce jour-là, loin de Saül, et s'en alla vers Akisch, roi de Gath.
\VS{11}Et les serviteurs d'Akisch lui dirent~: N'est-ce pas là David, roi du pays~? N'est-ce pas celui dont on chantait et répondait en dansant~: Saül a tué ses mille, et David ses dix mille~?
\VS{12}David mit ces paroles dans son cœur, et eut une grande peur à cause d'Akisch, roi de Gath.
\VS{13}Il changea son comportement à leurs yeux, il agit devant eux comme un insensé~; et il faisait des marques sur les battants des portes, et laissait couler sa salive sur sa barbe.
\VS{14}Et Akisch dit à ses serviteurs~: Ne voyez-vous pas que c'est un homme insensé~? Pourquoi me l'avez-vous amené~?
\VS{15}Est-ce que je manque d'hommes insensés, pour que vous m'ameniez celui-ci pour faire l'insensé devant moi~? Faudrait-il qu'il entre dans ma maison~?
\Chap{22}
\TextTitle{David se réfugie dans la caverne d'Adullam\FTNTT{1 Ch. 12:16-18.}}
\VerseOne{}Et David partit de là, et se sauva dans la caverne d'Adullam. Ses frères et toute la maison de son père l'ayant appris, ils descendirent vers lui.
\VS{2}Tous ceux qui étaient dans la détresse, qui avaient des créanciers, et qui avaient le cœur rempli d'amertume, se rassemblèrent auprès de lui, et il devint leur chef. Et il y eut avec lui environ quatre cents hommes.
\VS{3}David s'en alla de là à Mitspé dans le pays de Moab. Il dit au roi de Moab~: Permets, je te prie, à mon père et à ma mère de se retirer chez vous jusqu'à ce que je sache ce que Dieu fera de moi.
\VS{4}Il les amena devant le roi de Moab, et ils demeurèrent chez lui, tout le temps que David fut dans cette forteresse.
\VS{5}Or Gad, le prophète, dit à David~: Ne demeure pas dans cette forteresse, mais va-t'en, et entre dans le pays de Juda. David donc s'en alla, et vint dans la forêt de Héreth.
\TextTitle{Saül fait tuer les prêtres}
\VS{6}Saül apprit qu'on avait découvert David et ses gens. Or Saül était assis sous le tamaris, à Guibea, sur la hauteur~; il avait sa lance à la main, et tous ses serviteurs se tenaient devant lui.
\VS{7}Et Saül dit à ses serviteurs qui se tenaient près de lui~: Ecoutez Benjamites~! Le fils d'Isaï vous donnera-t-il à vous tous des champs et des vignes~? Vous établira-t-il tous chefs de mille, et chefs de cent~?
\VS{8}Pourquoi avez-vous tous conspiré contre moi, et n'y a-t-il personne qui m'informe de l'alliance que mon fils a faite avec le fils d'Isaï~? Pourquoi n'y a-t-il personne de vous qui souffre à mon sujet et qui m'avertisse que mon fils a suscité mon serviteur contre moi pour me dresser des embûches, comme il le fait aujourd'hui~?
\VS{9}Alors Doëg, l'Edomite, qui était établi sur les serviteurs de Saül, répondit, et dit~: J'ai vu le fils d'Isaï venir à Nob, auprès d'Achimélec, fils d'Achithub.
\VS{10}Il a consulté Yahweh pour lui, il lui a donné des vivres ainsi que l'épée de Goliath, le Philistin.
\VS{11}Alors le roi envoya appeler Achimélec, le prêtre, fils d'Achithub, la maison de son père, et les prêtres qui étaient à Nob. Et ils vinrent tous vers le roi.
\VS{12}Saül dit~: Ecoute, fils d'Achithub~! Et il répondit~: Me voici, mon seigneur~!
\VS{13}Alors Saül lui dit~: Pourquoi avez-vous conspiré contre moi, toi et le fils d'Isaï~? Pourquoi lui as-tu donné du pain et une épée, et as-tu consulté Dieu pour lui, pour qu'il s'élève contre moi comme il le fait aujourd'hui, pour me dresser des embûches~?
\VS{14}Et Achimélec répondit au roi, et dit~: Entre tous tes serviteurs y en a-t-il un comme David, fidèle, et gendre du roi, qui est parti sur ton commandement, et qui est si honoré dans ta maison~?
\VS{15}Est-ce aujourd'hui que j'ai commencé à consulter Dieu pour lui~? A Dieu ne plaise~! Que le roi n'impute aucun tort à son serviteur, à personne de la maison de mon père, car ton serviteur ne sait rien de tout cela, petite ou grande.
\VS{16}Le roi lui dit~: Tu mourras, tu mourras\FTNT{Le mot est répété deux fois. Voir commentaire en Ge. 2:16.}, Achimélec, toi et toute la maison de ton père.
\VS{17}Alors le roi dit aux coureurs qui se tenaient devant lui~: Approchez-vous, et mettez à mort les prêtres de Yahweh~; car leur main est avec David, parce qu'ils savaient qu'il s'enfuyait, et qu'ils ne m'ont pas averti. Mais les serviteurs du roi ne voulurent pas étendre leurs mains pour frapper les prêtres de Yahweh.
\VS{18}Alors Le roi dit à Doëg~: Approche-toi, et frappe les prêtres. Et Doëg, l'Edomite, se tourna, et frappa les prêtres~; il tua en ce jour-là quatre-vingt-cinq hommes qui portaient l'éphod de lin.
\VS{19}Il frappa encore du tranchant de l'épée Nob, ville des prêtres~; hommes et femmes, enfants et nourrissons, bœufs, ânes, et brebis, tombèrent sous le tranchant de l'épée.
\VS{20}Toutefois un des fils d'Achimélec, fils d'Achithub, qui s'appelait Abiathar, se sauva, et s'enfuit auprès de David.
\VS{21}Abiathar rapporta à David que Saül avait tué les prêtres de Yahweh.
\VS{22}David dit à Abiathar~: Je savais bien ce jour-là que Doëg, l'Edomite, qui était présent, ne manquerait pas d'informer Saül. Je suis la cause de la mort de toutes les personnes de la maison de ton père.
\VS{23}Reste avec moi, ne crains rien, car celui qui cherche ma vie cherche la tienne~; avec moi, tu seras bien gardé.
\Chap{23}
\TextTitle{David libère Keïla}
\VerseOne{}On fit ce rapport à David, en disant~: Voici, les Philistins font la guerre à Keïla, et pillent les aires.
\VS{2}Et David consulta Yahweh\FTNT{La clé du succès de David était Yahweh. Il consultait régulièrement Dieu avant de s'engager dans une guerre (Ps. 60:14.).} en disant~: Irai-je, et frapperai-je ces Philistins~? Et Yahweh répondit à David~: Va, et tu frapperas les Philistins, et tu délivreras Keïla.
\VS{3}Les gens de David lui dirent~: Voici, nous avons peur ici en Juda~; que sera-ce donc quand nous irons à Keïla contre les troupes des Philistins~?
\VS{4}C'est pourquoi David consulta encore Yahweh, et Yahweh lui répondit, et dit~: Lève-toi, descends à Keïla, car je vais livrer les Philistins entre tes mains.
\VS{5}Alors David s'en alla avec ses gens à Keïla, et combattit contre les Philistins~; et emmena leur bétail, et fit un grand carnage. Ainsi David délivra les habitants de Keïla.
\VS{6}Or il était arrivé que quand Abiathar, fils d'Achimélec, s'était enfui vers David à Keïla, il avait en main l'éphod.
\VS{7}On rapporta à Saül que David était venu à Keïla~; Saül dit~: Dieu l'a livré entre mes mains car il s'est enfermé en entrant dans une ville qui a des portes et des barres.
\VS{8}Saül convoqua tout le peuple pour aller à la guerre, afin de descendre à Keïla, et d'assiéger David et ses gens.
\VS{9}David ayant eu connaissance des mauvais desseins de Saül à son égard, dit au prêtre Abiathar~: Apporte l'éphod~!
\VS{10}Puis David dit~: Ô Yahweh, Dieu d'Israël~! Ton serviteur apprend que Saül cherche à venir à Keïla, pour détruire la ville à cause de moi.
\VS{11}Les chefs de Keïla me livreront-ils entre ses mains~? Saül descendra-t-il comme ton serviteur l'a entendu dire~? Ô Yahweh, Dieu d'Israël, je te prie, révèle-le à ton serviteur. Et Yahweh répondit~: Il descendra.
\VS{12}David dit encore~: Les chefs de Keïla me livreront-ils, moi et mes gens, entre les mains de Saül~? Et Yahweh répondit~: Ils te livreront.
\TextTitle{David échappe encore à Saül}
\VS{13}Alors David se leva avec ses gens au nombre d'environ six cents hommes~; et ils sortirent de Keïla, et s'en allèrent où ils purent. On rapporta à Saül que David s'était sauvé de Keïla, c'est pourquoi il cessa sa marche.
\VS{14}David resta au désert, dans des lieux forts, et il se tint sur la montagne au désert de Ziph. Et Saül le cherchait tous les jours, mais Dieu ne le livra pas entre ses mains.
\VS{15}David, sachant que Saül était sorti pour chercher à sa vie, se tint au désert de Ziph, dans la forêt.
\VS{16}Alors Jonathan, fils de Saül, se leva, et s'en alla dans la forêt vers David, et fortifia ses mains en Dieu~;
\VS{17}et lui dit~: Ne crains pas, car Saül, mon père, ne t'atteindra pas. Mais tu régneras sur Israël, et moi je serai le second après toi~; et même Saül, mon père, le sait bien.
\VS{18}Ils firent tous les deux alliance devant Yahweh~; et David resta dans la forêt, mais Jonathan retourna dans sa maison.
\VS{19}Or les Ziphiens montèrent auprès de Saül à Guibea, et lui dirent~: David ne se tient-il pas caché parmi nous dans des lieux forts, dans la forêt, sur la colline de Hakila, qui est au sud du désert~?
\VS{20}Maintenant donc, ô roi, puisque tout le désir de ton âme est de descendre, descends, et ce sera à nous de le livrer entre les mains du roi.
\VS{21}Et Saül dit~: Que Yahweh vous bénisse de ce que vous avez eu pitié de moi~!
\VS{22}Allez donc, je vous prie, assurez-vous encore davantage pour savoir et trouver le lieu où il a dirigé ses pas et qui l'a vu, car, m'a-t-on dit, il est fort rusé.
\VS{23}Examinez donc et reconnaissez tous les lieux où il se tient caché, puis retournez vers moi quand vous en serez assurés, et j'irai avec vous. S'il est dans le pays, je le chercherai soigneusement parmi tous les milliers de Juda.
\VS{24}Ils se levèrent donc et s'en allèrent à Ziph avant Saül. David et ses gens étaient dans le désert de Maon, dans la plaine, au sud du désert.
\VS{25}Saül et ses gens partirent à la recherche de David. Et l'on en informa David, qui descendit le rocher, et resta dans le désert de Maon. Saül, l'ayant appris, poursuivit David au désert de Maon.
\VS{26}Saül marchait d'un côté de la montagne, et David et ses gens de l'autre côté de la montagne. David fuyait précipitamment pour échapper à Saül. Mais Saül et ses gens entouraient David et ses gens pour s'emparer d'eux,
\VS{27}lorsqu'un messager vint à Saül, en disant~: Hâte-toi de venir, car les Philistins envahissent le pays.
\VS{28}Alors Saül cessa de poursuivre David, et s'en retourna au-devant des Philistins. C'est pourquoi on appela ce lieu Séla-Hammachlekoth.
\Chap{24}
\TextTitle{David épargne la vie de Saül à En-Guédi}
\VerseOne{}Puis David monta de là et demeura dans les lieux forts d'En-Guédi.
\VS{2}Lorsque Saül fut revenu de la poursuite des Philistins, on lui fit ce rapport, disant~: Voilà David au désert d'En-Guédi.
\VS{3}Alors Saül prit trois mille hommes d'élite de tout Israël, et il s'en alla chercher David et ses gens jusque sur le rocher des boucs sauvages.
\VS{4}Saül arriva à des parcs de brebis qui étaient près du chemin, où il y avait une caverne dans laquelle il entra pour se couvrir les pieds. David et ses gens se tenaient au fond de la caverne.
\VS{5}Et les gens de David lui dirent~: Voici le jour où Yahweh te dit~: Je te livre ton ennemi entre tes mains, afin que tu lui fasses selon ce qu'il te semblera bon. David se leva et coupa tout doucement le pan du manteau de Saül.
\VS{6}Après cela, le cœur de David battit, parce qu'il avait coupé le pan du manteau de Saül.
\VS{7}Et il dit à ses gens~: Que Yahweh me garde de commettre une telle action contre mon seigneur, l'oint de Yahweh, en mettant ma main sur lui~! Car il est l'oint de Yahweh\FTNT{David épargne Saül parce qu'il fait confiance à Yahweh. Il laisse Dieu agir plutôt que d'agir par lui-même.}.
\VS{8}Ainsi, David détourna ses gens par ses paroles, et il ne leur permit pas de s'élever contre Saül. Puis Saül se leva de la caverne et poursuivit son chemin.
\VS{9}Après cela, David se leva, sortit de la caverne, et cria après Saül, en disant~: Mon seigneur le roi~! Saül regarda derrière lui, et David s'inclina le visage contre terre et se prosterna.
\VS{10}David dit à Saül~: Pourquoi écouterais-tu les paroles des gens qui te disent~: Voici, David cherche ton malheur~?
\VS{11}Aujourd'hui, tes yeux ont vu que Yahweh t'avait livré entre mes mains dans la caverne, et on m'a dit de te tuer~; mais je t'ai épargné, et j'ai dit~: Je ne porterai pas la main sur mon seigneur, car il est l'oint de Yahweh.
\VS{12}Regarde donc, mon père, regarde, dis-je, le pan de ton manteau dans ma main. Car j'ai coupé le pan de ton manteau et je ne t'ai pas tué. Sache et reconnais qu'il n'y a ni mal ni injustice dans ma conduite, et que je n'ai pas péché contre toi. Cependant tu me dresses des embûches pour m'ôter la vie~!
\VS{13}Yahweh sera juge entre moi et toi, et Yahweh me vengera de toi, mais ma main ne sera pas sur toi.
\VS{14}C'est des méchants que vient la méchanceté, comme dit le proverbe des anciens~; c'est pourquoi ma main ne sera point sur toi.
\VS{15}Contre qui est sorti le roi d'Israël~? Qui poursuis-tu~? Un chien mort, une puce~!
\VS{16}Yahweh sera donc juge, et jugera entre moi et toi~; il regardera et plaidera ma cause, il me rendra justice en me délivrant de ta main.
\VS{17}Or il arriva qu'aussitôt que David eut achevé d'adresser ces paroles à Saül, Saül dit~: N'est-ce pas là ta voix, mon fils David~? Et Saül éleva la voix, et pleura.
\VS{18}Et il dit à David~: Tu es plus juste que moi~; car tu m'as rendu le bien pour le mal que je t'ai fait,
\VS{19}et tu m'as fait connaître aujourd'hui comment tu as usé de bonté envers moi, car Yahweh m'avait livré entre tes mains, et cependant tu ne m'as pas tué.
\VS{20}Si quelqu'un rencontre son ennemi, le laisse-t-il poursuivre tranquillement son chemin~? Que Yahweh donc te récompense pour la grâce que tu m'as faite aujourd'hui~!
\VS{21}Et maintenant voici, je sais que tu régneras certainement et que le royaume d'Israël sera ferme entre tes mains.
\VS{22}C'est pourquoi maintenant, jure-moi par Yahweh, que tu ne détruiras pas ma race après moi, et que tu n'extermineras pas mon nom de la maison de mon père.
\VS{23}Et David le jura à Saül. Puis Saül s'en alla dans sa maison, et David et ses gens montèrent au lieu fort.
\Chap{25}
\TextTitle{Israël pleure la mort de Samuel}
\VerseOne{}Samuel mourut. Et tout Israël s'assembla, et le pleura, et on l'enterra dans sa maison à Rama. David se leva, et descendit au désert de Paran.
\TextTitle{Méchanceté de Nabal~; bon sens d'Abigaïl}
\VS{2}Or il y avait à Maon un homme qui avait ses biens à Carmel, et cet homme-là était très puissant car il avait trois mille brebis, et mille chèvres, et il se trouvait à Carmel quand on tondait ses brebis.
\VS{3}Le nom de l'homme était Nabal, et le nom de sa femme, Abigaïl~; elle était une femme de bon sens et belle de visage, mais l'homme était cruel et méchant dans toutes ses actions. Il était de la race de Caleb.
\VS{4}Or David apprit au désert que Nabal tondait ses brebis.
\VS{5}Et il envoya dix jeunes gens, et leur dit~: Montez à Carmel, et rendez-vous auprès de Nabal. Vous le saluerez en mon nom,
\VS{6}et vous lui direz~: Puisses-tu faire autant l'année prochaine à la même saison, et que la paix soit avec ta maison et tout ce qui est à toi.
\VS{7}Et maintenant, j'ai appris que tu as les tondeurs. Or tes bergers ont été avec nous, et nous ne leur avons fait aucune injure, et ils n'ont subi aucune perte pendant tout le temps qu'ils ont été à Carmel.
\VS{8}Demande-le à tes serviteurs, et ils te le diront. Que ces jeunes gens trouvent donc grâce à tes yeux, puisque nous venons dans un jour favorable. Nous te prions de donner à tes serviteurs, et à David, ton fils, ce que tu trouveras sous ta main.
\VS{9}Les gens de David arrivèrent, et dirent à Nabal, au nom de David, toutes ces paroles. Puis ils se turent.
\VS{10}Nabal répondit aux serviteurs de David, et dit~: Qui est David, et qui est le fils d'Isaï~? Aujourd'hui le nombre des serviteurs qui s'échappent de leurs maîtres se multiplie.
\VS{11}Et prendrais-je mon pain et mon eau, et la viande que j'ai apprêtée pour mes tondeurs, afin de les donner à des gens qui viennent je ne sais d'où~?
\VS{12}Ainsi les gens de David rebroussèrent chemin. Ils s'en retournèrent, et lui firent leur rapport selon toutes ces paroles.
\VS{13}Et David dit à ses gens~: Que chacun de vous ceigne son épée. Et ils ceignirent chacun leur épée. David aussi ceignit son épée et environ quatre cents hommes montèrent avec David. Il en resta deux cents près des bagages.
\VS{14}Or un des serviteurs de Nabal fit ce rapport à Abigaïl, femme de Nabal, et lui dit~: Voici, David a envoyé du désert des messagers pour saluer notre maître, qui les a traités rudement.
\VS{15}Cependant ces hommes ont été très bons envers nous, et ne nous ont fait aucune injure, et rien ne nous a été enlevé, tout le temps que nous avons été avec eux lorsque nous étions dans les champs.
\VS{16}Ils nous ont servi de muraille nuit et jour, tout le temps que nous avons été avec eux, faisant paître les troupeaux.
\VS{17}Sache maintenant et vois ce que tu as à faire, car le mal est résolu contre notre maître et contre toute sa maison, et il est si méchant\FTNT{Littéralement «~Fils de Bélial~». Voir commentaire en 1 S. 2:12.} qu'on n'ose lui parler.
\VS{18}Abigaïl se hâta donc, et prit deux cents pains, deux outres de vin, cinq pièces de menu bétail, cinq mesures de grain rôti, cent paquets de raisins secs, deux cents de figues sèches, et les mit sur des ânes.
\VS{19}Puis elle dit à ses gens~: Passez devant moi, je vais vous suivre. Elle n'en dit rien à Nabal, son mari.
\VS{20}Et étant montée sur un âne, elle descendait de la montagne par un chemin couvert~; voici, David et ses gens descendaient en face d'elle, et elle les rencontra.
\VS{21}David avait dit~: Certainement c'est en vain que j'ai gardé tout ce que cet homme a dans le désert, en sorte qu'il ne s'est rien perdu de tout ce qu'il possède~; il m'a rendu le mal pour le bien.
\VS{22}Que Dieu traite son serviteur David dans toute sa rigueur, si d'ici au matin je laisse subsister de tout ce qui appartient à Nabal.
\VS{23}Lorsque Abigaïl aperçut David, elle se hâta de descendre de son âne, et tomba sur sa face devant David, et se prosterna contre terre.
\VS{24}Elle se jeta donc à ses pieds et lui dit~: A moi la faute, mon seigneur~! Permets à ta servante de parler devant toi, et écoute les paroles de ta servante.
\VS{25}Je t'en supplie, que mon seigneur ne prenne pas garde à ce méchant homme, à Nabal, car il est comme son nom~; Nabal est son nom, et il y a de la folie en lui. Et moi, ta servante, je n'ai pas vu les gens que mon seigneur a envoyés.
\VS{26}Maintenant donc, mon seigneur, aussi vrai que Yahweh est vivant, et que ton âme vit, Yahweh t'a empêché d'en venir au sang, et il a retenu ta main. Or que tes ennemis, et ceux qui cherchent à nuire à mon seigneur soient comme Nabal.
\VS{27}Mais maintenant voici un présent que ta servante a apporté à mon seigneur, afin qu'on le donne aux gens qui sont à la suite de mon seigneur.
\VS{28}Pardonne, je te prie, le crime de ta servante, vu que Yahweh ne manquera pas d'établir une maison ferme à mon seigneur~; car mon seigneur conduit les batailles de Yahweh, et il ne s'est trouvé en toi aucun mal pendant toute ta vie.
\VS{29}Si les hommes se lèvent pour te persécuter, et pour chercher ton âme, l'âme de mon seigneur sera liée au faisceau des vivants auprès de Yahweh, ton Dieu~; mais il lancera au loin, avec la fronde, l'âme de tes ennemis.
\VS{30}Il arrivera que Yahweh fera à mon seigneur selon tout le bien qu'il t'a prédit, et qu'il t'établira conducteur d'Israël,
\VS{31}ceci ne sera pas un obstacle, ni un sujet de regret dans l'âme de mon seigneur, pour avoir répandu le sang inutilement, et pour s'être vengé lui-même. Aussi, lorsque Yahweh aura fait du bien à mon seigneur, tu te souviendras de ta servante.
\VS{32}Alors David dit à Abigaïl~: Béni soit Yahweh, le Dieu d'Israël, qui t'a aujourd'hui envoyée à ma rencontre~!
\VS{33}Et béni soit ton bon sens, et bénie sois-tu, toi qui m'as aujourd'hui empêché d'en venir au sang, et qui as retenu ma main~!
\VS{34}Car certainement Yahweh, le Dieu d'Israël, qui m'a empêché de te faire du mal, est vivant~! Si tu ne t'étais hâtée de venir à ma rencontre, il ne serait resté qui que ce soit à Nabal d'ici à la lumière du matin.
\VS{35}David prit donc de sa main ce qu'elle lui avait apporté, et lui dit~: Remonte en paix dans ta maison~; regarde, j'ai écouté ta voix, et j'ai répondu favorablement à ta demande.
\TextTitle{Mort de Nabal~; Abigaïl devient la femme de David}
\VS{36}Puis Abigaïl revint auprès de Nabal. Et voici, il faisait un festin dans sa maison, comme un festin de roi~; et Nabal avait le cœur joyeux, et il était complètement ivre. C'est pourquoi elle ne lui dit aucune chose petite ou grande, jusqu'au matin.
\VS{37}Il arriva donc au matin, après que Nabal fut désenivré, que sa femme lui déclara toutes ces choses. Et son cœur s'amortit en lui, de sorte qu'il devint comme une pierre.
\VS{38}Or, il arriva qu'environ dix jours après, Yahweh frappa Nabal, et il mourut.
\VS{39}Et quand David eut appris que Nabal était mort, il dit~: Béni soit Yahweh, qui m'a vengé de l'outrage que j'avais reçu de la main de Nabal, et qui a préservé son serviteur de faire du mal, et a fait retomber le mal de Nabal sur sa tête~! Puis David envoya des gens pour parler à Abigaïl, afin de la prendre pour sa femme.
\VS{40}Les serviteurs de David vinrent auprès d'Abigaïl à Carmel, et lui parlèrent, en disant~: David nous a envoyés vers toi, afin de te prendre pour femme.
\VS{41}Alors elle se leva, et se prosterna le visage contre terre, et dit~: Voici, ta servante sera à ton service, afin de laver les pieds des serviteurs de mon seigneur.
\VS{42}Puis Abigaïl se leva promptement et monta sur un âne, et cinq de ses servantes la suivirent~; elle suivit les messagers de David, et fut sa femme.
\VS{43}Or David avait pris aussi Achinoam de Jizreel, et toutes les deux furent ses femmes.
\VS{44}Car Saül avait donné Mical, sa fille, femme de David, à Palthi, fils de Laïsch, qui était de Gallim.
\Chap{26}
\TextTitle{David épargne encore la vie de Saül}
\VerseOne{}Les Ziphiens allèrent encore auprès de Saül à Guibea, en disant~: David ne se tient-il pas caché sur la colline de Hakila, en face du désert~?
\VS{2}Saül se leva, et descendit au désert de Ziph, avec trois mille hommes de l'élite d'Israël, pour chercher David dans le désert de Ziph.
\VS{3}Saül campa sur la colline de Hakila, en face du désert, près du chemin. Or David se tenait dans le désert, et il aperçut que Saül marchait à sa poursuite au désert,
\VS{4}alors il envoya des espions, et apprit avec certitude que Saül était arrivé.
\VS{5}Alors David se leva, et alla au lieu où Saül campait, et David vit la place où couchait Saül, avec Abner, fils de Ner, chef de son armée. Saül couchait au milieu du camp, et le peuple campait autour de lui.
\VS{6}David prit la parole, et dit à Achimélec, le Héthien, et à Abischaï, fils de Tseruja et frère de Joab, il dit~: Qui veut descendre avec moi dans le camp vers Saül~? Et Abischaï répondit~: J'y descendrai avec toi.
\VS{7}David et Abischaï allèrent de nuit vers le peuple. Et voici, Saül dormait étant couché au milieu du camp, et sa lance était plantée en terre à son chevet. Et Abner, et le peuple étaient couchés autour de lui.
\VS{8}Alors Abischaï dit à David~: Aujourd'hui, Dieu a livré ton ennemi entre tes mains~; laisse-moi donc le frapper avec la lance, jusqu'en terre d'un seul coup, et je n'y retournerai pas une seconde fois.
\VS{9}Et David dit à Abischaï~: Ne le tue pas~! Car qui porterait impunément sa main sur l'oint de Yahweh~?
\VS{10}David dit encore~: Yahweh est vivant~! C'est Yahweh seul qui le frappera, soit que son jour vienne, soit qu'il descende au combat et qu'il y périsse.
\VS{11}Que Yahweh me garde de mettre ma main sur l'oint de Yahweh~! Mais prends maintenant la lance qui est à son chevet et la cruche d'eau, et allons-nous-en.
\VS{12}David donc prit la lance et la cruche d'eau qui étaient au chevet de Saül, puis ils s'en allèrent. Personne ne les vit, ni ne s'aperçut de rien, ni ne se réveilla, car ils dormaient tous d'un profond sommeil dans lequel Yahweh les avait plongés.
\VS{13}David passa de l'autre côté, et s'arrêta au loin sur le sommet de la montagne, et il y avait une grande distance entre eux.
\VS{14}Et il cria au peuple, et à Abner, fils de Ner, en disant~: Ne répondras-tu pas, Abner~? Abner répondit, et dit~: Qui es-tu, toi, qui cries vers le roi~?
\VS{15}Alors David dit à Abner~: N'es-tu pas un vaillant homme~? Qui est semblable à toi en Israël~? Pourquoi donc n'as-tu pas gardé le roi, ton seigneur~? Car quelqu'un du peuple est venu pour tuer le roi, ton seigneur.
\VS{16}Ce que tu as fait n'est pas bien. Yahweh est vivant~! Vous méritez la mort, pour avoir si mal gardé votre seigneur, l'oint de Yahweh. Et maintenant, regarde où sont la lance du roi et la cruche d'eau qui étaient à son chevet.
\VS{17}Alors Saül reconnut la voix de David, et dit~: N'est-ce pas là ta voix, mon fils David~? Et David dit~: C'est ma voix, ô roi, mon seigneur.
\VS{18}Il dit encore~: Pourquoi mon seigneur poursuit-il son serviteur~? Car qu'ai-je fait, et quel mal y a t-il dans ma main~?
\VS{19}Maintenant donc je te prie que le roi, mon seigneur, écoute les paroles de son serviteur~: Si c'est Yahweh qui te pousse contre moi, que ton offrande lui soit agréable~; mais si ce sont les hommes, qu'ils soient maudits devant Yahweh, car aujourd'hui ils m'ont chassé, afin que je ne puisse me joindre à l'héritage de Yahweh, et ils m'ont dit~: Va, sers les dieux étrangers~!
\VS{20}Que mon sang ne tombe pas en terre loin de la face de Yahweh~! Car le roi d'Israël est sorti pour chercher une puce, comme on poursuivrait une perdrix dans les montagnes.
\TextTitle{Saül se repent devant David}
\VS{21}Alors Saül dit~: J'ai péché~; reviens, mon fils David, car je ne te ferai plus de mal, parce qu'aujourd'hui ma vie t'a été précieuse. Voici, j'ai agi en insensé, et j'ai commis une très grande faute.
\VS{22}David répondit, et dit~: Voici la lance du roi~; que l'un de tes gens vienne la prendre.
\VS{23}Que Yahweh rende à chacun selon sa justice et selon sa fidélité~; car il t'avait livré aujourd'hui entre mes mains, mais je n'ai pas voulu mettre ma main sur l'oint de Yahweh.
\VS{24}Voici, comme ta vie a été aujourd'hui de grand prix à mes yeux, ainsi ma vie sera de grand prix aux yeux de Yahweh, et il me délivrera de toutes les angoisses.
\VS{25}Saül dit à David~: Béni sois-tu, mon fils David~! Tu auras du succès dans tes entreprises. Alors David continua son chemin, et Saül s'en retourna chez lui.
\Chap{27}
\TextTitle{David se réfugie dans le pays des Philistins}
\VerseOne{}Mais David dit en son cœur~: Certes je périrai un jour par les mains de Saül~; ne vaut-il pas mieux que je me sauve en hâte au pays des Philistins, afin que Saül renonce à me chercher encore dans tout le territoire d'Israël~? Ainsi j'échapperai à sa main.
\VS{2}David se leva, lui et les six cents hommes qui étaient avec lui, et il passa chez Akisch, fils de Maoc, roi de Gath.
\VS{3}David et ses gens restèrent à Gath auprès d'Akisch~; ils avaient chacun leur famille, David et ses deux femmes, Achinoam de Jizreel, et Abigaïl, qui avait été femme de Nabal, lequel était de Carmel.
\VS{4}Alors on informa Saül que David s'était enfui à Gath~; et il cessa de le chercher.
\VS{5}David dit à Akisch~: Si j'ai trouvé grâce à tes yeux, qu'on me donne dans l'une des villes du pays, un lieu où je puisse habiter~; car pourquoi ton serviteur habiterait-il dans la ville royale avec toi~?
\VS{6}Akisch lui donna ce même jour, Tsiklag. C'est pourquoi Tsiklag appartient aux rois de Juda jusqu'à ce jour.
\VS{7}Le temps que David demeura dans le pays des Philistins fut d'un an et quatre mois.
\VS{8}David montait avec ses gens faire des incursions chez les Gueschuriens, les Guirziens, et les Amalécites~; car ces nations-là habitaient dans le territoire dès les temps anciens, depuis Schur jusqu'au pays d'Egypte.
\VS{9}David ravageait ce territoire, il ne laissait en vie ni homme ni femme, et il prenait les brebis, les bœufs, les ânes, les chameaux, et les vêtements, puis il s'en retournait, et allait chez Akisch.
\VS{10}Akisch disait~: Où avez-vous fait vos incursions aujourd'hui~? Et David répondait~: Vers le sud de Juda, vers le sud des Jerachmeélites et vers le sud des Kéniens.
\VS{11}Mais David ne laissait en vie ni homme ni femme pour les amener à Gath, de peur, disait-il, qu'ils ne rapportent quelque chose contre nous, disant~: Ainsi a fait David. Et il agit ainsi tout le temps qu'il demeura dans le pays des Philistins.
\VS{12}Akisch croyait David, et il disait~: Il se rend odieux à Israël, son peuple, c'est pourquoi il sera mon serviteur à jamais.
\Chap{28}
\TextTitle{Les Philistins vont en guerre contre Saül}
\VerseOne{}Or il arriva qu'en ces jours-là, les Philistins rassemblèrent leurs armées pour faire la guerre, pour combattre Israël. Akisch dit à David~: Sache certainement que vous viendrez avec moi au camp, toi et tes gens.
\VS{2}David répondit à Akisch~: Certainement, tu verras ce que ton serviteur fera. Et Akisch dit à David~: C'est pour cela que je te confierai toujours la garde de ma personne.
\VS{3}Or Samuel était mort, et tout Israël avait fait le deuil, et on l'avait enseveli à Rama qui était sa ville. Saül avait ôté du pays ceux qui évoquaient les morts, et les devins.
\VS{4}Les Philistins se rassemblèrent et vinrent camper à Sunem~; Saül aussi rassembla tout Israël, et ils campèrent à Guilboa.
\VS{5}A la vue du camp des Philistins, Saül eut peur, et son cœur fut saisi de crainte.
\VS{6}Saül consulta Yahweh~; mais Yahweh ne lui répondit rien, ni par des songes, ni par l'urim, ni par les prophètes.
\TextTitle{Saül chez la femme qui évoque les morts}
\VS{7}Saül dit à ses serviteurs~: Cherchez-moi une femme qui évoque les morts, et j'irai vers elle, et je la consulterai. Ses serviteurs lui dirent~: Voici, il y a une femme à En-Dor qui évoque les morts.
\VS{8}Alors Saül se déguisa, prit d'autres vêtements, et il partit avec deux hommes. Ils arrivèrent de nuit chez cette femme et Saül lui dit~: Je te prie évoque les morts et fais monter vers moi celui que je te dirai.
\VS{9}Mais la femme lui répondit~: Voici, tu sais ce que Saül a fait, et comment il a exterminé du pays ceux qui évoquent les morts et les devins~; pourquoi donc dresses-tu un piège à mon âme pour me faire mourir~?
\VS{10}Saül lui jura par Yahweh, et lui dit~: Yahweh est vivant~! Il ne t'arrivera pas de mal pour cela.
\VS{11}Alors la femme dit~: Qui veux-tu que je te fasse monter~? Et il répondit~: Fais-moi monter Samuel.
\VS{12}Et la femme voyant Samuel s'écria à haute voix, en disant à Saül~: Pourquoi m'as-tu trompée~? Car tu es Saül~!
\VS{13}Et le roi lui répondit~: Ne crains pas~; mais que vois-tu~? La femme dit à Saül~: Je vois un dieu qui monte de la terre.
\VS{14}Il lui dit encore~: Comment est-il fait~? Elle répondit~: C'est un vieillard qui monte, et il est couvert d'un manteau. Et Saül comprit que c'était Samuel, il s'inclina le visage contre terre et se prosterna.
\VS{15}Samuel dit à Saül~: Pourquoi m'as-tu troublé en me faisant monter~? Et Saül répondit~: Je suis dans une grande angoisse~; car les Philistins me font la guerre, et Dieu s'est retiré de moi, et ne m'a plus répondu ni par les prophètes ni par des songes~; c'est pourquoi je t'ai invoqué\FTNT{Dieu interdit formellement ces pratiques (De. 18:9-14~; Es. 8:19.).}, afin que tu me fasses entendre ce que j'aurai à faire.
\VS{16}Samuel dit~: Pourquoi donc me consultes-tu, puisque Yahweh s'est retiré de toi, et qu'il est devenu ton ennemi~?
\VS{17}Yahweh te traite comme je te l'avais annoncé de sa part~; car Yahweh a déchiré le royaume d'entre tes mains, et l'a donné à un autre, à David.
\VS{18}Parce que tu n'as pas obéi à la voix de Yahweh, et que tu n'as pas exécuté l'ardeur de sa colère contre Amalek, à cause de cela, Yahweh te traite de cette manière aujourd'hui.
\VS{19}Et même Yahweh livrera Israël avec toi entre les mains des Philistins, et vous serez demain avec moi, toi et tes fils~; Yahweh livrera aussi le camp d'Israël entre les mains des Philistins.
\VS{20}Aussitôt Saül tomba à terre tout étendu, car il fut très effrayé des paroles de Samuel, et même les forces lui manquèrent parce qu'il n'avait rien mangé ce jour, ni toute cette nuit.
\VS{21}Alors la femme vint auprès de Saül, et voyant qu'il avait été très effrayé, elle lui dit~: Voici, ta servante a obéi à ta voix, j'ai exposé ma vie, et j'ai obéi aux paroles que tu m'as dites.
\VS{22}Maintenant, je te prie, écoute, toi aussi, ce que ta servante te dira~: Laisse-moi te servir avant un morceau de pain, afin que tu manges pour avoir la force de te remettre en route.
\VS{23}Et il le refusa et dit~: Je ne mangerai pas. Mais ses serviteurs et la femme aussi le pressèrent tellement qu'il écouta leur voix. Il se leva de terre, et s'assit sur un lit.
\VS{24}Or cette femme avait dans sa maison un veau qu'elle engraissait, et elle se hâta de le tuer~; puis elle prit de la farine, et la pétrit, et en cuisit des pains sans levain.
\VS{25}Elle les mit devant Saül et devant ses serviteurs. Et ils mangèrent. Puis s'étant levés, ils s'en allèrent cette nuit-là.
\Chap{29}
\TextTitle{Les Philistins refusent que David combatte Israël}
\VerseOne{}Or les Philistins rassemblèrent toutes leurs armées à Aphek, et Israël campa près de la fontaine de Jizreel.
\VS{2}Les princes des Philistins s'avancèrent avec leurs centaines et leurs milliers, et David et ses gens marchèrent à l'arrière-garde avec Akisch.
\VS{3}Les princes des Philistins dirent~: Que font ici ces Hébreux~? Et Akisch répondit aux princes des Philistins~: N'est-ce pas ce David, serviteur de Saül, roi d'Israël~? Il y a longtemps qu'il est avec moi, même quelques années, et je n'ai pas trouvé quelque chose à lui reprocher depuis son arrivée, jusqu'à ce jour.
\VS{4}Mais les princes des Philistins se mirent en colère contre lui, et lui dirent~: Renvoie cet homme, et qu'il retourne dans le lieu où tu l'as établi, et qu'il ne descende pas avec nous dans la bataille, de peur qu'il ne se tourne contre nous dans la bataille. Car comment pourrait-il se remettre en grâce auprès de son maître~? Ne serait-ce pas par le moyen des têtes de nos hommes~?
\VS{5}N'est-ce pas ce David pour qui l'on chantait et répondait en dansant~: Saül a frappé ses mille, et David ses dix mille~?
\VS{6}Akisch appela David, et lui dit~: Yahweh est vivant~! Tu es certainement un homme droit, et ta conduite dans le camp m'a paru bonne, car je n'ai pas trouvé de mal en toi depuis le jour où tu es arrivé auprès de moi jusqu'à ce jour~; mais tu ne plais pas aux princes.
\VS{7}Maintenant donc, retourne et va-t'en en paix, afin que tu ne fasses aucune chose qui déplaise aux princes des Philistins.
\VS{8}Et David dit à Akisch~: Mais qu'ai-je fait~? Et qu'as-tu trouvé en ton serviteur depuis que je suis avec toi jusqu'à ce jour, pour que je n'aille pas combattre contre les ennemis du roi, mon seigneur~?
\VS{9}Akisch répondit, et dit à David~: Je le sais, car tu es agréable à mes yeux, comme un ange de Dieu~; mais c'est seulement les chefs des Philistins qui disent~: Il ne montera pas avec nous dans la bataille.
\VS{10}C'est pourquoi lève-toi de bon matin, avec les serviteurs de ton maître qui sont venus avec toi~; levez-vous de bon matin, et partez dès que vous verrez le jour, allez-vous-en.
\VS{11}Ainsi David se leva de bonne heure, lui et ses gens, pour partir dès le matin, et retourner dans le pays des Philistins. Et les Philistins montèrent à Jizreel.
\Chap{30}
\TextTitle{David libère Tsiklag}
\VerseOne{}Or trois jours après, David et ses gens étant revenus à Tsiklag, trouvèrent que les Amalécites avaient fait une incursion dans le midi et à Tsiklag. Et ils avaient frappé et brûlé au feu Tsiklag,
\VS{2}après avoir fait prisonniers les femmes et tous ceux qui étaient là, petits et grands. Ils n'avaient tué personne, mais ils les avaient emmenés, et s'étaient remis en chemin.
\VS{3}David et ses gens revinrent dans la ville et voici, elle était brûlée, et leurs femmes, leurs fils, et leurs filles avaient été faits prisonniers.
\VS{4}C'est pourquoi David et le peuple qui était avec lui élevèrent leur voix, et pleurèrent tellement qu'il n'y avait plus en eux de force pour pleurer.
\VS{5}Les deux femmes de David avaient été emmenées, Achinoam de Jizreel, et Abigaïl de Carmel, qui avait été femme de Nabal.
\VS{6}David fut dans une grande angoisse, parce que le peuple parlait de le lapider, car tout le peuple avait de l'amertume dans l'âme à cause de leurs fils et de leurs filles. Toutefois, David se fortifia en Yahweh, son Dieu.
\VS{7}Et il dit à Abiathar, le prêtre, fils d'Achimélec~: Apporte-moi, je te prie, l'éphod~! Abiathar apporta l'éphod à David.
\VS{8}Et David consulta Yahweh, en disant~: Poursuivrai-je cette troupe~? L'atteindrai-je~? Et il lui répondit~: Poursuis, car tu l'atteindras, et tu délivreras.
\VS{9}David s'en alla avec les six cents hommes qui étaient avec lui, et ils arrivèrent au torrent de Besor, où s'arrêtèrent ceux qui restaient en arrière.
\VS{10}Ainsi David et quatre cents hommes continuèrent la poursuite, mais deux cents hommes s'arrêtèrent, trop fatigués pour pouvoir passer le torrent de Besor.
\VS{11}Ayant trouvé un homme égyptien dans les champs, ils l'amenèrent à David, et lui donnèrent du pain, il mangea, puis ils lui donnèrent de l'eau à boire.
\VS{12}Ils lui donnèrent aussi quelques figues sèches, et deux grappes de raisins secs. Et il mangea, et le cœur lui revint, car cela faisait trois jours et trois nuits qu'il n'avait pas mangé de pain ni bu d'eau.
\VS{13}Et David lui dit~: A qui es-tu~? Et d'où es-tu~? Et il répondit~: Je suis un garçon égyptien, serviteur d'un homme Amalécite, et mon maître m'a abandonné, parce que j'étais malade il y a trois jours.
\VS{14}Nous avons envahi le sud des Kéréthiens, et sur ce qui est à Juda, et sur le sud de Caleb, et nous avons brûlé Tsiklag par le feu.
\VS{15}David lui dit~: Me conduiras-tu vers cette troupe~? Et il répondit~: Jure-moi par le Nom de Dieu que tu ne me feras point mourir, et que tu ne me livreras point entre les mains de mon maître, et je te conduirai vers cette troupe.
\VS{16}Et il le conduisit. Et voici, ils étaient dispersés sur toute la contrée, mangeant, buvant, et dansant, à cause de ce grand butin qu'ils avaient pris au pays des Philistins, et au pays de Juda.
\VS{17}Et David les frappa depuis l'aube du jour, jusqu'au soir du lendemain, et il n'en échappa aucun d'eux, hormis quatre cents jeunes hommes qui montèrent sur des chameaux, et s'enfuirent.
\VS{18}David recouvra tout ce que les Amalécites avaient emporté, il délivra aussi ses deux femmes.
\VS{19}Il ne leur manqua personne, depuis le plus petit jusqu'au plus grand, ni fils ni filles, ni butin, ni rien de ce qu'ils leur avaient emporté~: David ramena tout.
\VS{20}David reprit aussi tout le gros et menu bétail, qu'on mena devant les troupeaux~; et on disait~: C'est ici le butin de David.
\TextTitle{David partage le butin}
\VS{21}Puis David arriva auprès des deux cents hommes qui avaient été tellement fatigués qu'ils n'avaient pu suivre David, et qu'on avait laissés au torrent de Besor. Ils sortirent au-devant de David, et au-devant du peuple qui était avec lui. David s'étant approché du peuple, il les salua aimablement.
\VS{22}Mais tous les mauvais et méchants hommes, qui étaient allés avec David, prirent la parole, et dirent~: Puisqu'ils ne sont point venus avec nous, nous ne leur donnerons rien du butin que nous avons récupéré, sinon à chacun sa femme et ses enfants, et qu'ils les emmènent, et s'en aillent.
\VS{23}Mais David dit~: Mes frères, n'agissez pas ainsi au sujet de ce que Yahweh nous a donné~; il nous a gardés, et a livré entre nos mains la troupe qui était venue contre nous.
\VS{24}Qui vous écouterait dans cette affaire~? Car celui qui est resté près des bagages doit avoir autant de part que celui qui est descendu sur le champ de bataille~: Ils partageront ensemble.
\VS{25}Il en fut ainsi depuis ce jour et dans la suite, il en fut fait une ordonnance et une loi en Israël, jusqu'à ce jour.
\VS{26}David revint à Tsiklag, et envoya une partie du butin aux anciens de Juda, à ses amis, en disant~: Voici, un présent pour vous du butin des ennemis de Yahweh~!
\VS{27}Il en envoya à ceux de Béthel, à ceux qui étaient à Ramoth du sud, à ceux de Jatthir,
\VS{28}à ceux d'Aroër, à ceux de Siphmoth, à ceux de Eschthemoa,
\VS{29}à ceux de Racal, à ceux des villes des Jerachmeélites, à ceux des villes des Kéniens,
\VS{30}à ceux d'Horma, à ceux de Cor-Aschan, à ceux d'Athac,
\VS{31}à ceux d'Hébron, et dans tous les lieux où David avait demeuré, lui et ses gens.
\Chap{31}
\TextTitle{Les Philistins battent Israël~; Saül et ses fils meurent\FTNTT{1 Ch. 10:1-14.}}
\VerseOne{}Or les Philistins livrèrent bataille à Israël, et les hommes d'Israël s'enfuirent devant les Philistins, et furent tués sur la montagne de Guilboa.
\VS{2}Les Philistins atteignirent Saül et ses fils, et tuèrent Jonathan, Abinadab et Malkischua, fils de Saül.
\VS{3}L'effort du combat se porta sur Saül, et les archers l'atteignirent et le blessèrent grièvement.
\VS{4}Alors Saül dit à celui qui portait ses armes~: Tire ton épée, et transperce-moi, de peur que ces incirconcis ne viennent, ne me transpercent, et ne m'outragent. Mais celui qui portait ses armes refusa, parce qu'il était saisi de crainte. Saül prit l'épée, et se jeta dessus.
\VS{5}Alors celui qui portait les armes de Saül, voyant que Saül était mort, se jeta aussi sur son épée, et mourut avec lui.
\VS{6}Ainsi périrent en ce jour, Saül et ses trois fils, celui qui portait ses armes, et tous ses gens.
\VS{7}Ceux d'Israël qui étaient de ce côté de la vallée et de ce côté du Jourdain, ayant vu que les Israélites s'étaient enfuis, que Saül et ses fils étaient morts, abandonnèrent les villes et s'enfuirent de sorte que les Philistins y entrèrent et s'y établirent.
\VS{8}Or il arriva que, dès le lendemain, les Philistins vinrent pour dépouiller les morts, et ils trouvèrent Saül et ses trois fils, étendus sur la montagne de Guilboa.
\VS{9}Ils coupèrent la tête de Saül et le dépouillèrent de ses armes, qu'ils envoyèrent au pays des Philistins, dans tous les environs, pour en faire savoir les nouvelles dans les maisons de leurs idoles et parmi le peuple.
\VS{10}Ils déposèrent les armes de Saül dans le temple d'Astarté, et ils attachèrent son cadavre sur les murs de Beth-Schan.
\VS{11}Lorsque les habitants de Jabès en Galaad apprirent ce que les Philistins avaient fait à Saül,
\VS{12}tous les vaillants hommes, se levèrent et marchèrent toute la nuit, et ils enlevèrent des murs de Beth-Schan le cadavre de Saül et les cadavres de ses fils. Ils revinrent à Jabès, où ils les brûlèrent~;
\VS{13}puis ils prirent leurs os, les ensevelirent sous un tamaris près de Jabès, et ils jeûnèrent sept jours.
\PPE{}
\end{multicols}

\clearpage%& -output-directory="../pdf"
% type document & taille police
\documentclass[11pt]{book}
% package format document
\usepackage[paperwidth=6.5in, paperheight=9.05in, top=0in, bottom=0in, left=0in, right=0in]{geometry}
% formatage marges, etc.
\setlength{\voffset}{-0.7in} % offset haut
%\setlength{\hoffset}{-0.3in} % offset gauche
\setlength{\topmargin}{0in} % marge en tête
\setlength{\headsep}{0.2in} % marge header/body
\setlength{\oddsidemargin}{-0.5in} % marge texte gauche
\setlength{\evensidemargin}{-0.5in} % marge texte droite
\setlength{\textheight}{8in} % hauteur du texte
\setlength{\textwidth}{5.5in} % largeur du texte
\setlength{\columnseprule}{0.4pt} % épaisseur séparateur colonne
\setlength{\parskip}{0pt} % espace entre paragraphes
% package pour afficher les cadres
%\usepackage{showframe}
% package langue
\usepackage[francais]{babel}
% package polices système
\usepackage{fontspec}
% définition police
\setmainfont[Ligatures=TeX,Scale=0.95]{Liberation Serif}
\setsansfont{Liberation Sans}
\setmonofont{Liberation Mono}
% package parskip pour espaces entre paragraphes
\usepackage{parskip}
% package multicolonne
\usepackage{multicol}
% package liens cliquables
\usepackage[xetex]{hyperref}
% package inclusion copyright (dépandant de hyperref)
\usepackage{hyperxmp}
% copyright
\hypersetup{
    pdfauthor = {ANJC Productions},
    pdftitle = {Bible de Jésus-Christ},
    pdfkeywords = {BJC, Bible, Jesus},
    pdfcopyright = {ANJC Productions. Distribution et Diffusion Libre - Pas d'Utilisation Commerciale - Pas de Dénaturation de l'Œuvre - International},
    pdflicenseurl = {http://www.bible-de-jesus.org/}
}
% ???
\setcounter{collectmore}{-1}
% style
\pagestyle{myheadings}
% ???
\sloppy\hyphenpenalty=2000
% titres de livres
\newcommand{\ShortTitle}[1]{\def\webbook{#1}\par\goodbreak\bigskip\setcounter{footnote}{0}}
\newcommand{\BookTitle}[1]{\par\goodbreak\bigskip{\parindent=0mm\begin{center}{\small\bfseries{\LARGE #1\nopagebreak}}\end{center}}\addcontentsline{toc}{subsection}{#1}\nopagebreak\par\nobreak}
% chapitres
\newcommand{\Chap}[1]{\def\webchap{#1:}\def\webvs{0}\def\vchap{#1}\ssubsection{\centerline{\textbf{{CHAPITRE\ #1}}}}}
% versets
\newcommand{\VerseOne}{\def\webvs{1}{\up{\footnotesize 1}}\markboth{\webbook\ \webchap 1}{\webbook\ \webchap 1}}
\newcommand{\VS}[1]{\def\webvs{#1}{\up{\footnotesize #1}}\markboth{\webbook\ \webchap #1}{\webbook\ \webchap #1}}
\newcommand{\vref}[1]{\NoAutoSpaceBeforeFDP{#1}}
% commentaires
%\interfootnotelinepenalty=10000 % longueur max commentaires
\renewcommand{\thefootnote}{\alph{footnote}} % repères alphabetiques
\renewcommand{\footnoterule}{\hrule width \textwidth} % longueur ligne
\newcommand{\FTNT}[1]{\ifnum\value{footnote}>25\setcounter{footnote}{0}\fi\footnote{[\webchap\webvs]\ #1}}
\newcounter{webvst}
\newcommand{\FTNTT}[1]{
    \ifnum\value{footnote}>25\setcounter{footnote}{0}\fi
    \setcounter{webvst}{\webvs}\addtocounter{webvst}{1}
    \footnote{[\webchap\thewebvst]\ #1}
}
% titres de paragraphes
\newcommand{\ssubsection}[1]{\subsection*{\centering\footnotesize\normalfont #1}}
\newcommand{\ssubsubsection}[1]{\subsubsection*{\centering\footnotesize\normalfont #1}}
\newcommand{\TextTitle}[1]{\ssubsubsection{[\textit{#1}]}}
% dictionnaire
\newcommand{\DicoEntry}[1]{\smallskip\parindent=0mm{\textbf{#1}}\markboth{#1}{#1}}
% commandes diverses
\newcommand{\BFont}{\normalfont\small}
\newcommand{\PP}{\par\parindent=0mm}
\newcommand{\PPE}{\par\parindent=4mm}
% debut document
\begin{document}
% en-tête vide
\makeatletter
    \def\@evenhead{}
    \def\@oddhead{}
\makeatother
% inclusion intro
%\begin{center}{\LARGE Introduction}\end{center}
\begin{small}
\subsection*{Pourquoi cette Bible révisée~?}

En novembre 2013, alors que j'étais en prière, je demandais au Seigneur ce qu'il attendait de moi. Ce dernier m'a répondu à travers plusieurs songes dans lesquels il me disait de réviser la Bible. Je dois dire que j'ai eu du mal à croire que Dieu puisse me demander une telle chose. De plus, je me sentais incapable d'assumer un si grand projet, aussi je lui ai demandé à plusieurs reprises de me confirmer que c'était bien sa volonté, chose qu'il a faite. J'ai ensuite parlé de ce que j'avais reçu à des frères et sœurs qui travaillent avec moi et ces derniers m'ont confirmé que cette vision venait bien du Seigneur. Une dynamique s'est créée aussitôt et bien qu'aucun d'entre nous ne se sentît à la hauteur de la tâche qui nous était confiée, nous nous sommes rapidement organisés pour concrétiser cette vision, comptant sur le Seigneur pour qu'il nous donne les capacités et la sagesse dont nous avions besoin.\bigskip

Deux constats majeurs nous ont amenés à la conclusion qu'une révision de la Bible était plus que nécessaire. Tout d'abord, la plupart des bibles modernes les plus diffusées sont basées sur le texte minoritaire comportant une quantité importante de fautes de traduction, d'omissions et de rajouts qui altèrent la compréhension du message et induisent par conséquent le lecteur en erreur. Or il est du devoir de tout chrétien de mettre en pratique la Parole, notamment en veillant sur son authenticité.\bigskip

«~\emph{Car, je vous le dis en vérité, tant que le ciel et la terre ne passeront point, il ne disparaîtra pas de la loi un seul iota ou un seul trait de lettre jusqu'à ce que tout soit arrivé. Celui donc qui aura violé l'un de ces petits commandements, et qui aura enseigné les hommes à faire de même, sera appelé le plus petit au Royaume des cieux~; mais celui qui les observera, et qui enseignera à les observer, celui-là sera appelé grand au Royaume des cieux.}~» Matthieu 5:18-19.\bigskip

«~\emph{Je le déclare à quiconque entend les paroles de la prophétie de ce livre~: Si quelqu'un y ajoute quelque chose, Dieu le frappera des fléaux décrits dans ce livre. Et si quelqu'un retranche quelque chose des paroles du livre de cette prophétie, Dieu retranchera sa part de l'arbre de vie, et de la ville sainte, décrits dans ce livre.}~» Apocalypse 22:18-19.\bigskip

Nous ne devons pas oublier que la Bible a été initialement écrite en trois langues, à savoir l'hébreu, le grec et quelques versets en araméen. En réalisant cette révision, notre but est de restituer le sens des mots d'origine et d'expurger toute l'influence de l'ennemi. Ce travail a permis de mettre en lumière une évidence~: la personne de Jésus-Christ occupe une place centrale de Genèse à Apocalypse, ce qui ne fait que confirmer et attester sa divinité.\bigskip

«~\emph{Puis il leur dit~: C'est là ce que je disais lorsque j'étais encore avec vous, qu'il fallait que s'accomplisse tout ce qui est écrit de moi dans la loi de Moïse, dans les prophètes, et dans les psaumes.}~» Luc 24:44.\bigskip

Ensuite, nous déplorons le fait que la majorité des bibles en circulation soient vendues alors que Jésus-Christ a dit «~\emph{Vous l'avez reçu gratuitement, donnez-le gratuitement}~» (Mt. 10:8). Il est donc impensable que celui qui a chassé du temple vendeurs et changeurs puisse approuver un seul instant le commerce qui est fait avec sa Parole (Jn. 2:14-16).\bigskip

«~\emph{Vous tous qui avez soif, venez aux eaux, et vous qui n'avez pas d'argent, venez, achetez et mangez~; venez, dis-je, achetez du vin et du lait sans argent et sans rien payer~!}~» Esaïe 55:1.\bigskip

«~\emph{Il me dit aussi~: Tout est accompli. Je suis l'Alpha et l'Oméga, le commencement et la fin. A celui qui a soif, je lui donnerai de la source d'eau vive, gratuitement.}~» Apocalypse 21:6.\bigskip

«~\emph{Et l'Esprit et l'épouse disent~: Viens. Et que celui qui entend dise~: Viens. Et que celui qui a soif vienne~; que celui qui veut prenne gratuitement de l'eau de la vie.}~» Apocalypse 22:17.\bigskip

Les apôtres ont scrupuleusement respecté l'ordre du Seigneur en Matthieu 10:8. Pierre a dénoncé avec la plus grande sévérité Simon, le magicien qui avait eu la folie de croire que le don de Dieu pouvait être monnayé. Et durant tout son service, Paul a enseigné l'Evangile gratuitement.\bigskip

«~\emph{Puis ils leur imposèrent les mains, et ils reçurent le Saint-Esprit. Lorsque Simon vit que le Saint-Esprit était donné par l’imposition des mains des apôtres, il leur présenta de l’argent, en leur disant : Donnez-moi aussi ce pouvoir, afin que tous ceux à qui j’imposerai les mains reçoivent le Saint-Esprit. Mais Pierre lui dit~: Que ton argent périsse avec toi, puisque tu as estimé que le don de Dieu s’acquérait avec de l’argent. Tu n’as point de part ni d’héritage en cette affaire ; car ton coeur n’est point droit devant Dieu. Repens-toi donc de cette méchanceté, et prie Dieu, afin que, s’il est possible, la pensée de ton coeur te soit pardonnée. Car je vois que tu es dans un fiel très amer et dans un lien d’iniquité.}~» Actes 8:17-23.\bigskip

«~\emph{Je n'ai désiré ni l'argent, ni l'or, ni les vêtements de personne.}~» Actes 20:33.\bigskip

«~\emph{Quelle récompense en ai-je donc~? C’est qu’en prêchant l’Evangile, je prêche l’Evangile de Christ sans qu’il en coûte rien, afin que je n’abuse pas de mon pouvoir dans l’Evangile.}~» 1 Corinthiens 9:18.\bigskip

Nous pensons qu'il est juste et honnête que la Bible porte le nom de son véritable auteur et qu'elle soit gratuitement diffusée selon sa volonté et l'ordre clair qu'il a donné. Cette Bible s'appelle donc La Bible de Jésus-Christ et est gratuitement mise à la disposition de ceux qui souhaitent se la procurer.\bigskip

\subsection*{Comment a été réalisée cette révision~?}

Pour réaliser cette révision, nous nous sommes appuyés sur le texte majoritaire (originaux et traductions). Ainsi, tout en essayant de conserver un vocabulaire qui soit à la portée de tous, certains mots et expressions ont été changés pour restituer pleinement leur signification initiale. A titre d'exemple, vous constaterez régulièrement que certains mots sont répétés deux fois de suite. Cela n'est pas une erreur mais la restitution littérale de certaines expressions qui insistent sur une vérité (voir commentaire en Gn. 2:15-17). En effet, Dieu parle une fois et une seconde fois pour avertir les hommes (Job. 33:14). «~Et quant à ce que le songe a été réitéré à Pharaon pour la seconde fois, c'est que la chose est arrêtée de la part de Dieu, et que Dieu se hâtera de l'exécuter~» (Genèse 41:32).\bigskip

Les Ecrits ont été classés dans l'ordre de la tradition juive pour le Tanakh et dans l'ordre chronologique de leur rédaction pour les épîtres afin de permettre au lecteur de mieux comprendre le contexte et le déroulement de la prophétie biblique. L'appellation «~Ancien Testament~» a été remplacée par l'acronyme hébreu Tanakh (voir sommaire). Quant à ce qu'on appelle communément le «~Nouveau Testament~», il sera désormais question du Testament de Jésus. En effet, l'Ancienne Alliance n'étant pas un testament, on ne peut donc pas parler de «~Nouveau Testament~» mais plutôt d'une Nouvelle Alliance (voir commentaires en Ex. 19:5~; Mt. 27:51~; Jn. 19:30).\bigskip

Je remercie tout d'abord le Seigneur pour son aide précieuse qu'il m'a apportée pour la révision de cette Bible, ainsi qu'à celles et ceux qui m’ont assisté dans ce travail.\newline

\begin{flushright}
Shora KUETU
\end{flushright}
\end{small}

%% formatage sommaire
%\makeatletter
%\renewcommand\tableofcontents{
%    \begin{center}{\LARGE Sommaire}\end{center}
%    \setlength{\columnseprule}{0pt} % désactivation séparateur colonne temp
%    \begin{multicols}{2}\medskip\footnotesize{\@starttoc{toc}}\end{multicols}
%    \setlength{\columnseprule}{0.4pt} % réactivation séparateur colonne
%}
%\makeatother
%% inclusion table des matières
%\clearpage\tableofcontents\clearpage
%% en-tête pages
%\makeatletter
\def\@evenhead{{\NoAutoSpaceBeforeFDP{\small{\rightmark\hfil\thepage\hfil\leftmark}}}}
\def\@oddhead{{\NoAutoSpaceBeforeFDP{\small{\rightmark\hfil\thepage\hfil\leftmark}}}}
\makeatother
% inclusion des livres
\addcontentsline{toc}{chapter}{Tanakh}\pagenumbering{arabic}\clearpage
%\addcontentsline{toc}{section}{Torah (Loi)}\clearpage
%\clearpage\ShortTitle{Genèse}\BookTitle{Genèse}\BFont
\noindent\hrulefill
{\footnotesize
\textit{
\bigskip
{\centering{}
\\Auteur : Probablement Moïse
\\(Heb. : Bereshit)
\\Signification : Au commencement
\\Thème : Le Messie d'Israël
\\Date de rédaction : Env. 1450-1410 av. J.-C.\\}
}
%\bigskip
\textit{
\\Premier livre du Tanakh, la Genèse est le livre des commencements.
Elle relate l’histoire des origines de l’humanité, la création des cieux, de la terre et de tout ce qui s’y trouve par Yahweh, le Dieu créateur.
%\bigskip
\\Il y est décrit le péché de l’homme et sa séparation d’avec Dieu, ainsi que la décadence de l’univers qui en résulta. En réponse à la méchanceté du cœur de l’homme, Yahweh  exerça sa justice en détruisant la terre par le déluge.
Dans sa prescience, Yahweh avait cependant résolu de se réconcilier avec l’homme. Il se révéla donc comme sauveur en accordant sa grâce à Noé et à sa famille. Après cet événement, les hommes se tournèrent une fois de plus vers le mal en tentant Dieu par la construction de la tour de Babel, œuvre à l’origine de la dispersion des nations.
%\bigskip
\\Ce livre présente aussi l’élection d’Abraham, originaire d’Ur en Chaldée - actuelle Mésopotamie - qui reçut la promesse divine de devenir une grande nation, en qui toutes les familles de la terre seraient bénies. Le récit se poursuit par l’histoire de ses descendants Isaac, Jacob et ses douze fils,  qui formèrent par la suite la nation d’Israël.\bigskip
}
}
\par\nobreak\noindent\hrulefill
\begin{multicols}{2}
\Chap{1}
\VerseOne{}Au commencement, Dieu créa les cieux et la terre.
\TextTitle{La terre devient informe et vide}
\VS{2}Et la terre devint informe et vide\FTNT{Les termes «~informe~» et «~vide~» viennent des mots hébreux «~tohuw~» et «~bohuw~»  qui désignent la confusion, le chaos, la vanité.}, les ténèbres étaient à la surface de l'abîme ; et l'Esprit de Dieu se mouvait au-dessus des eaux.
\TextTitle{Jour «~un~» : Apparition de la lumière}
\VS{3}Dieu dit : Que la lumière apparaisse\FTNT{«~Que la lumière apparaisse !~» (Es. 9:1 ; Mt. 4:16 ; Jn. 1:1-5). Cette lumière n’est autre que Yahweh lui-même qui va s’incarner en la personne de Jésus-Christ pour chasser les ténèbres (2 S. 22:9-12 ; Es. 60:1 ; 60:19-20 ; Jn. 1:1-2 ; 8:12-14 ; 2 Co. 4:6).} ; et la lumière apparut.
\VS{4}Et Dieu vit que la lumière était bonne ; et Dieu sépara la lumière des ténèbres.
\VS{5}Dieu appela la lumière jour, et il appela les ténèbres nuit\FTNT{La lumière et les ténèbres, ainsi que leurs champs lexicaux respectifs, personnifient  souvent Jésus et Satan.  Ainsi, Jésus est la Lumière du monde (Jn. 9:5), l’Etoile brillante du matin (Ap. 22:16), le Soleil levant ou le Soleil de la justice (Mal. 4:2 ; Ps. 19:6 ; Lu. 1:78). Il est associé au jour (Jn. 9:4) d’où les expressions  «~Jour du Seigneur~» (1 Th. 5:2) ou «~Jour de Yahweh~» (Joë. 1:15). A l’inverse, la Bible associe Satan aux ténèbres (Es. 8:23 ; Ps. 143:3 ; Ep. 6:12 ; Col. 1:13) et à la nuit (Jn. 9:4 ; Ro. 13:12).}. Ainsi fut le soir, ainsi fut le matin\FTNT{Contrairement au calendrier grégorien où le jour commence à minuit, selon Dieu et le calendrier hébraïque, le jour commence le soir à 18 heures pour se terminer le lendemain à la même heure. Voir commentaire en Mc. 16:9.} ; ce fut le  jour un\FTNT{L’hébreu utilise le terme «~ehad~» qui signifie «~un~», au sens de l’indivisible, pour qualifier le premier jour. Ce jour nous parle de Yahweh tel qu’il s’est présenté à son peuple sur le mont Sinaï en De. 6:4:«~Shema Yisrael Yahweh elohénou Yahweh ehad~» («~écoute Israël, Yahweh [est] notre Dieu, Yahweh [est] UN~»). Un n’est pas divisible sinon on obtient un zéro ce qui équivaut au néant. Dieu est tout sauf le néant, il remplit tout (Ep. 1:23), il est partout (Ps. 139:7-13), les cieux des cieux ne peuvent le contenir (1 R. 8:27).}.
\TextTitle{Second jour : Une étendue entre les eaux}
\VS{6}Puis Dieu dit : Qu'il y ait une étendue entre les eaux, et qu'elle sépare les eaux d'avec les eaux.
\VS{7}Dieu donc fit l'étendue, et il sépara les eaux qui sont au-dessous de l'étendue d'avec celles qui sont au-dessus de l'étendue, et il fut ainsi.
\VS{8}Et Dieu appela l'étendue cieux. Ainsi fut le soir, ainsi fut le matin ; ce fut le second jour.
\TextTitle{Troisième jour : Les mers, la terre et la végétation}
\VS{9}Puis Dieu dit : Que les eaux qui sont au-dessous des cieux soient rassemblées en un lieu, et que le sec paraisse ; et il fut ainsi.
\VS{10}Et Dieu appela le sec terre ; et il appela l'amas des eaux mers ; et Dieu vit que cela était bon.
\VS{11}Puis Dieu dit : Que la terre produise de la verdure, de l'herbe portant de la semence, et des arbres fruitiers portant du fruit selon leur espèce, qui aient leur semence en eux-mêmes sur la terre ; et il fut ainsi.
\VS{12}La terre donc produisit de la verdure, de l'herbe portant de la semence selon son espèce ; et des arbres portant du fruit qui avaient leur semence en eux-mêmes selon leur espèce ; et Dieu vit que cela était bon.
\VS{13}Ainsi fut le soir, ainsi fut le matin ; ce fut le troisième jour.
\TextTitle{Quatrième jour : Les luminaires du ciel}
\VS{14}Puis Dieu dit : Qu'il y ait des luminaires dans l'étendue du ciel pour séparer la nuit d'avec le jour, et qui servent de signes pour les saisons, pour les jours, et pour les années ;
\VS{15}et qu’ils servent de luminaires dans l'étendue du ciel afin d'éclairer la terre ; et il fut ainsi.
\VS{16}Dieu donc fit deux grands luminaires, le plus grand luminaire pour présider au jour, et le plus petit luminaire pour présider à la nuit ; il fit aussi les étoiles.
\VS{17}Dieu les plaça dans l'étendue du ciel pour éclairer la terre,
\VS{18}pour présider au jour et à la nuit, et pour séparer la lumière d’avec les ténèbres ; et Dieu vit que cela était bon.
\VS{19}Ainsi fut le soir, ainsi fut le matin ; ce fut le quatrième jour.
\TextTitle{Cinquième jour : Les animaux vivant dans les eaux et les airs\FTNTT{ Ge. 2:19}}
\VS{20}Puis Dieu dit : Que les eaux produisent en toute abondance des reptiles vivants ; et qu'il y ait des oiseaux qui volent sur la terre vers l'étendue du ciel.
\VS{21}Dieu créa les grands poissons et tous les animaux vivants qui se meuvent et que les eaux produisirent en toute abondance selon leur espèce ; il créa aussi tout oiseau ayant des ailes selon son espèce ; et Dieu vit que cela était bon.
\VS{22}Dieu les bénit en disant : Soyez féconds, multipliez, et remplissez les eaux des mers ; et que les oiseaux multiplient sur la terre.
\VS{23}Ainsi fut le soir, ainsi fut le matin ; ce fut le cinquième jour.
\TextTitle{Sixième jour : Les animaux terrestres}
\VS{24}Puis Dieu dit : Que la terre produise des animaux selon leur espèce, le bétail, les reptiles, et les bêtes de la terre selon leur espèce ; et il fut ainsi.
\VS{25}Dieu donc fit les animaux de la terre selon leur espèce, et le bétail selon son espèce, et les reptiles de la terre selon leur espèce ; et Dieu vit que cela était bon.
\TextTitle{Mission confiée à l’homme ; son autorité sur la création}
\VS{26}Puis Dieu dit : Faisons l'homme à notre image, selon notre ressemblance\FTNT{L’image de Dieu n’est autre que Jésus-Christ lui-même (Col. 1:15). Adam, qui signifie terrien, a été créé à  l’image du dernier Adam (1 Co. 15:40-49) qui est venu comme Fils, afin de nous montrer le modèle de fils et de filles que Dieu souhaite (Ro. 8:29). Nous avons ici une autre image de l’incarnation de Dieu en la personne de Jésus-Christ. Ainsi, avant que l’homme ne pèche, le projet de la rédemption était déjà là (1 Pi. 1:19-21).}, et qu'il domine sur les poissons de la mer, sur les oiseaux du ciel, sur le bétail, sur toute la terre, et sur tout reptile qui rampe sur la terre.
\VS{27}Dieu créa l'homme à son image, il le créa à l'image de Dieu, il les créa mâle et femelle.
\TextTitle{Autorité de l'homme sur la création}
\VS{28}Dieu les bénit et leur dit : Soyez féconds, multipliez, remplissez la terre, et assujettissez-la ; et dominez sur les poissons de la mer, sur les oiseaux du ciel, et sur toute bête qui se meut sur la terre.
\VS{29}Et Dieu dit : Voici, je vous donne toute herbe portant de la semence qui est sur toute la terre, et tout arbre ayant en lui du fruit d'arbre et portant de la semence, ce sera votre nourriture.
\VS{30}Et à tout animal de la terre, à tout oiseau du ciel, et à tout ce qui se meut sur la terre, ayant en soi un souffle de vie, je donne toute herbe verte pour nourriture. Et cela fut ainsi.
\VS{31}Dieu vit tout ce qu'il avait fait, et voici cela était très bon. Ainsi fut le soir, ainsi fut le matin ; ce fut le sixième jour.
\Chap{2}
\TextTitle{Septième jour : Le sabbat}
\VerseOne{}Les cieux donc et la terre furent achevés, avec toute leur armée.
\VS{2}Dieu acheva au septième jour son œuvre qu'il avait faite, et il se reposa au septième jour de toute son œuvre qu'il avait faite.
\VS{3}Dieu bénit le septième jour, et le sanctifia, parce qu'en ce jour-là il s'était reposé de toute son œuvre qu'il avait créée en la faisant.
\VS{4}Telles sont les origines des cieux et de la terre, lorsqu'ils furent créés.
\VS{5}Lorsque Yahweh Dieu fit la terre et les cieux, aucun arbuste des champs n’était encore sur la terre, et aucune herbe des champs ne germait encore ; car Yahweh Dieu n'avait pas fait pleuvoir sur la terre, et il n'y avait point d'homme pour cultiver la terre.
\TextTitle{Yahweh forme l’homme et le place en Eden\FTNTT{Job 10:8-9 ; Ps. 119:73}}
\VS{6}Et il monta une vapeur de la terre qui arrosa toute la surface de la terre.
\VS{7}Yahweh Dieu forma l'homme de la poussière de la terre, et il souffla dans ses narines un souffle de vie ; et l'homme devint une âme vivante.
\VS{8}Aussi Yahweh Dieu planta un jardin en Eden, du côté de l’orient, et il y mit l'homme qu'il avait formé.
\TextTitle{Description du jardin en Eden\FTNTT{Ge. 1:28-3:6}}
\VS{9}Yahweh Dieu fit germer de la terre des arbres de toute espèce, agréables à voir et bons à manger, et l'arbre de la vie au milieu du jardin, et l'arbre de la connaissance du bien et du mal.
\VS{10}Un fleuve sortait d'Eden pour arroser le jardin ; et de là il se divisait en quatre bras.
\VS{11}Le nom du premier est Pischon ; c'est le fleuve qui coule en entourant tout le pays de Havila où se trouve l'or.
\VS{12}L'or de ce pays est bon ; c'est là aussi que se trouvent le bdellium et la pierre d'onyx.
\VS{13}Le nom du second fleuve est Guihon ; c'est celui qui coule en entourant tout le pays de Cusch.
\VS{14}Le nom du troisième fleuve est Hiddékel, qui coule vers l'Assyrie ; et le quatrième fleuve est l'Euphrate.
\TextTitle{Commandement donné par Yahweh à l’homme\FTNTT{Ge. 1:28}}
\VS{15}Yahweh Dieu prit donc l'homme et le mit dans le jardin d'Eden pour le cultiver et pour le garder.
\VS{16}Puis Yahweh Dieu donna cet ordre à l'homme, en disant : Tu mangeras, tu mangeras\FTNT{En hebreu, le mot «akal» signifie «~manger~», «~se nourrir~», «~goûter~», «~jouir~», «~dévorer~», «~consumer~», et il a été utilisé deux fois de suite dans ce passage.} de tout arbre du jardin.
\VS{17}Mais quant à l'arbre de la connaissance du bien et du mal, tu n'en mangeras point, car le jour où tu en mangeras, tu mourras, tu mourras\FTNT{Dans la plupart des versions ce passage est mal traduit par «~tu mourras certainement~», alors que le terme mort en hébreu «~muwth~» est utilisé deux fois dans ce passage, et les écritures nous parlent de la mort physique et de la seconde mort, qui est le lac de feu (Ap. 2:11 ; Ap. 20:6,14). La mort physique précède la seconde physique.}.
\TextTitle{Yahweh forme une femme pour l'homme\FTNTT{Ge. 1:27}}
\VS{18}Yahweh Dieu dit : Il n'est pas bon que l'homme soit seul ; je lui ferai une aide semblable à lui.
\VS{19}Car Yahweh Dieu forma de la terre tous les animaux des champs et tous les oiseaux du ciel, puis il les fit venir vers Adam pour voir comment il les nommerait, et afin que le nom qu'Adam donnerait à tout animal fût, son nom.
\VS{20}Et Adam donna des noms à tout le bétail, et aux oiseaux du ciel, et à tous les animaux des champs ; mais pour Adam, il ne trouva point d'aide semblable à lui.
\VS{21}Et Yahweh Dieu fit tomber un profond sommeil sur Adam, qui s'endormit ; et Dieu prit une de ses côtes, et referma la chair à la place de cette côte.
\VS{22}Yahweh Dieu forma une femme de la côte qu'il avait prise d'Adam, et il l’amena vers Adam\FTNT{1 Co. 11:8.}.
\TextTitle{Union d’Adam et Eve}
\VS{23}Alors Adam dit : Voici cette fois celle qui est os de mes os et chair de ma chair ; on l’appellera femme, parce qu'elle a été prise de l'homme.
\VS{24}C'est pourquoi l'homme quittera son père et sa mère et s’attachera à sa femme, et ils deviendront une seule chair\FTNT{Ep. 5:30-31 ; Mt. 19:5 ; Mc. 10:7 ; 1 Co. 6:16.}.
\VS{25}Adam et sa femme étaient tous deux nus, et ils n’en avaient pas honte.
\Chap{3}
\TextTitle{Séduction du serpent et chute de l’homme}
\VerseOne{}Or le serpent\FTNT{Satan ou le serpent ancien (Ap. 12:9 ; Ap. 20:2).} était le plus prudent\FTNT{La prudence, la ruse, la subtilité du serpent, sont marquées dans l'Ecriture comme des qualités qui le distinguent des autres animaux (Mt. 10:16).} de tous les animaux des champs que Yahweh Dieu avait faits ; et il dit à la femme : Quoi ! Dieu a dit : Vous ne mangerez pas de tous les arbres du jardin ?
\VS{2}La femme répondit au serpent : Nous mangeons du fruit des arbres du jardin ;
\VS{3}mais quant au fruit de l'arbre qui est au milieu du jardin, Dieu a dit : Vous n'en mangerez point, et vous ne le toucherez point, de peur que vous ne mouriez.
\VS{4}Alors le serpent dit à la femme : Vous ne mourrez nullement ;
\VS{5}mais Dieu sait que le jour où vous en mangerez, vos yeux seront ouverts, et vous serez comme des dieux, connaissant le bien et le mal.
\VS{6}La femme donc voyant que le fruit de l'arbre était bon à manger et agréable à la vue, et que cet arbre était désirable pour donner de la science ; elle prit de son fruit, et en mangea, et elle en donna aussi à son mari qui était auprès d’elle, et il en mangea.
\TextTitle{La connaissance du bien et du mal}
\VS{7}Les yeux de tous les deux s’ouvrirent, ils connurent qu'ils étaient nus, et ils cousirent ensemble des feuilles de figuier, et s'en firent des ceintures.
\VS{8}Alors ils entendirent au vent du jour la voix de Yahweh Dieu qui se promenait par le jardin ; et Adam et sa femme se cachèrent loin de la face de Yahweh Dieu, au milieu des arbres du jardin.
\VS{9}Mais Yahweh Dieu appela Adam et lui dit : Où es-tu ?
\VS{10}Il répondit : J'ai entendu ta voix dans le jardin, et j'ai eu peur parce que je suis nu, et je me suis caché.
\VS{11}Et Dieu dit : Qui t'a appris que tu es nu ? Est-ce que tu as mangé du fruit de l'arbre dont je t'avais défendu de manger ?
\VS{12}Adam répondit : La femme que tu m'as donnée pour être avec moi m'a donné du fruit de l'arbre, et j'en ai mangé.
\VS{13}Et Yahweh Dieu dit à la femme : Pourquoi as-tu fait cela ? Et la femme répondit : Le serpent m'a séduite, et j'en ai mangé.
\TextTitle{La création soumise à la vanité\FTNTT{Ro. 8:20-22}}
\VS{14}Alors Yahweh Dieu dit au serpent : Parce que tu as fait cela, tu seras maudit entre tout le bétail et entre tous les animaux des champs ; tu marcheras sur ton ventre, et tu mangeras la poussière\FTNT{La poussière dont il est question n’est autre que l’homme pécheur (Ge. 3:19). Satan ne peut rien contre les véritables enfants de Dieu (Mt. 16:18 ; Lu. 10:19).} tous les jours de ta vie.
\VS{15}Je mettrai inimitié entre toi et la femme\FTNT{La femme représente en premier lieu Eve, la mère de tous les hommes. Ici, elle représente aussi Israël, l’épouse de Yahweh selon Ge. 37:5-11 et Ap. 12:1.}, et entre ta postérité\FTNT{La postérité du serpent regroupe l’homme impie (2 Th. 2:3-4 ; 1 Jn. 2:18-22), et tous ceux qui n’ont pas reçu Jésus-Christ comme Seigneur et Sauveur. En effet, seuls ceux qui ont reçu Jésus dans leur vie sont appelés enfants de Dieu (Jn. 1:12 ; 1 Jn. 3:8-10 ; 1 Jn. 5:19).} et sa postérité\FTNT{La postérité de la femme regroupe Jésus-Christ homme (Es. 7:14 ; Lu. 2:4-7), et l’Église, le Corps de Christ (Col. 1:24).} ; celle-ci te brisera la tête, et tu lui blesseras le talon.
\VS{16}Et il dit à la femme : J'augmenterai beaucoup la souffrance de tes grossesses ; tu enfanteras dans la douleur tes enfants ; tes désirs se porteront vers ton mari, et il dominera sur toi.
\VS{17}Puis il dit à Adam : Parce que tu as obéi à la parole de ta femme, et que tu as mangé le fruit de l'arbre au sujet duquel je t'avais donné cet ordre, en disant : Tu n'en mangeras point, la terre sera maudite à cause de toi ; tu en mangeras les fruits dans la peine, tous les jours de ta vie.
\VS{18}Et elle te produira des épines et des chardons ; et tu mangeras l'herbe des champs.
\VS{19}C’est à la sueur de ton visage que tu mangeras du pain, jusqu'à ce que tu retournes dans la terre, d’où tu as été pris ; car tu es poussière, et tu retourneras dans la poussière.
\TextTitle{L’homme et la femme revêtu de tuniques de peaux}
\VS{20}Et Adam appela sa femme Eve, parce qu'elle a été la mère de tous les vivants.
\VS{21}Yahweh Dieu fit à Adam et à sa femme des tuniques de peaux, et il les en revêtit.
\VS{22}Yahweh Dieu dit : Voici, l'homme est devenu comme l'un de nous, connaissant le bien et le mal. Mais maintenant il faut prendre garde, qu’il n’avance sa main, et aussi qu’il ne prenne de l'arbre de vie, et qu’il n’en mange, et ne vive éternellement.
\TextTitle{L’homme chassé du jardin}
\VS{23}Et Yahweh Dieu le chassa du jardin d’Eden pour qu’il cultive la terre d’où il avait été pris.
\VS{24}C’est ainsi qu’il chassa l'homme, et il mit à l’orient du jardin d’Eden des chérubins qui tournent ça et là une épée flamboyante pour garder le chemin de l'arbre de vie.
\Chap{4}
\TextTitle{La jalousie de Caïn contre son frère Abel}
\VerseOne{}Adam connut Eve sa femme ; elle conçut, et enfanta Caïn ; et elle dit : J'ai acquis un homme de par Yahweh.
\VS{2}Elle enfanta encore Abel, son frère ; et Abel fut berger, et Caïn laboureur.
\VS{3}Or, au bout de quelque temps, Caïn offrit à Yahweh une offrande des fruits de la terre\FTNT{Caïn était du diable, il est l’archétype du religieux qui pense pouvoir être sauvé par les œuvres (Lu. 11:51 ; 1 Jn. 3:12). Son offrande fut rejetée car il avait apporté devant Dieu le fruit de la terre qui avait été maudite (Ge. 3:17).  Cela revenait à offrir à Dieu le péché, la malédiction.} ;
\VS{4}et Abel, de son côté, offrit des premiers-nés de son troupeau, et de leur graisse\FTNT{Abel était juste et pieux, aussi il sut instinctivement apporter une offrande agréable à Dieu (Mt. 23:35 ; Lu. 11:51 ; Hé. 11:4). En l’occurrence, son offrande préfigurait le sacrifice du Seigneur.}. Yahweh eut égard à Abel, et à son offrande.
\VS{5}Mais il n'eut point d'égard à Caïn ni à son offrande ; et Caïn fut fort irrité, et son visage fut abattu.
\TextTitle{Yahweh avertit Caïn}
\VS{6}Et Yahweh dit à Caïn : Pourquoi es-tu irrité, et pourquoi ton visage est-il abattu ?
\VS{7}Si tu agis bien, tu relèveras ton visage, et si tu agis mal, le péché est couché à la porte, et ses désirs se portent vers toi ;  mais toi, domine sur lui.
\TextTitle{Caïn tue son frère Abel\FTNTT{Ge. 4:23}}
\VS{8}Et Caïn parla avec Abel son frère, et comme ils étaient dans les champs, Caïn se jeta sur Abel, son frère, et le tua.
\VS{9}Yahweh dit à Caïn : Où est Abel ton frère ? Et il lui répondit : Je ne sais, suis-je le gardien de mon frère, moi ?
\VS{10}Et Dieu dit : Qu'as-tu fait ? La voix du sang de ton frère crie de la terre à moi.
\VS{11}Maintenant donc tu seras maudit de la terre, qui a ouvert sa bouche pour recevoir de ta main le sang de ton frère.
\VS{12}Quand tu cultiveras la terre, elle ne te donnera plus son fruit, et tu seras vagabond et fugitif sur la terre.
\VS{13}Caïn dit à Yahweh : Mon châtiment est trop grand pour être supporté.
\VS{14}Voici, tu me chasses aujourd'hui de cette terre ; je serai caché loin de ta face, je serai vagabond et fugitif sur la terre, et quiconque me trouvera me tuera.
\VS{15}Yahweh lui dit : Si quelqu’un tuait Caïn, Caïn serait vengé sept fois. Ainsi Yahweh mit une marque sur Caïn afin que quiconque le trouverait ne le tue point.
\TextTitle{Caïn bâtit une cité loin de Yahweh}
\VS{16}Alors, Caïn s’éloigna de la face de Yahweh, et habita dans la terre de Nod, à l'orient d’Eden.
\VS{17}Puis Caïn connut sa femme ; elle conçut et enfanta Hénoc. Il bâtit une ville, et il donna à cette ville le nom de son fils Hénoc.
\VS{18}Hénoc engendra Irad, Irad engendra Mehujaël, Mehujaël engendra Metuschaël, et Métuschaël engendra Lémec.
\VS{19}Lémec prit deux femmes ; le nom de l'une était Ada, et le nom de l'autre Tsilla.
\VS{20}Ada enfanta Jabal : Il fut le père de ceux qui habitent dans les tentes et près des troupeaux.
\VS{21}Le nom de son frère était Jubal : Il fut le père de tous ceux qui jouent de la harpe et du chalumeau.
\VS{22}Tsilla aussi enfanta Tubal-Caïn, qui forgeait toutes sortes d'instruments d'airain et de fer. La soeur de Tubal-Caïn était Naama.
\VS{23}Lémec dit à Ada et à Tsilla ses femmes : Ecoutez ma voix femmes de Lémec, écoutez ma parole ! J’ai tué un homme pour ma blessure et un jeune homme pour ma meurtrissure.
\VS{24}Car si Caïn est vengé sept fois, Lémec le sera soixante-dix-sept fois.
\TextTitle{Naissance de Seth}
\VS{25}Adam connut encore sa femme ; elle enfanta un fils, et il l’appela du nom de Seth, car, dit-il, Dieu m'a donné un autre fils à la place d'Abel, que Caïn a tué.
\VS{26}Il naquit aussi un fils à Seth, et il l'appela du nom d’Enosch. C’est alors que l’on commença à proclamer le nom de Yahweh.
\Chap{5}
\TextTitle{La postérité d'Adam soumise à la mort\FTNTT{Ro. 5:12}}
\VerseOne{}Voici le livre de la postérité d'Adam, depuis le jour où Dieu créa l'homme, il le fit à la ressemblance de Dieu.
\VS{2}Il les créa mâle et femelle, et les bénit, et il leur donna le nom d'homme, le jour où ils furent créés.
\VS{3}Adam vécut cent trente ans, et engendra un fils à sa ressemblance, selon son image\FTNT{Désormais les hommes naissent à la ressemblance d’Adam, c’est-à-dire pécheurs (Ro. 3:23 ; Ro. 5:14-17).}, et il lui donna le nom de Seth.
\VS{4}Les jours d'Adam, après qu'il eut engendré Seth, furent de huit cents ans, et il engendra des fils et des filles.
\VS{5}Tous les jours qu'Adam vécut furent de neuf cent trente ans ; puis il mourut.
\TextTitle{De Seth aux fils de Noé\FTNTT{Ro. 5:12}}
\VS{6}Seth aussi vécut cent cinq ans, et engendra Enosch.
\VS{7}Seth, après qu'il eut engendré Enosch, vécut huit cent sept ans ; et il engendra des fils et des filles.
\VS{8}Tous les jours que Seth vécut furent de neuf cent douze ans ; puis il mourut.
\VS{9}Enosch, ayant vécu quatre-vingt-dix ans, engendra Kénan.
\VS{10}Enosch, après qu'il eut engendré Kénan, vécut huit cent quinze ans, et il engendra des fils et des filles.
\VS{11}Tous les jours qu'Enosch vécut furent de neuf cent cinq ans ; puis il mourut.
\VS{12}Kénan, ayant vécu soixante-dix ans, engendra Mahalaleel.
\VS{13}Kénan, après qu'il eut engendré Mahalaleel, vécut huit cent quarante ans ; et il engendra des fils et des filles.
\VS{14}Tous les jours que Kénan vécut furent de neuf cent dix ans ; puis il mourut.
\VS{15}Mahalaleel vécut soixante-cinq ans ; et il engendra Jéred.
\VS{16}Et Mahalaleel, après qu'il eut engendré Jéred, vécut huit cent trente ans, et il engendra des fils et des filles.
\VS{17}Tous les jours donc que Mahalaleel vécut furent de huit cent quatre-vingt-quinze ans ; puis il mourut.
\VS{18}Jéred, ayant vécu cent soixante-deux ans, engendra Hénoc.
\VS{19}Jéred, après avoir engendré Hénoc, vécut huit cents ans, et il engendra des fils et des filles.
\VS{20}Tous les jours que Jéred vécut furent de neuf cent soixante-deux ans ; puis il mourut.
\VS{21}Hénoc vécut soixante-cinq ans, et engendra Metuschélah.
\VS{22}Hénoc, après qu'il eut engendré Metuschélah, marcha avec Dieu trois cents ans ; et il engendra des fils et des filles.
\VS{23}Tous les jours qu'Hénoc vécut furent de trois cent soixante-cinq ans.
\VS{24}Hénoc marcha avec Dieu ; mais il ne parut plus parce que Dieu le prit.
\VS{25}Metuschélah, ayant vécu cent quatre-vingt-sept ans, engendra Lémec.
\VS{26}Metuschélah, après qu'il eut engendré Lémec, vécut sept cent quatre-vingt-deux ans ; et il engendra des fils et des filles.
\VS{27}Tous les jours que Metuschélah vécut furent de neuf cent soixante-neuf ans ; puis il mourut.
\VS{28}Lémec aussi vécut cent quatre-vingt-deux ans, et il engendra un fils.
\VS{29}Il l'appela Noé, en disant : Celui-ci nous consolera de notre oeuvre, et du travail pénible de nos mains, sur la terre que Yahweh a maudite.
\VS{30}Lémec, après qu'il eut engendré Noé, vécut cinq cent quatre-vingt-quinze ans ; et il engendra des fils et des filles.
\VS{31}Tous les jours que Lémec vécut furent de sept cent soixante-dix-sept ans ; puis il mourut.
\VS{32}Noé, âgé de cinq cents ans, engendra Sem, Cham, et Japhet.
\Chap{6}
\TextTitle{Le mal dans le cœur de l'homme\FTNTT{Ro. 5:12}}
\VerseOne{}Lorsque les hommes eurent commencé à se multiplier sur la face de la terre, et qu'ils eurent engendré des filles,
\VS{2}les fils de Dieu\FTNT{Ici, les fils de Dieu sont des anges qui ont quitté leur demeure (Jud. 1:5-7).} virent que les filles des hommes étaient belles, et ils en prirent pour femmes parmi toutes celles qu'ils choisirent.
\TextTitle{Yahweh ne conteste plus avec les hommes}
\VS{3}Yahweh dit : Mon Esprit ne contestera point à toujours avec les hommes\FTNT{C’est le Saint-Esprit qui nous convainc de péché, de jugement et de justice (Jn. 16:8). Lorsqu’il constate que le cœur d’une personne est définitivement endurci au point de refuser la repentance, il renonce à la convaincre de péché et il se retire. La génération antédiluvienne avait définitivement rejeté Dieu en choisissant de faire du mal son idole (Ge. 6:5). Elle était allée si loin dans l’abomination au point de s’accoupler avec des anges déchus (Ge. 6:4), ce qui laisse supposer un culte volontaire aux démons. Lorsque le Saint-Esprit est retiré d’une personne, il est remplacé par l’esprit d’égarement qui enferme le pécheur dans l’erreur et l’entraîne ainsi à sa condamnation éternelle (Mt. 12:31 ; 2 Th. 2:11 )}, car les hommes ne sont que chair, et leurs jours seront de cent vingt ans.
\TextTitle{Le monde avant le déluge\FTNTT{Lu. 17:27}}
\VS{4}Les géants étaient sur la terre en ce temps-là. Il en fut de même après que les fils de Dieu furent venus vers les filles des hommes, et qu’elles leur eurent donné des enfants. Ce sont ces hommes vaillants qui furent des gens de renom dans l’antiquité.
\TextTitle{Yahweh prépare un jugement}
\VS{5}Yahweh vit que la méchanceté des hommes était très grande sur la terre, et que toute l'imagination des pensées de leur cœur n'était que mal en tout temps.
\VS{6}Yahweh se repentit d'avoir fait l'homme sur la terre, et il fut affligé en son cœur.
\VS{7}Et Yahweh dit : J'exterminerai de la face de la terre les hommes que j'ai créés, depuis les hommes jusqu'au bétail, jusqu'aux reptiles, et même jusqu'aux oiseaux du ciel ; car je me repens de les avoir faits.
\TextTitle{La grâce de Yahweh sur Noé : Construction de l'arche}
\VS{8}Mais Noé trouva grâce aux yeux de Yahweh.
\VS{9}Voici la postérité de Noé. Noé était un homme juste et intègre en son temps ; Noé marchait avec Dieu.
\VS{10}Noé engendra trois fils : Sem, Cham, et Japhet.
\VS{11}Et la terre était corrompue devant Dieu, et remplie de violence.
\VS{12}Dieu donc regarda la terre, et voici elle était corrompue ; car toute chair avait corrompu sa voie sur la terre.
\VS{13}Et Dieu dit à Noé : La fin de toute chair est venue devant moi ; car ils ont rempli la terre de violence, et voici, je les détruirai avec la terre.
\VS{14}Fais-toi une arche\FTNT{L’arche  est un type de Christ et du salut en lui et par lui.  On peut voir plusieurs aspects de Jésus-Christ et de la rédemption dans la structure de l’arche :
- L’arche a été une révélation de Jésus-Christ donnée à Noé.  C’est en Jésus-Christ que nous avons le salut et la protection (Col. 1:12-13 ; Col. 3:3). 
-L’arche était faite de bois de gopher, probablement du cèdre.  Ce bois est un bois qui ne pourrit pas en condition normale. Ce bois préfigurait l’incorruptibilité de Jésus-Christ homme (Es. 53:9 ;  Hé. 4:15 ; 1 Pi. 2:22). 
-Au verset 14, on lit que Dieu demande à Noé d’enduire l’arche en dedans et en dehors avec de la  poix, c’est-à-dire du bitume.  Le mot «~poix~» vient de l’hébreu «~kaphar~», qui signifie «~expiation~».  Ce mot est traduit près de 70 fois dans le Tanakh par expiation.  Il est également traduit par «~réconciliation~», «~pardon~», «~miséricordieux~» et «~apaiser~».  L’allusion à l’expiation des péchés faite par Jésus-Christ est claire. Par son sacrifice, nous sommes rendus parfaits à jamais (Hé. 10:14-15).} de bois de gopher ; tu feras cette arche en cellules, et tu l’enduiras de poix en dedans et en dehors.
\VS{15}Et voici comment tu la feras : La longueur de l'arche sera de trois cents coudées ; sa largeur de cinquante coudées, et sa hauteur de trente coudées.
\VS{16}Tu feras une fenêtre à l'arche, et feras son comble d'une coudée de hauteur, et tu mettras la porte de l'arche à son côté, et tu la feras avec un bas, un second, et un troisième étage.
\VS{17}Et voici, je ferai venir un déluge d'eau sur la terre, pour détruire toute chair dans laquelle il y a souffle de vie sous les cieux ; et tout ce qui est sur la terre expirera.
\VS{18}Mais j'établirai mon alliance avec toi ; et tu entreras dans l'arche toi et tes fils, et ta femme, et les femmes de tes fils avec toi.
\VS{19}Et de tout ce qui a vie d'entre toute chair, tu en feras entrer deux de chaque espèce dans l'arche, pour les conserver en vie avec toi, à savoir le mâle et la femelle.
\VS{20}Des oiseaux, selon leur espèce, des bêtes à quatre pattes, selon leur espèce, et de tous les reptiles, selon leur espèce. Ils y entreront tous par paires avec toi, afin que tu les conserves en vie.
\VS{21}Prends aussi avec toi de tous les aliments que l’on mange, et rassemble-les auprès de toi, afin qu'ils servent pour ta nourriture et pour celle des animaux.
\VS{22}Et Noé fit selon tout ce que Dieu lui avait ordonné ; il le fit ainsi.
\Chap{7}
\TextTitle{Le jugement par le déluge}
\VerseOne{}Yahweh dit à Noé : Entre dans l’arche, toi et toute ta maison ; car je t'ai vu juste devant moi parmi cette génération. 
\VS{2} Tu prendras de toutes les bêtes pures sept de chaque espèce, le mâle et sa femelle ; mais des bêtes qui ne sont point pures, un couple, le mâle et la femelle.
\VS{3}Tu prendras aussi des oiseaux du ciel sept de chaque espèce, le mâle et sa femelle ; afin d'en conserver la race sur toute la terre.
\VS{4}Car dans sept jours, je ferai pleuvoir sur la terre pendant quarante jours et quarante nuits ; et j'exterminerai de la surface de la terre tous les êtres qui subsistent que j'ai faits.
\VS{5}Noé fit selon tout ce que Yahweh lui avait ordonné.
\VS{6}Noé était âgé de six cents ans quand le déluge des eaux vint sur la terre.
\VS{7}Noé donc entra dans l’arche avec ses fils, sa femme, et les femmes de ses fils, pour échapper aux eaux du déluge.
\VS{8}Des bêtes pures, des bêtes qui ne sont point pures, des oiseaux, et tout ce qui se meut sur la terre.
\VS{9}Elles entrèrent deux à deux vers Noé dans l'arche, le mâle et la femelle, comme Dieu l’avait ordonné à Noé.
\VS{10}Sept jours après, les eaux du déluge furent sur la terre.
\VS{11}En l'an six cent de la vie de Noé, au second mois, le dix-septième jour du mois, en ce jour-là toutes les sources du grand abîme furent rompues, et les écluses des cieux furent ouvertes.
\VS{12}La pluie tomba sur la terre pendant quarante jours et quarante nuits.
\VS{13}Ce même jour entrèrent dans l’arche Noé, Sem, Cham, et Japhet, fils de Noé, avec la femme de Noé, et les trois femmes de ses fils avec eux.
\VS{14}Eux, et tous les animaux selon leur espèce, et tout le bétail selon son espèce, et tous les reptiles qui se meuvent sur la terre selon leur espèce, et tous les oiseaux selon leur espèce ; et tout petit oiseau ayant des ailes, de quelque sorte que ce soit.
\VS{15}Ils entrèrent dans l'arche auprès de Noé, deux à deux, de toute chair ayant souffle de vie.
\VS{16}Il en entra mâle et femelle de toute chair comme Dieu l’avait ordonné à Noé, puis Yahweh ferma l'arche sur lui.
\VS{17}Le déluge fut pendant quarante jours sur la terre ; et les eaux crurent et élevèrent l'arche, et elle fut élevée au-dessus de la terre.
\VS{18}Les eaux  grossirent et s'accrurent beaucoup sur la terre, et l'arche flottait au-dessus des eaux.
\VS{19}Les eaux grossirent de plus en plus sur la terre, et toutes les hautes montagnes qui sont sous le ciel entier en furent couvertes.
\VS{20}Les eaux s’élevèrent de quinze coudées au-dessus des montagnes  qui furent couvertes.
\VS{21}Toute chair qui se mouvait sur la terre périt, tant les oiseaux que le bétail et les animaux, tous les reptiles qui rampaient sur la terre, et tous les hommes.
\VS{22}Tout ce qui avait respiration, souffle de vie dans ses narines, et qui était sur la terre sèche mourut.
\VS{23}Tous les êtres qui étaient sur la face de la terre furent donc exterminés, depuis les hommes jusqu’au bétail, aux reptiles et aux oiseaux du ciel ; ils furent exterminés de la face de la terre ; il ne resta seulement que Noé, et ce qui était avec lui dans l'arche.
\VS{24}Les eaux furent grosses sur la terre pendant cent cinquante jours.
\Chap{8}
\TextTitle{Fin du déluge}
\VerseOne{}Dieu se souvint de Noé, de tous les animaux et de tout le bétail qui étaient avec lui dans l'arche ; et Dieu fit passer un vent sur la terre, et les eaux s’apaisèrent.
\VS{2}Les sources de l'abîme et les écluses des cieux furent fermées et la pluie ne tomba plus du ciel.
\VS{3}Au bout de cent cinquante jours, les eaux se retirèrent sans interruption de dessus la terre, et diminuèrent.
\VS{4}Le dix-septième jour du septième mois, l'arche s'arrêta sur les montagnes d'Ararat.
\VS{5}Les eaux allèrent en diminuant de plus en plus jusqu'au dixième mois ; et au premier jour du dixième mois, les sommets des montagnes apparurent.
\VS{6}Au bout de quarante jours, Noé ouvrit la fenêtre qu’il avait faite à l'arche.
\VS{7}Il lâcha le corbeau, qui sortit, allant et revenant, jusqu'à ce que les eaux aient séché sur la terre.
\VS{8}Il lâcha aussi une colombe pour voir si les eaux avaient diminué à la surface de la terre.
\VS{9}Mais la colombe ne trouvant aucun lieu pour poser la plante de son pied, retourna à lui dans l'arche, car les eaux étaient sur toute la terre ; et Noé avançant sa main la reprit et la fit entrer dans l'arche.
\VS{10}Il attendit encore sept autres jours, il lâcha de nouveau la colombe hors de l'arche.
\VS{11}Sur le soir, la colombe revint à lui ; et voici, elle avait dans son bec une feuille d'olivier qu'elle avait arrachée ; et Noé connut que les eaux avaient diminué sur la terre.
\VS{12}Il attendit encore sept autres jours, puis il lâcha la colombe qui ne retourna plus à lui.
\VS{13}L’an six cent un de l'âge de Noé, le premier jour du premier mois, les eaux avaient diminué sur la terre. Noé ôta la couverture de l'arche, regarda, et voici, la surface de la terre avait séché.
\VS{14}Le vingt-septième jour du second mois la terre fut sèche.
\TextTitle{Noé sort de l'arche: Le règne des hommes\FTNTT{Ge. 8:11-15:32}}
\VS{15}Puis Dieu parla à Noé, en disant :
\VS{16}Sors de l'arche, toi et ta femme, tes fils, et les femmes de tes fils avec toi.
\VS{17}Fais sortir avec toi tous les animaux qui sont avec toi, de toute chair, tant les oiseaux que le bétail, et tous les reptiles qui rampent sur la terre ; qu'ils se répandent sur la terre, et qu'ils soient féconds et multiplient sur la terre.
\VS{18}Noé donc sortit, et avec lui ses fils, sa femme, et les femmes de ses fils.
\VS{19}Tous les animaux, tous les reptiles, tous les oiseaux, tout ce qui se meut sur la terre, selon leurs espèces, sortirent de l'arche.
\VS{20}Noé bâtit un autel à Yahweh, il prit de toutes les bêtes pures, et de tout oiseau pur, et il offrit des holocaustes sur l'autel.
\VS{21}Yahweh respira une odeur agréable, et dit en son cœur : Je ne maudirai plus la terre à cause des hommes, quoique les dispositions du coeur des hommes soient mauvaises dès leur jeunesse ; et je ne frapperai plus tout ce qui est vivant, comme je l’ai fait.
\VS{22}Tant que la terre subsistera, les semailles et les moissons, le froid et la chaleur, l'été et l'hiver, le jour et la nuit ne cesseront point.
\Chap{9}
\TextTitle{Yahweh établit une alliance avec Noé\FTNTT{Ge. 9:16}}
\VerseOne{}Dieu bénit Noé et ses fils, et leur dit : Soyez féconds, multipliez, et remplissez la terre.
\VS{2}Vous serez un sujet de crainte et d’effroi pour tout animal de la terre, pour tout oiseau du ciel, pour tout ce qui se meut sur la terre, et pour tous les poissons de la mer : Ils sont livrés entre vos mains.
\VS{3}Tout ce qui se meut et qui a vie sera votre nourriture ; je vous donne tout cela comme l'herbe verte.
\VS{4}Seulement, vous ne mangerez point de chair avec son âme, c'est-à-dire, son sang.
\VS{5}Sachez-le aussi, je redemanderai votre sang, le sang de vos âmes, je le redemanderai à tout animal ; et je redemanderai l’âme de l’homme de la main de l’homme, de la main de son frère.
\VS{6}Celui qui aura versé le sang de l'homme, par l'homme son sang sera versé ; car Dieu a fait l'homme à son image.
\VS{7}Vous donc, soyez féconds et multipliez, répandez-vous sur la terre et multipliez sur elle.
\VS{8}Dieu parla aussi à Noé et à ses fils qui étaient avec lui, en disant :
\VS{9}Et quant à moi, voici, j'établis mon alliance avec vous, et avec votre postérité après vous ;
\VS{10}avec tous les êtres vivants qui sont avec vous, tant les oiseaux que le bétail, et tous les animaux de la terre qui sont avec vous, tous ceux qui sont sortis de l'arche jusqu'à tous les animaux de la terre.
\VS{11}J'établis donc mon alliance avec vous ; aucune chair ne sera plus exterminée par les eaux du déluge, et il n'y aura plus de déluge pour détruire la terre.
\VS{12}Puis Dieu dit : C'est ici le signe de l'alliance que j’établis entre moi et vous, et tous les êtres vivants qui sont avec vous, pour les générations à toujours :
\VS{13}J’ai placé mon arc dans la nuée, et il servira de signe de l'alliance entre moi et la terre.
\VS{14}Quand j’aurai rassemblé des nuages au-dessus de la terre, l’arc paraîtra dans la nuée ;
\VS{15}et je me souviendrai de mon alliance entre moi et vous, et tous les êtres vivants de toute chair, et les eaux ne deviendront plus un déluge pour détruire toute chair.
\VS{16}L'arc donc sera dans la nuée, et je le regarderai, et je me souviendrai de l'alliance perpétuelle entre Dieu et tous les êtres vivants  de toute chair qui est sur la terre.
\VS{17}Dieu donc dit à Noé : C'est là le signe de l'alliance que j'ai établie entre moi et toute chair qui est sur la terre.
\VS{18}Les fils de Noé qui sortirent de l'arche étaient Sem, Cham, et Japhet. Cham fut père de Canaan.
\VS{19}Ce sont là les trois fils de Noé, et c’est leur postérité qui peupla toute la terre.
\TextTitle{Le péché de Noé}
\VS{20}Or, Noé commença à cultiver la terre, et planta de la vigne.
\VS{21}Il but du vin, s'enivra, et se découvrit au milieu de sa tente.
\VS{22}Cham, père de Canaan, vit la nudité de son père\FTNT{Lé. 18:6-19 ; Lé. 20:11-21.}, et il le rapporta dehors à ses deux frères.
\VS{23}Alors Sem et Japhet prirent un manteau qu'ils mirent sur leurs deux épaules, et marchant à reculons, ils couvrirent la nudité de leur père ; et leurs visages étaient tournés en arrière, de sorte qu'ils ne virent point la nudité de leur père.
\VS{24}Et quand Noé se réveilla de son vin, il apprit ce que lui avait fait son fils cadet.
\TextTitle{Noé prononce une malédiction contre Canaan}
\VS{25}C'est pourquoi il dit : Maudit soit Canaan\FTNT{Une idée erronée selon laquelle les noirs auraient été maudits par Dieu au travers de la malédiction de Canaan s’est répandue pendant des siècles. On a ainsi légitimé la domination des peuples africains par les puissances occidentales blanches, et par la même occasion l’esclavage. Il faut préciser que les descendants de Cham furent Cush (Ethiopie), Mitsraïm (Egypte), Puth (les Celtes) et Canaan (Palestine, pays que Dieu a donné aux descendants de Sem, selon Ge. 15). Cham est le fils cadet, c’est-à-dire le coupable aux yeux de Noé. Mais c’est à Canaan (Palestine), le fils de Cham, donc petit-fils de Noé, que s’adresse la malédiction. Selon la Bible, les peuples africains sont des descendants de Cham, mais par son fils Cush et non par Canaan. La prétendue malédiction des noirs n’a donc aucun fondement.} ; il sera serviteur des serviteurs de ses frères.
\VS{26}Il dit aussi : Béni soit Yahweh, Dieu de Sem ; et que Canaan soit leur serviteur.
\VS{27}Que Dieu étende en douceur Japhet, et qu’il habite dans les tentes de Sem ; et que Canaan soit leur serviteur.
\VS{28}Noé vécut après le déluge trois cent cinquante ans.
\VS{29}Tout le temps donc que Noé vécut fut de neuf cent cinquante ans ; puis il mourut.
\Chap{10}
\TextTitle{La postérité de Noé}
\VerseOne{}Voici la postérité des enfants de Noé, Sem, Cham et Japhet ; il leur naquit des fils après le déluge.
\VS{2}Les fils de Japhet furent : Gomer, Magog, Madaï, Javan, Tubal, Mésech, et Tiras.
\VS{3}Les fils de Gomer : Aschkenaz, Riphat, et Togarma.
\VS{4}Les fils de Javan : Elischa, Tarsis, Kittim, et Dodanim.
\VS{5}C’est par eux qu’ont été peuplées les îles des nations selon leurs terres, chacun selon sa langue, selon leurs familles, entre leurs nations.
\VS{6}Les fils de Cham furent : Cusch, Mitsraïm, Puth, et Canaan.
\VS{7}Les fils de Cusch : Saba, Havila, Sabta, Raema, et Sabteca. Les fils de Raema : Séba et Dedan.
\VS{8}Cusch engendra aussi Nimrod\FTNT{Nimrod ou Nemrod, dont le nom signifie «~rebelle~», fut le premier roi de l’histoire biblique. Fils de Cusch (Ethiopie), lui-même premier-né de Cham, fils de Noé (Ge. 10:8-10), il fut à la tête du premier empire après le déluge. Il se distingua en qualité de puissant chasseur  «~devant Yahweh~» ou «~contre Yahweh~». Le contexte du chapitre 10 laisse entendre que Nimrod était un puissant chasseur qui provoquait Dieu. Fondateur de Ninive, il est surtout connu pour avoir été à l’origine du projet de la tour de Babel.}, c’est lui qui commença à être puissant sur la terre.
\VS{9}Il fut un puissant chasseur devant Yahweh, c'est pourquoi l'on a dit : Comme Nimrod, le puissant chasseur devant Yahweh.
\VS{10}Il régna d’abord sur Babel\FTNT{Le nom Babel signifie confusion par le mélange.}, Erec, Accad, et Calné au pays de Schinear.
\VS{11}De ce pays-là sortit Assur, et il bâtit Ninive et les rues de la ville, Rehoboth-Hir et Calach,
\VS{12}et Résen, entre Ninive et Calach, qui est une grande ville.
\VS{13}Mitsraïm engendra les Ludim, les Anamim, les Lehabim, les Naphtuhim,
\VS{14}les Patrusim, les Casluhim, d’où sont sortis les Philistins, et les Caphtorim.
\VS{15}Canaan engendra Sidon, son premier-né, et Heth ;
\VS{16}et les Jébusiens, les Amoréens, les Guirgasiens,
\VS{17}les Héviens, les Arkiens, les Siniens,
\VS{18}les Arvadiens, les Tsemariens, les Hamathiens. Ensuite, les familles des Cananéens se sont dispersées.
\VS{19}Les limites des Cananéens furent depuis Sidon, quand on vient vers Guérar, jusqu'à Gaza, en allant vers Sodome et Gomorrhe, Adma, et Tseboïm, jusqu'à Léscha.
\VS{20}Ce sont là les fils de Cham selon leurs familles et leurs langues, selon leurs pays, et selon leurs nations.
\VS{21}Il naquit aussi des fils à Sem, père de tous les fils d'Héber, et frère aîné de Japhet.
\VS{22}Les fils de Sem furent : Elam, Assur, Arpacschad, Lud et Aram.
\VS{23}Les fils d'Aram : Uts, Hul, Guéter et Masch.
\VS{24}Arpacschad engendra Schélach ; et Schélach engendra Héber.
\VS{25}Il naquit à Héber deux fils : Le nom de l'un était Péleg, parce que de son temps la terre fut partagée ; et le nom de son frère était Jokthan.
\VS{26}Jokthan engendra Almodad, Schéleph, Hatsarmaveth, Jérach,
\VS{27}Hadoram, Uzal, Dikla,
\VS{28}Obal, Abimaël, Séba,
\VS{29}Ophir, Havila, et Jobab. Tous ceux-là sont les enfants de Jokthan.
\VS{30}Ils habitèrent depuis Méscha, du côté de Sephar jusqu’à la montagne de l’orient.
\VS{31}Ce sont là les fils de Sem, selon leurs familles, selon leurs langues, selon leurs pays, et selon leurs nations.
\VS{32}Telles sont les familles des fils de Noé, selon leurs lignées, selon leurs nations. Et c’est d’eux que sont sorties les nations qui se sont répandues sur la terre après le déluge.
\Chap{11}
\TextTitle{Un projet humain : La tour de Babel}
\VerseOne{}Alors toute la terre avait un même langage et les mêmes paroles.
\VS{2}Mais il arriva qu'étant partis d'orient, ils trouvèrent une vallée au pays de Schinear où ils habitèrent.
\VS{3}Et ils se dirent l'un à l'autre : Allons ! Faisons des briques, et cuisons-les très bien au feu. Et la brique leur servit de pierre, et le bitume leur servit d’argile.
\VS{4}Puis ils dirent : Allons ! Bâtissons-nous une ville, et une tour dont le sommet soit jusqu’aux cieux ; et faisons-nous un nom, de peur que nous ne soyons dispersés sur toute la terre.
\VS{5}Alors Yahweh descendit pour voir la ville et la tour que les fils des hommes bâtissaient.
\VS{6}Et Yahweh dit : Voici, ce n'est qu'un seul et même peuple, ils ont un même langage, et ils commencent à travailler ; et maintenant rien ne les empêchera d'exécuter ce qu'ils ont projeté.
\TextTitle{Yahweh confond le langage humain}
\VS{7}Allons ! Descendons, et là confondons leur langage afin qu'ils n'entendent point le langage les uns des autres.
\VS{8}Ainsi Yahweh les dispersa de là par toute la terre, et ils cessèrent de bâtir la ville.
\VS{9}C'est pourquoi on l’appela du nom de Babel, car c’est là que Yahweh confondit le langage de toute la terre, et c’est de là que Yahweh les dispersa sur toute la terre.
\TextTitle{La postérité de Sem, ancêtre d’Abram}
\VS{10}Voici la postérité de Sem : Sem, âgé de cent ans, engendra Arpacschad, deux ans après le déluge.
\VS{11}Sem, après qu'il eut engendré Arpacschad, vécut cinq cents ans, et engendra des fils et des filles.
\VS{12}Arpacschad vécut trente-cinq ans, et engendra Schélach.
\VS{13}Arpacschad, après qu'il eut engendré Schélach, vécut quatre cent trois ans, et engendra des fils et des filles.
\VS{14}Schélach, ayant vécu trente ans, engendra Héber.
\VS{15}Schélach, après qu'il eut engendré Héber, vécut quatre cent trois ans, et engendra des fils et des filles.
\VS{16}Héber, ayant vécu trente-quatre ans, engendra Péleg.
\VS{17}Héber, après qu'il eut engendré Péleg, vécut quatre cent trente ans, et engendra des fils et des filles.
\VS{18}Péleg, ayant vécu trente ans, engendra Rehu.
\VS{19}Péleg, après qu'il eut engendré Rehu, vécut deux cent neuf ans, et engendra des fils et des filles.
\VS{20}Rehu, ayant vécu trente-deux ans, engendra Serug.
\VS{21}Rehu, après qu'il eut engendré Serug, vécut deux cent sept ans, et engendra des fils et des filles.
\VS{22}Serug, ayant vécu trente ans, engendra Nachor.
\VS{23}Serug, après qu'il eut engendré Nachor, vécut deux cents ans, et engendra des fils et des filles.
\VS{24}Nachor, ayant vécu vingt-neuf ans, engendra Térach.
\VS{25}Nachor, après qu'il eut engendré Térach, vécut cent dix-neuf ans, et engendra des fils et des filles.
\VS{26}Térach, ayant vécu soixante-dix ans, engendra Abram, Nachor, et Haran.
\VS{27}Voici la postérité de Térach : Térach engendra Abram, Nachor, et Haran ; et Haran engendra Lot.
\VS{28}Et Haran mourut en présence de son père, au pays de sa naissance, à Ur en Chaldée.
\VS{29}Abram et Nachor prirent chacun une femme. Le nom de la femme d'Abram était Saraï ; et le nom de la femme de Nachor était Milca, fille de Haran, père de Milca et de Jisca.
\VS{30}Saraï était stérile, et n'avait point d'enfants.
\TextTitle{Séjour à Charan}
\VS{31}Térach prit son fils Abram, et Lot fils de son fils, qui était fils de Haran, et Saraï, sa belle-fille, femme d'Abram, son fils, et ils sortirent ensemble d'Ur en Chaldée pour aller au pays de Canaan, et ils vinrent jusqu'à Charan, et ils y habitèrent.
\VS{32}Les jours de Térach furent de deux cent cinq ans ; puis il mourut à Charan.
\Chap{12}
\TextTitle{Appel d'Abram : La promesse de Yahweh\FTNTT{Ge. 12:2 ; 13:14-18 ; 15:1-21 ; 17:4-8 ; 22:15-24 ; 26:1-5 ; 28:10-15}}
\VerseOne{}Yahweh dit à Abram : Va pour toi, hors de ta terre, de ta patrie, et de la maison de ton père, vers la terre que je te montrerai\FTNT{Ac. 7:3 ; Hé. 11:8.}.
\VS{2}Je te ferai devenir une grande nation, et je te bénirai, je rendrai ton nom grand, et tu seras béni.
\VS{3}Je bénirai ceux qui te béniront, et je maudirai ceux qui te maudiront ; et toutes les familles de la terre seront bénies en toi\FTNT{Ac. 3:25 ; Ga. 3:8.}.
\TextTitle{Abram sur la terre de Canaan}
\VS{4}Abram donc partit, comme Yahweh le lui avait dit, et Lot alla avec lui. Abram était âgé de soixante-quinze ans quand il sortit de Charan.
\VS{5}Abram prit aussi Saraï, sa femme, et Lot, fils de son frère, avec tous les biens qu'ils avaient acquis, et les personnes qu'ils avaient eues à Charan ; et ils partirent pour aller dans le pays de Canaan, et ils arrivèrent au pays de Canaan\FTNT{Ac. 7:4.}.
\VS{6}Abram parcourut le pays jusqu'au lieu nommé Sichem, et jusqu'aux chênes de Moré ; et les Cananéens étaient alors dans le pays.
\VS{7}Yahweh apparut à Abram, et lui dit : Je donnerai ce pays à ta postérité. Et Abram bâtit là un autel à Yahweh qui lui était apparu.
\VS{8}Il se transporta de là vers la montagne, à l'orient de Béthel, et il dressa ses tentes, ayant Béthel à l'occident, et Aï à l'orient ; et il bâtit là un autel à Yahweh, et invoqua le nom de Yahweh.
\VS{9}Puis Abram partit de là, marchant et s'avançant vers le midi.
\TextTitle{Abram en Egypte}
\VS{10}Mais la famine étant survenue dans le pays, Abram descendit en Egypte pour s'y retirer, car la famine était grande dans le pays.
\VS{11}Comme il était près d'entrer en Egypte, il dit à Saraï, sa femme : Voici, je sais que tu es une fort belle femme ;
\VS{12}c'est pourquoi, quand les Egyptiens te verront, ils diront : C'est la femme de cet homme, et ils me tueront, mais ils te laisseront vivre.
\VS{13}Dis donc, je te prie, que tu es ma sœur, afin que je sois bien traité à cause de toi, et que par ton moyen, ma vie soit préservée.
\VS{14}Il arriva donc qu'aussitôt qu'Abram fut arrivé en Egypte, les Egyptiens virent que cette femme était fort belle.
\VS{15}Les principaux de la cour de Pharaon la virent aussi et la vantèrent à Pharaon, et elle fut enlevée pour être menée dans la maison de Pharaon.
\VS{16}Il traita bien Abram à cause d'elle, de sorte qu'il en eut des brebis, des bœufs, des ânes, des serviteurs, des servantes, des ânesses, et des chameaux.
\VS{17}Mais Yahweh frappa de grandes plaies Pharaon et sa maison, à cause de Saraï, femme d'Abram.
\VS{18}Alors Pharaon appela Abram, et lui dit : Qu'est-ce que tu m'as fait ? Pourquoi ne m'as-tu pas déclaré que c'était ta femme ?
\VS{19}Pourquoi as-tu dit : C'est ma sœur ? Car je l'avais prise pour ma femme ; mais maintenant, voici ta femme, prends-la, et va-t'en.
\VS{20}Et Pharaon ayant donné ordre à ses gens, ils le renvoyèrent, lui, sa femme, et tout ce qui était à lui.
\Chap{13}
\TextTitle{Retour d'Abram à Canaan}
\VerseOne{}Abram donc monta d'Egypte vers le midi, lui, sa femme, et tout ce qui lui appartenait, et Lot avec lui.
\VS{2}Et Abram était très riche en bétail, en argent, et en or.
\VS{3}Et il s'en retourna en suivant la route qu'il avait suivie du midi à Béthel, jusqu'au lieu où il avait dressé ses tentes au commencement, entre Béthel et Aï,
\VS{4}au même lieu où était l'autel qu'il y avait bâti au commencement, et Abram invoqua là le nom de Yahweh.
\TextTitle{Abram se sépare de Lot\FTNTT{Ge. 13:12}}
\VS{5}Lot aussi, qui marchait avec Abram, avait des brebis, des boeufs, et des tentes.
\VS{6}Et le pays ne pouvait les porter pour demeurer ensemble ; car leurs biens étaient si grand qu'ils ne pouvaient demeurer ensemble.
\VS{7}Il y eut querelle entre les bergers du bétail d'Abram et les bergers du bétail de Lot ; or en ce temps-là, les Cananéens et les Phérésiens habitaient dans le pays.
\VS{8}Et Abram dit à Lot : Je te prie qu'il n'y ait point de dispute entre moi et toi, ni entre mes bergers et les tiens, car nous sommes frères.
\VS{9}Tout le pays n'est-il pas devant toi ? Sépare-toi je te prie d'avec moi. Si tu vas à gauche, j’irai à droite ; et si tu vas à droite, j’irai à gauche.
\TextTitle{Lot s'établit à Sodome\FTNTT{Ge. 13:10}}
\VS{10}Lot, levant les yeux, vit que toute la plaine du Jourdain était entièrement arrosée. Avant que Yahweh ait détruit Sodome et Gomorrhe, c’était, jusqu'à Tsoar, comme le jardin de Yahweh, et comme le pays d'Egypte.
\VS{11}Lot choisit pour lui toute la plaine du Jourdain, et alla du côté de l’orient ; ainsi ils se séparèrent l'un de l'autre.
\VS{12}Abram habita dans le pays de Canaan, et Lot habita dans les villes de la plaine, et dressa ses tentes jusqu'à Sodome.
\VS{13}Les habitants de Sodome étaient méchants et de grands pécheurs contre Yahweh.
\TextTitle{Yahweh confirme son alliance avec Abram}
\VS{14}Yahweh dit à Abram, après que Lot se fut séparé de lui : Lève maintenant tes yeux, et regarde du lieu où tu es vers le nord, le midi, l'orient, et l'occident.
\VS{15}Car je te donnerai, à toi et à ta postérité pour toujours, tout le pays que tu vois.
\VS{16}Je rendrai ta postérité comme la poussière de la terre ; en sorte que si quelqu'un peut compter la poussière de la terre, il comptera aussi ta postérité\FTNT{Ro. 4:18 ; Hé. 11:12.}.
\VS{17}Lève-toi donc et promène-toi dans le pays, dans sa longueur et dans sa largeur, car je te le donnerai.
\VS{18}Abram ayant transporté ses tentes, alla habiter dans les plaines de Mamré, qui sont près d’Hébron et là, il bâtit un autel à Yahweh.
\Chap{14}
\TextTitle{Abram va au secours de Lot}
\VerseOne{}Dans le temps d'Amraphel, roi de Schinear, d'Arjoc, roi d'Ellasar, de Kedorlaomer, roi d'Elam, et de Tideal, roi de Gojim,
\VS{2}il arriva qu’ils firent la guerre contre Béra, roi de Sodome, et contre Birscha, roi de Gomorrhe, et contre Schineab, roi d'Adma, et contre Schémeéber, roi de Tseboïm, et contre le roi de Béla, qui est Tsoar.
\VS{3}Tous ceux-ci se joignirent dans la vallée de Siddim, qui est la mer salée.
\VS{4}Ils avaient été asservis douze années à Kedorlaomer, et la treizième année, ils s'étaient révoltés.
\VS{5}A la quatorzième année, Kedorlaomer et les rois qui étaient avec lui vinrent et ils battirent les Rephaïm à Aschteroth-Karnaïm, les Zuzim à Ham, et les Emin à la plaine de Schavé-Kirjathaïm,
\VS{6}et les Horiens dans leur montagne de Séir, jusqu'au chêne de Paran, qui est près du désert.
\VS{7}Puis ils s’en retournèrent et vinrent à En-Mischpath, qui est Kadès ; et ils frappèrent tout le pays des Amalécites et des Amoréens qui habitaient dans Hatsatson-Thamar.
\VS{8}Alors le roi de Sodome, le roi de Gomorrhe, le roi d'Adma, le roi de Tseboïm, et le roi de Béla qui est Tsoar, sortirent et rangèrent leurs troupes contre eux dans la vallée de Siddim.
\VS{9}C'est-à-dire contre Kedorlaomer, roi d'Elam, et contre Tideal, roi de Gojim, et contre Amraphel, roi de Schinear, et contre Arjoc, roi d'Ellasar : Quatre rois contre cinq.
\VS{10}La vallée de Siddim était pleine de puits de bitume ; les rois de Sodome et de Gomorrhe s'enfuirent et y tombèrent, et le reste s'enfuit dans la montagne.
\VS{11}Ils prirent donc toutes les richesses de Sodome et de Gomorrhe, et tous leurs vivres ; puis ils se retirèrent.
\VS{12}Ils prirent aussi Lot, fils du frère d'Abram, qui habitait dans Sodome, et tous ses biens ; puis ils s'en allèrent.
\VS{13}Un fuyard vint avertir Abram, l’Hébreu, qui demeurait dans les plaines de Mamré, l’Amoréen, frère d'Eschcol, et frère d’Aner, qui avaient fait alliance avec Abram.
\VS{14}Dès qu’Abram eut appris que son frère avait été emmené prisonnier, il arma trois cent dix-huit de ses plus braves serviteurs, nés dans sa maison, et il poursuivit ces rois jusqu'à Dan.
\VS{15}Il divisa sa troupe, il se jeta sur eux de nuit, lui et ses serviteurs ; il les battit et les poursuivit jusqu'à Choba, qui est à la gauche de Damas.
\VS{16}Il ramena tous les biens qu'ils avaient pris ; il ramena aussi Lot, son frère, ses biens, les femmes et le peuple.
\TextTitle{Melchisédek, sacrificateur d'El Elyon (Dieu Très-Haut)}
\VS{17}Le roi de Sodome sortit à la rencontre d’Abram qui revenait vainqueur de Kedorlaomer, et des rois qui étaient avec lui, dans la vallée de la plaine, qui est la vallée royale.
\VS{18}Melchisédek\FTNT{Melchisédek est un type de Christ (Ps. 110:4 ; Hé. 5:5-6 ; Hé. 6:20 ; Hé. 7:1-2). Ce personnage nous montre l’aspect de Christ en tant que roi de Salem, ce qui signifie «~paix~», et Souverain Sacrificateur possédant un sacerdoce non transmissible (Hé. 7:24).}, roi de Salem, fit apporter du pain et du vin, or il était Sacrificateur du Dieu Très-Haut.
\VS{19}Il bénit Abram en disant : Béni soit Abram par le Dieu Très-Haut, Maître du ciel et de la terre.
\VS{20}Béni soit le Dieu Très-Haut qui a livré tes ennemis entre tes mains. Et Abram lui donna la dîme\FTNT{Voir commentaire sur la dîme en No. 18:21 et Mal. 3:10.} de tout.
\VS{21}Le roi de Sodome dit à Abram : Donne-moi les personnes, et prends pour toi les richesses.
\VS{22}Abram répondit au roi de Sodome : Je lève ma main vers Yahweh, le Dieu Très-Haut, Maître du ciel et de la terre :
\VS{23}Je ne prendrai rien de tout ce qui est à toi, pas même un fil, ni un cordon de soulier, afin que tu ne dises point : J'ai enrichi Abram.
\VS{24}Seulement, ce que les jeunes gens ont mangé, et la part des hommes qui sont venus avec moi, Aner, Eschcol, et Mamré, qui prendront leur part.
\Chap{15}
\TextTitle{Yahweh promet un enfant à Abram}
\VerseOne{}Après ces choses, la parole de Yahweh fut adressée à Abram dans une vision, en disant : Abram, ne crains point, je suis ton bouclier, et ta récompense sera très grande.
\VS{2}Abram répondit : Seigneur Yahweh, que me donneras-tu ? Je m'en vais sans laisser d'enfants après moi, et l’héritier de ma maison c'est Eliézer de Damas.
\VS{3}Abram dit aussi : Voici, tu ne m'as point donné d'enfants ; et voilà, le serviteur né dans ma maison sera mon héritier.
\VS{4}Alors la parole de Yahweh lui fut adressée ainsi : Ce n’est pas lui qui sera ton héritier, mais c’est celui qui sortira de tes entrailles qui sera ton héritier.
\VS{5}Puis l'ayant fait sortir dehors, il lui dit : Lève maintenant les yeux au ciel et compte les étoiles si tu peux les compter. Et il lui dit : Ainsi sera ta postérité.
\VS{6}Abram crut à Yahweh qui lui imputa cela à justice\FTNT{Ga. 3:6 ; Ja. 2:23 ; Ro. 4:3.}.
\TextTitle{Yahweh annonce l'esclavage de la postérité d'Abram}
\VS{7}Et il lui dit : Je suis Yahweh qui t'ai fait sortir d'Ur en Chaldée, afin de te donner ce pays-ci pour le posséder.
\VS{8}Abram répondit : Seigneur Yahweh, à quoi connaîtrai-je que je le posséderai ?
\VS{9}Et Yahweh lui répondit : Prends une génisse de trois ans,  une chèvre de trois ans, un bélier de trois ans, une tourterelle, et un pigeon.
\VS{10}Abram prit tous ces animaux, les coupa par le milieu, et mit chaque morceau l’un vis-à-vis de l’autre, mais il ne partagea point les oiseaux.
\VS{11}Les oiseaux de proie descendirent sur les cadavres, mais Abram les chassa.
\VS{12}Au coucher du soleil, un profond sommeil tomba sur Abram, et voici, une frayeur d'une grande obscurité tomba sur lui.
\VS{13}Et Yahweh dit à Abram : Sache comme une chose certaine que tes descendants habiteront quatre cents ans comme étrangers dans un pays qui ne leur appartiendra point, et qu’ils seront asservis aux habitants du pays qui les opprimera\FTNT{Ac. 7:6 ; Ga. 3:17.}.
\VS{14}Mais je jugerai la nation à laquelle ils seront asservis, et après cela ils sortiront avec de grands biens\FTNT{Ex. 3:22.}.
\VS{15}Et toi tu iras vers tes pères en paix, et tu seras enterré après une heureuse vieillesse.
\VS{16}A la quatrième génération, ils reviendront ici ; car l'iniquité des Amoréens n'est pas encore à son comble.
\VS{17}Quand le soleil fut couché, il y eut une obscurité profonde, et voici, ce fut une fournaise fumante, et des flammes passèrent entre les animaux qui avaient été partagés.
\VS{18}En ce jour-là, Yahweh traita alliance avec Abram, en disant : Je donne ce pays à ta postérité, depuis le fleuve d'Egypte jusqu'au grand fleuve, le fleuve d'Euphrate ;
\VS{19}le pays des Kéniens, des Keniziens, des Kadmoniens,
\VS{20}des Héthiens, des Phéréziens, des Rephaïm,
\VS{21}des Amoréens, des Cananéens, des Guirgasiens, et des Jébusiens.
\Chap{16}
\TextTitle{Saraï pousse Abram dans les bras de sa servante}
\VerseOne{}Saraï, femme d'Abram, ne lui avait enfanté aucun enfant, mais elle avait une servante égyptienne nommée Agar.
\VS{2}Et Saraï dit à Abram : Voici, Yahweh m'a rendue stérile ; viens je te prie vers ma servante, peut-être aurai-je des enfants par elle. Et Abram écouta la voix de Saraï.
\VS{3}Alors Saraï, femme d'Abram, prit Agar, sa servante égyptienne, et la donna pour femme à Abram, son mari, après qu’Abram eut habité dix ans dans le pays de Canaan.
\VS{4}Il alla donc vers Agar, et elle conçut. Quand Agar se vit enceinte, elle regarda sa maîtresse avec mépris.
\VS{5}Et Saraï dit à Abram : L'outrage qui m'est fait retombe sur toi. J’ai mis ma servante dans ton sein, mais quand elle a vu qu'elle avait conçu, elle m'a regardée avec mépris. Que Yahweh soit juge entre moi et toi !
\VS{6}Alors Abram répondit à Saraï : Voici, ta servante est entre tes mains, traite-la comme il te plaira. Saraï donc la maltraita, et Agar s'enfuit de devant elle.
\VS{7}Mais l'Ange de Yahweh la trouva auprès d'une fontaine d'eau dans le désert, près de la fontaine qui est sur le chemin de Schur.
\VS{8}Il lui dit : Agar, servante de Saraï, d'où viens-tu ? Et où vas-tu ? Et elle répondit : Je m'enfuis de devant Saraï, ma maîtresse.
\VS{9}L'Ange de Yahweh lui dit : Retourne vers ta maîtresse et humilie-toi sous sa main.
\VS{10}L'Ange de Yahweh lui dit : Je multiplierai beaucoup ta postérité, elle sera si nombreuse qu'on ne pourra la compter.
\VS{11}L'Ange de Yahweh lui dit aussi : Voici, tu as conçu, et tu enfanteras un fils que tu appelleras Ismaël, car Yahweh a entendu ton affliction.
\VS{12}Et ce sera un homme farouche comme un âne sauvage ; sa main sera contre tous, et la main de tous contre lui ; et il habitera en face de tous ses frères.
\VS{13}Alors elle appela Atta-El-roï (tu es le Dieu qui me voit) le nom de Yahweh qui lui avait parlé ; car elle dit : N'ai-je pas même, ici, vu celui qui me voyait ?
\VS{14}C'est pourquoi on a appelé ce puits le puits du vivant qui me voit ; lequel est entre Kadès et Bared.
\TextTitle{Naissance d'Ismaël}
\VS{15}Agar donc enfanta un fils à Abram ; et Abram donna le nom d’Ismaël  au fils qu'Agar lui avait enfanté\FTNT{Ga. 4:22.}.
\VS{16}Abram était âgé de quatre-vingt-six ans quand Agar enfanta Ismaël à Abram.
\Chap{17}
\TextTitle{El Schaddaï (Dieu Tout-Puissant) confirme sa promesse}
\VerseOne{}Lorsqu’Abram fut âgé de quatre-vingt-dix-neuf ans, Yahweh lui apparut et lui dit : Je suis le Dieu Tout-Puissant\FTNT{Dieu se révèle ici à Abraham comme le Dieu Tout-Puissant. Or Christ s’est présenté à l’apôtre Jean comme le Dieu Tout Puissant (Ap. 1:8).  Plus loin en Ap. 5:6, le Seigneur apparaît au milieu du trône céleste sous la forme d’un Agneau ayant sept cornes qui représentent sa toute-puissance. Jésus est bien le Dieu Tout-Puissant qui s’était révélé à Abraham (Da. 8:20-22).}. Marche devant ma face, et sois intègre.
\VS{2}J’établirai mon alliance entre moi et toi, et je te multiplierai très abondamment.
\VS{3}Alors Abram tomba sur sa face, et Dieu lui parla et lui dit :
\TextTitle{Abram devient Abraham}
\VS{4}Quant à moi, voici, mon alliance est avec toi, et tu deviendras père d'une multitude de nations\FTNT{Ro. 4:17.}.
\VS{5}On ne t’appellera plus Abram\FTNT{Né. 9:7.}, mais ton nom sera Abraham ; car je t'ai établi père d'une multitude de nations.
\TextTitle{Promesse d’une alliance éternelle}
\VS{6}Je te rendrai fécond  à l’extrême, et je te ferai devenir des nations ; même des rois sortiront de toi\FTNT{Mt. 1:6.}.
\VS{7}J'établirai donc mon alliance entre moi et toi, et entre ta postérité après toi, selon leurs générations, ce sera une alliance éternelle en vertu de laquelle je serai ton Dieu et celui de ta postérité après toi.
\VS{8}Je te donnerai, et à ta postérité après toi, le pays où tu demeures comme étranger, à savoir tout le pays de Canaan, en possession perpétuelle, et je serai leur Dieu.
\TextTitle{La circoncision, signe de l'alliance}
\VS{9}Dieu dit encore à Abraham : Tu garderas donc mon alliance, toi et ta postérité après toi, selon leurs générations.
\VS{10}C’est ici mon alliance entre moi et vous, et entre ta postérité après toi, que vous garderez : Tout mâle parmi vous sera circoncis.
\VS{11}Vous circoncirez la chair de votre prépuce ; et cela sera le signe de l'alliance entre moi et vous\FTNT{Ac. 7:8 ; Ro. 4:11.}.
\VS{12}Tout enfant mâle de huit jours sera circoncis parmi vous dans vos générations, tant celui qui est né dans la maison que l'esclave acquis à prix d’argent de tout étranger qui n'est point de ta race\FTNT{Lu. 2:21 ; Lé. 12:3.}.
\VS{13}On ne manquera donc point de circoncire celui qui est né dans ta maison, et celui qui est acquis à prix d’argent, et mon alliance sera dans votre chair pour être une alliance perpétuelle.
\VS{14}Et le mâle incirconcis qui n’aura pas été circoncis dans sa  chair sera retranché du milieu de son peuple parce qu'il aura violé mon alliance.
\TextTitle{Saraï devient Sara ; promesse de la naissance d’Isaac}
\VS{15}Dieu dit aussi à Abraham : Quant à Saraï, ta femme, tu n'appelleras plus son nom Saraï, mais son nom sera Sara.
\VS{16}Je la bénirai, et même je te donnerai un fils d'elle. Je la bénirai et elle deviendra des nations ; des rois, chefs de peuples sortiront d'elle.
\VS{17}Alors Abraham se prosterna la face contre terre, et sourit en disant en son cœur : Naîtrait-il un fils à un homme âgé de cent ans ? Et Sara, âgée de quatre-vingt-dix ans, aurait-elle un enfant ?
\VS{18}Et Abraham dit à Dieu : Je te prie, qu'Ismaël vive devant toi.
\VS{19}Et Dieu dit : Certainement Sara, ta femme, t'enfantera un fils, et tu appelleras son nom Isaac ; et j'établirai mon alliance avec lui pour être une alliance perpétuelle pour sa postérité après lui.
\TextTitle{Une nation sortira d'Ismaël}
\VS{20}Je t'ai aussi exaucé touchant Ismaël : Voici, je le bénirai, et je le ferai croître et multiplier très abondamment. Il engendrera douze princes, et je le ferai devenir une grande nation.
\VS{21}Mais j'établirai mon alliance avec Isaac, que Sara t'enfantera l'année qui vient, en cette même saison.
\VS{22}Et Dieu ayant achevé de parler, s’éleva au-dessus d'Abraham.
\VS{23}Et Abraham prit son fils Ismaël, avec tous ceux qui étaient nés dans sa maison, et tous ceux qu'il avait acquis à prix d’argent, tous les mâles qui étaient des gens de sa maison, et il circoncit la chair de leur prépuce en ce même jour-là, comme Dieu le lui avait dit.
\VS{24}Abraham était âgé de quatre-vingt-dix-neuf ans quand il circoncit la chair de son prépuce ;
\VS{25}et Ismaël, son fils, était âgé de treize ans lorsqu'il fut circoncis.
\VS{26}En ce même jour, Abraham fut circoncis, et son fils Ismaël aussi.
\VS{27}Et tous les gens de sa maison, tant ceux qui étaient nés dans sa maison que ceux qui avaient été acquis à prix d’argent des étrangers, furent circoncis avec lui.
\Chap{18}
\TextTitle{Abraham, ami de Yahweh\FTNTT{Jn 3:29 ; 15:13-15}}
\VerseOne{}Puis Yahweh lui apparut dans les plaines de Mamré, comme il était assis à la porte de sa tente, pendant la chaleur du jour.
\VS{2}Levant ses yeux, il regarda : Et voici, trois hommes parurent devant lui. Quand il les vit, il courut au-devant d'eux depuis la porte de sa tente, et se prosterna à terre\FTNT{Hé. 13:2.} ;
\VS{3}Et il dit : Mon Seigneur, je te prie, si j'ai trouvé grâce devant tes yeux, ne passe point outre, je te prie, et arrête-toi chez ton serviteur.
\VS{4}Qu'on prenne, je vous prie, un peu d'eau, et lavez vos pieds, et reposez-vous sous un arbre.
\VS{5}J’apporterai un morceau de pain pour fortifier votre cœur, après quoi vous passerez outre ; car c'est pour cela que vous êtes venus vers votre serviteur. Et ils dirent : Fais ce que tu as dit.
\VS{6}Abraham donc s'en alla en hâte dans la tente vers Sara, et lui dit : Hâte-toi, prends trois mesures de fleur de farine, pétris-les, et fais des gâteaux.
\VS{7}Puis Abraham courut au troupeau et prit un veau tendre et bon, et le donna à un serviteur qui se hâta de l'apprêter.
\VS{8}Ensuite, il prit du beurre et du lait, et le veau qu'on avait apprêté, et le mit devant eux ; et il se tint auprès d'eux sous l'arbre, et ils mangèrent.
\VS{9}Et ils lui dirent : Où est Sara ta femme ? Et il répondit : La voilà dans la tente.
\VS{10}Et l'un d'entre eux dit : Je ne manquerai pas de revenir vers toi en ce même temps où nous sommes, et voici, Sara, ta femme, aura un fils. Et Sara écoutait à la porte de la tente qui était derrière lui\FTNT{Ro. 9:9.}.
\VS{11}Or Abraham et Sara étaient vieux, fort avancés en âge ; et Sara n'avait plus ce que les femmes sont accoutumées d'avoir\FTNT{Ro. 4:19 ; Hé. 11:11.}.
\VS{12}Et Sara rit en elle-même et dit : Etant vieille, et mon Seigneur étant fort âgé, aurai-je encore des désirs ?
\VS{13}Et Yahweh dit à Abraham : Pourquoi Sara a-t-elle ri en disant : Serait-il vrai que j'aurais un enfant, étant vieille comme je suis ?
\VS{14}Y a-t-il quelque chose qui soit difficile à Yahweh ? Je reviendrai vers toi à cette époque, en ce même temps où nous sommes et Sara aura un fils\FTNT{Mt. 19:26 ; Lu. 1:37.}.
\VS{15}Et Sara le nia en disant : Je n'ai point ri ; car elle avait peur. Mais il dit : Cela n'est pas, car tu as ri.
\VS{16}Et ces hommes se levèrent de là, et regardèrent vers Sodome ; et Abraham alla  avec eux pour les accompagner.
\VS{17}Et Yahweh dit : Cacherai-je à Abraham ce que je vais faire ?
\VS{18}Abraham deviendra certainement une nation grande et puissante, et toutes les nations de la terre seront bénies en lui\FTNT{Ac. 3:25 ; Ga. 3:8.}.
\VS{19}Car je le connais, et je sais qu'il ordonnera à ses enfants, et à sa maison après lui, de garder la voie de Yahweh, pour faire ce qui est juste et droit ; afin que Yahweh fasse venir sur Abraham tout ce qu'il lui a dit.
\VS{20}Et Yahweh dit : Le cri contre Sodome et Gomorrhe s’est accru, et leur péché s’est fort aggravé.
\VS{21}Je descendrai maintenant, et je verrai s'ils ont fait entièrement selon le cri qui est venu jusqu'à moi ; et si cela n'est pas, je le saurai.
\VS{22}Ces hommes donc partant de là allèrent vers Sodome ; mais Abraham se tint encore devant Yahweh.
\TextTitle{Intercession d'Abraham}
\VS{23}Et Abraham s'approcha et dit : Feras-tu périr le juste avec le méchant ?
\VS{24}Peut-être y a-t-il cinquante justes dans la ville, les feras-tu périr aussi ? Ne pardonneras-tu point à la ville à cause des cinquante justes qui sont au milieu d’elle ?
\VS{25}Non, il n'arrivera pas que tu fasses une telle chose, que tu fasses mourir le juste avec le méchant, et que le juste soit traité comme le méchant ! Non, tu ne le feras point. Celui qui juge toute la terre ne fera-t-il point justice\FTNT{Ro. 3:5-6.} ?
\VS{26}Et Yahweh dit : Si je trouve dans Sodome cinquante justes au milieu de la ville, je pardonnerai à toute la ville à cause d'eux.
\VS{27}Et Abraham répondit en disant : Voici, j'ai pris maintenant la hardiesse de parler au Seigneur, moi qui ne suis que poussière et cendres.
\VS{28}Peut-être en manquera-t-il cinq des cinquante justes ; détruiras-tu toute la ville pour ces cinq-là ? Et Yahweh lui répondit : Je ne la détruirai point si j'y trouve quarante-cinq justes.
\VS{29}Abraham continua de lui parler en disant : Peut-être s'y trouvera-t-il quarante ? Et il dit : Je ne la détruirai point pour l'amour des quarante.
\VS{30}Abraham dit : Je prie le Seigneur de ne pas s'irriter si je parle encore. Peut-être s'en trouvera-t-il trente ? Et il dit : Je ne la détruirai point si j'y trouve trente.
\VS{31}Abraham dit : Voici, maintenant j'ai pris la hardiesse de parler au Seigneur : Peut-être s'en trouvera-t-il vingt ? Et il dit : Je ne la détruirai point pour l'amour des vingt.
\VS{32}Abraham dit : Je prie le Seigneur de ne pas s'irriter, je parlerai encore une seule fois : Peut-être s'y trouvera-t-il dix. Et Yahweh dit : Je ne la détruirai point pour l'amour des dix.
\VS{33}Yahweh s'en alla quand il eut achevé de parler avec Abraham. Et Abraham retourna dans sa demeure.
\Chap{19}
\TextTitle{Des anges chez Lot\FTNTT{Ge. 13:10, 12 ; 19:33}}
\VerseOne{}Sur le soir, les deux anges arrivèrent à Sodome, et Lot était assis à la porte de Sodome. Quand Lot les vit, il se leva pour aller au-devant d'eux, et se prosterna la face contre terre.
\VS{2}Et il leur dit : Voici, je vous prie, mes seigneurs, entrez maintenant dans la maison de votre serviteur, et passez-y la nuit ; lavez-vous les pieds ; puis vous vous lèverez dès le matin et continuerez votre chemin ; et ils dirent : Non, mais nous passerons la nuit dans la rue.
\VS{3}Mais il les pressa tellement qu'ils se retirèrent chez lui ; et quand ils furent entrés dans sa maison, il leur fit un festin, et fit cuire des pains sans levain, et ils mangèrent.
\VS{4}Ils n’étaient pas encore couchés que les hommes de la ville, les hommes de Sodome, environnèrent la maison, depuis les plus jeunes jusqu'aux vieillards, tout le peuple était ensemble.
\VS{5}Ils appelèrent Lot et ils lui dirent : Où sont les hommes qui sont venus cette nuit chez toi ? Fais-les sortir afin que nous les connaissions.
\VS{6}Mais Lot sortit de sa maison pour leur parler à la porte, et ayant fermé la porte après lui,
\VS{7}il leur dit : Je vous prie, mes frères, ne leur faites point de mal.
\VS{8}Voici, j'ai deux filles qui n'ont point encore connu d'homme ; je vous les amènerai et vous les traiterez comme il vous plaira. Seulement, ne faites pas de mal à ces hommes, car ils sont venus à l'ombre de mon toit.
\VS{9}Ils lui dirent : Retire-toi de là. Ils dirent aussi : Cet homme seul est venu pour habiter ici comme étranger, et il veut nous gouverner ? Maintenant nous te ferons pis qu'à eux. Et faisant violence à Lot,  ils s'approchèrent pour briser la porte\FTNT{2 Pi. 2:7-8.}.
\VS{10}Mais les hommes étendirent leurs mains, firent rentrer Lot vers eux dans la maison, et fermèrent la porte.
\VS{11}Et ils frappèrent d’aveuglement les hommes qui étaient à la porte de la maison, depuis le plus petit jusqu'au plus grand, de sorte qu'ils se lassèrent à chercher la porte.
\VS{12}Alors ces hommes dirent à Lot : Qui as-tu encore ici qui t'appartienne ? Gendres, fils et filles, et tout ce qui t'appartient dans la ville, fais-les sortir de ce lieu.
\VS{13}Car nous allons détruire ce lieu parce que le cri contre ses habitants est grand devant Yahweh. Yahweh nous a envoyés pour le détruire.
\VS{14}Lot sortit donc et parla à ses gendres, qui devaient prendre ses filles, et leur dit : Levez-vous, sortez de ce lieu, car Yahweh va détruire la ville. Mais aux yeux de ses gendres, il parut plaisanter.
\TextTitle{Jugement sur Sodome}
\VS{15}Dès l’aube du jour, les anges pressèrent Lot en disant : Lève-toi, prends ta femme et tes deux filles qui se trouvent ici, de peur que tu ne périsses dans le châtiment de la ville.
\VS{16}Et comme il tardait, ces hommes le prirent par la main, et ils prirent aussi par la main sa femme et ses deux filles, parce que Yahweh voulait l'épargner ; et ils l'emmenèrent et le mirent hors de la ville.
\VS{17}Après les avoir fait sortir, l'un d’eux dit : Sauve ta vie, ne regarde point derrière toi, et ne t'arrête en aucun endroit de la plaine ; sauve-toi sur la montagne, de peur que tu ne périsses.
\VS{18}Lot leur répondit : Non, Seigneur, je te prie.
\VS{19}Voici, ton serviteur a maintenant trouvé grâce devant toi, et tu as montré la grandeur de ta bonté à mon égard en préservant ma vie, mais je ne pourrai pas me sauver vers la montagne avant que le mal ne m'atteigne, et je mourrai.
\VS{20}Voici, je te prie, cette ville-là est proche ; je puis m'y enfuir, et elle est petite. Je te prie, que je m'y sauve ; n'est-elle pas petite ? Et mon âme vivra.
\VS{21}Et il lui dit : Voici, je t'ai exaucé encore en cela, de ne point détruire la ville dont tu as parlé.
\VS{22}Hâte-toi, sauve-toi là, car je ne pourrai rien faire jusqu'à ce que tu y sois entré ; c'est pourquoi cette ville fut appelée Tsoar.
\VS{23}Comme le soleil se levait sur la terre, Lot entra dans Tsoar.
\VS{24}Alors Yahweh fit pleuvoir du ciel, sur Sodome et sur Gomorrhe, du soufre et du feu, de la part de Yahweh\FTNT{De. 29:23 ; Lu. 17:29 ; Jud. 1:7.} ;
\VS{25}et il détruisit ces villes-là, et toute la plaine, et tous les habitants des villes, et les herbes de la terre.
\VS{26}Mais la femme de Lot regarda en arrière, et elle devint une statue de sel\FTNT{Lu. 17:31-33.}.
\VS{27}Abraham se leva de bon matin et vint au lieu où il s'était tenu devant Yahweh ;
\VS{28}et regardant vers Sodome et Gomorrhe, et vers toute la terre de cette plaine-là, il vit monter de la terre une fumée comme la fumée d'une fournaise.
\VS{29}Lorsque Dieu détruisit les villes de la plaine, il se souvint d'Abraham, et laissa Lot s’en aller  du milieu du désastre par lequel il détruisit les villes où Lot avait établi sa demeure.
\TextTitle{Une abomination commise dans la famille de Lot\FTNTT{Ge. 13:10,12 ; 19:1 ; Lu. 22:31-62}}
\VS{30}Lot quitta Tsoar et habita sur la montagne avec ses deux filles, car il craignait de demeurer dans Tsoar, et il se retira dans une caverne avec ses deux filles.
\VS{31}L'aînée dit à la plus jeune : Notre père est vieux, et il n'y a personne sur la terre pour venir vers nous, selon la coutume de tous les pays.
\VS{32}Viens, donnons du vin à notre père, et couchons avec lui  afin que nous conservions la race de notre père.
\VS{33}Elles donnèrent donc du vin à boire à leur père cette nuit-là ; et l'aînée vint, et coucha avec son père, mais il ne s'aperçut point ni quand elle se coucha ni quand elle se leva.
\VS{34}Le lendemain, l'aînée dit à la plus jeune : Voici, j'ai couché la nuit dernière avec mon père, donnons-lui encore du vin à boire cette nuit, puis va et couche avec lui, et nous conserverons la race de notre père.
\VS{35}Elles firent boire du vin à leur père encore cette nuit-là ; et la plus jeune se leva et coucha avec lui ; mais il ne s'aperçut point ni quand elle se coucha ni quand elle se leva.
\VS{36}Ainsi, les deux filles de Lot conçurent de leur père.
\VS{37}L’aînée enfanta un fils qu’elle appela du nom de Moab ; c'est le père des Moabites jusqu'à ce jour.
\VS{38}La plus jeune aussi enfanta un fils qu’elle appela du nom de Ben-Ammi ; c'est le père des Ammonites jusqu'à ce jour.
\Chap{20}
\TextTitle{Faute d'Abraham à Guérar\FTNTT{Ge. 26:6-32}}
\VerseOne{}Abraham s'en alla de là pour le pays du midi ; il demeura entre Kadès et Schur, et il habita comme étranger à Guérar.
\VS{2}Abraham disait de Sara sa femme : C'est ma sœur. Et Abimélec, roi de Guérar, envoya des gens prendre Sara.
\VS{3}Mais Dieu apparut la nuit dans un songe à Abimélec, et lui dit : Voici, tu vas mourir, à cause de la femme que tu as prise, car elle a un mari.
\VS{4}Abimélec, qui ne s'était point approché d'elle, répondit : Seigneur, feras-tu donc mourir une nation juste ?
\VS{5}Ne m'a-t-il pas dit : C'est ma sœur? Et elle-même aussi n'a-t-elle pas dit : C'est mon frère ? J'ai fait ceci dans l'intégrité de mon cœur et dans la pureté de mes mains.
\VS{6}Dieu lui dit en songe : Je sais que tu l'as fait dans l'intégrité de ton cœur, aussi ai-je empêché que tu ne pèches contre moi ; c'est pourquoi je n'ai pas permis que tu la touches.
\VS{7}Maintenant donc rends la femme de cet homme, car il est prophète ; et il priera pour toi et tu vivras. Mais si tu ne la rends pas, sache que tu mourras toi et tout ce qui est à toi.
\VS{8}Abimélec se leva de bon matin, appela tous ses serviteurs, et rapporta à leurs oreilles  toutes ces choses, et ils furent saisis de crainte.
\VS{9}Puis Abimélec appela Abraham et lui dit : Que nous as-tu fait ? Et en quoi t'ai-je offensé que tu aies fait venir sur moi et sur mon royaume un grand péché ? Tu m'as fait des choses qui ne doivent point se faire.
\VS{10}Abimélec dit aussi à Abraham : Qu'as-tu vu qui t'aie obligé de faire cela ?
\VS{11}Abraham répondit : C'est parce que je disais : Assurément, il n'y a point de crainte de Dieu dans ce pays, et ils me tueront à cause de ma femme.
\VS{12}De plus, il est vrai qu’elle est ma soeur, fille de mon père ; mais elle n'est pas fille de ma mère ; et elle m'a été donnée pour femme.
\VS{13}Lorsque Dieu me fit errer loin de la maison de mon père, je dis à Sara : Voici la grâce que tu me feras, dis de moi dans tous les lieux où nous irons : C'est mon frère.
\VS{14}Alors Abimélec prit des brebis, des bœufs, des serviteurs et des servantes, et les donna à Abraham, et lui rendit Sara, sa femme.
\VS{15}Abimélec lui dit : Voici, mon pays est à ta disposition, demeure où il te plaira.
\VS{16}Et il dit à Sara : Voici, je donne à ton frère mille pièces d'argent ; cela te sera un voile sur les yeux  pour tous ceux qui sont avec toi, et envers tous les autres ; et ainsi elle fut reprise.
\VS{17}Abraham pria Dieu, et Dieu guérit Abimélec, sa femme, et ses servantes ; et elles eurent des enfants.
\VS{18}Car Yahweh avait frappé de stérilité en  fermant toute matrice de la maison d'Abimélec, à cause de Sara, femme d'Abraham.
\Chap{21}
\TextTitle{Naissance d'Isaac}
\VerseOne{}Et Yahweh visita Sara, comme il avait dit ; et il agit selon ses paroles.
\VS{2}Sara donc conçut, et enfanta un fils à Abraham dans sa vieillesse, au temps précis que Dieu lui avait dit.
\VS{3}Abraham donna le nom d’Isaac au fils qui lui était né, que Sara lui avait enfanté.
\VS{4}Abraham circoncit son fils Isaac âgé de huit jours, comme Dieu le lui avait ordonné.
\VS{5}Abraham était âgé de cent ans quand Isaac, son fils, lui naquit.
\VS{6}Et Sara dit : Dieu m'a donné de quoi rire ; tous ceux qui l'apprendront riront avec moi.
\VS{7}Elle dit aussi : Qui aurait dit à Abraham que Sara allaiterait des enfants ? Car je lui ai enfanté un fils dans sa vieillesse.
\VS{8}L'enfant grandit et fut sevré ; et Abraham fit un grand festin le jour où Isaac fut sevré.
\TextTitle{Abraham chasse Agar avec Ismaël\FTNTT{Ga. 4:21-31}}
\VS{9}Sara vit rire le fils qu’Agar, l’Egyptienne, avait enfanté à Abraham ;
\VS{10}et elle dit à Abraham : Chasse cette servante et son fils, car le fils de cette servante n'héritera point avec mon fils, avec Isaac\FTNT{Ga. 4:30.}.
\VS{11}Cette parole déplut fort à Abraham à cause de son fils.
\VS{12}Mais Dieu dit à Abraham : N'aie point de chagrin au sujet de l'enfant ni de ta servante ;  écoute la parole de Sara dans toutes les choses qu’elle te dira, car en Isaac te sera donnée une postérité.
\VS{13}Je ferai aussi devenir le fils de la servante une nation, parce qu'il est ta semence.
\VS{14}Puis Abraham se leva de bon matin et prit du pain et une outre d'eau, et il les donna à Agar en les mettant sur son épaule. Il lui donna aussi l'enfant et la renvoya. Elle se mit en chemin et fut errante au désert de Beer-Schéba.
\VS{15}Quand l'eau de l’outre fut épuisée, elle jeta l'enfant sous un arbrisseau,
\VS{16}et elle alla s’asseoir vis-à-vis, à une portée d’arc, car elle dit : Que je ne voie pas mourir mon enfant. Elle s’assit donc vis-à-vis de lui, éleva la voix et pleura.
\VS{17}Dieu entendit la voix de l'enfant, et l'Ange de Dieu appela des cieux Agar et lui dit : Qu'as-tu Agar ? Ne crains point, car Dieu a entendu la voix de l'enfant du lieu où il est.
\VS{18}Lève-toi, lève l'enfant, et prends-le par la main, car je le ferai devenir une grande nation.
\VS{19}Et Dieu lui ouvrit les yeux et elle vit un puits d'eau ; elle alla remplir d'eau l’outre, et donna à boire à l'enfant.
\VS{20}Dieu fut avec l'enfant, qui devint grand, et demeura dans le désert ; et il fut tireur d'arc.
\VS{21}Il habita dans le désert de Paran ; et sa mère lui prit une femme du pays d'Egypte.
\TextTitle{Abraham à Beer-Schéba}
\VS{22}Et il arriva en ce temps-là qu'Abimélec, et Picol, chef de son armée, parla à Abraham en disant : Dieu est avec toi dans toutes les choses que tu fais.
\VS{23}Maintenant donc jure-moi ici par le nom de Dieu que tu ne me mentiras point, ni à mes enfants ni aux enfants de mes enfants, et que selon la faveur que je t'ai faite, tu agiras envers moi et envers le pays où tu séjournes comme étranger.
\VS{24}Abraham répondit : Je te le jurerai.
\VS{25}Mais Abraham fit des reproches à Abimélec au sujet d'un puits d'eau, dont les serviteurs d'Abimélec s'étaient emparés de force.
\VS{26}Abimélec répondit : J’ignore qui a fait cela, et aussi tu ne m'en as point informé, et moi, je ne l’apprends qu’aujourd’hui.
\VS{27}Alors Abraham prit des brebis et des bœufs, et les donna à Abimélec, et ils firent alliance ensemble.
\VS{28}Abraham mit à part sept jeunes brebis de son troupeau.
\VS{29}Et Abimélec dit à Abraham : Que veulent dire ces sept jeunes brebis que tu as mises à part ?
\VS{30}Il répondit : C'est que tu prendras ces sept jeunes brebis de ma main pour me servir de témoignage que j'ai creusé ce puits.
\VS{31}C'est pourquoi on appela ce lieu-là Beer-Schéba, car tous deux y jurèrent.
\VS{32}Ils traitèrent donc alliance à Beer-Schéba, puis Abimélec se leva avec Picol, chef de son armée, et ils retournèrent au pays des Philistins.
\VS{33}Abraham planta des tamaris à Beer-Schéba ; et là il invoqua le nom de Yahweh, le Dieu de l’éternité.
\VS{34}Abraham séjourna beaucoup de jours comme étranger dans le pays des Philistins.
\Chap{22}
\TextTitle{Abraham présente Isaac en sacrifice\FTNTT{Hé. 11:17-19}}
\VerseOne{}Or, il arriva après ces choses, que Dieu éprouva Abraham et lui dit : Abraham ! Et il répondit : Me voici.
\VS{2}Et Dieu lui dit : Prends maintenant ton fils, ton unique, celui que tu aimes, Isaac, et va-t'en au pays de Morija, et là offre-le en holocauste sur l'une des montagnes que je te dirai.
\VS{3}Abraham donc s'étant levé de bon matin, sella son âne, et prit deux de ses serviteurs avec lui, et Isaac son fils ; et ayant fendu le bois pour l'holocauste, il se mit en chemin et s'en alla au lieu que Dieu lui avait dit.
\VS{4}Le troisième jour, Abraham levant ses yeux, vit le lieu de loin.
\VS{5}Et Abraham dit à ses serviteurs : Restez ici avec l'âne ; moi et l'enfant nous irons jusque-là pour adorer, après quoi nous reviendrons auprès de vous.
\VS{6}Abraham prit le bois de l'holocauste et le mit sur Isaac, son fils, et prit le feu dans sa main, et un couteau ; et ils s'en allèrent tous deux ensemble.
\VS{7}Alors Isaac parla à Abraham, son père, et dit : Mon père ! Abraham répondit : Me voici mon fils. Et il dit : Voici le feu et le bois, mais où est l’agneau pour l'holocauste\FTNT{Isaac est un autre type de Christ qui s’offre en sacrifice pour l’expiation de nos péchés. La réponse à sa question au v. 7:«~Voici le feu et le bois, mais où est l’agneau pour l’holocauste ?~», a été apportée bien des siècles plus tard par Jean-Baptiste:«~Voici l’agneau de Dieu, qui ôte le péché du monde~». (Jn. 1:29).} ?
\VS{8}Abraham répondit : Mon fils, Dieu se pourvoira lui-même de l’agneau pour l'holocauste. Et ils marchèrent tous deux ensemble.
\VS{9}Et étant arrivés au lieu que Dieu lui avait dit, Abraham bâtit là un autel, et rangea le bois, et ensuite il lia Isaac, son fils, et le mit sur l'autel, par-dessus le bois\FTNT{Ja. 2:21.}.
\VS{10}Puis Abraham étendit sa main et prit le couteau pour égorger son fils.
\VS{11}Mais l'Ange de Yahweh l’appela des cieux et dit : Abraham, Abraham ! Il répondit : Me voici.
\VS{12}L’Ange lui dit : Ne porte pas ta main sur l'enfant, et ne lui fais rien ; car maintenant je sais que tu crains Dieu, puisque tu ne m’as point refusé ton fils, ton unique.
\VS{13}Abraham leva les yeux et regarda ; et voici,  il vit derrière lui un bélier qui était retenu à un buisson par ses cornes ; et Abraham alla prendre le bélier et l'offrit en holocauste à la place de son fils.
\VS{14}Abraham donna à ce lieu le nom de Yahweh-Jiré (Yahweh pourvoira) ; c'est pourquoi on dit aujourd'hui : Dans la montagne de Yahweh il y sera pourvu.
\VS{15}L'Ange de Yahweh appela des cieux Abraham pour la seconde fois,
\VS{16}et dit : Je le jure par moi-même\FTNT{Hé. 6:13-15.}, parole de Yahweh ! Parce que tu as fait cela, et que tu n'as point refusé ton fils, ton unique,
\VS{17}certainement je te bénirai, et je multiplierai très abondamment ta postérité, comme les étoiles du ciel et comme le sable qui est sur le bord de la mer ; et ta postérité possédera la porte de ses ennemis.
\VS{18}Toutes les nations de la terre seront bénies en ta postérité, parce que tu as obéi à ma voix.
\VS{19}Ainsi Abraham retourna vers ses serviteurs, et ils se levèrent et s'en allèrent ensemble à Beer-Schéba ; car Abraham demeurait à Beer-Schéba.
\VS{20}Après ces choses, quelqu'un apporta des nouvelles à Abraham, en disant : Voici, Milca a aussi enfanté des fils à Nachor, ton frère.
\VS{21}Uts, son premier-né, et Buz, son frère, Kemuel, père d'Aram,
\VS{22}Késed, Hazo, Pildasch, Jidlaph et Bethuel.
\VS{23}Bethuel a engendré Rebecca. Milca enfanta ces huit fils à Nachor, frère d'Abraham.
\VS{24}Sa concubine, nommée Réuma, enfanta aussi Thébach, Gaham, Tahasch, et Maaca.
\Chap{23}
\TextTitle{Mort de Sara}
\VerseOne{}Or, Sara vécut cent vingt-sept ans ; ce sont là les années de la vie de Sara.
\VS{2}Sara mourut à Kirjath-Arba, qui est Hébron, dans le pays de Canaan ; et Abraham vint pour mener deuil sur Sara et pour la pleurer.
\VS{3}Et Abraham se leva de devant son mort, il parla aux fils de Heth, en disant :
\VS{4}Je suis étranger et habitant parmi vous ; donnez-moi une possession de sépulcre parmi vous, afin que j'enterre mon mort et que je l'ôte de devant moi\FTNT{Ac. 7:5.}.
\VS{5}Les fils de Heth répondirent à Abraham et lui dirent :
\VS{6}Mon seigneur, écoute-nous ! Tu es un prince de Dieu parmi nous, enterre ton mort dans le plus distingué de nos sépulcres ; nul de nous ne te refusera son sépulcre afin que tu y enterres ton mort.
\VS{7}Alors Abraham se leva et se prosterna devant le peuple du pays, devant les Héthiens.
\VS{8}Et il leur parla et dit : S'il vous plaît que j'enterre mon mort et que je l'ôte de devant moi ; écoutez-moi, et intercédez pour moi envers Ephron, fils de Tsochar,
\VS{9}afin qu'il me cède sa caverne de Macpéla, qui est à l’extrémité de son champ ; qu'il me la cède contre sa valeur en argent, afin qu’elle me serve de possession sépulcrale au milieu de vous.
\VS{10}Ephron était assis parmi les fils de Heth. Et Ephron, l’Héthien, répondit à Abraham, en présence des fils de Heth qui l'écoutaient, devant tous ceux qui entraient par la porte de sa ville, et dit :
\VS{11}Non, mon seigneur, écoute-moi ! Je te donne le champ, je te donne aussi la caverne qui y est, je te la donne en présence des enfants de mon peuple ; enterres-y ton mort.
\VS{12}Abraham se prosterna devant le peuple du pays.
\VS{13}Et il parla ainsi à Ephron, en présence de tout le peuple du pays qui écoutait et dit : S'il te plaît, je te prie, écoute-moi ! Je donnerai l'argent du champ ; reçois-le de moi, et j'y enterrerai mon mort.
\VS{14}Et Ephron répondit à Abraham, en disant :
\VS{15}Mon seigneur, écoute-moi ! La terre vaut quatre cents sicles d'argent, qu’est-ce que cela entre moi et toi ? Enterre donc ton mort.
\VS{16}Abraham ayant entendu Ephron, lui paya l'argent dont il avait parlé, en présence des fils de Heth, à savoir quatre cents sicles d'argent ayant cours chez les marchands\FTNT{Ac. 7:16.}.
\VS{17}Le champ d'Ephron, qui était à Macpéla, vis-à-vis de Mamré, le champ et la caverne qui y est, et tous les arbres qui sont dans le champ et dans toutes ses limites alentour,
\VS{18}tout fut acquis comme propriété d’Abraham, en présence des fils de Heth, et de tous ceux qui entraient par la porte de la ville.
\VS{19}Après cela, Abraham enterra Sara, sa femme, dans la caverne du champ de Macpéla, vis-à-vis de Mamré, qui est Hébron, dans le pays de Canaan.
\VS{20}Le champ et la caverne qui y est demeurèrent à Abraham comme possession sépulcrale, acquise des fils de Heth.
\Chap{24}
\TextTitle{Abraham recherche une épouse pour Isaac}
\VerseOne{}Or, Abraham devint vieux et fort avancé en âge ; et Yahweh avait béni Abraham en toute chose.
\VS{2}Abraham dit à son serviteur, le plus ancien des serviteurs de sa maison, l’intendant de tout ce qui lui appartenait : Mets, je te prie, ta main sous ma cuisse ;
\VS{3}et je te ferai jurer par Yahweh, le Dieu du ciel et le Dieu de la terre, que tu ne prendras point de femme pour mon fils parmi les filles des Cananéens, au milieu desquels j'habite.
\VS{4}Mais tu iras dans mon pays et vers mes parents, et tu y prendras une femme pour mon fils Isaac.
\VS{5}Le serviteur lui répondit : Peut-être que la femme ne voudra-t-elle pas me suivre dans ce pays ; me faudra-t-il nécessairement ramener ton fils dans le pays d'où tu es sorti ?
\VS{6}Abraham lui dit : Garde-toi bien d'y ramener mon fils !
\VS{7}Yahweh, le Dieu du ciel, qui m'a fait sortir de la maison de mon père et de ma patrie, qui m'a parlé et qui m’a juré en disant : Je donnerai ce pays à ta postérité, enverra lui-même son ange devant toi ; et c’est là que tu prendras une femme pour mon fils.
\VS{8}Si la femme ne veut pas te suivre, tu seras quitte de ce serment que je te fais faire. Quoi qu'il en soit, tu n’y ramèneras point mon fils.
\VS{9}Le serviteur mit la main sous la cuisse d'Abraham, son Seigneur, et lui jura d’observer ces choses.
\VS{10}Alors le serviteur prit dix chameaux parmi les chameaux de son maître, et s'en alla, ayant à sa disposition tous les biens. Il partit donc et s'en alla en Mésopotamie, à la ville de Nachor.
\VS{11}Il fit reposer les chameaux sur leurs genoux hors de la ville, près d'un puits d'eau, sur le soir, au temps où sortent celles qui vont puiser de l'eau.
\VS{12}Et il dit : Ô Yahweh, Dieu de mon seigneur Abraham, fais que j'aie une heureuse rencontre aujourd'hui, et sois favorable à mon seigneur Abraham.
\VS{13}Voici, je me tiens près de la source d'eau, et les filles des gens de la ville vont sortir pour puiser de l'eau.
\VS{14}Fais donc que la jeune fille à laquelle je dirai : Penche ta cruche, je te prie, afin que je boive, et qui me répondra : Bois, et je donnerai aussi à boire à tes chameaux, soit celle que tu as destinée à ton serviteur Isaac, et par là je connaîtrai que tu es favorable à mon seigneur.
\VS{15}Il n’avait pas encore fini de parler que sortit sa cruche sur l’épaule, Rebecca, fille de Bethuel, fils de Milca, femme de Nachor, frère d'Abraham.
\VS{16}Et la jeune fille était très belle de figure ; elle était vierge, et aucun homme ne l'avait connue. Elle descendit donc à la source, et comme elle remontait après avoir rempli sa cruche,
\VS{17}le serviteur courut au-devant d'elle et lui dit : Laisse-moi boire, je te prie, un peu d’eau de ta cruche.
\VS{18}Elle répondit : Mon seigneur, bois. Elle s’empressa d’abaisser sa cruche sur sa main,  et elle lui donna à boire.
\VS{19}Quand elle eut achevé de lui donner à boire, elle dit : Je puiserai aussi pour tes chameaux jusqu'à ce qu'ils aient achevé de boire.
\VS{20}Et elle s’empressa de vider sa cruche dans l’abreuvoir ; elle courut encore au puits pour puiser de l'eau, et elle puisa pour tous ses chameaux.
\VS{21}L’homme la regardait avec étonnement et sans rien dire, pour voir si Yahweh faisait réussir son voyage ou non.
\VS{22}Quand les chameaux eurent fini de boire, l’homme prit un anneau d'or, du poids d'un demi-sicle, et deux bracelets, pour les mettre sur les mains de cette fille, pesant dix sicles d'or.
\VS{23}Et il lui dit : De qui es-tu fille ? Je te prie, fais-le-moi savoir. Y a-t-il dans la maison de ton père de la place pour nous loger ?
\VS{24}Elle lui répondit : Je suis fille de Bethuel, fils de Milca et de Nachor.
\VS{25}Elle lui dit encore : Il y a chez nous de la paille et du fourrage en abondance, et de la place pour loger.
\VS{26}Alors l’homme s'inclina et adora Yahweh,
\VS{27}et dit : Béni soit Yahweh, le Dieu de mon seigneur Abraham, qui n'a point cessé d'exercer sa bonté et sa fidélité envers mon Seigneur ! Lorsque j'étais en chemin, Yahweh m'a conduit dans la maison des frères de mon seigneur.
\VS{28}La jeune fille courut et rapporta toutes ces choses à la maison de sa mère.
\VS{29}Rebecca avait un frère nommé Laban, qui courut dehors vers l’homme près de la source.
\VS{30}Il avait vu l’anneau et les bracelets aux mains de sa sœur, et il avait entendu les paroles de Rebecca sa sœur, disant : Ainsi m’a parlé l’homme. Il vint donc à cet homme qui se tenait auprès des chameaux, près de la source,
\VS{31}et il lui dit : Entre, béni de Yahweh ! Pourquoi te tiens-tu dehors ? J'ai préparé la maison et une place pour tes chameaux.
\VS{32}L'homme donc entra dans la maison. Laban fit décharger les chameaux, et il donna de la paille et du fourrage  aux chameaux ; et il apporta de l'eau pour laver les pieds de l’homme et les pieds de ceux qui étaient avec lui.
\VS{33}Et il lui présenta à manger. Mais il dit : Je ne mangerai point avant d’avoir dit  ce que j'ai à dire. Parle ! dit Laban.
\VS{34}Alors il dit : Je suis serviteur d'Abraham.
\VS{35}Yahweh a comblé de bénédictions mon seigneur qui est devenu puissant. Il lui a donné des brebis, des bœufs, de l'argent, de l'or, des serviteurs, des servantes, des chameaux, et des ânes.
\VS{36}Sara, la femme de mon seigneur, a enfanté dans sa vieillesse un fils à mon seigneur ; et il lui a donné tout ce qu'il possède.
\VS{37}Mon seigneur m'a fait jurer en disant : Tu ne prendras point de femme pour mon fils parmi les filles des Cananéens dans le pays desquels j’habite ;
\VS{38}mais tu iras dans la maison de mon père et de ma famille prendre une femme pour mon fils.
\VS{39}J’ai dit à mon seigneur : Peut-être que la femme ne voudra-t-elle pas me suivre.
\VS{40}Et il m’a répondu : Yahweh, devant la face de qui j'ai marché, enverra son ange avec toi, et fera réussir ton voyage ; et tu prendras pour mon fils une femme de ma famille et de la maison de mon père.
\VS{41}Quand tu auras été vers ma famille, tu seras alors dégagé de la punition du serment que je te fais faire ; et si on ne te la donne pas, tu seras dégagé de la punition du serment que je te fais faire.
\VS{42}Je suis arrivé aujourd'hui à la source et j'ai dit : Ô Yahweh ! Dieu de mon seigneur Abraham, si tu daignes faire réussir le voyage que j'ai entrepris,
\VS{43}voici, je me tiendrai près de la source d'eau, et la jeune fille qui sortira pour puiser à qui je dirai : Laisse-moi boire, je te prie, un peu d’eau de ta cruche ; et qui me répondra :
\VS{44}Bois toi-même, et je puiserai aussi pour tes chameaux, que cette jeune fille soit la femme que Yahweh a destinée au fils de mon seigneur.
\VS{45}Avant que j’ai fini de parler en mon cœur, voici, Rebecca est sortie, ayant sa cruche sur son épaule ; elle est descendue à la source et a puisé de l'eau ; et je lui ai dit : Donne-moi, à boire, je te prie.
\VS{46}Elle s’est empressée d’abaisser sa cruche de dessus son épaule et m'a dit : Bois, et même je donnerai à boire à tes chameaux. J'ai donc bu, et elle a aussi donné à boire aux chameaux.
\VS{47}Puis je l'ai interrogée en disant : De qui es-tu fille ? Elle a répondu : Je suis fille de Bethuel, fils de Nachor et de Milca. Alors je lui ai mis un anneau à son nez et les bracelets à ses mains.
\VS{48}Puis je me suis incliné, j’ai adoré Yahweh, et j'ai béni Yahweh, le Dieu de mon seigneur Abraham, qui m'a conduit fidèlement, afin que je prenne la fille du frère de mon seigneur pour son fils.
\VS{49}Maintenant donc, si vous voulez user de bonté et de fidélité envers mon seigneur, déclarez-le-moi ; sinon, déclarez-le-moi aussi ; et je me tournerai à droite ou à gauche.
\VS{50}Laban et Bethuel répondirent et dirent : Cette affaire vient de Yahweh, nous ne pouvons te parler ni en bien ni en mal.
\VS{51}Voici Rebecca est devant toi, prends-la et va, et qu'elle soit la femme du fils de ton seigneur, comme Yahweh l’a dit.
\VS{52}Lorsque le serviteur d'Abraham eut entendu leurs paroles, il se prosterna à terre devant Yahweh.
\VS{53}Et le serviteur sortit des objets d'argent et d'or, et des vêtements, et les donna à Rebecca. Il donna aussi de riches présents à son frère et à sa mère.
\VS{54}Puis ils mangèrent et burent, lui et les gens qui étaient avec lui, et ils passèrent la nuit.  Le matin,  quand ils furent levés, le serviteur dit : Laissez-moi retourner vers mon seigneur.
\VS{55}Le frère et la mère lui dirent : Que la jeune fille reste avec nous quelques jours encore, une dizaine de jours ;  après quoi, elle s'en ira.
\VS{56}Il leur répondit : Ne me retardez pas puisque Yahweh a fait réussir mon voyage ; laissez-moi partir  afin que je m'en aille vers mon seigneur.
\VS{57}Alors ils dirent : Appelons la jeune fille et demandons-lui son avis.
\VS{58}Ils appelèrent donc Rebecca et lui dirent : Veux-tu aller avec cet homme ? Et elle répondit : J'irai.
\VS{59}Ainsi ils laissèrent partir Rebecca, leur sœur, et sa nourrice, avec le serviteur d'Abraham et ses gens.
\VS{60}Ils bénirent Rebecca et lui dirent : Tu es notre sœur, puisses-tu devenir des milliers de myriades, et que ta postérité possède la porte de ses ennemis !
\VS{61}Alors Rebecca se leva avec ses servantes, et elles montèrent sur les chameaux et suivirent l’homme. Et le serviteur prit Rebecca et s'en alla.
\VS{62}Or Isaac revenait du puits de Lachaï-roï, et il habitait dans le pays du midi.
\VS{63}Un soir qu’Isaac était sorti dans les champs pour prier, il leva les yeux et regarda, et voici, des chameaux arrivaient.
\VS{64}Rebecca leva aussi les yeux, vit Isaac, et descendit de son chameau ;
\VS{65}car elle avait dit au serviteur : Qui est cet homme qui marche dans les champs à notre rencontre ? Et le serviteur avait répondu : C'est mon seigneur ; et elle prit son voile et se couvrit.
\VS{66}Le serviteur raconta à Isaac toutes les choses qu'il avait faites.
\VS{67}Alors Isaac conduisit Rebecca dans la tente de Sara, sa mère ; il prit Rebecca pour sa femme\FTNT{Pr. 18:22 ; Pr. 31:10-31.} et l'aima. Ainsi Isaac fut consolé après la mort de sa mère.
\Chap{25}
\TextTitle{Ketura, femme d'Abraham}
\VerseOne{}Or, Abraham prit une autre femme nommée Ketura.
\VS{2}Elle lui enfanta Zimram, Jokschan, Medan, Madian, Jischbak, et Schuach.
\VS{3}Jokschan engendra Séba et Dedan. Les fils de Dedan furent Aschurim, Letuschim et Leummim.
\VS{4}Les fils de Madian furent Epha, Epher, Hénoc, Abida, Eldaa. Ce sont là tous les fils de Ketura.
\TextTitle{Isaac hérite d'Abraham\FTNTT{Hé. 1:2}}
\VS{5}Abraham donna tout ce qui lui appartenait à Isaac.
\VS{6}Mais il fit des dons aux fils de ses concubines, et tandis qu’il vivait encore, il les envoya loin de son fils Isaac, du côté de l'orient, dans le pays d’orient.
\TextTitle{Mort d'Abraham}
\VS{7}Voici les jours des années de la vie d’Abraham : Il vécut cent soixante-quinze ans.
\VS{8}Abraham expira et mourut après une heureuse vieillesse, fort âgé et rassasié de jours, et il fut recueilli auprès de son peuple.
\VS{9}Isaac et Ismaël, ses fils, l'enterrèrent dans la caverne de Macpéla, dans le champ d'Ephron, fils de Tschoar, le Héthien, qui est vis-à-vis de Mamré.
\VS{10}C’est le champ qu'Abraham avait acheté des fils de Heth. Là furent enterrés Abraham et Sara, sa femme.
\VS{11}Après la mort d'Abraham, Dieu bénit Isaac son fils.  Isaac habitait près du puits de Lachaï-roï.
\TextTitle{Postérité d'Ismaël}
\VS{12}Voici la postérité d'Ismaël, fils d'Abraham, qu'Agar l’Egyptienne, servante de Sara, avait enfanté à Abraham.
\VS{13}Voici les noms des fils d'Ismaël, par leurs noms, selon leurs générations. Le premier-né d'Ismaël fut Nebajoth, puis Kédar, Adbeel, Mibsam,
\VS{14}Mischma, Duma, Massa,
\VS{15}Hadad, Théma, Jethur, Naphisch, et Kedma.
\VS{16}Ce sont là les fils d'Ismaël, et ce sont là leurs noms, selon leurs parcs, et selon leurs enclos ; douze princes de leurs peuples.
\VS{17}Et voici les années de la vie d'Ismaël : Cent trente-sept ans. Il expira et mourut, et il fut recueilli auprès de son peuple.
\VS{18}Ses descendants habitèrent depuis Havila jusqu'à Schur, qui est vis-à-vis de l'Egypte, en allant vers l'Assyrie. Et le pays qui était échu à Ismaël était à la vue de tous ses frères.
\TextTitle{Postérité d'Isaac}
\VS{19}Voici la postérité d'Isaac, fils d'Abraham.
\VS{20}Abraham engendra Isaac. Isaac était âgé de quarante ans quand il épousa Rebecca, fille de Bethuel, le Syrien, de Paddan-Aram, sœur de Laban, le Syrien.
\VS{21}Isaac pria instamment Yahweh au sujet de sa femme parce qu'elle était stérile ; et Yahweh exauça ses prières ; et Rebecca, sa femme, conçut.
\VS{22}Mais les enfants se heurtaient dans son ventre, et elle dit : S'il en est ainsi, pourquoi suis-je enceinte ? Et elle alla consulter Yahweh.
\VS{23}Et Yahweh lui dit : Deux nations sont dans ton ventre, et deux peuples se sépareront au sortir de tes entrailles ; un de ces peuples sera plus fort que l'autre, et le plus grand sera asservi au plus petit\FTNT{Ro. 9:12.}.
\TextTitle{Naissance des jumeaux : Esaü et Jacob}
\VS{24}Les jours où elle devait accoucher s’accomplirent ; et voici, il y avait deux jumeaux dans son ventre.
\VS{25}Celui qui sortit le premier était roux et tout velu, comme un manteau de poil ; et on lui donna le nom d’Esaü.
\VS{26}Ensuite sortit son frère, tenant de sa main le talon d'Esaü ; c'est pourquoi il fut appelé Jacob\FTNT{Jacob:«~celui qui prend par le talon~» ou «~qui supplante~».}. Isaac était âgé de soixante ans quand ils naquirent.
\TextTitle{Esaü méprise son droit d'aînesse}
\VS{27}Depuis, les enfants devinrent grands. Esaü devint un habile chasseur, et un homme des champs ; mais Jacob fut un homme intègre, se tenant dans les tentes.
\VS{28}Isaac aimait Esaü ; car le gibier était sa nourriture. Mais Rebecca aimait Jacob.
\VS{29}Comme Jacob faisait cuire du potage, Esaü arriva des champs, et il était fatigué.
\VS{30}Et Esaü dit à Jacob : Donne-moi, je te prie, à manger de ce roux, de ce roux-là\FTNT{Probablement un plat de lentilles.} ; car je suis fatigué. C'est pourquoi on appela son nom, Edom\FTNT{Edom:«~rouge, de couleur rousse~».}.
\VS{31}Mais Jacob lui dit : Vends-moi aujourd'hui ton droit d'aînesse.
\VS{32}Et Esaü répondit : Voici, je m'en vais mourir ; et de quoi me servira le droit d'aînesse ?
\VS{33}Et Jacob dit : Jure-moi aujourd'hui ; et il lui jura ; ainsi il vendit son droit d'aînesse à Jacob\FTNT{Hé. 12:16.}.
\VS{34}Et Jacob donna à Esaü du pain et du potage de lentilles ; et il mangea et but ; puis il se leva et s'en alla ; ainsi Esaü méprisa son droit d'aînesse.
\Chap{26}
\TextTitle{Yahweh confirme son alliance à Isaac}
\VerseOne{}Or, il y eut une famine dans le pays, outre la première famine qui eut lieu du temps d'Abraham ; et Isaac s'en alla vers Abimélec, roi des Philistins, à Guérar.
\VS{2}Yahweh lui apparut et lui dit : Ne descends pas en Egypte ; demeure dans le pays que je te dirai.
\VS{3}Demeure dans ce pays-ci, et je serai avec toi, et je te bénirai ; car je donnerai toutes ces contrées à toi et à ta postérité, et j’accomplirai le serment que j'ai fait à ton père Abraham.
\VS{4}Je multiplierai ta postérité comme les étoiles du ciel ; et je donnerai ces contrées à ta postérité ; et toutes les nations de la terre seront bénies en ta postérité,
\VS{5}parce qu'Abraham a obéi à ma voix, et qu'il a gardé mon ordonnance, mes commandements, mes statuts et mes lois.
\TextTitle{Faute d'Isaac à Guérar\FTNTT{Ge. 20}}
\VS{6}Isaac donc demeura à Guérar.
\VS{7}Et quand les gens du lieu posaient des questions sur sa femme, il disait : C'est ma sœur ; car il craignait de dire : C'est ma femme ; de peur, disait-il, que les habitants du lieu ne me tuent à cause de Rebecca, car elle est belle de figure.
\VS{8}Comme son séjour se prolongeait, il arriva qu'Abimélec, roi des Philistins, regardant par la fenêtre, vit Isaac qui plaisantait avec Rebecca, sa femme\FTNT{Ge. 20.}.
\VS{9}Alors Abimélec appela Isaac et lui dit : Voici, c'est véritablement ta femme. Comment as-tu pu dire : C'est ma soeur ? Et Isaac lui répondit : C'est parce que j'ai dit : Il ne faut pas que je meure à cause d'elle.
\VS{10}Et Abimélec dit : Que nous as-tu fait ? Il s'en est peu fallu que quelqu'un du peuple n'ait couché avec ta femme, et tu nous aurais rendus coupables.
\VS{11}Abimélec donc fit une ordonnance à tout le peuple en disant : Celui qui touchera cet homme, ou à sa femme, sera certainement puni de mort.
\VS{12}Isaac sema dans cette terre-là et il recueillit cette année-là le centuple ; car Yahweh le bénit.
\VS{13}Cet homme devint riche, et il alla s’enrichissant de plus en plus, jusqu'à ce qu'il devint fort riche.
\VS{14}Il avait des troupeaux de menu bétail et des troupeaux de gros bétail, et un grand nombre de serviteurs ; et les Philistins lui portèrent envie ; 
\VS{15}Et tous les puits que les serviteurs de son père avaient creusés, du temps de son père Abraham, les Philistins les bouchèrent et les remplirent de terre.
\VS{16}Abimélec aussi dit à Isaac : Va-t’en de chez nous, car tu es devenu beaucoup plus puissant que nous.
\TextTitle{Les puits d'Isaac}
\VS{17}Isaac donc partit de là, et campa dans la vallée de Guérar, où il s’établit.
\VS{18}Isaac creusa de nouveau les puits d'eau qu'on avait creusés du temps d'Abraham, son père, et que les Philistins avaient bouchés après la mort d'Abraham, et il leur donna les mêmes noms que son père leur avait donnés.
\VS{19}Les serviteurs d'Isaac creusèrent dans cette vallée et y trouvèrent un puits d'eau vive.
\VS{20}Mais les bergers de Guérar eurent une querelle avec les bergers d'Isaac, disant : L'eau est à nous. Et il appela le nom du puits Esek parce qu'ils avaient contesté avec lui.
\VS{21}Ensuite, ils creusèrent un autre puits, pour lequel ils contestèrent aussi ; et il appela son nom Sitna.
\VS{22}Alors il se transporta de là et creusa un autre puits pour lequel ils ne contestèrent point, et il le nomma Rehoboth, en disant : C'est parce que Yahweh nous a maintenant mis au large, et nous fructifierons dans le pays.
\VS{23}Et de là il remonta à Beer-Schéba.
\VS{24}Yahweh lui apparut cette nuit-là et lui dit : Je suis le Dieu d'Abraham, ton père ; ne crains point, car je suis avec toi, je te bénirai et je multiplierai ta postérité à cause d'Abraham, mon serviteur.
\VS{25}Alors il bâtit là un autel, et invoqua le nom de Yahweh, et il y dressa ses tentes. Et les serviteurs d'Isaac y creusèrent un puits.
\VS{26}Abimélec vint à lui de Guérar avec Ahuzath, son ami, et Picol, chef de son armée.
\VS{27}Mais Isaac leur dit : Pourquoi venez-vous vers moi, puisque vous me haïssez et que vous m'avez renvoyé de chez vous ?
\VS{28}Ils répondirent : Nous avons vu clairement que Yahweh est avec toi ; et nous avons dit : Qu'il y ait maintenant un serment solennel entre nous, c'est-à-dire entre nous et toi ; et traitons alliance avec toi.
\VS{29}Jure que tu ne nous feras aucun mal, de même que nous ne t'avons point maltraité, que nous t'avons fait seulement du bien, et que nous t’avons laissé partir en paix. Toi qui es maintenant béni de Yahweh.
\VS{30}Alors il leur fit un festin, et ils mangèrent et burent.
\VS{31}Ils se levèrent de bon matin, et jurèrent l'un à l'autre. Puis Isaac les renvoya, et ils s'en allèrent en paix.
\VS{32}Ce même jour, les serviteurs d'Isaac vinrent lui parler du puits qu'ils avaient creusé, et lui dirent : Nous avons trouvé de l'eau.
\VS{33}Et il l'appela Schiba. C'est pourquoi le nom de la ville a été Beer-Schéba jusqu'à aujourd'hui.
\VS{34}Esaü, âgé de quarante ans, prit pour femmes Judith, fille de Beéri, le Héthien, et Basmath, fille d'Elon, le Héthien.
\VS{35}Elles furent un sujet d’amertume pour l’esprit d’Isaac et de Rebecca.
\Chap{27}
\TextTitle{Jacob prend la bénédiction d'Isaac à la place d'Esaü}
\VerseOne{}Et il arriva que quand Isaac fut devenu vieux, et que ses yeux furent si affaiblis qu'il ne pouvait plus voir, il appela Esaü, son fils aîné, et lui dit : Mon fils ! Et il lui répondit : Me voici.
\VS{2}Isaac lui dit : Voici, maintenant je suis devenu vieux, et je ne connais pas le jour de ma mort.
\VS{3}Maintenant donc, je te prie, prends tes armes, ton carquois et ton arc, va dans les champs, et chasse-moi du gibier.
\VS{4}Apprête-moi un mets comme j’aime, et apporte-le-moi, afin que je mange, et que mon âme te bénisse avant que je meure.
\VS{5}Or Rebecca écoutait pendant qu'Isaac parlait à Esaü, son fils. Esaü donc s'en alla dans les champs pour chasser du gibier et pour le rapporter.
\VS{6}Et Rebecca parla à Jacob, son fils, et lui dit : Voici, j'ai entendu parler ton père à Esaü, ton frère, disant :
\VS{7}Apporte-moi du gibier, et fais-moi un mets, afin que je le mange et je te bénirai devant Yahweh avant de mourir.
\VS{8}Maintenant donc, mon fils, obéis à ma parole, et fais ce que je vais te commander.
\VS{9}Va maintenant à la bergerie, et prends-moi là deux bons chevreaux parmi les chèvres, et j'en ferai un mets pour ton père comme il aime.
\VS{10}Et tu le porteras à ton père, afin qu'il le mange et qu'il te bénisse avant sa mort.
\VS{11}Jacob répondit à Rebecca sa mère : Voici, Esaü, mon frère, est un homme velu, et je suis un homme sans poil.
\VS{12}Peut-être que mon père me touchera-t-il, et il me regardera comme un homme qui a voulu le tromper, et j'attirerai sur moi sa malédiction et non pas sa bénédiction.
\VS{13}Sa mère lui dit : Mon fils, que la malédiction que tu crains retombe sur moi ! Obéis seulement à ma parole, et va me prendre ce que je t'ai dit.
\TextTitle{Déception d'Esaü\FTNTT{Hé. 12:16-17}}
\VS{14}Jacob alla les prendre et les apporta à sa mère ; et sa mère fit un mets comme son père aimait.
\VS{15}Puis Rebecca prit les plus précieux habits d'Esaü, son fils aîné, qu'elle avait dans la maison, et elle les fit mettre à Jacob, son fils cadet.
\VS{16}Elle couvrit ses mains et son cou, qui étaient sans poil, des peaux des chevreaux.
\VS{17}Puis elle mit entre les mains de son fils Jacob le mets et le pain qu'elle avait apprêtés.
\VS{18}Il vint vers son père, et lui dit : Mon père ! Il répondit : Me voici ; qui es-tu, mon fils ?
\VS{19}Jacob répondit à son père : Je suis Esaü, ton fils aîné ; j'ai fait ce que tu m’as dit. Lève-toi, je te prie, assieds-toi et mange de mon gibier, afin que ton âme me bénisse.
\VS{20}Isaac dit à son fils : Eh quoi ! Tu en as déjà trouvé, mon fils ! Et il dit : Yahweh ton Dieu l'a fait venir devant moi.
\VS{21}Isaac dit à Jacob : Approche-toi, je te prie, mon fils, et que je te touche, afin que je sache si tu es mon fils Esaü ou non.
\VS{22}Jacob donc s'approcha de son père Isaac, qui le toucha et dit : Cette voix est la voix de Jacob, mais ces mains sont les mains d'Esaü.
\VS{23}Et il ne le reconnut pas, car ses mains étaient velues comme les mains de son frère Esaü ; et il le bénit.
\VS{24}Il dit : C’est toi, mon fils Esaü ? Il répondit : Je le suis.
\VS{25}Isaac lui dit : Apporte-moi donc la viande, et que je mange du gibier de mon fils, afin que mon âme te bénisse. Jacob l'apporta, et Isaac mangea ; il lui apporta aussi du vin, et il but.
\VS{26}Puis Isaac, son père, lui dit : Approche-toi, je te prie, et embrasse-moi mon fils.
\VS{27}Jacob s'approcha et l’embrassa. Isaac sentit l'odeur de ses habits, et le bénit en disant : Voici l'odeur de mon fils, comme l'odeur d'un champ que Yahweh a béni.
\VS{28}Que Dieu te donne de la rosée du ciel, et de la graisse de la terre, du blé et du vin en abondance\FTNT{Hé. 11:20.} !
\VS{29}Que des peuples te servent, et que des nations se prosternent devant toi ! Sois le maître de tes frères, et que les fils de ta mère se prosternent devant toi ! Maudit soit quiconque te maudira, et béni soit quiconque te bénira.
\VS{30}Isaac avait fini de bénir Jacob, et Jacob avait à peine quitté son père Isaac, qu’Esaü, son frère, revint de la chasse.
\VS{31}Il apprêta aussi un mets, l’apporta à son père, et lui dit : Que mon père se lève et mange du gibier de son fils, afin que ton âme me bénisse.
\VS{32}Isaac, son père, lui dit : Qui es-tu ? Et il dit : Je suis ton fils, ton fils aîné, Esaü.
\VS{33}Isaac fut saisi d'une grande, d’une violente émotion, et dit : Qui est donc celui qui a chassé du gibier et me l’a apporté ? J'ai mangé de tout avant que tu ne viennes, et je l'ai béni. Aussi sera-t-il béni !
\VS{34}Dès qu'Esaü entendit les paroles de son père, il poussa de forts cris, pleins d’amertume, et il dit à son père : Bénis-moi aussi, bénis-moi, mon père !
\VS{35}Mais il dit : Ton frère est venu avec tromperie, et il a enlevé ta bénédiction.
\VS{36}Esaü dit : N'est-ce pas avec raison qu'on a appelé son nom Jacob ? Car il m'a déjà supplanté deux fois ; il m'a enlevé mon droit d'aînesse, et voici, maintenant il a enlevé ma bénédiction. Puis il dit : Ne m'as-tu point réservé de bénédiction ?
\VS{37}Isaac répondit à Esaü en disant : Voici, je l'ai établi ton maître, et lui ai donné tous ses frères pour serviteurs, et je l'ai pourvu de blé et de vin ; et que ferai-je maintenant pour toi, mon fils ?
\VS{38}Esaü dit à son père : N'as-tu qu'une bénédiction, mon père ? Bénis-moi aussi, bénis-moi, mon père ! Et Esaü éleva la voix et pleura\FTNT{Hé. 12:17.}.
\VS{39}Isaac, son père, répondit, et dit : Voici, ta demeure sera privée de la graisse de la terre, et de la rosée du ciel, d'en haut.
\VS{40}Tu vivras par ton épée, et tu seras asservi à ton frère ; mais il arrivera qu'étant devenu maître, tu briseras son joug de dessus ton cou.
\TextTitle{Fuite de Jacob chez Laban}
\VS{41}Esaü conçut de la haine contre Jacob, à cause de la bénédiction dont son père l'avait béni ; et Esaü dit en son cœur : Les jours du deuil de mon père approchent, et je tuerai Jacob, mon frère.
\VS{42}On rapporta à Rebecca les paroles d'Esaü, son fils aîné ; et elle fit alors appeler Jacob, son fils cadet, et lui dit : Voici, Esaü, ton frère, se console dans l'espérance qu'il a de te tuer.
\VS{43}Maintenant donc, mon fils, obéis à ma parole ! Lève-toi, et enfuis-toi à Charan, vers Laban, mon frère.
\VS{44}Et reste avec lui quelque temps, jusqu'à ce que la fureur de ton frère soit passée ;
\VS{45}jusqu’à ce que la colère de ton frère se détourne de toi, et qu'il oublie ce que tu lui as fait. Pourquoi serais-je privée de vous deux en un même jour ?
\VS{46}Rebecca dit à Isaac : Je suis dégoûtée de la vie, à cause de filles de Heth. Si Jacob prend une femme, comme celles-ci, parmi les filles de Heth, parmi les filles du pays, à quoi me sert la vie ?
\Chap{28}
\TextTitle{A Béthel, Yahweh confirme son alliance à Jacob}
\VerseOne{}Isaac donc appela Jacob, et le bénit, et lui donna cet ordre : Tu ne prendras point de femme parmi les filles de Canaan.
\VS{2}Lève-toi, va à Paddan-Aram, à la maison de Bethuel, père de ta mère, et prends-toi une femme de là, parmi les filles de Laban, frère de ta mère.
\VS{3}Que le Dieu Tout-Puissant te bénisse, te rende fécond et te multiplie, afin que tu deviennes une assemblée de peuples.
\VS{4}Qu’il te donne la bénédiction d'Abraham, à toi et à ta postérité avec toi, afin que tu obtiennes en héritage le pays où tu as été étranger, que Dieu a donné à Abraham.
\VS{5}Isaac donc fit partir Jacob, qui s'en alla à Paddan-Aram, vers Laban, fils de Bethuel, le Syrien, frère de Rebecca, mère de Jacob et d'Esaü.
\VS{6}Esaü vit qu'Isaac avait béni Jacob, et qu'il l'avait envoyé à Paddan-Aram afin qu'il prenne une femme de ce pays-là pour lui, et qu'il lui avait donné cet ordre, quand il le bénissait, disant : Ne prends point de femme parmi les filles de Canaan ;
\VS{7}il vit que Jacob avait obéi à son père et à sa mère, et qu’il était parti à Paddan-Aram.
\VS{8}Esaü comprit ainsi que les filles de Canaan déplaisaient à Isaac, son père.
\VS{9}Et Esaü s’en alla vers Ismaël. Il prit pour femme, outre ses autres femmes, Mahalath, fille d'Ismaël, fils d'Abraham, sœur de Nebajoth.
\VS{10}Jacob partit de Beer-Schéba et s'en alla à Charan.
\VS{11}Il arriva dans un lieu où il passa la nuit, parce que le soleil était couché. Il y prit donc une pierre\FTNT{1 Pi. 2:4. Voir  commentaire en Es. 8:13-15.}, et en fit son chevet, et il se coucha dans ce lieu-là.
\VS{12}Il eut un songe ; et voici, une échelle dressée sur la terre, dont le sommet touchait le ciel. Et voici, les anges de Dieu montaient et descendaient par cette échelle\FTNT{Jn. 1:51.}.
\VS{13}Et voici, Yahweh se tenait sur l'échelle, et il lui dit : Je suis Yahweh, le Dieu d'Abraham, ton père, et le Dieu d'Isaac ; je te donnerai à toi et à ta postérité, la terre sur laquelle tu es couché.
\VS{14}Ta postérité sera comme la poussière de la terre, et tu t'étendras à l'occident et à l'orient, au nord et au midi, et toutes les familles de la terre seront bénies en toi et en ta postérité.
\VS{15}Voici, je suis avec toi ; et je te garderai partout où tu iras ; et je te ramènerai dans ce pays ; car je ne t'abandonnerai point que je n'aie exécuté ce que je t'ai dit.
\VS{16}Et quand Jacob fut réveillé de son sommeil, il dit : Certainement, Yahweh est en ce lieu-ci, et moi, je ne le savais pas !
\VS{17}Il eut peur et dit : Que ce lieu-ci est effrayant ! C'est ici la maison de Dieu, et c'est ici la porte des cieux !
\VS{18}Et Jacob se leva de bon matin, prit la pierre dont il avait fait son chevet, il la dressa pour monument, et versa de l'huile sur son sommet.
\VS{19}Il donna à ce lieu le nom de Béthel ; mais auparavant la ville s'appelait Luz.
\VS{20}Jacob fit un vœu en disant : Si Dieu est avec moi, et s'il me garde pendant le voyage que je fais, s'il me donne du pain à manger, et des habits pour me vêtir,
\VS{21}et si je retourne en paix à la maison de mon père, certainement Yahweh sera mon Dieu.
\VS{22}Cette pierre que j'ai dressée pour monument sera la maison de Dieu ; et de tout ce que tu m'auras donné, je t'en donnerai entièrement la dîme\FTNT{Voir commentaire sur la dîme en No. 18:21 et Mal. 3:10.}.
\Chap{29}
\TextTitle{Jacob épouse Léa et Rachel chez Laban}
\VerseOne{}Jacob donc se mit en chemin, et s'en alla au pays des fils de l’orient.
\VS{2}Il regarda. Et voici, il y avait un puits dans un champ ; et voici il y avait à côté trois troupeaux de brebis couchées près du puits, car c’était à ce puits qu’on abreuvait les troupeaux.  Et il y avait une grosse pierre sur l'ouverture du puits.
\VS{3}Tous les troupeaux se rassemblaient là ; on roulait la pierre de dessus l'ouverture du puits, et on abreuvait les troupeaux ; et ensuite on remettait la pierre à sa place, sur l'ouverture du puits.
\VS{4}Jacob leur dit : Mes frères, d'où êtes-vous ? Ils répondirent : Nous sommes de Charan.
\VS{5}Il leur dit : Connaissez-vous Laban, fils de Nachor ? Ils répondirent : Nous le connaissons.
\VS{6}Il leur dit : Se porte-t-il bien ? Ils lui répondirent : Il se porte bien ; et voici Rachel, sa fille, qui vient avec le troupeau.
\VS{7}Il dit : Voici, il est encore grand jour, et il n'est pas temps de rassembler les troupeaux ; abreuvez les brebis, puis allez et faites-les paître.
\VS{8}Ils répondirent : Nous ne le pouvons pas, jusqu'à ce que tous les troupeaux soient rassemblés et qu'on ait ôté la pierre de dessus l'ouverture du puits, afin d'abreuver les troupeaux.
\VS{9}Comme il parlait encore avec eux, Rachel arriva avec le troupeau de son père ; car elle était bergère.
\VS{10}Lorsque Jacob vit Rachel, fille de Laban, frère de sa mère, et le troupeau de Laban, frère de sa mère, il s'approcha et roula la pierre de dessus l'ouverture du puits, et abreuva le troupeau de Laban, frère de sa mère.
\VS{11}Et Jacob embrassa Rachel, et il éleva sa voix et pleura.
\VS{12}Jacob apprit à Rachel qu'il était frère de son père, et qu'il était fils de Rebecca ; et elle courut le rapporter à son père.
\VS{13}Dès que Laban eut entendu parler de Jacob, fils de sa soeur, il courut au-devant de lui, il le prit dans ses bras et l’embrassa, et il le fit venir dans sa maison ; et Jacob raconta à Laban tout ce qui lui était arrivé.
\VS{14}Et Laban lui dit : Certainement, tu es mon os et ma chair. Jacob demeura un mois entier chez Laban.
\VS{15}Puis Laban dit à Jacob : Me serviras-tu pour rien parce que tu es mon frère ? Dis-moi quel sera ton salaire ?
\VS{16}Or Laban avait deux filles : L'aînée s'appelait Léa, et la cadette Rachel.
\VS{17}Léa avait les yeux délicats, mais Rachel était belle de taille et belle de figure.
\VS{18}Jacob aimait Rachel, et il dit : Je te servirai sept ans pour Rachel, ta cadette.
\VS{19}Et Laban répondit : Il vaut mieux que je te la donne que de la donner à un autre homme ; demeure avec moi.
\VS{20}Ainsi Jacob servit sept années pour Rachel ; et elles furent à ses yeux comme quelques jours, parce qu'il l'aimait.
\VS{21}Et Jacob dit à Laban : Donne-moi ma femme, car mon temps est accompli, et j’irai vers elle.
\VS{22}Laban réunit tous les gens du lieu et fit un festin.
\VS{23}Mais quand le soir fut venu, il prit Léa, sa fille, et l'amena vers Jacob qui s’approcha d’elle.
\VS{24}Et Laban donna Zilpa, sa servante, à Léa, sa fille, pour servante.
\VS{25}Le lendemain matin, voilà que c'était Léa. Alors Jacob dit à Laban : Qu'est-ce que tu m'as fait ? N'ai-je pas servi chez toi pour Rachel ? Et pourquoi m'as-tu trompé ?
\VS{26}Laban répondit : On ne fait pas ainsi dans ce lieu de donner la plus jeune avant l'aînée.
\VS{27}Achève la semaine avec celle-ci, et nous te donnerons aussi l'autre, pour le service que tu feras encore chez moi sept autres années.
\VS{28}Jacob donc fit ainsi, et il acheva la semaine avec Léa ; et Laban lui donna aussi pour femme Rachel, sa fille.
\VS{29}Et Laban donna Bilha, sa servante, à Rachel, sa fille, pour servante.
\VS{30}Jacob alla aussi vers Rachel, et il aima Rachel plus que Léa ; et il servit encore chez Laban sept autres années.
\VS{31}Yahweh vit que Léa était haïe, et il ouvrit sa matrice, tandis que Rachel était stérile.
\TextTitle{Les enfants de Jacob}
\VS{32}Léa conçut et enfanta un fils à qui elle donna le nom de Ruben, car elle dit : C'est parce que Yahweh a vu mon affliction, et maintenant mon mari m'aimera.
\VS{33}Elle conçut encore et enfanta un fils, et elle dit : Parce que Yahweh a entendu que j'étais haïe, il m'a aussi donné celui-ci. Et elle lui donna le nom de Siméon.
\VS{34}Elle conçut encore et enfanta un fils, et elle dit : Maintenant mon mari s'attachera à moi, car je lui ai enfanté trois fils. C'est pourquoi on lui donna le nom de Lévi.
\VS{35}Elle conçut encore et enfanta un fils, et elle dit : Cette fois je louerai Yahweh. C'est pourquoi elle lui donna le nom de Juda. Et elle cessa d'avoir des enfants.
\Chap{30}
\TextTitle{Les enfants de Jacob (suite)}
\VerseOne{}Alors Rachel, voyant qu’elle ne donnait point d'enfants à Jacob, fut jalouse de Léa, sa sœur, et elle dit à Jacob : Donne-moi des enfants, autrement je meurs !
\VS{2}La colère de Jacob s’enflamma contre Rachel, et il dit : Suis-je à la place de Dieu pour t’empêcher d'avoir des enfants ?
\VS{3}Elle dit : Voici ma servante Bilha ; va vers elle ; qu’elle enfante sur mes genoux, et que j’aie des fils par elle.
\VS{4}Et elle lui donna pour femme Bilha, sa servante, et Jacob alla vers elle.
\VS{5}Bilha conçut et enfanta un fils à Jacob.
\VS{6}Rachel dit : Dieu a jugé en ma faveur, et il a aussi exaucé ma voix, et m'a donné un fils ; c'est pourquoi elle l’appela du nom de Dan.
\VS{7}Bilha, servante de Rachel, conçut encore et enfanta un second fils à Jacob.
\VS{8}Rachel dit : J'ai fortement lutté contre ma sœur, aussi j'ai eu la victoire ; c'est pourquoi elle l’appela du nom de Nephthali.
\VS{9}Alors Léa, voyant qu'elle avait cessé de faire des enfants, prit Zilpa, sa servante, et la donna pour femme à Jacob.
\VS{10}Zilpa, servante de Léa, enfanta un fils à Jacob.
\VS{11}Léa dit : Le bonheur est arrivé, c'est pourquoi elle l’appela du nom de Gad.
\VS{12}Zilpa, servante de Léa, enfanta un second fils à Jacob.
\VS{13}Léa dit : C'est pour me rendre heureuse, car les filles me diront bienheureuse ; c'est pourquoi elle l’appela du nom d’Aser.
\VS{14}Ruben sortit au temps de la moisson des blés, trouva des mandragores\FTNT{La mandragore, appelée pomme d'amour, était utilisée comme excitant du désir sexuel ainsi que pour favoriser la procréation. On attribuait à cette plante aux propriétés hallucinogènes des vertus magiques.} aux champs, et les apporta à Léa, sa mère ; et Rachel dit à Léa : Donne-moi, je te prie, des mandragores de ton fils.
\VS{15}Elle lui répondit : Est-ce peu que tu aies pris mon mari, pour que tu prennes aussi les mandragores de mon fils ? Et Rachel dit : Qu'il couche donc cette nuit avec toi pour les mandragores de ton fils.
\VS{16}Le soir, comme Jacob revenait des champs, Léa sortit au-devant de lui et lui dit : Tu viendras vers moi, car je t'ai acheté pour les mandragores de mon fils ; et il coucha avec elle cette nuit-là.
\VS{17}Dieu exauça Léa, et elle conçut et enfanta à Jacob un cinquième fils.
\VS{18}Léa dit : Dieu m'a récompensée, parce que j'ai donné ma servante à mon mari ; c'est pourquoi elle l’appela du nom d’Issacar.
\VS{19}Léa conçut encore et enfanta un sixième fils à Jacob.
\VS{20}Léa dit : Dieu m'a donné un beau don ; maintenant mon mari habitera avec moi, car je lui ai enfanté six fils ; c'est pourquoi elle l’appela du nom de Zabulon.
\VS{21}Puis elle enfanta une fille et la nomma Dina.
\VS{22}Dieu se souvint de Rachel, il l’exauça et il ouvrit sa matrice.
\VS{23}Alors elle conçut et enfanta un fils, et elle dit : Dieu a ôté mon opprobre.
\VS{24}Et elle lui donna le nom de Joseph, en disant : Que Yahweh m'ajoute un autre fils !
\TextTitle{Jacob devient de plus en plus riche}
\VS{25}Lorsque Rachel eut enfanté Joseph, Jacob dit à Laban : Laisse-moi partir, pour que je m’en aille chez moi, dans mon pays.
\VS{26}Donne-moi mes femmes et mes enfants, pour lesquels je t'ai servi, et je m'en irai ; car tu sais de quelle manière je t'ai servi.
\VS{27}Laban lui répondit : Ecoute, je te prie, si j'ai trouvé grâce à tes yeux ; j’ai deviné que Yahweh m'a béni à cause de toi.
\VS{28}Il lui dit aussi : Fixe-moi le salaire que tu veux, et je te le donnerai.
\VS{29}Jacob lui répondit : Tu sais comment je t'ai servi et ce qu'est devenu ton bétail avec moi.
\VS{30}Car le peu que tu avais avant que je vienne s’est beaucoup accru, et Yahweh t'a béni depuis que j’ai mis mes pieds chez toi. Et maintenant, quand ferai-je aussi quelque chose pour ma maison ?
\VS{31}Laban lui dit : Que te donnerai-je ? Et Jacob répondit : Tu ne me donneras rien ; mais je ferai paître encore tes troupeaux, et je les garderai, si tu consens à ce que je vais te dire.
\VS{32}Je parcourrai aujourd'hui tes troupeaux, mets à part parmi toutes les brebis tachetées et marquetées, et tous les agneaux noirs, et les chèvres marquetées et tachetées. Ce sera mon salaire.
\VS{33}Ma justice me rendra témoignage à l’avenir devant toi ; quand tu viendras reconnaître mon salaire, en ta présence ; et tout ce qui ne sera pas marqueté ou tacheté parmi les chèvres, et noirs parmi les agneaux, sera considéré comme un vol s'il est trouvé chez moi.
\VS{34}Laban dit : Voici, qu'il te soit fait comme tu l'as dit.
\VS{35}Ce même jour, il sépara les boucs rayés et marquetés, et toutes les chèvres tachetées et marquetées, toutes celles où il y avait du blanc, et tous les agneaux noirs. Il les remit entre les mains de ses fils.
\VS{36}Puis il mit l'espace de trois journées de chemin entre lui et Jacob ; et Jacob fit paître le reste des troupeaux de Laban.
\VS{37}Mais Jacob prit des branches vertes de peuplier, d’amandier et de platane ; il y pela des bandes blanches,  mettant à nu le blanc qui était sur les branches.
\VS{38}Puis il plaça les branches qu’il avait pelées dans les auges, dans les abreuvoirs, sous les yeux des brebis qui venaient boire, et elles entraient en chaleur quand elles venaient boire.
\VS{39}Les brebis entraient en chaleur près des branches, et elles faisaient des brebis rayées, tachetées et marquetées.
\VS{40}Jacob séparait les agneaux, et il mettait ensemble ce qui était rayé et tout ce qui était noir dans les troupeaux de Laban. Il se fit ainsi des troupeaux à part, qu’il ne réunit point aux troupeaux de Laban.
\VS{41}Toutes les fois que les brebis vigoureuses entraient en chaleur, Jacob mettait les branches dans les auges sous les yeux des brebis, afin qu'elles entrent en chaleur près des  branches.
\VS{42}Mais pour les brebis chétives, il ne les mettait point ; de sorte que les chétives appartenaient à Laban et les vigoureuses à Jacob.
\VS{43}Ainsi cet homme devint de plus en plus riche ; il eut du menu bétail en abondance, des servantes et des serviteurs, des chameaux et des ânes.
\Chap{31}
\TextTitle{Yahweh demande à Jacob de rentrer dans la terre de se pères}
\VerseOne{}Or Jacob entendit les discours des fils de Laban qui disaient : Jacob a pris tout ce qui appartenait à notre père, et c’est de ce qui était à notre père qu’il s’est acquis toute cette richesse.
\VS{2}Jacob regarda le visage de Laban, et voici, il n'était plus à son égard comme auparavant.
\VS{3}Alors Yahweh dit à Jacob : Retourne au pays de tes pères et vers ta parenté, et je serai avec toi.
\VS{4}Jacob fit appeler Rachel et Léa qui étaient aux champs vers son troupeau,
\VS{5}et leur dit : Je vois au visage de votre père qu'il n'est plus envers moi comme il était auparavant ; toutefois le Dieu de mon père a été avec moi.
\VS{6}Vous savez que j'ai servi votre père de tout mon pouvoir.
\VS{7}Mais votre père s'est moqué de moi et a changé dix fois mon salaire ; mais Dieu ne lui a pas permis de me faire du mal.
\VS{8}Quand il disait : Les tachetées seront ton salaire, alors toutes les brebis faisaient des agneaux tachetés ; et quand il disait : Les marquetées seront ton salaire, alors toutes les brebis faisaient des agneaux marquetés.
\VS{9}Ainsi Dieu a ôté à votre père son bétail et me l'a donné.
\VS{10}Au temps où les brebis entraient en chaleur, je levai mes yeux et vis en songe que les boucs qui couvraient les brebis étaient rayés, tachetés, et marquetés.
\VS{11}Et l'Ange de Dieu\FTNT{Gn. 7:7} me dit en songe : Jacob ! Et je répondis : Me voici.
\VS{12}Il dit : Lève maintenant tes yeux et regarde : Tous les boucs qui couvrent les brebis sont rayés, tachetés et marquetés, car j'ai vu tout ce que te fait Laban.
\VS{13}Je suis le Dieu de Béthel, où tu oignis la pierre que tu dressas pour monument, où tu me fis un vœu.  Maintenant lève-toi, sors de ce pays, et retourne au pays de ta naissance.
\TextTitle{Jacob fuit de chez Laban avec sa famille}
\VS{14}Alors Rachel et Léa lui répondirent et dirent : Avons-nous encore quelque portion et quelque héritage dans la maison de notre père ?
\VS{15}Ne nous a-t-il pas traitées comme des étrangères ? Car il nous a vendues, et même il a entièrement mangé notre argent.
\VS{16}Car toutes les richesses que Dieu a ôtées à notre père nous appartenaient ainsi qu’à nos enfants. Maintenant donc fais tout ce que Dieu t'a dit.
\VS{17}Ainsi Jacob se leva, et fit monter ses enfants et ses femmes sur des chameaux.
\VS{18}Il emmena tout son bétail et tous les biens qu'il avait acquis, et tout ce qu'il possédait et qu'il avait acquis à Paddan-Aram, pour aller vers Isaac, son père, au pays de Canaan.
\VS{19}Or comme Laban était allé tondre ses brebis, Rachel déroba les théraphim de son père\FTNT{Les théraphim étaient des idoles utilisées dans un sanctuaire de maison ou dans un lieu de culte. Voir Jg. 18:14 ; 2 R. 23:24.}.
\VS{20}Et Jacob trompa Laban, le Syrien, en ne l’avertissant pas de son dessein, parce qu'il s'enfuyait.
\VS{21}Il s'enfuit avec tout ce qui lui appartenait ; il se leva, passa le fleuve, et se dirigea vers la montagne de Galaad.
\VS{22}Le troisième jour, on rapporta à Laban que Jacob s’était enfui.
\VS{23}Alors il prit avec lui ses frères, et il le poursuivit sept journées de marche, et l'atteignit à la montagne de Galaad.
\VS{24}Mais Dieu apparut à Laban, le Syrien, en songe la nuit, et lui dit : Garde-toi de parler à Jacob ni en bien ni en mal.
\VS{25}Laban donc atteignit Jacob. Jacob avait dressé ses tentes sur la montagne ; et Laban dressa aussi les siennes avec ses frères sur la montagne de Galaad.
\VS{26}Et Laban dit à Jacob : Qu'as-tu fait ? Tu m’as trompé, tu as emmené mes filles comme des prisonnières de guerre.
\VS{27}Pourquoi as-tu pris la fuite secrètement, m’as-tu trompé et ne m’as-tu pas averti ? Car je t'aurais laissé partir avec joie et avec des chansons, au son des tambours et des violons.
\VS{28}Tu ne m'as pas laissé embrasser mes fils et mes filles ! C’est en insensé que tu as agi.
\VS{29}J'ai en main le pouvoir de vous faire du mal, mais le Dieu de votre père m'a parlé la nuit passée et m'a dit : Garde-toi de ne parler à Jacob ni en bien ni en mal.
\VS{30}Maintenant que tu es parti, parce que tu  languissais après la maison de ton père, pourquoi as-tu dérobé mes dieux ?
\VS{31}Jacob répondit et dit à Laban : Je me suis enfui parce que je craignais ; car je me disais qu'il fallait prendre garde que tu ne me ravisses tes filles.
\VS{32}Mais celui chez qui tu trouveras tes dieux ne vivra point. En présence de nos frères, examine  s'il y a chez moi quelque chose qui t'appartienne, et prends-le ; car Jacob ignorait que Rachel les avait dérobés.
\VS{33}Alors Laban entra dans la tente de Jacob, et dans celle de Léa, et dans la tente des deux servantes, et il ne les trouva point ; et étant sorti de la tente de Léa, il entra dans la tente de Rachel.
\VS{34}Mais Rachel avait prit les théraphim et les avait mis dans le bât d'un chameau, et s’était assise dessus ; et Laban fouilla toute la tente et ne les trouva point.
\VS{35}Elle dit à son père : Que mon seigneur ne se fâche point de ce que je ne puis me lever devant lui, car j'ai ce que les femmes ont coutume d'avoir ; et il fouilla, mais il ne trouva point les théraphim.
\VS{36}Jacob se mit en colère et querella Laban. Il reprit la parole et lui dit : Quel est mon crime ? Quel est mon péché, pour que tu me poursuives avec tant d’ardeur ?
\VS{37}Car tu as fouillé tous mes effets, qu'as-tu trouvé des effets de ta maison ? Mets-les ici devant mes frères et les tiens, et qu'ils soient juges entre nous deux.
\VS{38}Voilà vingt ans que j’ai passés chez toi ; tes brebis et tes chèvres n'ont point avorté, je n'ai point mangé les moutons de tes troupeaux.
\VS{39}Je ne t'ai point rapporté de bêtes déchirées par les bêtes sauvages, j'en ai moi-même subi la perte ; et tu redemandais de ma main ce qui avait été dérobé de jour et ce qui avait été dérobé de nuit.
\VS{40}Le jour la chaleur me consumait, et la nuit le froid ; et le sommeil fuyait de mes yeux.
\VS{41}Voilà vingt ans que j’ai passés dans ta maison, quatorze ans pour tes deux filles, et six ans pour tes troupeaux, et tu m'as changé dix fois mon salaire.
\VS{42}Si je n’avais pas eu pour moi le Dieu de mon père, le Dieu d'Abraham, et celui que craint Isaac, certes tu m’aurais maintenant renvoyé à vide. Mais Dieu a regardé mon affliction et le travail de mes mains, et il t'a repris la nuit passée.
\VS{43}Laban répondit à Jacob et dit : Ces filles sont mes filles, et ces enfants sont mes enfants, et ces troupeaux sont mes troupeaux, et tout ce que tu vois est à moi ; et que ferais-je aujourd'hui à mes filles et aux enfants qu'elles ont enfantés ?
\VS{44}Maintenant donc, viens, faisons ensemble une alliance, et qu’elle serve de témoignage entre moi et toi.
\VS{45}Jacob prit une pierre et il la dressa pour monument.
\VS{46}Jacob dit à ses frères : Ramassez des pierres. Et ils prirent des pierres et ils en firent un monceau, et ils mangèrent là sur ce monceau.
\VS{47}Laban l'appela Jegar-Sahadutha, et Jacob l'appela Galed.
\VS{48}Et Laban dit : Ce monceau sera aujourd'hui témoin entre moi et toi ; c'est pourquoi il fut nommé Galed (poste d’observation).
\VS{49}Il fut aussi appelé Mitspa ; parce que Laban dit : Que Yahweh veille sur moi et sur toi, quand nous nous serons l'un et l'autre perdus de vue.
\VS{50}Si tu maltraites mes filles et si tu prends une autre femme que mes filles, ce n’est pas un homme qui sera témoin entre nous, prends-y garde ; c'est Dieu qui est témoin entre moi et toi.
\VS{51}Laban dit encore à Jacob : Regarde ce monceau, et considère le monument que j'ai dressé entre moi et toi.
\VS{52}Que ce monceau soit témoin et que ce monument soit témoin que je n’irai pas vers toi au-delà de ce monceau, et que tu ne viendras pas vers moi au-delà de ce monceau et de ce monument pour me faire du mal.
\VS{53}Que le Dieu d'Abraham et le Dieu de Nachor, le Dieu de leur père, juge entre nous ; mais Jacob jura par celui que craignait Isaac, son père.
\VS{54}Jacob offrit un sacrifice sur la montagne et invita ses frères pour manger du pain ; ils mangèrent donc du pain et passèrent la nuit sur la montagne.
\VS{55}Laban se leva de bon matin, embrassa ses fils et ses filles, et les bénit. Ensuite il s'en alla. Ainsi Laban retourna chez lui.
\Chap{32}
\TextTitle{Jacob devient Israël}
\VerseOne{}Et Jacob continua son chemin, et des anges de Dieu le rencontrèrent.
\VS{2}En les voyant, Jacob dit : C'est ici le camp de Dieu ! Et il donna à ce lieu le nom de Mahanaïm.
\VS{3}Jacob envoya devant lui des messagers vers Esaü, son frère, au pays de Séir, dans le territoire d'Edom.
\VS{4}Il leur donna cet ordre : Vous parlerez de cette manière à mon seigneur Esaü : Ainsi a dit ton serviteur Jacob : J'ai séjourné comme étranger chez Laban, et j’y ai habité jusqu'à présent ;
\VS{5}j’ai des bœufs, des ânes, des brebis, des serviteurs, et des servantes ; et j'envoie l’annoncer à mon seigneur, afin de trouver grâce à  tes yeux.
\VS{6}Et les messagers revinrent auprès de Jacob et lui dirent : Nous sommes allés vers ton frère Esaü, et il marche aussi à ta rencontre avec quatre cents hommes.
\VS{7}Alors Jacob fut très effrayé et rempli d’angoisse ; et il partagea le peuple qui était avec lui, et les brebis, et les boeufs, et les chameaux, en deux camps ; et  il dit :
\VS{8}Si Esaü attaque l'un des camps et le frappe, le camp qui restera pourra s’échapper.
\VS{9}Jacob dit aussi : Ô Dieu de mon père Abraham, Dieu de mon père Isaac, ô Yahweh qui m'as dit : Retourne dans ton pays, et vers ta parenté, et je te ferai du bien.
\VS{10}Je suis trop petit pour toutes les faveurs et pour toute la fidélité dont tu as usé envers ton serviteur ; car j'ai passé ce Jourdain avec mon bâton, et maintenant je forme deux camps.
\VS{11}Je te prie, délivre-moi de la main de mon frère Esaü ; car je crains qu'il ne vienne, et qu'il ne me frappe, et qu'il ne tue la mère avec les enfants.
\VS{12}Et toi, tu as dit : Certes, je te ferai du bien, et je rendrai ta postérité comme le sable de la mer, si abondant qu’on ne saurait le compter.
\VS{13}C’est dans ce lieu-là que Jacob passa la nuit.  Il prit de ce qu’il avait sous la main pour faire un présent à Esaü, son frère :
\VS{14}à savoir deux cents chèvres, vingt boucs, deux cents brebis et vingt béliers.
\VS{15}Trente femelles de chameaux qui allaitaient, et leurs petits ; quarante jeunes vaches, dix jeunes taureaux, vingt ânesses et dix ânes.
\VS{16}Il les mit entre les mains de ses serviteurs, chaque troupeau à part, et leur dit : Passez devant moi, et faites qu'il y ait un intervalle entre chaque troupeau.
\VS{17}Il donna cet ordre au premier, disant : Quand Esaü, mon frère, te rencontrera et te demandera, disant : A qui es-tu ? Et où vas-tu ? Et à qui sont ces choses qui sont devant toi ?
\VS{18}Alors tu diras : Je suis à ton serviteur Jacob ; c'est un présent qu'il envoie à mon seigneur Esaü ; et voici, il vient lui-même derrière nous.
\VS{19}Il donna le même ordre au deuxième, au troisième, et à tous ceux qui suivaient les troupeaux, disant : C’est ainsi que vous parlerez à mon seigneur Esaü, quand vous le rencontrerez.
\VS{20}Vous lui direz : Voici, ton serviteur Jacob vient aussi derrière nous. Car il se disait : J'apaiserai sa colère par ce présent qui va devant moi, et après cela, je verrai sa face ; peut-être qu'il me regardera favorablement.
\VS{21}Le présent passa devant lui ; mais il resta cette nuit-là dans le camp.
\VS{22}Il se leva cette nuit, et prit ses deux femmes, ses deux servantes, et ses onze enfants, et passa le gué de Jabbok.
\VS{23}Il les prit donc, et leur fit passer le torrent ; il fit aussi passer tout ce qu'il avait.
\VS{24}Jacob demeura seul. Alors un homme lutta avec lui jusqu'au lever de l’aurore.
\VS{25}Et quand cet homme vit qu'il ne pouvait pas le vaincre, il frappa à l'emboîture de la hanche de Jacob ; ainsi l'emboîture de l'os de la hanche de Jacob se démit pendant qu’il luttait avec lui.
\VS{26}Et cet homme lui dit : Laisse-moi, car l'aube du jour est levée. Mais il dit : Je ne te laisserai point que tu ne m'aies béni.
\VS{27}Cet homme lui dit : Quel est ton nom ? Il répondit : Jacob.
\VS{28}Alors il dit : Ton nom ne sera plus Jacob, mais tu seras appelé Israël ; car tu as été le vainqueur en luttant avec Dieu et avec les hommes, et tu as été le plus fort.
\VS{29}Jacob l’interrogea en disant : Je te prie, déclare-moi ton nom. Et il répondit : Pourquoi demandes-tu mon nom ? Et il le bénit là\FTNT{Jg. 13:18.}.
\VS{30}Jacob appela ce lieu du nom de Peniel ; car, dit-il, j’ai vu Dieu face à face, et mon âme a été délivrée.
\VS{31}Le soleil se levait lorsqu’il passa Peniel. Jacob boitait de la hanche.
\VS{32}C'est pourquoi, jusqu'à ce jour, les enfants d'Israël ne mangent point le tendon qui est à l’emboîture de la hanche ; parce que Dieu frappa Jacob à l'emboîture de la hanche, au tendon.
\Chap{33}
\TextTitle{Jacob demande pardon à son frère Esaü}
\VerseOne{}Et Jacob leva ses yeux et regarda ; et voici, Esaü arrivait avec quatre cents hommes. Et Jacob répartit les enfants entre Léa, Rachel, et les deux servantes.
\VS{2}Il plaça en tête les servantes avec leurs enfants ; Léa et ses enfants ensuite ; et Rachel et Joseph au dernier rang.
\VS{3}Quant à lui,  il passa devant eux et se prosterna à terre sept fois, jusqu'à ce qu'il soit près de son frère.
\VS{4}Esaü courut à sa rencontre ; il le prit dans ses bras, se jeta sur son cou, et l’embrassa. Et ils pleurèrent.
\VS{5}Esaü leva ses yeux, vit les femmes et les enfants, et dit : Qui sont ceux-là ? Sont-ils à toi ? Jacob lui répondit : Ce sont les enfants que Dieu, par sa grâce, a donnés à ton serviteur.
\VS{6}Les servantes s'approchèrent, elles et leurs enfants, et se prosternèrent.
\VS{7}Puis Léa aussi s'approcha avec ses enfants, et ils se prosternèrent, et ensuite Joseph et Rachel s'approchèrent et se prosternèrent aussi.
\VS{8}Esaü dit : Que veux-tu faire avec tout ce camp que j'ai rencontré ? Et Jacob répondit : C'est pour trouver grâce aux yeux de mon seigneur.
\VS{9}Esaü dit : Je suis dans l’abondance, mon frère ; garde ce qui est à toi.
\VS{10}Et Jacob répondit : Non, je te prie, si j'ai maintenant trouvé grâce à tes yeux, reçois ce présent de ma main ; parce que j'ai vu ta face comme si j'avais vu la face de Dieu, et parce que tu m’as accueilli favorablement.
\VS{11}Accepte, je te prie, mon présent qui t'a été offert ; car Dieu m’a comblé de grâce, et je ne manque de rien. Il le pressa tant qu'il le prit.
\VS{12}Esaü dit : Partons et marchons, et je marcherai devant toi.
\VS{13}Mais Jacob lui dit : Mon seigneur sait que ces enfants sont jeunes et que j’ai des brebis et des vaches qui allaitent ; si l’on forçait leur marche un seul jour, tout le troupeau mourra.
\VS{14}Je te prie que mon seigneur passe devant son serviteur, et je m’avancerai tout doucement, au pas de ce bétail qui est devant moi, et au pas de ces enfants, jusqu'à ce que j'arrive chez mon seigneur à Séir.
\VS{15}Esaü dit : Je te prie, je vais au moins laisser avec toi une partie de ce peuple qui est avec moi ; et il répondit : Pourquoi cela ? Je te prie que je trouve grâce aux yeux de mon seigneur.
\VS{16}Ainsi Esaü  retourna ce jour-là par son chemin à Séir.
\TextTitle{Jacob dresse un autel à El-Elohé-Israël (Dieu Fort, Dieu d'Israël)}
\VS{17}Jacob partit pour Succoth. Il bâtit une maison pour lui, et il fit des cabanes pour son bétail. C’est pourquoi il appela ce lieu du nom de Succoth.
\VS{18}A son retour de  Paddan-Aram, Jacob arriva sain et sauf à la ville de Sichem, dans le pays de Canaan, et il campa devant la ville.
\VS{19}Il acheta une portion du champ où il avait dressé sa tente de la main des fils de Hamor, père de Sichem, pour cent pièces d'argent.
\VS{20}Et là, il dressa un autel qu'il appela El-Elohé-Israël (le Dieu Fort, le Dieu d'Israël).
\Chap{34}
\TextTitle{Déshonneur de Dina et vengeance de ses frères}
\VerseOne{}Or Dina, la fille que Léa avait enfantée à Jacob, sortit pour voir les filles du pays.
\VS{2}Elle fut aperçue de Sichem, fils de Hamor, le Hévien, prince du pays. Il l'enleva et coucha avec elle, et la déshonora.
\VS{3}Son cœur fut attaché à Dina, fille de Jacob ; il aima la jeune fille et sut parler au cœur de la jeune fille.
\VS{4}Et Sichem parla à Hamor, son père, en disant : Prends-moi cette fille pour femme.
\VS{5}Jacob apprit qu'il avait déshonoré Dina, sa fille. Or ses fils étaient avec son bétail aux champs ; Jacob garda le silence jusqu’à leur retour.
\VS{6}Hamor, père de Sichem, sortit vers Jacob pour lui  parler.
\VS{7}Et les fils de Jacob revinrent des champs dès qu’ils apprirent ce qui était arrivé ; ces hommes furent dans  une grande douleur, et furent fort irrités de l'infamie que Sichem avait commise contre Israël, en couchant avec la fille de Jacob, ce qui ne devait point se faire.
\VS{8}Hamor leur parla en disant : L’âme de Sichem, mon fils, s’est attachée à votre fille ; donnez-la-lui je vous prie pour femme.
\VS{9}Alliez-vous avec nous, vous nous donnerez vos filles, et vous prendrez pour vous les nôtres.
\VS{10}Vous habiterez avec nous, et le pays sera à votre disposition ; restez pour y trafiquer et y acquérir des possessions.
\VS{11}Sichem dit aussi au père et aux frères de la fille : Que je trouve grâce à vos yeux, et je donnerai tout ce que vous me direz.
\VS{12}Exigez de moi une forte dot, et beaucoup de présents que vous voudrez, et je les donnerai comme vous me direz ; et donnez-moi la jeune fille pour femme.
\VS{13}Alors les fils de Jacob répondirent avec ruse à Sichem et à Hamor, son père ; ils parlèrent ainsi parce que Sichem avait déshonoré Dina, leur sœur.
\VS{14}Ils leur dirent : C’est une chose que nous ne pouvons pas faire, que de donner notre sœur à un homme incirconcis, car ce serait un opprobre pour nous.
\VS{15}Mais nous ne consentirons à ce que vous demandez que si vous deveniez semblables à nous en circoncisant tous les mâles qui sont parmi vous.
\VS{16}Alors nous vous donnerons nos filles, et nous prendrons vos filles pour nous, et nous habiterons avec vous, et nous ne serons qu'un seul peuple.
\VS{17}Mais si vous ne voulez pas nous écouter et vous circoncire, nous prendrons notre fille et nous nous en irons.
\VS{18}Leurs discours plurent à Hamor et à Sichem, fils d'Hamor.
\VS{19}Le jeune homme ne tarda point à faire ce qu'on lui avait proposé, car la fille de Jacob lui plaisait beaucoup ; et il était le plus considéré de tous ceux de la maison de son père.
\VS{20}Hamor et Sichem, son fils, se rendirent à la porte de leur ville et parlèrent aux gens de leur ville en leur disant :
\VS{21}Ces hommes sont paisibles à notre égard ; qu'ils habitent dans le pays et qu'ils y trafiquent ; car voici, le pays est assez vaste pour eux. Nous prendrons pour femmes leurs filles, et nous leur donnerons nos filles.
\VS{22}Mais ces hommes ne consentiront à habiter avec nous, pour former un seul peuple, que si tout mâle qui est parmi nous est circoncis, comme ils sont eux-mêmes circoncis.
\VS{23}Leur bétail, et leurs biens, et toutes leurs bêtes, ne seront-ils pas à nous ? Accordons-leur seulement cela, et qu'ils demeurent avec nous.
\VS{24}Tous ceux qui sortaient par la porte de leur ville obéirent à Hamor et à Sichem, son fils ; et tout mâle d'entre tous ceux qui sortaient par la porte de leur ville fut circoncis.
\VS{25}Le troisième jour, pendant qu’ils étaient souffrants, deux des fils de Jacob, Siméon et Lévi, frères de Dina, prirent leurs épées, entrèrent hardiment dans la ville et tuèrent tous les mâles.
\VS{26}Ils passèrent aussi au tranchant de l'épée Hamor et Sichem, son fils ; ils enlevèrent Dina de la maison de Sichem, et sortirent.
\VS{27}Les fils de Jacob se jetèrent sur les morts et pillèrent la ville, parce qu'on avait déshonoré leur sœur.
\VS{28}Ils prirent leurs troupeaux, leurs bœufs, leurs ânes, et ce qui était dans la ville et dans les champs ;
\VS{29}et toutes leurs richesses, leurs petits enfants, et ils emmenèrent prisonnières leurs femmes ; et ils les pillèrent avec tout ce qui était dans les maisons.
\VS{30}Alors Jacob dit à Siméon et Lévi : Vous m'avez troublé en me rendant odieux aux habitants du pays, aux Cananéens et aux Phérésiens, et je n'ai qu’un petit nombre d’hommes ; ils s'assembleront contre moi, et me frapperont, et me détruiront, moi et ma maison.
\VS{31}Ils répondirent : Doit-on traiter notre sœur comme une prostituée ?
\Chap{35}
\TextTitle{Jacob revient à Béthel pour adorer Yahweh}
\VerseOne{}Or Dieu dit à Jacob : Lève-toi, monte à Béthel, et demeures-y ; là, tu y dresseras un autel au Dieu qui t'apparut lorsque tu fuyais Esaü, ton frère.
\VS{2}Jacob dit à sa famille, et à tous ceux qui étaient avec lui : Otez les dieux des étrangers qui sont au milieu de vous, purifiez-vous, et changez de vêtements\FTNT{Jos. 24:23.}.
\VS{3}Levons-nous et montons à Béthel ; là je dresserai un autel au Dieu qui m'a exaucé dans le jour de ma détresse, et qui a été avec moi dans le chemin où j'ai marché.
\VS{4}Alors ils donnèrent à Jacob tous les dieux des étrangers qui étaient entre leurs mains, et les anneaux qui étaient à leurs oreilles, et il les cacha sous un térébinthe qui est près de Sichem.
\VS{5}Puis ils partirent. Et Dieu frappa de terreur les villes qui les entouraient, et l’on ne poursuivit point les fils de Jacob.
\VS{6}Ainsi Jacob, et tout le peuple qui était avec lui, arrivèrent à Luz, qui est Béthel, dans le pays de Canaan.
\VS{7}Il bâtit là un autel, et il appela ce lieu El-Béthel (le Dieu Puissant de Béthel) ; car c’est là que Dieu s’était révélé à lui lorsqu’il fuyait son frère.
\VS{8}Débora,  nourrice de Rebecca, mourut ; et elle fut ensevelie au-dessous de Béthel sous un chêne, auquel on donna le nom d’Allon-Bacuth (chêne des pleurs).
\VS{9}Dieu apparut encore à Jacob, après son retour de Paddan-Aram, et il le bénit\FTNT{Os. 12:5.}.
\VS{10}Dieu lui dit : Ton nom est Jacob, mais tu ne seras plus appelé Jacob, car ton nom sera Israël. Et il lui donna le nom d’Israël.
\VS{11}Dieu lui dit aussi : Je suis le Dieu Fort, Tout-Puissant. Sois fécond et multiplie : Une nation et une multitude de nations naîtront de toi, et des rois sortiront de tes reins.
\VS{12}Je te donnerai le pays que j'ai donné à Abraham et à Isaac, et je le donnerai à ta postérité après toi.
\VS{13}Dieu s’éleva au-dessus de lui dans le lieu où il lui avait parlé.
\VS{14}Et Jacob dressa un monument dans le lieu où Dieu lui avait parlé, à savoir un monument de pierre, et il fit dessus une aspersion et y versa de l'huile.
\VS{15}Jacob donna le nom de Béthel au lieu où Dieu lui avait parlé.
\VS{16}Puis ils partirent de Béthel, et il y avait encore une certaine distance jusqu’à Ephrata\FTNT{Ephrata : «~lieu de la fécondité~».} lorsque Rachel accoucha. Elle eut un accouchement difficile ;
\VS{17}et comme elle avait beaucoup de peine à accoucher, la sage-femme lui dit : Ne crains point, car tu as encore un fils.
\VS{18}Et comme elle rendait l'âme, car elle était mourante, elle lui donna le nom de Ben-Oni\FTNT{Ben-Oni : «~fils de ma douleur~».}, mais son père l’appela Benjamin\FTNT{Benjamin : «~fils de ma main droite~», «~fils de félicité~».}.
\VS{19}C'est ainsi que mourut Rachel, et elle fut ensevelie sur le chemin d'Ephrata, qui est Bethléhem.
\VS{20}Jacob dressa un monument sur son sépulcre. C'est le monument du sépulcre de Rachel qui subsiste encore aujourd'hui.
\VS{21}Puis Israël partit et dressa ses tentes au-delà de Migdal-Eder.
\VS{22}Pendant qu’Israël habitait dans ce pays, Ruben alla coucher avec Bilha, concubine de son père.  Et Israël l'apprit. Or Jacob avait douze fils.
\VS{23}Les fils de Léa étaient Ruben, premier-né de Jacob, Siméon, Lévi, Juda, Issacar, et Zabulon.
\VS{24}Les fils de Rachel : Joseph et Benjamin.
\VS{25}Les fils de Bilha, servante de Rachel : Dan et Nephthali.
\VS{26}Les fils de Zilpa, servante de Léa : Gad et Aser. Ce sont là les enfants de Jacob qui lui naquirent à Paddan-Aram.
\TextTitle{Jacob voit vers son père Isaac avant sa mort}
\VS{27}Jacob arriva auprès d’Isaac, son père, à la plaine de Mamré, à Kirjath-Arba, qui est Hébron, où Abraham et Isaac avaient séjourné comme étrangers.
\VS{28}Les jours d’Isaac furent de cent quatre-vingts ans.
\VS{29}Isaac expira et mourut, et fut recueilli auprès de son peuple, âgé et rassasié de jours ; et Esaü et Jacob ses fils l'ensevelirent.
\Chap{36}
\TextTitle{Postérité d'Esaü (Edom)}
\VerseOne{}Et voici la postérité d'Esaü, qui est Edom.
\VS{2}Esaü prit ses femmes parmi les filles de Canaan, à savoir Ada, fille d'Elon, le Héthien, Oholibama, fille d’Ana, petite-fille de Tsibeon, le Hévien.
\VS{3}Il prit aussi Basmath, fille d'Ismaël, sœur de Nebajoth.
\VS{4}Ada enfanta à Esaü Eliphaz ; et Basmath enfanta Réuel.
\VS{5}Et Oholibama enfanta Jéusch, Jaelam et Koré. Ce sont là les enfants d'Esaü qui lui naquirent dans le  pays de Canaan.
\VS{6}Esaü prit ses femmes, ses fils et ses filles, et toutes les personnes de sa maison, tous ses troupeaux, ses bêtes, et tout le bien qu'il avait acquis dans le pays de Canaan, et il s'en alla dans un autre pays, loin de Jacob, son frère.
\VS{7}Car leurs richesses étaient si grandes qu'ils n'auraient pas pu demeurer ensemble ; et le pays où ils séjournaient comme étrangers ne pouvait plus les contenir à cause de leurs troupeaux.
\VS{8}Ainsi Esaü habita dans la montagne de Séir ; Esaü est Edom.
\VS{9}Voici la postérité d'Esaü, père d'Edom, dans la montagne de Séir.
\VS{10}Voici les noms des fils d'Esaü : Eliphaz fils d’Ada, femme d'Esaü ; Réuel, fils de Basmath, femme d'Esaü.
\VS{11}Les fils d'Eliphaz furent : Théman, Omar, Tsepho, Gaetham et Kenaz.
\VS{12}Et Timna était la concubine d'Eliphaz, fils d'Esaü, et elle enfanta à Eliphaz Amalek. Ce sont là les fils d’Ada, femme d'Esaü.
\VS{13}Voici les fils de Réuel : Nahath, Zérach, Schamma et Mizza. Ce sont là les fils de Basmath, femme d'Esaü.
\VS{14}Voici les fils d'Oholibama, fille d’Ada, petite fille de Tsibeon, femme d'Esaü ; elle enfanta à Esaü Jéusch, Jaelam et Koré.
\VS{15}Voici les chefs des fils d'Esaü. Voici les fils d'Eliphaz, premier-né d'Esaü, le chef Théman, le chef Omar, le chef Tsepho, le chef Kenaz,
\VS{16}le chef Koré, le chef Gaetham, le chef Amalek. Ce sont là les chefs d'Eliphaz dans le  pays d'Edom. Ce sont les fils d’Ada.
\VS{17}Voici les fils de Réuel, fils d'Esaü : le chef Nahath, le chef Zérach, le chef Schamma, et le chef Mizza. Ce sont là les chefs sortis de Réuel, dans le pays d'Edom.  Ce sont là les fils de Basmath, femme d'Esaü.
\VS{18}Voici les fils d'Oholibama, femme d'Esaü : Le chef Jéusch, le chef Jaelam, le chef Koré. Ce sont là les chefs sortis d'Oholibama, fille d’Ana, femme d'Esaü.
\VS{19}Ce sont là les fils d'Esaü, qui est Edom, et ce sont là leurs chefs.
\VS{20}Voici les fils de Séir, le Horien, qui avaient habité dans le pays : Lothan, Schobal, Tsibeon, Ana,
\VS{21}Dischon, Etser, et Dischan. Ce sont là les chefs des Horiens, fils de Séir, dans le pays d'Edom.
\VS{22}Les fils de Lothan furent Hori et Héman.  Et Thimna était sœur de Lothan.
\VS{23}Voici les fils de Schobal : Alvan, Manahath, Ebal, Schepho et Onam.
\VS{24}Voici les fils de Tsibeon : Ajja et Ana. C’est cet Ana qui trouva les sources chaudes dans le désert, quand il faisait paître les ânes de Tsibeon, son père.
\VS{25}Voici les fils d’Ana : Dischon, et Oholibama, fille d’Ana.
\VS{26}Voici les fils de Dischon : Hemdan, Eschban, Jithran et Keran.
\VS{27}Voici les fils d'Etser : Bilhan, Zaavan et Akan.
\VS{28}Voci les fils de Dischan : Huts et Aran.
\VS{29}Voici les chefs des Horiens : Le chef Lothan, le chef Schobal, le chef Tsibeon, le chef Ana.
\VS{30}Le chef Dischon, le chef Etser, le chef Dischan. Ce sont là les chefs des Horiens, les chefs qu’ils établirent dans le pays de Séir.
\VS{31}Voici les rois qui ont régné dans le pays d'Edom, avant qu’un roi règne sur les enfants d'Israël.
\VS{32}Béla, fils de Béor, régna sur Edom, et le nom de sa ville était Dinhaba.
\VS{33}Béla mourut, et Jobab, fils de Zérach de Botsra, régna à sa place.
\VS{34}Jobab mourut, et Huscham, du pays des Thémanites, régna à sa place.
\VS{35}Huscham mourut, et Hadad, fils de Bédad, régna à sa place. C’est lui qui frappa Madian dans le territoire de Moab ; et le nom de sa ville était Avith.
\VS{36}Hadad mourut, et Samla, de Masréka, régna à sa place.
\VS{37}Samla mourut, et Saül de Réhoboth sur le fleuve, régna à sa place.
\VS{38}Saül mourut, et Baal-Hanan, fils d’Acbor, régna à sa place.
\VS{39}Baal-Hanan, fils de Hacbor mourut, et Hadar régna à sa place. Le nom de sa ville était Pau ; et le nom de sa femme Mehéthabeel, fille de Mathred, petite-fille de Mézahab.
\VS{40}Voici les noms des chefs d'Esaü selon leurs familles, selon leurs territoires, et d’après leurs noms : Le chef Thimna, le chef Alva, le chef Jétheth,
\VS{41}le chef Oholibama, le chef Ela, le chef Pinon,
\VS{42}Le chef Kenaz, le chef Théman, le chef Mibtsar,
\VS{43}le chef Magdiel, et le chef Iram. Ce sont là les chefs d'Edom, selon leurs habitations dans le pays qu’ils possédaient. C'est Esaü le père d'Edom.
\Chap{37}
\TextTitle{Jacob aime Joseph plus que ses autres fils}
\VerseOne{}Or Jacob demeura dans le pays de Canaan, pays où avait séjourné son père comme étranger.
\VS{2}Voici la postérité de Jacob. Joseph, âgé de dix-sept ans, faisait paître le troupeau avec ses frères ; et il était jeune garçon auprès des fils de Bilha et des fils de Zilpa, femmes de son père. Et Joseph rapportait à leur père leurs mauvais propos.
\VS{3}Or Israël aimait Joseph plus que tous ses autres fils, parce qu'il l'avait eu dans sa vieillesse, et il lui fit une tunique de plusieurs couleurs.
\VS{4}Ses frères voyant que leur père l'aimait plus qu'eux tous, le haïssaient et ne pouvaient lui parler paisiblement.
\VS{5}Joseph eut un songe et il raconta à ses frères ; et ils le haïrent encore davantage.
\VS{6}Il leur dit donc : Ecoutez, je vous prie, le songe que j'ai eu.
\VS{7}Voici, nous étions à lier des gerbes au milieu d'un champ ; et voici, ma gerbe se leva et se tint droite ; et voici, vos gerbes l’entourèrent et se prosternèrent devant elle.
\TextTitle{Joseph haï par ses frères}
\VS{8}Alors ses frères lui dirent : Régnerais-tu sur nous ? Et dominerais-tu sur nous ? Et ils le haïrent encore plus pour ses songes et pour ses paroles.
\VS{9}Il eut encore un autre songe, et il le raconta à ses frères, en disant : Voici, j'ai eu encore un songe ; et voici, le soleil, la lune et onze étoiles se prosternaient devant moi\FTNT{Ap. 12:1.}.
\VS{10}Il le raconta à son père et à ses frères. Son père le réprimanda et lui dit : Que veut dire ce songe que tu as eu ? Faut-il que nous venions moi, ta mère, et tes frères, nous prosterner à terre devant toi ?
\VS{11}Ses frères eurent de l'envie contre lui, mais son père garda ses discours\FTNT{Ac. 7:9.}.
\VS{12}Les frères de Joseph s'en allèrent paître les troupeaux de leur père à Sichem.
\VS{13}Israël dit à Joseph : Tes frères ne font-ils pas paître le troupeau à Sichem ? Viens, que je t'envoie vers eux ; et il lui répondit : Me voici.
\VS{14}Israël lui dit : Va maintenant, vois si tes frères se portent bien, et si le troupeau est en bon état, et rapporte-le-moi. Ainsi il l'envoya de la vallée d’Hébron, et il alla jusqu'à Sichem.
\VS{15}Un homme le rencontra, comme il errait dans les champs ; et cet homme le questionna et lui dit : Que cherches-tu ?
\VS{16}Joseph répondit : Je cherche mes frères ; je te prie, dis-moi où ils font paître leur troupeau.
\VS{17}Et l'homme dit : Ils sont partis d'ici, et je les ai entendus dire : Allons à Dothan. Joseph alla après ses frères et les trouva à Dothan.
\VS{18}Ils le virent de loin ; et avant qu'il soit près d’eux, ils complotèrent contre lui pour le tuer.
\VS{19}Ils se dirent l'un à l'autre : Voici ce maître songeur qui arrive.
\TextTitle{Joseph dans la citerne}
\VS{20}Venez maintenant, tuons-le, et jetons-le dans l’une de ces citernes ; et nous dirons qu'une bête féroce l'a dévoré, et nous verrons ce que deviendront ses songes.
\VS{21}Mais Ruben entendit cela et le délivra de leurs mains en disant : Ne lui ôtons point la vie.
\VS{22}Ruben leur dit encore : Ne répandez point le sang ; jetez-le dans cette citerne qui est au désert, mais ne mettez point la main sur lui. C'était pour le délivrer de leurs mains et le renvoyer à son père.
\VS{23}Lorsque Joseph fut arrivé auprès de ses frères, ils le dépouillèrent de sa tunique, de cette tunique de plusieurs couleurs qui était sur lui.
\VS{24}Ils le prirent et le jetèrent dans la citerne.  Cette citerne était vide, il n'y avait point d'eau.
\VS{25}Ensuite, ils s'assirent pour manger du pain ; et levant les yeux, ils virent une caravane d'Ismaélites qui passait et qui venait de Galaad ; et leurs chameaux étaient chargés d’aromates, du baume et de la myrrhe, qu’ils transportaient en Egypte.
\VS{26}Et Juda dit à ses frères : Que gagnerons-nous à tuer notre frère et à cacher son sang ?
\VS{27}Venez, vendons-le à ces Ismaélites, et ne mettons point notre main sur lui, car il est notre frère, notre chair ; et ses frères lui obéirent.
\TextTitle{Joseph vendu à des marchands et emmené en Egypte}
\VS{28}Et comme les marchands Madianites passaient, ils tirèrent et firent remonter Joseph de la citerne, et le vendirent pour vingt pièces d'argent aux Ismaélites, qui emmenèrent Joseph en Egypte\FTNT{Ps. 105:17.}.
\VS{29}Puis Ruben revint à la citerne, et voici, Joseph n'était plus dans la citerne. Alors il déchira ses vêtements.
\VS{30}Il retourna vers ses frères et leur dit : L'enfant n’y est plus ! Et moi ! Moi ! Où irai-je ?
\VS{31}Ils prirent la tunique de Joseph et tuèrent un bouc d'entre les chèvres, ils plongèrent la tunique dans le sang.
\VS{32}Puis ils envoyèrent et firent porter à leur père la tunique de plusieurs couleurs, en lui disant : Voici ce que nous avons trouvé ! Reconnais maintenant si c'est la tunique de ton fils ou non.
\VS{33}Jacob la reconnut, et dit : C'est la tunique de mon fils ! Une bête féroce l'a dévoré ! Certainement Joseph a été déchiré !
\VS{34}Et Jacob déchira ses vêtements, il mit un sac sur ses reins, et il porta le deuil de son fils durant plusieurs jours.
\VS{35}Tous ses fils et toutes ses filles vinrent pour le consoler, mais il rejeta toute consolation. Il disait : C’est en pleurant que je descendrai vers mon fils dans le scheol ! C'est ainsi que son père le pleurait.
\VS{36}Les Madianites le vendirent en Egypte à Potiphar, eunuque de Pharaon, chef des gardes.
\Chap{38}
\TextTitle{Péché de Juda}
\VerseOne{}Il arriva qu’en ce temps-là, Juda s’éloigna de ses frères et se retira vers un homme d’Adullam, nommé Hira.
\VS{2}Là, Juda vit la fille d'un Cananéen, nommé Schua, il la prit pour femme et alla vers elle.
\VS{3}Elle conçut et enfanta un fils qu’elle appela Er.
\VS{4}Elle conçut encore et enfanta un fils qu’elle appela Onan.
\VS{5}Elle enfanta de nouveau un fils qu’elle appela Schéla. Juda était à Czib quand elle l’enfanta.
\VS{6}Juda prit une femme pour Er, son premier-né, une femme nommée Tamar.
\VS{7}Mais Er le premier-né de Juda était méchant devant Yahweh, et Yahweh le fit mourir\FTNT{No. 26:19.}.
\VS{8}Alors Juda dit à Onan : Va vers la femme de ton frère, et prends-la pour femme, comme tu es son beau-frère, et suscite des enfants à ton frère\FTNT{Lé. 25:25  ; Lé. 25:48. Voir commentaire en Ru. 2:20.}.
\VS{9}Mais Onan, sachant que les enfants ne seraient pas à lui, se souillait à terre lorsqu’il allait vers la femme de son frère, afin de ne pas donner de postérité à son frère.
\VS{10}Ce qu'il faisait déplut à Yahweh, c'est pourquoi il le fit aussi mourir.
\VS{11}Et Juda dit à Tamar, sa belle-fille : Demeure veuve dans la maison de ton père, jusqu'à ce que Schéla, mon fils, soit grand ; car il dit : Il faut prendre garde qu'il ne meure comme ses frères. Ainsi Tamar s'en alla et demeura dans la maison de son père.
\VS{12}Et après plusieurs jours, la fille de Schua, femme de Juda, mourut ; lorsque Juda fut consolé, il monta vers ceux qui tondaient ses brebis à Thimna, avec Hira, l’Adullamite, son ami intime.
\VS{13}On en informa Tamar et on lui dit : Voici, ton beau-père monte à Thimna pour tondre ses brebis.
\VS{14}Alors elle ôta ses habits de veuve, se couvrit d'un voile, et s'enveloppa, et elle s’assit à l’entrée d’Enaïm, sur le chemin de Thimna ; car elle voyait que Schéla était devenu grand et qu’elle ne lui était point donnée pour femme.
\VS{15}Et quand Juda la vit, il s'imagina que c'était une prostituée, car elle avait couvert son visage.
\VS{16}Il l’aborda sur le chemin et lui dit : Permets, je te prie, que je vienne vers toi ; car il ne savait pas que c’était sa belle-fille. Et elle répondit : Que me donneras-tu pour venir vers moi ?
\VS{17}Il répondit : Je t'enverrai un chevreau d'entre les chèvres du troupeau. Elle répondit : Me donneras-tu un gage jusqu'à ce que tu l'envoies ?
\VS{18}Il répondit : Quel gage te donnerai-je ? Et elle répondit : Ton cachet, ton cordon, et ton bâton que tu as à la main. Et il les lui donna. Il alla vers elle, et elle devint enceinte de lui.
\VS{19}Puis elle se leva et s'en alla ; elle ôta son voile et remit ses habits de veuve.
\VS{20}Juda envoya un chevreau d'entre ses chèvres par son ami intime l’Adullamite, pour qu'il retire le gage de la main de la femme, mais il ne la trouva point.
\VS{21}Il interrogea les hommes du lieu où elle avait été, en disant : Où est cette prostituée qui était à Enaïm, sur le chemin ? Ils répondirent : Il n'y a point eu ici de prostituée.
\VS{22}Il retourna auprès de Juda et lui dit : Je ne l'ai point trouvée ; et même les gens du lieu m'ont dit : Il n'y a point eu ici de prostituée.
\VS{23}Juda dit : Qu'elle garde le gage, il ne faut pas nous faire mépriser. Voici, j'ai envoyé ce chevreau, mais tu ne l'as point trouvée.
\VS{24}Environ trois mois après, on fit un rapport à Juda, en disant : Tamar, ta belle-fille, a commis un adultère, et voici elle est même enceinte. Et Juda dit : Faites-la sortir, et qu'elle soit brûlée.
\VS{25}Comme on la faisait sortir, elle envoya dire à son beau-père : Je suis enceinte de l'homme à qui ces choses appartiennent. Elle dit aussi : Reconnais, je te prie, à qui est ce cachet, ce cordon, et ce bâton.
\VS{26}Alors Juda les reconnut et il dit : Elle est plus juste que moi, parce que je ne l'ai point donnée à Schéla, mon fils ; et il ne la connut plus.
\VS{27}Quand elle fut au moment d'accoucher, voici, des jumeaux étaient dans son ventre.
\VS{28}Et pendant qu’elle accouchait, il y en eut un qui présenta la main ; la sage-femme la prit et y attacha un fil cramoisi, en disant : Celui-ci sort le premier.
\VS{29}Mais il retira la main, et son frère sortit. Alors la sage-femme dit : Quelle brèche tu as faite ! Et elle lui donna le nom de Pérets.
\VS{30}Ensuite sortit son frère, qui avait à la main le fil cramoisi ; et on lui donna le nom de Zérach.
\Chap{39}
\TextTitle{Joseph fidèle à Yahweh devant la tentation}
\VerseOne{}Or, quand on fit descendre Joseph en Egypte, Potiphar, eunuque de Pharaon, chef des gardes, Egyptien, l'acheta de la main des Ismaélites qui l'y avaient amené.
\VS{2}Yahweh était avec Joseph ; et il prospéra, et demeura dans la maison de son maître,  l’Egyptien.
\VS{3}Son maître vit que Yahweh était avec lui, et que Yahweh faisait prospérer entre ses mains tout ce qu'il faisait.
\VS{4}C'est pourquoi Joseph trouva grâce aux yeux de son maître, qui l’employa à son service. Et son maître l'établit sur sa maison, et lui remit entre les mains tout ce qui lui appartenait.
\VS{5}Dès que Potiphar l’eut établi sur sa maison et sur tout ce qu’il possédait, Yahweh bénit la maison de l’Egyptien, à cause de Joseph ; et la bénédiction de Yahweh fut sur tout ce qui lui appartenait, soit à la maison, soit aux champs.
\VS{6}Il abandonna aux mains de Joseph tout ce qui lui appartenait, et il n’avait avec lui d’autre soin que celui de prendre sa nourriture. Or Joseph était beau de  taille et beau de figure.
\VS{7}Après ces choses, il arriva que la femme de son maître porta les yeux sur Joseph, et elle lui dit : Couche avec moi\FTNT{Pr. 7:9-13.} !
\VS{8}Mais il le refusa, et dit à la femme de son maître : Voici, mon maître ne prend avec moi connaissance de rien dans la maison, et il a remis entre mes mains tout ce qui lui appartient.
\VS{9}Il n'y a personne dans cette maison qui soit plus grand que moi, et il ne m'a rien interdit excepté toi, parce que tu es sa femme ; et comment ferais-je un si grand mal et pécherais-je contre Dieu ?
\VS{10}Quoiqu’elle parlât tous les jours à Joseph, il refusa de coucher auprès d’elle, d’être avec elle.
\VS{11}Un jour qu'il était entré dans la maison pour faire son ouvrage, et qu'il n'y avait là aucun des gens dans la maison,
\VS{12}elle le saisit par son vêtement et lui dit : Couche avec moi ! Mais il laissa son vêtement entre ses mains, s'enfuit, et sortit dehors\FTNT{1 Co. 6:18.}.
\TextTitle{Fausse accusation contre Joseph}
\VS{13}Et lorsqu'elle vit qu'il lui avait laissé son vêtement entre les mains, et qu'il s'était enfui dehors,
\VS{14}elle appela les gens de sa maison, et leur parla en disant : Voyez, on nous a amené un Hébreu pour se moquer de nous.  Cet homme est venu vers moi pour coucher avec moi ; mais j'ai crié à haute voix.
\VS{15}Et dès qu’il a entendu que j’élevais la voix et que je  criais, il a laissé son vêtement à côté de moi, et s’est enfui dehors.
\VS{16}Et elle garda le vêtement de Joseph jusqu'à ce que son maître rentre à la maison.
\VS{17}Alors elle lui parla en ces mêmes termes et dit : Le serviteur Hébreu que tu nous as amené est venu vers moi pour se moquer de moi.
\VS{18}Mais comme j'ai élevé ma voix et que j'ai crié, il a laissé son vêtement à côté de moi et s'est enfui.
\VS{19}Et dès que le maître de Joseph eut entendu les paroles de sa femme qui lui disait : Ton serviteur m'a fait ce que je t'ai dit, sa colère s'enflamma.
\VS{20}Et le maître de Joseph le prit et le mit dans une étroite prison ; dans l'endroit où les prisonniers du roi étaient enfermés, et il fut là en prison.
\VS{21}Mais Yahweh fut avec Joseph ; il étendit sa bonté sur lui et lui fit trouver grâce auprès du chef de la prison.
\VS{22}Et le chef de la prison mit entre les mains de Joseph tous les prisonniers qui étaient dans la prison, et tout ce qu'il y avait à faire, il le faisait.
\VS{23}Le chef de la prison ne prenait aucune connaissance de ce que Joseph avait en main, parce que Yahweh était avec lui. Et Yahweh faisait prospérer tout ce qu'il faisait.
\Chap{40}
\TextTitle{Joseph demeure en prison}
\VerseOne{}Après ces choses, il arriva que l'échanson et le panetier du roi d'Egypte offensèrent leur maître, le roi d'Egypte.
\VS{2}Pharaon fut fort irrité contre ces deux eunuques, contre le chef des échansons, et contre le chef des panetiers.
\VS{3}Et il les fit mettre dans la maison du chef des gardes, dans la prison étroite, dans le même lieu où Joseph était enfermé.
\VS{4}Le chef des gardes les mit entre les mains de Joseph qui les servait ; et ils furent quelques jours en prison.
\VS{5}Pendant une même nuit, l’échanson et le panetier du roi d’Egypte, qui étaient enfermés dans la prison, eurent tous les deux un songe, chacun le sien, pouvant recevoir une explication distincte.
\VS{6}Joseph, étant venu le matin vers eux, les regarda ; et voici, ils étaient fort tristes.
\VS{7}Et il interrogea ces deux eunuques de Pharaon, qui étaient avec lui dans la prison de son maître, et leur dit : Pourquoi avez-vous mauvais visage aujourd'hui ?
\VS{8}Ils lui répondirent : Nous avons eu des songes, et il n'y a personne qui les interprète. Et Joseph leur dit : Les interprétations n’appartiennent-elles pas à Dieu ? Je vous prie, racontez-moi vos songes\FTNT{1 Co. 12:8-10 ; Job. 33:15.}.
\VS{9}Le chef des échansons raconta son songe à Joseph et lui dit : Dans mon songe, voici, il y avait un cep devant moi.
\VS{10}Ce cep avait trois sarments. Quand il eut poussé, sa fleur se développa et ses grappes donnèrent des raisins mûrs.
\VS{11}La coupe de Pharaon était dans ma main. Je pris les raisins, je les pressai dans la coupe de Pharaon, et je mis la coupe dans la main de Pharaon.
\VS{12}Joseph lui dit : Voici son interprétation : Les trois sarments sont trois jours.
\VS{13}Dans trois jours Pharaon élèvera ta tête et te rétablira dans ta charge, et tu mettras la coupe dans sa main, comme tu le faisais auparavant, lorsque tu étais son échanson.
\VS{14}Mais souviens-toi de moi quand tu  seras heureux, et use de bonté envers moi je te prie ; fais mention de moi à Pharaon, afin  qu’il me fasse sortir de cette maison.
\VS{15}Car certainement j'ai été enlevé du pays des Hébreux ; ici non plus je n’ai rien fait  pour  être mis en prison.
\VS{16}Le chef des panetiers, voyant que Joseph avait interprété favorablement ce songe, lui dit : Voici, il y avait aussi dans mon songe trois corbeilles de pain blanc sur ma tête.
\VS{17}Dans la corbeille la plus élevée, il y avait pour Pharaon des mets de toute espèce, cuits au four ; et les oiseaux les mangeaient dans la corbeille au-dessus de ma tête.
\VS{18}Joseph répondit et dit : Voici son interprétation : Les trois corbeilles sont trois jours.
\VS{19}Dans trois jours Pharaon enlèvera ta tête de dessus toi et te fera pendre à un bois, et les oiseaux mangeront ta chair sur toi.
\VS{20}Le troisième jour, jour de la naissance de Pharaon, il fit un festin à tous ses serviteurs ; et il éleva la tête du chef des échansons et la tête du chef des panetiers, au milieu de ses serviteurs.
\VS{21}Il rétablit le chef des échansons dans sa charge d’échanson, pour qu’il mette la coupe dans la main de Pharaon.
\VS{22}Mais il fit pendre le chef des panetiers, selon l’explication que Joseph leur avait donnée.
\VS{23}Cependant, le chef des échansons ne pensa plus à Joseph. Il l’oublia.
\Chap{41}
\TextTitle{Les songes de Pharaon}
\VerseOne{}Mais il arriva qu’au bout de deux ans entiers, Pharaon eut un songe. Et il lui semblait qu'il était près du fleuve.
\VS{2}Et voici, sept jeunes vaches belles à voir, grasses de chair, montèrent hors du fleuve et se mirent à paître dans les  prairies.
\VS{3}Et voici sept autres jeunes vaches, laides à voir, et maigres de chair, montèrent hors du fleuve derrière les autres et se tinrent auprès des autres jeunes vaches sur le bord du fleuve.
\VS{4}Les jeunes vaches laides à voir, et maigres, mangèrent les sept jeunes vaches belles à voir, et grasses. Alors Pharaon s'éveilla.
\VS{5}Il se rendormit et il eut un second songe. Voici, sept épis gras et beaux montèrent sur une même tige.
\VS{6}Et sept épis maigres et brûlés par le vent d’orient poussèrent après eux.
\VS{7}Les épis maigres engloutirent les sept épis gras et pleins. Et Pharaon s'éveilla ; et voilà le songe.
\VS{8}Le matin, Pharaon eut l’esprit troublé, et il envoya appeler tous les magiciens et tous les sages d'Egypte, et leur raconta ses songes.  Mais personne ne put les interpréter à Pharaon.
\VS{9}Alors le chef des échansons parla à Pharaon en disant : Je rappellerai aujourd'hui le souvenir de mes fautes.
\VS{10}Lorsque Pharaon fut irrité contre ses serviteurs, et nous fit mettre, le chef des panetiers et moi, en prison, dans la maison du chef des gardes,
\VS{11}nous eûmes l’un et l’autre un songe dans une même nuit ; et chacun de nous reçut une interprétation en rapport avec le songe qu’il avait eu.
\VS{12}Il y avait là avec nous un garçon Hébreu, esclave du chef des gardes. Nous lui racontâmes nos songes, et il nous les expliqua.
\VS{13}Les choses sont arrivées comme il nous les avait interprétées ; car le roi me rétablit dans ma charge et fit pendre le chef des panetiers.
\TextTitle{Joseph sort de prison et est établi sur l'Egypte par Pharaon}
\VS{14}Alors Pharaon envoya appeler Joseph.  On le fit sortir en hâte de la prison ; on le rasa, et on lui fit changer de vêtements ; puis il se rendit vers Pharaon.
\VS{15}Pharaon dit à Joseph : J'ai eu un songe, et personne ne peut  l'expliquer ; or j'ai appris que tu sais expliquer les songes.
\VS{16}Joseph répondit à Pharaon en disant : Ce n’est pas moi !  C’est Dieu qui donnera une réponse concernant la paix de Pharaon.
\VS{17}Pharaon dit alors à Joseph : Dans mon songe, voici, je me tenais sur le bord du fleuve.
\VS{18}Et voici, sept vaches grasses de chair et belles d’apparence montèrent hors du fleuve et se mirent à paître dans la prairie.
\VS{19}Sept autres vaches montèrent derrière elles, maigres, fort laides d’apparence, et décharnées ; je n’en ai point vu d’aussi laides dans tout le pays d’Egypte.
\VS{20}Les vaches décharnées et laides mangèrent les sept premières vaches qui étaient grasses ;
\VS{21}elles les engloutirent dans leur ventre, sans qu’on s’aperçoive qu’elles y étaient entrées ; et leur apparence était laide comme auparavant. Et je m’éveillai.
\VS{22}Je vis encore en songe sept épis pleins et beaux, qui montèrent sur une même tige.
\VS{23}Et sept épis vides, maigres, brûlés par le vent d’orient, poussèrent après eux.
\VS{24}Les épis maigres engloutirent les sept beaux épis. Je l’ai dit aux magiciens, mais personne ne m’a donné l’explication. 25 Joseph dit à Pharaon : Ce qu’a rêvé Pharaon est une seule chose ; Dieu a fait connaître à Pharaon ce qu’il va faire.
\VS{26}Les sept vaches belles sont sept années ; et les sept épis beaux sont sept années ; c’est un seul songe.
\VS{27}Les sept vaches décharnées et laides, qui montaient derrière les premières, sont sept années ; et les sept épis vides, brûlés par le vent d’orient, seront sept années de famine.
\VS{28}Ainsi, comme je viens de le dire à Pharaon, Dieu a fait connaître à Pharaon ce qu’il va faire.
\VS{29}Voici, il y aura sept années de grande abondance dans tout le pays d’Egypte.
\VS{30}Sept années de famine viendront après elles ; et l’on oubliera toute cette abondance au pays d’Egypte, et la famine consumera le pays.
\VS{31}Cette famine qui suivra sera si forte qu’on ne s’apercevra plus de l’abondance dans le pays.
\VS{32}Si Pharaon a vu le songe se répéter une seconde fois, c’est que la chose est arrêtée de la part de Dieu, et que Dieu se hâtera de l’exécuter.
\VS{33}Maintenant que Pharaon choisisse un homme intelligent et sage, et qu'il l'établisse sur le pays d'Egypte.
\VS{34}Que Pharaon établisse et institue des commissaires sur le pays, et qu'ils prennent la cinquième partie du revenu du pays d'Egypte durant les sept années d'abondance.
\VS{35}Qu’ils rassemblent tous les produits de ces bonnes années  qui viennent ; qu’ils fassent sous l’autorité de Pharaon des amas de blé, des approvisionnements dans les villes, et qu’ils en aient la garde.
\VS{36}Ces provisions seront en réserve pour le pays durant les sept années de famine qui seront dans le  pays d'Egypte, afin que le pays ne soit pas consumé par la famine.
\VS{37}Ces paroles plurent à Pharaon et à tous ses serviteurs\FTNT{Ac. 7:10.}.
\VS{38}Et Pharaon dit à ses serviteurs : Trouverions-nous un homme semblable à celui-ci, qui a l'Esprit de Dieu ?
\VS{39}Et Pharaon dit à Joseph : Puisque Dieu t'a fait connaître toutes ces choses, il n'y a personne qui soit aussi intelligent et aussi sage que toi.
\VS{40}C’est toi qui seras sur ma maison, et tout mon peuple obéira à tes ordres ; je serai seulement plus grand que toi par le trône.
\VS{41}Pharaon dit encore à Joseph : Regarde, je t’établis sur tout le pays d'Egypte.
\VS{42}Alors Pharaon ôta son anneau de sa main et le mit à la main de Joseph ; il le fit revêtir d'habits de fin lin et lui mit un collier d'or au cou.
\VS{43}Il le fit monter sur le char qui suivait le sien, et on criait devant lui : A genoux ! Et il l'établit sur tout le pays d'Egypte.
\VS{44}Et Pharaon dit à Joseph : Je suis Pharaon ! Et sans toi nul ne lèvera la main ni le pied dans tout le pays d'Egypte.
\TextTitle{Joseph épouse une égyptienne}
\VS{45}Pharaon appela Joseph du nom de Tsaphnath-Paenéach ; et il lui donna pour femme Asnath, fille de Poti-Phéra, prêtre d'On. Et Joseph alla visiter le pays d'Egypte.
\VS{46}Joseph était âgé de trente ans lorsqu’il se présenta devant Pharaon, roi d'Egypte ; et il quitta Pharaon et parcourut tout le pays d'Egypte.
\VS{47}Et la terre rapporta très abondamment pendant les sept années de fertilité.
\VS{48}Joseph rassembla tous les produits de ces sept années dans le pays d’Egypte ; il fit des approvisionnements dans les villes, mettant dans l’intérieur de chaque ville les productions des champs d’alentour.
\VS{49}Ainsi Joseph amassa une grande quantité de blé, comme le sable de la mer ; tellement qu'on cessa de le compter, parce qu’il n’y avait plus de nombre.
\VS{50}Avant les années de famine, il naquit à Joseph deux fils, que lui enfanta Asnath, fille de Poti-Phéra, prêtre  d'On.
\VS{51}Joseph donna au premier-né le nom de Manassé, parce que, dit-il, Dieu m'a fait oublier toute ma peine et toute la maison de mon père.
\VS{52}Et il donna au second le nom d’Ephraïm, parce que, dit-il, Dieu m'a fait fructifier dans le  pays de mon affliction.
\VS{53}Alors finirent les sept années de l'abondance qui avaient été dans le pays d'Egypte.
\VS{54}Et les sept années de la famine commencèrent à venir comme Joseph l'avait prédit. Et la famine fut dans tous les pays ; mais il y avait du pain dans tout le pays d'Egypte.
\VS{55}Ensuite tout le pays d'Egypte fut affamé, et le peuple cria à Pharaon pour avoir du pain. Et Pharaon répondit à tous les Egyptiens : Allez vers Joseph, et faites ce qu'il vous dira.
\VS{56}La famine régnait dans tout le pays. Joseph ouvrit tous les lieux d’approvisionnements et vendit du blé aux Egyptiens. La famine augmentait dans le pays d’Egypte.
\VS{57}On venait de tous les pays jusqu’en Egypte, pour acheter du blé  auprès de Joseph ; car la famine était fort grande sur toute la terre.
\Chap{42}
\TextTitle{Les frères de Joseph viennent acheter des vivres en Egypte}
\VerseOne{}Et Jacob, voyant qu'il y avait du blé à vendre en Egypte, dit à ses fils : Pourquoi vous regardez-vous les uns les autres ?
\VS{2}Il leur dit aussi : Voici, j'ai appris qu'il y a du blé à vendre en Egypte, descendez-y pour nous en acheter là, afin que nous vivions, et que nous ne mourions point.
\VS{3}Alors les frères de Joseph descendirent pour acheter du blé en Egypte.
\VS{4}Mais Jacob n'envoya point Benjamin, frère de Joseph, avec ses frères ; car il disait : Il faut prendre garde qu’un malheur ne lui arrive.
\VS{5}Ainsi les fils d'Israël allèrent en Egypte pour acheter du blé avec ceux qui y allaient, car la famine était dans le pays de Canaan.
\TextTitle{Joseph met ses frères à l'épreuve}
\VS{6}Joseph commandait dans le pays, et c’était lui qui vendait le blé à tous les peuples de la terre. Les frères de Joseph vinrent et se prosternèrent devant lui la face contre terre.
\VS{7}Joseph vit ses frères et les reconnut ; mais il feignit d’être un étranger pour eux, et il leur parla rudement, en leur disant : D'où venez-vous ? Et ils répondirent : Du pays de Canaan, pour acheter des vivres.
\VS{8}Joseph reconnut ses frères, mais eux ne le connurent point.
\VS{9}Alors Joseph se souvint des songes qu'il avait eus à leur sujet et leur dit : Vous êtes des espions, vous êtes venus pour observer les lieux faibles du pays.
\VS{10}Et ils lui répondirent : Non, mon seigneur, mais tes serviteurs sont venus pour acheter des vivres.
\VS{11}Nous sommes tous enfants d'un même homme, nous sommes des gens de bien ; tes serviteurs ne sont pas des espions.
\VS{12}Et il leur dit : Nullement ; vous êtes venus pour observer les lieux faibles du pays.
\VS{13}Et ils répondirent : Nous, tes serviteurs, étions douze frères, fils d'un même homme, dans le pays de Canaan. Et voici, le plus jeune est aujourd'hui avec notre père, et l'un n'est plus.
\VS{14}Joseph leur dit : C'est ce que je vous disais, vous êtes des espions.
\VS{15}Voici comment vous serez éprouvés : Par la vie de Pharaon ! Vous ne sortirez pas d'ici que votre jeune frère ne soit venu ici.
\VS{16}Envoyez l’un de vous et qu’il amène votre frère ; et  vous, restez prisonniers. Vos paroles seront éprouvées et je saurai si vous avez dit la vérité. Autrement, par la vie de Pharaon ! Vous êtes des espions.
\VS{17}Et il les mit tous ensemble en prison pendant trois jours.
\VS{18}Le troisième jour, Joseph leur dit : Faites ceci, et vous vivrez. Je crains Dieu !
\VS{19}Si vous êtes sincères, que l'un de vos frères reste enfermé dans votre prison ; et vous, partez et emportez du blé pour nourrir vos familles.
\VS{20}Puis amenez-moi votre jeune frère afin que vos paroles soient éprouvées, et vous ne mourrez point ; et ils firent ainsi.
\TextTitle{Siméon gardé en Egypte en attendant que Benjamin soit présenté à Joseph}
\VS{21}Et ils se dirent alors l'un à l'autre : Nous sommes certainement coupables à l'égard de notre frère ; car nous avons vu l'angoisse de son âme quand il nous demandait grâce, et nous ne l'avons point écouté ; c'est pour cela que cette détresse nous est arrivée.
\VS{22}Ruben leur répondit en disant : Ne vous disais-je pas : Ne commettez point ce péché contre l'enfant ? Et vous ne m’avez point écouté ; et voici que son sang vous est redemandé.
\VS{23}Ils ne savaient pas que Joseph les comprenait, parce qu'il se servait d’un interprète pour leur parler.
\VS{24}Il s’éloigna d’eux pour pleurer. Et il revint, leur parla ; puis il prit parmi eux Siméon, et le fit enchaîner sous leurs yeux.
\VS{25}Et Joseph ordonna qu'on remplisse leurs sacs de blé, et qu'on remette l'argent de chacun d’eux dans son sac, et qu'on leur donne de la provision pour la route ; et cela fut fait ainsi.
\VS{26}Ils chargèrent donc leur blé sur leurs ânes, et s'en allèrent.
\VS{27}L’un d'eux ouvrit son sac pour donner du fourrage à son âne dans l'hôtellerie ; et il vit son argent qui était à l’entrée de son sac.
\VS{28}Il dit à ses frères : Mon argent m'a été rendu ; et le voici dans mon sac. Alors leur cœur fut en défaillance ; et ils furent saisis de peur, et se dirent l'un à l'autre : Qu'est-ce que Dieu nous a fait ?
\VS{29}Et étant arrivés dans le pays de Canaan, vers Jacob leur père, ils lui racontèrent toutes les choses qui leur étaient arrivées, en disant :
\VS{30}L'homme qui est le seigneur du pays, nous a parlé rudement et nous a pris pour des espions du pays.
\VS{31}Mais nous lui avons répondu : Nous sommes sincères, nous ne sommes point des espions.
\VS{32}Nous étions douze frères, fils de notre père ; l'un n'est plus, et le plus jeune est aujourd'hui avec notre père dans le pays de Canaan.
\VS{33}Et cet homme, qui est le seigneur  du pays, nous a dit : A ceci je connaîtrai que vous êtes sincères : Laissez-moi l'un de vos frères, et prenez de quoi nourrir vos familles et partez.
\VS{34}Puis amenez-moi votre jeune frère, et je saurai que vous n'êtes point des espions, que vous êtes sincères ; je vous rendrai votre frère, et vous pourrez librement trafiquer dans le  pays.
\VS{35}Lorsqu’ils vidèrent leurs sacs, voici, le paquet d’argent de chacun était dans son sac. Ils virent, eux et leur père, leurs paquets d’argent, et ils eurent peur.
\VS{36}Jacob leur père leur dit : Vous me privez de mes enfants ! Joseph n'est plus, et Siméon n'est plus, et vous prendriez Benjamin ! C’est sur moi que tout cela retombe.
\VS{37}Ruben parla à son père et lui dit : Fais mourir deux de mes fils si je ne te ramène pas Benjamin ! Remets-le entre mes mains et je te le ramènerai.
\VS{38}Jacob répondit : Mon fils ne descendra point avec vous, car son frère est mort, et il reste seul ; s’il lui arrivait un malheur dans le voyage que vous allez faire, vous feriez descendre mes cheveux blancs avec douleur dans le scheol.
\Chap{43}
\TextTitle{Jacob renvoie ses fils en Egypte\FTNTT{Ge. 37:26-28}}
\VerseOne{}Or la famine devint fort grande dans le pays.
\VS{2}Et quand  ils eurent achevé de manger le blé qu'ils avaient apporté d'Egypte, leur père leur dit : Retournez, achetez-nous un peu de vivres.
\VS{3}Juda lui répondit et lui dit : Cet homme nous a expressément déclaré, disant : Vous ne verrez point ma face, à moins que votre frère ne soit avec vous.
\VS{4}Si donc tu envoies notre frère avec nous, nous descendrons en Egypte et nous t'achèterons des vivres.
\VS{5}Mais si tu ne l'envoies pas, nous n'y descendrons point ; car cet homme nous a dit : Vous ne verrez point ma face, à moins que votre frère ne soit avec vous.
\VS{6}Et Israël dit : Pourquoi avez-vous mal agi à mon égard, en disant à cet homme que vous aviez encore un frère ?
\VS{7}Ils répondirent : Cet homme nous a interrogés sur nous et sur notre famille, en disant : Votre père vit-il encore ? N'avez-vous point de frère ? Et nous lui avons déclaré selon ce qu'il nous avait demandé ; pouvions-nous savoir qu'il dirait : Faites descendre votre frère ?
\VS{8}Juda dit à Israël, son père : Laisse venir l'enfant avec moi, afin que nous nous levions et que nous partions ; et nous vivrons et nous ne mourons point, nous, toi et nos enfants.
\VS{9}Je réponds de lui, tu le redemanderas de ma main. Si je ne te le ramène pas auprès de toi et si je ne le remets pas devant ta face, je serai coupable toute ma vie envers toi.
\VS{10}Car si nous n’avions pas tardé, certainement nous serions déjà de retour deux fois.
\VS{11}Alors Israël leur père leur dit : Si cela est ainsi, faites ceci, prenez dans vos bagages les meilleures productions du pays, pour en  porter un présent à cet homme, un peu de baume, et un peu de miel, des épices, de la myrrhe, des dattes, et des amandes.
\VS{12}Prenez avec vous de l'argent au double dans vos mains, et rapportez l’argent qu’on avait mis à l’entrée de vos sacs ; peut-être était-ce une erreur.
\VS{13}Prenez votre frère, et levez-vous, retournez vers cet homme.
\VS{14}Que le Dieu Tout-Puissant vous fasse trouver grâce devant cet homme, afin qu'il relâche votre autre frère et Benjamin ; et s'il faut que je sois privé de ces deux fils, que j'en sois privé.
\VS{15}Alors ils prirent le présent, et ayant pris de l'argent au double dans leurs mains, et Benjamin, ils se levèrent et descendirent en Egypte ; puis ils se présentèrent devant Joseph.
\VS{16}Dès que Joseph vit Benjamin avec eux, il dit à l’intendant de sa maison : Fais entrer ces gens dans la maison, tue et apprête quelques bêtes, car ils mangeront à midi avec moi.
\VS{17}Cet homme fit ce que Joseph lui avait dit ; et il conduit ces gens dans la maison de Joseph.
\VS{18}Ils eurent peur lorsqu’ils furent conduits dans la maison de Joseph, et ils dirent : Nous sommes emmenés à cause de l'argent remis l’autre fois dans nos sacs ; c’est pour se jeter sur nous, se précipiter sur nous ; c’est pour nous prendre comme esclaves et s’emparer de nos ânes.
\VS{19}Ils s’approchèrent de l’intendant de la maison de Joseph, et lui adressèrent la parole, à l’entrée de la maison.
\VS{20}Ils dirent : Pardon ! Mon seigneur, nous sommes déjà descendus une fois pour acheter des vivres.
\VS{21}Puis, quand nous arrivâmes, au lieu où nous devions passer la nuit, nous avons ouvert nos sacs ; et voici, l’argent de chacun était à l’entrée de son sac, notre argent selon son poids ;  nous le rapportons avec nous.
\VS{22}Nous avons aussi apporté d'autre argent dans nos mains pour acheter des vivres ; et nous ne savons point qui a remis notre argent dans nos sacs.
\VS{23}L’intendant leur dit : Tout va bien pour vous, ne craignez point. C’est votre Dieu, le Dieu de votre père vous a donné un trésor dans vos sacs ; votre argent est parvenu jusqu'à moi ; et il leur amena Siméon.
\VS{24}Cet homme les fit entrer dans la maison de Joseph, et leur donna de l'eau, et ils lavèrent leurs pieds ; il donna aussi à manger à leurs ânes.
\VS{25}Ils préparèrent leur présent en attendant que Joseph revienne à midi ; car ils avaient appris qu'ils mangeraient du pain chez lui.
\VS{26}Quand  Joseph fut arrivé à la maison, ils lui offrirent le présent qu'ils avaient dans leurs mains, et se prosternèrent à terre devant lui dans la maison.
\VS{27}Il leur demanda comment ils se portaient et leur dit : Votre vieux père, dont vous m'avez parlé, se porte-t-il bien ? Vit-il encore ?
\VS{28}Ils répondirent : Ton serviteur, notre père, se porte bien, il vit encore. Et ils s’inclinèrent et se prosternèrent.
\VS{29}Joseph leva les yeux, il vit Benjamin, son frère, fils de sa mère, et il dit : Est-ce là votre jeune frère dont vous m'avez parlé ? Et il ajouta : Mon fils, Dieu te fasse grâce !
\VS{30}Et Joseph se retira promptement, car ses entrailles étaient émues à la vue de son frère, et il cherchait un lieu pour pleurer ; il entra dans sa chambre et il y pleura.
\VS{31}Après s’être lavé le visage, il sortit de là, et faisant des efforts pour se contenir, il dit : Servez le pain.
\VS{32}On servit Joseph à part, et ses frères à part, et les Egyptiens qui mangeaient avec lui furent aussi servis à part, car les Egyptiens ne pouvaient manger du pain avec les Hébreux,  parce que c’est à leurs yeux une abomination.
\VS{33}Les frères de Joseph s’assirent en sa présence, le premier-né selon son droit d’aînesse, et le plus jeune selon son âge ; et ils se regardaient les uns les autres avec étonnement.
\VS{34}Joseph leur fit porter des mets qui étaient devant lui, et Benjamin en eut cinq fois plus que les autres. Ils burent et s’enivrèrent  avec lui.
\Chap{44}
\TextTitle{Juda se rend esclave de Joseph à la place de Benjamin\FTNTT{Ge. 43:9}}
\VerseOne{}Et Joseph donna un ordre à son intendant, en disant : Remplis de vivres les sacs de ces gens, autant qu'ils en pourront porter, et remets l'argent de chacun à l’entrée de son sac.
\VS{2}Tu mettras aussi ma coupe, la coupe d'argent, à l’entrée du sac du plus petit avec l'argent de son blé ; et il fit comme Joseph lui avait dit.
\VS{3}Le matin, dès qu'il fit jour, on renvoya ces hommes avec leurs ânes.
\VS{4}Ils étaient sortis de la ville, ils n’en étaient guère éloignés, lorsque Joseph dit à son intendant : Va, poursuis ces hommes, et quand tu les auras atteints, tu leur diras : Pourquoi avez-vous rendu le mal pour le bien ?
\VS{5}N'est-ce pas la coupe dont se sert mon seigneur pour boire et pour deviner ? Vous avez mal fait d’agir ainsi.
\VS{6}L’intendant les atteignit, et leur dit ces paroles.
\VS{7}Ils lui répondirent : Pourquoi mon seigneur parle-t-il ainsi ? Loin de tes serviteurs la pensée de faire pareille chose !
\VS{8}Voici, nous t'avons rapporté du pays de Canaan l'argent que nous avions trouvé à l’entrée de nos sacs, et comment aurions-nous dérobé de l'argent ou de l'or de la maison de ton maître ?
\VS{9}Que celui de tes serviteurs sur qui se trouvera la coupe meure ; et nous serons aussi esclaves de mon seigneur !
\VS{10}Il leur dit : Qu'il soit fait maintenant selon vos paroles ! Qu’il en soit ainsi ! Que celui sur qui se trouvera la coupe soit mon esclave, et vous, vous serez innocents.
\VS{11}Et ils se hâtèrent de déposer chacun son sac à terre ; et chacun ouvrit son sac.
\VS{12}L’intendant les fouilla, en commençant par le plus âgé, et finissant par le plus jeune ; et la coupe fut trouvée dans le sac de Benjamin.
\VS{13}Alors ils déchirèrent leurs vêtements, et chacun rechargea son âne, et ils retournèrent à la ville.
\VS{14}Juda et ses frères arrivèrent à la maison de Joseph, qui était encore là, et ils se jetèrent à terre devant lui.
\VS{15}Joseph leur dit : Quelle action avez-vous faite ? Ne savez-vous pas qu'un homme tel que moi ne manque pas de deviner ?
\VS{16}Juda lui répondit : Que dirons-nous à mon seigneur ? Comment parlerons-nous ? Et comment nous justifierons-nous ? Dieu a trouvé l'iniquité de tes serviteurs ; voici, nous sommes esclaves de mon seigneur, nous, et celui entre les mains de qui la coupe a été trouvée.
\VS{17}Mais il dit : Loin de moi la pensée d’agir ainsi ! L’homme dans la main duquel la coupe a été trouvée sera mon esclave ; mais vous, remontez en paix vers votre père.
\VS{18}Alors Juda s'approcha de lui en disant : Pardon mon seigneur ! Je te prie, que ton serviteur dise un mot, je te prie aux oreilles de mon seigneur, et que ta colère ne s'enflamme point contre ton serviteur, car tu es comme Pharaon.
\VS{19}Mon seigneur interrogea ses serviteurs en disant : Avez-vous un père ou un frère ?
\VS{20}Nous avons répondu à mon seigneur : Nous avons notre père qui est âgé, et un enfant de sa vieillesse, et qui est le plus jeune d'entre nous ; son frère est mort, et celui-ci est resté le seul enfant  de sa mère ; et son père l'aime.
\VS{21}Tu as dis à tes serviteurs : Faites-le descendre vers moi, et que je le voie de mes yeux.
\VS{22}Nous avons répondu  à mon seigneur : Cet enfant ne peut quitter son père, car s'il le quitte, son père mourra.
\VS{23}Alors tu dis à tes serviteurs : Si votre petit frère ne descend avec vous, vous ne verrez plus ma face.
\VS{24}Lorsque nous sommes remontés auprès de ton serviteur, mon père, nous lui avons rapporté les paroles de mon seigneur.
\VS{25}Notre père nous a dit : Retournez, et achetez-nous un peu de vivres.
\VS{26}Nous lui avons répondu : Nous ne pouvons pas descendre ; mais si notre petit frère est avec nous, nous descendrons, car nous ne pouvons pas voir la face de cet homme, à moins que notre jeune frère ne soit avec nous.
\VS{27}Ton serviteur, mon père, nous répondit : Vous savez que ma femme m'a enfanté deux fils.
\VS{28}L’un étant sorti de chez moi, je pense qu’il a été sans doute déchiré, car je ne l’ai pas revu jusqu’à présent.
\VS{29}Si vous me prenez encore celui-ci, et qu’il lui arrive un malheur, vous ferez descendre mes cheveux blancs avec douleur dans le scheol.
\VS{30}Maintenant, si je retourne auprès de ton serviteur, mon père, sans avoir avec nous l’enfant à l’âme duquel son âme est attachée,
\VS{31}il mourra, en voyant que l’enfant n’y est pas ; et tes serviteurs feront descendre avec douleur dans le scheol les cheveux blancs de ton serviteur, notre père.
\VS{32}De plus, ton serviteur a répondu pour l'enfant, en le prenant à mon père, en disant : Si je ne te le ramène pas, je serai pour toujours coupable envers mon père.
\VS{33}Permets donc, je te prie, à ton serviteur de rester à la place de l’enfant, comme esclave de mon seigneur ; et que l’enfant remonte avec ses frères.
\VS{34}Car comment pourrai-je remonter vers mon père, si l'enfant n'est pas avec moi ? Que je ne voie point l'affliction qu'en aurait mon père !
\Chap{45}
\TextTitle{Joseph révèle son identité à ses frères}
\VerseOne{}Alors Joseph, ne pouvant plus se contenir devant tous ceux qui étaient là présents, cria : Faites sortir tout le monde ! Et il ne resta personne quand il se fit connaître à ses frères.
\VS{2}Et en pleurant, il éleva sa voix, et les Egyptiens l'entendirent, et la maison de Pharaon l'entendit aussi.
\VS{3}Et Joseph dit à ses frères : Je suis Joseph ! Mon père vit-il encore ? Mais ses frères ne pouvaient lui répondre, car ils étaient tout troublés en sa présence.
\VS{4}Joseph dit encore à ses frères : Je vous prie, approchez-vous de moi ; et ils s'approchèrent, et il leur dit : Je suis Joseph, votre frère, que vous avez vendu pour être mené en Egypte\FTNT{Ac. 7:13.}.
\VS{5}Mais maintenant ne soyez pas en peine, et n'ayez point de regret de ce que vous m'avez vendu pour être mené ici, car Dieu m'a envoyé devant vous pour la conservation de votre vie.
\VS{6}Car voici, il y a déjà deux ans que la famine est sur la terre, et il y aura encore cinq ans pendant lesquels il n'y aura ni labour ni moisson.
\VS{7}Mais Dieu m'a envoyé devant vous, pour vous faire subsister sur la terre, et vous faire vivre par une grande délivrance.
\VS{8}Maintenant donc ce n'est pas vous qui m'avez envoyé ici, mais c'est Dieu ; il m'a établi père de Pharaon, et seigneur sur toute sa maison, et gouverneur de tout le pays d'Egypte.
\VS{9}Hâtez-vous d'aller vers mon père, et dites-lui : Ainsi a dit ton fils, Joseph : Dieu m'a établi seigneur sur toute l'Egypte, descends vers moi, ne t'arrête point.
\VS{10}Et tu habiteras dans la contrée de Gosen, et tu seras près de moi, toi, tes fils, et les fils de tes fils, tes brebis, et tes bœufs, et tout ce qui est à toi.
\VS{11}Là, je te nourrirai, car il y aura encore cinq années de famine ; et ainsi tu ne périras point, toi et ta maison, et tout ce qui est à toi.
\VS{12}Et voici, vous voyez de vos yeux, et Benjamin mon frère voit aussi de ses yeux, que c'est moi qui vous parle de ma propre bouche.
\VS{13}Rapportez donc à mon père quelle est ma gloire en Egypte, et tout ce que vous avez vu ; hâtez-vous, et faites descendre ici mon père.
\VS{14}Alors il se jeta sur le cou de Benjamin, son frère, et pleura. Benjamin pleura aussi sur son cou.
\VS{15}Puis il embrassa tous ses frères et pleura sur eux ; après cela ses frères parlèrent avec lui.
\TextTitle{Jacob pardonne ses frères et fait venir son père Jacob\FTNTT{Ge. 43:9}}
\VS{16}Et le bruit se répandit dans la maison de Pharaon que les frères de Joseph étaient venus, ce qui plut fort à Pharaon et à ses serviteurs.
\VS{17}Alors Pharaon dit à Joseph : Dis à tes frères : Faites ceci : Chargez vos bêtes, et allez, retournez dans le pays de Canaan ;
\VS{18}et prenez votre père et vos familles, et revenez vers moi, et je vous donnerai le meilleur du pays d'Egypte ; et vous mangerez la graisse de la terre.
\VS{19}Tu as ordre de leur dire: Faites ceci : Prenez dans le pays d’Egypte des chars pour vos enfants et pour vos femmes; amenez votre père, et venez.
\VS{20}Ne regrettez point ce que vous laisserez, car ce qu’il y a de meilleur dans tout le pays d’Egypte sera pour vous.
\VS{21}Et les fils d'Israël firent ainsi. Et Joseph leur donna des chars selon l'ordre de Pharaon ; il leur donna aussi de la provision pour la route.
\VS{22}Il leur donna à chacun des vêtements de rechange ; et il donna à Benjamin trois cents pièces d'argent et cinq vêtements de rechange.
\VS{23}Il envoya aussi à son père dix ânes chargés des plus excellentes choses qu'il y avait en Egypte, et dix ânesses portant du blé, du pain, et des vivres à son père pour la route.
\VS{24}Il renvoya donc ses frères, et ils partirent ; et il leur dit : Ne vous querellez point en chemin.
\VS{25}Ainsi ils remontèrent d'Egypte, et vinrent dans le  pays de Canaan auprès de Jacob, leur père.
\VS{26}Et ils lui rapportèrent et lui dirent : Joseph vit encore, et même c’est lui qui gouverne tout le pays d'Egypte ; mais le cœur de Jacob resta froid, parce qu’il ne les croyait pas.
\VS{27}Et ils lui dirent toutes les paroles que Joseph leur avait dites ; puis il vit les chars que Joseph avait envoyés pour le porter ; et l'esprit de Jacob, leur père, se ranima.
\VS{28}Alors Israël dit : C'est assez ! Joseph, mon fils, vit encore ! J'irai, et je le verrai avant que je meure.
\Chap{46}
\TextTitle{Jacob en Egypte}
\VerseOne{}Israël donc partit avec tout ce qui lui appartenait, et vint à Beer-Schéba, et il offrit des sacrifices au Dieu de son père Isaac.
\VS{2}Et Dieu parla à Israël dans une vision pendant la nuit et lui dit : Jacob, Jacob ! Et il répondit : Me voici.
\VS{3}Et Dieu lui dit : Je suis le Dieu, le Dieu de ton père. Ne crains point de descendre en Egypte, car là je te ferai devenir une grande nation.
\VS{4}Je descendrai avec toi en Egypte, et je t'en ferai aussi très certainement remonter ; et Joseph te fermera les yeux avec sa main.
\VS{5}Ainsi Jacob partit de Beer-Schéba, et les fils d'Israël mirent Jacob, leur père, et leurs petits enfants, et leurs femmes, sur les chars que Pharaon avait envoyés pour le porter.
\VS{6}Ils emmenèrent aussi leur bétail et leur bien qu'ils avaient acquis dans le pays de Canaan ; et Jacob et toute sa famille avec lui vinrent en Egypte.
\VS{7}Il amena avec lui en Egypte ses fils, et les fils de ses fils, ses filles, et les filles de ses fils, et toute sa famille.
\TextTitle{Les fils de Jacob en Egypte}
\VS{8}Voici les noms des fils d'Israël qui vinrent en Egypte : Jacob et ses fils. Le premier-né de Jacob fut Ruben.
\VS{9}Et les fils de Ruben : Hénoc, Pallu, Hetsron, et Carmi.
\VS{10}Et les fils de Siméon : Jemuel, Jamin, Ohad, Jakin, Tsochar, et Saül, fils d'une Cananéenne.
\VS{11}Et les fils de Lévi : Guerschon, Kehath, et Merari.
\VS{12}Et les fils de Juda : Er, Onan, Schéla, Pérets et Zérach ; mais Er et Onan moururent au pays de Canaan. Les fils de Pérets furent Hetsron et Hamul.
\VS{13}Et les fils d'Issacar : Thola, Puva, Job et Schimron.
\VS{14}Et les fils de Zabulon : Séred, Elon et Jahleel.
\VS{15}Ce sont là les fils de Léa, qu'elle enfanta à Jacob à Paddan-Aram, avec Dina, sa fille. Ses fils et ses filles formaient en tout trente-trois personnes.
\VS{16}Et les fils de Gad : Tsiphjon, Haggi, Schuni, Etsbon, Eri, Arodi et Areéli.
\VS{17}Et les fils d'Aser : Jimna, Jischva, Jischvi, Beria et Sérach, leur sœur. Les fils de Beria : Héber et Malkiel.
\VS{18}Ce sont là les fils de Zilpa que Laban donna à Léa, sa fille ; et elle les enfanta à Jacob. En tout seize personnes.
\VS{19}Les fils de Rachel, femme de Jacob, furent Joseph et Benjamin.
\VS{20}Et il naquit à Joseph dans le  pays d'Egypte, Manassé et Ephraïm, qu'Asnath, fille de Poti-Phéra, prêtre d'On, lui enfanta.
\VS{21}Et les fils de Benjamin étaient Béla, Béker, Aschbel, Guéra, Naaman, Ehi, Rosch, Muppim, Huppim et Ard.
\VS{22}Ce sont là les fils de Rachel, qu'elle enfanta à Jacob. En tout quatorze personnes.
\VS{23}Et les fils de Dan : Huschim.
\VS{24}Et les enfants de Nephthali : Jahtseel, Guni, Jetser, et Schillem.
\VS{25}Ce sont là les fils de Bilha, que Laban donna à Rachel, sa fille, et elle les enfanta à Jacob. En tout sept personnes.
\VS{26}Toutes les personnes appartenant à Jacob qui vinrent en Egypte, et qui étaient issues de lui, sans les femmes des fils de Jacob, furent en tout soixante-dix.
\VS{27}Et les fils de Joseph qui lui étaient nés en Egypte furent deux personnes. Toutes les personnes de la maison de Jacob qui vinrent en Egypte furent soixante-dix.
\VS{28}Jacob envoya Juda devant lui vers Joseph, pour l’informer qu’il se rendait en Gosen. Ils vinrent donc dans la contrée de Gosen.
\VS{29}Et Joseph fit atteler son char, et y monta pour aller à la rencontre d'Israël, son père, en Gosen. Dès qu’il le vit, il se jeta à son cou, et pleura longtemps sur son cou.
\VS{30}Et Israël dit à Joseph : Que je meure à présent, puisque j'ai vu ton visage, et que tu vis encore.
\VS{31}Puis Joseph dit à ses frères et à la famille de son père : Je monterai pour informer Pharaon, et je lui dirai : Mes frères et la famille de mon père, qui étaient au pays de Canaan, sont arrivés auprès de moi.
\VS{32}Et ces hommes sont bergers, ils se sont toujours occupés du bétail, et ils ont amené leurs brebis et leurs bœufs, et tout ce qui était à eux.
\VS{33}Et quand Pharaon vous fera appeler et vous dira : Quel est votre métier ?
\VS{34}Vous direz : Tes serviteurs se sont toujours occupés de bétail dès leur jeunesse jusqu'à maintenant, nous, et nos pères. De cette manière, vous habiterez dans le pays de Gosen, car les Egyptiens ont en abomination les bergers.
\Chap{47}
\TextTitle{La famille de Jacob honorée en Egypte}
\VerseOne{}Joseph alla avertir Pharaon, et lui dit : Mon père  et mes frères sont arrivés du pays de Canaan avec leurs troupeaux et leurs bœufs, et tout ce qui est à eux ; et voici, ils sont dans le pays de Gosen.
\VS{2}Et il prit une partie de ses frères, à savoir cinq, et il les présenta à Pharaon.
\VS{3}Et Pharaon dit aux frères de Joseph : Quel est votre métier ? Ils répondirent à Pharaon : Tes serviteurs sont bergers, comme l'ont été nos pères.
\VS{4}Ils dirent aussi à Pharaon : Nous sommes venus séjourner comme étrangers dans ce pays, parce qu'il n'y a plus de pâturages pour les troupeaux de tes serviteurs, et il y a une grande famine au pays de Canaan ; maintenant nous te prions que tes serviteurs demeurent dans le pays de Gosen.
\VS{5}Et Pharaon parla à Joseph et lui dit : Ton père et tes frères sont arrivés auprès de toi.
\VS{6}Le pays d'Egypte est à ta disposition ; fais habiter ton père et tes frères dans le meilleur endroit du pays ; qu'ils demeurent dans la terre de Gosen ; et si tu connais parmi eux des hommes habiles tu les établiras chefs de tous mes troupeaux.
\VS{7}Alors Joseph amena Jacob, son père, et le présenta à Pharaon ; et Jacob bénit Pharaon.
\VS{8}Et Pharaon dit à Jacob : Quel est le nombre de jours de tes années ?
\VS{9}Jacob répondit à Pharaon : Les jours des années de mes pèlerinages sont de cent trente ans ; les jours des années de ma vie ont été courts et mauvais et n'ont point atteint les jours des années de la vie de mes pères, du temps de leurs pèlerinages.
\VS{10}Jacob donc bénit Pharaon, et sortit de devant lui.
\VS{11}Et Joseph assigna une demeure à son père et à ses frères, et leur donna une possession au pays d'Egypte, au meilleur endroit du pays, dans le pays d'Egypte, comme Pharaon l'avait ordonné.
\VS{12}Et Joseph fournit du pain à son père et à ses frères, et à toute la maison de son père, selon le nombre de leurs familles.
\VS{13}Or il n'y avait point de pain sur toute la terre, car la famine était très grande ; et le pays d'Egypte et le pays de Canaan étaient épuisés par la famine.
\VS{14}Et Joseph amassa tout l'argent qui se trouva dans le pays d'Egypte, et dans le pays de Canaan, contre le blé qu'on achetait ; et il apporta l'argent à la maison de Pharaon.
\VS{15}Quand l'argent du pays d'Egypte et du pays de Canaan fut épuisé, tous les Egyptiens vinrent à Joseph en disant : Donne-nous du pain ; et pourquoi mourrions-nous en ta présence, parce que l'argent manque ?
\VS{16}Joseph répondit : Donnez votre bétail, et je vous en donnerai pour votre bétail, puisque l'argent manque.
\VS{17}Alors ils amenèrent à Joseph leur bétail, et Joseph leur donna du pain pour des chevaux, pour des troupeaux de brebis, pour des troupeaux de boeufs, et pour des ânes ; ainsi il leur fournit du pain en échange de leurs troupeaux cette année-là.
\VS{18}Lorsque cette année fut écoulée, ils revinrent à Joseph l'année suivante et lui dirent : Nous ne cacherons point à mon seigneur que l'argent est épuisé et les troupeaux de bétail ont été amenés à mon seigneur, il ne nous reste plus rien devant mon seigneur que nos corps et nos terres.
\VS{19}Pourquoi mourrions-nous sous tes yeux ? Achète-nous avec nos terres, pour du pain ; et nous serons esclaves de Pharaon, et nos terres seront à lui ; donne-nous aussi de quoi semer, afin que nous vivions et ne mourions point, et que nos terres ne soient point désolées.
\VS{20}Ainsi, Joseph acheta toutes les terres de l’Egypte pour Pharaon ; car les Egyptiens vendirent chacun son champ, parce que la famine les pressait. Et le pays devint la propriété de Pharaon.
\VS{21}Et il fit passer le peuple dans les villes, d’un bout à l’autre des frontières de l’Egypte.
\VS{22}Seulement, il n’acheta point les terres des prêtres, parce qu’il y avait une loi de Pharaon en faveur des prêtres, qui vivaient du revenu que leur assurait Pharaon, c’est pourquoi ils ne vendirent point leurs terres.
\VS{23}Et Joseph dit au peuple : Voici, je vous ai achetés aujourd'hui, vous et vos terres pour Pharaon, voilà de la semence pour ensemencer la terre.
\VS{24}Et quand le temps de la récolte viendra, vous donnerez la cinquième partie à Pharaon, et les quatre autres seront à vous, pour ensemencer les champs, et pour votre nourriture, et pour celle de ceux qui sont dans vos maisons, et pour la nourriture de vos petits enfants.
\VS{25}Et ils dirent : Tu nous sauves la vie ! Que nous trouvions grâce aux yeux de mon seigneur, et nous serons esclaves de Pharaon.
\VS{26}Et Joseph fit de cela une loi qui a subsisté jusqu’à ce jour, et d’après laquelle un cinquième du revenu des terres de l’Egypte appartient à Pharaon ; il n’y a que les terres des prêtres qui ne soient point à Pharaon.
\TextTitle{Jacob demande à être enterré à Canaan}
\VS{27}Israël habita dans le pays d’Egypte, dans le pays de Gosen. Ils eurent des possessions, ils furent féconds et multiplièrent beaucoup.
\VS{28}Jacob vécut dix-sept ans dans le pays d’Egypte ; et les jours des années de la vie de Jacob furent de cent quarante-sept ans.
\VS{29}Et quand le jour de la mort d'Israël approcha, il appela Joseph, son fils, et lui dit : Je te prie, si j'ai trouvé grâce à tes yeux, mets présentement ta main sous ma cuisse, et jure-moi que tu useras envers moi de bonté et de fidélité : Je te prie, ne m'enterre point en Egypte !
\VS{30}Quand  je serai couché avec mes pères, tu me transporteras hors de l'Egypte, et m'enterreras dans leur sépulcre. Et il répondit : Je le ferai selon ta parole.
\VS{31}Et Jacob lui dit : Jure-le-moi ; et il le lui jura. Et Israël se prosterna sur le chevet du lit.
\Chap{48}
\TextTitle{Bénédiction de Jacob sur les fils de Joseph}
\VerseOne{}Or il arriva après ces choses que l'on vint dire à Joseph : Voici, ton père est malade. Et il prit avec lui ses deux fils, Manassé et Ephraïm.
\VS{2}On avertit Jacob et on lui dit : Voici Joseph, ton fils, qui vient vers toi. Alors Israël rassembla ses forces et s’assit sur son lit.
\VS{3}Puis Jacob dit à Joseph : Le Dieu Tout-Puissant m’est apparu  à Luz, au pays de Canaan, et m’a béni.
\VS{4}Et il m’a dit : Voici, je te ferai croître et multiplier, et je te ferai devenir une assemblée de peuples, et je donnerai ce pays en possession perpétuelle à ta postérité après toi.
\VS{5}Et maintenant tes deux fils, qui te sont nés au pays d'Egypte, avant mon arrivée vers toi, seront à moi : Ephraïm et Manassé seront à moi comme Ruben et Siméon.
\VS{6}Mais les enfants que tu auras engendrés après eux, seront à toi, et ils seront appelés selon le nom de leurs frères dans leur héritage.
\VS{7}A mon retour de Paddan, Rachel mourut en route auprès de moi, dans le pays de Canaan, à quelque distance d’Ephrata ; et c’est là que je l’ai enterrée, sur le chemin d’Ephrata, qui est Bethléhem.
\VS{8}Puis Israël vit les fils de Joseph, et il dit : Qui sont ceux-ci ?
\VS{9}Et Joseph répondit à son père : Ce sont mes fils que Dieu m'a donnés ici ; et il dit : Amène-les-moi, je te prie, afin que je les bénisse.
\VS{10}Or les yeux d'Israël étaient appesantis par la vieillesse, et il ne pouvait plus voir ; et il les fit approcher de lui, les embrassa et les prit dans ses bras.
\VS{11}Et Israël dit à Joseph : Je ne pensais pas revoir ton visage ; et voici, Dieu m'a fait voir et toi et ta postérité.
\VS{12}Et Joseph les retira des genoux de son père, et se prosterna le visage contre terre.
\VS{13}Puis Joseph les prit tous deux, Ephraïm de sa main droite à la gauche d’Israël, et Manassé de sa main gauche à la droite d’Israël, et il les fit approcher de lui.
\VS{14}Israël étendit sa main droite et la posa sur la tête d’Ephraïm qui était le plus jeune, et il posa sa main gauche sur la tête de Manassé ; ce fut avec intention qu’il posa ses mains ainsi, car Manassé était le premier-né.
\VS{15}Il bénit Joseph et dit : Que le Dieu en présence duquel ont marché mes pères, Abraham et Isaac, que le Dieu qui m’a conduit depuis que j’existe jusqu’à ce jour\FTNT{Hé. 11:21.},
\VS{16}que l’Ange qui m’a délivré de tout mal, bénisse ces enfants ! Qu’ils soient appelés de mon nom et du nom de mes pères, Abraham et Isaac, et qu’ils multiplient en abondance comme les poissons au milieu du pays.
\VS{17}Joseph vit avec déplaisir que son père posait sa main droite sur la tête d’Ephraïm ; il saisit la main de son père, pour la détourner de dessus la tête d’Ephraïm, et la diriger sur celle de Manassé.
\VS{18}Et Joseph dit à son père : Ce n'est pas ainsi mon père ! Car celui-ci est l'aîné ; mets ta main droite sur sa tête.
\VS{19}Mais son père le refusa en disant : Je le sais, mon fils, je le sais. Celui-ci deviendra aussi un peuple, et même il sera grand ; mais toutefois son frère, qui est plus jeune, sera plus grand que lui, et sa postérité sera une multitude de nations.
\VS{20}Il les bénit ce jour-là et dit : C’est par toi qu’Israël bénira en disant : Que Dieu te traite comme Ephraïm et comme Manassé ! Et il mit Ephraïm avant Manassé.
\VS{21}Puis Israël dit à Joseph : Voici, je  vais mourir, mais Dieu sera avec vous, et vous fera retourner au pays de vos pères.
\VS{22}Et je te donne une portion  de plus qu'à tes frères, celle que j'ai prise avec mon épée et mon arc sur les Amoréens.
\Chap{49}
\TextTitle{Prophétie de Jacob qui bénit ses fils}
\VerseOne{}Puis Jacob appela ses fils et leur dit : Assemblez-vous, et je vous annoncerai ce qui vous arrivera dans les derniers jours\FTNT{L’expression «~dans les derniers jours~» vient de l’hébreu «~achariyth~» qui veut dire «~dernier~». Son équivalent grec est «~eschatos~» : «~dernier~», «~extrémité~» etc. Jacob est le premier homme à avoir utilisé  cette expression. Cette promesse de Jacob devait arriver à Israël dans les derniers jours, selon leurs tribus. Ainsi, les promesses du droit d'aînesse de Ge. 49 étaient pour l'âge messianique, lequel est associé aux derniers jours, et a commencé à la Fête de la Pentecôte (Ac. 2:14-21). 
Ces jours impliquent :
- L’effusion de l’Esprit, le réveil de l’Eglise de Christ (Mt. 25:1-13 ; Ac. 2)
- Le réveil des faux prophètes ou l’apostasie (2 Pi. 3:3 ; 1 Jn. 2)
- La dégradation de la moralité (2 Ti. 3)
- L’enrichissement des hommes de ce monde (Ja. 5:3 ; Ap. 3:14-22)
- Le fait que Dieu nous parle par le Fils (Hé. 1:2)
- La future résurrection des saints lors du retour du Messie (Jn. 6:39-54 ; 1 Th. 4:12-17).Le temps des nations (fin des temps) s’achèvera lors du retour visible de Jésus-Christ pour établir son règne sur toute la terre. Le temps des nations a commencé lorsque, à la suite de l’infidélité d’Israël, la gloire de Dieu a quitté le temple et la ville de Jérusalem (Ez. 11), la puissance fut confiée aux nations en la personne de Nebucadnetsar qui s’empara de Jérusalem (2 R. 24 et 25 ; 2 Ch. 36:6-21 ; Da. 1 ; Jé 39). Ces temps dureront jusqu’à la destruction finale du dernier empire des nations représenté par la Bête romaine ressuscitée (Ap. 13:3). Cette destruction n’aura lieu que lorsque Jésus-Christ, la pierre détachée sans le secours d’aucune main, deviendra une grande montagne qui remplira toute la terre (Da. 2:34 ; Mi. 4). Jérusalem ne sera délivrée du joug des nations qu’à ce moment-là. Les temps des nations ne seront accomplis que lorsque le trône de Dieu sera de nouveau établi à Jérusalem.}.
\VS{2}Rassemblez-vous, et écoutez, fils de Jacob ; écoutez Israël\FTNT{«~Ecoutez Israël~» : Le «~shema~» Israël est le texte principal de la liturgie juive. Composé de trois extraits de la Torah, on le récite matin et soir accompagné de bénédictions. Voir De. 6:4-9.}, votre père.
\VS{3}Ruben, tu es mon premier-né, ma force et le commencement de ma vigueur, qui excelle en dignité et qui excelle aussi en force ;
\VS{4}impétueux comme les eaux ; tu n'auras pas la prééminence, car tu es monté sur la couche de ton père, et tu as souillé mon lit en y montant.
\VS{5}Siméon et Lévi, sont frères, leurs glaives sont des instruments de violence dans leurs demeures.
\VS{6}Que mon âme n'entre point dans leur conseil secret, que ma gloire ne soit point jointe à leur compagnie, car ils ont tué les gens dans leur colère, et ont enlevé les bœufs pour leur plaisir.
\VS{7}Maudite soit leur colère, car elle a été violente ; et leur fureur, car elle a été cruelle ; je les diviserai dans Jacob, et les disperserai dans Israël.
\VS{8}Juda, quant à toi, tes frères te loueront ; ta main sera sur la nuque de tes ennemis ; les fils de ton père se prosterneront devant toi.
\VS{9}Juda est un jeune lion. Mon fils, tu reviens du carnage, mon fils ! Il ploie les genoux, il se couche comme un lion, comme une lionne : Qui le fera lever ?
\VS{10}Le sceptre ne s’éloignera point de Juda, ni le bâton de législateur d'entre ses pieds, jusqu'à ce que le Schilo vienne, et que  les peuples lui obéissent.
\VS{11}Il attache à la vigne son ânon, et au cep excellent le petit de son ânesse ; il lavera son vêtement dans le vin, et son vêtement dans le sang des raisins.
\VS{12}Il a les yeux rouges de vin, et les dents blanches de lait.
\VS{13}Zabulon habitera sur la côte des mers, il sera un port des navires ; et ses côtés s'étendront vers Sidon.
\VS{14}Issacar est un âne robuste, couché entre les barres des étables.
\VS{15}Il voit que le lieu où il repose est agréable et que la contrée est magnifique ; et il courbe son épaule sous le fardeau, il s’assujettit à un tribut.
\VS{16}Dan jugera son peuple, comme l’une des tribus d'Israël.
\VS{17}Dan sera un serpent sur le chemin, une vipère sur le sentier, mordant les talons du cheval, pour que le cavalier tombe à la renverse.
\VS{18}Ô Yahweh ! J’espère en ton salut\FTNT{Le mot secours se dit «~yeshuw`ah~» en hébreu et veut littéralement dire  «~salut~», «~délivrance~». Avant de mourir, Jacob a  donc  placé son espérance en Jésus-Christ qui est la résurrection et la vie (Jn. 11:25). Voir commentaire en Es. 26:1.} !
\VS{19}Quant à Gad, des troupes viendront l’attaquer, mais il ravagera leur arrière-garde.
\VS{20}Le pain excellent viendra d'Aser, et il fournira les mets délicats des rois.
\VS{21}Nephtali est une biche en liberté ; il profère des belles paroles.
\VS{22}Joseph est un fils fertile, un rameau fertile près d'une fontaine ; ses branches se sont étendues sur la muraille.
\VS{23}Des archers l’ont provoqué, ils ont lancé des traits ; les archers l’ont poursuivi de leur haine.
\VS{24}Mais son arc est demeuré ferme, et ses mains ont été fortifiées par les mains du Puissant de Jacob : Il est ainsi devenu le pasteur, le rocher d’Israël.
\VS{25}C’est l’œuvre du Dieu de ton père qui t’aidera ; c’est l’œuvre du Tout-Puissant qui te bénira des bénédictions des cieux en haut, des bénédictions des eaux en bas, des bénédictions des mamelles et du sein maternel.
\VS{26}Les bénédictions de ton père ont surpassé les bénédictions de ceux qui m'ont engendré, jusqu'à la cime des antiques collines ; elles seront sur la tête de Joseph, et sur le sommet de la tête du Nazaréen d'entre ses frères.
\VS{27}Benjamin est un loup qui déchirera ; le matin il dévorera la proie, et sur le soir il partagera le butin.
\VS{28}Ce sont là tous ceux qui forment les douze tribus d'Israël.  Et c’est là ce que leur père leur dit en les bénissant. Il bénit chacun d'eux selon la bénédiction qui lui était propre.
\VS{29}Il leur donna aussi cet ordre : Je vais être recueilli auprès de mon peuple, enterrez-moi avec mes pères dans la caverne qui est au champ d'Ephron, le Héthien,
\VS{30}dans la caverne du champ de Macpéla, vis-à-vis de Mamré, dans le pays de Canaan. C’est le champ qu’Abraham a acheté d’Ephron, le Héthien, comme propriété sépulcrale.
\VS{31}C'est là qu'on a enterré Abraham avec Sara, sa femme ; c'est là qu'on a enterré Isaac et Rebecca, sa femme ; et c'est là que j'ai enterré Léa.
\VS{32}Le champ a été acquis des fils de Heth avec la caverne qui s’y trouve.
\VS{33}Lorsque Jacob eut achevé de donner ses ordres à ses fils, il retira ses pieds dans le lit, il expira, et fut recueilli auprès de son peuple.
\Chap{50}
\TextTitle{Mort de Jacob}
\VerseOne{}Alors Joseph se jeta sur le visage de son père, pleura sur lui et l’embrassa.
\VS{2}Et Joseph ordonna à ceux de ses serviteurs qui étaient médecins d'embaumer son père ; et les médecins embaumèrent Israël.
\VS{3}Et on employa quarante jours à l'embaumer, car c'était la coutume d'embaumer les corps pendant quarante jours ; et les Egyptiens le pleurèrent soixante-dix jours.
\VS{4}Quand les jours du deuil furent passés, Joseph s’adressa aux gens de la maison de Pharaon, et leur dit : Si j’ai trouvé grâce à vos yeux, rapportez, je vous prie, à Pharaon ce que je vous dis.
\VS{5}Mon père m’a fait jurer en disant : Voici, je vais mourir ! Tu m’enterreras dans le sépulcre que je me suis acheté au pays de Canaan. Je voudrais donc y monter, pour enterrer mon père ; et je reviendrai.
\VS{6}Et Pharaon répondit : Monte, et enterre ton père comme il t'a fait jurer.
\VS{7}Alors Joseph monta pour enterrer son père, et les serviteurs de Pharaon, les anciens de la maison de Pharaon, et tous les anciens du pays d'Egypte montèrent avec lui.
\VS{8}Et toute la maison de Joseph, et ses frères, et la maison de son père y montèrent aussi, laissant seulement leurs familles, et leurs troupeaux, et leurs bœufs dans le pays de Gosen.
\VS{9}Il y avait encore avec Joseph des chars et des cavaliers, en sorte que le cortège était très nombreux.
\VS{10}Arrivés à l’aire d’Athad, qui est au-delà du Jourdain, ils firent entendre de grandes et profondes lamentations ; et Joseph fit en l’honneur de son père un deuil de sept jours.
\VS{11}Et les Cananéens, habitants du pays, voyant ce deuil dans l'aire d'Athad, dirent : Ce deuil est grand pour les Egyptiens ; c'est pourquoi cette aire, qui est au-delà du Jourdain, fut nommée Abel-Mitsraïm\FTNT{Abel-Mitsraïm : «~Pré du deuil de l’Egypte~».}.
\VS{12}Les fils de Jacob firent à l'égard de son corps ce qu'il leur avait ordonné.
\VS{13}Ils le transportèrent au pays de Canaan, et l’enterrèrent dans la caverne du champ de Macpéla, qu’Abraham avait achetée d’Ephron, le Héthien, comme propriété sépulcrale, et qui est vis-à-vis de Mamré.
\VS{14}Et après que Joseph eut enseveli son père, il retourna en Egypte avec ses frères et tous ceux qui étaient montés avec lui pour ensevelir son père.
\VS{15}Et les frères de Joseph, voyant que leur père était mort, se dirent entre eux : Peut-être que Joseph nous aura en haine, et ne manquera pas de nous rendre tout le mal que nous lui avons fait.
\VS{16}Et ils firent dire à Joseph : Ton père a donné cet ordre avant de mourir en disant :
\VS{17}Vous parlerez ainsi à Joseph : Je te prie, pardonne maintenant l'iniquité de tes frères, et leur péché, car ils t'ont fait du mal. Maintenant, je te supplie, pardonne le crime des serviteurs du Dieu de ton père. Et Joseph pleura quand on lui parla.
\VS{18}Ses frères vinrent eux-mêmes se prosterner devant lui, et ils dirent : Nous sommes tes serviteurs.
\VS{19}Et Joseph leur dit : Ne craignez point, car suis-je à la place de Dieu ?
\VS{20}Vous aviez médité de me faire du mal : Dieu l’a changé en bien, pour accomplir ce qui arrive aujourd’hui, pour sauver la vie à un peuple nombreux.
\VS{21}Soyez donc sans crainte ; je vous entretiendrai, vous et vos familles ; et il les consola en parlant à leur cœur.
\VS{22}Joseph demeura donc en Egypte, lui et la maison de son père, et vécut cent dix ans.
\VS{23}Et Joseph vit les fils d'Ephraïm jusqu'à la troisième génération. Makir aussi, fils de Manassé, eut des fils qui furent élevés sur les genoux de Joseph.
\VS{24}Et Joseph dit à ses frères : Je vais mourir ! Mais Dieu ne manquera pas de vous visiter, et il vous fera remonter de ce pays au pays qu’il a juré  de donner à Abraham, Isaac et à Jacob.
\VS{25}Et Joseph fit jurer les enfants d'Israël et leur dit : Dieu ne manquera pas de vous visiter, et alors vous transporterez mes os d'ici\FTNT{Hé. 11:22 ; Ex. 13:19.}.
\VS{26}Puis Joseph mourut, âgé de cent dix ans. On l'embauma, et on le mit dans un cercueil en Egypte.
\PPE{}
\end{multicols}

%\clearpage\ShortTitle{Ex.}\BookTitle{Exode}\BFont
\noindent\hrulefill
{\footnotesize
\textit{
\bigskip
{\centering{}
\\Auteur~: Probablement Moïse
\\(Heb.~: Shemot)
\\Signification~: Noms
\\Thème~: La délivrance
\\Date de rédaction~: Env. 1450-1410 av. J.-C.\\}
}
%\bigskip
\textit{
\\Les fils de Jacob s'étaient retrouvés en Egypte pour survivre à une famine qui avait frappé la terre entière pendant plusieurs années. Grâce à leur frère Joseph, alors gouverneur d'Egypte, ils bénéficièrent d'un bon traitement. Mais la mort de ce dernier et la montée au pouvoir d'un nouveau Pharaon (probablement Ramsès II) inaugurèrent une période de quatre siècles de souffrances pour le peuple élu.
%\bigskip
\\En effet, les Hébreux avaient été réduits en esclavage. En réponse aux cris de douleur de son peuple, Dieu suscita Moïse, dont le nom signifie «~tiré de~». Ce descendant de Lévi fut élevé dans le palais de Pharaon, mais dut s'enfuir parce qu'il avait tué un Egyptien. Après quarante ans passés dans le pays de Madian, le Dieu qui s'appelle «~Je suis~» se révéla à Moïse sur la montagne d'Horeb et lui confia la mission d'aller délivrer son peuple du joug égyptien.
%\bigskip
\\Ce livre retrace la sortie d'Egypte et le début de la traversée du désert, jalonnée de prodiges exceptionnels.\bigskip
}
}
\par\nobreak\noindent\hrulefill
\begin{multicols}{2}
\Chap{1}
\TextTitle{Après la mort de Joseph}
\VerseOne{}Et ce sont ici les noms des fils d'Israël qui entrèrent en Egypte avec Jacob. Ils y entrèrent chacun avec sa famille~: 
\VS{2}Ruben, Siméon, Lévi, et Juda,
\VS{3}Issacar, Zabulon, et Benjamin,
\VS{4}Dan, et Nephthali, Gad, et Aser.
\VS{5}Toutes les personnes issues des reins de Jacob étaient soixante-dix âmes. Joseph était alors en Egypte.
\VS{6}Joseph mourut ainsi que tous ses frères et toute cette génération-là.
\VS{7}Les enfants d'Israël fructifièrent et s'accrurent abondamment, et se multiplièrent et devinrent extrêmement puissants, de sorte que le pays en fut rempli\FTNT{De. 26:5~; Ac. 7:17.}.
\TextTitle{Israël esclave en Egypte}
\VS{8}Depuis, il s'éleva un nouveau roi sur l'Egypte, qui n'avait point connu Joseph.
\VS{9}Et il dit à son peuple~: Voici, le peuple des enfants d'Israël est plus grand et plus puissant que nous.
\VS{10}Agissons donc prudemment avec lui, de peur qu'il ne se multiplie, et que s'il survenait une guerre, il ne se joigne à nos ennemis, ne fasse la guerre contre nous, et qu'il ne s'en aille du pays.
\VS{11}Ils établirent donc sur le peuple des commissaires d'impôts, pour l'affliger en le surchargeant~; car le peuple bâtit des villes à greniers pour Pharaon~; à savoir Pithom et Ramsès.
\VS{12}Mais plus ils l'affligeaient et plus il multipliait et croissait en toute abondance~; c'est pourquoi ils haïssaient les enfants d'Israël\FTNT{Ps. 105:24.}.
\VS{13}Et les Egyptiens assujettirent les enfants d'Israël à une rude servitude\FTNT{Ge. 15:13.}.
\VS{14}Tellement qu'ils leur rendirent la vie amère par un rude travail, en les employant à faire du mortier, des briques, et toute sorte d'ouvrage qui se fait aux champs~; c'était avec cruauté qu'ils leur imposaient toutes ces charges.
\VS{15}Le roi d'Egypte parla aussi aux sages-femmes des Hébreux, nommées l'une Schiphra et l'autre Pua.
\VS{16}Il leur dit~: Quand vous accoucherez les femmes des Hébreux, et que vous les verrez sur les sièges, si c'est un fils, mettez-le à mort~; mais si c'est une fille, qu'elle vive.
\VS{17}Mais les sages-femmes craignirent Dieu et ne firent pas ce que le roi d'Egypte leur avait dit~; car elles laissèrent vivre les fils.
\VS{18}Alors le roi d'Egypte appela les sages-femmes et leur dit~: Pourquoi avez-vous fait cela, et avez-vous laissé vivre les fils~?
\VS{19}Les sages-femmes répondirent à Pharaon~: Parce que les femmes des Hébreux ne sont point comme les femmes Egyptiennes~; car elles sont vigoureuses, elles ont accouché avant que la sage-femme ne soit arrivée chez elles.
\VS{20}Dieu fit du bien aux sages-femmes~; et le peuple multiplia et devint très puissant.
\VS{21}Parce que les sages-femmes craignirent Dieu, il leur édifia des maisons.
\VS{22}Alors Pharaon donna cet ordre à tout son peuple~: Jetez dans le fleuve tous les fils qui naîtront, mais laissez vivre toutes les filles.
\Chap{2}
\TextTitle{Naissance de Moïse\FTNTT{Hé. 11:23-27.}}
\VerseOne{}Un homme de la maison de Lévi s'en alla et prit une fille de Lévi\FTNT{No. 26:59.}.
\VS{2}Cette femme conçut et enfanta un fils. Voyant qu'il était beau, elle le cacha pendant trois mois\FTNT{Hé. 11:23}.
\VS{3}Mais ne pouvant le tenir caché plus longtemps, elle prit une arche de jonc, et l'enduisit de bitume et de poix, mit l'enfant dedans, et le posa parmi des roseaux sur le bord du fleuve.
\VS{4}Et la sœur de cet enfant se tenait loin pour savoir ce qu'il en arriverait.
\VS{5}La fille de Pharaon descendit à la rivière pour se baigner, et ses servantes se promenaient sur le bord de la rivière, et ayant vu le coffret au milieu des roseaux, elle envoya une de ses servantes pour le prendre.
\VS{6}Et l'ayant ouvert, elle vit l'enfant et voici l'enfant pleurait. Elle en fut touchée de compassion et dit~: C'est un des enfants de ces Hébreux~!
\VS{7}Alors la sœur de l'enfant dit à la fille de Pharaon~: Irai-je appeler une femme d'entre les Hébreux, qui allaite~? Et elle t'allaitera cet enfant.
\VS{8}La fille de Pharaon lui répondit~: Va~! Et la jeune fille s'en alla et appela la mère de l'enfant.
\VS{9}Et La fille de Pharaon lui dit~: Emporte cet enfant, et allaite-le moi, je te donnerai ton salaire~; et la femme prit l'enfant et l'allaita.
\VS{10}Quand l'enfant fut devenu grand, elle l'amena à la fille de Pharaon~; il fut pour elle comme un fils. Elle lui donna le nom de Moïse parce que, dit-elle, je l'ai tiré des eaux.
\TextTitle{Moïse prend à cœur le sort d'Israël~; fuite à Madian}
\VS{11}Or il arriva, en ce temps-là, que Moïse, étant devenu grand, sortit vers ses frères et vit leurs travaux~; il vit aussi un Egyptien qui frappait un Hébreu d'entre ses frères\FTNT{Hé. 11:24-25.}.
\VS{12}Et ayant regardé çà et là, et voyant qu'il n'y avait personne, il tua l'Egyptien et le cacha dans le sable.
\VS{13}Il sortit encore le second jour~; et voici, deux hommes Hébreux se querellaient. Il dit à celui qui avait tort~: Pourquoi frappes-tu ton prochain~?
\VS{14}Lequel répondit~: Qui t'a établi prince et juge sur nous~? Veux-tu me tuer comme tu as tué l'Egyptien~? Et Moïse craignit, et dit~: Certainement le fait est connu.
\VS{15}Or Pharaon ayant appris ce fait-là, chercha à faire mourir Moïse~; mais Moïse s'enfuit de devant Pharaon, s'arrêta au pays de Madian et s'assit près d'un puits.
\VS{16}Or le prêtre de Madian avait sept filles qui vinrent puiser de l'eau, et elles remplirent les auges pour abreuver le troupeau de leur père.
\VS{17}Mais des bergers survinrent et les chassèrent~; et Moïse se leva et les secourut, et abreuva leur troupeau.
\VS{18}Et quand elles furent revenues chez Réuel, leur père, il leur dit~: Comment êtes-vous revenues si tôt aujourd'hui~?
\VS{19}Elles répondirent~: Un homme Egyptien nous a délivrées de la main des bergers~; et même il nous a puisé abondamment de l'eau et a abreuvé le troupeau.
\VS{20}Il dit à ses filles~: Où est-il~? Pourquoi avez-vous ainsi laissé cet homme~? Appelez-le, et qu'il mange du pain.
\VS{21}Et Moïse s'accorda de demeurer avec cet homme-là, qui donna Séphora, sa fille, à Moïse.
\VS{22}Et elle enfanta un fils, et il le nomma Guerschom~: Car, dit-il, je séjourne dans un pays étranger.
\TextTitle{Yahweh entend les cris de son peuple}
\VS{23}Or il arriva longtemps après que le roi d'Egypte mourut, et les enfants d'Israël soupirèrent à cause de la servitude, et ils crièrent~; et leur cri monta jusqu'à Dieu, à cause de la servitude\FTNT{No. 20:15-16.}.
\VS{24}Dieu entendit leurs gémissements, et Dieu se souvint de l'alliance qu'il avait traitée avec Abraham, Isaac et Jacob.
\VS{25}Ainsi Dieu regarda les enfants d'Israël et il fit attention à leur état.
\Chap{3}
\TextTitle{Yahweh se révèle à Moïse dans le buisson ardent}
\VerseOne{}Or Moïse fut berger du troupeau de Jéthro, son beau-père, prêtre de Madian~; il mena le troupeau derrière le désert, et vint à la montagne de Dieu à Horeb.
\VS{2}Et l'Ange de Yahweh lui apparut dans une flamme de feu, du milieu d'un buisson. Il regarda, et voici, le buisson était tout en feu, et le buisson ne se consumait point.
\VS{3}Alors Moïse dit~: Je me détournerai maintenant, et je regarderai cette grande vision, pourquoi le buisson ne se consume point.
\VS{4}Et Yahweh vit que Moïse s'était détourné pour regarder~; et Dieu l'appela du milieu du buisson, en disant~: Moïse~! Moïse~! Et il répondit~: Me voici~!
\VS{5}Et Dieu dit~: N'approche point d'ici~; déchausse tes souliers de tes pieds, car le lieu où tu es arrêté est une terre sainte.
\VS{6}Il dit aussi~: Je suis le Dieu de ton père, le Dieu d'Abraham, le Dieu d'Isaac et le Dieu de Jacob\FTNT{Mt. 22:32~; Mc. 12:26~; Lu. 20:37~; Ac. 7:32.}~; Moïse cacha son visage, parce qu'il craignait de regarder vers Dieu.
\VS{7} Et Yahweh dit~: J'ai très bien vu l'affliction de mon peuple qui est en Egypte et j'ai entendu le cri qu'ils ont jeté à cause de leurs oppresseurs, car je connais leurs douleurs.
\VS{8}C'est pourquoi je suis descendu pour le délivrer de la main des Egyptiens et pour le faire remonter de ce pays-là, dans un pays bon et vaste, dans un pays découlant de lait et de miel~; au lieu où sont les Cananéens, les Héthiens, les Amoréens, les Phéréziens, les Héviens et les Jébusiens.
\VS{9}Et maintenant, voici le cri des enfants d'Israël est parvenu à moi, et j'ai vu aussi l'oppression dont les Egyptiens les oppriment.
\VS{10}Maintenant donc viens, et je t'enverrai vers Pharaon~; et tu retireras mon peuple, les enfants d'Israël, hors d'Egypte\FTNT{Os. 12:14~; Mi. 6:4.}.
\VS{11}Et Moïse répondit à Dieu~: Qui suis-je, moi, pour aller vers Pharaon, et pour retirer de l'Egypte les enfants d'Israël~?
\VS{12}Et Dieu lui dit~: Va, car je serai avec toi. Et tu auras ce signe que c'est moi qui t'envoie~: C'est que quand tu auras retiré mon peuple d'Egypte, vous servirez Dieu près de cette montagne.
\TextTitle{Yahweh révèle son Nom à Moïse}
\VS{13}Et Moïse dit à Dieu~: Voici, quand je serai venu vers les enfants d'Israël, et que je leur aurai dit~: Le Dieu de vos pères m'a envoyé vers vous, s'ils me disent alors~: Quel est son Nom~? Que leur dirai-je~?
\VS{14} Et Dieu dit à Moïse~: JE SUIS CELUI QUI SUIS. Il dit aussi~: Tu diras ainsi aux enfants d'Israël~: Celui qui s'appelle JE SUIS\FTNT{Je suis («~Ehyeh~» en hébreu), c'est de là que vient le Nom de Yahweh. Or le Nom de Jésus signifie «~Yahweh est Salut~». Dieu révèle son Nom à Moïse~: «~Je suis celui qui suis~». Or Jésus-Christ s'est ouvertement attribué ce Nom en Jn. 8:58. N'ayant compris ni le plan de Dieu ni qui était celui qui les visitait, les religieux Juifs ont voulu le lapider car ils estimaient qu'il blasphémait. Car en déclarant être «~Je suis~», Jésus-Christ proclamait ouvertement sa divinité (Ro. 9:5), chose que les Juifs ne pouvaient concevoir. Dans l'évangile de Jean, Jésus déclare clairement qu'il est le «~JE SUIS~» d'Ex. 3:14. «~Je suis le pain de vie~» (Jn. 6:35), «~Je suis la lumière du monde~» (Jn. 8:12), «~Je suis le bon berger~» (Jn. 10:11), «~Je suis la porte~» (Jn. 10:7), «~Je suis la résurrection~» (Jn. 11:25), «~Je suis le chemin, la vérité et la vie~» (Jn. 14:6), «~Je suis la vraie vigne~» (Jn. 15:1).}, m'a envoyé vers vous.
\VS{15}Dieu dit encore à Moïse~: Tu diras ainsi aux enfants d'Israël~: Yahweh, le Dieu de vos pères, le Dieu d'Abraham, le Dieu d'Isaac et le Dieu de Jacob m'a envoyé vers vous. C'est ici mon Nom éternellement, et c'est ici le souvenir que vous aurez de moi de génération en génération.
\VS{16}Va, et rassemble les anciens d'Israël, et dis leur~: Yahweh, le Dieu de vos pères, le Dieu d'Abraham, d'Isaac et de Jacob, m'est apparu, en disant~: Certainement je vous ai visités, et j'ai vu ce qu'on vous fait en Egypte.
\VS{17}Et j'ai dit~: Je vous ferai remonter de l'Egypte où vous êtes affligés, dans le pays des Cananéens, des Héthiens, des Amoréens, des Phéréziens, des Héviens et des Jébusiens, qui est un pays découlant de lait et de miel.
\VS{18}Et ils obéiront à ta parole~; et tu iras, toi et les anciens d'Israël, vers le roi d'Egypte, et vous lui direz~: Yahweh, le Dieu des Hébreux, est venu nous rencontrer. Et maintenant donc, laisse-nous aller, nous te prions, à trois jours de marche dans le désert, afin que nous puissions sacrifier à Yahweh, notre Dieu.
\VS{19}Or je sais que le roi d'Egypte ne vous permettra point de vous en aller, si ce n'est par une main forte.
\VS{20}Mais j'étendrai ma main et je frapperai l'Egypte par toutes les merveilles que je ferai au milieu d'elle~; et après cela, il vous laissera aller.
\VS{21}Je ferai que ce peuple trouve grâce envers les Egyptiens, et il arrivera que, quand vous partirez, vous ne vous en irez point à vide.
\VS{22}Mais chacune demandera à sa voisine, et à l'hôtesse de sa maison, des vases d'argent, des vases d'or, et des vêtements, que vous mettrez sur vos fils et sur vos filles~: Ainsi vous dépouillerez les Egyptiens.
\Chap{4}
\TextTitle{Moïse résiste en évoquant l'incrédulité du peuple}
\VerseOne{}Et Moïse répondit, et dit~: Mais voici, ils ne me croiront pas et n'obéiront pas à ma parole~; car ils diront~: Yahweh ne t'est point apparu.
\VS{2}Et Yahweh lui dit~: Qu'est-ce que tu as dans ta main~? Il répondit~: Une verge.
\VS{3}Et Dieu lui dit~: Jette-la par terre~; il la jeta par terre et elle devint un serpent. Et Moïse s'enfuyait de devant lui.
\VS{4}Et Yahweh dit à Moïse~: Etends ta main et saisis sa queue~; et il étendit sa main et l'empoigna~; et il redevint une verge dans sa main.
\VS{5}C'est là ce que tu feras, afin qu'ils croient que Yahweh, le Dieu de leurs pères, le Dieu d'Abraham, le Dieu d'Isaac et le Dieu de Jacob, t'est apparu.
\VS{6}Yahweh lui dit encore~: Mets maintenant ta main dans ton sein, et il mit sa main dans son sein~; puis il la tira~; et voici, sa main était blanche de lèpre comme la neige.
\VS{7}Et Dieu lui dit~: Remets ta main dans ton sein~; et il remit sa main dans son sein~; puis il la retira hors de son sein~; et voici, elle était redevenue comme son autre chair.
\VS{8}Mais s'il arrive qu'ils ne te croient point, et qu'ils n'obéissent point à la voix du premier signe, ils croiront à la voix du second signe.
\VS{9}S'il arrive qu'ils ne croient point à ces deux signes et qu'ils n'obéissent point à ta parole, tu prendras de l'eau du fleuve et tu la répandras sur la terre, et les eaux que tu auras prises du fleuve deviendront du sang sur la terre.
\TextTitle{Moïse résiste en évoquant son incapacité à parler}
\VS{10}Et Moïse répondit à Yahweh~: Hélas~! Seigneur~! Je ne suis point un homme qui ait, ni d'hier ni d'avant-hier, la parole aisée, ni même depuis que tu parles à ton serviteur~; car j'ai la bouche et la langue empêchées.
\VS{11}Et Yahweh lui dit~: Qui a fait la bouche de l'homme~? Ou qui a fait le muet, ou le sourd, ou le voyant, ou l'aveugle~? N'est-ce pas moi Yahweh\FTNT{Ps. 94:9.}~?
\VS{12}Va donc maintenant, je serai avec ta bouche et je t'enseignerai ce que tu auras à dire\FTNT{Lu. 12:12~; Mt. 10:19~; Mc. 13:11.}.
\VS{13}Et Moïse répondit~: Hélas~! Seigneur~! Envoie, je te prie, celui que tu dois envoyer.
\VS{14}Et la colère de Yahweh s'enflamma contre Moïse, et il lui dit~: Aaron, le Lévite, n'est-il pas ton frère~? Je sais qu'il parlera très bien, et même le voilà qui sort à ta rencontre, et quand il te verra, il se réjouira dans son cœur.
\VS{15}Tu lui parleras donc et tu mettras ces paroles dans sa bouche~; je serai avec ta bouche et avec la sienne, et je vous enseignerai ce que vous aurez à faire.
\VS{16}Et il parlera pour toi au peuple, et ainsi il te sera pour bouche, et tu lui seras pour Dieu.
\VS{17}Tu prendras aussi dans ta main cette verge, avec laquelle tu feras ces signes-là.
\TextTitle{Moïse accepte sa mission et part en Egypte}
\VS{18}Ainsi Moïse s'en alla, et retourna vers Jéthro, son beau-père, et lui dit~: Je te prie, que je m'en aille, et que je retourne vers mes frères qui sont en Egypte, pour voir s'ils vivent encore. Et Jéthro lui dit~: Va en paix~!
\VS{19}Or Yahweh dit à Moïse au pays de Madian~: Va, et retourne en Egypte~; car tous ceux qui cherchaient ta vie sont morts.
\VS{20}Moïse prit sa femme et ses fils, les mit sur un âne, et retourna au pays d'Egypte. Moïse prit aussi la verge de Dieu dans sa main.
\VS{21}Et Yahweh avait dit à Moïse~: Quand tu t'en iras pour retourner en Egypte, tu prendras garde à tous les miracles que j'ai mis dans ta main~; et tu les feras devant Pharaon~; mais j'endurcirai son cœur et il ne laissera point aller le peuple.
\VS{22}Tu diras donc à Pharaon, ainsi parle Yahweh~: Israël est mon fils, mon premier-né\FTNT{Os. 11:1.}.
\VS{23}Et je t'ai dit~: Laisse aller mon fils, afin qu'il me serve. Mais tu as refusé de le laisser aller~: Voici, je m'en vais tuer ton fils, ton premier-né.
\VS{24}Or il arriva que, comme Moïse était en chemin dans l'hotellerie, Yahweh le rencontra et chercha à le faire mourir.
\VS{25}Et Séphora prit un couteau tranchant, coupa le prépuce de son fils et le jeta à ses pieds, et dit~: Certes, tu es pour moi un époux de sang~!
\VS{26}Alors Yahweh se retira de lui~; et Séphora dit~: Epoux de sang~; à cause de la circoncision.
\TextTitle{Yahweh envoie Aaron vers Moïse}
\VS{27}Et Yahweh dit à Aaron~: Va dans le désert, au-devant de Moïse. Il y alla donc, et le rencontra sur la montagne de Dieu et l'embrassa.
\VS{28}Et Moïse raconta à Aaron toutes les paroles de Yahweh qui l'avait envoyé, et tous les signes qu'il lui avait ordonné de faire.
\VS{29}Moïse donc poursuivit son chemin avec Aaron~; et ils assemblèrent tous les anciens des enfants d'Israël.
\VS{30}Et Aaron rapporta toutes les paroles que Yahweh avait dites à Moïse, et il exécuta les signes aux yeux du peuple.
\VS{31}Et le peuple crut. Ils apprirent que Yahweh avait visité les enfants d'Israël, qu'il avait vu leur affliction~; et ils s'inclinèrent et se prosternèrent.
\Chap{5}
\TextTitle{Pharaon s'oppose à Moïse\FTNTT{Ex. 5-14.}}
\VerseOne{}Après cela, Moïse et Aaron se rendirent ensuite auprès de Pharaon et lui dirent~: Ainsi parle Yahweh, le Dieu d'Israël~: Laisse aller mon peuple, afin qu'il me célèbre une fête solennelle dans le désert.
\VS{2}Mais Pharaon dit~: Qui est Yahweh pour que j'obéisse à sa voix et que je laisse aller Israël~? Je ne connais point Yahweh et je ne laisserai point aller Israël.
\VS{3}Et ils dirent~: Le Dieu des Hébreux est venu au-devant de nous. Permets-nous de faire trois journées de marche dans le désert, et que nous sacrifions à Yahweh, notre Dieu~; de peur qu'il ne se jette sur nous par la peste ou par l'épée.
\VS{4}Et le roi d'Egypte leur dit~: Moïse et Aaron, pourquoi détournez-vous le peuple de son ouvrage~? Allez maintenant à vos charges.
\VS{5}Pharaon dit aussi~: Voici, le peuple de ce pays est maintenant en grand nombre, et vous lui feriez cesser leur travail~!
\VS{6}Et ce jour-là, Pharaon donna cet ordre aux oppresseurs établis sur le peuple et à ses commissaires, en disant~:
\VS{7}Vous ne donnerez plus de paille à ce peuple pour faire des briques comme auparavant, mais qu'ils aillent s'amasser de la paille.
\VS{8}Néanmoins, vous leur imposerez la quantité de briques qu'ils faisaient auparavant, sans en rien diminuer~; car ils sont paresseux, et c'est pour cela qu'ils crient, en disant~: Allons et sacrifions à notre Dieu~!
\VS{9}Que la servitude soit aggravée sur ces gens-là, et qu'ils s'occupent, et ne s'amusent plus à des paroles de mensonge.
\VS{10}Alors les oppresseurs du peuple et ses commissaires sortirent et dirent au peuple~: Ainsi parle Pharaon~: Je ne vous donnerai plus de paille.
\VS{11}Allez vous-mêmes et prenez de la paille où vous en trouverez~; mais il ne sera rien diminué de votre travail.
\VS{12}Alors le peuple se répandit par tout le pays d'Egypte, pour ramasser du chaume au lieu de paille.
\VS{13}Et les oppresseurs les pressaient en disant~: Achevez vos ouvrages, chaque jour sa tâche, comme lorsque la paille vous était fournie.
\VS{14}Même les commissaires des enfants d'Israël, que les oppresseurs de Pharaon avaient établis sur eux, furent battus, et on leur dit~: Pourquoi n'avez-vous point achevé votre tâche en faisant des briques hier et aujourd'hui, comme auparavant~?
\VS{15}Alors les commissaires des enfants d'Israël vinrent crier à Pharaon, en disant~: Pourquoi fais-tu ainsi à tes serviteurs~?
\VS{16}On ne donne point de paille à tes serviteurs, et toutefois on nous dit~: Faites des briques. Et voici, tes serviteurs sont battus, et ton peuple est traité comme coupable.
\VS{17}Et il répondit~: Vous êtes des paresseux, des paresseux~! C'est pourquoi vous dites~: Allons, sacrifions à Yahweh~!
\VS{18}Maintenant donc allez, travaillez~; car on ne vous donnera point de paille, et vous rendrez la même quantité de briques.
\VS{19}Les commissaires des enfants d'Israël virent qu'ils souffraient, puisqu'on disait~: Vous ne diminuerez rien de vos briques sur la tâche de chaque jour.
\VS{20}Et en sortant de chez Pharaon, ils rencontrèrent Moïse et Aaron, qui se trouvèrent au-devant d'eux~;
\VS{21}et ils leur dirent~: Que Yahweh vous regarde, et en juge, vu que vous nous avez mis en mauvaise odeur devant Pharaon et devant ses serviteurs, leur mettant l'épée à la main pour nous tuer.
\VS{22}Alors Moïse retourna vers Yahweh, et dit~: Seigneur~! Pourquoi as-tu fait maltraiter ce peuple~? Pourquoi m'as-tu envoyé~?
\VS{23}Car depuis que je suis allé vers Pharaon pour parler en ton Nom, il a maltraité ce peuple, et tu n'as point délivré ton peuple.
\Chap{6}
\TextTitle{Yahweh fortifie Moïse et rappelle son alliance avec Israël}
\VerseOne{}Et Yahweh dit à Moïse~: Tu verras maintenant ce que je ferai à Pharaon~; car il les laissera aller y étant contraint par une main puissante, étant, dis-je contraint par ma main puissante, il les chassera de son pays.
\VS{2}Dieu parla encore à Moïse et lui dit~: Je suis Yahweh.
\VS{3}Je suis apparu à Abraham, à Isaac et à Jacob, comme le Dieu Tout-Puissant, mais je n'ai point été connu d'eux par mon Nom YAHWEH.
\VS{4}J'ai aussi fait cette alliance avec eux, que je leur donnerai le pays de Canaan, le pays de leurs pèlerinages, dans lequel ils ont demeuré comme étrangers.
\VS{5}Et j'ai entendu les sanglots des enfants d'Israël, que les Egyptiens tiennent esclaves, et je me suis souvenu de mon alliance~;
\VS{6}c'est pourquoi dis aux enfants d'Israël~: Je suis Yahweh, et je vous retirerai de dessous les charges des Egyptiens, et je vous délivrerai de leur servitude, je vous rachèterai à bras étendu, et par de grands jugements.
\VS{7}Et je vous prendrai pour être mon peuple, je vous serai Dieu~; et vous connaîtrez que je suis Yahweh, votre Dieu, qui vous retire de dessous les charges des Egyptiens.
\VS{8}Et je vous ferai entrer dans le pays au sujet duquel j'ai levé ma main que je le donnerai à Abraham, à Isaac et à Jacob, et je vous le donnerai en héritage~; je suis Yahweh.
\VS{9}Moïse donc parla de cette manière aux enfants d'Israël. Mais ils n'écoutèrent point Moïse, à cause de l'angoisse de leur esprit, et à cause de leur dure servitude.
\VS{10}Et Yahweh parla à Moïse, en disant~:
\VS{11}Va, et dis à Pharaon, roi d'Egypte, qu'il laisse sortir les enfants d'Israël de son pays.
\VS{12}Alors Moïse parla devant Yahweh, en disant~: Voici, les enfants d'Israël ne m'ont point écouté, et comment Pharaon m'écoutera-t-il, moi, qui suis incirconcis des lèvres~?
\VS{13} Mais Yahweh parla à Moïse et à Aaron, et leur ordonna d'aller trouver les enfants d'Israël, et Pharaon, roi d'Egypte, pour retirer les fils d'Israël du pays d'Egypte.
\TextTitle{Les chefs d'Israël}
\VS{14}Voici les chefs des pères~: Les fils de Ruben, premier-né d'Israël~: Hénoc et Pallu, Hetsron et Carmi~; ce sont là les familles de Ruben\FTNT{Ge. 46:9~; No. 26:5~; 1 Ch. 5:3.}.
\VS{15}Les fils de Siméon~: Jemuel, Jamin, Ohad, Jakin et Tsochar, et Saül, fils d'une Cananéenne~; ce sont là les familles de Siméon.
\VS{16}Voici les noms des fils de Lévi selon leur naissance~: Guerschon, Kehath et Merari. Les années de la vie de Lévi furent de cent trente-sept ans.
\VS{17}Les fils de Guerschon~: Libni et Schimeï, selon leurs familles.
\VS{18}Les fils de Kehath~: Amram, Jitsehar, Hébron et Uziel. Et les années de la vie de Kehath furent de cent trente-trois ans.
\VS{19}Les fils de Merari~: Machli et Muschi~; ce sont là les familles de Lévi selon leurs générations.
\VS{20}Or Amram prit Jokébed, sa tante, pour femme, qui lui enfanta Aaron et Moïse~; les années de la vie d'Amram furent de cent trente-sept ans.
\VS{21}Et les fils de Jitsehar~: Koré, Népheg et Zicri.
\VS{22}Et les fils d'Uziel~: Mischaël, Eltsaphan, et Sithri.
\VS{23}Aaron prit pour femme Elischéba, fille d'Amminadab, sœur de Nachschon, qui lui enfanta Nadab, Abihu, Eléazar et Ithamar.
\VS{24}Et les fils de Koré~: Assir, Elkana, et Abiasaph. Ce sont là les familles des Korites.
\VS{25}Eléazar, fils d'Aaron, prit pour femme une des filles de Puthiel, qui lui enfanta Phinées. Ce sont là les chefs des pères des Lévites selon leurs familles.
\VS{26}Or c'est là cet Aaron et ce Moïse à qui Yahweh dit~: Retirez les enfants d'Israël du pays d'Egypte selon leurs armées.
\VS{27}Ce sont eux qui parlèrent à Pharaon, roi d'Egypte, pour retirer d'Egypte les enfants d'Israël. C'est ce Moïse et c'est cet Aaron.
\VS{28}Le jour où Yahweh parla à Moïse dans le pays d'Egypte,
\VS{29}Yahweh parla à Moïse et dit~: Je suis Yahweh~; dis à Pharaon, roi d'Egypte, toutes les paroles que je t'ai dites.
\VS{30}Et Moïse dit en présence de Yahweh~: Voici, je suis incirconcis des lèvres, comment Pharaon m'écoutera-t-il~?
\Chap{7}
\TextTitle{L'appel de Moïse confirmé}
\VerseOne{}Et Yahweh dit à Moïse~: Voici, je t'ai établi pour être Dieu à Pharaon, et Aaron, ton frère, sera ton prophète.
\VS{2}Tu diras tout ce que je t'ordonnerai, et Aaron, ton frère, parlera à Pharaon pour qu'il laisse aller les enfants d'Israël hors de son pays.
\VS{3}J'endurcirai le cœur de Pharaon, et je multiplierai mes signes et mes miracles dans le pays d'Egypte.
\VS{4}Pharaon ne vous écoutera point~; je mettrai ma main sur l'Egypte, et je sortirai mes armées, mon peuple, les enfants d'Israël, du pays d'Egypte, par de grands jugements.
\VS{5}Les Egyptiens connaîtront\FTNT{Les nations reconnaîtront que Jésus-Christ est le Dieu d'Israël lorsqu'il reviendra en Sion pour délivrer et restaurer son peuple (Za. 14).} que je suis Yahweh quand j'aurai étendu ma main sur l'Egypte, et que j'aurai retiré du milieu d'eux les enfants d'Israël.
\VS{6}Et Moïse et Aaron firent comme Yahweh leur avait ordonné~; ils firent ainsi.
\VS{7}Or Moïse était âgé de quatre-vingts ans, et Aaron de quatre-vingt-trois ans quand ils parlèrent à Pharaon.
\TextTitle{La verge d'Aaron devient un serpent}
\VS{8}Yahweh parla à Moïse et à Aaron, en disant~:
\VS{9}Quand Pharaon vous parlera, en disant~: Faites un miracle~; tu diras alors à Aaron~: Prends ta verge, jette-la devant Pharaon et elle deviendra un serpent.
\VS{10}Moïse donc et Aaron allèrent auprès de Pharaon, et firent comme Yahweh avait ordonné~; Aaron jeta sa verge devant Pharaon et devant ses serviteurs, et elle devint un serpent.
\VS{11}Mais Pharaon fit venir aussi les sages et les enchanteurs~; et les magiciens d'Egypte, et eux aussi firent autant par leurs enchantements.
\VS{12}Ils jetèrent donc chacun leurs verges et elles devinrent des serpents~; mais la verge d'Aaron engloutit leurs verges.
\VS{13}Le cœur de Pharaon s'endurcit et il ne les écouta point~; selon ce que Yahweh avait dit.
\TextTitle{Les eaux du fleuve changées en sang}
\VS{14}Yahweh dit à Moïse~: Le cœur de Pharaon est endurci, il a refusé de laisser aller le peuple.
\VS{15}Va-t'en dès le matin vers Pharaon~; voici, il sortira pour aller près de l'eau~; tu te présenteras donc devant lui sur le bord du fleuve, et tu prendras dans ta main la verge qui a été changée en serpent.
\VS{16}Et tu lui diras~: Yahweh, le Dieu des Hébreux, m'avait envoyé vers toi pour te dire~: Laisse aller mon peuple, afin qu'il me serve au désert~; mais voici, tu ne m'as point écouté jusqu'ici.
\VS{17}Ainsi parle Yahweh~: A ceci tu sauras que je suis Yahweh~; je m'en vais frapper de la verge qui est dans ma main les eaux du fleuve, et elles seront changées en sang.
\VS{18}Et le poisson qui est dans le fleuve mourra, le fleuve deviendra puant, et les Egyptiens éprouveront du dégoût à boire des eaux du fleuve.
\VS{19}Yahweh parla aussi à Moïse~: Dis à Aaron~: Prends ta verge et étends ta main sur les eaux des Egyptiens, sur leurs rivières, sur leurs ruisseaux, et sur leurs marais, et sur tous les amas de leurs eaux, et elles deviendront du sang~; il y aura du sang par tout le pays d'Egypte, dans les vases de bois et de pierre.
\VS{20}Moïse donc et Aaron firent ce que Yahweh avait ordonné. Aaron, ayant levé la verge, en frappa les eaux du fleuve, sous les yeux de Pharaon et de ses serviteurs~; et toutes les eaux du fleuve furent changées en sang.
\VS{21}Et le poisson qui était dans le fleuve mourut, et le fleuve en devint puant, tellement que les Egyptiens ne pouvaient point boire les eaux du fleuve~; il y eut du sang dans tout le pays d'Egypte.
\VS{22}Et les magiciens d'Egypte en firent de même par leurs enchantements. Et le cœur de Pharaon s'endurcit tellement, qu'il ne les écouta point, selon ce que Yahweh avait dit.
\VS{23}Et Pharaon leur ayant tourné le dos, alla dans sa maison, et ne prit même pas à cœur ces choses qu'il avait vues.
\VS{24}Or tous les Egyptiens creusèrent autour du fleuve pour trouver de l'eau à boire, parce qu'ils ne pouvaient pas boire de l'eau du fleuve.
\VS{25}Il se passa sept jours depuis que Yahweh eut frappé le fleuve.
\TextTitle{Invasion de grenouilles}
\VS{26}Après cela, Yahweh dit à Moïse~: Va vers Pharaon et dis-lui~: Ainsi parle Yahweh~: Laisse aller mon peuple, afin qu'il me serve.
\VS{27}Si tu refuses de le laisser aller, voici, je m'en vais frapper de grenouilles toutes tes contrées~;
\VS{28}et le fleuve fourmillera de grenouilles, qui monteront et entreront dans ta maison, et dans la chambre où tu couches, et sur ton lit, et dans les maisons de tes serviteurs, et parmi tout ton peuple, dans tes fours et dans tes maies.
\VS{29}Ainsi, les grenouilles monteront sur toi, sur ton peuple et sur tous tes serviteurs.
\Chap{8}
\VerseOne{}Yahweh donc dit à Moïse~: Dis à Aaron~: Etends ta main avec ta verge sur les fleuves, sur les rivières, et sur les marais, et fais monter les grenouilles sur le pays d'Egypte.
\VS{2}Et Aaron étendit sa main sur les eaux de l'Egypte, et les grenouilles montèrent et couvrirent le pays d'Egypte.
\VS{3}Mais les magiciens firent de même par leurs enchantements et firent monter des grenouilles sur le pays d'Egypte.
\VS{4}Alors Pharaon appela Moïse et Aaron, et leur dit~: Fléchissez Yahweh par vos prières, afin qu'il retire les grenouilles de dessus moi et de dessus mon peuple~; et je laisserai aller le peuple, afin qu'ils sacrifient à Yahweh.
\VS{5}Et Moïse dit à Pharaon~: Glorifie-toi sur moi~! Pour quel temps fléchirai-je par mes prières Yahweh pour toi, et pour tes serviteurs et pour ton peuple, afin que Yahweh retire les grenouilles loin de toi et de tes maisons~? Il en demeurera seulement dans le fleuve.
\VS{6}Alors il répondit~: Pour demain. Et Moïse dit~: Il sera fait selon ta parole, afin que tu saches qu'il n'y a nul Dieu tel que Yahweh, notre Dieu.
\VS{7}Les grenouilles donc se retireront de toi, de tes maisons, de tes serviteurs et de ton peuple~; il en demeurera seulement dans le fleuve.
\VS{8}Alors Moïse et Aaron sortirent de chez Pharaon~; Moïse cria à Yahweh au sujet des grenouilles qu'il avait fait venir sur Pharaon.
\VS{9}Et Yahweh fit selon la parole de Moïse. Ainsi les grenouilles moururent dans les maisons, dans les villages et dans les champs.
\VS{10}On les amassa par monceaux, et la terre en fut infectée.
\VS{11}Mais Pharaon, voyant qu'il y avait du relâche, endurcit son cœur et ne les écouta point, selon ce que Yahweh avait dit.
\TextTitle{Invasion de poux}
\VS{12}Et Yahweh dit à Moïse~: Dis à Aaron~: Etends ta verge et frappe la poussière de la terre, et elle deviendra des poux dans tout le pays d'Egypte.
\VS{13}Et ils firent ainsi~; et Aaron étendit sa main avec sa verge, et frappa la poussière de la terre~; et elle fut changée en poux, sur les hommes et sur les bêtes~; toute la poussière du pays fut changée en poux dans tout le pays d'Egypte.
\VS{14}Et les magiciens voulurent faire de même par leurs enchantements, pour produire des poux, mais ils ne purent pas. Les poux furent donc tant sur les hommes que sur les bêtes.
\VS{15}Alors les magiciens dirent à Pharaon~: C'est ici le doigt de Dieu\FTNT{Lu. 11:20.}~! Toutefois, le cœur de Pharaon s'endurcit et il ne les écouta point, selon ce que Yahweh avait dit.
\TextTitle{Invasion de mouches}
\VS{16}Puis Yahweh dit à Moïse~: Lève-toi de bon matin, et présente-toi devant Pharaon~; voici, il sortira près de l'eau, et tu lui diras~: Ainsi parle Yahweh~: Laisse aller mon peuple, afin qu'il me serve.
\VS{17}Car si tu ne laisses pas aller mon peuple, voici, je m'en vais envoyer contre toi, contre tes serviteurs, contre ton peuple et contre tes maisons, un mélange d'insectes~; et les maisons des Egyptiens seront remplies de ce mélange, et la terre aussi sur laquelle ils seront \FTNT{Ps. 105:31~; Ps. 78:43.}.
\VS{18}Mais je distinguerai ce jour-là le pays de Gosen, où se tient mon peuple, tellement qu'il n'y aura nul mélange d'insectes~; afin que tu saches que je suis Yahweh au milieu de la terre.
\VS{19}Et je ferai la différence entre ton peuple et mon peuple~; demain, ce signe-là se fera.
\VS{20}Et Yahweh le fit ainsi~; et un grand mélange d'insectes entra dans la maison de Pharaon et dans chaque maison de ses serviteurs, et dans tout le pays d'Egypte, de sorte que la terre fut gâtée par ce mélange.
\TextTitle{Pharaon tente de compromettre Moïse}
\VS{21} Et Pharaon appela Moïse et Aaron, et leur dit~: Allez, sacrifiez à votre Dieu dans ce pays.
\VS{22}Mais Moïse dit~: Il n'est pas convenable de faire ainsi~; car nous sacrifierions à Yahweh, notre Dieu, l'abomination des Egyptiens. Voici, si nous sacrifions l'abomination des Egyptiens devant leurs yeux, ne nous lapideraient-ils pas~?
\VS{23}Nous irons le chemin de trois jours au désert, et nous sacrifierons à Yahweh, notre Dieu, comme il nous dira.
\VS{24}Alors Pharaon dit~: Je vous laisserai aller pour sacrifier dans le désert à Yahweh, votre Dieu~; toutefois, vous ne vous éloignerez pas en y allant. Fléchissez Yahweh pour moi, par vos prières.
\VS{25}Moïse dit~: Voici, je sors de chez toi et je supplierai Yahweh, afin que le mélange d'insectes se retire demain de Pharaon, de ses serviteurs, et de son peuple. Mais que Pharaon ne continue point à se moquer en ne laissant point aller le peuple pour sacrifier à Yahweh.
\VS{26}Alors Moïse sortit de chez Pharaon et fléchit Yahweh par la prière.
\VS{27}Et Yahweh fit selon la parole de Moïse~; et le mélange d'insectes se retira de Pharaon, de ses serviteurs et de son peuple~; il n'en resta pas un seul insecte.
\VS{28}Mais Pharaon endurcit son cœur cette fois encore et ne laissa point aller le peuple.
\Chap{9}
\TextTitle{La mort des troupeaux}
\VerseOne{}Alors Yahweh dit à Moïse~: Va vers Pharaon et dis-lui~: Ainsi parle Yahweh, le Dieu des Hébreux~: Laisse aller mon peuple, afin qu'il me serve.
\VS{2}Car si tu refuses de les laisser aller et si tu le retiens encore,
\VS{3}voici, la main de Yahweh sera sur ton bétail qui est dans les champs, tant sur les chevaux que sur les ânes, sur les chameaux, sur les bœufs, et sur les brebis, et il y aura une très grande mortalité.
\VS{4}Et Yahweh distinguera le bétail des Israélites du bétail des Egyptiens, afin que rien de ce qui est aux enfants d'Israël ne meure.
\VS{5}Et Yahweh fixa un temps, en disant~: Demain, Yahweh fera ceci dans le pays.
\VS{6}Yahweh donc fit cela dès le lendemain~; et tout le bétail des Egyptiens mourut~; mais du bétail des enfants d'Israël, il ne mourut pas une seule bête.
\VS{7}Et Pharaon envoya examiner, et voici, il n'y avait pas une seule bête morte du bétail des enfants d'Israël. Toutefois, le cœur de Pharaon s'endurcit, et il ne laissa point aller le peuple.
\TextTitle{Des ulcères sur les Egyptiens et les bêtes}
\VS{8}Alors Yahweh dit à Moïse et à Aaron~: Remplissez vos mains de cendre de fournaise~; et que Moïse les répande vers les cieux en la présence de Pharaon.
\VS{9}Et elles deviendront de la poussière sur tout le pays d'Egypte, et il s'en fera des ulcères bourgeonnant en pustules tant sur les hommes que sur les bêtes, dans tout le pays d'Egypte.
\VS{10}Ils prirent donc de la cendre de fournaise et se tinrent devant Pharaon~; Moïse la répandit vers les cieux et il se forma des ulcères bourgeonnant en pustules tant sur les hommes que sur les bêtes.
\VS{11} Et les magiciens ne purent se tenir devant Moïse, à cause des ulcères~; car les magiciens avaient des ulcères, comme tous les Egyptiens.
\VS{12} Et Yahweh endurcit le cœur de Pharaon, et il ne les écouta point selon ce que Yahweh avait dit à Moïse.
\TextTitle{L'Egypte frappée par la grêle et le feu}
\VS{13}Puis Yahweh parla à Moïse~: Lève-toi de bon matin, et présente-toi devant Pharaon, et dis-lui~: Ainsi parle Yahweh, le Dieu des Hébreux~: Laisse aller mon peuple, afin qu'il me serve.
\VS{14}Car cette fois, je vais faire venir toutes mes plaies contre ton cœur, sur tes serviteurs et sur ton peuple, afin que tu saches qu'il n'y a nul Dieu semblable à moi sur toute la terre.
\VS{15}Car maintenant si j'avais étendu ma main, je t'aurais frappé de la peste, toi et ton peuple, et tu serais effacé de la terre.
\VS{16}Mais certainement, je t'ai fait subsister pour te faire voir ma puissance, afin que mon Nom soit célébré sur toute la terre\FTNT{Ro. 9:17.}.
\VS{17}T'élèves-tu encore contre mon peuple, pour ne point le laisser aller~?
\VS{18}Voici, je m'en vais faire pleuvoir demain à cette même heure, une grêle tellement forte qu'il n'y en a point eu de semblable en Egypte, depuis le jour où elle fut fondée jusqu'à maintenant.
\VS{19}Maintenant envoie rassembler ton bétail et tout ce que tu as à la campagne~; car la grêle tombera sur tous les hommes, sur le bétail qui se trouvera à la campagne, et qu'on n'aura pas renfermé, et ils mourront.
\VS{20}Celui d'entre les serviteurs de Pharaon, qui craignit la parole de Yahweh, fit promptement retirer dans les maisons ses serviteurs et ses bêtes.
\VS{21}Mais celui qui n'appliqua point son cœur à la parole de Yahweh, laissa ses serviteurs et ses bêtes à la campagne.
\VS{22}Et Yahweh dit à Moïse~: Etends ta main vers les cieux, et il y aura de la grêle sur tout le pays d'Egypte, sur les hommes et sur les bêtes, et sur toutes les herbes des champs au pays d'Egypte.
\VS{23}Moïse donc étendit sa verge vers les cieux, et Yahweh envoya des tonnerres et de la grêle, et le feu se promenait sur la terre. Yahweh fit pleuvoir de la grêle sur le pays d'Egypte.
\VS{24}Il y eut donc de la grêle et du feu entremêlé avec la grêle, laquelle était si grosse qu'il n'y en avait point eu de semblable sur toute la terre d'Egypte, depuis qu'elle a été habitée.
\VS{25}La grêle frappa dans tout le pays d'Egypte tout ce qui était aux champs, depuis les hommes jusqu'aux bêtes. La grêle frappa aussi toutes les herbes des champs et brisa tous les arbres des champs.
\VS{26}Il n'y eut que la contrée de Gosen, dans laquelle étaient les enfants d'Israël, où il n'y eut point de grêle.
\TextTitle{Pharaon continue d'endurcir son cœur}
\VS{27}Alors Pharaon envoya appeler Moïse et Aaron, et leur dit~: J'ai péché cette fois~; Yahweh est juste, mais moi et mon peuple sommes méchants.
\VS{28}Fléchissez par des prières Yahweh~: Que ce soit assez, et que Dieu ne fasse plus tonner ni grêler, car je vous laisserai aller, et on ne vous arrêtera plus.
\VS{29}Alors Moïse dit~: Aussitôt que je sortirai de la ville, j'étendrai mes mains vers Yahweh et les tonnerres cesseront. Il n'y aura plus de grêle, afin que tu saches que la terre est à Yahweh\FTNT{Ps. 24:1.}.
\VS{30}Mais quant à toi et tes serviteurs, je sais que vous ne craindrez pas encore Yahweh Dieu.
\VS{31}Or le lin et l'orge avaient été frappés, car l'orge était en épis et c'était la floraison du lin.
\VS{32} Mais le blé et l'épeautre ne furent point frappés, parce qu'ils sont tardifs.
\VS{33}Moïse donc sortit de chez Pharaon pour aller hors de la ville. Il étendit ses mains vers Yahweh, et les tonnerres cessèrent, et la grêle et la pluie ne tombèrent plus sur la terre.
\VS{34}Pharaon, voyant que la pluie, la grêle, et les tonnerres avaient cessé, continua encore à pécher, et il endurcit son cœur, lui et ses serviteurs.
\VS{35}Le cœur donc de Pharaon s'endurcit et il ne laissa point aller les enfants d'Israël, selon ce que Yahweh avait dit par l'intermédiaire de Moïse.
\Chap{10}
\TextTitle{Invasion de sauterelles}
\VerseOne{} Et Yahweh dit à Moïse~: Va vers Pharaon, car j'ai endurci son cœur et le cœur de ses serviteurs, afin que je mette au-dedans de lui les signes que je m'en vais faire~;
\VS{2}et afin que tu racontes à ton fils et au fils de ton fils, les signes que j'accomplirai sur les Egyptiens et les prodiges que je ferai au milieu d'eux, et que vous sachiez que je suis Yahweh.
\VS{3}Moïse donc et Aaron vinrent vers Pharaon, et lui dirent~: Ainsi parle Yahweh, le Dieu des Hébreux~: Jusqu'à quand refuseras-tu de t'humilier devant moi~? Laisse aller mon peuple, afin qu'il me serve.
\VS{4}Car si tu refuses de laisser aller mon peuple, voici, je ferai venir demain des sauterelles dans ton territoire.
\VS{5}Elles couvriront la face de la terre, et l'on ne pourra plus voir la terre~; elles dévoreront le reste de ce qui a échappé, ce que la grêle vous a laissé~; et elles dévoreront tous les arbres qui poussent dans vos champs.
\VS{6}Et elles rempliront tes maisons, et les maisons de tous tes serviteurs, et les maisons de tous les Egyptiens~; ce que tes pères n'ont point vu ni les pères de tes pères, depuis qu'ils existent sur la terre jusqu'à ce jour. Puis, ayant tourné le dos à Pharaon, il sortit d'auprès de lui.
\VS{7}Et les serviteurs de Pharaon lui dirent~: Jusqu'à quand celui-ci nous sera-t-il un piège~? Laisse aller ces gens et qu'ils servent Yahweh, leur Dieu. Attendras-tu de savoir avant cela que l'Egypte est perdue~?
\VS{8}Alors on fit revenir Moïse et Aaron vers Pharaon, il leur dit~: Allez, servez Yahweh, votre Dieu. Qui sont tous ceux qui iront~?
\VS{9} Et Moïse répondit~: Nous irons avec nos jeunes gens et nos vieillards, avec nos fils et nos filles~; nous irons avec nos brebis et nos bœufs~; car nous avons à célébrer une fête solennelle à Yahweh.
\VS{10}Alors il leur dit~: Que Yahweh soit avec vous, comme je laisserai aller vos petits enfants~! Prenez garde, car le mal est devant vous.
\VS{11}Il n'en sera pas ainsi que vous l'avez demandé~; mais vous, hommes, allez maitenant et servez Yahweh~; car c'est ce que vous demandiez. Et on les chassa de la présence de Pharaon.
\VS{12}Alors Yahweh dit à Moïse~: Etends ta main sur le pays d'Egypte, pour faire venir les sauterelles, afin qu'elles montent sur le pays d'Egypte, qu'elles dévorent toute l'herbe de la terre, tout ce que la grêle a laissé.
\VS{13}Moïse étendit donc sa verge sur le pays d'Egypte~; et Yahweh amena sur le pays, tout ce jour-là et toute la nuit, un vent d'orient~; le matin vint, et le vent d'orient enleva les sauterelles.
\VS{14}Et il fit monter les sauterelles sur tout le pays d'Egypte, et les mit dans toutes les contrées d'Egypte~; elles étaient fort grosses et il y n'en avait point eu avant elles de semblables, et il n'y en aura point de semblables après elles.
\VS{15}Et elles couvrirent la face de tout le pays, tellement que le pays en fut obscurci~; elles dévorèrent toute l'herbe de la terre, tout le fruit des arbres que la grêle avait laissé~; il ne resta aucune verdure aux arbres ni aux herbes des champs, dans tout le pays d'Egypte.
\VS{16}Aussitôt Pharaon se hâta d'appeler Moïse et Aaron, et dit~: J'ai péché contre Yahweh, votre Dieu, et contre vous.
\VS{17}Mais pardonne, je te prie, mon péché, pour cette fois seulement~; et suppliez Yahweh, votre Dieu, par vos prières, afin qu'il retire de moi cette mort-ci seulement.
\VS{18}Il sortit donc de chez Pharaon, et fléchit Yahweh par ses prières.
\VS{19}Et Yahweh fit lever un vent d'occident très fort qui enleva les sauterelles et les précipita dans la Mer Rouge. Il ne resta pas une seule sauterelle dans tout le territoire de l'Egypte.
\VS{20}Mais Yahweh endurcit le cœur de Pharaon et il ne laissa point aller les enfants d'Israël.
\TextTitle{Les ténèbres sur les Egyptiens}
\VS{21}Puis Yahweh dit à Moïse~: Etends ta main vers les cieux, qu'il y ait sur le pays d'Egypte des ténèbres si épaisses, qu'on puisse les toucher à la main.
\VS{22}Moïse étendit donc sa main vers les cieux, et il y eut d'épaisses ténèbres dans tout le pays d'Egypte, pendant trois jours\FTNT{Ps. 105:28.}.
\VS{23}On ne se voyait pas l'un l'autre, et nul ne se leva de sa place pendant trois jours. Mais pour tous les enfants d'Israël, il y eut de la lumière dans le lieu de leurs demeures.
\TextTitle{Pharaon tente encore de compromettre Moïse}
\VS{24}Alors Pharaon appela Moïse et dit~: Allez, servez Yahweh~; que vos brebis et vos bœufs seuls demeurent~; vos petits enfants iront aussi avec vous.
\VS{25}Moïse répondit~: Tu mettras toi-même entre nos mains de quoi faire des sacrifices et des holocaustes, que nous ferons à Yahweh, notre Dieu.
\VS{26}Et même, nos troupeaux viendront aussi avec nous, il n'en restera pas un sabot. Car nous en prendrons pour servir Yahweh, notre Dieu~; car nous ne savons pas ce que nous choisirons pour offrir à Yahweh, jusqu'à ce que nous soyons arrivés en ce lieu là.
\VS{27}Mais Yahweh endurcit le cœur de Pharaon et il ne voulut point les laisser aller.
\VS{28}Et Pharaon lui dit~:Va-t-en~!Arrière de moi~! Garde-toi de revoir ma face, car le jour où tu verras ma face, tu mourras.
\VS{29}Alors Moïse répondit~: Tu as bien dit, je ne reverrai plus ta face\FTNT{Hé. 11:27.}.
\Chap{11}
\TextTitle{Pharaon méprise l'avertissement sur la mort des premiers-nés}
\VerseOne{}Or Yahweh dit à Moïse~: Je ferai venir encore une plaie sur Pharaon, et sur l'Egypte, et après cela il vous laissera aller d'ici~; il vous laissera entièrement aller, et vous chassera tout à fait d'ici.
\VS{2}Parle maintenant aux oreilles du peuple, et dis leur~: Que chacun demande à son voisin, et chacune à sa voisine, des vases d'argent et des vases d'or.
\VS{3}Or Yahweh fit trouver grâce au peuple devant les Egyptiens~; et même Moïse passait pour un grand homme dans le pays d'Egypte, tant parmi les serviteurs de Pharaon que parmi le peuple.
\VS{4}Et Moïse dit~: Ainsi parle Yahweh~: Vers le milieu de la nuit, je passerai au travers de l'Egypte~;
\VS{5}et tout premier-né mourra dans le pays d'Egypte, depuis le premier-né de Pharaon, qui devait être assis sur son trône, jusqu'au premier-né de la servante qui est derrière la meule, et jusqu'à tous les premiers-nés des bêtes.
\VS{6}Et il y aura un grand cri dans tout le pays d'Egypte, tel qu'il n'y en a jamais eu et qu'il n'y en aura jamais de semblable.
\VS{7}Mais contre tous les enfants d'Israël, un chien même ne remuera point sa langue, depuis l'homme jusqu'aux bêtes~; afin que vous sachiez que Dieu fera la différence entre les Egyptiens et les Israélites.
\VS{8}Et tous tes serviteurs viendront vers moi, et se prosterneront devant moi, en disant~: Sors, toi, et tout le peuple qui est avec toi. Après cela, je sortirai. Ainsi, Moïse sortit de chez Pharaon dans une ardente colère.
\VS{9}Yahweh donc dit à Moïse~: Pharaon ne vous écoutera point, afin que mes miracles soient multipliés dans le pays d'Egypte.
\VS{10}Et Moïse et Aaron firent tous ces miracles-là devant Pharaon. Et Yahweh endurcit le cœur de Pharaon, tellement qu'il ne laissa point aller les enfants d'Israël hors de son pays.
\Chap{12}
\TextTitle{La première Pâque}
\VerseOne{}Or Yahweh dit à Moïse et à Aaron dans le pays d'Egypte~:
\VS{2}Ce mois-ci sera pour vous le premier des mois, il sera pour vous le premier des mois de l'année.
\VS{3}Parlez à toute l'assemblée d'Israël, en disant~: Jusqu'au dixième jour de ce mois, que chacun prenne un petit d'entre les brebis ou d'entre les chèvres, selon les familles des pères~; un petit, dis-je, d'entre les brebis ou d'entre les chèvres, par famille.
\VS{4}Mais si la famille est moindre qu'il ne faut pour manger un petit d'entre les brebis ou d'entre les chèvres, qu'elle le prenne avec son voisin qui est près de sa maison, selon le nombre de personnes~; vous compterez combien il en faudra pour manger d'entre les brebis ou d'entre les chèvres, ayant égard à ce que chacun de vous peut manger.
\VS{5}Or le petit d'entre les brebis ou d'entre les chèvres sera sans défaut, et sera un mâle ayant un an\FTNT{La Pâque juive était célébrée le 14ème jour du premier mois de l'année juive soit, le 14 du mois de Nissan (Ex. 12:2~; No. 9:1-5). L'agneau pascal était une préfiguration de Jésus-Christ~: L'agneau de Dieu qui ôte le péché du monde (Jn. 1:29). Ses caractéristiques sont les suivantes~: \\- L'agneau devait nécessairement être un mâle sans défaut (Ex. 12:5). Jésus est l'enfant mâle mis au monde par une vierge, il n'a pas été affecté par le sang corrompu d'Adam, il est donc sans défaut (Es. 7:14~; Mt. 1:20-21). Pour être certains de la perfection de l'animal, les hébreux devaient l'examiner pendant quatre jours avant de l'immoler (Ex. 12:3-6). Il est à noter que la loi juive exigeait que deux ou trois témoins soient présents pour constater un crime ou un péché (De. 17:6~; De. 19:15), ces quatre jours font donc office de quatre témoins pour attester de la pureté de l'animal. De même, les quatre auteurs de l'évangile attestent la sainteté du Seigneur. De plus, avant sa mise à mort, le Seigneur a été examiné par deux législations~: juive (le sanhédrin) et romaine (Ponce Pilate). Ces deux législations attestèrent, malgré elles, son innocence (Mt. 25:60~; Mt. 27:24~; Mc. 14:55-56~; Mc. 15:14~; Lu. 23:4~; Jn. 18:31~; Jn. 19:6) et confirmèrent qu'il était sans défaut et donc digne d'être offert en sacrifice.\\- Yahweh avait prescrit aux hébreux d'immoler l'agneau entre les deux soirs (Ex. 12:6), c'est-à-dire avant le crépuscule, entre la neuvième et la onzième heure. Jésus fut arrêté la nuit de Pâque (Mc. 14:12-41). Sa crucifixion eut lieu le lendemain, à la troisième heure (Mc. 15:25), et sa mort survint à la neuvième heure (Mt. 27:45). L'agneau devait être rôti au feu puis consommé avec du pain sans levain et des herbes amères (Ex. 12:8). Le feu symbolise le jugement que le Seigneur a pris sur lui à cause de nos péchés (Es. 53:5~; Ro. 4:25~; 1 Pi. 1:18-20). Le pain sans levain est une autre image de Jésus, le pain de vie (Jn. 6:35) sans aucun péché (1 Co. 5:8). Les herbes amères préfigurent, quant à elles, l'affliction et la souffrance du Seigneur (Hé. 2:10).}~; vous le prendrez d'entre les brebis ou d'entre les chèvres~;
\VS{6}et vous le garderez jusqu'au quatorzième jour de ce mois~; et toute la congrégation de l'assemblée d'Israël l'égorgera entre les deux soirs.
\VS{7} Et ils prendront de son sang, et le mettront sur les deux poteaux et sur le linteau de la porte des maisons où ils le mangeront.
\VS{8} Et ils en mangeront la chair rôtie au feu cette nuit-là~; et ils la mangeront avec des pains sans levain, et avec des herbes amères.
\VS{9}N'en mangez rien à demi cuit, ni qui ait été bouilli dans l'eau~; mais qu'il soit rôti au feu, sa tête, ses jambes et ses entrailles.
\VS{10} Et ne laissez aucun reste jusqu'au matin, mais s'il en reste quelque chose le matin, vous le brûlerez au feu.
\VS{11}Et vous le mangerez ainsi~: Vos reins seront ceints, vous aurez vos souliers à vos pieds, et votre bâton à la main, et vous le mangerez à la hâte. C'est la Pâque de Yahweh.
\TextTitle{Le sang qui sauve~; l'instauration de la fête de la Pâque}
\VS{12}Car je passerai cette nuit-là par le pays d'Egypte, et je frapperai tout premier-né au pays d'Egypte, depuis les hommes jusqu'aux bêtes~; et j'exercerai des jugements sur tous les dieux de l'Egypte. Je suis Yahweh.
\VS{13}Et le sang sera pour vous un signe sur les maisons où vous serez~; car je verrai le sang et je passerai par-dessus vous, et il n'y aura point de plaie à destruction quand je frapperai le pays d'Egypte.
\VS{14}Et ce jour là, vous conserverez le souvenir de ce jour, et vous le célébrerez comme une fête solennelle à Yahweh~; vous le célébrerez comme une fête solennelle par une ordonnance perpétuelle de génération en génération.
\VS{15}Vous mangerez pendant sept jours des pains sans levain, et dès le premier jour, vous ôterez le levain de vos maisons~; car quiconque mangera du pain levé, depuis le premier jour jusqu'au septième, cette personne-là sera retranchée d'Israël.
\VS{16}Au premier jour il y aura une sainte convocation, et il y aura de même au septième jour une sainte convocation~; il ne se fera aucune œuvre dans ces jours-là~; seulement, on vous apprêtera à manger ce qu'il faudra pour chaque personne.
\VS{17}Vous prendrez donc garde aux pains sans levain, parce qu'en ce même jour, j'aurai retiré vos armées du pays d'Egypte~; vous observerez donc ce jour-là de génération en génération par une ordonance perpétuelle.
\VS{18}Au premier mois, le quatorzième jour du mois, au soir, vous mangerez des pains sans levain jusqu'au vingt et unième jour du mois, au soir.
\VS{19}Il ne se trouvera point de levain dans vos maisons pendant sept jours, car quiconque mangera du pain levé, cette personne-là sera retranchée de l'assemblée d'Israël, tant celui qui habite comme étranger que celui qui est né au pays.
\VS{20}Vous ne mangerez point de pain levé~; mais vous mangerez dans tous les lieux où vous demeurerez des pains sans levain.
\VS{21}Moïse donc appela tous les anciens d'Israël et leur dit~: Choisissez et prenez un petit d'entre les brebis ou d'entre les chèvres selon vos familles, et égorgez la Pâque.
\VS{22}Puis vous prendrez un bouquet d'hysope et le tremperez dans le sang qui sera dans un bassin, et vous arroserez du sang qui sera dans le bassin, le linteau et les deux poteaux~; et nul de vous ne sortira de la porte de sa maison jusqu'au matin.
\VS{23}Car Yahweh passera pour frapper l'Egypte et il verra le sang sur le linteau et sur les deux poteaux, et Yahweh passera par-dessus la porte, et ne permettra point que le destructeur entre dans vos maisons pour frapper.
\VS{24}Vous garderez ceci comme une ordonnance perpétuelle pour toi et pour tes fils.
\VS{25}Quand donc vous serez entrés dans le pays que Yahweh vous donnera, selon qu'il en a parlé, vous observerez ce service.
\VS{26}Et quand vos fils vous diront~: Que signifie pour vous ce service~?
\VS{27}Alors vous répondrez~: C'est le sacrifice de la Pâque à Yahweh, qui passa en Egypte par-dessus les maisons des enfants d'Israël, quand il frappa l'Egypte, et qu'il préserva nos maisons. Alors le peuple s'inclina et se prosterna.
\VS{28}Ainsi les enfants d'Israël s'en allèrent et firent comme Yahweh l'ordonna à Moïse et à Aaron, ils le firent ainsi.
\TextTitle{Les premiers-nés d'Egypte frappés}
\VS{29}Et il arriva qu'à minuit Yahweh frappa tous les premiers-nés du pays d'Egypte, depuis le premier-né de Pharaon, qui devait être assis sur son trône, jusqu'aux premiers-nés des captifs qui étaient dans la prison, et tous les premiers-nés des bêtes.
\VS{30}Et Pharaon se leva de nuit, lui et ses serviteurs, et tous les Egyptiens~; et il y eut un grand cri en Egypte, parce qu'il n'y avait point de maison où il n'y ait eu un mort\FTNT{Hé. 11:28~; No. 8:17~; Ps. 78:51~; Ps. 105:36.}.
\TextTitle{Israël sort d'Egypte}
\VS{31}Il appela donc Moïse et Aaron de nuit, et leur dit~: Levez-vous, sortez du milieu de mon peuple, tant vous que les enfants d'Israël, allez et servez Yahweh, comme vous en avez parlé.
\VS{32}Prenez aussi votre menu et gros bétail, comme vous en avez parlé, et allez-vous-en et bénissez-moi.
\VS{33}Et les Egyptiens pressaient le peuple et se hâtaient de les faire sortir du pays, car ils disaient~: Nous sommes tous morts.
\VS{34}Le peuple donc prit sa pâte avant qu'elle fût levée, ayant leurs mains liées avec leurs vêtements, sur leurs épaules.
\VS{35}Or les enfants d'Israël firent selon la parole de Moïse, et demandèrent aux Egyptiens des vases d'argent et d'or, et des vêtements.
\VS{36}Et Yahweh fit trouver grâce au peuple auprès des Egyptiens, qui les leur prêtèrent~; de sorte qu'ils dépouillèrent les Egyptiens.
\VS{37}Ainsi, les fils d'Israël étant partis de Ramsès, vinrent à Succoth, environ six cent mille hommes de pied, sans les enfants.
\VS{38}Il s'en alla aussi avec eux un grand nombre de toutes sortes de gens~; et du menu et du gros bétail, en fort grands troupeaux.
\VS{39} Or parce qu'ils avaient été chassés d'Egypte, et qu'ils n'avaient pas pu tarder plus longtemps, et que même ils n'avaient fait aucune provision, ils cuisirent par gâteaux sans levain, la pâte qu'ils avaient emportée d'Egypte~; car ils ne l'avaient point fait lever. 
\VS{40}Or le séjour des enfants d'Israël en Egypte fut de quatre cent trente ans\FTNT{Ge. 15:13~; Ac. 7:6~; Ga. 3:17.}.
\VS{41}Il arriva donc au bout de quatre cent trente ans, il arriva dis-je, en ce propre jour-là, que toutes les armées de Yahweh sortirent du pays d'Egypte.
\VS{42}C'est la nuit qui doit être soigneusement observée en l'honneur de Yahweh, parce qu'alors il les retira du pays d'Egypte~; cette nuit-là est à observer en l'honneur de Yahweh, par tous les enfants d'Israël de génération en génération\FTNT{De. 16:1-6.}.
\VS{43}Yahweh dit aussi à Moïse et à Aaron~: C'est ici l'ordonnance de la Pâque~: Aucun étranger n'en mangera~;
\VS{44}mais tout esclave qu'on aura acheté par argent sera circoncis, et alors il en mangera.
\VS{45}L'étranger et le mercenaire n'en mangeront point.
\VS{46}On la mangera dans une même maison, et vous n'emporterez point de sa chair hors de la maison, et vous n'en casserez point les os.
\VS{47}Toute l'assemblée d'Israël la fera.
\VS{48}Et si quelque étranger qui habite chez toi veut faire la Pâque à Yahweh, que tout mâle qui lui appartient soit circoncis~; et alors il s'approchera pour la faire, et il sera comme celui qui est né dans le pays~; mais aucun incirconcis n'en mangera.
\VS{49}Il y aura une même loi pour celui qui est né dans le pays et pour l'étranger qui habite parmi vous.
\VS{50}Tous les enfants d'Israël firent ce que Yahweh avait ordonné à Moïse et à Aaron~; ils le firent ainsi.
\VS{51}Il arriva donc en ce même jour que Yahweh retira les enfants d'Israël du pays d'Egypte, selon leurs armées.
\Chap{13}
\TextTitle{Consécration des premiers-nés à Yahweh}
\VerseOne{}Et Yahweh parla à Moïse, et dit~:
\VS{2}Sanctifie-moi tout premier-né, tout premier-né issu du sein maternel parmi les fils d'Israël, tant des hommes que des bêtes, car il est à moi\FTNT{Lé. 27:26-27~; No. 3:13~; No. 8:17~; Lu. 2:22-23.}.
\VS{3}Moïse donc dit au peuple~: Souvenez-vous de ce jour où vous êtes sortis d'Egypte, de la maison de servitude~; car Yahweh vous en a retirés par sa main puissante~; on ne mangera donc point de pain levé.
\VS{4}Vous sortez aujourd'hui dans le mois où les épis mûrissent.
\VS{5}Quand donc Yahweh t'aura introduit dans le pays des Cananéens, des Héthiens, des Amoréens, des Héviens et des Jébusiens, qu'il a juré à tes pères de te donner, et qui est un pays découlant de lait et de miel, alors tu feras ce service durant ce mois-ci.
\VS{6}Pendant sept jours tu mangeras des pains sans levain, et au septième jour il y aura une fête solennelle à Yahweh.
\VS{7}On mangera durant sept jours des pains sans levain~; il ne sera point vu chez toi de pain levé et même il ne sera point vu de levain dans toutes tes contrées.
\VS{8}Et ce jour-là, tu feras entendre ces choses à tes enfants, en disant~: C'est à cause de ce que Yahweh m'a fait en me retirant d'Egypte.
\VS{9}Et ceci te sera pour signe sur ta main, et comme un rappel entre tes yeux, afin que la loi de Yahweh soit dans ta bouche, car Yahweh t'aura retiré d'Egypte par sa main puissante\FTNT{De. 6:8~; De. 11:18.}.
\VS{10}Tu observeras cette ordonnance au jour fixé d'année en année.
\VS{11}Aussi, quand Yahweh t'aura introduit dans le pays des Cananéens, selon qu'il a juré à toi et à tes pères, et qu'il te l'aura donné,
\VS{12}tu consacreras à Yahweh tout premier-né issu du sein de sa mère, même tout premier-né des animaux que tu auras~; les mâles appartiendront à Yahweh.
\VS{13}Et tu rachèteras avec un petit d'entre les brebis ou d'entre les chèvres, tout premier-né de l'ânesse, et si tu ne le rachètes point, tu lui briseras la nuque. Tu rachèteras aussi tout premier-né des hommes parmi tes fils.
\VS{14}Et quand ton fils t'interrogera à l'avenir, en disant~: Que veut dire ceci~? Alors tu lui diras~: Yahweh nous a retirés par main forte hors d'Egypte, de la maison de servitude.
\VS{15}Car il arriva que, quand Pharaon s'obstinait à ne point nous laisser aller, Yahweh tua tous les premiers-nés au pays d'Egypte, depuis les premiers-nés des hommes jusqu'aux premiers-nés des bêtes. Voilà pourquoi je sacrifie à Yahweh tout premier-né mâle issu du sein de sa mère, et je rachète tout premier-né de mes fils.
\VS{16}Ceci te sera donc pour signe sur ta main, et pour fronteaux entre tes yeux, que Yahweh nous a retirés d'Egypte par sa main puissante.
\TextTitle{Début du voyage, Yahweh dirige son peuple}
\VS{17}Or lorsque Pharaon laissa aller le peuple, Dieu ne les conduisit point par le chemin du pays des Philistins, bien qu'il fût le plus court~; car Dieu dit~: C'est afin qu'il n'arrive que le peuple se repente quand il verra la guerre, et qu'il ne retourne en Egypte.
\VS{18}Mais Dieu fit tourner le peuple par le chemin du désert, vers la Mer Rouge. Ainsi, les enfants d'Israël montèrent en armes hors du pays d'Egypte.
\VS{19}Et Moïse avait pris avec lui les ossements de Joseph, parce que Joseph avait expressément fait jurer les enfants d'Israël, en leur disant~: Dieu vous visitera très certainement, et vous transporterez donc avec vous mes ossements d'ici\FTNT{Ge. 50:25~; Jos. 24:32.}.
\VS{20}Et ils partirent de Succoth, et campèrent à Etham, qui est à l'extrémité du désert.
\VS{21}Et Yahweh allait devant eux, de jour dans une colonne de nuée pour les conduire par le chemin~; et de nuit dans une colonne de feu pour les éclairer, afin qu'ils marchent jour et nuit\FTNT{No. 9:13-23~; No. 10:43~; De. 1:33~; Né. 9:12-19~; 1 Co. 10:1.}.
\VS{22}Et il ne retira point la colonne de nuée le jour, ni la colonne de feu la nuit de devant le peuple.
\Chap{14}
\TextTitle{Pharaon et son armée à la poursuite d'Israël}
\VerseOne{}Et Yahweh parla à Moïse et dit~:
\VS{2}Parle aux enfants d'Israël et dis-leur: Qu'ils se détournent, et qu'ils campent devant Pi-Hahiroth, entre Migdol et la mer, vis-à-vis de Baal-Tsephon. Vous camperez vis-à-vis de ce lieu-là près de la mer\FTNT{No. 33:7.}.
\VS{3}Pharaon dira des enfants d'Israël~: Ils sont confus dans le pays, le désert les a enfermés.
\VS{4}Et j'endurcirai le cœur de Pharaon, et il vous poursuivra. Ainsi je serai glorifié en Pharaon et en toute son armée et les Egyptiens sauront que je suis Yahweh~; et ils firent ainsi.
\VS{5}Or on avait rapporté au roi d'Egypte que le peuple s'enfuyait, et le cœur de Pharaon et de ses serviteurs fut changé à l'égard du peuple, et ils dirent~: Qu'est-ce que nous avons fait en laissant aller Israël, de sorte qu'il ne nous servira plus~?
\VS{6}Alors il fit atteler son char et il prit son peuple avec lui.
\VS{7}Il prit donc six cents chars d'élite et tous les chars de l'Egypte~; et il y avait des capitaines sur tout cela.
\VS{8}Et Yahweh endurcit le cœur de Pharaon, roi d'Egypte, qui poursuivit les enfants d'Israël. Or les fils d'Israël étaient sortis à main levée\FTNT{Lé. 26:13~; No. 33:3.}.
\VS{9}Les Egyptiens donc les poursuivirent~; et tous les chevaux des chars de Pharaon, ses cavaliers et son armée les atteignirent comme ils étaient campés près de la mer, vers Pi-Hahiroth vis-à-vis de Baal-Tsephon.
\VS{10}Et Pharaon approchait. Les enfants d'Israël levèrent leurs yeux, et voici, les Egyptiens marchaient après eux. Et les fils d'Israël eurent une grande frayeur et crièrent à Yahweh.
\VS{11}Ils dirent aussi à Moïse~: Est-ce qu'il n'y avait pas des sépulcres en Egypte pour que tu nous aies emmenés pour mourir au désert~? Que nous as-tu fait en nous faisant sortir d'Egypte~?
\VS{12}N'est-ce pas ce que nous te disions en Egypte, en disant~: Retire-toi de nous et que nous servions les Egyptiens~? Car nous aimons mieux les servir que de mourir au désert.
\TextTitle{Délivrance miraculeuse par Yahweh}
\VS{13}Et Moïse dit au peuple~: Ne craignez point, arrêtez-vous et voyez la délivrance que Yahweh vous donnera aujourd'hui~; car les Egyptiens que vous voyez aujourd'hui, vous ne les verrez plus.
\VS{14}Yahweh combattra pour vous et vous demeurerez tranquilles.
\VS{15}Or Yahweh avait dit à Moïse~: Que cries-tu à moi~? Parle aux enfants d'Israël, qu'ils marchent.
\VS{16}Et toi, élève ta verge, étends ta main sur la mer, et fends-la~; et que les enfants d'Israël entrent au milieu de la mer à sec.
\VS{17} Et quant à moi, voici, je m'en vais endurcir le cœur des Egyptiens, afin qu'ils entrent après eux~; et je serai glorifié en Pharaon, et en toute son armée, en ses chars et en ses cavaliers.
\VS{18}Et les Egyptiens sauront que je suis Yahweh, quand j'aurai été glorifié en Pharaon, avec ses chars et ses cavaliers.
\VS{19}Et l'Ange de Dieu qui allait devant le camp d'Israël partit, et s'en alla derrière eux~; et la colonne de nuée partit de devant eux et se tint derrière eux.
\VS{20}Et elle vint entre le camp des Egyptiens et le camp d'Israël. Elle était aux uns une nuée et une obscurité~; et pour les autres, elle les éclairait la nuit. L'un des camps n'approcha point de l'autre durant toute la nuit.
\VS{21}Or Moïse avait étendu sa main sur la mer, et Yahweh fit reculer la mer toute la nuit par un vent d'orient qui souffla avec puissance~; il mit la mer à sec, et les eaux se fendirent\FTNT{Jos. 4:23~; Ps. 66:6~; Ps. 106:9~; Hé. 11:29.}.
\VS{22}Et les enfants d'Israël entrèrent au milieu de la mer à sec, et les eaux leur servaient de mur à droite et à gauche.
\VS{23}Et les Egyptiens les poursuivirent~; et ils entrèrent après eux au milieu de la mer, à savoir tous les chevaux de Pharaon, ses chars et ses cavaliers.
\VS{24}Mais il arriva que sur la veille du matin, Yahweh étant dans la colonne de feu et dans la nuée, regarda le camp des Egyptiens et le mit en déroute.
\VS{25}Il ôta les roues de leurs chars et alourdit leur marche. Alors les Egyptiens dirent~: Fuyons de devant les Israëlites, car Yahweh combat pour eux contre les Egyptiens.
\VS{26}Et Yahweh dit à Moïse~: Etends ta main sur la mer, et les eaux retourneront sur les Egyptiens, sur leurs chars et sur leurs cavaliers.
\VS{27}Moïse donc étendit sa main sur la mer, et la mer reprit son impétuosité vers le matin. Et les Egyptiens s'enfuyant rencontrèrent la mer qui s'était rejointe~; et ainsi Yahweh jeta les Egyptiens au milieu de la mer.
\VS{28}Car les eaux retournèrent et couvrirent les chars et les cavaliers de toute l'armée de Pharaon, qui étaient entrés après les Israélites dans la mer, et il n'en resta pas un seul.
\VS{29}Mais les enfants d'Israël marchèrent au milieu de la mer à sec, et les eaux leur servaient de mur à droite et à gauche.
\VS{30}Ainsi, Yahweh délivra, en ce jour-là, Israël de la main des Egyptiens~; et Israël vit sur le bord de la mer les Egyptiens morts.
\VS{31}Israël vit donc la grande puissance que Yahweh avait déployée contre les Egyptiens~; et le peuple craignit Yahweh, ils crurent en Yahweh, et en Moïse, son serviteur.
\Chap{15}
\TextTitle{Cantique de délivrance}
\VerseOne{}Alors Moïse et les enfants d'Israël chantèrent ce cantique à Yahweh, et dirent~: Je chanterai à Yahweh, car il est hautement élevé~; il a jeté dans la mer le cheval et celui qui le montait.
\VS{2}Yahweh est ma force et ma louange, et il a été mon Sauveur, mon Dieu. Je lui dresserai un tabernacle, c'est le Dieu de mon père, je l'exalterai.
\VS{3}Yahweh est un vaillant guerrier, son Nom est Yahweh.
\VS{4}Il a jeté dans la mer les chars de Pharaon et son armée~; l'élite de ses capitaines a été submergée dans la Mer Rouge.
\VS{5}Les gouffres les ont couverts, ils sont descendus au fond des eaux comme une pierre\FTNT{Né. 9:11.}.
\VS{6}Ta droite, ô Yahweh, s'est montrée magnifique en force~! Ta droite, ô Yahweh, a brisé l'ennemi\FTNT{Ps. 118:15-16~; Ps. 77:16.}~!
\VS{7}Tu as ruiné par la grandeur de ta majesté ceux qui s'élevaient contre toi~; tu as lâché ta colère et elle les a consumés comme du chaume.
\VS{8}Par le souffle de tes narines, les eaux ont été amoncelées~; les eaux courantes se sont arrêtés comme un monceau~; les gouffres ont été gelés au milieu de la mer.
\VS{9}L'ennemi disait~: Je poursuivrai, j'atteindrai, je partagerai le butin~; mon âme sera assouvie d'eux, je tirerai mon épée, ma main les détruira.
\VS{10}Tu as soufflé de ton vent, la mer les a couverts~; ils ont été enfoncés comme du plomb au plus profond des eaux.
\VS{11}Qui est comme toi parmi les dieux, ô Yahweh~! Qui est comme toi, magnifique en sainteté, digne d'être révéré et célébré, faisant des choses merveilleuses~?
\VS{12}Tu as étendu ta droite, la terre les a engloutis.
\VS{13}Tu as conduit par ta miséricorde ce peuple que tu as racheté~; tu l'as conduit par ta force à la demeure de ta sainteté.
\VS{14}Les peuples l'ont entendu, et ils en ont tremblé~; la douleur a saisi les habitants du pays des Philistins.
\VS{15}Alors les princes d'Edom seront troublés, et le tremblement saisira les puissants de Moab, tous les habitants de Canaan se fondront.
\VS{16}La frayeur et l'épouvante tomberont sur eux~; ils seront rendus muets comme une pierre par la grandeur de ton bras, jusqu'à ce que ton peuple soit passé, ô Yahweh~! Jusqu'à ce que ce peuple que tu as acquis soit passé\FTNT{De. 2:25~; De. 11:25~; Jos. 2:9.}.
\VS{17}Tu les introduiras et les planteras sur la montagne de ton héritage, au lieu que tu as préparé pour ta demeure, ô Yahweh~! Au lieu saint, ô Seigneur, que tes mains ont établi~!
\VS{18}Yahweh régnera à jamais et à perpétuité.
\VS{19}Car les chevaux de Pharaon, ses chars et ses cavaliers sont entrés dans la mer, et Yahweh a fait retourner sur eux les eaux de la mer~; mais les enfants d'Israël ont marché à sec au milieu de la mer.
\VS{20}Et Marie, la prophétesse, sœur d'Aaron, prit un tambour dans sa main, et toutes les femmes sortirent après elle, avec des tambours et des flûtes.
\VS{21}Et Marie leur répondait~: Chantez à Yahweh, car il est hautement élevé~; il a jeté dans la mer le cheval et celui qui le montait.
\TextTitle{Yahweh pourvoit pour son peuple}
\VS{22}Après cela, Moïse fit partir les Israélites de la Mer Rouge, et ils partirent vers le désert de Schur~; et ayant marché trois jours dans le désert, ils ne trouvèrent point d'eau.
\VS{23}De là, ils vinrent à Mara, mais ils ne purent boire les eaux de Mara, parce qu'elles étaient amères~; c'est pourquoi ce lieu fut appelé Mara.
\VS{24}Et le peuple murmura contre Moïse en disant~: Que boirons-nous~?
\VS{25}Et Moïse cria à Yahweh, et Yahweh lui montra\FTNT{«~Montra~» de l'hébreu «~yarah~» qui veut également dire «~enseigner~», «~signaler~», «~lancer~», «~instruire~», «~informer~», «~montrer~», «~jeter~» etc.} un certain bois qu'il jeta dans les eaux~; et les eaux devinrent douces. Il lui proposa là une ordonnance et une loi, et il l'éprouva là,
\VS{26}et lui dit~: Si tu écoutes attentivement la voix de Yahweh, ton Dieu, si tu fais ce qui est droit devant lui, si tu prêtes l'oreille à ses commandements, si tu gardes toutes ses ordonnances, je ne ferai venir sur toi aucune des infirmités que j'ai fait venir sur l'Egypte, car je suis Yahweh qui te guérit\FTNT{De. 7:12-15.}.
\VS{27}Puis ils vinrent à Elim, où il y avait douze fontaines d'eau, et soixante-dix palmiers. Et ils campèrent là, près des eaux.
\Chap{16}
\TextTitle{Yahweh envoie la manne}
\VerseOne{}Et toute l'assemblée des enfants d'Israël étant partie d'Elim, vint au désert de Sin, qui est entre Elim et Sinaï, le quinzième jour du second mois après qu'ils furent sortis du pays d'Egypte.
\VS{2}Et toute l'assemblée des enfants d'Israël murmura dans ce désert contre Moïse et Aaron.
\VS{3}Et les enfants d'Israël leur dirent~: Ah~! Pourquoi ne sommes-nous point morts par la main de Yahweh dans le pays d'Egypte, quand nous étions assis près des pots de viande, et que nous mangions du pain à satiété~? Car vous nous avez amenés dans ce désert pour faire mourir de faim toute cette assemblée\FTNT{1 Co. 10:10~; No. 11:4.}.
\VS{4}Et Yahweh dit à Moïse~: Voici, je vais vous faire pleuvoir des cieux du pain, et le peuple sortira et en recueillera chaque jour la provision d'un jour, afin que je l'éprouve, pour voir s'il observera ma loi ou non.
\VS{5}Mais qu'ils apprêtent au sixième jour ce qu'ils auront apporté, et qu'il y ait le double de ce qu'ils recueilleront chaque jour.
\VS{6}Moïse donc et Aaron dirent à tous les enfants d'Israël~: Ce soir vous saurez que Yahweh vous a tirés du pays d'Egypte.
\VS{7}Et au matin vous verrez la gloire de Yahweh, parce qu'il a entendu vos murmures, qui sont contre Yahweh~; car que sommes-nous pour que vous murmuriez contre nous~?
\VS{8}Moïse dit donc~: Ce sera quand Yahweh vous aura donné ce soir de la chair à manger, et qu'au matin, il vous aura rassasiés de pain, parce qu'il a entendu vos murmures, par lesquels vous avez murmuré contre lui. Car que sommes-nous~? Vos murmures ne sont pas contre nous, mais contre Yahweh.
\VS{9}Et Moïse dit à Aaron~: Dis à toute l'assemblée des enfants d'Israël~: Approchez-vous de la présence de Yahweh, car il a entendu vos murmures.
\VS{10}Or il arriva qu'aussitôt qu'Aaron eut parlé à toute l'assemblée des enfants d'Israël, ils regardèrent vers le désert, et voici, la gloire de Yahweh se montra dans la nuée.
\VS{11} Et Yahweh parla à Moïse, en disant~:
\VS{12}J'ai entendu les murmures des enfants d'Israël. Parle-leur et dis-leur~: Entre les deux soirs, vous mangerez de la chair, et au matin vous serez rassasiés de pain~; et vous saurez que je suis Yahweh, votre Dieu.
\VS{13}Sur le soir donc, il monta des cailles qui couvrirent le camp, et au matin il y eut une couche de rosée autour du camp.
\VS{14}Et cette couche de rosée étant évanouie, voici, sur la surface du désert, quelque chose de menu et de rond, comme du grain sur la terre.
\VS{15}Ce que les enfants d'Israël ayant vu, ils se dirent l'un à l'autre~: Qu'est-ce~? Car ils ne savaient ce que c'était. Et Moïse leur dit~: C'est le pain que Yahweh vous donne à manger\FTNT{Ps. 105:40.}.
\TextTitle{Récolte de la manne}
\VS{16}Or ce que Yahweh a ordonné, c'est que chacun en recueille autant qu'il lui en faut pour sa nourriture, un homer par tête, selon le nombre de vos personnes~; chacun en prendra pour ceux qui sont dans sa tente.
\VS{17}Les enfants d'Israël firent donc ainsi~; et les uns en recueillirent plus, les autres moins.
\VS{18}Et ils le mesuraient par homer~; et celui qui en avait recueilli beaucoup n'en avait pas plus qu'il ne lui en fallait~; ni celui qui en avait recueilli peu, n'en avait pas moins~; mais chacun en recueillait selon ce qu'il en pouvait manger.
\VS{19}Et Moïse leur avait dit~: Que personne n'en laisse rien de reste jusqu'au matin.
\VS{20}Mais il y en eut qui n'obéirent point à Moïse, car quelques-uns en réservèrent jusqu'au matin~; et il s'y engendra des vers, et cela puait. Et Moïse se mit en grande colère contre eux.
\VS{21}Ainsi, chacun en recueillait tous les matins autant qu'il lui en fallait pour se nourrir, et lorsque la chaleur du soleil était venue, elle se fondait.
\VS{22}Mais le sixième jour, ils recueillirent du pain en double, deux homers pour chacun~; et les principaux de l'assemblée vinrent pour le rapporter à Moïse.
\TextTitle{Le sabbat\FTNTT{Né. 9:13-14~; Mt. 12:1.}}
\VS{23} Et il leur dit~: C'est ce que Yahweh a dit~: Demain est le repos, le sabbat consacré à Yahweh~; faites cuire ce que vous avez à cuire, et faites bouillir ce que vous avez à bouillir, et serrez tout ce qui sera de surplus, pour le garder jusqu'au matin.
\VS{24}Ils le serrèrent donc jusqu'au matin, comme Moïse l'avait ordonné, et il ne pua point, et il n'y eut point de vers dedans.
\VS{25}Alors Moïse dit~: Mangez-le aujourd'hui, car c'est aujourd'hui le repos de Yahweh~; aujourd'hui vous n'en trouverez point dans les champs.
\VS{26}Durant six jours vous le recueillerez, mais le septième est le sabbat, il n'y en aura point ce jour-là.
\VS{27}Et au septième jour, quelques-uns du peuple sortirent pour en recueillir, mais ils n'en trouvèrent point.
\VS{28}Et Yahweh dit à Moïse~: Jusqu'à quand refuserez-vous de garder mes commandements et mes lois~?
\VS{29}Considérez que Yahweh vous a ordonné le sabbat, c'est pourquoi il vous donne au sixième jour du pain pour deux jours~; que chacun demeure au lieu où il sera, et qu'aucun ne sorte du lieu où il est le septième jour.
\VS{30}Le peuple donc se reposa le septième jour.
\VS{31}Et la maison d'Israël nomma ce pain manne\FTNT{Le mot «~manne~» vient de l'hébreu «~man~» et veut dire «~Qu'est-ce que cela~?~». La manne est une image de Jésus, le Pain de vie descendu du ciel (Jn. 6:32-52). La consommation quotidienne du Pain de vie, qui est aussi la Parole de Dieu, apporte la vie éternelle.}. Elle était comme de la semence de coriandre blanche, et ayant le goût d'un gâteau au miel.
\VS{32}Et Moïse dit~: Voici ce que Yahweh a ordonné~: Qu'on en remplisse un homer pour le garder pour vos générations, afin qu'on voie le pain que je vous ai fait manger au désert, après vous avoir retirés du pays d'Egypte.
\VS{33}Moïse dit à Aaron~: Prends un vase, et mets-y un plein d'homer de manne, et pose-le devant Yahweh, afin qu'il soit conservé pour vos générations.
\VS{34}Et Aaron le posa devant le témoignage pour y être gardé, selon que le Seigneur l'avait ordonné à Moïse.
\VS{35}Et les enfants d'Israël mangèrent la manne durant quarante ans, jusqu'à leur arrivée dans un pays habité~; ils mangèrent, dis-je, la manne, jusqu'à leur arrivée aux frontières du pays de Canaan.
\VS{36}Or un homer est la dixième partie d'un épha.
\Chap{17}
\TextTitle{Miracle de l'eau qui sort du rocher}
\VerseOne{}Et toute l'assemblée des enfants d'Israël partit du désert de Sin, selon l'ordre de marche que Yahweh leur avait ordonné, et ils campèrent à Rephidim, où il n'y avait point d'eau à boire pour le peuple.
\VS{2}Et le peuple se souleva contre Moïse et ils lui dirent~: Donnez-nous de l'eau à boire. Et Moïse leur dit~: Pourquoi vous soulevez-vous contre moi~? Pourquoi tentez-vous Yahweh\FTNT{No. 20:2-5.}~?
\VS{3}Le peuple donc eut soif en ce lieu-là, par faute d'eau~; et ainsi le peuple murmura contre Moïse, en disant~: Pourquoi nous as-tu fait monter hors d'Egypte, pour nous faire mourir de soif, nous, nos enfants, et nos troupeaux~?
\VS{4}Et Moïse cria à Yahweh, en disant~: Que ferai-je à ce peuple~? Encore un peu, et ils me lapideront.
\VS{5}Et Yahweh répondit à Moïse: Passe devant le peuple, et prends avec toi des anciens d'Israël, prends aussi dans ta main la verge avec laquelle tu as frappé le fleuve, et viens~!
\VS{6}Voici, je vais me tenir là devant toi sur le rocher d'Horeb~; et tu frapperas le rocher, et il en sortira des eaux, et le peuple en boira. Moïse donc fit ainsi aux yeux des anciens d'Israël\FTNT{De. 9:8~; Ps. 78:15~; 1 Co. 10:4.}.
\VS{7}Et il nomma le lieu Massa et Meriba, à cause de la querelle des enfants d'Israël, et parce qu'ils avaient tenté Yahweh, en disant~: Yahweh est-il au milieu de nous ou non~?
\TextTitle{Bataille et victoire contre Amalek}
\VS{8}Alors Amalek vint et livra bataille contre Israël à Rephidim\FTNT{De. 25:17-18.}.
\VS{9}Et Moïse dit à Josué~: Choisis-nous des hommes, et sors pour combattre contre Amalek~; et je me tiendrai demain sur le sommet de la colline, et la verge de Dieu sera dans ma main.
\VS{10}Et Josué fit comme Moïse lui avait ordonné en combattant contre Amalek. Mais Moïse, Aaron et Hur montèrent au sommet de la colline.
\VS{11}Et il arrivait que lorsque Moïse élevait sa main, Israël était alors le plus fort, mais quand il reposait sa main, alors Amalek était le plus fort.
\VS{12}Et les mains de Moïse étant devenues pesantes, ils prirent une pierre et la mirent sous lui, et il s'assit dessus~; Aaron et Hur soutenaient ses mains, l'un d'un côté, et l'autre de l'autre côté~; et ainsi ses mains furent fermes jusqu'au soleil couchant.
\VS{13}Josué donc défit Amalek et son peuple au tranchant de l'épée.
\VS{14}Et Yahweh dit à Moïse~: Ecris ceci pour mémoire dans un livre, et fais entendre à Josué que j'effacerai entièrement la mémoire d'Amalek de dessous les cieux.
\VS{15}Et Moïse bâtit un autel et le nomma Yahweh, ma bannière.
\VS{16}Il dit aussi~: Parce que la main a été levée contre le trône de Yahweh, Yahweh aura toujours la guerre contre Amalek.
\Chap{18}
\TextTitle{Jéthro conseille Moïse}
\VerseOne{}Or Jéthro, prêtre de Madian, beau-père de Moïse, apprit toutes les choses que Yahweh avait faites à Moïse, et à Israël, son peuple, à savoir comment Yahweh avait retiré Israël de l'Egypte.
\VS{2}Jéthro, beau-père de Moïse, prit Séphora la femme de Moïse, après que Moïse l'eut renvoyée,
\VS{3}et les deux fils de cette femme, dont l'un s'appelait Guerschom, car il avait dit~: J'habite un pays étranger~;
\VS{4}et l'autre Eliézer, car il avait dit~: Le Dieu de mon père m'a secouru et m'a délivré de l'épée de Pharaon.
\VS{5}Jéthro donc, beau-père de Moïse, vint vers Moïse avec ses fils et sa femme au désert, où il était campé, à la montagne de Dieu.
\VS{6}Il fit dire à Moïse~: Jéthro, ton beau-père, vient vers toi, et ta femme et ses deux fils avec elle.
\VS{7}Et Moïse sortit au-devant de son beau-père, et s'étant prosterné, il l'embrassa~; et ils s'enquirent l'un de l'autre, de leur santé, puis ils entrèrent dans la tente.
\VS{8}Et Moïse raconta à son beau-père toutes les choses que Yahweh avait faites à Pharaon et aux Egyptiens en faveur d'Israël, et toute la fatigue qu'ils avaient soufferte en chemin, et comment Yahweh les avait délivrés.
\VS{9}Et Jéthro se réjouit de tout le bien que Yahweh avait fait à Israël, parce qu'il les avait délivrés de la main des Egyptiens.
\VS{10}Puis Jéthro dit~: Béni soit Yahweh qui vous a délivrés de la main des Egyptiens et de la main de Pharaon, qui a, dis-je, délivré le peuple de la main des Egyptiens~!
\VS{11}Je sais maintenant que le Seigneur est plus grand que tous les dieux, car la chose même en laquelle ils se sont enorgueillis, il a eu le dessus sur eux.
\VS{12}Jéthro, beau-père de Moïse, apporta aussi un holocauste et des sacrifices pour les offrir à Dieu. Et Aaron et tous les anciens d'Israël vinrent pour manger du pain avec le beau-père de Moïse dans la présence de Dieu.
\VS{13}Et il arriva, le lendemain, comme Moïse siégeait pour juger le peuple, et que le peuple se tenait devant Moïse depuis le matin jusqu'au soir,
\VS{14}que le beau-père de Moïse vit tout ce qu'il faisait au peuple, et il lui dit~: Qu'est-ce que tu fais à l'égard de ce peuple~? Pourquoi es-tu assis seul, et tout le peuple se tient devant toi depuis le matin jusqu'au soir~?
\VS{15} Et Moïse répondit à son beau-père~: C'est que le peuple vient à moi pour s'enquérir de Dieu.
\VS{16}Quand ils ont quelque affaire, ils viennent à moi, et je juge entre l'un et l'autre, et je leur fais entendre les ordonnances de Dieu et ses lois.
\VS{17}Mais le beau-père de Moïse lui dit~: Ce que tu fais n'est pas bien.
\VS{18}Certainement, tu succomberas, toi et ce peuple qui est avec toi~; car cela est trop pesant pour toi, tu ne saurais faire cela toi seul.
\VS{19}Ecoute donc mon conseil~; je te conseillerai et Dieu sera avec toi: Sois pour ce peuple auprès de Dieu, et rapporte les causes à Dieu.
\VS{20}Et instruis-les des ordonnances et des lois~; et fais-leur connaître la voie par laquelle ils auront à marcher et ce qu'ils auront à faire.
\VS{21}Et choisis-toi d'entre tout le peuple des hommes vertueux, craignant Dieu~; des hommes véritables, haïssant le gain déshonnête, et établis-les chefs de milliers, chefs de centaines, chefs de cinquantaines et chef des dizaines.
\VS{22}Et qu'ils jugent le peuple en tout temps, mais qu'ils te rapportent toutes les grandes affaires, et qu'ils jugent toutes les petites causes~; ainsi ils te soulageront et porteront une partie de la charge avec toi.
\VS{23}Si tu fais cela, et que Dieu te l'ordonne, tu pourras subsister, et tout le peuple parviendra en paix à destination.
\VS{24}Moïse donc obéit à la parole de son beau-père, et fit tout ce qu'il lui avait dit.
\VS{25}Ainsi, Moïse choisit de tout Israël des hommes vertueux, et les établit chefs sur le peuple, chefs de milliers, chefs de centaines, chefs de cinquantaines, et chefs de dizaines,
\VS{26}lesquels devaient juger le peuple en tout temps, mais ils devaient rapporter à Moïse les choses difficiles, et juger de toutes les petites affaires.
\VS{27}Puis Moïse laissa partir son beau-père, qui s'en alla dans son pays.
\Chap{19}
\TextTitle{DEBUT DE LA PÉRIODE DE LA LOI MOSAÏQUE OU DE LA PREMIÈRE ALLIANCE}
\VerseOne{}Au premier jour du troisième mois, après que les enfants d'Israël furent sortis du pays d'Egypte, en ce même jour-là, ils vinrent au désert de Sinaï.
\VS{2}Etant donc partis de Rephidim, ils vinrent au désert de Sinaï et campèrent au désert. Et Israël campa vis-à-vis de la montagne.
\VS{3}Et Moïse monta vers Dieu, car Yahweh l'avait appelé de la montagne pour lui dire~: Tu parleras ainsi à la maison de Jacob, et tu annonceras ceci aux enfants d'Israël~:
\VS{4}Vous avez vu ce que j'ai fait aux Egyptiens~; comment je vous ai portés comme sur des ailes d'aigle et vous ai amenés à moi.
\VS{5}Maintenant donc, si vous obéissez exactement à ma voix, et si vous gardez mon alliance, vous serez aussi d'entre tous les peuples mon plus précieux joyau, car toute la terre m'appartient\FTNT{C'est ici que débute la période de la Loi ou Première Alliance. Le fait d'avoir réuni les textes de Genèse à Malachie sous l'appellation «~Ancien Testament~» a induit beaucoup de personnes en erreur quant à leur compréhension du plan de Dieu pour nos vies. Tout d'abord, l'emploi du mot «~testament~» est inapproprié puisqu'on ne peut parler de testament sans qu'il y ait au préalable la mort du testateur (Hé. 9:16-17). Certes, des animaux étaient tués sous la Loi pour couvrir les péchés. Toutefois, ces sacrifices étaient imparfaits et par conséquent prévus pour ne durer qu'un temps, en attendant le sacrifice parfait de Jésus-Christ (Hé. 10:1-14). De plus, il est évident que les animaux sacrifiés ne nous ont rien légué.\\ Ensuite, il est à noter que tous les textes classés dans ce que l'on appelle à tort «~Ancien Testament~» ne se rapportent pas exclusivement et nécessairement à la Loi. Ainsi, des prophètes, en commençant par Moïse en personne, ayant vécu sous la Loi, ont prophétisé et écrit sur d'autres sujets que la Loi, notamment sur la grâce et la fin des temps. N'oublions pas non plus que Jésus-Christ est né et a vécu sous la Loi (Ga. 4:4). En tant que juif, il l'a scrupuleusement respectée de telle sorte qu'elle fut totalement accomplie en Lui (Mt. 5:17-18~; Jn. 19:30). En conséquence, la fin de la Loi mosaïque eut lieu après la mort du Seigneur, précisément au moment où le Seigneur a dit «~Tout est accompli~», et lorsque le voile du temple s'est déchiré de haut en bas (Mt. 27:50-51~; Jn. 19:30). La Nouvelle Alliance, ou le Testament de Jésus débuta avec l'effusion de l'Esprit (Ac. 2). De la mort du Seigneur à la Pentecôte, une période de transition de cinquante jours s'est écoulée. Jésus-Christ s'est présenté pendant ce temps dans le sanctuaire céleste pour présenter son sang dans le Saint des saints. Une fois son sacrifice examiné et accepté, le Saint-Esprit qui avait été retiré de l'homme (Ge. 6:3) put de nouveau revenir habiter le cœurs des croyants.\\Mais qu'est-ce que la loi exactement~? Beaucoup de chrétiens sont dans la confusion à ce sujet. En réalité, il n'y avait pas qu'une loi mais trois sortes de lois~: Les lois morales et les lois cérémonielles qui préexistaient depuis l'éternité~; et les lois civiles qui ont débuté avec Moïse car elles ne concernaient que son peuple.\\- Les lois civiles régissaient le fonctionnement de la vie en communauté des Hébreux. Elles étaient exclusivement réservées au peuple d'Israël dans le camp puis dans le pays de Canaan (Ex. 21:1-2~; De. 23).\\- Les lois morales font référence à la nature de Dieu~: Son amour, sa justice, sa sainteté, etc. Les dix commandements, à l'exception du sabbat tel que prescrit par Moïse (Ex. 16:28-29~; Lé. 15:32), font partie des lois morales (Ex. 20:1-17). Les dix paroles ne constituent qu'une base, un résumé. Ainsi, d'autres règles morales sont énoncées tout au long des Ecritures notamment sur la sexualité (Lé. 18:1-22), l'interdiction des sacrifices humains et de l'occultisme (De. 18:10-13), le respect d'autrui et l'entraide (Lé. 19:10-18~; Lé. 19:29-36). Comme il est impossible de consigner dans un livre tous les péchés moraux, le Seigneur a inscrit les lois morales dans le cœur de l'homme afin qu'il sache instinctivement faire la différence entre le bien et mal (Ro. 2:14-15). Jésus les a résumées en ces quelques mots~: «~Tu aimeras le Seigneur ton Dieu de tout ton cœur, et de toute ton âme, et de toute ta pensée. Celui-ci est le premier et le plus grand commandement. Et le second semblable à celui-là, est~: Tu aimeras ton prochain comme toi-même.~» (Mt. 22:37-39). Ces lois sont encore en vigueur aujourd'hui et le resteront pour toujours.\\- Les lois cérémonielles étaient relatives au culte et au sanctuaire terrestre, c'est-à-dire le tabernacle puis le temple de Jérusalem (Hé. 9:1-10). Elles regroupent toutes les ordonnances concernant les sacrifices, les ablutions, les sabbats, les fêtes de Yahweh, la dîme des Lévites et des prêtres (voir commentaire en No. 18:21 et Mal. 3:10). Les livres du Lévitique et des Nombres exposent en détail toutes les ordonnances reçues par Moïse d'après le modèle céleste que Yahweh lui avait montré sur le Mont Sinaï (Ex. 26:30). Les lois cérémonielles préexistaient donc depuis l'éternité.\\Les lois cérémonielles représentent la Première Alliance qui avait pour fondement la loi morale. Or cette alliance a vieilli puis disparu car elle n'était que l'ombre des choses à venir (Hé. 8:13). En effet, elle était basée sur quatre points principaux~: Le temple, le culte centralisé, le sacrifice et les prêtres. En Christ, nous n'avons plus besoin d'un temple physique puisque nous sommes devenus les temples vivants de Yahweh (1 Co. 6:19~; Ep. 2:22). Nous pouvons désormais adorer le Seigneur en Esprit et en vérité, à tout moment et en tout lieu (Jn. 4:23). Le sacerdoce lévitique ayant été aboli, chaque enfant de Dieu est devenu un prêtre (Ap. 5:10) qui offre en sacrifice sa propre vie consacrée au Seigneur (Ro. 12:1).\\Les lois cérémonielles ont donc trouvé leur parfait accomplissement en Jésus-Christ~: Tous les sacrifices sanglants le préfiguraient, toutes les solennités ont été réalisées en Lui (voir note en Lé. 23). Christ est donc la fin de la Loi, non pas morale, mais cérémonielle (Ro. 10:4).\\Un lien étroit existe entre les lois morales et les lois cérémonielles. La loi morale est comme un diagnostic qui révèle une pathologie incurable comme le sida~: Le péché (Ro. 5:13-20~; Ro. 7:7-14). En la découvrant, l'homme se sent condamné car il réalise qu'il ne peut pas répondre aux exigences de la justice divine. La loi cérémonielle (le sang des animaux - Hé. 9:1-13~; Hé. 10:11) a donné aux hommes une sorte de trithérapie pour les soulager provisoirement de leurs péchés mais sans pour autant les ôter (guérir, délivrer, nettoyer, laver) définitivement. Seul le sang de la Nouvelle Alliance, c'est-à-dire le sang de Jésus-Christ, a pu nous délivrer une fois pour toutes (Jn. 1:29~; Hé. 9:11-26~; Hé. 10:1-23~; Ap. 1:6).}.
\VS{6}Et vous me serez un royaume de prêtres, et une nation sainte~; ce sont là les discours que tu tiendras aux enfants d'Israël.
\VS{7}Puis Moïse vint et appela les anciens du peuple, et proposa devant eux toutes ces choses-là que Yahweh lui avait ordonné.
\VS{8}Et tout le peuple répondit d'un commun accord, en disant~: Nous ferons tout ce que Yahweh a dit. Et Moïse rapporta à Yahweh toutes les paroles du peuple.
\TextTitle{Moïse doit sanctifier le peuple pour qu'il rencontre Yahweh}
\VS{9}Et Yahweh dit à Moïse~: Voici, je viendrai à toi dans une nuée épaisse, afin que le peuple entende quand je parlerai avec toi, et qu'il te croie aussi toujours~; car Moïse avait rapporté à Yahweh les paroles du peuple.
\VS{10}Yahweh dit aussi à Moïse~: Va-t'en vers le peuple, et sanctifie-les aujourd'hui et demain, et qu'ils lavent leurs vêtements.
\VS{11}Et qu'ils soient tous prêts pour le troisième jour, car au troisième jour, Yahweh descendra sur la montagne de Sinaï, à la vue de tout le peuple.
\VS{12}Or tu mettras des bornes pour le peuple tout autour, et tu diras~: Gardez-vous de monter sur la montagne et de toucher aucune de ses extrémités. Quiconque touchera la montagne sera puni de mort.
\VS{13}Aucune main ne la touchera, et certainement il sera lapidé, ou percé de flèches~; soit bête, soit homme, il ne vivra point. Quand la trompette sonnera longuement, ils monteront vers la montagne.
\VS{14}Et Moïse descendit de la montagne vers le peuple, et sanctifia le peuple, et ils lavèrent leurs vêtements.
\VS{15}Et il dit au peuple~: Soyez tous prêts pour le troisième jour, et ne vous approchez point de vos femmes.
\VS{16}Et le troisième jour au matin, il y eut des tonnerres, et des éclairs, et une grosse nuée sur la montagne, avec un très fort son de shofar, et tout le peuple dans le camp fut effrayé.
\VS{17}Alors Moïse fit sortir le peuple du camp pour aller au-devant de Dieu~; et ils s'arrêtèrent au pied de la montagne.
\VS{18}Or le mont Sinaï était tout couvert de fumée, parce que Yahweh y était descendu en feu~; et sa fumée montait comme la fumée d'une fournaise, et toute la montagne tremblait fort.
\VS{19}Et comme le son du shofar se renforçait de plus en plus, Moïse parla, et Dieu lui répondit par une voix.
\VS{20}Yahweh donc étant descendu sur la montagne de Sinaï, au sommet de la montagne, Yahweh appela Moïse au sommet de la montagne~; et Moïse y monta.
\VS{21}Et Yahweh dit à Moïse~: Descends. Somme le peuple qu'il ne rompe point les barrières pour monter vers Yahweh afin de regarder~; de peur qu'un grand nombre d'entre eux ne périsse.
\VS{22}Et même, que les prêtres qui s'approchent de Yahweh se sanctifient aussi, de peur qu'il n'arrive que Yahweh se jette sur eux.
\VS{23} Et Moïse dit à Yahweh~: Le peuple ne pourra pas monter sur la montagne de Sinaï, parce que tu nous as sommés en me disant~: Mets des bornes sur la montagne, et sanctifie-la.
\VS{24}Et Yahweh lui dit~: Va, descends~; puis tu monteras, toi, et Aaron avec toi~; mais que les prêtres et le peuple ne rompent point les bornes pour monter vers Yahweh, de peur qu'il n'arrive qu'il se jette sur eux.
\VS{25}Moïse descendit donc vers le peuple, et lui dit ces choses.
\Chap{20}
\TextTitle{Les dix paroles}
\VerseOne{}Alors Dieu prononça toutes ces paroles, disant~:
\VS{2}Je suis Yahweh, ton Dieu, qui t'ai retiré du pays d'Egypte, de la maison de servitude.
\VS{3}Tu n'auras point d'autres dieux devant ma face.
\VS{4}Tu ne te feras point d'image taillée, ni aucune ressemblance des choses qui sont là-haut aux cieux, ni ici-bas sur la terre, ni dans les eaux sous la terre\FTNT{Lé. 26:1.}.
\VS{5}Tu ne te prosterneras point devant elles, et ne les serviras point~; car je suis Yahweh, ton Dieu~; le Dieu qui est jaloux, punissant l'iniquité des pères sur les fils, jusqu'à la troisième et à la quatrième génération de ceux qui me haïssent~;
\VS{6}et faisant miséricorde en mille générations à ceux qui m'aiment et qui gardent mes commandements.
\VS{7}Tu ne prendras point le Nom de Yahweh, ton Dieu, en vain~; car Yahweh ne tiendra point pour innocent celui qui aura pris son Nom en vain\FTNT{Lé. 19:12~; Mt. 5:33.}.
\VS{8}Souviens-toi du jour du repos pour le sanctifier.
\VS{9}Tu travailleras six jours, et tu feras toute ton œuvre.
\VS{10}Mais le septième jour est le repos de Yahweh ton Dieu. Tu ne feras aucune œuvre en ce jour-là, ni toi, ni ton fils, ni ta fille, ni ton serviteur, ni ta servante, ni ton bétail, ni ton étranger qui est dans tes portes.
\VS{11}Car Yahweh a fait en six jours les cieux, la terre, la mer, et tout ce qui est en eux, et s'est reposé le septième jour~; c'est pourquoi Yahweh a béni le jour du repos et l'a sanctifié\FTNT{Ge. 2:3~; Ex. 31:14~; Ez. 20:12.}.
\VS{12}Honore ton père et ta mère, afin que tes jours soient prolongés sur la terre que Yahweh, ton Dieu, te donne\FTNT{Lé. 19:3~; De. 5:16~; Mt. 15:4~; Ep. 6:2.}.
\VS{13}Tu ne commettras pas de meurtre\FTNT{Mt. 5:21.}.
\VS{14}Tu ne commettras pas d'adultère\FTNT{Lé. 20:10~; De. 5:18~; Pr. 6:32~; Mt. 5:32~; Ro. 7:3.}.
\VS{15}Tu ne déroberas pas.
\VS{16}Tu ne diras pas de faux témoignage contre ton prochain.
\VS{17}Tu ne convoiteras pas la maison de ton prochain~; tu ne convoiteras pas la femme de ton prochain, ni son serviteur, ni sa servante, ni son bœuf, ni son âne, ni aucune chose qui soit à ton prochain.
\TextTitle{Le peuple tout tremblant devant Yahweh}
\VS{18}Or tout le peuple apercevait les tonnerres, les éclairs, le son du shofar, et la montagne fumante. Et le peuple voyant cela tremblait et se tenait loin.
\VS{19}Et ils dirent à Moïse~: Parle, toi, avec nous, et nous écouterons~; mais que Dieu ne parle point avec nous, de peur que nous ne mourrions\FTNT{De. 5:23-24~; Hé. 12:18-19.}.
\VS{20}Et Moïse dit au peuple~: Ne craignez point car Dieu est venu pour vous éprouver, et afin que sa crainte soit devant vous, et que vous ne péchiez point.
\VS{21}Le peuple donc se tint loin, mais Moïse s'approcha de l'obscurité dans laquelle Dieu était.
\VS{22}Et Yahweh dit à Moïse~: Tu diras ainsi aux enfants d'Israël~: Vous avez vu que je vous ai parlé des cieux.
\VS{23}Vous ne vous ferez point avec moi de dieux d'argent ni de dieux d'or.
\VS{24}Tu me feras un autel de terre, sur lequel tu sacrifieras tes holocaustes, et tes offrandes de paix\FTNT{Voir commentaire en Lé. 3:1.}, ton menu et ton gros bétail. En quelque lieu que ce soit où je mettrai la mémoire de mon Nom, je viendrai là à toi, et je te bénirai.
\VS{25}Si tu me fais un autel de pierres, ne les taille point~; car si tu fais passer le fer dessus, tu le souillerais.
\VS{26}Et tu ne monteras point à mon autel par des marches, de peur que ta nudité ne soit découverte en y montant.
\Chap{21}
\TextTitle{Lois sur les maîtres et leurs esclaves}
\VerseOne{}Ce sont ici les lois que tu leur proposeras.
\VS{2}Si tu achètes un esclave Hébreu, il te servira six ans, et au septième il sortira pour être libre, sans rien payer\FTNT{Lé. 25:39-43~; De. 15:12~; Jé. 34:14.}.
\VS{3}S'il est venu avec son corps seulement, il sortira avec son corps~; s'il avait une femme, sa femme sortira aussi avec lui.
\VS{4}Si son maître lui a donné une femme qui lui ait enfanté des fils ou des filles, sa femme et les enfants qu'il en aura seront à son maître, mais il sortira avec son corps.
\VS{5}Si l'esclave dit positivement: J'aime mon maître, ma femme, et mes fils, je ne sortirai point pour être libre.
\VS{6}Alors son maître le fera venir devant les juges, et le fera approcher de la porte ou du poteau, et son maître lui percera l'oreille avec un poinçon~; et il le servira pour toujours.
\VS{7}Si quelqu'un vend sa fille pour être esclave, elle ne sortira point comme les esclaves sortent.
\VS{8}Si elle déplaît à son maître, qui ne l'aura point fiancée, il la fera acheter~; mais il n'aura pas le pouvoir de la vendre à un peuple étranger, après lui avoir été infidèle.
\VS{9}Mais s'il l'a fiancée à son fils, il fera pour elle selon le droit des filles.
\VS{10}S'il en prend une pour lui, il ne retranchera rien de sa nourriture, de ses vêtements et du droit conjugal.
\VS{11}S'il ne fait pas pour elle ces trois choses-là, elle sortira sans payer aucun argent.
\TextTitle{Lois sur les dommages corporels}
\VS{12}Si quelqu'un frappe un homme et qu'il en meure,on le fera mourir de mort \FTNT{Lé. 24:17~; No. 35:11-16~; De. 19:2-11~; Jos. 20:2.}.
\VS{13}S'il ne lui a point dressé d'embûches, mais que Dieu l'ait fait tomber entre ses mains, je t'établirai un lieu où il s'enfuira.
\VS{14}Mais si quelqu'un s'élève de propos délibéré contre son prochain, pour le tuer par ruse, tu le tireras de mon autel, afin qu'il meure.
\VS{15}Celui qui aura frappé son père ou sa mère sera puni de mort\FTNT{Lé. 20:9~; De. 27:16~; Mt. 15:4.}.
\VS{16}Si quelqu'un dérobe un homme et le vend, ou s'il est trouvé entre ses mains, on le fera mourir de mort.
\VS{17}Celui qui aura maudit son père ou sa mère sera puni de mort.
\VS{18}Si quelques uns ont une querelle, et que l'un ait frappé l'autre d'une pierre ou du poing, sans causer sa mort, mais qu'il soit obligé de se mettre au lit,
\VS{19}s'il se lève et marche dehors en s'appuyant sur son bâton, celui qui l'aura frappé sera absous~; toutefois, il le dédommagera de ce qu'il a chômé et le fera guérir entièrement.
\VS{20}Si quelqu'un a frappé du bâton son serviteur ou sa servante, et qu'il soit mort sous sa main, on ne manquera point de le venger.
\VS{21}Mais s'il survit un jour ou deux, il ne sera point vengé, car c'est son argent.
\VS{22}Si des hommes se querellent, et que l'un d'eux frappe une femme enceinte, et qu'elle en accouche, s'il n'y a pas cas de mort, il sera condamné à l'amende que le mari de la femme lui imposera, et il la donnera selon que les juges en ordonneront.
\VS{23}Mais s'il y a cas de mort, tu donneras vie pour vie,
\VS{24}œil pour œil, dent pour dent, main pour main, pied pour pied\FTNT{Lé. 24:20~; De. 19:21~; Mt. 5:38.},
\VS{25}brûlure pour brûlure, plaie pour plaie, meurtrissure pour meurtrissure.
\VS{26}Si quelqu'un frappe l'œil de son serviteur, ou l'œil de sa servante, et lui gâte l'œil, il le laissera aller libre pour son œil~;
\VS{27}et s'il fait tomber une dent à son serviteur, ou à sa servante, il le laissera aller libre pour sa dent.
\VS{28}Si un bœuf heurte de sa corne un homme ou une femme, et que la personne en meure, le bœuf sera lapidé sans nulle exception, et on ne mangera point de sa chair, mais le maître du bœuf sera absous.
\VS{29}Si le bœuf était auparavant sujet à frapper de sa corne, et que son maître en ait été averti avec protestation, et qu'il ne l'ait point surveillé, s'il tue un homme ou une femme, le bœuf sera lapidé, et on fera aussi mourir son maître.
\VS{30}Si on lui impose un prix pour se racheter, il donnera la rançon de sa vie, selon tout ce qui lui sera imposé.
\VS{31}Si le bœuf heurte de sa corne un fils ou une fille, il lui sera fait selon cette même loi.
\VS{32}Si le bœuf heurte de sa corne un esclave, soit homme, soit femme, celui à qui est le bœuf donnera trente sicles d'argent au maître de l'esclave, et le bœuf sera lapidé.
\VS{33}Si quelqu'un découvre une fosse, ou si quelqu'un creuse une fosse, et ne la couvre point, et qu'il y tombe un bœuf ou un âne,
\VS{34}le maître de la fosse donnera satisfaction, et rendra l'argent au maître du bœuf, mais la bête morte lui appartiendra.
\VS{35}Et si le bœuf de quelqu'un blesse le bœuf de son prochain, et qu'il en meure, ils vendront le bœuf vivant, et en partageront l'argent par moitié, ils partageront aussi par moitié le bœuf mort.
\VS{36}Mais s'il est connu que le bœuf avait auparavant l'habitude de heurter avec sa corne, et que le maître ne l'ait point gardé, il restituera bœuf pour bœuf~; mais le bœuf mort sera pour lui.
\Chap{22}
\TextTitle{Lois sur les torts causés à autrui}
\VerseOne{}Si quelqu'un dérobe un bœuf, ou un chevreau, ou un agneau, et qu'il le tue, ou le vende, il restituera cinq bœufs pour le bœuf, et quatre agneaux ou chevreaux pour l'agneau ou pour le chevreau.
\VS{2}Si le voleur est trouvé dérobant avec effraction, et est frappé de sorte qu'il en meure, celui qui l'aura frappé ne sera point coupable de meurtre.
\VS{3}Mais si le soleil est levé sur lui, il sera coupable de meurtre. Il fera donc une entière restitution~; et s'il n'a pas de quoi, il sera vendu pour son vol.
\VS{4}Si ce qui a été dérobé est trouvé vivant entre ses mains, soit bœuf, soit âne, soit brebis ou chèvre, il rendra le double.
\VS{5}Si quelqu'un fait brouter dans un champ ou dans une vigne, en lâchant son bétail qui aille paître dans le champ d'autrui, il rendra le meilleur de son champ et le meilleur de sa vigne.
\VS{6}Si un feu éclate et rencontre des épines, et que le blé qui est en tas, ou sur pied, ou le champ, soit consumé, celui qui aura allumé le feu rendra entièrement ce qui en aura été brûlé.
\VS{7}Si quelqu'un donne à son prochain de l'argent ou des vases à garder, et qu'on le dérobe de sa maison, et si l'on trouve le voleur, il rendra le double\FTNT{Lé. 5:20-26.}.
\VS{8}Mais si on ne trouve point le voleur, on fera venir le maître de la maison devant les juges pour jurer s'il n'a point mis sa main sur le bien de son prochain.
\VS{9}Dans toute affaire d'infidélité concernant un bœuf, un âne, une brebis, une chèvre, un vêtement ou tout objet perdu, dont quelqu'un dira qu'il lui appartient, la cause des deux parties viendra devant les juges~; et celui que les juges auront condamné, rendra le double à son prochain.
\VS{10}Si quelqu'un donne à garder à son prochain un âne, un bœuf, quelque menue ou grosse bête, et qu'elle meure, ou qu'elle se soit cassée quelque membre, ou qu'on l'ait emmenée sans que personne l'ait vue,
\VS{11}le serment de Yahweh interviendra entre les deux parties\FTNT{Hé. 6:16.}, pour savoir s'il n'a point mis sa main sur le bien de son prochain, et le maître de la bête se contentera du serment, et l'autre ne la rendra point.
\VS{12}Mais s'il est vrai qu'elle lui a été dérobée, il la rendra à son maître.
\VS{13}S'il est vrai qu'elle ait été déchirée par les bêtes sauvages, il la produira en témoignage, et il ne rendra point ce qui a été déchiré.
\VS{14}Si quelqu'un a emprunté de son prochain quelque bête, et qu'elle se casse quelque membre, ou qu'elle meure, son maître n'étant point présent, il ne manquera pas de la rendre.
\VS{15}Mais si son maître est avec lui, il ne la rendra point~; si elle a été louée, on payera seulement son louage.
\TextTitle{Lois diverses}
\VS{16}Si un homme séduit une vierge non fiancée, et couche avec elle, il faut qu'il la dote, et qu'il la prenne pour femme\FTNT{De. 22:28.}.
\VS{17}Mais si le père de la fille refuse absolument de la lui donner, il lui comptera autant d'argent qu'on en donne pour la dot des vierges.
\VS{18}Tu ne laisseras point vivre la sorcière\FTNT{De. 18:10-11~; Lé. 20:27.}.
\VS{19}Celui qui couche avec une bête sera puni de mort\FTNT{Lé. 18:23~; Lé. 20:15~; De. 27:21.}.
\VS{20}Celui qui sacrifie à d'autres dieux qu'à Yahweh seul sera dévoué à la façon de l'interdit\FTNT{Lé. 17:7~; De. 13:6-16~; De. 17:2-5.}.
\VS{21}Tu ne fouleras ni n'opprimeras point l'étranger~; car vous avez été étrangers au pays d'Egypte\FTNT{Lé. 19:34.}.
\VS{22}Vous n'affligerez point la veuve ni l'orphelin\FTNT{De. 24:17-18~; Za. 7:10.}.
\VS{23}Si vous les affligez en quoi que ce soit, et qu'ils crient à moi, certainement j'entendrai leur cri.
\VS{24}Et ma colère s'embrasera, et je vous ferai mourir par l'épée~; et vos femmes seront veuves, et vos fils orphelins.
\VS{25}Si tu prêtes de l'argent à mon peuple, au pauvre qui est avec toi, tu ne te comporteras point avec lui en créancier, vous ne lui exigerez point d'intérêt.
\VS{26}Si tu prends en gage le vêtement de ton prochain, tu le lui rendras avant que le soleil soit couché\FTNT{De. 24:10-13.}.
\VS{27}Car c'est sa seule couverture, c'est son vêtement pour couvrir sa peau~; où coucherait-il~? S'il arrive donc qu'il crie à moi, je l'entendrai~; car je suis miséricordieux.
\VS{28}Tu ne maudiras point les juges, et tu ne maudiras point le prince de ton peuple\FTNT{Lé. 24:15-16.}.
\VS{29}Tu ne différeras point de m'offrir de ton abondance et de tes liqueurs~; tu me donneras le premier-né de tes fils\FTNT{Ex. 13:12-15~; De. 26:2-11.}.
\VS{30}Tu feras la même chose de ta vache, de ta brebis, et de ta chèvre. Il sera sept jours avec sa mère, et le huitième jour tu me le donneras.
\VS{31}Vous me serez saints, et vous ne mangerez point de la chair déchirée dans les champs, mais vous la jetterez aux chiens.
\Chap{23}
\TextTitle{Lois diverses (suite)}
\VerseOne{}Tu ne léveras point de faux bruit, et tu ne te joindras point au méchant pour être un faux témoin, afin que violence soit faite\FTNT{Ex. 20:16~; De. 19:16-21.}.
\VS{2}Tu ne suivras point la multitude pour faire le mal~; et tu ne témoigneras point dans un procès en sorte que tu te détournes après un grand nombre pour pervertir le droit.
\VS{3}Tu n'honoreras point le pauvre dans son procès\FTNT{De. 1:17.}.
\VS{4}Si tu rencontres le bœuf de ton ennemi, ou son âne égaré, tu ne manqueras point de le lui ramener.
\VS{5}Si tu vois l'âne de celui qui te hait, abattu sous sa charge, tu t'arrêteras pour le secourir, et tu ne manqueras pas de l'aider.
\VS{6}Tu ne pervertiras point le droit de l'indigent, qui est au milieu de toi, dans son procès.
\VS{7}Tu t'éloigneras de toute parole fausse, et tu ne feras point mourir l'innocent et le juste~; car je ne justifierai point le méchant.
\VS{8}Tu ne prendras point de présent~; car le présent aveugle les plus éclairés, et pervertit les paroles des justes.
\VS{9}Tu n'opprimeras point l'étranger~; car vous savez ce que c'est que d'être étrangers, parce que vous avez été étrangers au pays d'Egypte.
\TextTitle{Le sabbat, le repos de la terre}
\VS{10}Pendant six ans tu ensemenceras ta terre, et en recueilleras le revenu.
\VS{11}Mais la septième année, tu lui donneras du relâche, et la laisseras reposer, afin que les pauvres de ton peuple en mangent, et que les bêtes des champs mangent ce qui restera. Tu en feras de même de ta vigne et de tes oliviers.
\VS{12}Tu travailleras six jours, mais tu te reposeras au septième jour, afin que ton bœuf et ton âne se reposent, et que le fils de ta servante et l'étranger reprennent courage.
\VS{13}Vous prendrez garde à toutes les choses que je vous ai ordonnées. Vous ne ferez point mention du nom des dieux étrangers, on ne l'entendra point de ta bouche\FTNT{Jos. 23:7~; Ps. 16:4.}.
\TextTitle{Les fêtes solennelles}
\VS{14}Trois fois l'an, tu me célébreras une fête solennelle\FTNT{Lé. 23:4-44.}.
\VS{15}Tu garderas la fête solennelle des pains sans levain\FTNT{Ex. 29:2.}~; tu mangeras des pains sans levain pendant sept jours, comme je t'ai ordonné, en la saison et au mois où les épis mûrissent~; car c'est en ce mois-là que tu es sorti d'Egypte~; et nul ne se présentera devant ma face à vide.
\VS{16}Et la fête solennelle de la moisson des premiers fruits de ton travail, de ce que tu auras semé au champ~; et la fête de la récolte, après la fin de l'année, quand tu auras recueilli du champ les fruits de ton travail\FTNT{Ex. 34:22.}.
\VS{17}Trois fois l'an, tous les mâles d'entre vous se présenteront devant le Seigneur Yahweh.
\VS{18}Tu ne sacrifieras point le sang de mon sacrifice avec du pain levé~; et la graisse de ma fête solennelle ne passera point la nuit jusqu'au matin\FTNT{Ex. 34:25-26.}.
\VS{19}Tu apporteras dans la maison de Yahweh, ton Dieu, les prémices des premiers fruits de ta terre. Tu ne feras point cuire le chevreau dans le lait de sa mère.
\TextTitle{Mises en garde et promesses de Yahweh}
\VS{20}Voici, j'envoie un Ange devant toi, afin qu'il te garde dans le chemin, et qu'il t'introduise dans le lieu que je t'ai préparé.
\VS{21}Garde-toi de provoquer sa colère, et écoute sa voix, et ne l'irrite point, car il ne pardonnera point votre péché~; car mon Nom est en lui.
\VS{22}Mais si tu écoutes attentivement sa voix, et si tu fais tout ce que je te dirai, je serai l'ennemi de tes ennemis, et j'affligerai ceux qui t'affligeront.
\VS{23}Car mon Ange marchera devant toi, et t'introduira au pays des Amoréens, des Héthiens, des Phéréziens, des Cananéens, des Héviens, et des Jébusiens, et je les exterminerai.
\VS{24}Tu ne te prosterneras point devant leurs dieux, et tu ne les serviras point, et tu ne feras point selon leurs œuvres, mais tu les détruiras entièrement, et tu briseras entièrement leurs statues\FTNT{Ex. 20:5~; Ex. 34:13~; No. 33:52.}.
\VS{25}Vous servirez Yahweh, votre Dieu. Et il bénira ton pain et tes eaux~; et j'ôterai les maladies du milieu de toi\FTNT{Ex. 15:26~; De. 6:13~; De. 7:15-16~; Mt. 4:10.}.
\VS{26}Il n'y aura point dans ton pays de femme qui avorte, ou qui soit stérile~; j'accomplirai le nombre de tes jours.
\VS{27}J'enverrai la terreur de mon Nom devant toi, et j'effrayerai tout peuple vers lequel tu arriveras, et je ferai que tous tes ennemis tourneront le dos devant toi\FTNT{De. 7:23.}.
\VS{28}Et j'enverrai des frelons devant toi, qui chasseront les Héviens, les Cananéens, et les Héthiens, de devant ta face\FTNT{De. 7:20~; Jos. 24:12.}.
\VS{29}Je ne les chasserai point loin de devant ta face en une année, de peur que le pays ne devienne un désert, et que les bêtes des champs ne se multiplient contre toi.
\VS{30}Mais je les chasserai peu à peu loin de devant toi, jusqu'à ce que tu te sois accru, et que tu possèdes le pays.
\VS{31}Et je mettrai des bornes depuis la Mer Rouge jusqu'à la mer des Philistins, et depuis le désert jusqu'au fleuve~; car je livrerai entre tes mains les habitants du pays et je les chasserai de devant toi.
\VS{32}Tu ne traiteras point d'alliance avec eux ni avec leurs dieux.
\VS{33}Ils n'habiteront point dans ton pays, de peur qu'ils ne te fassent pécher contre moi~; car tu servirais leurs dieux, et ce serait un piège pour toi.
\Chap{24}
\TextTitle{La loi lue au peuple~; le sang de l'Alliance}
\VerseOne{}Puis il dit à Moïse~: Monte vers Yahweh, toi et Aaron, Nadab et Abihu, et soixante-dix des anciens d'Israël, et vous vous prosternerez de loin.
\VS{2}Et Moïse s'approchera seul de Yahweh, mais eux ne s'en approcheront point, et le peuple ne montera point avec lui.
\VS{3}Alors Moïse vint, et récita au peuple toutes les paroles de Yahweh, et toutes ses lois, et tout le peuple répondit d'une voix, et dit~: Nous ferons toutes les choses que Yahweh a dites.
\VS{4}Or Moïse écrivit toutes les paroles de Yahweh, et s'étant levé de bon matin, il bâtit un autel au bas de la montagne, et dressa pour monument douze pierres pour les douze tribus d'Israël.
\VS{5}Et il envoya des jeunes hommes, des enfants d'Israël, qui offrirent des holocaustes et qui sacrifièrent des veaux à Yahweh en sacrifice d'offrande de paix.
\VS{6}Et Moïse prit la moitié du sang, et le mit dans des bassins, et répandit l'autre moitié sur l'autel.
\VS{7}Ensuite, il prit le livre de l'Alliance et le lut, et le peuple qui l'écoutait dit~: Nous ferons tout ce que Yahweh a dit, et nous obéirons.
\VS{8}Moïse donc prit le sang, et le répandit sur le peuple, en disant~: Voici le sang de l'Alliance que Yahweh a traitée avec vous, selon toutes ces paroles\FTNT{Mt. 26:28~; Mc. 14:24~; Lu. 22:20~; 1 Co. 11:25~; Hé. 9:20.}.
\TextTitle{Yahweh fait monter Moïse sur la montagne}
\VS{9}Puis Moïse, Aaron, Nadab, Abihu, et les soixante-dix anciens d'Israël montèrent.
\VS{10}Et ils virent le Dieu d'Israël, et sous ses pieds comme un ouvrage de saphir transparent, comme le ciel dans toute sa pureté.
\VS{11}Et il ne mit point sa main sur ceux qui avaient été choisis d'entre les enfants d'Israël~; ainsi, ils virent Dieu, et ils mangèrent et burent.
\VS{12}Et Yahweh dit à Moïse~: Monte vers moi sur la montagne, et demeure là~; et je te donnerai des tables de pierre, la loi et les commandements que j'ai écrits pour les enseigner.
\VS{13}Alors Moïse se leva avec Josué qui le servait~; et Moïse monta sur la montagne de Dieu.
\VS{14}Et il dit aux anciens d'Israël~: Demeurez ici en nous attendant jusqu'à ce que nous retournions vers vous. Et voici, Aaron et Hur seront avec vous~; quiconque aura quelque affaire, qu'il s'adresse à eux.
\VS{15}Moïse donc monta sur la montagne, et une nuée couvrit la montagne\FTNT{Ex. 19:9-16.}.
\VS{16}Et la gloire de Yahweh demeura sur la montagne de Sinaï, et la nuée la couvrit pendant six jours. Et au septième jour, il appela Moïse du milieu de la nuée.
\VS{17}Et ce qu'on voyait de la gloire de Yahweh au sommet de la montagne, était comme un feu dévorant aux yeux des enfants d'Israël\FTNT{De. 4:24~; De. 9:3~; Hé. 12:29.}.
\VS{18}Et Moïse entra dans la nuée et monta sur la montagne. Moïse fut sur la montagne quarante jours et quarante nuits.
\Chap{25}
\TextTitle{Des offrandes volontaires pour les matériaux du tabernacle}
\VerseOne{}Et Yahweh parla à Moïse, en disant~:
\VS{2}Parle aux enfants d'Israël, et qu'on prenne une offrande pour moi. Vous prendrez mon offrande de tout homme dont le cœur me l'offrira volontairement.
\VS{3}Et c'est ici l'offrande que vous prendrez d'eux~: De l'or, de l'argent, de l'airain,
\VS{4}de la pourpre, de l'écarlate, du cramoisi\FTNT{La couleur cramoisi s'obtient grâce à la femelle cochenille aptère qui contient dans son corps et dans ses œufs un pigment rouge à base d'acide carminique qui permet à l'insecte et à ses larves de se protéger des prédateurs. Au moment de la ponte, cette dernière fixe fermement son corps au tronc d'un arbre puis libère ses œufs qui demeurent ainsi protégés en dessous d'elle jusqu'à leur éclosion. Ensuite, l'insecte meurt en libérant cette substance rouge qui se propage sur tout son corps et sur le bois hôte. C'est ce fluide que l'homme récupère pour en faire un colorant à la couleur caractéristique. Une subtile analogie peut être faire entre la cochenille et le Seigneur qui a versé son sang à la croix pour nous donner la vie. «~Et moi, je suis un ver, et non un homme, l'opprobre des hommes et le méprisé du peuple~» (Ps. 22~:7).}, du fin lin, du poil de chèvre,
\VS{5}des peaux de béliers teintes en rouge, des peaux de taissons\FTNT{Le mot hébreu employé ici est «~tachash~», il désigne le matériau servant à fabriquer la couverture extérieure de la tente d'assignation. Si tout le monde s'accorde pour dire qu'il s'agissait d'une fourrure ou d'une peau d'animal, un doute subsiste sur la race exacte de l'animal. On hésite entre le marsouin, le dauphin, le blaireau (taisson) ou peut-être le mouton. Dans de nombreuses bibles, on a pris le parti de traduire par «~peaux de dauphins~». Cette hypothèse est cependant très peu probable. D'une part parce que le dauphin n'a pas de fourrure~; d'autre part parce que sa peau n'est absolument pas adaptée à la vie terrestre. Elle est donc impossible à conserver et à transformer, en particulier dans le contexte d'un climat propre au désert. Certains pensent qu'il s'agit tout simplement de peaux de béliers. Dans ce cas, comment expliquer qu'on n'ait pas employé le terme «~'ayil~» comme cela est mentionné pour les peaux teintes en rouge~? Il reste donc les peaux de taissons, c'est-à-dire de blaireaux, dont la fourrure est utilisée depuis des siècles. On peut objecter qu'il est impossible que le Seigneur puisse accepter la peau d'un animal impur pour construire son sanctuaire. Si tel était le cas, la peau du dauphin, qui est également un animal impur, n'aurait pas non-plus été autorisée. Toutefois, d'un point de vue prophétique, le symbole est important~: La présence de cet animal impur préfigurait Christ qui a pris une chair semblable à celle du péché (Ro. 8:3) mais aussi le levain que l'on peut trouver dans la pate nouvelle (1 Co. 5:6-7), l'ivraie qui se glisse parmi le blé (Mt. 13:25-43).}, du bois d'acacia,
\VS{6}de l'huile pour le luminaire, des aromates pour l'huile d'onction et pour le parfum odoriférant,
\VS{7}des pierres d'onyx, et d'autres pierres pour la garniture de l'éphod et pour le pectoral.
\VS{8}Et ils me feront un sanctuaire, et j'habiterai au milieu d'eux\FTNT{Ex. 29:45-46.}.
\VS{9}Ils le feront conformément à tout ce que je vais te montrer, selon le modèle du tabernacle et selon le modèle de tous ses ustensiles~; vous le ferez donc ainsi.
\TextTitle{L'arche de l'alliance}
\VS{10}Et ils feront une arche de bois d'acacia~; et sa longueur sera de deux coudées et demie, et sa largeur d'une coudée et demie, et sa hauteur d'une coudée et demie.
\VS{11}Et tu la couvriras d'or pur, tu l'en couvriras en dehors et en dedans~; et tu feras sur elle un couronnement d'or tout autour\FTNT{Ex. 37:1-9.}.
\VS{12}Et tu fondras pour elle quatre anneaux d'or, que tu mettras à ses quatre coins, deux anneaux à l'un de ses côtés, et deux autres de l'autre côté.
\VS{13}Tu feras aussi des barres de bois d'acacia, et tu les couvriras d'or.
\VS{14}Puis tu feras entrer les barres dans les anneaux aux côtés de l'arche, pour porter l'arche avec elles.
\VS{15}Les barres seront dans les anneaux de l'arche, et on ne les en tirera point.
\VS{16}Et tu mettras dans l'arche le témoignage que je te donnerai\FTNT{Hé. 9:4.}.
\VS{17}Tu feras aussi un propitiatoire d'or pur, dont la longueur sera de deux coudées et demie, et la largeur d'une coudée et demie.
\VS{18}Et tu feras deux chérubins d'or~; tu les feras d'ouvrage étendu au marteau, tirés des deux extrémités du propitiatoire.
\VS{19}Fais donc un chérubin tiré des extrémités et un chérubin tiré de l'autre extrémité~; vous ferez les chérubins tirés du propitiatoire à ses deux extrémités.
\VS{20}Et les chérubins étendront les ailes en haut, couvrant de leurs ailes le propitiatoire, et leurs faces seront vis-à-vis l'une de l'autre~; et le regard des chérubins sera vers le propitiatoire\FTNT{1 R. 8:6-7~; Hé. 9:5.}.
\VS{21}Et tu poseras le propitiatoire au-dessus de l'arche, et tu mettras dans l'arche le témoignage que je te donnerai.
\VS{22}Et je me rencontrerai là avec toi, et je te dirai de dessus le propitiatoire, d'entre les deux chérubins qui seront sur l'arche du témoignage, toutes les choses que je t'ordonnerai pour les enfants d'Israël\FTNT{Ex. 29:42-43~; No. 7:89.}.
\TextTitle{La table des pains de proposition}
\VS{23}Tu feras aussi une table de bois d'acacia. Sa longueur sera de deux coudées, et sa largeur d'une coudée, et sa hauteur d'une coudée et demie.
\VS{24}Tu la couvriras d'or pur, et tu lui feras un couronnement d'or tout autour.
\VS{25}Tu lui feras aussi à l'entour une clôture d'une largeur de main, et tout autour de sa clôture tu feras un couronnement d'or.
\VS{26}Tu lui feras aussi quatre anneaux d'or que tu mettras aux quatre coins qui seront à ses quatre pieds.
\VS{27}Les anneaux seront à l'endroit de la clôture, afin d'y mettre les barres pour porter la table.
\VS{28}Tu feras les barres de bois d'acacia, et tu les couvriras d'or, et on portera la table avec elles.
\VS{29}Tu feras aussi ses plats, ses tasses, ses gobelets, et ses bassins, avec lesquels on fera les aspersions~; tu les feras d'or pur\FTNT{Ex. 37:10-16.}.
\VS{30}Et tu mettras sur cette table le pain de proposition continuellement devant moi\FTNT{Lé. 24:5-9.}.
\TextTitle{Le chandelier d'or pur}
\VS{31}Tu feras aussi un chandelier d'or pur\FTNT{Le chandelier avait une double symbolique. D'une part, il préfigurait Jésus-Christ, notre Lumière (Jn. 1:4-5~; Jn. 8:12). Les sept lampes évoquaient l'omniscience de l'Esprit de Jésus-Christ (Za. 3:9~; Jn. 16:29-30~; Ap. 1:4~; Ap. 3:1~; Ap. 4:5~; Ap. 5:6). Il est à noter que ce chandelier comportait des calices en forme de fleurs, de pommes et d'amandes (Ex. 25:33) qui symbolisaient les fruits de l'Esprit que nous devons nécessairement porter (Ga. 5:22). D'autre part, il est une image de l'Eglise (Ap. 1:20). Voir commentaire Ex. 37:17-24~; Es. 8:13-17.}. Le chandelier sera étendu au marteau~; son pied, sa tige et ses branches, ses plats, ses pommeaux et ses fleurs seront tirés de lui.
\VS{32}Six branches sortiront de ses côtés~: Trois branches d'un côté du chandelier, et trois autres de l'autre côté du chandelier.
\VS{33}Il y aura sur l'une des branches trois petits plats en forme d'amande, un pommeau et une fleur~; sur l'autre branche trois petits plats en forme d'amande, un pommeau et une fleur~; il en sera de même des six branches sortant du chandelier.
\VS{34}ll y aura aussi au chandelier quatre petits plats en forme d'amande, ses pommeaux et ses fleurs.
\VS{35}Un pommeau sous deux branches tirées du chandelier, un pommeau sous deux autres branches tirées de lui, et un pommeau sous deux autres branches tirées de lui~; il en sera de même des six branches sortant du chandelier.
\VS{36}Leurs pommeaux et leurs branches seront tirés de lui, et tout le chandelier sera un seul ouvrage étendu au marteau, et d'or pur.
\VS{37}Tu feras aussi ses sept lampes, et on les allumera afin qu'elles éclairent vis-à-vis du chandelier.
\VS{38}Et ses mouchettes et ses encensoirs destinés à recevoir ce qui tombe des lampes seront d'or pur.
\VS{39}On le fera avec tous ses ustensiles d'un talent d'or pur.
\VS{40}Regarde donc, et fais selon le modèle qui t'est montré sur la montagne.
\Chap{26}
\TextTitle{Les tapis de fin lin}
\VerseOne{}Tu feras aussi le tabernacle de dix tapis de fin lin retors, de pourpre, d'écarlate, et de cramoisi~; et tu les feras semés de chérubins d'un ouvrage exquis\FTNT{Ex. 36:8-38.}.
\VS{2}La longueur d'un tapis sera de vingt-huit coudées, et la largeur du même tapis de quatre coudées~; tous les tapis auront une même mesure.
\VS{3}Cinq de ces tapis seront joints l'un à l'autre, et les cinq autres seront aussi joints l'un à l'autre.
\VS{4}Fais aussi des lacets de pourpre sur le bord d'un tapis, au bord du premier assemblage~; et tu feras la même chose au bord du dernier tapis dans l'autre assemblage.
\VS{5}Tu feras donc cinquante lacets au premier tapis, et tu feras cinquante lacets au bord du tapis qui est dans le second assemblage. Les lacets seront vis-à-vis l'un de l'autre.
\VS{6}Tu feras aussi cinquante crochets d'or, et tu attacheras les tapis l'un à l'autre avec les crochets~; ainsi le tabernacle ne fera qu'un.
\TextTitle{Les tapis de poils de chèvre}
\VS{7}Tu feras aussi des tapis de poils de chèvres pour servir de tente sur le tabernacle~; tu feras onze de ces tapis.
\VS{8}La longueur d'un tapis sera de trente coudées, et la largeur du même tapis sera de quatre coudées~; les onze tapis auront une même mesure.
\VS{9}Puis tu joindras séparément cinq de ces tapis, et les six tapis à part~; mais tu redoubleras le sixième tapis sur le devant du tabernacle.
\VS{10}Tu feras aussi cinquante lacets sur le bord de l'un des tapis, à savoir au dernier qui est assemblé, et cinquante lacets au bord du tapis du second assemblage.
\VS{11}Tu feras aussi cinquante crochets d'airain, et tu feras entrer les crochets dans les lacets~; et tu assembleras ainsi la tente qui fera un tout.
\VS{12}Mais ce qu'il y aura en surplus dans les tapis de la tente, à savoir la moitié du tapis de reste, retombera sur le derrière du tabernacle.
\VS{13}La coudée d'une part, et la coudée d'autre part, qui seront de reste sur la longueur des tapis de la tente, retomberont sur les deux côtés du tabernacle, pour le couvrir.
\TextTitle{Les couvertures de peaux de béliers}
\VS{14}Tu feras aussi pour ce tabernacle une couverture de peaux de béliers teintes en rouge, et une couverture de peaux de taissons par-dessus\FTNT{Ex. 35:7~; Ex. 35:23~; Ex. 36:19~; Ex. 39:34.}.
\TextTitle{Les planches et leurs bases}
\VS{15}Et tu feras pour le tabernacle, des planches de bois d'acacia, qu'on fera tenir debout\FTNT{Ex. 36:20-34.}.
\VS{16}La longueur d'une planche sera de dix coudées, et la largeur d'une même planche d'une coudée et demie.
\VS{17}Il y aura à chaque planche deux tenons joints l'un à l'autre~; et tu feras de même pour toutes les planches du tabernacle.
\VS{18}Tu feras donc les planches du tabernacle, à savoir vingt planches qui regardent vers le midi.
\VS{19}Et au-dessous des vingt planches, tu feras quarante bases d'argent~; deux bases sous une planche pour ses deux tenons, et deux bases sous l'autre planche pour ses deux tenons.
\VS{20}Et vingt planches de l'autre côté du tabernacle, du coté nord.
\VS{21}Et leurs quarante bases seront d'argent, deux bases sous une planche, et deux bases sous l'autre planche.
\VS{22}Et pour le fond du tabernacle, vers l'occident, tu feras six planches.
\VS{23}Tu feras aussi deux planches pour les angles du tabernacle, aux deux cotés du fond.
\VS{24}Et elles seront égales par le bas, et elles seront jointes et unies par le haut avec un anneau~; il en sera de même des deux planches qui seront aux deux angles.
\VS{25}Il y aura donc huit planches, et seize bases d'argent~; deux bases sous une planche et deux bases sous une autre planche.
\VS{26}Après cela, tu feras cinq barres de bois d'acacia, pour les planches d'un des côtés du tabernacle.
\VS{27}Pareillement, tu feras cinq barres pour les planches de l'autre côté du tabernacle~; et cinq barres pour les planches du côté du tabernacle, pour le fond, vers le côté de l'occident.
\VS{28}Et la barre du milieu sera au milieu des planches d'une extrémité à l'autre.
\VS{29}Tu couvriras aussi d'or les planches, et tu feras d'or leurs anneaux pour mettre les barres, et tu couvriras d'or les barres.
\VS{30}Tu dresseras le tabernacle selon le modèle qui t'est montré sur la montagne.
\TextTitle{Les voiles intérieurs et extérieurs}
\VS{31}Et tu feras un voile\FTNT{Le voile intérieur symbolisait la chair de Jésus-Christ qui a été brisée à cause de nos péchés (Es. 53:5~; Hé. 10:20). Ex. 36:35-38~; Mt. 27:51~; Hé. 9:3.} de pourpre, d'écarlate, de cramoisi, et de fin lin retors~; on le fera d'ouvrage exquis, avec des chérubins.
\VS{32}Et tu le mettras sur quatre piliers de bois d'acacia couverts d'or, ayant leurs crochets d'or. Et ils seront sur quatre bases d'argent.
\VS{33}Puis tu mettras le voile sous les crochets, et tu feras entrer là-dedans, c'est-à-dire au-dedans du voile, l'arche du témoignage~; et ce voile vous fera la séparation entre le lieu saint et le Saint des saints.
\VS{34}Et tu poseras le propitiatoire sur l'arche du témoignage, dans le Saint des saints.
\VS{35}Et tu mettras la table au dehors de ce voile, et le chandelier vis-à-vis de la table, au côté du tabernacle, vers le sud~; et tu placeras la table côté nord.
\VS{36}Et à l'entrée du tabernacle, tu feras un rideau de pourpre, d'écarlate, de cramoisi et de fin lin retors, d'ouvrage de broderie.
\VS{37}Tu feras aussi pour ce rideau cinq piliers de bois d'acacia, que tu couvriras d'or, et leurs crochets seront d'or~; et tu fondras pour eux cinq bases d'airain.
\Chap{27}
\TextTitle{L'autel d'airain}
\VerseOne{}Tu feras aussi un autel de bois d'acacia, ayant cinq coudées de long, et cinq coudées de large~; l'autel sera carré, et sa hauteur sera de trois coudées.
\VS{2}Tu feras ses cornes à ses quatre coins~; ses cornes seront tirées de lui, et tu le couvriras d'airain\FTNT{C'est sur l'autel d'airain que les animaux étaient sacrifiés. Il préfigurait la croix et le jugement que Jésus-Christ a pris sur lui à notre place (Es. 53:5~; 2 Co. 13:4~; Ph. 2:8).}.
\VS{3}Tu feras ses chaudrons pour recevoir ses cendres, et ses racloirs, ses bassins, ses fourchettes, et ses encensoirs~; tu feras tous ses ustensiles d'airain.
\VS{4}Tu lui feras une grille d'airain en forme de treillis, et tu feras au treillis quatre anneaux d'airain à ses quatre coins.
\VS{5}Et tu le mettras au-dessous de l'enceinte de l'autel en bas, et le treillis s'étendra jusqu'au milieu de l'autel.
\VS{6}Tu feras aussi des barres pour l'autel, des barres de bois d'acacia, et tu les couvriras d'airain.
\VS{7}Et on fera passer ses barres dans les anneaux~; les barres seront aux deux côtés de l'autel pour le porter.
\VS{8}Tu le feras creux avec des planches~; ils le feront ainsi qu'il t'a été montré sur la montagne\FTNT{Ex. 38:1-7.}.
\TextTitle{Le parvis}
\VS{9}Tu feras aussi le parvis du tabernacle, du côté qui regarde vers le sud~; il y aura pour former le parvis, des courtines de fin lin retors~; la longueur de l'un des côtés sera de cent coudées.
\VS{10}Il y aura vingt piliers avec leurs vingt bases d'airain, mais les crochets des piliers et leurs filets seront d'argent.
\VS{11}Ainsi du côté nord, il y aura également des courtines sur une longueur de cent coudées, avec vingt piliers avec leurs vingt bases d'airain~; mais les crochets des piliers avec leurs filets seront d'argent.
\VS{12}La largeur du parvis du côté de l'occident sera de cinquante coudées de courtines, qui auront dix piliers, avec leurs dix bases.
\VS{13}Et la largeur du parvis du côté de l'orient, directement vers le levant, sera de cinquante coudées.
\VS{14}A l'un des côtés, il y aura quinze coudées de courtines, avec leurs trois piliers et leurs trois bases.
\VS{15}Et de l'autre côté, quinze coudées de courtines, avec leurs trois piliers et leurs trois bases.
\TextTitle{La porte du parvis}
\VS{16}Il y aura aussi pour la porte du parvis un rideau de vingt coudées, fait de pourpre, d'écarlate, de cramoisi, et de fin lin retors, ouvrage de broderie, avec quatre piliers et quatre bases.
\VS{17}Tous les piliers du parvis seront ceints d'un filet d'argent, et leurs crochets seront d'argent, mais leurs bases seront d'airain.
\VS{18}La longueur du parvis sera de cent coudées, et la largeur de cinquante, de chaque côté~; et la hauteur de cinq coudées. Il sera de fin lin retors, et les bases des piliers seront d'airain.
\VS{19}Que tous les ustensiles du tabernacle, pour tout son service, et tous ses pieux, avec les pieux du parvis, soient d'airain\FTNT{Ex. 38:9-20.}.
\TextTitle{L'huile d'olive vierge pour les lampes}
\VS{20}Tu ordonneras aux fils d'Israël qu'ils t'apportent de l'huile d'olive vierge pour le luminaire, afin de faire luire les lampes continuellement\FTNT{Ex. 35:8-28~; Lé. 24:1-4.}.
\VS{21}Aaron avec ses fils les prépareront dans la présence de Yahweh, depuis le soir jusqu'au matin, dans la tente d'assignation, hors du voile qui est devant le témoignage~; ce sera une ordonnance perpétuelle pour les enfants d'Israël.
\Chap{28}
\TextTitle{La prêtrise}
\VerseOne{}Et toi, fais approcher de toi Aaron, ton frère, et ses fils avec lui, d'entre les enfants d'Israël, pour m'exercer la prêtrise, à savoir Aaron, Nadab et Abihu, Eléazar et Ithamar, fils d'Aaron.
\VS{2}Et tu feras à Aaron, ton frère, de saints vêtements pour gloire et pour ornement.
\TextTitle{Les vêtements sacrés des prêtres}
\VS{3}Et tu parleras à tous les hommes d'esprit, à chacun de ceux que j'ai remplis de l'esprit de science, afin qu'ils fassent des vêtements à Aaron pour le sanctifier, afin qu'il m'exerce la prêtrise.
\VS{4}Et ce sont ici les vêtements qu'ils feront~: Le pectoral, l'éphod, la robe, la tunique brodée, la tiare, et la ceinture. Ils feront donc les saints vêtements à Aaron, ton frère, et à ses fils, pour m'exercer la prêtrise.
\VS{5}Et ils prendront de l'or, de la pourpre, de l'écarlate, du cramoisi, et du fin lin.
\TextTitle{L'éphod}
\VS{6}Et ils feront l'éphod d'or, de pourpre, d'écarlate et de cramoisi, et de fin lin retors~; d'un ouvrage exquis.
\VS{7}Il aura deux épaulettes qui se joindront par les deux bouts~; et c'est ainsi qu'il sera joint.
\VS{8}La ceinture exquise dont il sera ceint, et qui sera par-dessus, sera de même ouvrage, et tirée de lui, étant d'or, de pourpre, d'écarlate, de cramoisi, et de fin lin retors.
\VS{9}Et tu prendras deux pierres d'onyx, et tu graveras sur elles les noms des enfants d'Israël~:
\VS{10}Six de leurs noms sur une pierre et les six noms des autres sur l'autre pierre, selon leur naissance.
\VS{11}Tu graveras sur les deux pierres les noms des enfants d'Israël, comme on grave les pierres et les cachets, tu les entoureras de montures d'or.
\VS{12}Et tu mettras les deux pierres sur les épaulettes de l'éphod, afin qu'elles soient des pierres de souvenir pour les enfants d'Israël~; car Aaron portera leurs noms sur ses deux épaules devant Yahweh, pour souvenir.
\VS{13}Tu feras aussi des montures d'or,
\VS{14}et deux chaînettes d'or pur que tu tresseras en forme de cordons, et tu fixeras aux montures les chaînettes ainsi tressées.
\TextTitle{Le pectoral}
\VS{15}Tu feras aussi le pectoral du jugement d'un ouvrage exquis, comme l'ouvrage de l'éphod, d'or, de pourpre, d'écarlate, de cramoisi, et de fin lin retors.
\VS{16}Il sera carré et double~; et sa longueur sera d'un empan, et sa largeur d'un empan.
\VS{17}Et tu le rempliras de garniture de pierres, à quatre rangées de pierres précieuses. A la première rangée, on mettra une sardoine, une topaze, et une émeraude.
\VS{18}Et à la seconde rangée, une escarboucle, un saphir, et un jaspe.
\VS{19}Et à la troisième rangée, une opale, une agate, et une améthyste.
\VS{20}Et à la quatrième rangée, un chrysolithe, un onyx et un béryl, qui seront enchâssés dans de l'or, selon leur garniture.
\VS{21}Et ces pierres-là seront selon les noms des enfants d'Israël, douze selon leurs noms, chacune d'elles gravées comme des cachets, selon le nom qu'elle doit porter, et elles seront pour les douze tribus.
\VS{22}Tu feras donc pour le pectoral des chaînettes d'or pur, tressées en forme de cordon.
\VS{23}Et tu feras sur le pectoral deux anneaux d'or, et tu mettras les deux anneaux aux deux bouts du pectoral.
\VS{24}Et tu mettras les deux chaînettes d'or, faites en cordon, dans les deux anneaux à l'extrémité du pectoral.
\VS{25}Et tu mettras les deux autres bouts des deux chaînettes en cordon sur les deux montures, et tu les mettras sur les épaulettes de l'éphod, sur le devant de l'éphod.
\VS{26}Tu feras aussi deux autres anneaux d'or, que tu mettras aux deux autres bouts du pectoral, sur le bord qui sera du côté de l'éphod à l'intérieur.
\VS{27}Et tu feras deux autres anneaux d'or, que tu mettras aux deux épaulettes de l'éphod par le bas, sur le devant, à l'endroit où il se joint, au-dessus de la ceinture exquise de l'éphod.
\VS{28}Et ils joindront le pectoral élevé par ses anneaux, aux anneaux de l'éphod, avec un cordon de pourpre, afin qu'il tienne au-dessus de la ceinture exquise de l'éphod, et que le pectoral ne puisse pas se séparer de l'éphod.
\VS{29}Ainsi, Aaron portera sur son cœur les noms des enfants d'Israël gravés sur le pectoral du jugement, quand il entrera dans le lieu saint, pour souvenir devant Yahweh continuellement.
\TextTitle{L'urim et le thummim}
\VS{30}Et tu mettras sur le pectoral de jugement l'urim et le thummim\FTNT{L'urim («~lumières~») et le thummim («~perfections~») étaient deux pierres du pectoral que l'on utilisait ensemble pour déterminer la décision de Dieu sur certaines questions.}, qui seront sur le cœur d'Aaron, quand il viendra devant Yahweh~; et Aaron portera le jugement des enfants d'Israël sur son cœur devant Yahweh continuellement.
\TextTitle{La robe de l'éphod}
\VS{31}Tu feras aussi la robe de l'éphod entièrement de pourpre.
\VS{32}Il y aura, au milieu, une ouverture pour la tête, et cette ouverture aura tout autour un bord tissé, comme l'ouverture d'une cotte de mailles, afin que la robe ne se déchire pas.
\VS{33}Tu feras à ses bords des grenades de pourpre, d'écarlate, et de cramoisi tout autour, et des clochettes d'or entre elles tout autour.
\VS{34}Une clochette d'or, puis une grenade, une clochette d'or, puis une grenade, aux bords de la robe tout autour.
\VS{35}Et Aaron en sera revêtu quand il fera le service, et on en entendra le son lorsqu'il entrera dans le lieu saint devant Yahweh, et quand il en sortira, afin qu'il ne meure pas.
\TextTitle{La lame d'or gravée~: La sainteté à Yahweh}
\VS{36}Et tu feras une lame d'or pur, sur laquelle tu graveras ces mots, comme on grave un cachet~: La sainteté à Yahweh.
\VS{37}Tu l'attacheras avec un cordon de pourpre sur la tiare, sur le devant de la tiare.
\VS{38}Et elle sera sur le front d'Aaron~; et Aaron portera l'iniquité commise par les enfants d'Israël, en faisant leurs saintes offrandes, elle sera continuellement sur son front devant Yahweh, pour qu'il leur soit favorable.
\TextTitle{Les vêtements de service d'Aaron et ses fils}
\VS{39}Tu feras aussi une tunique de fin lin qui s'appliquera sur le corps, et tu feras aussi la tiare de fin lin~; mais tu feras la ceinture d'ouvrage de broderie\FTNT{Ex. 39:1-32.}.
\VS{40}Tu feras aussi aux fils d'Aaron des tuniques, des ceintures, et des bonnets, pour leur gloire et leur ornement.
\VS{41}Et tu en revêtiras Aaron, ton frère, et ses fils avec lui~; tu les oindras, tu les consacreras et tu les sanctifieras~; puis ils exerceront la prêtrise pour moi\FTNT{Lé. 8:12~; Lé. 16:32~; No. 3:3.}.
\VS{42}Et tu leur feras des caleçons de lin, pour couvrir leur nudité, qui tiendront depuis les reins jusqu'au bas des cuisses.
\VS{43}Et Aaron et ses fils seront ainsi habillés quand ils entreront dans la tente d'assignation, ou quand ils approcheront de l'autel pour faire le service dans le lieu saint~; et ils ne porteront point la peine d'aucune iniquité, et ne mourront point. Ce sera une ordonnance perpétuelle pour lui et pour sa postérité après lui.
\Chap{29}
\TextTitle{Les prêtres consacrés au service de Yahweh}
\VerseOne{}Or c'est ici ce que tu leur feras, quand tu les sanctifieras pour exercer la prêtrise pour moi~: Prends un veau du troupeau, et deux béliers sans tare\FTNT{Lé. 8:2~; Lé. 9:2~; Hé. 7:26-28.}~;
\VS{2}et des pains sans levain, et des gâteaux sans levain pétris à l'huile, et des beignets sans levain, oints d'huile~; et tu les feras de fine farine de froment\FTNT{Lé. 6:13.}.
\VS{3}Tu les mettras dans une corbeille, et tu les présenteras dans la corbeille~; tu présenteras aussi le veau et les deux moutons.
\VS{4}Puis, tu feras approcher Aaron et ses fils à l'entrée de la tente d'assignation, et tu les laveras avec de l'eau\FTNT{Ex. 40:12.}.
\VS{5}Ensuite, tu prendras les vêtements, et tu feras vêtir à Aaron la tunique et la robe de l'éphod, l'éphod et pectoral, et tu le ceindras par-dessus avec la ceinture exquise de l'éphod.
\VS{6}Puis, tu mettras sur sa tête la tiare et la couronne de sainteté sur la tiare.
\VS{7}Et tu prendras l'huile d'onction et la répandras sur sa tête~; et tu l'oindras ainsi.
\VS{8}Puis, tu feras approcher ses fils, et tu leur feras vêtir les tuniques.
\VS{9}Et tu les ceindras des ceintures, Aaron, dis-je, et ses fils\FTNT{Es. 11:5~; Ep. 6:14.}, et tu leur attacheras des bonnets, et ils posséderont la prêtrise par ordonnance perpétuelle. Et tu consacreras ainsi Aaron et ses fils.
\VS{10}Et tu feras approcher le veau devant la tente d'assignation, et Aaron et ses fils poseront leurs mains sur la tête du veau.
\VS{11}Et tu égorgeras le veau devant Yahweh, à l'entrée de la tente d'assignation.
\VS{12}Puis, tu prendras du sang du veau, et le mettras avec ton doigt sur les cornes de l'autel, et tu répandras tout le reste du sang au pied de l'autel.
\VS{13}Tu prendras aussi toute la graisse qui couvre les entrailles, et le lobe du foie, les deux rognons, la graisse qui les entoure, et tu les feras fumer sur l'autel.
\VS{14}Mais tu brûleras au feu la chair du veau, sa peau, et ses excréments, hors du camp. C'est un sacrifice pour le péché\FTNT{Lé. 1:3-13~; Hé. 9:11~; Hé. 13:11.}.
\VS{15}Puis, tu prendras l'un des béliers, et Aaron et ses fils poseront leurs mains sur la tête du bélier.
\VS{16}Puis, tu égorgeras le bélier, et prenant son sang, tu le répandras sur l'autel tout autour.
\VS{17}Après, tu couperas le bélier par pièces, et ayant lavé ses entrailles et ses jambes, tu les mettras sur ses pièces et sur sa tête.
\VS{18} Et tu feras fumer tout le bélier sur l'autel~; c'est un holocauste à Yahweh, c'est un sacrifice consumé par le feu d'une agréable odeur à Yahweh.
\VS{19}Puis, tu prendras l'autre bélier, et Aaron et ses fils mettront leurs mains sur sa tête.
\VS{20}Et tu égorgeras le bélier, et prenant de son sang, tu le mettras sur le lobe de l'oreille droite d'Aaron, et sur le lobe de l'oreille droite de ses fils, sur le pouce de leur main droite, sur le gros orteil de leur pied droit, et tu répandras le reste du sang sur l'autel tout autour.
\VS{21}Et tu prendras du sang qui sera sur l'autel, de l'huile d'onction, et tu en feras l'aspersion sur Aaron, sur ses vêtements, sur ses fils, et sur les vêtements de ses fils avec lui. Ainsi, lui, ses vêtements, ses fils, et les vêtements de ses fils, seront sanctifiés avec lui.
\VS{22}Tu prendras aussi la graisse du bélier, la queue, et la graisse qui couvre les entrailles, le grand lobe du foie, les deux rognons, la graisse qui est dessus, et l'épaule droite~; car c'est le bélier de consécration.
\VS{23}Tu prendras aussi un pain, un gâteau à l'huile, et un beignet dans la corbeille où seront ces choses sans levain, laquelle sera devant Yahweh.
\VS{24}Et tu mettras toutes ces choses sur les mains d'Aaron et sur les mains de ses fils, et tu les agiteras de côté et d'autre devant Yahweh\FTNT{No. 6:19.}.
\VS{25}Puis, les recevant de leurs mains, tu les feras fumer sur l'autel, sur l'holocauste, pour être une odeur agréable devant Yahweh~; c'est un sacrifice consumé par le feu à Yahweh.
\TextTitle{La part des prêtres}
\VS{26}Tu prendras aussi la poitrine du bélier des consécrations, qui est pour Aaron, et tu l'agiteras de côté et d'autre en offrande agitée devant Yahweh. Ce sera ta part.
\VS{27}Tu sanctifieras donc la poitrine de l'offrande agitée, et l'épaule de l'offrande élevée, tant ce qui aura été agité que ce qui aura été élevé du bélier de consécration, de ce qui est pour Aaron et de ce qui est pour ses fils. \FTNT{Lé. 10:14~; No. 18:18.}.
\VS{28}Et ceci sera une ordonnance perpétuelle pour Aaron et pour ses fils, de ce qui sera offert par les enfants d'Israël~; car c'est une offrande élevée. Quand il y aura une offrande élevée de celles qui sont faites par les enfants d'Israël, de leurs offrandes de paix, leur offrande élevée sera à Yahweh.
\VS{29}Et les saints vêtements qui seront pour Aaron, seront pour ses fils après lui, afin qu'ils soient oints et consacrés dans ces vêtements.
\VS{30}Le prêtre qui succédera à sa place d'entre ses fils, et qui viendra à la tente d'assignation, pour faire le service dans le lieu saint, en sera revêtu durant sept jours.
\VS{31}Or tu prendras le bélier des consécrations, et tu feras bouillir sa chair dans un lieu saint~;
\VS{32}et Aaron et ses fils mangeront à l'entrée de la tente d'assignation la chair du bélier et le pain qui sera dans la corbeille.
\VS{33}Ils mangeront donc ces choses, par lesquelles la propitiation aura été faite, pour les consacrer et les sanctifier~; mais l'étranger n'en mangera point, parce qu'elles sont saintes.
\VS{34}S'il y a des restes de la chair des consécrations et du pain jusqu'au matin, tu brûleras ces restes-là au feu~; on n'en mangera point, parce que c'est une chose sainte.
\VS{35}Tu feras donc ainsi à Aaron et à ses fils, selon toutes les choses que je t'ai ordonnées~; tu les consacreras durant sept jours\FTNT{Lé. 8:31-35.}.
\VS{36}Et tu offriras comme sacrifice pour l'expiation tous les jours un veau pour faire l'expiation, et tu purifieras l'autel par cette propitiation, et tu l'oindras pour le sanctifier\FTNT{Ez. 43:19-20.}.
\VS{37}Pendant sept jours, tu feras propitiation pour l'autel, et tu le sanctifieras~; l'autel sera une chose très sainte~; tout ce qui touchera l'autel sera saint\FTNT{No. 28:3.}.
\TextTitle{L'holocauste perpétuel}
\VS{38}Or c'est ici ce que tu feras sur l'autel~: Tu offriras chaque jour continuellement deux agneaux d'un an.
\VS{39}Tu sacrifieras l'un des agneaux au matin, et l'autre agneau entre les deux soirs~;
\VS{40}avec un dixième de fine farine pétrie dans la quatrième partie d'un hin d'huile vierge, et avec une libation de vin de la quatrième partie d'un hin pour chaque agneau,
\VS{41}et tu sacrifieras l'autre agneau entre les deux soirs, avec un gâteau comme au matin, et tu lui feras la même libation, en bonne odeur~; c'est un sacrifice consumé par le feu à Yahweh.
\VS{42}Ce sera l'holocauste perpétuel qui sera offert en vos générations, à l'entrée de la tente d'assignation, devant Yahweh, où je me trouverai avec vous pour te parler.
\VS{43}Je me trouverai là pour les enfants d'Israël, et la tente sera sanctifiée par ma gloire.
\VS{44}Je sanctifierai donc la tente d'assignation et l'autel. Je sanctifierai aussi Aaron et ses fils, afin qu'ils exercent la prêtrise pour moi.
\VS{45} Et j'habiterai au milieu des enfants d'Israël, et je serai leur Dieu.
\VS{46}Et ils sauront que je suis Yahweh, leur Dieu, qui les ai tirés du pays d'Egypte, pour habiter au milieu d'eux. Je suis Yahweh leur Dieu.
\Chap{30}
\TextTitle{L'autel des parfums}
\VerseOne{}Tu feras aussi un autel pour les parfums, et tu le feras de bois d'acacia.
\VS{2}Sa longueur sera d'une coudée, et sa largeur d'une coudée~; il sera carré~; mais sa hauteur sera de deux coudées, et ses cornes seront tirées de lui.
\VS{3}Tu le couvriras d'or pur, tant le dessus, que ses côtés tout autour, et ses cornes. Et tu lui feras un couronnement d'or tout autour.
\VS{4}Tu lui feras aussi deux anneaux d'or au-dessous de son couronnement, à ses deux côtés, lesquels tu mettras aux deux coins, pour y faire passer les barres qui serviront à le porter.
\VS{5}Tu feras les barres de bois d'acacia, et tu les couvriras d'or.
\VS{6}Et tu les mettras devant le voile, qui est au-devant de l'arche du témoignage, à l'endroit du propitiatoire qui est sur le témoignage, où je me trouverai avec toi.
\VS{7}Et Aaron fera sur cet autel un parfum de choses aromatiques~; il y fera un parfum chaque matin, quand il préparera les lampes.
\VS{8}Et quand Aaron allumera les lampes entre les deux soirs, il y fera aussi le parfum, à savoir le parfum perpétuel devant Yahweh dans vos générations\FTNT{Ex. 37:25-29~; 2 Ch. 13:11.}.
\VS{9}Vous n'offrirez point sur cet autel aucun parfum étranger, ni d'holocauste, ni d'offrande, et vous n'y répandrez aucune libation.
\VS{10}Mais Aaron fera une fois l'an la propitiation sur les cornes de cet autel~; il fera, dis-je, la propitiation une fois l'an sur cet autel dans vos générations, avec le sang de l'offrande pour l'expiation faite pour les propitiations. C'est une chose très sainte à Yahweh.
\TextTitle{L'offrande du rachat\FTNTT{Ex. 15:1-21~; Ps. 107:1-2.}}
\VS{11}Yahweh parla aussi à Moïse, et lui dit~:
\VS{12}Quand tu feras le dénombrement des fils d'Israël, selon leur nombre, ils donneront chacun à Yahweh le rachat de sa personne, quand tu en feras le dénombrement, et il n'y aura point de plaie sur eux quand tu en feras le dénombrement\FTNT{No. 1:2.}.
\VS{13}Tous ceux qui passeront par le dénombrement donneront un demi-sicle, selon le sicle du sanctuaire, qui est de vingt guéras~; le demi-sicle donc sera l'offrande que l'on donnera à Yahweh\FTNT{Lé. 27:25~; No. 3:47~; Ez. 45:12.}.
\VS{14}Tous ceux qui passeront par le dénombrement, depuis l'âge de vingt ans et au-dessus, donneront cette offrande à Yahweh.
\VS{15}Le riche n'augmentera rien, et le pauvre ne diminuera rien du demi-sicle, quand ils donneront à Yahweh l'offrande pour faire le rachat de vos personnes.
\VS{16}Tu prendras donc des enfants d'Israël l'argent des expiations, et tu l'appliqueras à l'œuvre de la tente d'assignation. Ce sera pour les fils d'Israël, un souvenir devant Yahweh pour faire le rachat de vos personnes.
\TextTitle{Purification par l'eau de la cuve d'airain\FTNTT{Jn. 13:3-10~; Hé. 10:22~; 1 Jn. 1:9.}}
\VS{17}Yahweh parla encore à Moïse, en disant~:
\VS{18}Fais aussi une cuve d'airain, avec sa base d'airain, pour laver. Et tu la mettras entre la tente d'assignation et l'autel, et tu mettras de l'eau dedans~;
\VS{19}Aaron et ses fils y laveront leurs mains et leurs pieds.
\VS{20}Quand ils entreront dans la tente d'assignation ils se laveront avec de l'eau, afin qu'ils ne meurent point, et quand ils approcheront de l'autel pour faire le service, afin de faire fumer l'offrande consumée par le feu à Yahweh.
\VS{21}Ils laveront donc leurs pieds et leurs mains, afin qu'ils ne meurent point~; ce leur sera une ordonnance perpétuelle, tant pour Aaron que pour ses fils, en leur génération.
\TextTitle{L'huile pour l'onction sainte\FTNTT{Jn. 4:23~; Ep. 2:18, 5:18-19.}}
\VS{22}Yahweh parla aussi à Moïse, en disant~:
\VS{23}Prends des choses aromatiques les plus exquises~; de la myrrhe franche le poids de cinq cent sicles, la moitié de cinnamome odoriférant, c'est-à-dire, le poids de deux cent cinquante sicles, et du roseau aromatique deux cent cinquante sicles.
\VS{24}De la casse le poids de cinq cent sicles, selon le sicle du sanctuaire, et un hin d'huile d'olive.
\VS{25}Et tu en feras de l'huile pour l'onction sainte, un onguent composé selon l'art du parfumeur, ce sera l'huile de l'onction sainte.
\VS{26}Puis tu en oindras la tente d'assignation, et l'arche du témoignage.
\VS{27}La table et tous ses ustensiles, le chandelier et ses ustensiles, et l'autel du parfum,
\VS{28} et l'autel des holocaustes et tous ses ustensiles, la cuve et sa base.
\VS{29}Ainsi, tu les sanctifieras, et ils seront une chose très-sainte~; tout ce qui les touchera sera saint.
\VS{30}Tu oindras aussi Aaron et ses fils, et les sanctifieras pour m'exercer la prêtrise.
\VS{31}Tu parleras aussi aux enfants d'Israël, en disant~: Ce me sera une huile d'onction sainte dans toutes vos générations.
\VS{32}On n'en oindra point la chair d'aucun homme, et vous n'en ferez point d'autre de même composition~; elle est sainte, elle vous sera sainte.
\VS{33}Quiconque composera un onguent semblable, et qui en mettra sur un autre, sera retranché de ses peuples.
\TextTitle{L'encens pur parfumé}
\VS{34}Yahweh dit aussi à Moïse~: Prends des aromates, à savoir de la gomme, de l'ongle odorant, du galbanum, le tout préparé, et de l'encens pur, le tout à poids égal.
\VS{35}Et tu en feras un parfum aromatique selon l'art du parfumeur, et tu y mettras du sel~; vous le ferez pur, et ce sera pour vous une chose sainte.
\VS{36}Et quand tu l'auras pilé bien menu, tu en mettras dans la tente d'assignation devant le témoignage, où je me trouverai avec toi. Ce sera pour vous une chose très sainte.
\VS{37}Quant au parfum que tu feras, vous ne ferez point pour vous de semblable composition~; ce sera une chose sainte pour Yahweh.
\VS{38}Quiconque en fera un semblable pour le sentir sera retranché de ses peuples.
\Chap{31}
\TextTitle{Yahweh suscite des artisans}
\VerseOne{}Yahweh parla aussi à Moïse, en disant~:
\VS{2}Regarde, j'ai appelé par son nom Betsaleel, fils d'Uri, fils de Hur, de la tribu de Juda.
\VS{3}Et je l'ai rempli de l'Esprit de Dieu, de sagesse, d'intelligence et de science pour toutes sortes d'ouvrages,
\VS{4}afin d'inventer des dessins pour travailler  l'or, l'argent et l'airain~;
\VS{5}dans la sculpture des pierres précieuses, pour les mettre en œuvre, et dans la menuiserie pour travailler dans toutes sortes d'ouvrages.
\VS{6}Et voici, je lui ai donné pour compagnon Oholiab, fils d'Ahisamac, de la tribu de Dan~; et au cœur de tout homme sage, j'ai mis de l'intelligence, afin qu'ils fassent toutes les choses que je t'ai ordonnées,
\VS{7}à savoir la tente d'assignation, l'arche du témoignage, et le propitiatoire qui doit être au-dessus, et tous les ustensiles du tabernacle~;
\VS{8}et la table avec tous ses ustensiles~; et le chandelier pur avec tous ses ustensiles~; et l'autel du parfum~;
\VS{9}et l'autel de l'holocauste avec tous ses ustensiles, la cuve et sa base~;
\VS{10}et les vêtements du service~; les saints vêtements pour le prêtre Aaron, et les vêtements de ses fils pour exercer la prêtrise~;
\VS{11}et l'huile d'onction, et le parfum des choses aromatiques pour le sanctuaire, et ils feront toutes les choses que je t'ai ordonnées.
\TextTitle{Le sabbat comme signe entre Yahweh et Israël}
\VS{12}Yahweh parla encore à Moïse, en disant~:
\VS{13}Toi aussi parle aux enfants d'Israël, en disant~: Certes, vous garderez mes sabbats, car c'est un signe entre moi et vous, et parmi vos générations, afin que vous sachiez que je suis Yahweh qui vous sanctifie.
\VS{14}Gardez donc le sabbat, car il doit vous être saint. Quiconque le violera sera puni de mort~; quiconque,dis-je, fera une œuvre en ce jour-là sera retranché du milieu de son peuple.
\VS{15}On travaillera six jours, mais le septième jour est le sabbat du repos, consacré à Yahweh~; quiconque fera une œuvre le jour du repos sera puni de mort.
\VS{16}Ainsi, les enfants d'Israël garderont le sabbat pour célébrer le jour du repos, en leur génération, par une alliance perpétuelle.
\VS{17}C'est un signe entre moi et les enfants d'Israël à perpétuité~; car Yahweh a fait en six jours les cieux et la terre, et il a cessé au septième, et s'est reposé\FTNT{Ge. 2:2~; Ez. 20:12.}.
\VS{18}Et Dieu donna à Moïse, après qu'il eut achevé de parler avec lui sur la montagne de Sinaï, les deux tables du témoignage~; tables de pierre, écrites du doigt de Dieu\FTNT{De. 9:10.}.
\Chap{32}
\TextTitle{Le culte du veau d'or}
\VerseOne{}Mais le peuple, voyant que Moïse tardait tant à descendre de la montagne, s'assembla autour d'Aaron, et lui dit~: Lève-toi, fais-nous des dieux qui marchent devant nous, car quant à ce Moïse, cet homme qui nous a fait monter du pays d'Egypte, nous ne savons ce qui lui est arrivé\FTNT{Ac. 7:40.}.
\VS{2}Et Aaron leur répondit~: Mettez en pièces les anneaux d'or qui sont aux oreilles de vos femmes, de vos fils, et de vos filles, et apportez-les-moi\FTNT{Ex. 35:22.}.
\VS{3} Et aussitôt, tout le peuple mit en pièces les anneaux d'or qui étaient à leurs oreilles, et ils les apportèrent à Aaron, 
\VS{4}qui les ayant reçu de leurs mains, forma l'or avec un burin, et en fit un veau\FTNT{Selon toute vraisemblance, les Israélites s'étaient inspirés d'une idole égyptienne, le taureau sacré Apis, pour faire le veau d'or. Dieu de la puissance sexuelle, de la fertilité et de la force, il était souvent représenté sous la forme d'un homme avec une tête de taureau, puis avec un disque solaire entre les cornes à partir du Nouvel Empire.} en métal fondu. Et ils dirent~: Ce sont ici tes dieux, ô Israël, qui t'ont fait monter du pays d'Egypte.
\VS{5}Ce qu'Aaron ayant vu, il bâtit un autel devant le veau, et cria en disant~: Demain, il y aura une fête solennelle à Yahweh.
\VS{6}Ainsi, ils se levèrent le lendemain, dès le matin, et ils offrirent des holocaustes, et présentèrent des offrandes de paix. Et le peuple s'assit pour manger et pour boire, puis ils se levèrent pour jouer\FTNT{1 Co. 10:7.}.
\TextTitle{Yahweh condamne l'idolâtrie d'Israël}
\VS{7}Alors Yahweh dit à Moïse~: Va, descends, car ton peuple que tu as fait monter du pays d'Egypte s'est corrompu\FTNT{De. 32:5.}.
\VS{8}Ils se sont promptement détournés de la voie que je leur avais ordonnée, et ils se sont fait un veau en métal fondu, et se sont prosternés devant lui, ils lui ont offert des sacrifices, puis ont dit~: Ce sont ici tes dieux, ô Israël, qui t'ont fait monter du pays d'Egypte\FTNT{1 R. 12:28.}.
\VS{9}Yahweh dit encore à Moïse~: J'ai regardé ce peuple, et voici, c'est un peuple au cou raide.
\VS{10}Maintenant laisse-moi, et ma colère s'embrasera contre eux, et je les consumerai~; mais je te ferai devenir une grande nation.
\TextTitle{Moïse implore Yahweh pour le peuple}
\VS{11}Alors Moïse supplia Yahweh, son Dieu, et dit~: Ô Yahweh, pourquoi ta colère s'embraserait-elle contre ton peuple, que tu as retiré du pays d'Egypte par une grande puissance et par une main forte\FTNT{Ps. 106:23.}~?
\VS{12}Pourquoi les Egyptiens diraient~: Il les a retirés dans de mauvaises vues, pour les tuer sur les montagnes, et pour les consumer de dessus la terre~? Reviens de l'ardeur de ta colère, et repens-toi de ce mal que tu veux faire à ton peuple\FTNT{No. 14:11-15~; De. 9:28.}.
\VS{13}Souviens-toi d'Abraham, d'Isaac et d'Israël, tes serviteurs, auxquels tu as juré par toi-même en leur disant~: Je multiplierai votre postérité comme les étoiles des cieux, et je donnerai à votre postérité tout ce pays, dont j'ai parlé, et ils l'hériteront à jamais\FTNT{De. 34:4.}.
\VS{14}Et Yahweh se repentit du mal qu'il avait dit qu'il ferait à son peuple.
\TextTitle{Jugement sur le peuple}
\VS{15}Moïse regarda, et descendit de la montagne, ayant dans sa main les deux tables du témoignage, et les tables étaient écrites des deux côtés, écrites de l'un et de l'autre côté.
\VS{16}Et les tables étaient l'ouvrage de Dieu, et l'écriture était l'écriture de Dieu, gravée sur les tables.
\VS{17}Et Josué, entendant la voix du peuple qui faisait un grand bruit, dit à Moïse~: Il y a un bruit de bataille au camp.
\VS{18}Et Moïse lui répondit~: Ce n'est pas une voix ni un cri de gens qui soient les plus forts, ni une voix ni un cri de gens qui soient les plus faibles~; mais j'entends une voix de gens qui chantent.
\VS{19}Et il arriva que lorsque Moïse fut approché du camp, il vit le veau et les danses, et la colère de Moïse s'embrasa~; et il jeta de ses mains les tables, et les rompit au pied de la montagne.
\VS{20}Il prit ensuite le veau qu'ils avaient fait, et le brûla au feu, et le moulut jusqu'à ce qu'il fut en poudre, puis il répandit cette poudre dans de l'eau, et il en fit boire aux enfants d'Israël\FTNT{De. 9:17-21.}.
\VS{21}Et Moïse dit à Aaron~: Que t'a fait ce peuple pour que tu aies fait venir sur lui un si grand péché~?
\VS{22}Et Aaron lui répondit~: Que la colère de mon seigneur ne s'embrase point, tu sais que ce peuple est porté au mal.
\VS{23}Ils m'ont dit~: Fais-nous un dieu qui marche devant nous, car ce Moïse, cet homme qui nous a fait monter du pays d'Egypte, nous ne savons ce qui lui est arrivé.
\VS{24}Alors je leur ai dit~: Que celui qui a de l'or, le mette en pièces~! Et ils me l'ont donné~; et je l'ai jeté au feu, et ce veau en est sorti.
\VS{25}Or Moïse vit que le peuple était dénudé, car Aaron l'avait dénudé pour être en opprobre parmi leurs ennemis.
\VS{26}Et Moïse se tenant à la porte du camp, dit~: Qui est pour Yahweh~? Qu'il vienne vers moi~! Et tous les fils de Lévi s'assemblèrent vers lui.
\VS{27}Et il leur dit~: Ainsi parle Yahweh, le Dieu d'Israël~: Que chacun mette son épée à son côté, passez et repassez de porte en porte par le camp, et que chacun de vous tue son frère, son ami, et son voisin.
\VS{28}Et les fils de Lévi firent selon la parole de Moïse~; et ce jour-là il tomba parmi le peuple environ trois mille hommes.
\VS{29}Car Moïse avait dit~: Consacrez aujourd'hui vos mains à Yahweh, chacun même contre son fils, et contre son frère, afin que vous attiriez aujourd'hui sur vous la bénédiction.
\TextTitle{Moïse intercède pour Israël}
\VS{30}Et le lendemain, Moïse dit au peuple~: Vous avez commis un grand péché~; mais je monterai vers Yahweh, et peut-être je ferais propitiation pour votre péché.
\VS{31}Moïse donc retourna vers Yahweh et dit~: Hélas~! Je te prie, ce peuple a commis un grand péché, en se faisant des dieux d'or.
\VS{32}Maintenant pardonne leur péché~! Sinon, efface-moi maintenant de ton livre que tu as écrit.
\VS{33}Et Yahweh répondit à Moïse~: C'est celui qui aura péché contre moi que j'effacerai de mon livre\FTNT{Ap. 3:5~; Ap. 20:15~; Ap. 21:27.}.
\VS{34}Va maintenant, conduis le peuple au lieu duquel je t'ai parlé. Voici, mon Ange ira devant toi~; et le jour où je ferai punition, je punirai sur eux leur péché.
\VS{35} Ainsi, Yahweh frappa le peuple, parce qu'ils avaient été les auteurs du veau qu'Aaron avait fait.
\Chap{33}
\TextTitle{Yahweh ne veut plus marcher avec Israël}
\VerseOne{}Yahweh donc dit à Moïse~: Va, monte d'ici, toi et le peuple que tu as fait monter du pays d'Egypte, au pays que j'ai juré de donner à Abraham, à Isaac, et à Jacob, en disant~: Je le donnerai à ta postérité.
\VS{2}Et j'enverrai un Ange devant toi, et je chasserai les Cananéens, les Amoréens, les Héthiens, les Phéréziens, les Héviens et les Jébusiens,
\VS{3}pour vous conduire au pays découlant de lait et de miel, mais je ne monterai point au milieu de toi, parce que tu es un peuple au cou raide, de peur que je ne te consume en chemin.
\VS{4}Et le peuple entendit ces tristes nouvelles, et en mena le deuil, et aucun d'eux ne mit ses ornements sur soi.
\VS{5}Car Yahweh avait dit à Moïse~: Dis aux enfants d'Israël~: Vous êtes un peuple au cou raide~; je monterai en un moment au milieu de toi, et je te consumerai. Maintenant donc ôte tes ornements de dessus toi, et je saurai ce que je te ferai.
\VS{6}Ainsi, les enfants d'Israël se dépouillèrent de leurs ornements vers la montagne d'Horeb.
\TextTitle{Moïse dresse la tente d'assignation hors du camp}
\VS{7}Et Moïse prit une tente, et la tendit pour soi hors du camp, l'éloignant du camp~; et il l'appela la tente d'assignation~; et tous ceux qui cherchaient Yahweh sortaient vers la tente d'assignation qui était hors du camp.
\VS{8}Et il arrivait qu'aussitôt que Moïse sortait vers la tente, tout le peuple se levait, et chacun se tenait à l'entrée de sa tente, et regardait Moïse par-derrière, jusqu'à ce qu'il soit entré dans la tente.
\VS{9}Et sitôt que Moïse était entré dans la tente, la colonne de nuée descendait et s'arrêtait à la porte de la tente, et Yahweh parlait avec Moïse.
\VS{10}Et tout le peuple voyant la colonne de nuée s'arrêtant à la porte de la tente se levait, et chacun se prosternait à la porte de sa tente.
\VS{11}Et Yahweh parlait à Moïse face à face, comme un homme parle avec son intime ami. Puis Moïse retournait au camp, mais son serviteur Josué, fils de Nun, jeune homme, ne bougeait point de la tente\FTNT{No. 12:8~; De. 34:10~; Jn. 15:14-15.}.
\TextTitle{Moïse demande que Yahweh marche avec Israël}
\VS{12}Moïse donc dit à Yahweh~: Regarde, tu m'as dit~: Fais monter ce peuple, et tu ne m'as point fait connaître celui que tu dois envoyer avec moi~; tu as même dit~: Je te connais par ton nom, et aussi, tu as trouvé grâce devant mes yeux.
\VS{13}Or maintenant, je te prie, si j'ai trouvé grâce devant tes yeux, fais-moi connaître ton chemin, et je te connaîtrai, afin que je trouve grâce devant tes yeux~; considère aussi que cette nation est ton peuple\FTNT{Ps. 25:4.}.
\VS{14}Et Yahweh dit~: Ma face ira, et je te donnerai du repos.
\VS{15}Et Moïse lui dit~: Si ta face ne vient, ne nous fais point monter d'ici.
\VS{16}Car en quoi connaîtra-t-on que nous avons trouvé grace devant tes yeux, moi et ton peuple~? Ne sera-ce pas quand tu marcheras avec nous~? Et alors, moi et ton peuple serons en admiration plus que tous les peuples qui sont sur la terre~?
\VS{17}Et Yahweh dit à Moïse~: Je ferai aussi ce que tu dis~; car tu as trouvé grâce devant mes yeux, et je te connais par ton nom.
\TextTitle{Moïse veut voir la gloire de Yahweh}
\VS{18}Moïse dit aussi~: Je te prie, fais-moi voir ta gloire~!
\VS{19}Et Dieu dit~: Je ferai passer toute ma bonté devant ta face, et je crierai le Nom de Yahweh devant toi~; et je ferai grâce à qui je ferai grâce, et j'aurai compassion de celui de qui j'aurai compassion\FTNT{Ro. 9:15.}.
\VS{20}Puis il dit~: Tu ne pourras pas voir ma face, car nul homme ne peut me voir et vivre\FTNT{Jn. 1:18~; Jn. 14:8-11.}.
\VS{21}Yahweh dit aussi~: Voici, il y a un lieu près de moi, et tu t'arrêteras sur le rocher\FTNT{Le rocher préfigurait Jésus-Christ, le Roc sur lequel nous devons bâtir nos vies et le fondement de l'Eglise (Ps. 18:32~; Mt. 7:24-25~; Mt. 16:18~; 1 Co. 3:11). Voir commentaire Es. 8:13-17.}.
\VS{22}Et quand ma gloire passera, je te mettrai dans un creux du rocher, et te couvrirai de ma main, jusqu'à ce que je sois passé.
\VS{23}Puis je retirerai ma main, et tu me verras par-derrière, mais ma face ne se verra point.
\Chap{34}
\TextTitle{De nouvelles tables~; la gloire de Yahweh\FTNTT{Ex. 33:18-23.}}
\VerseOne{}Et Yahweh dit à Moïse~: Aplanis-toi deux tables de pierre comme les premières, et j'écrirai sur elles les paroles qui étaient sur les premières tables que tu as rompues\FTNT{De. 10:1.}.
\VS{2}Et sois prêt au matin, et monte au matin sur la montagne de Sinaï, et présente-toi là devant moi sur le haut de la montagne.
\VS{3}Mais que personne ne monte avec toi, et même que personne ne paraisse sur toute la montagne~; et que ni menu ni gros bétail ne paisse sur cette montagne\FTNT{Ex. 19:12-13.}.
\VS{4}Moïse donc aplanit deux tables de pierre comme les premières, et se leva de bon matin, et monta sur la montagne de Sinaï, comme Yahweh le lui avait ordonné, et il prit dans sa main les deux tables de pierre.
\VS{5}Et Yahweh descendit dans la nuée, et s'arrêta là avec lui, et cria le Nom de Yahweh.
\VS{6}Comme donc Yahweh passait par devant lui, il cria~: Yahweh, Yahweh~! Le Dieu compatissant, miséricordieux, lent à la colère, abondant en bonté et en fidélité\FTNT{No. 14:18~; 2 Ch. 30:9~; Né. 9:17~; Ps. 103:8.}.
\VS{7}Qui conserve sa bonté jusqu'à mille générations, ôtant l'iniquité, le crime, et le péché, qui ne tient point le coupable pour innocent, et qui punit l'iniquité des pères sur les fils, et sur les fils des fils, jusqu'à la troisième et à la quatrième génération\FTNT{Ex. 20:6~; De. 5:10~; Jé. 32:18.}.
\VS{8}Et Moïse se hatant, baissa la tête contre terre et se prosterna. 
\VS{9}Et il dit~: Ô Seigneur~! Je te prie, si j'ai trouvé grâce à tes yeux, que le Seigneur marche maintenant au milieu de nous, car c'est un peuple au cou raide. Pardonne donc nos iniquités et notre péché, et prends-nous pour ta possession.
\TextTitle{Yahweh renouvelle ses promesses\FTNTT{Ex. 33:18-23.}}
\VS{10}Et il répondit~: Voici, je traite alliance devant tout ton peuple, je ferai des merveilles qui n'ont point été faites sur toute la terre ni dans aucune nation. Et tout le peuple au milieu duquel tu es, verra l'œuvre de Yahweh, car ce que je m'en vais faire avec toi sera une chose redoutable.
\VS{11}Garde soigneusement ce que je t'ordonne aujourd'hui. Voici, je m'en vais chasser devant toi les Amoréens, les Cananéens, les Héthiens, les Phéréziens, les Héviens, et les Jébusiens.
\VS{12}Garde-toi de traiter alliance avec les habitants du pays où tu dois entrer, de peur que peut-être ils ne soient un piège pour toi\FTNT{De. 7:2~; Jos. 23:12-13~; 2 Co. 6:14.}.
\VS{13}Mais vous démolirez leurs autels, vous briserez leurs statues, et vous couperez leurs emblèmes d'Asherah\FTNT{Cité au moins quarante fois dans le Tanakh, le terme hébreu «~Asherah~» fait référence à un «~arbre sacré, un pieu près d'un autel~» ou encore «~une idole~», terme par lequel il est majoritairement traduit. Il s'agit également de l'objet en bois utilisé dans le culte de la parèdre de Baal. Manassé, roi de Juda, introduisit l'emblème d'Asherah dans le temple (2 R. 21:1-7) en dépit de l'interdiction formelle de Yahweh (De. 16:21). Il n'en fut enlevé que lors des réformes de Josias et d'Ezéchias (2 R. 18:3-4~; 2 R. 23:6-14). Pourtant, Yahweh a toujours exigé la destruction de celle qu'il a nommé «~l'abomination des Sidoniens~», de peur que son peuple y trouve une occasion de chute (Jg. 2:13~; Jg. 10:6~; 1 S. 31:10~; 1 R. 11:5-33~; 2 R. 23:13). Bien qu'étant majoritairement citée dans les Ecritures en tant qu'objet de culte, Asherah est également associée à la divinité Astarté, connue pour être la Diane des Ephésiens (Ac. 19:23-40), la reine du ciel (Jé. 7:18~; Jé. 44:15-30), l'Isis des Egyptiens et l'épouse de Baal (voir commentaire en Jg. 2:13).}.
\VS{14}Car tu ne te prosterneras point devant un autre dieu, parce que Yahweh se nomme le Dieu jaloux~; c'est le Dieu qui est jaloux.
\VS{15}Afin qu'il n'arrive que tu traites alliance avec les habitants du pays, et que quand ils viendront à se prostituer après leurs dieux et à sacrifier à leurs dieux, quelqu'un ne t'invite et que tu ne manges de leurs sacrifices~;
\VS{16}et que tu ne prennes de leurs filles pour tes fils, lesquelles se prostituant après leurs dieux, n'entraînent tes fils à se prostituer après leurs dieux.
\VS{17}Tu ne te feras aucun dieu de métal fondu.
\TextTitle{Les fêtes et le sabbat\FTNTT{Lé. 23:4-44.}}
\VS{18}Tu garderas la fête solennelle des pains sans levain~; tu mangeras les pains sans levain pendant sept jours, comme je te l'ai ordonné, dans la saison où les épis mûrissent~; car c'est dans le mois des épis que tu es sorti du pays d'Egypte.
\VS{19}Tout premier-né sera à moi~; même le premier mâle qui naîtra de toutes les bêtes, tant du gros que du menu bétail.
\VS{20}Mais tu rachèteras avec un agneau ou un chevreau le premier-né d'un âne. Si tu ne le rachètes pas, tu lui couperas le cou. Tu rachèteras tout premier-né de tes fils~; et nul ne se présentera devant ma face à vide.
\VS{21}Tu travailleras six jours, mais au septième tu te reposeras~; tu te reposeras au temps du labourage et de la moisson.
\VS{22}Tu feras la fête solennelle des semaines au temps des premiers fruits de la moisson du froment~; et la fête solennelle de la récolte à la fin de l'année.
\VS{23}Trois fois l'an, tout mâle d'entre vous comparaîtra devant le Seigneur Yahweh, le Dieu d'Israël.
\VS{24}Car je déposséderai les nations de devant toi, et j'étendrai tes limites, et nul ne convoitera ton pays lorsque tu monteras pour comparaître trois fois l'an devant Yahweh, ton Dieu.
\VS{25}Tu n'offriras point le sang de mon sacrifice avec du pain levé~; on ne gardera rien du sacrifice de la fête solennelle de la Pâque jusqu'au matin.
\VS{26}Tu apporteras les prémices des premiers fruits de la terre dans la maison de Yahweh, ton Dieu. Tu ne feras point cuire le chevreau dans le lait de sa mère.
\VS{27}Yahweh dit aussi à Moïse~: Ecris ces paroles, car suivant la teneur de ces paroles, j'ai traité alliance avec toi et avec Israël.
\VS{28}Et Moïse demeura là avec Yahweh quarante jours et quarante nuits, sans manger de pain et sans boire d'eau~; et Yahweh écrivit sur les tables les paroles de l'alliance, c'est-à-dire les dix paroles.
\TextTitle{La gloire de Yahweh sur le visage de Moïse}
\VS{29}Or il arriva que lorsque Moïse descendait de la montagne de Sinaï, tenant dans sa main les deux tables du témoignage, lorsque, dis-je, il descendait de la montagne, il ne s'aperçut point que la peau de son visage était devenue rayonnante pendant qu'il parlait avec Dieu.
\VS{30}Mais Aaron et tous les enfants d'Israël ayant vu Moïse, et s'étant aperçus que la peau de son visage était rayonnante, ils craignirent de s'approcher de lui.
\VS{31}Mais Moïse les appela, et Aaron et tous les principaux de l'assemblée retournèrent vers lui~; et Moïse parla avec eux.
\VS{32}Après quoi, tous les enfants d'Israël s'approchèrent, et il leur ordonna toutes les choses que Yahweh lui avait dites sur la montagne de Sinaï.
\VS{33}Ainsi, Moïse acheva de leur parler~; or il avait mis un voile sur son visage.
\VS{34}Et quand Moïse entrait vers Yahweh pour parler avec lui, il ôtait le voile jusqu'à ce qu'il sorte~; et étant sorti, il disait aux enfants d'Israël ce qui lui avait été ordonné.
\VS{35}Or les enfants d'Israël avaient vu que le visage de Moïse, la peau, dis-je, de son visage rayonnait. C'est pourquoi Moïse remettait le voile sur son visage, jusqu'à ce qu'il entre pour parler avec Yahweh.
\Chap{35}
\TextTitle{Rappels sur le sabbat}
\VerseOne{}Moïse donc assembla toute la congrégation des enfants d'Israël, et leur dit~: Ce sont ici les choses que Yahweh a ordonnées de faire.
\VS{2}On travaillera six jours, mais le septième jour il y aura sainteté pour vous, car c'est le sabbat du repos consacré à Yahweh~; quiconque travaillera en ce jour-là sera puni de mort.
\VS{3}Vous n'allumerez point de feu dans aucune de vos demeures le jour du repos.
\TextTitle{Les offrandes pour le tabernacle\FTNTT{Ex. 25:1-8.}}
\VS{4}Puis Moïse parla à toute l'assemblée des enfants d'Israël, et leur dit~: C'est ici ce que Yahweh vous a ordonné, en disant~:
\VS{5}Prenez des choses qui sont chez vous une offrande pour Yahweh. Quiconque sera de bonne volonté, apportera cette offrande pour Yahweh, à savoir de l'or, de l'argent, de l'airain\FTNT{Ex. 25:2~; 2 Co. 8:12.},
\VS{6}de la pourpre, de l'écarlate, du cramoisi, du fin lin, du poil de chèvre,
\VS{7}des peaux de béliers teintes en rouge, des peaux de taissons, du bois d'acacia,
\VS{8}de l'huile pour le chandelier, des aromates pour l'huile d'onction et pour le parfum odoriférant,
\VS{9}des pierres d'onyx, et des pierres pour la garniture de l'éphod et pour le pectoral.
\VS{10}Et tous les hommes d'esprit d'entre vous viendront et feront tout ce que Yahweh a ordonné, 
\VS{11}à savoir le tabernacle, sa tente, et sa couverture, ses agrafes, ses planches, ses barres, ses colonnes, et ses bases~;
\VS{12}l'arche et ses barres, le propitiatoire et le voile qui sert de rideau~;
\VS{13}la table et ses barres, et tous ses ustensiles, et le pain de proposition~;
\VS{14}et le chandelier du luminaire, ses ustensiles, ses lampes et l'huile du luminaire~;
\VS{15}l'autel du parfum et ses barres~; l'huile d'onction, le parfum odoriférant, le rideau de la porte pour l'entrée du tabernacle~;
\VS{16}l'autel de l'holocauste, sa grille d'airain, ses barres et tous ses ustensiles~; la cuve avec sa base~;
\VS{17}les courtines du parvis, ses colonnes, ses bases, et le rideau de la porte du parvis~;
\VS{18}les pieux du tabernacle, les pieux du parvis et leur cordage~;
\VS{19}les vêtements du service pour faire le service dans le sanctuaire, les saints vêtements d'Aaron, le prêtre, et les vêtements de ses fils pour exercer la prêtrise.
\VS{20}Alors toute l'assemblée des enfants d'Israël sortit de la présence de Moïse.
\VS{21}Et quiconque fut ému en son cœur, quiconque, dis-je, se sentit porté à la libéralité, apporta l'offrande de Yahweh pour l'ouvrage de la tente d'assignation et pour tout son service et pour les saints vêtements.
\VS{22}Et les hommes vinrent avec les femmes~; quiconque fut de cœur volontaire, apporta des boucles, des bagues, des anneaux, des bracelets, et des joyaux d'or~; et quiconque offrit quelque offrande d'or à Yahweh.
\VS{23}Tout homme aussi chez qui se trouvait de la pourpre, de l'écarlate, du cramoisi, du fin lin, du poil de chèvre, des peaux de béliers teintes en rouge et des peaux de taissons, les apportèrent.
\VS{24}Tout homme qui avait de quoi faire une offrande d'argent et d'airain, l'apporta pour l'offrande de Yahweh~; tout homme aussi chez qui fut trouvé du bois d'acacia pour tout l'ouvrage du service, l'apporta.
\VS{25}Toute femme adroite fila de sa main et apporta ce qu'elle avait filé~: De la pourpre, de l'écarlate, du cramoisi, et du fin lin\FTNT{Pr. 31:19.}.
\VS{26}Toutes les femmes aussi dont le cœur les y porta en sagesse, filèrent du poil de chèvre.
\VS{27}Les principaux aussi de l'assemblée apportèrent des pierres d'onyx, et d'autres pierres pour la garniture de l'éphod et du pectoral~;
\VS{28}et des aromates, et de l'huile tant pour le chandelier que pour l'huile d'onction, et pour le parfum odoriférant.
\VS{29}Tout homme donc et toute femme que le cœur incita à la libéralité pour apporter de quoi faire l'ouvrage que Yahweh avait ordonné par le moyen de Moïse, tous les enfants, dis-je, d'Israël apportèrent volontairement des présents à Yahweh.
\TextTitle{Betsaleel et Oholiab oints pour l'œuvre du tabernacle}
\VS{30}Alors Moïse dit aux fils d'Israël~: Voyez, Yahweh a appelé par son nom Betsaleel, fils d'Uri, fils de Hur, de la tribu de Juda.
\VS{31}Et il l'a rempli de l'Esprit de Dieu, de sagesse, d'intelligence, de science, pour toutes sortes d'ouvrages.
\VS{32}Même afin d'inventer des dessins, pour travailler l'or, l'argent et l'airain~;
\VS{33}dans la sculpture des pierres précieuses pour les mettre en œuvre, et dans la menuiserie pour travailler en tout ouvrage exquis.
\VS{34}Et il lui a mis aussi au cœur, tant à lui qu'à Oholiab, fils d'Ahisamac, de la tribu de Dan, de l'enseigner.
\VS{35}Et il les a remplis de sagesse pour faire toutes sortes d'ouvrages d'ouvrier, même d'ouvrier en ouvrage exquis, et en broderie, en pourpre, en écarlate, en cramoisi, et en fin lin, et d'ouvrage de tisserand, faisant toutes sortes d'ouvrages et inventant toutes sortes de dessins\FTNT{Es. 28:26.}.
\Chap{36}
\TextTitle{Construction du tabernacle d'après le modèle donné par Yahweh\FTNTT{Ex. 36-39.}}
\VerseOne{}Et Betsaleel et Oholiab, et tous les hommes au coeur sage auxquels Yahweh avait donné de la sagesse et de l'intelligence pour savoir faire tout l'ouvrage du service du sanctuaire, firent selon toutes les choses que Yahweh avait ordonnées.
\VS{2}Moïse donc appela Betsaleel et Oholiab, et tous les hommes d'esprit, dans le cœur desquels Yahweh avait mis de la sagesse, et tous ceux qui furent émus en leur cœur de se présenter pour faire cet ouvrage.
\VS{3}Lesquels emportèrent de devant Moïse toute l'offrande que les enfants d'Israël avaient apportée pour faire l'ouvrage du service du sanctuaire. Or on apportait encore chaque matin quelques offrandes volontaires.
\VS{4}C'est pourquoi, tous les hommes sages qui faisaient tout l'ouvrage du sanctuaire, vinrent chacun de l'ouvrage qu'ils faisaient,
\VS{5}et parlèrent à Moïse, en disant~: Le peuple ne cesse d'apporter plus qu'il ne faut pour le service et pour l'ouvrage que Yahweh a ordonné de faire.
\VS{6}Alors, par l'ordre de Moïse, on fit crier dans le camp que ni homme ni femme ne fasse plus d'ouvrage pour l'offrande du sanctuaire~; et ainsi, on empêcha le peuple d'offrir.
\VS{7}Car ils avaient du travail suffisant pour tout l'ouvrage à faire, et il y en avait même de reste.
\TextTitle{Les tapis de fin lin}
\VS{8}Tous les hommes donc au coeur sage d'entre ceux qui faisaient l'ouvrage, firent le tabernacle, à savoir dix tapis de fin lin retors, de pourpre, d'écarlate, et de cramoisi~; et ils les firent semés de chérubins d'un ouvrage exquis.
\VS{9}La longueur d'un tapis était de vingt-huit coudées, et la largeur du même tapis de quatre coudées~; tous les tapis avaient une même mesure\FTNT{Ex. 26:1-6.}.
\VS{10}Et ils joignirent cinq tapis l'un à l'autre, et cinq autres tapis l'un à l'autre.
\VS{11}Et ils firent des lacets de pourpre sur le bord d'un tapis, à savoir au bord de celui qui était attaché~; ils en firent ainsi au bord du dernier tapis dans l'assemblage de l'autre.
\VS{12}Ils firent cinquante lacets à un tapis, et cinquante lacets au bord du tapis qui était dans l'assemblage de l'autre~; les lacets étant vis-à-vis l'un de l'autre.
\VS{13}Puis on fit cinquante agrafes d'or, et on attacha les tapis l'un à l'autre avec les agrafes~; ainsi fut fait le tabernacle.
\TextTitle{Les tapis de poils de chèvres}
\VS{14}Puis on fit des tapis de poils de chèvres, pour servir de tente au-dessus du tabernacle~; on fit onze de ces tapis.
\VS{15}La longueur d'un tapis était de trente coudées, et la largeur du même tapis de quatre coudées~; et les onze tapis étaient d'une même mesure.
\VS{16}Et on assembla cinq de ces tapis à part, et six tapis à part.
\VS{17}On fit aussi cinquante lacets sur le bord de l'un des tapis, à savoir au dernier qui était attaché, et cinquante lacets sur le bord de l'autre tapis, qui était attaché.
\VS{18}On fit aussi cinquante agrafes d'airain pour assembler la tente, afin qu'il n'y en eut qu'une.
\TextTitle{Les couvertures de peaux de béliers et de taissons}
\VS{19}Puis, on fit pour la tente une couverture de peaux de béliers teintes en rouge, et une couverture de peaux de taissons par-dessus.
\TextTitle{Les planches et leurs bases}
\VS{20}Et on fit pour le tabernacle des planches de bois d'acacia, qu'on fit tenir debout.
\VS{21}La longueur d'une planche était de dix coudées, et la largeur de la même planche d'une coudée et demie.
\VS{22}Il y avait deux tenons à chaque planche en façon d'échelons l'un après l'autre~; on fit la même chose pour toutes les planches du tabernacle.
\VS{23}On fit donc les planches pour le tabernacle~; à savoir vingt planches au côté qui regardaient directement vers le sud.
\VS{24}Et au-dessous des vingt planches, on fit quarante bases d'argent, deux bases sous une planche, pour ses deux tenons, et deux bases sous l'autre planche, pour ses deux tenons.
\VS{25}On fit aussi vingt planches pour l'autre côté du tabernacle, du côté nord,
\VS{26}et leurs quarante bases d'argent~: Deux bases sous une planche, et deux bases sous l'autre planche.
\VS{27}Et pour le fond du tabernacle, vers l'occident, on fit six planches.
\VS{28}Et on fit deux planches pour les angles du tabernacle aux deux cotés du fond~;
\VS{29}qui étaient égales par le bas, et qui étaient jointes et unies par le haut avec un anneau~; on fit la même chose aux deux planches qui étaient aux deux angles.
\VS{30}Il y avait donc huit planches et seize bases d'argent, à savoir deux bases sous chaque planche.
\VS{31}Puis on fit cinq barres de bois d'acacia, pour les planches de l'un des côtés du tabernacle~;
\VS{32}et cinq barres pour les planches de l'autre côté du tabernacle~; et cinq barres pour les planches du tabernacle pour le fond, vers le côté de l'occident.
\VS{33}Et on fit que la barre du milieu passait par le milieu des planches d'une extrémité à l'autre.
\VS{34}Et on couvrit d'or les planches, et on fit leurs anneaux d'or pour y faire passer les barres, et on couvrit d'or les barres.
\TextTitle{Le voile et le rideau extérieur}
\VS{35}On fit aussi le voile de pourpre, d'écarlate, de cramoisi, et de fin lin retors~; on le fit d'ouvrage exquis, avec des chérubins.
\VS{36}Et on lui fit quatre piliers de bois d'acacia, qu'on couvrit d'or, ayant leurs crochets d'or~; et on fondit pour eux quatre bases d'argent.
\VS{37}On fit aussi à l'entrée de la tente un rideau de pourpre, d'écarlate, de cramoisi, et de fin lin retors~; d'ouvrage de broderie~;
\VS{38}et ses cinq piliers avec leurs crochets~; et on couvrit d'or leurs chapiteaux et leurs filets~; mais leurs cinq bases étaient d'airain.
\Chap{37}
\TextTitle{L'arche de l'alliance}
\VerseOne{}Puis Betsaleel fit l'arche de bois d'acacia. Sa longueur était de deux coudées et demie, et sa largeur d'une coudée et demie, et sa hauteur d'une coudée et demie\FTNT{Ex. 23:10-31.}.
\VS{2}Et il la couvrit par dedans et par dehors de pur or, et lui fit un couronnement d'or tout autour.
\VS{3}Et il lui fondit pour elle quatre anneaux d'or pour les mettre sur ses quatre coins, à savoir deux anneaux à l'un de ses côtés, et deux autres à l'autre côté.
\VS{4}Et il fit aussi des barres de bois d'acacia, et les couvrit d'or.
\VS{5}Et il fit entrer les barres dans les anneaux aux côtés de l'arche, pour porter l'arche.
\TextTitle{Le propitiatoire}
\VS{6}Il fit aussi le propitiatoire d'or pur~; sa longueur était de deux coudées et demie, et sa largeur d'une coudée et demie.
\VS{7}Et il fit deux chérubins d'or~; il les fit d'ouvrage étendu au marteau, tirés des deux extrémités du propitiatoire~;
\VS{8}à savoir un chérubin tiré de l'une des extrémités et un chérubin tiré de l'autre extrémité~; il fit, dis-je, les chérubins tirés du propitiatoire~; à savoir de ses deux extrémités. 
\VS{9}Et les chérubins étendaient leurs ailes en haut, couvrant de leurs ailes le propitiatoire~; et leurs faces étaient vis-à-vis l'une de l'autre, et les chérubins regardaient vers le propitiatoire.
\TextTitle{La table des pains de proposition}
\VS{10}Il fit aussi la table de bois d'acacia~; sa longueur était de deux coudées, et sa largeur d'une coudée, et sa hauteur d'une coudée et demie.
\VS{11}Et il la couvrit d'or pur, et lui fit un couronnement d'or tout autour.
\VS{12}Il lui fit aussi à l'entour un rebord d'une largeur d'une paume, et à l'entour de sa bordure un couronnement d'or.
\VS{13}Et il lui fondit quatre anneaux d'or, et il mit les anneaux aux quatre coins, qui étaient à ses quatre pieds.
\VS{14}Les anneaux étaient à coté du rebord, pour y mettre les barres afin de porter la table avec elles.
\VS{15}Et il fit les barres de bois d'acacia, et les couvrit d'or pour porter la table.
\VS{16}Il fit aussi d'or pur des ustensiles pour poser sur la table, ses plats, ses tasses, ses bassins et ses gobelets avec lesquels on devait faire les aspersions.
\TextTitle{Le chandelier}
\VS{17}Il fit aussi le chandelier d'or pur~; il le fit d'ouvrage façonné au marteau~; sa tige, ses branches, ses plats, ses pommeaux et ses fleurs étaient tirés de lui.
\VS{18}Et six branches sortaient de ses côtés, trois branches d'un côté du chandelier, et trois de l'autre côté du chandelier.
\VS{19}Il y avait sur l'une des branches trois plats en forme d'amande, un pommeau et une fleur~; et sur l'autre branche trois plats en forme d'amande, un pommeau et une fleur~; il fit la même chose aux six branches qui sortaient du chandelier.
\VS{20}Et il y avait sur le chandelier quatre plats en forme d'amande, ses pommeaux et ses fleurs.
\VS{21}Et un pommeau sous deux branches tirées du chandelier, et un pommeau sous deux autres branches, tirées de lui, et un pommeau sous deux autres branches, tirées de lui, à savoir des six branches qui procédaient du chandelier.
\VS{22}Leurs pommeaux et leurs branches étaient tirés de lui, et tout le chandelier était un ouvrage d'une seule pièce étendue au marteau et d'or pur.
\VS{23}Il fit aussi ses sept lampes, ses mouchettes, et ses encensoirs d'or pur.
\VS{24}Et il le fit avec toute sa garniture d'un talent d'or pur.
\TextTitle{L'autel des parfums}
\VS{25}Il fit aussi de bois d'acacia l'autel des parfums. Sa longueur était d'une coudée, et sa largeur d'une coudée. Il était carré, et sa hauteur était de deux coudées, et ses cornes procédaient de lui\FTNT{Ex. 30:1-10.}.
\VS{26} Et il couvrit d'or pur le dessus de l'autel, ses côtés tout à l'entour, et ses cornes~; et il lui fit tout à l'entour un couronnement d'or.
\VS{27}Il fit aussi au-dessous de son couronnement deux anneaux d'or à ses deux côtés, lesquels il mit aux deux coins, pour y faire passer les barres afin de le porter avec elles.
\VS{28}Et il fit les barres de bois d'acacia, et les couvrit d'or.
\TextTitle{L'huile d'onction et le parfum}
\VS{29}Il composa aussi l'huile pour l'onction, qui était une chose sainte, et le parfum pur odoriférant, d'ouvrage de parfumeur.
\Chap{38}
\TextTitle{L'autel des holocaustes}
\VerseOne{}Il fit aussi de bois d'acacia l'autel des holocaustes. Et sa longueur était de cinq coudées, et sa largeur de cinq coudées. Il était carré, et sa hauteur était de trois coudées\FTNT{Ex. 27:1-8.}.
\VS{2}Et il fit ses cornes à ses quatre coins. Ses cornes sortaient de lui, et il le couvrit d'airain.
\VS{3}Il fit aussi tous les ustensiles de l'autel~: Les chaudrons, les racloirs, les bassins, les fourchettes et les encensoirs~; il fit tous ses ustensiles d'airain.
\VS{4}Et il fit pour l'autel une grille d'airain en forme de treillis, au-dessous de l'enceinte de l'autel, depuis le bas jusqu'au milieu.
\VS{5}Et il fondit quatre anneaux aux quatre coins de la grille d'airain pour mettre les barres.
\VS{6} Et il fit les barres de bois d'acacia, et les couvrit d'airain.
\VS{7}Et il fit passer les barres dans les anneaux, au cotés de l'autel, pour le porter avec elles. Il le fit creux, avec des planches.
\TextTitle{La cuve d'airain}
\VS{8}Il fit aussi la cuve d'airain et sa base d'airain avec les miroirs des femmes qui s'assemblaient à l'entrée de la tente d'assignation\FTNT{Ex. 30:14-18.}.
\TextTitle{Le parvis}
\VS{9}Il fit aussi un parvis, pour le côté qui regarde vers le sud, et des courtines de fin lin retors, de cent coudées, pour le parvis.
\VS{10}Il fit d'airain leurs vingt colonnes avec leurs vingt bases, mais les crochets des colonnes et leurs filets étaient d'argent.
\VS{11}Et pour le côté nord, il fit des courtines de cent coudées, leurs vingt colonnes et leurs vingt bases étaient d'airain, mais les crochets des colonnes et leurs filets étaient d'argent.
\VS{12}Pour le côté de l'occident, des courtines de cinquante coudées, leurs dix colonnes, et leurs dix bases. Les crochets des colonnes et leurs filets étaient d'argent.
\VS{13}Pour le côté de l'orient droit vers le levant, des courtines de cinquante coudées.
\VS{14}Il fit pour l'un des côtés quinze coudées de courtines, et leurs trois colonnes avec leurs trois bases.
\VS{15}Et pour l'autre côté, quinze coudées de courtines, afin qu'il y en ait autant de part et d'autre de la porte du parvis, et leurs trois colonnes avec leurs trois bases.
\VS{16}Il fit donc toutes les courtines du parvis qui étaient tout autour de fin lin retors.
\VS{17}Il fit aussi d'airain les bases des colonnes, mais il fit d'argent les crochets des colonnes et les filets, et leurs chapiteaux furent couverts d'argent~; et toutes les colonnes du parvis furent ceintes tout autour d'un filet d'argent.
\TextTitle{La porte du parvis}
\VS{18} Et il fit le rideau de la porte du parvis de pourpre, d'écarlate, et de cramoisi et de fin lin retors, d'ouvrage de broderie, de la longueur de vingt coudées, et de la hauteur qui était comme la largeur de cinq coudées, à la correspondance des courtines du parvis~;
\VS{19}et ses quatre colonnes avec leurs bases d'airain, et leurs crochets d'argent, la couverture aussi de leur chapiteaux et leurs filets d'argent~; 
\VS{20} et tous les pieux du tabernacle et du parvis tout autour d'airain.
\TextTitle{Les comptes du tabernacle}
\VS{21}C'est ici le compte des choses qui furent employées au tabernacle, à savoir à la tente d'assignation, selon que le compte en fut fait par l'ordre de Moïse, à quoi furent employés les Lévites, sous la conduite d'Ithamar, fils du prêtre Aaron.
\VS{22}Et Betsaleel, fils d'Uri, fils de Hur, de la tribu de Juda, fit toutes les choses que Yahweh avait ordonnées à Moïse~;
\VS{23}et avec lui Oholiab, fils d'Ahisamac, de la tribu de Dan, les ouvriers, et ceux qui travaillaient en ouvrage exquis, et les brodeurs en pourpre, en écarlate, en cramoisi, et en fin lin.
\VS{24}Tout l'or qui fut employé pour l'ouvrage, à savoir pour tout l'ouvrage du sanctuaire, qui était l'or des offrandes, fut de vingt-neuf talents et sept cent trente sicles, selon le sicle du sanctuaire.
\VS{25} Et l'argent de ceux de l'assemblée qui furent dénombrés fut de cent talents et mille sept cent soixante-quinze sicles, selon le sicle du sanctuaire.
\VS{26}Un demi-sicle par tête, la moitié d'un sicle selon le sicle du sanctuaire. Tous ceux qui passèrent par le dénombrement depuis l'âge de vingt ans et au-dessus, furent six cent trois mille cinq cent cinquante.
\VS{27}Il y eut donc cent talents d'argent pour fondre les bases du sanctuaire, et les bases du voile, à savoir cent bases de cent talents, un talent pour chaque base.
\VS{28}Mais des mille sept cent soixante-quinze sicles, il fit les crochets pour les colonnes, et il couvrit leurs chapiteaux et en fit des filets tout autour.
\VS{29}L'airain des offrandes fut de soixante-dix talents et deux mille quatre cents sicles~;
\VS{30}dont on fit les bases de la porte de la tente d'assignation, et l'autel d'airain avec sa grille d'airain, et tous les ustensiles de l'autel~;
\VS{31}et les bases tout autour du parvis, les bases de la porte du parvis, et tous les pieux du tabernacle, et tous les pieux du parvis tout autour.
\Chap{39}
\TextTitle{Les vêtements sacrés d'Aaron}
\VerseOne{}Ils firent aussi de pourpre, d'écarlate, et de cramoisi les vêtements du service, pour faire le service du sanctuaire. Et ils firent les saints vêtements sacrés pour Aaron, comme Yahweh l'avait ordonné à Moïse\FTNT{Ex. 28.}.
\VS{2}On fit donc l'éphod d'or, de pourpre, d'écarlate, de cramoisi, et de fin lin retors.
\VS{3}Or on étendit des lames d'or, et on les coupa par filets pour les brocher parmi la pourpre, l'écarlate, le cramoisi, et le fin lin d'ouvrage exquis.
\VS{4}On fit à l'éphod des épaulettes\FTNT{Voir annexe «~Les habits du grand-prêtre~».} qui s'attachaient, en sorte qu'il était joint par ses deux extrémités.
\VS{5}Et la ceinture exquise de laquelle il était ceint, était tirée de lui, et de même ouvrage, d'or, de pourpre, d'écarlate, de cramoisi, et de fin lin retors, comme Yahweh l'avait ordonné à Moïse.
\VS{6}On enchassa aussi les pierres d'onyx dans leurs montures d'or, ayant les noms des enfants d'Israël gravés comme on grave les cachets.
\VS{7}Et on les mit sur les épaulettes de l'éphod, afin qu'elles soient des pierres de souvenir pour les enfants d'Israël, comme Yahweh l'avait ordonné à Moïse.
\VS{8}On fit aussi le pectoral\FTNT{Voir annexe «~Les habits du grand-prêtre~».} d'ouvrage exquis, comme l'ouvrage de l'éphod, d'or, de pourpre, d'écarlate, de cramoisi, et de fin lin retors.
\VS{9}On fit le pectoral carré et double~; sa longueur était d'une paume, et sa largeur d'une paume de part et d'autre.
\VS{10}Et on le garnit de quatre rangs de pierres~: A la première rangée on mit une sardoine, une topaze et une émeraude.
\VS{11}A la seconde rangée une escarboucle, un saphir, et un jaspe.
\VS{12}A la troisième rangée, une opale, une agate, et une améthyste.
\VS{13}A la quatrième rangée, un chrysolithe, un onyx, et un béryl\FTNT{Ap. 21:18-19.}, enchâssés dans leur monture d'or.
\VS{14}Ainsi, il y avait autant de pierres qu'il y avait de noms des enfants d'Israël, douze selon leurs noms, chacune d'elles gravées comme des cachets, selon le nom qu'elle devait porter, et elles étaient pour les douze tribus.
\VS{15}Et on fit sur le pectoral des chaînettes à bouts en façon de cordon, d'or pur.
\VS{16}On fit aussi deux montures d'or et deux anneaux d'or, on mit les deux anneaux aux deux extrémités du pectoral.
\VS{17}Et on mit les deux chaînettes d'or faites à cordon dans les deux anneaux à l'extrémité du pectoral.
\VS{18}Et on mit les deux autres bouts des deux chaînettes faites à cordon aux deux montures, sur les épaulettes de l'éphod, sur le devant de l'éphod.
\VS{19}On fit aussi deux autres anneaux d'or, et on les mit aux deux autres extrémités du pectoral sur son bord, qui était du côté de l'éphod à l'intérieur.
\VS{20}On fit aussi deux autres anneaux d'or, et on les mit aux deux épaulettes de l'éphod par le bas, répondant sur le devant de l'éphod, à l'endroit où il se joignait au-dessus de la ceinture exquise de l'éphod.
\VS{21}Et on joignit le pectoral élevé par ses anneaux aux anneaux de l'éphod, avec un cordon de pourpre, afin qu'il tienne au-dessus de la ceinture exquise de l'éphod, et que le pectoral ne bouge de dessus l'éphod, comme Yahweh l'avait ordonné à Moïse.
\VS{22}On fit aussi la robe de l'éphod d'ouvrage tissé et entièrement de pourpre.
\VS{23} Et l'ouverture pour passer la tête était au milieu de la robe, comme l'ouverture d'une cotte de mailles~; et il y avait un ourlet à l'ouverture de la robe tout autour, afin qu'elle ne se déchire pas.
\VS{24}Aux bordures de la robe, on fit des grenades de pourpre, d'écarlate et de cramoisi, à fil retors.
\VS{25}On fit aussi des clochettes d'or pur~; et on mit les clochettes entre les grenades aux bordures de la robe tout autour, parmi les grenades~;
\VS{26}à savoir une clochette puis une grenade, une clochette puis une grenade, sur la bordure de la robe tout autour, pour faire le service, comme Yahweh l'avait ordonné à Moïse.
\VS{27}On fit aussi à Aaron et à ses fils des tuniques de fin lin d'ouvrage tissé.
\VS{28}Et la tiare de fin lin, et les ornements des calottes de fin lin, et les caleçons de lin, de fin lin retors.
\VS{29}Et la ceinture de fin lin retors, de pourpre, d'écarlate et de cramoisi, d'ouvrage de broderie~; comme Yahweh l'avait ordonné à Moïse~;
\VS{30}et la lame du saint diadème d'or pur, sur laquelle on écrivit comme on grave un cachet~: La sainteté à Yahweh.
\VS{31}Et on mit sur elle un cordon de pourpre, pour l'appliquer à la tiare par-dessus, comme Yahweh l'avait ordonné à Moïse.
\TextTitle{Le matériel pour excercer la prêtrise est prêt}
\VS{32}Ainsi fut achevé tout l'ouvrage du tabernacle, de la tente d'assignation. Les enfants d'Israël firent selon toutes les choses que Yahweh avait ordonnées à Moïse~; ils les firent ainsi.
\VS{33}Et ils apportèrent à Moïse le tabernacle, la tente, et tous ses ustensiles, ses crochets, ses planches, ses barres, ses colonnes, et ses bases~;
\VS{34}la couverture de peaux de béliers teintes en rouge, la couverture de peaux de taissons, et le voile qui sert de rideau devant le Saint des saints~;
\VS{35}l'arche du témoignage et ses barres, et le propitiatoire~;
\VS{36}la table avec tous ses ustensiles, et les pains de proposition\FTNT{Ex. 31:8-10.}~;
\VS{37}et le chandelier d'or pur avec toutes ses lampes arrangées, et tous ses ustensiles, et l'huile du chandelier~;
\VS{38}et l'autel d'or, l'huile d'onction, le parfum odoriférant, et le rideau de l'entrée de la tente~;
\VS{39}l'autel d'airain, avec sa grille d'airain, ses barres et tous ses ustensiles~; la cuve et sa base~;
\VS{40}et les courtines du parvis, ses colonnes, ses bases, le rideau pour la porte du parvis, son cordage, ses pieux, et tous les ustensiles pour le service du tabernacle, pour la tente d'assignation~;
\VS{41}les vêtements du service pour faire le service du sanctuaire, les saints vêtements pour le prêtre Aaron, et les vêtements de ses fils pour exercer la prêtrise.
\VS{42}Les enfants d'Israël donc firent tout l'ouvrage, comme Yahweh l'avait ordonné à Moïse.
\VS{43}Et Moïse vit tout l'ouvrage, et voici, on l'avait fait ainsi que Yahweh l'avait ordonné, on l'avait, dis-je, fait ainsi. Et Moïse les bénit.
\Chap{40}
\TextTitle{Moïse dresse le tabernacle}
\VerseOne{}Et Yahweh parla à Moïse, en disant~:
\VS{2}Au premier jour du premier mois, tu dresseras le tabernacle de la tente d'assignation.
\VS{3}Et tu y mettras l'arche du témoignage, au-devant de laquelle tu tendras le voile.
\VS{4}Puis tu apporteras la table et y arrangeras ce qui doit y être arrangé. Tu apporteras aussi le chandelier et allumeras ses lampes.
\VS{5}Tu mettras aussi l'autel d'or pour le parfum au-devant de l'arche du témoignage, et tu mettras le rideau de l'entrée au tabernacle.
\VS{6}Tu mettras aussi l'autel de l'holocauste vis-à-vis de l'entrée du tabernacle de la tente d'assignation.
\VS{7}Tu mettras aussi la cuve entre la tente d'assignation et l'autel, et y mettras de l'eau.
\VS{8}Tu mettras aussi le parvis tout autour, et tu mettras le rideau à la porte du parvis.
\VS{9}Tu prendras aussi l'huile de l'onction, et tu en oindras le tabernacle, et tout ce qui y est, et tu le sanctifieras avec tous ses ustensiles~; et il sera saint.
\VS{10}Tu oindras aussi l'autel de l'holocauste, et tous ses ustensiles, et tu sanctifieras l'autel, et l'autel sera très saint.
\VS{11}Tu oindras aussi la cuve et sa base, et la sanctifieras.
\VS{12}Tu feras aussi approcher Aaron et ses fils à l'entrée de la tente d'assignation, et les laveras avec de l'eau.
\VS{13}Et tu feras vêtir à Aaron les saints vêtements, et tu l'oindras et le sanctifieras~; et il exercera la prêtrise pour moi.
\VS{14}Tu feras aussi approcher ses fils que tu revêtiras de tuniques.
\VS{15} Et tu les oindras comme tu auras oint leur père~; et ils m'exerceront la prêtrise, et leur onction leur sera pour exercer la prêtrise à toujours parmi leur génération.
\VS{16} Ce que Moïse fit selon toutes les choses que Yahweh lui avait ordonnées~; il le fit ainsi.
\VS{17}Car au premier jour du premier mois de la seconde année, le tabernacle fut dressé.
\VS{18}Moïse donc dressa le tabernacle, mit ses bases, posa ses planches, mit ses barres et dressa ses colonnes.
\VS{19}Et il étendit la tente sur le tabernacle, et mit la couverture de la tente au-dessus du tabernacle par le haut, comme Yahweh l'avait ordonné à Moïse.
\VS{20}Puis il prit et posa le témoignage dans l'arche et mit les barres à l'arche~; il mit aussi le propitiatoire au-dessus de l'arche.
\VS{21}Et il apporta l'arche dans le tabernacle, et posa le voile qui sert de rideau, et le mit au-devant de l'arche du témoignage, comme Yahweh l'avait ordonné à Moïse.
\VS{22}Il mit aussi la table dans la tente d'assignation, au côté du tabernacle vers le nord, en dehors du voile.
\VS{23}Et il arrangea sur elle les rangées de pains devant Yahweh, comme Yahweh l'avait ordonné à Moïse.
\VS{24}Il mit aussi le chandelier dans la tente d'assignation, vis-à-vis de la table, du côté du tabernacle, vers le sud.
\VS{25}Et il alluma les lampes devant Yahweh, comme Yahweh l'avait ordonné à Moïse.
\VS{26}Il posa aussi l'autel d'or dans la tente d'assignation, devant le voile.
\VS{27}Et il fit fumer sur lui le parfum odoriférant, comme Yahweh l'avait ordonné à Moïse.
\VS{28}Il mit aussi le rideau de l'entrée du tabernacle.
\VS{29}Et il mit l'autel de l'holocauste à l'entrée du tabernacle de la tente d'assignation~; et offrit sur lui l'holocauste et l'offrande, comme Yahweh l'avait ordonné à Moïse.
\VS{30}Et il plaça la cuve entre la tente d'assignation et l'autel, et y mit de l'eau pour se laver.
\VS{31}Et Moïse et Aaron avec ses fils en lavèrent leurs mains et leurs pieds.
\VS{32}Et quand ils entraient dans la tente d'assignation, et qu'ils approchaient de l'autel, ils se lavaient, selon que Yahweh l'avait ordonné à Moïse.
\VS{33}Il dressa aussi le parvis tout autour du tabernacle et de l'autel, et tendit le rideau de la porte du parvis. Ainsi Moïse acheva l'ouvrage.
\TextTitle{La gloire de Yahweh sur le tabernacle}
\VS{34}Et la nuée couvrit la tente d'assignation, et la gloire de Yahweh remplit le tabernacle\FTNT{No. 9:15~; 1 R. 8:10.},
\VS{35}tellement que Moïse ne put entrer dans la tente d'assignation, car la nuée se tenait dessus et la gloire de Yahweh remplissait le tabernacle.
\VS{36} Or quand la nuée se levait de dessus le tabernacle, les enfants d'Israël partaient dans toutes leurs marches.
\VS{37}Mais si la nuée ne se levait point, ils ne partaient point, jusqu'au jour où elle se levait.
\VS{38}Car la nuée de Yahweh était le jour sur le tabernacle, et le feu y était la nuit, devant les yeux de toute la maison d'Israël, dans toutes leurs marches.
\PPE
\end{multicols}

%\clearpage\ShortTitle{Lé.}\BookTitle{Lévitique}\BFont
\noindent\hrulefill
{\footnotesize
\textit{
\bigskip
{\centering{}
\\Auteur : Probablement Moïse
\\(Heb. : Vayiqra)
\\Signification : Et Il (Yahweh) appela
\\Thème : La sainteté
\\Date de rédaction : Env. 1450-1410 av. J.-C.\\}
}
%\bigskip
\textit{
\\Après avoir construit et dressé le tabernacle selon le modèle que Yahweh avait donné à Moïse, les fils d'Israël reçurent le détail des prescriptions relatives aux offrandes, aux sacrifices et aux fêtes en l'honneur de Yahweh. 
%\bigskip
\\Ce livre, dont le nom tire son origine de Lévi, explique la manière dont Aaron et ses fils devaient exercer la sacrificature et amener le peuple à s'approcher de Dieu dans le respect de ses ordonnances.
%\bigskip
\\Les lois que Moïse avait recueillies présentent la voie du pardon, laquelle est impossible sans effusion de sang. Bien que les mêmes sacrifices furent réitérés tous les ans, ces préceptes mettaient en évidence l'impuissance de l'homme à atteindre la justice de Dieu par ses propres moyens.\bigskip
}
}
\par\nobreak\noindent\hrulefill
\begin{multicols}{2}
\Chap{1}
\TextTitle{L'holocauste\FTNTT{voir Lé. 6:1-6.}}
\VerseOne{}Et Yahweh appela Moïse, et lui parla de la tente d'assignation, en disant :
\VS{2}Parle aux enfants d'Israël, et dis-leur : Quand quelqu'un d'entre vous offrira à Yahweh une offrande d'une bête à quatre pattes, il fera son offrande de gros ou de menu bétail.
\VS{3}Si son offrande pour un holocauste est de gros bétail, il offrira un mâle sans défaut\FTNT{L'holocauste était le sacrifice pour l'expiation par excellence. Contrairement aux autres sacrifices, l'holocauste était entièrement consumé sur l'autel. Il symbolisait d'une part le sacrifice parfait de Christ et d'autre part notre vie, volontairement offerte à Dieu (Ro. 12:1). Les animaux aptes à être offerts en holocauste devaient être des mâles sans défaut :
\\- Le veau (Lé. 1:5), image de Christ, l'humble serviteur, soumis et obéissant (Mt. 20:28 ; Ph. 2:5-8).
\\- L'agneau ou le chevreau, image de Christ qui livre sa vie à la croix sans résistance ni contestation, et qui prend sur lui nos péchés (Es. 53:7 ; Mt. 26:63 ; Ac. 8:32). 
\\- Les tourterelles ou les jeunes pigeons, image de la simplicité de Christ (Mt. 10:16).
\\Toutes les étapes de la réalisation de ce sacrifice enseignent le disciple sur la mort à soi-même et le dépouillement des œuvres de la chair (Ga. 5:19-21).
\\Le sang de l'animal égorgé devait être répandu sur l'autel (Lé. 1:5), image de la croix. L'âme (contenue dans le sang selon Lé. 17:14), liée à la chair et ses désirs, doit être crucifiée (Ga. 2:20 ; Ga. 5:24). L'objet de la mise à mort était certainement un couteau tranchant comme une épée, image de la Parole de Dieu (Hé. 4:12). La mise en pratique de la Parole nous amène nécessairement à nous séparer du monde et à renoncer à soi-même.} ; il l'offrira de son bon gré à l'entrée de la tente d'assignation ; devant Yahweh\FTNT{Ex. 29:10-11.}.
\VS{4}Et il posera sa main sur la tête de l'holocauste, et il sera agréé pour lui, afin de faire la propitiation pour lui.
\VS{5}Puis, on égorgera le jeune taureau devant Yahweh ; et les fils d'Aaron, les prêtres, en offriront le sang et ils répandront le sang sur l'autel tout autour, qui est à l'entrée de la tente d'assignation.
\VS{6}Et on égorgera l'holocauste et le coupera en morceaux.
\VS{7}Les fils du prêtre Aaron mettront le feu sur l'autel, et disposeront le bois sur le feu.
\VS{8}Et les fils d'Aaron, les prêtres, poseront les morceaux, la tête et la graisse sur le bois qui sera au feu sur l'autel.
\VS{9}Mais il lavera avec de l'eau les entrailles et les jambes ; et le prêtre brûlera toutes ces choses sur l'autel. C'est un holocauste, un sacrifice consumé par le feu, d'une bonne odeur à Yahweh.
\VS{10}Si son offrande est un holocauste de menu bétail, d'entre les agneaux ou d'entre les chèvres, il offrira un mâle sans défaut.
\VS{11}Et on l'égorgera à côté de l'autel, vers le nord, devant Yahweh ; et les prêtres, fils d'Aaron, en répandront le sang sur l'autel tout autour.
\VS{12}Puis on le coupera en morceaux, avec sa tête et sa graisse ; et le prêtre les posera sur le bois qui sera au feu sur l'autel.
\VS{13}Mais il lavera avec de l'eau les entrailles et les jambes. Puis le prêtre offrira toutes ces choses, et les brûlera sur l'autel. C'est un holocauste, un sacrifice consumé par le feu, d'une agréable odeur à Yahweh\FTNT{Ez. 40:38.}.
\VS{14}Si son offrande à Yahweh est un holocauste d'oiseaux, il offrira son offrande de tourterelles, ou de jeunes pigeons.
\VS{15}Le prêtre l'apportera sur l'autel, lui ouvrira la tête avec l'ongle, la brûlera sur l'autel, et il en exprimera le sang contre un côté de l'autel.
\VS{16}Il ôtera son jabot avec ses plumes, et le jettera près de l'autel, vers l'orient, dans le lieu où seront les cendres.
\VS{17}Il le déchirera avec ses ailes, sans le séparer ; et le prêtre le brûlera sur l'autel, sur le bois qui sera au feu. C'est un holocauste, un sacrifice consumé par le feu, d'une agréable odeur à Yahweh.
\Chap{2}
\TextTitle{L'offrande de gâteau\FTNTT{Lé. 6:7-16.}}
\VerseOne{}Lorsque quelqu'un offrira l'offrande de gâteau\FTNT{L'offrande de farine ou de gâteau correspond aux perfections de la vie du Seigneur Jésus-Christ en tant qu'homme. Ce sacrifice ne comporte ni victime ni sang, mais seulement de la farine, de l'huile, de l'encens et du sel. Jésus, le grain de blé (Jn. 12:24), a été complètement broyé, pétri et oint d'huile, éprouvé par toutes sortes de douleurs. Sa vie sainte était pour le Père un parfum de bonne odeur. Son amour pour les âmes, sa dépendance totale au Père, sa persévérance, sa douceur, sa sagesse et sa bonté, n'ont pas varié malgré toutes les souffrances par lesquelles il est passé. Voilà quelques-uns des fruits admirables qui correspondent à l'offrande de gâteau saupoudrée d'encens. Le levain, image du péché (1 Co. 5:6-8), n'y entrait pas, ni le miel, symbole des affections humaines (Pr. 5:3). Quant au sel, il préserve de la corruption des aliments, il est comparé à la saveur des disciples de Christ (Mt. 5:13).} à Yahweh, son offrande sera de fine farine ; il versera de l'huile dessus, et mettra de l'encens.
\VS{2}Il l'apportera aux fils d'Aaron, les prêtres, et le prêtre prendra une pleine poignée de cette fine farine, et d'huile, avec tout l'encens, et il brûlera son souvenir\FTNT{En hébreu « azkarah », offrande de souvenir, la portion de nourriture offerte et qui est consumée.} sur l'autel. C'est une offrande d'une bonne odeur à Yahweh.
\VS{3}Ce qui restera du gâteau sera pour Aaron et ses fils ; c'est une chose très sainte parmi les offrandes consumées par le feu à Yahweh.
\VS{4}Et quand tu offriras une offrande de gâteaux cuits au four, ce sera de fine farine, des gâteaux sans levain, pétris avec de l'huile, et des galettes sans levain, ointes d'huile.
\VS{5}Si ton offrande est un gâteau cuit sur la plaque, elle sera de fine farine pétrie à l'huile, sans levain.
\VS{6}Tu la rompras en morceaux, et tu verseras de l'huile sur elle ; c'est une offrande de gâteau.
\VS{7}Si ton offrande est un gâteau cuit sur le gril, elle sera faite de fine farine avec de l'huile.
\VS{8}Puis tu apporteras à Yahweh l'offrande de gâteaux qui sera faite de ces choses, et on la présentera au prêtre, qui l'apportera sur l'autel.
\VS{9}Le prêtre lèvera de l'offrande de gâteaux, son souvenir, et le brûlera sur l'autel. C'est une offrande consumée par le feu de bonne odeur à Yahweh.
\VS{10}Ce qui restera de l'offrande de gâteau sera pour Aaron et ses fils ; c'est une chose très sainte parmi les offrandes consumées par le feu devant Yahweh.
\VS{11}Aucune offrande de gâteau que vous offrirez à Yahweh ne sera faite avec du levain ; car vous ne brûlerez point de levain ni de miel, parmi l'offrande consumée par le feu devant Yahweh.
\VS{12}Vous pourrez bien les offrir à Yahweh dans l'offrande des prémices, mais ils ne seront point mis sur l'autel comme offrande d'une bonne odeur.
\VS{13}Tu mettras du sel\FTNT{Voir No. 18:19 ; 2 Ch. 13:5. Le sel est un agent purificateur (2 R. 2:19-22). Le sel préserve de la corruption et conserve les aliments. Les chrétiens sont le sel de la terre (Mt. 5:13). Nos paroles doivent être assaisonnées de sel (Col. 4:6).} sur toutes tes offrandes de gâteaux, et tu ne laisseras point ton offrande de gâteau manquer de sel, signe de l'alliance de ton Dieu ; mais sur toutes tes offrandes, tu offriras du sel.
\VS{14}Si tu offres à Yahweh une offrande de gâteau des premiers fruits, tu offriras, pour l'offrande de gâteau des premiers fruits, des épis qui commencent à mûrir, rôtis au feu, les grains de quelques épis bien grenés, broyés entre les mains.
\VS{15}Puis tu mettras de l'huile sur le gâteau, et tu mettras aussi de l'encens dessus : C'est une offrande de gâteaux.
\VS{16}Et le prêtre brûlera son souvenir, pris de ses grains broyés, et de son huile avec tout l'encens. C'est une offrande consumée par le feu à Yahweh.
\Chap{3}
\TextTitle{Le sacrifice d'offrande de paix\FTNTT{Lé. 7:11-21.}}
\VerseOne{}Si son offrande est un sacrifice d'offrande de paix\FTNT{La plupart des traducteurs ont traduit par « sacrifice d'actions de grâces », or l'étymologie hébraïque du mot grâce est « shelem », ce qui signifie d'abord « paix ». Ce terme peut aussi vouloir dire « remerciement » ou « reconnaissance ». La racine de « shelem » est « shalam » : « être dans une alliance de paix », « être en paix ».Il est donc question ici d'une offrande de paix qui préfigure l'ensemble de l'œuvre de la croix accomplie par le Messie, et grâce à laquelle nous sommes réconciliés avec le Père (Col. 1:20 ; Ep. 2:14-17). Cette offrande préfigure aussi la Pâque incarnée par le Messie (1 Co. 5:7) ainsi que le repas du Seigneur. En effet, sur cette offrande, Dieu prenait pour lui la graisse et la queue entière (Lé. 3:3 ; Lé. 3:9-17), le prêtre prenait la poitrine et l'épaule droite (Lé. 7:31-34), et celui qui offrait l'animal pouvait consommer le reste avec d'autres personnes pures (Lé. 7:20). Ainsi, comme pour le repas du Seigneur, tous ceux qui étaient saints pouvaient participer au repas (1 Co. 11:27-34).}, et qu'il offre du gros bétail, soit mâle, soit femelle, il l'offrira sans défaut devant Yahweh.
\VS{2}Il posera sa main sur la tête de son offrande, et l'égorgera à l'entrée de la tente d'assignation, et les fils d'Aaron, les prêtres, répandront le sang sur l'autel tout autour.
\VS{3}Puis on offrira de cette offrande de paix, un sacrifice consumé par le feu à Yahweh, à savoir la graisse qui couvre les entrailles et toute la graisse qui est sur les entrailles ;
\VS{4}les deux rognons avec la graisse qui est dessus et qui est sur les flancs ; et on ôtera le grand lobe qui est sur le foie pour le mettre avec les rognons.
\VS{5}Les fils d'Aaron brûleront tout cela sur l'autel, sur l'holocauste, qui sera sur le bois mis au feu. C'est une offrande consumée par le feu d'agréable odeur à Yahweh\FTNT{Ex. 29:13-25.}.
\VS{6}Si son offrande pour le sacrifice d'offrande de paix à Yahweh est de menu bétail, soit mâle, soit femelle, il l'offrira sans défaut.
\VS{7}S'il offre un agneau pour son offrande, il l'offrira devant Yahweh.
\VS{8}Il posera sa main sur la tête de son offrande, et l'égorgera devant la tente d'assignation, et les fils d'Aaron répandront son sang sur l'autel tout autour.
\VS{9}De ce sacrifice d'offrande de paix, il offrira en offrande consumée par le feu à Yahweh, sa graisse et sa queue entière, séparée jusqu'à l'échine, avec la graisse qui couvre les entrailles et toute la graisse qui est sur les entrailles,
\VS{10}les deux rognons avec la graisse qui est dessus, sur les flancs, et il ôtera le grand lobe qui est sur le foie, jusqu'aux rognons.
\VS{11}Le prêtre brûlera tout cela sur l'autel. C'est un aliment d'offrande consumée par le feu à Yahweh\FTNT{No. 28:2.}.
\VS{12}Si son offrande est une chèvre, il l'offrira devant Yahweh.
\VS{13}Il posera sa main sur sa tête, et l'égorgera devant la tente d'assignation ; et les fils d'Aaron répandront son sang sur l'autel tout autour.
\VS{14}Puis il offrira son offrande en sacrifice consumé par le feu à Yahweh, la graisse qui couvre les entrailles et toute la graisse qui est sur les entrailles,
\VS{15}les deux rognons, et la graisse qui est dessus, sur les flancs, et il ôtera le grand lobe qui est sur le foie, jusqu'aux rognons.
\VS{16}Puis le prêtre brûlera toutes ces choses sur l'autel. C'est un aliment d'offrande consumée par le feu de bonne odeur. Toute graisse appartient à Yahweh.
\VS{17}C'est une loi perpétuelle pour vos descendants, dans toutes vos demeures : Vous ne mangerez ni graisse ni sang\FTNT{Ge. 9:4 ; 1 S. 14:33.}.
\Chap{4}
\TextTitle{Le sacrifice pour l'expiation\FTNTT{Lé. 6:17-23.}}
\VerseOne{}Yahweh parla encore à Moïse en disant :
\VS{2}Parle aux enfants d'Israël, et dis-leur : Quand une personne aura péché involontairement\FTNT{Avant la promulgation de la loi, certains hommes péchaient par ignorance (Ro. 5:13). Néanmoins, ces péchés étaient tout de même punis et nécessitaient un sacrifice (Lé. 4:13-14. No. 15:22-36 ; Job. 1). Sous la grâce, l'excuse du péché par ignorance ne peut être invoquée puisque nous sommes scellés du Saint-Esprit qui nous enseigne toutes choses (1 Jn. 2:20 et 27).} contre l'un des commandements de Yahweh, en commettant des choses qui ne doivent point se faire, et qu'il aura fait une de ces choses ;
\VS{3} si c'est le prêtre oint qui ait commis un péché, semblable à quelque faute du peuple, il offrira à Yahweh pour son péché qu'il aura fait, un jeune taureau sans défaut, pris du troupeau en sacrifice pour l'expiation.
\VS{4}Il amènera le taureau à l'entrée de la tente d'assignation, devant Yahweh, il posera sa main sur la tête du taureau, et l'égorgera devant Yahweh.
\VS{5}Et le prêtre oint prendra du sang du taureau, et l'apportera dans la tente d'assignation.
\VS{6}Le prêtre trempera son doigt dans le sang, et fera sept fois l'aspersion du sang devant Yahweh, en face du voile du lieu saint\FTNT{No. 19:4.}.
\VS{7}Le prêtre mettra aussi devant Yahweh du sang sur les cornes de l'autel des parfums odoriférants, qui est dans la tente d'assignation ; et il répandra tout le reste du sang du taureau au pied de l'autel de l'holocauste, qui est à l'entrée de la tente d'assignation.
\VS{8}Il enlèvera toute la graisse du taureau du sacrifice pour l'expiation, à savoir, la graisse qui couvre les entrailles, et toute la graisse qui est sur les entrailles,
\VS{9}et les deux rognons avec la graisse qui les entoure, qui couvre les flancs, et il ôtera le grand lobe qui est sur le foie, pour le mettre sur les rognons.
\VS{10}Comme on les enlève du taureau du sacrifice d'offrande de paix\FTNT{Voir commentaire en Lé. 3:1.}, et le prêtre brûlera toutes ces choses-là sur l'autel de l'holocauste.
\VS{11}Mais quant à la peau du taureau et toute sa chair, avec sa tête, ses jambes, ses entrailles, et ses excréments,
\VS{12}et même tout le taureau, il l'emportera hors du camp, dans un lieu pur, où l'on répand les cendres, et il le brûlera au feu sur du bois : Il sera brûlé au lieu où l'on répand les cendres.
\VS{13}Et si toute l'assemblée d'Israël a péché involontairement, et que la chose soit restée cachée aux yeux de l'assemblée, et qu'ils aient violé l'un des commandements de Yahweh, en commettant des choses qui ne doivent pas se faire, et s'en soit rendu coupable,
\VS{14}et que le péché qu'ils ont fait vienne en évidence, l'assemblée offrira en sacrifice pour l'expiation un jeune taureau pris du troupeau, et on l'amènera devant la tente d'assignation.
\VS{15}Les anciens de l'assemblée poseront leurs mains sur la tête du taureau devant Yahweh, et on égorgera le taureau devant Yahweh.
\VS{16}Et le prêtre oint, apportera du sang du taureau dans la tente d'assignation ;
\VS{17}ensuite le prêtre trempera son doigt dans le sang, et en fera aspersion devant Yahweh en face du voile, par sept fois.
\VS{18}Et il mettra du sang sur les cornes de l'autel, qui est devant Yahweh dans la tente d'assignation ; et il répandra tout le reste du sang au pied de l'autel de l'holocauste, qui est à l'entrée de la tente d'assignation.
\VS{19}Il enlèvera toute sa graisse et la brûlera sur l'autel.
\VS{20}Et il fera de ce taureau comme il l'a fait du taureau pour le sacrifice d'expiation. Le prêtre fera ainsi ; il fera propitiation pour eux, et il leur sera pardonné.
\VS{21}Puis il emportera le taureau hors du camp, et le brûlera comme il a brûlé le premier taureau. Car c'est le sacrifice pour l'expiation de l'assemblée.
\VS{22}Que si un chef a péché involontairement, en violant l'un des commandements de Yahweh son Dieu, ce qui ne doit point se faire, et s'en soit rendu coupable,
\VS{23}et qu'on vienne à connaître le péché qu'il a commis, il amènera pour sacrifice un jeune bouc, mâle, sans défaut ;
\VS{24}et il posera sa main sur la tête du bouc, et l'égorgera au lieu où l'on égorge l'holocauste devant Yahweh. C'est un sacrifice pour expiation.
\VS{25}Puis le prêtre prendra avec son doigt du sang de l'offrande pour l'expiation, et le mettra sur les cornes de l'autel de l'holocauste, et il répandra le reste de son sang au pied de l'autel de l'holocauste.
\VS{26}Et il brûlera toute sa graisse sur l'autel, comme la graisse du sacrifice d'offrande de paix. Ainsi le prêtre fera propitiation pour lui de son péché, et il lui sera pardonné.
\VS{27}Que si quelqu'un du peuple du pays a péché involontairement, en violant l'un des commandements de Yahweh, et en commettant des choses qui ne doivent point se faire, et s'en soit rendu coupable,
\VS{28}et qu'on vienne à connaître le péché qu'il a commis, il amènera pour offrande une jeune chèvre, femelle, sans défaut, pour le péché qu'il a commis.
\VS{29}Et il posera sa main sur la tête de l'offrande pour le péché, et égorgera l'offrande pour l'expiation au lieu où l'on égorge l'holocauste.
\VS{30}Puis le prêtre prendra du sang de la chèvre avec son doigt, et le mettra sur les cornes de l'autel de l'holocauste, et il répandra tout le reste de son sang au pied de l'autel.
\VS{31}Et il ôtera toute sa graisse, comme on ôte la graisse de dessus le sacrifice d'offrande de paix, et le prêtre la brûlera sur l'autel, en bonne odeur à Yahweh. Il fera propitiation pour lui, et il lui sera pardonné.
\VS{32}Que s'il amène un agneau comme offrande, pour le sacrifice d'expiation, il amènera une femelle sans défaut.
\VS{33}Et il posera sa main sur la tête de l'offrande d'expiation, et on l'égorgera en sacrifice pour l'expiation au lieu où l'on égorge l'holocauste.
\VS{34}Puis le prêtre prendra avec son doigt du sang de l'offrande pour l'expiation, et le mettra sur les cornes de l'autel de l'holocauste, et il répandra tout le reste de son sang au pied de l'autel.
\VS{35}Et il ôtera toute sa graisse, comme on ôte la graisse de l'agneau du sacrifice d'offrande de paix, et le prêtre la brûlera sur l'autel, par-dessus les sacrifices de Yahweh consumés par le feu, et il fera propitiation pour lui, pour son péché qu'il aura commis, et il lui sera pardonné.
\Chap{5}
\TextTitle{Le sacrifice de culpabilité\FTNTT{Lé. 7:1-7.}}
\VerseOne{}Et quand quelqu'un, étant témoin, après avoir entendu la parole du serment, aura péché en ne déclarant pas ce qu'il a vu ou ce qu'il sait, il portera son iniquité\FTNT{Pr. 29:24.}.
\VS{2}Et quand quelqu'un, à son insu, aura touché une chose souillée, soit le cadavre d'un animal impur, soit le cadavre d'une bête sauvage impure, soit le cadavre d'un reptile impur, il sera souillé et coupable\FTNT{Ag. 2:14 ; 2 Co. 6:17.}.
\VS{3}Ou quand il aura touché à l'impureté d'un homme, quelle que soit son impureté par laquelle il se rend impur, et que cela lui soit resté caché, quand il le sait, alors il est coupable.
\VS{4}Ou quand quelqu'un, parlant légèrement de ses lèvres, a juré de faire du mal ou du bien, selon tout ce que l'homme profère légèrement en jurant, et que cela lui soit resté caché, quand il le sait, alors il est coupable dans l'un de ces points-là.
\VS{5}Quand donc quelqu'un sera coupable sur l'un de ces points là, il confessera ce en quoi il aura péché.
\VS{6}Et il amènera son sacrifice de culpabilité à Yahweh pour le péché qu'il a commis, à savoir, une femelle du menu bétail, soit une brebis, soit une chèvre, pour l'offrande d'expiation. Et le prêtre fera pour lui propitiation de son péché.
\VS{7}Et s'il n'a pas le moyen de trouver une brebis ou une chèvre, il apportera en offrande pour le péché à Yahweh, pour sa culpabilité, deux tourterelles ou deux jeunes pigeons, l'un comme sacrifice pour l'expiation, l'autre pour l'holocauste\FTNT{Lu. 2:24.}.
\VS{8}Il les apportera au prêtre, qui offrira premièrement celui qui est pour l'offrande d'expiation. Il leur ouvrira la tête avec l'ongle, près du cou, sans la séparer ;
\VS{9}puis il fera l'aspersion du sang du sacrifice d'expiation sur un côté de l'autel, et ce qui restera du sang sera exprimé au pied de l'autel : C'est un sacrifice pour l'expiation.
\VS{10}Et il fera de l'autre un holocauste, selon l'ordonnance. Et le prêtre fera pour lui la propitiation pour son péché qu'il aura commis, et il lui sera pardonné.
\VS{11}Si celui qui aura péché n'a pas le moyen de trouver deux tourterelles ou deux jeunes pigeons, il apportera pour son offrande un dixième d'épha de fine farine en offrande pour le sacrifice d'expiation ; il ne mettra ni huile ni encens, car c'est un sacrifice d'expiation.
\VS{12}Il l'apportera au prêtre, et le prêtre qui en prendra une pleine poignée pour souvenir\FTNT{Lé. 2:2.}, la brûlera sur l'autel, comme offrande consumée par le feu à Yahweh : C'est un sacrifice d'expiation.
\VS{13}Ainsi le prêtre fera propitiation pour lui, pour le péché qu'il a commis dans l'une de ces choses, et il lui sera pardonné. Le reste sera pour le prêtre, comme étant une offrande de gâteau.
\VS{14}Yahweh parla aussi à Moïse, en disant :
\VS{15}Quand quelqu'un aura commis une transgression et péchera involontairement, en retenant des choses consacrées à Yahweh, il amènera en sacrifice de culpabilité à Yahweh, à savoir un bélier sans défaut, pris du troupeau, avec l'estimation que tu feras de la chose sainte, la faisant en sicles d'argent, selon le sicle du sanctuaire, à cause de son péché.
\VS{16}Il restituera donc ce en quoi il aura péché en retenant de la chose sainte et il y ajoutera un cinquième par dessus, et le donnera au prêtre ; et le prêtre fera propitiation pour lui, par le bélier du sacrifice de culpabilité, et il lui sera pardonné.
\VS{17}Lorsque quelqu'un aura péché, en violant, sans le savoir, l'un des commandements de Yahweh, des choses qu'on ne doit point faire, il sera coupable et portera son iniquité.
\VS{18} Il amènera donc en sacrifice de culpabilité au prêtre un bélier sans tâche, pris du troupeau, avec l'estimation que tu feras du péché involontaire ; et le prêtre fera propitiation pour lui du péché involontaire qu'il a commis et dont il ne se sera point aperçu ; et ainsi il lui sera pardonné.
\VS{19}C'est un sacrifice de culpabilité. Il s'est rendu coupable contre Yahweh.
\TextTitle{La restitution au jour du sacrifice de culpabilité\FTNTT{Lé. 7:1-7.}}
\VS{20}Yahweh parla aussi à Moïse, en disant :
\VS{21}Quand quelqu'un aura péché et aura commis une transgression contre Yahweh, en mentant à son prochain pour un dépôt, pour une chose qu'on aura mise entre ses mains, un vol, ou qu'il ait extorqué son prochain,
\VS{22}ou s'il a trouvé quelque chose perdue, et qu'il mente à ce sujet, ou s'il jure faussement concernant l'une des choses qu'un homme fait en péchant ;
\VS{23}quand il péchera et se rendra coupable, il rendra la chose qu'il a volée ou extorquée, ou le dépôt qui lui a été donné en garde, ou la chose perdue qu'il a trouvée,
\VS{24}ou tout ce dont il aura juré faussement. Il le restituera totalement, et il y ajoutera un cinquième ; il le donnera à celui à qui il appartenait, le jour de son sacrifice de culpabilité.
\VS{25}Et il amènera pour Yahweh, au prêtre le sacrifice de culpabilité, à savoir un bélier sans défaut, pris du troupeau, avec l'estimation que tu feras de la culpabilité.
\VS{26}Et le prêtre fera propitiation pour lui devant Yahweh, et il lui sera pardonné, quelle que soit la faute dont il se sera rendu coupable. 
\Chap{6}
\TextTitle{Loi de l'holocauste\FTNTT{Lé. 1:1-17.}}
\VerseOne{}Yahweh parla aussi à Moïse, en disant :
\VS{2}Ordonne à Aaron et à ses fils, et dis-leur : C'est ici la loi de l'holocauste. L'holocauste demeurera sur le foyer de l'autel toute la nuit jusqu'au matin, et le feu brûlera sur l'autel.
\VS{3}Et le prêtre revêtira sa tunique de lin, mettra ses caleçons de lin sur son corps, et il enlèvera la cendre de l'holocauste que le feu aura consumé sur l'autel, puis il la mettra près de l'autel.
\VS{4}Alors il ôtera ses vêtements et portera d'autres vêtements pour transporter les cendres hors du camp, dans un lieu pur.
\VS{5}Et quant au feu qui brûle sur l'autel, il continuera de brûler, on ne l'éteindra point ; le prêtre y brûlera du bois tous les matins, il préparera l'holocauste sur le bois, et y brûlera les graisses des offrandes de paix.
\VS{6}Le feu brûlera continuellement sur l'autel, on ne le laissera point s'éteindre.
\TextTitle{Loi de l'offrande de gâteau\FTNTT{Lé. 2:1-16.}}
\VS{7}Et c'est ici la loi de l'offrande de gâteau. Les fils d'Aaron l'offriront devant Yahweh sur l'autel\FTNT{No. 15:4.}.
\VS{8}Et on lèvera une poignée de la fine farine du gâteau et de son huile, avec tout l'encens qui est sur le gâteau, et on le brûlera sur l'autel, en bonne odeur, en mémorial à Yahweh.
\VS{9}Aaron et ses fils mangeront ce qui en restera ; ils le mangeront sans levain dans un lieu saint, ils le mangeront dans le parvis de la tente d'assignation\FTNT{Ex. 29:26-37.}.
\VS{10}On ne le cuira point avec du levain. Je leur ai donné cela pour leur portion d'entre mes offrandes consumées par le feu. C'est une chose très sainte, comme le sacrifice d'expiation et le sacrifice de culpabilité.
\VS{11}Tout mâle d'entre les fils d'Aaron en mangera. C'est une ordonnance perpétuelle pour vos descendants concernant les offrandes consumées par le feu à Yahweh : Quiconque les touchera sera sanctifié.
\VS{12}Yahweh parla aussi à Moïse, en disant :
\VS{13}C'est ici l'offrande d'Aaron et de ses fils, qu'ils offriront à Yahweh le jour où il sera oint : Un dixième d'épha de fine farine, comme offrande de gâteau perpétuelle, une moitié le matin et une moitié le soir.
\VS{14}Elle sera apprêtée sur une plaque avec de l'huile, tu l'apporteras mélangée, et tu offriras les morceaux cuits du gâteau en bonne odeur à Yahweh.
\VS{15}Et le prêtre, d'entre ses fils, qui sera oint à sa place, fera cela. C'est une ordonnance perpétuelle devant Yahweh : On le brûlera tout entier.
\VS{16}Tout le gâteau du prêtre sera entièrement consumé ; on n'en mangera pas.
\TextTitle{Loi de l'offrande pour le péché\FTNTT{Lé. 4:1-35.}}
\VS{17}Yahweh parla aussi à Moïse, en disant :
\VS{18}Parle à Aaron et à ses fils, et dis-leur : C'est ici la loi du sacrifice d'expiation. L'offrande pour l'expiation sera égorgée devant Yahweh, dans le même lieu où l'on égorge l'holocauste : C'est une chose très sainte.
\VS{19}Le prêtre qui offrira l'offrande pour l'expiation la mangera ; elle se mangera dans un lieu saint, dans le parvis de la tente d'assignation\FTNT{No. 18:10.}.
\VS{20}Quiconque touchera sa chair sera saint. Et s'il en jaillit du sang sur le vêtement, ce sur quoi il aura jailli sera lavé dans un lieu saint.
\VS{21}Et le vase de terre dans lequel on l'aura fait cuire sera brisé ; mais si on l'a fait cuire dans un vase d'airain, il sera nettoyé et lavé dans l'eau.
\VS{22}Tout mâle d'entre les prêtres en mangera ; car c'est une chose très sainte.
\VS{23}Aucune offrande pour le sacrifice d'expiation, dont on portera le sang dans la tente d'assignation pour faire la propitiation dans le sanctuaire, ne sera mangée, mais elle sera brûlée au feu\FTNT{Hé. 13:11.}.
\Chap{7}
\TextTitle{Loi du sacrifice de culpabilité\FTNTT{Lé. 5:1-26.}}
\VerseOne{}Or c'est ici la loi du sacrifice de culpabilité : C'est une chose très sainte.
\VS{2}Au même lieu où l'on égorgera l'holocauste, on égorgera le sacrifice de culpabilité. On en répandra le sang sur l'autel tout autour.
\VS{3}Puis on en offrira toute la graisse, avec la queue, et toute la graisse qui couvre les entrailles,
\VS{4}les deux rognons, la graisse qui est dessus sur les flancs, et le grand lobe qui est sur le foie, qu'on ôtera jusqu'aux rognons.
\VS{5}Le prêtre brûlera toutes ces choses sur l'autel comme offrande consumée par le feu à Yahweh : C'est un sacrifice pour la culpabilité.
\VS{6}Tout mâle d'entre les prêtres en mangera ; il sera mangé dans un lieu saint ; car c'est une chose très sainte.
\VS{7}Le sacrifice pour l'expiation sera semblable au sacrifice de culpabilité, il y aura une même loi pour les deux ; et la victime appartiendra au prêtre qui aura fait propitiation par elle.
\VS{8}Et le prêtre qui offrira l'holocauste de quelqu'un aura la peau de l'holocauste qu'il aura offert.
\VS{9}Et toute offrande de gâteau cuit au four, apprêtée sur le gril ou sur la plaque, appartiendra au prêtre qui l'offre.
\VS{10}Et toute offrande pétrie à l'huile, ou sèche, sera pour tous les fils d'Aaron, pour l'un comme pour l'autre.
\TextTitle{Loi du sacrifice d'offrande de paix\FTNTT{Lé. 3:1-17.}} 
\VS{11}Et c'est ici, la loi du sacrifice d'offrande de paix\FTNT{Voir commentaire en Lé. 3:1.} qu'on offrira à Yahweh.
\VS{12}Si quelqu'un l'offre pour un sacrifice de reconnaissance, il offrira avec le sacrifice de reconnaissance, des gâteaux sans levain pétris à l'huile, des galettes sans levain ointes d'huile, et des gâteaux de fine farine mêlés et pétris à l'huile.
\VS{13}En plus des gâteaux, il offrira pour son offrande du pain levé avec le sacrifice de reconnaissance de ses offrandes de paix.
\VS{14}Il présentera une part de chaque offrande, qu'il offrira comme offrande élevée à Yahweh ; elle sera pour le prêtre qui a répandu le sang du sacrifice d'offrande de paix.
\VS{15}Mais la chair du sacrifice de reconnaissance de ses offrandes de paix sera mangée le jour où elle sera offerte ; on n'en laissera rien jusqu'au matin.
\VS{16}Que si le sacrifice de son offrande est un vœu ou une offrande volontaire, son sacrifice sera mangé le jour où il l'aura offert ; ce qui en restera sera mangé le lendemain.
\VS{17}Mais ce qui restera de la chair du sacrifice sera brûlé au feu le troisième jour.
\VS{18}Que si on mange de la chair du sacrifice d'offrande de paix le troisième jour, celui qui l'aura offert ne sera point agréé, il ne lui sera point imputé, ce sera une chose infâme, et la personne qui en mangera portera son iniquité\FTNT{Ez. 4:14.}.
\VS{19}Et la chair de ce sacrifice qui a touché quelque chose d'impure ne sera point mangée, elle sera brûlée au feu. Mais quiconque sera pur, mangera de cette chair.
\VS{20}Car une personne qui mangera de la chair du sacrifice d'offrande de paix, laquelle appartient à Yahweh, et qui aura sur elle son impureté, cette personne-là sera retranchée de son peuple.
\VS{21}Si une personne touche quelque chose d'impure, soit une impureté d'homme, soit une bête impure, ou quelque autre chose impure, et qu'il mange de la chair du sacrifice d'offrande de paix qui appartient à Yahweh, cette personne-là sera retranchée d'entre son peuple.
\VS{22}Yahweh parla à Moïse, en disant :
\VS{23}Parle aux enfants d'Israël, et dis-leur : Vous ne mangerez aucune graisse de bœuf, ni d'agneau, ni de chèvre.
\VS{24}On pourra se servir pour un usage quelconque de la graisse d'une bête morte ou de la graisse d'une bête déchirée ; mais vous n'en mangerez point.
\VS{25}Car quiconque mangera de la graisse d'une bête que l'on offre comme offrande consumée par le feu à Yahweh, la personne qui en mangera, sera retranché de son peuple.
\VS{26}Vous ne mangerez point de sang, ni d'oiseaux, ni d'autres bêtes, dans aucune de vos demeures.
\VS{27}Toute personne, qui aura mangé de quelque sang que ce soit, sera retranchée de son peuple.
\VS{28}Yahweh parla à Moïse, en disant :
\VS{29}Parle aux enfants d'Israël, et dis-leur : Celui qui offrira son sacrifice d'offrande de paix à Yahweh, apportera son offrande à Yahweh, prise sur son sacrifice d'offrande de paix.
\VS{30}Il apportera de ses mains les offrandes consumées par le feu devant Yahweh. Il apportera la graisse avec la poitrine, la poitrine pour l'agiter d'un côté et de l'autre devant Yahweh.
\VS{31}Puis le prêtre brûlera la graisse sur l'autel, mais la poitrine sera pour Aaron et ses fils.
\VS{32}Vous donnerez aussi au prêtre pour offrande élevée, l'épaule droite de vos sacrifices d'offrande de paix\FTNT{No. 18:18.}.
\VS{33}Celui des fils d'Aaron qui offrira le sang et la graisse de l'offrande de paix, aura pour sa part l'épaule droite.
\VS{34}Car je prends sur les enfants d'Israël, la poitrine qu'on agite d'un côté et de l'autre, et l'épaule qu'on présente par élévation, de tous les sacrifices d'offrande de paix, et je les donne à Aaron le prêtre et à ses fils, par une ordonnance perpétuelle, de la part des fils d'Israël.
\VS{35}C'est là, le droit de l'onction d'Aaron et de l'onction de ses fils sur ces offrandes consumées par le feu devant Yahweh, depuis le jour où on les aura présentés pour exercer la sacrificature à Yahweh.
\VS{36}Et c'est ce que Yahweh ordonne aux enfants d'Israël de leur donner, depuis le jour où on les aura oints ; par une loi perpétuelle parmi leurs descendants\FTNT{Ex. 40:15.}.
\VS{37}Telle est donc la loi de l'holocauste, du gâteau, du sacrifice pour l'expiation, du sacrifice pour la culpabilité, de la consécration et du sacrifice d'offrande de paix.
\VS{38}Yahweh l'ordonna à Moïse sur la montagne de Sinaï, le jour où il ordonna aux enfants d'Israël d'offrir leurs offrandes à Yahweh dans le désert de Sinaï.
\Chap{8}
\TextTitle{Consécration d'Aaron et ses fils}
\VerseOne{}Yahweh parla aussi à Moïse en disant :
\VS{2}Prends Aaron et ses fils avec lui, les vêtements, l'huile d'onction, un jeune taureau pour le sacrifice d'expiation, deux béliers et une corbeille de pains sans levain\FTNT{Ex. 29:1-2 ; Ex. 30:25.} ;
\VS{3}et convoque toute l'assemblée à l'entrée de la tente d'assignation.
\VS{4}Et Moïse fit comme Yahweh lui avait ordonné ; et l'assemblée se rassembla à l'entrée de la tente d'assignation.
\VS{5}Moïse dit à l'assemblée : Voici ce que Yahweh a ordonné de faire.
\TextTitle{La purification avec l'eau} 
\VS{6}Et Moïse fit approcher Aaron et ses fils, et les lava avec de l'eau.
\TextTitle{Les vêtements d'Aaron} 
\VS{7}Et il mit sur Aaron la tunique, il le ceignit de la ceinture, le revêtit de la robe, mit sur lui l'éphod, et le ceignit avec la ceinture de l'éphod dont il le lia.
\VS{8}Puis il mit sur lui le pectoral, après avoir mis au pectoral l'urim et le thummim.
\VS{9}Il lui mit aussi la tiare sur la tête, et il mit sur le devant de la tiare la lame d'or, la couronne de sainteté, comme Yahweh l'avait ordonné à Moïse\FTNT{Ex. 28.}.
\TextTitle{L'onction d'huile} 
\VS{10}Puis Moïse prit l'huile d'onction, et oignit le tabernacle et toutes les choses qui y étaient, et les sanctifia.
\VS{11}Et il en fit l'aspersion sur l'autel par sept fois, et il oignit l'autel, tous ses ustensiles, et la cuve avec sa base, pour les sanctifier.
\VS{12}Il versa aussi de l'huile d'onction sur la tête d'Aaron, et l'oignit pour le sanctifier\FTNT{Ps. 133:2.}.
\TextTitle{Les vêtements des fils d'Aaron}
\VS{13}Puis Moïse fit approcher les fils d'Aaron, les revêtit des tuniques, les ceignit des ceintures et leur attacha des turbans, comme Yahweh l'avait ordonné à Moïse.
\TextTitle{Les offrandes et les sacrifices}
\VS{14}Alors il fit approcher le jeune taureau pour le sacrifice d'expiation, et Aaron et ses fils posèrent leurs mains sur la tête du taureau pour le sacrifice d'expiation.
\VS{15}Et Moïse l'égorgea, prit de son sang, et en mit avec son doigt sur les cornes de l'autel tout autour, et purifia l'autel ; et il répandit le reste du sang au pied de l'autel, ainsi il le sanctifia pour faire la propitiation sur lui.
\VS{16}Puis il prit toute la graisse qui était sur les entrailles, le grand lobe du foie, les deux rognons avec leur graisse, et Moïse les brûla sur l'autel.
\VS{17}Mais il brûla au feu, hors du camp, le jeune taureau avec sa peau, sa chair, et ses excréments, comme Yahweh l'avait ordonné à Moïse.
\VS{18}Il fit aussi approcher le bélier de l'holocauste, et Aaron et ses fils posèrent leurs mains sur la tête du bélier.
\VS{19}Et Moïse l'égorgea et répandit le sang sur l'autel tout autour.
\VS{20}Puis il coupa le bélier en morceaux, et Moïse brûla la tête, les morceaux, et la graisse.
\VS{21}Et il lava dans l'eau les entrailles et les jambes, et brûla tout le bélier sur l'autel : Ce fut un holocauste d'une agréable odeur, c'était une offrande consumée par le feu à Yahweh, comme Yahweh l'avait ordonné à Moïse.
\VS{22}Il fit aussi approcher l'autre bélier, le bélier de consécration, et Aaron et ses fils posèrent les mains sur la tête du bélier.
\VS{23}Et Moïse l'égorgea, prit de son sang, et le mit sur le lobe de l'oreille droite d'Aaron, et sur le pouce de sa main droite et sur le gros orteil de son pied droit.
\VS{24}Il fit aussi approcher les fils d'Aaron, et mit du même sang sur le lobe de leur oreille droite, et sur le pouce de leur main droite, et sur le gros orteil de leur pied droit, et Moïse répandit le reste du sang sur l'autel tout autour.
\VS{25}Après, il prit la graisse, la queue, toute la graisse qui est sur les entrailles, et le grand lobe du foie, et les deux rognons avec leur graisse, et l'épaule droite.
\VS{26}Il prit aussi de la corbeille des pains sans levain, qui étaient devant Yahweh, un gâteau sans levain, et un gâteau de pain fait à l'huile et une galette, et il les mit sur les graisses, et sur l'épaule droite.
\VS{27}Puis il mit toutes ces choses sur les paumes des mains d'Aaron et sur les paumes des mains de ses fils, et les agita d'un côté et de l'autre devant Yahweh.
\VS{28}Puis Moïse les prit de leurs mains et les brûla sur l'autel, sur l'holocauste : Ce fut l'offrande de consécration de bonne odeur, c'est une offrande consumée par le feu devant Yahweh.
\VS{29}Moïse prit aussi la poitrine du bélier de consécration, et l'agita d'un côté et de l'autre devant Yahweh : Ce fut la part de Moïse, comme Yahweh l'avait ordonné à Moïse.
\TextTitle{L'aspersion d'huile et de sang}
\VS{30}Moïse prit de l'huile d'onction et du sang qui était sur l'autel, et il en fit l'aspersion sur Aaron et sur ses vêtements, sur ses fils et sur les vêtements de ses fils ; ainsi il sanctifia Aaron et ses vêtements, les fils d'Aaron et les vêtements de ses fils.
\TextTitle{La nourriture de consécration\FTNTT{Ex. 29:26 ; Lé. 7:31-34 ; 8:29.}}
\VS{31}Après cela, Moïse dit à Aaron et à ses fils : Faites cuire la chair à l'entrée de la tente d'assignation, et vous la mangerez là, avec le pain qui est dans la corbeille de consécration, comme je l'ai ordonné, en disant : Aaron et ses fils la mangeront.
\VS{32}Mais vous brûlerez au feu ce qui restera de la chair et du pain.
\TextTitle{Les prêtres mis à part}
\VS{33}Et vous ne sortirez point pendant sept jours, de l'entrée de la tente d'assignation, jusqu'à ce que vos jours de consécration soient accomplis ; car on emploiera sept jours à vous consacrer.
\VS{34}Yahweh a ordonné de faire en ces autres jours comme on a fait en celui-ci, pour faire la propitiation en votre faveur.
\VS{35}Vous resterez donc pendant sept jours à l'entrée de la tente d'assignation, jour et nuit, et vous observerez ce que Yahweh vous a ordonné d'observer, afin que vous ne mouriez pas ; car il m'a été ainsi ordonné.
\VS{36}Ainsi Aaron et ses fils firent toutes les choses que Yahweh avait ordonnées par Moïse.
\Chap{9}
\TextTitle{Aaron et ses fils commencent leur service dans le tabernacle}
\VerseOne{}Et il arriva au huitième jour, que Moïse appela Aaron et ses fils, et les anciens d'Israël.
\VS{2}Et il dit à Aaron : Prends un jeune taureau du troupeau pour l'offrande d'expiation, et un bélier pour l'holocauste, tous deux sans défaut, et offre-les devant Yahweh.
\VS{3}Et tu parleras aux enfants d'Israël, en disant : Prenez un bouc pour l'offrande d'expiation, un jeune taureau et un agneau, tous deux d'un an et sans défaut, pour l'holocauste ;
\VS{4}un bœuf et un bélier pour l'offrande de paix\FTNT{Voir commentaire en Lé. 3:1.}, pour les sacrifier devant Yahweh ; et un gâteau pétri à l'huile. Car aujourd'hui Yahweh vous apparaîtra.
\VS{5}Ils prirent donc les choses que Moïse avait ordonné et les amenèrent devant la tente d'assignation, et toute l'assemblée s'approcha, et se tint devant Yahweh.
\VS{6}Et Moïse dit : Faites ce que Yahweh vous a ordonné, et la gloire de Yahweh vous apparaîtra.
\VS{7}Moïse dit à Aaron : Approche-toi de l'autel, fais ton sacrifice pour l'expiation et ton holocauste, et fais propitiation pour toi et pour le peuple ; présente l'offrande pour le peuple, et fais propitiation pour eux, comme Yahweh l'a ordonné\FTNT{Hé. 7:26-27.}.
\VS{8}Alors Aaron s'approcha de l'autel et égorgea le veau de son sacrifice d'expiation.
\VS{9}Et les fils d'Aaron lui présentèrent le sang, et il trempa son doigt dans le sang et le mit sur les cornes de l'autel ; puis il répandit le reste du sang au pied de l'autel.
\VS{10}Mais il brûla sur l'autel la graisse, et les rognons, et le grand lobe du foie de l'offrande pour le péché, comme Yahweh l'avait ordonné à Moïse.
\VS{11}Et il brûla au feu la chair et la peau hors du camp.
\VS{12}Il égorgea aussi l'holocauste. Les fils d'Aaron lui présentèrent le sang, lequel il répandit sur l'autel tout autour.
\VS{13}Puis ils lui présentèrent l'holocauste coupé en morceaux, avec la tête, et il les brûla sur l'autel.
\VS{14}Et il lava les entrailles et les jambes, qu'il brûla sur l'holocauste, sur l'autel.
\VS{15}Il offrit l'offrande du peuple. Il prit le bouc pour le sacrifice d'expiation du peuple, il l'égorgea et l'offrit pour le péché, comme la première offrande.
\VS{16}Il l'offrit en holocauste, faisant selon l'ordonnance.
\VS{17}Ensuite, il offrit l'offrande du gâteau, et il en remplit la paume de sa main, et la brûla sur l'autel, outre l'holocauste du matin.
\VS{18}Il égorgea aussi le bœuf et le bélier pour le sacrifice d'offrande de paix, qui était pour le peuple. Les fils d'Aaron lui présentèrent le sang, lequel il répandit sur l'autel tout autour.
\VS{19}Ils présentèrent la graisse du bœuf et du bélier, la queue, ce qui couvre les entrailles, les rognons, et le grand lobe du foie ;
\VS{20}ils mirent les graisses sur les poitrines, et il brûla les graisses sur l'autel.
\VS{21}Et Aaron agita d'un côté et de l'autre devant Yahweh les poitrines et l'épaule droite, comme Yahweh l'avait ordonné à Moïse.
\VS{22}Aaron éleva aussi ses mains vers le peuple, et le bénit. Puis il descendit, après avoir offert le sacrifice pour l'expiation, l'holocauste et l'offrande de paix.
\VS{23}Moïse donc et Aaron entrèrent dans la tente d'assignation, puis ils sortirent et ils bénirent le peuple. Et la gloire de Yahweh apparut à tout le peuple.
\VS{24}Car le feu sortit de devant Yahweh, et consuma sur l'autel l'holocauste et les graisses. Tout le peuple le vit et ils poussèrent des cris de joie et tombèrent sur leur face\FTNT{1 R. 18:38 ; 2 Ch. 7:1.}.
\Chap{10}
\TextTitle{Un feu étranger présenté à Yahweh}
\VerseOne{}Or les fils d'Aaron, Nadab et Abihu, prirent chacun leur encensoir, mirent du feu, et ils posèrent dessus du parfum ; ils offrirent devant Yahweh un feu étranger\FTNT{Ce passage nous avertit du danger auquel s'exposent ceux qui apportent un feu étranger dans le temple. Les feux étrangers sont les fausses doctrines, le péché, les conceptions cartésiennes, pernicieuses, mercantiles, destinés à remplacer la Parole de Dieu et à conduire le chrétien dans les ténèbres.}, ce qu'il ne leur avait point été ordonné.
\VS{2}Et le feu sortit de devant Yahweh, et les dévora ; ils moururent devant Yahweh\FTNT{No. 3:4.}.
\VS{3}Moïse dit à Aaron : C'est ce dont Yahweh avait parlé, en disant : Je serai sanctifié par ceux qui s'approchent de moi, et je serai glorifié en présence de tout le peuple. Et Aaron se tut.
\VS{4}Et Moïse appela Mischaël et Eltsaphan, les fils d'Uziel, oncle d'Aaron, et leur dit : Approchez-vous, emportez vos frères de devant le sanctuaire, hors du camp.
\VS{5}Alors ils s'approchèrent et les emportèrent avec leurs tuniques hors du camp, comme Moïse l'avait dit.
\TextTitle{Instructions données par Moïse} 
\VS{6}Puis Moïse dit à Aaron, à Eléazar et à Ithamar, ses fils : Ne découvrez point vos têtes, et ne déchirez point vos vêtements, de peur que vous ne mouriez, et que Yahweh ne se mette en colère contre toute l'assemblée. Mais que vos frères, toute la maison d'Israël, pleurent à cause de l'embrasement que Yahweh a allumé\FTNT{Ez. 24:17.}.
\VS{7}Et ne sortez point de l'entrée de la tente d'assignation, de peur que vous ne mouriez, car l'huile de l'onction de Yahweh est sur vous. Et ils firent selon la parole de Moïse.
\VS{8}Et Yahweh parla à Aaron, en disant :
\VS{9}Vous ne boirez point de vin, ni de boisson forte, ni toi ni tes fils avec toi, quand vous entrerez dans la tente d'assignation, de peur que vous ne mouriez ; c'est une ordonnance perpétuelle pour vos descendants\FTNT{No. 6:3 ; Jg. 13:7.},
\VS{10}afin que vous puissiez discerner entre ce qui est saint et ce qui est profane, entre ce qui est impur et ce qui est pur,
\VS{11}afin que vous enseigniez aux enfants d'Israël toutes les ordonnances que Yahweh leur a prononcées par Moïse.
\VS{12}Puis Moïse parla à Aaron, à Eléazar et à Ithamar, ses fils qui lui restaient : Prenez l'offrande de gâteau, leur dit-il, ce qui reste des offrandes de Yahweh consumées par le feu, et mangez-la avec des pains sans levain auprès de l'autel, car c'est une chose très sainte.
\VS{13}Vous la mangerez dans un lieu saint, parce que c'est la portion qui est assignée à toi et à tes fils sur les offrandes consumées par le feu à Yahweh ; car il m'a été ainsi ordonné.
\VS{14}Vous mangerez aussi la poitrine offerte par agitation et l'épaule présentée par élévation dans un lieu pur, toi, tes fils et tes filles avec toi ; car ces choses-là t'ont été données, dans les sacrifices d'offrande de paix\FTNT{Voir commentaire en Lé. 3:1.} des enfants d'Israël, comme ton droit et le droit de tes fils.
\VS{15}Ils apporteront l'épaule présentée par élévation et la poitrine offerte par agitation, avec les offrandes consumées par le feu, qui sont les graisses, pour les agiter en offrande çà et là devant Yahweh : Cela t'appartiendra, et à tes fils avec toi, par une ordonnance perpétuelle, comme Yahweh l'a ordonné.
\VS{16}Or Moïse cherchait soigneusement le bouc de l'offrande pour l'expiation mais voici, il avait été brûlé. Et Moïse se mit en grande colère contre Eléazar et Ithamar, les fils d'Aaron qui lui restaient, et leur dit :
\VS{17}Pourquoi n'avez-vous point mangé l'offrande pour l'expiation dans un lieu saint ? Car c'est une chose très sainte ; vu qu'elle vous a été donnée pour porter l'iniquité de l'assemblée, afin de faire propitiation pour eux devant Yahweh.
\VS{18}Voici, son sang n'a point été porté dans l'intérieur du sanctuaire ; ne manquez donc plus à la manger dans le lieu saint, comme je l'avais ordonné.
\VS{19}Alors Aaron répondit à Moise : Voici, ils ont offert aujourd'hui leur offrande pour l'expiation et leur holocauste devant Yahweh, et ces choses-ci me sont arrivées. Si j'avais mangé aujourd'hui l'offrande pour le péché, cela aurait-il plu à Yahweh ?
\VS{20}Et Moïse l'entendit, et cela fut bon à ses yeux.
\Chap{11}
\TextTitle{Lois de purification : les bêtes pures et impures}
\VerseOne{}Et Yahweh parla à Moïse et à Aaron, et leur dit :
\VS{2}Parlez aux enfants d'Israël, et dites-leur : Ce sont ici les bêtes dont vous mangerez d'entre toutes les bêtes qui sont sur la terre\FTNT{De. 14:4 ; Ac. 10:11-14.}.
\VS{3}Vous mangerez d'entre les bêtes de tous ceux qui ont le sabot fendu, qui ont le pied fourchu, et qui ruminent.
\VS{4}Mais vous ne mangerez point de celles qui ruminent uniquement, ou qui ont uniquement le sabot fendu : Comme le chameau, car il rumine mais il n'a point le sabot fendu : Il vous sera impur.
\VS{5}Et le lapin, car il rumine mais il n'a point le sabot fendu : Il vous sera impur.
\VS{6}Le lièvre, car il rumine mais il n'a point le sabot fendu : Il vous sera impur.
\VS{7}Le porc, car il a bien le sabot fendu et le pied fourchu, mais il ne rumine pas : Il vous sera impur.
\VS{8}Vous ne mangerez point de leur chair, même vous ne toucherez point leur cadavre : Ils vous seront impurs.
\VS{9}Vous mangerez de ceci d'entre tout ce qui est dans les eaux. Vous mangerez de tout ce qui a des nageoires et des écailles dans les eaux, soit dans la mer, soit dans les fleuves.
\VS{10}Mais vous ne mangerez rien de ce qui n'a point de nageoires et d'écailles, soit dans la mer, soit dans les fleuves, tant des reptiles des eaux, que de toute chose vivante qui est dans les eaux, cela vous sera en abomination.
\VS{11}Elles vous seront donc en abomination, vous ne mangerez point de leur chair, et vous tiendrez pour une chose abominable leur cadavre.
\VS{12}Tout ce donc qui vit dans les eaux et qui n'a point de nageoires et d'écailles, vous sera en abomination.
\VS{13}Et d'entre les oiseaux vous tiendrez ceux-ci pour abominables, on n'en mangera point, ils vous seront en abomination : L'aigle, l'orfraie, l'aigle de mer ;
\VS{14}le vautour, et le milan, selon leur espèce ;
\VS{15}tout corbeau, selon son espèce ;
\VS{16}l'autruche, le hibou, la mouette, et l'épervier selon leur espèce ;
\VS{17}le chat-huant, le plongeon, la chouette ;
\VS{18}le cygne, le cormoran, le pélican ;
\VS{19}la cigogne, le héron selon leur espèce, la huppe et la chauve-souris,
\VS{20}et tout reptile volant qui marche sur quatre pattes vous sera en abomination.
\VS{21}Mais, vous pourrez manger de toute chose rampante qui vole et qui va sur quatre pattes qui ont des jambes au-dessus de leurs pieds, pour sauter avec celles-ci sur la terre.
\VS{22}Ce sont donc ici ceux dont vous mangerez : La sauterelle selon son espèce, le solam\FTNT{« Solam », « hargol » et « hagab » sont diverses espèces de sauterelles.} selon son espèce, le hargol, selon son espèce et le hagab, selon son espèce.
\VS{23}Mais tout autre reptile volant qui a quatre pattes vous sera en abomination.
\VS{24}Vous serez donc impurs par ces bêtes ; quiconque touchera leur cadavre sera impur jusqu'au soir,
\VS{25}et quiconque aussi portera leur cadavre lavera ses vêtements et sera impur jusqu'au soir.
\VS{26}Toute bête qui a le sabot fendu, et qui n'a point le pied fourchu et ne rumine point, vous sera impur : Quiconque les touchera sera impur.
\VS{27}Tout ce qui marche sur ses pattes, entre tous les animaux qui marchent à quatre pieds, vous sera impur : Quiconque touchera leur cadavre sera impur jusqu'au soir,
\VS{28}et celui qui portera leur cadavre lavera ses vêtements et sera impur jusqu'au soir. Ils vous seront impurs.
\VS{29}Ceci aussi vous sera impur entre les reptiles, qui rampent sur la terre : La taupe, la souris et la tortue, selon leur espèce ;
\VS{30}le hérisson, la grenouille, le lézard, la limace et le caméléon.
\VS{31}Ces choses vous seront impures entre les reptiles : Quiconque les touchera mortes sera impur jusqu'au soir.
\VS{32}Aussi, tout ce sur quoi il en tombera quelque chose quand elles seront mortes sera impur, soit ustensile de bois, soit vêtement, soit peau, ou sac, quelque objet que ce soit dont on se sert pour faire quelque chose ; il sera mis dans l'eau, et sera impur jusqu'au soir ; puis il sera pur.
\VS{33}Mais s'il en tombe quelque chose dans quelque vase de terre que ce soit, tout ce qui est dedans sera impur, et vous casserez le vase.
\VS{34}Et tout aliment qu'on mange, sur lequel il y aura eu de cette eau, sera impur ; tout breuvage qu'on boit dans quelque vase que ce soit, en sera impur.
\VS{35}Et s'il tombe quelque chose de leur cadavre sur quoi que ce soit, cela sera impur ; le four et le foyer seront détruits : Ils seront impurs, et ils vous seront impurs.
\VS{36}Toutefois, la source, le puits ou tel autre amas d'eaux resteront purs ; mais celui donc qui touchera leur cadavre sera impur.
\VS{37}Et s'il est tombé de leur cadavre sur quelque semence qui se sème, elle restera pure.
\VS{38}Mais si on avait mis de l'eau sur la semence, et que quelque chose de leur cadavre tombe sur elle, elle vous sera impure.
\VS{39}Et si une des bêtes qui vous servent pour nourriture meurt, celui qui en touchera le cadavre sera impur jusqu'au soir ;
\VS{40}celui qui mangera de son cadavre lavera ses vêtements et sera impur jusqu'au soir, et celui aussi qui portera le cadavre de cette bête, lavera ses vêtements et sera impur jusqu'au soir.
\VS{41}Tout reptile donc qui rampe sur la terre vous sera en abomination ; et on n'en mangera point\FTNT{cp. Ge. 3:14.}.
\VS{42}Vous ne mangerez point de tout ce qui rampe sur la poitrine, ni de tout ce qui marche sur les quatre pieds, ni de tout ce qui a plusieurs pieds entre tous les reptiles qui se traînent sur la terre ; car ils seront en abomination.
\VS{43}Ne rendez point vos personnes abominables par aucun reptile qui se traîne ; ne vous rendez point impurs par eux, ne vous souillez point par eux.
\VS{44}Car je suis Yahweh, votre Dieu ; vous vous sanctifierez donc et vous serez saints, car je suis saint\FTNT{1 Pi. 1:16.} ! Ainsi, vous ne rendrez point vos personnes impures par aucun reptile qui se traîne sur la terre.
\VS{45}Car je suis Yahweh, qui vous ai fait monter du pays d'Egypte, afin que je sois votre Dieu, et que vous soyez saints ; car je suis saint !
\VS{46}Telle est la loi touchant les animaux, les oiseaux, tout être vivant, qui se meut dans les eaux, et toute être vivant, qui se rampe sur la terre,
\VS{47}afin de discerner entre la chose impure et la chose pure, entre les animaux qu'on peut manger et les animaux dont on ne doit point manger.
\Chap{12}
\TextTitle{Lois de purification : Le flux de sang\FTNTT{Ps. 51:7.}}
\VerseOne{}Yahweh parla aussi à Moïse, en disant :
\VS{2}Parle aux enfants d'Israël, et dis-leur : Si la femme après avoir conçu, enfante un mâle, elle sera impure pendant sept jours ; elle sera impure comme au temps de son indisposition menstruelle.
\VS{3}Et au huitième jour, on circoncira la chair du prépuce de l'enfant\FTNT{Les parents de Jésus ont observé cette loi (Lu. 2:21-24). Jn. 7:22.}.
\VS{4}Et elle demeurera trente-trois jours à se purifier de son sang ; elle ne touchera aucune chose sainte, et ne viendra point au sanctuaire, jusqu'à ce que les jours de sa purification soient accomplis.
\VS{5}Si elle enfante une fille, elle sera impure deux semaines, comme au temps de son indisposition menstruelle, et elle restera soixante-six jours à se purifier de son sang.
\VS{6}Après que le temps de sa purification sera accompli, soit pour un fils ou pour une fille, elle présentera au prêtre un agneau d'un an en holocauste, et un jeune pigeon ou une tourterelle en sacrifice d'expiation, à l'entrée de la tente d'assignation\FTNT{No. 6:10.}.
\VS{7}Et le prêtre offrira ces choses devant Yahweh, et fera propitiation pour elle ; et elle sera purifiée du flux de son sang. Telle est la loi pour celle qui enfante un fils ou une fille.
\VS{8}Et que, si elle n'a pas le moyen de trouver un agneau, alors elle prendra deux tourterelles ou deux jeunes pigeons, l'un pour l'holocauste, et l'autre pour le sacrifice d'expiation. Le prêtre fera propitiation pour elle, et elle sera pure.
\Chap{13}
\TextTitle{Lois de purification : La lèpre}
\VerseOne{}Yahweh parla aussi à Moïse et à Aaron, en disant :
\VS{2}L'homme qui aura sur la peau de son corps une tumeur, une dartre, ou une tache blanche, et que cela paraîtra sur la peau de son corps comme une plaie de lèpre, on l'amènera à Aaron, le prêtre, ou à l'un de ses fils prêtres.
\VS{3}Et le prêtre regardera la plaie qui est sur la peau du corps. Si le poil de la plaie est devenu blanc, et si la plaie, à la voir, est plus profonde que la peau du corps, c'est une plaie de lèpre : Le prêtre donc le regardera et le jugera impur.
\VS{4}Mais si la tache est blanche sur la peau du corps, et qu'à la voir, elle n'est point plus profonde que la peau, et si son poil n'est pas devenu blanc, le prêtre fera enfermer pendant sept jours celui qui a la plaie.
\VS{5}Et le prêtre la regardera le septième jour. Si à ses yeux la plaie s'est arrêtée, et qu'elle ne s'est point étendue sur la peau, le prêtre le fera renfermer pendant sept autres jours.
\VS{6}Et le prêtre la regardera une seconde fois le septième jour suivant. Si la plaie est devenue pâle, et qu'elle ne s'est point étendue sur la peau, le prêtre le jugera pur : C'est de la dartre ; il lavera ses vêtements, et sera pur.
\VS{7}Mais si la dartre s'est étendue sur la peau, après avoir été vu par le prêtre pour être jugé pur, il se fera examiner pour la seconde fois par le prêtre.
\VS{8}Le prêtre le regardera encore. S'il aperçoit que la dartre s'est étendue sur la peau, le prêtre le jugera impur : C'est de la lèpre.
\VS{9}Quand il y aura une plaie de lèpre sur un homme, on l'amènera au prêtre.
\VS{10}Le prêtre le regardera. Et s'il aperçoit qu'il y a une tumeur blanche sur la peau, que le poil est devenu blanc, et qu'il y a une trace de chair vive dans la tumeur,
\VS{11}c'est une lèpre invétérée dans la peau du corps : Le prêtre le jugera impur ; il ne le fera point enfermer, car il est jugé impur.
\VS{12}Si la lèpre fait une éruption sur la peau, et qu'elle couvre toute la peau de celui qui a la plaie, depuis la tête de cet homme jusqu'à ses pieds, partout où pourra voir le prêtre, le prêtre le regardera,
\VS{13}et si le prêtre voit que la lèpre couvre tout le corps de cet homme, alors il jugera pur celui qui a la plaie : La plaie est devenue toute blanche, il est pur.
\VS{14}Mais le jour où l'on apercevra de la chair vive, il sera impur ;
\VS{15}alors le prêtre regardera la chair vive, et le jugera impur : La chair vive est impure, c'est de la lèpre.
\VS{16}Si la chair vive se change et devient blanche, alors il viendra vers le prêtre ;
\VS{17}et le prêtre le regardera, et s'il aperçoit que la plaie est devenue blanche, le prêtre jugera pur celui qui a la plaie : Il est pur.
\VS{18}Si le corps a eu sur la peau un ulcère qui soit guéri,
\VS{19}et qu'à l'endroit où était l'ulcère il y ait une tumeur blanche, ou une tache blanche rougeâtre, il sera regardé par le prêtre.
\VS{20}Le prêtre donc la regardera. Et s'il aperçoit, qu'à la voir, elle paraît plus enfoncée que la peau, et que son poil est devenu blanc, alors le prêtre le jugera impur : C'est une plaie de lèpre qui a fait éruption dans l'ulcère.
\VS{21}Si le prêtre la regardant, voit que le poil n'est point blanc, et qu'elle n'est point plus enfoncée que la peau, mais qu'elle est devenue pâle, le prêtre le fera enfermer pendant sept jours.
\VS{22}Si elle s'est étendue sur la peau en quelque sorte que ce soit, le prêtre le jugera impur : C'est une plaie.
\VS{23}Mais si la tache est restée à la même place et ne s'est pas étendue, c'est une cicatrice d'ulcère : Ainsi le prêtre le jugera pur.
\VS{24}Si le corps a sur la peau une brûlure par le feu, et que la chair vive de la partie brûlée soit une tache blanche rougeâtre ou blanc seulement, le prêtre la regardera,
\VS{25}et si le poil est devenu blanc dans la tache, et qu'à la voir, elle est plus profonde que la peau, c'est de la lèpre, elle a fait éruption dans la brûlure ; le prêtre donc le jugera impur : C'est une plaie de lèpre.
\VS{26}Mais si le prêtre la regardant aperçoit qu'il n'y a point de poil blanc dans la tache, et qu'elle n'est point plus basse que la peau, qu'elle est devenue pâle, le prêtre le fera enfermer pendant sept jours.
\VS{27}Puis le prêtre la regardera le septième jour. Si la tache s'est étendue sur la peau, le prêtre le jugera impur : C'est une plaie de lèpre.
\VS{28}Si la tache est restée à la même place, ne s'est pas étendue, et est devenue pâle, c'est la tumeur de la brûlure ; et le prêtre le jugera pur ; c'est la cicatrice de la brûlure.
\VS{29}Si l'homme ou la femme a une plaie à la tête, ou l'homme à la barbe,
\VS{30}le prêtre regardera la plaie, et si à la voir, elle est plus profonde que la peau, et qu'il y ait en elle du poil jaunâtre et fin, le prêtre le jugera impur : C'est de la teigne, c'est une lèpre de la tête ou de la barbe.
\VS{31}Si le prêtre regardant la plaie de la teigne, voit qu'elle n'est point plus profonde que la peau, et n'a en elle aucun poil noir, le prêtre fera enfermer pendant sept jours celui qui a la plaie de la teigne.
\VS{32}Et le septième jour le prêtre regardera la plaie. Si la teigne ne s'est point étendue, qu'elle n'a aucun poil jaunâtre, et, qu'à voir la teigne, elle n'est pas plus profonde que la peau,
\VS{33}celui qui a la plaie de la teigne se rasera, mais il ne se rasera point à l'endroit de la teigne, et le prêtre fera enfermer pendant sept autres jours celui qui a la teigne.
\VS{34}Puis le prêtre regardera la teigne au septième jour. Si la teigne ne s'est point étendue sur la peau et, qu'à la voir, elle n'est point plus profonde que la peau, le prêtre le jugera pur, et cet homme lavera ses vêtements, et il sera pur.
\VS{35}Mais si la teigne s'est étendue sur la peau, après sa purification, le prêtre la regardera,
\VS{36}et si la teigne s'est étendue sur la peau, le prêtre ne cherchera point de poil jaunâtre : Il est impur.
\VS{37}Mais si la teigne s'est arrêtée, et qu'il y ait poussé du poil noir, la teigne est guérie : Il est pur, et le prêtre le jugera pur.
\VS{38}Si l'homme ou la femme ont sur la peau de leur corps des taches, des taches qui sont blanches,
\VS{39}le prêtre les regardera. Si sur la peau de leur corps il y a des taches d'un blanc pâle, c'est une tache blanche qui a fait éruption sur la peau : Il est donc pur.
\VS{40}Si l'homme a la tête dépouillée de cheveux, c'est un chauve : Il est pur.
\VS{41}Et si sa tête est dépouillée de cheveux du côté de son visage, c'est un front chauve : Il est pur.
\VS{42}Et si dans la partie chauve de devant ou de derrière, il y a une plaie d'un blanc rougeâtre, c'est une lèpre qui a fait éruption dans sa partie chauve de derrière ou de devant.
\VS{43}Et le prêtre le regardera. S'il aperçoit que la tumeur de la plaie est d'un blanc rougeâtre dans sa partie chauve de derrière ou de devant, semblable à la lèpre de la peau du corps,
\VS{44}l'homme est lépreux, il est impur : Le prêtre ne manquera pas de le juger impur ; sa plaie est à la tête.
\VS{45}Or le lépreux en qui sera la plaie aura ses vêtements déchirés, et sa tête nue ; et il se couvrira sur la lèvre de dessus et il criera : Impur ! Impur !
\VS{46}Pendant tout le temps qu'il aura cette plaie, il sera jugé impur : Il est impur. Il demeurera seul ; sa demeure sera hors du camp\FTNT{2 R. 7:3 ; La. 4:15 ; Lu. 17:12-13.}.
\VS{47}Et si le vêtement est infecté de la plaie de la lèpre, soit sur un vêtement de laine, soit sur un vêtement de lin,
\VS{48}à la chaîne ou à la trame du lin, ou de laine, sur la peau ou sur quelque ouvrage de peau,
\VS{49}et si cette plaie est verdâtre ou rougeâtre sur le vêtement ou sur la peau, à la chaîne ou à la trame, ou sur un objet quelconque de peau, ce sera une plaie de lèpre, et elle sera montrée au prêtre.
\VS{50}Et le prêtre regardera la plaie, et fera enfermer pendant sept jours celui qui a la plaie.
\VS{51}Et au septième jour, il regardera la plaie. Si la plaie s'est étendue sur le vêtement, à la chaîne ou à la trame, sur la peau ou sur quelque ouvrage de peau, la plaie est une lèpre invétérée : La chose est impure.
\VS{52}Il brûlera le vêtement, la chaîne ou la trame de laine ou de lin, et toutes les choses de peau, qui auront cette plaie, car c'est une lèpre rongeuse : Cela sera brûlé au feu.
\VS{53}Mais si le prêtre regarde, et que la plaie ne s'est point étendue sur le vêtement, sur la chaîne ou sur la trame, ou sur quelque objet de peau,
\VS{54}le prêtre ordonnera qu'on lave la chose où est la plaie, et il le fera enfermer pendant sept autres jours.
\VS{55}Si le prêtre, après qu'on aura fait laver la plaie, la regarde, et s'il aperçoit que la plaie n'a point changé sa couleur, et qu'elle ne s'est point étendue, c'est une chose impure : Tu la brûleras au feu ; c'est une partie de l'endroit ou de l'envers qui a été rongée.
\VS{56}Si le prêtre regarde, et aperçoit que la plaie est devenue pâle, après qu'on l'ait fait laver, il la déchirera du vêtement ou de la peau, de la chaîne ou de la trame.
\VS{57}Si elle paraît encore sur le vêtement, à la chaîne ou à la trame, ou sur quelque chose de peau, c'est une lèpre qui a fait éruption : Vous brûlerez au feu la chose où est la plaie.
\VS{58}Mais si tu as lavé le vêtement, la chaîne ou la trame, ou quelque chose de peau, et que la plaie s'en est allée, il sera lavé une seconde fois, puis il sera pur.
\VS{59}Telle est la loi sur la plaie de la lèpre sur un vêtement de laine ou de lin, la chaîne ou la trame, ou quelque chose de peau, pour la juger pure ou impure.
\Chap{14}
\TextTitle{Loi du lépreux pour le jour de sa purification}
\VerseOne{}Yahweh parla aussi à Moïse, en disant :
\VS{2}C'est ici la loi du lépreux pour le jour de sa purification. Il sera amené au prêtre\FTNT{Mt. 8:2-4 ; Mc. 1:42-44 ; Lu. 5:12-14.}.
\VS{3}Le prêtre sortira hors du camp et le regardera. Si la plaie de la lèpre du lépreux est guérie,
\VS{4}le prêtre ordonnera qu'on prenne pour celui qui doit être purifié, deux oiseaux vivants et purs, avec du bois de cèdre, du cramoisi et de l'hysope\FTNT{Ex. 12:22.}.
\VS{5}Et le prêtre ordonnera qu'on égorge l'un des oiseaux sur un vase de terre, sur de l'eau vive.
\VS{6}Puis il prendra l'oiseau vivant, le bois de cèdre, le cramoisi et l'hysope ; et il trempera toutes ces choses avec l'oiseau vivant, dans le sang de l'autre oiseau qui aura été égorgé sur de l'eau vive.
\VS{7}Il en fera sept fois l'aspersion sur celui qui doit être purifié de la lèpre. Il le déclarera pur, et il laissera aller par les champs, l'oiseau vivant.
\VS{8}Et celui qui doit être purifié lavera ses vêtements, rasera tout son poil, et se lavera dans l'eau ; et il sera pur. Ensuite il entrera dans le camp, mais il demeurera sept jours hors de sa tente.
\VS{9}Au septième jour, il rasera tout son poil, sa tête, sa barbe, les sourcils de ses yeux, tout son poil ; il rasera tout son poil ; puis il lavera ses vêtements et son corps, et il sera pur.
\VS{10}Et au huitième jour, il prendra deux agneaux sans défaut, une brebis d'un an sans défaut, et trois dixièmes de fine farine en offrande de gâteau, pétrie à l'huile, et un log d'huile.
\VS{11}Le prêtre qui fait la purification présentera celui qui doit être purifié et ces choses-là devant Yahweh, à l'entrée de la tente d'assignation.
\VS{12}Puis le prêtre prendra l'un des agneaux et l'offrira en sacrifice pour la culpabilité avec un log d'huile ; il agitera ces choses devant Yahweh, en offrande agitée.
\VS{13}Et il égorgera l'agneau au lieu où l'on égorge l'offrande pour l'expiation et l'holocauste, dans le lieu saint ; car le sacrifice pour la culpabilité appartient au prêtre, comme le sacrifice pour l'expiation ; c'est une chose très sainte.
\VS{14}Le prêtre prendra du sang de l'offrande pour la culpabilité ; il le mettra sur le lobe de l'oreille droite de celui qui doit être purifié, sur le pouce de sa main droite et sur le gros orteil de son pied droit.
\VS{15}Puis le prêtre prendra du log d'huile et en versera dans la paume de sa main gauche.
\VS{16}Et le prêtre trempera le doigt de sa main droite dans l'huile qui est dans sa paume gauche, et fera l'aspersion de l'huile avec son doigt sept fois devant Yahweh.
\VS{17}Et du reste de l'huile qui sera dans sa paume, le prêtre en mettra sur le lobe de l'oreille droite de celui qui doit être purifié, sur le pouce de sa main droite et sur le gros orteil de son pied droit, sur le sang pris de l'offrande pour la culpabilité.
\VS{18}Mais ce qui restera de l'huile sur la paume du prêtre, il le mettra sur la tête de celui qui doit être purifié ; et ainsi le prêtre fera propitiation pour lui devant Yahweh.
\VS{19}Ensuite le prêtre offrira le sacrifice pour l'expiation et fera propitiation pour celui qui doit être purifié de sa souillure. Puis il égorgera l'holocauste.
\VS{20}Le prêtre offrira l'holocauste et le gâteau sur l'autel, et fera propitiation pour celui qui doit être purifié, et il sera pur.
\VS{21}Mais s'il est pauvre et s'il n'a pas le moyen de fournir ces choses, il prendra un agneau en offrande agitée pour la culpabilité, afin de faire propitiation pour lui. Et un dixième de fine farine pétrie à l'huile pour le gâteau, avec un log d'huile.
\VS{22}Et deux tourterelles ou deux jeunes pigeons, selon ce qu'il pourra fournir, dont l'un sera pour le péché et l'autre pour l'holocauste.
\VS{23}Et le huitième jour de sa purification, il les apportera au prêtre, à l'entrée de la tente d'assignation, devant Yahweh.
\VS{24}Et le prêtre recevra l'agneau du sacrifice pour la culpabilité et le log d'huile, et les agitera devant Yahweh en offrande agitée.
\VS{25}Et il égorgera l'agneau du sacrifice pour la culpabilité. Puis le prêtre prendra du sang de l'offrande pour la culpabilité, il le mettra sur le lobe de l'oreille droite de celui qui doit être purifié, sur le pouce de sa main droite et sur le gros orteil de son pied droit.
\VS{26}Puis le prêtre versera de l'huile dans la paume de sa main gauche.
\VS{27}Et avec le doigt de sa main droite, il fera l'aspersion de l'huile qui est dans sa main gauche sept fois devant Yahweh.
\VS{28}Il mettra de cette huile qui est dans sa paume, sur le lobe de l'oreille droite de celui qui doit être purifié et sur le pouce de sa main droite et sur le gros orteil de son pied droit, sur le lieu du sang pris de l'offrande pour la culpabilité.
\VS{29}Après il mettra le reste de l'huile qui est dans sa paume sur la tête de celui qui doit être purifié, afin de faire propitiation pour lui devant Yahweh.
\VS{30}Puis il sacrifiera l'une des tourterelles ou l'un des jeunes pigeons, selon ce qu'il aura pu fournir.
\VS{31}De ce donc qu'il aura pu fournir, l'un sera pour le sacrifice d'expiation et l'autre pour l'holocauste, avec le gâteau ; ainsi le prêtre fera propitiation devant Yahweh pour celui qui doit être purifié.
\VS{32}Telle est la loi de celui qui a une plaie de lèpre, et dont les ressources sont insuffisantes à sa purification.
\TextTitle{Lois de purification d'une maison lépreuse}
\VS{33}Puis Yahweh parla à Moïse et à Aaron, en disant :
\VS{34}Quand vous serez entrés dans le pays de Canaan, que je vous donne en possession, si j'envoie une plaie de lèpre sur une maison du pays que vous posséderez,
\VS{35}celui à qui la maison appartiendra viendra et le fera savoir au prêtre, en disant : Il me semble que j'aperçois comme une plaie dans ma maison.
\VS{36}Alors le prêtre ordonnera qu'on vide la maison avant qu'il y entre pour regarder la plaie, afin que rien de ce qui est dans la maison ne soit impur, puis le prêtre entrera pour voir la maison.
\VS{37}Et il regardera la plaie. Si la plaie qui est sur les murs de la maison a des creux verdâtres ou rougeâtres, qui soient, à les voir, plus enfoncés que le mur ;
\VS{38}le prêtre sortira de la maison, à l'entrée, et fera fermer la maison pendant sept jours.
\VS{39}Au septième jour, le prêtre retournera et la regardera. Si la plaie s'est étendue sur les murs de la maison,
\VS{40}alors il ordonnera de retirer les pierres sur lesquelles est la plaie, et de les jeter hors de la ville, dans un lieu impur.
\VS{41}Il fera aussi racler l'enduit de la maison à l'intérieur, tout autour ; et l'enduit qu'on aura raclé, on le jettera hors de la ville, dans un lieu impur.
\VS{42}Puis on prendra d'autres pierres, et on les mettra à la place des premières pierres ; et on prendra d'autres mortiers pour recrépir la maison.
\VS{43}Mais si la plaie revient et fait éruption dans la maison, après avoir retiré les pierres, après avoir raclé et recrépi la maison,
\VS{44}le prêtre y entrera et la regardera. Si la plaie s'est étendue dans la maison, c'est une lèpre invétérée dans la maison : Elle est impure.
\VS{45}On démolira la maison, ses pierres, son bois, et tout le mortier de la maison ; et on les transportera hors de la ville, dans un lieu impur.
\VS{46}Si quelqu'un est entré dans la maison pendant tout le temps que le prêtre l'avait faite fermer, il sera impur jusqu'au soir.
\VS{47}Celui qui dormira dans cette maison lavera ses vêtements. Celui aussi qui mangera dans cette maison lavera ses vêtements.
\VS{48}Mais quand le prêtre y sera entré, et qu'il aura aperçu que la plaie ne s'est point étendue dans cette maison, après l'avoir recrépie, il jugera la maison pure, car sa plaie est guérie.
\VS{49}Alors il prendra pour purifier la maison deux oiseaux, du bois de cèdre, du cramoisi et de l'hysope.
\VS{50}Il égorgera l'un des oiseaux sur un vase de terre, sur de l'eau vive.
\VS{51}Il prendra le bois de cèdre, l'hysope, le cramoisi et l'oiseau vivant ; il trempera le tout dans le sang de l'oiseau qu'on aura égorgé et dans l'eau vive, puis il fera sept fois l'aspersion sur la maison.
\VS{52}Il purifiera la maison avec le sang de l'oiseau, avec l'eau vive, avec l'oiseau vivant, le bois de cèdre, l'hysope et le cramoisi.
\VS{53}Puis il laissera aller hors de la ville par les champs l'oiseau vivant. C'est ainsi qu'il fera propitiation pour la maison, et elle sera pure.
\VS{54}Telle est la loi pour toute plaie de lèpre et de teigne,
\VS{55}de lèpre de vêtement et de maison,
\VS{56}de tumeur, de dartre, et de tache ;
\VS{57}pour enseigner quand une chose est impure et quand elle est pure. Telle est la loi sur la lèpre.
\Chap{15}
\TextTitle{Lois de purification : Gonorrhée et flux menstruel\FTNTT{Jn. 13:3-10 ; Ep. 5:25-27 ; 1 Jn. 1:9.}}
\VerseOne{}Yahweh parla aussi à Moïse et à Aaron, en disant :
\VS{2}Parlez aux enfants d'Israël et dites-leur : Tout homme qui a une gonorrhée\FTNT{Gonorrhée : infection des organes génito-urinaires.} sera impur à cause de son flux.
\VS{3}Et telle sera l'impureté de son flux : Quand sa chair laissera aller son flux, ou que sa chair retiendra son flux, c'est son impureté.
\VS{4}Tout lit sur lequel se couchera celui qui est atteint d'un flux sera impur ; et toute chose sur laquelle il se sera assis sera impure.
\VS{5}L'homme aussi qui touchera son lit lavera ses vêtements et se lavera avec de l'eau ; et il sera impur jusqu'au soir.
\VS{6}Et celui qui s'assiéra sur quelque chose sur laquelle celui qui a ce flux s'est assis, lavera ses vêtements et se lavera dans l'eau, et il sera impur jusqu'au soir.
\VS{7}Et celui qui touchera la chair de celui qui a ce flux lavera ses vêtements et se lavera dans l'eau, et il sera impur jusqu'au soir.
\VS{8}Si celui qui a ce flux crache sur celui qui est pur, celui qui était pur lavera ses vêtements et se lavera dans l'eau, et il sera impur jusqu'au soir.
\VS{9}Toute monture que celui qui a ce flux aura montée sera impure.
\VS{10}Quiconque touchera quelque chose qui aura été sous lui sera impur jusqu'au soir ; et quiconque portera une telle chose lavera ses vêtements, et se lavera dans l'eau ; il sera impur jusqu'au soir.
\VS{11}Quiconque aura été touché par celui qui a ce flux, sans qu'il ait lavé ses mains dans l'eau, lavera ses vêtements et il se lavera dans l'eau, et il sera impur jusqu'au soir.
\VS{12}Et le vase de terre que celui qui a ce flux aura touché sera cassé, mais tout vase de bois sera lavé dans l'eau.
\VS{13}Or quand celui qui a ce flux sera purifié de son flux, il comptera sept jours pour sa purification ; il lavera ses vêtements et sa chair avec de l'eau vive, et ainsi il sera pur.
\VS{14}Au huitième jour, il prendra pour lui deux tourterelles ou deux jeunes pigeons, et il viendra devant Yahweh à l'entrée de la tente d'assignation, et les donnera au prêtre.
\VS{15}Et le prêtre les sacrifiera, l'un en sacrifice pour l'expiation et l'autre en holocauste ; ainsi le prêtre fera propitiation pour lui devant Yahweh à cause de son flux.
\VS{16}L'homme aussi duquel sera sortie de la semence lavera dans l'eau tout son corps, et il sera impur jusqu'au soir.
\VS{17}Et tout vêtement et toute peau sur lequel il y aura de la semence seront lavés dans l'eau, et seront impurs jusqu'au soir.
\VS{18}Même la femme qui couchera avec un tel homme se lavera dans l'eau avec son mari, et ils seront impurs jusqu'au soir.
\VS{19}Et quand la femme aura un flux, un flux de sang en sa chair, elle sera séparée sept jours. Quiconque la touchera sera impur jusqu'au soir\FTNT{Mt. 9:18-22 ; Mc. 5:21-34 ; Lu. 8:41-48.}.
\VS{20}Toute chose sur laquelle elle aura couché durant sa séparation sera impure, toute chose aussi sur laquelle elle aura été assise sera impure.
\VS{21}Quiconque aussi touchera le lit de cette femme lavera ses vêtements et se lavera dans l'eau, et il sera impur jusqu'au soir.
\VS{22}Et quiconque touchera quelque chose sur laquelle elle se sera assise lavera ses vêtements et se lavera dans l'eau, et il sera impur jusqu'au soir.
\VS{23}Même si la chose que quelqu'un aura touchée était sur le lit ou sur quelque chose sur laquelle elle était assise, quand quelqu'un aura touché cette chose-là, il sera impur jusqu'au soir.
\VS{24}Et si un homme a couché avec elle et que son impureté soit sur lui, il sera impur sept jours, et toute couche sur laquelle il dormira sera impure.
\VS{25}La femme qui aura un flux de sang pendant plusieurs jours, hors de l'époque de ses menstruations, ou dont le flux durera plus longtemps que l'époque de ses menstruations, sera impure tout le temps du flux de son impureté, comme au temps de sa séparation.
\VS{26}Toute couche sur laquelle elle couchera tous les jours de son flux lui sera comme la couche de sa séparation, et toute chose sur laquelle elle s'assiéra sera impure comme pour l'impureté de sa séparation.
\VS{27}Et quiconque aura touché ces choses-là sera impur ; il lavera ses vêtements et se lavera dans l'eau, et il sera impur jusqu'au soir.
\VS{28}Mais si elle est purifiée de son flux, elle comptera sept jours, et après elle sera pure.
\VS{29}Au huitième jour, elle prendra deux tourterelles ou deux jeunes pigeons, et les apportera au prêtre à l'entrée de la tente d'assignation.
\VS{30}Et le prêtre en sacrifiera l'un en sacrifice pour l'expiation et l'autre en holocauste ; ainsi le prêtre fera propitiation pour elle devant Yahweh, à cause du flux de son impureté.
\VS{31}Ainsi, vous séparerez les enfants d'Israël de leurs impuretés, et ils ne mourront point à cause de leurs impuretés, en rendant impur mon tabernacle, qui est au milieu d'eux.
\VS{32}Telle est la loi pour celui qui a une gonorrhée ou de celui duquel sort la semence qui le rend impur.
\VS{33}Telle est aussi la loi pour celle qui a son indisposition menstruelle ou de toute personne qui découle et qui a son flux, soit mâle, soit femelle, et de celui qui couche avec celle qui est impure.
\Chap{16}
\TextTitle{Expiation pour le prêtre, sa maison et le peuple.\FTNTT{Hé. 9:1-14.}}
\VerseOne{}Or Yahweh parla à Moïse après la mort des deux fils d'Aaron, qui moururent lorsqu'ils s'étaient approchés de la présence de Yahweh.
\VS{2}Yahweh donc dit à Moïse : Parle à Aaron, ton frère, et dis-lui qu'il n'entre point en tout temps dans le sanctuaire, au-dedans du voile, devant le propitiatoire qui est sur l'arche, afin qu'il ne meure point ; car j'apparaîtrai dans une nuée sur le propitiatoire.
\VS{3}Aaron entrera dans le sanctuaire de cette manière, après avoir offert un jeune taureau du troupeau pour le péché, et un bélier pour l'holocauste.
\VS{4}Il se revêtira de la sainte tunique de lin, et portera les caleçons de lin sur son corps ; il se ceindra de la ceinture de lin\FTNT{La ceinture de vérité (Ep. 6:14).}, et se couvrira la tête de la tiare\FTNT{La tiare, le casque du salut (Ep. 6:17).} de lin, qui sont les saints vêtements, et il s'en vêtira après avoir lavé son corps avec de l'eau\FTNT{Le lavement préfigure ici la régénération (Tit. 3:5).}.
\VS{5}Et il prendra de l'assemblée des enfants d'Israël deux jeunes boucs en offrande pour le péché et un bélier pour l'holocauste.
\VS{6}Puis Aaron offrira son veau en sacrifice pour l'expiation, et fera propitiation tant pour lui que pour sa maison.
\TextTitle{Les deux boucs expiatoires\FTNTT{2 Co. 5:21.}}
\VS{7}Et il prendra les deux boucs, et les présentera devant Yahweh, à l'entrée de la tente d'assignation.
\VS{8}Puis Aaron jettera le sort sur les deux boucs, un sort pour Yahweh et un sort pour le bouc qui doit être Azazel.
\VS{9}Et Aaron offrira le bouc sur lequel le sort sera échu pour Yahweh, et l'offrira en sacrifice pour l'expiation.
\VS{10}Mais le bouc sur lequel le sort sera tombé pour être Azazel, sera présenté vivant devant Yahweh pour faire propitiation par lui, et on l'enverra dans le désert pour être Azazel.
\VS{11}Aaron donc, présentera le veau en sacrifice pour l'expiation, et fera propitiation pour lui et pour sa maison. Et il égorgera, dis-je, son veau qui est le sacrifice pour l'expiation.
\VS{12}Puis il prendra un encensoir plein de charbons ardents, de dessus l'autel devant Yahweh, et deux poignées de parfum odoriférant en poudre ; et il les apportera au-dedans du voile ;
\VS{13}et il mettra le parfum sur le feu devant Yahweh, afin que la nuée du parfum couvre le propitiatoire qui est sur le témoignage, ainsi il ne mourra point.
\VS{14}Il prendra aussi du sang du veau, et il en fera l'aspersion avec son doigt au-devant du propitiatoire vers l'orient ; il fera l'aspersion de ce sang-là sept fois avec son doigt devant le propitiatoire.
\VS{15}Il égorgera aussi le bouc du peuple, qui est l'offrande pour l'expiation, et il apportera son sang au-dedans du voile. Il fera de son sang comme il a fait du sang du veau, en faisant l'aspersion sur le propitiatoire et sur le devant du propitiatoire.
\VS{16}Et il fera propitiation pour le sanctuaire, le purifiant des impuretés des enfants d'Israël, et de leurs transgressions, selon tous leurs péchés. Il fera la même chose pour la tente d'assignation, qui demeure avec eux au milieu de leurs impuretés.
\VS{17}Et personne ne sera dans la tente d'assignation quand le prêtre y entrera pour faire propitiation dans le sanctuaire, jusqu'à ce qu'il en sorte, lorsqu'il fera propitiation pour lui et pour sa maison, et pour toute l'assemblée d'Israël.
\VS{18}Puis il sortira vers l'autel qui est devant Yahweh, et fera propitiation pour lui ; il prendra du sang du veau et du sang du bouc, il le mettra sur les cornes de l'autel tout autour.
\VS{19}Et il fera par sept fois l'aspersion du sang avec son doigt sur l'autel, et le purifiera et le sanctifiera des impuretés des enfants d'Israël.
\VS{20}Et quand il achèvera de faire propitiation pour le sanctuaire, pour la tente d'assignation et pour l'autel, alors il offrira le bouc vivant.
\VS{21}Et Aaron posera ses deux mains sur la tête du bouc vivant, et il confessera sur lui toutes les iniquités des enfants d'Israël et toutes leurs transgressions, selon tous leurs péchés ; et il les mettra sur la tête du bouc, et l'enverra au désert par un homme prêt pour cela.
\VS{22}Et le bouc portera sur lui toutes leurs iniquités dans une terre inhabitable, puis cet homme laissera aller le bouc par le désert.
\VS{23}Et Aaron reviendra dans la tente d'assignation ; il quittera les vêtements de lin dont il s'était vêtu quand il était entré dans le sanctuaire, et les posera là.
\VS{24}Il lavera aussi son corps avec de l'eau dans le lieu saint, et se revêtira de ses vêtements. Puis il sortira, il offrira son holocauste et l'holocauste du peuple, et fera propitiation pour lui et pour le peuple.
\VS{25}Il brûlera aussi sur l'autel la graisse de l'offrande pour le péché.
\VS{26}Et celui qui aura conduit le bouc pour être Azazel lavera ses vêtements et son corps avec de l'eau ; après cela, il rentrera dans le camp.
\VS{27}Mais on tirera hors du camp le veau et le bouc qui auront été offerts en sacrifice pour l'expiation, et desquels le sang aura été porté dans le sanctuaire pour y faire propitiation, et on brûlera au feu leurs peaux, leur chair et leurs excréments\FTNT{Hé. 13:11.}.
\VS{28}Et celui qui les aura brûlés lavera ses vêtements et son corps avec de l'eau ; après cela, il rentrera dans le camp.
\VS{29}Et ceci sera pour vous une ordonnance perpétuelle : Le dixième jour du septième mois, vous affligerez vos âmes, et vous ne ferez aucune œuvre, tant celui qui est du pays que l'étranger qui fait son séjour parmi vous\FTNT{La fête des expiations (ou yom kippour) avait lieu une fois par an, le dixième jour du septième mois (Ex. 30:10 ; Lé. 16:29). A cette occasion, le grand prêtre jetait le sort sur deux boucs : un sort pour Yahweh et un sort pour Azazel (Lé. 16:8-10). Le bouc pour Yahweh était sacrifié, il préfigurait la mort expiatoire de Christ. Le bouc émissaire, pour Azazel, n'avait lui-même rien fait de mal, mais il était choisi par Dieu pour porter le péché du peuple afin qu'il soit dégagé de toute accusation. Ce que l'on faisait de ce bouc, préfigurait l'œuvre de Jésus-Christ. Il symbolisait le Seigneur qui s'est chargé de nos péchés pour les emporter loin de nous (Es. 53 ; Ps. 103:12 ; Hé. 10:17 ; Hé. 13:12-14). Christ est mort et ressuscité hors du camp et c'est là qu'il nous appelle à le rejoindre : hors du monde et des systèmes religieux (Hé. 13:10-14).}.
\VS{30}Car en ce jour-là le prêtre fera propitiation pour vous, afin de vous purifier : Ainsi vous serez purifiés de tous vos péchés devant Yahweh.
\VS{31}Ce sera pour vous donc un sabbat, un jour de repos, et vous affligerez vos âmes. C'est une ordonnance perpétuelle.
\VS{32}Et le prêtre qu'on aura oint, et qu'on aura consacré pour exercer la sacrificature à la place de son père, fera propitiation, s'étant revêtu des vêtements de lin, qui sont les saints vêtements.
\VS{33}Et il fera propitiation pour le saint sanctuaire et il fera propitiation pour la tente d'assignation et pour l'autel, et pour les prêtres et pour tout le peuple de l'assemblée.
\VS{34}Ceci donc sera pour vous une ordonnance perpétuelle, afin de faire propitiation pour les enfants d'Israël de tous leurs péchés une fois par an. On fit comme Yahweh l'avait ordonné à Moïse.
\Chap{17}
\TextTitle{Les sacrifices apportés à l'entrée de la tente d'assignation}
\VerseOne{}Yahweh parla aussi à Moïse, en disant :
\VS{2}Parle à Aaron et à ses fils, et à tous les enfants d'Israël, et dis-leur : C'est ici ce que Yahweh a ordonné, en disant :
\VS{3}Quiconque de la maison d'Israël aura égorgé un bœuf, un agneau ou une chèvre dans le camp, ou qui l'aura égorgé hors du camp\FTNT{De. 12:6.},
\VS{4}et ne l'aura point amené à l'entrée de la tente d'assignation, pour en faire une offrande à Yahweh, devant le tabernacle de Yahweh, le sang sera imputé à cet homme-là ; il a répandu du sang, c'est pourquoi cet homme-là sera retranché du milieu de son peuple.
\VS{5}C'est afin que les enfants d'Israël amènent leurs sacrifices, qu'ils sacrifient dans les champs, qu'ils les amènent à Yahweh, à l'entrée de la tente d'assignation, vers le prêtre, et qu'ils les sacrifient en sacrifices d'offrande de paix\FTNT{Voir commentaire en Lé. 3:1.} à Yahweh ;
\VS{6}et que le prêtre en répande le sang sur l'autel de Yahweh, à l'entrée de la tente d'assignation, et en brûle la graisse en bonne odeur à Yahweh.
\VS{7}Et qu'ils n'offrent plus leurs sacrifices aux démons, avec lesquels ils se sont prostitués. Ceci leur sera une ordonnance perpétuelle pour eux et leurs descendants\FTNT{De. 32:17 ; Ps. 106:37.}.
\VS{8}Tu leur diras donc : Si un homme de la maison d'Israël, ou des étrangers qui font leur séjour parmi eux, aura offert un holocauste ou un sacrifice,
\VS{9}et qui ne l'aura point amené à l'entrée de la tente d'assignation, pour le sacrifier à Yahweh, cet homme-là sera retranché d'entre ses peuples.
\TextTitle{Importance du sang}
\VS{10}Quiconque de la maison d'Israël ou des étrangers qui font leur séjour parmi eux, aura mangé de quelque sang que ce soit, je mettrai ma face contre cette personne qui aura mangé du sang, et je la retrancherai du milieu de son peuple\FTNT{Ge. 9:4 ; De. 12:16-23 ; 1 S. 14:33.}.
\VS{11}Car l'âme de la chair est dans le sang. C'est pourquoi je vous ai ordonné qu'il soit mis sur l'autel, afin de faire propitiation pour vos âmes, car c'est le sang qui fera propitiation pour l'âme.
\VS{12}C'est pourquoi j'ai dit aux enfants d'Israël : Que personne d'entre vous ne mange du sang, que même l'étranger qui fait son séjour parmi vous ne mange point de sang.
\VS{13}Et quiconque des enfants d'Israël, et des étrangers qui font leur séjour parmi eux, aura pris à la chasse une bête sauvage ou un oiseau que l'on mange, il répandra leur sang et le couvrira de poussière.
\VS{14}Car l'âme de toute chair est dans son sang, c'est son âme. C'est pourquoi j'ai dit aux enfants d'Israël : Vous ne mangerez point le sang d'aucune chair ; car l'âme de toute chair est son sang : Quiconque en mangera sera retranché.
\VS{15}Et toute personne qui aura mangé de la chair de quelque bête morte d'elle-même ou déchirée par les bêtes sauvages, tant celui qui est né dans le pays que l'étranger, lavera ses vêtements et se lavera avec de l'eau, et il sera impur jusqu'au soir ; puis il sera pur.
\VS{16}S'il ne lave pas ses vêtements et son corps, il portera son iniquité.
\Chap{18}
\TextTitle{Condamnation des incestes}
\VerseOne{}Yahweh parla encore à Moïse, en disant :
\VS{2}Parle aux enfants d'Israël et dis-leur : Je suis Yahweh, votre Dieu.
\VS{3}Vous ne ferez point ce qui se fait dans le pays d'Egypte où vous avez habité, ni ce qui se fait dans le pays de Canaan, auquel je vous amène : Vous ne vivrez point selon leurs statuts\FTNT{Jé. 10:2.}.
\VS{4}Mais vous ferez selon mes statuts, et vous garderez mes ordonnances pour marcher en elles. Je suis Yahweh, votre Dieu.
\VS{5}Vous garderez donc mes statuts et mes ordonnances, l'homme qui les pratiquera vivra par elles. Je suis Yahweh\FTNT{Ez. 20:11-13 ; Ga. 3:12 ; Ro. 10:5.}.
\VS{6}Que nul ne s'approche de celle qui est sa proche parente pour découvrir sa nudité. Je suis Yahweh.
\VS{7}Tu ne découvriras point la nudité de ton père, ni la nudité de ta mère. C'est ta mère ; tu ne découvriras point sa nudité.
\VS{8}Tu ne découvriras point la nudité de la femme de ton père. C'est la nudité de ton père\FTNT{De. 22:30 ; 1 Co. 5:1.}.
\VS{9}Tu ne découvriras point la nudité de ta sœur, fille de ton père ou fille de ta mère, née dans la maison ou hors de la maison. Tu ne découvriras point leur nudité.
\VS{10}Quant à la nudité de la fille de ton fils ou de la fille de ta fille, tu ne découvriras point leur nudité. Car elles sont ta nudité.
\VS{11}Tu ne découvriras point la nudité de la fille de la femme de ton père, née de ton père. C'est ta sœur.
\VS{12}Tu ne découvriras point la nudité de la sœur de ton père. Elle est la proche parente de ton père.
\VS{13}Tu ne découvriras point la nudité de la sœur de ta mère ; car elle est la proche parente de ta mère.
\VS{14}Tu ne découvriras point la nudité du frère de ton père. Et tu ne t'approcheras point de sa femme. Elle est ta tante.
\VS{15}Tu ne découvriras point la nudité de ta belle-fille. Elle est la femme de ton fils ; tu ne découvriras point sa nudité.
\VS{16}Tu ne découvriras point la nudité de la femme de ton frère. C'est la nudité de ton frère.
\VS{17}Tu ne découvriras point la nudité d'une femme et de sa fille. Et tu ne prendras point la fille de son fils, ni la fille de sa fille pour découvrir leur nudité. Elles sont tes proches parentes : C'est un crime.
\VS{18}Tu ne prendras point aussi une femme avec sa sœur pour exciter une rivalité en découvrant sa nudité à côté d'elle pendant sa vie.
\TextTitle{Condamnation des abominations}
\VS{19}Tu ne t'approcheras point d'une femme durant son impureté menstruelle, pour découvrir sa nudité.
\VS{20}Tu ne coucheras point avec la femme de ton prochain pour te souiller avec elle\FTNT{Ex. 20:17 ; De. 5:21 ; Mt. 5:28.}.
\VS{21}Tu ne donneras point tes enfants pour les faire passer par le feu devant Moloc\FTNT{Moloc est le nom du dieu auquel les Ammonites, peuple issu de la relation incestueuse de Loth et sa fille, sacrifiaient leurs premiers-nés en les jetant dans un brasier. De. 18:9-10 ; 1 R. 11:5-7 ; 2 R. 23:10 ; Jé. 32:35.}, et tu ne profaneras point le nom de ton Dieu. Je suis Yahweh.
\VS{22}Tu ne coucheras pas aussi avec un homme, comme on couche avec une femme. C'est une abomination\FTNT{1 Co. 6:9-10 ; Ge. 13:13 ; Ro. 1:26-27.}.
\VS{23}Tu ne coucheras point aussi avec une bête pour te souiller avec elle ; et la femme ne se prostituera point à une bête ; c'est une confusion\FTNT{1 Co. 6:9-10 ; Ro. 1:26-27.}.
\VS{24}Ne vous rendez point impurs par aucune de ces choses, car les nations que je vais chasser de devant vous se sont rendues impures par toutes ces choses.
\VS{25}Le pays a été rendu impur ; et je punirai sur lui son iniquité, et le pays vomira ses habitants.
\VS{26}Mais quant à vous, vous garderez mes ordonnances et mes jugements, et vous ne ferez aucune de ces abominations, tant celui qui est né dans le pays que l'étranger qui fait son séjour parmi vous.
\VS{27}Car les gens de ce pays-là qui ont été avant vous, ont fait toutes ces abominations, et le pays en a été rendu impur.
\VS{28}Prenez garde que le pays ne vous vomisse, si vous le rendez impur, comme il aura vomi les nations qui y étaient avant vous.
\VS{29}Car tous ceux qui feront l'une de toutes ces abominations, seront retranchés du milieu de leur peuple.
\VS{30}Vous garderez donc ce que j'ai ordonné de garder, et vous ne pratiquerez aucune de ces coutumes abominables qui ont été pratiquées avant vous, et vous ne vous rendrez point impurs par elles. Je suis Yahweh, votre Dieu.
\Chap{19}
\TextTitle{Mise en garde contre l'idolâtrie}
\VerseOne{}Yahweh parla aussi à Moïse, en disant :
\VS{2}Parle à toute l'assemblée des enfants d'Israël, et dis-leur : Soyez saints, car je suis saint, moi, Yahweh, votre Dieu.
\VS{3}Chacun de vous craindra sa mère et son père, et vous garderez mes sabbats. Je suis Yahweh, votre Dieu\FTNT{Ex. 20:12 ; De. 5:16 ; Mt. 15:4.}.
\VS{4}Vous ne vous tournerez point vers les idoles, et vous ne vous ferez aucun dieu de fonte. Je suis Yahweh, votre Dieu\FTNT{Ex. 20:3-5.}.
\TextTitle{Recommandation pour les sacrifices}
\VS{5}Si vous offrez un sacrifice d'offrande de paix\FTNT{Voir commentaire en Lé. 3:1.} à Yahweh, vous le sacrifierez de votre bon gré.
\VS{6}II se mangera le jour où vous l'aurez sacrifié, et le lendemain, mais ce qui restera jusqu'au troisième jour sera brûlé au feu.
\VS{7}Si on en mange au troisième jour, ce sera une abomination : Il ne sera point agréé.
\VS{8}Quiconque aussi en mangera portera son iniquité ; car il aura profané la chose sainte de Yahweh : Cette personne-là sera retranchée d'entre ses peuples.
\TextTitle{La justice de Yahweh, l'amour pour son prochain}
\VS{9}Quand vous ferez la moisson de votre pays, tu n'achèveras point de moissonner le bout de ton champ, et tu ne glaneras point ce qui restera à cueillir de ta moisson.
\VS{10}Tu ne grappilleras point ta vigne, ni ne recueilleras point les grains tombés de ta vigne, mais tu les laisseras au pauvre et à l'étranger\FTNT{De. 24:19.}. Je suis Yahweh, votre Dieu.
\VS{11}Vous ne déroberez point, et vous ne vous tromperez point les uns les autres ; et aucun de vous ne mentira à son prochain\FTNT{Ex. 20:15 ; Ep. 4:25 ; Col. 3:9.}.
\VS{12}Vous ne jurerez point par mon Nom en mentant, car tu profanerais le Nom de ton Dieu\FTNT{Ex. 20:7 ; De. 5:11.}. Je suis Yahweh.
\VS{13}Tu n'opprimeras point ton prochain, et tu ne le pilleras point\FTNT{De. 24:14-15 ; Ja. 5:4.}. Le salaire de ton mercenaire ne demeurera point chez toi jusqu'au lendemain.
\VS{14}Tu ne maudiras point le sourd, et tu ne mettras point d'achoppement devant l'aveugle, mais tu craindras ton Dieu. Je suis Yahweh.
\VS{15}Vous ne ferez point d'iniquité dans vos jugements : Tu n'auras point d'égard à la personne du pauvre, et tu n'honoreras point la personne du grand, mais tu jugeras ton prochain selon la justice.
\VS{16}Tu ne répandras point de calomnies parmi ton peuple. Tu ne t'élèveras point contre le sang de ton prochain. Je suis Yahweh.
\VS{17}Tu ne haïras point ton frère dans ton cœur ; tu reprendras soigneusement ton prochain\FTNT{Ge. 4:8 ; Mt. 18:15 ; 1 Jn. 2:9-11.}, et tu ne te chargeras point d'un péché à cause de lui.
\VS{18}Tu n'useras point de vengeance, et tu ne la garderas point aux enfants de ton peuple ; mais tu aimeras ton prochain comme toi-même\FTNT{Mt. 7:12 ; Mc. 12:28-34.}. Je suis Yahweh.
\VS{19}Vous garderez mes ordonnances. Tu n'accoupleras point tes bêtes de deux espèces différentes ; tu ne sèmeras point ton champ de diverses sortes de grains ; et tu ne mettras point sur toi de vêtements de diverses espèces, comme de la laine et du lin.
\VS{20}Si un homme couche et a commerce avec une femme, si c'est une esclave, fiancée à un homme, qui n'a pas été rachetée, et que la liberté ne lui a pas été donnée, ils auront le fouet, mais on ne les fera point mourir, parce qu'elle n'a pas été affranchie.
\VS{21}L'homme amènera son sacrifice pour la culpabilité à Yahweh à l'entrée de la tente d'assignation, à savoir un bélier pour la culpabilité.
\VS{22}Et le prêtre fera propitiation pour lui devant Yahweh par le bélier du sacrifice pour la culpabilité, à cause de son péché qu'il aura commis, et son péché qu'il aura commis lui sera pardonné.
\TextTitle{Ordonnances diverses}
\VS{23}Et quand vous serez entrés dans le pays, et que vous y aurez planté quelque arbre fruitier, vous considérerez son fruit comme incirconcis ; il vous sera incirconcis pendant trois ans, on n'en mangera point.
\VS{24}Mais à la quatrième année, tout son fruit sera une chose sainte à la louange de Yahweh.
\VS{25}Et à la cinquième année, vous mangerez son fruit, afin qu'il vous multiplie son produit. Je suis Yahweh, votre Dieu.
\VS{26}Vous ne mangerez rien avec le sang. Vous n'userez point de divinations, et vous ne pronostiquerez point le temps\FTNT{De. 12:23.}.
\VS{27}Vous ne couperez point en rond les coins de votre chevelure, et vous ne raserez point les coins de votre barbe.
\VS{28}Vous ne ferez point d'incisions dans votre chair pour un mort, et vous n'imprimerez point de caractères sur vous. Je suis Yahweh.
\VS{29}Tu ne profaneras point ta fille en la prostituant ; afin que le pays ne se prostitue point et ne se remplisse point de crimes.
\VS{30}Vous garderez mes sabbats et vous aurez en révérence mon sanctuaire. Je suis Yahweh.
\VS{31}Ne vous tournez point vers ceux qui évoquent les morts, ni vers les devins\FTNT{Ac. 16:16.} ; ne cherchez point à vous rendre impurs avec eux. Je suis Yahweh, votre Dieu.
\VS{32}Lève-toi devant les cheveux blancs, et tu honoreras la personne du vieillard. Tu craindras ton Dieu. Je suis Yahweh.
\VS{33}Si quelque étranger séjourne dans votre pays, vous ne lui ferez point de tort.
\VS{34}L'étranger qui séjourne parmi vous, vous sera comme celui qui est né parmi vous, et vous l'aimerez comme vous-mêmes, car vous avez été étrangers dans le pays d'Egypte. Je suis Yahweh, votre Dieu.
\VS{35}Vous ne ferez point d'iniquité dans les jugements, ni dans les mesures de dimension, ni dans les poids, ni dans les mesures de capacité.
\VS{36}Vous aurez les balances justes, les pierres à peser justes, l'épha juste et le hin juste. Je suis Yahweh, votre Dieu, qui vous ai fait sortir du pays d'Egypte.
\VS{37}Gardez donc toutes mes ordonnances et mes jugements, et pratiquez-les. Je suis Yahweh.
\Chap{20}
\TextTitle{Abominations diverses et leurs châtiments}
\VerseOne{}Yahweh parla aussi à Moïse, en disant :
\VS{2}Tu diras aux enfants d'Israël : Quiconque des enfants d'Israël ou des étrangers qui demeurent en Israël, qui donnera de sa postérité à Moloc, sera puni de mort : Le peuple du pays le lapidera.
\VS{3}Et je mettrai ma face contre un tel homme, et je le retrancherai du milieu de son peuple, parce qu'il a donné de sa postérité à Moloc, pour rendre impur mon sanctuaire et profaner le Nom de ma sainteté.
\VS{4}Si le peuple du pays ferme les yeux en quelque manière que ce soit sur cet homme-là, qui donne de sa postérité à Moloc, et s'il ne le fait pas mourir,
\VS{5}je mettrai ma face contre cet homme-là, contre sa famille, et je le retrancherai du milieu de mon peuple, avec tous ceux qui se prostituent comme lui, en se prostituant après Moloc.
\VS{6}Quant à la personne qui se tournera vers ceux qui évoquent les morts, vers les devins, en se prostituant après eux, je mettrai ma face contre cette personne-là, et je la retrancherai du milieu de son peuple.
\VS{7}Sanctifiez-vous donc, et soyez saints, car je suis Yahweh, votre Dieu.
\VS{8}Gardez aussi mes lois et pratiquez-les. Je suis Yahweh, qui vous sanctifie.
\VS{9}Un homme qui maudit son père ou sa mère sera puni de mort ; il a maudit son père ou sa mère : Son sang retombera sur lui.
\VS{10}Quant à l'homme qui commet un adultère avec la femme d'un autre, parce qu'il a commis un adultère avec la femme de son prochain, l'homme et la femme adultères seront mis à mort.
\VS{11}L'homme qui couche avec la femme de son père, découvre la nudité de son père, les deux seront mis à mort, leur sang est sur eux.
\VS{12}Quant un homme couche avec sa belle-fille, ils seront mis à mort, tous deux ; ils ont fait une confusion : Leur sang est sur eux.
\VS{13}Quant un homme couche avec un homme comme on couche avec une femme, ils ont tous deux fait une chose abominable ; ils seront mis à mort : Leur sang est sur eux.
\VS{14}Et si un homme prend pour femmes la fille et la mère, c'est un crime : Il sera brûlé au feu avec elles, afin que ce crime n'existe pas au milieu de vous.
\VS{15}Si un homme couche avec une bête, il sera puni de mort ; et vous tuerez aussi la bête.
\VS{16}Et si une femme s'approche d'une bête, tu tueras cette femme et la bête ; ils seront mis à mort : Leur sang sera sur eux.
\VS{17}Si un homme prend sa sœur, fille de son père ou fille de sa mère, et voit sa nudité, et qu'elle voit la nudité de cet homme, c'est une chose infâme ; ils seront donc retranchés sous les yeux des fils de leur peuple : Il a découvert la nudité de sa sœur, il portera son iniquité.
\VS{18}Si un homme couche avec une femme qui a son indisposition menstruelle, et qu'il découvre la nudité de cette femme, en découvrant son flux, et qu'elle découvre le flux de son sang, ils seront tous deux retranchés du milieu de leur peuple.
\VS{19}Tu ne découvriras point la nudité de la sœur de ta mère, ni de la sœur de ton père, car c'est découvrir sa proche parente, ils porteront tous deux leur iniquité.
\VS{20}Si un homme couche avec sa tante, il a découvert la nudité de son oncle ; ils porteront leur péché, et ils mourront privés d'enfants.
\VS{21}Si un homme prend la femme de son frère, c'est une impureté ; il a découvert la nudité de son frère, ils seront privés d'enfants.
\VS{22}Vous garderez toutes mes ordonnances et mes jugements et vous les pratiquerez, afin que le pays où je vous fais entrer pour y habiter ne vous vomisse point.
\VS{23}Vous ne suivrez point les statuts des nations que je vais chasser devant vous ; car elles ont fait toutes ces choses-là, et je les ai eues en abomination.
\VS{24}Et je vous ai dit : Vous posséderez leur pays, je vous le donnerai en possession : C'est un pays où coulent le lait et le miel. Je suis Yahweh, votre Dieu, qui vous ai séparés des autres peuples.
\VS{25}C'est pourquoi séparez les bêtes pures de celles qui sont impures, les oiseaux purs de ceux qui sont impurs, et ne rendez point abominables vos personnes en mangeant des bêtes et des oiseaux impurs, ni rien qui rampe sur la terre, rien de ce que je vous ai défendu comme une chose impure.
\VS{26}Vous me serez donc saints, car je suis saint, moi, Yahweh ; je vous ai séparés des autres peuples afin que vous soyez à moi.
\VS{27}Si un homme ou une femme évoquent les morts ou se livrent à la divination, on les mettra à mort ; on les lapidera : Leur sang sera sur eux.
\Chap{21}
\TextTitle{Recommandations aux prêtres}
\VerseOne{}Yahweh dit aussi à Moïse : Parle aux prêtres, fils d'Aaron, et dis-leur : Aucun d'eux ne se rendra impur parmi son peuple pour un mort,
\VS{2}excepté pour son proche parent, pour sa mère, pour son père, pour son fils, pour sa fille, et pour son frère,
\VS{3}et aussi pour sa sœur vierge, qui lui est proche, et qui n'aura point eu de mari, il se rendra impur pour elle.
\VS{4}Chef parmi son peuple, il ne se rendra point impur en se profanant.
\VS{5}Ils ne se feront point de place chauve sur la tête, ils ne raseront point les coins de leur barbe, ni ne feront point d'incisions dans leur chair.
\VS{6}Ils seront consacrés à leur Dieu, et ils ne profaneront point le Nom de leur Dieu ; car ils offrent à Yahweh les sacrifices consumés par le feu, qui sont la nourriture de leur Dieu : C'est pourquoi ils seront très saints.
\VS{7}Ils ne prendront point une femme prostituée ou déshonorée ; ils ne prendront point une femme répudiée par son mari, car ils sont saints pour leur Dieu.
\VS{8}Tu regarderas chacun d'eux comme saint, parce qu'ils offrent la nourriture de ton Dieu ; ils seront saints, car je suis saint, moi, Yahweh, qui vous sanctifie.
\VS{9}Si la fille du prêtre se profane en se prostituant, elle déshonore son père : Qu'elle soit brûlée au feu.
\VS{10}Le grand prêtre d'entre ses frères, sur la tête duquel l'huile d'onction a été répandue, et qui se sera consacré pour vêtir les saints vêtements, ne découvrira point sa tête et ne déchirera point ses vêtements.
\VS{11}Il n'ira vers aucune personne morte, il ne se rendra point impur pour son père ni pour sa mère.
\VS{12}Il ne sortira point du sanctuaire, et ne profanera point le sanctuaire de son Dieu ; car l'huile d'onction de son Dieu est une couronne sur lui. Je suis Yahweh.
\VS{13}Il prendra pour femme une vierge.
\VS{14}Il ne prendra point une veuve, ni une répudiée, ni une femme déshonorée ou prostituée ; mais il prendra pour femme une vierge parmi son peuple.
\VS{15}Il ne profanera point sa postérité parmi son peuple ; car je suis Yahweh qui le sanctifie.
\VS{16}Yahweh parla aussi à Moïse, en disant :
\VS{17}Parle à Aaron, et dis-lui : Si quelqu'un de ta postérité, parmi tes descendants, qui a quelque défaut corporel, il ne s'approchera point pour offrir la nourriture de son Dieu.
\VS{18}Car tout homme en qui il y aura un défaut n'en approchera point ; l'homme aveugle, boiteux, ayant le nez camus ou qui aura un membre allongé ;
\VS{19}ou l'homme qui aura une fracture aux pieds ou aux mains ;
\VS{20}ou qui sera bossu ou grêle, qui aura une tache à l'œil, qui aura une gale sèche, une dartre, ou qui aura les testicules écrasés.
\VS{21}Nul homme de la postérité d'Aaron, le prêtre, en qui il y aura un défaut corporel, ne s'approchera pour offrir les offrandes consumées par le feu à Yahweh ; il y a un défaut en lui, il ne s'approchera donc point pour offrir la nourriture de son Dieu.
\VS{22}Il pourra manger la nourriture de son Dieu, des choses très saintes et des choses saintes.
\VS{23}Mais il n'entrera point vers le voile, ni ne s'approchera point de l'autel, car il a un défaut corporel, et il ne profanera point mes sanctuaires, car je suis Yahweh, qui les sanctifie.
\VS{24}Moïse parla ainsi à Aaron et à ses fils, et à tous les enfants d'Israël.
\Chap{22}
\TextTitle{Consécration d'Aaron et de ses fils}
\VerseOne{}Puis Yahweh parla à Moïse, en disant :
\VS{2}Parle à Aaron et à ses fils, afin qu'ils s'abstiennent des choses saintes des enfants d'Israël, et qu'ils ne profanent point le Nom de ma sainteté dans les choses qu'ils me consacrent. Je suis Yahweh.
\VS{3}Dis-leur donc : Tout homme parmi votre génération et de vos descendants qui, étant impur, s'approchera des choses saintes que les enfants d'Israël auront sanctifiées à Yahweh, cette personne-là sera retranchée de devant moi. Je suis Yahweh.
\VS{4}Tout homme de la postérité d'Aaron, qui aura la lèpre ou une gonorrhée, ne mangera point des choses saintes jusqu'à ce qu'il soit pur. Il en sera de même pour celui qui touchera quelqu'un s'étant rendu impur en touchant un mort, ou celui qui aura une perte séminale,
\VS{5}et celui qui touchera un reptile et qui en aura été impur, ou un homme atteint d'une impureté quelconque, il en sera rendu impur.
\VS{6}La personne qui touchera ces choses sera rendu impur jusqu'au soir ; il ne mangera point des choses saintes s'il n'a point lavé son corps dans l'eau ;
\VS{7}Ensuite il sera pur après le coucher du soleil, et il mangera des choses saintes, car c'est sa nourriture.
\VS{8}Il ne mangera de la chair d'aucune bête morte d'elle-même ou déchirée par les bêtes sauvages, pour se rendre impur par elle. Je suis Yahweh.
\VS{9}Ils garderont ce que j'ai ordonné de garder, et ils ne commettront point de péché au sujet de la nourriture sainte, afin qu'ils ne meurent point, pour l'avoir profanée. Je suis Yahweh, qui les sanctifie.
\VS{10}Aucun étranger ne mangera des choses saintes ; l'étranger logé chez le prêtre et le mercenaire ne mangeront point des choses saintes.
\VS{11}Mais si le prêtre achète une personne avec son argent, elle en mangera, de même pour celui qui sera né dans sa maison ; ils mangeront de sa nourriture.
\VS{12}Si la fille du prêtre est mariée à un homme étranger, elle ne mangera point des choses saintes présentées en offrande par élévation.
\VS{13}Mais si la fille du prêtre est veuve ou répudiée, et si elle n'a point d'enfants, et est retournée dans la maison de son père, comme dans sa jeunesse, elle mangera de la nourriture de son père. Mais aucun étranger n'en mangera.
\VS{14}Si quelqu'un, pèche involontairement en mangeant d'une chose sainte, il y ajoutera un cinquième et le donnera au prêtre avec la chose sainte.
\VS{15}Et ils ne profaneront point les choses sanctifiées des enfants d'Israël, qu'ils auront offertes à Yahweh.
\VS{16}Mais on leur fera porter la peine du péché, parce qu'ils auront mangé de leurs choses saintes : Car je suis Yahweh, qui les sanctifie.
\TextTitle{Des animaux sans défaut pour les sacrifices\FTNTT{Hé. 9:14.}}
\VS{17}Yahweh parla encore à Moïse, en disant :
\VS{18}Parle à Aaron, à ses fils, et à tous les enfants d'Israël, et dis-leur : Quiconque de la maison d'Israël ou des étrangers qui sont en Israël, offrira son offrande, selon tous ses vœux, ou toutes ses offrandes volontaires, qu'on offre en holocauste à Yahweh,
\VS{19}il offrira de son bon gré, un mâle sans défaut, parmi les bœufs, les agneaux ou les chèvres.
\VS{20}Vous n'offrirez aucune chose qui ait un défaut, car elle ne serait point agréée pour vous.
\VS{21}Si un homme offre à Yahweh un sacrifice d'offrande de paix\FTNT{Voir commentaire en Lé. 3:1.} en s'acquittant d'un vœu, ou en faisant une offrande volontaire, soit de gros ou de menu bétail, elle sera sans défaut pour être agréée ; il ne doit y avoir aucun défaut.
\VS{22}Vous n'offrirez point à Yahweh ce qui sera aveugle, estropié, ou mutilé, qui ait un ulcère, une gale sèche ou une dartre ; et vous n'en ferez point sur l'autel un sacrifice consumé par le feu pour Yahweh.
\VS{23}Tu pourras bien faire une offrande volontaire d'un bœuf, ou d'une brebis, ou d'une chèvre ayant quelques membres allongés, ou quelque défaut dans ses membres, mais ils ne seront point agréés pour le vœu.
\VS{24}Vous n'offrirez point à Yahweh, et ne sacrifierez point dans votre pays un animal qui ait les testicules froissés, cassés, arrachés ou taillés.
\VS{25}Vous ne prendrez point de la main de l'étranger aucune de toutes ces choses pour les offrir comme nourriture à votre Dieu ; car la corruption qui est en eux est un défaut en elles : Elles ne seront point agréées pour vous.
\VS{26}Yahweh parla encore à Moïse, en disant :
\VS{27}Quand un veau, un agneau ou une chèvre seront nés, et qu'ils auront été sept jours sous leur mère, depuis le huitième jour et les suivants, ils seront agréables pour l'offrande du sacrifice consumé par le feu à Yahweh.
\VS{28}Vous n'égorgerez point aussi en un même jour la vache, ou la brebis, ou la chèvre, avec son petit.
\VS{29}Quand vous offrirez un sacrifice de remerciement à Yahweh, vous le sacrifierez de votre bon gré.
\VS{30}Il sera mangé le jour même ; vous n'en laisserez rien jusqu'au matin. Je suis Yahweh.
\VS{31}Gardez mes commandements et pratiquez-les. Je suis Yahweh.
\VS{32}Ne profanez point le nom de ma sainteté, car je serai sanctifié entre les enfants d'Israël. Je suis Yahweh, qui vous sanctifie,
\VS{33}et qui vous ai fait sortir du pays d'Egypte, pour être votre Dieu. Je suis Yahweh.
\Chap{23}
\TextTitle{Les fêtes de Yahweh}
\VerseOne{}Yahweh parla aussi à Moïse en disant :
\VS{2}Parle aux enfants d'Israël et dis-leur : Les fêtes\FTNT{Les fêtes de Yahweh étaient des jours solennels, c'est-à-dire des temps fixés pour s'approcher de Dieu et présenter des sacrifices (Voir le tableau en annexe « Les 7 fêtes de Yahweh » et également le dictionnaire).} solennelles de Yahweh, que vous publierez, seront de saintes convocations. Ce sont ici mes fêtes solennelles.
\VS{3}On travaillera six jours ; mais au septième jour, qui est le sabbat, le jour du repos, il y aura une sainte convocation. Vous ne ferez aucune œuvre, car c'est le sabbat à Yahweh, dans toutes vos demeures.
\TextTitle{La Pâque}
\VS{4}Ce sont ici les fêtes solennelles de Yahweh, qui seront de saintes convocations, que vous publierez en leur saison.
\VS{5}Au premier mois, le quatorzième jour du mois, entre les deux soirs, sera la Pâque\FTNT{La pâque était une fête qui commémorait la sortie d'Egypte (Ex. 12:1-14). Elle préfigurait la rédemption en Jésus-Christ, notre Pâque (1 Co. 5:7). Elle était fixée au 14ème jour du mois de Nisan, le premier mois.} à Yahweh.
\TextTitle{La fête des pains sans levain\FTNTT{Ex. 12:18 ; 13:6-8 ; 1 Co. 11:23-26.}}
\VS{6}Et le quinzième jour de ce même mois, sera la fête solennelle des pains sans levain\FTNT{La fête des pains sans levain commençait le 15ème jour du même mois (Nisan) et durait sept jours. Elle annonçait Christ, notre Pain descendu du ciel (Jn. 6:32-35). Seul le Seigneur Jésus a été sans levain, c'est-à-dire sans aucun péché. Le croyant est sauvé à la Pâque de Christ et doit vivre une vie sans péché (la fête des pains sans levain).} à Yahweh ; vous mangerez des pains sans levain pendant sept jours.
\VS{7}Le premier jour, vous aurez une sainte convocation : Vous ne ferez aucune œuvre servile.
\VS{8}Mais vous offrirez à Yahweh pendant sept jours des offrandes consumées par le feu. Et au septième jour, il y aura une sainte convocation : Vous ne ferez aucune œuvre servile.
\TextTitle{La fête des prémices\FTNTT{1 Co. 15:23.}}
\VS{9}Yahweh parla aussi à Moïse, en disant :
\VS{10}Parle aux enfants d'Israël et dis-leur : Quand vous serez entrés dans le pays que je vous donne, et que vous en aurez fait la moisson, vous apporterez alors au prêtre une gerbe des premiers fruits\FTNT{La fête des prémices annonce d'abord la résurrection du Seigneur Jésus-Christ, ensuite celle de tous ceux qui lui appartiennent (1 Th. 4:13-18 ; 1 Co. 15:23). Elle commençait le premier jour de la semaine suivant le sabbat de la Pâque, au mois de Nisan.} de votre moisson.
\VS{11}Et il agitera cette gerbe-là devant Yahweh, afin qu'elle soit agréée pour vous : Le prêtre l'agitera le lendemain du sabbat.
\VS{12}Et le jour où vous agiterez cette gerbe, vous sacrifierez un agneau sans défaut et d'un an, en holocauste à Yahweh ;
\VS{13}et le gâteau de cet holocauste sera de deux dixièmes de fine farine, pétrie à l'huile, pour offrande consumée par le feu, en bonne odeur à Yahweh ; et sa libation de vin sera le quart d'un hin.
\VS{14}Vous ne mangerez ni pain, ni grain rôti, ni grain en épi, jusqu'à ce jour-là, même jusqu'à ce que vous ayez apporté l'offrande à votre Dieu. C'est une loi perpétuelle pour vos descendants, dans toutes vos demeures.
\TextTitle{La Pentecôte ou la fête des semaines}
\VS{15}Vous compterez aussi dès le lendemain du sabbat, à savoir dès le jour où vous aurez apporté la gerbe qu'on doit agiter, sept semaines entières.
\VS{16}Vous compterez donc cinquante jours\FTNT{La fête des semaines ou fête de la moisson est désignée également comme la Pentecôte. Elle avait lieu au mois de Sivan et préfigurait l'effusion du Saint-Esprit et l'inauguration de la Nouvelle Alliance (Ac. 2:1-4). Le levain autorisé lors de cette fête évoquait par avance la présence de l'ivraie, symbole du péché et des fils du malin, parmi le blé, c'est-à-dire les enfants de Dieu (Mt. 13:24-41). Cinquante jours séparent la Pâque de la Pentecôte. Cet intervalle correspond exactement à la période séparant la résurrection du Seigneur Jésus-Christ de la naissance de l'Eglise (Ac. 2:1-4).} jusqu'au lendemain du septième sabbat ; et vous offrirez à Yahweh un gâteau nouveau.
\VS{17}Vous apporterez de vos demeures deux pains pour en faire une offrande agitée, ils seront de deux dixièmes, et de fine farine, pétris avec du levain : Ce sont les premiers fruits à Yahweh.
\VS{18}Vous offrirez aussi avec ce pain-là sept agneaux sans défaut et d'un an, un jeune taureau pris du troupeau et deux béliers, qui seront un holocauste à Yahweh, avec leurs gâteaux et leurs libations, des sacrifices consumés par le feu, en bonne odeur à Yahweh.
\VS{19}Vous sacrifierez aussi un jeune bouc en sacrifice pour l'expiation, et deux agneaux d'un an pour le sacrifice d'offrande de paix\FTNT{Voir commentaire en Lé. 3:1.}.
\VS{20}Et le prêtre les agitera avec le pain des premiers fruits, et avec les deux agneaux, en offrande agitée devant Yahweh : Ils seront saints à Yahweh, pour le prêtre.
\VS{21}Vous publierez donc, en ce même jour-là, une sainte convocation : Vous ne ferez aucune œuvre servile. C'est une ordonnance perpétuelle dans toutes vos demeures, pour vos descendants.
\VS{22}Et quand vous ferez la moisson de votre pays, tu n'achèveras point de moissonner le bout de ton champ, et tu ne glaneras point les épis qui resteront de ta moisson. Mais tu les laisseras pour le pauvre et pour l'étranger. Je suis Yahweh, votre Dieu.
\TextTitle{La fête des trompettes}
\VS{23}Yahweh parla aussi à Moïse, en disant :
\VS{24}Parle aux enfants d'Israël et dis-leur : Au septième mois, le premier jour du mois, il y aura un jour de repos pour vous, un mémorial de jubilation\FTNT{La fête des trompettes préfigure le rassemblement futur du peuple d'Israël après sa longue dispersion et l'enlèvement de l'Eglise. Cette fête était fixée au premier jour du septième mois (Tishri).}, et une sainte convocation.
\VS{25}Vous ne ferez aucune œuvre servile, et vous offrirez à Yahweh des offrandes consumées par le feu.
\TextTitle{Le jour des expiations\FTNTT{Hé. 9:1-16.}}
\VS{26}Yahweh parla aussi à Moïse, en disant :
\VS{27}Pareillement en ce même mois, qui est le septième, le dixième jour sera le jour des expiations\FTNT{Le jour des expiations ou du grand pardon (Voir Lé. 16) était célébré le dixième jour du septième mois (Tishri). Le Seigneur Jésus-Christ a fait l'expiation de nos péchés afin de nous amener à Dieu. Le propitiatoire au lieu d'être le trône du jugement, devenait ainsi le lieu de rencontre de Dieu avec le croyant (Ex. 25:22). Christ est la propitiation pour nos péchés (1 Jn. 2:2), mais il est aussi lui-même le propitiatoire (Ro. 3:25). Le péché ôté, les fautes confessées, le pardon acquis, l'holocauste offert, le chemin est ouvert pour la joie de la fête des tabernacles.} : Vous aurez une sainte convocation, vous humilierez vos âmes, et vous offrirez à Yahweh des sacrifices consumés par le feu.
\VS{28}En ce jour-là, vous ne ferez aucune œuvre, car c'est le jour des expiations, afin de faire propitiation pour vous devant Yahweh, votre Dieu.
\VS{29}Toute personne qui ne s'humiliera point en ce jour-là sera retranchée d'entre son peuple.
\VS{30}Et toute personne qui aura fait quelque œuvre en ce jour-là, je ferai périr cette personne-là du milieu de son peuple.
\VS{31}Vous ne ferez donc aucune œuvre. C'est une ordonnance perpétuelle pour vos descendants dans toutes vos demeures.
\VS{32}Ce sera pour vous un sabbat, un jour de repos, et vous humilierez vos âmes. Le neuvième jour du mois, au soir, depuis le soir jusqu'à l'autre soir, vous célébrerez votre sabbat.
\TextTitle{La fête des tabernacles\FTNTT{Esd. 3:4.}}
\VS{33}Yahweh parla aussi à Moïse, en disant :
\VS{34}Parle aux enfants d'Israël, et dis-leur : Au quinzième jour de ce septième mois sera la fête solennelle des tabernacles\FTNT{La fête des tabernacles ou des récoltes, était la fête du souvenir et de la joie. Célébrée au mois de Tishri, elle était aussi celle du repos, dans l'accomplissement des promesses. Elle préfigure le Royaume millénaire (Za. 14).} pendant sept jours, à Yahweh.
\VS{35}Au premier jour, il y aura une sainte convocation : Vous ne ferez aucune œuvre servile.
\VS{36}Pendant sept jours, vous offrirez à Yahweh des offrandes consumées par le feu. Et au huitième jour, vous aurez une sainte convocation, et vous offrirez à Yahweh des offrandes consumées par le feu ; ce sera une assemblée solennelle : Vous ne ferez aucune œuvre servile.
\VS{37}Ce sont là les fêtes solennelles de Yahweh, que vous publierez pour être des convocations saintes, afin d'offrir à Yahweh des offrandes consumées par le feu ; à savoir un holocauste, un gâteau, un sacrifice et une libation, chacune de ces choses en son jour ;
\VS{38}outre les sabbats de Yahweh, et outre vos dons, outre tous vos vœux, outre toutes les offrandes volontaires que vous présenterez à Yahweh.
\VS{39}Et aussi au quinzième jour du septième mois, quand vous aurez recueilli le produit du pays, vous célébrerez la fête solennelle de Yahweh pendant sept jours : Le premier jour sera un jour de repos, le huitième aussi sera un jour de repos.
\VS{40}Et au premier jour, vous prendrez du fruit d'un bel arbre, des branches de palmier, des rameaux d'arbres touffus et des saules de rivière ; et vous vous réjouirez pendant sept jours, devant Yahweh, votre Dieu.
\VS{41}Et vous célébrerez à Yahweh cette fête solennelle pendant sept jours dans l'année. C'est une loi perpétuelle pour vos descendants. Vous la célébrerez le septième mois.
\VS{42}Vous demeurerez sept jours sous des tentes ; tous ceux qui seront nés entre les Israélites demeureront sous des tentes,
\VS{43}afin que votre postérité sache que j'ai fait habiter les enfants d'Israël sous des tentes, quand je les ai fait sortir du pays d'Egypte. Je suis Yahweh, votre Dieu.
\VS{44}Moïse déclara ainsi aux enfants d'Israël les fêtes solennelles de Yahweh.
\Chap{24}
\TextTitle{L'huile du chandelier\FTNTT{Ex. 25:6.}}
\VerseOne{}Yahweh parla aussi à Moïse, en disant :
\VS{2}Ordonne aux enfants d'Israël de t'apporter de l'huile pure d'olives pressées pour le chandelier, afin de faire brûler les lampes continuellement.
\VS{3}Aaron les arrangera devant Yahweh continuellement, depuis le soir jusqu'au matin, en dehors du voile du témoignage dans la tente d'assignation. C'est une ordonnance perpétuelle pour vos descendants.
\VS{4}Il arrangera, dis-je, continuellement les lampes sur le chandelier pur, devant Yahweh.
\TextTitle{Les pains de proposition\FTNTT{Ex. 25:23-30.}}
\VS{5}Tu prendras aussi de la fine farine\FTNT{La fine farine est une farine de blé très pure, la première qui passe à travers les tamis de bluterie.}, et tu en feras cuire douze gâteaux\FTNT{Les pains de proposition étaient au nombre de douze et ne pouvaient être consommés que par les prêtres (Lé. 24:9). Ils préfiguraient Christ, le véritable pain de vie descendu du ciel (Jn. 6:48-51). Sous la Nouvelle Alliance, chaque enfant de Dieu est également un prêtre (Ap. 1:6), et est invité par conséquent à manger ce pain. Le nombre douze nous parle du fondement sur lequel nous devons êtres bâtis, à savoir Jésus-Christ lui-même et l'enseignement des apôtres et des prophètes (1 Co. 3:11 ; Ep. 2:20).}, chaque gâteau sera de deux dixièmes.
\VS{6}Et tu les exposeras devant Yahweh en deux rangées sur la table d'or pur, six à chaque rangée.
\VS{7}Et tu mettras de l'encens pur sur chaque rangée, qui sera comme un souvenir\FTNT{Voir commentaire en Lé. 2:2.} pour le pain, c'est une offrande consumée par le feu à Yahweh.
\VS{8}On les arrangera chaque jour de sabbat continuellement devant Yahweh, de la part des enfants d'Israël : C'est une alliance perpétuelle.
\VS{9}Et ils appartiendront à Aaron et à ses fils, qui les mangeront dans un lieu saint ; car ce sera pour eux une chose très sainte d'entre les offrandes de Yahweh consumées par le feu. C'est une ordonnance perpétuelle.
\TextTitle{Le blasphème contre le Nom de Yahweh\FTNTT{Jn. 8:59 ; 10:31.}}
\VS{10}Or le fils d'une femme israélite, qui était aussi fils d'un homme égyptien, sortit parmi les fils d'Israël, et ce fils de la femme israélite se querella dans le camp avec un homme israélite.
\VS{11}Et le fils de la femme israélite blasphéma et maudit le Nom de Yahweh. On l'amena à Moïse. Or sa mère s'appelait Schelomith, fille de Dibri, de la tribu de Dan.
\VS{12}Et on le mit en prison, jusqu'à ce que Moïse ait déclaré ce qu'il devrait faire selon la parole de Yahweh.
\VS{13}Et Yahweh parla à Moïse, en disant :
\VS{14}Tire hors du camp celui qui a maudit ; et que tous ceux qui l'ont entendu mettent les mains sur sa tête, et que toute l'assemblée le lapide.
\VS{15}Tu parleras aux enfants d'Israël, et tu leur diras : Quiconque aura maudit son Dieu, portera la peine de son péché.
\VS{16}Et celui qui aura blasphémé le Nom de Yahweh sera puni de mort : Toute l'assemblée ne manquera pas de le lapider, on fera mourir tant l'étranger que celui qui est né au pays, lequel aura blasphémé le Nom de Yahweh.
\TextTitle{La violence punie}
\VS{17}On punira aussi de mort celui qui aura frappé à mort quelque personne que ce soit.
\VS{18}Celui qui aura frappé une bête à mort, la remplacera : Vie pour vie.
\VS{19}Et quand quelque homme aura fait une blessure à son prochain, on lui fera comme il a fait :
\VS{20}Fracture pour fracture, œil pour œil, dent pour dent, selon le mal qu'il aura fait à un homme, il lui sera fait de même.
\VS{21}Celui qui frappera une bête à mort, la remplacera ; mais on fera mourir celui qui aura frappé un homme à mort.
\VS{22}Vous rendrez un même jugement. Vous traiterez l'étranger comme celui qui est né au pays ; car je suis Yahweh, votre Dieu.
\VS{23}Moïse parla aux enfants d'Israël, qui firent sortir hors du camp celui qui avait maudit, et le lapidèrent. Ainsi les fils d'Israël firent comme Yahweh l'avait ordonné à Moïse.
\Chap{25}
\TextTitle{L'année sabbatique}
\VerseOne{}Yahweh parla aussi à Moïse sur la montagne de Sinaï, en disant :
\VS{2}Parle aux enfants d'Israël, et dis-leur : Quand vous serez entrés dans le pays que je vous donne, la terre se reposera : Ce sera un sabbat à Yahweh.
\VS{3}Pendant six ans tu sèmeras ton champ, et pendant six ans tu tailleras ta vigne ; et tu en recueilleras le produit.
\VS{4}Mais la septième année il y aura un sabbat, un temps de repos pour la terre, ce sera un sabbat à Yahweh : Tu ne sèmeras point ton champ, et tu ne tailleras point ta vigne.
\VS{5}Tu ne moissonneras point ce qui proviendra des grains tombés dans ta moisson, et tu ne vendangeras point les raisins de ta vigne non taillée : Ce sera une année de repos total pour la terre.
\VS{6}Mais ce qui proviendra de la terre l'année du sabbat vous servira de nourriture, à toi, à ton serviteur et à ta servante, à ton mercenaire et à l'étranger qui demeurent avec toi,
\VS{7}à ton bétail et aux animaux qui sont dans ton pays ; tout son produit servira de nourriture.
\TextTitle{L'année du jubilé}
\VS{8}Tu compteras aussi sept sabbats d'années, à savoir sept fois sept ans, et les jours de sept sabbats feront quarante-neuf ans.
\VS{9}Puis tu feras sonner le shofar de jubilation le dixième jour du septième mois ; le jour, dis-je, des expiations, vous ferez sonner le shofar dans tout votre pays.
\VS{10}Et vous sanctifierez la cinquantième année, et publierez la liberté dans le pays à tous ses habitants : Ce sera pour vous l'année du jubilé ; et vous retournerez chacun dans sa possession, et chacun dans sa famille.
\VS{11}Cette cinquantième année vous sera l'année du jubilé : Vous ne sèmerez point et vous ne moissonnerez point ce que la terre rapportera d'elle-même, et vous ne vendangerez point les fruits de la vigne non taillée.
\VS{12}Car c'est l'année du jubilé, elle vous sera sainte. Vous mangerez ce que les champs rapporteront cette année-là.
\VS{13}En cette année du jubilé chacun de vous retournera dans sa possession.
\VS{14}Et si tu fais une vente à ton prochain, ou si tu achètes quelque chose de ton prochain, que nul de vous ne trompe son frère.
\VS{15}Mais tu achèteras de ton prochain selon le nombre des années après le jubilé. Pareillement on te fera les ventes selon le nombre des années de rapport.
\VS{16}Selon qu'il y aura plus d'années, tu augmenteras le prix de ce que tu achètes ; et selon qu'il y aura moins d'années, tu le diminueras ; car on te vend le nombre des récoltes.
\VS{17}Que nul de vous ne trompe son prochain, mais craignez votre Dieu ; car je suis Yahweh, votre Dieu.
\VS{18}Pratiquez mes ordonnances, gardez mes jugements et observez-les, et vous habiterez en sécurité dans le pays.
\VS{19}Et le pays vous donnera ses fruits, vous en mangerez, vous en serez rassasiés, et vous y habiterez en sécurité.
\VS{20}Et si vous dites : Que mangerons-nous la septième année si nous ne semons point, et si nous ne recueillons point notre récolte ?
\VS{21}J'ordonnerai à ma bénédiction de se répandre sur vous dans la sixième année, et la terre rapportera pour trois ans.
\VS{22}Puis vous sèmerez la huitième année, et vous mangerez de l'ancienne récolte jusqu'à la neuvième année ; jusqu'à ce que sa récolte soit venue, vous mangerez de l'ancienne.
\VS{23}La terre ne sera point vendue à perpétuité ; car le pays est à moi, et vous êtes étrangers et forains\FTNT{Forain : Quelqu'un d'extérieur, d'étranger à un lieu.} chez moi.
\VS{24}C'est pourquoi dans tout le pays dont vous aurez la possession, vous donnerez le droit de rachat\FTNT{Pour voir un exemple de ce droit de rachat, voir Ru. 4:1-13.} pour la terre.
\TextTitle{Le droit de rachat}
\VS{25}Si ton frère est devenu pauvre et vend quelque chose de ce qu'il possède, celui qui a le droit de rachat, à savoir son plus proche parent, viendra et rachètera la chose vendue par son frère.
\VS{26}Si cet homme n'a personne qui ait le droit de rachat, et qu'il ait trouvé de lui-même suffisamment de quoi faire le rachat de ce qu'il a vendu,
\VS{27}il comptera les années du temps qu'il a fait la vente, et il restituera le surplus à l'homme auquel il l'avait faite, et ainsi il retournera dans sa possession.
\VS{28}Mais s'il n'a pas trouvé suffisamment de quoi lui rendre, la chose qu'il aura vendue sera dans les mains de celui qui l'aura acheté, jusqu'à l'année du jubilé ; puis l'acheteur en sortira au jubilé, et le vendeur retournera dans sa possession.
\VS{29}Et si quelqu'un a vendu une maison d'habitation dans quelque ville entourée de murs, il aura le droit de rachat jusqu'à la fin de l'année de sa vente ; son droit de rachat sera d'une année.
\VS{30}Mais si elle n'est point rachetée dans l'année accomplie, la maison qui est dans la ville entourée de murs, demeurera à l'acheteur absolument et à ses descendants ; il n'en sortira point au jubilé.
\VS{31}Mais les maisons des villages, qui ne sont point entourés de murs, seront comptées comme des fonds de terre ; le vendeur aura droit de rachat, et l'acheteur sortira au jubilé.
\VS{32}Et quant aux villes des Lévites, les Lévites auront un droit de rachat perpétuel des maisons des villes de leur possession.
\VS{33}Et celui qui achètera une maison des Lévites, sortira au jubilé de la maison vendue, qui est dans la ville de sa possession ; car les maisons des villes des Lévites sont leur possession parmi les enfants d'Israël.
\VS{34}Mais les champs situés autour des villes des Lévites ne seront point vendus ; car c'est leur possession perpétuelle.
\TextTitle{Les traitements du frère pauvre}
\VS{35}Quand ton frère sera devenu pauvre, et qu'il tendra vers toi ses mains tremblantes, tu le soutiendras, tu soutiendras aussi l'étranger, et le forain, afin qu'il vive avec toi.
\VS{36}Tu ne prendras point de lui d'usure ni d'intérêt, mais tu craindras ton Dieu, et ton frère vivra avec toi.
\VS{37}Tu ne lui prêteras point ton argent à intérêt ni ne lui prêteras de tes vivres pour en tirer du profit.
\VS{38}Je suis Yahweh, votre Dieu qui vous ai fait sortir du pays d'Egypte, pour vous donner le pays de Canaan, afin d'être votre Dieu.
\VS{39}Pareillement, quand ton frère sera devenu pauvre auprès de toi, et qu'il se sera vendu à toi, tu ne te serviras point de lui comme on se sert des esclaves.
\VS{40}Mais il sera chez toi comme serait le mercenaire et l'étranger, et il te servira jusqu'à l'année du jubilé.
\VS{41}Alors il sortira de chez toi avec ses fils, il s'en retournera dans sa famille, et rentrera dans la possession de ses pères.
\VS{42}Car ils sont mes serviteurs, parce que je les ai retirés du pays d'Egypte ; c'est pourquoi ils ne seront point vendus comme on vend les esclaves.
\VS{43}Tu ne domineras point sur lui avec dureté, et tu craindras ton Dieu.
\VS{44}C'est des nations qui vous entourent que tu prendras ton esclave et ta servante qui t'appartiendront ; c'est d'elles que vous achèterez l'esclave et la servante.
\VS{45}Vous pourrez aussi en acheter des fils des étrangers qui demeureront chez toi, et même de leurs familles qui seront parmi vous, qui auront engendré dans votre pays, et vous les posséderez.
\VS{46}Vous les aurez comme un héritage pour les laisser à vos enfants après vous, afin qu'ils en héritent la possession, et vous vous servirez d'eux à perpétuité. Mais quant à vos frères, les fils d'Israël, nul ne dominera avec dureté sur son frère.
\VS{47}Et lorsque l'étranger ou le forain qui est avec toi se sera enrichi, et que ton frère qui est avec lui sera devenu si pauvre qu'il se soit vendu à l'étranger, ou au forain qui est avec toi, ou à quelqu'un de la postérité de la famille de l'étranger,
\VS{48}après s'être vendu, il y aura droit de rachat pour lui : Un de ses frères le rachètera.
\VS{49}Son oncle, ou le fils de son oncle, ou quelque autre proche parent de son sang d'entre ceux de sa famille, le rachètera ; ou lui-même, s'il en trouve le moyen, se rachètera.
\VS{50}Et il comptera avec son acheteur depuis l'année qu'il s'est vendu à lui, jusqu'à l'année du jubilé ; de sorte que l'argent du prix pour lequel il s'est vendu, se comptera à raison du nombre des années, le temps qu'il aura servi lui sera compté comme les journées d'un mercenaire.
\VS{51}S'il y a encore plusieurs années, il restituera le prix de son achat à raison de ces années, selon le prix pour lequel il a été acheté ;
\VS{52}et s'il reste peu d'années jusqu'à l'année du jubilé, il comptera avec lui, et restituera le prix de son achat à raison des années qu'il a servi.
\VS{53}Il aura été avec lui comme un mercenaire qui se loue d'année en année, et cet étranger ne dominera point sur lui avec dureté en ta présence.
\VS{54}S'il n'est pas racheté par quelqu'un de ces moyens, il sortira l'année du jubilé, lui et ses fils avec lui.
\VS{55}Car c'est de moi que les enfants d'Israël sont esclaves ; ce sont mes esclaves que j'ai fait sortir du pays d'Egypte. Je suis Yahweh, votre Dieu.
\Chap{26}
\TextTitle{Mise en garde contre le péché}
\VerseOne{}Vous ne vous ferez point d'idoles, vous ne vous dresserez point d'image taillée, ni de statue, et vous ne mettrez point de pierre sculptée dans votre pays, pour vous prosterner devant elles ; car je suis Yahweh, votre Dieu.
\VS{2}Vous garderez mes sabbats et vous révérerez mon sanctuaire. Je suis Yahweh.
\TextTitle{La bénédiction conditionnelle à l'obéissance à Yahweh}
\VS{3}Si vous marchez dans mes ordonnances et si vous gardez mes commandements et les pratiquez,
\VS{4}je vous donnerai les pluies en leur temps, la terre donnera ses produits, et les arbres des champs donneront leurs fruits.
\VS{5}Le foulage des grains atteindra la vendange chez vous, et la vendange atteindra les semailles ; vous mangerez votre pain à satiété et vous habiterez en sécurité dans votre pays.
\VS{6}Je donnerai la paix au pays, vous dormirez sans que personne ne vous trouble ; je ferai disparaître les bêtes méchantes du pays, et l'épée ne passera point par votre pays.
\VS{7}Vous poursuivrez vos ennemis, et ils tomberont par l'épée devant vous.
\VS{8}Cinq d'entre vous en poursuivront cent, et cent en poursuivront dix mille, et vos ennemis tomberont par l'épée devant vous.
\VS{9}Je me tournerai vers vous, je vous ferai fructifier et multiplier, et j'établirai mon alliance avec vous.
\VS{10}Vous mangerez de vieilles provisions, et vous sortirez le vieux pour y loger le nouveau.
\VS{11}Même, je mettrai mon tabernacle au milieu de vous, et mon âme ne vous aura point en horreur.
\VS{12}Mais je marcherai au milieu de vous, je serai votre Dieu, et vous serez mon peuple.
\VS{13}Je suis Yahweh, votre Dieu, qui vous ai fait sortir du pays d'Egypte, afin que vous ne soyez point leurs esclaves ; j'ai brisé les liens de votre joug, et je vous ai fait marcher la tête levée.
\TextTitle{Les châtiments en cas de désobéissance à Yahweh}
\VS{14}Mais si vous ne m'écoutez point et que vous ne pratiquez pas tous ces commandements,
\VS{15}et si vous rejetez mes ordonnances, et que votre âme a en horreur mes jugements, afin de ne point pratiquer tous mes commandements, et que vous rompiez mon alliance,
\TextTitle{La domination par les ennemis}
\VS{16}aussi je vous ferai ceci : Je répandrai sur vous la frayeur, la langueur et l'ardeur, qui vous consumerons les yeux et feront languir votre âme ; et vous sèmerez en vain votre semence car vos ennemis la mangeront.
\VS{17}Je tournerai ma face contre vous, vous serez battus devant vos ennemis ; ceux qui vous haïssent domineront sur vous ; et vous fuirez sans que personne ne vous poursuive.
\TextTitle{Le manque de fertilité de la terre}
\VS{18}Si après ces choses vous ne m'écoutez point, je vous châtierai sept fois plus à cause de vos péchés.
\VS{19}Je briserai l'orgueil de votre force et je ferai que votre ciel soit pour vous comme du fer, et votre terre comme de l'airain.
\VS{20}Votre force se consumera en vain, votre terre ne donnera point ses produits, et les arbres de la terre ne donneront point leurs fruits.
\TextTitle{Les attaques des bêtes des champs}
\VS{21}Si vous marchez en opposition avec moi et que vous ne voulez point m'écouter, je vous frapperai sept fois plus, selon vos péchés.
\VS{22}J'enverrai contre vous les bêtes des champs, qui vous priveront de vos enfants, qui détruiront votre bétail, et vous réduiront à un petit nombre, et vos chemins seront déserts.
\TextTitle{La peste}
\VS{23}Si après ces choses, vous ne recevez pas ma correction, et que vous marchiez en opposition avec moi,
\VS{24}je marcherai aussi en opposition avec vous, et je vous frapperai sept fois plus, selon vos péchés.
\VS{25}Et je ferai venir sur vous l'épée qui fera la vengeance de mon alliance ; et quand vous vous rassemblerez dans vos villes, j'enverrai la peste au milieu de vous, et vous serez livrés entre les mains de l'ennemi.
\TextTitle{Le manque de nourriture}
\VS{26}Lorsque je vous briserai le bâton du pain, dix femmes cuiront votre pain dans un seul four, et vous rendront votre pain au poids ; vous en mangerez, et vous n'en serez point rassasiés.
\VS{27}Si avec cela vous ne m'écoutez point, et que vous marchiez en opposition avec moi,
\VS{28}je marcherai aussi en opposition avec vous, avec fureur, et je vous châtierai aussi sept fois plus, selon vos péchés ;
\VS{29}vous mangerez la chair de vos fils, et vous mangerez aussi la chair de vos filles\FTNT{La. 4:10.}.
\VS{30}Je détruirai vos hauts lieux, j'abattrai vos statues consacrées au soleil, je mettrai vos cadavres sur les cadavres de vos idoles, et mon âme vous aura en horreur.
\VS{31}Je réduirai vos villes en désert, je dévasterai vos sanctuaires, et je ne respirerai plus l'agréable odeur de vos parfums.
\TextTitle{La dispersion dans les nations\FTNTT{De. 28:58-67.}}
\VS{32}Je dévasterai le pays, et vos ennemis qui l'habiteront, en seront étonnés.
\VS{33}Je vous disperserai parmi les nations, et je tirerai l'épée après vous, et votre pays sera dévasté, et vos villes désertes.
\VS{34}Alors la terre prendra plaisir à ses sabbats\FTNT{2 Ch. 36:21.}, tout le temps qu'elle sera dévastée, et lorsque vous serez dans le pays de vos ennemis, la terre se reposera et prendra plaisir à ses sabbats.
\VS{35}Tout le temps qu'elle sera dévastée, elle se reposera parce qu'elle ne s'était point reposée dans vos sabbats, lorsque vous y habitiez.
\VS{36}Et quant à ceux d'entre vous qui survivront dans le pays de leurs ennemis, je rendrai leur cœur lâche, de sorte que le bruit d'une feuille agitée les poursuivra, ils fuiront comme on fuit devant l'épée, et ils tomberont sans que personne ne les poursuive.
\VS{37}Et ils trébucheront les uns sur les autres comme devant l'épée, sans que personne ne les poursuive ; et vous ne tiendrez point devant vos ennemis ;
\VS{38}vous périrez parmi les nations, et le pays de vos ennemis vous consumera.
\VS{39}Et ceux d'entre vous qui survivront, se fondront à cause de leurs iniquités, dans les pays de vos ennemis ; ils se fondront aussi à cause des iniquités de leurs pères.
\TextTitle{Repentance et restauration de l'alliance d'Abraham, d'Isaac et de Jacob}
\VS{40}Alors, ils confesseront leurs iniquités et les iniquités de leurs pères, selon les transgressions qu'ils auront commises contre moi ; et aussi parce qu'ils auront marché en opposition avec moi.
\VS{41}Moi aussi, je marcherai en opposition avec eux, je les amènerai dans le pays de leurs ennemis. Et alors, leur cœur incirconcis s'humiliera, et ils recevront la peine de leur iniquité.
\VS{42}Et alors je me souviendrai de mon alliance avec Jacob, et de mon alliance avec Isaac, et je me souviendrai aussi de mon alliance avec Abraham, et je me souviendrai de la terre.
\VS{43}Quand donc la terre sera abandonnée par eux, et prendra plaisir à ses sabbats, ayant été désolée à cause d'eux ; et qu'ils recevront la peine de leur iniquité, parce qu'ils ont rejeté mes ordonnances, et que leur âme a eu mes ordonnances en horreur.
\VS{44}Je m'en souviendrai, dis-je, lorsqu'ils seront dans le pays de leurs ennemis, je ne les rejetterai point, et je ne les aurai point en horreur pour les consumer entièrement jusqu'à rompre mon alliance avec eux ; car je suis Yahweh, leur Dieu.
\VS{45}Je me souviendrai en leur faveur de la Première Alliance, par laquelle je les ai fait sortir du pays d'Egypte, aux yeux des nations, pour être leur Dieu. Je suis Yahweh.
\VS{46}Ce sont là les statuts, les ordonnances, et les lois que Yahweh établit entre lui et les enfants d'Israël sur la montagne de Sinaï, par Moïse.
\Chap{27}
\TextTitle{Lois des personnes et des biens voués à Yahweh}
\VerseOne{}Yahweh parla aussi à Moïse, en disant :
\VS{2}Parle aux enfants d'Israël, et dis-leur : quand quelqu'un aura fait un vœu important, les personnes vouées à Yahweh seront mises à ton estimation.
\VS{3}Et l'estimation que tu feras d'un homme, depuis l'âge de vingt ans jusqu'à l'âge de soixante ans, sera du prix de cinquante sicles d'argent, selon le sicle du sanctuaire.
\VS{4}Mais si c'est une femme, alors ton estimation sera de trente sicles.
\VS{5}Si c'est un homme de cinq ans jusqu'à vingt ans, alors ton estimation sera de vingt sicles ; et quant à la femme, de dix sicles.
\VS{6}Et si c'est un homme d'un mois jusqu'à cinq ans, ton estimation sera de cinq sicles d'argent ; et l'estimation d'une femme sera de trois sicles d'argent.
\VS{7}Et lorsque c'est un homme de soixante ans et au-dessus, ton estimation sera de quinze sicles ; et si c'est une femme, de dix sicles.
\VS{8}Et si celui qui a fait le vœu est plus pauvre que ton estimation, on le présentera devant le prêtre, qui en fera l'estimation, et le prêtre fera l'estimation selon les ressources de celui qui a fait le vœu.
\VS{9}Si c'est d'une des bêtes que l'on présente en offrande à Yahweh, tout ce qu'on donnera à Yahweh de la sorte sera saint.
\VS{10}On ne la changera point, et on n'en mettra point une autre à la place, d'une bonne pour une mauvaise, ou une mauvaise pour une bonne ; si l'on remplace une bête par une autre bête, elles seront l'une et l'autre chose sainte.
\VS{11}Si c'est d'une bête impure, qu'on ne peut présenter en offrande à Yahweh, on présentera la bête devant le prêtre,
\VS{12}qui en fera l'évaluation selon qu'elle sera bonne ou mauvaise, et il en sera fait ainsi, selon l'estimation du prêtre.
\VS{13}Mais si on veut la racheter, on ajoutera un cinquième à ton estimation.
\VS{14}Et quand quelqu'un sanctifiera sa maison pour être sainte à Yahweh, le prêtre l'estimera selon qu'elle sera bonne ou mauvaise, et on se tiendra à l'estimation que le prêtre en aura faite.
\VS{15}Mais si celui qui l'a sanctifiée veut racheter sa maison, il ajoutera par-dessus un cinquième de l'argent de ton estimation, et elle lui appartiendra.
\VS{16}Et si l'homme sanctifie à Yahweh une partie du champ de sa possession, ton estimation sera selon ce qu'on y sème, le homer de semence d'orge à cinquante sicles d'argent.
\VS{17}S'il a sanctifié son champ dès l'année du jubilé, on s'en tiendra à ton estimation ;
\VS{18}mais s'il sanctifie son champ après le jubilé, le prêtre estimera l'argent selon le nombre des années qui restent jusqu'à l'année du jubilé, et il sera fait une réduction sur ton estimation.
\VS{19}Et si celui qui a sanctifié le champ veut le racheter en quelque sorte que ce soit, il ajoutera par-dessus un cinquième de l'argent de ton estimation, et il lui restera.
\VS{20}Mais s'il ne rachète point le champ, et que le champ se vende à un autre homme, il ne se rachètera plus.
\VS{21}Et ce champ-là ayant passé le jubilé sera consacré à Yahweh, comme un champ d'interdit, la possession en sera au prêtre.
\VS{22}Et s'il sanctifie à Yahweh un champ qu'il ait acheté, qui ne soit point des champs de sa possession,
\VS{23}le prêtre lui comptera la somme de ton estimation jusqu'à l'année du jubilé, et il donnera en ce jour-là ton estimation, afin que ce soit une chose consacrée à Yahweh.
\VS{24}Mais l'année du jubilé, le champ retournera à celui de qui il avait été acheté, et auquel était la possession de la terre.
\VS{25}Et toute estimation que tu auras faite, sera selon le sicle du sanctuaire : Le sicle est de vingt guéras.
\TextTitle{Consécration des premiers-nés du bétail}
\VS{26}Toutefois, nul ne pourra consacrer le premier-né d'entre les bêtes, car il appartient à Yahweh par droit de primogéniture, soit de bœuf, soit d'agneau, il est à Yahweh.
\VS{27}Mais s'il s'agit d'une bête impure, il le rachètera selon ton estimation, et il ajoutera à ton estimation un cinquième ; et s'il n'est point racheté, il sera vendu selon ton estimation.
\TextTitle{Consécration des choses et personnes dévouées par interdit à Yahweh}
\VS{28}Or toute chose dévouée que quelqu'un dévouera à la façon de l'interdit à Yahweh, de tout ce qui est sien, soit homme, ou bête, ou champ de sa possession, ne se revendra ni ne se rachètera ; toute chose dévouée sera entièrement consacrée à Yahweh.
\VS{29}Nul interdit dévoué par interdit d'entre les hommes ne pourra être racheté, mais on le fera mettre à mort.
\TextTitle{Consécration de la dîme de la terre et du bétail}
\VS{30}Toute dîme de la terre, tant du grain de la terre que du fruit des arbres, est à Yahweh ; c'est une chose consacrée à Yahweh.
\VS{31}Mais si quelqu'un veut racheter en quelque sorte que ce soit quelque chose de sa dîme, il y ajoutera un cinquième par-dessus.
\VS{32}Mais toute dîme de bœufs, de brebis et de chèvres, à savoir tout ce qui passe sous la verge, le dixième en sera consacrée à Yahweh.
\VS{33}On ne choisira point le bon ou le mauvais, et l'on ne fera point d'échange ; si on l'échange, la bête changée et l'autre seront consacrées, et ne seront point rachetées.
\VS{34}Ce sont là les commandements que Yahweh donna à Moïse sur la montagne de Sinaï, pour les enfants d'Israël.
\PPE{}
\end{multicols}

%\clearpage\ShortTitle{No.}\BookTitle{Nombres}\BFont
\noindent\hrulefill
{\footnotesize
\textit{
\bigskip
{\centering{}
\\Auteur~: Probablement Moïse
\\(Heb.~: Bamidbar)
\\Signification~: Dans le désert
\\Thème~: Pérégrination dans le désert
\\Date de rédaction~: Env. 1450-1410 av. J.-C.\\}
}
\textit{
\\Ce livre commence par le recensement des fils d'Israël et relate trente-huit des quarante années qu'ils passèrent dans le désert du Sinaï. Il couvre une période qui s'étend de la deuxième année après la sortie d'Egypte à la veille de l'entrée en Canaan, terre que Dieu avait promis de donner à la descendance d'Abraham. Ce pays où coulaient le lait et le miel s'étendait de Sidon jusqu'à Lesha, en passant par Gaza et Sodome. En plus des Cananéens, il accueillait en son sein des enfants d'Anak, les Amalécites, les Hétiens, les Jébusiens et les Amoréens.
\\Ces écrits retracent les premières victoires d'Israël et regroupent diverses lois et instructions sur le partage de la terre promise. Ils témoignent également de la révolte et de l'incrédulité de la génération sortie d'Egypte dont la quasi-totalité périt dans le désert.\bigskip
}
}
\par\nobreak\noindent\hrulefill
\begin{multicols}{2}
\Chap{1}
\TextTitle{Dénombrement des hommes de guerre}
\VerseOne{}Or Yahweh parla à Moïse dans le désert de Sinaï, dans la tente d'assignation, le premier jour du second mois, la seconde année, après qu'ils furent sortis du pays d'Egypte, en disant~:
\VS{2}Faites le dénombrement de toute l'assemblée des fils d'Israël, selon leurs familles, selon les maisons de leurs pères, en comptant nom par nom, savoir tous les mâles\FTNT{Ex. 30:12~; Ex. 38:26.}, chacun par tête~;
\VS{3}depuis l'âge de vingt ans et au-dessus, tous ceux d'Israël qui peuvent aller à la guerre, vous les compterez selon leurs armées, toi et Aaron.
\VS{4}Il y aura avec vous un homme par tribu, celui qui est le chef de la maison de ses pères.
\VS{5}Voici les noms des hommes qui vous assisteront. Pour la tribu de Ruben~: Elitsur, fils de Schedéur~;
\VS{6}pour celle de Siméon~: Schelumiel, fils de Tsurischaddaï~;
\VS{7}pour celle de Juda~: Nachschon, fils d'Amminadab~;
\VS{8}pour celle d'Issacar~: Nethaneel, fils de Tsuar~;
\VS{9}pour celle de Zabulon~: Eliab, fils de Hélon~;
\VS{10}pour les fils de Joseph, pour la tribu d'Ephraïm~: Elischama, fils d'Ammihud~; pour celle de Manassé~: Gamliel, fils de Pedahtsur~;
\VS{11}pour la tribu de Benjamin~: Abidan, fils de Guideoni~;
\VS{12}pour celle de Dan~: Ahiézer, fils d'Ammischaddaï~;
\VS{13}pour celle d'Aser~: Paguiel, fils d'Ocran~;
\VS{14}pour celle de Gad~: Eliasaph, fils de Déuel~;
\VS{15}pour celle de Nephthali~: Ahira, fils d'Enan.
\VS{16}C'étaient là ceux qu'on appelait pour tenir l'assemblée~; ils étaient les princes des tribus de leurs pères, chefs des milliers d'Israël.
\VS{17}Alors Moïse et Aaron prirent ces hommes qui avaient été désignés par leurs noms,
\VS{18}et ils convoquèrent toute l'assemblée, le premier jour du second mois. On les enregistra selon leurs familles et selon la maison de leurs pères, en comptant les noms depuis l'âge de vingt ans et au-dessus, chacun par tête.
\VS{19}Comme Yahweh l'avait commandé à Moïse, il les dénombra au désert de Sinaï.
\VS{20}Les fils donc de Ruben, premier-né d'Israël, selon leurs générations, leurs familles, et les maisons de leurs pères, dont on fit le dénombrement par leur nom, et par tête, savoir tous les mâles de l'âge de vingt ans, et au dessus, tous ceux qui pouvaient aller à la guerre.
\VS{21}Ceux, dis-je, de la tribu de Ruben, qui furent dénombrés, furent quarante-six mille cinq cents.
\VS{22}Des enfants de Siméon, selon leurs générations, leurs familles, et les maisons de leurs pères, ceux qui furent dénombrés par leur nom et par tête, savoir tous les mâles de l'âge de vingt ans, et au dessus, tous ceux qui pouvaient aller à la guerre~;
\VS{23}ceux, dis-je, de la tribu de Siméon, qui furent dénombrés, furent cinquante-neuf mille trois cents.
\VS{24}Des fils de Gad, selon leurs générations, leurs familles, et les maisons de leurs pères, dénombrés chacun par leur nom, depuis l'âge de vingt ans, et au dessus, tous ceux qui pouvaient aller à la guerre~;
\VS{25}ceux, dis-je, de la tribu de Gad, qui furent dénombrés, furent quarante-cinq mille six cent cinquante.
\VS{26}Des enfants de Juda, selon leurs générations, leurs familles, et les maisons de leurs pères, dénombrés chacun par leur nom, depuis l'âge de vingt ans, et au dessus, tous ceux qui pouvaient aller à la guerre~;
\VS{27}ceux, dis-je, de la tribu de Juda, qui furent dénombrés, furent soixante-quatorze mille six cents.
\VS{28}Des fils d'Issacar, selon leurs générations, leurs familles, et les maisons de leurs pères, dénombrés chacun par leur nom, depuis l'âge de vingt ans, et au dessus, tous ceux qui pouvaient aller à la guerre~;
\VS{29}ceux, dis-je, de la tribu d'Issacar, qui furent dénombrés, furent cinquante-quatre mille quatre cents.
\VS{30}Des enfants de Zabulon, selon leurs générations, leurs familles, et les maisons de leurs pères, dénombrés chacun par leur nom, depuis l'âge de vingt ans, et au dessus, tous ceux qui pouvaient aller à la guerre~;
\VS{31}ceux, dis-je, de la tribu de Zabulon, qui furent dénombrés, furent cinquante-sept mille quatre cents.
\VS{32}Quant aux fils de Joseph~; les fils d'Ephraïm, selon leurs générations, leurs familles, et les maisons de leurs pères, dénombrés chacun par leur nom, depuis l'âge de vingt ans, et au dessus, tous ceux qui pouvaient aller à la guerre~;
\VS{33}ceux, dis-je, de la tribu d'Ephraïm, qui furent dénombrés, furent quarante mille cinq cents.
\VS{34}Des fils de Manassé, selon leurs générations, leurs familles, et les maisons de leurs pères, dénombrés chacun par leur nom, depuis l'âge de vingt ans, et au dessus, tous ceux qui pouvaient aller à la guerre~;
\VS{35}ceux, dis-je, de la tribu de Manassé, qui furent dénombrés, furent trente-deux mille deux cents.
\VS{36}Des fils de Benjamin, selon leurs générations, leurs familles, et les maisons de leurs pères, dénombrés chacun par leur nom, depuis l'âge de vingt ans, et au dessus, tous ceux qui pouvaient aller à la guerre~;
\VS{37}ceux, dis-je, de la tribu de Benjamin, qui furent dénombrés, furent trente-cinq mille quatre cents.
\VS{38}Des fils de Dan, selon leurs générations, leurs familles, et les maisons de leurs pères, dénombrés chacun par leur nom, depuis l'âge de vingt ans, et au dessus, tous ceux qui pouvaient aller à la guerre~;
\VS{39}ceux, dis-je, de la tribu de Dan qui furent dénombrés, furent soixante-deux mille sept cents.
\VS{40}Des fils d'Aser, selon leurs générations, leurs familles, et les maisons de leurs pères, dénombrés chacun par leur nom, depuis l'âge de vingt ans, et au dessus, tous ceux qui pouvaient aller à la guerre~;
\VS{41}ceux, dis-je, de la tribu d'Aser, qui furent dénombrés, furent quarante et un mille cinq cents.
\VS{42}Des fils de Nephthali, selon leurs générations, leurs familles, et les maisons de leurs pères, dénombrés chacun par leur nom, depuis l'âge de vingt ans, et au dessus, tous ceux qui pouvaient aller à la guerre~;
\VS{43}ceux, dis-je, de la tribu de Nephthali, qui furent dénombrés, furent cinquante-trois mille quatre cents.
\VS{44}Ce sont là ceux dont Moïse et Aaron firent le dénombrement, les douze princes d'entre les enfants d'Israël y étant, un pour chaque maison de leurs pères.
\VS{45}Ainsi tous ceux des enfants d'Israël, dont on fit le dénombrement, selon les maisons de leurs pères, depuis l'âge de vingt ans, et au dessus, tous ceux d'entre les Israélites, qui pouvaient aller à la guerre~;
\VS{46}tous ceux, dis-je, dont on fit le dénombrement, furent six cent trois mille cinq cent cinquante.
\VS{47}Mais les Lévites ne furent point dénombrés avec eux, selon la tribu de leurs pères.
\VS{48}Car Yahweh avait parlé à Moïse, en disant~:
\VS{49}Tu ne feras aucun dénombrement de la tribu de Lévi, et tu n'en lèveras point la somme avec les autres enfants d'Israël.
\VS{50}Mais tu donneras aux Lévites la charge du tabernacle du témoignage, et de tous ses ustensiles, et de tout ce qui lui appartient~; ils porteront le tabernacle, et tous ses ustensiles~; ils y serviront, et camperont autour du tabernacle.
\VS{51}Et quand le tabernacle partira, les Lévites le démonteront, et quand le tabernacle campera, les Lévites le dresseront. Que si quelque étranger en approche, on le fera mourir\FTNT{Ez. 44:8-9.}.
\VS{52}Or les enfants d'Israël camperont chacun dans son camp, et chacun sous sa bannière, selon leurs armées.
\VS{53}Mais les Lévites camperont autour du tabernacle du témoignage, afin qu'il n'y ait point d'indignation sur l'assemblée des enfants d'Israël, et ils prendront en leur charge le tabernacle du Témoignage.
\VS{54}Et les enfants d'Israël firent selon toutes les choses que Yahweh avait commandées à Moïse~; ils le firent ainsi.
\Chap{2}
\TextTitle{Disposition du camp d'Israël par tribu}
\VerseOne{}Et Yahweh parla à Moïse et à Aaron, en disant~:
\VS{2}Les enfants d'Israël camperont chacun sous sa bannière, avec les enseignes des maisons de leurs pères, tout autour de la tente d'assignation, vis-à-vis de lui.
\VS{3}Ceux de la bannière du camp de Juda camperont droit vers l'est, selon ses armées~; et Nachschon, fils d'Amminadab, sera le chef des fils de Juda~;
\VS{4}et son armée, et ses dénombrés, soixante-quatorze mille six cents.
\VS{5}Près de lui campera la tribu d'Issacar, et Nethanaël, fils de Tsuar, sera le chef des enfants d'Issacar~;
\VS{6}et son armée, et ses dénombrés, cinquante-quatre mille quatre cents.
\VS{7}Puis la tribu de Zabulon, et Eliab, fils de Hélon, sera le chef des enfants de Zabulon~;
\VS{8}et son armée, et ses dénombrés, cinquante-sept mille quatre cents.
\VS{9}Tous les dénombrés du camp de Juda, cent quatre-vingt-six mille quatre cents, selon leurs armées, partiront les premiers.
\VS{10}La bannière du camp de Ruben, selon ses armées, sera vers le sud, et Elitsur, fils de Schedéur, sera le chef des enfants de Ruben~;
\VS{11}et son armée, et ses dénombrés, quarante-six mille cinq cents.
\VS{12}Près de lui campera la tribu de Siméon, et Schelumiel, fils de Tsurischaddaï, sera le chef des enfants de Siméon~;
\VS{13}et son armée, et ses dénombrés, cinquante-neuf mille trois cents.
\VS{14}Puis la tribu de Gad, et Eliasaph, fils de Déuel, sera le chef des enfants de Gad~;
\VS{15}et son armée, et ses dénombrés, quarante-cinq mille six cent cinquante.
\VS{16}Tous les dénombrés du camp de Ruben, cent cinquante et un mille quatre cent cinquante, selon leurs armées, partiront les seconds.
\VS{17}Ensuite la tente d'assignation partira avec le camp des Lévites, au milieu des camps qui partiront comme ils auront campés, chacune en sa place, selon leurs bannières.
\VS{18}La bannière du camp d'Ephraïm, selon ses armées, sera vers l'occident~; et Elischama, fils de Ammihud, sera le chef des enfants d'Ephraïm~;
\VS{19}et son armée, et ses dénombrés, quarante mille cinq cents.
\VS{20}Près de lui campera la tribu de Manassé, et Gamliel, fils de Pedahtsur, sera le chef des fils de Manassé~;
\VS{21}et son armée, et ses dénombrés, trente-deux mille deux cents.
\VS{22}Puis la tribu de Benjamin, et Abidan, fils de Guideoni, sera le chef des fils de Benjamin~;
\VS{23}et son armée, et ses dénombrés, trente-cinq mille et quatre cents.
\VS{24}Tous les dénombrés pour le camp d'Ephraïm, cent huit mille et cent, selon leurs armées, partiront les troisièmes.
\VS{25}La bannière du camp de Dan, selon ses armées, sera vers le nord, et Ahiézer, fils de Ammischaddaaï, sera le chef des fils de Dan~;
\VS{26}et son armée, et ses dénombrés, soixante-deux mille sept cents.
\VS{27}Près de lui campera la tribu d'Aser, et Paguiel, fils de Ocran, sera le chef des fils d'Aser~;
\VS{28}et son armée, et ses dénombrés, quarante et un mille cinq cents.
\VS{29}Puis la tribu de Nephthali, et Ahira, fils d'Enan, sera le chef des fils de Nephthali~;
\VS{30}et son armée, et ses dénombrés, cinquante-trois mille quatre cents.
\VS{31}Tous les dénombrés du camp de Dan, cent cinquante-sept mille six cents, partiront les derniers des bannières.
\VS{32}Ce sont là ceux des enfants d'Israël dont on fit le dénombrement selon les maisons de leurs pères. Tous les dénombrés des camps selon leurs armées furent six cent trois mille cinq cent cinquante.
\VS{33}Mais les Lévites ne furent point dénombrés avec les autres enfants d'Israël, comme Yahweh l'avait commandé à Moïse.
\VS{34}Et les enfants d'Israël firent selon toutes les choses que Yahweh avait commandées à Moïse, et campèrent ainsi selon leurs bannières, et partirent ainsi, chacun selon leurs familles, et selon la maison de leurs pères.
\Chap{3}
\TextTitle{Organisation des prêtres et des Lévites}
\VerseOne{}Or ce sont ici les générations d'Aaron et de Moïse, au temps que Yahweh parla à Moïse sur la montagne de Sinaï.
\VS{2}Et ce sont ici les noms des fils d'Aaron~; Nadab, qui était l'aîné, Abihu, Eléazar, et Ithamar.
\VS{3}Ce sont là les noms des fils d'Aaron, les prêtres, qui furent oints et consacrés pour exercer la prêtrise\FTNT{Ex. 40:15~; Lé. 8:30.}.
\VS{4}Mais Nadab et Abihu moururent en la présence de Yahweh, quand ils offrirent un feu étranger devant Yahweh au désert de Sinaï, et ils n'eurent point d'enfants~; mais Eléazar et Ithamar exercèrent la prêtrise en la présence d'Aaron leur père\FTNT{Lé. 10:1-2~; 1 Ch. 24:2.}.
\VS{5}Yahweh parla à Moïse, en disant~:
\VS{6}Fais approcher la tribu de Lévi, et fais qu'elle se tienne devant Aaron, le prêtre, afin qu'ils le servent.
\VS{7}Et qu'ils aient la charge de ce qu'il leur ordonnera de garder, et de ce que toute l'assemblée leur ordonnera de garder, devant la tente d'assignation, en faisant le service du tabernacle.
\VS{8}Et qu'ils gardent tous les ustensiles de la tente d'assignation, et ce qui leur sera donné en charge par les enfants d'Israël, pour faire le service du tabernacle.
\VS{9}Ainsi tu donneras les Lévites à Aaron et à ses fils~; ils lui sont complètement donnés d'entre les enfants d'Israël.
\VS{10}Tu établiras donc Aaron et ses fils, et ils exerceront leur prêtrise. Que si quelque étranger en approche, on le fera mourir.
\VS{11}Et Yahweh parla à Moïse, en disant~:
\VS{12}Voici, j'ai pris les Lévites d'entre les enfants d'Israël, à la place de tout premier-né qui ouvre la matrice parmi les enfants d'Israël~; c'est pourquoi les Lévites seront à moi.
\VS{13}Car tout premier-né m'appartient, depuis le jour où je frappai tout premier-né au pays d'Egypte~; je me suis sanctifié tout premier-né en Israël, depuis les hommes jusqu'aux bêtes~; ils seront à moi, je suis Yahweh\FTNT{Ex. 13:2~; Ex. 22:29~; Ex. 34:19~; Lé. 27:26.}.
\TextTitle{Les familles des Lévites}
\VS{14}Yahweh parla aussi à Moïse au désert de Sinaï, en disant~:
\VS{15}Dénombre les enfants de Lévi, par les maisons de leurs pères, et par leurs familles, en comptant tout mâle depuis l'âge d'un mois, et au dessus.
\VS{16}Et Moïse les dénombra, selon le commandement de Yahweh, ainsi qu'il lui avait été ordonné.
\VS{17}Or ce sont ici les fils de Lévi selon leurs noms~: Guerschon, Kehath, et Merari.
\VS{18}Et ce sont ici les noms des fils de Guerschon, selon leurs familles, Libni, et Schimeï.
\VS{19}Et les fils de Kehath selon leurs familles, Amram, Jitsehar, Hébron et Uziel~;
\VS{20}et les fils de Merari, selon leurs familles, Machli et Muschi~; ce sont là les familles de Lévi, selon les maisons de leurs pères.
\VS{21}De Guerschon est sortie la famille de Libni, et la famille de Schimeï~; ce sont les familles des Guerschonites.
\VS{22}Ceux dont on fit le dénombrement, en comptant de tous les mâles depuis l'âge d'un mois et au dessus, furent au nombre de sept mille cinq cents.
\VS{23}Les familles des Guerschonites camperont derrière le tabernacle à l'occident.
\VS{24}Et Eliasaph, fils de Laël, sera le chef de la maison des pères des Guerschonites.
\TextTitle{Les fonctions des Lévites}
\VS{25}Et les fils de Guerschon auront en charge à la tente d'assignation, la tente, le tabernacle, sa couverture, le rideau de l'entrée de la tente d'assignation.
\VS{26}Et les courtines du parvis avec le rideau de l'entrée du parvis, qui servent pour tabernacle et pour l'autel, tout autour, et son cordage, pour tout son service.
\VS{27}Et de Kehath est sortie la famille des Amramites, la famille des Jitseharites, la famille des Hébronites, et la famille des Uziélites~; ce furent là les familles des Kehathites,
\VS{28}dont tous les mâles depuis l'âge d'un mois, et au dessus, furent au nombre de huit mille six cents, ayant la charge du sanctuaire.
\VS{29}Les familles des fils de Kehath camperont du côté du tabernacle vers le sud.
\VS{30}Et Elitsaphan, fils d'Uziel, sera le chef de la maison des pères des familles des Kehathites.
\VS{31}Et ils auront en charge l'arche, la table, le chandelier, les autels, et les ustensiles du sanctuaire avec lesquels on fait le service, et le rideau, avec tout ce qui y sert.
\VS{32}Et le chef des chefs des Lévites sera Eléazar, fils d'Aaron, le prêtre~; qui aura la surveillance sur ceux qui auront la charge du sanctuaire.
\VS{33}Et de Merari est sortie la famille des Machlites, et la famille des Muschi~; ce furent là les familles de Merari~;
\VS{34}ceux dont on fit le dénombrement, après le compte qui fut fait de tous les mâles, depuis l'âge d'un mois et au dessus, furent six mille deux cents.
\VS{35}Et Esuriel, fils d'Abihaïl, sera le chef de la maison des pères des familles des Merarites~; ils camperont du côté du tabernacle vers le nord.
\VS{36}Et on donnera aux enfants de Merari la surveillance des planches du tabernacle, de ses barres, de ses piliers, de ses bases, et de tous ses ustensiles, avec tout ce qui y sera~;
\VS{37}et des piliers du parvis tout autour, avec leurs bases, leurs pieux, et leurs cordes.
\VS{38}Et Moïse, et Aaron et ses fils campaient devant le tabernacle, à l'orient, devant la tente d'assignation, vers l'orient~; ils avaient la garde et le soin du sanctuaire, remis à la garde des enfants d'Israël~; et si quelque étranger en approche, on le fera mourir.
\VS{39}Tous ceux des Lévites dont on fit le dénombrement, lesquels Moïse et Aaron comptèrent par leurs familles, suivant le commandement de Yahweh, tous les mâles de l'âge d'un mois et au dessus, furent de vingt-deux mille.
\TextTitle{Le rachat des premiers-nés}
\VS{40}Yahweh dit à Moïse~: Fais le dénombrement de tous les premiers-nés mâles des enfants d'Israël, depuis l'âge d'un mois, et au dessus, et relève le nombre de leurs noms.
\VS{41}Et tu prendras pour moi, je suis Yahweh, les Lévites, à la place de tous les premiers-nés qui sont entre les enfants d'Israël~; tu prendras aussi les bêtes des Lévites, à la place de tous les premiers-nés des bêtes des enfants d'Israël.
\VS{42}Moïse fit le dénombrement, comme Yahweh lui avait commandé, de tous les premiers-nés qui étaient parmi les enfants d'Israël.
\VS{43}Et tous les premiers-nés des mâles, selon le nombre des noms, depuis l'âge d'un mois et au dessus, selon leur dénombrement, furent vingt-deux mille deux cent soixante-treize.
\VS{44}Et Yahweh parla à Moïse, en disant~:
\VS{45}Prends les Lévites à la place de tous les premiers-nés qui sont parmi les enfants d'Israël, et les bêtes des Lévites, à la place de leurs bêtes~; et les Lévites seront à moi~; je suis Yahweh.
\VS{46}Et quant à ceux qu'il faut racheter, les deux cent soixante-treize parmi les premiers-nés des fils d'Israël, qui sont de plus que les Lévites,
\VS{47}tu prendras cinq sicles par tête, tu les prendras selon le sicle du sanctuaire~; le sicle est de vingt guéras\FTNT{Ex. 30:13~; Lé. 27:6~; Lé. 27:25~; Ez. 45:12.}.
\VS{48}Et tu donneras à Aaron et à ses fils l'argent de ceux qui auront été rachetés, dépassant le nombre des Lévites.
\VS{49}Moïse donc prit l'argent du rachat de ceux qui étaient de plus, outre ceux qui avaient été rachetés par l'échange des Lévites.
\VS{50}Et il reçut l'argent des premiers-nés des enfants d'Israël, qui fut mille trois cent soixante-cinq sicles, selon le sicle du sanctuaire.
\VS{51}Et Moïse donna l'argent des rachetés à Aaron, et à ses fils, selon le commandement de Yahweh, ainsi que Yahweh le lui avait commandé.
\Chap{4}
\TextTitle{Les fonctions des fils de Kehath}
\VerseOne{}Et Yahweh parla à Moïse et à Aaron, en disant~:
\VS{2}Faites le dénombrement des fils de Kehath d'entre les enfants de Lévi par leurs familles, et par les maisons de leurs pères,
\VS{3}depuis l'âge de trente ans et au dessus, jusqu'à l'âge de cinquante ans, tous ceux qui entrent en rang, pour s'employer à la tente d'assignation.
\VS{4}C'est ici le service des fils de Kehath à la tente d'assignation, c'est-à-dire, le Saint des saints.
\VS{5}Quand le camp partira, Aaron et ses fils viendront démonter le voile\FTNT{Le voile intérieur est l'image du corps humain de Christ (Mt. 26:26). Ce voile fut déchiré de haut en bas lorsque le Seigneur est mort sur la croix (Mt. 27:50-51). Désormais, le croyant peut pénétrer dans la présence du Père (Hé. 10:19-20).} qui sert de rideau, et en couvriront l'arche du témoignage~;
\VS{6}puis ils mettront au dessus une couverture de peaux de taissons, ils étendront par dessus un drap de pourpre, et ils y mettront ses barres.
\VS{7}Et ils étendront un drap de pourpre sur la table des pains de proposition, et mettront sur elle les plats, les tasses, les bassins, et les calices de libations. Le pain continuel sera sur elle.
\VS{8}Ils étendront au dessus un drap teint de cramoisi, ils le couvriront d'une couverture de peaux de taissons, et ils y mettront ses barres.
\VS{9}Et ils prendront un drap de pourpre, en couvriront le chandelier du luminaire avec ses lampes, ses mouchettes, ses vases à cendre, et tous ses vases à huile, dont on fait usage pour son service\FTNT{Ex. 25:30-38.}~;
\VS{10}ils le mettront avec tous ses ustensiles, dans une couverture de peaux de taissons, et le mettront sur une perche.
\VS{11}Ils étendront sur l'autel d'or un drap de pourpre, ils le couvriront d'une couverture de peaux de taissons, et ils y mettront ses barres.
\VS{12}Ils prendront aussi tous les ustensiles du service dont on se sert dans le lieu saint, ils les mettront dans un drap de pourpre, et ils les couvriront d'une couverture de peaux de taissons, et les mettront sur des perches.
\VS{13}Ils ôteront les cendres de l'autel, et étendront dessus un drap de pourpre.
\VS{14}Et ils mettront dessus les ustensiles dont on se sert pour l'autel, les brasiers, les fourchettes, les pelles, les bassins, et tous les ustensiles de l'autel~; ils étendront dessus une couverture de peaux de taissons, et ils y mettront ses barres.
\VS{15}Le camp partira après qu'Aaron et ses fils auront achevé de couvrir le lieu saint et tous ses ustensiles, et après cela les fils de Kehath viendront pour le porter, et ils ne toucheront point les choses saintes, de peur qu'ils ne meurent~; c'est là ce que les fils de Kehath porteront de la tente d'assignation.
\TextTitle{Les fonctions d'Eléazar}
\VS{16}Et Eléazar fils d'Aaron, le prêtre, aura la surveillance de l'huile du luminaire, du parfum odoriférant, de l'offrande continuelle, et de l'huile de l'onction~; la charge de tout le tabernacle, et de toutes les choses qui sont dans le lieu saint, et de ses ustensiles\FTNT{Ex. 30:23-35.}.
\VS{17}Yahweh parla à Moïse et à Aaron, en disant~:
\VS{18}Ne retranchez pas la tribu des familles des Kehathites d'entre les Lévites.
\VS{19}Mais faites ceci pour eux, afin qu'ils vivent et ne meurent point~; c'est que quand ils approcheront du Saint des saints, Aaron et ses fils viendront, qui les placeront chacun à son service, et à sa charge.
\VS{20}Et ils n'entreront point pour regarder quand on enveloppera les choses saintes, afin qu'ils ne meurent point.
\TextTitle{Les fonctions des fils de Guerschon}
\VS{21}Yahweh parla à Moïse, en disant~:
\VS{22}Fais aussi le dénombrement des fils de Guerschon selon les maisons de leurs pères, et selon leurs familles~;
\VS{23}depuis l'âge de trente ans, et au dessus, jusqu'à l'âge de cinquante ans, dénombrant tous ceux qui entrent pour tenir leur rang, afin de s'employer à servir à la tente d'assignation.
\VS{24}C'est ici le service des familles des Guerschonites, ce à quoi, ils doivent servir et en ce qu'ils doivent porter.
\VS{25}Ils porteront donc les tapis du tabernacle, et la tente d'assignation, sa couverture, la couverture de peaux de taissons qui est sur lui par dessus, et le rideau de l'entrée de la tente d'assignation~;
\VS{26}les courtines du parvis, et le rideau de l'entrée de la porte du parvis, qui servent pour le tabernacle et pour l'autel tout autour, leurs cordages, et tous les ustensiles de leur service, et tout ce qui est fait pour eux~; c'est ce en quoi ils serviront.
\VS{27}Tout le service des fils de Guerschonites en tout ce qu'ils doivent porter, et en tout ce à quoi ils doivent servir, sera réglé par les ordres d'Aaron et de ses fils, et vous les chargerez d'observer tout ce qu'ils doivent porter.
\VS{28}C'est là le service des familles des fils des Guerschonites dans la tente d'assignation~; et leur charge sera sous la conduite d'Ithamar, fils d'Aaron, le prêtre.
\TextTitle{Les fonctions des fils de Merari}
\VS{29}Tu dénombreras aussi les fils de Mérari selon leurs familles et selon les maisons de leurs pères.
\VS{30}Tu les dénombreras depuis l'âge de trente ans et au dessus, jusqu'à l'âge de cinquante ans, tous ceux qui entrent en rang pour s'employer au service dans la tente d'assignation.
\VS{31}Or c'est ici la charge de ce qu'ils auront à porter, selon tout le service qu'ils auront à faire à la tente d'assignation, savoir les planches du tabernacle, ses barres, et ses piliers, avec ses bases\FTNT{Ex. 26:15.},
\VS{32}et les piliers du parvis tout autour, et leurs bases, leurs pieux, leurs cordages, tous leurs ustensiles, et tout ce dont on se sert en ces choses-là, et vous leur compterez, en les désignant par nom, tous les ustensiles, qu'ils auront en charge de porter, pièce par pièce.
\VS{33}C'est là le service des familles des fils de Merari, pour tout leur service à la tente d'assignation, sous la conduite d'Ithamar, fils d'Aaron, le prêtre.
\VS{34}Moïse, Aaron et les princes de l'assemblée dénombrèrent les fils des Kéhathites, selon leurs familles, et selon les maisons de leurs pères.
\VS{35}Depuis l'âge de trente ans, et au dessus, jusqu'à l'âge de cinquante ans, tous ceux qui entraient en rang pour servir à la tente d'assignation.
\VS{36}Et ceux dont on fit le dénombrement selon leurs familles, étaient deux mille sept cent cinquante.
\VS{37}Ce sont là les dénombrés des familles des Kéhathites, tous servant à la tente d'assignation, que Moïse et Aaron dénombrèrent selon le commandement que Yahweh avait fait par le moyen de Moïse.
\VS{38}Or quant aux dénombrés des fils de Guerschon selon leurs familles, et selon les maisons de leurs pères,
\VS{39}depuis l'âge de trente ans, et au dessus, jusqu'à l'âge de cinquante ans, tous ceux qui entraient en rang pour servir à la tente d'assignation,
\VS{40}ceux, dis-je, qui en furent dénombrés selon leurs familles, et selon les maisons de leurs pères, étaient deux mille six cent trente.
\VS{41}Ce sont là, les dénombrés des familles des fils de Guerschon, tous servant dans la tente d'assignation, que Moïse et Aaron dénombrèrent selon le commandement de Yahweh.
\VS{42}Et quant aux dénombrés des familles des fils de Merari, selon leurs familles, et selon les maisons de leurs pères,
\VS{43}depuis l'âge de trente ans, et au dessus, jusqu'à l'âge de cinquante ans, tous ceux qui entraient en rang, pour servir à la tente d'assignation~;
\VS{44}ceux, dis-je, qui en furent dénombrés selon leurs familles, étaient trois mille deux cents.
\VS{45}Ce sont là, les dénombrés des familles des fils de Merari, que Moïse et Aaron dénombrèrent selon le commandement que Yahweh avait fait par le moyen de Moïse.
\VS{46}Ainsi tous ces dénombrés, que Moïse, Aaron et les princes d'Israël dénombrèrent d'entre les Lévites, selon leurs familles, et selon les maisons de leurs pères~;
\VS{47}depuis l'âge de trente ans, et au dessus, jusqu'à l'âge de cinquante ans, tous ceux qui entraient en service pour s'employer en ce à quoi il fallait servir, et à ce qu'il fallait porter de la tente d'assignation.
\VS{48}Tous ceux, dis-je, qui en furent dénombrés, étaient huit mille cinq cent quatre-vingts.
\VS{49}On les dénombra selon le commandement que Yahweh en avait fait par le moyen de Moïse, chacun selon ce en quoi il avait à servir, et ce qu'il avait à porter, et la charge de chacun fut telle que Yahweh l'avait commandé à Moïse.
\Chap{5}
\TextTitle{Mise en garde contre toute souillure~; lois diverses}
\VerseOne{}Et Yahweh parla à Moïse, en disant~:
\VS{2}Ordonne aux enfants d'Israël qu'ils mettent hors du camp tout lépreux, tout homme ayant une gonorrhée, et tout homme souillé pour un mort\FTNT{Lé. 13~; Lé. 15.}.
\VS{3}Vous les mettrez dehors, tant l'homme que la femme, vous les mettrez, dis-je, hors du camp, afin qu'ils ne souillent point le camp au milieu duquel j'habite.
\VS{4}Et les enfants d'Israël firent ainsi, et les envoyèrent hors du camp, comme Yahweh l'avait dit à Moïse~; les enfants d'Israël firent ainsi.
\VS{5}Et Yahweh parla à Moïse, en disant~:
\VS{6}Parle aux enfants d'Israël~; quand un homme ou une femme aura commis un des péchés que l'homme commet en faisant un crime contre Yahweh, et qu'une telle personne en sera trouvée coupable~;
\VS{7}alors ils confesseront leur péché, qu'ils auront commis~; et le coupable restituera la somme totale de ce en quoi il aura été trouvé coupable, et il y ajoutera un cinquième par-dessus, et le donnera à celui contre qui il aura commis le délit.
\VS{8}Que si cet homme n'a personne à qui appartienne le droit de restituer pour retirer ce en quoi aura été commis le délit, cette chose-là sera restituée à Yahweh, et elle appartiendra au prêtre, outre le bélier expiatoire avec lequel on fera propitiation pour lui.
\VS{9}De même, toute offrande élevée d'entre toutes les choses sanctifiées des enfants d'Israël, qu'ils présenteront\FTNT{Ez. 44:30.} au prêtre, lui appartiendra.
\VS{10} Les choses donc que quelqu'un aura sanctifiées appartiendront au prêtre~; ce que chacun lui aura donné, lui appartiendra\FTNT{Lé. 10:12-13.}.
\VS{11}Yahweh parla à Moïse, en disant~:
\VS{12}Parle aux enfants d'Israël, et dis leur~: Si la femme de quelqu'un se détourne et lui devienne infidèle~;
\VS{13}et que quelqu'un aura couché avec elle, et l'aura connue, sans que son mari en ait rien su, mais qu'elle se soit cachée, et qu'elle se soit souillée, et qu'il n'y ait point de témoin contre elle, et qu'elle n'ait point été surprise~;
\VS{14}et que l'esprit de jalousie saisisse son mari, tellement qu'il soit jaloux de sa femme, parce qu'elle s'est souillée~; ou que l'esprit de jalousie le saisisse tellement, qu'il soit jaloux de sa femme, encore qu'elle ne se soit point souillée~;
\VS{15}cet homme-là fera venir sa femme devant le prêtre, et il apportera l'offrande de cette femme pour elle, savoir la dixième partie d'un epha de farine d'orge~; mais il ne répandra point d'huile dessus~; et il n'y mettra point d'encens~; car c'est un gâteau de jalousie, un gâteau de souvenir, pour remettre en mémoire l'iniquité\FTNT{Lé. 5:11.}.
\VS{16}Le prêtre la fera approcher et la fera tenir debout devant Yahweh.
\VS{17}Puis le prêtre prendra de l'eau sainte dans un vase de terre, et il prendra de la poussière qui sera sur le sol du tabernacle, et la mettra dans l'eau.
\VS{18}Ensuite le prêtre fera tenir debout la femme devant Yahweh, il découvrira la tête de cette femme, et lui posera sur les paumes des mains le gâteau de souvenir, le gâteau de jalousie~; le prêtre tiendra dans sa main les eaux amères, qui apportent la malédiction.
\VS{19}Et le prêtre fera jurer la femme et lui dira~: Si aucun homme n'a couché avec toi, et si étant sous la puissance de ton mari tu ne t'es point détournée et souillée, sois exempte du mal de ces eaux amères qui apportent la malédiction.
\VS{20}Mais si, étant sous la puissance de ton mari, tu t'es détournée et souillée, et si un autre homme que ton mari a couché avec toi,
\VS{21}alors le prêtre fera jurer la femme avec un serment d'imprécation et lui dira~: Que Yahweh te livre à la malédiction et à l'exécration au milieu de ton peuple, en faisant flétrir ta cuisse et enfler ton ventre,
\VS{22}et que ces eaux qui apportent la malédiction, entrent dans tes entrailles pour te faire enfler le ventre et flétrir ta cuisse~! Alors la femme répondra~: Amen~! Amen~!
\VS{23}Ensuite le prêtre écrira dans un livre ces imprécations, et les effacera avec les eaux amères.
\VS{24}Et il fera boire à la femme les eaux amères qui apportent la malédiction, et les eaux qui apportent la malédiction entreront en elle pour être amères.
\VS{25}Le prêtre donc prendra des mains de la femme le gâteau de jalousie, et l'agitera de côté et d'autre devant Yahweh, et l'offrira sur l'autel~;
\VS{26}le prêtre prendra une poignée de cette offrande comme souvenir\FTNT{Voir commentaire en Lé. 2:2.}, et il la brûlera sur l'autel. C'est après cela qu'il fera boire les eaux à la femme.
\VS{27}Et après qu'il lui aura fait boire les eaux, s'il est vrai qu'elle se soit souillée et qu'elle a été infidèle à son mari, les eaux qui apportent la malédiction entreront en elle et lui seront amères, et son ventre enflera, sa cuisse se flétrira, et cette femme sera assujettie à l'exécration du serment au milieu de son peuple.
\VS{28}Mais si la femme ne s'est point souillée, mais qu'elle soit pure, elle sera reconnue innocente et aura des enfants.
\VS{29}Telle est la loi sur la jalousie, quand la femme qui est sous la puissance de son mari se détourne et se souille,
\VS{30}ou quand un mari saisi d'un esprit de jalousie a des soupçons sur sa femme~: Le prêtre la fera tenir debout devant Yahweh et fera à l'égard de cette femme tout ce qui est ordonné par cette loi.
\VS{31}Le mari sera exempt de faute, mais cette femme portera son iniquité.
\Chap{6}
\TextTitle{Le vœu de naziréat}
\VerseOne{}Yahweh parla à Moïse, en disant~:
\VS{2}Parle aux enfants d'Israël, et dis-leur~: Lorsqu'un homme ou une femme se consacrera en faisant un vœu de naziréat pour se consacrer à Yahweh,
\VS{3}il s'abstiendra de vin et de boisson forte, il ne boira ni vinaigre fait de vin, ni vinaigre fait avec une boisson forte~; il ne boira d'aucune liqueur de raisins, et il ne mangera point de raisins, frais ou secs.
\VS{4}Durant tous les jours de son naziréat il ne mangera d'aucun fruit de la vigne, depuis les pépins jusqu'à la peau du raisin\FTNT{Jg. 13:7~; Lu. 1:15.}.
\VS{5}Le rasoir ne passera point sur sa tête durant tous les jours de son naziréat. Il sera saint jusqu'à ce que les jours pour lesquels il s'est consacré à Yahweh soient accomplis, et il laissera croître les cheveux de sa tête\FTNT{Jg. 13:5~; 1 S. 1:11.}.
\VS{6}Durant tous les jours pour lesquels il s'est consacré à Yahweh il ne s'approchera d'aucune personne morte\FTNT{Lé. 21:1-4.}~;
\VS{7}il ne se souillera point à la mort de son père, ni de sa mère, ni de son frère, ni de sa sœur, car il porte sur sa tête la consécration de son Dieu.
\VS{8}Durant tous les jours de son naziréat, il sera consacré à Yahweh.
\VS{9}Que si quelqu'un vient à mourir subitement près de lui, la tête de son naziréat sera souillée, et il rasera sa tête au jour de sa purification, il la rasera le septième jour.
\VS{10}Le huitième jour, il apportera au prêtre deux tourterelles ou deux pigeonneaux, à l'entrée de la tente d'assignation\FTNT{Lé. 1~; Lé. 12:6.}.
\VS{11}Et le prêtre en sacrifiera l'un pour le sacrifice d'expiation et l'autre en holocauste, et il fera propitiation pour lui de ce qu'il a péché à l'occasion du mort. Il sanctifiera donc ainsi sa tête en ce jour-là.
\VS{12}Et il séparera à Yahweh les jours de son naziréat, offrant un agneau d'un an pour le délit, et les premiers jours seront comptés pour rien, car son naziréat a été souillé.
\VS{13}Or c'est ici la loi du naziréen. Lorsque les jours de son naziréat seront accomplis, on le fera venir à la porte de la tente d'assignation.
\VS{14}Il présentera son offrande à Yahweh~: Un agneau d'un an et sans défaut pour l'holocauste, une brebis d'un an et sans défaut pour le sacrifice d'expiation, et un bélier sans défaut pour le sacrifice d'offrande de paix\FTNT{Voir commentaire en Lé. 3:1.}.
\VS{15}Une corbeille de pains sans levain, de gâteaux de fine farine, pétrie à l'huile, et de galettes sans levain, oints d'huile, avec leur gâteau, et leurs libations~;
\VS{16}lesquels, le prêtre offrira devant Yahweh~; il sacrifiera aussi son offrande pour le péché, et son holocauste.
\VS{17}Et il offrira le bélier en sacrifice d'offrande de paix à Yahweh, avec la corbeille des pains sans levain~; le prêtre offrira aussi son gâteau, et sa libation.
\VS{18}Et le naziréen rasera la tête de son naziréat à l'entrée de la tente d'assignation, et prendra les cheveux de la tête de son naziréat, et les mettra sur le feu qui est sous le sacrifice d'offrande de paix.
\VS{19}Et le prêtre prendra l'épaule cuite du bélier, et un gâteau sans levain de la corbeille, et une galette sans levain, et les mettra sur les paumes des mains du naziréen, après qu'il se sera fait raser son naziréat.
\VS{20}Et le prêtre les agitera de côté et d'autre devant Yahweh~: C'est une chose sainte qui appartient au prêtre, avec la poitrine agitée et l'épaule offerte par élévation. Et après cela le naziréen boira du vin\FTNT{Lé. 7:32-34~; Ex. 29:24-27.}.
\VS{21}Telle est la loi du naziréen qui aura voué à Yahweh son offrande pour son naziréat, outre ce qu'il aura encore moyen d'offrir~; il fera selon son vœu qu'il aura voué, suivant la loi de son naziréat.
\TextTitle{Aaron et ses fils bénissent Israël}
\VS{22}Yahweh parla à Moïse, en disant~:
\VS{23}Parle à Aaron et à ses fils, et dis-leur~: Vous bénirez ainsi les enfants d'Israël, en leur disant~:
\VS{24}Yahweh te bénisse, et te garde~!
\VS{25}Yahweh fasse luire sa face sur toi, et te fasse grâce\FTNT{Ps. 67:2~; Ps. 119:135.}~!
\VS{26}Yahweh tourne sa face vers toi, et te donne la paix~!
\VS{27}Ils mettront donc mon Nom sur les enfants d'Israël, et je les bénirai.
\Chap{7}
\TextTitle{Les offrandes des princes}
\VerseOne{}Or il arriva le jour que Moïse eut achevé de dresser le tabernacle, et qu'il l'eut oint et sanctifié avec tous ses ustensiles, de même que l'autel avec tous ses ustensiles, il arriva, dis-je, après qu'il les eut oints et sanctifiés~;
\VS{2}que les princes d'Israël, et les chefs des maisons de leurs pères, qui sont les princes des tribus, et qui avaient assisté à faire les dénombrements, firent leur offrande.
\VS{3}Et ils amenèrent leur offrande devant Yahweh~: Six chars couverts et douze bœufs~; chaque char pour deux des princes, et chaque bœuf pour chacun d'eux~; ils les offrirent devant le tabernacle.
\VS{4}Alors Yahweh parla à Moïse, en disant~:
\VS{5}Prends d'eux ces choses, et elles seront employées pour le service de la tente d'assignation~; et tu les donneras aux Lévites, à chacun selon ses fonctions.
\VS{6}Moïse prit donc les chars et les bœufs, et il les remit aux Lévites.
\VS{7}Il donna aux fils de Guerschon deux chars et quatre bœufs, selon leurs fonctions.
\VS{8}Mais il donna aux fils de Merari quatre chars et huit bœufs, selon leurs fonctions, sous la conduite d'Ithamar, fils d'Aaron, le prêtre.
\VS{9}Or il n'en donna point aux fils de Kehath, parce que le service du sanctuaire était de leur charge~; ils portaient ces choses saintes sur les épaules.
\VS{10}Et les princes présentèrent leur offrande pour la dédicace de l'autel, le jour où on l'oignit~; les princes, dis-je, présentèrent leur offrande devant l'autel.
\VS{11}Et Yahweh dit à Moïse~: Un des princes offrira un jour, et un autre l'autre jour, son offrande pour la dédicace de l'autel.
\VS{12}Le premier jour donc, Nachschon, fils d'Amminadab, présenta son offrande pour la tribu de Juda.
\VS{13}Il offrit un plat d'argent du poids de cent trente sicles, un bassin d'argent de soixante-dix sicles, selon le sicle du sanctuaire, tous deux pleins de fine farine pétrie à l'huile, pour l'offrande~;
\VS{14}une coupe d'or de dix sicles pleine de parfum~;
\VS{15}un jeune taureau, un bélier, un agneau d'un an, pour l'holocauste~;
\VS{16}un jeune bouc pour le sacrifice d'expiation~;
\VS{17}et pour le sacrifice d'offrande de paix, deux bœufs, cinq béliers, cinq boucs, et cinq agneaux d'un an. Telle fut l'offrande de Nachschon, fils d'Amminadab.
\VS{18}Le second jour, Nethaneel, fils de Tsuar, chef de la tribu d'Issacar, présenta son offrande.
\VS{19}Et il offrit pour son offrande un plat d'argent du poids de cent trente sicles, un bassin d'argent de soixante-dix sicles, selon le sicle du sanctuaire, tous deux pleins de fine farine pétrie à l'huile, pour l'offrande~;
\VS{20}une coupe d'or de dix sicles pleine de parfum~;
\VS{21}un jeune taureau, un bélier, un agneau d'un an, pour l'holocauste~;
\VS{22}un jeune bouc pour le sacrifice d'expiation~;
\VS{23}et pour le sacrifice d'offrande de paix, deux bœufs, cinq béliers, cinq boucs, et cinq agneaux d'un an. Telle fut l'offrande de Nethaneel, fils de Tsuar.
\VS{24}Le troisième jour, Eliab, fils de Hélon, chef des fils de Zabulon, présenta son offrande.
\VS{25}Il offrit un plat d'argent du poids de cent trente sicles, un bassin d'argent de soixante-dix sicles, selon le sicle du sanctuaire, tous deux pleins de fine farine pétrie à l'huile, pour l'offrande~;
\VS{26}une coupe d'or de dix sicles pleine de parfum~;
\VS{27}un jeune taureau, un bélier, un agneau d'un an, pour l'holocauste~;
\VS{28}un jeune bouc pour le sacrifice d'expiation~;
\VS{29}et pour le sacrifice d'offrande de paix, deux bœufs, cinq béliers, cinq boucs, et cinq agneaux d'un an. Telle fut l'offrande d'Eliab, fils de Hélon.
\VS{30}Le quatrième jour, Elitsur, fils de Schedéur, prince des fils de Ruben, présenta son offrande.
\VS{31}Il offrit un plat d'argent du poids de cent trente sicles, un bassin d'argent de soixante-dix sicles, selon le sicle du sanctuaire, tous deux pleins de fine farine pétrie à l'huile, pour l'offrande~;
\VS{32}une coupe d'or de dix sicles pleine de parfum~;
\VS{33}un jeune taureau, un bélier, un agneau d'un an, pour l'holocauste~;
\VS{34}un jeune bouc pour le sacrifice d'expiation~;
\VS{35}et pour le sacrifice d'offrande de paix, deux bœufs, cinq béliers, cinq boucs, et cinq agneaux d'un an. Telle fut l'offrande d'Elitsur, fils de Schedéur.
\VS{36}Le cinquième jour, Schelumiel, fils de Tsurischaddaï, prince des fils de Siméon, présenta son offrande.
\VS{37}Il offrit un plat d'argent du poids de cent trente sicles, un bassin d'argent de soixante-dix sicles, selon le sicle du sanctuaire, tous deux pleins de fine farine pétrie à l'huile pour l'offrande~;
\VS{38}une coupe d'or de dix sicles pleine de parfum~;
\VS{39}un jeune taureau, un bélier, un agneau d'un an, pour l'holocauste~;
\VS{40}un jeune bouc pour le sacrifice d'expiation~;
\VS{41}et pour le sacrifice d'offrande de paix, deux bœufs, cinq béliers, cinq boucs, et cinq agneaux d'un an. Telle fut l'offrande de Schelumiel, fils de Tsurischaddaï.
\VS{42}Le sixième jour, Eliasaph, fils de Déuel, prince des fils de Gad, présenta son offrande.
\VS{43}Il offrit un plat d'argent du poids de cent trente sicles, un bassin d'argent de soixante-dix sicles, selon le sicle du sanctuaire, tous deux pleins de fine farine pétrie à l'huile pour l'offrande~;
\VS{44}une coupe d'or de dix sicles pleine de parfum~;
\VS{45}un jeune taureau, un bélier, un agneau d'un an, pour l'holocauste~;
\VS{46}un jeune bouc pour le sacrifice d'expiation~;
\VS{47}et pour le sacrifice d'offrande de paix, deux bœufs, cinq béliers, cinq boucs, et cinq agneaux d'un an. Telle fut l'offrande d'Eliasaph, fils de Déuel.
\VS{48}Le septième jour, Elischama, fils d'Ammihud, prince des fils d'Ephraïm, présenta son offrande.
\VS{49}Il offrit un plat d'argent, du poids de cent trente sicles, un bassin d'argent de soixante-dix sicles, selon le sicle du sanctuaire, tous deux pleins de fine farine pétrie à l'huile pour l'offrande~;
\VS{50}une coupe d'or de dix sicles pleine de parfum~;
\VS{51}un jeune taureau, un bélier, un agneau d'un an, pour l'holocauste~;
\VS{52}un jeune bouc pour le sacrifice d'expiation~;
\VS{53}et pour le sacrifice d'offrande de paix, deux bœufs, cinq béliers, cinq boucs, et cinq agneaux d'un an. Telle fut l'offrande d'Elischama, fils d'Ammihud.
\VS{54}Le huitième jour, Gamliel, fils de Pedahtsur, prince des fils de Manassé, présenta son offrande.
\VS{55}Il offrit un plat d'argent, du poids de cent trente sicles, un bassin d'argent de soixante-dix sicles, selon le sicle du sanctuaire, tous deux pleins de fine farine pétrie à l'huile pour l'offrande~;
\VS{56}une coupe d'or de dix sicles pleine de parfum~;
\VS{57}un jeune taureau, un bélier, un agneau d'un an, pour l'holocauste~;
\VS{58}un jeune bouc pour le sacrifice d'expiation~;
\VS{59}et pour le sacrifice d'offrande de paix, deux bœufs, cinq béliers, cinq boucs, et cinq agneaux d'un an. Telle fut l'offrande de Gamliel, fils de Pedahtsur.
\VS{60}Le neuvième jour, Abidan, fils de Guideoni, prince des fils de Benjamin, présenta son offrande.
\VS{61}Il offrit un plat d'argent, du poids de cent trente sicles, un bassin d'argent de soixante-dix sicles, selon le sicle du sanctuaire, tous deux pleins de fine farine pétrie à l'huile pour l'offrande~;
\VS{62}une coupe d'or de dix sicles pleine de parfum~;
\VS{63}un jeune taureau, un bélier, un agneau d'un an, pour l'holocauste~;
\VS{64}un jeune bouc pour le sacrifice d'expiation~;
\VS{65}et pour le sacrifice d'offrande de paix, deux bœufs, cinq béliers, cinq boucs, et cinq agneaux d'un an. Telle fut l'offrande d'Abidan, fils de Guideoni.
\VS{66}Le dixième jour, Ahiézer, fils d'Ammischaddaï, prince des fils de Dan, présenta son offrande.
\VS{67}Il offrit un plat d'argent du poids de cent trente sicles, un bassin d'argent de soixante-dix sicles, selon le sicle du sanctuaire, tous deux pleins de fine farine pétrie à l'huile pour l'offrande~;
\VS{68}une coupe d'or de dix sicles pleine de parfum~;
\VS{69}Un jeune taureau, un bélier, un agneau d'un an, pour l'holocauste~;
\VS{70}un jeune bouc pour le sacrifice d'expiation~;
\VS{71}et pour le sacrifice d'offrande de paix, deux bœufs, cinq béliers, cinq boucs, et cinq agneaux d'un an. Telle fut l'offrande d'Ahiézer, fils d'Ammischaddaï.
\VS{72}Le onzième jour, Paguiel, fils d'Ocran, prince des fils d'Aser, présenta son offrande.
\VS{73}Il offrit un plat d'argent, du poids de cent trente sicles, un bassin d'argent de soixante-dix sicles, selon le sicle du sanctuaire, tous deux pleins de fine farine pétrie à l'huile pour l'offrande~;
\VS{74}une coupe d'or de dix sicles pleine de parfum~;
\VS{75}un jeune taureau, un bélier, un agneau d'un an, pour l'holocauste~;
\VS{76}un jeune bouc pour le sacrifice d'expiation~;
\VS{77}et pour le sacrifice d'offrande de paix, deux bœufs, cinq béliers, cinq boucs, et cinq agneaux d'un an. Telle fut l'offrande de Paguiel, fils d'Ocran.
\VS{78}Le douzième jour, Ahira, fils d'Enan, prince des fils de Nephthali, présenta son offrande.
\VS{79}Il offrit un plat d'argent du poids de cent trente sicles, un bassin d'argent de soixante-dix sicles, selon le sicle du sanctuaire, tous deux pleins de fine farine pétrie à l'huile pour l'offrande~;
\VS{80}une coupe d'or de dix sicles pleine de parfum~;
\VS{81}un jeune taureau, un bélier, un agneau d'un an, pour l'holocauste~;
\VS{82}un jeune bouc pour le sacrifice d'expiation~;
\VS{83}et pour le sacrifice d'offrande de paix, deux bœufs, cinq béliers, cinq boucs, et cinq agneaux d'un an. Telle fut l'offrande d'Ahira, fils d'Enan.
\TextTitle{Les dons des princes}
\VS{84}Telle fut la dédicace de l'autel, qui fut faite par les princes d'Israël, lorsqu'il fut oint. Douze plats d'argent, douze bassins d'argent, douze tasses d'or~;
\VS{85}chaque plat d'argent était de cent trente sicles, et chaque bassin de soixante-dix~; tout l'argent de ces ustensiles montait à deux mille quatre cents sicles, selon le sicle du sanctuaire~;
\VS{86}douze coupes d'or pleines de parfum, chacune de dix sicles, selon le sicle du sanctuaire~; tout l'or des tasses montait à cent-vingt sicles.
\VS{87}Tous les animaux pour l'holocauste étaient douze veaux, douze béliers, et douze agneaux d'un an, avec leurs offrandes, et douze jeunes boucs pour le sacrifice d'expiation.
\VS{88}Tous les animaux du sacrifice d'offrande de paix étaient vingt-quatre veaux, avec soixante béliers, soixante boucs, et soixante agneaux d'un an. Telle fut donc la dédicace de l'autel, après qu'on l'eut oint.
\VS{89}Et quand Moïse entrait dans la tente d'assignation pour parler avec Yahweh, il entendait une voix qui lui parlait du haut du propitiatoire placé sur l'arche du témoignage, entre les deux chérubins. Et il lui parlait\FTNT{Ex. 25:22.}.
\Chap{8}
\TextTitle{Les lampes sur le chandelier}
\VerseOne{}Yahweh parla à Moïse, en disant~:
\VS{2}Parle à Aaron, et tu lui diras~: Quand tu allumeras les lampes, les sept lampes éclaireront sur le devant du chandelier\FTNT{Ex. 25:37.}.
\VS{3}Et Aaron fit ainsi~; il plaça les lampes pour éclairer sur le devant du chandelier, comme Yahweh l'avait commandé à Moïse.
\VS{4}Or le chandelier était fait de telle manière, qu'il était d'or battu au marteau, d'ouvrage fait au marteau, sa tige aussi, et ses fleurs. On fit ainsi le chandelier selon le modèle que Yahweh en avait fait voir à Moïse\FTNT{Ex. 25:31-40.}.
\TextTitle{Purification des Lévites}
\VS{5}Puis Yahweh parla à Moïse, en disant~:
\VS{6}Prends les Lévites du milieu des enfants d'Israël, et purifie-les.
\VS{7}Tu leur feras ainsi pour les purifier. Tu feras aspersion sur eux de l'eau de purification~; ils feront passer le rasoir sur toute leur chair, ils laveront leurs vêtements, et ils se purifieront.
\VS{8}Puis ils prendront un jeune taureau avec son offrande de gâteau de fine farine pétrie à l'huile~; et tu prendras un autre jeune taureau pour le sacrifice d'expiation.
\VS{9}Alors tu feras approcher les Lévites devant la tente d'assignation, et tu convoqueras toute l'assemblée des enfants d'Israël.
\VS{10}Tu feras, dis-je, approcher les Lévites devant Yahweh, et les enfants d'Israël poseront leurs mains sur les Lévites.
\VS{11}Et Aaron fera tourner de côté et d'autre les Lévites devant Yahweh, comme offrande de la part des enfants d'Israël, et ils seront employés au service de Yahweh.
\VS{12}Et les Lévites poseront leurs mains sur la tête des veaux~; puis tu offriras l'un en sacrifice pour l'expiation, et l'autre en holocauste à Yahweh, afin de faire propitiation pour les Lévites.
\VS{13}Après tu feras tenir les Lévites devant Aaron et devant ses fils, et tu les présenteras en offrande à Yahweh.
\VS{14}Ainsi tu sépareras les Lévites du milieu des enfants d'Israël, et les Lévites m'appartiendront.
\VS{15}Après cela, les Lévites viendront pour servir dans la tente d'assignation quand tu les auras purifiés et présentés en offrande.
\VS{16}Car ils me sont entièrement donnés du milieu des enfants d'Israël~; je les ai pris pour moi à la place des premiers-nés~; de tous les premiers-nés des fils d'Israël.
\VS{17}Car tout premier-né des enfants d'Israël est à moi, tant des hommes que des animaux~; je me les suis consacrés le jour où j'ai frappé tous les premiers-nés dans le pays d'Egypte.
\VS{18}Or j'ai pris les Lévites au lieu de tous les premiers-nés d'entre les enfants d'Israël.
\VS{19}Et j'ai entièrement donné, d'entre les enfants d'Israël, les Lévites à Aaron et à ses fils, pour faire le service des enfants d'Israël dans la tente d'assignation, et pour faire propitiation pour les enfants d'Israël~; afin qu'il n'y ait point de plaie sur les enfants d'Israël, comme il y aurait si les enfants d'Israël s'approchaient du sanctuaire.
\VS{20}Moïse, Aaron et toute l'assemblée des enfants d'Israël firent à l'égard des Lévites tout ce que Yahweh avait ordonné à Moïse touchant les Lévites~; ainsi firent les enfants d'Israël.
\VS{21}Les Lévites donc se purifièrent, et lavèrent leurs vêtements, et Aaron les fit tourner de côté et d'autre comme une offrande devant Yahweh, et il fit propitiation pour eux afin de les purifier.
\VS{22}Cela étant fait, les Lévites vinrent faire leur service dans la tente d'assignation devant Aaron, et devant ses fils, selon ce que Yahweh avait commandé à Moïse touchant les Lévites~; ainsi fut-il fait à leur égard.
\VS{23}Puis Yahweh parla à Moïse, en disant~:
\VS{24}Voici ce qui concerne les Lévites. Depuis l'âge de vingt-cinq ans et au-dessus, tout Lévite entrera en fonction dans la tente d'assignation~;
\VS{25}dès l'âge de cinquante ans, il sortira du service et ne servira plus.
\VS{26}Cependant il servira ses frères dans la tente d'assignation, pour garder ce qui leur a été commis, mais il ne fera plus de service. Tu agiras ainsi à l'égard des Lévites pour ce qui concerne leurs fonctions.
\Chap{9}
\TextTitle{La Pâque}
\VerseOne{}Yahweh avait aussi parlé à Moïse dans le désert de Sinaï, le premier mois de la seconde année, après qu'ils furent sortis du pays d'Egypte, en disant~:
\VS{2}Que les enfants d'Israël célèbrent la Pâque\FTNT{Ex. 12~; 1 Co. 5:7.} au temps fixé.
\VS{3}Vous la ferez en sa saison, le quatorzième jour de ce mois entre les deux soirs, selon toutes ses ordonnances et selon tout ce qu'il faut y faire.
\VS{4}Moïse donc, parla aux enfants d'Israël afin qu'ils célèbrent la Pâque.
\VS{5}Et ils firent la Pâque le quatorzième jour du premier mois, entre les deux soirs, dans le désert de Sinaï~; selon tout ce que Yahweh avait commandé à Moïse, les enfants d'Israël le firent ainsi.
\VS{6}Or il y eut quelques-uns qui étaient impurs à cause d'un mort et qui ne purent célébrer la Pâque ce jour-là. Ils se présentèrent ce même jour devant Moïse et devant Aaron,
\VS{7}et ces hommes leur dirent~: Nous sommes impurs à cause d'un mort, pourquoi serions-nous privés de présenter l'offrande à Yahweh dans sa saison au milieu des enfants d'Israël~?
\VS{8}Et Moïse leur dit~: Arrêtez-vous, et j'entendrai ce que Yahweh commandera sur votre sujet.
\VS{9}Alors Yahweh parla à Moïse, en disant~:
\VS{10}Parle aux enfants d'Israël, et dis-leur~: Si quelqu'un d'entre vous, ou de votre postérité, est impur à cause d'un mort, ou est en voyage dans un lieu éloigné, il célébrera cependant la Pâque en l'honneur de Yahweh.
\VS{11}Ils la feront le quatorzième jour du second mois, entre les deux soirs~; et ils la mangeront avec du pain sans levain et des herbes amères\FTNT{Ex. 12:10~; Ex. 23:18~; Ex. 34:25~; De. 16:4~; Jn. 19:33-36.}.
\VS{12}Ils n'en laisseront rien jusqu'au matin, et n'en briseront point les os. Ils la feront selon toutes les ordonnances de la Pâque.
\VS{13}Mais si celui qui est pur et qui n'est pas en voyage s'abstient de célébrer la Pâque, il sera retranché d'entre ses peuples parce qu'il n'a pas présenté l'offrande de Yahweh en sa saison.
\VS{14}Et si un étranger en séjour chez vous célèbre la Pâque de Yahweh, il la fera selon l'ordonnance de la Pâque. II y aura une même ordonnance entre vous, pour l'étranger comme pour celui qui est né au pays\FTNT{Ex. 12:49.}.
\TextTitle{La nuée conduit Israël}
\VS{15}Or le jour où le tabernacle fut dressé, la nuée couvrit le tabernacle de la tente d'assignation~; et le soir jusqu'au matin, elle parut sur le tabernacle avec l'apparence d'un feu\FTNT{Ex. 13:21-22~; Ex. 40:34-38~; De. 1:33.}.
\VS{16}Il en fut ainsi continuellement~; la nuée le couvrait, mais elle paraissait la nuit comme du feu.
\VS{17}Et selon que la nuée se levait de dessus le tabernacle, les enfants d'Israël partaient~; et au lieu où la nuée s'arrêtait, les enfants d'Israël y campaient.
\VS{18}Les enfants d'Israël marchaient sur le commandement de Yahweh, et ils campaient sur le commandement de Yahweh~; ils campaient aussi longtemps que la nuée se tenait sur le tabernacle.
\VS{19}Et quand la nuée restait plusieurs jours sur le tabernacle, les enfants d'Israël observaient l'ordre de Yahweh, et ne partaient point.
\VS{20}Et pour peu de jours que la nuée fût sur le tabernacle, ils campaient sur le commandement de Yahweh, et ils partaient sur le commandement de Yahweh.
\VS{21}Et quand la nuée y était depuis le soir jusqu'au matin, et que la nuée se levait au matin, ils partaient~; fût-ce de jour ou de nuit, quand la nuée se levait, ils partaient.
\VS{22}Si la nuée s'arrêtait sur le tabernacle deux jours, ou un mois, ou plus longtemps, les enfants d'Israël restaient campés, et ne partaient point~; mais quand elle se levait, ils partaient.
\VS{23}Ils campaient donc au commandement de Yahweh, et ils partaient au commandement de Yahweh~; et ils prenaient garde à Yahweh, suivant le commandement de Yahweh, qu'il leur faisait savoir par Moïse.
\Chap{10}
\TextTitle{Les trompettes d'argent}
\VerseOne{}Puis Yahweh parla à Moïse, en disant~:
\VS{2}Fais-toi deux trompettes d'argent, battues au marteau. Elles te serviront pour convoquer l'assemblée, et pour le départ des camps.
\VS{3}Quand on en sonnera, toute l'assemblée s'assemblera auprès de toi à l'entrée de la tente d'assignation.
\VS{4}Et quand on sonnera d'une seule, les princes, qui sont les chefs des milliers d'Israël, s'assembleront vers toi.
\VS{5}Mais quand vous sonnerez avec un retentissement bruyant, ceux qui campent à l'orient partiront.
\VS{6}Et quand vous sonnerez la seconde fois avec un retentissement bruyant, ceux qui campent au midi partiront, on sonnera avec un retentissement bruyant, pour leur départ.
\VS{7}Lorsque vous convoquerez l'assemblée, vous ne sonnerez pas avec un retentissement bruyant.
\VS{8}Or les fils d'Aaron, les prêtres, sonneront des trompettes. Ce sera une loi perpétuelle pour vous et pour vos descendants.
\VS{9}Et lorsque, dans votre pays, vous irez à la guerre contre l'ennemi qui vous combattra, vous sonnerez des trompettes avec un retentissement bruyant, et Yahweh votre Dieu, se souviendra de vous, et vous serez délivrés de vos ennemis.
\VS{10}Aussi dans vos jours de joie, dans vos fêtes solennelles, et au commencement de vos mois, vous sonnerez des trompettes en offrant vos holocaustes et vos sacrifices d'offrande de paix, et elles vous serviront de souvenir devant votre Dieu. Je suis Yahweh, votre Dieu.
\TextTitle{La nuée se lève, reprise de la marche dans le désert}
\VS{11}Or il arriva le vingtième jour du second mois de la seconde année, que la nuée se leva de dessus le tabernacle du témoignage.
\VS{12}Et les enfants d'Israël partirent du désert de Sinaï, selon l'ordre fixé pour leur marche. La nuée se posa dans le désert de Paran.
\VS{13}Ils partirent donc pour la première fois, suivant le commandement de Yahweh, déclaré par Moïse.
\VS{14}Et la bannière du camp des fils de Juda partit la première, selon leurs armées. Nachschon, fils d'Amminadab, commandait l'armée de Juda~;
\VS{15}et Nethaneel, fils de Tsuar, commandait l'armée de la tribu des fils d'Issacar~;
\VS{16}et Eliab, fils de Hélon, commandait l'armée de la tribu des fils de Zabulon.
\VS{17}Et le tabernacle fut démonté~; et les fils de Guerschon, et les fils de Merari, qui portaient le tabernacle, partirent.
\VS{18}Puis la bannière du camp de Ruben partit, selon leurs armées. Et Elitsur, fils de Schedéur, commandait l'armée de Ruben~;
\VS{19}et Schelumiel, fils de Tsurischaddaï, commandait l'armée de la tribu des fils de Siméon~;
\VS{20}et Eliasaph, fils de Déuel, commandait l'armée des fils de Gad.
\VS{21}Alors les Kehathites, qui portaient le sanctuaire, partirent~; cependant on dressait le tabernacle, en attendant leur arrivée.
\VS{22}Puis la bannière du camp des fils d'Ephraïm partit, selon leurs armées. Elischama, fils d'Ammihud, commandait l'armée d'Ephraïm~;
\VS{23}et Gamliel, fils de Pedahtsur, commandait l'armée de la tribu des fils de Manassé~;
\VS{24}et Abidan, fils de Guideoni, commandait l'armée de la tribu des fils de Benjamin.
\VS{25}Enfin la bannière des camps des fils de Dan, qui faisait l'arrière-garde, partit, selon leurs armées~; et Ahiézer, fils d'Ammischaddaï, commandait l'armée de Dan.
\VS{26}Et Paguiel, fils d'Ocran, commandait l'armée de la tribu des fils d'Aser~;
\VS{27}et Ahira, fils d'Enan, commandait l'armée de la tribu des fils de Nephthali.
\VS{28}Tel fut l'ordre d'après lequel les enfants d'Israël se mirent en marche selon leurs armées, c'est ainsi qu'ils partirent.
\VS{29}Or Moïse dit à Hobab, fils de Réuel, le Madianite, beau-père de Moïse~: Nous allons au lieu dont Yahweh a dit~: Je vous le donnerai. Viens avec nous, et nous te ferons du bien, car Yahweh a promis de faire du bien à Israël.
\VS{30}Et Hobab lui répondit~: Je n'irai point, mais je m'en irai dans mon pays, et vers ma parenté.
\VS{31}Et Moïse lui dit~: Je te prie, ne nous quitte pas~; car tu nous serviras de guide, parce que tu connais les lieux où nous aurons à camper dans le désert.
\VS{32}Et il arrivera que, quand tu seras venu avec nous, et que le bien que Yahweh doit nous faire sera arrivé, nous te ferons aussi du bien.
\VS{33}Ainsi ils partirent de la montagne de Yahweh et ils marchèrent trois jours~; et l'arche de l'alliance de Yahweh alla devant eux, et fit une marche de trois jours pour leur chercher un lieu de repos.
\VS{34}Et la nuée de Yahweh était sur eux le jour, quand ils partaient du camp.
\VS{35}Or il arrivait qu'au départ de l'arche, Moïse disait~: Lève-toi, ô Yahweh, et tes ennemis seront dispersés, et ceux qui te haïssent s'enfuiront de devant toi\FTNT{Ps. 68:2.}~!
\VS{36}Et quand on la posait, il disait~: Reviens Yahweh, aux dix mille milliers d'Israël~!
\Chap{11}
\TextTitle{Jugement contre les murmures du peuple}
\VerseOne{}Après, il arriva que le peuple murmura et cela déplut aux oreilles de Yahweh. Lorsque Yahweh l'entendit, sa colère s'enflamma, et le feu de Yahweh s'alluma parmi eux et en consuma l'extrémité du camp.
\VS{2}Alors le peuple cria à Moïse. Moïse pria Yahweh, et le feu s'éteignit.
\VS{3}Et on nomma ce lieu-là Tabeéra, parce que le feu de Yahweh s'était allumé parmi eux.
\TextTitle{Le peuple regrette l'Egypte}
\VS{4}Et le peuple nombreux qui se trouvaient au milieu d'Israël fut épris de convoitise~; et même, les enfants d'Israël se mirent à pleurer disant~: Qui nous donnera de la viande à manger\FTNT{Ex. 16:3~; Ps. 106:14~; 1 Co. 10:6.}~?
\VS{5}Nous nous souvenons des poissons que nous mangions en Egypte, et qui ne nous coûtaient rien, des concombres, des melons, des poireaux, des oignons, et de l'ail.
\VS{6}Et maintenant nos âmes sont asséchées~; nos yeux ne voient que de la manne\FTNT{Ps. 78:24.}.
\VS{7}Or la manne était comme la graine de coriandre, et avait l'apparence du bdellium\FTNT{Ex. 16:14-31~; Jn. 6:31-58.}.
\VS{8}Le peuple se dispersait et la ramassait, il la moulait aux meules, ou la pilait dans un mortier, il la cuisait au pot et en faisait des gâteaux. Elle avait le goût d'une liqueur d'huile fraîche.
\VS{9}Et quand la rosée descendait la nuit sur le camp, la manne y descendait aussi.
\TextTitle{Moïse dans l'affliction}
\VS{10}Moïse donc entendit le peuple qui pleurait, chacun dans sa famille et à l'entrée de sa tente. La colère de Yahweh s'enflamma fortement et Moïse en fut attristé.
\VS{11}Et Moïse dit à Yahweh~: Pourquoi affliges-tu ton serviteur et pourquoi n'ai-je pas trouvé grâce à tes yeux, que tu aies mis sur moi la charge de tout ce peuple~?
\VS{12}Est-ce moi qui ai conçu tout ce peuple ou l'ai-je engendré pour que tu me dises~: Porte-le dans ton sein comme le nourricier porte un enfant qui tète, porte-le jusqu'au pays que tu as juré à ses pères~?
\VS{13}D'où aurais-je de la viande pour en donner à tout ce peuple~? Car il pleure auprès de moi, en disant~: Donne-nous de la viande à manger~!
\VS{14}Je ne puis, à moi seul, porter tout ce peuple, car il est trop pesant pour moi\FTNT{De. 1:9-12.}.
\VS{15}Si tu agis ainsi à mon égard, tue-moi, je te prie donc, si j'ai trouvé grâce à tes yeux, et que je ne voie pas mon malheur.
\TextTitle{Yahweh établit soixante-dix anciens autour de Moïse\FTNTT{Ex. 18:19.}}
\VS{16}Alors Yahweh dit à Moïse~: Assemble-moi soixante-dix hommes des anciens d'Israël, que tu connais être les anciens du peuple et ses officiers, et amène-les à la tente d'assignation, et qu'ils s'y présentent avec toi.
\VS{17}Puis je descendrai, et je parlerai là avec toi, je mettrai de l'Esprit qui est sur toi sur eux~; afin qu'ils portent avec toi la charge du peuple, et que tu ne la portes pas toi seul.
\VS{18}Et tu diras au peuple~: Sanctifiez-vous pour demain, et vous mangerez de la viande~; puisque vous avez pleuré aux oreilles de Yahweh, en disant~: Qui nous fera manger de la viande~? Car nous étions bien en Egypte. Ainsi Yahweh vous donnera de la viande, et vous en mangerez.
\VS{19}Vous n'en mangerez pas un jour, ni deux jours, ni cinq jours, ni dix jours, ni vingt jours,
\VS{20}mais jusqu'à un mois entier, jusqu'à ce qu'elle vous sorte par les narines, et que vous en ayez du dégoût, parce que vous avez rejeté Yahweh qui est au milieu de vous~; vous avez pleuré devant lui, en disant~: Pourquoi sommes-nous sortis d'Egypte~?
\VS{21}Moïse dit~: Six cent mille hommes de pied forment ce peuple au milieu duquel je suis, et tu as dit~: Je leur donnerai de la viande afin qu'ils en mangent un mois entier~!
\VS{22}Leur tuera-t-on des brebis ou des bœufs, en sorte qu'il y en ait assez pour eux~? Ou leur assemblera-t-on tous les poissons de la mer, en sorte qu'ils en aient assez~?
\VS{23}Yahweh répondit à Moïse: La main de Yahweh serait-elle trop courte~? Tu verras maintenant si ce que je t'ai dit arrivera ou non\FTNT{Es. 50:2~; Es. 59:1-2.}.
\VS{24}Moïse donc sortit et rapporta au peuple les paroles de Yahweh. Il assembla soixante-dix hommes des anciens du peuple, et les plaça autour de la tente.
\VS{25}Yahweh descendit dans la nuée et parla à Moïse~; il prit de l'Esprit qui était sur lui et le mit sur les soixante-dix hommes anciens. Et dès que l'Esprit reposa sur eux, ils prophétisèrent~; mais ils ne continuèrent pas.
\TextTitle{Prophétie d'Eldad et de Médad}
\VS{26}Or il y eut deux hommes restés au camp, l'un s'appelait Eldad, et l'autre Médad, sur lesquels l'Esprit reposa. Ils étaient de ceux qui avaient été inscrits, mais ils n'étaient pas allés à la tente, et ils prophétisaient dans le camp.
\VS{27}Alors un garçon courut le rapporter à Moïse, en disant~: Eldad et Médad prophétisent dans le camp.
\VS{28}Et Josué, fils de Nun, qui servait Moïse, l'un de ses jeunes gens, répondit, en disant~: Mon seigneur Moïse, empêche-les.
\VS{29}Et Moïse lui répondit~: Es-tu jaloux pour moi~? Plût à Dieu que tout le peuple de Yahweh fût prophète, et que Yahweh mît son Esprit sur eux~!
\VS{30}Puis Moïse se retira au camp, lui et les anciens d'Israël.
\TextTitle{Les cailles et le jugement de Yahweh}
\VS{31}Alors Yahweh fit lever un vent de la mer qui amena des cailles et les répandit sur le camp environ le chemin d'une journée, de çà et de là, tout autour du camp~; et il y en avait presque la hauteur de deux coudées sur la terre\FTNT{Ex. 16:13-15~; Ps. 78:26-29~; Ps. 105:40.}.
\VS{32}Et le peuple se leva tout ce jour-là, et toute la nuit, et tout le jour suivant, et amassa des cailles~; celui qui en avait amassé le moins en avait dix homers~; et ils les étendirent soigneusement pour eux tout autour du camp.
\VS{33}Mais la chair était encore entre leurs dents, avant qu'elle fût mâchée, la colère de Yahweh s'embrasa contre le peuple, et il frappa le peuple d'une très grande plaie\FTNT{Ps. 78:30-31.}.
\VS{34}Et on nomma ce lieu-là Kibroth-Hattaava~; car on ensevelit là le peuple qui avait convoité.
\VS{35}Et de Kibroth-Hattaava le peuple s'en alla pour Hatséroth, et il s'arrêta à Hatséroth.
\Chap{12}
\TextTitle{Marie et Aaron murmurent contre Moïse}
\VerseOne{}Alors Marie et Aaron parlèrent contre Moïse au sujet de la femme éthiopienne\FTNT{Voir commentaire en Ge. 2:13.} qu'il avait prise, car il avait pris une femme éthiopienne.
\VS{2}Et ils dirent~: Est-ce seulement par Moïse que Yahweh parle~? N'est-ce pas aussi par nous qu'il parle~? Et Yahweh entendit cela. 
\VS{3}Or cet homme Moïse était un homme fort doux, plus que tous les hommes qui étaient sur la terre.
\VS{4}Et soudain Yahweh dit à Moïse, à Aaron, et à Marie~: Venez vous trois à la tente d'assignation~; et ils y allèrent eux trois.
\VS{5}Alors Yahweh descendit dans la colonne de nuée et se tint à l'entrée de la tente. Puis il appela Aaron et Marie, qui s'avancèrent tous les deux.
\VS{6}Et il dit~: Ecoutez maintenant mes paroles~! Lorsqu'il y aura parmi vous un prophète, moi qui suis Yahweh je me ferai bien connaître à lui en vision, et je lui parlerai en songe.
\VS{7}Il n'en est pas ainsi de mon serviteur Moïse, qui est fidèle dans toute ma maison\FTNT{Hé. 3:2.}.
\VS{8}Je parle avec lui bouche à bouche, et il me voit en effet, et non point en obscurité, ni dans aucune représentation de Yahweh. Pourquoi donc n'avez-vous pas craint de parler contre mon serviteur, contre Moïse~?
\FTNT{Ex. 33:11~; De. 34:10.} 
\VS{9}Ainsi la colère de Yahweh s'embrasa contre eux. Et il s'en alla.
\VS{10}Car la nuée se retira de dessus la tente. Et voici, Marie était frappée d'une lèpre blanche comme la neige~; et Aaron se tourna vers Marie et la vit lépreuse.
\VS{11}Alors Aaron dit à Moïse~: Hélas, de grâce, mon seigneur~! Je te prie ne mets point sur nous ce péché, car nous avons fait follement, et nous avons péché.
\VS{12}Je te prie qu'elle ne soit pas comme un enfant mort-né, dont la moitié de la chair est déjà consumée quand il sort du ventre de sa mère~!
\VS{13}Alors Moïse cria à Yahweh, en disant~: Ô Dieu, je te prie, guéris-la, je t'en prie.
\VS{14}Et Yahweh répondit à Moïse~: Si son père lui avait craché au visage, ne serait-elle pas dans l'ignominie pendant sept jours~? Qu'elle soit enfermée sept jours en dehors du camp, après quoi, elle y sera reçue\FTNT{Lé. 13:46.}.
\VS{15}Ainsi Marie fut enfermée hors du camp sept jours~; et le peuple ne partit pas de là jusqu'à ce que Marie fût rentrée.
\VS{16}Après cela le peuple partit de Hatséroth, et il campa dans le désert de Paran.
\Chap{13}
\TextTitle{Douze espions envoyés pour explorer Canaan}
\VerseOne{}Et Yahweh parla à Moïse, en disant~:
\VS{2}Envoie des hommes pour explorer le pays de Canaan, que je donne aux enfants d'Israël. Tu enverras un homme de chaque tribu de leurs pères, tous seront des principaux d'entre eux.
\VS{3}Moïse donc les envoya du désert de Paran, d'après l'ordre de Yahweh~; et tous ces hommes étaient chefs des enfants d'Israël.
\VS{4}Et ce sont ici leurs noms~: De la tribu de Ruben~: Schammua, fils de Zaccur~;
\VS{5}de la tribu de Siméon~: Schaphath, fils de Hori~;
\VS{6}de la tribu de Juda~: Caleb, fils de Jephunné~;
\VS{7}de la tribu d'Issacar~: Jigual, fils de Joseph~;
\VS{8}de la tribu d'Ephraïm~: Hosée, fils de Nun~;
\VS{9}de la tribu de Benjamin~: Palthi, fils de Raphu~;
\VS{10}de la tribu de Zabulon~: Gaddiel, fils de Sodi~;
\VS{11}de l'autre tribu de Joseph~: la tribu de Manassé, Gaddi, fils de Susi~;
\VS{12}de la tribu de Dan~: Ammiel, fils de Guemalli~;
\VS{13}de la tribu d'Aser~: Sethur, fils de Micaël~;
\VS{14}de la tribu de Nephthali~: Nachbi, fils de Vophsi~;
\VS{15}de la tribu de Gad~: Guéuel, fils de Maki.
\VS{16}Ce sont là les noms des hommes que Moïse envoya pour explorer le pays. Moïse donna à Hosée, fils de Nun, le nom de Josué\FTNT{Moïse changea le nom d'Hosée en y ajoutant le Nom de Yahweh. Hosée signifie «~sauveur~» et Josué (ou Jésus) «~Yahweh est salut~». Josué préfigurait Jésus-Christ qui nous a délivrés et transportés dans le Royaume des cieux (Col. 1:12-14). Moïse avait compris prophétiquement que seul Jésus peut nous faire rentrer dans notre héritage.}.
\VS{17}Moïse les envoya pour explorer le pays de Canaan, et il leur dit~: Montez de ce côté par le sud~; et vous monterez sur la montagne.
\VS{18}Et vous verrez quel est ce pays-là, et quel est le peuple qui l'habite, s'il est fort ou faible~; s'il est en petit ou en grand nombre.
\VS{19}Et quel est le pays où il habite, s'il est bon ou mauvais~; et quelles sont les villes dans lesquelles il habite, si c'est dans des camps, ou dans des villes fortifiées.
\VS{20}Et quelle est la terre, si elle est grasse ou maigre, s'il y a des arbres, ou non. Ayez bon courage, et prenez du fruit du pays. Or c'était alors le temps des premiers raisins.
\VS{21}Etant donc partis, ils examinèrent le pays, depuis le désert de Tsin jusqu'à Rehob, à l'entrée de Hamath.
\VS{22}Ils montèrent par le sud, et ils allèrent jusqu'à Hébron, où étaient Ahiman, Schéschaï, et Talmaï, enfants d'Anak. Hébron avait été bâtie sept ans avant Tsoan en Egypte.
\VS{23}Et ils vinrent jusqu'au torrent d'Eschcol, et coupèrent de là un sarment de vigne, avec une grappe de raisins~; ils étaient deux à le porter avec une perche. Ils apportèrent aussi des grenades et des figues.
\VS{24}Et on donna à ce lieu le nom de vallée d'Eschcol~; à cause de la grappe que les fils d'Israël y coupèrent.
\VS{25}Et au bout de quarante jours, ils furent de retour du pays qu'ils étaient allés explorer.
\TextTitle{Comptes rendus des envoyés}
\VS{26}Et à leur arrivée, ils se rendirent auprès de Moïse et d'Aaron, et de toute l'assemblée des enfants d'Israël, dans le désert de Padan à Kadès. Ils leur firent le rapport, ainsi qu'à toute l'assemblée, ils leur montrèrent les fruits du pays.
\VS{27}Ils firent donc leur rapport à Moïse, et lui dirent~: Nous avons été dans le pays où tu nous as envoyés. Véritablement, c'est un pays où coulent le lait et le miel, et en voici les fruits.
\VS{28}Seulement, le peuple qui habite ce pays est puissant, les villes sont fortifiées, très grandes~; nous y avons vu des enfants d'Anak\FTNT{De. 1:24-28.}.
\VS{29}Les Amalécites habitent la contrée du midi~; les Héthiens, les Jébusiens et les Amoréens habitent la montagne~; les Cananéens habitent le long de la mer, et vers le rivage du Jourdain.
\VS{30}Caleb fit taire le peuple devant Moïse, et il dit~: Montons, possédons ce pays, car nous y serons vainqueurs~!
\VS{31}Mais les hommes qui y étaient montés avec lui dirent~: Nous ne pouvons pas monter contre ce peuple-là, car il est plus fort que nous.
\VS{32}Et ils décrièrent devant les enfants d'Israël le pays qu'ils avaient exploré, en disant~: Le pays que nous avons parcouru pour l'explorer est un pays qui dévore ses habitants et tous ceux que nous y avons vus sont des gens de grande taille.
\VS{33}Et nous y avons vu aussi des géants, des enfants d'Anak, de la race des géants et nous étions à nos yeux et à leurs yeux comme des sauterelles.
\Chap{14}
\TextTitle{Rébellion et incrédulité d'Israël\FTNTT{1 Co. 10:1-5~; Hé. 3:7-19}}
\VerseOne{}Alors toute l'assemblée éleva la voix et se mit à pousser des cris, et le peuple pleura cette nuit-là.
\VS{2}Et tous les enfants d'Israël murmurèrent contre Moïse et Aaron, et toute l'assemblée leur dit~: Oh~! Si nous étions morts dans le pays d'Egypte~! Ou si nous étions morts dans ce désert\FTNT{De. 1:26-27.}~!
\VS{3}Et pourquoi Yahweh nous fait-il aller dans ce pays, où nous tomberons par l'épée, où nos femmes et nos petits enfants deviendront une proie~? Ne vaut-il pas mieux retourner en Egypte~?
\VS{4}Et ils se dirent l'un à l'autre~: Etablissons-nous un chef, et retournons en Egypte.
\VS{5}Alors Moïse et Aaron tombèrent sur leurs visages devant toute l'assemblée des enfants d'Israël.
\VS{6}Et Josué, fils de Nun, et Caleb, fils de Jephunné, qui étaient parmi ceux qui avaient exploré le pays, déchirèrent leurs vêtements,
\VS{7}et parlèrent à toute l'assemblée des enfants d'Israël, en disant~: Le pays que nous avons exploré est un très bon pays.
\VS{8}Si nous sommes agréables à Yahweh, il nous fera entrer dans ce pays, et il nous le donnera. C'est un pays où coulent le lait et le miel.
\VS{9}Seulement, ne soyez point rebelles contre Yahweh, et ne craignez point le peuple de ce pays-là, car ils seront notre pain, leur protection s'est retirée de dessus eux. Yahweh est avec nous, ne les craignez point\FTNT{De. 20:3-4.}~!
\VS{10}Alors toute l'assemblée parlait de les lapider~; mais la gloire de Yahweh apparut à tous les enfants d'Israël, devant la tente d'assignation.
\TextTitle{Moïse intercède pour le pardon d'Israël}
\VS{11}Et Yahweh dit à Moïse~: Jusqu'à quand ce peuple-ci m'irritera-t-il par mépris et jusqu'à quand ne croira-t-il point en moi, malgré tous les signes que j'ai faits au milieu de lui~?
\VS{12}Je le frapperai par la peste et je le détruirai, mais je ferai de toi une nation plus grande et plus puissante que lui.
\VS{13}Et Moïse dit à Yahweh~: Mais les Egyptiens l'entendront, car tu as fait monter par ta puissance ce peuple-ci du milieu d'eux\FTNT{Ex. 32:10-12.},
\VS{14}et ils diront avec les habitants de ce pays qui auront entendu que tu étais, ô Yahweh, au milieu de ce peuple, et que tu apparaissais, ô Yahweh à vue d'œil, que ta nuée s'arrêtait sur eux, et que tu marchais devant eux le jour dans la colonne de nuée, et la nuit dans la colonne de feu~;
\VS{15}si tu fais mourir ce peuple comme un seul homme, les nations qui ont entendu parler de toi diront~:
\VS{16}Yahweh n'avait pas le pouvoir de faire entrer ce peuple dans le pays qu'il avait juré de leur donner, il l'a égorgé dans le désert.
\VS{17}Maintenant, je te prie, que la puissance du Seigneur se montre dans sa grandeur, comme tu l'as déclaré en disant~:
\VS{18}Yahweh est lent à la colère et riche en bonté, il ôte l'iniquité et pardonne la rébellion, mais il ne tient point le coupable pour innocent, et il punit l'iniquité des pères sur les fils, jusqu'à la troisième et à la quatrième génération\FTNT{Ex. 20:5~; Ex. 34:6~; Ex. 34:7~; Ps. 86:15~; Ps. 103:8~; Ps. 145:8~; Jon. 4:2~; De. 5:9.}.
\VS{19}Pardonne, je te prie, l'iniquité de ce peuple, selon la grandeur de ta miséricorde, comme tu as pardonné à ce peuple depuis l'Egypte jusqu'ici.
\TextTitle{Réponse de Yahweh à Moïse}
\VS{20}Et Yahweh dit~: Je pardonne selon ta parole.
\VS{21}Mais certainement je suis vivant, et la gloire de Yahweh remplira toute la terre.
\VS{22}Car tous ceux qui ont vu ma gloire, et les prodiges que j'ai faits en Egypte et dans le désert, qui m'ont déjà tenté par dix fois, et qui n'ont point écouté ma voix,
\VS{23}tous ceux-là ne verront point le pays que j'ai juré à leurs pères de leur donner, tous ceux, dis-je, qui m'ont irrité par mépris, ne le verront pas\FTNT{De. 1:35-38.}.
\VS{24}Mais parce que mon serviteur Caleb a été animé d'un autre esprit, et qu'il a persévéré à me suivre, je le ferai entrer dans le pays où il a été, et ses descendants le posséderont en héritage.
\VS{25}Or les Amalécites et les Cananéens habitent la vallée. Demain, tournez-vous et partez pour le désert, dans la direction de la Mer Rouge.
\VS{26}Yahweh parla à Moïse et à Aaron, en disant~:
\VS{27}Jusqu'à quand laisserai-je cette méchante assemblée murmurer contre moi~? J'ai entendu les murmures des enfants d'Israël, qui murmuraient contre moi\FTNT{Ps. 106:25.}.
\VS{28}Dis-leur~: Je suis vivant, dit Yahweh, je vous ferai ainsi que vous avez parlé à mes oreilles.
\VS{29}Vos cadavres tomberont dans ce désert, et tous ceux d'entre vous qui ont été dénombrés, selon tout le compte que vous en avez fait, depuis l'âge de vingt ans, et au dessus, vous tous qui avez murmuré contre moi~;
\VS{30}vous n'entrerez pas dans le pays que j'avais juré de vous faire habiter, excepté Caleb, fils de Jephunné, et Josué, fils de Nun.
\VS{31}Et quant à vos petits enfants, dont vous avez dit~: Ils deviendront une proie~! Je les y ferai entrer, et ils connaîtront le pays que vous avez méprisé.
\VS{32}Mais quant à vous, vos cadavres tomberont dans ce désert~;
\VS{33}mais vos enfants paîtront dans ce désert quarante ans et ils porteront la peine de vos prostitutions, jusqu'à ce que vos cadavres soient tous consumés dans le désert.
\VS{34}Selon le nombre des jours que vous avez mis à reconnaître le pays, qui ont été quarante jours, un jour pour une année, vous porterez la peine de vos iniquités quarante ans, et vous connaîtrez ma rupture de promesse.
\VS{35}Je suis Yahweh, j'ai parlé~! C'est ainsi que je traiterai cette méchante assemblée, qui s'est assemblée contre moi~; ils seront consumés dans ce désert, et ils y mourront.
\VS{36}Les hommes donc que Moïse avait envoyés pour épier le pays, et qui étant de retour avaient fait murmurer contre lui toute l'assemblée, en diffamant le pays~;
\VS{37}ces hommes-là, qui avaient décrié le pays, moururent frappés d'une plaie devant Yahweh.
\VS{38}Mais Josué, fils de Nun, et Caleb, fils de Jephunné, restèrent seuls vivants parmi ceux qui étaient allés pour explorer le pays.
\TextTitle{Israël battu par les Amalécites et les Cananéens}
\VS{39}Or Moïse dit ces choses à tous les enfants d'Israël, et le peuple fut dans un grand deuil.
\VS{40}Puis ils se levèrent de bon matin et montèrent au sommet de la montagne, en disant~: Nous voici, et nous monterons au lieu dont Yahweh a parlé car nous avons péché.
\VS{41}Mais Moïse leur dit~: Pourquoi transgressez-vous le commandement de Yahweh~? Cela ne réussira point.
\VS{42}Ne montez pas~; car Yahweh n'est pas au milieu de vous~; afin que vous ne soyez pas battus devant vos ennemis\FTNT{De. 1:41-42.}.
\VS{43}Car les Amalécites et les Cananéens sont là devant vous, et vous tomberez par l'épée~; parce que vous vous êtes détournés de Yahweh, Yahweh ne sera point avec vous.
\VS{44}Toutefois ils s'obstinèrent à monter au sommet de la montagne~; mais l'arche de l'alliance de Yahweh et Moïse ne sortirent point du milieu du camp.
\VS{45}Alors les Amalécites et les Cananéens qui habitaient sur cette montagne descendirent, les battirent, et les taillèrent en pièces jusqu'à Horma.
\Chap{15}
\TextTitle{Consignes pour le pays de Canaan}
\VerseOne{}Puis Yahweh parla à Moïse, en disant~:
\VS{2}Parle aux enfants d'Israël, et dis-leur~: Quand vous serez entrés au pays que je vous donne, où vous devez demeurer,
\VS{3}et que vous voudrez faire un sacrifice consumé par le feu à Yahweh, un holocauste, ou un sacrifice en accompagnement d'un vœu, ou en offrande volontaire, ou bien dans vos fêtes, pour produire avec votre gros ou votre menu bétail une agréable odeur à Yahweh\FTNT{Ex. 29:18~; Lé. 22:21.},
\VS{4}celui qui offrira son offrande à Yahweh présentera en offrande un dixième de fleur de farine, pétrie dans un quart de hin d'huile\FTNT{Lé. 2:1-2.},
\VS{5}et un quart de hin de vin pour la libation que tu feras sur l'holocauste, ou sur un autre sacrifice pour chaque agneau.
\VS{6}Si c'est pour un bélier, tu feras en offrande deux dixièmes de fleur de farine, pétrie dans un tiers de hin d'huile,
\VS{7}et un tiers de hin de vin pour la libation, comme offrande d'une bonne odeur à Yahweh.
\VS{8}Et si tu sacrifies un veau, soit comme holocauste, soit comme sacrifice en accompagnement d'un vœu, ou comme sacrifice d'offrande de paix à Yahweh,
\VS{9}on présentera en offrande avec le veau trois dixièmes de fleur de farine, pétrie dans un demi-hin d'huile.
\VS{10}Et tu offriras la moitié d'un hin de vin pour la libation, en offrande consumée par le feu d'une bonne odeur à Yahweh.
\VS{11}On fera de même pour chaque bœuf, chaque bélier, et chaque petit des brebis ou des chèvres.
\VS{12}Selon le nombre que vous en sacrifierez, vous ferez ainsi à chacun, d'après leur nombre.
\VS{13}Tous ceux qui sont nés au pays feront ces choses de cette manière, en offrant un sacrifice consumé par le feu, d'une bonne odeur à Yahweh.
\TextTitle{Loi sur l'étranger vivant au milieu d'Israël}
\VS{14}Si un étranger séjournant chez vous, ou se trouvant au milieu de vous en vos générations, offre un sacrifice consumé par le feu d'une bonne odeur à Yahweh, il l'offrira de la même manière que vous.
\VS{15}Ô assemblée~! Il y aura une même ordonnance pour vous et pour l'étranger qui fait son séjour parmi vous, il y aura une même ordonnance perpétuelle en vos âges~; il en sera de l'étranger comme de vous en la présence de Yahweh.
\VS{16}Il y aura une même loi et une seule ordonnance pour vous et pour l'étranger qui séjourne au milieu de vous.
\TextTitle{Lois diverses}
\VS{17}Yahweh parla à Moïse, en disant~:
\VS{18}Parle aux enfants d'Israël, et dis-leur~: Quand vous serez arrivés dans le pays où je vous ferai entrer,
\VS{19}et que vous mangerez du pain de ce pays, vous en offrirez à Yahweh une offrande élevée.
\VS{20}Vous offrirez en offrande élevée un gâteau, les prémices de votre pâte~; vous l'offrirez comme ce qu'on prélève de l'aire.
\VS{21}Vous donnerez pour Yahweh une offrande des prémices de votre pâte, dans les temps à venir.
\VS{22}Et lorsque vous aurez péché involontairement\FTNT{Voir commentaire en Lé. 4:2.}, et que vous n'aurez pas fait tous ces commandements que Yahweh a fait connaître à Moïse,
\VS{23}tout ce que Yahweh vous a commandé par Moïse, depuis le jour où Yahweh a commencé de donner ses commandements, et dans la suite dans vos générations,
\VS{24}s'il arrive que la chose ait été faite involontairement, sans que l'assemblée s'en soit aperçue, toute l'assemblée sacrifiera un jeune taureau en holocauste d'une bonne odeur à Yahweh, avec l'offrande et la libation, d'après les règles établies~; elle offrira encore un jeune bouc en sacrifice pour l'expiation.
\VS{25}Ainsi le prêtre fera propitiation pour toute l'assemblée des enfants d'Israël, et il leur sera pardonné parce que c'est une chose arrivée involontairement, et ils ont apporté leur offrande, un sacrifice consumé par le feu à Yahweh et l'offrande pour l'expiation devant Yahweh, à cause de leur péché involontaire.
\VS{26}Alors il sera pardonné à toute l'assemblée des enfants d'Israël, et à l'étranger qui séjourne au milieu d'eux, car c'est involontairement que tout le peuple a péché.
\VS{27}Si c'est une seule personne qui a péché involontairement, elle offrira une chèvre d'un an en offrande pour le péché\FTNT{Lé. 4:27-28.}.
\VS{28}Et le prêtre fera propitiation pour la personne qui aura péché involontairement, de ce qu'elle aura péché involontairement devant Yahweh, et faisant propitiation pour elle, il lui sera pardonné.
\VS{29}Il y aura une même loi pour celui qui aura fait quelque chose involontairement, tant pour celui qui est né au pays des enfants d'Israël, que pour l'étranger qui fait son séjour parmi eux.
\VS{30}Mais quant à celui qui aura péché par fierté, tant celui qui est né au pays, que l'étranger, il a outragé Yahweh, cette personne-là sera retranchée du milieu de son peuple.
\VS{31}Parce qu'il a méprisé la parole de Yahweh, et qu'il a enfreint son commandement. Cette personne donc sera certainement retranchée~; son iniquité est sur elle.
\TextTitle{Un homme lapidé selon la loi\FTNTT{Ro. 3:19~; 7:7-11~; 2 Co. 3:7-9~; Ga. 3:10.}}
\VS{32}Or comme les enfants d'Israël étaient dans le désert, on trouva un homme qui ramassait du bois le jour du sabbat.
\VS{33}Et ceux qui l'avaient trouvé ramassant du bois, l'amenèrent à Moïse, à Aaron, et à toute l'assemblée.
\VS{34}Et on le mit sous garde, car ce qu'on devait lui faire n'avait pas été déclaré.
\VS{35}Alors Yahweh dit à Moïse~: On punira de mort cet homme, et toute l'assemblée le lapidera hors du camp.
\VS{36}Toute l'assemblée donc le mena hors du camp et le lapida, et il mourut, comme Yahweh l'avait ordonné à Moïse.
\VS{37}Et Yahweh parla à Moïse, en disant~:
\VS{38}Parle aux enfants d'Israël, et dis-leur~: Qu'ils se fassent de génération en génération des franges aux bords de leurs vêtements, et qu'ils mettent sur les franges au bords de leurs vêtements un cordon de couleur pourpre\FTNT{De. 22:12~; Mt. 23:5.}.
\VS{39}Quand vous aurez cette frange, vous la regarderez et vous vous souviendrez de tous les commandements de Yahweh, pour les mettre en pratique, et vous ne suivrez pas les désirs de vos cœurs et de vos yeux, pour vous laisser entraîner à la prostitution.
\VS{40}Afin que vous vous souveniez de tous mes commandements, et que vous les fassiez, et que vous soyez saints à votre Dieu.
\VS{41}Je suis Yahweh, votre Dieu, qui vous ai retiré du pays d'Egypte, pour être votre Dieu. Je suis Yahweh, votre Dieu.
\Chap{16}
\TextTitle{La révolte de Koré\FTNTT{Jud. 11.}}
\VerseOne{}Or Koré\FTNT{Koré, Dathan et Abiram, s'étaient révoltés contre Aaron et Moïse, car ils voulaient s'attribuer l'honneur d'offrir à Dieu des sacrifices. Ils voulaient exercer la prêtrise (sacerdoce) alors que Yahweh ne les avait pas établis pour le service du culte. Vouloir servir Dieu sans avoir reçu un appel divin est dangereux.}, fils de Jitsehar, fils de Kehath, fils de Lévi, se révolta avec Dathan et Abiram, fils d'Eliab, et On, fils de Péleth, tous trois fils de Ruben.
\VS{2}Et ils s'élevèrent contre Moïse, avec deux cent cinquante hommes des fils d'Israël, qui étaient des principaux de l'assemblée, de ceux que l'on convoquait pour tenir le conseil, et qui étaient des gens de renom.
\VS{3}Et ils s'assemblèrent contre Moïse et contre Aaron, et leur dirent~: C'en est assez~! Puisque tous ceux de l'assemblée sont saints, et que Yahweh est au milieu d'eux, pourquoi vous élevez-vous au-dessus de l'assemblée de Yahweh~?
\VS{4}Quand Moïse eut entendu cela, il se jeta sur son visage.
\VS{5}Et il parla à Koré et à tous ceux qui étaient assemblés avec lui, et leur dit~: Demain au matin, Yahweh fera connaître celui qui lui appartient, et celui qui est saint, et il le fera approcher de lui~; il fera, dis-je, approcher de lui celui qu'il aura choisi.
\VS{6}Faites ceci, prenez des encensoirs, Koré et toute son assemblée.
\VS{7}Et demain, mettez-y du feu, et mettez-y du parfum devant Yahweh~; et celui que Yahweh choisira, c'est celui-là qui sera saint. C'en est assez, fils de Lévi~!
\VS{8}Moïse dit aussi à Koré~: Ecoutez maintenant, fils de Lévi~:
\VS{9}Est-ce trop peu de chose pour vous, que le Dieu d'Israël vous ait séparés de l'assemblée d'Israël, pour vous faire approcher de lui, afin de faire le service du tabernacle de Yahweh, et pour vous tenir devant l'assemblée, afin de la servir~?
\VS{10}Et qu'il t'ait fait approcher de lui, toi et tous tes frères, les fils de Lévi, et vous recherchez encore la prêtrise~!
\VS{11}C'est pourquoi toi et toute ton assemblée, vous vous êtes rassemblés contre Yahweh~! Car qui est Aaron pour que vous murmuriez contre lui~?
\VS{12}Et Moïse envoya appeler Dathan et Abiram, fils d'Eliab, qui répondirent~: Nous n'y monterons point.
\VS{13}Est-ce peu de chose que tu nous aies fait monter hors d'un pays où coulent le lait et le miel, pour nous faire mourir dans le désert, que tu veuilles aussi dominer sur nous~?
\VS{14}Certes, tu ne nous as pas fait venir dans un pays où coulent le lait et le miel~! Et tu ne nous as pas donné un héritage de champs ni de vignes~! Veux-tu crever les yeux de ces gens~? Nous ne monterons pas.
\VS{15}Alors Moïse fut très irrité, et il dit à Yahweh~: N'aie point égard à leur offrande. Je n'ai point pris d'eux un seul âne, et je n'ai fait de mal à aucun d'eux.
\VS{16}Puis Moïse dit à Koré~: Toi et tous ceux qui sont assemblés avec toi, trouvez-vous demain devant Yahweh, toi et eux avec Aaron.
\VS{17}Et prenez chacun vos encensoirs, et mettez-y du parfum~; et que chacun présente devant Yahweh son encensoir~: Il y aura deux cent cinquante encensoirs~; toi et Aaron aussi, chacun avec son encensoir.
\VS{18}Ils prirent donc chacun son encensoir, et y mirent du feu, et ensuite y posèrent du parfum, et ils se tinrent à l'entrée de la tente d'assignation, avec Moïse et Aaron.
\VS{19}Et Koré fit assembler contre eux toute l'assemblée à l'entrée de la tente d'assignation~; et la gloire de Yahweh apparut à toute l'assemblée.
\VS{20}Puis Yahweh parla à Moïse et à Aaron, en disant~:
\VS{21}Séparez-vous du milieu de cette assemblée, et je les consumerai en un seul instant\FTNT{Ex. 32:10.}.
\VS{22}Mais ils tombèrent sur leur visage et dirent~: Ô Dieu~! Dieu des esprits de toute chair~! Un seul homme a péché, et tu te mettrais en colère contre toute l'assemblée\FTNT{Hé. 12:9.}~?
\VS{23}Et Yahweh parla à Moïse, en disant~:
\VS{24}Parle à l'assemblée, et dis lui~: Retirez-vous d'auprès de la demeure de Koré, de Dathan, et d'Abiram.
\VS{25}Moïse donc se leva, et alla vers Dathan et Abiram~; et les anciens d'Israël le suivirent.
\VS{26}Et il parla à l'assemblée, en disant~: Eloignez-vous, je vous prie, d'auprès des tentes de ces méchants hommes, et ne touchez à rien qui leur appartienne, de peur que vous ne périssiez punis pour tous leurs péchés.
\VS{27}Ils se retirèrent donc d'auprès des demeures de Koré, de Dathan et d'Abiram. Et Dathan et Abiram sortirent et se tinrent debout à l'entrée de leurs tentes, avec leurs femmes, leurs fils, et leurs petits-enfants.
\VS{28}Et Moïse dit~: Vous connaîtrez à ceci que Yahweh m'a envoyé pour faire toutes ces choses, et que je n'agis pas de moi-même.
\VS{29}Si ces gens meurent comme tous les hommes meurent, et s'ils subissent le sort commun à tous les hommes, Yahweh ne m'a point envoyé~;
\VS{30}mais si Yahweh fait une chose nouvelle, et si la terre ouvre sa bouche pour les engloutir avec tout ce qui leur appartient, et qu'ils descendent vivants dans le scheol, vous saurez alors que ces hommes-là ont irrité par mépris Yahweh.
\VS{31}Et il arriva qu'aussitôt qu'il eut achevé de dire toutes ces paroles, la terre qui était sous eux se fendit.
\VS{32}Et la terre ouvrit sa bouche et les engloutit, avec leurs tentes et tous les hommes qui étaient à Koré, et tous leurs biens\FTNT{De. 11:6~; Ps. 106:17.}.
\VS{33}Ils descendirent donc vivants dans le scheol, eux et tout ceux qui leur appartenait~; la terre les recouvrit, et ils disparurent au milieu de l'assemblée.
\VS{34}Et tout Israël qui était autour d'eux s'enfuit à leurs cris~; car ils disaient~: Prenons garde que la terre ne nous engloutisse~!
\VS{35}Un feu sortit de part Yahweh et consuma les deux cent cinquante hommes qui offraient le parfum.
\VS{36}Puis Yahweh parla à Moïse, en disant~:
\VS{37}Dis à Eléazar, fils d'Aaron, le prêtre, qu'il ramasse les encensoirs du milieu de l'embrasement, et d'en répandre au loin le feu, car ils sont sanctifiés.
\VS{38}Avec les encensoirs de ceux qui ont péché contre leurs âmes, que l'on fasse des lames étendues dont on couvrira l'autel. Puisqu'ils ont été offerts devant Yahweh et qu'ils sont sanctifiés, ils serviront de signe aux enfants d'Israël.
\VS{39}Ainsi Eléazar, le prêtre, prit les encensoirs d'airain, que ces hommes qui furent brûlés avaient présentés, et on en fit des lames pour couvrir l'autel.
\VS{40}C'est un souvenir pour les enfants d'Israël, afin qu'aucun étranger qui n'est pas de la race d'Aaron, ne s'approche pour offrir du parfum devant Yahweh, et ne soit comme Koré, et comme ceux qui ont été assemblés avec lui~; selon ce que Yahweh avait déclaré par Moïse.
\TextTitle{Le peuple frappé à cause des murmures}
\VS{41}Or dès le lendemain, toute l'assemblée des enfants d'Israël murmura contre Moïse et contre Aaron, en disant~: Vous avez fait mourir le peuple de Yahweh.
\VS{42}Et il arriva comme l'assemblée s'amassait contre Moïse et contre Aaron, et comme ils tournaient les regards vers la tente d'assignation, voici la nuée la couvrit, et la gloire de Yahweh apparut.
\VS{43}Moïse donc et Aaron vinrent donc devant la tente d'assignation.
\VS{44}Et Yahweh parla à Moïse, en disant~:
\VS{45}Retirez-vous du milieu de cette assemblée, et je les consumerai en un instant. Alors ils se prosternèrent le visage contre terre~;
\VS{46}puis Moïse dit à Aaron~: Prends l'encensoir, et mets-y du feu de dessus l'autel, mets-y aussi du parfum, et va promptement à l'assemblée, et fais propitiation pour eux~; car une grande colère est sortie de devant Yahweh, la plaie a commencé.
\VS{47}Et Aaron prit l'encensoir, comme Moïse lui avait dit, et il courut au milieu de l'assemblée, et voici la plaie avait déjà commencé sur le peuple. Alors il mit du parfum et fit propitiation pour le peuple.
\VS{48}Et comme il se tenait entre les morts et les vivants, la plaie fut arrêtée.
\VS{49}Et il y en eut quatorze mille sept cents qui moururent de cette plaie, outre ceux qui étaient morts à cause de Koré.
\VS{50}Et Aaron retourna auprès de Moïse, à l'entrée de la tente d'assignation, et la plaie s'arrêta.
\Chap{17}
\TextTitle{Yahweh confirme l'appel d'Aaron, sa verge fleurit}
\VerseOne{}Après cela Yahweh parla à Moïse, en disant~:
\VS{2}Parle aux enfants d'Israël, et prends une verge de chacun d'eux selon la maison de leur père, de tous ceux qui sont les princes, selon la maison de leurs pères, douze verges, puis tu écriras le nom de chacun sur sa verge,
\VS{3}mais tu écriras le nom d'Aaron sur la verge de Lévi\FTNT{La verge d'Aaron est une image du Messie ressuscité. Elle avait produit la vie tandis que celles des autres princes n'avaient produit aucun fruit. Cette histoire nous parle également de la confirmation de l'appel d'Aaron face aux critiques dont il était l'objet. On reconnaît l'arbre par ses fruits (Mt. 7:16-20~; Lu. 7:17-22).}~; car il y aura une verge pour chaque chef des maisons de leurs pères.
\VS{4}Et tu les déposeras dans la tente d'assignation, devant le témoignage, où je me rencontre avec vous.
\VS{5}Et la verge de l'homme que j'aurai choisi fleurira~; et je ferai cesser de devant moi les murmures des enfants d'Israël, par lesquels ils murmurent contre vous.
\VS{6}Quand Moïse parla aux enfants d'Israël, tous leurs princes lui donnèrent une verge, chaque prince une verge, selon les maisons de leurs pères, soit douze verges~; or la verge d'Aaron était au milieu des leurs.
\VS{7}Et Moïse mit les verges devant Yahweh, dans la tente du témoignage.
\VS{8}Et le lendemain, lorsque Moïse entra dans la tente du témoignage, voici, la verge d'Aaron, avait fleuri, pour la maison de Lévi, et elle avait poussé des boutons, produit des fleurs et mûri des amandes.
\VS{9}Alors Moïse ôta de devant Yahweh toutes les verges et les porta à tous les fils d'Israël, afin qu'ils les voient et qu'ils prennent chacun leurs verges.
\VS{10}Et Yahweh dit à Moïse~: Reporte la verge d'Aaron devant le témoignage, pour être conservée comme un signe pour les fils de rébellion, afin que tu fasses cesser de devant moi leurs murmures et qu'ils ne meurent point\FTNT{Hé. 9:3-5.}.
\VS{11}Et Moïse fit ainsi~; il se conforma à l'ordre que Yahweh lui avait donné.
\VS{12}Les enfants d'Israël parlèrent à Moïse, en disant~: Voici, nous expirons, nous périssons, nous périssons tous~!
\VS{13}Quiconque s'approche du tabernacle de Yahweh, meurt. Serons-nous tous entièrement expirés~?
\Chap{18}
\TextTitle{Droits et devoirs des prêtres et des Lévites}
\VerseOne{}Alors Yahweh dit à Aaron~: Toi et tes fils, et la maison de ton père avec toi, vous porterez l'iniquité du sanctuaire~; et toi, et tes fils avec toi, vous porterez l'iniquité de votre prêtrise.
\VS{2}Fais aussi approcher de toi tes frères, la tribu de Lévi, qui est la tribu de ton père, afin qu'ils te soient attachés et qu'ils te servent, mais toi et tes fils avec toi, vous servirez devant la tente du témoignage.
\VS{3}Ils garderont ce que tu leur ordonneras de garder, et ce qu'il faut garder de toute la tente, mais ils n'approcheront point des ustensiles du sanctuaire, ni de l'autel de peur qu'ils ne meurent, et que vous ne mouriez avec eux.
\VS{4}Ils te seront donc attachés, et ils garderont tout ce qu'il faut garder dans la tente d'assignation, selon tout le service du tabernacle et aucun étranger n'approchera de vous.
\VS{5}Mais vous prendrez garde à ce qu'il faut faire dans le sanctuaire, et à ce qu'il faut faire à l'autel, afin qu'il n'y ait plus d'indignation sur les enfants d'Israël.
\VS{6}Car quant à moi voici, j'ai pris vos frères, les Lévites, du milieu des enfants d'Israël, qui sont donnés en pur don pour Yahweh, afin qu'ils soient employés au service de la tente d'assignation.
\VS{7}Mais toi et tes fils avec toi, vous observerez la fonction de votre prêtrise en tout ce qui concerne l'autel et ce qui est au dedans du voile, et vous y ferez le service. J'établis votre prêtrise en office de pur don~; c'est pourquoi si un étranger en approche, on le fera mourir.
\VS{8}Yahweh dit encore à Aaron~: Voici, je t'ai donné la garde de mes offrandes élevées sur toutes les choses consacrées par les enfants d'Israël~; je te les ai données, et à tes enfants, par ordonnance perpétuelle, à cause de l'onction.
\VS{9}Ceci t'appartiendra d'entre les choses très saintes qui ne sont pas brûlées, savoir toutes leurs offrandes, soit de tous leurs gâteaux, soit de tous leurs sacrifices pour l'expiation, et tous leurs sacrifices pour la culpabilité qu'ils m'apporteront~; ce sont des choses très saintes pour toi et pour tes enfants.
\VS{10}Vous les mangerez dans un lieu très saint~; tout mâle en mangera~; vous les regarderez comme saintes\FTNT{Lé. 6:17-22~; Lé. 7:6~; Lé. 10:13.}.
\VS{11}Voici encore ce qui t'appartiendra~: Tous les dons que les enfants d'Israël présenteront par élévation et en les agitant de côté et d'autre, je te les donne à toi, à tes fils, et à tes filles avec toi, par une loi perpétuelle~; quiconque sera pur dans ta maison en mangera\FTNT{Lé. 7:34~; Lé. 10:14.}.
\VS{12}Je te donne aussi leurs prémices qu'ils offriront à Yahweh~: Tout ce qu'il y aura de meilleur en huile, et tout le meilleur du moût et du blé.
\VS{13}Les premiers fruits de toutes les choses que leur terre produira, et qu'ils apporteront à Yahweh t'appartiendront~; quiconque sera pur dans ta maison, en mangera.
\VS{14}Tout ce qui sera dévoué en Israël t'appartiendra\FTNT{Lé. 27:28~; Ez. 44:29.}.
\VS{15}Tout premier-né de toute chair, qu'ils offriront à Yahweh, tant des hommes que des animaux t'appartiendront. Mais, tu feras racheter le premier-né de l'homme, et tu feras racheter le premier-né d'un animal impur.
\VS{16}Et ceux qui doivent être rachetés, depuis l'âge d'un mois, tu les rachèteras selon ton estimation que tu en feras, au prix de cinq sicles d'argent, selon le sicle du sanctuaire, qui est de vingt guéras.
\VS{17}Mais tu ne feras point racheter le premier-né du bœuf, ni le premier-né de la brebis, ni le premier-né de la chèvre~: Ce sont des choses saintes. Tu répandras leur sang sur l'autel, et tu brûleras leur graisse~: Ce sera un sacrifice consumé par le feu d'une bonne odeur à Yahweh.
\VS{18}Mais leur chair t'appartiendra, comme la poitrine qu'on agite de côté et d'autre, et comme l'épaule droite.
\VS{19}Je t'ai donné, à toi et à tes fils, et à tes filles avec toi, par une loi perpétuelle, toutes les offrandes présentées par élévation des choses sanctifiées, que les enfants d'Israël offriront à Yahweh. C'est une alliance de sel\FTNT{Le sel est un aliment pratiquement impérissable et incorruptible. Dans l'Antiquité, il symbolisait l'incorruptibilité (Lé. 2:13).} et à perpétuité devant Yahweh, pour toi et pour ta postérité avec toi.
\VS{20}Puis Yahweh dit à Aaron~: Tu ne posséderas rien dans leur pays, et il n'y aura point de part pour toi au milieu d'eux~; c'est moi qui suis ta part et ta possession, au milieu des enfants d'Israël\FTNT{De. 10:9~; De. 18:2~; Ez. 44:28.}.
\TextTitle{Lois sur les dîmes (De. 14:22-29)}
\VS{21}Et je donne comme possession aux fils de Lévi, toutes les dîmes\FTNT{Il y avait plusieurs sortes de dîmes dans la loi Mosaïque~:
\\- La 1ère dîme~: Le peuple devait payer une dîme générale au bénéfice des Lévites (No. 18:21).
\\Toutes les tribus d'Israël, à l'exception des Lévites, eurent une possession géographique qu'ils reçurent comme héritage après leur entrée en Canaan. Mais les Lévites devaient accomplir une tâche particulière au sein de la nation. Ils devaient s'occuper du service dans la tente d'assignation. En compensation de ce service, ils devaient percevoir un impôt de 10\% des revenus de tous les Israélites.
\\- La 2ème dîme~: Les Lévites devaient payer la «~dîme de la dîme~», au bénéfice des prêtres (No. 18:25-31).
\\Tous les prêtres étaient des Lévites, mais tous les Lévites n'étaient pas des prêtres. Les prêtres descendaient d'Aaron et ils exerçaient des responsabilités particulières dans la tente d'assignation, puis dans le temple. Cette seconde dîme permettait aux prêtres d'être nourris et assurait donc le bon fonctionnement du service du temple.
\\- La 3ème dîme~: Tous les Israélites devaient conserver une dîme de toute leur production en prévision de leurs pèlerinages annuels à Jérusalem (De. 14:22-26).
\\Trois fois par an, tout le peuple devait s'assembler à Jérusalem, l'endroit choisi par le Seigneur, à l'occasion des principales fêtes. Dieu avait prévu que chacun puisse disposer de ressources suffisantes pour leur permettre de se réjouir pleinement à ces occasions. C'est pour cela qu'ils devaient mettre de côté 10\% de leurs productions agricoles annuelles. Il est intéressant de noter que la dîme n'était jamais payée en argent, mais toujours en nature.
\\- La 4ème dîme~: Il fallait payer une dîme spéciale à l'intention des pauvres, des orphelins et des veuves (De. 14:28-29). 
\\Certains affirment que la dîme existait bien avant la loi. Mais ils ignorent que la Bible parle de plusieurs sortes de lois.
\\- Les lois cérémonielles (Hé. 9:1)
\\Ces lois étaient relatives au culte et concernaient le tabernacle puis le temple, les sacrifices, les ablutions (Lé. 16~; Hé. 9:1-10). Les dîmes (la dîme des prêtres) devaient être amenées dans le temple (Mal. 3:10), elles faisaient donc partie des lois cérémonielles. Or les Lévites et les prêtres de la Première Alliance n'existent plus sous la Nouvelle Alliance car les enfants de Dieu sont un royaume de rois et de prêtres (Ap. 1:6~; Ap. 5:10).
\\- Les lois morales (Ex. 20:1-17). Dieu est saint et il veut un peuple saint qui marche dans sa crainte, dans la sainteté et dans l'obéissance. Lé. 18 nous parle des lois morales~; elles n'ont pas été abolies, elles existent toujours. Elles sont inscrites dans la conscience de l'homme, elles sont gravées dans notre cœur (Hé. 8:10).
\\- Les lois sociales (Ex. 21:1-24). Ce sont des lois civiles régissant la vie sociale d'Israël, comme nous pouvons le lire dans Ex. 21 par exemple. Ces lois n'ont rien à voir avec les croyants de la Nouvelle Alliance. Les lois morales témoignent de la nature de Dieu, ce sont des lois éternelles qui existaient bien avant Abraham. Les lois cérémonielles ont commencé dès la fondation du monde (Ap. 13:8) car l'Agneau de Dieu était immolé avant la fondation du monde (1 Pi. 1:19-20). Seules les lois sociales ont débuté avec Moïse car elles concernaient exclusivement les Israélites. Ces trois sortes de lois ont été institutionnalisées par Moïse, mais les deux premières (morales et cérémonielles) existaient avant ce dernier. Les quatre sortes de dîmes faisaient bel et bien partie des lois sociales et cérémonielles. Or ces lois ne sont plus d'actualité sous la Nouvelle Alliance. En conclusion, nous pouvons dire que Jésus nous a rachetés en accomplissant les lois cérémonielles afin que nous pratiquions les lois morales (Ep. 2:10). Voir également commentaire en Mal 3~: 10.} d'Israël, pour le service auquel ils sont employés, le service de la tente d'assignation.
\VS{22}Et les enfants d'Israël n'approcheront plus de la tente d'assignation, afin qu'ils ne se chargent d'un péché et qu'ils ne meurent point.
\VS{23}Mais les Lévites s'emploieront au service de la tente d'assignation, et ils resteront chargés de leurs iniquités. Cette loi sera perpétuelle parmi vos descendants, et ils ne posséderont point d'héritage parmi les enfants d'Israël.
\VS{24}Car je donne comme possession aux Lévites les dîmes que les enfants d'Israël présenteront à Yahweh en offrande élevée~; c'est pourquoi je dis d'eux qu'ils n'auront point d'héritage parmi les fils d'Israël.
\VS{25}Puis Yahweh parla à Moïse, en disant~:
\VS{26}Tu parleras aussi aux Lévites, et tu leur diras~: Quand vous recevrez des enfants d'Israël les dîmes que je vous donne de leur part comme possession, vous en offrirez l'offrande élevée à Yahweh, la dîme de la dîme~;
\VS{27}et votre offrande élevée vous sera comptée comme le blé qu'on prélève de l'aire, et comme l'abondance qu'on prélève de la cuve.
\VS{28}C'est ainsi que vous prélèverez une offrande pour Yahweh de toutes les dîmes que vous recevrez des enfants d'Israël, et vous donnerez au prêtre Aaron l'offrande que vous en aurez prélevée pour Yahweh.
\VS{29}Sur tous les dons qui vous seront faits, vous prélèverez toute l'offrande élevée pour Yahweh~; sur tout ce qu'il y aura de meilleur, vous prélèverez la portion consacrée.
\VS{30}Et tu leur diras~: Quand vous aurez offert en offrande élevée le meilleur de la dîme, pris de la dîme même, il sera imputé aux Lévites comme le revenu de l'aire, et comme le revenu de la cuve.
\VS{31}Et vous la mangerez en tout lieu, vous et votre maison~; car c'est votre salaire pour le service auquel vous êtes employés dans la tente d'assignation.
\VS{32}Vous ne serez point coupables de péché au sujet de la dîme, quand vous en aurez offert en offrande élevée sur ce qu'il y aura de meilleur et vous ne souillerez point les choses saintes des enfants d'Israël et vous ne mourrez point.
\Chap{19}
\TextTitle{La jeune vache rousse~; l'eau de purification}
\VerseOne{}Yahweh parla à Moïse et à Aaron, en disant~:
\VS{2}Voici ce qui est ordonné par la loi que Yahweh a commandé, en disant~: Parle aux enfants d'Israël, et dis-leur qu'ils t'amènent une jeune vache rousse, entière, sans défaut, et qui n'ait point porté le joug.
\VS{3}Puis vous la donnerez à Eléazar, le prêtre, qui la mènera hors du camp, et on l'égorgera en sa présence\FTNT{Lé. 4:12~; Hé. 13:11-12.}.
\VS{4}Ensuite, Eléazar, le prêtre, prendra de son sang avec son doigt, et fera sept fois l'aspersion du sang vers le devant de la tente d'assignation.
\VS{5}Et on brûlera la jeune vache en sa présence~; on brûlera sa peau, sa chair, son sang et ses excréments\FTNT{Ex. 29:14.}.
\VS{6}Le prêtre prendra du bois de cèdre, de l'hysope, et du cramoisi, et les jettera dans le feu où sera brûlée la jeune vache.
\VS{7}Puis le prêtre lavera ses vêtements et son corps avec de l'eau~; après cela, il rentrera au camp, et le prêtre sera impur jusqu'au soir.
\VS{8}Celui qui l'aura brûlé, lavera ses vêtements dans l'eau, il lavera aussi dans l'eau son corps~; et il sera impur jusqu'au soir.
\VS{9}Et un homme pur ramassera les cendres de la jeune vache, et les mettra hors du camp, dans un lieu pur~; elles seront gardées pour l'assemblée des enfants d'Israël~; afin d'en faire l'eau de purification. C'est une purification pour le péché.
\VS{10}Celui qui aura ramassé les cendres de la jeune vache, lavera ses vêtements, et sera impur jusqu'au soir~; ce sera une loi perpétuelle pour les enfants d'Israël, et pour l'étranger en séjour au milieu d'eux.
\VS{11}Celui qui touchera un mort, un corps humain quel qu'il soit, sera impur pendant sept jours\FTNT{Ag. 2:13.}.
\VS{12}Il se purifiera avec cette eau le troisième jour et le septième jour, et il sera pur~; mais s'il ne se purifie pas le troisième jour, il ne sera pas pur le septième jour.
\VS{13}Alors celui qui touchera un mort, le corps d'un homme qui sera mort et qui ne se purifiera pas, souille le tabernacle de Yahweh~; celui-là sera retranché d'Israël. Il est impur, car l'eau de purification n'a pas été répandue sur lui, son impureté demeure encore sur lui.
\VS{14}Voici la loi. Lorsqu'un homme mourra dans une tente, quiconque entrera dans la tente, et quiconque se trouvera dans la tente sera impur pendant sept jours.
\VS{15}Aussi tout vase découvert, sur lequel il n'y aura point de couvercle attaché, sera impur.
\VS{16}Et quiconque touchera, dans les champs, un homme qui aura été tué par l'épée, ou un mort, ou des ossements humains, ou un sépulcre, sera impur durant sept jours.
\VS{17}Et on prendra, pour celui qui est impur, de la poudre de la jeune vache brûlée pour faire la purification, et on la mettra dans un vase, avec de l'eau vive par-dessus.
\VS{18}Puis un homme pur prendra de l'hysope, et la trempera dans l'eau~; il en fera aspersion sur la tente, et sur tous les ustensiles, et sur toutes les personnes qui auront été là, et sur celui qui a touché des ossements ou un homme tué, ou un mort, ou un sépulcre.
\VS{19}Celui qui est pur fera l'aspersion sur celui qui est impur, le troisième jour et le septième jour, et il le purifiera le septième jour~; puis il lavera ses vêtements, et se lavera dans l'eau, et il sera pur le soir.
\VS{20}Mais l'homme qui sera impur, et qui ne se purifiera point, sera retranché du milieu de l'assemblée, parce qu'il a souillé le sanctuaire de Yahweh~; comme l'eau de purification n'a pas été répandue sur lui, il est impur.
\VS{21}Et ce sera pour eux une loi perpétuelle, et celui qui fera l'aspersion de l'eau de purification lavera ses vêtements~; et quiconque touchera l'eau de purification sera impur jusqu'au soir.
\VS{22}Et tout ce que l'homme impur touchera sera souillé, et la personne qui le touchera sera impure jusqu'au soir.
\Chap{20}
\TextTitle{Mort de Marie}
\VerseOne{}Or toute l'assemblée des enfants d'Israël arriva dans le désert de Tsin au premier mois, et le peuple s'arrêta à Kadès. Marie mourut là, et y fut ensevelie.
\TextTitle{Murmures du peuple à cause du manque d'eau\FTNTT{De. 32:51~; cp. Ex. 17:1-7.}}
\VS{2}Et il n'y avait point d'eau pour l'assemblée~; et ils se soulevèrent contre Moïse et contre Aaron.
\VS{3}Et le peuple contesta contre Moïse et ils lui dirent~: Pourquoi ne sommes-nous pas morts quand nos frères moururent devant Yahweh~?
\VS{4}Et pourquoi avez-vous fait venir l'assemblée de Yahweh dans ce désert, pour que nous y mourions, nous et notre bétail\FTNT{Ex. 17:3.}~?
\VS{5}Et pourquoi nous avez-vous fait monter hors d'Egypte, pour nous amener dans ce méchant lieu qui n'est pas un lieu où l'on puisse semer, ni un lieu pour des figuiers, ni pour des vignes, ni pour des grenadiers, et sans eau pour boire~?
\VS{6}Alors Moïse et Aaron se retirèrent de devant l'assemblée à l'entrée de la tente d'assignation et ils tombèrent sur leurs faces~; et la gloire de Yahweh apparut.
\TextTitle{Incrédulité de Moïse et d'Aaron à Meriba}
\VS{7}Yahweh parla à Moïse, en disant~:
\VS{8}Prends la verge, et convoque l'assemblée, toi et Aaron, ton frère. Vous parlerez en leur présence au rocher\FTNT{Christ, le rocher des âges ( Es. 8:13-17~; 1 Co. 10:1-4).}, et il donnera son eau~; ainsi tu leur feras sortir de l'eau du rocher, et tu donneras à boire à l'assemblée et à leur bétail.
\VS{9}Moïse prit la verge qui était devant Yahweh, comme il lui avait ordonné.
\VS{10}Moïse et Aaron convoquèrent l'assemblée devant le rocher. Et il leur dit~: Ecoutez donc, rebelles~! Est-ce de ce rocher que nous vous ferons sortir de l'eau~?
\VS{11}Puis Moïse leva sa main, et frappa deux fois le rocher avec sa verge et il en sortit des eaux en abondance. L'assemblée but, et leur bétail aussi.
\VS{12}Alors Yahweh dit à Moïse et à Aaron~: Parce que vous n'avez pas cru en moi, pour me sanctifier aux yeux des enfants d'Israël, ainsi vous ne ferez point entrer cette assemblée dans le pays que je lui donne.
\VS{13}Ce sont là, les eaux de Meriba, où les enfants d'Israël contestèrent avec Yahweh, qui fut sanctifié en eux.
\TextTitle{La méchanceté d'Edom\FTNTT{Ge. 25:30~; Ab. 10.}}
\VS{14}Puis Moïse envoya des ambassadeurs de Kadès au roi d'Edom, pour lui dire~: Ainsi parle ton frère Israël~: Tu sais toutes les souffrances que nous avons eu.
\VS{15}Comment nos pères descendirent en Egypte, où nous avons demeuré longtemps~; et comment les Egyptiens nous ont maltraités, nous et nos pères.
\VS{16}Et nous avons crié à Yahweh, et il a entendu nos cris. Il a envoyé l'Ange et nous a retirés d'Egypte. Et voici, nous sommes à Kadès, ville qui est à l'extrémité de ton territoire\FTNT{Ex. 2:23~; Ex. 23:20~; Ac. 7:30-38.}.
\VS{17}Je te prie, laisse-nous passer par ton pays~; nous ne traverserons ni les champs ni les vignes, et nous ne boirons l'eau d'aucun puits~; nous marcherons par le chemin royal~; nous ne nous détournerons ni à droite ni à gauche, jusqu'à ce que nous ayons passé ton territoire.
\VS{18}Et Edom lui dit~: Tu ne passeras point par mon pays, de peur que je ne sorte en armes à ta rencontre.
\VS{19}Les enfants d'Israël lui répondirent~: Nous monterons par le grand chemin, et si nous buvons de tes eaux, moi et mes bêtes, je t'en payerai le prix~; je veux seulement passer à pied.
\VS{20}Mais il lui répondit~: Tu ne passeras pas~! Et sur cela, Edom sortit à sa rencontre avec une grande multitude, et à main armée.
\VS{21}Ainsi Edom ne voulut point permettre à Israël de passer par ses frontières~; c'est pourquoi Israël se détourna de lui.
\VS{22}Et toute l'assemblée des enfants d'Israël partit de Kadès et arriva à la montagne de Hor.
\TextTitle{Mort d'Aaron}
\VS{23}Et Yahweh parla à Moïse et à Aaron à la montagne de Hor, près des frontières du pays d'Edom, en disant~:
\VS{24}Aaron sera recueilli auprès de son peuple, car il n'entrera pas dans le pays que je donne aux enfants d'Israël, parce que vous avez été rebelles à mon commandement aux eaux de la dispute\FTNT{«~Meriba~»}.
\VS{25}Prends donc Aaron et Eléazar, son fils, et fais-les monter sur la montagne de Hor.
\VS{26}Puis fais dépouiller Aaron de ses vêtements, et fais-les revêtir à Eléazar, son fils. C'est là qu'Aaron sera recueilli et qu'il mourra.
\VS{27}Moïse fit ce que Yahweh avait ordonné~; et ils montèrent sur la montagne de Hor, aux yeux de toute l'assemblée.
\VS{28}Et Moïse dépouilla Aaron de ses vêtements et en fit revêtir Eléazar, son fils. Aaron mourut là, au sommet de la montagne. Moïse et Eléazar descendirent de la montagne\FTNT{De. 10:6.}.
\VS{29}Toute l'assemblée, toute la maison d'Israël, voyant qu'Aaron était mort, le pleurèrent trente jours.
\Chap{21}
\TextTitle{Les Cananéens livrés à Israël}
\VerseOne{}Quand le roi d'Arad, Cananéen, qui habitait le midi, eut appris qu'Israël venait par le chemin d'Atharim, il combattit Israël et emmena des prisonniers.
\VS{2}Alors Israël fit un vœu à Yahweh, en disant~: Si tu livres ce peuple entre mes mains, je dévouerai ses villes par le moyen de l'interdit.
\VS{3}Et Yahweh exauça la voix d'Israël et livra entre ses mains les Cananéens. On les dévoua par interdit, avec leurs villes~; et on donna à ce lieu le nom de Horma.
\TextTitle{Le serpent d'airain\FTNTT{Jn. 3:14-15~; 2 Co. 5:20.}}
\VS{4}Puis ils partirent de la montagne de Hor, par le chemin de la Mer Rouge, pour faire le tour du pays d'Edom. Le cœur du peuple s'impatienta en route,
\VS{5}et parla contre Dieu, et contre Moïse, en disant~: Pourquoi nous as-tu fait monter hors d'Egypte, pour mourir dans ce désert~? Car il n'y a point de pain ni d'eau, et notre âme est dégoûtée de cette nourriture misérable.
\VS{6}Et Yahweh envoya contre le peuple des serpents brûlants qui mordaient le peuple~; tellement qu'il en mourut un grand nombre en Israël\FTNT{1 Co. 10:9.}.
\VS{7}Alors le peuple vint vers Moïse, et dit~: Nous avons péché, car nous avons parlé contre Yahweh et contre toi. Invoque Yahweh afin qu'il éloigne de nous les serpents et Moïse pria pour le peuple.
\VS{8}Et Yahweh dit à Moïse~: Fais-toi un serpent brûlant, et mets-le sur une perche~; quiconque aura été mordu et le regardera conservera la vie.
\VS{9}Moïse fit un serpent d'airain\FTNT{Voir Jn. 3:14-16. Ceux qui regardent à Jésus-Christ, et non aux hommes, obtiennent la délivrance. L'airain nous parle du jugement (Job 20:24), le serpent de la malédiction (Ge. 3:14), et la perche parle de la croix (1 Co. 1:18). Jésus a pris nos malédictions sur la croix de Golgotha (Ga. 3:13).}, et le mit sur une perche~; quiconque avait été mordu par un serpent et regardait le serpent d'airain conservait la vie.
\VS{10}Les enfants d'Israël partirent et campèrent à Oboth.
\VS{11}Et ils partirent d'Oboth et ils campèrent en Ijjé-Abarim, dans le désert qui est vis-à-vis de Moab, vers le soleil levant.
\VS{12}Puis ils partirent de là et campèrent vers le torrent de Zéred.
\VS{13}Et ils partirent de là et campèrent de l'autre côté de l'Arnon, qui est dans le désert, en sortant du territoire des Amoréens~; car l'Arnon est la frontière de Moab, entre les Moabites et les Amoréens\FTNT{Jg. 11:18.}.
\VS{14}C'est pourquoi il est dit dans le livre des batailles de Yahweh~: Vaheb en Supha, et les torrents de l'Arnon,
\VS{15}et le cours des torrents qui s'étend du côté d'Ar et touche à la frontière de Moab.
\VS{16}De là ils allèrent à Beer. C'est là le puit où Yahweh dit à Moïse~: Rassemble le peuple, et je leur donnerai de l'eau.
\VS{17}Alors Israël chanta ce cantique~: Monte, puits~! Chantez-lui en vous répondant les uns aux autres.
\VS{18}Puits que des princes ont creusés. Que les grands du peuple ont creusé, avec le législateur, avec leurs bâtons~! Du désert ils vinrent à Matthana~;
\VS{19}de Matthana à Nahaliel~; et de Nahaliel à Bamoth~;
\VS{20}de Bamoth à la vallée qui est dans le territoire de Moab, au sommet de Pisga, et qui regarde vers Jeshimon.
\TextTitle{Israël bat le roi des Amoréens et le roi de Basan}
\VS{21}Puis Israël envoya des messagers à Sihon, roi des Amoréens, pour lui dire~:
\VS{22}Laisse-moi passer par ton pays~; nous ne nous détournerons ni dans les champs, ni dans les vignes, et nous ne boirons l'eau d'aucun des puits~; mais nous marcherons par la route royale, jusqu'à ce que nous ayons passé ton territoire.
\VS{23}Mais Sihon ne permit pas à Israël de passer sur son territoire~; il rassembla tout son peuple et sortit à la rencontre d'Israël, dans le désert~; il vint à Jahats, et combattit Israël\FTNT{De. 2:26-30~; Jg. 11:29-30.}.
\VS{24}Israël le fit passer au fil de l'épée et conquit son pays, depuis l'Arnon jusqu'à Jabbok, et jusqu'à la frontière des fils d'Ammon~; car la frontière des fils d'Ammon était forte\FTNT{De. 2:30~; De. 29:7~; Ps. 135:11-12.}.
\VS{25}Et Israël prit toutes les villes qui étaient là, et habitat dans toutes les villes des Amoréens, à Hesbon, et dans toutes les villes de son ressort.
\VS{26}Or Hesbon était la ville de Sihon, roi des Amoréens, qui avait le premier fait la guerre au roi de Moab, et pris sur lui tout son pays jusqu'à l'Arnon.
\VS{27}C'est pourquoi les poètes disent~: Venez à Hesbon~! Que la ville de Sihon soit rebâtie et fortifiée~!
\VS{28}Car le feu est sorti de Hesbon, et la flamme de la cité de Sihon~; elle a consumé Ar-Moab, les habitants des hauteurs de l'Arnon.
\VS{29}Malheur à toi, Moab~! Peuple de Kemosch, tu es perdu~! Il a livré ses fils qui se sauvaient et ses filles en captivité à Sihon, roi des Amoréens\FTNT{Jé. 48:46.}.
\VS{30}Nous les avons défaits à coups de flèches~: De Hesbon à Dibon tout est détruit~; nous les avons mis en déroute jusqu'à Nophach, jusqu'à Médeba.
\VS{31}Israël s'établit dans le pays des Amoréens.
\VS{32}Puis Moïse envoya des gens pour reconnaître Jaezer, ils prirent les villes de son ressort, et chassèrent les Amoréens qui y étaient.
\VS{33}Ensuite, ils se tournèrent et montèrent par le chemin de Basan. Og, roi de Basan, sortit à leur rencontre, avec tout son peuple pour les combattre à Edréï.
\VS{34}Et Yahweh dit à Moïse~: Ne le crains point, car je le livre entre tes mains, lui et tout son peuple, et son pays~; tu le traiteras comme tu as traité Sihon, roi des Amoréens, qui habitait à Hesbon\FTNT{De. 3:1-2.}.
\VS{35}Ils le battirent donc, lui et ses fils, et tout son peuple, sans en laisser échapper un seul, et ils s'emparèrent de son pays.
\Chap{22}
\TextTitle{Balak cherche à maudire Israël~; Balaam\FTNTT{2 Pi. 2:15~; Jud. 11~; Ap. 2:14} séduit par les honneurs}
\VerseOne{}Puis les enfants d'Israël partirent, et ils campèrent dans les plaines de Moab, au-delà du Jourdain, vis-à-vis de Jéricho.
\VS{2}Balak, fils de Tsippor, vit tout ce qu'Israël avait fait aux Amoréens.
\VS{3}Et Moab eut une grande frayeur du peuple, parce qu'il était en grand nombre, il fut saisi de terreur en face des enfants d'Israël.
\VS{4}Et Moab dit aux anciens de Madian~: Maintenant cette multitude va brouter tout ce qui nous entoure, comme le bœuf broute l'herbe des champs. Balak, fils de Tsippor, était alors roi de Moab.
\VS{5}Il envoya des messagers auprès de Balaam, fils de Beor, à Pethor, située sur le fleuve, dans le pays des fils de son peuple, afin de l'appeler et de lui dire~: Voici, un peuple est sorti d'Egypte, il couvre la surface de la terre, et il habite vis-à-vis de moi.
\VS{6}Viens donc maintenant, je te prie, maudis-moi ce peuple, car il est plus puissant que moi~; peut-être que je serai le plus fort, et que nous le battrons, et que je le chasserai du pays~; car je sais que celui que tu bénis est béni, et que celui que tu maudis est maudit.
\VS{7}Les anciens de Moab s'en allèrent avec les anciens de Madian, ayant dans leurs mains de quoi payer le devin. Ils arrivèrent auprès de Balaam, et lui rapportèrent les paroles de Balak.
\VS{8}Il leur répondit~: Demeurez ici cette nuit, et je vous répondrai d'après ce que Yahweh me dira. Et les chefs des Moabites restèrent chez Balaam.
\VS{9}Et Dieu vint à Balaam et dit~: Qui sont ces hommes que tu as chez toi~?
\VS{10}Et Balaam répondit à Dieu~: Balak, fils de Tsippor, roi de Moab, les a envoyés pour me dire~:
\VS{11}Voici, un peuple qui est sorti d'Egypte, et qui couvre la face de la terre~; viens donc, maudis-le-moi~; peut-être qu'ainsi je pourrai le combattre, et je le chasserai.
\VS{12}Et Dieu dit à Balaam~: Tu n'iras point avec eux, et tu ne maudiras point ce peuple, car il est béni.
\VS{13}Et Balaam se leva le matin, et il dit aux chefs qui avaient été envoyés par Balak~: Retournez dans votre pays, car Yahweh refuse de me laisser venir avec vous.
\VS{14}Ainsi les chefs des Moabites se levèrent et retournèrent auprès de Balak, et dirent~: Balaam a refusé de venir avec nous.
\VS{15}Et Balak envoya encore des chefs en plus grand nombre, et plus considérés que les premiers.
\VS{16}Ils arrivèrent auprès de Balaam, et lui dirent~: Ainsi parle Balak, fils de Tsippor~: Que l'on ne t'empêche donc pas de venir vers moi~;
\VS{17}car je te rendrai beaucoup d'honneur, et je ferai tout ce que tu me diras~; je te prie donc viens, maudis-moi ce peuple.
\VS{18}Et Balaam répondit et dit aux serviteurs de Balak~: Quand Balak me donnerait sa maison pleine d'or et d'argent, je ne pourrais point transgresser l'ordre de Yahweh, mon Dieu~; je ne pourrais faire aucune chose, ni petite ni grande.
\VS{19}Toutefois, je vous prie, demeurez maintenant ici encore cette nuit, et je saurai ce que Yahweh aura de plus à me dire.
\VS{20}Dieu vint, la nuit à Balaam, et lui dit~: Puisque ces hommes sont venus t'appeler, lève-toi, et va avec eux~; mais quoi qu'il en soit, tu feras ce que je te dirai.
\VS{21}Ainsi Balaam se leva le matin, et sella son ânesse, et partit avec les chefs de Moab.
\VS{22}Mais la colère de Dieu s'enflamma parce qu'il était parti~; et l'Ange de Yahweh se plaça sur le chemin pour lui résister. Balaam était monté sur son ânesse, et ses deux serviteurs étaient avec lui.
\VS{23}L'ânesse vit l'Ange de Yahweh qui se tenait sur le chemin, son épée nue dans la main~; elle se détourna du chemin et alla dans les champs. Balaam frappa l'ânesse pour la ramener dans le chemin\FTNT{2 Pi. 2:16~; Jud. 1:11.}.
\VS{24}L'Ange de Yahweh se plaça dans un sentier entre les vignes~; il y avait un mur de chaque côté.
\VS{25}L'ânesse vit l'Ange de Yahweh~; elle se serra contre le mur, et elle serra le pied de Balaam contre le mur. Balaam la frappa de nouveau.
\VS{26}Et l'Ange de Yahweh passa plus loin et s'arrêta dans un lieu étroit où il n'y avait point d'espace pour se détourner à droite ou à gauche.
\VS{27}Et l'ânesse vit l'Ange de Yahweh, et elle s'abattit sous Balaam. Balaam se mit en grande colère, et il frappa l'ânesse avec son bâton.
\VS{28}Alors Yahweh fit parler l'ânesse, et elle dit à Balaam~: Que t'ai-je fait, pour que tu m'aies déjà frappée trois fois~?
\VS{29}Et Balaam répondit à l'ânesse~: C'est parce que tu t'es moquée de moi~; si j'avais une épée dans la main, je te tuerai sur le champ !
\VS{30}Et l'ânesse dit à Balaam: Ne suis-je pas ton ânesse, sur laquelle tu montes depuis que je suis à toi, jusqu'à aujourd'hui~? Ai-je l'habitude de te faire ainsi~? Et il répondit~: Non.
\VS{31}Alors Yahweh ouvrit les yeux de Balaam, et il vit l'Ange de Yahweh qui se tenait sur le chemin, et qui avait dans sa main son épée nue~; et il s'inclina et se prosterna sur son visage.
\VS{32}Et l'Ange de Yahweh lui dit~: Pourquoi as-tu frappé ton ânesse déjà trois fois~? Voici je suis sorti pour m'opposer à toi~; car ta voie est devant moi une voie de perdition.
\VS{33}Mais l'ânesse m'a vu et elle s'est détournée de devant moi déjà trois fois~; autrement, si elle ne s'était détournée de moi, je t'aurais même déjà tué, et je lui aurais laissé la vie.
\VS{34}Alors Balaam dit à l'Ange de Yahweh~: J'ai péché, car je ne savais point que tu t'étais placé au-devant de moi sur le chemin~; et maintenant, si cela te déplaît, je m'en retournerai.
\VS{35}L'Ange de Yahweh dit à Balaam~: Va avec ces hommes~; mais tu ne feras que répéter les paroles que je te dirai. Et Balaam alla avec les chefs envoyés par Balak.
\VS{36}Et quand Balak apprit que Balaam arrivait, il sortit à sa rencontre jusqu'à la ville de Moab, qui est sur la limite de l'Arnon, à l'extrême frontière.
\VS{37}Et Balak dit à Balaam~: N'ai-je pas auparavant envoyé vers toi pour t'appeler~? Pourquoi n'es-tu pas venu vers moi~? Ne puis-je donc pas te traiter avec honneur~?
\VS{38}Et Balaam répondit à Balak~: Je suis venu vers toi~; mais pourrais-je maintenant dire quelque chose~? Je ne dirai que les paroles que Dieu m'aura mis dans la bouche.
\VS{39}Et Balaam alla avec Balak, et ils arrivèrent dans la cité de Kirjath-Hutsoth.
\VS{40}Et Balak sacrifia des bœufs et des brebis, et il en envoya à Balaam et aux chefs qui étaient venus avec lui.
\VS{41}Quand le matin fut venu, il prit Balaam et le fit monter à Bamoth-Baal, et de là il vit une partie du peuple.
\Chap{23}
\TextTitle{Balaam ne maudit pas mais bénit Israël des hauts lieux de Baal}
\VerseOne{}Et Balaam dit à Balak~: Bâtis-moi ici sept autels, et prépare-moi ici sept veaux et sept béliers.
\VS{2}Et Balak fit ce que Balaam avait dit~; et Balak offrit avec Balaam un veau et un bélier sur chaque autel.
\VS{3}Balaam dit à Balak~: Tiens-toi près de ton holocauste, et je m'éloignerai~; peut-être que Yahweh viendra à ma rencontre, et je te rapporterai tout ce qu'il me révélera. Ainsi il se retira à l'écart.
\VS{4}Et Dieu vint au-devant de Balaam, et Balaam lui dit~: J'ai dressé sept autels, et j'ai sacrifié un veau et un bélier sur chaque autel.
\VS{5}Et Yahweh mit des paroles dans la bouche de Balaam et lui dit~: Retourne vers Balak, et tu parleras ainsi.
\VS{6}Il s'en retourna donc vers lui~; et voici, Balak se tenait près de son holocauste, tant lui que tous les chefs de Moab.
\VS{7}Alors Balaam prononça son discours sentencieux et dit~: Balak, roi de Moab, m'a fait descendre d'Aram\FTNT{De l'hébreu «~Aram~» traduit par «~Aram~» ou «~Syrie~» (1 R. 11:25).}, des montagnes d'orient, en me disant~: Viens, maudis-moi Jacob~! Viens, dis-je, déteste Israël~!
\VS{8}Mais comment le maudirai-je~? Dieu ne l'a point maudit. Et comment le détesterai-je~? Yahweh ne l'a point détesté.
\VS{9}Car je le regarderai du sommet des rochers, et je le contemplerai du haut des collines~: Voici, ce peuple habitera à part, et il ne sera pas compté parmi les nations\FTNT{De. 33:28.}.
\VS{10}Qui comptera la poussière de Jacob, et dira le nombre du quart d'Israël~? Que je meure de la mort des justes, et que ma fin soit semblable à la leur~!
\VS{11}Alors Balak dit à Balaam~: Que m'as-tu fait~? Je t'ai pris pour maudire mes ennemis, et voici, tu les a bénis, tu les a bénis\FTNT{Le verbe bénir vient de l'hébreu «~Barak~», il est utilisé deux fois de suite dans ce passage. Voir commentaire en Ge. 2:16-17.}.~!
\VS{12}Et il répondit, et dit~: Ne prendrais-je pas garde de dire les paroles que Yahweh aura mis dans ma bouche~?
\TextTitle{Balaam bénit Israël au sommet de Pisga}
\VS{13}Alors Balak lui dit~: Viens, je te prie, avec moi dans un autre lieu, d'où tu pourras le voir, car tu en voyais seulement une extrémité, et tu ne le voyais pas tout entier~; maudis-le moi de là.
\VS{14}Puis l'ayant conduit au territoire de Tsophim, sur le sommet de Pisga~; il bâtit sept autels, et offrit un taureau et un bélier sur chaque autel.
\VS{15}Alors Balaam dit à Balak~: Tiens-toi ici près de ton holocauste, et je m'en irai à la rencontre de Dieu, comme j'ai déjà fait.
\VS{16}Yahweh donc vint au-devant de Balaam, il mit des paroles dans sa bouche et lui dit~: Retourne vers Balak, et tu parleras ainsi.
\VS{17}Il retourna vers Barak~; et voici, il se tenait près de son holocauste, et les chefs de Moab avec lui. Et Balak lui dit~: Qu'est-ce que Yahweh a dit~?
\VS{18}Alors il prononça son discours sentencieux et dit~: Lève-toi, Balak, écoute~! Fils de Tsippor, prête-moi l'oreille~!
\VS{19}Dieu n'est point un homme pour mentir ni fils d'un homme pour se repentir. Ce qu'il a dit, ne le fera-t-il pas~? Ce qu'il a déclaré, ne l'exécutera-t-il pas\FTNT{Ja. 1:17.}~?
\VS{20}Voici, j'ai reçu la parole pour bénir~: Puisqu'il a béni, je ne le révoquerai point.
\VS{21}Il n'a point aperçu d'iniquité en Jacob, il ne voit point de perversité en Israël~; Yahweh, son Dieu, est avec lui, et il y a en lui un chant de triomphe royal\FTNT{Jé. 50:20~; Ro. 4:7.}.
\VS{22}Dieu les a tirés d'Egypte, il est pour eux comme la vigueur du buffle.
\VS{23} Car il n'y a pas d'enchantement contre Jacob, ni la divination contre Israël. Au temps marqué, il sera dit à Jacob et à Israël~: Qu'est-ce que Dieu a fait~?
\VS{24}Voici, ce peuple se lèvera comme un vieux lion, et se dressera comme un lion qui est dans sa force~; il ne se couchera pas jusqu'à ce qu'il ait dévoré la proie, et bu le sang des blessés à mort.
\VS{25}Balak dit à Balaam~: Et bien~! Ne le maudis pas, mais du moins ne le bénis pas.
\VS{26}Et Balaam répondit à Balak~: Ne t'ai-je pas dit que tout ce que Yahweh dira, je le ferai~? 
\TextTitle{Balaam bénit Israël de Peor}
\VS{27}Balak dit encore à Balaam~: Viens maintenant, je te conduirai dans un autre lieu~; peut-être que Dieu trouvera bon que tu me le maudisses de là.
\VS{28}Balak conduisit donc Balaam sur le sommet de Peor, qui regarde du côté de Jeshimon.
\VS{29}Et Balaam lui dit~: Bâtis-moi ici sept autels, et apprête-moi ici sept veaux et sept béliers.
\VS{30}Et Balak fit donc comme Balaam lui avait dit~; puis il offrit un taureau et un bélier sur chaque autel.
\Chap{24}
\VerseOne{}Or Balaam, voyant que Yahweh voulait bénir Israël, n'alla plus comme les autres fois chercher des enchantements~; mais il tourna son visage du côté du désert.
\VS{2}Et Balaam leva les yeux, il vit Israël qui se tenait rangé selon ses tribus. Alors l'Esprit de Dieu fut sur lui.
\VS{3}Et il prononça à haute voix son discours sentencieux et dit~: Balaam, fils de Beor, dit, et l'homme qui a l'œil ouvert dit:
\VS{4}Celui qui entend les paroles de Dieu, qui voit la vision du Tout-Puissant, qui tombe à terre, et qui a les yeux ouverts dit~:
\VS{5}Que tes tentes sont belles, ô Jacob~! Et tes tabernacles, ô Israël~!
\VS{6}Ils sont étendus comme des torrents, comme des jardins près d'un fleuve, comme des arbres d'aloès que Yahweh a plantés, comme des cèdres auprès des eaux.
\VS{7}L'eau coule de ses seaux, et sa semence est parmi d'abondantes eaux. Et son roi s'élève au-dessus d'Agag, et son royaume sera haut élevé.
\VS{8}Dieu, qui l'a tiré d'Egypte, est pour lui comme la vigueur du buffle~; il consumera les nations qui sont ses ennemies~; il brisera leurs os, et les percera de ses flèches.
\VS{9}Il s'est courbé, il s'est couché comme un lion qui est dans sa force, et comme un vieux lion~; qui le réveillera~? Quiconque te bénit, sera béni, et quiconque te maudit, sera maudit.
\VS{10}Alors Balak se mit très en colère contre Balaam, il frappa des mains et Balak parla ainsi à Balaam~: Je t'ai appelé pour maudire mes ennemis, et voici, tu les as bénis, tu les a bénis\FTNT{voir le commentaire en Ge. 2:16-17.} trois fois déjà.
\VS{11}Et maintenant, fuis dans ton pays~! J'avais dit que je t'honorerais, je t'honorerais\FTNT{Le mot hébreu est utilisé deux fois dans ce passage. Voir le commentaire en Ge. 2:17.}, mais Yahweh t'empêche d'être honoré.
\VS{12}Et Balaam répondit à Balak~: N'ai-je pas dit à tes messagers que tu m'as envoyés~:
\VS{13}Quand Balak me donnerait sa maison pleine d'argent et d'or, je ne pourrais transgresser l'ordre de Yahweh pour faire de moi-même du bien ou du mal~; mais ce que Yahweh dira, je le dirai.
\VS{14}Maintenant donc je m'en vais vers mon peuple. Viens, je te donnerai un conseil, et je te dirai ce que ce peuple fera à ton peuple, dans les derniers jours.
\TextTitle{Prophétie sur le Roi qui sort de Jacob, le Messie}
\VS{15}Alors il prononça son discours sentencieux et dit~: Balaam, fils de Beor dit, et l'homme qui a l'œil ouvert dit~:
\VS{16}Celui qui entend les paroles de Dieu, qui connaît la science du Très-Haut, qui voit la vision du Tout-Puissant, qui tombe à terre, et qui a les yeux ouverts.
\VS{17}Je le vois, mais non pas maintenant~; je le regarde, mais non pas de près~; une Etoile est sortie de Jacob\FTNT{L'Etoile en question est Jésus-Christ, qui se révéla à Jean comme l'Etoile brillante du matin (Ap. 22:16).}, et un Sceptre s'est élevé d'Israël. Il transpercera les cotés de Moab, il détruira tous les enfants de Seth.
\VS{18}Edom sera sa possession, Séir sera possédé par ses ennemis, et Israël se portera vaillamment.
\VS{19}Et il y en aura un de Jacob qui dominera, il fera périr le reste de la ville.
\VS{20}Il vit aussi Amalek, il prononça son discours sentencieux et dit~: Amalek est le premier des nations, mais à la fin il sera détruit.
\VS{21}Il vit aussi les Kéniens\FTNT{Il y a plusieurs sens à ce mot~:
\\Caïn = «~possession~», «~artisan, forgeron~», fils d'Adam.
\\Kéniens = «~forgerons~» tribu du beau-père de Moïse qui vivait dans la région du sud de la Palestine.}. Il prononça à haute voix son discours sentencieux et dit~: Ta demeure est dans un lieu solide, et tu as mis ton nid dans le rocher~;
\VS{22}toutefois, le Kénien sera consumé, jusqu'à ce que l'Assyrien t'emmène en captivité.
\VS{23}Il continua à prononcer à haute voix son discours sentencieux, et il dit~: Malheur à celui qui vivra quand Dieu fera ces choses.
\VS{24}Et des navires viendront de Kittim, et ils humilieront l'Assyrien et l'Hébreu~; et lui aussi sera détruit.
\VS{25}Puis Balaam se leva, et s'en alla pour retourner chez lui. Balak aussi s'en alla son chemin.
\Chap{25}
\TextTitle{Prostitution d'Israël à Baal-Peor\FTNTT{No. 31:16~; Ja. 4:4~; Ap. 2:14.}}
\VerseOne{}Alors Israël demeurait à Sittim~; et le peuple commença à commettre la fornication avec les filles de Moab.
\VS{2}Car elles convièrent le peuple aux sacrifices de leurs dieux~; et le peuple mangea et se prosterna devant leurs dieux.
\VS{3}Et Israël s'accoupla à Baal-Peor, c'est pourquoi la colère de Yahweh s'enflamma contre Israël\FTNT{Ps. 106:28~; Os. 9:10.}.
\VS{4}Et Yahweh dit à Moïse~: Prends tous les chefs du peuple, et fais-les pendre devant Yahweh en face du soleil, afin que la colère de Yahweh se détourne d'Israël\FTNT{De. 4:3~; Jos. 22:17.}.
\VS{5}Moïse donc dit aux juges d'Israël~: Que chacun de vous fasse mourir les hommes qui sont à sa charge, et qui se sont joints à Baal-Peor.
\VS{6}Et voici, un homme des enfants d'Israël vint, et amena à ses frères une Madianite, devant Moïse et devant toute l'assemblée des fils d'Israël, tandis qu'ils pleuraient à l'entrée de la tente d'assignation.
\VS{7}Ce que Phinées, fils d'Eléazar, fils d'Aaron le prêtre, ayant vu, se leva du milieu de l'assemblée et prit une lance dans sa main.
\VS{8}Et il entra dans la tente de l'homme Israélite et les transperça tous deux, l'homme Israélite puis la femme, par le ventre. Et la plaie s'arrêta parmi les enfants d'Israël\FTNT{Ps. 106:30.}.
\VS{9}Or il y en eut vingt-quatre mille qui moururent de cette plaie.
\VS{10}Et Yahweh parla à Moïse, en disant~:
\VS{11}Phinées, fils d'Eléazar, fils d'Aaron, le prêtre, a détourné ma colère de dessus les enfants d'Israël, parce qu'il a été animé de mon zèle au milieu d'eux~; et je n'ai point, dans mon ardeur, consumé les fils d'Israël.
\VS{12}C'est pourquoi, dis-lui~: Voici, je lui donne mon alliance de paix.
\VS{13}Et l'alliance de prêtrise perpétuelle sera tant pour lui que pour sa postérité après lui, parce qu'il a été animé de zèle pour son Dieu, et qu'il a fait propitiation pour les enfants d'Israël.
\VS{14}Et le nom de l'homme Israélite tué, lequel fut tué avec la Madianite, était Zimri, fils de Salu, chef d'une maison de père des Siméonites.
\VS{15}Et le nom de la femme Madianite qui fut tuée était Cozbi, fille de Tsur, chef du peuple, et d'une maison de père en Madian.
\VS{16}Yahweh parla à Moïse, en disant~:
\VS{17}Mettez en détresse les Madianites, tuez-les~;
\VS{18}car ils vous ont serrés les premiers par leurs ruses, par lequelles ils vous surpris dans l'affaire de Peor, et dans l'affaire de Cozbi, fille d'un chef d'entre les Madianites, leur sœur, qui a été tuée le jour de la plaie, causée par l'affaire de Peor.
\Chap{26}
\TextTitle{Nouveau dénombrement des hommes de guerre}
\VerseOne{} Or il arriva qu'après cette plaie-là, que Yahweh parla à Moïse, et à Eléazar, fils d'Aaron, le prêtre, en disant~:
\VS{2}Faites le dénombrement de toute l'assemblée des enfants d'Israël, depuis l'âge de vingt ans et au-dessus, selon les maisons de leurs pères, à savoir de tous ceux d'Israël qui peuvent aller à la guerre.
\VS{3}Moïse donc et Eléazar, le prêtre, leur parlèrent donc dans les plaines de Moab, près du Jourdain de Jéricho, en disant~:
\VS{4}Qu'on fasse le dénombrement depuis l'âge de vingt ans et au-dessus, comme Yahweh l'avait ordonné à Moïse et aux enfants d'Israël, quand ils furent sortis du pays d'Egypte.
\VS{5}Ruben, premier-né d'Israël. Fils de Ruben~: Hénoc, de qui descend la famille des Hénokites~; Pallu, de qui descend la famille des Palluites~;
\VS{6}Hetsron, de qui descend la famille des Hetsronites~; Carmi, de qui descend la famille des Carmites.
\VS{7}Ce sont là les familles des Rubénites~: Ceux qui furent dénombrés étaient quarante-trois mille sept cent trente.
\VS{8}Et les fils de Pallu~: Eliab.
\VS{9}Fils d'Eliab~: Nemuel, Dathan et Abiram. Ce Dathan et cet Abiram, qui étaient de ceux qu'on appelait pour tenir l'assemblée, et qui se révoltèrent contre Moïse et contre Aaron dans l'assemblée de Koré, lors de leur révolte contre Yahweh.
\VS{10}Et lorsque la terre ouvrit sa bouche et les engloutit, ainsi que Koré, ceux qui s'étaient assemblés avec lui moururent. Et le feu dévora les deux cent cinquante hommes qui servirent d'avertissement.
\VS{11}Mais les fils de Koré ne moururent pas.
\VS{12}Les fils de Siméon selon leurs familles~: De Nemuel descend la famille des Némuélites~; de Jamin, la famille des Jaminites~; de Jakin, la famille des Jakinites~;
\VS{13}de Zérach, la famille des Zérachites~; de Saül, la famille des Saülites.
\VS{14}Ce sont là les familles des Siméonites, qui furent vingt-deux mille deux cents.
\VS{15}Fils de Gad selon leurs familles. De Tsephon, descend la famille des Tsephonites~; de Haggi, la famille des Haggites~; de Schuni, la famille des Schunites~;
\VS{16}d'Ozni, la famille des Oznites~; d'Eri, la famille des Erites~;
\VS{17}d'Arod, la famille des Arodites~; d'Areéli, la famille des Areélites.
\VS{18}Ce sont là les familles des fils de Gad, d'après leur dénombrement~: Quarante mille cinq cents.
\VS{19}Fils de Juda, Er, et Onan~; mais Er et Onan moururent au pays de Canaan\FTNT{Ge. 38:7-10~; Ge. 46:12.}.
\VS{20}Voici les fils de Juda selon leurs familles~: De Schéla descend la famille des Schélanites~; de Pérets, la famille des Péretsites~; de Zérach, la famille des Zérachites.
\VS{21}Les fils de Pérets furent~: Hetsron, de qui descend la famille des Hetsronites~; Hamul, de qui descend la famille des Hamulites.
\VS{22}Ce sont là les familles de Juda, selon leur dénombrement: Soixante-seize mille cinq cents.
\VS{23}Fils d'Issacar, selon leurs familles~: De Thola descend la famille des Tholaïtes~; de Puva, la famille des Puvites~;
\VS{24}de Jaschub, la famille des Jaschubites~; de Schimron, la famille des Schimronites.
\VS{25}Ce sont là les familles d'Issacar, d'après leur dénombrement~: Soixante-quatre mille trois cents.
\VS{26}Fils de Zabulon, selon leurs familles~: De Séred, descend la famille des Sardites~; d'Elon, la famille des Elonites~; de Jahleel, la famille des Jahleélites.
\VS{27}Ce sont là les familles des Zabulonites, d'après leur dénombrement~: Soixante mille cinq cents.
\VS{28}Fils de Joseph, selon leurs familles~: Manassé et Ephraïm.
\VS{29}Fils de Manassé. De Makir descend la famille des Makirites. Makir engendra Galaad. De Galaad descend la famille des Galaadites.
\VS{30}Voici les fils de Galaad~: Jézer, de qui descend la famille des Jézerites~; Hélek, la famille des Hélekites.
\VS{31}Asriel, la famille des Asriélites~; Sichem, la famille des Sichémites~;
\VS{32}Schemida, la famille des Schemidaïtes~; Hépher, la famille des Héphrites.
\VS{33}Tselophchad, fils de Hépher, n'eut point de fils, mais des filles. Voici les noms des filles de Tselophchad~: Machla, Noa, Hogla, Milca, et Thirtsa.
\VS{34}Ce sont là les familles de Manassé, d'après leur dénombrement~: Cinquante-deux mille sept cents.
\VS{35}Voici les fils d'Ephraïm, selon leurs familles~: De Schutélach descend la famille des Schutalchites~; de Béker, la famille des Bakrites~; de Thachan, la famille des Thachanites.
\VS{36}Voici les fils de Schutélach~: D'Eran est descendue la famille des Eranites.
\VS{37}Ce sont là les familles des fils d'Ephraïm, d'après leur dénombrement~: Trente-deux mille cinq cents. Ce sont là les fils de Joseph, selon leurs familles.
\VS{38}Fils de Benjamin, selon leurs familles~: De Béla descend la famille des Balites~; d'Aschbel, la famille des Aschbélites~; d'Achiram, la famille des Achiramites~;
\VS{39}De Schupham, la famille des Schuphamites~; de Hupham, la famille des Huphamites.
\VS{40}Les fils de Béla furent Ard et Naaman. D'Ard descend la famille des Ardites~; et de Naaman la famille des Naamanites.
\VS{41}Ce sont là les fils de Benjamin, d'après leurs familles~; et leur dénombrement~: Quarante-cinq mille six cents.
\VS{42}Voici les fils de Dan, selon leurs familles~: De Schucham descend la famille des Schuchamites. Ce sont là les familles de Dan, selon leurs familles.
\VS{43}Toutes les familles des Schuchamites, selon leur dénombrement~: Soixante-quatre mille quatre cents.
\VS{44}Fils d'Aser, selon leurs familles~: De Jimna descend la famille des Jimnites~; de Jischvi, la famille des Jischvites~; de Beria la famille des Beriites.
\VS{45}Des fils de Beria descendent~: De Héber, la famille des Hébrites~; de Malkiel, la famille des Malkiélites.
\VS{46}Et le nom de la fille d'Aser était Sérach.
\VS{47}Ce sont là les familles des fils d'Aser, d'après leur dénombrement~: Cinquante-trois mille quatre cents.
\VS{48}Fils de Nephthali, selon leurs familles~: De Jahtseel descend la famille des Jahtseélites~; de Guni, la famille des Gunites~;
\VS{49}de Jetser la famille des Jitsrites~; de Schillem, la famille des Schillémites.
\VS{50}Ce sont là les familles de Nephthali, selon leurs familles, et leur dénombrement~: Quarante-cinq mille quatre cents.
\VS{51}Voici les dénombrés des fils d'Israël, qui furent six cent un mille sept cent trente.
\VS{52}Yahweh parla à Moïse, en disant~:
\VS{53}Le pays sera partagé entre ceux-ci en héritage, selon le nombre des noms.
\VS{54}A ceux qui sont en plus grand nombre, tu donneras plus d'héritage, et à ceux qui sont en plus petit nombre tu donneras moins d'héritage~; on donnera à chacun son héritage selon le nombre de ses dénombrés.
\VS{55}Toutefois, que le pays soit partagé par le sort~; et qu'ils prennent leur héritage selon les noms des tribus de leurs pères\FTNT{Jos. 11:23~; Jos. 14:2~; Jos. 18:6-8.}.
\VS{56}L'héritage de chacun sera selon que le sort le montrera, et on aura égard au plus grand et au plus petit nombre.
\VS{57}Et ce sont ici les dénombrés de Lévi selon leurs familles~; de Guerschon, la famille des Guerschonites~; de Kehath, la famille des Kehathites~; de Merari, la famille des Merarites.
\VS{58}Ce sont ici les familles de Lévi~; la famille des Libnites, la famille des Hébronites, la famille des Machlites, la famille des Muschites, la famille des Korites. Kehath engendra Amram.
\VS{59}Et le nom de la femme d'Amram était Jokébed, fille de Lévi, qui naquit à Lévi en Egypte~; et elle enfanta à Amram: Aaron, Moïse, et Marie, leur sœur.
\VS{60}Et il naquit à Aaron~: Nadab et Abihu, Eléazar et Ithamar.
\VS{61}Nadab et Abihu moururent lorsqu'ils apportèrent du feu étranger devant Yahweh\FTNT{Lé. 10:1-2~; 1 Ch. 24:2.}.
\VS{62}Et tous les dénombrés des Lévites furent vingt-trois mille, tous mâles, depuis l'âge d'un mois, et au dessus, qui ne furent point dénombrés avec les autres enfants d'Israël, car on ne leur donna point d'héritage entre les enfants d'Israël.
\VS{63}Ce sont là ceux qui furent dénombrés par Moïse et Eléazar, le prêtre, qui firent le dénombrement des fils d'Israël dans les plaines de Moab, près du Jourdain de Jéricho.
\VS{64}Entre lesquels il ne s'en trouva aucun de ceux qui avaient été dénombrés par Moïse et Aaron le prêtre, quand ils firent le dénombrement des enfants d'Israël au désert de Sinaï.
\VS{65}Car Yahweh avait dit d'eux~: ils mourront certainement dans le désert, et qu'ainsi il n'en restera pas un, excepté Caleb, fils de Jephunné, et Josué, fils de Nun\FTNT{1 Co. 10:5.}.
\Chap{27}
\TextTitle{Loi sur les héritages\FTNTT{No. 36.}}
\VerseOne{}Or les filles de Tselophchad, fils de Hépher, fils de Galaad, fils de Makir, fils de Manassé, d'entre les familles de Manassé, fils de Joseph, s'approchèrent~; et ce sont ici les noms de ses filles~: Machla, Noa, Hogla, Milca, et Thirtsa.
\VS{2}Elles se présentèrent devant Moïse, devant Eléazar, le prêtre, et devant les princes et toute l'assemblée, à l'entrée de la tente d'assignation. Elles dirent~:
\VS{3}Notre père est mort dans le désert~; il n'était toutefois pas dans la troupe de ceux qui s'assemblèrent contre Yahweh, dans l'assemblée de Koré, mais il est mort dans son péché, et il n'avait point de fils.
\VS{4}Pourquoi le nom de notre père serait-il retranché de sa famille, parce qu'il n'a point eu de fils~? Donne-nous une possession parmi les frères de notre père.
\VS{5}Moïse rapporta leur cause devant Yahweh.
\VS{6}Et Yahweh parla à Moïse, en disant~:
\VS{7}Les filles de Tselophchad ont parlé droitement. Tu ne manqueras pas de leur donner un héritage à posséder parmi les frères de leur père, et tu leur feras passer l'héritage de leur père.
\VS{8}Tu parleras aussi aux enfants d'Israël, et tu leur diras~: Lorsqu'un homme mourra sans avoir de fils, vous ferez passer son héritage à sa fille.
\VS{9}S'il n'a pas de fille, vous donnerez son héritage à ses frères.
\VS{10}S'il n'a pas de frères, vous donnerez son héritage aux frères de son père.
\VS{11}Et si son père n'a pas de frère, vous donnerez son héritage à son parent le plus proche de sa famille, et il le possédera. Et ce sera pour les enfants d'Israël une ordonnance de droit, comme Yahweh l'a ordonné à Moïse.
\TextTitle{Moïse voit de loin le pays promis aux fils d'Israël}
\VS{12}Yahweh dit aussi à Moïse~: Monte sur cette montagne d'Abarim, et regarde le pays que je donne aux enfants d'Israël\FTNT{De. 32:48-49.}.
\VS{13}Tu le regarderas donc~; et puis tu seras toi aussi recueilli auprès de ton peuple, comme Aaron ton frère y a été recueilli~;
\VS{14}parce que vous avez été rebelles à mon ordre dans le désert de Tsin, lors de la contestation de l'assemblée, vous ne m'avez point sanctifié au sujet des eaux devant eux~; ce sont les eaux de Meriba, à Kadès, dans le désert de Tsin.
\TextTitle{Yahweh désigne Josué comme successeur de Moïse}
\VS{15}Moïse parla à Yahweh, en disant~:
\VS{16}Que Yahweh, le Dieu des esprits de toute chair, établisse sur l'assemblée un homme\FTNT{Hé. 12:9.},
\VS{17}qui sorte devant eux et qui entre devant eux, et qui les fasse sortir et qui les fasse entrer, afin que l'assemblée de Yahweh ne soit pas comme des brebis qui n'ont point de berger\FTNT{1 R. 22:17~; Mt. 9:36~; Mc. 6:34.}.
\VS{18}Alors Yahweh dit à Moïse~: Prends Josué, fils de Nun, un homme en qui est l'Esprit, et tu poseras ta main sur lui\FTNT{De. 34:9.}.
\VS{19}Tu le présenteras devant Eléazar, le prêtre, et devant toute l'assemblée~; et tu lui donneras des instructions sous leurs yeux.
\VS{20}Et tu lui feras part de ton autorité, afin que toute l'assemblée des enfants d'Israël l'écoute.
\VS{21}Et il se présentera devant Eléazar, le prêtre, qui consultera pour lui les jugements de l'urim\FTNT{Lé. 8:8.} devant Yahweh~; et à sa parole ils sortiront, et à sa parole ils entreront, lui, les enfants d'Israël, avec lui, et toute l'assemblée.
\VS{22}Moïse donc fit comme Yahweh lui avait ordonné. Il prit Josué et le présenta devant Eléazar, le prêtre, et devant toute l'assemblée.
\VS{23}Puis il posa ses mains sur lui, et lui donna des instructions, comme Yahweh l'avait dit par Moïse.
\Chap{28}
\TextTitle{Consignes relatives au temps des sacrifices}
\VerseOne{}Yahweh parla à Moïse, en disant~:
\VS{2}Donne cet ordre aux enfants d'Israël, et dis-leur~: Vous aurez soin de m'offrir en leur temps, mon offrande, ma nourriture, pour mes sacrifices consumés par le feu, qui me sont d'une bonne odeur\FTNT{Lé. 3:11~; Lé. 21:6.}.
\VS{3}Tu leur diras~: Voici le sacrifice consumé par le feu que vous offrirez à Yahweh~: Deux agneaux d'un an sans défaut, chaque jour, en holocauste perpétuel\FTNT{Ex. 29:38.}.
\VS{4}Tu sacrifieras l'un des agneaux le matin, et l'autre agneau entre les deux soirs,
\VS{5}et la dixième partie d'épha de fine farine pour le gâteau pétrie avec le quart d'un hin d'huile vierge\FTNT{Lé. 2:1~; Ex. 29:40~; Ex. 16:36.}.
\VS{6}C'est l'holocauste perpétuel, qui a été offert à la montagne de Sinaï, c'est un sacrifice consumé par le feu, d'une bonne odeur à Yahweh.
\VS{7}Et sa libation sera d'un quart de hin pour chaque agneau~: Et tu verseras dans le lieu saint la libation de boisson forte à Yahweh.
\VS{8}Et tu sacrifieras l'autre agneau entre les deux soirs, tu feras le même gâteau qu'au matin, et la même libation, en sacrifice consumé par le feu d'une bonne odeur à Yahweh.
\VS{9}Mais le jour du sabbat vous offrirez deux agneaux d'un an sans défaut, et deux dixièmes de fine farine pétrie à l'huile pour le gâteau, avec sa libation.
\VS{10}C'est l'holocauste du sabbat, pour chaque sabbat, outre l'holocauste perpétuel avec sa libation.
\VS{11}Et au commencement de vos mois, vous offrirez en holocauste à Yahweh deux jeunes taureaux, un bélier, et sept agneaux d'un an sans défaut~;
\VS{12}et trois dixièmes de fine farine pétrie à l'huile, pour le gâteau de chaque taureau, et deux dixièmes de fine farine pétrie à l'huile pour le gâteau du bélier~;
\VS{13}et un dixième de fine farine pétrie à l'huile, comme gâteau pour chaque agneau, en holocauste, d'une bonne odeur, et en sacrifice consumé par le feu à Yahweh.
\VS{14}Et leurs libations seront d'un demi-hin de vin pour chaque veau, d'un tiers de hin pour un bélier, et d'un quart de hin pour chaque agneau, c'est l'holocauste du commencement de chaque mois, selon tous les mois de l'année.
\VS{15}On sacrifiera aussi à Yahweh un jeune bouc en sacrifice d'expiation, outre l'holocauste perpétuel, et sa libation.
\VS{16}Au quatorzième jour du premier mois, ce sera la Pâque à Yahweh.
\VS{17}Et au quinzième jour du même mois sera un jour de fête. On mangera pendant sept jours des pains sans levain\FTNT{Ex. 12~; Lé. 23:5-6.}.
\VS{18}Au premier jour, il y aura une sainte convocation~: Vous ne ferez aucune œuvre servile.
\VS{19}Et vous offrirez un sacrifice consumé par le feu en holocauste à Yahweh~: Deux jeunes taureaux, un bélier, et sept agneaux d'un an, sans défaut.
\VS{20}Leur gâteau sera de fine farine pétrie à l'huile, vous en offrirez trois dixièmes pour chaque jeune taureau, et deux dixièmes pour un bélier~;
\VS{21}tu en offriras aussi un dixième pour chacun des sept agneaux,
\VS{22}et un bouc en sacrifice pour l'expiation, afin de faire propitiation pour vous.
\VS{23}Vous offrirez ces choses là, outre l'holocauste du matin, qui est l'holocauste perpétuel.
\VS{24}Vous offrirez ces choses-là chaque jour, pendant sept jours, comme l'aliment d'un sacrifice consumé par le feu, d'une bonne odeur à Yahweh. On offrira cela outre l'holocauste perpétuel, et sa libation.
\VS{25}Et au septième jour, vous aurez une sainte convocation~: Vous ne ferez aucune œuvre servile.
\VS{26}Et au jour des prémices, quand vous offrirez à Yahweh une offrande nouvelle de gâteau à votre fête des semaines, vous aurez une sainte convocation~: Vous ne ferez aucune œuvre servile.
\VS{27}Et vous offrirez en holocauste d'une bonne odeur à Yahweh, deux jeunes taureaux, un bélier, et sept agneaux d'un an.
\VS{28}Et leur gâteau sera de fine farine pétrie à l'huile, de trois dixièmes pour chaque jeune taureau, et de deux dixièmes pour le bélier,
\VS{29}et d'un dixième pour chacun des sept agneaux~;
\VS{30}et un jeune bouc, afin de faire propitiation pour vous.
\VS{31}Vous les offrirez, outre l'holocauste perpétuel et son offrande, lesquels seront sans défaut, avec leurs libations.
\Chap{29}
\TextTitle{Consignes relatives au temps des sacrifices - suite}
\VerseOne{}Et le premier jour du septième mois, vous aurez une sainte convocation~: Vous ne ferez aucune œuvre servile. Ce jour sera publié parmi vous au son des trompettes\FTNT{Lé. 23:24-25.}.
\VS{2}Et vous offrirez en holocauste de bonne odeur à Yahweh, un jeune taureau, un bélier, et sept agneaux d'un an, sans défaut.
\VS{3}Et leur gâteau sera de fine farine pétrie à l'huile, de trois dixièmes pour le jeune taureau, de deux dixièmes pour le bélier,
\VS{4}et un dixième pour chacun des sept agneaux.
\VS{5}Et un jeune bouc en sacrifice pour l'expiation, afin de faire propitiation pour vous,
\VS{6}outre l'holocauste du commencement du mois et son gâteau, et l'holocauste perpétuel et son gâteau, et leurs libations selon leur ordonnance. Ce sont des sacrifices consumés par le feu en bonne odeur à Yahweh.
\VS{7}Et au dixième jour de ce septième mois, vous aurez une sainte convocation, et vous affligerez vos âmes~: Vous ne ferez aucune œuvre\FTNT{Lé. 16:29-31~; Lé. 23:27.}.
\VS{8}Et vous offrirez en holocauste, de bonne odeur à Yahweh, un jeune taureau, un bélier, et sept agneaux d'un an, qui seront sans défaut.
\VS{9}Et leur gâteau sera de fine farine pétrie à l'huile, de trois dixièmes pour le taureau, et de deux dixièmes pour le bélier,
\VS{10}et d'un dixième pour chacun des sept agneaux.
\VS{11}Un jeune bouc aussi en sacrifice d'expiation, outre le sacrifice des expiations, l'holocauste perpétuel et son gâteau, avec leurs libations.
\VS{12}Et au quinzième jour du septième mois, vous aurez une sainte convocation~: Vous ne ferez aucune œuvre servile. Vous célébrerez une fête à Yahweh, pendant sept jours\FTNT{Lé. 23:34-43.}.
\VS{13}Et vous offrirez en holocauste un sacrifice consumé par le feu, d'une agréable odeur à Yahweh, treize jeunes taureaux, deux béliers, et quatorze agneaux d'un an, sans défaut.
\VS{14}Et leur gâteau sera de fine farine pétrie à l'huile, de trois dixièmes pour chacun des treize jeunes taureaux, de deux dixièmes pour chacun des deux béliers,
\VS{15}et d'un dixième pour chacun des quatorze agneaux.
\VS{16}Et un jeune bouc en sacrifice d'expiation, outre l'holocauste perpétuel, son gâteau, et sa libation.
\VS{17}Et au second jour, vous offrirez douze jeunes taureaux, deux béliers, et quatorze agneaux d'un an, sans défaut,
\VS{18}avec les gâteaux et les libations pour les jeunes taureaux, pour les béliers, et pour les agneaux, selon leur nombre, d'après les ordonnances.
\VS{19}Vous offrirez un jeune bouc en sacrifice d'expiation, outre l'holocauste perpétuel, et son offrande, avec leurs libations.
\VS{20}Et au troisième jour, vous offrirez onze taureaux, deux béliers, et quatorze agneaux d'un an, sans défaut~;
\VS{21}et les gâteaux et les libations pour les jeunes taureaux, les béliers et les agneaux, selon leur nombre, selon leur ordonnance.
\VS{22}Et un bouc en sacrifice d'expiation, outre l'holocauste continuel, son gâteau et sa libation.
\VS{23}Et au quatrième jour, vous offrirez dix jeunes taureaux, deux béliers, et quatorze agneaux d'un an, sans défaut,
\VS{24}les gâteaux et les libations pour les taureaux, les béliers, et les agneaux, selon leur nombre et leur ordonnance.
\VS{25}Et un jeune bouc en sacrifice d'expiation, outre l'holocauste perpétuel, son offrande, et sa libation.
\VS{26}Et au cinquième jour, vous offrirez neuf jeunes taureaux, deux béliers, et quatorze agneaux d'un an, sans défaut,
\VS{27}avec les gâteaux et les libations pour les taureaux, les béliers, et les agneaux, selon leur nombre et leur ordonnance.
\VS{28}Et un bouc en sacrifice d'expiation, outre l'holocauste continuel, son gâteau, et sa libation.
\VS{29}Et le sixième jour, vous offrirez huit jeunes taureaux, deux béliers et quatorze agneaux d'un an, sans défaut,
\VS{30}et les gâteaux, les libations pour les taureaux, les béliers, et les agneaux selon leur nombre leur ordonnance.
\VS{31}Et un bouc en sacrifice d'expiation, outre l'holocauste continuel, son offrande, et sa libation.
\VS{32}Et au septième jour, vous offrirez sept jeunes taureaux, deux béliers, et quatorze agneaux d'un an, sans défaut,
\VS{33}avec les gâteaux et les libations pour les jeunes taureaux, les béliers, et les agneaux, selon leur nombre et leur ordonnance.
\VS{34}Et un bouc en sacrifice d'expiation, outre l'holocauste continuel, son gâteau, et sa libation.
\VS{35}Et au huitième jour, vous aurez une assemblée solennelle~: Vous ne ferez aucune œuvre servile.
\VS{36}Et vous offrirez en holocauste un sacrifice consumé par le feu, d'une agréable odeur à Yahweh~: Un jeune taureau, un bélier, et sept agneaux d'un an, sans défaut,
\VS{37}avec les gâteaux et les libations pour le jeune taureau, le bélier, et les agneaux, selon leur nombre et leur ordonnance.
\VS{38}Et un bouc en sacrifice d'expiation, outre l'holocauste perpétuel, son offrande, et sa libation.
\VS{39}Vous offrirez ces choses à Yahweh dans vos fêtes solennelles, outre vos vœux, et vos offrandes volontaires, selon vos holocaustes, vos gâteaux, vos libations, et vos sacrifices d'offrande de paix.
\Chap{30}
\TextTitle{Les vœux}
\VerseOne{}Et Moïse parla aux enfants d'Israël selon toutes les choses que Yahweh lui avait ordonné.
\VS{2}Moïse parla aussi aux chefs des tribus des enfants d'Israël, en disant~: Voici ce que Yahweh ordonne.
\VS{3}Quand un homme fera un vœu à Yahweh, ou aura juré par serment, pour lier son âme par un vœu, il ne violera pas sa parole~; il fera selon toutes les choses qui sont sorties de sa bouche\FTNT{De. 23:21.}.
\VS{4}Mais quand une femme fera un vœu à Yahweh, et qu'elle se liera par un serment, dans sa jeunesse, étant encore dans la maison de son père,
\VS{5}et que son père aura entendu son vœu et le serment par lequel elle a lié son âme, si son père ne lui dit rien, tous ses vœux seront valables, et tout serment par lequel elle aura lié son âme sera valable~;
\VS{6}mais si son père la désapprouve le jour où il l'a entendue, aucun de ses vœux ou de ses serments par lesquels elle a lié son âme ne sera valable, et Yahweh lui pardonnera~; parce que son père l'a désapprouvée.
\VS{7}Et si elle a un mari, et qu'elle s'est engagée par quelque vœu ou par une parole échappée de ses lèvres par laquelle elle aura lié son âme,
\VS{8}et que son mari l'aura entendue, et que le jour même où il l'a entendue, il ne lui a rien dit, ses vœux alors seront valables, et ses serments par lesquels elle aura lié son âme seront valables~;
\VS{9}mais si son mari la désapprouve le jour où il l'a entendue, alors il annulera le vœu par lequel elle s'est engagée et la parole échappée de ses lèvres, par laquelle elle avait lié son âme~; et Yahweh lui pardonnera.
\VS{10}Mais le vœu de la veuve ou de la répudiée, tout ce par quoi elle aura lié son âme, sera valable pour elle.
\VS{11}Que si étant encore dans la maison de son mari elle a fait un vœu, ou si elle a lié son âme par serment,
\VS{12}et que son mari l'ait entendue, et ne lui en ait rien dit, et ne l'ait pas désapprouvée, alors tous ses vœux seront valables, et tout serment par lequel elle a lié son âme sera valable.
\VS{13}Mais si son mari les a entièrement annulés le jour où il les a entendus, alors rien de ce qui est sorti de ses lèvres, soit ses vœux, soit le serment par lequel elle a lié son âme ne seront valables~; parce que son mari les a annulés, et Yahweh lui pardonnera.
\VS{14}Son mari ratifiera ou son mari annulera tout vœu et toute obligation faite par serment, pour affliger l'âme.
\VS{15}Mais si son mari ne lui en a absolument rien dit, d'un jour à l'autre, il aura ratifié tous ses vœux ou toutes ses obligations dont elle était tenue~; il les aura, dis-je, ratifiés, parce qu'il ne lui en a rien dit le jour où il les a entendus.
\VS{16}Mais s'il les a expressément annulés après les avoir entendus, alors il portera l'iniquité de sa femme.
\VS{17}Telles sont les ordonnances que Yahweh ordonna à Moïse, entre un mari et sa femme~; entre un père et sa fille, étant encore dans la maison de son père, dans sa jeunesse.
\Chap{31}
\TextTitle{Jugements sur Madian\FTNTT{No. 25:6-18.}}
\VerseOne{}Yahweh parla à Moïse, en disant~:
\VS{2}Fais la vengeance des enfants d'Israël sur les Madianites, puis tu seras recueilli auprès de ton peuple.
\VS{3}Moïse donc parla au peuple, en disant~: Que quelques-uns d'entre vous s'équipent pour aller à la guerre, et qu'ils aillent contre Madian, pour exécuter la vengeance de Yahweh sur Madian.
\VS{4}Vous enverrez à la guerre mille hommes de chaque tribu, de toutes les tribus d'Israël.
\VS{5}On donna d'entre les milliers d'Israël mille hommes de chaque tribu, qui furent douze mille hommes équipés pour la guerre.
\VS{6}Moïse les envoya à la guerre, savoir mille de chaque tribu, et avec eux Phinées, fils d'Eléazar, le prêtre, qui portait les instruments sacrés et les trompettes retentissantes.
\VS{7}Ils s'avancèrent donc contre Madian, comme Yahweh l'avait ordonné à Moïse, et ils en tuèrent tous les mâles.
\VS{8}Ils tuèrent aussi les rois de Madian, outre les autres qui y furent tués, Evi, Rékem, Tsur, Hur, et Réba, cinq rois de Madian~; ils firent aussi passer au fil de l'épée Balaam, fils de Beor\FTNT{Jos. 13:21-22.}.
\VS{9}Et les fils d'Israël emmenèrent prisonniers les femmes de Madian, avec leurs petits enfants, et pillèrent tout leur gros et menu bétail, et tous leurs biens.
\VS{10}Ils brûlèrent par le feu toutes leurs villes, leurs demeures, et tous leurs châteaux.
\VS{11}Ils prirent tout le butin et tout le pillage, tant des hommes que du bétail\FTNT{De. 20:14.}~;
\VS{12}puis ils amenèrent les captifs, le pillage, et le butin, à Moïse, à Eléazar le prêtre, et à l'assemblée des enfants d'Israël, au camp, dans les plaines de Moab, qui sont près du Jourdain, vis-à-vis de Jéricho.
\VS{13}Moïse, Eléazar, le prêtre, et tous les princes de l'assemblée sortirent au-devant d'eux, hors du camp.
\VS{14}Et Moïse se mit en grande colère contre les officiers de l'armée, les chefs des milliers, et les chefs des centaines, qui revenaient de cet exploit de guerre.
\VS{15}Et Moïse leur dit~: N'avez-vous pas gardé en vie toutes les femmes~?
\VS{16}Voici ce sont elles qui, à la parole de Balaam, ont donné l'occasion aux fils d'Israël de pécher contre Yahweh dans l'affaire de Peor~; ce qui attira la plaie sur l'assemblée de Yahweh\FTNT{2 Pi. 2:15~; Ap. 2:14.}.
\VS{17}Or maintenant, tuez tous les mâles d'entre les petits enfants, et tuez toute femme qui a connu un homme en couchant avec lui\FTNT{Jg. 21:11.}~;
\VS{18}mais vous garderez en vie toutes les jeunes filles qui n'ont point connu la couche d'un homme.
\VS{19}Au reste, demeurez sept jours hors du camp~; quiconque aura tué quelqu'un, et quiconque aura touché quelqu'un qui aura été tué, se purifiera le troisième et le septième jour, tant vous que vos prisonniers.
\VS{20}Vous purifierez aussi tous vos vêtements, et tout ce qui sera fait de peau, et tout ouvrage de poil de chèvre, et toute vaisselle de bois.
\VS{21}Eléazar, le prêtre, dit aux hommes de guerre qui étaient allés au combat~: Voici l'ordonnance et la loi que Yahweh a ordonné à Moïse.
\VS{22}En général l'or, l'argent, l'airain, le fer, l'étain, le plomb~;
\VS{23}tout ce qui peut passer par le feu, vous le ferez passer par le feu pour le rendre pur. Seulement on purifiera avec l'eau de purification toutes les choses qui ne peuvent aller au feu, vous les ferez passer dans l'eau.
\VS{24}Vous laverez aussi vos vêtements le septième jour, ensuite vous serez purs~; puis vous entrerez au camp.
\TextTitle{Partage du butin}
\VS{25}Et Yahweh parla à Moïse, en disant~:
\VS{26}Fais le compte du butin et de tout ce qu'on a emmené, tant des personnes que des bêtes, toi et Eléazar, le prêtre, et les chefs des pères de l'assemblée.
\VS{27}Et partage par moitié le butin entre les combattants qui sont allés à la guerre et toute l'assemblée\FTNT{1 S. 30:24.}.
\VS{28}Tu prélèveras aussi pour Yahweh un tribut sur les hommes de guerre qui sont allés à la bataille, savoir un sur cinq cents, tant des personnes, que des bœufs, des ânes et des brebis.
\VS{29}On le prendra sur leur moitié, et tu le donneras à Eléazar, le prêtre, en offrande présentée par élévation à Yahweh.
\VS{30}Et sur la moitié qui appartient aux enfants d'Israël, tu prendras un sur cinquante, tant des personnes que des bœufs, des ânes, des brebis et de tous les autres animaux, et tu le donneras aux Lévites qui ont la charge de garder le tabernacle de Yahweh.
\VS{31}Moïse et Eléazar, le prêtre, firent comme Yahweh l'avait ordonné à Moïse.
\VS{32}Or le butin qui était resté du pillage du peuple qui était allé à la guerre, était de six cent soixante-quinze mille brebis~;
\VS{33}de soixante-douze mille bœufs~;
\VS{34}de soixante et un mille ânes,
\VS{35}quant aux femmes qui n'avaient point connu la couche d' un homme, elles étaient en tout trente-deux mille âmes.
\VS{36}Et la moitié du butin, à savoir la part de ceux qui étaient allés à la guerre, montait à trois cent trente-sept mille cinq cents brebis~;
\VS{37}dont le tribut pour Yahweh, quant aux brebis, était de six cent soixante-quinze.
\VS{38}Trente-six mille bœufs~; dont le tribut pour Yahweh, quant aux bœufs, était de soixante-douze bœufs,
\VS{39}trente mille cinq cents ânes~; dont le tribut pour Yahweh, quant aux ânes, était de soixante et un ânes~;
\VS{40}et de seize mille personnes, dont le tribut pour Yahweh était de trente-deux personnes.
\VS{41}Et Moïse donna à Eléazar, le prêtre, le tribut de l'offrande présentée par élévation à Yahweh, comme Yahweh le lui avait ordonné.
\VS{42}Et de l'autre moitié qui appartenait aux enfants d'Israël, que Moïse avait tiré des hommes qui étaient allés à la guerre~;
\VS{43}or de cette moitié qui fut pour l'assemblée, et qui montait à trois cent trente-sept mille cinq cents brebis,
\VS{44}trente-six mille bœufs,
\VS{45}trente mille cinq cents ânes,
\VS{46}et à seize mille personnes~;
\VS{47}de cette moitié, dis-je, qui appartenait aux enfants d'Israël, Moïse prit un sur cinquante, tant des personnes que des bêtes, et les donna aux Lévites qui avaient la charge de garder le tabernacle de Yahweh, comme Yahweh le lui avait ordonné.
\VS{48}Les commandants des milliers de l'armée, tant les chefs des milliers que les chefs des centaines, s'approchèrent de Moïse,
\VS{49}et lui dirent~: Tes serviteurs ont fait le compte des hommes de guerre qui étaient sous nos ordres, il ne manque pas un homme d'entre nous.
\VS{50}C'est pourquoi, nous offrons l'offrande de Yahweh, chacun les objets que nous avons trouvés~: Des joyaux d'or, des chaînes de cheville, des bracelets, des anneaux, des pendants d'oreilles et des colliers, afin de faire propitiation pour nos personnes devant Yahweh.
\VS{51}Moïse et Eléazar, le prêtre, reçurent d'eux cet or, tous ces objets travaillés.
\VS{52}Et tout l'or de l'offrande présentée par élévation à Yahweh, de la part des chefs de milliers et des chefs de centaines, montait à seize mille sept cent cinquante sicles.
\VS{53}Or les hommes de guerre gardèrent chacun pour soi ce qu'ils avaient pillé.
\VS{54}Moïse donc et Eléazar, le prêtre, prirent l'or des chefs des milliers et des chefs de centaines, et l'apportèrent à la tente d'assignation, comme souvenir pour les enfants d'Israël, devant Yahweh.
\Chap{32}
\TextTitle{Ruben et Gad en Galaad}
\VerseOne{}Les fils de Ruben et les fils de Gad avaient beaucoup de bétail, en très grande quantité, et ils virent que le pays de Jaezer et le pays de Galaad étaient un lieu propre pour du bétail.
\VS{2}Ainsi les fils de Gad et les fils de Ruben vinrent, et parlèrent à Moïse et à Eléazar, le prêtre, et aux princes de l'assemblée, en disant~:
\VS{3}Atharoth, Dibon, Jaezer, Nimra, Hesbon, et Elealé, Sebam, Nebo, et Beon,
\VS{4}ce pays-là que Yahweh a frappé devant l'assemblée d'Israël, est un pays propre pour le bétail, et tes serviteurs ont des troupeaux.
\VS{5}Ils dirent donc: Si nous avons trouvé grâce à tes yeux, que ce pays soit donné en possession à tes serviteurs~; et ne nous fais point passer le Jourdain.
\VS{6}Mais Moïse répondit aux fils de Gad, et aux fils de Ruben~: Vos frères iront-ils à la guerre, et vous, demeurerez-vous ici~?
\VS{7}Pourquoi voulez-vous décourager les enfants d'Israël de passer dans le pays que Yahweh leur a donné~?
\VS{8}C'est ainsi que firent vos pères quand je les envoyai de Kadès-Barnéa pour examiner le pays.
\VS{9}Car ils montèrent jusqu'à la vallée d'Eschcol, virent le pays, puis découragèrent les enfants d'Israël, afin qu'ils n'entrent point dans le pays que Yahweh leur avait donné.
\VS{10}C'est pourquoi la colère de Yahweh s'enflamma ce jour-là, et il jura en disant~:
\VS{11}Les hommes qui sont montés hors d'Egypte, depuis l'âge de vingt ans et au-dessus, ne verront point le pays que j'ai juré de donner à Abraham, Isaac, et à Jacob~; car ils n'ont point persévéré à me suivre\FTNT{De. 1:35.},
\VS{12}excepté Caleb, fils de Jephunné, le Kénizien, et Josué, fils de Nun, car ils ont persévéré à suivre Yahweh.
\VS{13}Ainsi la colère de Yahweh s'enflamma contre Israël et il les fit errer dans le désert pendant quarante ans, jusqu'à ce que toute la génération qui avait fait le mal aux yeux de Yahweh, ait été consumée.
\VS{14}Et voici, vous vous êtes levés à la place de vos pères, comme une race d'hommes pécheurs, pour augmenter encore l'ardeur de la colère de Yahweh contre Israël.
\VS{15}Si vous vous détournez de lui, il continuera encore à vous laisser au désert, et vous ferez détruire tout ce peuple.
\VS{16}Mais ils s'approchèrent de lui et lui dirent~: Nous bâtirons ici des cloisons pour nos troupeaux, et des villes pour nos petits enfants~;
\VS{17}et nous nous équiperons pour marcher promptement devant les enfants d'Israël, jusqu'à ce que nous les ayons introduits en leur lieu~; mais nos petits enfants demeureront dans les villes fortes, à cause des habitants du pays.
\VS{18}Nous ne retournerons point dans nos maisons avant que chacun des enfants d'Israël n'ait pris possession de son héritage~;
\VS{19}et nous ne posséderons rien en héritage avec eux au-delà du Jourdain, ni plus avant~; parce que nous aurons notre héritage de ce côté-ci du Jourdain, à l'orient.
\VS{20}Et Moïse leur dit~: Si vous faites cela, si vous vous équipez devant Yahweh pour aller à la guerre,
\VS{21}si chacun de vous étant équipé passe le Jourdain devant Yahweh, jusqu'à ce qu'il ait chassé ses ennemis loin de devant lui,
\VS{22}et que le pays soit assujetti devant Yahweh, et qu'ensuite vous vous en retournez, alors vous serez innocents envers Yahweh, et envers Israël~; et ce pays-ci vous appartiendra pour le posséder devant Yahweh.
\VS{23}Mais si vous ne faites point cela, vous péchez contre Yahweh~; et sachez que votre péché vous atteindra.
\VS{24}Bâtissez donc des villes pour vos petits enfants, et des cloisons pour vos troupeaux, et faites ce que vous avez dit.
\VS{25}Alors les fils de Gad et les fils de Ruben parlèrent à Moïse, en disant~: Tes serviteurs feront ce que mon seigneur a ordonné.
\VS{26}Nos petits enfants, nos femmes, nos troupeaux, et tout notre bétail demeureront ici dans les villes de Galaad~;
\VS{27}et tes serviteurs passeront chacun armés pour aller à la guerre devant Yahweh, prêts à combattre, comme mon seigneur a parlé.
\VS{28}Alors Moïse donna des ordres à leur sujet à Eléazar, le prêtre, à Josué, fils de Nun, et aux chefs des pères des tribus des fils d'Israël.
\VS{29}Il leur dit~: Si les fils de Gad et les fils de Ruben passent avec vous le Jourdain tous armés, prêts à combattre devant Yahweh, et que le pays vous soit assujetti, vous leur donnerez le pays de Galaad en possession.
\VS{30}Mais s'ils ne marchent point en armes avec vous, qu'ils s'établissent au milieu de vous dans le pays de Canaan.
\VS{31}Les fils de Gad et les fils de Ruben répondirent, en disant~: Nous ferons ce que Yahweh a dit à tes serviteurs.
\VS{32}Nous passerons en armes devant Yahweh au pays de Canaan, afin que nous possédions pour notre héritage ce qui est de ce côté-ci du Jourdain.
\VS{33}Ainsi Moïse donna aux fils de Gad et aux fils de Ruben, et à la demi-tribu de Manassé, fils de Joseph, le royaume de Sihon, roi des Amoréens~; et le royaume de Og, roi de Basan, le pays avec ses villes, selon les bornes des villes du pays tout autour.
\VS{34}Alors les fils de Gad rebâtirent Dibon, Atharoth, Aroër,
\VS{35}Athroth-Schophan, Jaezer, Jogbeha,
\VS{36}Beth-Nimra et Beth-Haran, villes fortifiées. Ils firent aussi des cloisons pour les troupeaux.
\VS{37}Et les fils de Ruben rebâtirent Hesbon, Elealé, Kirjathaim,
\VS{38}Nébo, Baal-Meon, et Sibma, dont ils changèrent les noms, et ils donnèrent des noms aux villes qu'ils rebâtirent.
\VS{39}Or les fils de Makir, fils de Manassé, allèrent en Galaad, le prirent et dépossédèrent les Amoréens qui y étaient.
\VS{40}Moïse donc donna Galaad à Makir, fils de Manassé, qui y habita\FTNT{De. 3:15.}.
\VS{41}Jaïr, fils de Manassé, se mit en marche, prit leurs villages, et les appela villages de Jaïr\FTNT{De. 3:14~; 1 Ch. 2:22.}.
\VS{42}Et Nobach se mit en marche, prit Kenath avec les villes de son ressort, et l'appela Nobach d'après son nom.
\Chap{33}
\TextTitle{Les stations de l'Egypte jusqu'au Jourdain}
\VerseOne{}Ce sont ici les étapes des enfants d'Israël, qui sortirent du pays d'Egypte, selon leurs armées, sous la main de Moïse et d'Aaron.
\VS{2}Moïse écrivit leurs départs, et leurs étapes, d'après l'ordre de Yahweh~! Et voici leurs étapes selon leurs départs.
\VS{3}Les enfants d'Israël donc partirent de Ramsès le quinzième jour du premier mois, dès le lendemain de la Pâque, et ils sortirent à main levée, à la vue de tous les Egyptiens\FTNT{Ex. 14:8.}.
\VS{4}Et les Egyptiens ensevelissaient ceux que Yahweh avait frappés parmi eux, à savoir tous les premiers-nés~; même Yahweh exerçait aussi ses jugements contre leurs dieux\FTNT{Ex. 12:12~; Ex. 18:11.}.
\VS{5}Et les enfants d'Israël partirent de Ramsès, et campèrent à Succoth\FTNT{Ex. 12:37.}.
\VS{6}Et ils partirent de Succoth et campèrent à Etham, qui est au bout du désert\FTNT{Ex. 13:20.}.
\VS{7}Et ils partirent d'Etham et se détournèrent vers Pi-Hahiroth, qui est vis-à-vis de Baal-Tsephon, et campèrent devant Migdol\FTNT{Ex. 14:2.}.
\VS{8}Et ils partirent de devant Pi-Hahiroth et passèrent au travers de la mer vers le désert, et firent trois journées de marche par le désert d'Etham et campèrent à Mara.
\VS{9}Puis ils partirent de Mara et vinrent à Elim où il y avait douze fontaines d'eaux et soixante-dix palmiers, et ils y campèrent\FTNT{Ex. 15:27.}.
\VS{10}Et ils partirent d'Elim et campèrent près de la Mer Rouge.
\VS{11}Puis ils partirent de la Mer Rouge et campèrent au désert de Sin\FTNT{Ex. 16:1.}.
\VS{12}Ils partirent du désert de Sin et campèrent à Dophka.
\VS{13}Puis ils partirent de Dophka et campèrent à Alusch.
\VS{14}Et ils partirent d'Alusch et campèrent à Rephidim où il n'y avait point d'eau à boire pour le peuple\FTNT{Ex. 17:1.}.
\VS{15}Puis ils partirent de Rephidim et campèrent dans le désert de Sinaï\FTNT{Ex. 17:1.}.
\VS{16}Ils partirent du désert de Sinaï et campèrent à Kibroth-Hattaava.
\VS{17}Et ils partirent de Kibroth-Hattaava et campèrent à Hatséroth.
\VS{18}Puis ils partirent de Hatséroth et campèrent à Rithma.
\VS{19}Et ils partirent de Rithma et campèrent à Rimmon-Pérets.
\VS{20}Ils partirent de Rimmon-Pérets et campèrent à Libna.
\VS{21}Et ils partirent de Libna et campèrent à Rissa.
\VS{22}Puis ils partirent de Rissa et campèrent vers Kehélatha.
\VS{23}Et ils partirent de Kehélatha et campèrent à la montagne de Schapher.
\VS{24}Ils partirent de la montagne de Schapher et campèrent à Harada.
\VS{25}Et ils partirent de Harada et campèrent à Makhéloth.
\VS{26}Puis ils partirent de Makhéloth et campèrent à Tahath.
\VS{27}Ils partirent de Tahath et campèrent à Tarach.
\VS{28}Et ils partirent de Tarach et campèrent à Mithka.
\VS{29}Puis ils partirent de Mithka et campèrent à Haschmona.
\VS{30}Ils partirent de Haschmona et campèrent à Moséroth.
\VS{31}Et ils partirent de Moséroth et campèrent à Bené-Jaakan.
\VS{32}Ils partirent de Bené-Jaakan et campèrent à Hor-Guidgad.
\VS{33}Puis ils partirent de Hor-Guidgad et campèrent vers Jothbatha.
\VS{34}Ils partirent de Jothbatha et campèrent à Abrona.
\VS{35}Et ils partirent d'Abrona et campèrent à Etsjon-Guéber.
\VS{36}Ils partirent d'Etsjon-Guéber et campèrent dans le désert de Tsin, qui est Kadès.
\VS{37}Puis ils partirent de Kadès et campèrent à la montagne de Hor, qui est au bout du pays d'Edom.
\VS{38}Et Aaron le prêtre, monta sur la montagne de Hor, suivant l'ordre de Yahweh, et mourut là, la quarantième année après que les enfants d'Israël furent sortis du pays d'Egypte, le premier jour du cinquième mois.
\VS{39}Et Aaron était âgé de cent vingt-trois ans quand il mourut sur la montagne de Hor.
\VS{40}Alors le Cananéen, roi d'Arad, qui habitait vers le midi au pays de Canaan, apprit que les enfants d'Israël venaient.
\VS{41}Et ils partirent de la montagne de Hor et campèrent à Tsalmona.
\VS{42}Puis ils partirent de Tsalmona et campèrent à Punon.
\VS{43}Et Ils partirent de Punon et campèrent à Oboth.
\VS{44}Ils partirent d'Oboth et campèrent à Ijjé-Abarim, sur les frontières de Moab.
\VS{45}Puis ils partirent d'Ijjé-Abarim et campèrent à Dibon-Gad.
\VS{46}Et ils partirent de Dibon-Gad, et campèrent à Almon-Diblathaïm.
\VS{47}Ils partirent d'Almon-Diblathaïm et campèrent aux montagnes de Abarim devant Nébo.
\VS{48}Et ils partirent des montagnes d'Abarim et campèrent aux plaines de Moab, près du Jourdain de Jéricho.
\VS{49}Pui ils campèrent près du Jourdain, depuis Beth-Jeschimoth jusqu'à Abel-Sittim, dans les plaines de Moab.
\TextTitle{Consignes pour les possessions attribuées à Israël}
\VS{50}Et Yahweh parla à Moïse dans les plaines de Moab, près du Jourdain de Jéricho, en disant~:
\VS{51}Parle aux enfants d'Israël, et dis-leur~: Puisque vous allez passer le Jourdain pour entrer au pays de Canaan,
\VS{52}vous chasserez de devant vous tous les habitants du pays, vous détruirez toutes leurs peintures, et vous ruinerez toutes leurs images de fonte, et vous démolirez tous leurs hauts lieux\FTNT{De. 7:5~; De. 12:2.}.
\VS{53}Et vous prendrez possession du pays, et vous y habiterez. Car je vous ai donné le pays pour le posséder.
\VS{54}Or vous recevrez le pays en héritage par le sort, selon vos familles. A ceux qui sont en plus grand nombre, vous donnerez plus d'héritage, et à ceux qui sont en plus petit nombre, vous donnerez moins d'héritage. Chacun aura selon ce qui lui sera échu par le sort, et vous hériterez selon les tribus de vos pères.
\VS{55}Mais si vous ne chassez pas de devant vous les habitants du pays, il arrivera que ceux d'entre eux que vous aurez laissés comme reste, seront comme des épines à vos yeux, et comme des pointes à vos côtés, et ils vous serreront de près dans le pays auquel vous habiterez\FTNT{Jos. 23:13.}.
\VS{56}Et il arrivera que je vous ferai tout comme j'ai eu dessein de leur faire.
\Chap{34}
\TextTitle{Consignes sur les limites de chaque tribu}
\VerseOne{}Yahweh parla aussi à Moïse, en disant~:
\VS{2}Donne l'ordre aux enfants d'Israël, et dis-leur~: Parce que vous allez entrer au pays de Canaan, ce pays deviendra votre héritage, le pays de Canaan selon ses limites.
\VS{3}Votre frontière du côté du sud sera depuis le désert de Tsin, le long d'Edom, et votre frontière du côté du sud commencera au bout de la mer salée, vers l'orient~;
\VS{4}et cette frontière tournera du sud vers la montée d'Akrabbim, et passera jusqu'à Tsin~; et elle aboutira du côté du sud de Kadès-Barnéa~; et sortira aussi par Hatsar-Addar, et passera jusqu'à Atsmon.
\VS{5}Et cette frontière tournera depuis Atsmon jusqu'au torrent d'Egypte~; et elle aboutira à la mer.
\VS{6}Quant à la frontière d'occident, vous aurez la grande mer et ses limites~; ce sera votre frontière occidentale.
\VS{7}Et ce sera ici votre frontière au nord~; depuis la grande mer, vous marquerez pour vos limites la montagne de Hor~;
\VS{8}et depuis la montagne de Hor, vous marquerez pour vos limites l'entrée de Hamath, et cette frontière aboutira vers Tsedad~;
\VS{9}cette frontière passera jusqu'à Ziphron, et elle aboutira à Hatsar-Enan~; telle sera votre frontière au nord.
\VS{10}Puis vous marquerez pour vos limites vers l'orient de Hatsar-Enan à Schepham.
\VS{11}Et cette frontière descendra de Schepham à Ribla, du côté de l'orient d'Aïn~; et cette frontière descendra et s'étendra le long de la Mer de Kinnéreth vers l'orient.
\VS{12}Cette frontière descendra au Jourdain pour aboutir à la Mer Salée~; tel sera le pays que vous aurez avec ses limites tout autour.
\VS{13}Et Moïse donna l'ordre aux enfants d'Israël, en disant~: C'est là le pays que vous hériterez par le sort, et que Yahweh a ordonné de donner à neuf tribus, et à la demi-tribu.
\VS{14}Car la tribu des fils de Ruben selon les familles de leurs pères, et la tribu des fils de Gad, selon les familles de leurs pères, ont pris leur héritage~; et la demi-tribu de Manassé a pris aussi son héritage.
\VS{15}Deux tribus, dis-je, et la demi-tribu ont pris leur héritage de l'autre côté du Jourdain, vis-à-vis de Jéricho, du côté du levant.
\VS{16}Et Yahweh parla à Moïse, en disant~:
\VS{17}Ce sont ici les noms des hommes qui vous partageront le pays~: Eléazar le prêtre, et Josué fils de Nun.
\VS{18}Vous prendrez aussi un prince de chaque tribu pour faire le partage du pays.
\VS{19}Et voici les noms de ces hommes. Pour la tribu de Juda~: Caleb, fils de Jephunné~;
\VS{20}pour la tribu des fils de Siméon~: Samuel, fils d'Ammihud~;
\VS{21}pour la tribu de Benjamin~: Elidad, fils de Kislon~;
\VS{22}pour la tribu des fils de Dan~: Celui qui en est le chef, Buki, fils de Jogli~;
\VS{23}pour les fils de Joseph, pour la tribu des fils de Manassé~: Celui qui en est le chef, Hanniel, fils d'Ephod~;
\VS{24}et pour la tribu des fils d'Ephraïm~: Celui qui en est le chef, Kemuel, fils de Schiphtan~;
\VS{25}pour la tribu des fils de Zabulon~: Celui qui en est le chef, Elitsaphan, fils de Parnac~;
\VS{26}pour la tribu des fils d'Issacar~: Celui qui en est le chef, Paltiel, fils d'Azzan~;
\VS{27}pour la tribu des fils d'Aser~: Celui qui en est le chef, Ahihud, fils de Schelomi~;
\VS{28}pour la tribu des fils de Nephthali~: Celui qui en est le chef, Pedahel, fils d'Ammihud.
\VS{29}Ce sont là, ceux à qui Yahweh donna l'ordre de partager l'héritage aux enfants d'Israël dans le pays de Canaan.
\Chap{35}
\TextTitle{Quarante-huit villes pour les Lévites dont six villes de refuge}
\VerseOne{}Yahweh parla à Moïse dans les plaines de Moab, près du Jourdain, vis-à-vis de Jéricho, en disant~:
\VS{2}Donne l'ordre aux enfants d'Israël qu'ils donnent aux Lévites, sur l'héritage qu'ils posséderont, des villes pour y habiter. Vous leur donnerez aussi les faubourgs qui sont autour de ces villes\FTNT{Jos. 21:2.}.
\VS{3}Ils auront donc les villes pour y habiter~; et les faubourgs de ces villes seront pour leurs bétails, pour leurs biens, et pour tous leurs animaux.
\VS{4}Les faubourgs des villes que vous donnerez aux Lévites, seront de mille coudées tout autour depuis la muraille de la ville en dehors.
\VS{5}Et vous mesurerez depuis le dehors de la ville du côté de l'orient, deux mille coudées~; et du côté du sud, deux mille coudées~; et du côté de l'occident, deux mille coudées~; et du côté du nord, deux mille coudées~; et la ville sera au milieu~; tels seront les faubourgs de leurs villes.
\VS{6}Et des villes que vous donnerez aux Lévites, il y aura six villes de refuge que vous donnerez pour que le meurtrier s'y enfuie, et outre celles-là, vous leur donnerez quarante-deux villes.
\VS{7}Toutes les villes que vous donnerez aux Lévites seront quarante-huit villes, elles et leurs faubourgs.
\VS{8}Et quant aux villes que vous leur donnerez sur la possession des enfants d'Israël, de ceux qui en auront plus vous en prendrez plus, et de ceux qui en auront moins vous en prendrez moins~; chacun donnera de ses villes aux Lévites, en proportion de l'héritage qu'il possédera.
\VS{9}Puis Yahweh parla à Moïse, en disant~:
\VS{10}Parle aux enfants d'Israël, et dis-leur~: Quand vous aurez passé le Jourdain, pour entrer au pays de Canaan~;
\VS{11}établissez-vous des villes qui vous soient des villes de refuge, afin que le meurtrier qui aura frappé à mort quelqu'un involontairement, s'y enfuie\FTNT{Jos. 20:2-3~; Ex. 21:13.}.
\VS{12}Et ces villes seront pour vous des villes de refuge contre le vengeur, afin que le meurtrier ne meure pas, jusqu'à ce qu'il ait comparu en jugement devant l'assemblée.
\VS{13}De ces villes que vous donnerez, il y en aura six de refuge pour vous.
\VS{14}Vous donnerez trois de ces villes au-delà du Jourdain, et les trois autres dans le pays de Canaan, qui seront des villes de refuge\FTNT{De. 19:2~; De. 4:41-42.}.
\VS{15}Ces six villes serviront de refuge aux enfants d'Israël, à l'étranger et à celui qui séjourne au milieu de vous, afin que quiconque aura frappé à mort quelqu'un involontairement, s'y enfuie.
\VS{16}Mais si un homme en frappe un autre avec un instrument de fer, et qu'il en meure, il est meurtrier~; on punira de mort le meurtrier.
\VS{17}Et s'il le frappe avec une pierre qu'il tenait à la main, dont on puisse mourir, et qu'il en meure, c'est un meurtrier~; le meurtrier sera puni de mort.
\VS{18}De même s'il le frappe d'un instrument de bois qu'il tenait à la main, dont on puisse mourir, et qu'il en meure, il est un meurtrier~; on punira de mort le meurtrier.
\VS{19}Et le vengeur du sang fera mourir le meurtrier quand il le rencontrera, il pourra le faire mourir.
\VS{20}Et s'il le pousse par haine, ou s'il jette quelque chose sur lui avec préméditation, et qu'il en meure~;
\VS{21}ou si par inimitié il le frappe de sa main, et qu'il en meure, on punira de mort celui qui l'a frappé, car il est meurtrier~; le vengeur du sang pourra le faire mourir quand il le rencontrera\FTNT{De. 19:11-12.}.
\VS{22}Mais s'il le pousse subitement, sans inimitié, ou s'il jette quelque chose sur lui, sans préméditation,
\VS{23}ou s'il fait tomber sur lui quelque pierre sans l'avoir vu, et qu'il en meure, n'étant pas son ennemi et ne lui cherchant pas du mal,
\VS{24}alors l'assemblée jugera entre celui qui a frappé et le vengeur du sang, selon ces ordonnances~;
\VS{25}l'assemblée délivrera le meurtrier de la main du vengeur de sang, et le fera retourner dans la ville de refuge où il s'était enfui. Il y demeurera jusqu'à la mort du grand-prêtre, qui aura été oint de la sainte huile.
\VS{26}Mais si le meurtrier sort de quelque manière que ce soit hors des bornes de la ville de son refuge, où il s'est enfui,
\VS{27}et si le vengeur du sang le rencontre hors des bornes de la ville de son refuge, et qu'il tue le meurtrier, il ne sera point coupable de meurtre.
\VS{28}Car il doit demeurer dans la ville de son refuge jusqu'à la mort du grand-prêtre~; et après la mort du grand-prêtre, le meurtrier pourra retourner dans sa possession.
\VS{29}Et ces choses-ci seront des ordonnances de jugement pour vous et pour vos générations, dans toutes vos demeures.
\VS{30}Celui qui fera mourir le meurtrier, le fera mourir sur la parole de deux témoins~; mais un seul témoin ne sera point reçu en témoignage contre quelqu'un, pour le faire mourir\FTNT{De. 17:6~; De. 19:15.}.
\VS{31}Et vous ne prendrez point de rançon pour la vie du meurtrier, qui est coupable et digne de mort~; mais il doit être puni de mort.
\VS{32}Vous ne prendrez point de rançon pour le laisser s'enfuir de sa ville de refuge, pour qu'il retourne habiter dans le pays, jusqu'à la mort du prêtre.
\VS{33}Et vous ne souillerez point le pays où vous serez, car le sang souille le pays~; et il ne se fera point de propitiation pour le pays, du sang qui y sera répandu que par le sang de celui qui l'aura répandu.
\VS{34}Vous ne souillerez donc point le pays où vous allez demeurer, et au milieu duquel j'habiterai~; car je suis Yahweh qui habite au milieu des enfants d'Israël.
\Chap{36}
\TextTitle{Loi sur les héritages\FTNTT{No. 27:1-11.}}
\VerseOne{}Or les chefs des pères de la famille des fils de Galaad, fils de Makir, fils de Manassé, d'entre les familles des fils de Joseph, s'approchèrent et parlèrent devant Moïse, et devant les princes, les chefs des pères des enfants d'Israël,
\VS{2}et ils dirent~: Yahweh a donné l'ordre à mon seigneur de donner aux enfants d'Israël le pays en héritage par le sort~; et mon seigneur a reçu l'ordre de Yahweh de donner l'héritage de Tselophchad, notre frère, à ses filles.
\VS{3}Si elles se marient à l'un des fils des autres tribus d'Israël, leur héritage sera retranché de l'héritage de nos pères et sera ajouté à l'héritage de la tribu de laquelle elles seront~; ainsi sera diminué l'héritage qui nous est échu par le sort.
\VS{4}Même quand viendra le jubilé pour les enfants d'Israël, on ajoutera leur héritage à l'héritage de la tribu à laquelle elles appartiendront, ainsi leur héritage sera retranché de l'héritage de la tribu de nos pères\FTNT{Lé. 25:10-13.}.
\VS{5}Et Moïse ordonna aux enfants d'Israël, suivant l'ordre de la bouche de Yahweh, en disant~: Ce que la tribu des fils de Joseph dit est juste.
\VS{6}C'est ici ce que Yahweh ordonne au sujet des filles de Tselophchad~: Elles se marieront à qui bon leur semblera, toutefois elles se marieront dans l'une des familles de la tribu de leurs pères.
\VS{7}Ainsi l'héritage ne sera point transporté entre les enfants d'Israël de tribu en tribu~; car chacun des enfants d'Israël se tiendra à l'héritage de la tribu de ses pères.
\VS{8}Et toute fille, qui possédera un héritage d'entre les tribus des enfants d'Israël, se mariera à quelqu'un de la famille de la tribu de son père, afin que chacun des enfants d'Israël possède l'héritage de ses pères.
\VS{9}L'héritage donc ne sera point transporté d'une tribu à une autre, mais chacune des tribus des enfants d'Israël se tiendra à son héritage.
\VS{10}Les filles de Tselophchad firent comme Yahweh avait donné à Moïse.
\VS{11}Machla, Thirtsa, Hogla, Milca, et Noa, filles de Tselophchad, se marièrent aux fils de leurs oncles.
\VS{12}Ainsi elles se marièrent à ceux qui étaient des familles des fils de Manassé, fils de Joseph~; et leur héritage demeura dans la tribu de la famille de leur père.
\VS{13}Ce sont là les ordonnances et les jugements que Yahweh ordonna par Moïse aux enfants d'Israël, dans les plaines de Moab, près du Jourdain, vis-à-vis de Jéricho.
\PPE{}
\end{multicols}

%\clearpage\ShortTitle{De.}\BookTitle{Deutéronome}\BFont
\noindent\hrulefill
{\footnotesize
\textit{
\bigskip
{\centering{}
\\Auteur~: Probablement Moïse
\\(Heb.~: Devarim)
\\Signification~: Paroles
\\Thème~: Rappel de la loi
\\Date de rédaction~: 1450-1410 av. J.-C.\\}
}
\textit{
\\Ce livre est un rappel de la loi de Yahweh. Après quarante années d'errance dans le désert, Moïse s'adresse à la nouvelle génération par des discours et des exhortations, depuis les plaines de Moab. Au travers de son serviteur, Dieu rappelle ainsi la loi donnée sur le mont Sinaï, les expériences vécues par la génération passée et par conséquent, l'importance de la soumission à Dieu. De leur obéissance dépendraient les bénédictions ou les malédictions contenues dans ce livre.\bigskip
}
}
\par\nobreak\noindent\hrulefill
\begin{multicols}{2}
\Chap{1}
\TextTitle{Rappel de l'infidélité d'Israël\FTNTT{No. 14.}}
\VerseOne{}Ce sont ici les paroles que Moïse déclara à tout Israël de l'autre côté du Jourdain, dans le désert, dans la plaine, qui est vis-à-vis de Suph, entre Paran, Tophel, Laban, Hatséroth, et Di-Zahab.
\VS{2}Il y a onze journées depuis Horeb, par le chemin de la montagne de Séir, jusqu'à Kadès-Barnéa.
\VS{3}Or il arriva dans la quarantième année, au onzième mois, le premier jour du mois, que Moïse parla aux enfants d'Israël selon tout ce que Yahweh lui avait ordonné de leur dire,
\VS{4}après qu'il eut battu Sihon, roi des Amoréens, qui habitait à Hesbon, et Og, roi de Basan, qui demeurait à Aschtaroth et à Edréi\FTNT{No. 21:23-24.}.
\VS{5}Moïse donc commença à expliquer cette loi, de l'autre côté du Jourdain, dans le pays de Moab, en disant~:
\VS{6}Yahweh, notre Dieu, nous a parlé à Horeb, en disant~: Vous avez assez demeuré dans cette montagne.
\VS{7}Tournez-vous, partez, et allez à la montagne des Amoréens et dans tous les lieux voisins, dans la plaine, dans la montagne, dans la vallée, vers le sud, sur le rivage de la mer, au pays des Cananéens et au Liban, jusqu'au grand fleuve, le fleuve d'Euphrate.
\VS{8}Regardez, j'ai mis devant vous le pays~; entrez et prenez possession du pays que Yahweh a juré de donner à vos pères, Abraham, Isaac et Jacob, et à leur postérité après eux.
\VS{9}Et je vous parlai en ce temps-là, et je vous dis~: Je ne puis pas, à moi seul, vous porter.
\VS{10}Yahweh, votre Dieu, vous a multipliés, et vous voici aujourd'hui comme les étoiles du ciel par le nombre.
\VS{11}Que Yahweh, le Dieu de vos pères, vous fasse croître mille fois au delà de ce que vous êtes et vous bénisse, comme il vous l'a dit~!
\VS{12}Comment porterais-je moi seul vos chagrins, vos charges, et vos procès~?
\VS{13}Prenez dans vos tribus des hommes sages, intelligents et connus, et je les établirai chefs sur vous.
\VS{14}Et vous me répondîtes, et dîtes~: Il est bon de faire ce que tu as dit.
\VS{15}Alors je pris les chefs de vos tribus, des hommes sages et connus, et je les établis chefs sur vous, chefs de milliers, chefs de centaines, chefs de cinquantaines, chefs de dizaines et officiers selon vos tribus. 
\VS{16}Puis j'ordonnai en ce temps-là à vos juges, en disant~: Ecoutez les différends qui seront entre vos frères, et jugez droitement entre l'homme et son frère, et entre l'étranger qui est avec lui\FTNT{Lé. 19:15~; De. 16:19~; Pr. 24:23.}.
\VS{17}Vous n'aurez point d'égard à l'apparence de la personne en jugement~; vous entendrez autant le petit que le grand~; vous ne craindrez personne, car le jugement est à Dieu~; et vous ferez venir devant moi la cause qui sera trop difficile pour vous, et je l'entendrai. 
\VS{18}Et en ce temps-là, je vous ordonnai toutes les choses que vous deviez faire.
\VS{19}Puis nous partîmes d'Horeb, et nous marchâmes dans tout ce grand et affreux désert que vous avez vu~; par le chemin de la montagne des Amoréens, ainsi que Yahweh, notre Dieu, nous l'avait ordonné, et nous vînmes jusqu'à Kadès-Barnéa.
\VS{20}Alors je vous dis~: Vous êtes arrivés jusqu'à la montagne des Amoréens, que Yahweh, notre Dieu, nous donne.
\VS{21}Regarde, Yahweh, ton Dieu, met le pays devant toi~; monte et prends-en possession, comme Yahweh, le Dieu de tes pères, te l'a dit~; ne crains point et ne t'effraie point.
\VS{22}Et vous vous approchâtes tous de moi, et dîtes~: Envoyons devant nous des hommes pour explorer le pays, et qui nous rapportent des nouvelles du chemin par lequel nous devrons monter, et des villes où nous devrons aller\FTNT{No. 13:2.}.
\VS{23}Et ce discours sembla bon à mes yeux~; et je pris douze hommes parmi vous, un homme par tribu.
\VS{24}Et ils se mirent en chemin et montèrent dans la montagne, et vinrent jusqu'au torrent d'Eschcol et explorèrent le pays.
\VS{25}Et ils prirent dans leurs mains des fruits du pays, et ils nous les apportèrent~; ils nous donnèrent des nouvelles, et nous dirent~: Le pays que Yahweh, notre Dieu, nous donne est bon. 
\VS{26}Mais vous refusâtes d'y monter, et vous fûtes rebelles à l'ordre de Yahweh, votre Dieu.
\VS{27}Et vous murmurâtes dans vos tentes, en disant~: C'est parce que Yahweh nous hait qu'il nous a fait sortir du pays d'Egypte, afin de nous livrer entre les mains des Amoréens pour nous exterminer.
\VS{28}Où monterions-nous~? Nos frères nous ont fait fondre le cœur, en disant~: Le peuple est plus grand que nous, et de plus haute taille~; les villes sont grandes et closes jusqu'au ciel~; et même nous avons vu là les fils des Anakim.
\VS{29}Mais je vous dis~: Ne tremblez point et ne les craignez point.
\VS{30}Yahweh, votre Dieu, qui marche devant vous, lui-même combattra pour vous, selon tout ce que vous avez vu qu'il a fait pour vous en Egypte~;
\VS{31}et au désert, où tu as vu de quelle manière Yahweh, ton Dieu, t'a porté comme un homme porterait son fils, sur tout le chemin où vous avez marché, jusqu'à ce que vous soyez arrivés dans ce lieu-ci.
\VS{32}Mais malgré cela, vous ne crûtes point encore en Yahweh, votre Dieu,
\VS{33}qui marchait devant vous sur le chemin afin de vous chercher un lieu pour camper, marchant de nuit dans la colonne de feu pour vous éclairer dans le chemin par lequel vous deviez marcher et de jour dans la nuée.
\VS{34}Et Yahweh entendit la voix de vos paroles et se mit en grande colère et jura, disant~:
\VS{35}Aucun des hommes de cette méchante génération ne verra ce bon pays que j'ai juré de donner à vos pères,
\VS{36}à l'exception de Caleb, fils de Jephunné~; lui le verra, et je donnerai à lui et à ses fils le pays sur lequel il a marché, parce qu'il a persévéré à suivre Yahweh\FTNT{No. 14:22-24.}.
\VS{37}Même Yahweh s'est mis en colère contre moi à cause de vous, disant~: Et toi aussi tu n'y entreras pas.
\VS{38}Josué, fils de Nun, qui te sert, y entrera~; fortifie-le, car c'est lui qui mettra les enfants d'Israël en possession de ce pays\FTNT{De. 34:4.}.
\VS{39}Et vos petits-enfants, dont vous avez dit qu'ils seront en proie, vos enfants, dis-je, qui aujourd'hui ne savent pas ce que c'est le bien ou le mal, eux y entreront, et je leur donnerai ce pays et ils le posséderont.
\VS{40}Mais vous, retournez vous-en en arrière, et allez dans le désert par le chemin de la Mer Rouge.
\VS{41}Et vous répondîtes et me dîtes~: Nous avons péché contre Yahweh, nous monterons et nous combattrons, comme Yahweh, notre Dieu, nous l'a ordonné. Et vous ceignîtes chacun vos armes de guerre, et vous entreprîtes hardiment de monter à la montagne.
\VS{42}Et Yahweh me dit~: Dis-leur~: Ne montez point et ne combattez point, car je ne suis point au milieu de vous~; afin que vous ne soyez point battus par vos ennemis.
\VS{43}Je vous parlai, mais vous ne m'écoutâtes point et vous vous rebellâtes contre l'ordre de Yahweh, et vous fûtes orgueilleux et vous montâtes sur la montagne.
\VS{44}Et les Amoréens, qui demeuraient sur cette montagne, sortirent contre vous et vous poursuivirent comme font les abeilles~; et ils vous battirent depuis Séir jusqu'à Horma.
\VS{45}Et étant retournés vous pleurâtes devant Yahweh~; mais Yahweh n'écouta point votre voix, et ne vous prêta point l'oreille.
\VS{46}Ainsi, vous demeurâtes à Kadès plusieurs jours, autant de temps que vous y aviez demeuré.
\Chap{2}
\TextTitle{Périple du peuple dans le désert}
\VerseOne{}Alors nous retournâmes en arrière, et nous partîmes pour le désert, par le chemin de la Mer Rouge, comme Yahweh me l'avait dit, et nous tournâmes autour de la montagne de Séir plusieurs jours.
\VS{2}Et Yahweh me parla, en disant~:
\VS{3}Vous avez assez tourné autour de cette montagne. Tournez-vous vers le nord.
\VS{4}Ordonne au peuple, en disant~: Vous allez passer la frontière de vos frères, les fils d'Esaü, qui demeurent en Séir. Ils auront peur de vous~; mais soyez bien sur vos gardes.
\VS{5}N'ayez pas de démêlé avec eux~; car je ne vous donnerai rien dans leur pays, pas même de quoi poser la plante du pied~: J'ai donné à Esaü la montagne de Séir en héritage.
\VS{6}Vous achèterez d'eux la nourriture à prix d'argent et vous en mangerez, et vous achèterez d'eux l'eau à prix d'argent et vous en boirez.
\VS{7}Car Yahweh, ton Dieu, t'a béni dans tout le travail de tes mains, il a connu ta marche dans ce grand désert. Yahweh, ton Dieu, a été avec toi pendant ces quarante années, et tu n'as manqué de rien.
\VS{8}Nous passâmes à distance de nos frères, les fils d'Esaü, qui demeuraient en Séir, à distance du chemin de la plaine, d'Elath et d'Etsjon-Guéber, et nous nous tournâmes, et nous passâmes par le chemin du désert de Moab.
\VS{9}Yahweh me dit~: N'assiège point Moab, et ne t'engage pas dans un combat avec lui~; car je ne te donnerai rien en héritage dans son pays~: J'ai donné Ar en héritage aux fils de Lot\FTNT{Ge. 19:36-38.}.
\VS{10}Les Emim y habitaient auparavant~; c'était un peuple grand, nombreux et de haute taille comme les Anakim.
\VS{11}Ils étaient considérés comme des Rephaïm, de même que les Anakim~; mais les Moabites les appelaient Emim.
\VS{12}Séir était habité autrefois par les Horiens~; mais les fils d'Esaü les en dépossédèrent, les détruisirent devant eux, et y habitèrent à leur place, comme l'a fait Israël dans le pays de son héritage que Yahweh lui a donné.
\VS{13}Mais maintenant, levez-vous, et passez le torrent de Zéred. Et nous passâmes le torrent de Zéred.
\VS{14}Or le temps que nous avons marché de Kadès-Barnéa, jusqu'à ce que nous ayons passé le torrent de Zéred, fut de trente-huit ans, jusqu'à ce que toute la génération des hommes de guerre eût été consumée du milieu du camp, comme Yahweh le leur avait juré.
\VS{15}La main de Yahweh fut aussi sur eux pour les détruire du milieu du camp, jusqu'à ce qu'ils eussent été consumés.
\VS{16}Or il est arrivé qu'après que tous les hommes de guerre eurent été consumés par la mort du milieu du peuple,
\VS{17}Yahweh me parla, et dit~:
\VS{18}Tu vas passer aujourd'hui la frontière de Moab, à savoir Har.
\VS{19}Tu t'approcheras en face des fils d'Ammon, mais ne les assiège point, et ne t'engage point dans un combat avec eux~; car je ne te donnerai rien en possession dans le pays des fils d'Ammon~: Je l'ai donné en héritage aux fils de Lot.
\VS{20} Ce pays était aussi considéré comme un pays de Rephaïm~; car les Rephaïm y habitaient auparavant, et les Ammonites les appelaient Zamzummim~;
\VS{21}c'était un peuple grand, nombreux, et de haute taille, comme les Anakim, Yahweh les détruisit devant eux, et ils les dépossédèrent, et habitèrent à leur place.
\VS{22}Comme il fit pour les fils d'Esaü qui demeurent en Séir, quand il détruisit les Horiens devant eux~; ils les dépossédèrent et habitèrent à leur place jusqu'à ce jour.
\VS{23}Or quant aux Avviens, qui demeuraient en Hatserim jusqu'à Gaza, ils furent détruits par les Caphtorim, sortis de Caphtor, qui demeurèrent à leur place.
\TextTitle{Yawheh livre Sihon, roi de Hesbon, entre les mains d'Israël}
\VS{24}Levez-vous, partez et passez le torrent de l'Arnon. Regarde, j'ai livré entre tes mains Sihon, roi de Hesbon, l'Amoréen, et son pays. Commence à en prendre possession, et fais-lui la guerre~!
\VS{25}Aujourd'hui, je vais commencer à mettre la frayeur et la crainte de toi sur les peuples qui sont sous les cieux~; et ayant entendu parler de toi, ils trembleront et seront dans l'angoisse à cause de ta présence.
\VS{26}J'envoyai, du désert de Kedémoth, des messagers à Sihon, roi de Hesbon, avec des paroles de paix, disant\FTNT{No. 21:21.}~:
\VS{27}Permets que je passe par ton pays~; et j'irai par le grand chemin, sans me détourner ni à droite ni à gauche.
\VS{28}Tu me vendras de la nourriture à prix d'argent, afin que je mange, et tu me donneras de l'eau à prix d'argent, afin que je boive~; seulement que j'y passe de mes pieds.
\VS{29}C'est ce qu'ont fait les fils d'Esaü qui demeurent en Séir, et les Moabites qui demeurent à Ar, jusqu'à ce que je passe le Jourdain pour entrer au pays que Yahweh, notre Dieu, nous donne.
\VS{30}Mais Sihon, roi de Hesbon, ne voulut point nous laisser passer par son pays~; car Yahweh, ton Dieu, avait endurci son esprit, et raidit son cœur afin de le livrer entre tes mains, comme tu le vois aujourd'hui.
\VS{31}Yahweh me dit~: Regarde, j'ai commencé à te livrer Sihon et son pays~; commence à posséder son pays, pour le tenir en héritage.
\VS{32}Sihon donc, sortit nous rencontrer avec tout son peuple pour nous combattre à Jahats.
\VS{33}Mais Yahweh, notre Dieu, nous le livra en face, et nous le battîmes, lui, ses fils, et tout son peuple.
\VS{34}Et en ce temps-là, nous prîmes toutes ses villes, et nous détruisîmes par le moyen de l'interdit les villes, les hommes, les femmes, et les petits enfants, sans laisser de survivants.
\VS{35}Seulement nous pillâmes les bêtes pour nous, et le butin des villes que nous avions prises.
\VS{36}Depuis Aroër, qui est sur le bord du torrent de l'Arnon, et la ville qui est dans la vallée, jusqu'à Galaad, il n'y eut pas une ville qui fût trop haute pour nous~: Yahweh, notre Dieu, nous livra tout.
\VS{37}Seulement tu n'approchas point du pays des fils d'Ammon, de tous les bords du torrent de Jabbok, des villes de la montagne, ni d'aucun lieu que Yahweh, notre Dieu, t'avait ordonné de ne point attaquer.
\Chap{3}
\TextTitle{Yawheh livre Og, roi de Basan, entre les mains d'Israël}
\VerseOne{}Alors nous nous tournâmes, et nous montâmes par le chemin de Basan. Et Og, roi de Basan, sortit nous rencontrer, avec tout son peuple, pour nous combattre à Edréi.
\VS{2}Et Yahweh me dit~: Ne le crains point~; car je le livre entre tes mains, lui, tout son peuple, et son pays~; et tu lui feras comme tu as fait à Sihon, roi des Amoréens, qui demeurait à Hesbon.
\VS{3}Ainsi Yahweh, notre Dieu, livra aussi entre nos mains Og, roi de Basan, avec tout son peuple~; nous le battîmes sans laisser de survivants.
\VS{4}En ce même temps, nous prîmes aussi toutes ses villes, et il n'y eut point de ville que nous ne lui prîmes pas~: Soixante villes, toute la contrée d'Argob, le royaume d'Og en Basan.
\VS{5}Toutes ces villes-là étaient fortifiées, avec de hautes murailles, des portes et des barres. Il y avait aussi des villes sans murailles en fort grand nombre.
\VS{6}Et nous les détruisîmes par le moyen de l'interdit, comme nous l'avions fait à Sihon, roi de Hesbon~; nous dévouâmes par le moyen de l'interdit toutes les villes, les hommes, les femmes, et les petits enfants.
\VS{7}Mais nous pillâmes pour nous toutes les bêtes et le butin des villes.
\VS{8}Nous prîmes donc, en ce temps-là, le pays de la main des deux rois des Amoréens, qui étaient de l'autre côté du Jourdain, depuis le torrent de l'Arnon jusqu'à la montagne de l'Hermon~;
\VS{9}or les Sidoniens donnent à l'Hermon le nom de Sirion, mais les Amoréens le nomment Senir~;
\VS{10}toutes les villes de la plaine, tout Galaad, et tout Basan jusqu'à Salca et Edréi, les villes du royaume d'Og en Basan.
\VS{11}Og, roi de Basan, avait survécu seul du reste des Rephaïm. Voici, son lit, un lit de fer, n'est-il pas dans Rabbath, ville des fils d'Ammon~? Sa longueur est de neuf coudées, et sa largeur de quatre coudées, en coudées d'homme.
\TextTitle{Premières terres attribuées à Ruben, Gad et à la demi-tribu de Manassé}
\VS{12}En ce temps-là donc, nous prîmes possession de ce pays. Je donnai aux Rubénites et aux Gadites le territoire à partir d'Aroër, sur le torrent de l'Arnon, et la moitié de la montagne de Galaad, avec ses villes\FTNT{Jos. 13:23-32.}.
\VS{13}Je donnai à la demi-tribu de Manassé le reste de Galaad et tout le royaume d'Og, en Basan~: Toute la contrée d'Argob avec tout le Basan, c'est ce qu'on appelait le pays des Réphaïm.
\VS{14}Jaïr, fils de Manassé, prit toute la contrée d'Argob jusqu'à la frontière des Gueschuriens et des Maacathiens, et il donna son nom à Basan, appelé villages de Jaïr jusqu'à aujourd'hui.
\VS{15}Je donnai aussi Galaad à Makir.
\VS{16}Mais aux Rubénites et aux Gadites, je donnai de Galaad jusqu'au torrent de l'Arnon, dont le milieu du torrent sert de frontière, et jusqu'au torrent de Jabbok, frontière des fils d'Ammon~;
\VS{17}la plaine, et le Jourdain, de la frontière de Kinnéreth jusqu'à la mer de la plaine, la Mer Salée, aux pieds de Pisga vers l'orient.
\VS{18}Or en ce temps-là, je vous ordonnai, en disant~: Yahweh, votre Dieu, vous donne ce pays pour le posséder. Vous tous, qui êtes vaillants, vous passerez armés devant vos frères, les fils d'Israël.
\VS{19}Seulement vos femmes, vos petits-enfants, et vos troupeaux, car je sais que vous avez beaucoup de troupeaux, resteront dans les villes que je vous ai données,
\VS{20}jusqu'à ce que Yahweh ait accordé du repos à vos frères comme à vous, et qu'eux aussi possèdent le pays que Yahweh, votre Dieu, leur donne de l'autre côté du Jourdain. Puis vous retournerez chacun dans l'héritage que je vous ai donné.
\VS{21}En ce temps-là, j'ordonnai à Josué, en disant~: Tes yeux ont vu tout ce que Yahweh, votre Dieu, a fait à ces deux rois~: Yahweh en fera de même à tous les royaumes vers lesquels tu vas passer.
\VS{22}Ne les craignez point~; car Yahweh, votre Dieu, combattra lui-même pour vous.
\TextTitle{Moïse n'entrera pas dans la terre promise}
\VS{23}En ce même temps, j'implorai la grâce de Yahweh, en disant~:
\VS{24}Seigneur Yahweh, tu as commencé à montrer à ton serviteur ta grandeur et ta main puissante~; car quel est le dieu dans le ciel et sur la terre qui puisse faire selon tes œuvres et selon ta puissance~?
\VS{25}Que je passe, je te prie, et que je voie ce bon pays de l'autre côté du Jourdain, ces bonnes montagnes et le Liban.
\VS{26}Mais Yahweh s'irrita contre moi, à cause de vous, et ne m'écouta point. Yahweh me dit~: C'est assez, ne me parle plus de cette affaire.
\VS{27}Monte au sommet du Pisga, et lève tes yeux à l'occident, au nord, au sud, et à l'orient, et regarde de tes yeux~; car tu ne passeras point ce Jourdain.
\VS{28}Donnes-en la charge à Josué, fortifie-le et affermis-le~; car c'est lui qui passera devant ce peuple et qui le mettra en possession du pays que tu verras.
\VS{29}Ainsi nous demeurâmes dans la vallée, vis-à-vis de Beth-Peor.
\Chap{4}
\TextTitle{Encouragement à garder la loi de Yahweh}
\VerseOne{}Et maintenant Israël, écoute les lois et les ordonnances que je vous enseigne, pour les pratiquer afin que vous viviez, que vous entriez et possédiez le pays que Yahweh, le Dieu de vos pères, vous donne.
\VS{2}Vous n'ajouterez\FTNT{De. 12:32~; Pr. 30:6~; Ap. 22:18-19.} rien à la parole que je vous ordonne, et vous n'en retrancherez rien~; afin de garder les commandements de Yahweh, votre Dieu, que je vous ordonne.
\VS{3}Vos yeux ont vu ce que Yahweh a fait à cause de Baal-Peor~: Yahweh, ton Dieu, a détruit du milieu de toi tout homme qui était allé après Baal-Peor\FTNT{No. 25:4-9.}.
\VS{4}Mais vous, qui vous êtes attachés à Yahweh, votre Dieu, vous êtes tous vivants aujourd'hui.
\VS{5}Regardez, je vous ai enseigné des lois et des ordonnances, comme Yahweh, mon Dieu, me l'a ordonné, afin que vous les pratiquiez au milieu du pays où vous allez pour le posséder.
\VS{6}Vous les garderez et vous les pratiquerez, car c'est là votre sagesse et votre intelligence aux yeux de tous les peuples, qui entendront ces lois, et qui diront~: Cette grande nation est un peuple sage et intelligent~!
\TextTitle{Israël, privilégié parmi tous les peuples}
\VS{7}Car quelle est la grande nation qui ait ses dieux proches d'elle, comme nous avons Yahweh, notre Dieu, toutes les fois que nous l'invoquons~?
\VS{8}Et quelle est la grande nation qui ait des lois et des ordonnances justes, comme toute cette loi que je mets aujourd'hui devant vous~?
\VS{9}Seulement, prends garde à toi et garde soigneusement ton âme, afin que tous les jours de ta vie tu n'oublies point les choses que tes yeux ont vues, et qu'elles ne sortent de ton cœur\FTNT{Pr. 4:23.}~; enseigne-les à tes fils, et aux fils de tes fils.
\VS{10}Rappelle-toi du jour où tu te tins face à Yahweh, ton Dieu, à Horeb, après que Yahweh me dit~: Convoque le peuple~! Je veux leur faire entendre mes paroles, pour qu'ils apprennent à me craindre tout le temps qu'ils seront vivants sur la terre~; et pour qu'ils les enseignent à leurs fils.
\VS{11}Et que vous vous approchâtes, et vous vous tîntes au pied de la montagne. Or la montagne était embrasée de feu jusqu'au milieu du ciel. Il y avait des ténèbres, une nuée, et une obscurité.
\VS{12}Et Yahweh vous parla du milieu du feu~; vous entendîtes le son de ses paroles, mais vous ne vîtes aucune image, vous entendîtes seulement la voix\FTNT{Ex. 19:17-19.}.
\VS{13}Et il déclara son alliance, qu'il vous ordonna d'observer, les dix paroles, qu'il écrivit sur deux tables de pierre.
\VS{14}Yahweh m'ordonna aussi, en ce temps-là, de vous enseigner les lois et les ordonnances, afin que vous les pratiquiez dans le pays que vous allez posséder.
\VS{15}Prenez bien garde à vos âmes, puisque vous n'avez vu aucune image le jour où Yahweh, votre Dieu, vous parla du milieu du feu à Horeb,
\VS{16}de peur que vous ne vous corrompiez et que vous ne vous fassiez une image taillée, une représentation d'idole ayant la forme d'un mâle ou d'une femelle,
\VS{17}ou la forme d'un animal qui soit sur la terre, ou la forme d'un oiseau ailé qui vole dans les cieux,
\VS{18}ou la forme d'un animal qui rampe sur la terre, ou la forme d'un poisson qui soit dans les eaux au-dessous de la terre.
\VS{19}De peur aussi qu'élevant tes yeux vers les cieux, et voyant le soleil, la lune, et les étoiles, toute l'armée des cieux, tu ne sois poussé à te prosterner devant elles, et que tu ne les serves~: C'est ce que Yahweh, ton Dieu, a donné en partage à tous les peuples, sous tous les cieux.
\VS{20}Mais vous, Yahweh vous a pris, et vous a fait sortir d'Egypte, du fourneau de fer, afin que vous fussiez un peuple de son héritage, comme vous l'êtes aujourd'hui.
\TextTitle{Conséquences de la désobéissance et de l'idolâtrie}
\VS{21}Or Yahweh s'irrita contre moi, à cause de vos paroles, et il jura que je ne passerais point le Jourdain, et que je n'entrerais point dans ce bon pays que Yahweh, ton Dieu, te donne en héritage.
\VS{22}Je mourrai dans ce pays-ci, je ne passerai point le Jourdain~; mais vous le passerez, et vous posséderez ce bon pays.
\VS{23}Gardez-vous d'oublier l'alliance de Yahweh, votre Dieu, qu'il a traitée avec vous, et que vous ne vous fassiez d'image taillée, de représentation quelconque, que Yahweh, votre Dieu, vous a défendues.
\VS{24}Car Yahweh, ton Dieu, est un feu dévorant\FTNT{Hé. 12:29.}, un Dieu jaloux.
\VS{25}Quand tu auras engendré des fils, et des fils de tes fils, et que vous serez depuis longtemps dans le pays, si vous vous corrompez, et que vous faites des images taillées, ou des représentations de quelque chose que ce soit, si vous faites ce qui est mal aux yeux de Yahweh, votre Dieu, afin de l'irriter
\VS{26}j'appelle aujourd'hui à témoin les cieux et la terre contre vous, certainement vous périrez promptement dans ce pays que vous allez posséder au-delà du Jourdain, vous n'y prolongerez point vos jours, car vous serez entièrement détruits.
\VS{27}Yahweh vous dispersera parmi les peuples, et vous ne resterez qu'un petit nombre parmi les nations, chez lesquelles Yahweh vous emmènera.
\VS{28}Et là, vous servirez des dieux, œuvres de main d'homme, du bois et de la pierre, qui ne peuvent voir, ni entendre, ni manger, ni sentir\FTNT{Es. 44:9~; Es. 46:7~; Ps. 115:4-7}.
\TextTitle{Yahweh, puissant, miséricordieux et fidèle à son alliance}
\VS{29}Mais de là, tu chercheras Yahweh, ton Dieu, et tu le trouveras, si tu le cherches de tout ton cœur et de toute ton âme.
\VS{30}Quand tu seras dans la détresse, et que toutes ces choses te seront arrivées, alors, dans les derniers jours, tu retourneras à Yahweh, ton Dieu, et tu obéiras à sa voix~;
\VS{31}parce que Yahweh, ton Dieu, est le Dieu puissant et miséricordieux, il ne t'abandonnera point et ne te détruira point, il n'oubliera point l'alliance de tes pères qu'il leur a jurée.
\VS{32}Interroge les premiers temps, qui ont été avant toi, depuis le jour que Dieu créa l'homme sur la terre, et d'une extrémité des cieux à l'autre, s'il a jamais été rien fait de semblable à cette grande chose, et s'il a été jamais rien entendu de semblable.
\VS{33}Est-ce qu'un peuple a entendu la voix de Dieu parlant du milieu du feu, comme tu l'as entendue, et qui soit demeuré en vie~?
\VS{34}Où Dieu a-t-il essayé de venir prendre pour lui une nation du milieu d'une nation, par des épreuves, des signes, des miracles, et des batailles, à main forte, et à bras étendu, et par des choses grandes et terribles, comme tout ce que Yahweh, notre Dieu, a fait pour vous en Egypte, sous vos yeux~?
\VS{35}Cela t'a été montré afin que tu reconnaisses que Yahweh est Dieu et qu'il n'y en a point d'autre.
\VS{36}Il t'a fait entendre sa voix des cieux pour t'instruire~; et il t'a montré son grand feu sur la terre, et tu as entendu ses paroles du milieu du feu.
\VS{37}Et parce qu'il a aimé tes pères, il a choisi leur postérité après eux et il t'a retiré d'Egypte en sa présence, par sa grande puissance~;
\VS{38}pour chasser de devant toi des nations plus grandes et plus puissantes que toi, pour te faire entrer dans leur pays, et pour te le donner en héritage, comme tu le vois aujourd'hui.
\VS{39}Sache donc aujourd'hui, et rappelle dans ton cœur que Yahweh est Dieu, en haut dans les cieux et sur la terre, et qu'il n'y en a point d'autre.
\VS{40}Garde donc ses lois et ses commandements que je t'ordonne aujourd'hui, afin que tu sois heureux, toi et tes fils après toi, et que tu prolonges tes jours sur la terre que Yahweh, ton Dieu, te donne\FTNT{Ex 20.}.
\TextTitle{Trois villes de refuge à l'est du Jourdain}
\VS{41}Alors Moïse sépara trois villes de l'autre côté du Jourdain vers le soleil levant,
\VS{42}afin que le meurtrier qui aurait tué son prochain involontairement, sans l'avoir haï auparavant, s'y enfuie~; et qu'en s'enfuyant dans l'une de ces villes-là, il eût sa vie sauve.
\VS{43}C'étaient~: Betser dans le désert, dans la plaine du pays, chez les Rubénites~; Ramoth en Galaad, chez les Gadites~; Golan en Basan, chez les Manassites.
\VS{44}C'est ici la loi que Moïse plaça face aux enfants d'Israël.
\VS{45}Voici les témoignages, les lois, et les ordonnances que Moïse déclara aux enfants d'Israël, après qu'ils furent sortis d'Egypte.
\VS{46}C'était de l'autre côté du Jourdain, dans la vallée, vis-à-vis de Beth-Peor, au pays de Sihon, roi des Amoréens, qui demeurait à Hesbon, et qui fut battu par Moïse et les enfants d'Israël après être sortis d'Egypte.
\VS{47}Et ils s'emparèrent de son pays avec le pays d'Og, roi de Basan, deux rois des Amoréens qui étaient de l'autre côté du Jourdain, vers le soleil levant.
\VS{48}Depuis Aroër, sur le bord du torrent de l'Arnon, jusqu'à la montagne de Sion, qui est l'Hermon,
\VS{49}et toute la plaine de l'autre côté du Jourdain vers l'orient, jusqu'à la mer de la plaine, au pied du Pisga.
\Chap{5}
\TextTitle{L'alliance établie à Horeb rappelée à la nouvelle génération}
\VerseOne{}Moïse appela tout Israël, et leur dit~: Ecoute, Israël, les lois et les ordonnances que je prononce aujourd'hui à vos oreilles, apprenez-les, et veillez à les mettre en pratique.
\VS{2}Yahweh, notre Dieu, a traité avec nous une alliance en Horeb\FTNT{Ex. 19:5.}.
\VS{3}Dieu n'a point traité cette alliance avec nos pères, mais avec nous, qui sommes ici aujourd'hui tous vivants.
\VS{4}Yahweh vous parla face à face sur la montagne du milieu du feu.
\VS{5}Je me tenais en ce temps-là entre Yahweh et vous, pour vous rapporter la parole de Yahweh~; parce que vous aviez peur face à ce feu, et vous ne montâtes point sur la montagne. Il dit\FTNT{Les dix paroles (Ex. 20).}~:
\VS{6}Je suis Yahweh, ton Dieu, qui t'ai fait sortir du pays d'Egypte, de la maison de servitude.
\VS{7}Tu n'auras point d'autres dieux devant ma face.
\VS{8}Tu ne te feras point d'image taillée, ni de représentation des choses qui sont en haut dans les cieux, ni sur la terre, ni dans les eaux sous la terre.
\VS{9}Tu ne te prosterneras point devant elles, et tu ne les serviras point~; car je suis Yahweh, ton Dieu, un Dieu jaloux, qui punis l'iniquité des pères sur les enfants jusqu'à la troisième et à la quatrième génération de ceux qui me haïssent,
\VS{10}et qui fais miséricorde jusqu'à mille générations à ceux qui m'aiment et qui gardent mes commandements.
\VS{11}Tu ne prendras point le Nom de Yahweh, ton Dieu, en vain~; car Yahweh ne tiendra pas pour innocent celui qui prendra son Nom en vain.
\VS{12}Garde le jour du sabbat pour le sanctifier, comme Yahweh, ton Dieu, te l'a ordonné.
\VS{13}Tu travailleras six jours, et tu feras toute ton œuvre,
\VS{14}mais le septième jour est le sabbat de Yahweh, ton Dieu~: Tu ne feras aucune œuvre, ni ton fils, ni ta fille, ni ton serviteur, ni ta servante, ni ton bœuf, ni ton âne, ni aucune de tes bêtes, ni l'étranger qui est dans tes portes, afin que ton serviteur et ta servante se reposent comme toi.
\VS{15}Et tu te souviendras que tu as été esclave au pays d'Egypte, et que Yahweh, ton Dieu, t'en a fait sortir à main forte et à bras étendu~: C'est pourquoi Yahweh, ton Dieu, t'a ordonné d'observer le jour du sabbat.
\VS{16}Honore ton père et ta mère, comme Yahweh, ton Dieu, te l'a ordonné, afin que tes jours se prolongent et que tu sois heureux sur la terre que Yahweh, ton Dieu, te donne.
\VS{17}Tu ne tueras point.
\VS{18}Tu ne commettras point d'adultère.
\VS{19}Tu ne déroberas point.
\VS{20}Tu ne diras point de faux témoignage contre ton prochain.
\VS{21}Tu ne convoiteras point la femme de ton prochain~; tu ne désireras point la maison de ton prochain, ni son champ, ni son serviteur, ni sa servante, ni son bœuf, ni son âne, ni aucune chose qui soit à ton prochain.
\VS{22}Yahweh déclara ces paroles à toute votre assemblée sur la montagne, du milieu du feu, des nuées et de l'obscurité, à voix forte sans rien ajouter. Il les écrivit sur deux tables de pierre qu'il me donna.
\TextTitle{Moïse, intermédiaire entre Yahweh et le peuple}
\VS{23}Or il arriva qu'aussitôt que vous eûtes entendu la voix du milieu de l'obscurité, parce que la montagne était embrasée par le feu, vos chefs de tribus et vos anciens s'approchèrent de moi,
\VS{24}et vous dîtes~: Voici, Yahweh, notre Dieu, nous a fait voir sa gloire et sa grandeur, et nous avons entendu sa voix du milieu du feu~; aujourd'hui, nous avons vu que Dieu a parlé avec l'homme, et qu'il est resté en vie.
\VS{25}Et maintenant pourquoi mourrions-nous~? Car ce grand feu là nous dévorera~; si nous entendons encore la voix de Yahweh, notre Dieu, nous mourrons.
\VS{26}Car qui, de toute chair, a entendu comme nous la voix du Dieu vivant parlant du milieu du feu, et qui soit resté en vie~?
\VS{27}Approche-toi et écoute tout ce que Yahweh, notre Dieu, dira~; puis tu nous diras tout ce que Yahweh, notre Dieu, t'aura dit~; nous l'entendrons, et nous le ferons.
\VS{28}Yahweh entendit la voix de vos paroles pendant que vous me parliez. Et Yahweh me dit~: J'ai entendu les paroles que ce peuple t'ont adressées~: Tout ce qu'ils ont dit est bien.
\VS{29}Ô~! S'ils avaient toujours ce même cœur pour me craindre et pour garder tous mes commandements, afin qu'ils fussent heureux, eux et leurs enfants, pour toujours~!
\VS{30}Va, dis-leur~: Retournez dans vos tentes.
\VS{31}Mais toi, reste ici avec moi, et je te dirai tous les commandements, les lois, et les ordonnances que tu leur enseigneras, afin qu'ils les pratiquent dans le pays que je leur donne en possession.
\VS{32}Vous prendrez donc garde de faire ce que Yahweh, votre Dieu, vous a ordonné~; vous ne vous en détournerez ni à droite ni à gauche.
\VS{33}Vous marcherez dans toute la voie que Yahweh, votre Dieu, vous a ordonnée, afin que vous viviez et que vous soyez heureux, et que vous prolongiez vos jours sur la terre que vous posséderez.
\Chap{6}
\TextTitle{Obéissance à la loi, source de bénédictions}
\VerseOne{}Voici les commandements, les lois et les ordonnances que Yahweh, votre Dieu, m'a ordonné de vous enseigner, afin que vous les pratiquiez dans le pays dans lequel vous allez passer pour le posséder~;
\VS{2}afin que tu craignes Yahweh, ton Dieu, en gardant durant tous les jours de ta vie, toi, ton fils, et le fils de ton fils, toutes ses lois et ses commandements que je t'ordonne, pour que tes jours soient prolongés.
\VS{3}Tu les écouteras donc, ô Israël, et tu auras soin de les mettre en pratique, afin que tu sois heureux, et que vous vous multipliiez sur la terre où coulent le lait et le miel, comme Yahweh, le Dieu de tes pères, l'a dit\FTNT{Ex. 3:8.}.
\VS{4}Ecoute Israël~! Yahweh, notre Dieu, Yahweh est Un\FTNT{Jacob fut le premier à faire cette prière qui affirme l'unicité de Dieu. Dieu est UN (en hébreu «~Echad~» ou «~Ehad~»). Loin de l'infirmer ou de la contredire, Jésus a confirmé cette prière et l'enseignement capital qu'elle contient (Mc. 12:29). Dieu n'est pas trois personnes en une, mais UN. Cette parole annonce un monothéisme absolu. Elle s'oppose catégoriquement au polythéisme des Cananéens qui adoraient de multiples dieux, les étoiles, la lune, le soleil, les arbres, les rois etc. Aussi les Hébreux avaient reçu l'ordre de la part de Yahweh de détruire toutes les idoles qu'ils trouveraient en la terre promise (De. 16:21). Voir également commentaire en Ge. 1:5.}.
\VS{5}Tu aimeras donc Yahweh, ton Dieu, de tout ton cœur, de toute ton âme, et de toute ta force\FTNT{Mt. 22:37~; Mc. 12:30.}.
\TextTitle{La loi de Yahweh doit être enseignée aux enfants}
\VS{6}Et ces paroles, que je t'ordonne aujourd'hui, seront dans ton cœur.
\VS{7}Tu les enseigneras soigneusement à tes enfants, et tu en parleras quand tu te tiendras dans ta maison, quand tu iras en voyage, quand tu te coucheras et quand tu te lèveras.
\VS{8}Et tu les lieras comme un signe sur tes mains, et elles seront comme des fronteaux entre tes yeux.
\VS{9}Tu les écriras aussi sur les poteaux de ta maison et sur tes portes.
\VS{10}Yahweh, ton Dieu, te fera entrer dans le pays qu'il a juré à tes pères, Abraham, Isaac, et Jacob, de te donner. Tu posséderas de grandes et bonnes villes que tu n'as point bâties,
\VS{11}des maisons pleines de toutes sortes de biens que tu n'as point remplies, des puits creusés que tu n'as point creusés, des vignes et des oliviers que tu n'as point plantés, tu mangeras, et tu te rassasieras.
\VS{12}Prends garde à toi, de peur que tu n'oublies Yahweh, qui t'a fait sortir du pays d'Egypte, de la maison de servitude.
\VS{13}Tu craindras Yahweh, ton Dieu, tu le serviras et tu jureras par son Nom.
\VS{14}Vous n'irez point après d'autres dieux, d'entre les dieux des peuples qui sont autour de vous~;
\VS{15}car Yahweh, ton Dieu, est un Dieu jaloux au milieu de toi~; de peur que la colère de Yahweh, ton Dieu, ne s'enflamme contre toi, et qu'il ne t'extermine de dessus la terre.
\VS{16}Vous ne tenterez point Yahweh, votre Dieu, comme vous l'avez tenté à Massa.
\VS{17}Vous garderez soigneusement les commandements de Yahweh, votre Dieu, ses ordonnances et ses lois qu'il vous a ordonnées.
\VS{18}Tu feras ce qui est droit et bon aux yeux de Yahweh, afin que tu sois heureux, que tu entres et que tu possèdes le bon pays que Yahweh a juré à tes pères,
\VS{19}après qu'il aura chassé tous tes ennemis de devant toi, comme Yahweh l'a dit.
\VS{20}Quand ton enfant t'interrogera à l'avenir, en disant~: Que veulent dire ces préceptes, ces lois, et ces ordonnances que Yahweh, notre Dieu, vous a ordonnés~?
\VS{21}Tu diras à ton enfant~: Nous étions esclaves de Pharaon en Egypte, et Yahweh nous a fait sortir de l'Egypte par sa main puissante.
\VS{22}Yahweh a fait sous nos yeux des signes et des miracles, grands et désastreux contre l'Egypte, contre Pharaon et contre toute sa maison~;
\VS{23}et il nous a fait sortir de là pour nous conduire dans le pays qu'il avait juré à nos pères de nous donner.
\VS{24}Yahweh nous a ordonné de pratiquer toutes ces lois, et de craindre Yahweh, notre Dieu, afin que nous soyons toujours heureux, et qu'il préserve notre vie, comme aujourd'hui.
\VS{25}Et ceci sera notre justice, que nous prenions garde de pratiquer tous ces commandements devant Yahweh, notre Dieu, comme il nous l'a ordonné.
\Chap{7}
\TextTitle{Yahweh interdit les alliances avec les peuples païens}
\VerseOne{}Quand Yahweh, ton Dieu, t'aura fait entrer dans le pays où tu vas entrer pour le posséder, et qu'il aura chassé de devant toi beaucoup de nations~: Les Héthiens, les Guirgasiens, les Amoréens, les Cananéens, les Phéréziens, les Héviens, et les Jébusiens, sept nations plus grandes et plus puissantes que toi~;
\VS{2}et que Yahweh, ton Dieu, te les aura livrées en face et que tu les auras battues, tu les dévoueras complètement à la façon de l'interdit, tu ne traiteras point d'alliance avec elles, et tu ne leur feras point de grâce.
\VS{3}Tu ne t'allieras point par mariage avec elles, tu ne donneras point tes filles à leurs fils, et tu ne prendras point leurs filles pour tes fils\FTNT{Jos. 23:12-13.}~;
\VS{4}car elles détourneraient de moi tes fils, et ils serviraient d'autres dieux, et la colère de Yahweh s'enflammerait contre vous~: Il te détruirait promptement.
\VS{5}Mais vous les traiterez de cette manière~: Vous renverserez leurs autels, vous briserez leurs statues, vous abattrez leurs Asherah\FTNT{Le mot idole vient de l'hébreu «~Asherah~». Il est cité au moins quarante fois dans le Tanakh. Il fait référence à un objet en bois utilisé dans le culte d'Astarté, l'épouse de Baal. Voir De. 19:21.}, et vous brûlerez au feu leurs images taillées.
\VS{6}Car tu es un peuple saint pour Yahweh, ton Dieu. Yahweh, ton Dieu, t'a choisi pour que tu sois pour lui un peuple précieux entre tous les peuples qui sont sur la face de la terre.
\VS{7}Ce n'est pas parce que vous êtes plus nombreux que tous les peuples que Yahweh vous a aimé et qu'il vous a choisis~; car vous êtes le plus petit de tous les peuples.
\VS{8}Mais c'est parce que Yahweh vous aime, et qu'il garde le serment qu'il a juré à vos pères, Yahweh vous a fait sortir par sa main puissante, et vous a rachetés de la maison de servitude, de la main de Pharaon, roi d'Egypte.
\VS{9}Sache que c'est Yahweh, ton Dieu, qui est Dieu. Ce Dieu fidèle garde son alliance et sa miséricorde jusqu'à mille générations envers ceux qui l'aiment et qui gardent ses commandements,
\VS{10}et qui rend la pareille en face à ceux qui le haïssent, et les fait périr~; il ne diffère point envers celui qui le hait, il lui rend la pareille en face.
\VS{11}Garde les commandements, les lois, et les ordonnances que je t'ordonne aujourd'hui, et mets-les en pratique.
\TextTitle{L'obéissance à Yahweh, source de bénédictions et de victoires}
\VS{12}Et il arrivera que si vous écoutez ces ordonnances, si vous les gardez et les mettez en pratique, Yahweh, ton Dieu, gardera l'alliance et la bonté qu'il a jurées à tes pères.
\VS{13}Et il t'aimera, te bénira, et te multipliera~; il bénira le fruit de tes entrailles, et le fruit de ta terre, ton blé, ton vin, et ton huile, les portées de ton gros et de ton menu bétail, sur la terre qu'il a juré de donner à tes pères.
\VS{14}Tu seras béni plus que tous les peuples~; il n'y aura chez toi et parmi tes bêtes, ni mâle ni femelle stérile\FTNT{Ex. 23:26.}.
\VS{15}Yahweh détournera de toi toute maladie~; il ne t'enverra aucun de ces mauvais maux d'Egypte qui te sont connus, mais il les fera venir sur tous ceux qui te haïssent.
\VS{16}Tu détruiras donc tous les peuples que Yahweh, ton Dieu, va te livrer, ton œil n'aura point de pitié, et tu ne serviras point leurs dieux, car cela te serait un piège.
\VS{17}Si tu dis dans ton cœur~: Ces nations sont plus nombreuses que moi, comment pourrai-je les déposséder~?
\VS{18}Ne les crains point. Rappelle-toi bien ce que Yahweh, ton Dieu, a fait à Pharaon, et à tous les Egyptiens,
\VS{19}de ces grandes épreuves que tes yeux ont vues, les signes et les miracles, la main forte et le bras étendu par lesquels Yahweh, ton Dieu, t'a fait sortir~; ainsi fera Yahweh, ton Dieu, à tous ces peuples que tu crains.
\VS{20}Yahweh, ton Dieu, enverra contre eux les frelons, jusqu'à ce que périssent ceux qui resteront, et ceux qui se seront cachés de devant toi.
\VS{21}Ne t'effraie point devant eux, car Yahweh, ton Dieu, le Dieu grand et terrible est au milieu de toi.
\VS{22}Or Yahweh, ton Dieu, chassera peu à peu ces nations de devant toi~; tu ne pourras pas les exterminer promptement, de peur que les bêtes des champs ne se multiplient contre toi.
\VS{23}Mais Yahweh, ton Dieu, les livrera devant toi~; et il les troublera par de grandes confusions, jusqu'à ce qu'elles soient détruites.
\VS{24}Et il livrera leurs rois entre tes mains, et tu feras disparaître leurs noms de dessous les cieux~; aucun homme ne tiendra face à toi, jusqu'à ce que tu les aies détruits.
\VS{25}Tu brûleras au feu les images taillées de leurs dieux. Tu ne convoiteras point et tu ne prendras point pour toi l'argent et l'or qui seront sur elles, de peur que tu en sois pris au piège~; car c'est une abomination pour Yahweh, ton Dieu.
\VS{26}Ainsi tu n'introduiras point de choses abominables dans ta maison, afin que tu ne sois pas, comme cette chose, dévoué par interdit~; tu la détesteras fortement, et tu l'auras en abomination, car c'est une chose dévouée par interdit.
\Chap{8}
\TextTitle{Le désert, lieu de formation, d'humiliation et d’épreuve}
\VerseOne{}Vous observerez et vous mettrez en pratique tous les commandements que je vous ordonne aujourd'hui, afin que vous viviez, que vous multipliiez, et que vous entriez en possession du pays que Yahweh a juré de donner à vos pères.
\VS{2}Et souviens-toi de tout le chemin par lequel Yahweh, ton Dieu, t'a fait marcher pendant ces quarante ans dans ce désert, afin de t'humilier et de t'éprouver, pour connaître ce qui était dans ton cœur, et si tu garderais ses commandements ou non.
\VS{3}Il t'a donc humilié, il t'a laissé avoir faim, mais il t'a nourri de la manne, que tu ne connaissais pas et que tes pères n'avaient pas connue, afin de te faire connaître que l'homme ne vivra pas de pain seulement, mais que l'homme vivra de tout ce qui sort de la bouche de Yahweh\FTNT{Mt. 4:4~; Lu. 4:4.}.
\VS{4}Ton vêtement ne s'est point usé sur toi, et ton pied ne s'est point enflé durant ces quarante années\FTNT{Né. 9:21.}.
\VS{5}Reconnais dans ton cœur que Yahweh, ton Dieu, te châtie comme un homme châtie son enfant\FTNT{Hé. 12:5-12.}.
\TextTitle{Se garder d'oublier Yahweh}
\VS{6}Et garde les commandements de Yahweh, ton Dieu, pour marcher dans ses voies, et pour le craindre.
\VS{7}Car Yahweh, ton Dieu, va te faire entrer dans un bon pays, un pays de torrents d'eaux, de fontaines et d'abîmes, qui jaillissent des vallées et des montagnes~;
\VS{8}un pays de blé, d'orge, de vignes, de figuiers, et de grenadiers~; un pays d'oliviers donnant de l'huile et du miel~;
\VS{9}un pays où tu ne mangeras point le pain avec disette, où tu ne manqueras de rien~; un pays dont les pierres sont du fer, et des montagnes desquelles tu tailleras l'airain.
\VS{10}Tu mangeras et tu te rassasieras, tu béniras Yahweh, ton Dieu, pour le bon pays qu'il t'a donné.
\VS{11}Prends garde à toi de peur que tu n'oublies Yahweh, ton Dieu, en ne gardant point ses commandements, ses ordonnances, et ses lois que je t'ordonne aujourd'hui~;
\VS{12}de peur que quand tu mangeras et que tu seras rassasié~; que tu bâtiras et habiteras de belles maisons~;
\VS{13}que ton gros et menu bétail se multipliera~; que ton argent et ton or augmentera, et que tout ce qui est à toi se multipliera,
\VS{14}que ton cœur ne s'élève point et que tu n'oublies point Yahweh, ton Dieu, qui t'a fait sortir du pays d'Egypte, de la maison de servitude,
\VS{15}qui t'a fait marcher dans ce grand et affreux désert de serpents brûlants et de scorpions, dans des lieux arides et sans eau, et qui a fait jaillir pour toi de l'eau du rocher le plus dur,
\VS{16}qui t'as fait manger dans ce désert la manne que tes pères n'avaient point connue, afin de t'humilier et de t'éprouver, pour te faire ensuite du bien,
\VS{17}et que tu ne dises dans ton cœur~: Ma force et la puissance de ma main m'ont acquis ces richesses.
\VS{18}Mais tu te souviendra de Yahweh, ton Dieu, car c'est lui qui te donne de la force pour acquérir ces richesses, afin de confirmer son alliance, qu'il a jurée à tes pères, comme tu le vois aujourd'hui.
\VS{19}Mais si tu oublies Yahweh, ton Dieu, et que tu vas après d'autres dieux, si tu les sers, et que tu te prosternes devant eux, je vous avertis aujourd'hui que vous périrez certainement.
\VS{20}Vous périrez comme les nations que Yahweh fait périr devant vous, parce que vous n'aurez pas obéi à la voix de Yahweh, votre Dieu.
\Chap{9}
\TextTitle{Yahweh, fidèle à son alliance malgré la rébellion du peuple}
\VerseOne{}Ecoute, Israël~! Tu vas passer aujourd'hui le Jourdain, pour aller posséder des nations plus grandes et plus puissantes que toi, des villes grandes et fortifiées jusqu'au ciel,
\VS{2}un peuple grand et de haute taille, les fils d'Anak, que tu connais, et dont tu as entendu dire~: Qui tiendra face aux fils d'Anak~?
\VS{3}Sache donc aujourd'hui que Yahweh, ton Dieu, passera devant toi, comme un feu dévorant, c'est lui qui les détruira, qui les humiliera devant toi~; tu les chasseras, et tu les feras périr promptement, comme Yahweh te l'a dit.
\VS{4}Ne parle pas en ton cœur, quand Yahweh, ton Dieu, les chassera de devant toi, en disant~: C'est à cause de ma justice que Yahweh me fait entrer en possession de ce pays. Car c'est à cause de la méchanceté de ces nations-là que Yahweh les chasse devant toi.
\VS{5}Ce n'est point pour ta justice ni pour la droiture de ton cœur que tu entres en possession de leur pays, mais c'est pour la méchanceté de ces nations-là que Yahweh, ton Dieu, les chasse de devant toi, et pour confirmer la parole que Yahweh a jurée à tes pères, Abraham, Isaac, et Jacob.
\VS{6}Sache donc que ce n'est point pour ta justice que Yahweh, ton Dieu, te donne ce bon pays pour que tu le possèdes~; car tu es un peuple au cou raide.
\VS{7}Souviens-toi, n'oublie pas que tu as excité la colère de Yahweh, ton Dieu, dans le désert. Depuis le jour où tu es sorti du pays d'Egypte jusqu'à ce que vous arriviez dans ce lieu, vous avez été rebelles contre Yahweh.
\VS{8}Même à Horeb, vous avez excité la colère de Yahweh~; et Yahweh s'irrita contre vous, pour vous détruire.
\VS{9}Quand je montai sur la montagne, pour prendre les tables de pierre, les tables de l'alliance que Yahweh a traitée avec vous, je demeurai sur la montagne quarante jours et quarante nuits, sans manger de pain et sans boire d'eau~;
\VS{10}et Yahweh me donna les deux tables de pierre écrites du doigt de Dieu, et contenant toutes les paroles que Yahweh avait déclarées sur la montagne, du milieu du feu, le jour de l'assemblée.
\VS{11}Et il arriva qu'au bout de quarante jours et quarante nuits, Yahweh me donna les deux tables de pierre, qui sont les tables de l'alliance.
\VS{12}Puis Yahweh me dit~: Lève-toi, descends promptement d'ici~; car ton peuple, que tu as fait sortir d'Egypte, s'est corrompu. Ils se sont détournés promptement de la voie que je leur avais ordonnée, ils se sont fait une image en métal fondu.
\VS{13}Yahweh me parla, en disant~: Je vois que ce peuple est un peuple au cou raide.
\VS{14}Laisse-moi les détruire et effacer leur nom de dessous les cieux~; et je te ferai devenir une nation plus puissante et plus grande que celle-ci.
\VS{15}Je retournai et je descendis de la montagne~; or la montagne était toute en feu, et j'avais les deux tables de l'alliance dans mes deux mains.
\VS{16}Puis je regardai, et voici, vous aviez péché contre Yahweh, votre Dieu, vous vous étiez fait un veau en métal fondu, vous vous étiez détournés promptement de la voie que vous avait ordonnée Yahweh.
\VS{17}Alors je saisis les deux tables, je les jetai de mes deux mains, et je les brisai devant vos yeux.
\TextTitle{Moïse, intercède pour Israël devant Yahweh}
\VS{18}Puis je me prosternai devant Yahweh, comme auparavant, quarante jours et quarante nuits, sans manger de pain et sans boire d'eau, à cause de tout votre péché, que vous aviez commis en faisant ce qui est mal aux yeux de Yahweh, afin de l'irriter.
\VS{19}Car je craignais face à la colère et à la fureur dont Yahweh était enflammé contre vous, pour vous détruire. Et Yahweh m'exauça encore cette fois.
\VS{20}Yahweh était très irrité contre Aaron, voulant le faire périr, mais j'intercédai pour Aaron en ce temps-là.
\VS{21}Puis je pris le veau\FTNT{Le veau d'or (Ex. 32).} que vous aviez fait, votre péché, et je le brûlai au feu, je le brisai en le broyant, jusqu'à ce qu'il soit réduit en poudre, et je jetai cette poudre dans le torrent qui descend de la montagne.
\VS{22}Vous avez fort irrité la colère de Yahweh à Tabeéra, à Massa, et à Kibroth-Hattaava.
\VS{23}Et quand Yahweh vous envoya à Kadès-Barnéa, en disant~: Montez, et prenez possession du pays que je vous donne~! Vous fûtes rebelles à la parole de Yahweh, votre Dieu, vous n'eûtes point confiance, et vous n'obéîtes point à sa voix.
\VS{24}Vous avez été rebelles à Yahweh depuis le jour où je vous ai connu.
\VS{25}Je me prosternai donc devant Yahweh, je me prosternai quarante jours et quarante nuits, parce que Yahweh avait dit qu'il vous détruirait.
\VS{26}Et je priai Yahweh, et je dis~: Ô Seigneur, Yahweh, ne détruis point ton peuple, ton héritage que tu as racheté par ta grandeur, et que tu as fait sortir d'Egypte par ta main puissante.
\VS{27}Souviens-toi de tes serviteurs Abraham, Isaac, et Jacob. Ne regarde point à l'obstination de ce peuple, ni à sa méchanceté, ni à son péché,
\VS{28}de peur que le pays d'où tu nous as fait sortir ne dise~: Parce que Yahweh n'était pas capable de les conduire dans le pays qu'il leur avait promis, et parce qu'il les haïssait, il les a fait sortir pour les faire mourir dans le désert.
\VS{29}Cependant ils sont ton peuple et ton héritage, que tu as fait sortir par ta grande puissance et par ton bras étendu.
\Chap{10}
\TextTitle{Rappel du remplacement des tables de la loi}
\VerseOne{}En ce temps-là Yahweh me dit~: Taille deux tables de pierre comme les premières, et monte vers moi sur la montagne~; tu feras une arche de bois\FTNT{Ex. 25:10~; 34:1-4.}.
\VS{2}Et j'écrirai sur ces tables les paroles qui étaient sur les premières tables que tu as brisées, et tu les mettras dans l'arche.
\VS{3}Ainsi je fis une arche de bois d'acacia, je taillai deux tables de pierre comme les premières, et je montai sur la montagne, les deux tables dans ma main\FTNT{Ex. 34:4.}.
\VS{4}Et Yahweh écrivit sur ces tables ce qui avait été écrit sur les premières, les dix paroles qu'il avait dites sur la montagne, du milieu du feu, le jour de l'assemblée~; puis Yahweh me les donna.
\VS{5}Je me retournai et je descendis de la montagne~; je mis les tables dans l'arche que j'avais faite, et elles y sont demeurées, comme Yahweh me l'avait ordonné.
\VS{6}Or les enfants d'Israël partirent de Beéroth-Bené-Jaakan pour Moséra. Là mourut Aaron, et il fut enseveli~; Eléazar, son fils, exerça la prêtrise à sa place.
\VS{7}De là ils partirent pour Gudgoda, et de Gudgoda pour Jothbatha, qui est un pays de torrents d'eau.
\VS{8}Or en ce temps-là, Yahweh sépara la tribu de Lévi afin de porter l'arche de l'alliance de Yahweh, de se tenir devant Yahweh, de le servir, et de bénir en son Nom, jusqu'à ce jour.
\VS{9}C'est pourquoi Lévi n'a ni portion ni d'héritage avec ses frères~: Yahweh est son héritage, comme Yahweh, ton Dieu, lui a dit.
\VS{10}Je restai sur la montagne, comme la première fois, quarante jours et quarante nuits. Yahweh m'exauça encore cette fois~; Yahweh ne voulut point te détruire.
\VS{11}Mais Yahweh me dit~: Lève-toi, va, marche devant ce peuple. Qu'ils aillent prendre possession du pays que j'ai juré à leurs pères de leur donner.
\TextTitle{Une alliance basée sur l'amour de Yahweh}
\VS{12}Maintenant donc, ô Israël, que demande de toi Yahweh, ton Dieu, sinon que tu craignes Yahweh, ton Dieu, afin de marcher dans toutes ses voies, d'aimer et de servir Yahweh, ton Dieu, de tout ton cœur, et de toute ton âme~;
\VS{13}de garder les commandements de Yahweh et ses lois que je t'ordonne aujourd'hui, afin que tu sois heureux~?
\VS{14}Voici, les cieux, et les cieux des cieux appartiennent à Yahweh, ton Dieu, la terre et tout ce qu'elle renferme.
\VS{15}Et Yahweh s'est attaché à tes pères, pour les aimer~; et après eux, il vous a choisis, vous, leur postérité, entre tous les peuples, comme vous le voyez aujourd'hui.
\VS{16}Circoncisez donc le prépuce de votre cœur, et vous ne raidirez plus votre cou.
\VS{17}Car Yahweh, votre Dieu, est le Dieu des dieux, le Seigneur des seigneurs\FTNT{Yahweh, le Dieu des dieux et le Seigneur des seigneurs n'est autre que Jésus-Christ, notre Seigneur qui s'est révélé à Jean comme le Seigneur des seigneurs et le Roi des rois (Ap. 19:16).}, le Fort, le Grand, le Puissant et le Redoutable, qui n'a point d'égard à l'apparence des personnes, et qui ne prend point de présents~;
\VS{18}qui fait justice à l'orphelin et à la veuve, qui aime l'étranger et lui donne le pain et le vêtement.
\VS{19}Vous aimerez donc l'étranger~; car vous avez été étrangers dans le pays d'Egypte.
\VS{20}Tu craindras Yahweh, ton Dieu, tu le serviras, tu t'attacheras à lui, et tu jureras par son Nom.
\VS{21}Il est ta louange, il est ton Dieu, qui a fait pour toi des choses grandes et redoutables que tes yeux ont vues.
\VS{22}Tes pères descendirent en Egypte au nombre de soixante-dix âmes~; et maintenant Yahweh, ton Dieu, t'a fait devenir comme les étoiles des cieux, tant tu es en grand nombre.
\Chap{11}
\TextTitle{Exhortation à la reconnaissance et à l'obéissance}
\VerseOne{}Tu aimeras donc Yahweh, ton Dieu, et tu garderas toujours ses lois, ses ordonnances, et ses commandements.
\VS{2}Et reconnaissez aujourd'hui, ce que n'ont point connu ni vu vos fils, le châtiment de Yahweh, votre Dieu, sa grandeur, sa main puissante, et son bras étendu,
\VS{3}ses signes, et les œuvres qu'il a accomplies au milieu de l'Egypte contre Pharaon, roi d'Egypte, et contre tout son pays~;
\VS{4}et ce que Yahweh a fait à l'armée d'Egypte, à ses chevaux et à ses chars, quand il a fait déborder sur eux les eaux de la Mer Rouge, car Yahweh les a détruits jusqu'à ce jour\FTNT{Ex. 14:28.}~;
\VS{5}ce qu'il a fait dans le désert, jusqu'à votre arrivée en ce lieu-ci~;
\VS{6}ce qu'il a fait à Dathan et à Abiram, fils d'Eliab, fils de Ruben, comment la terre ouvrit sa bouche et les engloutit, avec leurs maisons et leurs tentes, et tous les êtres qui les suivaient, au milieu de tout Israël\FTNT{No. 16:1-33.}.
\VS{7}Car ce sont vos yeux qui ont vu toutes les grandes œuvres que Yahweh a faites.
\TextTitle{Les bienfaits de la terre promise sont pour un peuple fidèle}
\VS{8}Vous garderez donc tous les commandements que je vous ordonne aujourd'hui, afin que vous ayez la force d'entrer et de vous emparer du pays où vous allez passer pour en prendre possession,
\VS{9}et afin que vous prolongiez vos jours sur la terre que Yahweh a juré à vos pères de leur donner, ainsi qu'à leur postérité, pays où coulent le lait et le miel.
\VS{10}Car le pays où tu vas entrer afin de le posséder n'est pas comme le pays d'Egypte, d'où vous êtes sortis, où tu semais ta semence, et l'arrosais avec ton pied, comme un jardin potager.
\VS{11}Mais le pays où vous allez passer pour le posséder est un pays de montagnes et de vallées, qui boit les eaux de la pluie du ciel~;
\VS{12}c'est un pays dont Yahweh, ton Dieu, prend soin, et sur lequel Yahweh, ton Dieu, a continuellement ses yeux, du commencement de l'année jusqu'à la fin de l'année.
\VS{13}Il arrivera donc que, si vous obéissez attentivement à mes commandements que je vous ordonne aujourd'hui, si vous aimez Yahweh, votre Dieu, et que vous le servez de tout votre cœur et de toute votre âme,
\VS{14}alors je donnerai à votre pays la pluie en son temps, la pluie de la première et de l'arrière-saison, et tu recueilleras ton blé, ton vin, et ton huile.
\VS{15}Je mettrai aussi dans ton champ de l'herbe pour ton bétail, tu mangeras et tu seras rassasié.
\VS{16}Prenez garde à vous, de peur que votre cœur ne soit trompé, et que vous ne vous détourniez, et ne serviez d'autres dieux, et ne vous prosterniez devant eux.
\VS{17}Et que la colère de Yahweh s'enflamme contre vous, et qu'il ne ferme les cieux, tellement qu'il n'y aurait point de pluie, la terre ne donnerait plus son produit, et vous péririez promptement dans ce bon pays que Yahweh vous donne.
\VS{18}Mettez donc dans votre cœur et dans votre âme ces paroles. Liez-les comme un signe sur vos mains, et qu'elles soient comme des fronteaux entre vos yeux.
\VS{19}Et enseignez-les à vos enfants, en leur en parlant, quand tu seras dans ta maison, quand tu partiras en voyage, quand tu te coucheras et quand tu te lèveras.
\VS{20}Tu les écriras aussi sur les poteaux de ta maison, et sur tes portes.
\VS{21}Afin que vos jours et les jours de vos fils, sur la terre que Yahweh a juré à vos pères de leur donner, soient aussi nombreux que les jours des cieux sur la terre.
\VS{22}Car si vous gardez et si vous pratiquez tous ces commandements que je vous ordonne de faire, aimant Yahweh, votre Dieu, marchant dans toutes ses voies, et vous attachant à lui,
\VS{23}alors Yahweh chassera devant vous toutes ces nations et vous prendrez possession de nations plus grandes et plus puissantes que vous.
\VS{24}Tout lieu que foulera la plante de votre pied sera à vous\FTNT{Jos. 1:3~; 14:9.}~: Votre territoire s'étendra du désert au Liban, et du fleuve, le fleuve de l'Euphrate, jusqu'à la Mer Occidentale.
\VS{25}Aucun homme ne tiendra face à vous. Yahweh, votre Dieu, mettra, comme il vous l'a dit, la frayeur et la crainte de vous sur tout le pays où vous marcherez.
\TextTitle{La malédiction et la bénédiction}
\VS{26}Regardez, je mets aujourd'hui devant vous la bénédiction et la malédiction~:
\VS{27}La bénédiction, si vous obéissez aux commandements de Yahweh, votre Dieu, que je vous ordonne aujourd'hui~;
\VS{28}la malédiction, si vous n'obéissez point aux commandements de Yahweh, votre Dieu, et si vous vous détournez du chemin que je vous ordonne aujourd'hui, pour aller après d'autres dieux que vous ne connaissez point.
\VS{29}Et quand Yahweh, ton Dieu, t'aura fait entrer dans le pays dont tu vas prendre possession, tu prononceras alors les bénédictions, étant sur la montagne de Garizim, et les malédictions, étant sur la montagne d'Ebal.
\VS{30}Ces montagnes ne sont-elles pas de l'autre côté du Jourdain, derrière le chemin du soleil couchant, au pays des Cananéens qui demeurent dans la plaine, vis-à-vis de Guilgal, près des chênes de Moré~?
\VS{31}Car vous allez passer le Jourdain, pour entrer et prendre possession du pays que Yahweh, votre Dieu, vous donne~; vous le posséderez, et vous y habiterez.
\VS{32}Vous garderez et pratiquerez toutes les lois et les ordonnances que je mets aujourd'hui devant vous.
\Chap{12}
\TextTitle{Lois sur les sacrifices offerts au lieu où résidera le Nom de Yahweh}
\VerseOne{}Ce sont ici les lois et les ordonnances que vous garderez et pratiquerez dans le pays que Yahweh, le Dieu de vos pères, vous a donné à posséder, tout le temps que vous vivrez sur cette terre.
\VS{2}Vous détruirez, vous détruirez tous les lieux où les nations que vous allez déposséder servent leurs dieux, sur les hautes montagnes et sur les collines, et sous tout arbre verdoyant.
\VS{3}Vous démolirez aussi leurs autels, vous briserez leurs statues, vous brûlerez au feu leurs asheras, vous mettrez en pièces les images taillées de leurs dieux, et vous ferez périr leur nom de ce lieu-là.
\VS{4}Vous ne ferez pas ainsi à Yahweh, votre Dieu.
\VS{5}Mais vous le chercherez dans sa demeure, et vous irez au lieu que Yahweh, votre Dieu, aura choisi d'entre toutes vos tribus, pour y mettre son Nom.
\VS{6}Et vous y apporterez vos holocaustes, vos sacrifices, vos dîmes, vos offrandes élevées, vos vœux, vos offrandes volontaires de vos mains, et les premiers-nés de votre gros et de votre menu bétail\FTNT{Lé. 17:3-4.}.
\VS{7}Et là, vous mangerez devant Yahweh, votre Dieu, et vous vous réjouirez, vous et vos familles, de toutes les choses auxquelles vous aurez mis la main, et dans lesquelles Yahweh, votre Dieu, vous aura bénis.
\VS{8}Vous ne ferez pas comme nous faisons ici aujourd'hui, où chacun fait ce qui lui semble juste à ses yeux,
\VS{9}car vous n'êtes point encore entrés dans le lieu de repos, et dans l'héritage que Yahweh, votre Dieu, vous donne.
\VS{10}Vous passerez le Jourdain, et vous habiterez dans le pays que Yahweh, votre Dieu, vous donne en héritage~; il vous donnera du repos de tous vos ennemis qui vous entourent, et vous y habiterez en sécurité.
\VS{11}Et il y aura un lieu que Yahweh, votre Dieu, choisira pour y faire habiter son Nom. Vous y apporterez tout ce que je vous ordonne, vos holocaustes, vos sacrifices, vos dîmes, vos offrandes élevées de vos mains, et toutes offrandes de choix pour les vœux que vous aurez voués à Yahweh.
\VS{12}Et là, vous vous réjouirez devant Yahweh, votre Dieu, vous, vos fils et vos filles, vos serviteurs et vos servantes, et le Lévite qui sera dans vos portes~; car il n'a ni part ni héritage avec vous.
\VS{13}Garde-toi d'offrir tes holocaustes dans tous les lieux que tu verras~;
\VS{14}mais tu offriras tes holocaustes dans le lieu que Yahweh choisira dans l'une de tes tribus, et tu y feras tout ce que je t'ordonne.
\VS{15}Toutefois, selon le désir de ton âme, tu pourras tuer et manger de la viande dans toutes tes portes, selon la bénédiction que t'accordera Yahweh, ton Dieu~; celui qui sera impur et celui qui sera pur en mangeront, comme on mange de la gazelle et du cerf.
\VS{16}Seulement, vous ne mangerez point de sang. Tu le répandras sur la terre, comme de l'eau.
\VS{17}Tu ne pourras pas manger dans tes portes la dîme de ton blé, de ton vin, de ton huile, ni les premiers-nés de ton gros et menu bétail, ni aucune de tes offrandes en accomplissement d'un vœu, ni tes offrandes volontaires, ni les offrandes élevées de tes mains.
\VS{18}Mais tu les mangeras devant Yahweh, ton Dieu, au lieu que Yahweh, ton Dieu, choisira~; toi, ton fils, ta fille, ton serviteur et ta servante, et le Lévite qui sera dans tes portes~; et tu te réjouiras devant Yahweh, ton Dieu, de tout ce à quoi tu auras mis la main.
\VS{19}Garde-toi, tout le temps que tu vivras sur la terre, d'abandonner le Lévite.
\VS{20}Quand Yahweh, ton Dieu, aura élargi tes frontières, comme il te l'a promis, et que tu diras~: Je mangerai de la chair, parce que ton âme désirera manger de la chair, tu en mangeras selon tous les désirs de ton âme.
\VS{21}Si le lieu que Yahweh, ton Dieu, aura choisi pour y mettre son Nom, est loin de toi, alors tu tueras de ton gros et menu bétail, comme je te l'ai ordonné, et tu en mangeras dans tes portes selon tous les désirs de ton âme.
\VS{22}Tu en mangeras comme on mange de la gazelle et du cerf~; celui qui sera impur et celui qui sera pur en mangeront également.
\VS{23}Seulement, garde-toi de manger le sang, car le sang c'est l'âme~; et tu ne mangeras point l'âme avec la chair\FTNT{Lé. 7:26.}.
\VS{24}Tu n'en mangeras point~: Tu le répandras sur la terre comme de l'eau.
\VS{25}Tu n'en mangeras point, afin que tu sois heureux, toi et tes enfants après toi, parce que tu auras fait ce qui est droit aux yeux de Yahweh.
\VS{26}Mais tu prendras les choses que tu auras consacrées, qui seront à toi, et ce que tu auras voué, tu les prendras et tu viendras au lieu que Yahweh aura choisi.
\VS{27}Et tu offriras tes holocaustes, la chair et le sang, sur l'autel de Yahweh, ton Dieu~; mais le sang de tes autres sacrifices sera versé sur l'autel de Yahweh, ton Dieu, et tu en mangeras la chair.
\VS{28}Garde et écoute toutes ces paroles que je t'ordonne, afin que tu sois heureux, toi et tes enfants après toi, à jamais, en faisant ce qui est bon et droit aux yeux de Yahweh, ton Dieu.
\TextTitle{Mise en garde contre la séduction et les dieux étrangers}
\VS{29}Quand Yahweh, ton Dieu, aura exterminé de devant toi les nations que tu vas prendre en possession, que tu les auras possédées, et que tu habiteras dans leur pays,
\VS{30} prends garde à toi, de peur que tu ne sois pris au piège après elles, quand elles auront été détruites de devant toi~; et que tu ne recherches leurs dieux, en disant~: Comme ces nations-là servaient leurs dieux, je le ferai aussi tout de même.
\VS{31}Tu ne feras point ainsi à Yahweh, ton Dieu~; car elles ont fait à leurs dieux tout ce qui est en abomination et qui est odieux à Yahweh, et même ils brûlaient au feu leurs fils et leurs filles à leurs dieux.
\VS{32}Vous prendrez garde de faire tout ce que je vous commande. Vous n'y ajouterez rien, et vous n'en retrancherez rien.
\Chap{13}
\TextTitle{Eprouver les faux prophètes, ôter le méchant du milieu de l'assemblée}
\VerseOne{}S'il s'élève au milieu de toi un prophète ou un songeur de songes, qui te donne un signe ou miracle,
\VS{2}et que ce signe ou ce miracle dont il t'a parlé, arrive, et qu'il te dise~: Allons après d'autres dieux que tu ne connais point, et servons-les~!
\VS{3}Tu n'écouteras point les paroles de ce prophète ni de ce songeur de songes, car Yahweh, votre Dieu, vous met à l'épreuve pour savoir si vous aimez Yahweh, votre Dieu, de tout votre cœur et de toute votre âme.
\VS{4}Vous marcherez après Yahweh, votre Dieu, vous le craindrez~; vous garderez ses commandements, vous obéirez à sa voix, vous le servirez, et vous vous attacherez à lui.
\VS{5}Mais on fera mourir ce prophète-là ou ce songeur de songes, parce qu'il a parlé de révolte\FTNT{Le mot «~révolte~» utilisé ici, traduit le terme hébreu «~carah~» et signifie «~apostasie~». L'apostasie est une déviation progressive. C'est tout d'abord l'abandon d'une vérité reçue. Paul, l'apôtre enseigne que deux événements doivent avoir lieu avant le retour du Seigneur sur la terre~: L'apostasie et la révélation de l'homme du péché, le fils de perdition, c'est-à-dire l'Antichrist (2 Th. 2:1-3~; 2 Ti. 4:1).} contre Yahweh, votre Dieu, qui vous a fait sortir du pays d'Egypte et vous a délivrés de la maison de servitude, pour vous conduire loin de la voie que Yahweh, votre Dieu, vous a ordonné de marcher. Tu ôteras le méchant du milieu de toi.
\VS{6}Quand ton frère, fils de ta mère, ou ton fils, ou ta fille, ou ta femme bien-aimée, ou ton intime ami, qui est comme ton âme, t'incitera, en te disant en secret~: Allons, et servons d'autres dieux, que tu n'as point connus, ni tes pères,
\VS{7}d'entre les dieux des peuples qui sont autour de vous, près ou loin de toi, d'une extrémité de la terre jusqu'à l'autre,
\VS{8}tu ne t'accorderas pas avec lui, et tu ne l'écouteras point. Ton œil ne le regardera pas avec pitié, tu ne l'épargneras point, et tu ne le cacheras point.
\VS{9}Mais tu le feras mourir, tu le feras mourir\FTNT{Répétition du mot «~mourir~», voir commentaire en Ge. 2:17}~; ta main sera la première sur lui pour le mettre à mort, et ensuite la main de tout le peuple.
\VS{10}Tu le lapideras avec des pierres, et il mourra, parce qu'il a cherché à t'éloigner loin de Yahweh, ton Dieu, qui t'a fait sortir du pays d'Egypte, de la maison de servitude.
\VS{11}Afin que tout Israël entende et craigne, et que l'on ne fasse plus une action aussi méchante au milieu de toi.
\TextTitle{Jugement des villes idolâtres}
\VS{12}Si tu entends dire dans l'une des villes que Yahweh, ton Dieu, t'a données pour y habiter~:
\VS{13}Des hommes, fils de Bélial, sont sortis du milieu de toi, et ont chassé les habitants de leur ville, en disant~: Allons et servons d'autres dieux, des dieux que tu ne connais point~!
\VS{14}Tu chercheras, tu examineras, tu t'enquerras bien. Et si c'est la vérité, si la chose est établie, si cette abomination a été faite au milieu de toi,
\VS{15}tu frapperas, tu frapperas\FTNT{Répétition du mot «~frapperas~». Dans les écrits hébraïque, la répétition de mots est utilisée afin d'accentuer une action, pour appuyer un fait précis et le renforcer.} du tranchant de l'épée les habitants de cette ville, tu la dévoueras par interdit, et tu passeras le bétail au fil de l'épée.
\VS{16}Tu assembleras tout son butin au milieu de la place, et tu brûleras entièrement au feu cette ville et tout son butin, devant Yahweh, ton Dieu~: Elle sera pour toujours un monceau de ruines, sans être jamais rebâtie.
\VS{17}Rien de ce qui sera dévoué ne s'attachera à ta main, afin que Yahweh revienne de l'ardeur de sa colère, qu'il te fasse miséricorde et grâce, et qu'il te multiplie, comme il a juré à tes pères,
\VS{18}si tu obéis à la voix de Yahweh, ton Dieu, en gardant tous ses commandements que je t'ordonne aujourd'hui, et en faisant ce qui est droit aux yeux de Yahweh, ton Dieu.
\Chap{14}
\TextTitle{Israël, peuple mis en part}
\VerseOne{}Vous êtes les enfants de Yahweh, votre Dieu. Vous ne vous ferez aucune incision, et vous ne vous ferez point de place chauve entre les yeux pour aucun mort.
\VS{2}Car tu es un peuple saint pour Yahweh, ton Dieu~; et Yahweh t'a choisi pour que tu lui sois un peuple qui lui appartienne entre tous les peuples qui sont sur la face de la terre.
\TextTitle{Lois sur l'alimentation}
\VS{3}Tu ne mangeras d'aucune chose abominable.
\VS{4}Ce sont ici les bêtes que vous mangerez~: Le bœuf, la brebis et la chèvre~;
\VS{5}le cerf, la gazelle et le daim~; le bouquetin, le chevreuil, la chèvre sauvage, et le mouflon.
\VS{6}Vous mangerez donc toute bête qui a le sabot divisé, le pied fendu, et qui rumine.
\VS{7}Mais vous ne mangerez point de ceux qui ruminent seulement, ou qui ont le sabot divisé et le pied fendu seulement, comme le chameau, le lièvre et le lapin, car ils ruminent bien, mais ils n'ont pas le sabot qui est fendu~: Ils vous seront impurs.
\VS{8}Le porc aussi, car il a le sabot fendu, mais il ne rumine point~: Il vous sera impur. Vous ne mangerez point de leur chair, et vous ne toucherez point à leur cadavre.
\VS{9}Voici ce que vous mangerez de tout ce qui est dans les eaux~: Vous mangerez de tout ce qui a des nageoires et des écailles.
\VS{10}Mais vous ne mangerez point de ce qui n'a ni nageoires ni écailles~: Cela vous sera impur.
\VS{11}Vous mangerez tout oiseau pur.
\VS{12}Mais voici ceux dont vous ne mangerez point~: L'aigle, l'orfraie, l'aigle de mer~;
\VS{13}le vautour, le milan, et l'autour, selon leur espèce~;
\VS{14}le corbeau, selon son espèce~;
\VS{15}l'autruche, le hibou, la mouette, l'épervier, selon son espèce~;
\VS{16}le chat-huant, la chouette et le cygne~;
\VS{17}le cormoran, le pélican, le plongeon~;
\VS{18}la cigogne, le héron, selon leur espèce, la huppe et la chauve-souris.
\VS{19}Et tout reptile qui vole sera impur pour vous~; on n'en mangera point.
\VS{20}Mais vous mangerez de tout ce qui vole et qui est pur.
\VS{21}Vous ne mangerez aucun cadavre~; tu le donneras à l'étranger qui sera dans tes portes, et il le mangera, ou tu le vendras à un étranger~; car tu es un peuple saint pour Yahweh, ton Dieu. Tu ne feras point cuire le chevreau dans le lait de sa mère.
\TextTitle{Lois sur les dîmes\FTNT{No. 18:21-32.}}
\VS{22}Tu prendras la dîme, tu prendras la dîme\FTNT{Il est question ici de la dîme que les Hébreux consommaient chaque année.} de tout le produit de ta semence, de ce qui sortira de ton champ, chaque année.
\VS{23}Et tu mangeras devant Yahweh, ton Dieu, au lieu qu'il aura choisi pour y faire habiter son Nom, la dîme de ton blé, de ton vin et de ton huile, et les premiers-nés de ton gros et menu bétail, afin que tu apprennes à toujours craindre Yahweh, ton Dieu.
\VS{24}Mais quand le chemin sera trop long pour que tu puisses les transporter, parce que le lieu que Yahweh, ton Dieu, aura choisi pour y mettre son Nom, sera trop loin de toi, lorsque Yahweh, ton Dieu, t'aura béni,
\VS{25}alors tu l'échangeras contre de l'argent, tu serreras l'argent dans ta main, et tu iras au lieu que Yahweh, ton Dieu, aura choisi.
\VS{26}Et tu donneras l'argent contre tout ce que ton âme désirera, des bœufs, des brebis, du vin et des liqueurs fortes, tout ce que ton âme demandera, tu le mangeras devant Yahweh, ton Dieu, et tu te réjouiras, toi et ta famille.
\VS{27}Tu n'abandonneras point le Lévite qui sera dans tes portes, parce qu'il n'a ni portion ni héritage avec toi\FTNT{Ce verset fait référence à la première dîme qui devait être donnée aux Lévites (Voir commentaire en No. 18:21 et Mal. 3:1).}.
\VS{28}Au bout de trois ans, tu feras sortir toutes les dîmes de tes produits de cette année-là, et tu les déposeras dans tes portes.
\VS{29}Alors le Lévite, qui n'a ni portion ni héritage avec toi, l'étranger, l'orphelin, et la veuve qui seront dans tes portes, viendront, mangeront et se rassasieront, afin que Yahweh, ton Dieu, te bénisse dans toute l'œuvre que tu feras de tes mains.
\Chap{15}
\TextTitle{Lois sur l'année de relâche~: la justice et la bonté de Yahweh}
\VerseOne{}Tous les sept ans, tu célébreras l'année de relâche\FTNT{Ex. 21:2, Jé. 34:14.}.
\VS{2}Et c'est ici la manière de célébrer l'année de relâche. Que tout homme ayant droit d'exiger quelque chose que ce soit, qu'il puisse exiger de son prochain, donnera relâche, et ne l'exigera point de son prochain ni de son frère, quand on aura proclamé le relâche, en l'honneur de Yahweh.
\VS{3}Tu l'exigeras de l'étranger~; mais ta main relâchera tout ce qui t'appartiendra chez ton frère,
\VS{4}afin qu'il n'y ait point d'indigent chez toi, car Yahweh te bénira, te bénira\FTNT{Voir commentaire en Ge. 2:16} abondamment dans le pays que Yahweh, ton Dieu, te donnera à posséder pour héritage~;
\VS{5}pourvu que tu obéisses, que tu obéisses\FTNT{Voir commentaire en Ge. 2:16} bien à la voix de Yahweh, ton Dieu, en prenant garde de pratiquer tous ces commandements que je t'ordonne aujourd'hui.
\VS{6}Parce que Yahweh, ton Dieu, te bénira comme il te l'a promis, tu prêteras sur gage à beaucoup de nations, et tu n'emprunteras point sur gage~; tu domineras sur beaucoup de nations, et elles ne domineront point sur toi.
\VS{7}Quand un de tes frères sera indigent au milieu de toi, dans l'une de tes portes, dans le pays que Yahweh, ton Dieu, te donne, tu n'endurciras point ton cœur, et tu ne fermeras point ta main à ton frère indigent.
\VS{8}Mais tu lui ouvriras, tu lui ouvriras\FTNT{Voir commentaire en Ge. 2:16} ta main, et tu lui prêteras, lui prêteras\FTNT{Voir commentaire en Ge. 2:16} sur gage autant qu'il en aura besoin pour son indigence, dans laquelle il se trouvera.
\VS{9}Prends garde à toi, de peur que tu n'aies dans ton cœur quelque chose de Bélial, et que tu ne dises~: La septième année, l'année du relâche approche~! Et que ton œil soit méchant envers ton frère indigent, afin de ne rien lui donner et qu'il ne crie à Yahweh contre toi, et qu'il n'y ait du péché en toi.
\VS{10}Tu lui donneras, lui donneras\FTNT{Voir commentaire en Ge. 2:16} et que ton cœur ne lui donne point à regret~; car à cause de cela, Yahweh, ton Dieu, te bénira dans toutes tes œuvres, et dans tout ce à quoi tu mettras tes mains.
\VS{11}Car il y aura toujours des indigents dans le pays~; c'est pourquoi je t'ordonne, et je te dis~: Tu ouvriras, tu ouvriras\FTNT{Voir commentaire en Ge. 2:16} ta main à ton frère, à l'affligé, et à l'indigent dans ton pays.
\TextTitle{Loi sur les esclaves}
\VS{12}Quand l'un de tes frères Hébreux, homme ou femme, te sera vendu, il te servira six ans~; mais la septième année, tu le renverras libre de chez toi.
\VS{13}Et quand tu le renverras libre de chez toi, tu ne le renverras point à vide.
\VS{14}Tu chargeras, chargeras\FTNT{Voir commentaire en Ge. 2:16} de quelque chose de ton menu bétail, de ton aire, de ton pressoir, et tu lui donneras de ce que Yahweh, ton Dieu, t'aura béni.
\VS{15}Et tu te souviendras que tu as été esclave au pays d'Egypte, et que Yahweh, ton Dieu, t'en a racheté~; et c'est pour cela que je t'ordonne ceci aujourd'hui.
\VS{16}Mais s'il arrive qu'il te dise~: Je ne sortirai point de chez toi~; parce qu'il t'aime, toi et ta maison, et qu'il se trouve bien chez toi,
\VS{17}alors tu prendras un poinçon\FTNT{Ex. 21:6} et tu lui perceras l'oreille contre la porte, et il sera ton serviteur pour toujours. Tu en feras de même à ta servante.
\VS{18}Ce ne sera point, à tes yeux, dur de le renvoyer libre de chez toi, car il t'a servi six ans, ce qui est le double salaire d'un mercenaire~; et Yahweh, ton Dieu, te bénira en tout ce que tu feras.
\TextTitle{Loi sur les premiers-nés des animaux}
\VS{19}Tu consacreras à Yahweh, ton Dieu, tout premier-né mâle qui naîtra parmi ton gros et ton menu bétail. Tu ne travailleras point avec le premier-né de ton bœuf, et tu ne tondras point le premier-né de tes brebis\FTNT{Ex. 13:2.}.
\VS{20}Tu le mangeras, toi et ta famille, chaque année devant Yahweh, ton Dieu, dans le lieu que Yahweh aura choisi.
\VS{21}Mais s'il a quelque défaut, boiteux ou aveugle, ou qu'il ait quelque autre mauvais défaut, tu ne le sacrifieras point à Yahweh, ton Dieu.
\VS{22}Mais tu le mangeras dans tes portes~; celui qui sera impur et celui qui sera pur en mangeront également, comme on mange de la gazelle et du cerf.
\VS{23}Seulement, tu n'en mangeras point le sang~; mais tu le répandras sur la terre comme de l'eau.
\Chap{16}
\TextTitle{La Pâque et la fête des pains sans levain}
\VerseOne{}Observe le mois des épis, et fais la Pâque à Yahweh, ton Dieu~; car c'est au mois des épis que Yahweh, ton Dieu, t'a fait sortir, de nuit, d'Egypte\FTNT{Ex. 12:2-29.}.
\VS{2}Et tu sacrifieras la Pâque à Yahweh, ton Dieu, du gros et du menu bétail, au lieu que Yahweh choisira pour y faire habiter son Nom.
\VS{3}Tu ne mangeras point de pain levé, mais tu mangeras sept jours des pains sans levain, du pain d'affliction, parce que tu es sorti précipitamment du pays d'Egypte, afin que tous les jours de ta vie tu te souviennes du jour où tu es sorti du pays d'Egypte.
\VS{4}Et il ne se verra point de levain chez toi, sur tout le territoire de ton pays pendant sept jours\FTNT{1 Co. 5:7.}~; et aucune chair que tu sacrifieras le soir du premier jour ne restera jusqu'au matin.
\VS{5}Tu ne pourras point sacrifier la Pâque dans l'une de tes portes que Yahweh, ton Dieu, te donne~;
\VS{6}mais c'est au lieu que Yahweh, ton Dieu, choisira pour y faire habiter son Nom, que tu sacrifieras la Pâque, le soir, au coucher du soleil, moment où tu es sorti d'Egypte.
\VS{7}Tu la cuiras et tu la mangeras dans le lieu que Yahweh, ton Dieu, aura choisi. Et le matin, tu t'en retourneras et tu t'en iras dans tes tentes.
\VS{8}Pendant six jours, tu mangeras des pains sans levain~; et le septième jour, il y aura une assemblée solennelle à Yahweh, ton Dieu~: Tu ne feras aucune œuvre.
\TextTitle{La fête des semaines}
\VS{9}Tu te compteras sept semaines~; tu commenceras à compter ces sept semaines dès que la faucille sera mise dans les blés.
\VS{10}Puis tu feras la fête des semaines à Yahweh, ton Dieu, en présentant l'offrande volontaire de ta main, que tu donneras, selon que Yahweh, ton Dieu, t'aura béni.
\VS{11}Et tu te réjouiras devant Yahweh, ton Dieu, toi, ton fils et ta fille, ton serviteur et ta servante, le Lévite qui sera dans tes portes, l'étranger, l'orphelin et la veuve qui seront au milieu de toi, dans le lieu que Yahweh, ton Dieu, aura choisi pour y faire habiter son Nom.
\VS{12}Et tu te souviendras que tu as été esclave en Egypte, et tu garderas et pratiqueras ces lois.
\TextTitle{La fête des tabernacles}
\VS{13}Tu feras la fête des tabernacles pendant sept jours, après que tu auras recueilli le produit de ton aire et de ton pressoir.
\VS{14}Et tu te réjouiras à cette fête, toi, ton fils et ta fille, ton serviteur et ta servante, le Lévite, l'étranger, l'orphelin, et la veuve qui seront dans tes portes.
\VS{15}Tu célébreras la fête pendant sept jours à Yahweh, ton Dieu, dans le lieu que Yahweh aura choisi~; car Yahweh, ton Dieu, te bénira dans toute ta récolte, et dans tout le travail de tes mains, et tu vivras dans la joie.
\TextTitle{Offrandes à Yahweh selon ses moyens}
\VS{16}Trois fois l'an, tout mâle d'entre vous se présentera devant Yahweh, ton Dieu, dans le lieu qu'il aura choisi, à la fête des pains sans levain, à la fête des semaines, et à la fête des tabernacles. On ne se présentera point devant Yahweh à vide.
\VS{17}Mais chacun donnera à proportion de ce qu'il aura, selon la bénédiction de Yahweh,ton Dieu, qu'il t'aura donnée. 
\TextTitle{Des juges établis pour faire respecter la justice de Yahweh}
\VS{18}Tu t'établiras des juges et des officiers dans toutes les villes que Yahweh, ton Dieu, te donne, selon tes tribus~; et ils jugeront le peuple d'un juste jugement.
\VS{19}Tu ne te détourneras point de la justice, tu ne prêteras point attention à l'apparence des personnes, et tu ne recevras point de présents, car les présents aveuglent les yeux des sages et corrompent les paroles des justes.
\VS{20}Tu suivras fermement la justice, afin que tu vives et que tu possèdes le pays que Yahweh, ton Dieu, te donne.
\TextTitle{Prescriptions sur les cultes}
\VS{21}Tu ne planteras point d'arbre d'Asherah\FTNT{Bois ou arbre d'Asherah~: Il est question d'un objet en bois, pieu sacré ou arbre utilisé dans le culte d'Astarté, l'épouse de Baal (Ex. 34:13~; De. 7:5~; De. 12:3~; Jg. 3:7~; Jg. 6:25-30~; 1 R. 14:15-23).}, près de l'autel que tu feras à Yahweh, ton Dieu.
\VS{22}Tu ne dresseras point non plus de statue~; Yahweh, ton Dieu, hait ces choses.
\Chap{17}
\VerseOne{}Tu ne sacrifieras à Yahweh, ton Dieu, ni bœuf, ni agneau qui ait quelque défaut ou quelque chose de mauvais~; car c'est une abomination à Yahweh, ton Dieu.
\TextTitle{Punition de l'idolâtrie}
\VS{2}S'il se trouve au milieu de toi dans l'une des villes que Yahweh, ton Dieu, te donne, un homme ou une femme faisant ce qui est mal aux yeux de Yahweh, ton Dieu, en transgressant son alliance,
\VS{3}et allant servir d'autres dieux et se prosterner devant eux, devant le soleil, devant la lune, ou devant toute l'armée des cieux, ce que je n'ai pas ordonné~;
\VS{4}et que cela t'aura été rapporté, et que tu l'auras entendu, alors tu feras des recherches avec soin. Si la chose est vraie, que le fait est établi, et que cette abomination a été commise en Israël,
\VS{5}alors tu feras sortir vers tes portes cet homme ou cette femme, qui aura fait cette mauvaise action, cet homme, dis-je, ou cette femme, et tu les lapideras avec des pierres, et ils mourront.
\VS{6}On fera mourir sur la parole de deux témoins ou de trois témoins\FTNT{Mt. 18:15-17.}, celui qui doit être mis à mort~; il ne sera pas mis à mort sur la parole d'un seul témoin.
\VS{7}La main des témoins sera la première sur lui pour le faire mourir, et ensuite la main de tout le peuple. Et ainsi tu ôteras le mal du milieu de toi.
\TextTitle{Soumission aux autorités}
\VS{8}Quand une affaire te paraîtra trop difficile à juger entre meurtre et meurtre, entre cause et cause, entre plaie et plaie, qui sont des affaires de procès dans tes portes, alors tu te lèveras et tu monteras au lieu que Yahweh, ton Dieu, aura choisi.
\VS{9}Et tu iras vers les prêtres, les Lévites, et vers le juge qu'il y aura en ce temps-là, tu les consulteras, et ils te feront connaître et te déclareront la sentence du jugement.
\VS{10}Tu feras conformément à la sentence qu'ils t'auront déclarée de leur bouche dans le lieu que Yahweh aura choisi, et tu prendras garde de faire tout ce qu'ils t'enseigneront.
\VS{11}Tu feras conformément à la loi qu'ils t'auront enseignée de leur bouche et selon la sentence qu'ils t'auront prononcée~; tu ne te détourneras ni à droite ni à gauche de ce qu'ils t'auront déclaré.
\VS{12}Mais l'homme qui agira par orgueil et n'obéira pas au prêtre qui se tient là pour servir Yahweh, ton Dieu, ou au juge, cet homme mourra. Tu ôteras le mal d'Israël,
\VS{13}et tout le peuple l'entendra et craindra, et n'agira plus par orgueil.
\TextTitle{Instructions sur la royauté}
\VS{14}Quand tu seras entré dans le pays que Yahweh, ton Dieu, te donne, que tu le posséderas, que tu y demeureras, et que tu diras~: J'établirai un roi sur moi, comme toutes les nations qui sont autour de moi,
\VS{15}tu ne manqueras pas de t'établir pour roi celui que Yahweh, ton Dieu, aura choisi, tu établiras un roi du milieu de tes frères, tu ne pourras point désigner un homme étranger qui ne soit pas ton frère\FTNT{Dans sa prescience, Yahweh savait que le peuple se détournerait de ses voies et réclamerait un roi, à l'identique des nations alentour. (1 S. 8). Or depuis leur sortie d'Egypte, seul Yahweh était leur Dieu et leur Roi.}.
\VS{16}Seulement, il n'aura pas de nombreux chevaux, et il ne ramènera point le peuple en Egypte pour augmenter le nombre de chevaux~; car Yahweh vous a dit~: Vous ne retournerez plus par ce chemin.
\VS{17}Il n'aura point un grand nombre de femmes, afin que son cœur ne se détourne point~; et qu'il n'accumule point beaucoup d'argent et d'or.
\VS{18}Et dès qu'il sera assis sur le trône de son royaume, il écrira pour lui, dans un livre, une copie de cette loi, qu'il prendra des prêtres, les Lévites.
\VS{19}Il l'aura auprès de lui et la lira tous les jours de sa vie, afin qu'il apprenne à craindre Yahweh, son Dieu, à prendre garde à toutes les paroles de cette loi, et à ces ordonnances, afin de les pratiquer~;
\VS{20}afin que son cœur ne s'élève point au-dessus de ses frères, et qu'il ne se détourne point de ce commandement ni à droite ni à gauche~; afin qu'il prolonge ses jours dans son royaume, lui et ses fils, au milieu d'Israël.
\Chap{18}
\TextTitle{Héritage des Lévites et des prêtres}
\VerseOne{}Les prêtres, les Lévites, et même toute la tribu de Lévi, n'auront ni part ni héritage avec Israël~; ils mangeront les sacrifices consumés par le feu de Yahweh, et de son héritage.
\VS{2}Ils n'auront point d'héritage parmi leurs frères~: Yahweh sera leur héritage, comme il leur a dit.
\VS{3}Or c'est ici le droit que les prêtres prendront du peuple, sur ceux qui offriront un sacrifice, un bœuf ou un agneau~: On donnera au prêtre l'épaule, les mâchoires et l'estomac.
\VS{4}Tu lui donneras les prémices de ton blé, de ton vin et de ton huile, et les prémices de la toison de tes brebis.
\VS{5}Car Yahweh, ton Dieu, l'a choisi d'entre toutes les tribus, afin qu'il se tienne devant lui, et qu'il fasse le service au Nom de Yahweh, lui et ses fils, à toujours.
\VS{6}Or quand le Lévite viendra de l'une de tes portes, de tout lieu où il habite en Israël, et qu'il viendra selon tout le désir de son âme, au lieu que Yahweh aura choisi,
\VS{7}et qu'il fera le service au Nom de Yahweh, son Dieu, comme tous ses frères Lévites qui se tiennent là devant Yahweh,
\VS{8}il mangera une portion égale à la leur, outre ce qu'il aura vendu de son patrimoine.
\TextTitle{Les abominations des nations interdites en Israël}
\VS{9}Quand tu seras entré dans le pays que Yahweh, ton Dieu, te donne, tu n'apprendras point à faire les abominations de ces nations-là.
\VS{10}Qu'on ne trouve au milieu de toi personne qui fasse passer par le feu son fils ou sa fille, personne qui pratique la divination, l'astrologie, l'augure, la sorcellerie,
\VS{11}ni d'enchanteur qui use d'enchantements, personne qui consulte les médiums ou disent la bonne aventure, personne qui interroge les morts\FTNT{Yahweh interdit tout contact avec le monde des esprits et des démons. Le croyant qui accepte l'Evangile comprendra sans peine et simplement en obéissant à la Parole que ce domaine est interdit. Voir Ex. 22:18~; Lé. 19:26~; Lé. 19:31~; Lé. 20:6~; Lé. 20:27~; Es. 8:19~; 2 Ch. 33:6~; Ac. 19:13-20.}.
\VS{12}Car quiconque fait ces choses est en abomination à Yahweh~; et à cause de ces abominations, Yahweh, ton Dieu, va chasser ces nations-là devant toi.
\VS{13}Tu seras intègre avec Yahweh, ton Dieu.
\VS{14}Car ces nations, que tu vas déposséder, écoutent les pronostiqueurs et les devins~; mais à toi, Yahweh, ton Dieu, ne le permet point.
\TextTitle{Annonce sur la venue du Messie}
\VS{15}Yahweh, ton Dieu, te suscitera du milieu de toi, d'entre tes frères, un prophète comme moi\FTNT{Moïse a annoncé la venue d'un prophète comme lui, c'est-à-dire un prophète de la délivrance et de l'exode. Ce prophète n'est autre que Jésus-Christ qui nous délivre de l'emprise de Satan et nous sort du monde pour nous amener dans la Nouvelle Jérusalem (Jn. 14:2~; Col. 1:13). Notons qu'au moment de la transfiguration, Elie et Moïse parlaient avec Jésus de son départ («~exodus~» en grec~; Lu. 9:31).}: Vous l'écouterez.
\VS{16}Selon tout ce que tu as demandé à Yahweh, ton Dieu, à Horeb, le jour de l'assemblée, quand tu disais~: Que je n'entende plus la voix de Yahweh, mon Dieu, et que je ne voie plus ce grand feu, de peur de mourir.
\VS{17}Alors Yahweh me dit~: Ce qu'ils ont dit est bien.
\VS{18}Je leur susciterai un prophète comme toi du milieu de leurs frères, je mettrai mes paroles dans sa bouche, et il leur dira tout ce que je lui ordonnerai.
\VS{19}Et il arrivera que si un homme n'écoute pas mes paroles qu'il dira en mon Nom, je lui en demanderai compte.
\TextTitle{Comment éprouver les prophètes~?}
\VS{20}Mais le prophète qui agira de manière orgueilleuse pour dire en mon Nom une parole que je ne lui aurai point ordonnée de dire, ou qui parlera au nom des autres dieux, ce prophète-là mourra.
\VS{21}Et si tu dis dans ton cœur~: Comment connaîtrons-nous la parole que Yahweh n'aura point dite~?
\VS{22}Quand le prophète parlera au Nom de Yahweh, et que ce qu'il aura dit n'arrivera pas, ce sera une parole que Yahweh ne lui aura point dite. C'est par orgueil que le prophète l'a dite~: N'aie point peur de lui.
\Chap{19}
\TextTitle{Les villes de refuge\FTNTT{No. 35:1-34.}}
\VerseOne{}Quand Yahweh, ton Dieu, aura exterminé les nations dont Yahweh, ton Dieu, te donne le pays, et que tu les auras dépossédées et que tu demeureras dans leurs villes, et dans leurs maisons,
\VS{2}alors tu sépareras trois villes au milieu du pays que Yahweh, ton Dieu, te donne à posséder.
\VS{3}Tu établiras des chemins, et tu diviseras en trois le territoire de ton pays, que Yahweh, ton Dieu, te donnera en héritage. Ce sera afin que tout meurtrier s'y enfuie.
\VS{4}Or voici comment on procédera envers le meurtrier qui s'enfuira pour sauver sa vie. Celui qui aura frappé son prochain involontairement, et sans l'avoir haï dans le passé~;
\VS{5}ainsi, si quelqu'un va couper du bois dans la forêt avec une autre personne, la hache à la main pour couper du bois, si le fer glisse du manche, trouve son compagnon, et s'il en meurt~; il s'enfuira alors dans une de ces villes, afin qu'il vive.
\VS{6}De peur que celui qui venge le sang ne poursuive le meurtrier, parce que son cœur est échauffé, et qu'il ne le rattrape, si le chemin est trop long, et ne le frappe à mort, alors qu'il ne mérite pas la mort, parce qu'il ne le haïssait pas auparavant\FTNT{No. 35:1-34.}.
\VS{7}C'est pourquoi je t'ordonne, en disant~: Sépare-toi trois villes.
\VS{8}Lorsque Yahweh, ton Dieu, aura élargi tes frontières, comme il l'a juré à tes pères, et qu'il t'aura donné tout le pays qu'il a promis à tes pères de te donner,
\VS{9}parce que tu auras gardé et mis en pratique tous ces commandements que je t'ordonne aujourd'hui, en aimant Yahweh, ton Dieu, et en marchant toujours dans ses voies, alors tu ajouteras encore trois villes à ces trois-là,
\VS{10}afin que le sang innocent ne soit versé au milieu du pays que Yahweh, ton Dieu, te donne en héritage, et que tu ne sois pas coupable de meurtre.
\VS{11}Mais si un homme hait son prochain, lui dresse un piège, se lève contre lui et frappe cette personne, de sorte qu'il meure, et qu'il s'enfuit dans l'une de ces villes,
\VS{12}alors les anciens de sa ville l'enverront saisir, et le livreront entre les mains du vengeur de sang, afin qu'il meure.
\VS{13}Ton œil ne l'épargnera point, mais tu feras disparaître d'Israël le sang innocent, et tu seras heureux.
\VS{14}Tu ne déplaceras point les bornes de ton prochain, fixées par tes ancêtres, dans l'héritage que tu posséderas, dans le pays que Yahweh, ton Dieu, te donne à posséder.
\TextTitle{Résoudre des différends}
\VS{15}Un seul témoin ne sera point valable contre un homme pour constater un crime ou un péché, quel que soit le péché~; mais sur la parole de deux témoins ou de trois témoins la chose sera valable.
\VS{16}Quand un faux témoin s'élèvera contre un homme pour témoigner contre lui d'un crime,
\VS{17}ces deux hommes en contestation comparaîtront devant Yahweh, en présence des prêtres et des juges qui seront là en ce temps-là.
\VS{18}Et les juges feront des recherches avec soin. Si le témoin est un faux témoin, s'il a donné un faux témoignage contre son frère,
\VS{19}tu lui feras comme il avait pensé faire à son frère. Tu ôteras ainsi le mal du milieu de toi.
\VS{20}Et les autres entendront et craindront, et ne feront plus une chose aussi méchante au milieu de toi.
\VS{21}Ton œil ne l'épargnera point~: Vie pour vie, œil pour œil, dent pour dent, main pour main, pied pour pied.
\Chap{20}
\TextTitle{Instructions diverses pour la guerre}
\VerseOne{}Quand tu iras à la guerre contre tes ennemis, et que tu verras des chevaux et des chars, et un peuple plus grand que toi, tu ne les craindras point, car Yahweh, ton Dieu, qui t'a fait monter du pays d'Egypte, est avec toi.
\VS{2}Et quand vous vous approcherez du combat, le prêtre s'avancera et parlera au peuple.
\VS{3}Et leur dira~: Ecoute Israël~: Vous vous approchez aujourd'hui pour combattre vos ennemis. Que votre cœur ne faiblisse pas~; ne craignez point, ne soyez point effrayés et ne soyez point terrifiés face à eux.
\VS{4}Car Yahweh, votre Dieu, marche avec vous, pour combattre vos ennemis, pour vous sauver.
\VS{5}Les officiers parleront au peuple, en disant~: Qui est l'homme qui a bâti une maison neuve et ne l'a pas inaugurée~? Qu'il s'en aille et retourne dans sa maison, de peur qu'il ne meure dans la bataille et qu'un autre homme ne l'inaugure.
\VS{6}Qui est celui qui a planté une vigne et n'en a point encore cueilli le fruit~? Qu'il s'en aille et retourne dans sa maison, de peur qu'il ne meure dans la bataille et qu'un autre homme n'en cueille le fruit.
\VS{7}Qui est celui qui a fiancé une femme et ne l'a point prise en mariage~? Qu'il s'en aille et retourne dans sa maison, de peur qu'il ne meure dans la bataille et qu'un autre homme ne la prenne en mariage.
\VS{8}Et les officiers continueront à parler au peuple, et diront~: Si un homme a peur et est timide, qu'il s'en aille et retourne dans sa maison, de peur que le cœur de ses frères ne devienne craintif comme le sien.
\VS{9}Quand les officiers auront fini de parler au peuple, ils désigneront les chefs des armées à la tête du peuple.
\VS{10}Quand tu t'approcheras d'une ville pour lui faire la guerre, tu l'inviteras à la paix.
\VS{11}Et si elle te donne une réponse de paix et s'ouvre à toi, tout le peuple qui s'y trouvera te sera tributaire et te servira.
\VS{12}Si elle ne fait pas la paix avec toi et qu'elle te fait la guerre, alors tu l'assiègeras.
\VS{13}Et quand Yahweh, ton Dieu, l'aura livrée entre tes mains, tu frapperas tous les mâles au fil de l'épée.
\VS{14}Mais les femmes, les enfants, le bétail, tout ce qui sera dans la ville, et tout son butin, tu le prendras pour toi et tu mangeras le butin de tes ennemis, que Yahweh, ton Dieu, t'aura donné.
\VS{15}Tu feras ainsi à toutes les villes qui sont très éloignées de toi, et qui ne sont point des villes de ces nations.
\VS{16}Mais dans les villes de ces peuples que Yahweh, ton Dieu, te donne en héritage, tu ne laisseras vivre personne qui respire.
\VS{17}Car tu ne manqueras point de les dévouer par interdit~: Héthiens, Amoréens, Cananéens, Phéréziens, Héviens, et Jébusiens, comme Yahweh, ton Dieu, te l'a ordonné.
\VS{18}Afin qu'ils ne vous enseignent point à faire toutes les abominations qu'ils font pour leurs dieux, et que vous ne péchiez point contre Yahweh, votre Dieu.
\VS{19}Quand tu assiégeras une ville durant plusieurs jours, en lui faisant la guerre pour la saisir, tu ne détruiras point les arbres à coups de hache, tu t'en nourriras et tu ne les couperas point, car l'arbre des champs est-il un homme pour être assiégé par toi~?
\VS{20}Mais seulement tu détruiras et tu couperas les arbres que tu sauras ne point être des arbres fruitiers, et tu construiras des retranchements contre la ville qui te fait la guerre, jusqu'à ce qu'elle tombe.
\Chap{21}
\TextTitle{Lois sur le meurtre anonyme}
\VerseOne{}S'il se trouve sur la terre que Yahweh, ton Dieu, te donne à posséder, un homme tué, étendu dans un champ, sans que l'on sache qui l'a frappé,
\VS{2}tes anciens et tes juges sortiront, et ils mesureront de l'homme tué jusqu'aux villes qui sont autour.
\VS{3}Puis les anciens de la ville la plus proche de l'homme tué prendront une génisse du troupeau qui n'a pas travaillé et qui n'a point tiré au joug.
\VS{4}Et les anciens de cette ville feront descendre cette génisse vers un torrent intarissable, où on ne travaille ni ne sème~; et là, ils briseront la nuque à la génisse dans le torrent.
\VS{5}Et les prêtres, fils de Lévi, s'approcheront~; car Yahweh, ton Dieu, les a choisis pour qu'ils le servent, et qu'ils bénissent au Nom de Yahweh~; et leur bouche doit décider de toute contestation et toute blessure.
\VS{6}Et tous les anciens de cette ville, qui seront les plus proches de l'homme qui aura été tué, laveront leurs mains sur la génisse à laquelle on aura brisé la nuque dans le torrent.
\VS{7}Et prenant la parole, ils diront~: Nos mains n'ont point répandu ce sang et nos yeux ne l'ont point vu.
\VS{8}Ô Yahweh~! Sois propice à ton peuple d'Israël que tu as racheté~; ne lui impute point le sang innocent qui a été répandu au milieu de ton peuple d'Israël~; et le meurtre sera expié pour eux.
\VS{9}Et tu ôteras le sang innocent du milieu de toi, en faisant ce qui est droit aux yeux de Yahweh.
\TextTitle{Lois sur le mariage et l'héritage}
\VS{10}Quand tu iras en guerre contre tes ennemis, que Yahweh, ton Dieu, les aura livrés entre tes mains, et que tu en auras emmené des captifs,
\VS{11}si tu vois parmi les captifs une femme belle de figure, et que tu désires la prendre pour femme,
\VS{12}alors tu la conduiras à l'intérieur de ta maison, et elle rasera sa tête et fera ses ongles,
\VS{13}elle ôtera les vêtements de sa captivité, elle demeurera dans ta maison, et pleurera son père et sa mère durant un mois. Puis tu iras vers elle, tu l'épouseras, et elle sera ta femme.
\VS{14}Si il arrive qu'elle ne te plaise plus, tu la renverras où elle voudra, mais tu ne la vendras certainement pas pour de l'argent ni la traiteras en esclave, parce que tu l'auras humiliée.
\VS{15}Quand un homme, qui a deux femmes, aime l'une et hait l'autre, si celle qu'il aime et celle qu'il hait enfantent des fils, et que le fils aîné est de celle qui est haïe,
\VS{16}alors, le jour où il laissera en héritage ce qu'il aura, il ne pourra pas reconnaître comme premier-né le fils de celle qu'il aime, à la place du fils de celle qui est haïe, et qui est le premier-né.
\VS{17}Mais il reconnaîtra pour premier-né le fils de celle qui est haïe, et il lui donnera la double portion de tout ce qui s'y trouvera être à lui~; car il est le commencement de sa vigueur, le droit d'aînesse lui appartient.
\TextTitle{Le fils indocile sous la loi\FTNTT{cp. Lu. 15:11-23.}}
\VS{18}Si un homme a un fils indocile et rebelle, n'obéissant point à la voix de son père, ni à la voix de sa mère, et qui, bien qu'ils l'aient châtié, ne les écoute point,
\VS{19}alors le père et la mère le prendront et le mèneront aux anciens de sa ville, et à la porte du lieu de sa demeure.
\VS{20}Et ils diront aux anciens de sa ville~: Voici notre fils qui est indocile et rebelle, qui n'obéit point à notre voix, et qui se livre à l'excès et à l'ivrognerie.
\VS{21}Et tous les gens de la ville le lapideront avec des pierres, et il mourra. Tu ôteras le mal du milieu de toi, afin que tout Israël entende et craigne.
\VS{22}Si un homme a commis un péché digne de mort, et qu'on le fait mourir, et que tu l'aies pendu à un bois,
\VS{23}son cadavre ne passera point la nuit sur le bois~; mais tu ne manqueras point de l'ensevelir le même jour, car celui qui est pendu est malédiction de Dieu\FTNT{Ga. 3:13.}, et tu ne souilleras point la terre que Yahweh, ton Dieu, te donne en héritage.
\Chap{22}
\TextTitle{Lois sur la vie en société}
\VerseOne{}Si tu vois le bœuf ou la brebis de ton frère s'égarer, tu ne t'en cacheras point, tu ne manqueras point de les ramener à ton frère.
\VS{2}Si ton frère ne demeure point près de toi, et que tu ne le connais point, tu les recueilleras dans ta maison et il sera chez toi jusqu'à ce que ton frère les cherche~; et alors tu les lui rendras.
\VS{3}Tu feras de même pour son âne, tu feras de même pour son vêtement, et tu feras de même pour tout ce que ton frère aura perdu et que tu trouveras~; tu ne devras point t'en détourner.
\VS{4}Si tu vois l'âne de ton frère ou son bœuf tombé dans le chemin, tu ne t'en détourneras point, et tu ne manqueras point de le relever.
\VS{5}La femme ne portera point l'habit d'un homme ni l'homme ne se vêtira point d'un habit de femme~; car celui qui fait ces choses est en abomination à Yahweh, ton Dieu\FTNT{Dans ce passage, Yahweh condamne le travestisme. Cette pratique était répandue chez les Cananéens. Le travestisme consiste à adopter le comportement, les habitudes sociales et la tenue vestimentaire du sexe opposé dans le but de lui ressembler.}.
\VS{6}Si tu rencontres sur le chemin, sur un arbre ou sur la terre, un nid d'oiseaux, ayant des petits ou des œufs, et la mère couchée sur les petits ou les œufs, tu ne prendras point la mère et les petits,
\VS{7}mais tu ne manqueras point de laisser aller la mère et tu ne prendras que les petits, afin que tu sois heureux et que tu prolonges tes jours.
\VS{8}Si tu bâtis une maison neuve, tu feras un parapet tout autour de ton toit, afin que tu ne mettes point de sang sur ta maison, si quelqu'un tombait de là.
\TextTitle{Lois sur les mélanges}
\VS{9}Tu ne sèmeras point dans ta vigne diverses sortes de grains~; de peur que le tout, à savoir les grains, que tu auras semés, et le rapport de ta vigne, ne soit souillé. 
\VS{10}Tu ne laboureras point avec un âne et un bœuf ensemble.
\VS{11}Tu ne te vêtiras point d'un tissu mélangé de laine et de lin ensemble.
\VS{12}Tu te feras des franges aux quatre pans du vêtement dont tu te couvriras.
\TextTitle{Lois sur la virginité, l'adultère et la fidélité}
\VS{13}Si un homme a pris une femme et est allé vers elle, et qu'il la haïsse,
\VS{14}et qu'il lui impute des choses qui donnent l'occasion de parler d'elle et de la diffamer, en disant~: J'ai pris cette femme, et quand je me suis approché d'elle, je ne l'ai point trouvé vierge,
\VS{15}alors le père et la mère de la jeune femme prendront et produiront les signes de la virginité de la jeune femme devant les anciens de la ville, à la porte.
\VS{16}Et le père de la jeune femme dira aux anciens~: J'ai donné ma fille à cet homme pour femme, et il l'a haïe~;
\VS{17}et voici, il lui impute des choses qui lui donnent l'occasion de parler d'elle, disant~: Je n'ai point trouvé ta fille vierge. Cependant, voici les signes de la virginité de ma fille. Et ils étendront le drap devant les anciens de la ville.
\VS{18}Alors les anciens de la ville prendront le mari, et le châtieront~;
\VS{19}et parce qu'il aura répandu une mauvaise réputation sur une vierge d'Israël, ils le condamneront à une amende de cent sicles d'argent, qu'ils donneront au père de la jeune femme. Elle sera sa femme, et il ne pourra pas la répudier, tant qu'il vivra.
\VS{20}Mais si la chose est vraie, si la jeune femme ne s'est point trouvée vierge,
\VS{21}alors ils feront sortir la jeune femme à l'entrée de la maison de son père~; les gens de sa ville la lapideront de pierres et elle mourra, car elle a commis une infamie en Israël, en se prostituant dans la maison de son père. Tu ôteras le mal du milieu de toi.
\VS{22}Si l'on trouve un homme couché avec une femme mariée, ils mourront tous les deux, l'homme qui a couché avec la femme, et la femme aussi. Tu ôteras ainsi le mal d'Israël.
\VS{23}Si une jeune fille vierge est fiancée à un homme, et qu'un homme la rencontre dans la ville, et couche avec elle,
\VS{24}vous les conduirez tous deux à la porte de la ville, vous les lapiderez de pierres, et ils mourront~; la jeune fille, parce qu'elle n'a point crié étant dans la ville, et l'homme parce qu'il a humilié la femme de son prochain. Tu ôteras le mal du milieu de toi.
\VS{25}Si l'homme rencontre dans les champs la jeune fille fiancée, et que l'homme lui fait violence et couche avec elle, alors l'homme qui aura couché avec elle mourra lui seul.
\VS{26}Mais tu ne feras rien à la jeune fille~; la jeune fille n'a point commis de péché digne de mort, car c'est comme si un homme s'élevait contre son prochain et lui ôtait la vie.
\VS{27}Parce que l'ayant trouvée dans les champs, la jeune fille fiancée a pu crier, sans que personne ne l'ait délivrée.
\VS{28}Si un homme rencontre une jeune fille vierge non fiancée, lui fait violence et couche avec elle, et qu'ils soient découverts,
\VS{29}l'homme qui aura couché avec elle donnera au père de la jeune fille cinquante sicles d'argent~; et il la prendra pour femme, parce qu'il l'a humiliée, et il ne pourra point la répudier, tant qu'il vivra.
\VS{30}Un homme ne prendra point la femme de son père ni ne découvrira le pan de la robe de son père.
\Chap{23}
\TextTitle{Lois sur l'accès à l'assemblée de Yahweh}
\VerseOne{}Celui dont les testicules ont été écrasés ou l'urètre coupé n'entrera point dans l'assemblée de Yahweh.
\VS{2}Le bâtard\FTNT{Le mot bâtard, «~mamzer~» en hébreu, désigne l'enfant illégitime, celui issu de l'inceste, celui né d'une population mélangée ou d'un père Juif et d'une mère païenne, et inversement.} n'entrera point dans l'assemblée de Yahweh~; même sa dixième génération n'entrera point dans l'assemblée de Yahweh.
\VS{3}L'Ammonite et le Moabite n'entreront point dans l'assemblée de Yahweh, même leur dixième génération, à jamais,
\VS{4}parce qu'ils ne sont point venus à votre rencontre avec du pain et de l'eau, sur le chemin, lorsque vous sortiez d'Egypte, et parce qu'ils ont engagé à prix d'argent contre vous Balaam, fils de Beor, de Pethor en Mésopotamie, pour qu'il vous maudisse.
\VS{5}Mais Yahweh, ton Dieu, n'a point voulu écouter Balaam~; et Yahweh, ton Dieu, a changé la malédiction en bénédiction, parce que Yahweh, ton Dieu, t'aime.
\VS{6}Tu ne chercheras jamais, tant que tu vivras, leur paix ni leur bien.
\VS{7}Tu n'auras point en abomination l'Edomite, car il est ton frère~; tu n'auras point en abomination l'Egyptien, car tu as été étranger dans son pays~:
\VS{8}Les enfants qui leur naîtront à la troisième génération entreront dans l'assemblée de Yahweh.
\TextTitle{La sainteté et la justice dans le camp de Yahweh}
\VS{9}Quand le camp sortira contre tes ennemis, garde-toi de toute chose mauvaise.
\VS{10}S'il y a parmi vous un homme qui ne soit point pur, par suite d'un accident nocturne, il sortira hors du camp, et n'entrera point dans le camp.
\VS{11}Et sur le soir, il se lavera dans l'eau, et dès que le soleil sera couché, il rentrera dans le camp.
\VS{12}Tu auras un endroit hors du camp, et tu sortiras là dehors.
\VS{13}Tu auras un pieu parmi tes bagages, et quand tu voudras aller dehors, tu creuseras, puis tu recouvriras tes excréments.
\VS{14}Car Yahweh, ton Dieu, marche au milieu de ton camp pour te délivrer et pour livrer tes ennemis devant toi~; que tout ton camp soit saint, afin qu'il ne voie chez toi aucune chose honteuse, et qu'il ne se détourne point de toi.
\VS{15}Tu ne livreras point à son maître l'esclave qui se sera sauvé chez toi d'auprès de son maître.
\VS{16}Il demeurera avec toi, au milieu de toi, dans le lieu qu'il choisira, dans l'une de tes villes, là où bon lui semblera~: Tu ne l'opprimeras point.
\VS{17}Il n'y aura, parmi les filles d'Israël, aucune prostituée, et il n'y aura, parmi les fils d'Israël, aucun qui se prostitue.
\VS{18}Tu n'apporteras point dans la maison de Yahweh, ton Dieu, le salaire d'une prostituée, ni le prix d'un chien, pour quelque vœu que ce soit~; car tous les deux sont en abomination devant Yahweh, ton Dieu.
\VS{19}Tu n'exigeras aucun intérêt à ton frère, ni intérêt pour de l'argent, ni intérêt pour des vivres, ni intérêt pour quelque chose que ce soit que l'on prête avec intérêt.
\VS{20}Tu prêteras avec intérêt à l'étranger, mais tu ne prêteras point avec intérêt à ton frère, afin que Yahweh, ton Dieu, te bénisse dans tout ce que ta main entreprendra dans le pays où tu vas entrer en possession.
\TextTitle{Vœux faits à Yahweh}
\VS{21}Si tu fais un vœu à Yahweh, ton Dieu, tu ne tarderas point à l'accomplir, car Yahweh, ton Dieu, ne manquerait point de te le redemander, ainsi il y aurait du péché en toi.
\VS{22}Mais si tu t'abstiens de faire un vœu, il n'y aura pas de péché en toi.
\VS{23}Mais tu prendras garde de faire ce qui sortira de tes lèvres, l'offrande volontaire que tu auras vouée à Yahweh, ton Dieu, et que ta bouche aura prononcée.
\TextTitle{Lois diverses}
\VS{24}Si tu entres dans la vigne de ton prochain, tu pourras manger des raisins selon ton appétit, jusqu'à en être rassasié~; mais tu n'en mettras point dans ton vase.
\VS{25}Si tu entres dans les blés de ton prochain, tu pourras arracher des épis avec ta main~; mais tu n'agiteras point la faucille sur les blés de ton prochain.
\Chap{24}
\TextTitle{Loi sur le divorce}
\VerseOne{}Quand un homme aura pris et épousé une femme, s'il arrive qu'elle ne trouve pas grâce à ses yeux, parce qu'il aura trouvé en elle quelque chose de honteux, il lui écrira une lettre de divorce, et après la lui avoir remise en main, il la renverra de sa maison.
\VS{2}Elle sortira de sa maison, s'en ira, et elle pourra devenir la femme d'un autre homme.
\VS{3}Si ce dernier homme la hait, écrit une lettre de divorce, la lui donne dans sa main, et la renvoie de sa maison, ou que ce dernier homme qui l'a prise pour femme, meure,
\VS{4}alors son premier mari qui l'avait renvoyée ne pourra pas la reprendre pour femme après avoir été souillée, car c'est une abomination devant Yahweh, ainsi tu ne feras point pécher le pays que Yahweh, ton Dieu, te donne en héritage.
\TextTitle{Lois diverses sur l'organisation de la société}
\VS{5}Quand un homme aura nouvellement épousé une femme, il n'ira point à la guerre, et on ne lui imposera aucune charge~; il en sera libre pour sa maison pendant un an, et il réjouira la femme qu'il a prise.
\VS{6}On ne prendra point pour gage les deux meules, pas même la meule de dessus~; parce qu'on prendrait pour gage la vie.
\VS{7}Si l'on trouve un homme qui ait dérobé l'un de ses frères, l'un des enfants d'Israël, qui en ait fait son esclave ou qui l'ait vendu, ce voleur mourra. Tu ôteras le mal du milieu de toi.
\VS{8}Prends garde à la plaie de la lèpre, afin de bien observer et de faire tout ce que les prêtres, les Lévites, vous enseigneront~; vous prendrez garde de faire selon ce que je leur ai ordonné.
\VS{9}Souviens-toi de ce que Yahweh, ton Dieu, fit à Marie, en chemin, après votre sortie d'Egypte.
\TextTitle{Lois en faveur des nécessiteux}
\VS{10}Lorsque tu feras à ton prochain un prêt quelconque, tu n'entreras point dans sa maison pour prendre son gage~;
\VS{11}mais tu te tiendras dehors, et l'homme à qui tu feras le prêt t'apportera le gage dehors.
\VS{12}Si cet homme est pauvre, tu ne te coucheras point ayant encore son gage~;
\VS{13}tu ne manqueras point de lui rendre le gage dès que le soleil sera couché, afin qu'il se couche dans son vêtement et qu'il te bénisse~; et cela te sera imputé à justice devant Yahweh, ton Dieu.
\VS{14}Tu n'opprimeras point le mercenaire, le pauvre et l'indigent, d'entre tes frères, ou d'entre les étrangers qui demeurent dans ton pays, dans tes portes.
\VS{15}Tu lui donneras son salaire le jour même avant que le soleil se couche~; car il est pauvre, et son désir s'y porte. Afin qu'il ne crie point contre toi à Yahweh, et que tu ne pèches point.
\VS{16}On ne fera point mourir les pères pour les fils, et on ne fera point mourir les fils pour les pères~; mais on fera mourir chacun pour son péché.
\VS{17}Tu ne feras pas d'injustice à l'étranger ni à l'orphelin, et tu ne prendras point en gage le vêtement de la veuve.
\VS{18}Et tu te souviendras que tu as été esclave en Egypte, et que Yahweh, ton Dieu, t'a racheté de là~; c'est pourquoi je t'ordonne de faire ces choses.
\VS{19}Quand tu moissonneras dans ton champ, et que tu auras oublié une gerbe dans ton champ, tu ne retourneras point la prendre~: Elle sera pour l'étranger, pour l'orphelin et pour la veuve, afin que Yahweh, ton Dieu, te bénisse dans toute l'œuvre de tes mains.
\VS{20}Quand tu secoueras tes oliviers, tu n'y retourneras point pour cueillir ce qui reste aux branches~: Ce sera pour l'étranger, pour l'orphelin et pour la veuve.
\VS{21}Quand tu vendangeras ta vigne, tu ne grappilleras point après~: Ce sera pour l'étranger, pour l'orphelin et pour la veuve.
\VS{22}Et tu te souviendras que tu as été esclave dans le pays d'Egypte~; c'est pourquoi je t'ordonne de faire ces choses.
\Chap{25}
\TextTitle{Le juste justifié et le méchant condamné}
\VerseOne{}Quand il y aura un différend entre des hommes et qu'ils viendront en jugement afin qu'on les juge, on justifiera le juste, et on condamnera le méchant.
\VS{2}Si le méchant mérite d'être battu, le juge le fera jeter par terre et frapper en sa présence par un certain nombre de coups, selon l'exigence de son crime.
\VS{3}Il le fera battre de quarante coups, pas plus, de peur que si l'on continuait à le frapper avec plus de coups, ton frère ne soit méprisé à tes yeux.
\VS{4}Tu n'emmuselleras point ton bœuf lorsqu'il foulera le grain.
\TextTitle{Loi sur la continuité de la postérité}
\VS{5}Quand des frères demeureront ensemble, et que l'un d'entre eux mourra sans fils, alors la femme du défunt ne se mariera point dehors avec un homme qui est étranger, mais son beau-frère viendra vers elle, la prendra pour femme, et l'épousera comme son beau-frère.
\VS{6}Et le premier-né qu'elle enfantera succédera au frère mort et portera son nom, afin que son nom ne soit point effacé d'Israël.
\VS{7}Et s'il ne plaît pas à cet homme-là de prendre sa belle-sœur, alors sa belle-sœur montera à la porte vers les anciens\FTNT{Ru. 4:1-10}, et dira~: Mon beau-frère refuse de relever le nom de son frère en Israël, et ne veut point m'épouser par droit de beau-frère.
\VS{8}Alors les anciens de la ville l'appelleront et lui parleront. S'il demeure ferme, et qu'il dit~: Il ne me plaît point de la prendre,
\VS{9}alors sa belle-sœur s'approchera de lui à la vue des anciens, lui ôtera son soulier du pied, et lui crachera au visage. Et prenant la parole, elle dira~: C'est ainsi qu'on fera à l'homme qui ne bâtit point la maison de son frère.
\VS{10}Et son nom sera appelé en Israël la maison du déchaussé.
\TextTitle{L'abomination sévèrement et justement punie}
\VS{11}Quand des hommes se querelleront ensemble, l'un contre l'autre, si la femme de l'un s'approche pour délivrer son mari de la main de celui qui le frappe, et qu'étendant sa main elle saisisse ses parties intimes,
\VS{12}tu lui couperas la main, et ton œil ne l'épargnera point.
\VS{13}Tu n'auras point dans ton sac deux poids différents, un grand et un petit.
\VS{14}Il n'y aura point dans ta maison deux épha différents, un grand et un petit\FTNT{Lé. 19:35-37.}.
\VS{15}Mais tu auras un poids exact et juste, tu auras un épha exact et juste, afin que tes jours se prolongent sur la terre que Yahweh, ton Dieu, te donne.
\VS{16}Car celui qui fait ces choses, celui qui commet une injustice, est en abomination à Yahweh, ton Dieu.
\TextTitle{Yahweh confirme le sort d'Amalek}
\VS{17}Souviens-toi ce que te fit Amalek en chemin, quand vous sortiez d'Egypte\FTNT{Ex. 17:8.},
\VS{18}comment il est venu te rencontrer sur le chemin, et, sans aucune crainte de Dieu, attaqua par derrière ceux qui étaient fatigués, quand toi-même tu étais épuisé.
\VS{19}Quand Yahweh, ton Dieu, t'aura accordé du repos de tous tes ennemis qui t'entourent, dans le pays que Yahweh, ton Dieu, te donne en héritage afin que tu le possèdes, alors tu effaceras la mémoire d'Amalek de dessous les cieux~: Ne l'oublie point.
\Chap{26}
\TextTitle{La loi des prémices\FTNTT{cp. Ex. 23:16-19.}}
\VerseOne{}Quand tu seras entré dans le pays que Yahweh, ton Dieu, te donne en héritage, et quand tu le posséderas et y habiteras,
\VS{2}alors tu prendras des prémices de tous les fruits que tu retireras de la terre dans le pays que Yahweh, ton Dieu, te donne~; tu les mettras dans une corbeille, et tu iras au lieu que Yahweh, ton Dieu, choisira pour y faire habiter son Nom\FTNT{Ex. 23:16-19.}.
\VS{3}Et tu viendras vers le prêtre qui sera en ce temps-là, et tu lui diras~: Je déclare aujourd'hui à Yahweh, ton Dieu, que je suis entré dans le pays que Yahweh a juré à nos pères de nous donner.
\VS{4}Et le prêtre prendra la corbeille de ta main, et la posera devant l'autel de Yahweh, ton Dieu.
\VS{5}Puis tu prendras la parole, et tu diras devant Yahweh, ton Dieu~: Mon père était un Araméen qui périssait, il descendit en Egypte avec un petit nombre de gens, il y séjourna et il y devint une nation grande, puissante, et nombreuse.
\VS{6}Puis les Egyptiens nous maltraitèrent, nous humilièrent, et nous imposèrent une dure servitude.
\VS{7}Nous criâmes à Yahweh, le Dieu de nos pères. Yahweh entendit notre voix, et il vit notre souffrance, notre travail, et notre oppression.
\VS{8}Et Yahweh nous fit sortir d'Egypte, à main forte et à bras étendu, avec une grande frayeur, avec des signes et des miracles.
\VS{9}Et il nous a conduits dans ce lieu, et nous a donné ce pays où coulent le lait et le miel.
\VS{10}Maintenant donc voici, j'apporte les prémices des fruits de la terre que tu m'as donnée, ô Yahweh~! Tu les poseras devant Yahweh, ton Dieu, et tu te prosterneras devant Yahweh, ton Dieu.
\VS{11}Et tu te réjouiras de tout le bien que Yahweh, ton Dieu, t'aura donné, et à ta maison, toi et le Lévite, et l'étranger qui sera au milieu de toi.
\VS{12}Quand tu auras achevé de lever toute la dîme de ta récolte, la troisième année, l'année de la dîme, tu la donneras au Lévite, à l'étranger, à l'orphelin, et à la veuve~; ils en mangeront dans tes portes, et ils en seront rassasiés.
\VS{13}Tu diras en la présence de Yahweh, ton Dieu~: J'ai fait disparaître de ma maison ce qui est consacré, et je l'ai donné au Lévite, à l'étranger, à l'orphelin, et à la veuve, selon tous tes commandements que tu m'as ordonnés~; je n'ai transgressé ni oublié aucun de tes commandements.
\VS{14}Je n'en ai point mangé dans mon affliction, et je n'en ai rien fait disparaître pour un usage impur, et je n'en ai point donné pour un mort~; j'ai obéi à la voix de Yahweh, mon Dieu~; j'ai fait selon tout ce que tu m'avais ordonné.
\VS{15}Regarde de ta sainte demeure, des cieux, et bénis ton peuple d'Israël et la terre que tu nous as donnée, comme tu l'avais juré à nos pères, pays où coulent le lait et le miel.
\VS{16}Aujourd'hui, Yahweh, ton Dieu, t'ordonne de mettre en pratique ces lois et ces ordonnances~; prends garde de les faire de tout ton cœur et de toute ton âme.
\VS{17}Tu as fait promettre aujourd'hui à Yahweh qu'il sera ton Dieu, pour que tu marches dans ses voies, que tu observes ses lois, ses commandements et ses ordonnances, et que tu obéisses à sa voix.
\VS{18}Et aujourd'hui, Yahweh t'a fait promettre que tu seras un peuple précieux, comme il te l'a dit, et que tu observeras tous ses commandements,
\VS{19}pour qu'il te donne sur toutes les nations qu'il a créées la supériorité en louange, en renom, et en beauté, et pour que tu sois un peuple saint à Yahweh, ton Dieu, comme il te l'a dit.
\Chap{27}
\TextTitle{La loi gravée sur des pierres au mont Ebal}
\VerseOne{}Or Moïse et les anciens d'Israël ordonnèrent au peuple, en disant~: Gardez tous les commandements que je vous ordonne aujourd'hui.
\VS{2}Le jour où vous aurez traversé le Jourdain, pour entrer dans le pays que Yahweh, ton Dieu, te donne, tu dresseras de grandes pierres, et tu les enduiras de chaux.
\VS{3}Puis tu écriras sur elles toutes les paroles de cette loi, quand tu auras traversé le Jourdain, pour entrer dans le pays que Yahweh, ton Dieu, te donne, pays où coulent le lait et le miel, comme te l'a dit Yahweh, le Dieu de tes pères.
\VS{4}Quand donc vous aurez traversé le Jourdain, vous dresserez ces pierres-là sur le mont Ebal, selon ce que je vous ordonne aujourd'hui, et tu les enduiras de chaux.
\VS{5}Tu bâtiras aussi là un autel à Yahweh, ton Dieu~; un autel, dis-je, de pierres, sur lesquelles tu ne lèveras point le fer.
\VS{6}Tu bâtiras l'autel de Yahweh, ton Dieu, de pierres entières. Tu y offriras des holocaustes à Yahweh, ton Dieu~;
\VS{7}tu y offriras aussi des offrandes de paix\FTNT{Voir commentaire en Lé. 3:1.}, et tu mangeras là et te réjouiras devant Yahweh, ton Dieu.
\VS{8}Et tu écriras sur ces pierres toutes les paroles de cette loi, en les gravant bien distinctement.
\TextTitle{Les malédictions prononcées sur le mont Ebal}
\VS{9}Et Moïse et les prêtres, les Lévites, parlèrent à tout Israël, en disant~: Ecoute et garde le silence, Israël~! Aujourd'hui, tu es devenu le peuple de Yahweh, ton Dieu.
\VS{10}Tu obéiras à la voix de Yahweh, ton Dieu, et tu feras ses commandements et ses lois que je t'ordonne aujourd'hui.
\VS{11}Moïse ordonna au peuple ce jour-là, disant~:
\VS{12}Quand vous aurez traversé le Jourdain, Siméon, Lévi, Juda, Issacar, Joseph, et Benjamin, se tiendront sur le mont Garizim, pour bénir le peuple~;
\VS{13}et Ruben, Gad, Aser, Zabulon, Dan et Nephthali, se tiendront sur le mont Ebal, pour maudire.
\VS{14}Et les Lévites prendront la parole, et diront à haute voix à tous les hommes d'Israël~:
\VS{15}Maudit soit l'homme qui fait une image taillée ou une image en métal fondu, car c'est une abomination à Yahweh, œuvre des mains d'un artisan, et qui la met dans un lieu secret~! Et tout le peuple répondra, et dira~: Amen~!
\VS{16}Maudit soit celui qui méprise son père et sa mère~! Et tout le peuple dira~: Amen~!
\VS{17}Maudit soit celui qui déplace les bornes de son prochain~! Et tout le peuple dira~: Amen~!
\VS{18}Maudit soit celui qui égare un aveugle dans le chemin~! Et tout le peuple dira~: Amen~!
\VS{19}Maudit soit celui qui fait injustice à l'étranger, à l'orphelin, et à la veuve~! Et tout le peuple dira~: Amen~!
\VS{20}Maudit soit celui qui couche avec la femme de son père, car il découvre le pan de la robe de son père~! Et tout le peuple dira~: Amen~!
\VS{21}Maudit soit celui qui couche avec une bête~! Et tout le peuple dira~: Amen~!
\VS{22}Maudit soit celui qui couche avec sa sœur, fille de son père, ou fille de sa mère~! Et tout le peuple dira~: Amen~!
\VS{23}Maudit soit celui qui couche avec sa belle-mère~! Et tout le peuple dira~: Amen~!
\VS{24}Maudit soit celui qui frappe son prochain en secret~! Et tout le peuple dira~: Amen~!
\VS{25}Maudit soit celui qui reçoit un présent pour mettre à mort un homme, en versant le sang innocent~! Et tout le peuple dira~: Amen~!
\VS{26}Maudit soit celui qui n'accomplit point les paroles de cette loi et ne les met pas en pratique~! Et tout le peuple dira~: Amen~!
\Chap{28}
\TextTitle{Les bénédictions accompagnent l'obéissance}
\VerseOne{}Or il arrivera que si tu écoutes attentivement la voix de Yahweh, ton Dieu, et que tu prennes garde de pratiquer tous ses commandements que je t'ordonne aujourd'hui, Yahweh, ton Dieu, te donnera la supériorité sur toutes les nations de la terre.
\VS{2}Voici toutes les bénédictions qui viendront sur toi, et qui t'atteindront, quand tu obéiras à la voix de Yahweh, ton Dieu~:
\VS{3}Tu seras béni dans la ville, et tu seras aussi béni aux champs.
\VS{4}Le fruit de tes entrailles, le fruit de ta terre, le fruit de tes troupeaux, les portées de ton gros et de ton menu bétail seront bénis.
\VS{5}Ta corbeille et ta huche seront bénies.
\VS{6}Tu seras béni en entrant, et tu seras béni en sortant.
\VS{7}Yahweh fera que tes ennemis qui s'élèveront contre toi seront battus devant toi, ils sortiront contre toi par un chemin, et ils s'enfuiront devant toi par sept chemins.
\VS{8}Yahweh ordonnera à la bénédiction d'être avec toi dans tes greniers et dans tout ce à quoi tu mettras ta main~; il te bénira dans le pays que Yahweh, ton Dieu, te donne.
\VS{9}Yahweh t'établira pour lui être un peuple saint, comme il te l'a juré, quand tu garderas les commandements de Yahweh, ton Dieu, et que tu marcheras dans ses voies.
\VS{10}Et tous les peuples de la terre verront que tu es appelé du Nom de Yahweh, et ils te craindront.
\VS{11}Yahweh te fera abonder de biens dans le fruit de tes entrailles, le fruit de tes troupeaux, et le fruit de ton sol, sur la terre que Yahweh a juré à tes pères de te donner.
\VS{12}Yahweh t'ouvrira son bon trésor, les cieux, pour donner à ton pays la pluie en sa saison et pour bénir tout le travail de tes mains~; tu prêteras à beaucoup de nations, et tu n'emprunteras point.
\VS{13}Yahweh te mettra à la tête et non à la queue, tu seras toujours en haut et jamais en bas, lorsque tu obéiras aux commandements de Yahweh, ton Dieu, que je t'ordonne aujourd'hui, afin que tu prennes garde de les faire,
\VS{14}et que tu ne te détournes ni à droite ni à gauche de toutes les paroles que je t'ordonne aujourd'hui, pour aller après d'autres dieux et pour les servir.
\TextTitle{Les malédictions accompagnent la désobéissance}
\VS{15}Mais si tu n'obéis point à la voix de Yahweh, ton Dieu, pour prendre garde de pratiquer tous ses commandements et ses lois que je t'ordonne aujourd'hui, voici toutes les malédictions qui viendront sur toi, et qui t'atteindront~:
\VS{16}Tu seras maudit dans la ville, et tu seras maudit dans les champs.
\VS{17}Ta corbeille et ta huche seront maudites.
\VS{18}Le fruit de tes entrailles, le fruit de ta terre, les portées de ton gros et de ton menu bétail seront maudits.
\VS{19}Tu seras maudit à ton entrée, et tu seras maudit à ta sortie.
\VS{20}Yahweh enverra sur toi la malédiction, la confusion, et la ruine dans tout ce à quoi tu mettras ta main et que tu feras, jusqu'à ce que tu sois détruit, et que tu périsses promptement, à cause de la méchanceté de tes pratiques, par lesquelles tu m'auras abandonné.
\VS{21}Yahweh fera que la peste s'attachera à toi, jusqu'à ce qu'elle te consume sur la terre où tu vas entrer pour en prendre possession.
\VS{22}Yahweh te frappera de tuberculose, de fièvre, d'inflammation, de chaleur brûlante, de l'épée, de sécheresse et de rouille, qui te poursuivront jusqu'à ce que tu périsses.
\VS{23}Les cieux sur ta tête seront d'airain, et la terre sous toi sera de fer.
\VS{24}Yahweh te donnera pour pluie à ton pays de la poussière et de la poudre, qui descendra des cieux sur toi jusqu'à ce que tu sois détruit.
\VS{25}Yahweh fera que tu seras battu devant tes ennemis~; tu sortiras par un chemin contre eux, et tu t'enfuiras devant eux par sept chemins~; et tu seras tremblant face à tous les royaumes de la terre.
\VS{26}Ton cadavre sera la viande de tous les oiseaux des cieux et des bêtes de la terre~; et il n'y aura personne qui les effraye.
\VS{27}Yahweh te frappera de l'ulcère d'Egypte, d'hémorroïdes, de gale, et de teigne, dont tu ne pourras guérir.
\VS{28}Yahweh te frappera de folie, d'aveuglement, et d'égarement d'esprit~;
\VS{29}et tu tâtonneras en plein midi comme tâtonne un aveugle dans l'obscurité, tu ne prospéreras pas dans tes voies, et tu seras opprimé et dépouillé tous les jours, et il n'y aura personne pour venir te sauver.
\VS{30}Tu fianceras une femme, mais un autre homme couchera avec elle et la violera~; tu bâtiras une maison, mais tu ne l'habiteras point~; tu planteras une vigne, mais tu n'en jouiras point.
\VS{31}Ton bœuf sera tué sous tes yeux, et tu n'en mangeras point~; ton âne sera enlevé devant toi, et on ne te le rendra point~; tes brebis seront livrées à tes ennemis, et il n'y aura personne pour te sauver.
\VS{32}Tes fils et tes filles seront livrés à un autre peuple, tes yeux le verront, et languiront tout le jour après eux, et tu n'auras aucun pouvoir en ta main.
\VS{33}Un peuple que tu n'auras point connu mangera le fruit de ta terre et tout ton travail, et tu seras opprimé et écrasé tous les jours.
\VS{34}Tu deviendras fou à cause de ce que tu verras de tes yeux.
\VS{35}Yahweh te frappera d'un ulcère malin sur les genoux et sur les cuisses dont tu ne pourras guérir, il t'en frappera depuis la plante du pied jusqu'au sommet de ta tête.
\VS{36}Yahweh te fera marcher, toi et ton roi que tu auras établi sur toi, vers une nation que tu n'auras point connue, ni toi ni tes pères. Et là, tu serviras d'autres dieux, du bois et de la pierre.
\VS{37}Et tu seras un sujet d'étonnement, de proverbes, de railleries, parmi tous les peuples vers lesquels Yahweh t'aura emmené.
\VS{38}Tu jetteras beaucoup de semence dans ton champ, et tu recueilleras peu, car les sauterelles la consumeront.
\VS{39}Tu planteras des vignes et tu les cultiveras~; mais tu n'en boiras point le vin et tu n'en recueilleras rien, car les vers la mangeront.
\VS{40}Tu auras des oliviers sur tout le territoire~; mais tu ne t'oindras point d'huile, car tes olives perdront leurs fruits.
\VS{41}Tu engendreras des fils et des filles~; mais ils ne seront pas à toi, car ils iront en captivité.
\VS{42}Les insectes posséderont tous tes arbres et le fruit de ta terre.
\VS{43}L'étranger qui sera au milieu de toi montera toujours plus au-dessus de toi, et toi, tu descendras toujours plus bas.
\VS{44}Il te prêtera, et tu ne lui prêteras point~; il sera à la tête, et tu seras à la queue.
\VS{45}Toutes ces malédictions viendront sur toi, elles te poursuivront et t'atteindront jusqu'à ce que tu sois détruit, parce que tu n'auras pas obéi à la voix de Yahweh, ton Dieu, pour garder ses commandements et ses lois qu'il t'a ordonnés.
\VS{46}Et ces choses seront à jamais pour toi et ta postérité comme des signes et des prodiges.
\VS{47}Et parce que tu n'auras pas servi Yahweh, ton Dieu, avec joie, et de bon cœur, malgré l'abondance de toutes choses,
\VS{48}tu serviras, dans la faim, dans la soif, dans la nudité, et dans la disette de toutes choses, ton ennemi que Yahweh enverra contre toi. Il mettra un joug de fer sur ton cou, jusqu'à ce qu'il t'ait détruit.
\TextTitle{Prophétie sur l'invasion babylonienne et la dispersion d'Israël}
\VS{49}Yahweh fera lever de loin, des extrémités de la terre, une nation qui volera comme l'aigle, une nation dont tu ne comprendras pas la langue,
\VS{50}une nation au visage féroce, et qui ne soutiendra point le vieillard et n'aura point pitié pour l'enfant\FTNT{Cette prophétie s'est accomplie en 587 av. J.-C. Voir 2 R. 24-25.}.
\VS{51}Elle mangera le fruit de tes troupeaux et les fruits de ta terre, jusqu'à ce que tu sois détruit~; elle n'épargnera ni blé, ni vin, ni huile, ni portée de ton gros et de ton menu bétail, jusqu'à ce qu'elle t'ait fait périr.
\VS{52}Et elle t'assiégera dans toutes tes portes, jusqu'à ce que tombent ces hautes et fortes murailles dans lesquelles tu auras mis ta confiance dans tout ton pays~; elle t'assiégera, dis-je, dans toutes tes portes, dans tout le pays que Yahweh, ton Dieu, te donne.
\VS{53}Tu mangeras le fruit de tes entrailles, la chair de tes fils et de tes filles que Yahweh, ton Dieu, t'aura donnés, durant le siège et la détresse dont ton ennemi te serrera.
\VS{54}L'homme le plus tendre et le plus délicat d'entre vous regardera d'un œil malin son frère, sa femme bien-aimée, et le reste de ses fils qu'il a épargnés~;
\VS{55}pour ne donner à aucun d'eux de la chair de ses fils, qu'il mangera, parce qu'il ne lui restera rien du tout, à cause du siège et de la détresse dont ton ennemi te serrera dans toutes tes portes.
\VS{56}La femme la plus tendre et la plus délicate d'entre vous, qui n'a point osé mettre la plante de son pied sur la terre, par délicatesse et par mollesse, regardera d'un œil malin son mari bien-aimé, son fils, et sa fille~;
\VS{57}et le placenta qui sortira d'entre ses jambes, et les fils qu'elle enfantera~; car manquant de tout, elle les mangera secrètement, à cause du siège et de la détresse, dont ton ennemi te serrera dans toutes les villes.
\VS{58}Si tu ne prends pas garde d'observer toutes les paroles de cette loi, qui sont écrites dans ce livre, en craignant le Nom glorieux et redoutable de Yahweh, ton Dieu,
\VS{59}alors Yahweh rendra difficile tes plaies et les plaies de ta postérité, par des plaies grandes et persistantes, des maladies malignes et persistantes. 
\VS{60}Et il fera retourner sur toi toutes les maladies d'Egypte, devant lesquelles tu avais peur~; et elles s'attacheront à toi.
\VS{61}Même Yahweh fera venir sur toi toutes maladies et toutes plaies, qui ne sont point écrites dans le livre de cette loi, jusqu'à ce que tu sois détruit.
\VS{62}Et vous resterez en petit nombre, après avoir été aussi nombreux que les étoiles des cieux, parce que tu n'auras point obéi à la voix de Yahweh, ton Dieu.
\VS{63}Et il arrivera que comme Yahweh s'est réjoui sur vous, en vous faisant du bien et en vous multipliant, de même Yahweh se réjouira sur vous en vous faisant périr et en vous détruisant~; et vous serez arrachés de la terre dans laquelle vous allez entrer en possession.
\VS{64}Et Yahweh te dispersera parmi tous les peuples, d'un bout de la terre jusqu'à l'autre~; et là, tu serviras d'autres dieux que ni toi ni tes pères n'avez connus, le bois et la pierre.
\VS{65}Tu n'auras aucun repos parmi ces nations, même la plante de ton pied n'aura aucun repos. Car Yahweh te donnera un cœur tremblant, des yeux languissants, et une âme souffrante.
\VS{66}Et ta vie sera en suspens devant toi, tu trembleras la nuit et le jour, et tu ne seras point sûr de ta vie.
\VS{67}Tu diras le matin~: Qui me fera voir le soir~? Et le soir tu diras~: Qui me fera voir le matin~? A cause de l'effroi dont ton cœur sera effrayé, et à cause des choses que tu verras de tes yeux.
\VS{68}Et Yahweh te fera retourner en Egypte sur des navires, pour faire le chemin dont je t'ai dit~: Tu ne le verras plus~; et là, vous vous vendrez à vos ennemis, comme esclaves et servantes~; et il n'y aura personne pour vous acheter.
\Chap{29}
\TextTitle{Yahweh rappelle sa fidélité à Israël}
\VerseOne{}Voici les paroles de l'alliance que Yahweh ordonna à Moïse de traiter avec les enfants d'Israël au pays de Moab, outre l'alliance qu'il avait traitée avec eux à Horeb.
\VS{2}Moïse donc appela tout Israël, et leur dit~: Vous avez vu tout ce que Yahweh a fait sous vos yeux, dans le pays d'Egypte, à Pharaon, à tous ses serviteurs, et à tout son pays,
\VS{3}les grandes épreuves que tes yeux ont vues, ces signes et ces grands miracles.
\VS{4}Mais, jusqu'à ce jour, Yahweh ne vous a point donné un cœur pour connaître, ni des yeux pour voir, ni des oreilles pour entendre.
\VS{5}Je t'ai conduit pendant quarante ans par le désert~; tes vêtements ne se sont point usés, et ton soulier ne s'est point usé à ton pied.
\VS{6}Vous n'avez point mangé de pain, ni bu de vin ni de liqueur forte, afin que vous connaissiez que je suis Yahweh, votre Dieu.
\VS{7}Et vous êtes parvenus dans ce lieu~; Sihon, roi de Hesbon, et Og, roi de Basan, sont sortis à notre rencontre, pour nous combattre, et nous les avons battus.
\VS{8}Et nous avons pris leur pays, et nous l'avons donné en héritage aux Rubénites, aux Gadites, et à la demi-tribu des Manassites.
\TextTitle{Béni celui qui reste fidèle à l'alliance}
\VS{9}Vous garderez donc les paroles de cette alliance, et vous les pratiquerez, afin de réussir dans tout ce que vous ferez.
\VS{10}Vous vous tiendrez aujourd'hui devant Yahweh, votre Dieu, vos chefs de tribus, vos anciens, vos officiers, tous les hommes d'Israël,
\VS{11}vos enfants, vos femmes, et l'étranger qui est au milieu de ton camp, depuis celui qui coupe ton bois jusqu'à celui qui puise ton eau~;
\VS{12}afin que tu entres dans l'alliance de Yahweh, ton Dieu, dans ce serment, que Yahweh, ton Dieu, traite aujourd'hui avec toi,
\VS{13}afin qu'il t'établisse aujourd'hui pour son peuple et qu'il soit ton Dieu, comme il te l'a dit, et comme il l'a juré à tes pères, Abraham, Isaac et Jacob.
\VS{14}Et ce n'est pas seulement avec vous que je traite cette alliance, ce serment.
\VS{15}Mais c'est avec ceux qui sont ici, avec nous aujourd'hui devant Yahweh, notre Dieu, et avec ceux qui ne sont point ici, avec nous aujourd'hui.
\TextTitle{Mise en garde contre celui qui abandonne l'alliance}
\VS{16}Car vous savez comment nous avons habité dans le pays d'Egypte, et comment nous sommes passés au milieu des nations, que vous avez traversées.
\VS{17}Et vous avez vu leurs abominations et leurs idoles, le bois et la pierre, l'argent et l'or qui sont parmi eux.
\VS{18}Qu'il n'y ait parmi vous ni homme, ni femme, ni famille, ni tribu qui détourne son cœur aujourd'hui de Yahweh, notre Dieu, pour aller servir les dieux de ces nations. Qu'il n'y ait parmi vous de racine qui produise du poison et de l'absinthe.
\VS{19}Et qu'il n'arrive que quelqu'un en entendant les paroles de cette malédiction, ne se bénisse dans son cœur, en disant~: J'aurai la paix, même si je marche dans les penchants de mon cœur, et que j'ajoute l'ivresse à la soif.
\VS{20}Yahweh ne voudra point lui pardonner. Mais la colère de Yahweh et la jalousie s'enflammeront contre cet homme, et toutes les malédictions écrites dans ce livre reposeront sur lui, et Yahweh effacera son nom de dessous les cieux.
\VS{21}Et Yahweh le séparera de toutes les tribus d'Israël, pour son malheur, selon toutes les malédictions de l'alliance écrite dans ce livre de la loi.
\VS{22}Et la génération à venir, vos fils qui se lèveront après vous, et l'étranger qui viendra d'un pays lointain, quand ils verront les plaies et les maladies, dont Yahweh aura frappé ce pays~;
\VS{23}et que toute la terre de ce pays-là ne sera que soufre, que sel, et qu'embrasement, qu'elle ne sera point semée, et qu'elle ne fera rien germer, et que nulle herbe n'en sortira, ainsi qu'en la subversion de Sodome, et de Gomorrhe, et d'Adma, et de Tseboïm, que Yahweh détruisit dans sa colère et dans sa fureur.
\VS{24}Mais toutes les nations diront~: Pourquoi Yahweh a-t-il traité ainsi ce pays~? D'où vient l'ardeur de cette grande colère~?
\VS{25}Et on répondra~: C'est parce qu'ils ont abandonné l'alliance de Yahweh, le Dieu de leurs pères, qu'il a traitée avec eux quand il les fit sortir du pays d'Egypte~;
\VS{26}car ils sont allés servir d'autres dieux et se sont prosternés devant eux~; des dieux qu'ils ne connaissaient point et qu'il ne leur avait point donnés en partage.
\VS{27}A cause de cela, la colère de Yahweh s'est enflammée contre ce pays, et il a fait venir sur lui toutes les malédictions écrites dans ce livre.
\VS{28}Et Yahweh les a arrachés de leur terre avec colère, avec fureur, avec une grande indignation, et il les a chassés sur un autre pays, comme on le voit aujourd'hui.
\VS{29}Les choses cachées sont à Yahweh, notre Dieu~; les choses révélées sont à nous et à nos fils, à jamais, afin que nous pratiquions toutes les paroles de cette loi.
\Chap{30}
\TextTitle{Yahweh bénira et restaurera le peuple repentant}
\VerseOne{}Or il arrivera que lorsque toutes ces choses seront venues sur toi, la bénédiction et la malédiction, que je mets devant toi, si tu les rappelles dans ton cœur, parmi toutes les nations vers lesquelles Yahweh, ton Dieu, t'aura chassé~;
\VS{2}si tu reviens à Yahweh, ton Dieu, et si tu obéis à sa voix de tout ton cœur, de toute ton âme, toi et tes fils, selon tout ce que je t'ordonne aujourd'hui,
\VS{3}Yahweh, ton Dieu, ramènera tes captifs et aura compassion de toi~; il te rassemblera encore du milieu de tous les peuples parmi lesquels Yahweh, ton Dieu, t'aura dispersé.
\VS{4}Quand tu seras dispersé à l'extrémité des cieux, Yahweh, ton Dieu, te rassemblera de là, et de là, il te prendra.
\VS{5}Yahweh, ton Dieu, dis-je, te ramènera dans le pays que tes pères possédaient, et tu le posséderas~; il te fera du bien, et te rendra plus nombreux que tes pères.
\VS{6}Yahweh, ton Dieu, circoncira ton cœur, et le cœur de ta postérité, pour que tu aimes Yahweh, ton Dieu, de tout ton cœur, et de toute ton âme, afin que tu vives\FTNT{Ro. 2:29.}.
\VS{7}Et Yahweh, ton Dieu, mettra toutes ces malédictions sur tes ennemis, et sur ceux qui te haïront et te persécuteront.
\VS{8}Ainsi tu retourneras à Yahweh, tu obéiras à sa voix, et tu feras tous ses commandements que je t'ordonne aujourd'hui.
\TextTitle{Faire connaître la loi aux futures générations}
\VS{9}Et Yahweh, ton Dieu, te fera abonder en bien dans toute l'œuvre de ta main, dans le fruit de tes entrailles, dans le fruit de tes troupeaux et dans le fruit de ta terre~; car Yahweh se réjouira de nouveau de ton bonheur, comme il s'est réjoui de celui de tes pères,
\VS{10}quand tu obéiras à la voix de Yahweh, ton Dieu, en gardant ses commandements et ses ordonnances écrites dans ce livre de la loi, quand tu reviendras à Yahweh, ton Dieu, de tout ton cœur et de toute ton âme.
\TextTitle{Le peuple devant un choix}
\VS{11}Car ce commandement que je t'ordonne aujourd'hui n'est pas trop difficile pour toi et hors de ta portée.
\VS{12}Il n'est pas aux cieux, pour dire~: Qui montera pour nous aux cieux, nous l'apportera et nous le fera entendre, pour que nous le fassions~?
\VS{13}Il n'est point aussi de l'autre côté de la mer pour dire~: Qui passera de l'autre côté de la mer pour nous, et nous l'apportera, et nous le fera entendre pour que nous le fassions~?
\VS{14}Car cette parole est fort près de toi, dans ta bouche et dans ton cœur, afin que tu la pratiques\FTNT{Ro. 10:6.}.
\VS{15}Regarde, je mets aujourd'hui devant toi la vie et le bien, la mort et le mal.
\VS{16}Car je t'ordonne aujourd'hui d'aimer Yahweh, ton Dieu, de marcher dans ses voies, de garder ses commandements, ses lois, et ses ordonnances, afin que tu vives, que tu multiplies, et que Yahweh, ton Dieu, te bénisse dans le pays où tu vas entrer en possession.
\VS{17}Mais si ton cœur se détourne, si tu n'obéis point, et si tu te laisses entraîner à te prosterner devant d'autres dieux et à les servir,
\VS{18}je vous déclare aujourd'hui que vous périrez certainement, et que vous ne prolongerez point vos jours sur la terre dont vous allez entrer en possession, après avoir passé le Jourdain.
\VS{19}J'en prends aujourd'hui à témoin les cieux et la terre contre vous~: J'ai mis devant toi la vie et la mort, la bénédiction et la malédiction. Choisis donc la vie\FTNT{cp. Mt. 7:13-14.}, afin que tu vives, toi et ta postérité.
\VS{20}en aimant Yahweh, ton Dieu, en obéissant à sa voix, et en t'attachant à lui~: Car c'est lui qui est ta vie et la longueur de tes jours, afin que tu demeures sur la terre que Yahweh a juré à tes pères, Abraham, Isaac, et Jacob, de leur donner.
\Chap{31}
\TextTitle{Moïse encourage et affermit le peuple}
\VerseOne{}Puis Moïse s'en alla, et dit ces paroles à tout Israël~:
\VS{2}Aujourd'hui, leur dit-il, je suis âgé de cent vingt ans, je ne pourrai plus sortir ni entrer, et Yahweh m'a dit~: Tu ne passeras point ce Jourdain.
\VS{3}Yahweh, ton Dieu, passera lui-même devant toi, il détruira ces nations devant toi, et tu les posséderas. Josué passera aussi devant toi, comme Yahweh l'a dit.
\VS{4}Et Yahweh leur fera comme il a fait à Sihon et à Og, rois des Amoréens, qu'il a détruits avec leurs pays.
\VS{5}Et Yahweh les livrera devant vous, et vous leur ferez selon tout le commandement que je vous ai ordonné.
\VS{6}Fortifiez-vous donc et prenez courage~! Ne craignez point et ne soyez point effrayés devant eux~; car Yahweh, ton Dieu, marchera avec toi, il ne te délaissera point et ne t'abandonnera point.
\VS{7}Et Moïse appela Josué, et lui dit en présence de tout Israël~: Fortifie-toi et prends courage, car tu entreras avec ce peuple dans le pays que Yahweh a juré à leurs pères de leur donner, et c'est toi qui les en mettras en possession.
\VS{8}Yahweh est celui qui marchera devant toi, il sera lui-même avec toi, il ne te délaissera point, il ne t'abandonnera point~; ne crains point, et ne t'effraie point.
\VS{9}Or Moïse écrivit cette loi, et il la donna aux prêtres, fils de Lévi, qui portaient l'arche de l'alliance de Yahweh, et à tous les anciens d'Israël.
\VS{10}Moïse leur ordonna, en disant~: Tous les sept ans, au temps fixé de l'année du relâche, à la fête des tabernacles,
\VS{11}quand tout Israël viendra se présenter devant Yahweh, ton Dieu, dans le lieu qu'il aura choisi, tu liras alors cette loi devant tout Israël, à leurs oreilles.
\VS{12}Tu rassembleras le peuple, les hommes, les femmes, les enfants et l'étranger qui sera dans tes portes, pour qu'ils t'entendent, et qu'ils apprennent à craindre Yahweh, votre Dieu, et qu'ils prennent garde de faire toutes les paroles de cette loi.
\VS{13}Et leurs fils qui ne la connaîtront point l'entendront, et ils apprendront à craindre Yahweh, votre Dieu, tous les jours que vous vivrez sur cette terre que vous allez posséder après avoir passé le Jourdain.
\TextTitle{Yahweh annonce les événements à venir}
\VS{14}Alors Yahweh dit à Moïse~: Voici, le jour où tu vas mourir est proche. Appelle Josué, et tenez-vous dans la tente d'assignation. Je lui donnerai mes ordres. Moïse et Josué allèrent et se présentèrent dans la tente d'assignation.
\VS{15}Et Yahweh apparut dans la tente, dans une colonne de nuée~; et la colonne de nuée s'arrêta à l'entrée de la tente.
\VS{16}Yahweh dit à Moïse~: Voici, tu vas te coucher avec tes pères. Et ce peuple se lèvera et se prostituera après les dieux étrangers du pays au milieu duquel il va entrer. Il m'abandonnera et violera mon alliance que j'ai traitée avec lui.
\VS{17}En ce jour-là ma colère s'enflammera contre lui. Je les abandonnerai, et je leur cacherai ma face. Il sera dévoré, une multitude de maux et d'angoisses l'atteindront, et il dira en ce jour-là~: N'est-ce pas parce que mon Dieu n'est point au milieu de moi, que ces maux m'ont atteint~?
\VS{18}En ce jour-là, dis-je, je cacherai entièrement ma face, à cause de tout le mal qu'il aura fait, parce qu'il se sera tourné vers d'autres dieux.
\VS{19}Maintenant donc, écrivez ce cantique. Enseigne-le aux enfants d'Israël, mets-le dans leur bouche, afin que ce cantique me serve de témoignage contre les fils d'Israël.
\VS{20}Car je le conduirai sur la terre que j'ai juré à ses pères, où coulent le lait et le miel~; il mangera, se rassasiera, et s'engraissera~; puis il se tournera vers d'autres dieux, et il les servira, il m'irritera par mépris et violera mon alliance~;
\VS{21}et il arrivera qu'il sera atteint par une multitude de maux et d'angoisses, ce cantique, qui ne sera point oublié et qui sera dans la bouche de la postérité, répondra comme témoin contre eux. Je connais ses desseins, qu'il a déjà préparés aujourd'hui, avant même que je l'aie fait entrer dans le pays que j'ai juré.
\VS{22}Ainsi Moïse écrivit ce cantique en ce jour-là, et l'enseigna aux enfants d'Israël.
\VS{23}Et Yahweh commanda à Josué, fils de Nun, en disant~: Fortifie-toi et prends courage, car c'est toi qui feras entrer les enfants d'Israël dans le pays que je leur ai juré~; et je serai avec toi.
\VS{24}Et il arrivera que quand Moïse eut achevé d'écrire dans un livre les paroles de cette loi jusqu'à ce qu'elle soit complète,
\VS{25}Moïse ordonna aux Lévites qui portaient l'arche de l'alliance de Yahweh, en disant~:
\VS{26}Prenez ce livre de la loi, et mettez-le à côté de l'arche de l'alliance de Yahweh, votre Dieu, et il sera là comme témoin contre toi.
\VS{27}Car je connais ta rébellion et ton cou raide. Voici, déjà aujourd'hui étant en vie avec vous, vous avez été rebelles contre Yahweh, combien plus le serez-vous après ma mort~?
\VS{28}Faites assembler devant moi tous les anciens de vos tribus, et vos officiers, et je dirai ces paroles en leur présence, et j'appellerai à témoin contre eux les cieux et la terre.
\VS{29}Car je sais qu'après ma mort vous vous corromprez, et que vous vous détournerez de la voie que je vous ai ordonnée~; mais à la fin, le malheur vous atteindra, parce que vous aurez fait ce qui déplaît aux yeux de Yahweh, en l'irritant par les œuvres de vos mains.
\VS{30}Ainsi Moïse prononça entièrement les paroles de ce cantique-ci, en présence de toute l'assemblée d'Israël.
\Chap{32}
\TextTitle{Cantique de Moïse}
\VerseOne{}Cieux~! Prêtez l'oreille, et je parlerai. Terre~! écoute les paroles de ma bouche.
\VS{2}Que mon enseignement tombe comme la pluie, que ma parole se répande comme la rosée, comme une pluie fine sur l'herbe naissante, et comme une averse sur la verdure~!
\VS{3}Car j'invoquerai le Nom de Yahweh~; attribuez la grandeur à notre Dieu.
\VS{4}L'œuvre du rocher\FTNT{Voir commentaire en Es. 8:13-17.} est parfaite, car toutes ses voies sont justes. C'est un Dieu fidèle et sans iniquité, il est juste et droit.
\VS{5}Ils se sont corrompus, à lui n'est point la faute~; la faute est à ses fils, c'est une génération fausse et tortueuse.
\TextTitle{Israël, le choix de Yahweh}
\VS{6}Est-ce ainsi que tu récompenses Yahweh, peuple insensé et dépourvu de sagesse~? N'est-il pas ton père, celui qui t'a acquis~? Il t'a fait et t'a façonné.
\VS{7}Souviens-toi des anciens jours, considère les années, de génération en génération, interroge ton père, et il te l'apprendra, et tes anciens, et ils te le diront.
\VS{8}Quand le Très-Haut laissa un héritage aux nations, quand il sépara les enfants des hommes, il fixa les limites des peuples selon le nombre des fils d'Israël~;
\VS{9}car la portion de Yahweh, c'est son peuple, Jacob est le lot de son héritage.
\VS{10}Il l'a trouvé dans un pays désert, dans la désolation des hurlements d'une solitude, il l'a entouré, il l'a dirigé, il l'a gardé comme la prunelle de son œil,
\VS{11}comme l'aigle éveille sa nichée, couve ses petits, étend ses ailes, les prend, les porte sur ses ailes.
\VS{12}Yahweh seul l'a conduit, et il n'y a point eu avec lui de dieu étranger.
\VS{13}Il l'a fait monter à cheval sur les hauteurs du pays, et il a mangé les fruits des champs~; il lui a donné à sucer le miel du rocher, l'huile du rocher le plus dur,
\VS{14}la crème des vaches, le lait des brebis, et la graisse des agneaux, des béliers de Basan, et des boucs, et la fleur du froment~; et tu as bu le vin qui était le sang de la grappe.
\TextTitle{Condamnation de l'apostasie d'Israël}
\VS{15}Jeshurun\FTNT{Littéralement «~Jeshurun~» en hébreu~: «~celui qui est droit~». Nom symbolique donné à Israël pour décrire son caractère idéal.} s'est engraissé, et a regimbé~; tu es devenu gras, gros et épais~! Et il a abandonné Dieu qui l'a fait, et il a méprisé le rocher de son salut.
\VS{16}Ils ont provoqué sa jalousie par des dieux étrangers, ils l'ont irrité par des abominations.
\VS{17}Ils ont sacrifié à des démons, qui ne sont point Dieu~; aux dieux qu'ils ne connaissaient point, dieux nouveaux, venus depuis peu, et que vos pères n'ont point redoutés.
\VS{18}Tu as oublié le rocher qui t'a engendré, et tu as oublié le Dieu qui t'a fait naître.
\VS{19}Yahweh l'a vu, et a été irrité, parce que ses fils et ses filles l'ont provoqué à la colère.
\VS{20}Et il a dit~: Je cacherai ma face, je verrai quelle sera leur fin~; car ils sont une génération perverse, des fils infidèles.
\VS{21}Ils ont excité ma jalousie par ce qui n'est point Dieu, ils m'ont irrité par leurs vanités~; ainsi je provoquerai leur jalousie par ce qui n'est point un peuple, et je les offenserai par une nation insensée.
\VS{22}Car le feu de ma colère s'est allumé, et brûlera jusqu'au fond du scheol, dévorera la terre et son fruit, et embrasera les fondements des montagnes.
\VS{23}Je rassemblerai sur eux des maux, et je détruirai toutes mes flèches sur eux.
\VS{24}Ils seront consumés par la famine, rongés par des charbons ardents, et par une destruction amère~; j'enverrai contre eux la dent des bêtes et le venin des serpents qui rampent sur la poussière.
\VS{25}L'épée venant de dehors les privera les uns des autres~; et au-dedans, la terreur les privera d'enfants. Il en sera du jeune homme comme de la vierge, de l'enfant à la mamelle comme de l'homme aux cheveux blancs.
\TextTitle{A Yahweh la vengeance et la rétribution}
\VS{26}Je dirais~: Je les détruirai, et je ferai disparaître leur mémoire d'entre les hommes~!
\VS{27}Si je ne craignais la colère de l'ennemi, de peur que leurs adversaires ne se méprennent, et ne disent~: Notre main est élevée, et ce n'est pas Yahweh qui a fait tout ceci.
\VS{28}Car c'est une nation qui se perd par ses conseils, et il n'y a en eux aucune intelligence.
\VS{29}Ô s'ils étaient sages, ils comprendraient ceci, et ils considéreraient leur fin.
\VS{30}Comment un seul en poursuivrait-il mille, et deux en mettraient-ils dix mille en fuite, si ce n'était que leur Rocher les avait vendus, et que Yahweh ne les avait enserrés~?
\VS{31}Car leur rocher n'est pas comme notre Rocher, nos ennemis en sont juges.
\VS{32}Car leur vigne est du plant de Sodome, et du terroir de Gomorrhe~; leurs raisins sont des raisins empoisonnés, leurs grappes sont amères.
\VS{33}Leur vin est un venin de dragon, et du poison cruel d'aspic.
\VS{34}Cela n'est-il pas caché près de moi, scellé dans mes trésors~?
\VS{35}A moi la vengeance et la rétribution, le temps où leur pied glissera~! Car le jour de leur calamité est près, et les choses qui doivent leur arriver se hâtent.
\VS{36}Mais Yahweh jugera son peuple~; et il se repentira en faveur de ses serviteurs, quand il verra que leur force a disparu, et qu'il n'y a personne de retenu ni d'abandonné.
\VS{37}Et il dira~: Où sont leurs dieux, le rocher en qui ils se confiaient,
\VS{38}qui mangeaient la graisse de leurs sacrifices, qui buvaient le vin de leurs libations~? Qu'ils se lèvent, qu'ils vous aident, et qu'ils vous servent de refuge~!
\VS{39}Voyez maintenant que moi,JE SUIS\FTNT{«~JE SUIS. Il s'agit ici du Nom que Dieu a révélé à Moïse et à Esaïe (Ex. 3:14) et sous lequel Jésus s'est présenté (Jn. 18:5-8).}, et il n'y a point de dieu avec moi\FTNT{Ce verset confirme que Dieu est un puisqu'il n'y a pas d'autres dieux à ses côtés.}~; je fais mourir et je fais vivre, je blesse et je guéris~; et il n'y a personne qui puisse délivrer de ma main.
\VS{40}Car je lève ma main au ciel, et je dis~: Je vis éternellement.
\VS{41}Si j'aiguise l'éclair de mon épée, et si ma main saisit la justice, je rendrai la vengeance à mes adversaires et je rétribuerai ceux qui me haïssent.
\VS{42} J'enivrerai mes flèches de sang et mon épée dévorera la chair, j'enivrerai, dis-je, mes flèches du sang des tués et des captifs, de la tête des chefs de l'ennemi.
\VS{43}Nations, réjouissez-vous avec son peuple~! Car il venge le sang de ses serviteurs, il tire vengeance de ses ennemis, et fait propitiation pour sa terre et pour son peuple.
\TextTitle{Fin du cantique, invitation à demeurer fidèle}
\VS{44}Moïse donc vint et prononça toutes les paroles de ce cantique, à l'oreille du peuple, lui et Josué, fils de Nun.
\VS{45}Et quand Moïse eut achevé de prononcer toutes ces paroles à tout Israël,
\VS{46}il leur dit~: Appliquez votre cœur à toutes ces paroles que je vous conjure aujourd'hui d'ordonner à vos fils, afin qu'ils prennent garde de faire toutes les paroles de cette loi.
\VS{47}Car ce n'est pas une parole vaine pour vous, mais c'est votre vie~; et par cette parole vous prolongerez vos jours sur la terre que vous posséderez, après avoir passé le Jourdain.
\TextTitle{Moïse, invité à monter sur le mont Nebo}
\VS{48}En ce même jour-là, Yahweh parla à Moïse, en disant~:
\VS{49}Monte sur cette montagne d'Abarim, sur le mont Nebo, au pays de Moab, vis-à-vis de Jéricho~; et regarde le pays de Canaan, que je donne en possession aux enfants d'Israël.
\VS{50}Tu mourras sur la montagne où tu vas monter, et tu seras recueilli vers ton peuple, comme Aaron, ton frère, est mort sur la montagne d'Hor, et a été recueilli vers son peuple,
\VS{51}parce que vous avez péché contre moi au milieu des fils d'Israël, aux eaux de Meriba, à Kadès, dans le désert de Tsin~; car vous ne m'avez point sanctifié au milieu des enfants d'Israël.
\VS{52}Tu verras le pays devant toi, mais tu n'entreras point dans le pays, que je donne aux enfants d'Israël.
\Chap{33}
\TextTitle{Moïse bénit les tribus d'Israël}
\VerseOne{}Or c'est ici la bénédiction dont Moïse, homme de Dieu, bénit les enfants d'Israël avant sa mort.
\VS{2}Il dit donc~: Yahweh est venu de Sinaï, il s'est levé sur eux de Séir, il a resplendi de la montagne de Paran, et il est sorti d'entre les dix milliers des saints, et de sa droite le feu de la loi est sorti vers eux.
\VS{3}En effet, il aime les peuples~; tous ses saints sont dans ta main. Ils se sont mis à tes pieds pour recevoir tes paroles.
\VS{4}Moïse nous a donné la loi, héritage de l'assemblée de Jacob.
\VS{5}Il était roi de Jeshurun\FTNT{Voir commentaire en De. 32:15.}, quand les chefs du peuple s'assemblaient ensemble, avec les tribus d'Israël.
\VS{6}Que Ruben vive et qu'il ne meure point, encore que ses hommes soient en petit nombre.
\VS{7}Et voici ce qu'il dit pour Juda~: Ô Yahweh~! Ecoute la voix de Juda, et ramène-le vers son peuple. Que ses mains soient puissantes, et sois-lui en aide contre ses ennemis.
\VS{8}Il dit aussi touchant Lévi~: Tes thummim et tes urim sont à l'homme fidèle que tu as éprouvé à Massa, et avec qui tu as contesté aux eaux de Meriba.
\VS{9}Il dit de son père et de sa mère~: Je ne les ai point vus~! Il ne reconnait point ses frères, et ne connait point ses fils. Car ils gardent tes paroles, et ils gardent ton alliance.
\VS{10}Ils enseignent tes ordonnances à Jacob, et ta loi à Israël~; ils mettent l'encens sous tes narines, et l'holocauste sur ton autel.
\VS{11}Ô Yahweh, bénis sa force~! Agrée l'œuvre de ses mains~! Brise les reins de ceux qui s'élèvent contre lui, et que ceux qui le haïssent ne se relèvent plus~!
\VS{12}Il dit de Benjamin~: Le bien-aimé de Yahweh habitera en sécurité avec lui~; il le protégera toujours, et demeurera entre ses épaules.
\VS{13}Il dit de Joseph~: Son pays est béni par Yahweh, de ce qu'il y a de plus précieux au ciel, de la rosée, et de l'abîme qui est en bas,
\VS{14}et du plus précieux des produits du soleil, et du plus précieux des produits de la lune, 
\VS{15}et de ce qui croît sur le sommet des montagnes d'ancienneté, du plus précieux des collines éternelles,
\VS{16}et du plus précieux de la terre et de sa plénitude. Que la grâce de celui qui demeura dans le buisson vienne sur la tête de Joseph, sur le sommet, sur le sommet de la tête de celui qui est consacré d'entre ses frères~!
\VS{17}Sa majesté est comme le premier-né de son taureau~; et ses cornes comme les cornes du buffle~; il poussera tous les peuples ensemble jusqu'aux extrémités de la terre~: Ce sont les dix milliers d'Ephraïm, et ce sont les milliers de Manassé.
\VS{18}Il dit de Zabulon~: Réjouis-toi, Zabulon, dans ta sortie, et toi, Issacar, dans tes tentes.
\VS{19}Ils appelleront les peuples sur la montagne, ils y offriront des sacrifices de justice, car ils suceront l'abondance des mers, et les trésors cachés dans le sable.
\VS{20}Il dit aussi de Gad~: Béni soit celui qui élargit Gad~! Il habite comme un lion, et il déchire le bras et la tête.
\VS{21}Il a choisi les prémices, parce que c'était là qu'était cachée la portion du législateur, et il est venu en tête du peuple~; il a exécuté la justice de Yahweh et ses jugements envers Israël.
\VS{22}Et il dit de Dan~: Dan est un jeune lion, il s'élance de Basan.
\VS{23}Il dit de Nephthali~: Nephthali, rassasié de faveur, et rempli de la bénédiction de Yahweh, possède l'occident et le sud.
\VS{24}Il dit aussi d'Aser~: Aser sera béni entre les fils~; il sera agréable à ses frères, et il trempera son pied dans l'huile.
\VS{25}Tes verrous seront de fer et d'airain, et ta force durera autant que tes jours.
\VS{26}Nul n'est comme le Dieu de Jeshurun\FTNT{Voir commentaire en De. 32:15.}, porté sur les cieux pour te venir en aide, et sur les nuées dans sa majesté.
\VS{27}Le Dieu d'éternité est un refuge, et au-dessous de toi sont ses bras éternels~; car il a chassé de devant toi tes ennemis, et il a dit~: Extermine.
\VS{28}Israël donc habitera en sécurité, la source de Jacob est à part dans un pays de blé et de vin, et ses cieux distilleront la rosée.
\VS{29}Ô que tu es heureux, Israël~! Qui est le peuple semblable à toi, qui ait été sauvé par Yahweh, le bouclier de ton secours et l'épée de ta majesté~? Tes ennemis dissimuleront devant toi, et tu fouleras de tes pieds leurs lieux élevés.
\Chap{34}
\TextTitle{Moïse voit le pays mais n'y entre pas}
\VerseOne{}Alors Moïse monta des plaines de Moab sur le mont Nebo, au sommet du Pisga, vis-à-vis de Jéricho. Et Yahweh lui fit voir tout le pays~: De Galaad jusqu'à Dan,
\VS{2}tout Nephthali, le pays d'Ephraïm et de Manassé, tout le pays de Juda, jusqu'à la Mer Occidentale,
\VS{3}le sud, les environs du Jourdain, la plaine de Jéricho, la ville des palmiers, jusqu'à Tsoar.
\VS{4}Yahweh lui dit~: C'est ici le pays que j'ai juré à Abraham, à Isaac, et à Jacob, en disant~: Je le donnerai à ta postérité. Je te l'ai fait voir de tes yeux~; mais tu n'y entreras point.
\TextTitle{Mort de Moïse}
\VS{5}Ainsi Moïse, serviteur de Yahweh, mourut là, dans le pays de Moab, selon la parole de Yahweh.
\VS{6}Et il l'ensevelit dans la vallée, au pays de Moab, vis-à-vis de Beth-Peor. Personne n'a connu son sépulcre jusqu'à aujourd'hui\FTNT{Jud. 1:9.}.
\VS{7}Or Moïse était âgé de cent vingt ans quand il mourut~; sa vue n'était point affaiblie, et sa vigueur n'était point passée.
\VS{8}Les enfants d'Israël pleurèrent Moïse trente jours dans les plaines de Moab~; et ces jours de pleurs et de deuil sur Moïse furent accomplis.
\TextTitle{Josué, successeur de Moïse}
\VS{9}Et Josué, fils de Nun, fut rempli de l'Esprit de sagesse, parce que Moïse lui avait imposé les mains\FTNT{Jos. 1:9.}. Les enfants d'Israël lui obéirent, et firent ce que Yahweh avait ordonné à Moïse.
\VS{10}Et il ne s'est plus levé en Israël de prophète comme Moïse, que Yahweh connaissait face à face.
\VS{11}Selon tous les signes et les miracles que Yahweh l'envoya faire au pays d'Egypte, devant Pharaon, et tous ses serviteurs, et tout son pays,
\VS{12}et selon toute cette main forte, et tous ces terribles prodiges, que Moïse fit sous les yeux de tout Israël.
\PPE{}
\end{multicols}

%\addcontentsline{toc}{section}{Nevi'im (Prophètes)}\clearpage
%\clearpage\ShortTitle{Josué}\BookTitle{Josué}\BFont
\noindent\hrulefill
{\footnotesize
\textit{
\bigskip
{\centering{}
\\Auteur : Probablement Josué
\\(Heb. : Yehowshuwa)
\\Signification : Yahweh est salut
\\Thème : La conquête de Canaan
\\Date de rédaction : 14\up{ème} siècle av. J.-C.\\}
}
%\bigskip
\textit{
\\Né en Egypte, Josué, fils de Nun, originaire de la tribu d'Ephraïm, servit Moïse de la sortie d'Egypte jusqu'à sa mort. Choisi par Dieu pour succéder au prophète, il fut le seul de l'ancienne génération, avec Caleb, à avoir survécu à la longue épreuve du désert. Ce livre relate les étapes du voyage du peuple et sa conquête de la terre promise. Il présente par ailleurs les victoires acquises par la puissance de Yahweh sous la conduite de Josué. C'est l'histoire de la prise de Canaan et de son partage aux douze tribus d'Israël.\bigskip
}
}
\par\nobreak\noindent\hrulefill
\begin{multicols}{2}
\Chap{1}
\TextTitle{Josué succède à Moïse à sa mort\FTNTT{De. 34:9.}}
\VerseOne{}Or, il arriva après la mort de Moïse, serviteur de Yahweh, que Yahweh parla à Josué, fils de Nun, qui avait servi Moïse, en disant :
\VS{2}Moïse, mon serviteur est mort ; maintenant donc, lève-toi, passe ce Jourdain, toi et tout ce peuple, pour entrer dans le pays que je donne aux enfants d'Israël.
\VS{3}Tout lieu que foulera la plante de votre pied, je vous l'ai donné, comme je l'ai déclaré à Moïse\FTNT{De. 11:24.}.
\VS{4}Vos frontières seront depuis ce désert et le Liban, jusqu'au grand fleuve, le fleuve de l'Euphrate, tout le pays des Héthiens jusqu'à la grande mer, vers le soleil couchant.
\VS{5}Nul ne tiendra devant toi, tous les jours de ta vie. Je serai avec toi comme j'ai été avec Moïse ; je ne te délaisserai point, et je ne t'abandonnerai point\FTNT{De. 31:6 ; Hé. 13:5-6.}.
\VS{6}Fortifie-toi et prends courage, car c'est toi qui mettras ce peuple en possession du pays dont j'ai juré à leurs pères de leur donner.
\VS{7}Seulement fortifie-toi et renforce-toi de plus en plus, afin que tu prennes garde de faire selon toute la loi que Moïse mon serviteur t'a ordonnée ; ne t'en détourne point ni à droite ni à gauche, afin que tu prospères partout où tu iras.
\VS{8}Que ce livre de la loi ne s'éloigne point de ta bouche, mais médite-le jour et nuit, pour agir fidèlement selon tout ce qui y est écrit\FTNT{La clé d'une vie chrétienne épanouie est la Parole de Dieu. Méditer signifie : 
\\- Murmurer la Parole de Dieu : Partout où nous sommes, nous pouvons dans nos cœurs murmurer les promesses du Seigneur (Ps. 63:5-8 ; Ps. 119:11).
\\- Proclamer à haute voix : Il est intéressant de noter que le mot hébreu traduit dans Jos. 1:8 par méditer est traduit par « proclamer » ou « dire » dans Pr. 8:7 ; Ps. 35:8 ; Ps. 77:13. 
\\- Réfléchir profondément : Il faut être dans le lieu secret (Mt. 6:5-6). En Israël il est de coutume d'aller étudier la Torah à l'ombre d'un figuier. Voir Jn. 1:43-51.} ; car c'est alors que tu auras du succès dans tes entreprises, c'est alors que tu réussiras.
\VS{9}Ne t'ai-je pas donné cet ordre, fortifie-toi et prends courage ? Ne t'épouvante point et ne t'effraie point ; car Yahweh ton Dieu est avec toi partout où tu iras.
\TextTitle{Josué prend la direction du peuple}
\VS{10}Après cela, Josué donna cet ordre aux officiers du peuple, en disant :
\VS{11}Passez par le camp, ordonnez au peuple et dites-lui : Préparez-vous des provisions, car dans trois jours vous passerez ce Jourdain pour aller prendre possession du pays que Yahweh, votre Dieu, vous donne afin que vous le possédiez.
\VS{12}Josué parla aussi aux Rubénites, aux Gadites et à la demi-tribu de Manassé, en disant :
\VS{13}Souvenez-vous de la parole que Moïse, serviteur de Yahweh, vous a prescrite, en disant : Yahweh votre Dieu vous a accordé du repos, et vous a donné ce pays.
\VS{14}Vos femmes, vos petits-enfants, et vos bêtes resteront dans le pays que Moïse vous a donné de l'autre côté du Jourdain ; mais vous tous, hommes vaillants, vous passerez en armes devant vos frères, et vous les aiderez\FTNT{Ex. 13:18.} ;
\VS{15}jusqu'à ce que Yahweh ait accordé du repos à vos frères comme à vous, et qu'ils soient aussi en possession du pays que Yahweh, votre Dieu, leur donne. Puis vous reviendrez prendre possession du pays qui est votre propriété, et que vous a donné Moïse, serviteur de Yahweh, de l'autre côté du Jourdain, vers l'orient.
\VS{16}Ils répondirent à Josué, en disant : Nous ferons tout ce que tu nous as ordonné, et nous irons partout où tu nous enverras.
\VS{17}Nous t'obéirons comme nous avons obéi à Moïse ; seulement que Yahweh ton Dieu soit avec toi, comme il a été avec Moïse.
\VS{18}Tout homme qui sera rebelle à ton ordre, et qui n'obéira point à tes paroles dans tout ce que tu lui commanderas, sera mis à mort ; seulement, fortifie-toi, et sois courageux !
\Chap{2}
\TextTitle{Josué envoie deux espions à Jéricho ; ils sont reçus par Rahab\FTNTT{Ja. 2:25.}}
\VerseOne{}Or, Josué fils de Nun, envoya secrètement de Sittim deux hommes, pour épier secrètement le pays, et il leur dit : Allez, examinez le pays, et Jéricho. Ils partirent donc et entrèrent dans la maison d'une femme prostituée, nommée Rahab\FTNT{Rahab avait entendu parler du Dieu des Hébreux et avait placé son espérance de salut en lui (Ro. 10 :11). Par cet acte de foi, sa destinée a changé. Cette femme qui était vouée à une double condamnation du fait de sa condition de prostituée (De.23:17) et de son appartenance à une nation païenne qui devait être dévouée à la façon de l'interdit (Jos. 6), a été sauvée avec sa famille (Ac. 2:21 ; Ac. 16:31 ). Ainsi, bien des siècles plus tard, on ne la mentionnera plus comme une prostituée, mais comme une ancêtre du Sauveur et une héroïne de la foi (Mt. 1:5 ; Hé. 11 :21). Rahab est donc l'archétype des païens qui sont rentrés dans l'alliance de Dieu par la foi.}, et ils y couchèrent.
\VS{2}Alors on dit au roi de Jéricho : Voici, des hommes sont venus ici cette nuit de la part des enfants d'Israël pour explorer le pays.
\VS{3}Et le roi de Jéricho envoya dire à Rahab : Fais sortir les hommes qui sont venus chez toi et qui sont entrés dans ta maison ; car ils sont venus pour explorer tout le pays.
\VS{4}Or la femme prit les deux hommes et les cacha ; et elle dit : Il est vrai que des hommes sont venus chez moi, mais je ne savais pas d'où ils étaient ;
\VS{5}et comme on fermait la porte sur le soir, ces hommes sont sortis ; je ne sais pas où ces hommes sont allés ; poursuivez-les bien vite car vous les atteindrez.
\VS{6}Or elle les avait fait monter sur le toit et les avait cachés sous des tiges de lin qu'elle avait arrangées sur le toit.
\VS{7}Et quelques gens les poursuivirent par le chemin du Jourdain jusqu'aux passages ; et on ferma la porte après que ceux qui les poursuivaient furent sortis.
\VS{8}Or, avant qu'ils se couchent, elle monta vers eux sur le toit ;
\VS{9}et leur dit : Je sais que Yahweh vous a donné ce pays, et que la terreur de votre nom nous a saisis, et que tous les habitants du pays perdent courage à cause de vous\FTNT{Ex. 23:27.}.
\VS{10}Car nous avons entendu que Yahweh a mis à sec devant vous les eaux de la Mer Rouge à votre sortie du pays d'Egypte ; et ce que vous avez fait aux deux rois des Amoréens qui étaient de l'autre côté du Jourdain, à Sihon et à Og, que vous avez détruits complètement en les dévouant par le moyen de l'interdit.
\VS{11}Nous l'avons entendu, et notre cœur a fondu, et depuis aucun homme n'a eu le courage à cause de vous. Car Yahweh, votre Dieu, est le Dieu des cieux en haut et de la terre\FTNT{De. 4:39.} en bas.
\VS{12}Maintenant donc, je vous prie, jurez-moi par Yahweh, que puisque j'ai usé de bonté envers vous, vous userez aussi de bonté envers la maison de mon père, 
\VS{13}et que vous me donnerez un signe de votre fidélité\FTNT{La couleur cramoisi s'obtient grâce à la femelle cochenille aptère qui contient dans son corps et dans ses œufs un pigment rouge à base d'acide carminique qui permet à l'insecte et à ses larves de se protéger des prédateurs. Au moment de la ponte, cette dernière fixe fermement son corps au tronc d'un arbre puis libère ses œufs qui demeurent ainsi protégés en dessous d'elle jusqu'à leur éclosion. Ensuite, l'insecte meurt en libérant cette substance rouge qui se propage sur tout son corps et sur le bois hôte. C'est ce fluide que l'homme récupère pour en faire un colorant à la couleur caractéristique. Une subtile analogie peut être faire entre la cochenille et le Seigneur qui a versé son sang à la croix pour nous donner la vie. « Et moi, je suis un ver, et non un homme, l'opprobre des hommes et le méprisé du peuple » (Ps. 22 :7).} d'une ferme assurance que vous laisserez vivre mon père, ma mère, mes frères, mes sœurs, et tous ceux qui leur appartiennent, et que vous sauverez nos âmes de la mort.
\VS{14}Et ces hommes lui répondirent : Nos personnes répondront pour vous jusqu'à la mort, pourvu que vous ne divulguez pas cette affaire ; et quand Yahweh nous aura donné le pays nous userons envers toi de bonté et de vérité. 
\TextTitle{Les espions s'enfuient aidés par Rahab}
\VS{15}Elle les fit donc descendre avec une corde par la fenêtre ; car sa maison était sur la muraille de la ville, et elle habitait sur la muraille de la ville. 
\VS{16}Et elle leur dit : Allez à la montagne, de peur que ceux qui vous poursuivent ne vous rencontrent, et cachez-vous là pendant trois jours jusqu'à ce qu'ils soient de retour. Après cela vous suivrez votre chemin.
\VS{17}Et ces hommes lui dirent : Voici comment nous serons quittes de ce serment que tu nous as fait faire.
\VS{18}Voici, quand nous entrerons dans le pays, tu lieras ce cordon de fil d'écarlate à la fenêtre par laquelle tu nous auras fait descendre, et tu recueilleras chez toi, dans cette maison, ton père et ta mère, tes frères, et toute la famille de ton père.
\VS{19}Et quiconque sortira hors de la porte de ta maison, son sang sera sur sa tête, et nous en serons quittes ; mais quiconque sera avec toi, dans la maison, son sang sera sur notre tête si quelqu'un met la main sur lui.
\VS{20}Et si tu divulgues cette affaire, nous serons quittes du serment que tu nous as fait faire.
\VS{21}Et elle répondit : Que cela soit ainsi que vous l'avez dit. Alors elle les laissa aller. Ils s'en allèrent et elle lia le cordon de fil d'écarlate à la fenêtre.
\VS{22}Et ils marchèrent et arrivèrent à la montagne, où ils restèrent trois jours, jusqu'à ce que ceux qui les poursuivaient soient de retour. Ceux qui les poursuivaient les cherchèrent par tout le chemin, mais ils ne les trouvèrent pas.
\VS{23}Ainsi ces deux hommes s'en retournèrent, descendirent de la montagne, passèrent le Jourdain. Ils vinrent auprès de Josué, fils de Nun. Ils lui racontèrent toutes les choses qui leur étaient arrivées.
\VS{24}Et ils dirent à Josué : Certainement, Yahweh a livré tout le pays entre nos mains, et même tous les habitants ont perdu le courage à notre vue.
\Chap{3}
\TextTitle{Israël traverse le Jourdain à sec}
\VerseOne{}Or Josué se leva de bon matin, lui et tous les enfants d'Israël partirent de Sittim, ils vinrent jusqu'au Jourdain, et ils logèrent là cette nuit, avant de le traverser.
\VS{2}Et au bout de trois jours les officiers traversèrent le milieu du camp,
\VS{3}et donnèrent cet ordre au peuple en disant : Dès que vous verrez l'arche de l'alliance de Yahweh, votre Dieu, portée par les prêtres, les Lévites, vous partirez de votre quartier, et vous marcherez après elle.
\VS{4}Et afin que vous n'approchez pas d'elle, il y aura entre vous et elle une distance de la mesure d'environ deux mille coudées. Elle vous fera connaître le chemin par lequel vous devez marcher ; car vous n'avez pas encore passé par ce chemin.
\VS{5}Josué dit au peuple : Sanctifiez-vous, car Yahweh fera demain des choses merveilleuses au milieu de vous\FTNT{Ex. 19:10-11.}.
\VS{6}Josué parla aussi aux prêtres, en disant : Portez l'arche de l'alliance, et passez devant le peuple. Ainsi ils portèrent l'arche de l'alliance, et marchèrent devant le peuple.
\VS{7}Or Yahweh dit à Josué : Aujourd'hui je commencerai à t'élever aux yeux de tout Israël, afin qu'ils sachent que je serai aussi avec toi, comme j'ai été avec Moïse.
\VS{8}Tu donneras cet ordre aux prêtres qui portent l'arche de l'alliance, en leur disant : Dès que vous arriverez au bord des eaux du Jourdain, vous vous arrêterez dans le Jourdain.
\VS{9}Et Josué dit aux enfants d'Israël : Approchez-vous d'ici, et écoutez les paroles de Yahweh, votre Dieu.
\VS{10}Puis Josué dit : Vous reconnaîtrez à ceci que le Dieu vivant est au milieu de vous et qu'il chassera et déshéritera devant vous les Cananéens, les Héthiens, les Héviens, les Phéréziens, les Guirgasiens, les Amoréens et les Jébusiens.
\VS{11}Voici, l'arche de l'alliance du Seigneur de toute la terre va passer devant vous dans le Jourdain.
\VS{12}Maintenant, prenez douze hommes des tribus d'Israël, un homme de chaque tribu.
\VS{13}Et il arrivera qu'aussitôt que les plantes des pieds des prêtres qui portent l'arche de Yahweh, le Seigneur de toute la terre, seront posés dans les eaux du Jourdain, les eaux du Jourdain seront coupées, les eaux, dis-je, qui descendent d'en haut, et elles s'arrêteront en un monceau\FTNT{Ps. 114:3.}.
\VS{14}Et il arriva que le peuple étant parti de ses tentes pour passer le Jourdain, et les prêtres qui portaient l'arche de l'alliance, étaient devant le peuple.
\VS{15}Aussitôt que ceux qui portaient l'arche furent arrivés au Jourdain, et que les pieds des prêtres qui portaient l'arche furent mouillés au bord de l'eau. Le Jourdain regorge par-dessus toutes ses rives durant tout le temps de la moisson\FTNT{1 Ch. 12:15}.
\VS{16}Les eaux qui descendent d'en haut, s'arrêtèrent, et s'élevèrent en un monceau, à une très grande distance, depuis la ville d'Adam, qui est à côté de Tsarthan ; et celles d'en bas, qui descendaient vers la mer de la plaine, qui est la mer salée, furent totalement coupées. Le peuple passa vis-à-vis de Jéricho.
\VS{17}Mais les prêtres qui portaient l'arche de l'alliance de Yahweh, s'arrêtèrent de pied ferme sur le sec, au milieu du Jourdain, pendant que tout Israël passait à sec, jusqu'à ce que tout le peuple ait achevé de passer le Jourdain.
\Chap{4}
\TextTitle{Josué dresse un monument de pierres en souvenir de la traversée}
\VerseOne{}Or il arriva que quand tout le peuple eut achevé de passer le Jourdain, que Yahweh parla à Josué et dit :
\VS{2}Prenez douze hommes parmi le peuple, un homme de chaque tribu.
\VS{3}et donnez-leur cet ordre, en disant : Prenez ici, du milieu du Jourdain, de la place où les prêtres se sont arrêtés de pied ferme, douze pierres, que vous emporterez avec vous, et vous les poserez au lieu où vous passerez cette nuit.
\VS{4}Josué appela les douze hommes qu'il choisit parmi les enfants d'Israël, un homme de chaque tribu.
\VS{5}Et il leur dit : Passez devant l'arche de Yahweh, votre Dieu, au milieu du Jourdain, et que chacun de vous charge une pierre sur son épaule, selon le nombre des tribus des enfants d'Israël ;
\VS{6}afin que cela soit un signe au milieu de vous. Et quand vos fils interrogeront à l'avenir leurs pères, en disant : Que signifient ces pierres-ci ?
\VS{7}Alors vous leur répondrez : Les eaux du Jourdain ont été coupées devant l'arche de l'alliance de Yahweh ; lorsqu'elle passa le Jourdain, les eaux du Jourdain ont été arrêtées ; c'est pourquoi ces pierres-là seront à jamais un souvenir pour les enfants d'Israël.
\VS{8}Les enfants d'Israël firent donc comme Josué leur avait ordonné. Ils prirent douze pierres du milieu du Jourdain, comme Yahweh l'avait ordonné à Josué, selon le nombre des tribus des enfants d'Israël. Ils les emportèrent avec eux et les posèrent au lieu où ils devaient passer la nuit.
\VS{9}Josué dressa aussi douze pierres au milieu du Jourdain, à l'endroit où les pieds des prêtres qui portaient l'arche de l'alliance s'étaient arrêtés ; et elles y sont restées jusqu'à ce jour.
\VS{10}Les prêtres donc qui portaient l'arche se tinrent debout au milieu du Jourdain, jusqu'à ce que tout ce que Yahweh avait ordonné à Josué de dire au peuple soit accompli, selon tout ce que Moïse avait prescrit à Josué. Et le peuple se hâta de passer.
\VS{11}Et quand tout le peuple eut achevé de passer, alors l'arche de Yahweh et les prêtres passèrent devant le peuple.
\VS{12}Et les fils de Ruben, les fils de Gad, et la demi-tribu de Manassé passèrent en armes devant les enfants d'Israël, comme Moïse le leur avait dit\FTNT{No. 32:20-29.}.
\VS{13}Ils passèrent, dis-je, dans les plaines de Jérico environ quarante mille hommes en équipage de guerre, devant Yahweh, pour combattre. 
\VS{14}Ce jour-là, Yahweh éleva Josué à la vue de tout Israël, et ils le craignirent, comme ils avaient craint Moïse, tous les jours de sa vie.
\VS{15}Yahweh parla à Josué, et dit :
\VS{16}Ordonne aux prêtres qui portent l'arche du témoignage qu'ils montent hors du Jourdain.
\VS{17}Et Josué donna cet ordre aux prêtres, en disant : Montez hors du Jourdain.
\VS{18}Or sitôt que les prêtres, qui portaient l'arche de l'alliance de Yahweh furent montés hors du milieu du Jourdain, et qu'ils eurent mis la plante de leurs pieds sur le sec, les eaux du Jourdain retournèrent à leur place, et coulèrent comme auparavant sur tous les rivages.
\VS{19}Le peuple donc monta hors du Jourdain le dixième jour du premier mois, et il campa à Guilgal, à l'orient de Jéricho.
\VS{20}Josué aussi dressa à Guilgal les douze pierres qu'ils avaient prises du Jourdain.
\VS{21}Et il parla aux enfants d'Israël et leur dit : Quand vos enfants interrogeront à l'avenir leurs pères, et leur diront : Que signifient ces pierres-ci ?
\VS{22}Vous l'apprendrez à vos enfants, en leur disant : Israël a passé ce Jourdain à sec.
\VS{23}Car Yahweh, votre Dieu, a fait tarir les eaux du Jourdain devant vous jusqu'à ce que vous eussiez passé, comme Yahweh, votre Dieu, l'avait fait à la Mer Rouge, qu'il mit à sec devant nous, jusqu'à ce que nous eussions passé,
\VS{24}afin que tous les peuples de la terre sachent que la main de Yahweh est puissante, et afin que vous ayez toujours la crainte de Yahweh, votre Dieu.
\Chap{5}
\TextTitle{La crainte s'empare des Amoréens}
\VerseOne{}Or il arriva qu'aussitôt que tous les rois des Amoréens qui étaient au-delà du Jourdain, vers l'occident, et tous les rois des Cananéens qui étaient près de la mer, apprirent que Yahweh avait mis à sec les eaux du Jourdain devant les enfants d'Israël, jusqu'à ce que nous eussions passé, leur cœur fut fondu, et il n'y avait plus de courage en eux à cause des enfants d'Israël.
\TextTitle{Israël circoncis à nouveau ; la fin de la manne}
\VS{2}En ce temps-là, Yahweh dit à Josué : Fais-toi des couteaux de pierre tranchants, et circoncis de nouveau les enfants d'Israël, une seconde fois.
\VS{3}Et Josué se fit des couteaux de pierre tranchants, et circoncit les enfants d'Israël sur la colline d'Araloth.
\VS{4}Or la raison pour laquelle Josué les circoncit, c'est que tout le peuple sorti d'Egypte, tous les mâles, dis-je, hommes de guerre étaient morts en chemin dans le désert, après leur sortie d'Egypte.
\VS{5}Et tout le peuple sorti d'Egypte était circoncis, mais aucun du peuple né dans le désert en chemin n'avait été circoncis, après leur sortie d'Egypte.
\VS{6}Car les enfants d'Israël avaient marché dans le désert quarante ans jusqu'à ce que soit consummée toute la nation des hommes de guerre qui étaient sortis d'Egypte, et qui n'avaient point écouté la voix de Yahweh ; auxquels Yahweh avait juré qu'il ne leur laisserait point voir le pays qu'il avait juré à leurs pères de nous donner, pays où coulent le lait et le miel\FTNT{No. 14:32-33.}.
\VS{7}Et il a suscité à leur place leurs enfants que Josué circoncit, parce qu'ils étaient incirconcis ; car on ne les avait pas circoncis pendant le voyage.
\VS{8}Et quand on eut achevé de circoncire tout le peuple, ils restèrent dans leur camp, jusqu'à ce qu'ils soient guéris.
\VS{9}Et Yahweh dit à Josué : Aujourd'hui j'ai roulé de dessus vous l'opprobre de l'Egypte. Et ce lieu-là fut appelé Guilgal jusqu'à ce jour.
\VS{10}Ainsi les enfants d'Israël campèrent à Guilgal, et célébrèrent la Pâque le quatorzième jour du mois, sur le soir, dans les plaines de Jéricho\FTNT{Ex. 12:6.}.
\VS{11}Et dès le lendemain de la Pâque, ils mangèrent du blé du pays, savoir, des pains sans levain et du grain rôti, en ce même jour\FTNT{Ex. 12:39 ; Lé. 2:14}.
\VS{12}Et la manne cessa dès le lendemain de la Pâque, après qu'ils eurent manger du blé du pays ; les enfants d'Israël n'eurent plus de manne, mais ils mangèrent les récoltes de la terre de Canaan cette année-là\FTNT{Ex. 16:35.}.
\TextTitle{Rencontre avec le chef de l'armée de Yahweh}
\VS{13}Or il arriva, comme Josué était près de Jéricho, qu'il leva les yeux et regarda. Voici, un homme qui avait son épée nue à la main, se tenait debout devant lui. Josué alla vers lui et lui dit : Es-tu des nôtres ou de nos ennemis ?
\VS{14}Et il répondit : Non, mais je suis le Chef de l'armée de Yahweh, je viens maintenant. Josué tomba à terre sur son visage, l'adora, et lui dit : Qu'est-ce que mon Seigneur dit à son serviteur ?
\VS{15}Et le Chef de l'armée de Yahweh dit à Josué : Délie tes souliers de tes pieds ; car le lieu sur lequel tu te tiens est saint\FTNT{Ex. 3:5.}. Et Josué fit ainsi.
\Chap{6}
\TextTitle{Jéricho miraculeusement livré à Israël ; Rahab sauvée}
\VerseOne{}Or Jéricho était barricadée et fermée soigneusement, à cause des enfants d'Israël. Personne ne sortait, et personne n'entrait.
\VS{2}Et Yahweh dit à Josué : Regarde, j'ai livré entre tes mains Jéricho et son roi, ses hommes vaillants.
\VS{3}Vous tous donc, hommes de guerre, vous ferez le tour de la ville, en tournant une fois autour d'elle. Tu feras ainsi durant six jours.
\VS{4}Et Sept prêtres porteront sept shofars retentissants devant l'arche. Mais au septième jour, vous ferez sept fois le tour de la ville et les prêtres sonneront des shofars.
\VS{5}Et quand ils sonneront avec la corne de bélier, aussitôt que vous entendrez le son du shofar retentissant, tout le peuple poussera un grand cri de joie et la muraille de la ville tombera sur elle. Et le peuple montera, les hommes devant lui.
\VS{6}Josué donc, fils de Nun, appela les prêtres et leur dit : Portez l'arche de l'alliance et que sept prêtres portent sept shofars devant l'arche de Yahweh.
\VS{7}Il dit aussi au peuple : Passez et faites le tour de la ville, que tous ceux qui seront armés passent devant l'arche de Yahweh.
\VS{8}Et quand Josué eut parlé au peuple, les sept prêtres qui portaient les sept cornes de béliers devant Yahweh passèrent et sonnèrent des cornes. Et l'arche de l'alliance de Yahweh les suivait.
\VS{9}Et les hommes qui étaient armés marchaient devant les prêtres qui sonnaient des shofars ; mais l'arrière-garde suivait derrière l'arche ; on sonnait des shofars en marchant.
\VS{10}Or Josué avait donné cet ordre au peuple, en disant : Vous ne pousserez point de cris de joie et vous ne ferez point entendre votre voix. Et il ne sortira point un seul mot de votre bouche, jusqu'au jour où je vous dirai : Poussez des cris de joie ! Alors vous crierez.
\VS{11}L'arche de Yahweh fit ainsi le tour de la ville, en tournant tout autour une fois, puis on revint au camp, et on y passa la nuit.
\VS{12}Ensuite Josué se leva de bon matin, et les prêtres portèrent l'arche de Yahweh.
\VS{13}Et les sept prêtres qui portaient les sept cornes de bélier devant l'arche de Yahweh se mirent en marche et sonnèrent du shofar. Et les hommes armés allaient devant eux ; puis l'arrière-garde suivait l'arche de Yahweh ; on sonnait des shofars en marchant.
\VS{14}Ainsi ils firent une fois le tour de la ville le deuxième jour, et ils retournèrent au camp. Ils firent de même durant six jours.
\VS{15}Mais quand le septième jour fut venu, ils se levèrent dès le matin à l'aube du jour, et ils firent sept fois le tour de la ville de la même manière ; ce fut le seul jour où ils firent sept fois le tour de la ville.
\VS{16}Et à la septième fois, comme les prêtres sonnaient des shofars, Josué dit au peuple : Poussez des cris de joie, car Yahweh vous a donné la ville !
\VS{17}La ville sera dévouée par le moyen de l'interdit à Yahweh, elle et toutes les choses qui y sont ; seulement Rahab, la prostituée\FTNT{Rahab sauva sa famille par sa foi en Dieu (Ac. 16:31). Voir Josué 2.}, vivra, elle et tous ceux qui seront avec elle dans la maison, parce qu'elle a caché soigneusement les messagers que nous avions envoyés.
\VS{18}Mais quoi qu'il en soit gardez-vous de l'interdit, de peur que vous ne vous mettiez en interdit, et que vous ne mettriez le camp d'Israël en interdit et que vous le troubliez\FTNT{De. 7:26.}.
\VS{19}Mais tout l'argent et tout l'or, tous les objets d'airain et de fer seront consacrés à Yahweh, ils entreront dans le trésor de Yahweh\FTNT{No. 31:54.}.
\VS{20}Le peuple donc poussa de cris de joie et on sonna des shofars. Et quand le peuple entendit le son des shofars, il poussa de grands cris de joie et la muraille tomba sur elle-même\FTNT{Hé. 11:30.}. Alors le peuple monta dans la ville, les hommes devant le peuple. Et ils prirent la ville. 
\VS{21}Et ils la dévouèrent entièrement par le moyen de l'interdit, et passèrent au fil de l'épée tout ce qui était dans la ville, depuis l'homme jusqu'à la femme, depuis l'enfant jusqu'au vieillard, même jusqu'aux bœufs, aux brebis et aux ânes.
\VS{22}Mais Josué dit aux deux hommes qui avaient espionné le pays : Entrez dans la maison de cette femme prostituée, et faites-la sortir de là, avec tous ceux qui lui appartiennent, selon que vous lui avez juré.
\VS{23}Les jeunes hommes donc qui avaient espionné le pays, entrèrent et firent sortir Rahab, et son père, et sa mère et ses frères, avec tous ceux qui lui appartenaient ; ils firent aussi sortir toutes les familles qui lui appartenaient, et les mirent hors du camp d'Israël.
\VS{24}Puis ils allumèrent le feu et brûlèrent la ville et tout ce qui s'y trouvait ; seulement ils mirent l'argent et l'or, les objets d'airain et de fer dans le trésor de la maison de Yahweh.
\VS{25}Ainsi Josué sauva la vie à Rahab la prostituée, la maison de son père, et tous ceux qui lui appartenaient ; et elle a habité au milieu d'Israël jusqu'à ce jour, parce qu'elle avait caché les messagers que Josué avait envoyés pour explorer Jéricho.
\VS{26}Et en ce temps-là Josué jura, en disant : Maudit soit devant Yahweh l'homme qui se mettra à rebâtir cette ville de Jéricho ! Il la fondera sur son premier-né, et il posera ses portes sur son plus jeune fils\FTNT{Cette parole s'est accomplie en 1 R. 16:34.}.
\VS{27}Yahweh fut avec Josué, et sa renommée se répandit dans tout le pays.
\Chap{7}
\TextTitle{Israël battu à Aï suite au péché d'Acan}
\VerseOne{}Mais les enfants d'Israël se rendirent coupables au sujet de l'interdit. Car Acan, fils de Carmi, fils de Zabdi, fils de Zérach, de la tribu de Juda, prit de l'interdit, et la colère de Yahweh s'enflamma contre les enfants d'Israël.
\VS{2}Car Josué envoya de Jéricho des hommes vers Aï, qui est près de Beth-Aven, à l'orient de Béthel. Il leur parla, et dit : Montez, et reconnaissez le pays. Ces hommes donc montèrent et reconnurent Aï.
\VS{3}Et étant retournés vers Josué, ils lui dirent : Que tout le peuple n'y monte point mais qu'environ deux mille ou trois mille hommes y montent, et ils battront Aï. Ne fatigue pas tout le peuple en l'envoyant là, car ils sont en petit nombre.
\VS{4}Ainsi, environ trois mille hommes du peuple y montèrent, mais ils s'enfuirent devant les gens d'Aï.
\VS{5}Et les gens d'Aï leur tuèrent environ trente-six hommes ; car ils les poursuivirent depuis la porte jusqu'à Schebarim, et les battirent à la descente. Le cœur du peuple se fondit et devint comme de l'eau.
\VS{6}Alors Josué déchira ses vêtements, et se jeta sur le visage contre terre, devant l'arche de Yahweh jusqu'au soir, lui et les anciens d'Israël, et ils jettèrent de la poussière sur leur tête.
\VS{7}Et Josué dit : Helas ! Seigneur Yahweh, pourquoi as-tu fait si magnifiquement passer le Jourdain à ce peuple, pour nous livrer entre les mains des Amoréens, et nous faire périr ? Oh ! Que n'avons-nous eu dans l'esprit de demeurer de l'autre côté du Jourdain !
\VS{8}Hélas ! Seigneur, que dirai-je, puisqu'Israël a tourné le dos devant ses ennemis ?
\VS{9}Les Cananéens et tous les habitants du pays l'entendront ; ils nous envelepperont, et ils retrancheront notre nom de dessus la terre. Et que feras-tu à ton grand Nom ?
\VS{10}Alors Yahweh dit à Josué : Lève-toi ! Pourquoi te jettes-tu ainsi le visage contre terre ?
\VS{11}Israël a péché ; ils ont transgressé mon alliance que je leur avais prescrite, même ils ont pris de l'interdit, même ils en ont dérobé, même ils ont menti, et même ils l'ont caché parmi leurs objets\FTNT{Il est impossible de remporter une victoire contre Satan en ayant avec soi des choses qui lui appartiennent (Jn. 14:30). Celui qui pèche est du diable nous dit la Parole de Dieu (1 Jn. 3:4-10). Les grandes victoires sont remportées par ceux qui se sanctifient et invoquent le Nom de Jésus-Christ.}.
\VS{12}C'est pourquoi les enfants d'Israël ne pourront subsister devant leurs ennemis ; ils tourneront le dos devant leurs ennemis ; car ils sont devenus un interdit. Je ne serai plus avec vous si vous ne détruisez pas l'interdit du milieu de vous.
\VS{13}Lève-toi, sanctifie le peuple, et dis : Sanctifiez-vous pour demain ; car ainsi parle Yahweh, le Dieu d'Israël : Il y a de l'interdit au milieu de toi, Israël ! Tu ne pourras subsister et faire face à tes ennemis jusqu'à ce que vous ayez ôté l'interdit du milieu de vous.
\VS{14}Vous vous approcherez donc le matin selon vos tribus ; et la tribu que Yahweh aura saisi s'approchera selon les familles, et la famille que Yahweh aura saisie s'approchera selon les maisons, et la maison que Yahweh aura saisie s'approchera selon les hommes.
\VS{15}Alors celui qui aura été saisi avec l'interdit sera brûlé au feu, lui et tout ce qui lui appartient parce qu'il a transgressé l'alliance de Yahweh, et qu'il a commis une infamie en Israël.
\VS{16}Josué donc se leva de bon matin, et fit approcher Israël selon ses tribus, et la tribu de Juda fut saisie.
\VS{17}Puis il fit approcher les familles de Juda, et la famille de Zérach fut saisie. Puis il fit approcher les hommes de la famille de ceux qui étaient descendants de Zérach, et Zabdi fut saisie.
\VS{18}Et quand il fit approcher la maison de Zabdi par hommes, Acan fils de Carmi, fils de Zabdi, fils de Zérach, de la tribu de Juda, fut saisi.
\VS{19}Josué dit à Acan : Mon fils, je te prie donne gloire à Yahweh, le Dieu d'Israël, et fais-lui confession. Déclare-moi je te prie ce que tu as fait, ne me le cache point.
\VS{20}Et Acan répondit à Josué, et dit : J'ai péché il est vrai, contre Yahweh, le Dieu d'Israël, et voici ce que j'ai fait.
\VS{21}J'ai vu parmi le butin un beau manteau de Schinear\FTNT{Ge. 10:6-12.}, deux cents sicles d'argent et un lingot d'or du poids de cinquante sicles ; je les ai convoités, je les ai pris et voilà, ces choses sont cachées dans la terre au milieu de ma tente, et l'argent est sous le manteau.
\VS{22}Alors Josué envoya des messagers qui coururent à cette tente ; et voici, le manteau était caché dans la tente d'Acan, et l'argent sous le manteau.
\VS{23}Ils les tirèrent donc du milieu de la tente et les apportèrent à Josué et à tous les enfants d'Israël, et ils les déposèrent devant Yahweh.
\VS{24}Alors Josué et tout Israël avec lui, prirent Acan, fils de Zérach, l'argent, le manteau, le lingot d'or, ses fils et ses filles, ses bœufs, ses ânes et ses brebis, sa tente et tout ce qui lui appartenait, et ils les firent monter dans la vallée d'Acor.
\VS{25}Et Josué dit : Pourquoi nous as-tu troublés ? Yahweh te troublera aujourd'hui. Et tout Israël le lapida avec des pierres, et les brûlèrent au feu, après les avoir lapidés avec des pierres.
\VS{26}Et ils dressèrent sur lui un grand monceau de pierres, qui dure jusqu'à ce jour. Et Yahweh apaisa l'ardeur de sa colère. C'est pourquoi ce lieu-là a été appelé jusqu'à aujourd'hui, la vallée d'Acor\FTNT{2 S. 18:17.}.
\Chap{8}
\TextTitle{Victoire d'Israël à Aï}
\VerseOne{}Puis Yahweh dit à Josué : Ne crains point, et ne t'effraie de rien\FTNT{De. 1:21 ; De. 7:18.} ! Prends avec toi tout le peuple propre à la guerre et lève-toi, et monte contre Aï. Regarde, j'ai livré entre tes mains le roi d'Aï et son peuple, sa ville et son pays.
\VS{2}Et tu traiteras Aï et son roi, comme tu as fais Jéricho et son roi : Seulement vous pillerez pour vous le butin et les bêtes. Place des gens en embuscade derrière la ville.
\VS{3}Josué donc se leva avec tout le peuple propre à la guerre, pour monter contre Aï. Josué choisit trente mille vaillants hommes armés, et les envoya de nuit.
\VS{4}Et il leur donna cet ordre en disant : Voyez, vous qui serez en embuscade derrière la ville ; ne vous éloignez pas beaucoup de la ville, mais tenez-vous prêts.
\VS{5}Et moi et tout le peuple qui est avec moi, nous nous approcherons de la ville. Et quand ils sortiront à notre rencontre, comme ils ont fait la première fois, nous nous enfuirons devant eux.
\VS{6}Ainsi ils sortiront après nous, jusqu'à ce que nous les ayons attirés hors de la ville ; car ils diront : Ils fuient devant nous comme la première fois ; parce que nous fuirons devant eux.
\VS{7}Alors vous vous lèverez de l'embuscade, et vous vous saisirez de la ville ; car Yahweh, votre Dieu, la livrera entre vos mains.
\VS{8}Et quand vous aurez pris la ville, vous y mettrez le feu ; vous agirez selon la parole de Yahweh. Regardez, je vous l'ai ordonné.
\VS{9}Josué donc les envoya, et ils allèrent se mettre en embuscade, et se tinrent entre Béthel et Aï, à l'occident d'Aï. Mais Josué resta cette nuit-là au milieu du peuple.
\VS{10}Puis Josué se leva de bon matin, et dénombra le peuple ; et il monta lui et les anciens d'Israël, devant le peuple vers Aï.
\VS{11}Et tout le peuple propre à la guerre qui étaient avec lui, monta et s'approcha ; et ils vinrent en face de la ville et campèrent du côté du nord d'Aï ; et la vallée était entre lui et Aï.
\VS{12}Il prit aussi environ cinq mille hommes, et les mit en embuscade entre Béthel et Aï, à l'occident de la ville.
\VS{13}Après que tout le camp eut pris position au nord de la ville, et l'embuscade à l'occident de la ville, cette nuit-là, Josué s'avança au milieu de la vallée.
\VS{14}Or il arriva qu'aussitôt que le roi de Aï l'eut vu, les hommes de la ville se hâtèrent, et se levèrent de bon matin, et au temps marqué, le Roi et tout son peuple sortirent à la campagne contre Israël pour le combattre. Or il ne savait pas qu'il y eût des gens en embuscade contre lui derrière la ville.
\VS{15}Alors Josué et tout Israël feignirent d'être battus devant eux, et ils s'enfuirent par le chemin du désert.
\VS{16}Alors tout le peuple qui était dans la ville d'Aï, fut assemblé à grand cri pour les poursuivre. Ils poursuivirent Josué, et ils furent ainsi attirés loin de la ville.
\VS{17}Il ne resta pas un seul homme dans Aï ni dans Béthel qui ne sortit contre Israël. Ils laissèrent la ville ouverte, et ils poursuivirent Israël.
\VS{18}Alors Yahweh dit à Josué : Etends vers Aï l'étandard qui est dans ta main, car je la livrerai entre tes mains. Et Josué étendit vers la ville l'étandard qui était dans sa main.
\VS{19}Aussitôt qu'il eut étendu sa main, les hommes qui étaient en embuscade se levèrent précipitamment du lieu où ils étaient ; ils pénétrèrent dans la ville, la prirent, et se hâtèrent de mettre le feu dans la ville.
\VS{20}Et les gens d'Aï, se tournant derrière eux, regardèrent ; et voici, la fumée de la ville montait vers le ciel, et ils n'y eut en eux aucune force pour fuir ça ou là. Et le peuple qui fuyait vers le désert se tourna contre ceux qui le poursuivaient ;
\VS{21}Et Josué et tout Israël, voyant que ceux qui étaient en embuscade avaient pris la ville, et que la fumée de la ville montait, se retournèrent, et frappèrent les gens d'Aï.
\VS{22}Les autres aussi sortirent de la ville contre eux, et ils furent enveloppés par les Israélites ayant les uns d'un côté et les autres de l'autre. Ils furent tellement battus qu'il n'en laissa aucun qui resta en vie ou qui échappât\FTNT{De 7:2.} ;
\VS{23}ils prirent aussi vivant le roi d'Aï, et le présentèrent à Josué.
\VS{24}Et quand les Israélites eurent achevé de tuer tous les habitants d'Aï dans la campagne, dans le désert, où ils les avaient poursuivis, et que tous furent tombés sous le tranchant de l'épée, jusqu'à être entièrement défaits, tous les Israélites revinrent vers Aï, et la frappèrent au tranchant de l'épée.
\VS{25}Et tous ceux qui tombèrent ce jour-là, tant des hommes que des femmes, furent au nombre de douze mille, tous gens d'Aï.
\VS{26}Et Josué ne retira point sa main qu'il tenait étendue avec l'étandard, jusqu'à ce que tous les habitants d'Aï aient été entièrement dévoués par le moyen de l'interdit.
\VS{27}Seulement les Israélites pillèrent pour eux les bêtes et le butin de cette ville-là, suivant ce que Yahweh avait prescrit à Josué\FTNT{No. 31:22-26.}.
\VS{28}Josué donc brûla Aï, et en fit un monceau perpétuel de ruines, jusqu'à aujourd'hui.
\VS{29}Puis il fit pendre le roi d'Aï à un arbre jusqu'au temps du soir. Et comme le soleil se couchait, Josué ordonna qu'on descende de l'arbre son cadavre ; on le jeta à l'entrée de la porte de la ville, puis on dressa sur lui un grand amas de pierres, qui subsiste encore aujourd'hui.
\TextTitle{Sacrifices offerts à Yahweh et lecture de la loi de Moïse}
\VS{30}Alors Josué bâtit un autel à Yahweh, le Dieu d'Israël, sur la montagne d'Ebal,
\VS{31}comme Moïse, serviteur de Yahweh, l'avait ordonné aux enfants d'Israël, ainsi qu'il est écrit dans le livre de la loi de Moïse : Il fit cet autel de pierres brutes sur lesquelles personne ne porta le fer\FTNT{L'autel devait être construit avec des pierres taillées par Dieu lui-même dans la nature (Ex. 20:25). L'Eglise du Seigneur est construite avec des pierres vivantes, taillées par Dieu et non par les hommes (Mt. 16:18). Babylone est construite avec des briques, œuvre des hommes (Ge. 11:1-3).} ; et ils offrirent dessus des holocaustes à Yahweh, et sacrifièrent des sacrifices d'offrande de paix\FTNT{Voir commentaire en Lé. 3:1.}.
\VS{32}Il écrivit aussi là, sur les pierres une copie de la loi que Moïse avait mise par écrit devant les enfants d'Israël.
\VS{33}Et tout Israël, ses anciens, ses officiers et ses juges étaient des deux côtés de l'arche, en face des prêtres qui sont de la race de Lévi, qui portaient l'arche de l'alliance de Yahweh, les étrangers comme les Hébreux naturels, une moitié du côté du mont Garizim\FTNT{Voir Jn. 4:19-24.}, et l'autre moitié du côté du mont Ebal, selon l'ordre qu'avait précédemment donné Moïse, serviteur de Yahweh, de bénir le peuple d'Israël.
\VS{34}Et après cela, il lut tout haut toutes les paroles de la loi, tant les bénédictions que les malédictions, selon tout ce qui est écrit dans le livre de la loi.
\VS{35}Il n'y eut rien de tout ce que Moïse avait prescrit, que Josué ne lise tout haut devant toute l'assemblée d'Israël, des femmes et des petits-enfants, et des étrangers qui marchaient au milieu d'eux.
\Chap{9}
\TextTitle{Josué tombe dans la ruse des Gabaonites}
\VerseOne{}Or, dès que tous les rois qui étaient au-delà du Jourdain, dans la montagne et dans la plaine, et sur toute la côte de la grande mer, jusque près du Liban, les Héthiens, les Amoréens, les Cananéens, les Phéréziens, les Héviens et les Jébusiens, eurent appris ces choses,
\VS{2}ils s'assemblèrent tous d'un commun accord pour faire la guerre à Josué et à Israël.
\VS{3}Mais les habitants de Gabaon\FTNT{Les Gabaonites étaient rusés. Ils poussèrent les Hébreux à faire alliance avec eux, comme le font les faux chrétiens aujourd'hui (Esd. 4 ; Es. 30:1). Il n'y a pas de rapport entre la lumière et les ténèbres (2 Co. 6:14-18). Combien de chrétiens ne se font-ils pas avoir par des loups ravisseurs dans le domaine du mariage ?}, ayant entendu ce que Josué avait fait à Jéricho et à Aï,
\VS{4}usèrent de ruse, car ils se mirent en chemin et contrefirent les ambassadeurs et prirent de vieux sacs pour leurs ânes, et de vieilles outres de vin déchirées et recousues,
\VS{5}Et ils avaient à leurs pieds de vieux souliers raccommodés et de vieux habits sur eux ; et tout le pain qu'ils avaient pour nourriture était sec et moisi.
\VS{6}Et ils arrivèrent auprès de Josué au camp de Guilgal, et lui dirent, ainsi qu'à tous les hommes d'Israël : Nous sommes venus d'un pays éloigné, maintenant donc traitez alliance avec nous.
\VS{7}Et les hommes d'Israël répondirent à ces Héviens : Peut-être que vous habitez au milieu de nous, et comment traiterions-nous alliance avec vous ?
\VS{8}Mais ils dirent à Josué : Nous sommes tes serviteurs. Alors Josué leur dit : Qui êtes-vous ? Et d'où venez-vous ?
\VS{9}Ils lui répondirent : Tes serviteurs sont venus d'un pays très éloigné, sur la renommée de Yahweh, ton Dieu ; car nous avons entendu sa renommée, et toutes les choses qu'il a faites en Egypte,
\VS{10}et tout ce qu'il a fait aux deux rois des Amoréens, qui étaient au-delà du Jourdain, Sihon, roi de Hesbon, et Og, roi de Basan, qui demeurait à Aschtaroth.
\VS{11}Et nos anciens et tous les habitants de notre pays nous ont dit : Prenez avec vous des provisions pour le chemin, et allez au-devant d'eux, et dites-leur : Nous sommes vos serviteurs, et maintenant traitez alliance avec nous.
\VS{12}Voci notre pain : Nous l'avons pris dans nos maisons tout chaud pour notre provision, le jour où nous sommes partis pour venir vers vous, mais maintenant voici, il est devenu sec et moisi.
\VS{13}Et voici aussi les outres de vin neuves que nous avons remplies, elles se sont déchirées ; nos habits et nos souliers sont usés à cause de la longueur de la marche.
\VS{14}Les hommes d'Israël prirent de leur provision, et aucun d'eux ne consulta la bouche de Yahweh\FTNT{Josué et les chefs ne consultèrent pas Yahweh avant de traiter alliance avec les Gabaonites. Prenez le temps dans la prière afin de connaître le cœur de la personne avec laquelle vous voulez marcher.}.
\VS{15}Car Josué fit la paix avec eux, et traita avec eux une alliance par laquelle il devait leur laisser la vie, et les chefs de l'assemblée le leur jurèrent.
\TextTitle{Les Gabaonites démasqués}
\VS{16}Mais il arriva, trois jours après l'alliance traitée avec eux, qu'ils apprirent que c'étaient leurs voisins et qu'ils habitaient parmi eux.
\VS{17}Car les enfants d'Israël partirent, et arrivèrent à leurs villes le troisième jour. Leurs villes étaient Gabaon, Kephira, Beéroth, et Kirjath-Jearim.
\VS{18}Et les enfants d'Israël ne les frappèrent point, parce que les chefs de l'assemblée leur avaient juré par Yahweh, le Dieu d'Israël. Mais toute l'assemblée murmura contre les chefs.
\VS{19}Alors tous les chefs dirent à toute l'assemblée : Nous leur avons juré par Yahweh, le Dieu d'Israël, c'est pourquoi maintenant nous ne pouvons pas les frapper.
\VS{20}Faisons-leur ceci, et qu'on les laisse vivre afin qu'il n'y ait pas de colère contre nous, à cause du serment que nous leur avons fait.
\VS{21}Ils vivront, leur dirent les chefs. Mais ils furent employés à couper le bois et à puiser l'eau pour toute l'assemblée, comme les chefs le leur avaient dit\FTNT{2 S. 21:1-14. La présence des Gabaonites en plein centre de Canaan tendait à isoler les tribus du nord de celles du sud, favorisant ainsi le schisme des deux royaumes (1 R. 12).}.
\VS{22}Car Josué les fit appeler, et leur parla, en disant : Pourquoi nous avez-vous trompés, en nous disant : Nous sommes très éloignés de vous, alors que vous habitez au milieu de nous ?
\VS{23}Maintenant vous êtes maudits ; il y aura toujours des esclaves parmi vous, des coupeurs de bois et des puiseurs d'eau pour la maison de mon Dieu.
\VS{24}Et ils répondirent à Josué, et dirent : Après qu'il ait été exactement rapporté à tes serviteurs les ordres que Yahweh, ton Dieu, avait ordonnés à Moïse, son serviteur, pour vous donner tout le pays et pour en exterminer tous les habitants devant vous ; nous avons extrêmement crains pour nos personnes à cause de vous et nous avons fait ceci. 
\VS{25}Et maintenant nous voici entre tes mains ; fais-nous comme il te semblera bon et juste de nous faire.
\VS{26}Il leur fit donc ainsi et il les délivra de la main des enfants d'Israël, de sorte qu'il ne les tuèrent point.
\VS{27}Et en ce jour-là, Josué les établit coupeurs de bois et puiseurs d'eau pour l'assemblée, et pour l'autel de Yahweh, jusqu'à aujourd'hui, dans le lieu qu'il choisirait.
\Chap{10}
\TextTitle{Josué secoure Gabaon des cinq rois des Amoréens}
\VerseOne{}Or quand Adoni-Tsédek, roi de Jérusalem, entendit que Josué avait pris Aï, et qu'il l'avait entièrement détruite par le moyen de l'interdit, ayant fait à Aï et à son roi, comme il avait fait à Jéricho et à son roi, et que les habitants de Gabaon avaient fait la paix avec Israël, et étaient au milieu d'eux.
\VS{2}Il eut une grande frayeur, parce que Gabaon était une grande ville, comme une ville royale, et elle était plus grande qu'Aï, et parce que tous ses hommes étaient vaillants.
\VS{3}C'est pourquoi Adoni-Tsédek, roi de Jérusalem, envoya dire à Hoham, roi d'Hébron, et à Piream, roi de Jarmuth, et à Japhia, roi de Lakis, et à Debir, roi d'Eglon :
\VS{4}Montez vers moi, et aidez-moi afin que nous frappions Gabaon, car elle a fait la paix avec Josué et avec les enfants d'Israël.
\VS{5}Ainsi cinq rois des Amoréens, savoir, le roi de Jérusalem, le roi d'Hébron, le roi de Jarmuth, le roi de Lakis, et le roi d'Eglon, s'assemblèrent et montèrent avec toutes leurs armées ; et ils campèrent près de Gabaon, et lui firent la guerre.
\VS{6}Alors les gens de Gabaon dirent à Josué au camp de Guilgal : Ne retire point tes mains de tes serviteurs, monte rapidement vers nous, délivre-nous, et donne-nous du secours ; car tous les rois des Amoréens qui habitent aux montagnes se sont rassemblés contre nous.
\VS{7}Josué donc monta de Guilgal, et avec lui tout le peuple qui était propre à la guerre, et tous les hommes forts et vaillants.
\TextTitle{Yahweh accorde à Israël une grande victoire à Makkéda}
\VS{8}Et Yahweh dit à Josué : Ne les crains point, car je les ai livré entre tes mains, et aucun d'eux ne tiendra devant toi.
\VS{9}Josué arriva subitement sur eux, après avoir marché toute la nuit depuis Guilgal.
\VS{10}Yahweh les mit en déroute devant Israël, qui en fit un grand carnage près de Gabaon, et les poursuivit par le chemin de la montagne de Beth-Horon, les battit jusqu'à Azéka, et jusqu'à Makkéda.
\VS{11}Et comme ils s'enfuyaient devant Israël, et qu'ils étaient à la descente de Beth-Horon, Yahweh fit tomber du ciel sur eux de grosses pierres jusqu'à Azéka, et ils périrent ; ceux qui moururent des pierres de grêle furent plus nombreux que ceux qui furent tués avec l'épée par les enfants d'Israël.
\VS{12}Alors Josué parla à Yahweh, le jour où Yahweh livra les Amoréens aux enfants d'Israël, et dit en présence d'Israël : Soleil, arrête-toi sur Gabaon, et toi lune, sur la vallée d'Ajalon !
\VS{13}Et le soleil s'arrêta, et la lune aussi s'arrêta, jusqu'à ce que le peuple ait tiré vengeance de ses ennemis. Cela n'est-il pas écrit dans le livre du Juste ? Le soleil s'arrêta au milieu du ciel et ne se hâta point de se coucher environ un jour entier\FTNT{Ha. 3:11.}.
\VS{14}Et il n'y a point eu de jour semblable à celui-là, ni avant ni après, où Yahweh exauça la voix d'un homme ; car Yahweh combattait pour Israël.
\VS{15}Et Josué, et tout Israël avec lui, retourna au camp à Guilgal.
\VS{16}Au reste, ces cinq rois restants s'enfuirent, et se cachèrent dans une caverne à Makkéda.
\VS{17}Et on le rapporta à Josué, en disant : On a trouvé les cinq rois cachés dans une caverne à Makkéda.
\VS{18}Et Josué dit : Roulez de grosses pierres à l'entrée de la caverne et mettez près d'elle quelques hommes pour les garder.
\VS{19}Mais vous, ne vous arrêtez pas, poursuivez vos ennemis, attaquez-les par-derrière jusqu'au dernier, ne les laissez pas entrer dans leurs villes, car Yahweh, votre Dieu, les a livrés entre vos mains.
\VS{20}Et quand Josué et les enfants d'Israël eurent achevé d'en faire une très grande boucherie, jusqu'à les détruire entièrement, ceux d'entre eux qui s'étaient échappés se retirèrent dans les villes fortifiées,
\VS{21}tout le peuple revint en paix au camp vers Josué à Makkéda, et personne ne remua sa langue contre les enfants d'Israël.
\VS{22}Alors Josué dit : Ouvrez l'entrée de la caverne, et amenez-moi ces cinq rois hors de la caverne.
\VS{23}Et ils firent ainsi, et ils lui amenèrent hors de la caverne ces cinq rois : Le roi de Jérusalem, le roi d'Hébron, le roi de Jarmuth, le roi de Lakis et le roi d'Eglon.
\VS{24}Et après qu'ils eurent amené à Josué ces cinq rois hors de la caverne, Josué appela tous les hommes d'Israël, et dit aux chefs des gens de guerre qui étaient allés avec lui : Approchez-vous, mettez vos pieds sur les cous de ces rois. Ils s'approchèrent, et mirent leurs pieds sur leurs cous\FTNT{Ps. 110:1.}.
\VS{25}Alors Josué leur dit : Ne craignez point, et ne soyez point effrayés, fortifiez-vous, et ayez du courage, car Yahweh traitera ainsi tous vos ennemis contre lesquels vous combattez.
\VS{26}Et après cela, Josué les frappa et les fit mourir, il les fit pendre à cinq arbres, et ils restèrent pendus à ces arbres jusqu'au soir.
\VS{27}Et comme le soleil se couchait, Josué ordonna qu'on les descende de ces arbres, et on les jeta dans la caverne où ils s'étaient cachés, et on mit à l'entrée de la caverne de grosses pierres qui y sont demeurées jusqu'à ce jour\FTNT{De. 21:23.}.
\VS{28}Josué prit aussi Makkéda le même jour, la frappa du tranchant de l'épée, et dévoua à la façon de l'interdit son roi et ses habitants, et ne laissa échapper personne qui était dans cette ville. Et il fit au roi de Makkéda comme il fait au roi de Jéricho.
\TextTitle{Conquête des territoires du sud}
\VS{29}Après cela, Josué, et tout Israël avec lui, passa de Makkéda à Libna, et fit la guerre à Libna.
\VS{30}Et Yahweh la livra aussi entre les mains d'Israël, avec son roi, et il la frappa du tranchant de l'épée, elle et tous ceux qui s'y trouvaient ; il n'en laissa échapper aucune personne qui était dans cette ville ; et il fit à son roi comme il avait fait au roi de Jéricho.
\VS{31}Ensuite Josué, et tout Israël avec lui, passa de Libna à Lakis, campa devant elle, et lui fit la guerre.
\VS{32}Et Yahweh livra Lakis entre les mains d'Israël, qui la prit le deuxième jour, et la frappa du tranchant de l'épée, et toutes les personnes qui s'y trouvaient, comme il avait fait à Libna.
\VS{33}Alors Horam, roi de Guézer, monta pour secourir Lakis. Josué le frappa, lui et son peuple, de sorte qu'il n'en laissa pas échapper un seul homme.
\VS{34}Après cela Josué, et tout Israël avec lui, passa de Lakis à Eglon ; ils campèrent devant elle, et lui firent la guerre.
\VS{35}Ils la prirent le jour même, la frappèrent du tranchant de l'épée ; et Josué dévoua à la façon de l'interdit ce jour-là toutes les personnes qui y étaient, comme il avait fait à Lakis.
\VS{36}Puis Josué, et tout Israël avec lui, monta d'Eglon à Hébron, et ils lui firent la guerre.
\VS{37}Et ils la prirent, et la frappèrent du tranchant de l'épée, avec son roi, toutes ses villes, et toutes les personnes qui y étaient ; il n'en laissa échapper aucune, comme il avait fait à Eglon ; et il dévoua à la façon de l'interdit, toutes les personnes qui y étaient.
\VS{38}Ensuite Josué, et tout Israël avec lui, retourna vers Debir, et ils lui firent la guerre.
\VS{39}Et il la prit, avec son roi et toutes ses villes ; et ils les frappèrent du tranchant de l'épée, et dévouèrent à la façon de l'interdit toutes les personnes qui y étaient ; il n'en laissa échapper aucune. Il fait à Debir et son roi comme il avait fait à Hébron, et comme il avait fait à Libna et à son roi.
\VS{40}Josué donc frappa tout ce pays, la montagne et le midi, la plaine et les coteaux, et tous leurs rois ; il n'en laissa échapper aucun, et il dévoua par le moyen de l'interdit toutes les personnes qui y respiraient, comme Yahweh, le Dieu d'Israël, l'avait ordonné\FTNT{De. 20:16-17.}.
\VS{41}Ainsi Josué les battit depuis Kadès-Barnéa jusqu'à Gaza, et tout le pays de Gosen jusqu'à Gabaon.
\VS{42}Josué prit tous ces rois en même temps et leur pays, parce que Yahweh, le Dieu d'Israël, combattait pour Israël.
\VS{43}Après quoi Josué, et tout Israël avec lui, retourna au camp à Guilgal.
\Chap{11}
\TextTitle{Conquête des territoires du nord}
\VerseOne{}Et aussitôt que Jabin, roi de Hatsor, eut appris ces choses, il envoya des messagers à Jobab, roi de Madon, au roi de Schimron, et au roi d'Acschaph,
\VS{2}et aux rois qui habitaient vers le nord, aux montagnes et dans la plaine, vers le midi de Kinnéreth, dans la vallée, et sur les hauteurs de Dor vers l'occident,
\VS{3}aux Cananéens qui étaient à l'orient et à l'occident, aux Amoréens, aux Héthiens, aux Phéréziens, aux Jébusiens dans les montagnes, et aux Héviens au pied de la montagne de l'Hermon, dans le pays de Mitspa.
\VS{4}Ils sortirent donc avec toutes leurs armées, un grand peuple par leur grand nombre, comme le sable qui est sur le bord de la mer, il y avait aussi des chevaux et des chars en très grand nombre.
\VS{5}Tous ces rois se réunirent, et campèrent ensemble près des eaux de Mérom, pour combattre contre Israël.
\VS{6}Et Yahweh dit à Josué : Ne les crains point, car demain, à cette même heure, je les livrerai tous, blessés à mort, devant Israël. Tu couperas les jarrets à leurs chevaux, et brûleras au feu leurs chars\FTNT{2 S. 8:4.}.
\VS{7}Josué donc, et tous les gens de guerre avec lui vinrent subitement sur eux près des eaux de Mérom, et ils se précipitèrent au milieu d'eux.
\VS{8}Et Yahweh les livra entre les mains d'Israël ; ils les battirent, et les poursuivirent jusqu'à Sidon la grande, jusqu'aux eaux de Misrephoth-Maïm, et jusqu'à la vallée de Mitspa vers l'orient, et ils les battirent tellement qu'ils ne laissèrent aucun survivant.
\VS{9}Et Josué leur fit comme Yahweh lui avait dit ; il coupa les jarrets de leurs chevaux, et brûla au feu leurs chars.
\VS{10}A son retour, et dans le même temps, Josué prit Hatsor, et frappa son roi avec l'épée ; car Hatsor avait été auparavant la capitale de tous ces royaumes.
\VS{11}On frappa aussi du tranchant de l'épée et l'on dévoua à la façon de l'interdit tous ceux qui s'y trouvaient, il ne resta rien de ce qui respirait, et l'on brûla au feu Hatsor.
\VS{12}Josué prit aussi toutes les villes de ces rois, et tous leurs rois, et les frappa du tranchant de l'épée, et il les dévoua à la façon de l'interdit, comme Moïse, serviteur de Yahweh, l'avait ordonné.
\VS{13}Mais Israël ne brûla aucune des villes situées sur des collines, excepté de Hatsor seule, que Josué brûla.
\VS{14}Et les enfants d'Israël pillèrent pour eux tout le butin de ces villes et le bétail ; mais ils frappèrent du tranchant de l'épée tous les hommes, jusqu'à ce qu'ils les aient exterminés, ils n'y laissèrent aucun qui respirait.
\VS{15}Comme Yahweh l'avait ordonné à Moïse son serviteur, ainsi Moïse l'avait ordonné à Josué ; et Josué le fit ainsi ; de sorte qu'il n'omit rien de tout ce que Yahweh avait ordonné à Moïse. 
\TextTitle{Josué s'empare de tout le pays}
\VS{16}Josué donc prit tout ce pays-là, la montagne et tout le pays du midi, avec tout le pays de Gosen, la vallée et la plaine, la montagne d'Israël et ses vallées.
\VS{17}Depuis la montagne de Halak, qui s'élève vers Séir, jusqu'à Baal-Gad dans la vallée du Liban, au pied de la montagne d'Hermon. Il prit aussi tous leurs rois, les battit et les fit mourir.
\VS{18} Josué fit la guerre plusieurs jours contre tous ces rois.
\VS{19}Il n'y eut aucune ville qui fit la paix avec les enfants d'Israël, excepté les Héviens qui habitaient à Gabaon ; ils les prirent toutes par la guerre.
\VS{20}Car cela venait de Yahweh, qu'ils endurcissent leur cœur pour qu'ils sortent en bataille contre Israël, afin qu'il les dévoue à la façon de l'interdit, sans qu'il y ait pour eux de miséricorde, et qu'il les extermine, comme Yahweh l'avait ordonné à Moïse\FTNT{Ex. 4:21 ; De. 2:30 ; 1 R. 12:15.}.
\VS{21}En ce même temps-là aussi, Josué se mit en marche, et il extermina les Anakim des montagnes d'Hébron, de Debir, d'Anab, et de toute la montagne de Juda, et de toute la montagne d'Israël ; Josué, dis-je, les dévoua à la façon de l'interdit avec leurs villes.
\VS{22}Il ne resta aucun Anakim dans le pays des enfants d'Israël ; il n'en resta seulement qu'à Gaza, à Gath et à Asdod\FTNT{2 S. 21:20.}.
\VS{23}Josué donc prit tout le pays, suivant tout ce que Yahweh avait dit à Moïse. Et Josué le donna en héritage à Israël, selon leurs portions, et leurs tribus. Et le pays fut en repos et sans avoir guerre.
\Chap{12}
\TextTitle{Liste des rois vaincus par Moïse et Josué}
\VerseOne{}Voici les rois du pays que les enfants d'Israël frappèrent, et dont ils possédèrent le pays de l'autre côté du Jourdain, vers l'orient, depuis le torrent de l'Arnon jusqu'à la montagne de l'Hermon, et toute la plaine vers l'orient.
\VS{2}Savoir, Sihon, roi des Amoréens, qui habitait à Hesbon, et qui dominait depuis Aroër, qui est sur le bord du torrent de l'Arnon, et depuis le milieu du torrent, sur la moitié de Galaad, jusqu'au torrent de Jabbok, qui est la frontière des enfants d'Ammon\FTNT{De. 3:8-16.} ;
\VS{3}et depuis la plaine jusqu'à la mer de Kinnéreth vers l'orient, et jusqu'à la mer de la plaine, qui est la mer salée, vers l'orient, au chemin de Beth-Jeschimoth ; et depuis le midi sur le pied du Pisga.
\VS{4}Et les contrées d'Og, roi de Basan, qui était seul reste des Rephaïm, et qui habitait à Aschtaroth et à Edréï.
\VS{5}Et sa domination s'étendait sur la montagne de l'Hermon, sur Salca, et sur tout Basan, jusqu'à la frontière des Gueschuriens et des Maacathiens, et sur la moitié de Galaad, frontière de Sihon, roi de Hesbon.
\VS{6}Moïse, serviteur de Yahweh, et les enfants d'Israël, les battirent ; et Moïse, serviteur de Yahweh, en donna la possession aux Rubénites, aux Gadites, et à la demi-tribu de Manassé\FTNT{No. 32:33.}.
\VS{7}Voici les rois du pays que Josué et les enfants d'Israël frappèrent de ce côté-ci du Jourdain vers l'occident, depuis Baal-Gad, dans la vallée du Liban, jusqu'à la montagne de Halak qui monte vers Séir, et que Josué donna aux tribus d'Israël en possession, selon leurs portions,
\VS{8}pays consistant en montagnes et en vallées, en plaines et en collines, en pays de désert et de midi: Les Héthiens, les Amoréens, les Cananéens, les Phéréziens, les Héviens et les Jébusiens.
\VS{9}Le roi de Jéricho, un ; le roi d'Aï, près de Béthel, un ;
\VS{10}le roi de Jérusalem, un ; le roi d'Hébron, un ;
\VS{11}le roi de Jarmuth, un ; le roi de Lakis, un ;
\VS{12}le roi d'Eglon, un ; le roi de Guézer, un ;
\VS{13}le roi de Debir, un ; le roi de Guéder, un ;
\VS{14}le roi de Horma, un ; le roi d'Arad, un ;
\VS{15}le roi de Libna, un ; le roi d'Adullam, un ;
\VS{16}le roi de Makkéda, un ; le roi de Béthel, un ;
\VS{17}le roi de Tappuach, un ; le roi de Hépher, un ;
\VS{18}le roi d'Aphek, un ; le roi de Lascharon, un ;
\VS{19}le roi de Madon, un ; le roi de Hatsor, un ;
\VS{20}le roi de Schimron-Meron, un ; le roi d'Acschaph, un ;
\VS{21}le roi de Taanac, un ; le roi de Meguiddo, un ;
\VS{22}le roi de Kédesch, un ; le roi de Jokneam, au Carmel, un ;
\VS{23}le roi de Dor, sur les hauteurs de Dor, un ; le roi de Gojim, près de Guilgal, un ;
\VS{24}le roi de Thirtsa, un ; en tout trente et un rois.
\Chap{13}
\TextTitle{Les territoires de Ruben, de Gad et de la demi-tribu de Manassé}
\VerseOne{}Or, quand Josué fut devenu vieux, fort avancé en âge, Yahweh lui dit : Tu es devenu vieux, fort avancé en âge, et il te reste encore un très grand pays à posséder.
\VS{2}Voici le pays qui reste, toutes les contrées des Philistins, et des Gueschuriens,
\VS{3}depuis le Schichor, qui coule devant l'Egypte, jusqu'à la frontière d'Ekron au nord, contrée qui doit être tenue pour Cananéenne, et qui est occupée par les cinq princes des Philistins, celui de Gaza, celui d'Asdod, celui d'Askalon, celui de Gath, celui d'Ekron, et par les Avviens ;
\VS{4}du côté du midi, tout le pays des Cananéens, et Meara qui est aux Sidoniens, jusqu'à Aphek, jusqu'à la frontière des Amoréens ;
\VS{5}le pays qui appartient aux Guibliens, et tout le Liban, vers l'orient, depuis Baal-Gad, au pied de la montagne d'Hermon, jusqu'à l'entrée de Hamath ;
\VS{6}tous les habitants de la montagne, depuis le Liban jusqu'aux eaux de Misrephoth-Maïm, tous les Sidoniens. Je les chasserai moi-même devant les fils d'Israël. Donne seulement ce pays en héritage par le sort à Israël, comme je te l'ai prescrit.
\VS{7}Maintenant donc divise ce pays en héritage aux neuf tribus, et à la demi-tribu de Manassé.
\VS{8}Avec l'autre moitié de laquelle les Rubénites et les Gadites ont pris leur héritage, lequel Moïse leur a donné au delà du Jourdain, vers l'orient, selon que Moïse, serviteur de Yahweh, le leur a donné ;
\VS{9}depuis Aroër, qui est sur le bord du torrent de l'Arnon, et la ville qui est au milieu de la vallée, et toute la plaine de Médeba, jusqu'à Dibon ;
\VS{10}et toutes les villes de Sihon, roi des Amoréens, qui régnait à Hesbon, jusqu'à la frontière des enfants d'Ammon ;
\VS{11}et Galaad, et les territoires des Gueschuriens et des Maacathiens, toute la montagne de l'Hermon, et tout Basan jusqu'à Salca ;
\VS{12}tout le royaume d'Og en Basan, qui régnait à Aschtaroth, et à Edréï, et qui était resté le seul reste des Rephaïm ; Moïse battit ces rois, et les chassa.
\VS{13}Or les fils d'Israël ne chassèrent point les Gueschuriens et les Maacathiens, mais les Gueschuriens et les Maacathiens ont habité au milieu d'Israël jusqu'à ce jour.
\VS{14}Seulement il ne donna point d'héritage à la tribu de Lévi ; les sacrifices consumés par le feu devant Yahweh, le Dieu d'Israël, tel fut son héritage, comme il le lui avait dit\FTNT{No. 18:20-24 ; De. 10:9 ; De. 18:2 ; Ez. 44:28.}.
\VS{15}Moïse donc donna un héritage à la tribu des fils de Ruben selon leurs familles.
\VS{16}Et leurs frontières furent depuis Aroër qui est sur le bord du torrent d'Arnon, et de la ville qui est au milieu du torrent, et toute la plaine qui est près de Médeba.
\VS{17}Hesbon et toutes ses villes, qui étaient dans la plaine, Dibon, Bamoth-Baal, Beth-Baal-Meon,
\VS{18}Jahats, Kedémoth et Méphaath,
\VS{19}Kirjathaïm, Sibma, Tséreth-Haschachar sur la montagne de la vallée,
\VS{20}Beth-Peor, les coteaux du Pisga et Beth-Jeschimoth,
\VS{21}et toutes les villes de la plaine, et tout le royaume de Sihon, roi des Amoréens qui régnait à Hesbon ; Moïse l'avait battu, lui et les princes de Madian, Evi, Rékem, Tsur, Hur, et Réba, princes qui relevaient de Sihon, et qui habitaient dans le pays.
\VS{22}Les enfants d'Israël firent passer aussi par l'épée Balaam\FTNT{Voir No. 22. Balaam était l'exemple type du prophète corrompu, soucieux de tirer profit de son service.}, fils de Beor, le devin, avec les autres qui y furent tués.
\VS{23}Et les frontières des enfants d'Israël fut le Jourdain et sa frontière. Tel fut l'héritage des fils de Ruben, selon leurs familles ; savoir, ces villes-là et leurs villages\FTNT{No. 34:14-15.}.
\VS{24}Moïse donna aussi un héritage à la tribu de Gad, pour les fils de Gad, selon leurs familles.
\VS{25}Et leur pays fut Jaezer, et toutes les villes de Galaad et la moitié du pays des enfants d'Ammon, jusqu'à Aroër, qui est vis-à-vis de Rabba,
\VS{26}et depuis Hesbon jusqu'à Ramath-Mitspé, et Bethonim, et depuis Mahanaïm jusqu'à la frontière de Debir,
\VS{27}et, dans la vallée, Beth-Haram, Beth-Nimra, Succoth et Tsaphon, reste du royaume de Sihon, roi de Hesbon, ayant le Jourdain pour frontière jusqu'à l'extrémité de la mer de Kinnéreth, de l'autre côté du Jourdain, vers l'orient.
\VS{28}Tel fut l'héritage des fils de Gad, selon leurs familles ; savoir, les villes et leurs villages.
\VS{29}Moïse donna aussi à la demi-tribu de Manassé un héritage, qui est resté à la demi-tribu des fils de Manassé, selon leurs familles.
\VS{30}Leur pays fut depuis Mahanaïm, tout Basan, et tout le royaume d'Og, roi de Basan, et tous les villages de Jaïr qui sont en Basan, soixante villes.
\VS{31}Et la moitié de Galaad, Aschtaroth et Edréï, villes du royaume d'Og en Basan, furent aux fils de Makir, fils de Manassé, à la moitié des enfants de Makir, selon leurs familles.
\VS{32}Ce sont là les pays que Moïse avait donnés en héritage, lorsqu'il était dans les plaines de Moab, de l'autre côté du Jourdain, vis-à-vis de Jéricho, à l'orient.
\VS{33}Mais Moïse ne donna point d'héritage à la tribu de Lévi ; car Yahweh, le Dieu d'Israël, fut leur héritage, comme il le lui avait dit.
\Chap{14}
\TextTitle{Caleb reçoit Hébron}
\VerseOne{}Voici les terres que les enfants d'Israël eurent pour héritage dans le pays de Canaan, ce que partagèrent entre eux le prêtre Eléazar, Josué, fils de Nun, et les chefs des familles des tribus des enfants d'Israël.
\VS{2}Selon le sort de leur héritage ; comme Yahweh l'avait ordonné par le moyen de Moïse ; savoir, à neuf tribus et à la demi-tribu\FTNT{No. 26:55.}.
\VS{3}Car Moïse avait donné un héritage aux deux tribus et à la demi-tribu de l'autre côté du Jourdain, mais il n'avait point donné de part aux Lévites parmi eux.
\VS{4}Parce que les fils de Joseph, savoir, Manassé et Ephraïm, formaient deux tribus ; et l'on ne donna point de part aux Lévites dans le pays, excepté des villes pour habitation, et les faubourgs pour leurs troupeaux, et pour le reste de leurs biens.
\VS{5}Les enfants d'Israël firent comme Yahweh l'avait ordonné à Moïse, et ils partagèrent le pays.
\VS{6}Or les fils de Juda s'approchèrent de Josué à Guilgal ; et Caleb, fils de Jephunné, le Kenizien, lui dit : Tu sais la parole que Yahweh a déclarée à Moïse, homme de Dieu, à mon sujet et au tien à Kadès-Barnéa\FTNT{No. 14:24 ; No. 32:12 ; De. 1:36.}.
\VS{7}J'étais âgé de quarante ans quand Moïse, serviteur de Yahweh, m'envoya à Kadès-Barnéa pour espionner le pays, et je lui fis un rapport avec droiture de cœur.
\VS{8}Et mes frères qui étaient montés avec moi découragèrent le cœur du peuple, mais moi je persévérai à suivre Yahweh, mon Dieu.
\VS{9}Et ce jour-là Moïse jura, en disant : La terre que ton pied a foulée sera ton héritage à perpétuité, pour toi et pour tes fils, parce que tu as persévéré à suivre Yahweh, mon Dieu.
\VS{10}Or maintenant voici, Yahweh m'a fait vivre comme il l'a dit. Il y a déjà quarante-cinq ans que Yahweh déclarait cette parole à Moïse, lorsqu'Israël marchait dans le désert. Et maintenant voici, je suis aujourd'hui âgé de quatre-vingt-cinq ans.
\VS{11}Et je suis encore aujourd'hui aussi vigoureux que j'étais le jour où Moïse m'envoya ; et j'ai maintenant la même force que j'avais alors pour le combat, soit pour sortir et pour entrer.
\VS{12}Maintenant, donne-moi donc cette montagne, dont Yahweh a parlé ce jour-là ; car tu as appris en ce jour qu'il s'y trouve des Anakim, et qu'il y a de grandes villes fortifiées. Yahweh sera peut-être avec moi, et je les chasserai, comme Yahweh a dit.
\VS{13}Josué donc bénit Caleb, fils de Jephunné, et lui donna Hébron pour héritage.
\VS{14}C'est ainsi que Caleb, fils de Jephunné, le Kenizien, a eu jusqu'à ce jour Hébron pour héritage, parce qu'il avait persévéré à suivre Yahweh, le Dieu d'Israël.
\VS{15}Or Hébron s'appelait autrefois Kirjath-Arba ; et Arba avait été le plus grand homme parmi les Anakim. Le pays fut en repos et sans guerre.
\Chap{15}
\TextTitle{Le territoire de Juda}
\VerseOne{} Ce sont ici la part échue par le sort à la tribu des enfants de Juda, selon leurs familles ; à la frontière d'Edom, au désert de Tsin, vers le midi, fut la dernière extrémité de leurs pays vers le midi ;
\VS{2}tellement que leur frontière, du côté du midi, fut la dernière extrémité de la mer Salée, depuis le bras qui regarde vers le midi. 
\VS{3}Et elle devait sortir vers le midi de la montée d'Akrabbim, et passer vers Tsin ; et montant du midi de Kadès-Barnéa, passer à Hetsron ; puis montant vers Addar, se tourner vers Karkaa ; 
\VS{4}puis, passant vers Atsmon, sortir au torrent d'Egypte ; tellement que les extrémités de cette frontière devaient se rendre à la mer. Ce sera là, dit Josué, votre frontière, du côté du midi.
\VS{5}Et la frontière vers l'orient était la mer salée jusqu'à l'embouchure du Jourdain. La frontière du côté du nord sera depuis la langue de mer, qui est à l'embouchure du Jourdain.
\VS{6}Et cette frontière montera jusqu'à Beth-Hogla, et passera du côté du nord de Beth-Araba ; et cette frontière montera jusqu'à la pierre de Bohan, fils de Ruben.
\VS{7}Puis cette frontière montera vers Debir, depuis la vallée d'Acor, et même vers le nord, du côté de Guilgal, qui est vis-à-vis de la montée d'Adummim, au sud du torrent. Puis cette frontière passera près des eaux d'En-Schémesch, et ses extrémités se prolongeront à En-Roguel.
\VS{8}Puis cette frontière montera de là par la vallée de Ben-Hinnom, au côté du midi de Jebus, qui est Jérusalem, puis cette frontière montera jusqu'au sommet de la montagne, qui est vis-à-vis de la vallée de Hinnom, à l'occident, et à l'extrémité de la vallée des Rephaïm, au nord.
\VS{9}Et cette frontière s'alignera, depuis le sommet de la montagne jusqu'à la source des eaux de Nephthoach, et continuera vers les villes de la montagne d'Ephron, puis cette frontière s'alignera à Baala, qui est Kirjath-Jearim.
\VS{10}Et cette frontière se tournera depuis Baala, vers l'occident, jusqu'à la montagne de Séir, puis elle traversera le côté nord de la montagne de Jearim, à Kesalon, puis descendait à Beth-Schémesch, et passera par Thimna.
\VS{11}Et cette frontière sortira jusqu'au côté d'Ekron, vers le nord et cette frontière s'alignera vers Schicron, puis ayant passé la montagne de Baala, elle se sortira jusqu'à Jabneel ; tellement que les extrémités de cette frontière se rendront à la mer. 
\VS{12}Or la frontière du côté de l'occident sera ce qui est vers la grande mer et ses limites. Telles furent de tous les côtés les frontières des fils de Juda, selon leurs familles.
\VS{13}Au reste, on donna à Caleb, fils de Jephunné, une part au milieu des fils de Juda, comme Yahweh l'avait ordonné à Josué ;savoir, Kirjath-Arba, or Arba était père d'Anak ; et Kirjath-Arba c'est Hébron.
\VS{14}Et Caleb chassa de là les trois fils d'Anak : Schéschaï, Ahiman, et Talmaï, fils d'Anak.
\VS{15}Et de là il monta contre les habitants de Debir ; Debir s'appelait autrefois Kirjath-Sépher.
\VS{16}Et Caleb dit : Je donnerai ma fille Acsa pour femme à celui qui battra Kirjath-Sépher, et la prendra\FTNT{Jg. 1:12-14.}.
\VS{17}Et Othniel, fils de Kenaz, frère de Caleb, la prit ; et Caleb lui donna sa fille Acsa pour femme.
\VS{18}Et il arriva que comme elle s'en allait, elle l'incita à demander à son père un champ ; puis elle descendit impétueusement de dessus son âne, et Caleb lui dit : Qu'as-tu ? 
\VS{19}Elle répondit : Donne-moi un présent, puisque tu m'as donné une terre du sud, donne-moi aussi des sources d'eau. Et il lui donna les sources supérieures et les sources inférieures.
\VS{20}Tel fut l'héritage de la tribu des fils de Juda, selon leurs familles.
\VS{21}Les villes situées dans la contrée du midi, à l'extrémité de la tribu des fils de Juda, près de la frontière d'Edom, étaient : Kabtseel, Eder, Jagur,
\VS{22}Kina, Dimona, Adada,
\VS{23}Kédesch, Hatsor, Ithnan,
\VS{24}Ziph, Thélem, Bealoth,
\VS{25}Hatsor-Hadattha, Kerijoth-Hetsron qui est Hatsor,
\VS{26}Amam, Schema, Molada,
\VS{27}Hatsar-Gadda, Heschmon, Beth-Paleth,
\VS{28}Hatsar-Schual, Beer-Schéba, Bizjothja,
\VS{29}Baala, Ijjim, Atsem,
\VS{30}Eltholad, Kesil, Horma,
\VS{31}Tsiklag, Madmanna, Sansanna,
\VS{32}Lebaoth, Schilhim, Aïn et Rimmon. Total des villes : Vingt-neuf villes, et leurs villages.
\VS{33}Dans la plaine : Eschthaol, Tsorea, Aschna,
\VS{34}Zanoach, En-Gannim, Tappuach, Enam,
\VS{35}Jarmuth, Adullam, Soco, Azéka,
\VS{36}Schaaraïm, Adithaïm, Guedéra et Guedérothaïm ; quatorze villes, et leurs villages.
\VS{37}Tsenan, Hadascha, Migdal-Gad,
\VS{38}Dilean, Mitspé, Joktheel,
\VS{39}Lakis, Botskath, Eglon,
\VS{40}Cabbon, Lachmas, Kithlisch,
\VS{41}Guedéroth, Beth-Dagon, Naama, et Makkéda ; seize villes, et leurs villages.
\VS{42}Libna, Ether, Aschan,
\VS{43}Jiphtach, Aschna, Netsib,
\VS{44}Keïla, Aczib et Maréscha ; neuf villes, et leurs villages.
\VS{45}Ekron, et les villes de son ressort, et ses villages.
\VS{46}Depuis Ekron et à l'occident, toutes les villes près d'Asdod, et leurs villages.
\VS{47}Asdod, les villes de son ressort, et ses villages, Gaza, les villes de son ressort, et ses villages, jusqu'au torrent d'Egypte, et à la grande mer, qui sert de limite.
\VS{48}Dans la montagne : Schamir, Jatthir, Soco,
\VS{49}Danna, Kirjath-Sanna, qui est Debir,
\VS{50}Anab, Eschthemo, Anim,
\VS{51}Gosen, Holon, et Guilo ; onze villes et leurs villages.
\VS{52}Arab, Duma, Eschean,
\VS{53}Janum, Beth-Tappuach, Aphéka,
\VS{54}Humta, Kirjath-Arba, qui est Hébron, et Tsior ; neuf villes, et leurs villages.
\VS{55}Maon, Carmel, Ziph, Juta,
\VS{56}Jizreel, Jokdeam, Zanoach,
\VS{57}Kaïn, Guibea, et Thimna ; dix villes, et leurs villages.
\VS{58}Halhul, Beth-Tsur, Guedor,
\VS{59}Maarath, Beth-Anoth, et Elthekon ; six villes, et leurs villages.
\VS{60}Kirjath-Baal, qui est Kirjath-Jearim, et Rabba ; deux villes, et leurs villages.
\VS{61}Au désert : Beth-Araba, Middin, Secaca,
\VS{62}Nibschan, Ir-Hammélach, et En-Guédi : Six villes et leurs villages.
\VS{63}Au reste, les fils de Juda ne purent pas chasser les Jébusiens qui habitaient à Jérusalem, c'est pourquoi les Jébusiens ont habité avec les fils de Juda à Jérusalem jusqu'à ce jour.
\Chap{16}
\TextTitle{Le territoire d'Ephraïm}
\VerseOne{}La part échue par le sort aux fils de Joseph depuis le Jourdain près de Jéricho, aux eaux de Jéricho, vers l'orient qui est le désert ; montant de Jéricho par la montagne jusqu'à Béthel.
\VS{2}Et cette frontière devait sortir de Béthel à Luz, puis passer vers la frontière des Arkiens jusqu'à Atharoth.
\VS{3}Et elle devait descendre tirant vers l'occident, vers la frontière des Japhléthiens, jusqu'à celle de Beth-Horon la basse et jusqu'à Guézer, de sorte que ses extrémités aboutissent à la mer.
\VS{4}Ainsi les fils de Joseph, savoir, Manassé et Ephraïm, reçurent leur héritage.
\VS{5}Or la frontière des fils d'Ephraïm, selon leurs familles, la frontière de leur héritage était à l'orient, Atharoth-Addar, jusqu'à Beth-Horon la haute.
\VS{6}Et cette frontière devait sortir vers la mer à Micmethath, du côté du nord ; et cette frontière devait se tourner vers l'orient jusqu'à Thaanath-Silo, et passant du côté d'orient, se rendre à Janoach.
\VS{7}Puis descendre de Janoach à Atharoth et à Naaratha, se rencontrer à Jéricho, et sortir au Jourdain.
\VS{8}Et cette frontière devait aller de Tappuach, vers l'occident, jusqu'au torrent de Kana, tellement que ses extrémités devaient se rendre à la mer. Ce fut là l'héritage de la tribu des fils d'Ephraïm, selon leurs familles.
\VS{9}Les fils d'Ephraïm avaient aussi des villes séparées au milieu de l'héritage des fils de Manassé, toutes ces villes, avec leurs villages.
\VS{10}Or ils ne chassèrent point les Cananéens qui habitaient à Guézer, c'est pourquoi les Cananéens ont habité parmi Ephraïm jusqu'à ce jour, mais ils furent réduits à la servitude et assujettis à un tribut\FTNT{Jg. 1:29 ; 1 R. 9:16.}.
\Chap{17}
\TextTitle{Le territoire de Manassé}
\VerseOne{}Il y eut aussi une part échut par le sort à la tribu de Manassé qui était le premier-né de Joseph. Quant à Makir, premier-né de Manassé, et père de Galaad, il avait eu Galaad et Basan parce qu'il était un homme de guerre.
\VS{2}Puis on jeta donc le sort pour les autres enfants de Manassé, selon ses familles ; aux fils d'Abiézer, aux fils de Hélek, aux fils d'Asriel, aux fils de Sichem, aux fils de Hépher, et aux fils de Schemida. Ce sont là les enfants mâles de Manassé fils de Joseph, selon leurs familles.
\VS{3}Or Tselophchad, fils de Hépher, fils de Galaad, fils de Makir, fils de Manassé, n'eut point de fils, mais il eut des filles dont voici les noms : Machla, Noa, Hogla, Milca et Thirtsa.
\VS{4}Elles vinrent se présenter devant le prêtre Eléazar, devant Josué, fils de Nun, et devant les princes, en disant : Yahweh a ordonné à Moïse de nous donner un héritage parmi nos frères. C'est pourquoi on leur donna un héritage parmi les frères de leur père, selon l'ordre de Yahweh\FTNT{No. 27:7 ; No. 36:2.}.
\VS{5}Et dix portions échurent à Manassé, outre le pays de Galaad et de Basan, qui est de l'autre côté du Jourdain.
\VS{6}Car les filles de Manassé eurent un héritage parmi ses fils, et le pays de Galaad fut pour les autres des fils de Manassé.
\VS{7}Or la frontière de Manassé fut du côté d'Aser, venant à Micmethath, qui est près de Sichem ; puis cette frontière devait aller à main droite vers les habitants d'En-Tappuach.
\VS{8}Or le pays de Tappuach appartenait à Manassé, mais Tappuach qui était près de la frontière de Manassé, appartenait aux fils d'Ephraïm.
\VS{9}De là, cette frontière devait descendre au torrent de Kana, au midi du torrent. Ces villes étaient à Ephraïm parmi les villes de Manassé. La frontière de Manassé était au côté du nord du torrent, et ses extrémités devaient se rendre à la mer.
\VS{10}Ce qui était vers le midi était à Ephraïm, et celui qui était vers le nord était à Manassé, et la mer leur servait de frontière ; et du côté du nord, les frontières se rencontraient à Aser, à Issacar, vers l'orient.
\VS{11}Car Manassé possédait dans Issacar et dans Aser : Beth-Schean et les villes de son ressort, Jibleam et les villes de son ressort, les habitants de Dor et les villes de son ressort, les habitants d'En-Dor, et les villes de son ressort, les habitants de Thaanac et les villes de son ressort, les habitants de Meguiddo et les villes de son ressort, qui sont trois contrées.
\VS{12}Au reste, les fils de Manassé ne purent pas chasser les habitants de ces villes, et les Cananéens voulurent rester dans le même pays.
\VS{13}Mais lorsque les fils d'Israël furent assez forts, ils assujettirent les Cananéens à un tribut, mais ils ne les chassèrent pas entièrement.
\VS{14}Or les fils de Joseph parlèrent à Josué, et dirent : Pourquoi nous as-tu donné en héritage un seul lot, et une seule part, vu que nous sommes un peuple nombreux, et que Yahweh nous a bénis jusqu'à présent ?
\VS{15}Et Josué leur dit : Si vous êtes un peuple nombreux, montez à la forêt, et vous l'abattrez, pour vous y faire de la place dans le pays des Phéréziens et des Rephaïm, si la montagne d'Ephraïm est trop étroite pour vous.
\VS{16}Et les fils de Joseph répondirent : Cette montagne ne sera pas suffisante pour nous, et tous les Cananéens qui habitent la vallée ont des chars de fer, et ceux qui sont à Beth-Schean, et dans les villes de son ressort, et ceux qui habitent dans la vallée de Jizreel\FTNT{Jg. 1:19 ; Jg. 4:3.}.
\VS{17}Donc Josué parla à la maison de Joseph, à Ephraïm et à Manassé, et dit : Vous êtes un peuple nombreux, et vous avez de grandes forces, vous n'aurez pas qu'une seule part.
\VS{18}Mais vous aurez la montagne, car c'est une forêt que vous abattrez et dont les extrémités vous appartiendront, et vous chasserez les Cananéens, quoiqu'ils aient des chars de fer, et qu'ils soient puissants.
\Chap{18}
\TextTitle{La tente d'assignation à Silo}
\VerseOne{}Or toute l'assemblée des enfants d'Israël s'assembla à Silo\FTNT{Silo fut pendant la période des Juges le centre religieux d'Israël car c'est dans cette ville que l'on avait déposé l'arche jusqu'à ce que le roi David l'amène à Jérusalem (Jos. 18:1 ; 2 S. 6 ; 1 Ch. 15:3). Durant le schisme, Silo, située en Samarie, fit office de capitale du royaume de sud. La ville fut finalement détruite par les Philistins aux alentours de 1050 av. J.-C.}, et ils y posèrent la tente d'assignation, après que le pays leur ait été assujetti. 
\VS{2}Mais il restait sept tribus des enfants d'Israël qui n'avaient pas encore reçu leur héritage.
\VS{3}Josué dit aux enfants d'Israël : Jusqu'à quand négligerez-vous de prendre possession du pays que Yahweh, le Dieu de vos pères, vous a donné ?
\VS{4}Prenez trois hommes de chaque tribu, que j'enverrai. Ils se lèveront, traverseront le pays, traceront un plan en vue de l'héritage, puis ils reviendront auprès de moi.
\VS{5}Ils le diviseront en sept parts ; Juda restera dans ses limites au midi, et la maison de Joseph restera dans ses limites au nord.
\VS{6}Vous donc faites-vous un plan du pays en sept parts, et apportez-le-moi ici. Puis je jetterai pour vous le sort devant Yahweh, notre Dieu.
\VS{7}Et il n'y aura point de part pour les Lévites au milieu de vous, parce que le sacerdoce de Yahweh est leur héritage. Quant à Gad et à Ruben, et à la demi-tribu de Manassé, ils ont reçu leur héritage de l'autre côté du Jourdain, vers l'orient, que Moïse, serviteur de Yahweh, leur a donné.
\VS{8}Ces hommes-là donc se levèrent et s'en allèrent pour tracer un plan du pays, Josué leur donna cet ordre en disant : Allez et traversez le pays, et tracez-en un plan, puis revenez auprès de moi, et je jetterai ici le sort pour vous devant Yahweh, à Silo.
\VS{9}Ces hommes-là donc s'en allèrent, parcoururent le pays, et en tracèrent un plan dans un livre en sept parts selon les villes ; puis ils revinrent auprès de Josué dans le camp à Silo.
\VS{10}Et Josué jeta le sort pour eux à Silo devant Yahweh, et Josué fit le partage du pays entre les enfants d'Israël, selon leurs parts.
\TextTitle{Le territoire de Benjamin}
\VS{11}Et le sort tomba sur la tribu des fils de Benjamin selon leurs familles, et la part qui leur échut par le sort avait ses frontières entre les fils de Juda et les fils de Joseph.
\VS{12}Et leur frontière du côté du nord fut depuis le Jourdain ; et cette frontière devait monter à côté de Jéricho vers le nord, puis monter en la montagne tirant vers l'Occident ; de sorte que ses extrémités devaient se rendre au désert de Beth-aven. 
\VS{13}Puis cette frontière devait passer de là vers Luz, à côté de Luz, qui est Béthel tirant vers le midi ; et cette frontière devait descendre à Hatroth-addar, près de la montagne qui est du côté du midi de Beth-horon la basse. 
\VS{14}Et cette frontière devait s'aligner et tourner du côté occidental qui regarde vers le midi, depuis la montagne qui est vis-à-vis de Beth-horon, vers le midi ; tellement que ses extrémités devaient se rendre à Kirjath Baal, qui est Kirjath Jearim, ville des enfants de Juda. C'est là le côté d'occident. 
\VS{15}Mais le côté méridional est l'extrémité de Kirjath Jearim ; et cette frontière devait sortir vers l'Occident, puis elle devait sortir à la fontaine des eaux de Nephtoah. 
\VS{16}Et cette frontière devait descendre à l'extrémité de la montagne qui est vis-à-vis de la vallée de Ben-Hinnom, dans la vallée des Rephaïm, vers le nord, et descendre par la vallée de Hinnom, sur le côté méridional des Jébusiens, puis descendre jusqu'à En-Roguel.
\VS{17}Et elle devait s'aligner vers le nord, et sortir à En-Schémesch, de là à Gueliloth, qui est vis-à-vis de la montée d'Adummim, et descendre à la pierre de Bohan, fils de Ruben,
\VS{18}et passer sur le côté nord en face d'Araba, et descendre à Araba,
\VS{19}puis cette frontière devait passer à côté de Beth-Hogla vers le nord ; de sorte que les extrémités de cette frontière aboutissent à la langue de la mer salée vers le nord, à l'embouchure du Jourdain vers le midi. C'était la frontière du midi.
\VS{20}Et le Jourdain devait borner du côté de l'orient. Ce fut là l'héritage des fils de Benjamin avec ses frontières tout autour, selon leurs familles.
\VS{21}Les villes de la tribu des fils de Benjamin, selon leurs familles, étaient : Jéricho, Beth-Hogla, Emek-Ketsits,
\VS{22}Beth-Araba, Tsemaraïm, Béthel,
\VS{23}Avvim, Para, Ophra,
\VS{24}Kephar-Ammonaï, Ophni et Guéba ; douze villes et leurs villages.
\VS{25}Gabaon, Rama, Beéroth,
\VS{26}Mitspé, Kephira, Motsa,
\VS{27}Rékem, Jirpeel, Thareala,
\VS{28}Tséla, Eleph, Jebus, qui est Jérusalem, Guibeath et Kirjath ; quatorze villes et leurs villages. Tel fut l'héritage des fils de Benjamin selon leurs familles.
\Chap{19}
\TextTitle{Le territoire de Siméon}
\VerseOne{}La deuxième part échut par le sort à Siméon, pour la tribu des fils de Siméon, selon leurs familles. Leur héritage était parmi l'héritage des fils de Juda\FTNT{Ge. 49:5-7.}.
\VS{2}Ils eurent dans leur héritage Beer-Schéba, Schéba, Molada,
\VS{3}Hatsar-Schual, Bala, Atsem,
\VS{4}Eltholad, Bethul, Horma,
\VS{5}Tsiklag, Beth-Marcaboth, Hatsar-Susa,
\VS{6}Beth-Lebaoth et Scharuchen ; treize villes et leurs villages.
\VS{7}Aïn, Rimmon, Ether, et Aschan ; quatre villes et leurs villages ;
\VS{8}et tous les villages qui étaient autour de ces villes-là jusqu'à Baalath-Beer, qui est Ramath du midi. Tel fut l'héritage de la tribu des fils de Siméon, selon leurs familles.
\VS{9}L'héritage des fils de Siméon fut pris sur la portion des fils de Juda ; car la portion des fils de Juda était trop grande pour eux ; c'est pourquoi les fils de Siméon reçurent leur héritage parmi le leur.
\TextTitle{Le territoire de Zabulon}
\VS{10}La troisième part échut par le sort aux fils de Zabulon, selon leurs familles.
\VS{11}Et leur frontière devait monter vers le quartier devers la mer, même jusqu'à Mareala, puis se rencontrer à Dabbéscheth, et de là au torrent qui est vis-à-vis de Jokneam.
\VS{12}Or cette frontière devait retourner vers Sarid à l'orient, vers le soleil levant, jusqu'à la frontière de Kisloth-Thabor, puis continuer à Dabrath, et monter à Japhia.
\VS{13}De là passer à l'orient, par Guittha-Hépher, par Ittha-Katsin, puis continuer à Rimmon, jusqu'à Néa.
\VS{14}Puis cette frontière devait tourner du côté du nord vers Hannathon, et ses extrémités devaient se rendre à la vallée de Jiphthach-El.
\VS{15}Avec Katthath, Nahalal, Schimron, Jideala, et Bethléhem ; il y avait douze villes et leurs villages.
\VS{16}Tel fut l'héritage des fils de Zabulon selon leurs familles, ces villes-là, et leurs villages.
\TextTitle{Le territoire d'Issacar}
\VS{17}La quatrième part échut par le sort à Issacar, aux fils d'Issacar, selon leurs familles.
\VS{18}Et leur frontière devaient passer par Jizreel, Kesulloth, Sunem,
\VS{19}Hapharaïm, Schion, Anacharath,
\VS{20}Rabbith, Kischjon, Abets,
\VS{21}Rémeth, En-Gannim, En-Hadda et Beth-Patsets ;
\VS{22}elle devait se rencontrer à Thabor, et vers Schachatsima et Beth-Schémesch, et les extrémités de leur frontière devaient se rendre au Jourdain. Seize villes et leurs villages.
\VS{23}Tel fut l'héritage de la tribu des fils d'Issacar, selon leurs familles, ces villes-là et leurs villages.
\TextTitle{Le territoire d'Aser}
\VS{24}La cinquième part échut par le sort à la tribu des fils d'Aser, selon leurs familles.
\VS{25}Et leur frontière fut Helkath, Hali, Béthen, Acschaph,
\VS{26}Allammélec, Amead et Mischeal ; et elle devait se rencontrer à Carmel, au quartier vers la mer, et à Schichor-Libnath.
\VS{27}Puis elle devait retourner vers l'orient, à Beth-Dagon, et se rencontrer à Zabulon, et à la vallée de Jiphthach-El, vers le nord de Beth-Emek et de Neïel, puis sortir vers Cabul, à gauche,
\VS{28}et vers Ebron, Rehob, Hammon et Kana, jusqu'à Sidon la grande.
\VS{29}Puis la frontière devait retourner à Rama, jusqu'à la ville forte de Tyr, et cette frontière devait retourner à Hosa ; de sorte que ses extrémités se rencontrent au quartier qui est vers la mer, par la contrée d'Aczib.
\VS{30}Avec Umma, Aphek et Rehob ; vingt-deux villes et leurs villages.
\VS{31}Tel fut l'héritage de la tribu des fils d'Aser, selon leurs familles ; ces villes-là et leurs villages.
\TextTitle{Le territoire de Nephthali}
\VS{32}La sixième part échut par le sort aux fils de Nephthali, selon leurs familles.
\VS{33}Leur frontière fut depuis Héleph, depuis Allon par Tsaanannim, Adami-Nékeb et Jabneel, jusqu'à Lakkum, et ses extrémités devaient se rendre au Jourdain.
\VS{34}Puis cette frontière devait retourner du côté d'occident, vers Aznoth-Thabor, et sortir de là à Hukkok ; de sorte que du côté du midi elle devait se rencontrer à Zabulon, et du côté d'occident elle devait se rencontrer à Aser et à Juda ; le Jourdain était du côté au soleil levant.
\VS{35}Au reste, les villes fortifiées étaient : Tsiddim, Tser, Hammath, Rakkath, Kinnéreth,
\VS{36}Adama, Rama, Hatsor,
\VS{37}Kédesch, Edréï, En-Hatsor,
\VS{38}Jireon, Migdal-El, Horem, Beth-Anath et Beth-Schémesch ; dix-neuf villes et leurs villages.
\VS{39}Tel fut l'héritage de la tribu des fils de Nephthali, selon leurs familles ; ces villes-là, et leurs villages.
\TextTitle{Le territoire de Dan}
\VS{40}La septième part échut par le sort à la tribu des fils de Dan selon leurs familles.
\VS{41}La limite de leur héritage fut, Tsorea, Eschthaol, Ir-Schémesch,
\VS{42}Schaalabbin, Ajalon, Jithla,
\VS{43}Elon, Thimnatha, Ekron,
\VS{44}Eltheké, Guibbethon, Baalath,
\VS{45}Jehud, Bené-Berak, Gath-Rimmon,
\VS{46}Mé-Jarkon et Rakkon, avec le territoire qui est vis-à-vis de Japho.
\VS{47}Le territoire échu aux fils de Dan était trop petit pour eux. C'est pourquoi les fils de Dan montèrent, et combattirent contre Léschem ; ils s'en emparèrent et la frappèrent du tranchant de l'épée ; ils en prirent possession, s'y établirent, et l'appelèrent Léschem, Dan, du nom de Dan leur père.
\VS{48}Tel fut l'héritage de la tribu des fils de Dan selon leurs familles ; ces villes-là et leurs villages.
\TextTitle{Josué reçoit Thimnath-Sérach}
\VS{49}Après qu'on eut achevé de partager le pays selon ses frontières, les enfants d'Israël donnèrent à Josué, fils de Nun, une possession au milieu d'eux.
\VS{50}Selon l'ordre de Yahweh, ils lui donnèrent la ville qu'il demanda, Thimnath-Sérach, dans la montagne d'Ephraïm. Il rebâtit la ville, et y habita.
\VS{51}Ce sont là les héritages que le prêtre Eléazar, Josué, fils de Nun, et les chefs de pères des tribus des enfants d'Israël partagèrent par le sort à Silo, devant Yahweh, à l'entrée de la tente d'assignation, et ils achevèrent ainsi le partage du pays.
\Chap{20}
\TextTitle{Les six villes de refuge\FTNTT{No. 35.}}
\VerseOne{}Puis Yahweh parla à Josué et dit :
\VS{2}Parle aux enfants d'Israël et dis : Etablissez-vous des villes de refuge comme je vous l'ai ordonné par le moyen de Moïse,
\VS{3}où pourra s'enfuir le meurtrier qui aura tué quelqu'un involontairement, sans intention, et elles vous serviront de refuge devant celui qui a le droit de venger le sang.
\VS{4}Et le meurtrier s'enfuira dans l'une de ces villes, s'arrêtera à l'entrée de la porte de la ville, et il exposera son affaire aux anciens de cette ville-là, ils l'écouteront, et le recevront chez eux dans la ville, et lui donneront une demeure, afin qu'il habite avec eux.
\VS{5}Et quand celui qui a le droit de venger le sang le poursuivra, ils ne livreront pas le meurtrier entre ses mains ; puisque c'est involontairement qu'il a tué son prochain, et qu'il ne le haïssait point auparavant.
\VS{6}Mais il demeurera dans cette ville-là, jusqu'à ce qu'il comparaisse devant l'assemblée pour être jugé, jusqu'à la mort du grand prêtre qui sera en fonction en ce temps-là. Alors le meurtrier s'en retournera, et reviendra dans sa ville et dans sa maison, dans la ville d'où il s'était enfui\FTNT{Ex. 21:13 ; No. 35:9-34 ; De. 19.}.
\VS{7}Ils consacrèrent donc Kédesch, en Galilée, dans la montagne de Nephthali ; Sichem dans la montagne d'Ephraïm ; et Kirjath-Arba, qui est Hébron, dans la montagne de Juda.
\VS{8}Et de l'autre côté du Jourdain, à l'orient de Jéricho, ils choisirent Betser, dans la tribu de Ruben, dans le désert, dans la plaine ; Ramoth en Galaad, dans la tribu de Gad ; et Golan en Basan, dans la tribu de Manassé\FTNT{De. 4:43.}.
\VS{9}Telles furent les villes désignées pour tous les enfants d'Israël et pour l'étranger en séjour au milieu d'eux, afin que quiconque aurait tué quelqu'un involontairement puisse s'y réfugier, et qu'il ne meure pas de la main de celui qui a le droit de venger le sang, avant d'avoir comparu devant l'assemblée.
\Chap{21}
\TextTitle{Les quarante-huit villes des Lévites}
\VerseOne{}Or les chefs des pères de famille des Lévites s'approchèrent d'Eléazar, le prêtre, de Josué, fils de Nun, et des chefs des pères de famille des tribus des enfants d'Israël.
\VS{2}Et leur parlèrent à Silo, dans le pays de Canaan, et dirent : Yahweh a ordonné par Moïse qu'on nous donne des villes pour habiter, et leurs faubourgs pour nos bêtes\FTNT{No. 35:2-3.}.
\VS{3}Alors les enfants d'Israël donnèrent aux Lévites, sur leur héritage, les villes suivantes et leurs faubourgs, d'après l'ordre de Yahweh.
\VS{4}Et on tira au sort pour les familles des Kehathites ; et les Lévites, fils d'Aaron, le prêtre eurent par le sort treize villes de la tribu de Juda, de la tribu de Siméon, et de la tribu de Benjamin.
\VS{5}Les autres fils de Kehath eurent par le sort dix villes des familles de la tribu d'Ephraïm, de la tribu de Dan, et de la demi-tribu de Manassé.
\VS{6}Et les fils de Guerschon eurent par le sort treize villes, des familles de la tribu d'Issacar, de la tribu d'Aser, de la tribu de Nephthali, et de la demi-tribu de Manassé en Basan.
\VS{7}Et les fils de Merari selon leurs familles, eurent douze villes, de la tribu de Ruben, de la tribu de Gad, et de la tribu de Zabulon.
\VS{8}Les enfants d'Israël donnèrent donc par le sort aux Lévites ces villes-là avec leurs faubourgs, comme Yahweh l'avait ordonné par Moïse.
\VS{9}Ils donnèrent donc de la tribu des fils de Juda et de la tribu des fils de Siméon, ces villes, qui vont être nommées par leurs noms,
\VS{10}et qui furent pour les fils d'Aaron, qui étaient des familles des Kehathites, et des fils de Lévi, car le sort les avait indiqués les premiers.
\VS{11}Ils leur donnèrent Kirjath-Arba, qui est Hébron, dans la montagne de Juda, avec ses faubourgs tout autour : Arba était le père d'Anak.
\VS{12}Mais quant au territoire de la ville, et à ses villages, on les donna à Caleb, fils de Jephunné, pour sa possession.
\VS{13}Ils donnèrent donc aux fils d'Aaron, le prêtre, les villes de refuge pour les meurtriers, Hébron, avec ses faubourgs, et Libna avec ses faubourgs.
\VS{14}Jatthir, avec ses faubourgs, Eschthemoa, avec ses faubourgs,
\VS{15}Holon, avec ses faubourgs, Debir, avec ses faubourgs,
\VS{16}Aïn, avec ses faubourgs, Jutta, avec ses faubourgs ; et Beth-Schémesch, avec ses faubourgs ; neuf villes de ces deux tribus-là ;
\VS{17}et de la tribu de Benjamin, Gabaon, avec ses faubourgs, et Guéba, avec ses faubourgs,
\VS{18}Anathoth, avec ses faubourgs, et Almon, avec ses faubourgs ; quatre villes.
\VS{19}Toutes les villes des prêtres, fils d'Aaron, furent treize villes, avec leurs faubourgs.
\VS{20}Quant aux Lévites, appartenant aux familles des autres fils de Kehath, ils eurent par le sort des villes de la tribu d'Ephraïm.
\VS{21}On leur donna donc les villes de refuge pour les meurtriers, Sichem, avec ses faubourgs, dans la montagne d'Ephraïm, et Guézer avec ses faubourgs ;
\VS{22}Kibtsaïm, avec ses faubourgs, et Beth-Horon, avec ses faubourgs ; quatre villes ;
\VS{23}et de la tribu de Dan, Eltheké, avec ses faubourgs ; Guibbethon, avec ses faubourgs,
\VS{24}Ajalon, avec ses faubourgs, Gath-Rimmon, avec ses faubourgs ; quatre villes.
\VS{25}Et de la demi-tribu de Manassé, Thaanac, avec ses faubourgs ; et Gath-Rimmon, avec ses faubourgs, deux villes.
\VS{26}Total des villes : Dix villes avec leurs faubourgs, pour les familles des autres fils de Kehath.
\VS{27}On donna aussi aux fils de Guerschon, d'entre les familles des Lévites : De la demi-tribu de Manassé les villes de refuge pour les meurtriers, Golan en Basan, avec ses faubourgs, et Beeschthra, avec ses faubourgs ; deux villes ;
\VS{28}et de la tribu d'Issacar, Kischjon, avec ses faubourgs, Dabrath, avec ses faubourgs,
\VS{29}Jarmuth, avec ses faubourgs, En-Gannim, avec ses faubourgs ; quatre villes ;
\VS{30}et de la tribu d'Aser, Mischeal, avec ses faubourgs, Abdon, avec ses faubourgs,
\VS{31}Helkath, avec ses faubourgs, et Rehob, avec ses faubourgs ; quatre villes ;
\VS{32}et de la tribu de Nephthali, les villes de refuge pour les meurtriers, Kédesch en Galilée avec ses faubourgs, Hammoth-Dor, avec ses faubourgs, et Karthan, avec ses faubourgs ; trois villes.
\VS{33}Total des villes des Guerschonites, selon leurs familles : Treize villes, et leurs faubourgs.
\VS{34}On donna aussi au reste des Lévites, qui appartenaient aux familles des fils de Merari : De la tribu de Zabulon, Jokneam, avec ses faubourgs, Kartha, avec ses faubourgs,
\VS{35}Dimna, avec ses faubourgs, et Nahalal, avec ses faubourgs ; quatre villes ;
\VS{36}et de la tribu de Ruben, Betser, avec ses faubourgs, et Jahtsa, avec ses faubourgs ;
\VS{37}Kedémoth, avec ses faubourgs, et Méphaath, avec ses faubourgs ; quatre villes ;
\VS{38}et de la tribu de Gad, les villes de refuge pour les meurtriers, Ramoth en Galaad, avec ses faubourgs, et Mahanaïm, avec ses faubourgs,
\VS{39}Hesbon, avec ses faubourgs, et Jaezer, avec ses faubourgs ; en tout quatre villes.
\VS{40}Total des villes qui échurent par le sort aux fils de Merari, selon leurs familles, formant le reste des familles des Lévites : Douze villes.
\VS{41}Total des villes des Lévites qui étaient parmi la possession des enfants d'Israël : Quarante-huit villes, et leurs faubourgs.
\VS{42}Chacune de ces villes avait ses faubourgs autour d'elle ; il en était ainsi de toutes ces villes-là.
\TextTitle{Yahweh accomplit sa promesse}
\VS{43}Yahweh donna donc à Israël tout le pays qu'il avait juré de donner à leurs pères ; ils le possédèrent, et y habitèrent\FTNT{Dieu accomplit toujours ses promesses (Jé. 1:12).}.
\VS{44}Yahweh leur accorda un parfait repos tout autour, selon tout ce qu'il avait juré à leurs pères ; aucun de leurs ennemis ne put leur résister, car Yahweh les livra entre leurs mains.
\VS{45}Il ne tomba pas un seul mot de toutes les bonnes paroles que Yahweh avait dites à la maison d'Israël : Toutes s'accomplirent.
\Chap{22}
\TextTitle{Ruben, Gad et la demi-tribu de Manassé retournent sur leurs terres}
\VerseOne{}Alors Josué appela les Rubénites, les Gadites et la demi-tribu de Manassé.
\VS{2}Et il leur dit : Vous avez gardé tout ce que Moïse, serviteur de Yahweh, vous a prescrit, et vous avez obéi à ma voix dans tout ce que je vous ai ordonné.
\VS{3}Vous n'avez pas abandonné vos frères, depuis une très longue période jusqu'à ce jour ; et vous avez gardé les ordres, les commandements de Yahweh votre Dieu.
\VS{4}Maintenant que Yahweh, votre Dieu, a donné du repos à vos frères, comme il le leur avait dit, retournez et allez dans vos tentes, dans le pays qui vous appartient, et que Moïse, serviteur de Yahweh, vous a donné de l'autre côté du Jourdain\FTNT{No. 32:33 ; De. 3:13 ; De. 29:8.}.
\VS{5}Prenez seulement bien garde d'observer les ordonnances et les lois que Moïse, serviteur de Yahweh, vous a prescrites : Aimez Yahweh votre Dieu, marchez dans toutes ses voies, gardez ses commandements, attachez-vous à lui, et servez-le de tout votre cœur et de toute votre âme\FTNT{De. 10:12.}.
\VS{6}Puis Josué les bénit et les renvoya ; et ils s'en allèrent vers leurs tentes.
\VS{7}Moïse avait donné à la moitié de la tribu de Manassé son héritage en Basan ; et Josué donna à l'autre moitié son héritage avec leurs frères de l'autre côté du Jourdain vers l'occident. Josué les renvoya dans leurs tentes, et les bénit.
\VS{8}Et il leur parla et dit : Vous retournez à vos tentes avec de grandes richesses, une très nombreuse quantité de bétail, avec une quantité considérable d'argent, d'or, d'airain, de fer, et de vêtements. Partagez avec vos frères le butin de vos ennemis.
\VS{9}Ainsi donc les fils de Ruben, les fils de Gad, et la demi-tribu de Manassé s'en retournèrent, et partirent de Silo, dans le pays de Canaan, après avoir quitté les enfants d'Israël, pour s'en aller dans le pays de Galaad, sur la terre de leur possession et où ils s'établirent, suivant ce que Yahweh avait ordonné par Moïse.
\TextTitle{L'autel Ed, sujet d'incompréhension}
\VS{10}Quand ils furent arrivés aux frontières du Jourdain, qui appartiennent au pays de Canaan, les fils de Ruben, les fils de Gad, et la demi-tribu de Manassé y bâtirent un autel, près du Jourdain, un autel dont la grandeur frappait les regards.
\VS{11}Les enfants d'Israël apprirent que l'on disait : Voici, les fils de Ruben, les fils de Gad, et la demi-tribu de Manassé ont bâti un autel en face du pays de Canaan, sur les frontières du Jourdain, du côté des enfants d'Israël.
\VS{12}Lorsque les enfants d'Israël entendirent cela, toute l'assemblée des enfants d'Israël se réunit à Silo, pour monter en guerre contre eux.
\VS{13}Cependant les enfants d'Israël envoyèrent vers les fils de Ruben, vers les fils de Gad, et vers la demi-tribu de Manassé, au pays de Galaad, Phinées, fils du prêtre Eléazar,
\VS{14}et avec lui dix princes, un prince par maison paternelle pour chacune des tribus d'Israël ; tous étaient chefs de maison paternelle parmi les milliers d'Israël.
\VS{15}Ils se rendirent auprès des fils de Ruben, des fils de Gad et de la demi-tribu de Manassé au pays de Galaad, et leur parlèrent, en disant :
\VS{16}Ainsi parle toute l'assemblée de Yahweh : Quelle est cette infidélité que vous avez commise contre le Dieu d'Israël, et pourquoi vous détournez-vous aujourd'hui de Yahweh, en vous bâtissant un autel, pour vous rebeller aujourd'hui contre Yahweh ?
\VS{17}Regardons-nous comme peu de chose l'iniquité de Peor\FTNT{Peor : No. 25:1-9.}, dont nous ne nous sommes pas encore bien purifiés jusqu'à présent, malgré la plaie qu'il attira sur l'assemblée de Yahweh ?
\VS{18}Et vous vous détournez aujourd'hui de Yahweh ! Si vous vous rebellez aujourd'hui contre Yahweh, demain il s'irritera contre toute l'assemblée d'Israël.
\VS{19}Si vous tenez pour impure la terre qui est votre propriété, passez sur la terre qui est la possession de Yahweh, où est fixé le tabernacle de Yahweh, ayez votre possession parmi nous, mais ne vous révoltez point contre Yahweh, et ne soyez point rebelles contre nous, en vous bâtissant un autel, outre l'autel de Yahweh notre Dieu.
\VS{20}Acan\FTNT{Acan : Jos. 7:1-26.}, fils de Zérach, ne commit-il pas une infidélité en prenant des choses dévouées par le moyen de l'interdit, et la colère de Yahweh ne s'enflamma-t-elle pas contre toute l'assemblée d'Israël ? Cependant, cet homme ne fut pas le seul qui périt à cause de son iniquité.
\VS{21}Mais les fils de Ruben, les fils de Gad, et la demi-tribu de Manassé répondirent, et dirent aux chefs des milliers d'Israël :
\VS{22}Dieu\FTNT{Dieu : de l'hébreu « El » : puissant, etc.}, Dieu\FTNT{Dieu : de l'hébreu « elohim » : juge, ange.} Yahweh, Dieu\FTNT{Dieu : de l'hébreu « El » : puissant, etc.}, Dieu\FTNT{Dieu : de l'hébreu « elohim » : juge, ange.}Yahweh, le sait, et Israël lui-même le saura ! Si c'est par rébellion et par infidélité envers Yahweh, alors qu'il ne nous vienne point en aide aujourd'hui.
\VS{23}Si nous nous sommes bâti un autel pour nous détourner de Yahweh, si c'est pour y offrir des holocaustes, ou des offrandes, ou si c'est pour y faire des sacrifices d'offrande de paix, que Yahweh lui-même nous en demande compte !
\VS{24}C'est bien plutôt par une sorte d'inquiétude que nous avons fait cela, en pensant que vos fils pourraient un jour parler à nos fils et leur dire : Qu'y a-t-il de commun entre vous et Yahweh, le Dieu d'Israël ?
\VS{25}Puisque Yahweh a mis le Jourdain pour frontière entre nous et vous, fils de Ruben, et fils de Gad ; vous n'avez point de part à Yahweh ! Et ainsi vos fils feraient qu'un jour nos fils cesseraient de craindre Yahweh\FTNT{Né. 2:20 ; Ac. 8:21.}.
\VS{26}C'est pourquoi nous avons dit : Mettons-nous maintenant à bâtir un autel, non pour des holocaustes ni pour des sacrifices ;
\VS{27}mais afin qu'il serve de témoignage entre nous et vous, et entre nos descendants et les vôtres, que nous voulons servir Yahweh devant sa face par nos holocaustes et nos sacrifices d'expiation et d'offrande de paix, afin que vos fils ne disent pas un jour à nos fils : Vous n'avez point de part à Yahweh\FTNT{Ge. 31:48.} !
\VS{28}C'est pourquoi nous avons dit : Lorsqu'ils nous tiendront ce discours, ou à nos descendants, nous leur dirons : Voyez la forme de l'autel de Yahweh qu'ont fait nos pères, non pour des holocaustes, ni pour des sacrifices, mais afin qu'il soit témoin entre nous et vous.
\VS{29}A Dieu ne plaise que nous nous révoltions contre Yahweh et que nous nous détournions aujourd'hui de Yahweh, en bâtissant un autel pour des holocaustes, pour des offrandes, et pour des sacrifices, outre l'autel de Yahweh notre Dieu, qui est devant son tabernacle !
\VS{30}Or, après que le prêtre Phinées, et les princes de l'assemblée, les chefs des milliers d'Israël qui étaient avec lui, eurent entendu les paroles que les fils de Ruben, les fils de Gad, et les fils de Manassé leur dirent, ils furent satisfaits.
\VS{31}Et Phinées, fils du prêtre Eléazar, dit aux fils de Ruben, aux fils de Gad, et aux fils de Manassé : Nous reconnaissons aujourd'hui que Yahweh est au milieu de nous, puisque vous n'avez point commis cette infidélité contre Yahweh ; vous avez ainsi délivré les enfants d'Israël de la main de Yahweh.
\VS{32}Ainsi Phinées, fils du prêtre Eléazar, et les princes, quittèrent les fils de Ruben, les fils de Gad, et revinrent du pays de Galaad dans le pays de Canaan, auprès des enfants d'Israël, auxquels ils firent un rapport.
\VS{33}Et la chose plut aux enfants d'Israël ; ils bénirent Dieu, et ne parlèrent plus de monter en armes contre eux pour détruire le pays où habitaient les fils de Ruben, et les fils de Gad.
\VS{34}Les fils de Ruben, et les fils de Gad appelèrent l'autel Ed ; car, dirent-ils, il est témoin entre nous que Yahweh est Dieu.
\Chap{23}
\TextTitle{Avertissements de Josué}
\VerseOne{}Or il arriva, plusieurs jours après, que Yahweh ayant donné du repos à Israël de tous les ennemis qui l'entouraient, Josué était vieux, fort avancé en âge.
\VS{2}Et Josué convoqua tout Israël, ses anciens, ses chefs, ses juges, ses officiers, et leur dit : Je suis devenu vieux, fort avancé en âge.
\VS{3}Vous avez vu tout ce que Yahweh, votre Dieu, a fait à toutes ces nations devant vous ; car Yahweh, votre Dieu, est celui qui combat pour vous.
\VS{4}Voyez, je vous ai donné en héritage par le sort, selon vos tribus, ces nations qui sont restées, depuis le Jourdain, et toutes les nations que j'ai exterminées, jusqu'à la grande mer vers le soleil couchant.
\VS{5}Yahweh, votre Dieu, les repoussera devant vous et les chassera ; et vous posséderez leur pays en héritage, comme Yahweh, votre Dieu, vous l'a dit\FTNT{Ex. 14:14 ; Ex. 23:27 ; No. 33:53 ; De. 6:18-19.}.
\VS{6}Appliquez-vous avec force à observer et à mettre en pratique tout ce qui est écrit dans le livre de la loi de Moïse, sans vous en détourner ni à droite ni à gauche\FTNT{De. 5:32 ; De. 28:14.}.
\VS{7}Ne vous mêlez point avec ces nations qui sont restées parmi vous ; et ne faites point mention du nom de leurs dieux, et ne faites jurer personne par eux, ne les servez point, et ne vous prosternez point devant eux\FTNT{Ex. 23:13 ; De. 12:3 ; Jé. 5:7 ; De. 6:14.}.
\VS{8}Mais attachez-vous à Yahweh, votre Dieu, comme vous l'avez fait jusqu'à ce jour\FTNT{De. 11:22.}.
\VS{9}C'est pour cela que Yahweh a chassé devant vous des nations grandes et puissantes ; nul n'a pu vous résister jusqu'à ce jour.
\VS{10}Un seul homme d'entre vous en poursuivait mille ; car Yahweh votre Dieu est celui qui combat pour vous, comme il vous l'a dit\FTNT{Lé. 26:8 ; De. 32:30.}.
\VS{11}Veillez donc attentivement sur vos âmes, afin d'aimer Yahweh, votre Dieu.
\VS{12}Autrement, si vous vous détournez et que vous vous attachez au reste de ces nations qui sont demeurées parmi vous, si vous faites alliance par des mariages avec elles, et si vous formez ensemble des relations,
\VS{13}sachez certainenement que Yahweh, votre Dieu, ne continuera pas à chasser ces nations devant vous ; mais elles seront pour vous un piège et un filet, un fouet dans vos côtés et des épines dans vos yeux, jusqu'à ce que vous ayez péri de dessus cette bonne terre que Yahweh, votre Dieu, vous a donnée\FTNT{Ex. 23:33 ; De. 7:16 ; Jg. 2:3.}.
\VS{14}Voici, je m'en vais aujourd'hui par le chemin de toute la terre. Reconnaissez de tout votre cœur et de toute votre âme qu'aucune de toutes les bonnes paroles prononcées sur vous par Yahweh, votre Dieu, n'est restée sans effet ; toutes se sont accomplies pour vous, aucune n'est restée sans effet\FTNT{Jos. 21:45 ; 2 R. 10:10.}.
\VS{15}Et il arrivera que comme toutes les bonnes paroles que Yahweh, votre Dieu, vous a dites vous sont arrivées ; ainsi Yahweh fera venir sur vous toutes les paroles mauvaises, jusqu'à ce qu'il vous ait exterminés de dessus cette bonne terre que Yahweh, votre Dieu, vous a donnée.
\VS{16}Si vous transgressez l'alliance que Yahweh, votre Dieu, vous a prescrite, et si vous allez servir d'autres dieux et vous prosterner devant eux, la colère de Yahweh s'enflammera contre vous, et vous périrez promptement de dessus cette bonne terre qu'il vous a donnée.
\Chap{24}
\TextTitle{Josué rappelle à Israël son histoire}
\VerseOne{}Josué assembla toutes les tribus d'Israël à Sichem, et il convoqua les anciens d'Israël, ses chefs, ses juges, et ses officiers, qui se présentèrent devant Dieu.
\VS{2}Et Josué dit à tout le peuple : Ainsi parle Yahweh, le Dieu d'Israël : Vos pères, Térach père d'Abraham, et père de Nachor, ont anciennement habité de l'autre côté du fleuve, où ils servaient d'autres dieux.
\VS{3}Mais j'ai pris votre père Abraham de l'autre côté du fleuve, je lui fis parcourir tout le pays de Canaan, je multipliai sa postérité, et lui donnai Isaac\FTNT{Ge. 12 ; Ge. 21:2.}.
\VS{4}Je donnai à Isaac, Jacob et Esaü ; et je donnai à Esaü le mont de Séir, pour le posséder ; mais Jacob et ses fils descendirent en Egypte\FTNT{Ge. 25:24 ; Ge. 36:6.}.
\VS{5}Puis j'envoyai Moïse et Aaron, et je frappai l'Egypte, par les prodiges que j'opérai au milieu d'elle ; puis je vous en fis sortir\FTNT{Ex. 3:10.}.
\VS{6}Je fis donc sortir vos pères hors de l'Egypte, et vous arrivâtes à la mer. Les Egyptiens poursuivirent vos pères avec des chars et des cavaliers, jusqu'à la Mer Rouge\FTNT{Ex. 14:9.}.
\VS{7}Alors ils crièrent à Yahweh. Et il mit des ténèbres entre vous et les Egyptiens, et ramena sur eux la mer, qui les couvrit. Vos yeux ont vu ce que j'ai fait aux Egyptiens. Puis vous restâtes longtemps dans le désert.
\VS{8}Ensuite je vous conduisis dans le pays des Amoréens, qui habitaient de l'autre côté du Jourdain, et ils combattirent contre vous. Mais je les livrai entre vos mains ; vous prîtes possession de leur pays, et je les détruisis devant vous.
\VS{9}Balak\FTNT{Balaak : Voir No. 22:2-14.} aussi, fils de Tsippor, roi de Moab, se leva, et fit la guerre à Israël. Il fit appeler Balaam\FTNT{Balaam : Voir No. 22.}, fils de Beor, pour qu'il vous maudisse.
\VS{10}Mais je ne voulus point écouter Balaam ; il s'agenouilla et vous bénit, et je vous délivrai de la main de Balak.
\VS{11}Et vous passâtes le Jourdain, et arrivâtes près de Jéricho. Les habitants de Jéricho, les Amoréens, les Phéréziens, les Cananéens, les Héthiens, les Guirgasiens, les Héviens et les Jébusiens vous firent la guerre. Je les livrai entre vos mains,
\VS{12}et j'envoyai devant vous des frelons qui les chassèrent loin de votre face, comme les deux rois des Amoréens : Ce ne fut ni par ton épée, ni par ton arc\FTNT{Ex. 23:28 ; De. 7:20.}.
\VS{13}Je vous donnai une terre que vous n'aviez point cultivée, des villes que vous n'aviez point bâties, et que vous habitez, et vous mangez les fruits des vignes et des oliviers que vous n'avez point plantés\FTNT{De. 6:10 ; Ps. 105:44 ; Né. 9:25.}.
\TextTitle{Le peuple choisit de servir Yahweh}
\VS{14}Maintenant, craignez Yahweh, et servez-le avec intégrité et avec fidélité. Ôtez les dieux que vos pères ont servis de l'autre côté du fleuve et en Egypte, et servez Yahweh\FTNT{1 S. 12:23-24 ; Ez. 20:7-44.}.
\VS{15}Et s'il vous déplaît de servir Yahweh, choisissez aujourd'hui qui vous voulez servir, ou les dieux que servaient vos pères au-delà du fleuve, ou les dieux des Amoréens dans le pays desquels vous habitez. Mais moi et ma maison, nous servirons Yahweh.
\VS{16}Alors le peuple répondit, et dit : Que Dieu nous garde d'abandonner Yahweh pour servir d'autres dieux !
\VS{17}Car Yahweh, notre Dieu, est celui qui nous a fait monter, nous et nos pères, hors du pays d'Egypte, de la maison de servitude, qui a fait devant nos yeux ces grands signes, qui nous a gardés dans tout le chemin par lequel nous avons marché, et entre tous les peuples parmi lesquels nous avons passé.
\VS{18}Yahweh a chassé devant nous tous les peuples, et même les Amoréens qui habitaient ce pays. Nous servirons aussi Yahweh, car il est notre Dieu.
\VS{19}Josué dit au peuple : Vous ne pourrez pas servir Yahweh, car c'est un Dieu Saint, qui est jaloux, il ne pardonnera point votre rébellion et vos péchés.
\VS{20}Lorsque vous abandonnerez Yahweh et que vous servirez les dieux des étrangers, il reviendra vous faire du mal, et il vous consumera après vous avoir fait du bien.
\VS{21}Le peuple dit à Josué : Non ! Car nous servirons Yahweh.
\VS{22}Et Josué dit au peuple : Vous êtes témoins contre vous-mêmes que c'est vous qui avez choisi Yahweh pour le servir. Et ils répondirent : Nous en sommes témoins.
\VS{23}Maintenant donc ôtez les dieux étrangers qui sont au milieu de vous, et tournez votre cœur vers Yahweh, le Dieu d'Israël.
\VS{24}Et le peuple répondit à Josué : Nous servirons Yahweh notre Dieu et nous obéirons à sa voix.
\VS{25}Ce jour-là, Josué traita alliance avec le peuple, et lui donna des lois et des ordonnances à Sichem.
\VS{26}Josué écrivit ces paroles dans le livre de la loi de Dieu. Il prit aussi une grande pierre\FTNT{Cette Pierre entend selon Josué, elle est également appelée « témoin ». Jésus-Christ, la Pierre angulaire (Es. 8:13-16) est le témoin fidèle (Ap. 19:11). Cette Pierre suivait les Hébreux dans le désert (1 Co. 10:1-3).}, qu'il dressa là sous le chêne qui était dans le lieu consacré à Yahweh.
\VS{27}Josué dit à tout le peuple : Voici, cette pierre servira de témoin contre nous, car elle a entendu toutes les paroles que Yahweh nous a déclarées ; elle servira de témoin contre vous, afin que vous ne reniiez pas votre Dieu.
\VS{28}Puis Josué renvoya le peuple, chacun dans son héritage.
\TextTitle{Mort de Josué et d'Eléazar ; ensevelissement des os de Joseph (Ge. 50 :26)}
\VS{29}Or il arriva, après ces choses, que Josué, fils de Nun, serviteur de Yahweh, mourut, âgé de cent dix ans.
\VS{30}Et on l'ensevelit dans le territoire de son héritage, à Thimnath-Sérach, dans la montagne d'Ephraïm, du côté du nord de la montagne de Gaasch.
\VS{31}Et Israël servit Yahweh tout le temps de Josué, et tout le temps des anciens qui survécurent à Josué, qui avaient connu toutes les œuvres que Yahweh avait faites pour Israël.
\VS{32}Les os de Joseph\FTNT{(Ge. 50:25 ; Ex. 13:19 ; Hé. 11:22).}, que les enfants d'Israël avaient rapportés d'Egypte, furent ensevelis à Sichem, dans la portion du champ que Jacob avait achetée des fils de Hamor, père de Sichem, pour cent kesita, et qui appartint à l'héritage des fils de Joseph.
\VS{33}Et Eléazar, fils d'Aaron, mourut, on l'enterra à Guibeath-Phinées, qui avait été donnée à son fils Phinées, dans la montagne d'Ephraïm.
\PPE{}
\end{multicols}

%\clearpage\ShortTitle{Juges}\BookTitle{Juges}\BFont
\noindent\hrulefill
{\footnotesize
\textit{
\bigskip
{\centering{}
\\Auteur : Inconnu
\\(Heb. : Shoftim)
\\Signification : Être juge, prononcer, punir
\\Thème : Défaites et délivrances
\\Date de rédaction : Environ 1100 av. J.-C.\\}
}
%\bigskip
\textit{
\\A la mort de Josué et des anciens, il s’éleva en Israël une nouvelle génération qui n'avait pas connu l’expérience du désert. Elle fit ce qui est mal aux yeux de Dieu, l’abandonna et tomba dans l’idolâtrie. Ainsi, la colère de Yahweh s’abattit sur Israël et il livra le peuple entre les mains de ses ennemis. Dans ces temps de troubles, Dieu suscita des juges - douze hommes et une femme - pour délivrer Israël de ses oppresseurs. Aussi longtemps que le juge était en vie, Israël était en paix. Mais dès qu’il venait à mourir, le peuple se corrompait de nouveau et ses oppressions recommençaient.\bigskip
}
}
\par\nobreak\noindent\hrulefill
\begin{multicols}{2}
\Chap{1}
\TextTitle{Poursuite de la conquête de Canaan}
\VerseOne{}Or il arriva qu'après la mort de Josué, les enfants d'Israël consultèrent Yahweh, en disant : Qui de nous montera le premier contre les Cananéens pour leur faire la guerre ?
\VS{2}Et Yahweh répondit : Juda montera ; voici, j'ai livré le pays entre ses mains.
\VS{3}Juda dit à Siméon son frère : Monte avec moi dans mon lot et nous ferons la guerre aux Cananéens ; et j'irai aussi avec toi dans ton lot. Ainsi Siméon alla avec lui.
\TextTitle{Victoires de Juda ; Caleb prend possession d’Hébron}
\VS{4}Juda monta, et Yahweh livra les Cananéens et les Phéréziens entre leurs mains ; ils battirent dix mille hommes à Bézek.
\VS{5}Et ils trouvèrent Adoni-Bézek à Bézek ; ils l'attaquèrent et frappèrent les Cananéens et les Phéréziens.
\VS{6}Adoni-Bézek s'enfuit mais ils le poursuivirent ; et l'ayant pris, ils lui coupèrent les pouces des mains et des pieds.
\VS{7}Alors Adoni-Bézek dit : Soixante-dix rois, dont les pouces des mains et des pieds avaient été coupés, ramassaient du pain sous ma table ; Dieu me rend ce que j’ai fait. On l’amena à Jérusalem et il y mourut\FTNT{Es. 33:1}.
\VS{8}Les fils de Juda firent la guerre contre Jérusalem et la prirent, ils frappèrent ses habitants du tranchant de l'épée et mirent le feu à la ville.
\VS{9}Puis les fils de Juda descendirent pour faire la guerre aux Cananéens, qui habitaient la montagne, la contrée du midi et la plaine.
\VS{10}Juda marcha contre les Cananéens qui habitaient à Hébron ; or le nom d'Hébron était auparavant Kirjath-Arba ; et il battit Schéschaï, Ahiman et Talmaï\FTNT{Jos. 15:14.}.
\VS{11}De là, il marcha contre les habitants de Debir ; Debir s’appelait auparavant Kirjath-Sépher\FTNT{Jos. 15:15.}.
\VS{12}Caleb dit : Je donnerai ma fille Acsa pour femme à celui qui frappera Kirjath-Sépher et qui la prendra\FTNT{Jos. 15:16.}.
\VS{13}Othniel, fils de Kenaz, frère cadet de Caleb, s’en empara ; et Caleb lui donna sa fille Acsa pour femme.
\VS{14}Et il arriva que comme elle s'en allait, elle l'incita à demander à son père un champ. Puis elle descendit  impétueusement de dessus son âne ; et Caleb lui dit : Qu'as-tu ?\FTNT{Jos. 15:18.}
\VS{15}Elle lui répondit : Donne-moi un présent, puisque tu m'as donné une terre du midi ; donne-moi aussi des sources d'eau. Et Caleb lui donna les sources supérieures et les sources inférieures.
\VS{16}Les fils du Kénien, beau-père de Moïse, montèrent de la ville des palmiers avec les fils de Juda, dans le désert de Juda, qui est au midi d'Arad, et ils allèrent et demeurèrent avec le peuple\FTNT{Jg. 4:11}.
\VS{17}Puis Juda se mit en marche avec Siméon son frère et ils frappèrent les Cananéens qui habitaient à Tsephath ; et ils détruisirent la ville par le moyen de l'interdit, c'est pourquoi on appela la ville du nom de Horma.
\VS{18}Juda prit aussi Gaza avec ses territoires ; Askalon avec ses territoires ; et Ekron avec ses territoires.
\TextTitle{Des victoires en demi-teintes}
\VS{19}Yahweh fut avec Juda et il se rendit maître de la montagne, mais il ne pût chasser les habitants de la vallée, parce qu'ils avaient des chars de fer.
\VS{20}On donna Hébron à Caleb, comme Moïse l'avait dit ; et il en chassa les trois fils d'Anak\FTNT{No. 14:24}.
\VS{21}Quant aux fils de Benjamin, ils ne chassèrent pas les Jébusiens qui habitaient à Jérusalem ; c'est pourquoi les Jébusiens ont habité avec les fils de Benjamin à Jérusalem jusqu'à ce jour.
\VS{22}Ceux de la maison de Joseph montèrent aussi contre Béthel, et Yahweh fut avec eux.
\VS{23}Ceux de la maison de Joseph firent explorer Béthel, dont le nom était auparavant Luz.
\VS{24}Les espions virent un homme qui sortait de la ville, et ils dirent : Nous te prions de nous montrer un endroit par où l’on puisse entrer dans la ville, et nous te ferons grâce.
\VS{25}Il leur montra par où ils pourraient entrer dans la ville. Et ils frappèrent la ville du tranchant de l'épée ; mais ils laissèrent aller cet homme et toute sa famille.
\VS{26}Puis cet homme se rendit dans le pays des Héthiens ; il bâtit une ville et lui donna le nom de Luz, nom qu’elle a porté jusqu'à ce jour.
\VS{27}Manassé ne chassa pas les habitants de Beth-Schean et des villes de son ressort, de Thaanac et des villes de son ressort, de Dor et des villes de son ressort, les habitants de Jibleam et des villes de son ressort, les habitants de Meguiddo et des villes de son ressort ; et les Cananéens persistèrent à habiter dans ce pays-là.
\VS{28}Il est vrai qu’il arriva que quand Israël fut devenu plus fort, il assujettit les Cananéens à un tribut mais il ne les chassa pas entièrement.
\VS{29}Ephraïm ne chassa pas les Cananéens qui habitaient à Guézer, et les Cananéens habitèrent avec lui à Guézer.
\VS{30}Zabulon ne chassa pas les habitants de Kitron, ni les habitants de Nahalol ; et les Cananéens habitèrent avec lui et lui furent assujettis à un tribut.
\VS{31}Aser ne chassa pas les habitants d’Acco, ni les habitants de Sidon, ni ceux d’Achlal, ni d'Aczib, ni d'Helba, ni d'Aphik, ni de Rehob ;
\VS{32}Mais ceux d'Aser habitèrent parmi les Cananéens, habitants du pays ; car ils ne les chassèrent pas.
\VS{33}Nephthali ne chassa pas les habitants de Beth-Schémesch, ni les habitants de Beth-Anath, mais il habita parmi les Cananéens habitants du pays ; et les habitants de Beth-Schémesch, et de Beth-Anath lui furent assujettis au tribut.
\VS{34}Les Amoréens repoussèrent les enfants de Dan dans la montagne et ne les laissèrent pas descendre dans la vallée.
\VS{35}Les Amoréens voulurent encore habiter à Har-Hérès, à Ajalon et à Schaalbim ; mais la main de la maison de Joseph étant devenue plus forte, ils furent assujettis au tribut.
\VS{36}Le territoire des Amoréens s'étendait depuis la montée d’Akrabbim, depuis Séla et en dessus.
\Chap{2}
\TextTitle{Le peuple repris pour sa désobéissance}
\VerseOne{}Or l'Ange de Yahweh monta de Guilgal à Bokim, et dit : Je vous ai fait monter hors d'Egypte, et je vous ai fait entrer dans le pays que j’avais juré à vos pères, et j’ai dit : Je n’enfreindrai jamais mon alliance que j’ai traitée avec vous\FTNT{Ge. 17:7.} ;
\VS{2}Et vous aussi vous ne traiterez pas alliance avec les habitants de ce pays, vous démolirez leurs autels. Mais vous n'avez pas obéi à ma voix. Pourquoi avez-vous fait cela\FTNT{Ex. 23:32 ; De. 7:2 ; De. 12:3.} ?
\VS{3}J’ai dit alors : Je ne les chasserai pas devant vous, mais ils seront à vos côtés, et leurs dieux vous seront un piège\FTNT{Ex. 23:33 ; Jos. 23:13.}.
\VS{4}Et il arriva que, comme l'Ange de Yahweh disait ces paroles à tous les enfants d'Israël, le peuple éleva la voix et pleura.
\VS{5}C'est pourquoi ils appelèrent ce lieu Bokim et ils y offrirent des sacrifices à Yahweh.
\VS{6}Josué renvoya le peuple, et les enfants d'Israël allèrent chacun dans son héritage pour prendre possession du pays\FTNT{Jos. 24:28–32.}.
\VS{7}Le peuple servit Yahweh tout le temps de Josué, et tout le temps des anciens qui survécurent à Josué et qui avaient vu toutes les grandes œuvres que Yahweh avait faites en faveur d’Israël\FTNT{Jos. 24:31.}.
\VS{8}Puis Josué, fils de Nun, serviteur de Yahweh, mourut, âgé de cent dix ans\FTNT{Jos. 24:29.}.
\VS{9}On l’ensevelit dans le territoire qu’il avait eu en partage à Thimnath-Hérès, dans la montagne d'Ephraïm, au nord de la montagne de Gaasch\FTNT{Jos. 24:30.}.
\TextTitle{La nouvelle génération abandonne Yahweh}
\VS{10}Toute cette génération fut recueillie auprès de ses pères, puis il s’éleva après elle une autre génération, qui ne connaissait pas Yahweh ni les œuvres qu'il avait faites en faveur d’Israël.
\VS{11}Les enfants d'Israël firent alors ce qui est mal aux yeux de Yahweh et ils servirent les Baals\FTNT{Baal est un dieu phénicien qui, sous les Ramessides, était assimilé dans la mythologie égyptienne à Seth et à Montou. Baal est un dieu d’origine sémite. Il est le dieu de la pluie. Son nom – «~le maître~» ou «~l’époux~»- se retrouve partout dans le Moyen-Orient, depuis les zones peuplées par les sémites jusqu’aux colonies phéniciennes, dont Carthage. Il était invariablement accompagné d’une divinité féminine (Astarté, Ishtar, Tanit...). Voir Jg. 3:7 ; Jg. 8:33 ; Jg. 10:6.}.
\VS{12}Et ils abandonnèrent Yahweh, le Dieu de leurs pères, qui les avait fait sortir du pays d’Égypte, ils allèrent après d'autres dieux, d'entre les dieux des peuples qui les entouraient ; et ils se prosternèrent devant eux, irritant ainsi Yahweh.
\VS{13}Ils abandonnèrent donc Yahweh, et servirent Baal et les Astartés\FTNT{Astarté ou Ashtart en punico-phénicien, ou Ishtar, dérivé de la déesse de Babylone, était  généralement assimilée à la déesse Mésopotamienne Innana. Déesse phénicienne présentant un caractère belliqueux, elle était souvent représentée à califourchon sur son cheval, accompagnant et protégeant le souverain. Elément féminin du couple suprême qu’elle formait avec Baal, celle-ci assumait des fonctions variées : protectrice du souverain et de sa dynastie ou encore des marins.  Comme pour la plupart des divinités féminines primordiales de l’antiquité (et de la proto-histoire), son culte était  lié à la fertilité et à la fécondité. Parfois vénérée sous le nom de Tanit, elle sera assimilée à Vénus par les Romains sous le nom officiel de Venere Ericina.}.
\VS{14}La colère de Yahweh s'enflamma contre Israël. Il les livra entre les mains de pillards\FTNT{Lorsqu’un enfant de Dieu ouvre la porte au péché, il s’expose aux pillards, c’est-à-dire à Satan et ses démons (Jn. 10:10).} qui les pillèrent, il les vendit entre les mains de leurs ennemis d'alentour, de sorte qu'ils ne purent plus résister face à leurs ennemis\FTNT{Ps. 44:12-13 ; Es. 50:1.}.
\VS{15}Partout où ils allaient, la main de Yahweh était contre eux pour leur faire du mal, comme Yahweh l’avait dit et leur avait juré. Ils furent dans une grande détresse\FTNT{Lé. 26:25 ; De. 28:25.}.
\TextTitle{Yahweh suscite des libérateurs : Les juges}
\VS{16}Yahweh leur suscita des juges\FTNT{Les Juges étaient principalement des libérateurs de l’oppression des ennemis d’Israël.} et ils les délivrèrent de la main de ceux qui les pillaient.
\VS{17}Mais ils ne voulurent pas écouter leurs juges, ils se prostituèrent auprès d'autres dieux, se prosternèrent devant eux. Ils se détournèrent promptement du chemin qu’avaient suivi leurs pères et ils n’obéirent pas comme eux aux commandements de Yahweh.
\VS{18}Quand Yahweh leur suscitait des juges, Yahweh était avec le juge, et il les délivrait de la main de leurs ennemis pendant tout le temps de la vie du juge ; car Yahweh se repentait à cause de leurs gémissements contre ceux qui les opprimaient et les tourmentaient.
\VS{19}Puis il arrivait que quand le juge mourrait, ils se corrompaient de nouveau plus que leurs pères en allant après d'autres dieux pour les servir et se prosterner devant eux, et ils persévéraient dans la même conduite et dans la même voie obstinée\FTNT{Jg. 3:12.}.
\TextTitle{IYahweh éprouve Israël et ne chasse pas ses ennemis}
\VS{20}C'est pourquoi la colère de Yahweh s'enflamma contre Israël, et il dit : Puisque cette nation a transgressé mon alliance que j'avais prescrite à leurs pères et puisqu’ils n'ont pas obéi à ma voix,
\VS{21}aussi je ne chasserai plus devant eux aucune des nations que Josué laissa quand il mourut\FTNT{Jos. 23:13.},
\VS{22}afin d'éprouver par elles Israël, pour savoir s'ils prendront garde ou non de suivre la voie de Yahweh, comme leurs pères y ont pris garde.
\VS{23}Yahweh laissa en repos ces nations qu'il n'avait pas livrées entre les mains de Josué et il ne se hâta pas de les chasser\FTNT{Jg. 3:1-3.}.
\Chap{3}
\VerseOne{}Voici les nations que Yahweh laissa pour éprouver par elles Israël, tous ceux qui n'avaient pas connu toutes les guerres de Canaan\FTNT{Jg. 2:21-23.} ; 
\VS{2}afin qu’au moins les générations des enfants d'Israël connaissent et apprennent la guerre, ceux qui ne l’avaient pas connue auparavant.
\VS{3}Ces nations étaient : Les cinq princes des Philistins, tous les Cananéens, les Sidoniens et les Héviens qui habitaient la montagne du Liban depuis la montagne de Baal-Hermon, jusqu'à l'entrée de Hamath\FTNT{No. 13:22.}.
\VS{4}Ces nations, dis-je, servirent à éprouver Israël pour voir s'ils obéiraient aux commandements que Yahweh avait donnés à leurs pères par le moyen de Moïse.
\TextTitle{Israël se mélange aux nations païennes}
\VS{5}Ainsi les enfants d'Israël habitèrent parmi les Cananéens, les Héthiens, les Amoréens, les Phéréziens, les Héviens et les Jébusiens.
\VS{6}Ils prirent leurs filles pour femmes, ils donnèrent leurs filles à leurs fils et servirent leurs dieux.
\VS{7}Les enfants d'Israël firent ce qui est mal aux yeux de Yahweh, ils oublièrent Yahweh et servirent les Baals et les Astartés\FTNT{Jg. 2:11.}.
\TextTitle{Othniel, premier juge suscité par Yahweh}
\VS{8}C'est pourquoi la colère de Yahweh s'enflamma contre Israël, et il les vendit entre la main de Cuschan-Rischeathaïm, roi de Mésopotamie. Et les enfants d'Israël furent asservis à Cuschan-Rischeathaïm durant huit ans.
\VS{9}Puis les enfants d'Israël crièrent à Yahweh, et Yahweh leur suscita un libérateur qui les délivra, Othniel, fils de Kenaz, frère cadet de Caleb.
\VS{10}L’Esprit de Yahweh fut sur lui. Il devint juge en Israël, et il sortit pour la guerre. Yahweh livra entre ses mains Cuschan-Rischeathaïm, roi de Mésopotamie ; et sa main fut puissante contre Cuschan-Rischeathaïm.
\VS{11}Le pays fut en repos pendant quarante ans. Puis Othniel, fils de Kenaz, mourut.
\TextTitle{Ehud, juge en Israël}
\VS{12}Les enfants d'Israël firent encore ce qui est mal aux yeux de Yahweh ; et Yahweh fortifia Eglon, roi de Moab, contre Israël, parce qu'ils avaient fait ce qui est mauvais aux yeux de Yahweh.
\VS{13}Eglon réunit auprès de lui les fils d'Ammon et les Amalécites et il se mit en marche. Il battit Israël et ils s'emparèrent de la ville des palmiers\FTNT{Palmiers: un autre nom de Jéricho.}.
\VS{14}Et les enfants d'Israël furent asservis à Eglon, roi de Moab, durant dix-huit ans.
\VS{15}Puis les enfants d'Israël crièrent à Yahweh, et Yahweh leur suscita un libérateur, Ehud, fils de Guéra, Benjamite, qui ne se servait pas de sa main droite. Les enfants d'Israël envoyèrent par lui un présent à Eglon, roi de Moab.
\VS{16}Ehud se fit une épée à deux tranchants, de la longueur d'une coudée\FTNT{Une coudée correspond environ à 45 cm.} et il la ceignit sous ses vêtements, sur sa cuisse droite.
\VS{17}Il offrit le présent à Eglon, roi de Moab ; et Eglon était un homme fort gras.
\VS{18}Or il arriva que lorsqu’il eut achevé d’offrir le présent, il renvoya le peuple qui avait apporté le présent.
\VS{19}Mais Ehud revint depuis les idoles de pierre, qui étaient près de Guilgal et il dit : Ô roi ! J’ai quelque chose de secret à te dire. Et il lui répondit : Tais-toi ! Et tous ceux qui étaient auprès de lui sortirent de là.
\VS{20}Ehud s'approcha de lui, comme il était assis seul dans sa chambre d'été, et il dit : J'ai un mot à te dire de la part de Dieu, alors le roi se leva du trône.
\VS{21}Et Ehud avança sa main gauche, tira l'épée de son côté droit et la lui enfonça dans le ventre.
\VS{22}Et la poignée entra après la lame, et la graisse serra tellement la lame, qu’il ne pouvait retirer l’épée du ventre, et il en sortit de l’excrément.
\VS{23}Après cela, Ehud sortit par le portique, ferma après lui les portes de la chambre et tira le verrou.
\VS{24}Quand il fut sorti, les serviteurs d'Eglon vinrent et regardèrent ; et voici, les portes de la chambre étaient fermées au verrou. Ils dirent : Sans doute il se couvre les pieds dans sa chambre d’été.
\VS{25}Et ils attendirent tant qu'ils en furent déconcertés ; et voyant qu'il n'ouvrait pas les portes de la chambre, ils prirent la clef et ouvrirent ; et voici, leur maître était mort, étendu à terre.
\VS{26}Mais Ehud s'échappa pendant qu’ils hésitaient ; et il dépassa les carrières de pierre et se sauva à Seïra.
\VS{27}Dès qu’il fut arrivé, il sonna du shofar dans la montagne d'Ephraïm. Les enfants d'Israël descendirent avec lui de la montagne et il marchait à leur tête.
\VS{28}Il leur dit : Suivez-moi, car Yahweh a livré entre vos mains les Moabites, vos ennemis. Ainsi ils descendirent après lui, s’emparèrent des passages du Jourdain vis-à-vis de Moab et ne laissèrent passer personne.
\VS{29}Ils battirent dans ce temps-là environ dix mille hommes de Moab, tous robustes, tous vaillants et il n'en échappa aucun.
\VS{30}En ce jour, Moab fut humilié sous la main d'Israël. Et le pays fut en repos pendant quatre-vingts ans.
\TextTitle{Schamgar, juge en Israël}
\VS{31}Après lui, il y eut Schamgar, fils d'Anath. Il battit six cents Philistins avec un aiguillon à bœufs et délivra Israël.
\Chap{4}
\TextTitle{Débora et Barak, juges en Israël}
\VerseOne{}Mais les enfants d'Israël firent encore ce qui est mal aux yeux de Yahweh après qu'Ehud fut mort.
\VS{2}C'est pourquoi Yahweh les vendit entre la main de Jabin, roi de Canaan, qui régnait à Hatsor. Le chef de son armée était Sisera, qui habitait à Haroscheth-Goïm\FTNT{Jg. 3:8-16 ; Jos. 11:11-13 ; 1 S. 12:9.}.
\VS{3}Les enfants d'Israël crièrent à Yahweh car Jabin avait neuf cents chars de fer, et il avait violemment opprimé les enfants d'Israël durant vingt ans\FTNT{Jg. 1:19.}.
\VS{4}Dans ce temps-là, Débora, prophétesse, femme de Lappidoth, était juge en Israël.
\VS{5}Débora se tenait sous un palmier, entre Rama et Béthel, dans la montagne d'Ephraïm ; et les enfants d'Israël montaient vers elle pour être jugés.
\VS{6}Elle envoya appeler Barak, fils d'Abinoam, de Kédesch-Nephthali et elle lui dit : Yahweh, le Dieu d'Israël, n'a-t-il pas donné cet ordre ? En disant : Va, et dirige-toi sur la montagne de Thabor et prends avec toi dix mille hommes des enfants de Nephthali, et des enfants de Zabulon\FTNT{Hé. 11:32.} ;
\VS{7}J’attirerai vers toi, au torrent de Kison, Sisera, chef de l'armée de Jabin, avec ses chars et ses troupes et je le livrerai entre tes mains\FTNT{Ps. 83:9-10.}.
\VS{8}Barak lui dit : Si tu viens avec moi, j'irai ; mais si tu ne viens pas avec moi, je n’irai pas.
\VS{9}Elle répondit : J'irai, j'irai avec toi, mais tu n'auras pas d'honneur sur le chemin où tu marches ; car Yahweh livrera Sisera entre les mains d'une femme. Débora se leva et elle alla avec Barak à Kédesch.
\VS{10}Barak convoqua Zabulon et Nephthali à Kédesch ; dix mille hommes marchèrent à sa suite ; et Débora monta avec lui.
\VS{11}Héber, le Kénien, s’était séparé des fils de Hobab, beau-père de Moïse et il avait dressé ses tentes jusqu'au chêne de Tsaannaïm, près de Kédesch\FTNT{No. 10:29.}.
\TextTitle{Yahweh accorde la victoire à Israël}
\VS{12}On rapporta à Sisera que Barak, fils d'Abinoam, s’était dirigé sur la montagne de Thabor.
\VS{13}Et Sisera rassembla tous ses chars, neuf cents chars de fer, et tout le peuple qui était avec lui, depuis Haroscheth-Goïm, jusqu'au torrent de Kison.
\VS{14}Alors Débora dit à Barak : Lève-toi, car voici le jour où Yahweh livre Sisera entre tes mains. Yahweh ne marche-t-il pas devant toi ? Barak descendit de la montagne de Thabor, ayant dix mille hommes à sa suite.
\VS{15}Yahweh mit en déroute devant Barak, Sisera, tous ses chars et toute l'armée, par le tranchant de l'épée. Sisera descendit du char et s'enfuit à pied\FTNT{Ps. 83:9-10.}.
\VS{16}Barak poursuivit les chars et l'armée jusqu'à Haroscheth-Goïm ; et toute l'armée de Sisera fut passée au fil de l'épée ; il n'en resta pas un seul.
\VS{17}Sisera se sauva à pied dans la tente de Jaël, femme de Héber, le Kénien ; car il y avait paix entre Jabin, roi de Hatsor et la maison de Héber, le Kénien.
\VS{18}Jaël étant sortie au-devant de Sisera, lui dit : Entre, mon seigneur, entre chez moi, ne crains pas. Il entra donc chez elle dans la tente et elle le cacha sous une couverture.
\VS{19}Puis il lui dit : Je te prie, donne-moi un peu d'eau à boire, car j'ai soif. Et elle ouvrit une outre de lait, lui donna à boire et le couvrit\FTNT{Jg. 5:25.}.
\VS{20}Il lui dit encore : Tiens-toi à l'entrée de la tente et si l’on vient t’interroger, en disant : Y a-t-il ici quelqu'un ? Alors tu répondras : Non.
\VS{21}Jaël, femme de Héber, saisit un pieu de la tente, prit en sa main un marteau, s’approcha de lui doucement, et lui enfonça dans la tempe le pieu, qui pénétra en terre, pendant qu'il dormait profondément, car il était accablé de fatigue. Et ainsi il mourut.
\VS{22}Et voici, Barak poursuivait Sisera, Jaël sortit au-devant de lui et lui dit : Viens, et je te montrerai l'homme que tu cherches. Barak entra chez elle, et voici, Sisera était étendu mort, et le pieu était dans sa tempe.
\VS{23}En ce jour-là, Dieu humilia Jabin, roi de Canaan, devant les enfants d'Israël.
\VS{24}Et la main des enfants d'Israël s’appesantit et se renforça de plus en plus sur Jabin, roi de Canaan, jusqu'à ce qu'ils aient exterminé Jabin, roi de Canaan.
\Chap{5}
\TextTitle{Cantique à la gloire de Yahweh, le Dieu qui délivre}
\VerseOne{}En ce jour-là, Débora chanta ce cantique avec Barak, fils d'Abinoam, en disant :
\VS{2}Bénissez Yahweh de ce qu’il a fait de telles vengeances en Israël et de ce que le peuple s’est offert volontairement.
\VS{3}Vous, rois, écoutez ! Vous, princes, prêtez l'oreille ! Moi, je chanterai à Yahweh, je chanterai un hymne à Yahweh, le Dieu d'Israël.
\VS{4}Ô Yahweh ! Quand tu sortis de Séir, quand tu t’avanças des champs d'Edom, la terre trembla, les cieux se fondirent, les nuées fondirent en eaux ;
\VS{5}Les montagnes s'ébranlèrent devant Yahweh, ce Sinaï devant Yahweh, le Dieu d'Israël\FTNT{Ps. 68:8-9}.
\VS{6}Aux jours de Schamgar, fils d’Anath, aux jours de Jaël, les grandes routes étaient délaissées, et ceux qui voyageaient prenaient des chemins détournés.
\VS{7}Les villes non murées n’étaient plus habitées en Israël, elles n’étaient point habitées, jusqu’à ce que je me suis levée, moi Débora, jusqu’à ce que je me suis levée pour être mère en Israël.
\VS{8}Israël choisissait-il des dieux nouveaux aussitôt la guerre était aux portes. On ne voyait ni bouclier ni lance chez quarante milliers en Israël.
\VS{9}J’ai mon cœur vers les chefs d'Israël, qui se sont portés volontairement d’entre le peuple. Bénissez Yahweh !
\VS{10}Vous qui montez sur les ânesses blanches, vous qui avez pour sièges des tapis et vous qui marchez sur le chemin, méditez !
\VS{11}Le bruit des archers ayant cessé dans les abreuvoirs, qu’on s’y entretienne des justices de Yahweh et des justices de ses villes non murées en Israël ; alors le peuple de Dieu descendra aux portes.
\VS{12}Réveille-toi, réveille-toi, Débora !  Réveille-toi, réveille-toi, dit le cantique, lève-toi Barak et emmène en captivité ceux que tu as faits captifs, toi fils d'Abinoam\FTNT{Jg. 4:6.}.
\VS{13}Yahweh a fait dominer un reste du peuple sur les puissants ; Yahweh m'a fait dominer sur les héros.
\VS{14}Leur racine est depuis Ephraïm jusqu’à Amalek. A ta suite marcha Benjamin parmi ta troupe. De Makir descendirent les chefs, et de Zabulon ceux qui manient la plume du scribe.
\VS{15}Et les chefs d’Issacar ont été avec Débora, et Issacar ainsi que Barak ; il a été envoyé avec sa suite dans la vallée ; il y a eu aux ruisseaux de Ruben, de grandes considérations dans leur cœur.
\VS{16}Pourquoi es-tu resté entre les barres des étables, à écouter le bêlement des troupeaux ? Aux ruisseaux de Ruben, grandes furent les résolutions du cœur !
\VS{17}Galaad est resté au-delà du Jourdain ; et pourquoi Dan est-il resté sur ses navires ? Aser s'est tenu sur le rivage de la mer, et s’est reposé dans ses ports.
\VS{18}Mais pour Zabulon, c'est un peuple qui a exposé son âme à la mort ; et Nephthali de même, sur les hauteurs des champs.
\VS{19}Les rois vinrent, ils combattirent. Alors combattirent les rois de Canaan, à Thaanac, près des eaux de Meguiddo ; mais ils ne remportèrent nul butin, nul argent.
\VS{20}On a combattu des cieux, les étoiles, dis-je, ont combattu du lieu de leur cours contre Sisera\FTNT{Jg. 4:7.}.
\VS{21}Le torrent de Kison les a emportés, le torrent des anciens temps, le torrent de Kison. Mon âme tu as foulé aux pieds les héros.
\VS{22}Alors les talons des chevaux battirent le sol à cause de la course rapide, de la course rapide de ses puissants chevaux.
\VS{23}Maudissez Méroz, dit l'Ange de Yahweh ; maudissez, maudissez ses habitants, car ils ne sont pas venus au secours de Yahweh, au secours de Yahweh, avec les héros.
\VS{24}Bénie soit par-dessus toutes les femmes Jaël, femme de Héber, le Kénien ! Qu'elle soit bénie entre les femmes qui habitent sous les tentes !
\VS{25}Il demanda de l'eau, elle lui a donné du lait ; elle lui a présenté de la crème dans la coupe des chefs.
\VS{26}Elle a saisi de sa main gauche le pieu et de sa main droite le marteau des ouvriers ; elle a frappé Sisera et lui a fendu la tête ; elle a fracassé et transpercé ses tempes.
\VS{27}Il s'est affaissé aux pieds de Jaël, il est tombé, il s’est couché aux pieds de Jaël ; il s'est affaissé, il est tombé ; là où il s'est affaissé, il est tombé là tout défiguré.
\VS{28}La mère de Sisera regardait par la fenêtre et s'écriait en regardant par les treillis : Pourquoi son char tarde-t-il à venir ? Pourquoi ses chars vont-ils si lentement ?
\VS{29}Les plus sages de ses dames lui répondent, et elle se répond à elle-même :
\VS{30}N’ont-ils pas trouvé ? ils partagent le butin ; une fille, deux filles à chacun par tête. Le butin des vêtements de couleurs est à Sisera, le butin de couleurs de broderie ; couleur de broderie à deux endroits, autour du cou de ceux du butin.
\VS{31}Périssent ainsi, tous tes ennemis ô Yahweh ! Et que ceux qui t'aiment soient comme le soleil quand il sort dans sa force. Et le pays fut en repos pendant quarante ans.
\Chap{6}
\TextTitle{Israël assujetti par Madian}
\VerseOne{}Or, les enfants d'Israël firent ce qui est mal aux yeux de Yahweh ; et Yahweh les livra entre les mains de Madian pendant sept ans.
\VS{2}La main de Madian fut puissante contre Israël. Pour échapper aux Madianites, les enfants d'Israël se retiraient dans les ravins des montagnes, dans des cavernes et sur les rochers fortifiés.
\VS{3}Car il arrivait que quand Israël avait semé, Madian montait avec Amalek et les fils de l’orient, et ils montaient contre lui.
\VS{4}Ils faisaient un camp contre lui, ravageaient les fruits du pays jusqu'à Gaza et ne laissaient en Israël ni vivres, ni brebis, ni bœufs, ni ânes.
\VS{5}Car ils montaient avec leurs troupeaux et leurs tentes, ils arrivaient comme une multitude de sauterelles, ils étaient innombrables, eux et leurs chameaux et ils venaient dans le pays pour le ravager.
\VS{6}Israël fut très appauvri par Madian, et les enfants d'Israël crièrent à Yahweh.
\VS{7}Lorsque les enfants d'Israël crièrent à Yahweh au sujet de Madian,
\VS{8}Yahweh envoya un prophète aux enfants d'Israël, qui leur dit : Ainsi parle Yahweh, le Dieu d'Israël : Je vous ai fait monter hors d’Égypte et je vous ai retirés de la maison de servitude.
\VS{9}Je vous ai délivrés de la main des Egyptiens et de la main de tous ceux qui vous opprimaient ; je les ai chassés devant vous et je vous ai donné leur pays.
\VS{10}Je vous ai dit : Je suis Yahweh, votre Dieu ; vous ne craindrez pas les dieux des Amoréens, dans le pays desquels vous habitez. Mais vous n'avez pas obéi à ma voix.
\TextTitle{Gédéon rencontre l’Ange de Yahweh}
\VS{11}Puis l'Ange de Yahweh vint et s'assit sous le térébinthe d’Ophra, qui appartenait à Joas, de la famille d'Abiézer. Gédéon, son fils, battait du froment au pressoir pour le mettre à l'abri de Madian.
\VS{12}Alors l'Ange de Yahweh lui apparut et lui dit : Très fort et vaillant héros, Yahweh est avec toi !
\VS{13}Gédéon lui répondit : Hélas mon Seigneur ! Est-il possible que Yahweh soit avec nous ? Pourquoi donc toutes ces choses nous sont-elles arrivées ? Et où sont tous ces prodiges que nos pères nous ont racontés, en disant : Yahweh ne nous a-t-il pas fait monter hors d'Egypte ? Car maintenant Yahweh nous a abandonnés et nous a livrés entre les mains des Madianites.
\VS{14}Yahweh le regarda et lui dit : Va avec cette force que tu as et tu délivreras Israël de la main des Madianites ; ne t'ai-je pas envoyé\FTNT{Hé.11:32}?
\VS{15}Et il lui répondit : Hélas, mon Seigneur ! Avec quoi délivrerai-je Israël ? Voici, mon millier de bétail est le plus pauvre en Manassé et je suis le plus petit de la maison de mon père\FTNT{1 S. 9:21 ; 1 S. 16:11.}.
\VS{16}Yahweh lui dit : Parce que je serai avec toi, tu frapperas les Madianites comme s'ils n'étaient qu'un seul homme.
\VS{17}Et il lui répondit : Je te prie, si j'ai trouvé grâce à tes yeux, donne-moi un signe pour montrer que c'est toi qui me parles.
\VS{18}Je te prie, ne t’éloigne pas d’ici jusqu'à ce que je revienne auprès de toi, que j'apporte mon offrande et que je la dépose devant toi. Yahweh dit : Je resterai jusqu'à ce que tu reviennes.
\VS{19}Alors Gédéon rentra et apprêta un chevreau de lait, et fit avec un épha de farine des pains sans levain. Il mit la chair dans un panier, le jus dans un pot et il les lui apporta sous le térébinthe, et les présenta.
\VS{20}L'Ange de Dieu lui dit : Prends la chair et les pains sans levain et pose-les sur ce rocher\FTNT{Voir commentaire en  Es. 8:13-17} et répands le jus. Et il fit ainsi.
\VS{21}Alors l'Ange de Yahweh avança l’extrémité du bâton qu'il avait à la main, et toucha la chair et les pains sans levain. Le feu monta du rocher, et consuma la chair et les pains sans levain. Puis l'Ange de Yahweh disparut à ses yeux.
\VS{22}Gédéon, voyant que c'était l'Ange de Yahweh, dit : Ah, malheur à moi, Seigneur Yahweh ! Car j'ai vu l’Ange de Yahweh face à face.
\VS{23}Et Yahweh lui dit : Sois en paix, ne crains pas, tu ne mourras pas.
\VS{24}Gédéon bâtit là un autel à Yahweh, et lui donna pour nom Yahweh-Shalom. Cet autel, qui appartenait à la famille d'Abiézer, existe encore aujourd'hui à Ophra.
\TextTitle{Gédéon détruit les idole ; Yahweh lui confirme sa mission}
\VS{25}Or il arriva dans cette nuit-là que Yahweh lui dit : Prends un jeune taureau d'entre les bœufs qui sont à ton père et un deuxième taureau de sept ans ; et démolis l'autel de Baal qui est à ton père, et abats l’idole d'Astarté qui est dessus.
\VS{26}Tu bâtiras ensuite et tu disposeras, sur le haut de ce rocher, un autel à Yahweh, ton Dieu. Tu prendras ce deuxième taureau, et tu l'offriras en holocauste avec le bois de l’emblème d’Astarté que tu auras démoli.
\VS{27}Gédéon ayant pris dix hommes parmi ses serviteurs, fit comme Yahweh lui avait dit ; et parce qu'il craignait la maison de son père et les gens de la ville, il l’exécuta de nuit et non de jour.
\VS{28}Lorsque les gens de la ville se levèrent de bon matin, voici, l'autel de Baal avait été démoli, et l'idole d'Astarté qui est dessus était abattue, et le deuxième taureau était offert en holocauste sur l'autel qui avait été bâti.
\VS{29}Ils se dirent les uns aux autres : Qui a fait cela ? Et ils s’informèrent et firent des recherches. On leur dit : C’est Gédéon, fils de Joas, qui a fait cela.
\VS{30}Puis les gens de la ville dirent à Joas : Fais sortir ton fils et qu'il meure ; car il a démoli l'autel de Baal et abattu l'idole d'Astarté qui est dessus.
\VS{31}Joas répondit à tous ceux qui s'adressèrent à lui : Est-ce à vous de prendre parti pour Baal, est-ce à vous de venir à son secours ? Quiconque prendra parti pour Baal sera mis à mort avant le matin. Si Baal est un dieu, qu'il défende lui-même sa cause puisqu'on a démoli son autel.
\VS{32}Et en ce jour on donna à Gédéon le nom de Jerubbaal, en disant : Que Baal défende sa cause, puisque Gédéon a démoli son autel.
\VS{33}Tout Madian, Amalek, et les fils de l’orient se rassemblèrent ;  ils passèrent le Jourdain et campèrent dans la vallée de Jizréel.
\VS{34}Gédéon fut revêtu de l'Esprit de Yahweh ; il sonna du shofar et Abiézer fut convoqué pour marcher à sa suite\FTNT{Jg. 11:29 ; Jg. 13:25.}.
\VS{35}Il envoya des messagers dans tout Manassé qui fut aussi convoqué pour marcher à sa suite. Puis il envoya des messagers dans Aser, dans Zabulon et dans Nephthali, qui montèrent à leur rencontre.
\VS{36}Gédéon dit à Dieu : Si tu veux délivrer Israël par ma main, comme tu l'as dit,
\VS{37}voici, je vais mettre une toison de laine dans l'aire de battage ; si la toison seule se couvre de rosée et que tout le terrain reste sec, je connaîtrai que tu délivreras Israël par ma main, comme tu l’as dit.
\VS{38}Et il arriva ainsi. Le jour suivant, il se leva de bon matin, pressa la toison et en fit sortir la rosée qui donna de l’eau plein une coupe.
\VS{39}Gédéon dit encore à Dieu : Que ta colère ne s'enflamme pas contre moi, et je ne parlerai plus que cette fois : Je te prie, je voudrais seulement faire encore une épreuve avec la toison : Que la toison seule reste sèche et que tout le terrain se couvre de rosée.
\VS{40}Et Dieu fit ainsi cette nuit-là. La toison seule resta sèche, et tout le terrain se couvrit de rosée.
\Chap{7}
\TextTitle{Yahweh sélectionne un petit nombre pour le combat}
\VerseOne{}Jerubbaal qui est Gédéon, et tout le peuple qui était avec lui, se levèrent de bon matin et campèrent près de la source de Harod. Le camp de Madian était au nord, vers la colline de Moré, dans la vallée.
\VS{2}Yahweh dit à Gédéon : Le peuple qui est avec toi est trop nombreux pour que je livre Madian entre ses mains, de peur qu'Israël ne se glorifie contre moi, en disant : C’est ma main qui m'a délivré.
\VS{3}Maintenant donc fais plublier ceci aux oreilles du peuple, et qu'on dise : Que celui qui est craintif et qui a peur s’en retourne et s’éloigne de la montagne de Galaad. Vingt-deux mille hommes parmi le peuple s'en retournèrent et il en resta dix mille\FTNT{De. 20:8.}.
\VS{4}Yahweh dit à Gédéon : Le peuple est encore trop nombreux. Fais-les descendre vers l'eau et là je les épurerai\FTNT{C’est Dieu qui qualifie ses ouvriers, il les éprouve et les épure pour les rendre inébranlables. Voir le test de l’épreuve des Hébreux dans le désert de Sinaï (De. 8).} ; et celui dont je te dirai : Que celui-ci aille avec toi, ira avec toi ; et celui dont je te dirai : Que celui-ci n’aille pas avec toi, n’ira pas avec toi.
\VS{5}Il fit donc descendre le peuple vers l'eau ; et Yahweh dit à Gédéon : Tous ceux qui laperont l'eau avec la langue comme lape le chien, tu les sépareras de tous ceux qui se mettront à genoux pour boire\FTNT{Ps. 110:7.}.
\VS{6}Ceux qui lapèrent l’eau en la portant à la bouche avec leur main furent au nombre de trois cents hommes et tout le reste du peuple se mit à genoux pour boire.
\VS{7}Alors Yahweh dit à Gédéon : C’est par les trois cents hommes qui ont lapé, que je vous délivrerai et que je livrerai Madian entre tes mains. Que tout le reste du peuple s'en aille donc chacun chez soi.
\VS{8}Ainsi le peuple prit entre ses mains des provisions et ses shofars. Gédéon renvoya tous les hommes d'Israël chacun dans sa tente et il retint les trois cents hommes. Or le camp de Madian était au-dessous de lui, dans la vallée.
\TextTitle{Victoire de Gédéon sur Madian}
\VS{9}Et il arriva cette nuit-là que Yahweh lui dit : Lève-toi, descends au camp car je l'ai livré entre tes mains.
\VS{10}Si tu crains de descendre, descends-y avec Pura, ton serviteur.
\VS{11}Tu écouteras ce qu'ils diront et après cela, tes mains seront fortifiées ; descends donc au camp. Il descendit avec Pura, son serviteur, jusqu'aux avant-postes du camp.
\VS{12}Or Madian, Amalek et tous les fils de l'orient étaient répandus dans la vallée comme des sauterelles, tant il y en avait, et leurs chameaux étaient sans nombre, comme le sable qui est sur le bord de la mer, tant il y en avait\FTNT{Jg. 6:3-33.}.
\VS{13}Gédéon arriva ; et voici, un homme racontait à son compagnon un songe. Il lui disait : Voici, j'ai eu un songe ; il me semblait qu'un gâteau de pain d'orge roulait dans le camp de Madian ; et il est venu heurter jusqu’à la tente et elle est tombée ; il l’a retournée sens dessus dessous et elle a été renversée.
\VS{14}Alors son compagnon répondit et dit : Ce n'est pas autre chose que l'épée de Gédéon, fils de Joas, homme d'Israël ; Dieu a livré Madian et tout le camp entre ses mains.
\VS{15}Lorsque Gédéon eut entendu le récit du songe et son interprétation, il se prosterna, revint au camp d'Israël et dit : Levez-vous car Yahweh a livré le camp de Madian entre vos mains.
\VS{16}Puis il divisa les trois cents hommes en trois corps et il leur donna à chacun des shofars à la main et des cruches vides, avec des flambeaux dans les cruches.
\VS{17}Il leur dit : Regardez-moi et faites comme je ferai. Dès que je serai arrivé à l’extrémité du camp, vous ferez comme je ferai.
\VS{18}Quand je sonnerai du shofar, moi et tous ceux qui sont avec moi, alors vous sonnerez aussi du shofar tout autour du camp et vous direz : Pour Yahweh et pour Gédéon !
\VS{19}Gédéon et les cent hommes qui étaient avec lui arrivèrent à l’extrémité du camp, au commencement de la veille de la nuit, comme on venait de placer les gardes. Ils sonnèrent du shofar et  brisèrent les cruches qu'ils avaient à la main.
\VS{20}Ainsi les trois corps sonnèrent du shofar, et brisèrent les cruches ; ils saisirent de la main gauche les flambeaux et de la main droite les shofars pour sonner et ils s’écrièrent : L'épée de Yahweh et de Gédéon !
\VS{21}Ils restèrent chacun à sa place autour du camp, et tout le camp se mit à courir ça et là, à pousser des cris et à prendre la fuite.
\VS{22}Car comme les trois cents hommes sonnèrent encore du shofar, Yahweh leur fit tourner l'épée les uns contre les autres. Le camp s'enfuit jusqu'à Beth-Schitta, vers Tseréra, jusqu'au bord d'Abel-Mehola, près de Tabbath\FTNT{1 S. 14:20 ; Ez. 38:21.}.
\VS{23}Les hommes d'Israël, à savoir ceux de Nephthali, d'Aser et de tout Manassé, se rassemblèrent, et ils poursuivirent Madian.
\VS{24}Alors Gédéon envoya des messagers dans toute la montagne d'Ephraïm, pour leur dire : Descendez pour aller à la rencontre de Madian, et coupez-leur les premiers le passage des eaux jusqu'à Beth-Bara et celui du Jourdain. Tous les hommes d'Ephraïm se rassemblèrent, et ils s’emparèrent du passage des eaux jusqu’à Beth-Bara et de celui du Jourdain.
\VS{25}Ils saisirent deux des chefs de Madian, Oreb et Zeeb ; ils tuèrent Oreb au rocher d’Oreb, et ils tuèrent Zeeb au pressoir de Zeeb. Ils poursuivirent Madian, et ils apportèrent les têtes de Oreb et de Zeeb à Gédéon, de l’autre côté du Jourdain\FTNT{Ps. 83:11 ; Es. 10:26.}.
\Chap{8}
\TextTitle{Poursuite de Zébach et Tsalmunna ; exécution des rois de Madian}
\VerseOne{}Alors les hommes d'Ephraïm dirent à Gédéon : Que signifie cette manière d’agir envers nous ? Pourquoi ne pas nous avoir appelés quand tu es allé à la guerre contre Madian ? Et ils s'emportèrent fortement contre lui\FTNT{Jg. 12:1.}.
\VS{2}Et il leur répondit : Qu'ai-je fait maintenant au prix de ce que vous avez fait ? Les grappillages d'Ephraïm ne sont-ils pas meilleurs que la vendange d'Abiézer ?
\VS{3}Dieu a livré entre vos mains les chefs de Madian, Oreb et Zeeb. Qu'ai-je pu faire au prix de ce que vous avez fait ? Et leur esprit fut apaisé envers lui lorsqu’il eut ainsi parlé.
\VS{4}Gédéon arriva au Jourdain, et il le passa, lui et les trois cents hommes qui étaient avec lui, fatigués, mais poursuivant toujours l'ennemi.
\VS{5}C'est pourquoi il dit aux gens de Succoth : Donnez, je vous prie, quelques pains aux hommes qui m’accompagnent, car ils sont fatigués, et ainsi je poursuivrai Zébach et Tsalmunna, rois de Madian.
\VS{6}Mais les chefs de Succoth répondirent : La main de Zébach et celle de Tsalmunna sont-elles déjà en ton pouvoir, pour que nous donnions du pain à ton armée ?
\VS{7}Et Gédéon dit : Eh bien ! Quand Yahweh aura livré Zébach et Tsalmunna entre mes mains, je foulerai au pied votre chair avec des épines du désert et avec des chardons.
\VS{8}Puis de là il monta à Penuel, et il fit la même demande aux gens de Penuel. Les gens de Penuel lui répondirent comme avaient répondu ceux de Succoth.
\VS{9}Et il dit aussi aux gens de Penuel : Quand je reviendrai en paix, je démolirai cette tour.
\VS{10}Zébach et Tsalmunna étaient à Karkor et leurs armées avec eux, environ quinze mille hommes, tous ceux qui étaient restés de l'armée entière des fils de l’orient ; cent vingt mille hommes tirant l’épée avaient été tués.
\VS{11}Gédéon monta par le chemin de ceux qui habitent sous les tentes, à l’orient de Nobach et de Jogbeha, et il battit l'armée, qui se croyait en sûreté.
\VS{12}Et comme Zébach et Tsalmunna s'enfuyaient, il les poursuivit, et prit les deux rois de Madian, Zébach et Tsalmunna, et mit en déroute toute l'armée\FTNT{Ps. 83:11.}.
\TextTitle{Vengeance sur Succoth et Penuel ; exécution de Zébach et Tsalmunna}
\VS{13}Puis Gédéon, fils de Joas, revint de la bataille par la montée de Hérès.
\VS{14}Il saisit un garçon d’entre les hommes de Succoth, il l'interrogea, et ce garçon lui donna par écrit le nom des chefs et des anciens de Succoth, au nombre de soixante-dix-sept hommes.
\VS{15}Et il vint auprès de gens de Succoth, et leur dit : Voici Zébach et Tsalmunna, au sujet desquels vous m'avez insulté, en disant : La main de Zébach et celle de Tsalmunna sont-elles déjà en ton pouvoir, pour que nous donnions du pain à tes hommes fatigués ?
\VS{16}Il prit donc les anciens de la ville et châtia les hommes de Succoth avec des épines du désert et des chardons.
\VS{17}Il démolit la tour de Penuel, et tua les gens de la ville.
\VS{18}Puis il dit à Zébach et à Tsalmunna : Comment étaient les hommes que vous avez tués à Thabor ? Ils répondirent : Ils étaient entièrement comme toi, chacun d'eux avait l'air d'un fils de roi.
\VS{19}Il leur dit : C'étaient mes frères, fils de ma mère. Yahweh est vivant, si vous les aviez laissés vivre, je ne vous tuerais pas.
\VS{20}Puis il dit à Jéther, son premier-né : Lève-toi, tue-les ! Mais le jeune garçon ne tira pas son épée, car il avait peur, car il était encore un enfant.
\VS{21}Et Zébach et Tsalmunna dirent : Lève-toi toi-même, et jette-toi sur nous ! Car tel est l'homme, telle est sa force. Et Gédéon se leva, et tua Zébach et Tsalmunna. Il prit ensuite les croissants qui étaient aux cous de leurs chameaux.
\TextTitle{Gédéon recommande au peuple le règne de Yahweh}
\VS{22}Les hommes d'Israël dirent tous d'un commun accord à Gédéon : Domine sur nous, tant toi que ton fils, et le fils de ton fils, car tu nous as délivrés de la main de Madian.
\VS{23}Gédéon leur répondit : Je ne dominerai pas sur vous, et mon fils ne dominera pas sur vous ; c'est Yahweh qui dominera sur vous\FTNT{De. 17:15.}.
\TextTitle{Gédéon introduit une occasion de chute en Israël}
\VS{24}Mais Gédéon leur dit : J’ai une demande à vous faire : Donnez-moi chacun les anneaux que vous avez eus pour butin. Les ennemis avaient des anneaux d'or, car ils étaient Ismaélites.
\VS{25}Ils répondirent : Nous les donnerons volontiers. Et ils étendirent un manteau sur lequel chacun jeta les anneaux de son butin.
\VS{26}Le poids des anneaux d'or que Gédéon demanda fut de mille sept cents sicles d'or, sans les croissants, les pendants d'oreilles, et les vêtements de pourpre que portaient les rois de Madian, et sans les colliers qui étaient aux cous de leurs chameaux.
\VS{27}Puis Gédéon en fit un  éphod\FTNT{Sous Moïse, il y avait deux sortes d'éphods, le premier était de simple lin pour les sacrificateurs, et le deuxième de broderie pour le souverain sacrificateur. Comme celui des simples sacrificateurs n'avait rien de particulier, Moïse ne s'est pas arrêté à le décrire. Mais il décrit longuement celui du souverain sacrificateur. (Ex. 28:6-9). Il était composé d'or, d'hyacinthe, de pourpre, de cramoisi, de coton retors ; c'était un tissu de différentes couleurs. Il y avait à l'endroit de l'éphod qui venait sur les deux épaules du souverain sacrificateur, deux grosses pierres précieuses, qui étaient chargées du nom des douze tribus d’Israël, six noms sur chaque pierre. A l'endroit où l'éphod se croisait sur la poitrine du grand prêtre, il y avait un ornement carré, nommé le rational, en hébreu «~choschen~», dans lequel étaient enchâssées douze pierres précieuses, où l'on avait gravé les noms des douze tribus d'Israël ; un sur chacune des pierres.}, et le mit dans sa ville, à Ophra, où il devint un objet de prostitution pour tout Israël ; il fut un piège pour Gédéon et pour sa maison.
\TextTitle{Fin de la vie de Gédéon ; rechute d’Israël après sa mort}
\VS{28}Ainsi Madian fut humilié devant les enfants d'Israël, et il ne leva plus la tête. Le pays fut en repos pendant quarante ans, durant les jours de Gédéon.
\VS{29}Jerubbaal, fils de Joas s’en retourna dans sa ville, et demeura dans sa maison.
\VS{30}Gédéon eut soixante-dix fils, issus de ses reins, car il eut plusieurs femmes.
\VS{31}Sa concubine, qui était à Sichem, lui enfanta aussi un fils, et il lui donna le nom d’Abimélec.
\VS{32}Puis Gédéon, fils de Joas, mourut après une heureuse vieillesse ; et il fut enseveli dans le sépulcre de Joas, son père, à Ophra, qui appartenait à la famille d’Abiézer.
\TextTitle{Rechute dans l'idolâtrie}
\VS{33}Et il arriva après que Gédéon fut mort, que les enfants d'Israël se détournèrent et se prostituèrent aux Baals, et ils établirent Baal-Berith pour leur dieu\FTNT{Jg. 2:11-17 ; 10:6.}.
\VS{34}Ainsi les enfants d'Israël ne se souvinrent pas de Yahweh, leur Dieu, qui les avait délivrés de la main de tous leurs ennemis qui les entouraient.
\VS{35}Et ils n'usèrent d'aucune loyauté envers la maison de Jerubbaal, de Gédéon, après tout le bien qu'il avait fait à Israël.
\Chap{9}
\TextTitle{Conspiration d’Abimélec pour régner sur Israël}
\VerseOne{}Et Abimélec, fils de Jerubbaal, s'en alla à Sichem vers les frères de sa mère, et leur parla, ainsi qu'à toute la maison du père de sa mère :
\VS{2}Je vous prie, faites entendre ces paroles à tous les seigneurs de Sichem : Lequel vous semble le meilleur, que soixante-dix hommes, tous fils de Jerubbaal, dominent sur vous, ou qu'un seul homme domine sur vous ? Et souvenez-vous que je suis votre os et votre chair\FTNT{Ge. 29:14.}.
\VS{3}Les frères de sa mère dirent de sa part toutes ces paroles aux oreilles de tous les seigneurs de Sichem, et leur cœur se tourna après Abimélec, car ils disaient : C'est notre frère.
\VS{4}Ils lui donnèrent soixante-dix sicles d'argent de la maison de Baal-Berith. Abimélec s'en servit pour acheter des hommes misérables et turbulents, qui allèrent après lui.
\VS{5}Et il vint dans la maison de son père à Ophra, et tua sur une seule pierre ses frères, fils de Jerubbaal, qui étaient soixante-dix hommes. Il ne resta que Jotham, le plus jeune fils de Jerubbaal, parce qu'il s'était caché.
\VS{6}Et tous les seigneurs de Sichem s'assemblèrent avec toute la maison de Millo ; ils vinrent, et firent d'Abimélec leur roi près du chêne à Sichem.
\VS{7}On le rapporta à Jotham, qui alla se tenir au sommet de la montagne de Garizim, et les appelant, il dit en élevant la voix : Écoutez-moi, seigneurs de Sichem, et que Dieu vous entende !
\VS{8}Les arbres allèrent pour oindre un roi, et ils dirent à l'olivier : Règne sur nous.
\VS{9}Mais l'olivier leur répondit : Renoncerai-je à mon huile, par laquelle Dieu et les hommes sont honorés, pour aller m'agiter sur les arbres\FTNT{Ps. 104:15.} ?
\VS{10}Puis les arbres dirent au figuier : Viens, toi, règne sur nous.
\VS{11}Mais le figuier leur répondit : Renoncerai-je à ma douceur, et à mon bon fruit, pour aller m'agiter sur les arbres ?
\VS{12}Puis les arbres dirent à la vigne : Viens, toi, et règne sur nous.
\VS{13}Mais la vigne répondit : Renoncerai-je à mon vin, qui réjouit Dieu et les hommes, pour aller m'agiter sur les arbres ?
\VS{14}Alors tous les arbres dirent à l'épine : Viens, toi, et règne sur nous.
\VS{15}Et l'épine répondit aux arbres : Si c'est en vérité que vous m'oignez pour roi, venez, et réfugiez-vous sous mon ombrage ; sinon, que le feu sorte de l'épine, et qu'il dévore les cèdres du Liban.
\VS{16}Maintenant donc, est-ce en vérité et avec intégrité que vous avez agi en établissant Abimélec pour roi ? Avez-vous bien fait envers Jerubbaal et sa maison ? L'avez-vous fait selon les bienfaits qu'il a rendus de sa main ?
\VS{17}Car mon père a combattu pour vous, il a exposé sa vie devant vous, et vous a délivrés de la main de Madian ;
\VS{18}Mais vous vous êtes levés aujourd'hui contre la maison de mon père, et avez tué sur une pierre ses fils, soixante-dix hommes, et avez établi pour roi Abimélec, fils de sa servante, sur les habitants de Sichem, parce qu'il est votre frère.
\VS{19}Si, dis-je, vous avez agi aujourd'hui en vérité et avec intégrité envers Jerubbaal, et sa maison, réjouissez-vous d'Abimélec, et qu'il se réjouisse aussi de vous !
\VS{20}Sinon, que le feu sorte d'Abimélec et qu'il dévore les seigneurs de Sichem, et la maison de Millo ; et que le feu sorte des seigneurs de Sichem, et de la maison de Millo, et qu'il dévore Abimélec !
\VS{21}Puis Jotham s'enfuit rapidement ; il s'en alla à Beer, où il demeura loin d'Abimélec, son frère.
\TextTitle{Sichem se retourne contre Abimélec}
\VS{22}Abimélec gouverna sur Israël durant trois ans.
\VS{23}Alors Dieu envoya un mauvais esprit entre Abimélec et les seigneurs de Sichem, et les seigneurs de Sichem furent infidèles à Abimélec.
\VS{24}Afin que la violence faite aux soixante-dix fils de Jerubbaal vienne et que leur sang se tourne contre Abimélec, leur frère, qui les avait tués, et sur les seigneurs de Sichem, qui l'avaient aidé par leur main à tuer ses frères.
\VS{25}Les seigneurs de Sichem mirent des embûches sur le sommet des montagnes, des gens pillaient tous ceux qui passaient près d'eux sur le chemin. Cela fut rapporté à Abimélec.
\VS{26}Alors Gaal, fils d'Ebed, vint avec ses frères, et ils passèrent à Sichem. Les seigneurs de Sichem eurent confiance en lui.
\VS{27}Puis étant sortis aux champs, ils vendangèrent leurs vignes, foulèrent les raisins, et se donnèrent à des réjouissances ; ils entrèrent dans la maison de leur dieu, ils mangèrent et burent, et ils maudirent Abimélec.
\VS{28}Alors Gaal, fils d'Ebed, dit : Qui est Abimélec, et qui est Sichem pour que nous servions Abimélec ? N'est-il pas le fils de Jerubbaal et Zebul n'est-il pas son commissaire ? Servez plutôt les hommes de Hamor, père de Sichem ; mais pour quelle raison servirions-nous Abimélec ?
\VS{29}Plaise à Dieu ! Qu'on mette ce peuple sous mon pouvoir, et je chasserais Abimélec. Et il disait d'Abimélec : Multiplie ton armée, et sors !
\VS{30}Zebul, gouverneur de la ville, entendit les paroles de Gaal, fils d'Ebed, et sa colère s'enflamma.
\VS{31}Puis il envoya astucieusement des messagers vers Abimélec, pour lui dire : Voici, Gaal fils d'Ebed, et ses frères, sont entrés dans Sichem, et voici, ils assiègent la ville contre toi.
\VS{32}Maintenant donc, lève-toi de nuit, toi et le peuple qui est avec toi, et mets-toi en embuscade dans les champs.
\VS{33}Et le matin, au lever du soleil, tu te lèveras et tu te jetteras sur la ville. Gaal et le peuple qui est avec lui sortiront contre toi, ta main lui fera selon les forces que tu trouveras.
\VS{34}Abimélec et tout le peuple qui était avec lui se levèrent de nuit, et ils se mirent en embuscade contre Sichem, divisés en quatre bandes.
\VS{35}Alors Gaal, fils d'Ebed, sortit, et il se tint à l'entrée de la porte de la ville. Abimélec et tout le peuple qui était avec lui se levèrent de l'embuscade.
\VS{36}Gaal voyant le peuple, dit à Zebul : Voici un peuple qui descend du sommet des montagnes. Zebul lui dit : Tu vois l'ombre des montagnes comme des hommes.
\VS{37}Gaal, parla encore, et dit : C'est bien un peuple qui descend des hauteurs du pays, et une bande vient du chemin du chêne des devins.
\VS{38}Et Zebul lui dit : Où est donc ta bouche, toi qui disais : Qui est Abimélec, pour que nous le servions ? N'est-ce pas ici ce peuple que tu méprisais ? Sors maintenant, je te prie, et combats !
\VS{39}Alors, Gaal sortit conduisant les seigneurs de Sichem, et combattit contre Abimélec.
\VS{40}Abimélec le poursuivit, et il s'enfuit de devant lui, et plusieurs tombèrent morts jusqu'à l'entrée de la porte.
\VS{41}Abimélec s'arrêta à Aruma. Zebul repoussa Gaal et ses frères, afin qu'ils ne restent plus à Sichem.
\VS{42}Et il arriva, dès le lendemain, que le peuple sortit aux champs. Cela fut rapporté à Abimélec,
\VS{43}qui prit son peuple, et le divisa en trois bandes, et les mit en embuscade dans les champs. Ayant vu que le peuple sortait de la ville, il se leva contre eux, et les battit.
\VS{44}Abimélec et la bande qui était avec lui se répandirent, et se tinrent à l'entrée de la porte de la ville ; mais les deux autres bandes se jetèrent sur tous ceux qui étaient aux champs, et les battirent.
\VS{45}Ainsi Abimélec combattit contre la ville toute la journée ; il prit la ville, et tua le peuple qui y était. Il la rasa, et y sema du sel.
\VS{46}Ayant appris cela, tous les seigneurs de la tour de Sichem entrèrent dans la forteresse de la maison du dieu Baal-Berith\FTNT{Jg. 9:4 ; 8:33.}.
\VS{47}On rapporta à Abimélec que tous les seigneurs de la tour de Sichem s'étaient assemblés dans la forteresse.
\VS{48}Alors Abimélec monta sur la montagne de Tsalmon, lui et tout le peuple qui était avec lui. Il prit en main une hache, coupa une branche d'arbre, et l'ayant mise sur son épaule, la porta, et dit au peuple qui était avec lui : Avez-vous vu ce que j'ai fait ? Hâtez-vous de faire comme moi.
\VS{49}Chacun donc de tout le peuple coupa une branche, et ils marchèrent derrière Abimélec ; ils mirent ces branches tout autour de la forteresse, et y mirent le feu. Il brûlèrent la forteresse, et toutes les personnes de la tour de Sichem moururent ; au nombre d'environ mille, tant hommes que femmes.
\TextTitle{Abimélec meurt}
\VS{50}Puis Abimélec marcha contre Thébets, y mit son camp, et la prit.
\VS{51}Il y avait au milieu de la ville une forte tour, où s'enfuirent tous les hommes et toutes les femmes, et tous les seigneurs de la ville, et ayant fermé les portes après eux, ils montèrent sur le toit de la Tour.
\VS{52}Alors Abimélec alla jusqu'à la tour, l'attaqua, et s'approcha jusqu'à la porte pour la brûler par le feu.
\VS{53}Mais une femme jeta une pièce de meule de moulin sur la tête d'Abimélec, et lui brisa le crâne\FTNT{2 S. 11:21.}.
\VS{54}Rapidement, il appela le garçon qui portait ses armes, et lui dit : Tire ton épée, et tue-moi, de peur qu'on ne dise de moi : C'est une femme qui l'a tué. Le garçon le transperça, et il mourut\FTNT{1 S. 31:4.}.
\VS{55}Quand les hommes d'Israël virent qu'Abimélec était mort, ils s'en allèrent chacun en son lieu.
\VS{56}Ainsi Dieu rendit à Abimélec le mal qu'il avait fait contre son père, en tuant ses soixante-dix frères,
\VS{57}Et toute la méchanceté des hommes de Sichem ; Dieu, dis-je, la fit retourner sur leurs têtes ; et ainsi la malédiction de Jotham, fils de Jérubbaal, vint sur eux.
\Chap{10}
\TextTitle{Thola, juge en Israël}
\VerseOne{}Après Abimélec, Thola fils de Pua, fils de Dodo, homme d'Issacar, se leva pour délivrer Israël ; il habitait à Schamir, dans la montagne d'Ephraïm.
\VS{2}Il fut juge en Israël pendant vingt-trois ans ; puis il mourut, et fut enterré à Schamir.
\TextTitle{Jaïr, juge en Israël}
\VS{3}Après lui se leva Jaïr, le Galaadite, qui fut juge en Israël pendant vingt-deux ans.
\VS{4}Il avait trente fils, qui montaient sur trente ânons, et qui avaient trente villes, qu'on appelle jusqu'à ce jour bourgs de Jaïr, lesquelles sont situées au pays de Galaad\FTNT{Jg. 5:10.}.
\VS{5}Et Jaïr mourut, et fut enterré à Kamon.
\TextTitle{Idolâtrie d’Israël et oppression par ses ennemis}
\VS{6}Puis les enfants d'Israël firent encore ce qui est mal aux yeux de Yahweh ; et servirent les Baals et les Astartés, les dieux de Syrie, les dieux de Sidon, les dieux de Moab, les dieux des fils d'Ammon, et les dieux des Philistins, et ils abandonnèrent Yahweh, et ne le servirent plus\FTNT{Jg. 2:11 ; 3:7 ; 8:33.}.
\VS{7}Alors la colère de Yahweh s'enflamma contre Israël, et il les vendit entre les mains des Philistins, et des fils d'Ammon.
\VS{8}Ils opprimèrent et écrasèrent les enfants d'Israël cette année-là, et pendant dix-huit ans tous les enfants d'Israël qui étaient au-delà du Jourdain, au pays des Amoréens en Galaad.
\VS{9}Même les fils d'Ammon passèrent le Jourdain pour combattre contre Juda, contre Benjamin, et contre la maison d'Ephraïm. Israël fut dans une grande détresse.
\VS{10}Alors les enfants d'Israël crièrent à Yahweh, en disant : Nous avons péché contre toi, et certes, nous avons abandonné notre Dieu et nous avons servi les Baals.
\VS{11}Mais Yahweh répondit aux enfants d'Israël : N'avez-vous pas été opprimés par les Egyptiens, les Amoréens, les fils d'Ammon et les Philistins ?
\VS{12}Et lorsque les Sidoniens, Amalek et Maon, vous opprimèrent, et que vous criâtes à moi, ne vous ai-je pas délivrés de leurs mains ?
\VS{13}Mais vous, vous m'avez abandonné, et vous avez servi d'autres dieux. C'est pourquoi je ne vous délivrerai plus.
\VS{14}Allez et criez vers les dieux que vous avez choisis ; qu'ils vous délivrent au temps de votre détresse !
\VS{15}Mais les enfants d'Israël répondirent à Yahweh : Nous avons péché ; traite-nous comme tu le trouveras bon. Nous te prions seulement que tu nous délivres aujourd'hui !
\VS{16}Alors ils ôtèrent du milieu d'eux les dieux des étrangers, et servirent Yahweh, qui fut affligé des souffrances d'Israël.
\VS{17}Les fils d'Ammon se rassemblèrent et campèrent en Galaad, et les enfants d'Israël se rassemblèrent et campèrent à Mitspa.
\VS{18}Le peuple, les chefs de Galaad se dirent l'un à l'autre : Qui sera l'homme qui commencera à combattre contre les fils d'Ammon ? Il sera chef de tous les habitants de Galaad.
\Chap{11}
\TextTitle{Jephté, juge en Israël}
\VerseOne{}Or Jephthé, le Galaadite, était un fort et vaillant homme. Il était le fils d'une femme prostituée ; et c'est Galaad qui l'avait engendré.
\VS{2}La femme de Galaad lui enfanta des fils ; et quand les fils de cette femme furent grands, ils chassèrent Jephthé, en lui disant : Tu n'auras pas d'héritage dans la maison de notre père, car tu es fils d'une autre femme.
\VS{3}Jephthé s'enfuit donc de devant ses frères, et habita au pays de Tob. Des misérables se rassemblèrent auprès de Jephthé, et ils sortirent dehors avec lui\FTNT{Jg. 9:4 ; 1 S. 22:2 ; 1 S 10:6-8.}.
\VS{4}Et il arriva, quelque temps après, les fils d'Ammon firent la guerre à Israël.
\VS{5}Et comme les fils d'Ammon faisaient la guerre à Israël, les anciens de Galaad s'en allèrent pour emmener Jephthé du pays de Tob.
\VS{6}Ils dirent à Jephthé : Viens, et sois notre chef, afin que nous combattions contre les fils d'Ammon.
\VS{7}Jephthé répondit aux anciens de Galaad : N'est-ce pas vous qui m'avez haï et chassé de la maison de mon père ? Pourquoi êtes-vous venus à moi maintenant que vous êtes dans la détresse ?
\VS{8}Alors les anciens de Galaad dirent à Jephthé : La raison pour laquelle nous retournons à toi maintenant, c'est afin que tu viennes avec nous, que tu combattes contre les fils d'Ammon, et que tu sois notre chef, celui de tous les habitants de Galaad.
\VS{9}Jephthé répondit aux anciens de Galaad : Si vous me ramenez pour combattre contre les fils d'Ammon, et que Yahweh les livre devant moi, je serai votre chef.
\VS{10}Les anciens de Galaad dirent à Jephthé : Que Yahweh nous entende, et qu'il juge, si nous ne faisons pas ce que tu dis.
\VS{11}Jephthé donc s'en alla avec les anciens de Galaad. Le peuple le mit à sa tête et l'établit pour chef, et Jephthé déclara devant Yahweh, à Mitspa, toutes les paroles qu'il avait dites.
\VS{12}Puis Jephthé envoya des messagers au roi des fils d'Ammon, pour lui dire : Qu'y a-t-il entre toi et moi, que tu viennes contre moi pour faire la guerre à mon pays ?
\VS{13}Le roi des fils d'Ammon répondit aux messagers de Jephthé : C'est parce qu'Israël a pris mon pays quand il est monté d'Egypte, depuis l'Arnon jusqu'à Jabbok, et même jusqu'au Jourdain. Maintenant rends-le de bon gré.
\VS{14}Mais Jephthé envoya encore des messagers au roi des fils d'Ammon,
\VS{15}qui lui dirent : Ainsi parle Jephthé : Israël n'a rien pris du pays de Moab, ni du pays des fils d'Ammon.
\VS{16}Mais lorsqu’Israël est monté d'Egypte, il est venu par le désert jusqu'à la Mer Rouge et il a atteint Kadès.
\VS{17}Alors Israël envoya des messagers au roi d'Edom, pour lui dire : Que je passe, je te prie, par ton pays. Le roi d'Edom ne voulut pas l'entendre. Il en envoya aussi au roi de Moab, qui ne voulut pas non plus l'entendre. Et Israël demeura à Kadès.
\VS{18}Puis il marcha par le désert, tourna le pays d'Edom et le pays de Moab, et vint à l'orient du pays de Moab ; il campa au-delà de l'Arnon, et n'entra pas sur les frontières de Moab, car l'Arnon est la frontière de Moab.
\VS{19}Mais Israël envoya des messagers à Sihon, roi des Amoréens, roi de Hesbon, auquel Israël dit : Laisse-nous passer par ton pays jusqu'au lieu où nous allons.
\VS{20}Mais Sihon n'eut pas assez confiance en Israël pour le laisser passer sur son territoire ; il rassembla tout son peuple, ils campèrent vers Jahats, et combattirent contre Israël.
\VS{21}Et Yahweh, le Dieu d'Israël, livra Sihon et tout son peuple entre les mains d'Israël, qui les battit. Israël prit possession de tout le pays des Amoréens qui habitaient cette terre.
\VS{22}Ils conquirent donc tout le pays des Amoréens, depuis l'Arnon jusqu'à Jabbok, et depuis le désert jusqu'au Jourdain.
\VS{23}Et maintenant que Yahweh, le Dieu d'Israël, a dépossédé les Amoréens de devant son peuple d'Israël, aurais-tu la possession de leur pays ?
\VS{24}Ce que ton dieu Kemosch te donne à posséder, ne le posséderais-tu pas ? Et tout ce que Yahweh, notre Dieu, a mis en notre possession devant nous, nous ne le posséderions pas !
\VS{25}Or maintenant vaux-tu mieux en quelque sorte que ce soit que Balak, fils de Tsippor, roi de Moab ? A-t-il contesté et combattu contre Israël ?
\VS{26}Voilà trois cents ans qu'Israël demeure à Hesbon, et dans les villes de son ressort, à Aroër, et dans les villes de son ressort, et dans toutes les villes qui sont le long de l'Arnon : Pourquoi ne les avez-vous pas saisies pendant ce temps-là ?
\VS{27}Je ne t'ai pas offensé, mais tu fais mal de me faire la guerre. Que Yahweh, qui est le juge, juge aujourd'hui entre les enfants d'Israël et les fils d'Ammon !
\VS{28}Le roi des fils d'Ammon n'écouta pas les paroles que Jephthé lui fit dire.
\VS{29}L'Esprit de Yahweh fut sur Jephthé. Il passa au travers de Galaad et de Manassé ; il passa jusqu'à Mitspé de Galaad, et de Mitspé de Galaad, il passa jusqu'aux fils d'Ammon.
\TextTitle{Jephté fait un vœu ; Ammon livré entre ses mains}
\VS{30}Jephthé fit un vœu à Yahweh, et dit : Si tu livres les fils d'Ammon entre mes mains,
\VS{31}alors tout ce qui sortira des portes de ma maison au-devant de moi, quand je retournerai en paix chez les fils d'Ammon, sera consacré à Yahweh, et je l'offrirai en holocauste.
\VS{32}Jephthé passa jusqu'où étaient les fils d'Ammon, et Yahweh les livra entre ses mains.
\VS{33}Il les battit par une grande défaite, depuis Aroër jusqu'à Minnith, espace qui renfermait vingt villes, et jusqu'à Abel-Keramim. Et les fils d'Ammon furent humiliés devant les fils d'Israël.
\VS{34}Puis comme Jephthé retourna à Mitspa dans sa maison, voici, sa fille, qui était seule et unique, sans qu'il eût d'autres fils ou filles, sortit au-devant de lui avec des tambourins et des danses.
\VS{35}Et il arriva qu'aussitôt qu'il l'eut aperçue, il déchira ses vêtements, et dit : Ha ! Ma fille ! Tu m'as entièrement abaissé, tu es du nombre de ceux qui me troublent ! J'ai ouvert ma bouche à Yahweh, et je ne puis le révoquer.
\VS{36}Elle répondit : Mon père, si tu as ouvert ta bouche à Yahweh, fais-moi selon ce qui est sorti de ta bouche, puisque Yahweh t'a fait vengeance de tes ennemis, des fils d'Ammon.
\VS{37}Toutefois, elle dit à son père : Que ceci me soit fait : Laisse-moi pendant deux mois ! Je m'en irai, je descendrai par les montagnes, et je pleurerai ma virginité, avec mes compagnes.
\VS{38}Il répondit : Va ! Et il la laissa aller pour deux mois. Elle s'en alla donc avec ses compagnes, et pleura sa virginité dans les montagnes.
\VS{39}Et au bout de deux mois, elle retourna vers son père ; et il lui fit selon le vœu qu'il avait fait\FTNT{Yahweh interdit les sacrifices humains (Lé. 20:2-5 ; Lé. 21 ; De. 12:31 ; De. 18:10).}. Elle n'avait pas connu d'homme. Dès lors, ce fut une coutume en Israël,
\VS{40}tous les ans les filles d'Israël allaient pour célébrer la fille de Jephthé, le Galaadite, quatre jours par an.
\Chap{12}
\TextTitle{Querelle entre Jephté et Ephraïm}
\VerseOne{}Or les hommes d'Ephraïm se rassemblèrent, passèrent par le nord, et dirent à Jephthé : Pourquoi es-tu passé pour combattre contre les enfants d'Ammon, sans nous avoir appelés pour aller avec toi ? Nous brûlerons ta maison, et toi aussi\FTNT{Jg. 8:1.}.
\VS{2}Et Jephthé leur dit : J’ai eu un grand différend avec les enfants de Ammon, moi et mon peuple, et quand je vous ai appelés, vous ne m’avez point délivré de leurs mains.
\VS{3}Voyant que vous ne me délivriez pas, j'ai exposé ma vie, et je suis passé jusqu'où étaient les fils d'Ammon. Yahweh les a livrés entre mes mains. Pourquoi donc aujourd'hui montez-vous vers moi pour me faire la guerre ?
\VS{4}Puis Jephthé assembla tous les hommes de Galaad, et combattit contre Ephraïm. Les hommes de Galaad battirent Ephraïm, parce qu'ils disaient : Vous êtes des fugitifs d'Ephraïm ! Galaad est au milieu d'Ephraïm, au milieu de Manassé !
\VS{5}Les Galaadites se saisirent des gués du Jourdain du côté d'Ephraïm. Et quand l'un des fuyards d'Ephraïm disait : Que je passe ! Les hommes de Galaad lui disaient : Es-tu Ephraïmite ? Il répondait : Non.
\VS{6}Alors ils lui disaient : dis un peu « Schibboleth ». Et il disait « Sibboleth », car il ne pouvait pas le prononcer. Sur quoi, se saisissant de lui, ils le tuaient aux gués du Jourdain. En ce temps-là quarante-deux mille hommes d'Ephraïm périrent.
\VS{7}Jephthé fut juge en Israël pendant six ans ; puis Jephthé le Galaadite mourut, et fut enterré dans l'une des villes de Galaad.
\TextTitle{Ibstan, juge en Israël}
\VS{8}Après lui, Ibtsan de Bethléhem fut juge en Israël.
\VS{9}Il eut trente fils, il envoya trente filles au-dehors, et il fit venir du dehors trente filles pour ses fils. Il fut juge en Israël pendant sept ans.
\VS{10}Puis Ibtsan mourut, et fut enterré à Bethléhem.
\TextTitle{Le juge Elon}
\VS{11}Après lui, Elon de Zabulon fut juge en Israël pendant dix ans.
\VS{12}Puis Elon de Zabulon mourut, et fut enterré à Ajalon, dans le pays de Zabulon.
\TextTitle{Le juge Abdon}
\VS{13}Après lui, Abdon, fils d'Hillel, le Pirathonite, fut juge en Israël.
\VS{14}Il eut quarante fils et trente petits-fils, qui montaient sur soixante-dix ânons. Il fut juge en  Israël pendant huit ans\FTNT{Jg. 10:4. }.
\VS{15}Puis Abdon, fils d'Hillel, le Pirathonite, mourut, et fut enterré à Pirathon, dans le pays d'Ephraïm, sur la montagne des Amalécites.
\Chap{13}
\TextTitle{Israël à nouveau asservi pas les Philistins}
\VerseOne{}Et les enfants d’Israël recommencèrent à faire ce qui est mauvais aux yeux de Yahweh ; et Yahweh les livra entre les mains des Philistins, pendant quarante ans.
\TextTitle{Naissance du juge Samson}
\VS{2}Or il y avait un homme de Tsorea, de la famille des Danites, dont le nom était Manoach. Sa femme était stérile, et n'enfantait pas.
\VS{3}L’Ange de Yahweh apparut à la femme, et lui dit : Voici, tu es stérile, et tu n'as jamais eu d'enfants ; mais tu concevras, et tu enfanteras un fils.
\VS{4}Prends donc bien garde dès maintenant de ne boire ni vin ni liqueur forte, et de ne manger aucune chose impure.
\VS{5}Car voici tu vas être enceinte et tu enfanteras un fils. Le rasoir ne s'élèvera pas sur sa tête, parce que l'enfant sera Naziréen\FTNT{Naziréen vient du mot  «~nazir~» qui signifie «~consacré~» ou «~séparé~». Voir No. 6.} pour Dieu dès le ventre de sa mère ; et ce sera lui qui commencera à délivrer Israël de la main des Philistins.
\VS{6}Et La femme vint, et parla à son mari en disant : Un homme de Dieu est venu vers moi et il avait l'aspect d'un ange de Dieu, un aspect fort redoutable. Je ne lui ai pas demandé d'où il était, et il ne m'a pas déclaré son nom.
\VS{7}Mais il m'a dit : Tu vas être enceinte, et tu enfanteras un fils ; maintenant donc ne bois ni vin ni liqueur forte, et ne mange aucune chose impure, car cet enfant sera Naziréen pour Dieu dès le ventre de sa mère jusqu'au jour de sa mort.
\TextTitle{Prière de Manoach}
\VS{8}Et Manoach pria instamment Yahweh, et dit : Ah ! Seigneur, que l'homme de Dieu que tu as envoyé vienne encore vers nous, et qu'il nous enseigne ce que nous devons faire à l'enfant quand il naîtra !
\VS{9}Et Dieu exauça la prière de Manoach, et l'Ange de Dieu vint encore vers la femme lorsqu'elle était assise dans un champ ; mais Manoach, son mari, n'était pas avec elle.
\VS{10}Et la femme courut vite le rapporter à son mari, en lui disant : Voici, l'homme qui était venu vers moi l'autre jour m'est apparu.
\VS{11}Manoach se leva, suivit sa femme, et venant vers l'homme, il lui dit : Es-tu cet homme qui a parlé à cette femme ? Il répondit : C'est moi.
\VS{12}Manoach dit : Tout ce que tu as dit arrivera, quelle conduite faudra-t-il tenir envers l'enfant, et que lui faudra-t-il faire ?
\VS{13}L'Ange de Yahweh répondit à Manoach : La femme se gardera de tout ce que je lui ai dit.
\VS{14}Elle ne mangera rien qui sorte de la vigne, elle ne boira ni vin ni liqueur forte, et ne mangera aucune chose impure ; elle prendra garde à tout ce que je lui ai ordonné.
\VS{15}Alors Manoach dit à l'Ange de Yahweh : Permets que nous te retenions, et que nous apprêtions un chevreau en ta présence.
\VS{16}Et l'Ange de Yahweh répondit à Manoach : Quand tu me retiendrais, je ne mangerai pas de ton mets ; mais si tu fais un holocauste, tu l'offriras à Yahweh. Manoach ne savait pas que ce fût un Ange de Yahweh.
\VS{17}Et Manoach dit à l'Ange de Yahweh : Quel est ton nom, afin que nous te rendions les honneurs lorsque ta parole viendra ?
\VS{18}Et l'Ange de Yahweh lui répondit : Pourquoi demandes-tu mon nom ? Il est merveilleux.
\VS{19}Alors Manoach prit un chevreau, et une offrande, et les offrit à Yahweh sur le rocher. Il se produisit une chose merveilleuse à la vue de Manoach et de sa femme.
\VS{20}Comme la flamme montait de dessus l'autel vers les cieux, l'Ange de Yahweh monta aussi avec la flamme de l'autel. A cette vue, Manoach et sa femme tombèrent la face contre terre.
\VS{21}L'Ange de Yahweh n'apparut plus à Manoach ni à sa femme. Alors Manoach sut que c'était l'Ange de Yahweh.
\VS{22}Et Manoach dit à sa femme : Certainement nous mourrons, car nous avons vu Dieu.
\VS{23}Mais sa femme lui répondit : Si Yahweh avait voulu nous faire mourir, il n'aurait pas pris de nos mains l'holocauste ni l'offrande, il ne nous aurait pas fait voir toutes ces choses ni fait entendre les choses que nous avons entendues.
\VS{24}Puis cette femme enfanta un fils, et elle l'appela du nom de Samson. L'enfant devint grand, et Yahweh le bénit.
\VS{25}Et l'Esprit de Yahweh commença à l'agiter à Machané-Dan, entre Tsorea et Eschthaol.
\Chap{14}
\TextTitle{Yahweh, le maître des évènements}
\VerseOne{}Samson descendit à Thimna, et il y vit une femme d'entre les filles des Philistins.
\VS{2}Etant remonté dans sa maison, il le déclara à son père et à sa mère, en disant : J'ai vu une femme à Thimna d'entre les filles des Philistins ; prenez-la maintenant, afin qu'elle soit ma femme.
\VS{3}Son père et sa mère lui dirent : N'y a-t-il pas de femme parmi les filles de tes frères et parmi tout notre peuple, pour que tu ailles prendre une femme d'entre les Philistins, ces incirconcis ? Et Samson dit à son père : Prenez-la pour moi, car elle est droite à mes yeux.
\VS{4}Mais son père et sa mère ne savaient pas que cela venait de Yahweh : Car Samson cherchait une occasion de dispute de la part des Philistins. Or en ce temps-là, les Philistins dominaient sur Israël.
\TextTitle{L'énigme de Samson}
\VS{5}Samson descendit avec son père et sa mère à Thimna. Ils allèrent jusqu'aux vignes de Thimna, et voici, un jeune lion rugissant vint à sa rencontre.
\VS{6}Et l'Esprit de Yahweh saisit Samson ; sans avoir rien en sa main, il déchira le lion comme on déchire un chevreau. Il ne déclara pas à son père ni à sa mère ce qu'il avait fait\FTNT{1 S. 17:34-35.}.
\VS{7}Il descendit et parla à la femme, et elle fut trouvée droite à ses yeux.
\VS{8}Puis quelque temps après, il retourna à Thimna pour la prendre, et se détourna pour voir la carcasse du lion. Et voici, il y avait dans la carcasse du lion un essaim d'abeilles et du miel.
\VS{9}Il en prit entre ses mains, et s'en alla en mangeant ; et lorsqu'il fut arrivé vers son père et sa mère, il leur en donna, et ils en mangèrent. Mais il ne leur déclara pas qu'il avait pris ce miel dans la carcasse du lion.
\VS{10}Son père descendit chez la femme. Samson fit là un festin ; car c'est ainsi que les jeunes gens faisaient.
\VS{11}Dès qu'on le vit, on prit trente compagnons qui furent avec lui.
\VS{12}Samson leur dit : Je vous propose une énigme. Si vous me l'expliquez au cours des sept jours du festin, et si vous la trouvez, je vous donnerai trente chemises et trente vêtements de rechange.
\VS{13}Mais si vous ne pouvez pas me l'expliquer, vous me donnerez trente chemises et trente vêtements de rechange. Ils lui répondirent : Propose ton énigme, et nous l'écouterons.
\VS{14}Et il leur dit : De celui qui mange est sorti ce qui se mange, et du fort est sorti le doux. Pendant trois jours, ils ne purent pas expliquer l'énigme.
\VS{15}Et au septième jour, ils dirent à la femme de Samson : Persuade ton mari de nous expliquer l'énigme ; de peur que nous ne te brûlions au feu, toi et la maison de ton père. C'est pour nous déposséder que vous nous avez appelés ici, n'est-ce pas ?
\VS{16}La femme de Samson pleurait auprès de lui, et disait : Certainement tu me hais, et tu ne m'aimes pas ; tu as proposé une énigme aux enfants de mon peuple, et tu ne me l'as pas expliquée ! Et il lui répondait : Je ne l'ai expliquée ni à mon père ni à ma mère ; est-ce à toi que je l'expliquerais ?
\VS{17}Elle pleura ainsi auprès de lui durant les sept jours du festin ; mais au septième jour, il la lui expliqua, parce qu'elle le tourmentait. Puis elle l'expliqua aux enfants de son peuple.
\VS{18}Les gens de la ville lui dirent au septième jour, avant le coucher du soleil : Qu'y a-t-il de plus doux que le miel, et qu'y a-t-il de plus fort que le lion ? Et il leur dit : Si vous n'aviez pas labouré avec ma génisse vous n'auriez pas trouvé mon énigme.
\VS{19}L'Esprit de Yahweh le saisit, et il descendit à Askalon. Il tua trente hommes, il prit leurs dépouilles, et donna les vêtements de rechange à ceux qui avaient expliqué l'énigme. Sa colère s'enflamma, et il monta à la maison de son père.
\VS{20}Et la femme de Samson fut donnée à son compagnon, avec lequel il était lié.
\Chap{15}
\TextTitle{Samson utilisé pour le jugement des Philistins}
\VerseOne{}Et il arriva quelque jours après, au jour de la moisson des blés, que Samson alla visiter sa femme, et lui porta un chevreau. Il dit : J'entrerai vers ma femme dans sa chambre. Mais le père de sa femme ne lui permit pas d'y entrer.
\VS{2}Car il lui dit : J'ai cru que tu avais de la haine pour elle, c'est pourquoi je l'ai donnée à ton compagnon. Sa jeune sœur n'est-elle pas plus belle qu'elle ? Prends-la donc à sa place.
\VS{3}Samson leur dit : Cette fois je serai innocent à l'égard des Philistins si je leur fais du mal.
\VS{4}Samson s'en alla donc. Il prit trois cents renards, il prit aussi des torches ; puis il tourna les renards queue contre queue, et mit une torche entre les deux queues, au milieu.
\VS{5}Puis il mit le feu aux torches, et lâcha les renards dans les blés des Philistins, et brûla le tas de gerbes, le blé sur pied, jusqu'aux plantations d'oliviers.
\VS{6}Les Philistins dirent : Qui a fait cela ? On répondit : Samson, le gendre du Thimnien, parce qu'il lui a pris sa femme et l'a donnée à son compagnon. Les Philistins montèrent, et ils la brûlèrent au feu, elle avec son père.
\VS{7}Alors Samson leur dit : Est-ce donc ainsi que vous faites ? Je ne cesserai qu'après m'être vengé de vous.
\VS{8}Il les battit par une grande défaite, dos et ventre ; puis il descendit, et demeura dans une caverne du rocher d'Etam.
\VS{9}Alors les Philistins montèrent, campèrent en Juda, et s'étendirent jusqu'à Léchi.
\VS{10}Les hommes de Juda dirent : Pourquoi êtes-vous montés contre nous ? Ils répondirent : Nous sommes montés pour lier Samson, afin que nous lui fassions comme il nous a fait.
\VS{11}Alors trois mille hommes de Juda descendirent à la caverne du rocher d'Etam, et dirent à Samson : Ne sais-tu pas que les Philistins dominent sur nous ? Que nous as-tu donc fait ? Il leur répondit : Je leur ai fait comme ils m'ont fait.
\VS{12}Ils lui dirent : Nous sommes descendus pour te lier, afin de te livrer entre les mains des Philistins. Samson leur dit : Jurez-moi que vous ne me tuerez pas.
\VS{13}Ils lui répondirent, en disant : Non ; mais nous te lierons, afin de te livrer entre leurs mains, mais nous ne te tuerons pas. Ils le lièrent avec deux cordes neuves, et le firent monter hors du rocher.
\VS{14}Lorsqu'il entra à Léchi, les Philistins poussèrent des cris de joie à sa rencontre. Alors l'Esprit de Yahweh le saisit. Les cordes qui étaient sur ses bras devinrent comme du lin brûlé par le feu, et les liens tombèrent de ses mains.
\VS{15}Il trouva une mâchoire d'âne fraîche, il étendit sa main, la prit, et il en tua mille hommes.
\VS{16}Puis Samson dit : Avec une mâchoire d'âne, un monceau, deux monceaux ; avec une mâchoire d'âne, j'ai tué mille hommes.
\VS{17}Quand il cessa de parler, il jeta de sa main la mâchoire. On appela ce lieu Ramath-Léchi.
\VS{18}Il eut extrêmement soif, et invoqua Yahweh en disant : Tu as accordé par la main de ton serviteur cette grande délivrance ; et maintenant mourrais-je de soif, et tomberais-je entre les mains des incirconcis\FTNT{1 S. 17:26.} ?
\VS{19}Alors Dieu fendit la cavité du rocher qui est à Léchi, et il en sortit de l'eau. Samson but, l'Esprit lui revint, et il reprit vie. C'est pourquoi on a appelé cette source du nom d'En-Hakkoré ; elle existe encore aujourd'hui  à Léchi.
\VS{20}Samson fut juge en Israël, au temps des Philistins, pendant vingt ans\FTNT{Jg. 16:31.}.
\Chap{16}
\TextTitle{Faiblesse de Samson}
\VerseOne{}Or Samson s'en alla à Gaza ; il y vit une femme prostituée, et il entra chez elle.
\VS{2}On dit aux gens de Gaza : Samson est venu ici. Ils l'entourèrent, et se tinrent en embuscade toute la nuit à la porte de la ville. Ils restèrent tranquilles toute la nuit, en disant : Au point du jour, nous le tuerons.
\VS{3}Samson demeura couché jusqu'à minuit. Au milieu de la nuit, il se leva ; et il saisit les battants des portes de la ville et les deux poteaux, les retira avec la barre, les mit sur ses épaules, et les porta sur le sommet de la montagne qui est en face d'Hébron.
\VS{4}Après cela, il aima une femme dans la vallée de Sorek. Elle se nommait Delila.
\VS{5}Les princes des Philistins montèrent vers elle, et lui dirent : Séduis-le, jusqu'à ce que tu saches de lui en quoi consiste sa grande force, et comment pourrions-nous le vaincre ; afin que nous le lions pour l'abattre, et nous te donnerons chacun mille cent sicles d'argent.
\VS{6}Delila dit à Samson : Dis-moi, je te prie, en quoi consiste ta grande force, et avec quoi il faudrait te lier pour t'abattre.
\VS{7}Samson lui répondit : Si on me liait avec sept cordes fraîches, qui ne soient pas encore sèches, je deviendrais faible et je serais comme un autre homme.
\VS{8}Les princes des Philistins emmenèrent à Delila sept cordes fraîches, qui n'étaient pas encore sèches. Et elle le lia.
\VS{9}Or il y avait chez elle, dans une chambre, des gens qui se tenaient en embuscade. Elle lui dit : Les Philistins sont sur toi, Samson ! Alors il rompit les cordes comme se romprait un cordon d'étoupe dès qu'il sent le feu. Et l'on ne connut pas d'où lui venait sa force.
\VS{10}Puis Delila dit à Samson : Voici, tu t'es moqué de moi, car tu m'as dit des mensonges. Je te prie, déclare-moi maintenant avec quoi il faut te lier.
\VS{11}Il lui répondit : Si on me liait avec des cordes neuves, dont on ne se serait jamais servi pour un quelconque ouvrage, je deviendrais faible, et je serais comme un autre homme.
\VS{12}Delila prit des cordes neuves avec lesquelles elle le lia. Puis elle lui dit : Les Philistins sont sur toi, Samson ! Or il y avait des gens en embuscade dans une chambre. Et il rompit les cordes comme un fil.
\VS{13}Puis Delila dit à Samson : Tu t'es moqué de moi, jusqu'ici tu m'as dit des mensonges. Déclare-moi avec quoi il faut te lier. Il lui dit : Tu n'as qu'à tresser les sept tresses de ma tête avec la chaîne du tissu.
\VS{14}Et elle les fixa par la cheville. Puis elle dit : Les Philistins sont sur toi, Samson ! Alors il se réveilla de son sommeil, et il retira la chaîne du tissu.
\TextTitle{Samson révèle son secret}
\VS{15}Alors elle lui dit : Comment peux-tu dire : Je t'aime ! Puisque ton cœur n'est pas avec moi ? Tu t'es moqué de moi par trois fois, et tu ne m'as pas déclaré en quoi consiste ta grande force.
\VS{16}Comme elle le tourmentait et l'importunait tous les jours par ses paroles, son âme en fut affligée jusqu'à la mort,
\VS{17}alors il lui ouvrit tout son cœur, et lui dit : Le rasoir n'est jamais passé sur ma tête, car je suis Naziréen de Dieu dès le ventre de ma mère. Si j'étais rasé, ma force partirait, je me trouverais faible, et je serais comme tous les autres hommes.
\VS{18}Delila, voyant qu'il lui avait ouvert tout son cœur, envoya appeler les princes des Philistins, et leur fit dire : Montez cette fois, car il m'a ouvert tout son cœur. Les princes des Philistins montèrent vers elle, et emmenèrent l'argent dans leurs mains.
\VS{19}Elle l'endormit sur ses genoux. Et ayant appelé un homme, elle rasa les sept tresses de la tête de Samson, et commença à le dompter. Sa force partit.
\VS{20}Alors elle dit : Les Philistins sont sur toi, Samson ! Et il se réveilla de son sommeil, et dit : Je m'en sortirai comme les autres fois, et je me dégagerai. Mais il ne savait pas que Yahweh s'était retiré de lui\FTNT{L’immoralité sexuelle de Samson et sa désobéissance à Yahweh, dues à son manque de caractère, ont ruiné à jamais son ministère et compromis l’avenir du peuple d’Israël qu’il devait diriger (Jg. 16). Cet homme avait reçu un appel puissant dès le sein de sa mère, mais il ne vivait pas dans la crainte de Dieu. Le manque de discernement de Samson lui coûta ainsi toutes les grâces que le Seigneur lui avait accordées : La sainteté symbolisée par ses sept tresses, la force ou l’onction,  la vision, la liberté (Jg. 16:21).}.
\VS{21}Les Philistins donc le saisirent, et lui crevèrent les yeux ; ils le descendirent à Gaza, et le lièrent de deux chaînes d'airain. Il tournait la meule dans la prison\FTNT{2 S. 3:34.}.
\VS{22}Les cheveux de sa tête commencèrent à repousser, depuis qu'il avait été rasé.
\TextTitle{Samson achève le jugement des Philistins}
\VS{23}Or les princes des Philistins s'assemblèrent pour offrir un grand sacrifice à Dagon, leur dieu, et pour se réjouir. Ils disaient : Notre dieu a livré en nos mains Samson, notre ennemi.
\VS{24}Et quand le peuple le vit, il loua son dieu, en disant : Notre dieu a livré entre nos mains notre ennemi, celui qui ravageait notre pays, et qui multipliait nos morts.
\VS{25}Comme ils avaient le cœur joyeux, ils dirent : Qu'on appelle Samson, afin qu'il nous fasse rire ! Ils appelèrent Samson et le tirèrent de la prison ; et il joua devant eux. Ils le firent tenir entre les colonnes.
\VS{26}Alors Samson dit au garçon qui le tenait par la main : Laisse-moi afin que je puisse toucher les colonnes sur lesquelles repose la maison pour que je m'y appuie.
\VS{27}Or la maison était remplie d'hommes et de femmes ; tous les princes des Philistins y étaient, et il y avait même sur le toit près de trois mille personnes, hommes et femmes, qui regardaient Samson jouer.
\VS{28}Alors Samson invoqua Yahweh, et dit : Seigneur Yahweh ! Je te prie, souviens-toi de moi ; ô Dieu ! Fortifie-moi seulement cette fois, et que par un coup je me venge des Philistins pour mes deux yeux\FTNT{Hé. 11:32.} !
\VS{29}Samson embrassa les deux colonnes du milieu sur lesquelles reposait la maison, et il s'appuya contre elles ; l'une à sa droite, et l'autre à sa gauche.
\VS{30}Et il dit : Que mon âme meure avec les Philistins ! Il se pencha donc de toute sa force, et la maison tomba sur les princes et sur tout le peuple qui y était. Et il fit mourir beaucoup plus de gens à sa mort, qu'il n'en avait fait mourir pendant sa vie.
\VS{31}Ensuite ses frères et toute la maison de son père descendirent, et le transportèrent. Lorsqu'ils furent montés, ils l'enterrèrent entre Tsorea et Eschthaol dans le sépulcre de Manoach, son père. Il avait été juge en Israël pendant vingt ans\FTNT{Jg. 13:2.}.
\Chap{17}
\TextTitle{Confusion en Israël}
\VerseOne{}Il y avait un homme de la montagne d'Ephraïm, nommé Mica.
\VS{2}Il dit à sa mère : Les mille cent sicles d'argent qu'on t'a pris, et pour lesquels tu as fait des imprécations même à mes oreilles, voici, j'ai cet argent, c'est moi qui l'avais pris. Alors sa mère dit : Béni soit mon fils par Yahweh !
\VS{3}Et il rendit à sa mère les mille cent sicles d'argent ; sa mère dit : Je consacre de ma main cet argent à Yahweh, afin d'en faire pour mon fils une image taillée, et une image en métal fondu ; et c'est ainsi que je te le rendrai.
\VS{4}Et il rendit l'argent à sa mère. Elle prit deux cents sicles d'argent et les donna au fondeur, qui en fit une image taillée, et une image en métal fondu.  On les plaça dans la maison de Mica.
\VS{5}Ainsi cet homme, savoir Mica, avait une maison de Dieu ; il fit un éphod et des téraphim, et il consacra par sa main l'un de ses fils, qui lui servit de sacrificateur.
\VS{6}En ce temps-là, il n'y avait pas de roi en Israël. Chacun faisait ce qui lui semblait être droit à ses yeux\FTNT{Jg. 18:1}.
\VS{7}Or il y avait un jeune homme de Bethléhem de Juda, de la famille de la tribu de Juda ; il était Lévite, et il séjournait là.
\VS{8}Cet homme partit de la ville de Bethléhem de Juda, pour trouver une demeure qui lui convienne.  En chemin, il entra dans la montagne d'Ephraïm jusqu'à la maison de Mica.
\VS{9}Mica lui dit : D'où viens-tu ? Il lui répondit : Je suis Lévite, de Bethléhem de Juda, et je voyage pour trouver une demeure qui me convienne.
\VS{10}Mica lui dit : Demeure avec moi ; tu me serviras de père et de sacrificateur, et je te donnerai dix sicles d'argent par an, les vêtements d'ordre dont tu auras besoin, et ton entretien. Et le Lévite vint\FTNT{Jg. 18:19.}.
\VS{11}Ainsi le Lévite convint de demeurer avec cet homme, qui regarda le jeune homme comme l'un de ses fils.
\VS{12}Mica consacra\FTNT{"Consacrer" signifie litteralement "remplir la main".} le Lévite, qui lui servit de sacrificateur, et qui demeura dans sa maison.
\VS{13}Mica dit : Maintenant je sais que Yahweh me fera du bien, parce que j'ai un Lévite pour sacrificateur.
\Chap{18}
\TextTitle{Dan recherche un territoire}
\VerseOne{}En ce temps-là, il n'y avait pas de roi en Israël ; et en ce même temps la tribu des Danites cherchait un héritage afin de pouvoir s'établir, car jusqu'à ce jour il ne lui était pas échu d'héritage au milieu des tribus d'Israël\FTNT{Jg. 17:6.}.
\VS{2}C'est pourquoi les fils de Dan envoyèrent de leur famille cinq hommes vaillants, de Tsorea et d'Eschthaol, pour explorer le pays et l'examiner. Ils leur dirent : Allez examiner le pays. Ils entrèrent dans la montagne d'Ephraïm jusqu'à la maison de Mica, et ils y passèrent la nuit.
\VS{3}Comme ils étaient près de la maison de Mica, ils reconnurent la voix du jeune homme Lévite et lui dirent : Qui t'a amené ici ? Qu'y fais-tu ? Que fais-tu ici ?
\VS{4}Il leur répondit : Mica fait pour moi telle et telle chose, il me donne un salaire, et je lui sers de sacrificateur.
\VS{5}Ils lui dirent : Nous te prions consulte Dieu, afin que nous sachions si le voyage que nous entreprenons prospérera.
\VS{6}Et le sacrificateur leur répondit : Allez en paix ; Yahweh a sous ses yeux le voyage que vous mènerez.
\VS{7}Ces cinq hommes s'en allèrent, et entrèrent à Laïs. Ils virent le peuple qui y habitait en sécurité selon les coutumes des Sidoniens, tranquille et en confiance ; il n'y avait personne au pays qui les humiliait en quelque chose en dominant sur eux ; ils étaient éloignés des Sidoniens, et ils n'avaient aucune affaire avec d'autres hommes.
\VS{8}Puis ils vinrent auprès de leurs frères à Tsorea et à Eschthaol, et leurs frères leur dirent : Quelle nouvelle rapportez-vous ?
\VS{9}Et ils répondirent : Allons ! Montons contre eux ; car nous avons vu le pays, et nous l'avons trouvé très bon. Quoi ! Vous restez sans rien faire ? Ne soyez pas paresseux pour aller posséder ce pays.
\VS{10}Quand vous y entrerez, vous irez vers un peuple en sécurité. Le pays est vaste, Dieu l'a livré entre vos mains ; c'est un lieu où il ne manque rien de tout ce qui est sur la terre.
\VS{11}Il partit de Tsorea et d'Eschthaol, six cents hommes de la famille de Dan, munis de leurs armes de guerre.
\VS{12}Ils montèrent, et campèrent à Kirjath-Jearim en Juda ; c'est pourquoi on a appelé ce lieu qui est derrière Kirjath-Jearim jusqu'à ce jour, Machané-Dan.
\VS{13}Puis ils passèrent par la montagne d'Ephraïm, et ils entrèrent dans la maison de Mica.
\TextTitle{Campagnes de la tribu de Dan}
\VS{14}Alors les cinq hommes qui étaient allés explorer le pays de Laïs prirent la parole et dirent à leurs frères : Savez-vous qu'il y a dans ces maisons-là un éphod, des théraphim, une image taillée et une image en métal fondu ? Voyez maintenant ce que vous avez à faire.
\VS{15}Alors ils se détournèrent de ce lieu, et entrèrent dans la maison où était le jeune homme Lévite, dans la maison de Mica, et lui demandèrent comment il se portait.
\VS{16}Et les six cents hommes d'entre les fils de Dan, qui étaient munis de leurs armes de guerre, se tenaient à l'entrée de la porte.
\VS{17}Mais les cinq hommes, qui étaient allés explorer le pays, montèrent et entrèrent dans la maison ; ils prirent l'image taillée, l'éphod, les théraphim, et l'image en métal fondu, pendant que le sacrificateur était à l'entrée de la porte avec les six cents hommes munis de leurs armes de guerre.
\VS{18}Etant entrés dans la maison de Mica, ils prirent l'image taillée, l'éphod, les théraphim, et l'image en métal fondu. Le sacrificateur leur dit : Que faites-vous ?
\VS{19}Ils lui répondirent : Tais-toi, mets ta main sur ta bouche, et viens avec nous ; sois pour nous un père et un sacrificateur. Vaut-il mieux que tu serves de sacrificateur à la maison d'un homme seul, ou que tu serves de sacrificateur à une tribu et à une famille en Israël\FTNT{Jg. 17:10.} ?
\VS{20}Le sacrificateur eut de la joie dans son cœur ; il prit l'éphod, les théraphim, et l'image taillée, et vint au milieu du peuple.
\VS{21}Après quoi ils se retournèrent et marchèrent, en mettant devant eux les petits enfants, le bétail, et les bagages.
\VS{22}Comme ils étaient loin de la maison de Mica, les gens qui  habitaient les maisons voisines de celle de Mica furent assemblés à grand cri ; et poursuivirent les fils de Dan.
\VS{23}Et ils crièrent aux fils de Dan, qui se tournèrent de face et dirent à Mica : Qu’as-tu, que tu te sois ainsi écrié pour rassembler ces gens ?
\VS{24}Il répondit : Vous avez enlevé mes dieux que j'avais faits, vous avez pris le sacrificateur, et vous vous en êtes allés : Que me reste-t-il ? Comment pouvez-vous me dire : Qu'as-tu\FTNT{Ge. 31:30.} ?
\VS{25}Les fils de Dan lui dirent : Ne fais pas entendre ta voix après nous, de peur que des hommes exaspérés ne se jettent sur vous, et que vous n’y laissiez la vie, toi, et tous ceux de ta famille.
\VS{26}Les fils de Dan firent leur chemin. Mica, voyant qu'ils étaient plus forts que lui, s'en retourna et revint dans sa maison.
\VS{27}Ainsi ils prirent les choses que Mica avait faites, et le sacrificateur qu'il avait, et ils entrèrent à Laïs, vers un peuple tranquille et en sécurité ; ils les firent passer au fil de l'épée, et ils brûlèrent la ville.
\VS{28}Et il n'y eut personne qui la délivrât, car elle était éloignée de Sidon, et ses habitants n'avaient pas d'affaires avec les autres hommes : Elle était située dans la vallée qui appartenait au pays de Beth-Rehob. Les fils de Dan rebâtirent la ville, et y demeurèrent.
\VS{29}Ils appelèrent la ville Dan, selon le nom de Dan, leur père qui était né à Israël ; mais la ville s'appelait auparavant Laïs\FTNT{Jos. 19:47.}.
\VS{30}Et les fils de Dan dressèrent l'image taillée ; et Jonathan, fils de Guerschom, fils de Manassé, lui et ses fils, furent sacrificateurs pour la tribu des Danites, jusqu'au jour de la captivité du pays.
\VS{31}Ils y dressèrent donc l'image taillée que Mica avait faite, pendant tout le temps que la maison de Dieu fut à Silo.
\Chap{19}
\TextTitle{Dégradation morale}
\VerseOne{}Il arriva aussi en ce temps-là, où il n'y avait pas de roi en Israël, qu'un Lévite qui habitait aux côtés de la montagne d'Ephraïm, prit pour concubine une femme de Bethléhem de Juda\FTNT{Jg. 17:6 ; 21:25.}.
\VS{2}Mais sa concubine se prostitua chez lui, et elle s'en alla pour aller dans la maison de son père à Bethléhem de Juda, où elle resta pendant quatre mois.
\VS{3}Puis son mari se leva et alla après elle, pour parler à son cœur, et la ramener. Il avait avec lui son serviteur et deux ânes. Elle le fit entrer dans la maison de son père ; et quand le père de la jeune femme le vit, il s'approcha avec joie.
\VS{4}Son beau-père, le père de la jeune femme, le retint avec grande instance, de sorte qu'il demeura trois jours avec lui. Ils mangèrent et burent, et logèrent là.
\VS{5}Le quatrième jour, ils se levèrent de bon matin, et le Lévite se levait pour s'en aller. Mais le père de la jeune femme dit à son gendre : Fortifie ton cœur avec un morceau de pain, et vous partirez ensuite.
\VS{6}Ils s'assirent, et ils mangèrent et burent eux deux ensemble. Puis le père de la jeune femme dit au mari : Je te prie consens à passer encore ici cette nuit, et que ton cœur se réjouisse.
\VS{7}Le mari se levait pour s'en aller ; mais son beau-père le pressa tellement, qu'il s'en retourna, et y passa encore la nuit.
\VS{8}Le cinquième jour, il se leva de bon matin pour s'en aller. Alors le père de la jeune femme dit : Fortifie ton cœur ; et attendez le déclin du jour. Et ils mangèrent eux deux.
\VS{9}Puis le mari se levait pour s'en aller, avec sa concubine et son serviteur ; mais son beau-père, le père de la jeune femme, lui dit : Voici, maintenant le jour baisse, il se fait tard, je vous prie passez ici la nuit ; voici le jour est sur son déclin, passe ici la nuit, et que ton cœur se réjouisse ; demain matin vous vous mettrez en route, et tu t'en iras à ta tente.
\VS{10}Mais le mari ne voulut pas y passer la nuit, il se leva, et s'en alla.  Il vint jusque vis-à-vis de Jébus, qui est Jérusalem, avec les deux ânes bâtés et sa concubine.
\VS{11}Comme ils étaient près de Jébus, le jour avait beaucoup baissé. Le serviteur dit à son maître : Allons, détournons-nous vers cette ville des Jébusiens, afin que nous y passions la nuit.
\VS{12}Son maître lui répondit : Nous ne nous détournerons pas vers une ville d'étrangers, où il n'y a pas d'enfants d'Israël, mais nous passerons par Guibea.
\VS{13}Il dit aussi à son serviteur : Allons, approchons-nous de l'un de ces lieux, Guibea ou Rama, et passons-y la nuit.
\VS{14}Ils continuèrent à marcher, et le soleil se coucha quand ils furent près de Guibea, qui appartient à Benjamin.
\VS{15}Alors ils se détournèrent vers Guibea, et y entrèrent pour passer la nuit. Le Lévite entra, et il s'assit sur la place de la ville. Il n'y eut aucun homme qui les reçut dans sa maison afin qu'ils y passent la nuit.
\VS{16}Et voici, sur le soir, un vieil homme venait de travailler dans les champs ; cet homme était de la montagne d'Ephraïm, il séjournait à Guibea, et les gens du lieu étaient Benjamites.
\VS{17}Et levant ses yeux, il vit le voyageur sur la place de la ville. Le vieil homme lui dit : Où vas-tu, et d'où viens-tu ?
\VS{18}Il lui répondit : Nous passons de Bethléhem de Juda vers les côtés de la montagne d'Ephraïm, d'où je suis. J'étais allé jusqu'à Bethléhem de Juda, mais maintenant je m'en vais à la maison de Yahweh. Mais il n'y a aucun homme qui me reçoive dans sa maison.
\VS{19}Nous avons pourtant de la paille et du fourrage pour nos ânes ; du pain et du vin pour moi,  pour ta servante, et pour le garçon qui est avec tes serviteurs. Nous n'avons besoin d'aucune chose.
\VS{20}Le vieil homme dit : Pourvu que la paix soit ! Quoi qu'il en soit, je me charge de tous tes besoins, je te prie seulement de ne pas passer la nuit sur la place.
\VS{21}Alors il les fit entrer dans sa maison, et il donna du fourrage aux ânes. Les voyageurs se lavèrent les pieds ; puis ils mangèrent et burent\FTNT{Ge. 43:24.}.
\VS{22}Comme ils se réjouissaient, voici, les hommes de la ville, fils d'hommes pervers, environnèrent la maison, frappèrent à la porte, et dirent au vieil homme, maître de la maison : Fais sortir l'homme qui est entré dans ta maison, afin que nous le connaissions\FTNT{Jg. 20:13 ; Os. 9:9 ; 10:9 ; Ge. 19:4.}.
\VS{23}Mais cet homme, savoir le maître de la maison, sortit vers eux, et leur dit : Non, mes frères, ne lui faites pas de mal, je vous prie ; puisque cet homme est entré dans ma maison, ne faites pas une telle infamie.
\VS{24}Voici, j'ai une fille vierge, et cet homme a une concubine ; je vous les amènerai dehors ; vous les déshonorerez, et vous ferez d'elles comme il semblera bon à vos yeux. Mais ne faites pas cette action infâme à l'égard de cet homme.
\VS{25}Mais ces gens ne voulurent pas l'écouter. C'est pourquoi l'homme saisit sa concubine, et la leur amena dehors. Ils la connurent, et abusèrent d'elle toute la nuit jusqu'au matin ; puis ils la renvoyèrent au lever de l'aurore.
\VS{26}Vers le matin, cette femme alla tomber à la porte de la maison de l'homme où était son mari, et elle y demeura jusqu'au jour.
\VS{27}Et le matin, son mari se leva, et ayant ouvert la porte de la maison, il sortit pour poursuivre son chemin. Mais voici, la femme concubine était tombée à la porte de la maison, et avait les mains sur le seuil.
\VS{28}Il lui dit : Lève-toi, et allons-nous-en. Mais elle ne répondit pas. Alors il l'emmena sur un âne, se mit en chemin, et s'en alla dans sa demeure.
\VS{29}En entrant en sa maison, il prit un couteau, et saisissant sa concubine, il la coupa avec ses os en douze morceaux, qu'il envoya dans tout le territoire d'Israël.
\VS{30}Et il arriva que tous ceux qui virent cela dirent : Une telle chose n'a été faite ni vue depuis le jour où les enfants d'Israël sont montés hors du pays d'Egypte, jusqu'à ce jour ; prenez la chose à cœur, consultez-vous, et parlez !
\Chap{20}
\TextTitle{Israël devant Yahweh à Mitspa}
\VerseOne{}Alors tous les fils d'Israël sortirent, et toute l'assemblée se réunit comme un seul homme, depuis Dan jusqu'à Beer-Schéba et jusqu'au pays de Galaad, devant Yahweh, à Mitspa.
\VS{2}Les chefs de tout le peuple, toutes les tribus d'Israël, se présentèrent à l'assemblée du peuple de Dieu, au nombre de quatre cent mille hommes de pied, tirant l'épée.
\VS{3}Les fils de Benjamin entendirent que les fils d'Israël étaient montés à Mitspa. Les fils d'Israël dirent : Parlez, comment ce mal est arrivé ?
\VS{4}Alors le Lévite, mari de la femme tuée, répondit, et dit : J'étais venu à Guibea de Benjamin, avec ma concubine, pour y passer la nuit.
\VS{5}Les seigneurs de Guibea se sont élevés contre moi, et ont encerclé de nuit la maison où j'étais. Ils avaient l'intention de me tuer, et ils ont tellement violé ma concubine qu'elle en est morte.
\VS{6}C'est pourquoi j'ai saisi ma concubine, je l'ai coupée en morceaux, et je les ai envoyés dans tout le territoire de l'héritage d'Israël ; car ils ont fait un crime et une infamie en Israël.
\VS{7}Vous voici tous, fils d'Israël ; consultez-vous sur la question, et prenez ici une décision !
\VS{8}Tout le peuple se leva comme un seul homme, et ils dirent : Aucun homme n'ira dans sa tente, et aucun homme ne se retirera dans sa maison.
\VS{9}Et maintenant voici ce que nous ferons à Guibea : Nous marcherons contre elle d'après le sort.
\VS{10}Nous prendrons dans toutes les tribus d'Israël dix hommes sur cent, cent sur mille, et mille sur dix mille ; nous prendrons des provisions pour le peuple, afin qu'en entrant à Guibea de Benjamin, on leur fasse selon toute l'infamie qu'elle a commise en Israël.
\VS{11}Ainsi tous les hommes d'Israël s'assemblèrent contre la ville, unis comme un seul homme.
\VS{12}Alors les tribus d'Israël envoyèrent des hommes vers la maison de Benjamin, pour dire : Quelle méchanceté a été faite parmi vous ?
\VS{13}Maintenant donc livrez-nous les fils des hommes pervers qui sont à Guibea, afin que nous les fassions mourir et que nous ôtions le mal du milieu d'Israël. Mais les fils de Benjamin ne voulurent pas écouter la voix de leurs frères, les enfants d'Israël.
\VS{14}Et les fils de Benjamin s'assemblèrent à Guibea pour sortir en guerre contre les fils d'Israël.
\VS{15}En ce jour-là, on fit le dénombrement des fils de Benjamin qui étaient dans ces villes, et il se trouva vingt-six mille hommes, tirant l'épée, sans compter les habitants de Guibea formant sept cents hommes d'élite.
\VS{16}De tout ce peuple, il y avait sept cents hommes d'élite qui ne se servaient pas de la main droite ; tous tirant la pierre avec la fronde,  à un cheveu près,  ils n'y manquaient pas.
\VS{17}On fit aussi le dénombrement des hommes d'Israël, excepté ceux de Benjamin, et l'on en trouva quatre cent mille hommes tirant l'épée, tous gens de guerre.
\TextTitle{Coalition pour monter contre Benjamin}
\VS{18}Et les fils d'Israël se levèrent, montèrent vers Dieu à Béthel pour le consulter, en disant : Qui d'entre nous montera le premier pour faire la guerre aux fils de Benjamin ? Yahweh répondit : Juda montera le premier.
\VS{19}Puis les fils d'Israël se levèrent de bon matin, et campèrent près de Guibea.
\VS{20}Et les hommes d'Israël sortirent pour combattre ceux de Benjamin, et se rangèrent en bataille près de Guibea.
\VS{21}Les fils de Benjamin sortirent de Guibea, et ils tuèrent ce jour-là vingt-deux mille hommes d'Israël.
\VS{22}Toutefois le peuple, les hommes d'Israël, se fortifièrent et se rangèrent de nouveau en bataille au lieu où ils s'étaient rangés le premier jour.
\VS{23}Et les fils d'Israël montèrent, et ils pleurèrent devant Yahweh jusqu'au soir ; ils consultèrent Yahweh en disant : M'approcherai-je encore pour combattre contre les fils de Benjamin, mon frère ? Yahweh dit : Montez contre lui.
\VS{24}Le second jour, les fils d'Israël s'approchèrent des fils de Benjamin.
\VS{25}Ce même jour, les Benjamites sortirent de Guibea à leur rencontre, et ils tuèrent encore dix-huit mille hommes des fils d'Israël, tous tirant l'épée.
\VS{26}Alors tous les fils d'Israël et tout le peuple montèrent et vinrent vers Dieu à Béthel ; ils pleurèrent, et restèrent là devant Yahweh. Ce jour-là ils jeûnèrent jusqu'au soir, et ils offrirent des holocaustes, et des sacrifices de paix devant Yahweh.
\VS{27}Ensuite les fils d'Israël consultèrent Yahweh, c'était là que se trouvait l'arche de l'alliance de Dieu ;
\VS{28}et Phinées, fils d'Eléazar, fils d'Aaron, se tenait devant Yahweh en ce temps-là en disant : Sortirai-je encore en guerre contre les fils de Benjamin, mon frère, ou dois-je m'en abstenir ? Yahweh répondit : Montez, car demain je les livrerai entre vos mains.
\VS{29}Alors Israël mit une embuscade autour de Guibea.
\VS{30}Le troisième jour, les fils d'Israël montèrent contre les fils de Benjamin, et ils se rangèrent en bataille contre Guibea, comme les autres fois.
\VS{31}Alors les fils de Benjamin sortirent à la rencontre du peuple, et ils furent attirés hors de la ville. Ils commencèrent à frapper à mort quelques-uns du peuple comme les autres fois, environ trente hommes d'Israël, sur les routes dont l'une monte à Béthel et l'autre à Guibea, par les champs.
\VS{32}Les fils de Benjamin disaient : Ils tombent battus devant nous, comme la première fois ! Mais les fils d'Israël disaient : Fuyons, et attirons-les hors de la ville dans les chemins.
\VS{33}Tous les hommes d'Israël se levant de leur lieu, se rangèrent à Baal-Thamar ; et l'embuscade sortit du lieu où ils étaient, de Maaré-Guibea.
\VS{34}Dix mille hommes choisis sur tout Israël vinrent contre Guibea. La bataille fut rude, et les Benjamites ne surent pas que le mal les atteindrait.
\VS{35}Yahweh battit Benjamin devant Israël, et les fils d'Israël tuèrent ce jour-là vingt-cinq mille cent hommes de Benjamin, tous tirant l'épée.
\VS{36}Les fils de Benjamin regardaient comme battus les hommes d'Israël, qui cédaient du terrain à Benjamin et se reposaient sur l'embuscade qu'ils avaient mise près de Guibea.
\VS{37}Ceux qui étaient en embuscade se jetèrent promptement sur Guibea, ils se portèrent en avant et frappèrent toute la ville au tranchant de l'épée.
\VS{38}Et le signal convenu entre les hommes d’Israël et l’embuscade était qu’ils fassent monter beaucoup de fumée de la ville.
\VS{39}Les hommes d’Israël avaient donc tourné le dos dans la bataille, et les Benjamites avaient commencé de frapper et de blesser à mort environ trente hommes de ceux d’Israël ; et ils disaient : Certainement ils tombent devant nous comme à la première bataille !
\VS{40}Mais quand l'épaisse colonne de fumée commençait à monter de la ville, les Benjamites se tournèrent ; et voici, derrière eux toute la ville disparaissait montant en feu vers le ciel.
\VS{41}Les hommes d'Israël tournèrent le visage ; et ceux de Benjamin furent épouvantés en voyant le mal qui allait les  atteindre.
\VS{42}Ils tournèrent le dos devant les hommes d'Israël par le chemin du désert. Mais les assaillants s'attachaient à leurs pas, et ils détruisirent ceux qui étaient sortis des villes.
\VS{43}Ils environnèrent Benjamin, le poursuivirent, l'écrasèrent dès qu'il voulut se reposer jusqu'en face de Guibea, du côté du soleil levant.
\VS{44}Il tomba dix-huit mille hommes de Benjamin, tous des vaillants hommes.
\VS{45}Et parmi ceux de Benjamin qui tournèrent le dos pour s'enfuir vers le désert au rocher de Rimmon, les hommes d'Israël en firent périr cinq mille hommes sur les routes ; et les poursuivant de près jusqu'à Guideom, ils frappèrent deux mille hommes.
\TextTitle{La tribu de Benjamin décimée}
\VS{46}En ce jour-là, le nombre de Benjamites qui tombèrent fut de vingt-cinq mille hommes tirant l'épée, et tous étaient des vaillants hommes.
\VS{47}Et il y eut six cents hommes de ceux qui avaient tourné le dos, qui s'échappèrent vers le désert au rocher de Rimmon, et qui demeurèrent au rocher de Rimmon pendant quatre mois.
\VS{48}Les hommes d'Israël retournèrent vers les fils de Benjamin, et ils les frappèrent du tranchant de l'épée, depuis les hommes des villes jusqu'aux bêtes, et tout ce qui s'y trouva. Ils brûlèrent toutes les villes qu'ils trouvaient.
\Chap{21}
\TextTitle{Deuil national}
\VerseOne{}Les hommes d'Israël avaient juré à Mitspa, en disant : Aucun homme ne donnera sa fille pour femme à un Benjamite.
\VS{2}Puis le peuple vint vers Dieu à Béthel, jusqu'au soir. Ils élevèrent leurs voix, et pleurèrent grandement,
\VS{3}Et ils dirent : Ô Yahweh, Dieu d'Israël, pourquoi est-il arrivé en Israël qu'une tribu d'Israël ait été aujourd'hui punie ?
\VS{4}Le lendemain, le peuple se leva de bon matin ; ils bâtirent là un autel, et ils offrirent des holocaustes et des sacrifices d'offrande de paix.
\VS{5}Alors les fils d'Israël dirent : Quel est celui d'entre toutes les tribus d'Israël qui n'est pas monté à l'assemblée vers Yahweh ? Car on avait fait un grand serment contre tout homme qui ne monterait pas vers Yahweh à Mitspa, en disant : Il sera puni de mort.
\VS{6}Les fils d'Israël se repentaient de ce qui était arrivé à Benjamin, leur frère, et ils disaient : Aujourd'hui une tribu a été retranchée d'Israël.
\VS{7}Comment ferons-nous pour donner des femmes à ceux qui ont survécu, puisque nous avons juré par Yahweh que nous ne leur donnerions pas nos filles pour femmes ?
\TextTitle{Avenir de la tribu de Benjamin}
\VS{8}Ils dirent donc : Y a-t-il quelqu'un d'entre les tribus d'Israël qui ne soit pas monté vers Yahweh à Mitspa ? Et voici, aucun homme de Jabès en Galaad n'était venu au camp, à l'assemblée.
\VS{9}Quand on fit le dénombrement du peuple, il n'y avait aucun des hommes habitant à Jabès en Galaad.
\VS{10}C'est pourquoi l'assemblée envoya contre eux douze mille hommes des fils vaillants, en leur donnant cet ordre : Allez, et frappez du tranchant de l'épée les habitants de Jabès en Galaad, tant les femmes que les enfants.
\VS{11}Voici les choses que vous ferez : Vous détruirez par le moyen de l'interdit tout mâle et toute femme qui a connu la couche d'un homme.
\VS{12}Ils trouvèrent parmi les habitants de Jabès en Galaad quatre cents filles vierges, qui n'avaient pas connu d'homme en couchant avec lui, et ils les amenèrent au camp de Silo, qui est sur la terre de Canaan.
\VS{13}Alors toute l'assemblée envoya parler aux fils de Benjamin qui étaient au rocher de Rimmon,  pour leur proclamer la paix.
\VS{14}En ce temps-là, les Benjamites revinrent, et on leur donna pour femmes celles qui avaient été conservées en vie d'entre les femmes de Jabès en Galaad. Mais ils n’en trouvèrent pas assez pour eux.
\VS{15}Le peuple se repentit de ce qui avait été fait à Benjamin, car Yahweh avait fait une brèche dans les tribus d'Israël.
\VS{16}Les anciens de l'assemblée dirent : Comment ferons-nous pour donner des femmes à ceux qui restent, car les femmes de Benjamin ont été détruites ?
\VS{17}Et ils dirent : Que ceux qui sont réchappés de Benjamin possèdent leur héritage, afin qu'une tribu d'Israël ne soit pas effacée.
\VS{18}Cependant, nous ne pouvons pas leur donner des femmes d'entre nos filles, car les fils d'Israël ont juré, en disant : Maudit soit celui qui donnera une femme à un Benjamite !
\VS{19}Et ils dirent : Voici, il y a chaque année une fête de Yahweh à Silo, qui est au nord de Béthel, à l'orient qui monte à Béthel, à Sichem, et au midi de Lebona.
\VS{20}Puis ils ordonnèrent aux fils de Benjamin : Allez, et placez-vous en embuscade dans les vignes.
\VS{21}Vous verrez, et voici, lorsque les filles de Silo sortiront pour danser, alors vous sortirez des vignes, vous enlèverez chacun une des filles de Silo pour en faire votre femme, et vous vous en irez dans le pays de Benjamin.
\VS{22}Si leurs pères ou leurs frères viennent se plaindre auprès de nous, nous leur dirons : Accordez-nous cette faveur, puisque nous n'avons pas pris de femmes pour chaque homme dans cette guerre.  Ce n'est pas vous qui les leur avez données ; sinon vous en seriez coupables en ce temps.
\VS{23}Les fils de Benjamin firent ainsi ; ils prirent des femmes selon leur nombre, parmi les danseuses qu'ils saisirent, puis ils s'en allèrent et retournèrent dans leur héritage ; ils rebâtirent les villes, et y habitèrent.
\VS{24}Ainsi en ce temps-là chacun des enfants d’Israël s’en alla de là dans sa tribu, et dans sa famille, et ils se retirèrent de là chacun dans son héritage.
\VS{25}En ce temps-là, il n'y avait pas de roi en Israël. L'homme faisait ce qui lui semblait être droit à ses yeux.
\PPE{}
\end{multicols}

%\clearpage\ShortTitle{1 Samuel}\BookTitle{1 Samuel}\BFont
\noindent\hrulefill
{\footnotesize
\textit{
\bigskip
{\centering{}
\\(Shemouel)
\\Signifie : Entendu, Exaucé de Dieu
\\Thème : Samuel, Saül et David
\\Auteur : Inconnu
\\Date de rédaction : 10ème siècle av. J.-C.\\}
}
%\bigskip
\textit{
\\Samuel naquit de l’union entre Elkana, de la montagne d’Ephraïm, et Anne.  Sa mère, que Yahweh avait rendu stérile, fit une alliance avec Dieu et lui promit de lui consacrer son premier fils. Ainsi, Samuel fut dès son plus jeune âge amené à la maison de Dieu où il grandit aux côtés d’Eli, le sacrificateur. A la mort de ce dernier, Samuel exerça les fonctions de juge, sacrificateur et prophète sur Israël. C’est en son temps qu’Israël manifesta le désir d’avoir un roi, marquant ainsi la fin de l’ère des juges et le début de la monarchie en Israël.
%\bigskip
\\Ce livre relate l’histoire de Saül, premier roi de l’histoire d’Israël, à qui Yahweh accorda de puissantes victoires notamment sur les philistins, grand ennemi du peuple de Dieu. Le parcours de Saul ne fut pas sans erreur, aussi Yahweh le disqualifia et choisit pour lui succéder sur le trône un homme de la tribu de Juda, David fils d’Isaï. L’accès à la royauté de ce dernier ne fut pas immédiat comme en témoignent ces écrits. David dut faire preuve de patience, de courage et de confiance en Dieu au milieu de nombreuses persécutions.
%\bigskip
\\Au travers de la vie des deux premiers rois d’Israël, est mise en évidence l’importance de l’obéissance à Dieu ; au travers de la vie de Samuel est mis en exergue l’impact de la prière dans une vie, une nation.\bigskip
}
}
\par\nobreak\noindent\hrulefill
\begin{multicols}{2}
\TextTitle{[Stérilité d'Anne la mère de Samuel]}
\Chap{1}
\VerseOne{}Il y avait un homme de Ramathaïm-Tsophim, de la montagne d'Ephraïm, nommé Elkana, fils de Jeroham, fils d'Elihu, fils de Thohu, fils de Tsuph, Ephratien.
\VS{2}Il avait deux femmes, dont l'une s'appelait Anne, et l'autre Peninna. Peninna avait des enfants, mais Anne n'en avait pas.
\VS{3}Or cet homme-là montait tous les ans, de sa ville à Silo\FTNT{Jos. 18:1.}, pour adorer Yahweh des armées, et lui offrir des sacrifices. Là étaient les deux fils d’Eli, Hophni et Phinées, sacrificateurs de Yahweh.
\VS{4}Le jour où Elkana offrait son sacrifice, il donnait des portions à Peninna, sa femme, à tous les fils et à toutes les filles qu'il avait d'elle.
\VS{5}Mais il donnait à Anne une portion double ; car il aimait Anne, mais Yahweh avait fermé sa matrice\FTNT{Dieu est celui qui ferme et ouvre les portes des bénédictions.}.
\VS{6}Sa rivale lui portait envie et la mortifiait fort aigrement afin de l’irriter, car Yahweh avait fermé sa matrice.
\VS{7}Et Elkana faisait donc ainsi tous les ans. Mais quand Anne montait à la maison de Yahweh, Peninna la mortifiait de la même manière, et Anne pleurait et ne mangeait pas.
\VS{8}Elkana, son mari, lui disait : Anne, pourquoi pleures-tu ? Et pourquoi ne manges-tu pas ? Pourquoi ton cœur est-il triste ? Est-ce que je ne vaux pas pour toi mieux que dix fils ?
\TextTitle{[Prière et voeu d'Anne à Yahweh]}
\VS{9}Anne se leva, après avoir mangé et bu à Silo. Et le sacrificateur Eli était assis sur un siège, près de l’un des poteaux du temple de Yahweh.
\VS{10}Elle donc, ayant le coeur rempli d'amertume, pria Yahweh en pleurant abondamment.
\VS{11}Et elle fit un vœu, en disant : Yahweh des armées ! Si tu regardes attentivement l'affliction de ta servante, et si tu te souviens de moi, et n'oublies pas ta servante, et que tu donnes à ta servante un enfant mâle, je le donnerai à Yahweh pour tous les jours de sa vie ; et aucun rasoir ne passera sur sa tête.
\VS{12}Il arriva, comme elle continuait à prier devant Yahweh, Eli observait sa bouche.
\VS{13}Or Anne parlait dans son cœur, elle ne faisait que remuer ses lèvres et on n'entendait pas sa voix. C’est pourquoi Eli estima qu'elle était ivre,
\VS{14}et Eli lui dit : Jusqu'à quand seras-tu ivre ? Eloigne-toi du vin.
\VS{15}Mais Anne répondit et dit : Je ne suis pas ivre, mon seigneur, je suis une femme affligée en son esprit, je n'ai bu ni vin ni boisson forte, mais je répandais mon âme devant Yahweh.
\VS{16}Ne mets pas ta servante au rang d'une femme pervertie, car c'est l’excès de ma douleur et de mon affliction qui m’a fait parler jusqu'à présent.
\VS{17}Alors Eli répondit et dit : Va en paix, et que le Dieu d'Israël veuille t’accorder la demande que tu lui as faite.
\VS{18}Et elle dit : Que ta servante trouve grâce à tes yeux ! Puis cette femme poursuivit son voyage. Elle mangea, et son visage ne fut plus le même.
\TextTitle{[Naissance de Samuel]}
\VS{19}Après cela, ils se levèrent de bon matin, et se prosternèrent devant Yahweh, puis ils s'en retournèrent et revinrent dans leur maison à Rama. Elkana connut Anne, sa femme, et Yahweh se souvint d'elle.
\VS{20}Il arriva donc, quelque temps après, qu'Anne conçut et enfanta un fils ; elle le nomma Samuel, parce que dit-elle, je l'ai demandé à Yahweh.
\VS{21}Puis Elkana, son mari, monta avec toute sa maison, pour offrir à Yahweh le sacrifice annuel et son vœu.
\VS{22}Mais Anne n'y monta pas, car elle dit à son mari : Je n’irai pas jusqu'à ce que le petit enfant soit sevré, et alors je le mènerai afin qu'il soit présenté devant Yahweh et qu'il demeure toujours-là.
\VS{23}Elkana, son mari, lui dit : Fais ce qui te semblera bon, reste jusqu'à ce que tu l'aies sevré. Seulement que Yahweh accomplisse sa parole. Ainsi cette femme resta et allaita son fils, jusqu'à ce qu'elle l’ait sevré.
\TextTitle{[Samuel chez Elie, Anne accomplie son voeu]}
\VS{24}Et dès qu'elle l'eut sevré, elle le fit monter avec elle, et ayant pris trois taureaux, un épha de farine et une outre de vin, elle le mena dans la maison de Yahweh à Silo ; l'enfant était très jeune.
\VS{25}Puis ils égorgèrent le veau, et ils amenèrent l'enfant à Eli.
\VS{26}Elle dit : Pardon, mon seigneur ! Aussi vrai que ton âme vit, mon seigneur, je suis cette femme qui me tenais en ta présence pour prier Yahweh.
\VS{27}J'ai prié pour avoir cet enfant, et Yahweh m’a accordé la demande que je lui ai faite.
\VS{28}C'est pourquoi je le prête à Yahweh ; il sera prêté à Yahweh pour tous les jours de sa vie. Et ils se prosternèrent là devant Yahweh.
\TextTitle{[Prière et prophésie d'Anne]}
\Chap{2}
\VerseOne{}Alors Anne pria, et dit : Mon cœur se réjouit en Yahweh ; ma force a été relevée par Yahweh ; ma bouche s'est ouverte contre mes ennemis, parce que je me suis réjouie de ton salut\FTNT{Le mot « salut » vient  de l’hébreu « yeshuw`ah » c’est-à-dire « Jésus ». Voir commentaire en Es. 26:1.}.
\VS{2}Nul n’est saint comme Yahweh ; car il n'y en a pas d'autre que toi, et il n'y a pas de rocher\FTNT{Voir commentaire en Es. 8:13-17.} tel que notre Dieu.
\VS{3}Ne proférez pas tant de paroles hautaines ; qu'il ne sorte pas de votre bouche des paroles arrogantes ; car Yahweh est le Dieu qui sait tout ; c'est lui qui pèse toutes les actions.
\VS{4}L'arc des puissants est brisé, mais ceux qui chancèlent ont la force pour ceinture.
\VS{5}Ceux qui étaient rassasiés, se louent pour du pain, mais les affamés ont cessé de l'être ; même la stérile en a enfanté sept et celle qui avait beaucoup de fils est devenue languissante.
\VS{6}Yahweh est celui qui fait mourir et qui fait vivre, qui fait descendre au scheol et qui en fait remonter.
\VS{7}Yahweh appauvrit et il enrichit, il abaisse et il élève.
\VS{8}De la poussière il retire le pauvre, du fumier il relève l’indigent, pour le faire asseoir avec les nobles, avec les nobles de son peuple, et il leur donne en héritage un trône de gloire ; car les colonnes de la terre sont à Yahweh, et il a posé le monde sur elles.
\VS{9}Il gardera les pieds de ses bien-aimés, et les méchants se tairont dans les ténèbres. Car l'homme ne triomphera pas par sa force.
\VS{10}Ceux qui contestent contre Yahweh seront effrayés ; des cieux il lancera son tonnerre sur chacun d'eux ; Yahweh jugera les extrémités de la terre ; et il donnera la force à son Roi\FTNT{Le Roi dont il est question ici est le Seigneur Jésus-Christ, le Roi des rois (Za. 14:9 ; Ap. 19:16). }, et élèvera la corne de son Messie\FTNT{Anne a annoncé la glorification ou la résurrection du Seigneur Jésus, le Messie (Jn. 3:14).}.
\VS{11}Puis Elkana s'en alla à Rama dans sa maison, et le jeune garçon vaquait au service de Yahweh, en présence du sacrificateur Eli.
\TextTitle{[Corruption des fils d'Eli]}
\VS{12}Or les fils d'Eli\FTNT{Les fils d’Eli, Hophni et Phinées étaient corrompus. Ils volaient les offrandes de Dieu, couchaient avec les femmes qui venaient adorer Dieu. L’esprit qui animait ces sacrificateurs  n’a pas disparu après leur mort, mais il opère  encore dans beuacoup d’institutions religieuses actuelles. Beaucoup de dirigeants d’églises continuent à s’aproprier ce qui appartient à Dieu (l’adoration, les âmes... ) Ils ne craignent pas Yahweh. Ils abusent de leur position et de leur autorité pour contraindre leurs fidèles à leur donner la dîme et toutes sortes d’offrandes. Ils font payer les entretiens, les prières, et les divers dons qu’ils peuvent avoir. Non seulement l'esprit qui animait les fils d'Eli existe encore, mais il s'est accru en ces temps actuels.} étaient des fils de Bélial\FTNT{Les fils d’Eli étaient qualifiés de « fils de Bélial ». Ce mot vient de l’hébreu « beliya`al » qui signifie « indigne », « bon à rien », « méchant », « ruine », « destruction ». Il est à noter que Bélial est aussi  un nom de Satan. (2 Co. 6:15). Les fils d’Eli servaient Dieu sans le connaître. En fait, ils étaient au service de Satan. Ce terme est également utilisé au sujet des méchants qui incitèrent les Israélites à servir les dieux étrangers (De. 13:14), les hommes iniques de Guivéa (Jg. 19:22 ;  Jg. 20:13), les deux vauriens qui accusèrent Naboth (1 R. 21:10-13) et les individus qui s’opposèrent à la monarchie (1 S. 10:27 ; 2 S. 20:1 ; 2 Ch. 13:7). Voir aussi  De. 13:13 ; De. 15:9 ; Job. 34:18 ; Ps. 18:4 ;  Ps. 34:8 ; Ps. 111:3 ; Pr. 6:12 ; Pr. 16:27 ; Pr. 19:28 ; Na. 1:11 ; Na. 1:18.} et ils ne connaissaient pas Yahweh,
\VS{13}et voici la coutume de ces sacrificateurs envers le peuple : Lorsque quelqu'un faisait quelque sacrifice, le serviteur du sacrificateur venait lorsqu'on faisait bouillir la chair, ayant à la main une fourchette à trois dents,
\VS{14}avec laquelle il piquait dans la chaudière, dans le chaudron, dans la marmite, dans le pot ; et le sacrificateur prenait pour lui tout ce que la fourchette enlevait. C’est ainsi qu’ils agissaient envers tous ceux d'Israël qui venaient à Silo.
\VS{15}Même avant qu'on fasse brûler la graisse, le serviteur du sacrificateur venait et disait à l'homme qui sacrifiait : Donne-moi de la chair à rôtir pour le sacrificateur ; car il ne prendra pas de toi de chair bouillie, mais de la chair crue.
\VS{16}Et si l'homme lui répondait : On va d’abord faire brûler la graisse, et après cela tu prendras ce que ton âme souhaitera, alors le serviteur lui disait : Quoi qu'il en soit, tu en donneras maintenant, sinon j'en prendrai de force.
\VS{17}Et le péché de ces jeunes hommes fut très grand devant Yahweh, car ils méprisaient l’offrande de Yahweh.
\TextTitle{[Samuel au service de Yahweh]}
\VS{18}Samuel faisait le service en présence de Yahweh, étant jeune garçon, vêtu d'un éphod de lin.
\VS{19}Sa mère lui faisait une petite tunique, qu'elle lui apportait tous les ans, quand elle montait avec son mari pour offrir le sacrifice annuel.
\VS{20}Eli bénit Elkana, et sa femme, et dit : Que Yahweh te donne des enfants de cette femme, pour le prêt qu’elle a fait à Yahweh. Et ils s'en retournèrent chez eux.
\VS{21}Et Yahweh visita Anne, elle conçut et enfanta trois fils et deux filles ; et le jeune garçon Samuel grandissait en présence de Yahweh.
\TextTitle{[Eli avertit ses fils]}
\VS{22}Or Eli était très vieux, il apprit tout ce que faisaient ses fils à tout Israël, et qu'ils couchaient avec les femmes qui s'assemblaient à la porte de la tente d'assignation.
\VS{23}Et il leur dit : Pourquoi commettez-vous de telles choses ? Car j'apprends vos méchantes actions de tout le peuple.
\VS{24}Ne faites pas ainsi, mes fils, car ce que j'entends dire de vous n'est pas bon ; vous faites pécher le peuple de Yahweh.
\VS{25}Si un homme a péché contre un autre homme, le juge interviendra ; mais si quelqu'un pèche contre Yahweh, qui interviendra pour lui ? Mais ils n'obéirent pas à la voix de leur père parce que Yahweh voulait les faire mourir.
\VS{26}Cependant le jeune garçon Samuel croissait et il était agréable à Yahweh et aux hommes.
\TextTitle{[Yahweh annonce un jugement sur la maison d'Eli]}
\VS{27}Or un homme de Dieu vint auprès d’Eli, et lui dit : Ainsi parle Yahweh : Ne me suis-je pas clairement manifesté à la maison de ton père, quand ils étaient en Egypte, dans la maison de Pharaon ?
\VS{28}Je l'ai choisie parmi toutes les tribus d'Israël pour être mon sacrificateur, afin d'offrir sur mon autel, et faire brûler les parfums, et porter l'éphod devant moi, et j'ai donné à la maison de ton père toutes les offrandes des enfants d'Israël consumées par le feu.
\VS{29}Pourquoi avez-vous foulé aux pieds mes sacrifices et mes offrandes que j'ai ordonné de faire dans ma demeure ? Et pourquoi as-tu honoré tes fils plus que moi, afin de vous engraisser du meilleur de toutes les offrandes d'Israël mon peuple ?
\VS{30}C'est pourquoi voici ce que dit Yahweh, le Dieu d'Israël : J'avais dit et promis que ta maison et la maison de ton père marcheraient devant moi éternellement. Et maintenant, dit Yahweh : Il n’en sera pas ainsi ; car j'honorerai ceux qui m'honorent, mais ceux qui me méprisent seront méprisés.
\VS{31}Voici, les jours viennent où je couperai ton bras, et le bras de la maison de ton père, de telle sorte qu'il n'y ait plus de vieillard dans ta maison.
\VS{32}Et tu verras un adversaire dans ma demeure, au temps où Dieu enverra toutes sortes de biens à Israël ; et il n'y aura plus jamais de vieillard dans ta maison.
\VS{33}Celui de tes descendants que je n'aurai pas retranché d'auprès de mon autel, subsistera pour consumer tes yeux et affliger ton âme ; et tous les enfants de ta maison mourront dans la fleur de l'âge.
\VS{34}Et ceci sera pour toi un signe, à savoir ce qui arrivera à tes deux fils, Hophni et Phinées, ils mourront tous les deux le même jour.
\VS{35}Et je m'établirai un sacrificateur fidèle\FTNT{Hé. 2:17 ; Hé 7:26-28). }, qui agira selon mon cœur, et selon mon âme ; et je lui édifierai une maison stable\FTNT{La maison stable fait premièrement allusion à Israël (Mi. 4) et ensuite à l’Eglise (Mt. 16:18). Cette prophétie sera pleinement réalisée lors du millénium (Za. 14).}, et il marchera à toujours devant mon Messie.
\VS{36}Et quiconque restera de ta maison, viendra se prosterner devant lui pour avoir une pièce d'argent et un morceau de pain et dira : Attache-moi, je te prie, à l’une des fonctions du sacerdoce pour manger un morceau de pain.
\TextTitle{[Yahweh appelle Samuel]}
\Chap{3}
\VerseOne{}Le jeune garçon Samuel servait Yahweh en présence d'Eli. La parole de Yahweh était rare en ce temps-là, et les visions n’étaient pas fréquentes.
\VS{2}Il arriva en ce temps qu'Eli était couché à sa place, ses yeux commençaient à se ternir et il ne pouvait plus voir.
\VS{3}Et avant que les lampes\FTNT{Le chandelier d'or à sept branches du tabernacle et du temple de Jérusalem a été décrit avec une extrême minutie dans plusieurs passages de la Bible. Il a été réalisé selon le modèle imposé par Dieu à Moïse au Sinaï (Ex. 25:31-40 ; Ex. 37:17-24 ; No. 8:4).} de Dieu soient éteintes, Samuel était aussi couché dans le temple de Yahweh, où était l'arche de Dieu.
\VS{4}Yahweh appela Samuel. Et il répondit : Me voici !
\VS{5}Et il courut vers Eli, et lui dit : Me voici, car tu m'as appelé ; mais Eli dit : Je ne t'ai pas appelé, retourne te coucher. Et il s'en alla, et se coucha.
\VS{6}Yahweh appela encore Samuel. Et Samuel se leva, et s'en alla vers Eli, et lui dit : Me voici, car tu m'as appelé ! Et Eli dit : Mon fils, je ne t'ai pas appelé, retourne, et couche-toi.
\VS{7}Or Samuel ne connaissait pas encore Yahweh, et la parole de Yahweh ne lui avait pas encore été révélée.
\VS{8}Et Yahweh appela encore Samuel pour la troisième fois ; et Samuel se leva, et s'en alla vers Eli, et dit : Me voici, car tu m'as appelé. Eli reconnut que Yahweh appelait ce jeune garçon.
\VS{9}Alors Eli dit à Samuel : Va et couche-toi ; et si on t'appelle, tu diras : Parle Yahweh , car ton serviteur écoute. Samuel donc s'en alla, et se coucha à sa place.
\VS{10}Yahweh donc vint, et se tint là ; et appela comme les autres fois : Samuel, Samuel ! Et Samuel dit : Parle, car ton serviteur écoute.
\TextTitle{[Autre avertissement de Yahweh à Eli par Samuel]}
\VS{11}Alors Yahweh dit à Samuel : Voici, je vais faire une chose en Israël, qui étourdira les oreilles de quiconque l’entendra.
\VS{12}En ce jour-là, j’accomplirai sur Eli tout ce que j’ai déclaré contre sa maison ; je commencerai, et j’achèverai.
\VS{13}Car je l'ai averti que je vais punir sa maison à perpétuité, à cause de l'iniquité dont il a connaissance, par laquelle ses fils se sont rendus infâmes, sans qu’ils les ait réprimés.
\VS{14}C'est pourquoi j'ai juré contre la maison d'Eli que jamais l’iniquité de la la maison d'Eli, ne sera expiée ni par des sacrifices ni par des offrandes.
\VS{15}Et Samuel resta couché jusqu'au matin, puis il ouvrit les portes de la maison de Yahweh. Or Samuel craignait de rapporter cette vision à Eli.
\VS{16}Mais Eli appela Samuel, et lui dit : Samuel mon fils ! Il répondit : Me voici !
\VS{17}Et Eli dit : Quelle est la parole qui t'a été adressée ? Je te prie ne me la cache pas. Que Dieu te traite avec rigueur, si tu me caches un seul mot de tout ce qui t'a été dit.
\VS{18}Samuel lui déclara donc toutes ces paroles, et ne lui en cacha rien. Et Eli répondit : C'est Yahweh, qu'il fasse ce qui lui semblera bon !
\TextTitle{[Samuel, prophète de Yahweh]}
\VS{19}Samuel grandissait. Et Yahweh était avec lui, il ne laissa pas tomber à terre une seule de ses paroles.
\VS{20}Tout Israël, depuis Dan jusqu'à Beer-Schéba, reconnut que Samuel était établi prophète de Yahweh.
\VS{21}Yahweh continuait de se manifester dans Silo ; car Yahweh se manifestait à Samuel dans Silo par la parole de Yahweh.
\TextTitle{[Les philistins prennent l'arche de Yahweh, jugement sur la maison d'Eli]}
\Chap{4}
\VerseOne{}La parole de Samuel s’adressait à tout Israël. Car Israël sortit en bataille pour aller à la rencontre des Philistins. Ils campèrent près d'Eben-Ezer, et les Philistins campaient à Aphek.
\VS{2}Les Philistins se rangèrent en bataille contre d'Israël, et le combat s’engagea, Israël fut battu par les Philistins, qui en tuèrent environ quatre mille hommes sur le champ de bataille.
\VS{3}Quand le peuple rentra au camp, les anciens d'Israël dirent : Pourquoi Yahweh nous a-t-il laissé battre aujourd'hui par les Philistins ? Ramenons de Silo l'arche de l'alliance de Yahweh, et qu'elle vienne au milieu de nous, et nous délivre de la main de nos ennemis.
\VS{4}Le peuple envoya donc à Silo, d’où l’on apporta l'arche de l'alliance de Yahweh des armées, qui habite entre les chérubins. Les deux fils d’Eli, Hophni et Phinées étaient là, avec l'arche de l'alliance de Dieu.
\VS{5}Et comme l'arche de Yahweh entrait dans le camp, tout Israël poussa de grands cris de joie et la terre en fut ébranlée.
\VS{6}Les Philistins entendirent le bruit de ces cris de joie, et ils dirent : Que veut dire ce bruit, et que signifient ces grands cris de joie dans le camp de ces Hébreux ? Et ils apprirent que l'arche de Yahweh était arrivée dans le camp.
\VS{7}Les Philistins eurent peur, car ils disaient : Dieu est entré dans le camp. Et ils dirent : Malheur à nous ! Car il n’en a pas été ainsi auparavant.
\VS{8}Malheur à nous ! Qui nous délivrera de la main de ces dieux puissants\FTNT{Le terme hébreu « elohim », généralement traduit par « dieu » ou « dieux », signifie également « dirigeants », « juges » ou encore « anges ».  Dans les textes bibliques, « Elohim » est employé pour désigner Moïse, qui a été fait « dieu » (« Elohim ») pour Pharaon (Ex. 7:1), ainsi que pour  les dieux païens  Baal, Kemosh et Dagaon (Jg. 6:31 ; Jg. 11:24 ; 1 S. 5:7)  Les Philistins avaient une vision polythéiste de la divinité et n’avaient pas la révélation du Dieu des hébreux qui est Un (Dt. 6:4).} ? C’est le Dieu qui a frappé les Egyptiens de toutes sortes de plaies dans le désert.
\VS{9}Philistins prenez courage, et agissez en hommes, de peur que vous ne soyez esclaves des Hébreux, comme ils vous ont été asservis ; agissez en hommes, et combattez !
\VS{10}Les Philistins donc combattirent, et Israël fut battu. Et chacun s’enfuit dans sa tente. La défaite fut très grande, trente mille hommes de pied d'Israël périrent .
\VS{11}L'arche de Dieu fut prise, et les deux fils d'Eli, Hophni et Phinées moururent.
\VS{12}Un homme de Benjamin s'enfuit de la bataille, et arriva à Silo ce même jour, ayant ses vêtements déchirés et la tête recouverte de terre.
\VS{13}Au moment où il arriva, Eli était dans l’attente, assis sur un siège au bord du chemin ; car son cœur tremblait à cause de l'arche de Dieu. Cet homme entra donc dans la ville, et donna les nouvelles , et toute la ville se mit à crier.
\VS{14}Eli, entendant les cris, dit : Que veut dire ce grand tumulte ? Et aussitôt cet homme vint à Eli, et lui raconta tout.
\VS{15}Or Eli était âgé de quatre-vingt-dix-huit ans, ses yeux étaient fixes, il ne pouvait plus voir.
\VS{16}L’homme dit à Eli : Je viens de la bataille, car je me suis enfui aujourd'hui de la bataille. Et Eli dit : Qu'est-il arrivé, mon fils ?
\VS{17}Celui qui apportait les nouvelles répondit : Israël a fui devant les Philistins, e il y a eu une grande défaite du peuple ; tes deux fils, Hophni et Phinées sont morts et l'arche de Dieu a été prise.
\VS{18}Et dès qu'il eut fait mention de l'arche de Dieu, Eli tomba à la renverse, de dessus son siège, à côté de la porte, se rompit le cou et mourut ; car cet homme était vieux et pesant. Il avait été juge en Israël pendant quarante ans.
\VS{19}Sa belle-fille, femme de Phinées, qui était enceinte, et sur le point d'accoucher. Lorsqu’elle apprit la nouvelle de la prise de l'arche de Dieu, de la mort de son beau-père et de son mari, elle se coucha et enfanta, car les douleurs la surprirent.
\VS{20}Comme elle mourait, celles qui l'assistaient lui dirent : Ne crains pas, car tu as enfanté un fils ; mais elle ne répondit rien, et n'en tint pas compte.
\VS{21}Mais elle appela l'enfant I-Kabod, en disant : La gloire s’en est allée d'Israël parce que l'arche de Yahweh était prise à cause de son beau-père et de son mari.
\VS{22}Elle dit donc : La gloire s’en allée d'Israël, car l'arche de Dieu est prise !
\TextTitle{[Jugements de Yahweh sur les philistins]}
\Chap{5}
\VerseOne{}Les Philistins prirent l'arche de Dieu, et l'emmenèrent d'Eben-Ezer à Asdod.
\VS{2}Les Philistins donc prirent l'arche de Dieu, et l'emmenèrent dans la maison de Dagon\FTNT{L'étymologie du nom Dagon  avait justifié la représentation qu’on faisait de ce dieu : une sorte de sirène mâle ou un homme avec queue de poisson. En effet, «  dâg », en hébreu signifie « poisson ». Il était le dieu des semences et de l'agriculture chez les peuples d’origine sémites, mais également l’un des principaux dieux des Philistins.}, et la posèrent auprès de Dagon.
\VS{3}Le lendemain les Asdodiens s'étant levés de bon matin, trouvèrent Dagon le visage contre terre, devant l'arche de Yahweh ; mais ils le prirent et le remirent à sa place.
\VS{4}Ils se levèrent encore le lendemain de bon matin, et voici, Dagon était tombé le visage contre terre, devant l'arche de Yahweh ; la tête de Dagon et les deux paumes de ses mains découpées étaient sur le seuil, et il ne lui restait que le tronc.
\VS{5}C'est pour cela que les sacrificateurs de Dagon, et tous ceux qui entrent dans la maison de Dagon, à Asdod, ne marchent pas sur le seuil jusqu'à aujourd'hui.
\VS{6}Puis la main de Yahweh s'appesantit sur le Asdodiens et les dévasta ; et il les frappa d’hémorroïdes à Asdod et dans tout son territoire.
\VS{7}Ceux donc d'Asdod, voyant qu'il en allait ainsi, dirent : L'arche du Dieu d'Israël ne demeurera pas chez nous ; car sa main s’est appesantie sur nous, et sur Dagon, notre dieu.
\VS{8}Et ils firent appeler et assemblèrent auprès d’eux tous les princes des Philistins, et dirent : Que ferons-nous de l'arche du Dieu d'Israël ? Et ils répondirent : Qu'on transporte à Gath l'arche du Dieu d'Israël. Ainsi on transporta l'arche du Dieu d'Israël.
\VS{9}Mais il arriva après qu'on l'eut transportée, la main de Yahweh fut sur la ville et il y eut une très grande terreur ; et il frappa les gens de la ville depuis le plus petit jusqu'au plus grand, par une éruption d’hémorroïdes.
\VS{10}Ils envoyèrent donc l'arche de Dieu à Ekron. Or comme l'arche de Dieu entrait à Ekron, ceux d’Ekron s'écrièrent, en disant : Ils ont transporté vers nous l'arche du Dieu d'Israël, pour nous faire mourir, nous et notre peuple !
\VS{11}C'est pourquoi ils firent appeler, et assemblèrent tous les princes des Philistins, en disant : Renvoyez l'arche du Dieu d'Israël, et qu'elle retourne en son lieu, afin qu'elle ne nous fasse pas mourir, nous et notre peuple. Car il y régnait une terreur mortelle dans toute la ville, et la main de Dieu s’y appesantissait fortement.
\VS{12}Les hommes qui n'en mouraient pas étaient frappés d’hémorroïdes, de sorte que le cri de la ville montait jusqu'au ciel.
\TextTitle{[L'arche de Yahweh revient en Israël]}
\Chap{6}
\VerseOne{}L'arche de Yahweh ayant été pendant sept mois dans le pays des Philistins.
\VS{2}Les Philistins appelèrent les sacrificateurs et les devins, et leur dirent : Que ferons-nous de l'arche de Yahweh ? Dites-nous comment nous devons la renvoyer en son lieu.
\VS{3}Ils répondirent : Si vous renvoyez l'arche du Dieu d'Israël, ne la renvoyez pas à vide, et n’oubliez pas de lui payer une offrande de culpabilité ; alors vous serez guéris, et vous saurez pourquoi sa main ne s’est pas retirée de dessus vous.
\VS{4}Et ils dirent : Quelle offrande lui payerons-nous pour le péché ? Et ils répondirent : Selon le nombre des princes des Philistins, vous donnerez cinq hémorroïdes d'or, et cinq souris d'or ; car une même plaie a été sur vous tous, et sur vos princes.
\VS{5}Vous ferez donc des figures de vos hémorroïdes, et des figures des souris qui ravagent le pays, et vous donnerez gloire au Dieu d'Israël. Peut-être retirera-t-il sa main de dessus vous, et de dessus vos dieux, et de dessus votre pays.
\VS{6}Et pourquoi endurciriez-vous votre cœur, comme l'Egypte et Pharaon ont endurci leur cœur ? Après qu'il eut fait de merveilleux exploits parmi eux, ne les laissèrent-ils pas partir et s’en aller ?
\VS{7}Maintenant, donc prenez de quoi faire un char tout neuf, et deux jeunes vaches qui allaitent leurs veaux et qui n’aient point porté le joug ; et attelez au char les deux jeunes vaches, et ramenez leurs petits à la maison.
\VS{8}Vous prendrez l'arche de Yahweh et vous la mettrez sur le char ; et vous déposerez dans un coffre, à côté de l’arche, les objets d'or que vous donnez à Yahweh en offrande pour le péché ; vous la renverrez, et elle s'en ira.
\VS{9}Et vous observerez ; si l'arche monte vers Beth-Schémesch, par le chemin de sa frontière, c'est Yahweh qui nous a fait tout ce grand mal ; si elle n'y va pas, nous saurons alors que sa main ne nous a pas touchés, mais que ceci nous est arrivé par hasard.
\VS{10}Ces gens firent ainsi. Ils prirent donc deux jeunes vaches qui allaitaient, ils les attelèrent au char, et ils enfermèrent leurs petits dans l'étable.
\VS{11}Ils mirent sur le char l'arche de Yahweh, et le coffre avec les souris d'or, et les figures de leurs hémorroïdes.
\VS{12}Alors les jeunes vaches prirent tout droit le chemin de Beth-Schémesch, elles suivirent toujours le même chemin en marchant et en mugissant ; et elles ne se détournèrent ni à droite ni à gauche. Les princes des Philistins allèrent après elles jusqu'à la frontière de Beth-Schémesch.
\VS{13}Or ceux de Beth-Schémesch, moissonnaient les blés dans la vallée ; et ayant élevé leurs yeux, ils virent l'arche, et se réjouirent en la voyant.
\VS{14}Le char arriva dans le champ de Josué de Beth-Schémesch, et s'arrêta là. Or il y avait là une grande pierre, et on fendit le bois du char, et on offrit les jeunes vaches en holocauste à Yahweh.
\VS{15}Les Lévites descendirent l'arche de Yahweh, et le coffre dans lequel étaient les objets d'or, et ils les mirent sur cette grande pierre. En ce même jour, ceux de Beth-Schémesch offrirent des holocaustes et des sacrifices à Yahweh.
\VS{16}Les cinq princes des Philistins, après avoir vu cela, retournèrent le même jour à Ekron.
\VS{17}Voici les hémorroïdes d'or que les Philistins donnèrent à Yahweh en offrande pour le péché ; un pour Asdod, un pour Gaza, un pour Askalon, un pour Gath, un pour Ekron.
\VS{18}Les souris d’or, selon le nombre de toutes les villes des Philistins, appartenant aux cinq princes, tant des villes fortifiées, que des villages sans murailles. Et ils les amenèrent jusqu'à la grande pierre sur laquelle on posa l'arche de Yahweh, et qui jusqu'à ce jour est dans le champ de Josué de Beth-Schémesch.
\VS{19}Yahweh frappa des gens de Beth-Schémesch parce qu'ils avaient regardé dans l'arche de Yahweh ; il frappa (cinquante mille) et soixante-dix hommes\FTNT{Ce nombre est généralement considéré comme une erreur des copistes.} et le peuple mena le deuil parce que Yahweh l'avait frappé d'une grande plaie.
\VS{20}Alors ceux de Beth-Schémesch dirent : Qui pourrait subsister en présence de Yahweh, ce Dieu Saint ? Et vers qui montera-t-il en s'éloignant de nous ?
\VS{21}Et ils envoyèrent des messagers aux habitants de Kirjath-Jearim, en disant : Les Philistins ont ramené l'arche de Yahweh ; descendez, et faites-la monter vers vous.
\TextTitle{[Un réveil après l'apostasie]}
\Chap{7}
\VerseOne{}Ceux donc de Kirjath-Jearim vinrent et firent monter l'arche de Yahweh, et la mirent dans la maison d'Abinadab sur la colline ; et ils consacrèrent Eléazar, son fils, pour garder l'arche de Yahweh.
\VS{2}Il s’écoula un long moment, depuis le jour où l'arche de Yahweh fut déposée à Kirjath-Jearim. Vingt années s’étaient écoulées. Toute la maison d'Israël soupira après Yahweh.
\VS{3}Et Samuel parla à toute la maison d'Israël, en disant : Si vous revenez à Yahweh de tout votre cœur, ôtez du milieu de vous les dieux étrangers, et les Astartés, dirigez votre cœur vers Yahweh, et servez-le lui seul ; et il vous délivrera de la main des Philistins.
\VS{4}Alors les enfants d'Israël ôtèrent les Baals, et les Astartés, et ils servirent Yahweh seul\FTNT{Jg. 2:13.}.
\VS{5}Samuel dit : Assemblez tout Israël à Mitspa, et je prierai Yahweh pour vous.
\VS{6}Ils s'assemblèrent donc à Mitspa ; ils puisèrent de l'eau qu'ils répandirent devant Yahweh et ils jeûnèrent ce jour-là, en disant : Nous avons péché contre Yahweh ! Et Samuel jugea les enfants d'Israël à Mitspa.
\VS{7}Or quand les Philistins eurent appris que les enfants d'Israël étaient assemblés à Mitspa, les princes des Philistins montèrent contre Israël. Les enfants d'Israël l’apprirent et ils eurent peur des Philistins.
\VS{8}Les enfants d'Israël dirent à Samuel : Ne cesse pas de crier pour nous à Yahweh, notre Dieu, afin qu'il nous délivre de la main des Philistins.
\TextTitle{[Victoire d'Israël contre les Philistins]}
\VS{9}Alors Samuel prit un agneau de lait, et l'offrit tout entier à Yahweh en holocauste. Et Samuel cria à Yahweh pour Israël, et Yahweh l'exauça.
\VS{10}Comme Samuel offrait l'holocauste, les Philistins s'approchèrent pour combattre contre Israël, mais Yahweh fit gronder, en ce jour-là, un grand tonnerre sur les Philistins, et les mit en déroute, et ils furent battus devant Israël.
\VS{11}Les hommes d'Israël sortirent de Mitspa, et poursuivirent les Philistins, et les frappèrent jusqu'au-dessous de Beth-Car.
\VS{12}Alors Samuel prit une pierre, et la mit entre Mitspa et Schen, et il appela ce lieu Eben-Ezer, en disant : Yahweh nous a secourus jusqu'en ce lieu-ci.
\VS{13}Les Philistins furent humiliés, et ils ne vinrent plus sur le territoire d'Israël. La main de Yahweh fut contre les Philistins durant la vie de Samuel.
\VS{14}Les villes que les Philistins avaient prises sur Israël, retournèrent à Israël, depuis Ekron jusqu'à Gath, avec leurs territoires. Israël les délivra donc de la main des Philistins, et il y eut paix entre Israël et les Amoréens.
\VS{15}Samuel fut juge en Israël tous les jours de sa vie.
\VS{16}Il allait tous les ans faire le tour de Béthel, de Guilgal et de Mitspa, et il jugeait Israël dans tous ces lieux.
\VS{17}Puis il revenait à Rama, où était sa maison ; et là il jugeait Israël, et il y bâtit un autel à Yahweh.
\TextTitle{[Israël veut un roi]}
\Chap{8}
\VerseOne{}Lorsque Samuel devint vieux, il établit ses fils juges sur Israël.
\VS{2}Son fils premier-né s’appelait Joël, et le second Abija ; ils jugeaient à Beer-Schéba.
\VS{3}Mais ses fils ne marchèrent pas dans ses voies, ils s’en détournèrent pour les profits acquis par la violence ; ils recevaient des présents et violaient la justice.
\VS{4}C'est pourquoi tous les anciens d'Israël s'assemblèrent, et vinrent auprès de Samuel à Rama.
\VS{5}Ils lui dirent : Voici, tu es devenu vieux, et tes fils ne suivent pas tes voies ; maintenant, établis sur nous un roi pour nous juger comme il y en a chez toutes les nations.
\TextTitle{[Prière de Samuel et réponse de Yahweh]}
\VS{6}Samuel fut affligé de ce qu'ils lui avaient dit : Etablis sur nous un roi pour nous juger. Et Samuel pria Yahweh.
\VS{7}Yahweh dit à Samuel : Obéis à la voix du peuple dans tout ce qu'il te dira, car ce n'est pas toi qu'ils ont rejeté, mais c'est moi qu'ils ont rejeté, afin que je ne règne plus sur eux.
\VS{8}Ils agissent à ton égard comme ils ont agi depuis le jour où je les ai fait monter hors d'Egypte jusqu’à ce jour ; ils m’ont abandonné, pour servir d'autres dieux.
\VS{9}Maintenant donc, obéis à leur voix ; mais ne manque pas de les avertir, en leur déclarant comment le roi qui régnera sur eux, les traitera.
\TextTitle{[Avertissements aux enfants d'Israël qui demandent un roi]}
\VS{10}Ainsi Samuel dit toutes les paroles de Yahweh, au peuple qui lui avait demandé un roi.
\VS{11}Il leur dit donc : Voici comment vous traitera le roi qui régnera sur vous. Il prendra vos fils et les mettra sur ses chars et parmi ses cavaliers, afin qu’ils courent devant son char ;
\VS{12}il en établira des chefs de mille, et des chefs de cinquante, pour labourer ses terres, pour récolter ses moissons, et pour fabriquer ses armes de guerre et l’équipement de ses chars.
\VS{13}Il prendra aussi vos filles pour en faire des parfumeuses, des cuisinières, et des boulangères.
\VS{14}Il prendra ce qu’il y a de meilleur parmi vos champs, vos vignes et vos oliviers, et il les donnera à ses serviteurs.
\VS{15}Il prélèvera la dîme de ce que vous aurez semé et de ce que vous aurez vendangé, et il la donnera à ses eunuques, et à ses serviteurs.
\VS{16}Il prendra vos serviteurs et vos servantes, l'élite de vos jeunes gens, vos ânes, et les emploiera à ses ouvrages.
\VS{17}Il prélèvera la dîme de vos troupeaux, et vous serez ses esclaves.
\VS{18}En ce jour-là, vous crierez à cause du roi que vous vous serez choisi, mais Yahweh ne vous exaucera pas.
\VS{19}Mais le peuple refusa d’écouter la voix de Samuel, et ils dirent : Non ! Mais il y aura un roi sur nous.
\VS{20}Nous serons aussi comme toutes les nations ; et notre roi nous jugera, il sortira devant nous, et il conduira nos guerres.
\VS{21}Samuel entendit donc toutes les paroles du peuple, et les rapporta à Yahweh.
\VS{22}Et Yahweh dit à Samuel : Obéis à leur voix, et établis un roi sur eux. Et Samuel dit aux hommes d'Israël : Allez-vous-en chacun dans sa ville.
\TextTitle{[Dieu leur donne un roi : Saül]}
\Chap{9}
\VerseOne{}Il y avait un homme de Benjamin, nommé Kis, fort et vaillant, fils d Abiel, fils de Tseror, fils de Becorath, fils d'Aphiach, fils d'un Benjamite.
\VS{2}Il avait un fils nommé Saül, jeune et beau, et aucun des enfants d'Israël n’était plus beau que lui, des épaules en haut, il dépassait tout le peuple.
\VS{3}Les ânesses de Kis, père de Saül, s’égarèrent; et Kis dit à Saül, son fils : Prends maintenant avec toi un des serviteurs et lève-toi, et va chercher les ânesses.
\VS{4}Il passa donc par la montagne d'Ephraïm et traversa le pays de Schalischa ; mais ils ne les trouvèrent pas ; puis ils passèrent par le pays de Schaalim, mais elles n'y étaient pas ; ils passèrent ensuite par le pays de Benjamin, mais ils ne les trouvèrent pas.
\VS{5}Quand ils furent arrivés dans le pays de Tsuph, Saül dit à son serviteur qui était avec lui : Viens, et retournons, de peur que mon père oublie les ânesses, et s’inquiète pour nous.
\VS{6}Le serviteur lui dit : Voici, je te prie, il y a dans cette ville un homme de Dieu, qui est un homme très honoré ; tout ce qu'il déclare ne manque pas d’arriver ; allons y maintenant, peut-être nous renseignera-t-il sur le chemin que nous devons prendre.
\VS{7}Et Saül dit à son serviteur : Mais si nous y allons, que porterons-nous à l'homme de Dieu, nous n’avons plus de provisions, et nous n'avons aucun présent pour l'homme de Dieu ? Qu’est-ce que nous avons ?
\VS{8}Le serviteur reprit la parole et dit à Saül : Voici j'ai encore entre mes mains le quart d'un sicle d'argent, et je le donnerai à l'homme de Dieu, et il nous indiquera notre chemin.
\VS{9}Autrefois en Israël quand on allait consulter Dieu, on se disait l'un à l'autre : Venez, allons vers le voyant ! Car le prophète, s'appelait autrefois le voyant.
\VS{10}Saül dit à son serviteur : Tu as bien dit ; viens, allons ! Et ils s'en allèrent dans la ville où était l'homme de Dieu.
\VS{11}Et comme ils montaient à la ville, ils trouvèrent de jeunes filles qui sortaient pour puiser de l'eau, et ils leur dirent : Le voyant n'est-il pas ici ?
\VS{12}Elles leur répondirent, et dirent : Il y est, le voilà devant toi ; hâte-toi maintenant, car il est venu aujourd'hui à la ville, parce qu'il y a aujourd'hui un sacrifice pour le peuple sur le haut lieu.
\VS{13}Quand vous entrerez dans la ville, vous le trouverez avant qu'il monte au haut lieu pour manger ; car le peuple ne mangera pas jusqu'à ce qu'il soit venu, parce qu'il doit bénir le sacrifice ; après quoi, les conviés mangeront. Montez donc maintenant, car vous le trouverez aujourd'hui.
\VS{14}Ils montèrent donc à la ville. Comme ils entraient dans la ville, Samuel, qui sortait pour monter au haut lieu, les rencontra.
\VS{15}Or, un jour avant l’arrivée de Saül, Yahweh avait fait une révélation à Samuel, en disant :
\VS{16}Demain, à cette même heure, je t'enverrai un homme du pays de Benjamin, et tu l'oindras pour être le conducteur de mon peuple d'Israël. Il délivrera mon peuple de la main des Philistins ; car j'ai regardé mon peuple parce que son cri est venu jusqu'à moi.
\VS{17}Et dès que Samuel eut aperçu Saül, Yahweh lui dit : Voici l'homme dont je t'ai parlé ; c'est lui qui dominera sur mon peuple.
\VS{18}Et Saül s'approcha de Samuel au milieu de la porte, et dit : Indique-moi je te prie, où est la maison du voyant.
\VS{19}Et Samuel répondit à Saül, et dit : Je suis le voyant. Monte devant moi au haut lieu, et vous mangerez aujourd'hui avec moi. Je te laisserai partir demain, et je te dirai tout ce que tu as sur le cœur.
\VS{20}Mais quant aux ânesses que tu as perdues il y a trois jours, ne t'en inquiète pas, parce qu'elles ont été retrouvées. Et vers qui tend tout le désir d’Israël ? N’est-ce pas vers toi, et vers toute la maison de ton père ?
\VS{21}Saül répondit : Ne suis-je pas de Benjamin, l’une des moindres tribus d'Israël, et ma famille n'est-elle pas la plus petite de toutes tribus de Benjamin ? Pourquoi m’as-tu tenu de tels discours ?
\VS{22}Samuel prit Saül et son serviteur, et les fit entrer dans la salle, et les plaça à la tête des conviés, qui étaient environ trente hommes.
\VS{23}Et Samuel dit au cuisinier : Apporte la portion que je t'ai donnée, en te disant : Mets-la à part.
\VS{24}Le cuisinier prit l’épaule, et ce qui l’entoure, et il la servit à Saül. Et Samuel dit : Voici ce qui a été réservé, mets-le devant toi, et mange, car il t’a été gardé expressément pour cette heure, lorsque j'ai résolu de convier le peuple ; et Saül mangea avec Samuel ce jour-là.
\VS{25}Puis ils descendirent du haut lieu dans la ville, et Samuel parla avec Saül sur le toit.
\VS{26}Puis ils se levèrent de bon matin ; et, dès l’aurore, Samuel appela Saül sur le toit, et lui dit : Lève-toi, et je te laisserai aller. Saül donc se leva, et ils sortirent tous deux dehors, lui et Samuel.
\VS{27}Et comme ils descendaient à l’extrémité de la ville, Samuel dit à Saül : Dis au serviteur de passer devant nous, et le serviteur passa devant. Arrête-toi maintenant, afin que je te fasse entendre la parole de Dieu.
\TextTitle{[Samuel oint Saül comme roi]}
\Chap{10}
\VerseOne{}Or, Samuel prit une fiole d'huile, qu’il répandit sur la tête de Saül. Il l’embrassa, et lui dit : Yahweh ne t'a-t-il pas oint, pour être le conducteur de son héritage?
\VS{2}Aujourd’hui, après m’avoir quitté, tu trouveras deux hommes près du sépulcre de Rachel, sur la frontière de Benjamin à Tseltsach, qui te diront : Les ânesses que tu étais allé chercher sont retrouvées ; et voici, ton père ne pense plus aux ânesses, mais il s’inquiète pour vous, disant : Que dois-je faire à propos de mon fils ?
\VS{3}En allant plus loin, tu arriveras au chêne de Thabor, où tu seras rencontré par trois hommes qui montent vers Dieu, à Béthel, et l'un porte trois chevreaux, l'autre trois pains, et l'autre une outre de vin.
\VS{4}Ils te demanderont comment tu te portes, et ils te donneront deux pains, que tu recevras de leurs mains.
\VS{5}Après cela tu arriveras à Guibea-Elohim, où se trouve une garnison des Philistins. Et il arrivera qu’en entrant dans la ville, tu rencontreras une troupe de prophètes descendant du haut lieu, précédés du luth, du tambourin, de la flûte, et de la harpe, et qui prophétisent.
\VS{6}Alors l'Esprit de Yahweh te saisira, et tu prophétiseras avec eux, et tu seras changé en un autre homme.
\VS{7}Et quand ces signes te seront arrivés, fais avec force ce que tu trouveras, car Dieu est avec toi.
\VS{8}Puis tu descendras devant moi à Guilgal, et voici, je descendrai vers toi pour offrir des holocaustes, et des sacrifices d’offrande de paix, tu m'attendras là sept jours, jusqu'à ce que je vienne, et que je te déclare ce que tu devras faire.
\VS{9}Aussitôt que Saül eut tourné le dos pour se séparer de Samuel, Dieu changea son cœur, et tous ces signes s’accomplir le même jour.
\VS{10}Quand ils arrivèrent à Guibea, voici une troupe de prophètes vint à sa rencontre. L'Esprit de Dieu le saisit, et il prophétisa au milieu d'eux.
\VS{11}Tous ceux qui le connaissaient de longue date, le virent prophétiser avec les prophètes. Il se dirent l'un à l'autre : Qu'est-il arrivé au fils de Kis ? Saül est-il aussi parmi les prophètes ?
\VS{12}Un homme répondit : Et qui est leur père ? De là le proverbe : Saül est-il aussi parmi les prophètes ?
\VS{13}Lorsqu’il eut cessé de prophétiser, il se rendit au haut lieu.
\VS{14}L'oncle de Saül dit à Saül et à son serviteur : Où êtes-vous allés ? Et il répondit : Chercher les ânesses, mais ne les trouvant pas nous sommes allés vers Samuel.
\VS{15}Et l’oncle de Saül dit : Déclare-moi, je te prie, ce que vous a dit Samuel.
\VS{16}Saül répondit à son oncle : Il nous a assuré que les ânesses étaient retrouvées ; mais il ne lui déclara rien concernant la royauté dont Samuel lui avait parlé.
\VS{17}Samuel convoqua le peuple devant Yahweh, à Mitspa.
\VS{18}Et il dit aux enfants d'Israël : Ainsi parle Yahweh, le Dieu d'Israël : J'ai fait monter Israël hors d'Egypte, et je vous ai délivrés de la main des Egyptiens, et de la main de tous les royaumes qui vous opprimaient.
\VS{19}Mais aujourd'hui, vous avez rejeté votre Dieu, celui qui vous a délivrés de tous vos malheurs, et de vos afflictions, et vous avez dit : Non, établis-nous un roi. Présentez-vous donc maintenant, devant Yahweh, par tribus, et par familles.
\VS{20}Ainsi Samuel fit approcher toutes les tribus d'Israël ; et la tribu de Benjamin fut désignée.
\VS{21}Après il fit approcher la tribu de Benjamin selon ses familles ; et la famille de Matri fut désignée; puis Saül fils de Kis fut désigné, on le chercha, mais on ne le trouva pas.
\VS{22}On consulta de nouveau Yahweh : Est-il encore venu quelqu’un ici ? Yahweh répondit : Il est caché parmi les bagages.
\VS{23}Ils coururent donc le chercher, et il se présenta au milieu du peuple, et il était plus grand que tout le peuple, depuis les épaules en haut.
\VS{24}Et Samuel dit à tout le peuple : Voyez-vous celui que Yahweh a choisi, il n'y a personne dans tout le peuple qui soit semblable à lui. Et le peuple poussa des cris de joie, et dit : Vive le roi !
\VS{25}Alors Samuel fit connaître au peuple les règles de la royauté, et les écrivit dans un livre, qu’il déposa devant Yahweh. Puis Samuel renvoya le peuple, chacun dans sa maison.
\VS{26}Saül aussi s'en alla chez lui à Guibea. Il fut accompagné par des vaillants hommes dont Dieu avait touché le cœur.
\VS{27}Mais il y eut des fils de Bélial\FTNT{1 S. 2:12.} qui dirent : Comment celui-ci nous délivrerait-il ? Et ils le méprisèrent, et ne lui apportèrent pas de présent. Mais Saül fit le sourd.
\TextTitle{[Saül vainqueur des Ammonites]}
\Chap{11}
\VerseOne{}Nachasch, l’Ammonite, vint et assiégea Jabès, en Galaad. Les habitants de Jabès dirent à Nachasch : Traite alliance avec nous et nous te servirons.
\VS{2}Mais Nachasch, l’Ammonite, leur répondit : Je traiterai avec vous à la condition que je vous crève à tous l’œil droit, et que je mette cet opprobre sur tout Israël.
\VS{3}Les anciens de Jabès lui dirent : Donne-nous sept jours de trêve, et nous enverrons des messagers dans tout le territoire d'Israël, et s'il n'y a personne qui nous délivre, nous nous rendrons à toi.
\VS{4}Les messagers arrivèrent à Guibea de Saül, et dirent ces paroles devant le peuple. Tout le peuple éleva sa voix, et pleura.
\VS{5}Et voici, Saül revenait des champs derrière ses bœufs, et il dit : Qu'est-ce qu'a ce peuple pour pleurer ainsi ? Et on lui raconta ce qu'avaient dit ceux de Jabès.
\VS{6}Et l'Esprit de Dieu saisit Saül, lorsqu'il entendit ces paroles, et sa colère s’enflamma fortement.
\VS{7}Il prit une paire de bœufs, et les coupa en morceaux qu’il envoya dans tous le territoire d'Israël, par des messagers, en disant : Les boeufs de tous ceux qui ne sortiront pas pour suivre Saül et Samuel, seront traités de la même manière. Et la frayeur de Yahweh tomba sur le peuple, et ils sortirent comme un seul homme.
\VS{8}Saül en fit la revue à Bézek ; les enfants d'Israël étaient trois cents mille et ceux de Juda trente mille.
\VS{9}Puis, ils dirent aux messagers qui étaient venus : Vous parlerez ainsi à ceux de Jabès en Galaad : Vous serez délivrés demain, quand le soleil sera dans sa force. Les messagers rapportèrent donc cela à ceux de Jabès, qui s'en réjouirent ;
\VS{10}et ils dirent aux Ammonites : Demain nous nous rendrons à vous, et vous nous traiterez selon votre bon plaisir.
\VS{11}Le lendemain, Saül disposa le peuple en trois corps. Ils entrèrent dans le camp des Ammonites à la veille du matin, et ils les battirent jusqu’à la chaleur du jour. Ceux qui échappèrent furent dispersés si bien qu'il n'en resta pas deux ensemble.
\TextTitle{[Le peuple reconnait Saül comme roi]}
\VS{12}Le peuple dit à Samuel : Qui est-ce qui dit : Saül régnera-t-il sur nous ? Donnez-nous ces hommes-là, et nous les ferons mourir.
\VS{13}Saül répondit : Personne ne sera mis à mort en ce jour, car Yahweh a délivré Israël aujourd'hui .
\VS{14}Et Samuel dit au peuple : Venez, allons à Guilgal, et nous y renouvellerons la royauté.
\VS{15}Et tout le peuple se rendit à Guilgal, et là, ils établirent Saül pour roi devant Yahweh, à Guilgal. Et ils offrirent des sacrifices d’offrande de paix, devant Yahweh ; Saül et tous ceux d'Israël se réjouirent beaucoup.
\TextTitle{[Le peuple atteste l'intégrité de Samuel]}
\Chap{12}
\VerseOne{}Alors Samuel dit à tout Israël : Voici, j'ai obéi à votre voix dans tout ce que vous m'avez dit, et j'ai établi un roi sur vous.
\VS{2}Et maintenant, voici le roi qui marchera devant vous. Car moi, je suis vieux et tout blanc, et voici, mes fils aussi sont avec vous ; pour moi j'ai marché devant vous, depuis ma jeunesse jusqu’à ce jour.
\VS{3}Me voici, témoignez contre moi, devant Yahweh, et devant son oint. De qui ai-je pris le bœuf ? Et de qui ai-je pris l'âne ? Qui ai-je opprimé ? Qui ai-je traité durement ? Et de la main de qui ai-je reçu des présents, afin de fermer les yeux sur lui ? Et je vous le rendrai.
\VS{4}Et ils répondirent : Tu ne nous as pas opprimés, tu ne nous as pas traités durement et tu n'as rien reçu de la main de personne.
\VS{5}Il leur dit encore : Yahweh est témoin contre vous, et son oint aussi est témoin aujourd'hui, que vous n'avez rien trouvé entre mes mains. Et ils répondirent : Il en est témoin.
\TextTitle{[Exhortation de Samuel]}
\VS{6}Alors Samuel dit au peuple : Yahweh est celui qui a établi Moïse et Aaron, et qui a fait monter vos pères hors du pays d'Egypte.
\VS{7}Maintenant donc, présentez-vous, et je vous jugerai devant Yahweh sur tous les bienfaits que Yahweh vous a accordés, à vous et à vos pères.
\VS{8}Après que Jacob fut entré en Egypte, vos pères crièrent à Yahweh, et Yahweh envoya Moïse et Aaron qui firent sortir vos pères hors d'Egypte, et les firent habiter en ce lieu.
\VS{9}Mais ils oublièrent Yahweh, leur Dieu, et il les livra entre les mains de Sisera, chef de l'armée de Hatsor, et entre les mains des Philistins, et entre les mains du roi de Moab, qui leur firent la guerre.
\VS{10}Ils crièrent encore à Yahweh, et dirent : Nous avons péché ; car nous avons abandonné Yahweh, et nous avons servi les Baals et les Astartés. Maintenant donc, délivre-nous de la main de nos ennemis, et nous te servirons.
\VS{11}Et Yahweh envoya Jerubbaal, Bedan, Jephthé et Samuel, et il vous délivra de la main de tous vos ennemis d'alentour, et vous demeurâtes en sécurité.
\VS{12}Mais voyant que Nachasch, roi des fils d’Ammon, marchait contre vous, vous m'avez dit : Non ! Mais un roi régnera sur nous. Alors que Yahweh, votre Dieu, était votre Roi.
\VS{13}Maintenant donc, voici le roi que vous avez choisi, que vous avez demandé, et voici Yahweh l'a établi roi sur vous.
\VS{14}Si vous craignez Yahweh, si vous le servez, et obéissez à sa voix, et que vous n’êtes pas rebelles au commandement de Yahweh, alors vous et votre roi qui règne sur vous, vous serez sous la conduite de Yahweh, votre Dieu.
\VS{15}Mais si vous n'obéissez pas à la voix de Yahweh, et si vous êtes rebelles au commandement de Yahweh, la main de Yahweh sera aussi contre vous, comme elle a été contre vos pères.
\VS{16}Maintenant, préparez-vous, et voyez cette grande chose que Yahweh va opérer sous vos yeux.
\VS{17}N'est-ce pas aujourd'hui la moisson des blés ? Je crierai à Yahweh, et il enverra des tonnerres et de la pluie. Sachez alors et voyez combien vous avez mal agi aux yeux de Yahweh en demandant un roi.
\VS{18}Alors Samuel cria à Yahweh, et Yahweh envoya des tonnerres et de la pluie ce même jour. Tout le peuple eut une grande crainte de Yahweh, et de Samuel.
\VS{19}Et tout le peuple dit à Samuel : Prie Yahweh, ton Dieu, pour tes serviteurs, afin que nous ne mourions pas ; car nous avons ajouté à nos péchés, celui d'avoir demandé un roi.
\VS{20}Alors Samuel dit au peuple : Ne craignez pas ! Vous avez fait tout ce mal, néanmoins ne vous détournez pas de Yahweh, mais servez Yahweh de tout votre cœur.
\VS{21}Ne vous en détournez pas car vous iriez après des choses de néant, qui ne vous apportent ni profit ni délivrance ; puisque ce sont des choses de néant.
\VS{22}Car Yahweh n’abandonne pas son peuple, pour l'amour de son grand Nom, car Yahweh a résolu de faire de vous son peuple.
\VS{23}Et pour moi, Dieu me garde de pécher contre Yahweh, et de cesser de prier pour vous ! Je vous enseignerai le bon et le droit chemin.
\VS{24}Craignez seulement Yahweh, et servez-le en vérité, de tout votre cœur ; car vous avez vu les choses magnifiques qu'il a faites pour vous.
\VS{25}Mais si vous persévérez à faire le mal, vous serez détruits vous et votre roi.
\TextTitle{[Saül pèche contre Yahweh en offrant l'holocauste]}
\Chap{13}
\VerseOne{}Saül régna un an sur Israël et après deux années, 
\VS{2} Saül choisit trois mille hommes d'Israël, deux mille avec lui à Micmasch, et sur la montagne de Béthel, et mille étaient avec Jonathan à Guibea de Benjamin. Il renvoya le reste du peuple, chacun à sa tente.
\VS{3}Et Jonathan battit le poste des Philistins qui était à Guéba, et les Philistins en furent informés ; et Saül fit sonner le shofar dans tout le pays, en disant : Que les Hébreux écoutent !
\VS{4}Tout Israël apprit donc que Saül avait battu le poste des Philistins, et Israël se rendit odieux aux Philistins. Et le peuple fut convoqué auprès de Saül, à Guilgal.
\VS{5}Les Philistins s'assemblèrent pour combattre Israël, ayant trente mille chars et six mille cavaliers ; et le peuple était aussi nombreux que le sable au bord de la mer, tant il était en grand nombre ; ils allèrent prendre position à Micmasch, à l'orient de Beth-Aven.
\VS{6}Les hommes d'Israël furent pris d’une grande angoisse ; car ils étaient oppressés, c'est pourquoi le peuple se cacha dans les cavernes, dans les buissons, dans les rochers, dans les tours et dans des citernes.
\VS{7}Les Hébreux passèrent le Jourdain pour aller au pays de Gad, et de Galaad. Saül était encore à Guilgal, aussi tout le peuple effrayé le rejoignit.
\VS{8}Il attendit sept jours selon le terme fixé par Samuel ; mais Samuel ne venait pas à Guilgal et le peuple se dispersait.
\VS{9}Et Saül dit : Amenez-moi un holocauste et des sacrifices d’offrande de paix, et il offrit l'holocauste.
\VS{10}Comme il achevait d'offrir l'holocauste, Samuel arriva, et Saül sortit au-devant de lui pour le saluer.
\VS{11}Et Samuel lui dit : Qu'as-tu fait ? Saül répondit : Lorsque j’ai vu que le peuple se dispersait, que tu ne venais pas au jour fixé, et que les Philistins étaient assemblés à Micmasch ;
\VS{12}J'ai dit : Les Philistins descendront maintenant contre moi à Guilgal, et je n'ai pas supplié Yahweh ! Je me suis maîtrisé un temps, mais j'ai fini par offrir l'holocauste.
\VS{13}Samuel répondit à Saül : C’est en insensé que tu as agi, car tu n'as pas gardé le commandement que Yahweh, ton Dieu t'avait donné ; car Yahweh aurait maintenu à jamais ta royauté sur Israël.
\VS{14}Et maintenant ta royauté ne subsistera pas ; Yahweh s'est choisi un homme selon son cœur, et Yahweh l’a destiné à être le chef de son peuple parce que tu n'as pas respecté le commandement de Yahweh.
\VS{15}Puis Samuel se leva, et monta de Guilgal à Guibea de Benjamin. Et Saül passa en revue le peuple qui se trouvait avec lui, qui fut d'environ six cents hommes.
\VS{16}Or Saül vint s’établir avec son fils Jonathan, et le peuple qui était sous ses ordres à Guibea de Benjamin, et les Philistins étaient campés à Micmasch.
\VS{17}Les Philistins sortirent du camp en trois divisions pour ravager ; l'une de ces divisions prit le chemin d’Ophra, vers le pays de Schual ;
\VS{18}l'autre division prit le chemin de Beth-Horon ; et la troisième prit le chemin de la frontière qui regarde vers la vallée de Tseboïm, du côté du désert.
\VS{19}Or dans tout le pays d'Israël, il ne se trouvait aucun forgeron ; car les Philistins avaient dit : Empêchons les Hébreux de faire des épées ou des lances.
\VS{20}C'est pourquoi chaque homme descendait vers les Philistins, pour aiguiser son soc, son hoyau, sa hache, et sa bêche ;
\VS{21}lorsque le tranchant des bêches, des hoyaux, des tridents, et des haches était émoussé, même pour redresser un aiguillon.
\VS{22}De sorte qu’il arriva qu’au jour du combat, nul n’avait d’ épée ni de lance dans toute l’armée qui était avec Saül et Jonathan ; si ce n’est Saül lui-même et Jonathan, son fils.
\VS{23}Un poste de Philistins s’établit au passage de Micmasch.
\TextTitle{[Courage de Jonathan]}
\Chap{14}
\VerseOne{}Jonathan, fils de Saül, dit un jour au garçon qui portait ses armes : Viens et allons jusqu’au poste de garde des Philistins qui est au-delà de ce lieu-là ; mais il ne dit rien à son père.
\VS{2}Saül se tenait à l'extrémité de Guibea sous un grenadier, à Migron, entouré d'environ six cents hommes.
\VS{3}Achija, fils d'Achithub, frère d'I-Kabnod, fils de Phinées, fils d'Eli, sacrificateur de Yahweh à Silo, portait l'éphod ; et le peuple ignorait que Jonathan s'en était allé.
\VS{4}Or entre les passages par lesquels Jonathan voulait arriver au poste de garde des Philistins, il y avait une dent de rocher d’un côté, et une dent de rocher de l’autre ; l'une s’appelait Botsets et l'autre Séné.
\VS{5}L'une de ces dents était située du côté nord vis-à-vis de Micmasch ; et l'autre, du côté sud vis-à-vis de Guéba.
\VS{6}Jonathan dit au garçon qui portait ses armes : Viens, poursuivons jusqu’au poste de garde de ces incirconcis ; peut-être que Yahweh agira-t-il pour nous : car on ne saurait empêcher Yahweh de délivrer avec peu ou beaucoup de gens.
\VS{7}Et celui qui portait ses armes lui dit : Fais tout ce que tu as dans le cœur, vas-y , voici je serai avec toi où tu voudras.
\VS{8}Et Jonathan lui dit : Allons vers ces hommes, et montrerons-nous à eux.
\VS{9}S'ils nous disent : Attendez jusqu'à ce que nous venions à vous, alors nous resterons sur place, et nous ne monterons pas vers eux.
\VS{10}Mais s'ils disent : Montez vers nous, nous irons ; car Yahweh les aura livrés entre nos mains. Que cela soit pour nous un signe.
\VS{11}Ils se montrèrent donc tous deux au poste de garde des Philistins, et les Philistins dirent : Voici, les Hébreux sortent des trous où ils s'étaient cachés.
\VS{12}Et ceux du poste de garde dirent à Jonathan, et à celui qui portait ses armes : Montez vers nous, nous avons quelque chose à vous apprendre. Alors Jonathan dit à celui qui portait ses armes : Monte avec moi ; car Yahweh les a livrés entre les mains d'Israël.
\VS{13}Et Jonathan monta en s’aidant des mains et des pieds ; celui qui portait ses armes le suivit. Puis ceux du poste de garde tombèrent sous les coups de Jonathan, et celui qui portait ses armes les tuait à sa suite.
\VS{14}Dans cette première victoire, Jonathan et celui qui portait ses armes, tuèrent environ vingt hommes, dans un espace d'environ une moitié d’un arpent de terre.
\VS{15}Et il y eut un grand effroi au camp, à la campagne, et parmi tout le peuple ; le poste de garde aussi, et ceux qui avaient ravagé furent effrayés et le pays fut tellement troublé que cela fut comme une frayeur de Dieu.
\TextTitle{[Victoire d'Israël]}
\VS{16}Les sentinelles de Saül qui étaient à Guibea de Benjamin virent que la multitude se dispersait et courait éperdue.
\VS{17}Alors Saül dit au peuple qui était avec lui : Faites donc la revue et voyez qui s’en est allé du milieu de nous. Ils firent donc la revue, et voici Jonathan n'y était pas, ni celui qui portait ses armes.
\VS{18}Et Saül dit à Achija : Fais approcher l'arche de Dieu ; - car l'arche de Dieu était en ce jour-là avec les enfants d'Israël.-
\VS{19}Pendant que Saül parlait au sacrificateur, le tumulte venant du camp des Philistins augmentait de plus en plus ; et Saül dit au sacrificateur : Retire ta main !
\VS{20}Saül et tout le peuple se rassemblèrent ; et vinrent au champ de bataille, les Philistins tournaient les épées les uns contre les autres, la confusion était extrême.
\VS{21}Les Hébreux, qui étaient montés auparavant dans le camp des Philistins et qui étaient dispersés, se joignirent aux Israëlites qui étaient avec Saül et Jonathan.
\VS{22}Et tous les Israëlites qui s'étaient cachés dans la montagne d'Ephraïm, ayant appris que les Philistins s'enfuyaient, les poursuivirent aussi pour les combattre.
\VS{23}Ce jour-là, Yahweh délivra Israël, et le combat s’étendit jusqu'à Beth-Aven.
\TextTitle{[Jonathan épargné des conséquences du vœu de Saül]}
\VS{24}Les hommes d'Israël furent épuisés cette journée-là. Mais Saül avait fait jurer le peuple, en disant : Maudit soit l'homme qui prendra de la nourriture avant le soir, avant que je me sois vengé de mes ennemis ! Et le peuple n’avait pris de pain.
\VS{25}Tout le peuple arriva dans une forêt, où il y avait du miel à la surface du sol.
\VS{26}Lorsque le peuple entra dans la forêt, il vit le miel qui coulait, mais nul ne porta la main à sa bouche ; car le peuple craignait le serment.
\VS{27}Or Jonathan n'avait pas entendu son père lorsqu'il avait fait faire le serment au peuple, il étendit le bout du bâton qu'il avait à la main, le trempa dans un rayon de miel et porta sa main à sa bouche et ses yeux furent éclaircis.
\VS{28}Alors quelqu'un du peuple lui dit : Ton père a fait jurer le peuple en disant : Maudit soit l'homme qui mangera aujourd'hui quelque chose ; quoique le peuple soit très fatigué.
\VS{29}Et Jonathan dit : Mon père trouble le peuple ; voyez comment mes yeux sont éclaircis après avoir goûté un peu de ce miel ;
\VS{30}combien plus si le peuple s’était aujourd'hui restauré du butin de ses ennemis ; la défaite des Philistins n'en aurait-elle pas été plus considérable ?
\VS{31}En ce jour-là donc ils frappèrent les Philistins de Micmasch à Ajalon. Le peuple était très fatigué.
\VS{32}Puis il se jeta sur le butin, il prit des brebis, des bœufs, et des veaux, et les égorgea sur la terre ; et le peuple les mangeait avec le sang.
\VS{33}On le rapporta à Saül, en disant : Voici, le peuple pèche contre Yahweh, en mangeant, avec le sang ; et il dit : Vous avez péché, roulez-moi ici une grosse pierre.
\VS{34}Allez parmi le peuple, ajouta-t-il, et dites à chacun d’amener son bœuf et ses brebis ; vous les égorgerez ici, vous les mangerez, et vous ne pécherez plus contre Yahweh, en mangeant avec le sang. Et chacun amena cette nuit-là son bœuf à la main, et ils les égorgèrent.
\VS{35}Saül bâtit un autel à Yahweh ; ce fut le premier autel qu'il bâtit à Yahweh.
\VS{36}Puis Saül dit : Descendons et poursuivons de nuit les Philistins, afin de les piller jusqu'au matin, et n’en laissons pas un homme de reste. Ils lui répondirent : Fais tout ce qui te semble bon ; mais le sacrificateur dit : Approchons-nous d’abord de Dieu.
\VS{37}Saül consulta donc Dieu : Descendrai-je à la poursuite des Philistins ? Les livreras-tu entre les mains d'Israël ? Mais il ne lui répondit pas.
\VS{38}Et Saül dit : Approchez ici, vous tous les chefs du peuple, recherchez et voyez par qui ce péché est arrivé aujourd'hui.
\VS{39}Car Yahweh est vivant, lui qui délivre Israël, quand il s’agirait de mon fils Jonathan, il en mourrait. Mais du peuple, personne ne répondit.
\VS{40}Puis il dit à tout Israël : Mettez-vous d'un côté, et nous serons de l'autre, moi et mon fils, Jonathan. Le peuple répondit à Saül : Fais ce qui te semble bon.
\VS{41}Et Saül dit à Yahweh, le Dieu d'Israël : Fais connaître la vérité. Jonathan et Saül furent désignés; et le peuple fut écarté.
\VS{42}Et Saül dit : Jetez le sort entre moi et Jonathan, mon fils. Et Jonathan fut désigné.
\VS{43}Alors Saül dit à Jonathan : Déclare-moi ce que tu as fait. Et Jonathan lui déclara et dit : Il est vrai que j'ai goûté un peu de miel avec le bout de mon bâton que j'avais à la main ; me voici, je mourrai.
\VS{44}Et Saül dit : Que Dieu agisse à mon égard comme il le veut, si tu ne meurs pas, Jonathan.
\VS{45}Mais le peuple dit à Saül : Jonathan qui a accompli cette grande délivrance en Israël, mourrait-il ? Garde-toi bien ! Yahweh est vivant, il ne tombera pas à terre un seul des cheveux de sa tête ; car c’est avec Dieu qu’il a agi en ce jour. Le peuple délivra Jonathan de la mort.
\VS{46}Saül renonça à poursuivre les Philistins, qui regagnèrent leur pays.
\TextTitle{[Les guerres sous le règne de Saül]}
\VS{47}Après que Saül eut pris possession de la royauté sur Israël, il fit la guerre de tous côtés contre ses ennemis, Moab, les enfants d’Ammon, Edom, les rois de Tsoba et les Philistins ; partout où il se tournait, il était vainqueur.
\VS{48}Il manifesta sa puissance en frappant Amalek et délivra Israël de la main de ceux qui le pillaient.
\VS{49}Les fils de Saül étaient Jonathan, Jischvi et Malkischua ; et quant aux noms de ses deux filles, le nom de l'aînée était Mérab, et la plus jeune, Mical.
\VS{50}Et le nom de la femme de Saül était Achinoam, fille d'Achimaats ; et le nom du chef de son armée était Abner, fils de Ner, oncle de Saül.
\VS{51}Kis, père de Saül, et Ner père d'Abner étaient fils d'Abiel.
\VS{52}La guerre contre les Philistins fut violente durant toute la vie de Saül ; et chaque fois que Saül remarquait un homme fort et vaillant, il le prenait auprès de lui.
\TextTitle{[Saül désobéit à Yahweh]}
\Chap{15}
\VerseOne{}Samuel dit à Saül : Yahweh m'a envoyé pour t'oindre afin que tu sois roi sur son peuple, sur Israël ; maintenant donc, écoute les paroles de Yahweh.
\VS{2}Ainsi parle Yahweh des armées : Je me rappelle de ce qu'Amalek a fait à Israël, comment il s'opposa à lui sur le chemin, à sa sortie d'Egypte.
\VS{3}Va maintenant, et frappe Amalek, et dévouez par interdit tout ce qui lui appartient ; ne l’épargne pas, mais fais mourir hommes et femmes, enfants et nourrissons, bœufs et menu bétail, chameaux et ânes.
\VS{4}Saül donc convoqua le peuple, et en fit la revue à Thelaïm, il y avait deux cent mille hommes de pied, et de dix mille hommes de Juda.
\VS{5}Et Saül marcha jusqu'à la ville d’Amalek, et mit une embuscade dans la vallée.
\VS{6}Et Saül dit aux Kéniens : Allez retirez-vous, séparez-vous des Amalécites, de peur que je ne vous détruise avec eux ; car vous avez agi avec bonté envers tous les enfants d'Israël, quand ils montèrent d'Egypte. Et les Kéniens se séparèrent des Amalécites.
\VS{7}Et Saül frappa les Amalécites depuis Havila jusqu'à Schur, qui est face à l'Egypte.
\VS{8}Il fit passer tout le peuple au fil de l'épée, le dévouant par interdit ; mais il épargna Agag, roi d'Amalek.
\VS{9}Saül et le peuple épargnèrent Agag, les meilleures brebis, les meilleurs boeufs, les bêtes grasses, les agneaux, ce qu’il y avait de meilleur ; ils ne voulurent pas les dévouer par interdit ; détruisant seulement tout ce qui est chétif et méprisable.
\VS{10}Alors la parole de Yahweh fut adressée à Samuel en disant :
\VS{11}Je me repens d'avoir établi Saül pour roi car il s’est détourné de moi et n'a pas exécuté mes paroles. Samuel fut très irrité, et il cria à Yahweh toute la nuit.
\TextTitle{[Samuel annonce à Saül que Yahweh le rejette]}
\VS{12}Puis Samuel se leva de bon matin pour aller rencontrer Saül. On lui rapporta que Saül venu à Carmel, s'est érigé un monument, puis s'en est retourné, pour enfin descendre à Guilgal.
\VS{13}Samuel se rendit auprès de Saül, et Saül lui dit : Sois béni de Yahweh ! J’ai exécuté la parole de Yahweh.
\VS{14}Samuel dit : Quel est donc ce bêlement de brebis qui parvient à mes oreilles, et ce mugissement de bœufs que j'entends ?
\VS{15}Et Saül répondit : Ils les ont amenés de chez les Amalécites ; car le peuple a épargné les meilleures brebis et les meilleurs bœufs, pour les sacrifier à Yahweh, ton Dieu ; et nous avons détruit le reste, nous l’avons dévoué par interdit.
\VS{16}Samuel dit à Saül : Laisse-moi te déclarerai ce que Yahweh m'a dit cette nuit ; et il lui répondit : Parle !
\VS{17}Samuel dit : N'est-il pas vrai que, quand tu étais petit à tes yeux, tu as été fait chef des tribus d'Israël, et Yahweh t'a oint pour roi sur Israël ?
\VS{18}Yahweh t'avait envoyé dans cette expédition, et t'avait dit : Va, et détruis ces pécheurs, les Amalécites, et fais-leur la guerre, jusqu'à ce qu'ils soient exterminés.
\VS{19}Pourquoi n'as-tu pas obéi à la voix de Yahweh, tu t'es jeté sur le butin, et as fait ce qui déplaît à Yahweh ?
\VS{20}Et Saül répondit à Samuel : J'ai pourtant obéi à la voix de Yahweh, et je suis allé par le chemin par lequel Yahweh m'a envoyé, et j'ai amené Agag, roi des Amalécites, et j'ai dévoué les Amalécites, par interdit.
\VS{21}Mais le peuple a pris des brebis, des bœufs, du butin, comme prémices de ce qui devait être dévoué, pour le sacrifier à Yahweh, ton Dieu à Guilgal.
\VS{22}Samuel répondit : Yahweh prend-il plaisir aux holocaustes et aux sacrifices, autant qu’à l’obéissance à sa voix ? Voici, l'obéissance vaut mieux que les sacrifices, et l’observation de sa parole vaut mieux que la graisse des béliers.
\VS{23}Car la rébellion est un péché autant que la divination, et la résistance ne l’est pas moins que l’idolâtrie et les théraphim. Puisque tu as rejeté la parole de Yahweh, il te rejette aussi afin que tu ne sois plus roi.
\VS{24}Et Saül répondit à Samuel : J'ai péché parce que j'ai transgressé le commandement de Yahweh, ainsi que tes paroles ; car je craignais le peuple et j'ai obéi à sa voix.
\VS{25}Mais maintenant, je te prie, pardonne-moi mon péché, et reviens avec moi, que je me prosterne devant Yahweh.
\VS{26}Et Samuel dit à Saül : Je n’irai pas avec toi ; parce que tu as rejeté la parole de Yahweh, Yahweh te rejette afin que tu sois plus roi d’Israël.
\VS{27}Comme Samuel se détournait pour s'en aller, Saül le saisit par le pan de son manteau qui se déchira.
\VS{28}Alors Samuel lui dit : Yahweh déchire aujourd'hui le royaume d'Israël de dessus toi, et le donne à un autre, qui est meilleur que toi.
\VS{29}En effet, le Puissant d'Israël ne ment pas, il ne se repent pas ; car il n'est pas un homme pour se repentir.
\VS{30}Et Saül répondit : J'ai péché ; mais honore-moi maintenant, je te prie, en présence des anciens de mon peuple, et en présence d'Israël, et reviens avec moi, et je me prosternerai devant Yahweh ton Dieu.
\VS{31}Samuel retourna et suivit Saül ; et Saül se prosterna devant Yahweh.
\VS{32}Puis Samuel dit : Amenez-moi Agag, roi d'Amalek. Et Agag s’avança vers lui, faisant le gracieux ; car Agag disait : certainement l'amertume de la mort est passée.
\VS{33}Mais Samuel dit : Comme ton épée a privé les femmes de leurs enfants, ainsi ta mère entre les femmes sera privée d'enfants. Et Samuel mit Agag en pièces devant Yahweh à Guilgal.
\VS{34}Puis il s'en alla à Rama ; et Saül monta dans sa maison à Guibea de Saül.
\VS{35}Et Samuel n'alla plus voir Saül jusqu'au jour de sa mort ; car Samuel pleurait sur Saül, de ce que Yahweh s'était repenti d'avoir établi Saül, roi sur Israël.
\TextTitle{[Yahweh envoie Samuel à Bethléhem pour oindre David]}
\Chap{16}
\VerseOne{}Yahweh dit à Samuel : Jusqu'à quand mèneras-tu deuil sur Saül, vu que je l'ai rejeté, afin qu'il ne règne plus sur Israël ? Remplis ta corne d'huile, et viens ; je t’enverrai chez Isaï, Bethléhémite ; car je me suis pourvu d'un de ses fils pour roi.
\VS{2}Et Samuel dit : Comment irai-je ? Car Saül l’apprendra et il me tuera. Et Yahweh répondit : Tu emmèneras avec toi une jeune vache du troupeau ; et tu diras : Je suis venu pour sacrifier à Yahweh.
\VS{3}Et tu inviteras Isaï au sacrifice, et je te ferai savoir ce que tu auras à faire, et tu m'oindras celui que je te dirai.
\VS{4}Samuel fit donc comme Yahweh lui avait dit, et il alla à Bethléhem. Les anciens de la ville tout effrayés accoururent au-devant de lui et lui dirent : Ton arrivée annonce-t-elle la paix ?
\VS{5}Et il répondit : Soyez en paix ; je suis venu pour sacrifier à Yahweh, sanctifiez-vous, et venez avec moi au sacrifice. Il fit sanctifier aussi Isaï et ses fils, et les invita au sacrifice.
\VS{6}A son entrée, il remarqua Eliab, et se dit : L'oint de Yahweh est certainement devant lui.
\VS{7}Mais Yahweh dit à Samuel : Ne prête pas attention à son apparence, ni à la hauteur de sa taille, car je l'ai rejeté ; Yahweh ne considère pas ce que l'homme considère ; car l'homme considère ce que voient ses yeux ; mais Yahweh regarde au cœur.
\VS{8}Isaï appela Abinadab, et le fit passer devant Samuel, et Samuel dit : Yahweh n'a pas non plus choisi celui-ci.
\VS{9}Isaï fit passer Schamma, et Samuel dit : Yahweh n'a pas non plus choisi celui-ci.
\VS{10}Ainsi Isaï fit passer ses sept fils devant Samuel et Samuel dit à Isaï : Yahweh n’a pas choisi ceux-ci.
\VS{11}Puis Samuel dit à Isaï : Sont-ce là tous tes garçons? Et il dit : Il reste encore le plus jeune, seulement, il fait paître les brebis. Alors Samuel dit à Isaï : Envoie-le chercher ; car nous ne retournerons pas avant qu’il ne soit venu ici.
\VS{12}Il le fit donc venir. Il était roux, avec de beaux yeux et une belle apparence. Et Yahweh dit à Samuel : Lève-toi, et oins-le ; car c'est lui !
\VS{13}Alors Samuel prit la corne d'huile, et l'oignit au milieu de ses frères ; et depuis ce jour-là l'Esprit de Yahweh saisit David. Et Samuel se leva, et s'en alla à Rama.
\TextTitle{[David chez Saül]}
\VS{14}L'Esprit de Yahweh se retira de Saül, et un mauvais esprit\FTNT{Saül a été frappé d’un esprit d’égarement (2 Th. 2:9-12)} envoyé par Yahweh le terrifiait.
\VS{15}Les serviteurs de Saül lui dirent : Voici, un mauvais esprit envoyé de Dieu te tourmente.
\VS{16}Que le roi notre seigneur parle ! Tes serviteurs sont devant toi. Ils chercheront un homme qui sache jouer de la harpe ; et quand le mauvais esprit envoyé par Dieu sera sur toi, il jouera de sa main, et tu seras soulagé.
\VS{17}Saül répondit à ses serviteurs : Trouvez-moi un homme qui sache bien jouer et amenez-le-moi.
\VS{18}L'un des serviteurs répondit : Voici, j'ai vu l’un des fils d'Isaï, le Bethléhémite, qui sait jouer des instruments, il est fort et vaillant, c’est un guerrier qui parle bien, bel homme, et Yahweh est avec lui.
\VS{19}Alors Saül envoya des messagers à Isaï, pour lui dire : Envoie-moi David, ton fils, qui est avec les brebis.
\VS{20}Isaï prit un âne, qu’il chargea de pain, et une outre de vin, et un jeune chevreau, et les envoya par David, son fils, à Saül.
\VS{21}David arrivé chez Saül, se présenta devant lui ; et Saül l'aima beaucoup, et il lui servit à porter ses armes.
\VS{22}Saül fit dire à Isaï : Je te prie que David demeure à mon service ; car il a trouvé grâce devant moi.
\VS{23}Il arrivait donc que quand le mauvais esprit envoyé de Dieu, était sur Saül, David prenait la harpe, et en jouait de sa main ; et Saül en était soulagé, parce que le mauvais esprit se retirait de lui.
\TextTitle{[Goliath défie Israël]}
\Chap{17}
\VerseOne{}Les Philistins réunirent leurs armées pour faire la guerre, et ils se rassemblèrent à Soco, qui est de Juda ; et ils campèrent entre Soco et Azéka, à Ephès-Dammim.
\VS{2}Saül et ceux d'Israël se rassemblèrent aussi ; et ils campèrent dans la vallée du chêne, et ils se mirent en ordre de bataille contre les Philistins.
\VS{3}Les Philistins étaient sur une montagne d’un côté, et les Israëlites sur une montagne de l’autre côté ; de sorte que la vallée les séparait.
\VS{4}Il sortit du camp des Philistins un homme qui se présentait entre les deux armées, il s’appelait Goliath, de la ville de Gath, haut de six coudées et d'un empan.
\VS{5}Il avait un casque d'airain sur sa tête, et était armé d'une cuirasse à écailles pesant cinq mille sicles d'airain.
\VS{6}Il avait aussi des jambières d'airain, et un javelot d'airain entre ses épaules.
\VS{7}Le bois de sa lance était comme une ensouple d'un tisserand, et le fer de sa lance pesait six cents sicles de fer. Celui qui portait son bouclier marchait devant lui.
\VS{8}Il se présenta donc, et cria aux troupes d'Israël rangées en bataille, il leur disait : Pourquoi sortez-vous pour vous ranger en bataille ? Ne suis-je pas Philistin, et n'êtes-vous pas esclaves de Saül ? Choisissez l'un d'entre vous, et qu'il descende contre moi.
\VS{9}S’il peut me battre et qu'il me tue, nous serons vos esclaves ; mais si j'ai l'avantage sur lui, et que je le tue, vous serez nos esclaves, et vous nous serez asservis.
\VS{10}Le Philistin disait : Je jette un défi en ce jour aux troupes rangées d'Israël : Donnez-moi un homme, et nous combattrons ensemble.
\VS{11}Saül et tous les Israëlites ayant entendu les paroles du Philistin furent épouvantés et saisis d’une grande frayeur.
\VS{12}David, était le fils d'un homme Ephratien, de Bethléhem de Juda, nommé Isaï, qui avait huit fils, et qui du temps de Saül, était vieux et mis au rang de personnes de qualité.
\VS{13}Et les trois fils aînés d'Isaï avaient suivi Saül à la guerre. Les noms de ses trois fils qui s'en étaient allés à la guerre, étaient Eliab, le premier-né ; Abinadab, le second ; et Schamma, le troisième.
\VS{14}David était le plus jeune, et les trois plus grands suivaient Saül.
\VS{15}David allait et revenait d'auprès de Saül, pour paître les brebis de son père à Bethléhem.
\VS{16}Et le Philistin s'approchant le matin et le soir, se présenta pendant quarante jours.
\TextTitle{[David veut combattre contre Goliath]}
\VS{17}Isaï dit à David, son fils : Prends maintenant pour tes frères un épha de ce blé rôti, et ces dix pains, et porte-les promptement au camp, à tes frères.
\VS{18}Tu porteras aussi ces dix fromages au chef de leur millier, tu t’informeras du bien-être de tes frères et tu m'en apporteras des nouvelles sûres.
\VS{19}Or Saül était avec eux et les hommes d'Israël, combattant les Philistins dans la vallée du chêne.
\VS{20}David se leva de bon matin, et laissa les brebis aux soins d’un gardien ; puis ayant pris sa charge, s'en alla, comme son père Isaï le lui avait ordonné. Lorsqu’il arriva au lieu où était le camp, l'armée sortait pour se ranger en bataille, et on poussait des cris de guerre.
\VS{21}Car les Israëlites et les Philistins se rangèrent armée contre armée.
\VS{22}Alors David se déchargea de son bagage, le laissant entre les mains de celui qui gardait le bagage, et courut vers les rangs de l’armée. Aussitôt arrivé, il demanda à ses frères s'ils se portaient bien.
\VS{23}Et comme il parlait avec eux, le Philistin de Gath, nommé Goliath, sortit des rangs de l'armée des Philistins, se présenta entre les deux armées et proféra les mêmes paroles qu'il avait proférées auparavant et David les entendit.
\VS{24}A la vue de cet homme, tous ceux d'Israël s'enfuirent devant lui, saisis d’une grande frayeur.
\VS{25}Et les Israélites disaient : Avez-vous vu s’avancer cet homme ? Il est monté pour jeter un défi à Israël, mais si quelqu'un le tue, le roi le comblera de richesses, et lui donnera sa fille, et affranchira la maison de son père en Israël.
\VS{26}Alors David parla aux personnes qui étaient là avec lui, en disant : Quel bien fera-t-on à l'homme qui frappera ce Philistin, et qui ôtera l'opprobre de dessus Israël ? Car qui est ce Philistin, cet incirconcis, pour insulter l’armée du Dieu vivant ?
\VS{27}Et le peuple lui répéta ces mêmes paroles et lui dit : C'est le bien qu'on fera à l'homme qui l'aura tué.
\VS{28}Et quand Eliab son frère aîné entendit qu'il parlait à ces personnes, sa colère s'enflamma contre David, et il lui dit : Pourquoi es-tu descendu, et à qui as-tu laissé ce peu de brebis au désert ? Je connais ton orgueil et la malice de ton cœur, car tu es descendu pour voir la bataille.
\VS{29}Et David répondit : Qu'ai-je donc fait ? Ne puis-je pas parler ainsi ?
\VS{30}Puis il se détourna de lui vers un autre, et lui posa les mêmes questions ; et le peuple lui répondit comme la première fois.
\VS{31}Les paroles que David avait dites furent entendues et rapportées devant Saül qui le fit venir.
\VS{32}David dit à Saül : Que personne ne perde courage à cause de ce Philistin ! Ton serviteur ira et se battra contre lui.
\VS{33}Mais Saül dit à David : Tu ne peux aller te battre contre ce Philistin, car tu n'es qu'un enfant, et il est un homme de guerre depuis sa jeunesse.
\VS{34}David répondit à Saül : Ton serviteur faisait paître les brebis de son père, quand un lion ou un ours venait emporter une brebis du troupeau,
\VS{35}je le poursuivais, je le frappais, et j’arrachais la brebis de la gueule, s’il se jetait sur moi, je le saisissais par la mâchoire, je le frappais, et je le tuais.
\VS{36}Ton serviteur a tué et le lion, et l’ours ; et ce Philistin, cet incirconcis, sera comme l'un d'eux ; car il a déshonoré l’armée du Dieu vivant.
\VS{37}David dit encore : Yahweh qui m'a délivré de la griffe du lion, et de la patte de l'ours, me délivrera de la main de ce Philistin. Alors Saül dit à David : Va, et que Yahweh soit avec toi.
\TextTitle{[David tue Goliath]}
\VS{38}Saül fit revêtir David de ses vêtements, et lui mit son casque d'airain sur sa tête, et lui fit endosser une cuirasse.
\VS{39}Puis David ceignit l'épée par-dessus ses vêtements, et voulut marcher, car il n’avait pas encore essayé. Et David dit à Saül : Je ne saurais marcher ainsi, je ne l’ai jamais essayé. Et il s’en débarrassa.
\VS{40}Alors il prit en main son bâton, et se choisit dans le torrent cinq pierres bien polies, et les mit dans sa mallette de berger et dans sa poche, puis sa fronde en main, il s'approcha du Philistin.
\VS{41}Le Philistin aussi s'approcha lentement de David, précédé de l'homme qui portait son bouclier.
\VS{42}Le Philistin regarda, et lorsqu’il vit David, il le méprisa, car ce n'était qu'un jeune garçon, roux et beau de figure.
\VS{43}Le Philistin dit à David : Suis-je un chien, pour que tu viennes contre moi avec des bâtons ? Et le Philistin maudit David par ses dieux.
\VS{44}Le Philistin ajouta : Viens vers moi et je donnerai ta chair aux oiseaux du ciel, et aux bêtes des champs.
\VS{45}Et David dit au Philistin : Tu marches contre moi avec l'épée, la lance, et le javelot ; mais moi, je marche contre toi au Nom de Yahweh des armées, le Dieu de l’armée d'Israël, que tu as blasphémé.
\VS{46}Aujourd'hui Yahweh te livrera entre mes mains, je t’abattrai, je te couperai la tête ; aujourd'hui je donnerai les cadavres du camp des Philistins aux oiseaux du ciel, et aux animaux de la terre ; et toute la terre saura qu'Israël a un Dieu.
\VS{47}Et toute cette assemblée saura que Yahweh ne délivre pas par l'épée ni par la lance ; car la victoire est à Yahweh, qui vous livrera entre nos mains.
\VS{48}Voyant le Philistin se mettre en mouvement et s'approcher de lui, David s’élança, et courut au milieu du champ de bataille en direction du Philistin.
\VS{49}Il mit la main à sa mallette, prit une pierre, et la lança avec sa fronde ; il frappa le Philistin au front, tellement que la pierre s'enfonça dans son front, il tomba le visage contre terre.
\VS{50}Ainsi avec une fronde et une pierre, David fut plus fort que le Philistin, il le frappa, et le tua, sans avoir une épée à la main.
\VS{51}Alors David courut, se jeta sur le Philistin, prit son épée, la tira de son fourreau, le tua, et lui coupa la tête. Les Philistins, voyant que leur héros était mort, prirent la fuite.
\VS{52}Alors les hommes d'Israël et de Juda se levèrent, et poussèrent des cris de joie, et poursuivirent les Philistins, jusqu'à la vallée, et jusqu'aux portes d’Ekron. Les Philistins blessés à mort tombèrent dans le chemin de Schaaraïm, jusqu'à Gath, et jusqu'à Ekron.
\VS{53}Et les enfants d'Israël revinrent de la poursuite des Philistins, et pillèrent leurs camps.
\VS{54}David prit la tête du Philistin et la porta à Jérusalem, et il mit aussi dans sa tente les armes du Philistin.
\VS{55}Quand Saül vit David sortant à la rencontre du Philistin, il dit à Abner, chef de l'armée : Abner, de qui ce jeune homme est-il le fils ? Abner répondit : Que ton âme vive, ô roi! Je n'en sais rien.
\VS{56}Le roi lui dit : Informe-toi de qui ce jeune garçon est fils.
\VS{57}Et quand David fut de retour après avoir tué le Philistin, Abner le prit, et le mena devant Saül. David avait la tête du Philistin à la main.
\VS{58}Et Saül lui dit : Jeune garçon, de qui es-tu fils ? David répondit : Je suis fils d'Isaï Bethléhémite, ton serviteur.
\TextTitle{[Jonathan et David font alliance]}
\Chap{18}
\VerseOne{}Dès que David eut achevé de parler à Saül, l'âme de Jonathan fut attachée à l'âme de David, et Jonathan l'aima comme son âme.
\VS{2}Ce jour-là donc Saül le retint, et ne lui permit plus de retourner à la maison de son père.
\VS{3}Alors Jonathan fit alliance avec David, parce qu'il l'aimait comme son âme.
\VS{4}Jonathan se dépouilla du manteau qu'il portait, et le donna à David, avec ses habits, jusqu'à son épée, son arc, et sa ceinture.
\TextTitle{[Saül jaloux veut tuer David]}
\VS{5}David envoyé par Saül, réussissait partout où il allait, de sorte que Saül l'établit sur son armée, et il plaisait à tout le peuple, même aux serviteurs de Saül.
\VS{6}Or quand ils rentraient, lors du retour de David après qu’il eut tué le Philistin, des femmes sortirent de toutes les villes d'Israël, en chantant et dansant devant le roi Saül, avec des tambourins, des triangles et en poussant des cris de joie.
\VS{7}Les femmes chantaient, se répondant les unes aux autres, en disant : Saül a frappé ses mille, et David ses dix mille.
\VS{8}Saül fut très irrité, car cette parole lui déplut. Il dit : Elles en ont donné dix mille à David, et à moi, mille ! Il ne lui manque plus que le royaume.
\VS{9}Depuis ce jour-là, Saül regardait David d’un mauvais œil.
\VS{10}Dès le lendemain, le mauvais esprit envoyé de Dieu saisit Saül qui prophétisait dans sa maison, et David joua de sa main, comme les autres jours, et Saül avait une lance à la main.
\VS{11}Saül jeta sa lance, se disant : Je frapperai David, contre le mur ; mais David l’évita deux fois.
\VS{12}Saül craignait la présence de David, parce que Yahweh était avec David, et qu'il s'était retiré de Saül.
\VS{13}C'est pourquoi Saül éloigna David de lui, et l'établit chef de mille ; et David allait et venait devant le peuple.
\VS{14}David réussissait dans tout ce qu'il entreprenait, car Yahweh était avec lui.
\VS{15}Saül, voyant que David réussissait beaucoup, avait peur de sa présence.
\VS{16}Mais tout Israël et Juda aimaient David, parce qu'il allait et venait devant eux.
\TextTitle{[David épouse Mérab]}
\VS{17}Saül dit à David : Voici, je te donnerai Mérab, ma fille aînée pour femme ; sois pour moi un fils vaillant, et conduis les guerres de Yahweh ; car Saül disait : Que ma main ne le touche pas, mais que ce soit celle des Philistins.
\VS{18}David répondit à Saül : Qui suis-je, et quelle est ma vie, et la famille de mon père en Israël, pour que je devienne gendre du roi ?
\VS{19}Or, au temps où l’on devait donner Mérab fille de Saül à David, elle fut donnée pour femme à Adriel de Mehola.
\VS{20}Mais Mical, fille de Saül, aima David ; ce qu'on rapporta à Saül, et la chose lui plut.
\VS{21}Et Saül dit : Je la lui donnerai afin qu'elle soit pour lui un piège, et que par ce moyen la main des Philistins l’atteigne. Saül donc dit à David pour la seconde fois : Tu seras aujourd'hui mon gendre.
\VS{22}Et Saül ordonna à ses serviteurs de parler à David en secret, et de lui dire : Voici, le roi prend plaisir en toi, et tous ses serviteurs t'aiment ; sois donc maintenant gendre du roi.
\VS{23}Les serviteurs de Saül répétèrent toutes ces paroles à David, et David répondit : Pensez-vous qu’il soit facile de devenir le gendre du roi, moi qui suis un homme pauvre, et peu important ?
\VS{24}Et les serviteurs de Saül lui rapportèrent ce que David avait répondu.
\VS{25}Saül dit : Vous parlerez ainsi à David : Le roi ne désire pas de dot, mais cent prépuces de Philistins, afin d’être vengé de ses ennemis. Or Saül avait pour but de faire tomber David aux mains des Philistins.
\VS{26}Les serviteurs de Saül rapportèrent tous ces discours à David à qui il plut de devenir gendre du roi. Le temps n’était pas encore écoulé,
\VS{27}que David se leva, et s'en alla, lui et ses gens, et tua deux cents hommes parmi les Philistins ; il apporta leurs prépuces, et on les livra au complet au roi, afin qu'il devienne gendre du roi. Alors Saül lui donna pour femme Mical, sa fille.
\VS{28}Saül vit et comprit que Yahweh était avec David et Mical, fille de Saül l'aimait.
\VS{29}Saül craignait David de plus en plus, et devint son ennemi toute sa vie durant.
\VS{30}Les chefs des Philistins firent des incursions, mais chaque fois qu’ils sortaient, David remportait du succès mieux que tous les serviteurs de Saül et son nom devint célèbre.
\TextTitle{[David échappe aux assauts de Saül]}
\Chap{19}
\VerseOne{}Saül parla à Jonathan, son fils, et à tous ses serviteurs de faire mourir David.
\VS{2}Mais Jonathan, fils de Saül, avait une grande affection pour David. C'est pourquoi Jonathan le fit savoir à David, et lui dit : Saül, mon père, cherche à te faire mourir ; maintenant donc, tiens-toi sur tes gardes jusqu'au matin, demeure dans un lieu secret, et cache-toi.
\VS{3}Je me tiendrai auprès de mon père, je sortirai dans le champ où tu seras ; car je parlerai de toi à mon père ; je verrai ce qu'il en sera, et je te le rapporterai.
\VS{4}Jonathan parla favorablement de David à Saül, son père, et lui dit : Que le roi ne pèche pas contre son serviteur David, car il n'a pas péché contre toi ; au contraire, il a agi pour ton bien.
\VS{5}Car il a exposé sa vie, il a tué le Philistin, et Yahweh a opéré une grande délivrance pour tout Israël, tu l'as vu, et tu t'en es réjoui ; pourquoi donc pécherais-tu contre le sang innocent en faisant mourir David sans cause ?
\VS{6}Saül écouta la voix de Jonathan et jura : Yahweh est vivant, il ne mourra pas.
\VS{7}Alors Jonathan appela David, et lui répéta toutes ces choses. Jonathan l’introduisit auprès de Saül, et il fut à son service comme auparavant.
\VS{8}La guerre ayant recommencé, David se mit en campagne et frappa les Philistins, et leur infligea une grande défaite, de sorte qu'ils prirent la fuite.
\VS{9}Le mauvais esprit envoyé de Yahweh fut sur Saül, comme il était assis dans sa maison, ayant sa lance à la main, et David jouait de sa main.
\VS{10}Saül voulut frapper David avec sa lance contre le mur ; mais il se glissa de devant Saül, qui frappa le mur de la lance, David s'enfuit et s'échappa cette nuit-là.
\VS{11}Saül envoya des messagers à la maison de David pour le garder, et le faire mourir au matin. Mical, femme de David l’en informa, en disant : Si tu ne te sauves pas demain on te fera mourir.
\VS{12}Mical fit descendre David par une fenêtre, et ainsi il s'en alla et s'enfuit.
\VS{13}Ensuite Mical prit un théraphim, qu’elle plaça dans le lit ; elle mit une peau de chèvre à son chevet et l’enveloppa d'une couverture.
\VS{14}Lorsque Saül envoya des gens pour prendre David, elle dit : Il est malade.
\VS{15}Saül envoya encore des gens pour prendre David, en leur disant : Apportez-le-moi dans son lit, afin que je le fasse mourir.
\VS{16}Ces gens donc vinrent, et voici, un théraphim était au lit, et la peau de chèvre à son chevet.
\VS{17}Saül dit à Mical : Pourquoi m'as-tu trompé de la sorte, et as-tu laissé aller mon ennemi, de sorte qu'il s’est échappé ? Et Mical, répondit à Saül : Il m'a dit : Laisse-moi aller ou je te tue !
\VS{18}C’est ainsi que David prit la fuite et qu’il s’échappa. Il se rendit auprès de Samuel à Rama, et lui raconta tout ce que Saül lui avait fait. Puis il s'en alla avec Samuel, et ils demeurèrent à Najoth.
\VS{19}On le rapporta à Saül, en lui disant : Voici, David est à Najoth près de Rama.
\VS{20}Alors Saül envoya des gens pour s’emparer de David. Ils virent une assemblée de prophètes qui prophétisaient, et Samuel à leur tête, se tenait là. L'Esprit de Dieu saisit les envoyés de Saül, qui prophétisèrent aussi.
\VS{21}On le rapporta à Saül, qui envoya d'autres gens, et eux aussi prophétisèrent. Saül en envoya encore pour la troisième fois et ils prophétisèrent également.
\VS{22}Alors il alla lui-même à Rama. Arrivé à la grande citerne qui est à Sécou, il s'informa disant : Où sont Samuel et David ? Et on lui répondit : Ils sont à Najoth, près de Rama.
\VS{23}Il se dirigea vers Najoth près de Rama ; et l'Esprit de Dieu le saisit à son tour, et il continua son chemin en prophétisant, jusqu'à son arrivée à Najoth près de Rama.
\VS{24}Il se dépouilla lui aussi de ses vêtements et prophétisa devant Samuel ; et il se jeta à terre nu, tout ce jour-là et toute la nuit. C'est pourquoi on dit : Saül est-il aussi parmi les prophètes ?
\TextTitle{[David et Jonathan renouvellent leur serment]}
\Chap{20}
\VerseOne{}David s'enfuit de Najoth près de Rama. Il alla voir Jonathan et lui dit : Qu'ai-je fait ? Quelle est mon iniquité, et quel est mon péché devant ton père, pour qu'il en veuille à ma vie ?
\VS{2}Jonathan lui dit : Loin de là ! Tu ne mourras pas. Voici, mon père ne fait aucune chose, ni grande, ni petite, qu'il ne m’en informe ; pourquoi mon père me cacherait-il cette chose-là ? Il n'en est rien.
\VS{3}Alors David jurant, dit encore : Ton père sait certainement que j’ai trouvé grâce à tes yeux, et il aura dit : Que Jonathan ne sache rien de ceci, de peur qu'il n'en soit attristé ; mais Yahweh est vivant, et ton âme vit ! Qu'il n'y a qu'un pas entre moi et la mort.
\VS{4}Alors Jonathan dit à David : Que désires-tu que je fasse ? Et je le ferai pour toi.
\VS{5}Et David dit à Jonathan : Voici, c'est demain la nouvelle lune, et je devrais m'asseoir auprès du roi pour manger, laisse-moi donc aller et je me cacherai aux champs, jusqu'au troisième soir.
\VS{6}Si ton père me cherche, tu lui répondras : David m'a demandé la permission de courir Bethléhem sa ville, parce que toute sa famille fait un sacrifice annuel.
\VS{7}S'il dit ainsi : C’est bien ! Ton serviteur n’a rien à craindre. Mais s'il se met en colère, sache qu’il a résolu mon malheur.
\VS{8}Use donc de bonté envers ton serviteur, puisque tu as conclu une alliance avec ton serviteur devant Yahweh. S'il y a de l’iniquité en moi, tue-moi toi-même ; car pourquoi me mènerais-tu jusqu’à ton père ?
\VS{9}Jonathan lui dit : Loin de toi cette pensée ! Si je savais ta perte arrêtée dans la pensée de mon père, ne t’en informerais-je pas ?
\VS{10}David répondit à Jonathan : Qui m’avertira si la réponse que t'aura faite ton père est sévère ?
\VS{11}Et Jonathan dit à David : Viens et sortons dans les champs. Ils sortirent donc eux deux dans les champs.
\VS{12}Alors Jonathan dit à David : Par Yahweh, le Dieu d'Israël, je sonderai mon père demain, environ à cette heure ou après demain, et s’il est favorable envers David, et que je n'envoie personne vers toi pour t’en informer,
\VS{13}que Yahweh traite Jonathan dans toute sa rigueur ! Si mon père a résolu de te faire du mal, je t’en informerai, et je te laisserai aller, et tu t'en iras en paix, de sorte que Yahweh sera avec toi comme il a été avec mon père.
\VS{14}Si je vis encore, tu useras de la bonté de Yahweh envers moi, en sorte que je ne meure pas.
\VS{15}Ne retire jamais ta bonté de ma maison, pas même quand Yahweh retranchera tous les ennemis de David de dessus la surface de la terre.
\VS{16}Ainsi Jonathan traita alliance avec la maison de David, en disant : Que Yahweh tire vengeance des ennemis de David.
\VS{17}Jonathan se lia encore par serment à David pour l'amour qu'il lui portait ; car il l'aimait comme son âme.
\VS{18}Puis Jonathan lui dit : C'est demain la nouvelle lune, et on s'informera sur toi ; car ta place sera vide.
\VS{19}Le troisième jour au soir, tu descendras en hâte, jusqu’au fond du lieu où tu t’étais caché le jour de l’affaire et tu resteras près de la pierre d'Ezel.
\VS{20}Je tirerai trois flèches à côté de cette pierre, comme si je visais un but.
\VS{21}Et voici, j'enverrai un jeune homme, et je lui dirai : Va, trouve les flèches. Si je dis au jeune homme : Voici, les flèches sont au deçà de toi, prends-les ! Alors viens, car la paix est avec toi et tu n’as rien à craindre ; Yahweh est vivant.
\VS{22}Mais si je dis ainsi au jeune homme : Voici, les flèches sont au-delà de toi ; va-t'en, car Yahweh te renvoie.
\VS{23}Et quant à la parole que nous nous sommes donnée toi et moi ; voici, Yahweh est entre moi et toi à jamais.
\TextTitle{[Saül en colère contre Jonathan]}
\VS{24}David donc se cacha dans le champ. La nouvelle lune étant venue, le roi s'assit pour prendre son repas.
\VS{25}Et le roi s’assit à sa place, comme à l’ordinaire, sur son siège près du mur, Jonathan se leva, et Abner s'assit à côté de Saül ; mais la place de David resta vide.
\VS{26}Saül ne dit rien ce jour-là, car il se disait : Il lui est arrivé quelque chose ; il n'est pas pur, certainement il n'est pas pur.
\VS{27}Mais le lendemain, le second jour de la nouvelle lune, la place de David était encore vide. Et Saül dit à Jonathan, son fils : Pourquoi le fils d'Isaï n'a-t-il été ni hier ni aujourd'hui au repas ?
\VS{28}Et Jonathan répondit à Saül : David m'a instamment demandé la permission d’aller à Bethléhem.
\VS{29}Même il m'a dit : Je te prie, laisse-moi aller ; car notre famille fait un sacrifice dans la ville, et mon frère m'a ordonné de m'y trouver ; maintenant donc si je suis dans tes bonnes grâces, je te prie que j'y aille, afin que je voie mes frères. C'est pour cela qu'il n'est pas venu à la table du roi.
\VS{30}Alors la colère de Saül s'enflamma contre Jonathan et il lui dit : Fils perfide et rebelle, ne sais-je pas que tu as choisi le fils d'Isaï à ta honte et à la honte de ta mère ?
\VS{31}Car aussi longtemps que le fils d'Isaï sera vivant sur la terre, tu ne seras pas stable, ni toi, ni ta royauté ; c'est pourquoi maintenant amène-le-moi, car il est digne de mort.
\VS{32}Et Jonathan répondit à Saül son père, et lui dit : Pourquoi le ferait-on mourir ? Qu'a-t-il fait ?
\VS{33}Et Saül lança sa lance contre lui pour le frapper. Alors Jonathan reconnut que son père avait résolu la mort de David.
\VS{34}Jonathan se leva de table dans une ardente colère, et ne mangea pas le pain le deuxième jour de la nouvelle lune ; car il était affligé à cause de David, parce que son père l'avait insulté.
\VS{35}Le matin venu, Jonathan sortit dans les champs, au lieu convenu avec David, et il amena avec lui un petit garçon.
\VS{36}Et il dit à son garçon : Cours, trouve maintenant les flèches que je m'en vais tirer. Et le garçon courut, et Jonathan tira une flèche qui le dépassa.
\VS{37}Lorsque le garçon arriva au lieu où était la flèche que Jonathan avait tirée, Jonathan cria après lui, et lui dit : La flèche n'est-elle pas plus loin de toi ?
\VS{38}Jonathan cria encore après le garçon : Hâte-toi, ne t'arrête pas ; et le garçon ramassa les flèches, et revint vers son maître.
\VS{39}Le garçon ne savait rien de cette affaire ; seuls David et Jonathan le savaient.
\VS{40}Jonathan remit ses armes au garçon et lui dit : Va, porte-les à la ville.
\VS{41}Le garçon parti, David se leva du côté du midi, se jeta le visage contre terre et se prosterna à trois reprises. Ils s’embrassèrent et pleurèrent ensemble, David versa d’abondantes larmes.
\VS{42}Jonathan dit à David : Va en paix ; comme nous l’avons juré au nom de Yahweh, en disant : Que Yahweh soit entre moi et toi, entre ma postérité et ta postérité.
\VS{43}David donc se leva, s'en alla et Jonathan rentra dans la ville.
\TextTitle{[David s'enfuit]}
\Chap{21}
\VerseOne{}David se rendit à Nob, vers Achimélec le sacrificateur, qui tout effrayé courut au-devant de David, et lui dit : Pourquoi es-tu seul et n'y a-t-il personne avec toi ?
\VS{2}David répondit au sacrificateur Achimélec : Le roi m'a donné un ordre et m'a dit : Que personne ne sache rien de l'affaire pour laquelle je t'envoie, ni de l’ordre que je t'ai donné. J’ai donné rendez-vous à mes hommes en un certain lieu.
\VS{3}Maintenant donc qu'as-tu sous la main ? Donne-moi cinq pains ou ce qui se trouvera.
\VS{4}Le sacrificateur répondit à David et dit : Je n'ai pas de pain ordinaire sous la main, mais du pain sacré\FTNT{Mt. 12:4.} ; pourvu que tes gens se soient abstenus de femmes !
\VS{5}David répondit au sacrificateur : Il est vrai que depuis que je suis parti, il y a trois jours, les femmes ont été éloignées de nous, et les vases des serviteurs sont restés purs, et si c’est là un acte profane, à plus forte raison, il sera aujourd’hui sanctifié par les vases.
\VS{6}Alors le sacrificateur lui donna du pain sacré, car il n'y avait pas là d'autre pain que les pains de proposition qui avaient été ôtés de devant Yahweh, pour le remplacer par du pain chaud le jour où on l’avait pris.
\VS{7}Or il y avait là un homme d'entre les serviteurs de Saül, retenu ce jour-là devant Yahweh ; il s’appelait Doëg, un Edomite, le plus puissant des bergers de Saül.
\VS{8}David dit à Achimélec : Mais n'as-tu pas ici sous la main quelque lance, ou quelque épée ? Car je n’ai pas pris mon épée ni mes armes sur moi, parce que l’ordre du roi était pressant.
\VS{9}Et le sacrificateur dit : Voici l'épée de Goliath, le Philistin, que tu as tué dans la vallée du chêne, elle est enveloppée d'un drap, derrière l'éphod ; si tu veux la prendre pour toi, prends-la ; car il n'y en a pas ici d'autre que celle-là. Et David dit : Il n'y en a pas de pareille ; donne-la-moi.
\TextTitle{[David se rend à Gad]}
\VS{10}Alors David se leva, et s'enfuit ce jour-là, loin de Saül, et s'en alla vers Akisch, roi de Gath.
\VS{11}Et les serviteurs d'Akisch lui dirent : N'est-ce pas là David, roi du pays ? N’est-ce pas celui duquel on chantait et répondait en dansant : Saül a tué ses mille, et David ses dix mille ?
\VS{12}David mit ces paroles dans son cœur, et eut une grande crainte d'Akisch, roi de Gath.
\VS{13}Il se montra comme un insensé à leurs yeux, il agit devant eux comme un fou ; et il faisait des marques sur les battants des portes, et laissait couler sa salive sur sa barbe.
\VS{14}Et Akisch dit à ses serviteurs : Vous voyez que cet homme a perdu la raison. Pourquoi me l'avez-vous amené ?
\VS{15}Est-ce que je manque de fous, pour que vous m’ameniez celui-ci pour faire l'insensé devant moi ? Faudrait-il qu’il entre dans ma maison ?
\TextTitle{[David se réfugie dans la caverne d'Adullam]
\\(1 Ch. 12:16-18)}
\Chap{22}
\VerseOne{}David partit de là, et se sauva dans la caverne d'Adullam. Ses frères et toute la maison de son père l’ayant appris, ils descendirent vers lui.
\VS{2}Tous ceux qui étaient dans la détresse, qui avaient des créanciers, et qui avaient le cœur rempli d'amertume, se rassemblèrent auprès de lui, et il devint leur chef. Ainsi se joignirent à lui environ quatre cents hommes.
\VS{3}David s'en alla de là à Mitspé dans le pays de Moab. Il dit au roi de Moab : Permets, je te prie à mon père et ma mère de se retirer chez vous jusqu'à ce que je sache ce que Dieu fera de moi.
\VS{4}Il les amena devant le roi de Moab, et ils demeurèrent chez lui, tout le temps que David fut dans cette forteresse.
\VS{5}Gad, le prophète, dit à David : Ne demeure pas dans cette forteresse, va-t'en, et entre dans le pays de Juda. David donc s'en alla, et vint dans la forêt de Héreth.
\TextTitle{[Saül tue les sacrificateurs]}
\VS{6}Saül apprit qu'on avait découvert David et ses gens. Or Saül était assis sous le tamaris, à Guibea, sur la hauteur ; il avait sa lance à la main, et tous ses serviteurs se tenaient devant lui.
\VS{7}Saül dit à ses serviteurs qui se tenaient près de lui : Ecoutez Benjamites ! Le fils d'Isaï vous donnera-t-il à vous tous des champs et des vignes ? Vous établira-t-il tous chefs de mille, et chefs de cent ?
\VS{8}Pourquoi avez-vous tous conspiré contre moi, et n'y a-t-il personne qui m’informe de l’alliance que mon fils a faite avec le fils d'Isaï ? Pourquoi n’y a-t-il personne de vous qui souffre à mon sujet et qui m'avertisse que mon fils a suscité mon serviteur contre moi pour me dresser des embûches, comme il le fait aujourd'hui.
\VS{9}Alors Doëg, l’Edomite, qui était établi sur les serviteurs de Saül, répondit et dit : J'ai vu le fils d'Isaï venir à Nob, auprès d’Achimélec, fils d'Achithub.
\VS{10}Il a consulté Yahweh pour lui, il lui a donné des vivres ainsi que l'épée de Goliath, le Philistin.
\VS{11}Alors le roi envoya appeler Achimélec, le sacrificateur, fils d'Achithub, la maison de son père, et les sacrificateurs qui étaient à Nob ; et ils vinrent tous vers le roi.
\VS{12}Saül dit : Ecoute, fils d'Achithub ! Il répondit : Me voici, mon seigneur.
\VS{13}Saül lui dit : Pourquoi avez-vous conspiré contre moi, toi et le fils d'Isaï ? Pourquoi lui as-tu donné du pain et une épée, et as-tu consulté Dieu pour lui, pour qu'il s'élève contre moi comme il le fait aujourd'hui, pour me dresser des embûches ?
\VS{14}Achimélec répondit au roi et dit : Entre tous tes serviteurs y en a-t-il un comme David, fidèle et gendre du roi, qui est parti sur ton commandement, et honoré dans ta maison ?
\VS{15}Est-ce d’aujourd'hui que j’ai commencé à consulter Dieu pour lui ? Loin de moi ! Que le roi n’impute aucun tort à son serviteur, à personne de la maison de mon père ; car ton serviteur ne sait rien de tout cela, petite ou grande.
\VS{16}Le roi lui dit : Tu mourras, Achimélec, toi et toute la maison de ton père.
\VS{17}Alors le roi dit aux coureurs qui se tenaient devant lui : Approchez-vous et mettez à mort les sacrificateurs de Yahweh ; car leur main est avec David, parce qu'ils savaient qu'il s'enfuyait, et qu'ils ne m'ont pas averti. Mais les serviteurs du roi ne voulurent pas étendre la main pour frapper les sacrificateurs de Yahweh.
\VS{18}Le roi dit à Doëg : Approche-toi, et frappe les sacrificateurs. Et Doëg, l’Edomite se tourna, et frappa les sacrificateurs ; il tua en ce jour-là quatre-vingt-cinq hommes qui portaient l'éphod de lin.
\VS{19}Il frappa encore du tranchant de l’épée Nob, ville des sacrificateurs ; hommes et femmes, enfants et nourrissons, bœufs, ânes, et brebis, tombèrent sous le tranchant de l'épée.
\VS{20}Toutefois un des fils d'Achimélec, fils d'Achithub, qui s’appelait Abiathar, se sauva, et s'enfuit auprès de David.
\VS{21}Abiathar rapporta à David que Saül avait tué les sacrificateurs de Yahweh.
\VS{22}David dit à Abiathar : Je savais bien ce jour-là, que Doëg, l’Edomite qui était présent, ne manquerait pas d’informer Saül. Je suis la cause de la mort de toutes les personnes de la maison de ton père.
\VS{23}Reste avec moi, ne crains rien, car celui qui cherche ma vie, cherche la tienne ; avec moi, tu seras bien gardé.
\TextTitle{[David libère Kéïla]}
\Chap{23}
\VerseOne{}On fit ce rapport à David, en disant : Voici, les Philistins font la guerre à Keïla, et pillent les aires.
\VS{2}David consulta Yahweh\FTNT{La clé du succès de David était Yahweh. Il consultait régulièrement Dieu avant de s’engager dans une guerre (Ps. 60:14).} en disant : Irai-je, et frapperai-je ces Philistins ? Et Yahweh répondit à David : Va, et tu frapperas les Philistins, et tu délivreras Keïla.
\VS{3}Les gens de David lui dirent : Voici, nous avons peur ici en Juda ; que sera-ce donc quand nous irons à Keïla contre les troupes des Philistins ?
\VS{4}C'est pourquoi David consulta encore Yahweh, et Yahweh lui répondit et dit : Lève-toi, descends à Keïla, car je livre les Philistins entre tes mains.
\VS{5}Alors David s'en alla avec ses gens à Keïla, et combattit contre les Philistins, et emmena leur bétail, et fit un grand carnage ; ainsi David délivra les habitants de Keïla.
\VS{6}Lorsque Abiathar, fils d'Achimélec, s'était enfui vers David à Keïla, il avait en main l'éphod.
\VS{7}On rapporta à Saül que David était venu à Keïla ; Saül dit : Dieu l'a livré entre mes mains car il s'est enfermé en entrant dans une ville qui a des portes et des barres.
\VS{8}Saül convoqua tout le peuple pour aller à la guerre, afin de descendre à Keïla, et d'assiéger David et ses gens.
\VS{9}David ayant eu connaissance des mauvais desseins de Saül à son égard, dit au sacrificateur Abiathar : Apporte l'éphod.
\VS{10}Puis David dit : Yahweh, Dieu d'Israël ! Ton serviteur apprend que Saül cherche à venir à Keïla, pour détruire la ville à cause de moi.
\VS{11}Les chefs de Keïla me livreront-ils entre ses mains ? Saül descendra-t-il comme ton serviteur l'a entendu dire ? Yahweh, Dieu d'Israël ! Je te prie, révèle-le à ton serviteur. Et Yahweh répondit : Il descendra.
\VS{12}David dit encore : Les chefs de Keïla me livreront-ils, moi et mes gens, entre les mains de Saül ? Et Yahweh répondit : Ils te livreront.
\TextTitle{[David échappe encore à Saül]}
\VS{13}Alors David se leva avec ses gens au nombre d’environ six cents hommes ; et ils sortirent de Keïla, et s'en allèrent où ils purent. On rapporta à Saül que David s'était sauvé de Keïla, c'est pourquoi il cessa sa marche.
\VS{14}David resta au désert, dans des lieux forts, et il se tint sur la montagne au désert de Ziph. Et Saül le cherchait tous les jours, mais Dieu ne le livra pas entre ses mains.
\VS{15}David sachant que Saül était sorti pour attenter à sa vie, se tint au désert de Ziph, dans la forêt.
\VS{16}Alors Jonathan, fils de Saül, se leva, et s'en alla dans la forêt vers David, et fortifia son autorité en Dieu.
\VS{17}Et lui dit : Ne crains pas, car Saül mon père ne t’atteindra pas, mais tu régneras sur Israël, et moi je serai le second après toi ; et même Saül mon père le sait bien.
\VS{18}Ils firent tous les deux, alliance devant Yahweh ; et David resta dans la forêt, mais Jonathan retourna dans sa maison.
\VS{19}Or les Ziphiens montèrent auprès de Saül à Guibea, et lui dirent : David ne se tient-il pas caché parmi nous dans des lieux forts, dans la forêt, sur la colline de Hakila, qui est au midi du désert ?
\VS{20}Maintenant donc, ô roi ! Puisque tout le désir de ton âme est de descendre, descends, et ce sera à nous de le livrer entre les mains du roi.
\VS{21}Et Saül dit : Que Yahweh vous bénisse de ce que vous avez eu pitié de moi !
\VS{22}Allez donc, je vous prie, assurez-vous encore davantage pour savoir et trouver le lieu où il a dirigé ses pas et qui l’a vu ; car m’a-t-on dit, il est fort rusé.
\VS{23}Examinez donc et reconnaissez tous les lieux où il se tient caché, puis retournez vers moi quand vous en serez assurés, et j'irai avec vous. S'il est dans le pays, je le chercherai parmi tous les milliers de Juda.
\VS{24}Ils se levèrent donc et s'en allèrent à Ziph avant Saül. David et ses gens étaient dans le désert de Maon, dans la plaine, au midi du désert.
\VS{25}Saül et ses gens partirent à la recherche de David. Et l’on en informa David, qui descendit le rocher, et resta dans le désert de Maon. Saül l’ayant appris, poursuivit David au désert de Maon.
\VS{26}Saül marchait d’un côté de la montagne, et David et ses gens de l'autre côté de la montagne. David fuyait précipitamment pour échapper à Saül. Mais Saül et ses gens entouraient David et ses gens pour s’emparer d’eux.
\VS{27}Lorsqu’un messager vint à Saül, en disant : Hâte-toi de venir, car les Philistins envahissent le pays.
\VS{28}Alors Saül cessa de poursuivre David, et s'en retourna au-devant des Philistins : C'est pourquoi on appela ce lieu Séla-Hammachlekoth.
\TextTitle{[David épargne la vie de Saül à En-Guédi]}
\Chap{24}
\VerseOne{}Puis David monta de là et demeura dans les lieux forts d'En-Guédi.
\VS{2}Lorsque Saül fut revenu de la poursuite des Philistins, on lui fit ce rapport disant : David est dans le désert d’En-Guédi.
\VS{3}Saül prit trois mille hommes d'élite de tout Israël, et il s'en alla chercher David et ses gens jusque sur le rocher des boucs sauvages.
\VS{4}Saül arriva à des parcs de brebis qui étaient près du chemin, où il y avait une caverne dans laquelle il entra pour se couvrir les pieds. David et ses gens se tenaient au fond de la caverne.
\VS{5}Et les gens de David lui dirent : Voici le jour où Yahweh te dit : Je te livre ton ennemi entre tes mains, afin que tu lui fasses selon ce qu'il te semblera bon. David se leva et coupa tout doucement le pan du manteau de Saül.
\VS{6}Après cela, le cœur de David battit, parce qu'il avait coupé le pan du manteau de Saül.
\VS{7}Et il dit à ses gens : Que Yahweh me garde de commettre une telle action contre mon seigneur, l'oint de Yahweh, en mettant ma main sur lui ; car il est l'oint de Yahweh\FTNT{David épargne Saül parce qu'il fait confiance à Yahweh. David laisse Dieu agir plutôt que d'agir lui-même. C'est ce que Paul, apôtre du Seigneur Jésus-Christ, a écrit en Ro. 11 :17. }.
\VS{8}Ainsi David détourna ses gens par ses paroles, et il ne leur permit pas de s'élever contre Saül. Puis Saül se leva de la caverne et poursuivit son chemin.
\VS{9}Après cela, David se leva, sortit de la caverne, et cria après Saül, en disant : Mon seigneur le roi ! Saül regarda derrière lui, et David s'inclina le visage contre terre et se prosterna.
\VS{10}David dit à Saül : Pourquoi écouterais-tu les paroles des gens qui te disent : Voici, David cherche ton malheur ?
\VS{11}Aujourd'hui, tes yeux ont vu que Yahweh t'avait livré entre mes mains dans la caverne, et on m'a dit de te tuer ; mais je t'ai épargné, et j'ai dit : Je ne porterai pas la main sur mon seigneur ; car il est l'oint de Yahweh.
\VS{12}Regarde donc, mon père, regarde le pan de ton manteau dans ma main. Car, j’ai coupé le pan de ton manteau et je ne t'ai pas tué. Sache et reconnais qu'il n'y a ni mal ni injustice dans ma conduite ; et que je n'ai pas péché contre toi. Mais cependant tu me dresses des embûches pour me tuer.
\VS{13}Yahweh sera juge entre moi et toi, et Yahweh me vengera de toi, mais ma main ne sera pas sur toi.
\VS{14}Des méchants vient la méchanceté, dit l’ancien proverbe. C'est pourquoi je ne porterai pas la main sur toi.
\VS{15}Contre qui est sorti le roi d'Israël ? Qui poursuis-tu ? Un chien mort, une puce ?
\VS{16}Yahweh sera donc juge, et jugera entre moi et toi ; il regardera et plaidera ma cause, il me rendra justice en me délivrant de ta main.
\VS{17}Dès que David eut achevé d’adresser ces paroles à Saül, Saül dit : N'est-ce pas là ta voix, mon fils David ? Et Saül éleva la voix, et pleura.
\VS{18}Et il dit à David : Tu es plus juste que moi ; car tu m'as rendu le bien pour le mal que je t'ai fait,
\VS{19}et tu m'as fait connaître aujourd'hui comment tu as usé de bonté envers moi, car Yahweh m'avait livré entre tes mains, et cependant tu ne m'as pas tué.
\VS{20}Si quelqu’un rencontre son ennemi le laisse-t-il poursuivre tranquillement son chemin ? Que Yahweh donc te récompense pour la grâce que tu m'as faite aujourd'hui !
\VS{21}Et maintenant voici, je sais que tu régneras certainement et que le royaume d'Israël restera entre tes mains.
\VS{22}C'est pourquoi maintenant, jure-moi par Yahweh, que tu ne détruiras pas ma race après moi, et que tu n'extermineras pas mon nom de la maison de mon père.
\VS{23}Et David le jura à Saül. Puis Saül s'en alla dans sa maison, et David et ses gens montèrent au lieu fort.
\TextTitle{[Israël pleure la mort de Samuel]}
\Chap{25}
\VerseOne{}Samuel mourut, et tout Israël s'assembla, et le pleura, et on l'enterra dans sa maison à Rama. David se leva, et descendit au désert de Paran.
\TextTitle{[Ingratitude de Nabal, Abigail une femme de bon sens]}
\VS{2}Il y avait à Maon un homme qui avait ses biens à Carmel, et cet homme-là était très puissant, il avait trois mille brebis, et mille chèvres ; et il se trouvait à Carmel quand on tondait ses brebis.
\VS{3}Cet homme s’appelait Nabal, sa femme Abigaïl, elle était une femme de bon sens, et belle de visage, mais l’homme était cruel et méchant dans toutes ses actions. Il était de la race de Caleb.
\VS{4}David apprit au désert, que Nabal tondait ses brebis.
\VS{5}Il envoya dix jeunes gens, et leur dit : Montez à Carmel, et rendez-vous auprès de Nabal. Vous le saluerez en mon nom,
\VS{6}et vous lui direz : Puisses-tu faire autant l’année prochaine à la même saison, et que la paix soit avec ta maison et tout ce qui est à toi.
\VS{7}Et maintenant j'ai appris que tu as les tondeurs. Or tes bergers ont été avec nous, et nous ne leur avons fait aucune injure, et ils n’ont subi aucune perte pendant tout le temps qu'ils ont été à Carmel.
\VS{8}Demande-le à tes serviteurs, et ils te le diront. Que ces jeunes gens trouvent donc grâce à tes yeux, puisque nous venons dans un jour favorable. Nous te prions de donner à tes serviteurs, et à David, ton fils, ce que tu trouveras sous ta main.
\VS{9}Les gens de David arrivèrent et dirent à Nabal, au nom de David, toutes ces paroles ; puis ils se turent.
\VS{10}Nabal répondit aux serviteurs de David, et dit : Qui est David, et qui est le fils d'Isaï ? Aujourd'hui le nombre des serviteurs qui s’échappent de leurs maîtres se multiplie.
\VS{11}Et prendrais-je mon pain, mon eau, et la viande que j'ai apprêtée pour mes tondeurs, afin de les donner à des gens qui viennent je ne sais d'où ?
\VS{12}Ainsi les gens de David rebroussèrent chemin. Ils s'en retournèrent, et firent leur rapport à David.
\VS{13}Et David dit à ses gens : Que chacun de vous ceigne son épée. Et ils ceignirent chacun leur épée. David aussi ceignit son épée et environ quatre cents hommes montèrent avec David. Il en resta deux cents près des bagages.
\VS{14}Or un des serviteurs de Nabal fit ce rapport à Abigaïl, femme de Nabal, et lui dit : Voici, David a envoyé du désert des messagers pour saluer notre maître, qui les a traités rudement.
\VS{15}Cependant ces hommes ont été très bon envers nous, et ne nous ont fait aucune injure, et rien ne nous été enlevé, tout le temps que nous avons été avec eux lorsque nous étions dans les champs.
\VS{16}Ils nous ont servi de muraille nuit et jour, tout le temps que nous avons été avec eux, faisant paître les troupeaux.
\VS{17}Sache maintenant, et vois ce que tu as à faire, car le mal est résolu contre notre maître, et contre toute sa maison, et il est si méchant qu'on n'ose lui parler.
\VS{18}Abigaïl se hâta donc, et prit deux cents pains, deux outres de vin, cinq pièces de menu bétail, cinq mesures de grain rôti, cent paquets de raisins secs, deux cents de figues sèches, et les mit sur des ânes.
\VS{19}Puis elle dit à ses gens : Passez devant moi, je vais vous suivre. Elle n'en dit rien à Nabal, son mari.
\VS{20}Et étant montée sur un âne, elle descendait de la montagne par un chemin couvert ; voici, David et ses gens descendaient en face d’elle, et elle les rencontra.
\VS{21}David avait dit : C'est en vain que j'ai gardé tout ce que cet homme a dans le désert, en sorte qu'il ne s'est rien perdu de tout ce qu’il possède ; il m'a rendu le mal pour le bien.
\VS{22}Que Dieu traite son serviteur David dans toute sa rigueur, si d'ici au matin je laisse subsister de tout ce qui appartient à Nabal.
\VS{23}Lorsque Abigaïl aperçut David, elle se hâta de descendre de son âne, et tomba sur sa face devant David, et se prosterna contre terre.
\VS{24}Elle se jeta donc à ses pieds et lui dit : A moi la faute, mon seigneur ! Permets à ta servante de parler devant toi, et écoute les paroles de ta servante.
\VS{25}Que mon seigneur ne prenne pas garde à ce méchant homme, à Nabal, car il est comme son nom ; Nabal est son nom, et il y a de la folie chez lui. Et moi, ta servante, je n'ai pas vu les gens que mon seigneur a envoyés.
\VS{26}Maintenant, mon seigneur, aussi vrai que Yahweh est vivant, et que ton âme vit, Yahweh t'a empêché d'en venir au sang, et il a retenu ta main. Or que tes ennemis, et ceux qui cherchent à nuire à mon seigneur, soient comme Nabal.
\VS{27}Voici un présent, que ta servante a apporté à mon seigneur, afin qu'on le donne aux gens qui sont à la suite de mon seigneur.
\VS{28}Pardonne, je te prie, le crime de ta servante ; vu que Yahweh ne manquera pas d'établir une maison ferme à mon seigneur ; car mon seigneur conduit les batailles de Yahweh, et il ne s'est trouvé en toi aucun mal pendant toute ta vie.
\VS{29}Si les hommes se lèvent pour te persécuter, et pour chercher ton âme, l'âme de mon seigneur sera liée au faisceau des vivants auprès de Yahweh ton Dieu ; mais il lancera au loin, avec la fronde, l'âme de tes ennemis.
\VS{30}Lorsque Yahweh fera à mon seigneur selon tout le bien qu'il t'a prédit, et qu’il t'établira conducteur d'Israël,
\VS{31}ceci ne sera pas un obstacle, ni un sujet de regret dans l'âme de mon seigneur, pour avoir répandu le sang inutilement, et pour s'être vengé lui-même. Aussi lorsque Yahweh aura fait du bien à mon seigneur, tu te souviendras de ta servante.
\VS{32}Alors David dit à Abigaïl : Béni soit Yahweh, le Dieu d'Israël, qui t'a aujourd'hui envoyée à ma rencontre !
\VS{33}Et béni soit ton bon sens, et bénie sois-tu, toi qui m'as aujourd'hui empêché d'en venir au sang, et qui as retenu ma main !
\VS{34}Car Yahweh, le Dieu d'Israël qui m'a empêché de te faire du mal, est vivant ! Si tu ne t’étais hâtée de venir à ma rencontre, il ne serait resté qui que ce soit à Nabal d'ici au matin.
\VS{35}David prit donc de sa main ce qu'elle lui avait apporté, et lui dit : Remonte en paix dans ta maison ; regarde, j'ai écouté ta voix, et j'ai répondu favorablement à ta demande.
\TextTitle{[Mort de Nabal]}
\VS{36}Alors Abigaïl revint auprès de Nabal ; et voici, il faisait un festin dans sa maison, comme un festin de roi ; et Nabal avait le cœur joyeux, et il était complètement ivre ; c'est pourquoi elle ne lui dit aucune chose petite ou grande, jusqu'au matin.
\VS{37}Mais le matin, l’ivresse de Nabal étant dissipée, sa femme lui raconta toutes ces choses. Le cœur de Nabal reçut un coup mortel, de sorte qu'il devint comme une pierre.
\VS{38}Environ dix jours après, Yahweh frappa Nabal, et il mourut.
\VS{39}Lorsque David apprit que Nabal était mort, il dit : Béni soit Yahweh, qui m'a vengé de l'outrage que j'avais reçu de la main de Nabal, et qui a préservé son serviteur de faire du mal, et a fait retomber le mal de Nabal sur sa tête ! Puis David envoya des gens pour parler à Abigaïl, afin de la prendre pour sa femme.
\VS{40}Les serviteurs de David vinrent auprès d'Abigaïl à Carmel, et lui parlèrent, en disant : David nous a envoyés vers toi, afin de te prendre pour femme.
\VS{41}Alors elle se leva, et se prosterna le visage contre terre, et dit : Voici, ta servante sera à ton service afin de laver les pieds des serviteurs de mon seigneur.
\VS{42}Aussitôt, Abigaïl se leva et monta sur un âne, accompagnée de cinq jeunes filles ; elle suivit les messagers de David, et fut sa femme.
\VS{43}Or David avait pris aussi Achinoam, de Jizreel, et toutes les deux furent ses femmes.
\VS{44}Et Saül avait donné Mical, sa fille, femme de David, à Palthi, fils de Laïsch, qui était de Gallim.
\TextTitle{[David épargne encore la vie de Saül]}
\Chap{26}
\VerseOne{}Les Ziphiens allèrent encore auprès de Saül à Guibea, en disant : David ne se tient-il pas caché sur la colline de Hakila, en face du désert ?
\VS{2}Saül se leva, et descendit au désert de Ziph, avec trois mille hommes de l'élite d'Israël, pour chercher David dans le désert de Ziph.
\VS{3}Saül campa sur la colline de Hakila, en face du désert, près du chemin. David se tenait dans le désert, et il aperçut que Saül marchait à sa poursuite au désert,
\VS{4}alors il envoya des espions et apprit avec certitude que Saül était arrivé.
\VS{5}Alors David se leva, et alla au lieu où Saül campait, et David vit la place où couchait Saül, avec Abner, fils de Ner, chef de son armée. Saül couchait au milieu du camp, et le peuple campait autour de lui.
\VS{6}David prit la parole et dit à Achimélec, Héthien, et à Abischaï, fils de Tseruja et frère de Joab, il dit : Qui veut descendre avec moi dans le camp vers Saül ? Et Abischaï répondit : J'y descendrai avec toi.
\VS{7}David et Abischaï allèrent de nuit vers le peuple, et voici, Saül dormait étant couché au milieu du camp, et sa lance était plantée en terre à son chevet ; et Abner, et le peuple étaient couchés autour de lui.
\VS{8}Alors Abischaï dit à David : Aujourd'hui, Dieu a livré ton ennemi entre tes mains ; laisse-moi donc le frapper avec la lance, jusqu'en terre d'un seul coup, et je n'y retournerai pas une seconde fois.
\VS{9}Et David dit à Abischaï : Ne le tues pas ! Car qui porterait impunément sa main sur l'oint de Yahweh ?
\VS{10}David dit encore : Yahweh est vivant ! C’est Yahweh seul qui le frappera, soit que son jour vienne, soit qu'il descende au combat et qu'il y périsse.
\VS{11}Que Yahweh me garde de mettre ma main sur l'oint de Yahweh ; mais prends maintenant la lance qui est à son chevet et la cruche d’eau, et allons-nous-en.
\VS{12}David donc prit la lance et la cruche d’eau qui étaient au chevet de Saül, puis ils s'en allèrent. Personne ne les vit, ni ne s’aperçut de rien, ni ne se réveilla ; car ils dormaient tous d’un profond sommeil dans lequel Yahweh les avait plongés.
\VS{13}David passa de l'autre côté, et s'arrêta au loin sur le sommet de la montagne, et il y avait une grande distance entre eux.
\VS{14}Et il cria au peuple, et à Abner, fils de Ner, en disant : Ne répondras-tu pas, Abner ? Abner répondit, et dit : Qui es-tu toi qui cries vers le roi ?
\VS{15}Alors David dit à Abner : N'es-tu pas un vaillant homme ? Qui est semblable à toi en Israël ? Pourquoi donc n'as-tu pas gardé le roi ton seigneur ? Car quelqu'un du peuple est venu pour tuer le roi ton seigneur.
\VS{16}Ce que tu as fait n’est pas bien ; Yahweh est vivant ! Vous méritez la mort, pour avoir si mal gardé votre seigneur, l'oint de Yahweh. Et maintenant regarde où sont la lance du roi et la cruche d’eau qui était à son chevet.
\VS{17}Alors Saül reconnut la voix de David, et dit : N'est-ce pas là ta voix, mon fils David ? Et David dit : C'est ma voix, ô roi mon seigneur.
\VS{18}Il dit encore : Pourquoi mon seigneur poursuit-il son serviteur ? Car qu'ai-je fait, de quoi suis-je coupable ?
\VS{19}Maintenant donc je te prie, que le roi mon seigneur écoute les paroles de son serviteur. Si c'est Yahweh qui te pousse contre moi, que ton offrande lui soit agréable ; mais si ce sont les hommes, qu’ils soient maudits devant Yahweh ; car aujourd'hui ils m'ont chassé, afin que je ne puisse me joindre à l'héritage de Yahweh, et ils m'ont dit : Va, sers les dieux étrangers.
\VS{20}Que mon sang ne tombe pas en terre loin de la face de Yahweh ! Car le roi d’Israël est sorti pour chercher une puce, comme on poursuivrait une perdrix dans les montagnes.
\TextTitle{[Saül se repend devant David]}
\VS{21}Saül dit : J'ai péché, reviens mon fils David ; car je ne te ferai plus de mal, parce qu'aujourd'hui ma vie t'a été précieuse. Voici, j'ai agi en insensé, et j'ai commis une très grande faute.
\VS{22}David répondit et dit : Voici la lance du roi que l'un de tes gens vienne la prendre.
\VS{23}Que Yahweh rende à chacun selon sa justice et selon sa fidélité ; car il t'avait livré aujourd'hui entre mes mains, mais je n'ai pas voulu mettre ma main sur l'oint de Yahweh.
\VS{24}Voici, comme ta vie a été aujourd'hui de grand prix à mes yeux, ainsi ma vie sera de grand prix aux yeux de Yahweh, et il me délivrera de toutes les angoisses.
\VS{25}Saül dit à David : Béni sois-tu, mon fils David ! Tu auras du succès dans tes entreprises. Alors David continua son chemin, et Saül s'en retourna chez lui.
\TextTitle{[David se réfugie dans le pays des Philistins]}
\Chap{27}
\VerseOne{}David dit en son cœur : Certes je périrai un jour par les mains de Saül ; ne vaut-il pas mieux que je me sauve en hâte au pays des Philistins, afin que Saül renonce à me chercher encore dans tout le territoire d'Israël ? Ainsi j’échapperai à sa main.
\VS{2}David se leva, lui et les six cents hommes qui étaient avec lui, et il passa chez Akisch, fils de Maoc, roi de Gath.
\VS{3}David et ses gens restèrent à Gath auprès d’Akisch ; ils avaient chacun leur famille, David et ses deux femmes, Achinoam de Jizreel, et Abigaïl, femme de Nabal, qui était de Carmel.
\VS{4}Alors on informa Saül que David s'était enfui à Gath ; et il cessa de le chercher.
\VS{5}David dit à Akisch : Si j'ai trouvé grâce à tes yeux, qu'on me donne dans l'une des villes du pays, un lieu où je puisse habiter. Car pourquoi ton serviteur habiterait-il dans la ville royale avec toi ?
\VS{6}Akisch lui donna ce même jour, Tsiklag. C'est pourquoi Tsiklag appartient aux rois de Juda jusqu'à ce jour.
\VS{7}Le temps que David demeura dans le pays des Philistins fut d’un an et quatre mois.
\VS{8}David montait avec ses gens faire des incursions chez les Gueschuriens, les Guirziens, et les Amalécites ; car ces nations habitaient dans le territoire dès les temps anciens, depuis Schur jusqu'au pays d'Egypte.
\VS{9}David ravageait ce territoire, il ne laissait en vie ni homme ni femme, et il prenait les brebis, les bœufs, les ânes, les chameaux, et les vêtements, puis il s'en retournait, et allait chez Akisch.
\VS{10}Akisch disait : Où avez-vous fait vos incursions aujourd'hui ? Et David répondait : Vers le midi de Juda, vers le midi des Jerachmeélites, et vers le midi des Kéniens.
\VS{11}Mais David ne laissait en vie ni homme ni femme pour les amener à Gath, de peur, disait-il, qu'ils ne rapportent quelque chose contre nous, disant : Ainsi a fait David. Et il agit ainsi tout le temps qu'il demeura dans le pays des Philistins.
\VS{12}Akisch croyait David, et il disait : Il se rend odieux à Israël, son peuple ; c'est pourquoi il sera mon serviteur à jamais.
\TextTitle{[Les Philistins vont en guerre contre Saül]}
\Chap{28}
\VerseOne{}En ces temps-là, les Philistins rassemblèrent leurs armées pour faire la guerre, pour combattre Israël. Akisch dit à David : Sache certainement que vous viendrez avec moi au camp, toi et tes gens.
\VS{2}David répondit à Akisch : Certainement tu verras ce que ton serviteur fera. Et Akisch dit à David : C'est pour cela que je te confierai toujours la garde de ma personne.
\VS{3}Or Samuel était mort, et tout Israël avait fait le deuil, et on l'avait enseveli à Rama qui était sa ville. Saül avait ôté du pays ceux qui avaient des esprits de Python\FTNT{L’esprit de python possède les faux prophètes (Ac. 16:16-19).}, et les médiums.
\VS{4}Les Philistins se rassemblèrent et vinrent camper à Sunem ; Saül aussi rassembla tout Israël, et ils campèrent à Guilboa.
\VS{5}A la vue du camp des Philistins, Saül eut peur, et son cœur fut saisi de crainte.
\VS{6}Saül consulta Yahweh ; mais Yahweh ne lui répondit rien, ni par des songes, ni par l'urim, ni par les prophètes.
\TextTitle{[Saül consulte une femme qui évoque les morts]}
\VS{7}Saül dit à ses serviteurs : Cherchez-moi une femme qui ait un esprit de Python, et j'irai vers elle, et je la consulterai. Ses serviteurs lui dirent : Voilà, il y a une femme à En-Dor qui évoque les morts.
\VS{8}Alors Saül se déguisa, prit d'autres vêtements, et il partit avec deux hommes. Ils arrivèrent de nuit chez cette femme et Saül lui dit : Je te prie devine-moi par l’esprit de Python, et fais-moi monter vers moi celui que je te dirai.
\VS{9}Mais la femme lui répondit : Voici, tu sais ce que Saül a fait, et comment il a exterminé du pays ceux qui ont l'esprit de Python et les médiums ; pourquoi donc dresses-tu un piège à mon âme pour me faire mourir ?
\VS{10}Saül lui jura par Yahweh, et lui dit : Yahweh est vivant ! Il ne t’arrivera pas de mal pour cela.
\VS{11}Alors la femme dit : Qui veux-tu que je te fasse monter ? Et il répondit : Fais-moi monter Samuel.
\VS{12}Et la femme voyant Samuel s'écria à haute voix, en disant à Saül : Pourquoi m'as-tu trompée ? Car tu es Saül.
\VS{13}Et le roi lui répondit : Ne crains pas, mais que vois-tu ? La femme dit à Saül : Je vois un dieu qui monte de la terre.
\VS{14}Il lui dit encore : Comment est-il fait ? Elle répondit : C'est un vieillard qui monte, et il est couvert d'un manteau. Et Saül comprit que c'était Samuel, il s’inclina le visage contre terre et se prosterna.
\VS{15}Samuel dit à Saül : Pourquoi m'as-tu troublé en me faisant monter ? Et Saül répondit : Je suis dans une grande angoisse ; car les Philistins me font la guerre, et Dieu s'est retiré de moi, et ne m'a plus répondu ni par les prophètes ni par des songes ; c'est pourquoi je t'ai invoqué\FTNT{}, afin que tu me fasses entendre ce que j'aurai à faire.
\VS{16}Samuel dit : Pourquoi donc me consultes-tu, puisque Yahweh s'est retiré de toi, et qu'il est devenu ton ennemi ?
\VS{17}Yahweh te traite comme je te l’avais annoncé de sa part ; car Yahweh a déchiré le royaume d'entre tes mains, et l'a donné à un autre, à David.
\VS{18}Parce que tu n'as pas obéi à la voix de Yahweh, et que tu n'as pas exécuté l'ardeur de sa colère contre Amalek, à cause de cela, Yahweh te traite de cette manière aujourd'hui.
\VS{19}Yahweh livrera Israël avec toi entre les mains des Philistins, et vous serez demain avec moi, toi et tes fils; Yahweh livrera aussi le camp d'Israël entre les mains des Philistins.
\VS{20}Saül s’écroula à terre tout étendu, très effrayé des paroles de Samuel, les forces lui manquèrent parce qu'il n'avait rien mangé ce jour, ni toute cette nuit.
\VS{21}Alors la femme vint auprès de Saül, et voyant qu'il avait été très effrayé, elle lui dit : Voici, ta servante a obéi à ta voix, j'ai exposé ma vie, et j'ai obéi aux paroles que tu m'as dites.
\VS{22}Maintenant, je te prie, écoute toi aussi ce que ta servante te dira : Laisse-moi de servir avant un morceau de pain, afin que tu manges pour avoir la force de te remettre en route.
\VS{23}Et il le refusa et dit : Je ne mangerai pas. Mais ses serviteurs et la femme aussi le pressèrent tellement qu'il écouta leur voix. Il se leva de terre, et s'assit sur un lit.
\VS{24}Cette femme avait dans sa maison un veau qu'elle engraissait ; et elle se hâta de le tuer, puis elle prit de la farine, et la pétrit, et en cuisit des pains sans levain.
\VS{25}Elle les mit devant Saül et devant ses serviteurs. Et ils mangèrent. Puis s'étant levés, ils s'en allèrent cette nuit-là.
\TextTitle{[Les philistins refusent que David combattent contre Israël]}
\Chap{29}
\VerseOne{}Les Philistins rassemblèrent toutes leurs armées à Aphek, et Israël campa près de la fontaine de Jizreel.
\VS{2}Les princes des Philistins s’avancèrent avec leurs centaines et leurs milliers, et David et ses gens marchèrent à l'arrière-garde avec Akisch.
\VS{3}Les princes des Philistins dirent : Que font ici ces Hébreux ? Et Akisch répondit aux princes des Philistins : N'est-ce pas David, serviteur de Saül, roi d'Israël, il y a longtemps qu’il est avec moi, même quelques années, et je n'ai pas trouvé quelque chose à lui reprocher depuis son arrivée, jusqu'à ce jour.
\VS{4}Mais les princes des Philistins se mirent en colère contre lui, et lui dirent : Renvoie cet homme, et qu'il retourne dans le lieu où tu l'as établi, et qu'il ne descende pas avec nous dans la bataille, de peur qu'il ne se tourne contre nous dans la bataille ; car comment pourrait-il se remettre en grâce auprès de son maître ? Ne serait-ce pas par le moyen des têtes de nos hommes ?
\VS{5}N'est-ce pas ce David, pour qui l’on chantait et répondait en dansant : Saül a frappé ses mille, et David ses dix mille ?
\VS{6}Akisch appela David, et lui dit : Yahweh est vivant ! Tu es certainement un homme droit, et ta conduite dans le camp m'a paru bonne, car je n'ai pas trouvé de mal en toi, depuis le jour où tu es arrivé auprès de moi jusqu'à ce jour ; mais tu ne plais pas aux princes.
\VS{7}Maintenant retourne, et va-t'en en paix, afin que tu ne fasses aucune chose qui déplaise aux princes des Philistins.
\VS{8}David dit à Akisch : Mais qu'ai-je fait ? Et qu'as-tu trouvé en ton serviteur depuis que je suis avec toi jusqu'à ce jour, pour que je n'aille pas combattre contre les ennemis du roi, mon seigneur ?
\VS{9}Akisch répondit et dit à David : Je le sais, car tu es agréable à mes yeux, comme un ange de Dieu ; mais c'est seulement les chefs des Philistins qui disent : Il ne montera pas avec nous dans la bataille.
\VS{10}C'est pourquoi lève-toi de bon matin, avec les serviteurs de ton maître qui sont venus avec toi ; levez-vous de bon matin, et partez dès que vous verrez le jour, allez-vous-en.
\VS{11}Ainsi David se leva de bonne heure, lui et ses gens, pour partir dès le matin, et retourner dans le pays des Philistins. Et les Philistins montèrent à Jizreel.
\TextTitle{[David libère Tsiklag]}
\Chap{30}
\VerseOne{}Lorsque David et ses gens arrivèrent à Tsiklag, le troisième jour, les Amalécites avaient fait une invasion dans le midi, et à Tsiklag, et ils avaient frappé et brûlé Tsiklag.
\VS{2}Après avoir fait prisonniers les femmes et tous ceux qui étaient là, petits et grands. Ils n’avaient tué personne, mais ils les avaient emmenés, et s’étaient remis en chemin.
\VS{3}David et ses gens revinrent dans la ville et voici, elle était brûlée, et leurs femmes, leurs fils, et leurs filles avaient été faits prisonniers.
\VS{4}C’est pourquoi David et le peuple qui était avec lui élevèrent leur voix, et pleurèrent tellement qu’il n’y avait plus en eux de force pour pleurer.
\VS{5}Les deux femmes de David avaient été emmenées, Achinoam, de Jizreel, et Abigaïl, de Carmel, femme de Nabal.
\VS{6}David fut dans une grande angoisse, parce que le peuple parlait de le lapider ; car tout le peuple avait de l’amertume dans l’âme à cause de leurs fils et de leurs filles ; toutefois David se fortifia en Yahweh, son Dieu.
\VS{7}Et il dit au sacrificateur Abiathar, fils d’Achimélec : Apporte-moi, je te prie, l’éphod ! Abiathar apporta l'éphod à David.
\VS{8}Et David consulta Yahweh, en disant : Poursuivrai-je cette troupe ? L’atteindrai-je ? Et il lui répondit : Poursuis, car tu l’atteindras, et tu délivreras.
\VS{9}David s’en alla avec les six cents hommes qui étaient avec lui, et ils arrivèrent au torrent de Besor, où s’arrêtèrent ceux qui restaient en arrière.
\VS{10}Ainsi David et quatre cents hommes, continuèrent la poursuite, mais deux cents hommes s’arrêtèrent, trop fatigués pour pouvoir passer le torrent de Besor.
\VS{11}Ayant trouvé un homme Egyptien dans les champs, ils l’amenèrent à David, et lui donnèrent du pain, il mangea, puis ils lui donnèrent de l’eau à boire.
\VS{12}Ils lui donnèrent aussi quelques figues sèches, et deux grappes de raisins secs, et il mangea, et le coeur lui revint ; car cela faisait trois jours et trois nuits qu’il n’avait pas mangé de pain, ni bu d’eau.
\VS{13}Et David lui dit : A qui es-tu ? Et d’où es-tu ? Et il répondit : Je suis un garçon égyptien, serviteur d’un homme amalécite ; et mon maître m’a abandonné, parce que j’étais malade il y a trois jours.
\VS{14}Nous avons envahi le midi des Kéréthiens, et sur ce qui est à Juda, et sur le midi de Caleb, et nous avons mis le feu et brûlé Tsiklag.
\VS{15}David lui dit : Me conduiras-tu vers cette troupe ? Et il répondit : Jure-moi par le Nom de Dieu que tu ne me feras point mourir, et que tu ne me livreras point entre les mains de mon maître, et je te conduirai vers cette troupe.
\VS{16}Et il le conduisit. Et voici, ils étaient dispersés sur toute la contrée, mangeant, buvant, et dansant, à cause de ce grand butin qu’ils avaient pris au pays des Philistins, et au pays de Juda.
\VS{17}Et David les frappa depuis l’aube du jour, jusqu’au soir du lendemain, et il n’en échappa aucun d’eux, hormis quatre cents jeunes hommes qui montèrent sur des chameaux, et s’enfuirent.
\VS{18}David recouvra tout ce que les Amalécites avaient emporté ; il délivra aussi ses deux femmes.
\VS{19}Il ne leur manqua personne, depuis le plus petit jusqu’au plus grand, ni fils ni filles, ni butin, ni rien de ce qu’ils leur avaient emporté ; David ramena tout.
\VS{20}David reprit aussi tout le gros et menu bétail, qu’on mena devant les troupeaux ; et on disait : C’est ici le butin de David.
\TextTitle{[David partage le butin]}
\VS{21}Puis David arriva auprès de deux cents hommes qui avaient été tellement fatigués qu’ils n’avaient pu suivre David, et qu’on avait laissés au torrent de Besor. Ils sortirent au-devant de David, et au-devant du peuple qui était avec lui. David s’étant approché du peuple, il les salua aimablement.
\VS{22}Mais tous les mauvais et méchants hommes qui étaient allés avec David, prirent la parole, et dirent : Puisqu’ils ne sont point venus avec nous, nous ne leur donnerons rien du butin que nous avons récupéré, sinon à chacun sa femme et ses enfants, et qu’ils les emmènent, et s’en aillent.
\VS{23}Mais David dit : Mes frères, n’agissez pas ainsi au sujet de ce que Yahweh nous a donné, il nous a gardés, et a livré entre nos mains la troupe qui était venue contre nous.
\VS{24}Qui vous écouterait dans cette affaire ? Car celui qui est resté près des bagages doit avoir autant que celui qui est descendu sur le champ de bataille ; ils partageront ensemble.
\VS{25}Il en fut ainsi depuis ce jour et dans la suite, il en fut fait de même une ordonnance et une loi en Israël.
\VS{26}David revint à Tsiklag, et envoya une partie du butin aux anciens de Juda, à ses amis, en disant : Voici, un présent pour vous, du butin des ennemis de Yahweh.
\VS{27}Il en envoya à ceux de Béthel, à ceux qui étaient à Ramoth du midi, à ceux de Jatthir,
\VS{28}à ceux d’Aroër, à ceux de Siphmoth, à ceux de Eschthemoa,
\VS{29}à ceux de Racal, à ceux des villes des Jerachmeélites, à ceux des villes des Kéniens,
\VS{30}à ceux d’Horma, à ceux de Cor-Aschan, à ceux d’Athac,
\VS{31}à ceux d’Hébron, et dans tous les lieux où David avait demeuré, lui et ses gens.
\TextTitle{[Mort de Jonathan et Saül à Guilboa]
\\(1 Ch. 10:1-14)}
\Chap{31}
\VerseOne{}Les Philistins livrèrent bataille à Israël, et les hommes d'Israël s'enfuirent devant les Philistins, et furent tués sur la montagne de Guilboa.
\VS{2}Les Philistins atteignirent Saül et ses fils, et tuèrent Jonathan, Abinadab et Malkischua, fils de Saül.
\VS{3}L’effort du combat se porta sur Saül, et les archers l’atteignirent et le blessèrent grièvement.
\VS{4}Alors Saül dit à celui qui portait ses armes : Tire ton épée, et transperce-moi, de peur que ces incirconcis ne viennent, ne me transpercent, et ne m’outragent. Mais celui qui portait ses armes refusa, parce qu'il était saisi de crainte. Saül prit l'épée, et se jeta dessus.
\VS{5}Alors celui qui portait les armes de Saül, voyant que Saül était mort, se jeta aussi sur son épée, et mourut avec lui.
\VS{6}Ainsi périrent en ce jour, Saül et ses trois fils, celui qui portait ses armes, et tous ses gens.
\VS{7}Ceux d'Israël qui étaient de ce côté de la vallée et de ce côté du Jourdain, ayant vu que les Israëlites s'étaient enfuis que Saül et ses fils étaient morts, abandonnèrent les villes et s'enfuirent ; de sorte que les Philistins y entrèrent et s’y établirent.
\VS{8}Le lendemain, les Philistins vinrent pour dépouiller les morts, et ils trouvèrent Saül et ses trois fils, étendus sur la montagne de Guilboa.
\VS{9}Ils coupèrent la tête de Saül et le dépouillèrent de ses armes. Ils firent annoncer ces bonnes nouvelles par tout le pays des Philistins, dans les maisons de leurs idoles et parmi le peuple.
\VS{10}Ils déposèrent les armes de Saül dans le temple d’Astarté, et ils attachèrent son cadavre sur les murs de Beth-Schan.
\VS{11}Lorsque les habitants de Jabès en Galaad apprirent ce que les Philistins avaient fait à Saül,
\VS{12}tous les vaillants hommes, se levèrent et marchèrent toute la nuit, et ils enlevèrent des murs de Beth-Schan le cadavre de Saül et les cadavres de ses fils. Ils revinrent à Jabès, où ils les brûlèrent.
\VS{13}Puis ils prirent leurs os, les ensevelirent sous un tamaris près de Jabès, et ils jeûnèrent sept jours.
\PPE{}
\end{multicols}

\clearpage\ShortTitle{2 Samuel}\BookTitle{2 Samuel}\BFont
\noindent\hrulefill
{\footnotesize
\textit{
\bigskip
{\centering{}
\\Auteur : Inconnu
\\(Heb. : Shemuw'el)
\\Signification : Entendu, exaucé de Dieu
\\Thème : Le règne de David
\\Date de rédaction : 10\up{ème} siècle av. J.-C.\\}
}
%\bigskip
\textit{
\\Suite du premier livre de Samuel, ce livre commence par le récit de la mort de Saül et l'accession progressive à la royauté de David. La faveur de Dieu dans sa vie lui donna du succès et lui permit d'étendre son royaume jusqu'au nord de Damas. Au détriment de sa piété et de son alliance avec Dieu, David commit de lourdes erreurs. Il s'en repentit sincèrement, mais il dut en assumer les conséquences…\bigskip
}
}
\par\nobreak\noindent\hrulefill
\begin{multicols}{2}
\Chap{1}
\TextTitle{Attitude de David à la mort de Saül}
\VerseOne{}Et il arriva qu'après la mort de Saül, David, qui était revenu vainqueur des Amalécites, resta deux jours à Tsiklag.
\VS{2}Le troisième jour, un homme arriva du camp de Saül, avec ses vêtements déchirés et de la terre sur sa tête. Il se présenta à David, se jeta par terre et se prosterna.
\VS{3}David lui dit : D'où viens-tu ? Il lui répondit : Je me suis échappé du camp d'Israël.
\VS{4}David lui dit : Qu'est-il arrivé ? Je te prie, raconte-le-moi! Il répondit : Le peuple s'est enfui de la bataille, et il y a eu beaucoup du peuple qui sont tombés morts ; Saül aussi et Jonathan, son fils, sont morts.
\VS{5}David dit à ce jeune garçon qui lui disait ces nouvelles : Comment sais-tu que Saül et Jonathan, son fils, sont morts ?
\VS{6}Le jeune garçon qui lui disait ces nouvelles lui répondit : Je me trouvais par hasard sur la montagne de Guilboa ; et voici, Saül s'appuyait sur sa lance, et voici, les chars et quelques chefs des cavaliers le poursuivaient.
\VS{7}S'étant retourné, il m'aperçut et m'appela. Je lui répondis : Me voici !
\VS{8}Il me dit : Qui es-tu ? Je lui répondis : Je suis Amalécite.
\VS{9}Et il dit : Approche-toi de moi et tue-moi ; car je suis dans une grande angoisse, et ma vie est encore toute en moi.
\VS{10}Je m'approchai de lui et je lui donnai la mort\FTNT{Cet homme Amalécite a peut-être menti afin de gagner la faveur de David (1 S. 31:3-5).}, sachant bien qu'il ne survivrait après s'être jeté sur sa lance. J'ai pris la couronne qu'il avait sur sa tête, et le bracelet qui était à son bras, et je les apporte ici à mon seigneur.
\VS{11}Alors David saisit ses vêtements et les déchira, et tous les hommes qui étaient avec lui firent de même.
\VS{12}Ils furent dans le deuil, ils pleurèrent et ils jeûnèrent jusqu'au soir, à cause de Saül, de Jonathan, son fils, à cause du peuple de Yahweh, et de la maison d'Israël, parce qu'ils étaient tombés par l'épée.
\VS{13}David dit au jeune garçon qui lui avait apporté ces nouvelles : D'où es-tu ? Et il répondit : Je suis fils d'un homme étranger, d'un Amalécite.
\VS{14}David lui dit : Comment n'as-tu pas craint d'avancer ta main pour tuer l'oint de Yahweh ?
\VS{15}Et David appela l'un de ses serviteurs et lui dit : Approche-toi, jette-toi sur lui ! Ce dernier le frappa et il mourut.
\VS{16}Et David lui dit : Que ton sang retombe sur ta tête, car ta bouche a témoigné contre toi, en disant : J'ai fait mourir l'oint de Yahweh !
\TextTitle{Chant funèbre de David}
\VS{17}Alors David composa sur Saül et sur Jonathan, son fils, un chant funèbre,
\VS{18}qu'il ordonna d'enseigner aux fils de Juda. C'est le cantique de l'arc : Il est écrit dans le livre du Juste.
\VS{19}L'élite d'Israël a succombé sur tes collines ! Comment des héros sont-ils tombés ?
\VS{20}Ne l'annoncez pas dans Gath et n'en publiez point la nouvelle dans les rues d'Askalon, de peur que les filles des Philistins ne se réjouissent, de peur que les filles des incirconcis n'en tressaillent de joie.
\VS{21}Montagnes de Guilboa ! Qu'il n'y ait sur vous ni rosée, ni pluie, ni champs qui donnent des prémices pour les offrandes ! Car là ont été jetés les boucliers des héros, le bouclier de Saül ; l'huile a cessé de les oindre.
\VS{22}L'arc de Jonathan ne revenait jamais sans être teint du sang des blessés et de la graisse des hommes forts ; et l'épée de Saül ne retournait jamais sans effet.
\VS{23}Saül et Jonathan, aimables et agréables pendant leur vie, n'ont point été séparés dans leur mort ; ils étaient plus légers que les aigles, ils étaient plus forts que des lions.
\VS{24}Filles d'Israël ! Pleurez sur Saül, qui vous revêtait magnifiquement de cramoisi, qui mettait des ornements d'or à vos habits.
\VS{25}Comment les héros sont-ils tombés au milieu du combat ? Comment Jonathan a-t-il été tué sur tes collines ?
\VS{26}Jonathan, mon frère, je suis dans la douleur à cause de toi ! Tu faisais tout mon plaisir ; l'amour que j'avais pour toi était plus grand que celui qu'on a pour les femmes.
\VS{27}Comment sont tombés les héros ? Comment se sont perdus les instruments de guerre ?
\Chap{2}
\TextTitle{David oint roi de Juda}
\VerseOne{}Et il arriva après cela que David consulta Yahweh, en disant : Monterai-je dans l'une des villes de Juda ? Yahweh lui répondit : Monte. David dit : Où monterai-je ? Yahweh répondit : A Hébron.
\VS{2}David y monta, avec ses deux femmes, Achinoam de Jizreel, et Abigaïl de Carmel, qui avait été femme de Nabal.
\VS{3}David fit monter aussi les hommes qui étaient avec lui, chacun avec sa famille ; et ils habitèrent dans les villes d'Hébron.
\VS{4}Les hommes de Juda vinrent, et là ils oignirent David pour roi sur la maison de Juda. On fit un rapport à David en disant : Les gens de Jabès en Galaad ont enseveli Saül.
\VS{5}Alors David envoya des messagers vers les gens de Jabès en Galaad, pour leur dire : Soyez bénis de Yahweh, puisque vous avez montré de la bienveillance envers Saül, votre seigneur, et que vous l'avez enseveli.
\VS{6}Maintenant donc que Yahweh use envers vous de bonté et de fidélité. Moi aussi je vous ferai du bien, parce que vous avez agi de la sorte.
\VS{7}Que maintenant vos mains se fortifient, et soyez des vaillants hommes ; car Saül, votre maître, est mort, et la maison de Juda m'a oint pour être roi sur elle.
\TextTitle{Isch-Boscheth établi roi d'Israël}
\VS{8}Cependant Abner, fils de Ner, chef de l'armée de Saül, prit Isch-Boscheth, fils de Saül, et le fit passer à Mahanaïm.
\VS{9}Il l'établit roi sur Galaad, sur les Gueschuriens, sur Jizreel, sur Ephraïm, sur Benjamin, et sur tout Israël.
\VS{10}Isch-Boscheth, fils de Saül, était âgé de quarante ans quand il commença à régner sur Israël, et il régna deux ans. Il n'y eut que la maison de Juda qui suivit David.
\VS{11}Le nombre de jours pendant lesquels David régna à Hébron sur la maison de Juda fut de sept ans et six mois.
\TextTitle{Guerre entre Juda et Israël}
\VS{12}Abner, fils de Ner, et les gens d'Isch-Boscheth, fils de Saül, sortirent de Mahanaïm pour marcher vers Gabaon.
\VS{13}Joab, fils de Tseruja, et les gens de David, sortirent aussi. Ils se rencontrèrent ensemble près de l'étang de Gabaon, et les uns se tinrent d'un côté de l'étang, et les autres du côté opposé de l'étang.
\VS{14}Alors Abner dit à Joab : Que ces jeunes gens se lèvent maintenant, et qu'ils se battent devant nous ! Et Joab répondit : Qu'ils se lèvent !
\VS{15}Ils se levèrent donc, et s'avancèrent en nombre égal, douze pour Benjamin et pour Isch-Boscheth, fils de Saül, et douze des gens de David.
\VS{16}Alors, chacun saisissant son adversaire par la tête, lui enfonça son épée dans le flanc, et ils tombèrent tous ensemble. Et l'on donna à ce lieu, qui est près de Gabaon, le nom de Helkath-Hatsurim.
\VS{17}Il y eut ce jour-là un combat très rude, dans lequel Abner et les hommes d'Israël furent battus par les gens de David.
\VS{18}Les trois fils de Tseruja, Joab, Abischaï et Asaël étaient là. Asaël avait les pieds légers comme une gazelle des champs.
\VS{19}Asaël poursuivit Abner, sans se détourner de lui ni à droite ni à gauche.
\VS{20}Abner regarda derrière lui, et dit : Est-ce toi, Asaël ? Et il répondit : C'est moi.
\VS{21}Abner lui dit : Détourne-toi à droite ou à gauche ; saisis-toi de l'un de ces jeunes gens, et prends sa dépouille. Mais Asaël ne voulut point se détourner de lui.
\VS{22}Et Abner dit encore à Asaël : Détourne-toi de moi ; pourquoi te frapperais-je et t'abattrais-je à terre ? Comment ensuite lèverais-je le visage devant ton frère Joab ?
\VS{23}Mais Asaël refusa de se détourner. Abner le frappa au ventre avec l'extrémité inférieure de sa lance, qui sortit par-derrière. Il tomba là, raide mort sur place. Tous ceux qui arrivaient au lieu où Asaël était tombé mort, s'y arrêtaient.
\VS{24}Joab et Abischaï poursuivirent Abner, et le soleil se couchait quand ils arrivèrent au coteau d'Amma, qui est en face de Guiach, sur le chemin du désert de Gabaon.
\VS{25}Les fils de Benjamin s'assemblèrent auprès d'Abner et formèrent un corps de troupe, et ils s'arrêtèrent sur le sommet d'une colline.
\VS{26}Alors Abner appela Joab, et dit : L'épée dévorera-t-elle sans cesse ? Ne sais-tu pas qu'il y aura de l'amertume à la fin ? Jusqu'à quand tarderas-tu à dire au peuple qu'il cesse de poursuivre ses frères ?
\VS{27}Joab répondit : Dieu est vivant ! Si tu n'avais parlé ainsi, le peuple n'aurait pas cessé avant le matin de poursuivre ses frères.
\VS{28}Joab sonna du shofar, et tout le peuple s'arrêta ; ils ne poursuivirent plus Israël, et ils ne continuèrent plus à se battre.
\VS{29}Ainsi, Abner et ses gens marchèrent toute la nuit dans la plaine ; ils passèrent le Jourdain, traversèrent tout le Bithron, et arrivèrent à Mahanaïm.
\VS{30}Joab aussi revint de la poursuite d'Abner, et rassembla tout le peuple ; il manquait dix-neuf hommes des gens de David, et Asaël.
\VS{31}Mais les gens de David avaient frappé à mort trois cent soixante hommes de Benjamin, et des gens d'Abner.
\VS{32}Ils emportèrent Asaël, et l'ensevelirent dans le sépulcre de son père à Bethléhem. Joab et ses gens marchèrent toute la nuit et arrivèrent à Hébron au point du jour.
\Chap{3}
\TextTitle{L'autorité de David s'accroît\FTNTT{1 Ch. 3:1-4.}}
\VerseOne{}Or il y eut une longue guerre entre la maison de Saül et la maison de David. David devenait de plus en plus fort, et la maison de Saül allait en s'affaiblissant.
\VS{2}Il naquit à David des fils à Hébron. Son premier-né fut Amnon, d'Achinoam de Jizreel ;
\VS{3}le second, Kileab, d'Abigaïl de Carmel, femme de Nabal ; le troisième, Absalom, fils de Maaca, fille de Talmaï, roi de Gueschur ;
\VS{4}le quatrième, Adonija, fils de Haggith ; le cinquième, Schephathia, fils d'Abithal ;
\VS{5}et le sixième, Jithream, d'Egla, femme de David. Ce sont là ceux qui naquirent à David à Hébron.
\TextTitle{Abner fait alliance avec David}
\VS{6}Et il arriva que pendant la guerre entre la maison de Saül et la maison de David, Abner tint ferme pour la maison de Saül.
\VS{7}Or Saül avait eu une concubine, nommée Ritspa, fille d'Ajja. Et Isch-Boscheth dit à Abner : Pourquoi es-tu venu vers la concubine de mon père ?
\VS{8}Abner fut très irrité à cause du discours d'Isch-Boscheth, et il lui dit : Suis-je une tête de chien, au service de Juda ? Je fais aujourd'hui preuve de bienveillance envers la maison de Saül, ton père, envers ses frères et ses amis, je ne t'ai pas livré entre les mains de David, et c'est aujourd'hui que tu me reproches une faute avec cette femme ?
\VS{9}Que Dieu punisse sévèrement Abner, si je n'agis pas avec David selon ce que Yahweh a juré à David,
\VS{10}en disant qu'il ferait passer la royauté de la maison de Saül à la sienne, et qu'il établirait le trône de David sur Israël et sur Juda depuis Dan jusqu'à Beer-Schéba.
\VS{11}Isch-Boscheth n'osa pas répondre un seul mot à Abner, parce qu'il le craignait.
\VS{12}Abner envoya des messagers à David pour lui dire de sa part : A qui est le pays ? Fais alliance avec moi, et voici, ma main sera avec toi, pour tourner vers toi tout Israël.
\VS{13}David répondit : Je le veux bien ! Je ferai alliance avec toi ; je te demande seulement une chose, c'est que tu ne voies point ma face, à moins que tu n'amènes d'abord Mical, fille de Saül, quand tu viendras me voir.
\VS{14}Et David envoya des messagers à Isch-Boscheth, fils de Saül, pour lui dire : Rends-moi ma femme Mical, que j'ai épousée pour cent prépuces des Philistins.
\VS{15}Isch-Boscheth envoya et l'ôta à son mari Palthiel, fils de Laïsch.
\VS{16}Et son mari la suivit, marchant et pleurant continuellement après elle jusqu'à Bachurim. Alors Abner lui dit : Va, retourne-t'en ! Et il s'en retourna.
\VS{17}Abner parla aux anciens d'Israël, et leur dit : Vous désiriez autrefois avoir David pour roi ;
\VS{18}établissez-le maintenant, car Yahweh a parlé de David et a dit : C'est par David, mon serviteur, que je délivrerai mon peuple d'Israël de la main des Philistins et de la main de tous ses ennemis.
\VS{19}Abner parla aussi aux oreilles de ceux de Benjamin, puis il alla faire entendre expressément à David, qui était à Hébron, ce qui semblait bon aux yeux d'Israël et aux yeux de toute la maison de Benjamin.
\VS{20}Abner vint donc vers David à Hébron, accompagné de vingt hommes ; et David fit un festin à Abner et aux hommes qui étaient avec lui.
\VS{21}Abner dit à David : Je me lèverai, et je partirai pour rassembler tout Israël auprès du roi, mon seigneur ; ils feront alliance avec toi, et tu régneras selon le désir de ton âme. David renvoya Abner, qui s'en alla en paix.
\VS{22}Voici, les gens de David et Joab revinrent d'une excursion, et amenèrent avec eux un grand butin. Abner n'était plus avec David à Hébron, car David l'avait renvoyé, et il s'en était allé en paix.
\VS{23}Lorsque Joab et toute l'armée qui était avec lui revinrent, on fit ce rapport à Joab en ces mots : Abner, fils de Ner, est venu auprès du roi, qui l'a renvoyé, et il s'en est allé en paix.
\VS{24}Joab vint vers le roi, et dit : Qu'as-tu fait ? Voici, Abner est venu vers toi ; pourquoi l'as-tu ainsi renvoyé, en sorte qu'il s'en est allé ?
\VS{25}Tu connais Abner, fils de Ner ! C'est pour te tromper qu'il est venu, pour épier tes démarches, tes allées et venues, et pour savoir tout ce que tu fais.
\VS{26}Puis Joab, après avoir quitté David, envoya sur les traces d'Abner des messagers, qui le ramenèrent de la fosse de Sira, sans que David n'en sache rien.
\TextTitle{Mort d'Abner}
\VS{27}Lorsque Abner revint à Hébron, Joab le tira à l'écart au milieu de la porte, comme pour lui parler en secret ; et là il le frappa à la cinquième côte ; et ainsi Abner mourut à cause du sang d'Asaël, frère de Joab.
\VS{28}David apprit ce qui était arrivé et dit : Je suis à jamais innocent, mon royaume et moi, devant Yahweh, du sang d'Abner, fils de Ner.
\VS{29}Que ce sang retombe sur la tête de Joab, et sur toute la maison de son père ! Que soit retranchée la maison de Joab, qu'il y ait toujours un homme qui soit atteint d'un flux ou de la lèpre, ou qui s'appuie sur un bâton, ou qui tombe par l'épée, ou qui manque de pain !
\VS{30}Ainsi Joab et Abischaï, son frère, tuèrent Abner, parce qu'il avait tué Asaël, leur frère, à Gabaon, dans la bataille.
\VS{31}David dit à Joab et à tout le peuple qui était avec lui : Déchirez vos vêtements, ceignez-vous de sacs, et menez le deuil en marchant devant Abner ! Et le roi David marcha derrière le cercueil.
\VS{32}On ensevelit Abner à Hébron. Le roi éleva la voix et pleura sur la tombe d'Abner, et tout le peuple pleura.
\VS{33}Le roi fit une complainte sur Abner, et dit : Abner devait-il mourir comme meurt un insensé ?
\VS{34}Tes mains n'étaient pas liées, et tes pieds n'étaient pas mis dans des chaînes ! Tu es tombé comme on tombe devant les méchants. Et tout le peuple recommença à pleurer sur Abner.
\VS{35}Puis tout le peuple vint pour faire prendre quelque nourriture à David, pendant qu'il était encore jour ; mais David jura, en disant : Que Dieu me punisse sévèrement, si je goûte du pain ou quelque chose d'autre avant le coucher du soleil !
\VS{36}Tout le peuple l'entendit, et l'approuva, et tout le peuple trouva bon tout ce qu'avait fait le roi.
\VS{37}En ce jour, tout le peuple et tout Israël surent que ce n'était pas par ordre du roi qu'Abner, fils de Ner, avait été tué.
\VS{38}Le roi dit à ses serviteurs : Ne savez-vous pas qu'un chef, un grand homme, est tombé aujourd'hui en Israël ?
\VS{39}Je suis encore faible aujourd'hui, bien que j'aie été oint roi ; et ces gens, les fils de Tseruja, sont trop puissants pour moi. Que Yahweh rende à celui qui fait le mal selon sa méchanceté !
\Chap{4}
\TextTitle{Mort d'Ish-Boscheth}
\VerseOne{}Quand le fils de Saül apprit qu'Abner était mort à Hébron, ses mains restèrent sans force, et tout Israël fut dans l'épouvante.
\VS{2}Le fils de Saül avait deux chefs de bandes, dont l'un s'appelait Baana et l'autre Récab ; ils étaient fils de Rimmon de Beéroth, d'entre les fils de Benjamin. Car Beéroth était regardée comme appartenant à Benjamin,
\VS{3}et les Beérothiens s'étaient enfuis à Guitthaïm, où ils y ont habité jusqu'à ce jour.
\VS{4}Jonathan, fils de Saül, avait un fils perclus des pieds ; il était âgé de cinq ans lorsque la nouvelle de la mort de Saül et de Jonathan arriva de Jizreel ; sa nourrice le prit et s'enfuit, et comme elle se hâtait de fuir, il tomba et devint boiteux ; son nom était Mephiboscheth.
\VS{5}Les fils de Rimmon de Beéroth, Récab et Baana, se rendirent pendant la chaleur du jour à la maison d'Isch-Boscheth, qui était couché pour son repos du midi.
\VS{6}Ils pénétrèrent jusqu'au milieu de la maison, comme pour y prendre du froment, et ils le frappèrent à la cinquième côte ; puis Récab et Baana, son frère, se sauvèrent.
\VS{7}Ils entrèrent donc dans la maison lorsqu'Isch-Boscheth était couché sur son lit dans la chambre à coucher où il dormait, ils le frappèrent et le tuèrent, puis ils lui coupèrent la tête. Ils prirent sa tête, et ils marchèrent toute la nuit au travers de la plaine.
\VS{8}Ils apportèrent la tête d'Isch-Boscheth à David dans Hébron, et ils dirent au roi : Voici la tête d'Isch-Boscheth, fils de Saül, ton ennemi, qui en voulait à ta vie ; Yahweh venge aujourd'hui le roi mon seigneur de Saül et de sa race.
\VS{9}Mais David répondit à Récab et à Baana, son frère, fils de Rimmon de Beéroth, et leur dit : Yahweh, qui a délivré mon âme de toute angoisse est vivant !
\VS{10}J'ai saisi celui qui est venu m'annoncer et me dire : Voilà, Saül est mort, et qui pensait m'apprendre de bonnes nouvelles, je l'ai fait saisir et tuer à Tsiklag, pour lui donner le salaire de ses bonnes nouvelles ;
\VS{11}combien plus, quand des méchants ont tué un homme juste dans sa maison et sur sa couche, ne redemanderai-je pas maintenant son sang de vos mains et ne vous exterminerai-je pas de la terre ?
\VS{12}David ordonna à ses gens de les tuer ; ils leur coupèrent les mains et les pieds, et les pendirent près de l'étang d'Hébron. Ils prirent la tête d'Isch-Boscheth, et l'ensevelirent dans le sépulcre d'Abner à Hébron.
\Chap{5}
\TextTitle{David oint roi sur tout Israël\FTNTT{1 Ch. 11:1-3.}}
\VerseOne{}Alors toutes les tribus d'Israël vinrent auprès de David, à Hébron, et dirent : Voici, nous sommes tes os et ta chair.
\VS{2}Autrefois déjà, quand Saül était roi sur nous, c'est toi qui conduisais et qui ramenais Israël. Yahweh t'a dit : Tu paîtras mon peuple d'Israël, et tu seras le chef d'Israël.
\VS{3}Tous les anciens d'Israël vinrent donc vers le roi à Hébron, et le roi David fit alliance avec eux à Hébron, devant Yahweh. Ils oignirent David pour roi sur Israël.
\VS{4}David était âgé de trente ans lorsqu'il commença à régner ; il régna quarante ans.
\VS{5}Il régna sur Juda à Hébron sept ans et six mois, puis il régna trente-trois ans à Jérusalem sur tout Israël et Juda.
\TextTitle{Jérusalem, capitale de tout Israël\FTNTT{1 Ch. 11:4-9.}}
\VS{6}Le roi marcha avec ses gens sur Jérusalem contre les Jébusiens qui habitaient ce pays. Ils dirent à David : Tu n'entreras point ici, car les aveugles mêmes et les boiteux te repousseront ! Ce qui voulait dire : David n'entrera point ici.
\VS{7}Mais David s'empara de la forteresse de Sion : C'est la cité de David.
\VS{8}David avait dit en ce jour-là : Quiconque battra les Jébusiens et atteindra le canal, ces aveugles et ces boiteux, qui sont haïs de l'âme de David, sera récompensé… C'est pourquoi l'on dit : Aucun aveugle ni boiteux n'entrera dans cette maison.
\VS{9}Et David habita dans la forteresse, et l'appela la cité de David. Il bâtit tout autour, depuis Millo jusqu'au-dedans.
\VS{10}David devenait de plus en plus grand, et Yahweh, le Dieu des armées, était avec lui.
\TextTitle{Yahweh affermit le règne de David}
\VS{11}Hiram, roi de Tyr, envoya des messagers à David, du bois de cèdre, des charpentiers et des tailleurs de pierres à bâtir, et ils bâtirent la maison de David.
\VS{12}David reconnut que Yahweh l'affermissait comme roi sur Israël, et qu'il élevait son royaume à cause de son peuple d'Israël.
\TextTitle{Fils de David nés à Jérusalem\FTNTT{2 S. 3:2-5 ; 1 Ch. 3:1-4.}}
\VS{13}David prit encore des concubines et des femmes de Jérusalem, après qu'il fut venu d'Hébron, et il lui naquit encore des fils et des filles.
\VS{14}Voici les noms de ceux qui lui naquirent à Jérusalem : Schammua, Schobad, Nathan, Salomon,
\VS{15}Jibhar, Elischua, Népheg, Japhia,
\VS{16}Elischama, Eliada et Eliphéleth.
\TextTitle{Yahweh livre les Philistins à David\FTNTT{2 S. 23:13-17 ; 1 Ch. 14:8-17 ; 11:15-19 ; 12:8-15.}}
\VS{17}Or quand les Philistins apprirent qu'on avait oint David pour roi sur Israël, ils montèrent tous pour chercher David. Et David l'ayant appris, il descendit vers la forteresse.
\VS{18}Les Philistins arrivèrent et se répandirent dans la vallée des Rephaïm.
\VS{19}Alors David consulta Yahweh, en disant : Monterai-je contre les Philistins ? Les livreras-tu entre mes mains ? Et Yahweh parla à David : Monte, car certainement je livrerai les Philistins entre tes mains.
\VS{20}Alors David vint à Baal-Peratsim, où il les battit. Puis il dit : Yahweh a dispersé mes ennemis devant moi, comme des eaux qui s'écoulent. C'est pourquoi il nomma ce lieu-là Baal-Peratsim.
\VS{21}Ils laissèrent là leurs faux dieux que David et ses gens emportèrent.
\VS{22}Les Philistins montèrent encore une autre fois, et se répandirent dans la vallée des Rephaïm.
\VS{23}David consulta Yahweh. Et Yahweh dit : Tu ne monteras pas ; contourne-les par-derrière, et tu les atteindras vis-à-vis des mûriers.
\VS{24}Quand tu entendras un bruit comme des gens qui marchent au sommet des mûriers, alors hâte-toi, car c'est Yahweh qui sort devant toi pour battre l'armée des Philistins.
\VS{25}David fit ce que Yahweh lui avait ordonné, et il battit les Philistins depuis Guéba jusqu'à Guézer.
\Chap{6}
\TextTitle{Désobéissance dans le transport de l'arche\FTNTT{1 Ch. 13:1-14.}}
\VerseOne{}David rassembla encore toute l'élite d'Israël, au nombre de trente mille hommes.
\VS{2}Puis David se leva, ainsi que tout le peuple qui était avec lui, et se mit en marche de Baalé-Juda, pour faire monter de là l'arche de Dieu, qui est appelée du Nom, du Nom de Yahweh des armées, qui siège entre les chérubins.
\VS{3}Ils mirent l'arche\FTNT{L'arche ne devait être portée que par les Lévites. Les meilleures intentions pour le service de Yahweh ne suffisent pas pour que le Seigneur nous agrée. Nous devons nous conformer à la Parole de Dieu (1 R. 18:36-39).} de Dieu sur un char tout neuf, et l'emmenèrent de la maison d'Abinadab qui était sur la colline ; Uzza et Achjo, fils d'Abinadab, conduisaient le char neuf.
\VS{4}Ils l'emportèrent donc de la maison d'Abinadab sur la colline ; et Achjo allait devant l'arche.
\VS{5}David et toute la maison d'Israël jouaient devant Yahweh de toutes sortes d'instruments faits de bois de cyprès, de harpes, de luths, de tambourins, de sistres et de cymbales.
\VS{6}Quand ils furent arrivés à l'aire de Nacon, Uzza étendit la main vers l'arche de Dieu et la saisit parce que les bœufs la faisaient pencher.
\VS{7}La colère de Yahweh s'enflamma contre Uzza et Dieu le frappa là à cause de sa faute. Il mourut là, près de l'arche de Dieu.
\VS{8}David fut irrité de ce que Yahweh avait fait une brèche en la personne d'Uzza. C'est pourquoi on a appelé ce lieu jusqu'à ce jour Pérets-Uzza.
\VS{9}David eut peur de Yahweh en ce jour-là, et il dit : Comment l'arche de Yahweh entrerait-elle chez moi ?
\VS{10}David ne voulut pas déposer l'arche de Yahweh chez lui dans la cité de David, mais il la fit conduire dans la maison d'Obed-Edom de Gath.
\VS{11}L'arche de Yahweh resta trois mois dans la maison d'Obed-Edom de Gath, et Yahweh bénit Obed-Edom et toute sa maison.
\TextTitle{Accueil de l'arche à Jérusalem\FTNTT{1 Ch. 15:26-16:1.}}
\VS{12}Puis on vint dire au roi David : Yahweh a béni la maison d'Obed-Edom et tout ce qui lui appartient, pour l'amour de l'arche de Dieu. Alors David s'y rendit, et il fit monter l'arche de Dieu depuis la maison d'Obed-Edom jusqu'à la cité de David, au milieu des réjouissances.
\VS{13}Et il arriva que quand ceux qui portaient l'arche de Dieu eurent fait six pas, on sacrifia des taureaux et des béliers gras.
\VS{14}David dansait de toute sa force devant Yahweh, et il était ceint d'un éphod de lin.
\VS{15}Ainsi, David et toute la maison d'Israël firent monter l'arche de Yahweh avec des cris de joie et au son du shofar.
\VS{16}Comme l'arche de Yahweh entrait dans la cité de David, Mical, fille de Saül, regardait par la fenêtre, et voyant le roi David sauter et danser devant Yahweh, elle le méprisa en son cœur.
\VS{17}Ils amenèrent l'arche de Yahweh, et la posèrent au milieu de la tente que David avait dressée pour elle ; et David offrit des holocaustes et des sacrifices d'offrande de paix\FTNT{Voir commentaire en Lé. 3:1.} devant Yahweh.
\VS{18}Quand David eut achevé d'offrir des holocaustes et des sacrifices d'offrande de paix, il bénit le peuple au Nom de Yahweh des armées.
\VS{19}Et il partagea à tout le peuple, à toute la multitude d'Israël, tant aux hommes qu'aux femmes, à chacun un pain, une portion de viande, un gâteau de raisins, et une ration de vin. Puis tout le peuple s'en alla, chacun dans sa maison.
\VS{20}David s'en retourna pour bénir aussi sa maison, et Mical, fille de Saül, sortit à sa rencontre. Elle dit : Quel honneur s'est fait aujourd'hui le roi d'Israël, en se découvrant aux yeux des servantes et de ses serviteurs, comme se découvrirait un homme de néant sans en avoir honte !
\VS{21}David répondit à Mical : C'est devant Yahweh, qui m'a choisi plutôt que ton père et toute sa maison pour m'établir chef sur le peuple de Yahweh, sur Israël, c'est devant Yahweh que je me suis réjoui.
\VS{22}Je me rendrai encore plus insignifiant que je n'ai été cette fois, et je m'estimerai encore moins à mes propres yeux ; malgré cela, je serai en honneur auprès des servantes dont tu parles.
\VS{23}Or Mical, fille de Saül, n'eut point d'enfants jusqu'au jour de sa mort.
\Chap{7}
\TextTitle{David veut construire une maison à Yahweh\FTNTT{1 Ch. 17:1-2.}}
\VerseOne{}Et il arriva, lorsque le roi fut établi dans sa maison, et que Yahweh lui eut donné du repos de tous ses ennemis qui l'entouraient,
\VS{2}qu'il dit à Nathan le prophète : Regarde maintenant ! J'habite dans une maison de cèdres, et l'arche de Dieu habite sous des tapis\FTNT{Dans la plupart des versions, on a traduit ce mot par « tente », alors que le terme hébreu est « yeriy'ah », ce qui signifie « rideau », « drap », « tapis ». Voir Ex. 26.}.
\VS{3}Alors Nathan répondit au roi : Va, fais tout ce qui est dans ton cœur, car Yahweh est avec toi.
\TextTitle{Yahweh traite alliance avec David et sa postérité\FTNTT{1 Ch. 17:3-15.}}
\VS{4}Mais il arriva cette nuit-là que la parole de Yahweh fut adressée à Nathan, en disant :
\VS{5}Va, et dis à David, mon serviteur : Ainsi parle Yahweh : Me bâtirais-tu une maison afin que j'y habite ?
\VS{6}Puisque je n'ai point habité dans une maison depuis le jour où j'ai fait monter les enfants d'Israël hors d'Egypte jusqu'à ce jour ; mais j'ai marché ça et là sous une tente et dans un tabernacle.
\VS{7}Partout où j'ai marché avec tous les enfants d'Israël, ai-je dit un seul mot à quelqu'une des tribus d'Israël à qui j'avais ordonné de paître mon peuple d'Israël, ai-je dit : Pourquoi ne me bâtissez-vous pas une maison de cèdres ?
\VS{8}Maintenant tu diras à David, mon serviteur : Ainsi parle Yahweh des armées : Je t'ai pris d'une cabane, d'auprès des brebis, afin que tu sois le conducteur de mon peuple, Israël ;
\VS{9}j'ai été avec toi partout où tu as marché, j'ai exterminé tous tes ennemis devant toi, et j'ai rendu ton nom grand, comme le nom des grands qui sont sur la terre ;
\VS{10}j'ai établi une demeure à mon peuple, à Israël, et je l'ai planté pour qu'il y habite et ne soit plus agité, pour que les méchants ne l'affligent plus comme auparavant,
\VS{11}et comme du temps où j'avais établi des juges sur mon peuple d'Israël. Je t'ai accordé du repos face à tous tes ennemis. Et Yahweh t'annonce qu'il te bâtira une maison.
\VS{12}Quand tu seras endormi avec tes pères, je susciterai après toi, ton fils, qui sera sorti de tes entrailles, et j'affermirai son règne.
\VS{13}Ce sera lui qui bâtira une maison à mon Nom, et j'affermirai pour toujours le trône de son règne\FTNT{Le royaume millénaire était promis à David et à sa postérité. Il fut proclamé par Jean-Baptiste (Mt. 3:1-12), le Messie (Mt. 4:17) et les apôtres (Mt. 10:5-7) comme étant proche. Présentement, le royaume de Dieu se manifeste par la vie sanctifiée des saints en Christ (Lu. 17:20 ; Jn. 3:1-8 ; Ro. 14:17). Il n'apparaîtra pas de manière visible avant la «moisson», c'est-à-dire le jugement des nations (Mt. 13:39-50). En effet, ce n'est qu'après cette moisson que le royaume sera installé ici-bas, lorsque le Messie rétablira la monarchie et la dynastie de David en sa propre personne. Il rassemblera alors les enfants d'Israël dispersés dans le monde entier et établira sa domination sur toute la terre pendant mille ans. Ce royaume sera remis au Père par le Messie après avoir vaincu le dernier ennemi, c'est-à-dire la mort (1 Co. 15:24-26). De ce fait, personne ne mourra pendant le millénium. Toutes les nations monteront tous les ans à Jérusalem pour adorer Yahweh et célébrer la fête des tabernacles qui sera restaurée (Za. 14). Le gouvernement théocratique en Israël sera alors restauré (Es. 1:26).}.
\VS{14}Je serai pour lui un père, et il sera pour moi un fils. S'il fait le mal, je le châtierai avec une verge d'hommes et avec des plaies des fils des hommes ;
\VS{15}mais ma grâce ne se retirera point de lui, comme je l'ai retirée de Saül, que j'ai ôté de devant toi.
\VS{16}Ainsi ta maison et ton règne seront assurés à jamais devant tes yeux, et ton trône sera pour toujours affermi.
\VS{17}Nathan rapporta à David toutes ces paroles et toute cette vision.
\TextTitle{Louange et reconnaissance de David envers Yahweh\FTNTT{1 Ch. 17:16-27.}}
\VS{18}Alors le roi David alla se présenter devant Yahweh, et dit : Qui suis-je, Seigneur Yahweh, et quelle est ma maison, que tu m'aies fait arriver au point où je suis ?
\VS{19}C'est encore peu de choses à tes yeux, ô Seigneur Yahweh ! Car tu as même parlé sur la maison de ton serviteur pour les temps éloignés. Est-ce là la manière d'agir des hommes, ô Seigneur Yahweh ?
\VS{20}Et que pourrait dire de plus David ? Car, Seigneur Yahweh, tu connais ton serviteur !
\VS{21}Tu as fait toutes ces grandes choses pour l'amour de ta parole, et selon ton cœur, pour les révéler à ton serviteur.
\VS{22}C'est pourquoi tu t'es montré grand, ô Yahweh Dieu ! Car nul n'est semblable à toi, et il n'y a point d'autre Dieu que toi, d'après tout ce que nous avons entendu de nos oreilles.
\VS{23}Et qui est comme ton peuple, comme Israël, la seule nation de la terre que Dieu est venu racheter pour en faire son peuple, y mettre son Nom et pour accomplir dans ton pays, devant ton peuple que tu t'es racheté d'Egypte, des choses grandes et terribles contre les nations et contre leurs dieux ?
\VS{24}Tu as affermi ton peuple d'Israël pour qu'il soit ton peuple pour toujours ; et toi, Yahweh, tu es devenu son Dieu.
\VS{25}Maintenant donc, ô Yahweh Dieu, confirme pour toujours la parole que tu as prononcée sur ton serviteur et sur sa maison, et agis selon ta parole.
\VS{26}Que ton Nom soit à jamais glorifié, et que l'on dise : Yahweh des armées est le Dieu d'Israël ! Et que la maison de David, ton serviteur, demeure stable devant toi !
\VS{27}Car toi, Yahweh des armées, Dieu d'Israël, tu as révélé ces choses à l'oreille de ton serviteur, en disant : Je te bâtirai une maison ! C'est pourquoi ton serviteur a pris courage pour t'adresser cette prière.
\VS{28}Maintenant, Seigneur Yahweh, tu es Dieu, tes paroles sont vérité, et tu as promis cette grâce à ton serviteur.
\VS{29}Veuille donc bénir la maison de ton serviteur, afin qu'elle soit éternellement devant toi ! Car c'est toi, Seigneur Yahweh, qui a parlé, et par ta bénédiction la maison de ton serviteur sera comblée de bénédictions éternellement.
\Chap{8}
\TextTitle{Yahweh donne à David la victoire sur ses ennemis\FTNTT{1 Ch. 18:1-17.}}
\VerseOne{}Après cela, il arriva que David battit les Philistins et les humilia, et il prit Métheg-Amma de la main des Philistins.
\VS{2}Il battit aussi les Moabites, et les mesura au cordeau, en les faisant coucher par terre ; il en mesura deux cordeaux pour les faire mourir, et un plein cordeau pour leur laisser la vie. Et les Moabites furent assujettis à David, et lui payèrent un tribut.
\VS{3}David battit aussi Hadadézer, fils de Rehob, roi de Tsoba, lorsqu'il alla rétablir sa domination sur le fleuve de l'Euphrate.
\VS{4}David lui prit mille sept cents cavaliers, et vingt mille hommes de pied ; il coupa les jarrets aux chevaux de tous les chars, et ne conserva que cent attelages.
\VS{5}Les Syriens de Damas vinrent au secours d'Hadadézer, roi de Tsoba, et David battit vingt-deux mille Syriens.
\VS{6}David mit des garnisons dans la Syrie de Damas. Et les Syriens furent assujettis à David, et lui payèrent un tribut. Yahweh protégeait David partout où il allait.
\VS{7}Et David prit les boucliers d'or qui étaient aux serviteurs d'Hadadézer, et les apporta à Jérusalem.
\VS{8}Le roi David emporta aussi une grande quantité d'airain de Béthach, et de Bérothaï, villes d'Hadadézer.
\VS{9}Thoï, roi de Hamath, apprit que David avait battu toute l'armée d'Hadadézer,
\VS{10}et il envoya Joram, son fils, vers le roi David, pour le saluer et pour le féliciter d'avoir fait la guerre contre Hadadézer et de l'avoir battu. Car Hadadézer était continuellement en guerre avec Thoï. Joram apporta des vases d'argent, des vases d'or, et des vases d'airain.
\VS{11}Le roi David les consacra à Yahweh, avec l'argent et l'or qu'il avait déjà consacrés du butin de toutes les nations qu'il s'était assujetties,
\VS{12}de la Syrie, de Moab, des fils d'Ammon, des Philistins, d'Amalek, et du butin d'Hadadézer, fils de Rehob, roi de Tsoba.
\VS{13}Au retour de la défaite des Syriens, David se fit encore un nom, en battant dans la vallée du sel dix-huit mille Edomites.
\VS{14}Il mit des garnisons dans Edom, il mit des garnisons dans tout Edom. Et tout Edom fut assujetti à David. Yahweh protégeait David partout où il allait.
\VS{15}Ainsi David régna sur tout Israël, et il faisait droit et justice à tout son peuple.
\VS{16}Joab, fils de Tseruja, commandait l'armée ; Josaphat, fils d'Achilud, était archiviste ;
\VS{17}Tsadok, fils d'Achithub, et Achimélec, fils d'Abiathar, étaient sacrificateurs ; Seraja était secrétaire ;
\VS{18}Benaja, fils de Jehojada, était chef des Kéréthiens et des Péléthiens ; et les fils de David étaient ministres d'Etat.
\Chap{9}
\TextTitle{Mephiboscheth à la table de David}
\VerseOne{}Alors David dit : Ne reste-t-il donc personne de la maison de Saül, afin que je lui fasse du bien pour l'amour de Jonathan ?
\VS{2}Il y avait dans la maison de Saül un serviteur nommé Tsiba, que l'on fit venir auprès de David. Le roi lui dit : Es-tu Tsiba ? Et il répondit : Je suis ton serviteur !
\VS{3}Le roi dit : N'y a-t-il plus personne de la maison de Saül, pour que j'use envers lui de la bonté de Dieu ? Tsiba répondit au roi : Il y a encore un des fils de Jonathan, qui est perclus des pieds.
\VS{4}Le roi lui dit : Où est-il ? Et Tsiba répondit au roi : Il est dans la maison de Makir, fils d'Ammiel, à Lodebar.
\VS{5}Alors le roi David l'envoya chercher dans la maison de Makir, fils d'Ammiel, à Lodebar.
\VS{6}Quand Mephiboscheth, fils de Jonathan, fils de Saül, vint auprès de David, il tomba sur sa face et se prosterna. David dit : Mephiboscheth ! Et il répondit : Voici ton serviteur.
\VS{7}David lui dit : Ne crains point, car certainement je te ferai du bien pour l'amour de Jonathan, ton père. Je te restituerai toutes les terres de Saül, ton père, et tu mangeras toujours du pain à ma table.
\VS{8}Il se prosterna, et dit : Qui suis-je, moi ton serviteur, pour que tu regardes un chien mort tel que moi ?
\VS{9}Le roi appela Tsiba, serviteur de Saül, et lui dit : Je donne au fils de ton maître tout ce qui appartenait à Saül et à toute sa maison.
\VS{10}Tu cultiveras pour lui ces terres, toi, tes fils, et tes serviteurs, et tu en recueilleras les fruits, afin que le fils de ton maître ait du pain à manger ; et Mephiboscheth, fils de ton maître, mangera toujours du pain à ma table. Or Tsiba avait quinze fils et vingt serviteurs.
\VS{11}Tsiba dit au roi : Ton serviteur fera tout ce que le roi, mon seigneur, ordonne à son serviteur. Et Mephiboscheth mangea à la table de David comme l'un des fils du roi.
\VS{12}Mephiboscheth avait un jeune fils, nommé Mica, et tous ceux qui demeuraient dans la maison de Tsiba étaient serviteurs de Mephiboscheth.
\VS{13}Mephiboscheth habitait à Jérusalem parce qu'il mangeait toujours à la table du roi. Il était boiteux des deux pieds.
\Chap{10}
\TextTitle{Double bataille contre les Ammonites et les Syriens}
\VerseOne{}Or il arriva après cela que le roi des fils d'Ammon mourût, et Hanun, son fils, régna à sa place.
\VS{2}Et David dit : J'userai de bonté envers Hanun, fils de Nachasch, comme son père en a usé envers moi. Ainsi David lui envoya ses serviteurs pour le consoler au sujet de son père. Lorsque les serviteurs de David arrivèrent dans le pays des fils d'Ammon,
\VS{3}les chefs des fils d'Ammon dirent à Hanun, leur maître : Penses-tu que ce soit pour honorer ton père que David t'envoie des consolateurs ? N'est-ce pas pour reconnaître exactement la ville et pour l'épier, afin de la détruire, que David envoie ses serviteurs auprès de toi ?
\VS{4}Alors Hanun saisit les serviteurs de David, et fit raser la moitié de leur barbe, et couper la moitié de leurs habits jusqu'aux hanches. Puis il les renvoya.
\VS{5}David en fut informé et envoya des gens à leur rencontre, car ces hommes étaient accablés de honte ; et le roi leur fit dire : Restez à Jéricho jusqu'à ce que votre barbe ait repoussé, et revenez ensuite.
\VS{6}Les fils d'Ammon, voyant qu'ils s'étaient rendus odieux à David, firent enrôler à leur solde vingt mille hommes de pied chez les Syriens de Beth-Rehob, et chez les Syriens de Tsoba, mille hommes chez le roi de Maaca, et douze mille hommes chez les gens de Tob.
\VS{7}David l'ayant appris, envoya Joab et toute l'armée, les hommes les plus vaillants.
\VS{8}Les fils d'Ammon sortirent et se rangèrent en bataille à l'entrée de la porte ; les Syriens de Tsoba de Rehob, et les hommes de Tob et de Maaca étaient à part dans la campagne.
\VS{9}Joab, voyant que leur armée était tournée contre lui devant et derrière, choisit alors des hommes d'élite parmi tous ceux d'Israël, et les rangea contre les Syriens ;
\VS{10}et il donna le commandement du reste du peuple à Abischaï, son frère, pour le ranger en bataille contre les fils d'Ammon.
\VS{11}Il dit : Si les Syriens sont plus forts que moi, tu viendras à mon secours ; et si les fils d'Ammon sont plus forts que toi, j'irai te secourir.
\VS{12}Sois vaillant, et portons-nous vaillamment pour notre peuple et pour les villes de notre Dieu, et que Yahweh fasse ce qu'il lui semblera bon !
\VS{13}Alors Joab et le peuple qui était avec lui s'approchèrent pour livrer bataille aux Syriens, et ils s'enfuirent devant lui.
\VS{14}Quand les fils d'Ammon virent que les Syriens avaient pris la fuite, ils s'enfuirent aussi devant Abischaï et rentrèrent dans la ville. Joab s'éloigna des fils d'Ammon et revint à Jérusalem.
\VS{15}Les Syriens, voyant qu'ils avaient été battus par Israël, se rassemblèrent.
\VS{16}Hadarézer envoya chercher les Syriens qui étaient de l'autre côté du fleuve ; et ils arrivèrent à Hélam, et Schobac, chef de l'armée d'Hadarézer, les conduisait.
\VS{17}Cela fut rapporté à David, qui assembla tout Israël, passa le Jourdain, et vint à Hélam. Les Syriens se rangèrent en bataille contre David, et combattirent contre lui.
\VS{18}Mais les Syriens s'enfuirent devant Israël. Et David défit sept cents chars des Syriens et quarante mille cavaliers ; il frappa aussi Schobac, le chef de leur armée, qui mourut sur place.
\VS{19}Tous les rois soumis à Hadarézer, se voyant battus par Israël, firent la paix avec Israël et lui furent assujettis. Et les Syriens craignirent désormais de secourir les fils d'Ammon.
\Chap{11}
\TextTitle{Péché de David avec Bath-Schéba}
\VerseOne{}Et il arriva, l'année suivante, au temps où les rois partaient en guerre, que David envoya Joab, avec ses serviteurs et tout Israël, pour détruire les fils d'Ammon et assiéger Rabba. Mais David resta à Jérusalem\FTNT{Au lieu d'aller en guerre et de diriger les troupes, David resta à Jérusalem. Cette négligence l'a conduit à la convoitise, à l'adultère et au meurtre d'Urie. La distraction peut conduire à la mort. Il y a un temps pour toutes choses (Ec. 3).}.
\VS{2}Et il arriva, sur le soir, que David se leva de sa couche ; et comme il se promenait sur le toit de la maison royale, il aperçut de là une femme qui se baignait, et cette femme était très belle de figure.
\VS{3}David envoya demander qui était cette femme, et on lui dit : N'est-ce pas Bath-Schéba, fille d'Eliam, femme d'Urie, le Héthien ?
\VS{4}Et David envoya des messagers pour la chercher. Elle vint vers lui, et il coucha avec elle. Après s'être purifiée de sa souillure, elle retourna dans sa maison.
\VS{5}Cette femme devint enceinte, et elle fit dire à David : Je suis enceinte.
\VS{6}Alors David envoya dire à Joab : Envoie-moi Urie, le Héthien. Et Joab envoya Urie à David.
\VS{7}Urie se rendit auprès de David, qui l'interrogea sur l'état de Joab, sur l'état du peuple, et sur l'état de la guerre.
\VS{8}Puis David dit à Urie : Descends dans ta maison, et lave tes pieds. Urie sortit de la maison du roi, et on fit porter après lui un présent royal.
\VS{9}Mais Urie se coucha à la porte de la maison du roi, avec tous les serviteurs de son maître, et il ne descendit point dans sa maison.
\VS{10}On le rapporta à David, et on lui dit : Urie n'est pas descendu dans sa maison. David dit à Urie : N'arrives-tu pas de voyage ? Pourquoi n'es-tu pas descendu dans ta maison ?
\VS{11}Urie répondit à David : L'arche et Israël et Juda habitent sous des tentes, mon seigneur Joab et les serviteurs de mon seigneur campent aux champs, et moi j'entrerais dans ma maison pour manger et boire et pour coucher avec ma femme ! Tu es vivant, et ton âme est vivante, je ne ferai point une telle chose.
\VS{12}David dit à Urie : Reste ici encore aujourd'hui, et demain je te renverrai. Urie resta donc ce jour-là et le lendemain à Jérusalem.
\VS{13}David l'invita à manger et à boire en sa présence, et il l'enivra ; néanmoins le soir, Urie sortit pour dormir sur sa couche, avec tous les serviteurs de son maître, et il ne descendit point dans sa maison.
\VS{14}Le lendemain matin, David écrivit une lettre à Joab, et l'envoya par la main d'Urie.
\VS{15}Il écrivit en ces termes : Placez Urie à l'endroit où sera le plus fort de la bataille et éloignez-vous de lui, afin qu'il soit frappé et qu'il meure.
\VS{16}Joab, en observant la ville, plaça Urie à l'endroit qu'il savait défendu par de vaillants soldats.
\VS{17}Les hommes de la ville sortirent et combattirent contre Joab ; et quelques-uns du peuple qui étaient des serviteurs de David moururent, et Urie, le Héthien, mourut aussi.
\VS{18}Alors Joab envoya un messager à David pour lui faire savoir tout ce qui était arrivé dans ce combat.
\VS{19}Il donna cet ordre au messager : Quand tu auras achevé de raconter au roi tout ce qui est arrivé au combat,
\VS{20}peut-être se mettra-t-il en fureur et te dira : Pourquoi vous êtes-vous approchés de la ville pour combattre ? Ne savez-vous pas bien qu'on tire de dessus la muraille ?
\VS{21}Qui a tué Abimélec, fils de Jerubbéscheth ? N'est-ce pas une femme qui lança sur lui de dessus la muraille une pièce de meule de moulin, et n'en est-il pas mort à Thébets ? Pourquoi vous êtes-vous approchés de la muraille ? Alors tu lui diras : Ton serviteur Urie, le Héthien, est mort aussi.
\VS{22}Le messager partit. A son arrivée, il fit savoir à David tout ce pourquoi Joab l'avait envoyé.
\VS{23}Le messager dit à David : Ces gens ont été plus forts que nous; ils avaient fait une sortie contre nous dans les champs, mais nous les avons repoussés jusqu'à l'entrée de la porte ;
\VS{24}les archers ont tiré sur tes serviteurs du haut de la muraille, et plusieurs des serviteurs du roi ont été tués, ton serviteur Urie, le Héthien, est mort aussi.
\VS{25}David dit au messager : Tu diras ainsi à Joab : Ne sois point peiné de cette affaire, car l'épée dévore tantôt l'un, tantôt l'autre ; attaque vigoureusement la ville, et détruis-la. Et toi, encourage-le !
\VS{26}La femme d'Urie apprit qu'Urie, son mari, était mort, et elle pleura son mari.
\VS{27}Quand le deuil fut passé, David l'envoya chercher et la recueillit dans sa maison. Elle devint sa femme, et lui enfanta un fils. Ce que David avait fait était mal aux yeux de Yahweh.
\Chap{12}
\TextTitle{Le prophète Nathan envoyé pour reprendre David}
\VerseOne{}Yahweh envoya Nathan vers David. Nathan vint à lui, et lui dit : Il y avait deux hommes dans une ville, l'un riche et l'autre pauvre.
\VS{2}Le riche avait des brebis et des bœufs en très grand nombre.
\VS{3}Le pauvre n'avait rien du tout sauf une petite brebis, qu'il avait achetée ; il la nourrissait et elle grandissait chez lui avec ses enfants ; elle mangeait de son pain, buvait dans sa coupe, dormait sur son sein et elle était comme sa fille.
\VS{4}Un voyageur arriva chez l'homme riche. Ce riche a épargné ses brebis et ses bœufs, pour préparer un repas au voyageur qui était venu chez lui ; il a pris la brebis du pauvre homme, et l'a apprêtée pour l'homme qui était venu chez lui.
\VS{5}Alors la colère de David s'enflamma violemment contre cet homme, et il dit à Nathan : Yahweh est vivant ! L'homme qui a fait cela mérite la mort.
\VS{6}Parce qu'il a fait cela et qu'il n'a pas épargné cette brebis, pour une brebis il en rendra quatre.
\VS{7}Alors Nathan dit à David : Tu es cet homme-là ! Ainsi parle Yahweh, le Dieu d'Israël : Je t'ai oint pour roi sur Israël, et je t'ai délivré de la main de Saül ;
\VS{8}je t'ai même donné la maison de ton maître, et les femmes de ton maître dans ton sein, et je t'ai donné la maison d'Israël, et de Juda. Et si cela avait été peu, j'y aurais encore ajouté.
\VS{9}Pourquoi donc as-tu méprisé la parole de Yahweh, en faisant ce qui est mal à ses yeux ? Tu as frappé de l'épée Urie, le Héthien ; tu as pris sa femme pour en faire ta femme, et tu l'as tué par l'épée des fils d'Ammon.
\VS{10}Maintenant, l'épée ne s'éloignera jamais de ta maison, parce que tu m'as méprisé, et que tu as pris la femme d'Urie, le Héthien, pour en faire ta femme.
\VS{11}Ainsi parle Yahweh : Voici, je vais faire sortir de ta propre maison le malheur contre toi, et je vais prendre sous tes yeux tes propres femmes pour les donner à un homme de ta maison, qui couchera avec elles à la vue de ce soleil.
\VS{12}Car tu as agi en secret ; mais moi, je le ferai en présence de tout Israël et à la face du soleil.
\TextTitle{Repentance de David}
\VS{13}David dit à Nathan : J'ai péché contre Yahweh ! Et Nathan dit à David : Yahweh passe par-dessus ton péché, tu ne mourras point.
\VS{14}Toutefois, parce qu'en commettant cela, tu as donné l'occasion aux ennemis de Yahweh de le blasphémer, à cause de cela le fils qui t'est né mourra certainement.
\VS{15}Et Nathan retourna dans sa maison. Yahweh frappa l'enfant que la femme d'Urie avait enfanté à David, et il devint gravement malade.
\VS{16}David pria Dieu pour l'enfant, et David jeûna ; et quand il rentra, il passa la nuit couché par terre.
\VS{17}Les anciens de sa maison se levèrent et vinrent vers lui pour le faire lever de terre ; mais il ne voulut point, et il ne mangea rien avec eux.
\VS{18}Et il arriva que l'enfant mourut le septième jour. Les serviteurs de David craignaient de lui annoncer que l'enfant était mort. Car ils disaient : Voici, quand l'enfant vivait encore, nous lui avons parlé, et il n'a pas écouté notre voix ; comment donc lui dirions-nous : L'enfant est mort ? Il s'affligera bien davantage.
\VS{19}David vit que ses serviteurs parlaient à voix basse, et il comprit que l'enfant était mort. David dit à ses serviteurs : L'enfant est-il mort ? Ils répondirent : Il est mort.
\VS{20}Alors David se leva de terre. Il se lava, s'oignit, et changea de vêtements ; il alla dans la maison de Yahweh, et se prosterna. De retour chez lui, il demanda à manger ; on mit de la viande devant lui et il mangea.
\VS{21}Ses serviteurs lui dirent : Qu'est-ce que tu fais ? Tu jeûnais et pleurais pour l'amour de l'enfant lorsqu'il vivait encore ; et maintenant que l'enfant est mort, tu te lèves et tu manges !
\VS{22}Mais il répondit : Quand l'enfant vivait encore, je jeûnais et pleurais, car je disais : Qui sait si Yahweh n'aura pas pitié de moi et si l'enfant ne vivra pas ?
\VS{23}Maintenant qu'il est mort, pourquoi jeûnerais-je ? Puis-je le faire revenir ? J'irai vers lui, mais il ne reviendra pas vers moi.
\TextTitle{Naissance de Salomon}
\VS{24}David consola sa femme Bath-Schéba, et il alla auprès d'elle et coucha avec elle. Elle lui enfanta un fils qu'il nomma Salomon et qui fut aimé de Yahweh.
\VS{25}Il le remit entre les mains de Nathan, le prophète, qui lui donna le nom de Jedidja, à cause de Yahweh.
\TextTitle{Le pays et le roi de Rabba livrés à Joab et David (1 Ch. 20:1-3)}
\VS{26}Joab combattait contre Rabba, qui appartenait aux fils d'Ammon, il s'empara de la ville royale,
\VS{27}et envoya des messagers à David pour lui dire : J'ai attaqué Rabba, et j'ai pris la ville des eaux ;
\VS{28}rassemble maintenant le reste du peuple, campe contre la ville, et prends-la, de peur que je ne m'en empare et que la gloire m'en soit attribuée.
\VS{29}David rassembla tout le peuple et marcha contre Rabba ; il l'attaqua et la prit.
\VS{30}Il enleva la couronne de dessus la tête de son roi ; elle pesait un talent d'or et était garnie de pierres précieuses. On la mit sur la tête de David, qui emporta de la ville un très grand butin.
\VS{31}Il fit sortir aussi le peuple qui s'y trouvait, et il les plaça sous des scies, des herses de fer, des haches de fer et le fit passer par un fourneau où l'on cuit les briques ; il traita ainsi toutes les villes des fils d'Ammon. Puis David retourna avec tout le peuple à Jérusalem.
\Chap{13}
\TextTitle{David subit les conséquences de son péché}
\VerseOne{}Or il arriva après cela qu'Absalom, fils de David, avait une sœur qui était belle et qui se nommait Tamar ; et Amnon, fils de David, l'aima.
\TextTitle{Inceste au sein de la famille royale}
\VS{2}Et Amnon fut si tourmenté qu'il tomba malade à cause de Tamar sa sœur, car elle était vierge ; et il paraissait trop difficile à Amnon d'obtenir la moindre chose d'elle.
\VS{3}Amnon avait un ami, nommé Jonadab, fils de Schimea, frère de David, et Jonadab était un homme très rusé.
\VS{4}Il lui dit : Fils de roi, pourquoi maigris-tu ainsi de jour en jour ? Ne veux-tu pas me le dire ? Amnon lui dit : J'aime Tamar, la sœur de mon frère, Absalom.
\VS{5}Jonadab lui dit : Couche-toi dans ton lit et fais le malade. Quand ton père viendra te voir, tu lui diras : Permets à Tamar, ma sœur, de venir pour me donner à manger ; qu'elle prépare un mets sous mes yeux, afin que je le voie et que je le prenne de sa main.
\VS{6}Amnon se coucha et fit le malade. Le roi vint le voir, et Amnon dit au roi : Je te prie, que ma sœur Tamar vienne faire deux beignets sous mes yeux, et que je les mange de sa main.
\VS{7}David envoya dire à Tamar dans la maison : Va dans la maison de ton frère Amnon, et prépare-lui quelque chose d'appétissant.
\VS{8}Tamar alla dans la maison de son frère Amnon, qui était couché. Elle prit de la pâte, la pétrit, et en fit devant lui des beignets et les fit cuire.
\VS{9}Puis elle prit la poêle, et elle les versa devant lui. Mais Amnon refusa d'en manger. Il dit : Faites sortir tous ceux qui sont auprès de moi. Et tout le monde se retira.
\VS{10}Alors Amnon dit à Tamar : Apporte-moi le mets dans la chambre, et que je le mange de ta main. Tamar prit les beignets qu'elle avait faits, et les apporta à Amnon, son frère dans la chambre.
\VS{11}Comme elle les lui présentait pour qu'il en mange, il se saisit d'elle et lui dit : Viens, couche avec moi, ma sœur !
\VS{12}Elle lui répondit : Non, mon frère, ne me déshonore pas, car cela ne se fait point en Israël ; ne commets pas cette infamie.
\VS{13}Et moi, où irais-je avec mon opprobre ? Et toi, tu serais comme l'un des infâmes en Israël. Maintenant, je te prie, parle au roi, et il ne s'opposera pas à ce que je sois ta femme.
\VS{14}Mais il ne voulut pas écouter sa parole ; il fut plus fort qu'elle, lui fit violence et coucha avec elle\FTNT{Le viol et l'inceste que commit Amnon, fils de David, sur Tamar, sa demi-sœur, furent les conséquences du péché de David avec Bath-Schéba.}.
\VS{15}Après cela, Amnon eut pour elle une très grande haine, en sorte que la haine qu'il lui portait était plus grande que l'amour qu'il avait eu pour elle. Ainsi, Amnon lui dit : Lève-toi, va-t'en !
\VS{16}Elle lui répondit : Tu n'as aucune raison de me faire ce mal, que de me chasser, ce mal est plus grand que l'autre que tu m'as fait.
\VS{17}Mais il ne voulut point l'écouter, et appelant le garçon qui le servait, il dit : Qu'on chasse cette femme loin de moi, qu'on la mette dehors. Et ferme la porte après elle !
\VS{18}Elle était habillée d'une tunique de couleurs ; car les filles du roi, qui étaient encore vierges, s'habillaient ainsi. Le serviteur d'Amnon la mit dehors, et ferma la porte après elle.
\VS{19}Alors Tamar répandit de la cendre sur sa tête, et déchira sa tunique de couleurs ; elle mit la main sur sa tête, et s'en alla en poussant des cris.
\VS{20}Et son frère Absalom lui dit : Ton frère, Amnon, a-t-il été avec toi ? Maintenant, ma sœur, tais-toi, c'est ton frère ; ne prends pas cette affaire à cœur. Et Tamar, désolée, demeura dans la maison d'Absalom, son frère.
\VS{21}Quand le roi David eut appris toutes ces choses, il fut très irrité.
\VS{22}Absalom ne parla ni en bien ni en mal avec Amnon ; mais il le prit en haine, parce qu'il avait déshonoré Tamar, sa sœur.
\TextTitle{Vengeance d'Absalom sur Amnon}
\VS{23}Et il arriva au bout de deux années entières, qu'Absalom ayant les tondeurs à Baal-Hatsor, près d'Ephraïm, invita tous les fils du roi.
\VS{24}Absalom alla vers le roi, et dit : Voici, ton serviteur a les tondeurs ; je te prie que le roi et ses serviteurs viennent avec ton serviteur.
\VS{25}Et le roi dit à Absalom : Non, mon fils, nous n'irons pas tous, de peur que nous ne te soyons à charge. Absalom le pressa ; mais le roi ne voulut point aller, et il le bénit.
\VS{26}Absalom dit : Permets au moins à Amnon, mon frère, de venir avec nous. Le roi lui répondit : Pourquoi irait-il ?
\VS{27}Absalom le pressa tellement qu'il laissa aller Amnon et tous les fils du roi avec lui.
\VS{28}Or Absalom avait donné cet ordre à ses serviteurs, en disant : Prenez bien garde, je vous prie, quand le cœur d'Amnon sera égayé par le vin et que je vous dirai : Frappez Amnon ! Tuez-le ; ne craignez point, n'est-ce pas moi qui vous l'ordonne ? Fortifiez-vous et portez-vous en vaillants hommes !
\VS{29}Les serviteurs d'Absalom traitèrent Amnon comme Absalom l'avait ordonné. Et tous les fils du roi se levèrent, montèrent chacun sur son mulet, et s'enfuirent.
\VS{30}Et il arriva, comme ils étaient en chemin, que le bruit parvint à David qu'Absalom avait tué tous les fils du roi, et qu'il n'en était pas resté un seul d'entre eux.
\VS{31}Le roi se leva, déchira ses vêtements, et se coucha par terre ; et tous ses serviteurs étaient là, avec leurs vêtements déchirés.
\VS{32}Jonadab, fils de Schimea, frère de David, prit la parole, et dit : Que mon seigneur ne dise point que tous les jeunes hommes, fils du roi, ont été tués, car seul Amnon est mort ; car c'était là le dessein d'Absalom, depuis le jour où Amnon a violé Tamar, sa sœur ; car il a été exécuté selon son commandement.
\VS{33}Maintenant donc, que le roi mon seigneur ne prenne point la chose à cœur, en disant que tous les fils du roi sont morts, car Amnon seul est mort.
\VS{34}Absalom prit la fuite. Or le jeune homme placé en sentinelle leva les yeux et regarda. Et voici, un grand peuple venait par le chemin qui était derrière lui, du côté de la montagne.
\VS{35}Jonadab dit au roi : Voici les fils du roi qui arrivent ! Ainsi se confirme ce que disait ton serviteur.
\VS{36}Comme il achevait de parler, voici, les fils du roi arrivèrent. Ils élevèrent la voix et pleurèrent ; le roi aussi et tous ses serviteurs versèrent d'abondantes larmes.
\TextTitle{Absalom s'enfuit loin de son père}
\VS{37}Absalom s'était enfui, et il alla chez Talmaï, fils d'Ammihur, roi de Gueschur\FTNT{Absalom s'était réfugié chez Talmaï, roi de Gueschur (Transjordanie, au nord de la Syrie), qui était le père de Maaca, sa mère (2 S. 3:3). Il est donc allé chez son grand-père maternel.}. Et David pleurait tous les jours son fils.
\VS{38}Absalom resta trois ans à Gueschur, où il était allé, après avoir pris la fuite.
\VS{39}Le roi David cessa de poursuivre Absalom, car il était consolé de la mort d'Amnon.
\Chap{14}
\TextTitle{Joab convainc le roi de faire revenir Absalom}
\VerseOne{}Alors Joab, fils de Tseruja, s'aperçut que le cœur du roi était pour Absalom.
\VS{2}Il envoya chercher à Tekoa une femme habile, et il lui dit : Fais semblant de te lamenter, et revêts des habits de deuil ; ne t'oins pas d'huile, mais sois comme une femme qui depuis longtemps pleure un mort.
\VS{3}Ensuite va vers le roi, et tu lui parleras de cette manière. Joab lui mit dans la bouche ce qu'elle devait dire.
\VS{4}La femme de Tekoa alla parler au roi. Elle tomba la face contre terre, se prosterna et dit : Ô roi, sauve-moi !
\VS{5}Le roi lui dit : Qu'as-tu ? Elle répondit : Certainement, je suis une femme veuve, et mon mari est mort !
\VS{6}Or ta servante avait deux fils ; ils se sont tous deux querellés dans les champs, et il n'y avait personne pour les séparer ; l'un a frappé l'autre et l'a tué.
\VS{7}Et voici, toute la famille s'est élevée contre ta servante, en disant : Donne-nous le meurtrier de son frère ! Nous voulons le faire mourir, pour la vie de son frère qu'il a tué ; et que nous exterminions même l'héritier ! Ils veulent ainsi éteindre le charbon vif qui me restait, pour ne laisser à mon mari ni nom ni survivant sur la face de la terre.
\VS{8}Le roi dit à la femme : Va-t-en dans ta maison, et je donnerai des ordres en ta faveur.
\VS{9}Alors la femme de Tekoa dit au roi : Mon seigneur et mon roi ! Que l'iniquité soit sur moi et sur la maison de mon père, et que le roi et son trône en soient innocents.
\VS{10}Et le roi répondit : Si quelqu'un parle contre toi, amène-le-moi, et jamais il ne lui arrivera de te toucher.
\VS{11}Et elle dit : Je te prie, que le roi se souvienne de Yahweh, son Dieu, afin que le vengeur de sang n'augmente pas la ruine et qu'on ne fasse pas périr mon fils. Et il répondit : Yahweh est vivant ! Il ne tombera pas à terre un seul des cheveux de ton fils.
\VS{12}La femme dit : Je te prie que ta servante dise un mot au roi, mon seigneur. Et il répondit : Parle !
\VS{13}La femme dit : Mais pourquoi as-tu pensé une chose comme celle-ci contre le peuple de Dieu ? Puisqu'en tenant ce discours, le roi se déclare coupable en ce qu'il n'a pas fait revenir celui qu'il a banni ?
\VS{14}Car nous mourrons certainement, et nous sommes comme l'eau versée sur la terre qu'on ne peut recueillir. Dieu n'ôte pas la vie, mais il médite les moyens de ne pas repousser loin de lui celui qui est banni de sa présence.
\VS{15}Maintenant, si je suis venue pour tenir ce discours au roi, mon seigneur, c'est parce que le peuple m'a effrayée. Et ta servante a dit : Je veux parler maintenant au roi ; peut-être que le roi fera ce que sa servante lui dira.
\VS{16}Oui, car le roi écoutera sa servante pour la délivrer de la main de celui qui veut nous exterminer, moi et mon fils, de l'héritage de Dieu.
\VS{17}Ta servante a dit : Que la parole du roi, mon seigneur, nous apporte du repos. Car le roi mon seigneur est comme un ange de Dieu, pour entendre le bien et le mal. Que Yahweh, ton Dieu, soit avec toi !
\VS{18}Le roi répondit, et dit à la femme : Je te prie, ne me cache rien de ce que je vais te demander. Et la femme dit : Que le roi mon seigneur parle !
\VS{19}Et le roi dit : La main de Joab n'est-elle pas avec toi dans tout ceci ? Et la femme répondit et dit : Ton âme vit, ô mon seigneur, qu'on ne saurait se détourner ni à droite ni à gauche de tout ce que dit le roi mon seigneur. C'est en effet ton serviteur Joab qui m'a donné des ordres et qui a mis dans la bouche de ta servante toutes ces paroles.
\VS{20}C'est ton serviteur Joab qui a fait que j'ai ainsi tourné ce discours. Mais mon seigneur est sage comme un ange de Dieu, pour savoir tout ce qui se passe sur la terre.
\TextTitle{Retour d'Absalom à Jérusalem}
\VS{21}Alors le roi dit à Joab : Voici, maintenant c'est toi qui as conduit cette affaire ; va donc, et fais revenir le jeune homme Absalom.
\VS{22}Et Joab tomba la face contre terre et se prosterna, et il bénit le roi. Puis il dit : Aujourd'hui, ton serviteur sait qu'il a trouvé grâce à tes yeux, ô roi mon seigneur, puisque le roi agit selon ce que son serviteur lui a dit.
\VS{23}Joab se leva et partit pour Gueschur, et il ramena Absalom à Jérusalem.
\VS{24}Mais le roi dit : Qu'il se retire dans sa maison, et qu'il ne voie point ma face. Et Absalom se retira dans sa maison, et ne vit point la face du roi.
\VS{25}Il n'y avait point d'homme dans tout Israël aussi renommé qu'Absalom pour sa beauté ; depuis la plante des pieds jusqu'au sommet de la tête, il n'y avait point en lui de défaut.
\VS{26}Et quand il faisait couper ses cheveux, or il arrivait tous les ans qu'il les faisait couper, parce que sa chevelure lui pesait trop, le poids de sa chevelure était de deux cents sicles, poids du roi.
\VS{27}Il naquit à Absalom trois fils, et une fille nommée Tamar, qui était une femme belle de figure.
\VS{28}Et Absalom demeura deux ans entiers à Jérusalem, sans voir la face du roi.
\VS{29}Absalom fit demander Joab, pour l'envoyer vers le roi ; mais Joab ne voulut pas venir vers lui ; il le fit demander encore pour la seconde fois ; mais Joab ne voulut point venir.
\VS{30}Absalom dit alors à ses serviteurs : Voyez le champ de Joab qui est à côté du mien ; il y a de l'orge ; allez et mettez-y le feu. Et les serviteurs d'Absalom mirent le feu au champ.
\VS{31}Alors Joab se leva et vint vers Absalom dans sa maison. Il lui dit : Pourquoi tes serviteurs ont-ils mis le feu à mon champ?
\VS{32}Et Absalom répondit à Joab : Voici, je t'ai fait dire : Viens ici, et je t'enverrai vers le roi, afin que tu lui dises : Pourquoi suis-je revenu de Gueschur ? Il vaudrait mieux pour moi que j'y fusse encore. Je désire maintenant voir la face du roi ; et s'il y a de l'iniquité en moi, qu'il me fasse mourir.
\VS{33}Joab alla vers le roi, et lui rapporta cela. Et le roi appela Absalom, qui vint vers lui et se prosterna le visage contre terre devant le roi. Le roi embrassa Absalom.
\Chap{15}
\TextTitle{Mauvaises intentions d'Absalom}
\VerseOne{}Or il arriva qu'après cela, Absalom se procura des chars et des chevaux, et il avait cinquante hommes qui couraient devant lui\FTNT{La révolte d'Absalom était une autre conséquence du péché de David avec Bath-Schéba.}.
\VS{2}Absalom se levait de bon matin et se tenait au bord du chemin de la porte. Et chaque fois qu'un homme ayant une contestation se rendait auprès du roi pour obtenir justice, Absalom l'appelait, et lui disait : De quelle ville es-tu ? Et il répondait : Ton serviteur est de l'une des tribus d'Israël.
\VS{3}Absalom lui disait : Vois, ta cause est bonne et droite ; mais personne de chez le roi ne t'écoutera.
\VS{4}Absalom disait encore : Qui m'établira juge dans le pays ? Tout homme qui aurait une contestation et un procès viendrait vers moi, et je lui ferais justice.
\VS{5}Et il arrivait aussi que quand quelqu'un s'approchait de lui pour se prosterner, il lui tendait sa main, le saisissait, et l'embrassait.
\VS{6}Absalom faisait ainsi à tous ceux d'Israël qui venaient vers le roi pour demander justice. Et Absalom gagnait les cœurs des hommes d'Israël.
\TextTitle{Conspiration d'Absalom}
\VS{7}Et il arriva qu'au bout de quarante ans, Absalom dit au roi : Permets que j'aille à Hébron, pour accomplir le vœu que j'ai fait à Yahweh.
\VS{8}Car quand ton serviteur demeurait à Gueschur en Syrie, il fit un vœu, en disant : Si Yahweh me ramène à Jérusalem, j'en témoignerai ma reconnaissance à Yahweh.
\VS{9}Et le roi lui répondit : Va en paix. Et Absalom se leva et s'en alla à Hébron.
\VS{10}Absalom envoya des espions dans toutes les tribus d'Israël, pour dire : Aussitôt que vous entendrez le son du shofar, vous direz : Absalom est établi roi à Hébron !
\VS{11}Deux cents hommes de Jérusalem, qui avaient été invités, s'en allèrent avec Absalom ; ils y allèrent en toute simplicité de coeur, ne sachant rien de cette affaire.
\VS{12}Pendant qu'Absalom offrait les sacrifices, il envoya chercher à la ville de Guilo, Achitophel, le Guilonite, conseiller de David. Il se forma une puissante conspiration, parce que le peuple était de plus en plus nombreux auprès d'Absalom.
\TextTitle{David fuit son fils Absalom}
\VS{13}Un messager se rendit auprès de David, et lui dit : Le cœur des hommes d'Israël s'est tourné vers Absalom.
\VS{14}Et David dit à tous ses serviteurs qui étaient avec lui à Jérusalem : Levez-vous, fuyons, car nous ne pourrons échapper à Absalom. Hâtez-vous de partir ; sinon, il ne tarderait pas à nous atteindre, et il nous précipiterait dans le malheur et frapperait la ville du tranchant de l'épée.
\VS{15}Les serviteurs du roi lui répondirent : Tes serviteurs feront tout ce que le roi, notre seigneur, voudra.
\VS{16}Le roi sortit, et toute sa maison le suivait, mais le roi laissa dix femmes, des concubines, pour garder la maison.
\VS{17}Le roi sortit, et tout le peuple le suivait, et ils s'arrêtèrent à Beth-Merkhak.
\VS{18}Tous ses serviteurs marchaient à côté de lui ; tous les Kéréthiens, tous les Péléthiens, et tous les Gathiens, qui étaient six cents hommes venus de Gath, pour être à sa suite, marchaient devant le roi.
\VS{19}Mais le roi dit à Ittaï de Gath : Pourquoi viendrais-tu aussi avec nous ? Retourne et reste avec le roi, car tu es étranger, et même tu vas retourner bientôt en ton lieu.
\VS{20}Tu es arrivé hier, et te ferais-je aujourd'hui errer çà et là avec nous ? Quant à moi, je m'en vais où je pourrai ! Retourne et emmène tes frères avec toi. Que la bonté et la vérité t'accompagnent !
\VS{21}Mais Ittaï répondit au roi, et dit : Yahweh est vivant, et le roi mon seigneur est vivant ! Quel que soit le lieu où le roi mon seigneur sera, soit pour mourir, soit pour vivre, ton serviteur y sera aussi.
\VS{22}David donc dit à Ittaï : Viens, et marche ! Alors Ittaï de Gath marcha avec tous ses gens et tous les enfants qui étaient avec lui.
\VS{23}Et tout le pays pleurait à grands cris et tout le peuple passait plus avant. Puis le roi passa le torrent de Cédron, et tout le peuple passa en face du chemin qui mène au désert.
\TextTitle{L'arche de l'alliance à Jérusalem}
\VS{24}Tsadok était aussi là, et avec lui tous les Lévites portant l'arche de l'alliance de Dieu ; et ils posèrent là l'arche de Dieu, et Abiathar montait, pendant que tout le peuple achevait de sortir de la ville.
\VS{25}Le roi dit à Tsadok : Rapporte l'arche de Dieu dans la ville. Si je trouve grâce aux yeux de Yahweh, il me ramènera, et il me fera voir l'arche et sa demeure.
\VS{26}Mais s'il dit : Je ne prends point de plaisir en toi ! Me voici, qu'il fasse de moi ce qui lui semblera bon.
\VS{27}Le roi dit encore au sacrificateur Tsadok : N'es-tu pas le voyant ? Retourne en paix dans la ville, avec Achimaats, ton fils, et Jonathan, fils d'Abiathar, vos deux fils.
\VS{28}Voyez, j'attendrai dans les plaines du désert, jusqu'à ce qu'on vienne m'apporter des nouvelles de votre part.
\VS{29}Ainsi, Tsadok et Abiathar rapportèrent l'arche de Dieu à Jérusalem, et ils y restèrent.
\VS{30}David monta par la montée des oliviers. Il montait en pleurant, la tête couverte, et marchait pieds nus ; tout le peuple qui était avec lui se couvrit aussi la tête, et il montait en pleurant.
\VS{31}Alors on vint dire à David : Achitophel est parmi ceux qui ont conspiré avec Absalom. Et David dit : Je te prie, ô Yahweh, abolis les conseils d'Achitophel !
\TextTitle{Huschaï, espion pour David dans la cour d'Absalom}
\VS{32}Et il arriva que quand David fut arrivé au sommet de la montagne, où il se prosterna devant Dieu, Huschaï, l'Arkien, vint au-devant de lui, la tunique déchirée et de la terre sur sa tête.
\VS{33}David lui dit : Tu me seras à charge si tu viens avec moi.
\VS{34}Et au contraire, tu anéantiras en ma faveur les conseils d'Achitophel, si tu retournes à la ville, et que tu dis à Absalom : Ô roi, je serai ton serviteur, comme je fus autrefois le serviteur de ton père ; mais maintenant je serai ton serviteur.
\VS{35}Les sacrificateurs Tsadok et Abiathar ne seront-ils pas là avec toi ? Tout ce que tu entendras de la maison du roi, tu le rapporteras aux sacrificateurs Tsadok et Abiathar.
\VS{36}Voici, ils ont là avec eux leurs deux fils, Achimaats, fils de Tsadok, et Jonathan, fils d'Abiathar ; c'est par eux que vous me ferez savoir tout ce que vous aurez entendu.
\VS{37}Huschaï, l'ami de David, retourna donc dans la ville, et Absalom entra à Jérusalem.
\Chap{16}
\TextTitle{Tsiba retrouve David en fuite}
\VerseOne{}Quand David eut un peu dépassé le sommet, voici, Tsiba, serviteur de Mephiboscheth, vint au-devant de lui avec deux ânes bâtés, sur lesquels il y avait deux cents pains, cent paquets de raisins secs, cent de fruits d'été, et une outre de vin.
\VS{2}Le roi dit à Tsiba : Que veux-tu faire de cela ? Et Tsiba répondit : Les ânes serviront de montures pour la maison du roi, le pain et les autres fruits d'été sont pour nourrir les jeunes gens, et le vin pour désaltérer ceux qui se seront fatigués dans le désert.
\VS{3}Le roi lui dit : Mais où est le fils de ton maître ? Et Tsiba répondit au roi : Voici, il est resté à Jérusalem, car il a dit : Aujourd'hui, la maison d'Israël me rendra le royaume de mon père.
\VS{4}Alors le roi dit à Tsiba : Voici, tout ce qui est à Mephiboscheth est à toi. Et Tsiba dit : Je me prosterne ! Que je trouve grâce à tes yeux, ô roi, mon seigneur !
\TextTitle{Schimeï maudit le roi David}
\VS{5}Le roi David était arrivé jusqu'à Bachurim. Et voici, il sortit de là un homme de la famille et de la maison de Saül, nommé Schimeï, fils de Guéra. Il s'avança en prononçant des malédictions,
\VS{6}il jeta des pierres contre David, contre tous ses serviteurs, et contre tout le peuple ; tous les hommes vaillants étaient à la droite et à la gauche du roi.
\VS{7}Schimeï parlait ainsi en le maudissant : Sors, sors, homme de sang, méchant homme !
\VS{8}Yahweh fait retomber sur toi tout le sang de la maison de Saül, à la place duquel tu régnais, et Yahweh a mis le royaume entre les mains de ton fils, Absalom ; et voilà, tu souffres le mal que tu as fait, parce que tu es un homme de sang !
\VS{9}Alors Abischaï, fils de Tseruja, dit au roi : Pourquoi ce chien mort maudit-il le roi, mon seigneur ? Permets que je m'avance et que je lui ôte la tête.
\VS{10}Mais le roi répondit : Qu'ai-je à faire avec vous, fils de Tseruja ? S'il maudit, c'est que Yahweh lui a dit : Maudis David ! Qui donc lui dira : Pourquoi agis-tu ainsi ?
\VS{11}Et David dit à Abischaï et à tous ses serviteurs : Voici, mon propre fils, qui est sorti de mes entrailles, en veut à ma vie ; à plus forte raison ce Benjamite ! Laissez-le, et qu'il maudisse, car Yahweh lui a parlé.
\VS{12}Peut-être Yahweh regardera mon affliction, et que Yahweh me rendra le bien au lieu des malédictions d'aujourd'hui.
\VS{13}David donc, et ses gens, continuèrent leur chemin. Et Schimeï marchait sur le flanc de la montagne vis-à-vis de lui, continuant à maudire, jetant des pierres contre lui et de la poussière en l'air.
\VS{14}Le roi David et tout le peuple qui était avec lui arrivèrent fatigués et là ils se rafraîchirent.
\TextTitle{Abominations d'Absalom à Jérusalem}
\VS{15}Absalom, et tout le peuple, les hommes d'Israël, étaient entrés dans Jérusalem ; et Achitophel était avec lui.
\VS{16}Quand Huschaï, l'Arkien, ami de David, fut arrivé auprès d'Absalom, il lui dit : Vive le roi ! Vive le roi !
\VS{17}Et Absalom dit à Huschaï : Est-ce donc là l'affection que tu as pour ton ami ? Pourquoi n'es-tu pas allé avec ton ami ?
\VS{18}Huschaï répondit à Absalom : Non, mais je serai à celui qui a été choisi par Yahweh, par ce peuple et par tous les hommes d'Israël, et je demeurerai avec lui.
\VS{19}D'ailleurs, qui servirai-je ? Ne sera-ce pas son fils ? Je serai ton serviteur, comme j'ai été le serviteur de ton père.
\VS{20}Absalom dit à Achitophel : Donnez un conseil sur ce que nous ferons.
\VS{21}Achitophel dit à Absalom : Va vers les concubines que ton père a laissées pour garder la maison ; ainsi tout Israël saura que tu t'es rendu odieux envers ton père, et les mains de tous ceux qui sont avec toi se fortifieront.
\VS{22}On dressa une tente pour Absalom sur le toit et Absalom alla vers les concubines de son père, aux yeux de tout Israël\FTNT{2 S. 12:11-12.}.
\VS{23}Les conseils que donnait Achitophel en ce temps-là étaient autant estimés que si l'on eût demandé la parole de Dieu. C'est ainsi qu'on considérait tous les conseils qu'Achitophel donnait, tant à David qu'à Absalom.
\Chap{17}
\TextTitle{Schimeï maudit le roi David}
\VerseOne{}Après cela, Achitophel dit à Absalom : Je choisirai maintenant douze mille hommes, et je me lèverai, et je poursuivrai David cette nuit.
\VS{2}Je l'atteindrai pendant qu'il est fatigué, et que ses mains sont affaiblies ; je l'épouvanterai tellement que tout le peuple qui est avec lui s'enfuira, et je frapperai seulement le roi ;
\VS{3}et je ramènerai à toi tout le peuple ; car l'homme que tu cherches vaut autant que si tous retournaient à toi ; ainsi tout le peuple sera en paix.
\VS{4}Cette parole plut à Absalom, et à tous les anciens d'Israël.
\VS{5}Cependant Absalom dit : Qu'on appelle maintenant aussi Huschaï, l'Arkien, et que nous entendions aussi son avis.
\VS{6}Huschaï vint vers Absalom et Absalom lui dit : Achitophel a donné un tel avis ; devons-nous faire ce qu'il a dit ou non ? Parle, toi aussi.
\VS{7}Alors Huschaï dit à Absalom : Cette fois, le conseil qu'Achitophel a donné n'est pas bon.
\VS{8}Huschaï dit encore : Tu connais ton père et ses gens, ce sont des hommes forts, et ils ont l'amertume dans l'âme comme une ourse des champs privée de ses petits. Ton père est un homme de guerre, il ne passera pas la nuit avec le peuple.
\VS{9}Voici, il est maintenant caché dans quelque fosse, ou dans quelque autre lieu ; et si, dès le commencement, il en est qui tombent sous leurs coups, on ne tardera pas à l'apprendre et l'on dira : Il y a une défaite parmi le peuple qui suit Absalom !
\VS{10}Alors le plus vaillant, celui-là même qui avait le cœur comme un lion, se découragera ; car tout Israël sait que ton père est un homme de cœur, et que ceux qui sont avec lui sont vaillants.
\VS{11}Je conseille donc que tout Israël se rassemble auprès de toi, depuis Dan jusqu'à Beer-Schéba, multitude pareille au sable qui est sur le bord de la mer, et qu'en personne tu marches au combat.
\VS{12}Alors nous viendrons à lui en quelque lieu que nous le trouvions, et nous nous jetterons sur lui, comme la rosée tombe sur la terre ; et il ne lui restera aucun de tous les hommes qui sont avec lui.
\VS{13}S'il se retire dans une ville, tout Israël portera des cordes vers cette ville-là, et nous la traînerons jusqu'au torrent, jusqu'à ce qu'on n'en trouve plus une pierre.
\VS{14}Alors Absalom et tous les hommes d'Israël dirent : Le conseil de Huschaï, l'Arkien, est meilleur que le conseil d'Achitophel. Car Yahweh avait résolu de dissiper le conseil d'Achitophel, qui était bon, afin de faire venir le mal sur Absalom.
\TextTitle{Huschaï avertit David du danger}
\VS{15}Alors Huschaï dit aux sacrificateurs Tsadok et Abiathar : Achitophel a donné tel et tel conseil à Absalom, et aux anciens d'Israël ; mais moi, j'ai conseillé telle et telle chose.
\VS{16}Maintenant donc, envoyez tout de suite informer David, en disant : Ne passe point la nuit dans les plaines du désert, mais va plus loin, de peur que le roi et tout le peuple qui est avec lui ne soient exposés au péril.
\VS{17}Jonathan et Achimaats se tenaient à En-Roguel (la fontaine du foulon). Une servante vint leur dire d'aller informer le roi David ; car ils n'osaient pas se montrer et entrer dans la ville.
\VS{18}Mais un garçon les aperçut, et le rapporta à Absalom. Et ils partirent tous deux en hâte et ils arrivèrent à Bachurim, à la maison d'un homme qui avait un puits dans sa cour, dans lequel ils descendirent.
\VS{19}La femme de cet homme prit une couverture, qu'elle étendit sur l'ouverture du puits, et y répandit dessus du grain pilé en sorte qu'on ne s'aperçut de rien.
\VS{20}Les serviteurs d'Absalom entrèrent dans la maison auprès de cette femme, et lui dirent : Où sont Achimaats et Jonathan ? La femme leur répondit : Ils ont passé le ruisseau. Ils cherchèrent, et ne les trouvant pas, ils retournèrent à Jérusalem.
\VS{21}Après leur départ, Achimaats et Jonathan remontèrent du puits et allèrent informer le roi David. Ils lui dirent : Levez-vous, et hâtez-vous de passer l'eau, car Achitophel a conseillé telle chose contre vous.
\VS{22}Alors David et tout le peuple qui était avec lui se levèrent et ils passèrent le Jourdain ; à la lumière du matin, il n'en manqua pas un qui n'eût passé le Jourdain.
\VS{23}Or Achitophel voyant qu'on n'avait point fait ce qu'il avait conseillé, fit seller son âne, se leva, et s'en alla en sa maison, dans sa ville. Après avoir donné des ordres à sa maison, il s'étrangla et mourut. On l'enterra dans le sépulcre de son père.
\TextTitle{Absalom et Israël en marche contre David}
\VS{24}David arriva à Mahanaïm. Et Absalom passa le Jourdain, lui et tous les hommes d'Israël avec lui.
\VS{25}Absalom établit Amasa sur l'armée, à la place de Joab. Or Amasa était fils d'un homme nommé Jithra, l'Israélite, qui était allé vers Abigaïl, fille de Nachasch, et soeur de Tseruja, mère de Joab.
\VS{26}Israël et Absalom campèrent dans le pays de Galaad.
\TextTitle{Mahanaïm bienveillant envers David}
\VS{27}Or il arriva qu'aussitôt que David fut arrivé à Mahanaïm, Schobi, fils de Nachasch de Rabba, des fils d'Ammon, Makir, fils d'Ammiel de Lodebar, et Barzillaï, le Galaadite de Roguelim,
\VS{28}apportèrent des lits, des bassins, des vases de terre, du froment, de l'orge, de la farine, du grain rôti, des fèves, des lentilles, des pois rôtis,
\VS{29}du miel, de la crème, des brebis, et des fromages de vache. Ils apportèrent ces choses à David et au peuple qui était avec lui, afin qu'ils mangent, car ils disaient : Ce peuple a dû souffrir de la faim, de la fatigue et de la soif dans le désert.
\Chap{18}
\TextTitle{Bataille dans la forêt d'Ephraïm ; instructions de David sur Absalom}
\VerseOne{}David fit le dénombrement du peuple qui était avec lui, et il établit sur eux des chefs de milliers et des chefs de centaines.
\VS{2}David envoya le peuple, un tiers sous le commandement de Joab, un tiers sous le commandement d'Abischaï, fils de Tseruja, frère de Joab, et un tiers sous le commandement d'Ittaï, de Gath. Et le roi dit au peuple : Moi aussi, je veux sortir avec vous.
\VS{3}Mais le peuple lui dit : Tu ne sortiras point ! Car si nous prenons la fuite, ce n'est pas sur nous que l'attention se portera ; et même quand la moitié d'entre nous y serait tuée, on n'y ferait pas attention ; mais toi, tu es comme dix mille de nous, et maintenant il vaut mieux que de la ville tu puisses venir à notre secours.
\VS{4}Le roi leur répondit : Je ferai ce qui est bon à vos yeux. Le roi s'arrêta donc à la place de la porte, pendant que tout le peuple sortait par centaines et par milliers.
\VS{5}Le roi donna cet ordre à Joab, à Abischaï, et à Ittaï, et dit : Epargnez-moi le jeune homme Absalom ! Et tout le peuple entendit ce que le roi commandait à tous les chefs au sujet d'Absalom.
\VS{6}Ainsi le peuple sortit dans les champs à la rencontre d'Israël, et la bataille eut lieu dans la forêt d'Ephraïm.
\VS{7}Là, le peuple d'Israël fut battu par les serviteurs de David, et il y eut en ce jour-là dans ce même lieu, une grande défaite de vingt mille hommes.
\VS{8}La bataille s'étendit sur toute la contrée, et la forêt dévora ce jour-là beaucoup plus de peuple que l'épée.
\TextTitle{Joab tue Absalom}
\VS{9}Absalom se retrouva devant les serviteurs de David. Il était monté sur un mulet. Le mulet entra sous les branches entrelacées d'un grand chêne, et la tête d'Absalom fut prise dans le chêne ; il demeura suspendu entre le ciel et la terre, et le mulet qui était sous lui passa outre.
\VS{10}Un homme ayant vu cela, le rapporta à Joab, et lui dit : Voici, j'ai vu Absalom suspendu à un chêne.
\VS{11}Et Joab répondit à l'homme qui lui rapportait cela : Tu l'as vu ! Pourquoi ne l'as-tu pas tué là, le jetant par terre ? Je t'aurais donné dix sicles d'argent et une ceinture.
\VS{12}Mais cet homme dit à Joab : Quand je pèserais dans ma main mille pièces d'argent, je ne mettrais pas ma main sur le fils du roi ; car nous avons entendu ce que le roi vous a ordonné, à toi, à Abischaï et à Ittaï, en disant : Prenez garde chacun au jeune homme Absalom !
\VS{13}Autrement j'aurais commis une lâcheté au péril de ma vie, car rien ne serait caché au roi, et toi-même tu te lèverais contre moi.
\VS{14}Joab répondit : Je ne m'attarderai pas auprès de toi ! Et il prit en sa main trois javelots, et les enfonça dans le cœur d'Absalom qui était encore vivant au milieu du chêne.
\VS{15}Puis dix jeunes hommes, qui portaient les armes de Joab, entourèrent Absalom, le frappèrent et le firent mourir\FTNT{La mort d'Absalom fut une conséquence du péché de David avec Bath-Schéba. Le péché a donc des conséquences graves et cause beaucoup de souffrances.}.
\VS{16}Alors Joab fit sonner la trompette ; et le peuple cessa de poursuivre Israël, parce que Joab le retint.
\VS{17}Ils prirent Absalom, le jetèrent dans la forêt dans une grande fosse, et mirent sur lui un très grand monceau de pierres. Tout Israël s'enfuit, chacun dans sa tente.
\VS{18}Or Absalom s'était fait ériger, de son vivant, un monument dans la vallée du roi ; car il disait : Je n'ai point de fils pour conserver la mémoire de mon nom. Et il donna son propre nom au monument, qu'on appelle encore aujourd'hui la place d'Absalom.
\TextTitle{David apprend la mort d'Absalom}
\VS{19}Et Achimaats, fils de Tsadok, dit : Laisse-moi courir, et porter au roi la bonne nouvelle que Yahweh lui a rendu justice en jugeant ses ennemis.
\VS{20}Joab lui répondit : Tu ne seras pas aujourd'hui porteur de bonnes nouvelles ; tu le seras un autre jour ; car aujourd'hui tu ne porterais pas de bonnes nouvelles, puisque le fils du roi est mort.
\VS{21}Et Joab dit à Cuschi : Va, et annonce au roi ce que tu as vu. Cuschi se prosterna devant Joab, puis il se mit à courir.
\VS{22}Achimaats, fils de Tsadok, dit encore à Joab : Quoi qu'il arrive, laisse-moi courir après Cuschi. Joab lui dit : Pourquoi veux-tu courir, mon fils, puisque tu n'as pas de bonnes nouvelles à apporter ?
\VS{23}Quoiqu'il arrive, je veux courir, reprit Achimaats. Et Joab lui dit : Cours ! Achimaats courut par le chemin de la plaine, et il devança Cuschi.
\VS{24}David était assis entre les deux portes. La sentinelle alla sur le toit de la porte vers la muraille ; elle leva les yeux et elle regarda. Et voici un homme qui courait tout seul.
\VS{25}Alors la sentinelle cria, et avertit le roi. Le roi dit : S'il est seul, il apporte des bonnes nouvelles. Et cet homme marchait incessamment et approchait.
\VS{26}Puis la sentinelle vit un autre homme qui courait ; et elle cria au portier : Voici un homme qui court tout seul. Le roi dit : Il apporte aussi des bonnes nouvelles.
\VS{27}La sentinelle dit : La manière de courir du premier me paraît celle d'Achimaats, fils de Tsadok. Et le roi dit : C'est un homme de bien, il vient quand il y a des bonnes nouvelles.
\VS{28}Achimaats cria, et il dit au roi : Tout va bien ! Et il se prosterna devant le roi, le visage contre terre, et dit : Béni soit Yahweh, ton Dieu, qui a livré les hommes qui levaient leurs mains contre le roi, mon seigneur !
\VS{29}Le roi dit : Le jeune homme Absalom se porte-t-il bien ? Achimaats lui répondit : J'ai vu s'élever un grand tumulte au moment où Joab envoya le serviteur du roi et moi ton serviteur ; mais je ne sais pas exactement ce que c'était.
\VS{30}Et le roi lui dit : Mets-toi là de côté. Et Achimaats se tint de côté.
\VS{31}Aussitôt arriva Cuschi. Et il dit : Que le roi, mon seigneur apprenne, ces bonnes nouvelles ! Aujourd'hui, Yahweh t'a rendu justice en jugeant tous ceux qui s'élevaient contre toi.
\VS{32}Le roi dit à Cuschi : Le jeune homme Absalom se porte-t-il bien ? Et Cuschi lui répondit : Que les ennemis du roi, mon seigneur, et tous ceux qui s'élèvent contre toi pour te faire du mal soient comme ce jeune homme !
\VS{33}Alors le roi, saisi d'émotion, monta à la chambre haute de la porte, et alla pleurer. Il disait ainsi en marchant : Mon fils Absalom ! Mon fils, mon fils Absalom ! Plaise à Dieu que je sois moi-même mort à ta place ! Absalom, mon fils, mon fils !
\Chap{19}
\TextTitle{Souffrance de David ; indignation de Joab}
\VerseOne{}Et on fit ce rapport à Joab : Voici, le roi pleure et se lamente à cause d'Absalom.
\VS{2}Ainsi, la victoire fut en ce jour-là changée en deuil pour tout le peuple, car en ce jour-là le peuple entendait dire : Le roi est affligé à cause de son fils.
\VS{3}Ce même jour, le peuple rentra dans la ville à la dérobée, comme l'auraient fait des gens honteux d'avoir pris la fuite dans la bataille.
\VS{4}Le roi s'était couvert le visage, et il criait à haute voix : Mon fils Absalom ! Absalom, mon fils, mon fils !
\VS{5}Joab entra dans la chambre où était le roi, et lui dit : Tu couvres aujourd'hui de confusion les faces de tous tes serviteurs, qui ont en ce jour sauvé ta vie, celle de tes fils et de tes filles, celle de tes femmes et de tes concubines.
\VS{6}Tu aimes ceux qui te haïssent, et tu hais ceux qui t'aiment, car tu montres aujourd'hui que tes chefs et tes serviteurs ne te sont rien ; et je sais maintenant que si Absalom vivait, et que nous tous fussions morts aujourd'hui, cela serait agréable à tes yeux.
\VS{7}Maintenant donc lève-toi, sors, et parle selon le coeur de tes serviteurs ! Car je jure par Yahweh que si tu ne sors pas, il ne restera pas un seul homme avec toi cette nuit ; et ce mal sera pire que tous ceux qui te sont arrivés depuis ta jeunesse jusqu'à présent.
\TextTitle{Retour du roi David à Jérusalem}
\VS{8}Alors le roi se leva et s'assit à la porte. On fit dire à tout le peuple : Voici, le roi est assis à la porte. Et tout le peuple vint devant le roi. Cependant, Israël s'était enfui, chacun dans sa tente.
\VS{9}Et dans toutes les tribus d'Israël, tout le peuple était en contestation, disant : Le roi nous a délivrés de la main de nos ennemis, c'est lui qui nous a sauvés de la main des Philistins, et maintenant il a dû fuir du pays devant Absalom.
\VS{10}Or Absalom, que nous avions oint pour roi sur nous, est mort dans la bataille. Maintenant donc, pourquoi ne parlez-vous pas de faire revenir le roi ?
\VS{11}Le roi David envoya dire aux sacrificateurs Tsadok et Abiathar : Parlez aux anciens de Juda, et dites-leur : Pourquoi seriez-vous les derniers à ramener le roi en sa maison ? Car les discours que tout Israël avait tenus étaient parvenus jusqu'au roi dans sa maison.
\VS{12}Vous êtes mes frères, vous êtes mes os et ma chair ; pourquoi seriez-vous les derniers à ramener le roi ?
\VS{13}Dites même à Amasa : N'es-tu pas mon os et ma chair ? Que Dieu me traite dans toute sa rigueur si tu ne deviens pas devant moi pour toujours chef de l'armée à la place de Joab !
\VS{14}Ainsi David fléchit le cœur de tous les hommes de Juda, comme s'ils n'eussent été qu'un seul homme ; et ils envoyèrent dire au roi : Reviens, toi, et tous tes serviteurs.
\VS{15}Le roi revint et arriva jusqu'au Jourdain ; et Juda se rendit jusqu'à Guilgal, pour aller à la rencontre du roi afin de lui faire repasser le Jourdain.
\VS{16}Et Schimeï, fils de Guéra, Benjamite, qui était de Bachurim, se hâta de descendre avec les hommes de Juda à la rencontre du roi David.
\VS{17}Il avait avec lui mille hommes de Benjamin, et Tsiba, serviteur de la maison de Saül, ses quinze enfants, et ses vingt serviteurs étaient aussi avec lui. Ils passèrent le Jourdain en présence du roi.
\VS{18}Le bateau, mis à la disposition du roi, faisait la traversée pour transporter sa maison ; et au moment où le roi allait passer le Jourdain, Schimeï, fils de Guéra, se prosterna devant lui.
\VS{19}Et il dit au roi : Que mon seigneur ne m'impute pas mon iniquité, et ne se souvienne pas de ce que ton serviteur a fait de mal le jour où le roi mon seigneur sortait de Jérusalem, et que le roi ne le prenne point à cœur !
\VS{20}Car ton serviteur sait qu'il a péché. Et voici, je viens aujourd'hui le premier de toute la maison de Joseph à la rencontre du roi, mon seigneur.
\VS{21}Mais Abischaï, fils de Tseruja, répondit et dit : A cause de cela, ne fera-t-on pas mourir Schimeï, puisqu'il a maudit l'oint de Yahweh ?
\VS{22}Et David dit : Qu'ai-je à faire avec vous, fils de Tseruja ? Et pourquoi vous montrez-vous aujourd'hui mes adversaires ? Ferait-on mourir aujourd'hui quelqu'un en Israël ? Ne sais-je donc pas que je règne aujourd'hui sur Israël ?
\VS{23}Et le roi dit à Schimeï : Tu ne mourras point ! Et le roi le lui jura.
\VS{24}Après cela, Mephiboscheth, fils de Saül, descendit aussi à la rencontre du roi. Il n'avait point lavé ses pieds, ni fait sa barbe, ni lavé ses vêtements, depuis que le roi s'en était allé, jusqu'au jour où il revenait en paix.
\VS{25}Il se trouva donc au-devant du roi comme il entrait dans Jérusalem, et le roi lui dit : Pourquoi n'es-tu pas venu avec moi, Mephiboscheth ?
\VS{26}Et il lui répondit : Ô roi, mon seigneur, mon serviteur m'a trompé, car ton serviteur qui est boiteux avait dit : Je ferai seller mon âne, je monterai dessus, et j'irai avec le roi.
\VS{27}Et il a calomnié ton serviteur auprès du roi, mon seigneur. Mais le roi mon seigneur est comme un ange de Dieu. Fais donc ce qui semblera bon à tes yeux.
\VS{28}Car bien que tous ceux de la maison de mon père n'ont été que des gens dignes de mort devant le roi mon seigneur ; cependant tu as mis ton serviteur parmi ceux qui mangent à ta table. Quel droit puis-je encore avoir, pour me plaindre encore au roi ?
\VS{29}Et le roi lui dit : Pourquoi toutes ces paroles ? Je l'ai dit : Toi et Tsiba, vous partagerez les terres.
\VS{30}Et Mephiboscheth dit au roi : Qu'il prenne même tout, puisque le roi mon seigneur rentre en paix dans sa maison.
\VS{31}Barzillaï, le Galaadite, descendit de Roguelim, et passa le Jourdain avec le roi, pour l'accompagner jusqu'au-delà du Jourdain.
\VS{32}Barzillaï était très vieux, âgé de quatre-vingts ans. Il avait nourri le roi pendant qu'il avait séjourné à Mahanaïm, car c'était un homme fort riche.
\VS{33}Le roi dit à Barzillaï : Viens avec moi, je te nourrirai chez moi à Jérusalem.
\VS{34}Mais Barzillaï répondit au roi : Combien d'années vivrai-je encore pour que je monte avec le roi à Jérusalem ?
\VS{35}Je suis aujourd'hui âgé de quatre-vingts ans. Puis-je encore discerner ce qui est bon de ce qui est mauvais ? Ton serviteur peut-il savourer ce qu'il mange et ce qu'il boit ? Puis-je encore entendre la voix des chanteurs et des chanteuses ? Et pourquoi ton serviteur serait-il encore à charge à mon seigneur, le roi ?
\VS{36}Ton serviteur ira un peu au-delà du Jourdain avec le roi. Pourquoi le roi voudrait-il me donner une telle récompense ?
\VS{37}Je te prie que ton serviteur s'en retourne, et que je meure dans ma ville, près du sépulcre de mon père et de ma mère ! Mais voici ton serviteur Kimham, passera avec le roi mon seigneur ; fais-lui ce qui semblera bon à tes yeux.
\VS{38}Le roi dit : Que Kimham passe avec moi, et je lui ferai ce qui sera bon à tes yeux ; et tout ce que tu voudras de moi, je te l'accorderai.
\VS{39}Tout le peuple passa donc le Jourdain avec le roi. Puis le roi embrassa Barzillaï et le bénit. Et Barzillaï retourna dans sa demeure.
\TextTitle{Juda et Israël se disputent le roi}
\VS{40}De là, le roi passa à Guilgal, et Kimham passa avec lui. Ainsi, tout le peuple de Juda, et même la moitié du peuple d'Israël ramenèrent le roi.
\VS{41}Mais voici, tous les hommes d'Israël vinrent vers le roi, et lui dirent : Pourquoi nos frères, les hommes de Juda, t'ont-ils enlevé, et ont-ils fait passer le Jourdain au roi et à sa maison, et à tous les gens de David ?
\VS{42}Alors tous les hommes de Juda répondirent aux hommes d'Israël : Parce que le roi nous est plus proche ; pourquoi vous fâchez-vous de cela ? Avons-nous vécu aux dépens du roi ? Nous a-t-il fait des présents ?
\VS{43}Les hommes d'Israël répondirent aux hommes de Juda, et dirent : Le roi nous appartient dix fois autant, et David même plus qu'à vous. Pourquoi nous avez-vous méprisés ? N'avons-nous pas parlé les premiers de ramener notre roi ? Mais les hommes de Juda parlèrent avec plus de violence que les hommes d'Israël.
\Chap{20}
\TextTitle{Juda reste fidèle au roi David}
\VerseOne{}Et il se trouvait là un méchant\FTNT{Littéralement « beliya`al » : « méchant, pervers », « ruine, destruction ». Voir commentaire en 1 S. 2:12.} homme, nommé Schéba, fils de Bicri, Benjamite. Il sonna du shofar et dit : Nous n'avons point de part avec David ni d'héritage avec le fils d'Isaï ! Israël, chacun à ses tentes !
\VS{2}Ainsi tous les hommes d'Israël se séparèrent de David, et suivirent Schéba, fils de Bicri. Mais les hommes de Juda s'attachèrent à leur roi, et l'accompagnèrent depuis le Jourdain jusqu'à Jérusalem.
\VS{3}David rentra dans sa maison à Jérusalem. Il prit les dix femmes concubines qu'il avait laissées pour garder sa maison, et les mit en un lieu où elles étaient gardées ; il pourvut à leur entretien, mais il n'alla point vers elles. Ainsi, elles furent enfermées jusqu'au jour de leur mort, vivant dans le veuvage.
\TextTitle{Bataille contre Schéba ; Joab tue Amasa}
\VS{4}Puis le roi dit à Amasa : Rassemble-moi dans trois jours les hommes de Juda ; et toi, sois ici présent.
\VS{5}Amasa donc s'en alla pour rassembler Juda ; mais il tarda au-delà du temps que le roi lui avait fixé.
\VS{6}Alors David dit à Abischaï : Maintenant Schéba, fils de Bicri, nous fera plus de mal qu'Absalom. Prends toi-même les serviteurs de ton maître et poursuis-le, de peur qu'il ne trouve des villes fortes, et que nous ne le perdions de vue.
\VS{7}Et Abischaï partit, suivi des gens de Joab, des Kéréthiens et des Péléthiens, et de tous les hommes forts ; ils sortirent de Jérusalem, pour poursuivre Schéba, fils de Bicri.
\VS{8}Et comme ils furent près de la grande pierre qui est à Gabaon, Amasa vint au-devant d'eux. Joab était ceint d'une épée par-dessus les habits dont il était revêtu ; elle était attachée à ses reins dans le fourreau, et comme il s'avançait, elle tomba.
\VS{9}Joab dit à Amasa : Te portes-tu bien, mon frère ? Puis Joab prit de sa main droite la barbe d'Amasa pour l'embrasser.
\VS{10}Amasa ne prit point garde à l'épée qui était dans la main de Joab ; et Joab l'en frappa au ventre et répandit ses entrailles à terre, sans le frapper une seconde fois. Et il mourut. Après cela, Joab et Abischaï, son frère, poursuivirent Schéba, fils de Bicri.
\VS{11}Un des serviteurs de Joab resta près d'Amasa, et il disait : Qui aime Joab et qui est pour David ? Qu'il suive Joab !
\VS{12}Amasa était vautré dans son sang au milieu de la route ; et cet homme-là, ayant vu que tout le peuple s'arrêtait, poussa Amasa hors de la route dans un champ, et jeta un vêtement sur lui, lorsqu'il vit que tous ceux qui arrivaient près de lui s'arrêtaient.
\VS{13}Quand il fut ôté de la route, tous les hommes qui suivaient Joab passaient au-delà, afin de poursuivre Schéba, fils de Bicri.
\TextTitle{La révolte de Schéba}
\VS{14}Joab passa par toutes les tribus d'Israël jusqu'à Abel-Beth-Maaca, avec tous les Bériens, qui s'étaient assemblés et qui l'avaient suivi.
\VS{15}Les gens donc de Joab vinrent assiéger Schéba dans Abel-Beth-Maaca, et ils élevèrent contre la ville une terrasse qui atteignait le rempart. Tout le peuple qui était avec Joab rompait la muraille pour la faire tomber.
\VS{16}Lorsqu'une femme sage de la ville se mit à crier : Ecoutez, écoutez ! Dites, je vous prie, à Joab : Approche jusqu'ici, je veux te parler !
\VS{17}Il s'approcha d'elle, et la femme dit : Es-tu Joab ? Il répondit : Je le suis. Elle lui dit : Ecoute les paroles de ta servante. Il répondit : J'écoute.
\VS{18}Et elle dit : Autrefois on avait coutume de dire : Que l'on consulte Abel ! Et tout se terminait ainsi.
\VS{19}Je suis une des cités paisibles et fidèles en Israël ; tu cherches à détruire une ville qui est une mère en Israël ! Pourquoi détruirais-tu l'héritage de Yahweh ?
\VS{20}Joab lui répondit : A Dieu ne plaise, à Dieu ne plaise que je détruise et que je ruine !
\VS{21}La chose n'est pas ainsi. Mais un homme de la montagne d'Ephraïm, nommé Schéba, fils de Bicri, a levé sa main contre le roi David ; livrez-le, lui seul, et je m'éloignerai de la ville. La femme dit à Joab : Voici, sa tête te sera jetée par-dessus la muraille.
\VS{22}Et la femme alla vers tout le peuple, et leur parla sagement ; et ils coupèrent la tête de Schéba, fils de Bicri, et la jetèrent à Joab. Alors il sonna du shofar ; et on se dispersa loin de la ville, et chacun s'en alla dans sa tente. Puis Joab retourna vers le roi à Jérusalem.
\VS{23}Joab était le chef de toute l'armée d'Israël ; Benaja, fils de Jehojada, était à la tête des Kéréthiens et des Péléthiens ;
\VS{24}et Adoram était préposé aux impôts ; Josaphat, fils d'Achilud, était archiviste.
\VS{25}Scheja était le secrétaire ; Tsadok et Abiathar étaient les sacrificateurs ;
\VS{26}et Ira de Jaïr était ministre d'Etat de David.
\Chap{21}
\TextTitle{Vengeance des Gabaonites sur la maison de Saül}
\VerseOne{}Or il y eut du temps de David, une famine qui dura trois ans de suite. David chercha la face de Yahweh, et Yahweh lui répondit : C'est à cause de Saül et de sa maison sanguinaire, parce qu'il a fait mourir les Gabaonites.
\VS{2}Alors le roi appela les Gabaonites pour leur parler. Or les Gabaonites n'étaient point des enfants d'Israël, mais un reste des Amoréens ; les enfants d'Israël leur avaient juré de les laisser vivre\FTNT{Jos. 9.}, mais Saül dans son zèle pour les enfants d'Israël et de Juda, avait cherché à les faire mourir.
\VS{3}Et David dit aux Gabaonites : Que ferais-je pour vous, et par quel moyen vous apaiserai-je, afin que vous bénissiez l'héritage de Yahweh ?
\VS{4}Les Gabaonites lui répondirent : Il ne s'agit pas pour nous d'argent ou d'or avec Saül et avec sa maison, et ce n'est pas à nous de faire mourir un homme en Israël. Le roi leur dit : Que voulez-vous donc que je fasse pour vous ?
\VS{5}Ils répondirent au roi : Puisque cet homme nous a consumés, et qu'il avait résolu de nous exterminer pour nous faire disparaître de tout le territoire d'Israël,
\VS{6}qu'on nous livre sept hommes d'entre ses fils, et nous les pendrons devant Yahweh à Guibea de Saül, l'élu de Yahweh. Et le roi dit : Je vous les livrerai.
\VS{7}Le roi épargna Mephiboscheth, fils de Jonathan, fils de Saül, à cause du serment que David et Jonathan, fils de Saül, avaient fait entre eux, devant Yahweh.
\VS{8}Mais le roi prit les deux fils que Ritspa, fille d'Ajja, avait enfantés à Saül, Armoni et Mephiboscheth, et les cinq fils que Mérab, fille de Saül, avait enfantés à Adriel de Mehola, fils de Barzillaï,
\VS{9}et il les livra entre les mains des Gabaonites, qui les pendirent sur la montagne, devant Yahweh. Tous les sept furent tués ensemble ; on les fit mourir dans les premiers jours de la moisson, au commencement de la moisson des orges.
\VS{10}Alors Ritspa, fille d'Ajja, prit un sac et l'étendit sous elle au-dessus d'un rocher, depuis le commencement de la moisson jusqu'à ce que l'eau du ciel tombât sur eux ; et elle ne permit pas aux oiseaux du ciel de s'approcher d'eux pendant le jour, ni aux bêtes des champs pendant la nuit.
\VS{11}On informa David de ce qu'avait fait Ritspa, fille d'Ajja, concubine de Saül.
\VS{12}Et David alla prendre les os de Saül et les os de Jonathan, son fils, chez les habitants de Jabès en Galaad, qui les avaient enlevés de la place de Beth-Schan, où les Philistins les avaient pendus lorsqu'ils tuèrent Saül à Guilboa.
\VS{13}Il emporta de là les os de Saül et les os de Jonathan, son fils ; on recueillit aussi les os de ceux qui avaient été pendus.
\VS{14}On les enterra avec les os de Saül et de Jonathan, son fils, au pays de Benjamin, à Tséla, dans le sépulcre de Kis, père de Saül. Et l'on fit tout ce que le roi avait ordonné. Après cela, Dieu fut apaisé envers le pays.
\TextTitle{Nouvelles batailles contre les Philistins}
\VS{15}Il y eut encore une guerre entre les Philistins et Israël. David y était allé, et ses serviteurs avec lui, et ils combattirent tellement contre les Philistins que David défaillait.
\VS{16}Et Jischbi-Benob, qui était un des enfants de Rapha, eut l'intention de tuer David ; il avait une lance dont le fer pesait trois cents sicles d'airain, et il était ceint d'une armure neuve.
\VS{17}Mais Abischaï, fils de Tseruja, vint au secours de David, frappa le Philistin, et le tua. Alors les gens de David jurèrent, en disant : Tu ne sortiras plus avec nous à la bataille, de peur que tu n'éteignes la lampe d'Israël.
\VS{18}Après cela, il y eut encore une autre guerre à Gob avec les Philistins. Sibbecaï, le Huschatite, tua Saph, qui était un des enfants de Rapha.
\VS{19}Il y eut encore une autre guerre à Gob avec les Philistins. Et Elchanan, fils de Jaaré-Oreguim, de Bethléhem, tua Goliath de Gath, qui avait une lance dont le bois était comme une ensouple de tisserand.
\VS{20}Il y eut encore une guerre à Gath. Il s'y trouva un homme de haute taille, qui avait six doigts à chaque main, et six orteils à chaque pied, en tout vingt-quatre, lequel était aussi issu de Rapha.
\VS{21}Il jeta un défi à Israël ; et Jonathan, fils de Schimea, frère de David, le tua.
\VS{22}Ces quatre-là étaient nés à Gath, de la race de Rapha. Ils moururent par les mains de David, ou par les mains de ses serviteurs.
\Chap{22}
\TextTitle{Louange à Yahweh, le Dieu qui délivre}
\VerseOne{}Après cela, David adressa à Yahweh les paroles de ce cantique, le jour où Yahweh l'eut délivré de la main de tous ses ennemis, et de la main de Saül.
\VS{2}Il dit : Yahweh est mon rocher, ma forteresse, mon libérateur.
\VS{3}Dieu est mon rocher, où je trouve un abri, mon bouclier et la force qui me sauve, ma haute retraite et mon refuge. Ô mon Sauveur ! Tu me délivres de la violence.
\VS{4}Je m'écrie : Loué soit Yahweh! Et je suis délivré de mes ennemis\FTNT{Ps. 18:4.}.
\VS{5}Car les flots de la mort m'avaient environné, les torrents des méchants m'avaient épouvanté ;
\VS{6}les liens du scheol m'avaient entouré, les filets de la mort m'avaient surpris.
\VS{7}Dans ma détresse, j'ai invoqué Yahweh, j'ai crié à mon Dieu ; de son palais, il a entendu ma voix, et mon cri est parvenu à ses oreilles.
\VS{8}Alors la terre fut ébranlée et trembla, les fondements des cieux s'agitèrent, et ils furent ébranlés, parce qu'il était irrité.
\VS{9}Une fumée montait de ses narines, et de sa bouche sortait un feu dévorant : Il en jaillissait des charbons embrasés.
\VS{10}Il abaissa les cieux, et descendit : Il y avait une épaisse nuée sous ses pieds.
\VS{11}Il était monté sur un chérubin, et il volait, il paraissait sur les ailes du vent.
\VS{12}Il mit autour de lui les ténèbres pour tabernacle, des amas d'eaux, des nuées épaisses.
\VS{13}Des charbons de feu étaient embrasés de la splendeur qui le précédait.
\VS{14}Yahweh tonna des cieux, et le Très-Haut fit retentir sa voix ;
\VS{15}il lança des flèches, et dispersa mes ennemis ;  il lança des éclairs, et les mit en déroute.
\VS{16}Alors le fond de la mer apparut, et les fondements de la terre habitable furent mis à découvert, par la menace de Yahweh, par le souffle du vent de sa colère.
\VS{17}Il étendit sa main d'en haut, il me saisit, il me retira des grandes eaux ;
\VS{18}il me délivra de mon ennemi puissant, de ceux qui me haïssaient, car ils étaient plus forts que moi.
\VS{19}Ils m'avaient surpris au jour de ma détresse, mais Yahweh fut mon appui.
\VS{20}Il m'a mis au large, il m'a sauvé, parce qu'il a pris son plaisir en moi.
\VS{21}Yahweh m'a traité selon ma droiture, il m'a rendu selon la pureté de mes mains ;
\VS{22}parce que j'ai gardé les voies de Yahweh, et que je ne me suis point détourné de mon Dieu.
\VS{23}Toutes ses ordonnances ont été devant moi, et je ne me suis point écarté de ses lois.
\VS{24}J'ai été intègre envers lui, et je me suis gardé de mon iniquité.
\VS{25}Yahweh donc m'a rendu selon ma droiture, selon ma pureté devant ses yeux.
\VS{26}Avec celui qui est bon tu es bon, avec l'homme intègre tu es intègre,
\VS{27}avec celui qui est pur tu te montres pur, mais avec le pervers tu agis selon sa perversité.
\VS{28}Tu sauves le peuple qui s'humilie, et de ton regard, tu abaisses les orgueilleux.
\VS{29}Tu es ma lampe, ô Yahweh ! Et Yahweh éclaire mes ténèbres.
\VS{30}Avec toi je me précipite sur une troupe en armes, avec mon Dieu je franchis une muraille.
\VS{31}La voie de Dieu est parfaite, la parole de Yahweh est éprouvée ; il est le bouclier de tous ceux qui se confient en lui.
\VS{32}Car qui est Dieu, si ce n'est Yahweh ? Et qui est un rocher, si ce n'est notre Dieu ?
\VS{33}C'est Dieu qui est ma puissante forteresse, et qui me conduit dans la voie droite.
\VS{34}Il a rendu mes pieds semblables à ceux des biches, et il me fait tenir debout sur mes lieux élevés.
\VS{35}Il exerce mes mains au combat, et mes bras tendent l'arc d'airain.
\VS{36}Tu me donnes le bouclier de ton salut, et ta bonté me fait devenir plus grand.
\VS{37}Tu élargis le chemin sous mes pas, et mes pieds ne chancellent point.
\VS{38}Je poursuis mes ennemis, et je les détruis ; je ne reviens qu'après les avoir exterminés.
\VS{39}Je les anéantis, je les transperce, et ils ne se relèvent plus ; ils tombent sous mes pieds.
\VS{40}Tu me ceins de force pour le combat, tu fais plier sous moi mes adversaires.
\VS{41}Tu fais tourner le dos à mes ennemis devant moi, et j'extermine ceux qui me haïssent.
\VS{42}Ils regardent autour d'eux, et il n'y a point de sauveur ! Ils crient à Yahweh, mais il ne leur répond pas !
\VS{43}Je les broie comme la poussière de la terre, je les écrase, je les foule, comme la boue des rues.
\VS{44}Tu me délivres des dissensions de mon peuple ; tu me gardes pour être chef des nations ; un peuple que je ne connaissais pas m'est asservi.
\VS{45}Les fils de l'étranger me flattent, dès qu'ils ont entendu parler de moi, ils se sont rendus obéissants.
\VS{46}Les fils de l'étranger défaillent, et sortent tremblants de leurs forteresses.
\VS{47}Yahweh est vivant, et béni soit mon rocher ! Que Dieu, le rocher de mon salut, soit exalté,
\VS{48}le Dieu qui me donne vengeance, qui m'assujettit les peuples,
\VS{49}et qui me fait échapper à mes ennemis ! Tu m'élèves au-dessus de mes adversaires, tu me délivres de l'homme violent.
\VS{50}C'est pourquoi, ô Yahweh, je te louerai parmi les nations, et je chanterai des psaumes à ton Nom.
\VS{51}C'est lui qui est la tour de délivrance de son roi, et qui fait miséricorde à son oint, à David, et à sa postérité, à jamais.
\Chap{23}
\TextTitle{Paroles prophétiques de David}
\VerseOne{}Voici les dernières paroles de David. Parole de David, fils d'Isaï, parole de l'homme qui a été élevé, de l'oint du Dieu de Jacob, du chantre agréable d'Israël :
\VS{2}L'Esprit de Yahweh parle par moi, et sa parole est sur ma langue.
\VS{3}Le Dieu d'Israël a parlé, le Rocher\FTNT{Voir commentaire Es. 8 :13-17.} d'Israël m'a dit : Celui qui règne parmi les hommes avec justice, celui qui règne dans la crainte de Dieu,
\VS{4}est comme la lumière du matin quand le soleil se lève, un matin sans nuage ; son éclat fait germer de la terre la verdure après la pluie.
\VS{5}N'en est-il pas ainsi de ma maison devant Dieu, puisqu'il a traité avec moi une alliance éternelle, bien ordonnée, et gardée ? Tout mon salut et tout mon plaisir, ne les fera-t-il pas germer ?
\VS{6}Mais les méchants sont tous comme des épines que l'on jette au loin, parce qu'on ne les prend pas avec la main ;
\VS{7}celui qui les touche, s'arme du fer ou du bois d'une lance, et on les brûle au feu sur place.
\TextTitle{Les vaillants hommes de David\FTNTT{1 Ch. 11:10-47.}}
\VS{8}Voici les noms des vaillants hommes qui étaient au service de David. Joscheb-Basschébeth, le Tachkemonite, était l'un des principaux chefs. C'était Hadino le Hetsnite, qui eut le dessus sur huit cents hommes qu'il tua en une seule fois.
\VS{9}Après lui, Eléazar, fils de Dodo, fils d'Achochi. Il était l'un des trois vaillants hommes qui étaient avec David lorsqu'ils défièrent les Philistins rassemblés pour combattre, tandis que les hommes d'Israël se retiraient.
\VS{10}Il se leva, et frappa les Philistins jusqu'à ce que sa main fut lasse et qu'elle restât attachée à l'épée. Ce jour-là, Yahweh opéra une grande délivrance. Le peuple revint après Eléazar, seulement pour prendre les dépouilles.
\VS{11}Après lui, Schamma, fils d'Agué d'Harar. Les Philistins s'étaient rassemblés en troupe. Il y avait là une parcelle de champ pleine de lentilles ; et le peuple fuyait devant les Philistins.
\VS{12}Schamma se mit au milieu de cette parcelle, la défendit, et frappa les Philistins. Et Yahweh opéra une grande délivrance.
\VS{13}Trois des trente chefs descendirent au temps de la moisson et vinrent vers David, dans la caverne d'Adullam, lorsqu'une troupe de Philistins était campée dans la vallée des Rephaïm.
\VS{14}David était alors dans la forteresse, et la garnison des Philistins était en ce temps-là à Bethléhem.
\VS{15}Et David eut un désir, et dit : Qui est-ce qui me fera boire de l'eau de la citerne qui est à la porte de Bethléhem ?
\VS{16}Alors ces trois vaillants hommes passèrent au travers du camp des Philistins, et puisèrent de l'eau de la citerne qui est à la porte de Bethléhem. Ils l'apportèrent, et ils la présentèrent à David ; mais il ne voulut pas la boire, et il la répandit devant Yahweh.
\VS{17}Car il dit : Loin de moi, ô Yahweh, de faire une telle chose ! N'est-ce pas le sang de ces hommes qui sont allés au péril de leur vie ? Il ne voulut pas la boire. Voilà ce que firent ces trois vaillants hommes.
\VS{18}Il y avait aussi Abischaï, frère de Joab, fils de Tseruja, qui était le chef des trois. Il brandit sa lance sur trois cents hommes, les blessa à mort ; et il eut du renom parmi les trois.
\VS{19}Il était le plus considéré des trois, et il fut leur chef ; cependant il n'égala point les trois premiers.
\VS{20}Benaja, fils de Jehojada, fils d'un vaillant homme de Kabtseel, rempli de force, avait fait de grands exploits. Il frappa deux des plus puissants hommes de Moab. Il descendit au milieu d'une fosse, où il frappa un lion, un jour de neige.
\VS{21}Il frappa aussi un Egyptien d'un aspect formidable et ayant une lance à la main ; Benaja descendit contre lui avec un bâton, arracha la lance de la main de l'Egyptien, et s'en servit pour le tuer.
\VS{22}Benaja, fils de Jehojada, fit ces choses-là ; et fut illustre parmi les trois vaillants hommes.
\VS{23}Il était le plus considéré des trente ; mais il n'égala pas les trois premiers. C'est pourquoi David l'établit dans son conseil secret.
\VS{24}Asaël, frère de Joab, était des trente. Elchanan, fils de Dodo, de Bethléhem.
\VS{25}Schamma, de Harod. Elika, de Harod.
\VS{26}Hélets, de Péleth. Ira, fils d'Ikkesch, de Tekoa.
\VS{27}Abiézer, d'Anathoth. Mebunnaï, de Huscha.
\VS{28}Tsalmon, d'Achoach. Maharaï, de Nethopha.
\VS{29}Héleb, fils de Baana, de Nethopha. Ittaï, fils de Ribaï, de Guibea des fils de Benjamin.
\VS{30}Benaja, de Pirathon. Hiddaï, de Nachalé-Gaasch.
\VS{31}Abi-Albon, d'Araba. Azmaveth, de Barchum.
\VS{32}Eliachba, de Schaalbon. Bené-Jaschen. Jonathan.
\VS{33}Schamma, d'Harar. Achiam, fils de Scharar, d'Arar.
\VS{34}Eliphéleth, fils d'Achasbaï, fils d'un Maacathien. Eliam, fils d'Achitophel, de Guilo.
\VS{35}Hetsro, de Carmel. Paaraï, d'Arab.
\VS{36}Jigueal, fils de Nathan, de Tsoba. Bani, de Gad.
\VS{37}Tsélek, l'Ammonite. Naharaï, de Beéroth, qui portait les armes de guerre de Joab, fils de Tseruja.
\VS{38}Ira, de Jéther. Gareb, de Jéther.
\VS{39}Urie, le Héthien. En tout, trente-sept.
\Chap{24}
\TextTitle{Péché de David ; plaie mortelle sur Israël\FTNTT{1 Ch. 21:1-17.}}
\VerseOne{}La colère de Yahweh s'enflamma encore contre Israël, parce que David fut incité contre eux, en disant : Va, fais le dénombrement d'Israël et de Juda\FTNT{1 Ch. 21.}.
\VS{2}Le roi dit donc à Joab, chef de l'armée qui se trouvait près de lui : Parcours toutes les tribus d'Israël, depuis Dan jusqu'à Beer-Schéba ; et dénombre le peuple, afin que je sache le nombre du peuple.
\VS{3}Joab dit au roi : Que Yahweh, ton Dieu, veuille augmenter ton peuple cent fois plus, et que les yeux du roi mon seigneur le voient ! Mais pourquoi le roi mon seigneur prend-il plaisir à cela ?
\VS{4}Néanmoins, la parole du roi l'emporta sur Joab, et sur les chefs de l'armée ; et Joab et les chefs de l'armée sortirent de la présence du roi pour dénombrer le peuple d'Israël.
\VS{5}Ils passèrent le Jourdain, et ils campèrent à Aroër, à droite de la ville qui est au milieu de la vallée du torrent de Gad, et vers Jaezer.
\VS{6}Ils allèrent en Galaad et dans le territoire de ceux qui habitent vers le bas du pays de Thachthim-Hodschi. Ils allèrent à Dan-Jaan, et aux environs de Sidon.
\VS{7}Ils vinrent jusqu'à la forteresse de Tyr, et dans toutes les villes des Héviens et des Cananéens. Ils sortirent vers le midi de Juda à Beer-Schéba.
\VS{8}Ainsi ils parcoururent tout le pays, et arrivèrent à Jérusalem au bout de neuf mois et vingt jours.
\VS{9}Et Joab donna au roi le rôle du dénombrement du peuple : Il y avait en Israël huit cent mille hommes de guerre tirant l'épée, et en Juda cinq cent mille hommes.
\VS{10}Alors David sentit battre son cœur, après qu'il eut fait ainsi dénombrer le peuple. Et David dit à Yahweh : J'ai commis un grand péché en faisant cela ! Mais, je te prie, ô Yahweh, de pardonner l'iniquité de ton serviteur, car j'ai agi en insensé !
\VS{11}Après cela, David se leva dès le matin, et la parole de Yahweh fut adressée à Gad le prophète, qui était le voyant de David :
\VS{12}Va dire à David : Ainsi parle Yahweh : J'apporte trois choses contre toi ; choisis l'une d'elles afin que je te la fasse.
\VS{13}Gad alla vers David, et lui rapporta cela en disant : Que veux-tu qu'il t'arrive : Sept ans de famine sur ton pays, ou que durant trois mois tu fuies devant tes ennemis qui te poursuivront, ou que durant trois jours la peste soit dans ton pays ? Choisis maintenant, et regarde ce que tu veux que je réponde à celui qui m'a envoyé.
\VS{14}David répondit à Gad : Je suis dans une très grande détresse ! Tombons entre les mains de Yahweh, car ses compassions sont en grand nombre ; mais que je ne tombe pas entre les mains des hommes !
\VS{15}Yahweh envoya donc la peste en Israël, depuis le matin jusqu'au temps fixé ; et depuis Dan jusqu'à Beer-Schéba, il mourut soixante-dix mille hommes parmi le peuple.
\VS{16}Mais quand l'ange étendait sa main sur Jérusalem pour la ravager, Yahweh se repentit de ce mal et dit à l'ange qui ravageait le peuple : C'est assez ! Retire maintenant ta main. Or l'Ange de Yahweh était près de l'aire d'Aravna, le Jébusien.
\VS{17}Car David voyant l'ange qui frappait le peuple, parla à Yahweh, et dit : Voici, c'est moi qui ai péché ! C'est moi qui ai commis l'iniquité ; mais ces brebis, qu'ont-elles fait ? Je te prie que ta main soit contre moi et contre la maison de mon père !
\TextTitle{Sacrifice de David ; Yahweh met fin à la plaie\FTNTT{1 Ch. 21:18-30.}}
\VS{18}Ce jour-là, Gad vint vers David, et lui dit : Monte, et dresse un autel à Yahweh dans l'aire d'Aravna, le Jébusien.
\VS{19}Et David monta, selon la parole de Gad, comme Yahweh l'avait ordonné.
\VS{20}Aravna regarda, et vit le roi et ses serviteurs qui venaient vers lui ; et Aravna sortit, et se prosterna devant le roi, le visage contre terre.
\VS{21}Aravna dit : Pourquoi le roi mon seigneur vient-il vers son serviteur ? Et David répondit : Pour acheter ton aire, et y bâtir un autel à Yahweh, afin que cette plaie se retire de dessus le peuple.
\VS{22}Aravna dit à David : Que le roi mon seigneur prenne et offre ce qu'il lui plaira ; vois les bœufs seront pour l'holocauste, et les chars avec l'attelage de bœufs serviront de bois.
\VS{23}Aravna donna tout cela au roi. Et Aravna dit au roi : Que Yahweh, ton Dieu, te soit favorable !
\VS{24}Mais le roi répondit à Aravna : Non ! Je veux l'acheter de toi pour un certain prix, et je n'offrirai point à Yahweh, mon Dieu, des holocaustes qui ne me coûtent rien. Ainsi, David acheta l'aire et les bœufs pour cinquante sicles d'argent.
\VS{25}David bâtit là un autel à Yahweh, et offrit des holocaustes et des sacrifices d'offrande de paix. Alors Yahweh fut apaisé envers le pays, et la plaie se retira d'Israël.
\PPE{}
\end{multicols}

%\clearpage\ShortTitle{1 Rois}\BookTitle{1 Rois}\BFont
\noindent\hrulefill
{\footnotesize
\textit{
\bigskip
{\centering{}
\\Auteur : Inconnu
\\(Heb. : Melakhim)
\\Signification : Roi, Règne
\\Thème : Unité du royaume après le schisme
\\Date de rédaction : 6ème siècle av. J.-C.\\}
}
%\bigskip
\textit{
\\Ce livre relate la vie de Salomon : son accession à la royauté après la mort de son père David, son alliance avec Dieu qui lui accorda une sagesse exceptionnelle ainsi que la construction du temple de Yahweh et du palais royal.
%\bigskip
\\Les premières années du règne de Salomon furent exemplaires. Malheureusement, il ne fit pas preuve de la même piété
que son père et développa une affection particulière pour les femmes étrangères qui l’entrainèrent dans l’idolâtrie. A sa
mort, son fils Roboam accéda au pouvoir et provoqua la division du royaume en deux : d’un côté les dix tribus du nord qui gardèrent le nom d’Israël, gouvernées par Jéroboam, et de l’autre côté les deux tribus du sud, Juda et Benjamin, qui demeurèrent sous l’autorité de Roboam.
%\bigskip
\\Ce livre raconte également le règne et la conduite parfois abominable des rois d’Israël et de Juda jusqu’à Achab et Josaphat.
Il présente la puissance de l’appel prophétique d’Elie, le Tischbite, que Dieu suscita pour ramener son peuple à lui et
montrer sa souveraineté.\bigskip
}
}
\par\nobreak\noindent\hrulefill
\begin{multicols}{2}
\Chap{1}
\TextTitle{Fin de la vie de David}
\VerseOne{}Le roi David était vieux et avancé en âge ; on le couvrait de vêtements parce qu’il ne parvenait point à se réchauffer.
\VS{2}Ses serviteurs lui dirent : Que l'on cherche pour le roi notre seigneur, une jeune fille vierge ; qu’elle se tienne devant le roi, qu’elle le soigne et qu'elle dorme en son sein, afin que le roi notre seigneur, se réchauffe.
\VS{3}On chercha donc dans toutes les contrées d'Israël une jeune et belle femme, et on trouva Abischag la Sunamite, que l’on amena auprès du roi.
\VS{4}Cette jeune femme était fort belle. Elle prit soin du roi et le servit, mais le roi ne la connut point.
\TextTitle{Conspiration d’Adonija pour régner sur Israël}
\VS{5}Alors Adonija, fils de Haggith, se laissa emporter par l’orgueil en disant : Je suis le roi ! Il se procura un char, des cavaliers et cinquante hommes qui couraient devant lui.
\VS{6}Son père ne lui avait jamais fait un reproche jusqu’à ce jour-là, en disant : Pourquoi agis-tu ainsi ? Adonija était très beau de figure, il était né après Absalom.
\VS{7}Il s’entendit avec Joab, fils de Tseruja, et avec le sacrificateur Abiathar, qui embrassèrent son parti.
\VS{8}Mais le sacrificateur Tsadok, Benaja fils de Jehojada, Nathan le prophète, Schimeï, Reï et les vaillants hommes de David ne furent point du parti d'Adonija.
\VS{9}Or, Adonija fit tuer des brebis, des bœufs et des veaux gras près de la pierre de Zohéleth, qui est auprès d’En-Roguel ; il invita tous ses frères, fils du roi, et tous les hommes de Juda qui étaient au service du roi.
\VS{10}Mais il ne convia point Nathan le prophète, ni Benaja, ni les vaillants hommes, ni Salomon, son frère.
\TextTitle{Opposition de Nathan et Bath-Schéba}
\VS{11}Alors Nathan parla à Bath-Schéba, mère de Salomon, en disant : N'as-tu pas entendu qu'Adonija, fils de Haggith, a été fait roi ? Et David notre Seigneur n'en sait rien.
\VS{12}Maintenant donc viens, je t’en donne le conseil afin que tu sauves ta vie et la vie de ton fils Salomon.
\VS{13}Va, entre chez le roi David et dis-lui : Ô roi, mon seigneur, n'as-tu pas fait serment à ta servante, en disant : ton fils Salomon régnera après moi et sera assis sur mon trône ? Pourquoi donc Adonija règne-t-il ?
\VS{14}Et voici, lorsque tu seras encore là et que tu parleras avec le roi, je viendrai après toi et je confirmerai tes dires.
\VS{15}Bath-Schéba se rendit dans la chambre du roi. Or, le roi était très vieux et Abischag, la Sunamite, le servait.
\VS{16}Bath-Schéba s'inclina et se prosterna devant le roi. Et le roi lui dit : Qu'as-tu ?
\VS{17}Et elle lui répondit : Mon seigneur, tu as juré par Yahweh, ton Dieu à ta servante, en lui disant : Ton fils Salomon régnera après moi et s’assiéra sur mon trône.
\VS{18}Mais maintenant voici, Adonija est proclamé roi ! Et tu ne le sais pas, ô roi, mon seigneur !
\VS{19}Il a fait tuer des bœufs, des veaux gras et des brebis en grand nombre, il a convié tous les fils du roi, avec Abiathar, le sacrificateur, et Joab, chef de l'armée, mais il n'a point convié ton serviteur Salomon.
\VS{20}Ô roi mon seigneur ! Les yeux de tout Israël sont sur toi, afin que tu lui fasses connaître qui s’assiéra sur le trône du roi mon seigneur après lui.
\VS{21}Aussi, lorsque le roi mon seigneur sera endormi avec ses pères, nous serons traités comme des coupables, moi et mon fils Salomon.
\VS{22}Tandis qu’elle parlait encore avec le roi, Nathan le prophète se présenta.
\VS{23}On l’annonça au roi en disant : Voici Nathan le prophète ! Il se présenta devant le roi et se prosterna devant lui, le visage contre terre.
\VS{24}Et Nathan dit : Ô roi mon seigneur ! Tu as dit : Adonija régnera après moi et sera assis sur mon trône !
\VS{25}Car il est descendu aujourd'hui, il a sacrifié des bœufs, des veaux gras et des brebis en grand nombre. Il a convié tous les fils du roi, les chefs de l'armée et le sacrificateur Abiathar. Et voici, ils mangent et boivent devant lui ; ils disent : Vive le roi Adonija !
\VS{26}Mais il n'a convié ni moi, ton serviteur, ni le sacrificateur Tsadok, ni Benaja, fils de Jehojada, ni Salomon ton serviteur.
\VS{27}Est-ce bien par ordre de mon seigneur le roi que cette chose a lieu et sans que tu aies fait connaître à ton serviteur quel est celui qui doit s'asseoir sur le trône du roi mon seigneur après lui ?
\VS{28}Et le roi David répondit, en disant : Appelez-moi Bath-Schéba ; elle entra et se présenta devant le roi.
\VS{29}Alors le roi jura et dit : Yahweh, qui m'a délivré de toute détresse, est vivant !
\VS{30}Comme je te l'ai juré par Yahweh, le Dieu d'Israël, en disant : Ton fils Salomon régnera après moi et sera assis sur mon trône à ma place ; ainsi ferai-je aujourd'hui.
\VS{31}Alors Bath-Schéba s'inclina le visage contre terre et se prosterna devant le roi en disant : Que le roi David mon seigneur vive éternellement !
\VS{32}Et le roi David dit : Appelez-moi le sacrificateur Tsadok, le prophète Nathan et Benaja, fils de Jehojada ; et ils se présentèrent devant le roi.
\VS{33}Le roi leur dit : Prenez avec vous les serviteurs de votre seigneur, faites monter mon fils Salomon sur ma mule, et faites-le descendre à Guihon.
\VS{34}Que Tsadok le sacrificateur et Nathan le prophète, l'oignent en ce lieu-là pour roi sur Israël, puis vous sonnerez du shofar et vous direz : Vive le roi Salomon !
\VS{35}Vous monterez après lui et il viendra, il s'assiéra sur mon trône et il régnera à ma place ; car j'ai ordonné qu'il soit le chef d'Israël et de Juda.
\VS{36}Et Benaja fils de Jehojada répondit au roi : Amen ! Ainsi parle Yahweh, le Dieu de mon seigneur le roi !
\VS{37}Comme Yahweh a été avec mon seigneur le roi, qu'il soit aussi avec Salomon, et qu'il élève son trône encore plus que le trône du roi David mon seigneur !
\TextTitle{Salomon oint roi d’Israël par Tsadok\FTNTT{cp. 1 Ch. 29:22}}
\VS{38}Puis Tsadok le sacrificateur descendit avec Nathan le prophète et Benaja, fils de Jehojada, les Kéréthiens et les Péléthiens ; ils firent monter Salomon sur la mule du roi David et le menèrent à Guihon.
\VS{39}Tsadok le sacrificateur prit du tabernacle une corne d'huile dont il oignit Salomon. On sonna du shofar et tout le peuple dit : Vive le roi Salomon !
\VS{40}Et tout le monde monta après lui et le peuple jouait de la flûte, en se livrant à une grande joie, au point que la terre se fendait par leurs cris.
\VS{41}Ce bruit fut entendu d’Adonija et de tous les conviés qui étaient avec lui comme ils achevaient de manger ; et Joab entendant le son du shofar, dit : Pourquoi ce bruit de la ville en tumulte ?
\VS{42}Et comme il parlait encore, voici Jonathan, fils du sacrificateur Abiathar, arriva et Adonija lui dit : Entre, car tu es un vaillant homme et tu apportes de bonnes nouvelles.
\VS{43}Oui ! répondit Jonathan à Adonija : Le roi David, notre seigneur, a établi Salomon roi.
\VS{44}Et le roi a envoyé avec lui Tsadok le sacrificateur, Nathan le prophète, Benaja, fils de Jehojada, les Kéréthiens, et les Péléthiens, et ils l'ont fait monter sur la mule du roi.
\VS{45}Tsadok le sacrificateur, et Nathan le prophète l'ont oint pour roi à Guihon, d'où ils sont remontés avec joie, et la ville est ainsi émue ; c'est là le bruit que vous avez entendu.
\VS{46}Salomon s'est même assis sur le trône royal.
\VS{47}Et les serviteurs du roi sont venus pour bénir le roi David notre seigneur, en disant : Que ton Dieu rende le nom de Salomon encore plus grand que ton nom, et qu'il élève son trône encore plus que ton trône ! Et le roi s'est prosterné sur son lit.
\VS{48}Le roi a ainsi parlé : Béni soit Yahweh, le Dieu d'Israël, qui a aujourd’hui établi sur mon trône un successeur, et qui m’a permis de le voir !
\VS{49}Alors tous les conviés d’Adonija furent saisis de frayeur, ils se levèrent et s'en allèrent chacun son chemin.
\VS{50}Adonija eut peur de Salomon ; il se leva aussi et s'en alla empoigner les cornes de l'autel.
\VS{51}On vint l’apprendre à Salomon, en disant : Voici Adonija a peur du roi Salomon et il a saisi les cornes de l'autel, en disant : Que le roi Salomon me jure aujourd'hui qu'il ne fera point mourir son serviteur par l'épée.
\VS{52}Et Salomon dit : A l’avenir, s’il se comporte en homme de bien il ne tombera pas un seul de ses cheveux à terre ; mais s'il se trouve du mal en lui, il mourra.
\VS{53}Alors le roi Salomon envoya des personnes qui le firent descendre de l'autel. Il vint et se prosterna devant le roi Salomon, et Salomon lui dit : Va dans ta maison.
\Chap{2}
\TextTitle{Dernières paroles de David à Salomon}
\VerseOne{}David approchait du moment de sa mort, et il donna ses ordres à Salomon, son fils, en disant :
\VS{2}Je m'en vais par le chemin de toute la terre, fortifie-toi et comporte-toi en homme.
\VS{3}Observe les commandements de Yahweh, ton Dieu, en marchant dans ses voies, en gardant ses statuts, ses commandements, ses ordonnances et ses préceptes, selon ce qui est écrit dans la loi de Moïse, afin que tu réussisses dans tout ce que tu feras et dans tout ce que tu entreprendras ;
\VS{4}et afin que s’accomplisse cette parole de Yahweh déclarée sur moi : Si tes fils prennent garde à leur voie, pour marcher devant moi dans la vérité, de tout leur cœur et de toute leur âme, tu ne manqueras jamais de successeur sur le trône d'Israël.
\VS{5}Tu sais ce que m'a fait Joab, fils de Tseruja et ce qu'il a fait aux deux chefs des armées d'Israël, Abner, fils de Ner, et à Amasa, fils de Jéther, qu'il a tués, en versant pendant la paix le sang de la guerre ; il a mis de ce sang sur la ceinture qu'il avait sur ses reins et sur les chaussures qu'il avait aux pieds.
\VS{6}Tu agiras selon ta sagesse, en sorte que tu ne laisseras point ses cheveux blancs descendre en paix dans le scheol.
\VS{7}Tu traiteras avec bienveillance les fils de Barzillaï, le Galaadite, et ils seront du nombre de ceux qui mangent à ta table ; car ils se sont approchés de moi quand je fuyais Absalom, ton frère.
\VS{8}Voici, tu as avec toi Schimeï, fils de Guéra, le benjamite de Bachurim, qui proféra contre moi des malédictions violentes le jour où je m'en allais à Mahanaïm. Mais il descendit au-devant de moi vers le Jourdain et je lui jurai par Yahweh, en disant : Je ne te ferai point mourir par l'épée.
\VS{9}Maintenant donc tu ne le laisseras point impuni, car tu es sage, pour savoir comment tu dois le traiter ; et tu feras descendre ses cheveux blancs ensanglantés au scheol.
\TextTitle{Mort de David ; début du règne de Salomon\FTNTT{1 Ch. 29:23-30}}
\VS{10}Ainsi David se coucha avec ses pères, il fut enseveli dans la cité de David.
\VS{11}Et le temps que David régna sur Israël fut quarante ans. Il régna sept ans à Hébron et il régna trente-trois ans à Jérusalem.
\VS{12}Et Salomon s'assit sur le trône de David, son père, et son règne fut très affermi.
\TextTitle{Mort d'Adonija}
\VS{13}Alors Adonija, fils de Haggith, vint vers Bath-Schéba, mère de Salomon et elle dit : Amènes-tu la paix ? Et il répondit : Je viens en paix.
\VS{14}Il ajouta : J'ai un mot à te dire. Elle répondit : Parle !
\VS{15}Et il dit : Tu sais bien que le royaume m'appartenait et que tout Israël s'attendait à ce que je règne. Mais la royauté s’est détournée de moi, elle est échue à mon frère parce que Yahweh la lui a donnée.
\VS{16}Maintenant donc je te demande une chose, ne me la refuse point. Elle lui répondit : Parle !
\VS{17}Et il dit : Je te prie, dis au roi Salomon, car il ne te refusera rien, qu'il me donne Abischag, la Sunamite, pour femme.
\VS{18}Bath-Schéba répondit : Et bien, je parlerai pour toi au roi.
\VS{19}Bath-Schéba se rendit auprès du roi Salomon pour lui parler en faveur d’Adonija ; et le roi se leva pour aller au-devant d’elle, il se prosterna devant elle, puis il s'assit sur son trône. On plaça un siège pour la mère du roi, et elle s'assit à sa droite.
\VS{20}Elle dit alors : J'ai une petite demande à te faire : ne me la refuse pas ! Et le roi lui répondit : Demande, ma mère, car je ne te la refuserai point.
\VS{21}Et elle dit : Qu'on donne Abischag, la Sunamite, pour femme à Adonija, ton frère.
\VS{22}Mais le roi Salomon répondit à sa mère et dit : Et pourquoi demandes-tu Abischag, la Sunamite, pour Adonija ? Demande plutôt le royaume pour lui, parce qu'il est mon frère aîné ; demande-le pour lui, pour Abiathar, le sacrificateur, et pour Joab, fils de Tseruja !
\VS{23}Alors le roi Salomon jura par Yahweh, en disant : Que Dieu me traite dans toute sa rigueur, si Adonija n'a dit cette parole contre sa propre vie !
\VS{24}Maintenant Yahweh est vivant, lui qui m'a établi, qui m'a fait asseoir sur le trône de David, mon père, et qui m'a donné une maison, selon sa promesse ! Aujourd’hui Adonija mourra.
\VS{25}Et le roi Salomon envoya Benaja, fils de Jehojada, qui le frappa, et Adonija mourut.
\TextTitle{Abiathar dépouillé de ses fonctions au temple}
\VS{26}Puis le roi dit à Abiathar, le sacrificateur : Va-t'en à Anathoth sur tes terres, car tu mérites la mort ; toutefois je ne te ferai point mourir aujourd'hui, parce que tu as porté l'arche du Seigneur Yahweh devant David, mon père ; et parce que tu as eu part à toutes les afflictions de mon père.
\VS{27}Ainsi Salomon dépouilla Abiathar de ses fonctions, afin qu'il ne fût plus sacrificateur de Yahweh pour accomplir la parole de Yahweh, qu'il avait prononcée à Silo contre la maison d'Eli.
\TextTitle{Mort de Joab ; Benaja à la tête de l’armé}
\VS{28}Le bruit en parvint à Joab, qui avait suivi le parti d’Adonija, quoiqu'il n’eût pas suivi le parti d’Absalom. Joab s'enfuit au tabernacle de Yahweh et empoigna les cornes de l'autel.
\VS{29}On alla l’apprendre au roi Salomon, en disant : Joab s'en est enfui dans la tente de Yahweh et il est auprès de l'autel. Salomon envoya Benaja, fils de Jehojada, et lui dit : Va et frappe-le.
\VS{30}Benaja entra dans la tente de Yahweh et dit à Joab : Ainsi a parlé le roi : Sors de là ! Mais il répondit : Non ! Je veux mourir ici. Et Benaja rapporta la chose au roi en disant : Joab m'a parlé ainsi et c’est ainsi qu’il m’a répondu.
\VS{31}Et le roi dit à Benaja : Fais comme il t'a dit, frappe-le et enterre-le ; tu ôteras ainsi de dessus moi et de dessus la maison de mon père le sang que Joab a répandu sans cause.
\VS{32}Et Yahweh fera retomber son sang sur sa tête, car il a frappé deux hommes plus justes et meilleurs que lui et les a tués par l'épée, sans que mon père David n’en sût rien : Abner, fils de Ner, chef de l'armée d'Israël, et Amassa, fils de Jéther, chef de l'armée de Juda.
\VS{33}Leur sang retombera sur la tête de Joab et sur la tête de sa postérité à perpétuité ; mais il y aura paix à toujours de par Yahweh, pour David, pour sa postérité, pour sa maison et pour son trône.
\VS{34}Donc Benaja, fils de Jehojada, monta, et il frappa Joab à mort. On l'ensevelit dans sa maison, dans le désert.
\VS{35}Alors le roi établit Benaja, fils de Jehojada, sur l'armée à la place de Joab ; le roi établit aussi Tsadok sacrificateur à la place d'Abiathar.
\TextTitle{Mort de Schimeï}
\VS{36}Puis le roi fit appeler Schimeï et lui dit : Bâtis-toi une maison à Jérusalem, et demeures-y, et n'en sors point pour aller de côté ou d'autre.
\VS{37}Car sache que le jour où tu en sortiras et que tu passeras le torrent de Cédron, tu mourras certainement ; ton sang sera sur ta tête.
\VS{38}Schimeï répondit au roi : Cette parole est bonne ! Ton serviteur fera tout ce que le roi mon Seigneur a dit. Ainsi Schimeï demeura à Jérusalem plusieurs jours.
\VS{39}Mais il arriva qu'au bout de trois ans, deux serviteurs de Schimeï s'enfuirent vers Akisch, fils de Maaca, roi de Gath, et on le rapporta à Schimeï en disant : Voilà tes serviteurs sont à Gath.
\VS{40}Alors Schimeï se leva, sella son âne, et s'en alla à Gath vers Akisch pour chercher ses serviteurs. Schimeï s'en alla donc et ramena de Gath ses serviteurs.
\VS{41}On rapporta à Salomon que Schimeï était allé de Jérusalem à Gath, et qu'il était de retour.
\VS{42}Et le roi envoya appeler Schimeï, et lui dit : Ne t'avais-je pas fait jurer par Yahweh, et ne t'avais-je pas fait cette déclaration formelle : Sache-le, sache bien que le jour que tu sortiras pour aller de côté ou d’autre, tu mourras ? Et ne me répondis-tu pas : La parole que j'ai entendue est bonne ?
\VS{43}Pourquoi donc n'as-tu pas observé le serment que tu as fait par Yahweh et le commandement que je t'avais donné ?
\VS{44}Le roi dit aussi à Schimeï : Tu sais en ton cœur tout le mal que tu as fait à David, mon père ; c'est pourquoi Yahweh a fait retomber ta méchanceté sur ta tête.
\VS{45}Mais le roi Salomon sera béni, et le trône de David sera affermi devant Yahweh à jamais.
\VS{46}Et le roi donna commission à Benaja fils de Jehojada, qui sortit, et frappa Schimeï et Schimeï mourut. La royauté fut ainsi affermie entre les mains de Salomon.
\Chap{3}
\TextTitle{Salomon s'allie à Pharaon}
\VerseOne{}Or, Salomon s'allia avec Pharaon roi d'Egypte. Il prit pour femme la fille de Pharaon, et l'amena en la cité de David, jusqu'à ce qu'il eût achevé de bâtir sa maison, la maison de Yahweh, et la muraille de Jérusalem tout alentour.
\VS{2}Seulement le peuple sacrifiait dans les hauts lieux, parce que jusqu’alors on n’avait pas bâti de maison au nom de Yahweh.
\Chap{3}
\TextTitle{Salomon demande la sagesse à Yahweh\FTNTT{2 Ch. 1:2-10}}
\VS{3}Salomon aimait Yahweh, il marchait selon les ordonnances de David son père. Seulement, c’était sur les hauts lieux qu’il offrait des sacrifices et des parfums.
\VS{4}Le roi se rendit à Gabaon pour y sacrifier, car c'était le plus grand des hauts lieux. Et Salomon offrit mille holocaustes sur cet autel.
\VS{5}Et Yahweh apparut de nuit à Salomon à Gabaon dans un songe, et Dieu lui dit : Demande ce que tu veux que je te donne.
\VS{6}Et Salomon répondit : Tu as usé d'une grande bienveillance envers ton serviteur David, mon père, parce qu’il a marché devant toi fidèlement, dans la justice, et dans la droiture de cœur envers toi. Tu as gardé cette grande bienveillance envers lui en lui donnant un fils qui est assis sur son trône, comme on le voit aujourd'hui.
\VS{7}Or, maintenant, ô Yahweh mon Dieu ! Tu as fait régner ton serviteur à la place de David, mon père, et je ne suis qu'un jeune homme, je ne sais comment me conduire.
\VS{8}Ton serviteur est parmi ce peuple que tu as choisi, un peuple nombreux qui ne peut être compté ni dénombré à cause de sa multitude.
\VS{9}Accorde donc à ton serviteur un cœur intelligent pour juger ton peuple, pour discerner le bien du mal ! Car qui pourrait juger ce peuple si grand ?
\TextTitle{Yahweh exauce Salomon\FTNTT{2 Ch. 1:11-13}}
\VS{10}Cette demande de Salomon plut à Yahweh.
\VS{11}Et Dieu lui dit : Puisque c’est là ta demande et que tu n'as point demandé une longue vie, ni les richesses, ni la mort de tes ennemis, mais que tu as demandé de l'intelligence pour rendre justice,
\VS{12}voici, je fais selon ta parole. Voici, je te donne un cœur sage et intelligent, de sorte qu'il n'y aura eu personne de semblable avant toi et qu’il n'y en aura jamais de semblable après toi.
\VS{13}Et même, je te donne ce que tu n'as point demandé, les richesses et la gloire, de sorte qu'il n'y aura point de roi semblable à toi entre les rois, tant que tu vivras.
\VS{14}Et si tu marches dans mes voies pour garder mes ordonnances et mes commandements, comme David, ton père, je prolongerai tes jours.
\VS{15}Salomon s’éveilla. Et voilà le songe. Puis il s'en retourna à Jérusalem et se tint devant l'Arche de l'alliance de Yahweh. Là, il offrit des holocaustes et des offrandes de paix et fit un festin à tous ses serviteurs.
\VS{16}Alors deux femmes prostituées vinrent au roi et se présentèrent devant lui.
\VS{17}Et l'une de ces femmes dit : Hélas, mon Seigneur ! Nous demeurions cette femme-ci et moi dans une même maison et j'ai accouché près d’elle dans cette maison-là.
\VS{18}Trois jours après, cette femme a aussi accouché. Et nous étions ensemble, il n'y avait aucun étranger avec nous dans cette maison, il n’y avait que nous deux.
\VS{19}Or, l'enfant de cette femme est mort la nuit, parce qu'elle s'était couchée sur lui.
\VS{20}Elle s'est levée au milieu de la nuit, et a pris mon fils à mes côtés pendant que ta servante dormait, et l'a couché dans son sein. Et son fils mort, elle l’a couché dans mon sein.
\VS{21}Le matin, je me suis levée pour allaiter mon fils. Et voici, il était mort. Je l’ai regardé attentivement ce matin-là ; et voici, ce n'était point mon fils que j'avais enfanté.
\VS{22}L’autre femme dit : Non, c’est mon fils qui est vivant, et c’est ton fils qui est mort. Mais la première répliqua : Nullement ! Celui qui est mort est ton fils, et c’est mon fils qui vit. Elles parlaient ainsi devant le roi.
\VS{23}Et le roi dit : L’une dit : C’est mon fils qui est vivant, et c’est ton fils qui est mort ; l’autre dit : Nullement ! C’est ton fils qui est mort, et c’est mon fils qui est vivant.
\VS{24}Alors le roi dit : Apportez-moi une épée ! Et on apporta une épée devant le roi.
\VS{25}Puis le roi dit : Partagez en deux l'enfant qui vit, et donnez-en la moitié à l'une et la moitié à l'autre.
\VS{26}Alors la femme dont le fils était vivant sentit ses entrailles s’émouvoir pour son fils, et elle dit au roi : Ah ! Mon seigneur, qu'on donne à celle-ci l'enfant qui vit et qu'on ne le fasse pas mourir ! Mais l'autre dit : Il ne sera ni à moi ni à toi ; qu'on le partage.
\VS{27}Alors le roi répondit et dit : Donnez à la première l'enfant qui vit, et ne le faites pas mourir. C’est elle qui est sa mère.
\VS{28}Tout Israël entendit parler du jugement que le roi avait prononcé. Et l’on craignit le roi, car l’on reconnut que la sagesse divine était en lui pour rendre justice.
\Chap{4}
\TextTitle{Salomon établit onze chefs et douze intendants}
\VerseOne{}Le roi Salomon était roi sur tout Israël.
\VS{2}Voici les chefs qu’il avait à son service. Azaria, fils du sacrificateur Tsadok,
\VS{3}Elihoreph et Achija, enfants de Schischa, secrétaires ; Josaphat, fils d'Achilud, archiviste ;
\VS{4}Benaja, fils de Jehojada, commandait l'armée ; Tsadok et Abiathar étaient sacrificateurs ;
\VS{5}Azaria, fils de Nathan était chef des intendants ; Zabud, fils de Nathan, était le ministre d’état, favori du roi ;
\VS{6}Achischar, chef de la maison du roi ; et Adoniram, fils d’Abda, préposé sur les impôts.
\VS{7}Or, Salomon avait douze intendants sur tout Israël, qui veillaient à l’entretien du roi et de sa maison ; et chacun pendant un mois de l'année.
\VS{8}Voici leurs noms : Le fils de Hur, sur la montagne d'Ephraïm.
\VS{9}Le fils de Déker, sur Makats, sur Saalbim, sur Beth-Schémesch, à Elon de Beth-Hanan.
\VS{10}Le fils de Hésed, à Arubboth ; il avait Soco et tout le pays de Hépher.
\VS{11}Le fils d'Abinadab avait toute la contrée de Dor ; il avait Thaphath, fille de Salomon, pour femme.
\VS{12}Baana, fils d'Achilud, avait Thaanac et Meguiddo, et tout le pays de Beth-Schean qui est près de Tsarthan au-dessous de Jizreel, depuis Beth-Schean jusqu'à Abel-Mehola et jusqu'au-delà de Jokmeam.
\VS{13}Le fils de Guéber, à Ramoth en Galaad ; il avait les bourgs de Jaïr, fils de Manassé, en Galaad ; il avait aussi toute la contrée d'Argob en Basan, soixante grandes villes à murailles et garnies de barres d'airain.
\VS{14}Achinadab, fils d’Iddo, à Mahanaïm.
\VS{15}Achimaats, qui avait pour femme Basmath, fille de Salomon, en Nephthali.
\VS{16}Baana, fils de Huschaï, en Aser et sur Bealoth.
\VS{17}Josaphat, fils de Paruach, à Issacar.
\VS{18}Schimeï, fils d'Ela, en Benjamin.
\VS{19}Guéber, fils d'Uri, dans le pays de Galaad, le pays de Sihon, roi des Amoréens, et d’Og, roi de Basan ; et il était seul intendant de ce pays-là.
\TextTitle{L'étendue de la domination du royaume}
\VS{20}Juda et Israël étaient en grand nombre, semblable au sable sur le bord de la mer ; ils mangeaient, buvaient et se réjouissaient.
\VS{21}Et Salomon dominait sur tous les royaumes depuis le fleuve jusqu'au pays des Philistins et jusqu'à la frontière d'Egypte ; ils apportaient des présents, et lui furent assujettis pendant toute sa vie.
\VS{22}Or, les vivres de Salomon pour chaque jour étaient de trente cors de fine farine et soixante d'autre farine,
\VS{23}dix bœufs gras, vingt bœufs de pâturages, et cent moutons, outre les cerfs, les daims et les volailles engraissées.
\VS{24}Il dominait sur toutes les contrées de l’autre côté du fleuve, depuis Thiphsach jusqu'à Gaza, sur tous les rois qui étaient de l’autre côté du fleuve. Il était en paix avec tous les pays alentour.
\VS{25}Juda et Israël habitèrent en sécurité chacun sous sa vigne et sous son figuier, depuis Dan jusqu'à Beer-Schéba, durant toute la vie de Salomon.
\VS{26}Salomon avait aussi quarante mille crèches pour les chevaux destinés à ses chars et douze mille hommes de cheval.
\VS{27}Or, les intendants pourvoyaient à l’entretien du roi Salomon et de tous ceux qui s'approchaient de sa table, chacun en son mois ; ils ne les laissaient manquer de rien.
\VS{28}Ils faisaient aussi venir de l'orge et de la paille pour les chevaux et les coursiers dans le lieu où se trouvait le roi, chacun selon les ordres qu'il avait reçus.
\TextTitle{La sagesse de Salomon connue de toute la terre}
\VS{29}Dieu donna à Salomon de la sagesse, une très grande intelligence, et des connaissances multipliées comme le sable qui est sur le bord de la mer.
\VS{30}La sagesse de Salomon surpassait la sagesse de tous les fils de l’orient et toute la sagesse des égyptiens.
\VS{31}Il était plus sage qu’aucun homme, plus qu'Ethan, l’Ezrachite, plus qu'Héman, Calcol et Darda, les fils de Machol ; et sa renommée était répandue parmi toutes les nations d'alentour.
\VS{32}Il a prononcé trois mille paraboles et composa cinq mille cantiques.
\VS{33}Il a aussi parlé des arbres, depuis le cèdre du Liban jusqu'à l'hysope qui sort de la muraille ; il a aussi parlé sur les animaux, sur les oiseaux, sur les reptiles et sur les poissons.
\VS{34}Il venait des gens d'entre tous les peuples pour entendre la sagesse de Salomon, de la part de tous les rois de la terre qui avaient entendu parler de sa sagesse.
\Chap{5}
\TextTitle{Salomon prépare la construction du temple\FTNTT{2 Ch. 2:1 ; 13:16}}
\VerseOne{}Hiram, roi de Tyr, envoya ses serviteurs vers Salomon, car il apprit qu'on l'avait oint pour roi à la place de son père, car Hiram avait toujours aimé David.
\VS{2}Et Salomon fit dire à Hiram :
\VS{3}Tu sais que David, mon père, n'a pu bâtir une maison à Yahweh, son Dieu, à cause des guerriers qui l'ont encerclé, jusqu'à ce que Yahweh les ait mis sous la plante de ses pieds.
\VS{4}Maintenant Yahweh, mon Dieu, m'a donné du repos de toutes parts, et je n'ai plus d’adversaires, plus de calamités !
\VS{5}Voici donc j’ai l’intention de bâtir une maison au nom de Yahweh, mon Dieu, comme Yahweh l’a promis à David, mon père, en disant : Ton fils que je mettrai à ta place sur ton trône sera celui qui bâtira une maison à mon nom.
\VS{6}Ordonne maintenant que l’on coupe des cèdres du Liban pour moi. Mes serviteurs seront avec les tiens, et je donnerai pour tes serviteurs le salaire que tu auras fixé ; car tu sais qu'il n'y a personne parmi nous qui sache couper le bois comme les Sidoniens.
\VS{7}Lorsque Hiram eut entendu les paroles de Salomon, il eut une grande joie et il dit : Béni soit aujourd'hui Yahweh, qui a donné à David un fils sage pour chef de ce grand peuple !
\VS{8}Hiram fit répondre à Salomon : J'ai entendu ce que tu m'as envoyé dire et je ferai tout ce qui te plaira au sujet des bois de cèdre et des bois de cyprès.
\VS{9}Mes serviteurs les descendront du Liban à la mer, puis je les expédierai sur la mer par radeaux jusqu'au lieu que tu m'auras indiqué ; là je les ferai délier, et tu les prendras. Ce que je désire en retour, c’est que tu fournisses des vivres à ma maison.
\VS{10}Hiram donna du bois de cèdre et du bois de cyprès à Salomon autant qu'il en voulait.
\VS{11}Et Salomon donna à Hiram vingt mille cors de froment pour la nourriture de sa maison et vingt cors d'huile d’olives concassées ; Salomon en donna autant à Hiram chaque année.
\VS{12}Et Yahweh donna de la sagesse à Salomon, comme il le lui avait promis ; et il y eut paix entre Hiram et Salomon, et ils firent alliance ensemble.
\TextTitle{Les hommes de corvée\FTNTT{2 Ch. 2:2 ; 17:18}}
\VS{13}Le roi Salomon leva sur tout Israël des hommes de corvée ; ils étaient au nombre de trente mille hommes.
\VS{14}Il en envoya dix mille au Liban chaque mois, tour à tour, ils étaient un mois au Liban, et deux mois chez eux. Adoniram était préposé sur les hommes de corvée.
\VS{15}Salomon avait aussi soixante-dix mille hommes qui portaient les fardeaux et quatre-vingt mille qui taillaient les pierres dans la montagne,
\VS{16}sans compter les chefs au nombre de trois mille trois cents, préposés par Salomon sur le suivi des travaux, et chargés de surveiller les ouvriers.
\VS{17}Le roi ordonna d’extraire de grandes et précieuses pierres, pour faire le fondement de la maison, qui soient toutes taillées,
\VS{18}de sorte que les maçons de Salomon et ceux d'Hiram, taillèrent les pierres et préparèrent le bois et les pierres pour bâtir la maison.
\Chap{6}
\TextTitle{Construction du temple de Yahweh\FTNTT{2 Ch. 3:1-14}}
\VerseOne{}Ce fut la quatre cent quatre-vingtième année après la sortie des enfants d'Israël du pays d'Egypte que Salomon bâtit la maison de Yahweh\FTNTT{Voir les annexes «~Le temple de Salomon~»}, la quatrième année du règne de Salomon sur Israël, au mois de Ziv, qui est le second mois.
\VS{2}La maison que le roi Salomon bâtit à Yahweh avait soixante coudées de long, vingt de large, et trente de haut.
\VS{3}Le portique devant le temple de la maison avait vingt coudées de longueur, répondant à la largeur de la maison, et il avait dix coudées de profondeur sur le devant de la maison.
\VS{4}Il fit placer des fenêtres à la maison, fenêtres solidement grillées.
\VS{5}Il bâtit contre la muraille de la maison, à l’entour, des étages qui entouraient les murs de la maison, le temple et le sanctuaire ainsi il fit des chambres latérales tout autour.
\VS{6}L’étage inférieur était large de cinq coudées, celui du milieu de six coudées et le troisième de sept coudées ; car il avait aménagé des retraites à la maison tout autour en dehors, afin que la charpente n'entrât pas dans les murailles de la maison.
\VS{7}Pour bâtir la maison, on se servit de pierres déjà taillées, de sorte qu'en bâtissant la maison on n'entendit ni marteau, ni hache, ni aucun outil de fer.
\VS{8}L'entrée des chambres de l’étage inférieur était au côté droit de la maison, et on montait à l’étage du milieu par un escalier tournant, et de l’étage du milieu au troisième.
\VS{9}Après avoir achevé de bâtir la maison, Salomon couvrit la maison de planches et de poutres de cèdre.
\VS{10}Et il bâtit les étages joignant toute la maison, avec chacun cinq coudées de haut, et il les lia à la maison par des bois de cèdre.
\VS{11}Alors la parole de Yahweh fut adressée à Salomon, en ces termes :
\VS{12}Quant à cette maison que tu bâtis, si tu marches dans mes statuts, si tu pratiques mes ordonnances et que tu gardes tous mes commandements pour y marcher, j’accomplirai en ta faveur la parole que j'ai dite à David, ton père.
\VS{13}Et j'habiterai au milieu des enfants d'Israël, et je n'abandonnerai point mon peuple d'Israël.
\VS{14}Ainsi Salomon bâtit la maison et l'acheva.
\VS{15}Il revêtit de cèdre les murs de la maison, depuis le sol jusqu'au plafond ; il revêtit ainsi de bois l’intérieur, et il couvrit le sol de la maison de planches de cyprès.
\VS{16}Il revêtit aussi l'espace de vingt coudées de planches de cèdre à partir du fond de la maison, depuis le sol jusqu'au haut des murailles, et il bâtit cet espace au dedans pour en faire le sanctuaire, le saint des saints.
\VS{17}Les quarante coudées sur le devant formaient la maison, c’est-à-dire le temple.
\VS{18}Le bois de cèdre à l’intérieur de la maison était sculpté en coloquintes et en fleurs épanouies ; tout l’intérieur était de cèdre, on ne voyait aucune pierre.
\VS{19}Salomon disposa aussi le sanctuaire, au dedans de la maison vers le fond, pour y mettre l'arche de l'alliance de Yahweh.
\VS{20}Le sanctuaire avait par devant vingt coudées de long, vingt coudées de large, et vingt coudées de haut, et on le couvrit d’or pur ; on en couvrit aussi l'autel, fait de planches de bois de cèdre.
\VS{21}Salomon couvrit d’or pur l’intérieur de la maison, et fit passer un voile avec des chaînes d'or au-devant du sanctuaire, qu’il couvrit également d'or.
\VS{22}Ainsi il couvrit d'or la maison tout entière. Il couvrit aussi d'or tout l'autel qui était devant le sanctuaire.
\VS{23}Et il fit dans le sanctuaire deux chérubins de bois d'olivier sauvage, qui avaient chacun dix coudées de haut.
\VS{24}Chacune des ailes de l'un des chérubins avait cinq coudées et les ailes de l’autre chérubin avaient aussi cinq coudées ; depuis le bout d'une aile jusqu'au bout de l'autre aile il y avait donc dix coudées.
\VS{25}Le second chérubin était aussi de dix coudées. Les deux chérubins étaient d'une même mesure et taillés l'un comme l'autre.
\VS{26}La hauteur de chacun des deux chérubins était de dix coudées.
\VS{27}Salomon plaça les chérubins à l’intérieur, au milieu de la maison. Les ailes des chérubins étaient déployées : l'aile de l'un touchait à l’un des murs, l'aile de l'autre chérubin touchait à l'autre mur ; et leurs autres ailes se rencontraient par l’extrémité au milieu de la maison.
\VS{28}Salomon couvrit d'or les chérubins.
\VS{29}Il fit sculpter sur tout le pourtour des murs de la maison, à l’intérieur et à l’extérieur, des sculptures en relief de chérubins, des palmes et des fleurs épanouies.
\VS{30}Il couvrit aussi d'or le sol de la maison, tant à l’intérieur qu’au-dehors.
\VS{31}A l'entrée du sanctuaire, il fit une porte à deux battants de bois d'olivier sauvage, dont les linteaux avec les poteaux équivalaient à un cinquième du mur.
\VS{32}Les deux battants étaient de bois d'olivier sauvage. Il y fit sculpter des chérubins, des palmes et des fleurs épanouies qu’il couvrit d'or, étendant également l'or sur les chérubins et sur les palmes.
\VS{33}Il fit aussi, à l'entrée du temple, des poteaux de bois d'olivier sauvage, du quart de la dimension du mur.
\VS{34}Les deux battants étaient de bois de sapin ; chacun des battants était formé de deux planches brisées.
\VS{35}Il y fit sculpter des chérubins, des palmes et des fleurs épanouies, et les couvrit d'or, proprement posé sur la sculpture.
\VS{36}Il bâtit aussi le parvis de l’intérieur de trois rangées de pierres de taille et d'une rangée de poutres de cèdre.
\VS{37}La quatrième année, au mois de Ziv, les fondements de la maison de Yahweh furent posés.
\VS{38}Et la onzième année, au mois de Bul, qui est le huitième mois, la maison fut achevée dans toutes ses parties et telle qu’elle devait être. Salomon la construisit en l’espace de sept années.
\Chap{7}
\TextTitle{Construction du palais royal}
\VerseOne{}Salomon bâtit aussi sa maison, et l'acheva complètement en treize ans.
\VS{2}Il bâtit d’abord la maison de la forêt du Liban, de cent coudées de long, de cinquante coudées de large, et de trente coudées de haut, sur quatre rangées de colonnes de cèdre ; et sur les colonnes il y avait des poutres de cèdre.
\VS{3}On couvrit de bois de cèdre les chambres qui portaient sur les colonnes qui étaient au nombre de quarante-cinq, quinze par étages.
\VS{4}Et il y avait trois rangées de fenêtrages ; et une fenêtre répondait à l'autre en trois endroits.
\VS{5}Toutes les portes et tous les poteaux étaient formés de poutres carrées, avec les fenêtres ; et à chacun des trois étages, les ouvertures étaient en vis-à-vis les unes des autres.
\VS{6}Il fit aussi le portique de colonnes, long de cinquante coudées, et large de trente coudées ; et un autre portique en avant avec des colonnes et des degrés sur leur front.
\VS{7}Il fit aussi le portique du trône sur lequel il rendait ses jugements, appelé le portique du jugement ; on le couvrit de cèdre depuis un bout du sol jusqu'à l'autre.
\VS{8}La maison où il demeurait fut construite de la même manière, dans une autre cour, derrière le portique. Salomon fit une maison bâtie comme ce portique à la fille de Pharaon, qu'il avait prise pour femme.
\VS{9}Toutes ces constructions étaient de pierres de prix, taillées d’après des mesures, sciées à la scie, en dedans et en dehors, depuis les fondements jusqu'aux corniches, et par dehors jusqu'au grand parvis.
\VS{10}Le fondement était en pierres magnifiques et de grand prix, de grandes pierres, des pierres de dix coudées et des pierres de huit coudées.
\VS{11}Et par-dessus il y avait des pierres de prix, taillées d’après des mesures, et du bois de cèdre.
\VS{12}Et le grand parvis avait aussi tout alentour trois rangées de pierres de taille et une rangée de poutres de cèdre, comme le parvis intérieur de la maison de Yahweh, et le portique de la maison.
\TextTitle{Hiram, artisan spécialiste en airain\FTNTT{2 Ch. 2:12-13}}
\VS{13}Or, le roi Salomon fit venir de Tyr Hiram ;
\VS{14}fils d'une femme veuve de la tribu de Nephthali, et d’un père tyrien, Hiram travaillait le cuivre ; fort expert, intelligent et savant pour faire toutes sortes d'ouvrages d'airain ; il arriva auprès du roi Salomon, et il fit tout son ouvrage.
\TextTitle{Les colonnes du temple\FTNTT{2 Ch. 3:15-17}}
\VS{15}Il fit les deux colonnes d'airain, la première avait dix-huit coudées de hauteur ; et un cordon de douze coudées mesurait le tour de la seconde.
\VS{16}Il fit aussi deux chapiteaux d'airain fondu pour mettre sur les sommets des colonnes ; le premier chapiteau était de cinq coudées de hauteur, le second était aussi de cinq coudées.
\VS{17}Il fit des treillis en forme de maillages, des festons façonnés en forme de chaînes, pour les chapiteaux qui étaient sur le sommet des colonnes, sept pour le premier des chapiteaux, et sept pour le second.
\VS{18}Il fit deux rangs de grenades autour de l’un des treillis, pour couvrir le chapiteau qui était sur le sommet d'une des colonnes ; et il fit de même pour l'autre chapiteau.
\VS{19}Dans le portique, les chapiteaux qui étaient sur le sommet des colonnes figuraient des fleurs de lis hautes de quatre coudées au porche.
\VS{20}Ces chapiteaux placés sur les deux colonnes étaient entourés de deux cents grenades, en haut, depuis le renflement qui était au-delà du treillis ; il y avait aussi deux cents grenades, disposées par rangs, autour du second chapiteau.
\VS{21}Il dressa donc les colonnes au portique du temple. Il dressa la colonne de droite qu’il nomma Jakin ; puis il dressa la colonne de gauche qu’il nomma Boaz.
\VS{22}Et l’on mit sur le chapiteau des colonnes l'ouvrage figurant des fleurs de lis ; ainsi l'ouvrage des colonnes fut achevé.
\TextTitle{La mer de fonte\FTNTT{2 Ch. 4:2-5}}
\VS{23}Il fit aussi la mer de fonte. Elle avait dix coudées d'un bord à l'autre, ronde tout autour, avec cinq coudées de haut ; et un cordon de trente coudées en mesurait le tour.
\VS{24}Au-dessous de son bord, des coloquintes l'environnaient, dix à chaque coudée, lesquelles faisaient tout le tour de la mer. Il y avait deux rangées de coloquintes, jetées en fonte.
\VS{25}Et elle était posée sur douze bœufs, dont trois regardaient le nord et trois regardaient l'occident, trois regardaient le sud et trois regardaient l'orient. La mer était sur eux et toute la partie postérieure de leur corps était tournée en dedans.
\VS{26}Son épaisseur était d'une paume, et son bord était comme le bord d'une coupe en fleur de lis ; elle contenait deux mille baths.
\TextTitle{Les dix socles d'airain}
\VS{27}Il fit aussi dix socles d'airain, ayant chacun quatre coudées de long, quatre coudées de large et trois coudées de haut.
\VS{28}Ces socles étaient réalisés de telle manière qu'il y avait des panneaux enchâssés entre leurs bordures.
\VS{29}Sur les panneaux qui étaient entre les bordures, il y avait des lions, des bœufs et des chérubins. Et sur les bordures, au-dessus et en dessous des lions et des bœufs, il y avait des ornements qui pendaient en festons.
\VS{30}Chaque socle avait quatre roues d'airain avec des essieux d'airain. Ses quatre pieds leur servaient d’appuis. Ces appuis étaient fondus au-dessous de la cuve, et au-dessus étaient les festons.
\VS{31}Le couronnement offrait à son intérieur une ouverture avec un prolongement d'une coudée vers le haut ; cette ouverture était arrondie comme pour les ouvrages de ce genre et elle avait une coudée et demie de largeur. Il s’y trouvait aussi des sculptures ; les panneaux étaient carrés, et non arrondis.
\VS{32}Les quatre roues étaient sous les panneaux, et les essieux des roues fixés à la base ; chaque roue était haute d'une coudée et demie.
\VS{33}Les roues étaient faites comme les roues de chars ; leurs essieux, leurs jantes, leurs rais et leurs moyeux étaient tous de fonte.
\VS{34}Il y avait aux quatre angles de chaque socle quatre consoles d’une même pièce que la base.
\VS{35}La partie supérieure de la base se terminait par un cercle d’une demi-coudée de hauteur, et elle avait ses appuis et ses panneaux de la même pièce.
\VS{36}Puis, on sculpta sur la surface de ses appuis et sur ses panneaux, des chérubins, des lions et des palmes, selon les espaces libres, et des ornements tout autour.
\VS{37}Ainsi les dix socles étaient tous d’une même fonte, d’une même mesure et d’une même forme.
\TextTitle{Les dix cuves d'airain\FTNTT{2 Ch. 4:6}}
\VS{38}Il fit aussi dix cuves d'airain, dont chacune contenait quarante baths, et chaque cuve était de quatre coudées, chaque cuve était sur l’un des dix socles.
\VS{39}Il mit cinq socles au côté droit de la maison, et cinq au côté gauche de la maison ; quant à la mer, il l’a mis au côté droit de la maison, vers l'orient du côté sud.
\TextTitle{Totalité de l’œuvre d’Hiram}
\VS{40}Ainsi Hiram fit les cuves, les pelles et les bassins, et il acheva tout l'ouvrage qu'il faisait au roi Salomon pour la maison de Yahweh.
\VS{41}Savoir, deux colonnes avec les deux chapiteaux qui étaient sur le sommet des colonnes ; et deux maillages pour couvrir les deux bourrelets des chapiteaux qui étaient sur le sommet des colonnes ;
\VS{42}les quatre cents grenades pour les deux maillages, deux rangs de grenades pour chaque réseau, pour couvrir les deux renflements des chapiteaux, qui étaient sur les colonnes ;
\VS{43}les dix socles ; et les dix cuves pour mettre sur les socles ;
\VS{44}la mer avec les douze bœufs sous la mer ;
\VS{45}les pots, les pelles et les bassins. Tous ces ustensiles que Hiram fit au roi Salomon pour la maison de Yahweh étaient d'airain poli.
\VS{46}Le roi les fit fondre dans la plaine du Jourdain, dans un sol argileux, entre Succoth et Tsarthan.
\VS{47}Et Salomon ne pesa aucun de ces ustensiles, parce qu'ils étaient en trop grand nombre, de sorte qu'on ne rechercha point le poids de l’airain.
\TextTitle{Divers ustensiles d’or pour la maison de Yahweh}
\VS{48}Salomon fit aussi tous les ustensiles pour la maison de Yahweh, savoir l'autel d'or, et les tables d'or, sur lesquelles étaient les pains de proposition ;
\VS{49}les chandeliers d’or pur, cinq à droite et cinq à gauche devant le sanctuaire, avec les fleurs, les lampes et les mouchettes d'or ;
\VS{50}les coupes, les couteaux, les bassins, les tasses et les brasiers d’or pur. Les gonds, même des portes de la maison, à l’entrée du saint des saints, à la porte de la maison et à l’entrée du temple, étaient d'or.
\VS{51}Ainsi fut achevé tout l'ouvrage que le roi Salomon fit pour la maison de Yahweh ; puis il y fit apporter l'or, l’argent et les ustensiles que David, son père, avait consacrés ; il les mit dans les trésors de la maison de Yahweh.
\Chap{8}
\TextTitle{L’arche de l’alliance placée dans le saint des saints ; la gloire de Yahweh remplit le temple \FTNTT{2 Ch. 5:2-14}}
\VerseOne{}Alors le roi Salomon convoqua près de lui à Jérusalem les anciens d'Israël, tous les chefs des tribus et les chefs de famille des fils d'Israël, pour transporter l'arche de l'alliance de Yahweh de la cité de David, qui est Sion.
\VS{2}Tous les hommes d'Israël s’assemblèrent auprès du roi Salomon, au mois d'Ethanim, qui est le septième mois, pendant la fête.
\VS{3}Une fois tous les anciens d'Israël arrivés, les sacrificateurs portèrent l'arche.
\VS{4}Ils transportèrent l'arche de Yahweh, la tente d'assignation, et tous les ustensiles qui étaient dans le tabernacle ; les sacrificateurs et les Lévites les emportèrent.
\VS{5}Le roi Salomon et toute l'assemblée d'Israël convoquée auprès de lui se tinrent devant l'arche. Ils sacrifièrent du gros et du menu bétail en si grand nombre, qu'on ne pouvait ni nombrer ni compter.
\VS{6}Et les sacrificateurs portèrent l'arche de l'alliance de Yahweh à sa place, dans le sanctuaire de la maison, dans le saint des saints, sous les ailes des chérubins.
\VS{7}Car les chérubins avaient les ailes étendues sur l’emplacement de l'arche, et ils couvraient l'arche et ses barres par-dessus.
\VS{8}On avait donné aux barres une longueur telle que leurs extrémités se voyaient du lieu saint devant le sanctuaire, mais elles ne se voyaient point du dehors. Elles sont demeurées là jusqu'à ce jour.
\VS{9}Il n'y avait rien dans l'arche que les deux tables de pierre que Moïse y déposa en Horeb, lorsque Yahweh fit alliance avec les enfants d'Israël à leur sortie du pays d'Egypte.
\VS{10}Au moment où les sacrificateurs sortirent du lieu saint, la nuée remplit la maison de Yahweh.
\VS{11}Les sacrificateurs ne purent pas y rester pour faire le service, à cause de la nuée ; car la gloire de Yahweh remplissait la maison de Yahweh.
\TextTitle{Discours de Salomon\FTNTT{2 Ch. 6:1-11}}
\VS{12}Alors Salomon dit : Yahweh veut habiter dans l'obscurité !
\VS{13}J'ai achevé de bâtir une maison pour ta demeure ô Yahweh ! Ce sera une demeure, un lieu où tu résideras éternellement.
\VS{14}Le roi tourna son visage, et bénit toute l'assemblée d'Israël ; car toute l'assemblée d'Israël se tenait là debout.
\VS{15}Et il dit : Béni soit Yahweh, le Dieu d'Israël, qui a parlé de sa propre bouche à David, mon père, et qui a accompli par sa puissance ce qu’il avait déclaré en disant :
\VS{16}Depuis le jour où je fis sortir mon peuple d'Israël hors d'Egypte, je n'ai choisi aucune ville d'entre toutes les tribus d'Israël pour y bâtir une maison afin que mon nom y fût, mais j'ai choisi David pour qu’il règne sur mon peuple d'Israël.
\VS{17}David, mon père, avait à cœur de bâtir une maison au nom de Yahweh, le Dieu d'Israël.
\VS{18}Et Yahweh dit à David, mon père : Puisque tu as eu à cœur de bâtir une maison à mon nom, tu as bien fait d’avoir eu cette intention.
\VS{19}Néanmoins, tu ne bâtiras point cette maison, mais ton fils qui sortira de tes entrailles sera celui qui bâtira cette maison à mon Nom.
\VS{20}Yahweh a donc accompli la parole qu'il avait prononcée. Je me suis élevé à la place de David, mon père, et me suis assis sur le trône d'Israël, comme Yahweh l’avait annoncé, et j'ai bâti cette maison au Nom de Yahweh, le Dieu d'Israël.
\VS{21}J'y ai établi ici un lieu pour l'arche, dans lequel est l'alliance de Yahweh, qu'il traita avec nos pères quand il les fit sortir hors du pays d'Egypte.
\TextTitle{Prière de Salomon\FTNTT{2 Ch. 6:12-42}}
\VS{22}Ensuite Salomon se tint devant l'autel de Yahweh en la présence de toute l'assemblée d'Israël, et étendant ses mains vers les cieux,
\VS{23}il dit : Ô Yahweh, Dieu d'Israël ! Il n'y a point de Dieu semblable à toi en haut dans les cieux, ni en bas sur la terre ; tu gardes l'alliance et la miséricorde envers tes serviteurs qui marchent devant ta face de tout cœur !
\VS{24}Ainsi tu as tenu parole à ton serviteur David, mon père, car ce que tu as déclaré de ta bouche, tu l'as accompli en ce jour par ta main puissante.
\VS{25}Maintenant donc, ô Yahweh, Dieu d'Israël, prête attention à la promesse faite à ton serviteur David, mon père, en lui disant : Tu ne manqueras jamais devant moi d’un successeur assis sur le trône d'Israël, pourvu seulement que tes fils prennent garde à leur voie et qu’ils marchent devant ma face, comme tu y as marché.
\VS{26}Et maintenant, ô Dieu d'Israël ! Je te prie, que s’accomplisse la promesse que tu as faite à ton serviteur David, mon père.
\VS{27}Mais Dieu habiterait-il véritablement sur la terre ? Voilà, les cieux, même les cieux des cieux ne peuvent te contenir ; combien moins cette maison que j'ai bâtie !
\VS{28}Toutefois, ô Yahweh, mon Dieu, sois attentif à la prière que t’adresse ton serviteur et à sa supplication, pour entendre le cri et la prière que ton serviteur t’adresse aujourd'hui.
\VS{29}Que tes yeux soient ouverts jour et nuit sur cette maison, sur le lieu dont tu as dit : Là sera mon Nom ! Ecoute la prière que ton serviteur fait en ce lieu.
\VS{30}Daigne exaucer la supplication de ton serviteur et de ton peuple d'Israël lorsqu’ils te prieront en ce lieu ; exauce du lieu de ta demeure. Des cieux, exauce, et pardonne !
\VS{31}Si quelqu'un pèche contre son prochain et qu’on lui impose un serment pour le faire jurer, et que le serment aura été fait devant ton autel dans cette maison ;
\VS{32}écoute-le des cieux, et agis. Juge tes serviteurs, condamne le coupable en lui rendant selon sa conduite ; rends justice à l’innocent, et traite-le selon son innocence !
\VS{33}Quand ton peuple d'Israël sera battu par l'ennemi, pour avoir péché contre toi, s’il revient à toi et rend gloire à ton Nom, en t’adressant des prières et des supplications dans cette maison,
\VS{34}exauce-le des cieux, et pardonne le péché de ton peuple d'Israël, et ramène-le dans la terre que tu as donnée à leurs pères.
\VS{35}Quand les cieux seront fermés et qu'il n'y aura point de pluie, à cause de ses péchés contre toi, s'il te fait une prière en ce lieu-ci, qu’il loue ton Nom, et s’il se détourne de ses péchés, parce que tu les auras affligés,
\VS{36}exauce-le des cieux, pardonne le péché de tes serviteurs et de ton peuple d'Israël, à qui tu enseigneras quel est le chemin par lequel ils doivent marcher et envoie-leur la pluie sur la terre que tu as donnée à ton peuple pour héritage !
\VS{37}Quand il y aura dans le pays, famine, peste, jaunisse, nielle, sauterelles d’une espèce ou d’une autre, même quand les ennemis assiégeront ton peuple dans son propre pays, quand il y aura un fléau ou une maladie quelconque ;
\VS{38}si un homme, si tout ton peuple d'Israël fait entendre des prières et des supplications, que chacun reconnaisse la plaie de son cœur et étende les mains vers cette maison,
\VS{39}exauce-le des cieux, du lieu de ta demeure, pardonne, et agis. Rends à chacun selon toutes ses voies, parce que tu auras connu leurs cœurs ; car toi seul connais le cœur de tous les fils des hommes ;
\VS{40}et ils te craindront toute leur vie dans le pays que tu as donné à nos pères !
\VS{41}Et même lorsque l'étranger, qui n’est pas de ton peuple d'Israël, viendra d'un pays éloigné à cause de ton Nom,
\VS{42}car on saura que ton Nom est grand, ta main puissante et ton bras étendu, quand il viendra prier dans cette maison,
\VS{43}exauce-le des cieux, du lieu de ta demeure, et fais à cet étranger selon ce qu’il t’aura demandé, afin que tous les peuples de la terre connaissent ton Nom pour te craindre, comme ton peuple d'Israël ; et pour connaître que ton Nom est invoqué sur cette maison que j'ai bâtie !
\VS{44}Quand ton peuple sortira pour combattre son ennemi, par la voie par laquelle tu l’auras envoyé, s'ils prient Yahweh en regardant vers cette ville que tu as choisie et vers cette maison que j'ai bâtie à ton Nom !
\VS{45}Exauce des cieux leurs prières et leurs supplications, et fais leur justice !
\VS{46}Quand ils pécheront contre toi, car il n'y a point d'homme qui ne pèche, et que tu seras irrité contre eux et que tu les auras livrés à leurs ennemis, qui les emmènera captifs dans un pays ennemi, lointain ou proche ;
\VS{47}si dans le pays où ils auront été menés captifs, ils reviennent à toi et t’adressent des supplications, se repentent et te prient au pays de ceux qui les auront emmenés captifs, en disant : Nous avons péché, nous avons commis l’iniquité, nous avons fait le mal !
\VS{48}S'ils reviennent à toi de tout leur cœur et de toute leur âme, dans le pays de leurs ennemis, qui les auront emmenés captifs, et s'ils t'adressent leurs prières, les regards tournés vers le pays que tu as donné à leurs pères, vers la ville que tu as choisie, vers la maison que j'ai bâtie à ton Nom,
\VS{49}exauce des cieux, du lieu de ta demeure, leurs prières et leurs supplications, et fais-leur justice.
\VS{50}Pardonne à ton peuple ses offenses et ses péchés envers toi, et fais que ceux qui les auront emmenés captifs aient pitié d'eux et leur fassent grâce,
\VS{51}car ils sont ton peuple et ton héritage, et tu les as fait sortir hors d'Egypte, du milieu d'une fournaise de fer !
\VS{52}Que tes yeux donc soient ouverts sur la supplication de ton serviteur et celle de ton peuple d'Israël, pour les exaucer dans tout ce pourquoi ils crieront à toi !
\VS{53}Car tu les as séparés de tous les autres peuples de la terre pour être ton héritage, comme tu l’as déclaré par Moïse, ton serviteur, quand tu fis sortir nos pères hors d'Egypte, ô Seigneur Yahweh !
\TextTitle{Bénédictions et réjouissances\FTNTT{2 Ch. 7:4-10}}
\VS{54}Lorsque Salomon eut achevé de faire cette prière et cette supplication à Yahweh, il se leva de devant l'autel de Yahweh où il était agenouillé et les mains étendues vers les cieux.
\VS{55}Il se tint debout, et bénit toute l'assemblée d'Israël à haute voix, en disant :
\VS{56}Béni soit Yahweh, qui a donné du repos à son peuple d'Israël, comme il l’avait annoncé ! De toutes les paroles qu'il avait prononcées par le moyen de Moïse, son serviteur, aucune n’est restée sans effet.
\VS{57}Que Yahweh, notre Dieu, soit avec nous, comme il a été avec nos pères ; qu'il ne nous abandonne point et qu'il ne nous délaisse point,
\VS{58}mais qu'il incline nos cœurs vers lui, afin que nous marchions dans toutes ses voies, et que nous observions ses commandements, ses statuts et ses ordonnances, qu'il a prescrits à nos pères !
\VS{59}Que ces paroles, par lesquelles j'ai fait supplication à Yahweh, soient présentes devant Yahweh, notre Dieu, jour et nuit ; afin qu'il fasse justice à son serviteur et à son peuple d’Israël en tout temps,
\VS{60}afin que tous les peuples de la terre reconnaissent que c'est Yahweh qui est Dieu et qu'il n'y en a point d'autre !
\VS{61}Que votre cœur soit intègre envers Yahweh, notre Dieu, comme aujourd'hui, pour marcher dans ses statuts et pour garder ses commandements.
\VS{62}Le roi et tout Israël avec lui offrirent des sacrifices devant Yahweh.
\VS{63}Salomon offrit un sacrifice d'offrande de paix à Yahweh, savoir vingt-deux mille bœufs et cent vingt mille brebis. Ainsi le roi et tous les enfants d'Israël firent la dédicace de la maison de Yahweh.
\VS{64}En ce jour-là, le roi consacra le milieu du parvis, qui est devant la maison de Yahweh ; car il offrit là les holocaustes, les offrandes et les graisses des sacrifices d’offrandes de paix, parce que l'autel d'airain qui est devant Yahweh, était trop petit pour contenir les holocaustes, les offrandes et les graisses des offrandes de paix.
\VS{65}Et en ce temps-là, Salomon célébra une fête solennelle ; et tout Israël avec lui, venu en grande multitude depuis les environs de Hamath jusqu'au torrent d'Egypte, devant Yahweh, notre Dieu, pendant sept jours, et sept autres jours, soit quatorze jours.
\VS{66}Le huitième jour, il renvoya le peuple. Et ils bénirent le roi, et s'en allèrent dans leurs demeures, en se réjouissant, et le cœur heureux pour tout le bien que Yahweh avait fait à David, son serviteur, et à Israël, son peuple.
\Chap{9}
\TextTitle{Yahweh apparaît à Salomon une seconde fois\FTNTT{2 Ch. 7:11-22}}
\VerseOne{}Lorsque Salomon eut achevé de bâtir la maison de Yahweh, la maison royale, et tout ce que Salomon prit plaisir à faire,
\VS{2}Yahweh apparut à Salomon une seconde fois, comme il lui était apparu à Gabaon.
\VS{3}Et Yahweh lui dit : J'exauce ta prière, et la supplication que tu as faite devant moi, j'ai sanctifié cette maison que tu as bâtie pour y mettre mon Nom à jamais, et mes yeux et mon cœur seront toujours là.
\VS{4}Quant à toi, si tu marches devant moi comme David, ton père, a marché, avec intégrité et de cœur et avec droiture, en faisant tout ce que je t'ai commandé, et si tu gardes mes statuts et mes ordonnances,
\VS{5}j’affermirai le trône de ton royaume sur Israël à jamais, comme je l’ai déclaré à David, ton père, en disant : Tu ne manqueras jamais d’un successeur sur le trône d'Israël.
\VS{6}Mais si vous et vos fils, vous vous détournez de moi et que vous ne gardiez pas mes commandements, mes lois que je vous ai prescrites, et si vous allez servir d'autres dieux et vous prosterner devant eux,
\VS{7}je retrancherai Israël de la terre que je lui ai donnée, je rejetterai loin de moi cette maison que j'ai consacrée à mon Nom et Israël sera un sujet de sarcasme et de moquerie parmi tous les peuples.
\VS{8}Et si haut placée qu’ait été cette maison, quiconque passera auprès d'elle sera étonné et sifflera. Et on dira : Pourquoi Yahweh a-t-il ainsi traité ce pays et cette maison ?
\VS{9}Et on répondra : Parce qu'ils ont abandonné Yahweh, leur Dieu, qui avait tiré leurs pères hors du pays d'Egypte, qu'ils se sont attachés à d'autres dieux, se sont prosternés devant eux et les ont servis, voilà pourquoi Yahweh a fait venir sur eux tous ces maux.
\TextTitle{Les réalisations de Salomon\FTNTT{2 Ch. 8:1-18}}
\VS{10}Au bout de vingt ans, Salomon avait bâti les deux maisons, la maison de Yahweh et la maison royale.
\VS{11}Hiram, roi de Tyr, avait fourni à Salomon du bois de cèdre, du bois de sapin et de l'or, autant qu'il en avait voulu, le roi Salomon donna à Hiram vingt villes dans le pays de Galilée.
\VS{12}Hiram sortit de Tyr, pour voir les villes que Salomon lui avait données. Mais elles ne lui plurent point,
\VS{13}et il dit : Quelles villes m'as-tu assignées, mon frère ? Et il les appela, pays de Cabul, nom qu’elles ont conservé jusqu'à ce jour.
\VS{14}Hiram avait aussi envoyé au roi cent vingt talents d'or.
\VS{15}Voici ce qui concerne les hommes de corvée que le roi Salomon leva pour bâtir la maison de Yahweh, sa maison, Millo, la muraille de Jérusalem, Hatsor, Meguiddo et Guézer.
\VS{16}Pharaon, roi d'Egypte, était venu s’emparer de Guézer et l'avait incendiée, il avait tué les Cananéens qui habitaient dans la ville. Puis il la donna pour dot à sa fille, femme de Salomon.
\VS{17}Salomon donc bâtit Guézer, et Beth-Horon la basse,
\VS{18}Baalath et Thadmor, dans le désert qui est au pays,
\VS{19}toutes les villes servant de magasins et lui appartenant, les villes pour les chars et les villes pour la cavalerie, et tout ce qu’il plut à Salomon de bâtir à Jérusalem, au Liban, et dans tout le pays dont il était le souverain.
\VS{20}Tout le peuple qui était resté des Amoréens, des Héthiens, des Phéréziens, des Héviens et des Jébusiens ne faisaient point partie des fils d'Israël,
\VS{21}leurs descendants qui étaient demeurés après eux dans le pays et que les fils d'Israël n'avaient pu dévouer par le moyen de l'interdit, Salomon les fit placer à son service comme gens de corvée à toujours.
\VS{22}Mais Salomon n’employa aucun des fils d'Israël comme esclaves ; car ils étaient ses hommes de guerre, ses serviteurs, ses chefs, ses officiers, les chefs de ses chars et ses hommes d'armes.
\VS{23}Les chefs préposés aux travaux par Salomon étaient au nombre de cinq cent cinquante, lesquels géraient l'intendance des ouvriers.
\VS{24}La fille de Pharaon monta de la cité de David dans la maison que Salomon lui avait bâtie. Ce fut alors qu’il bâtit Millo.
\VS{25}Trois fois par an, Salomon offrait des holocaustes et des offrandes de paix sur l'autel qu'il avait bâti à Yahweh, et il brûlait des parfums sur celui qui était devant Yahweh. Et il acheva la maison.
\VS{26}Le roi Salomon construisit des navires à Etsjon-Guéber, près d'Eloth, sur le rivage de la Mer Rouge, au pays d'Edom.
\VS{27}Et Hiram envoya sur ces navires, auprès des serviteurs de Salomon, ses propres serviteurs, des hommes connaissant la mer.
\VS{28}Ils allèrent en Ophir, et ils prirent de là quatre cent vingt talents d'or qu’ils apportèrent au roi Salomon.
\Chap{10}
\TextTitle{La reine de Séba chez Salomon\FTNTT{2 Ch. 7:1-12}}
\VerseOne{}Or, la reine de Séba ayant appris la renommée de Salomon, à cause du Nom de Yahweh, vint l’éprouver par des énigmes.
\VS{2}Elle entra dans Jérusalem avec une suite fort nombreuse, et avec des chameaux qui portaient des aromates, une grande quantité d'or, et des pierres précieuses. Elle se rendit auprès de Salomon, et lui parla de tout ce qu'elle avait dans le cœur.
\VS{3}Salomon répondit à toutes ses questions, et il n’y eut aucune parole à laquelle le roi ne put fournir une explication.
\VS{4}La reine de Séba vit toute la sagesse de Salomon et la maison qu'il avait bâtie,
\VS{5}les mets de sa table, la demeure de ses serviteurs, l’ordre de service, leurs vêtements, ses échansons, et les holocaustes qu'il offrait dans la maison de Yahweh.
\VS{6}Elle fut toute ravie en elle-même, elle parla ainsi au roi : Ce que j'ai entendu dire dans mon pays au sujet de ta sagesse était donc vrai !
\VS{7}Je ne croyais pas ce qu’on en disait avant d’être venue et que mes yeux ne l'aient vu. Et voici, on ne m'en avait point rapporté la moitié. Ta sagesse et ta prospérité surpassent tout ce que j'en avais entendu.
\VS{8}Heureux sont tes gens ! Heureux tes serviteurs qui se tiennent continuellement devant toi, et qui entendent ta sagesse !
\VS{9}Béni soit Yahweh, ton Dieu, qui t’a accordé la faveur de t’établir sur le trône d'Israël ! Car Yahweh a aimé Israël à toujours ; et t'a établi roi pour faire droit et justice.
\VS{10}Puis elle donna au roi cent vingt talents d'or, une très grande quantité d’aromates et des pierres précieuses. Il ne vint jamais depuis une aussi grande abondance d’aromates que la reine de Séba en donna au roi Salomon.
\VS{11}Et les navires de Hiram, qui amenèrent de l'or d'Ophir, amenèrent aussi d’Ophir une grande quantité de bois de santal et de pierres précieuses.
\VS{12}Le roi fit des supports de ce bois de santal pour la maison de Yahweh et pour la maison royale ; il en fit aussi des harpes et des luths pour les chantres ; il ne vint plus de ce bois de santal et on n’en a plus vu jusqu'à ce jour-là.
\VS{13}Le roi Salomon donna à la reine de Séba tout ce qu'elle désira et répondit à tout et ce qu'elle lui demanda. Il lui fit en outre des présents dignes d'un roi tel que Salomon. Puis elle s'en retourna et alla dans son pays, elle et ses serviteurs.
\TextTitle{Les richesses de Salomon\FTNTT{2 Ch. 9:13-28}}
\VS{14}Le poids de l'or qui revenait à Salomon chaque année, était de six cent soixante-six talents d'or,
\VS{15}outre ce qui lui revenait des négociants, du trafic des marchands, de tous les rois d'Arabie, et des gouverneurs de ce pays-là.
\VS{16}Le roi Salomon fit aussi deux cents grands boucliers d'or battu au marteau, employant six cents sicles d'or pour chaque bouclier,
\VS{17}et trois cents autres boucliers d'or battu au marteau, pour chacun desquels il employa trois mines d'or ; et le roi les mit dans la maison de la forêt du Liban.
\VS{18}Le roi fit aussi un grand trône d'ivoire, qu'il couvrit d’or pur.
\VS{19}Ce trône avait six degrés, et la partie supérieure, le haut du trône était arrondi par derrière. Il y avait des accoudoirs de chaque côté du siège et deux lions se tenaient auprès des accoudoirs.
\VS{20}Il y avait aussi douze lions sur les six degrés du trône, de part et d'autre. Il ne s'est rien fait de tel dans aucun royaume.
\VS{21}Toute la vaisselle du buffet du roi Salomon était d'or, et toutes les coupes de la maison de la forêt du Liban étaient d’or pur. Il n'y en avait point en argent ; on n’en faisait aucun cas du temps de Salomon.
\VS{22}Car le roi avait en mer des navires de Tarsis avec la flotte d'Hiram ; et tous les trois ans la flotte de Tarsis revenait, apportant de l'or, de l'argent, de l'ivoire, des singes et des paons.
\VS{23}Le roi Salomon fut plus grand que tous les rois de la terre, tant en richesses qu'en sagesse.
\VS{24}Tous les habitants de la terre cherchaient à voir la face de Salomon, pour écouter la sagesse que Dieu avait mise en son cœur.
\VS{25}Et chacun d'eux lui apportait son présent, des vases d’or et d'argent, des vêtements, des armes, des aromates, des chevaux et des mulets, tous les ans.
\VS{26}Salomon rassembla ses chars et sa cavalerie ; il y avait mille quatre cents chars et douze mille chevaliers, qu'il plaça dans les villes où il tenait ses chars et à Jérusalem près du roi.
\VS{27}Le roi rendit l'argent aussi commun à Jérusalem que les pierres ; et les cèdres que les sycomores qui croissent dans les plaines, tant il y en avait.
\VS{28}C’est d’Egypte que provenaient les chevaux de Salomon ; une caravane de marchands du roi allait les chercher par troupes, à un prix fixe :
\VS{29}Un char montait et sortait d'Egypte pour six cents sicles d'argent et chaque cheval pour cent cinquante sicles ; ils en amenaient de même avec eux pour tous les rois des Héthiens et pour les rois de Syrie.
\Chap{11}
\TextTitle{Salomon détourne son cœur de Yahweh}
\VerseOne{}Le roi Salomon aima plusieurs femmes étrangères, outre la fille de Pharaon ; savoir des Moabites, des Ammonites, des Edomites, des Sidoniennes et des Héthiennes.
\VS{2}Elles étaient d'entre les nations dont Yahweh avait dit aux enfants d'Israël : Vous n'irez point vers elles, et elles ne viendront point vers vous ; car certainement elles feraient détourner vos cœurs pour suivre leurs dieux. Salomon s'attacha à elles et les aima.
\VS{3}Il eut donc pour femmes sept cents princesses et trois cents concubines ; et ses femmes détournèrent son cœur.
\VS{4}Au temps de la vieillesse de Salomon, ses femmes firent détourner son cœur vers d'autres dieux ; et son cœur ne fut point intègre devant Yahweh, son Dieu, comme David, son père.
\VS{5}Salomon alla après Astarté, la divinité des Sidoniens, et après Milcom, l'abomination des Ammonites.
\VS{6}Ainsi Salomon fit ce qui est mal aux yeux de Yahweh, et il ne persévéra point à suivre Yahweh, comme David, son père.
\VS{7}Et Salomon bâtit un haut lieu à Kemosch, l'abomination des Moabites, sur la montagne qui est vis-à-vis de Jérusalem ; et à Moloc, l'abomination des fils d’Ammon.
\VS{8}Il en fit de même pour toutes ses femmes étrangères, qui offraient des parfums et des sacrifices à leurs dieux.
\VS{9}C'est pourquoi Yahweh fut irrité contre Salomon, parce qu'il avait détourné son cœur de Yahweh, le Dieu d'Israël, qui lui était apparu deux fois.
\VS{10}Il lui avait donné cet ordre de ne point aller après d'autres dieux ; mais il ne garda point ce que Yahweh lui avait ordonné.
\VS{11}Et Yahweh dit à Salomon : Puisque tu as agi de la sorte, et que tu n'as pas observé l’alliance et les ordonnances que je t'avais prescrites, je déchirerai le royaume afin qu'il ne soit plus à toi et je le donnerai à ton serviteur.
\VS{12}Toutefois je ne le ferai point en ton temps, pour l’amour de David, ton père. Ce sera d'entre les mains de ton fils que je déchirerai le royaume.
\VS{13}Néanmoins je ne déchirerai pas tout le royaume, j'en donnerai une tribu à ton fils, pour l'amour de David, mon serviteur, et pour l'amour de Jérusalem, que j'ai choisie.
\TextTitle{Dieu suscite des ennemis à Salomon}
\VS{14}Yahweh donc suscita un ennemi à Salomon, savoir Hadad, l’Edomite, qui était de la race royale d'Edom.
\VS{15}Car il était arrivé qu'au temps que David était en Edom, Joab, chef de l'armée, étant monté pour ensevelir les morts, tua tous les mâles qui étaient en Edom ;
\VS{16}Joab demeura là six mois avec tout Israël, jusqu'à ce qu'il eût exterminé tous les mâles d'Edom.
\VS{17}Ce fut alors qu’Hadad prit la fuite avec des Edomites d'entre les serviteurs de son père, pour se retirer en Egypte. Hadad était alors un jeune garçon.
\VS{18}Une fois partis de Madian, ils allèrent à Paran, prirent avec eux des hommes de Paran, et arrivèrent en Egypte auprès de Pharaon, roi d'Egypte, qui lui donna une maison, pourvut à sa subsistance et lui donna aussi une terre.
\VS{19}Et Hadad trouva grâce aux yeux de Pharaon, de sorte que Pharaon lui donna pour femme la sœur de sa propre femme, la sœur de la reine Thachpenès.
\VS{20}Et la sœur de Thachpenès lui enfanta son fils Guenubath. Thachpenès le sevra dans la maison de Pharaon. Ainsi Guenubath fut dans la maison de Pharaon, parmi les fils de Pharaon.
\VS{21}Lorsque Hadad apprit en Egypte que David s'était endormi avec ses pères, et que Joab, chef de l'armée, était mort, il dit à Pharaon : Laisse-moi partir dans mon pays.
\VS{22}Et Pharaon lui répondit : Que te manque-t-il auprès de moi, pour désirer ainsi t'en aller dans ton pays ? Et il répondit : Je n’ai besoin de rien, mais cependant laisse-moi partir.
\VS{23}Dieu suscita aussi un autre ennemi à Salomon, savoir Rezon, fils d'Eliada, qui s'était enfui de chez son maître Hadadézer, roi de Tsoba,
\VS{24}Il avait rassemblé des gens auprès de lui, et était devenu chef de bandes, lorsque David les fit périr ; et ils s'en allèrent à Damas, s’y établirent et y régnèrent.
\VS{25}Rezon fut ennemi d'Israël au temps de Salomon, en même temps qu’Hadad le mettait à mal, il avait en aversion Israël et il régna sur la Syrie.
\VS{26}Jéroboam aussi, serviteur de Salomon, s'éleva également contre le roi. Il était fils de Nebath, Ephratien, de Tseréda, dont la mère s’appelait Tserua, femme veuve.
\VS{27}Voici à quelle occasion il s'éleva contre le roi. Salomon bâtissait Millo, et fermait la brèche de la cité de David, son père.
\VS{28}Jéroboam était un homme fort et vaillant ; et Salomon, voyant ce jeune homme à l’ouvrage, lui assigna la charge de toute la maison de Joseph.
\VS{29}Dans ce même temps, Jéroboam, étant sorti de Jérusalem, rencontra en chemin le prophète Achija de Silo, revêtu d'un manteau neuf, et ils étaient eux deux tout seuls dans les champs.
\VS{30}Et Achija prit le manteau neuf qu'il avait sur lui et le déchira en douze morceaux,
\VS{31}et il dit à Jéroboam : Prends-en pour toi dix morceaux ! Car ainsi parle Yahweh, le Dieu d'Israël : Voici, je vais arracher le royaume d'entre les mains de Salomon, et je t'en donnerai dix tribus.
\VS{32}Mais il aura une tribu, pour l'amour de David, mon serviteur, et pour l'amour de Jérusalem, qui est la ville que j'ai choisie d'entre toutes les tribus d'Israël.
\VS{33}Parce qu'ils m'ont abandonné, et se sont prosternés devant Astarté, la déesse des Sidoniens, devant Kemosch, dieu de Moab, et devant Milcom, le dieu des fils d’Ammon, et qu'ils n'ont point marché dans mes voies, pour faire ce qui est droit à mes yeux et garder mes statuts, et mes ordonnances, comme l’a fait David, père de Salomon.
\VS{34}Toutefois, je n'ôterai pas de sa main tout le royaume, car pendant toute sa vie je le maintiendrai prince, pour l'amour de David, mon serviteur, que j'ai choisi et qui a observé mes commandements et mes lois.
\VS{35}Mais j'ôterai le royaume d'entre les mains de son fils, je t'en donnerai dix tribus ;
\VS{36}j'en donnerai une tribu à son fils, afin que David, mon serviteur, ait une lampe à toujours devant moi dans Jérusalem, qui est la ville que j'ai choisie pour y mettre mon Nom.
\VS{37}Je te prendrai donc, tu régneras sur tout ce que ton âme désirera, tu seras roi sur Israël.
\VS{38}Et il arrivera que si tu m'obéis en tout ce que je te commanderai, que tu marches dans mes voies, en faisant tout ce qui est droit à mes yeux, en gardant mes statuts et mes commandements, comme l’a fait David, mon serviteur, je serai avec toi, je te bâtirai une maison qui sera stable, comme j'en ai bâti une à David, et je te donnerai Israël.
\VS{39}Ainsi j’humilierai la postérité de David à cause de cela, mais non pas à toujours.
\VS{40}Salomon chercha à faire mourir Jéroboam, mais Jéroboam se leva et s'enfuit en Egypte vers Schischak, roi d'Egypte ; et il demeura en Egypte jusqu'à la mort de Salomon.
\TextTitle{Mort de Salomon\FTNTT{2 Ch. 9:29-31}}
\VS{41}Or, le reste des faits de Salomon, tout ce qu'il a fait et sa sagesse, cela n'est-il pas écrit dans le livre des actes de Salomon ?
\VS{42}Salomon régna à Jérusalem sur tout Israël pendant quarante ans.
\VS{43}Ainsi Salomon s'endormit avec ses pères, il fut enseveli dans la cité de David, son père. Et Roboam, son fils, régna en sa place.
\Chap{12}
\TextTitle{Règne de Roboam\FTNTT{2 Ch. 10:1 ; cp. Ec. 2:18-19}}
\VerseOne{}Roboam se rendit à Sichem, parce que tout Israël était venu à Sichem pour l'établir roi.
\VS{2}Or, Jéroboam, fils de Nebath, était encore en Egypte, où il s'était enfui de devant le roi Salomon, quand il l'apprit, et c’était en Egypte qu’il habitait.
\VS{3}On l'envoya appeler. Ainsi Jéroboam et toute l'assemblée d'Israël vinrent, ils parlèrent à Roboam, en disant :
\VS{4}Ton père a mis sur nous un pesant joug ; mais toi allège maintenant cette rude servitude de ton père et ce pesant joug qu'il a mis sur nous ; et nous te servirons.
\VS{5}Il leur répondit : Allez, et dans trois jours revenez vers moi. Et le peuple s'en alla.
\VS{6}Le roi Roboam consulta les vieillards qui avaient été auprès de Salomon, son père, pendant sa vie et leur dit : Que me conseillez-vous de répondre à ce peuple ?
\VS{7}Et ils lui répondirent, en disant : Si aujourd'hui tu rends service à ce peuple et que tu leur cèdes, et si tu leur réponds avec des paroles bienveillantes, ils seront tes serviteurs à toujours.
\VS{8}Mais Roboam laissa le conseil que les vieillards lui avaient donné et consulta les jeunes gens qui avaient grandi avec lui et qui se tenaient près de lui.
\VS{9}Il leur dit : Que me conseillez-vous de répondre à ce peuple qui m'a parlé, en disant : Allège le joug que ton père a mis sur nous ?
\VS{10}Alors les jeunes gens qui avaient grandi avec lui, lui dirent : Tu parleras ainsi à ce peuple qui t'est venu dire : Ton père a mis sur nous un pesant joug, mais toi allège-le-nous ! Tu leur parleras ainsi : Mon petit doigt est plus gros que les reins de mon père.
\VS{11}Or, mon père a mis sur vous un pesant joug, mais moi je rendrai votre joug encore plus pesant ; mon père vous a châtiés avec des fouets, mais moi je vous châtierai avec des scorpions.
\VS{12}Or, trois jours après, Jéroboam avec tout le peuple vint vers Roboam, selon que le roi leur avait dit : Retournez vers moi dans trois jours.
\VS{13}Mais le roi répondit durement au peuple, laissant le conseil que les anciens lui avaient donné.
\VS{14}Il leur parla selon le conseil des jeunes gens, en leur disant : Mon père a mis sur vous un pesant joug, mais moi, je rendrai votre joug plus pesant encore ; mon père vous a châtiés avec des fouets, mais moi, je vous châtierai avec des scorpions.
\VS{15}Le roi donc n'écouta point le peuple ; car cela était ainsi conduit par Yahweh, en vue d’accomplir la parole qu'il avait prononcée par le ministère d'Achija de Silo, à Jéroboam, fils de Nebath.
\TextTitle{Schisme du royaume ; Jéroboam devient roi d’Israël\FTNTT{2 Ch. 10:12-19 ; 11:1-4}}
\VS{16}Et quand tout Israël vit que le roi ne les avait point écoutés, le peuple fit cette réponse au roi, en disant : Quelle part avons-nous avec David ? Nous n'avons point de propriété avec le fils d'Isaï ! A tes tentes, Israël ! Et toi David, pourvois maintenant à ta maison ! Ainsi Israël s'en alla dans ses tentes.
\VS{17}Les fils d'Israël qui habitaient dans les villes de Juda furent les seuls sur qui Roboam régna.
\VS{18}Or, le roi Roboam envoya Adoram, qui était préposé aux impôts, mais tout Israël le lapida, et il mourut. Alors le roi Roboam se hâta de monter sur un char pour s'enfuir à Jérusalem.
\VS{19}C’est ainsi qu’Israël s’est détaché de la maison de David jusqu'à ce jour.
\VS{20}Tout Israël apprit que Jéroboam était de retour, ils l'envoyèrent appeler dans l'assemblée, et l'établirent roi sur tout Israël. La tribu de Juda fut la seule qui suivit la maison de David.
\VS{21}Roboam arriva à Jérusalem, il rassembla toute la maison de Juda et la tribu de Benjamin, savoir cent quatre-vingt mille hommes d’élite choisis et disposés à faire la guerre, pour combattre contre la maison d'Israël, et ramener la domination à Roboam, fils de Salomon.
\VS{22}Mais la parole de Dieu fut ainsi adressée à Schemaeja, homme de Dieu, disant :
\VS{23}Parle à Roboam, fils de Salomon, roi de Juda, et à toute la maison de Juda, et de Benjamin, et au reste du peuple, en disant :
\VS{24}Ainsi parle Yahweh : Vous ne monterez point et vous ne combattrez point contre vos frères, les fils d'Israël ! Que chacun de vous retourne dans sa maison, car ceci a été fait de par moi. Ils obéirent à la parole de Yahweh, et s'en retournèrent, selon la parole de Yahweh.
\TextTitle{Idolâtrie de Jéroboam}
\VS{25}Or, Jéroboam bâtit Sichem sur la montagne d'Ephraïm, et y demeura, puis il en sortit et bâtit Penuel.
\VS{26}Et Jéroboam dit en son cœur : Maintenant le royaume pourrait bien retourner à la maison de David.
\VS{27}Si ce peuple monte à Jérusalem pour faire des sacrifices dans la maison de Yahweh, le cœur de ce peuple se tournera vers son seigneur, Roboam, roi de Juda, et ils me tueront, et ils retourneront à Roboam, roi de Juda.
\VS{28}Sur quoi le roi ayant pris conseil, fit deux veaux d'or et dit au peuple : Vous êtes longtemps montés à Jérusalem ! Voici ton dieu, ô Israël, qui t'a fait sortir hors du pays d'Egypte.
\VS{29}Il plaça un de ces veaux à Béthel, et il mit l'autre à Dan.
\VS{30}Et cela fut une occasion de péché, car le peuple allait jusqu'à Dan, pour se prosterner devant l'un des veaux.
\VS{31}Il fit aussi des maisons dans les hauts lieux, et établit des sacrificateurs pris parmi tout le peuple, qui n'étaient point des enfants de Lévi.
\VS{32}Jéroboam ordonna aussi une fête solennelle au huitième mois, le quinzième jour du mois, à l'imitation de la fête solennelle qu'on célébrait en Juda, et il offrait des sacrifices sur un autel. Il fit ainsi à Béthel, sacrifiant aux veaux qu'il avait faits, et il établit à Béthel des sacrificateurs des hauts lieux qu'il avait élevés.
\VS{33}Or, le quinzième jour du huitième mois, savoir au mois qu'il avait choisi lui-même, il monta sur l'autel qu'il avait fait à Béthel, et célébra cette fête solennelle pour les enfants d'Israël ; et fit brûler des parfums sur l'autel.
\Chap{13}
\TextTitle{Un homme de Dieu envoyé vers Jéroboam}
\VerseOne{}Et voici, un homme de Dieu vint de Juda à Béthel avec la parole de Yahweh, pendant que Jéroboam se tenait près de l'autel pour brûler des parfums.
\VS{2}Et il cria contre l'autel selon la parole de Yahweh, et dit : Autel ! Autel ! Ainsi parle Yahweh : Voici, un fils naîtra à la maison de David, qui aura pour nom Josias ; il immolera sur toi les sacrificateurs des hauts lieux qui brûlent des parfums sur toi, et on brûlera sur toi des ossements d’hommes !
\VS{3}Le même jour il donna un signe, en disant : C'est ici le signe dont Yahweh a parlé : Voici, l'autel se fendra, et la cendre qui est dessus sera répandue.
\VS{4}Lorsque le roi entendit la parole que l'homme de Dieu avait criée contre l'autel de Béthel, Jéroboam étendit sa main de l'autel, en disant : Saisissez-le ! Et la main qu'il étendit contre lui devint sèche, et il ne put la ramener à lui.
\VS{5}L'autel aussi se fendit, et la cendre qui était sur l'autel fut répandue, selon le signe que l'homme de Dieu avait donné par la parole de Yahweh.
\VS{6}Alors le roi prit la parole et dit à l'homme de Dieu : Implore Yahweh, ton Dieu, et prie pour moi, afin que ma main revienne à moi. L'homme de Dieu implora Yahweh, et la main du roi put revenir à lui et elle fut comme auparavant.
\VS{7}Alors le roi dit à l'homme de Dieu : Entre avec moi dans la maison, tu prendras quelque nourriture et je te donnerai un présent.
\VS{8}Mais l'homme de Dieu répondit au roi : Quand tu me donnerais la moitié de ta maison, je n'entrerais point chez toi, je ne mangerais point de pain, ni ne boirais d'eau en ce lieu.
\VS{9}Car cela m'a été ordonné par Yahweh, qui m'a dit : Tu ne mangeras point de pain, tu ne boiras point d'eau et tu ne t'en retourneras point par le chemin par lequel tu y seras allé.
\VS{10}Il s'en alla donc par un autre chemin, et ne s'en retourna point par le chemin par lequel il était venu à Béthel.
\TextTitle{L’homme de Dieu séduit par un vieux prophète}
\VS{11}Or, il y avait un vieux prophète qui demeurait à Béthel. Ses fils vinrent raconter toutes les choses que l'homme de Dieu avait faites ce jour-là à Béthel, et les paroles qu'il avait dites au roi ; et comme les fils de ce prophète les rapportaient à leur père,
\VS{12}il leur demanda : Par quel chemin s'en est-il allé ? Or, ses fils avaient vu le chemin par lequel l'homme de Dieu qui était venu de Juda s'en était allé.
\VS{13}Et il dit à ses fils : Sellez-moi un âne. Ils lui sellèrent, puis il monta dessus.
\VS{14}Et il s'en alla après l'homme de Dieu, et le trouva assis sous un chêne. Et il lui dit : Es-tu l'homme de Dieu qui est venu de Juda ? Et il lui répondit : C'est moi.
\VS{15}Alors il lui dit : Viens avec moi dans la maison, et tu prendras de quoi te nourrir.
\VS{16}Mais il répondit : Je ne puis retourner avec toi, ni entrer chez toi et je ne mangerai point de pain, ni ne boirai d'eau avec toi en ce lieu ;
\VS{17}Car il m'a été dit de la part de Yahweh : Tu ne mangeras point de pain, tu ne boiras point d'eau, et tu ne t'en retourneras point par le chemin par lequel tu seras allé.
\VS{18}Et il lui dit : Et moi aussi je suis prophète comme toi ; et un ange m'a parlé de la part de Yahweh, en disant : Ramène-le avec toi dans ta maison, qu'il mange du pain, et qu'il boive de l'eau ; mais il lui mentait.
\VS{19}Il s'en retourna donc avec lui, il mangea du pain et but de l'eau dans sa maison.
\VS{20}Et il arriva que comme ils étaient assis à table, la parole de Yahweh fut adressée au prophète qui l'avait ramené.
\VS{21}Et il cria à l'homme de Dieu qui était venu de Juda, en disant : Ainsi a parlé Yahweh : Parce que tu as été rebelle au commandement de Yahweh et que tu n'as point gardé l’ordre que Yahweh, ton Dieu, t'avait donné ;
\VS{22}mais tu t'en es retourné, tu as mangé du pain et bu de l'eau dans le lieu dont Yahweh t'avait dit : N'y mange point de pain et n'y bois point d'eau, ton cadavre n'entrera point au sépulcre de tes pères.
\VS{23}Et quand le prophète qu'il avait ramené eut mangé du pain et bu de l’eau, il sella l’âne pour lui.
\VS{24}L’homme de Dieu s'en alla, et un lion le rencontra dans le chemin, et le tua. Son corps était étendu dans le chemin, l'âne resta auprès du corps, et le lion aussi resta à côté du cadavre.
\VS{25}Et voici des passants virent le corps étendu dans le chemin et le lion qui se tenait auprès du corps ; et ils vinrent le dire dans la ville où le vieux prophète demeurait.
\VS{26}Et le prophète qui avait ramené du chemin l'homme de Dieu, l'ayant appris, dit : C'est l'homme de Dieu qui a été rebelle au commandement de Yahweh, c'est pourquoi Yahweh l'a livré au lion, qui l'aura déchiré après l'avoir tué, selon la parole que Yahweh avait dite à ce prophète.
\VS{27}Et il parla à ses fils, en disant : Sellez-moi un âne. Ils le lui sellèrent,
\VS{28}Et il s'en alla et trouva le corps de l'homme de Dieu étendu dans le chemin, l'âne et le lion qui se tenaient auprès du corps. Le lion n'avait pas dévoré le cadavre, ni déchiré l'âne.
\VS{29}Alors le prophète leva le corps de l'homme de Dieu, le plaça sur l'âne et le ramena ; et ce vieux prophète revint dans la ville pour le pleurer et l'enterrer.
\VS{30}Il mit le corps de ce prophète dans le sépulcre, et il pleura sur lui, en disant : Hélas, mon frère !
\VS{31}Après l’avoir enterré, il parla à ses fils, en disant : Quand je serai mort, enterrez-moi au sépulcre où est enterré l'homme de Dieu, et vous déposerez mes os à côté de ses os.
\VS{32}Car elle s’accomplira, la parole qu’il a criée de la part de Yahweh, contre l'autel qui est à Béthel et contre toutes les maisons des hauts lieux qui sont dans les villes de Samarie.
\TextTitle{Jéroboam continue dans le mal}
\VS{33}Néanmoins, Jéroboam ne se détourna point de sa mauvaise voie, mais il établit de nouveau des sacrificateurs de hauts lieux pris parmi tout le peuple ; quiconque le voulait, Jéroboam le consacrait sacrificateur des hauts lieux.
\VS{34}Cela fut une occasion de péché pour la maison de Jéroboam, qui fut effacée et exterminée de dessus la terre.
\Chap{14}
\TextTitle{Maladie et mort du fils de Jéroboam}
\VerseOne{}En ce temps-là, Abija, fils de Jéroboam, devint malade.
\VS{2}Et Jéroboam dit à sa femme : Lève-toi maintenant et déguise-toi, en sorte qu'on ne reconnaisse point que tu es la femme de Jéroboam, et va à Silo. Voici, là est Achija, le prophète, qui m'a dit que je serais roi sur ce peuple.
\VS{3}Emmène avec toi dix pains, des gâteaux et un vase de miel, et entre chez lui ; il te dira ce qui arrivera à l’enfant.
\VS{4}La femme de Jéroboam fit donc ainsi ; elle se leva et s'en alla à Silo puis elle entra dans la maison d'Achija. Or, Achija ne pouvait plus voir, parce qu’il avait les yeux figés à cause de sa vieillesse.
\VS{5}Et Yahweh dit à Achija : Voici, la femme de Jéroboam, qui vient te consulter concernant l’état de son fils, parce qu'il est malade. Tu lui parleras de telle et de telle manière. Quand elle arrivera, elle se sera déguisée.
\VS{6}Lorsque Achija eut entendu le bruit de ses pas, comme elle franchissait la porte, il dit : Entre, femme de Jéroboam. Pourquoi fais-tu semblant d'être quelqu’un d’autre ? Je suis chargé de t’annoncer des choses dures.
\VS{7}Va, dis à Jéroboam : Ainsi parle Yahweh, le Dieu d'Israël : Parce que je t'ai élevé du milieu du peuple et que je t'ai établi pour chef sur mon peuple d'Israël,
\VS{8}j'ai arraché le royaume de la maison de David et je te l'ai donné ; mais parce que tu n'as point été comme David, mon serviteur, qui a gardé mes commandements et qui a marché après moi de tout son cœur, ne faisant que ce qui est droit à mes yeux.
\VS{9}Tu as fait pire que tous ceux qui ont été devant toi, tu es allé te faire d'autres dieux et des images de fonte, pour m'irriter, et tu m'as rejeté derrière ton dos !
\VS{10}A cause de cela, voici, je vais faire venir le malheur sur la maison de Jéroboam ; je retrancherai ce qui appartient à Jéroboam, ce qu’il détient et ce qu’il néglige en Israël, et je brûlerai la maison de Jéroboam, comme on brûle les ordures, jusqu'à ce qu'il n'en reste plus.
\VS{11}Celui de la maison de Jéroboam qui mourra dans la ville, les chiens le mangeront, et celui qui mourra aux champs, les oiseaux du ciel le mangeront. Car Yahweh a parlé.
\VS{12}Toi donc lève-toi, va dans ta maison. Dès que tes pieds entreront dans la ville, l'enfant mourra.
\VS{13}Tout Israël le pleurera et on l’enterrera ; car lui seul de la famille de Jéroboam entrera au sépulcre, parce que Yahweh, le Dieu d'Israël, a trouvé quelque chose de bon en lui seul dans toute la maison de Jéroboam.
\VS{14}Yahweh s'établira un roi sur Israël qui retranchera la maison de Jéroboam. Ce jour-là, n’est-ce pas déjà ce qui arrive ?
\VS{15}Yahweh frappera Israël, l'agitant comme le roseau est agité dans l'eau ; et il arrachera Israël de ce bon pays qu'il a donné à leurs pères, et les dispersera au-delà du fleuve, parce qu'ils se sont fait des idoles, irritant Yahweh.
\VS{16}Il livrera Israël à cause des péchés que Jéroboam a commis et qu’il a fait commettre à Israël.
\VS{17}Alors la femme de Jéroboam se leva et s'en alla, elle vint à Thirtsa : et comme elle franchit le seuil de la maison, le jeune garçon mourut.
\VS{18}Il fut enseveli et tout Israël le pleura, selon la parole de Yahweh, proférée par son serviteur Achija, le prophète.
\TextTitle{Règne de Nadab sur Israël\FTNTT{cp. 2 Ch. 13:20}}
\VS{19}Quant au reste des faits de Jéroboam, comment il a fait la guerre et comment il a régné, cela est écrit dans le livre des Chroniques des rois d'Israël.
\VS{20}Jéroboam régna vingt-deux ans, puis il s'endormit avec ses pères. Et Nadab, son fils, régna à sa place.
\TextTitle{Juda dans l'apostasie\FTNTT{2 Ch. 12:1}}
\VS{21}Roboam, fils de Salomon, régna en Juda. Il avait quarante et un ans quand il devint roi, et il régna dix-sept ans à Jérusalem, la ville que Yahweh avait choisie d'entre toutes les tribus d'Israël pour y mettre son nom. Sa mère s’appelait Naama, l’Ammonite.
\VS{22}Juda fit ce qui est mal aux yeux de Yahweh ; et par les péchés qu'ils commirent, ils excitèrent sa jalousie plus que leurs pères ne l'avaient jamais fait.
\VS{23}Ils se bâtirent, eux aussi, des hauts lieux avec des statues et des idoles sur toute colline élevée, et sous tout arbre verdoyant.
\VS{24}Il y avait dans le pays des prostitués. Et ils firent selon toutes les abominations des nations que Yahweh avait chassées devant les enfants d'Israël.
\TextTitle{Le roi d’Egypte emporte les trésors de Juda ; mort de Roboam\FTNTT{2 Ch. 12:2-16}}
\VS{25}La cinquième année du roi Roboam, Schischak, roi d'Egypte, monta contre Jérusalem.
\VS{26}Il prit les trésors de la maison de Yahweh et les trésors de la maison royale, et il emporta tout. Il prit aussi tous les boucliers d'or que Salomon avait faits.
\VS{27}Le roi Roboam fit des boucliers d'airain au lieu de ceux-là, et les mit entre les mains des chefs des coureurs, qui gardaient l’entrée de la maison du roi.
\VS{28}Toutes les fois où le roi entrait dans la maison de Yahweh, les coureurs les portaient, et ensuite ils les rapportaient dans la chambre des coureurs.
\VS{29}Le reste des actions de Roboam, et tout ce qu'il a fait, n'est-il pas écrit au livre des Chroniques des rois de Juda ?
\VS{30}Il y eut toujours guerre entre Roboam et Jéroboam.
\VS{31}Roboam s'endormit avec ses pères et fut enseveli avec eux dans la cité de David. Sa mère avait pour nom Naama, l’Ammonite. Et Abijam, son fils, régna à sa place.
\Chap{15}
\TextTitle{Règne d'Abijam (ou Abija) sur Juda\FTNTT{2 Ch. 13:1-2}}
\VerseOne{}La dix-huitième année du roi Jéroboam, fils de Nebath, Abijam commença à régner sur Juda.
\VS{2}Il régna trois ans à Jérusalem. Sa mère s’appelait Maaca et était fille d'Abisalom.
\VS{3}Il marcha dans tous les péchés que son père avait commis avant lui ; son cœur ne fut point intègre envers Yahweh, son Dieu, comme l'avait été le cœur de David, son père.
\VS{4}Mais pour l'amour de David, Yahweh, son Dieu, lui donna une lampe dans Jérusalem, lui suscitant son fils après lui et laissant subsister Jérusalem ;
\VS{5}Parce que David avait fait ce qui est droit devant Yahweh, et que pendant toute sa vie il ne s'était point détourné d’aucun de ses commandements, hormis dans l'affaire d'Urie, le Héthien.
\VS{6}Or, il y eut toujours guerre entre Roboam et Jéroboam, pendant toute la vie de Roboam.
\VS{7}Le reste des actions d'Abijam, et même tout ce qu'il fit, n'est-il pas écrit au livre des Chroniques des rois de Juda ? Il y eut aussi guerre entre Abijam et Jéroboam.
\VS{8}Ainsi Abijam s'endormit avec ses pères, et on l'enterra dans la cité de David. Et Asa, son fils, régna à sa place.
\TextTitle{Règne d’Asa sur Juda\FTNTT{2 Ch. 14:1-5 ; 15:1-19}}
\VS{9}La vingtième année de Jéroboam, roi d'Israël, Asa commença à régner sur Juda.
\VS{10}Il régna quarante et un ans à Jérusalem. Sa mère avait pour nom Maaca, elle était fille d'Abisalom.
\VS{11}Asa fit ce qui est droit devant Yahweh, comme David, son père.
\VS{12}Il ôta du pays les prostitués, et ôta toutes les idoles que ses pères avaient faites.
\VS{13}Et même il ôta la dignité de reine à sa mère Maaca, parce qu'elle avait fait une idole pour Astarté. Asa mit en pièces l’idole qu'elle avait faite, et la brûla au torrent de Cédron.
\VS{14}Mais les hauts lieux ne furent point ôtés. Néanmoins, le cœur d'Asa fut intègre envers Yahweh pendant toute sa vie.
\TextTitle{Guerre entre Juda et Israël ; Asa s’allie avec la Syrie\FTNTT{1 Ch. 14:6-15 ; 16:1-10}}
\VS{15}Il remit dans la maison de Yahweh les choses qui avaient été consacrées par son père et par lui-même, de l'argent, de l'or et les ustensiles.
\VS{16}Or, il y eut guerre entre Asa et Baescha, roi d'Israël, pendant toute leur vie.
\VS{17}Baescha, roi d'Israël, monta contre Juda, et bâtit Rama, pour empêcher quiconque de sortir et entrer vers Asa, roi de Juda.
\VS{18}Asa prit tout l'argent et l'or qui était resté dans les trésors de Yahweh et dans les trésors de la maison royale, et les donna à ses serviteurs ; le roi Asa les envoya vers Ben-Hadad, fils de Thabrimmon, fils de Hezjon, roi de Syrie, qui demeurait à Damas, pour lui dire :
\VS{19}Qu’il y ait alliance entre moi et toi, comme entre mon père et le tien. Voici, je t'envoie un présent en argent et en or. Va, romps l'alliance que tu as avec Baescha, roi d'Israël, afin qu'il se retire de moi.
\VS{20}Et Ben-Hadad écouta le roi Asa ; il envoya les chefs de son armée contre les villes d'Israël, et il battit Ijjon, Dan, Abel-Beth-Maaca, tout Kinneroth, et tout le pays de Nephthali.
\VS{21}Lorsque Baescha l’apprit, il cessa de bâtir Rama et demeura à Thirtsa.
\VS{22}Alors le roi Asa fit publier par tout Juda que tous, sans en excepter aucun, eussent à emporter les pierres et le bois de Rama, que Baescha faisait bâtir, et le roi Asa s’en servit pour bâtir Guéba de Benjamin, et Mitspa.
\TextTitle{Mort d’Asa ; Josaphat règne sur Juda\FTNTT{1 Ch. 16:11-17:1}}
\VS{23}Le reste de toutes les actions d'Asa, tous ses exploits, tout ce qu'il fit, et les villes qu'il a bâties, cela n'est-il pas écrit au livre des Chroniques des rois de Juda ? Au reste, il fut malade de ses pieds au temps de sa vieillesse.
\VS{24}Et Asa s'endormit avec ses pères, avec lesquels il fut enseveli en la cité de David, son père. Et, son fils, Josaphat, régna à sa place.
\TextTitle{Baescha tue Nadab et devient roi d'Israël}
\VS{25}Or, Nadab, fils de Jéroboam, régna sur Israël la seconde année d'Asa, roi de Juda, et il régna deux ans sur Israël.
\VS{26}Il fit ce qui est mal aux yeux de Yahweh ; et il marcha dans la voie de son père, se livrant aux péchés que son père avait fait commettre à Israël.
\VS{27}Et Baescha, fils d'Achija, de la maison d'Issacar, fit une conspiration contre lui. Il le tua devant Guibbethon, qui était aux Philistins, lorsque Nadab et tout Israël assiégeaient Guibbethon.
\VS{28}Baescha le fit donc mourir la troisième année d'Asa, roi de Juda, et il régna à sa place.
\VS{29}Une fois proclamé roi, il frappa toute la maison de Jéroboam et ne laissa échapper aucune âme vivante, il détruisit tout ce qui respirait, selon la parole de Yahweh qu'il avait proférée par son serviteur Achija, Silonite,
\VS{30}A cause des péchés que Jéroboam avait commis et fait commettre à Israël, irritant ainsi Yahweh, le Dieu d'Israël.
\VS{31}Le reste des faits de Nadab, et même tout ce qu'il a fait, n'est-il pas écrit au livre des Chroniques des rois d'Israël ?
\VS{32}Or, il y eut guerre entre Asa et Baescha, roi d'Israël, pendant toute leur vie.
\VS{33}La troisième année d'Asa, roi de Juda, Baescha, fils d'Achija, commença à régner sur tout Israël à Thirtsa, il régna vingt-quatre ans.
\VS{34}Et il fit ce qui est mal aux yeux de Yahweh, et marcha dans la voie de Jéroboam, en se livrant aux péchés que Jéroboam avait fait commettre à Israël.
\Chap{16}
\TextTitle{Yahweh avertit Baescha avant sa mort}
\VerseOne{}Alors la parole de Yahweh fut adressée à Jéhu, fils de Hanani, contre Baescha, en ces mots :
\VS{2}Je t'ai élevé de la poussière et je t'ai établi chef de mon peuple d'Israël ; malgré cela tu as suivi la voie de Jéroboam et fait pécher mon peuple d'Israël, pour m'irriter par leurs péchés.
\VS{3}Voici, je m'en vais entièrement consumer Baescha et sa maison, et je rendrai ta maison semblable à la maison de Jéroboam, fils de Nebath.
\VS{4}Celui de la maison de Baescha qui mourra dans la ville, les chiens le mangeront, et celui des siens qui mourra aux champs, les oiseaux du ciel le mangeront.
\VS{5}Le reste des faits de Baescha, ce qu'il a fait et ses exploits, n'est-il pas écrit au livre des Chroniques des rois d'Israël ?
\VS{6}Ainsi Baescha s'endormit avec ses pères et fut enseveli à Thirtsa. Ela, son fils, régna à sa place.
\VS{7}La parole de Yahweh fut aussi adressée par le moyen de Jéhu, fils d'Hanani, le prophète, contre Baescha et contre sa maison, tant à cause de tout le mal qu'il avait fait devant Yahweh, en l'irritant par l'œuvre de ses mains et en devenant comme la maison de Jéroboam, que parce qu'il l'avait détruite.
\TextTitle{Ela puis Zimri règnent sur Israël}
\VS{8}La vingt-sixième année d'Asa, roi de Juda, Ela, fils de Baescha, commença à régner sur Israël et il régna deux ans à Thirtsa.
\VS{9}Son serviteur, Zimri, capitaine de la moitié des chars, fit une conspiration contre Ela, lorsqu'il était à Thirtsa, buvant et s'enivrant dans la maison d'Artsa, chef de la maison du roi à Thirtsa.
\VS{10}Alors, Zimri vint, le frappa et le tua, la vingt-septième année d'Asa, roi de Juda, et il régna à sa place.
\VS{11}Dès qu’il fut roi et qu'il fut assis sur son trône, il frappa toute la maison de Baescha, il n'en laissa échapper personne qui lui appartint, ni parent, ni ami.
\VS{12}Ainsi Zimri extermina toute la maison de Baescha, selon la parole que Yahweh avait proférée contre Baescha, par Jéhu, le prophète,
\VS{13}A cause de tous les péchés de Baescha, et des péchés d'Ela, son fils, qu’ils avaient commis et qu’ils avaient fait commettre à Israël, irritant Yahweh, le Dieu d'Israël, par leurs idoles.
\VS{14}Le reste des faits d'Ela, et même tout ce qu'il a fait, n'est-il pas écrit au livre des Chroniques des rois d'Israël ?
\VS{15}La vingt-septième année d'Asa, roi de Juda, Zimri régna sept jours à Thirtsa. Or, le peuple était campé contre Guibbethon qui appartenait aux Philistins.
\VS{16}Et le peuple qui était campé là entendit que l'on disait : Zimri a fait une conspiration, et il a même tué le roi ! En ce même jour, tout Israël établit dans le camp pour roi d’Israël Omri, chef de l'armée d'Israël.
\VS{17}Omri et tout Israël avec lui partirent de Guibbethon, et assiégèrent Thirtsa.
\VS{18}Mais dès que Zimri vit que la ville était prise, il entra au palais de la maison royale et brûla sur lui la maison royale, il mourut ainsi,
\VS{19}A cause des péchés qu’il avait commis, faisant ce qui est mal aux yeux de Yahweh, en suivant la voie de Jéroboam et le péché qu'il avait fait commettre à Israël.
\VS{20}Le reste des actions de Zimri et la conspiration qu'il forma, cela n’est-il pas écrit dans le livre des Chroniques des rois d'Israël ?
\TextTitle{Omri règne sur Israël}
\VS{21}Alors le peuple d'Israël se divisa en deux partis : la moitié du peuple voulait faire roi Thibni, fils de Guinath ; et l'autre moitié suivait Omri.
\VS{22}Mais le peuple qui suivait Omri, fut plus fort que le peuple qui suivait Thibni, fils de Guinath. Thibni mourut et Omri régna.
\VS{23}La trente et unième année d'Asa, roi de Juda, Omri commença à régner sur Israël et il régna douze ans après avoir régné six ans à Thirtsa.
\VS{24}Puis il acheta de Schémer la montagne de Samarie, deux talents d'argent ; il bâtit une ville sur cette montagne et nomma la ville qu'il bâtit, du nom de Schémer, seigneur de la montagne.
\VS{25}Omri fit ce qui est mal aux yeux de Yahweh ; il agit même plus mal que tous ceux qui avaient été avant lui.
\VS{26}Il marcha dans la voie de Jéroboam, fils de Nebath, et se livra aux péchés que Jéroboam avait fait commettre à Israël, irritant Yahweh, le Dieu d'Israël, par leurs idoles.
\VS{27}Le reste des actions d’Omri, tout ce qu'il a fait et ses exploits, cela n’est-il pas écrit au livre des Chroniques des rois d'Israël ?
\VS{28}Ainsi Omri s'endormit avec ses pères et fut enseveli à Samarie. Achab, son fils, régna à sa place.
\TextTitle{Achab règne sur Israël et épouse Jézabel}
\VS{29}Achab, fils d’Omri, régna sur Israël la trente-huitième année d'Asa, roi de Juda. Et Achab, fils d’Omri, régna sur Israël à Samarie vingt-deux ans.
\VS{30}Et Achab, fils d’Omri, fit ce qui est mal aux yeux de Yahweh, plus que tous ceux qui avaient été avant lui.
\VS{31}Et il arriva que, comme si ce lui eût été peu de chose de marcher dans les péchés de Jéroboam, fils de Nebath, il prit pour femme Jézabel, fille d'Ethbaal, roi des Sidoniens, puis il alla servir Baal et se prosterna devant lui.
\VS{32}Il dressa un autel à Baal, dans la maison de Baal, qu'il bâtit à Samarie.
\VS{33}Et Achab fit une idole d’Astarté. De sorte qu'Achab fit plus encore que tous les rois d'Israël qui avaient été avant lui, pour irriter Yahweh, le Dieu d'Israël.
\VS{34}En son temps, Hiel de Béthel bâtit Jéricho ; il en jeta les fondements au prix d’Abiram, son premier-né, et posa ses portes sur Segub, son plus jeune fils, selon la parole que Yahweh avait proférée par le moyen de Josué, fils de Nun.
\Chap{17}
\TextTitle{Elie annonce trois ans de sécheresse\FTNTT{1 R. 17-2 R. 1}}
\VerseOne{}Alors Elie, Thischbite, l’un des habitants de Galaad, dit à Achab : Yahweh, le Dieu d'Israël, en la présence duquel je me tiens, est vivant ! Il n'y aura ces années-ci ni rosée ni pluie, sinon à ma parole.
\TextTitle{Elie au torrent de Kerith}
\VS{2}Puis la parole de Yahweh fut adressée à Elie, en disant :
\VS{3}Va-t'en d'ici et tourne-toi vers l'orient ; cache-toi près du torrent de Kerith, qui est en face du Jourdain.
\VS{4}Tu boiras de l’eau du torrent, et j'ai commandé aux corbeaux de t'y nourrir.
\VS{5}Il partit donc et fit selon la parole de Yahweh, il s'en alla et demeura au torrent de Kerith, vis-à-vis du Jourdain.
\VS{6}Les corbeaux lui apportaient du pain et de la viande le matin, et du pain et de la viande le soir, et il buvait de l’eau du torrent.
\VS{7}Mais il arriva qu'au bout d’un certain temps le torrent tarit, parce qu'il n'y avait point eu de pluie dans le pays.
\TextTitle{Elie chez la veuve de Sarepta}
\VS{8}Alors la parole de Yahweh lui fut adressée, en ces mots :
\VS{9}Lève-toi, va à Sarepta, qui appartient à Sidon, et demeure-là. Voici, j'ai commandé là à une femme veuve de t'y nourrir.
\VS{10}Il se leva donc et s'en alla à Sarepta. Et comme il fut arrivé à l’entrée de la ville, voici, une femme veuve était là, qui ramassait du bois. Et il l'appela et lui dit : Apporte-moi, je te prie, un peu d'eau dans un vase et que je boive.
\VS{11}Elle alla en chercher. Il l’appela de nouveau et dit : Apporte-moi, je te prie, un morceau de pain de ta main.
\VS{12}Mais elle répondit : Yahweh, ton Dieu, est vivant ! Je n'ai rien de cuit, je n'ai qu’une poignée de farine dans un pot et un peu d'huile dans une cruche. Et voici, j'amasse deux morceaux de bois, puis je rentrerai, je l'apprêterai pour moi et pour mon fils, nous le mangerons, après quoi nous mourrons.
\VS{13}Et Elie lui dit : Ne crains point, va, fais comme tu dis. Seulement, fais-moi d’abord avec cela un petit gâteau et tu me l’apporteras, tu en feras ensuite pour toi et pour ton fils.
\VS{14}Car ainsi parle Yahweh, le Dieu d'Israël : La farine qui est dans le pot ne finira point et l'huile qui est dans la cruche ne diminuera point, jusqu'à ce que Yahweh donne de la pluie sur la terre.
\VS{15}Elle s'en alla donc, et fit selon la parole d'Elie. Et elle eut à manger, elle et sa famille, ainsi qu’Elie pendant plusieurs jours.
\VS{16}La farine du pot ne finit point, et l'huile de la cruche ne diminua point, selon la parole que Yahweh avait prononcée par le moyen d'Elie.
\TextTitle{Résurrection du fils de la veuve de Sarepta}
\VS{17}Après ces choses, il arriva que le fils de la femme, maîtresse de la maison, devint malade ; et la maladie fut si forte, qu'il expira.
\VS{18}Et elle dit à Elie : Qu'y a-t-il entre moi et toi, homme de Dieu ? Es-tu venu chez moi pour rappeler le souvenir de mon iniquité, et pour faire mourir mon fils ?
\VS{19}Et il lui dit : Donne-moi ton fils. Et il le prit du sein de cette femme, le porta dans la chambre haute où il demeurait, et le coucha sur son lit.
\VS{20}Puis il cria à Yahweh, et dit : Yahweh, mon Dieu ! Affligeras-tu cette veuve au point de faire mourir son fils, elle qui a m’a reçu comme un hôte ?
\VS{21}Et il s'étendit sur l'enfant par trois fois, et cria à Yahweh, en disant : Yahweh, mon Dieu ! Je te prie que l'âme de cet enfant revienne au-dedans de lui.
\VS{22}Et Yahweh écouta la voix d'Elie, l'âme de l'enfant revint au-dedans de lui, et il fut rendu à la vie.
\VS{23}Elie prit l'enfant, le descendit de la chambre haute dans la maison et le donna à sa mère, en lui disant : Regarde, ton fils est vivant.
\VS{24}Et la femme dit à Elie : Je reconnais maintenant, que tu es un homme de Dieu et que la parole de Yahweh, qui est dans ta bouche, est vérité.
\Chap{18}
\TextTitle{Elie à la rencontre d’Abdias puis d’Achab}
\VerseOne{}Et il arriva, après bien des jours, que la parole de Yahweh fut adressée à Elie, dans la troisième année, en disant : Va, montre-toi à Achab et je ferai tomber de la pluie sur la terre.
\VS{2}Et Elie s'en alla pour se présenter devant Achab. Il y avait alors une grande famine en Samarie.
\VS{3}Achab avait appelé Abdias, chef de sa maison ; or, Abdias craignait beaucoup Yahweh ;
\VS{4}quand Jézabel exterminait les prophètes de Yahweh, Abdias prit cent prophètes et les cacha, cinquante dans une caverne et cinquante dans une autre, et il les y nourrit de pain et d'eau.
\VS{5}Achab dit alors à Abdias : Va par le pays vers toutes les sources d'eaux et vers tous les torrents ; peut-être que nous trouverons de l'herbe, nous garderons ainsi en vie les chevaux et les mulets, et nous n’aurons pas besoin d’abattre du bétail.
\VS{6}Ils se partagèrent donc entre eux le pays pour le parcourir ; Achab allait seul par un chemin et Abdias allait seul par un autre chemin.
\VS{7}Comme Abdias était en chemin, voici, Elie le rencontra. Abdias reconnut Elie, il tomba sur son visage et lui dit : N'es-tu pas mon seigneur Elie ?
\VS{8}Il lui répondit : C'est moi ; va et dis à ton seigneur : Voici Elie !
\VS{9}Et Abdias dit : Quel péché ai-je commis, pour que tu livres ton serviteur entre les mains d'Achab pour me faire mourir ?
\VS{10}Yahweh, ton Dieu, est vivant ! Il n'y a ni nation, ni royaume, où mon seigneur n'ait envoyé pour te chercher ; et quand on répondait que tu n'y étais pas, il faisait jurer aux rois et au peuple que l'on ne t’avait pas trouvé.
\VS{11}Et maintenant tu dis : Va, dis à ton seigneur, voici Elie !
\VS{12}Puis, lorsque je t’aurai quitté, l'Esprit de Yahweh te transportera je ne sais où et j’irai informer Achab qui ne te trouvera pas et qui me tuera. Or, ton serviteur craint Yahweh dès sa jeunesse.
\VS{13}N'a-t-on point dit à mon seigneur ce que je fis quand Jézabel tuait les prophètes de Yahweh, comment j'en cachai cent, cinquante dans une caverne et cinquante dans une autre et les y ai nourris de pain et d'eau ?
\VS{14}Et maintenant tu dis : Va, dis à ton seigneur : Voici Elie ! Il me tuera !
\VS{15}Mais Elie lui répondit : Yahweh des armées, devant lequel je me tiens, est vivant ! Aujourd'hui, je me montrerai à Achab.
\VS{16}Abdias étant allé à la rencontre d'Achab, l’informa de la chose ; puis Achab alla au-devant d'Elie.
\VS{17}Et aussitôt qu'Achab eut vu Elie, il lui dit : Est-ce toi qui jettes le trouble en Israël ?
\VS{18}Et Elie lui répondit : Je n'ai point troublé Israël ; c'est toi et la maison de ton père, puisque vous avez abandonné les commandements de Yahweh et que vous êtes allés après les Baals.
\VS{19}Fais maintenant se rassembler tout Israël auprès de moi, sur le mont Carmel, les quatre cent cinquante prophètes de Baal et les quatre cents prophètes d’Astarté qui mangent à la table de Jézabel.
\TextTitle{Confrontation entre Elie et les prophètes de Baal sur le mont Carmel}
\VS{20}Ainsi Achab envoya des messagers vers tous les fils d'Israël et, il rassembla les prophètes sur le mont Carmel.
\VS{21}Alors Elie s'approcha de tout le peuple et dit : Jusqu'à quand clocherez-vous des deux côtés ? Si Yahweh est Dieu, suivez-le ; mais si Baal est dieu, suivez-le. Et le peuple ne lui répondit pas un seul mot.
\VS{22}Alors Elie dit au peuple : Je suis demeuré seul prophète de Yahweh ; et voici quatre cent cinquante prophètes de Baal.
\VS{23}Que l’on nous donne deux veaux, qu'ils en choisissent l'un pour eux, qu'ils le coupent en pièces et qu'ils le mettent sur du bois ; mais qu'ils n'y mettent point de feu ; et je préparerai l'autre veau, je le mettrai sur du bois, sans y mettre le feu.
\VS{24}Puis invoquez le nom de vos dieux, et moi j'invoquerai le nom de Yahweh ; que le dieu qui répondra par le feu, soit reconnu pour être Dieu. Et tout le peuple répondit et dit : C'est bien !
\VS{25}Et Elie dit aux prophètes de Baal : Choisissez un veau et préparez-le les premiers, car vous êtes en plus grand nombre et invoquez le nom de vos dieux ; mais n'y mettez point de feu.
\VS{26}Ils prirent donc un veau qu'on leur donna, ils l'apprêtèrent et ils invoquèrent le nom de Baal depuis le matin jusqu'à midi, en disant : Baal exauce-nous ! Mais il n'y avait ni voix ni réponse et ils sautaient devant l'autel qu'ils avaient fait.
\VS{27}A midi, Elie se moqua d'eux et dit : Criez à haute voix, puisqu’il est dieu ; mais il pense à quelque chose, ou il est occupé, ou il est en voyage ; peut-être qu'il dort et il se réveillera.
\VS{28}Ils criaient donc à haute voix ; ils se faisaient des incisions avec des couteaux et des lances, selon leur coutume, en sorte que le sang coulait sur eux.
\VS{29}Lorsque midi fut passé et qu'ils eurent fait les prophètes jusqu'au temps où l’on offre l'oblation, sans qu'il y eût ni voix, ni réponse, ni signe d’attention.
\VS{30}Elie dit alors à tout le peuple : Approchez-vous de moi ! Et tout le peuple s'approcha de lui et il répara l'autel de Yahweh, qui avait été renversé.
\VS{31}Puis Elie prit douze pierres, selon le nombre des tribus des fils de Jacob, auquel la parole de Yahweh avait été adressée, en disant : Israël sera ton nom.
\VS{32}Et il rebâtit de ces pierres l'autel au nom de Yahweh. Puis il fit un fossé de la capacité de deux mesures de semence autour de l'autel.
\VS{33}Il rangea le bois, il coupa le veau en pièces, et il le plaça sur le bois.
\VS{34}Puis il dit : Remplissez quatre cruches d'eau, puis versez-les sur l'holocauste et sur le bois. Puis il dit : Faites-le encore une seconde fois. Et ils le firent une seconde fois. Il dit : Faites-le une troisième fois. Et ils le firent pour la troisième fois ;
\VS{35}de sorte que les eaux allaient à l'entour de l'autel ; et il remplit aussi d’eau le fossé.
\VS{36}Et au moment de la présentation de l’offrande, Elie, le prophète, s'approcha et dit : Ô Yahweh ! Dieu d'Abraham, d'Isaac et d'Israël ! Que l’on sache aujourd'hui que tu es Dieu en Israël et que je suis ton serviteur ; et que j'ai fait toutes ces choses par ta parole !
\VS{37}Réponds-moi, Ô Yahweh ! Réponds-moi, afin que ce peuple connaisse que c’est toi, Yahweh, qui es Dieu et que c'est toi qui ramènes leur cœur.
\VS{38}Alors le feu de Yahweh tomba et consuma l'holocauste, le bois, les pierres et la terre, et il absorba toute l'eau qui était dans le fossé.
\VS{39}Quand tout le peuple vit cela, ils tombèrent sur leur visage et dirent : C'est Yahweh qui est Dieu ! C'est Yahweh qui est Dieu !
\VS{40}Et Elie leur dit : Saisissez les prophètes de Baal et qu'il n'en échappe aucun ! Ils les saisirent. Elie les fit descendre au torrent de Kison, où il les fit égorger là.
\TextTitle{Retour de la pluie selon la parole d’Elie\FTNTT{Ja. 5:17-18}}
\VS{41}Puis Elie dit à Achab : Monte, mange et bois ; car il se fait un bruit qui annonce la pluie.
\VS{42}Ainsi Achab monta pour manger et pour boire tandis qu’Elie monta au sommet du Carmel ; et, se penchant contre terre, il mit son visage entre ses genoux ;
\VS{43}Et il dit à son serviteur : Monte maintenant et regarde vers la mer. Le serviteur monta, il regarda et dit : Il n'y a rien. Elie dit par sept fois : Retournes-y.
\VS{44}A la septième fois, il dit : Voici un petit nuage qui s’élève de la mer et qui est comme la paume de la main d'un homme, laquelle monte de la mer. Elie dit : Monte et dis à Achab : Attelle ton char et descends de peur que la pluie ne t’arrête.
\VS{45}Ici et là, les cieux s'obscurcirent de nuages accompagnés de vent et il y eut une forte pluie. Achab monta sur son char et partit pour Jizreel.
\VS{46}Et la main de Yahweh fut sur Elie, qui se ceignit les reins et courut devant Achab, jusqu'à l'entrée de Jizreel.
\Chap{19}
\TextTitle{Fuite d’Elie devant les menaces de Jézabel}
\VerseOne{}Achab rapporta à Jézabel tout ce qu'Elie avait fait, et comment il avait tué par l'épée tous les prophètes.
\VS{2}Et Jézabel envoya un messager vers Elie, pour lui dire : Que les dieux me traitent dans toute leur rigueur, si demain, à cette heure-ci, je ne fais de ta vie ce que tu as fait de la vie de chacun d'eux !
\VS{3}Elie, voyant cela, se leva et s'en alla pour sauver sa vie. Il arriva à Beer-Schéba, qui appartient à Juda ; et il laissa là son serviteur.
\TextTitle{L’ange de Yahweh fortifie Elie}
\VS{4}Mais lui s'en alla dans le désert où, après une journée de marche, il s'assit sous un genêt et demanda la mort, en disant : C'en est assez, Ô Yahweh ! Prends mon âme, car je ne suis pas meilleur que mes pères.
\VS{5}Puis il se coucha et s'endormit sous un genêt. Voici un ange le toucha et lui dit : Lève-toi, mange.
\VS{6}Et il regarda, et voici à son chevet, un gâteau cuit sur des pierres chauffées et une cruche d'eau. Il mangea et but, puis se recoucha.
\VS{7}Et l'ange de Yahweh vint une seconde fois, le toucha et lui dit : Lève-toi, mange, car le chemin est trop long pour toi.
\TextTitle{Elie à Horeb, visitation et instructions de Yahweh}
\VS{8}Il se leva donc, mangea et but ; puis avec la force que lui donna cette nourriture, il marcha quarante jours et quarante nuits jusqu'à Horeb, la montagne de Dieu.
\VS{9}Et là, il entra dans une caverne et y passa la nuit. Et voici, la parole de Yahweh lui fut adressée en ces mots : Que fais-tu ici, Elie ?
\VS{10}Et il répondit : J'ai déployé mon zèle pour Yahweh, le Dieu des armées, parce que les enfants d'Israël ont abandonné ton alliance, ils ont renversé tes autels, ils ont tué tes prophètes par l'épée ; je suis resté, moi seul et ils me cherchent pour m'ôter la vie.
\VS{11}Yahweh lui dit : Sors et tiens-toi sur la montagne devant Yahweh. Et voici, Yahweh passa. Et devant Yahweh, il y eut un grand vent impétueux qui déchirait les montagnes et brisait les rochers, mais Yahweh n'était point dans ce vent. Après le vent, ce fut un tremblement de terre ; mais Yahweh n'était point dans ce tremblement de terre.
\VS{12}Après le tremblement de terre, un feu ; mais Yahweh n'était pas dans le feu. Et après le feu vint un murmure doux et léger.
\VS{13}Quand Elie l'entendit, il s’enveloppa le visage de son manteau, il sortit et se tint à l'entrée de la caverne. Et voici, une voix lui fit entendre ces paroles : Que fais-tu ici Elie ?
\VS{14}Et il répondit : J'ai déployé mon zèle pour Yahweh, le Dieu des armées, parce que les enfants d'Israël ont abandonné ton alliance, ils ont renversé tes autels, ils ont tué par l'épée tes prophètes ; je suis resté moi seul, et ils cherchent ma vie pour me l'ôter.
\VS{15}Yahweh lui dit : Va, retourne-t'en par ton chemin vers le désert de Damas ; et quand tu seras arrivé, tu oindras Hazaël pour roi de Syrie.
\VS{16}Tu oindras aussi Jéhu, fils de Nimschi, pour roi d’Israël ; et tu oindras Elisée, fils de Schaphath, d'Abel-Mehola, pour prophète à ta place.
\VS{17}Et il arrivera que quiconque échappera de l'épée de Hazaël, Jéhu le fera mourir ; et quiconque échappera de l'épée de Jéhu, Elisée le fera mourir.
\VS{18}Mais je me suis réservé sept mille hommes de reste en Israël, tous ceux qui n'ont point fléchi les genoux devant Baal, et dont la bouche ne l'a point baisé.
\TextTitle{Elisée devient disciple d’Elie}
\VS{19}Elie partit donc de là, et il trouva Elisée, fils de Schaphath, qui labourait. Il y avait douze paires de bœufs devant soi et il était avec la douzième. Quand Elie passa près de lui, il jeta sur lui son manteau.
\VS{20}Elisée laissa ses bœufs et courut après Elie, en disant : Je t’en prie, laisse-moi embrasser mon père et ma mère, et je te suivrai. Elie lui répondit : Va, et reviens ; car pense à ce que je t'ai fait.
\VS{21}Après s’être éloigné d’Elie, il revint prendre une paire de bœufs qu’il offrit en sacrifice ; et avec l'attelage des bœufs, il en fit bouillir la chair, et la donna au peuple ; ils mangèrent ; puis il se leva et suivit Elie. Dès lors, il fut à son service.
\Chap{20}
\TextTitle{Achab monte contre Ben-Hadad}
\VerseOne{}Alors Ben-Hadad, roi de Syrie rassembla toute son armée ; il avait avec lui trente-deux rois, des chevaux et des chars. Puis il monta, assiégea Samarie et il lui fit la guerre.
\VS{2}Il envoya des messagers à Achab, roi d'Israël, dans la ville ;
\VS{3}Et il lui fit dire : Ainsi parle Ben-Hadad : Ton argent et ton or sont à moi, tes femmes aussi et tes beaux enfants sont à moi.
\VS{4}Et le roi d'Israël répondit, et dit : Mon seigneur, je suis à toi, comme tu le dis, avec tout ce que j'ai.
\VS{5}Ensuite les messagers retournèrent, et dirent : Ainsi parle Ben-Hadad : Puisque je t'ai envoyé dire : Donne-moi ton argent et ton or, ta femme et tes enfants ;
\VS{6}A la même heure demain, j'enverrai chez toi mes serviteurs, ils fouilleront ta maison et les maisons de tes serviteurs, et se saisiront de tout ce que tu as de précieux, et ils l'emporteront.
\VS{7}Alors le roi d'Israël appela tous les anciens du pays, et il dit : Sachez et considérez, je vous prie, combien cet homme nous veut du mal ; car il m’a envoyé demander mes femmes, mes enfants, mon argent et mon or, et je ne lui avais rien refusé.
\VS{8}Et tous les anciens et tout le peuple lui dirent : Ne l'écoute point et ne consens pas.
\VS{9}Il répondit donc aux messagers de Ben-Hadad : Dites au roi, mon seigneur : Je ferai tout ce que tu as envoyé demander la première fois à ton serviteur, mais je ne pourrai faire ceci. Les messagers s'en allèrent et lui rapportèrent cette réponse.
\VS{10}Et Ben-Hadad envoya dire à Achab : Que les dieux me traitent dans toute leur rigueur, si la poudre de Samarie suffit pour remplir le creux de la main de tout le peuple qui me suit.
\VS{11}Mais le roi d'Israël répondit, et dit : Dites-lui : Que celui qui revêt une armure ne se glorifie point comme celui qui la dépose.
\VS{12}Lorsque Ben-Hadad entendit cette réponse, il était à boire avec les rois sous les tentes et il dit à ses serviteurs : Rangez-vous en bataille ! Et ils se rangèrent en bataille contre la ville.
\TextTitle{Victoire Achab}
\VS{13}Alors voici, un prophète s’approcha d’Achab, roi d'Israël et lui dit : Ainsi parle Yahweh : N'as-tu pas vu cette grande multitude ? Voilà, je m'en vais la livrer aujourd'hui entre tes mains, et tu sauras que je suis Yahweh.
\VS{14}Et Achab dit : Par qui ? Et il lui répondit : Ainsi parle Yahweh : Ce sera par les serviteurs des chefs des provinces. Et Achab dit : Qui engagera le combat ? Et il lui répondit : Toi.
\VS{15}Alors il passa en revue les serviteurs des chefs des provinces, qui furent deux cent trente-deux ; et après eux, il dénombra tout le peuple de tous les enfants d'Israël qui furent sept mille.
\VS{16}Ils firent une sortie en plein midi, lorsque Ben-Hadad buvait et s'enivrait dans les tentes, lui et les trente-deux rois qui étaient ses auxiliaires.
\VS{17}Les serviteurs des chefs des provinces sortirent les premiers et Ben-Hadad envoya quelques-uns qui le lui rapportèrent en disant : Des hommes sont sortis de Samarie.
\VS{18}Et il dit : Qu’ils soient sortis pour la paix, ou qu'ils soient sortis pour faire la guerre, saisissez-les tous vivants.
\VS{19}Les serviteurs des chefs de province sortirent de la ville puis l'armée qui était après eux.
\VS{20}Chacun d'eux frappa son homme, de sorte que les Syriens s'enfuirent et Israël les poursuivit. Ben-Hadad, roi de Syrie, se sauva sur un cheval, avec des cavaliers.
\VS{21}Et le roi d'Israël sortit et frappa les chevaux et les chars, en sorte qu'il fit éprouver une grande défaite aux Syriens.
\TextTitle{Achab monte de nouveau contre les Syriens}
\VS{22}Alors le prophète s’approcha du roi d'Israël, et lui dit : Va, fortifie-toi ; considère et vois ce que tu auras à faire ; car l’année révolue, le roi de Syrie montera contre toi.
\VS{23}Or, les serviteurs du roi de Syrie lui dirent : Leur dieu est un dieu de montagnes, c'est pourquoi ils ont été plus forts que nous. Mais combattons contre eux dans la plaine, et certainement, nous serons plus forts qu'eux.
\VS{24}Fais donc ceci : Ote chacun de ces rois de sa place, et remplace-les par des chefs ;
\VS{25}Puis lève une armée pareille à celle que tu as perdue, avec autant de chevaux et de chars, puis nous les combattrons dans la plaine et l’on verra si nous ne sommes pas plus forts qu'eux. Il les écouta, et fit ainsi.
\VS{26}L’année suivante, Ben-Hadad dénombra les Syriens et monta à Aphek pour combattre contre Israël.
\VS{27}On fit aussi le dénombrement des enfants d'Israël ; ils reçurent des vivres, et ils marchèrent à la rencontre des Syriens. Les enfants d'Israël campèrent vis-à-vis d'eux ; semblables à deux petits troupeaux de chèvres, tandis que les Syriens remplissaient le pays.
\VS{28}Alors l'homme de Dieu vint, et dit au roi d'Israël : Ainsi parle Yahweh : Parce que les Syriens ont dit : Yahweh est un dieu des montagnes et non un dieu des vallées, je livrerai entre tes mains toute cette grande multitude, et vous saurez que je suis Yahweh.
\VS{29}Sept jours durant ils campèrent vis-à-vis les uns des autres. Le septième jour, ils entrèrent en bataille, et les enfants d'Israël tuèrent en un seul jour cent mille hommes de pied des Syriens.
\VS{30}Le reste s'enfuit à la ville d'Aphek, où la muraille tomba sur vingt-sept mille hommes demeurés de reste. Ben-Hadad s'était réfugié dans la ville où il allait de chambre en chambre.
\TextTitle{Faute d'Achab qui épargne Ben-Hadad}
\VS{31}Ses serviteurs lui dirent : Voici maintenant, nous avons appris que les rois de la maison d'Israël sont des rois miséricordieux ; maintenant donc mettons des sacs sur nos reins et des cordes à nos têtes, sortons vers le roi d'Israël, peut-être qu'il te laissera la vie sauve.
\VS{32}Ils se mirent donc des sacs autour des reins et des cordes autour de leurs têtes. Ils allèrent auprès du roi d'Israël. Ils lui dirent : Ton serviteur Ben-Hadad dit : Laisse-moi la vie ! Achab répondit : Est-il encore vivant ? Il est mon frère.
\VS{33}Ces hommes tirèrent de là un bon augure, ils se hâtèrent de le prendre au mot et ils dirent : Ben-Hadad est-il ton frère ! Et il répondit : Allez, amenez-le. Ben-Hadad vint vers lui, et il le fit monter sur son char.
\VS{34}Et Ben-Hadad lui dit : Je te rendrai les villes que mon père avait prises à ton père ; et tu te feras des rues en Damas comme mon père avait fait en Samarie. Et moi, répondit Achab, je te laisserai aller en faisant alliance. Il traita donc alliance avec lui, et le laissa aller.
\VS{35}Alors un homme d'entre les fils des prophètes dit à son compagnon, sur l’ordre de Yahweh : Frappe-moi, je te prie ! Mais celui-là refusa de le frapper.
\VS{36}Et il lui dit : Parce que tu n'as point obéi à la parole de Yahweh, voilà, quand tu m’auras quitté, un lion te frappera. Quand il se fut séparé de lui, un lion survint et le frappa.
\VS{37}Puis il trouva un autre homme, et lui dit : Frappe-moi, je te prie. Cet homme-là le frappa et il le blessa.
\VS{38}Après cela le prophète s'en alla, et se plaça sur le chemin du roi ; il se déguisa avec un bandeau sur ses yeux.
\VS{39}Lorsque le roi passa, il cria vers lui, et dit : Ton serviteur était allé au milieu de la bataille ; et voici quelqu'un s'étant retiré, m'a amené un homme, en disant : Garde cet homme, s'il vient à s'échapper, ta vie en répondra, ou tu paieras un talent d'argent.
\VS{40}Et pendant que ton serviteur faisait quelques affaires çà et là, cet homme a disparu. Et le roi d'Israël lui répondit : Telle est ta condamnation, tu l’as toi-même prononcée.
\VS{41}Alors le prophète ôta promptement le bandeau de dessus ses yeux et le roi d'Israël reconnut que c'était l’un des prophètes.
\VS{42}Et il dit : Ainsi parle Yahweh : Parce que tu as laissé échapper de tes mains l'homme que j'avais dévoué par la voie de l'interdit, ta vie répondra de sa vie, et ton peuple de son peuple.
\VS{43}Mais le roi d'Israël se retira en sa maison, triste et irrité ; et il arriva en Samarie.
\Chap{21}
\TextTitle{Achab convoite la vigne de Naboth}
\VerseOne{}Après ces choses, voici ce qui arriva. Naboth de Jizreel, avait une vigne à Jizreel, près du palais d'Achab, roi de Samarie.
\VS{2}Achab parla à Naboth et lui dit : Cède-moi ta vigne, afin que j'en fasse un jardin potager, car elle est proche de ma maison et je te donnerai à la place une vigne meilleure ; ou, si cela te semble bon, je te paierai l'argent qu'elle vaut.
\VS{3}Mais Naboth répondit à Achab : Que Yahweh me garde de te donner l'héritage de mes pères !
\VS{4}Et Achab vint en sa maison tout triste et irrité, à cause de cette parole que lui avait dite Naboth de Jizreel, en disant : Je ne te donnerai point l'héritage de mes pères ! Il se coucha sur son lit, détourna son visage, et ne mangea rien.
\TextTitle{Manigance meurtrière de Jézabel}
\VS{5}Alors Jézabel, sa femme, vint auprès de lui, et lui dit : D'où vient que ton esprit est si triste ? Et pourquoi ne manges-tu point ?
\VS{6}Et il lui répondit : J’ai parlé à Naboth de Jizreel, et je lui ai dit : Donne-moi ta vigne pour de l'argent, ou si tu le désires, je te donnerai une autre vigne pour celle-là, mais il m'a dit : Je ne te céderai point ma vigne !
\VS{7}Alors Jézabel, sa femme, lui dit : Est-ce bien toi maintenant qui exerces la royauté sur Israël ? Lève-toi, prends un repas et que ton cœur se réjouisse ; je te ferai avoir la vigne de Naboth de Jizreel.
\VS{8}Et elle écrivit au nom d'Achab des lettres qu’elle scella du sceau du roi, et elle envoya aux anciens et magistrats qui habitaient avec Naboth, dans sa ville.
\VS{9}Voici ce qu’elle écrivit dans ces lettres : Publiez un jeûne et placez Naboth à la tête du peuple.
\VS{10}Mettez face à lui deux méchants hommes et qu'ils témoignent contre lui, en disant : Tu as maudit Dieu et le roi ! Puis vous le mènerez dehors et le lapiderez afin qu'il meure.
\VS{11}Les gens donc de la ville de Naboth, les anciens et les magistrats qui habitaient dans sa ville, agirent comme Jézabel le leur avait dit, et d’après ce qui était écrit dans les lettres qu'elle leur avait envoyées.
\VS{12}Ils publièrent un jeûne et ils placèrent Naboth à la tête du peuple.
\VS{13}Les deux méchants hommes vinrent et se mirent face à lui, et ces méchants hommes déclarèrent contre Naboth en la présence du peuple : Naboth a maudit Dieu et le roi ! Puis ils le menèrent hors de la ville, ils le lapidèrent, et il mourut.
\VS{14}Après cela, ils envoyèrent dire à Jézabel : Naboth a été lapidé, et il est mort.
\VS{15}Lorsque Jézabel apprit que Naboth avait été lapidé et qu'il était mort, elle dit à Achab : Lève-toi, mets-toi en possession de la vigne de Naboth de Jizreel, qu’il avait refusé de te donner pour de l'argent ; car Naboth n'est plus en vie, il est mort.
\VS{16}Ainsi dès qu'Achab eut entendu que Naboth était mort, il se leva pour descendre à la vigne de Jizreel et pour s'en mettre en possession.
\TextTitle{Jugement d’Achab et de Jézabel ; Achab s’humilie devant Dieu}
\VS{17}Alors la parole de Yahweh fut adressée à Elie, le Thischbite, en ces mots :
\VS{18}Lève-toi, descends au-devant d'Achab, roi d'Israël, lorsqu'il sera à Samarie. Le voilà dans la vigne de Naboth, où il est descendu pour en prendre possession.
\VS{19}Et tu lui diras : Ainsi parle Yahweh : N’es-tu pas un meurtrier et un voleur ? Puis tu lui diras : Ainsi parle Yahweh : Comme les chiens ont léché le sang de Naboth, les chiens lécheront aussi ton propre sang.
\VS{20}Et Achab dit à Elie : M'as-tu trouvé mon ennemi ? Mais il lui répondit : Oui, je t'ai trouvé, parce que tu t'es vendu pour faire ce qui est mal aux yeux de Yahweh.
\VS{21}Voici je vais faire venir le malheur sur toi, et je te consumerai, j’exterminerai quiconque appartient à Achab, tant celui qui est esclave, que celui qui est libre en Israël.
\VS{22}Je rendrai ta maison semblable à la maison de Jéroboam, fils de Nebath, et la maison de Baescha, fils d'Achija, parce que tu m'as irrité et fait pécher Israël.
\VS{23}Yahweh parla aussi contre Jézabel en disant : Les chiens mangeront Jézabel près du rempart de Jizreel.
\VS{24}Celui de la maison d’Achab qui mourra dans la ville, les chiens le mangeront, et celui qui mourra aux champs, les oiseaux des cieux le mangeront.
\VS{25}En effet, il n'y en avait point eu de personne comme Achab, qui se soit vendu pour faire ce qui est mal aux yeux de Yahweh, et sa femme Jézabel l’y excitait ;
\VS{26}de sorte qu'il se rendit fort abominable, allant après les idoles, comme l'avaient fait les Amoréens, que Yahweh avait chassés de devant les enfants d'Israël.
\VS{27}Après avoir entendu les paroles d’Elie, Achab déchira ses vêtements, il mit un sac sur son corps, et jeûna. Il se tenait couché avec ce sac, et il marchait lentement.
\VS{28}Et la parole de Yahweh fut adressée à Elie, le Thischbite, en disant :
\VS{29}As-tu vu comment Achab s'est humilié devant moi ? Parce qu'il s'est humilié devant moi, je ne ferai pas venir le malheur pendant sa vie, ce sera aux jours de son fils que je ferai venir le malheur sur sa maison.
\Chap{22}
\TextTitle{Josaphat aide Achab contre les Syriens}
\VerseOne{}Et on resta trois ans sans qu'il y eût guerre entre la Syrie et Israël.
\VS{2}Puis il arriva, dans la troisième année, que Josaphat, roi de Juda, descendit vers le roi d'Israël.
\VS{3}Le roi d'Israël dit à ses serviteurs : Ne savez-vous pas que Ramoth de Galaad nous appartient ? Et nous ne nous inquiétons pas de la reprendre des mains du roi de Syrie !
\VS{4}Puis il dit à Josaphat : Viendras-tu avec moi à la guerre contre Ramoth de Galaad ? Et Josaphat répondit au roi d'Israël : Nous irons, moi comme toi, mon peuple comme ton peuple, et mes chevaux comme tes chevaux.
\TextTitle{Les prophètes de mensonge\FTNTT{2 Ch. 18:4-5, 9-11}}
\VS{5}Josaphat dit encore au roi d'Israël : Consulte aujourd'hui, je te prie, la parole de Yahweh.
\VS{6}Et le roi d'Israël assembla les prophètes, au nombre de quatre cents environ, auxquels il dit : Irai-je à la guerre contre Ramoth de Galaad, ou dois-je y renoncer ? Et ils répondirent : Monte, car le Seigneur la livrera entre les mains du roi.
\VS{7}Mais Josaphat dit : N'y a-t-il point ici encore quelque prophète de Yahweh, afin que nous le consultions ?
\VS{8}Et le roi d'Israël dit à Josaphat : Il y a encore un homme par qui l’on puisse consulter Yahweh, mais je le hais, car il ne prophétise rien de bon, mais seulement du mal, c'est Michée, fils de Jimla. Josaphat dit : Que le roi ne parle point ainsi !
\VS{9}Alors le roi d'Israël appela un eunuque auquel il dit : Fais venir promptement Michée, fils de Jimla.
\VS{10}Or, le roi d'Israël et Josaphat, roi de Juda, étaient assis chacun sur son trône, revêtus de leurs habits, dans la place, vers l'entrée de la porte de Samarie ; et tous les prophètes prophétisaient en leur présence.
\VS{11}Sédécias, fils de Kenaana, s'était fait des cornes de fer et il dit : Ainsi parle Yahweh : De ces cornes-ci tu heurteras les Syriens, jusqu'à les détruire.
\VS{12}Et tous les prophètes prophétisaient de même, en disant : Monte à Ramoth de Galaad et tu réussiras ; et Yahweh la livrera entre les mains du roi.
\TextTitle{Michée annonce la défaite et la mort d'Achab\FTNTT{2 Ch. 18:6-8, 12-27, 28-34}}
\VS{13}Le messager qui était allé appeler Michée, lui parla ainsi : Voici, les prophètes parlent d'un commun accord au sujet du roi ; je te prie que ta parole soit semblable à celle de chacun d’eux ! Annonce du bien !
\VS{14}Mais Michée lui répondit : Yahweh est vivant ! J’annoncerai ce que Yahweh me dira.
\VS{15}Il vint donc vers le roi, et le roi lui dit : Michée, irons-nous à la guerre contre Ramoth de Galaad, ou devons-nous y renoncer ? Et il lui dit : Monte et tu réussiras, et Yahweh la livrera entre les mains du roi.
\VS{16}Et le roi lui dit : Jusqu'à combien de fois te conjurerai-je de ne me dire que la vérité au nom de Yahweh ?
\VS{17}Et il répondit : J'ai vu tout Israël dispersé par les montagnes, comme un troupeau de brebis qui n'a point de berger ; et Yahweh a dit : Ces gens n’ont point de maître, que chacun retourne en paix dans sa maison !
\VS{18}Alors le roi d'Israël dit à Josaphat : Ne t'ai-je pas bien dit que quand il est question de moi il ne prophétise rien de bon, mais seulement du mal ?
\VS{19}Et Michée lui dit : Ecoute néanmoins la parole de Yahweh ! J'ai vu Yahweh assis sur son trône, et toute l'armée des cieux se tenant devant lui, à sa droite et à sa gauche.
\VS{20}Et Yahweh a dit : Quel est celui qui séduira Achab, afin qu'il monte et qu'il périsse en Ramoth de Galaad ? Et ils répondaient, l'un parlait d'une manière et l'autre d'une autre.
\VS{21}Alors un esprit s'avança et se tint devant Yahweh, il déclara : Je le séduirai. Et Yahweh lui dit : Comment ?
\VS{22}Et il répondit : Je sortirai et je serai un esprit de mensonge dans la bouche de tous ses prophètes. Et Yahweh dit : Tu le séduiras et même tu en viendras à bout ; sors et fais ainsi !
\VS{23}Et maintenant, voici, Yahweh a mis un esprit de mensonge dans la bouche de tous tes prophètes que voilà et Yahweh a prononcé du mal contre toi.
\VS{24}Alors Sédécias, fils de Kenaana, s'approcha et frappa Michée sur la joue et dit : Par où l'Esprit de Yahweh est-il sorti de moi pour s'adresser à toi ?
\VS{25}Et Michée répondit : Voici, tu le verras le jour où tu iras de chambre en chambre pour te cacher.
\VS{26}Alors le roi d'Israël dit : Qu'on prenne Michée et qu'on le mène vers Amon, capitaine de la ville et vers Joas, le fils du roi.
\VS{27}Et tu diras : Ainsi a parlé le roi : Mettez cet homme en prison, nourrissez-le de pain et de l’eau d’affliction, jusqu'à ce que je revienne en paix.
\VS{28}Et Michée répondit : Si tu reviens en paix, Yahweh n'a point parlé par moi. Il dit aussi : Vous tous, peuples, entendez !
\VS{29}Le roi d'Israël monta avec Josaphat, roi de Juda, contre Ramoth de Galaad.
\VS{30}Et le roi d'Israël dit à Josaphat : Que je me déguise et que j'aille à la bataille ; mais toi, revêts-toi de tes habits. Le roi d'Israël donc se déguisa et alla au combat.
\VS{31}Or, le roi de Syrie avait donné un ordre aux trente-deux chefs de ses chars, en disant : Vous n’attaquerez ni petits ni grands, mais seulement contre le roi d'Israël.
\VS{32}Quand les chefs des chars aperçurent Josaphat, ils dirent : C'est certainement le roi d'Israël. Et ils s’approchèrent de lui pour le combattre, mais Josaphat s'écria.
\VS{33}Et quand les chefs des chars virent que ce n'était pas le roi d'Israël, ils se détournèrent de lui.
\TextTitle{Mort d’Achab}
\VS{34}Alors un homme tira de son arc au hasard, et frappa le roi d'Israël entre les jointures de la cuirasse. Et le roi dit à son conducteur de char : Tourne et fais-moi sortir du champ de bataille, car je suis blessé.
\VS{35}Or, le combat devint acharné ce jour-là. Le roi d'Israël fut arrêté dans son char en face des Syriens et il mourut sur le soir. Le sang de sa blessure coulait à l’intérieur du char.
\VS{36}Au coucher du soleil, on cria par tout le camp, en disant : Que chacun se retire en sa ville et chacun en son pays !
\VS{37}Ainsi mourut le roi, qui fut ramené à Samarie ; et l’on enterra le roi à Samarie.
\VS{38}Lorsqu’on lava le char à l’étang de Samarie, les chiens léchèrent le sang d’Achab, et les prostituées s’y baignèrent, selon la parole que Yahweh avait prononcée.
\VS{39}Le reste des actions d'Achab, tout ce qu’il a fait, la maison d'ivoire qu'il construisit et toutes les villes qu'il a bâties, toutes ces choses ne sont-elles pas écrites au livre des Chroniques des rois d'Israël ?
\TextTitle{Règne de Josaphat sur Juda\FTNTT{2 Ch. 17:19-20}}
\VS{40}Ainsi Achab se coucha avec ses pères. Et Achazia, son fils, régna à sa place.
\VS{41}Josaphat, fils d'Asa, régna sur Juda, la quatrième année d'Achab, roi d'Israël.
\VS{42}Josaphat avait trente-cinq ans lorsqu’il devint roi, et il régna vingt-cinq ans à Jérusalem. Sa mère s’appelait Azuba, fille de Schilchi.
\VS{43}Il suivit entièrement la voie d'Asa, son père, et ne s'en détourna point, faisant tout ce qui est droit aux yeux de Yahweh.
\VS{44}Toutefois les hauts lieux ne disparurent pas ; le peuple offrait encore des sacrifices et offrait encore des parfums sur les hauts lieux.
\VS{45}Josaphat fit aussi la paix avec le roi d'Israël.
\VS{46}Le reste des actions de Josaphat, ses exploits et les guerres qu'il mena ne sont-elles pas écrites au livre des Chroniques des rois de Juda ?
\VS{47}Il extermina du pays le reste des prostitués, qui étaient demeurés là depuis le temps d'Asa, son père.
\VS{48}Il n'y avait point alors de roi en Edom : c’était un intendant qui gouvernait.
\VS{49}Josaphat construisit des navires de Tarsis pour aller chercher de l'or à Ophir ; mais il n'y alla point, parce que les navires se brisèrent à Etsjon-Guéber.
\VS{50}Alors Achazia, fils d'Achab, dit à Josaphat : Que mes serviteurs aillent sur les navires avec les tiens, mais Josaphat ne le voulut point.
\TextTitle{Joram règne sur Juda\FTNTT{2 Ch. 21:1}}
\VS{51}Et Josaphat s’endormit avec ses pères et fut enterré avec eux en la cité de David, son père. Et Joram, son fils, régna à sa place.
\TextTitle{Achazia règne sur Israël}
\VS{52}Achazia, fils d'Achab, régna sur Israël à Samarie, la dix-septième année de Josaphat, roi de Juda. Et il régna deux ans sur Israël.
\VS{53}Il fit ce qui est mal aux yeux de Yahweh : il marcha dans la voie de son père, de sa mère et celle de Jéroboam, fils de Nebath, qui avait fait pécher Israël.
\VS{54}Il servit Baal, il se prosterna devant lui et il irrita Yahweh, le Dieu d'Israël, comme l’avait fait son père.
\PPE{}
\end{multicols}

%\clearpage\ShortTitle{2 R.}\BookTitle{2 Rois}\BFont
\noindent\hrulefill
{\footnotesize
\textit{
\bigskip
{\centering{}
\\Auteur~: Inconnu
\\(Heb.~: Melakhim)
\\Signification~: Roi, Règne
\\Thème~: Suite de l'histoire d'Israël et de Juda
\\Date de rédaction~: 6\up{ème} siècle av. J.-C.\\}
}
\textit{
\\Le second livre des rois s'articule autour de la vie d'Elisée, serviteur d'Elie, devenu dorénavant son successeur. On y découvre le service prophétique au travers duquel Dieu se révéla comme le Tout-Puissant, le Dieu compatissant, le Maître des temps et des circonstances, le Libérateur, le Dieu de la résurrection, le Puissant Guerrier et aussi le Juge.
\\Ce livre relate l'histoire des derniers rois, la chute d'Israël et sa captivité, la destruction de Jérusalem par Nebudcanetsar, roi de Babylone, en 586 av. J.-C., et la captivité de Juda.\bigskip
}
}
\par\nobreak\noindent\hrulefill
\begin{multicols}{2}
\Chap{1}
\TextTitle{Jugement de Yahweh sur Achazia, roi d'Israël}
\VerseOne{}Or après la mort d'Achab, Moab se révolta contre Israël.
\VS{2}Or Achazia tomba par le treillis de sa chambre haute qui était à Samarie, et il en fut malade. Il envoya des messagers et leur dit~: Allez, consultez Baal-Zebub\FTNT{Baal-Zebub était une divinité des Philistins adorée à Ekron qui se nommait aussi Béelzébul (Mt. 10:25).}, dieu d'Ekron, pour savoir si je guérirai de cette maladie.
\VS{3}Mais l'Ange de Yahweh dit à Elie\FTNT{Elie~: Voir 1 R. 17.}, le Thischbite~: Lève-toi, monte à la rencontre des messagers du roi de Samarie, et dis-leur~: N'y a-t-il point de Dieu en Israël pour que vous alliez consulter Baal-Zebub, dieu d'Ekron~?
\VS{4}C'est pourquoi ainsi parle Yahweh~: Tu ne descendras pas du lit sur lequel tu es monté, mais tu mourras, tu mourras\FTNT{Voir en Gn. 2:16.}. Et Elie s'en alla.
\VS{5}Les messagers retournèrent vers Achazia. Et il leur dit~: Pourquoi revenez-vous~?
\VS{6}Ils lui répondirent~: Un homme est monté à notre rencontre et nous a dit~: Allez, retournez vers le roi qui vous a envoyés et dites-lui~: Ainsi parle Yahweh~: N'y a-t-il point de Dieu en Israël, pour que tu envoies consulter Baal-Zebub, dieu d'Ekron~? A cause de cela, tu ne descendras pas du lit sur lequel tu es monté, mais certainement tu mourras.
\VS{7}Achazia leur dit~: Comment était cet homme qui est monté à votre rencontre et qui vous a dit ces paroles~?
\VS{8}Ils lui répondirent~: C'était un homme vêtu de poil, ayant une ceinture de cuir, ceinte sur ses reins. Et Achazia dit~: C'est Elie, le Thischbite.
\TextTitle{Affirmation de l'autorité d'Elie}
\VS{9}Alors il envoya vers lui un chef de cinquante avec ses cinquante hommes. Ce chef monta auprès d'Elie, qui demeurait au sommet d'une montagne, et il lui dit~: Homme de Dieu, le roi a dit~: Descends~!
\VS{10}Mais Elie répondit et dit au chef de cinquante~: Si je suis un homme de Dieu, que le feu descende du ciel et te consume, toi et tes cinquante hommes~! Et le feu descendit du ciel et le consuma, lui et ses cinquante hommes.
\VS{11}Achazia envoya encore un autre chef de cinquante avec ses cinquante hommes. Ce chef prit la parole et dit à Elie~: Homme de Dieu, ainsi parle le roi~: Hâte-toi de descendre~!
\VS{12}Mais Elie répondit, et leur dit~: Si je suis un homme de Dieu, que le feu descende du ciel et te consume, toi et tes cinquante hommes~! Et le feu de Dieu descendit du ciel et le consuma, lui et ses cinquante hommes.
\VS{13}Achazia envoya encore un troisième chef de cinquante avec ses cinquante hommes. Ce troisième chef de cinquante hommes monta, et vint se mettre à genoux devant Elie, le suppliant, en disant~: Homme de Dieu, je te prie, que ma vie et la vie de ces cinquante hommes, tes serviteurs, soit précieuse à tes yeux~!
\VS{14}Voici, le feu est descendu du ciel et a consumé les deux premiers chefs de cinquante, avec leurs cinquante hommes~; mais maintenant, je te prie, que ma vie soit précieuse à tes yeux~!
\VS{15}Et l'Ange de Yahweh dit à Elie~: Descends avec lui, n'aie pas peur de lui. Elie se leva donc et descendit avec lui vers le roi.
\VS{16}Il lui dit~: Ainsi parle Yahweh~: Parce que tu as envoyé des messagers pour consulter Baal-Zebub, dieu d'Ekron, comme s'il n'y avait point de Dieu en Israël, pour consulter sa parole, tu ne descendras pas du lit sur lequel tu es monté, mais certainement tu mourras.
\TextTitle{Mort d'Achazia~; Joram règne sur Israël}
\VS{17}Achazia mourut, selon la parole de Yahweh prononcée par Elie. Et Joram régna à sa place, la seconde année de Joram, fils de Josaphat, roi de Juda, parce qu'Achazia n'avait point de fils.
\VS{18}Le reste des actions d'Achazia et ce qu'il a fait, cela n'est-il pas écrit dans le livre des Chroniques des rois d'Israël~?
\Chap{2}
\TextTitle{Enlèvement d'Elie au ciel}
\VerseOne{}Or il arriva lorsque Yahweh enleva Elie au ciel dans un tourbillon, Elie et Elisée partaient de Guilgal.
\VS{2}Elie dit à Elisée~: Je te prie, reste ici, car Yahweh m'envoie jusqu'à Béthel. Mais Elisée répondit Yahweh est vivant et ton âme est vivante~! Je ne te quitterai pas~! Ainsi ils descendirent à Béthel.
\VS{3}Les fils des prophètes qui étaient à Béthel sortirent vers Elisée, et lui dirent~: Ne sais-tu pas qu'aujourd'hui Yahweh va enlever ton maître au-dessus de ta tête~? Et il répondit~: Je le sais aussi~; taisez-vous~!
\VS{4}Elie lui dit~: Elisée, je te prie, reste ici, car Yahweh m'envoie à Jéricho. Mais Elisée lui répondit~: Yahweh est vivant et ton âme est vivante~! Je ne te quitterai pas~! Ainsi, ils arrivèrent à Jéricho.
\VS{5}Les fils des prophètes qui étaient à Jéricho s'approchèrent d'Elisée, et lui dirent~: Ne sais-tu pas qu'aujourd'hui Yahweh va enlever ton maître au-dessus de ta tête~? Et il répondit~: Je le sais aussi~; taisez-vous~!
\VS{6}Elie lui dit~: Elisée, je te prie demeure ici, car Yahweh m'envoie jusqu'au Jourdain. Mais Elisée répondit~: Yahweh est vivant et ton âme est vivante~! Je ne te quitterai pas~! Ainsi, ils s'en allèrent tous les deux.
\VS{7}Cinquante hommes d'entre les fils des prophètes arrivèrent et s'arrêtèrent à distance vis-à-vis d'eux, et eux deux s'arrêtèrent au bord du Jourdain.
\VS{8}Alors Elie prit son manteau, le roula et en frappa les eaux, qui se divisèrent çà et là, et ils passèrent tous deux à sec.
\VS{9}Quand ils furent passés, Elie dit à Elisée~: Demande ce que tu veux que je fasse pour toi, avant que je sois enlevé d'avec toi. Elisée répondit~: Je te prie, que j'aie, une double portion\FTNT{Le fils aîné recevait une double portion par rapport aux autres fils (De. 21:15-17).} de ton esprit~!
\VS{10}Elie lui dit~: Tu demandes une chose difficile. Mais si tu me vois pendant que je serai enlevé d'avec toi, cela te sera accordé~; mais si tu ne me vois pas, cela ne te sera pas accordé.
\VS{11}Comme ils continuaient à marcher en parlant, voici, un char de feu et des chevaux de feu les séparèrent l'un de l'autre, et Elie monta au ciel dans un tourbillon.
\TextTitle{La double portion de l'esprit d'Elie sur Elisée}
\VS{12}Elisée le regardait et criait~: Mon père~! Mon père~! Char d'Israël et sa cavalerie~! Et il ne le vit plus. Puis saisissant ses vêtements, il les déchira en deux morceaux.
\VS{13}Il releva le manteau qu'Elie avait laissé tomber. Puis il retourna et s'arrêta sur le bord du Jourdain.
\VS{14}Ensuite il prit le manteau qu'Elie avait laissé tomber et il en frappa les eaux, et dit~: Où est Yahweh, le Dieu d'Elie, Yahweh lui-même~? Lui aussi frappa les eaux qui se divisèrent en deux~; et Elisée passa.
\TextTitle{Le service d'Elisée est reconnu par les hommes}
\VS{15}Quand les fils des prophètes qui étaient à Jéricho, vis-à-vis, l'eurent vu, ils dirent~: L'esprit d'Elie repose sur Elisée~! Ils vinrent à sa rencontre et se prosternèrent contre terre devant lui.
\VS{16}Ils lui dirent~: Voici, il y a parmi tes serviteurs cinquante hommes vaillants~; veux-tu qu'ils aillent chercher ton maître, de peur que l'Esprit de Yahweh ne l'ait enlevé et ne l'ait jeté sur quelque montagne ou dans quelque vallée~? Elisée répondit~: Ne les envoyez pas.
\VS{17}Mais ils le pressèrent tant par leurs paroles, qu'il en était embarrassé. Il leur dit donc~: Envoyez-les. Ils envoyèrent cinquante hommes, qui pendant trois jours cherchèrent Elie, mais ils ne le trouvèrent point.
\VS{18}Puis ils retournèrent vers Elisée, qui était à Jéricho, et il leur dit~: Ne vous avais-je pas dit~: N'y allez pas~?
\VS{19}Les gens de la ville dirent à Elisée~: Voici, le séjour dans cette ville est bon, comme mon seigneur le voit~; mais les eaux sont mauvaises et le pays est stérile.
\VS{20}Il dit~: Apportez-moi un vase neuf et mettez-y du sel. Et ils le lui apportèrent.
\VS{21}Puis il alla vers la source des eaux, et il y jeta le sel, et dit~: Ainsi parle Yahweh~: J'assainis ces eaux~; elles ne causeront plus ni mort ni stérilité.
\VS{22}Les eaux furent assainies, jusqu'à ce jour, selon la parole qu'Elisée avait prononcée.
\TextTitle{Jugement des moqueurs}
\VS{23}Elisée monta de là à Béthel~; et comme il montait par le chemin, des petits garçons sortirent de la ville et se moquèrent de lui. Ils lui disaient~: Monte chauve~! Monte chauve~!
\VS{24}Il se retourna pour les regarder, et il les maudit au nom de Yahweh. Alors deux ours sortirent de la forêt et déchirèrent quarante-deux de ces enfants.
\VS{25}De là il alla sur la montagne de Carmel, d'où il retourna à Samarie.
\Chap{3}
\TextTitle{Joram règne sur Israël}
\VerseOne{}La dix-huitième année de Josaphat, roi de Juda, Joram, fils d'Achab, régna sur Israël à Samarie. Il régna douze ans.
\VS{2}Il fit ce qui est mal aux yeux de Yahweh, non pas toutefois comme son père et sa mère, car il ôta la statue de Baal que son père avait faite~;
\VS{3}mais il s'attacha aux péchés de Jéroboam, fils de Nebath, qui avait fait pécher Israël et il ne s'en détourna point.
\TextTitle{Rébellion de Moab~; Israël et Juda s'allient pour combattre}
\VS{4}Or Méscha, roi de Moab, possédait des troupeaux, et il payait au roi d'Israël un tribut de cent mille agneaux et cent mille béliers avec leur laine.
\VS{5}Mais aussitôt qu'Achab mourut, le roi de Moab se révolta contre le roi d'Israël.
\VS{6}C'est pourquoi le roi Joram sortit ce jour-là de Samarie et passa en revue tout Israël.
\VS{7}Il se mit en marche et fit dire à Josaphat, roi de Juda~: Le roi de Moab s'est rebellé contre moi~; veux-tu venir avec moi faire la guerre à Moab~? Josaphat répondit~: Je monterai, moi comme toi, mon peuple comme ton peuple, mes chevaux comme tes chevaux.
\VS{8}Ensuite il dit~: Par quel chemin monterons-nous~? Joram répondit~: Par le chemin du désert d'Edom.
\TextTitle{Les rois d'Israël, de Juda et d'Edom en marche~; ils consultent Elisée}
\VS{9}Ainsi, le roi d'Israël, le roi de Juda et le roi d'Edom, partirent~; ils firent un détour, et après une marche de sept jours, ils manquèrent d'eau pour l'armée et pour les bêtes qui la suivaient.
\VS{10}Alors le roi d'Israël dit~: Hélas~! Yahweh a appelé ces trois rois pour les livrer entre les mains de Moab.
\VS{11}Et Josaphat dit~: N'y a-t-il ici aucun prophète de Yahweh, par qui nous puissions consulter Yahweh~? Et un des serviteurs du roi d'Israël répondit, et dit~: Il y a ici Elisée, fils de Schaphath, qui versait de l'eau sur les mains d'Elie.
\VS{12}Alors Josaphat dit~: La parole de Yahweh est avec lui. Le roi d'Israël, Josaphat et le roi d'Edom descendirent vers lui.
\VS{13}Mais Elisée dit au roi d'Israël~: Qu'y a-t-il entre moi et toi~? Va-t'en vers les prophètes de ton père et vers les prophètes de ta mère. Et le roi d'Israël lui répondit~: Non~! Car Yahweh a appelé ces trois rois pour les livrer entre les mains de Moab.
\VS{14}Elisée dit~: Yahweh des armées, devant lequel je me tiens, est vivant~! Si je n'avais de la considération pour Josaphat, roi de Juda, je ne ferais aucune attention à toi et je ne te regarderais même pas.
\VS{15}Mais maintenant, amenez-moi un joueur d'instruments à cordes. Et comme le joueur jouait des instruments à cordes, la main de Yahweh fut sur Elisée.
\TextTitle{Prophétie sur la défaite de Moab}
\VS{16}Et il dit~: Ainsi parle Yahweh~: Faites des tranchées dans toute cette vallée.
\VS{17}Car ainsi parle Yahweh~: Vous ne verrez ni vent, ni pluie, et néanmoins cette vallée sera remplie d'eaux, et vous boirez, vous et vos bêtes.
\VS{18}Mais cela est peu de chose aux yeux de Yahweh. Il livrera Moab entre vos mains~;
\VS{19}Vous frapperez toutes les villes fortes et toutes les villes d'élite, vous abattrez tous les bons arbres, vous boucherez toutes les sources d'eau et vous ruinerez avec des pierres tous les meilleurs champs.
\VS{20}Il arriva donc au matin, environ à l'heure de l'offrande, que l'eau arriva du chemin d'Edom, en sorte que ce pays fut rempli d'eau.
\VS{21}Cependant, tous les Moabites ayant appris que ces rois étaient montés pour leur faire la guerre s'étaient assemblés. On convoqua tous ceux qui étaient en âge de porter les armes, et même au-dessus, et ils se tinrent sur la frontière.
\VS{22}Et le lendemain, ils se levèrent de bon matin, et comme le soleil se levait sur les eaux, les Moabites virent en face d'eux les eaux rouges comme du sang.
\VS{23}Ils dirent~: C'est du sang~! Certainement, ces rois-là se sont entretués, et chacun a frappé son compagnon~; maintenant, Moabites, au butin~!
\VS{24}Ainsi ils marchèrent contre le camp d'Israël. Mais Israël se leva et frappa Moab, qui prit la fuite devant eux. Puis ils pénétrèrent dans le pays et frappèrent Moab.
\VS{25}Ils détruisirent les villes, et chacun jetait des pierres dans les meilleurs champs, de sorte qu'ils les en remplirent, ils bouchèrent toutes les sources d'eaux et abattirent tous les bons arbres~; et les frondeurs entourèrent et frappèrent Kir-Haréseth, dont on ne laissa que les pierres.
\VS{26}Le roi de Moab, voyant qu'il n'était pas le plus fort dans la bataille, prit avec lui sept cents hommes tirant l'épée pour se frayer un passage jusqu'au roi d'Edom~; mais ils ne purent pas.
\VS{27}Alors il prit son fils premier-né, qui devait régner à sa place, et l'offrit en holocauste sur la muraille. Et il y eut une grande indignation en Israël~; ainsi ils se retirèrent du roi de Moab et retournèrent dans leur pays.
\Chap{4}
\TextTitle{Miracle~: Le vase d'huile de la veuve}
\VerseOne{}Or une femme d'un des fils des prophètes cria à Elisée, en disant~: Ton serviteur mon mari est mort, et tu sais que ton serviteur craignait Yahweh~; or son créancier est venu pour prendre mes deux enfants, afin qu'ils soient ses esclaves.
\VS{2}Elisée lui répondit~: Que puis-je faire pour toi~? Dis-moi ce que tu as à la maison. Et elle dit~: Ta servante n'a rien dans toute la maison qu'un vase d'huile.
\VS{3}Alors il lui dit~: Va, demande des vases dans la rue à tous tes voisins, des vases vides, et n'en demande pas un petit nombre.
\VS{4}Puis rentre et ferme la porte sur toi et sur tes enfants, et verse dans tous ces vases, et tu mettras de côté ceux qui seront pleins.
\VS{5}Alors elle le quitta. Ayant fermé la porte sur elle et sur ses enfants~; ils lui présentaient les vases, et elle versait.
\VS{6}Lorsqu'elle eut rempli les vases, elle dit à son fils~: Présente-moi encore un vase. Mais il répondit~: Il n'y a plus de vase. Et l'huile s'arrêta.
\VS{7}Elle alla le raconter à l'homme de Dieu, qui lui dit~: Va, vends l'huile, et paye ta dette~; et vous vivrez, toi et tes fils, de ce qui restera.
\TextTitle{Yahweh se souvient de la Sunamite}
\VS{8}Et il arriva un jour qu'Elisée passait par Sunem, où il y avait une femme importante~; elle le retint avec grande instance à manger du pain chez elle. Et toutes les fois qu'il passait, il s'y retirait pour manger du pain.
\VS{9}Elle dit à son mari~: Voilà, je sais que cet homme qui passe souvent chez nous est un saint homme de Dieu.
\VS{10}Faisons-lui, je te prie, une petite chambre haute avec des murs, et mettons-y pour lui un lit, une table, un siège et un chandelier, afin que quand il viendra chez nous, il s'y retire.
\VS{11}Un jour, Elisée étant revenu à Sunem, il se retira dans cette chambre haute et s'y coucha.
\VS{12}Puis il dit à Guéhazi, son serviteur~: Appelle cette Sunamite. Guéhazi l'appela, et elle se présenta devant lui.
\VS{13}Et Elisée dit à Guéhazi~: Dis maintenant à cette femme~: Voici, tu nous as montré tout cet empressement~; que pourrait-on faire pour toi~? Faut-il parler pour toi au roi ou au chef de l'armée~? Elle répondit~: J'habite au milieu de mon peuple.
\VS{14}Et il dit~: Que faudrait-il faire pour elle~? Guéhazi répondit~: Mais elle n'a point de fils et son mari est vieux.
\VS{15}Et il dit~: Appelle-la. Guéhazi l'appela, et elle se présenta à la porte.
\VS{16}Elisée lui dit~: L'année prochaine, à cette même époque, tu embrasseras un fils. Elle répondit~: Mon seigneur, homme de Dieu, ne trompe pas, ne trompe pas ta servante~!
\VS{17}Cette femme devint enceinte et enfanta un fils un an après, à la même époque, comme Elisée lui avait dit.
\TextTitle{Foi de la Sunamite, résurrection de son fils}
\VS{18}L'enfant grandit. Il sortit un jour pour aller trouver son père vers les moissonneurs.
\VS{19}Et il dit à son père~: Ma tête~! Ma tête~! Et le père dit au serviteur~: Porte-le à sa mère.
\VS{20}Il le porta donc et l'amena à sa mère. Et l'enfant resta sur les genoux de sa mère jusqu'à midi, puis il mourut.
\VS{21}Elle monta et le coucha sur le lit de l'homme de Dieu~; et ayant fermé la porte sur lui, elle sortit.
\VS{22}Elle appela son mari et dit~: Je te prie envoie-moi un des serviteurs et une ânesse~; j'irai chez l'homme de Dieu et je reviendrai.
\VS{23}Et il dit~: Pourquoi vas-tu vers lui aujourd'hui~? Ce n'est point la nouvelle lune ni le sabbat. Elle répondit~: Tout va bien~!
\VS{24}Elle fit donc seller l'ânesse, et dit à son serviteur~: Conduis-moi et ne m'arrête pas en route sans que je te le dise.
\VS{25}Ainsi elle s'en alla et se rendit vers l'homme de Dieu sur la montagne de Carmel. L'homme de Dieu, l'ayant aperçue, dit à Guéhazi son serviteur~: Voilà la Sunamite~!
\VS{26}Va, cours à sa rencontre et dis-lui~: Te portes-tu bien~? Ton mari se porte-t-il bien~? L'enfant se porte-t-il bien~? Et elle répondit~: Nous nous portons bien.
\VS{27}Dès qu'elle fut arrivée auprès de l'homme de Dieu sur la montagne, elle embrassa ses pieds. Guéhazi s'approcha pour la repousser, mais l'homme de Dieu lui dit~: Laisse-la, car son âme est dans l'amertume, et Yahweh me l'a caché, et ne me l'a pas révélé.
\VS{28}Alors elle dit~: Ai-je demandé un fils à mon seigneur~? N'ai-je pas dit~: Ne me trompe pas~?
\VS{29}Et Elisée dit à Guéhazi~: Ceins tes reins, prends mon bâton dans ta main, et pars. Si tu rencontres quelqu'un, ne le salue pas~; et si quelqu'un te salue, ne lui réponds pas. Tu mettras mon bâton sur le visage de l'enfant.
\VS{30}Mais la mère de l'enfant dit~: Yahweh est vivant, et ton âme est vivante~! Je ne te quitterai point. Il se leva donc et la suivit.
\VS{31}Or Guéhazi les avait devancés et il avait mis le bâton sur le visage de l'enfant~; mais il n'y eut ni voix ni signe d'attention. Guéhazi retourna à la rencontre d'Elisée et l'en informa en disant~: L'enfant ne s'est pas réveillé.
\VS{32}Lorsqu'Elisée entra dans la maison, l'enfant, mort, était couché sur son lit.
\VS{33}Il ferma la porte sur eux deux et pria Yahweh.
\VS{34}Puis, il monta et se coucha sur l'enfant~; il mit sa bouche sur la bouche de l'enfant, ses yeux sur ses yeux, ses mains sur ses mains, et il s'étendit sur lui. La chair de l'enfant se réchauffa.
\VS{35}Puis il s'éloigna et marcha dans la maison, tantôt dans un lieu, tantôt dans un autre, et il remonta et s'étendit encore sur lui. L'enfant éternua sept fois et ouvrit ses yeux.
\VS{36}Alors, Elisée appela Guéhazi, et lui dit~: Appelle cette Sunamite. Guéhazi l'appela, et elle vint vers Elisée qui lui dit~: Prends ton fils~!
\VS{37}Elle se jeta à ses pieds et se prosterna contre terre. Puis elle prit son fils et sortit.
\TextTitle{Les coloquintes sauvages}
\VS{38}Après cela, Elisée revint à Guilgal. Or il y avait une famine\FTNT{Par le passé, Israël a connu plusieurs famines, dont celle relatée en 2 R. 4:38-41.. Dans ce passage, l'un des fils des prophètes trouva une vigne sauvage dans un champ et y cueillit des coloquintes sauvages. Il les ajouta au potage qui mijotait dans un pot, ne sachant pas que c'était du poison. Le pot est l'image des églises de Laodicée dans lesquelles il y a un mélange mortel de fausses doctrines et de préceptes mondains qui viennent altérer la vérité de la parole de Dieu. Ce mélange impur est absorbé par des millions de personnes ignorantes à travers le monde. Celles-ci se rendent compte qu'elles ont été empoisonnées spirituellement, et une fois le mélange ingéré, elles constatent les effets pervers et dévastateurs souvent tardivement. Le champ tout comme la vigne sauvage, selon Mt. 13:38 et Ro. 11:17, symbolise le monde. Il est par ailleurs intéressant de noter que le mot herbe, «~owrah~» en hébreu, signifie aussi lumière (Ps. 139:12). Cette histoire n'est pas sans nous rappeler le feu étranger introduit par les fils d'Aaron dans le tabernacle, et ce, malgré l'interdiction formelle de Yahweh (Ex. 30:9~; Lé. 10:1-5). C'est exactement ce qui se passe de nos jours. Les églises importent de plus en plus en leur sein la lumière luciférienne du monde (musique, marketing, philosophie, etc.). Beaucoup de pasteurs et de musiciens cherchent malheureusement leur inspiration dans le monde à cause de la famine qui sévit dans les églises. Ce feu étranger représente la plupart des doctrines et pratiques promues par l'église de Laodicée.} dans le pays, et les fils des prophètes étaient assis devant lui~; et il dit à son serviteur~: Mets le grand pot et fais cuire du potage pour les fils des prophètes.
\VS{39}Mais quelqu'un étant sorti dans les champs pour cueillir des herbes, trouva de la vigne sauvage, et cueillit des coloquintes sauvages plein sa robe, et étant revenu, il les coupa en morceaux dans le pot où était le potage, car on ne savait pas ce que c'était.
\VS{40}Et on servit à manger de ce potage à quelques-uns~; mais aussitôt qu'ils eurent mangé de ce potage, ils s'écrièrent et dirent~: Homme de Dieu, la mort est dans le pot~! Et ils ne purent en manger.
\VS{41}Et il dit~: Apportez-moi de la farine~; et il en jeta dans le pot, puis il dit~: Qu'on en verse à ce peuple, afin qu'il mange~; et il n'y avait plus rien de mauvais dans le pot.
\TextTitle{Multiplication de pains}
\VS{42}Un homme venant de Baal-Schalischa apporta à l'homme de Dieu du pain des prémices, à savoir vingt pains d'orge et des épis nouveaux. Elisée dit~: Donne cela à ces gens, et qu'ils mangent.
\VS{43}Son serviteur répondit~: Comment pourrais-je en donner à cent hommes~? Mais Elisée lui répondit~: Donne-les à ces gens, et qu'ils mangent~; car ainsi parle Yahweh~: Ils mangeront et il en restera encore.
\VS{44}Il mit donc les pains devant eux. Ils mangèrent et en eurent de reste, selon la parole de Yahweh.
\Chap{5}
\TextTitle{Guérison miraculeuse de Naaman}
\VerseOne{}Or Naaman, chef de l'armée du roi de Syrie, était un homme puissant et très considéré aux yeux de son maître~; car c'était par lui que Yahweh avait délivré les Syriens. Mais cet homme fort et vaillant était lépreux.
\VS{2}Et les Syriens étaient sortis par troupes, et ils avaient emmené prisonnière une petite fille du pays d'Israël, qui était au service de la femme de Naaman.
\VS{3}Elle dit à sa maîtresse~: Oh~! Si mon seigneur se présentait devant le prophète qui est à Samarie, il le guérirait de sa lèpre~!
\VS{4}Naaman le rapporta à son maître, en disant~: La fille qui est du pays d'Israël a dit telle et telle chose.
\VS{5}Et le roi de Syrie dit à Naaman~: Va, rends-toi à Samarie et j'enverrai une lettre au roi d'Israël. Naaman donc s'en alla et prit avec lui dix talents d'argent et six mille pièces d'or, et dix vêtements de rechange.
\VS{6}Il porta au roi d'Israël la lettre, où il était dit~: Dès que cette lettre te sera parvenue, sache que je t'ai envoyé Naaman, mon serviteur, afin que tu le guérisses de sa lèpre.
\VS{7}Et dès que le roi d'Israël eut lu la lettre, il déchira ses vêtements et dit~: Suis-je Dieu pour faire mourir et pour rendre la vie, pour qu'il s'adresse à moi afin que je guérisse un homme de sa lèpre~? Voyez et comprenez qu'il cherche certainement une occasion de dispute avec moi.
\VS{8}Et il arriva qu'aussitôt qu'Elisée, homme de Dieu, apprit que le roi d'Israël avait déchiré ses vêtements, il envoya dire au roi~: Pourquoi as-tu déchiré tes vêtements~? Laisse-le venir vers moi et il saura qu'il y a un prophète en Israël.
\VS{9}Naaman vint avec ses chevaux et son char, et il s'arrêta à la porte de la maison d'Elisée.
\VS{10}Elisée envoya un messager vers lui, pour lui dire~: Va, et lave-toi sept fois dans le Jourdain, et ta chair redeviendra saine, et tu seras pur.
\VS{11}Mais Naaman se mit dans une grande colère, et s'en alla en disant~: Voilà, je me disais~: Il sortira et viendra vers moi, il se présentera lui-même, il invoquera le Nom de Yahweh, son Dieu, puis, il agitera sa main sur la plaie, et guérira le lépreux.
\VS{12}Les fleuves de Damas, l'Abana et le Parpar ne sont-ils pas meilleurs que toutes les eaux d'Israël~? Ne pourrais-je pas m'y laver et devenir pur~? Ainsi donc, il s'en retourna et s'en alla furieux.
\VS{13}Mais ses serviteurs s'approchèrent et lui parlèrent en disant~: Mon père, si le prophète t'avait imposé quelque chose de difficile, ne l'aurais-tu pas fait~? Combien plus dois-tu faire ce qu'il t'a dit~: Lave-toi, et tu deviendras pur~!
\VS{14}Alors il descendit et se plongea sept fois dans le Jourdain, selon la parole de l'homme de Dieu~; et sa chair redevint comme la chair d'un petit enfant~; et il fut pur.
\VS{15}Il retourna vers l'homme de Dieu, lui et tout son camp, et il vint se présenter devant lui et dit~: Voici, maintenant je sais qu'il n'y a point d'autre Dieu sur toute la terre, si ce n'est en Israël. Maintenant donc, je te prie, accepte ce présent de ton serviteur.
\VS{16}Elisée répondit~: Yahweh, devant lequel je me tiens, est vivant~! Je ne l'accepterai pas~! Naaman le pressa fort de l'accepter, mais Elisée refusa~!
\VS{17}Alors, Naaman dit~: Je te prie, permets que l'on donne de la terre à ton serviteur, une charge de deux mulets~; car ton serviteur ne fera plus d'holocauste ni de sacrifice à d'autres dieux, mais seulement à Yahweh.
\VS{18}Voici toutefois, que Yahweh pardonne ceci à ton serviteur. Quand mon maître entre dans la maison de Rimmon pour s'y prosterner et qu'il s'appuie sur ma main, je me prosterne aussi dans la maison de Rimmon~: Que Yahweh me pardonne, quand je me prosternerai dans la maison de Rimmon.
\TextTitle{Convoitise et mensonge de Guéhazi~; jugement de Dieu}
\VS{19}Elisée lui dit~: Va en paix. Lorsque Naaman eut quitté Elisée et qu'il fut à une certaine distance,
\VS{20}Guéhazi\FTNT{Guéhazi, dont le nom hébreu signifie «~vallée de la vision~».}, le serviteur d'Elisée, homme de Dieu, se dit en lui-même~: Voici, mon maître a ménagé Naaman, ce Syrien, et n'a pas accepté de sa main ce qu'il avait apporté~; Yahweh est vivant~! Je vais courir après lui et j'en obtiendrai quelque chose.
\VS{21}Et Guéhazi courut après Naaman. Naaman, le voyant courir après lui, descendit de son char pour aller à sa rencontre. Il dit~: Tout va bien~?
\VS{22}Guéhazi répondit~: Tout va bien. Mon maître m'envoie te dire~: Voici, il vient d'arriver chez moi deux jeunes hommes de la montagne d'Ephraïm, d'entre les fils des prophètes. Je te prie donne-leur un talent d'argent et deux vêtements de rechange.
\VS{23}Et Naaman dit~: Consens à prendre deux talents. Il insista, puis il serra deux talents d'argent dans deux sacs avec deux vêtements de rechange et les fit porter devant Guéhazi par deux de ses serviteurs.
\VS{24}Et quand il fut arrivé dans un lieu secret, il les prit de leurs mains, et les déposa dans la maison, et il renvoya ces gens qui s'en allèrent.
\VS{25}Puis il entra et se présenta devant son maître. Elisée lui dit~: D'où viens-tu, Guéhazi~? Et il répondit~: Ton serviteur n'est allé nulle part.
\VS{26}Mais Elisée lui dit~: Mon cœur n'est-il pas allé là, lorsque cet homme a quitté son char pour venir à ta rencontre~? Est-ce le temps de prendre de l'argent, de prendre des vêtements, des oliviers, des vignes, du menu et du gros bétail, des serviteurs et des servantes~?
\VS{27}C'est pourquoi la lèpre de Naaman s'attachera à toi et à ta postérité à jamais. Et Guéhazi sortit de la présence d'Elisée avec une lèpre comme de la neige.
\Chap{6}
\TextTitle{Miracle du fer de hache}
\VerseOne{}Les fils des prophètes dirent à Elisée~: Voici, le lieu où nous sommes assis devant toi est trop étroit pour nous.
\VS{2}Allons jusqu'au Jourdain~; nous prendrons là chacun une poutre et nous y ferons un lieu d'habitation. Elisée répondit~: Allez~!
\VS{3}Et l'un d'eux dit~: Veuille, je te prie, venir avec tes serviteurs. Il répondit~: J'irai.
\VS{4}Il partit donc avec eux. Arrivés au Jourdain, ils coupèrent du bois.
\VS{5}Mais il arriva que comme l'un d'eux abattait une poutre, le fer de sa cognée tomba dans l'eau. Il s'écria et dit~: Ah~! Mon seigneur~! Je l'avais emprunté~!
\VS{6}L'homme de Dieu dit~: Où est-il tombé~? Et il lui montra l'endroit. Alors Elisée coupa un morceau de bois, le jeta au même endroit, et fit surnager le fer.
\VS{7}Et il dit~: Retire-le~! Et cet homme étendit sa main et le prit.
\TextTitle{Yahweh révèle à Elisée les plans militaires des Syriens}
\VS{8}Le roi de Syrie était en guerre avec Israël, et, dans un conseil qu'il tint avec ses serviteurs, il dit~: Mon camp sera dans un tel lieu.
\VS{9}L'homme de Dieu envoya dire au roi d'Israël~: Garde-toi de passer dans ce lieu, car les Syriens y descendent.
\VS{10}Et le roi d'Israël envoya des gens, pour s'y tenir en observation, vers le lieu que l'homme de Dieu lui avait mentionné et signalé. Et il y était sur ses gardes. Et cela n'arriva pas seulement une fois ni deux fois.
\VS{11}Le roi de Syrie en eut le cœur troublé~; et il appela ses serviteurs et leur dit~: Ne voulez-vous pas me déclarer lequel de vous est pour le roi d'Israël~?
\VS{12}Et l'un de ses serviteurs répondit~: Personne~! Ô roi, mon seigneur~! Mais Elisée, le prophète qui est en Israël, révèle au roi d'Israël les paroles même que tu déclares dans ta chambre à coucher.
\VS{13}Et il dit~: Allez et voyez où il est, et je le ferai prendre. On vint lui dire~: Voici, il est à Dothan.
\VS{14}Il envoya là des chevaux et des chars, et une grande armée, qui arrivèrent de nuit, et qui entourèrent la ville.
\TextTitle{L'armée de Yahweh plus grande que celle des Syriens}
\VS{15}Le serviteur de l'homme de Dieu se leva de grand matin et sortit~; et voici, une armée, entourait la ville, avec des chevaux et des chars. Le serviteur dit à l'homme de Dieu~: Ah~! Mon seigneur, comment ferons-nous~?
\VS{16}Il lui répondit~: Ne crains point, car ceux qui sont avec nous sont en plus grand nombre que ceux qui sont avec eux.
\VS{17}Elisée pria et dit~: Je te prie, ô Yahweh~! Ouvre ses yeux, afin qu'il voie. Et Yahweh ouvrit les yeux du serviteur et il vit. Et voici la montagne était pleine de chevaux et de chars de feu autour d'Elisée.
\TextTitle{Dieu aveugle les Syriens à la prière d'Elisée}
\VS{18}Les Syriens descendirent vers Elisée. Il adressa alors cette prière à Yahweh~: Je te prie, frappe ces gens d'aveuglement~! Et Dieu les frappa d'aveuglement, selon la parole d'Elisée.
\VS{19}Elisée leur dit~: Ce n'est pas ici le chemin, et ce n'est pas ici la ville~; suivez-moi et je vous conduirai vers l'homme que vous cherchez. Et il les conduisit à Samarie.
\VS{20}Et il arriva qu'aussitôt qu'ils furent entrés dans Samarie, Elisée dit~: Ô Yahweh ouvre leurs yeux afin qu'ils voient. Et Yahweh ouvrit leurs yeux et ils virent qu'ils étaient au milieu de Samarie.
\VS{21}Et dès que le roi d'Israël le vit, il dit à Elisée~: Frapperai-je, frapperai-je, mon père~?
\VS{22}Et Elisée répondit~: Tu ne frapperas point~; frapperais-tu de ton épée et de ton arc ceux que tu as fait prisonniers~? Sers-leur du pain et de l'eau afin qu'ils mangent et boivent~; et après cela, qu'ils s'en aillent vers leur maître.
\VS{23}Le roi d'Israël leur fit servir un grand repas et ils mangèrent et burent~; puis il les renvoya et ils s'en allèrent vers leur maître. Alors, les armées de Syrie ne revinrent plus au pays d'Israël.
\TextTitle{Siège des Syriens et famine en Samarie}
\VS{24}Et il arriva après cela que Ben-Hadad, roi de Syrie, rassembla toute son armée, monta et assiégea Samarie.
\VS{25}Il y eut une grande famine\FTNT{Cette histoire est riche en enseignements pour notre génération. Le siège de la Samarie par les étrangers, la famine qui frappait les Hébreux, le cannibalisme de certaines femmes, la cherté des produits alimentaires, la consommation d'excréments d'animaux à cause de la famine, sont des conséquences du péché. Aujourd'hui, beaucoup d'églises sont assiégées par les choses du monde, les démons, les fausses doctrines, etc.} dans Samarie~; ils l'assiégèrent tellement qu'une tête d'âne se vendait quatre-vingts pièces d'argent, et le quart d'un kab de fiente de pigeon cinq pièces d'argent.
\VS{26}Et comme le roi d'Israël passait sur la muraille, une femme lui cria~: Ô roi, mon seigneur~! Sauve-moi.
\VS{27}Il répondit~: Si Yahweh ne te sauve pas, comment pourrais-je te sauver~? Serait-ce avec le produit de l'aire ou de la cuve~?
\VS{28}Il lui dit encore~: Qu'as-tu~? Elle répondit~: Cette femme-là m'a dit~: Donne ton fils, et mangeons-le aujourd'hui, et nous mangerons mon fils demain\FTNT{Lé. 26:29~; De. 28:53-57.}.
\VS{29}Ainsi nous avons fait bouillir mon fils et l'avons mangé. Et le jour suivant, je lui ai dit~: Donne ton fils et nous le mangerons. Mais elle a caché son fils.
\VS{30}Dès que le roi entendit les paroles de cette femme, il déchira ses vêtements et passa sur la muraille. Le peuple vit qu'il avait en dessous un sac sur son corps.
\VS{31}C'est pourquoi le roi dit~: Que Dieu me traite dans toute sa rigueur, si aujourd'hui la tête d'Elisée, fils de Schaphath, reste sur lui.
\VS{32}Or Elisée était assis dans sa maison, et les anciens étaient assis avec lui. Le roi envoya un homme devant lui. Mais avant que le messager soit arrivé, Elisée dit aux anciens~: Ne voyez-vous pas que le fils de ce meurtrier envoie quelqu'un pour m'ôter la tête~? Lorsque le messager viendra, fermez la porte et repoussez-le avec la porte. N'entendez-vous pas le bruit des pas de son maître derrière lui~?
\VS{33}Et comme il parlait encore avec eux, voici le messager descendit vers lui et dit~: Voici, ce mal vient de Yahweh~; qu'ai-je à espérer encore de Yahweh~?
\Chap{7}
\TextTitle{Prophétie d'Elisée~; les lépreux dans le camp des Syriens}
\VerseOne{}Alors Elisée dit~: Ecoutez la parole de Yahweh~! Ainsi parle Yahweh~: Demain, à cette heure, on aura une mesure de fleur de farine pour un sicle, et deux mesures d'orge pour un sicle, à la porte de Samarie.
\VS{2}Mais l'officier sur la main duquel le roi s'appuyait répondit à l'homme de Dieu et dit~: Quand Yahweh ferait des fenêtres au ciel, cela arriverait-il~? Et Elisée dit~: Tu le verras de tes yeux, mais tu n'en mangeras pas.
\VS{3}Or il y avait à l'entrée de la porte quatre hommes lépreux\FTNT{Dieu s'est servi de ces quatre lépreux comme messagers de bonnes nouvelles. Le Seigneur utilise souvent les personnes rejetées et déconsidérées (1 Co. 1:26-31).}, et ils se dirent l'un à l'autre~: Pourquoi resterions-nous ici jusqu'à ce que nous mourions~?
\VS{4}Si nous pensons à entrer dans la ville, la famine est dans la ville et nous y mourrons~; et si nous restons ici, nous mourrons également. Allons-nous jeter dans le camp des Syriens~; s'ils nous laissent vivre, nous vivrons, et s'ils nous font mourir, nous mourrons.
\VS{5}Ils se levèrent donc au crépuscule pour entrer au camp des Syriens. Lorsqu'ils furent arrivés à l'extrémité du camp, voici, il n'y avait personne.
\VS{6}Car le Seigneur avait fait entendre dans le camp des Syriens un bruit de chars, et un bruit de chevaux, et un bruit d'une grande armée~; de sorte qu'ils s'étaient dit l'un à l'autre~: Voici, le roi d'Israël a payé les rois des Héthiens et les rois des Egyptiens pour venir contre nous.
\VS{7}C'est pourquoi ils s'étaient levés au crépuscule et s'étaient enfuis. Ils avaient abandonné leurs tentes, leurs chevaux, leurs ânes, et le camp tel qu'il était, et ils s'étaient enfuis pour sauver leur vie.
\VS{8}Les lépreux donc arrivèrent jusqu'à l'extrémité du camp. Ils entrèrent dans une tente, mangèrent, burent, emportèrent de l'argent, de l'or, des vêtements, et ils s'en allèrent et les cachèrent. Ils revinrent et entrèrent dans une autre tente et emportèrent de là aussi des objets, s'en allèrent et les cachèrent.
\VS{9}Alors ils se dirent l'un à l'autre~: Nous n'agissons pas bien~! Ce jour est un jour de bonnes nouvelles~; si nous gardons le silence et si nous attendons jusqu'à lumière du matin, le châtiment nous atteindra. Venez maintenant et allons informer la maison du roi.
\VS{10}Ils partirent et appelèrent les portiers de la ville, et leur racontèrent, en disant~: Nous sommes entrés dans le camp des Syriens, et voici, il n'y a personne. On n'y entend aucune voix d'homme~; il n'y a que des chevaux attachés, des ânes attachés et les tentes sont comme elles étaient.
\VS{11}Alors les portiers crièrent et transmirent ce rapport à la maison du roi.
\TextTitle{Accomplissement de la prophétie d'Elisée}
\VS{12}Le roi se leva de nuit et dit à ses serviteurs~: Je veux vous dire ce que les Syriens ont préparé contre nous. Ils savent que nous sommes affamés et ils sont sortis du camp pour se cacher dans les champs, disant~: Quand ils sortiront hors de la ville, nous les saisirons vivants et nous entrerons dans la ville.
\VS{13}L'un des serviteurs du roi répondit et dit~: Qu'on prenne cinq des chevaux qui restent encore dans la ville~; c'est presque tout ce qui est resté du grand nombre des chevaux d'Israël~; ils sont comme toute la multitude d'Israël, qui est consumée. Envoyons voir ce qui se passe.
\VS{14}Ils prirent donc deux chars avec les chevaux, et le roi envoya des messagers après l'armée des Syriens, en disant~: Allez et voyez.
\VS{15}Et ils allèrent après eux jusqu'au Jourdain~; et voici, le chemin était plein de vêtements et d'objets que les Syriens avaient jetés dans leur précipitation. Les messagers revinrent et le rapportèrent au roi.
\VS{16}Alors le peuple sortit et pilla le camp des Syriens, de sorte qu'il eut une mesure de fleur de farine pour un sicle, et deux mesures d'orge pour un sicle, selon la parole de Yahweh.
\VS{17}Le roi donna à l'officier, sur la main duquel il s'appuyait, la charge de garder la porte. Mais cet officier fut écrasé à la porte par le peuple et il en mourut selon la parole qu'avait prononcée l'homme de Dieu, quand le roi était descendu vers lui.
\VS{18}Car lorsque l'homme de Dieu avait parlé au roi, en disant~: Demain matin, à cette heure-ci, on donnera à la porte de Samarie deux mesures d'orge pour un sicle et une mesure de fleur de farine pour un sicle~;
\VS{19}cet officier avait répondu à l'homme de Dieu~: Quand Yahweh ferait des fenêtres au ciel, ce que tu dis pourrait-il arriver~? Et l'homme de Dieu avait dit~: Voici, tu le verras de tes yeux, mais tu n'en mangeras pas.
\VS{20}C'est en effet ce qui lui arriva~; car le peuple l'écrasa à la porte et il mourut.
\Chap{8}
\TextTitle{Elisée annonce une famine de sept ans}
\VerseOne{}Elisée parla à la femme dont il avait fait revivre le fils, en disant~: Lève-toi et va-t'en, toi et ta famille, et séjourne où tu pourras~; car Yahweh a appelé la famine, et même elle vient sur le pays pour sept ans.
\VS{2}La femme se leva et elle fit selon la parole de l'homme de Dieu. Elle s'en alla, elle et sa famille, et séjourna sept ans au pays des Philistins.
\TextTitle{La Sunamite retrouve ses terres}
\VS{3}Mais il arriva qu'au bout des sept ans, la femme revint du pays des Philistins, et alla implorer le roi au sujet de sa maison et de ses champs.
\VS{4}Le roi parlait à Guéhazi\FTNT{Voir 2 R. 5.}, serviteur de l'homme de Dieu, en disant~: Je te prie raconte-moi toutes les grandes choses qu'Elisée a faites.
\VS{5}Et il arriva que comme il racontait au roi comment Elisée avait rendu la vie à un mort, la femme dont Elisée avait fait revivre le fils vint implorer le roi au sujet de sa maison et de ses champs. Guéhazi dit~: Ô roi, mon seigneur, voici la femme et voici son fils, à qui Elisée a rendu la vie.
\VS{6}Alors le roi interrogea la femme, et elle lui raconta ce qui s'était passé. Le roi lui donna un eunuque, auquel il dit~: Fais restituer tout ce qui lui appartenait, même tous les revenus de ses champs, depuis le jour où elle a quitté le pays jusqu'à maintenant.
\TextTitle{Prophétie sur le règne d'Hazaël sur la Syrie}
\VS{7}Elisée se rendit à Damas. Ben-Hadad, roi de Syrie, était malade et on lui fit ce rapport~: L'homme de Dieu est venu ici.
\VS{8}Le roi dit à Hazaël~: Prends avec toi un présent et va au-devant de l'homme de Dieu, et consulte par lui Yahweh, en disant~: Guérirai-je de cette maladie~?
\VS{9}Et Hazaël s'en alla au-devant d'Elisée, ayant pris avec lui un présent, à savoir quarante chameaux chargés de tout ce qu'il y avait de meilleur à Damas. Il vint se présenter devant Elisée et dit~: Ton fils, Ben-Hadad, roi de Syrie, m'a envoyé vers toi, pour te dire~: Guérirai-je de cette maladie~?
\VS{10}Et Elisée lui répondit~: Va, dis-lui~: Tu guériras~! Tu guériras~! Toutefois, Yahweh m'a révélé qu'il mourra, qu'il mourra.
\VS{11}L'homme de Dieu arrêta son regard sur Hazaël et le fixa longtemps, puis il pleura.
\VS{12}Hazaël dit~: Pourquoi mon seigneur pleure-t-il~? Et il répondit~: Parce que je sais le mal que tu feras aux enfants d'Israël~; tu mettras le feu à leurs villes fortes, tu tueras avec l'épée leurs jeunes gens, tu écraseras leurs petits-enfants et tu fendras le ventre de leurs femmes enceintes.
\VS{13}Hazaël dit~: Mais qu'est-ce que ton serviteur, ce chien, pour faire de si grandes choses~? Et Elisée répondit~: Yahweh m'a révélé que tu seras roi de Syrie.
\VS{14}Alors Hazaël quitta Elisée et revint vers son maître, qui lui demanda~: Que t'a dit Elisée~? Et il répondit~: Il m'a dit que tu guériras~! Tu guériras~!
\VS{15}Mais le lendemain, Hazaël prit une couverture et l'ayant plongé dans l'eau, il l'étendit sur le visage de Ben-Hadad, qui mourut. Et Hazaël régna à sa place.
\TextTitle{Joram règne sur Juda\FTNTT{2 Ch. 21:1-7.}}
\VS{16}La cinquième année de Joram, fils d'Achab, roi d'Israël, Josaphat était encore roi de Juda et Joram, fils de Josaphat, roi de Juda, commença à régner sur Juda.
\VS{17}Il était âgé de trente-deux ans lorsqu'il commença à régner. Il régna huit ans à Jérusalem.
\VS{18}Il marcha dans la voie des rois d'Israël comme avait fait la maison d'Achab, car il avait pour femme la fille d'Achab\FTNT{Le mariage de Joram, fils de Josaphat, avec Athalie, fille d'Achab, était une grande erreur. Cette union qui était contractée dans le but de favoriser la paix entre les deux royaumes, entraîna le déclin de Juda~; or Dieu est contre les alliances contre nature. Voir Es. 30-31.}, et il fit ce qui est mal aux yeux de Yahweh.
\VS{19}Mais Yahweh ne voulut point détruire Juda, par amour pour David, son serviteur, selon la promesse qu'il lui avait faite de lui donner toujours une lampe parmi ses fils.
\TextTitle{Révoltes contre l'autorité de Juda}
\VS{20}De son temps, Edom se révolta contre l'autorité de Juda et se donna un roi.
\VS{21}Joram passa à Tsaïr, avec tous ses chars~; il se leva de nuit, et frappa les Edomites qui l'entouraient, et les chefs des chars, mais le peuple s'enfuit dans ses tentes.
\VS{22}Néanmoins, les Edomites ont été rebelles à Juda jusqu'à ce jour. En ce même temps, Libna aussi se révolta.
\TextTitle{Achazia règne sur Juda\FTNTT{2 Ch. 21:18-22:4.}}
\VS{23}Le reste des actions de Joram et tout ce qu'il a fait, cela n'est-il pas écrit dans le livre des Chroniques des rois de Juda~?
\VS{24}Joram se coucha avec ses pères et il fut enterré avec ses pères dans la cité de David. Et Achazia, son fils, régna à sa place.
\VS{25}La douzième année de Joram, fils d'Achab, roi d'Israël, Achazia, fils de Joram, roi de Juda, commença à régner.
\VS{26}Achazia était âgé de vingt-deux ans lorsqu'il commença à régner. Il régna un an à Jérusalem. Sa mère s'appelait Athalie, fille d'Omri, roi d'Israël.
\VS{27}Il marcha dans la voie de la maison d'Achab et il fit ce qui est mal aux yeux de Yahweh, comme avait fait la maison d'Achab, car il était gendre de la maison d'Achab.
\VS{28}Il alla avec Joram, fils d'Achab, à la guerre contre Hazaël, roi de Syrie, à Ramoth en Galaad. Et les Syriens blessèrent Joram.
\VS{29}Le roi Joram s'en retourna pour se faire guérir à Jizreel des blessures que les Syriens lui avaient faites à Rama, lorsqu'il se battait contre Hazaël, roi de Syrie. Achazia, fils de Joram, roi de Juda, descendit pour voir Joram, fils d'Achab, à Jizreel, parce qu'il était malade.
\Chap{9}
\TextTitle{Jéhu oint roi d'Israël}
\VerseOne{}Alors Elisée, le prophète, appela l'un des fils des prophètes et lui dit~: Ceins tes reins, prends cette fiole d'huile dans ta main, et va à Ramoth en Galaad.
\VS{2}Quand tu y seras entré, vois Jéhu, fils de Josaphat, fils de Nimschi. Tu iras le faire lever du milieu de ses frères et tu le conduiras dans une chambre secrète.
\VS{3}Tu prendras la fiole d'huile, tu la verseras sur sa tête et tu diras~: Ainsi parle Yahweh~: Je t'ai oint pour être roi sur Israël. Après quoi tu ouvriras la porte, tu t'enfuiras et tu ne t'arrêteras pas.
\VS{4}Le jeune homme, serviteur du prophète, s'en alla à Ramoth en Galaad.
\VS{5}Quand il arriva, voici, les chefs de l'armée étaient là assis. Il dit~: Chef, j'ai à te parler. Et Jéhu répondit~: Auquel de nous parles-tu~? Et il répondit~: A toi, chef.
\VS{6}Alors Jéhu se leva, et entra dans la maison, et le jeune homme répandit l'huile sur la tête, et lui dit~: Ainsi parle Yahweh, le Dieu d'Israël~: Je t'ai oint pour être roi sur Israël, le peuple de Yahweh.
\VS{7}Tu frapperas la maison d'Achab, ton maître, et je vengerai sur Jézabel\FTNT{1 R. 16:31~; 1 R. 17,18,19.} le sang de mes serviteurs les prophètes, et le sang de tous les serviteurs de Yahweh.
\VS{8}Et toute la maison d'Achab périra, et je retrancherai quiconque appartient à Achab, celui qui est esclave et celui qui est libre en Israël.
\VS{9}Je rendrai la maison d'Achab semblable à la maison de Jéroboam, fils de Nebath, et à la maison de Baescha, fils d'Achija.
\VS{10}Les chiens mangeront Jézabel dans le champ de Jizreel, et il n'y aura personne pour l'enterrer. Puis il ouvrit la porte et s'enfuit.
\VS{11}Jéhu sortit pour rejoindre les serviteurs de son maître et on lui dit~: Tout va bien~? Pourquoi ce fou est-il venu vers toi~? Jéhu leur répondit~: Vous connaissez l'homme et ses rêveries.
\VS{12}Mais ils répliquèrent~: Mensonge~! Réponds-nous donc. Et il dit~: Il m'a parlé de telle et telle manière, disant~: Ainsi parle Yahweh, je t'ai oint pour être roi sur Israël.
\VS{13}Alors ils se hâtèrent, et prirent chacun leurs vêtements, et les mirent sous lui au plus haut des degrés. Ils sonnèrent du shofar et dirent~: Jéhu a été fait roi~!
\TextTitle{Mort de Joram}
\VS{14}Ainsi Jéhu, fils de Josaphat, fils de Nimschi, forma une conspiration contre Joram. Or Joram et tout Israël défendaient Ramoth en Galaad contre Hazaël, roi de Syrie.
\VS{15}Le roi Joram s'en était retourné pour se faire guérir à Jizreel des blessures que les Syriens lui avaient faites, lorsqu'il se battait contre Hazaël, roi de Syrie. Jéhu dit~: Si vous le trouvez bon, que personne ne sorte ni ne s'échappe de la ville pour aller porter cette nouvelle à Jizreel.
\VS{16}Alors, Jéhu monta à cheval et s'en alla à Jizreel, car Joram était là, malade, et Achazia, roi de Juda, y était descendu pour le visiter.
\VS{17}Or il y avait une sentinelle sur une tour à Jizreel, qui voyant venir la troupe de Jéhu dit~: Je vois une troupe de gens. Et Joram dit~: Prends un cavalier et envoie-le à leur rencontre, et qu'il dise~: Est-ce la paix~?
\VS{18}Le cavalier s'en alla à sa rencontre, et dit~: Ainsi parle le roi~: Est-ce la paix~? Et Jéhu répondit~: Qu'as-tu à faire de la paix~? Mets-toi derrière moi. La sentinelle le rapporta, en disant~: Le messager est allé jusqu'à eux et il ne revient pas.
\VS{19}Joram envoya un second cavalier, qui arriva jusqu'à eux et dit~: Ainsi parle le roi~: Est-ce la paix~? Et Jéhu répondit~: Qu'as-tu à faire de la paix~? Mets-toi derrière moi.
\VS{20}La sentinelle le rapporta et dit~: Il est arrivé jusqu'à eux et il ne revient pas~; mais la manière de conduire le char est comme celle de Jéhu, fils de Nimschi~; car il le conduit avec furie.
\VS{21}Alors Joram dit~: Attelle~! Et on attela son char. Ainsi Joram, roi d'Israël, sortit avec Achazia, roi de Juda, chacun dans son char, et ils allèrent à la rencontre de Jéhu, et ils le trouvèrent dans le champ de Naboth de Jizreel\FTNT{1 R. 21.}.
\VS{22}Dès que Joram vit Jéhu, il dit~: Est-ce la paix, Jéhu~? Jéhu répondit~: Quelle paix~! Tant que durent les prostitutions de Jézabel, ta mère, et la multitude de ses enchantements~!
\VS{23}Alors Joram tourna sa main et s'enfuit, et il dit à Achazia~: Trahison, Achazia~!
\VS{24}Mais Jéhu saisit l'arc de sa main, et il frappa Joram entre ses épaules, de sorte que la flèche transperça son cœur, et il tomba sur ses genoux dans son char.
\VS{25}Jéhu dit à Bidkar, son officier~: Prends-le et jette-le dans le champ de Naboth de Jizreel~; car souviens-toi, lorsque nous étions à cheval moi et toi, ensemble, derrière Achab, son père, Yahweh prononça cette sentence contre lui~:
\VS{26}N'ai-je pas vu hier le sang de Naboth et le sang de ses fils, dit Yahweh~? Et je te le rendrai dans ce champ-ci, dit Yahweh~! C'est pourquoi prends-le donc, et jette-le dans ce champ, selon la parole de Yahweh.
\TextTitle{Mort d'Achazia\FTNTT{2 Ch. 22:7,9.}}
\VS{27}Achazia, roi de Juda, ayant vu cela, s'enfuit par le chemin de la maison du jardin~; mais Jéhu le poursuivit et dit~: Frappez-le sur le char~! Et on le frappa à la montée de Gur, près de Jibleam. Puis il se réfugia à Meguiddo, et il y mourut.
\VS{28}Ses serviteurs le transportèrent sur un char à Jérusalem, et ils l'enterrèrent dans son sépulcre avec ses pères, dans la cité de David.
\VS{29}Achazia avait commencé à régner sur Juda la onzième année de Joram, fils d'Achab.
\TextTitle{Mort de Jézabel}
\VS{30}Jéhu entra dans Jizreel. Jézabel, l'ayant appris, mit du fard à ses yeux, orna sa tête et regarda par la fenêtre.
\VS{31}Comme Jéhu franchissait la porte, elle dit~: Est-ce la paix, Zimri, assassin de son maître~?
\VS{32}Il leva sa tête vers la fenêtre et dit~: Qui est avec moi~? Qui~? Alors deux ou trois des eunuques regardèrent vers lui.
\VS{33}Et il leur dit~: Jetez-la en bas~! Et ils la jetèrent, de sorte qu'il rejaillit de son sang sur la muraille et sur les chevaux. Jéhu la foula aux pieds~;
\VS{34}puis il entra, mangea et but, et il dit~: Allez voir maintenant cette maudite et enterrez-la, car elle est fille de roi.
\VS{35}Ils allèrent donc pour l'enterrer~; mais ils ne trouvèrent d'elle que le crâne, les pieds et les paumes des mains.
\VS{36}Ils retournèrent l'annoncer à Jéhu, qui dit~: C'est la parole que Yahweh avait déclarée par son serviteur Elie\FTNT{1 R. 21:23.}, le Thischbite, en disant~: Dans le champ de Jizreel les chiens mangeront la chair de Jézabel~;
\VS{37}et le cadavre de Jézabel sera comme du fumier sur la face des champs, dans le champ de Jizreel, de sorte qu'on ne pourra dire~: C'est Jézabel.
\Chap{10}
\TextTitle{Accomplissement du jugement de Dieu sur la maison d'Achab}
\VerseOne{}Achab avait soixante-dix fils dans Samarie. Jéhu écrivit des lettres qu'il envoya à Samarie aux chefs de Jizreel, aux anciens et aux gouverneurs d'Achab. Il y était dit~:
\VS{2}Dès que cette lettre vous sera parvenue, puisque vous avez avec vous les fils de votre maître, avec vous les chars et les chevaux, la ville forte et les armes,
\VS{3}choisissez qui est le plus considérable et le plus sincère parmi les fils de votre maître, mettez-le sur le trône de son père et combattez pour la maison de votre maître.
\VS{4}Ils eurent une très grande peur et ils dirent~: Voici, deux rois n'ont point pu tenir contre lui, comment donc résisterions-nous~?
\VS{5}Et le chef de la maison, le chef de la ville, les anciens et les gouverneurs envoyèrent dire à Jéhu~: Nous sommes tes serviteurs, nous ferons tout ce que tu nous diras~; nous n'établirons personne roi, fais ce qui te semblera bon.
\VS{6}Jéhu leur écrivit une seconde lettre, où il était dit~: Si vous êtes pour moi et si vous obéissez à ma voix, prenez les têtes des fils de votre maître et venez auprès de moi demain à cette heure-ci, à Jizreel. Or les soixante-dix hommes, fils du roi, étaient avec les plus grands de la ville qui les élevaient.
\VS{7}Aussitôt que la lettre leur fut parvenue, ils prirent les fils du roi et ils égorgèrent ces soixante-dix hommes~; et ayant mis leurs têtes dans des corbeilles, ils les envoyèrent à Jéhu, à Jizreel.
\VS{8}Un messager vint l'en informer, en disant~: Ils ont apporté les têtes des fils du roi. Et il répondit~: Mettez-les en deux tas à l'entrée de la porte, jusqu'au matin.
\VS{9}Le matin, il sortit~; et se présentant à tout le peuple, il dit~: Vous êtes justes~! Voici, j'ai conspiré contre mon maître et je l'ai tué~; mais qui a frappé tous ceux-ci~?
\VS{10}Sachez maintenant qu'il ne tombera rien à terre de la parole de Yahweh\FTNT{1 R. 21:19-24.}, de la parole que Yahweh a prononcée contre la maison d'Achab~; Yahweh accomplit ce qu'il avait déclaré par son serviteur Elie.
\VS{11}Jéhu tua aussi tous ceux qui restaient de la maison d'Achab à Jizreel, tous ses grands, ses familiers et ses prêtres, sans en laisser échapper un seul.
\TextTitle{Mise à mort des frères d'Achazia et de la lignée d'Achab\FTNTT{2 Ch. 22:8.}}
\VS{12}Puis il se leva et partit pour aller à Samarie. Et comme il était près d'une maison de bergers sur le chemin,
\VS{13}Jéhu trouva les frères d'Achazia, roi de Juda, et leur dit~: Qui êtes-vous~? Ils répondirent~: Nous sommes les frères d'Achazia et nous sommes descendus pour saluer les fils du roi et les fils de la reine.
\VS{14}Jéhu dit~: Saisissez-les vivants. Ils les saisirent vivants et les égorgèrent, à savoir quarante-deux hommes, auprès du puits de la maison des bergers, sans en laisser échapper un seul.
\VS{15}Jéhu étant parti de là, il rencontra Jonadab, fils de Récab, qui venait au-devant de lui. Il le salua, et lui dit~: Ton cœur est-il aussi droit envers moi comme mon cœur l'est à ton égard~? Et Jonadab répondit~: Il l'est. Donne-moi ta main répliqua Jéhu. Et Jonadab lui donna sa main, et Jéhu le fit monter auprès de lui dans son char.
\VS{16}Puis il dit~: Viens avec moi et tu verras le zèle que j'ai pour Yahweh. Il l'emmena ainsi dans son char.
\VS{17}Et quand Jéhu fut arrivé à Samarie, il tua tous ceux qui restaient de la maison d'Achab à Samarie, et il les extermina entièrement, selon la parole que Yahweh avait dite à Elie.
\TextTitle{Mise à mort de tous les prophètes de Baal}
\VS{18}Puis Jéhu assembla tout le peuple, et leur dit~: Achab a peu servi Baal\FTNT{Jg. 2:13.}, mais Jéhu le servira beaucoup.
\VS{19}Maintenant donc, convoquez-moi tous les prophètes de Baal, tous ses serviteurs, et tous ses prêtres, sans qu'il en manque un seul, car je veux offrir un grand sacrifice à Baal~: Quiconque manquera ne vivra pas. Jéhu agissait avec ruse, pour faire périr les serviteurs de Baal.
\VS{20}Jéhu dit~: Publiez une fête solennelle en l'honneur de Baal. Et ils la publièrent.
\VS{21}Jéhu envoya des messagers dans tout Israël~; et tous les serviteurs de Baal arrivèrent, il n'y en eut pas un qui ne vînt~; et ils entrèrent dans le temple de Baal, qui fut rempli d'un bout à l'autre.
\VS{22}Alors Jéhu dit à celui qui avait la charge du vestiaire~: Sors des vêtements pour tous les serviteurs de Baal. Et cet homme sortit des vêtements.
\VS{23}Alors Jéhu, et Jonadab, fils de Récab, entrèrent dans le temple de Baal, et Jéhu dit aux serviteurs de Baal~: Cherchez et regardez afin qu'il n'y ait pas ici de serviteurs de Yahweh. Prenez garde qu'il n'y ait seulement que les serviteurs de Baal.
\VS{24}Ils entrèrent donc pour offrir des sacrifices et des holocaustes. Or Jéhu avait placé dehors quatre-vingts hommes, et leur avait dit~: Celui qui laissera échapper un de ces hommes que je remets entre vos mains, sa vie répondra de la sienne.
\VS{25}Et il arriva que dès qu'on eut achevé d'offrir l'holocauste, Jéhu dit aux gardes et aux officiers~: Entrez, tuez-les, et que nul n'échappe. Les gardes et les officiers les frappèrent du tranchant de l'épée, et les jetèrent là~; puis ils allèrent jusqu'à la ville du temple de Baal.
\VS{26}Ils tirèrent dehors les statues de la maison de Baal, et les brûlèrent.
\VS{27}Et ils démolirent la statue de Baal. Ils démolirent aussi la maison de Baal, et ils en firent un cloaque qui subsiste jusqu'à ce jour.
\VS{28}Ainsi Jéhu extermina Baal d'Israël.
\TextTitle{L'idolâtrie dans la vie de Jéhu}
\VS{29}Toutefois, Jéhu ne se détourna point des péchés que Jéroboam, fils de Nebath, avait fait commettre à Israël, à savoir les veaux d'or\FTNT{1 R. 12:28-29.} qui étaient à Béthel et à Dan.
\VS{30}Yahweh dit à Jéhu~: Parce que tu as fort bien exécuté ce qui était droit à mes yeux, et que tu as fait à la maison d'Achab tout ce qui était conforme à ma volonté, tes fils seront assis sur le trône d'Israël jusqu'à la quatrième génération.
\VS{31}Mais Jéhu ne prit point garde à marcher de tout son cœur dans la loi de Yahweh, le Dieu d'Israël~; il ne se détourna point des péchés que Jéroboam avait fait commettre à Israël.
\TextTitle{Hazaël règne sur la Syrie}
\VS{32}Dans ce temps-là, Yahweh commença à entamer le territoire d'Israël, et Hazaël battit les Israélites sur toutes les frontières.
\VS{33}Depuis le Jourdain, jusqu'au soleil levant, il battit tout le pays de Galaad, les Gadites, les Rubénites et ceux de Manassé, depuis Aroër sur le torrent de l'Arnon, jusqu'à Galaad et à Basan.
\TextTitle{Joachaz règne sur Israël}
\VS{34}Le reste des actions de Jéhu, tout ce qu'il a fait, et tous ses exploits, ne sont-ils pas écrits dans le livre des Chroniques des rois d'Israël~?
\VS{35}Jéhu se coucha avec ses pères, et on l'enterra à Samarie. Et Joachaz, son fils, régna à sa place.
\VS{36}Jéhu avait régné vingt-huit ans sur Israël à Samarie.
\Chap{11}
\TextTitle{Athalie fait périr la race royale de Juda\FTNTT{2 Ch. 22:9-12.}}
\VerseOne{}Athalie, mère d'Achazia, ayant vu que son fils était mort, se leva et extermina toute la race royale.
\VS{2}Mais Joschéba, fille du roi Joram, sœur d'Achazia, prit Joas, fils d'Achazia, et l'enleva du milieu des fils du roi, quand on les fit mourir~: Elle le mit avec sa nourrice dans la chambre des lits. Il fut ainsi dérobé aux regards d'Athalie, de sorte qu'on ne le fit point mourir.
\VS{3}Il resta caché six ans avec Joschéba dans la maison de Yahweh. Cependant Athalie régnait sur le pays.
\TextTitle{Joas devient roi de Juda\FTNTT{2 Ch. 23:1-11.}}
\VS{4}La septième année, Jehojada envoya chercher les chefs de centaines des Kéréthiens et des archers, et il les fit venir auprès de lui dans la maison de Yahweh. Il traita alliance avec eux, les fit jurer dans la maison de Yahweh, et leur montra le fils du roi.
\VS{5}Puis il leur donna cet ordre, en disant~: Voici ce que vous ferez. Parmi ceux d'entre vous qui entrent en service le jour du sabbat, un tiers doit monter la garde à la maison du roi,
\VS{6}un tiers sera à la porte de Sur, et un tiers à la porte derrière les archers~; ainsi vous veillerez à la garde de la maison, afin que personne n'y entre par force.
\VS{7}Vos deux autres compagnies, tous ceux qui sortent de service le jour du sabbat feront la garde de la maison de Yahweh, auprès du roi~:
\VS{8}Et vous entourerez le roi de toutes parts, chacun ayant ses armes à la main, et l'on mettra à mort quiconque s'avancera dans les rangs~; vous serez avec le roi quand il sortira et quand il entrera.
\VS{9}Les chefs de centaines firent donc tout ce que Jehojada, le prêtre, avait ordonné. Ils prirent chacun leurs gens, ceux qui entraient en service et ceux qui sortaient de service le jour du sabbat, et ils se rendirent vers le prêtre Jehojada.
\VS{10}Le prêtre donna aux chefs de centaine les lances et les boucliers qui provenaient du roi David, et qui étaient dans la maison de Yahweh.
\VS{11}Les archers, chacun les armes à la main, entourèrent le roi, en se plaçant depuis le côté droit de la maison, jusqu'au côté gauche, près de l'autel et près de la maison.
\VS{12}Jehojada fit amener le fils du roi, et il mit sur lui la couronne\FTNT{Couronne ou consacrer.} et le témoignage. Ils l'établirent roi et l'oignirent, et frappant des mains, ils dirent~: Vive le roi~!
\TextTitle{Mort d'Athalie\FTNTT{2 Ch. 23:12-15,21.}}
\VS{13}Athalie entendit le bruit des archers et du peuple, et elle vint vers le peuple à la maison de Yahweh.
\VS{14}Elle regarda. Et voici, le roi se tenait sur l'estrade, selon la coutume des rois. Les chefs et les trompettes étaient près du roi~: Tout le peuple du pays éclatait de joie, et on sonnait des trompettes. Alors Athalie déchira ses vêtements, et cria~: Conspiration~! Conspiration~!
\VS{15}Alors le prêtre Jehojada donna cet ordre aux chefs de centaines, qui avaient la charge de l'armée~: Faites-la sortir hors des rangs, et que celui qui la suivra soit mis à mort par l'épée. Car le prêtre avait dit~: Qu'elle ne soit pas mise à mort dans la maison de Yahweh~!
\VS{16}Ils lui firent donc place, et elle retourna dans la maison du roi par le chemin de l'entrée des chevaux~: C'est là qu'elle fut tuée.
\TextTitle{Alliance entre Jehojada, Yahweh et le peuple~; réveil sous le règne de Joas\FTNTT{2 Ch. 23:16-21.}}
\VS{17}Jehojada traita entre Yahweh, le roi et le peuple l'alliance par laquelle ils devaient être le peuple de Yahweh~; il traita aussi l'alliance entre le roi et le peuple.
\VS{18}Alors tout le peuple du pays entra dans la maison de Baal, et ils la démolirent avec ses autels~; et ils brisèrent entièrement ses images~; ils tuèrent aussi Matthan, prêtre de Baal, devant les autels. Le prêtre Jehojada établit des gardes dans la maison de Yahweh.
\VS{19}Il prit les chefs de centaines, les Kéréthiens et les archers, et tout le peuple du pays~; et ils firent descendre le roi de la maison de Yahweh, et ils entrèrent dans la maison du roi par le chemin de la porte des archers, et Joas s'assit sur le trône des rois.
\VS{20}Tout le peuple du pays fut dans la joie, et la ville fut en repos, après qu'on eût mis à mort Athalie par l'épée dans la maison du roi.
\VS{21}Joas était âgé de sept ans lorsqu'il commença à régner.
\Chap{12}
\TextTitle{Joas ordonne des réparations dans le temple\FTNTT{2 Ch. 24:2.}}
\VerseOne{}La septième année de Jéhu, Joas, commença à régner. Il régna quarante ans à Jérusalem. Sa mère s'appelait Tsibja, elle était de Beer-Schéba.
\VS{2}Joas fit ce qui est droit aux yeux de Yahweh pendant tout le temps qu'il suivit les instructions de Jehojada, le prêtre.
\VS{3}Toutefois, les hauts lieux ne disparurent point~; le peuple offrait encore des sacrifices et des parfums sur les hauts lieux.
\VS{4}Joas dit aux prêtres~: Tout l'argent consacré qu'on apporte dans la maison de Yahweh, l'argent ayant cours, à savoir l'argent pour l'évaluation des personnes d'après l'estimation qui en est faite, et tout l'argent que chacun apporte volontairement à la maison de Yahweh,
\VS{5}que les prêtres le prennent, chacun de la part des gens de sa connaissance, et qu'ils l'emploient à réparer ce qui est à réparer dans la maison, partout où l'on trouvera quelque chose à réparer.
\VS{6}Mais il arriva que, la vingt-troisième année du roi Joas, les prêtres n'avaient point encore réparé les brèches de la maison.
\VS{7}Le roi Joas appela le prêtre Jehojada et les autres prêtres, et il leur dit~: Pourquoi n'avez-vous pas réparé ce qui était à réparer dans la maison~? Maintenant, vous ne prendrez plus l'argent de vos connaissances, mais vous le livrerez pour les réparations de la maison.
\VS{8}Les prêtres convinrent de ne plus prendre l'argent du peuple et de ne pas être chargés des réparations de la maison.
\TextTitle{Offrandes volontaires pour réparer le temple\FTNTT{2 Ch. 24:8-14.}}
\VS{9}Alors le prêtre Jehojada prit un coffre, et le perça dans son couvercle, et le plaça à côté de l'autel, à droite, à l'endroit par lequel on entrait à la maison de Yahweh. Les prêtres qui avaient la garde du seuil y mettaient tout l'argent qu'on apportait à la maison de Yahweh.
\VS{10}Et dès qu'ils voyaient qu'il y avait beaucoup d'argent dans le coffre, le secrétaire du roi montait avec le grand-prêtre, et ils mettaient dans des sacs l'argent qui se trouvait dans la maison de Yahweh, puis ils le comptaient.
\VS{11}Ils remettaient cet argent bien compté entre les mains de ceux qui étaient chargés de faire exécuter l'ouvrage dans la maison de Yahweh. Et l'on employait cet argent pour les charpentiers et pour les architectes qui travaillaient à la maison de Yahweh,
\VS{12}pour les maçons et les tailleurs de pierres, pour acheter du bois et des pierres de taille, afin de réparer les brèches de la maison de Yahweh, et pour acheter tout ce qu'il fallait pour la réparation de la maison.
\VS{13}Mais, avec l'argent qu'on apportait dans la maison de Yahweh, on ne fit pour la maison de Yahweh ni bassins d'argent ni de couteaux, ni coupes, ni trompettes, ni aucun autre ustensile d'or, ou ustensile d'argent~;
\VS{14}on le distribuait à ceux qui avaient la charge de l'ouvrage et qui réparaient la maison de Yahweh.
\VS{15}On ne demandait pas de comptes aux hommes entre les mains desquels on remettait l'argent pour qu'ils le donnent à ceux qui faisaient l'ouvrage, car ils le faisaient fidèlement.
\VS{16}L'argent des sacrifices pour la culpabilité et l'argent des sacrifices pour les expiations n'était point apporté dans la maison de Yahweh~: Car il était pour les prêtres.
\TextTitle{Invasion syrienne évitée~; mort de Joas}
\VS{17}Alors Hazaël\FTNT{Hazaël envahit Juda à deux reprises. Ce passage fait mention de la première invasion~; la deuxième invasion est relatée en 2 Ch. 24:23.}, roi de Syrie, monta et fit la guerre à Gath, dont il s'empara. Hazaël avait l'intention de monter contre Jérusalem.
\VS{18}Mais Joas, roi de Juda, prit tout ce qui était consacré, que Josaphat, Joram, et Achazia, ses pères, rois de Juda, avaient consacré, tout ce que lui-même avait consacré, tout l'or qui se trouva dans les trésors de la maison de Yahweh et de la maison du roi~; et il envoya le tout à Hazaël, roi de Syrie, qui ne monta pas contre Jérusalem.
\VS{19}Le reste des actions de Joas, tout ce qu'il a fait, cela n'est-il pas écrit dans le livre des Chroniques des rois de Juda~?
\VS{20}Ses serviteurs se soulevèrent et se liguèrent~; ils frappèrent Joas dans la maison de Millo, qui est à la descente de Silla.
\VS{21}Jozacar, fils de Schimeath, et Jozabad fils de Schomer, ses serviteurs, le frappèrent, et il mourut. On l'enterra avec ses pères dans la cité de David. Et Amatsia, son fils, régna à sa place.
\Chap{13}
\TextTitle{Joachaz règne sur Israël}
\VerseOne{}La vingt-troisième année de Joas, fils d'Achazia, roi de Juda, Joachaz, fils de Jéhu, commença à régner sur Israël à Samarie. Il régna dix-sept ans.
\VS{2}Il fit ce qui est mal aux yeux de Yahweh~; car il suivit les péchés de Jéroboam, fils de Nebath, par lesquels il avait fait pécher Israël, et il ne s'en détourna point.
\TextTitle{L'idolâtrie perdure dans le pays}
\VS{3}La colère de Yahweh s'enflamma contre Israël, et il les livra entre les mains de Hazaël, roi de Syrie, et entre les mains de Ben-Hadad, fils de Hazaël, tout le temps que ces rois vécurent.
\VS{4}Mais Joachaz implora Yahweh. Et Yahweh l'exauça, parce qu'il vit l'oppression sous laquelle le roi de Syrie tenait Israël.
\VS{5}Yahweh donna donc un libérateur à Israël, et ils échappèrent aux mains des Syriens~; ainsi les enfants d'Israël habitèrent dans leurs tentes comme auparavant.
\VS{6}Mais ils ne se détournèrent point des péchés de la maison de Jéroboam, par lesquels il avait fait pécher Israël~; ils s'y livrèrent, et même l'idole d'Asherah\FTNT{Voir commentaire Jg. 2:13.} resta debout à Samarie.
\VS{7}De tout le peuple de Joachaz, Dieu ne lui avait laissé que cinquante cavaliers, dix chars, et dix mille hommes de pied~; car le roi de Syrie les avait fait périr et les avait rendus semblables à la poussière qu'on foule aux pieds.
\TextTitle{Mort de Joachaz~; Joas règne sur Israël}
\VS{8}Le reste des actions de Joachaz, tout ce qu'il a fait, et ses exploits, cela n'est-il pas écrit dans le livre des Chroniques des rois d'Israël~?
\VS{9}Ainsi Joachaz se coucha avec ses pères, et on l'ensevelit à Samarie. Et Joas, son fils, régna à sa place.
\VS{10}La trente-septième année de Joas, roi de Juda, Joas, fils de Joachaz, commença à régner sur Israël à Samarie. Il régna seize ans.
\VS{11}Et il fit ce qui est mal aux yeux de Yahweh~; il ne se détourna d'aucun des péchés de Jéroboam, fils de Nebath, par lesquels il avait fait pécher Israël, il s'y livra comme lui.
\TextTitle{Mort de Joas}
\VS{12}Le reste des actions de Joas, tout ce qu'il a fait, ses exploits, et la guerre qu'il eut avec Amatsia, roi de Juda, tout cela n'est-il pas écrit dans le livre des Chroniques des rois d'Israël~?
\VS{13}Joas se coucha avec ses pères, et Jéroboam s'assit sur son trône. Joas fut enterré à Samarie avec les rois d'Israël.
\TextTitle{Fin de la vie d'Elisée~; récit de la visite de Joas roi d'Israël}
\VS{14}Elisée était atteint de la maladie dont il mourut~; et Joas, roi d'Israël, descendit vers lui, pleura sur son visage, en disant~: Mon père~! Mon père~! Char d'Israël et sa cavalerie~!
\VS{15}Elisée lui dit~: Prends un arc et des flèches. Il prit donc un arc et des flèches.
\VS{16}Puis Elisée dit au roi d'Israël~: Bande l'arc avec ta main. Mets ta main sur l'arc. Et quand il y eut mis sa main, Elisée mit ses mains sur les mains du roi,
\VS{17}et il lui dit~: Ouvre la fenêtre à l'orient. Et il l'ouvrit. Elisée lui dit~: Tire. Après qu'il eut tiré, il lui dit~: C'est la flèche de la délivrance de la part de Yahweh, la flèche de la délivrance contre les Syriens~; tu frapperas les Syriens à Aphek, jusqu'à leur extermination.
\VS{18}Elisée lui dit encore~: Prends les flèches. Et il les prit. Elisée dit au roi d'Israël~: Frappe contre terre. Et le roi frappa trois fois, puis il s'arrêta.
\VS{19}Et l'homme de Dieu se mit dans une très grande colère contre lui, et lui dit~: Il fallait frapper cinq ou six fois~; alors tu aurais battu les Syriens jusqu'à leur extermination~; mais maintenant tu ne les frapperas que trois fois.
\TextTitle{Mort d'Elisée~; ses os rendent la vie à un mort}
\VS{20}Elisée mourut, et on l'ensevelit. L'année suivante, quelques troupes de Moabites entrèrent dans le pays.
\VS{21}Et comme on enterrait un homme, voici, on aperçut l'une des troupes de soldats, et l'on jeta l'homme dans le sépulcre d'Elisée. L'homme alla toucher les os d'Elisée, il reprit vie et se leva sur ses pieds.
\TextTitle{Fin de l'oppression syrienne}
\VS{22}Pendant toute la vie de Joachaz, Hazaël, roi de Syrie, avait opprimé Israël.
\VS{23}Mais Yahweh eut compassion d'eux, leur fit miséricorde, il tourna sa face vers eux par amour pour son alliance avec Abraham, Isaac et Jacob, de sorte qu'il ne voulut point les exterminer, et il ne les rejeta pas de sa face, jusqu'à maintenant.
\VS{24}Puis Hazaël, roi de Syrie, mourut, et Ben-Hadad, son fils, régna à sa place.
\VS{25}Joas, fils de Joachaz, reprit des mains de Ben-Hadad, fils d'Hazaël, les villes enlevées par Hazaël, à Joachaz, son père, pendant la guerre. Joas le battit trois fois et recouvra les villes d'Israël.
\Chap{14}
\TextTitle{Amatsia règne sur Juda\FTNTT{2 Ch. 25:1-4.}}
\VerseOne{}La deuxième année de Joas, fils de Joachaz, roi d'Israël, Amatsia, fils de Joas, roi de Juda, commença à régner.
\VS{2}Il était âgé de vingt-cinq ans lorsqu'il commença à régner, et il régna vingt-neuf ans à Jérusalem. Sa mère s'appelait Joaddan, elle était de Jérusalem.
\VS{3}Il fit ce qui est droit aux yeux de Yahweh, non pas toutefois comme David, son père~; il agit entièrement comme avait agi Joas, son père.
\VS{4}Seulement, les hauts lieux ne furent point ôtés~; le peuple offrait encore des sacrifices et des parfums sur les hauts lieux.
\VS{5}Et il arriva que dès que le royaume fut affermi entre ses mains, il frappa ses serviteurs qui avaient tué le roi, son père.
\VS{6}Mais il ne fit point mourir les fils des meurtriers, suivant ce qui est écrit dans le livre de la loi de Moïse, où Yahweh donne ce commandement~: On ne fera point mourir les pères pour les enfants, et l'on ne fera pas mourir les enfants pour les pères~; mais on fera mourir chacun pour son péché\FTNT{De. 24:16~; Ez. 18:4,20.}.
\VS{7}Il frappa dix mille hommes d'Edom dans la vallée du sel~; et il prit Séla durant la guerre, et l'appela Joktheel, nom qu'elle a conservé jusqu'à ce jour.
\VS{8}Alors Amatsia envoya des messagers vers Joas, fils de Joachaz, fils de Jéhu, roi d'Israël, pour lui dire~: Viens, voyons-nous en face~!
\VS{9}Et Joas, roi d'Israël, envoya dire à Amatsia, roi de Juda~: L'épine du Liban envoya dire au cèdre du Liban~: Donne ta fille en mariage à mon fils~! Et les bêtes sauvages qui sont au Liban passèrent et foulèrent l'épine.
\VS{10}Parce que tu as frappé et ravagé Edom, ton cœur s'est élevé. Contente-toi de ta gloire et reste dans ta maison. Pourquoi exciterais-tu le mal par lequel tu tomberas, toi et Juda avec toi~?
\VS{11}Mais Amatsia ne l'écouta pas. Et Joas, roi d'Israël, monta~: Et ils s'affrontèrent, lui et Amatsia, roi de Juda, à Beth-Schémesch, qui est à Juda.
\VS{12}Juda fut battu par Israël, et ils s'enfuirent chacun dans leurs tentes.
\VS{13}Joas, roi d'Israël, prit Amatsia, roi de Juda, fils de Joas, fils d'Achazia, à Beth-Schémesch. Puis il vint à Jérusalem et fit une brèche de quatre cents coudées dans la muraille de Jérusalem, depuis la porte d'Ephraïm, jusqu'à la porte de l'angle.
\VS{14}Il prit tout l'or et tout l'argent et tous les vases qui se trouvaient dans la maison de Yahweh et dans les trésors de la maison royale~; il prit aussi des enfants en otages, et il retourna à Samarie.
\TextTitle{Jéroboam II règne sur Israël}
\VS{15}Le reste des actions de Joas, ses exploits, et comment il combattit contre Amatsia, tout cela n'est-il pas écrit dans le livre des Chroniques des rois d'Israël~?
\VS{16}Et Joas se coucha avec ses pères et fut enseveli à Samarie avec les rois d'Israël. Et Jéroboam, son fils, régna à sa place.
\TextTitle{Mort d'Amatsia~; Azaria (Ozias) règne sur Juda (2 Ch. 25:26-28)}
\VS{17}Amatsia, fils de Joas, roi de Juda, vécut quinze ans après la mort de Joas, fils de Joachaz, roi d'Israël.
\VS{18}Le reste des actions d'Amatsia n'est-il pas écrit dans le livre des Chroniques des rois de Juda~?
\VS{19}On forma une conspiration contre lui à Jérusalem, et il s'enfuit à Lakis~; mais on le poursuivit à Lakis, où on le fit mourir.
\VS{20}On le transporta sur des chevaux, et il fut enseveli à Jérusalem avec ses pères, dans la cité de David.
\VS{21}Alors tout le peuple de Juda prit Azaria, âgé de seize ans, et ils l'établirent roi à la place d'Amatsia, son père.
\VS{22}Azaria bâtit Elath et la fit rentrer sous la puissance de Juda, après que le roi se coucha avec ses pères.
\TextTitle{Prophétie de Jonas accomplie par Jéroboam II}
\VS{23}La quinzième année d'Amatsia, fils de Joas, roi de Juda, Jéroboam, fils de Joas, commença à régner sur Israël à Samarie, et il régna quarante et un ans.
\VS{24}Il fit ce qui est mal aux yeux de Yahweh, et ne se détourna d'aucun des péchés de Jéroboam, fils de Nebath, par lesquels il avait fait pécher Israël.
\VS{25}Il rétablit les frontières d'Israël depuis l'entrée de Hamath, jusqu'à la mer de la plaine, selon la parole de Yahweh, le Dieu d'Israël, qu'il avait prononcée par son serviteur Jonas\FTNT{Jon. 1:1.}, fils d'Amitthaï, le prophète, de Gath-Hépher.
\VS{26}Car Yahweh vit que l'affliction d'Israël était à son comble, et l'extrémité à laquelle se trouvaient réduits esclaves et hommes libres, sans qu'il n'y ait personne pour venir au secours d'Israël.
\VS{27}Or Yahweh n'avait point résolu d'effacer le nom d'Israël de dessous les cieux, à cause de cela, il les délivra par les mains de Jéroboam, fils de Joas.
\TextTitle{Zacharie règne sur Israël}
\VS{28}Le reste des actions de Jéroboam, tout ce qu'il a fait, ses exploits de guerre, et comment il reconquit pour Israël, Damas et Hamath qui avaient appartenu à Juda, cela n'est-il pas écrit dans le livre des Chroniques des rois d'Israël~?
\VS{29}Puis Jéroboam se coucha avec ses pères, avec les rois d'Israël. Et Zacharie, son fils, régna à sa place.
\Chap{15}
\TextTitle{Juda demeure dans l'idolâtrie sous le règne d'Azaria (Ozias)\FTNTT{2 R. 14:21-22~; 2 Ch. 26:1-15.}}
\VerseOne{}La vingt-septième année de Jéroboam, roi d'Israël, Azaria\FTNT{Azaria (Ozias, selon 2 Ch. 26:1-15~; à ne pas confondre avec le prophète du même nom que son grand-père avait fait assassiner) fut couronné à l'âge de seize ans et mourut à l'âge de soixante-huit ans.}, fils d'Amatsia, roi de Juda, régna.
\VS{2}Il était âgé de seize ans lorsqu'il commença à régner, et il régna cinquante-deux ans à Jérusalem. Sa mère s'appelait Jecolia, elle était de Jérusalem.
\VS{3}Il fit ce qui est droit aux yeux de Yahweh, entièrement comme avait fait Amatsia, son père.
\VS{4}Seulement, les hauts lieux ne disparurent pas~; le peuple offrait encore des sacrifices et des parfums sur les hauts lieux.
\TextTitle{Jugement de Yahweh sur Ozias par la lèpre\FTNTT{2 Ch. 26:16-21.}}
\VS{5}Alors Yahweh frappa le roi, qui fut lépreux jusqu'au jour de sa mort, et il demeura dans une maison à l'écart. Et Jotham, fils du roi, avait la charge de la maison, jugeant le peuple du pays.
\VS{6}Le reste des actions d'Azaria, tout ce qu'il a fait, cela n'est-il pas écrit dans le livre des Chroniques des rois de Juda~?
\VS{7}Azaria se coucha avec ses pères, et fut enseveli avec ses pères dans la cité de David, et Jotham, son fils, régna à sa place.
\TextTitle{Conspiration de Schallum contre Zacharie, roi d'Israël}
\VS{8}La trente-huitième année d'Azaria, roi de Juda, Zacharie, fils de Jéroboam, commença à régner sur Israël à Samarie, et il régna six mois.
\VS{9}Il fit ce qui est mal aux yeux de Yahweh, comme avaient fait ses pères~; il ne se détourna point des péchés de Jéroboam, fils de Nebath, par lesquels il avait fait pécher Israël.
\VS{10}Schallum, fils de Jabesch, fit une conspiration contre lui, et le frappa devant le peuple. Il le tua, et régna à sa place.
\VS{11}Quant au reste des actions de Zacharie, voilà, elles sont écrites dans le livre des Chroniques des rois d'Israël.
\VS{12}Ainsi s'accomplit la parole que Yahweh avait déclarée à Jéhu, en disant~: Tes fils seront assis sur le trône d'Israël jusqu'à la quatrième génération, et il en fut ainsi\FTNT{2 R. 10:30.}
\TextTitle{Schallum règne sur Israël~; sa mort}
\VS{13}Schallum, fils de Jabesch, commença à régner la trente-neuvième année d'Ozias, roi de Juda. Il régna pendant un mois à Samarie.
\VS{14}Menahem, fils de Gadi, monta de Thirtsa et vint dans Samarie, et frappa à Samarie, Schallum, fils de Jabesch, et le fit mourir~; et il régna à sa place.
\VS{15}Le reste des actions de Schallum, et la conspiration qu'il forma, cela est écrit dans le livre des Chroniques des rois d'Israël.
\TextTitle{Menahem règne sur Israël}
\VS{16}Alors Menahem frappa Thiphsach et tous ceux qui y étaient, avec son territoire depuis Thirtsa~; il la frappa parce qu'elle ne lui avait point ouvert ses portes. Il fendit le ventre de toutes les femmes enceintes.
\VS{17}La trente-neuvième année d'Azaria, roi de Juda, Menahem, fils de Gadi, commença à régner sur Israël. Il régna dix ans à Samarie.
\VS{18}Il fit ce qui est mal aux yeux de Yahweh~; il ne se détourna point des péchés de Jéroboam, fils de Nebath, par lesquels il avait fait pécher Israël.
\TextTitle{Invasion d'Israël par le roi d'Assyrie\FTNTT{1 Ch. 5:26.}}
\VS{19}Alors Pul, roi d'Assyrie, vint contre le pays~; et Menahem donna mille talents d'argent à Pul, afin qu'il l'aide à affermir son royaume entre ses mains.
\VS{20}Menahem leva cet argent sur tous ceux d'Israël qui avaient de la richesse pour le donner au roi d'Assyrie~; chacun cinquante sicles d'argent. Ainsi, le roi d'Assyrie s'en retourna, et ne s'arrêta point dans le pays.
\VS{21}Le reste des actions de Menahem, tout ce qu'il a fait, cela n'est-il pas écrit dans le livre des Chroniques des rois d'Israël~?
\TextTitle{Mort de Menahem~; Pekachia règne sur Israël}
\VS{22}Menahem se coucha avec ses pères, et Pekachia, son fils, régna à sa place.
\VS{23}La cinquantième année d'Azaria, roi de Juda, Pekachia, fils de Menahem, commença à régner sur Israël à Samarie. Il régna deux ans.
\VS{24}Il fit ce qui est mal aux yeux de Yahweh~; il ne se détourna point des péchés de Jéroboam, fils de Nebath, par lesquels il avait fait pécher Israël.
\TextTitle{Pékach tue Pekachia et devient roi d'Israël}
\VS{25}Pékach, fils de Remalia, son officier, conspira contre lui~; il le frappa à Samarie, dans le palais de la maison royale, de même qu'Argob et Arié~; il avait avec lui cinquante hommes d'entre les fils des Galaadites. Il fit ainsi mourir Pekachia, et il régna à sa place.
\VS{26}Le reste des actions de Pekachia tout ce qu'il a fait, cela est écrit dans le livre des Chroniques des rois d'Israël.
\VS{27}La cinquante-deuxième année d'Azaria, roi de Juda, Pékach, fils de Remalia, commença à régner sur Israël à Samarie. Il régna vingt ans.
\VS{28}Il fit ce qui est mal aux yeux de Yahweh et ne se détourna point des péchés de Jéroboam, fils de Nebath, par lesquels il avait fait pécher Israël.
\VS{29}Du temps de Pékach, roi d'Israël, Tiglath-Piléser, roi d'Assyrie, vint et prit Ijjon, Abel-Beth-Maaca, Janoach, Kédesch, Hatsor, Galaad et la Galilée, et même tout le pays de Nephthali, et il emmena captifs les habitants en Assyrie.
\TextTitle{Osée conspire contre Pékach et règne sur Israël}
\VS{30}Osée, fils d'Ela, forma une conspiration contre Pékach, fils de Remalia, le frappa et le fit mourir. Il régna à sa place la vingtième année de Jotham, fils d'Ozias.
\VS{31}Le reste des actions de Pékach, tout ce qu'il a fait, cela est écrit dans le livre des Chroniques des rois d'Israël.
\TextTitle{Jotham règne sur Juda~; sa mort\FTNTT{2 R. 15:2~; 2 Ch. 26:23~; 27:1-9.}}
\VS{32}La seconde année de Pékach, fils de Remalia, roi d'Israël, Jotham, fils d'Ozias, roi de Juda, commença à régner.
\VS{33}Il était âgé de vingt-cinq ans lorsqu'il commença à régner. Il régna seize ans à Jérusalem. Sa mère s'appelait Jeruscha, fille de Tsadok.
\VS{34}Il fit ce qui est droit aux yeux de Yahweh~; il agit entièrement comme avait agi Ozias, son père.
\VS{35}Seulement, les hauts lieux ne disparurent point~; et le peuple offrait encore des sacrifices et des parfums sur les hauts lieux. Jotham bâtit la porte supérieure de la maison de Yahweh.
\VS{36}Le reste des actions de Jotham, tout ce qu'il a fait, cela n'est-il pas écrit dans le livre des Chroniques des rois de Juda~?
\VS{37}Dans ce temps-là, Yahweh commença à envoyer contre Juda, Retsin, roi de Syrie, et Pékach, fils de Remalia.
\VS{38}Jotham se coucha avec ses pères, et il fut enseveli dans la cité de David, son père. Et Achaz, son fils, régna à sa place.
\Chap{16}
\TextTitle{Achaz règne sur Juda\FTNTT{2 R. 15:38~; 2 Ch. 28:1-4.}}
\VerseOne{}La dix-septième année de Pékach, fils de Remalia, Achaz, fils de Jotham, roi de Juda, commença à régner.
\VS{2}Achaz était âgé de vingt ans lorsqu'il commença à régner. Il régna seize ans à Jérusalem. Il ne fit point ce qui est droit aux yeux de Yahweh, son Dieu, comme avait fait David, son père.
\VS{3}Mais il suivit la voie des rois d'Israël et il fit même passer son fils par le feu, selon les abominations des nations que Yahweh avait chassées devant les enfants d'Israël.
\VS{4}Il offrait aussi des sacrifices et des parfums sur les hauts lieux, sur les coteaux et sous tout arbre vert.
\TextTitle{Juda envahi par les rois d'Assyrie et d'Israël\FTNTT{2 Ch. 28:5-19.}}
\VS{5}Alors Retsin, roi de Syrie, et Pékach, fils de Remalia, roi d'Israël, montèrent contre Jérusalem pour lui faire la guerre. Ils assiégèrent Achaz~; mais ne purent en venir à bout par les armes.
\VS{6}Dans ce même temps, Retsin, roi de Syrie, fit rentrer Elath au pouvoir des Syriens~; il expulsa les Juifs d'Elath, et les Syriens vinrent à Elath, où ils ont demeuré jusqu'à ce jour.
\TextTitle{Le roi d'Assyrie vient en aide à Achaz et s'empare de Damas\FTNTT{2 Ch. 28:16-25.}}
\VS{7}Achaz envoya des messagers à Tiglath-Piléser, roi d'Assyrie, pour lui dire~: Je suis ton serviteur et ton fils~; monte et délivre-moi de la main du roi des Syriens, et de la main du roi d'Israël, qui s'élèvent contre moi.
\VS{8}Alors Achaz prit l'argent et l'or qui se trouvaient dans la maison de Yahweh, et dans les trésors de la maison royale, et il les envoya en présent au roi d'Assyrie.
\VS{9}Le roi d'Assyrie l'écouta~; il monta contre Damas, la prit, emmena les habitants en captivité à Kir et fit mourir Retsin.
\VS{10}Alors le roi Achaz s'en alla à la rencontre de Tiglath-Piléser, roi d'Assyrie, à Damas. Et ayant vu l'autel\FTNT{Achaz, roi de Juda, se rendit chez le roi d'Assyrie et il fut fasciné par l'autel de son dieu au point de le convoiter. Il demanda au prêtre Urie de fabriquer un autel identique, dont le modèle n'était pas celui que Yahweh avait décrit à Moïse. Il introduisit un objet de culte d'origine païenne dans le temple de Jérusalem, sous prétexte d'honorer Yahweh. Certains «~Pères de l'Eglise~», comme les empereurs Constantin I (285-337) et Théodose I (347-395), se sont comportés exactement comme Achaz en adoptant les pratiques païennes. Les historiens s'accordent pour dire que la diffusion de la Parole de Dieu sous l'empire de Constantin I (285-337), empereur de Rome, avait des fins strictement politiques. Cette politique a eu deux conséquences essentielles concernant l'influence de l'Eglise chrétienne et son fonctionnement de plus en plus éloigné de la Parole de Dieu~:\\- Les peuples païens ont introduit leurs rites idolâtres au sein de l'Eglise. En effet, les dogmes de l'institution devaient plaire à la majorité.\\- L'Eglise chrétienne cessant d'être persécutée, son fonctionnement intimiste fondé sur l'implication de chaque croyant et l'exercice de la prêtrise universelle des chrétiens, a changé à cause de l'effet de masse. Devenant numériquement très importante, il a fallu imposer une autorité capable de contenir un nombre de fidèles de plus en plus élevé. Mais à cause de cette augmentation numérique et de la présence de «~faux convertis~» lié au fait que l'adhésion au christianisme (religion chrétienne fondée par les hommes) devenait une obligation, l'étude de la Parole, la fraction du pain et la prière ne pouvaient plus perdurer. C'est ainsi que beaucoup d'églises ont commencé à subir l'influence du monde.} qui était à Damas, le roi Achaz envoya au prêtre Urie, la forme et le modèle exact de cet autel.
\VS{11}Le prêtre Urie construisit un autel entièrement d'après le modèle envoyé de Damas par le roi Achaz, et le prêtre Urie le fit avant que le roi Achaz soit de retour de Damas.
\VS{12}Quand le roi Achaz revint de Damas et vit l'autel, il s'en approcha et y monta~;
\VS{13}Il fit brûler son holocauste et son sacrifice, versa ses libations et répandit sur l'autel le sang de ses sacrifices d'offrande de paix\FTNT{Voir commentaire en Lé. 3:1.}.
\VS{14}Il éloigna de la face de la maison l'autel d'airain qui était devant Yahweh, afin qu'il ne soit pas entre le nouvel autel et la maison de Yahweh~; et il le plaça à côté du nouvel autel, vers le nord.
\VS{15}Et le roi Achaz donna cet ordre au prêtre Urie~: Fais brûler l'holocauste du matin et l'offrande du soir, l'holocauste du roi et son offrande, les holocaustes de tout le peuple du pays et leurs offrandes, verses-y leurs libations, et répands-y tout le sang des holocaustes et tout le sang des sacrifices~; mais pour ce qui concerne l'autel d'airain, je m'en occuperai.
\VS{16}Le prêtre Urie, exécuta tout ce que le roi Achaz lui avait ordonné.
\VS{17}Le roi Achaz brisa les panneaux des bases et en ôta les cuves qui étaient dessus. Il descendit la mer de dessus les bœufs d'airain qui étaient sous elle et il la posa sur un pavé de pierre.
\VS{18}Il changea aussi dans la maison de Yahweh, à cause du roi d'Assyrie, le portique du sabbat qu'on y avait bâti et l'entrée extérieure du roi.
\TextTitle{Mort d'Achaz~; Ezéchias devient roi de Juda\FTNTT{2 Ch. 28:26-27.}}
\VS{19}Le reste des actions d'Achaz et tout ce qu'il a fait, cela n'est-il pas écrit dans le livre des Chroniques des rois de Juda~?
\VS{20}Achaz se coucha avec ses pères, et il fut enseveli avec ses pères dans la cité de David. Et Ezéchias, son fils, régna à sa place.
\Chap{17}
\TextTitle{Osée devient le dernier roi d'Israël}
\VerseOne{}La douzième année d'Achaz, roi de Juda, Osée, fils d'Ela, régna à Samarie sur Israël. Il régna neuf ans.
\VS{2}Il fit ce qui est mal aux yeux de Yahweh, non pas toutefois comme les rois d'Israël qui avaient été avant lui.
\TextTitle{Osée tente de s'affranchir du joug de l'Assyrie}
\VS{3}Salmanasar\FTNT{Le royaume d'Israël a été détruit en 722 av. J.-C., par l'empereur assyrien Salmanasar V (règne~: 727-722 av. J.-C.), après avoir assiégé trois ans le roi Osée (règne~: 732-722 av. J.-C.) dans sa capitale Samarie. Celui-ci ne payait plus le tribut et essayait d'obtenir l'appui de l'Egypte pour retrouver l'indépendance. Le royaume d'Israël a disparu au début du 8ème siècle av. J.-C., provoquant la dispersion dans le monde de plusieurs juifs issus des dix tribus. L'origine des Samaritains remonte à cette déportation, après que le royaume du Nord soit tombé aux mains de Salmanasar, roi d'Assyrie. Malgré les déportations, les Assyriens n'avaient pas laissé déserte cette région appelée «~Samarie~»~; plusieurs Israélites y étaient restés et des colons d'autres provinces assyriennes vinrent s'y établir. Les Samaritains sont issus du mélange de ces populations, et leur religion est un mélange entre le culte à Yahweh avec celui des dieux étrangers.}, roi d'Assyrie, monta contre lui~; et Osée lui fut assujetti et lui paya un tribut.
\VS{4}Mais le roi d'Assyrie découvrit une conspiration chez Osée, qui avait envoyé des messagers vers So, roi d'Egypte, et qui ne payait plus le tribut tous les ans au roi d'Assyrie. C'est pourquoi le roi d'Assyrie le fit enfermer et enchaîner dans une prison.
\TextTitle{Siège de Samarie par le roi d'Assyrie}
\VS{5}Le roi d'Assyrie parcourut tout le pays et monta contre Samarie qu'il assiégea pendant trois ans.
\TextTitle{Les causes de la captivité d’Israël par l'Assyrie}
\VS{6}La neuvième année d'Osée, le roi d'Assyrie prit Samarie et emmena captifs les Israélites en Assyrie. Il les fit habiter à Chalach, et sur le Chabor, fleuve de Gozan, et dans les villes des Mèdes.
\VS{7}Cela arriva parce que les enfants d'Israël péchèrent contre Yahweh, leur Dieu, qui les avait fait monter hors du pays d'Egypte, de dessous la main de Pharaon, roi d'Egypte, et parce qu'ils craignirent d'autres dieux.
\VS{8}Ils suivirent les coutumes des nations que Yahweh avait chassées devant les enfants d'Israël, et celles des rois d'Israël qu'ils avaient établis.
\VS{9}Les enfants d'Israël firent en secret des choses qui n'étaient point droites, contre Yahweh, leur Dieu. Ils se bâtirent des hauts lieux dans toutes leurs villes, depuis la tour des gardes jusqu'aux villes fortes.
\VS{10}Ils se dressèrent des statues et des Asherah sur toutes les hautes collines et sous tout arbre vert.
\VS{11}Et là, ils brûlèrent des parfums sur tous les hauts lieux, comme les nations que Yahweh avait chassées devant eux, et ils firent des choses mauvaises pour irriter Yahweh.
\VS{12}Ils servirent les idoles, au sujet desquelles Yahweh leur avait dit~: Vous ne ferez pas cela\FTNT{1 R. 12:28.}.
\VS{13}Yahweh fit avertir Israël et Juda par tous ses prophètes, tous les voyants, en disant~: Détournez-vous de toutes vos mauvaises voies, revenez, et gardez mes commandements et mes ordonnances, en suivant entièrement la loi que j'ai prescrite à vos pères et que je vous ai envoyée par mes serviteurs les prophètes.
\VS{14}Mais ils n'écoutèrent point et raidirent leur cou, comme leurs pères avaient raidi leur cou, et n'avaient pas cru en Yahweh, leur Dieu.
\VS{15}Ils rejetèrent ses lois, et son alliance qu'il avait traitée avec leurs pères, et ses avertissements, qu'il leur avait adressés. Ils allèrent après des choses de néant et ne furent eux-mêmes que néant, après les nations qui les entouraient et que Yahweh leur avait défendu d'imiter.
\VS{16}Ils abandonnèrent tous les commandements de Yahweh, leur Dieu, ils firent deux veaux en métal fondu, ils fabriquèrent des idoles d'Asherah\FTNT{Voir commentaire en Jg. 2:13.}, ils se prosternèrent devant toute l'armée des cieux et ils servirent Baal.
\VS{17}Ils firent aussi passer leurs fils et leurs filles par le feu, ils s'adonnèrent à la divination et aux enchantements, et ils se vendirent pour faire ce qui est mal aux yeux de Yahweh afin de l'irriter.
\VS{18}C'est pourquoi, Yahweh fut très irrité contre Israël et il les rejeta~; il n'est resté que la seule tribu de Juda.
\VS{19}Même Juda n'avait pas gardé les commandements de Yahweh, son Dieu, mais ils avaient suivi les ordonnances qu'Israël avait établies.
\VS{20}C'est pourquoi Yahweh rejeta toute la race d'Israël~; il les a humiliés, il les a livrés entre les mains des pillards, il a fini par les chasser loin de sa face.
\VS{21}Car Israël s'était détaché de la maison de David, et avait établi roi Jéroboam, fils de Nebath. Jéroboam avait détourné Israël de Yahweh, afin qu'il ne le suive plus, et lui avait fait commettre un grand péché.
\VS{22}C'est pourquoi les enfants d'Israël s'étaient livrés à tous les péchés que Jéroboam avait commis~; ils ne s'en détournèrent pas,
\VS{23}jusqu'à ce que Yahweh ait chassé Israël de devant sa face, comme il l'avait annoncé par tous ses serviteurs les prophètes. Et Israël fut emmené captif loin de son pays en Assyrie, jusqu'à ce jour.
\TextTitle{Jugement sur les étrangers occupant les villes d'Israël}
\VS{24}Le roi d'Assyrie fit venir des gens de Babylone, de Cutha, d'Avva, de Hamath et de Sepharvaïm. Il les fit habiter dans les villes de Samarie, à la place des enfants d'Israël. Ils prirent possession de la Samarie et habitèrent dans ses villes.
\VS{25}Lorsqu'ils commencèrent à y habiter, ils ne craignirent point Yahweh, et Yahweh envoya contre eux des lions, qui les tuaient.
\VS{26}Et on dit au roi d'Assyrie~: Les nations que tu as transportées et fait habiter dans les villes de Samarie ne connaissent pas la manière de servir le dieu du pays, c'est pourquoi il a envoyé contre eux des lions, et voilà, ces lions les tuent, parce qu'ils ne connaissent pas la manière de servir le dieu du pays.
\TextTitle{L'idolâtrie dans les villes occupées}
\VS{27}Alors le roi d'Assyrie donna cet ordre, en disant~: Faites-y aller quelqu'un des prêtres que vous avez emmenés de là en captivité~; qu'il parte pour s'y établir et qu'il leur enseigne la manière de servir le dieu du pays.
\VS{28}Alors l'un des prêtres, qui avaient été emmenés captifs de Samarie vint s'établir à Béthel et leur enseigna comment ils devaient craindre Yahweh.
\VS{29}Mais les nations firent chacune leurs dieux dans les villes qu'elles habitaient et les placèrent dans les maisons des hauts lieux bâties par les Samaritains.
\VS{30}Les gens de Babylone firent Succoth-Benoth, les gens de Cuth firent Nergal, et les gens de Hamath firent Aschima.
\VS{31}Ceux d'Avva firent Nibchaz et Thartak~; ceux de Sepharvaïm brûlaient leurs enfants par le feu à Adrammélec et Anammélec, les dieux de Sepharvaïm.
\VS{32}Toutefois, ils redoutaient Yahweh et ils établirent des prêtres des hauts lieux pris parmi tout le peuple~; ces prêtres offraient pour eux des sacrifices dans les maisons des hauts lieux.
\VS{33}Ils redoutaient Yahweh et en même temps, ils servaient leurs dieux à la manière des nations d'où on les avait transportés.
\VS{34}Et jusqu'à ce jour, ils font encore selon leurs premières coutumes~: Ils ne craignirent point Yahweh, et ils ne se conforment ni à leurs lois et à leurs ordonnances, ni à la loi et aux commandements prescrits par Yahweh Dieu aux enfants de Jacob, qu'il appela du nom d'Israël.
\VS{35}Yahweh avait traité alliance avec eux et leur avait donné cet ordre, en disant~: Vous ne craindrez point d'autres dieux~; et ne vous prosternerez point devant eux~; vous ne les servirez point et vous ne leur offrirez point de sacrifices.
\VS{36}Mais vous craindrez Yahweh, qui vous a fait monter hors du pays d'Egypte avec une grande puissance et à bras étendu~; et vous vous prosternerez devant lui, et vous lui offrirez des sacrifices.
\VS{37}Vous observerez et mettrez toujours en pratique les statuts, les ordonnances, la loi et les commandements, qu'il a écrits pour vous, et vous ne craindrez pas d'autres dieux.
\VS{38}Vous n'oublierez pas l'alliance que j'ai traitée avec vous et vous ne craindrez point d'autres dieux.
\VS{39}Mais vous craindrez Yahweh, votre Dieu, et il vous délivrera de la main de tous vos ennemis.
\VS{40}Ils n'écoutèrent pas et ils firent selon leurs premières coutumes.
\VS{41}Ainsi ces nations-là redoutaient Yahweh et servaient leurs images~; leurs enfants et les enfants de leurs enfants font jusqu'à ce jour ce que firent leurs pères.
\Chap{18}
\TextTitle{Ezéchias règne sur Juda\FTNTT{2 R. 16:20~; 2 Ch. 29:1-31:21.}}
\VerseOne{}La troisième année d'Osée, fils d'Ela, roi d'Israël, Ezéchias, fils d'Achaz, roi de Juda, commença à régner.
\VS{2}Il était âgé de vingt-cinq ans lorsqu'il commença à régner, il régna vingt-neuf ans à Jérusalem. Sa mère s'appelait Abi, fille de Zacharie.
\VS{3}Il fit ce qui est droit aux yeux de Yahweh, entièrement comme avait fait David, son père.
\TextTitle{Mouvement de réveil sous Ezéchias\FTNTT{2 Ch. 29:3-31:21.}}
\VS{4}Il fit disparaître les hauts lieux, mit en pièces les statues, abattit les idoles d'Asherah, et il brisa le serpent d'airain que Moïse avait fait, car les enfants d'Israël avaient jusqu'alors brûlé des parfums devant lui~; ils l'appelaient Nehuschtan.
\VS{5}Il se confia en Yahweh, le Dieu d'Israël~; et parmi tous les rois de Juda qui vinrent après ou qui le précédèrent, il n'y en eut point de semblable à lui.
\VS{6}Il s'attacha à Yahweh, il ne se détourna point de lui et il observa les commandements que Yahweh avait prescrits à Moïse.
\TextTitle{Révolte contre l'Assyrie~; victoire sur les Philistins}
\VS{7}Et Yahweh fut avec Ezéchias, qui réussit dans toutes ses entreprises. Il se révolta contre le roi d'Assyrie et ne lui fut plus assujetti.
\VS{8}Il frappa les Philistins jusqu'à Gaza et ravagea leur territoire depuis les tours des gardes jusqu'aux villes fortes.
\TextTitle{Captivité d'Israël par l'Assyrie\FTNTT{2 R. 17:4-6.}}
\VS{9}La quatrième année du roi Ezéchias, qui était la septième du règne d'Osée, fils d'Ela, roi d'Israël, Salmanasa, roi d'Assyrie, monta contre Samarie et l'assiégea.
\VS{10}Il la prit au bout de trois ans~; la sixième année du règne d'Ezéchias, qui était la neuvième d'Osée, roi d'Israël, Samarie fut prise.
\VS{11}Le roi d'Assyrie emmena Israël en Assyrie et il les établit à Chalach, sur le Chabor, fleuve de Gozan, et dans les villes des Mèdes,
\VS{12}parce qu'ils n'avaient point obéi à la voix de Yahweh, leur Dieu, et qu'ils avaient transgressé son alliance, parce qu'ils n'avaient ni écouté ni mis en pratique tout ce qu'avait ordonné Moïse, serviteur de Yahweh.
\TextTitle{Invasion de Juda par Sanchérib\FTNTT{2 Ch. 32:1-15,30~; Es. 36:1-10.}}
\VS{13}La quatorzième année du roi Ezéchias, Sanchérib, roi d'Assyrie, monta contre toutes les villes fortes de Juda et les prit.
\VS{14}Ezéchias, roi de Juda, envoya dire au roi d'Assyrie à Lakis~: J'ai commis une faute~! Eloigne-toi de moi. Je payerai tout ce que tu m'imposeras. Et le roi d'Assyrie imposa à Ezéchias, roi de Juda, trois cents talents d'argent et trente talents d'or.
\VS{15}Ezéchias donna tout l'argent qui se trouvait dans la maison de Yahweh et dans les trésors de la maison royale.
\VS{16}En ce temps-là, Ezéchias enleva les lames d'or dont il avait couvert les portes et les linteaux du temple de Yahweh, pour les livrer au roi d'Assyrie.
\VS{17}Puis le roi d'Assyrie envoya de Lakis à Jérusalem, vers le roi Ezéchias, Tharthan, Rab-Saris et Rabschaké avec une puissante armée. Ils montèrent et arrivèrent à Jérusalem. Lorsqu'ils furent montés et arrivés~; ils s'arrêtèrent à l'aqueduc de l'étang supérieur, qui est sur le chemin du champ du foulon.
\VS{18}Ils appelèrent le roi tout haut~; alors Eliakim, fils de Hilkija, chef de la maison du roi, Schebna, le secrétaire et Joach, fils d'Asaph, l'archiviste, se rendirent auprès d'eux.
\VS{19}Rabschaké leur dit~: Dites maintenant à Ezéchias~: Ainsi parle le grand roi, le roi d'Assyrie~: Quelle est cette confiance sur laquelle tu t'appuies~?
\VS{20}Tu as dit~: Il faut pour la guerre le conseil et la force. Mais ce ne sont que des paroles. Mais en qui donc as-tu placé ta confiance, pour te rebeller contre moi~?
\VS{21}Voici maintenant, tu l'as placée dans l'Egypte, dans ce roseau cassé, qui pénètre et perce la main de quiconque s'appuie dessus~: Tel est Pharaon, roi d'Egypte, pour tous ceux qui se confient en lui.
\VS{22}Peut-être me direz-vous~: Nous nous confions en Yahweh, notre Dieu, mais n'est-ce pas celui dont Ezéchias a détruit les hauts lieux et les autels, en disant à Juda et à Jérusalem~: Vous vous prosternerez devant cet autel à Jérusalem~?
\VS{23}Maintenant, donne des otages au roi d'Assyrie, mon maître, et je te donnerai deux mille chevaux, si tu peux donner autant de cavaliers pour les monter.
\VS{24}Comment donc repousserais-tu un seul gouverneur d'entre les serviteurs de mon maître~? Mais tu mets ta confiance dans l'Egypte, à cause des chars et des cavaliers.
\VS{25}D'ailleurs, est-ce sans l'ordre de Yahweh que je suis monté contre ce lieu, pour le détruire~? Yahweh m'a dit~: Monte contre ce pays et détruis-le.
\TextTitle{Menaces de Rabschaké\FTNTT{2 Ch. 32:16,18-19~; Es. 36:11-21.}}
\VS{26}Alors Eliakim, fils de Hilkija, Schebna et Joach dirent à Rabschaké~: Nous te prions de parler en araméen à tes serviteurs, car nous le comprenons~; et ne nous parle pas en langue judaïque, aux oreilles du peuple qui est sur la muraille.
\VS{27}Rabschaké leur répondit~: Est-ce à ton maître et à toi que mon maître m'a envoyé dire ces paroles~? Ne m'a-t-il pas envoyé vers les hommes qui se tiennent sur la muraille pour leur dire qu'ils mangeront leurs propres excréments et qu'ils boiront leur urine avec vous~?
\VS{28}Rabschaké, s'étant avancé, cria à haute voix en langue judaïque, il parla et dit~: Ecoutez la parole du grand roi, le roi d'Assyrie.
\VS{29}Ainsi parle le roi~: Qu'Ezéchias ne vous trompe pas, car il ne pourra pas vous délivrer de ma main.
\VS{30}Qu'Ezéchias ne vous amène pas à vous confier en Yahweh, en disant~: Yahweh nous délivrera certainement et cette ville ne sera pas livrée entre les mains du roi d'Assyrie.
\VS{31}N'écoutez pas Ezéchias~; car ainsi parle le roi d'Assyrie~: Faites la paix avec moi et rendez-vous à moi~; et chacun de vous mangera de sa vigne, et de son figuier, et chacun boira de l'eau de sa citerne,
\VS{32}jusqu'à ce que je vienne et que je vous emmène dans un pays comme le vôtre, dans un pays de blé et de bon vin, un pays de pain et de vignes, un pays d'oliviers qui portent de l'huile, et de miel, et vous vivrez et vous ne mourrez pas. Mais n'écoutez pas Ezéchias~; car il pourrait vous séduire, en disant~: Yahweh nous délivrera.
\VS{33}Les dieux des nations ont-ils délivré chacun leur pays de la main du roi d'Assyrie~?
\VS{34}Où sont les dieux de Hamath et d'Arpad~? Où sont les dieux de Sépharvaïm, d'Héna et d'Ivva~? Et même ont-ils délivré Samarie de ma main~?
\VS{35}Parmi tous les dieux de ces pays, quels sont ceux qui ont délivré leur pays de ma main, pour dire que Yahweh délivrera Jérusalem de ma main~?
\VS{36}Le peuple se tut et on ne lui répondit pas un mot~; car le roi avait donné cet ordre~: Vous ne lui répondrez point.
\VS{37}Après cela, Eliakim, fils de Hilkija, chef de la maison du roi, et Schebna le secrétaire, et Joach, fils d'Asaph, l'archiviste, vinrent auprès d'Ezéchias, les vêtements déchirés, et ils lui rapportèrent les paroles de Rabschaké.
\Chap{19}
\TextTitle{Ezéchias demande à Esaïe de consulter Yahweh\FTNTT{2 Ch. 32:20-22~; Es. 36:22-37:5.}}
\VerseOne{}Et il arriva qu'aussitôt que le roi Ezéchias entendit ces choses, il déchira ses vêtements, se couvrit d'un sac et entra dans la maison de Yahweh.
\VS{2}Puis il envoya Eliakim, chef de la maison du roi, et Schebna le secrétaire, et les plus anciens des prêtres, couverts de sacs, vers Esaïe, le prophète, fils d'Amots.
\VS{3}Ils lui dirent~: Ainsi parle Ezéchias~: Ce jour est un jour d'angoisse, de châtiment et d'opprobre~; car les enfants sont près du sein maternel, mais il n'y a point de force pour enfanter.
\VS{4}Peut-être Yahweh, ton Dieu, a-t-il entendu toutes les paroles de Rabschaké, que le roi d'Assyrie, son maître, a envoyé pour blasphémer le Dieu vivant, et peut-être Yahweh, ton Dieu, exercera-t-il ses châtiments à cause des paroles qu'il a entendues. Fais donc une prière pour le reste qui subsiste encore.
\VS{5}Les serviteurs du roi Ezéchias vinrent donc vers Esaïe.
\TextTitle{Réponse de Yahweh\FTNTT{Es. 37:6-7.}}
\VS{6}Et Esaïe leur dit~: Voici ce que vous direz à votre maître~: Ainsi parle Yahweh~: Ne t'effraie point des paroles que tu as entendues, par lesquelles les serviteurs du roi d'Assyrie m'ont blasphémé.
\VS{7}Voici, je vais mettre en lui un esprit, tel que sur une nouvelle qu'il recevra, il retournera dans son pays~; et je le ferai tomber par l'épée dans son pays.
\TextTitle{Défi du roi d'Assyrie au Dieu d'Israël\FTNTT{2 Ch. 32:17~; Es. 37:8-13.}}
\VS{8}Rabschaké s'étant retiré, trouva le roi d'Assyrie qui attaquait Libna, car il avait appris qu'il était parti de Lakis.
\VS{9}Le roi d'Assyrie reçut une nouvelle au sujet de Tirhaka, roi d'Ethiopie~; on lui dit~: Voici, il est sorti pour te combattre. C'est pourquoi le roi d'Assyrie retourna dans son pays, mais il envoya des messagers à Ezéchias, en leur disant~:
\VS{10}Vous parlerez ainsi à Ezéchias, roi de Juda, et lui direz~: Que ton Dieu, en qui tu te confies, ne t'abuse pas en te disant~: Jérusalem ne sera point livrée entre les mains du roi d'Assyrie.
\VS{11}Voici, tu as entendu ce que les rois d'Assyrie ont fait à tous les pays, et comment ils les ont détruits entièrement~; et tu échapperais~?
\VS{12}Les dieux des nations que mes ancêtres ont détruites, savoir de Gozan, de Charan, de Retseph, et des fils d'Eden, qui sont en Telassar, les ont-ils délivrées~?
\VS{13}Où sont le roi de Hamath, le roi d'Arpad, et le roi de la ville de Sepharvaïm, d'Héna et d'Ivva~?
\TextTitle{Ezéchias dans le temple, sa prière à Yawheh\FTNTT{2 Ch. 32:20~; Es. 37:14-20.}}
\VS{14}Quand Ezéchias reçut la lettre de la main des messagers, il la lut. Puis il monta à la maison de Yahweh et la déploya devant Yahweh~;
\VS{15}puis Ezéchias lui adressa cette prière et dit~: Ô Yahweh, Dieu d'Israël~! Qui est assis entre les chérubins, c'est toi qui es le seul Dieu de tous les royaumes de la terre, c'est toi qui as fait les cieux et la terre.
\VS{16}Ô Yahweh~! Incline ton oreille et écoute. Ouvre tes yeux et regarde. Ecoute les paroles de Sanchérib, et de celui qu’il a envoyé pour blasphémer le Dieu vivant.
\VS{17}Il est vrai, ô Yahweh~! Que les rois d'Assyrie ont détruit ces nations et ravagé leurs pays,
\VS{18}et qu'ils ont jeté dans le feu leurs dieux~; mais ils n'étaient pas des dieux, mais des ouvrages de mains d'homme, du bois, et de la pierre, c'est pourquoi ils les ont détruits.
\VS{19}Maintenant donc, ô Yahweh, notre Dieu~! Je te prie, délivre-nous de la main de Sanchérib, afin que tous les royaumes de la terre sachent que c'est toi, ô Yahweh, qui es le seul Dieu.
\TextTitle{Yahweh répond au travers d'Esaïe\FTNTT{Es. 37:21-35.}}
\VS{20}Alors Esaïe, fils d'Amots, envoya dire à Ezéchias~: Ainsi parle Yahweh, le Dieu d'Israël~: Je t'ai exaucé dans ce que tu m'as demandé au sujet de Sanchérib, roi d'Assyrie.
\VS{21}Voici la parole que Yahweh a prononcée contre lui~: Elle te méprise, elle se moque de toi, la fille, vierge de Sion~; elle hoche la tête après toi, la fille de Jérusalem.
\VS{22}Qui as-tu outragé et blasphémé~? Contre qui as-tu élevé la voix~? Tu as porté tes yeux en haut, vers le Saint d'Israël~!
\VS{23}Tu as insulté le Seigneur par le moyen de tes messagers et tu as dit~: J'ai gravi le sommet des montagnes avec la multitude de mes chars, les extrémités du Liban~; je couperai les plus hauts de ses cèdres et les plus beaux de ses cyprès, et j'atteindrai sa dernière cime, la forêt de son verger.
\VS{24}J'ai creusé des sources, après avoir bu les eaux étrangères et je tarirai avec la plante de mes pieds tous les fleuves de l'Egypte.
\VS{25}N'as-tu pas appris que j'ai préparé cette ville déjà dès longtemps, et que dès les temps anciens je l'ai ainsi formée~? Et maintenant l'aurais-je conservée pour être réduite en désolation, et les villes fortes, en monceaux de ruines~?
\VS{26}Il est vrai que leurs habitants sont impuissants, épouvantés et confus~; ils sont devenus comme l'herbe des champs et la tendre verdure, comme le gazon des toits et le blé brûlé avant la formation de sa tige.
\VS{27}Mais je connais ta demeure, ta sortie et ton entrée, et comment tu es furieux contre moi.
\VS{28}Parce que tu es furieux contre moi et que ton insolence est montée à mes oreilles, je mettrai ma boucle à tes narines, et mon mors entre tes lèvres, et je te ferai retourner par le chemin par lequel tu es venu.
\VS{29}Que ceci soit un signe pour toi, ô Ezéchias~: On mangera cette année le produit du grain tombé, et la deuxième année, ce qui croît de soi-même~; mais la troisième année, vous sèmerez et vous moissonnerez, vous planterez des vignes et vous en mangerez le fruit.
\VS{30}Ce qui aura été épargné de la maison de Juda, ce qui sera resté poussera encore des racines par-dessous et produira du fruit par-dessus.
\VS{31}Car il sortira de Jérusalem un reste, et de la montagne de Sion des réchappés. Voilà ce que fera le zèle de Yahweh des armées.
\VS{32}C'est pourquoi ainsi parle Yahweh, sur le roi d'Assyrie~: Il n'entrera point dans cette ville, il n'y lancera aucune flèche, il ne se présentera point contre elle avec le bouclier et il n'élèvera point des retranchements contre elle.
\VS{33}Il s'en retournera par le chemin par lequel il est venu et il n'entrera point dans cette ville, dit Yahweh.
\VS{34}Car je protègerai cette ville, afin de la délivrer, par amour pour moi et par amour pour David, mon serviteur.
\TextTitle{L'ange de Yahweh dans le camp des Assyriens\FTNTT{Es. 37:36-38.}}
\VS{35}Il arriva nuit-là que l'Ange de Yahweh sortit et frappa cent quatre-vingt-cinq mille hommes dans le camp des Assyriens. Et quand on se leva de bon matin, voici, ils étaient tous morts.
\TextTitle{Mort de Sanchérib, roi d'Assyrie\FTNTT{Es. 37:37-38~; 2 Ch. 32:21.}}
\VS{36}Alors Sanchérib, roi d'Assyrie, leva son camp, partit et s'en retourna~; et il resta à Ninive.
\VS{37}Il arriva, comme il était prosterné dans la maison de Nisroc, son dieu, qu'Adrammélec et Scharetser, ses fils, le tuèrent avec l'épée, puis ils se sauvèrent au pays d'Ararat~; et Esar-Haddon, son fils, régna à sa place.
\Chap{20}
\TextTitle{Ezéchias malade puis guéri par Yahweh\FTNTT{2 Ch. 32:24~; Es. 38.}}
\VerseOne{}En ce temps-là, Ezéchias fut malade à la mort. Le prophète Esaïe, fils d'Amots, vint auprès de lui, et lui dit~: Ainsi parle Yahweh~: Donne tes ordres à ta maison, car tu vas mourir et tu ne vivras plus.
\VS{2}Alors Ezéchias tourna son visage contre le mur et fit sa prière à Yahweh, en disant~:
\VS{3}Je te prie, ô Yahweh~! Souviens-toi que j'ai marché devant toi avec fidélité et intégrité de cœur, et que j'ai fait ce qui est agréable à tes yeux~! Et Ezéchias pleura abondamment.
\VS{4}Esaïe n'était pas encore sorti de la cour du milieu, que la parole de Yahweh lui fut adressée, en disant~:
\VS{5}Retourne et dis à Ezéchias, chef de mon peuple~: Ainsi parle Yahweh, le Dieu de David, ton père~: J'ai exaucé ta prière, j'ai vu tes larmes. Voici je te guérirai~; dans trois jours tu monteras à la maison de Yahweh.
\VS{6}J'ajouterai quinze ans à tes jours, je te délivrerai, toi et cette ville, de la main du roi d'Assyrie~; et je protégerai cette ville, par amour pour moi et par amour pour David, mon serviteur.
\VS{7}Puis Esaïe dit~: Prenez une masse de figues sèches. Et ils la prirent et l'appliquèrent sur l'ulcère. Et Ezéchias fut guéri.
\VS{8}Ezéchias avait dit à Esaïe~: A quel signe connaîtrai-je que Yahweh me guérira et qu'au troisième jour, je monterai à la maison de Yahweh~?
\VS{9}Esaïe répondit~: Voici, de la part de Yahweh, le signe auquel tu connaîtras que Yahweh accomplira la parole qu'il a prononcée~: L'ombre s'avancera-t-elle de dix degrés, ou reculera-t-elle en arrière de dix degrés~?
\VS{10}Ezéchias dit~: C'est peu de chose que l'ombre s'avance de dix degrés~; mais plutôt que l'ombre recule en arrière de dix degrés.
\VS{11}Alors Esaïe, le prophète, invoqua Yahweh, qui fit reculer l'ombre de dix degrés sur les degrés d'Achaz, où elle était descendue.
\TextTitle{Visite des ambassadeurs babyloniens~; prophétie sur la captivité babylonienne\FTNTT{2 Ch. 32:25-31~; Es. 39.}}
\VS{12}En ce temps-là, Berodac-Baladan, fils de Baladan, roi de Babylone, envoya une lettre avec un présent à Ezéchias, parce qu'il avait appris la maladie d'Ezéchias.
\VS{13}Et Ezéchias, donna audience aux envoyés et il leur montra tous les lieux où étaient ses objets les plus précieux, l'argent, l'or, les aromates, l'huile précieuse, tout son arsenal et tout ce qui se trouvait dans ses trésors. Il n'y eut rien qu'Ezéchias ne leur montra dans sa maison et dans tous ses domaines.
\VS{14}Esaïe, le prophète, vint ensuite auprès du roi Ezéchias, et lui dit~: Qu'ont dit ces gens-là~? Et d'où sont-ils venus vers toi~? Ezéchias répondit~: Ils sont venus d'un pays très éloigné, ils sont venus de Babylone.
\VS{15}Esaïe dit~: Qu'ont-ils vu dans ta maison~? Et Ezéchias répondit~: Ils ont vu tout ce qui est dans ma maison~; il n'y a rien dans mes trésors que je ne leur aie montré.
\VS{16}Alors Esaïe dit à Ezéchias~: Ecoute la parole de Yahweh~:
\VS{17}Voici, les jours viendront où tout ce qui est dans ta maison et ce que tes pères ont amassé dans leurs trésors jusqu'à ce jour, sera emporté à Babylone~; il n'en restera rien dit Yahweh\FTNT{La déportation des juifs à Babylone~: Voir 2 R. 24-25.}.
\VS{18}On prendra même de tes fils\FTNT{2 R. 24:12~; 2 Ch. 33:11~; Da. 1.} qui seront sortis de toi, que tu auras engendrés, afin qu'ils soient eunuques dans le palais du roi de Babylone.
\VS{19}Ezéchias répondit à Esaïe~: La parole de Yahweh que tu as prononcée est bonne. Et il ajouta~: N'y aura-t-il pas paix et sécurité pendant mes jours~?
\TextTitle{Mort d'Ezéchias~; Manassé règne sur Juda\FTNTT{2 Ch. 32:32-33.}}
\VS{20}Le reste des actions d'Ezéchias, tous ses exploits, et comment il fit l'étang et l'aqueduc par lequel il fit entrer les eaux dans la ville, cela n'est-il pas écrit dans le livre des Chroniques des rois de Juda~?
\VS{21}Ezéchias se coucha avec ses pères. Et Manassé, son fils, régna à sa place.
\Chap{21}
\TextTitle{Abominations et idolâtrie de Manassé\FTNTT{2 Ch. 33:1-9.}}
\VerseOne{}Manassé était âgé de douze ans, lorsqu'il commença à régner. Il régna cinquante-cinq ans à Jérusalem. Sa mère s'appelait Hephtsiba.
\VS{2}Il fit ce qui est mal aux yeux de Yahweh, selon les abominations des nations que Yahweh avait chassées devant les enfants d'Israël.
\VS{3}Car il rebâtit les hauts lieux qu'Ezéchias, son père, avait détruits, et redressa des autels à Baal, il fit une idole d'Asherah\FTNT{Voir commentaire en Jg. 2:13.}, comme avait fait Achab, roi d'Israël, il se prosterna devant toute l'armée des cieux et les servit.
\VS{4}Il bâtit aussi des autels dans la maison de Yahweh, quoique Yahweh ait dit~: C'est dans Jérusalem que j'établirai mon nom.
\VS{5}Il bâtit des autels à toute l'armée des cieux dans les deux parvis de la maison de Yahweh.
\VS{6}Il fit aussi passer son fils par le feu, il pratiquait l'astrologie et la divination, il établit des gens qui évoquaient les esprits des morts et qui prédisaient l'avenir. Il fit de plus en plus ce qui est mal aux yeux de Yahweh pour l'irriter.
\VS{7}Il plaça aussi l'idole d'Asherah qu'il avait faite, dans la maison de laquelle Yahweh avait dit à David, et à Salomon, son fils~: C'est dans cette maison, et c'est dans Jérusalem, que j'ai choisie parmi toutes les tribus d'Israël, que je veux à toujours établir mon nom.
\VS{8}Je ne ferai plus errer le pied d'Israël hors de cette terre que j'ai donnée à leurs pères, pourvu seulement qu'ils aient soin de mettre en pratique tout ce que je leur ai ordonné et toute la loi que Moïse, mon serviteur, leur a prescrite.
\VS{9}Mais ils n'obéirent point~; car Manassé les fit s'égarer, jusqu'à faire le mal plus que les nations que Yahweh avait exterminées devant les enfants d'Israël.
\TextTitle{Jugement de Yahweh contre Juda~; mort d'Ezéchias\FTNTT{2 Ch. 33:10-20.}}
\VS{10}Alors Yahweh parla par ses serviteurs les prophètes, en disant~:
\VS{11}Parce que Manassé, roi de Juda, a commis ces abominations, parce qu'il a fait pire que tout ce qu'avaient fait avant lui les Amoréens, et parce qu'il a aussi fait pécher Juda par ses idoles,
\VS{12}à cause de cela, Yahweh, le Dieu d'Israël, dit~: Voici, je vais faire venir sur Jérusalem et sur Juda des malheurs qui étourdiront les oreilles de quiconque en entendra parler.
\VS{13}Car j'étendrai sur Jérusalem le cordeau de Samarie, et le niveau de la maison d'Achab, et je nettoierai Jérusalem comme un plat qu'on nettoie, et qu’on renverse sur son fond après l'avoir nettoyé.
\VS{14}J'abandonnerai le reste de mon héritage, et je les livrerai entre les mains de leurs ennemis~; et ils seront le butin et la proie de tous leurs ennemis~;
\VS{15}parce qu'ils ont fait ce qui est mal à mes yeux, et qu'ils m'ont irrité depuis le jour où leurs pères sont sortis d'Egypte, jusqu'à ce jour.
\TextTitle{Meurtres de Manassé~; sa mort\FTNTT{2 Ch. 33:11-20.}}
\VS{16}Manassé répandit aussi beaucoup de sang innocent, jusqu'à en remplir Jérusalem d'un bout à l'autre, outre son péché par lequel il fit pécher Juda en faisant ce qui est mal aux yeux de Yahweh.
\VS{17}Le reste des actions de Manassé, tout ce qu'il a fait~; et les péchés auxquels il se livra, cela n'est-il pas écrit dans le livre des Chroniques des rois de Juda~?
\VS{18}Manassé se coucha avec ses pères, et il fut enseveli dans le jardin de sa maison, dans le jardin d'Uzza. Amon, son fils, régna à sa place.
\TextTitle{Amon règne sur Juda~; sa mort\FTNTT{2 Ch. 33:20-25.}}
\VS{19}Amon était âgé de vingt-deux ans lorsqu'il commença à régner. Il régna deux ans à Jérusalem. Sa mère s'appelait Meschullémeth, fille de Haruts, de Jotba.
\VS{20}Il fit ce qui est mal aux yeux de Yahweh, comme avait fait Manassé, son père.
\VS{21}Car il marcha dans toute la voie où avait marché son père, il servit les idoles que son père avait servies et se prosterna devant elles.
\VS{22}Il abandonna Yahweh, le Dieu de ses pères et il ne marcha point dans la voie de Yahweh.
\TextTitle{Josias, roi de Juda\FTNTT{2 Ch. 33:24-25.}}
\VS{23}Les serviteurs d'Amon firent une conspiration contre lui et le tuèrent dans sa maison.
\VS{24}Mais le peuple du pays frappa tous ceux qui avaient conspiré contre le roi Amon~; et ils établirent Josias, son fils, roi à sa place.
\VS{25}Le reste des actions d'Amon, ce qu'il a fait, cela n'est-il pas écrit dans le livre des Chroniques des rois de Juda~?
\VS{26}On l'ensevelit dans son sépulcre, dans le jardin d'Uzza. Et Josias, son fils, régna à sa place.
\Chap{22}
\TextTitle{Droiture de Josias~; réparations dans le temple\FTNTT{2 Ch. 34:2-13.}}
\VerseOne{}Josias était âgé de huit ans lorsqu'il commença à régner. Il régna trente et un ans à Jérusalem. Sa mère s'appelait Jedida, fille d'Adaja, de Botskath.
\VS{2}Il fit ce qui est droit aux yeux de Yahweh et il marcha dans toute la voie de David, son père~; il ne s'en détourna ni à droite ni à gauche.
\VS{3}La dix-huitième année du roi Josias, le roi envoya dans la maison de Yahweh, Schaphan, le secrétaire, fils d'Atsalia, fils de Meschullam.
\VS{4}Il lui dit~: Monte vers Hilkija, le grand-prêtre, et dis-lui d'amasser l'argent qui a été apporté dans la maison de Yahweh et que ceux qui ont la garde du seuil ont recueilli du peuple.
\VS{5}On remettra cet argent entre les mains de ceux qui sont chargés de faire exécuter l'ouvrage dans la maison de Yahweh. Et ils l'emploieront pour ceux qui travaillent dans la maison de Yahweh, pour réparer les brèches de la maison,
\VS{6}pour les charpentiers, les architectes et les maçons, pour les achats du bois et des pierres de taille pour réparer la maison.
\VS{7}Mais on ne leur demandera pas de comptes pour l'argent remis entre leurs mains, parce qu'ils agissent fidèlement.
\TextTitle{Découverte et lecture du livre de la loi\FTNTT{2 Ch. 34:14-19.}}
\VS{8}Alors Hilkija, le grand-prêtre, dit à Schaphan, le secrétaire~: J'ai trouvé le livre de la loi dans la maison de Yahweh. Et Hilkija donna ce livre à Schaphan qui le lut.
\VS{9}Schaphan, le secrétaire, alla vers le roi et lui rapporta la chose, et dit~: Tes serviteurs ont amassé l'argent qui se trouvait dans la maison et l'ont remis entre les mains de ceux qui sont chargés de faire l'ouvrage dans la maison de Yahweh.
\VS{10}Schaphan, le secrétaire, dit aussi au roi~: Le prêtre Hilkija m'a donné un livre. Et Schaphan le lut devant le roi.
\VS{11}Lorsque le roi eut entendu les paroles du livre de la loi, il déchira ses vêtements.
\TextTitle{Annonce du jugement de Yahweh par Hulda\FTNTT{2 Ch. 34:20-28.}}
\VS{12}Il donna cet ordre au prêtre Hilkija, à Achikam, fils de Schaphan, à Acbor, fils de Michée, à Schaphan, le secrétaire, et à Asaja, serviteur du roi~:
\VS{13}Allez, consultez Yahweh pour moi, pour le peuple et pour tout Juda, au sujet des paroles de ce livre qui a été trouvé~; car grande est la colère de Yahweh, qui s'est enflammée contre nous, parce que nos pères n'ont point obéi aux paroles de ce livre et n'ont pas mis en pratique tout ce qui nous y est prescrit.
\VS{14}Le prêtre Hilkija, Achikam, Acbor, Schaphan et Asaja, allèrent auprès de la prophétesse Hulda, femme de Schallum, fils de Thikva, fils de Harhas, gardien des vêtements. Elle habitait dans un autre quartier de Jérusalem.
\TextTitle{Yahweh rassure Josias par la prophétesse Hulda\FTNTT{2 Ch. 34:22-28.}}
\VS{15}Après qu'ils eurent parlé avec elle, elle leur répondit~: Ainsi parle Yahweh, le Dieu d'Israël~: Dites à l'homme qui vous a envoyé vers moi~:
\VS{16}Ainsi parle Yahweh~: Voici, je vais faire venir le malheur sur cette ville et sur ses habitants, selon toutes les paroles du livre que le roi de Juda a lu.
\VS{17}Parce qu'ils m'ont abandonné et qu'ils ont offert des parfums à d'autres dieux, pour m'irriter par toutes les actions de leurs mains, ma colère s'est enflammée contre cette ville et elle ne s'éteindra point.
\VS{18}Mais quant au roi de Juda qui vous a envoyé pour consulter Yahweh, vous lui direz~: Ainsi parle Yahweh, le Dieu d'Israël, au sujet des paroles que tu as entendues~:
\VS{19}Parce que ton cœur a été touché, et que tu t'es humilié devant Yahweh en entendant ce que j'ai prononcé contre cette ville et contre ses habitants, qui seront un objet d'épouvante et de malédiction, et parce que tu as déchiré tes vêtements, et que tu as pleuré devant moi, je t'ai exaucé, dit Yahweh.
\VS{20}C'est pourquoi voici, je vais te recueillir auprès de tes pères, et tu seras recueilli dans ton sépulcre en paix, et tes yeux ne verront point tout ce mal que je vais faire venir sur cette ville. Ils rapportèrent toutes ces paroles au roi.
\Chap{23}
\TextTitle{Le livre de la loi lu au peuple\FTNTT{2 Ch. 34:29-30.}}
\VerseOne{}Alors, le roi Josias, fit assembler auprès de lui tous les anciens de Juda et de Jérusalem.
\VS{2}Le roi monta à la maison de Yahweh, avec tous les hommes de Juda, tous les habitants de Jérusalem, les prêtres, les prophètes, et tout le peuple, depuis le plus petit jusqu'au plus grand. Il lut devant eux toutes les paroles du livre de l'alliance, qui avait été trouvé dans la maison de Yahweh.
\TextTitle{Engagement de Josias et du peuple à suivre la loi de Yahweh\FTNTT{2 Ch. 34:31-32.}}
\VS{3}Le roi se tenait sur l'estrade et il traita alliance devant Yahweh, s'engageant à suivre Yahweh, à observer ses ordonnances, ses préceptes et ses lois, de tout son cœur, à persévérer dans les paroles de cette alliance, écrites dans ce livre. Et tout le peuple entra dans cette alliance.
\TextTitle{Josias débarrasse Juda de tous ses faux dieux\FTNTT{2 Ch. 34:33.}}
\VS{4}Alors le roi donna cet ordre à Hilkija, le grand-prêtre, aux prêtres du second ordre et à ceux qui gardaient le seuil, de sortir hors du temple de Yahweh tous les ustensiles qui avaient été faits pour Baal\FTNT{Voir commentaire en Jg. 2:12.}, pour Asherah\FTNT{Voir commentaire en Jg. 2:13.}, et pour toute l'armée des cieux~; et il les brûla hors de Jérusalem, dans les champs de Cédron, et en fit porter la poussière à Béthel.
\VS{5}Il chassa les prêtres des idoles, que les rois de Juda avaient établis pour brûler des parfums sur les hauts lieux, dans les villes de Juda et aux environs de Jérusalem, et ceux qui offraient des parfums à Baal, au soleil, à la lune, au zodiaque et à toute l'armée des cieux.
\VS{6}Il sortit de la maison de Yahweh l'idole d'Asherah, qu'il transporta hors de Jérusalem vers le torrent de Cédron~; il la brûla au torrent de Cédron et la réduisit en poudre, et il en jeta la poussière sur le sépulcre des fils du peuple.
\VS{7}Ensuite, il démolit les maisons des prostituées qui étaient dans la maison de Yahweh, où les femmes tissaient des tentes pour Asherah.
\VS{8}Il fit venir des villes de Juda tous les prêtres~; il profana les hauts lieux où les prêtres brûlaient des parfums, depuis Guéba jusqu'à Beer-Schéba~; il renversa les hauts lieux des portes, celui qui était à l'entrée de la porte de Josué, chef de la ville, et celui qui était à gauche de la porte de la ville.
\VS{9}Toutefois, les prêtres des hauts lieux ne montaient pas à l'autel de Yahweh à Jérusalem, mais ils mangeaient des pains sans levain parmi leurs frères.
\VS{10}Le roi profana aussi Topheth, dans la vallée des fils de Hinnom, afin que personne ne fasse plus passer son fils ou sa fille par le feu, en l'honneur de Moloc\FTNT{Lé. 20:2-3.}.
\VS{11}Il fit disparaître de l'entrée de la maison de Yahweh les chevaux que les rois de Juda avaient consacrés au soleil, près de la chambre de l'eunuque Nethan-Mélec, situé à Parvarim, et il brûla au feu les chars du soleil.
\VS{12}Le roi démolit les autels qui étaient sur le toit de la chambre haute d'Achaz, que les rois de Juda avaient faits et les autels que Manassé avait faits dans les deux parvis de la maison de Yahweh~; après les avoir brisés et enlevés de là, il en jeta la poussière dans le torrent de Cédron.
\VS{13}Le roi profana aussi les hauts lieux qui étaient en face de Jérusalem, sur la droite de la montagne de perdition, que Salomon, roi d'Israël, avait bâtis à Astarté, l'abomination des Sidoniens, à Kemosch, l'abomination des Moabites, et à Milcom, l'abomination des fils d'Ammon.
\VS{14}Il brisa aussi les statues, et abattit les Asherah, et il remplit d'ossements d'hommes les lieux où elles étaient.
\VS{15}Il renversa l'autel qui était à Béthel et le haut lieu qu'avait fait Jéroboam, fils de Nebath, qui avait fait pécher Israël~; il brûla le haut lieu et le réduisit en poudre, et il brûla l'Asherah.
\VS{16}Josias s'étant tourné et ayant vu les sépulcres qui étaient là dans la montagne, envoya prendre les ossements des sépulcres, et il les brûla sur l'autel et le profana, selon la parole de Yahweh prononcée à haute voix par l'homme de Dieu.
\VS{17}Le roi dit~: Quel est ce monument que je vois~? Et les hommes de la ville lui répondirent~: C'est le sépulcre de l'homme de Dieu qui est venu de Juda qui a crié contre l'autel de Béthel ces choses que tu as accomplies.
\VS{18}Et il dit~: Laissez-le~; que personne ne remue ses os~! Ils conservèrent ainsi ses os, avec les os du prophète qui était venu de Samarie.
\VS{19}Josias fit encore disparaître toutes les maisons des hauts lieux, qui étaient dans les villes de Samarie, et qu'avaient faites les rois d'Israël pour irriter Yahweh~; et il fit à leur égard entièrement comme il avait fait à Béthel.
\VS{20}Il immola sur les autels tous les prêtres des hauts lieux qui étaient là, et il y brûla des ossements d'hommes. Puis il retourna à Jérusalem.
\TextTitle{Josias rétablit la fête de la Pâque\FTNTT{2 Ch. 35:1-19.}}
\VS{21}Alors le roi donna cet ordre à tout le peuple, en disant~: Célébrez la Pâque en l'honneur de Yahweh, votre Dieu, comme il est écrit dans le livre de cette alliance\FTNT{Jésus-Christ est notre Pâque. Voir Ex. 12 et 1 Co. 5:7.}.
\VS{22}Aucune Pâque pareille à celle-ci n'avait été célébrée depuis le temps où les juges jugeaient Israël et pendant tous les jours des rois d'Israël et des rois de Juda.
\VS{23}Ce fut la dix-huitième année du roi Josias qu'on célébra cette Pâque en l'honneur de Yahweh à Jérusalem.
\VS{24}Josias, extermina aussi, ceux qui évoquaient les esprits des morts et les devins, les théraphim, les idoles, et toutes les abominations qui se voyaient dans le pays de Juda et à Jérusalem, afin de mettre en pratique les paroles de la loi, écrites dans le livre que Hilkija, le prêtre, avait trouvé dans la maison de Yahweh.
\TextTitle{Témoignage de Josias~; confirmation du jugement de Yahweh}
\VS{25}Avant Josias, il n'y eut point de roi qui, comme lui, revienne à Yahweh de tout son cœur, de toute son âme et de toute sa force, selon toute la loi de Moïse~; et après lui, il n'en a point paru de semblable.
\VS{26}Toutefois, Yahweh ne se détourna point de l'ardeur de sa grande colère dont il était enflammé contre Juda, à cause de tout ce que Manassé avait fait pour l'irriter.
\VS{27}Et Yahweh dit~: J'ôterai Juda de devant ma face, comme j'ai ôté Israël, et je rejetterai cette ville de Jérusalem que j'avais choisie, et la maison de laquelle j'avais dit~: Là sera mon Nom.
\VS{28}Le reste des actions de Josias, tout ce qu'il a fait, cela n'est-il pas écrit dans le livre des Chroniques des rois de Juda~?
\TextTitle{Mort de Josias~; Joachaz règne sur Juda\FTNTT{2 Ch. 35:20-27~; 2 Ch. 36:1-2.}}
\VS{29}De son temps, Pharaon Néco, roi d'Egypte, monta contre le roi d'Assyrie, vers le fleuve d'Euphrate. Le roi Josias s'en alla au-devant de lui~; mais dès que pharaon le vit, il le tua à Meguiddo.
\VS{30}Ses serviteurs l'emportèrent mort sur un char~; ils l'amenèrent de Meguiddo à Jérusalem et l'ensevelirent dans son sépulcre. Et le peuple du pays prit Joachaz, fils de Josias, ils l'oignirent et l'établirent roi à la place de son père.
\TextTitle{Joachaz mis en prison par Pharaon\FTNTT{2 Ch. 36:3.}}
\VS{31}Joachaz était âgé de vingt-trois ans, lorsqu'il commença à régner. Il régna trois mois à Jérusalem. Sa mère s'appelait Hamuthal, fille de Jérémie, de Libna.
\VS{32}Il fit ce qui est mal aux yeux de Yahweh, entièrement comme avaient fait ses pères.
\VS{33}Et pharaon Néco l'emprisonna à Ribla, dans le pays de Hamath, afin qu'il ne règne plus à Jérusalem~; et il imposa sur le pays un tribut de cent talents d'argent et d'un talent d'or.
\TextTitle{Pharaon établit Jojakim roi de Juda\FTNTT{2 Ch. 36:4-5.}}
\VS{34}Puis pharaon Néco établit roi Eliakim, fils de Josias, à la place de Josias, son père, et il changea son nom en celui de Jojakim. Il prit Joachaz, qui alla en Egypte, où il mourut.
\VS{35}Jojakim donna cet argent et cet or à pharaon~; mais il taxa le pays pour fournir cet argent, selon l'ordre de pharaon~; il détermina la part de chacun et exigea du peuple du pays l'argent et l'or qu'il devait livrer à pharaon Néco.
\VS{36}Jojakim était âgé de vingt-cinq ans lorsqu'il commença à régner. Il régna onze ans à Jérusalem. Sa mère s'appelait Zebudda, fille de Pedaja, de Ruma.
\VS{37}Il fit ce qui est mal aux yeux de Yahweh, entièrement comme avaient fait ses pères.
\Chap{24}
\TextTitle{Asservissement de Jojakim au roi de Babylone~; destruction de Juda\FTNTT{2 Ch. 36:6-7.}}
\VerseOne{}De son temps, Nebucadnetsar, roi de Babylone, monta contre Jojakim, et Jojakim lui fut asservi pendant trois ans~; mais il se révolta de nouveau contre lui.
\VS{2}Alors Yahweh envoya contre Jojakim des troupes de Chaldéens, des armées de Syriens, des troupes de Moabites et des troupes des fils d'Ammon~; il les envoya contre Juda, pour le détruire, selon la parole que Yahweh avait prononcée par ses serviteurs les prophètes.
\VS{3}Cela arriva uniquement sur l'ordre de Yahweh, qui voulait ôter Juda de devant sa face, à cause de tous les péchés commis par Manassé,
\VS{4}et à cause aussi du sang innocent qu'il avait répandu, et dont il avait rempli Jérusalem. C'est pourquoi Yahweh ne voulut point lui pardonner.
\TextTitle{Mort de Jojakim~; Jojakin règne sur Juda\FTNTT{2 Ch. 36:8-9.}}
\VS{5}Le reste des actions de Jojakim et tout ce qu'il a fait, cela n'est-il pas écrit dans le livre des Chroniques des rois de Juda~?
\VS{6}Ainsi Jojakim se coucha avec ses pères. Et Jojakin, son fils, régna à sa place.
\VS{7}Le roi d'Egypte ne sortit plus de son pays, parce que le roi de Babylone avait pris tout ce qui était au roi d'Egypte, depuis le torrent d'Egypte jusqu'au fleuve d'Euphrate.
\VS{8}Jojakin était âgé de dix-huit ans lorsqu'il commença à régner. Il régna trois mois à Jérusalem. Sa mère s'appelait Nehuschtha, fille d'Elnathan, de Jérusalem.
\VS{9}Il fit ce qui est mal aux yeux de Yahweh, entièrement comme avait fait son père.
\TextTitle{Jérusalem et son roi en captivité à Babylone~; les pauvres restent\FTNTT{2 Ch. 36:10.}}
\VS{10}En ce temps-là, les serviteurs de Nebucadnetsar, roi de Babylone, montèrent contre Jérusalem, et la ville fut assiégée.
\VS{11}Nebucadnetsar, roi de Babylone, arriva devant la ville, pendant que ses serviteurs l'assiégeaient.
\VS{12}Alors Jojakin, roi de Juda, se rendit vers le roi de Babylone, avec sa mère, ses serviteurs, ses chefs et ses eunuques. Et le roi de Babylone le fit prisonnier, la huitième année de son règne.
\VS{13}Il emporta de là, tous les trésors de la maison de Yahweh et les trésors de la maison royale~; et il mit en pièces tous les ustensiles d'or que Salomon, roi d'Israël, avait faits pour le temple de Yahweh, comme Yahweh l'avait ordonné.
\VS{14}Il emmena en captivité tout Jérusalem\FTNT{Première déportation~: 2 R. 24:1-4 et 2 Ch. 36:6-7. La première déportation eut lieu en 597 av. J.-C. pendant le règne de Jojakim, roi de Juda. Les premiers exilés furent installés dans la région du fleuve Kebar (Ez. 1:1-3), un canal de 90 km de long reliant l'Euphrate au nord de Babylone au même fleuve au sud d'Ur en Chaldée. Jérémie savait que leur séjour à l'étranger serait long. Il avait prophétisé qu'il durerait soixante-dix ans (Jé. 25:1~; Jé. 25:11-12) et leur conseilla de se construire des maisons, de cultiver des jardins et de se multiplier (Jé. 29). Daniel et ses compagnons furent déportés à Babylone lors de la première déportation (Da. 1). Daniel fut déporté environ huit ans avant Ezéchiel.}, à savoir, tous les chefs, et tous les vaillants hommes de guerre, au nombre de dix mille captifs, avec les charpentiers et les serruriers, de sorte qu'il ne resta plus que le peuple pauvre du pays.
\VS{15}Ainsi il transporta Jojakin à Babylone, avec la mère du roi, les femmes du roi et ses eunuques. Il emmena captifs à Babylone tous les grands du pays, de Jérusalem à Babylone,
\VS{16}avec tous les guerriers au nombre de sept mille, les charpentiers, les serruriers au nombre de mille, tous les hommes vaillants et propres à la guerre. Le roi de Babylone les emmena captifs à Babylone.
\TextTitle{Nebucadnetsar établit Sédécias roi de Juda\FTNTT{2 Ch. 36:10-12.}}
\VS{17}Et le roi de Babylone établit roi, à la place de Jojakin, Matthania, son oncle, et il changea son nom en celui de Sédécias.
\VS{18}Sédécias était âgé de vingt et un ans lorsqu'il commença à régner. Il régna onze ans à Jérusalem. Sa mère s'appelait Hamuthal, fille de Jérémie, de Libna.
\VS{19}Il fit ce qui est mal aux yeux de Yahweh, entièrement comme avait fait Jojakim.
\TextTitle{Sédécias se révolte\FTNTT{2 Ch. 36:13-16.}}
\VS{20}Cela arriva à cause de la colère de Yahweh contre Jérusalem et contre Juda, qu'il voulait rejeter de devant sa face. Et Sédécias se révolta contre le roi de Babylone.
\Chap{25}
\TextTitle{Siège de Jérusalem\FTNTT{Jé. 39:1.}}
\VerseOne{}Et il arriva dans la neuvième année du règne de Sédécias, le dixième jour du dixième mois, que Nebucadnetsar\FTNT{Jérusalem fut assiégée pendant deux ans. Lors de ce siège, des femmes juives faisaient cuire leurs enfants pour les consommer (La. 2:20~; La. 4:10).}, roi de Babylone, vint avec toute son armée contre Jérusalem~; il campa devant elle et éleva des retranchements tout autour.
\VS{2}La ville fut assiégée jusqu'à la onzième année du roi Sédécias.
\VS{3}Le neuvième jour du 4ème mois, la famine\FTNT{La. 4:10.} augmenta dans la ville, de sorte qu'il n'y avait pas de pain pour le peuple du pays.
\TextTitle{Sédécias lié et emmené à Babylone\FTNTT{Jé. 39:2-7.}}
\VS{4}Alors la brèche fut faite à la ville~; et tous les gens de guerre s'enfuirent de nuit par le chemin de la porte entre les deux murailles près du jardin du roi, pendant que les Chaldéens environnaient la ville. Les fuyards et le roi prirent le chemin de la plaine.
\VS{5}Mais l'armée des Chaldéens poursuivit le roi et l'atteignit dans les plaines de Jéricho, et toute son armée se dispersa loin de lui.
\VS{6}Ils saisirent donc le roi et le firent monter vers le roi de Babylone à Ribla~; et l'on prononça contre lui un jugement.
\VS{7}Et on égorgea les fils de Sédécias en sa présence~; puis on creva les yeux à Sédécias, et on le lia de doubles chaînes d'airain, et on le mena à Babylone.
\TextTitle{Destruction de Jérusalem, du temple et des murailles\FTNTT{2 Ch. 36:17-21~; Jé. 39~:8-10.}}
\VS{8}Le septième jour du cinquième mois, c'était la dix-neuvième année du roi Nebucadnetsar, roi de Babylone, Nebuzaradan, chef des gardes, serviteur du roi de Babylone,\FTNT{Troisième déportation~: Le temple fut brûlé, la ville de Jérusalem fut totalement rasée et ses habitants furent déportés (De. 28:49-68). Contrairement à ce que l'on pense, il y a eu d'autres déportations. Voir Jé. 52.} entra dans Jérusalem.
\VS{9}Il brûla la maison de Yahweh, la maison royale et toutes les maisons de Jérusalem~; il brûla par le feu toutes les grandes maisons.
\VS{10}Toute l'armée des Chaldéens, qui était avec le chef des gardes, démolit les murailles qui entouraient Jérusalem.
\VS{11}Et Nebuzaradan, chef des gardes, emmena captifs le reste du peuple, ceux qui étaient restés dans la ville, ceux qui s'étaient rendus au roi de Babylone et le reste de la multitude.
\VS{12}Cependant le chef des gardes laissa quelques-uns des plus pauvres du pays comme vignerons et comme laboureurs.
\VS{13}Les Chaldéens brisèrent les colonnes d'airain qui étaient dans la maison de Yahweh, les bases, la mer d'airain qui était dans la maison de Yahweh, et ils en emportèrent l'airain à Babylone.
\VS{14}Ils prirent aussi les cendriers, les pelles, les couteaux, les tasses et tous les ustensiles d'airain avec lesquels on faisait le service.
\VS{15}Le chef des gardes emporta aussi les encensoirs et les coupes, ce qui était d'or et ce qui était d'argent.
\VS{16}Les deux colonnes, la mer et les bases, que Salomon avait faits pour la maison de Yahweh, tous ces ustensiles d'airain avaient un poids inconnu.
\VS{17}La hauteur d'une colonne était de dix-huit coudées, et il y avait au-dessus un chapiteau d'airain dont la hauteur était de trois coudées~; autour du chapiteau il y avait un treillis et des grenades, le tout d'airain~; il en était de même pour la seconde colonne avec le treillis.
\VS{18}Le chef des gardes emmena aussi Seraja, le premier prêtre, et Sophonie, le second prêtre, et les trois gardiens du seuil.
\VS{19}Et dans la ville, il prit un eunuque qui avait sous son commandement des hommes de guerre, cinq hommes de ceux qui voyaient la face du roi et qui furent trouvés dans la ville, il prit aussi le secrétaire du chef de l'armée qui était chargé d'enrôler le peuple du pays, et soixante hommes du peuple du pays qui se trouvaient dans la ville.
\VS{20}Nebuzaradan, chef des gardes, les prit et les conduisit vers le roi de Babylone à Ribla.
\VS{21}Le roi de Babylone les frappa, et les fit mourir à Ribla, dans le pays de Hamath. Ainsi Juda fut transporté captif hors de sa terre.
\TextTitle{Guedalia nommé gouverneur de Juda\FTNTT{Jé. 40:7-11.}}
\VS{22}Nebucadnetsar, roi de Babylone, plaça le reste du peuple, qu'il laissa dans le pays de Juda, sous le commandement de Guedalia, fils d'Achikam, fils de Schaphan.
\VS{23}Lorsque tous les chefs des troupes et leurs hommes, eurent appris que le roi de Babylone avait établi Guedalia pour gouverneur, ils allèrent trouver Guedalia à Mitspa, à savoir Ismaël, fils de Nethania, Jochanan, fils de Karéach, Seraja, fils de Thanhumeth, de Nethopha, Jaazania, fils du Maacathien, eux et leurs hommes.
\VS{24}Guedalia leur jura, à eux et à leurs hommes, et leur dit~: Ne craignez pas d'être serviteurs des Chaldéens~; demeurez dans le pays et servez le roi de Babylone, et vous vous en trouverez bien.
\TextTitle{Fuite du peuple en Egypte\FTNTT{Jé. 41:1-3~; Jé. 43:4-7.}}
\VS{25}Mais il arriva au septième mois, qu'Ismaël, fils de Nethania, fils d'Elischama, qui était de race royale, vint, accompagné de dix hommes, et ils frappèrent mortellement Guedalia, ainsi que les Juifs et les Chaldéens qui étaient avec lui à Mitspa.
\VS{26}Alors tout le peuple, depuis le plus petit jusqu'au plus grand, avec les chefs des troupes, se levèrent et s'en allèrent en Egypte, parce qu'ils avaient peur des Chaldéens.
\TextTitle{Jojakin à la table du roi de Babylone\FTNTT{Jé. 52:31-34.}}
\VS{27}La trente-septième année de la captivité de Jojakin, roi de Juda, le vingt-septième jour du douzième mois, Evil-Merodac, roi de Babylone, dans la première année de son règne, releva la tête de Jojakin, roi de Juda et le tira de prison.
\VS{28}Il lui parla avec bonté et il mit son trône au-dessus du trône des rois qui étaient avec lui à Babylone.
\VS{29}Il lui fit changer ses vêtements de prison, et Jojakin mangea du pain tout le temps de sa vie en sa présence.
\VS{30}Et quant à son entretien, un entretien perpétuel, lui fut accordé par le roi pour chaque jour, tous les jours de sa vie.
\PPE{}
\end{multicols}

%\clearpage
\ShortTitle{Esaïe}\BookTitle{Esaïe}\BFont
\noindent\hrulefill
{\footnotesize
\textit{
\bigskip
{\centering{}
\\Auteur : Esaïe
\\(Heb. : Yesha'yah)
\\Signification : YAHWEH a sauvé
\\Thème : Le Messie d'Israël
\\Date de rédaction : 8\up{ème} siècle av. J.-C.\\}
}
%\bigskip
\textit{
\\Prophète en Israël, Esaïe fut une figure marquante en raison du contenu et de l'impact de son message. Véritable porte-parole de Dieu, il parla de la ruine morale d'Israël, de la déportation à Babylone et des jugements de Dieu sur son peuple. Il prophétisa également sur le retour de l'exil, la restauration finale et la reconstruction de Jérusalem. Plus qu'aucun autre livre, les écrits d'Esaïe annoncent clairement la naissance du Messie, son service, sa mission rédemptrice, son sacrifice et son futur règne millénaire. 
%\bigskip
\\L'autorité et l'exactitude de ses prophéties ont été une source d'édification au fil des siècles.\bigskip
}
}
\par\nobreak\noindent\hrulefill
\begin{multicols}{2}
\Chap{1}
\TextTitle{Prophéties concernant Juda}
\VerseOne{}La vision d'Esaïe, fils d'Amots, qu'il a vue touchant Juda et Jérusalem, au jour d'Ozias, de Jotham, d'Achaz, et d'Ezéchias, rois de Juda.
\VS{2}Cieux, écoutez ! Et toi, terre, prête l'oreille ! Car Yahweh parle. J'ai nourri des enfants, je les ai élevés, mais ils se sont rebellés contre moi.
\VS{3}Le bœuf connaît son possesseur, et l'âne la crèche de son maître, mais Israël n'a point de connaissance, mon peuple n'a point d'intelligence.
\VS{4}Ah! Nation pécheresse, peuple chargé d'iniquités, race de gens méchants, enfants qui ne font que se corrompre ! Ils ont abandonné Yahweh, ils ont irrité par leur mépris le Saint d'Israël, ils se sont retirés en arrière.
\VS{5}Pourquoi serez-vous encore frappés ? Vous ajouterez la révolte ! La tête entière est malade, et tout le cœur est languissant.
\VS{6}Depuis la plante du pied jusqu'à la tête, il n'y a rien de sain en lui : Il n'y a que blessures, meurtrissures et plaies pourries, qui n'ont été ni nettoyées, ni bandées, et dont aucune n'a été adoucie par l'huile.
\VS{7}Votre pays n'est que désolation, et vos villes sont en feu ; des étrangers dévorent votre terre sous vos yeux, et cette désolation est comme un bouleversement fait par des étrangers.
\VS{8}Car la fille de Sion est restée comme une cabane dans une vigne, comme une cabane dans un champ de concombres, comme une ville assiégée.
\VS{9}Si Yahweh des armées ne nous avait pas laissé un petit reste, qui est même bien peu, nous serions comme Sodome, nous ressemblerions à Gomorrhe.
\TextTitle{Yahweh rejette la religiosité et recherche la justice}
\VS{10}Ecoutez la parole de Yahweh, chefs de Sodome, prêtez l'oreille à la loi de notre Dieu, peuple de Gomorrhe !
\VS{11}Qu'ai-je à faire, dit Yahweh, de la multitude de vos sacrifices ? Je suis rassasié des holocaustes de béliers et de la graisse des veaux ; je ne prends point plaisir au sang des taureaux, ni des agneaux, ni des boucs\FTNT{1 S. 15:22 ; Os. 8:13 ; Mt. 9:13.}.
\VS{12}Quand vous entrez pour vous présenter devant ma face, qui a requis cela de votre main, que vous fouliez de vos pieds mes parvis ?
\VS{13}Ne continuez plus à m'apporter de vaines offrandes : Le parfum m'est en abomination, quant aux nouvelles lunes, aux sabbats et à la publication de vos convocations ; je ne puis plus supporter votre méchanceté ni vos assemblées solennelles.
\VS{14}Mon âme hait vos nouvelles lunes et vos fêtes solennelles ; elles me sont fâcheuses, je suis las de les supporter.
\VS{15}C'est pourquoi, quand vous étendez vos mains, je cache mes yeux de vous ; quand vous multipliez vos prières, je ne les exauce pas ; vos mains sont pleines de sang\FTNT{Es. 59:1-3 ;Mi. 3:4.}.
\VS{16}Lavez-vous, purifiez-vous, ôtez de devant mes yeux la méchanceté de vos actions ; cessez de faire le mal.
\VS{17}Apprenez à bien faire, recherchez la droiture, redressez celui qui est foulé ; faites justice à l'orphelin, défendez la cause de la veuve.
\TextTitle{Mise en garde ; appel à la justice de Yahweh}
\VS{18}Venez maintenant, dit Yahweh, et débattons nos droits. Si vos péchés sont comme l'écarlate, ils seront blanchis comme la neige ; s'ils sont rouges comme le vermillon ils seront blanchis comme la laine.
\VS{19}Si vous obéissez volontairement, vous mangerez le meilleur du pays.
\VS{20}Mais si vous refusez d'obéir et si vous êtes rebelles, vous serez dévorés par l'épée, car la bouche de Yahweh a parlé.
\VS{21}Comment la cité fidèle est-elle devenue une prostituée ? Elle était pleine de droiture et la justice y habitait ; mais maintenant elle est pleine de meurtriers !
\VS{22}Ton argent s'est changé en scories; ton breuvage est mêlé d'eau. 
\VS{23}Les chefs de ton peuple sont rebelles et compagnons des voleurs ; chacun d'eux aime les présents, ils courent après les récompenses ; ils ne font point droit à l'orphelin, et la cause de la veuve ne vient point devant eux. 
\VS{24}C'est pourquoi le Seigneur, Yahweh des armées, le Puissant d'Israël dit : Ah ! Je me satisferai en punissant mes adversaires, et je me vengerai de mes ennemis. 
\VS{25}Et je remettrai ma main sur toi, je refondrai tes scories comme avec la potasse, et j'ôterai tout ton étain ;
\VS{26}mais je rétablirai tes juges, tels qu'ils étaient autrefois, et tes conseillers, tels qu'ils étaient au commencement\FTNT{Dans le royaume messianique, le gouvernement théocratique sera restauré et la fonction des juges sera rétablie (voir livre des Juges ; Mt. 19:28 ; 1 Co. 6:2-3).}. Après cela, on t'appellera cité de la justice, ville fidèle.
\VS{27}Sion sera rachetée par la droiture et ceux qui s'y convertiront seront rachetés par la justice.
\VS{28}Mais les rebelles et les pécheurs seront détruits ensemble, et ceux qui abandonnent Yahweh seront consumés.
\VS{29}Car on sera honteux à cause des térébinthes que vous avez désirés, et vous rougirez à cause des jardins que vous avez choisis\FTNT{Des cultes idolâtres avaient lieu autour des térébinthes et dans des jardins (De. 16:21 ; Es. 57:4-5 ; Es. 65:3 ; Jé. 2:20 ; Ez. 20:28 ; Os. 4:13).}.
\VS{30}Car vous serez comme le térébinthe dont le feuillage tombe, et comme un jardin qui n'a pas d'eau.
\VS{31}Et le fort sera de l'étoupe, et son œuvre une étincelle ; et tous deux brûleront ensemble, et il n'y aura personne pour éteindre le feu.
\Chap{2}
\TextTitle{Vision du règne messianique}
\VerseOne{}La parole qu'Esaïe, fils d'Amots a vue touchant Juda et Jérusalem.
\VS{2}Or il arrivera, dans les derniers jours\FTNT{Voir Ge. 49:1-2.}, que la montagne de la maison de Yahweh sera affermie au  sommet des montagnes, qu'elle sera élevée par-dessus les collines et que toutes les nations y afflueront.
\VS{3}Et plusieurs peuples iront et diront : Venez, et montons à la montagne de Yahweh, à la maison du Dieu de Jacob ; et il nous instruira ses voies, et nous marcherons dans ses sentiers ; car la loi sortira de Sion, et la parole de Yahweh sortira de Jérusalem. 
\VS{4}Il exercera le jugement parmi les nations, et reprendra plusieurs peuples. De leurs épées ils forgeront des hoyaux, et de leurs lances des serpes ; une nation ne lèvera plus l'épée contre une autre et ils ne s'adonneront plus à la guerre.
\VS{5}Venez, ô maison de Jacob, et marchons dans la lumière de Yahweh.
\TextTitle{L'orgueilleux abaissé au jour de Yahweh}
\VS{6}Certes tu as rejeté ton peuple, la maison de Jacob, parce qu'ils se sont remplis d'orient et adonnés à la divination comme les Philistins, et parce qu'ils s'allient aux enfants des étrangers\FTNT{De. 18:8-13 ; Os. 13:2 ; Mi. 5:11-13.}.
\VS{7}Son pays est rempli d'argent et d'or, et il n'y a pas de fin à ses trésors ; son pays est rempli de chevaux, et il n'y a pas de fin à ses chars.
\VS{8}Son pays est rempli d'idoles ; ils se prosternent devant l'ouvrage de leurs mains et devant ce que leurs doigts ont fabriqué.
\VS{9}Et ceux du commun sont abattus, et les personnes de qualité sont abaissées ; ne leur pardonne donc point.
\VS{10}Entre dans les rochers et cache-toi dans la poussière, à cause de la frayeur de Yahweh, et à cause de la gloire de sa majesté\FTNT{Ap. 6:15-16.}.
\VS{11}Les yeux hautains des hommes seront abaissés et les hommes qui s'élèvent seront humiliés, Yahweh sera seul haut élevé en ce jour-là.
\VS{12}Car il y a un jour assigné par Yahweh des armées contre tout homme orgueilleux et hautain, et contre tout homme qui s'élève, afin qu'il soit abaissé ;
\VS{13}contre tous les cèdres du Liban, hauts et élevés, et contre tous les chênes de Basan ;
\VS{14}contre toutes les hautes montagnes, et contre toutes les collines élevées ;
\VS{15}contre toutes les hautes tours, et contre toutes les murailles fortes ;
\VS{16}contre tous les navires de Tarsis, et contre toutes les peintures de plaisance.
\VS{17}Et l'arrogance des hommes sera humiliée, et les hommes qui s'élèvent seront abaissés :
\VS{18}Yahweh seul sera élevé en ce jour-là. Quant aux idoles, elles tomberont toutes.
\VS{19}Et les hommes entreront dans les cavernes des rochers et dans les trous de la terre, à cause de la frayeur de Yahweh et à cause de sa gloire magnifique, lorsqu'il se lèvera pour faire trembler la terre.
\VS{20}En ce jour-là, les hommes jetteront aux taupes et aux chauves-souris leurs idoles d'argent et leurs idoles d'or, qu'ils s'étaient faites pour se prosterner devant elles ;
\VS{21}et ils entreront dans les fentes des rochers et dans les creux des rochers, à cause de la frayeur de Yahweh, et à cause de sa gloire magnifique, quand il se lèvera pour punir la terre.
\VS{22}Retirez-vous de l'homme, dans les narines duquel il n'y a qu'un souffle : Car quel cas mérite-t-il qu'on en fasse ?
\Chap{3}
\TextTitle{Le péché, cause de dissolution nationale}
\VerseOne{}Car voici, le Seigneur, Yahweh des armées, va ôter de Jérusalem et de Juda tout appui et toute ressource, toute ressource de pain et toute ressource d'eau.
\VS{2}L'homme fort et l'homme de guerre, le juge et le prophète, le devin et l'ancien,
\VS{3}le chef de cinquante et l'homme d'autorité, le conseiller, l'expert d'entre les artisans et l'habile enchanteur.
\VS{4}Et je leur donnerai de jeunes gens pour chefs, et des enfants domineront sur eux.
\VS{5}Le peuple sera opprimé ; l'un opprimera l'autre, chacun son prochain. Le jeune homme se portera arrogamment contre le vieillard, et l'homme de rien contre l'honorable.
\VS{6}Même un homme ira jusqu'à saisir son frère dans la maison paternelle et lui dira : Tu as un manteau, sois notre chef ! Et prends en main ces ruines !
\VS{7}Ce jour même il répondra : Je ne suis pas médecin, et dans ma maison il n'y a ni pain ni manteau ; ne m'établissez donc pas chef du peuple.
\VS{8}Certes Jérusalem est renversée, et Juda est tombée, parce que leurs langues et leurs actions sont contre Yahweh, pour braver les regards de sa gloire.
\VS{9}L'aspect de leur visage témoigne contre eux, ils publient leur péché comme Sodome, ils ne le cachent pas. Malheur à leur âme, car ils ont attiré le mal sur eux !
\VS{10}Dites au juste que du bien lui arrivera, car il mangera le fruit de ses œuvres.
\VS{11}Malheur au méchant qui ne cherche qu'à faire le mal, car la rétribution de ses mains lui sera rendue.
\VS{12}Quant à mon peuple, il a pour oppresseur des enfants, et des femmes dominent sur lui. Mon peuple, ceux qui te conduisent t'égarent, ils corrompent le chemin dans lequel tu marches.
\VS{13}Yahweh se présente pour plaider, il se tient debout pour juger les peuples.
\VS{14}Yahweh entre en jugement avec les anciens de son peuple et avec ses chefs ; car vous avez brouté la vigne, et ce que vous avez ravi au pauvre est dans vos maisons.
\VS{15}Que vous revient-il de fouler mon peuple, et d'écraser le visage des affligés ? Dit le Seigneur, Yahweh des armées.
\TextTitle{Les filles hautaines de Sion}
\VS{16}Yahweh dit aussi : Parce que les filles de Sion sont hautaines, et qu'elles marchent le cou tendu et les yeux pleins de convoitise, parce qu'elles marchent avec une fière démarche faisant du bruit avec leurs pieds,
\VS{17}Yahweh rendra chauve le sommet de la tête des filles de Sion, Yahweh découvrira leur nudité.
\VS{18}En ce temps-là, le Seigneur ôtera l'ornement de leurs anneaux de cheville, et les filets et les croissants ;
\VS{19}les pendants d'oreilles, les bracelets et les voiles ;
\VS{20}les parures de la tête, les chaînettes des pieds et les ceintures, les boîtes à parfum et les amulettes ;
\VS{21}les anneaux et les bagues qui leur pendent sur le nez ;
\VS{22}les vêtements de fête et les larges tuniques, les manteaux et les gibecières ;
\VS{23}les miroirs et les chemises fines, les tiares et les voiles légers.
\VS{24}Et il arrivera qu'au lieu du parfum, il y aura de la puanteur ; au lieu de ceintures, des cordes ; au lieu de cheveux bouclés, des têtes chauves ; au lieu de robes flottantes, des sacs étroits ; et au lieu d'un beau teint, un teint tout hâlé.
\VS{25}Tes hommes tomberont par l'épée et ta force par la guerre.
\VS{26}Et ses portes gémiront et mèneront deuil ; désolée, elle s'assiéra par terre.
\Chap{4}
\TextTitle{Vision du règne messianique\FTNTT{Es. 11:1-16.}}
\VerseOne{}Et en ce jour sept femmes saisiront un seul homme, et diront : Nous mangerons notre pain, et nous nous vêtirons de nos habits ; seulement fais-nous porter ton nom ; ôte notre opprobre.
\VS{2}En ce temps-là, le germe de Yahweh\FTNT{Jésus est le « germe » de Yahweh (Es. 4:2) et le germe de David (Jé. 23:5 ; Za. 3:8 ; Za. 6:12). Ce germe a été placé par la vertu du Saint-Esprit dans le sein d'une vierge (Es. 7:14 ; Lu. 1:34-35) et l'enfant qui naquit d'elle fut appelé « Fils de Dieu » tout en étant le Dieu Tout-Puissant. Il existe de toute éternité en forme de Dieu (Jn. 1:1 ; Es. 9:5), mais il a été fait chair pour nous sauver (Jn. 1:14. 1 Ti. 3:16).
C'est le plus grand des miracles et la démonstration de sa divinité, de sa sagesse et de son amour envers les hommes.
} sera plein de noblesse et de gloire, et le fruit de la terre plein de grandeur et d'excellence pour les réchappés d'Israël.
\VS{3}Et il arrivera que les restes de Sion, et les restes de Jérusalem, seront appelés saints; et ceux de Jérusalem seront inscrits parmi les vivants\FTNT{Es. 10:20-22 ; Ro. 9:27 ; Ro. 11:5 ;}.
\VS{4}Quand le Seigneur aura lavé la souillure des filles de Sion, et purifié Jérusalem du sang qui est au milieu d'elle, par l'esprit de jugement et par l'esprit qui consume;
\VS{5}aussi Yahweh créera, sur toute l'étendue du mont Sion et sur ses assemblées, une nuée avec une fumée pendant le jour, et une splendeur de feu flamboyant pendant la nuit, car la gloire se répandra partout.
\VS{6}Et il y aura un tabernacle pour donner de l'ombre contre la chaleur du jour, pour servir de refuge et d'asile contre la tempête et la pluie\FTNT{Ap. 21:3.}.
\Chap{5}
\TextTitle{Israël, vigne de Yahweh}
\VerseOne{}Je chanterai maintenant pour mon bien-aimé le cantique de mon bien-aimé sur sa vigne. Mon bien-aimé avait une vigne sur un coteau fertile.
\VS{2}Il l'environna d'une haie, en ôta les pierres, et y planta des ceps exquis ; il bâtit une tour au milieu d'elle, et il y creusa aussi une cuve. Puis il espéra qu'elle produirait des raisins, mais elle a produit des grappes sauvages\FTNT{Lu. 13:6-9.}.
\VS{3}Maintenant donc, vous habitants de Jérusalem et vous hommes de Juda, jugez, je vous prie, entre moi et ma vigne.
\VS{4}Qu'y avait-il encore à faire à ma vigne que je ne lui aie fait ? Pourquoi, quand j'ai attendu qu'elle produirait des raisins, a-t-elle produit des grappes sauvages ?
\VS{5}Maintenant donc je vous dirai ce que je vais faire à ma vigne : J'ôterai sa haie, et elle sera broutée ; je romprai sa clôture et elle sera foulée.
\VS{6}Et je la réduirai en désert, elle ne sera plus taillée, ni cultivée ; les ronces et les épines y croîtront ; et je commanderai aux nuées qu'elles ne laissent plus tomber de pluie sur elle.
\VS{7}Or la maison d'Israël est la vigne de Yahweh des armées, et les hommes de Juda sont la plante en laquelle il prenait plaisir. Il en attendait de la droiture, et voici du saccagement ! De la justice, et voici des cris de détresse !
\TextTitle{Six malheurs en punition de l'infidélité d'Israël}
\VS{8}Malheur à ceux qui ajoutent maison à maison, et qui joignent champ à champ, jusqu'à ce qu'il n'y ait plus d'espace et qu'ils habitent seuls au milieu du pays.
\VS{9}Yahweh des armées m'a fait entendre : Certainement, ces maisons nombreuses seront réduites en désolation, ces maisons grandes et belles seront sans habitants.
\VS{10}Même dix arpents de vigne ne produiront qu'un bath, et un homer de semence ne produira qu'un épha.
\VS{11}Malheur à ceux qui se lèvent de bon matin, qui recherchent les boissons fortes, qui demeurent jusqu'au soir, et jusqu'à ce que le vin les échauffe !
\VS{12}La harpe et le luth, le tambourin, la flûte et le vin sont dans leurs festins ; mais ils ne regardent pas l'œuvre de Yahweh, et ils ne voient pas l'ouvrage de ses mains.
\VS{13}C'est pourquoi mon peuple sera emmené captif, parce qu'il n'a pas de connaissance\FTNT{2 R. 24:14-16 ; Os. 4:6.} ; et les plus honorables parmi eux seront des pauvres qui mourront de faim, et leur multitude sera asséchée par la soif.
\VS{14}C'est pourquoi le scheol s'élargit, il ouvre sa gueule outre mesure ; et sa magnificence y descend, sa multitude, sa pompe et tous ceux qui s'y réjouissent.
\VS{15}Ceux du commun seront abattus, les personnes de qualité seront humiliées, et les yeux des hautains seront humiliés.
\VS{16}Et Yahweh des armées sera haut élevé en jugement, et le Dieu saint sera sanctifié dans la justice.
\VS{17}Les agneaux paîtront selon qu'ils seront parqués, et les étrangers dévoreront les champs désolés des riches.
\VS{18}Malheur à ceux qui tirent l'iniquité avec des cordes de vanité, et le péché avec les traits d'un char,
\VS{19}et qui disent : Qu'il hâte et qu'il fasse venir son œuvre bientôt, afin que nous la voyions ! Que le conseil du Saint d'Israël s'avance et vienne, afin que nous le connaissions !
\VS{20}Malheur à ceux qui appellent le mal bien et le bien mal\FTNT{Mi. 7:2.} ; qui font les ténèbres lumière, et la lumière ténèbres ; qui font l'amertume douceur, et la douceur amertume.
\VS{21}Malheur à ceux qui sont sages à leurs yeux, en se considérant eux-mêmes intelligents !
\VS{22}Malheur à ceux qui sont forts pour boire le vin et vaillants pour mêler des boissons fortes ;
\VS{23}qui justifient le méchant pour des présents, et qui ôtent à chacun des justes sa justice.
\VS{24}C'est pourquoi, comme le flambeau de feu consume le chaume, et la flamme consume l'herbe sèche, ainsi leur racine sera comme la pourriture, et leur fleur sera détruite comme la poussière ; parce qu'ils ont rejeté la loi de Yahweh des armées, et ils ont méprisé la parole du Saint d'Israël.
\VS{25}C'est pourquoi la colère de Yahweh s'enflamme contre son peuple, il étend sa main sur lui, et il le frappe ; les montagnes tremblent, et leurs cadavres ont été mis en pièces au milieu des rues. Malgré tout cela, sa colère ne se détourne pas, mais sa main est encore étendue.
\VS{26}Il élève une bannière pour les nations éloignées, et il siffle à chacune d'elles depuis les extrémités de la terre ; et voici chacune viendra promptement et légèrement.
\VS{27}Nul n'est fatigué, nul ne chancelle de lassitude, personne ne sommeille ni ne dort ; et la ceinture de leurs reins ne sera point déliée, et la courroie de leurs souliers ne sera point rompue.
\VS{28}Leurs flèches sont aiguës et tous leurs arcs tendus ; les sabots de leurs chevaux ressemblent à des cailloux, et les roues de leurs chars à un tourbillon.
\VS{29}Leur rugissement est comme celui d'un vieux lion ; ils rugissent comme des lionceaux ; ils grondent et saisissent la proie, il l'emportent et personne ne vient à son secours.
\VS{30}En ce jour-là, on mènera un bruit sur lui, semblable au mugissement de la mer ; en regardant la terre, on ne verra que ténèbres et détresse ; la lumière sera obscurcie dans le ciel.
\Chap{6}
\TextTitle{Révélation de Yahweh à Esaïe}
\VerseOne{}L'année de la mort du roi Ozias, je vis le Seigneur assis sur un trône haut et élevé, et les pans de sa robe remplissaient le temple\FTNT{2 Ch. 26:23.}.
\VS{2}Les séraphins se tenaient au-dessus de lui ; et chacun d'eux avait six ailes ; deux dont ils se couvraient la face, deux dont ils se couvraient les pieds et deux dont ils se servaient pour voler.
\VS{3}Et ils criaient l'un à l'autre, et disaient : Saint, saint, saint est Yahweh des armées ! Toute la terre est pleine de sa gloire !
\VS{4}Et les poteaux des seuils furent ébranlés dans leurs fondements par la voix de celui qui criait ; et la maison fut remplie de fumée.
\VS{5}Alors je dis : Malheur à moi ! Je suis perdu, car je suis un homme dont les lèvres sont impures, j'habite au milieu d'un peuple dont les lèvres sont impures et mes yeux ont vu le Roi, Yahweh des armées\FTNT{Jg. 13:21-22.}.
\VS{6}Mais l'un des séraphins vola vers moi, tenant à la main un charbon ardent, qu'il avait pris sur l'autel avec des pincettes.
\VS{7}Il en toucha ma bouche, et dit : Voici, ceci a touché tes lèvres, c'est pourquoi ton iniquité est ôtée, et la propitiation est faite pour ton péché.
\VS{8}Puis j'entendis la voix du Seigneur, disant : Qui enverrai-je et qui marchera pour nous ? Je répondis : Me voici, envoie-moi.
\TextTitle{Mission d'Esaïe}
\VS{9}Et il dit : Va et dis à ce peuple : En entendant vous entendrez, mais vous ne comprendrez point ; et en voyant vous verrez, mais vous n'apercevrez point.
\VS{10}Engraisse le cœur de ce peuple, et rends ses oreilles pesantes, et bouche-lui les yeux ; de peur qu'il ne voie de ses yeux, et qu'il n'entende de ses oreilles, et que son cœur ne comprenne, et qu'il ne se convertisse, et qu'il ne recouvre la santé\FTNT{Mt. 13:15 ; Mc. 4:12 ; Jn. 12:40 ; Ac. 28:27.}.
\VS{11}Je dis : Jusqu'à quand, Seigneur ? Et il répondit : Jusqu'à ce que les villes soient dévastées, jusqu'à ce qu'il n'y ait plus d'habitants, ni d'hommes dans les maisons, et que la terre soit mise en entière désolation ;
\VS{12} et que Yahweh ait dispersé au loin les hommes, et que l'abandon ait été grand au milieu du pays.
\VS{13}Toutefois s'il y reste un dixième des habitants, ils reviendront pour être la proie des flammes. Mais comme le térébinthe et le chêne conservent leur tronc quand ils sont abattus, une sainte postérité renaîtra de ce peuple\FTNT{Ro. 11:17-25.}.
\Chap{7}
\TextTitle{Retsin et Pékach complote contre Juda}
\VerseOne{}Or il arriva du temps d'Achaz, fils de Jotham, fils d'Ozias, roi de Juda, que Retsin, roi de Syrie, et Pékach, fils de Remalia, roi d'Israël, montèrent contre Jérusalem pour lui faire la guerre ; mais ils ne purent l'assiéger.
\VS{2}Et on rapporta à la maison de David : La Syrie s'est reposée sur Ephraïm. Et le cœur d'Achaz, et le cœur de son peuple furent ébranlés comme les arbres des forêts qui sont ébranlés par le vent.
\VS{3}Alors Yahweh dit à Esaïe : Sors maintenant au devant d'Achaz, toi et Schear-Jaschub, ton fils, vers l'extrémité de l'aqueduc de l'étang supérieur, sur la route du champ du foulon.
\VS{4}Et dis-lui : Prends garde à toi, et demeure tranquille, ne crains point, et que ton cœur ne devienne point lâche à cause des deux queues de ces tisons fumants, à cause de l'ardeur, dis-je, de la colère de Retsin et de la Syrie, et du fils de Remalia,
\VS{5}de ce que la Syrie délibère avec Ephraïm et le fils de Remalia de te faire du mal, en disant :
\VS{6}Montons contre Juda, assiégeons la ville, battons-la en brèche, et établissons pour roi le fils de Tabeel au milieu d'elle.
\VS{7}Ainsi parle le Seigneur, Yahweh : Cela n'aura point d'effet, et cela ne se fera point.
\VS{8}Car la tête de la Syrie c'est Damas, et le chef de Damas c'est Retsin. Encore soixante-cinq ans, Ephraïm sera froissé pour n'être plus un peuple.
\VS{9}Et la tête d'Ephraïm c'est la Samarie, et le chef de la Samarie c'est le fils de Remalia. Si vous ne croyez pas, certainement vous ne serez point affermis.
\TextTitle{Annonce de la naissance d'Emmanuel}
\VS{10}Et Yahweh parla de nouveau à Achaz, en disant :
\VS{11}Demande pour toi un signe à Yahweh ton Dieu, demande-le, soit dans les bas lieux, soit dans les lieux élevés.
\VS{12}Et Achaz répondit : Je ne demanderai rien, et je ne tenterai point Yahweh.
\VS{13}Alors Esaïe dit : Ecoutez maintenant, ô maison de David ! Est-ce trop peu pour vous de lasser les hommes, que vous lassiez aussi mon Dieu ?
\VS{14}C'est pourquoi le Seigneur lui-même vous donnera un signe : Voici, une vierge sera enceinte, et elle enfantera un fils, et elle lui donnera le nom d'Emmanuel\FTNT{Le nom « Emmanuel » est dérivé de l'hébreu « Immanuw'el » qui signifie « Dieu est avec nous ». Jésus a dit aux disciples dans Mt. 28:20 : « Et moi, je suis avec vous tous les jours jusqu'à la fin des temps ». Jésus est Emmanuel, Dieu avec nous jusqu'à la fin des temps.}.
\VS{15}Il mangera du lait et du miel, jusqu'à ce qu'il sache rejeter le mal et choisir le bien.
\VS{16}Mais avant que l'enfant sache rejeter le mal et choisir le bien, la terre que tu as en détestation sera abandonnée par ses deux rois.
\TextTitle{Prophétie sur l'imminente invasion de Juda\FTNTT{2 Ch. 28:1-20.}}
\VS{17}Yahweh fera venir sur toi, sur ton peuple et sur la maison de ton père, par le roi d'Assyrie, des jours tels qu'il n'y en a point eu de semblable depuis le jour où Ephraïm s'est séparé de Juda.
\VS{18}Et il arrivera qu'en ce jour-là, Yahweh sifflera aux mouches qui sont à l'extrémité des ruisseaux d'Egypte, et aux abeilles qui sont au pays d'Assyrie.
\VS{19}Elles viendront, et se poseront dans toutes les vallées désertes, et dans les fentes des rochers, et par tous les buissons, et par tous les halliers.
\VS{20}En ce jour-là, le Seigneur rasera avec le rasoir pris à louage au-delà du fleuve, avec le roi d'Assyrie, la tête et les poils des pieds, et il enlèvera aussi la barbe\FTNT{2 R. 16:5-9.}.
\VS{21}Et il arrivera, en ce jour-là, qu'un homme nourrira une jeune vache et deux brebis.
\VS{22}Et il arrivera que de l'abondance du lait qu'elles rendront, il mangera du beurre ; car tous ceux qui seront restés dans le pays mangeront du beurre et du miel.
\VS{23}Et il arrivera, en ce jour-là, que tout lieu où il y aura mille vignes, valant mille sicles d'argent, sera réduit en ronces et en épines.
\VS{24}On y entrera avec des flèches et avec l'arc, car tout le pays ne sera que ronces et épines.
\VS{25}Et dans toutes les montagnes que l'on cultivait avec la bêche, on ne craindra plus de voir des ronces et des épines ; mais on y lâchera les bœufs, et la brebis en foulera le sol.
\Chap{8}
\TextTitle{Annonce de la défaite de Damas et de la Samarie}
\VerseOne{}Et Yahweh me dit : Prends un grand rouleau et écris dessus en grosses lettres : Qu'on se dépêche de butiner, qu'on se hâte de piller.
\VS{2}Et je pris avec moi des témoins fidèles : Urie, le sacrificateur, et Zacharie, fils de Bérékia.
\VS{3}Puis je m'étais approché de la prophétesse ; elle conçut et elle enfanta un fils. Et Yahweh me dit : Donne-lui pour nom Maher-Schalal-Chasch-Baz\FTNT{« Maher-Schalal-Chasch-Baz » signifie « rapide au butin, rapide sur la proie ».}.
\VS{4}Car avant que l'enfant sache dire : Mon père ! Ma mère ! On enlèvera la puissance de Damas et le butin de Samarie, devant le roi d'Assyrie.
\VS{5}Et Yahweh continua encore de me parler, en disant :
\VS{6}Parce que ce peuple a rejeté les eaux de Siloé qui coulent doucement, et qu'il s'est réjoui au sujet de Retsin, et du fils de Remalia,
\VS{7}à cause de cela, voici, le Seigneur va faire monter contre eux les puissantes et grandes eaux du fleuve : Le roi d'Assyrie et toute sa gloire. Il s'élèvera partout au-dessus de son lit, et il se répandra sur toutes ses rives.
\VS{8}Et il pénétrera dans Juda, il débordera et inondera, il atteindra jusqu'au cou. Et les étendues de ses ailes rempliront la largeur de ton pays, ô Emmanuel !
\TextTitle{Exhortation aux disciples de Yahweh à rester fidèles}
\VS{9}Alliez-vous, peuples ! Et vous serez brisés ; prêtez l'oreille, vous tous qui êtes d'un pays éloigné ! Equipez-vous, et vous serez brisés ; équipez-vous, et vous serez brisés.
\VS{10}Prenez conseil, et il sera dissipé ; dites la parole, et elle sera sans effet : Car Dieu est avec nous.
\VS{11}Car ainsi m'a parlé Yahweh, avec une main forte, et il m'instruisit de ne point aller par le chemin de ce peuple-ci, en me disant :
\VS{12}Ne dites point : Conjuration, toutes les fois que ce peuple dit conjuration ; ne craignez point ce qu'il craint, et ne vous en épouvantez point.
\VS{13}Sanctifiez Yahweh des armées, lui-même, c'est lui que vous devez craindre et redouter.
\VS{14}Et il sera un sanctuaire, mais aussi une pierre d'achoppement\FTNT{Yahweh s'est présenté comme une pierre d'achoppement et un rocher de scandale. En Es. 44:8 il affirme d'ailleurs ne pas connaître d'autre rocher que lui. Esaïe n'est pas le seul prophète à qui le Seigneur s'est révélé comme étant une pierre et un rocher. Dans le Ps. 118:22-23, il est dit : « La pierre qu'ont rejetée ceux qui bâtissaient est devenue la principale de l'angle ». Daniel et Zacharie ont également prophétisé au sujet de cette pierre : « Tu regardais, lorsqu'une pierre se détacha sans le secours d'aucune main, frappa les pieds de fer et d'argile de la statue, et les mit en pièces. Mais la pierre qui avait frappé la statue devint une grande montagne, et remplit toute la terre » (Da. 2:34-35). « Car voici, pour ce qui est de la pierre que j'ai placée devant Josué, il y a sept yeux sur cette seule pierre ; voici, je graverai moi-même ce qui doit y être gravé, dit Yahweh des armées; et j'enlèverai l'iniquité de ce pays, en un jour » (Za. 3:9). Ces prophéties se sont accomplies en Jésus-Christ, l'Agneau de Dieu qui ôte le péché du monde (Jn. 1:29). Le Seigneur s'est d'ailleurs clairement identifié à la pierre angulaire, affirmant ainsi sa divinité (Lu. 20:17-19). En Mt. 16:18, il s'est présenté comme le rocher inébranlable sur lequel il allait bâtir son Eglise. De plus, il est à noter que dans le livre de l'Apocalypse, l'Agneau possède sept yeux comme la pierre vue par Zacharie (Ap. 5:6). Ces sept yeux sont aussi les sept lampes du chandelier d'or que Zacharie et Jean avaient également vues (Za. 4:2 ; Ap. 4:5 ). Or le chiffre sept symbolise la plénitude et la perfection divines. Esaïe prophétisa encore en ces termes : « Voici, j'ai mis pour fondement en Sion une pierre, une pierre éprouvée, une pierre angulaire de prix, solidement posée; celui qui la prendra pour appui n'aura point hâte de fuir » (Es. 28:16). Les écrits de la Nouvelle Alliance attestent l'accomplissement de cette prophétie en Jésus-Christ, notamment par la bouche de Paul et de Pierre : « Vous avez été édifiés sur le fondement des apôtres et des prophètes, Jésus-Christ lui-même étant la pierre angulaire » (Ep. 2:20). « Car personne ne peut poser un autre fondement que celui qui a été posé, savoir Jésus-Christ » (1 Co. 3:11). « Approchez-vous de lui, pierre vivante, rejetée par les hommes, mais choisie et précieuse devant Dieu ; et vous-mêmes, comme des pierres vivantes, édifiez-vous pour former une maison spirituelle, un saint sacerdoce, afin d'offrir des victimes spirituelles, agréables à Dieu, par Jésus-Christ ». (1 Pi. 2:4-5).}, un rocher de scandale pour les deux maisons d'Israël, un filet et un piège pour les habitants de Jérusalem.
\VS{15}Plusieurs d'entre eux trébucheront, ils tomberont et se briseront, ils seront enlacés et pris.
\VS{16}Enveloppe ce témoignage, scelle cette loi\FTNTT{"towrah" ou "torah" en hébreu.} parmi mes disciples.
\VS{17}Je m'attends à Yahweh, qui cache sa face à la maison de Jacob, et je regarde à lui.
\VS{18}Me voici, avec les enfants que Yahweh m'a donnés, pour être un signe et un miracle en Israël, de la part de Yahweh des armées, qui habite sur la montagne de Sion.
\VS{19}Si l'on vous dit : Consultez ceux qui évoquent les morts et les diseurs de bonne aventure, qui poussent des sifflements et des soupirs, répondez : Un peuple ne consultera-t-il pas son Dieu ? S'adressera-t-il aux morts en faveur des vivants ?
\VS{20}A la loi et au témoignage ! Si l'on ne parle pas ainsi, il n'y aura certainement point d'aurore pour le peuple.
\VS{21}Et il sera errant dans le pays, accablé et affamé ; et il arrivera que dans sa faim, il s'irritera, maudira son roi et son Dieu, et tournera les yeux en haut ;
\VS{22}puis il regardera vers la terre, et voici, il n'y aura que détresse, ténèbres et de sombres angoisses : Il sera enfoncé dans l'obscurité.
\VS{23}Mais l'obscurité ne sera pas autant qu'elle avait été dans son humiliation ; quand au commencement, il affligea légèrement le pays de Zabulon et le pays de Nephthali, et ensuite, l'affligea plus sévèrement près de la mer, au-delà du Jourdain, dans la Galilée des Gentils.
\Chap{9}
\TextTitle{Annonce de la naissance et du règne du Messie}
\VerseOne{}Le peuple qui marchait dans les ténèbres voit une grande lumière, et la lumière resplendit sur ceux qui habitaient le pays de l'ombre de la mort\FTNT{Mt. 4:15-16.}.
\VS{2}Tu multiplies la nation, tu lui accordes de grandes joies, ils se réjouissent devant toi, comme on se réjouit à la moisson, comme on s'égaye quand on partage le butin.
\VS{3}Car tu as mis en pièces le joug dont il était chargé, et le bâton dont on lui battait ordinairement les épaules, et la verge de celui qui l'opprimait, comme au jour de Madian.
\VS{4}Parce que toute bataille de guerrier se fait dans un bruit confus, et que le vêtement est vautré dans le sang ; mais ceci sera comme un embrasement, quand le feu dévore quelque chose.
\VS{5}Car un enfant nous est né, un Fils nous a été donné\FTNT{Jésus-Christ est 100\% Dieu  et 100\% homme. Il  existe depuis toute éternité en tant que Dieu. Il est devenu homme au moment de son incarnation (Ph. 2 :5-7).}, et l'empire reposera sur son épaule : On l'appellera l'Admirable, le Conseiller, le Dieu Puissant, le Père d'éternité\FTNT{Philippe, disciple de Jésus-Christ voulait rencontrer le Père. Il posa au Seigneur cette question « Seigneur, montre-nous le Père, et cela nous suffit » (Jn. 14:8). Jésus lui répondit : « Il y a si longtemps que je suis avec vous, et tu ne m'as pas connu, Philippe ! » (Jn. 14:9).}, le Prince de paix,
\VS{6}pour accroître l'empire, et une paix sans fin au trône de David et à son royaume, pour l'affermir et le soutenir par le droit et par la justice, dès maintenant et à toujours\FTNT{Lu. 1:32-33.}. Voilà ce que fera le zèle de Yahweh des armées.
\TextTitle{Jugement sur le royaume du nord}
\VS{7}Le Seigneur envoie une parole à Jacob, et elle tombe sur Israël\FTNT{Ge. 32:28.}.
\VS{8}Et tout le peuple en aura connaissance, Ephraïm et les habitants de Samarie, qui disent avec orgueil et avec un cœur hautain :
\VS{9}Des briques sont tombées, mais nous bâtirons en pierres de taille ; des sycomores ont été coupés, mais nous les changerons en cèdres.
\VS{10}Yahweh élèvera contre eux les ennemis de Retsin, et il armera les ennemis d'Israël ;
\VS{11}la Syrie à l'orient, et les Philistins à l'occident ; et ils dévoreront Israël à gueule ouverte. Malgré tout cela, sa colère ne s'apaise point, et sa main est encore étendue.
\VS{12}Parce que le peuple ne revient pas à celui qui le frappe, et il ne cherche pas Yahweh des armées.
\VS{13}A cause de cela Yahweh retranchera d'Israël en un seul jour la tête et la queue, la branche de palmier et le roseau.
\VS{14}L'ancien et le magistrat, c'est la tête ; et le prophète qui enseigne le mensonge, c'est la queue.
\VS{15}Ceux donc qui font croire à ce peuple qu'il est heureux sont des séducteurs\FTNT{1 Ti. 4:1 ; Tit. 1:10.}; et ceux qui se laissent diriger par eux se perdent.
\VS{16}C'est pourquoi le Seigneur ne saurait prendre plaisir à leurs jeunes hommes ni avoir pitié de leurs orphelins et de leurs veuves, car tous sont des hypocrites et des méchants, et toute bouche ne profère que des infamies. Malgré tout cela, sa colère ne s'apaise point et sa main est encore étendue.
\VS{17}Car la méchanceté consume comme un feu, elle dévore les ronces et les épines ; elle embrase l'épaisseur de la forêt, d'où s'élèvent des colonnes de fumée.
\VS{18}A cause de la fureur de Yahweh des armées, la terre est obscurcie, et le peuple est comme la proie du feu ; nul n'a compassion de son frère.
\VS{19}On pille à droite, et l'on a faim ; on dévore à gauche, et l'on n'est pas rassasié ; chacun mange la chair de son bras.
\VS{20}Manassé dévore Ephraïm, Ephraïm dévore Manassé, et ensemble ils fondent sur Juda. Malgré tout cela, sa colère ne s'apaise point, et sa main est encore étendue.
\Chap{10}
\VerseOne{}Malheur à ceux qui décrètent des ordonnances iniques, et à ceux qui écrivent pour ordonner l'oppression,
\VS{2}pour refuser la justice aux pauvres et ravir leur droit aux malheureux de mon peuple, afin d'avoir les veuves pour leur butin, et de piller les orphelins !
\VS{3}Et que ferez-vous au jour de la visitation, et de la ruine éclatante qui viendra de loin ? Vers qui fuirez-vous pour avoir du secours et où laisserez-vous votre gloire\FTNT{Os. 9:7 ; Mt. 24:17-21 ; Lu. 19:41-44.} ?
\VS{4}Les uns seront courbés parmi les prisonniers, les autres tomberont parmi les morts. Malgré tout cela, sa colère ne s'apaise point, et sa main est encore étendue.
\TextTitle{Jugement sur l'Assyrie}
\VS{5}Malheur à l'Assyrie, verge de ma colère ! La verge dans leur main c'est l'instrument de ma colère.
\VS{6}Je l'ai envoyé contre une nation impie, et je l'ai fait marcher contre le peuple de ma fureur, afin qu'il se livre au pillage et fasse du butin, pour qu'il le foule aux pieds comme la boue des rues.
\VS{7}Mais il n'en juge pas ainsi, et ce n'est pas là la pensée de son cœur ; il ne songe qu'à détruire, qu'à exterminer beaucoup de nations.
\VS{8}Car il dit : Mes princes ne sont-ils pas autant de rois ?
\VS{9}Calno n'est-elle pas comme Carkemisch ? Hamath n'est-elle pas comme Arpad ? Et Samarie n'est-elle pas comme Damas ?
\VS{10}Puisque ma main a soumis les royaumes qui avaient des idoles, où il y avait plus d'images taillées qu'à Jérusalem et à Samarie,
\VS{11}ne ferai-je pas aussi à Jérusalem et à ses dieux, comme j'ai fait à Samarie et à ses idoles ?
\VS{12}Mais il arrivera que, quand le Seigneur aura achevé toute son œuvre sur la montagne de Sion et à Jérusalem, je punirai le roi d'Assyrie pour le fruit de son cœur orgueilleux, et pour la gloire de ses regards hautains.
\VS{13}Parce qu'il dit : C'est par la force de ma main que j'ai agi, c'est par ma sagesse, car je suis intelligent ; j'ai reculé les bornes des peuples, et j'ai pillé ce qu'ils avaient de plus précieux ; et comme un homme vaillant, j'ai fait descendre ceux qui étaient assis.
\VS{14}Ma main a trouvé les richesses des peuples, comme on trouve un nid ; comme on rassemble des œufs délaissés, ainsi ai-je rassemblé toute la terre ; nul n'a remué l'aile, ni ouvert le bec, ni poussé un cri.
\VS{15}La hache se glorifie-t-elle envers celui qui s'en sert ? Ou la scie s'élève-t-elle au-dessus de celui qui la manie ? Comme si la verge faisait mouvoir celui qui la lève, et que le bâton se levait comme s'il n'était pas du bois !
\VS{16}C'est pourquoi le Seigneur, Yahweh des armées, enverra la maigreur sur ses hommes gras ; et sous sa gloire éclatera l'embrasement d'un feu.
\VS{17}Car la lumière d'Israël deviendra un feu, et son Saint une flamme qui embrasera et consumera ses épines et ses ronces tout en un jour ;
\VS{18}et il consumera la gloire de sa forêt et de ses campagnes, depuis l'âme jusqu'à la chair. Il en sera comme quand celui qui porte l'étendard est défait.
\VS{19}Le reste des arbres de sa forêt pourra être compté, et un enfant en écrirait le nombre.
\TextTitle{Conversion et délivrance du reste d'Israël}
\VS{20}Et il arrivera en ce jour-là, que le reste d'Israël et les réchappés de la maison de Jacob ne s'appuieront plus sur celui qui les frappait, mais ils s'appuieront avec confiance sur Yahweh, le Saint d'Israël.
\VS{21}Le reste se convertira, le reste, dis-je, de Jacob se convertira au Dieu puissant.
\VS{22}Car quand ton peuple, ô Israël, serait comme le sable de la mer, un reste seulement se convertira ; la destruction est résolue, elle fera déborder la justice.
\VS{23}Car la destruction qu'il a résolue, le Seigneur, Yahweh des armées, va l'exécuter au milieu de toute la terre.
\VS{24}C'est pourquoi ainsi parle le Seigneur, Yahweh des armées : Mon peuple qui habites en Sion, ne crains pas le roi d'Assyrie ; il te frappe de la verge, et il lève son bâton sur toi comme faisait l'Egypte.
\VS{25}Mais encore un peu de temps, un peu de temps, et le châtiment cessera, puis ma colère se tournera contre lui pour l'exterminer.
\VS{26}Et Yahweh des armées lèvera le fouet contre lui, comme il frappa Madian au rocher d'Oreb ; et de même qu'il leva son bâton sur la mer, il le lèvera aussi comme contre les Egyptiens.
\VS{27}En ce jour-là, son fardeau sera ôté de dessus ton épaule et son joug de dessus ton cou ; et l'onction fera rompre le joug.
\TextTitle{Défaite des Assyriens\FTNTT{Es. 35-36 ; 37.7.}}
\VS{28}Il marche sur Ajjath, traverse Migron et il met ses bagages à Micmasch.
\VS{29}Ils passent le défilé, ils couchent à Guéba ; Rama est effrayée ; Guibea de Saül prend la fuite.
\VS{30}Pousse des cris, fille de Gallim ! Malheur à toi Anathoth ! Prends garde Laïs !
\VS{31}Madména se disperse, les habitants de Guébim se sauvent en foule.
\VS{32}Encore un jour d'arrêt à Nob, et il menace de sa main la montagne de la fille de Sion, la colline de Jérusalem.
\VS{33}Voici, le Seigneur, Yahweh des armées, brise les rameaux avec force ; et ceux qui sont les plus hauts élevés sont coupés, et les hauts montés sont abaissés.
\VS{34}Et il taille avec le fer les lieux les plus épais de la forêt, et le Liban tombe sous le Puissant.
\Chap{11}
\TextTitle{Rétablissement du règne de David par le Messie}
\VerseOne{}Mais il sortira un rameau du tronc d'Isaï, et un rejeton naîtra de ses racines\FTNT{Mt. 1:6-16 ; Lu. 1:31-32 ; Ro. 15:12 ; Ap. 5:5.}.
\VS{2}L'Esprit de Yahweh reposera sur lui, Esprit de sagesse et d'intelligence, Esprit de conseil et de force, Esprit de connaissance et de crainte de Yahweh\FTNT{Es. 61:1 ; Lu. 4:18}.
\VS{3}Il respirera la crainte de Yahweh, il ne jugera point sur l'apparence et il ne reprendra point sur un ouï-dire\FTNT{Jé. 11:20 ; Mt. 22:16 ; Ap. 2:23.}.
\VS{4}Mais il jugera les pauvres avec justice, et il prononcera avec droiture un jugement sur les malheureux de la terre, et il frappera la terre par la verge de sa bouche, et il fera mourir le méchant par le souffle de ses lèvres\FTNT{Job. 4:9 ; Job. 15:30 ; 2 Thess. 2:8.}.
\VS{5}La justice sera la ceinture de ses reins, et la fidélité, la ceinture de ses flancs\FTNT{Ep. 6:14.}.
\VS{6}Le loup habitera avec l'agneau, et le léopard se couchera avec le chevreau ; le veau, et le lionceau, et le bétail qu'on engraisse seront ensemble, et un petit enfant les conduira.
\VS{7}La jeune vache paîtra avec l'ourse, leurs petits auront un même gîte, et le lion, comme le bœuf, mangera de la paille\FTNT{Es. 65:25.}.
\VS{8}Le nourrisson s'ébattra sur l'antre de l'aspic, et l'enfant sevré mettra sa main dans la caverne du vipère.
\VS{9}Il ne se fera ni tort ni dommage sur toute ma montagne sainte, car la terre sera remplie de la connaissance de Yahweh, comme le fond de la mer des eaux qui le couvrent.
\VS{10}En ce jour-là, les nations rechercheront le rejeton d'Isaï qui sera comme une bannière\FTNT{Jésus, notre bannière doit être élevé afin que les pécheurs soient sauvés (Jn. 3:14-15). Le livre du Cantique des cantiques est une superbe image de l'amour de Dieu manifesté en Jésus-Christ, pour nous son Église, qui sommes sa bien-aimée. « Il m'a fait entrer dans la maison du vin ; et la bannière qu'il déploie sur moi, c'est l'amour » (Ca. 2:4). Le vin dont il est question c'est le Saint-Esprit que le Seigneur déverse sur nous jour après jour et qui nous désaltère spirituellement. « Moïse bâtit un autel, et lui donna pour nom : Yahweh ma bannière » (Ex. 17:15). Autrefois, lors des combats, les différentes armées portaient bien haut leur bannière en tête des troupes pour témoigner de leur appartenance et pour indiquer pour quel pays elles combattaient. En portant en nous Jésus, nous proclamons notre appartenance à Dieu et à son Royaume. Le Ps. 20:6 dit ceci : « Nous nous réjouirons de ton salut, nous lèverons l'étendard au nom de notre Dieu ; Yahweh exaucera tous tes vœux ». Et dans le Ps. 60:6 il est dit : « Tu as donné à ceux qui te craignent une bannière pour qu'elle s'élève à cause de la vérité ». La vérité se trouve en Jésus-Christ qui est le chemin, la vérité et la vie (Jn. 14:6). En élevant Jésus comme notre bannière, nous proclamons l'œuvre parfaite accomplie à la croix. Jésus, notre bannière, est le point de rassemblement des chrétiens de tous horizons. Ce rassemblement forme le corps de Christ, l'Eglise dont nous sommes les membres. Tout comme les douze tribus d'Israël se réunissaient pour combattre, nous nous réunissons tous sous la même bannière. Jésus, notre bannière, est le signe de la victoire contre les puissances des ténèbres. En élevant le nom de Jésus comme une bannière, nous faisons fuir toute l'armée de Satan. Dans Jn. 12:32 le Seigneur nous dit : « Et moi, quand j'aurai été élevé de la terre, j'attirerai tous les hommes à moi ». Jésus a été élevé comme étant la bannière qui a réconcilié Dieu avec les pécheurs. Lorsque cette bannière est élevée, les pécheurs sont attirés vers Dieu, ils passent des ténèbres à la lumière, de la mort à la vie.} pour les peuples, et son séjour ne   sera que gloire.
\TextTitle{Etablissement du règne du Messie}
\VS{11}Et il arrivera en ce jour-là, que le Seigneur mettra encore sa main une seconde fois pour acquérir le reste de son peuple dispersé en Assyrie, en Egypte, à Pathros, en Ethiopie, à Elam, à Schinear, à Hamath et dans les îles de la mer.
\VS{12}Il élèvera une bannière parmi les nations, il rassemblera les exilés d'Israël qui auront été chassés, et il recueillera les dispersés de Juda des quatre extrémités de la terre.
\VS{13}Et la jalousie d'Ephraïm sera ôtée, et les oppresseurs de Juda seront retranchés ; Ephraïm ne sera plus jaloux de Juda, et Juda n'opprimera plus Ephraïm.
\VS{14}Mais ils voleront sur l'épaule des Philistins vers la mer ; ils pilleront ensemble les fils de l'orient ; Edom et Moab seront la proie de leurs mains et les enfants d'Ammon leur obéiront.
\VS{15}Yahweh exterminera aussi à la façon de l'interdit la langue de la mer d'Egypte, et il lèvera sa main contre le fleuve par la force de son vent, et il le frappera sur les sept rivières, et fera qu'on y marche avec des souliers.
\VS{16}Et il y aura un chemin pour le reste de son peuple, qui sera échappé de l'Assyrie, comme il y en eut un pour Israël le jour où il remonta du pays d'Egypte.
\Chap{12}
\TextTitle{Louange au sein du royaume}
\VerseOne{}Tu diras en ce jour-là : Je te loue, ô Yahweh ! Car tu as été irrité contre moi, ta colère s'est apaisée, et tu m'as consolé.
\VS{2}Voici, Dieu est ma délivrance, j'aurai confiance et je ne craindrai rien ; car Yahweh, Yahweh est ma force et ma louange ; il est mon Sauveur.
\VS{3}Et vous puiserez de l'eau avec joie aux sources du salut\FTNT{Jn. 4:10-14.},
\VS{4}et vous direz en ce jour-là : Louez Yahweh, invoquez son Nom, publiez ses œuvres parmi les peuples, rappelez que son Nom est une haute retraite !
\VS{5}Psalmodiez à Yahweh car il a fait des choses magnifiques : Cela est connu dans toute la terre !
\VS{6}Habitante de Sion, égaye-toi, et réjouis-toi avec chant de triomphe ! Car le Saint d'Israël est grand au milieu de toi.
\Chap{13}
\TextTitle{Yahweh lève une armée}
\VerseOne{}Prophétie sur Babylone, révélé à Esaïe, fils d'Amots.
\VS{2}Elevez la bannière sur la haute montagne, élevez la voix vers eux, faites des signes avec la main, et qu'on entre dans les portes des magnifiques !
\VS{3}C'est moi qui ai donné des ordres à ceux qui me sont consacrés, j'ai appelé mes hommes forts pour exécuter ma colère, ceux qui se réjouissent de ma grandeur.
\VS{4}Il y a sur les montagnes un bruit d'une multitude, comme celui d'un grand peuple ; on entend un tumulte de royaumes, de nations rassemblées : Yahweh des armées passe en revue l'armée pour le combat.
\VS{5}D'un pays éloigné, de l'extrémité des cieux, Yahweh vient avec les instruments de sa colère pour détruire tout le pays.
\TextTitle{Jugement de Yahweh sur Babylone}
\VS{6}Hurlez, car le jour de Yahweh est proche, il vient comme un ravage du Tout-Puissant.
\VS{7}C'est pourquoi toutes les mains deviennent lâches, et tout cœur d'homme se fond.
\VS{8}Ils sont épouvantés ; les détresses et les douleurs les saisissent ; ils sont en travail comme celle qui enfante ; ils se regardent les uns les autres avec stupeur, leurs visages sont comme des visages enflammés.
\VS{9}Voici, le jour de Yahweh arrive, jour cruel, jour de colère et d'ardente fureur\FTNT{Mal. 4:1 ; Ap. 19:15}, qui réduira le pays en désolation, et en exterminera les pécheurs.
\VS{10}Même les étoiles des cieux et leurs astres ne feront plus briller leur lumière ; le soleil s'obscurcira dès son lever, et la lune ne fera plus resplendir sa lueur\FTNT{Joë. 2:31 ; Mt. 24:29 ; Mc. 13:24.}.
\VS{11}Je punirai le monde habitable à cause de sa malice, et les méchants à cause de leur iniquité ; je ferai cesser l'orgueil des hautains et j'abaisserai l'arrogance des tyrans.
\VS{12}Je ferai qu'un homme sera plus précieux que l'or fin, et une personne plus que l'or d'Ophir.
\VS{13}C'est pourquoi j'ébranlerai les cieux, et la terre sera secouée de sa base\FTNT{Ag. 2:6}, à cause de la fureur de Yahweh des armées, et à cause du jour de son ardente colère.
\VS{14}Et chacun sera comme un chevreuil qui est chassé, et comme une brebis que personne ne retire, chacun se tournera vers son peuple, chacun fuira vers son pays.
\VS{15}Quiconque sera trouvé, sera transpercé ; et quiconque s'y sera joint, tombera par l'épée.
\VS{16}Et leurs petits enfants seront écrasés sous leurs yeux\FTNT{Na. 3:10.}, leurs maisons seront pillées, et leurs femmes violées.
\TextTitle{Yahweh envoie les Mèdes contre Babylone}
\VS{17}Voici, je vais susciter contre eux les Mèdes, qui ne font point cas de l'argent, et qui ne convoitent point l'or.
\VS{18}Leurs arcs écraseront les jeunes gens, et ils seront sans pitié pour le fruit des entrailles, leur œil n'épargnera point les enfants.
\VS{19}Ainsi Babylone, l'ornement des royaumes, la parure et l'orgueil des Chaldéens, sera comme Sodome et Gomorrhe que Dieu détruisit.
\VS{20}Elle ne sera plus jamais habitée, elle ne sera point habitée de génération en génération ; même les Arabes n'y dresseront point leurs tentes, et les bergers n'y feront plus reposer leurs troupeaux.
\VS{21}Mais les bêtes sauvages des déserts y prendront leur gîte, et les hiboux rempliront ses maisons, les autruches en feront leur demeure, et les boucs y sauteront.
\VS{22}Les chacals hurleront dans ses palais, et les dragons dans ses maisons de plaisance. Son temps est près d'arriver et ses jours ne se prolongeront pas.
\Chap{14}
\TextTitle{Chant d'Israël après la chute de Babylone}
\VerseOne{}Car Yahweh aura pitié de Jacob, il choisira encore Israël, et il les rétablira dans leur terre ; les étrangers se joindront à eux et s'attacheront à la maison de Jacob.
\VS{2}Et les peuples les prendront, et les ramèneront à leur demeure, et la maison d'Israël les possédera en droit d'héritage sur la terre de Yahweh, comme serviteurs et comme servantes ; ils retiendront captifs ceux qui les avaient tenus captifs, et ils domineront sur leurs oppresseurs.
\VS{3}Et il arrivera qu'au jour que Yahweh fera cesser ton travail, ton tourment, et la dure servitude qui te fut imposée,
\VS{4}alors tu prononceras ce proverbe sur le roi de Babylone, et tu diras : Comment a-t-il fini le tyran ? Comment se repose celle qui était si avide de richesses ?
\VS{5}Yahweh a brisé le bâton des méchants, et la verge des dominateurs.
\VS{6}Celui qui frappait avec fureur les peuples de coups qu'on ne pouvait point détourner, qui dominait sur les nations avec colère, est poursuivi sans ménagement.
\VS{7}Toute la terre jouit du repos et de la paix ; on éclate en chants de triomphe à gorge déployée.
\VS{8}Même les cyprès et les cèdres du Liban se réjouissent de toi en disant : Depuis que tu es tombé, personne n'est monté pour nous abattre.
\TextTitle{Le roi de Babylone dépouillé de sa gloire}
\VS{9}Le scheol s'émeut jusque dans ses profondeurs, pour t'accueillir à ton arrivée ; il réveille à cause de toi les morts, et il fait lever de leurs sièges tous les principaux de la terre.
\VS{10}Tous prennent la parole pour te dire : Toi aussi, tu es sans force comme nous, tu es devenu semblable à nous !
\VS{11}Ta hauteur est descendue dans le scheol, avec le son de tes luths ; tu es couché sur une couche de vers, et la vermine est ta couverture.
\TextTitle{Orgueil, rébellion et chute de Satan}
\VS{12}Comment es-tu tombé du ciel, astre brillant, fils de l'aurore ? Toi qui foulais les nations, tu es abattu jusqu'à terre !
\VS{13}Tu disais en ton cœur : Je monterai aux cieux, je placerai mon trône au-dessus des étoiles de Dieu ; je m'assiérai sur la montagne de l'assemblée, du côté d'Aquilon\FTNT{Aquilon est un dieu des vents septentrionaux, froids et violents, dans la mythologie romaine.} ;
\VS{14}je monterai au dessus des hauts lieux des nuées, je serai semblable au Très-Haut.
\VS{15}Et cependant tu as été précipité dans le scheol, dans les profondeurs de la fosse\FTNT{Voir commentaire Ge. 1:1-2.}.
\VS{16}Ceux qui te voient fixent sur toi leurs regards, ils te considèrent attentivement, en disant : N'est-ce pas celui qui faisait trembler la terre, qui ébranlait les royaumes,
\VS{17}qui réduisait le monde habitable en désert, qui détruisait les villes, et ne relâchait pas ses prisonniers, ni ne les renvoyait chez eux ?
\TextTitle{Babylone anéantie}
\VS{18}Tous les rois des nations, oui, tous, reposent avec honneur, chacun dans sa maison.
\VS{19}Mais toi, tu as été jeté loin de ton sépulcre, comme un rejeton pourri, comme une dépouille de gens tués, transpercés avec l'épée, qu'on jette sous les pierres d'une fosse, comme un cadavre foulé aux pieds.
\VS{20}Tu ne seras point rangé comme eux dans le sépulcre, car tu as ravagé ta terre, tu as tué ton peuple. La race des méchants ne sera point renommée à toujours.
\VS{21}Préparez la tuerie pour ses enfants, à cause de l'iniquité de leurs pères ; afin qu'ils ne se relèvent point, et qu'ils n'héritent point la terre, et ne remplissent point de villes le dessus de la terre habitable.
\VS{22}Je m'élèverai contre eux, dit Yahweh des armées, et je retrancherai à Babylone le nom et le reste qu'elle a, ses descendants et sa postérité\FTNT{Ap. 14:8 ; Ap. 18:2.}, dit Yahweh.
\VS{23}J'en ferai l'habitation du butor et un marécage, et je la balayerai avec le balai de la destruction, dit Yahweh des armées.
\TextTitle{Jugement sur le roi d'Assyrie}
\VS{24}Yahweh des armées l'a juré, en disant : Certainement ce que j'ai décidé arrivera, ce que j'ai résolu s'accomplira.
\VS{25}Je briserai le roi d'Assyrie dans ma terre, je le foulerai aux pieds sur mes montagnes ; et son joug leur sera ôté, et son fardeau sera ôté de dessus leurs épaules.
\VS{26}C'est là le conseil arrêté contre toute la terre, c'est là la main étendue sur toutes les nations.
\TextTitle{Jugement sur le pays des Philistins}
\VS{27}Car Yahweh des armées l'a arrêté en son conseil : Qui l'empêchera ? Sa main est étendue : Qui la détournera\FTNT{Ec. 7:13.} ?
\VS{28}L'année de la mort du roi Achaz, cette prophétie fut prononcée : 
\VS{29}Ne te réjouis pas, toi pays des Philistins, de ce que la verge de celui qui te frappait est brisée ! Car de la racine du serpent sortira un vipère, et son fruit sera un serpent brûlant qui vole.
\VS{30}Alors les plus misérables seront repus, et les pauvres reposeront en assurance ; mais je ferai mourir de faim ta racine, et ce qui restera de toi sera tué.
\VS{31}Porte, hurle ! Ville, crie ! Tremble, pays tout entier des Philistins ! Car d'Aquilon, vient une fumée, et il ne restera pas un homme dans ses habitations.
\VS{32}Et que répondra-t-on aux envoyés de cette nation ? On répondra que Yahweh a fondé Sion, et que les affligés de son peuple y trouvent un refuge.
\Chap{15}
\TextTitle{Jugement sur Moab}
\VerseOne{}Prophétie sur Moab. La nuit même où elle est ravagée, Ar-Moab est détruite ! La nuit même où elle est saccagée, Kir-Moab est détruite !
\VS{2}Il monte à Bajith et à Dibon, dans les hauts lieux, pour pleurer ; Moab est en lamentations sur Nebo et sur Médeba : Toutes les têtes sont rasées et toutes les barbes sont coupées.
\VS{3}On sera couvert de sacs dans les rues ; chacun hurle, fondant en larmes sur ses toits et dans ses places\FTNT{Jé. 48:38.}.
\VS{4}Hesbon et Elealé poussent des cris, et l'on entend leur voix jusqu'à Jahats ; c'est pourquoi les guerriers de Moab se lamentent, ils ont l'effroi dans l'âme.
\VS{5}Mon cœur crie à cause de Moab, dont les fugitifs s'enfuient jusqu'à Tsoar, comme une génisse de trois ans ; car ils montent par la montée de Luchith avec des pleurs, et ils jettent des cris de détresse sur le chemin de Choronaïm.
\VS{6}Même les eaux de Nimrim ne sont que désolations, même le foin est déjà séché, l'herbe est consumée, et il n'y a point de verdure.
\VS{7}C'est pourquoi ils surveillent les richesses abondantes qu'ils ont acquises, afin que ce qu'ils ont réservé soit porté dans la vallée des saules.
\VS{8}Car les cris environnent les frontières de Moab, ses lamentations retentissent jusqu'à Eglaïm, ses lamentations retentissent jusqu'à Beer-Elim.
\VS{9}Même les eaux de Dimon sont pleines de sang ; car j'ajouterai un surcroît sur Dimon : Des lions contre les réchappés de Moab, et le reste du pays.
\Chap{16}
\TextTitle{Lamentation sur Moab}
\VerseOne{}Envoyez l'agneau au souverain du pays, envoyez-le du rocher du désert, à la montagne de la fille de Sion.
\VS{2}Car il arrivera que les filles de Moab seront au passage de l'Arnon, comme un oiseau volant ça et là, comme une nichée chassée de son nid.
\VS{3}Mets en avant le conseil, fais l'ordonnance, sers d'ombre comme une nuit au milieu de midi ; cache ceux qui ont été chassés, et ne trahis pas ceux qui sont errants.
\VS{4}Que ceux de mon peuple qui ont été chassés séjournent chez toi, ô Moab ! Sois pour eux un refuge contre le dévastateur ! Car celui qui use d'extorsion cessera, la dévastation finira, celui qui foule le pays sera consumé de dessus la terre.
\VS{5}Et le trône s'affermira par la clémence ; et sur ce trône sera assis en vérité, dans le tabernacle de David, un juge recherchant le droit, et se hâtant de faire justice\FTNT{Mi. 4:7 ; Da. 7:14 ; Lu. 1:33 ; Ap. 11:15}.
\VS{6}Nous avons entendu l'orgueil de Moab, le peuple extrêmement orgueilleux, sa fierté, son orgueil, son arrogance et ses vains discours.
\VS{7}C'est pourquoi Moab gémit sur Moab, chacun gémit ; vous soupirez pour les fondements de Kir-Haréseth, il n'y aura que des gens blessés à mort.
\VS{8}Car les campagnes de Hesbon et le vignoble de Sibma languissent ; les maîtres des nations ont foulé ses meilleurs ceps, qui s'étendaient jusqu'à Jaezer, qui couraient ça et là par le désert ; ses rameaux s'étendaient et passaient au-delà de la mer.
\VS{9}C'est pourquoi je pleure sur la vigne de Sibma, comme sur Jaezer ; je vous arrose de mes larmes, ô Hesbon et Elealé ! Car l'ennemi avec des cris s'est jeté sur tes fruits d'été et sur ta moisson.
\VS{10}Et la joie et l'allégresse se sont retirées du champ fertile ; on ne se réjouit plus et on ne s'égaye plus dans les vignes, le vendangeur ne foule plus dans les cuves, j'ai fait cesser la chanson de la vendange\FTNT{Jé. 48:31-34.}.
\VS{11}C'est pourquoi mes entrailles gémissent sur Moab, comme une harpe, et mon intérieur sur Kir-Harès.
\VS{12}Et on voit Moab qui se fatigue sur les hauts lieux ; il entre dans son sanctuaire pour prier mais il ne peut rien obtenir.
\VS{13}Telle est la parole que Yahweh a prononcée depuis longtemps sur Moab.
\VS{14}Et maintenant Yahweh a parlé, en disant : Dans trois ans, comme les années d'un mercenaire, la gloire de Moab sera avilie, avec toute cette grande multitude ; et le reste sera petit, ce sera peu de chose, ce ne sera rien de considérable.
\Chap{17}
\TextTitle{Prophétie sur la chute de Damas et de ses alliés}
\VerseOne{}Prophétie sur Damas. Voici, Damas est détruite pour ne plus être une ville, et elle ne sera qu'un monceau de ruines\FTNT{Jé. 49:23-27.}.
\VS{2}Les villes d'Aroër sont abandonnées, elles sont livrées aux troupeaux qui s'y reposent, et il n'y a personne qui les effraie.
\VS{3}Il n'y aura plus de forteresse en Ephraïm, ni de royaume à Damas et dans le reste de la Syrie ; ils seront comme la gloire des enfants d'Israël, dit Yahweh des armées.
\VS{4}Et il arrivera en ce jour-là que la gloire de Jacob sera affaiblie et la graisse de sa chair sera fondue.
\VS{5}Il en sera comme quand le moissonneur cueille les blés, et qu'il moissonne les épis avec son bras\FTNT{Joë. 3:13 ; Mt. 13:24-30.} ; comme quand on ramasse les épis dans la vallée de Rephaïm.
\VS{6}Mais il en restera quelques grappillages, comme quand on secoue l'olivier, et qu'il reste deux ou trois olives en haut de la cime, et qu'il y en a quatre ou cinq que l'olivier a produites dans ses branches fruitières, dit Yahweh, le Dieu d'Israël.
\VS{7}En ce jour-là, l'homme regardera vers celui qui l'a fait, et ses yeux se tourneront vers le Saint d'Israël.
\VS{8}Et il ne regardera plus vers les autels, qui sont l'ouvrage de ses mains, et il ne regardera plus ce que ses doigts ont fabriqué, ni les images d'Asherah, ni les statues du soleil.
\VS{9}En ce jour-là, ses villes fortes seront abandonnées à cause des enfants d'Israël, ils seront comme un bois taillis et des rameaux abandonnés, et ce sera un désert.
\VS{10}Parce que tu as oublié le Dieu de ton salut, et que tu ne t'es pas souvenue du rocher\FTNT{Voir commentaire Es. 8:13-14.} de ta force, à cause de cela tu as transplanté des plantes de plaisance, et tu as planté des ceps étrangers.
\VS{11}De jour tu as fais croître ce que tu as planté, et le matin tu as fait levé ta semence; mais la moisson a été enlevée au jour que l'on voulait en jouir, et il y a eu une douleur désespérée.
\VS{12}Malheur à la multitude de peuples nombreux, qui font un bruit comme le bruit des mers ; et à la tempête éclatante des nations, qui font du bruit comme une tempête éclatante d'eaux impétueuses !
\VS{13}Les nations font un bruit comme une tempête éclatante de grosses eaux, mais il les menace, et elles s'enfuient ; elles seront poursuivies comme la balle des montagnes chassée par le vent, et comme une boule poussée par un tourbillon.
\VS{14}Au temps du soir, voici une terreur soudaine ; mais avant le matin, ils ne sont plus ! C'est là le partage de ceux qui nous dépouillent, et le lot de ceux qui nous pillent.
\Chap{18}
\TextTitle{Jugement sur l'Ethiopie}
\VerseOne{}Malheur à la terre qui fait ombre avec des ailes, qui est au-delà des fleuves de l'Ethiopie ;
\VS{2}qui envoie par mer des messagers, dans des navires de jonc, voguant à la surface des eaux ! Allez, messagers rapides, vers la nation robuste et vigoureuse, vers le peuple redoutable, depuis là où il est et par delà ; nation puissante et qui écrase tout, et dont les fleuves ravagent son pays.
\VS{3}Vous tous, habitants du monde, et vous qui habitez dans le pays, quand la bannière sera élevée sur les montagnes, regardez ; et quand le shofar sonnera, écoutez !
\VS{4}Car ainsi m'a parlé Yahweh : Je me tiens tranquillement, et je regarde de ma demeure, par la chaleur de la lumière, et par la vapeur de la rosée, au temps de la chaude moisson.
\VS{5}Car avant la moisson, quand le bourgeon vient en sa perfection, et que la fleur devient un raisin qui mûrit, il coupe les sarments avec des serpes, il enlève les sarments, les ayant retranchés.
\VS{6}Ils seront tous ensemble abandonnés aux oiseaux de proie qui demeurent dans les montagnes, et aux bêtes de la terre ; les oiseaux de proie seront sur eux tout le long de l'été, et toutes les bêtes de la terre y passeront l'hiver.
\VS{7}En ce temps-là, un présent sera apporté à Yahweh des armées, par le peuple robuste et vigoureux, de la part, dis-je, du peuple terrible depuis là où il est et au-delà, nation puissante et qui écrase tout, et dont le pays est ravagé par ses fleuves ; il sera apporté dans la demeure du Nom de Yahweh des armées, sur la montagne de Sion.
\Chap{19}
\TextTitle{Chute de l'Egypte}
\VerseOne{}Prophétie sur l'Egypte. Voici, Yahweh est monté sur une nuée rapide, il entre en Egypte ; et les idoles d'Egypte s'enfuient de toutes parts devant sa face, et le cœur des Egyptiens se fond au milieu d'elle\FTNT{Jé. 43:12.}.
\VS{2}Et je ferai venir pêle-mêle l'Egyptien contre l'Egyptien, et chacun fera la guerre contre son frère, et chacun contre son ami, ville contre ville, et royaume contre royaume.
\VS{3}L'esprit de l'Egypte disparaîtra du milieu d'elle, et je dissiperai son conseil ; et ils consulteront les idoles et les enchanteurs, ceux qui évoquent les morts et ceux qui prédisent l'avenir.
\VS{4}Et je livrerai l'Egypte entre les mains d'un maître sévère ; et un roi cruel dominera sur eux, dit le Seigneur, Yahweh des armées.
\VS{5}Les eaux de la mer tariront, le fleuve séchera et tarira\FTNT{Jé. 51:36.}.
\VS{6}Et on fera détourner les fleuves ; les ruisseaux des digues s'abaisseront et sécheront ; les roseaux et les joncs seront coupés.
\VS{7}Les prairies qui sont près des ruisseaux, et sur l'embouchure du fleuve, tout ce qui aura été semé le long des ruisseaux, séchera, sera jeté au loin, et ne sera plus.
\VS{8}Et les pêcheurs gémiront, tous ceux qui jettent l'hameçon dans le fleuve mèneront deuil, et ceux qui étendent des filets sur les eaux languiront.
\VS{9}Ceux qui travaillent en fin lin et en fin crêpe, et ceux qui tissent les filets seront confus.
\VS{10}Les fondements du pays seront rompus, et tous ceux qui font des écluses de viviers auront l'âme attristée.
\VS{11}Certes les chefs de Tsoan ne sont que des insensés, les sages d'entre les conseillers de Pharaon forment un conseil stupide. Comment osez-vous dire à Pharaon : Je suis fils des sages, fils des anciens rois ?
\VS{12}Où sont-ils maintenant? Où sont, dis-je, tes sages ? Qu'ils t'annoncent, je te prie, s'ils le savent, ce que Yahweh des armées a décrété contre l'Egypte.
\VS{13}Les chefs de Tsoan sont devenus insensés, les chefs de Noph se sont trompés, les chefs des tribus font égarer l'Egypte.
\VS{14}Yahweh a versé au milieu d'elle un esprit de vertige\FTNT{1 R. 22:18-22.}, pour qu'ils fassent chanceler les Egyptiens dans toutes leurs actions, comme un homme ivre se vautre dans son vomissement.
\VS{15}Et l'Egypte sera hors d'état de faire ce que font la tête et la queue, la branche de palmier et le roseau.
\TextTitle{L'Egypte et l'Assyrie dans le royaume du Messie}
\VS{16}En ce jour-là, l'Egypte sera comme des femmes : Elle sera étonnée et épouvantée à cause de la main de Yahweh des armées, quand il élèvera la main contre elle.
\VS{17}Et la terre de Juda sera pour l'Egypte un objet d'effroi ; quiconque fera mention d'elle, en sera épouvanté en lui-même, à cause du conseil décrété contre elle par Yahweh des armées.
\VS{18}En ce jour-là, il y aura cinq villes au pays d'Egypte, qui parleront la langue de Canaan, et qui jureront par Yahweh des armées ; l'une sera appelée ville de la destruction.
\VS{19}En ce jour-là, il y aura un autel à Yahweh au milieu du pays d'Egypte, et un monument dressé à Yahweh sur la frontière.
\VS{20}Et ce sera un signe et un témoignage pour Yahweh des armées dans le pays d'Egypte ; car ils crieront à Yahweh à cause des oppresseurs, et il leur enverra un sauveur, quelqu'un de grand, et il les délivrera\FTNT{Es. 43:11.}.
\VS{21}Et Yahweh se fera connaître aux Egyptiens, et les Egyptiens connaîtront Yahweh en ce jour-là ; ils le serviront, ils offriront des sacrifices et des offrandes, et ils feront des vœux à Yahweh et les accompliront.
\VS{22}Ainsi Yahweh frappera les Egyptiens, il les frappera, mais il les guérira ; et ils retourneront à Yahweh, qui les exaucera et les guérira.
\VS{23}En ce jour-là, il y aura un chemin battu de l'Egypte en Assyrie ; et l'Assyrie viendra en Egypte, et l'Egypte en Assyrie, et l'Egypte servira avec l'Assyrie.
\VS{24}En ce même temps, Israël sera, lui troisième, uni à l'Egypte et à l'Assyrie, et la bénédiction sera au milieu de la terre.
\VS{25}Yahweh des armées les bénira, en disant : Bénis soit l'Egypte mon peuple, et l'Assyrie œuvre de mes mains, et Israël mon héritage !
\Chap{20}
\TextTitle{Conquête de l'Egypte et de l'Ethiopie}
\VerseOne{}L'année où Tharthan, envoyé par Sargon, roi d'Assyrie, vint et combattit contre Asdod, et la prit.
\VS{2}En ce temps-là, Yahweh parla par Esaïe, fils d'Amots, et lui dit : Va, délie le sac de dessus tes reins et ôte tes souliers de tes pieds. Il fit ainsi, marchant nu et déchaussé.
\VS{3}Puis Yahweh dit : De même que mon serviteur Esaïe marche nu et déchaussé, ce qui sera dans trois ans un signe et un prodige contre l'Egypte et contre l'Ethiopie,
\VS{4}de même le roi d'Assyrie emmènera de l'Egypte et de l'Ethiopie prisonniers et captifs les jeunes et les vieux, nus et déchaussés, ayant les hanches découvertes, ce qui sera l'opprobre de l'Egypte\FTNT{2 S. 10:4 ; Es. 3:17 ; Jé. 13:22-26.}.
\VS{5}Ils seront effrayés, et ils seront honteux à cause de l'Ethiopie à qui ils s'attendaient, et à cause de l'Egypte dont ils se glorifiaient.
\VS{6}Et les habitants de cette côte diront en ce jour-là : Voilà ce qu'est devenu le peuple à qui nous nous attendions, celui vers qui nous courions chercher du secours, afin d'être délivrés du roi d'Assyrie ! Comment pourrons-nous échapper ?
\Chap{21}
\TextTitle{Annonce de la conquête de Babylone}
\VerseOne{}Prophétie sur le désert de la mer. Il vient du désert, de la terre redoutable, comme des tourbillons qui s'élèvent au pays du midi pour traverser.
\VS{2}Une vision terrible m'a été révélée. Le traître demeure traître, celui qui saccage, saccage toujours. Monte, Elam ! Assiège, Médie ! Je fais cesser tous les soupirs.
\VS{3}C'est pourquoi mes reins sont remplis de douleur ; les angoisses me saisissent comme les douleurs de celle qui enfante ; je suis tourmenté à cause de ce que j'ai entendu, et j'ai été tout troublé à cause de ce que j'ai vu.
\VS{4}Mon cœur est agité de toutes parts, la terreur s'empare de moi ; la nuit de mes plaisirs devient une nuit de crainte.
\VS{5}Qu'on dresse la table, que la sentinelle veille, qu'on mange, qu'on boive! Levez-vous, chefs ! Oignez le bouclier !
\VS{6}Car ainsi m'a parlé le Seigneur : Va, place la sentinelle, et qu'elle rapporte ce qu'elle verra\FTNT{Ez. 33:1-19.}.
\VS{7}Et elle vit un char, un couple de cavaliers, un char tiré par des ânes, un char tiré par des chameaux ; et elle les considéra fort attentivement.
\VS{8}Et elle s'écria : C'est un lion ! Seigneur, je me tiens en sentinelle toute la journée et je suis à mon poste toutes les nuits ;
\VS{9}et voici venir le char d'un homme et un couple de cavaliers ! Alors elle parla et dit : Elle est tombée, elle est tombée, Babylone\FTNT{Prophétie sur la chute de Babylone. Voir Jé. 50 et 51 ; Ap.18.}, et toutes les images taillées de ses dieux sont brisées par terre.
\VS{10}C'est ce que j'ai foulé, et le grain que j'ai battu dans mon aire. Je vous ai annoncé ce que j'ai entendu de Yahweh des armées, du Dieu d'Israël.
\VS{11}Prophétie sur Duma. On me crie de Séir : Ô sentinelle ! Qu'en est-il de la nuit ? Ô sentinelle ! Qu'en est-il de la nuit ?
\VS{12}La sentinelle répond : Le matin vient et la nuit aussi. Si vous demandez, demandez. Retournez, venez.
\TextTitle{Jugement sur l'Arabie}
\VS{13}Prophétie contre l'Arabie. Vous passerez pêle-mêle la nuit dans la forêt, caravanes de Dedan !
\VS{14}Les habitants du pays de Théma portent de l'eau à ceux qui ont soif ; ils  viennent au-devant du fugitif avec du pain pour lui.
\VS{15}Car ils fuient devant les épées, devant l'épée dégainée, devant l'arc tendu, devant le fort de la bataille.
\VS{16}Car ainsi m'a parlé le Seigneur : Encore une année, comme les années d'un mercenaire, et toute la gloire de Kédar prendra fin.
\VS{17}Et le reste du nombre des forts archers des fils de Kédar sera diminué, car Yahweh, le Dieu d'Israël, a parlé.
\Chap{22}
\TextTitle{Malédiction sur la vallée des visions, Jérusalem}
\VerseOne{}Prophétie sur la vallée des visions. Qu'as-tu maintenant, que tu sois toute montée sur les toits ?
\VS{2}Toi ville bruyante, pleine de tumulte, ville joyeuse ! Tes blessés à morts ne seront pas blessés à mort par l'épée, et ils ne mourront pas par la guerre.
\VS{3}Tous tes chefs fuient ensemble, ils sont liés par les archers ; tous ceux des tiens qui sont trouvés sont liés ensemble tandis qu'ils s'enfuient au loin.
\VS{4}C'est pourquoi je dis : Détournez de moi vos regards, que je pleure amèrement. Ne vous empressez pas pour me consoler du désastre de la fille de mon peuple.
\VS{5}Car c'est le jour de trouble, d'oppression et de confusion\FTNT{Lam. 1:5 ; Lam. 2:2.}, envoyé par le Seigneur, Yahweh des armées, dans la vallée des visions. Il démolit la muraille et les cris retentissent jusqu'à la montagne.
\VS{6}Même Elam prend son carquois, il y a des hommes montés sur des chars et des cavaliers ; Kir découvre le bouclier.
\VS{7}Et tes plus belles vallées sont remplies de chars, et les cavaliers se rangent tous en bataille à tes portes.
\VS{8}Et on découvre ce qui couvrait Juda, et en ce jour là tu regardes vers les armes de la maison de la forêt.
\VS{9}Vous voyez que les brèches de la cité de David sont nombreuses ; et vous assemblez les eaux de l'étang inférieur.
\VS{10}Vous faites le dénombrement des maisons de Jérusalem, et vous démolissez les maisons pour fortifier la muraille.
\VS{11}Et vous faites aussi un réservoir d'eau entre les deux murailles, pour les eaux de l'ancien étang. Mais vous ne regardez pas à celui qui a fait ces choses, qui les a formées il y a longtemps.
\VS{12}Le Seigneur, Yahweh des armées, vous appelle ce jour-là aux pleurs et au deuil, à vous raser la tête, et à ceindre le sac\FTNT{Ez. 7:18 ; Joë 1:13}.
\VS{13}Et voici il y a de la joie et de l'allégresse ! On égorge des bœufs et l'on tue des moutons, on mange la viande et l'on boit du vin ; puis on dit : Mangeons et buvons, car demain nous mourrons\FTNT{Es. 56:12 ; 1 Co. 15:32.} !
\VS{14}Or il m'a été révélé à l'oreille, par Yahweh des armées : Sûrement cette iniquité ne vous sera pas pardonnée jusqu'à ce que vous mouriez, a dit le Seigneur, Yahweh des armées.
\TextTitle{Eliakim succède à Schebna}
\VS{15}Ainsi parle le Seigneur, Yahweh des armées : Va, entre chez ce trésorier, chez Schebna, gouverneur du palais et dis-lui :
\VS{16}Qu'as-tu à faire ici, et qu'as-tu ici qui t'appartienne, que tu te tailles ici un sépulcre ? Il taille un sépulcre en hauteur, il se taille une demeure dans le rocher.
\VS{17}Voici, ô homme ! Yahweh te chassera au loin d'un bras vigoureux ; il t'enveloppera entièrement.
\VS{18}Il te fera rouler fort vite, comme une balle sur une terre large et spacieuse ; là tu mourras, là seront les chars de ta gloire, ô toi qui es la honte de la maison de ton Seigneur !
\VS{19}Je te jetterai hors de ton rang, et on t'arrachera de ton service.
\VS{20}Et il arrivera en ce jour-là que j'appellerai mon serviteur Eliakim, fils de Hilkija.
\VS{21}Je le revêtirai de ta tunique, je le ceindrai de ta ceinture, et je remettrai ton autorité entre ses mains, il sera un père pour les habitants de Jérusalem et pour la maison de Juda.
\VS{22}Et je mettrai la clef de la maison de David sur son épaule ; et il ouvrira, et il n'y aura personne qui ferme ; et il fermera, et il n'y aura personne qui ouvre\FTNT{La clé de David est le symbole de l'autorité du Messie (Es. 9:5 ; Mt. 28:18 ; Ap. 3:7-8)}.
\VS{23}Je l'enfoncerai comme un clou dans un lieu sûr, et il sera un trône de gloire pour la maison de son père.
\VS{24}Et on y pendra toute la gloire de la maison de son père, de ses parents et de celles qui lui appartiennent ; tous les ustensiles des plus petites choses,  des bassins comme des vases. 
\VS{25}En ce jour-là, dit Yahweh des armées, le clou enfoncé dans un lieu sûr sera ôté ; et étant retranché il tombera, et le fardeau qui était sur lui sera retranché, car Yahweh a parlé.
\Chap{23}
\TextTitle{Effondrement de Tyr}
\VerseOne{}Prophétie sur Tyr. Hurlez, navires de Tarsis ! Car elle est détruite, il n'y a plus de maisons, on n'y entre plus ! Ceci leur a été révélé du pays de Kittim.
\VS{2}Vous qui habitez dans l'île, taisez-vous ! Toi qui étais remplie de marchands de Sidon, et de ceux qui traversaient la mer !
\VS{3}A travers les grandes eaux, les grains de Shichor, la moisson du Nil était pour elle son revenu ; elle était le marché des nations\FTNT{Ez. 27.}.
\VS{4}Sois honteuse, ô Sidon ! Car la mer, la forteresse de la mer, a parlé en disant : Je n'ai point eu de douleurs, je n'ai point enfanté, je n'ai point nourri de jeunes gens ni élevé aucune vierge.
\VS{5}Selon la nouvelle qui a été touchant l'Egypte, ainsi sera-t-on en travail quand on entendra la nouvelle touchant Tyr.
\VS{6}Passez à Tarsis, hurlez, vous qui habitez dans l'île !
\VS{7}N'est-ce pas ici votre ville joyeuse ? Elle avait une origine antique et ses propres pieds la mènent séjourner dans un pays étranger.
\VS{8}Qui a pris ce conseil contre Tyr, celle qui couronnait les siens, dont les marchands étaient des princes, et dont les trafiquants étaient les plus honorables de la terre\FTNT{Ap. 18:9-18.} ?
\VS{9}Yahweh des armées a pris ce conseil, pour flétrir l'orgueil de toute la noblesse, et pour avilir tous les honorables de la terre.
\VS{10}Traverse ton pays, comme une rivière, ô fille de Tarsis ! Il n'y a plus de ceinture.
\VS{11}Il a étendu sa main sur la mer, il a fait trembler les royaumes ; Yahweh a ordonné la destruction des forteresses de Canaan.
\VS{12}Il a dit : Tu ne te livreras plus à la joie, vierge opprimée, fille de Sidon ! Lève-toi, passe au pays de Kittim ! Même là, il n'y aura pas de repos pour toi.
\VS{13}Voilà le pays des Chaldéens ; ce peuple-là n'était pas autrefois ; Assur\FTNT{Assur : Le second fils de Sem (Ge. 10:22). L'ancêtre des Assyriens.} l'a fondé pour les gens du désert ; on a dressé ses forteresses, on a élevé ses palais, et il l'a mis en ruines.
\VS{14}Hurlez, navires de Tarsis ! Car votre force est détruite !
\VS{15}Et il arrivera en ce jour-là que Tyr tombera dans l'oubli durant soixante-dix ans, selon les jours d'un roi. Mais au bout de soixante-dix ans\FTNT{Jé. 25 : 11-12.}, on chantera une chanson à Tyr comme à une femme prostituée :
\VS{16}Prends la harpe, fais le tour de la ville, ô prostituée qu'on oublie ! Sonne avec force, chante et rechante, afin qu'on se ressouvienne de toi!
\VS{17}Et il arrivera au bout de soixante-dix ans que Yahweh visitera Tyr, mais elle retournera au salaire de sa prostitution, et elle se prostituera avec tous les royaumes de la terre, sur le dessus de la terre.
\VS{18}Mais son trafic et son salaire seront sanctifiés à Yahweh ; il n'en sera rien réservé, ni serré ; car son trafic sera pour ceux qui habitent dans la présence de Yahweh, pour en manger à satiété, et pour avoir des vêtements durables.
\Chap{24}
\TextTitle{Désastre après l'invasion babylonienne}
\VerseOne{}Voici, Yahweh s'en va rendre le pays vide et l'épuiser, il en renverse le dessus, et disperse ses habitants\FTNT{Ge. 11:1-8.}.
\VS{2}Et il en est du sacrificateur comme du peuple, du maître comme de son serviteur, de la dame comme de sa servante, du vendeur comme de l'acheteur, de celui qui prête comme de celui qui emprunte, du créancier comme du débiteur.
\VS{3}Le pays est entièrement vidé et entièrement pillé, car Yahweh a prononcé cet arrêt.
\VS{4}La terre mène le deuil, elle est déchue ; le pays habité est devenu languissant, il est déchu ; les plus distingués du peuple de la terre sont languissants.
\VS{5}Le pays était profané par ses habitants qui marchent sur lui ; car ils ont transgressé les lois, ils ont changé les ordonnances et ont enfreint l'alliance éternelle\FTNT{Da. 7:25.}.
\VS{6}C'est pourquoi la malédiction dévore le pays, et ses habitants portent la peine de leurs crimes ; c'est pourquoi les habitants du pays sont brûlés et il n'en reste qu'un petit nombre.
\VS{7}Le vin excellent pleure, la vigne languit, et tous ceux qui avaient le cœur joyeux soupirent.
\VS{8}La joie des tambours a cessé ; le bruit de ceux qui s'égayent a pris fin, la joie de la harpe a cessé.
\VS{9}On ne boit plus de vin en chantant ; les boissons fortes sont amères à ceux qui les boivent.
\VS{10}La ville confuse est en ruines ; toutes les maisons sont fermées, on n'y entre plus.
\VS{11}On crie dans les rues parce que le vin manque ; toute la joie est tournée en obscurité, l'allégresse du pays s'en est allée.
\VS{12}La désolation est restée dans la ville et la porte est frappée d'une ruine éclatante.
\VS{13}Car il arrivera au milieu de la terre et parmi les peuples, comme quand on secoue l'olivier, et comme quand on grappille après la vendange.
\TextTitle{Un reste de rescapés célèbre Yahweh}
\VS{14}Ils élèvent leur voix, ils se réjouissent avec chant de triomphe ; et s'égayent du côté de la mer, ils célèbrent la majesté de Yahweh.
\VS{15}C'est pourquoi glorifiez Yahweh dans les vallées, le Nom de Yahweh, le Dieu d'Israël, dans les îles de la mer !
\VS{16}De l'extrémité de la terre, nous entendons des cantiques à la gloire du Juste ; mais moi je dis : Maigreur sur moi ! Maigreur sur moi ! Malheur à moi ! Les perfides ont agi perfidement ; et ils ont imité la mauvaise foi des perfides.
\TextTitle{Manifestation des jugements de Yahweh}
\VS{17}La frayeur, la fosse, et le piège sont sur toi, habitant du pays !
\VS{18}Et il arrivera que celui qui fuit à cause du bruit de la frayeur tombe dans la fosse, et celui qui remonte hors de la fosse se prend au filet ; car les écluses d'en haut s'ouvrent et les fondements de la terre tremblent.
\VS{19}La terre est entièrement brisée, la terre s'écrase entièrement, la terre se remue de sa place.
\VS{20}La terre chancelle entièrement comme un homme ivre, elle est transportée comme une cabane ; son péché pèse sur elle, elle tombe et ne se relève plus.
\VS{21}Et il arrivera en ce jour là, que Yahweh punira dans le lieu élevé l'armée d'en haut, et sur la terre les rois de la terre.
\VS{22}Ils seront assemblés en troupes comme des prisonniers dans une fosse, et ils seront enfermés dans une prison, et après plusieurs jours ils seront visités.
\VS{23}La lune rougira et le soleil sera honteux quand Yahweh des armées régnera sur la montagne de Sion et à Jérusalem, resplendissant de gloire en présence de ses anciens\FTNT{Mt. 24:29-30 ; 2 Pi. 3:10-12 ; Ap. 6:12.}.
\Chap{25}
\TextTitle{Le royaume de Yahweh}
\VerseOne{}Ô Yahweh, tu es mon Dieu ; je t'exalterai, je célébrerai ton nom, car tu as fait des choses merveilleuses ; tes conseils conçus d'avance sont fidèlement accomplis.
\VS{2}Car tu as fait de la ville un monceau de pierres, et de la cité forte une ruine ; le palais des étrangers qui était dans la ville ne sera jamais rebâti.
\VS{3}C'est pourquoi le peuple fort te glorifie, la ville des nations redoutables te révère.
\VS{4}Parce que tu as été la force du faible, la force du misérable dans sa détresse, le refuge contre la tempête, l'ombrage contre la chaleur ; car le souffle des tyrans est comme la tempête qui abat une muraille.
\VS{5}Tu as rabaissé la tempête éclatante des étrangers ; comme la chaleur, dis-je, dans un pays sec, comme la chaleur par l'ombre d'une nuée, le branchage des tyrans sera abattu.
\VS{6}Et Yahweh des armées prépare à tous les peuples sur cette montagne un banquet de choses grasses, un banquet de vins vieux, un banquet, dis-je, de choses grasses et moelleuses, et de vins vieux bien purifiés\FTNT{Mt. 22:2 ; Ap. 3:20.}.
\VS{7}Et il détruit sur cette montagne l'enveloppe redoublée qu'on voit sur tous les peuples, et la couverture qui est étendue sur toutes les nations.
\VS{8}Il détruit la mort par sa victoire\FTNT{1 Co. 15:54.} ; et le Seigneur Yahweh essuie les larmes de tous les visages\FTNT{Ap. 7:17.}, et il ôte l'opprobre de son peuple de toute la terre\FTNT{Lu. 1:25.}, car Yahweh a parlé.
\VS{9}Et l'on dira en ce jour-là : Voici, c'est ici notre Dieu, auquel nous nous attendons, aussi c'est lui qui nous sauve ; c'est ici Yahweh, auquel nous nous attendons ; soyons dans l'allégresse, et réjouissons-nous de son salut !
\VS{10}Car la main de Yahweh repose sur cette montagne ; mais Moab est foulé aux pieds sous lui, comme on foule la paille pour en faire du fumier.
\VS{11}Et il étend ses mains au milieu d'eux, comme le nageur étend ses mains pour nager ; et Yahweh abat son orgueil, ainsi que l'artifice de ses mains.
\VS{12}Il abaisse la forteresse des plus hautes retraites de tes murailles, il les renverse, il les fait crouler à terre, et les réduit en poussière.
\Chap{26}
\TextTitle{Adoration à Yahweh}
\VerseOne{}En ce jour-là, ce cantique sera chanté dans le pays de Juda : Nous avons une ville forte ; le salut\FTNT{Le mot salut vient du mot « Yeshuw'ah ». Cette même racine a donné le prénom Jésus qui signifie Yahweh sauve. Jésus est notre muraille et notre rempart. Dans Ex. 15:2, Moïse identifie Yahweh à « Yeshuw'ah » c'est-à-dire à Jésus. Dans 1 Ch. 16:23, il est dit que « Yeshuw'ah » doit être annoncé tous les jours. Dans Ps. 62:2, il est présenté comme Dieu et le Rocher. Dans Es. 12:2, il est le Dieu qui sauve. Jacob et David avaient mis en lui leur espoir (Ge. 49:18 ; Ps. 119:166). Dans Es. 49:6, il est dit que le salut (« Yeshuw'ah » ou Jésus) doit être annoncé aux extrémités de la terre, et cela est répété et confirmé en Mt. 28:18-20. Es. 56:1 nous apprend que celui qui vient s'appelle « Yeshuw'ah ». Es. 59:17 le présente comme notre casque, ce qui fait écho au casque du salut en Ep. 6:17. Les murs de la Nouvelle Jérusalem portent son Nom (Es. 60:18). Ha. 3:8 nous dit que « Yeshuw'ah » montera sur ses chevaux, corroborant le récit de son retour en gloire dans Ap. 19:11-20. « Yeshuw'ah » est notre flambeau selon Es. 62:1 et Ap. 21:23.} y sera mis pour muraille et pour rempart.
\VS{2}Ouvrez les portes, et la nation juste, celle qui garde la fidélité, y entrera.
\VS{3}Tu gardes dans une paix parfaite celui dont l'esprit s'appuie sur toi, parce qu'il se confie en toi\FTNT{Es. 57:19 ; Ph. 4:6-7.}.
\VS{4}Confiez-vous en Yahweh à perpétuité, car le Rocher \FTNT{Voir commentaire en Es. 8:13-14. } des siècles est en Yahweh Dieu.
\VS{5}Car il a abaissé ceux qui habitaient aux lieux haut élevés, il a renversé la ville de haute retraite, il l'a renversée jusqu'à terre, il l'a réduite jusqu'à la poussière.
\VS{6}Le pied marchera dessus ; les pieds, dis-je, des pauvres, les plantes des misérables marcheront dessus.
\VS{7}Le sentier du juste est la droiture ; toi qui est juste, tu dresses au niveau le chemin du juste.
\VS{8}Aussi t'avons-nous attendu, ô Yahweh, dans le sentier de tes jugements ! Ton Nom et ton souvenir sont le désir de notre âme.
\VS{9}De nuit, je te désire de mon âme, et dès le point du jour, mon esprit qui est en moi te recherche ; car lorsque tes jugements s'exercent sur la terre, les habitants du monde apprennent la justice.
\VS{10}Est-il fait grâce au méchant ? Il n'en apprend point la justice, mais il agit méchamment sur la terre de la droiture, et il ne regarde pas à la majesté de Yahweh.
\VS{11}Yahweh, quand ta main est élevée, ils ne le voient pas. Mais ils verront et seront honteux à cause de leur jalousie pour ton peuple ; et le feu dont tu punis tes ennemis les dévorera.
\VS{12}Yahweh, tu ordonnes la paix pour nous, car aussi tout ce que nous faisons, c'est toi qui l'accomplis en nous.
\VS{13}Yahweh, notre Dieu, d'autres seigneurs que toi nous ont maîtrisés, mais c'est par toi seul que nous pouvons faire mention de ton Nom.
\VS{14}Ils sont morts, ils ne revivront plus, ils sont trépassés, ils ne se relèveront pas ; car tu les as châtiés et exterminés, et tu as fait périr toute mémoire d'eux\FTNT{Ec. 9:5.}.
\VS{15}Yahweh, tu avais accru la nation, tu avais accru la nation, tu as été glorifié, mais tu les as jetés loin dans toutes les extrémités de la terre.
\TextTitle{Un reste épargné de la colère de Yahweh}
\VS{16}Yahweh, étant en détresse ils se sont rendus auprès de toi ; ils se sont répandus en prières quand ton châtiment a été sur eux.
\VS{17}Comme celle qui est enceinte est en travail, et crie dans ses tranchées, lorsqu'elle est prête d'enfanter, ainsi avons-nous été devant ta face ô Yahweh !
\VS{18}Nous avons conçu et nous avons éprouvé des douleurs et nous avons comme enfanté du vent. Nous ne saurions en aucune manière délivrer le pays et les habitants de la terre habitable ne tomberaient point par notre force.
\VS{19}Tes morts vivront ! Même mon corps mort vivra ! Ils se relèveront. Réveillez-vous et réjouissez-vous avec des chants de triomphe, vous, habitants de la poussière ; car ta rosée est comme la rosée des herbes, et la terre jettera dehors les morts\FTNT{Os. 13:14 ; Da. 12:2 ; 1 Co. 15:52.}.
\VS{20}Va, mon peuple, entre dans tes cabinets et ferme ta porte derrière toi\FTNT{Mt. 6:6.} ; cache-toi pour un petit moment, jusqu'à ce que l'indignation soit passée.
\VS{21}Car voici, Yahweh s'en va sortir de son lieu pour visiter l'iniquité des habitants de la terre, commise contre lui ; alors la terre découvrira le sang qu'elle aura reçu et ne couvrira plus ceux qu'on a mis à mort.
\Chap{27}
\TextTitle{Israël rétabli}
\VerseOne{}En ce jour-là, Yahweh frappera de sa dure, grande et forte épée le Léviathan\FTNT{Ps. 104:26 ; Job. 40:20}, le serpent fuyard, le Léviathan, dis-je, le serpent tortueux, et il tuera le monstre qui est dans la mer.
\VS{2}En ce jour-là, chantez sur la vigne désirable\FTNT{Esaïe annonce ici le rétablissement d'Israël. Voir également Ro. 11:1-24.}.
\VS{3}C'est moi Yahweh qui la garde, je l'arrose à chaque instant, je la garde nuit et jour, afin que personne ne lui fasse du mal.
\VS{4}Il n'y a point de fureur en moi ; qu'on me donne des ronces, des épines pour les combattre ! Je marcherai contre elles, je les brulerai toutes ensemble.
\VS{5}Ou bien, qu'il saisisse ma force, qu'il fasse la paix avec moi, qu'il fasse la paix avec moi.
\VS{6}Il fera que Jacob prendra racine, Israël  fleurira, et s'épanouira ; et il remplira de fruits le dessus de la terre habitable.
\VS{7}L'a-t-il frappé comme il a frappé celui qui le frappaient ? L'a-t-il tué comme il a tué ceux qui le tuaient ?
\VS{8}Tu as plaidé avec elle modérément, quand tu l'as renvoyée ; en l'emportant par le vent rude au jour du vent d'orient.
\VS{9}C'est pourquoi l'expiation de l'iniquité de Jacob sera faite par ce moyen, et ceci en sera le fruit entier, que son péché sera ôté ; quand il aura transformé toutes les pierres des autels comme des pierres de chaux réduites en poussière ; et lorsque les idoles d'Asherah et les statues consacrées au soleil ne seront plus debout.
\VS{10}Car la ville fortifiée est désolée, la demeure agréable est abandonnée et délaissée comme le désert. Là pâture le veau, il y gîte et broute les branches.
\VS{11}Quand son branchage est sec, il est brisé ; et les femmes y venant en allument un feu. Car c'est un peuple sans intelligence\FTNT{De. 32:28 ; Es. 1:3.},  c'est pourquoi celui qui l'a fait n'a point eu pitié de lui, et celui qui l'a formé ne lui a point fait grâce.
\VS{12}Il arrivera en ce jour-là que Yahweh secouera, depuis le cours du fleuve jusqu'au torrent d'Egypte ; mais vous serez glanés un à un, ô enfants d'Israël.
\VS{13}Et il arrivera en ce jour-là qu'on sonnera du grand shofar, et ceux qui étaient exilés au pays d'Assyrie, et ceux qui avaient été chassés au pays d'Egypte, reviendront et se prosterneront devant Yahweh, sur la sainte montagne, à Jérusalem.
\Chap{28}
\TextTitle{Malheur et captivité d'Ephraïm en Assyrie}
\VerseOne{}Malheur à la couronne de fierté des ivrognes d'Ephraïm, la noblesse de la gloire qui n'est qu'une fleur qui tombe ; ceux qui sont sur le sommet de la grasse vallée sont étourdis de vin !
\VS{2}Voici, le Seigneur a dans sa main un homme fort et puissant, semblable à une tempête de grêle, à un tourbillon destructeur, à une tempête de grosses eaux débordées ; il la fera tomber à terre avec la main.
\VS{3}Elle seront foulées aux pieds, la couronne de fierté et les ivrognes d'Ephraïm.
\VS{4}Et la noblesse de sa gloire qui est sur le sommet de la fertile vallée, ne sera qu'une fleur qui tombe ; ils seront comme les fruits précoces avant l'été, aussitôt que celui qui regarde les voit, à peine ils sont dans sa main, il les dévore.  
\VS{5}En ce jour-là, Yahweh des armées sera une couronne de noblesse et un diadème de gloire pour le reste de son peuple ;
\VS{6}et un esprit de jugement pour celui qui sera assis au siège de jugement, et une force à ceux qui dans le combat repousseront l'ennemi jusqu'à la porte.
\VS{7}Mais eux aussi, s'oublient dans le vin, et se fourvoient dans les boissons fortes ; le sacrificateur et le prophète s'oublient dans les boissons fortes ; ils sont engloutis par le vin, ils se fourvoient à cause des boissons fortes ; ils s'oublient dans la vision, ils vacillent dans le jugement.
\VS{8}Car toutes leurs tables sont couvertes de vomissements et d'ordures ; aussi il n'y a plus de place !
\VS{9}A qui enseigne-t-on la connaissance ? A qui fait-on comprendre l'enseignement ? Est-ce à ceux qu'on vient de sevrer et de retirer de la mamelle ?
\VS{10}Car il faut leur donner précepte après précepte, précepte après précepte, règle après règle, règle après règle, un peu ici, un peu là\FTNT{Hé. 5:12.}.
\VS{11}C'est pourquoi, il parlera à ce peuple par des lèvres qui balbutient et une langue étrangère.
\VS{12}Il leur disait : Voici le repos, donnez du repos à celui qui est fatigué ;  voici le soulagement ! Mais ils n'ont point voulu écouter.
\VS{13}Ainsi la parole de Yahweh sera pour eux précepte après précepte, précepte après précepte, règle après règle, règle après règle, un peu ici, un peu là ; afin qu'ils aillent et tombent à la renverse, et qu'ils soient brisés, et afin qu'ils tombent dans le piège et qu'ils soient pris.
\TextTitle{Yahweh rompt le pacte du scheol par une pierre angulaire}
\VS{14}C'est pourquoi écoutez la parole de Yahweh, vous hommes moqueurs, qui dominez sur ce peuple qui est à Jérusalem !
\VS{15}Car vous dites : Nous avons fait un pacte avec la mort, et nous avons un accord avec le scheol ; quand le fléau débordé passera, il ne viendra pas sur nous, car nous avons le mensonge pour refuge et nous nous sommes cachés sous la fausseté.
\VS{16}C'est pourquoi ainsi parle le Seigneur Yahweh : Voici, je mettrai pour fondement en Sion une pierre\FTNT{Voir commentaire en Es. 8:13-16.}, une pierre éprouvée, la pierre angulaire la plus précieuse, pour être un fondement solide ; celui qui croira ne se hâtera point.
\VS{17}Et je mettrai le jugement à l'équerre, et la justice au niveau ; et la grêle détruira le refuge du mensonge, et les eaux inonderont le lieu où l'on se retirait.
\VS{18}Et votre pacte avec la mort sera détruit, votre accord avec le scheol ne tiendra pas ; quand le fléau débordé passera, vous en serez foulés.
\VS{19}Dès qu'il passera, il vous emportera. Or il passera tous les matins, le jour et la nuit ; et dès qu'on en entendra le bruit, il n'y aura que terreur.
\VS{20}Car le lit sera trop court, et on ne pourra pas s'y étendre, et la couverture trop étroite pour s'en envelopper.
\VS{21}Car Yahweh se lèvera comme à la montagne de Peratsim, et il sera ému comme dans la vallée de Gabaon, pour faire son œuvre, son œuvre extraordinaire, et pour faire son travail, son travail non accoutumé.
\VS{22}Maintenant donc, ne vous moquez plus, de peur que vos liens ne soient renforcés, car j'ai entendu de par le Seigneur, Yahweh des armées, que la destruction est déterminée sur tout le pays. 
\VS{23}Prêtez l'oreille, et écoutez ma voix ; soyez attentifs, et écoutez mon discours !
\VS{24}Celui qui laboure pour semer, laboure-t-il tous les jours ? Ne casse-t-il pas et ne rompt-il pas les mottes de sa terre ? 
\VS{25}Quand il en aura aplani la surface, ne sèmera-t-il pas la vesce\FTNT{La vesce est un genre de plante herbacées de la famille des légumineuse} ; ne répandra-t-il pas le cumin, ne mettra-t-il pas le froment au meilleur endroit, et l'orge en son lieu assigné, et l'épeautre\FTNT{L'épeautre est une espèce de blé} en son quartier ?
\VS{26}Parce que son Dieu l'a instruit, et lui a enseigné ce qu'il faut faire.
\VS{27}Car on ne foule pas la vesce avec la herse\FTNT{La herse est un instrument agricole permettant de travailler la terre en surface}, et on ne tourne point la roue du chariot sur le cumin ; mais on bat la vesce avec la verge, et le cumin avec le bâton.
\VS{28}Le blé avec lequel on fait le pain se menuise, car le laboureur ne le foule pas entièrement ; et quoiqu'il l'écrase avec la roue de son chariot, néanmoins il ne le menuisera pas avec ses chevaux.
\VS{29}Cela aussi vient de Yahweh des armées qui est admirable en conseil et magnifique en moyens.
\Chap{29}
\TextTitle{Avertissement d'un châtiment imminent}
\VerseOne{}Malheur à Ariel\FTNT{Ariel : Lion de Dieu, nom appliqué à Jérusalem.}, à Ariel, la ville dont David fit sa demeure ! Ajoutez année à année, qu'on égorge des victimes pour les fêtes.
\VS{2}Mais je mettrai Ariel à l'étroit, il n'y aura que tristesse et deuil ; et elle sera pour moi comme Ariel.
\VS{3}Car je camperai en rond contre toi, et je t'assiégerai avec des tours, et je dresserai contre toi des retranchements.
\VS{4}Et tu seras abaissée, et tu parleras depuis la terre, et ta parole sortira étouffée par la poussière ; et ta voix sortira de terre comme celle d'un esprit de Python , et ta parole marmottera comme si elle sortait de la poussière.
\VS{5}La multitude de tes étrangers sera comme une fine poussière ; et la multitude des guerriers sera comme la balle qui passe, et cela sera pour un petit moment.
\VS{6}Elle sera visitée par Yahweh des armées avec des tonnerres, des tremblements de terre, et un grand bruit\FTNT{Za. 14:13-14 ; Ap. 16:18-19.} ; avec la tempête, le tourbillon, et avec la flamme d'un feu dévorant.
\VS{7}Et la multitude de toutes les nations qui feront la guerre à Ariel, et tous ceux qui la combattront, et ceux qui la serreront de près seront comme un songe d'une vision de nuit.
\VS{8}Et il arrivera que comme celui qui a faim rêve qu'il mange, mais quand il se réveille son âme est vide ; et comme celui qui a soif rêve qu'il boit, mais quand il se réveille il est épuisé, et son âme est altérée ; ainsi sera-t-il de la multitude de toutes les nations qui combattront contre la montagne de Sion.
\TextTitle{Yahweh donne les raisons du châtiment}
\VS{9}Arrêtez-vous et soyez étonnés ! Ecriez-vous et criez ! Ils sont ivres, mais non de vin ; ils chancellent, mais non pas à cause des boissons fortes.
\VS{10}Car Yahweh a répandu sur vous un esprit d'un profond sommeil\FTNT{Ro. 11:8.} ; il a fermé vos yeux, il a bandé ceux de vos prophètes et de vos principaux voyants.
\VS{11}Et toute vision est pour vous comme les paroles d'un livre cacheté que l'on donne à un homme de lettres en lui disant : Nous te prions, lis donc cela ! Et qui répond : Je ne le puis, car il est cacheté ;
\VS{12}puis si on le donne à quelqu'un qui n'est pas un homme de lettres, en lui disant : Nous te prions, lis donc cela ! Et qui répond : Je ne sais pas lire.
\VS{13}C'est pourquoi le Seigneur dit : Parce que ce peuple s'approche de moi de sa bouche et qu'il m'honore de ses lèvres, mais que son cœur est éloigné de moi ; et parce que la crainte qu'il a de moi lui a été enseigné par un commandement d'hommes\FTNT{Mt. 15:8-9 ; Mc. 7:6-7.}.
\VS{14}A cause de cela, voici, je continuerai de faire à l'égard de ce peuple-ci des merveilles et des prodiges étranges ; et la sagesse de ses sages périra, et l'intelligence de ses hommes intelligents disparaîtra.
\VS{15}Malheur à ceux qui cachent profondément leurs desseins, pour les dissimuler à Yahweh, et dont les œuvres sont dans les ténèbres, et qui disent : Qui nous voit, et qui nous connaît\FTNT{Es. 47:10 ; Ez. 8:12 ; Ps. 10:11 ; Ps. 94:7.} ?
\VS{16}Ce que vous renversez ne sera-t-il pas réputé comme l'argile d'un potier ? Même l'ouvrage dira-t-il de celui qui l'a fait : Il ne m'a point fait ? Et la chose formée dira-t-elle de celui qui l'a formée : Il n'a point d'intelligence\FTNT{Ps. 100:3.} ?
\TextTitle{Yahweh rachète Jacob}
\VS{17}Le Liban ne sera-t-il pas encore dans très peu de temps changé en un Carmel ? Et Carmel ne sera-t-il pas considéré comme une forêt ?
\VS{18}En ce jour-là, les sourds entendront les paroles du livre, et les yeux des aveugles, étant délivrés de l'obscurité et des ténèbres, verront\FTNT{Mt. 11:5 ; Lu. 7:22.}.
\VS{19}Les humbles auront joie sur joie en Yahweh, et les pauvres d'entre les hommes se réjouiront dans le Saint d'Israël\FTNT{Mt. 5:3-11.}.
\VS{20}Car l'oppresseur prendra fin, le moqueur sera consumé, et tous ceux qui veillaient pour commettre l'iniquité seront retranchés\FTNT{Ap. 20:10.},
\VS{21}ceux qui rendaient coupable les hommes pour une parole, qui tendaient des pièges à celui qui les reprenait à la porte, et qui faisaient tomber le juste en confusion. 
\VS{22}C'est pourquoi ainsi parle Yahweh, lui qui a racheté Abraham, à la maison de Jacob : Jacob ne sera plus honteux, et sa face ne pâlira plus.
\VS{23}Car quand il verra ses fils, ouvrage de mes mains, au milieu de lui, ils sanctifieront mon Nom ; ils sanctifieront, dis-je, le Saint de Jacob, et ils craindront le Dieu d'Israël.
\VS{24}Et ceux dont l'esprit s'était fourvoyé deviendront intelligents, et ceux qui murmuraient apprendront la doctrine.
\Chap{30}
\TextTitle{Mise en garde contre les alliances étrangères}
\VerseOne{}Malheur aux enfants rebelles, dit Yahweh, qui prennent des conseils, et non pas de moi, et qui se forgent des idoles de métal où mon esprit n'est point, afin d'ajouter péché sur péché.
\VS{2}Qui sans avoir interrogé ma bouche, marchent pour descendre en Egypte, afin de se fortifier de la force de Pharaon et se retirer sous l'ombre de l'Egypte\FTNT{Jé. 42:19}.
\VS{3}Car la force de Pharaon sera pour vous une honte, et le refuge sous l'ombre de l'Egypte votre confusion.
\VS{4}Car ses princes sont à Tsoan, et ses messagers ont atteint Hanès.
\VS{5}Tous seront rendus honteux par un peuple qui ne leur profitera de rien, ils n'en recevront aucun secours ni aucun avantage, il sera leur honte et leur opprobre.
\VS{6}Les bêtes sont chargées pour aller au midi, ils portent leurs richesses sur les dos des ânons, et leurs trésors sur la bosse des chameaux, vers le peuple qui ne leur profitera point dans le pays de détresse et d'angoisse, d'où viennent le vieux lion et le lion, la vipère et le serpent volant ; .
\VS{7}Car le secours de l'Egypte n'est que vanité et néant ; c'est pourquoi je crie ceci : Leur force est de se tenir tranquille.
\VS{8}Va maintenant, et écris-le en leur présence sur une table, et rédige-le par écrit dans un livre, afin que cela demeure pour le temps à venir, à perpétuité, à jamais ;
\VS{9}que c'est ici un peuple rebelle, des enfants menteurs, des enfants qui ne veulent point écouter la loi de Yahweh\FTNT{No. 20: 3-5 ; De. 9:7 ; Ac. 7:51.} ;
\VS{10}qui disent aux voyants : Ne voyez pas ! Et aux prophètes : Ne nous prophétisez pas des choses droites, mais dites-nous des choses agréables, voyez des choses trompeuses\FTNT{2 Ti. 4:3-4 ; Mi. 2:6.} !
\VS{11}Retirez-vous du chemin, détournez-vous du sentier, éloignez de notre présence le Saint d'Israël\FTNT{Jn. 14:6.}.
\VS{12}C'est pourquoi ainsi dit le Saint d'Israël : Parce que vous rejetez cette parole et que vous vous confiez dans l'oppression et dans les détours, et que vous vous êtes appuyés sur ces choses,
\VS{13}à cause de cela, cette iniquité sera pour vous comme la fente d'une muraille qui va tomber, un renflement dans un mur élevé, dont la ruine vient soudainement, et en un instant.
\VS{14}Il la brise donc comme on brise un vase de terre, que l'on n'épargne point, et de ses pièces, il ne se trouve pas un tesson pour prendre du feu au foyer, ou pour puiser de l'eau à la citerne.
\TextTitle{La confiance en Yahweh, la vraie force}
\VS{15}Car ainsi a parlé le Seigneur Yahweh, le Saint d'Israël : En vous tenant tranquille et en repos vous serez sauvés ; votre force sera en vous tenant en repos et en espérance. Mais vous ne l'avez point voulu.
\VS{16}Et vous avez dit : Non, mais nous nous enfuirons sur des chevaux ; à cause de cela vous vous enfuirez. Et vous avez dit : Nous monterons sur des chevaux rapides ; à cause de cela ceux qui vous poursuivront seront rapides.
\VS{17}Mille d'entre vous s'enfuiront à la menace d'un seul ; vous vous enfuirez à la menace de cinq ; jusqu'à ce que vous soyez abandonnés comme un arbre tout ébranché au sommet d'une montagne, et comme un étendard sur la colline.
\VS{18}Cependant Yahweh attend pour vous faire grâce, et ainsi il sera exalté pour vous faire miséricorde ; car Yahweh est le Dieu de jugement : Ô bienheureux sont tous ceux qui se confient en lui !
\VS{19}Car le peuple demeurera dans Sion et dans Jérusalem. Tu ne pleureras point ! Certes, il te fera grâce dès qu'il entendra ton cri ; dès qu'il aura entendu, il t'exaucera.
\VS{20}Le Seigneur vous donnera du pain de détresse, et de l'eau d'angoisse, mais tes enseignants ne s'envoleront plus, et tes yeux verront tes enseignants.
\VS{21}Et tes oreilles entendront la parole de celui qui sera derrière toi, disant : Voici le chemin, marchez-y ,soit que vous tiriez à droite, soit que vous tiriez à gauche !
\VS{22}Et vous tiendrez pour souillés les chapiteaux des images taillées faites d'argent, et les ornements faits d'or fondu ; tu les jetteras au loin comme un sang impur, et tu leur diras : Hors d'ici ! 
\VS{23}Alors il donnera la pluie sur la semence que tu auras semées en terre, et le grain du revenu de la terre sera abondant et bien nourri ; en ce jour-là, ton bétail paîtra dans un pâturage spacieux\FTNT{Jn. 14:6.}.
\VS{24}Les bœufs et les ânes qui labourent la terre mangeront le pur fourrage de ce qui aura été vanné avec la pelle et le van.
\VS{25}Et il y aura des ruisseaux d'eau courante sur toute haute montagne, et sur toute colline haut élevée, au jour de la grande tuerie, quand les tours tomberont.
\VS{26}Et la lumière de la lune sera comme la lumière du soleil ; et la lumière du soleil sera sept fois plus grande, comme si c'était la lumière de sept jours, le jour où Yahweh bandera la blessure de son peuple, et qu'il guérira la blessure de sa plaie.
\TextTitle{Jugement de Yahweh sur les Assyriens}
\VS{27}Voici, le Nom de Yahweh vient de loin, sa colère est ardente, et une pesante charge ; ses lèvres sont pleines d'indignation, et sa langue est comme un feu dévorant.
\VS{28}Son Esprit est comme un torrent qui déborde et atteint jusqu'au milieu du cou, pour disperser les nations d'une telle dispersion qu'elles seront réduites à néant, et il est comme une bride aux mâchoires des peuples, qui les fera errer.
\VS{29}Vous aurez un cantique comme la nuit où l'on célèbre une fête solennelle ; vous aurez le cœur joyeux comme celui qui marche au son de la flûte, pour aller à la montagne de Yahweh, vers le Rocher d'Israël.
\VS{30}Et Yahweh fera entendre sa voix, pleine de majesté, et il montrera où aura assené son bras dans l'indignation de sa colère, avec une flamme de feu dévorant, avec éclat, tempête, et pierres de grêle.
\VS{31}Car l'Assyrien, qui frappait du bâton, sera effrayé par la voix de Yahweh.
\VS{32}Et partout où passe le bâton dont Yahweh l'a assené, et par lequel il combattra dans les batailles à bras élevé, on entendra les tambourins et les harpes.
\VS{33}Car Topheth\FTNT{Topheth : Lieu pour brûler. Un lieu à l'extrémité sud-est de la vallée de Hinnom au sud de Jérusalem.} est déjà préparée, et même elle est apprêtée pour le roi ; on a fait son bûcher profond et large ; son bûcher c'est du feu et du bois en abondance ; le souffle de Yahweh l'allume comme un torrent de soufre.
\Chap{31}
\TextTitle{Le secours de Yahweh préférable à celui de l'Egypte}
\VerseOne{}Malheur à ceux qui descendent en Egypte pour avoir de l'aide, et qui s'appuient sur les chevaux, et qui mettent leur confiance dans leurs chars parce qu'ils sont nombreux, et en leurs cavaliers quand ils sont bien forts, mais qui ne regardent pas vers le Saint d'Israël, et ne recherchent pas Yahweh.
\VS{2}Et cependant, c'est lui qui est sage, et il fait venir le malheur et ne révoque point sa parole ; il s'élève contre la maison des méchants et contre ceux qui aident les ouvriers d'iniquité.
\VS{3}Or les Egyptiens sont des hommes et non Dieu ; et leurs chevaux sont chair et non esprit. Quand Yahweh étendra sa main, et celui qui donne du secours sera renversé ; et celui à qui le secours est donné tombera ; et eux tous ensemble seront consumés.
\VS{4}Mais ainsi m'a dit Yahweh : Comme le lion, comme le lionceau rugit sur sa proie, et quoiqu'on appelle contre lui un grand nombre de bergers, il ne se laisse ni effrayer par leur cri, ni abaisser par leur bruit ; ainsi Yahweh des armées descendra pour combattre en faveur de la montagne de Sion et de sa colline.
\VS{5}Comme les oiseaux volent, ainsi Yahweh des armées défendra Jérusalem, la défendant et la délivrant, passant outre et la sauvant\FTNT{De. 32:11 ; Ps. 91:4 ; Mt. 23:37.}.
\VS{6}Retournez vers celui de qui les enfants d'Israël se sont étrangement éloignés.
\VS{7}Car en ce jour-là, chacun rejettera ses idoles d'argent et ses idoles d'or que vos propres mains ont fabriquées pour vous faire pécher.
\VS{8}Et l'Assyrien tombera par l'épée qui n'est pas celle d'un vaillant homme, et l'épée qui n'est pas celle d'un homme le dévorera ; et il s'enfuira devant l'épée, et ses jeunes hommes seront rendus tributaires.
\VS{9}Et saisi de frayeur, il s'enfuira à sa forteresse, et ses chefs seront effrayés à cause de la bannière, dit Yahweh, qui a son feu dans Sion et son fourneau dans Jérusalem.
\Chap{32}
\TextTitle{La venue de l'Esprit annonce la paix et la justice}
\VerseOne{}Voici, un roi régnera selon la justice, et les princes gouverneront avec équité.
\VS{2}Et un homme sera comme le lieu où l'on se cache du vent et comme un asile contre la tempête ; comme des ruisseaux d'eau dans un pays sec, et l'ombre d'un grand rocher dans une terre altérée.
\VS{3}Alors les yeux de ceux qui voient ne seront point retenus, et les oreilles de ceux qui entendent seront attentives.
\VS{4}Et le cœur des étourdis entendra la science, et la langue de ceux qui balbutient parlera aisément et nettement.
\VS{5}Le chiche ne sera plus appelé libéral, et l'avare trompeur ne sera plus nommé magnifique.
\VS{6}Car l'homme vil dira des choses viles, et son cœur ne machine qu'iniquité, pour exécuter son hypocrisie et pour proférer des faussetés contre Yahweh, pour rendre vide l'âme de celui qui a faim, et faire tarir la boisson de celui qui a soif\FTNT{Jn. 10:10.}.
\VS{7}Les instruments de l'avare sont pernicieux ; il prend des conseils pleins de machinations, pour attraper par des paroles de mensonge les affligés, même quand la cause du pauvre est juste\FTNT{2 Pi. 2:3.}.
\VS{8}Mais le libéral forme des conseils de libéralité et se lève pour user de libéralité.
\VS{9}Femmes qui êtes à votre aise, levez-vous, écoutez ma voix ! Filles qui vous tenez assurées, prêtez l'oreille à ma parole !
\VS{10}Dans un an et quelques jours, vous qui vous tenez assurées serez troublées ; car la vendange a manqué, la récolte n'arrivera plus.
\VS{11}Vous qui êtes à votre aise, tremblez ! Vous qui vous tenez assurées, soyez troublées ! Dépouillez-vous, quittez vos habits et ceignez de sacs vos reins !
\VS{12}On se frappe la poitrine à cause de la vigne abondante en fruits.
\VS{13}Les épines et les ronces montent sur la terre de mon peuple, même sur toutes les maisons où il y a de la joie et sur la ville joyeuse.
\VS{14}Car le palais est abandonné, la multitude de la cité est délaissée ; les lieux inaccessibles du pays et les forteresses serviront de cavernes à toujours ; les ânes sauvages y joueront, et les troupeaux y paîtront,
\VS{15}jusqu'à ce que l'Esprit soit répandu d'en haut sur nous\FTNT{Joë. 2:28 ; Za.12:10 ; Ac. 2:17-18.}, et que le désert devienne un Carmel et que Carmel  soit considéré comme une forêt.
\VS{16}Le jugement habitera dans le désert et la justice se tiendra en Carmel.
\VS{17}La justice produira de la paix, et le fruit de la justice sera le repos et la sécurité pour toujours.
\VS{18}Mon peuple habitera dans une demeure paisible, et dans des habitations assurées, et dans un repos fort tranquille.
\VS{19}Mais la grêle tombera sur la forêt, et la ville sera entièrement abaissée.
\VS{20}Heureux vous qui semez sur toutes les eaux, et qui laissez sans entraves le pied du bœuf et de l'âne !
\Chap{33}
\TextTitle{Yahweh se lève}
\VerseOne{}Malheur à toi qui dépouilles et qui n'as pas été dépouillé ! Qui pilles et qu'on n'a pas encore pillé ! Quand tu auras fini de dépouiller, tu seras dépouillé ; et quand tu auras achevé de piller, on te pillera.
\VS{2}Yahweh, aie pitié de nous ! Nous nous attendons à toi ! Sois leur bras dès le matin et notre délivrance au temps de la détresse !
\VS{3}Au son du tumulte, les peuples s'enfuient ; quand tu te lèves, les nations se dispersent.
\VS{4}Et votre butin est recueilli comme on rassemble les sauterelles ; on saute  dessus comme sautellent les sauterelles.
\VS{5}Yahweh est élevé, car il habite dans les lieux élevés ; il remplit Sion de jugement et de justice\FTNT{Ps. 97:9.}.
\VS{6}Et la sagesse et la science seront la certitude de ta durée, et la force de ton salut ; la crainte de Yahweh est son trésor.
\VS{7}Voici, leurs hérauts poussent des cris au-dehors, et les messagers de paix pleurent amèrement.
\VS{8}Les routes sont réduites en désolation, les passants n'y passent plus. Il a rompu l'alliance, il rejette les villes, il ne fait plus cas des hommes.
\VS{9}On mène le deuil, la terre languit. Le Liban est honteux et flétri. Le Saron est comme un désert. Le Basan et le Carmel secouent leur feuillage.
\VS{10}Maintenant je me lèverai, dit Yahweh, maintenant je serai exalté, maintenant je serai élevé.
\VS{11}Vous avez conçu du foin, et vous enfanterez de la paille ; votre souffle vous dévorera comme le feu.
\VS{12}Et les peuples seront des fourneaux de chaux ; ils seront brûlés au feu comme des épines coupées.
\VS{13}Vous qui êtes loin, écoutez ce que j'ai fait ! Et vous qui êtes près, connaissez ma force !
\TextTitle{Yahweh assure la paix aux justes}
\VS{14}Les pécheurs sont effrayés dans Sion, et le tremblement saisit les hypocrites, tellement qu'ils disent : Qui de nous pourra séjourner avec le feu dévorant\FTNT{Hé. 12:29.} ? Qui de nous pourra séjourner avec les flammes éternelles ?
\VS{15}Celui qui observe la justice et qui profère des choses droites ; celui qui rejette le gain déshonnête d'extorsion, et qui secoue ses mains pour ne pas accepter un présent ; celui qui bouche ses oreilles pour ne pas entendre des propos sanguinaires, et qui ferme ses yeux pour ne pas voir le mal,
\VS{16}Celui-là habitera dans des lieux élevés, des forteresses assises sur des rochers seront sa haute retraite ; son pain lui sera donné, et ses eaux ne lui manqueront point\FTNT{Jn. 4:14 ; Jn. 6:33-35 ; Ap 21:6.}.
\VS{17}Tes yeux contempleront le roi dans sa beauté ; et ils regarderont la terre éloignée.
\VS{18}Ton cœur méditera-il la frayeur, en disant : Où est le secrétaire, où est le trésorier ? Où est celui qui tient le compte des tours ?
\VS{19}Tu ne verras plus le peuple fier, le peuple au langage inconnu qu'on n'entend pas, et de langue bégayante qu'on ne comprend pas.
\VS{20}Regarde Sion, la ville de nos fêtes solennelles ! Que tes yeux voient Jérusalem, séjour tranquille, tabernacle qui ne sera pas transportée, et dont les pieux ne seront jamais ôtés, et dont les cordages ne seront point rompus\FTNT{Ap. 21:2.}.
\VS{21}C'est là que Yahweh nous est glorieux ; c'est le lieu de fleuves, de vastes rivières, où n'ira pas de navire à rame et où aucun gros navire passera.
\VS{22}Parce que Yahweh est notre Juge, Yahweh est notre Législateur, Yahweh est notre Roi\FTNT{Jésus-Christ exerce toutes les fonctions gouvernementales : législatives, exécutives et judiciaires.} ; c'est lui qui vous sauvera.
\VS{23}Tes cordages sont lâchés ; et ainsi ils ne tiennent point ferme leur mât et on n'étendra point la voile. Alors la dépouille d'un grand butin est partagé ; même les boiteux pillent le butin.
\VS{24}Et celui qui fait sa demeure dans la maison ne dit point : Je suis malade ! Le peuple qui habite en elle reçoit le pardon de ses iniquités.
\Chap{34}
\TextTitle{Le jugement des nations\FTNTT{Ap. 19:17-21.}}
\VerseOne{}Approchez-vous nations, pour écouter ! Et vous peuples, soyez attentifs ! Que la terre et tout ce qui la remplit écoute ! Que le monde habitable et tout ce qui y est produit écoute !
\VS{2}Car l'indignation de Yahweh est sur toutes les nations, et sa fureur sur toute leur armée ; il les voue à l'interdit, il les livre pour être tuées.
\VS{3}Leurs blessés à morts sont jetés là, et la puanteur de leurs corps morts se répand et les montagnes découlent de leur sang.
\VS{4}Et toute l'armée des cieux se fond ; les cieux sont roulés comme un livre\FTNT{Ap. 6:14.}, et toute leur armée tombe, comme tombe la feuille de la vigne, et comme tombe celle du figuier\FTNT{Mt. 24:28 ; Mc. 13:25.}.
\VS{5}Parce que mon épée s'est enivrée dans les cieux, voici, elle va descendre en jugement contre Edom, et contre le peuple que j'ai voué à l'interdit.
\VS{6}L'épée de Yahweh est pleine de sang ; engraissée de graisse, et du sang des agneaux et des boucs, et de la graisse des reins de béliers ; car il y a des sacrifices de Yahweh à Botsra, et une grande tuerie dans le pays d'Edom.
\VS{7}Les licornes descendent avec eux, et les bœufs avec les taureaux ; leur terre est enivrée de sang, et leur poussière engraissée de graisse.
\VS{8}Car c'est un jour de vengeance pour Yahweh, une année de rétribution pour maintenir la cause de Sion\FTNT{Jé. 46:10 ; Joë. 2:2 ; So. 1:15.}.
\VS{9}Et ces torrents d'Edom seront changés en poix, et sa poussière en soufre, et sa terre deviendra de la poix ardente.
\VS{10}Elle ne sera point éteinte ni jour ni nuit ; sa fumée montera éternellement, elle sera désolée de génération en génération ; il n'y aura personne qui passe par elle à jamais.
\VS{11}Le pélican et le hérisson la posséderont, la chouette et le corbeau  y habiteront ; et on étendra sur elle la ligne de la désolation et le niveau de désordre.
\VS{12}Ses magistrats crieront qu'il n'y a plus là de royaume, et tous ses princes seront réduits à néant.
\VS{13}Les épines croîtront dans ses palais, les chardons et les buissons dans ses forteresses; elle sera la demeure des dragons, et le parvis des hiboux.
\VS{14}Les bêtes sauvages des déserts rencontreront les bêtes sauvages des îles ; et les boucs s'y appelleront les uns les autres ; là aussi, la Lilith\FTNT{Lilith est le nom d'une déesse de la nuit connue pour être un démon nocturne qui hantait les lieux déserts d'Edom.} aura sa demeure et trouvera son lieu de repos ;
\VS{15}là le martinet fera son nid, déposera ses œufs, les couvera, et recueillera ses petits à son ombre ; et là aussi se rassembleront tous les vautours.
\VS{16}Consultez le livre de Yahweh et lisez : Il n'en manquera pas un seul point ; ni l'un ni l'autre ne manqueront ; car c'est ma bouche qui l'a ordonné, et son Esprit qui les rassemblera.
\VS{17}Car il leur a jeté le sort, et sa main leur a partagé cette terre au cordeau, ils la posséderont toujours, ils l'habiteront d'âge en âge.
\Chap{35}
\TextTitle{Yahweh se révèle et sauve son peuple}
\VerseOne{}Le désert et le lieu aride seront dans la joie ; le lieu solitaire se réjouira et fleurira comme une rose.
\VS{2}Il fleurira abondamment, et se réjouira, se réjouissant même et chantant en triomphe. La gloire du Liban lui est donnée, avec la magnificence de Carmel et de Saron ; ils verront la gloire de Yahweh et la magnificence de notre Dieu.
\VS{3}Renforcez les mains lâches, et fortifiez les genoux tremblants\FTNT{Hé. 12:12.}.
\VS{4}Dites à ceux qui ont le cœur troublé : Prenez courage et ne craignez plus\FTNT{Jn. 14:1 ; Jn. 16:33.} ; voici votre Dieu, la vengeance viendra, la rétribution de Dieu ; il viendra lui-même et vous délivrera.
\VS{5}Alors les yeux des aveugles seront ouverts, et les oreilles des sourds seront débouchées.
\VS{6}Alors le boiteux sautera comme un cerf, et la langue du muet chantera en triomphe\FTNT{Esaïe a annoncé la venue de Yahweh lui-même. Cette prophétie s'est parfaitement accomplie en Jésus-Christ qui a réalisé tout ce qui avait été prédit. « Allez rapporter à Jean ce que vous entendez et ce que vous voyez : Les aveugles voient, les boiteux marchent, les lépreux sont purifiés, les sourds entendent, les morts ressuscitent, et l'Evangile est annoncé aux pauvres » (Mt. 11:4-5).}. Car des eaux jailliront dans le désert, et des torrents dans le lieu solitaire.
\VS{7}Et les lieux secs deviendront des étangs, et la terre desséchée deviendra des sources d'eaux ; et dans les repaires où des dragons faisaient leur gîte, il y aura un parvis à roseaux et à joncs.
\VS{8}Il y aura là un sentier et un chemin, qu'on appellera le chemin de sainteté ; celui qui est souillé n'y passera point, mais il sera pour ceux-là ; celui qui va son chemin, et les insensés ne s'y égareront point\FTNT{Mt. 7:13-14 ; Jn. 14:6.}.
\VS{9}Là il n'y aura point de lion ; et aucune des bêtes qui ravissent les autres, n'y montera, et ne s'y trouvera ; mais les rachetés y marcheront.
\VS{10}Ceux dont Yahweh a payé la rançon\FTNT{Jésus-Christ est Yahweh qui a payé notre rançon (Mc. 10:45).}, retourneront, et viendront en Sion avec chant de triomphe, et une joie éternelle sera sur leur tête ; ils obtiendront la joie et l'allégresse ; la douleur et le gémissement s'enfuiront.
\Chap{36}
\TextTitle{Invasion de Sanchérib, menaces de Rabschaké\FTNTT{2 R. 18:9-37 ; 2 Ch. 32:1-19.}}
\VerseOne{}La quatorzième année du roi Ezéchias, Sanchérib, roi d'Assyrie, monta contre toutes les villes fortes de Juda et les prit\FTNT{2 R. 18:17.}.
\VS{2}Puis le roi d'Assyrie envoya de Lakis à Jérusalem, vers le roi Ezéchias, Rabschaké avec une puissante armée. Rabschaké s'arrêta à l'aqueduc de l'étang supérieur, sur le chemin du champ du foulon.
\VS{3}Alors Eliakim, fils de Hilkija, chef de la maison du roi, Schebna, le secrétaire, et Joach, fils d'Asaph, l'archiviste, sortirent vers lui.
\VS{4}Rabschaké leur dit : Dites maintenant à Ezéchias : Ainsi parle le grand roi, le roi d'Assyrie : Quelle est cette confiance que tu as ?
\VS{5}Je te le dis, ce ne sont là que des paroles ; mais il faut pour la guerre de la prudence et de la force. Or maintenant en qui t'es tu confié pour t'être rebellé contre moi ?
\VS{6}Voici, tu t'es confié sur ce bâton qui n'est qu'un roseau cassé, sur l'Egypte, qui perce et traverse la main de celui qui s'appuie dessus ; tel est Pharaon, roi d'Egypte, à tous ceux qui se confient en lui.
\VS{7}Que si tu me dis : Nous nous confions en Yahweh, notre Dieu. Mais n'est-ce pas lui dont Ezéchias a ôté les hauts lieux et les autels, en disant à Juda et à Jérusalem : Vous vous prosternerez devant cet autel-ci ?
\VS{8}Maintenant donc, donne des otages au roi d'Assyrie, mon maître ; et je te donnerai deux mille chevaux, si tu peux donner autant d'hommes pour monter dessus.
\VS{9}Et comment ferais-tu tourner le visage à un seul gouverneur d'entre les moindres serviteurs de mon maître ? Mais tu te confies en l'Egypte pour les chars et pour les cavaliers.
\VS{10}Mais suis-je monté sans Yahweh dans ce pays pour le détruire ? Yahweh m'a dit : Monte contre ce pays et détruis-le.
\VS{11}Alors Eliakim, Schebna et Joach dirent à Rabschaké : Nous te prions de parler en langue araméenne à tes serviteurs, car nous la comprenons ; mais ne parle pas en langue judaïque, pendant que le peuple qui est sur la muraille l'écoute.
\VS{12}Et Rabschaké répondit : Mon maître m'a-t-il envoyé vers ton maître ou vers toi, pour dire ces paroles là ? Ne m'a-t-il pas envoyé vers les hommes qui se tiennent sur la muraille, pour leur dire qu'ils mangeront leur propre fiente, et qu'ils boiront leur urine avec vous ?
\VS{13}Puis Rabschaké se dressa et s'écria à haute voix en langue judaïque, et dit : Ecoutez les paroles du grand roi, du roi d'Assyrie !
\VS{14}Ainsi parle le roi : Qu'Ezéchias ne vous séduise pas, car il ne pourra pas vous délivrer.
\VS{15}Qu'Ezéchias ne vous fasse pas confier en Yahweh, en disant : Yahweh nous délivrera certainement ; cette ville ne sera point livrée entre les mains du roi d'Assyrie.
\VS{16}N'écoutez point Ezéchias ; car ainsi parle le roi d'Assyrie : Faites un accord avec moi pour votre bien, et sortez vers moi, et vous mangerez chacun  de sa vigne, et chacun de son figuier, et vous boirez chacun de l'eau de sa citerne,
\VS{17}jusqu'à ce que je vienne, et que je vous emmène dans un pays qui est comme votre pays, un pays de blé et de bon vin, un pays de pain et de vignes.
\VS{18}Qu'Ezéchias donc ne vous séduise point, en disant : Yahweh nous délivrera. Les dieux des nations ont-ils délivré chacun leur pays de la main du roi d'Assyrie ?
\VS{19}Où sont les dieux de Hamath et d'Arpad ? Où sont les dieux de Sepharvaïm ? Ont-ils délivré Samarie de ma main ?
\VS{20}Qui sont ceux d'entre tous les dieux de ces pays qui aient délivré leur pays de ma main, pour que Yahweh délivre Jérusalem de ma main ?
\VS{21}Mais ils se turent et ne lui répondirent pas un mot ; car le roi avait donné cet ordre, disant : Vous ne lui répondrez pas.
\TextTitle{Ezéchias informé des menaces}
\VS{22}Après cela, Eliakim fils de Hilkija, chef de la maison du roi, Schebna, le secrétaire, et Joach, fils d'Asaph l'archiviste, s'en revinrent auprès d'Ezéchias, les vêtements déchirés, et lui rapportèrent les paroles de Rabschaké.
\Chap{37}
\TextTitle{Ezéchias recherche Yahweh auprès d'Esaïe\FTNTT{2 R. 19:1-7 ; 2 Ch. 32:20.}}
\VerseOne{}Et il arriva qu'aussitôt que le roi Ezéchias eut entendu ces choses, il déchira ses vêtements, se couvrit d'un sac, et entra dans la maison de Yahweh\FTNT{2 R. 19:1-7 ; 2 Ch. 32:20.}.
\VS{2}Puis il envoya Eliakim, chef de la maison du roi, et Schebna, le secrétaire, et les plus anciens des sacrificateurs couverts de sacs, vers Esaïe, le prophète, fils d'Amots.
\VS{3}Et ils lui dirent : Ainsi parle Ezéchias : Ce jour est un jour d'angoisse, de répréhension et de blasphème ; car les enfants sont près de sortir du sein maternel, mais il n'y a point de force pour enfanter.
\VS{4}Peut-être que Yahweh, ton Dieu, a-t-il entendu les paroles de Rabschaké, que le roi d'Assyrie, son maître, a envoyé pour blasphémer le Dieu vivant et lui faire outrage ; selon les paroles que Yahweh, ton Dieu, a entendues ; fais donc requête pour le reste qui subsiste encore.
\VS{5}Les serviteurs du roi Ezéchias vinrent vers Esaïe.
\VS{6}Et Esaïe leur dit : Voici ce que vous direz à votre maître : Ainsi parle Yahweh : Ne crains point pour les paroles que tu as entendues, par lesquelles les serviteurs du roi d'Assyrie m'ont blasphémé.
\VS{7}Voici, je vais mettre en lui un esprit tel qu'ayant entendu une certaine rumeur, il retournera dans son pays, et je le ferai tomber par l'épée dans son pays.
\TextTitle{Provocation et menace de Sanchérib\FTNTT{2 R. 19:8-13 ; 2 Ch. 32:17-19.}}
\VS{8}Or quand Rabschaké s'en fut retourné, il alla trouver le roi d'Assyrie qui attaquait Libna, car il avait appris qu'il était parti de Lakis.
\VS{9}Alors le roi d'Assyrie ayant entendu dire au sujet de Tirhaka, roi d'Ethiopie : Il est sorti pour te faire la guerre. Dès qu'il eut entendu cela, il envoya des messagers à Ezéchias, en leur disant :
\VS{10}Vous parlerez ainsi à Ezéchias, roi de Juda : Que ton Dieu, auquel tu te confies, ne te séduise point, en disant : Jérusalem ne sera point livrée entre les mains du roi d'Assyrie.
\VS{11}Voilà, tu as entendu ce que les rois d'Assyrie ont fait à tous les pays, en les détruisant entièrement ; et toi, tu échapperais ?
\VS{12}Les dieux des nations que mes ancêtres ont détruites, à savoir Gozan, Charan, Retseph, et les fils d'Eden, qui sont à Telassar, les ont-ils délivrées ?
\VS{13}Où sont le roi de Hamath, le roi d'Arpad, et le roi de la ville de Sepharvaïm, d'Héna et d'Ivva ?
\TextTitle{Prière d'Ezéchias à Yahweh\FTNTT{2 R. 19:14-19 ; 2 Ch. 32:20.}}
\VS{14}Et quand Ezéchias reçut les lettres de la main des messagers et les lut, il monta à la maison de Yahweh, et Ezéchias les déploya devant Yahweh.
\VS{15}Puis Ezéchias fit sa prière à Yahweh, en disant :
\VS{16}Ô Yahweh des armées ! Dieu d'Israël qui es assis entre les chérubins ! C'est toi qui es le seul Dieu de tous les royaumes de la terre, c'est toi qui as fait les cieux et la terre.
\VS{17}Ô Yahweh ! Incline ton oreille et écoute ! Ô Yahweh ! Ouvre tes yeux et regarde ! Ecoute les paroles de Sanchérib, qu'il m'a envoyé dire pour blasphémer le Dieu vivant.
\VS{18}Il est bien vrai, ô Yahweh, que les rois d'Assyrie ont détruit tous les pays et leurs contrées ;
\VS{19}et qu'ils ont jeté dans le feu leurs dieux ; mais ce n'étaient point des dieux, mais un ouvrage de mains d'homme, du bois et de la pierre ; c'est pourquoi ils les ont détruits.
\VS{20}Maintenant donc, ô Yahweh notre Dieu ! Délivre-nous de la main de Sanchérib, afin que tous les royaumes de la terre sachent que toi seul es Yahweh.
\TextTitle{Esaïe transmet la réponse de Yahweh\FTNTT{2 R. 19:20-34.}}
\VS{21}Alors Esaïe, fils d'Amots, envoya dire à Ezéchias : Ainsi parle Yahweh, le Dieu d'Israël : J'ai entendu la prière que tu m'as faite au sujet de Sanchérib, roi d'Assyrie.
\VS{22}C'est ici la parole que Yahweh a prononcée contre lui : La vierge, fille de Sion, te méprise et se moque de toi ; la fille de Jérusalem hoche la tête après toi.
\VS{23}Contre qui as-tu élevé ta voix, et levé tes yeux en haut ? C'est contre le Saint d'Israël.
\VS{24}Tu as outragé le Seigneur par le moyen de tes serviteurs, et tu as dit : Je suis monté avec la multitude de mes chars sur le haut des montagnes, aux côtés du Liban, je couperai les plus hauts cèdres, et les plus beaux cyprès qui y soient, et j'entrerai jusqu'en son plus haut bout, et en la forêt de son Carmel.
\VS{25}J'ai creusé des sources, et j'en ai bu les eaux, et je tarirai avec la plante de mes pieds tous les fleuves de l'Egypte.
\VS{26}N'as-tu pas appris, qu'il y a déjà longtemps, j'ai fait cette ville, et que dès les temps anciens je l'ai ainsi formée ? Et maintenant l'aurais-je conservée pour être réduite en désolation, et les villes fortes en monceaux de ruines ?
\VS{27}Or leurs habitants, étant dénués de force, ont été épouvantés et confus ; ils sont devenus comme l'herbe des champs ; et l'herbe verte, comme le foin des toits, et le blé brûlé avant la formation de sa tige.
\VS{28}Mais je sais quand tu t'assieds, quand tu sors et quand tu entres, et comment tu es furieux contre moi\FTNT{Ps. 139:2.}.
\VS{29}Parce que tu es furieux contre moi, et que ton insolence est montée à mes oreilles, je mettrai ma boucle à tes narines, et mon mors en ta bouche, et je te ferai retourner par le chemin par lequel tu es venu.
\VS{30}Et ceci te sera pour signe, ô Ezéchias, c'est qu'on mangera cette année ce qui viendra de soi-même aux champs ; et en la deuxième année ce qui croîtra encore sans semer ; mais la troisième année, vous sèmerez, vous moissonnerez, vous planterez des vignes, et vous en mangerez le fruit.
\VS{31}Et ce qui est réchappé, et demeuré de reste dans la maison de Juda, étendra sa racine par-dessous, et elle produira du fruit par-dessus.
\VS{32}Car il sortira de Jérusalem un reste, et de la montagne de Sion quelques réchappés, la jalousie de Yahweh des armées fera cela.
\VS{33}C'est pourquoi ainsi parle Yahweh sur le roi d'Assyrie : Il n'entrera point dans cette ville, il n'y jettera aucune flèche, il ne se présentera point contre elle avec le bouclier, et il ne dressera point de retranchements contre elle.
\VS{34}Il s'en retournera par le chemin par lequel il est venu, et il n'entrera point dans cette ville, dit Yahweh.
\VS{35}Car je protégerai cette ville pour la délivrer pour l'amour de moi, et pour l'amour de David, mon serviteur.
\TextTitle{Yahweh frappe Sanchérib\FTNTT{2 R. 19:35-37 ; 2 Ch. 32:21.}}
\VS{36}L'ange de Yahweh\FTNT{Ge. 16:7.} sortit et frappa cent quatre-vingt-cinq mille hommes dans le camp des Assyriens. Et quand on se leva le matin, voici, ils étaient tous morts.
\VS{37}Alors Sanchérib, roi d'Assyrie, partit de là ; il s'en alla et s'en retourna, et il se tint à Ninive.
\VS{38}Et il arriva qu'étant prosterné dans la maison de Nisroc\FTNT{Le nom Nisroc signifie « le grand aigle ». C'était une idole de Ninive adorée par Sanchérib, symbolisée par un aigle à figure humaine.}, son dieu, Adrammélec et Scharetser, ses fils, le tuèrent avec l'épée ; puis ils s'enfuirent au pays d'Ararat. Et Esar-Haddon, son fils, régna à sa place.
\Chap{38}
\TextTitle{Maladie et guérison d'Ezéchias\FTNTT{2 R. 20:1-11 ; 2 Ch. 32:24-30.}}
\VerseOne{}En ces jours-là, Ezéchias fut malade à la mort\FTNT{2 R. 20:1-11 ; 2 Ch. 32:24-30.}. Et Esaïe le prophète, fils d'Amots, vint auprès de lui, et lui dit : Ainsi parle Yahweh : Donne tes ordres à ta maison, car tu vas mourir et tu ne vivras plus.
\VS{2}Alors Ezéchias tourna sa face contre la muraille et fit sa prière à Yahweh,
\VS{3}et dit : Ô Yahweh, souviens-toi maintenant je te prie que j'ai marché devant toi en vérité et en intégrité de cœur, et que j'ai fait ce qui est agréable à tes yeux ! Et Ezéchias pleura abondamment.
\VS{4}Puis la parole de Yahweh fut adressée à Esaïe, en disant :
\VS{5}Va, et dis à Ezéchias ainsi parle Yahweh, le Dieu de David, ton père : J'ai exaucé ta prière, j'ai vu tes larmes. Voici, j'ajouterai à tes jours quinze années.
\VS{6}Et je te délivrerai de la main du roi d'Assyrie, toi et cette ville, et je défendrai cette ville.
\VS{7}Et ce signe t'est donné par Yahweh, pour voir que Yahweh accomplira la parole qu'il a prononcée.
\VS{8}Voici, je ferai retourner de dix degrés en arrière avec le soleil l'ombre des degrés qui est descendue sur les degrés d'Achaz. Et le soleil retourna de dix degrés par les degrés par lesquels il était descendu.
\VS{9}Or c'est ici l'écrit d'Ezéchias, roi de Juda, sur sa maladie et sur son rétablissement.
\VS{10}J'avais dit dans le retranchement de mes jours : Je m'en irai aux portes du scheol, je suis privé de ce qui restait de mes années.
\VS{11}Je disais : Je ne contemplerai plus Yahweh, Yahweh sur la terre des vivants ; je ne verrai plus aucun homme parmi les habitants du monde !
\VS{12}Ma durée s'en est allée, et a été transportée loin de moi, comme une cabane de berger ; ma vie est coupée je suis retranché comme la toile que le tisserand détache de sa trame. Du matin au soir tu m'auras enlevé\FTNT{Aux versets 12 et 13, le mot qui a été traduit par « enlevé » est « shalam » : « être dans une alliance de paix, être en paix ».} !
\VS{13}Je pensais en moi-même jusqu'au matin ; comme un lion, qui briserait ainsi tous mes os ; du matin au soir tu m'auras enlevé !
\VS{14}Je grommelais comme la grue et l'hirondelle ; je gémissais comme la colombe ; mes yeux défaillaient à force de regarder en haut : Ô Yahweh, je suis opprimé, sois mon garant !
\VS{15}Que dirai-je ? Il m'a parlé et lui-même l'a fait. Je m'en irai tout doucement tous les ans de ma vie, dans l'amertume de mon âme.
\VS{16}Seigneur, par ces choses-là on a la vie, et dans toutes ces choses est la vie de mon esprit. Ainsi tu me rétabliras et me feras revivre.
\VS{17}Voici, dans ma paix, une grande amertume m'est survenue, mais tu as embrassé mon âme afin qu'elle ne tombe pas dans la fosse de la pourriture, car tu as jeté tous mes péchés derrière ton dos.
\VS{18}Car le scheol ne te loue point,  la mort ne te célèbre point ; ceux qui sont descendus dans la fosse ne s'attendent plus à ta vérité\FTNT{Ps. 115:17.}.
\VS{19}Mais le vivant, le vivant est celui qui te célèbre, comme moi aujourd'hui ; le père conduira ses enfants à la connaissance de ta vérité\FTNT{Pr. 22:6 ; Ep. 6:4.}.
\VS{20}Yahweh est venu me délivrer, et à cause de cela, nous jouerons sur les instruments mes cantiques, tous les jours de notre vie dans la maison de Yahweh.
\VS{21}Or Esaïe avait dit : Qu'on prenne une masse de figues sèches et qu'on en fasse un emplâtre sur l'ulcère ; et Ezéchias guérira.
\VS{22}Et Ezéchias avait dit : Quel est le signe que je monterai à la maison de Yahweh ?
\Chap{39}
\TextTitle{Ezéchias montre toutes ses richesses aux Babyloniens\FTNTT{2 R. 20:12-19}}
\VerseOne{}En ce temps-là\FTNT{2 R. 20:12-19.}, Mérodac-Baladan, fils de Baladan, roi de Babylone, envoya des lettres avec un présent à Ezéchias, parce qu'il avait entendu qu'il avait été malade, et qu'il était guéri.
\VS{2}Et Ezéchias en eut de la joie, et il leur montra les cabinets où étaient ses choses précieuses, l'argent, l'or, et les aromates, et l'huile précieuse, tout son arsenal, et tout ce qui se trouvait dans ses trésors ; il n'y eut rien qu'Ezéchias ne leur montra dans sa maison et dans tous ses domaines.
\VS{3}Puis le prophète Esaïe vint vers le roi Ezéchias, et lui dit : Qu'ont dit ces hommes-là, et d'où sont-ils venus vers toi ? Et Ezéchias répondit : Ils sont venus vers moi d'un pays éloigné, de Babylone.
\VS{4}Puis Esaïe dit : Qu'ont-ils vu dans ta maison ? Ezéchias répondit : Ils ont vu tout ce qui est dans ma maison ; il n'y a rien dans mes trésors que je ne leur aie montré.
\VS{5}Et Esaïe dit à Ezéchias : Ecoute la parole de Yahweh des armées :
\VS{6}Voici, les jours viennent où l'on emportera à Babylone tout ce qui est dans ta maison, et ce que tes pères ont amassé dans leurs trésors jusqu'à aujourd'hui ; il n'en restera rien, dit Yahweh\FTNT{2 R. 24:13 ; 2 R. 25:13-15 ; Jé. 20:5.}.
\VS{7}Même on prendra de tes fils qui sortiront de toi, et que tu auras engendrés afin qu'ils soient eunuques dans le palais du roi de Babylone\FTNT{Da. 1:3-4.}.
\VS{8}Et Ezéchias répondit à Esaïe : La parole de Yahweh, que tu as prononcée, est bonne ; et, il ajouta, au moins qu'il y ait paix et sécurité pendant mes jours.
\Chap{40}
\TextTitle{Un nouveau message pour Esaïe}
\VerseOne{}Consolez, consolez mon peuple, dit votre Dieu.
\VS{2}Parlez à Jérusalem selon son cœur, et criez-lui que son temps marqué est accompli, que son iniquité est tenue pour acquittée, qu'elle a reçu de la main de Yahweh le double pour tous ses péchés.
\TextTitle{Mission de Jean-Baptiste\FTNTT{Mt. 3:3.}}
\VS{3}La voix de celui qui crie au désert\FTNT{L'accomplissement de cette prophétie se trouve en Mt 3:3, où il nous est dit que la voix qui devait crier ces choses était celle de Jean-Baptiste (voir aussi Mal. 3:1 ; Mal. 4:5-6 ; Mt. 17:10-13).} est : Préparez le chemin de Yahweh\FTNT{Les évangiles nous enseignent que Jean-Baptiste a été envoyé pour préparer le chemin du Seigneur Jésus (Jn. 1:19-27 ; Jn. 1:29-34 ; Jn. 3:28-31).}, aplanissez parmi les lieux arides un chemin pour notre Dieu.
\VS{4}Toute vallée sera comblée, toute montagne et toute colline seront abaissées, et les lieux tortueux seront redressés, et les lieux raboteux seront aplanis.
\VS{5}Alors la gloire de Yahweh sera manifestée, et toute chair en même temps la verra, car la bouche de Yahweh a parlé.
\TextTitle{La grandeur de Dieu échappe à l'homme}
\VS{6}La voix dit : Crie ! Et on a répondu : Que crierai-je ? Toute chair est comme l'herbe, et toute sa grâce est comme la fleur d'un champ\FTNT{Ja. 1:10 ; 1 Pi. 1:24-25.}.
\VS{7}L'herbe sèche, et la fleur tombe, parce que le vent de Yahweh souffle dessus. Certainement le peuple est comme l'herbe.
\VS{8}L'herbe sèche, et la fleur tombe, mais la parole de notre Dieu demeure éternellement.
\VS{9}Sion, qui annonce de bonnes nouvelles, monte sur une haute montagne ; Jérusalem, qui annonce de bonnes nouvelles, élève ta voix avec force ; élève-la, ne crains point ; dis aux villes de Juda : Voici votre Dieu !
\VS{10}Voici, le Seigneur Yahweh\FTNT{Jésus-Christ est Yahweh qui vient (Es. 35:4 ; Es. 40:10-11 ; Es. 60:1 ; Es. 62:11-12 ; Es. 66:15-16 ; Za. 14:1-7 ; Mt. 24 ; Jn. 14:1-3; Ac. 1:10-12 ; Ap. 3:11 ; Ap. 19:11-12 ; Ap. 22:7 ; Ap. 22:12 ; Ap. 22:20).} viendra contre le fort, et son bras dominera sur lui ; voici son salaire est avec lui, et ses rétributions sont devant lui.
\VS{11}Il paîtra son troupeau comme un berger, il rassemblera les agneaux dans ses bras, il les placera dans son sein ; il conduira celles qui allaitent\FTNT{Jn. 10.}.
\VS{12}Qui est celui qui a mesuré les eaux avec le creux de sa main, et qui a pris les dimensions des cieux avec la paume, qui a rassemblé toute la poussière de la terre dans un boisseau, et qui a pesé au crochet les montagnes et les collines à la balance ?
\VS{13}Qui a dirigé l'Esprit de Yahweh, ou qui a été son conseiller pour l'enseigner\FTNT{1 Co. 2:16 ; Ro. 11:34.} ?
\VS{14}Avec qui a-t-il pris conseil, et qui l'a instruit, et lui a enseigné le sentier de jugement ? Qui lui a enseigné la science, et lui a montré le chemin de l'intelligence ?
\VS{15}Voilà, les nations sont comme une goutte qui tombe d'un seau, et elles sont réputée comme la menue poussière d'une balance ; voila, il a jeté çà et là les îles comme de la poudre.
\VS{16}Et le Liban ne suffirait pas pour faire le feu, et les bêtes qui y sont ne seraient pas suffisantes pour l'holocauste.
\VS{17}Toutes les nations sont devant lui comme un rien, et il ne les considère que comme de la poussière, et comme un néant.
\VS{18}A qui donc ferez-vous ressembler Dieu ? Et à quelle ressemblance l'égalerez-vous ?
\VS{19}L'ouvrier fond l'image, et l'orfèvre la couvre d'or, et y soude des chaînettes d'argent.
\VS{20}Celui qui est si pauvre qu'il n'a pas de quoi faire une offrande, choisit un bois qui ne pourrisse point ; il se cherche un habile ouvrier pour faire une image taillée qui ne bouge pas\FTNT{Es. 44:9-20.}.
\VS{21}Ne le savez-vous pas ? Ne l'avez-vous pas entendu ? Cela ne vous a-t-il pas été déclaré dès le commencement ? Ne l'avez-vous pas entendu dès les fondements de la terre ?
\VS{22}C'est lui qui est assis au-dessus du globe de la terre, et à qui ses habitants sont comme des sauterelles ; c'est lui qui étend les cieux comme un voile, il les déploie même comme une tente pour y demeurer.
\VS{23}C'est lui qui réduit les princes à rien, et qui fait des chefs de la terre une chose de néant.
\VS{24}Ils ne sont pas même plantés, pas même semés, même leur tronc n'a point de racine en terre ; il souffle sur eux, et ils sèchent, et le tourbillon les emporte comme de la paille.
\VS{25}A qui donc me ferez-vous ressembler, et à qui serais-je égalé ? Dit le Saint.
\VS{26}Elevez vos yeux en haut et regardez ! Qui a créé ces choses ? C'est lui qui fait sortir leur armée par ordre, et qui les appelle toutes par leur nom ; il n'y en a pas une qui fait défaut, à cause de la grandeur de sa force, et parce qu'il excelle en puissance.
\VS{27}Pourquoi donc dis-tu, ô Jacob, pourquoi dis-tu, ô Israël : Ma voie est cachée à Yahweh, et mon jugement passe inaperçu devant mon Dieu ?
\VS{28}Ne sais-tu pas ? N'as-tu pas entendu que le Dieu d'éternité, Yahweh, a créé les extrémités de la terre ; il ne se fatigue point, il ne se lasse point, et il n'y a pas moyen de sonder son intelligence.
\VS{29}C'est lui qui donne de la force à celui qui est las, et il multiplie la force de celui qui n'a aucune vigueur.
\VS{30}Les jeunes gens se lassent et se fatiguent, même les jeunes hommes tombent sans force.
\VS{31}Mais ceux qui s'attendent à Yahweh renouvellent leur force. Ils s'élèvent avec des ailes, comme des aigles ; ils courent, et ne se fatiguent point ; ils marchent, et ne se lassent point.
\Chap{41}
\TextTitle{Dénonciation des idoles}
\VerseOne{}Iles, faites moi silence ! Que les peuples renouvellent leurs forces ; qu'ils s'approchent et qu'alors ils parlent ; allons ensemble en jugement.
\VS{2}Qui a fait levé l'homme droit de l'orient ? Qui l'a appelé à sa suite ? Qui a soumis à son commandement les nations ? Qui lui a donné la domination sur les rois ? Qui les a livrés à son épée comme de la poussière, et à son arc comme de la paille poussée par le vent ?
\VS{3}Il les a poursuivis, il est passé en paix par le chemin que son pied n'avait jamais foulé.
\VS{4}Qui est celui qui a opéré et fait ces choses ? C'est celui qui a appelé les âges dès le commencement. Moi, Yahweh, JE SUIS le premier, et JE SUIS avec les derniers\FTNT{Ap. 1:8 ; Ap. 21:6 ; Ap. 22:13.}.
\VS{5}Les îles voient, et sont dans la crainte, les extrémités de la terre sont effrayées, ils s'approchent, ils viennent.
\VS{6}Chacun aide son prochain, et chacun dit à son frère : Fortifie-toi.
\VS{7}L'ouvrier encourage le fondeur ; celui qui frappe doucement du marteau encourage celui qui frappe sur l'enclume, et il dit : Cela est bon pour souder, puis il fixe l'idole avec des clous, afin qu'elle ne bouge pas.
\VS{8}Mais toi, Israël, tu es mon serviteur, et toi, Jacob, tu es celui que j'ai élu, la race d'Abraham qui m'a aimé !
\VS{9}Car je t'ai pris aux extrémités de la terre, je t'ai appelé en te préférant aux plus excellents qui sont en elle, et je t'ai dit : C'est toi qui est mon serviteur, je t'ai élu, et je ne te rejette point\FTNT{De. 7:6 ; Ps. 77:8.}.
\VS{10}Ne crains rien, car je suis avec toi ; ne sois pas étonné, car je suis ton Dieu ; je te fortifie, et je t'aide, même je te soutiens par la droite de ma justice.
\VS{11}Voici, tous ceux qui sont indignés contre toi seront honteux et confus ; ils seront réduits à néant, et les hommes qui ont querelle avec toi périront.
\VS{12}Tu les chercheras, et tu ne les trouveras plus, ceux qui te suscitaient querelle ; ils seront réduits à néant, et ceux qui te font la guerre seront comme ce qui n'est plus.
\VS{13}Car je suis Yahweh, ton Dieu, qui soutient ta main droite, et te dis : Ne crains rien, c'est moi qui te secours.
\VS{14}Ne crains point, vermisseau de Jacob, hommes mortels d'Israël ; je viens à ton secours dit Yahweh, et ton défenseur, le Saint d'Israël.
\VS{15}Voici, je fais de toi un traîneau aigu, tout neuf, ayant des dents ; tu fouleras les montagnes et les menuiseras, et tu rendras les collines semblables à de la balle.
\VS{16}Tu les vanneras, et le vent les emportera, et le tourbillon les dispersera. Mais toi, tu te réjouiras en Yahweh, tu te glorifiera au Saint d'Israël.
\VS{17}Quant aux affligés et aux misérables qui cherchent des eaux, et n'en ont point ; dont la langue est tellement altérée qu'elle n'en peut plus ; moi, Yahweh, je les exaucerai ; moi, le Dieu d'Israël, je ne les abandonnerai pas\FTNT{Ge. 28:15 ; Jos. 1:5 ; Hé. 13:5.}.
\VS{18}Je ferai jaillir des fleuves sur les hauteurs et des fontaines au milieu des vallées ; et je ferai du désert des étang d'eaux et de la terre sèche des sources d'eaux.
\VS{19}Je ferai croître au désert le cèdre, l'acacia, le myrte et l'olivier ; je mettrai dans les lieux stériles le cyprès, l'orme et le buis ensemble,
\VS{20}afin qu'on voit, qu'on sache, qu'on pense, et qu'on comprenne que la main de Yahweh a fait cela, et que le Saint d'Israël a créé cela.
\VS{21}Plaidez votre cause, dit Yahweh ; et mettez en avant les fondements de votre cause, dit le Roi de Jacob.
\VS{22}Qu'ils les amènent et qu'ils nous déclarent ce qui doit arriver. Déclarez-nous que veulent dire les choses qui ont été auparavant et nous y prendrons garde, et nous saurons leur issue, ou faites-nous entendre ce qui est prêt à arriver. 
\VS{23}Déclarez les choses qui doivent arriver dorénavant, et nous saurons que vous êtes des dieux ; faites aussi du bien ou du mal, et nous en serons tout étonnés puis nous regarderons ensemble.
\VS{24}Voici, vous n'êtes rien, et votre œuvre est le néant ; celui qui vous choisit n'est qu'abomination.
\VS{25}Je l'ai suscité du nord, et il est venu ; il invoque mon Nom de devant le soleil levant ; et marche sur les princes comme sur le mortier, et les foule comme le potier foule la boue.
\VS{26}Qui est celui qui a manifesté ces choses dès le commencement, afin que nous le connaissions ? Et longtemps d'avance, que nous puissions dire : Il est juste. Mais il n'y a personne qui les annonce, même il n'y a personne qui les donne à entendre, même il n'y a personne qui entende vos paroles.
\VS{27}Le premier sera pour Sion, disant : Voici, les voici ! Et je donnerai quelqu'un à Jérusalem qui annoncera de bonnes nouvelles\FTNT{Es. 52:7 ; Ap. 14:6.}.
\VS{28}Je regarde, et il n'y a point d'homme même entre ceux-là, et il n'y a aucun homme de conseil ; je les interroge aussi afin qu'il réponde quelque chose. 
\VS{29}Voici, quant à eux tous, leurs œuvres ne sont que vanité, leurs idoles de fonte sont du vent et de la confusion.
\Chap{42}
\TextTitle{Le messie, serviteur de Yahweh}
\VerseOne{}Voici mon serviteur, que je soutiens, c'est mon élu, en qui mon âme prend son bon plaisir ; j'ai mis mon Esprit sur lui, il manifestera le jugement aux nations\FTNT{Mt. 3:17 ; Mt. 17:5 ; Mc. 9:7.}.
\VS{2}Il ne criera point, et il ne haussera, ni ne fera entendre sa voix dans les rues.
\VS{3}Il ne brisera point le roseau cassé, et il n'éteindra point le lumignon qui fume\FTNT{Mt. 12:18-20.} ; il mettra en avant le jugement en vérité.
\VS{4}Il ne se retirera point et ne s'affaiblira point, jusqu'à ce qu'il ait établi la justice sur la terre, et que les îles s'attendent à sa loi.
\VS{5}Ainsi parle Dieu, Yahweh, qui a créé les cieux, et qui les a étendus, qui a aplani la terre avec ce qu'elle produit, qui donne la respiration au peuple qui est sur elle, et l'esprit à ceux qui y marchent.
\VS{6}Moi Yahweh, je t'ai appelé en justice, et je prendrai ta main et te garderai, et je te ferai être l'alliance du peuple et la lumière des nations\FTNT{Voir commentaire en Ge. 1:3-5.},
\VS{7}afin d'ouvrir les yeux des aveugles, et de faire sortir les prisonniers hors du lieu où on les tient enfermés, et ceux qui habitent dans les ténèbres hors de la prison.
\TextTitle{Israël n'a pas été attentif à Yahweh}
\VS{8}Je suis Yahweh, c'est là mon Nom ; et je ne donnerai pas ma gloire à un autre, ni ma louange aux images taillées\FTNT{Es. 48:11.}.
\VS{9}Voici, les choses qui ont été prédites auparavant se sont accomplies. Et je vous en annonce de nouvelles ; et je vous les fait entendre avant qu'elles arrivent.
\VS{10}Chantez à Yahweh un cantique nouveau, et que sa louange éclate aux extrémités de la terre, vous qui descendez en la mer, et tout ce qui est en elle,  les îles et leurs habitants !
\VS{11}Que le désert et ses villes élèvent la voix ! Que les villages où habite Kédar et ceux qui habitent dans les rochers éclatent en chant de triomphe ! Qu'ils s'écrient du sommet des montagnes !
\VS{12}Qu'on donne gloire à Yahweh, et qu'on publie sa louange dans les îles !
\VS{13}Yahweh sort comme un homme vaillant, il réveille sa jalousie comme un homme de guerre, il jette, dis-je, des cris de joie, il jette de grands cris, et il prévaut sur ses ennemis.
\VS{14}Je me suis tu dès longtemps ; me tiendrais-je en repos ? Me retiendrais-je ? Je crierai comme celle qui enfante, je détruirai, et j'engloutirai tout à la fois.
\VS{15}Je réduirai les montagnes et les collines en désert, et j'en dessécherai toute la verdure, je réduirai les fleuves en îles, et je ferai tarir les étangs.
\VS{16}Je conduirai les aveugles sur un chemin qu'ils ne connaissent pas, je les ferai marcher par des sentiers qu'ils ne connaissent pas ; je réduirai devant eux les ténèbres en lumière, et les choses tortues en choses droites ; voilà ce que je ferai, et je ne les abandonnerai point.
\VS{17}Ils se retireront en arrière, et ils seront tout honteux, ceux qui se confient aux images taillées, et qui disent aux images de fonte : Vous êtes nos dieux !
\VS{18}Sourds, écoutez ! Et vous aveugles, regardez et voyez !
\VS{19}Qui, dis-je, est aveugle, sinon mon serviteur ? Et qui est sourd, comme mon messager que j'envoie ? Qui est aveugle, comme celui que j'ai comblé de grâces ? Qui est aveugle, comme le serviteur de Yahweh ?
\VS{20}Vous voyez beaucoup de choses, mais vous ne prenez garde à rien ; vous avez les oreilles ouvertes, mais vous n'entendez rien.
\VS{21}Yahweh a plaisir en lui à cause de sa justice ; il a magnifié la loi et l'a  rendu honorable. 
\VS{22}Mais c'est ici un peuple pillé et dépouillé ! Ils sont enlacés dans les cavernes, et sont cachés dans des prisons ; ils sont un butin, et il n'y a personne qui les délivre ; une proie, et il n'y a personne qui dise : Restituez !
\VS{23}Qui est celui d'entre vous qui prêtera l'oreille à ces choses ? Qui s'y rendra attentif et l'écoutera à l'avenir ?
\VS{24}Qui est-ce qui a livré Jacob au pillage, et Israël aux pillards\FTNT{Jg. 2:13-16.} ? N'est-ce pas Yahweh, contre lequel nous avons péché ? Car on n'a point voulu marcher dans ses voies et on n'a point obéi à sa loi.
\VS{25}C'est pourquoi il a répandu sur lui la fureur de sa colère, et une forte guerre ; et il l'a embrasé tout alentour, mais Israël ne l'a point connu ; et il l'a brûlé, mais il n'y a point pris garde.
\Chap{43}
\TextTitle{Yahweh veut racheter Israël}
\VerseOne{}Mais maintenant ainsi parle Yahweh, qui t'a créé, ô Jacob ! Celui qui t'a formé, ô Israël ! Ne crains point, car je te rachète, je t'appelle par ton nom, tu es à moi !
\VS{2}Si tu passes par les eaux, je serai avec toi ; et si tu passes par les fleuves, ils ne te noieront pas ; si tu marches dans le feu, tu ne seras pas brûlé, et la flamme ne t'embrasera pas.
\VS{3}Car je suis Yahweh, ton Dieu, le Saint d'Israël, ton Sauveur. Je donne l'Egypte pour ta rançon, l'Ethiopie et Saba à ta place.
\VS{4}Parce que tu es précieux à mes yeux, tu es rendu honorable et je t'aime, je donne des hommes à ta place, et des peuples pour ta vie.
\VS{5}Ne crains point, car je suis avec toi ; je ferai venir ta postérité de l'orient, et je t'assemblerai de l'occident.
\VS{6}Je dirai au nord : Donne ! Et au midi : Ne retiens point ! Fais venir mes fils de loin, et mes filles du bout de la terre,
\VS{7}savoir tous ceux qui s'appellent de mon Nom\FTNT{Dans les Ecritures, le Nom de Dieu le plus cité est YHWH. Jésus, dont le nom signifie « YHWH est salut » correspond au nom et à l'identité que Dieu a révélé à tous ceux qui l'ont rencontré quand il était sur cette terre. Dans sa dernière prière à Gethsémané, Jésus dit : « J'ai fait connaître ton Nom » (Jn. 17:6), et « Je leur ai fait connaître ton Nom » (Jn. 17:26). Ce nom n'est autre que le sien puisque Jésus (YHWH est salut) était et est le Nom de Dieu. Moïse n'avait pas reçu la révélation de ce Nom (Ex. 3:13-14) car cette révélation était réservée à l'Eglise. En tant qu'épouse de Christ, l'Eglise porte le Nom du Seigneur et bénéficie de l'autorité qu'il confère. Ainsi, Jésus est le seul Nom par lequel nous pouvons être sauvés (Ac. 4:12). C'est aussi en son Nom que nous devons être baptisés (Ac. 8:16 ; Ac. 19:5), que nous recevons l'exaucement de nos prières (Jn. 14:13-14 ; Jn. 16:24), que nous sommes délivrés de l'ennemi et que nous obtenons la victoire sur le camp de l'ennemi (Mc. 16:17 ; Ph. 2:9-11).} ; car je les ai créés pour ma gloire ; je les ai formés et les ai faits.
\TextTitle{Yahweh appelle ses témoins}
\VS{8}Amène dehors le peuple aveugle qui a des yeux, et les sourds qui ont des oreilles.
\VS{9}Que toutes les nations soient ramassées ensemble, et que les peuples soient assemblés. Lequel d'entre eux a annoncé ces choses-là ? Et qui sont ceux qui nous ont fait entendre les choses qui ont été ci-devant ? Qu'ils produisent leurs témoins et qu'ils se justifient ; qu'on les entende et qu'on dise : C'est vrai !
\VS{10}Vous êtes mes témoins\FTNT{Ac. 1:8.}, dit Yahweh, et mon serviteur que j'ai élu, afin que vous connaissiez, que vous me croyiez et que vous compreniez que JE SUIS. Avant moi il n'a pas été formé de Dieu, et il n'y en aura point après moi.
\VS{11}Moi, JE SUIS Yahweh, et à part moi il n'y a point de Sauveur\FTNT{Yahweh dit qu'à part lui, il n'y a pas d'autres sauveurs. Or les écrits de la nouvelle alliance affirment que Jésus-Christ est le seul Sauveur (Lu. 1:67-80 ; Ac. 4:11-12).}.
\VS{12}C'est moi qui ai prédit ce qui devait arriver, qui vous ai sauvés, et qui vous ai fait entendre l'avenir, quand il n'y avait point de dieu étranger parmi vous ; et vous êtes mes témoins, dit Yahweh, que je suis Dieu.
\VS{13}Et même avant que le jour fût, JE SUIS, et il n'y a personne qui puisse délivrer de ma main ; je ferai l'œuvre, qui m'en empêchera ?
\TextTitle{Yahweh fera une chose nouvelle car Jacob ne l'a pas honoré}
\VS{14}Ainsi parle Yahweh, votre Rédempteur\FTNT{Es. 60:16 ; 1 Co. 1:30 ; Ro. 3:24 ; Ep. 1:7.}, le Saint d'Israël : J'envoie pour l'amour de vous contre Babylone, et je les fais descendre tous fugitifs, et le cri des Chaldéens sera dans les navires.
\VS{15}Je suis Yahweh, votre Saint, le Créateur d'Israël, votre Roi.
\VS{16}Ainsi parle Yahweh, qui fraya un chemin dans la mer, et un sentier parmi les eaux impétueuses ;
\VS{17}qui amena des chars et des chevaux, et de grandes forces ; ils ont été étendus ensemble, et ils ne se relèveront point, ils ont été étouffés, ils ont été éteints comme un lumignon :
\VS{18}Ne pensez plus aux choses passées, et ne considérez point les choses anciennes.
\VS{19}Voici, je m'en vais faire une chose nouvelle\FTNT{2 Co. 5:17.}, qui paraîtra bientôt, ne la connaîtrez-vous pas ? Je mettrai un chemin dans le désert, et des fleuves dans le lieu de désolation.
\VS{20}Les bêtes des champs me glorifieront, les serpents et les autruches, parce que j'aurai mis des eaux dans le désert, et des fleuves dans la solitude, pour abreuver mon peuple que j'ai élu.
\VS{21}Ce peuple que je me suis formé racontera mes louanges.
\VS{22}Mais toi, Jacob, tu ne m'as pas invoqué, car tu t'es lassé de moi, ô Israël !
\VS{23}Tu ne m'as pas offert le menu bétail de tes holocaustes, et tu ne m'as pas glorifié dans tes sacrifices ; je ne t'ai point asservi pour me faire des offrandes, et je ne t'ai point fatigué pour de l'encens.
\VS{24}Tu ne m'as pas acheté à prix d'argent du roseau aromatique, et tu ne m'as pas rassasié de la graisse de tes sacrifices ; mais tu m'as asservi par tes péchés, et tu m'as peiné par tes iniquités.
\VS{25}Moi, JE SUIS celui qui efface tes transgressions pour l'amour de moi, et je ne me souviendrai plus de tes péchés.
\VS{26}Réveille ma mémoire, et plaidons ensemble ; toi, déclare pour que tu puisses être justifié.
\VS{27}Ton premier père a péché, et tes docteurs se sont rebellés contre moi.
\VS{28}C'est pourquoi j'ai profané les chefs du lieu saint, et j'ai livré Jacob à la destruction, et Israël à l'opprobre.
\Chap{44}
\TextTitle{Promesse de l'Esprit, folie de l'idolâtrie}
\VerseOne{}Ecoute maintenant, ô Jacob, mon serviteur, et toi Israël que j'ai choisi !
\VS{2}Ainsi parle Yahweh, qui t'a fait et formé dès le ventre, celui qui te soutient : Ne crains point, ô Jacob, mon serviteur ! Et toi Jeshurun que j'ai élu.
\VS{3}Car je répandrai des eaux sur celui qui est altéré, et des rivières sur la terre sèche ; je répandrai mon esprit sur ta postérité, et ma bénédiction sur ta descendance.
\VS{4}Et ils germeront comme au milieu de l'herbe, comme les saules auprès des courants d'eau.
\VS{5}L'un dira : Je suis à Yahweh ; et l'autre se réclamera du nom de Jacob ; et un autre écrira de sa main : Je suis à Yahweh, et se nommera du nom d'Israël.
\VS{6}Ainsi parle Yahweh, le Roi d'Israël et son Rédempteur, Yahweh des armées : Je suis le premier, et je suis le dernier ; et à part moi il n'y a point de Dieu.
\VS{7}Et qui, comme moi, a appelé, déclaré et ordonné cela, depuis que j'ai établi le peuple ancien ? Qu'ils déclarent les choses à venir, les choses qui arriveront ci-après !
\VS{8}Ne soyez point effrayés et ne soyez point troublés ; ne te l'ai-je pas fait entendre et déclarer dès ce temps-là ? Vous êtes mes témoins ; y a-t-il un autre Dieu que moi ? Certes il n'y a pas d'autre Rocher\FTNT{Yahweh dit qu'il ne connaît pas d'autre rocher. Jésus-Christ est ce rocher qui suivait les Hébreux dans le désert (Mt. 16:18 ; 1 Co. 10:1-4. Voir aussi commentaire en Es. 8:14). }, je n'en connais pas.
\VS{9}Les ouvriers d'images taillées ne sont tous que vanité, et leurs choses les plus désirables ne sont d'aucun profit ; elles le témoignent elles-mêmes, elles ne voient point, et ne connaissent point, afin qu'ils soient honteux.
\VS{10}Mais qui est-ce qui fabrique un dieu, ou fond une image taillée, pour n'en avoir aucun profit ?
\VS{11}Voici, tous ses compagnons seront honteux, car ces ouvriers-là sont d'entre les hommes. Qu'ils s'assemblent tous, qu'ils se tiennent là ! Ils seront effrayés et rendus honteux tous ensemble.
\VS{12}Le forgeron fait une hache, et il travaille avec le charbon, et il le forme à coups de marteau ; il le fait à force de bras, même il a faim et il est sans force, il ne boit point d'eau, et il est tout fatigué.
\VS{13}Le charpentier étend sa règle, il trace sa forme au crayon avec de la craie ; il le fait avec des équerres, et le forme au compas, et le fait à la ressemblance d'un homme, selon la beauté d'un homme, afin qu'il demeure dans la maison.
\VS{14}Il se coupe des cèdres, et prend un cyprès, ou un chêne, qu'il a laissé croître parmi les arbres de la forêt ; il plante des pins, et la pluie les fait croître.
\VS{15}Ces arbres servent à l'homme pour brûler, car il en prend et il s'en chauffe. Il en fait du feu, dis-je, et en cuit du pain ; et il en fait aussi un dieu et se prosterne devant lui ; il en fait une image taillée et l'adore.
\VS{16}Il en brûle au feu une partie, et d'une autre partie il mange sa chair, laquelle il rôtit, et s'en rassasie ; il s'en chauffe aussi, et il dit : Ah ! Ah ! Je me chauffe, je vois la flamme !
\VS{17}Puis avec le reste il fait un dieu pour être son image taillée ; il se prosterne devant elle, il l'adore, il lui fait sa requête et dit : Délivre-moi, car tu es mon dieu !
\VS{18}Ils ne savent et n'entendent rien, car on leur a plâtré les yeux afin qu'ils ne voient point, et les cœurs pour qu'ils ne comprennent point.
\VS{19}Nul ne rentre en lui-même\FTNT{So. 2:1 ; 2 Co. 13:5.}, et il n'a ni la connaissance ni l'intelligence pour dire : J'en ai brûlé une partie au feu, et même j'ai cuit du pain sur les charbons, j'ai rôti de la viande et je l'ai mangée ; et avec le reste ferais-je une abomination ? Adorerais-je une branche de bois ?
\VS{20}Il se repaît de cendres, et son cœur abusé l'égare, et il ne délivrera point son âme, et ne dira point : N'est-ce pas du mensonge que j'ai dans ma main droite ?
\TextTitle{Yahweh rachète son peuple}
\VS{21}Souviens-toi de ces choses, ô Jacob ! Ô Israël, car tu es mon serviteur ; je t'ai formé, tu es mon serviteur, ô Israël ! Je ne t'oublierai pas.
\VS{22}J'efface tes transgressions comme une nuée épaisse, et tes péchés comme une nuée ; reviens à moi, car je t'ai racheté.
\VS{23}Ô cieux ! Réjouissez-vous avec chants de triomphe, car Yahweh a opéré ; profondeurs de la terre, jetez des cris de réjouissance ! Montagnes, éclatez de joie avec chant de triomphe ! Et vous aussi forêts et tous les arbres qui êtes en elles ! Parce que Yahweh a racheté Jacob, et s'est manifesté glorieusement en Israël.
\VS{24}Ainsi parle Yahweh, ton Rédempteur, celui qui t'a formé dès le ventre : Je suis Yahweh qui ai fait toutes choses, qui seul ai étendu les cieux, et qui ai par moi-même étendu la terre ;
\VS{25}qui dissipe les signes des menteurs, qui rends insensés les devins ; qui renverse l'esprit des sages, et qui fait que leur science devient une folie.
\VS{26}C'est lui qui confirme la parole de son serviteur, et accomplit le conseil  de ses messagers ; qui dit à Jérusalem : Tu seras encore habitée ! Et aux villes de Juda : Vous serez rebâties ! Et je redresserai ses lieux déserts.
\VS{27}Qui dit à l'abîme : Sois asséchée, et je tarirai tes fleuves.
\TextTitle{Prophétie sur le rétablissement d'Israël par Cyrus}
\VS{28}Qui dit de Cyrus\FTNT{Esaïe prophétisa la destruction de Babylone deux siècles avant la réalisation de cet événement  le 5 octobre 539 av. J.-C. Fait remarquable : il précisa même le nom du commandant Cyrus qui dompta le lion babylonien. L’historien Hérodote donnera  par ailleurs raison au prophète sur le déroulement de la prise de Babylone.} : Il est mon berger, et il accomplira tout mon bon plaisir ; disant même à Jérusalem : Tu seras rebâtie ! Et au temple : Tu seras fondé.
\Chap{45}
\TextTitle{Cyrus suscité par Yahweh}
\VerseOne{}Ainsi parle Yahweh à son oint, à Cyrus\FTNT{Cyrus le Grand (580 av. J.-C. - 530 av. J.-C.). Voir Esd. 1.},
\VS{2}que je tiens par la main droite, pour terrasser les nations devant lui, et pour délier les ceintures des rois, pour ouvrir devant lui les portes, afin qu'elles ne soient point fermées.
\VS{3}J'irai devant toi, et j'aplanirai les lieux tortueux ; je romprai les portes d'airain, et je mettrai en pièces les barres de fer. Et je te donnerai des trésors cachés, et des richesses le plus secrètement gardées, afin que tu saches que je suis Yahweh, le Dieu d'Israël, qui t'appelle par ton nom.
\VS{4}Pour l'amour de Jacob, mon serviteur, et d'Israël mon élu ; je t'ai, dis-je, appelé par ton nom, et je t'ai surnommé avant que tu me connaisses.
\TextTitle{Yahweh, le seul Dieu}
\VS{5}Je suis Yahweh, et il n'y en a point d'autre ; à part moi, il n'y a point de Dieu. Je t'ai ceint avant que tu me connaisses,
\VS{6}afin que l'on sache, du soleil levant au soleil couchant, qu'à part moi, il n'y a point de Dieu. Je suis Yahweh, et il n'y en a point d'autre.
\VS{7}Je forme la lumière, et je crée les ténèbres ; je fais la paix et je crée l'adversité ; moi, Yahweh, je fais toutes ces choses.
\VS{8}Ô cieux ! Répandez la rosée d'en haut, et que les nuées laissent couler la justice ! Que la terre s'ouvre, qu'elle produise le salut, qu'elle fasse également germer la justice ! Moi, Yahweh, je crée ces choses.
\VS{9}Malheur à celui qui conteste avec celui qui l'a façonné ! Vase parmi des vases de terre ! L'argile dit-elle à celui qui la façonne : Que fais-tu ? Et l'œuvre dit-elle à l'ouvrier : Tu n'as point de mains\FTNT{Jé. 18:6 ; Ro. 9:21.} ?
\VS{10}Malheur à celui qui dit à son père : Qu'engendres-tu ? Et à sa mère : Qu'enfantes-tu ?
\VS{11}Ainsi parle Yahweh, le Saint d'Israël, qui est son Créateur : Interrogez-moi sur les choses à venir, mes fils ; me commanderez-vous sur l'œuvre de mes mains ?
\VS{12}C'est moi qui ai fait la terre et qui ai créé l'homme sur elle ; c'est moi qui ai étendu les cieux de mes mains, et qui ai donné la loi à toute leur armée.
\VS{13}C'est moi qui ai suscité Cyrus dans ma justice, et j'aplanirai toutes ses voies ; il rebâtira ma ville, et libérera mes captifs\FTNT{Cyrus le grand libéra les Juifs après 70 ans de captivité (Esd. 1).}, sans rançon ni présents, dit Yahweh des armées.
\TextTitle{Les autres peuples reconnaîtront la main de Yahweh sur Israël}
\VS{14}Ainsi parle Yahweh : Le travail de l'Egypte, et le trafic de l'Ethiopie, et ceux des Sabéens, gens de grande stature, passeront chez toi Jérusalem, et seront à toi ; ils marcheront à ta suite, ils passeront enchaînés, ils se prosterneront devant toi, ils te diront en suppliant : Certainement, Dieu est au milieu de toi, et il n'y a point d'autre Dieu que lui.
\VS{15}En vérité, tu es le Dieu qui te caches, le Dieu d'Israël, le Sauveur.
\VS{16}Ils sont tous honteux et confus, ils s'en vont tous avec ignominie, les fabricants d'idoles.
\VS{17}Mais Israël a été sauvé par Yahweh, d'un salut éternel ; vous ne serez ni honteux ni confus jusque dans l'éternité.
\VS{18}Car ainsi parle Yahweh qui a créé les cieux, Dieu lui-même qui a formé la terre, qui l'a faite et qui l'a affermie ; qui l'a créée pour qu'elle ne soit pas informe\FTNT{Informe : de l'hébreu « tohuw » qui signifie « informe, confusion, solitude, désert, néant ». On retrouve ce mot dès Ge. 1:2.}, qui l'a formée pour qu'elle soit habitée ; je suis Yahweh, et il n'y en a point d'autre.
\VS{19}Je n'ai point parlé en secret ni dans quelque lieu ténébreux de la terre ; je n'ai point dit à la postérité de Jacob : Cherchez-moi vainement ! Je suis Yahweh, qui prononce ce qui est juste, qui déclare ce qui est droit.
\VS{20}Assemblez-vous et venez, approchez-vous ensemble, vous les réchappés des nations ! Ceux qui portent le bois de leur image taillée ne savent rien, et invoquent un dieu qui ne sauve pas.
\VS{21}Déclarez-le, et faites-les approcher ! Qu'ils prennent conseil ensemble ! Qui a fait entendre ces choses dès l'origine, et les a déclarées dès longtemps ? N'est-ce pas moi, Yahweh ? Or il n'y a point d'autre Dieu à part moi ; un Dieu juste et un Sauveur, il n'y en a pas d'autre à part moi.
\VS{22}Vous tous qui êtes aux extrémités de la terre, regardez vers moi, et soyez sauvés ; car je suis Dieu, et il n'y en a point d'autre.
\VS{23}Je le jure par moi-même, la parole sort en justice de ma bouche, et elle ne sera point révoquée : Tout genou fléchira devant moi, et toute langue jurera par moi\FTNT{Ph. 2:9-11.}.
\VS{24}Certainement, on dira de moi : En Yahweh seul sont la justice et la force ; à lui viendront, pour être confondus, tous ceux qui étaient irrités contre lui.
\VS{25}Toute la postérité d'Israël sera justifiée, et elle se glorifiera en Yahweh.
\Chap{46}
\TextTitle{La puissance de Yahweh, l'incapacité des idoles}
\VerseOne{}Bel s'incline sur ses genoux, Nebo est renversé ; leurs faux dieux sont mis sur leurs bêtes et leur bétail ; les idoles que vous portiez, ont été chargées, elles sont un fardeau pour la bête fatiguée !
\VS{2}Elles se sont courbées, elles se sont inclinées ensemble sur leurs genoux, et ne peuvent échapper au fardeau, et elles-mêmes s'en vont en captivité.
\VS{3}Ecoutez-moi, maison de Jacob, et vous tous, tout le reste de la maison d'Israël, dont je me suis chargé dès le ventre, et que j'ai porté dès le sein maternel.
\VS{4}Jusqu'à votre vieillesse, JE SUIS ; et je vous chargerai sur moi jusqu'à votre blanche vieillesse, je l'ai fait, et je vous porterai encore, je vous chargerai sur moi et vous sauverai.
\VS{5}A qui me ferez-vous ressembler, et à qui m'égalerez-vous ? A qui me comparerez-vous pour que nous soyons semblable ? 
\VS{6}Ils tirent l'or de la bourse, et pèsent l'argent à la balance, et ils engagent un orfèvre pour en faire un dieu ; ils l'adorent, et se prosternent devant lui.
\VS{7}On le porte sur les épaules, on s'en charge ; on le pose en sa place où il se tient debout et ne bouge point de son lieu, puis on crie à lui, mais il ne répond pas, et il ne délivre pas de la détresse ceux qui crient vers lui.
\VS{8}Souvenez-vous de cela, et montrez-vous des hommes ; rappelez-le à votre pensée, ô vous transgresseurs !
\VS{9}Souvenez-vous des premières choses d'autrefois ; je suis Dieu, et il n'y en a point d'autre, je suis Dieu et il n'y en a point comme moi ;
\VS{10}qui déclare dès le commencement ce qui doit arriver à la fin, et longtemps auparavant, les choses qui n'ont pas encore été faites ; qui dis : Mon conseil tiendra, et j'exécuterai tout mon bon plaisir ;
\VS{11}qui appelle de l'orient l'oiseau de proie, et d'une terre éloignée un homme pour exécuter mon conseil. Oui, j'ai parlé, aussi je ferai venir la chose ; je l'ai formé, aussi je l'accomplirai. 
\VS{12}Ecoutez-moi, vous qui avez le cœur endurci et qui êtes éloignés de la justice.
\VS{13}Je fais approcher ma justice, elle ne s'éloignera point loin ; et mon salut, il ne tardera pas. Je mettrai le salut en Sion pour Israël, qui est ma gloire.
\Chap{47}
\TextTitle{Jugement sur Babylone}
\VerseOne{}Descends, et assieds-toi dans la poussière, vierge, fille de Babylone ! Assieds-toi à terre, il n'y a plus de trône pour la fille des Chaldéens ! Car tu ne te feras plus appeler la délicate et la voluptueuse.
\VS{2}Mets la main aux meules, et fais moudre la farine ; délie tes tresses, déchausse-toi, découvre tes jambes et traverse les fleuves !
\VS{3}Ta honte sera découverte et ton opprobre sera vue ; je prendrai vengeance, je n'irai point contre toi en homme.
\VS{4}Quant à notre Rédempteur, son Nom est Yahweh des armées, le Saint d'Israël.
\VS{5}Assieds-toi sans dire mot, et entre dans les ténèbres, fille des Chaldéens, car tu ne te feras plus appeler la dame des royaumes.
\VS{6}J'ai été embrasé de colère contre mon peuple, j'ai profané mon héritage, c'est pourquoi je les ai livrés entre tes mains, mais tu n'as point usé de miséricorde envers eux, tu as durement appesanti ton joug sur le vieillard.
\VS{7}Et tu as dit : Je serai dame à toujours ! De sorte que tu n'as point mis ces choses-là dans ton cœur, tu ne t'es point souvenue ce qu'en serait la fin.
\VS{8}Maintenant donc écoute ceci, toi voluptueuse qui habite avec assurance, et qui dis en ton cœur : C'est moi, et il n'y en a point d'autre que moi ; je ne deviendrai point veuve, et je ne saurai point ce que c'est que d'être privée d'enfants.
\VS{9}Mais ces deux choses t'arriveront en un moment, en un même jour, la privation d'enfants et le veuvage ; elles viendront sur toi dans leur perfection, pour le  grand nombre de tes sortilèges, et pour la grande abondance de tes enchantements\FTNT{Ap. 18:7-8.}.
\VS{10}Et tu t'es confiée dans ta méchanceté, et disais : Personne ne me voit ! Ta sagesse et ta science t'ont pervertie, et tu disais en ton cœur : C'est moi, et il n'y en a point d'autre que moi.
\VS{11} C'est pourquoi le mal viendra sur toi, et tu ne sauras pas quand il sera près d'arriver, et le malheur qui tombera sur toi sera tel, que tu ne pourras pas le détourner ; et la ruine éclatante que tu n'as pas soupçonnée viendra sur toi subitement.
\VS{12}Tiens-toi maintenant avec tes enchantements, et avec le grand nombre de tes sortilèges, après lesquels tu as travaillé dès ta jeunesse ; peut-être pourras-tu en tirer quelque profit ; peut-être en seras-tu renforcé.
\VS{13}Tu t'es lassée à force de demander des conseils. Que les spectateurs des cieux qui contemplent les étoiles, et qui font leurs prédictions selon les lunes, comparaissent maintenant, et qu'ils te délivrent des choses qui viendront sur toi.
\VS{14}Voici, ils sont devenus comme de la paille, le feu les consume, ils ne délivreront pas leur vie du pouvoir de la flamme ; il n'y a point de charbon pour se chauffer et il n'y a point de lueur de feu pour s'asseoir vis-à-vis. 
\VS{15}Tels te sont devenus ceux avec lesquels tu as travaillé et avec lesquels tu as trafiqué dès ta jeunesse, chacun s'en est fui en son quartier comme un vagabon ; il n'y a personne pour te sauver.
\Chap{48}
\TextTitle{Yahweh rappelle ses promesses}
\VerseOne{}Ecoutez ceci, maison de Jacob, qui êtes appelés du nom d'Israël, et qui êtes sortis des eaux de Juda ; qui jurez par le nom de Yahweh, et qui faites mention du Dieu d'Israël, mais non pas conformément à la vérité et à la justice\FTNT{Jé. 5:2.}.
\VS{2}Car ils prennent leur nom de la sainte cité, et ils s'appuient sur le Dieu d'Israël, dont le nom est Yahweh des armées\FTNT{Ex. 20:7.}.
\VS{3}J'ai déclaré les premières choses dès le commencement, elles sont sorties de ma bouche et je les ai publiées ; je les ai faites subitement et elles se sont accomplies.
\VS{4}Parce que j'ai connu que tu es obstiné, que ton cou est une barre de fer, et que ton front est d'airain,
\VS{5}je t'ai déclaré ces choses dès lors, et je les ai faites entendre avant qu'elles arrivent, de peur que tu ne dises : Mes dieux ont fait ces choses ; mon image taillée, et mon image de fonte les ont ordonnées.
\VS{6}Tu l'entends ! Vois tout ceci ! Et vous, ne l'annoncerez-vous pas ? Je te fais entendre dès maintenant des choses nouvelles, et qui étaient en réserve et que tu ne savais pas.
\VS{7}Elles sont créées maintenant, et non pas depuis le commencement ; et avant ce jour-ci tu n'en avais rien entendu, afin que tu ne dises pas : Voici, je les savais bien.
\VS{8}Oui, tu n'en avais pas entendu parler, oui, tu ne savais pas ; oui, depuis ce temps ton oreille n'a pas été ouverte ; car j'ai connu que tu agirais perfidement; aussi tu as été appelé transgresseur dès le ventre.
\VS{9}Pour l'amour de mon Nom, je diffère ma colère ; et pour l'amour de ma louange, je retiens mon courroux contre toi, afin de ne pas te retrancher.
\VS{10}Voici, je t'ai épuré, mais non pas comme on épure l'argent ; je t'ai éprouvé au creuset de l'affliction.
\VS{11}Pour l'amour de moi, pour l'amour de moi, je le ferai, car comment mon Nom serait-il profané ? Certes, je ne donnerai pas ma gloire à un autre
\VS{12}Ecoute-moi, Jacob ! Et toi Israël, mon appelé ; moi, JE SUIS le premier, JE SUIS aussi le dernier.
\VS{13}Ma main aussi a fondé la terre et ma droite a étendu les cieux ; quand je les appelle, ils comparaissent ensemble.
\VS{14}Vous tous, assemblez-vous et écoutez ! Lequel parmi eux a déclaré ces choses ? Yahweh l'aime et exécutera son bon plaisir contre Babylone, et son bras sera contre sur les Chaldéens.
\VS{15}Moi, JE SUIS celui qui ai parlé, je l'ai aussi appelé, je l'ai amené, et ses desseins réussiront.
\VS{16}Approchez-vous de moi et écoutez ceci ! Dès le commencement, je n'ai point parlé en secret, depuis l'origine de ces choses, JE SUIS. Or maintenant, le Seigneur, Yahweh, et son Esprit m'ont envoyé.
\VS{17}Ainsi parle Yahweh, ton Rédempteur, le Saint d'Israël : Je suis Yahweh, ton Dieu, qui t'enseigne pour ton profit, et qui te guide dans le chemin où tu dois marcher.
\VS{18}Ô ! Si tu étais attentif à mes commandements, ta paix serait comme un fleuve, et ta justice comme les flots de la mer\FTNT{Jos. 1:8 ; Ps. 1:2 ; Jn. 14:21 ; Ja. 1:22.},
\VS{19}ta postérité serait comme le sable, et ceux qui sortent de tes entrailles comme les grains de sable\FTNT{Ge. 15:5 ; Ge. 22:17 ; Ge. 32:12.} ; son nom ne serait point retranché ni effacé de devant ma face.
\VS{20}Sortez de Babylone, fuyez loin des Chaldéens ! Publiez ceci avec une voix de chant de triomphe, annoncez le, portez ceci jusqu'aux extrémités de la terre, dites : Yahweh a racheté son serviteur Jacob !
\VS{21}Et ils n'auront pas soif quand il les fera marcher dans les déserts ; il fera découler pour eux l'eau hors du rocher, même il leur fendra le rocher, et les eaux couleront.
\VS{22}Il n'y a point de paix pour les méchants, dit Yahweh.
\Chap{49}
\TextTitle{Le Messie, la lumière de tous les peuples}
\VerseOne{}Iles, écoutez-moi ! Soyez attentifs, vous peuples éloignés ! Yahweh m'a appelé dès le ventre, il a fait mention de mon nom dès les entrailles de ma mère\FTNT{Jé. 1:5 ; Ps. 139:16.}.
\VS{2}Et il a rendu ma bouche semblable à une épée aiguë ; il m'a caché dans l'ombre de sa main, et m'a rendu semblable à une flèche bien polie, il m'a serré dans son carquois.
\VS{3}Et il m'a dit : Tu es mon serviteur, ô Israël, en qui je serai glorifié.
\VS{4}Et moi j'ai dit : J'ai travaillé en vain, j'ai consumé ma force pour néant et sans fruit ; toutefois mon jugement est auprès de Yahweh, et ma récompense est auprès de mon Dieu.
\VS{5}Maintenant donc, Yahweh, qui m'a formé dès le ventre pour être à son service, m'a dit que je lui ramène Jacob, mais Israël ne se rassemble point ; toutefois je serai honoré aux yeux de Yahweh, et mon Dieu sera ma force.
\VS{6}Il me dit : C'est peu de chose que tu sois serviteur pour relever les tribus de Jacob et pour ramener les restes d'Israël ; c'est pourquoi je te donne pour lumière aux nations, afin que tu sois mon salut jusqu'aux extrémités de la terre.
\VS{7}Ainsi parle Yahweh, le Rédempteur, le Saint d'Israël, à celui qu'on méprise, à celui qui est abominable au peuple, au serviteur de ceux qui dominent ; les rois le verront, et se lèveront, et les princes aussi, et ils se prosterneront devant lui, pour l'amour de Yahweh, qui est fidèle, et du Saint d'Israël qui t'a élu.
\VS{8}Ainsi parle Yahweh : Je t'ai exaucé au temps de la bienveillance, et je t'ai aidé au jour du salut ; je te garderai, et je te donnerai pour être l'alliance du peuple, pour relever la terre, afin que tu possèdes les héritages désolés ;
\VS{9}disant à ceux qui sont emprisonnés : Sortez ! Et à ceux qui sont dans les ténèbres : Montrez-vous ! Ils paîtront sur les chemins, et leurs pâturages seront sur tous les lieux élevés.
\VS{10}Ils n'auront pas faim et ils n'auront pas soif ; la chaleur et le soleil ne les frapperont plus, car celui qui a pitié d'eux sera leur guide, et les conduira vers des sources d'eaux\FTNT{Ps. 121:6 ; Lu. 1:67-79.}.
\VS{11}Et je réduirai toutes mes montagnes en chemins, et mes sentiers seront relevés.
\VS{12}Voici, ceux-ci viennent de loin, et voici ceux-là viennent du nord et de l'occident, et les autres du pays de Sinim.
\VS{13}Ô cieux, réjouissez-vous avec des chants de triomphe! Et toi, ô terre, sois dans l'allégresse ! Et vous, ô montagnes, éclatez de joie avec des chants de triomphe ! Car Yahweh console son peuple, il a compassion de ceux qu'il a affligés.
\VS{14}Mais Sion disait : Yahweh me délaisse, le Seigneur m'oublie !
\VS{15}Une femme peut-elle oublier son enfant qu'elle allaite de sorte qu'elle n'ait pas pitié du fils de ses entrailles ? Mais quand les femmes les oublieraient, moi je ne t'oublierai point.
\VS{16}Voici, je t'ai gravé sur les paumes de mes mains ; tes murs sont continuellement devant moi.
\VS{17}Tes enfants viennent à grande hâtent, mais ceux qui te détruisaient et ceux qui te réduisaient en désert, sortiront du milieu de toi.
\VS{18}Elève tes yeux autour de toi, et regarde : Tous ceux-ci s'assemblent, ils viennent à toi. Je suis vivant, dit Yahweh, tu te revêtiras de tous comme d'une parure, et tu t'en orneras comme une épouse.
\VS{19}Car tes déserts, tes ruines, et ton pays détruit seront désormais trop étroits pour ses habitants, et ceux qui t'engloutissaient s'éloigneront.
\VS{20}Les enfants que tu auras après avoir perdu les autres diront encore, à tes oreilles : Le lieu est trop étroit pour moi, fais-moi de la place pour que je puisse y demeurer.
\VS{21}Et tu diras en ton cœur : Qui m'a engendré ceux-ci vu que j'avais perdu mes enfants et que j'étais stérile, emmenée en captivité et agitée ? Et qui m'a nourri ceux-ci ? Voici, j'étais restée toute seule, et ceux-ci où étaient-ils ?
\VS{22}Ainsi parle le Seigneur Yahweh : Voici, je lèverai ma main vers les nations et je dresserai ma bannière vers les peuples ; et ils ramèneront tes fils entre leurs bras, et ils porteront tes filles sur les épaules.
\VS{23}Et les rois seront tes nourriciers et leurs princesses, leurs femmes, tes nourrices ; ils se prosterneront devant toi le visage contre terre, et ils lécheront la poussière de tes pieds ; et tu sauras que je suis Yahweh, et que ceux qui se confient en moi ne seront point confus\FTNT{Ps. 22:5-6 ; Ps. 69:7 ; Ro. 9:33 ; 1 Pi. 2:6.}.
\VS{24}Le butin sera-t-il ôté à l'homme puissant ? Et les captifs du juste seront-ils délivrés ?
\VS{25}Car ainsi parle Yahweh : Même les captifs pris par l'homme puissant lui seront ôtés, et le butin de l'homme fort lui sera enlevé ; car je plaiderai moi-même avec ceux qui plaident contre toi, et je délivrerai tes enfants.
\VS{26}Et je ferai manger leur propre chair à ceux qui t'oppriment ; et ils s'enivreront de leur sang comme du moût, et toute chair connaîtra que je suis Yahweh, ton Sauveur, ton Rédempteur, le Puissant de Jacob.
\Chap{50}
\TextTitle{Avertissements de Yahweh par son serviteur}
\VerseOne{}Ainsi parle Yahweh : Où est la lettre de divorce par laquelle j'ai répudié votre mère\FTNT{De. 24:1 ; Jé. 3:8 ; Mt. 5:31.} ? Ou bien, auquel de mes créanciers vous ai-je vendus ? Voici, vous avez été vendus à cause de vos iniquités, et votre mère a été répudiée à cause de vos transgressions.
\VS{2}Je suis venu : Pourquoi ne s'est-il trouvé personne ? J'ai appelé : Pourquoi personne n'a-t-il répondu ? Ma main est-elle trop courte pour racheter\FTNT{No. 11:23 ; Es. 59:1.} ? Ou n'y a-t-il plus de force en moi pour délivrer ? Voici, par ma menace, je dessèche la mer, je réduis les fleuves en désert ; leurs poissons se corrompent faute d'eau, et ils meurent de soif.
\VS{3}Je revêts les cieux de noirceur, et je fais d'un sac leur couverture.
\VS{4}Le Seigneur, Yahweh, m'a donné la langue des savants, pour que je sache soutenir par la parole celui qui est accablé de maux\FTNT{Job. 6:14 ; 1 Th. 5:14. } ; chaque matin il me réveille soigneusement afin que je prête l'oreille aux discours des sages.
\VS{5}Le Seigneur Yahweh m'a ouvert l'oreille et je n'ai pas été rebelle, et je ne me suis pas retiré en arrière.
\VS{6}J'ai exposé mon dos à ceux qui me frappaient et mes joues à ceux qui me tiraient le poil ; je n'ai pas caché mon visage aux opprobres et aux crachats\FTNT{Mt. 5:39 ; Mt. 26:67 ; Lu. 6:29 ; Lu. 18:32.}.
\VS{7}Mais le Seigneur, Yahweh m'a aidé, c'est pourquoi je n'ai point été confus, et ainsi j'ai rendu mon visage semblable à un caillou\FTNT{Ez. 3:8-9.}, car je sais que je ne serais point rendu honteux.
\VS{8}Celui qui me justifie est proche ; qui plaidera contre moi ? Comparaissons ensemble ! Qui est mon adversaire ? Qu'il s'approche de moi.
\VS{9}Voici, le Seigneur, Yahweh m'aidera, qui est celui qui me condamnera ? Voici, tous seront usés comme un vêtement, la teigne les dévorera.
\VS{10}Qui est celui d'entre vous qui craint Yahweh, et qui obéit à la voix de son serviteur ! Que celui qui marche dans les ténèbres, et qui n'a pas de clarté, se confie dans le Nom de Yahweh, et qu'il s'appuie sur son Dieu.
\VS{11}Voici, vous tous qui allumez le feu, et qui vous ceignez d'étincelles, marchez à la lueur de votre feu et des étincelles que vous avez embrasées ; voici ce que vous aurez de ma main ; vous vous coucherez dans les tourments.
\Chap{51}
\TextTitle{Exhortation à ceux qui recherchent Yahweh}
\VerseOne{}Ecoutez-moi, vous qui poursuivez la justice et qui cherchez Yahweh ! Regardez au rocher d'où vous avez été taillés, et au creux de la citerne dont vous avez été tirés.
\VS{2}Regardez à Abraham, votre père, et à Sara qui vous a enfantés ; car lui seul je l'ai appelé, je l'ai béni et multiplié\FTNT{Ro. 4:1-16 ; Hé. 11:8-12.}.
\VS{3}Car Yahweh console Sion, il console de toutes ses désolations, il rendra son désert semblable à Eden, et sa terre aride à un jardin de Yahweh. En elle sera trouvée la joie et l'allégresse, la reconnaissance et la voie de mélodie.
\VS{4}Ecoutez-moi donc attentivement, mon peuple, et prêtez-moi l'oreille, vous ma nation ; car la loi sortira de moi, et j'établirai mon jugement pour être la lumière des peuples.
\VS{5}Ma justice est proche, mon salut va paraître, et mes bras jugeront les peuples ; les îles espéreront en moi, elles se confieront en mon bras.
\VS{6}Levez les yeux vers les cieux et regardez en bas sur la terre ! Car les cieux s'évanouiront comme la fumée, et la terre tombera en lambeaux comme un vêtement, et ses habitants périront pareillement ; mais mon salut demeurera éternellement, et ma justice ne sera point anéantie.
\VS{7}Ecoutez-moi, vous qui connaissez la justice, peuple dans le cœur duquel est ma loi ! Ne craignez point l'opprobre des hommes et ne soyez point effrayés devant leurs outrages.
\VS{8}Car la teigne les rongera comme un vêtement\FTNT{Mt. 6:19 ; Lu. 12:33 ; Ja. 5:2.}, et la gerce les dévorera comme de la laine ; mais ma justice demeurera toujours, et mon salut d'âge en âge.
\VS{9}Réveille-toi, réveille-toi, revêts-toi de force, bras de Yahweh ! Réveille-toi comme aux jours anciens, aux siècles passés. N'es-tu pas celui qui tailla en pièce l'Egypte, et qui blessa mortellement le dragon ?
\VS{10}N'est-ce pas toi qui fis tarir la mer, les eaux du grand abîme ? Qui réduisit les lieux les plus profonds de la mer en un chemin afin que les rachetés y passent ?
\VS{11}Ainsi ceux dont Yahweh aura payé la rançon, retourneront, ils iront à Sion avec chants de triomphe ; et une allégresse éternelle couronnera leurs têtes ; ils obtiendront la joie et l'allégresse ; la douleur et le gémissement s'enfuiront.
\VS{12}C'est moi qui suis celui qui vous console. Qui es-tu pour avoir peur de l'homme mortel qui mourra, et du fils de l'homme qui deviendra comme du foin ?
\VS{13}Et tu oublierais Yahweh qui t'a fait, qui a étendu les cieux et fondé la terre ; et chaque jour tu tremblerais continuellement à cause de la fureur de ton oppresseur parce qu'il s'apprête à détruire ! Et où est maintenant la fureur de ton oppresseur ?
\VS{14}Il se hâtera de faire que celui qui aura été transporté d'un lieu à l'autre, soit mis en liberté, afin qu'il ne meure point dans la fosse, et que son pain ne lui manque pas.
\VS{15}Car je suis Yahweh, ton Dieu, qui fend la mer, et les flots rugissants. Yahweh des armées est son Nom.
\VS{16}Or je mets mes paroles dans ta bouche, et je te couvre de l'ombre de ma main, afin que j'affermisse les cieux, que je fonde la terre, et que je dise à Sion : Tu es mon peuple !
\VS{17}Réveille-toi, réveille-toi ! Lève-toi, Jérusalem, qui as bu de la main de Yahweh la coupe de sa fureur ; tu as bu, tu as sucé la lie de la coupe d'étourdissement\FTNT{Ps. 60:5 ; Ap. 14:10.} !
\VS{18}Il n'y a pas un de tous les enfants qu'elle a enfantés qui te conduise, et de tous les enfants qu'elle a nourris, il n'y en a pas un qui la prenne par la main.
\VS{19}Ces deux choses te sont arrivées ; qui te plaindra ? Le ravage et la ruine, la famine et l'épée ; par qui te consolerai-je ?
\VS{20}Tes enfants en défaillance gisaient aux carrefours de toutes les rues, comme un bœuf sauvage pris dans les filets, pleins de la fureur de Yahweh, de la répréhension de ton Dieu.
\VS{21}C'est pourquoi, écoute maintenant ceci, ô affligée, ivre, mais non pas de vin.
\VS{22}Ainsi parle Yahweh, ton Seigneur et ton Dieu, qui plaide la cause de son peuple : Voici, je prends de la main la coupe d'étourdissement, la lie de la coupe de ma fureur, tu n'en boiras plus désormais !
\VS{23}Car je la mettrai dans la main de ceux qui t'ont affligée, et qui disaient à ton âme : Courbe-toi, et nous passerons ! C'est pourquoi tu as exposé ton corps  comme la terre, comme une rue pour les passants.
\Chap{52}
\TextTitle{Le réveil de Jérusalem, la ville sainte}
\VerseOne{}Réveille-toi, réveille-toi, Sion ! Revêts-toi de ta force ! Jérusalem, ville sainte ! Revêts-toi de tes vêtements magnifiques ! Car l'incirconcis et le souillé ne passeront plus désormais parmi toi. 
\VS{2}Jérusalem, secoue ta poussière, lève-toi, et assieds-toi ! Détache les liens de ton cou, captive, fille de Sion !
\VS{3}Car ainsi parle Yahweh : Vous avez été vendus pour rien, et vous serez aussi rachetés sans argent.
\VS{4}Car ainsi parle le Seigneur, Yahweh : Mon peuple descendit jadis en Egypte pour y séjourner ; mais les Assyriens l'opprimèrent sans cause.
\VS{5}Et maintenant, qu'ai-je à faire ici, dit Yahweh, quand mon peuple a été enlevé pour rien ? Ceux qui dominent sur lui le font hurler, dit Yahweh, et mon Nom est blasphémé continuellement chaque jour.
\VS{6}C'est pourquoi mon peuple connaîtra mon Nom ; c'est pourquoi il saura, en ce jour-là, que JE SUIS parle : Voici JE SUIS !
\VS{7}Combien sont beaux sur les montagnes les pieds de celui qui apporte de bonnes nouvelles, qui publie la paix\FTNT{Na. 2:1 ; Ro. 10:15.}, qui apporte de bonnes nouvelles concernant le bien, qui publie le salut, qui dit à Sion : Ton Dieu règne !
\VS{8}Tes sentinelles élèvent leurs voix, elles se réjouissent ensemble avec chants de triomphe ; car de leurs propres yeux elles voient comment Yahweh ramène Sion.
\VS{9}Déserts de Jérusalem, éclatez, réjouissez-vous ensemble avec chants de triomphe ! Car Yahweh console son peuple, il rachète Jérusalem.
\VS{10}Yahweh manifeste le bras de sa sainteté aux yeux de toutes les nations\FTNT{Es. 53:1.}, et toutes les extrémités de la terre verront le salut\FTNT{Toutes les extrémités de la terre verront le salut de Yahweh, c'est-à-dire Jésus (Mt. 28:18-20). } de notre Dieu.
\VS{11}Retirez-vous, retirez-vous, sortez de là ! Ne touchez rien d'impur ! Sortez du milieu d'elle\FTNT{Jé. 51:45 ; 2 Co. 6:17 ; Ap. 18:4.} ! Nettoyez-vous, vous qui portez les vases de Yahweh.
\VS{12}Car vous ne sortirez pas en hâte, et vous ne marcherez pas en fuyant, car Yahweh ira devant vous, et le Dieu d'Israël sera votre arrière-garde.
\TextTitle{Le serviteur de Yahweh}
\VS{13}Voici, mon serviteur prospérera, il sera fort exalté, élevé et glorifié.
\VS{14}Comme plusieurs ont été étonnés en te voyant, son visage était défiguré plus que celui d'aucun homme, et son apparence plus que celle d'aucun fils d'homme ;
\VS{15}ainsi, il aspergera plusieurs nations, et les rois fermeront la bouche sur lui ; car ceux auxquels on n'en avait point parlé le verront ; et ceux qui ne l'avait point entendu l'entendront.
\Chap{53}
\TextTitle{Le sacrifice du Messie, serviteur de Yahweh}
\VerseOne{}Qui a cru à notre prédication ? Et à qui le bras de Yahweh\FTNT{Jésus-Christ homme est le bras de Yahweh. Le bras de Yahweh est le symbole de la puissance divine. Cette puissance s'est manifestée dans l'œuvre du Messie accomplissant le salut du monde. Le prophète est transporté au moment où le peuple juif, après avoir rejeté son Messie, ouvrira enfin les yeux et acceptera celui qu'il a percé (Za. 12:10 ; Ap. 1:7). Voir aussi Jé. 27:4-5 ; Jé. 32:17.} a-t-il été révélé ?
\VS{2}Toutefois il s'est élevé devant lui comme une jeune plante, comme un rejeton qui sort d'une terre desséchée ; il n'y avait en lui ni beauté, ni splendeur, quand nous le regardions, ni apparence qui nous le fasse désirer.
\VS{3}Il était le méprisé et le rejeté des hommes\FTNT{Ps. 22:6-7 ; Mt. 27:27-31 ; Mc. 9:12 ; Jn. 16:32.}, homme de douleur, et sachant ce que c'est que la maladie ; et nous avons comme caché notre visage arrière de lui, tant il était méprisé ; et nous ne l'avons pas estimé.
\VS{4}En vérité, il a porté nos maladies, et il s'est chargé de nos douleurs\FTNT{Mt. 8:17 ; 1 Pi. 2:24.} ; et nous l'avons considéré comme frappé, battu par Dieu et humilié.
\VS{5}Mais il était transpercé pour nos péchés, brisé pour nos iniquités, le châtiment qui nous apporte la paix est tombé sur lui, et c'est par ses meurtrissures que nous avons la guérison.
\VS{6}Nous avons tous été errants\FTNT{Pierre, apôtre de l'Agneau, confirme que le Messie est bel et bien le Bon Berger (1 Pi. 2:25).} comme des brebis, nous nous sommes détournés, chacun suivait son propre chemin, et Yahweh a fait venir sur lui l'iniquité de nous tous.
\VS{7}Opprimé et humilié, il n'a point ouvert sa bouche\FTNT{Mt. 26:62-63 ; Mc. 15:3-5 ; Jn. 19:9 ; Ac. 8:32-33.}, semblable à un agneau qu'on mène à la boucherie, à une brebis muette devant celui qui la tond, et il n'a point ouvert sa bouche.
\VS{8}Il a été enlevé de la force de l'angoisse et de la condamnation ; mais qui racontera sa durée ? Car il a été retranché de la terre des vivants, et la plaie lui a été faite pour les péchés de mon peuple.
\VS{9}On a mis son sépulcre parmi les méchants, et dans sa mort, il a été avec le riche, quoiqu'il n'ait point commis de violence, et qu'il n'y ait point eu de fraude dans sa bouche\FTNT{Mc. 15:28 ; Lu. 23:32-33.}.
\VS{10}Toutefois il a plu à Yahweh de le briser ; il l'a mis dans la souffrance. Après avoir mis son âme en sacrifice pour le péché, il verra une postérité et prolongera ses jours ; et le bon plaisir de Yahweh prospérera en sa main\FTNT{Jé. 23:5.}.
\VS{11}Il jouira du travail de son âme et en sera rassasié ; mon serviteur juste justifiera beaucoup d'hommes par la connaissance qu'ils auront de lui ; et lui-même portera leurs iniquités.
\VS{12}C'est pourquoi je lui donnerai sa part parmi les grands ; il partagera le butin avec les puissants, parce qu'il a livré son âme à la mort, qu'il a été mis au rang des transgresseurs, et que lui-même a porté les péchés de plusieurs, et qu'il a intercédé pour les transgresseurs.
\Chap{54}
\TextTitle{Yahweh réhabilite Israël la délaissée}
\VerseOne{}Réjouis-toi avec chants de triomphe, stérile, toi qui n'enfantes point, toi qui n'a pas connu les douleurs de l'accouchement! Eclate de joie avec chant de triomphe et réjouis-toi  ! Car les enfants de la délaissée seront plus nombreux que les enfants de celle qui est mariée, dit Yahweh.
\VS{2}Elargis l'espace de ta tente, et qu'on étende les couvertures de ton tabernacle : Ne retiens rien ! Allonge tes cordages et affermis tes pieux !
\VS{3}Car tu te répandras à droite et à gauche, et ta postérité possédera les nations et peuplera les villes désertes.
\VS{4}Ne crains pas, car tu ne seras point honteuse, ni confuse, et tu ne rougiras pas ; mais tu oublieras la honte de ta jeunesse, et tu ne te souviendras plus de l'opprobre de ton veuvage.
\VS{5}Car ton Créateur est ton époux : Yahweh des armées est son Nom ; et ton Rédempteur est le Saint d'Israël : Il sera appelé le Dieu de toute la terre.
\VS{6}Car Yahweh t'appelle comme une femme délaissée et à l'esprit affligé, comme une femme qu'on a épousée dans la jeunesse, et qui a été répudiée, dit ton Dieu.
\VS{7}Je t'avais délaissée pour un petit moment, mais je te rassemblerai avec de grandes compassions.
\VS{8}Dans une courte colère, je t'avais un moment caché ma face, mais j'aurai compassion de toi avec une bonté éternelle, dit Yahweh, ton Rédempteur.
\VS{9}Car il en sera pour moi comme les eaux de Noé : De même que j'avais juré que les eaux de Noé ne se répandraient plus sur la terre\FTNT{Ge. 9:11 ; Ge. 8:21.} ; je jure de ne plus m'irriter contre toi, et de ne plus te menacer.
\VS{10}Car quand les montagnes s'en iraient, quand les collines chancelleraient, ma bonté ne s'en ira point de toi, et mon alliance de paix ne chancellera point, dit Yahweh, qui a compassion de toi.
\VS{11}Ô affligée, agitée de la tempête, dénuée de consolation, voici, je coucherai tes pierres d'antimoine, et je te fonderai sur des saphirs ;
\VS{12}et je ferai tes fenêtrages d'agates, et tes portes de rubis, et toute ton enceinte de pierres précieuses.
\VS{13}Aussi tous tes enfants seront enseignés de Yahweh, et grande sera la paix de tes fils.
\VS{14}Tu seras établie en justice, tu seras loin de l'oppression, et tu ne craindras rien ; tu seras, dis-je, loin de la frayeur, car elle n'approchera pas de toi.
\VS{15}Voici, on ne manquera pas de comploter contre toi, cela ne viendra pas de moi ; quiconque complotera contre toi tombera pour l'amour de toi\FTNT{Ps. 91:7 ; Ge. 37.}.
\VS{16}Voici, c'est moi qui ai créé le forgeron soufflant le charbon au feu, et formant un instrument pour son travail, et j'ai créé aussi le destructeur pour détruire.
\VS{17}Aucune arme forgée contre toi ne réussira, et toute langue qui se lèvera en jugement contre toi, tu la condamneras\FTNT{Ps. 23:4.}. Tel est l'héritage des serviteurs de Yahweh, et telle est la justice qui leur viendra de moi, dit Yahweh.
\Chap{55}
\TextTitle{Le salut gratuit par la grâce de Dieu}
\VerseOne{}Vous tous qui avez soif, venez aux eaux, et vous qui n'avez pas d'argent, venez, achetez et mangez ; venez, dis-je, achetez du vin et du lait sans argent, et sans rien payer !
\VS{2}Pourquoi dépensez-vous de l'argent pour ce qui ne nourrit pas ? Pourquoi travaillez-vous pour ce qui ne rassasie pas\FTNT{Ro. 14:17.} ? Ecoutez-moi attentivement, et vous mangerez de ce qui est bon, et votre âme se délectera de la graisse.
\VS{3}Inclinez l'oreille, et venez à moi\FTNT{Mt. 11:28.}, écoutez, et votre âme vivra ; et je traiterai avec vous une alliance éternelle, les miséricordes immuables promises à David.
\VS{4}Voici, je l'ai donné comme témoin auprès des peuples, comme chef et dominateur des peuples.
\VS{5}Voici, tu appelleras des nations que tu ne connais pas, et les nations qui ne te connaissent pas accourront vers toi, à cause de Yahweh, ton Dieu, et du Saint d'Israël, qui t'auras glorifié.
\VS{6}Cherchez Yahweh pendant qu'il se trouve, invoquez-le tandis qu'il est près.
\VS{7}Que le méchant abandonne sa voie, et l'homme injuste ses pensées ; et qu'il retourne à Yahweh, qui aura pitié de lui, et à notre Dieu qui pardonne abondamment\FTNT{Jé. 18:11 ; Ez. 33:11 ; Jon. 3:10 ; 1 Ti. 2:1-4 ; 2 Pi. 3:9.}.
\VS{8}Car mes pensées ne sont pas vos pensées, et mes voies ne sont pas vos voies, dit Yahweh.
\VS{9}Mais autant les cieux sont élevés au-dessus de la terre, autant mes voies sont élevées au-dessus de vos voies, et mes pensées au-dessus de vos pensées.
\VS{10}Car comme la pluie et la neige descendent des cieux et n'y retournent plus, mais arrosent la terre, et la font produire et germer, afin de donner de la semence au semeur, et du pain à celui qui mange,
\VS{11}ainsi en est-il de ma parole qui sort de ma bouche, elle ne retourne point vers moi sans effet, mais elle fait tout ce en quoi je prends plaisir, et prospérera dans l'œuvre pour laquelle je l'ai envoyée.
\VS{12}Car vous sortirez avec joie, et vous serez conduits en paix ; les montagnes et les collines éclateront de joie avec chants de triomphe devant vous, et tous les arbres des champs battront des mains.
\VS{13}Au lieu de l'épine s'élèvera le cyprès, au lieu de la ronce croîtra le myrte ; et ceci fera connaître le nom de Yahweh, et ce sera un signe perpétuel, qui ne sera jamais retranché.
\Chap{56}
\TextTitle{Exhortation à s'attacher à Yahweh}
\VerseOne{}Ainsi parle Yahweh : Observez le jugement, faites ce qui est juste, car mon salut ne tardera pas à venir, et ma justice à être révélée.
\VS{2}Bienheureux l'homme qui fait cela, et le fils de l'homme qui s'y tient, observant le sabbat pour ne pas le profaner, et gardant ses mains pour ne faire aucun mal.
\VS{3}Et que l'enfant de l'étranger qui se joint à Yahweh ne parle pas en disant : Yahweh me séparera entièrement de son peuple ! Et que l'eunuque ne dise pas : Voici, je suis un arbre sec.
\VS{4}Car ainsi parle Yahweh touchant les eunuques : Ceux qui garderont mes sabbats, et qui choisiront ce en quoi je prends plaisir, et qui tiendront dans mon alliance,
\VS{5}je leur donnerai dans ma maison et dans mes murailles une place et un nom meilleur que le nom de fils ou de filles ; je leur donnerai à chacun un nom éternel qui ne périra jamais\FTNT{Ap. 2:17.}.
\VS{6}Et les enfants des étrangers qui se joindront à Yahweh pour le servir, pour aimer le Nom de Yahweh, pour être ses serviteurs, savoir tous ceux qui garderont le sabbat pour ne pas le profaner et qui tiendront dans mon alliance\FTNT{Ex. 31:14.},
\VS{7}je les amènerai sur ma montagne sainte, et je les réjouirai dans ma maison de prière ; leurs holocaustes et leurs sacrifices seront agréés sur mon autel, car ma maison sera appelée une maison de prière\FTNT{Mt. 21:13 ; Mc. 11:17 ; Lu. 19:46.} pour tous les peuples.
\VS{8}Le Seigneur, Yahweh, parle, lui qui rassemble les exilés d'Israël : Je réunirai d'autres peuples à lui, outre ceux déjà rassemblés.
\VS{9}Bêtes des champs, bêtes des forêts, venez toutes pour manger !
\VS{10}Toutes ses sentinelles sont aveugles, elles ne connaissent rien ; ce sont tous des chiens muets, qui ne peuvent aboyer, dormant et demeurant couchés, et aimant à sommeiller.
\VS{11}Ce sont des chiens voraces et insatiables ; ce sont des pasteurs qui ne savent rien comprendre ; tous suivent leur propre voie, chacun à son gain injuste dans son quartier, en disant\FTNT{Mt. 23:24 ; Tit. 1:7-11 ; 1 Pi. 5:2.} :
\VS{12}Venez, je vais chercher du vin, et nous nous enivrerons de boissons fortes ! Nous en ferons autant demain, et même beaucoup plus encore !
\Chap{57}
\TextTitle{Yahweh expose la fausseté et défend le juste}
\VerseOne{}Le juste périt, et nul ne le prend à cœur ; et les gens de bien sont recueillis, sans qu'on y soit attentif, sans qu'on considère que le juste a été recueilli devant le mal\FTNT{Mi. 7:2 ; Ec. 7:15.}.
\VS{2}Il entrera en paix, il reposera sur sa couche, celui qui aura marché dans la droiture\FTNT{Mt. 25:23 ; Lu. 19:17.}.
\VS{3}Mais vous, approchez ici, enfants de l'enchanteresse, race de l'adultère et de la prostituée !
\VS{4}De qui vous êtes-vous moqués ? Contre qui avez-vous ouvert la bouche et tirez-vous la langue ? N'êtes-vous pas des enfants de rébellion, une race de mensonge ?
\VS{5}S'échauffant près des faux dieux, sous tout arbre vert ; égorgeant les enfants dans les vallées, sous les fentes des rochers\FTNT{Lé. 18:21 ; 1 R. 14:23 ; Jé. 2:20 ; Jé. 32:35.}.
\VS{6}Parmi les pierres polies des torrents est ta portion, ce sont elles, ce sont elles qui sont ton lot ; tu leur a aussi répandu ton aspersion, tu leur as aussi offert des offrandes ; puis-je être content de ces choses ?
\VS{7}Tu dresses ta couche sur les montagnes hautes et élevées ; c'est aussi là que tu montes pour offrir des sacrifices.
\VS{8}Et tu mets ton souvenir derrière la porte et les poteaux ; car tu te découvres loin de moi et tu montes, tu élargis ta couche, et tu te l'est taillé plus grande que n'ont fait ceux-là ; tu as aimé leur couche, tu as pris garde aux belles places.
\VS{9}Tu voyages vers le roi avec de l'huile précieuse, et tu ajoutes parfums sur parfums ; tu envoies au loin tes ambassades, tu t'abaisses jusqu'au scheol.
\VS{10}Tu te fatigues par la longueur du chemin, et tu ne dis pas : C'est sans espoir ! Tu trouves encore de la vigueur dans ta main ; c'est pourquoi tu n'as pas été languissante.
\VS{11}Et qui redoutais-tu, qui craignais-tu pour que tu me mentes, pour ne pas te souvenir et te soucier de moi ? N'ai-je pas gardé le silence, et même depuis longtemps, et tu ne me crains pas.
\VS{12}Je vais déclarer ta justice et tes œuvres, qui ne te profiteront pas.
\VS{13}Quand tu crieras, que ceux que tu assembles te délivrent ! Mais le vent les emmènera tous, la vanité les enlèvera ; mais celui qui met sa confiance en moi, héritera la terre et possédera ma montagne sainte\FTNT{Es. 2:3 ; Ps. 2:6 ;  Hé. 12:22.}.
\VS{14}On dira : Frayez, frayez, préparez le chemin, enlevez tout obstacle loin du chemin de mon peuple !
\TextTitle{Yahweh aime l'homme contrit}
\VS{15}Car ainsi parle celui qui est haut et élevé, qui habite dans l'éternité et dont le nom est le Saint : J'habiterai dans les lieux hauts et saints, avec celui qui a le cœur brisé et qui est humble d'esprit, afin de vivifier l'esprit des humbles, et afin de vivifier ceux qui ont le cœur brisé\FTNT{Ps. 34:19 ; Ps. 51:19.}.
\VS{16}Parce que je ne veux pas contester à toujours, et que je ne serai pas irrité à jamais ;  car devant moi tombent en défaillance les esprits, et les âmes que j'ai faites\FTNT{Mi. 7:18 ; Ps. 85:6 ; Ps. 103:9.}.
\VS{17}A cause de l'iniquité de ses gains déshonnêtes, je me suis irrité et je l'ai frappé, je me suis caché ma dans ma colère ; et le rebelle a suivi la voie de son cœur.
\VS{18}J'ai vu ses voies, et toutefois je le guérirai ; je le conduirai et je le restaurerai, lui et ceux qui mènent deuil avec lui.
\VS{19}Je crée les fruits des lèvres. Paix, paix à celui qui est loin et à celui qui est près ! dit Yahweh, car je le guérirai.
\VS{20}Mais les méchants sont comme la mer agitée, quand elle ne peut se calmer, et dont les eaux rejettent la boue et le bourbier.
\VS{21}Il n'y a point de paix pour les méchants, dit mon Dieu.
\Chap{58}
\TextTitle{Le vrai et le faux jeûne}
\VerseOne{}Crie à plein gosier, ne te retiens pas, élève ta voix comme un shofar, et annonce à mon peuple ses iniquités et à la maison de Jacob ses péchés !
\VS{2}Car ils me cherchent tous les jours, ils prennent plaisir à connaître mes voies ; comme une nation qui aurait pratiqué la justice, et qui n'aurait pas abandonné les ordonnances de son Dieu ; ils me demandent des jugements justes, ils prennent plaisir à s'approcher de Dieu, et puis ils disent :
\VS{3}Pourquoi jeûnons-nous, et tu ne le vois pas ? Pourquoi affligeons-nous nos âmes, si tu n'y as point connaissance ? Voici, le jour de votre jeûne, vous trouvez votre plaisir, et vous oppressez tous vos travailleurs.
\VS{4}Voici, vous jeûnez pour faire des querelles et vous disputer, et pour frapper du poing méchamment ; vous ne jeûnez pas comme le veut ce jour, pour que votre voix soit exaucée d'en haut.
\VS{5}Est-ce là le jeûne que j'ai choisi, que l'homme afflige son âme un jour ? Est-ce en courbant sa tête comme le jonc et en étendant le sac et la cendre ? Appelleras-tu cela un jeûne et un jour agréable à Yahweh ?
\VS{6}N'est-ce pas plutôt ici le jeûne que j'ai choisi : Que tu détaches les liens de la méchanceté, que tu délies les cordages du joug, que tu laisses aller libres les opprimés, et que l'on rompe toute espèce de joug ?
\VS{7}N'est ce-pas que tu partages ton pain avec celui qui a faim ? Et que tu fasses venir dans ta maison les affligés errants ? Quand tu vois un homme nu, que tu le couvres, et que tu ne te caches pas de ta propre chair ?
\TextTitle{Bénédiction pour ceux qui pratiquent le bien}
\VS{8}Alors ta lumière éclatera comme l'aurore, et ta guérison germera rapidement ; ta justice ira devant toi, et la gloire de Yahweh sera ton arrière-garde.
\VS{9}Alors tu prieras, et Yahweh t'exaucera ; tu crieras, et il dira : Me voici ! Si tu ôtes du milieu de toi le joug, si tu cesses de lever le doigt et de dire des outrages ;
\VS{10}si tu ouvres ton âme à celui qui a faim, si tu rassasies l'âme affligée ; ta lumière se lèvera sur les ténèbres, et l'obscurité sera comme le midi.
\VS{11}Et Yahweh te conduira continuellement, il rassasiera ton âme dans les grandes sécheresses, il fortifiera tes os, et tu seras comme un jardin arrosé, et comme une source dont les eaux ne tarissent pas\FTNT{Jn. 4:14 ; Ap. 21:6.}.
\VS{12}Et ceux qui sortiront de toi rebâtiront les lieux déserts depuis longtemps, tu rétabliras les fondements ruinés depuis plusieurs générations ; et on t'appellera le réparateur des brèches et le restaurateur des chemins, afin qu'on habite au pays.
\VS{13}Si tu détournes ton pied pendant le sabbat pour ne pas faire ta volonté en mon saint jour ; si tu appelles le sabbat tes délices, et honorable ce qui est saint à Yahweh, et si tu l'honores en ne suivant point tes voies, en ne te livrant pas à tes désirs et à des vains discours,
\VS{14}alors tu prendras plaisir en Yahweh, et je te ferai monter comme à cheval par-dessus les lieux haut élevés de la terre, et je te donnerai à manger de l'héritage de Jacob, ton père ; car la bouche de Yahweh a parlé.
\Chap{59}
\TextTitle{Le péché sépare de Yahweh}
\VerseOne{}Voici, la main de Yahweh n'est pas trop courte pour pouvoir sauver, ni son oreille trop pesante pour pouvoir entendre.
\VS{2}Mais ce sont vos iniquités qui mettent une séparation entre vous et votre Dieu ; ce sont vos péchés qui vous cachent sa face, afin qu'il ne vous entende point\FTNT{De. 31:17-18 ; Ez. 39:23-24.}.
\VS{3}Car vos mains sont souillées de sang, et vos doigts d'iniquité ; vos lèvres profèrent le mensonge, et votre langue déclare la perversité.
\VS{4}Nul ne crie pour la justice, nul ne plaide pour la vérité ; ils s'appuient sur des choses vaines et disent des faussetés, ils conçoivent le mal et enfantent l'iniquité.
\VS{5}Ils font éclore des œufs de vipère, et ils tissent des toiles d'araignée ; celui qui mange de leurs œufs meurt ; et si on les écrase, il en sort une vipère.
\VS{6}Leurs toiles ne servent point à faire des vêtements, et on ne se couvre pas de leurs ouvrages ; car leurs ouvrages sont des ouvrages d'iniquité, et il y a en leurs mains des actions de violence.
\VS{7}Leurs pieds courent au mal, et se hâtent pour répandre le sang innocent ; leurs pensées sont des pensées d'iniquité ; le ravage et la ruine sont sur leurs voies.
\VS{8}Ils ne connaissent point le chemin de la paix, et il n'y a point de jugement dans leurs voies, ils se sont pervertis dans leurs sentiers, tous ceux qui y marchent ignorent la paix\FTNT{Pr. 1:16 ; Pr. 6:16-19.}.
\VS{9}C'est pourquoi le jugement s'est éloigné de nous, et la justice ne parvient pas jusqu'à nous ; nous attendions la lumière, et voici les ténèbres, la clarté, et nous marchons dans l'obscurité.
\VS{10}Nous tâtonnons comme des aveugles le long du mur, nous tâtonnons comme ceux qui sont sans yeux ; nous chancelons en plein midi comme la nuit, et nous sommes dans les lieux abondants comme y sont des morts.
\VS{11}Nous rugissons tous comme des ours, et nous ne cessons de gémir comme des colombes ; nous attendons le jugement, et il n'y en a point, la délivrance, et elle est éloignée de nous.
\VS{12}Car nos transgressions se sont multipliées devant toi, et chacun de nos péchés témoignent contre nous ; parce que nos transgressions sont avec nous, et nous connaissons nos iniquités ;
\VS{13}qui sont de pécher et de mentir contre Yahweh, de s'éloigner de notre Dieu, de proférer l'oppression et la révolte, de concevoir et prononcer du cœur des paroles de mensonge.
\VS{14}C'est pourquoi le jugement s'est éloigné et la justice se tient éloignée ; car la vérité est tombée par les rues, et la droiture ne peut y entrer.
\VS{15}Même la vérité a disparu, et quiconque se retire du mal est exposé au pillage ; Yahweh voit, et cela lui a déplu, parce qu'il n'y a plus de droiture.
\TextTitle{Yahweh cherche un homme, il suscite le Messie}
\VS{16}Il voit aussi qu'il n'y a aucun homme, il s'étonne que personne ne se tienne à la brèche ; c'est pourquoi son bras lui vient en aide, et sa propre justice lui sert d'appui\FTNT{Es. 53:1 ; Es. 63:5 ; Ps. 77:15-16 ; Ac. 13:17.}.
\VS{17}Car il se revêt de la justice comme d'une cuirasse, et le casque du salut est sur sa tête\FTNT{Ep. 6:14-17.} ; il se revêt de la vengeance comme d'un vêtement, et se couvre de la jalousie comme d'un manteau.
\VS{18}Selon leurs actes, il rendra à chacun la pareille\FTNT{Jé. 17:10 ; Job. 34:11 ; Mt. 16:27 ; Ap. 2:23 ; Ap. 20:13.}, la fureur à ses adversaires, la rétribution à ses ennemis ; il rendra ainsi la rétribution aux îles.
\VS{19}Et on craindra le Nom de Yahweh depuis l'occident, et sa gloire depuis le soleil levant ; car l'ennemi viendra comme un fleuve, mais l'Esprit de Yahweh lèvera la bannière\FTNT{En hébreu « Yahweh Nissi », c'est-à-dire « Yahweh est ma bannière ». C'est le nom donné par Moïse à l'autel qu'il construisit pour célébrer la défaite d'Amalek (Ex. 17:15). En No. 21:8-9, Moïse éleva une bannière sur laquelle il avait fixé un serpent d'airain pour la guérison des malades. } contre lui.
\VS{20}Et le Rédempteur\FTNT{Le Rédempteur qui viendra pour Sion est le Seigneur Jésus-Christ (Ro. 11:26). Voir aussi Es. 60 : 16. } viendra en Sion, et vers ceux de Jacob qui se convertiront de leur péché, dit Yahweh.
\VS{21}Et quant à moi, c'est ici mon alliance que je ferai avec eux, dit Yahweh : Mon Esprit qui est sur toi, et mes paroles que j'ai mises dans ta bouche, ne se retireront point de ta bouche, ni de la bouche de ta postérité, ni de la bouche de la postérité de ta postérité, dit Yahweh, dès maintenant et à jamais.
\Chap{60}
\TextTitle{La gloire de Yahweh se lèvera sa gloire sur Sion}
\VerseOne{}Lève-toi, sois illuminée, car ta lumière arrive, et la gloire de Yahweh se lève sur toi.
\VS{2}Car voici, les ténèbres couvrent la terre, et l'obscurité couvre les peuples ; mais Yahweh se lève sur toi, et sa gloire apparaît sur toi.
\VS{3}Des nations marchent à ta lumière, et des rois à la splendeur qui se lève sur toi\FTNT{Ap. 21:24.}.
\VS{4}Elève tes yeux alentour, et regarde : Tous ceux-ci s'assemblent, ils viennent vers toi ; tes fils viennent de loin, et tes filles sont nourries par des nourriciers, étant portées sur les côtés.
\VS{5}Alors tu verras et tu seras éclairée, et ton cœur s'étonnera et s'épanouira de joie, quand l'abondance de la mer se sera tournée vers toi, et que la puissance des nations sera venue chez toi.
\VS{6}Tu seras couverte d'une foule de chameaux, des dromadaires de Madian et d'Epha ; et tous ceux de Séba viendront, ils apporteront de l'or et de l'encens, et publieront les louanges de Yahweh.
\VS{7}Toutes les brebis de Kédar seront assemblées vers toi, les béliers de Nebajoth seront à ton service ; ils seront agréables étant offerts sur mon autel, et je rendrai magnifique la maison de ma gloire.
\VS{8}Qui sont ceux-là qui volent comme des nuées, comme des colombes vers leur colombier ?
\VS{9}Car les îles s'attendent à moi, et les navires de Tarsis les premiers, afin d'amener de loin tes enfants, avec leur argent et leur or, à cause du Nom de Yahweh, ton Dieu, et du Saint d'Israël qui te glorifie.
\VS{10}Les fils des étrangers rebâtiront tes murailles, et leurs rois seront employés à ton service ; car je t'ai frappée dans ma colère, mais j'ai eu pitié de toi au temps de mon bon plaisir.
\VS{11}Tes portes seront continuellement ouvertes, elles ne seront fermées ni nuit ni jour, afin que les forces des nations te soient amenées et que leur roi y soient conduits\FTNT{Ap. 21:25-26.}.
\VS{12}Car la nation et le royaume qui ne te serviront pas périront, et ces nations-là seront réduites en une entière désolation.
\VS{13}La gloire du Liban viendra vers toi, le cyprès, l'orme, et le buis, tous ensemble pour rendre honorable le lieu de mon sanctuaire ; et je rendrais glorieux le lieu de mes pieds.
\VS{14}Mais les enfants de tes oppresseurs viendront vers toi en se courbant, et tous ceux qui te méprisaient se prosterneront à tes pieds et t'appelleront la ville de Yahweh, la Sion du Saint d'Israël.
\VS{15}Au lieu d'avoir été délaissée et haïe, si bien que personne ne passait par toi, je te mettrai dans une élévation éternelle et dans une joie qui sera de génération en génération.
\VS{16}Et tu suceras le lait des nations, et tu suceras la mamelle des rois, et tu sauras que je suis Yahweh, ton Sauveur, ton Rédempteur\FTNT{Le verbe « ga'al » et le nom correspondant « go'el », ont été traduits respectivement en français par « racheter » et « rédempteur ». Selon la loi de Moïse, si quelqu'un perdait son héritage à cause d'une dette ou s'il se vendait comme esclave, lui et ses biens pouvaient être rachetés par un proche parent qui devait payer le prix de la rédemption (Lé. 25:23-55). Yahweh se présente comme le Rédempteur par excellence (Es. 49:26 ; Es. 60:16 ; Ps. 78:35 ; Ps. 130:7; Job. 19:25). Or Jésus-Christ «[…] a été fait pour nous sagesse, justice, sanctification et rédemption » (1 Co. 1:30). Les épîtres nous révèlent la rédemption qu'il a acquise pour nous : « nous avons la rédemption par son sang » (Ep. 1:7). La rédemption est le paiement d'une rançon, or il est écrit : « Jésus-Christ s'est donné en rançon pour nous tous » (1 Ti. 2:6). « Vous avez été rachetés à grand prix » (1 Co. 6:20).}, le Puissant de Jacob.
\VS{17}Je ferai venir de l'or au lieu de l'airain, et de l'argent au lieu du fer, et de l'airain au lieu du bois, et du fer au lieu des pierres ; et je ferai régner la paix et dominer la justice.
\VS{18}On n'entendra plus parler de violence dans ton pays ni de ravage et de ruine dans ton territoire ; mais tu appelleras tes murailles : Salut ; et tes portes : Louange.
\VS{19}Tu n'auras plus le soleil pour la lumière du jour, et la lueur de la lune ne t'éclairera plus, mais Yahweh sera pour toi la lumière éternelle\FTNT{Voir le commentaire en Ge 1:3.}, et ton Dieu sera ta gloire.
\VS{20}Ton soleil ne se couchera plus, et ta lune ne se retirera plus, car Yahweh te sera pour lumière perpétuelle, et les jours de ton deuil seront finis.
\VS{21}Quant à ton peuple, ils seront tous justes, ils posséderont la terre à toujours ; savoir le germe de mes plantes, l'œuvre de mes mains pour y être glorifié\FTNT{Es. 11:1 ; Ro. 15:12 ; Ap. 5:5 ; Ap. 22:16.}.
\VS{22}La petite famille deviendra un millier de personnes, et la moindre deviendra une nation puissante. Je suis Yahweh, je hâterai ces choses en leur temps.
\Chap{61}
\TextTitle{La mission du Messie}
\VerseOne{}L'Esprit du Seigneur Yahweh est sur moi, car Yahweh m'a oint pour évangéliser les malheureux ; il m'a envoyé pour guérir ceux qui ont le cœur brisé, pour proclamer aux captifs la liberté, et aux prisonniers l'ouverture de la prison ;
\VS{2}pour publier une année de grâce de Yahweh, et le jour de vengeance de notre Dieu ; pour consoler tous ceux qui mènent deuil\FTNT{Lu. 4:14-19.} ;
\VS{3}pour annoncer à ceux de Sion qui mènent deuil, que la magnificence leur sera donnée au lieu de la cendre, une huile de joie au lieu du deuil, un manteau de louange au lieu d'un esprit abattu\FTNT{Job. 29:14 ; Ja. 1:12 ; 1 Co.9:25 ; 2 Ti. 4:8.}, afin qu'on les appelle des térébinthes de la justice, une plantation de Yahweh, pour servir à sa gloire.
\VS{4}Et ils rebâtiront les ruines antiques, ils relèveront les lieux qui étaient auparavant désolés, et ils renouvelleront des villes ravagées, et les choses désolées d'âge en âge.
\VS{5}Et des étrangers s'y tiendront là et feront paître vos troupeaux, et les enfants de l'étranger seront vos laboureurs et vos vignerons.
\VS{6}Mais vous, vous serez appelés sacrificateurs de Yahweh, et on vous nommera serviteurs de notre Dieu\FTNT{Ap. 1:6 ; Ap. 5:10.} ; vous mangerez les richesses des nations, et vous vous glorifierez de leur gloire.
\VS{7}Au lieu de la honte que vous avez eue, les nations en auront le double, et elles crieront tout haut que la confusion est leur portion ; c'est pourquoi ils posséderont le double dans leur pays, et leur joie sera éternelle.
\VS{8}Car je suis Yahweh qui aime le jugement et qui hait la rapine pour l'holocauste ; j'établirai leur œuvre dans la vérité et je traiterai avec eux une alliance éternelle.
\VS{9}Et leur race sera connue parmi les nations, et ceux qui seront sortis d'eux seront connus parmi les peuples ; tous ceux qui les verront connaîtront qu'ils sont la race que Yahweh aura bénie.
\VS{10}Je me réjouirai extrêmement en Yahweh, et mon âme se réjouira en mon Dieu ; car il m'a revêtu des vêtements du salut, il m'a couvert du manteau de la justice, comme un époux qui se pare de magnificence, et comme une épouse qui s'orne de ses joyaux\FTNT{Os. 2:21-22 ; Ap. 19:7-8.}.
\VS{11}Car comme la terre fait éclore son germe, et comme un jardin fait germer ses semences, ainsi le Seigneur Yahweh fera germer la justice, et la louange en présence de toutes les nations.
\Chap{62}
\TextTitle{Yahweh proclamme la restauration d'Israël}
\VerseOne{}Pour l'amour de Sion, je ne me tiendrai pas tranquille, et pour l'amour de Jérusalem je ne prendrai point de repos, jusqu'à ce que sa justice sorte dehors comme une splendeur, et que sa délivrance ne soit allumée comme une lampe.
\VS{2}Alors les nations verront ta justice, et tous les rois ta gloire ; et on t'appellera d'un nouveau nom\FTNT{Ap. 2:17.}, que la bouche de Yahweh aura expressément déclaré.
\VS{3}Tu seras une couronne de gloire dans la main de Yahweh, un turban royal dans la main de ton Dieu.
\VS{4}On ne te nommera plus la délaissée, et on ne nommera plus ta terre la désolation ; mais on t'appellera mon bon plaisir en elle ; et on appellera ta terre l'épouse ; car Yahweh prend son bon plaisir en toi, et ta terre aura un époux.
\VS{5}Car comme le jeune homme épouse la vierge, comme tes enfants se marient chez toi, ainsi ton Dieu se réjouira en toi, de la joie qu'un époux a de son épouse.
\VS{6}Jérusalem, j'ai placé des gardes sur tes murailles tout le jour et toute la nuit, et ils ne se tairont point. Vous qui faites mention de Yahweh, ne gardez point le silence !
\VS{7}Et ne vous arrêtez pas de l'invoquer jusqu'à ce qu'il rétablisse Jérusalem et lui rende sa renommée sur la terre.
\VS{8}Yahweh l'a juré par sa droite et par son bras puissant : Je ne donnerai plus ton froment pour nourriture à tes ennemis, et les enfants des étrangers ne boiront plus ton vin excellent pour lequel tu as travaillé.
\VS{9}Mais ceux qui auront amassé le froment le mangeront et loueront Yahweh, et ceux qui auront récolté le vin le boiront dans les parvis de ma sainteté.
\VS{10}Passez, passez les portes ! Disant : Préparez le chemin du peuple ! Frayez, frayez la route, et ôtez-en les pierres ! Elevez une bannière vers les peuples.
\VS{11}Voici ce que Yahweh proclame aux extrémités de la terre : Dites à la fille de Sion : Voici, ton Sauveur vient\FTNT{De nombreux passages, notamment dans le livre d'Esaïe, présentent Dieu comme le sauveur, le seul sauveur (Es. 43:3 ; Es. 43:11 ; Os. 13:4) qui viendra pour délivrer son peuple (Es. 35:4 ; Es. 60:1 ; Za. 14:1-7). Jésus-Christ a accompli en tous points les prophéties relatives à la venue de Yahweh. Dieu est bel et bien venu sur terre il y a plus de 2000 ans et ce même Dieu revient bientôt (Ac. 1:11 ; Ap. 1:7).} ; voici, son salaire est avec lui, et sa récompense marche devant lui.
\VS{12}Et on les appellera le peuple saint, les rachetés de Yahweh\FTNT{1 Pi. 2:9 ; Ap. 5:9.} ; et toi, on t'appellera la recherchée, la ville non abandonnée.
\Chap{63}
\TextTitle{Le jour de vengeance du Messie\FTNTT{Es. 2:10-22 ; Ap. 19:11-21.}}
\VerseOne{}Qui est celui-ci qui vient d'Edom, de Botsra, en habits rouges, magnifiquement paré en son vêtement, marchant selon la grandeur de sa force ? C'est moi qui parle en justice et qui ai tout pouvoir de sauver.
\VS{2}Pourquoi tes vêtements sont-ils rouges, et pourquoi tes habits sont comme les habits de ceux qui foulent dans la cuve ?
\VS{3}J'ai été seul à fouler au pressoir, et nul homme d'entre les peuples n'était avec moi. Cependant, j'ai marché sur eux dans ma colère, et je les ai foulés dans ma fureur ; et leur sang a rejailli sur mes vêtements, et j'ai souillé tous mes habits.
\VS{4}Car le jour de la vengeance était dans mon cœur, et l'année de mes rachetés est venue.
\VS{5}Je regardais donc, il n'y avait personne pour m'aider ; et j'étais étonné, et  il n'y avait personne pour me soutenir ; mais mon bras m'a sauvé et ma fureur m'a soutenu.
\VS{6}Ainsi j'ai foulé des peuples dans ma colère, et je les ai enivrés dans ma fureur ; et j'ai abattu leur force par terre.
\TextTitle{Esaïe confesse les péchés du peuple}
\VS{7}Je ferai mention des bontés de Yahweh, qui sont les louanges de Yahweh, pour tous les bienfaits que Yahweh nous a faits ; car grande est la bonté envers la maison d'Israël, qu'il a traitée selon ses compassions et la richesse de sa miséricorde.
\VS{8}Car il a dit : Certainement, ils sont mon peuple, des enfants qui ne tricheront pas ! Et il a été pour eux un Sauveur.
\VS{9}Et dans toutes leurs détresses, il a été en détresse, et l'ange qui est devant sa face les a délivrés\FTNT{Ge. 16:7-10 ; Jg. 6:11-14 ; Za.1:11.} ; lui-même les a rachetés dans son amour et sa miséricorde, et il les a soutenus et portés, tous les jours d'autrefois.
\VS{10}Mais ils ont été rebelles, et ils ont attristé son Esprit saint\FTNT{Ep. 4:30.}, c'est pourquoi il est devenu leur ennemi, et il a lui-même combattu contre eux.
\VS{11}Et on se souvint des anciens jours de Moïse et de son peuple. Où est celui, a-t-on dit, qui les fit monter de la mer, avec les pasteurs de son troupeau ? Où est celui qui mit au milieu d'eux son Esprit saint ;
\VS{12}qui les dirigea par la droite de Moïse et par son bras glorieux ; qui fendit les eaux devant eux pour se faire un nom éternel ?
\VS{13}Qui les dirigea à travers les flots, comme un cheval dans le désert, sans qu'ils ne bronchent ?
\VS{14}L'Esprit de Yahweh les a menés au repos comme on mène une bête qui descend dans la vallée. C'est ainsi que tu as conduit ton peuple, afin de t'acquérir un nom glorieux.
\VS{15}Regarde du ciel et vois de ta demeure sainte et glorieuse : Où sont ton zèle et ta puissance ? Le son de tes entrailles et de tes compassions se retiennent-ils envers moi ?
\VS{16}Certes tu es notre Père, encore qu'Abraham ne nous connaisse pas, et qu'Israël ne nous reconnaisse pas ; Yahweh, c'est toi qui es notre Père, et ton Nom est notre Rédempteur de tout temps.
\VS{17}Pourquoi nous as-tu fait égarer loin de tes voies, ô Yahweh, et endurcis-tu notre cœur contre ta crainte ? Reviens, pour l'amour de tes serviteurs, des tribus de ton héritage !
\VS{18}Ton peuple saint n'a possédé le pays que peu de temps ; nos ennemis ont foulé ton sanctuaire.
\TextTitle{Prière du reste d'Israël à Yahweh pour sa délivrance}
\VS{19}Nous sommes comme ceux sur lesquels tu ne domines pas depuis longtemps, et sur lesquels ton Nom n'est point réclamé. Ô ! Si tu fendais les cieux, et si tu descendais, les montagnes s'ébranleraient devant toi !
\Chap{64}
\VerseOne{}Comme un feu de fonte est ardent, le feu fait bouillir l'eau, afin de faire connaître ton Nom à tes ennemis, et que les nations tremblent en ta présence.
\VS{2}Lorsque tu fis les choses redoutables que nous n'attendions pas, tu descendis et les montagnes tremblèrent devant toi.
\VS{3}Jamais on n'a appris ni entendu dire, et jamais l'œil n'a vu qu'un autre dieu que toi fît de telles choses pour ceux qui s'attendent à lui\FTNT{1 Co. 2:9.}.
\VS{4}Tu viens à la rencontre de celui qui se réjouit et qui agit avec justice, et  se souviennent de toi dans tes voies. Voici tu as été irrité parce que nous avons péché ; tes compassions sont éternelles, c'est pourquoi nous serons sauvés.
\VS{5}Or nous sommes tous devenus comme une chose souillée, et toute notre justice est comme le linge le plus souillé\FTNT{Ap. 19:8.} ; nous sommes tous flétris comme la feuille, et nos iniquités nous emportent comme le vent.
\VS{6}Il n'y a personne qui invoque ton Nom, qui se réveille pour s'attacher fortement à toi ; c'est pourquoi tu nous as caché ta face, et tu nous fais fondre par l'effet de nos iniquités.
\VS{7}Cependant, ô Yahweh, tu es notre Père ; nous sommes l'argile, et c'est toi qui nous as formés, et nous sommes tous l'ouvrage de ta main\FTNT{Es. 29:16 ; Es. 45:9 ; Jé. 18:6 ; Ro. 9:20-21.}.
\VS{8}Ne t'irrite pas à l'extrême, ô Yahweh, et ne te souviens pas à toujours de notre iniquité. Voici, regarde, nous te prions, nous sommes tous ton peuple.
\VS{9}Tes villes saintes sont devenues un désert ; Sion est devenue un désert, et Jérusalem une désolation.
\VS{10}Notre maison sainte et glorieuse, où nos pères te louaient, a été brûlée par le feu ; tout ce que nous avions de précieux a été dévasté.
\VS{11}Après cela, ô Yahweh, ne te retiendras-tu pas ? Ne cesseras-tu pas, et nous affligeras-tu à l'excès ?
\Chap{65}
\TextTitle{Réponse de Yahweh}
\VerseOne{}Je me suis fait recherché de ceux qui ne me demandaient point, et je me suis laissé trouver par ceux qui ne me cherchaient pas\FTNT{Mt. 7:7 ; Lu. 11:9.} ; j'ai dit à la nation qui ne s'appelait pas de mon Nom : Me voici, me voici !
\VS{2}J'ai tendu mes mains tous les jours vers un peuple rebelle, à ceux qui marche dans une mauvaise voie, au gré de ses pensées ;
\VS{3}vers un peuple qui m'irrite continuellement en face, qui sacrifie dans les jardins, et qui fait des parfums sur les autels de briques,
\VS{4}qui habite les sépulcres et passe la nuit dans les lieux désolés, qui mangent la chair de porc, et ayant dans ses vases le jus des choses abominables.
\VS{5}Qui dit : Retire-toi, ne m'approche pas, car je suis plus saint que toi ! Ceux-là sont une fumée dans mes narines, un feu ardent tout le jour.
\VS{6}Voici, ceci est écrit devant moi, je ne me tairai point, mais je leur ferai porter la peine, oui je leur ferai porter la peine
\VS{7}de vos iniquités, dit Yahweh, et les iniquités de vos pères ensemble, qui ont brûlé de l'encens sur les montagnes, et qui m'ont blasphémé sur les collines ; c'est pourquoi je leur mesurerai aussi dans leur sein le salaire de ce qu'ils ont fait au commencement.
\VS{8}Ainsi parle Yahweh : Comme quand on trouve du vin dans une grappe, on dit : Ne la détruis pas, car il y a là une bénédiction ! J'agirai de même à cause de mes serviteurs, afin de ne pas tous les détruire.
\VS{9}Je ferai sortir de Jacob une postérité, et de Juda celui qui héritera de mes montagnes ; et mes élus hériteront le pays, et mes serviteurs y habiteront.
\VS{10}Et Saron servira de pâturage au menu bétail, et la vallée d'Acor sera le gîte du gros bétail, pour mon peuple qui m'aura recherché.
\VS{11}Mais vous, qui abandonnez Yahweh et qui oubliez ma montagne sainte, qui dressez la table pour Gad\FTNT{Gad : Dieu de la fortune.}, et qui remplissez une coupe pour Meni\FTNT{Meni : divinité païenne assimilée à la lune et dont le nom signifie « destin, sort ou fortune ».},
\VS{12}je vous destine aussi à l'épée, et vous serez tous courbés pour être égorgés ; parce que j'ai appelé, et vous n'avez point répondu ; j'ai parlé, et vous n'avez point écouté ; mais vous avez fait ce qui me déplaît, et vous avez choisi les choses auxquelles je ne prends pas plaisir.
\VS{13}C'est pourquoi, ainsi parle le Seigneur, Yahweh : Voici, mes serviteurs mangeront, et vous aurez faim ; voici, mes serviteurs boiront, et vous aurez soif ; voici mes serviteurs se réjouiront, et vous serez honteux.
\VS{14}Voici, mes serviteurs se réjouiront avec chants de triomphe pour la joie qu'ils auront au cœur ; mais vous, vous crierez pour la douleur que vous aurez au cœur, et vous crierez à cause de l'accablement de votre esprit.
\VS{15}Et vous laisserez votre nom à mes élus comme malédiction ; et le Seigneur Yahweh vous fera mourir ; et il donnera à ses serviteurs un autre nom.
\VS{16}Celui qui se bénira sur la terre, se bénira par le Dieu de vérité ; et celui qui jurera sur la terre jurera par le Dieu de vérité ; car les détresses du passé seront oubliées, et même elles seront cachées devant mes yeux.
\TextTitle{De nouveaux cieux et une nouvelle terre}
\VS{17}Car voici, je vais créer de nouveaux cieux et une nouvelle terre\FTNT{Es. 66:22 ; 2 Pi. 3:13 ; Ap. 21:1.} ; et on ne se souviendra plus des choses précédentes, elles ne reviendront plus au cœur.
\VS{18}Réjouissez-vous plutôt et soyez à toujours dans l'allégresse, à cause de ce que je vais créer ; car voici je vais créer Jérusalem pour n'être que joie, et son peuple pour n'être qu'allégresse.
\VS{19}Je ferai de Jérusalem mon allégresse, et de mon peuple ma joie ; on n'y entendra plus le bruit des pleurs et le bruit des clameurs.
\VS{20}Il n'y aura plus désormais ni nourrisson ni vieillard qui n'accomplissent leurs jours ; car celui qui mourra âgé de cent ans sera encore jeune ; mais le pécheur âgé de cent ans sera maudit.
\VS{21}Ils bâtiront des maisons et y habiteront ; ils planteront des vignes et ils en mangeront le fruit.
\VS{22}Ils ne bâtiront pas des maisons pour qu'un autre y habite ; ils ne planteront pas des vignes pour qu'un autre en mange le fruit ; car les jours de mon peuple seront comme les jours des arbres ; et mes élus jouiront de l'œuvre de leurs mains.
\VS{23}Ils ne travailleront plus en vain, et ils n'engendreront plus des enfants pour être exposés à la frayeur ; car ils seront la postérité des bénis de Yahweh, et ceux qui sortiront d'eux seront avec eux.
\VS{24}Et il arrivera qu'avant qu'ils crient, je les exaucerai ; et lorsqu'encore ils parleront, je les aurai déjà entendus.
\VS{25}Le loup et l'agneau paîtront ensemble, le lion comme le bœuf mangeront de la paille, et la poussière sera la nourriture du serpent\FTNT{Es. 2:4 ; Es. 11:6-7.}. On ne nuira point et on ne fera aucun dommage sur toute ma montagne sainte, dit Yahweh.
\Chap{66}
\TextTitle{Yahweh réprouve l'hypocrisie et agrée ceux qui le craignent}
\VerseOne{}Ainsi parle Yahweh : Le ciel est mon trône, et la terre est le marchepied de mes pieds\FTNT{Mt. 5:34-35 ; Ac. 7:49.}. Quelle maison me bâtiriez-vous, et quel serait le lieu de mon repos ?
\VS{2}Car ma main a fait toutes ces choses, et c'est par moi que toutes ces choses ont eu leur être, dit Yahweh. Mais à qui regarderai-je ? A celui qui est affligé, qui a l'esprit abattu, et qui tremble à ma parole.
\VS{3}Celui qui égorge un bœuf est comme celui qui tuerait un homme ; celui qui sacrifie une brebis est comme celui qui romprait la nuque à un chien ; celui qui présente une offrande est comme celui qui offrirait le sang d'un pourceau ; celui qui fait un parfum d'encens est comme celui qui bénirait une idole ; tous ceux-là ont choisi leurs voies, et leur âme trouve du plaisir dans leurs abominations.
\VS{4}Moi aussi je ferai attention à leurs tromperies, et je ferai venir sur eux les choses qu'ils craignent ; parce que j'ai appelé, et personne n'a répondu, parce que j'ai parlé, et qu'ils n'ont point écouté ; mais ils ont fait ce qui est mal à mes yeux, et ils ont choisi les choses auxquelles je ne prends pas de plaisir. 
\VS{5}Ecoutez la parole de Yahweh, vous qui tremblez à sa parole ; vos frères, qui vous haïssent et qui vous repoussent comme une chose abominable, à cause de mon Nom disent : Que Yahweh montre sa gloire ! Il sera donc vu à votre joie mais eux seront honteux. 
\VS{6}Un son éclatant sort de la ville, un son sort du temple, le son de Yahweh, qui rend à ses ennemis selon leurs œuvres.
\TextTitle{Israël renaît en un jour}
\VS{7}Elle a enfanté, avant d'éprouver les douleurs de l'enfantement ; elle a donné naissance à un enfant mâle, avant que les souffrances lui viennent.
\VS{8}Qui a jamais entendu une telle chose ? Qui en a jamais vu de semblable ? Ferait-on qu'un pays naisse en un jour ? Ou une nation naîtrait-elle d'un seul coup\FTNT{Cette prophétie fait allusion à la création de l'Etat d'Israël le 14 mai 1948.} ? Car dès que Sion a été en travail, elle a enfanté ses enfants !
\VS{9}Moi qui fais enfanter les autres, ne ferais-je point enfanter Sion ? Dit Yahweh. Moi qui donne de la postérité aux autres, l'empêcherais-je d'enfanter ? Dit ton Dieu.
\TextTitle{Réjouissance à Jérusalem et consolation}
\VS{10}Réjouissez-vous avec Jérusalem, faites d'elle le sujet de votre allégresse, vous tous qui l'aimez ; vous tous qui menez deuil sur elle, réjouissez-vous avec elle d'une grande joie ;
\VS{11}afin que vous soyez allaités et rassasiés de la mamelle de ses consolations, afin que vous suciez le lait et que vous jouissiez à plaisir de la plénitude de sa gloire.
\VS{12}Car ainsi parle Yahweh : Voici, je ferai couler vers elle la paix comme un fleuve, et la gloire des nations comme un torrent débordé, et vous serez allaités, vous serez portés sur les côtés et caressés sur les genoux.
\VS{13}Je vous consolerai pour vous apaiser, comme quelqu'un que sa mère caresse pour l'apaiser, vous serez consolés dans Jérusalem.
\VS{14}Vous le verrez et votre cœur se réjouira, et vos os germeront comme l'herbe ; et la main de Yahweh sera connue de ses serviteurs ; mais il sera indigné contre ses ennemis.
\TextTitle{Jugement de Yahweh}
\VS{15}Car voici, Yahweh viendra avec le feu, et ses chars seront comme la tempête ; afin qu'il tourne sa colère en fureur, et sa menace en flamme de feu.
\VS{16}Car Yahweh exercera jugement contre toute chair par le feu et avec son épée ; et le nombre de ceux qui seront mis à mort par Yahweh sera grand.
\VS{17}Ceux qui se sanctifient et se purifient au milieu des jardins, l'un après l'autre, qui mangent de la chair de porc et des choses abominables, comme des souris, seront ensemble consumés, dit Yahweh.
\VS{18}Mais pour moi, voyant leurs œuvres et leurs pensées, le temps est venu de rassembler toutes les nations et les langues ; ils viendront et verront ma gloire.
\TextTitle{Toutes les nations adoreront Yahweh}
\VS{19}Car je mettrai un signe en eux, et j'enverrai ceux d'entre eux qui seront réchappés, vers les nations, à Tarsis, à Pul, à Lud, gens tirant de l'arc, à Tubal et à Javan, et vers les îles lointaines, qui n'ont point entendu ma renommée, et qui n'ont pas vu ma gloire ; et ils annonceront ma gloire parmi les nations.
\VS{20}Et ils amèneront tous vos frères d'entre toutes les nations, sur des chevaux, sur des chars et dans des litières, sur des mulets et sur des dromadaires, en offrande à Yahweh, à la montagne sainte, à Jérusalem, dit Yahweh, comme lorsque les enfants d'Israël apportent l'offrande dans un vase pur, à la maison de Yahweh.
\VS{21}Et même je prendrai aussi parmi eux des sacrificateurs, des Lévites, dit Yahweh.
\VS{22}Car comme les nouveaux cieux et la nouvelle terre que je vais faire subsisteront devant moi, dit Yahweh, ainsi subsistera votre postérité et votre nom.
\VS{23}Et il arrivera que de nouvelle lune en nouvelle lune, et de sabbat en sabbat, toute chair viendra se prosterner devant ma face, dit Yahweh.
\VS{24}Et quand ils sortiront dehors, ils verront les cadavres des hommes qui se sont rebellés contre moi ; car leur ver ne mourra point, et leur feu ne s'éteindra point\FTNT{Mc. 9:48.} ; et ils seront méprisés de tout le monde.
\PPE{}
\end{multicols}

%\clearpage\ShortTitle{Jérémie}\BookTitle{Jérémie}\BFont
\noindent\hrulefill
{\footnotesize
\textit{
\bigskip
{\centering{}
\\Auteur : Jérémie
\\(Heb. : Yirmeyah)
\\Signification : Celui que Yahweh a désigné
\\Thème : Avertissements et jugements
\\Date de rédaction : 7\up{ème} siècle av J.C\\}
}
%\bigskip
\textit{
\\Issu d'une famille de sacrificateurs, Jérémie fut appelé dès son plus jeune âge au service de Yahweh et exerça un ministère prophétique avant et pendant les premières années de déportation. Outre son message à Israël et aux nations, le livre de Jérémie révèle sa personnalité. On découvre alors que l'opposition de ses pairs fut l'une de ses expériences les plus douloureuses. En effet, ce récit raconte ses combats contre les faux prophètes et met en évidence les signes accompagnant les prophètes authentiques, à savoir la souffrance, la solitude, l'incompréhension et le rejet.
%\bigskip
\\Son message annonçait le jugement imminent de Dieu et invitait le peuple à la repentance pour éviter le châtiment de Yahweh. Après la chute de Jérusalem, alors que Nebucadnetsar lui avait laissé le choix, Jérémie décida de rester avec les plus pauvres plutôt que de partir pour Babylone. Cependant, des Israélites décidèrent de s'expatrier en Egypte et l'entraînèrent avec eux de force. En terre étrangère, Jérémie continua de porter le fardeau de son peuple, l'exhortant à réformer ses voies. 
%\bigskip
\\Parmi les prophéties de Jérémie, figure le retour du peuple d'Israël sur la terre promise avant la seconde venue de Christ.\bigskip
}
}
\par\nobreak\noindent\hrulefill
\begin{multicols}{2}
\Chap{1}
\TextTitle{Yahweh appelle Jérémie à son service}
\VerseOne{}Les Paroles de Jérémie, fils de Hilkija, d'entre les sacrificateurs qui étaient à Anathoth, dans le pays de Benjamin;
\VS{2}auquel fut adressée la parole de Yahweh aux jours de Josias, fils d'Amon, roi de Juda, la treizième année de son règne,
\VS{3}laquelle lui fut aussi adressée aux jours de Jojakim, fils de Josias, roi de Juda, jusqu'à la fin de la onzième année de Sédécias, fils de Josias, roi de Juda; savoir jusqu'au temps où  Jérusalem fut transportée, ce qui arriva au cinquième mois.
\VS{4}La parole de Yahweh me fut adressée, en disant :
\VS{5}Avant que je t'aie formé dans le ventre de ta mère, je te connaissais, et avant que tu sois sorti de son sein, je t'avais consacré, je t'avais établi prophète pour les nations\FTNT{Es. 49:5 ; Ga. 1:15.}.
\VS{6}Je répondis : Ah ! Seigneur Yahweh ! Voici, je ne sais pas parler, car je suis un enfant\FTNT{Ex. 4:10-11.}.
\VS{7}Et Yahweh me dit : Ne dis pas : Je suis un enfant. Car tu iras partout où je t'enverrai, et tu diras tout ce que je t'ordonnerai.
\VS{8}Ne crains pas de te montrer devant eux, car je suis avec toi pour te délivrer, dit Yahweh.
\VS{9}Puis Yahweh avança sa main et toucha ma bouche ; et Yahweh me dit : Voici, je mets mes paroles dans ta bouche.
\VS{10}Regarde, je t'établis aujourd'hui sur les nations et sur les royaumes, pour que tu arraches et que tu démolisses, pour que tu ruines et que tu détruises, pour que tu bâtisses et que tu plantes\FTNT{Jérémie devait d'abord arracher, démolir, ruiner et détruire avant de bâtir et de planter. Il y avait dans le temple de Jérusalem les autels de Baal et le pieu d'Asherah (2 R. 21). De même, avant de planter la Parole de Dieu qui est une semence plantée dans les cœurs (Mc. 4 : 3-17), il est nécessaire au préalable d'arracher et de renverser les fausses doctrines et le péché en les dénonçant.}.
\TextTitle{Yahweh confirme la mission de Jérémie et l'établit sur Juda}
\VS{11}Puis la parole de Yahweh me fut adressée, en disant : Que vois-tu, Jérémie ? Et je répondis : Je vois une branche d'amandier.
\VS{12}Et Yahweh me dit : Tu as bien vu ; car je me hâte d'exécuter ma parole.
\VS{13}La parole de Yahweh me fut adressée pour la seconde fois, en disant : Que vois-tu ? Et je répondis : Je vois un pot bouillant dont le devant est tourné vers le nord.
\VS{14}Et Yahweh me dit : le mal se découvrira du côté du nord sur tous les habitants de ce pays-ci.
\VS{15}Car voici, je vais appeler toutes les familles des royaumes du nord, dit Yahweh ; elles viendront et mettront chacune leur trône à l'entrée des portes de Jérusalem, contre toutes ses murailles à l'entour, et contre toutes les villes de Juda.
\VS{16}Et je prononcerai mes jugements contre eux, à cause de toute leur méchanceté, par laquelle ils m'ont délaissé, et ont fait des parfums à d'autres dieux, et se sont prosternés devant l'ouvrage de leurs mains. 
\VS{17}Toi donc, ceins tes reins, lève-toi, et dis-leur tout ce que je t'ordonnerai. Ne crains pas de te montrer devant eux, de peur que je ne te mette en pièces en leur présence.
\VS{18}Car voici, je t'établis aujourd'hui sur tout le pays comme une ville forte, une colonne de fer, et un mur d'airain, contre les rois de Juda, contre les chefs du pays, contre ses sacrificateurs, et contre le peuple du pays.
\VS{19}Et ils combattront contre toi, mais ils ne seront pas plus forts que toi ; car je suis avec toi, dit Yahweh, pour te délivrer.
\Chap{2}
\TextTitle{Yahweh dénonce l'attitude d'Israël et l'avertit}
\VerseOne{}La parole de Yahweh me fut adressée, en disant :
\VS{2}Va et crie aux oreilles de Jérusalem, et dis : Ainsi parle Yahweh : Je me souviens de la fidélité de ta jeunesse, de l'amour de tes fiançailles, quand tu me suivais au désert, dans une terre qu'on n'ensemence pas. 
\VS{3}Israël était une chose sainte à Yahweh, il était les prémices de son revenu\FTNT{Lé. 23:20 ; Pr. 3:9 ; Né. 10:35.}; tous ceux qui le dévoraient étaient coupables, il leur en arrivait du mal dit Yahweh.
\VS{4}Ecoutez la parole de Yahweh, maison de Jacob, et vous toutes, familles de la maison d'Israël !
\VS{5}Ainsi parle Yahweh : Quelle iniquité vos pères ont-ils trouvée en moi, pour qu’ils se soient éloignés de moi, et qu’ils aient marché après la vanité et soient devenus vains ?
\VS{6}Ils n'ont pas dit : Où est Yahweh qui nous a fait remonter du pays d'Egypte, qui nous a conduits par un désert, par un pays de landes et montagneux, par un pays aride et d'ombre de mort, par un pays où aucun homme n'avait passé, et où personne n'avait habité ? 
\VS{7}Je vous ai fait entrer dans un pays de verger, pour que vous en mangiez les fruits et les biens ; mais sitôt vous y êtes entrés, vous avez souillé mon pays, et vous avez rendu abominable mon héritage.
\VS{8}Les sacrificateurs n'ont pas dit : Où est Yahweh ? Les dépositaires de la loi ne m'ont pas connu, les pasteurs se sont révoltés contre moi, les prophètes ont prophétisé par Baal\FTNT{Baal. Voir Jg. 2:13.}, et sont allés après ce qui n'est d'aucun profit.
\VS{9}A cause de cela, je veux encore contester avec vous, dit Yahweh, je veux contester avec les fils de vos fils.
\VS{10}Passez par les îles de Kittim et voyez ! Envoyez quelqu'un à Kédar ; observez bien, et voyez s'il n'y a rien de semblable !
\VS{11}Y a-t-il une nation qui change ses dieux, quoiqu'ils ne soient pas des dieux ? Et mon peuple a changé sa gloire contre ce qui n'est d'aucun profit\FTNT{Ro. 1:23.} !
\VS{12}Cieux, soyez étonnés de cela ; frémissez d'horreur et soyez stupéfaits ! dit Yahweh.
\VS{13}Car mon peuple a commis doublement le mal : Ils m'ont abandonné, moi qui suis la source d'eaux vives\FTNT{Yahweh est la Source d'eaux vives. Jésus-Christ se présente aussi comme la Source d'eau vive (Jn. 4:13-14 ; Ap. 21:6).}, pour se creuser des citernes, des citernes crevassées qui ne peuvent pas retenir l'eau.
\VS{14}Israël est-il un esclave, ou un esclave né dans la maison ? Pourquoi donc est-il mis au pillage ?
\VS{15}Les lionceaux rugissent, poussent leurs cris contre lui, et ils mettent son pays en désolation ; ses villes sont brûlées, de sorte que personne n'y habite.
\VS{16}Même les fils de Noph et de Tachpanès te casseront le sommet de la tête.
\VS{17}Cela ne t'arrive-t-il pas parce que tu as abandonné Yahweh, ton Dieu, à l'époque où il te conduisait par le chemin ?
\VS{18}Et maintenant, qu'as-tu à faire d'aller en Egypte, pour boire l'eau du Schichor\FTNT{Schichor : Sombre, noir, boueux. Le Nil, une rivière ou un canal affluent du fleuve. Les Israélites préféraient ces eaux à Yahweh.} ? Qu'as-tu à faire d'aller en Assyrie, pour boire l'eau du fleuve ?
\VS{19}Ta méchanceté te châtiera, et tes débauches te jugeront, tu sauras et tu verras que c'est une chose mauvaise et amère d'abandonner Yahweh, ton Dieu, et de n'avoir de moi aucune crainte, dit le Seigneur, Yahweh des armées.
\VS{20}Tu as dès longtemps brisé ton joug, rompu tes liens, et tu as dit : Je ne veux plus être dans la servitude ! Mais sur toute haute colline et sous tout arbre vert tu t'es incliné, tu t'es prostitué.
\VS{21}Je t'avais moi-même plantée comme une vigne exquise, dont tout le plant était franc ; comment t'es-tu changée en sarments d'une vigne étrangère ?
\VS{22}Quand tu te laverais avec du nitre, et que tu prendrais beaucoup de savon, ton iniquité resterait encore marquée devant moi, dit le Seigneur, Yahweh.
\VS{23}Comment dirais-tu : Je ne me suis pas souillée, je ne suis pas allée après les Baals ? Regarde tes pas dans la vallée, reconnais ce que tu as fait, dromadaire à la course légère et vagabonde !
\VS{24}Anesse sauvage, accoutumée au désert, humant le vent à son plaisir. Qui l'arrêtera dans son ardeur ? Tous ceux qui la cherchent n'ont pas à se fatiguer ; ils la trouvent pendant son mois.
\VS{25}Garde ton pied de se déchausser, ton gosier d'avoir soif ! Mais tu dis : C'est en vain, non ! Car j'aime les dieux étrangers, et j'irai après eux.
\VS{26}Comme un voleur est confus quand il est surpris, ainsi seront confus ceux de la maison d'Israël, eux, leurs rois, leurs chefs, leurs sacrificateurs et leurs prophètes.
\VS{27}Ils disent au bois : Tu es mon père ! Et à la pierre : Tu m'as enfanté ! Car ils me tournent le dos, et non la face. Et ils disent dans le temps de leur malheur : Lève-toi, et sauve-nous !
\VS{28}Où donc sont tes dieux que tu t'es faits ? Qu'ils se lèvent, s'ils peuvent te sauver au temps de ton malheur ! Car tu as autant de dieux que de villes, ô Juda !
\VS{29}Pourquoi contesteriez-vous avec moi ? Vous vous êtes tous rebellés contre moi, dit Yahweh.
\VS{30}En vain ai-je frappé vos fils ; ils n'ont pas reçu d'instruction ; votre épée a dévoré vos prophètes comme un lion destructeur.
\VS{31}Hommes de cette génération, considérez la parole de Yahweh ! Ai-je été un désert pour Israël, ou un pays de ténèbres ? Pourquoi mon peuple dit-il : Nous sommes libres, nous ne viendrons plus à toi ?
\VS{32}La vierge oublie-t-elle ses ornements, la fiancée sa ceinture ? Mais mon peuple m'a oublié depuis des jours sans nombre.
\VS{33}Comme tu es habile dans tes voies pour chercher ce que tu aimes ! C'est pourquoi aussi tu accoutumes tes voies aux crimes.
\VS{34}Même sur les pans de ta robe se trouve le sang des pauvres, des innocents que tu n'as pas trouvés en effraction.
\VS{35}Malgré cela, tu dis : Oui, je suis innocent ! Certainement sa colère s'est détournée de moi ! Voici, je vais entrer en jugement avec toi, sur ce que tu as dit : Je n'ai pas péché.
\VS{36}Pourquoi tant te précipiter pour changer ton chemin ? Tu auras autant de confusion de l'Egypte que tu en as eu de l'Assyrie.
\VS{37}Tu sortiras même d'ici, ayant tes mains sur la tête ; car Yahweh rejette ceux en qui tu te confies, et tu n'auras aucune prospérité par eux.
\Chap{3}
\TextTitle{Israël comparé à une prostituée}
\VerseOne{}Il dit : Si un homme répudie sa femme, qu'elle le quitte et se joigne à un autre, cet homme retourne-t-il encore vers elle\FTNT{Lé. 21:7 ; De. 24:2.} ? Le pays même n'en serait-il pas entièrement souillé ? Or toi, tu t'es prostituée à plusieurs amants, et tu reviendrais à moi ! dit Yahweh.
\VS{2}Lève tes yeux vers les lieux élevés et regarde ! Où ne t'es-tu pas prostituée ! Tu te tenais sur les chemins, comme un Arabe dans le désert, et tu as souillé le pays par tes prostitutions et par ta méchanceté.
\VS{3}Aussi les pluies ont été retenues, et il n'y a pas eu de pluie de l'arrière-saison ; mais tu as eu le front d'une femme prostituée, tu n'as pas voulu avoir honte.
\VS{4}Maintenant, n'est-ce pas ? Tu cries vers moi : Mon père ! Tu as été l'ami de ma jeunesse !
\VS{5}Gardera-t-il à toujours sa colère ? La conservera-t-il à jamais\FTNT{Es. 57:16 ; Ps.103:9.} ? Voici, tu as ainsi parlé, tu as fait ces maux-là autant que tu as pu.
\TextTitle{Yahweh appelle Israël à la repentance}
\VS{6}Yahweh me dit au temps du roi Josias : As-tu vu ce qu'a fait Israël, l'infidèle ? Elle est allée sur toute haute colline et sous tout arbre vert, et elle s'y est prostituée.
\VS{7}Je disais : Après avoir fait toutes ces choses, elle reviendra à moi. Mais elle n'est pas revenue. Et sa sœur Juda, la perfide, l'a vu.
\VS{8}Quoique j'aie répudié Israël, l'infidèle, à cause de tous ses adultères, et que je lui aie donné sa lettre de divorce, j'ai vu que la perfide Juda, sa sœur, n'a pas eu de crainte, mais elle s'en est allée et s'est aussi prostituée.
\VS{9}Par le bruit de sa prostitution, elle a souillé le pays, elle a commis un adultère avec la pierre et le bois.
\VS{10}Malgré tout cela, sa sœur Juda, la perfide, n'est pas revenue à moi de tout son cœur ; c'est avec fausseté qu'elle l'a fait, dit Yahweh.
\VS{11}Et Yahweh me dit : Israël, l'infidèle, se montre plus juste que Juda, la perfide.
\VS{12}Va, crie ces paroles vers le nord, et dis : Reviens, Israël, l'infidèle, dit Yahweh. Je ne jetterai pas sur vous un regard sévère ; car je suis miséricordieux, dit Yahweh, je ne garde pas ma colère à toujours.
\VS{13}Reconnais seulement ton iniquité, que tu t'es rebellée contre Yahweh, ton Dieu, que tu as tourné çà et là tes pas vers les étrangers, sous tout arbre vert, et que tu n'as pas écouté ma voix, dit Yahweh.
\VS{14}Fils rebelles, convertissez-vous, dit Yahweh, car je suis votre maître. Je vous prendrai, un d'une ville, deux d'une famille, et je vous ferai entrer dans Sion.
\VS{15}Je vous donnerai des pasteurs selon mon cœur, qui vous paîtront avec intelligence et avec sagesse\FTNT{Jé. 23:5.}.
\VS{16}Lorsque vous aurez multiplié et fructifié dans le pays, en ces jours-là, dit Yahweh, on ne parlera plus de l'arche de l'alliance de Yahweh, elle ne viendra plus à la pensée ; on ne s'en souviendra plus, on ne s'apercevra plus de son absence, et l'on n'en fera pas une autre.
\VS{17}En ce temps-là, on appellera Jérusalem le trône de Yahweh ; toutes les nations s'assembleront à Jérusalem, au Nom de Yahweh, et elles ne marcheront plus suivant les penchants de leur mauvais cœur.
\VS{18}En ces jours-là, la maison de Juda marchera avec la maison d'Israël ; elles viendront ensemble du pays du nord au pays que j'ai donné en héritage à vos pères.
\VS{19}Je disais : Comment te mettrai-je parmi mes fils et te donnerai-je un pays désirable, le plus bel héritage des armées des nations ? Je disais : Tu m'appelleras : Mon père ! Et tu ne te détourneras pas de moi.
\VS{20}Mais, comme une femme est infidèle à son compagnon, ainsi vous m'avez été infidèles, maison d'Israël, dit Yahweh.
\VS{21}Une voix se fait entendre sur les lieux élevés ; ce sont les pleurs, les supplications des fils d'Israël ; car ils ont perverti leur voie, ils ont oublié Yahweh, leur Dieu.
\VS{22}Fils rebelles, convertissez-vous, je guérirai vos infidélités. Nous voici, nous venons à toi, car tu es Yahweh, notre Dieu.
\VS{23}Certainement, on s'attend en vain aux collines et à la multitude des montagnes ; mais c'est en Yahweh, notre Dieu, qu'est la délivrance d'Israël.
\VS{24}Car la honte a dévoré dès notre jeunesse le travail de nos pères, leurs brebis et leurs bœufs, leurs fils et leurs filles.
\VS{25}Nous serons gisants dans notre honte, et notre ignominie nous couvrira ; parce que nous avons péché contre Yahweh, notre Dieu, nous et nos pères, dès notre jeunesse jusqu'à ce jour, et nous n'avons pas obéi à la voix de Yahweh, notre Dieu.
\Chap{4}
\TextTitle{Prophétie sur l'invasion du pays}
\VerseOne{}Israël, si tu reviens, dit Yahweh, si tu reviens à moi, si tu ôtes tes abominations de devant moi, tu ne seras plus errant ça et là.
\VS{2}Alors tu jureras avec vérité, avec droiture et avec justice : Yahweh est vivant ! Et les nations seront bénies en lui, et se glorifieront en lui.
\VS{3}Car ainsi parle Yahweh aux hommes de Juda et de Jérusalem : Labourez pour vous une terre arable et ne semez pas parmi les épines\FTNT{Mt. 13:7 ; Mt. 13:22 ; Mc. 4:7 ; Mc. 4:18 ; Lu. 8:14.}.
\VS{4}Hommes de Juda, et vous habitants de Jérusalem, circoncisez-vous pour Yahweh, circoncisez vos cœurs\FTNT{Ro. 2:29.}, de peur que ma fureur ne sorte comme un feu et qu'elle ne brûle sans qu'on puisse l'éteindre, à cause de la méchanceté de vos actions.
\VS{5}Annoncez en Juda, publiez dans Jérusalem, et dites : Sonnez du shofar dans le pays ! Criez à pleine voix et dites : Assemblez-vous et nous entrerons dans les villes fortes !
\VS{6}Elevez une bannière vers Sion, fuyez, ne vous arrêtez pas ! Car je fais venir du nord le malheur et une grande calamité.
\VS{7}Le lion\FTNT{Lion est ici une allusion à Nebucadnetsar, roi de Babylone. Voir 2 R. 24 et 25 ; Da. 7:4.} est sorti de la caverne, le destructeur des nations est en marche, il est sorti de son lieu, pour réduire ton pays en désert ; tes villes seront ruinées, il n'y aura personne pour y habiter.
\VS{8}C'est pourquoi ceignez-vous de sacs, lamentez-vous et gémissez ; car l'ardeur de la colère de Yahweh ne se détourne pas de nous.
\VS{9}Et il arrivera ce jour-là, dit Yahweh, que le cœur du roi et le cœur des chefs seront épouvantés et que les sacrificateurs seront étonnés, et que les prophètes seront stupéfaits.
\VS{10}C'est pourquoi je dis : Ah ! Seigneur Yahweh ! Oui certainement tu as abusé ce peuple et Jérusalem, en disant : Vous aurez la paix ! Et cependant l'épée est venue jusqu'à l'âme.
\VS{11}En ce temps-là on dira à ce peuple et à Jérusalem : Un vent brûlant souffle des lieux élevés du désert sur le chemin de la fille de mon peuple, non pas pour vanner ni pour nettoyer.
\VS{12}C'est un vent impétueux qui vient de là jusqu'à moi et je leur ferai maintenant leur procès. 
\VS{13}Voici, il monte comme des nuées ; ses chars sont comme un tourbillon, ses chevaux sont plus légers que les aigles. Malheur à nous, car nous sommes détruits !
\VS{14}Jérusalem, lave ton cœur du mal afin que tu sois délivrées ! Jusqu'à quand séjourneront-tu au-dedans de toi les pensées de ton injustice ?
\VS{15}Car une voix apporte des nouvelles de Dan, elle publie depuis la montagne d'Ephraïm le tourment.
\VS{16}Rappelez-le aux nations, faites-le entendre à Jérusalem : Des observateurs viennent d'un pays éloigné ; ils poussent des cris contre les villes de Juda.
\VS{17}Ils se sont mis tout autour d'elle comme ceux qui gardent un champ, parce qu'elle s'est rebellée contre moi, dit Yahweh.
\VS{18}Ta conduite et tes actions t'ont produit ces choses, telle a été ta méchanceté, parce que cela a été une chose amère, certainement elle t'atteindra jusqu'à ton cœur.
\VS{19}Mes entrailles ! Mes entrailles : Je suis dans la douleur au-dedans de mon cœur, mon cœur bat, je ne puis me taire ; car, ô mon âme, tu entends le son du shofar, la clameur de la guerre.
\VS{20}On annonce brèche sur brèche, car tout le pays est dévasté ; mes tentes sont détruites tout à coup, mes pavillons en un moment.
\VS{21}Jusqu'à quand verrai-je la bannière et entendrai-je le son du shofar ?
\VS{22}Car mon peuple est insensé ; ils ne m'ont pas reconnu, ce sont des enfants insensés qui n'ont pas d'intelligence ; ils sont habiles pour faire le mal, et ils ne savent pas faire le bien.
\VS{23}Je regarde la terre, et voici, elle est informe et vide\FTNT{Dieu n'a pas créé la terre informe et vide, mais elle l'est devenue à cause du péché. Voir le commentaire en Gn. 1:2.} ; les cieux et leur lumière ne sont plus.
\VS{24}Je regarde les montagnes, et voici, elles sont ébranlées ; et toutes les collines sont renversées.
\VS{25}Je regarde, et voici, il n'y a pas un seul homme et tous les oiseaux des cieux se sont enfuis.
\VS{26}Je regarde, et voici, le Carmel est un désert ; et toutes ses villes sont détruites, devant Yahweh, devant l'ardeur de sa colère.
\VS{27}Car ainsi parle Yahweh : Tout le pays sera dévasté, mais je ne ferai pas une entière destruction.
\VS{28}C'est pourquoi le pays mènera deuil et les cieux en haut seront obscurcis, parce que je l'ai dit, je l'ai résolu, et je ne m'en repentirai pas et je le révoquerai pas.
\VS{29}Toute la ville s'enfuit à cause du bruit des cavaliers et des archers ; ils entrent dans les bois fourrés et montent sur les rochers ; toute la ville est abandonnée, et aucun homme n'y habite.
\VS{30}Et quand tu auras été détruite que fais-tu ? Quoique tu te revêtes de pourpre, que tu te pares d'ornements d'or, et que tu bordes tes yeux de fard, tu t'embellis en vain: tes amants t'ont méprisée; c'est ta vie qu'ils cherchent.
\VS{31}Car j'entends un cri comme celui d'une femme qui est en travail, et une angoisse comme celle d'une femme qui est en travail de son premier-né ; c'est le cri de la fille de Sion ; elle soupire, elle étend ses mains, en disant : Malheur maintenant à moi, car mon âme a défailli à cause des meurtriers. 
\Chap{5}
\TextTitle{Raisons du jugement de Yahweh}
\VerseOne{}Parcourez les rues de Jérusalem et regardez maintenant, sachez et cherchez dans les places, si vous y trouvez un homme de bien, s'il y a quelqu'un qui fasse ce qui est droit, qui cherche la vérité, et je pardonne à Jérusalem\FTNT{Es. 59:15 ; Mi. 7:2 ; Pr. 20:6.}.
\VS{2}Même s'ils disent : Yahweh est vivant ! En cela, ils jurent faussement.
\VS{3}Yahweh, tes yeux ne regardent-ils pas à la fidélité ? Tu les frappes, et ils ne sentent pas de douleur ; tu les consumes, et ils refusent de recevoir l'instruction ; ils endurcissent leurs faces plus qu'un rocher, ils refusent de se convertir.
\VS{4}Je disais : Certainement ce ne sont que les plus petits ; ils se montrent insensés parce qu'ils ne connaissent pas la voie de Yahweh, le droit de leur Dieu.
\VS{5}J'irai donc vers les plus grands, et je leur parlerai ; car cela connaissent la voie de Yahweh, le droit de leur Dieu ; mais ceux-là même ont brisés le joug et ont rompu les liens.
\VS{6}C'est pourquoi le lion de la forêt les tue, le loup du soir les détruit, et le léopard est aux aguets contre leurs villes ; quiconque en sortira sera déchiré ; car leurs transgressions sont nombreuses, et leurs infidélités se sont renforcées.
\VS{7}Comment te pardonnerais-je en cela ? Tes fils m'ont abandonné, et ils jurent par ce qui ne sont pas dieux. Je les ai rassasiés, mais ils commettent l'adultère et ils se pressent en foule dans la maison de la prostituée.
\VS{8}Ils sont comme des chevaux bien nourris, quand ils se lèvent le matin, chacun hennit après la femme de son prochain.
\VS{9}Ne punirais-je pas ces choses-là, dit Yahweh ? Et mon âme ne se vengerait-elle pas d'une telle nation ?
\VS{10}Montez sur ses murailles et détruisez-les, mais ne les achevez pas entièrement ! Otez ses sarments car ils ne sont pas à Yahweh\FTNT{Jn. 15:5.} !
\VS{11}Car la maison d'Israël et la maison de Juda m'ont été infidèles, dit Yahweh.
\VS{12}Ils démentent Yahweh, et disent : Cela n'arrivera pas, et le malheur ne viendra pas sur nous, nous ne verrons ni l'épée ni la famine.
\VS{13}Et les prophètes sont légers comme le vent, et la parole n'est pas en eux. Qu'il leur soit fait ainsi !
\VS{14}C'est pourquoi ainsi parle Yahweh, le Dieu des armées : Parce que vous avez prononcé cette parole-là, voici, je vais mettre mes paroles dans ta bouche pour y être comme une feu, et ce peuple sera comme le bois, et ce feu les consumera.
\VS{15}Maison d'Israël, voici, je fais venir contre vous une nation d'un pays éloigné\FTNT{Il s'agit de Babylone. Voir 2 R. 24 et 25.}, dit Yahweh, une nation puissante, une nation ancienne, une nation dont tu ne connais pas la langue, et dont tu ne comprendras pas ce qu'elle dira.
\VS{16}Son carquois est comme un sépulcre ouvert, et ils sont tous des hommes vaillants.
\VS{17}Et elle dévorera ta moisson et ton pain, que tes fils et tes filles devaient manger; elle dévorera tes brebis et tes bœufs; elle dévorera les fruits ta vigne et ton figuier et réduira à la pauvreté par l'épée tes villes fortes dans lesquelles tu te confies.
\VS{18}Toutefois en ces jours-là, dit Yahweh, je ne vous achèverai pas entièrement.
\VS{19}Et il arrivera que vous direz : Pourquoi Yahweh, notre Dieu, nous a-t-il fait toutes ces choses ? Tu leur diras ainsi : Comme vous m'avez abandonné et que vous avez servi les dieux étrangers dans votre pays, ainsi vous servirez des étrangers dans un pays qui n'est pas le vôtre.
\VS{20}Annoncez ceci dans la maison de Jacob, et publiez-le dans Juda, en disant :
\VS{21}Ecoutez maintenant ceci, peuple insensé, et qui n'avez pas d'intelligence; qui avez des yeux et ne voyez pas; et qui avez des oreilles et n'entendez pas\FTNT{Ez. 12:2 ; Jn. 12:40.}.
\VS{22}Ne me craindrez-vous pas, dit Yahweh, ne tremblerez-vous pas devant ma face ? C'est moi qui ai mis le sable pour limite à la mer, par une ordonnance perpétuelle et qui ne passera pas ; ses vagues s'agitent, mais elles sont impuissantes ; elles grondent, mais elles ne la passent pas\FTNT{Pr. 8:29 ; Job. 38:8.}.
\VS{23}Mais ce peuple-ci a un cœur indocile et rebelle ; ils reculent en arrière et s'en vont.
\VS{24}Et ils ne disent pas dans leur cœur : Craignons maintenant Yahweh, notre Dieu, qui nous donne la pluie en son temps, de la première et de l'arrière-saison, et qui nous réserve les semaines ordonnées pour la moisson.
\VS{25}Vos iniquités ont détourné ces choses, vos péchés retiennent loin de vous le bien.
\VS{26}Car il se trouve parmi mon peuple des méchants ; ils épient comme l'oiseleur qui dresse des pièges, ils tendent des filets et prennent des hommes\FTNT{Ps. 91:3 ; Ps. 124:7.}.
\VS{27}Comme la cage est remplie d'oiseaux, ainsi leurs maisons sont remplies de fraude ; c'est par ce moyen qu'ils deviennent grands et riches.
\VS{28}Ils s'engraissent, ils sont brillants ; ils surpassent les actions des méchants, ils ne jugent pas la cause, la cause de l'orphelin, et ils prospèrent ; ils ne font pas droit aux pauvres.
\VS{29}Ne punirais-je pas ces choses-là, dit Yahweh ? Et mon âme ne se vengerait-elle pas d'une telle nation ?
\VS{30}Il est arrivé dans le pays une chose étonnante et horrible :
\VS{31}C'est que les prophètes prophétisent le mensonge, et les sacrificateurs dominent par leur moyen, et mon peuple prend plaisir à cela. Que ferez-vous donc quand elle prendra fin ?
\Chap{6}
\TextTitle{Jérusalem dans la confusion}
\VerseOne{}Fils de Benjamin, fuyez par troupes du milieu de Jérusalem, et sonnez du shofar à Tekoa, et élevez un signal de feu à Beth-Hakkérem ! Car on voit venir du nord un malheur et une grande ruine.
\VS{2}La belle et la délicate, la fille de Sion, je la détruis !
\VS{3}Les pasteurs avec leurs troupeaux viennent contre elle ; ils plantent leurs tentes autour d'elle, chacun paîtra en son quartier.
\VS{4}Préparez le combat contre elle ! Levez-vous, et montons en plein midi !… Malheur à nous, car le jour décline, les ombres du soir s'étendent.
\VS{5}Levez-vous ! Montons de nuit, et ruinons ses palais !
\VS{6}Car ainsi parle Yahweh des armées : Coupez des arbres, élevez des terrasses contre Jérusalem ! C'est la ville qui doit être visitée ; tout est oppression au milieu d'elle.
\VS{7}Comme le puits fait jaillir ses eaux, ainsi elle fait jaillir sa méchanceté ; on n'entend continuellement en elle, devant moi, que violence et ruine, avec des maladies et des plaies.
\VS{8}Jérusalem, reçois l'instruction, de peur que mon âme ne se retire de toi, et que je ne fasse de toi un désert, et une terre inhabitée !
\VS{9}Ainsi parle Yahweh des armées : On grappillera entièrement comme une vigne les restes d'Israël. Remets ta main dans les paniers, comme un vendangeur.
\VS{10}A qui parlerai-je, et qui prendrai-je à témoin, pour qu'ils écoutent ? Voici, leur oreille est incirconcise, et ils ne peuvent entendre ; voici, la parole de Yahweh leur est en opprobre, ils n'y prennent point de plaisir.
\VS{11}C'est pourquoi je suis plein de la fureur de Yahweh, et je suis las de la contenir. Répands-la sur les enfants dans la rue, et sur les assemblées des jeunes gens. Car tant le mari que la femme seront pris, le vieillard et celui qui est chargé de jours.
\VS{12}Et leurs maisons passeront à d'autres, les champs et les femmes aussi, quand j'étendrai ma main sur les habitants du pays, dit Yahweh.
\VS{13}Car depuis le plus petit d'entre eux jusqu'au plus grand, chacun s'adonne au gain déshonnête, tant le prophète que le sacrificateur, tous se agissent faussement.
\VS{14}Et ils pansent à la légère la plaie de la fille de mon peuple, disant : Paix ! Paix ! et il n'y a pas de paix\FTNT{1 Th. 5:3.}.
\VS{15}Sont-ils confus d'avoir commis des abominations ? Ils n'en ont même aucune honte, et ils ne savent pas ce que c'est que de rougir ; c'est pourquoi ils tomberont parmi ceux qui tombent, ils seront renversés au temps où je les visiterai, dit Yahweh.
\VS{16}Ainsi parle Yahweh : Tenez-vous sur les chemins, regardez et enquérez-vous des sentiers des siècles passés, quel est le bon chemin ; et marchez-y, et vous trouverez le repos de vos âmes ! Et ils répondent : Nous n'y marcherons pas.
\VS{17}J'ai aussi établi sur vous des sentinelles\FTNT{Es. 21:6 ; Ez. 33:1-19.} qui disent : Soyez attentifs au son du shofar ! Mais ils répondent : Nous n'y serons pas attentifs.
\VS{18}Vous donc, nations, écoutez, et toi assemblée, connais ce qui est entre eux.
\VS{19}Ecoute, terre ! Voici, je fais venir un mal sur ce peuple, à savoir le fruit de leurs pensées ; car ils n'ont pas été attentifs à mes paroles, et qu'ils ont rejeté ma loi.
\VS{20}Pourquoi m'offrir de l'encens venu de Séba, et le bon roseau aromatique du pays éloigné ? Vos holocaustes ne me plaisent pas, et vos sacrifices ne me sont pas agréables.
\VS{21}C'est pourquoi ainsi parle Yahweh : Voici, je mettrai devant ce peuple des pierres d'achoppement, auxquels les pères et les fils, le voisin et son compagnon, se heurteront ensemble et ils périront.
\VS{22}Ainsi parle Yahweh : Voici, un peuple vient du pays du nord, et une grande nation se réveille des extrémités de la terre.
\VS{23}Ils prendront l'arc et le javelot ; ils sont cruels et n'ont pas de pitié ; leur voix gronde comme la mer ; ils sont montés sur des chevaux, ils sont rangés comme un seul homme en bataille contre toi, fille de Sion !
\VS{24}Nous en entendons le bruit, nos mains en deviennent lâches, l'angoisse nous saisit, et une douleur comme celle d'une femme qui enfante.
\VS{25}Ne sortez pas dans les champs, n'allez pas par les chemins ; car l'épée de l'ennemi, la terreur est partout.
\VS{26}Fille de mon peuple, ceins-toi d'un sac et roule-toi dans la cendre, prends le deuil comme pour un fils unique, fais une lamentation très amère ! Car le dévastateur vient subitement sur nous.
\VS{27}Je t'avais établi en observateur au milieu de mon peuple, comme une forteresse, pour que tu connaisses et que tu éprouves leur voie.
\VS{28}Ils sont tous rebelles et plus que rebelles, des calomniateurs, ils sont comme de l'airain et du fer ; ils sont tous corrompus.
\VS{29}Le soufflet est brûlant, le plomb est consumé par le feu ; c'est en vain que l'on fond et refond, car les mauvais ne sont pas séparés.
\VS{30}On les appelle de l'argent réprouvé, car Yahweh les a réprouvés.
\Chap{7}
\TextTitle{Hypocrisie de Juda}
\VerseOne{}La parole fut adressée à Jérémie de la part de Yahweh, en disant :
\VS{2}Tiens-toi debout à la porte de la maison de Yahweh, et là, crie cette parole, et dis : Ecoutez la parole de Yahweh, vous tous, hommes de Juda, qui entrez par ces portes, pour vous prosterner devant Yahweh !
\VS{3}Ainsi parle Yahweh des armées, le Dieu d'Israël : Amendez vos voies et vos actions, et je vous ferai habiter en ce lieu-ci.
\VS{4}Ne vous confiez pas en des paroles trompeuses, en disant : C'est ici le temple de Yahweh, le temple de Yahweh, le temple de Yahweh !
\VS{5}Mais amendez sérieusement vos voies et vos actions, et appliquez-vous à faire droit à ceux qui plaident l'un contre l'autre,
\VS{6}et ne faites pas de tort à l'étranger, ni à l'orphelin, ni à la veuve, et ne répandez pas en ce lieu-ci le sang innocent, et ne marchez pas après les dieux étrangers, pour votre malheur.
\VS{7}Et je vous ferai habiter depuis un siècle jusqu'à l'autre siècle en ce lieu-ci, dans le pays que j'ai donné à vos pères.
\VS{8}Voici, vous vous confiez en des paroles trompeuses, sans aucun profit.
\VS{9}Ne dérobez-vous pas ? Ne tuez-vous pas ? Ne commettez-vous pas adultère ? Ne jurez-vous pas faussement ? Ne faites-vous pas des encensements à Baal ? N'allez-vous pas après les dieux étrangers, que vous ne connaissez point ?
\VS{10}Toutefois vous venez et vous vous présentez devant moi, dans cette maison sur laquelle mon Nom est invoqué, et vous dites : Nous sommes délivrés !… Pour faire toutes ces abominations !
\VS{11}N'est-elle plus à vos yeux qu'une caverne de voleurs\FTNT{Mt. 21:13 ; Mc. 11:17 ; Lu. 19:46.}, cette maison sur laquelle mon Nom est invoqué ? Et voici, moi-même je le vois, dit Yahweh.
\VS{12}Mais allez maintenant à mon lieu qui était à Silo, où j'avais fait demeurer mon Nom au commencement. Et regardez ce que je lui ai fait, à cause de la méchanceté de mon peuple d'Israël.
\VS{13}Maintenant donc, puisque vous avez fait toutes ces actions, dit Yahweh, puisque je vous ai parlé, parlé dès le matin, et que vous n'avez pas écouté, puisque je vous ai appelés et que vous n'avez pas répondu ;
\VS{14}je ferai à cette maison sur laquelle mon Nom est invoqué, et sur laquelle vous vous confiez, et à ce lieu que je vous ai donné à vous et à vos pères, comme j'ai fait à Silo ;
\VS{15}et je vous chasserai de devant ma face, comme j'ai chassé tous vos frères, avec toute la postérité d'Ephraïm.
\VS{16}Toi donc ne prie pas pour ce peuple, et n'élève pour eux ni cri ni prière, et n'intercède pas auprès de moi\FTNT{Ez. 3:26-27.} ; car je ne t'écouterai pas.
\VS{17}Ne vois-tu pas ce qu'ils font dans les villes de Juda et dans les rues de Jérusalem ?
\VS{18}Les fils ramassent le bois, et les pères allument le feu, et les femmes pétrissent la pâte pour faire des gâteaux à la reine des cieux\FTNT{La reine des cieux est une déesse qui change de nom en fonction des pays. Asherah, Astarté, Isis, Junon, Cybèle, Diane ou encore la vierge Marie, proclamée mère de Dieu en 431 au concile d'Ephèse. Voir De. 16:2-3.}, et pour faire des libations aux dieux étrangers, afin de m'irriter.
\VS{19}Est-ce moi qu'ils irritent ? dit Yahweh ; n'est-ce pas contre eux-mêmes, à la confusion de leurs faces ?
\VS{20}C'est pourquoi ainsi parle le Seigneur Yahweh : Voici, ma colère et ma fureur se répandent sur ce lieu-ci, sur les hommes et sur les bêtes, sur les arbres des champs et sur le fruit de la terre ; ma colère brûlera et ne s'éteindra pas.
\VS{21}Ainsi parle Yahweh des armées, le Dieu d'Israël : Ajoutez vos holocaustes à vos sacrifices, et mangez-en la chair !
\VS{22}Car je n'ai pas parlé avec vos pères et je ne leur ai pas donné d'ordre au sujet des holocaustes et des sacrifices, le jour où je les ai fait sortir du pays d'Egypte.
\VS{23}Mais voici la parole que je leur ai commandée, disant : Ecoutez ma voix, et je serai votre Dieu, et vous serez mon peuple ; marchez dans toutes les voies que je vous ordonne, afin que vous soyez heureux\FTNT{Ex. 15:26.}.
\VS{24}Mais ils n'ont pas écouté, et n'ont pas prêté l'oreille ; mais ils ont suivi d'autres conseils, les penchants de leur mauvais cœur ; ils se sont éloignés et ne sont pas revenus à moi.
\VS{25}Depuis le jour où vos pères sont sortis du pays d'Egypte, jusqu'à ce jour, je vous ai envoyé tous mes serviteurs les prophètes, je les ai envoyés chaque jour, dès le matin.
\VS{26}Mais ils ne m'ont pas écouté, et ils n'ont pas prêté l'oreille ; mais ils ont raidi leur cou, ils ont fait le mal plus que leurs pères.
\VS{27}Tu leur diras toutes ces paroles, mais ils ne t'écouteront pas ; et tu crieras après eux, mais ils ne te répondront pas.
\VS{28}C'est pourquoi tu leur diras : C'est ici la nation qui n'écoute pas la voix de Yahweh, son Dieu, et qui ne reçoit pas d'instruction ; la vérité a disparu, elle s'est retirée de leur bouche.
\VS{29}Coupe ta chevelure, ô Jérusalem ! Et jette-la au loin, et prononce à haute voix ta complainte sur les lieux élevés ! Car Yahweh rejette et abandonne la génération qui a provoqué sa fureur.
\VS{30}Car les fils de Juda ont fait ce qui est mal à mes yeux, dit Yahweh ; ils ont mis leurs abominations dans cette maison sur laquelle mon Nom est invoqué, afin de la souiller.
\VS{31}Et ils ont bâti les hauts lieux de Topheth, qui est dans la vallée de Ben-Hinnom\FTNT{Voir commentaire en Ap. 16:16.}, pour brûler au feu leurs fils et leurs filles\FTNT{Lé. 18:21. Voir commentaire en Lé. 20:2.} : Ce que je n'avais pas ordonné, et à quoi je n'ai jamais pensé.
\VS{32}C'est pourquoi voici, les jours viennent, dit Yahweh, qu'elle ne sera plus appelée Topheth, ni la vallée de Ben-Hinnom, mais la vallée de la tuerie ; et on enterrera les morts à Topheth, à cause qu'il n'y aura plus d'autre lieu.
\VS{33}Et les cadavres de ce peuple seront la pâture des oiseaux des cieux et des bêtes de la terre ; sans qu'il n'y ait personne qui les effraye.
\VS{34}Je ferai aussi cesser dans les villes de Juda et dans les rues de Jérusalem les cris de joie et les cris d'allégresse, la voix de l'époux et la voix de l'épouse ; car le pays sera un désert.
\Chap{8}
\TextTitle{Juda dans l'égarement}
\VerseOne{}En ce temps-là, dit Yahweh, on sortira les os des rois de Juda, et les os de ses chefs, les os des sacrificateurs, et les os des prophètes, et les os des habitants de Jérusalem, hors de leurs sépulcres.
\VS{2}Et on les étendra devant le soleil, et devant la lune, et devant toute l'armée des cieux, qui sont des choses qu'ils ont aimées, qu'ils ont servies et après lesquelles ils ont marché ; des choses qu'ils ont recherchées, et devant lesquelles ils se sont prosternés ; ils ne seront pas recueillis ni ensevelis, ils seront comme du fumier sur la face du sol.
\VS{3}Et la mort sera plus désirable que la vie pour tous ceux qui resteront de cette race mauvaise, ceux, dis-je, qui seront restés dans tous les lieux où je les aurai chassés, dit Yahweh des armées.
\VS{4}Dis-leur donc : Ainsi parle Yahweh : Si on tombe, ne se relève-t-on pas ? Et si on se détourne, ne revient-on pas ?
\VS{5}Pourquoi donc ce peuple de Jérusalem s'abandonne-t-il à de perpétuels égarements ? Ils tiennent ferme à la tromperie, et ils refusent de convertir.
\VS{6}Je suis attentif et j'écoute, mais nul ne parlent selon la justice ; il n'y a personne qui se repente de sa méchanceté, disant : Qu'ai-je fait ? Ils retournent tous vers les objets qui les entraînent, comme le cheval qui se jette avec impétuosité parmi la bataille.
\VS{7}Même la cigogne connaît dans les cieux ses saisons ; la tourterelle et l'hirondelle, et la grue observent le temps où elles doivent venir ; mais mon peuple ne connaît pas les ordonnances de Yahweh.
\VS{8}Comment dites-vous : Nous sommes les sages, et la loi de Yahweh est avec nous ? Voilà, certes on a agi faussement, et la plume des scribes est une plume de fausseté.
\VS{9}Les sages sont confus, ils sont épouvantés et pris ; car ils ont rejeté la parole de Yahweh, et quelle sagesse ont-ils ?
\VS{10}C'est pourquoi je donnerai leurs femmes à d'autres, et leurs champs à des gens qui les posséderont en héritage. Car depuis le plus petit jusqu'au plus grand, chacun s'adonne au gain déshonnête, tant le prophète que le sacrificateur, tous agissent faussement.
\VS{11}Ils pansent à la légère la plaie de la fille de mon peuple, en disant : Paix ! Paix ! Et il n'y a pas de paix.
\VS{12}Sont-ils confus d'avoir commis des abominations ? Ils n'en ont même aucune honte, et ils ne savent pas ce que c'est que de rougir ; c'est pourquoi ils tomberont parmi ceux qui tombent, ils seront renversés au temps où je les visiterai, dit Yahweh.
\VS{13}Je les ramasserai, j'en finirai avec eux, dit Yahweh ; il n'y aura plus de raisins à la vigne, et il n'y aura plus de figues au figuier, les feuilles se flétriront ; et ce que je leur avais donné sera transporté avec eux.
\VS{14}Pourquoi restons-nous assis ? Assemblez-vous et entrons dans les villes fortes, et nous serons là en repos ! Car Yahweh, notre Dieu, nous réduit au silence, et il nous fait boire des eaux empoisonnées, parce que nous avons péché contre Yahweh.
\VS{15}On attendait la paix, et il n'y a rien de bon ; on attend le temps de guérison, et voici la terreur !
\VS{16}Le hennissement de ses chevaux se fait entendre de Dan, et tout le pays tremble au bruit des hennissements de ses puissants chevaux ; ils viennent et dévorent le pays et ce qu'il contient, la ville et ceux qui l'habitent.
\VS{17}Qui plus est, voici, j'envoie contre vous des serpents, des basilics, contre lesquels il n'y a pas d'enchantement, et ils vous mordront, dit Yahweh.
\VS{18}J'ai voulu prendre des forces pour soutenir la douleur, mais mon cœur est languissant au dedans de moi.
\VS{19}Voici la voix du cri de la fille de mon peuple, qui crie d'un pays éloigné : Yahweh n'est-il plus à Sion ? Son Roi n'est-il plus au milieu d'elle ? Pourquoi m'ont-ils irrité par leurs images taillées, par les vanités\FTNT{Idoles que Dieu appelle vanité, vapeur ou souffle} (2) étrangères ?
\VS{20}La moisson est passée, l'été est fini, et nous ne sommes pas sauvés !
\VS{21}Je suis brisé par la blessure de la fille de mon peuple, je suis sombre, l'épouvante me saisit.
\VS{22}N'y a-t-il pas de baume en Galaad ? N'y a-t-il pas là de médecin ? Pourquoi donc la guérison de la fille de mon peuple ne s'opère-t-elle pas ?
\Chap{9}
\TextTitle{Jérémie pleure sur son peuple}
\VerseOne{}Plaise à Dieu que ma tête soit comme un réservoir d'eau, et que mes yeux soient une vive fontaine de larmes, et je pleurerais jour et nuit les blessés à mort de la fille de mon peuple !
\VS{2}Plaise à Dieu que j'aie au désert une cabane de voyageurs, j'abandonnerais mon peuple, je m'en irais loin de lui ! Car ils sont tous des adultères, et une assemblée de perfides.
\VS{3}Ils ont tendu leur langue, qui a été comme leur arc pour décrocher le mensonge\FTNT{Ps. 64:3-4.} ; et ils se sont renforcés dans la terre contre la fidélité ; car ils sont allés de méchanceté en méchanceté, et ne m'ont pas reconnu, dit Yahweh.
\VS{4}Gardez-vous chacun de son intime ami, et ne vous confiez en aucun frère\FTNT{Mi. 7:5.} ; car tout frère fait métier de supplanter, et tout intime ami marche dans la calomnie.
\VS{5}Et chacun se moque de son intime ami, et on ne parle pas selon la vérité ; ils ont instruit leur langue à dire le mensonge, ils se tourmentent extrêmement pour faire le mal.
\VS{6}Ta demeure est au milieu de la tromperie ; ils refusent, à cause de la tromperie, de me connaître, dit Yahweh.
\VS{7}C'est pourquoi, ainsi parle Yahweh des armées : Voici, je vais les fondre, je les éprouverai\FTNT{Mal. 3:3.}. Car comment en agirais-je autrement à l'égard de la fille de mon peuple ?
\VS{8}Leur langue est une flèche meurtrière, elle profère des tromperies ; chacun de sa bouche parle de la paix avec son ami, mais au-dedans il lui dresse des embûches\FTNT{Ps. 12:3; Ps. 28:3.}.
\VS{9}Ne les punirais-je pas pour ces choses-là, dit Yahweh ? Mon âme ne se vengerait-elle pas d'une telle nation ?
\VS{10}J'élèverai ma voix avec larmes, et je prononcerai à haute voix une lamentation à cause des montagnes, et une complainte à cause des cabanes du désert, parce qu'elles sont brûlées, de sorte que personne n'y passe et qu'on n'y entend plus la voix des troupeaux ; les oiseaux des cieux et le bétail ont fui, ils s'en sont allés.
\VS{11}Et je ferai de Jérusalem des monceaux de ruines, elle sera un repaire de serpents, et je ferai des villes de Juda un désert sans habitants.
\VS{12}Qui est l'homme sage qui comprenne ceci ? Qui est celui à qui la bouche de Yahweh a parlé ? Qu'il le déclare et qu'il dise pourquoi le pays est-il détruit, brûlé comme un désert, sans que personne y passe ?
\VS{13}Yahweh donc dit : Parce qu'ils ont abandonné ma loi que j'avais mise devant eux ; parce qu'ils n'ont pas écouté ma voix, et qu'ils n'ont pas marché selon elle ;
\VS{14}mais parce qu'ils ont marché suivant les penchants de leur cœur, et après les Baals, comme leurs pères le leur ont enseigné.
\VS{15}C'est pourquoi, ainsi parle Yahweh des armées, le Dieu d'Israël : Voici, je vais faire manger de l'absinthe à ce peuple-ci, et je leur ferai boire des eaux empoisonnées.
\VS{16}Je les disperserai parmi les nations que n'ont connues ni eux ni leurs pères, et j'enverrai après eux l'épée, jusqu'à ce que je les aie exterminés.
\VS{17}Ainsi parle Yahweh des armées : Considérez, et appelez des pleureuses, afin qu'elles viennent, et mandez les femmes sages, et qu'elles viennent !
\VS{18}Qu'elles se hâtent, et qu'elles prononcent à haute voix une lamentation sur nous ! Et que nos larmes tombent de nos yeux et que l'eau coule de nos paupières !
\VS{19}Car une voix de lamentation se fait entendre de Sion, disant : Eh quoi ! Nous sommes dévastés ! Nous sommes couverts de honte ! Car nous avons abandonné le pays, car nos demeures nous ont jetés dehors !
\VS{20}C'est pourquoi, vous, femmes, écoutez la parole de Yahweh, et que votre oreille reçoive la parole de sa bouche ! Enseignez vos filles à se lamenter, et chacune sa compagne à faire des complaintes !
\VS{21}Car la mort est montée par nos fenêtres, elle est entrée dans nos palais, pour exterminer les enfants dans les rues, et les jeunes hommes dans les places.
\VS{22}Dis : Ainsi parle Yahweh : Même les cadavres des hommes tomberont comme du fumier sur le dessus des champs, et comme une gerbe après le moissonneur, sans que personne les ramasse !
\VS{23}Ainsi parle Yahweh : Que le sage ne se glorifie pas de sa sagesse, que le fort ne se glorifie pas de sa force, et que le riche ne se glorifie pas de sa richesse.
\VS{24}Mais que celui qui se glorifie, se glorifie d'avoir de l'intelligence et de me connaître, car je suis Yahweh, qui fais miséricorde, droit et justice sur la terre ; car je prends plaisir en ces choses-là, dit Yahweh\FTNT{Ps. 62:10 ; 1 Co. 1:31 ; 2 Co. 10:17 ; 1 Ti. 6:17.}.
\VS{25}Voici, les jours viennent, dit Yahweh, où je punirai tout circoncis incirconcis,
\VS{26}l'Egypte, Juda, Edom, les fils d'Ammon, Moab, et tous ceux qui se coupent les coins de leur barbe et qui habitent dans le désert ; car toutes les nations sont incirconcises, et toute la maison d'Israël a le cœur incirconcis.
\Chap{10}
\TextTitle{Dénonciation de l'idolâtrie en Israël}
\VerseOne{}Ecoutez la parole que Yahweh vous adresse, maison d'Israël !
\VS{2}Ainsi parle Yahweh : N'apprenez pas les façons de faire des nations\FTNT{Lé. 18:3 ; De. 12:30.}, et ne craignez pas les signes des cieux, parce que les nations les craignent.
\VS{3}Car les lois des peuples ne sont que vanité\FTNT{Les lois des peuples, ou encore statuts, coutumes, ordonnances ne sont que vanité. Nous devons nous soumettre aux lois des nations tant que celles-ci ne s'opposent pas à la Loi de Dieu (1 Pi. 2 :13). Quand celles-ci sont contraires aux règles morales établies par le Seigneur, nous devons obéir à Dieu, car il vaut mieux obéir à Dieu plutôt qu'aux hommes (Ac. 4:19 ; Ac. 5:29).}. On coupe le bois dans la forêt ; la main de l'ouvrier le travaille avec la hache\FTNT{Es. 40 :20 ; Es. 44 :12-18.} ;
\VS{4}on l'embellit avec de l'argent et de l'or, on le fait tenir avec des clous et à coups de marteau, afin qu'il ne vacille pas.
\VS{5}Ils sont façonnés tout droits comme des colonnes massives, et ils ne parlent pas ; on les porte par nécessité, parce qu'ils ne peuvent pas marcher. Ne les craignez pas, car ils ne sauraient faire aucun mal, et aussi ils sont incapables de faire du bien.
\VS{6}Nul n'est semblable à toi, ô Yahweh ! Tu es grand, et ton Nom est grand par ta puissance.
\VS{7}Qui ne te craindrait, Roi des nations ? Car cela t'est dû ; car, parmi tous les sages des nations et dans tous leurs royaumes, nul n'est semblable à toi\FTNT{Ap. 15:4.}.
\VS{8}Et ils sont tous ensemble stupides et insensés ; le bois ne leur enseigne que des vanités\FTNT{Ha. 2:18.}.
\VS{9}L'argent qui est étendu en plaques est apporté de Tarsis, et l'or d'Uphaz, pour être mis en œuvre par l'ouvrier et par les mains du fondeur ; et la pourpre et l'écarlate sont leur vêtement ; toutes ces choses sont l'ouvrage de gens habiles.
\VS{10}Mais Yahweh est le Dieu de vérité, c'est le Dieu vivant et le Roi éternel ; la terre tremble devant sa colère, et les nations ne supportent pas sa fureur.
\VS{11}Vous leur parlerez ainsi : Les dieux qui n'ont pas fait les cieux et la terre périront de la terre et de dessous les cieux.
\VS{12}Mais Yahweh est celui qui a fait la terre par sa puissance, qui a fondé le monde habitable par sa sagesse, et qui a étendu les cieux par son intelligence.
\VS{13}Sitôt qu'il fait retentir sa voix, il y a un tumulte d'eaux dans les cieux ; il fait monter les vapeurs des extrémités de la terre, il fait les éclairs et la pluie, et il fait sortir le vent de ses réservoirs.
\VS{14}Tout homme devient stupide par sa connaissance, tout fondeur est honteux par les images taillées ; car les idoles en métal fondu ne sont que mensonge, il n'y a pas de souffle en elles ;
\VS{15}elles ne sont que vanité, une œuvre de tromperie ; elles périront au temps de leur châtiment.
\VS{16}La portion de Jacob n'est pas comme ces choses-là ; car c'est lui qui a tout formé, et Israël est la tribu de son héritage. Son Nom est Yahweh des armées.
\VS{17}Toi qui es assise dans la détresse, rassemble du pays tes paquets !
\VS{18}Car ainsi parle Yahweh : Voici, cette fois je vais lancer au loin, comme avec une fronde, les habitants du pays ; je vais les mettre à l'étroit, afin qu'on les atteigne.
\VS{19}Malheur à moi, diront-ils, à cause de ma blessure ! Ma plaie est douloureuse ! Mais moi, je dis : Quoi qu'il en soit, c'est une maladie qu'il faut que je supporte.
\VS{20}Ma tente est dévastée, tous mes cordages sont rompus ; mes fils m'ont quittée, et ils ne sont plus ; il n'y a plus personne qui dresse ma tente, qui relève mes pavillons.
\VS{21}Car les pasteurs ont été stupides, ils n'ont pas cherché Yahweh ; c'est pour cela qu'ils n'ont pas réussi et que tous leurs troupeaux s'éparpillent.
\VS{22}Voici, une rumeur se fait entendre ; avec une grande secousse qui vient du pays du nord, pour faire des villes de Juda un désert, un repaire de serpents.
\VS{23}Yahweh ! Je sais que la voie de l'homme ne dépend pas de lui\FTNT{Pr. 16:1.}, et qu'il n'est pas au pouvoir de l'homme qui marche de diriger ses pas.
\VS{24}Ô Yahweh ! Châtie-moi, mais avec équité, et non dans ta colère, de peur que tu ne me réduises à rien\FTNT{Es. 27:8 ; Ps. 38:2.}.
\VS{25}Répands ta fureur sur les nations qui ne te connaissent pas, et sur les familles qui n'invoquent pas ton Nom ! Car ils ont dévoré Jacob, ils l'ont, dis-je, dévoré et consumé, et ils ont mis en désolation son agréable demeure.
\Chap{11}
\TextTitle{Yahweh dénonce la prostitution de Juda}
\VerseOne{}La parole fut adressée à Jérémie de la part de Yahweh, en disant :
\VS{2}Ecoutez les paroles de cette alliance, et parlez aux hommes de Juda et aux habitants de Jérusalem !
\VS{3}Dis-leur : Ainsi parle Yahweh, le Dieu d'Israël : Maudit soit l'homme qui n'écoute pas les paroles de cette alliance\FTNT{De. 27:26 ; Ga. 3:10.},
\VS{4}que j'ai ordonnée à vos pères, le jour où je les ai fait sortir du pays d'Egypte, de la fournaise de fer, en disant : Ecoutez ma voix et faites toutes les choses que je vous ordonnerai ; alors vous serez mon peuple, et je serai votre Dieu\FTNT{Lé. 26:12 ; De. 4:20.},
\VS{5}afin que j'accomplisse le serment que j'ai juré à vos pères, de leur donner un pays où coulent le lait et le miel, comme vous le voyez aujourd'hui. Et je répondis et dis : Amen ! Ô Yahweh !
\VS{6}Puis Yahweh me dit : Crie toutes ces paroles dans les villes de Juda et dans les rues de Jérusalem, en disant : Ecoutez les paroles de cette alliance et observez-les !
\VS{7}Car j'ai averti vos pères, depuis le jour où je les ai fait monter du pays d'Egypte jusqu'à ce jour, je les ai avertis dès le matin, en disant : Ecoutez ma voix !
\VS{8}Mais ils n'ont pas écouté, ils n'ont pas prêté l'oreille, ils ont marché chacun suivant les penchants de leur mauvais cœur ; c'est pourquoi j'ai fait venir sur eux toutes les paroles de cette alliance, que je leur avais donné l'ordre d'observer, et qu'ils n'ont pas observée.
\VS{9}Yahweh me dit : Il y a une conspiration entre les hommes de Juda et entre les habitants de Jérusalem.
\VS{10}Ils sont retournés aux iniquités de leurs premiers pères, qui ont refusé d'écouter mes paroles, et ils sont allés après d'autres dieux pour les servir. La maison d'Israël et la maison de Juda ont rompu mon alliance, que j'avais faite avec leurs pères.
\VS{11}C'est pourquoi ainsi parle Yahweh : Voici, je fais venir sur eux un mal dont ils ne pourront sortir. Ils crieront vers moi, et je ne les écouterai pas\FTNT{Es. 1:15 ; Ez. 8:18 ; Mi. 3:4 ; Pr. 1:28.}.
\VS{12}Et les villes de Juda et les habitants de Jérusalem s'en iront et crieront vers les dieux auxquels ils brûlent de l'encens, mais ces dieux-là ne les sauveront pas au temps de leur malheur.
\VS{13}Car, ô Juda ! Tu as eu autant de dieux que de villes ; et toi, Jérusalem, tu as dressé autant d'autels aux choses honteuses que tu as de rues, des autels, dis-je, pour brûler de l'encens à Baal\FTNT{Ez. 16:24-31 ; Ac. 17:23.}…
\VS{14}Toi donc, n'intercède pas pour ce peuple, et n'élève pour eux ni cri ni prière ; car je ne les écouterai pas au temps où ils crieront vers moi dans leur malheur.
\VS{15}Qu'est-ce que mon bien-aimé a à faire dans ma maison, que tant de gens se servent d'elle pour y faire leurs complots? la chair sainte est transportée loin de toi, et encore quand tu fais le mal, c'est alors que tu triomphes !
\VS{16}Yahweh avait appelé ton nom Olivier verdoyant et beau par la forme de ton fruit ; mais au bruit d'un grand fracas, il y a mis le feu, et ses rameaux sont brisés.
\VS{17}Yahweh des armées, qui t'a plantée, prononce le mal contre toi, à cause de la méchanceté de la maison d'Israël et de la maison de Juda, qui ont agi pour m'irriter, en brûlant de l'encens à Baal.
\TextTitle{Jugement des ennemis de Jérémie}
\VS{18}Et Yahweh me l'a fait savoir, et je l'ai su ; alors tu m'as fait voir leurs actions.
\VS{19}Mais moi, comme un agneau, ou comme un bœuf qu'on mène pour être égorgé, je ne savais pas qu'ils projetaient de mauvais desseins contre moi, en disant : Détruisons l'arbre avec son fruit ! Exterminons-le de la terre des vivants, et qu'on ne se souvienne plus de son nom !
\VS{20}Mais toi, Yahweh des armées, qui juges justement, et qui éprouve les reins et le cœur ! Fais que je voie ta vengeance s'exercer contre eux, car je t'ai découvert ma cause\FTNT{1 S. 16:7 ; Ps. 26:2 ; 1 Ch. 28:9 ; Ap. 2:23.}.
\VS{21}C'est pourquoi ainsi parle Yahweh contre les gens d'Anathoth, qui cherchent ta vie et qui disent : Ne prophétise plus au Nom de Yahweh, et tu ne mourras pas par nos mains\FTNT{Es. 30:10 ; Mi. 2:6.} !
\VS{22}C'est pourquoi donc ainsi parle Yahweh des armées : Voici, je vais les punir ; les jeunes hommes mourront par l'épée, leurs fils et leurs filles mourront par la famine.
\VS{23}Et il ne restera rien d'eux ; car je ferai venir le mal sur les gens d'Anathoth, l'année de leur châtiment.
\Chap{12}
\TextTitle{Prière de Jérémie et réponse de Yahweh}
\VerseOne{}Yahweh, quand je contesterai avec toi, tu seras trouvé juste ; mais toutefois j'entrerai en contestation avec toi : Pourquoi la voie des méchants est-elle prospère ? Pourquoi tous les perfides vivent-ils en paix\FTNT{Job. 21:7-9 ; Ro. 3:4.} ?
\VS{2}Tu les as plantés, et ils ont pris racine, ils s'avancent, et ils portent du fruit. Tu es près de leur bouche, mais tu es loin de leurs cœurs\FTNT{Es. 29:13 ; Job. 21:7-8.}.
\VS{3}Mais, ô Yahweh, tu me connais, tu me vois, tu éprouves mon cœur qui est avec toi. Traîne-les comme des brebis qu'on mène pour être égorgées, et mets-les à part pour le jour de la tuerie !
\VS{4}Jusqu'à quand le pays mènera-t-il deuil, et l'herbe de tous les champs séchera-t-elle à cause de la méchanceté des habitants qui sont en la terre ? Les bêtes et les oiseaux ont été consumés par la disette, parce que ces méchants ont dit : On ne verra pas notre dernière fin. 
\VS{5}Si tu cours avec des piétons et qu'ils te fatiguent, comment lutteras-tu avec les chevaux ? Et si tu te crois en sûreté dans une terre de paix, que feras-tu devant l'orgueil du Jourdain ?
\VS{6} Certainement, mêmes tes frères et la maison de ton père, ceux-là mêmes ont agi perfidement contre toi, eux-mêmes ont crié après toi à plein gosier ; ne les crois point, quoiqu'ils te parlent amicalement\FTNT{Pr. 26:25.}.
\VS{7}J'ai abandonné ma maison, j'ai quitté mon héritage, ce que mon âme aimait le plus je l'ai livré aux mains de ses ennemis.
\VS{8}Mon héritage a été pour moi comme un lion dans la forêt, il a poussé contre moi ses rugissements ; c'est pourquoi je l'ai pris en haine.
\VS{9}Mon héritage a-t-il donc été pour moi comme un oiseau de proie tacheté ? Les oiseaux de proie ne sont-ils pas autour de lui ? Venez, assemblez-vous, vous tous les animaux des champs, venez pour le dévorer\FTNT{Es. 56:9.} !
\VS{10}Plusieurs pasteurs ravagent ma vigne, ils foulent mon champ ; ils réduisent le champ de mes délices en un désert, en une désolation.
\VS{11}Ils le réduisent en un désert ; il est en deuil, il est désolé devant moi. Tout le pays est ravagé, car nul n'y prend garde.
\VS{12}Les destructeurs viennent sur tous les lieux élevés du désert, car l'épée de Yahweh dévore le pays d'un bout à l'autre ; il n'y a de paix pour aucune chair.
\VS{13}Ils ont semé du froment, et ils moissonnent des épines, ils se sont fatigués sans profit. Soyez honteux de vos récoltes, à cause de l'ardeur de la colère de Yahweh\FTNT{Lé. 26:16.}.
\VS{14}Ainsi parle Yahweh contre tous mes mauvais voisins, qui mettent la main sur l'héritage que j'ai donné à mon peuple d'Israël : Voici, je les arracherai de leur pays, et j'arracherai la maison de Juda du milieu d'eux.
\VS{15}Mais il arrivera qu'après que je les avoir arrachés, j'aurai encore compassion d'eux, et je les ramènerai chacun dans son héritage, chacun dans son pays\FTNT{De. 30:3.}.
\VS{16}Et il arrivera que s'ils apprennent bien les voies de mon peuple, pour jurer par mon Nom, en disant : Yahweh est vivant ! Comme ils ont enseigné à mon peuple à jurer par Baal, ils seront édifiés au milieu de mon peuple.
\VS{17}Mais s'ils n'écoutent pas, j'arracherai entièrement une telle nation, et je la ferai périr, dit Yahweh\FTNT{Es. 60:12.}.
\Chap{13}
\TextTitle{La ceinture pourrie, illustration du jugement}
\VerseOne{}Ainsi m'a parlé Yahweh : Va, et achète-toi une ceinture de lin et mets-la sur tes reins ; et ne la mets pas dans l'eau.
\VS{2}J'achetai donc une ceinture, selon la parole de Yahweh, et je la mis sur mes reins.
\VS{3}Et la parole de Yahweh me fut adressée pour la seconde fois, en disant :
\VS{4}Prends la ceinture que tu as achetée et qui est sur tes reins ; lève-toi, va-t'en vers l'Euphrate, et là, cache-la dans la fente d'un rocher.
\VS{5}J'allai donc et je la cachai près de l'Euphrate, comme Yahweh me l'avait ordonné.
\VS{6}Et il arriva que plusieurs jours après Yahweh me dit : Lève-toi, va vers l'Euphrate et reprends la ceinture que je t'avais ordonné d'y cacher.
\VS{7}Et j'allai vers l'Euphrate, je creusai, et je pris la ceinture dans le lieu où je l'avais cachée ; mais voici, la ceinture était pourrie, elle n'était plus bonne à rien.
\VS{8}Alors la parole de Yahweh me fut adressée, en disant :
\VS{9}Ainsi parle Yahweh : Je ferai ainsi pourrir l'orgueil de Juda et le grand orgueil de Jérusalem.
\VS{10}L'orgueil de ce peuple très méchant, qui refuse d'écouter mes paroles, qui marche selon les penchants de son cœur, et qui va après d'autres dieux, pour les servir et pour se prosterner devant eux, qu'il devienne comme cette ceinture qui n'est plus bonne à rien !
\VS{11}Car comme une ceinture est attachée aux reins d'un homme, ainsi je m'étais attaché toute la maison d'Israël et toute la maison de Juda, dit Yahweh, afin qu'elles soient mon peuple, mon Nom, ma louange, et ma gloire. Mais ils ne m'ont pas écouté.
\VS{12}Tu leur diras donc cette parole-ci : Ainsi parle Yahweh, le Dieu d'Israël : Toute outre sera remplie de vin. Et ils te diront : Ne savons-nous pas que toute outre sera remplie de vin ?
\VS{13}Mais tu leur diras : Ainsi parle Yahweh : Voici, je vais remplir d'ivresse tous les habitants de ce pays, les rois qui sont assis sur le trône de David, les sacrificateurs, les prophètes, et tous les habitants de Jérusalem.
\VS{14}Et je les briserai les uns contre les autres, les pères et les fils ensemble, dit Yahweh\FTNT{Es. 51:17-20 ; Ps. 60:5.} ; je n'aurai pas de compassion, je n'épargnerai pas, et je n'aurai pas de miséricorde ; rien ne m'empêchera de les détruire.
\VS{15}Écoutez et prêtez l'oreille ! Ne vous élevez pas ! Car Yahweh parle.
\VS{16}Donnez gloire à Yahweh, votre Dieu, avant qu'il fasse venir les ténèbres, avant que vos pieds se heurtent contre les montagnes du crépuscule ; vous attendrez la lumière, et il la changera en ombre de la mort, il la réduira en obscurité profonde\FTNT{Es. 59:9 ; Jn. 12:35.}.
\VS{17}Que si vous n'écoutez pas ceci, mon âme pleurera en secret, à cause de votre orgueil ; mes yeux verseront des larmes en abondance, ils se fondront en larmes, parce que le troupeau de Yahweh sera emmené captif\FTNT{La. 1:2-16.}.
\VS{18}Dis au roi et à la reine : Humiliez-vous et asseyez-vous sur la cendre ! Car elle est tombée de vos têtes, la couronne de votre gloire.
\VS{19}Les villes du midi sont fermées, il n'y a personne qui les ouvre ; tout Juda est transporté en captivité, il est transporté entièrement.
\VS{20}Levez vos yeux et voyez ceux qui viennent du nord. Où est le troupeau qui t'avait été donné, le troupeau qui faisait ta gloire ?
\VS{21}Que diras-tu quand il te punit ? Car tu les as enseignés à dominer en maîtres sur toi. Les douleurs ne te saisiront-elles pas, comme elles saisissent une femme qui enfante ?
\VS{22}Que si tu dis en ton cœur : Pourquoi cela m'arrive-t-il ? C'est à cause de la multitude de tes iniquités que les pans de ta robe sont relevés, et que tes talons sont violemment mis à nu\FTNT{Es. 47:2-3.}.
\VS{23}L'éthiopien peut-il changer sa peau et le léopard ses taches ? Pourriez-vous, aussi, faire quelque bien, vous qui êtes accoutumés à faire le mal ?
\VS{24}C'est pourquoi je les disperserai, comme du chaume, qui est emporté çà et là par le vent du désert.
\VS{25}Voilà ton sort, la portion que je te mesure, dit Yahweh, parce que tu m'as oublié, et que tu as mis ta confiance dans le mensonge.
\VS{26}A cause de cela, je relèverai les pans de ta robe sur ton visage, et ta honte se verra.
\VS{27}Tes adultères et tes hennissements, l'énormité de tes prostitutions sur les collines et dans les champs, tes abominations, je les ai vues. Malheur à toi, Jérusalem ! Ne seras-tu pas purifiée ? Jusqu'à quand cela durera-t-il ?
\Chap{14}
\TextTitle{Le pays frappé par la sécheresse}
\VerseOne{}La parole de Yahweh, qui fut adressée à Jérémie, à l'occasion de la sécheresse.
\VS{2}Juda est dans le deuil, et ses portes sont dans un état pitoyable. Ils sont tous en deuil, gisant par terre ; et les cris de Jérusalem montent au ciel.
\VS{3}Et les personnes distinguées envoient les petits chercher de l'eau, et les petits vont aux citernes, ne trouvent pas d'eau, et reviennent leurs vases vides ; ils sont honteux et confus, ils couvrent leur tête.
\VS{4}Parce que la terre est crevassée, parce qu'il n'y a pas eu de pluie dans le pays, les laboureurs sont honteux, ils se couvrent la tête.
\VS{5}Même la biche met bas son faon dans le champ et l'abandonne, parce qu'il n'y a pas d'herbe.
\VS{6}Et les ânes sauvages se tiennent sur les lieux élevés, humant l'air comme des serpents ; leurs yeux se consument, parce qu'il n'y a pas d'herbe.
\VS{7}Si nos iniquités témoignent contre nous, agis à cause de ton Nom, ô Yahweh\FTNT{Es. 59:12.} ! Car nos infidélités sont nombreuses, c'est contre toi que nous avons péché.
\VS{8}Toi qui es l'espérance d'Israël, son sauveur au temps de la détresse, pourquoi serais-tu dans le pays comme un étranger, comme un voyageur qui se détourne pour passer la nuit ?
\VS{9}Pourquoi serais-tu comme un homme stupéfait, et comme un héros qui ne peut sauver ? Or tu es au milieu de nous, ô Yahweh, et ton Nom est invoqué sur nous : Ne nous abandonne pas !
\VS{10}Voici ce que Yahweh dit de ce peuple : Parce qu'ils aiment à errer ainsi çà et là, et qu'ils ne savent retenir leurs pieds, Yahweh ne prend pas plaisir en eux, il se souvient maintenant de leurs iniquités, et il punit leurs péchés\FTNT{Os. 8:13.}.
\VS{11}Puis Yahweh me dit : N'intercède pas en faveur de ce peuple.
\VS{12}Quand ils jeûnent, je n'écouterai pas leurs cris ; et quand ils offrent des holocaustes et des offrandes, je n'y prendrai pas plaisir ; mais je les consumerai par l'épée, par la famine et par la peste.
\VS{13}Et je répondis : Ah ! ah ! Seigneur Yahweh ! Voici, les prophètes leur disent : Vous ne verrez pas l'épée, et vous n'aurez pas de famine ; mais je vous donnerai dans ce lieu-ci une paix assurée.
\VS{14}Et Yahweh me dit : C'est le mensonge ce que ces prophètes prophétisent en mon Nom ; je ne les ai pas envoyés, je ne leur ai pas donné d'ordre, je ne leur ai pas parlé ; ils vous prophétisent des visions de mensonge, des divinations, de l'idolâtrie et des tromperies de leur cœur\FTNT{De. 18:20-22 ; Ez. 13:2-3.}.
\VS{15}C'est pourquoi ainsi parle Yahweh sur les prophètes qui prophétisent en mon Nom, sans que je les ai envoyés, et qui disent : Il n'y aura ni épée ni la famine dans ce pays : Ces prophètes-là seront consumés par l'épée et par la famine.
\VS{16}Et le peuple à qui ils prophétisent sera jeté dans les rues de Jérusalem à cause de la famine et de l'épée ; et il n'y aura personne pour les enterrer, ni eux, ni leurs femmes, ni leurs fils, ni leurs filles ; je répandrai sur eux leur méchanceté.
\VS{17}Tu leur diras donc cette parole-ci : Que mes yeux se fondent en larmes nuit et jour, et qu'ils ne cessent pas\FTNT{La. 1:16.} ; car la vierge, fille de mon peuple, a été frappée d'un grand coup, d'une plaie très douloureuse.
\VS{18}Si je sors dans les champs, voici les gens tués par l'épée ; si j'entre dans la ville, voici les gens consumés par la faim ; même le prophète et le sacrificateur parcourent le pays, sans savoir où ils vont.
\VS{19}As-tu entièrement rejeté Juda, et ton âme a-t-elle Sion en horreur ? Pourquoi nous frappes-tu sans qu'il y ait pour nous de guérison ? On attend la paix, mais il n'y a rien de bon, un temps de guérison, et voici la terreur !
\VS{20}Yahweh, nous reconnaissons notre méchanceté, l'iniquité de nos pères ; car nous avons péché contre toi\FTNT{Ps. 106:6 ; Da. 9:8.}.
\VS{21}Ne nous rejette pas, à cause de ton Nom, et ne déshonore pas le trône de ta gloire ! Souviens-toi de ton alliance avec nous, et ne la romps pas !
\VS{22}Parmi les vanités\FTNT{Ce terme veut aussi dire « idole ».} des nations, y en a-t-il qui fassent pleuvoir, et les cieux donnent-ils des ondées\FTNT{Es. 30:23 ; Ac. 14:17.} ? N'est-ce pas toi, ô Yahweh, notre Dieu ? C'est pourquoi nous nous attendons à toi, car c'est toi qui as fait toutes ces choses.
\Chap{15}
\TextTitle{Yahweh fermement décidé à juger son peuple}
\VerseOne{}Et Yahweh me dit : Quand Moïse et Samuel se tiendraient devant moi, je n'aurais pourtant point d'affection pour ce peuple ; chasse-les de devant ma face, et qu'ils sortent.
\VS{2}Que s'ils te disent : Où irons-nous ? Tu leur répondras : Ainsi parle Yahweh : Ceux qui sont destinés à la mort iront à la mort ; et ceux qui sont destinés à l’épée iront à l’épée ; et ceux qui sont destinés à la famine, iront à la famine ; et ceux qui sont destinés à la captivité iront en captivité\FTNT{Za. 11:9.} !
\VS{3}J'établirai aussi sur eux quatre espèces de punitions, dit Yahweh, l'épée pour tuer, et les chiens pour traîner, et les oiseaux des cieux, et les bêtes de la terre pour dévorer et pour détruire. 
\VS{4}Et je les livrerai à être agités par tous les Royaumes de la terre, à cause de Manassé, fils d'Ezéchias, Roi de Juda, pour les choses qu'il a faites dans Jérusalem. 
\VS{5}Car qui aurait compassion de toi, Jérusalem, ou qui te plaindrait ? Ou qui se détournerait pour s'informer de ta paix ? 
\VS{6}Tu m'as abandonné, dit Yahweh, et tu t'en es allée en arrière ; c'est pourquoi j'étends ma main sur toi, et je te détruis, je suis las d'avoir compassion.
\VS{7}Je les vanne avec un van aux portes du pays\FTNT{Mt. 3:12.} ; je prive d'enfants, je fais périr mon peuple, et ils ne se sont pas détournés de leurs voies.
\VS{8}Je multiplie ses veuves plus que le sable de la mer ; je fais venir sur eux, sur la mère du jeune homme, le dévastateur en plein midi ; je fais tomber subitement sur elle l'angoisse et les frayeurs.
\VS{9}Celle qui en avait enfanté sept languit, elle rend l'âme ; son soleil se couche pendant qu'il est encore jour\FTNT{Am. 8:9.} ; elle est confuse, couverte de honte. Ceux qui restent, je les livre à l'épée devant leurs ennemis, dit Yahweh.
\VS{10}Malheur à moi, ô ma mère, de ce que tu m'as enfanté\FTNT{Job. 3:1-2.} pour être un homme de contestation et un homme de dispute pour tout le pays ! Je n'emprunte ni ne prête, et néanmoins tous me maudissent et me méprisent.
\VS{11}Alors Yahweh dit : En vérité tout ira bien pour ton reste ; en vérité je ferai que l’ennemi te traite bien au temps du malheur, et au temps de la détresse.
\VS{12}Le fer brisera-t-il le fer du nord et l'airain ?
\VS{13}Je livre au pillage, sans en faire le prix, tes richesses et tes trésors, et cela à cause de tous tes péchés, sur tout ton territoire.
\VS{14}Je te fais passer avec tes ennemis dans un pays que tu ne connais pas, car le feu de ma colère s'est allumé, il brûle sur vous\FTNT{De. 32:22.}.
\TextTitle{La mise à part de Jérémie}
\VS{15}Yahweh ! Tu sais tout, souviens-toi de moi, visite-moi, venge-moi de ceux qui me persécutent\FTNT{Ps. 106:4.} ! Ne m'enlève pas, tandis que tu te montres lent à la colère ! Sache que je supporte l'opprobre à cause de toi.
\VS{16}J'ai trouvé tes paroles, je les ai aussitôt dévorées\FTNT{Ez. 3:3 ; Ap. 10:9.} ; tes paroles ont fait la joie et l'allégresse de mon cœur ; car ton Nom est invoqué sur moi, ô Yahweh, Dieu des armées !
\VS{17}Je ne me suis pas assis dans l'assemblée des moqueurs, et je ne m'y suis pas réjoui ; mais je me suis assis tout seul à cause de ta main, car tu me remplissais d'indignation.
\VS{18}Pourquoi ma douleur est-elle continuelle ? Pourquoi ma plaie est-elle incurable et refuse-t-elle d'être guérie ? Serais-tu pour moi comme une source trompeuse, comme des eaux qui ne durent pas ?
\VS{19}C'est pourquoi ainsi parle Yahweh : Si tu reviens, je te ramènerai, et tu te tiendras devant moi ; et si tu sépares la chose précieuse de la méprisable, tu seras comme ma bouche. Qu'ils reviennent vers toi, mais toi, ne retourne pas vers eux.
\VS{20}Je ferai que tu sois pour ce peuple une muraille d'airain bien forte ; ils combattront contre toi, mais il n'auront pas le dessus contre toi ; car je suis avec toi pour te sauver et te délivrer, dit Yahweh.
\VS{21}Et je te délivrerai de la main des malins, et te rachèterai de la main des méchants.
\Chap{16}
\TextTitle{Célibat de Jérémie, illustration du jugement sur Juda}
\VerseOne{}Puis la parole de Yahweh me fut adressée, en disant :
\VS{2}Tu ne prendras pas de femme, et tu n'auras pas de fils ni de filles dans ce lieu-ci.
\VS{3}Car ainsi parle Yahweh sur les fils et les filles qui naîtront en ce lieu-ci, sur leurs mères qui les auront enfantés, et sur leurs pères qui les auront engendrés dans ce pays :
\VS{4}Ils mourront de maladie mortelle ; ils ne seront ni pleurés ni enterrés ; ils seront comme du fumier sur la face du sol ; ils seront consumés par l'épée et par la famine ; et leurs cadavres seront la pâture des oiseaux des cieux et des bêtes de la terre.
\VS{5}Car ainsi parle Yahweh : N'entre pas dans une maison de deuil, ne vas pas te lamenter ni te plaindre avec eux ; car j'ai retiré de ce peuple dit Yahweh, ma paix, ma miséricorde et mes compassions.
\VS{6}Et les grands et petits mourront dans ce pays ; ils ne seront pas enterrés ; on ne les pleurera pas, on ne se fera pas d'incision, et on ne se rasera pas pour eux\FTNT{Lé. 19:28 ; De. 14:1 ; Ez. 7:11}.
\VS{7}On ne rompra pas le pain dans le deuil pour consoler quelqu'un au sujet d'un mort, et on ne leur donnera pas à boire de la coupe de consolation pour leur père ou pour leur mère.
\VS{8}Aussi n'entre pas non plus dans une maison de festin pour t'asseoir avec eux, pour manger et pour boire.
\VS{9}Car ainsi parle Yahweh des armées, le Dieu d'Israël : Voici, je vais faire cesser dans ce lieu-ci, devant vos yeux et en vos jours, les cris de joie et les cris d'allégresse, la voix de l'époux et la voix de l'épouse.
\VS{10}Et il arrivera que quand tu annonceras à ce peuple toutes ces paroles-là, ils te diront : Pourquoi Yahweh parle-t-il de tout ce grand mal contre nous ? Quelle est notre iniquité ? Quel est le péché que nous avons commis contre Yahweh, notre Dieu ?
\VS{11}Et tu leur diras : Parce que vos pères m'ont abandonné, dit Yahweh, et sont allés après d'autres dieux, et les ont servis, et se sont prosternés devant eux, et m'ont abandonné et n'ont pas gardé ma loi ; 
\VS{12}et que vous avez fait le mal plus encore que vos pères. Car voici, chacun de vous marche selon les penchants de son mauvais cœur pour ne pas m'écouter.
\VS{13}A cause de cela, je vous jetterai de ce pays dans un pays que vous n'avez pas connu, ni vous ni vos pères ; et là, vous servirez jour et nuit les autres dieux, car je ne vous aurai pas fait grâce\FTNT{De. 28:64-65.}.
\VS{14}Néanmoins voici, les jours viennent, dit Yahweh, qu'on ne dira plus : Yahweh est vivant, lui qui a fait monter les fils d'Israël du pays d'Egypte !
\VS{15}Mais on dira : Yahweh est vivant, lui qui a fait monter les fils d'Israël du pays du nord et de tous les pays où il les avait chassés ; après que je les aurai ramenés dans leur pays, que j'avais donné à leurs pères.
\VS{16}Voici, j'envoie plusieurs pêcheurs, dit Yahweh, et ils les pêcheront ; et ensuite, j'enverrai plusieurs chasseurs, et ils les chasseront de toutes les montagnes et de toutes les collines, et des fentes des rochers.
\VS{17}Car mes yeux sont sur toutes leurs voies, elles ne sont pas cachées devant ma face, et leur iniquité n'est pas couverte devant mes yeux\FTNT{Pr. 5:21 ; Job. 34:21.}.
\VS{18}Mais premièrement je leur rendrai le double de leur iniquité et de leur péché, parce qu'ils ont souillé mon pays par les cadavres de leurs idoles, et parce qu'ils ont rempli mon héritage de leurs abominations.
\VS{19}Yahweh, qui est ma force et ma forteresse, et mon refuge au jour de la détresse ! Les nations viendront à toi des extrémités de la terre, et diront : Certes nos pères ont hérité le mensonge et la vanité, et les choses auxquelles il n'y a pas de profit.
\VS{20}L'homme se fera-t-il lui-même des dieux, qui ne sont pas dieux ?
\VS{21}C'est pourquoi voici, je leur fais connaître, cette fois, je leur fais connaître ma main et ma force ; et ils sauront que mon Nom est Yahweh.
\Chap{17}
\TextTitle{Le caractère sinueux du cœur}
\VerseOne{}Le péché de Juda est écrit avec un burin de fer, et avec une pointe de diamant ; il est gravé sur la table de leur cœur, et sur les cornes de leurs autels.
\VS{2}De sorte que leurs fils se souviennent de leurs autels, et de leurs poteaux d'Asherah, auprès des arbres verts sur les hautes collines.
\VS{3}Ma montagne, je livre par les champs tes richesses et tous tes trésors au pillage ; tes hauts lieux sont pleins de péché sur tout ton territoire.
\VS{4}Et toi, et ceux qui sont avec toi, vous laisserez vacant l'héritage que je t'avais donné ; et je t'asservirai à tes ennemis dans un pays que tu ne connais pas ; car vous avez allumé le feu de ma colère, et il brûlera à toujours.
\VS{5}Ainsi parle Yahweh : Maudit soit l'homme qui se confie dans l'homme, et qui fait de la chair sa force, et dont le cœur se retire de Yahweh !
\VS{6}Car il sera comme la bruyère dans le désert, et il ne voit pas venir le bien ; mais il demeure dans des lieux brûlés du désert, dans une terre salée et inhabitable.
\VS{7}Béni soit l'homme qui se confie en Yahweh, et dont Yahweh est l'espérance !
\VS{8}Il est comme un arbre planté près des eaux\FTNT{Ps. 23.}, et qui étend ses racines le long d'une eau courante ; quand la chaleur vient, il ne s'en aperçoit pas, et sa feuille reste verte ; il n'est pas en peine dans l'année de la sécheresse, et ne cesse de porter du fruit.
\VS{9}Le cœur est rusé et désespérément malin par-dessus tout : Qui peut le connaître\FTNT{Ps. 64:7.} ?
\VS{10}Je suis Yahweh, qui sonde le cœur, et qui éprouve les reins ; même pour rendre à chacun selon sa voie, et selon le fruit de ses actions.
\VS{11}Celui qui acquiert des richesses, sans observer la justice, est une perdrix qui couve ce qu'elle n'a pas pondu ; il les laissera au milieu de ses jours, et à la fin il sera trouvé insensé\FTNT{Ec. 4:8}.
\VS{12}Le lieu de notre sanctuaire est un trône de gloire, un lieu haut élevé dès le commencement.
\VS{13}Yahweh, qui es l'espérance d'Israël ! Tous ceux qui t'abandonnent seront honteux : Ceux qui se détournent de moi seront écrits sur la terre, car ils abandonnent la source des eaux vives, Yahweh\FTNT{Es. 1:28 ; Ps. 73:28.}.
\VS{14}Yahweh, guéris-moi, et je serai guéri ; sauve-moi, et je serai sauvé ; car tu es ma louange.
\VS{15}Voici, ceux-ci me disent : Où est la parole de Yahweh ? Qu'elle vienne présentement\FTNT{Es. 5:19 ; Ez. 12:23 ; 2 Pi. 3:3-4.} !
\VS{16}Mais je ne me suis pas avancé plus qu’un pasteur après toi, je n'ai pas non plus désiré le jour du malheur, tu le sais ; et ce qui est sorti de mes lèvres est présent devant toi.
\VS{17}Ne sois pas pour moi un sujet d'effroi, toi, mon refuge au jour du malheur !
\VS{18}Que ceux qui me persécutent soient honteux, mais que je ne sois pas honteux ; qu'ils soient brisés, mais que je ne sois pas brisé ! Fais venir sur eux le jour du malheur, frappe-les d'une double plaie !
\TextTitle{Message à propos du sabbat}
\VS{19}Ainsi m'a parlé Yahweh : Va, et tiens-toi debout à la porte des fils du peuple, par laquelle les rois de Juda entrent et par laquelle ils sortent, et à toutes les portes de Jérusalem.
\VS{20}Tu leur diras : Ecoutez la parole de Yahweh, rois de Juda, et vous tous homme de Juda, et vous tous habitants de Jérusalem qui entrez par ces portes !
\VS{21}Ainsi parle Yahweh : Prenez garde à vos âmes ; ne portez aucun fardeau le jour du sabbat, et ne les faites pas passer par les portes de Jérusalem\FTNT{Né. 13:19.}.
\VS{22}Ne faites sortir de vos maisons aucun fardeau le jour du sabbat, et ne faites aucune œuvre ; mais sanctifiez le jour du sabbat, comme je l'ai ordonné à vos pères\FTNT{Ex. 20:8 ; Ex. 23:12.}.
\VS{23}Mais ils n'ont pas écouté, ils n'ont pas prêté l'oreille ; ils ont raidi leur cou, pour ne pas écouter et ne pas recevoir d'instruction.
\VS{24}Il arrivera donc, si vous m'écoutez attentivement, dit Yahweh, pour ne faire passer aucun fardeau par les portes de cette ville le jour du sabbat, et si vous sanctifiez le jour du sabbat, en ne faisant aucune œuvre ce jour-là,
\VS{25}que les rois et les chefs, ceux qui sont assis sur le trône de David, montés sur des chars et sur des chevaux, eux et les chefs d'entre eux, les hommes de Juda et les habitants de Jérusalem, entreront par les portes de cette ville, et cette ville sera habitée à toujours.
\VS{26}On viendra aussi des villes de Juda et des environs de Jérusalem, et du pays de Benjamin, et du bas pays, des montagnes et du midi, pour apporter des holocaustes, des sacrifices, des offrandes et de l'encens ; pour apporter aussi des sacrifices de louanges dans la maison de Yahweh.
\VS{27}Mais si vous ne m'écoutez pas pour sanctifier le jour du sabbat, pour ne porter aucun fardeau, et n'en faire entrer aucun par les portes de Jérusalem le jour du sabbat, je mettrai le feu à ses portes, et il consumera les palais de Jérusalem et ne s'éteindra pas\FTNT{2 R. 25:9.}.
\Chap{18}
\TextTitle{La maison du potier ; appel à la repentance et avertissement}
\VerseOne{}Cette parole fut adressée à Jérémie de la part de Yahweh, disant :
\VS{2}Lève-toi et descends dans la maison d'un potier ; et là, je te ferai entendre mes paroles.
\VS{3}Je descendis donc dans la maison d'un potier, et voici, il faisait son ouvrage, assis sur sa selle.
\VS{4}Et le vase qu'il faisait avec l'argile qu'il tenait dans sa main, fut gâté ; et il en fit encore un autre vase, comme il lui sembla bon de le faire.
\VS{5}Alors la parole de Yahweh me fut adressée, en disant :
\VS{6}Maison d'Israël, ne puis-je pas faire de vous comme a fait ce potier ? Dit Yahweh. Voici, comme l'argile est dans la main d'un potier, ainsi vous êtes dans ma main, maison d'Israël !
\VS{7}En un instant je parle contre une nation et contre un royaume, pour arracher, pour démolir, et pour détruire ;
\VS{8}mais si cette nation, contre laquelle j'ai parlé, revient de sa méchanceté, je me repentirai aussi du mal que j'avais pensé de lui faire\FTNT{Jon. 3:6-10.}.
\VS{9}Et si en un instant je parle d'une nation et d'un royaume, pour l'édifier et pour le planter ;
\VS{10} et que cette nation fasse ce qui est mal à mes yeux, en sorte qu'elle n'écoute pas ma voix, je me repentirai aussi du bien que j'avais dit que je lui ferais.
\VS{11}Or donc, parle maintenant aux hommes de Juda et aux habitants de Jérusalem, en disant : Ainsi parle Yahweh : Voici, je projette du mal contre vous, et je forme un dessein contre vous. Détournez-vous donc chacun de votre mauvaise voie, et amendez votre voie et vos actions !
\VS{12}Et ils répondent : Il n'y a plus d'espérance ; c'est pourquoi nous suivrons nos pensées, chacun de nous fera selon les penchants de son mauvais cœur.
\VS{13}C'est pourquoi ainsi parle Yahweh : Demandez maintenant aux nations ! Qui a entendu de telles choses ? La vierge d'Israël a fait une chose très horrible\FTNT{1 Co. 5:1.}.
\VS{14}La neige du Liban abandonnerait-elle le rocher du champ ? Ou les eaux fraîches et ruisselantes qui viennent de loin tariraient-elles ?
\VS{15}Mais mon peuple m'a oublié, et il brûle de l'encens à ce qui n'est que vanité, et qui les a fait chanceler leurs voies, pour les faire retirer des anciens sentiers, afin de marcher dans les sentiers d'un chemin non frayé ;
\VS{16}pour faire venir sur leur pays une désolation et un opprobre perpétuel ; quiconque passera par là, en sera étonné et secouera la tête.
\VS{17}Je les disperserai devant l'ennemi, comme par le vent d'orient ; je leur tournerai le dos, et non pas la face, au jour de leur calamité\FTNT{Es. 27:8.}.
\VS{18}Et ils ont dit : Venez, et faisons des complots contre Jérémie ! Car la loi ne périra pas chez le sacrificateur, ni le conseil chez le sage, ni la parole chez le prophète. Venez, et tuons-le avec la langue, et ne soyons pas attentifs à ses discours !
\VS{19}Yahweh ! Fais attention à moi, et écoute la voix de ceux qui contestent avec moi !
\VS{20}Le mal sera-t-il rendu pour le bien\FTNT{Ps. 35:12 ; Ps. 109:5.} ? Car ils ont creusé une fosse pour mon âme. Souviens-toi que je me suis tenu devant toi, afin de parler pour leur bien, et afin de détourner d'eux ta grande colère.
\VS{21}C'est pourquoi livre leurs fils à la famine, et fais couler leur sang à coups d’épée ; que leurs femmes soient privées d'enfants, et deviennent veuves, et que leurs maris soient enlevés par la mort ; et leurs jeunes gens frappés par l'épée dans la bataille\FTNT{Ps. 109:9-13.} !
\VS{22}Qu'on entende le cri de leurs maisons, quand tu feras venir subitement des troupes contre eux ! Car ils ont creusé une fosse pour me prendre, et ils ont caché des pièges pour mes pieds.
\VS{23}Or tu sais, ô Yahweh ! Que tout leur conseil est contre moi pour me mettre à mort ; ne sois pas apaisé à l'égard de leur iniquité, et n'efface pas leur péché de devant ta face, mais qu'on les fasse tomber en ta présence ; agis contre eux au temps de ta colère.
\Chap{19}
\TextTitle{Le vase brisé : Image de Juda}
\VerseOne{}Ainsi a parlé Yahweh : Va, et achète un vase de terre d'un potier, et prends avec toi des anciens du peuple et des anciens des sacrificateurs.
\VS{2}Et sors à la vallée de Ben-Hinnom, qui est auprès de l'entrée de la porte de la poterie, et crie là les paroles que je te dirai.
\VS{3}Dis donc : Rois de Juda, et vous, habitants de Jérusalem, écoutez la parole de Yahweh ! Ainsi parle Yahweh des armées, le Dieu d'Israël : Voici, je vais faire venir sur ce lieu-ci un mal, tel que quiconque l'entendra, les oreilles lui tinteront\FTNT{1 S. 3:11 ; 2 R. 21:12.} ; 
\VS{4}parce qu'ils m'ont abandonné, et qu'ils ont profané ce lieu, et y ont brûlé de l'encens à d'autres dieux, que ni eux, ni leurs pères, ni les rois de Juda n'ont connus, et parce qu'ils ont rempli ce lieu du sang des innocents ;
\VS{5}et qu'ils ont bâti des hauts lieux à Baal, afin de brûler au feu leurs fils pour en faire des holocaustes à Baal : Ce que je n'avais pas ordonné, et dont je n'avais pas parlé, et qui ne m'était pas monté à cœur.
\VS{6}A cause de cela, voici les jours viennent, dit Yahweh, que ce lieu-ci ne sera plus appelé Topheth, ni la vallée de Ben-Hinnom, mais la vallée de la tuerie.
\VS{7}Et j'anéantirai dans ce lieu-ci le conseil de Juda et de Jérusalem ; et je les ferai tomber par l'épée devant leurs ennemis et par la main de ceux qui cherchent leur vie ; et je donnerai leurs cadavres en pâture aux oiseaux des cieux et aux bêtes de la terre.
\VS{8}Je ferai de cette ville un objet de désolation et de moquerie ; quiconque passera près d'elle sera étonné et sifflera à cause de toutes ses plaies.
\VS{9}Et je leur ferai manger la chair de leurs fils et la chair de leurs filles ; et chacun mangera la chair de son compagnon durant le siège, et dans la détresse où les réduiront leurs ennemis et ceux qui cherchent leur vie\FTNT{Lé. 26:29 ; De. 28:53 ; La. 2:20.}.
\VS{10}Puis tu briseras le vase, sous les yeux des hommes qui seront allés avec toi.
\VS{11}Et tu leur diras : Ainsi parle Yahweh des armées : Je briserai ce peuple et cette ville, de même qu'on brise un vase de potier, qui ne peut être réparé. Et ils seront enterrés à Topheth parce qu'il n'y aura plus d'autre lieu pour les enterrer.
\VS{12}Je ferai ainsi à ce lieu-ci, dit Yahweh, et à ses habitants, et je rendrai cette ville semblable à Topheth ;
\VS{13}et les maisons de Jérusalem, et les maisons des rois de Juda, seront impures comme le lieu de Topheth, à cause de toutes les maisons sur les toits desquelles ils brûlaient de l'encens à toute l'armée des cieux, et faisaient des libations à d'autres dieux.
\VS{14}Puis Jérémie revint de Topheth, là où Yahweh l'avait envoyé pour prophétiser. Et il se tint debout dans le parvis de la maison de Yahweh, et il dit à tout le peuple :
\VS{15}Ainsi parle Yahweh des armées, le Dieu d'Israël : Voici, je vais faire venir sur cette ville et sur toutes ses villes tout le mal que j'ai prononcés contre elle, parce qu'ils ont raidi leur cou pour ne pas écouter mes paroles.
\Chap{20}
\TextTitle{Paschhur outrage Jérémie}
\VerseOne{}Alors Paschhur, fils d'Immer, qui était sacrificateur et inspecteur en chef dans la maison de Yahweh, entendit Jérémie qui prophétisait ces choses.
\VS{2}Et Paschhur frappa le prophète Jérémie, et le mit dans la prison qui était à la porte supérieure de Benjamin, dans la maison de Yahweh.
\VS{3}Et il arriva que dès le lendemain, que Paschhur tira Jérémie hors de la prison. Et Jérémie lui dit : Yahweh ne t'appelle pas du nom de Paschhur, mais Magor-Missabib\FTNT{« Magor-Missabib » veut dire « terreur de chaque côté ».}.
\VS{4}Car ainsi parle Yahweh : Voici, je vais te livrer à la terreur, toi et tous tes amis qui tomberont par l'épée de leurs ennemis, et tes yeux le verront. Je livrerai tous ceux de Juda entre les mains du roi de Babylone, qui les transportera à Babylone et les frappera de l'épée.
\VS{5}Et je livrerai toutes les richesses de cette ville, et tout son travail, et tout ce qu'elle a de précieux, je livrerai, dis-je, tous les trésors des rois de Juda entre les mains de leurs ennemis, qui les pilleront, les enlèveront et les conduiront à Babylone.
\VS{6}Et toi, Paschhur, et tous ceux qui demeurent dans ta maison, vous irez en captivité ; tu iras à Babylone, tu y mourras, et y seras enterré, toi et tous tes amis auxquels tu as prophétisé le mensonge.
\TextTitle{Jérémie gémit auprès de Yahweh}
\VS{7}Ô Yahweh ! Tu m'as persuadé, et je me suis laissé persuader ; tu m'as saisi, et tu m'as vaincu. Je suis un objet de moquerie chaque jour, chacun se moque de moi.
\VS{8}Car depuis que je parle, je crie, je crie violence et dévastation ! Et la parole de Yahweh est pour moi un sujet d'opprobre et de moquerie chaque jour\FTNT{Es. 57:4.}.
\VS{9}C'est pourquoi j'ai dit : Je ne ferai plus mention de lui, je ne parlerai plus en son Nom, mais il y a eu dans mon cœur comme un feu ardent, renfermé dans mes os ; je me fatigue à le contenir, et je ne le puis.
\VS{10}Car j'entends les mauvais propos de plusieurs, la frayeur m'a saisi de tous côtés ; rapportez, disent-ils, et nous le rapporterons ! Tous ceux qui étaient en paix avec moi observent si je bronche, et disent : Peut-être se laissera-t-il séduire, et nous le vaincrons, nous tirerons vengeance de lui !
\VS{11}Mais Yahweh est avec moi comme un héros puissant ; c'est pourquoi ceux qui me persécutent seront renversés, ils ne me vaincront pas ; ils seront honteux, car ils n'ont pas réussi : Ce sera une honte éternelle qui ne s'oubliera jamais.
\VS{12}Yahweh des armées qui éprouve les justes, qui voit les reins et les cœurs, fais que je voie ta vengeance s'exercer contre eux, car je t'ai découvert ma cause.
\VS{13}Chantez à Yahweh, louez Yahweh ! Car il délivre l'âme des pauvres de la main des méchants.
\VS{14}Maudit soit le jour où je suis né ! Que le jour où ma mère m'a enfanté ne soit pas béni !
\VS{15}Maudit soit l'homme qui porta cette nouvelle à mon père, en lui disant : Un fils mâle t'est né, et qui le combla de joie !
\VS{16}Que cet homme-là soit comme les villes que Yahweh a renversées sans s'en repentir ! Qu'il entende la clameur le matin, et le cri de guerre au temps du midi\FTNT{Ge. 19:24-25 ; So. 2:4.} !
\VS{17}Que ne m'a-t-on fait mourir dans le sein de ma mère ! Pourquoi ma mère ne m'a-t-elle pas servi de sépulcre ? Et pourquoi n'est-elle pas restée éternellement enceinte ?
\VS{18}Pourquoi suis-je sorti de son sein pour ne voir que peine et douleur, et pour consumer mes jours dans la honte ?
\Chap{21}
\TextTitle{Prophétie sur les rois de Juda : Sédécias}
\VerseOne{}La parole qui fut adressée à Jérémie de la part de Yahweh, lorsque le roi Sédécias envoya vers lui Paschhur, fils de Malkija, et Sophonie, fils de Maaséja, le sacrificateur, pour lui dire :
\VS{2}Consulte maintenant Yahweh pour nous ; car Nebucadnetsar, roi de Babylone, combat contre nous ; peut-être que Yahweh fera-t-il en notre faveur un de ses miracles, afin qu'il se retire de nous.
\VS{3}Et Jérémie leur dit : Vous direz ainsi à Sédécias :
\VS{4}Ainsi parle Yahweh, le Dieu d'Israël : Voici, je vais détourner les armes de guerre qui sont dans vos mains, et avec lesquelles vous combattez en dehors des murailles contre le roi de Babylone et contre les Chaldéens qui vous assiègent, et je les rassemblerai au milieu de cette ville.
\VS{5}Et je combattrai contre vous, avec une main étendue, et avec un bras puissant, avec colère, avec fureur, et avec une grande indignation.
\VS{6}Et je frapperai les habitants de cette ville, les hommes, et les bêtes ; et ils mourront d'une grande peste.
\VS{7}Et après cela, dit Yahweh, je livrerai Sédécias, roi de Juda, et ses serviteurs, et le peuple, et ceux qui dans cette ville survivront à la peste, à l'épée et à la famine, entre les mains de Nebucadnetsar\FTNT{2 R. 24 et 25. }, roi de Babylone, et entre les mains de leurs ennemis, et entre les mains de ceux qui cherchent leur vie ; et il les frappera au tranchant de l'épée, il ne les épargnera pas, il n'en aura pas de compassion, il n'en aura pas de pitié.
\VS{8}Tu diras aussi à ce peuple : Ainsi parle Yahweh : Voici, je mets devant vous le chemin de la vie et le chemin de la mort\FTNT{De. 30:19.}.
\VS{9}Quiconque restera dans cette ville mourra par l'épée, ou par la famine, ou par la peste, mais celui qui en sortira, et se rendra aux Chaldéens qui vous assiègent vivra et aura sa vie pour butin.
\VS{10}Car je dresse ma face en mal et non en bien contre cette ville, dit Yahweh ; elle sera livrée entre les mains du roi de Babylone, et il la brûlera par le feu.
\VS{11}Et quant à la maison du roi de Juda : Ecoutez la parole de Yahweh !
\VS{12}Maison de David ! Ainsi parle Yahweh : Rendez la justice dès le matin, et délivrez celui qui aura été pillé d'entre les mains de l'oppresseur, de peur que ma fureur ne sorte comme un feu, et qu'elle ne brûle sans qu'on puisse l'éteindre, à cause de la méchanceté de vos actions.
\VS{13}Voici, j'en veux à toi qui habites dans la vallée, et qui es le rocher de la plaine, dit Yahweh ; à vous qui dites : Qui descendra contre nous, et qui entrera dans nos demeures ?
\VS{14}Et je vous punirai selon le fruit de vos actions, dit Yahweh ; et je mettrai le feu dans sa forêt qui consumera tout ce qui est autour d'elle\FTNT{Ez. 21:2-3.}.
\Chap{22}
\TextTitle{Sédécias averti de la destruction de Jérusalem}
\VerseOne{}Ainsi parle Yahweh : Descends dans la maison du roi de Juda, et là prononce cette parole.
\VS{2}Tu diras donc : Ecoute la parole de Yahweh, ô roi de Juda qui es assis sur le trône de David, toi et tes serviteurs, et ton peuple, qui entrez par ces portes !
\VS{3}Ainsi parle Yahweh : Faites droit et justice ; et délivrez celui qui aura été pillé d'entre les mains de l'oppresseur ; ne maltraitez pas l'orphelin, ni l'étranger, ni la veuve ; et n'usez d'aucune violence, et ne répandez pas le sang innocent dans ce lieu-ci.
\VS{4}Car si vous mettez exactement en effet cette parole, alors les rois qui sont assis à la place de David sur son trône, montés sur des chars et sur des chevaux, entreront par les portes de cette maison, eux et leurs serviteurs, et leur peuple.
\VS{5}Mais si vous n'écoutez pas ces paroles, je le jure par moi-même\FTNT{Es. 45:23 ; Hé. 6:13.}, dit Yahweh, que cette maison deviendra une ruine.
\VS{6}Car ainsi parle Yahweh sur la maison du roi de Juda : Tu es pour moi un Galaad, et le sommet du Liban ; mais certainement, je ferai de toi un désert, une ville sans habitants.
\VS{7}Je prépare contre toi des destructeurs, chacun avec ses armes, qui couperont tes cèdres de choix, et les jetteront au feu.
\VS{8}Et plusieurs nations passeront près de cette ville, et chacun dira à son compagnon : Pourquoi Yahweh a-t-il fait ainsi à cette grande ville\FTNT{De. 29:24-28 ; 1 R. 9:8.} ?
\VS{9}Et on dira : C'est parce qu'ils ont abandonné l'alliance de Yahweh, leur Dieu, et qu'ils se sont prosternés devant d'autres dieux et les ont servis.
\TextTitle{Prophétie sur les rois de Juda : Joachaz (Schallum)}
\VS{10}Ne pleurez pas celui qui est mort, et ne vous lamentez pas sur lui ; mais pleurez amèrement celui qui s'en va, car il ne reviendra plus, il ne reverra plus le pays de sa naissance.
\VS{11}Car ainsi parle Yahweh sur Schallum, fils de Josias, roi de Juda, qui régnait à la place de Josias, son père, et qui est sorti de ce lieu : Il n'y reviendra plus ;
\VS{12}mais il mourra dans le lieu où on l'a transporté, et ne verra plus ce pays.
\TextTitle{Prophétie sur les rois de Juda : Jojakim}
\VS{13}Malheur à celui qui bâtit sa maison par l'injustice, et ses chambres hautes sans droiture ; qui fait travailler son prochain pour rien, sans lui donner le salaire de son travail\FTNT{Lé. 19:13 ; De. 24:14-15 ; Ha. 2:9.}.
\VS{14}Qui dit : Je me bâtirai une grande maison et des chambres spacieuses, et qui s'y fait percer des fenêtres ; elle est lambrissée de cèdre, et peinte de vermillon.
\VS{15}Régneras-tu, parce que tu t’enfermes dans du cèdre ? Ton père n'a-t-il pas mangé et bu ? Quand il a fait jugement et justice, alors il a prospéré.
\VS{16}Il jugeait la cause du pauvre et de l'indigent, alors il a prospéré. N'est-ce pas là me connaître ? Dit Yahweh.
\VS{17}Mais tes yeux et ton cœur ne sont adonnés qu'à ton gain déshonnête, qu'à répandre le sang innocent, qu'à faire du tord et qu'à opprimer.
\VS{18}C'est pourquoi ainsi parle Yahweh sur Jojakim, fils de Josias, roi de Juda : On ne le pleurera pas en disant : Hélas, mon frère ! et hélas, ma sœur ! On ne le pleurera pas en disant : Hélas, seigneur ! Et hélas, sa majesté !
\VS{19}Il sera enterré de la sépulture d'un âne, étant traîné et jeté hors des portes de Jérusalem.
\TextTitle{Prophétie sur les rois de Juda : Jojakin}
\VS{20}Monte sur le Liban, et crie ! Donne de la voix sur le Basan ! Crie du haut d'Abarim ! A cause que tous ceux qui t'aiment sont brisés.
\VS{21}Je t'ai parlé durant ta grande prospérité, mais tu disais : Je n'écouterai pas ; telle est ta voie depuis ta jeunesse, que tu n'as pas écouté ma voix.
\VS{22}Tous tes pasteurs seront la pâture du vent, et ceux qui t'aiment iront en captivité ; certainement tu seras alors honteuse et confuse, à cause de toute ta méchanceté.
\VS{23}Toi qui habites sur le Liban, et qui fais ton nid dans les cèdres, que tu seras à plaindre quand les douleurs t'atteindront, les douleurs comme celles d'une femme qui enfante.
\VS{24}Je suis vivant, dit Yahweh, que quand Jéconia, fils de Jojakim, roi de Juda, serait une bague à ma main droite, je t'arracherais de là.
\VS{25}Je te livrerai entre les mains de ceux qui cherchent ta vie, et entre les mains devant qui tu es craintif, et entre les mains de Nebucadnetsar, roi de Babylone, et entre les mains des Chaldéens\FTNT{2 R. 24:14 ; Ez. 17:12 ; 2 Ch. 36:10.}.
\VS{26}Et je te jetterai, toi et ta mère qui t'a enfanté, dans un autre pays où vous n'êtes pas nés, et vous y mourrez.
\VS{27}Et quant au pays qu'ils désirent pour y retourner, ils n'y retourneront pas.
\VS{28}Cet homme, Jéconia, est-il un vase méprisé et brisé ? Est-il un objet qui ne fait plus plaisir ? Pourquoi sont-ils jetés là, lui et sa postérité, lancés, dis-je, dans un pays qu'ils ne connaissent pas\FTNT{Os. 8:8.} ?
\VS{29}Ô terre, terre, terre ! Ecoute la parole de Yahweh !
\VS{30}Ainsi parle Yahweh : Ecrivez que cet homme-là est privé d'enfant, que c'est un homme qui ne prospérera pas pendant ses jours, et que même il n'y aura pas d'homme de sa postérité qui prospère, et qui soit assis sur le trône de David, ni qui domine plus en dominer sur Juda\FTNT{2 R. 24:8-16.}.
\Chap{23}
\TextTitle{Israël sera rassemblé par le Messie}
\VerseOne{}Malheur aux pasteurs qui détruisent et dispersent le troupeau de mon pâturage ! Dit Yahweh.
\VS{2}C'est pourquoi ainsi parle Yahweh, le Dieu d'Israël, sur les pasteurs qui paissent mon peuple : Vous avez dispersé\FTNT{Les brebis du Seigneur sont dispersées ou éparpillées par les faux pasteurs (Ez. 34).} mes brebis, et vous les avez chassées, et ne vous en êtes pas occupés ; voici, je vous punirai à cause de la méchanceté de vos actions, dit Yahweh.
\VS{3}Mais je rassemblerai le reste de mes brebis de tous les pays où je les ai chassées ; et je les ramènerai à leur pâturage, et elles seront fécondes et multiplieront.
\VS{4}Je susciterai aussi sur elles des pasteurs qui les paîtront, et elles n'auront plus de peur, et ne s'épouvanteront plus, et il n'en manquera aucune, dit Yahweh.
\VS{5}Voici, les jours viennent, dit Yahweh, où je susciterai à David un Germe juste, qui régnera en Roi ; il prospérera, et exercera le droit et la justice dans le pays\FTNT{Es. 4:2 ; Za. 6:12-13 ; Ps. 96:13 ; Lu. 1:32-33.}.
\VS{6}En son temps, Juda sera sauvé, Israël demeurera en sécurité ; et c'est ici le nom dont on l'appellera : Yahweh notre justice.
\VS{7}C'est pourquoi, voici, les jours viennent, dit Yahweh, qu'on ne dira plus : Yahweh est vivant, lui qui a fait monter les fils d'Israël du pays d'Egypte !
\VS{8}Mais : Yahweh est vivant, lui qui a fait monter et qui a ramené la postérité de la maison d'Israël, du pays du nord et de tous les pays où je les avais chassés, et ils habiteront dans leur pays.
\TextTitle{Jugement sur les faux prophètes}
\VS{9}A cause des prophètes mon cœur est brisé au-dedans de moi, tous mes os se relâchent ; je suis comme un homme ivre, et comme un homme que le vin a surmonté, à cause de Yahweh, et à cause des paroles de sa sainteté.
\VS{10}Car le pays est rempli d'hommes qui commettent l'adultère ; et le pays est en deuil à cause de la malédiction : Les pâturages du désert sont desséchés, leur course ne va qu'au mal, et leur force à ce qui n'est pas droit.
\VS{11}Car le prophète et le sacrificateur sont corrompus ; j'ai même trouvé dans ma maison leur méchanceté, dit Yahweh.
\VS{12}C'est pourquoi leur chemin sera comme des lieux glissants dans l'obscurité, ils y seront poussés et ils tomberont\FTNT{Ps. 35:6 ; Pr. 4:19.} ; car je ferai venir du mal sur eux, dans l'année de leur châtiment, dit Yahweh.
\VS{13}Or j'ai vu de la folie dans les prophètes de Samarie, car ils prophétisaient par Baal, et faisaient égarer mon peuple Israël.
\VS{14}Mais j'ai vu des choses horribles dans les prophètes de Jérusalem car ils commettent des adultères, et ils marchent dans le mensonge ; ils fortifient les mains de ceux qui font le mal, afin qu'aucun ne se détourne de sa méchanceté ; ils me sont tous comme Sodome, et les habitants de la ville comme Gomorrhe\FTNT{Es. 1:9.}.
\VS{15}C'est pourquoi, ainsi parle Yahweh des armées sur les prophètes : Voici, je vais leur faire manger de l'absinthe, et leur ferai boire des eaux empoisonnées ; car c'est par les prophètes que la profanation est venue dans tout le pays.
\VS{16}Ainsi parle Yahweh des armées : N'écoutez pas les paroles des prophètes qui vous prophétisent ! Ils vous font devenir vains, ils disent les visions de leur cœur, et ils ne les tiennent pas de la bouche de Yahweh.
\VS{17}Ils ne cessent de dire à ceux qui me méprisent : Yahweh a dit : Vous aurez la paix ; et ils disent à tous ceux qui marchent suivant les penchants de leur cœur : Il ne vous arrivera aucun mal\FTNT{Ez. 13:10.}.
\VS{18}Car qui s'est trouvé au conseil secret de Yahweh ? Et qui a aperçu et entendu sa parole\FTNT{Es. 40:13 ; Job. 15:8 ; 1 Co. 2:16.} ? Qui a été attentif à sa parole, et l'a entendue ?
\VS{19}Voici la tempête de Yahweh, sa fureur va se montrer, et le tourbillon prêt à fondre tombera sur la tête des méchants.
\VS{20}La colère de Yahweh ne se détournera pas jusqu'à ce qu'il ait accompli, exécuté les desseins de son cœur. Vous aurez une claire intelligence de ceci dans les derniers jours\FTNT{Gn. 49:1-2.}.
\VS{21}Je n'ai pas envoyé ces prophètes-là, et ils ont couru ; je ne leur ai pas parlé, et ils ont prophétisé.
\VS{22}S'ils s'étaient trouvés dans mon conseil secret, ils auraient aussi fait entendre mes paroles à mon peuple, et ils les auraient ramenés de leur mauvaise voie, de la méchanceté de leurs actions.
\VS{23}Suis-je un Dieu de près, dit Yahweh, et ne suis-je pas aussi un Dieu de loin ?
\VS{24}Quelqu'un se cachera-t-il dans un lieu secret sans que je ne le voie ? Dit Yahweh. Ne remplis-je pas, moi, les cieux et la terre ? Dit Yahweh\FTNT{Ps. 139:7-8 ; Am. 9:2-3.}.
\VS{25}J'ai entendu ce que les prophètes disent, prophétisant le mensonge en mon nom, et disant : J'ai eu un songe ! J'ai eu un songe !
\VS{26}Jusqu'à quand ceci sera-t-il au cœur des prophètes qui prophétisent le mensonge, et qui prophétisent la tromperie de leur cœur ?
\VS{27}Qui pensent comment ils feront oublier mon nom à mon peuple, par les songes que chacun d'eux raconte à son compagnon, comme leurs pères ont oublié mon Nom pour Baal\FTNT{Jg. 2:13.}.
\VS{28}Que le prophète qui a eu un songe raconte ce songe ; et que celui qui a ma parole proclame ma parole en vérité. Quelle convenance y a-t-il entre la paille et le froment ? Dit Yahweh.
\VS{29}Ma parole n'est-elle pas comme un feu, dit Yahweh, et comme un marteau qui brise le roc ?
\VS{30}C'est pourquoi voici, dit Yahweh, j'en veux aux prophètes qui se dérobent mes paroles l'un à l'autre.
\VS{31}Voici, dit Yahweh, j'en veux aux prophètes qui accommodent leurs langues, et qui disent : Il dit.
\VS{32}Voici, dit Yahweh, j'en veux à ceux qui prophétisent des songes faux, et qui les racontent, et font égarer mon peuple par leurs mensonges, et par leur témérité, quoique je ne les ai pas envoyés, et que je ne leur aie pas donné d'ordre ; c'est pourquoi ils ne sont d'aucune utilité à ce peuple, dit Yahweh\FTNT{So. 3:4.}.
\VS{33}Si donc ce peuple t'interroge, ou qu'il interroge le prophète, ou le sacrificateur, en disant : Quel est l'oracle de Yahweh ? Tu leur diras : Quel est cet oracle ? Je vous abandonnerai, dit Yahweh.
\VS{34}Et quant au prophète, et au sacrificateur, et au peuple qui dira : Oracle de Yahweh ; je punirai cet homme-là et sa maison.
\VS{35}Vous direz ainsi chacun à son compagnon, et chacun à son frère : Qu'a répondu Yahweh ? Qu'a dit Yahweh ?
\VS{36}Et vous ne mentionnerez plus : Oracle de Yahweh ; car la parole de chacun sera pour lui un oracle ; parce que vous tordez les paroles du Dieu vivant\FTNT{2 Pi. 3:15-16.}, les paroles de Yahweh des armées, notre Dieu.
\VS{37}Tu diras au prophète : Que t'a répondu Yahweh et que t'a dit Yahweh ?
\VS{38}Et si vous dites : Oracle de Yahweh ; à cause de cela, parle Yahweh, parce que vous dites cette parole : Oracle de Yahweh ; et que j'ai envoyé vers vous pour dire : Ne dites plus : Oracle de Yahweh !
\VS{39}A cause de cela, me voici, et je vous oublierai entièrement, et je vous rejetterai loin de ma présence, vous et la ville que j'ai donnée à vous et à vos pères.
\VS{40}Et je mettrai sur vous un opprobre éternel et une honte éternelle, qui ne s'oublieront pas.
\Chap{24}
\TextTitle{Bonnes figues: Futur retour en Juda des captifs de Babylone ; mauvaises figues: Jugement sur Jérusalem}
\VerseOne{}Yahweh me fit voir une vision, et voici deux paniers de figues posés devant le temple de Yahweh, après que Nebucadnetsar, roi de Babylone, eut transporté de Jérusalem, Jéconia, fils de Jojakim, roi de Juda, les chefs de Juda, avec les charpentiers et les serruriers, et les eut conduits à Babylone.
\VS{2}L'un des paniers avait de très bonnes figues, comme les figues de la première récolte ; et l'autre panier avait de très mauvaises figues, qu'on ne pouvait manger à cause de leur mauvaise qualité.
\VS{3}Et Yahweh me dit : Que vois-tu, Jérémie ? Et je répondis : Des figues. Les bonnes figues sont très bonnes, et les mauvaises sont très mauvaises et ne peuvent être mangées à cause de leur mauvaise qualité.
\VS{4}Alors la parole de Yahweh me fut adressée, en disant :
\VS{5}Ainsi parle Yahweh, le Dieu d'Israël : Comme tu distingues ces bonnes figues, ainsi je me souviendrai, pour leur faire du bien, des captifs de Juda, que j'ai envoyés hors de ce lieu dans le pays des Chaldéens.
\VS{6}Et je les regarderai d'un œil favorable, et je les ramènerai dans ce pays, je les y rétablirai et je ne les détruirai plus, je les planterai et je ne les arracherai pas.
\VS{7}Et je leur donnerai un cœur pour me connaître, pour connaître, dis-je, que je suis Yahweh ; et ils seront mon peuple, et je serai leur Dieu : Car ils reviendront à moi de tout leur cœur\FTNT{De. 30:6 ; Ez. 11:19.}.
\VS{8}Et comme les mauvaises figues, qui ne peuvent être mangées à cause de leur mauvaise qualité ; ainsi certainement, dit Yahweh, je ferai devenir Sédécias, roi de Juda, ses chefs, et le reste de Jérusalem, ceux qui sont restés dans ce pays, et ceux qui habitent dans le pays d'Egypte.
\VS{9}Et je les livrerai pour être agités pour leur malheur par tous les royaumes de la terre, et pour être en opprobre, en de proverbe, en raillerie, et en malédiction, par tous les lieux où je les aurai chassé\FTNT{De. 28:37.}.
\VS{10}Et j'enverrai sur eux l'épée, la famine et la peste, jusqu'à ce qu'ils soient consumés du pays que j'avais donné à eux et à leurs pères.
\Chap{25}
\TextTitle{Prophétie sur les soixante-dix ans de captivité babylonienne\FTNTT{Da.9:2.}}
\VerseOne{}La parole qui fut adressée à Jérémie touchant tout le peuple de Juda, la quatrième année de Jojakim, fils de Josias, roi de Juda, qui est la première année de Nebucadnetsar, roi de Babylone,
\VS{2}parole que Jérémie, le prophète, prononça à tout le peuple de Juda, et à tous les habitants de Jérusalem, en disant :
\VS{3}Depuis la treizième année de Josias, fils d'Amon, roi de Juda, jusqu'à ce jour, qui est la vingt-troisième année, la parole de Yahweh m'a été adressée ; et je vous ai parlé, me levant dès le matin et parlant, et vous n'avez pas écouté.
\VS{4}Et Yahweh vous a envoyé tous ses serviteurs, les prophètes, se levant dès le matin et les envoyant ; et vous ne les avez pas écoutés, vous n'avez pas prêté l'oreille pour écouter.
\VS{5}Lorsqu'ils disaient : Détournez-vous maintenant chacun de sa mauvaise voie et de la méchanceté de vos actions, et vous habiterez d'un siècle à l'autre dans le pays que Yahweh a donné à vous et à vos pères\FTNT{Jon. 3:8 ; 2 R. 17:13.} ;
\VS{6}et n'allez pas après d'autres dieux pour les servir et pour vous prosterner devant eux, ne m'irritez pas par les œuvres de vos mains, et je ne vous ferai aucun mal.
\VS{7}Mais vous ne m'avez désobéi, dit Yahweh, pour m'irriter par les œuvres de vos mains, pour votre malheur.
\VS{8}C'est pourquoi ainsi parle Yahweh des armées : Parce que vous n'avez pas écouté mes paroles,
\VS{9}voici j'enverrai et j'assemblerai toutes les familles du nord, dit Yahweh, et j'enverrai, dis-je, vers Nebucadnetsar, roi de Babylone, mon serviteur ; et je les ferai venir contre ce pays et contre ses habitants, et contre toutes ces nations d'alentour, je les détruirai à la façon de l'interdit, je les mettrai en désolation, en opprobre et en ruines éternelles\FTNT{De. 28:37 ; Es. 10:6.}.
\VS{10}Et je ferai cesser parmi eux les cris de joie et les cris d'allégresse, la voix de l'époux et la voix de l'épouse, le bruit des meules et la lumière des lampes\FTNT{Es. 24:7 ; Ez. 26:13.}.
\VS{11}Et tout ce pays sera une ruine jusqu'à s'en étonner, et ces nations seront asservies au roi de Babylone pendant soixante-dix ans\FTNT{Voir Jé. 29 : 10. Les soixante-dix ans se rapportent également au temps de la domination mondiale babylonienne. Le peuple avait une dette envers Yahweh de 70 ans de sabbats (Lé. 26 34-43 ; 2 Ch. 36:21).}.
\TextTitle{Jugement sur Babylone et les nations impies}
\VS{12}Et il arrivera que quand ces soixante-dix ans seront accomplis, je punirai le roi de Babylone et cette nation, dit Yahweh, à cause de leurs iniquités ; je punirai le pays des Chaldéens, que je mettrai en désolations éternelles\FTNT{Da. 9:2.}.
\VS{13}Et je ferai venir sur ce pays-là toutes mes paroles que j'ai prononcées contre lui, toutes les choses qui sont écrites dans ce livre, ce que Jérémie a prophétisé contre toutes ces nations.
\VS{14}Car de grands rois aussi et de grandes nations se serviront d'eux, et je leur rendrai selon leurs actions et selon l'œuvre de leurs mains.
\VS{15}Car ainsi m'a parlé Yahweh, le Dieu d'Israël : Prends de ma main cette coupe du vin, savoir de cette fureur-ci, et fais-la boire à toutes les nations vers lesquelles je t'enverrai\FTNT{Ab. 16.}.
\VS{16}Ils en boiront, et ils chancelleront et seront comme fous, à cause de l'épée que j'enverrai parmi eux.
\VS{17}Je pris donc la coupe de la main de Yahweh, et je la fis boire à toutes les nations vers lesquelles Yahweh m'envoyait :
\VS{18}Savoir : A Jérusalem et aux villes de Juda, à ses rois et à ses chefs, pour les mettre en désolation, en étonnement, en opprobre et en malédiction, comme il paraît aujourd'hui ;
\VS{19}à Pharaon, roi d'Egypte, à ses serviteurs, à ses chefs, et à tout son peuple ;
\VS{20}et à tout le mélange des peuples d'Arabie, à tous les rois du pays d'Uts, à tous les rois du pays des Philistins, à Askalon, à Gaza, à Ekron, et au reste d'Asdod ;
\VS{21}à Edom, à Moab, et aux fils d'Ammon ;
\VS{22}à tous les rois de Tyr, à tous les rois de Sidon, et aux rois des îles qui sont au-delà de la mer ;
\VS{23}à Dedan, à Théma, à Buz, et à tous ceux qui se coupent les coins de la barbe ;
\VS{24}à tous les rois d'Arabie, et à tous les rois des Arabes qui habitent au désert ;
\VS{25}à tous les rois de Zimri, à tous les rois d'Elam, et à tous les rois de Médie ;
\VS{26}à tous les rois du nord, tant proches qu'éloignés l'un de l'autre, et à tous les royaumes du monde qui sont sur la face de la terre. Et le roi de Schéschac boira après eux.
\VS{27}Et tu leur diras : Ainsi parle Yahweh des armées, le Dieu d'Israël : Buvez et soyez enivrés, même vomissez, et tombez sans vous relever, à cause de l'épée que j'enverrai parmi vous !
\VS{28}Or il arrivera qu'ils refuseront de prendre la coupe de ta main pour boire; mais tu leur diras : Ainsi parle Yahweh des armées : Vous en boirez certainement !
\VS{29}Car voici, je commence à envoyer du mal sur la ville sur laquelle mon nom est invoqué ; et vous, en seriez exempts en quelque sorte ? Vous n'en serez pas exempts ; car je m'en vais appeler l'épée sur tous les habitants de la terre, dit Yahweh des armées\FTNT{1 Pi. 4:17-18.}.
\VS{30}Tu prophétiseras donc contre eux toutes ces paroles-là, et tu leur diras : Yahweh rugira d'en haut ; il fera entendre sa voix de la demeure de sa sainteté ; il rugira, il rugira contre son son agréable demeure ; il poussera un cri  contre tous les habitants de la terre\FTNT{Joë. 3:16 ; Am. 1:2.}, comme ceux qui foulent au pressoir,
\VS{31}le son éclatant est parvenu jusqu'à l'extrémité de la terre ; car Yahweh plaide avec les nations, et il conteste contre toute chair. On livrera les méchants à l'épée, dit Yahweh.
\VS{32}Ainsi parle Yahweh des armées : Voici, le mal va sortir d'une nation à l'autre, et une grande tempête se se lèvera des extrémités de la terre.
\VS{33}Et en ce jour-là, ceux qui auront été mis à mort par Yahweh seront étendus d'un bout de la terre à l'autre bout ; ils ne seront ni pleurés, ni recueillis, ni enterrés, mais ils seront comme du fumier sur la face du sol.
\VS{34}Vous, pasteurs, hurlez et criez ! Et vous, les nobles du troupeau, roulez-vous dans la cendre ; car les jours pour vous massacrer sont accomplis. Je vous disperserai et vous tomberez comme un vase précieux.
\VS{35}Et les pasteurs n'auront aucun moyen de s'enfuir, ni les nobles d'échapper. 
\VS{36}Il y aura la voix du cri des bergers, les hurlements des nobles du troupeau ; parce que Yahweh s'en va ravager leur pâturage.
\VS{37}Et les demeures paisibles seront abattues, à cause de l'ardeur de la colère de Yahweh.
\VS{38}Il a abandonné son tabernacle comme un lionceau ; car leur pays est réduit en désert, à cause de l'ardeur du destructeur et à cause, dis-je, de l'ardeur de sa colère.
\Chap{26}
\TextTitle{Avertissement dans le parvis du temple}
\VerseOne{}Au commencement du règne de Jojakim, fils de Josias, roi de Juda, cette parole fut adressée à Jérémie part Yahweh, en disant :
\VS{2}Ainsi parle Yahweh : Tiens-toi debout dans le parvis de la maison de Yahweh, et prononce à toutes les villes de Juda qui viennent pour se prosterner dans la maison de Yahweh toutes les paroles que je t'ordonne de leur dire ; n'en retranche pas un mot.
\VS{3}Peut-être qu'ils écouteront et qu'ils se détourneront chacun de sa mauvaise voie ; et je me repentirai du mal que j'avais pensé leur faire à cause de la méchanceté de leurs actions.
\VS{4}Tu leur diras donc : Ainsi parle Yahweh : Si vous ne m'écoutez pas pour marcher selon ma loi que je vous ai proposée,
\VS{5}pour obéir aux paroles des prophètes, mes serviteurs, que je vous envoie, me levant dès le matin, et les envoyant, lesquels que vous n'avez pas écoutés,
\VS{6}je mettrai cette maison dans le même état que Silo, et je livrerai cette ville à la malédiction, à toutes les nations de la terre.
\VS{7}Or les sacrificateurs, les prophètes, et tout le peuple, entendirent Jérémie prononcer ces paroles dans la maison de Yahweh.
\TextTitle{Jérémie menacé de mort par les sacrificateurs et les prophètes}
\VS{8}Et il arrivera qu'aussitôt que Jérémie eut achevé prononcer tout ce que Yahweh lui avait ordonné de dire à tout le peuple, les sacrificateurs, les prophètes, et tout le peuple, le saisirent en disant : Tu mourras, tu mourras\FTNT{Gn. 2:17.} !
\VS{9}Pourquoi as-tu prophétisé au nom de Yahweh, en disant : Cette maison sera comme Silo, et cette ville sera déserte tellement que personne n'y habitera ? Et tout le peuple s'assembla autour de Jérémie dans la maison de Yahweh.
\VS{10}Et les les chefs de Juda ayant entendu toutes ces choses, montèrent de la maison du roi à la maison de Yahweh, et s'assirent à l'entrée de la porte neuve de la maison de Yahweh.
\VS{11}Et les sacrificateurs et les prophètes parlèrent aux chefs et à tout le peuple, en disant : Cet homme mérite d'être condamné à la mort ; car il a prophétisé contre cette ville, comme vous l'avez entendu de vos oreilles.
\VS{12}Et Jérémie parla à tous les chefs et à tout le peuple, en disant : Yahweh m'a envoyé pour prophétiser contre cette maison et contre cette ville toutes les paroles que vous avez entendues.
\VS{13}Maintenant donc, amendez votre conduite et vos actions, écoutez la voix de Yahweh, votre Dieu, et Yahweh se repentira du mal qu'il a prononcé contre vous.
\VS{14}Pour moi, me voici entre vos mains ; faites-moi ce qui vous semblera bon et juste.
\VS{15}Mais sachez comme une chose certaine, que si vous me faites mourir, vous mettrez du sang innocent sur vous, sur cette ville et sur ses habitants ; car en vérité Yahweh m'a envoyé vers vous pour prononcer à vos oreilles toutes ces paroles.
\VS{16}Alors les chefs et tout le peuple dirent aux sacrificateurs et aux prophètes : Cet homme ne mérite pas d'être condamné à la mort, car il nous a parlé au nom de Yahweh, notre Dieu.
\VS{17}Et quelques-uns des anciens du pays se levèrent et parlèrent à toute l'assemblée du peuple, en disant :
\VS{18}Michée, de Moréscheth, prophétisait aux jours d'Ezéchias, roi de Juda, et il parlait à tout le peuple de Juda, en disant : Ainsi parle Yahweh des armées : Sion sera labourée comme un champ, Jérusalem sera réduite en un monceau de pierres, et la montagne du temple en des hauts lieux d'une forêt\FTNT{Mi. 1:1 ; Mi. 3:12.}.
\VS{19}Ezéchias, roi de Juda, et tous ceux de Juda l'ont-ils fait mourir ? Ezéchias ne craignit-il pas Yahweh ? N'implora-t-il pas Yahweh ? Et Yahweh se repentit du mal qu'il avait prononcé contre eux. Et nous, nous ferions donc un grand mal contre nos âmes\FTNT{2 Ch. 32:26.} !
\VS{20}Mais aussi dirent les autres, y eut aussi un homme qui prophétisait au nom de Yahweh, savoir, Urie, fils de Schemaeja, de Kirjath-Jearim. Il prophétisa contre cette même ville et contre ce même pays, de la même manière que Jérémie.
\VS{21}Et le roi Jojakim, tous ses vaillants hommes, et tous ses chefs entendirent ses paroles, et le roi chercha à le faire mourir. Urie, qui en fut informé, eut peur, prit la fuite, et se retira en Egypte.
\VS{22}Et le roi Jojakim envoya des hommes en Egypte, savoir, Elnathan, fils d'Acbor, et quelques hommes avec lui, qui allèrent en Egypte.
\VS{23}Et ils firent sortir d'Egypte Urie et l'amenèrent au roi Jojakim qui le frappa avec l'épée et jeta son cadavre sur les sépulcres des fils du peuple.
\VS{24}Toutefois la main d'Achikam, fils de Schaphan, fut avec Jérémie, et empêcha qu'il ne soit livré au peuple pour être mis à mort.
\Chap{27}
\TextTitle{Prophétie : Les nations seront asservies à Nebucadnetsar}
\VerseOne{}Au commencement du règne de Jojakim\FTNT{Il est probable que ce soit une erreur de copiste, car bien que l'hébreu dit « Jojakim », le contexte se rapporte à Sédécias. Voir Jé. 27:3 ; Jé. 27:12 ; Jé. 27:20 ; Jé 28:1.}, fils de Josias, roi de Juda, cette parole fut adressée à Jérémie de la part de Yahweh, en disant :
\VS{2}Ainsi m'a parlé Yahweh : Fais-toi des liens et des jougs, et mets-les sur ton cou\FTNT{Ez. 7:23.}.
\VS{3}Et envoie-les au roi d'Edom, et au roi de Moab, et au roi des fils d'Ammon, et au roi de Tyr et au roi de Sidon, par les mains des messagers qui sont venus à Jérusalem vers Sédécias, roi de Juda ;
\VS{4}et tu leur donneras mes ordres pour leurs maîtres, en disant : Ainsi parle Yahweh des armées, le Dieu d'Israël : Vous direz ainsi à vos maîtres :
\VS{5}J'ai fait la terre, les hommes et les bêtes qui sont sur la terre, par ma grande force et par mon bras étendu, et je la donne à qui cela me plaît\FTNT{De. 32:8.}.
\VS{6}Et maintenant j'ai livré tous ces pays entre les mains de Nebucadnetsar, roi de Babylone, mon serviteur ; et même je lui ai donné les bêtes des champs pour qu'elles lui soient asservies\FTNT{Da. 2:38.}.
\VS{7}Et toutes les nations lui seront asservies, à lui, à son fils, et au fils de son fils, jusqu'à ce que le temps de son pays vienne aussi, et que plusieurs nations et de grands rois l'asservissent.
\VS{8}Et il arrivera que la nation ou le royaume qui ne se soumettra pas à lui, à Nebucadnetsar, roi de Babylone, et qui ne soumettra pas son cou au joug du roi de Babylone, je punirai cette nation par l'épée, par la famine et par la peste, dit Yahweh, jusqu'à ce que je les aie consumés par sa main.
\VS{9}Vous donc, n'écoutez pas vos prophètes, ni vos devins, ni vos songeurs, ni vos augures, ni vos magiciens, qui vous parlent, en disant : Vous ne serez pas asservis au roi de Babylone.
\VS{10}Car ils vous prophétisent le mensonge pour vous faire aller loin de votre pays, afin que je vous chasse et que vous périssiez.
\VS{11}Mais la nation qui livrera son cou au joug du roi de Babylone, et qui le servira, je la laisserai dans son pays, dit Yahweh, pour qu'elle le cultive et qu'elle y demeure.
\VS{12}Puis j'ai parlé à Sédécias, roi de Juda, selon toutes ces paroles-là, en disant : Soumettez votre cou au joug du roi de Babylone, et rendez-vous sujets, à lui et son peuple, et vous vivrez.
\VS{13}Pourquoi mourriez-vous, toi et ton peuple, par l'épée, par la famine et par la peste, selon que Yahweh a parlé contre la nation qui ne sera pas soumise au roi de Babylone ?
\VS{14}N'écoutez donc pas les paroles des prophètes qui vous parlent en disant : Vous ne serez pas asservis au roi de Babylone ! Car ils vous prophétisent le mensonge.
\VS{15}Même je ne les ai pas envoyés, dit Yahweh, et ils vous prophétisent faussement en mon nom, afin que je vous rejette et que vous périssiez, vous et les prophètes qui vous prophétisent.
\VS{16}J'ai aussi parlé aussi aux sacrificateurs et à tout ce peuple, en disant : Ainsi parle Yahweh : N'écoutez pas les paroles de vos prophètes qui vous prophétisent, en disant : Voici, les ustensiles de la maison de Yahweh seront bientôt rapportés de Babylone ! Car ils vous prophétisent le mensonge.
\VS{17}Ne les écoutez donc pas, rendez-vous sujets au roi de Babylone, et vous vivrez. Pourquoi cette ville serait-elle réduite en un désert ?
\VS{18}Et s'ils sont prophètes et si la parole de Yahweh est en eux, qu'ils intercèdent maintenant auprès de Yahweh des armées, afin que les ustensiles qui restent dans la maison de Yahweh, dans la maison du roi de Juda, et dans Jérusalem, n'aillent pas à Babylone.
\VS{19}Car ainsi parle Yahweh des armées au sujet des colonnes, de la mer, des bases, et des autres ustensiles qui sont restés dans cette ville,
\VS{20}que Nebucadnetsar, roi de Babylone, n'a pas emportés, quand il a transporté de Jérusalem à Babylone, Jéconia, fils de Jojakim, roi de Juda, et tous les nobles de Juda et de Jérusalem,
\VS{21}Yahweh, dis-je, des armées le Dieu d'Israël, parle ainsi au sujet des ustensiles qui restent dans la maison de Yahweh, dans la maison du roi de Juda et dans Jérusalem :
\VS{22}Ils seront emportés à Babylone, et ils y demeureront jusqu'au jour où je les visiterai, dit Yahweh, et où je les ferai remonter et revenir dans ce lieu\FTNT{2 R. 24:14-15 ; Esd. 1:7-11 ; 2 Ch. 25:13-16 ; 2 Ch. 36:18.}.
\Chap{28}
\TextTitle{Hanania meurt suite à sa prophétie mensongère}
\VerseOne{}Il arriva aussi, en cette même année, au commencement du règne de Sédécias, roi de Juda, savoir, au cinquième mois de la quatrième année, que Hanania, fils d'Azzur, prophète de Gabaon, me parla dans la maison de Yahweh, aux yeux des sacrificateurs et de tout le peuple, en disant :
\VS{2}Ainsi parle Yahweh des armées, le Dieu d'Israël : Je romps le joug du roi de Babylone !
\VS{3}Dans deux années accomplis, et je ferai rapporter dans ce lieu tous les ustensiles de la maison de Yahweh, que Nebucadnetsar, roi de Babylone, a pris de ce lieu, et qu'il a transportés à Babylone.
\VS{4}Et je ferai revenir dans ce lieu, dit Yahweh, Jéconia, fils de Jojakim, roi de Juda, et tous les captifs de Juda qui sont allés à Babylone ; car je romprai le joug du roi de Babylone.
\VS{5}Alors Jérémie, le prophète, répondit à Hanania, le prophète, aux yeux des sacrificateurs, et aux yeux de tout le peuple qui se tenait dans la maison de Yahweh.
\VS{6}Et Jérémie, le prophète, dit : Ainsi soit-il ! Que Yahweh fasse ainsi ! Que Yahweh accomplisse les paroles que tu as prophétisées, et qu'il fasse revenir de Babylone dans ce lieu-ci les ustensiles de la maison de Yahweh, et tous les captifs de Babylone !
\VS{7}Toutefois, écoute maintenant cette parole que je prononce, à tes oreilles et aux oreilles de tout le peuple :
\VS{8}Les prophètes qui ont été avant moi et avant toi, dès les temps anciens, ont prophétisé contre plusieurs pays et de grands royaumes, la guerre, le malheur et la peste ;
\VS{9}Le prophète qui aura prophétisé la paix, quand la parole de ce prophète sera accomplie, ce prophète-là sera reconnu pour avoir été véritablement envoyé par Yahweh.
\VS{10}Alors Hanania, le prophète, prit le joug de dessus le cou de Jérémie, le prophète, et le rompit.
\VS{11}Puis Hanania parla aux yeux de tout le peuple, en disant : Ainsi parle Yahweh : C'est ainsi que dans deux années, je romprai le joug de Nebucadnetsar, roi de Babylone, de dessus le cou de toutes les nations. Et Jérémie, le prophète, alla au loin par la route.
\VS{12}Mais la parole de Yahweh fut adressée à Jérémie, après que Hanania, le prophète, eut rompu le joug de dessus le cou de Jérémie, le prophète, en disant :
\VS{13}Va, et parle à Hanania, en disant : Ainsi parle Yahweh : Tu as rompu les jougs de bois, et tu auras à la place un joug de fer.
\VS{14}Car ainsi parle Yahweh des armées, le Dieu d'Israël : Je mets un joug de fer sur le cou de toutes ces nations, afin qu'elles servent Nebucadnetsar, roi de Babylone, et elles le serviront ; et je lui donne aussi les bêtes des champs\FTNT{De. 28:48.}.
\VS{15}Puis Jérémie, le prophète, dit à Hanania, le prophète : Ecoute maintenant, ô Hanania ! Yahweh ne t'a pas envoyé, et tu as fait que ce peuple se confie au mensonge\FTNT{Ez. 13:3-9.}.
\VS{16}C'est pourquoi ainsi parle Yahweh : Voici, je te chasse de la face de la terre ; et tu mourras cette année ; car tu as parlé de révolte contre Yahweh.
\VS{17}Et Hanania, le prophète, mourut cette année-là, dans le septième mois.
\Chap{29}
\TextTitle{Message à l'attention des Juifs captifs à Babylone}
\VerseOne{}Or ce sont ici les paroles de la lettre que Jérémie, le prophète, envoya de Jérusalem au reste des anciens en captivité, aux sacrificateurs et aux prophètes, et à tout le peuple, que Nebucadnetsar avait transportés de Jérusalem à Babylone,
\VS{2}après que le roi Jéconia fut sorti de Jérusalem, avec la reine, et les eunuques, et les chefs de Juda et de Jérusalem, et les charpentiers et les serruriers\FTNT{2 R. 24:12.}.
\VS{3}C'est par la main d'Eleasa, fils de Schaphan, et Guemaria, fils de Hilkija, que Sédécias, roi de Juda, l'envoya à Babylone vers Nebucadnetsar, roi de Babylone. La lettre disait :
\VS{4}Ainsi parle Yahweh des armées, le Dieu d'Israël, à tous les captifs que j'ai fait transporter de Jérusalem à Babylone.
\VS{5}Bâtissez des maisons, et habitez-les ; plantez des jardins, et mangez-en les fruits.
\VS{6}Prenez des femmes, et engendrez des fils et des filles ; prenez aussi des femmes pour vos fils, et donnez vos filles à des hommes, afin qu'elles enfantent des fils et des filles ; multipliez-vous là, et ne diminuez pas.
\VS{7}Et cherchez la paix de la ville où je vous ai transporté, et priez Yahweh pour elle ; parce que dans sa paix vous aurez la paix.
\VS{8}Car ainsi parle Yahweh des armées, le Dieu d'Israël : Que vos prophètes qui sont au milieu de vous, et vos devins, ne vous séduisent pas, et n'écoutez pas vos songes que vous vous songez\FTNT{Tous les songes ne viennent pas toujours du Seigneur. Les visions et les songes doivent être en accord avec la Parole de Dieu.}.
\VS{9}Parce qu'ils vous prophétisent faussement en mon nom. Je ne les ai pas envoyés, dit Yahweh.
\VS{10}Car ainsi parle Yahweh : Lorsque les soixante-dix ans seront accomplis pour Babylone, je vous visiterai, et j'accomplirai ma bonne parole à votre égard, pour vous faire revenir dans ce lieu.
\VS{11}Car je sais que les pensées que j'ai pour vous, dit Yahweh, sont des pensées de paix et non pas d'adversité, pour vous donner une fin telle que vous espérez\FTNT{Jos. 1:8.}.
\VS{12}Alors vous m'invoquerez, et vous partirez ; vous me prierez, et je vous exaucerai\FTNT{Os. 5:15.}.
\VS{13}Vous me chercherez, et vous me trouverez, après que vous m'aurez recherché de tout votre cœur\FTNT{Mt. 7:7.}.
\VS{14}Car je me laisserai trouver par vous, dit Yahweh, je ramènerai vos captifs ; et je vous rassemblerai d'entre toutes les nations et de tous les lieux où je vous ai chassés, dit Yahweh, et je vous ramènerai dans le lieu d'où je vous ai transportés.
\VS{15}Cependant si vous dites : Yahweh nous a suscité des prophètes à Babylone !
\VS{16} A cause de cela, ainsi parle Yahweh sur le roi qui est assis sur le trône de David, sur tout le peuple qui habite dans cette ville, sur vos frères qui ne sont pas allés avec vous en captivité ;
\VS{17}ainsi parle Yahweh des armées : Voici, je vais envoyer sur eux l'épée, la famine, et la peste, et je les ferai devenir comme des figues affreuses qui ne peuvent être mangées à cause de leur mauvaise qualité.
\VS{18}Et je les poursuivrai par l'épée, par la famine et par la peste, je les abandonnerai pour être agités par tous les royaumes de la terre, et pour être une malédiction, un étonnement, une moquerie et un opprobre parmi toutes les nations où je les chasserai\FTNT{De. 28:25-37.},
\VS{19}parce qu'ils n'ont pas écouté mes paroles, dit Yahweh, eux à qui j'ai envoyé mes serviteurs, les prophètes, en me levant dès le matin ; et ils n'ont pas écouté, dit Yahweh.
\VS{20}Vous tous donc, écoutez la parole de Yahweh, vous les captifs que j'ai envoyés de Jérusalem à Babylone !
\VS{21}Ainsi parle Yahweh des armées, le Dieu d'Israël sur Achab, fils de Kolaja, et sur Sédécias, fils de Maaséja, qui vous prophétisent faussement en mon nom : Voici, je vais les livrer entre les mains de Nebucadnetsar, roi de Babylone ; et il les frappera sous vos yeux.
\VS{22}Et on se servira d'eux comme une formule de malédiction, parmi tous les captifs de Juda qui sont à Babylone, en disant : Que Yahweh te mette dans un tel état, comme Sédécias et comme Achab, que le roi de Babylone a fait rôtir au feu !
\VS{23}parce qu'ils ont commis des impuretés en Israël, parce qu'ils ont commis l'adultère avec les femmes de leurs prochains, et qu'ils ont dit en mon nom des paroles fausses, alors que je ne leur avais pas commandées. Je le sais, et j'en suis témoin, dit Yahweh.
\VS{24}Parle aussi à Schemaeja, Néchélamite, en disant :
\VS{25}Ainsi parle Yahweh des armées, le Dieu d'Israël : Tu as envoyé en ton nom une lettre à tout le peuple de Jérusalem, à Sophonie, fils de Maaséja, le sacrificateur, et à tous les sacrificateurs, en disant :
\VS{26}Yahweh t'a établi sacrificateur à la place de Jehojada, le sacrificateur, afin qu'il y ait dans la maison de Yahweh des inspecteurs pour surveiller tout homme qui est fou et se donne pour prophète, et afin que tu le mettes en prison et dans les fers.
\VS{27}Et maintenant, pourquoi n'as-tu pas réprimé pas Jérémie d'Anathoth, qui prophétise parmi vous,
\VS{28}car à cause de cela il nous a envoyé dire à Babylone : La captivité sera longue ; bâtissez des maisons, et habitez-les ; plantez des jardins, et mangez-en les fruits !
\VS{29}Or Sophonie, le sacrificateur, lut cette lettre aux oreilles de Jérémie, le prophète.
\VS{30}C'est pourquoi la parole de Yahweh fut adressée à Jérémie, en disant :
\VS{31}Envoie dire à tous les captifs : Ainsi parle Yahweh sur Schemaeja, Néchélamite : Parce que Schemaeja vous prophétise, quoique que je ne l'aie envoyé, et qu'il vous a fait vous confier dans le mensonge,
\VS{32}à cause de cela, dit Yahweh : Je vais punir Schemaeja, Néchélamite, et sa postérité ; il n'y aura personne de sa race qui  habite au milieu de ce peuple, et il ne verra pas le bien que je ferai à mon peuple, dit Yahweh ; car il a parlé de révolte contre Yahweh.
\Chap{30}
\TextTitle{Le jour de Yahweh}
\VerseOne{}La parole qui fut adressée à Jérémie de la part de Yahweh, en disant :
\VS{2}Ainsi parle Yahweh, le Dieu d'Israël : Ecris pour toi dans un livre toutes les paroles que je t'ai dites.
\VS{3}Car voici, les jours viennent, dit Yahweh, où je ramènerai les captifs de mon peuple d'Israël et de Juda, dit Yahweh ; je les ramènerai dans le pays que j'ai donné à leurs pères, et ils le posséderont.
\VS{4}Ce sont ici les paroles que Yahweh a prononcées sur Israël et Juda.
\VS{5}Ainsi parle Yahweh : Nous entendons des cris d'effroi et de terreur, il n'y a pas de paix.
\VS{6}Informez-vous, je vous prie et voyez si un mâle enfante ! Pourquoi vois-je les hommes les mains sur leurs reins, comme une femme qui enfante ? Pourquoi tous les visages sont-ils devenus pâles ?
\VS{7}Malheur ! Que ce jour est grand ; il n'y en a pas eu de semblable. Il sera un temps de détresse pour Jacob ; mais il en sera pourtant délivré\FTNT{Joë. 2:11 ; So. 1:15 ; Da. 12:1 ; Mt. 24:21.}.
\VS{8}Et il arrivera en ce jour-là, dit Yahweh des armées, que je briserai son joug de dessus ton cou, je romprai tes liens, et les étrangers ne t'asserviront plus.
\VS{9}Mais ils serviront Yahweh, leur Dieu, et David, leur roi, que je leur susciterai\FTNT{Ez. 34:23-24.}.
\VS{10}Toi donc, mon serviteur Jacob, ne crains pas, dit Yahweh, et ne t'épouvante pas, ô Israël ! Car, voici, je te délivrerai du pays éloigné, et ta postérité du pays de leur captivité ; et Jacob reviendra, il sera en repos et sera en paix, et il n'y aura personne qui lui fasse peur\FTNT{Es. 41:13.}.
\VS{11}Car je suis avec toi, dit Yahweh, pour te délivrer ; et même je consumerai entièrement toutes les nations parmi lesquelles je t'ai dispersé, mais quand à toi, je ne te consumerai pas entièrement; je te châtierai avec équité, je ne te tiens pas entièrement pour innocent\FTNT{Es. 27:7-8.}.
\VS{12}Ainsi parle Yahweh : Ta blessure est incurable, ta plaie est très douloureuse\FTNT{Mi. 1:9 ; 2 Ch. 36:16.}.
\VS{13}Il n'y a personne qui défende ta cause, pour panser ta plaie ; il n'y a pour toi aucun remède, aucun moyen de guérison.
\VS{14}Tous tes amoureux t'oublient, ils ne te recherchent pas ; car je t'ai frappée d'une plaie d'ennemi, d'un châtiment d'homme cruel, à cause de la multitude de tes iniquités, tes péchés se sont renforcés\FTNT{La. 1:2.}.
\VS{15}Pourquoi cries-tu à cause de ta plaie ? Ta douleur est hors d'espérance ; je t'ai fait ces choses à cause de la grandeur de tes iniquités, du grand nombre de tes péchés.
\VS{16}Néanmoins tous ceux qui te dévorent seront dévorés, et tous ceux qui te mettent dans la détresse, iront en captivité ; ceux qui te dépouillent seront dépouillés, et je livrerai au pillage tous ceux qui te pillent\FTNT{Es. 41:11 ; Ab. 15.}.
\VS{17}Même je guérirai tes plaies, et je te guérirai tes blessures, dit Yahweh. Car ils t'appellent la repoussée, cette Sion que personne ne recherche.
\TextTitle{Israël délivré par Yahweh}
\VS{18}Ainsi parle Yahweh : Voici, je ramène les captifs des tentes de Jacob, j'ai compassion de ses demeures ; la ville sera rebâtie sur le monceau de ses ruines et le palais sera rétabli comme il était.
\VS{19}Et il en sortira des remerciements et des cris de joie ; et je les multiplierai, et ils ne diminueront pas ; je les honorerai, et ils ne seront pas amoindris.
\VS{20}Et ses enfants seront comme autrefois, son assemblée sera affermie devant moi, et je punirai tous ceux qui l'oppriment.
\VS{21}Et son chef sera tiré de son sein, son dominateur sortira du milieu de lui ; je le ferai approcher, et il viendra vers moi ; car qui disposerait son cœur pour venir vers moi ? dit Yahweh.
\VS{22}Et vous serez mon peuple, et je serai votre Dieu.
\VS{23}Voici, la tempête de Yahweh, la fureur éclate, un tourbillon qui s'entasse ; il tombera sur la tête des méchants. 
\VS{24}L'ardeur de la colère de Yahweh ne se détournera pas, jusqu'à ce qu'il ait exécuté, accompli les desseins de son cœur ; vous le comprendrez dans les derniers jours.
\Chap{31}
\TextTitle{Communion retrouvée : La paix et et la joie}
\VerseOne{}En ce temps-là, dit Yahweh, je serai le Dieu de toutes les familles d'Israël, et ils seront mon peuple.
\VS{2}Ainsi parle Yahweh : Le peuple survivant à l'épée a trouvé grâce dans le désert ; Israël marche vers son lieu de repos.
\VS{3}De loin Yahweh m'est apparu, et m'a dit : Je t'aime d'un amour éternel, c'est pourquoi j'ai prolongé ma bonté envers toi.
\VS{4}Je te rétablirai encore, et tu seras rétablie, ô vierge d'Israël ! Tu te pareras encore de tes tambours, et tu sortiras au milieu des danses joyeuses.
\VS{5}Tu planteras encore des vignes sur les montagnes de Samarie ; les vignerons planteront et recueilleront les fruits pour leur usage\FTNT{Es. 65:21.}.
\VS{6}Car il y a un jour où les gardes crieront sur la montagne d'Ephraïm : Levez-vous, et montons à Sion, vers Yahweh, notre Dieu !
\VS{7}Car ainsi parle Yahweh : Réjouissez-vous avec chant de triomphe, et avec allégresse à cause de Jacob, et vous égayez à cause du chef des nations ! Faites-le entendre, chantez des louanges, et dites : Yahweh, délivre ton peuple, le reste d’Israël !
\VS{8}Voici, je vais les faire venir du pays du nord, et je les rassemblerai des extrémités de la terre ; l'aveugle et le boiteux, la femme enceinte et celle qui enfante seront ensemble parmi eux ; une grande assemblée qui reviendra ici.
\VS{9}Ils y seront allés en pleurant, mais je les ferai retourner avec des supplications, et je les conduirai aux torrents d'eaux, et par un droit chemin, où ils ne broncheront pas ; car je suis un père pour Israël, et Ephraïm est mon premier-né\FTNT{Ex. 4:22.}.
\VS{10}Nations, écoutez la parole de Yahweh, et annoncez-la aux îles éloignées ! Dites : Celui qui a dispersé Israël le rassemblera, et il le gardera comme un berger garde son troupeau.
\VS{11}Car Yahweh rachète Jacob, et le retire de la main d'un ennemi plus fort que lui.
\VS{12}Ils viendront donc, et se réjouiront avec des chants de triomphe sur les hauteurs de Sion ; ils afflueront vers les biens de Yahweh, le blé, le vin, l'huile, et le fruit du gros et du menu bétail ; et leur âme sera comme un jardin arrosé, et ils ne seront plus dans la souffrance\FTNT{Es. 61:11.}.
\VS{13}Alors la vierge se réjouira à la danse, les jeunes hommes et les anciens ensemble ; je changerai leur deuil en joie, et je les consolerai ; et je les réjouirai en les délivrant de leur douleur.
\VS{14}Je rassasierai aussi de graisse l'âme des sacrificateurs, et mon peuple sera rassasié de mes biens, dit Yahweh.
\VS{15}Ainsi parle Yahweh : On entend des cris à Rama, des lamentations, des larmes amères ; Rachel pleure ses fils ; elle refuse d'être consolée sur ses fils, car ils ne sont plus\FTNT{Mt. 2:17-18.}.
\VS{16}Ainsi parle Yahweh : Retiens ta voix de pleurer, et tes yeux de verser des larmes, car ton œuvre aura son salaire, dit Yahweh ; et ils reviendront des terres de l'ennemi.
\VS{17}Et il y a de l'espérance pour tes derniers jours, dit Yahweh ; et tes fils reviendront dans leur territoire.
\VS{18}J'ai très bien entendu Ephraïm se plaignant, et disant : Tu m'as châtié, et j'ai été châtié comme un veau qui n'est pas dompté. Fais-moi revenir, et je reviendrai, car tu es Yahweh, mon Dieu\FTNT{Ps. 119:67-71.}.
\VS{19}Certes, après m'être détourné, je me repens ; et après avoir reconnu mes fautes, je frappe sur ma cuisse ; je suis honteux et confus, car je porte l'opprobre de ma jeunesse\FTNT{Ez. 21:17.}.
\VS{20}Ephraïm est-il donc pour moi un cher fils, un fils qui fait mes délices ? Car plus je parle de lui, plus encore son souvenir est en moi ; aussi mes entrailles sont émues en sa faveur : J'aurai certainement pitié de lui, dit Yahweh\FTNT{Es. 5:7.}.
\VS{21}Dresse-toi des signes sur les chemins, place des poteaux, prends garde à la route, au chemin par lequel tu es venue… Reviens, vierge d'Israël, reviens dans tes villes !
\VS{22}Jusqu'à quand seras-tu errante, fille rebelle ? Car Yahweh crée une chose nouvelle sur la terre : La femme entourera l'homme.
\VS{23}Ainsi parle Yahweh des armées, le Dieu d'Israël : On dira encore cette parole-ci dans le pays de Juda et dans ses villes, quand j'aurai ramené leurs captifs : Que Yahweh te bénisse, ô agréable demeure de la justice, montagne de sainteté !
\VS{24}Juda et toutes ses villes ensemble, les laboureurs, et ceux qui conduisent les troupeaux, y habiteront.
\VS{25}Car j'abreuverai l'âme épuisée par le travail, et je remplirai toute âme languissante.
\VS{26}C'est pourquoi je me suis réveillé, et j'ai regardé ; mon sommeil m'avait été agréable.
\TextTitle{Promesse d'une nouvelle alliance}
\VS{27}Voici, les jours viennent, dit Yahweh, où j'ensemencerai la maison d'Israël et la maison de Juda d'une semence d'hommes et d'une semence de bêtes.
\VS{28}Et il arrivera que comme j'ai veillé sur eux pour arracher et démolir, pour détruire, pour perdre et pour faire du mal ; ainsi je veillerai sur eux pour bâtir et pour planter, dit Yahweh.
\VS{29}En ces jours-là, on ne dira plus : Les pères ont mangé des raisins verts, et les dents des fils en ont été agacées\FTNT{Ez. 18:2-3.}.
\VS{30}Mais chacun mourra pour son iniquité ; tout homme qui mangera des raisins verts, ses dents en seront agacées.
\VS{31}Voici, les jours viennent, dit Yahweh, où je traiterai une nouvelle alliance\FTNT{Il s'agit de l'alliance du sang que Jésus, notre Messie, est venu inaugurer en prenant sur lui tous nos péchés et en mourant sur la croix à notre place (Mt. 26:27-29 ; Hé. 8:7-13).} avec la maison d'Israël et avec la maison de Juda,
\VS{32}non comme l'alliance que je traitai avec leurs pères, le jour où je les pris par la main, pour les faire sortir du pays d'Egypte, mon alliance qu'ils ont violée ; et toutefois j’avais été pour eux un mari, dit Yahweh.
\VS{33}Car c'est ici l'alliance que je traiterai avec la maison d'Israël, après ces jours-là, dit Yahweh, je mettrai ma loi au-dedans d'eux, je l'écrirai dans leur cœur ; et je serai leur Dieu, et ils seront mon peuple.
\VS{34}Aucun homme parmi eux n'enseignera plus son prochain, ni personne son frère, en disant : Connaissez Yahweh ! Car tous me connaîtront, depuis le plus petit jusqu'au plus grand, dit Yahweh ; parce que je pardonnerai leur iniquité, et que je ne me souviendrai plus de leur péché\FTNT{Es. 54:13 ; Ha. 2:14 ; Jn. 6:45.}.
\VS{35}Ainsi parle Yahweh, qui a donné le soleil pour être la lumière du jour, et qui a réglé la lune et les étoiles pour être la lumière de la nuit, qui remue la mer, et fait gronder ses flots, lui dont le nom est Yahweh des armées\FTNT{Ge. 1:16 ; Es. 51:15.} :
\VS{36}Si ces lois\FTNT{Les lois de l'univers ont été établies par Yahweh. Ces lois sont : la loi de la gravité, la loi de l'attraction et la loi de la résonance (voir Ps. 148:5-6 ; Job. 38:33).} viennent à cesser devant moi, dit Yahweh, la race d'Israël aussi cessera d'être à jamais une nation devant moi.
\VS{37}Ainsi parle Yahweh : Si les cieux en haut peuvent être mesurés, si les fondements de la terre en bas peuvent être sondés, alors je rejetterai toute la race d'Israël, à cause de toutes les choses qu'ils ont faites, dit Yahweh.
\VS{38}Voici, les jours viennent, dit Yahweh, où cette ville sera rebâtie à Yahweh, depuis la tour de Hananeel, jusqu'à la porte de l'angle\FTNT{Za. 14:10 ; Né. 3:1 ; 2 Ch. 26:9.}.
\VS{39}Le cordeau à mesurer sera encore tiré vis-à-vis d'elle, sur la colline de Gareb, et tournera vers Goath.
\VS{40}Et toute la vallée des cadavres et des cendres, et tous les champs jusqu'au torrent de Cédron, jusqu'à l'angle de la porte des chevaux à l'orient, seront consacrés à Yahweh, et ne seront plus jamais arrachés ni détruits.
\Chap{32}
\TextTitle{Le champ de Hanameel : La pérennité d'Israël}
\VerseOne{}La parole qui fut adressée à Jérémie de la part de Yahweh, la dixième année de Sédécias, roi de Juda. C'était la dix-huitième année de Nebucadnetsar.
\VS{2}Or l'armée du roi de Babylone assiégeait alors Jérusalem ; et Jérémie le prophète était enfermé dans la cour de la prison, qui était dans la maison du roi de Juda ;
\VS{3}car Sédécias, roi de Juda, l'avait fait enfermer, et lui avait dit : Pourquoi prophétises-tu, en disant : Ainsi parle Yahweh : Voici, je vais livrer cette ville entre les mains du roi de Babylone, et il la prendra ;
\VS{4}et Sédécias, roi de Juda, n'échappera pas aux mains des Chaldéens ; mais il sera livré entre les mains du roi de Babylone, et lui parlera bouche à bouche, et ses yeux verront les yeux de ce roi ;
\VS{5}il emmènera Sédécias à Babylone, qui y demeurera jusqu'à ce que je le visite, dit Yahweh ; si vous combattez contre les Chaldéens, vous ne prospérerez pas.
\VS{6}Jérémie donc dit : La parole de Yahweh m'a été adressée, en disant :
\VS{7}Voici Hanameel, fils de Schallum, ton oncle, qui vient vers toi pour te dire : Achète mon champ qui est à Anathoth, car tu as le droit de rachat pour l'acquérir\FTNT{Lé. 25:48 ; Ru. 3:12.}.
\VS{8}Hanameel donc, fils de mon oncle, vint à moi, selon la parole de Yahweh, dans la cour de la prison, et me dit : Achète, je te prie, mon champ, qui est à Anathoth, dans le pays de Benjamin, car tu as le droit d'héritage et de rachat, achète-le ! Et je connus alors que c'était la parole de Yahweh.
\VS{9}Ainsi j'achetai de Hanameel, fils de mon oncle, le champ qui est à Anathoth, et je lui pesai l'argent, qui fut dix-sept sicles d'argent.
\VS{10}Puis j'écrivis le contrat, que je cachetai, je pris des témoins après avoir pesé l'argent sur la balance.
\VS{11}Et je pris le contrat d'acquisition, celui qui était cacheté, selon les ordonnances et les statuts, et celui qui était ouvert ;
\VS{12}Et je remis le contrat d'acquisition à Baruc, fils de Nérija, fils de Machséja, sous les yeux de Hanameel, fils de mon oncle, des témoins qui avaient signé le contrat d'acquisition, et sous les yeux de tous les juifs qui étaient assis dans la cour de la prison.
\VS{13}Puis je donnai sous leurs yeux cet ordre à Baruc, en disant :
\VS{14}Ainsi parle Yahweh des armées, le Dieu d'Israël : Prends ces contrats-ci, à savoir, ce contrat d'acquisition, celui qui est scellé, et celui qui est ouvert, et mets-les dans un vase de terre, afin qu'ils se conservent longtemps.
\VS{15}Car ainsi parle Yahweh des armées, le Dieu d'Israël : On achètera encore des maisons, des champs et des vignes, dans ce pays.
\TextTitle{Promesse du retour des Juifs en Israël}
\VS{16}Et après que j'eus donné à Baruc, fils de Nérija, le contrat d'acquisition, je fis cette prière à Yahweh, en disant :
\VS{17}Ah ! Ah ! Seigneur Yahweh, voici, tu as fait les cieux et la terre par ta grande puissance et par ton bras étendu : Aucune chose n'est étonnante de ta part.
\VS{18}Tu fais miséricorde jusqu'à la millième génération, et tu punis l'iniquité des pères dans le sein de leurs fils après eux\FTNT{Ex. 34:7 ; Es. 65:7 ; Ps. 79:12.}. Tu es le Dieu, le Grand, le Puissant, dont le nom est Yahweh des armées.
\VS{19}Tu es grand en conseil et puissant en actions ; tes yeux sont ouverts sur toutes les voies des fils des hommes, pour rendre à chacun selon ses voies, et selon le fruit de ses œuvres.
\VS{20}Tu as fait dans le pays d'Egypte des miracles et des prodiges qui sont connus jusqu'à ce jour, et en Israël et parmi les hommes, tu t'es fait un nom tel qu'il est aujourd'hui.
\VS{21}Car tu as fait sortir du pays d'Egypte ton peuple d'Israël, avec des miracles et des prodiges, et avec une main forte, et avec un bras étendu, et en répandant partout une grande terreur ;
\VS{22}Et tu leur as donné ce pays, que tu avais juré à leurs pères de leur donner, pays où coulent le lait et le miel.
\VS{23}Et ils y sont entrés, ils l'ont possédé ; mais ils n'ont pas obéi à ta voix, et n'ont pas marché dans ta loi, et n'ont pas fait tout ce que tu leur avais ordonné de faire. C'est pourquoi tu as fait arriver sur eux tout ce mal-ci !
\VS{24}Voilà, les terrasses sont élevées, on est venu contre la ville pour la prendre, et à cause de l’épée, de la famine, et de la peste, la ville est livrée entre la main des Chaldéens qui combattent contre elle ; et ce que tu as dit est arrivé, et voici, tu le vois.
\VS{25}Et cependant, Seigneur Yahweh ! Tu m'as dit : Achète-toi ce champ à prix d'argent, et prends-en des témoins, quoique la ville soit livrée entre les mains des Chaldéens.
\VS{26}Mais la parole de Yahweh fut adressée à Jérémie, en disant :
\VS{27}Voici, je suis Yahweh, le Dieu de toute chair. Y a-t-il quelque chose d'étonnant de ma part ?
\VS{28}C'est pourquoi ainsi parle Yahweh : Voici, je vais livrer cette ville entre les mains des Chaldéens, et entre les mains de Nebucadnetsar, roi de Babylone, qui la prendra.
\VS{29}Et les Chaldéens qui combattent contre cette ville, y entreront, et mettront le feu à cette ville, et la brûleront, avec les maisons sur les toits desquelles on a brûlé de l'encens à Baal, et où l'on a fait des libations à d'autres dieux pour m'irriter.
\VS{30}Car les fils d'Israël et les fils de Juda n'ont fait, dès leur jeunesse, que ce qui est mal à mes yeux ; les fils d'Israël n'ont fait que m'irriter par les œuvres de leurs mains, dit Yahweh.
\VS{31}Car cette ville a été portée à provoquer ma colère et ma fureur, depuis le jour qu'ils l'ont bâtie, jusqu'à ce jour, afin que je l'ôte de devant ma face ;
\VS{32}à cause de tout le mal que les fils d'Israël et les fils de Juda ont fait pour m'irriter, eux, leurs rois, leurs chefs, leurs sacrificateurs et leurs prophètes, les hommes de Juda et les habitants de Jérusalem.
\VS{33}Ils m'ont tourné le dos, et non la face ; je les ai enseignés, je les ai enseignés dès le matin, mais ils n'ont pas écouté pour recevoir l'instruction.
\VS{34}Mais ils ont mis leurs abominations dans la maison sur laquelle mon Nom est invoqué, pour la souiller.
\VS{35}Et ils ont bâti les hauts lieux de Baal, qui sont dans la vallée de Ben-Hinnom, pour faire passer par le feu leurs fils et leurs filles à Moloc\FTNT{Voir commentaire en Lé. 20:2.} ; ce que je ne leur avais pas ordonné, et il ne m'était pas monté à la pensée qu'ils feraient cette abomination pour faire pécher Juda.
\VS{36}Et maintenant, à cause de cela Yahweh, le Dieu d'Israël, ainsi parle sur cette ville dont vous dites qu’elle est livrée entre les mains du roi de Babylone, à cause que l’épée, la famine, et la peste sont en elle :
\VS{37}Voici, je vais les rassembler de tous les pays où je les ai chassés, dans ma colère, dans ma fureur et dans mon grand courroux ; et je les ramènerai dans ce lieu-ci, et je les y ferai habiter en sécurité.
\VS{38}Et Ils seront mon peuple, et je serai leur Dieu.
\VS{39}Et je leur donnerai un même cœur et une même voie, afin qu'ils me craignent à toujours, pour leur bien et celui de leurs fils après eux.
\VS{40}Et je traiterai avec eux une alliance éternelle, à savoir, que je ne me détournerai plus d'eux pour leur faire du bien ; et je mettrai ma crainte dans leur cœur, afin qu'ils ne se détournent pas de moi\FTNT{Es. 54:10.}.
\VS{41}Et je me réjouirai à leur faire du bien, et je les planterai dans ce pays-ci solidement, de tout mon cœur, et de toute mon âme.
\VS{42}Car ainsi parle Yahweh : Comme j'ai fait venir tous ce grand mal sur ce peuple, ainsi je ferai venir sur eux tout le bien que je prononce en leur faveur.
\VS{43}Et on achètera des champs dans ce pays, duquel vous dites que ce n'est que désolation, sans hommes ni bêtes, et qui est livré entre les mains des Chaldéens.
\VS{44}On achètera, dis-je, des champs à prix d'argent, et on en écrira les contrats, et on les cachettera, et on en prendra des témoins dans le pays de Benjamin, et aux environs de Jérusalem, dans les villes de Juda, tant dans les villes des montagnes, que dans les villes de la plaine, et dans les villes du midi. Car je ramènerai leurs captifs, dit Yahweh.
\Chap{33}
\TextTitle{Jésus, le Germe appelé à régner\FTNTT{Voir 2 S. 7:8-16.}}
\VerseOne{}Et la parole de Yahweh fut adressée une seconde fois à Jérémie, quand il était encore enfermé dans la cour de la prison, en disant :
\VS{2}Ainsi parle Yahweh, qui fait ces choses, Yahweh qui les forme et les établit, lui dont le nom est Yahweh :
\VS{3}Crie vers moi\FTNT{Yahweh, qui demandait qu'on l'invoque, n'est autre que Jésus-Christ, notre Seigneur (Joë. 2:32 ; 1 Co. 1:2 ; Ro. 10:13).}, je te répondrai, et je t'annoncerai des choses grandes, des choses cachées, que tu ne connais pas.
\VS{4}Car ainsi parle Yahweh, le Dieu d'Israël, touchant les maisons de cette ville-ci et les maisons des rois de Juda ; elles seront abattues par les terrasses et par l'épée.
\VS{5}Ils sont venus pour combattre contre les Chaldéens, mais ça été pour remplir leurs maisons des cadavres des hommes que j'ai frappé dans ma colère et dans ma fureur, et parce que j'ai caché ma face de cette ville à cause toute leur méchanceté.
\VS{6}Voici, je vais lui donner la santé et la guérison, je les guérirai, et je leur ferai découvrir une abondance de paix et de fidélité\FTNT{Ap. 22:1-2.}.
\VS{7}Et je ramènerai les captifs de Juda, et les captifs d'Israël, et je les rétablirai comme autrefois.
\VS{8}Et je les purifierai de toute leur iniquité, par laquelle ils ont péché contre moi ; et je pardonnerai toutes leurs iniquités par lesquelles ils ont péché contre moi, et par lesquelles ils se sont révoltés contre moi\FTNT{Ez. 37:23.}.
\VS{9}Et cette ville sera pour moi un sujet de joie, de louange et de gloire, parmi toutes les nations de la terre qui entendront parler de tout le bien que je leur ferai, et elles seront dans la crainte et trembleront à cause de tout le bien et de toute la prospérité que je vais lui donner.
\VS{10}Ainsi parle Yahweh : Dans ce lieu-ci duquel vous dites : Il est désert, il n'y a plus d'hommes, plus de bêtes, dans les villes de Juda, et dans les rues de Jérusalem, qui sont désolées, privées d'hommes, d'habitants, de bêtes,
\VS{11}on y entendra encore les cris de joie et les cris d'allégresse, la voix de l'époux et la voix de l'épouse, et la voix de ceux qui disent : Louez Yahweh des armées ; car Yahweh est bon, parce que sa miséricorde demeure à toujours, lorsqu'ils offriront des offrandes de reconnaissance dans la maison de Yahweh ; car je ramènerai les captifs de ce pays, et je les rétablirai comme autrefois, dit Yahweh.
\VS{12}Ainsi parle Yahweh des armées : Dans ce lieu désert, où il n'y a ni hommes ni bêtes, et dans toutes ses villes, il y aura encore des demeures de bergers qui y feront reposer leurs troupeaux ;
\VS{13}dans les villes des montagnes, et dans les villes de la plaine, dans les villes du midi, dans le pays de Benjamin et aux environs de Jérusalem, et dans les villes de Juda ; les brebis passeront encore sous les mains de celui qui les compte, dit Yahweh.
\VS{14}Voici, les jours viennent, dit Yahweh, où j'accomplirai la bonne parole que j'ai prononcée sur la maison d'Israël et la maison de Juda.
\VS{15}En ces jours et en ce temps-là, je ferai germer à David le Germe de justice, qui exercera le jugement et la justice dans le pays.
\VS{16}En ces jours-là, Juda sera sauvé, Jérusalem habitera en sécurité ; et voici comment on l'appellera : Yahweh notre justice.
\VS{17}Car ainsi parle Yahweh : David ne manquera jamais d'un successeur assis sur le trône de la maison d'Israël ;
\VS{18}et d'entre les sacrificateurs Lévites, il ne manquera jamais d'y avoir devant moi d'homme offrant des holocaustes, brûlant de l'encens avec les offrandes, et faisant des sacrifices tous les jours.
\VS{19}La parole de Yahweh fut encore adressée à Jérémie, en disant :
\VS{20}Ainsi parle Yahweh : Si vous pouvez rompre mon alliance avec le jour et mon alliance avec la nuit, de sorte que le jour et la nuit ne soient plus en leur temps,
\VS{21}alors aussi mon alliance avec David, mon serviteur, sera rompue ; de sorte qu'il n'aura plus de fils régnant sur son trône ; et avec les Lévites sacrificateurs, faisant mon service.
\VS{22}Car comme on ne peut compter l'armée des cieux, ni mesurer le sable de la mer, ainsi je multiplierai la postérité de David mon serviteur, et les Lévites qui font mon service\FTNT{Ge. 2:1 ; Ge. 15:5.}.
\VS{23}La parole de Yahweh fut encore adressée à Jérémie, en disant :
\VS{24}N'as-tu pas vu ce que ce peuple prononce, en disant : Yahweh a rejeté les deux familles qu'il avait élues ? Ainsi ils méprisent mon peuple, ils ne sont plus une nation devant eux.
\VS{25}Ainsi parle Yahweh : Si je n'ai pas fait mon alliance avec le jour et la nuit, et si je n'ai pas établi les ordonnances des cieux et de la terre ;
\VS{26}aussi rejetterai-je la postérité de Jacob, et celle de David mon serviteur, pour ne plus prendre de sa postérité des gens qui dominent sur les descendants d'Abraham, d'Isaac et de Jacob ; car je ramènerai leurs captifs, et j'aurai compassion d'eux.
\Chap{34}
\TextTitle{Désobéissance du peuple : Jérusalem dévastée}
\VerseOne{}La parole qui fut adressée à Jérémie de la part de Yahweh, lorsque Nebucadnetsar, roi de Babylone, et toute son armée, et tous les royaumes de la terre, et tous les peuples qui étaient sous la puissance de sa main, combattaient contre Jérusalem, et contre toutes ses villes, en disant\FTNT{2 R. 25:1-2.} :
\VS{2}Ainsi parle Yahweh, le Dieu d'Israël : Va, et parle à Sédécias, roi de Juda, et dis-lui : Ainsi parle Yahweh : Voici, je vais livrer cette ville entre les mains du roi de Babylone, et il la brûlera par le feu.
\VS{3}Et tu n'échapperas pas de sa main, car certainement tu seras pris et tu seras livré entre ses mains, et tes yeux verront les yeux du roi de Babylone, et il te parlera bouche à bouche, et tu iras à Babylone.
\VS{4}Toutefois écoute la parole de Yahweh, ô Sédécias, roi de Juda ! Ainsi parle Yahweh sur toi : Tu ne mourras pas par l'épée ;
\VS{5}mais tu mourras en paix, et on brûlera pour toi des parfums aromatiques, comme on en a brûlé pour tes pères, les rois précédents qui ont été avant toi ; et on te pleurera, en disant : Hélas, Seigneur ! Car j'ai prononcé cette parole, dit Yahweh\FTNT{2 Ch. 16:14.}.
\VS{6}Jérémie, le prophète, dit toutes ces paroles à Sédécias, roi de Juda, à Jérusalem.
\VS{7}Et l'armée du roi de Babylone combattait contre Jérusalem et contre toutes les villes de Juda qui restaient, à savoir, contre Lakis et contre Azéka, car c'étaient les seules villes fortifiées qui restaient parmi les villes de Juda\FTNT{2 R. 18:13.}.
\TextTitle{Jérusalem deviendra une désolation à cause de la désobéissance}
\VS{8}La parole fut adressée à Jérémie de la part de Yahweh, après que le roi Sédécias eut traité une alliance avec tout le peuple de Jérusalem, pour proclamer la liberté,
\VS{9}afin que chacun renvoie libre son esclave et chacun sa servante, l'hébreu ou la femme de l'hébreu, et qu'aucun juif ne soit l'esclave de son frère.
\VS{10}Tous les chefs et tout le peuple, qui étaient entrés dans cette alliance, entendirent que chacun devait renvoyer libre son serviteur et chacun sa servante, sans plus les asservir ; ils obéirent et les renvoyèrent.
\VS{11}Mais ensuite, ils changèrent d'avis ; ils firent revenir leurs esclaves et leurs servantes, qu'ils avaient renvoyés libres, et les assujettirent pour être leurs esclaves et leurs servantes.
\VS{12}Et la parole de Yahweh fut adressée à Jérémie en disant :
\VS{13}Ainsi parle Yahweh, le Dieu d'Israël : J'ai traité une alliance avec vos pères, le jour où je les ai fait sortir du pays d'Egypte, de la maison de servitude, en disant :
\VS{14}A la fin de la septième année, chacun renverra libre son frère hébreu qui aura été vendu ; il te servira six années, puis tu le renverras libre de chez toi. Mais vos pères ne m'ont pas écouté, ils n'ont pas prêté l'oreille\FTNT{Ex. 21:2 ; Lé. 25:10-15 ; De. 15:12.}.
\VS{15}Et vous, qui aujourd'hui étiez revenus à vous-mêmes, et vous aviez fait ce qui était droit à mes yeux, en publiant la liberté chacun pour son prochain, vous aviez traité une alliance devant moi, dans la maison sur laquelle mon Nom est invoqué.
\VS{16}Mais vous êtes revenus en arrière, et vous avez souillé mon Nom ; vous avez fait revenir chacun ses esclaves et ses servantes, que vous aviez renvoyés libres, rendus à eux-mêmes, et vous les avez assujettis, afin qu'ils soient pour vous des serviteurs et des servantes.
\VS{17}C'est pourquoi ainsi parle Yahweh : Vous ne m'avez pas obéi, en publiant la liberté chacun à son frère, et chacun à son prochain. Voici, je vais publier contre vous, dit Yahweh, la liberté contre vous à l'épée, à la peste, et à la famine ;  et je vous livrerai pour être transportés par tous les royaumes de la terre.
\VS{18}Et je livrerai les hommes qui ont transgressé mon alliance, et qui n'ont pas observé les paroles de l'alliance qu'ils avaient traitée devant moi, lorsqu'ils sont passés entre les morceaux du veau qu'ils ont coupé en deux ;
\VS{19}les chefs de Juda, et les chefs de Jérusalem, les eunuques, et les sacrificateurs, et tout le peuple du pays, qui sont passés au travers des morceaux du veau ;
\VS{20}je les livrerai, dis-je, entre les mains de leurs ennemis, entre les mains de ceux qui cherchent leur vie ; et leurs cadavres seront la pâture des oiseaux des cieux et des bêtes de la terre.
\VS{21}Je livrerai aussi Sédécias, roi de Juda, et les chefs de sa cour, entre les mains de leurs ennemis, entre les mains de ceux qui cherchent leur vie, entre les mains de l'armée du roi de Babylone, qui s'est retiré de devant vous.
\VS{22}Voici, je vais leur donner mes ordres, dit Yahweh, et je les ramènerai contre cette ville-ci ; et ils combattront contre elle, et la prendront, et la brûleront au feu ; et je ferai des villes de Juda un désert sans habitants.
\Chap{35}
\TextTitle{L'obéissance des Récabites}
\VerseOne{}C'est ici la parole qui fut adressée à Jérémie de la part de Yahweh, au temps de Jojakim, fils de Josias, roi de Juda, en disant :
\VS{2}Va à la maison des Récabites, et parle-leur, et fais les venir à la maison de Yahweh, dans l'une des chambres, et présente-leur du vin à boire\FTNT{2 S. 4:2 ; 1 Ch. 2:55.}.
\VS{3}Je pris donc Jaazania, fils de Jérémie, fils de Habazinia, et ses frères, et tous ses fils, et toute la maison des Récabites,
\VS{4}et je les fis venir dans la maison de Yahweh, dans la chambre des fils de Hanan, fils de Jigdalia, homme de Dieu, qui était près de la chambre des chefs, au-dessus de la chambre de Maaséja, fils de Schallum, garde du seuil.
\VS{5}Et je mis devant les fils de la maison des Récabites des coupes pleines de vin, et des calices, et je leur dis : Buvez du vin !
\VS{6}Et ils répondirent : Nous ne buvons pas de vin ; car Jonadab, fils de Récab, notre père, nous a donné cet ordre en disant : Vous ne boirez jamais de vin, ni vous ni vos fils\FTNT{Lé. 10:9 ; No. 6:2-4.} ;
\VS{7}vous ne bâtirez aucune maison, vous ne sèmerez aucune semence, vous ne planterez aucune vigne, et vous n'en aurez pas ; mais vous habiterez sous des tentes toute votre vie, afin que vous viviez longtemps sur la terre où vous êtes étrangers.
\VS{8}Nous avons donc obéi à la voix de Jonadab, fils de Récab, notre père dans toutes les choses qu'il nous a ordonnées, de sorte que nous n'avons pas bu de vin tous les jours de notre vie, ni nous, ni nos femmes, ni nos fils, ni nos filles.
\VS{9}Nous n'avons bâti aucune maison pour notre demeure, et nous n'avons eu ni vigne, ni champ, ni semence.
\VS{10}Mais nous avons habité sous des tentes, et nous avons obéi, et nous avons fait selon toutes les choses que Jonadab, notre père, nous a ordonnées.
\VS{11}Mais il est arrivé que quand Nebucadnetsar, roi de Babylone, est monté au pays, nous avons dit : Venez, et entrons dans Jérusalem, pour fuir de devant l'armée des Chaldéens, et de devant l'armée de Syrie. C'est ainsi que nous habitons à Jérusalem.
\VS{12}Alors la parole de Yahweh fut adressée à Jérémie, en disant :
\VS{13}Ainsi parle Yahweh des armées, le Dieu d'Israël : Va, et dis aux hommes de Juda, et aux habitants de Jérusalem : Ne recevrez-vous pas d'instruction pour obéir à mes paroles ? Dit Yahweh.
\VS{14}Toutes les paroles de Jonadab, fils de Récab, qui a ordonné à ses fils de ne pas boire de vin, ont été observées, et ils n'en ont pas bu jusqu'à ce jour ; mais ils ont obéi au commandement de leur père ; mais moi, je vous ai parlé, je vous ai parlé dès le matin, et vous ne m'avez pas obéi.
\VS{15}Car je vous ai envoyé tous les prophètes, mes serviteurs, je les ai envoyés dès le matin, pour vous dire : Revenez maintenant chacun de votre mauvaise voie, et amendez vos actions, et n'allez pas après d'autres dieux pour les servir, afin que vous demeureriez dans le pays que j'ai donné à vous et à vos pères. Mais vous n'avez pas prêté l'oreille, et vous ne m'avez pas écouté.
\VS{16}Parce que les fils de Jonadab, fils de Récab, ont observé le commandement que leur avait donné leur père, et que ce peuple ne m'écoute pas ;
\VS{17}à cause de cela, Yahweh le Dieu des armées, le Dieu d'Israël, parle ainsi : Voici, je vais faire venir sur Juda et sur tous les habitants de Jérusalem tout le mal que j'ai prononcés contre eux ; parce que je leur ai parlé, et ils n'ont pas écouté ; et que je les ai appelés, et ils n'ont pas répondu.
\VS{18}Et Jérémie dit à la maison des Récabites: Ainsi parle Yahweh des armées, le Dieu d'Israël : Parce que vous avez obéi au commandement de Jonadab, votre père, et que vous avez gardé tous ses commandements, et avez fait selon tout ce qu'il vous a ordonné\FTNT{Les Récabites furent bénis parce qu'ils obéirent aux commandements de leur père (Ep. 6:1-3)} ;
\VS{19}c'est pourquoi, ainsi parle Yahweh des armées, le Dieu d'Israël : Jonadab, fils de Récab, ne manquera jamais de descendants qui se tiennent debout devant moi.
\Chap{36}
\TextTitle{Le roi Jojakim brûle le manuscrit de Jérémie}
\VerseOne{}Or il arriva, dans la quatrième année de Jojakim, fils de Josias, roi de Juda, que cette parole fut adressée à Jérémie de la part de Yahweh, en disant :
\VS{2}Prends-toi un rouleau de livre, et tu y écriras toutes les paroles que je t'ai dites contre Israël et contre Juda, et contre toutes les nations, depuis le jour où je t'ai parlé, c'est à dire, depuis le jours de Josias, jusqu'à ce jour.
\VS{3}Peut-être que la maison de Juda entendra tout le mal que je pense de leur faire, afin que chaque homme se détourne de sa mauvaise voie, et que je leur pardonne leur iniquité, et leur péché.
\VS{4}Jérémie donc appela Baruc, fils de Nérija, et Baruc écrivit, sous la dictée de Jérémie, dans le rouleau de livre, toutes les paroles que Yahweh lui avait dites.
\VS{5}Puis Jérémie donna cet ordre à Baruc, en disant : Je suis retenu, et je ne peux pas entrer dans la maison de Yahweh.
\VS{6}Tu y entreras donc et tu liras dans le rouleau que tu as écrit sous ma dictée, toutes les paroles de Yahweh, aux oreilles du peuple dans la maison de Yahweh le jour du jeûne ; tu les liras, dis-je, aussi aux oreilles de tous ceux de Juda qui seront venus de leurs villes.
\VS{7}Peut-être que Yahweh écoutera leur supplication et qu'ils reviendront chacun de leur mauvaise voie ; car grande est la colère, la fureur que Yahweh a déclarée contre ce peuple.
\VS{8}Baruc donc, fils de Nérija, fit selon tout ce que lui avait ordonné Jérémie le prophète, lisant dans le rouleau les paroles de Yahweh, dans la maison de Yahweh.
\VS{9}Or il arriva dans la cinquième année de Jojakim, fils de Josias, roi de Juda, le neuvième mois, qu'on publia le jeûne devant Yahweh à tout le peuple de Jérusalem et à tout le peuple venu des villes de Juda à Jérusalem.
\VS{10}Et Baruc lut dans le livre les paroles de Jérémie, aux oreilles de tout le peuple, dans la maison de Yahweh, dans la chambre de Guemaria, fils de Schaphan, le secrétaire, dans le parvis supérieur, à l'entrée de la porte neuve de la maison de Yahweh.
\VS{11}Et quand Michée fils de Guemaria, fils de Schaphan, eut entendu toutes les paroles de Yahweh contenues dans le livre ;
\VS{12}il descendit dans la maison du roi, vers la chambre du secrétaire, et voici tous les chefs y étaient assis, à savoir, Elischama le secrétaire, Delaja, fils de Schemaeja, Elnathan, fils de Acbor, et Guemaria, fils de Schaphan, et Sédécias, fils de Hanania, et tous les chefs.
\VS{13}Et Michée leur rapporta toutes les paroles qu'il avait entendues, quand Baruc lisait dans le livre, aux oreilles du peuple.
\VS{14}C'est pourquoi tous les chefs envoyèrent vers Baruc, Jehudi, fils de Nethania, fils de Schélémia, fils de Cuschi, pour lui dire : Prends en ta main le rouleau que tu as lu aux oreilles du peuple, et viens ici ! Baruc donc, fils de Nérija, prit le rouleau en sa main, et vint vers eux.
\VS{15}Et ils lui dirent : Assieds-toi maintenant, et lis-le à nos oreilles ; et Baruc le lut à leurs oreilles.
\VS{16}Et il arriva que sitôt qu'ils eurent entendu toutes les paroles, ils furent effrayés entre eux, et dirent à Baruc : Nous ne manquerons pas de rapporter au roi toutes ces paroles.
\VS{17}Et ils interrogèrent Baruc, en disant : Dis-nous comment tu as écrit toutes ces paroles sous sa dictée.
\VS{18}Et Baruc leur dit : Il me dictait de sa bouche toutes ces paroles, et je les écrivais avec de l'encre dans le livre.
\VS{19}Alors les chefs dirent à Baruc : Va, et cache-toi, ainsi que Jérémie, et que personne ne sache où vous serez.
\VS{20}Puis ils s'en allèrent vers le roi dans la cour, mais ils prirent soin de laisser le rouleau dans la chambre d'Elischama le secrétaire ; et ils racontèrent toutes ces paroles aux oreilles du roi.
\VS{21}Et le roi envoya Jehudi pour prendre le rouleau ; et quand Jehudi l'eut prit de la chambre d'Elischama le secrétaire, et il le lut aux oreilles du roi et de tous les chefs qui étaient autour de lui.
\VS{22}Or le roi était assis dans la maison d'hiver, au neuvième mois, et un brasier était allumé devant lui.
\VS{23}Et il arriva qu'aussitôt que Jehudi en eut lu trois ou quatre feuilles, le roi déchira le rouleau avec le canif du secrétaire, et le jeta au feu du brasier, jusqu'à ce que tout le rouleau fut consumé au feu du brasier.
\VS{24}Et ni le roi ni tous ses serviteurs qui entendirent toutes ces paroles, n'en furent pas effrayés, et ne déchirèrent pas leurs vêtements.
\VS{25}Toutefois Elnathan, et Delaja et Guemaria intercédèrent envers le roi, afin qu'il ne brûle pas le rouleau ; mais il ne les écouta pas.
\VS{26}Même le roi ordonna à Jerachmeel, fils de Hammélec, et à Seraja, fils d'Azriel, et à Schélémia, fils de Abdeel, de saisir Baruc, le secrétaire, et Jérémie le prophète ; mais Yahweh les cacha.
\TextTitle{Remplacement du manuscrit brûlé ; jugement sur Jojakim}
\VS{27}Et la parole de Yahweh fut adressée à Jérémie, après que le roi eut brûlé le rouleau contenant les paroles que Baruc avait écrites sous la dictée de Jérémie, en disant :
\VS{28}Prends encore un autre rouleau, et tu y écriras toutes les premières paroles qui étaient dans le premier rouleau que Jojakim, roi de Juda, a brûlé.
\VS{29}Et tu diras à Jojakim, roi de Juda : Ainsi parle Yahweh : Tu as brûlé ce rouleau, et tu as dit : Pourquoi y as-tu écrit ces paroles : Le roi de Babylone viendra certainement, il détruira ce pays, et il exterminera les hommes et les bêtes ?
\VS{30}C'est pourquoi ainsi parle Yahweh sur Jojakim, roi de Juda : Aucun des siens ne sera assis sur le trône de David, et son cadavre sera jeté de jour à la chaleur et de nuit à la gelée.
\VS{31}Je le punirai, lui, sa postérité, et ses serviteurs, à cause de leur iniquité ; et je ferai venir sur eux, et sur les habitants de Jérusalem, et sur les hommes de Juda, tout le mal que je leur ai prononcé, et qu'ils n'ont pas écouter.
\VS{32}Jérémie donc prit un autre rouleau, et le donna à Baruc, fils de Nérija secrétaire, lequel y écrivit, sous la dictée de Jérémie, toutes les paroles du rouleau que Jojakim, roi de Juda, avait brûlé au feu. Beaucoup de paroles semblables y furent encore ajoutées.
\Chap{37}
\TextTitle{Sédécias sollicite l'intercession de Jérémie}
\VerseOne{}Or le roi Sédécias, fils de Josias, régna à la place de Jéconia, fils de Jojakim, et il fut établi roi dans le pays de Juda par Nebucadnetsar, roi de Babylone.
\VS{2}Mais, ni lui, ni ses serviteurs, ni le peuple du pays, n'obéirent pas aux paroles que Yahweh prononça par Jérémie le prophète.
\VS{3}Toutefois le roi Sédécias envoya Jucal, fils de Schélémia, et Sophonie, fils de Maaséja sacrificateur, vers Jérémie le prophète, pour lui dire : Intercède pour nous auprès de Yahweh, notre Dieu.
\VS{4}Car Jérémie allait et venait parmi le peuple, parce qu'on ne l'avait pas encore mis en prison.
\VS{5}Alors l'armée de Pharaon sortit d'Egypte, et quand les Chaldéens qui assiégeaient Jérusalem en entendirent cette nouvelle, ils se retirèrent de devant Jérusalem.
\VS{6}Et la parole de Yahweh fut adressée à Jérémie le prophète, en disant :
\VS{7}Ainsi parle Yahweh, le Dieu d'Israël : Vous direz ainsi au roi de Juda, qui vous a envoyés me consulter : Voici, l'armée de Pharaon, qui était sortie à votre secours, retourne dans son pays, en Egypte ;
\VS{8}et les Chaldéens reviendront, et combattront contre cette ville, et la prendront, et la brûleront au feu.
\VS{9}Ainsi parle Yahweh : Ne vous abusez pas vous-mêmes, en disant : Les Chaldéens s'en iront loin de nous ; car ils ne s'en iront pas.
\VS{10}Même quand vous auriez battu toute l'armée des Chaldéens qui combattent contre vous, et qu'il n'y aurait de reste entre eux que des hommes percés de blessures, ils se relèveront pourtant chacun dans sa tente, et brûleront cette ville au feu.
\TextTitle{Jérémie calomnié et emprisonné}
\VS{11}Or il arriva que quand l'armée des Chaldéens se fut retirée de Jérusalem, à cause de l'armée de Pharaon,
\VS{12}Jérémie sortit de Jérusalem, pour s'en aller dans le pays de Benjamin, se glissant hors de là au milieu du peuple.
\VS{13}Mais quand il fut à la porte de Benjamin, il y avait là un commandant de la garde, nommé Jireija, fils de Schélémia, fils de Hanania, qui saisit Jérémie le prophète, en lui disant : Tu vas te rendre aux Chaldéens !
\VS{14}Et Jérémie répondit : C'est un mensonge ! Je ne vais pas me rendre aux Chaldéens. Mais il ne l'écouta pas, et Jireija prit Jérémie, et l'amena vers les chefs.
\VS{15}Et les chefs se mirent en colère contre Jérémie, et le frappèrent et le mirent en prison dans la maison de Jonathan le secrétaire, car ils en avaient fait une prison.
\VS{16}Et ainsi Jérémie entra dans la fosse de la maison et dans les cachots ; et Jérémie y demeura plusieurs jours.
\VS{17}Mais le roi Sédécias y envoya, et l'en tira, et il l'interrogea en secret dans sa maison, et lui dit : Y a-t-il une parole de la part de Yahweh ? Et Jérémie répondit : Il y en a une ; Et lui dit : Tu seras livré entre les mains du roi de Babylone.
\VS{18}Puis Jérémie dit au roi Sédécias : Quel péché ai-je commis contre toi, contre tes serviteurs, et contre ce peuple, pour que vous m'ayez mis en prison ?
\VS{19}Mais où sont vos prophètes qui vous prophétisaient, en disant : Le roi de Babylone ne reviendra pas contre vous, ni contre ce pays ?
\VS{20}Or écoute maintenant, je te prie, ô roi, mon seigneur ! Et que maintenant ma supplication soit reçue devant ta face, et ne me renvoie pas dans la maison de Jonathan le secrétaire, de peur que je n'y meure !
\VS{21}C'est pourquoi le roi Sédécias ordonna qu'on garde Jérémie dans la cour de la prison, et qu'on lui donne chaque jour un pain de la rue des boulangers, jusqu'à ce que tout le pain de la ville soit épuisé. Ainsi Jérémie demeura dans la cour de la prison.
\Chap{38}
\TextTitle{Jérémie jeté dans la fosse puis délivré par Ebed-Mélec l'éthiopien}
\VerseOne{}Mais Schephathia, fils de Matthan, et Guedalia, fils de Paschhur, et Jucal, fils de Schélémia, et Paschhur, fils de Malkija, entendirent les paroles que Jérémie prononçait à tout le peuple, en disant :
\VS{2}Ainsi parle Yahweh : Celui qui restera dans cette ville mourra par l'épée, par la famine, ou par la peste ; mais celui qui sortira vers les Chaldéens vivra, et sa vie sera son butin, et il vivra.
\VS{3}Ainsi parle Yahweh : Cette ville sera livrée certainement aux mains de l'armée du roi de Babylone, qui la prendra.
\VS{4}Et les chefs dirent au roi : Qu'on fasse mourir cet homme ! Car il décourage les mains des hommes de guerre qui restent dans cette ville, et les mains de tout le peuple, en leur disant de telles paroles ; parce que cet homme ne cherche pas le bien de ce peuple, mais le mal.
\VS{5}Et le roi Sédécias dit : Voici, il est entre vos mains ; car le roi ne peut rien contre vous.
\VS{6}Ils prirent donc Jérémie, et le jetèrent dans la fosse de Malkija, fils de Hammélec, laquelle était dans la cour de la prison, et ils descendirent Jérémie avec des cordes dans cette fosse où Il n'y avait pas d'eau mais de la boue ; et ainsi Jérémie enfonça dans la boue.
\VS{7}Mais Ebed-Mélec l'éthiopien, eunuque, qui était dans la maison du roi, apprit qu'ils avaient mis Jérémie dans cette fosse ; et le roi était assis à la porte de Benjamin.
\VS{8}Et Ebed-Mélec sortit de la maison du roi, et parla au roi, en disant :
\VS{9}Ô roi, mon seigneur ! Ces hommes-là ont mal fait dans tout ce qu'ils ont fait contre Jérémie le prophète, en le jetant dans la fosse, car il serait déjà mort de faim dans le lieu où il était parce qu'il n'y a plus de pain dans la ville.
\VS{10}C'est pourquoi le roi donna cet ordre à Ebed-Mélec l'éthiopien, en disant : Prends ici trente hommes avec toi, et fais remonter hors de la fosse Jérémie le prophète, avant qu'il meure.
\VS{11}Ebed-Mélec donc prit des hommes avec lui, et entra dans la maison du roi, dans un lieu au-dessous du trésor, d'où il prit de vieux lambeaux et de vieux chiffons, et les descendit avec des cordes à Jérémie dans la fosse.
\VS{12}Et Ebed-Mélec l'éthiopien dit à Jérémie : Mets ces vieux lambeaux et ces chiffons sous les aisselles de tes bras, au-dessous des cordes. Et Jérémie fit ainsi.
\VS{13}Ainsi ils tirèrent Jérémie dehors avec les cordes, et le firent remonter hors de la fosse ; et Jérémie demeura dans la cour de la prison.
\TextTitle{Jérémie appelle Sédécias à la repentance}
\VS{14}Et le roi Sédécias envoya chercher Jérémie le prophète, et le fit amener vers lui à la troisième entrée qui était dans la maison de Yahweh. Et le roi dit à Jérémie : Je vais te demander une chose, ne me cache rien.
\VS{15}Et Jérémie répondit à Sédécias : Quand je te l'aurais déclarée, n'est-il pas vrai que tu me feras mourir ? Et quand je t'aurai donné conseil, tu ne m'écouteras pas.
\VS{16}Alors le roi Sédécias jura secrètement à Jérémie, en disant : Yahweh est vivant, qui nous a fait cette âme, je ne te ferai pas mourir, et que je ne te livrerai pas entre les mains de ces hommes qui cherchent ta vie.
\VS{17}Alors Jérémie dit à Sédécias : Ainsi parle Yahweh, le Dieu des armées, le Dieu d'Israël : Si tu sors volontairement pour aller vers les chefs du roi de Babylone, tu auras la vie, et cette ville ne sera pas brûlée par le feu ; et tu vivras toi et ta maison.
\VS{18}Mais si tu ne sors pas vers les chefs du roi de Babylone, cette ville sera livrée entre les mains des Chaldéens, qui la brûleront par le feu ; et tu n'échapperas pas à leurs mains.
\VS{19}Et le roi Sédécias dit à Jérémie : Je crains à cause des Juifs qui se sont rendus aux Chaldéens, je crains qu'on ne me livre entre leurs mains et qu’ils ne se moquent de moi.
\VS{20}Et Jérémie lui répondit : On ne te livrera pas à eux. Je te prie, écoute la voix de Yahweh dans ce que je te dis ; tu t'en trouveras bien, et tu auras la vie.
\VS{21}Que si tu refuses de sortir, voici ce que Yahweh m'a fait voir :
\VS{22}C'est que, voici toutes les femmes qui restent dans la maison du roi de Juda seront menées aux chefs du roi de Babylone, et elles diront : Tu as été séduit, vaincu, par les hommes qui te prédisaient la paix ; et quand tes pieds sont enfoncés dans la boue, ils se sont retirés en arrière.
\VS{23}Toutes tes femmes et tes fils seront menés dehors aux Chaldéens ; et tu n'échapperas pas à leurs mains, mais tu seras pris, pour être livré entre les mains du roi de Babylone, et à cause de toi, cette ville sera brûlée par le feu.
\VS{24}Alors Sédécias dit à Jérémie : Que personne ne sache rien de ces paroles, et tu ne mourras pas.
\VS{25}Et si les chefs entendent que je t'ai parlé, et qu'ils viennent vers toi, et te dise : Déclare-nous maintenant ce que tu as dit au roi, et ce que le roi t'a dit, ne nous en cache rien, et nous ne te ferons pas mourir ;
\VS{26}tu leur diras : J'ai présenté ma supplication devant le roi afin qu'il ne me renvoie pas dans la maison de Jonathan, pour y mourir.
\VS{27}Tous les chefs donc vinrent vers Jérémie, et l'interrogèrent ; mais il leur répondit exactement comme le roi lui avait ordonné ; et ils gardèrent le silence, car l'affaire n'avait pas été divulguée.
\VS{28}Ainsi Jérémie demeura dans la cour de la prison, jusqu'au jour où Jérusalem fut prise, et il y était lorsque Jérusalem fut prise.
\Chap{39}
\TextTitle{Prise de Jérusalem ; Sédécias déporté à Babylone\FTNTT{2 R. 25:1-7; Jé. 52:4-17 ; 2 Ch. 36:17-21.}}
\VerseOne{}La neuvième année de Sédécias, roi de Juda, au dixième mois, Nebucadnetsar, roi de Babylone, vint avec toute son armée contre Jérusalem, et ils l'assiégèrent.
\VS{2}Et la onzième année de Sédécias, le neuvième jour du quatrième mois, une brèche fut faite à la ville.
\VS{3}Et tous les chefs du roi de Babylone y entrèrent, et s'assirent à la porte du milieu, à savoir, Nergal-Scharetser, Samgar-Nebu, Sarsekim, chef des eunuques, Nergal-Scharetser, chef des devins et tous les autres chefs du roi de Babylone.
\VS{4}Or il arriva qu'aussitôt que Sédécias, roi de Juda, et tous les hommes de guerre les eurent vus, ils s'enfuirent et sortirent de nuit hors de la ville, par le chemin du jardin du roi, par la porte entre les deux murailles, et ils s'en allèrent par le chemin de la plaine.
\VS{5}Mais l'armée des Chaldéens les poursuivit et atteignit Sédécias dans les plaines de Jéricho. Ils le prirent, et le firent monter vers Nebucadnetsar, roi de Babylone, à Ribla, dans le pays de Hamath, où il prononça contre lui une sentence.
\VS{6}Et le roi de Babylone fit égorger à Ribla les fils de Sédécias sous ses yeux ; le roi de Babylone fit aussi égorger tous les nobles de Juda.
\VS{7}Puis il fit crever les yeux à Sédécias, et le fit lier de doubles chaînes d'airain, pour le conduire à Babylone.
\VS{8}Les Chaldéens brûlèrent par le feu la maison royale et les maisons du peuple, et démolirent\FTNT{Ici débute le « temps des nations » (587-586 av J.-C.), Jérusalem est foulée aux pieds par les nations. Voir aussi 2 R. 25:8-24 ; 2 Ch. 36:17-21.} les murailles de Jérusalem.
\VS{9}Et Nebuzaradan, chef des gardes, transporta à Babylone le reste du peuple qui était resté dans la ville, et ceux qui s'étaient rendus à lui, le reste, dis-je, du peuple qui avait été épargné.
\VS{10}Mais Nebuzaradan, chefs des gardes, laissa dans le pays de Juda les plus pauvres du peuple qui n'avaient rien ; et en ce jour-là, il leur donna des vignes et des champs.
\TextTitle{Jérémie libéré de prison}
\VS{11}Or Nebucadnetsar, roi de Babylone, avait donné cet ordre au sujet de Jérémie, à Nebuzaradan, chef des gardes, en disant :
\VS{12}Prends cet homme et veille sur lui ; ne lui fais aucun mal, mais fais pour lui tout ce qu'il te dira.
\VS{13}Nebuzaradan donc chefs des gardes, envoya, et aussi Nebuschazban, Rabsaris, chef des eunuques, Nergal-Scharetser, Rabmag, chef des devins, et tous les chefs du roi de Babylone ;
\VS{14}ils envoyèrent, dis-je, chercher Jérémie dans la cour de la prison, et le remirent à Guedalia, fils d'Achikam, fils de Schaphan, pour qu'il le conduise dans sa maison. Ainsi il demeura au milieu du peuple.
\TextTitle{Yahweh épargne Ebed-Mélec}
\VS{15}Or la parole de Yahweh fut adressée à Jérémie pendant qu'il était enfermé dans la cour de la prison, en disant :
\VS{16}Va, et parle à Ebed-Mélec l'éthiopien, et dis-lui : Ainsi parle Yahweh des armées, le Dieu d'Israël : Voici, je vais faire venir mes paroles sur cette ville pour son malheur et non pas pour son bien, et elles s'accompliront en ce jour-là devant toi.
\VS{17}Mais je te délivrerai en ce jour-là, dit Yahweh, et tu ne seras pas livré entre les mains des hommes que tu crains.
\VS{18}Car certainement je te ferai échapper, et tu ne tomberas pas sous l'épée ; mais ta vie sera ton butin, parce que tu as eu confiance en moi, dit Yahweh.
\Chap{40}
\TextTitle{Assassinat de Guedalia et meurtres en série d'Ismaël}
\VerseOne{}La parole qui fut adressée à Jérémie de la part de Yahweh, quand Nebuzaradan, chef des gardes, l'eut renvoyé de Rama, après l'avoir pris lorsqu'il était lié de chaînes parmi tous les captifs de Jérusalem et de Juda qu'on transportait à Babylone.
\VS{2}Quand donc le chef des gardes prit Jérémie, et il lui dit : Yahweh, ton Dieu, a prononcé ce mal contre ce lieu-ci ;
\VS{3}et Yahweh l'a fait venir et a fait comme il avait dit, parce que vous avez péché contre Yahweh, et que vous n'avez pas écouté sa voix, à cause de cela ceci vous est arrivé.
\VS{4}Maintenant donc voici, je t'affranchis aujourd'hui des chaînes que tu as aux mains ; s'il est bon à tes yeux de venir avec moi à Babylone, viens, et j'aurai les yeux sur toi ; mais s'il est mauvais de venir avec moi à Babylone, ne viens pas ; regarde, tout le pays est à ta disposition, va où il te semblera bon et convenable d'aller.
\VS{5}Or Guedalia ne retournera plus ici ; retourne, dit-il, vers Guedalia, fils d'Achikam, fils de Schaphan, que le roi de Babylone a établi sur les villes de Juda, et demeure avec lui parmi le peuple ; ou bien, va partout où il conviendra à tes yeux d'aller. Et le chef des gardes lui donna des vivres et quelques présents, et le renvoya.
\VS{6}Jérémie donc alla vers Guedalia, fils d'Achikam, à Mitspa, et demeura avec lui parmi le peuple qui était resté dans le pays.
\VS{7}Et tous les chefs des armées qui étaient dans les champs, eux et leurs hommes, entendirent que le roi de Babylone avait établi Guedalia, fils d'Achikam, sur le pays, et qu'il lui avait commis les hommes, et les femmes, et les enfants, et ceux-là d'entre les plus pauvres du pays, à savoir, de ceux qui n'avaient pas été transportés à Babylone.
\VS{8}Alors ils allèrent vers Guedalia à Mitspa ; à savoir, Ismaël fils de Nethania, et Jochanan et Jonathan fils de Karéach, et Seraja fils de Thanhumeth, et les fils d'Ephaï de Nethopha, et Jezania fils du Maacatite, eux et leurs hommes.
\VS{9}Et Guedalia, fils d'Achikam, fils de Schaphan, leur jura, à eux et à leurs hommes, en disant : Ne craignez pas de servir les Chaldéens ; demeurez dans le pays, et servez le roi de Babylone, et vous vous en trouverez bien.
\VS{10}Et pour moi, voici, je resterai à Mitspa, pour me tenir prêt à recevoir les ordres des Chaldéens qui viendront vers nous ; mais vous, recueillez le vin, les fruits d'été et l'huile, et mettez-les dans vos vases, et demeurez dans vos villes que vous avez prises pour votre demeure.
\VS{11}Pareillement aussi tous les Juifs qui étaient au pays de Moab, et parmi les Ammonites, et au pays d'Edom, et dans toutes ces contrées, quand il eurent entendu que le roi de Babylone avait laissé quelque reste à Juda, et qu'il avait établi sur eux Guedalia, fils d'Achikam, fils de Schaphan ;
\VS{12}tous ces juifs-là retournèrent de tous les lieux où ils avaient été chassés, et vinrent dans le pays de Juda vers Guedalia à Mitspa, et recueillirent du vin et des fruits d'été en grande abondance.
\VS{13}Mais Jochanan, fils de Karéach, et tous les chefs des armées qui étaient dans les champs, vinrent vers Guedalia à Mitspa,
\VS{14}et lui dirent : Ne sais-tu pas certainement que Baalis, roi des Ammonites, a envoyé Ismaël, le fils de Nethania, pour t'ôter la vie ? Mais Guedalia, fils d'Achikam, ne les crut pas.
\VS{15}Et Jochanan, fils de Karéach parla en secret à Guedalia à Mitspa, en disant : Laisse-moi aller et frapper Ismaël, fils de Nethania, et personne ne le saura. Pourquoi t'ôterait-il la vie, afin que tous les Juifs qui se sont rassemblés vers toi soient dissipés, et que les restes de Juda périssent ?
\VS{16}Mais Guedalia, fils d'Achikam, dit à Jochanan, fils de Karéach : Ne fais pas cela, car tu parles faussement d'Ismaël.
\Chap{41}
\TextTitle{Assassinat de Guedalia}
\VerseOne{}Or il arriva, au septième mois, qu'Ismaël, fils de Nethania, fils d'Elischama, de la race royale, et l'un des grands du roi et dix hommes avec lui, vinrent vers Guedalia, fils d'Achikam, à Mitspa ; et ils mangèrent là du pain ensemble à Mitspa\FTNT{2 R. 25:25.}.
\VS{2}Mais Ismaël, fils de Nethania, se leva, et les dix hommes qui étaient avec lui, et ils frappèrent avec l'épée Guedalia, fils d'Achikam, fils de Schaphan, et on le fit mourir, lui que le roi de Babylone avait établi sur le pays.
\VS{3}Ismaël frappa aussi tous les juifs qui étaient avec Guedalia à Mitspa, et les Chaldéens, gens de guerre, qui se trouvaient là.
\VS{4}Et il arriva que le second jour après après qu'on eut fait mourir Guedalia, avant que personne le sût,
\VS{5}quelques hommes de Sichem, de Silo et de Samarie, au nombre de quatre-vingts hommes, ayant la barbe rasée et les vêtements déchirés, et s'étant fait des incisions, vinrent avec des dons et de l'encens dans leurs mains pour les apporter dans la maison de Yahweh.
\VS{6}Alors Ismaël, fils de Nethania, sortit de Mitspa au-devant d'eux, et il marchait en pleurant, et quand il les rencontra, il leur dit : Venez vers Guedalia, fils d'Achikam.
\VS{7}Mais sitôt qu'ils arrivèrent au milieu de la ville, Ismaël, fils de Nethania, accompagné des hommes qui étaient avec lui, les égorgea et les jeta dans une fosse.
\VS{8}Mais il se trouva parmi eux dix hommes, qui dirent à Ismaël : Ne nous fais pas mourir, car nous avons dans les champs des provisions cachées de froment, d'orge, d'huile et de miel ; et il les laissa, et ne les fit pas mourir avec leurs frères.
\VS{9} Et la fosse dans laquelle Ismaël jeta les cadavres des hommes qu'il tua, à l'occasion de Guedalia, est celle que le roi Asa avait faite, lorsqu'il craignait Baescha, roi d'Israël ; et Ismaël, fils de Nethania, la remplit de gens tués\FTNT{1 R. 15:22.}.
\VS{10}Et Ismaël emmena captif tout le reste du peuple qui était à Mitspa, les filles du roi et tous ceux du peuple qui demeuraient à Mitspa, que Nebuzaradan, chef des gardes, avait commis à Guedalia, fils d'Achikam ; Ismaël, fils de Nethania, les emmena captifs, et s'en alla pour passer vers les Ammonites.
\TextTitle{Jochanan délivre le peuple ; fuite d'Ismaël}
\VS{11}Mais Jochanan, fils de Karéach, et tous les chefs des armées qui étaient avec lui, entendirent tout le mal qu'Ismaël, fils de Nethania, avait fait ;
\VS{12}et ils prirent tous les hommes, et s'en allèrent pour combattre contre Ismaël, fils de Nethania. Ils le trouvèrent près des grandes eaux qui sont à Gabaon.
\VS{13}Et il arriva qu'aussitôt que tout le peuple qui était avec Ismaël vit Jochanan, fils de Karéach, et tous les chefs des armées qui étaient avec lui, ils s'en réjouirent ;
\VS{14}et tout le peuple qu'Ismaël avait emmené captif de Mitspa tourna visage, et revenant sur leur pas, il s'en alla vers Jochanan, fils de Karéach.
\VS{15}Mais Ismaël, fils de Nethania, échappa avec huit hommes devant Jochanan, et s'en alla vers les Ammonites.
\VS{16}Et Jochanan, fils de Karéach, et tous les chefs des armées qui étaient avec lui, prirent tout le reste du peuple qu'ils avaient retiré des mains d'Ismaël, fils de Nethania, qu'il emmenait captif de Mitspa, après avoir tué Guedalia, fils d'Achikam, à savoir, les vaillants hommes de guerre, et les femmes, et les enfants et les eunuques ; et les ramenèrent de Gabaon.
\VS{17}Et ils s'en allèrent et demeurèrent à l'hôtellerie de Kimham, près de Bethléhem, pour se retirer ensuite en Egypte,
\VS{18}à cause des Chaldéens ; car ils avaient peur d'eux, parce qu'Ismaël, fils de Nethania, avait tué Guedalia, fils d'Achikam, qui avait été établi sur le pays par le roi de Babylone.
\Chap{42}
\TextTitle{Yahweh défend au reste du peuple de se réfugier en Egypte }
\VerseOne{}Alors tous les chefs des armées, et Jochanan, fils de Karéach, et Jezania, fils d'Hosée, et tout le peuple, depuis le plus petit jusqu'au plus grand, s'approchèrent,
\VS{2}et dirent à Jérémie le prophète : Que notre supplication soit favorable devant toi ! Intercède auprès de Yahweh, ton Dieu, pour nous, à savoir, pour tout ce reste-ci ; car de beaucoup de monde que nous étions, nous sommes restés peu, comme tes yeux nous voient ;
\VS{3}et que Yahweh, ton Dieu, nous déclare le chemin par lequel nous aurons à marcher, et ce que nous avons à faire !
\VS{4}Et Jérémie le prophète, leur répondit : J'ai entendu votre demande ; voici, je vais prier Yahweh, votre Dieu, selon vos paroles ; et il arrivera que je vous déclarerai tout ce que Yahweh vous répondra, et je ne vous en cacherai pas un mot.
\VS{5}Et ils dirent à Jérémie : Yahweh soit entre nous un témoin véritable et fidèle, si nous ne faisons pas selon toutes les paroles que Yahweh, ton Dieu, t'enverra vers nous !
\VS{6}Soit bien, soit mal, nous obéirons à la voix de Yahweh, notre Dieu, vers qui nous t'envoyons, afin qu'il nous arrive du bien, quand nous aurons obéi à la voix de Yahweh, notre Dieu.
\VS{7}Et il arriva, au bout de dix jours, que la parole de Yahweh fut adressée à Jérémie.
\VS{8}Et il appela Jochanan, fils de Karéach, tous les chefs des armées qui étaient avec lui, et tout le peuple, depuis le plus petit jusqu'au plus grand ;
\VS{9}et leur dit : Ainsi parle Yahweh, le Dieu d'Israël, vers qui vous m'avez envoyé, pour présenter votre supplication devant lui :
\VS{10}Si vous continuez à demeurer dans ce pays, je vous rétablirai et je ne vous détruirai pas ; je vous y planterai et je ne vous arracherai pas, car je me repens du mal que je vous ai fait.
\VS{11}Ne craignez pas le roi de Babylone, dont vous avez peur, ne craignez pas, dit Yahweh, car je suis avec vous pour vous sauver et pour vous délivrer de sa main.
\VS{12}Même je vous ferai obtenir miséricorde, tellement qu'il aura pitié de vous, et vous fera retourner dans votre pays.
\VS{13}Que si vous dites : Nous ne demeurerons pas dans ce pays, et nous n'écouterons pas la voix de Yahweh, notre Dieu,
\VS{14}en disant : Non ; mais nous irons au pays d'Egypte, afin que nous ne voyons pas de guerre, et que nous n'entendions pas le son du shofar, et que nous ne manquions pas de pain, et nous demeurerons là.
\VS{15}A cause de cela écoutez maintenant la parole de Yahweh, vous les restes de Juda ! Ainsi parle Yahweh des armées, le Dieu d'Israël : Si vous tournez le visage pour aller en Egypte, et que vous y entriez pour y demeurer ;
\VS{16}il arrivera que l'épée dont vous avez peur vous attrapera là au pays d'Egypte ; et la famine que vous craignez si fort vous suivra en Egypte, et vous y mourrez\FTNT{Ez. 30:9-11.}.
\VS{17}Et il arrivera que tous les hommes qui tourneront le visage pour aller en Egypte afin d'y demeurer, mourront par l'épée, par la famine et par la peste ; et il n'y aura ni survivant ni réchappé devant le mal que je vais faire venir sur eux.
\VS{18}Car ainsi parle Yahweh des armées, le Dieu d'Israël : Comme ma colère et ma fureur se sont répandues sur les habitants de Jérusalem, ainsi ma fureur sera versée sur vous, quand vous serez entrés en Egypte ; et vous serez un sujet d'exécration, d'épouvante, de malédiction et d'opprobre, et vous ne verrez plus ce lieu-ci.
\VS{19}Vous, les restes de Juda, Yahweh dit contre vous : N'allez pas en Egypte ! Sachez certainement que je vous ai avertis aujourd'hui.
\VS{20}Car vous vous êtes séduits vous-mêmes dans vos âmes, quand vous m'avez envoyé vers Yahweh, votre Dieu, en me disant : Intercède pour nous auprès de Yahweh, notre Dieu, et déclare-nous tout ce que Yahweh, notre Dieu, te dira, et nous le ferons.
\VS{21}Et je vous l'ai déclaré aujourd'hui ; mais vous n'écoutez pas la voix de Yahweh, votre Dieu, ni rien de tout ce pour quoi il m'a envoyé vers vous.
\VS{22}Maintenant donc sachez certainement que vous mourrez par l'épée, par la famine et par la peste, dans le lieu où vous avez désiré d’aller pour y demeurer.
\Chap{43}
\TextTitle{Désobéissance des Hébreux ; jugement sur l'Egypte}
\VerseOne{}Or il arriva qu'aussitôt que Jérémie eut achevé de prononcer à tout le peuple toutes les paroles de Yahweh, leur Dieu, pour lesquelles Yahweh, leur Dieu, l'avait envoyé vers eux, à savoir, toutes ces choses-là ;
\VS{2}Azaria, fils d'Hosée, et Jochanan, fils de Karéach, et tous ces hommes orgueilleux, dirent à Jérémie : Tu dis un mensonge ; Yahweh, notre Dieu, ne t'a pas envoyé nous dire : N'allez pas en Egypte pour y demeurer.
\VS{3}Mais Baruc, fils de Nérija, t'incite contre nous, afin de nous livrer entre les mains des Chaldéens, pour nous faire mourir, et pour nous faire transporter à Babylone.
\VS{4}Ainsi Jochanan, fils de Karéach, et tous les chefs des armées, et tout le peuple, n'obéirent pas à la voix de Yahweh, pour demeurer dans le pays de Juda.
\VS{5}Car Jochanan, fils de Karéach, et tous les chefs des armées, prirent tous les restes de Juda qui étaient revenus de toutes les nations, parmi lesquelles ils avaient été chassés, pour demeurer dans le pays de Juda ;
\VS{6}les hommes, et les femmes, et les enfants, et les filles du roi, et toutes les personnes que Nebuzaradan, chef des gardes, avait laissées avec Guedalia, fils d'Achikam, fils de Schaphan ; ils prirent aussi Jérémie le prophète et Baruc, fils de Nérija.
\VS{7}Et ils entrèrent dans le pays d'Egypte, car ils n'obéirent pas à la voix de Yahweh, et ils vinrent jusqu'à Tachpanès.
\VS{8}Alors la parole de Yahweh fut adressée à Jérémie, à Tachpanès, en disant :
\VS{9}Prends dans ta main de grandes pierres, et cache-les dans l'argile, dans le four à briques qui est à l'entrée de la maison de Pharaon à Tachpanès, sous les yeux des Juifs ;
\VS{10}et dis-leur : Ainsi parle Yahweh des armées, le Dieu d'Israël : Voici, j'enverrai chercher Nebucadnetsar, roi de Babylone, mon serviteur, et je mettrai son trône sur ces pierres que j'ai cachées, et il étendra son dais sur elles ;
\VS{11}et Il viendra et frappera le pays d'Egypte. Ceux qui sont destinés à la mort, iront à la mort ; et ceux qui sont destinés à la captivité, iront en captivité ; et ceux qui sont destinés à l’épée, seront livrés à l’épée\FTNT{Ez. 29:9.} !
\VS{12}Et j'allumerai le feu dans les maisons des dieux d'Egypte, Nebucadnetsar les brûlera, et il emmènera captifs ceux d'Egypte, et il se parera des richesses du pays d'Egypte, comme le pasteur s'enveloppe de son vêtement, et il en sortira en paix\FTNT{Es. 19:1 ; Ez. 30:13.}.
\VS{13}Il brisera aussi les statues de Beth-Schémesch, qui est au pays d'Egypte, et il brûlera par le feu les maisons des dieux d'Egypte.
\Chap{44}
\TextTitle{Yahweh avertit les Juifs d'Egypte\FTNTT{Jé. 43:8-13}}
\VerseOne{}La parole qui fut adressée à Jérémie sur tous les Juifs qui demeuraient au pays d'Egypte, qui habitaient à Migdol, à Tachpanès, à Noph, et au pays de Pathros, en disant :
\VS{2}Ainsi parle Yahweh des armées, le Dieu d'Israël : Vous avez vu tous les malheurs que j'ai fait venir sur Jérusalem et sur toutes les villes de Juda : Voici, elles ne sont plus aujourd'hui que des ruines, et personne n'y habite,
\VS{3}à cause des méchancetés qu'ils ont faites pour m'irriter, en allant brûler de l'encens pour servir d'autres dieux, qu'ils n'ont pas connu, ni eux, ni vous, ni vos pères.
\VS{4}Et je vous ai envoyé tous mes serviteurs, les prophètes, me levant dès le matin, et les envoyant, pour vous dire : Ne commettez pas maintenant cette chose abominable que je hais.
\VS{5}Mais ils n'ont pas écouté, ils n'ont pas prêté l'oreille pour se détourner de leur méchanceté, afin de ne pas faire brûler de l'encens à d'autres dieux.
\VS{6}C'est pourquoi ma fureur et ma colère se sont répandues sur eux, et ont embrasé les villes de Juda et les rues de Jérusalem, qui ne sont réduites en désert et en désolation, comme il paraît aujourd'hui.
\VS{7}Maintenant donc, ainsi parle Yahweh, le Dieu des armées, le Dieu d'Israël : Pourquoi faites-vous ce grand mal contre vos âmes, pour vous faire exterminer du milieu de Juda, hommes et femmes, petits enfants et ceux qui tètent, afin qu'on ne vous laisse aucun reste ?
\VS{8}En m'irritant par les œuvres de vos mains, en brûlant de l'encens à d'autres dieux au pays d'Egypte, où vous êtes venus pour y demeurer, afin de vous faire exterminer et d'être un objet de malédiction et d'opprobre parmi toutes les nations de la terre ?
\VS{9}Avez-vous oublié les crimes de vos pères, les crimes des rois de Juda, les crimes de leurs femmes, vos propres crimes et les crimes de vos femmes, commis dans le pays de Juda et dans les rues de Jérusalem ?
\VS{10}Jusqu'à ce jour, ils ne se sont pas humiliés, ils n'ont pas eu de crainte, ils n'ont pas marché dans ma loi ni dans mes ordonnances, que j'ai mises devant vous et devant vos pères.
\VS{11}C'est pourquoi ainsi parle Yahweh des armées, le Dieu d'Israël : Voici, je tourne ma face contre vous pour vous nuire et vous retrancher tout Juda\FTNT{Am. 9:4.}.
\VS{12}Et je prendrai les restes de ceux de Juda qui ont tourné le visage pour aller au pays d'Egypte afin d'y demeurer ; ils seront tous consumés, ils tomberont dans le pays d'Egypte ; ils seront consumés par l'épée, par la famine, depuis le plus petit jusqu'au plus grand ; ils mourront par l'épée et par la famine ; et ils seront en exécration, en étonnement, en malédiction et en opprobre.
\VS{13}Et je punirai ceux qui demeurent au pays d'Egypte, comme j'ai puni Jérusalem, par l'épée, par la famine, et par la peste.
\VS{14}Il n'y aura personne du reste de Juda qui est venu dans le pays d’Égypte pour y séjourner, échappera ou restera, pour retourner dans le pays de Juda, où ils aspirent à retourner pour y demeurer ; car pas un ne retournera, sinon les rescapés.
\VS{15}Mais les tous hommes qui savaient que leurs femmes brûlaient de l'encens à d'autres dieux, toutes les femmes qui se tenaient là en grande compagnie, et tout le peuple qui demeurait au pays d'Egypte, à Pathros, répondirent à Jérémie, en disant :
\VS{16}Quant à la parole que tu nous as dite au nom de Yahweh, nous ne t'écouterons pas.
\VS{17}Mais nous ferons assurément selon toute parole qui est sortie de notre bouche, brûler de l'encens à la reine des cieux\FTNT{Voir commentaire en Jé. 7:18.}, et lui faire des libations, comme nous l'avons fait, nous et nos pères, nos rois et nos chefs, dans les villes de Juda et dans les rues de Jérusalem. Alors nous étions rassasiés de pain, nous étions heureux, et nous ne voyions pas le malheur\FTNT{Ez. 16:24 ; Ez. 20:32.}.
\VS{18}Mais depuis le temps que nous avons cessé de brûler de l'encens à la reine des cieux et de lui faire des libations, nous avons manqué de tout, et nous avons été consumés par l'épée et par la famine…
\VS{19}Quand nous brûlions de l'encens à la reine des cieux et que nous lui faisions des libations, est-ce à l'insu de nos maris que nous lui faisons des gâteaux sur lesquels elle est représentée et que nous lui faisons des libations ?
\VS{20}Alors Jérémie parla à tout le peuple, aux hommes, aux femmes, et à tous ceux qui lui avaient donné cette réponse, et leur dit :
\VS{21}Yahweh ne s'est-il pas souvenu, ne lui est-il pas monté à cœur l'encens que vous avez brûlé dans les villes de Juda et dans les rues de Jérusalem, vous et vos pères, vos rois et vos chefs, et le peuple du pays ?
\VS{22}Yahweh n'a pas pu le supporter davantage, à cause de la méchanceté de vos actions, à cause des abominations que vous avez faites ; et votre pays est devenu une ruine, un désert et un objet de malédiction, sans que personne y habite, comme on le voit aujourd'hui.
\VS{23}C'est parce que vous avez brûlé de l'encens et que vous avez péché contre Yahweh, parce que vous n'avez pas écouté la voix de Yahweh, et que vous n'avez pas marché dans sa loi, ni dans ses ordonnances, ni dans ses témoignages, c'est pour cela que ces malheurs vous sont arrivés, comme on le voit aujourd'hui.
\VS{24}Jérémie dit à tout le peuple et à toutes les femmes : Vous tous de Juda, qui êtes au pays d'Egypte, écoutez la parole de Yahweh !
\VS{25}Ainsi parle Yahweh des armées, le Dieu d'Israël, en disant : Vous et vos femmes, vous avez parlé de vos bouches et accompli de vos mains, en disant : Certainement, nous accomplirons nos vœux que nous avons faits, brûler de l'encens à la reine des cieux, et lui faire des libations. Vous avez entièrement accompli vos vœux, vous les avez effectués très exactement.
\VS{26}C'est pourquoi, écoutez la parole de Yahweh, vous tous de Juda, qui demeurez au pays d'Egypte ! Voici, je le jure par mon grand Nom, dit Yahweh, mon Nom ne sera plus invoqué par la bouche d'aucun homme de Juda, et dans tout le pays d'Egypte aucun ne dira : Le Seigneur Yahweh est vivant !
\VS{27}Voici, je veille sur eux pour leur mal et non pour leur bien ; et tous les hommes de Juda qui sont dans le pays d'Egypte seront consumés par l'épée et par la famine, jusqu'à ce qu'ils soient exterminés\FTNT{Da. 9:14.}.
\VS{28}Et ceux qui seront échappés à l'épée, retourneront du pays d'Egypte au pays de Juda en fort petit nombre. Mais tout le reste de Juda, tous ceux qui sont venus dans le pays d'Egypte pour y demeurer, sauront quelle est la parole qui s'accomplira, la mienne ou la leur.
\VS{29}Et ceci sera pour vous le signe, dit Yahweh, que je vous punirai dans ce lieu, afin que vous sachiez que mes paroles s'accompliront infailliblement pour votre malheur.
\VS{30}Ainsi parle Yahweh : Voici, je livrerai Pharaon Hophra, roi d'Egypte, entre les mains de ses ennemis, entre les mains de ceux qui cherchent sa vie, comme j'ai livré Sédécias, roi de Juda, entre les mains de Nebucadnetsar, roi de Babylone, son ennemi, et qui cherchait sa vie.
\Chap{45}
\TextTitle{Yahweh explique son dessein à Baruc}
\VerseOne{}La parole que Jérémie, le prophète, adressa à Baruc, fils de Nérija, quand il écrivit dans un livre ces paroles, sous la dictée de Jérémie, la quatrième année de Jojakim, fils de Josias, roi de Juda. Il dit :
\VS{2}Ainsi parle Yahweh, le Dieu d'Israël, sur toi, Baruc :
\VS{3}Tu dis : Malheur à moi ! Car Yahweh ajoute la tristesse à ma douleur ; je me suis lassé dans mon gémissement, et je ne trouve pas de repos.
\VS{4}Tu lui diras : Ainsi parle Yahweh : Voici, je vais détruire ce que j'ai bâti, et arracher ce que j'ai planté, à savoir tout ce pays.
\VS{5}Et toi, chercherais-tu de grandes choses ? Ne les cherche pas ! Car voici, je vais faire venir du mal sur toute chair, dit Yahweh ; je te donnerai ta vie pour butin, dans tous les lieux où tu iras.
\Chap{46}
\TextTitle{Prophétie contre l'Egypte}
\VerseOne{}La parole de Yahweh qui fut adressée à Jérémie, le prophète, sur les nations.
\VS{2}A l'égard de l'Egypte, contre l'armée de Pharaon Neco, roi d'Egypte, qui était près du fleuve de l'Euphrate, à Carkemisch, et qui fut battue par Nebucadnetsar, roi de Babylone, la quatrième année de Jojakim, fils de Josias, roi de Juda\FTNT{2 R. 24:7.}.
\VS{3}Préparez le bouclier et l'écu et approchez-vous pour la bataille !
\VS{4}Attelez les chevaux, montez, cavaliers ! Présentez-vous avec vos casques, polissez vos lances, revêtez l'armure !
\VS{5}D'où vient que je vois ceci ? Ils sont effrayés, ils reviennent en arrière ; leurs hommes vaillants sont battus ; ils s'enfuient avec précipitation sans regarder derrière eux… La frayeur les environne, dit Yahweh.
\VS{6}Que l'homme léger à la course ne s'enfuie pas, et que le fort ne se sauve pas\FTNT{Am. 2:14-16.} ! Ils sont renversés et tombés vers le nord, auprès du rivage du fleuve de l'Euphrate.
\VS{7}Qui est celui-ci qui s'élève comme le Nil et dont les eaux sont agitées comme les fleuves ?
\VS{8}C'est l'Egypte. Elle s'élève comme le Nil et ses eaux agitées comme les fleuves ; et elle dit : Je m'élèverai et je couvrirai la terre ; je détruirai la ville et ceux qui y habitent.
\VS{9}Montez, chevaux ! Agissez en insensés, chars ! Que les hommes vaillants sortent, ceux d'Ethiopie et de Puth qui manient le bouclier, et ceux de Lud qui manient et tendent l'arc\FTNT{Ez. 30:5-9 ; Na. 3:9-10.} !
\VS{10}Car c'est le jour du Seigneur,  Yahweh des armées ; c'est un jour de vengeance, où il se venge de ses ennemis. L'épée dévore, elle se rassasie, elle s'enivre de leur sang. Car il y a des sacrifices pour le Seigneur, Yahweh des armées, dans le pays du nord, sur le fleuve de l'Euphrate\FTNT{Es. 34:5-6 ; Ez. 39:17 ; So. 1:7.}.
\VS{11}Monte en Galaad, prends du baume, vierge, fille de l'Egypte ! En vain tu multiplies les remèdes, il n'y a pas de guérison pour toi\FTNT{Ez. 30:21-25 ; Na. 3:19.}.
\VS{12}Les nations apprennent ta honte, et tes cris remplissent la terre, car les hommes forts chancellent l'un sur l'autre, ils tombent tous deux ensemble.
\VS{13}La parole que Yahweh prononça à Jérémie, le prophète, sur la venue de Nebucadnetsar, roi de Babylone, pour frapper le pays d'Egypte :
\VS{14}Déclarez-le en Egypte, et publiez-le à Migdol, à Noph, et à Tachpanès et Dites : Présente-toi, tiens-toi prêt car l'épée dévore ce qui est autour de toi !
\VS{15}Pourquoi tes vaillants hommes sont-ils emportés ? Ils ne tiennent pas ferme, parce que Yahweh les pousse.
\VS{16}Il en a terrassé un grand nombre, et même chacun tombe sur son compagnon, et ils disent : Levons-nous, retournons vers notre peuple, au pays de notre naissance, loin de l'épée de l'oppresseur !
\VS{17}Là, ils s'écrient : Pharaon, roi d'Egypte, n'est qu'un bruit ; il a laissé passer le temps fixé.
\VS{18}Je suis vivant ! dit le Roi, dont le Nom est Yahweh des armées ; comme le Thabor entre les montagnes, comme le Carmel qui s'avance dans la mer, ainsi viendra-t-il.
\VS{19}O fille, habitante de l'Egypte, fais tes bagages pour la captivité ! Car Noph sera un désert, elle sera brûlée, elle n'aura plus d'habitants.
\VS{20}L'Egypte est une très belle génisse… la destruction vient, elle vient du nord.
\VS{21}Memes les mercenaires aussi sont au milieu d'elle comme des veaux engraissés. Et eux aussi tournent le dos, ils fuient tous sans résister. Car le jour de leur malheur, le temps de leur châtiment est venu sur eux.
\VS{22}Elle sifflera comme un serpent ; car ils marcheront avec une puissante armée, ils viendront contre elle avec des haches, comme des bûcherons.
\VS{23}Ils couperont sa forêt, dit Yahweh, quoiqu'elle soit impénétrable ; parce que leur armée est en plus grand nombre que les sauterelles, on ne saurait la compter.
\VS{24}La fille de l'Egypte est confuse, elle est livrée entre les mains du peuple du nord.
\VS{25}Yahweh des armées, le Dieu d'Israël, dit : Voici, je vais punir Amon de No, Pharaon, l'Egypte, ses dieux et ses rois, Pharaon et ceux qui se confient en lui.
\VS{26}Et je les livrerai entre les mains de ceux qui cherchent leur vie, entre les mains, dis-je, de Nebucadnetsar, roi de Babylone, et entre les mains de ses serviteurs ; mais après cela, l'Egypte sera habitée comme aux temps passés, dit Yahweh.
\VS{27}Et toi, Jacob, mon serviteur, ne crains pas ; ne t'épouvante pas Israël ! Car voici, je te sauverai de la terre lointaine, je sauverai ta postérité du pays de leur captivité ; Jacob reviendra, il sera en repos et en paix, et il n'y aura personne qui lui fasse peur.
\VS{28}Toi donc,Jacob, mon serviteur, ne crains pas ! dit Yahweh ; car je suis avec toi. Et même je consumerai entièrement  toutes les nations parmi lesquelles je t'ai chassé, mais je ne te consumerai pas entièrement ; et je te châtierai avec justice, je ne te tiendrai pas tout à fait pour innocent.
\Chap{47}
\TextTitle{Prophétie contre la Philistie et la Phénicie}
\VerseOne{}La parole de Yahweh, qui fut adressée à Jérémie, le prophète, contre les Philistins, avant que Pharaon frappe Gaza.
\VS{2}Ainsi parle Yahweh : Voici des eaux montent du nord, elles sont comme un torrent qui déborde ; elles inondent le pays et ce qu'il contient, les villes et leurs habitants. Les hommes poussent des cris, et tous les habitants du pays se lamentent,
\VS{3}à cause du bruit des battements de sabots de ses puissants chevaux, du bruit de ses chars et au son de ses roues ; les pères ne se tournent pas vers leurs fils, tant les mains sont affaiblies,
\VS{4}parce que le jour vient où seront détruits tous les Philistins, exterminés tout le reste de ceux qui servaient de secours à Tyr et à Sidon ; car Yahweh va détruire les Philistins, les restes de l'île de Caphtor.
\VS{5}Gaza est devenue chauve, Askalon est perdue, le reste de leur plaine aussi. Jusqu'à quand te feras-tu des incisions ?
\VS{6}Ah ! Epée de Yahweh, quand te reposeras-tu ? Rentre dans ton fourreau, repose-toi, et sois tranquille !
\VS{7}Mais comment te reposerais-tu ? Car Yahweh lui donne ses ordres, il l'a assignée contre Askalon et contre le rivage de la mer.
\Chap{48}
\TextTitle{Prophétie sur Moab}
\VerseOne{}Sur Moab. Ainsi parle Yahweh des armées, le Dieu d'Israël : Malheur à Nebo, car elle est dévastée ! Kirjathaïm est honteuse, elle est prise ; Misgab est honteuse et brisée.
\VS{2} Moab ne se glorifiera plus à Hesbon, car on a machiné du mal contre elle en disant : Allons, exterminons-la, qu'elle ne soit plus une nation ! Toi aussi, Madmen, tu seras détruite ; l'épée te poursuivra.
\VS{3}Il y a un bruit de clameur qui vient de Choronaïm ; c'est un ravage, une grande ruine.
\VS{4}Moab est brisé ! On entend les cris des plus jeunes.
\VS{5}Pleurs sur pleurs s’élèveront à la montée de Luchith, car on entendra à la descente de Choronaïm\FTNT{Es. 15:5.}. ceux qui crieront à cause des plaies que les ennemis leur auront faites.
\VS{6}Fuyez, dira-t-on, sauvez vos vies, et soyez comme un misérable dans le désert !
\VS{7}Car, parce que tu as eu confiance dans tes ouvrages, et dans tes trésors, tu seras pris, et Kemosch sortira pour être transporté, avec ses sacrificateurs et ses chefs\FTNT{Es. 46:1-7.}.
\VS{8}Et le dévastateur entrera dans toutes les villes, et aucune ville n'échappera ; la vallée périra et la plaine sera détruite, comme Yahweh l'a dit.
\VS{9}Donnez des ailes à Moab, et qu'il parte en volant ! Ses villes seront réduites en désert, elles n'auront plus d'habitants.
\VS{10}Maudit soit celui qui fait l'œuvre de Yahweh avec paresse, maudit soit celui qui garde son épée pour répandre le sang !
\VS{11}Moab était tranquille depuis sa jeunesse, il reposait sur sa lie, il n'était pas vidé de vase en vase, et il n'allait pas en captivité. C'est pourquoi son goût lui est resté, et son odeur ne s'est pas changée.
\VS{12}Mais voici, les jours viennent, dit Yahweh, où je lui enverrai des gens qui le transvaseront, qui videront ses vases, et qui briseront ses outres.
\VS{13}Moab aura honte à cause de Kemosch, comme la maison d'Israël a eu honte à cause de Béthel, qui était sa confiance.
\VS{14}Comment dites-vous : Nous sommes de vaillants hommes, des soldats prêts à combattre ?
\VS{15}Moab est dévasté, et chacune de ses villes monte en fumée, l'élite de sa jeunesse est descendue pour être égorgée, dit le Roi, dont le nom est Yahweh des armées.
\VS{16}La calamité de Moab est proche, son malheur avance à grands pas.
\VS{17}Vous tous qui êtes autour de lui, soyez-en émus à compassion, et vous tous qui connaissez son nom, dites : Comment a été rompue cette forte verge, et ce sceptre d'honneur ? 
\VS{18}Toi qui te tiens chez la fille de Dibon, descends de ta gloire, et assieds-toi dans un lieu desséché ! Car le dévastateur de Moab monte contre toi, il détruit tes forteresses.
\VS{19}Habitante d'Aroër, tiens-toi sur le chemin, et regarde ! Interroge celui qui s'enfuit, celui qui s'échappe, et dis : Qu'est-il arrivé ?
\VS{20}Moab est rendu honteux, car il est brisé. Poussez des gémissements et des cris\FTNT{Es. 15:5 ; Es. 16:7.} ! Rapportez dans Arnon que Moab est dévasté !
\VS{21}Et que jugement est venu sur le pays de la plaine, sur Holon, sur Jahats, sur Méphaath,
\VS{22}Et sur Dibon, sur Nebo, sur Beth-Diblathaïm,
\VS{23}Et sur Kirjathaïm, sur Beth-Gamul, sur Beth-Meon,
\VS{24}Et sur Kerijoth, sur Botsra, sur toutes les villes du pays de Moab, éloignées et proches.
\VS{25}La force de Moab est abattue, et son bras est brisé, dit Yahweh.
\VS{26}Enivrez-le, car il s'est élevé contre Yahweh ! Moab se vautrera dans le vin qu'il aura rendu et deviendra aussi un sujet de moquerie !
\VS{27}Car ô Moab! Israël n'a-t-il pas été pour toi un objet de moquerie ? Avait-il été trouvé parmi les voleurs, pour que tu ne dises des paroles qu'en secouant la tête ?
\VS{28}Habitants de Moab, quittez les villes, et demeurez dans les rochers ! Soyez comme les colombes qui font leur nid aux côtés de l'entrée des cavernes !
\VS{29}Nous avons appris l'extrême orgueil de Moab, son arrogance, sa fierté, et son cœur hautain\FTNT{Es. 16:6 ; So. 2:9-10.}.
\VS{30}J'ai connu son orgueil, dit Yahweh ; mais il n'en sera pas ainsi ; j'ai connu ceux sur lesquels il s'appuie ; ils n'ont rien fait de droit. 
\VS{31}Je hurlerai donc à cause de Moab, même je crierai à cause de Moab tout entier ; on gémira sur les gens de Kir-Hérès.
\VS{32}O vignoble de Sibma, je pleurerai sur toi du pleur de Jaezer ; tes rameaux allaient au-delà de la mer, ils atteignaient la mer de Jaezer ; le dévastateur s'est jeté sur tes fruits d'été et sur ta vendange.
\VS{33}L'allégresse ausi et la et la joie se sont retirées loin des campagnes et du pays de Moab ; j'ai fait cesser le vin des cuves ; on ne foule plus gaîment au pressoir ; il y a des cris de guerre, et non des cris de joie\FTNT{Es. 16:10.}.
\VS{34}A cause de cris de Hesbon qui est parvenu jusqu'à Elealé, et ils font entendre leurs cris jusqu'à Jahats, même depuis Tsoar jusqu'à Choronaïm, jusqu'à Eglath-Schelischija ; car les eaux de Nimrim seront aussi réduites en désolation.
\VS{35}Je ferai cesser en Moab, dit Yahweh, celui qui offre sur les hauts lieux et celui qui brûle de l’encens à ses dieux. 
\VS{36}C’est pourquoi mon cœur mènera un bruit sur Moab, comme des flûtes ; mon cœur mènera un bruit comme des flûtes sur les hommes de Kir-Hérès, parce que tous les biens qu'ils ont acquis ont péri.
\VS{37}Car toutes les têtes sont chauves, toutes les barbes sont coupées ; et il y a des incisions sur toutes les mains, et sur les reins des sacs.
\VS{38}Il y aura des lamentations sur tous les toits de Moab et dans ses places, parce que j'aurai brisé Moab comme un vase auquel on ne prend nul plaisir, dit Yahweh.
\VS{39}Hurlez, en disant : Comment a-t-il été mis en pièces ? brisé ! Comment Moab a-t-il tourné honteusement le dos ! Car Moab sera un objet de moquerie et de frayeur pour tous ceux qui sont autour de lui.
\VS{40}Car ainsi parle Yahweh : Voici, il volera comme un aigle, et il étendra ses ailes sur Moab.
\VS{41}Kerijoth est prise, les forteresses sont saisies, et le cœur des hommes forts de Moab est en ce jour comme le cœur d'une femme qui est en travail.
\VS{42}Et Moab sera exterminé, il ne sera plus un peuple, parce qu'il s'est élevé contre Yahweh.
\VS{43}Habitant de Moab, la frayeur, la fosse, et le filet sont sur toi ! dit Yahweh.
\VS{44}Celui qui s'enfuira à cause de la frayeur tombera dans la fosse, et celui qui remontera de la fosse sera au filet ; car je fera venir sur lui, sur Moab, l'année de son châtiment, dit Yahweh\FTNT{Es. 24:18.}.
\VS{45}Ils se sont arrêtés à l'ombre de Hesbon, voulant éviter la force ; mais le feu sort de Hesbon, une flamme du milieu de Sihon ; elle dévore les flancs de Moab, et le sommet de la tête des fils du tumulte\FTNT{No. 21:28.}.
\VS{46}Malheur à toi, Moab ! Le peuple de Kemosch est perdu ! Car tes fils sont enlevés et emmenés captifs, et tes filles ont été emmenées captives.
\VS{47}Toutefoi je ramènerai et mettrai en repos les captifs de Moab, aux derniers jours\FTNT{Gn. 49:1-2.}, dit Yahweh. Là est le jugement de Moab.
\Chap{49}
\TextTitle{Prophétie sur Ammon}
\VerseOne{}Sur les fils d'Ammon. Ainsi parle Yahweh : Israël n'a-t-il pas de fils ? N'a-t-il pas d'héritier ? Pourquoi donc Malcom hérite-t-il de Gad, et pourquoi son peuple demeure-t-il dans ses villes ?
\VS{2}C'est pourquoi, les jours viennent, dit Yahweh, où je ferai entendre le cri de guerre contre Rabbath des fils d'Ammon ; elle sera réduite en un monceau de ruines, et les villes de son ressort seront brûlées par le feu ; Israël possédera ceux qui l'auront possédé, dit Yahweh.
\VS{3}Hurle, ô Hesbon, car Aï est dévastée ! Poussez des cris, filles de Rabba, ceignez-vous de sacs, lamentez-vous, courez ça et là le long des murailles ! Car Malcom s'en va en captivité avec ses sacrificateurs et ses chefs.
\VS{4}Pourquoi te glorifies-tu de tes vallées ? Ta vallée se fond, fille rebelle, qui te confiais dans tes trésors : Qui viendra contre moi ?
\VS{5}Voici, je fais venir sur toi la terreur, dit le Seigneur, Yahweh des armées, de tous les alentours ; vous serez chassés chacun çà et là, et il n'y aura personne qui rassemblera les fuyards.
\VS{6}Mais après cela, je ramènerai les captifs des fils d'Ammon, dit Yahweh.
\TextTitle{Prophétie sur Edom}
\VS{7}Sur Edom. Ainsi parle Yahweh des armées : N'y a-t-il plus de sagesse dans Théman ? Le conseil a-t-il manqué aux hommes intelligents ? Leur sagesse s'est-elle évanouie\FTNT{Ab. 1:8.} ?
\VS{8}Fuyez, tournez le dos, demeurez dans les cavernes, habitants de Dedan ! Car je fais venir la détresse sur Esaü, le temps de son châtiment.
\VS{9}Si des vendangeurs entrent chez toi, ne laissent-ils rien à grappiller ? Si des voleurs viennent de nuit, ils ne pillent que ce qu'ils peuvent.
\VS{10}Mais je dépouillerai Esaü, je découvrirai ses lieux secrets, il ne pourra se cacher ; sa postérité, ses frères, et ses voisins, périront, et il ne sera plus.
\VS{11}Laisse tes orphelins, je les ferai vivre, et que tes veuves se confient en moi !
\VS{12}Car ainsi parle Yahweh : Voici, ceux dont le jugement n'était pas de boire la coupe, la boiront certainement ; et toi, tu resterais impuni ! Tu ne resteras pas impuni, tu la boiras.
\VS{13}Car je le jure par moi-même, dit Yahweh, que Botsra sera un objet de désolation, d'opprobre, de dévastation, et de malédiction, et que toutes ses villes deviendront des ruines éternelles.
\VS{14}J'ai entendu de Yahweh une nouvelle, et un messager a été envoyé parmi les nations : Assemblez-vous, et venez contre elle ! Levez-vous pour la guerre !
\VS{15}Car voici, je te rendrai petit entre les nations, méprisé entre les hommes.
\VS{16}Mais ta présomption, l'orgueil de ton cœur t'a séduit, toi qui habites dans le creux des rochers, et qui occupes le sommet des collines. Quand tu aurais élevé ton nid comme l'aigle, je t'en ferai descendre, dit Yahweh.
\VS{17}Edom sera un objet de désolation ; quiconque passera près de lui sera étonné, et sifflera à cause de toutes ses plaies.
\VS{18}Comme Sodome et Gomorrhe et les villes voisines qui furent détruites, dit Yahweh, il ne sera plus habité par des hommes, il ne sera le séjour d'aucun fils d'homme\FTNT{Ge. 19:25 ; Am. 4:11.}…
\VS{19}Voici, il monte comme un lion des rives orgueilleuses du Jourdain, vers la demeure forte ; soudain, j'en ferai fuir Edom, et j'établirai sur elle celui que j'ai choisi. Car qui est semblable à moi ? Qui me donnera des ordres ? Et quel est le chef qui me résistera en face\FTNT{Job. 41:1.} ?
\VS{20}C'est pourquoi écoutez le conseil que Yahweh a donné contre Edom, et les desseins qu'il a projetés contre les habitants de Théman ! Certainement, on les traînera comme les plus petits du troupeau, certainement on dévastera leur demeure.
\VS{21}La terre tremble au bruit de leur chute ; le bruit de leur cri se fait entendre jusqu'à la Mer Rouge…
\VS{22}Voici, il monte comme un aigle, il vole, il étend ses ailes sur Botsra, et le cœur des hommes forts d'Edom est en ce jour comme le cœur d'une femme en travail.
\TextTitle{Prophétie sur Damas}
\VS{23}Sur Damas. Hamath et Arpad sont honteuses parce qu'elles ont entendu des mauvaises nouvelles, elles tremblent ; il y a une tourmente dans la mer qui ne peut se calmer.
\VS{24}Damas est défaillante, elle se tourne pour fuir, et la panique la saisit ; l'angoisse et les douleurs la saisissent comme une femme qui enfante.
\VS{25}Ah ! Elle n'est pas abandonnée, la ville glorieuse, ma ville de plaisance !
\VS{26}C'est pourquoi ses jeunes gens tomberont dans les places, et tous ses hommes de guerre périront en ce jour, dit Yahweh des armées.
\VS{27}Je mettrai le feu à la muraille de Damas, qui dévorera les palais de Ben-Hadad.
\TextTitle{Prophétie sur Kédar (les Arabes) et Hatsor}
\VS{28}Sur Kédar et les royaumes de Hatsor, que Nebucadnetsar, roi de Babylone, frappa. Ainsi parle Yahweh : Levez-vous, montez vers Kédar, et détruisez les fils d'orient !
\VS{29}On prendra leurs tentes et leurs troupeaux, on prendra leurs tentes, tous leurs bagages et leurs chameaux, et l'on jettera de toutes parts contre eux des cris de terreur.
\VS{30}Fuyez, fuyez de toutes vos forces, cherchez une demeure dans les cavernes, vous habitants de Hatsor ! dit Yahweh ; car Nebucadnetsar, roi de Babylone, a pris une résolution contre vous, il a imaginé un plan contre vous.
\VS{31}Levez-vous, montez vers la nation tranquille qui habite en sécurité, dit Yahweh ; elle n'a ni portes ni barres, elle habite seule\FTNT{Ez. 38:11.}.
\VS{32}Leurs chameaux seront au pillage, et la multitude de leur bétail sera une proie ; je les disperserai à tout vent, vers ceux qui se coupent le coin de la barbe, et je ferai venir de tous les côtés leur détresse, dit Yahweh.
\VS{33}Hatsor sera le repaire des serpents, un désert pour toujours ; personne n'y habitera, et aucun fils d'homme n'y séjournera.
\TextTitle{Prophétie sur Elam}
\VS{34}La parole de Yahweh fut adressée à Jérémie, le prophète, sur Elam, au commencement du règne de Sédécias, roi de Juda, en disant :
\VS{35}Ainsi parle Yahweh des armées : Voici, je vais briser l'arc d'Elam, qui est sa principale force\FTNT{Ez. 32:24-27.}.
\VS{36}Je ferai venir contre Elam les quatre vents des quatre extrémités des cieux, je les disperserai par tous ces vents ; et il n'y aura pas une nation où ne viennent ceux qui seront chassés d'Elam éternellement.
\VS{37}Je ferai trembler les habitants d'Elam devant leurs ennemis, et devant ceux qui cherchent leur vie, je ferai venir le malheur sur eux, l'ardeur de ma colère, dit Yahweh, et j'enverrai l'épée après eux, jusqu'à ce que je les aie consumés.
\VS{38}Je mettrai mon trône dans Elam, et j'en détruirai les rois et les chefs, dit Yahweh.
\VS{39}Mais dans les derniers jours\FTNT{Gn. 49:1-2.}, je ramènerai les captifs d'Elam, dit Yahweh.
\Chap{50}
\TextTitle{Prophétie sur Babylone}
\VerseOne{}La parole que Yahweh prononça sur Babylone, sur le pays des Chaldéens, par le moyen de Jérémie, le prophète :
\VS{2}Annoncez-le parmi les nations, entendez-le, levez une bannière ! Entendez-le, ne le cachez pas ! Dites : Babylone est prise ! Bel est confus, Merodac est brisé ! Ses idoles sont confuses et brisées\FTNT{Es. 46:1.} !
\VS{3}Car une nation monte contre elle du nord, qui mettra son pays en ruines, il n'y aura plus personne qui y habite ; les hommes et les bêtes fuient, ils s'en vont.
\VS{4}En ces jours, et en ce temps-là, dit Yahweh, les fils d'Israël et les fils de Juda viendront ensemble ; ils marcheront en pleurant, et en cherchant Yahweh, leur Dieu.
\VS{5}Ils demanderont la route de Sion, ils tourneront leur visage vers elle : Venez, attachez-vous à Yahweh, par une alliance éternelle qui ne soit jamais oubliée !
\VS{6}Mon peuple était comme un troupeau de brebis perdues ; leurs bergers les égaraient, les rendaient errantes par les montagnes ; elles allaient de montagne en colline, oubliant leur bercail\FTNT{Ez. 34:5-6 ; Za. 10:2 ; Mt. 9:36.}.
\VS{7}Tous ceux qui les trouvaient les dévoraient, et leurs ennemis disaient : Nous ne sommes coupables d'aucun mal, parce qu'ils ont péché contre Yahweh, contre la demeure de la justice, contre Yahweh, l'espérance de leurs pères.
\VS{8}Fuyez du milieu de Babylone, sortez du pays des Chaldéens, et soyez comme les boucs qui vont devant le troupeau\FTNT{Es. 48:20 ; 2 Co. 6:17 ; Ap. 18:4.} !
\VS{9}Car voici, je vais susciter et faire monter contre Babylone une multitude de grandes nations du pays du nord ; elles se rangeront en bataille contre elle, de sorte qu'elle sera prise ; leurs flèches font des ravages comme celles d'un habile guerrier qui ne retourne pas à vide\FTNT{Es. 13:18.}.
\VS{10}Et la Chaldée sera abandonnée au pillage ; tous ceux qui la pilleront seront rassasiés, dit Yahweh.
\VS{11}Oui, réjouissez-vous, soyez dans l'allégresse, vous qui avez pillé mon héritage ! Oui, bondissez comme une génisse qui est dans l'herbe, hennissez comme de puissants chevaux !
\VS{12}Votre mère est fort honteuse, celle qui vous a enfantés rougit de honte ; voici, elle est la dernière entre les nations, c'est un désert, un pays sec et aride.
\VS{13}Elle ne sera plus habitée à cause de la colère de Yahweh, elle ne sera plus qu'une désolation. Quiconque passera près de Babylone sera étonné et sifflera à cause de toutes ses plaies.
\VS{14}Rangez-vous en bataille contre Babylone, mettez-vous tout alentour vous tous qui tendez l'arc ! Tirez contre elle, n'épargnez pas les flèches ! Car elle a péché contre Yahweh.
\VS{15}Poussez des cris de guerre contre elle tout alentour ! Elle tend les mains ; ses fondements tombent ; ses murs sont renversés. Car c'est ici la vengeance de Yahweh. Vengez-vous sur elle ! Faites-lui comme elle a fait\FTNT{Ab. 1:15 ; Ps. 137:8 ; Lu. 6:38.} !
\VS{16}Retranchez de Babylone le semeur, et celui qui manie la faucille au temps de la moisson ! Devant l'épée de l'oppresseur, que chacun se tourne vers son peuple, que chacun s'enfuie vers son pays.
\VS{17}Israël est comme une brebis égarée que les lions ont chassée ; le roi d'Assyrie l'a dévorée le premier ; mais ce dernier, Nebucadnetsar, roi de Babylone, lui a brisé les os.
\VS{18}C'est pourquoi ainsi parle Yahweh des armées, le Dieu d'Israël : Voici, je punirai le roi de Babylone et son pays, comme j'ai puni le roi d'Assyrie\FTNT{Es. 37:36 ; 2 R. 19:35.}.
\VS{19}Je ramènerai Israël dans sa demeure ; il paîtra au Carmel et au Basan, et son âme se rassasiera sur la montagne d'Ephraïm et de Galaad.
\VS{20}En ces jours, et en ce temps-là, dit Yahweh, on cherchera l'iniquité d'Israël, mais il n'y en aura pas, les péchés de Juda ne seront pas trouvés ; car je pardonnerai au reste que j'aurai fait demeurer.
\VS{21}Monte contre ce pays doublement rebelle, contre les habitants, et châtie-les ! Massacre, extermine-les ! dit Yahweh, fais selon toutes les choses que je t'ai ordonnées.
\VS{22}Des cris de guerre retentissent dans le pays, et la ruine est grande.
\VS{23}Eh quoi ! Il est rompu, brisé, le marteau de toute la terre ! Babylone est réduite en une désolation parmi les nations !
\VS{24}Je t'ai tendu un piège, et tu as été prise, Babylone, et tu n'en savais rien ; tu as été trouvée, et même attrapée, parce que tu as lutté contre Yahweh.
\VS{25}Yahweh a ouvert son arsenal et en a sorti les armes de sa colère ; c'est là une œuvre du Seigneur, de Yahweh des armées, dans le pays des Chaldéens.
\VS{26}Venez de toutes parts dans Babylone, ouvrez ses greniers, faites-y des monceaux, comme des tas de gerbes, et détruisez-la ! Qu'il ne reste plus rien d'elle !
\VS{27}Tuez tous ses taureaux et qu'ils descendent à l'abattage ! Malheur à eux ! Car le jour est venu, le temps de leur châtiment.
\VS{28}Ecoutez la voix de ceux qui s'enfuient, de ceux qui sont échappés du pays de Babylone pour annoncer dans Sion la vengeance de Yahweh, notre Dieu, la vengeance de son temple !
\VS{29}Appelez les archers contre Babylone, vous tous qui tendez l'arc ! Campez-vous contre elle tout alentour, que personne n'échappe, rendez-lui selon ses œuvres, faites-lui selon tout ce qu'elle a fait ! Car elle s'est fièrement élevée contre Yahweh, contre le Saint d'Israël\FTNT{Es. 13:11 ; Joë. 3:4-9 ; La. 1:22.}.
\VS{30}C'est pourquoi ses jeunes gens tomberont dans les places, et tous ses hommes de guerre périront en ce jour, dit Yahweh.
\VS{31}Voici, j'en veux à toi, orgueilleuse ! dit le Seigneur, Yahweh des armées ; car ton jour est venu, le temps de ton châtiment.
\VS{32}L'orgueilleuse chancellera et tombera, et il n'y aura personne pour la relever ; je mettrai le feu à ses villes, et il dévorera tous ses environs.
\VS{33}Ainsi parle Yahweh des armées : Les fils d'Israël et les fils de Juda sont ensemble opprimés ; tous ceux qui les ont emmenés captifs les retiennent, et refusent de les laisser aller.
\VS{34}Leur Rédempteur est fort, son nom est Yahweh des armées ; il défendra certainement leur cause, pour donner du repos au pays, et pour faire trembler les habitants de Babylone.
\VS{35}L'épée est sur les Chaldéens ! dit Yahweh, sur les habitants de Babylone, ses chefs, et ses sages !
\VS{36}L'épée est tirée contre ses devins de mensonges ! Qu'ils soient comme des insensés ! L'épée contre ses hommes forts ! Qu'ils soient épouvantés !
\VS{37}L'épée est sur ses chevaux et sur ses chars ! Contre les foules de toute espèce qui sont au milieu d'elle ! Qu'ils deviennent comme des femmes ! L'épée est sur ses trésors ! Qu'ils soient pillés !
\VS{38}La sécheresse est sur ses eaux ! Qu'elles soient mises à sec ! Parce que c'est un pays d'idoles ; ils agissent en insensés à l'égard de leurs idoles\FTNT{Es. 2:8.}.
\VS{39}C'est pourquoi les animaux du désert y habiteront avec les chacals, et les autruches y habiteront aussi ; elle ne sera plus jamais habitée, et on n'y demeurera plus jamais.
\VS{40}Comme Sodome et Gomorrhe, et les villes voisines que Dieu détruisit, dit Yahweh, elle ne sera plus habitée par des hommes, elle ne sera le séjour d'aucun fils d'homme.
\VS{41}Voici, un peuple vient du nord, une grande nation et plusieurs rois se lèvent des extrémités de la terre.
\VS{42}Ils saisissent l'arc et le javelot ; ils sont cruels, et ils n'ont pas de compassion ; leur voix mugit comme la mer ; ils sont montés sur des chevaux, chacun d'eux est rangé en bataille comme un seul homme, contre toi, fille de Babylone !
\VS{43}Le roi de Babylone entend la nouvelle, et ses mains s'affaiblissent, l'angoisse le saisit comme la douleur de celle qui enfante…
\VS{44}Voici, il monte comme un lion des rives orgueilleuses du Jourdain vers la demeure forte ; soudain, je les ferai courir, et je désignerai sur elle celui que j'ai choisi. Car qui est semblable à moi ? Qui me donnera des ordres ? Et quel est le chef qui me résistera en face ?
\VS{45}C'est pourquoi écoutez le conseil que Yahweh a donné contre Babylone, et les desseins qu'il a projetés contre le pays des Chaldéens ! Certainement, on les traînera comme les plus petits du troupeau, certainement on dévastera leur demeure.
\VS{46}La terre tremble au bruit de la prise de Babylone, et le cri se fait entendre parmi les nations.
\Chap{51}
\TextTitle{Le jugement de Babylone par Yahweh}
\VerseOne{}Ainsi parle Yahweh : Voici, je fais lever un vent destructeur contre Babylone et contre ceux qui habitent au cœur du royaume.
\VS{2}J'envoie contre Babylone des vanneurs qui la vanneront, qui videront son pays ; car de tous côtés ils seront contre elle, au jour du malheur.
\VS{3}Qu'on bande l'arc contre celui qui bande son arc, contre celui qui s'élève dans son armure ! N'épargnez pas ses jeunes hommes ! Exterminez toute son armée !
\VS{4}Qu'ils tombent les blessés à mort dans le pays des Chaldéens, percés de coups dans les rues de Babylone !
\VS{5}Car Israël et Juda ne sont pas abandonnés de leur Dieu, de Yahweh des armées, quoique leur pays ai été trouvé par le Saint d'Israël plein de crimes.
\VS{6}Fuyez hors de Babylone, et que chacun sauve sa vie, ne soyez point exterminés dans son iniquité ; car c'est le temps de la vengeance de Yahweh ; il lui rend ce qu'elle a mérité.
\VS{7}Babylone était comme une coupe d'or dans la main de Yahweh, enivrant toute la terre ; les nations ont bu de son vin : C'est pourquoi les nations ont agi comme des insensées.
\VS{8}Babylone est tombée\FTNT{Ap. 18.} en un instant, elle est brisée ! Gémissez sur elle, prenez du baume pour sa douleur : Peut-être qu'elle guérira.
\VS{9}Nous avons pansé Babylone, mais elle n'a pas guéri. Laissons-la et allons-nous-en chacun dans son pays ; car son jugement atteint les cieux et s'élève jusqu'aux nues.
\VS{10}Yahweh a rendu la justice de notre cause ; venez et racontons dans Sion l'œuvre de Yahweh, notre Dieu.
\VS{11}Aiguisez les flèches, remplissez vos mains avec les boucliers ! Yahweh a réveillé l'esprit des rois de Médie, car sa pensée est de détruire Babylone ; c'est ici la vengeance de Yahweh, la vengeance de son temple.
\VS{12}Elevez une bannière contre les murs de Babylone ! Fortifiez les postes, levez des gardes, préparez des embuscades ! Car Yahweh a formé un projet, il fait ce qu'il a dit contre les habitants de Babylone.
\VS{13}Toi qui habites près des grandes eaux, abondantes en trésors, ta fin est venue, ta cupidité est à son terme !
\VS{14}Yahweh des armées a juré par lui-même, en disant : Je te remplirai d'hommes comme de sauterelles, et ils pousseront contre toi des cris de guerre.
\VS{15}C'est lui qui a fait la terre par sa puissance, qui a fondé le monde habitable par sa sagesse, et qui a étendu les cieux par son intelligence\FTNT{Ge. 1:1 ; Es. 40:22 ; Ps. 104:2; Job. 9:8.}.
\VS{16}Lorsqu'il donne de la voix, il y a un tumulte d'eaux dans les cieux, il fait monter les vapeurs des extrémités de la terre, il fait les éclairs et la pluie, il fait sortir le vent de ses réservoirs.
\VS{17}Tout homme devient stupide par sa connaissance, tout fondeur est honteux par les images taillées ; car ses idoles en métal fondu ne sont que mensonge, il n'y a pas de souffle en elles.
\VS{18}Elles ne sont que vanité, une œuvre de tromperie ; elles périront au temps de leur châtiment.
\VS{19}La portion de Jacob n'est pas comme ces choses-là ; car c'est lui qui a tout formé, et Israël est la tribu de son héritage. Son nom est Yahweh des armées.
\VS{20}Tu as été pour moi un marteau, un instrument de guerre. Par toi j'ai brisé des nations, par toi j'ai détruit des royaumes.
\VS{21}Par toi j'ai brisé le cheval et son cavalier ; par toi j'ai brisé le char et celui qui était monté dessus.
\VS{22}Par toi j'ai brisé l'homme et la femme ; par toi j'ai brisé le vieillard et le jeune garçon ; par toi j'ai brisé le jeune homme et la jeune fille.
\VS{23}Par toi j'ai brisé le berger et son troupeau ; par toi j'ai brisé le laboureur et ses bœufs ; par toi j'ai brisé les gouverneurs et les chefs.
\VS{24}Mais je rendrai à Babylone et à tous les habitants de la Chaldée tout le mal qu'ils ont fait à Sion sous vos yeux, dit Yahweh\FTNT{La. 1:21.}.
\VS{25}Voici, j'en veux à toi, montagne de destruction, dit Yahweh, à toi qui détruisais toute la terre ! J'étendrai ma main sur toi, je te roulerai du haut des rochers, je ferai de toi une montagne embrasée.
\VS{26}On ne pourra prendre de toi aucune pierre pour la placer à l'angle de l'édifice ni aucune pierre pour servir de fondement ; car tu seras une ruine éternelle, dit Yahweh\FTNT{Es. 13:19-20.}…
\VS{27}Elevez une bannière dans le pays ! Sonnez du shofar parmi les nations ! Préparez les nations contre elle, appelez contre elle les royaumes d'Ararat, de Minni et d'Aschkenaz ! Établissez contre elle des chefs ! Faites monter ses chevaux comme des sauterelles hérissées !
\VS{28}Préparez contre elle les nations, les rois de Médie, ses gouverneurs et tous ses chefs, et tout le pays sous leur domination\FTNT{Es. 13:17.} !
\VS{29}La terre tremble, elle se tord ; car la pensée de Yahweh se dresse contre Babylone ; il va faire du pays de Babylone un désert sans habitants\FTNT{Es. 13:14 ; Joë. 3:16.}.
\VS{30}Les hommes forts de Babylone cessent de combattre, ils demeurent dans les forteresses ; leur force est épuisée, ils sont comme des femmes. On met le feu aux demeures, on brise les barres.
\VS{31}Le courrier rencontre le courrier, et le messager rencontre le messager, pour annoncer au roi de Babylone que sa ville est prise par tous les côtés,
\VS{32}que les gués sont saisis, les marais brûlés par le feu, et les hommes de guerre épouvantés.
\VS{33}Car ainsi parle Yahweh des armées, le Dieu d'Israël : La fille de Babylone est comme une aire dans le temps où on la foule ; encore un peu de temps, et le moment de la moisson sera venu pour elle.
\VS{34}Nebucadnetsar, roi de Babylone, m'a dévorée, m'a détruite ; il a fait de moi un vase vide ; il m'a engloutie tel un dragon, il a rempli son ventre de mes délices ; il m'a chassée au loin.
\VS{35}Que la violence envers moi et ma chair déchirée retombe sur Babylone ! dit l'habitante de Sion. Que mon sang retombe sur les habitants de la Chaldée ! dit Jérusalem.
\VS{36}C'est pourquoi ainsi parle Yahweh : Voici, je défendrai ta cause, je te vengerai ! Je dessécherai la mer de Babylone, et je ferai tarir sa source.
\VS{37}Babylone sera un monceau de ruines, un repaire de serpents, un objet d'épouvante et de moquerie ; sans que personne n'y habite.
\VS{38}Ils rugiront ensemble comme des lions, ils pousseront des cris comme des lionceaux.
\VS{39}Quand ils seront échauffés, je les ferai boire, et les enivrerai, afin qu'ils se réjouissent et qu'ils dorment d'un sommeil éternel et qu'ils ne se réveillent plus, dit Yahweh.
\VS{40}Je les ferai descendre comme des agneaux à la boucherie, comme des béliers et des boucs.
\VS{41}Eh quoi ! Schéschac est prise ! Celle dont la louange remplissait toute la terre est conquise ! Eh quoi ! Babylone est réduite en désolation parmi les nations !
\VS{42}La mer est montée sur Babylone, elle a été couverte de la multitude de ses flots\FTNT{Es. 8:8 ; Ez. 26:3-19 ; Lu. 21:25.}.
\VS{43}Ses villes sont en ruines, une terre sèche et déserte ; c'est un pays où personne ne demeure, et où il ne passe aucun fils d'homme.
\VS{44}Je punirai aussi Bel à Babylone, je sortirai de sa bouche ce qu'il a englouti, et les nations n'aborderont plus vers lui. Le mur même de Babylone est tombé !
\VS{45}Mon peuple, sortez du milieu d'elle, et que chacun sauve sa vie de l'ardeur de la colère de Yahweh.
\VS{46}Que votre cœur ne se trouble pas, et ne craignez pas les nouvelles qu'on entendra dans tout le pays ; car cette année viendra une nouvelle, et l'année d'après une autre nouvelle, et il y aura violence dans le pays, et un dominateur s'élèvera contre un autre dominateur.
\VS{47}C'est pourquoi voici, les jours viennent où je punirai les idoles de Babylone, et tout son pays sera honteux ; tous ses morts tomberont au milieu d'elle.
\VS{48}Les cieux, la terre, et tout ce qui y est, pousseront des cris de joie contre Babylone ; car du nord les dévastateurs viendront contre elle, dit Yahweh.
\VS{49}Babylone tombera, ô morts d'Israël, comme Babylone a fait tomber les morts de tout le pays.
\VS{50}Vous qui avez échappé à l'épée, allez, ne vous arrêtez pas ! Souvenez-vous de Yahweh dans ces pays éloignés, et que Jérusalem revienne à vos cœurs !
\VS{51}Nous étions honteux des reproches que nous entendions ; la honte couvrait nos visages, quand les étrangers sont venus dans le sanctuaire de la maison de Yahweh.
\VS{52}C'est pourquoi, voici, les jours viennent, dit Yahweh, où je châtierai ses idoles ; et les blessés gémiront dans tout son pays.
\VS{53}Quand Babylone monterait jusqu'aux cieux et qu'elle rendrait inaccessible le plus haut de sa forteresse, alors les dévastateurs viendront contre elle, dit Yahweh\FTNT{Am. 9:2 ; Ab. 1:4.}…
\VS{54}Un grand cri s'entend de Babylone, et la ruine est grande dans le pays des Chaldéens.
\VS{55}Parce que Yahweh dévaste Babylone, il en fait échapper de grands cris ; les flots des dévastateurs mugissent comme de grandes eaux, le bruit du mugissement s'étend.
\VS{56}Car le destructeur est venu contre elle, contre Babylone ; ses hommes forts sont pris, leurs arcs sont brisés. Car Yahweh est un Dieu qui rend à chacun selon ses œuvres, qui paie à chacun son salaire.
\VS{57}J'enivrerai ses princes et ses sages, ses gouverneurs, ses chefs, et ses hommes forts ; ils dormiront d'un sommeil éternel, et ils ne se réveilleront plus, dit le Roi dont le nom est Yahweh des armées.
\VS{58}Ainsi parle Yahweh des armées : Les larges murs de Babylone seront renversés, ses portes qui sont si hautes, seront brûlées par le feu ; ainsi les peuples auront travaillé en vain, les nations se seront fatiguées pour le feu.
\VS{59}C'est ici l'ordre que Jérémie, le prophète, donna à Seraja, fils de Nérija, fils de Machséja, lorsqu'il alla avec Sédécias, roi de Juda, la quatrième année du règne de Sédécias. Or Seraja était premier chambellan.
\VS{60}Jérémie écrivit dans un livre tous les malheurs qui devaient venir sur Babylone, toutes ces paroles qui sont écrites contre Babylone.
\VS{61}Jérémie dit à Seraja : Lorsque tu seras venu à Babylone, et que tu auras vu, tu liras toutes ces paroles,
\VS{62}et tu diras : Yahweh, c'est toi qui as déclaré que ce lieu serait exterminé, en sorte qu'il n'y ait aucun habitant, depuis l'homme jusqu'à la bête, mais qu'il deviendrait un désert pour toujours.
\VS{63}Et quand tu auras achevé de lire ce livre, tu le lieras à une pierre et tu le jetteras dans l'Euphrate,
\VS{64}et tu diras : Ainsi Babylone sera submergée, elle ne se lèvera pas des malheurs que je ferai venir sur elle ; ils seront épuisés. Jusqu'ici sont les paroles de Jérémie.
\Chap{52}
\TextTitle{Chute de Jérusalem et destruction du temple ; Juda déporté à Babylone\FTNTT{2 R. 25:1-26; Jé. 39:1-10.}}
\VerseOne{}Sédécias avait vingt et un ans lorsqu'il devint roi, et il régna onze ans à Jérusalem. Sa mère se nommait Hamuthal, fille de Jérémie, de Libna\FTNT{2 R. 24 et 25.}.
\VS{2}Il fit ce qui est mal aux yeux de Yahweh, comme avait fait Jojakim.
\VS{3}Et cela arriva à cause de la colère de Yahweh contre Jérusalem et Juda, qu'il voulait chasser de devant sa face. Sédécias se rebella contre le roi de Babylone.
\VS{4}La neuvième année de son règne, le dixième jour du dixième mois, Nebucadnetsar, roi de Babylone, vint contre Jérusalem, lui et toute son armée ; ils campèrent devant elle et construisirent des retranchements tout alentour.
\VS{5}La ville fut assiégée jusqu'à la onzième année du roi Sédécias.
\VS{6}Le neuvième jour du quatrième mois, la famine était forte dans la ville, et il n'y avait pas de pain pour le peuple du pays\FTNT{La. 2:11-12.}.
\VS{7}Alors la brèche fut faite à la ville ; et tous les gens de guerre s'enfuirent et sortirent de nuit hors de la ville par le chemin de la porte entre les deux murailles près du jardin du roi, tandis que les Chaldéens entouraient la ville. Ils s'en allèrent par le chemin de la plaine.
\VS{8}Mais l'armée des Chaldéens poursuivit le roi, et ils atteignirent Sédécias dans les plaines de Jéricho ; et toute son armée se dispersa loin de lui.
\VS{9}Ils prirent le roi, et le firent monter vers le roi de Babylone à Ribla, dans le pays de Hamath ; et il prononça contre lui une sentence.
\VS{10}Le roi de Babylone fit égorger les fils de Sédécias sous ses yeux ; il fit aussi égorger tous les chefs de Juda à Ribla.
\VS{11}Puis il fit crever les yeux à Sédécias, et le fit lier avec des chaînes d'airain ; le roi de Babylone l'emmena à Babylone, et le mit en prison jusqu'au jour de sa mort.
\VS{12}Le dixième jour du cinquième mois, c'était la dix-neuvième année du règne de Nebucadnetsar, roi de Babylone, Nebuzaradan, chef des gardes, qui se tenait devant le roi de Babylone, entra dans Jérusalem.
\VS{13}Il brûla la maison de Yahweh, la maison du roi, et toutes les maisons de Jérusalem ; il brûla toutes les maisons des personnes considérées.
\VS{14}Toute l'armée des Chaldéens, qui était avec le chef des gardes, renversa toutes les murailles qui entouraient Jérusalem.
\VS{15}Nebuzaradan, chef des gardes, transporta à Babylone une partie des plus pauvres du peuple, le reste du peuple qui était demeuré dans la ville, ceux qui s'étaient rendus au roi de Babylone, et le reste de la multitude.
\VS{16}Toutefois, Nebuzaradan, chef des gardes, laissa quelques-uns des plus pauvres du pays pour être vignerons et laboureur.
\VS{17}Les Chaldéens brisèrent les colonnes d'airain qui étaient dans la maison de Yahweh, les bases, la mer d'airain qui était dans la maison de Yahweh, et ils emmenèrent tout l'airain à Babylone.
\VS{18}Ils prirent les cendriers, les pelles, les couteaux, les coupes, les tasses, et tous les ustensiles d'airain avec lesquels on faisait le service.
\VS{19}Le chef des gardes prit aussi les coupes, les encensoirs, les cendriers, les chandeliers, les tasses et les calices, ce qui était d'or et ce qui était d'argent.
\VS{20}Les deux colonnes, la mer, et les douze bœufs d'airain qui servaient de base, et que le roi Salomon avait faits pour la maison de Yahweh, tous ces ustensiles d'airain ne pouvaient être pesés.
\VS{21}La hauteur de l'une des colonnes était de dix-huit coudées, et un cordon de douze coudées l'entourait ; elle était épaisse de quatre doigts, et creuse;
\VS{22}il y avait par-dessus un chapiteau d'airain, et la hauteur d'un des chapiteaux était de cinq coudées ; il y avait aussi un treillis et des grenades tout autour du chapiteau, le tout d'airain ; il en était de même pour la seconde colonne avec des grenades\FTNT{1 R. 7:15-20.}.
\VS{23}Il y avait quatre-vingt-seize grenades de chaque côté, et les grenades qui étaient autour du treillis étaient au nombre de cent.
\VS{24}Le chef des gardes prit Seraja, qui était le premier sacrificateur, Sophonie, qui était le second sacrificateur, et les trois gardiens du seuil.
\VS{25}Il prit de la ville un eunuque qui avait sous son commandement des hommes de guerre, sept hommes de ceux qui voyaient la face du roi et qui furent trouvés dans la ville, le secrétaire du chef de l'armée qui enrôlait le peuple du pays, et soixante hommes du peuple du pays, qui se trouvèrent dans la ville.
\VS{26}Nebuzaradan, chef des gardes, les prit et les emmena vers le roi de Babylone à Ribla.
\VS{27}Le roi de Babylone les frappa et les fit mourir à Ribla, dans le pays de Hamath. Ainsi Juda fut transporté hors de son pays.
\VS{28}Et c'est ici le peuple que Nebucadnetsar emmena en captivité : La septième année, trois mille vingt-trois juifs ;
\VS{29}la dix-huitième année de Nebucadnetsar, il emmena de Jérusalem huit cent trente-deux personnes ;
\VS{30}La vingt-troisième année de Nebucadnetsar, Nebuzaradan, chef des gardes, transporta en exil sept cent quarante-cinq personnes des Juifs ; en tout quatre mille six cents personnes.
\VS{31}La trente-septième année de la captivité de Jojakin, roi de Juda, le vingt-cinquième jour du douzième mois, Evil-Merodac, roi de Babylone, dans la première année de son règne, leva la tête de Jojakin, roi de Juda, et le fit sortir de prison.
\VS{32}Il lui parla avec bonté, et mit son trône au-dessus du trône des autres rois qui étaient avec lui à Babylone.
\VS{33}Il fit changer ses vêtements de prison, il mangea du pain tous les jours de sa vie en présence du roi.
\VS{34}Le roi de Babylone lui donna continuellement des vivres pour chaque jour, jusqu'au jour de sa mort, tout le temps de sa vie.
\PPE{}
\end{multicols}

%\clearpage\ShortTitle{Ezéchiel}\BookTitle{Ezéchiel}\BFont
\noindent\hrulefill
{\footnotesize
\textit{
\bigskip
{\centering{}
\\Auteur : Ezéchiel
\\(Heb. : Yechezqe'l)
\\Signification : Dieu fortifie
\\Thème : Jugements et gloire
\\Date de rédaction : 6\up{ème} siècle av. J.-C.\\}
}
%\bigskip
\textit{
\\Déporté à Babylone alors qu'il remplissait la fonction de sacrificateur, Ezéchiel eut la particularité d'exercer le ministère prophétique hors de la terre d'Israël. Sa mission était en partie d'affermir la foi des déportés par la promesse du jugement de leurs ennemis et du rétablissement de la nation. Il leur rappela aussi que les péchés de leurs pères étaient la raison de leur captivité et que l'occasion leur était donnée de réformer leurs voies.
%\bigskip
\\Ezéchiel reçut aussi des oracles concernant ceux qui étaient restés à Jérusalem et dont la condition n'était guère meilleure que celle des exilés et pour lesquels le pire était à venir. Ses prophéties s'exprimaient en songes et en visions, c'est donc ainsi qu'il vit la gloire de Yahweh quittant le temple de Jérusalem. Il reçut plusieurs prophéties sur les derniers temps, notamment la promesse d'un cœur nouveau en vue de la conversion, le retour de la gloire de Dieu lors du règne millénaire et aussi le rétablissement total d'Israël.\bigskip
}
}
\par\nobreak\noindent\hrulefill
\begin{multicols}{2}
\Chap{1}
\TextTitle{Vision de la gloire de Yahweh}
\VerseOne{}Or il arriva en la trentième année, au cinquième jour du quatrième mois, comme j'étais parmi ceux qui avaient été transportés sur le fleuve de Kebar, que les cieux furent ouverts, et je vis des visions de Dieu.
\VS{2}Au cinquième jour du mois de cette année, qui fut la cinquième après que le roi Jojakin\FTNT{2 R. 24:12-16.}ait été mené en captivité,
\VS{3}la parole de Yahweh fut adressée expressément à Ezéchiel, le sacrificateur, fils de Buzi, dans le pays des Chaldéens\FTNT{Ezéchiel était en exil à Babylone.}, sur le fleuve de Kebar, et la main de Yahweh fut là sur lui.
\VS{4}Je regardai donc, et voici, un vent impétueux vint du nord, une grosse nuée, et un feu qui prenait de tous côtés. Il y avait autour de la nuée une splendeur, et au milieu de la nuée paraissait comme de l'airain poli, lorsqu'il sort du milieu du feu.
\VS{5}Et du milieu aussi paraissait une ressemblance de quatre animaux\FTNT{Les quatre animaux représentent quatre aspects de Jésus. La face de l'homme correspond à l'humanité du Seigneur mise en exergue dans l'évangile de Luc. La face de lion symbolise la royauté de Christ, mise en évidence dans l'évangile de Matthieu. La face de bœuf fait écho à l'évangile de Marc où le Seigneur y est présenté comme serviteur. La face d'aigle symbolise la divinité du Messie mise en évidence dans l'évangile de Jean. Le Seigneur y est présenté comme le Fils de Dieu et le Dieu véritable.} et voici leur forme: Ils avaient une ressemblance humaine.
\VS{6}Et chacun d'eux avait quatre faces, et chacun avait quatre ailes.
\VS{7}Et leurs pieds étaient des pieds droits, et la plante de leurs pieds était comme la plante d'un pied de veau, ils étincelaient comme la couleur d'un airain poli.
\VS{8}Et il y avait des mains d'homme sous leurs ailes à leurs quatre côtés ; et tous quatre avaient leurs faces et leurs ailes.
\VS{9}Leurs ailes étaient jointes l'une à l'autre ; ils ne se tournaient point quand ils marchaient, mais chacun marchait droit devant soi.
\VS{10}Leurs faces ressemblaient à la face d'un homme, à la face de lion à la main droite, à la face de bœuf à la gauche des quatre, et à la face d'aigle à tous les quatre.
\VS{11}Leurs faces et leurs ailes étaient divisées par le haut ; chacun avait des ailes qui se joignaient l'une à l'autre, et deux couvraient leurs corps.
\VS{12}Chacun marchait droit devant soi ; ils allaient partout où l'Esprit les poussait à aller, et ils ne se tournaient point lorsqu'ils marchaient.
\VS{13}Et quant à l'aspect des animaux, leur regard était comme des charbons de feu ardent, comme des torches ; le feu courait parmi les animaux ; et le feu avait une splendeur, et de ce feu sortait un éclair.
\VS{14}Et les animaux couraient et revenaient selon que l'éclair paraissait.
\VS{15}Je regardais les animaux, et voici, une roue apparut sur la terre auprès des animaux, devant leurs quatre faces.
\VS{16}Et l'aspect et la forme des roues étaient comme la couleur d'un chrysolithe, et toutes les quatre avaient un même aspect ; leur aspect et leur structure étaient comme si chaque roue avait été au milieu d'une autre roue.
\VS{17}En marchant, elles allaient de leurs quatre côtés, et elles ne se tournaient point quand elles marchaient.
\VS{18}Elles avaient des jantes, elles étaient si hautes qu'elles faisaient peur, et leurs jantes étaient pleines d'yeux tout autour des quatre roues.
\VS{19}Quand ils marchaient, elles marchaient auprès d'eux ; et quand ils s'élevaient au-dessus de la terre, elles aussi s'élevaient.
\VS{20}Ils allaient partout où l'Esprit les poussait à aller ; l'Esprit tendait-il là, ils y allaient, et les roues s'élevaient avec eux, car l'esprit des animaux était dans les roues.
\VS{21}Quand ils marchaient, elles marchaient ; et quand ils s'arrêtaient, elles s'arrêtaient ; et quand ils s'élevaient au-dessus de la terre, les roues aussi s'élevaient avec eux, car l'esprit des animaux était dans les roues.
\VS{22}L'aspect de ce qui était au-dessus des têtes des animaux, était une étendue semblable à un cristal étincelant et terrible à voir, laquelle s'étendait au-dessus de leurs têtes.
\VS{23}Sous l'étendue, leurs ailes se tenaient droites l'une contre l'autre ; ils avaient chacun deux ailes dont ils se couvraient, chacun, dis-je, en avait deux qui couvraient leurs corps.
\VS{24}Puis j'entendis le bruit que faisaient leurs ailes quand ils marchaient, semblable au bruit des grandes eaux, et au bruit du Tout-Puissant, un bruit éclatant comme le bruit d'une armée ; quand ils s'arrêtaient, ils baissaient leurs ailes.
\VS{25}Et lorsqu'ils s'arrêtaient et laissaient tomber leurs ailes, il se faisait un bruit au-dessus de l'étendue qui était sur leurs têtes.
\VS{26}Et au-dessus de cette étendue, qui était sur leurs têtes, il y avait quelque chose de semblable à une pierre de saphir, en forme de trône ; et sur cette forme de trône apparaissait comme une figure d'homme\FTNT{Il est question ici de la manifestation du Messie} placée dessus en hauteur.
\VS{27}Je vis encore comme de l'airain poli semblable à un feu, au dedans duquel était cet homme, et qui l'environnait ; depuis la forme de ses reins jusqu'en haut et depuis la forme de ses reins jusqu'en bas, je vis comme du feu, et il y avait une lumière éclatante autour de lui.
\VS{28}Et la splendeur qui se voyait autour de lui, était comme l'arc qui se fait dans la nuée en un jour de pluie. C'est là la vision de la représentation de la gloire de Yahweh. A sa vue je tombai sur ma face, et j'entendis une voix qui parlait.
\Chap{2}
\TextTitle{Mandat d'Ezechiel}
\VerseOne{}Il me dit : Fils de l'homme, tiens-toi sur tes pieds, et je parlerai avec toi.
\VS{2}Alors l'Esprit entra en moi, après qu'il m'eut parlé, et il me releva sur mes pieds, et j'entendis celui qui me parlait.
\VS{3}Il me dit : Fils de l'homme, je t'envoie vers les fils d'Israël, vers des nations rebelles qui se sont rebellées contre moi. Eux et leurs pères ont péché contre moi jusqu'à ce jour même\FTNT{Jé. 3:25.}.
\VS{4}Ce sont des enfants à la face dure et au cœur obstiné, vers lesquels je t'envoie vers eux ; c'est pourquoi tu leur diras que le Seigneur Yahweh a ainsi parlé.
\VS{5}Et soit qu'ils écoutent, ou qu'ils n'en fassent rien, car ils sont une maison rebelle ; ils sauront pourtant qu'il y aura eu un prophète parmi eux\FTNT{Es. 6:9-10.}.
\VS{6}Mais toi, fils de l'homme, ne les crains point, et ne crains point leurs paroles ; quoique des gens rebelles et dont les langues sont perçantes comme des épines soient avec toi, et que tu habites parmi des scorpions ; ne crains point leurs paroles, et ne t'effraie point à cause d'eux, quoiqu'ils soient une maison rebelle\FTNT{Jé. 1:8 ; 1 Pi. 3:14.}.
\VS{7}Tu leur prononceras mes paroles, qu'ils écoutent ou qu'ils n'en fassent rien, car ils ne sont que rébellion.
\VS{8}Mais toi, fils de l'homme, écoute ce que je te dis, et ne sois point rebelle comme cette maison rebelle ; ouvre ta bouche et mange ce que je vais te donner\FTNT{Ap. 10:9 ; Jé. 15:16.}.
\VS{9}Alors je regardai, et voici, une main fut envoyée vers moi, et voici, elle avait un livre en rouleau.
\VS{10}Et elle l'ouvrit devant moi, et voici, il était écrit dedans et dehors ; des lamentations, des soupirs, et des gémissements y étaient écrits.
\Chap{3}
\TextTitle{Yahweh établit Ezéchiel comme sentinelle}
\VerseOne{}Puis il me dit : Fils de l'homme, mange ce que tu trouveras, mange ce rouleau, et va, parle à la maison d'Israël !
\VS{2}J'ouvris donc ma bouche, et il me fit manger ce rouleau.
\VS{3}Il me dit : Fils de l'homme, nourris ton ventre et remplis tes entrailles de ce rouleau que je te donne ! Je le mangeai, et il fut doux dans ma bouche comme du miel\FTNT{Ps. 119:103.}.
\VS{4}Puis il me dit : Fils de l'homme, lève-toi et va vers la maison d'Israël, et prononce-leur mes paroles !
\VS{5}Car tu n'es point envoyé vers un peuple au langage inconnu, ou à la langue barbare ; c'est vers la maison d'Israël ;
\VS{6}ni vers plusieurs peuples ayant un langage inconnu ou une langue barbare, dont tu ne puisses comprendre les paroles. Si je t'envoyais vers eux, ils t'écouteraient.
\VS{7}Mais la maison d'Israël ne voudra pas t'écouter, parce qu'ils ne veulent point m'écouter ; car toute la maison d'Israël a le front dur et le cœur obstiné.
\VS{8}Voici, j'endurcirai ta face contre leurs faces, et j'endurcirai ton front contre leurs fronts\FTNT{Jé. 1:18 ; Mi. 3:8.}.
\VS{9}Et j'ai rendu ton front semblable à un diamant, plus dur que le roc. Ne les crains donc point, et ne t'effraie point à cause d'eux, quoiqu'ils soient une maison rebelle.
\VS{10}Puis il me dit : Fils de l'homme, reçois dans ton cœur et écoute de tes oreilles toutes les paroles que je te dirai.
\VS{11}Lève-toi donc, va vers ceux qui ont été emmenés captifs, vers les enfants de ton peuple, parle-leur et dis-leur que le Seigneur Yahweh a ainsi parlé ; soit qu'ils écoutent ou qu'ils n'en fassent rien.
\VS{12}Puis l'Esprit m'enleva, et j'entendis derrière moi le bruit d'un grand tremblement, disant : Bénie soit la gloire de Yahweh du lieu de sa demeure !
\VS{13}Et j'entendis le bruit des ailes des animaux, qui s'entre-touchaient les unes les autres, et le bruit des roues auprès d'eux, et le bruit d'un grand tremblement.
\VS{14}L'Esprit donc m'enleva, et me prit, et j'allai, l'esprit rempli d'amertume et de colère, mais la main de Yahweh me fortifia.
\VS{15}Je vins donc vers ceux qui avaient été transportés à Thel-Abib, vers ceux qui demeuraient auprès du fleuve de Kebar ; et je me tins là où ils se tenaient, même je me tins là parmi eux sept jours, tout étonné.
\VS{16}Et au bout de sept jours, la parole de Yahweh me fut adressée, en disant :
\VS{17}Fils de l'homme, je t'établis pour être sentinelle sur la maison d'Israël ; tu écouteras donc la parole de ma bouche, et tu les avertiras de ma part\FTNT{Es. 52:8 ; Es. 62:6 ; Jé. 6:17.}.
\VS{18}Quand je dirai au méchant : Tu mourras, tu mourras ! Si tu ne l'avertis pas, et si tu ne parles pas pour l'avertir de se détourner de ses mauvaises voies, afin de lui sauver la vie ; ce méchant-là mourra dans son iniquité, mais je redemanderai son sang de ta main.
\VS{19}Et si tu avertis le méchant, et qu'il ne se détourne pas de sa méchanceté ni de ses mauvaises voies, il mourra dans son iniquité, mais toi, tu sauveras ton âme\FTNT{Ez. 18:23-24 ; Ez. 33:6.}.
\VS{20}Pareillement, si le juste se détourne de sa justice et commet l'iniquité, lorsque j'aurai mis quelque obstacle devant lui, il mourra ; parce que tu ne l'auras point averti, il mourra dans son péché, et il ne sera point fait mention de ses justices qu'il aura faites ; mais je te redemanderai son sang de ta main.
\VS{21}Et si tu avertis le juste de ne point pécher, et qu'il ne pèche point, il vivra, il vivra parce qu'il aura été averti, et toi pareillement tu sauveras ton âme.
\VS{22}Et la main de Yahweh fut sur moi, et il me dit : Lève-toi, et sors vers la vallée, et là je te parlerai.
\VS{23}Je me levai donc, et sortis dans la vallée ; voici, la gloire de Yahweh se tenait là, telle que je l'avais vue près du fleuve de Kebar, et je tombai sur ma face.
\VS{24}Alors l'Esprit entra en moi et me releva sur mes pieds ; il me parla et me dit : Entre, et enferme-toi dans ta maison.
\VS{25}Fils de l'homme, voici, on mettra des cordes sur toi, on te liera, et tu ne sortiras point pour aller parmi eux.
\VS{26} Et j'attacherai ta langue à ton palais, tu seras muet, et tu ne les reprendras point ; parce qu'ils sont une maison  rebelle\FTNT{Jn. 1:20-22.}.
\VS{27}Mais quand je te parlerai, j'ouvrirai ta bouche, et tu leur diras : Ainsi parle le Seigneur Yahweh : Que celui qui écoute, écoute ; et que celui qui n'écoute pas, n'écoute pas ; car ils sont une maison rebelle.
\Chap{4}
\TextTitle{Signes annonciateurs du jugement de Jérusalem : 
\\La brique, la plaque de fer et les cordes}
\VerseOne{}Toi, fils de l'homme, prends une brique et place-la devant toi, et traces-y la ville de Jérusalem.
\VS{2}Puis tu mettras le siège contre elle, tu bâtiras contre elle des retranchements, tu élèveras contre elle des terrasses, tu mettras des camps contre elle, et tu mettras autour d'elle des béliers pour la battre\FTNT{2 R. 25:1.}.
\VS{3}Tu prendras aussi une plaque de fer, et tu la mettras comme un mur de fer entre toi et la ville ; tu dresseras ta face contre elle, elle sera assiégée, et tu l'assiégeras ; ce sera un signe pour la maison d'Israël.
\VS{4}Après, tu dormiras sur ton côté gauche, mets-y l'iniquité de la maison d'Israël, et tu porteras leur iniquité autant de jours que tu seras couché sur ce côté.
\VS{5}Et je t'ai assigné un nombre de jours égal à celui des années de leur iniquité : Trois cent quatre-vingt-dix jours ; ainsi tu porteras l'iniquité de la maison d'Israël.
\VS{6}Et quand tu auras accompli ces jours-là, tu dormiras une seconde fois sur ton côté droit, et tu porteras l'iniquité de la maison de Juda pendant quarante jours ; un jour pour chaque année, car je t'ai assigné un jour pour chaque année.
\VS{7}Tu tourneras ta face et ton bras nu vers Jérusalem assiégée, et tu prophétiseras contre elle.
\VS{8}Et voici, j'ai mis sur toi des cordes, afin que tu ne puisses pas te tourner d'un côté sur l'autre, jusqu'à ce que tu aies accompli les jours de ton siège.
\TextTitle{Le pain impur}
\VS{9}Tu prendras aussi du froment, de l'orge, des fèves, des lentilles, du millet, et de l'épeautre ; tu les mettras dans un vase, et tu en feras du pain autant de jours que tu seras couché sur ton côté ; tu en mangeras pendant trois cent quatre-vingt-dix jours.
\VS{10}La viande que tu mangeras sera du poids de vingt sicles par jour ; et tu la mangeras de temps à autre.
\VS{11}Et tu boiras de l'eau par mesure; savoir la sixième de hin ; tu la boiras de temps à autre.
\VS{12}Et tu mangeras aussi des gâteaux d'orge, que tu feras cuire avec des excréments humains en leur présence.
\VS{13}Puis Yahweh dit : Les fils d'Israël mangeront ainsi leur pain souillé parmi les nations vers lesquelles je les chasserai\FTNT{Os. 9:3 ; Da. 1:8.}.
\VS{14}Et je dis : Ah ! Seigneur Yahweh, voici, mon âme n'a point été souillée, et je n'ai mangé d'aucune bête morte d'elle-même, ou déchirée par les bêtes sauvages, depuis ma jeunesse jusqu'à présent, et aucune chair impure n'est entrée dans ma bouche\FTNT{Lé 17:15 ; De. 14:3 ; Ac. 10:14.}.
\VS{15}Il me répondit : Voici, je te donne des excréments de bœuf au lieu d'excréments humains, et tu y feras cuire ton pain.
\VS{16}Puis il me dit : Fils de l'homme, voici, je m'en vais rompre le bâton du pain dans Jérusalem ; et ils mangeront leur pain au poids et avec chagrin ; et ils boiront de l'eau par mesure et avec horreur\FTNT{Lé. 26:26 ; Es. 3:1 ; Ps. 105:16 ; La. 5:4.}.
\VS{17}Parce que le pain et l'eau leur manqueront, ils seront épouvantés, se regardant les uns les autres, et ils se décomposeront à cause de leur iniquité.
\Chap{5}
\TextTitle{Les cheveux coupés et divisés en trois}
\VerseOne{}Et toi, fils de l'homme, prends un couteau tranchant, prends un rasoir de barbier, et fais-le passer sur ta tête et sur ta barbe. Puis, tu prendras une balance à peser, et tu partageras ce que tu auras rasé\FTNT{Lé. 21:5 ; Ez. 44:20.}.
\VS{2}Brûles-en un tiers dans le feu, au milieu de la ville, lorsque les jours du siège seront accomplis ; prends-en un tiers, et frappe-le avec l'épée tout autour de la ville ; disperses-en un tiers au vent, car je tirerai l'épée derrière eux\FTNT{Lé. 26:25 ; La. 1:20.}.
\VS{3}Tu en prendras une petite quantité que tu serreras aux pans de ton manteau.
\VS{4}De ceux-là, tu en prendras encore, les jetteras au milieu du feu, et les brûleras au feu. De là sortira un feu contre toute la maison d'Israël.
\VS{5}Ainsi parle le Seigneur Yahweh : C'est là cette Jérusalem que j'avais placée au milieu des nations et des pays qui sont autour d'elle.
\VS{6}Elle a changé mes ordonnances et s'est rendue plus coupable que les nations et les pays d'alentour ; car ils ont rejeté mes ordonnances, et n'ont point marché dans mes ordonnances.
\VS{7}C'est pourquoi ainsi parle le Seigneur Yahweh : Parce que vous avez multiplié vos méchancetés plus que les nations qui vous entourent, et que vous n'avez point suivi mes ordonnances et observé mes lois, et que vous n'avez pas agi selon les ordonnances des nations qui vous entourent ;
\VS{8}à cause de cela, ainsi parle le Seigneur Yahweh : Voici, j'en veux à toi et j'exécuterai au milieu de toi mes jugements, sous les yeux des nations.
\VS{9}Je te ferai, à cause de toutes tes abominations, des choses que je n'ai jamais faites, et ce que je ne ferai jamais\FTNT{Da. 9:12 ; Mt. 24:21.}.
\VS{10}Des pères mangeront leurs fils au milieu de toi, et des fils mangeront leurs pères ; j'exécuterai mes jugements sur toi, et je disperserai à tous les vents tout ce qui restera de toi\FTNT{Lé. 26:33 ; De. 28:64 ; Jé. 9:16 ; Za. 2:6.}.
\VS{11}Je suis vivant, dit le Seigneur Yahweh, parce que tu as souillé mon lieu saint par toutes tes infamies, et par toutes tes abominations, moi-même je te raserai, et mon oeil ne t'épargnera point, et je n'aurai point de compassion\FTNT{Jé. 7:9-11.}.
\VS{12}Un tiers d'entre vous mourra de la peste, et sera consumé par la famine au milieu de toi ; un tiers tombera par l'épée autour de toi ; et je disperserai à tous les vents l'autre tiers, je tirerai l'épée derrière eux.
\VS{13}Car ma colère sera portée à son comble, je ferai reposer ma fureur sur eux, et je me donnerai satisfaction ; ils sauront que moi, Yahweh, j'ai parlé dans ma jalousie, quand j'aurai consumé ma fureur sur eux.
\VS{14}Je ferai de toi un désert, un sujet d'opprobre parmi les nations qui sont autour de toi, aux yeux de tous les passants\FTNT{Lé. 26:31-32 ; Né. 2:17.}.
\VS{15}Tu seras en opprobre, en ignominie, un exemple et un sujet d'étonnement pour les nations qui t'entourent, quand j'aurai exécuté mes jugements sur toi, avec colère, avec fureur, et par des châtiments pleins de fureur ; moi, Yahweh, j'ai parlé\FTNT{De. 28:37 ; 1 R. 9:7 ; Ps. 79:4 ; Jé. 24:9 ; Es. 26:9.}.
\VS{16}Quand je lancerai sur eux les flèches douloureuses de la famine, qui seront mortelles, quand je les enverrai pour vous détruire, j'ajouterai la famine sur vous, et romprai pour vous le bâton du pain\FTNT{De. 32:24.}.
\VS{17}Je vous enverrai la famine, et des bêtes féroces, qui te priveront d'enfants ; la peste et le sang passeront au milieu de toi, et je ferai venir l'épée sur toi. Moi,Yahweh, j'ai parlé.
\Chap{6}
\TextTitle{Grâce de Yahweh pour quelques réchappés d'Israël}
\VerseOne{}La parole de Yahweh me fut encore adressée, en disant :
\VS{2}Fils de l'homme, tourne ta face contre les montagnes d'Israël, et prophétise contre elles !
\VS{3}Et dis : Montagnes d'Israël, écoutez la parole du Seigneur Yahweh. Ainsi parle le Seigneur Yahweh aux montagnes et aux collines, aux cours des rivières, et aux vallées : Me voici, je vais faire venir l'épée sur vous, et je détruirai vos hauts lieux\FTNT{Lé. 26:30.}.
\VS{4}Vos autels seront dévastés, vos autels d'encens seront brisés, et je ferai tomber vos morts devant vos idoles.
\VS{5}Car je mettrai les cadavres des fils d'Israël devant leurs idoles, et je disperserai vos os autour de vos autels\FTNT{2 R. 23:14-20.}.
\VS{6}Les villes seront désertes, là où sont vos demeures, et les hauts lieux seront dévastés, vos autels seront délaissés et abandonnés, et vos idoles seront brisées et ne seront plus ; vos autels d'encens abattus, et vos ouvrages seront nettoyés.
\VS{7}Les tués tomberont parmi vous ; et vous saurez que je suis Yahweh.
\VS{8}Mais je laisserai quelques restes d'entre vous, afin que vous ayez quelques réchappés de l'épée parmi les nations, quand vous serez dispersés parmi les pays.
\VS{9}Vos réchappés se souviendront de moi\FTNT{Jé. 51:50.} parmi les nations où ils seront captifs, parce que j'aurai brisé leur cœur adonné à la fornication, qui s'est détourné de moi, et à cause de leurs yeux qui se sont livrés à la prostitution après leurs idoles ; ils se prendront eux-mêmes en dégoût, à cause du mal qu'ils ont commis, à cause de leurs abominations.
\VS{10}Ils sauront que je suis Yahweh, que ce n'est point en vain que je les ai menacés.
\TextTitle{Sentence envers les idolâtres}
\VS{11}Ainsi parle le Seigneur Yahweh : Frappe de ta main et bats de ton pied, et dis : Hélas ! A cause de toutes les abominations, des maux de la maison d'Israël ; car ils tomberont par l'épée, par la famine, et par la peste.
\VS{12}Celui qui sera loin mourra de la peste, et celui qui sera près tombera par l'épée ; et celui qui restera et sera assiégé, mourra par la famine, ainsi je consumerai ma fureur sur eux\FTNT{Am. 4:10.}.
\VS{13}Vous saurez que je suis Yahweh quand les blessés à morts seront au milieu de leurs idoles, autour de leurs autels, sur toute colline élevée, sur tous les sommets des montagnes, sous tout arbre vert, et sous tout chêne touffu, là où ils offraient des parfums de bonne odeur à toutes leurs idoles\FTNT{Os. 4:13.}.
\VS{14}J'étendrai donc ma main sur eux, et je rendrai leur pays désolé et désert dans toutes leurs demeures, plus que le désert qui est vers Dibla. Et ils sauront que je suis Yahweh.
\Chap{7}
\TextTitle{Attaque babylonienne imminente}
\VerseOne{}Puis la parole de Yahweh me fut adressée, en disant :
\VS{2}Et toi, fils de l'homme, écoute : Ainsi parle le Seigneur Yahweh à la terre d'Israël : La fin, la fin vient sur les quatre coins de la terre !
\VS{3}Maintenant la fin vient sur toi, j'enverrai sur toi ma colère, et je te jugerai selon ta voie, et je mettrai sur toi toutes tes abominations\FTNT{Ro. 2:6.}.
\VS{4}Et mon œil ne t'épargnera point, et je n'aurai point de compassion ; mais je te chargerai de tes voies, et tes abominations seront au milieu de toi ; et vous saurez que je suis Yahweh.
\VS{5}Ainsi parle le Seigneur Yahweh : Voici un mal, un seul mal qui vient !
\VS{6}La fin vient, la fin vient, elle se réveille contre toi ; voici, le mal vient !
\VS{7}Ton tour arrive, habitant du pays ! Le temps vient, le jour est près de toi, il ne sera que frayeur, et non pas une invitation des montagnes\FTNT{So. 1:14-15.} à s'entre-réjouir.
\VS{8}Maintenant, je répandrai bientôt ma fureur sur toi, et je consumerai ma colère sur toi ; je te jugerai selon ta voie, je mettrai sur toi toutes tes abominations.
\VS{9}Mon œil ne t'épargnera point, et je n'aurai point de compassion, je te punirai selon ta voie, et tes abominations seront au milieu de toi ; et vous saurez que je suis Yahweh qui frappe.
\VS{10}Voici le jour, voici il vient, le matin paraît, la verge fleurit, l'orgueil bourgeonne.
\VS{11}La violence s'élève pour servir de verge à la méchanceté ; il ne restera rien d'eux, ni de leur multitude, ni de leur tumulte, et on ne se lamentera point sur eux.
\VS{12}Le temps vient, le jour est tout proche : Que celui donc qui achète ne se réjouisse point, et que celui qui vend ne se lamente point ; car il y a une ardente colère sur toute leur multitude.
\VS{13}Car le vendeur ne recouvrera pas ce qu'il a vendu, serait-il encore parmi les vivants ; car la vision touchant toute leur multitude ne sera point révoquée ; et à cause de son iniquité, nul ne conservera sa vie.
\VS{14}On sonne de la trompette, tout est prêt, mais il n'y a personne pour aller au combat, parce que l'ardeur de ma colère est sur toute leur multitude.
\VS{15}L'épée est au-dehors, la peste et la famine au-dedans ! Celui qui est aux champs mourra par l'épée ; et celui qui est dans la ville, la famine et la peste le dévoreront.
\VS{16}Les réchappés s'enfuiront et seront sur les montagnes comme les pigeons des vallées, tous gémissant, chacun sur son iniquité.
\VS{17}Toutes les mains deviendront lâches, et tous les genoux se fondront en eau\FTNT{Es. 13:7 ; Jé. 6:24.}.
\VS{18}Ils se ceindront de sacs, et le tremblement les couvrira, la confusion sera sur tous leurs visages, et leurs têtes deviendront chauves\FTNT{Es. 3:24 ; Jé. 48:37 ; Am. 8:10.}.
\VS{19}Ils jetteront leur argent par les rues, et leur or s'en ira au loin ; leur argent et leur or ne pourront pas les délivrer au jour de la grande colère de Yahweh\FTNT{Pr. 11:4 ; So. 1:18.} ; ils ne rassasieront point leurs âmes, et ne rempliront point leurs entrailles, parce que leur iniquité aura été leur ruine.
\TextTitle{Violation du temple}
\VS{20}Ils étaient fiers de leur magnifique parure ; mais ils y ont placé des images de leurs abominations et de leurs infamies, c'est pourquoi je la rendrai pour eux un objet d'horreur.
\VS{21}Je l'ai livrée au pillage dans la main des étrangers, et en proie aux méchants de la terre qui la profaneront\FTNT{Jé. 20:5.}.
\VS{22}Je détournerai aussi ma face d'eux, et on violera mon lieu secret, et des furieux entreront et le profaneront.
\VS{23}Fais une chaîne ! Car le pays est plein de crimes, de meurtre, et la ville est pleine de violence.
\VS{24}C'est pourquoi je ferai venir les plus méchants des nations, qui possèderont leurs maisons, et je ferai cesser l'orgueil des puissants, et leurs saints lieux seront profanés.
\VS{25}La destruction vient, et ils chercheront la paix, mais il n'y en aura point.
\VS{26}Il viendra malheur sur malheur, et il y aura rumeur sur rumeur ; ils demanderont la vision aux prophètes\FTNT{La. 2:9.} ; la loi périra chez le sacrificateur, et le conseil chez les anciens.
\VS{27}Le roi se lamentera, les princes se vêtiront de désolation, et les mains du peuple du pays tomberont de frayeur. Je les traiterai selon leur voie, je les jugerai comme ils le méritent et ils sauront que je suis Yahweh.
\Chap{8}
\TextTitle{Visions divines}
\VerseOne{}Puis il arriva dans la sixième année, au cinquième jour du sixième mois, comme j'étais assis dans ma maison, et que les anciens de Juda étaient assis devant moi, que la main du Seigneur Yahweh tomba là sur moi.
\VS{2}Je regardai, et voici c'était une figure ayant l'aspect d'un feu qui frappe les regards ; depuis ses reins jusqu'en bas c'était du feu, et depuis ses reins jusqu'en haut, c'était d'un aspect brillant comme de l'airain poli.
\VS{3}Il étendit une forme de main et me prit par les cheveux de ma tête. L'Esprit m'enleva entre la terre et le ciel et me transporta à Jérusalem, dans des visions de Dieu, à l'entrée de la porte intérieure, du côté nord, où était posée l'idole de jalousie\FTNT{L'idole de la jalousie : Dans le temple de Jérusalem à l'époque d'Ezéchiel, l'idolâtrie s'y développait sans retenue (2 R. 21, 22 et 23). Il y avait dans ce temple les idoles d'Astarté et les autels de Baal. Le temple était souillé.} qui provoque la jalousie.
\VS{4}Voici, la gloire du Dieu d'Israël était là, telle que je l'avais vue en vision dans la vallée.
\TextTitle{Abominations dans le temple}
\VS{5}Il me dit : Fils de l'homme, lève maintenant tes yeux vers le chemin qui tend vers le nord ! J'élevai mes yeux vers le chemin qui tend vers le nord, et voici du côté nord, à la porte de l'autel, était cette idole de jalousie, à l'entrée.
\VS{6}Il me dit : Fils de l'homme, ne vois-tu pas ce qu'ils font, les grandes abominations que la maison d'Israël commet ici, pour que je me retire de mon lieu saint ? Mais tourne-toi encore, tu verras de grandes abominations.
\VS{7}Il me conduisit donc à l'entrée du parvis. Je regardai, et voici, il y avait un trou dans le mur.
\VS{8}Il me dit : Fils de l'homme, perce maintenant le mur ; et quand je perçai le mur, il y avait une porte.
\VS{9}Puis il me dit : Entre et regarde les méchantes abominations qu'ils commettent ici.
\VS{10}J'entrai donc et je regardai ; et voici, toutes sortes de figures de reptiles et de bêtes abominables, et toutes les idoles de la maison d'Israël étaient peintes sur le mur tout autour\FTNT{Ex. 20:4 ; De. 4:16-18 ; Ro. 1:23.}.
\VS{11}Soixante-dix hommes des anciens de la maison d'Israël, au milieu desquels était Jaazania, fils de Schaphan, se tenaient debout devant ces idoles, chacun l'encensoir à la main, d'où s'élevait une épaisse nuée d'encens.
\VS{12}Alors il me dit : Fils de l'homme, n'as-tu pas vu ce que les anciens de la maison d'Israël font dans les ténèbres, chacun dans sa chambre pleine de figures ? Car ils disent : Yahweh ne nous voit point, Yahweh a abandonné le pays\FTNT{Es. 29:15.}.
\VS{13}Puis il me dit : Tourne-toi encore, et tu verras les grandes abominations qu'ils commettent.
\VS{14}Il me conduisit donc à l'entrée de la porte de la maison de Yahweh qui est vers le nord. Et voici, il y avait là des femmes assises qui pleuraient Thammuz\FTNT{Thammuz ou Adonis.}.
\VS{15}Il me dit : Fils de l'homme, n'as-tu pas vu ? Tourne-toi encore, et tu verras des abominations plus grandes que celles-ci.
\VS{16}Il me fit donc entrer dans le parvis intérieur de la maison de Yahweh. Et voici, à l'entrée du temple de Yahweh, entre le portique et l'autel, environ vingt-cinq hommes avaient le dos tourné contre le temple de Yahweh, leurs visages tournés vers l'orient ; et ils se prosternaient vers l'orient, devant le soleil\FTNT{De. 4:19.}.
\VS{17}Alors il me dit : Fils de l'homme, n'as-tu pas vu ? Est-ce une chose légère à la maison de Juda de commettre ces abominations qu'ils commettent ici ? Car ils ont rempli le pays de violence, et ils se sont ainsi tournés pour m'irriter ; mais voici ils approchent le rameau de leurs nez.
\VS{18}Et moi, j'agirai dans ma fureur ; mon œil ne les épargnera point, et je n'aurai point de compassion ; quand ils crieront à haute voix à mes oreilles, je ne les exaucerai point\FTNT{Pr. 1:28 ; Es. 1:15 ; Jé. 11:11 ; Mi. 3:4 ; Za. 7:13.}.
\Chap{9}
\TextTitle{Marque de Yahweh sur les justes ; extermination des impies}
\VerseOne{}Puis il cria d'une voix forte à mes oreilles : Faites approcher ceux qui châtient la ville, chacun avec son instrument de destruction à la main !
\VS{2}Et voici, six hommes venaient par le chemin de la haute porte qui regarde vers le nord, et chacun avait dans sa main son instrument de destruction. Il y avait au milieu d'eux un homme vêtu de lin, qui avait une écritoire sur ses reins ; ils entrèrent et se tinrent près de l'autel d'airain.
\VS{3}Alors la gloire du Dieu d'Israël s'éleva du chérubin sur lequel elle était, et vint sur le seuil de la maison. Il cria à l'homme qui était vêtu de lin et qui avait l'écritoire sur ses reins.
\VS{4}Yahweh lui dit : Passe par le milieu de la ville, par le milieu de Jérusalem, et marque la lettre Thau sur les fronts des hommes qui gémissent et qui soupirent à cause de toutes les abominations qui s'y commettent\FTNT{Ap. 7:3 ; Ap. 9:4 ; Ap. 13:16-17 ; Ap. 20:4 ; Ex. 12:7-23.}.
\VS{5}Et s'adressant aux autres en ma présence, il dit : Passez dans la ville après lui, et frappez ; que votre oeil soit sans pitié et n'ayez point de compassion !
\VS{6}Tuez-les tous, les vieillards, les jeunes gens, les vierges, les enfants et les femmes\FTNT{2 Ch. 36:17.} ; mais n'approchez pas de ceux qui ont la lettre Thau\FTNT{La lettre Thau ou Tav est la marque, le signe, le symbole ou le sceau Divin. La lettre Tav est formée par la réunion des lettres Daleth et Nun. Ces deux lettres forment le mot « dan » qui veut dire « juge ». Selon la Bible, la marque des chrétiens est représentée par : Le Saint-Esprit, le nom de Jésus-Christ (Ep. 1:13-14 ; Ep. 4:30 ; Ap. 14:1), le nom de la Nouvelle Jérusalem (Ap. 3 : 12) et le nom du Père. Les chrétiens fidèles à Dieu sont marqués par l'Esprit de Dieu qui est notre sceau. Le Saint-Esprit est saint, la sainteté est donc la marque des chrétiens (1 Pi. 1:2). Il est aussi l'Esprit de vérité, donc la vérité est aussi la marque des chrétiens (Jn. 16:13). Il est aussi amour, l'amour étant également la marque distinctive des véritables chrétiens (Ro. 5:5).}, et commencez par mon lieu saint\FTNT{Le jugement commence par la maison de Dieu (1 Pi. 4:17-18).}. Ils commencèrent donc par les vieillards qui étaient devant la maison.
\VS{7}Il leur dit : Profanez la maison, et remplissez de morts les parvis !… Sortez !… Et ils sortirent, et ils frappèrent dans la ville.
\VS{8}Or il arriva que comme ils frappaient, je restai là, et m'étant prosterné le visage contre terre, je criai et dis : Ah ! Seigneur Yahweh ! Vas-tu donc détruire tous les restes d'Israël en répandant ta fureur sur Jérusalem ?
\VS{9}Il me dit : L'iniquité de la maison d'Israël et de Juda est excessivement grande, le pays est rempli de meurtres et la ville remplie de crimes ; car ils ont dit : Yahweh a abandonné le pays, Yahweh ne nous voit point.
\VS{10}Quant à moi, mon oeil aussi ne les épargnera point, et je n'en aurai point compassion ; je mettrai leur voie sur leur tête.
\VS{11}Et voici, l'homme vêtu de lin, qui avait une écritoire sur ses reins, rapporta ce qui avait été fait, et il dit : J'ai fait comme tu m'as ordonné.
\Chap{10}
\TextTitle{La gloire de Yahweh quitte le temple}
\VerseOne{}Je regardai, et voici, sur l'étendue qui était au-dessus de la tête des chérubins, parut comme une pierre de saphir ; on voyait au-dessus d'eux quelque chose de semblable à un trône.
\VS{2}On parla à l'homme vêtu de lin, et on lui dit : Va entre les roues, sous les chérubins, et remplis tes mains de charbons ardents que tu prendras entre les chérubins, et répands-les sur la ville\FTNT{Es. 6:6 ; Ap. 8:5.} ; il y entra devant mes yeux.
\VS{3}Les chérubins étaient à la droite de la maison quand l'homme entra ; et une nuée remplit le parvis intérieur\FTNT{1 R. 8:10-11.}.
\VS{4}Puis la gloire de Yahweh s'éleva de dessus les chérubins pour venir sur le seuil de la maison, et la maison fut remplie d'une nuée, et le parvis fut rempli de la splendeur de la gloire de Yahweh.
\VS{5}On entendit le bruit des ailes des chérubins jusqu'au parvis extérieur, pareil à la voix du Dieu Tout-Puissant lorsqu'il parle.
\VS{6}Ainsi Yahweh donna cet ordre à l'homme qui était vêtu de lin : Prends du feu d'entre les roues des chérubins ; il entra et se tint auprès des roues.
\VS{7}L'un des chérubins étendit sa main entre les chérubins, vers le feu qui était entre les chérubins ; il en prit et le mit entre les mains de l'homme vêtu de lin. Et cet homme le prit et sortit.
\VS{8}On voyait aux chérubins une forme de main d'homme sous leurs ailes.
\VS{9}Puis je regardai, et voici, il y avait quatre roues près des chérubins, une roue près de chaque chérubin ; et ces roues avaient l'aspect d'une pierre de chrysolithe.
\VS{10}A leur aspect, toutes les quatre avaient la même forme ; chaque roue paraissait être au milieu d'une autre roue.
\VS{11}Quand elles marchaient, elles allaient de leurs quatre côtés, et elles ne se tournaient point dans leur marche ; mais elles allaient dans la direction de la tête, sans se tourner dans leur marche.
\VS{12}Tout le corps des chérubins, leur dos, leurs mains, leurs ailes, étaient remplis d'yeux, aussi bien que les roues tout autour, les quatre roues\FTNT{Ap. 4:6-8.}.
\VS{13}J'entendis qu'on appela les roues tourbillon.
\VS{14}Chaque animal avait quatre faces : La première face était la face d'un chérubin ; la seconde face était la face d'un homme ; la troisième était la face d'un lion ; et la quatrième la face d'un aigle\FTNT{Ez. 1 ; Ap. 4:7.}.
\VS{15}Puis les chérubins s'élevèrent. Ce sont là les animaux que j'avais vus près du fleuve de Kebar.
\VS{16}Lorsque les chérubins marchaient, les roues aussi marchaient à côté d'eux ; et quand les chérubins élevaient leurs ailes pour s'élever de terre, les roues ne se détournaient point d'eux.
\VS{17}Lorsqu'ils s'arrêtaient, elles s'arrêtaient ; et lorsqu'ils s'élevaient, elles s'élevaient ; car l'esprit des animaux était dans les roues.
\VS{18}Puis la gloire de Yahweh se retira de dessus le seuil de la maison, et se tint au-dessus des chérubins.
\VS{19}Les chérubins élevant leurs ailes, s'élevèrent de terre sous mes yeux quand ils partirent ; les roues s'élevèrent aussi. Et chacun d'eux s'arrêta à l'entrée de la porte orientale de la maison de Yahweh ; la gloire du Dieu d'Israël était sur eux en haut.
\VS{20}C'étaient les animaux que j'avais vus sous le Dieu d'Israël près du fleuve de Kebar ; et je reconnus que c'étaient des chérubins.
\VS{21}Chacun avait quatre faces, et chacun quatre ailes, une forme de main d'homme était sous leurs ailes.
\VS{22}Quant à l'aspect de leurs faces, c'étaient les faces que j'avais vues près du fleuve de Kebar, c'était le même aspect, c'étaient eux-mêmes. Et chacun marchait droit devant soi.
\Chap{11}
\TextTitle{Sentences sur les princes infidèles}
\VerseOne{}Puis l'Esprit m'enleva et me transporta à la porte orientale de la maison de Yahweh, à celle qui regarde vers l'orient. Et il y avait vingt-cinq hommes à l'entrée de la porte, et je vis au milieu d'eux Jaazania, fils d'Azzur, et Pelathia, fils de Benaja, les princes du peuple.
\VS{2}Il me dit : Fils de l'homme, ce sont les hommes qui ont des pensées d'iniquité, et qui donnent un mauvais conseil dans cette ville\FTNT{Mi. 2:1.}.
\VS{3}Ils disent : Ce n'est pas le moment ! Bâtissons des maisons ! La ville est la chaudière et nous sommes la viande.
\VS{4}C'est pourquoi prophétise contre eux, prophétise, fils de l'homme !
\VS{5}L'Esprit de Yahweh tomba sur moi. Et il me dit : Ainsi parle Yahweh : Vous parlez de la sorte, maison d'Israël, et je connais toutes les pensées de votre esprit.
\VS{6}Vous avez multiplié les meurtres dans cette ville, et vous avez rempli ses rues de gens que vous avez tués.
\VS{7}C'est pourquoi, ainsi parle le Seigneur Yahweh : Les gens que vous avez tués, et que vous avez mis au milieu d'elle, sont la viande, et elle est la chaudière, mais je vous tirerai hors du milieu d'elle\FTNT{Mi. 3:3.}.
\VS{8}Vous avez eu peur de l'épée, mais je ferai venir l'épée sur vous, dit le Seigneur Yahweh\FTNT{Jé. 42:16.}.
\VS{9}Je vous tirerai hors de la ville, je vous livrerai entre les mains des étrangers, et j'exécuterai mes jugements contre vous.
\VS{10}Vous tomberez par l'épée ; je vous jugerai dans le pays d'Israël, et vous saurez que je suis Yahweh.
\VS{11}Elle ne sera point une chaudière pour vous, et vous ne serez point au dedans d'elle comme la viande ; je vous jugerai dans le pays d'Israël.
\VS{12}Et vous saurez que je suis Yahweh ; car vous n'avez point suivi mes ordonnances, et vous n'avez pas observé mes lois, mais vous avez agi selon les ordonnances des nations qui sont autour de vous.
\VS{13}Or il arriva comme je prophétisais, que Pelathia, fils de Benaja, mourut. Alors je me prosternai sur mon visage, je criai à haute voix, et dis : Ah ! Seigneur Yahweh ! Vas-tu consumer entièrement le reste d'Israël ?
\TextTitle{Restauration d'Israël et de ses exilés}
\VS{14}La parole de Yahweh me fut adressée, en disant :
\VS{15}Fils de l'homme, tes frères, tes frères, les hommes de ta parenté, et la maison d'Israël tout entière, à qui les habitants de Jérusalem ont dit : Eloignez-vous de Yahweh, la terre nous a été donnée en héritage.
\VS{16}C'est pourquoi dis-leur : Ainsi parle le Seigneur Yahweh : Quoique je les aie éloignés des nations, et que je les aie dispersés dans divers pays, je serai pour eux quelque temps un lieu saint\FTNT{Alors que le lieu saint ou maison terrestre était souillée, Yahweh se présente comme le Lieu Sacré pour son peuple.} dans les pays où ils sont venus.
\VS{17}C'est pourquoi dis-leur : Ainsi parle le Seigneur Yahweh : Je vous rassemblerai du milieu des peuples, et je vous recueillerai des pays auxquels vous avez été dispersés, et je vous donnerai la terre d'Israël\FTNT{Es. 11:11-16 ; Jé. 24:6 ; Ez. 28:25 ; Ez. 34:13 ; Ez. 36:24.}.
\VS{18}C'est là qu'ils iront, et ils ôteront hors d'elle toutes ses infamies et toutes ses abominations.
\VS{19}Je leur donnerai un même cœur, et je mettrai en eux un esprit nouveau ; j'ôterai de leur corps le cœur de pierre, et je leur donnerai un cœur de chair\FTNT{Il s'agit d'une allusion à la nouvelle alliance (Jé. 31:31-34 ; Hé. 8).},
\VS{20}afin qu'ils suivent mes ordonnances, et qu'ils gardent et observent mes lois ; ils seront mon peuple, et je serai leur Dieu.
\VS{21}Quant à ceux dont le cœur se plaît à leurs idoles et à leurs abominations, quant à ceux-là, je ferai tomber sur leur tête les peines que mérite leur conduite, dit le Seigneur Yahweh.
\TextTitle{La gloire de Dieu en mouvement vers le Mont des Oliviers\FTNTT{Cp. Ez. 43:1-4.}}
\VS{22}Puis les chérubins élevèrent leurs ailes, accompagnés des roues ; et la gloire du Dieu d'Israël était sur eux, en haut.
\VS{23}La gloire de Yahweh s'éleva du milieu de la ville\FTNT{Le départ de la gloire de Dieu du temple de Jérusalem marque la fin de la théocratie (règne de Dieu) en Israël. Cet événement, comparable au retrait de l'Esprit de Dieu en Ge. 6:3, fut consécutif à la décadence morale d'Israël (voir Ez. 8) qui fut désormais livré aux nations. Certains estiment que la théocratie a cessé au moment où les israélites ont demandé un roi (voir 1 S. 8). Or bien que cette demande déplut à Yahweh, il continua néanmoins à diriger Israël au travers des souverains tels que David, qu'il établissait à la tête de son peuple. Les Hébreux avaient déjà reçu un sérieux avertissement avec la destruction du temple lors de la première déportation babylonienne (2 R. 24). Cet événement, bien que traumatisant pour beaucoup, n'avait cependant pas provoqué une réelle repentance, c'est pourquoi les israélites retombèrent rapidement dans leurs travers. Ainsi, comme en témoigne Mal. 2 :17 qui rapporte les propos de certains juifs : « Où est le Dieu de la justice ? », en dépit de la reconstruction du temple sous Néhémie et Esdras, la gloire de Dieu ne s'y manifestait plus depuis longtemps. Ezéchiel ne fait donc qu'assister à la conséquence de plusieurs siècles d'infidélité des juifs à l'égard de leur Dieu.}, et s'arrêta sur la montagne qui est à l'orient de la ville.
\VS{24}Puis l'Esprit m'enleva et me transporta en Chaldée, vers ceux qui avaient été emmenés captifs, le tout en vision par l'Esprit de Dieu ; et la vision que j'avais vue disparut au-dessus de moi.
\VS{25}Alors je dis à ceux qui avaient été emmenés captifs toutes les paroles que Yahweh m'avait révélées.
\Chap{12}
\TextTitle{Fuite d'Ezéchiel, un signe pour Israël}
\VerseOne{}La parole de Yahweh me fut encore adressée en ces mots :
\VS{2}Fils de l'homme : Tu habites au milieu d'une maison rebelle, au milieu de gens qui ont des yeux pour voir, et ne voient point ; et qui ont des oreilles pour entendre, et n'entendent point ; parce qu'ils sont une maison de rebelles\FTNT{Es. 6:9 ; Es. 49:19-20 ; Jé. 5:21 ; Ac. 28:26.}.
\VS{3}Toi donc fils d’homme, fais-toi des bagages d’un homme qui s'exile et pars en exil de jour, sous leurs yeux, pars en exil, dis-je de ton lieu pour aller dans un autre lieu, sous leurs yeux. Peut-être qu'ils y prendront garde, quoi qu’ils soient une maison rebelle.
\VS{4}Tu mettras donc dehors pendant le jour tes bagages comme les bagages d'un homme qui s'exile sous leurs yeux, et le soir, tu sortiras sous leurs yeux, comme quand on sort pour s'exiler.
\VS{5}Perce-toi le mur sous leurs yeux et sors par là tes bagages.
\VS{6}Tu les porteras sur tes épaules, sous leurs yeux, et tu sortiras tes bagages pendant l'obscurité. Tu couvriras aussi ton visage, afin que tu ne voies point la terre ; car je t'ai mis pour être un signe pour la maison d'Israël.
\VS{7}Je fis donc ce qui m'avait été ordonné : Je portai dehors pendant le jour mes bagages comme des bagages d'exil ; le soir je perçai le mur avec la main et je les sortis pendant l'obscurité, je les portai sur l'épaule, sous leurs yeux.
\VS{8}Au matin, la parole de Yahweh me fut adressée en ces mots :
\VS{9}Fils de l'homme, la maison d'Israël, maison rebelles, ne t'a-t-elle pas dit : Qu'est-ce que tu fais ?
\VS{10}Dis-leur : Ainsi parle le Seigneur Yahweh : Cet ordre dont je suis chargé s'adresse au prince qui est à Jérusalem et à toute la maison d'Israël qui s'y trouve.
\VS{11}Dis : Je suis pour vous un signe ; comme j'ai fait, ainsi il leur sera fait ; ils iront en exil, en captivité.
\VS{12}Et le prince qui est parmi eux, mettra son bagage sur l'épaule et sortira ; on percera le mur pour le tirer dehors ; il couvrira son visage, afin qu'il ne voie point de ses yeux la terre\FTNT{2 R. 25:4.}.
\VS{13}J'étendrai mon rets sur lui, et il sera pris dans mes filets ; je le ferai entrer dans Babylone, au pays des Chaldéens, mais il ne la verra point, et il y mourra.
\VS{14}Je disperserai à tout vent tout ce qui est autour de lui, son secours, et tous ses corps d'armées ; et je tirerai l'épée sur eux.
\VS{15}Ils sauront que je suis Yahweh, quand je les aurai répandus parmi les nations, et que je les aurai dispersés dans divers pays.
\VS{16}Je laisserai un reste d'entre eux, quelques hommes, préservés de l'épée, de la famine, et de la peste, afin qu'ils racontent toutes leurs abominations parmi les nations où ils iront ; et ils sauront que je suis Yahweh.
\TextTitle{La captivité du peuple imminente\FTNTT{Cp. 2 R. 25:1-10.}}
\VS{17}Puis la parole de Yahweh me fut adressée en ces mots :
\VS{18}Fils de l'homme, mange ton pain dans l'agitation, et bois ton eau en tremblant et avec inquiétude.
\VS{19}Puis tu diras au peuple du pays : Ainsi parle le Seigneur Yahweh, sur les habitants de Jérusalem, à la terre d'Israël : Ils mangeront leur pain avec chagrin, et ils boiront leur eau avec frayeur, parce que son pays sera désolé, étant privé de son abondance, à cause de la violence de tous ceux qui y habitent.
\VS{20}Les villes peuplées seront désertes, et le pays ne sera que désolation ; et vous saurez que je suis Yahweh.
\VS{21}La parole de Yahweh me fut encore adressée en ces mots :
\VS{22}Fils de l'homme, que signifient ces discours moqueurs que vous tenez sur la terre d'Israël, en disant : Les jours seront prolongés, et toute vision périra\FTNT{Es. 5:19 ; Am. 6:3 ; 2 Pi. 3:3.} ?
\VS{23}C'est pourquoi dis-leur : Ainsi parle le Seigneur Yahweh : Je ferai cesser ce proverbe, et on ne s'en servira plus comme proverbe en Israël ; et dis-leur : Les jours approchent, et toutes les visions s'accompliront.
\VS{24}Car il n'y aura plus désormais aucune vision de vanité ni aucune divination de flatteur, au milieu de la maison d'Israël.
\VS{25}Car moi, Yahweh, je parlerai, et la parole que j'aurai prononcée sera mis en exécution, elle ne sera plus différée ; mais ô maison rebelle! Je prononcerai en vos jours la parole, et je l'exécuterai dit le Seigneur Yahweh.
\VS{26}La parole de Yahweh me fut encore adressée en ces mots :
\VS{27}Fils de l'homme, voici, ceux de la maison d'Israël disent : La vision que celui-ci voit n'arrivera pas avant longtemps, et il prophétise pour des temps qui sont encore éloignés.
\VS{28}C'est pourquoi dis-leur : Ainsi parle le Seigneur Yahweh : Aucune de mes paroles ne sera plus différée, mais la parole que j'aurai prononcée sera exécutée incessament, dit le Seigneur Yahweh.
\Chap{13}
\TextTitle{Jugement sur ceux qui égarent le peuple de Dieu}
\VerseOne{}La parole de Yahweh me fut encore adressée en ces mots :
\VS{2}Fils de l'homme, prophétise contre les prophètes d'Israël qui prophétisent, et dis à ces prophètes qui prophétisent selon leur propre cœur : Ecoutez la parole de Yahweh !
\VS{3}Ainsi parle le Seigneur Yahweh : Malheur aux prophètes insensés qui suivent leur propre esprit, et qui n'ont point eu de vision.
\VS{4}Israël, tes prophètes ont été comme des renards dans les déserts.
\VS{5}Vous n'êtes point montés devant les brèches, et vous n'avez point réparé les murs pour la maison d'Israël, afin de vous tenir debout pour le combat au jour de Yahweh.
\VS{6}Ils ont eu des visions vaines et des divinations de mensonge, ils disent : Yahweh a dit ; et toutefois Yahweh ne les a point envoyés ; et ils font espérer que leur parole s'accomplira\FTNT{Les faux prophètes (Jé. 23) ; Jé. 14:14 ; Jé. 28:15.}.
\VS{7}N'avez-vous pas vu des visions de vanité, et prononcé des divinations de mensonge ? Cependant vous dites : Yahweh a parlé ; et je n'ai point parlé.
\VS{8}C'est pourquoi ainsi parle le Seigneur Yahweh : Parce que vous avez prononcé des choses vaines, et que vous avez eu des visions de mensonge, à cause de cela j'en veux à vous, dit le Seigneur Yahweh.
\VS{9}Et ma main sera sur les prophètes qui ont des visions de vanité et des divinations de mensonge ; ils ne seront plus admis dans le conseil de mon peuple, ils ne seront plus écrits dans les registres de la maison d'Israël, ils n'entreront plus dans la terre d'Israël ; et vous saurez que je suis le Seigneur Yahweh.
\VS{10}Parce, oui parce qu'ils ont abusé mon peuple, en disant : Paix ! Et il n'y avait point de paix\FTNT{Jé. 6:14 ; Jé. 8:11.}. L'un bâtissait le mur, et les autres l'induissaient de mortier mal lié.
\VS{11}Dis à ceux qui enduisent le mur de mortier mal lié, qu'il tombera ; il y aura une pluie débordante, et vous, pierres de grêle, vous tomberez sur lui, et un vent de tempête le fendra.
\VS{12}Et voici, le mur est tombé ; ne vous sera-t-il donc pas dit : Où est l'enduit dont vous l'avez couvert ?
\VS{13}C'est pourquoi, ainsi parle le Seigneur Yahweh : Je ferai dans ma fureur éclater un vent impétueux, et dans ma colère, il surviendra une pluie débordante et des pierres de grêle dans ma fureur, pour détruire entièrement.
\VS{14}Je démolirai le mur que vous avez enduit de mortier mal lié, je le jetterai par terre, tellement que son fondement sera découvert, et il tombera ; vous serez consumés au milieu de lui, et vous saurez que je suis Yahweh.
\VS{15}Ainsi j'accomplirai ma colère contre le mur, et contre ceux qui l'enduisent de mortier mal lié ; et je vous dirai : Le mur n'est plus ni ceux qui l'ont enduit ;
\VS{16}à savoir les prophètes d'Israël, qui prophétisent sur Jérusalem et qui voient pour elle des visions de paix ; et néanmoins il n'y a point de paix, dit le Seigneur Yahweh.
\VS{17}Aussi, toi, fils de l'homme, tourne ta face contre les filles de ton peuple qui prophétisent selon leur propre cœur, prophétise contre elles !
\VS{18} Et dis : ainsi parle le Seigneur Yahweh : Malheur à celles qui cousent des coussins\FTNT{Le mot « coussin » vient du terme hébreu « keceth », et signifie : bande, filet, faux phylactères,  tissu utilisé par les fausses prophétesses en Israël pour étayer leurs plans démoniaques de diseuses de bonne aventure. } pour s'accouder le long du bras jusqu'aux mains, et qui font des voiles pour mettre sur la tête des personnes de toute taille, pour séduire les âmes. Séduiriez-vous les âmes de mon peuple\FTNT{Ge. 10:9.} ; et conserveriez-vous vos âmes ?
\VS{19}Et me profaneriez-vous envers mon peuple pour des poignées d'orge et pour des morceaux de pain, en faisant mourir les âmes qui ne devaient point mourir et en faisant vivre les âmes qui ne devaient point vivre, en mentant à mon peuple qui écoute le mensonge ?
\VS{20}C'est pourquoi ainsi parle le Seigneur Yahweh : Voici, j'en veux à vos coussins, par lesquels vous séduisez les âmes pour les faire voler vers vous ; et je déchirerai ces coussins de vos bras, et je ferai échapper les âmes que vous avez attirées afin qu'elles volent vers vous\FTNT{Ap. 18:11-13 ; 1 Co. 6:10 ; 2 Pi. 2:14}.
\VS{21}Je déchirerai aussi vos voiles, et je délivrerai mon peuple d'entre vos mains, et ils ne seront plus entre vos mains pour en faire votre proie ; et vous saurez que je suis Yahweh.
\VS{22}Parce que vous avez affligé sans raison le cœur du juste, quand moi-même je ne l'ai point attristé, et que vous avez renforcé les mains du méchant, afin qu'il ne se détourne point de son mauvais chemin, et que je lui sauve la vie.
\VS{23}C'est pourquoi, vous n'aurez plus aucune vision de vanité ni aucune divination, mais je délivrerai mon peuple d'entre vos mains ; et vous saurez que je suis Yahweh.
\Chap{14}
\TextTitle{Jugement sur les anciens idolâtres}
\VerseOne{}Or quelques-uns des anciens d'Israël vinrent auprès de moi et s'assirent devant moi.
\VS{2}Et la parole de Yahweh me fut adressée en ces mots :
\VS{3}Fils de l'homme, ces gens élèvent leurs idoles dans leurs cœurs, et ils attachent les regards sur ce qui les fait tomber dans l'iniquité. Serais-je consulté par eux sérieusement ?
\VS{4}C'est pourquoi parle-leur et dis-leur : Ainsi parle le Seigneur Yahweh. Quiconque de la maison d'Israël aura élevé ses idoles dans son cœur, et aura mis devant sa face ce qui l'a fait tomber dans son iniquité, s'il vient vers le prophète, je suis Yahweh, je lui répondrai puisqu'il vient avec la multitude de ses idoles,
\VS{5}afin que je saisisse la maison d'Israël par leur propre cœur; car eux tous se sont éloignés de moi par leurs idoles.
\VS{6}C'est pourquoi dis à la maison d'Israël : Ainsi parle le Seigneur Yahweh : Revenez, et détournez-vous de vos idoles, détournez les regards de toutes vos abominations\FTNT{Es. 55:6-7.}.
\VS{7}Car quiconque de la maison d'Israël, ou des étrangers qui séjournent en Israël, qui s'est séparé de moi, qui éleve ses idoles dans son cœur, et attache ses regards sur ce qui l'a fait tomber dans l'iniquité, s'il vient vers le prophète pour me consulter par de lui, je suis Yahweh, on lui répondra tout ce qu'on a à lui répondre.
\VS{8}Je me tournerai contre cet homme\FTNT{Lé. 17:10 ; Lé. 20:3-6 ; Jé. 44:11.}, et je ferai de lui un signe, et un sujet de sarcasme\FTNT{No. 26:10 ; De. 28:37}. Je le retrancherai du milieu de mon peuple ; et vous saurez que je suis Yahweh.
\VS{9}S'il arrive que le prophète soit séduit, et qu'il profère quelque parole, moi, Yahweh, je séduirai ce prophète-là\FTNT{1 R. 22:23 ; Job. 12:16 ; 2 Th. 2:11.} ; et j'étendrai ma main sur lui, et je l'exterminerai du milieu de mon peuple d'Israël.
\VS{10}Et ils porteront la peine de leur iniquité ; la peine de l'iniquité du prophète sera comme la peine de celui qui l'aura interrogé ;
\VS{11}afin que la maison d'Israël ne s'éloigne plus de moi, et qu'ils ne se souillent plus par tous leurs crimes\FTNT{Jé. 31:18-19 ; Hé. 12:11 ; Ja. 1:1-3.} ; alors ils seront mon peuple, et je serai leur Dieu, dit le Seigneur Yahweh.
\TextTitle{Châtiments d'Israël ; Yahweh épargne un reste}
\VS{12}Puis la parole de Yahweh me fut adressée en ces mots :
\VS{13}Fils de l'homme, lorsqu'un pays aura péché contre moi, en commettant une infidélité, et que j'aurai étendu ma main contre lui, et que je lui aurai rompu le bâton du pain, envoyé la famine et retranché du milieu de lui tant les hommes que les bêtes,
\VS{14}et que ces trois hommes, Noé, Daniel et Job s'y trouvent, ils sauveraient leurs âmes par leur justice, dit le Seigneur Yahweh.
\VS{15}Si je fais passer les bêtes féroces par ce pays-là et qu'elles le privent d'enfants, tellement qu'il soit devenu un désert où personne ne passe à cause des bêtes,
\VS{16}et que ces trois hommes-là s'y trouvent, je suis vivant, dit le Seigneur Yahweh, ils ne sauveraient ni fils ni filles, eux seulement seraient sauvés, et le pays sera un désert.
\VS{17}Si je faisais venir l'épée sur ce pays-là et si je disais : Que l'épée passe par le pays, et qu'elle en retranche les hommes et les bêtes !
\VS{18}Si ces trois hommes-là se trouvent au milieu du pays, je suis vivant, dit le Seigneur, ils ne sauveraient ni fils ni filles ; mais eux seulement seraient sauvés.
\VS{19}Ou si j'envoyais la peste dans ce pays, et que je répandais ma colère contre lui jusqu'à faire ruisseler le sang, au point de retrancher du milieu de lui les hommes et les bêtes,
\VS{20}et que Noé\FTNT{Ge. 6:8.}, Daniel\FTNT{Da. 1:8-12.} et Job\FTNT{Job. 1:8.}, s'y trouvent, je suis vivant, dit le Seigneur Yahweh, ils ne sauveraient ni fils ni filles ; mais ils sauveraient leurs âmes par leur justice.
\VS{21}Car ainsi parle le Seigneur Yahweh : J'envoie mes quatre plaies mortelles, l'épée, la famine, les bêtes féroces, et la peste, contre Jérusalem, pour en retrancher les hommes et les bêtes\FTNT{Jé. 15:2-3.} ;
\VS{22}Et toutefois, il y aura un reste qui échappera, qui en sortira, des fils et des filles. Voici, ils viennent vers vous, et vous verrez leur conduite et leurs actions, et vous serez consolés du malheur que je fais venir contre Jérusalem, de tout ce que je fais venir sur elle.
\VS{23}Vous serez consolés, lorsque vous verrez leur conduite et leurs actions ; et vous reconnaîtrez que ce n'est pas sans raison que je fais tout ce que je lui fais, dit le Seigneur Yahweh\FTNT{Jé. 22:8-9.}.
\Chap{15}
\TextTitle{Infidélités d'Israël\FTNTT{Cp. Es. 5:1-24.}}
\VerseOne{}La parole de Yahweh me fut encore adressée, en disant :
\VS{2}Fils de l'homme, que vaut le bois de la vigne de plus que les autres bois ? Et les sarments de plus que les branches des arbres d'une forêt ?
\VS{3}Et prendra-t-on du bois pour en faire quelque ouvrage ? Ou prendra-t-on un clou pour y pendre quelque chose ?
\VS{4}Voici, on le met au feu pour être consumé ; le feu consume aussitôt ses deux bouts, et le milieu est en feu ; serait-il bon pour quelque ouvrage ?
\VS{5}Voici, quand il est entier, on n'en fait aucun ouvrage ; à plus forte raison quand le feu l'aura consumé et qu'il sera brûlé, sera-t-il bon pour quelque ouvrage ?
\VS{6}C'est pourquoi ainsi parle le Seigneur Yahweh : Comme le bois de la vigne est parmi les arbres d'une forêt, que j'ai assigné au feu pour être consumé, ainsi je livrerai les habitants de Jérusalem.
\VS{7}Je me tournerai contre eux, seront-ils sortis du feu ? Encore le feu les consumera ; et vous saurez que je suis Yahweh, quand je tournerai ma face contre eux.
\VS{8}Je ferai de ce pays une désolation, parce qu'ils ont commis une infidélité, dit le Seigneur Yahweh.
\Chap{16}
\TextTitle{Bonté de Yahweh, prostitutions d'Israël}
\VerseOne{}La parole de Yahweh me fut aussi adressée en ces mots :
\VS{2}Fils de l'homme, fais connaître à Jérusalem ses abominations.
\VS{3}Et dis : Ainsi parle le Seigneur Yahweh à Jérusalem : Tu as tiré ton origine et ta naissance du pays de Canaan ; ton père était Amoréen, et ta mère Héthienne.
\VS{4}Quant à ta naissance, le jour où tu naquis, ton cordon ombilical n'a pas été coupé, tu n'as pas été lavée dans l'eau pour être nettoyée ; tu n'as pas été salée de sel ni emmaillotée.
\VS{5}Il n'y a pas eu d'œil qui ait eu pitié de toi pour te faire une seule de ces choses, en ayant compassion pour toi ; mais tu as été jetée sur la face des champs le jour de ta naissance, parce qu'on avait horreur de toi.
\VS{6}Et passant près de toi, je te vis gisante par terre, dans ton sang, et je te dis : Vis dans ton sang ! Et je te redis encore : Vis dans ton sang !
\VS{7}Je t'ai fait croître par millions comme l'herbe des champs. Et tu pris de l'accroissement et tu devins grande, tu parvins à une parfaite bauté, tes seins se formèrent, ta chevelure poussa, tu devins nubile; mais tu étais abandonnée et sans habits.
\VS{8}Je passai près de toi, je te regardai, et voici, le temps était là, le temps des amours. J'étendis sur toi le pan de ma robe, et je couvris ta nudité. Je te jurai, j'entrai en alliance avec toi, dit le Seigneur Yahweh, et tu devins mienne.
\VS{9}Je te lavai dans l'eau en t'y plongeant, j'ôtai le sang de dessus toi, et je t'oignis d'huile.
\VS{10}Je te revêtis de vêtements brodés, je te chaussai de fourrure, je te ceignis de fin lin, et je te couvris de soie.
\VS{11}Je te parai d'ornements : Je mis des bracelets sur tes mains, et un collier à ton cou.
\VS{12}Je mis un anneau à ton nez, des pendants à tes oreilles, et une couronne de gloire sur ta tête.
\VS{13}Tu fus donc parée d'or et d'argent, et ton vêtement était de fin lin, de soie, et de broderie ; tu mangeas la fleur de farine, le miel, et l'huile ; tu devins extrêmement belle, et tu prospéras jusqu'à régner.
\VS{14}Ta renommée se répandit parmi les nations à cause de ta beauté, car elle était parfaite, à cause de ma gloire que j'avais mise sur toi, dit le Seigneur Yahweh.
\VS{15}Mais tu t'es confiée dans ta beauté, et tu t'es prostituée à cause de ta renommée, tu t'es abandonnée à tous les passants\FTNT{Es. 1:21 ; Jé. 2:20 ; Jé. 3:2-6 ; Os. 1:2.}.
\VS{16}Tu as pris tes vêtements pour t'en faire des hauts lieux de diverses couleurs, tels qu'il n'y en a point eu et n'en aura jamais, et tu t'y es prostituée.
\VS{17}Tu as pris ta magnifique parure d'or et d'argent, que je t'avais donnée, et tu t'en es fait des images d'hommes, tu as commis la fornication avec elles.
\VS{18}Tu as pris tes vêtements brodés, tu les en as couvertes, et tu as mis mon huile et mon encens devant elles.
\VS{19}Mon pain que je t'avais donné, la fleur de farine, l'huile, et le miel que je t'avais donné à manger, tu as mis cela devant elle en sacrifice de bonne odeur ; il a été fait ainsi, dit le Seigneur Yahweh.
\VS{20}Tu as aussi pris tes fils et tes filles que tu m'avais enfantés, et tu les as sacrifiés pour être mangés\FTNT{Lé. 18:21 ; Lé. 20:2 ; Es. 57:5 ; Jé. 19:5, Jé. 32:35.}. N'était-ce pas assez de tes prostitutions ?
\VS{21}Tu as égorgé mes fils, et tu les as livrés pour les faire passer par le feu, en l'honneur de ces idoles\FTNT{2 R. 17:17.}.
\VS{22}Et parmi toutes tes abominations et tes adultères, tu ne t'es point souvenue du temps de ta jeunesse, quand tu étais sans habits et toute nue, gisante par terre dans ton sang.
\VS{23}Après toutes tes méchancetés, malheur, malheur à toi ! dit le Seigneur Yahweh.
\VS{24}Tu t'es bâti un lieu éminent, et tu t'es fait des hauts lieux dans toutes les places.
\VS{25}A l'entrée de chaque chemin tu as bâti un haut lieu, et tu as rendu ta beauté abominable, tu t'es prostituée à tous les passants, tu as multiplié tes adultères.
\VS{26}Tu t'es abandonnée aux fils d'Egypte, tes voisins au corps avantageux ; tu as multiplié tes adultères pour m'irriter.
\VS{27}Et voici, j'ai étendu ma main sur toi, j'ai diminué la portion que je t'avais prescrite, et je t'ai abandonnée à la volonté de celles qui te haïssaient, des filles des Philistins, lesquelles ont honte de tes voies qui ne sont que méchanceté.
\VS{28}Tu t'es aussi abandonnée aux fils des Assyriens\FTNT{2 R. 16:7-10 ; Jé. 2:18-36.}, parce que tu n'étais pas encore rassasiée ; et après avoir commis l'adultère avec eux, tu n'as point encore été rassasiée.
\VS{29}Tu as multiplié tes adultères dans le pays de Canaan jusqu'en Chaldée, et avec cela tu n'as pas encore été rassasiée.
\VS{30}Quelle faiblesse de cœur tu as eue, dit le Seigneur, Yahweh, d'avoir fait toutes ces choses-là, qui sont les actions d'une femme qui se prostitue avec arrogance.
\VS{31}De t'être bâti un lieu éminent à chaque entrée de chemin, et d'avoir fait ton haut lieu dans toutes les places. Et tu n'as pas été comme la prostituée, car tu n'as point tenu compte du salaire.
\VS{32}Femme adultère, tu prends des étrangers au lieu de ton mari.
\VS{33}On donne un salaire à toutes les prostituées, mais toi tu as donné à tous tes amants des présents\FTNT{Es. 57:8-9 ; Os. 8:9-10.}, tu les as gagnés par des présents, afin que de toutes parts ils viennent vers toi, pour se plonger avec toi dans le crime.
\VS{34}Tu as été le contraire des autres prostituées, parce qu'on ne te recherchait pas ; et en donnant un salaire au lieu d'en recevoir un, tu as été le contraire des autres.
\TextTitle{Conséquences de l'infidélité de Jérusalem}
\VS{35}C'est pourquoi, ô adultère, écoute la parole de Yahweh !
\VS{36}Ainsi parle le Seigneur, Yahweh : Parce que ton venin s'est répandu, et que dans tes excès tu t'es abandonnée à ceux que tu aimais, à tes abominables idoles, et que tu as mis à mort tes fils que tu leur as donnés ;
\VS{37}à cause de cela, voici, je vais rassembler tous tes amants, avec lesquels tu te plaisais, et tous ceux que tu as aimés, avec tous ceux que tu as haïs ; je les assemblerai de toutes parts contre toi, et je découvrirai ta honte à leurs yeux et ils verront ton infamie.
\VS{38}Et je te jugerai comme on juge les femmes adultères, et celles qui répandent le sang\FTNT{Lé. 20:10 ; De. 22:22-30.} ; je te livrerai pour être mise à mort selon ma fureur et ma jalousie.
\VS{39}Je te livrerai, dis-je, entre leurs mains ; et ils détruiront tes maisons de prostitution, et ils détruiront tes hauts lieux ; ils te dépouilleront de tes vêtements, emporteront ta magnifique parure, et ils te laisseront sans habits et entièrement nue.
\VS{40}Et on fera monter contre toi une foule de gens qui te lapideront de pierres, et qui te perceront avec leurs épées.
\VS{41}Puis ils brûleront tes maisons, et feront justice de toi aux yeux d'un grand nombre de femmes, je ferai cesser tes prostitutions et tu ne donneras plus de salaires.
\VS{42}J'abandonnerai alors ma colère contre toi, et ma jalousie se retirera de toi ; je serai en repos, et je ne m'irriterai plus.
\VS{43}Parce que tu ne t'es point souvenue du temps de ta jeunesse, et que tu m'as provoqué par toutes ces choses-là ; à cause de cela, voici, j'ai fait tomber la peine de tes crimes sur ta tête, dit le Seigneur,Yahweh ; et tu ne feras plus de mauvais projets avec toutes tes abominations.
\VS{44}Voici, tous ceux qui usent de proverbes feront un proverbe de toi, en disant : Telle mère, telle fille !
\VS{45}Tu es la fille de ta mère, qui a dédaigné son mari et ses fils ; et tu es la sœur de chacune de tes sœurs, qui ont dédaigné leurs maris et leurs fils. Votre mère était Héthienne, et votre père était Amoréen.
\VS{46}Ta grande sœur qui demeure à ta gauche, c'est Samarie avec ses filles ; et ta petite sœur qui demeure à ta droite, c'est Sodome avec ses filles.
\VS{47}Et tu n'as pas seulement marché dans leurs voies et fait selon leurs abominations, c'était fort peu ; mais tu t'es corrompue plus qu'elles dans toutes tes voies.
\VS{48}Je suis vivant ! dit le Seigneur Yahweh, Sodome, ta sœur et tes filles, n'ont point fait comme tu as fait, toi et tes filles.
\VS{49}Voici quel a été le crime de Sodome, ta sœur : Elle avait de l'orgueil, elle vivait dans l'abondance de pain, et dans une insouciante tranquillité, elle et ses filles, elle ne fortifiait pas la main du pauvre et de l'indigent.
\VS{50}Elles se sont élevées, elles ont commis des abominations devant moi, et je me suis détourné quand j'ai vu cela.
\VS{51}Quant à Samarie, elle n'a pas commis la moitié de tes péchés ; car tu as multiplié tes abominations plus qu'elle, et tu as justifié tes sœurs par toutes les abominations que tu as commises.
\VS{52}Porte ta honte, toi qui as jugé chacune de tes sœurs, à cause de tes péchés, par lesquels tu as été rendue plus abominable qu'elles ; elles sont plus justes que toi ; c'est pourquoi sois honteuse, et porte ta confusion, vu que tu as justifié tes sœurs.
\VS{53}Quand je ramènerai leurs captifs, les captifs, dis-je, de Sodome et des villes de son ressort; et les captifs de Samarie et de villes de son ressort; je ramenèrai aussi les captifs de la captivité parmi elles!
\VS{54}afin que tu portes ta honte, et que tu sois confuse à cause de tout ce que tu as fait, et que tu les consoles.
\VS{55}Quand ta sœur Sodome, et les villes de son ressort retourneront à leur état précédent ; Samarie et les villes de son ressort retourneront à leur état précédent ; toi aussi, et les villes de ton ressort retournerez à votre état précédent.
\VS{56}Or ta bouche n'a point fait mention de ta sœur, Sodome, au jour de tes fiertés,
\VS{57}avant que ta méchanceté soit découverte ; lorsque tu as reçu les outrages des filles de Syrie, et de tous ses alentours, des filles des Philistins, qui te pillèrent de tous côtés !
\VS{58}Tu portes sur toi tes méchancetés et tes abominations, dit Yahweh.
\VS{59}Car ainsi parle le Seigneur Yahweh : Je te ferai comme tu as fait, quand tu as méprisé le serment en rompant l'alliance.
\TextTitle{Fidélité de Yahweh à son alliance}
\VS{60}Mais je me souviendrai de l'alliance que j'ai traitée avec toi dans les jours de ta jeunesse, et j'établirai avec toi une alliance éternelle\FTNT{Lé. 26:42-45 ; Ps. 106:45.}.
\VS{61}Et tu te souviendras de tes voies, et en seras confuse, lorsque tu recevras tes sœurs, tant tes plus grandes que tes plus petites, et je te les donnerai pour filles ; mais pas selon ton alliance.
\VS{62}Car j'établirai mon alliance avec toi, et tu sauras que je suis Yahweh,
\VS{63}afin que tu te souviennes de ta vie passée, que tu en sois honteuse, et que tu n'ouvres plus la bouche, à cause de ta confusion, après que j'aurai été apaisé envers toi, pour tout ce que tu auras fait, dit le Seigneur Yahweh.
\Chap{17}
\TextTitle{Enigme de Yahweh}
\VerseOne{}La parole de Yahweh me fut adressée en ces mots :
\VS{2}Fils de l'homme, propose une énigme, une parabole à la maison d'Israël.
\VS{3}Tu diras : Ainsi parle le Seigneur Yahweh : Un grand aigle à grandes ailes, aux ailes déployées, couvert de plumes de toutes les couleurs, vint au Liban, et enleva la cime d'un cèdre.
\VS{4}Il arracha la tête de ses rameaux, l'emmena dans un pays de commerce, et la mit dans une ville marchande.
\VS{5}Il prit de la semence du pays, et la mit dans un champ propre à semer, l'apporta près des grosses eaux, la planta comme un saule.
\VS{6}Cette semence poussa et devint un cep de vigne étendu, mais de peu d'élévation ; ses rameaux étaient tournés vers l'aigle, et ses racines étaient sous lui ; il devint une vigne, donna des jets, et produisit des branches.
\VS{7}Mais il y avait un autre grand aigle, aux longues ailes, et au plumage épais. Et voici, cette vigne serra vers lui ses racines, et étendit ses rameaux vers lui, afin qu'il l'arrose des eaux qui coulent des terrasses.
\VS{8}Elle était donc plantée dans une bonne terre, près des grosses eaux, en sorte qu'il y sortait des sarments et portait du fruit\FTNT{Mt. 13:8-23 ; Mc. 4:8-20 ; Lu. 8:8-15.}. Elle était devenue une vigne magnifique.
\VS{9}Dis : Ainsi parle le Seigneur Yahweh, prospèrera-t-elle ? N'arrachera-t-il pas ses racines, et ne coupera-t-il pas ses fruits pour qu'ils deviennent secs ? Tous les sarments qu'il a jetés sécheront, et il ne faudra pas un grand effort et beaucoup de monde pour l'enlever de dessus ses racines.
\VS{10}Mais voici, quoique plantée, prospèrera-t-elle ? Quand le vent d'orient l'aura touchée, ne séchera-t-elle pas entièrement ? Elle séchera sur le terrain où elle était plantée.
\TextTitle{Jugement de Dieu sur Sédécias\FTNTT{2 R. 24:17-20 ; 25:1-10.}}
\VS{11}Puis la parole de Yahweh me fut adressée en ces mots :
\VS{12}Parle maintenant à la maison rebelle : Ne savez-vous pas ce que veulent dire ces choses ? Dis : Voici, le roi de Babylone est venu à Jérusalem. Il a pris le roi, et les princes, et les a emmenés avec lui à Babylone.
\VS{13}Il a pris un de la race royale, il a traité alliance avec lui, il lui a fait prêter serment, et il a retenu les puissants du pays,
\VS{14}afin que le royaume soit tenu dans l'abaissement, et qu'il ne s'élève point, mais qu'en gardant son alliance, il subsiste.
\VS{15}Mais celui-ci s'est rebellé contre lui, envoyant ses messagers en Egypte, pour qu'on lui donne des chevaux et un grand peuple. Celui qui fait de telles choses prospérera-t-il, échappera-t-il ? Ayant enfreint l'alliance, échappera-t-il ?
\VS{16}Je suis vivant, dit le Seigneur Yahweh, c'est dans le pays du roi qui l'a établi pour roi, envers qui il a violé son serment et dont il a rompu l'alliance, c'est près de lui, au milieu de Babylone, qu'il mourra\FTNT{Référence à Nebucadnetsar. Sédécias eut les yeux crevés avant d'être emmené captif (2 R. 25:7 ; Jé. 34:3 ; Jé. 52:11).}.
\VS{17}Pharaon n'ira pas avec une grande armée et un peuple nombreux pour le secourir dans cette guerre, lorsque l'ennemi élèvera des terrasses et fera des retranchements pour exterminer beaucoup d'âmes.
\VS{18}Car il a méprisé le serment en violant l'alliance ; car voici, après avoir donné sa main, il a fait néanmoins toutes ces choses-là ; il n'échappera point !
\VS{19}C'est pourquoi ainsi parle le Seigneur Yahweh : Je suis vivant, si je ne fais tomber sur sa tête mon serment qu'il a méprisé, et mon alliance qu'il a enfreinte.
\VS{20}Et j'étendrai mon rets sur lui, et il sera pris dans mes filets, je le ferai entrer dans Babylone, et là j'entrerai en jugement contre lui pour le crime qu'il a commis contre moi.
\VS{21}Et tous ses fugitifs avec toutes ses troupes tomberont par l'épée, et ceux qui resteront seront dispersés à tout vent ; et vous saurez que moi, Yahweh, j'ai parlé.
\VS{22}Ainsi parle le Seigneur Yahweh : Je prendrai aussi un rameau de la cime de ce haut cèdre, et je le planterai ; je couperai, dis-je, du bout de ses jeunes branches, un tendre rameau, et je le planterai sur une montagne haute et éminente.
\VS{23}Je le planterai sur la haute montagne d'Israël, et là il produira des branches et produira du fruit, et il deviendra un excellent cèdre ; et des oiseaux de tout plumage demeureront sous lui, et habiteront sous l'ombre de ses branches.
\VS{24}Et tous les bois des champs connaîtront que moi, Yahweh, j'aurai abaissé le grand arbre, et élevé le petit arbre, fait sécher le bois vert, et fait reverdir le bois sec ; moi, Yahweh, j'ai parlé, et je le ferai.
\Chap{18}
\TextTitle{Chacun responsable de son péché}
\VerseOne{}La parole de Yahweh me fut encore adressée, en disant :
\VS{2}Que voulez-vous dire, vous qui usez ordinairement de ce proverbe touchant le pays d'Israël, en disant : Les pères ont mangé des raisins verts et les dents des enfants ont été agacées\FTNT{Jé. 31:29 ; La. 5:7.} ?
\VS{3}Je suis vivant, dit le Seigneur Yahweh et vous n'userez plus de ce proverbe en Israël.
\VS{4}Voici, toutes les âmes sont à moi ; l'âme du fils est à moi comme l'âme du père ; l'âme qui pèche sera celle qui mourra.
\VS{5}Mais l'homme qui est juste, et qui pratique la droiture et la justice,
\VS{6}qui ne mange pas sur les montagnes, et qui ne lève pas ses yeux vers les idoles de la maison d'Israël, et qui ne souille pas la femme de son prochain et ne s'approche pas de la femme dans son état d'impureté\FTNT{Lé 18:18 ; Lé. 20:18.},
\VS{7}qui n'opprime personne, qui rend le gage à son débiteur\FTNT{Ex. 22:26 ; De. 24:12-13.}, qui ne ravit pas le bien d'autrui, qui donne son pain à celui qui a faim et qui couvre d'un vêtement celui qui est nu\FTNT{De. 15:11 ; Es. 58:7.},
\VS{8}qui ne prête pas à intérêt, et ne tire pas d'usure, qui détourne sa main de l'iniquité et qui juge selon la vérité entre les parties qui plaident ensemble\FTNT{Ex. 22:25 ; Lé. 25:35-37 ; De. 23:19.},
\VS{9}qui suit mes lois et garde mes ordonnances pour agir avec fidélité, celui-là est juste, certainement il vivra, dit le Seigneur Yahweh.
\VS{10}Et s'il a engendré un fils qui soit un meurtrier, répandant le sang, et commettant des choses semblables ;
\VS{11}et qui ne fasse aucune de ces choses que j'ai ordonnées, s'il mange sur les montagnes, s'il déshonore la femme de son prochain,
\VS{12}s'il opprime le malheureux et le pauvre, s'il ravit le bien d'autrui, s'il ne rend pas le gage, s'il lève ses yeux vers les idoles et commet des abominations,
\VS{13}s'il prête à intérêt, et tire une usure, ce fils-là, vivrait ? Il ne vivra pas, quand il aura commis toutes ces abominations, on le fera mourir, et son sang retombera sur lui.
\VS{14}Mais s'il engendre un fils qui voie tous les péchés que commet son père, qui les voie et n'agisse pas de la même manière ;
\VS{15}S'il ne mange pas sur les montagnes et qu'il ne lève point ses yeux vers les idoles de la maison d'Israël, s'il ne déshonore pas la femme de son prochain,
\VS{16}s'il n'opprime personne, s'il ne prend point de gages, s'il ne ravit point le bien d'autrui, s'il donne de son pain à celui qui a faim et couvre celui qui est nu,
\VS{17}s'il retire sa main du pauvre, s'il n'exige ni usure ni intérêt, s'il garde mes ordonnances, et s'il suit mes lois ; il ne mourra point pour l'iniquité de son père, mais certainement il vivra.
\VS{18}Mais son père, parce qu'il a usé de fraude, et qu'il a ravi ce qui était à son frère, et fait parmi son peuple ce qui n'est pas bon, voici, il mourra pour son iniquité.
\VS{19}Mais, direz-vous : Pourquoi le fils ne porte-t-il pas l'iniquité de son père\FTNT{Ex. 20:5 ; De. 5:9.} ? Parce que le fils a fait ce qui était juste et droit, et qu'il a gardé toutes mes lois et les a observées, certainement il vivra.
\VS{20}L'âme qui pèche est celle qui mourra. Le fils ne portera point l'iniquité du père, et le père ne portera point l'iniquité du fils. La justice du juste sera sur le juste, et la méchanceté du méchant sera sur le méchant.
\VS{21}Si le méchant se détourne de tous ses péchés qu'il aura commis, et qu'il garde toutes mes lois, et fasse ce qui est juste et droit, certainement il vivra, il ne mourra point.
\VS{22}Il ne lui sera point fait mention de tous ses crimes qu'il aura commis, mais il vivra pour sa justice, à laquelle il se sera adonné.
\VS{23}Ce que je désire, est-ce que le méchant meure ? dit le Seigneur Yahweh. N'est-ce pas qu'il se détourne de ses mauvaises voies et qu'il vive ?
\VS{24}Mais si le juste se détourne de sa justice, et commet l'iniquité, selon toutes les abominations que le méchant a l'habitude de commettre, vivra-t-il ? Il ne sera point fait mention de toutes ses justices qu'il aura faites, à cause de son crime qu'il aura commis, et à cause de son péché qu'il aura fait ; il mourra à cause de ces choses-là.
\VS{25}Et vous, vous dites : La voie du Seigneur n'est pas bien réglée. Ecoutez maintenant maison d'Israël, ma voie n'est-elle pas bien réglée ? Ne sont-ce pas plutôt vos voies qui ne sont pas bien réglées ?
\VS{26}Si le juste se détourne de sa justice, et commet l'iniquité, il mourra à cause de ces choses-là ; il mourra à cause de son iniquité qu'il aura commise.
\VS{27}Si le méchant se détourne de sa méchanceté qu'il aura commise, et pratique ce qui est juste et droit, il fera vivre son âme.
\VS{28}Ayant donc considéré sa conduite, et s'étant détourné de tous ses crimes qu'il aura commis, certainement il vivra, il ne mourra point.
\VS{29}La maison d'Israël dit : La voie du Seigneur Yahweh n'est pas bien réglée. Ô maison d'Israël ! Mes voies ne sont-elles pas bien réglées ? Ne sont-ce pas plutôt vos voies qui ne sont pas bien réglées ?
\VS{30}C'est pourquoi je jugerai chacun de vous selon ses voies, ô maison d'Israël ! dit le Seigneur. Revenez, et détournez-vous de tous vos péchés, et l'iniquité ne vous ruinera pas.
\VS{31}Rejetez loin de vous tous les crimes par lesquels vous avez péché ; et faites-vous un nouveau cœur et un esprit nouveau ; pourquoi mourriez-vous, ô maison d'Israël ?
\VS{32}Car je ne désire pas la mort de celui qui meurt, dit le Seigneur Yahweh. Convertissez-vous donc, et vivez\FTNT{Ac. 3:19-20.}.
\Chap{19}
\TextTitle{Complaintes sur les dirigeants d'Israël}
\VerseOne{}Et toi, prononce à haute voix une complainte touchant les princes d'Israël.
\VS{2}Et dis : Ta mère, qu'était-ce ? C'était une lionne couchée parmi les lions, et qui a élevé ses petits parmi les jeunes lions.
\VS{3}Elle fit croître un de ses petits, qui devint un jeune lion, et qui apprit à déchirer la proie et a dévorer les hommes.
\VS{4}Les nations en entendirent parler, il fut attrapé dans leur fosse ; et elles l'emmenèrent avec des boucles au pays d'Egypte\FTNT{2 R. 23 : 33-34.}.
\VS{5}Puis ayant vu qu'elle attendait en vain, qu'elle était trompée dans son espérance, elle prit un autre de ses petits, et en fit un jeune lion.
\VS{6}Il marcha parmi les lions et devint un jeune lion, il apprit à déchirer la proie et a dévorer les hommes.
\VS{7}Il désola leurs palais, il ravagea leurs villes, de sorte que le pays, et tout ce qui y est, fut épouvanté par le cri de son rugissement.
\VS{8}Les nations s'armèrent contre lui de toutes les provinces, elles étendirent leurs rets contre lui, et il fut attrapé dans leur fosse\FTNT{2 R. 24:2.}.
\VS{9}Puis ils l'enfermèrent et l'enchaînèrent, pour l'amener au roi de Babylone, et le mettre dans une forteresse, afin que sa voix ne soit plus entendue sur les montagnes d'Israël.
\VS{10}Ta mère était comme une vigne dans ton sang plantée auprès des eaux, et elle est devenue chargée de fruits et de rameaux, à cause des grandes eaux.
\VS{11}Elle avait de puissantes branches pour en faire des sceptres de souverains ; son tronc s'était élevé jusqu'à ses branches touffues, et on la voyait dans sa hauteur avec la multitude de ses rameaux.
\VS{12}Mais elle a été arrachée avec fureur, et jetée par terre ; et le vent d'orient a séché son fruit ; ses puissantes branches se sont rompues et ont séché ; le feu les a consumées.
\VS{13}Maintenant elle est plantée dans le désert, dans une terre sèche et aride.
\VS{14}Le feu est sorti de ses branches, et a consumé son fruit ; et il n'y a plus en elle de puissantes branches pour un sceptre de souverain. C'est là une complainte, et cela servira de complainte.
\Chap{20}
\TextTitle{Compassions de Yahweh face aux infidélités d'Israël}
\VerseOne{}Or il arriva la septième année, au dixième jour du cinquième mois, que quelques-uns des anciens d'Israël vinrent pour consulter Yahweh, et s'assirent devant moi.
\VS{2}La parole de Yahweh me fut adressée en ces mots :
\VS{3}Fils de l'homme, parle aux anciens d'Israël, et dis-leur : Ainsi parle le Seigneur Yahweh : Est-ce pour me consulter que vous venez ? Je suis vivant, dit le Seigneur Yahweh, si vous me consultez.
\VS{4}Ne les jugeras-tu pas, ne les jugeras-tu pas, fils de l'homme ? Donne-leur à connaître les abominations de leurs pères.
\VS{5}Et dis-leur : Ainsi parle le Seigneur Yahweh : Le jour où j'ai choisi Israël, j'ai levé ma main vers la postérité de la maison de Jacob, et je me suis fait connaître à eux dans le pays d'Egypte, et j'ai levé ma main vers eux, en disant : Je suis Yahweh, votre Dieu.
\VS{6}En ce jour, j'ai levé ma main vers eux, pour les faire sortir du pays d'Egypte, pour les amener dans un pays que j'avais cherché pour eux, pays où coulent le lait et le miel, et qui est la noblesse de tous les pays\FTNT{Ex. 3:8 ; Ex. 6:7.}.
\VS{7}Alors je leur dis : Que chacun de vous rejette les abominations qui attirent ses regards, et ne vous souillez point par les idoles d'Egypte ! Je suis Yahweh, votre Dieu\FTNT{Jos. 24:14-23.}.
\VS{8}Mais ils se rebellèrent contre moi, et ils ne voulurent point m'écouter. Aucun ne rejeta les abominations qui attiraient ses regards, et ils n'abandonnèrent point les idoles de l'Egypte. Et je dis que je répandrais ma fureur sur eux, que je consumerais ma colère sur eux au milieu du pays d'Egypte.
\VS{9}Mais je les ai tirés hors du pays d'Egypte, je l'ai fait pour l'amour de mon Nom, afin qu'il ne soit point profané aux yeux des nations parmi lesquelles ils se trouvaient, et aux yeux desquelles je m'étais fait connaître à eux, pour les faire sortir du pays d'Egypte.
\VS{10}Je les fit donc sortir du pays d'Egypte, et je les conduisis dans le désert.
\VS{11}Je leur donnai mes lois et leur fis connaître mes ordonnances, que l'homme doit mettre en pratique, afin de vivre par elles\FTNT{Lé. 18:5 ; Ro. 10:5 ; Ga. 3:12.}.
\VS{12}Je leur donnai aussi mes sabbats, pour être un signe entre moi et eux, afin qu'ils sachent que je suis Yahweh qui les sanctifie\FTNT{Ex. 20:8 ; Ex. 31:13.}.
\VS{13}Mais ceux de la maison d'Israël se rebellèrent contre moi dans le désert. Ils ne suivirent point mes lois, et ils rejetèrent mes ordonnances que l'homme doit mettre en pratique, afin de vivre par elles, et ils profanèrent à l'excès mes sabbats. C'est pourquoi je dis que je répandrais sur eux ma fureur dans le désert pour les consumer\FTNT{Ex. 16:28.}.
\VS{14}Je l'ai fait pour l'amour de mon Nom, afin qu'il ne soit point profané devant les nations, en présence desquelles je les avais fait sortir d'Egypte\FTNT{Ex. 32:12 ; No. 14:13-14 ; De. 9:28 ; Jos. 7:9.}.
\VS{15}Je levai ma main vers eux dans le désert pour ne pas les amener dans le pays que je leur avais donné, pays où coulent le lait et le miel, et qui est la noblesse de tous les pays,
\VS{16}parce qu'ils ont rejeté mes ordonnances, qu'ils n'ont point suivi mes lois, et qu'ils ont profané mes sabbats, car leur cœur ne s'est pas éloigné de leurs idoles.
\VS{17}Toutefois, j'eus pour eux un regard de pitié pour ne point les détruire, et je ne les consumai point entièrement dans le désert.
\VS{18}Mais je dis à leurs fils dans le désert : Ne marchez point dans les statuts de vos pères, et ne gardez point leurs ordonnances, et ne vous souillez point par leurs idoles.
\VS{19}Je suis Yahweh, votre Dieu. Marchez dans mes statuts, et gardez mes ordonnances et accomplissez-les.
\VS{20}Sanctifiez mes sabbats, et ils seront un signe entre moi et vous, afin que vous reconnaissiez que je suis Yahweh, votre Dieu.
\VS{21}Mais les fils se rebellèrent aussi contre moi, et ils ne marchèrent point dans mes statuts, et ne gardèrent point mes ordonnances pour les faire ; ce que l'homme doit accomplir, pour vivre par elles. Ils profanèrent mes sabbats ; c'est pourquoi je dis que je répandrais ma fureur sur eux, et que je consumerais ma colère sur eux dans le désert.
\VS{22}Toutefois, je retirai ma main, et je le fis pour l'amour de mon Nom, afin qu'il ne soit point profané devant les nations, en présence desquelles je les avais sortis d'Egypte.
\VS{23}Néanmoins, je levai ma main vers eux dans le désert, pour les répandre parmi les nations, et les disperser dans les pays\FTNT{Lé. 26:13-33.},
\VS{24}parce qu'ils n'ont point accompli mes ordonnances, et qu'ils ont rejeté mes statuts, profané mes sabbats, et que leurs yeux se sont attachés aux idoles de leurs pères.
\VS{25}A cause de cela, je leur donnai des statuts qui n'étaient pas bons, et des ordonnances par lesquelles ils ne vivraient point.
\VS{26}Je les souillai par leurs dons, quand ils firent passer par le feu tous les premiers-nés, afin de les punir, et que l'on sache que je suis Yahweh.
\VS{27}C'est pourquoi, toi fils de l'homme, parle à la maison d'Israël, et dis-leur : Ainsi parle le Seigneur Yahweh : Vos pères m'ont encore outragé, car ils ont commis un crime contre moi.
\VS{28}Je les ai conduits dans le pays que j'avais juré de leur donner, et ils ont regardé toute haute colline, et tout arbre touffu, ils y ont fait leurs sacrifices, ils y ont posé leur oblation pour m'irriter, ils y ont mis leurs parfums, et ils y ont répandu leurs aspersions.
\VS{29}Je leur ai dit : Que veulent dire ces hauts lieux où vous allez ? Et le nom de hauts lieux leur a été donné jusqu'à ce jour.
\VS{30}C'est pourquoi dis à la maison d'Israël : Ainsi parle le Seigneur Yahweh : Ne vous souillez-vous pas selon les voies de vos pères, et ne vous prostituez-vous point à leurs idoles abominables,
\VS{31}en offrant vos dons, en faisant passer vos fils par le feu, en vous souillant par toutes vos idoles jusqu'à ce jour ? Est-ce ainsi que vous me consultez, ô maison d'Israël ? Je suis vivant, dit le Seigneur Yahweh, vous ne me consultez point.
\VS{32}Ce que vous pensez n'arrivera nullement, quand vous dites : Nous serons comme les nations, et comme les familles des pays, en servant le bois et la pierre.
\TextTitle{Restauration future d'Israël}
\VS{33}Je suis vivant ! dit le Seigneur Yahweh. Je règnerai sur vous avec une main forte, et un bras étendu, et avec effusion de colère.
\VS{34}Je vous sortirai du milieu des peuples, et vous rassemblerai hors des pays dans lesquels vous êtes dispersés, avec une main forte, et un bras étendu et avec effusion de colère.
\VS{35}Je vous ferai venir dans le désert des peuples, et je contesterai là contre vous, face à face,
\VS{36}comme j'ai contesté contre vos pères dans le désert du pays d'Egypte, ainsi je contesterai contre vous, dit le Seigneur Yahweh.
\VS{37}Je vous ferai passer sous la verge, et vous ramènerai au lieu de l'alliance\FTNT{Es. 65:12.}.
\VS{38}Je séparerai de vous les rebelles, et ceux qui se révoltent contre moi ; je les ferai sortir du pays dans lequel ils séjournent, mais ils n'entreront point dans la terre d'Israël ; et vous saurez que je suis Yahweh.
\VS{39}Vous donc, ô maison d'Israël, ainsi parle le Seigneur Yahweh : Allez, servez chacun vos idoles, puisque vous ne voulez pas m'écouter ! Ainsi vous ne profanerez plus mon saint Nom par vos dons et par vos idoles.
\VS{40}Mais ce sera sur ma sainte montagne, sur la haute montagne d'Israël, dit le Seigneur Yahweh, que toute la maison d'Israël me servira, dans le pays\FTNT{Jn. 4:21-24.}. Là, je prendrai plaisir en eux, et là je demanderai vos offrandes et les prémices de vos dons, et tout ce que vous me consacrerez.
\VS{41}Je prendrai plaisir en vous par vos parfums d'une agréable odeur, quand je vous aurai fait sortir du milieu des peuples, et que je vous aurai rassemblés des pays dans lesquels vous êtes dispersés ; je serai sanctifié par vous, aux yeux des nations.
\VS{42}Vous saurez que je suis Yahweh, quand je vous aurai fait revenir dans le pays d'Israël, dans le pays où j'ai levé ma main pour le donner à vos pères.
\VS{43}Et là, vous vous souviendrez de vos voies, et de toutes vos actions, par lesquelles vous vous êtes souillés ; et vous vous prendrez vous-mêmes en dégoût à cause de tous vos maux que vous aurez faits.
\VS{44}Vous saurez que je suis Yahweh, par tout ce que j'aurai fait pour vous, à cause de mon Nom, et non pas selon vos méchantes voies et vos actions corrompues, ô maison d'Israël ! dit le Seigneur Yahweh.
\Chap{21}
\TextTitle{L'épée de Yahweh}
\VerseOne{}La parole de Yahweh me fut encore adressée en ces mots :
\VS{2}Fils de l'homme, tourne ta face vers Jérusalem, parle en direction du sud, et prophétise contre la forêt du champ du sud.
\VS{3}Dis à la forêt du sud : Ecoute la parole de Yahweh. Ainsi parle le Seigneur Yahweh : Voici, je m'en vais allumer au dedans de toi un feu qui consumera tout bois vert et tout bois sec au dedans de toi ; la flamme de l'embrasement ne s'éteindra point, et tout le dessus en sera brûlé, depuis le sud jusqu'au nord\FTNT{Jé. 21:14 ; Jé. 22:7 ; Jé. 46:23 ; Lu. 23:31.}.
\VS{4}Toute chair verra que moi, Yahweh, j'ai allumé le feu ; et il ne s'éteindra point.
\VS{5}Je dis : Ah ! Seigneur Yahweh, ils disent de moi : N'est-il pas vrai que celui-ci ne fait que mettre en avant des similitudes ?
\TextTitle{Parabole de l'épée de Yahweh}
\VS{6}La parole de Yahweh me fut adressée en ces mots :
\VS{7}Fils de l'homme, tourne ta face vers Jérusalem, et parle en direction du lieu saint, et prophétise contre la terre d'Israël.
\VS{8}Dis à la terre d'Israël : Ainsi parle Yahweh : Voici, j'en veux à toi, je tirerai mon épée de son fourreau, et je retrancherai du milieu de toi le juste et le méchant.
\VS{9}Parce que je retrancherai du milieu de toi le juste et le méchant, à cause de cela mon épée sortira de son fourreau contre toute chair, depuis le sud jusqu'au nord.
\VS{10}Toute chair saura que moi, Yahweh, j'ai tiré mon épée de son fourreau, et elle n'y retournera plus.
\VS{11}Aussi, toi, fils de l'homme, gémis en te rompant les reins de douleur, et soupire avec amertume dans leur présence.
\VS{12}Quand ils te diront : Pourquoi gémis-tu ? Alors tu répondras : C'est à cause d'une nouvelle, car elle vient, et tout cœur se fondra, et toutes les mains seront baissées, tout esprit sera affaibli, et tous les genoux se fondront en eau ; voici, elle vient, elle arrive, dit le Seigneur Yahweh\FTNT{Jé. 6:24 ; Jé. 49:23.}.
\VS{13}Puis la parole de Yahweh me fut adressée en ces mots :
\VS{14}Fils de l'homme, prophétise, et dis : Ainsi parle Yahweh : Dis : L'épée ! L'épée a été aiguisée, elle est polie !
\VS{15}Elle a été aiguisée pour faire un grand carnage, elle a été polie afin qu'elle brille… Nous réjouirons-nous ? C'est la verge de mon fils, elle dédaigne tout bois.
\VS{16}Yahweh l'a donnée à polir, afin qu'on la tienne à la main ; l'épée a été aiguisée, et elle a été polie pour la mettre dans la main du destructeur.
\VS{17}Crie et hurle, fils de l'homme, car elle est contre mon peuple, elle est contre tous les princes d'Israël ; ils sont livrés à l'épée à cause de mon peuple. C'est pourquoi frappe sur ta cuisse !
\VS{18}Oui l'épreuve sera faite ; et que sera-ce si ce sceptre qui méprise tout est anéanti ? dit le Seigneur Yahweh.
\VS{19}Toi donc, fils de l'homme, prophétise, et frappe d'une main contre l'autre, que les coups de l'épée soient doublés, soient triplés, c'est l'épée du carnage, l'épée du grand carnage, l'épée qui doit les poursuivre.
\VS{20}J'ai mis à toutes leurs portes l'épée étincelante, afin que le cœur se fonde, et que les ruines soient multipliées. Ah ! Elle est faite pour briller et réservée pour tuer.
\VS{21}Joins-toi épée, frappe à la droite ! Avance-toi, frappe à la gauche, à tous côtés que tu rencontres !
\VS{22}Je frapperai aussi d'une main contre l'autre, je donnerai du repos à ma colère. Moi, Yahweh, j'ai parlé.
\VS{23}La parole de Yahweh me fut adressée en ces mots :
\VS{24}Toi, fils de l'homme, pose deux chemins où l'épée du roi de Babylone pourrait venir ; que les deux chemins sortent d'un même pays, et forme-les, forme-les de ta main à l'endroit où commence le chemin de la ville.
\VS{25}Tu poseras le chemin par lequel l'épée doit venir contre Rabbath des fils d'Ammon, et le chemin qui va en Judée, et à Jérusalem, ville forte.
\VS{26}Car le roi de Babylone se tient au carrefour, à l'entrée des deux chemins, pour consulter les devins ; il aiguise les flèches, il interroge les théraphim, il examine le foie.
\VS{27}Dans sa main droite est la divination contre Jérusalem, pour y dresser des béliers, pour publier le carnage, pour pousser des cris de guerre, pour ranger les béliers contre les portes, pour élever des terrasses et construire des remparts.
\VS{28}Mais ce sera pour eux, à leurs yeux, une divination vaine ; il y a de grands serments entre eux. Mais lui, il se souvient de leur iniquité, en sorte qu'ils seront pris.
\VS{29}C'est pourquoi ainsi parle le Seigneur Yahweh : Parce que vous avez fait revenir le souvenir de votre iniquité, lorsque vos crimes se sont découverts, au point de voir vos péchés dans toutes vos actions ; parce que vous avez fait qu'on se souvienne de vous, vous serez saisis par sa main.
\TextTitle{Quand l'iniquité arrive à son terme\FTNTT{Ap. 19:11-20:6.}}
\VS{30}Et toi, profane, méchant, prince d'Israël, dont le jour arrive au temps où l'iniquité est à son terme !
\VS{31}Ainsi parle le Seigneur Yahweh : Qu'on ôte cette tiare, et qu'on enlève cette couronne. Ce ne sera plus celle-ci ; j'élèverai ce qui est bas, et j'abaisserai ce qui est haut\FTNT{Job. 5:11 ; 1 Co. 1:27.}.
\VS{32}J'en ferai une ruine, une ruine, une ruine, et elle ne sera plus. Mais cela n'aura lieu qu'à la venue de celui à qui appartient le jugement et à qui je le lui donnerai.
\VS{33}Toi, fils de l'homme, prophétise, et dis : Ainsi parle le Seigneur Yahweh, au sujet des fils d'Ammon, et de leur opprobre : Dis donc, épée, épée dégainée, polie pour le massacre, pour dévorer avec son éclat !
\VS{34}Au milieu de tes visions vaines et de tes oracles menteurs, elle te fera tomber parmi les cadavres des méchants, dont le jour arrive au temps où l'iniquité est à son terme.
\VS{35}La remettrait-on dans son fourreau ? Je te jugerai sur le lieu où tu as été créé, au pays de ta naissance.
\VS{36}Je répandrai ma colère sur toi, j'allumerai sur toi le feu de ma fureur, et je te livrerai entre les mains d'hommes brutaux, qui ne travaillent qu'à détruire\FTNT{Jé. 25:11 ; Jé. 52:30.}.
\VS{37}Tu seras destiné au feu pour être dévoré ; ton sang sera au milieu de la terre : On ne se souviendra plus de toi, car c'est moi, Yahweh, qui parle.
\Chap{22}
\TextTitle{Les péchés d'Israël}
\VerseOne{}La parole de Yahweh me fut encore adressée en ces mots :
\VS{2}Et toi, fils de l'homme, ne jugeras-tu pas, ne jugeras-tu pas la ville sanguinaire, et ne lui donneras-tu pas à connaître toutes ses abominations\FTNT{Na. 3:1-4 ; Ha. 1:13 ; Ez. 24:6-9.} ?
\VS{3}Tu diras donc, ainsi parle le Seigneur Yahweh : Ville qui répands le sang au milieu de toi, afin que ton temps vienne, et qui as fait des idoles à ton préjudice, pour en être souillée.
\VS{4}Tu t'es rendue coupable par ton sang que tu as répandu, et tu t'es souillée par tes idoles que tu as faites ; tu as fait approcher tes jours, et tu es venue au terme de tes années ; c'est pourquoi je t'ai exposée en opprobre aux nations, et en dérision dans tous les pays\FTNT{2 R. 21:16 ; Jé. 26:21-23.}.
\VS{5}Celles qui sont près de toi, et celles qui en sont loin, se moqueront de toi, infâme de réputation, et remplie de troubles.
\VS{6}Voici, les princes d'Israël ont contribué au dedans de toi, chacun selon sa force, à répandre le sang.
\VS{7}Au dedans de toi, on méprise père et mère, on use de tromperie à l'égard de l'étranger, on opprime l'orphelin et la veuve.
\VS{8}Tu méprises ma sainteté, et profanes mes sabbats.
\VS{9}Des gens médisants sont au milieu de toi pour répandre le sang, ceux qui sont chez toi mangent sur les montagnes ; on commet des actions énormes au milieu de toi\FTNT{Es. 57:7 ; Jé. 2:20.}.
\VS{10}L'enfant découvre la nudité du père au milieu de toi, et on humilie au milieu de toi la femme dans le temps de son impureté\FTNT{Lé. 18:6-9 ; Ge. 9:22-23.}.
\VS{11}L'un commet l'abomination avec la femme de son prochain ; et l'autre se souille par l'inceste avec sa belle-fille ; chacun humilie sa sœur, fille de son père\FTNT{Ge. 19:32-36 ; Lé. 18:15-20 ; Jé. 5:8.}.
\VS{12}Chez toi, on reçoit des présents pour répandre le sang ; tu exiges un intérêt et une usure, tu dépouilles ton prochain par l'extorsion, et tu m'oublies, dit le Seigneur Yahweh\FTNT{Ex. 23:8 ; De. 27:25.}.
\VS{13}Voici, je frappe de mes mains l'une contre l'autre à cause de ton gain déshonnête que tu fais, et à cause de ton sang qui se répand au milieu de toi.
\VS{14}Ton cœur pourra-t-il tenir ferme, tes mains seront-elles fortes dans les jours où j'agirai contre toi ? Moi, Yahweh, j'ai parlé, et je le ferai.
\VS{15}Je te disperserai parmi les nations, je t'éparpillerai en divers pays, et je consumerai ta souillure, jusqu'à ce qu'il n'y en ait plus en toi.
\VS{16}Tu seras souillée par toi-même aux yeux des nations, et tu sauras que je suis Yahweh.
\TextTitle{La fureur de Yahweh}
\VS{17}Puis la parole de Yahweh me fut adressée en ces mots :
\VS{18}Fils de l'homme, la maison d'Israël m'est devenue comme de l'écume ; eux tous sont de l'airain, de l'étain, du fer et du plomb dans un creuset ; ils sont devenus comme une écume d'argent.
\VS{19}C'est pourquoi ainsi parle le Seigneur Yahweh : Parce que vous êtes tous devenus comme de l'écume, voici, je vais à cause de cela vous rassembler au milieu de Jérusalem,
\VS{20}comme on assemble de l'argent, de l'airain, du fer, du plomb, et de l'étain dans un creuset, afin d'y souffler le feu pour les fondre ; je vous rassemblerai ainsi dans ma colère et dans ma fureur, et je vous fondrai.
\VS{21}Je vous assemblerai, je soufflerai contre vous le feu de ma fureur, et vous serez fondus au milieu de Jérusalem.
\VS{22}Comme l'argent se fond dans le creuset, ainsi vous serez fondus au milieu d'elle, et vous saurez que moi,Yahweh, j'ai répandu ma fureur sur vous.
\TextTitle{Se tenir à la brèche devant Yahweh}
\VS{23}La parole de Yahweh me fut encore adressée en ces mots :
\VS{24}Fils de l'homme, dis-lui : Tu es une terre qui n'est pas purifiée ni arrosée de pluie au jour de la colère.
\VS{25}Il y a un complot de ses prophètes au milieu d'elle ; ils seront comme des lions rugissants, qui ravissent la proie : Ils dévorent les âmes, ils emportent les richesses et la gloire, ils multiplient les veuves au milieu d'elle\FTNT{Mt. 23:13 ; 1 Pi. 5:8.}.
\VS{26}Ses sacrificateurs ont fait violence à ma loi et ont profané mes choses saintes ; ils ne font pas de différence entre la chose sainte et profane; ils ne donnent pas à connaître la diffrence qu'il y a entre la chose impure et la pure, et ils cachent leurs yeux de mes sabbats, et je suis profané au milieu d'eux.
\VS{27}Ses princes sont au milieu d'elle comme des loups qui ravissent la proie, pour répandre le sang et pour détruire les âmes, pour s'adonner au gain déshonnête\FTNT{2 Pi. 2:16 ; Mt. 10:16 ; Mi. 3:11.}.
\VS{28}Ses prophètes ont pour eux des enduits de plâtre, des visions fausses, et des oracles menteurs, en disant : Ainsi parle le Seigneur Yahweh ; et cependant Yahweh n'a point parlé.
\VS{29}Le peuple du pays use de tromperies, ravit le bien d'autrui, opprime l'affligé et le pauvre, et foule l'étranger contre tout droit.
\VS{30}Je cherche parmi eux un homme\FTNT{Dieu n'a pas besoin d'une foule de gens avant d'agir. Une seule personne suffit. } qui élève un mur, et qui se tient à la brèche devant moi pour le pays, afin que je ne le détruise point ; mais je n'en trouve pas.
\VS{31}C'est pourquoi je répandrai sur eux ma colère, et je les consumerai par le feu de ma fureur ; je mettrai leur voie sur leur tête, dit le Seigneur Yahweh.
\Chap{23}
\TextTitle{Prostitutions d'Israël et de Juda}
\VerseOne{}La parole de Yahweh me fut encore adressée en ces mots :
\VS{2}Fils de l'homme, il y a eu deux femmes, filles d'une même mère,
\VS{3}qui se sont prostituées en Egypte, elles se sont prostituées dans leur jeunesse. Là, leur sein fut déshonoré et leur virginité touchée.
\VS{4}Et c'était ici leurs noms, celui de la plus grande était Ohola, et celui de sa sœur Oholiba\FTNT{Ohola signifie « sa propre tente », et Oholiba « la femme de la tente ». 2 R. 17:23-24.} ; elles étaient à moi, et elles ont enfanté des fils et des filles ; leurs noms donc étaient Ohola, qui était Samarie, et Oholiba, qui est Jérusalem.
\VS{5}Or Ohola a commis adultère étant ma femme, et s'est rendue amoureuse de ses amoureux, c'est-à-dire, des Assyriens ses voisins,
\VS{6}vêtus de pourpre, gouverneurs et magistrats, tous jeunes et aimables, tous cavaliers, montés sur des chevaux.
\VS{7}Elle a commis ses adultères avec toute l'élite des fils des Assyriens, et avec tous ceux pour qui elle s'était enflammée, et s'est souillée avec toutes leurs idoles.
\VS{8}Elle n'a pas abandonné ses fornications d'Egypte, car ils avaient couché avec elle dans sa jeunesse, ils avaient déshonoré sa virginité et s'étaient livrés à l'impureté avec elle\FTNT{Ac. 7:42.}.
\VS{9}C'est pourquoi je l'ai livrée entre les mains de ses amoureux, entre les mains des fils des Assyriens, dont elle s'était rendue amoureuse.
\VS{10}Ils l'ont couverte d'opprobre, ils ont enlevé ses fils et ses filles, et l'ont tuée elle-même avec l'épée ; elle a été en renom parmi les femmes, après avoir exercé des jugements sur elle.
\VS{11}Quand sa sœur Oholiba a vu cela, et fut plus déréglée qu'elle dans ses passions ; ses prostitutions dépassèrent celles de sa sœur.
\VS{12}Elle s'enflamma pour les fils des Assyriens, des gouverneurs et des magistrats, ses voisins, vêtus magnifiquement, et des cavaliers montés sur des chevaux, tous jeunes et bien faits.
\VS{13}J'ai vu qu'elle s'était souillée, et que l'une et l'autre avaient suivi la même voie.
\VS{14}Elle alla même plus loin dans ses prostitutions. Elle aperçut contre des murailles des peintures d'hommes, des images de Chaldéens peints en couleur rouge,
\VS{15}avec des ceintures autour des reins, avec des turbans de couleurs variées flottant sur la tête. Tous ayant l'apparence de princes, et figurant des fils de Babylone.
\VS{16}Elle s'enflamma pour eux au premier regard, et leur envoya des messagers en Chaldée.
\VS{17}Les fils de Babylone vinrent vers elle au lit de ses prostitutions, et la souillèrent par leurs adultères ; elle s'est aussi souillée avec eux, et après cela son cœur s'est détaché d'eux.
\VS{18}Elle a manifesté ses fornications et fait connaître son opprobre ; et mon cœur s'est détaché d'elle, comme mon cœur s'était détaché de sa sœur.
\VS{19}Car elle a multiplié ses adultères, jusqu'à rappeler le souvenir des jours de sa jeunesse, lorsqu'elle s'était abandonnée au pays d'Egypte.
\VS{20}Elle s'est enflammée pour des impudiques dont la chair était comme celle des ânes, et dont la force égale celle des chevaux.
\VS{21}Tu as donc repris les méchancetés de ta jeunesse, lorsque tu as été déshonorée, depuis que tu étais en Egypte, à cause du sein de ta jeunesse.
\VS{22}C'est pourquoi, Oholiba, ainsi parle le Seigneur Yahweh : Voici, je m'en vais réveiller contre toi tous tes amants, ceux dont ton cœur s'est détaché, et je les amènerai contre toi de toutes parts.
\VS{23}Les fils de Babylone, et tous les Chaldéens, Pekod, Shoa, Koa, et tous les Assyriens avec eux, tous jeunes gens d'élite, gouverneurs et magistrats, grands seigneurs et renommés, tous montant à cheval.
\VS{24}Ils viendront contre toi avec des armes, des chars, et des roues, avec une multitude de peuples, avec le grand bouclier et le petit bouclier, avec les casques, et je leur mettrai le jugement en main, ils te jugeront selon leur jugement.
\VS{25}Je mettrai ma jalousie contre toi, et ils agiront contre toi avec fureur ; ils te retrancheront le nez et les oreilles, et ce qui restera de toi tombera par l'épée. Ils enlèveront tes fils et tes filles, et ce qui restera de toi sera dévoré par le feu.
\VS{26}Ils te dépouilleront de tes vêtements, et t'enlèveront les ornements dont tu te pares.
\VS{27}Je mettrai un terme à tes méchancetés et tes prostitutions du pays d'Egypte ; tu ne lèveras plus tes yeux vers eux, et tu ne te souviendras plus de l'Egypte.
\VS{28}Car ainsi parle le Seigneur Yahweh : Voici, je te livre entre les mains de ceux que tu hais, entre les mains de ceux dont ton cœur s'est détaché.
\VS{29}Ils te traiteront avec haine ; ils enlèveront tout ton travail, et te laisseront sans habits et découverte ; et la turpitude de tes adultères, de ton énormité, et de tes fornications, sera découverte.
\VS{30}On te fera ces choses-là parce que tu t'es prostituée aux nations, avec lesquelles tu t'es souillée par leurs idoles.
\VS{31}Tu as marché dans la voie de ta sœur, c'est pourquoi je mets sa coupe dans ta main.
\VS{32}Ainsi parle le Seigneur Yahweh : Tu boiras la coupe profonde et large de ta sœur ; elle sera une coupe d'une grande mesure ; tu seras un sujet de risée et de moquerie\FTNT{Ps. 75:9 ; Es. 51:17 ; Jé. 25:15.}.
\VS{33}Tu seras remplie d'ivresse et de douleur, par la coupe de désolation et de dégât, qui est la coupe de ta sœur Samarie.
\VS{34}Tu la boiras et la videras, tu briseras ce pot de terre et tu déchireras ton sein. Car j'ai parlé, dit le Seigneur Yahweh.
\VS{35}C'est pourquoi ainsi parle le Seigneur Yahweh : Parce que tu m'as oublié, et que tu m'as jeté derrière ton dos, aussi porteras-tu la peine de ta méchanceté, et de tes prostitutions.
\TextTitle{Jugement sur Israël et Juda}
\VS{36}Puis Yahweh me dit : Fils de l'homme, ne jugeras-tu pas Ohola et Oholiba ? Déclare-leur donc leurs abominations.
\VS{37}Déclare-leur comment elles ont commis l'adultère et comment il y a du sang dans leurs mains ; comment, dis-je, elles ont commis l'adultère avec leurs idoles, et ont même fait passer par le feu leurs fils pour les consumer, ces enfants qu'elles m'avaient enfantés.
\VS{38}Voici encore ce qu'elles m'ont fait : Elles ont souillé mon lieu saint ce même jour, et ont profané mes sabbats.
\VS{39}Car après avoir égorgé leurs fils à leurs idoles, elles sont entrées ce même jour-là dans mon lieu saint pour le profaner ; et voilà, comment elles ont fait au milieu de ma maison\FTNT{2 R. 21:4.}.
\VS{40}Et qui plus est, elles ont fait chercher des hommes venant de loin, elles leur ont envoyé des messagers, et voici, ils sont venus. Pour eux tu t'es lavée, tu as fardé ton visage, et t'es parée d'ornements.
\VS{41}Tu t'es assise sur un lit magnifique, devant lequel a été apprêtée une table, sur laquelle tu as mis mon encens et mon huile.
\VS{42}On entendait le bruit d'une multitude tranquille ; et parmi cette foule d'hommes, on a fait venir du désert des Sabéens, qui ont mis des bracelets aux mains des deux sœurs, et de superbes couronnes sur leurs têtes.
\VS{43}J'ai dit au sujet de celle qui avait vieilli dans l'adultère : Maintenant ses impudicités prendront fin, et elle aussi.
\VS{44}Toutefois on est venu vers elle comme on vient vers une femme prostituée ; ils sont ainsi venus vers Ohola, et vers Oholiba, femmes pleines de méchanceté.
\VS{45}Les hommes justes donc les jugeront comme on juge les femmes adultères, et comme on juge celles qui répandent le sang ; car elles sont adultères, et le sang est dans leurs mains.
\VS{46}C'est pourquoi ainsi parle le Seigneur Yahweh : Qu'on fasse monter l'assemblée contre elles, et qu'elles soient abandonnées au tumulte et au pillage.
\VS{47}Que l'assemblée les lapide de pierres et les taille en pièces avec leurs épées ; qu'ils tuent leurs fils et leurs filles, et qu'ils brûlent au feu leurs maisons.
\VS{48}Et ainsi je ferai cesser la méchanceté dans le pays, et toutes les femmes seront enseignées à ne point faire selon votre méchanceté.
\VS{49}On mettra votre méchanceté sur vous, et vous porterez les péchés de vos idoles ; et vous saurez que je suis le Seigneur Yahweh.
\Chap{24}
\TextTitle{Malheur à la ville sanguinaire}
\VerseOne{}La neuvième année, au dixième jour du dixième mois, la parole de Yahweh me fut adressée en ces mots :
\VS{2}Fils de l'homme, mets par écrit la date de ce jour, de ce jour-ci ! Car en ce même jour le roi de Babylone s'approche contre Jérusalem\FTNT{2 R. 25:1.}.
\VS{3}Propose une parabole à la famille de rebelles, et dis-leur : Ainsi parle le Seigneur Yahweh : Mets, mets la chaudière, et verse de l'eau dedans.
\VS{4}Mets-y les morceaux, tous les bons morceaux, la cuisse, et l'épaule, et remplis-la des meilleurs os.
\VS{5}Prends la meilleure bête du troupeau, et fais brûler des os sous la chaudière, fais-la bouillir à gros bouillons, et que les os cuisent au-dedans.
\VS{6}Car ainsi parle le Seigneur Yahweh : Malheur à la ville sanguinaire, à la chaudière pleine de rouille, et de laquelle la rouille n'est point sortie ! Vide-la morceau par morceau, et que le sort ne soit point jeté sur elle.
\VS{7}Parce que son sang est au milieu d'elle, qu'elle l'a mis sur le rocher brillant, et qu'elle ne l'a point répandu sur la terre pour le couvrir de poussière,
\VS{8}j'ai mis son sang sur un rocher brillant, afin qu'il ne soit point couvert, pour faire monter la fureur, et pour me venger.
\VS{9}C'est pourquoi ainsi parle le Seigneur Yahweh : Malheur à la ville sanguinaire ! J'en ferai aussi un grand tas de bois à brûler !
\VS{10}Amasse beaucoup de bois, allume le feu, fais cuire la chair entièrement, et fais-la consumer, et que les os soient brûlés.
\VS{11}Puis mets sur les charbons ardents la chaudière toute vide, afin qu'elle s'échauffe et que son airain se brûle, et que sa souillure soit fondue à l'intérieur, et que sa rouille soit consumée.
\VS{12}Les efforts sont inutiles, sa rouille dont elle est pleine n'est point sortie d'elle ; sa rouille ne s'en ira que par le feu.
\VS{13}L'impureté est dans ta souillure ; car je t'avais purifiée, et tu n'as point été pure ; tu ne seras pas encore nettoyée de ta souillure, jusqu'à ce que j'aie assouvi sur toi ma fureur.
\VS{14}Moi, Yahweh, j'ai parlé, cela arrivera, et je le ferai ; et je ne me retirerai point en arrière, je n'épargnerai point, et je ne serai point apaisé. On t'a jugée selon tes voies et selon tes actions, dit le Seigneur Yahweh.
\TextTitle{La vie d'Ezéchiel, un signe pour Israël}
\VS{15}La parole de Yahweh me fut adressée en ces mots :
\VS{16}Fils de l'homme, voici, je vais t'ôter par une plaie ce que tes yeux voient avec le plus de plaisir. Ne mène point de deuil, ne pleure point, ne fais point couler tes larmes\FTNT{Jé. 16:6-7.}.
\VS{17}Garde-toi de gémir, et ne fais pas le deuil des morts ; attache ton turban sur ta tête, mets tes souliers à tes pieds, ne te couvre pas la barbe, et ne mange pas le pain des autres\FTNT{Lé. 10:6.}.
\VS{18}Je parlai au peuple le matin, et ma femme mourut le soir ; le lendemain matin je fis comme il m'avait été ordonné.
\VS{19}Le peuple me dit : Ne nous déclareras-tu point ce que nous signifient ces choses-là que tu fais ?
\VS{20}Je leur répondis : La parole de Yahweh m'a été adressée en ces mots :
\VS{21}Parle à la maison d'Israël : Ainsi parle le Seigneur Yahweh : Voici, je m'en vais profaner mon lieu saint, la magnificence de votre force, ce qui est le plus agréable à vos yeux, ce que vous voudriez épargner sur toutes choses ; et vos fils et vos filles, que vous aurez laissés, tomberont par l'épée.
\VS{22}Vous ferez alors comme j'ai fait ; vous ne couvrirez point vos barbes, et vous ne mangerez point le pain des autres.
\VS{23}Vos turbans seront sur vos têtes, et vos souliers à vos pieds ; vous ne mènerez point de deuil ni ne pleurerez ; mais vous pourrirez à cause de vos iniquités, et vous gémirez les uns avec les autres.
\VS{24}Ezéchiel sera pour vous un signe ; vous ferez selon toutes les choses qu'il a faites ; et quand cela sera arrivé, vous saurez que je suis le Seigneur Yahweh.
\VS{25}Quant à toi, fils de l'homme, au jour que je leur ôterai leur force, la joie de leur ornement, l'objet le plus agréable à leurs yeux, et l'objet de leurs cœurs, leurs fils et leurs filles,
\VS{26}ce jour-là un fuyard ne viendra-t-il pas vers toi pour te le raconter ?
\VS{27}En ce jour-là ta bouche sera ouverte envers celui qui sera échappé, et tu parleras, et ne seras plus muet ; ainsi tu seras pour eux un signe, et ils sauront que je suis Yahweh.
\Chap{25}
\TextTitle{Jugement de Dieu sur Ammon}
\VerseOne{}Puis la parole de Yahweh me fut adressée en ces mots :
\VS{2}Fils de l'homme, tourne ta face vers les fils d'Ammon, et prophétise contre eux\FTNT{Jé. 49:1.}.
\VS{3}Dis aux fils d'Ammon : Ecoutez la parole du Seigneur Yahweh : Parce que vous avez dit : Ah ! Ah ! contre mon lieu saint, parce qu'il était profané ; et contre la terre d'Israël, parce qu'elle était désolée ; et contre la maison de Juda, parce qu'ils allaient en captivité\FTNT{Am. 1:13 ; So. 2:8.};
\VS{4}à cause de cela, voici, je m'en vais te donner en héritage aux fils d'orient, et ils bâtiront des palais dans tes villes, et ils demeureront chez toi ; ils mangeront tes fruits et boiront ton lait.
\VS{5}Je livrerai Rabba pour être le repaire des chameaux, et le pays des fils d'Ammon pour être le gîte des brebis, et vous saurez que je suis Yahweh.
\VS{6}Car ainsi parle le Seigneur Yahweh : Parce que tu as frappé des mains, que tu as battu des pieds, et que tu t'es réjoui de bon cœur avec tout le mépris que tu as eu pour la terre d'Israël,
\VS{7}à cause de cela voici, j'ai étendu ma main sur toi, et je te livrerai pour être pillée par les nations, et je te retrancherai du milieu des peuples, je te ferai périr d'entre les pays ; je te détruirai ; et tu sauras que je suis Yahweh.
\TextTitle{Jugement sur Moab}
\VS{8}Ainsi parle le Seigneur Yahweh : Parce que Moab et Séir ont dit : Voici, la maison de Juda est comme toutes les autres nations ;
\VS{9}à cause de cela voici, j'ouvre le territoire de Moab du côté des villes, de ses villes frontières, la beauté du pays de Beth-Jeschimoth, de Baal-Meon et de Kirjathaïm\FTNT{Jos. 12:3 ; No. 32:38.},
\VS{10}je l'ouvre aux fils d'orient, qui sont au-delà du pays des fils d'Ammon, je leur donne en possession, afin qu'on ne se souvienne plus des fils d'Ammon parmi les nations.
\VS{11}J'exercerai aussi des jugements contre Moab, et ils sauront que je suis Yahweh.
\TextTitle{Jugement sur Edom}
\VS{12}Ainsi parle le Seigneur Yahweh : A cause de ce qu'Edom a fait quand il s'est vengé de la maison de Juda, et parce qu'il s'en est rendu coupable en se vengeant d'eux\FTNT{Ps. 137:7.},
\VS{13}à cause de cela, le Seigneur Yahweh dit : J'étendrai ma main sur Edom, j'en retrancherai les hommes et les bêtes, et j'en ferai un désert ; depuis Théman à Dedan ils tomberont par l'épée\FTNT{Jé. 49:7-9 ; Am. 1:12 ; Ab. 1:9.}.
\VS{14}J'exercerai ma vengeance sur Edom à cause de mon peuple d'Israël, et on traitera Edom selon ma colère, et selon ma fureur, et ils reconnaîtront ma vengeance, dit le Seigneur Yahweh.
\TextTitle{Jugement sur les Philistins}
\VS{15}Ainsi parle le Seigneur Yahweh : Puisque les Philistins ont agi par vengeance, et qu'ils se sont vengés avec mépris et du fond de leur âme, voulant tout détruire dans leur haine éternelle ;
\VS{16}à cause de cela le Seigneur Yahweh dit : Voici, je m'en vais étendre ma main sur les Philistins, j'exterminerai les Kéréthiens, et je ferai périr le reste sur le rivage de la mer.
\VS{17}J'exercerai sur eux de grandes vengeances par des châtiments de fureur ; et ils sauront que je suis Yahweh, quand j'aurai exécuté sur eux ma vengeance\FTNT{Es. 14:29 ; Jé. 25:20 ; So. 2:7.}.
\Chap{26}
\TextTitle{Jugement sur Tyr}
\VerseOne{}Il arriva dans la onzième année, le premier jour du mois, que la parole de Yahweh me fut adressée en ces mots :
\VS{2}Fils de l'homme, parce que Tyr a dit au sujet de Jérusalem : Ah ! Ah ! Celle qui était la porte des peuples a été rompue, elle s'est réfugiée chez moi, je serai remplie parce qu'elle a été rendue déserte\FTNT{Am. 1:9 ; Za. 9:2-3.} !
\VS{3}A cause de cela, ainsi parle le Seigneur Yahweh : Voici, j'en veux à toi, Tyr, et je ferai monter contre toi plusieurs nations, comme la mer fait monter ses flots\FTNT{Jé. 51:42.}.
\VS{4}Elles détruiront les murailles de Tyr, et démoliront ses tours ; j'en raclerai sa poussière, et la rendrai semblable à un rocher nu\FTNT{ Es. 23:15.}.
\VS{5}Elle servira à étendre les filets au milieu de la mer ; car j'ai parlé, dit le Seigneur Yahweh, et elle sera en pillage aux nations.
\VS{6}Ses filles sur sa terre seront tuées par l'épée, et elles sauront que je suis Yahweh.
\VS{7}Car ainsi parle le Seigneur Yahweh : Voici, je m'en vais faire venir du nord contre Tyr, Nebucadnetsar, roi de Babylone, le roi des rois, avec des chevaux, des chars, des cavaliers, et un grand peuple assemblé de toutes parts.
\VS{8}Il tuera par l'épée tes filles sur ta terre, il fera des remparts contre toi, il dressera des terrasses contre toi, et il lèvera les boucliers contre toi.
\VS{9}Il donnera des coups de béliers contre tes murs, et renversera tes tours avec ses épées.
\VS{10}La multitude de ses chevaux te couvrira de poussière, tes murs trembleront au bruit des cavaliers, des roues, et des chars, quand il entrera par tes portes, comme on entre dans une ville qu'on a divisée.
\VS{11}Il foulera toutes tes rues avec les sabots de ses chevaux, il tuera ton peuple avec l'épée, et les trophées de ta force tomberont par terre\FTNT{Jé. 47:3 ; Es. 5:8.}.
\VS{12}Puis ils retireront tes biens, et pilleront ta marchandise ; ils renverseront tes murs, et renverseront tes maisons de plaisance ; et ils mettront tes pierres, ton bois et ta poussière au milieu des eaux.
\VS{13}Je ferai cesser le bruit de tes chansons, et le son de tes harpes ne sera plus entendu.
\VS{14}Je te rendrai semblable à un rocher nu ; tu seras un lieu pour étendre les filets, et tu ne seras plus rebâtie, parce que moi,Yahweh, j'ai parlé, dit le Seigneur Yahweh.
\VS{15}Ainsi parle le Seigneur Yahweh, à Tyr : Les îles ne trembleront-elles pas du bruit de ta ruine, quand ceux qui seront blessés à mort gémiront, quand le carnage se fera au milieu de toi ?
\VS{16}Tous les princes de la mer descendront de leurs trônes, ôteront leurs manteaux, dépouilleront leurs vêtements brodés, et s'envelopperont de frayeur ; ils s'assiéront sur la terre, ils seront effrayés à chaque instant, et seront désolés à cause de toi.
\VS{17}Ils prononceront à haute voix une complainte sur toi, et te diront : Comment as-tu péri, toi qui étais fréquentée par ceux qui vont sur la mer, ville renommée, qui étais forte dans la mer, toi et tes habitants qui inspiraient la terreur à tous ceux qui habitent chez elle\FTNT{Es. 23:15-16 ; Ap. 18:9.}?
\VS{18}Maintenant les îles seront effrayées au jour de ta ruine, et les îles qui sont dans la mer seront terrifiées à cause de ta fuite.
\VS{19}Car ainsi parle le Seigneur Yahweh : Quand je ferai de toi une ville désolée, comme sont les villes qui ne sont point habitées, quand j'aurai fait tomber sur toi l'abîme, et que les grosses eaux t'auront couverte ;
\VS{20}alors je te ferai descendre avec ceux qui descendent dans la fosse, vers le peuple d'autrefois, et je te placerai aux lieux les plus bas de la terre, aux endroits désolés depuis longtemps, avec ceux qui descendent dans la fosse, afin que tu ne sois plus habitée, mais je donnerai la gloire pour la terre des vivants.
\VS{21}Je ferai qu'on sera épouvanté à cause de toi, de ce que tu n'es plus ; et quand on te cherchera, on ne te trouvera plus jamais, dit le Seigneur Yahweh.
\Chap{27}
\TextTitle{Lamentation sur Tyr\FTNTT{Cp. Ap. 18:1-24.}}
\VerseOne{}La parole de Yahweh me fut encore adressée en ces mots :
\VS{2}Toi donc, fils de l'homme, prononce à haute voix une complainte sur Tyr.
\VS{3}Tu diras à Tyr : Toi qui demeures au bord de la mer, qui trafiques avec les peuples dans plusieurs îles ; ainsi parle le Seigneur Yahweh : Tyr, tu disais : Je suis parfaite en beauté !
\VS{4}Ton territoire est au cœur de la mer, ceux qui t'ont bâtie t'ont rendue parfaite en beauté.
\VS{5}Ils t'ont bâti de tous les côtés des navires de sapins de Senir ; ils ont pris les cèdres du Liban pour te faire des mâts.
\VS{6}Ils ont fait tes rames de chênes de Basan, et la troupe des Assyriens a fait tes bancs d'ivoire, apporté des îles de Kittim.
\VS{7}Le fin lin d'Egypte, avec des broderies, te servait de voiles et de pavillon ; des étoffes teintes en bleu et en pourpre des îles d'Elischa formaient tes couvertures.
\VS{8}Les habitants de Sidon et d'Arvad étaient tes rameurs, ô Tyr ! Les plus sages du milieu de toi étaient tes pilotes.
\VS{9}Les anciens de Guebal et ses hommes experts furent parmi toi, réparant tes brèches ; tous les navires de la mer, et leurs mariniers étaient chez toi, pour faire l'échange de tes marchandises.
\VS{10}Ceux de Perse, de Lud, et de Puth servaient dans ton armée. C'étaient des hommes de guerre, ils suspendaient chez toi le bouclier et le casque ; ils t'ont rendue magnifique.
\VS{11}Les fils d'Arvad avec ton armée étaient autour de tes murs, et des hommes braves étaient dans tes tours ; ils ont suspendu leurs boucliers à tous tes murs, ils ont achevé de te rendre parfaite en beauté.
\VS{12}Ceux de Tarsis ont trafiqué avec toi de toutes sortes de richesses, d'argent, de fer, d'étain et de plomb.
\VS{13}Javan, Tubal, et Méschec trafiquaient avec toi ; ils donnaient des personnes et des ustensiles d'airain en échange de tes marchandises.
\VS{14}Ceux de la maison de Togarma pourvoyaient tes marchés de chevaux, de cavaliers, et de mulets.
\VS{15}Les fils de Dedan trafiquaient avec toi ; tu avais dans ta main le commerce de plusieurs îles ; et on t'a rendu en échange des dents d'ivoire et de l'ébène.
\VS{16}La Syrie trafiquait avec toi, en quantité d'ouvrages faits pour toi ; elle pourvoyait tes marchés d'escarboucles, d'écarlate, de broderie, de fin lin, de corail, et d'agate.
\VS{17}Juda et le pays d'Israël trafiquaient avec toi, faisant valoir ton commerce en blé de Minnith, en pâtisseries, en miel, en huile, et en baume.
\VS{18}Damas trafiquait avec toi en quantité d'ouvrages faits pour toi, en toutes sortes de richesses, en vin de Helbon, et en laine blanche.
\VS{19}Vedan, et Javan depuis Uzal, pourvoyaient tes marchés ; le fer luisant, la casse et le roseau aromatique furent dans ton commerce.
\VS{20}Dedan trafiquait avec toi en couvertures pour s'asseoir à cheval.
\VS{21}Les arabes, et tous les princes de Kédar, étaient des marchands dans ta main, trafiquant avec toi en agneaux, en moutons, et en boucs.
\VS{22}Les marchands de Séba et de Raema trafiquaient avec toi de tous les meilleurs aromates, de toute sorte de pierres précieuses et d'or.
\VS{23}Charan, Canné, et Eden, les marchands de Séba, d'Assyrie, de Kilmad, trafiquaient avec toi.
\VS{24}Ils trafiquaient avec toi toutes sortes de belles choses, des manteaux teints en bleu, en broderie, en riches étoffes contenues dans des coffres attachés avec des cordes, faits en bois de cèdre, et amenés sur tes marchés.
\VS{25}Les navires de Tarsis naviguaient pour ton commerce ; tu étais au comble de la force et de la richesse, au cœur des mers.
\VS{26}Tes rameurs t'ont amenée dans de grosses eaux, le vent d'orient t'a brisée au cœur de la mer.
\VS{27}Tes richesses, tes marchés et tes marchandises, tes mariniers et tes pilotes, ceux qui réparaient tes brèches, et ceux qui s'occupaient de ton commerce, tous tes hommes de guerre qui étaient chez toi, et toute ta multitude au milieu de toi, tomberont dans le cœur de la mer au jour de ta ruine\FTNT{Ap. 18:9.}.
\VS{28}Les faubourgs trembleront au bruit du cri de tes pilotes.
\VS{29}Tous ceux qui manient la rame descendront de leurs navires, les mariniers, et tous les pilotes de la mer ; ils se tiendront sur la terre ;
\VS{30}ils feront entendre leur voix, et crieront amèrement ; ils jetteront de la poussière sur leurs têtes, et se vautreront dans la cendre ;
\VS{31}ils arracheront leurs cheveux, et rendront leur tête chauve à cause de toi, ils se ceindront de sacs, et te pleureront avec l'amertume dans leur âme, en menant un deuil amer.
\VS{32}Ils prononceront à haute voix sur toi une complainte dans leur lamentation, et feront leur complainte sur toi, en disant : Qui fut jamais comme Tyr, comme cette ville détruite au cœur de la mer ?
\VS{33}Tu as rassasié plusieurs peuples par la traite des marchandises qu'on apportait de tes marchés au-delà des mers ; et tu as enrichi les rois de la terre par la multitude de tes richesses et de ton commerce.
\VS{34}Quand tu as été brisée par la mer au fond des eaux, ton commerce et toute ta multitude sont tombés avec toi.
\VS{35}Tous les habitants des îles sont désolés à cause de toi ; et leurs rois sont saisis d'épouvante, et leur visage pâlit.
\VS{36}Les marchands parmi les peuples t'insultent, tu es réduite au néant, tu ne seras plus à jamais !
\Chap{28}
\TextTitle{Yahweh réprime l'arrogance du roi de Tyr}
\VerseOne{}La parole de Yahweh me fut encore adressée en ces mots :
\VS{2}Fils de l'homme, dis au prince de Tyr : Ainsi parle le Seigneur Yahweh : Parce que ton cœur s'est élevé et que tu as dit : Je suis Dieu, je suis assis sur le siège de Dieu, au cœur de la mer, quoique tu sois un homme, et non Dieu, et parce que tu as élevé ton cœur comme si tu étais un Dieu.
\VS{3}Voici, tu es plus sage que Daniel, rien de caché ne t'a été rendu obscur.
\VS{4}Tu t'es acquis de la puissance par ta sagesse et par ton intelligence ; et tu as amassé de l'or et de l'argent dans tes trésors\FTNT{Za. 9:2-3.}.
\VS{5}Tu as multiplié ta puissance par la grandeur de ta sagesse dans ton commerce, puis ton cœur s'est élevé à cause de ta puissance.
\VS{6}C'est pourquoi ainsi parle le Seigneur Yahweh : Parce que tu as élevé ton cœur, comme si tu étais un Dieu,
\VS{7}à cause de cela voici, je m'en vais faire venir contre toi des étrangers, les plus terribles parmi les nations, qui tireront leurs épées sur la beauté de ta sagesse, et souilleront ta splendeur.
\VS{8}Ils te feront descendre dans la fosse, et tu mourras comme ceux qui tombent percés de coups, au milieu de la mer.
\VS{9}En face de ton meurtrier diras-tu : Je suis Dieu ? Tu seras homme et non Dieu sous la main de celui qui te tuera.
\VS{10}Tu mourras de la mort des incirconcis par la main des étrangers ; car j'ai parlé, dit le Seigneur Yahweh.
\TextTitle{Chute du roi de Tyr représentant satan\FTNTT{Cp. Es. 14:12-17.}}
\VS{11}La parole de Yahweh me fut encore adressée en ces mots :
\VS{12}Fils de l'homme, prononce à haute voix une complainte sur le roi de Tyr, et dis-lui : Ainsi parle le Seigneur Yahweh : Toi à qui rien ne manquait, plein de sagesse, et parfait en beauté ;
\VS{13}tu étais en Eden, le jardin de Dieu ; ta couverture était de pierres précieuses de toutes sortes, de sardoine, de topaze, de diamant, de chrysolithe, d'onyx, de jaspe, de saphir, d'escarboucle, d'émeraude, et d'or ; tes tambourins et tes flûtes étaient à ton service ; préparés pour le jour où tu fus créé.
\VS{14}Tu étais un chérubin, oint pour servir de protection ; je t'avais établi, et tu étais sur la sainte montagne de Dieu ; tu marchais entre les pierres éclatantes.
\VS{15}Tu étais parfait dans tes voies dès le jour où tu fus créé, jusqu'à celui où l'injustice fut trouvée en toi.
\VS{16}Selon la grandeur de ton trafic\FTNT{Satan est le premier commerçant. Il avait transformé ses sanctuaires célestes en un lieu de trafic, en un marché. Il avait reçu gratuitement du Seigneur, notre Dieu, plusieurs dons : la beauté, des pierres précieuses, des instruments de musique, la sagesse, un sanctuaire. Au lieu de les utiliser pour la gloire de Dieu, il en fit un trafic pour son propre profit égoïste. Il est le père de tous ceux qui vendent les dons de Dieu pour s'enrichir, de tous ceux qui font du commerce avec l'Evangile. Or Jésus nous a donné cet ordre formel : « Vous avez reçu gratuitement, donnez gratuitement » (Mt. 10 : 8). De même que le temple de Dieu était devenu une caverne de voleurs, plusieurs pasteurs ont transformé les bâtiments de leurs églises en véritables boutiques pour vendre toutes sortes de produits dérivés qui ne servent pas à l'avancement du Royaume de Dieu mais à enrichir des dirigeants cupides esclaves du dieu Mammon (Jn. 2:13-17 ; Mt. 6:24 ; Lu. 16:13 ; 1 Ti. 6:10 ; Hé. 13:5).}, tu as été rempli de violence, et tu as péché ; c'est pourquoi je te jette comme une chose souillée hors de la montagne de Dieu\FTNT{Ap. 12:1-12.}, et je te détruis d'entre les pierres éclatantes, ô chérubin protecteur !
\VS{17}Ton cœur s'est élevé à cause de ta beauté, tu as corrompu ta sagesse à cause de ton éclat ; je te jette par terre, je te donne en spectacle aux rois, afin qu'ils te regardent.
\VS{18}Tu as profané tes sanctuaires par la multitude de tes iniquités, par l'injustice de ton commerce ; et je fais sortir du milieu de toi un feu qui te consume, je te réduis en cendres sur la terre, dans la présence de tous ceux qui te regardent.
\VS{19}Tous ceux qui te connaissent parmi les peuples sont désolés à cause de toi ; tu es réduit à néant, tu ne seras plus à jamais.
\TextTitle{Jugement sur Sidon}
\VS{20}Puis la parole de Yahweh me fut adressée en ces mots :
\VS{21}Fils de l'homme, tourne ta face vers Sidon, et prophétise contre elle.
\VS{22}Tu diras : Ainsi parle le Seigneur Yahweh : Voici j'en veux à toi, Sidon ! Je serai glorifié au milieu de toi ; et on saura que je suis Yahweh, quand j'aurai exercé des jugements contre elle et que je serai sanctifié.
\VS{23}J'enverrai la peste dans son sein, je ferai couler le sang dans ses rues. Les morts tomberont au milieu d'elle par l'épée qui viendra de toutes parts sur elle ; et ils sauront que je suis Yahweh.
\VS{24}Elle ne sera plus pour la maison d'Israël une épine qui blesse, une ronce déchirante, parmi tous ceux qui l'entourent et qui la méprisent. Et ils sauront que je suis le Seigneur Yahweh.
\TextTitle{Rétablissement d'Israël}
\VS{25}Ainsi parle le Seigneur Yahweh : Quand j'aurai rassemblé la maison d'Israël d'entre les peuples parmi lesquels ils auront été dispersés, je manifesterai en elle ma sainteté, aux yeux des nations, et ils habiteront sur leur terre que j'ai donnée à mon serviteur Jacob.
\VS{26}Ils y habiteront en sûreté, ils bâtiront des maisons, ils planteront des vignes ; ils y habiteront, dis-je, en sûreté, lorsque j'aurai exercé des jugements contre ceux qui les auront pillés de toutes parts ; et ils sauront que je suis Yahweh leur Dieu.
\Chap{29}
\TextTitle{Jugement sur l'Egypte}
\VerseOne{}La dixième année, au douzième jour du dixième mois, la parole de Yahweh me fut adressée en ces mots :
\VS{2}Fils de l'homme, tourne ta face contre Pharaon, roi d'Egypte, prophétise contre lui, et contre toute l'Egypte\FTNT{Jé. 43:8-11.}.
\VS{3}Parle, et dis : Ainsi parle le Seigneur Yahweh : Voici, j'en veux à toi, Pharaon, roi d'Egypte, grand serpent couché au milieu de tes fleuves, qui dis : Mes fleuves sont à moi, et je me les suis faits\FTNT{Ps. 74:13-14 ; Es. 27:1.} !
\VS{4}C'est pourquoi je mettrai des crocs dans ta mâchoire, j'attacherai à tes écailles les poissons de tes fleuves ; je te tirerai hors de tes fleuves, avec tous les poissons de tes fleuves, qui seront attachés à tes écailles.
\VS{5}Et t'ayant tiré dans le désert, je te laisserai là, toi, et tous les poissons de tes fleuves ; tu tomberas sur la face des champs, tu ne seras point recueilli ni ramassé ; je te livrerai aux bêtes de la terre, et aux oiseaux des cieux, pour en être dévoré.
\VS{6}Et tous les habitants d'Egypte sauront que je suis Yahweh ; parce qu'ils ont été un soutien de roseau pour la maison d'Israël\FTNT{2 R. 18:21 ; Es. 36:6.}.
\VS{7}Quand ils t'ont pris par la main, tu t'es rompu, et tu leur as percé toute l'épaule ; et quand ils se sont appuyés sur toi, tu t'es cassé, et tu les as fait tomber à la renverse.
\VS{8}C'est pourquoi ainsi parle le Seigneur Yahweh : Voici, je m'en vais faire venir l'épée sur toi, et j'exterminerai du milieu de toi les hommes et les bêtes.
\VS{9}Le pays d'Egypte sera dans la désolation et dans le désert, et ils sauront que je suis Yahweh, parce que le roi d'Egypte a dit : Les fleuves sont à moi, et je les ai faits !
\VS{10}C'est pourquoi voici, j'en veux à toi, et à tes fleuves, et je réduirai le pays d'Egypte en désert de sécheresse et de désolation, depuis Migdol jusqu'à Syène, et aux frontières de l'Ethiopie.
\VS{11}Nul pied d'homme ne passera par là, et il n'y passera non plus aucun pied d'animal, elle sera quarante ans sans être habitée.
\VS{12}Car je réduirai le pays d'Egypte en désolation entre les pays désolés, et ses villes entre les villes réduites en désert ; elles seront en désolation durant quarante ans, je disperserai les Egyptiens parmi les nations, et je les répandrai parmi les pays.
\VS{13}Toutefois, ainsi parle le Seigneur Yahweh : Au bout de quarante ans, je ramasserai les Egyptiens d'entre les peuples parmi lesquels ils auront été dispersés ;
\VS{14}je ramènerai les captifs d'Egypte, et les ferai retourner au pays de Pathros, au pays de leur origine, mais ils seront là un royaume rabaissé.
\VS{15}Il sera le plus bas des royaumes, et il ne s'élèvera plus au-dessus des nations, je le diminuerai, afin qu'il ne domine point sur les nations.
\VS{16}Ce royaume ne sera plus pour la main d'Israël un sujet de confiance ; il lui rappellera son iniquité, quand elle se tournait vers eux ; et ils sauront que je suis le Seigneur Yahweh.
\VS{17}Il arriva la vingt-septième année, au premier jour du premier mois, que la parole de Yahweh me fut adressée en ces mots :
\VS{18}Fils de l'homme, Nebucadnetsar, roi de Babylone, a fait servir son armée dans un service pénible contre Tyr ; toute tête en est devenue chauve, et toute épaule en a été foulée, mais il n'a point eu de salaire, ni lui ni son armée, à cause de Tyr, pour le service qu'il a fait contre elle.
\VS{19}C'est pourquoi ainsi parle le Seigneur Yahweh : Voici, je m'en vais donner à Nebucadnetsar, roi de Babylone, le pays d'Egypte ; il enlèvera la multitude, il emportera le butin et fera le pillage ; ce sera là le salaire de son armée.
\VS{20}Pour prix du service qu'il a fait contre Tyr, je lui ai donné le pays d'Egypte, parce qu'ils ont travaillé pour moi, dit le Seigneur Yahweh.
\VS{21}En ce jour-là, je ferai germer la corne de la maison d'Israël, et j'ouvrirai ta bouche au milieu d'eux, et ils sauront que je suis Yahweh.
\Chap{30}
\TextTitle{Disgrâce de l'Egypte}
\VerseOne{}La parole de Yahweh me fut encore adressée en ces mots :
\VS{2}Fils de l'homme, prophétise, et dis : Ainsi parle le Seigneur Yahweh : Hurlez, et dites : Malheureux jour !
\VS{3}Car le jour est proche, oui le jour de Yahweh est proche, c'est un jour ténébreux ; ce sera le temps des nations.
\VS{4}L'épée viendra sur l'Egypte, et il y aura de l'effroi en Ethiopie, quand ceux qui seront blessés à mort tomberont dans l'Egypte, et quand on enlèvera la multitude de son peuple, et que ses fondements seront ruinés.
\VS{5}L'Ethiopie, Puth, Lud, toute l'Arabie, Cub, et les fils du pays allié tomberont par l'épée avec eux\FTNT{Jé. 46:9 ; Na. 3:9-10.}.
\VS{6}Ainsi parle Yahweh : Ceux qui soutiendront l'Egypte, tomberont ; et l'orgueil de sa force sera renversé ; ils tomberont par l'épée de Migdol à Syène, dit le Seigneur Yahweh.
\VS{7}Ils seront désolés au milieu des pays désolés, et ses villes seront au milieu des villes désertes.
\VS{8}Ils sauront que je suis Yahweh, quand j'aurai mis le feu en Egypte ; et tous ceux qui lui donneront du secours, seront brisés.
\VS{9}En ce jour-là, des messagers sortiront de ma part sur des navires pour effrayer l'Ethiopie dans sa sécurité, et il y aura entre eux un tourment au jour de l'Egypte ; car voici, il vient.
\VS{10}Ainsi parle le Seigneur Yahweh : Je ferai périr la multitude d'Egypte par la puissance de Nebucadnetsar, roi de Babylone.
\VS{11}Lui et son peuple avec lui, les plus terribles d'entre les nations, seront amenés pour ruiner le pays, et ils tireront leurs épées contre les Egyptiens, et rempliront la terre de morts.
\VS{12}Je mettrai à sec les fleuves et je livrerai le pays entre les mains des méchants ; je désolerai le pays, et tout ce qui y est, par la puissance des étrangers ; moi, Yahweh, j'ai parlé.
\VS{13}Ainsi parle le Seigneur Yahweh : Je détruirai aussi les idoles, j'anéantirai les faux dieux de Noph, et il n'y aura point de prince qui soit du pays d'Egypte ; je mettrai la frayeur dans le pays d'Egypte\FTNT{Es. 19:1-13 ; Jé. 43:12 ; Jé. 46:13.}.
\VS{14}Je désolerai Pathros, je mettrai le feu à Tsoan, et j'exercerai mes jugements sur No\FTNT{Jé. 44:1.}.
\VS{15}Je répandrai ma fureur sur Sin, qui est la place forte de l'Egypte, et j'exterminerai la multitude qui est à No.
\VS{16}Quand je mettrai le feu en Egypte, Sin sera grièvement tourmentée, et No sera rompue par diverses brèches, et il n'y aura à Noph que détresses en plein jour.
\VS{17}Les jeunes hommes d'On et de Pi-Béseth tomberont par l'épée, et ces villes iront en captivité.
\VS{18}Le jour s'obscurcira à Tachpanès, lorsque j'y romprai le joug de l'Egypte, et que l'orgueil de sa force aura cessé ; un nuage la couvrira, et les villes de son ressort iront en captivité.
\VS{19}J'exercerai des jugements en Egypte ; et ils sauront que je suis Yahweh.
\TextTitle{Chute et dispersion de l'Egypte}
\VS{20}Dans la onzième année, au septième jour du premier mois, la parole de Yahweh me fût adressée en ces mots :
\VS{21}Fils de l'homme, j'ai rompu le bras de Pharaon, roi d'Egypte ; et voici on ne l'a point bandé pour le guérir, on ne lui a point mis de linges pour le bander, et pour le fortifier, afin qu'il puisse manier l'épée.
\VS{22}C'est pourquoi ainsi parle le Seigneur Yahweh : Voici, j'en veux à Pharaon, roi d'Egypte, et je romprai ses bras, tant celui qui est fort que celui qui est rompu, et je ferai tomber l'épée de sa main.
\VS{23}Je disperserai les Egyptiens parmi les nations, et les répandrai parmi les pays.
\VS{24}Je fortifierai les bras du roi de Babylone, je lui mettrai mon épée dans la main ; mais je romprai les bras de Pharaon, et il gémira devant lui comme gémissent les mourants.
\VS{25}Je fortifierai donc les bras du roi de Babylone, mais les bras de Pharaon tomberont ; et on saura que je suis Yahweh, quand j'aurai mis mon épée dans la main du roi de Babylone, et qu'il l'a tournera contre le pays d'Egypte.
\VS{26}Je disperserai les Egyptiens parmi les nations, les répandrai parmi les pays ; et ils sauront que je suis Yahweh.
\Chap{31}
\TextTitle{Avertissement contre l'arrogance de Pharaon}
\VerseOne{}Il arriva aussi dans la onzième année, au premier jour du troisième mois, que la parole de Yahweh me fut adressée en ces mots :
\VS{2}Fils de l'homme, parle à Pharaon, roi d'Egypte, et à la multitude de son peuple : A qui ressembles-tu dans ta grandeur ?
\VS{3}Voici, le roi d'Assyrie a été comme un cèdre du Liban, ayant de belles branches, et des rameaux qui faisaient une grande ombre, et qui étaient d'une grande hauteur ; sa cime était fort touffue.
\VS{4}Les eaux l'ont fait croître, l'abîme l'a fait pousser en hauteur, ses fleuves ont coulé autour de ses plantes, et il a envoyé ses eaux abondantes vers tous les arbres des champs.
\VS{5}C'est pourquoi il s'est élevé au-dessus de tous les autres arbres des champs, ses branches se sont multipliées, et ses rameaux croissaient par les grandes eaux qui faisaient pousser ses branches.
\VS{6}Tous les oiseaux des cieux ont fait leurs nids dans ses branches, toutes les bêtes des champs ont fait leurs petits sous ses rameaux, et toutes les grandes nations ont habité sous son ombre.
\VS{7}Il était beau par sa grandeur, et par l'étendue de ses branches, parce que sa racine était sur de grandes eaux.
\VS{8}Les cèdres du jardin de Dieu ne le surpassaient point ; les cyprès n'égalaient point ses branches, et les platanes n'égalaient point comme ses rameaux ; aucun arbre du jardin de Dieu ne lui était comparable en beauté.
\VS{9}Je l'avais embelli par la multitude de ses rameaux, au point que tous les arbres d'Eden, qui étaient dans le jardin de Dieu, lui portaient envie.
\VS{10}C'est pourquoi le Seigneur Yahweh dit : Parce qu'il s'est élevé, parce qu'il lançait sa cime au milieu d'épais rameaux et que son cœur était fier de sa hauteur,
\VS{11}je l'ai livré entre les mains du plus fort des nations, qui l'a traité comme il fallait, et je l'ai chassé à cause de sa méchanceté.
\VS{12}Les étrangers les plus terrifiants parmi les nations l'ont coupé et l'ont laissé là, ses branches sont tombées sur les montagnes et sur toutes les vallées ; ses rameaux se sont rompus dans tous les ravins de la terre, et tous les peuples de la terre se sont retirés de dessous son ombre, et l'ont laissé là.
\VS{13}Tous les oiseaux des cieux se sont tenus sur ses ruines, et toutes les bêtes des champs se sont retirées vers ses rameaux,
\VS{14}afin que tous les arbres près des eaux n'élèvent plus leur hauteur, et qu'ils ne lancent plus leur cime au milieu d'épais rameaux, afin que tous les chênes arrosés d'eau ne gardent plus leur hauteur ; car tous sont livrés à la mort, aux profondeurs de la terre, parmi les fils des hommes, avec ceux qui descendent dans la fosse.
\VS{15}Ainsi parle le Seigneur Yahweh : Le jour qu'il descendit dans le scheol, j'ai répandu deuil sur lui, j'ai couvert l'abîme devant lui, j'ai empêché ses fleuves de couler, et les grosses eaux ont été retenues ; j'ai fait que le Liban soit en deuil à cause de lui, et tous les arbres des champs ont été desséchés.
\VS{16}J'ai ébranlé les nations par le bruit de sa ruine, quand je l'ai fait descendre dans le scheol, avec ceux qui descendent dans la fosse\FTNT{Es. 14:9.} ; et tous les arbres d'Eden, les plus beaux et les plus agréables du Liban, tous arrosés par les eaux, ont été consolés dans les profondeurs de la terre.
\VS{17}Eux aussi sont descendus avec lui dans le scheol, vers ceux qui ont péri par l'épée ; ils étaient son bras et ils habitaient sous son ombre parmi les nations.
\VS{18}A qui ressembles-tu ainsi en gloire et en grandeur parmi les arbres d'Eden ? Tu seras précipité avec les arbres d'Eden dans les profondeurs de la terre, tu seras gisant au milieu des incirconcis, avec ceux qui ont péri par l'épée. Voilà Pharaon et toute sa multitude ! dit le Seigneur Yahweh.
\Chap{32}
\TextTitle{Lamentation sur le pays d'Egypte}
\VerseOne{}Dans la douzième année, le premier jour du douzième mois, la parole de Yahweh me fut adressée en ces mots :
\VS{2}Fils de l'homme, prononce à haute voix une complainte sur Pharaon, roi d'Egypte, et dis-lui : Tu as été parmi les nations semblable à un lionceau, et comme un serpent dans les mers ; tu t'élançais dans tes fleuves, et tu troublais les eaux avec tes pieds, et remplissais de bourbe leurs fleuves.
\VS{3}Ainsi parle le Seigneur Yahweh : J'étendrai mon rets sur toi dans une assemblée nombreuse de peuples qui te tireront dans mes filets.
\VS{4}Je te laisserai à l'abandon sur la terre ; je te jetterai sur le dessus des champs, et je ferai demeurer sur toi tous les oiseaux des cieux, et rassasierai de toi les bêtes de toute la terre.
\VS{5}Car je mettrai ta chair sur les montagnes, et je remplirai les vallées de tes débris.
\VS{6}J'arroserai de ton sang jusqu'aux montagnes, la terre où tu nages, et les lits des eaux seront remplis de toi.
\VS{7}Quand je t'éteindrai, je couvrirai les cieux et j'obscurcirai leurs étoiles, je couvrirai le soleil de nuages, et la lune ne donnera plus sa lumière\FTNT{Es. 13:10 ; Joë. 2:31 ; Mt. 24:29.}.
\VS{8}J'obscurcirai à cause de toi tous les luminaires des cieux, et je répandrai les ténèbres sur ton pays, dit le Seigneur Yahweh.
\VS{9}J'affligerai le cœur de beaucoup de peuples, quand j'annoncerai ta ruine parmi les nations, à des pays que tu ne connaissais pas.
\VS{10}Je frapperai de stupeur beaucoup de peuples à cause de toi, et leurs rois seront saisis d'épouvante à cause de toi, quand je ferai luire mon épée à leurs yeux ; ils seront effrayés à chaque instant, chacun pour sa vie, au jour de ta ruine.
\VS{11}Car ainsi parle le Seigneur Yahweh : L'épée du roi de Babylone viendra sur toi.
\VS{12}J'abattrai ta multitude par les épées des hommes forts, qui tous sont les plus terribles d'entre les nations ; ils détruiront l'orgueil de l'Egypte, et toute la multitude de son peuple sera ruinée.
\VS{13}Je ferai périr tout son bétail près des grandes eaux, et aucun pied d'homme ne les troublera plus, ni aucun pied d'animaux ne les agitera plus.
\VS{14}Alors je rendrai profondes leurs eaux, et je ferai couler leurs fleuves comme de l'huile, dit le Seigneur Yahweh.
\VS{15}Quand j'aurai réduit le pays d'Egypte en désolation, et que le pays sera dénué des choses dont il était rempli ; quand je frapperai tous ceux qui y habitent, ils sauront alors que je suis Yahweh.
\VS{16}C'est ici la complainte qu'on fera sur elle, les filles des nations feront cette complainte sur elle ; elles feront cette complainte sur l'Egypte et sur toute la multitude de son peuple, dit le Seigneur.
\VS{17}Il arriva aussi dans la douzième année, le quinzième jour du mois, que la parole de Yahweh me fut adressée en ces mots :
\VS{18}Fils de l'homme, dresse une lamentation sur la multitude d'Egypte, et fais-la descendre, elle et les filles des nations magnifiques, aux plus bas lieux de la terre, avec ceux qui descendent dans la fosse\FTNT{Jé. 1:10 ; Jé. 18:7.}.
\VS{19}Qui surpasses-tu en beauté ? Descends, et couche-toi avec les incirconcis !
\VS{20}Ils tomberont au milieu de ceux qui seront tués par l'épée. L'épée a déjà été donnée : Entraînez l'Egypte et toute sa multitude !
\VS{21}Les plus forts d'entre les puissants lui parleront du milieu du scheol, avec ceux qui lui donnaient du secours, et diront : Ils sont descendus, ils sont couchés, les incirconcis, tués par l'épée.
\VS{22}Là est l'Assyrien, et toute son assemblée ; ses sépulcres sont autour de lui, eux tous, mis à mort, sont tombés par l'épée.
\VS{23}Car ses sépulcres sont posés au fond de la fosse et son assemblée autour de sa sépulture ; eux tous qui avaient répandu leur terreur sur la terre des vivants sont tombés morts par l'épée.
\VS{24}Là est Elam et toute sa multitude autour de son sépulcre ; eux tous sont tombés morts par l'épée, ils sont descendus incirconcis dans les plus bas lieux de la terre ; et après avoir répandu leur terreur sur la terre des vivants, ils ont porté leur ignominie avec ceux qui descendent dans la fosse.
\VS{25}On a mis sa couche parmi ceux qui ont été tués, avec toute sa multitude ; ses sépulcres sont autour de lui ; eux tous incirconcis, tués par l'épée, quoiqu'ils aient répandu leur terreur sur la terre des vivants, toutefois ils ont porté leur ignominie avec ceux qui descendent dans la fosse ; ils ont été placés parmi les morts.
\VS{26}Là est Méschec, Tubal, et toute leur multitude ; leurs sépulcres sont autour d'eux ; eux tous incirconcis, tués par l'épée, quoiqu'ils aient répandu leur terreur sur la terre des vivants.
\VS{27}Ils ne se sont point couchés avec les hommes vaillants qui sont tombés d'entre les incirconcis, lesquels sont descendus dans le scheol avec leurs armes de guerre, dont on a mis les épées sous leurs têtes, et dont les iniquités ont reposé sur leurs os ; parce que la terreur des hommes forts est dans la terre des vivants.
\VS{28}Toi aussi tu seras brisé au milieu des incirconcis, et tu seras couché avec ceux qui sont tués par l'épée.
\VS{29}Là est Edom, ses rois, et tous ses princes, qui ont été placés malgré leur force avec ceux qui sont tués par l'épée ; ils seront couchés avec les incirconcis, et avec ceux qui sont descendus dans la fosse.
\VS{30}Là sont tous les princes du nord, et tous les Sidoniens, qui sont descendus avec ceux qui sont tués, malgré la terreur qu'inspirait leur force ; ils sont couchés incirconcis avec ceux qui sont tués par l'épée ; ils ont porté leur ignominie avec ceux qui sont descendus dans la fosse.
\VS{31}Pharaon les verra, et il se consolera au sujet de toute la multitude de son peuple ; Pharaon, dit le Seigneur Yahweh, verra les blessés par l'épée et toute son armée.
\VS{32}Car je mettrai ma terreur dans la terre des vivants, c'est pourquoi Pharaon avec toute la multitude de son peuple se couchera au milieu des incirconcis, avec ceux qui sont tués par l'épée, dit le Seigneur Yahweh.
\Chap{33}
\TextTitle{Ezéchiel établi comme sentinelle pour avertir le pécheur}
\VerseOne{}La parole de Yahweh me fut encore adressée en ces mots :
\VS{2}Fils de l'homme, parle aux fils de ton peuple, et dis-leur : Quand je ferai venir l'épée sur un pays, et que le peuple du pays aura choisi quelqu'un d'entre eux, et l'aura établi pour leur servir de sentinelle,
\VS{3}et que voyant venir l'épée sur le pays, il sonnera du shofar et avertira le peuple,
\VS{4}si le peuple ayant bien entendu le son du shofar, ne se tient pas sur ses gardes, et qu'ensuite l'épée vienne le prendre, son sang sera sur sa tête.
\VS{5}Car il a entendu le son du shofar, et ne s'est point tenu sur ses gardes ; son sang sera sur lui ; mais s'il se tient sur ses gardes, il sauvera sa vie.
\VS{6}Si la sentinelle voit venir l'épée, et qu'elle ne sonne point du shofar, en sorte que le peuple ne se tienne point sur ses gardes, et qu'ensuite l'épée survienne et ôte la vie à l'un d'entre eux, celui-ci sera emmené en captivité à cause de son iniquité, mais je redemanderai son sang de la main de la sentinelle.
\VS{7}Toi donc, fils de l'homme, je t'ai établi pour sentinelle sur la maison d'Israël ; tu écouteras donc la parole qui sort de ma bouche, et tu les avertiras de ma part.
\VS{8}Quand j'aurai dit au méchant : Méchant, tu mourras ! et que tu n'auras point parlé au méchant pour l'avertir de se détourner de sa voie, ce méchant mourra dans son iniquité ; mais je redemanderai son sang de ta main.
\VS{9}Mais si tu as averti le méchant de se détourner de sa voie, et qu'il ne se détourne pas de sa voie, il mourra dans son iniquité ; mais toi tu auras délivré ton âme.
\VS{10}Toi donc, fils de l'homme, dis à la maison d'Israël : Vous avez parlé ainsi, en disant : Puisque nos crimes et nos péchés sont sur nous, et que nous périssons à cause d'eux, comment pourrions-nous vivre\FTNT{Lé. 26:39.} ?
\VS{11}Dis-leur : Je suis vivant, dit le Seigneur Yahweh, je ne prends point plaisir dans la mort du méchant, mais que le méchant se détourne de sa voie et qu'il vive. Détournez-vous, détournez-vous de votre méchante voie ! Pourquoi mourriez-vous, maison d'Israël ?
\VS{12}Toi donc, fils de l'homme, dis aux fils de ton peuple : La justice du juste ne le délivrera point au jour de son péché, le méchant ne tombera point par sa méchanceté au jour où il s'en détournera ; et le juste ne pourra pas vivre par sa justice au jour de son péché.
\VS{13}Quand j'aurai dit au juste qu'il vivra certainement, et que lui, se confiant sur sa justice, aura commis l'iniquité, on ne se souviendra plus d'aucune de ses justices, mais il mourra dans son iniquité qu'il aura commise.
\VS{14}Aussi quand j'aurai dit au méchant : Tu mourras ! s'il se détourne de son péché, et qu'il fasse ce qui est juste et droit ;
\VS{15}si le méchant rend le gage et qu'il restitue ce qu'il aura ravi, et qu'il marche dans les statuts de la vie, sans commettre d'iniquité, certainement il vivra, il ne mourra point.
\VS{16}On ne se souviendra plus des péchés qu'il aura commis ; il a fait ce qui est juste et droit ; certainement il vivra.
\VS{17}Or les fils de ton peuple ont dit : La voie du Seigneur n'est pas bien réglée ; mais c'est plutôt leur voie qui n'est pas bien réglée.
\VS{18}Quand le juste se détournera de sa justice, et qu'il commettra l'iniquité, il mourra à cause de cela.
\VS{19}Quand le méchant se détournera de sa méchanceté, et qu'il fera ce qui est juste et droit, il vivra à cause de cela.
\VS{20}Vous avez dit : La voie du Seigneur n'est pas bien réglée ! Je vous jugerai, maison d'Israël, chacun selon sa voie.
\TextTitle{Exécution du jugement de Yahweh}
\VS{21}Or il arriva dans la douzième année de notre captivité, au cinquième jour du dixième mois, qu'un homme qui s'était échappé de Jérusalem vint vers moi, en disant : La ville est prise !
\VS{22}La main de Yahweh fut sur moi le soir, avant l'arrivée du fugitif, et Yahweh ouvrit ma bouche lorsqu'il vint auprès de moi le matin. Ma bouche était ouverte et je n'étais plus muet.
\TextTitle{Ne pas se contenter d'écouter la Parole de Dieu}
\VS{23}La parole de Yahweh me fut adressée en ces mots :
\VS{24}Fils de l'homme, ceux qui habitent dans ces ruines, sur la terre d'Israël, discourent en disant : Abraham était seul, et il a possédé le pays\FTNT{Ge. 15:7.}; mais nous sommes un grand nombre de gens, et le pays nous a été donné en héritage.
\VS{25}C'est pourquoi tu leur diras : Ainsi parle le Seigneur Yahweh : Vous mangez la chair avec le sang, et vous levez vos yeux vers vos idoles, vous répandez le sang ; et vous posséderiez le pays\FTNT{Ge. 9:4 ; Lé. 3:17 ; Lé. 17:10.} ?
\VS{26}Vous vous appuyez sur votre épée, vous commettez des abominations, vous souillez chacun de vous la femme de son prochain ; et vous posséderiez le pays ?
\VS{27}Tu leur diras : Ainsi parle le Seigneur Yahweh : Je suis vivant, ceux qui sont dans ces ruines tomberont par l'épée, et je livrerai aux bêtes celui qui est dans les champs, afin qu'elles le mangent ; et ceux qui sont dans les forteresses et dans les cavernes mourront par la peste.
\VS{28}Ainsi je réduirai le pays en désolation et en désert, l'orgueil de sa force sera aboli, et les montagnes d'Israël seront désolées, en sorte qu'il n'y passera plus personne.
\VS{29}Ils sauront que je suis Yahweh, quand j'aurai réduit leur pays en désolation et en désert, à cause de toutes leurs abominations qu'ils ont commises.
\VS{30}Quant à toi, fils de l'homme, les fils de ton peuple parlent de toi près des murs et aux entrées des maisons, et parlent l'un à l'autre, chacun avec son prochain, en disant : Venez maintenant, et écoutez la parole qui vient de Yahweh.
\VS{31}Ils viennent vers toi en foule, et mon peuple s'assied devant toi, ils écoutent tes paroles, mais ils ne les mettent point en pratique ; ils les répètent comme si c'était une chanson profane, mais leur cœur marche toujours après leur gain déshonnête.
\VS{32}Voici tu es pour eux comme un homme qui leur chante une chanson profane avec une belle voix, qui résonne bien ; car ils écoutent bien tes paroles, mais ils ne les mettent point en pratique.
\VS{33}Mais quand ces choses arriveront, et voici, elles arrivent, ils sauront qu'il y avait un prophète au milieu d'eux.
\Chap{34}
\TextTitle{Jugement de Dieu sur les faux bergers}
\VerseOne{}La parole de Yahweh me fut encore adressée en ces mots :
\VS{2}Fils de l'homme, prophétise contre les pasteurs d'Israël\FTNT{Les faux pasteurs prennent en otage les brebis du Seigneur (Jé. 23). La véritable fonction pastorale consiste au service envers les frères et sœurs et non le contraire. Un vrai pasteur sert les autres, il n'aime pas être servi comme un roi. Il ne dit pas aux autres de faire les choses, mais il les fait et les autres l'imitent (Jn. 10).} ! Prophétise, et dis à ces pasteurs : Ainsi parle le Seigneur Yahweh : Malheur aux pasteurs d'Israël ! Qui ne paissent qu'eux-mêmes ! Les pasteurs ne paissent-ils pas le troupeau ?
\VS{3}Vous en mangez la graisse, et vous vous habillez de laine ; vous tuez ce qui est gras, vous ne paissez point le troupeau !
\VS{4}Vous n'avez point fortifié les brebis languissantes, vous n'avez point donné de remède à celle qui était malade, vous n'avez point bandé la plaie de celle qui avait la jambe rompue, vous n'avez point ramené celle qui était chassée, et vous n'avez point cherché celle qui était perdue\FTNT{Lu. 15:4-6 ; 1 Pi. 5:1-3.} ; mais vous les avez maîtrisées avec dureté et rigueur.
\VS{5}Elles se sont dispersées, parce qu'elles n'avaient pas de pasteurs, et elles se sont exposées à toutes les bêtes des champs, pour en être dévorées, étant dispersées.
\VS{6}Mes brebis sont errantes sur toutes les montagnes, et sur toutes les collines élevées ; mes brebis sont dispersées sur toute la surface de la terre ; et il n'y a personne qui les recherche, et il n'y a personne pour s'en soucier\FTNT{Mc. 14:27 ; Za. 13:7 ; Mt. 26:31.}.
\VS{7}C'est pourquoi pasteurs, écoutez la parole de Yahweh :
\VS{8}Je suis vivant, dit le Seigneur Yahweh, parce que mes brebis sont pillées, et que mes brebis sont la nourriture de toutes les bêtes des champs, parce qu'elles n'ont point de pasteur ; car mes pasteurs n'ont point recherché mes brebis, mais les pasteurs se sont nourris simplement eux-mêmes, et n'ont point fait paître mes brebis.
\VS{9}C'est pourquoi pasteurs, écoutez la parole de Yahweh !
\VS{10}Ainsi parle le Seigneur Yahweh : Voici, j'en veux à ces pasteurs-là, et je redemanderai mes brebis de leur main ; ils cesseront de paître les brebis, et les pasteurs ne se repaîtront plus eux-mêmes, mais je délivrerai mes brebis de leur bouche, et elles ne seront plus dévorées par eux.
\TextTitle{Yahweh, le bon berger qui restaure son troupeau\FTNT{Jn. 10:1-18}}
\VS{11}Car ainsi parle le Seigneur Yahweh : Me voici, je redemanderai mes brebis, et je les rechercherai.
\VS{12}Comme le pasteur prend soin de son troupeau quand il est au milieu de ses brebis dispersées, ainsi je rechercherai mes brebis, et les retirerai de tous les lieux où elles auront été dispersées au jour des nuages et de l'obscurité.
\VS{13}Je les retirerai d'entre les peuples et les rassemblerai des territoires, les ramènerai dans leur terre, et les nourrirai sur les montagnes d'Israël, auprès des cours d'eau et dans toutes les demeures du pays.
\VS{14}Je les paîtrai dans de bons pâturages, et leur demeure sera sur les hautes montagnes d'Israël ; et là elles coucheront dans une agréable demeure, et paîtront dans de gras pâturages, sur les montagnes d'Israël.
\VS{15}Moi-même je paîtrai mes brebis et les ferai reposer, dit le Seigneur Yahweh\FTNT{Ps. 23.}.
\VS{16}Je chercherai celle qui était perdue, et je ramènerai celle qui était chassée, je banderai la plaie de celle qui a la jambe rompue, et je fortifierai celle qui est malade ; mais je détruirai la grasse et la forte ; je les paîtrai avec justice.
\VS{17}Quant à vous, mes brebis, ainsi parle le Seigneur Yahweh : Voici, je m'en vais mettre à part les brebis, les béliers, et les boucs.
\VS{18}Et vous, est-ce peu de chose de vous faire paître dans de bons pâturages, pour que vous fouliez de vos pieds le reste de votre pâture ? Et de boire des eaux claires, pour que vous troubliez le reste avec vos pieds ?
\VS{19}Mais mes brebis sont nourries du pâturage que vous foulez de vos pieds, et boivent ce que vos pieds ont troublé.
\VS{20}C'est pourquoi le Seigneur Yahweh leur dit : Me voici, je mettrai moi-même à part la brebis grasse et la brebis maigre.
\VS{21}Parce que vous poussez du côté et de l'épaule, et que vous heurtez de vos cornes toutes celles qui sont languissantes, jusqu'à ce que vous les ayez chassées dehors,
\VS{22}je sauverai mes brebis, au point qu'elles ne seront plus au pillage. Voici, je jugerai entre brebis et brebis.
\VS{23}Je susciterai sur elles un pasteur qui les paîtra, mon serviteur David ; il les paîtra, et lui-même sera leur pasteur.
\VS{24}Moi, Yahweh, je serai leur Dieu, et mon serviteur David sera prince au milieu d'elles ; moi,Yahweh, j'ai parlé.
\VS{25}Je traiterai avec elles une alliance de paix ; et je détruirai dans le pays les mauvaises bêtes ; les brebis habiteront dans le désert en sécurité, et dormiront dans les forêts.
\VS{26}Je les comblerai de bénédictions, elles, et tous les environs de mes collines ; je ferai tomber la pluie en sa saison ; ce seront des pluies de bénédiction.
\VS{27}Les arbres des champs produiront leur fruit, et la terre rapportera son revenu ; elles seront dans leur terre en sécurité, et sauront que je suis Yahweh, quand j'aurai rompu les bois de leur joug, et que je les aurai délivrées de la main de ceux qui se les asservissent.
\VS{28}Elles ne seront plus au pillage parmi les nations, et les bêtes de la terre ne les dévoreront plus ; mais elles habiteront en sécurité, et il n'y aura personne pour les effrayer.
\VS{29}Je leur susciterai une plantation de renom ; elles ne mourront plus de faim sur la terre, et ne porteront plus l'opprobre des nations.
\VS{30}Ils sauront que moi, Yahweh, leur Dieu, suis avec eux, et qu'eux, la maison d'Israël, sont mon peuple, dit le Seigneur Yahweh.
\VS{31}Or vous êtes mes brebis, vous hommes, les brebis de mon pâturage, et je suis votre Dieu, dit le Seigneur Yahweh.
\Chap{35}
\TextTitle{Jugement sur Edom}
\VerseOne{}La parole de Yahweh me fut encore adressée en ces mots :
\VS{2}Fils de l'homme, tourne ta face contre la montagne de Séir, et prophétise contre elle\FTNT{Am. 1:11.}.
\VS{3}Dis-lui : Ainsi parle le Seigneur Yahweh : Voici, j'en veux à toi, montagne de Séir, et j'étendrai ma main contre toi, et te réduirai en désolation et en désert.
\VS{4}Je réduirai tes villes en désert, tu ne seras que désolation, et tu sauras que je suis Yahweh.
\VS{5}Parce que tu as eu une inimitié immortelle, et que tu as fait couler le sang des fils d'Israël à coups d'épée, au temps de leur détresse, au temps où l'iniquité était à son terme\FTNT{Ps. 137:7.}.
\VS{6}C'est pourquoi je suis vivant, dit le Seigneur Yahweh, je te mettrai à sang, et le sang te poursuivra ; parce que tu n'as point haï le sang, le sang aussi te poursuivra.
\VS{7}Je réduirai la montagne de Séir en désolation et en désert, et j'en éloignerai tous ceux qui la fréquentaient.
\VS{8}Je remplirai de morts ses montagnes ; tes hommes tués par l'épée tomberont sur tes collines, dans tes vallées, et dans tous tes courants d'eau.
\VS{9}Je te réduirai en désolations éternelles, et tes villes ne seront plus habitées ; vous saurez que je suis Yahweh.
\VS{10}Parce que tu as dit : Les deux nations, et les deux pays seront à moi, nous les posséderons, quand même Yahweh était là ;
\VS{11}à cause de cela, je suis vivant, dit le Seigneur Yahweh, j'agirai avec la colère et la jalousie que tu as montrées dans ta haine contre eux ; et je me ferai connaître au milieu d'eux, quand je te jugerai.
\VS{12}Tu sauras que moi, Yahweh, j'ai entendu toutes les paroles insultantes que tu as prononcées contre les montagnes d'Israël, en disant : Elles sont dévastées, elles nous sont livrées comme une proie.
\VS{13}Vous m'avez bravé par vos discours, et vous avez multiplié vos paroles contre moi ; je l'ai entendu.
\VS{14}Ainsi parle le Seigneur Yahweh : Quand toute la terre se réjouira, je te réduirai en désolation.
\VS{15}Comme tu t'es réjouie sur l'héritage de la maison d'Israël et de sa désolation, j'en ferai de même envers toi ; tu ne seras que désolation, ô montagne de Séir ! Ainsi qu'Edom tout entier ; et ils sauront que je suis Yahweh\FTNT{Ab. 1:11-16.}.
\Chap{36}
\TextTitle{Yahweh rétablit Israël}
\VerseOne{}Toi, fils de l'homme, prophétise sur les montagnes d'Israël, et dis : Montagnes d'Israël, écoutez la parole de Yahweh !
\VS{2}Ainsi parle le Seigneur Yahweh : Parce que l'ennemi a dit contre vous : Ah ! Ah ! Tous ces hauts lieux éternels sont devenus notre possession !
\VS{3}Prophétise, et dis : Ainsi parle le Seigneur Yahweh : Oui, parce qu'on vous a réduites en désolation, et que de toutes parts, on vous a englouties pour que vous soyez la propriété des autres nations, et qu'on vous a exposées à la langue et aux insultes des nations,
\VS{4}à cause de cela, montagnes d'Israël, écoutez la parole du Seigneur Yahweh : Ainsi parle le Seigneur Yahweh, aux montagnes, aux collines, aux courants d'eau, aux vallées, aux lieux détruits et désolés, et aux villes abandonnées qui sont pillées et sont un sujet de moquerie aux autres nations d'alentour ;
\VS{5}à cause de cela, ainsi parle le Seigneur Yahweh : Je parle dans le feu de ma jalousie contre les autres nations, et contre tous ceux d'Edom qui se sont attribués ma terre en possession, avec toute la joie de leur cœur et le mépris de leur âme, afin d'en piller le butin\FTNT{Lé. 25:23 ; Es. 14:2 ; Jé. 2:7.}.
\VS{6}C'est pourquoi prophétise sur la terre d'Israël, et dis aux montagnes et aux collines, aux courants d'eau et aux vallées : Ainsi parle le Seigneur Yahweh : Voici, j'ai parlé avec jalousie, et avec fureur, parce que vous avez porté l'ignominie des nations.
\VS{7}C'est pourquoi ainsi parle le Seigneur Yahweh : J'ai levé ma main, si les nations qui sont tout autour de vous ne portent leur ignominie.
\VS{8}Mais vous, montagnes d'Israël, vous pousserez vos branches, et vous porterez votre fruit pour mon peuple d'Israël ; car ils sont prêts à venir.
\VS{9}Car me voici, je viens à vous, et je retournerai vers vous, et vous serez labourées et semées.
\VS{10}Je mettrai sur vous des hommes en grand nombre, la maison d'Israël tout entière, et les villes seront habitées, les lieux déserts seront rebâtis.
\VS{11}Je multiplierai sur vous les hommes et les animaux, ils multiplieront et seront féconds ; je veux que vous soyez habitées comme auparavant, et je vous ferai plus de bien que vous n'en avez eu au commencement ; et vous saurez que je suis Yahweh.
\VS{12}Je ferai marcher sur vous des hommes, mon peuple d'Israël, qui vous posséderont, vous serez leur héritage, et vous ne les consumerez plus.
\VS{13}Ainsi parle le Seigneur Yahweh : Parce qu'on dit de vous : Tu es un pays qui dévore les hommes, et tu as consumé tes habitants ;
\VS{14}à cause de cela, tu ne dévoreras plus les hommes et ne consumeras plus tes habitants, dit le Seigneur Yahweh.
\VS{15}Je ne te ferai plus entendre l'ignominie des nations, tu ne porteras plus l'opprobre des peuples ; et tu ne feras plus périr tes habitants, dit le Seigneur Yahweh.
\VS{16}Puis la parole de Yahweh me fut adressée en ces mots :
\VS{17}Fils de l'homme, ceux de la maison d'Israël habitant sur leur terre l'ont souillée par leur voie et par leurs actions ; leur voie est devenue devant moi comme la souillure d'une femme pendant son impureté\FTNT{Lé. 12:2 ; Lé. 15:19.} ;
\VS{18}j'ai répandu ma fureur sur eux à cause du sang qu'ils ont répandu sur le pays, et parce qu'ils l'ont souillé par leurs idoles.
\VS{19}Je les ai dispersés parmi les nations, et ils ont été disséminés en divers pays ; je les ai jugés selon leur voie, et selon leurs actions.
\VS{20}Ils sont arrivés chez les nations où ils allaient, ils ont profané mon saint Nom en sorte qu'on disait d'eux : Ceux-ci sont le peuple de Yahweh, c'est de son pays qu'ils sont sortis\FTNT{Ro. 2:24.}.
\VS{21}Mais j'ai épargné mon saint Nom, que la maison d'Israël avait profané parmi les nations où elle est allée.
\VS{22}C'est pourquoi dis à la maison d'Israël : Ainsi parle le Seigneur Yahweh : Je ne le fais point à cause de vous, ô maison d'Israël ! Mais à cause de mon saint Nom, que vous avez profané parmi les nations où vous êtes allés\FTNT{De. 7:7 ; De. 9:5 ; Ps. 25:11 ; Es. 43:25.}.
\VS{23}Je sanctifierai mon grand nom, qui a été profané parmi les nations, et que vous avez profané au milieu d'elles ; et les nations sauront que je suis Yahweh, dit le Seigneur Yahweh, quand je serai sanctifié par vous, sous leurs yeux.
\VS{24}Je vous retirerai d'entre les nations, je vous rassemblerai de tous les pays, et je vous ramènerai dans votre terre.
\VS{25}Je répandrai sur vous une eau pure\FTNT{Il est question ici de la Nouvelle Alliance (Jé. 31:31-34 ; Hé. 8:7-13).}, et vous serez nettoyés ; je vous nettoierai de toutes vos souillures et de toutes vos idoles.
\TextTitle{Prophétie sur la naissance d'en haut}
\VS{26}Je vous donnerai un nouveau cœur, je mettrai au dedans de vous un Esprit nouveau ; j'ôterai de votre chair le cœur de pierre, et je vous donnerai un cœur de chair\FTNT{Jé. 32:39 ; 2 Co. 3:3 ; Ez. 11:19.}.
\VS{27}Je mettrai mon Esprit au dedans de vous, je ferai en sorte que vous suiviez mes ordonnances, et que vous observiez et pratiquiez mes lois.
\VS{28}Vous habiterez le pays que j'ai donné à vos pères, vous serez mon peuple, et je serai votre Dieu.
\VS{29}Je vous délivrerai de toutes vos souillures, j'appellerai le blé, je le multiplierai, et je ne vous enverrai plus la famine.
\VS{30}Je multiplierai le fruit des arbres et le revenu des champs, afin que vous ne portiez plus l'opprobre de la famine parmi les nations.
\VS{31}Vous vous souviendrez de votre mauvaise voie et de vos actions, qui n'étaient pas bonnes, et vous prendrez vous-mêmes en dégoût vos iniquités et vos abominations.
\VS{32}Je ne le fais point par amour pour vous, dit le Seigneur Yahweh ; sachez-le ! Soyez honteux et confus à cause de votre voie, ô maison d'Israël !
\VS{33}Ainsi parle le Seigneur Yahweh : Le jour où je vous aurai purifiés de toutes vos iniquités, je vous ferai habiter dans des villes, et les lieux déserts seront rebâtis.
\VS{34}La terre désolée sera cultivée, tandis qu'elle n'était que désolation aux yeux de tous les passants.
\VS{35}On dira : Cette terre-ci qui était désolée est devenue comme le jardin d'Eden ; et ces villes qui étaient désertes, désolées, et détruites, sont fortifiées et habitées\FTNT{Jé. 22:8-9 ; Es. 33:20.}.
\VS{36}Les nations qui resteront autour de vous sauront que moi, Yahweh, j'ai rebâti les lieux détruits et planté le pays désolé ; moi, Yahweh, j'ai parlé, et je le ferai.
\VS{37}Ainsi parle le Seigneur Yahweh : Je me laisserai rechercher par la maison d'Israël. Voici ce que je ferai pour eux : Je multiplierai les hommes comme un troupeau de brebis.
\VS{38}Les villes qui sont désertes seront remplies de troupeaux d'hommes, pareils aux troupeaux consacrés, aux troupeaux qu'on amène à Jérusalem pendant ses fêtes solennelles ; et ils sauront que je suis Yahweh.
\Chap{37}
\TextTitle{Vision des ossements desséchés, image de la restauration d'Israël}
\VerseOne{}La main de Yahweh fut sur moi, et Yahweh me transporta par son Esprit et me déposa au milieu d'une vallée remplie d'ossements\FTNT{Les ossements desséchés représentent les Israélites dispersés dans les nations.}.
\VS{2}Il me fit passer auprès d'eux, tout autour ; et voici, ils étaient fort nombreux à la surface de cette vallée et complètement secs.
\VS{3}Puis il me dit : Fils de l'homme, ces os pourront-ils revivre ? Et je répondis : Seigneur Yahweh, tu le sais.
\VS{4}Alors il me dit : Prophétise sur ces os, et dis-leur : Ossements desséchés, écoutez la parole de Yahweh !
\VS{5}Ainsi parle le Seigneur Yahweh à ces os : Voici, je ferai entrer un esprit en vous, et vous vivrez\FTNT{Ro. 8:11 ; Ps. 71:20.};
\VS{6}je mettrai des nerfs sur vous, je ferai croître de la chair sur vous, et j'étendrai la peau sur vous ; puis je mettrai un esprit en vous, et vous vivrez. Et vous saurez que je suis Yahweh.
\VS{7}Alors je prophétisai selon l'ordre que j'avais reçu. Et comme je prophétisais, il se fit un bruit, et voici, il se fit un mouvement, et ces os s'approchèrent les uns des autres.
\VS{8}Puis je regardai, et voici, il vint des nerfs sur eux, et il y crût de la chair, la peau fut étendue par dessus ; mais il n'y avait pas en eux d'esprit.
\VS{9}Alors il me dit : Prophétise à l'Esprit ! Prophétise, fils de l'homme ! Et dis à l'Esprit : Ainsi parle le Seigneur Yahweh : Esprit, viens des quatre vents, et souffle sur ces morts, et qu'ils revivent !
\VS{10}Je prophétisai donc selon l'ordre qu'il m'avait donné. Et l'Esprit entra en eux, ils reprirent vie, et se tinrent sur leurs pieds ; c'était une armée extrêmement grande.
\VS{11}Alors il me dit : Fils de l'homme, ces os sont toute la maison d'Israël ; voici, ils disent : Nos os sont desséchés, et notre attente est perdue, c'en est fait de nous !
\VS{12}C'est pourquoi prophétise, et dis-leur : Ainsi parle le Seigneur Yahweh : Mon peuple, voici, je m'en vais ouvrir vos sépulcres, je vous tirerai hors de vos sépulcres, et vous ferai entrer dans la terre d'Israël\FTNT{Les sépulcres représentent les nations dans lesquelles les Israélites se sont établis. Dieu annonce le retour de son peuple sur la terre d'Israël. Es. 26:19 ; Os. 13:14.}.
\VS{13}Et vous, mon peuple, vous saurez que je suis Yahweh quand j'aurai ouvert vos sépulcres, et que je vous aurai tirés hors de vos sépulcres.
\VS{14}Je mettrai mon Esprit en vous, et vous vivrez, je vous rétablirai sur votre terre ; et vous saurez que moi, Yahweh, j'ai parlé et que je l'ai fait, dit Yahweh.
\TextTitle{Prophétie sur l'unité d'Israël}
\VS{15}Puis la parole de Yahweh me fut adressée en ces mots :
\VS{16}Et toi, fils de l'homme, prends un bois et écris dessus : Pour Juda, et pour les fils d'Israël ses compagnons. Prends encore un autre bois, et écris dessus : Le bois d'Ephraïm et de toute la maison d'Israël, ses compagnons, pour Joseph.
\VS{17}Puis tu les joindras l'un à l'autre pour ne former qu'un même bois, ils seront unis dans ta main.
\VS{18}Quand les fils de ton peuple demanderont, en disant : Ne nous déclareras-tu pas ce que tu veux dire par ces choses ?
\VS{19}Dis-leur : Ainsi parle le Seigneur Yahweh : Voici, je m'en vais prendre le bois de Joseph qui est dans la main d'Ephraïm, et des tribus d'Israël, ses compagnons ; je les joindrai au bois de Juda, et j'en formerai un seul bois, ils ne seront qu'un seul bois dans ma main.
\VS{20}Ainsi les bois sur lesquels tu écriras seront dans ta main, sous leurs yeux.
\VS{21}Dis-leur : Ainsi parle le Seigneur Yahweh : Voici, je m'en vais prendre les fils d'Israël d'entre les nations parmi lesquelles ils sont allés, je les rassemblerai de toutes parts, et je les ferai entrer dans leur terre.
\VS{22}Je ferai d'eux une seule nation dans le pays, sur les montagnes d'Israël ; un seul roi sera leur roi à tous, ils ne seront plus deux nations, et ils ne seront plus divisés en deux royaumes\FTNT{Os. 2:2 ; Es. 11:12-13 ; Jn. 10:16.}.
\VS{23}Ils ne se souilleront plus par leurs idoles, ni par leurs infamies, ni par tous leurs crimes, et je les retirerai de toutes leurs demeures dans lesquelles ils ont péché, et je les purifierai ; ils seront mon peuple, et je serai leur Dieu\FTNT{Es. 1:18 ; Jé. 33:8 ; Jé. 24:7 ; Jé. 32:38 ; Za. 8:8 ; 2 Co. 6:16.}.
\VS{24}David, mon serviteur, sera leur roi, et ils auront tous un seul pasteur ; ils suivront mes ordonnances, ils garderont mes lois et les mettront en pratique.
\VS{25}Ils habiteront dans le pays que j'ai donné à Jacob, mon serviteur, dans lequel vos pères ont habité ; ils y habiteront, dis-je, eux, et leurs fils, et les fils de leurs fils, pour toujours ; et David mon serviteur sera leur prince pour toujours.
\VS{26}Je traiterai avec eux une alliance de paix, et il y aura une alliance éternelle avec eux ; je les établirai, et les multiplierai, je mettrai mon lieu saint au milieu d'eux pour toujours.
\VS{27}Ma demeure sera parmi eux ; je serai leur Dieu, et ils seront mon peuple.
\VS{28}Les nations sauront que je suis Yahweh qui sanctifie Israël, quand mon lieu saint sera au milieu d'eux pour toujours.
\Chap{38}
\TextTitle{Jugement sur Gog}
\VerseOne{}La parole de Yahweh me fut encore adressée en ces mots :
\VS{2}Fils de l'homme, tourne ta face vers Gog au pays de Magog\FTNT{Gog est un prince et Magog le pays. Ce chapitre doit être mis en parallèle avec Za. 12:1-4 ; Za. 14:1-9 ; Mt. 24:14-30 ; Ap. 14:14-20 ; Ap. 20:8.}, vers le prince de Rosch, de Méschec et de Tubal, et prophétise contre lui !
\VS{3}Tu diras : Ainsi parle le Seigneur Yahweh : Voici, j'en veux à toi, Gog, prince des chefs de Méschec et de Tubal !
\VS{4}Je te ferai retourner en arrière, et je mettrai des boucles dans tes mâchoires, et te ferai sortir avec toute ton armée, avec les chevaux, et les cavaliers, tous parfaitement bien équipés, une grande multitude portant le grand et le petit bouclier, et tous maniant l'épée ;
\VS{5}ceux de Perse, d'Ethiopie, et de Puth avec eux, qui tous ont des boucliers et des casques.
\VS{6}Gomer et toutes ses troupes, la maison de Togarma à l'extrême nord, avec toutes ses troupes, et plusieurs peuples avec toi.
\VS{7}Apprête-toi, tiens-toi prêt, toi, et toute la multitude assemblée autour de toi ! Sois leur chef !
\VS{8}Après plusieurs jours, tu seras à leur tête, et dans la suite des années, tu marcheras contre le pays dont les habitants, délivrés de l'épée, auront été rassemblés d'entre plusieurs peuples sur les montagnes d'Israël longtemps désertes ; retirés du milieu des peuples, ils seront en sécurité dans leurs demeures.
\VS{9}Tu monteras, tu viendras comme une dévastation, tu seras comme une nuée pour couvrir la terre, toi, toutes tes troupes, et plusieurs peuples avec toi\FTNT{Da. 11:40.}.
\VS{10}Ainsi parle le Seigneur Yahweh : Il arrivera dans ces jours-là que des pensées s'élèveront dans ton cœur, et que tu formeras un dessein pernicieux.
\VS{11}Car tu diras : Je monterai contre le pays dont les villes sont sans murailles ; j'envahirai ceux qui sont en repos, qui habitent en sécurité, qui demeurent tous dans des villes sans murs, lesquelles n'ont ni barres ni portes\FTNT{Jé. 49:31.} ;
\VS{12}pour enlever un grand butin et faire un grand pillage ; pour remettre ta main sur les déserts qui de nouveau étaient habités, et sur le peuple rassemblé d'entre les nations, ayant des troupeaux et des biens, et occupant les lieux élevés du pays.
\VS{13}Séba, et Dedan, les marchands de Tarsis, et tous ses lionceaux, te diront : Ne vas-tu pas pour faire du butin, et n'as-tu pas assemblé ta multitude pour faire un grand pillage, pour emporter de l'argent et de l'or, pour prendre le bétail et les biens, pour enlever un grand butin ?
\VS{14}Toi donc, fils de l'homme, prophétise, et dis à Gog : Ainsi parle le Seigneur Yahweh : En ce jour-là, quand mon peuple d'Israël habitera en sécurité, ne le sauras-tu pas ?
\VS{15}Ne viendras-tu pas de ton lieu, de l'extrême nord, toi, et plusieurs peuples avec toi, tous montés sur des chevaux, une grande multitude, et une grosse armée ?
\VS{16}Ne monteras-tu pas contre mon peuple d'Israël, comme une nuée pour couvrir la terre ? Dans la suite de ces jours, je te ferai venir sur ma terre, afin que les nations me connaissent, quand je serai sanctifié par toi sous leurs yeux, ô Gog !
\VS{17}Ainsi parle le Seigneur Yahweh : N'est-ce pas de toi que j'ai parlé autrefois par le ministère de mes serviteurs, les prophètes d'Israël, qui ont prophétisé dans ces jours-là pendant plusieurs années, qu'on te ferait venir contre eux ?
\VS{18}Mais il arrivera dans ce jour-là, au jour de la venue de Gog sur la terre d'Israël, dit le Seigneur Yahweh, que ma colère éclatera.
\VS{19}Je le déclare, dans ma jalousie, dans l'ardeur de ma fureur, en ce jour-là, il y aura une grande agitation sur la terre d'Israël.
\VS{20}Les poissons de la mer, les oiseaux des cieux, et les bêtes des champs, et tous les reptiles qui rampent sur la terre, et tous les hommes qui sont sur la surface de la terre seront épouvantés par ma présence ; les montagnes seront renversées, les parois des rochers tomberont, et tous les murs chuteront par terre.
\VS{21}J'appellerai contre lui l'épée sur toutes mes montagnes, dit le Seigneur Yahweh ; l'épée de chacun d'eux sera contre son frère.
\VS{22}J'entrerai en jugement avec lui par la peste, et par le sang ; je ferai pleuvoir sur lui, sur ses troupes, et sur les grands peuples qui seront avec lui, des torrents d'eau, des pierres de grêle, du feu et du soufre\FTNT{Ap. 8:7 ; Ps. 11:6 ; Ap. 16:21 ; Ap. 11:19.}.
\VS{23}Je me glorifierai, je me sanctifierai, je serai connu aux yeux de plusieurs nations ; et elles sauront que je suis Yahweh.
\Chap{39}
\TextTitle{Jugement sur Gog, suite}
\VerseOne{}Toi donc, fils de l'homme, prophétise contre Gog, et dis : Ainsi parle le Seigneur Yahweh : Voici, j'en veux à toi, Gog, prince des chefs de Méschec et de Tubal !
\VS{2}Je te ferai retourner en arrière, je te conduirai, je te ferai monter de l'extrême nord, et je t'amènerai sur les montagnes d'Israël.
\VS{3}Car je frapperai ton arc dans ta main gauche, et je ferai tomber tes flèches de ta main droite.
\VS{4}Tu tomberas sur les montagnes d'Israël, toi et toutes tes troupes, et les peuples qui seront avec toi ; je te livrerai aux oiseaux de proie, à tout ce qui a des ailes, et aux bêtes des champs, pour en être dévoré.
\VS{5}Tu tomberas sur la face des champs, parce que j'ai parlé, dit le Seigneur Yahweh.
\VS{6}Je mettrai le feu dans Magog, et parmi ceux qui demeurent en sécurité dans les îles ; et ils sauront que je suis Yahweh.
\VS{7}Je ferai connaître mon saint Nom au milieu de mon peuple d'Israël ; et je ne profanerai plus mon saint Nom ; les nations sauront que je suis Yahweh, le Saint d'Israël.
\VS{8}Voici, cela arrive et sera fait, dit le Seigneur Yahweh ; c'est ici le jour dont j'ai parlé.
\VS{9}Les habitants des villes d'Israël sortiront, allumeront le feu, brûleront les armes, les petits et les grands boucliers, les arcs, les flèches, les bâtons qu'on lance de la main, et les javelots ; ils en feront du feu pendant sept ans.
\VS{10}On n'apportera point du bois des champs, et on n'en coupera point dans les forêts, parce qu'ils feront du feu de ces armes, lorsqu'ils dépouilleront ceux qui les avaient dépouillés, et qu'ils pilleront ceux qui les avaient pillés, dit le Seigneur Yahweh.
\VS{11}Il arrivera ce jour-là que je donnerai à Gog dans ces quartiers-là un lieu pour sépulcre en Israël, à savoir la vallée des passants, qui est au-devant de la mer ; elle réduira les passants au silence ; on enterrera là Gog, et toute la multitude de son peuple, et on l'appellera la vallée d'Hamon-Gog\FTNT{La vallée d'Hamon-Gog : La vallée de la multitude de Gog.}.
\VS{12}Ceux de la maison d'Israël les enterreront, et cela durera sept mois, afin de purifier le pays.
\VS{13}Tout le peuple du pays les enterrera, et il en aura du renom, le jour où je serai glorifié, dit le Seigneur Yahweh.
\VS{14}Ils mettront à part des gens qui ne feront autre chose que parcourir le pays, et qui enterreront, avec l'aide des passants, les corps restés à la surface de la terre, pour la purifier, et ils seront à la recherche pendant sept mois.
\VS{15}Ils parcourront le pays, et celui qui verra l'os d'un homme, dressera auprès de lui un signal ; jusqu'à ce que les fossoyeurs l'aient enterré dans la vallée d'Hamon-Gog.
\VS{16}Il y aura aussi une ville nommée Hamona\FTNT{Hamona signifie « multitude ».}, et on nettoiera le pays.
\VS{17}Toi donc, fils de l'homme, ainsi parle le Seigneur Yahweh : Dis aux oiseaux de toutes espèces, et à toutes les bêtes des champs : Assemblez-vous et venez ; amassez-vous de toutes parts vers mon sacrifice que je fais pour vous, qui est un grand sacrifice sur les montagnes d'Israël ! Vous mangerez de la chair, et vous boirez du sang.
\VS{18}Vous mangerez la chair des hommes puissants, et vous boirez le sang des princes de la terre, le sang des moutons, des agneaux, des boucs, et des veaux engraissés sur le Basan\FTNT{Es. 34:6 ; Jé. 46:10 ; So. 1:7 ; Mt. 24:28 ; Job. 39:33.}.
\VS{19}Vous mangerez de la graisse jusqu'à en être rassasiés, et vous boirez du sang jusqu'à en être ivres, de la graisse et du sang de mon sacrifice, que j'aurai sacrifié pour vous.
\VS{20}Vous serez rassasiés à ma table, de chevaux et de bêtes d'attelage, d'hommes forts, et de tous hommes de guerre, dit le Seigneur Yahweh.
\VS{21}Je mettrai ma gloire parmi les nations, et toutes les nations verront mon jugement que j'aurai exercé, et comment j'aurai mis ma main sur eux.
\VS{22}La maison d'Israël connaîtra dès ce jour-là, et dans la suite, que je suis Yahweh, leur Dieu.
\VS{23}Les nations sauront que la maison d'Israël avait été emmenée en captivité à cause de son iniquité, parce qu'ils avaient péché contre moi, et que je leur avais caché ma face ; aussi je les avais livrés entre les mains de leurs ennemis pour qu'ils périssent par l'épée\FTNT{De. 31:17-18 ; Ps. 13:2.}.
\VS{24}Je leur avais fait selon leurs souillures, et selon leurs crimes, et je leur avais caché ma face.
\TextTitle{Rétablissement et conversion d'Israël}
\VS{25}C'est pourquoi, ainsi parle le Seigneur Yahweh : Maintenant, je ramènerai la captivité de Jacob, et j'aurai pitié de toute la maison d'Israël, et je serai jaloux de mon saint Nom,
\VS{26}après avoir porté leur ignominie, et tout leur crime, lorsqu'ils avaient péché contre moi, quand ils demeuraient en sûreté dans leur terre, sans qu'il y eût personne pour les effrayer.
\VS{27}Parce que je les ramènerai d'entre les peuples, que je les rassemblerai des pays de leurs ennemis, et que je serai sanctifié par eux, sous les yeux de plusieurs nations.
\VS{28}Ils sauront que je suis Yahweh, leur Dieu, lorsqu'après les avoir enlevés parmi les nations, je les rassemblerai sur leurs terres, et que je n'en laisserai chez elles aucun d'eux.
\VS{29}Je ne leur cacherai plus ma face, car je répandrai mon Esprit sur la maison d'Israël, dit le Seigneur Yahweh\FTNT{Joë. 2:28 ; Ac. 2:17.}.
\Chap{40}
\TextTitle{Mesures du futur temple}
\VerseOne{}Dans la vingt-cinquième année de notre captivité, au commencement de l'année, au dixième jour du mois, la quatorzième année après que la ville fut prise, en ce même jour, la main de Yahweh fut sur moi, et il m'amena là.
\VS{2}Il m'amena par des visions de Dieu, au pays d'Israël, et me posa sur une montagne fort élevée, sur laquelle du côté sud il y avait comme une ville construite.
\VS{3}Après qu'il m'y fît entrer, voici un homme, dont l'aspect était comme de l'airain, qui avait dans sa main un cordeau de lin, et une canne à mesurer, et qui se tenait debout à la porte.
\VS{4}Cet homme me parla ainsi : Fils de l'homme, regarde de tes yeux, écoute de tes oreilles, et applique ton cœur à toutes les choses que je m'en vais te faire voir, car tu as été amené ici afin que je te les fasse voir, et que tu fasses savoir à la maison d'Israël toutes les choses que tu vas voir.
\VS{5}Voici, un mur extérieur entourait la maison. Cet homme avait dans la main une canne à mesurer longue de six coudées, chaque coudée étant d'une coudée normale et une largeur de main en plus. Il mesura la largeur de ce mur bâti, laquelle était d'une canne, et sa hauteur d'une autre canne.
\VS{6}Puis il vint vers la porte orientale, et monta par ses étages. Il mesura l'un des poteaux de la porte d'une canne en largeur, et l'autre poteau d'une autre canne en largeur.
\VS{7}Puis il mesura chaque chambre d'une canne en longueur, et d'une canne en largeur. L'espace entre les deux chambres était de cinq coudées. Il mesura d'une canne chacun des poteaux de la porte près du vestibule qui menait à la porte la plus intérieure.
\VS{8}Puis il mesura d'une canne le vestibule qui menait à la porte la plus intérieure.
\VS{9}Il mesura de huit coudées le vestibule de la porte et ses poteaux, le vestibule de la porte était en dedans.
\VS{10}Les chambres de la porte orientale étaient au nombre de trois d'un côté et de trois de l'autre, toutes les trois avaient la même mesure.
\VS{11}Puis il mesura de dix coudées la largeur de l'ouverture de la première porte, et de treize coudées la longueur de la même porte.
\VS{12}Ensuite, il mesura d'un côté un espace limité au-devant des chambres d'une coudée, et une autre coudée d'espace limité de l'autre côté ; chaque chambre avait six coudées d'un côté, et six coudées de l'autre.
\VS{13}Après cela, il mesura le portail depuis le toit d'une chambre jusqu'au toit de l'autre, de la largeur de vingt-cinq coudées entre les deux ouvertures opposées.
\VS{14}Il compta soixante coudées pour les poteaux, près desquels était une cour, autour de la porte.
\VS{15}L'espace entre la porte d'entrée et le vestibule de la porte intérieure était de cinquante coudées.
\VS{16}Il y avait des fenêtres closes aux chambres et à leurs poteaux, à l'intérieur de la porte tout autour. Il y avait aussi des fenêtres dans les vestibules tout autour intérieurement, des palmes étaient sculptées sur les poteaux.
\VS{17}Il me mena dans le parvis extérieur, où se trouvaient des chambres et un pavé tout autour. Il y avait trente chambres sur ce pavé.
\VS{18}Le pavé était au côté des portes et répondait à la longueur des portes ; c'était le pavé inférieur.
\VS{19}Il mesura la largeur du parvis depuis la porte qui menait vers le bas jusqu'au parvis intérieur en dehors. Il y avait cent coudées à l'orient et au nord.
\VS{20}Après cela, il mesura la longueur et la largeur de la porte nord du parvis extérieur.
\VS{21}Quant aux chambres, au nombre de trois d'un côté et trois de l'autre, ses poteaux et ses vestibules avaient la même mesure que la première porte, cinquante coudées en longueur, et vingt-cinq coudées en largeur.
\VS{22}Ses fenêtres, son vestibule, et ses palmes avaient la même mesure que la porte orientale ; on y montait par sept étages, devant lesquels étaient son vestibule.
\VS{23}La porte du parvis intérieur était vis-à-vis de la première porte du nord, et vis-à-vis de la porte orientale. Il mesura depuis une porte jusqu'à l'autre cent coudées.
\VS{24}Après cela, il me conduisit du côté sud, où se trouvait la porte méridionale, il en mesura les poteaux et les vestibules qui avaient la même mesure.
\VS{25}Cette porte et ses vestibules avaient des fenêtres tout autour, comme les autres fenêtres, cinquante coudées de long, et vingt-cinq coudées de large.
\VS{26}On y montait par sept étages, devant lesquels était son vestibule ; il avait de chaque côté des palmes sur ses poteaux.
\VS{27}Pareillement, le parvis intérieur avait sa porte du côté sud ; il mesura d'une porte à l'autre au sud cent coudées.
\VS{28}Après cela il me fit entrer dans le parvis intérieur par la porte sud, et il mesura la porte sud, selon les mesures précédentes.
\VS{29}Ses chambres, ses poteaux et ses vestibules avaient la même mesure. Cette porte et ses vestibules avaient des fenêtres tout autour, cinquante coudées de long, et vingt-cinq coudées de large.
\VS{30}Il y avait tout autour des vestibules de vingt-cinq coudées de long, et cinq coudées de large.
\VS{31}Les vestibules de la porte aboutissaient au parvis extérieur ; il y avait des palmes sur ses poteaux, et huit étages pour y monter.
\VS{32}Il me conduisit dans le parvis intérieur, par l'entrée orientale. Il mesura la porte, qui avait la même mesure.
\VS{33}Ses chambres, ses poteaux et ses vestibules avaient la même mesure. Cette porte et ses vestibules avaient des fenêtres tout autour, cinquante coudées de long, et vingt-cinq de large.
\VS{34}Ses vestibules aboutissaient au parvis extérieur ; il y avait de chaque côté des palmes sur ses poteaux, et huit étages pour y monter.
\VS{35}Il me conduisit vers la porte nord. Il la mesura et trouva la même mesure.
\VS{36}Ainsi qu'à ses chambres, à ses poteaux et à ses vestibules ; elle avait des fenêtres tout autour, cinquante coudées de long, et vingt-cinq coudées de large.
\VS{37}Ses vestibules aboutissaient au parvis extérieur ; il y avait de chaque côté des palmes sur ses poteaux, et huit étages pour y monter.
\VS{38}Il y avait une chambre qui s'ouvrait vers les poteaux des portes, et où l'on devait laver les holocaustes.
\VS{39}Il y avait aussi dans le vestibule de la porte de chaque côté deux tables, pour y égorger les bêtes qu'on sacrifierait pour l'holocauste, et le sacrifice pour l'expiation et le sacrifice pour la culpabilité.
\VS{40}Vers l'un des côtés de la porte, au dehors, vers le lieu où l'on montait, à l'entrée de la porte nord, il y avait deux tables, et de l'autre côté, vers le vestibule de la porte, deux autres tables.
\VS{41}Il se trouvait ainsi, aux côtés de la porte, quatre tables d'une part, et quatre tables de l'autre, en tout huit tables, sur lesquelles on devait abattre les victimes.
\VS{42}Les quatre tables qui étaient pour l'offrande entièrement consumée, étaient en pierres de taille, de la longueur d'une coudée et demie, et de la largeur d'une coudée et demie, et de la hauteur d'une coudée ; et même on devait poser sur elles les instruments avec lesquels on tuait les victimes pour les offrandes entièrement consumées, et les autres sacrifices.
\VS{43}Il y avait aussi à l'intérieur de la maison tout autour, des chevilles pour accrocher, larges d'une paume, bien adaptées, d'où l'on apportait la chair des sacrifices sur les tables.
\TextTitle{Répartition des pièces du futur temple}
\VS{44}En dehors de la porte intérieure, il y avait des chambres pour les chantres dans le parvis intérieur, l'une était à côté de la porte nord et avait la face au sud, l'autre était à côté de la porte orientale et avait la face au nord.
\VS{45}Il me dit : Ces chambres, dont la face est au sud, sont pour les sacrificateurs qui ont la charge de la maison.
\VS{46}Mais ces chambres, dont la face est au nord, sont pour les sacrificateurs qui ont la charge de l'autel, qui sont les fils de Tsadok, qui, parmi les fils de Lévi, s'approchent de Yahweh pour faire son service.
\VS{47}Puis il mesura un parvis de la longueur et de la largeur de cent coudées, en carré ; et l'autel était devant la maison.
\VS{48}Ensuite, il me fit entrer dans le vestibule de la maison ; et il mesura les poteaux du vestibule de cinq coudées d'un côté, et de cinq coudées de l'autre, puis la largeur de la porte de trois coudées d'un côté, et de trois coudées de l'autre.
\VS{49}Le vestibule avait une longueur de vingt coudées, et une largeur de onze coudées ; on y montait par des étages. Il y avait des colonnes près des poteaux, l'une d'un côté, et l'autre de l'autre.
\Chap{41}
\TextTitle{Description du temple}
\VerseOne{}Puis il me fit entrer dans le temple, et il mesura des poteaux de six coudées de largeur d'un côté, et de six coudées de largeur de l'autre côté, largeur de la tente.
\VS{2}Ensuite il mesura la largeur de l'ouverture de la porte qui était de dix coudées, et les côtés de l'ouverture de cinq coudées, d'une part, et de cinq coudées de l'autre part. Puis il mesura la longueur du temple, quarante coudées, et la largeur, vingt coudées.
\VS{3}Il entra à l'intérieur, et il mesura un poteau d'une ouverture de porte, deux coudées, la hauteur de cette ouverture, six coudées, et la largeur de cette ouverture, sept coudées.
\VS{4}Puis il mesura une longueur de vingt coudées, et une largeur de vingt coudées en face du temple ; et il me dit : C'est ici le Saint des saints.
\VS{5}Il mesura l'épaisseur du mur de la maison, qui fut de six coudées, et la largeur des chambres qui étaient tout autour de la maison, de quatre coudées.
\VS{6}Les chambres latérales étaient les unes à côté des autres, au nombre de trente, et il y avait trois poutres ; elles entraient dans un mur construit pour ces chambres tout autour de la maison, elles y étaient appuyées sans entrer dans le mur même de la maison.
\VS{7}Les chambres occupaient plus d'espace, à mesure qu'elles s'élevaient, et l'on allait en tournant, car on montait autour de la maison par un escalier tournant. Il y avait plus d'espace dans le haut de la maison, et l'on montait de l'étage inférieur à l'étage supérieur par celui du milieu.
\VS{8}Je considérai la hauteur autour de la maison. Les chambres latérales, à partir de leur fondement avaient une canne pleine, six grandes coudées.
\VS{9}La largeur du mur extérieur des chambres latérales était de cinq coudées ; l'espace libre entre les chambres latérales de la maison,
\VS{10}et les chambres autour de la maison avait une largeur de vingt coudées.
\VS{11}L'ouverture des chambres latérales donnait sur l'espace libre, une ouverture au nord, et une autre ouverture au sud ; la largeur de l'espace libre était de cinq coudées tout autour.
\VS{12}Le bâtiment qui était devant la place vide, du côté de l'occident, avait une largeur de soixante-dix coudées, un mur de cinq coudées de largeur tout autour, et une longueur de quatre-vingt-dix coudées.
\VS{13}Il mesura la maison, qui avait cent coudées de longueur ; la place vide, le bâtiment et les murs avaient une longueur de cent coudées.
\VS{14}La largeur de la face de la maison et de la place vide, du côté oriental, était de cent coudées.
\VS{15}Il mesura la longueur du bâtiment devant la place vide, sur le derrière, et ses galeries de chaque côté : Il y avait cent coudées.
\VS{16}Les seuils, les fenêtres closes, les galeries du pourtour aux trois étages, en face des seuils, étaient recouverts de bois tout autour. Depuis le sol jusqu'aux fenêtres fermées,
\VS{17}jusqu'au-dessus des ouvertures, et jusqu'à la maison au-dedans comme au dehors, tout le mur du pourtour, à l'intérieur et à l'extérieur, tout était d'après la mesure,
\VS{18}et fait de chérubins et de palmes. Il y avait une palme entre deux chérubins, et chaque chérubin avait deux faces.
\VS{19}Une face d'homme était tournée vers la palme d'un côté, et une face de jeune lion était tournée vers la palme de l'autre côté ; il en était ainsi tout autour de la maison.
\VS{20}Depuis le sol jusqu'au-dessus des ouvertures il y avait des chérubins et des palmes et aussi sur le mur du temple.
\VS{21}Les poteaux du temple étaient carrés ; et la face du lieu saint avait la même apparence.
\VS{22}L'autel était de bois, de la hauteur de trois coudées, et de deux coudées de longueur ; ses angles, ses pieds et ses côtés étaient de bois. Puis il me dit : C'est ici la table qui est devant Yahweh.
\VS{23}Le temple et le lieu saint avaient deux portes.
\VS{24}Il y avait deux portes, deux battants, qui tous deux tournaient sur les portes, deux battants pour une porte et deux pour l'autre.
\VS{25}Il y avait aussi des chérubins et des palmes façonnés sur les portes du temple, comme sur les murs. Un entablement en bois était sur le front du vestibule en dehors.
\VS{26}Il y avait des fenêtres fermées, et des palmes de part et d'autre, ainsi qu'aux côtés du vestibule, aux chambres latérales de la maison, et aux entablements.
\Chap{42}
\TextTitle{Mesures supplémentaires du temple}
\VerseOne{}Après cela, il me fit sortir vers le parvis extérieur, du côté nord ; et il me conduisit vers les chambres qui étaient vis-à-vis de la place vide et vis-à-vis du bâtiment, au nord.
\VS{2}Sur la face où se trouvait une ouverture au nord, il y avait une longueur de cent coudées, et la largeur était de cinquante coudées.
\VS{3}C'était vis-à-vis des vingt coudées du parvis intérieur, et vis-à-vis du pavé extérieur, là où se trouvaient les galeries des trois étages.
\VS{4}Devant les chambres, il y avait une promenade large de dix coudées, et une voie d'une coudée ; leurs ouvertures donnaient au nord.
\VS{5}Les chambres supérieures étaient plus étroites que les inférieures et que celles du milieu du bâtiment parce que les galeries leur ôtaient de la place.
\VS{6}Car elles étaient à trois étages, et n'avaient point de colonnes, comme les colonnes des parvis ; c'est pourquoi à partir du sol, les chambres du haut étaient plus étroites que celles du bas et du milieu.
\VS{7}Le mur extérieur parallèle aux chambres, du côté du parvis extérieur devant les chambres, avait cinquante coudées de long.
\VS{8}Car la longueur des chambres du côté du parvis extérieur était de cinquante coudées. Mais sur la face du temple, il y avait cent coudées.
\VS{9}Au bas de ces chambres était l'entrée orientale quand on y venait du parvis extérieur.
\VS{10}Il y avait encore des chambres sur la largeur du mur du parvis du côté oriental, vis-à-vis de la place vide et vis-à-vis du bâtiment.
\VS{11}Devant elles, il y avait un chemin, comme devant les chambres qui étaient du côté nord. La longueur et la largeur étaient les mêmes ; leurs issues, leur disposition et leurs ouvertures étaient semblables.
\VS{12}Il en était de même pour les ouvertures des chambres du côté sud. Il y avait une ouverture à la tête du chemin, du chemin qui se trouvait droit devant le mur du côté oriental par où l'on y entrait.
\VS{13}Après cela, il me dit : Les chambres du parvis nord et les chambres du parvis sud, qui sont devant la place vide, ce sont les chambres du lieu saint où les sacrificateurs qui s'approchent de Yahweh, mangeront les choses très saintes. Ils déposeront là les choses très saintes, savoir les gâteaux, les offrandes pour l'expiation et les offrandes pour la culpabilité ; car ce lieu est saint.
\VS{14}Quand les sacrificateurs seront entrés, ils ne sortiront point du lieu saint pour venir au parvis extérieur, mais ils déposeront là leurs vêtements avec lesquels ils font le service ; car ces vêtements sont saints ; ils en mettront d'autres pour s'approcher du peuple.
\VS{15}Lorsqu'il eut achevé de mesurer la maison intérieure, il me fit sortir par la porte qui était du côté oriental, puis il mesura l'enceinte tout autour.
\VS{16}Il mesura le côté oriental avec la canne qui servait de mesure, et il y avait tout autour cinq cents cannes.
\VS{17}Ensuite il mesura le côté nord, avec la canne qui servait de mesure, et il y avait tout autour cinq cents cannes.
\VS{18}Puis il mesura le côté sud avec la canne qui servait de mesure, et il y avait cinq cents cannes.
\VS{19}Il se tourna du côté occidental, et mesura cinq cents cannes avec la canne qui servait de mesure.
\VS{20}Il mesura des quatre côtés le mur formant l'enceinte de la maison ; la longueur était de cinq cents cannes, et la largeur de cinq cents cannes, ce mur marquait la séparation entre le saint et le profane.
\Chap{43}
\TextTitle{La gloire de Yahweh remplit la maison\FTNTT{Cp. Ez. 11:22-24.}}
\VerseOne{}Puis il me ramena à la porte, à la porte qui était du côté oriental.
\VS{2}Et voici, la gloire du Dieu d'Israël s'avançait de l'orient, sa voix était pareille au bruit des grandes eaux, et la terre resplendissait de sa gloire\FTNT{Ap. 1:15.}.
\VS{3}La vision que j'eus alors était semblable à celle que j'avais vue lorsque j'étais venu pour détruire la ville, ces visions étaient comme la vision que j'avais vue sur le fleuve de Kebar ; et je me prosternai le visage contre terre.
\VS{4}Puis la gloire de Yahweh entra dans la maison par la porte qui était du côté oriental.
\VS{5}L'Esprit m'enleva et me fit entrer dans le parvis intérieur, et voici la gloire de Yahweh remplissait la maison.
\TextTitle{Le trône de Yahweh}
\VS{6}Je l'entendis s'adressant à moi depuis la maison, et l'homme qui me conduisait était debout près de moi.
\VS{7}Yahweh me dit : Fils de l'homme, c'est ici le lieu de mon trône, et le lieu des plantes de mes pieds, dans lequel je ferai ma demeure éternellement parmi les fils d'Israël ; et la maison d'Israël ne souillera plus mon saint Nom, ni eux, ni leurs rois, par leurs fornications ; mais ils souilleront leurs hauts lieux par les cadavres de leurs rois.
\VS{8}Car ils ont mis leur seuil près de mon seuil, et leur poteau près de mon poteau, il y avait un mur entre moi et eux ; ils ont souillé mon saint Nom par leurs abominations qu'ils ont faites, c'est pourquoi je les ai consumés dans ma colère.
\VS{9}Maintenant ils rejetteront loin de moi leurs adultères et les cadavres de leurs rois, et je ferai ma demeure éternellement parmi eux.
\VS{10}Toi donc, fils de l'homme, montre ce temple à la maison d'Israël ; et qu'ils soient confus à cause de leurs iniquités ; et qu'ils aient honte de leur iniquité.
\VS{11}S'ils rougissent de tout ce qu'ils ont fait, fais-leur connaître la forme de ce temple, sa disposition, avec ses sorties et ses entrées, toutes ses figures et toutes ses ordonnances, toutes ses formes, toutes ses lois, et écris-les sous leurs yeux, afin qu'ils gardent toutes ses formes, et toutes les ordonnances, et qu'ils les pratiquent.
\VS{12}Tel est la loi de la maison. Sur le sommet de la montagne, tout le territoire sera un lieu très saint tout autour. Voilà donc la loi de la maison.
\TextTitle{L'autel pour les holocaustes et les sacrifices}
\VS{13}Voici les mesures de l'autel, d'après les coudées dont chacune était d'une largeur de main plus longue que la coudée ordinaire. Le fond avait une coudée de hauteur et une coudée de largeur, et le rebord qui terminait son contour avait un empan de largeur ; c'était le dos de l'autel.
\VS{14}Depuis le fond sur le sol jusqu'à l'encadrement inférieur, il y avait deux coudées, et une coudée de largeur, et depuis le petit jusqu'au grand encadrement, il y avait quatre coudées et une coudée de largeur.
\VS{15}L'autel avait quatre coudées ; et quatre cornes s'élevaient de l'autel.
\VS{16}L'autel avait douze coudées de longueur, douze coudées de largeur, et formait un carré par ses quatre côtés.
\VS{17}L'encadrement avait quatorze coudées de longueur sur quatorze coudées de largeur à ses quatre côtés, le rebord qui terminait son contour avait une demi-coudée, le fond avait une coudée tout autour, et les étages étaient tournés vers l'orient.
\VS{18}Il me dit : Fils de l'homme, ainsi parle le Seigneur Yahweh : Ce sont ici les lois au sujet de l'autel pour le jour où on le fera, afin qu'on y offre l'holocauste, et qu'on y répande le sang.
\VS{19}C'est que tu donneras aux sacrificateurs, aux Lévites, qui sont de la race de Tsadok, et qui s'approchent de moi, dit le Seigneur Yahweh, afin qu'ils y fassent mon service, un jeune veau en sacrifice pour le péché.
\VS{20}Et tu prendras de son sang, et en mettras sur les quatre cornes de l'autel, et sur les quatre angles de l'encadrement et sur le rebord qui l'entoure, ainsi tu purifieras l'autel, et tu feras propitiation pour lui\FTNT{Ex. 29:36-39.}.
\VS{21}Tu prendras le jeune taureau expiatoire, et on le brûlera dans un lieu réservé de la maison, en dehors du lieu saint.
\VS{22}Le second jour, tu offriras en expiation un bouc, sans défaut, et on purifiera l'autel comme on l'aura purifié avec le jeune taureau.
\VS{23}Quand tu auras achevé de purifier l'autel, tu offriras un jeune taureau sans défaut, et un bélier du troupeau sans défaut.
\VS{24}Tu les offriras devant Yahweh, et les sacrificateurs jetteront du sel par dessus, et les offriront en holocauste à Yahweh\FTNT{Lé. 2:13.}.
\VS{25}Durant sept jours, tu sacrifieras chaque jour un bouc comme victime expiatoire, et les sacrificateurs sacrifieront un jeune taureau et un bélier du troupeau sans défaut.
\VS{26}Pendant sept jours, les sacrificateurs feront la propitiation pour l'autel, on le purifiera et chacun d'eux sera consacré\FTNT{Le terme « consacré » veut dire littéralement « remplir sa main ». Voir aussi Jg.17:5 et 17:12}
\VS{27}Lorsque ces jours seront accomplis, dès le huitième jour, et à l'avenir, les sacrificateurs offriront sur cet autel vos holocaustes et vos sacrifices d'offrande de paix. Et je serai apaisé envers vous, dit le Seigneur Yahweh.
\Chap{44}
\TextTitle{La porte fermée du sanctuaire}
\VerseOne{}Puis il me ramena vers la porte extérieure du lieu saint, du côté oriental, mais elle était fermée.
\VS{2}Yahweh me dit : Cette porte-ci sera fermée, et ne sera point ouverte, personne n'y passera, parce que Yahweh, le Dieu d'Israël, est entré par cette porte ; elle sera donc fermée\FTNT{Ap. 3:8.}.
\VS{3}Elle sera pour le prince ; le prince sera le seul qui s'y assiéra pour manger le pain devant Yahweh ; il entrera par le chemin du vestibule de la porte, et sortira par le même chemin.
\TextTitle{[La gloire dans la maison de Yahweh]}
\VS{4}Il me fit revenir par le chemin de la porte nord jusque sur le devant de la maison, je regardai, et voici, la gloire de Yahweh avait rempli la maison de Yahweh, et je me prosternai sur ma face.
\VS{5}Alors Yahweh me dit : Fils de l'homme, applique ton cœur, et regarde de tes yeux, écoute de tes oreilles tout ce dont je vais te parler, concernant toutes les ordonnances et toutes les lois qui concernent la maison de Yahweh. Applique ton cœur en ce qui concerne l'entrée de la maison et toutes les sorties du lieu saint.
\VS{6}Tu diras aux rebelles, à la maison d'Israël : Ainsi parle le Seigneur Yahweh : Maison d'Israël ! Assez de toutes vos abominations !
\VS{7}Vous avez fait entrer les fils de l'étranger, incirconcis de cœur et incirconcis de chair, pour être dans mon lieu saint, pour profaner ma maison. Vous avez offert mon pain, la graisse et le sang, à toutes vos abominations, vous avez enfreint mon alliance\FTNT{Lé. 3:11-16 ; Lé. 22:25 ; No. 28:2.}.
\VS{8}Vous n'avez pas observé l'office de mon lieu saint, mais vous les avez mis à votre place pour faire l'office dans mon lieu saint.
\TextTitle{Recommandations aux sacrificateurs du futur temple}
\VS{9}Ainsi parle le Seigneur Yahweh : Pas un de tous ceux qui seront fils d'étranger, incirconcis de cœur et incirconcis de chair, n'entrera dans mon lieu saint, pas même un d'entre tous les fils d'étrangers qui seront parmi les fils d'Israël.
\VS{10}Mais les Lévites qui se sont éloignés de moi, lorsque Israël s'est égaré, et qui se sont égarés de moi pour suivre leurs idoles, porteront la peine de leur iniquité.
\VS{11}Toutefois, ils seront employés dans mon lieu saint aux charges qui sont vers les portes de la maison, et ils feront le service de la maison ; ils égorgeront pour le peuple les bêtes pour l'holocauste, et pour les autres sacrifices, et se tiendront prêts devant lui pour le servir.
\VS{12}Parce qu'ils l'ont servi se présentant devant leurs idoles, et qu'ils ont fait tomber dans l'iniquité la maison d'Israël, à cause de cela j'ai levé ma main en jurant contre eux, dit le Seigneur Yahweh, qu'ils porteront la peine de leur iniquité.
\VS{13}Ils n'approcheront plus de moi pour exercer la sacrificature, ni pour approcher mes sanctuaires, mes lieux très saints ; mais ils porteront leur confusion et leurs abominations qu'ils ont commises.
\VS{14}C'est pourquoi je les établirai pour avoir la garde de la maison pour tout son service, et pour tout ce qui s'y fait.
\VS{15}Mais quant aux sacrificateurs et aux Lévites, fils de Tsadok, qui ont soigneusement administré ce qu'il fallait faire dans mon lieu saint, lorsque les fils d'Israël se sont éloignés de moi, ceux-là s'approcheront de moi pour faire mon service, et se tiendront devant moi pour m'offrir la graisse et le sang, dit le Seigneur Yahweh.
\VS{16}Ceux-là entreront dans mon lieu saint, et s'approcheront de ma table, pour faire mon service, et ils administreront soigneusement ce que j'ai ordonné de faire.
\VS{17}Lorsqu'ils franchiront les portes des parvis intérieurs, ils se vêtiront de robes de lin ; et il n'y aura point de laine sur eux pendant qu'ils feront le service aux portes des parvis intérieurs et dans la maison.
\VS{18}Ils auront des ornements de lin sur leur tête, et des caleçons de lin sur leurs reins, et ne se ceindront point de manière à provoquer la sueur.
\VS{19}Quand ils sortiront pour aller dans le parvis extérieur, dans le parvis extérieur, vers le peuple, ils se dévêtiront de leurs habits, avec lesquels ils font le service, et les poseront dans les chambres saintes, et se revêtiront d'autres habits, afin qu'ils ne sanctifient point le peuple avec leurs habits.
\VS{20}Ils ne se raseront point la tête, ni ne laisseront point croître leurs cheveux, mais simplement ils tondront leur tête\FTNT{Lé. 19:27.}.
\VS{21}Pas un des sacrificateurs ne boira du vin quand ils entreront au parvis intérieur.
\VS{22}Ils ne prendront point pour femme une veuve, ni une répudiée ; mais ils prendront des vierges, de la race de la maison d'Israël, ou une veuve qui soit veuve d'un sacrificateur\FTNT{Lé. 21:13-14.}.
\VS{23}Ils enseigneront à mon peuple la différence qu'il y a entre le saint et le profane, et leur feront entendre la différence qu'il y a entre ce qui est souillé et ce qui est pur.
\VS{24}Quand il surviendra quelque procès, ils assisteront au jugement, et jugeront suivant les lois que j'ai données ; et ils garderont mes lois et mes statuts dans toutes mes fêtes, et ils sanctifieront mes sabbats.
\VS{25}Un sacrificateur n'ira pas vers un mort, de peur d'en être souillé, il pourra se rendre impur que pour un père, pour une mère, pour un fils, pour une fille, pour un frère, et pour une sœur qui n'aura point eu de mari\FTNT{Lé. 21:1-2.}.
\VS{26}Et après que chacun d'eux se sera purifié, on lui comptera sept jours.
\VS{27}Le jour où il entrera dans le lieu saint, dans le parvis intérieur pour faire le service dans le lieu saint, il offrira son sacrifice pour son péché, dit le Seigneur Yahweh.
\VS{28}Et cela leur sera pour héritage. Ce sera moi leur héritage, car vous ne leur donnerez aucune possession en Israël, ce sera moi leur possession\FTNT{No. 18:20 ; De. 18:1-2.}.
\VS{29}Ils mangeront donc les gâteaux et ce qui s'offrira pour l'expiation, et ce qui s'offrira pour la culpabilité ; et tout interdit en Israël leur appartiendra.
\VS{30}Les prémices de tous les fruits et toutes les offrandes que vous présenterez par élévation, appartiendront aux sacrificateurs ; vous donnerez aussi les prémices de votre pâte aux sacrificateurs, afin que la bénédiction repose sur votre maison.
\VS{31}Les sacrificateurs ne mangeront aucune créature volante, aucun animal mort ou déchiré\FTNT{Ex. 22:31 ; Lé. 22:8.}.
\Chap{45}
\TextTitle{Zone réservée à Yahweh et aux sacrificateurs}
\VerseOne{}Quand vous partagerez au sort le pays en héritage, vous prélèverez comme une offrande en élévation pour Yahweh, une portion du pays longue de vingt-cinq mille cannes, et large de dix mille ; ce sera une chose sainte dans tous ses territoires et aux environs.
\VS{2}De cette portion, vous prendrez pour le lieu saint cinq cents cannes sur cinq cents en carré, et cinquante coudées tout autour pour ses faubourgs.
\VS{3}Sur cette étendue de vingt-cinq mille en longueur, et de dix mille en largeur, tu mesureras un emplacement pour le lieu saint, pour le Saint des saints.
\VS{4}C'est la portion sainte du pays, elle appartiendra aux sacrificateurs qui font le service du lieu saint, qui s'approchent de Yahweh pour le servir ; c'est là que seront leur maison, et ce sera un lieu saint pour le lieu saint.
\VS{5}Vingt-cinq mille cannes en longueur, et dix mille en largeur, formeront la propriété des Lévites, serviteurs de la maison, avec vingt chambres.
\VS{6}Vous donnerez pour la possession de la ville la largeur de cinq mille et la longueur de vingt-cinq mille, suivant la proportion de la portion sanctifiée, qui aura été levée pour toute la maison d'Israël.
\TextTitle{Zone réservée au prince}
\VS{7}Pour le prince vous réserverez un espace aux deux côtés de la portion sainte et de la propriété de la ville, le long de la portion sainte et le long de la propriété de la ville, du côté de l'occident vers l'occident, et du côté de l'orient vers l'orient, sur une longueur parallèle à l'une des parts, depuis la limite de l'occident jusqu'à la limite de l'orient.
\VS{8}Ce sera sa terre, sa propriété en Israël ; et mes princes que j'établirai ne fouleront plus mon peuple, mais ils distribueront le pays à la maison d'Israël, selon leurs tribus.
\TextTitle{Le prince, exemple au milieu du peuple ; prescriptions sur les offrandes}
\VS{9}Ainsi parle le Seigneur Yahweh : Assez, princes d'Israël ! Otez la violence et le pillage, et jugez avec justice ; ôtez vos extorsions de dessus mon peuple ! dit le Seigneur Yahweh.
\VS{10}Ayez la balance juste, l'épha juste, et le bath juste\FTNT{Lé. 19:35-36.}.
\VS{11}L'épha et le bath seront de même mesure ; on prendra un bath pour la dixième partie d'un homer, et l'épha sera la dixième partie d'un homer, la mesure de l'un et de l'autre se rapportera à l'homer.
\VS{12}Le sicle sera de vingt guéras ; vingt sicles, vingt-cinq sicles et quinze sicles feront la mine\FTNT{Ex. 30:13 ; Lé. 27:25.}.
\VS{13}Voici l'offrande que vous élèverez en offrande : La sixième partie d'un épha d'un homer de blé ; et vous donnerez la sixième partie d'un épha d'un homer d'orge.
\VS{14}Le bath est la mesure pour l'huile, l'offrande ordonnée pour l'huile sera la dixième partie d'un bath sur un cor, qui est égal à un homer de dix baths ; car dix baths feront un homer.
\VS{15}Pareillement l'offrande ordonnée des bêtes du menu bétail sera de deux cents l'une, même des meilleurs pâturages d'Israël ; toute cette offrande sera employée en gâteaux et en holocaustes, et en offrandes de paix, afin de faire propitiation pour vous, dit le Seigneur Yahweh.
\VS{16}Tout le peuple du pays sera tenu à cette offrande élevée, pour celui qui sera prince en Israël.
\VS{17}Mais le prince sera tenu de fournir les holocaustes, les offrandes et les libations qu'il faudra offrir aux fêtes solennelles, aux nouvelles lunes et aux sabbats, et dans toutes les solennités de la maison d'Israël. Il tiendra prêtes les bêtes qu'on sacrifiera pour l'expiation, et les gâteaux, et les bêtes qu'on sacrifiera pour l'holocauste, et les bêtes qu'on sacrifiera pour les offrandes de paix, afin de faire propitiation pour la maison d'Israël.
\VS{18}Ainsi parle le Seigneur Yahweh : Au premier mois, au premier jour du mois, tu prendras un jeune taureau sans défaut, et tu feras l'expiation du lieu saint.
\VS{19}Le sacrificateur prendra du sang de ce sacrifice offert pour le péché, et en mettra sur les poteaux de la maison, et sur les quatre angles de l'encadrement de l'autel, et sur les poteaux de la porte du parvis intérieur.
\VS{20}Tu en feras ainsi au septième jour du même mois, à cause des hommes qui pèchent involontairement et à cause des hommes simples ; et vous ferez ainsi propitiation pour la maison.
\VS{21}Au premier mois, au quatorzième jour du mois, vous aurez la Pâque, fête solennelle qui durera sept jours, pendant lesquels on mangera des pains sans levain\FTNT{Lé. 25:5 ; No. 9:3 ; Ex. 12.}.
\VS{22}En ce jour-là, le prince offrira un taureau pour le sacrifice d'expiation, tant pour lui que pour tout le peuple du pays.
\VS{23}Pendant les sept jours de cette fête solennelle, il offrira chaque jour sept taureaux et sept béliers sans défaut, pour l'holocauste qu'on offrira à Yahweh, et un bouc en sacrifice d'expiation, chaque jour.
\VS{24}Il offrira un épha pour chaque taureau, et un épha pour chaque bélier, avec un hin d'huile par épha.
\VS{25}Au septième mois, le quinzième jour du mois, à la fête solennelle, il offrira durant sept jours les mêmes choses, le même sacrifice expiatoire, le même holocauste, et la même offrande avec l'huile.
\Chap{46}
\TextTitle{Le service le jour du sabbat et les jours de fêtes}
\VerseOne{}Ainsi parle le Seigneur Yahweh : La porte du parvis intérieur, du côté oriental, sera fermée les six jours ouvrables, mais elle sera ouverte le jour du sabbat, elle sera aussi ouverte le jour de la nouvelle lune.
\VS{2}Et le prince y entrera par le chemin du vestibule de la porte du parvis extérieure, et se tiendra près de l'un des poteaux de l'autre porte, les sacrificateurs prépareront son holocauste et ses sacrifices d'offrande de paix ; il se prosternera sur le seuil de cette porte, et ensuite il sortira ; et la porte ne sera point fermée jusqu'au soir.
\VS{3}Tellement que le peuple du pays se prosternera devant Yahweh à l'entrée de cette porte, les jours de sabbat et des nouvelles lunes.
\VS{4}L'holocauste que le prince offrira à Yahweh le jour du sabbat sera de six agneaux sans défaut, et d'un bélier sans défaut.
\VS{5}L'offrande pour le bélier sera d'un épha, et l'offrande pour chacun des agneaux sera selon ce qu'il pourra donner ; mais il y aura un hin d'huile pour chaque épha.
\VS{6}Au jour de la nouvelle lune, son holocauste sera d'un jeune taureau, sans défaut, de six agneaux et d'un bélier, aussi sans défaut.
\VS{7}Son offrande pour le taureau sera d'un épha, pour l'offrande du bélier, un autre épha, et pour chacun des agneaux selon ce qu'il pourra donner ; mais il y aura un hin d'huile pour chaque épha.
\VS{8}Lorsque le prince entrera, il entrera par le chemin du vestibule de la porte, et il sortira par le même chemin.
\VS{9}Quand le peuple du pays entrera pour se présenter devant Yahweh, aux fêtes solennelles, celui qui y entrera par le chemin de la porte nord pour y adorer Yahweh, sortira par le chemin de la porte sud ; et celui qui y entrera par le chemin de la porte sud, sortira par le chemin de la porte nord ; personne ne retournera par le chemin de la porte par laquelle il sera entré, mais il sortira par celle qui lui est opposée.
\VS{10}Alors le prince entrera parmi eux, quand ils entreront ; et quand ils sortiront, ils sortiront ensemble.
\VS{11}Or,dans ces fêtes solennelles et dans ces solennités, l'offrande d'un taureau sera d'un épha, et l'offrande d'un bélier d'un autre épha, l'offrande de chacun des agneaux sera selon ce que le prince pourra donner, et il y aura un hin d'huile pour chaque épha.
\VS{12}Et si le prince offre un sacrifice volontaire, quelque un holocauste, soit quelques un sacrifices d'offrande de paix en offrande à Yahweh, on lui ouvrira la porte qui est du côté oriental, et il offrira son holocauste et ses sacrifices d'offrande de paix comme il les offre le jour du sabbat, puis il sortira, et après qu'il sera sorti, on fermera cette porte.
\VS{13}Tu sacrifieras chaque jour en holocauste à Yahweh un agneau d'un an sans défaut, tu le sacrifieras tous les matins.
\VS{14}Tu lui offriras tous les matins l'offrande, faite de la sixième partie d'un épha, et de la troisième d'un hin d'huile pour pétrir la farine ; c'est l'offrande à Yahweh qu'il faut offrir par ordonnances perpétuelles.
\VS{15}Ainsi on offrira tous les matins en holocauste perpétuel cet agneau et l'offrande avec cette huile.
\VS{16}Ainsi a dit le Seigneur Yahweh : quand le Prince aura fait un don de quelque pièce de son héritage à quelqu'un de ses fils, ce don appartiendra à ses fils ; parce qu'ils ont droit de possession en l'héritage.
\VS{17}Mais s'il fait un don pris de son héritage à l'un de ses serviteurs, le don lui appartiendra bien, mais seulement jusqu'à l'année de la liberté, puis il retournera au prince ; ses fils seuls posséderont ce qu'il leur donnera de son héritage\FTNT{Lé. 25:10.}.
\VS{18}Et le prince ne prendra pas de l'héritage du peuple en les opprimant, les chassant de leur possession : c'est de sa propre possession qu'il fera hériter ses fils, afin qu'aucun de mon peuple ne soit pas dispersé loin de sa possession.
\VS{19}Puis il me mena par l'entrée qui était vers le côté de la porte, aux chambres saintes qui appartenaient aux sacrificateurs, vers le nord. Et voici, il y avait un certain lieu dans le fond du côté occidental.
\VS{20}Il me dit : C'est là le lieu où les sacrificateurs feront bouillir le reste de la bête qu'on aura sacrifiée pour la culpabilité, et le reste de la bête qu'on aura sacrifiée pour l'expiation, et où ils feront cuire les offrandes, afin qu'ils ne les emportent point au parvis extérieur de manière à sanctifier le peuple\FTNT{No. 18:9.}.
\VS{21}Puis il me fit sortir vers le parvis extérieur, et me fit traverser vers les quatre angles du parvis, et voici, il y avait une cour à chacun des angles du parvis.
\VS{22}Aux quatre angles de ce parvis, il y avait des cours voûtées, longues de quarante coudées, et larges de trente ; et toutes les quatre avaient la même mesure dans les angles.
\VS{23}Un mur les entourait toutes les quatre, et des foyers étaient faits au bas du campement tout autour.
\VS{24}Et il me dit : Ce sont ici les cuisines, où ceux qui font le service de la maison cuiront les sacrifices du peuple.
\Chap{47}
\TextTitle{Les eaux pures du sanctuaire\FTNTT{Cp. Za. 14:8-9 ; Ap. 22:1-2.}}
\VerseOne{}Puis il me ramena vers l'entrée de la maison, et voici, des eaux sortaient sous le seuil de la maison, vers l'orient, car la face de la maison était vers l'orient ; et ces eaux-là descendaient du côté droit de la maison, du côté sud de l'autel\FTNT{Ps. 46:5 ; Joë. 3:18 ; Za. 13:1 ; Za. 14:8 ; Ap. 22:1.}.
\VS{2}Puis il me fit sortir par le chemin de la porte nord, et me fit faire le tour par dehors, jusqu'à la porte extérieure, du côté de l'orient, et voici, les eaux coulaient du côté droit.
\VS{3}Quand cet homme s'avança vers l'orient, il avait dans sa main un cordeau ; et il mesura mille coudées, puis il me fit traverser ces eaux-là, et j'avais de l'eau jusqu'aux chevilles.
\VS{4}Puis il mesura mille autres coudées, et me fit traverser les eaux, j'avais de l'eau jusqu'aux genoux ; puis il mesura mille autres coudées, et me fit traverser, et j'avais de l'eau jusqu' aux reins.
\VS{5}Il mesura mille autres coudées ; mais ces eaux-là étaient déjà un torrent que je ne pouvais traverser ; car ces eaux-là étaient si profondes qu'il fallait y traverser à la nage, c'était un torrent que l'on ne pouvait traverser.
\VS{6}Alors il me dit : Fils de l'homme, as-tu vu ? Puis il me fit aller et revenir vers le bord du torrent.
\VS{7}Quand je revins, il y avait un grand nombre d'arbres sur les deux bords du torrent.
\VS{8}Il me dit : Ces eaux couleront vers la Galilée orientale, et elles descendront à la campagne, puis elles entreront dans la mer, et quand elles se seront jetées dans la mer, les eaux deviendront saines.
\VS{9}Il arrivera que tout être vivant qui se meut vivra partout où les deux torrents couleront, et il y aura une grande quantité de poissons ; car là où ces eaux entreront, les eaux deviendront saines, et tout vivra là où ce torrent parviendra.
\VS{10}Il arrivera que des pêcheurs se tiendront le long de cette mer, depuis En-Guédi jusqu'à En-Eglaïm ; on étendra les filets, il y aura des poissons de diverses espèces, comme les poissons de la grande mer, et ils seront très nombreux.
\VS{11}Ses marais et ses fosses ne seront pas assainis, ils seront abandonnés au sel.
\VS{12}Auprès de ce torrent et sur ses deux bords, il croîtra des arbres fruitiers de toutes sortes. Leur feuillage ne se flétrira point, et l'on trouvera toujours du fruit. Tous les mois, ils produiront des fruits mûrs, parce que les eaux de ce torrent sortent du lieu saint, et à cause de cela leur fruit sera bon à manger, et leur feuillage servira de remède\FTNT{Ap. 22:2.}.
\TextTitle{Délimitations du pays\FTNTT{Cp. Ge. 15:18-21.}}
\VS{13}Ainsi parle le Seigneur Yahweh : Voici les frontières du pays que vous aurez en héritage, selon les douze tribus d'Israël. Joseph aura deux portions.
\VS{14}Vous en aurez la possession l'un comme l'autre de ce pays. J'ai levé ma main de le donner à vos pères ; et ce pays-là vous sera donc échu en héritage\FTNT{Ge. 12:7 ; Ge. 17:8.}.
\VS{15}Voici la frontière du pays, du côté nord, depuis la grande mer, le chemin de Hethlon jusqu'à Tsedad,
\VS{16}Hamath, Bérotha, et Sibraïm, entre la frontière de Damas et la frontière de Hamath, Hatzer-Hatthicon, vers la frontière de Havran.
\VS{17}La frontière sera depuis la mer, Hatsar-Enon ; la frontière de Damas, Tsaphon au nord et la frontière de Hamath : Ce sera le côté nord.
\VS{18}Le côté oriental sera le Jourdain entre Havran, Damas et Galaad, et le pays d'Israël ; vous mesurerez depuis la frontière nord jusqu'à la mer orientale : Ce sera le côté oriental.
\VS{19}Le côté méridional, au midi, ira depuis Thamar jusqu'aux eaux de Meriba à Kadès, jusqu'au torrent vers la grande mer : Ce sera le côté méridional.
\VS{20}Le côté occidental sera la grande mer, depuis la frontière jusque vis-à-vis de Hamath : Ce sera le côté occidental.
\VS{21}Vous partagerez ce pays entre vous, selon les tribus d'Israël.
\VS{22}Vous le diviserez en héritage par le sort pour vous et pour les étrangers qui séjourneront au milieu de vous, qui engendreront des fils au milieu de vous ; vous les regarderez comme natifs des fils d'Israël ; ils partageront au sort l'héritage avec vous parmi les tribus d'Israël.
\VS{23}Vous donnerez à l'étranger son héritage dans la tribu où il séjournera, dit le Seigneur Yahweh.
\Chap{48}
\TextTitle{Héritage de sept tribus\FTNTT{Cp. Jos. 13:1-19:51.}}
\VerseOne{}Voici les noms des tribus. Depuis l'extrémité qui regarde vers le nord, le long de la contrée du chemin de Hethlon du quartier par lequel on entre à Hamath, jusqu'à Hatsar-Enon, qui est la frontière de Damas, du côté qui regarde vers le nord, le long de la contrée de Hamath, tellement que cette extrémité est le canton de l'orient et celui de l'occident: il y aura une portion pour Dan.
\VS{2}Et tout joignant les frontières de Dan, depuis le canton de l'orient jusqu'au canton qui regarde vers l'occident : il y aura une partion pour Aser.
\VS{3}Et tout joignant les frontières d'Aser, depuis le canton qui regarde vers l'orient jusqu'au canton qui regarde vers l'occident : il y aura une partion pour Nephthali.
\VS{4}Et tout les frontières de Nephthali, depuis le canton qui regarde vers l'orient jusqu'au canton qui regarde vers  l'occident : il y aura une partion pour Manassé. 
\VS{5}Et tout joignant les frontières de Manassé, depuis le canton qui regarde vers de l'occident jusqu'au canton qui regarde vers  l'orient : il y aura une part pour Ephraïm. 
\VS{6}Et tout joignant les frontières d'Ephraïm, encore depuis le canton de l'orient,jusqu'au canton qui regarde vers l'occident: il y aura une partion pour Ruben. 
\VS{7}Et joignant les frontièrs de Ruben, depuis le canton de l'orient jusqu'au canton qui regarde vers l'occident : il y aura une partion pour Juda.
\VS{8}Et tout le long des frontières de Juda, depuis le canton de l'orient, jusqu'au canton qui regarde vers l'occident; il y aura une portion que vous prélèverez sur toute la masse du pays, comme une offrande élevée et elle aura vingt-cinq mille cannes de largueur et de longueur, autant que l'une des autres partions depuis le canton qui regarde vers l'orient jusqu'au canton qui regarde vers l'occident; de sorte que le sanctuaire sera au milieu.
\VS{9}La portion que vous lèverez pour Yahweh et offerte en offrande élevée, aura vingt-cinq mille cannes de longueur, et dix mille de largeur.
\TextTitle{Territoire réservé aux sacrificateurs et aux Lévites}
\VS{10}Et cette portion sainte sera pour les sacrificateurs, vingt-cinq mille cannes de longueur au nord, et dix mille de largeur, à l'occident, dix mille en largeur à l'orient, et vingt-cinq mille en longueur au sud, et le sanctuaire de Yahweh sera au milieu.
\VS{11}Elle sera pour les sacrificateurs, et quiconque aura été sanctifié d'entre les fils de Tsadok, qui ont fait ce que j'ai ordonné, et qui ne se sont point égarés quand les fils d'Israël se sont égarés, comme se sont égarés les autres Lévites,
\VS{12}Ceux-là auront une portion ainsi levée sur l'autre très sainte, prélevée sur la portion du pays qui aura été prélevée, à côté de la frontière des Lévites.
\VS{13}Les Lévites auront, parallèlement à la frontière des sacrificateurs, vingt-cinq mille cannes en longueur et dix mille de largeur, vingt-cinq mille pour toute la longueur et dix mille pour toute la largeur.
\VS{14}Ils n'en pourront ni vendre, ni échanger, et les prémices du pays ne seront point transgressées car elles sont mises à part pour Yahweh.
\VS{15}Les cinq mille cannes qui resteront en largeur sur les vingt-cinq mille cannes seront destinées à la ville, pour les habitations et le faubourg, et la ville sera au milieu.
\VS{16}En voici les mesures : Du côté nord, quatre mille cinq cents cannes, du côté sud, quatre mille cinq cents, du côté oriental, quatre mille cinq cents, et du côté occidental, quatre mille cinq cents.
\VS{17}Puis il y aura des faubourgs pour la ville, vers le nord. La ville aura un faubourg au nord de deux cent cinquante cannes, de deux cent cinquante au sud, de deux cent cinquante à l'orient, et de deux cent cinquante à l'occident.
\VS{18}Quant à ce qui sera de reste sur la longueur et qui sera tout joignant à la portion sanctifiée, et qui aura dix mille cannes à l'orient, et dix mille autres cannes à l'occident, parallèlement à la portion sanctifiée, le revenu qu'on en tirera sera pour nourriture de ceux qui feront le service qu'il faut dans la ville.
\VS{19}Le sol sera travaillé par ceux de toutes les tribus d'Israël qui travailleront pour la ville.
\VS{20}Toute la portion prélevée sera de vingt-cinq mille cannes en longueur sur vingt-cinq mille en largeur ; vous en séparerez un carré pour la propriété de la ville.
\VS{21}PUis le reste sera pour le prince aux deux côtés de la portion sainte et de la possession de la ville, des vingt-cinq mille coudées de la portion prélevée jusqu'à la frontière de l'orient, et à l'occident, des vingt-cinq mille coudées jusqu'à la frontière de l'occident, le long des parts, pour le prince, et la portion sainte et le sanctuaire de la Maison seront au milieu de tout le pays. 
\VS{22}Ce qui sera donc pour le prince sera l'espace compris depuis la possession des Lévites, et depuis la possession de la ville ; ce qui sera entre ces possessions là et la frontière de Juda, et la frontière de Benjamin, sera pour le prince.
\TextTitle{Héritage de cinq tribus}
\VS{23}Or ce qui sera de reste sera pour les autres tribus; depuis le canton de ce qui regarde vers l'orient, jusqu'au canton de ce qui regarde vers l'occident, il y aura une portion pour Benjamin.
\VS{24}Puis tout joignant les frontières de Benjamin, depuis le canton de ce qui regarde vers l'orient, jusqu'au canton de ce qui regarde vers l'occident, il y aura une autre portion pour Siméon.
\VS{25}Puis tout joignant les frontières de Siméon, depuis le canton de ce qui regarde vers l'orient, jusqu'au canton de ce qui regarde vers l'occident, il y aura une autre portion pour Issacar.
\VS{26}Puis tout joignant sur les frontières d'Issacar, depuis le canton de ce qui regarde vers l'orient, jusqu'au canton de ce qui regarde vers l'occident, il y aura une autre portion pour Zabulon.
\VS{27}Puis joignant les frontières de Zabulon, depuis le canton de ce qui regarde vers l'orient, jusqu'au canton de ce qui regarde vers l'occident, il y aura une autre portion pour Gad.
\VS{28}Or ce qui appartient au côté du midi qui regarde proprement le vent d'autant, et sur la frontière de Gad, et cette frontière sera depuis Thamar jusqu'aux eaux de contestation, à Kadès, le long du torrent jusqu'à la grande mer.
\VS{29}C'est là le pays que vous partagerez par le sort en héritage aux tribus d'Israël, et ce sont là leurs portions, dit le Seigneur Yahweh.
\VS{30}Et ce sont ici les sorties de la ville : Du côté du nord, il y aura quatre mille cinq cents mesures.
\VS{31}Et les portes de la ville seront selon les noms des tribus d'Israël : Trois portes vers le nord, une porte de Ruben, une porte de Juda, une porte de Lévi.
\VS{32}Et du côté de l'orient, quatre mille cinq cents mesures, et trois portes : Une porte de Joseph, une porte de Benjamin, une porte de Dan.
\VS{33}Et du côté sud, quatre mille cinq cents mesures, et trois portes : Une porte de Simeon, une porte d'Issacar, une porte de Zabulon.
\VS{34}Et du côté ouest, quatre mille cinq cents mesures, avec leurs trois portes : Une porte de Gad, une porte d'Aser, une porte de Nephthali.
\VS{35}Ainsi le circuit de la ville sera de dix-huit mille mesures ; et le nom de la ville depuis ce jour-là sera : Yahweh est ici.
\PPE{}
\end{multicols}

%\clearpage\ShortTitle{Osée}\BookTitle{Osée}\BFont
\noindent\hrulefill
{\footnotesize
\textit{
\bigskip
{\centering{}
\\Auteur : Osée
\\(Heb. : Hoswhéa')
\\Signification : Salut, Sauve
\\Thème : Israël sera rejeté à cause de son apostasie. D'autres nations seront appelées à sa place.
\\Date de rédaction : 8\up{ème} siècle av. J.-C.\\}
}
%\bigskip
\textit{
\\Osée, fils de Béeri, exerça son service dans le royaume du nord au temps de Joas, roi d'Israël. Il était contemporain des prophètes Amos, Michée et Esaïe.
%\bigskip
\\Yahweh demanda à Osée d'épouser une prostituée pour que le prophète puisse partager plus profondément son fardeau
et la tristesse qu'il subissait en raison de l'infidélité du peuple qu'il aimait tant. Malgré la rébellion et les mauvais
agissements d'Israël, Yahweh manifesta une fois de plus sa patience et utilisa Osée pour avertir et inviter les fils de Jacob à la repentance.\bigskip
}
}
\par\nobreak\noindent\hrulefill
\begin{multicols}{2}
\Chap{1}
\VerseOne{}La parole de Yahweh qui fut adressée à Osée fils de Béeri, au temps d'Ozias, de Jotham, d'Achaz et d'Ezéchias, rois de Juda, et au temps de Jéroboam, fils de Joas, roi d'Israël.
\TextTitle{Mariage d'Osée et naissance de Jizreel}
\VS{2}La première fois que Yahweh parla à Osée, Yahweh dit à Osée : Va, prends une femme prostituée, et aie d'elle des enfants de prostitution ; car le pays s'est entièrement prostitué en abandonnant Yahweh !
\VS{3}Il alla, et il prit Gomer, fille de Diblaïm. Elle conçut, et lui enfanta un fils.
\VS{4}Et Yahweh lui dit : Donne-lui le nom de Jizreel ; car encore un peu de temps, et je punirai la maison de Jéhu pour le sang versé à Jizreel, et je ferai cesser le royaume de la maison d'Israël\FTNT{Cette prophétie s'est accomplie en 722 av. J-C. Voir 2 R.17.}.
\VS{5}Et il arrivera qu'en ce jour-là, je briserai l'arc d'Israël dans la vallée de Jizreel.
\TextTitle{Naissance de Lo-Ruchama}
\VS{6}Elle conçut encore, et enfanta une fille. Et Yahweh lui dit : Donne-lui le nom de Lo-Ruchama\FTNT{Tout au long de ce livre, certains mots sont symboliquement utilisées pour nommer Israël. A savoir : 
\\- « ruchama » : miséricorde ;
\\- « Lo-Ruchama » : celle dont on ne fait pas miséricorde ; 
\\- « ammi » : mon peuple ;
\\- « Lo-Ammi » : pas mon peuple.
\\Ces noms illustrent ainsi l'infidélité d'Israël envers Yahweh.} ; car je ne continuerai plus à faire miséricorde à la maison d'Israël, mais je les enlèverai entièrement.
\VS{7}Mais je ferai miséricorde à la maison de Juda; et je les délivrerai par Yahweh, leur Dieu, et je ne les délivrerai ni par l'arc, ni par l'épée, ni par les combats, ni par les chevaux, ni par les cavaliers.
\TextTitle{Naissance de Lo-Ammi}
\VS{8}Puis quand elle sevra Lo-Ruchama ; elle conçut, et enfanta un fils.
\VS{9}Et Yahweh dit : Appelle-le du nom de Lo-Ammi ; car vous n'êtes point mon peuple, et je ne suis pas votre Dieu.
\Chap{2}
\TextTitle{Futur rétablissement d'Israël}
\VerseOne{}Cependant le nombre des fils d'Israël sera comme le sable de la mer, qui ne peut ni se mesurer ni se compter ; et dans la ville où il leur est dit : Vous n'êtes pas mon peuple ! On leur dira : Vous êtes les fils du Dieu vivant !
\VS{2}Aussi les fils de Juda et les fils d'Israël se rassembleront, et ils s'établiront un chef, et monteront hors du pays ; car la journée de Jizreel sera grande.
\VS{3}Dites à vos frères : Ammi ! Et à vos sœurs : Ruchama !
\TextTitle{Châtiment d'Israël, la prostituée\FTNTT{2 R. 17:1-18.}}
\VS{4}Plaidez, plaidez contre votre mère, car elle n'est point ma femme, et je ne suis point son mari ! Qu'elle ôte ses prostitutions de son visage, et ses adultères de son sein !
\VS{5}De peur que je ne la dépouille à nu, et que je ne l'expose comme au jour de sa naissance, et que je ne la rende semblable à un désert, à une terre aride, et ne la fasse mourir de soif ;
\VS{6}et je n'aurai point de miséricorde pour ses enfants, car ce sont des enfants de prostitution.
\VS{7}Car leur mère s'est prostituée, celle qui les a conçus s'est déshonorée, car elle a dit : Je m'en irai après mes amants, qui me donnent mon pain et mes eaux, ma laine et mon lin, mon huile, et mes boissons.
\VS{8}C'est pourquoi voici, je vais fermer ton chemin avec des épines, j'y élèverai un mur, afin qu'elle ne trouve plus ses sentiers.
\VS{9}Elle poursuivra ses amants, mais ne les atteindra pas ; elle les cherchera, mais elle ne les trouvera point. Puis elle dira : Je m'en irai, et je retournerai vers mon premier mari, car alors j'étais plus heureuse que maintenant.
\VS{10}Mais elle n'a pas reconnu que c'était moi qui lui donnais le blé, le vin et l'huile ; et l'on a fait des offrandes à Baal\FTNT{Baal : Voir commentaire en Jg. 2:13.} avec l'argent et l'or que je lui prodiguais.
\VS{11}C'est pourquoi je reprendrai mon froment en son temps, mon vin en sa saison, et je retirerai ma laine et mon lin qui couvraient sa nudité.
\VS{12}Et maintenant je découvrirai sa honte aux yeux de ses amants, et personne ne la délivrera de ma main.
\VS{13}Je ferai cesser toute sa joie, ses fêtes, ses nouvelles lunes, ses sabbats, et toutes ses solennités.
\VS{14}Je ravagerai ses vignes et ses figuiers, dont elle disait : Voici le salaire que mes amants m'ont donné ! Je les réduirai en une forêt, et les bêtes des champs les dévoreront.
\VS{15}Je la punirai pour les jours où elle encensait les Baals, où elle se parait de ses anneaux et de ses colliers, et s'en allait après ses amants, et m'oubliait, dit Yahweh.
\TextTitle{La femme adultère revient dans son foyer : Israël revient à Yahweh}
\VS{16}Néanmoins, voici, je veux l'attirer et la mener au désert, là je parlerai à son cœur.
\VS{17}Là, je lui accorderai ses vignes et la vallée d'Acor, telle une porte d'espérance, et là, elle chantera comme au temps de sa jeunesse, et comme au jour où elle remonta du pays d'Egypte.
\VS{18} Et il arrivera en ce jour-là, dit Yahweh, tu m'appelleras: Mon Mari ! Et tu ne m'appelleras plus: Mon Maître\FTNT{Littéralement : Baal.} !
\VS{19}Car j'ôterai de sa bouche les noms des Baals, et on ne fera plus mention de leurs noms.
\VS{20}Aussi en ce temps-là, je traiterai pour eux une alliance avec les bêtes des champs, avec les oiseaux du ciel, et avec les reptiles de la terre ; je briserai et j'ôterai du pays l'arc, l'épée et la guerre, et je les ferai se reposer en sécurité.
\VS{21}Et je te fiancerai pour moi à toujours ; je te fiancerai, dis-je, pour moi, par la justice, la droiture, la grâce et la miséricorde;
\VS{22}je serai ton fiancé par la fidélité, et tu reconnaîtras Yahweh.
\VS{23}Et il arrivera en ce jour-là, j'exaucerai, dit Yahweh, je témoignerai aux cieux, et les cieux exauceront la terre;
\VS{24}la terre exaucera le blé, le bon vin et l'huile, et ils exauceront Jizreel. 
\VS{25}Puis, je la sèmerai pour moi dans ce pays, et je ferai miséricorde à Lo-Ruchama ; je dirai à Lo-Ammi : Tu es mon peuple ! Et il me répondra : Mon Dieu !
\Chap{3}
\TextTitle{Soumission d'Israël à Yahweh}
\VerseOne{}Après cela Yahweh me dit : Va encore, et aime une femme aimée d'un ami, et adultère; aime-la comme Yahweh aime les enfants d'Israël, qui se tournent toutefois vers d'autres dieux et aiment les gâteaux de raisins.
\VS{2}J'achetai donc cette femme pour quinze pièces d'argent, un homer et demi d'orge\FTNT{Voir dans les annexes les tableaux des poids et mesures.}.
\VS{3}Et je lui dis : Assieds-toi avec moi pendant plusieurs jours, ne t'abandonne plus à la prostitution, ne sois à aucun homme, et je serai fidèle envers toi.
\VS{4}Car les enfants d'Israël demeureront plusieurs jours sans roi, sans chef, sans sacrifice, sans statue, sans éphod, et sans théraphim\FTNT{Cette prophétie s'est accomplie d'une manière extraordinaire à travers l'histoire du peuple d'Israël depuis la première venue de Jésus-Christ. Les Israélites étaient dispersés, sans unité politique faute de roi, empêchés d'offrir des sacrifices depuis la destruction du temple par Titus (39-81), fils de l'empereur romain Vespasien (9-79), en l'an 70.}.
\VS{5}Mais après cela, les enfants d'Israël se repentiront\FTNT{La repentance et la conversion nationale d'Israël auront lieu lors du retour du Messie (Es. 59:20-21 ; Ro. 11:26-27). Dieu n'a pas abandonné son peuple, il viendra lui-même le délivrer.}; et rechercheront Yahweh, leur Dieu, et David, leur roi; ils seront dans la crainte à la vue de Yahweh et de sa bonté, dans les derniers jours\FTNT{Les derniers jours : Voir commentaire en Ge. 49:1.}.
\Chap{4}
\TextTitle{Israël, la nation pécheresse}
\VerseOne{}Ecoutez la parole de Yahweh, fils d'Israël ! Car Yahweh a un procès avec les habitants du pays, parce qu'il n'y a ni de vérité, ni de miséricorde, ni de connaissance de Dieu dans le pays.
\VS{2}Il n'y a que parjures et mensonges, meurtres, vols et adultères ; on use de violence, et un meurtre touche l'autre.
\VS{3}C'est pourquoi le pays sera dans le deuil, et tous ceux qui l'habitent seront languissants, et avec eux toutes les bêtes des champs et tous les oiseaux du ciel ; même les poissons de la mer périront.
\VS{4}Mais que nul ne conteste, et que nul ne reprenne ; car ton peuple est comme ceux qui disputent avec le sacrificateur.
\VS{5}Tu tomberas donc en plein jour, et le prophète aussi tombera avec toi de nuit, et j'exterminerai ta mère.
\TextTitle{Israël dans l'ignorance}
\VS{6}Mon peuple est détruit, parce qu'il lui manque la connaissance\FTNT{Le verbe « détruire » vient de l'hébreu « damah » qui signifie aussi « égorger ». Satan est celui qui vient égorger, dérober et détruire, notamment avec ses faux prophètes (Jn. 10:10). Chaque disciple de Jésus-Christ doit avoir une vie de prière et de méditation quotidienne afin de résister aux attaques de l'ennemi.}. Parce que tu as rejeté la connaissance, je te rejetterai, afin que tu n'exerces plus la sacrificature; puisque tu as oublié la loi de ton Dieu, moi aussi j'oublierai tes enfants.
\VS{7}Plus ils se sont multipliés, plus ils ont péché contre moi : Je changerai leur gloire en ignominie.
\VS{8}Ils se nourrissent des péchés de mon peuple, leur âme soutient leur iniquité.
\VS{9}C'est pourquoi le sacrificateur sera traité comme le peuple; je le châtierai selon ses voies, et je lui rendrai selon ses œuvres.
\VS{10}Et ils mangeront mais ils ne seront point rassasiés, ils se prostitueront mais ils ne multiplieront point, parce qu'ils ont cessé de prendre garde à Yahweh.
\VS{11}La prostitution, le vin et le moût, font perdre l'entendement.
\TextTitle{Israël dans l'idolâtrie}
\VS{12}Mon peuple consulte son bois, et c'est son bâton qui lui répond ; car l'esprit de prostitution égare, et ils se prostituent loin de leur Dieu.
\VS{13}Ils sacrifient sur le sommet des montagnes, ils brûlent de l'encens sur les collines, sous les chênes, sous les peupliers, et les térébinthes, parce que leur ombrage est agréable. C'est pourquoi vos filles se prostituent, et vos belles-filles commettent l'adultère.
\VS{14}Je ne punirai pas vos filles parce qu'elles se prostituent, ni vos belles-filles parce qu'elles commettent l'adultère, car eux-mêmes se retirent avec des prostituées, et sacrifient avec des femmes débauchées. Ainsi le peuple qui est sans intelligence sera ruiné.
\VS{15}Si tu te prostitues, ô Israël, au moins que Juda ne se rende point coupable ! N'entrez donc point dans Guilgal, et ne montez pas à Beth-Aven, et ne jurez point : Yahweh est vivant !
\VS{16}Parce qu'Israël se révolte comme une vache indomptable, maintenant Yahweh les fera paître comme des agneaux dans de vastes plaines.
\VS{17}Ephraïm s'est associé aux idoles ; abandonne-le !
\VS{18}Leur breuvage est devenu aigre ; ils n'ont fait que se prostituer ; ils n'aiment qu'à dire : apportez ; ce n'est qu'ignominie que ses protecteurs.
\VS{19}Le vent l'a enfermé dans ses ailes, et ils auront honte de leurs sacrifices.
\Chap{5}
\TextTitle{Yahweh abandonne son peuple}
\VerseOne{}Ecoutez ceci, sacrificateurs ! Maison d'Israël, sois attentive ! Maison du roi, tendez l'oreille ! Car c'est à vous que s'adresse le jugement, parce que vous avez été un piège à Mitspa, et un filet tendu sur le Thabor.
\VS{2}Les infidèles s'enfoncent dans le crime. Et moi, je les châtierai tous.
\VS{3}Je connais Ephraïm, et Israël ne m'est point caché; car maintenant, Ephraïm, tu t'es prostitué, et Israël est souillé.
\VS{4}Leurs œuvres ne leur permettent pas de revenir à leur Dieu, parce que l'esprit de prostitution est au milieu d'eux, et parce qu'ils ne connaissent point Yahweh.
\VS{5}L'orgueil d'Israël témoigne contre lui; Israël et Ephraïm tomberont par leur iniquité ; Juda aussi tombera avec eux.
\VS{6}Ils iront avec leurs brebis et leurs bœufs chercher Yahweh, mais ils ne le trouveront point, il s'est retiré du milieu d'eux.
\VS{7}Ils se sont montrés infidèles envers Yahweh, car ils ont engendré des fils étrangers ; maintenant un mois suffira pour les dévorer avec leurs biens.
\VS{8}Sonnez du shofar à Guibea. Sonnez de la trompette à Rama ! Poussez des cris de guerre à Beth-Aven ! Derrière toi, Benjamin !
\VS{9}Ephraïm sera un sujet d'épouvante au jour du châtiment ; je le fais savoir parmi les tribus d'Israël comme une chose certaine.
\VS{10}Les chefs de Juda sont comme ceux qui déplacent les bornes; je répandrai sur eux ma fureur comme un torrent.
\VS{11}Ephraïm est opprimé, brisé par le jugement, car il a vécu selon les préceptes qui lui plaisaient.
\VS{12}Je serai comme une teigne pour Ephraïm, comme de la pourriture pour la maison de Juda.
\VS{13}Ephraïm voit sa maladie, et Juda ses plaies ; Ephraïm s'en est allé vers le roi d'Assyrie, et s'est adressé au roi Jareb. Mais ce roi ne pourra ni vous guérir, ni panser vos plaies.
\VS{14}Je serai comme un lion pour Ephraïm, comme un lionceau pour la maison de Juda. Moi, moi je déchirerai, puis je m'en irai, j'emporterai la proie, et nul ne me l'enlèvera.
\TextTitle{Israël revient à Yahweh}
\VS{15}Je m'en irai, je reviendrai dans ma demeure, jusqu'à ce qu'ils se reconnaissent coupables, et qu'ils cherchent ma face. Quand ils seront dans la détresse, dans leur angoisse, ils me chercheront.
\Chap{6}
\VerseOne{}Venez, retournons à Yahweh ! Car il a déchiré, mais il nous guérira ; il a frappé, mais il bandera nos plaies.
\VS{2}Il nous rendra la vie dans deux jours; et le troisième jour il nous relèvera, et nous vivrons en sa présence.
\VS{3}Car nous connaîtrons Yahweh, et nous continuerons à le connaître ; sa venue\FTNT{Il est question ici du retour du Seigneur Jésus-Christ : Voir  commentaire en Za. 14:1.} est aussi certaine que celle de l'aurore. Il viendra pour nous comme la pluie, comme la pluie de l'arrière-saison\FTNT{Voir commentaire en Joë. 2:23.} qui arrose la terre.
\TextTitle{Yahweh dénonce le péché d'Ephraïm}
\VS{4}Que te ferai-je, Ephraïm ? Que te ferai-je, Juda ? Votre piété est comme la nuée du matin, comme la rosée qui se dissipe dès le matin.
\VS{5}C'est pourquoi je les taillerai en pièces par mes prophètes, je les tuerai par les paroles de ma bouche\FTNT{Hé. 4:12 ; Ap. 1:16 ; Ap. 19:15.}, et mes jugements apporteront la lumière.
\VS{6}Car je prends plaisir à la miséricorde et non aux sacrifices, et à la connaissance de Dieu plus qu'aux holocaustes.
\VS{7}Mais ils ont transgressé l'alliance, comme si elle avait été d'un homme, en quoi ils se sont portés perfidement contre moi.
\VS{8}Galaad est une ville d'ouvriers d'iniquité, couverte de traces de sang.
\VS{9}Et comme les bandes des voleurs attendent quelqu'un, ainsi les sacrificateurs, après avoir comploté, tuent les gens sur le chemin du côté de Sichem ; car ils exécutent leurs méchants desseins.
\VS{10}J'ai vu des choses infâmes dans la maison d'Israël: Là Ephraïm se prostitue, Israël en est souillé.
\VS{11}A toi aussi Juda, une moisson est préparée, quand je ramènerai les captifs de mon peuple.
\Chap{7}
\TextTitle{Transgression d'Ephraïm}
\VerseOne{}Lorsque je guérissais Israël, l'iniquité d'Ephraïm et la méchanceté de Samarie se sont révélées, car ils ont agi frauduleusement ; le voleur vient tandis que la bande dépouille au-dehors.
\VS{2}Ils n'ont point pensé dans leur cœur que je me souviens de toute leur méchanceté ; maintenant leurs œuvres les entourent, elles sont devant ma face.
\VS{3}Ils réjouissent le roi par leur méchanceté, et les chefs par leurs mensonges.
\VS{4}Ils sont tous adultères, comme un four allumé par le boulanger : Il cesse d'attiser le feu depuis qu'il a pétri la pâte jusqu'à ce qu'elle soit levée.
\VS{5}Au jour de notre roi, les chefs se rendent malades par les excès de vin ; il tend la main aux moqueurs.
\VS{6}Lorsqu'ils dressent des embuscades, leur cœur s'embrase comme un four ; leur boulanger dort toute la nuit, le matin le four est embrasé comme un feu accompagné de flammes.
\VS{7}Ils sont tous ardents comme un four, et ils dévorent leurs chefs ; tous leurs rois tombent, et il n'y a aucun d'entre eux qui crie à moi.
\VS{8}Ephraïm se mêle avec les peuples, Ephraïm est un gâteau qui n'a pas été retourné.
\VS{9}Les étrangers ont dévoré sa force, et il ne s'en doute pas ; les cheveux gris sont aussi parsemés sur lui, et il ne s'en doute pas.
\VS{10}L'orgueil d'Israël rendra témoignage contre lui ; car ils ne reviennent pas à Yahweh, leur Dieu, et ils ne le recherchent pas malgré tout cela. 
\VS{11}Ephraïm est comme une colombe troublée, sans intelligence ; car ils appellent l'Egypte, et s'en vont vers le roi d'Assyrie.
\VS{12}Quand ils s'en iront, j'étendrai mon filet sur eux, et je les précipiterai comme les oiseaux du ciel ; je les châtierai, comme ils en ont été avertis au sein de leurs assemblées.
\VS{13}Malheur à eux, parce qu'ils me fuient ! Ruine sur eux, car ils se révoltent contre moi ! Je voudrais les sauver, mais ils profèrent contre moi des paroles mensongères.
\VS{14}Ils ne crient pas vers moi dans leur cœur, mais ils gémissent sur leurs couches ; ils se rassemblent pour le froment et le bon vin, et ils s'éloignent de moi.
\VS{15}Je les ai châtiés, et j'ai fortifié leurs bras, mais ils méditent le mal contre moi.
\VS{16}Ce n'est pas au Très-Haut qu'ils retournent ; ils sont comme un arc trompeur. Leurs chefs tomberont par l'épée, à cause de l'insolence de leur langue. C'est ce qui en fera un objet de moquerie dans le pays d'Egypte.
\Chap{8}
\TextTitle{Conséquences de la désobéissance}
\VerseOne{}Crie comme si tu avais un shofar dans ta bouche ! Il vient comme un aigle contre la maison de Yahweh, parce qu'ils ont transgressé mon alliance, et qu'ils ont agi méchamment contre ma loi.
\VS{2}Ils crieront à moi : Mon Dieu, nous te connaissons, dira Israël !
\VS{3}Israël a rejeté le bien ; l'ennemi le poursuivra.
\VS{4}Ils ont fait régner, mais non pas de ma part, ils ont établi des chefs, et je n'en ai rien su ; ils se sont fait des idoles avec leur argent et leur or ; c'est pourquoi ils seront retranchés.
\VS{5}Samarie, ton veau t'a chassée loin ! Ma colère s'est embrasée contre eux. Jusqu'à quand ne pourront-ils pas s'adonner à l'innocence ?
\VS{6}Car il vient d'Israël, c'est un orfèvre qui l'a fait, et il n'est pas Dieu ; c'est pourquoi le veau de Samarie sera mis en pièces.
\VS{7}Parce qu'ils ont semé du vent, ils moissonneront la tempête ; ils n'auront pas un épi de blé ; le grain qui poussera ne donnera point de farine, et s'il en faisait, les étrangers la dévoreraient.
\VS{8}Israël est dévoré ! Il est maintenant parmi les nations comme un vase dont on ne se soucie pas.
\VS{9}Car ils sont montés vers le roi d'Assyrie, qui est un âne sauvage se tenant seul à part ; Ephraïm a fait des présents à ceux qui l'aimait.
\VS{10}Et parce qu'ils ont fait des présents aux nations, je les rassemblerai maintenant ; et ils commenceront à être amoindris à cause de l'impôt pour le roi des princes.
\VS{11}Parce qu'Ephraïm a fait plusieurs autels pour pécher, ils auront des autels pour pécher.
\VS{12}Je lui ai écrit les grandes choses de ma loi, mais elles sont estimées comme des lois étrangères.
\VS{13}Quant aux sacrifices qui me sont offerts, ils sacrifient de la chair, et la mangent ; mais Yahweh ne les accepte point. Et maintenant il se souviendra de leur iniquité, et punira leurs péchés ; ils retourneront en Egypte.
\VS{14}Israël a oublié celui qui l'a fait, et il a bâti des palais ; et Juda a multiplié les villes fortes ; c'est pourquoi j'enverrai le feu dans les villes de celui-ci, quand il aura dévoré les palais de celui-là.
\Chap{9}
\TextTitle{Ephraïm châtié et rejeté}
\VerseOne{}Israël, ne te réjouis point, ne sois pas dans l'allégresse, comme les autres peuples, de ce que tu t'es prostitué en abandonnant ton Dieu, de ce que tu as obtenu un salaire de tes amants dans toutes les aires à blé !
\VS{2}L'aire et la cuve ne les nourriront pas, et le vin doux les trompera.
\VS{3}Ils ne resteront pas dans le pays de Yahweh; Ephraïm retournera en Egypte, et ils mangeront en Assyrie ce qui est impur.
\VS{4}Ils ne feront pas d'aspersions de vin à Yahweh : Elles ne lui seraient point agréables. Leurs sacrifices seront pour eux comme le pain de deuil ; tous ceux qui en mangeront se rendront impurs ; car leur pain ne sera que pour eux, il n'entrera point dans la maison de Yahweh.
\VS{5}Que ferez-vous aux jours des fêtes solennelles, aux jours des fêtes de Yahweh ?
\VS{6}Car voici, ils partent à cause de la dévastation ; l'Egypte les recueillera, Moph les enterrera ; ce qu'ils ont de précieux, leur argent, sera la proie des ronces, et l'épine sera dans leurs tentes.
\VS{7}Les jours du châtiment sont venus, les jours de la rétribution sont venus, et Israël le saura ! Les prophètes sont fous, les hommes de révélation sont insensés, à cause de la grandeur de ton iniquité, et de ta grande aversion.
\VS{8}Ephraïm est une sentinelle avec mon Dieu ; mais le prophète est un filet d'oiseleur sur toutes ses voies, en rébellion contre la maison de son Dieu.
\VS{9}Ils se sont profondément corrompus, comme aux jours de Guibea ; Yahweh se souviendra de leur iniquité, il punira leurs péchés.
\VS{10}J'ai, dira-t-il, trouvé Israël comme des raisins dans le désert ; j'ai vu vos pères comme les premiers fruits d'un figuier ; mais ils sont allés vers Baal-Peor\FTNT{Baal-Peor, « seigneur de la brèche », était une divinité adorée à Peor avec des rites licencieux (No. 23:28 ; No. 25:1-3 ; Ps. 106:28-29).}, ils se sont consacrés à l'infâme idole, et ils sont devenus abominables comme ce qu'ils ont aimé.
\VS{11}La gloire d'Ephraïm s'envolera comme un oiseau : Point d'enfantement, point de grossesse, point de conception.
\VS{12}Que s'ils élèvent leurs enfants, je les en priverai tellement, que pas un d'entre eux ne deviendra homme ; malheur à eux, quand je me retirerai d'eux !
\VS{13}Ephraïm était comme j'ai vu Tyr, plantée dans un lieu agréable ; mais Ephraïm mènera ses fils à celui qui les tuera.
\VS{14}Ô Yahweh, donne-leur ! Mais que leur donnerais-tu ? Donne-leur un sein qui avorte et des mamelles desséchées.
\VS{15}Toute leur méchanceté s'est manifestée à Guilgal ; c'est là que je les ai pris en aversion. Je les chasserai de ma maison à cause de la malice de leurs actions. Je ne les aimerai plus ; tous leurs chefs sont des rebelles.
\VS{16}Ephraïm est frappé, sa racine est devenue sèche ; ils ne porteront plus de fruit ; et s'ils engendrent des enfants, je mettrai à mort les fruits désirables de leur ventre.
\VS{17}Mon Dieu les rejettera, parce qu'ils ne l'ont point écouté, et ils seront vagabonds parmi les nations.
\Chap{10}
\TextTitle{Yahweh annonce la destruction du royaume d'Israël}
\VerseOne{}Israël était une vigne dévastée, elle ne fait de fruit que pour elle-même. Selon l'abondance de son fruit, il a multiplié les autels ; selon la beauté de son pays, il a rendu belles ses statues.
\VS{2}Leur cœur est partagé. Ils vont être déclarés coupables. Yahweh renversera leurs autels, il détruira leurs statues.
\VS{3}Car bientôt ils diront : Nous n'avons point de roi, parce que nous n'avons point craint Yahweh ; et le roi, que pourrait-il faire pour nous ?
\VS{4}Ils prononcent des paroles vaines, des faux serments, lorsqu'ils concluent une alliance. C'est pourquoi le châtiment germera dans les sillons des champs, comme une plante vénéneuse.
\VS{5}Les habitants de Samarie seront épouvantés à cause des jeunes vaches de Beth-Aven; car le peuple mènera deuil sur son idole ; et les prêtres de ses idoles, qui s'en étaient réjouis, mèneront deuil parce que sa gloire est transportée loin d'elle.
\VS{6}Elle sera transportée en Assyrie, pour en faire un présent au roi Jareb. Ephraïm sera dans la confusion, et Israël aura honte de ses desseins.
\VS{7}C'en est fait de Samarie, et de son roi, qui sera retranché comme l'écume qui est à la surface des eaux.
\VS{8}Les hauts lieux de Beth-Aven, qui sont le péché d'Israël, seront détruits ; l'épine et la ronce croîtront sur leurs autels. Et on dira aux montagnes : Couvrez-nous ! Et aux collines : Tombez sur nous !
\VS{9}Israël, tu as péché dès les jours de Guibea ! Là ils restèrent debout, la guerre contre les fils d'iniquité ne les atteignit pas à Guibea.
\VS{10}Je les châtierai selon ma volonté, et les peuples s'assembleront contre eux, lorsqu'on les enchaînera pour leur double iniquité.
\VS{11}Ephraïm est une génisse bien dressée, qui aime à fouler le blé, mais je m'approcherai de son superbe cou ; j'attellerai Ephraïm, Juda labourera, Jacob brisera ses mottes.
\VS{12}Semez selon la justice, moissonnez selon la miséricorde, défrichez-vous un champ nouveau ! Car il est temps de chercher Yahweh, jusqu'à ce qu'il vienne, et répande sur vous sa justice.
\VS{13}Vous avez cultivé la méchanceté, et vous avez moissonné l'iniquité, vous avez mangé le fruit du mensonge; parce que vous avez eu confiance dans vos voies, dans la multitude de vos vaillants hommes.
\VS{14}Il s'élèvera un tumulte parmi ton peuple, et on détruira toutes tes forteresses, comme Schalman a détruit Beth-Arbel, au jour de la bataille, où la mère fut écrasée avec les enfants.
\VS{15}Béthel vous fera de même, à cause de votre extrême méchanceté ; le roi d'Israël sera entièrement exterminé dès l'aurore.
\Chap{11}
\TextTitle{L'amour de Yahweh pour Israël}
\VerseOne{}Quand Israël était jeune enfant, je l'aimais, et j'appelai mon fils hors d'Egypte\FTNT{La sortie des Hébreux de l'Egypte sous Moïse était une préfiguration de celle de Jésus-Christ lorsqu'il fuyait le massacre décrété par Hérode (Mt. 2:15).}.
\VS{2}Lorsqu'on les appelait ils se sont éloignés ; ils ont sacrifié aux Baals, et offert de l'encens aux idoles.
\VS{3}J'appris à Ephraïm à marcher en le prenant par les bras; et ils n'ont pas vu que je les guérissais.
\VS{4}Je les tirai avec des liens d'humanité, et avec des cordages d'amour, et je fus pour eux comme ceux qui enlèveraient le joug de dessus leur mâchoire, et je leur présentai de la nourriture.
\VS{5}Ils ne retourneront pas au pays d'Egypte ; mais le roi d'Assyrie sera leur roi, parce qu'ils n'ont point voulu revenir à moi.
\VS{6}L'épée fondra sur leurs villes, les réduira à néant, consumera leurs forces, et les dévorera, à cause des desseins qu'ils ont eus.
\VS{7}Mon peuple tient à se détourner de moi ; on les appelle vers le Très-Haut, mais aucun d'eux ne l'exalte.
\VS{8}Que ferai-je de toi, Ephraïm ? Te livrerais-je, Israël ? Te traiterai-je comme Adma ? Te rendrai-je semblable à Tseboïm ? Mon cœur s'agite au-dedans de moi, mes compassions sont émues.
\VS{9}Je n'exécuterai pas l'ardeur de ma colère, je ne reviendrai pas pour détruire Ephraïm ; car je suis Dieu, et non pas un homme, je suis le Saint au milieu de toi; et je n'entrerai point dans la ville.
\VS{10}Ils marcheront après Yahweh, qui rugira comme un lion\FTNT{Yahweh rugit comme un lion : Jésus-Christ est le lion de la tribu de Juda, car selon la chair, il est issu de la postérité de Juda (Lu. 3:23-38 ; Ap. 5:5). Le lion est le roi des animaux, or Jacob fut le premier a avoir annoncé la venue du Schilo, c'est-à-dire celui à qui appartient le sceptre (Ge. 49:8-12).}, et quand il rugira, les enfants accourront en hâte de la mer.
\VS{11}Ils accourront en hâte hors d'Egypte, comme des oiseaux, et hors du pays d'Assyrie, comme des colombes. Et je les ferai habiter dans leurs maisons, dit Yahweh.
\Chap{12}
\TextTitle{Dénonciation du péché d'Ephraïm}
\VerseOne{}Ephraïm m'entoure avec des mensonges, et la maison d'Israël avec des tromperies ; lorsque Juda erre sans frein vis-à-vis du Dieu Puissant, vis-à-vis du Saint fidèle.
\VS{2}Ephraïm se repaît de vent, et poursuit le vent d'orient ; il multiplie chaque jour le mensonge et la violence, et il traite alliance avec l'Assyrie, et l'on porte des huiles de senteur en Egypte.
\VS{3}Yahweh a aussi un procès avec Juda, et il punira Jacob pour sa conduite, il lui rendra selon ses œuvres.
\VS{4}Dans le ventre Jacob saisit son frère par le talon\FTNT{Ge. 25:26}, puis dans sa vigueur, il lutta avec Dieu\FTNT{Ge. 32:24-28.}.
\VS{5}Il lutta avec l'Ange, et il fut vainqueur, il pleura, et lui demanda grâce. Jacob l'avait rencontré à Béthel, et c'est là que Dieu nous a parlé.
\VS{6}Yahweh est le Dieu des armées ; Yahweh est son mémorial.
\VS{7}Et toi donc, reviens à ton Dieu, garde la miséricorde et la justice, et espère toujours en ton Dieu.
\VS{8}Ephraïm est un marchand\FTNT{Ephraïm est appelé « marchand », littéralement « Canaan ». Notez que l'ange de Laodicée s'exprime comme Ephraïm : « Je suis riche, je me suis enrichi… » (Ap. 3:14-19).}, qui a dans sa main des balances fausses, il aime à frauder.
\VS{9}Et Ephraïm dit : Quoi qu'il en soit, je suis devenu riche ; je me suis acquis des richesses ; c'est entièrement le produit de mon travail ; on ne trouvera en moi aucune iniquité, rien qui soit un péché.
\VS{10}Et moi, je suis Yahweh, ton Dieu, dès le pays d'Egypte ; je te ferai encore habiter dans des tentes, comme aux jours des fêtes solennelles.
\VS{11}Je parlerai par les prophètes, et je multiplierai les visions, et par les prophètes, je proposerai des paraboles.
\VS{12}Certainement Galaad n'est qu'iniquité, certainement ils ne seront que vanité. Ils sacrifient des bœufs dans Guilgal ; même leurs autels seront comme des monceaux de pierres sur les sillons des champs.
\VS{13}Jacob s'enfuit au pays de Syrie, et Israël servit pour une femme, et pour une femme il garda les troupeaux.
\VS{14}Par un prophète, Yahweh fit monter Israël hors d'Egypte, et par un prophète, Israël fut gardé.
\VS{15}Mais Ephraïm a provoqué Yahweh à une amère colère ; son Seigneur laissera sur lui le sang qu'il a répandu, et lui rendra ses mépris.
\Chap{13}
\TextTitle{Ephraïm persiste dans sa méchanceté}
\VerseOne{}Quand Ephraïm parlait, c'était une terreur ; il s'éleva en Israël. Mais il se rendit coupable par Baal, et mourut.
\VS{2}Et maintenant ils continuent de pécher, et se sont fait avec leur argent des images de fonte, des idoles selon leurs pensées; toutes sont un travail d'artisans, desquelles ils disent : Que les hommes qui sacrifient embrassent\FTNT{C'est une expression d'hommage.} les veaux !
\VS{3}C'est pourquoi ils seront comme la nuée du matin, et comme la rosée qui bientôt disparaît ; comme la balle qui est emportée par le vent hors de l'aire, comme la fumée sortant de la cheminée.
\VS{4}Et moi, je suis Yahweh, ton Dieu, dès le pays d'Egypte. Et tu ne devrais reconnaître d'autre dieu que moi, et il n'y a pas d'autre Sauveur que moi\FTNT{Yahweh dit qu'il n'y a pas d'autre sauveur que lui (Es. 43:11). Et les Ecrits de la Nouvelle Alliance nous présentent clairement notre Sauveur : Jésus-Christ (Mt.1:21; Ac.13:23 ; 2 Ti. 1:10 ; Tit : 1:4).}.
\VS{5}Je t'ai connu dans le désert, dans une terre aride.
\VS{6}Ils se sont rassasiés dans leurs pâturages ; ils se sont rassasiés, et leur cœur s'est enflé ; alors ils m'ont oublié.
\VS{7}Je serai pour eux comme un lion ; je les épierai sur la route comme un léopard.
\VS{8}Je les attaquerai, comme une ourse à qui on a enlevé ses petits, et je déchirerai l'enveloppe de leur cœur ; et là, je les dévorerai comme un lion ; les bêtes des champs les mettront en pièces.
\TextTitle{Châtiment d'Ephraïm}
\VS{9}Ta ruine, ô Israël, c'est que tu as été contre moi, alors que moi seul pouvais te secourir !
\VS{10}Où donc est ton roi ? Qu'il te délivre dans toutes tes villes ! Où sont tes juges, au sujet desquels tu as dit : Donne-moi un roi et des princes ?
\VS{11}Je t'ai donné un roi\FTNT{Ce passage concerne Saül, premier roi d'Israël (1 S. 8, 9 et 10)} dans ma colère, et je l'ôterai dans ma fureur.
\VS{12}L'iniquité d'Ephraïm est enveloppée, et son péché est mis en réserve.
\VS{13}Les douleurs comme de celle qui enfante le surprendront ; c'est un enfant qui n'est pas sage, qui, au temps marqué, ne sort pas du sein maternel.
\VS{14}Je les rachèterai de la puissance du scheol, je les délivrerai de la mort\FTNT{L'auteur de l'épître aux Hébreux applique ce passage à la victoire que le Seigneur Jésus-Christ a remportée face à la mort lors de sa résurrection (Hé. 2 :14-18). Depuis la chute d'Adam, les hommes ont toujours eu peur de la mort. Cette peur est d'autant plus forte de nos jours, car la plupart des gens sont angoissés par son aspect imprévisible, inévitable et par son non-sens. Et bien que beaucoup ne croient pas à l'existence de la vie après la mort (au paradis ou à l'enfer), la mort associée à l'annihilation, au non-être, apparaît d'autant plus monstrueuse et insupportable. Or notre Seigneur Jésus-Christ a vaincu la mort et il promet la vie éternelle à ceux qui croient en lui (Jn. 3:16 ; Jn. 5:24-29 ; Ap. 1:18). En plaçant notre foi en lui, nous avons non seulement la victoire sur la mort, mais aussi sur l'angoisse qu'elle produit dans le cœur de tout homme.} ; ô mort, où est ta peste ? Scheol, où est ta destruction\FTNT{1 Co. 15:55-57} ? Mais le repentir se cache à mes yeux !
\VS{15}Ephraïm a beau être fertile au milieu de ses frères, le vent d'orient, le vent de Yahweh s'élèvera du désert, viendra, desséchera ses sources et tarira ses fontaines. On pillera le trésor de tous ses objets précieux.
\VS{16}Samarie sera châtiée, car elle s'est rebellée contre son Dieu. Ils tomberont par l'épée; leurs petits enfants seront écrasés, et l'on fendra le ventre de leurs femmes enceintes.
\Chap{14}
\TextTitle{Bénédiction future d'Israël}
\VerseOne{}Israël, reviens à Yahweh ton Dieu ; car tu es tombé par ton iniquité.
\VS{2}Apportez avec vous des paroles, et revenez à Yahweh. Dites-lui : Pardonne toutes nos iniquités, et reçois le bien, pour le mettre à sa place! Et nous t'offrirons pour sacrifices la louange de nos lèvres.
\VS{3}L'Assyrie ne nous sauvera pas, nous ne monterons pas sur des chevaux, et nous ne dirons plus à l'ouvrage de nos mains : Notre dieu ! Car c'est auprès de toi que l'orphelin trouve de la compassion.
\VS{4}Je guérirai leur rébellion, et les aimerai volontairement ; parce que ma colère s'est détournée d'eux.
\VS{5}Je serai comme la rosée pour Israël ; il fleurira comme le lis, et il poussera ses racines comme le Liban.
\VS{6}Ses branches s'étendront, et sa magnificence sera comme celle de l'olivier, avec un parfum comme celui du Liban.
\VS{7}Ils reviendront s'asseoir à son ombre, et ils redonneront la vie au froment, et ils fleuriront comme la vigne ; et l'odeur de chacun d'eux sera comme celle du vin du Liban.
\VS{8}Ephraïm dira : Qu'ai-je à faire encore avec les idoles ? Je l'exaucerai, je le regarderai, je serai pour lui comme un cyprès verdoyant. C'est de moi que tu recevras ton fruit.
\VS{9}Qui est celui qui est sage ? Qu'il entende ces choses ! Et qui est celui qui est prudent ? Qu'il les connaisse ! Car les voies de Yahweh sont droites ; aussi les justes y marcheront, mais les rebelles y tomberont.
\PPE{}
\end{multicols}

%\clearpage\ShortTitle{Joël}\BookTitle{Joël}\BFont
\noindent\hrulefill
{\footnotesize
\textit{
\bigskip
{\centering{}
\\Auteur : Joël
\\(Heb. : Yow'el)
\\Signification : Yahweh est Dieu
\\Thème : Le jour de Yahweh
\\Date de rédaction : 9ème ou 8ème siècle av. J.-C.\\}
}
%\bigskip
\textit{
\\Joël, fils de Pethuel, exerça son ministère dans le royaume de Juda. Son message faisait suite à deux fléaux qui s’étaient abattus sur Juda, à savoir une invasion de sauterelles et la sécheresse. Il s’agissait d’un avertissement de Yahweh qui appelait le peuple à revenir à lui avec la promesse de le restaurer dans tout ce qu’il avait perdu. Joël annonça en outre l’effusion de l’Esprit sur toute chair dans un avenir lointain, prophétie ayant trouvé son accomplissement à la naissance de l’Eglise lors de la Pentecôte.\bigskip
}
}
\par\nobreak\noindent\hrulefill
\begin{multicols}{2}
\Chap{1}
\VerseOne{}La parole de Yahweh qui fut adressée à Joël, fils de Pethuel.
\VS{2}Anciens écoutez ceci ! Et vous, tous les habitants du pays, prêtez l'oreille ! Rien de pareil est-il arrivé de votre temps, ou même du temps de vos pères ?
\VS{3}Racontez-le à vos enfants, et que vos enfants le racontent à leurs enfants, et leurs enfants à la génération suivante !
\TextTitle{Désolation  après  l'invasion des sauterelles}
\VS{4}La sauterelle a dévoré les restes du gazam, le jélek a dévoré les restes de la sauterelle, et le hasil a dévoré les restes du jélek.
\VS{5}Ivrognes, réveillez-vous, et pleurez ; et vous tous buveurs de vin hurlez à cause du vin nouveau, parce qu’il est retranché de votre bouche.
\VS{6}Car une nation puissante et innombrable est montée contre mon pays. Elle a les dents d’un lion et les mâchoires d’un vieux lion.
\VS{7}Elle a réduit ma vigne en désert ; et a ôté l’écorce de mes figuiers ; elle les a entièrement dépouillés, et les a abattus, leurs branches en sont devenues blanches.
\VS{8}Lamente-toi, comme une jeune fille qui se revêt d'un sac pour pleurer le mari de sa jeunesse !
\VS{9}L'offrande et la libation sont retranchées de la maison de Yahweh, et les sacrificateurs qui font le service de Yahweh mènent deuil.
\VS{10}Les champs sont ravagés, la terre est dans le deuil ; parce que le blé est détruit, le moût est tari, l'huile est desséchée.
\VS{11}Les laboureurs sont confus, les vignerons gémissent, à cause du froment et de l'orge, car la moisson des champs est perdue.
\VS{12}La vigne est desséchée, le figuier languissant ; le grenadier, le palmier, le pommier, tous les arbres des champs ont séché, c'est pourquoi la joie a cessé parmi les fils de l’homme !
\VS{13}Sacrificateurs, ceignez-vous et pleurez ! Poussez des gémissements, vous qui faites le service de l’autel, hurlez, vous qui faites le service de mon Dieu ; entrez, passez la nuit vêtus de sacs car il est défendu à l'offrande et à la libation d’entrer dans la maison de votre Dieu.
\TextTitle{Désolation après la sécheresse et la famine}
\VS{14}Sanctifiez le jeûne, publiez l’assemblée solennelle, assemblez les anciens, et tous les habitants du pays dans la maison de Yahweh votre Dieu, et criez à Yahweh en disant :
\VS{15}Hélas ! Quel jour ! Car le jour de Yahweh\FTNT{Jour de Yahweh : Voir commentaire en Za. 14:1.} est proche : Il vient comme un ravage fait par le Tout-Puissant.
\VS{16}La nourriture n’est-elle pas retranchée sous nos yeux ? Et la joie et l'allégresse de la maison de notre Dieu ?
\VS{17}Les semances sont pourries sous leurs mottes, les magasins sont dévastés, les greniers sont renversés parce que le blé a manqué.
\VS{18}Ô combien ont gémi les bêtes, et dans quelle peine ont été les troupeaux de bœufs, parce qu’ils n’ont point de pâturage ! Aussi les troupeaux de brebis sont dévastés.
\VS{19}Ô Yahweh, je crierai à toi, car le feu a consumé les pâturages du désert, et la flamme a brûlé tous les arbres des champs.
\VS{20}Même toutes les bêtes des champs crient aussi vers toi ; car les torrents d’eau sont à sec, et le feu a consumé les pâturages du désert.
\Chap{2}
\TextTitle{Le jour de Yahweh, invasion future}
\VerseOne{}Sonnez du shofar en Sion, et sonnez avec un retentissement bruyant dans la montagne de ma sainteté ; que tous les habitants du pays tremblent ; car le jour de Yahweh vient ; car il est proche,
\VS{2}jour de ténèbres et d'obscurité, jour de nuées et de brouillards, il vient comme l'aurore s'étend sur les montagnes. Voici un peuple nombreux et puissant, tel qu’il n’y en a jamais eu, et qu’il n’y en aura jamais dans la suite des siècles.
\VS{3}Devant lui est un feu dévorant, et derrière lui la flamme brûle ; le pays était, avant sa venue, comme le jardin d’Eden, et après qu’il sera parti il sera comme un désert affreux ; et même il n’y aura rien qui lui échappe.
\VS{4}Leur aspect est comme l’aspect des chevaux, et ils courent comme des cavaliers.
\VS{5}C’est comme le bruit de chariots, quand ils sautent au sommet des montagnes, comme le bruit d’une flamme de feu, qui dévore le chaume, comme un peuple puissant rangé en bataille.
\VS{6}Les peuples tremblent en le voyant ; tous les visages en deviennent pâles et livides.
\VS{7}Ils courent comme des hommes vaillants, et montent sur les murailles comme des gens de guerre ; chacun va son chemin, sans se détourner de son chemin.
\VS{8}Ils ne se pressent point les uns les autres, chacun va son chemin ; ils se jettent au travers des épées sans être blessés.
\VS{9}Ils courent çà et là dans la ville, se précipitent sur les murailles, montent sur les maisons, entrent par les fenêtres comme le voleur.
\VS{10}La terre tremble devant eux, les cieux sont ébranlés, le soleil et la lune s’obscurcissent, et les étoiles retirent leur éclat.
\VS{11}Aussi Yahweh fait entendre sa voix devant son armée ; parce que son camp est très grand, car l'exécuteur de sa parole est puissant. Certainement le jour de Yahweh est grand et terrible. Qui peut le supporter ?
\TextTitle{Repentance et miséricorde}
\VS{12}Maintenant encore, dit Yahweh, revenez à moi de tout votre cœur, avec des jeûnes, avec des pleurs et des lamentations !
\VS{13}Déchirez vos cœurs et non vos vêtements, et revenez à Yahweh, votre Dieu ; car il est compatissant et miséricordieux, lent à la colère et riche en bonté, et il se repent d’avoir affligé.
\VS{14}Qui sait si Yahweh, votre Dieu, ne reviendra pas et ne se repentira pas, et s'il ne laissera point après lui la bénédiction, des offrandes et des libations ?
\VS{15}Sonnez du shofar en Sion ! Sanctifiez le jeûne, publiez l'assemblée solennelle !
\VS{16}Assemblez le peuple, sanctifiez la congrégation ! Réunissez les anciens, assemblez les enfants, même les nourrissons à la mamelle ! Que l’époux sorte de sa demeure, et l’épouse de sa chambre nuptiale !
\VS{17}Que les sacrificateurs qui font le service de Yahweh pleurent entre le portique et l'autel, et qu'ils disent : Yahweh ! Epargne ton peuple ! N’expose pas ton héritage à l'opprobre, que les nations n’en fassent pas un sujet de railleries ! Pourquoi dirait-on parmi les peuples : Où est leur Dieu ?
\TextTitle{Promesse de restauration}
\VS{18}Or Yahweh est jaloux pour son pays, et il est ému de compassion envers son peuple.
\VS{19}Yahweh répond et il dit à son peuple : Voici, je vous enverrai du blé, du moût, et de l'huile, et vous en serez rassasiés ; et je ne vous exposerai plus à l'opprobre parmi les nations.
\VS{20}J'éloignerai de vous l’armée venue du nord, je la chasserai vers une terre aride et déserte, son avant-garde dans la mer orientale, son arrière-garde dans la mer occidentale ; et sa puanteur montera, et son infection s’élèvera, après avoir fait de grandes choses.
\VS{21}Terre, ne crains pas, sois dans l’allégresse et réjouis-toi, car Yahweh fait de grandes choses !
\VS{22}Ne craignez point, bêtes des champs, car les pâturages du désert ont poussé leur jet, et même les arbres portent leur fruit ; le figuier et la vigne ont poussé avec vigueur.
\VS{23}Et vous, enfants de Sion, soyez dans l’allégresse et réjouissez-vous en Yahweh, votre Dieu, car il vous donnera la pluie selon sa justice, il vous enverra la pluie de la première\FTNT{La pluie de la première saison : En Orient, la première pluie tombe au moment des semailles d’automne. Elle est nécessaire afin que la semence puisse germer. Sous l'influence des pluies fertilisantes, les tendres pousses sortent du sol.} et de l’arrière-saison\FTNT{La pluie de l’arrière-saison : Elle tombe vers la fin de la saison, mûrit le grain et le prépare pour la moisson. C’est la pluie du printemps. Voir Jé. 5:24 ; Os. 6:1-3 ; Za. 10:1.}, au premier mois.
\VS{24}Et les aires se rempliront de blé, et les cuves regorgeront de moût et d'huile.
\VS{25}Ainsi je vous rendrai les fruits des années qu'ont dévoré la sauterelle, le jélek, le hasil et le gazam, ma grande armée que j’avais envoyée contre vous.
\VS{26}Vous aurez donc abondamment de quoi manger et être rassasiés, et vous louerez le Nom de Yahweh votre Dieu, qui aura fait pour vous des choses merveilleuses ; et mon peuple ne sera plus jamais dans la confusion.
\VS{27}Et vous saurez que je suis au milieu d'Israël, que je suis Yahweh, votre Dieu, et qu'il n'y en a point d'autre, et mon peuple ne sera plus jamais dans la confusion.
\TextTitle{La promesse de l'Esprit}
\VS{28}Et il arrivera après cela, que je répandrai mon Esprit sur toute chair\FTNT{Cette promesse s’est réalisée dans Actes 2. Elle se réalise encore aujourd’hui dans la vie de chaque enfant de Dieu. Enfin, elle sera pleinement réalisée lors du retour du Messie en Israël (Za. 12:10-14) puisque cette prophétie annonce la repentance nationale d’Israël (Ro. 11:26-27).} ; et vos fils et vos filles prophétiseront ; vos vieillards songeront des songes, et vos jeunes gens verront des visions.
\VS{29}Et même en ces jours-là, je répandrai mon Esprit sur les serviteurs et sur les servantes.
\TextTitle{Prodiges précédant le jour de Yahweh\FTNTT{Es. 13:9-10 ; 24:21-23 ; Ez. 32:7-10 ; Mt. 24:29-30.}}
\VS{30}Je ferai des prodiges dans les cieux et sur la terre, du sang, et du feu, et des colonnes de fumée ;
\VS{31}Le soleil se changera en ténèbres, et la lune en sang, avant que le grand et terrible jour de Yahweh vienne.
\VS{32}Et il arrivera que quiconque invoquera le Nom de Yahweh\FTNT{Quiconque invoquera le Nom de Yahweh sera sauvé. Ce passage nous confirme que Jésus-Christ est vraiment Yahweh. En effet, Paul, apôtre des païens, attribue le Nom de Yahweh et cette prophétie à Jésus-Christ (Ro. 10:9-13). C’est bien le Nom de Jésus-Christ qu'il faut invoquer pour être sauvé (Ac. 4:12 ; Ac. 9:21 ; 1 Co. 1:2). Les éditeurs de la Traduction du Monde Nouveau (bible des témoins de Jéhovah) se sont permis de « restituer » le Nom divin YHWH qui apparaît près de 6000 fois dans le Tanakh, en 237 endroits dans les écrits de la nouvelle alliance, alors qu’aucun ancien manuscrit de la nouvelle alliance (testament de Jésus) ne le contient. Ils affirment, sur la base d’éléments de preuves indirectes, que les scribes du IIème siècle remplaçaient le Nom divin dans la Nouvelle Alliance par « Seigneur » ou « Dieu ». Pour restituer ce Nom (YHWH), ils se basent sur les citations du Tanakh où celui-ci figure et sur des versions hébraïques de la nouvelle alliance dont la plus ancienne date du XIVème siècle pour la plupart des copies de textes plus anciens. On constate cependant qu’ils n’ont pas restitué le Nom divin en 1 Pierre 2:3 qui est pourtant une citation du Psaumes 34:8. Pourquoi ? 
Parce que l’application de ce texte à Jésus-Christ, la pierre rejetée, est évidente. Si ce texte du Tanakh mentionnant Yahweh est appliqué à Jésus que penser des autres ? Jésus-Christ est vraiment Yahweh qui s’est incarné pour nous sauver. D'ailleurs, le Nom de Jésus veut dire « YHWH est Sauveur » (Es. 7:14 ; Es. 9:5 ; Mt. 1 ; Lu. 1 ; 1 Ti. 3:16).} sera sauvé ; car le salut sera sur la montagne de Sion et dans Jérusalem, comme l’a dit Yahweh, et parmi les réchappés que Yahweh appellera.
\Chap{3}
\TextTitle{Rétablissement d'Israël\FTNTT{Es. 11:10-12 ; Jé. 23:5-8 ; Ez. 37:21-28 ; Ac. 15:15-17.}}
\VerseOne{}Car voici, en ces jours-là, et en ce temps-là, quand je ramènerai les captifs de Juda et de Jérusalem,
\TextTitle{Jugements des nations étrangères\FTNTT{Za. 12:2-3.}}
\VS{2}J'assemblerai toutes les nations\FTNT{Dieu rassemblera les nations dans la vallée de Josaphat (de l'hébreu « Yehowshaphat », « Yahweh a jugé ») pour leur jugement. Cette vallée est peut-être celle où le roi Josaphat remporta une grande victoire, avec beaucoup de facilité, sur les Moabites, les Ammonites et les Maonites (2 Ch. 20). Cette vallée s'étend à l'orient de Jérusalem, entre la ville et le Mont des Oliviers, et traverse le torrent de Cédron.}, et je les ferai descendre dans la vallée de Josaphat ; là, j'entrerai en jugement avec elles, à cause de mon peuple, et d'Israël, mon héritage, lequel ils ont dispersé parmi les nations, et parce qu’ils ont partagé entre eux mon pays ;
\VS{3}et qu'ils ont tiré mon peuple au sort ; ils ont donné l’enfant pour une prostituée, ils ont vendu la jeune fille pour du vin, et ils ont bu.
\VS{4}Et qu’ai-je aussi affaire de vous, Tyr et Sidon, et de vous, toutes les limites de la Palestine, me rendrez-vous ma récompense, ou voulez-vous m'irriter ? Je vous rendrai promptement et sans délai votre récompense sur votre tête.
\VS{5}Car vous avez pris mon argent et mon or ; et vous avez emporté dans vos temples ce que j’avais de plus précieux et de plus beau.
\VS{6}Vous avez vendu les enfants de Juda et de Jérusalem aux enfants des Grecs, afin de les éloigner de leur territoire.
\VS{7}Voici, je les ferai lever\FTNT{Le verbe lever vient de l'hébreu «'uwr » qui signifie « se réveiller », « éveiller », « être éveillé » , « inciter », « veiller »,  « se lever », « sortir de l’assoupissement », «prendre courage». Yahweh annonce le réveil des hébreux depuis les nations, d'où ils sont établis. Ce réveil est une prise de conscience qui aboutira au retour à la terre sainte.} du lieu où ils ont été transportés après que vous les avez vendus ; et je ferai retourner votre récompense sur votre tête.
\VS{8}Je vendrai donc vos fils et vos filles entre les mains des enfants de Juda, et ils les vendront à ceux de Séba, qui les transporteront vers une nation éloignée ; car Yahweh a parlé.
\VS{9}Publiez ceci parmi les nations ! Préparez la guerre ! Réveillez les hommes vaillants ! Qu’ils s’approchent, et qu’ils montent, tous les hommes de guerre !
\VS{10}Forgez des épées de vos hoyaux, et des lances de vos serpes ! Et que le faible dise : Je suis fort !
\VS{11}Hâtez-vous et venez, vous toutes les nations d'alentour, et rassemblez-vous ! Là, ô Yahweh, fais descendre tes hommes vaillants !
\VS{12}Que les nations se réveillent, et qu'elles montent à la vallée de Josaphat ! Car là je siégerai pour juger toutes les nations d'alentour.
\VS{13}Saisissez la faucille, car la moisson est mûre ! Venez, et descendez, car le pressoir est plein, les cuves regorgent ! Car leur méchanceté est grande,
\VS{14}Des multitudes, des multitudes, dans la vallée du jugement ; car le jour de Yahweh est proche, dans la vallée du jugement.
\VS{15}Le soleil et la lune s’obscurcissent, et les étoiles retirent leur éclat.
\VS{16}De Sion Yahweh rugit, de Jérusalem il fait entendre sa voix ; les cieux et la terre sont ébranlés. Mais Yahweh est le refuge pour son peuple, et la forteresse\FTNT{Jésus-Christ est notre rocher (commentaire Es. 8:14 ; Ps. 78:35 ; 1 Co. 10:4).} pour les enfants d’Israël.
\VS{17}Et vous saurez que je suis Yahweh, votre Dieu, qui habite à Sion, ma sainte montagne. Jérusalem sera sainte, et les étrangers n'y passeront plus.
\TextTitle{Restauration finale et pleine bénédiction du royaume}
\VS{18}Et il arrivera en ce jour-là, le moût ruissellera des montagnes, le lait coulera des collines, il y aura de l’eau dans tous les torrents de Juda ; et une source\FTNT{Jésus est celui qui fait jaillir en nous une source d’eau qui étanche notre soif à jamais et nous donne la vie éternelle (Jé. 2:13 ; Jé. 17:13 ; Ez. 47:1-12 ; Za. 14:8 ; Jn. 4:14 ; Ap. 22:1).} sortira de la maison de Yahweh, et arrosera la vallée de Sittim.
\VS{19}L'Egypte sera dévastée, Edom sera réduit en désert de désolation, à cause de la violence faite aux enfants de Juda, dont ils ont répandu le sang innocent dans leur pays.
\VS{20}Mais là, Judas sera éternellement habitée, et Jérusalem, d’âge en âge.
\VS{21}Et je nettoierai leur sang que je n’avais point nettoyé ; car Yahweh habite en Sion.

\PPE{}
\end{multicols}

%\clearpage\ShortTitle{Amos}\BookTitle{Amos}\BFont
\noindent\hrulefill
{\footnotesize
\textit{
\bigskip
{\centering{}
\\Auteur : Amos
\\(Heb. : Amowc)
\\Signification : Fardeau, porteur de fardeau
\\Thème : Jugement sur le péché
\\Date de rédaction : 8ème siècle av. J.-C.\\}
}
%\bigskip
\textit{
\\Originaire de Tekoa, Amos exerça son ministère dans le royaume du nord, au temps d’Ozias,  roi de Juda, et Jéroboam II, roi d’Israël. Il fut aussi le contemporain des prophètes Osée, Michée, Jonas et Esaïe.
%\bigskip
\\Alors que le peuple juif jouissait d’une certaine prospérité, l’immoralité et les sacrilèges prirent place dans le royaume. Amos avertit le peuple de son péché et du jugement qu'il encourait. Il lui rappela la bonté de Dieu et l’invita à revenir à Yahweh et à lui rester fidèle.\bigskip
}
}
\par\nobreak\noindent\hrulefill
\begin{multicols}{2}
\Chap{1}
\VerseOne{}Paroles d'Amos, berger de Tekoa, qui prophétisa sur Israël, du temps d’Ozias, roi de Juda, et de Jéroboam, fils de Joas, roi d'Israël, deux ans avant le tremblement de terre\FTNT{Za. 14:5}.
\VS{2}Il dit : Yahweh rugit de Sion, et fait entendre sa voix de Jérusalem. Les habitations des bergers sont en deuil, et le sommet du Carmel est desséché\FTNT{Jé. 25:30 ; Joë 3:16}.
\TextTitle{Yahweh annonce ses jugements sur les villes et les pays d'alentour}
\VS{3}Ainsi parle Yahweh : A cause de trois crimes de Damas, et même de quatre, je ne rappellerai point cela, mais je le ferai\FTNT{Il est question du jugement de Dieu.}, parce qu'ils ont foulé Galaad avec des herses de fer\FTNT{Es. 17:1}.
\VS{4}J'enverrai le feu dans la maison de Hazaël, et il dévorera le palais de Ben-Hadad.
\VS{5}Je briserai aussi les verrous de Damas, j'exterminerai de Bikath-Aven ses habitants, et de Beth-Eden celui qui tient le sceptre. Et le peuple de Syrie sera mené captif à Kir, dit Yahweh.
\VS{6}Ainsi parle Yahweh : A cause de trois crimes de Gaza, et même de quatre, je ne rappellerai point cela, mais je le ferai\FTNT{Il est question du jugement de Dieu.} parce qu'ils ont emmené des captifs en grand nombre pour les livrer à Edom\FTNT{Ez. 25:13-17}.
\VS{7}J'enverrai le feu dans les murs de Gaza, et il dévorera ses palais.
\VS{8}J'exterminerai d'Asdod les habitants, et d'Askalon celui qui tient le sceptre ; je tournerai ma main contre Ekron, et le reste des Philistins périra, dit le Seigneur, Yahweh.
\VS{9}Ainsi parle Yahweh : A cause de trois crimes de Tyr, et même de quatre, je ne rappellerai point cela, mais je le ferai\FTNT{Il est question du jugement de Dieu.}, parce qu'ils ont livré à Edom des captifs en grand nombre sans se souvenir de l'alliance fraternelle\FTNT{Ez. 26:2}.
\VS{10}J'enverrai le feu dans les murs de Tyr, et il dévorera ses palais.
\VS{11}Ainsi parle Yahweh : A cause de trois crimes d'Edom, et même de quatre, je ne rappellerai point cela,, mais je le ferai\FTNT{Il est question du jugement de Dieu.}, parce qu'il a poursuivi son frère avec l'épée, refoulant toute compassion, parce que sa colère déchire continuellement et qu'il garde sa fureur éternellement.
\VS{12}J'enverrai le feu dans Théman, et il dévorera les palais de Botsra\FTNT{Jé. 49:7 ; Abd. 1:9}.
\VS{13}Ainsi parle Yahweh : A cause de trois crimes des enfants d'Ammon, et même de quatre, je ne rappellerai point cela, mais je le ferai\FTNT{Il est question du jugement de Dieu.}, parce qu’ils ont fendu le ventre des femmes enceintes de Galaad pour étendre leurs frontières\FTNT{Ez. 21:33 ; So. 2:8}.
\VS{14}J'allumerai le feu dans les murs de Rabba, et il dévorera les palais, au bruit des cris de guerre au jour du combat, et au milieu de l’ouragan au jour de la tempête.
\VS{15}Et leur roi ira en captivité, lui et ses chefs, dit Yahweh.
\Chap{2}
\TextTitle{Suite des jugements prononcés sur les villes et les pays d'alentour}
\VerseOne{}Ainsi parle Yahweh : A cause de trois crimes de Moab, et même de quatre, je ne rappellerai point cela, mais je le ferai, parce qu'il a brûlé les os du roi d'Edom jusqu’à les calciner.
\VS{2}J'enverrai le feu dans Moab, et il dévorera les palais de Kerijoth ; et Moab périra dans le tumulte, au milieu des cris de guerre et du bruit du shofar\FTNT{Ez. 25:8-9}.
\VS{3}J'exterminerai les juges de son pays, et je tuerai tous ses chefs, dit Yahweh.
\TextTitle{Juda et Israël jugés à cause de leurs iniquités}
\VS{4}Ainsi parle Yahweh : A cause de trois crimes de Juda, et même de quatre, je ne rappellerai point cela, mais je le ferai, parce qu'ils ont rejeté la loi de Yahweh et n'ont point gardé ses ordonnances ; parce qu’ils ont été égarés par les mensonges après lesquels leurs pères ont marché.
\VS{5}J'enverrai le feu dans Juda, et il dévorera les palais de Jérusalem.
\VS{6}Ainsi parle Yahweh : A cause de trois crimes d'Israël, et même de quatre, je ne rappellerai point cela, mais je le ferai, parce qu'ils ont vendu le juste pour de l'argent, et le pauvre pour une paire de souliers.
\VS{7}Ils aspirent à voir la poussière de la terre sur la tête des misérables, et ils pervertissent la voie des pauvres. Le fils et le père vont vers la même jeune fille, pour profaner mon Saint Nom.
\VS{8}Ils se couchent près de chaque autel, sur les vêtements qu'ils ont pris en gage, et boivent dans la maison de leurs dieux le vin de ceux qu’ils châtient.
\VS{9}Pourtant j'ai détruit devant eux les Amoréens qui étaient hauts comme les cèdres et forts comme les chênes ; j'ai détruit son fruit en haut, et ses racines en bas\FTNT{No. 21:24 ; Jos. 24:8}.
\VS{10}Je vous ai fait monter du pays d'Egypte et je vous ai conduits dans le désert quarante ans pour que vous possédiez le pays des Amoréens.
\VS{11}J'ai suscité quelques-uns d'entre vos fils pour être prophètes, et quelques-uns d'entre vos jeunes gens pour être nazaréens\FTNT{Le mot nazaréen vient de l'hébreu nâzîr, de la racine nâzar qui signifie séparer. Il y avait deux types de nazaréens. Premièrement ceux qui étaient appelés par Dieu. Par exemple : Samson Jg. 13:1-7 ; Samuel 1 S. 1:11 ; Jean-Baptiste Lu. 1:15. Deuxièmement les personnes qui voulaient se consacrer à Dieu. No. 6:13.}. N'en est-il pas ainsi, enfants d'Israël ? dit Yahweh.
\VS{12}Mais vous avez fait boire du vin aux nazaréens, et vous avez donné cet ordre aux prophètes disant : Ne prophétisez pas\FTNT{Es. 30:10 ; Jé. 11:21 ; Mi. 2:6} !
\VS{13}Voici, je m’en vais fouler le lieu où vous habitez, comme un chariot plein de gerbes foule tout par où il passe.
\VS{14}Tellement que l’homme agile ne pourra pas fuir, et le fort ne pourra pas se servir de sa vigueur, et l’homme vaillant ne sauvera pas sa vie\FTNT{Jé. 46:6}.
\VS{15}Celui qui manie l'arc, ne pourra pas tenir ferme, et celui qui a les pieds légers n'échappera pas, et le cavalier ne sauvera pas sa vie.
\VS{16}Le plus courageux d’entre les hommes vaillants s'enfuira tout nu en ce jour-là, dit Yahweh.
\Chap{3}
\TextTitle{La maison de Jacob coupable devant Yahweh}
\VerseOne{}Enfants d’Israël écoutez la parole que Yahweh prononce contre vous, contre toutes les familles que j'ai fait monter du pays d'Egypte.
\VS{2}Je vous ai connu vous seuls d'entre toutes les familles de la terre ; c'est pourquoi je vous châtierai pour toutes vos iniquités\FTNT{Ex. 19:5-6 ; Ps. 147:19-20}.
\VS{3}Deux hommes marchent-ils ensemble s’ils ne sont pas accordés ?
\VS{4}Le lion rugit-il dans la forêt sans qu’il n'ait de proie ? Le lionceau jette-t-il son cri de sa tanière sans qu’il n'ait rien attrapé ?
\VS{5}L'oiseau tombe-t-il dans le filet posé à terre sans que ce ne soit un piège ? Le filet est-il ramassé par terre sans qu’il n’y ait rien de capturé ?
\VS{6}Le shofar sonne-t-il dans une ville sans que le peuple en étant tout effrayé s’assemble ? Arrive-t-il un malheur dans une ville sans que Yahweh ne l’ait causé\FTNT{Es. 45:7 ; La. 3:37-38} ?
\VS{7}Car le Seigneur, Yahweh, ne fait aucune chose qu'il n'ait révélé son secret aux prophètes ses serviteurs.
\VS{8}Le lion rugit, qui ne serait pas effrayé ? Le Seigneur, Yahweh, parle, qui ne prophétiserait\FTNT{Lorsque Yahweh parle, des prophètes sont suscités : Jé. 20:9 ; Mi. 3:8 ; Ac. 4:20.} ?
\VS{9}Faites entendre votre voix dans les palais d'Asdod, et dans les palais du pays d'Egypte, et dites : Assemblez-vous sur les montagnes de Samarie, et voyez l’important tumulte interne et quelles oppressions dans son sein !
\VS{10}Ils ne savent pas faire ce qui est droit, dit Yahweh, ils amassent la violence et la rapine dans leurs palais.
\VS{11}C'est pourquoi ainsi parle le Seigneur, Yahweh : L'ennemi viendra, il cernera le pays, il t'ôtera ta force et tes palais seront pillés.
\VS{12}Ainsi parle Yahweh : Comme un berger arrache de la gueule d'un lion deux jambes ou un bout d'oreille, ainsi les enfants d'Israël qui habitent dans Samarie seront arrachés de l’angle d’un lit et de l’asile de Damas.
\VS{13}Ecoutez et soyez mes témoins contre la maison de Jacob, dit le Seigneur, Yahweh, le Dieu des armées :
\VS{14}Le jour où je punirai Israël pour ses péchés, j’exercerai mon châtiment sur les autels de Béthel ; les cornes de l'autel seront brisées, et tomberont à terre.
\VS{15}J’abattrai la maison d'hiver et la maison d'été ; les maisons d'ivoire seront détruites, et un grand nombre de maisons disparaîtront, dit Yahweh.
\Chap{4}
\TextTitle{Yahweh condamne les sacrifices du peuple}
\VerseOne{}Ecoutez cette parole, vaches de Basan, qui êtes sur la montagne de Samarie, vous qui opprimez les faibles, qui maltraitez les pauvres, qui dites à leurs maîtres : Apportez, et que nous buvions !
\VS{2}Le Seigneur, Yahweh, l’a juré par sa sainteté : Voici, les jours viennent sur vous, où l’on vous enlèvera avec des hameçons, et votre postérité avec des crochets de pêche\FTNT{Jé. 16:16 ; Ha. 1.14-16}.
\VS{3}Vous sortirez dehors par les brèches, chacune devant soi, et vous serez jetées dans la forteresse, dit Yahweh.
\VS{4}Entrez dans Béthel, et péchez ! Multipliez vos péchés dans Guilgal ! Amenez vos sacrifices dès le matin, et vos dîmes tous les trois ans\FTNT{Voir commentaires en Mal. 3:10 et No. 18:21.} !
\VS{5}Brûlez de l’encens avec du pain levé pour l’offrande de remerciement ; proclamez et publiez les offrandes volontaires ; car c’est là ce que vous aimez, enfants d'Israël, dit le Seigneur, Yahweh\FTNT{Lé. 2:1}.
\TextTitle{Endurcissement du peuple malgré les châtiments de Yahweh}
\VS{6}C'est pourquoi je vous ai envoyé la famine dans toutes vos villes, et la disette de pain dans toutes vos demeures ; mais malgré cela, vous n’êtes pas revenus vers moi, dit Yahweh.
\VS{7}Je vous ai aussi privés de pluie, alors qu’il restait encore trois mois jusqu'à la moisson ; j'ai fait pleuvoir sur une ville et je n'ai pas fait pleuvoir sur une autre ville ; une parcelle a été arrosée par la pluie, et l'autre parcelle, sur laquelle il n'a pas plu, est desséchée\FTNT{1 R. 8:35 ; 1 R. 17:1 ; Es. 5:6 ; Ag. 1:11}.
\VS{8}Et deux, même trois villes sont allées vers une autre ville pour boire de l'eau et n'ont pas été désaltérées, mais vous n’êtes pas revenus vers moi, dit Yahweh.
\VS{9}Je vous ai frappés par la rouille et par la nielle, et la sauterelle a brouté autant de jardins et de vignes, de figuiers et d'oliviers que vous aviez, mais vous n’êtes pas revenus vers moi, dit Yahweh\FTNT{De. 28:22-39 ; 1 R. 8:37 ; Ag. 2:17 ; 2 Ch. 6:28}.
\VS{10}J’ai envoyé parmi vous la peste comme celle en Egypte ; j'ai tué par l'épée vos jeunes hommes et vos chevaux en captivité ; j'ai fait remonter, jusque dans votre nez, la puanteur de vos camps ; mais vous n’êtes pas revenus vers moi, dit Yahweh\FTNT{Ez. 14:19}.
\VS{11}Je vous ai détruits comme Dieu détruisit Sodome et Gomorrhe, et vous avez été comme un tison arraché du feu, mais vous n’êtes pas revenus vers moi, dit Yahweh\FTNT{Ge. 19:24 ; Jé. 49:18 ; Za. 3:2},
\VS{12}C'est pourquoi je te traiterai de la même manière ô Israël ; et parce que je te traiterai ainsi, prépare-toi à la rencontre de ton Dieu, ô Israël !
\VS{13}Car voici celui qui a formé les montagnes et créé le vent, et qui déclare à l'homme quelle est sa pensée, qui fait l'aube et l'obscurité, et qui marche sur les hauteurs de la terre ; son nom est Yahweh, le Dieu des armées.
\Chap{5}
\TextTitle{Israël invité à revenir entièrement à Yahweh}
\VerseOne{}Ecoutez cette parole, cette complainte que je prononce sur vous, maison d'Israël !
\VS{2}Elle est tombée, elle ne se relèvera plus, la vierge d'Israël ; elle est couchée par terre, et personne ne la relève.
\VS{3}Car ainsi a parlé le Seigneur, Yahweh : La ville qui mettait en campagne mille hommes n'en aura de reste que cent ; et celle qui mettait en campagne cent hommes n'en aura de reste que dix dans la maison d’Israël.
\VS{4}Car ainsi a parlé Yahweh à la maison d'Israël : Cherchez-moi, et vous vivrez !
\VS{5}Ne cherchez pas Béthel, et n'allez pas à Guilgal, et ne passez point à Beer-Schéba. Car Guilgal sera transportée en captivité, et Béthel sera détruite\FTNT{Os. 4:15}.
\VS{6}Cherchez Yahweh, et vous vivrez, de peur qu'il ne saisisse comme un feu la maison de Joseph, et que ce feu ne la consume, sans qu'il y ait personne à Béthel pour l’éteindre.
\VS{7}Ils changent le jugement en absinthe, et ils foulent à terre la justice\FTNT{Es. 5:26-28 ; Ha. 1:1-3}.
\VS{8}Celui qui a créé les Pléiades et l'Orion, qui change les profonds ténèbres en aurore, et qui obscurcit le jour en nuit, qui appelle les eaux de la mer, et les répand sur la surface de la Terre, Yahweh est son nom\FTNT{Es. 58:8-10 ; Job 9:9 ; Job 38:31}.
\VS{9}Il fait éclater la ruine sur les puissants, et la ruine vient sur les forteresses.
\VS{10}Ils haïssent à la porte ceux qui les reprennent, et ils ont en abomination celui qui parle en intégrité.
\VS{11}C'est pourquoi, puisque vous opprimez le pauvre, et que vous prenez de lui du blé en présent, vous avez bâti des maisons en pierres de taille, mais vous n'y habiterez pas ; vous avez planté des vignes délicieuses, mais vous n'en boirez pas le vin.
\VS{12}Car j'ai connu vos crimes, ils sont en grand nombre, et vos péchés se sont multipliés : Vous opprimez le juste, vous recevez des présents, et vous violez à la porte le droit des pauvres.
\VS{13}C'est pourquoi, en ce temps-ci, le sage se tait, car les temps sont mauvais.
\VS{14}Recherchez le bien et non le mal, afin que vous viviez ; et qu’ainsi Yahweh, le Dieu des armées, soit avec vous, comme vous l'avez dit.
\VS{15}Haïssez le mal, et aimez le bien, faites régner la justice à la porte ; peut-être Yahweh, le Dieu des armées, aura pitié des restes de Joseph.
\TextTitle{Le jour de Yahweh}
\VS{16}C'est pourquoi ainsi parle Yahweh, le Dieu des armées, le Seigneur, parle ainsi : Dans toutes les places on se lamentera, dans toutes les rues on dira : Hélas ! Hélas ! On appellera au deuil le laboureur, et à la lamentation ceux qui en savent le métier.
\VS{17}Dans toutes les vignes on se lamentera, quand je passerai au milieu de toi, dit Yahweh.
\VS{18}Malheur à ceux qui désirent le jour de Yahweh\FTNT{L’expression «~le jour du Seigneur~» ou «~le jour de Yahweh~» est une période durant laquelle Jésus-Christ interviendra ouvertement dans les affaires des hommes. Elle est utilisé dix-neuf fois dans le Tanakh (Es. 2:12 ; Es. 13:6-9 ; Ez. 13:5 ; Ez. 30:3 ; Joë. 1:15 ; Joë. 2:1, 11, 31 ; Joë. 3:14 ; Am. 5:18, 20 ; Ab. 1:15 ; So. 1:7, 14 ; Za. 14:1 ; Mal. 4:5) et quatre fois dans le Testament de Jésus (Ac. 2:20 ; 2 Th. 2:2 ; 2 Pi. 3:10). On y fait également allusion dans d’autres passages (Ap. 6:17 ; Ap. 16:14).}. Qu’attendez-vous du jour de Yahweh ? Ce sont des ténèbres, et non pas une lumière.
\VS{19}Vous serez comme un homme qui fuit devant un lion et qui rencontre un ours, ou qui entre dans sa maison, appuie sa main sur le mur et un serpent le mord.
\TextTitle{Mépris du droit et de la justice}
\VS{20}Le jour de Yahweh n’est-il pas ténèbres et non lumière ? Obscurité et non clarté ?
\VS{21}Je hais, je méprise vos fêtes, je ne prends pas plaisir à vos assemblées solennelles.
\VS{22}Si vous me présentez des holocaustes, je n’agréerai pas vos offrandes, je ne regarderai pas les bêtes grasses de vos offrandes de paix.
\VS{23}Eloigne de moi le bruit de tes cantiques ; je n’écouterai pas la mélodie de tes luths.
\VS{24}Mais le jugement roule comme de l'eau, et la justice comme un torrent intarissable.
\VS{25}Est-ce à moi, maison d'Israël, que vous avez offert des sacrifices et des gâteaux dans le désert pendant quarante ans ? 
\VS{26}Au contraire, vous avez porté la tente de votre roi, et de vos idoles Kijun\FTNT{«~Kijun~» : Probablement une statue d'un dieu Assyro-Babylonien de la planète Saturne et utilisé pour symboliser l'apostasie d'Israél}, l'étoile de votre dieu que vous vous êtes fabriqué.
\VS{27}C'est pourquoi je vous transporterai au-delà de Damas, dit Yahweh, dont le nom est le Dieu des armées.
\Chap{6}
\TextTitle{Ceux qui prospèrent seront emmenés captifs}
\VerseOne{}Hélas vous qui êtes à votre aise en Sion, et qui vous confiez en la montagne de Samarie, lieux les plus renommés d’entre les principaux des nations, auprès desquels va la maison d'Israël.
\VS{2}Passez à Calné, et regardez ; allez de là à Hamath la grande, puis descendez à Gath chez les Philistins. Ces villes sont-elles plus prospères que vos deux royaumes, ou leur pays n'est-il pas plus étendu que votre pays ?
\VS{3}Vous qui éloignez le jour du malheur, et qui approchez le règne de la violence.
\VS{4}Vous qui vous couchez sur des lits d'ivoire, et qui sont étendus sur vos coussins ; qui mangez les agneaux du troupeau, et les veaux pris du lieu où on les engraisse ;
\VS{5}qui fredonnez au son du luth ; qui inventez des instruments de musique comme David.
\VS{6}Qui buvez le vin dans de grandes coupes, et qui parfumez des parfums les plus exquis, et qui n’êtes pas affligés pour la plaie de Joseph.
\VS{7}A cause de cela ils vont être emmenés à la tête des captifs, et les cris de joie de ces personnes voluptueuses prendront fin.
\VS{8}Le Seigneur, Yahweh, l’a juré par lui-même. Yahweh, Dieu des armées, dit : J'ai en détestation l'orgueil de Jacob, et j'ai en haine ses palais, c'est pourquoi je livrerai la ville, et tout ce qui est en elle.
\VS{9}Et si il reste dix hommes dans une maison, ils mourront.
\VS{10}Un proche parent prendra un mort et le brûlera pour emporter les os hors de la maison ; il dira à celui qui est au fond de la maison : Y a-t-il encore quelqu'un avec toi ? Et il répondra : Il n’y a plus personne. Puis il dira : Silence ! Ce n’est pas le moment de prononcer le nom de Yahweh.
\VS{11}Car voici, Yahweh ordonne : Et il frappera les grandes maisons par des débordements d'eau, et la petite maison en débris.
\VS{12}Les chevaux courent-ils sur les rochers, y laboure-t-on avec des bœufs, pour que vous ayez changé la droiture en poison, et le fruit de la justice en absinthe ?
\VS{13}Vous vous réjouissez de choses qui ne sont que néant, vous dites : N’est-ce pas par notre force que nous avons acquis de la puissance ?
\VS{14}Voici, je ferai lever contre vous, maison d’Israël, dit Yahweh, le Dieu des armées, une nation qui vous opprimera depuis l'entrée de Hamath jusqu'au torrent du désert.
\Chap{7}
\TextTitle{Avertissement\FTNTT{Am. 8:1 ; 9:10}}
\VerseOne{}Le Seigneur, Yahweh, me fit voir cette vision : Voici, il formait des sauterelles au temps où le regain commençait à croître ; et voici le regain poussait après les récoltes du roi.
\VS{2}Et quand elles eurent achevé de dévorer l'herbe de la terre, je dis : Seigneur Yahweh, pardonne, je te prie ! Comment Jacob subsistera-t-il ? Car il est faible.
\VS{3}Yahweh se repentit de cela. Cela n'arrivera pas, dit Yahweh.
\VS{4}Le Seigneur, Yahweh, me fit voir cette vision : Voici, le Seigneur, Yahweh, proclamait le jugement par le feu. Et le feu dévorait le grand abîme et dévorait les champs.
\VS{5}Et je dis : Seigneur Yahweh ! Arrête, je te prie ! Comment Jacob subsistera-t-il ? Car il est faible.
\VS{6}Yahweh se repentit de cela. Cela non plus n'arrivera pas, dit le Seigneur, Yahweh.
\VS{7}Il me fit voir cette vision : Voici, le Seigneur se tenait debout sur un mur fait au niveau, et il avait un niveau dans la main.
\VS{8}Et Yahweh me dit : Que vois-tu, Amos ? Et je répondis : Un niveau. Et le Seigneur me dit : Je mettrai le niveau au milieu de mon peuple d'Israël, je ne lui pardonnerai plus.
\VS{9}Et les hauts lieux d'Isaac seront ravagés, et les sanctuaires d'Israël seront détruits ; et je me lèverai contre la maison de Jéroboam avec l'épée.
\TextTitle{Amatsia accuse Amos devant Jéroboam}
\VS{10}Alors Amatsia, sacrificateur de Béthel, fit dire à Jéroboam roi d'Israël : Amos conspire contre toi au milieu de la maison d'Israël ; le pays ne saurait supporter toutes ses paroles.
\VS{11}Car voici ce que dit Amos : Jéroboam mourra par l'épée, et Israël sera emmené captif hors de son pays.
\VS{12}Et Amatsia dit à Amos : Voyant\FTNT{Voyant ou prophète.}, va-t’en, fuis dans le pays de Juda, et manges-y ton pain, et là tu prophétiseras.
\VS{13}Mais ne continue pas à prophétiser à Béthel\FTNT{Bethel, qui signifie «~maison de Dieu~», était devenue le sanctuaire du roi Jéroboam. De même aujourd’hui des églises du Seigneur sont devenues la propriété des hommes et les brebis de Dieu sont devenues la propriété des pasteurs.}, car c'est le sanctuaire du roi, et c'est une maison royale.
\TextTitle{Amos répond}
\VS{14}Amos répondit à Amatsia : Je n'étais ni prophète ni fils de prophète ; j'étais un berger, et je cueillais des figues sauvages.
\VS{15}Or Yahweh m’a pris derrière le troupeau, et Yahweh m’a dit : Va, prophétise à mon peuple d'Israël.
\VS{16}Ecoute maintenant la parole de Yahweh, tu dis : Ne prophétise pas contre Israël, et ne parle pas contre la maison d'Isaac.
\VS{17}C'est pourquoi ainsi parle Yahweh : Ta femme se prostituera dans la ville, tes fils et tes filles tomberont par l'épée, ton champ sera partagé au cordeau, et toi, tu mourras sur une terre souillée, et Israël sera emmené captif hors de son pays.
\Chap{8}
\TextTitle{Vision du panier de fruit, la fin pour le peuple d'Israël}
\VerseOne{}Le Seigneur, Yahweh, me fit voir cette vision : Voici, je vis un panier de fruits d'été.
\VS{2}Il dit : Que vois-tu, Amos ? Et je répondis : Un panier de fruits. Et Yahweh me dit : La fin est venue pour mon peuple d'Israël, je ne continuerai plus à lui pardonner.
\VS{3}En ce jour-là, les chants du palais seront des gémissements, dit le Seigneur, Yahweh ; en tout lieu, il y aura beaucoup de cadavres que l'on jettera en silence.
\VS{4}Ecoutez ceci vous qui dévorez les pauvres et qui faites périr les pauvres misérables du pays,
\VS{5}et qui dites : Quand la nouvelle lune sera-t-elle passée pour que nous vendions du blé ? Quand finira le sabbat pour que nous ouvrions les greniers ? Nous diminuerons l’épha, nous augmenterons le sicle, nous falsifierons les balances pour tromper,
\VS{6}nous achèterons les faibles pour de l’argent, et le pauvre pour une paire de souliers, et nous vendrons la criblure du froment.
\VS{7}Yahweh l’a juré par la gloire de Jacob : Jamais je n’oublierai toutes leurs actions !
\VS{8}La terre ne sera-t-elle point émue d'une telle chose, et tous ses habitants ne se lamenteront-ils point ? Le pays tout entier montera comme le fleuve. Il se soulèvera et s’affaissera comme le fleuve d'Egypte.
\VS{9}Il arrivera en ce jour-là, dit le Seigneur, Yahweh, que je ferai coucher le soleil à midi, et que j’obscurcirai la terre en plein jour.
\VS{10}Je changerai vos fêtes en deuil, et tous vos chants en lamentations ; je couvrirai de sacs tous les reins, et je rendrai chauves toutes les têtes ; je mettrai le pays dans le deuil comme pour un fils unique, et sa fin sera un jour d’amertume.
\VS{11}Voici, les jours viennent, dit le Seigneur, Yahweh, où j'enverrai la famine\FTNT{Nous sommes dans une époque où la Parole de Dieu, l’Evangile véritable a presque disparu au profit de l’évangile de prospérité. L’esprit de commerce a pris place au sein de beaucoup d’églises. C’est le temps de l’église de Laodicée, une église qui fait l’apologie de la richesse matérielle.} dans le pays ; non une famine de pain, ni une soif d'eau, mais d’entendre les paroles de Yahweh.
\VS{12}Ils erreront d’une mer jusqu'à l'autre, du nord à l'orient, ils iront çà et là pour chercher la parole de Yahweh, et ils ne la trouveront pas.
\VS{13}En ce jour-là, les belles vierges et les jeunes hommes mourront de soif.
\VS{14}Ceux qui jurent par le péché de Samarie disent : Vive ton Dieu, ô Dan ! Vive la voie de Beer-Schéba ! Mais ils tomberont et ne se relèveront plus.
\Chap{9}
\TextTitle{Prophétie annonçant la destruction\FTNTT{De. 28:63-68.}}
\VerseOne{}Je vis le Seigneur qui se tenait debout sur l'autel. Et il dit : Frappe le chapiteau et que les seuils s’ébranlent ; et brise-les sur leurs têtes à tous ! Je tuerai par l'épée ce qui restera d'eux. Il ne s’enfuira pas un fugitif, il ne s’échappera pas un fuyard.
\VS{2}S’ils pénètrent dans le séjour des morts, ma main les enlèvera de là ; s’ils montent aux cieux, je les en ferai descendre.
\VS{3}S’ils se cachent au sommet du Carmel, je les y rechercherai et je les enlèverai de là ; s’ils se dérobent à mes yeux dans le fond de la mer, là j’ordonnerai au serpent de les mordre.
\VS{4}Lorsqu'ils s'en iront en captivité devant leurs ennemis, là j’ordonnerai à l’épée de les tuer ; je fixerai mon regard sur eux pour leur faire du mal et non du bien.
\VS{5}Le Seigneur, Yahweh des armées, touche la terre, et elle tremble, et tous ses habitants sont dans le deuil ; elle monte tout entière comme le fleuve, et elle s’affaisse comme le fleuve d'Egypte.
\VS{6}Il a bâti sa demeure dans les cieux, et fondé sa voûte sur la terre ; il appelle les eaux de la mer, et les répand sur la surface de la Terre. Son nom est Yahweh.
\VS{7}N'êtes-vous pas pour moi comme les enfants des Ethiopiens, enfants d'Israël ? dit Yahweh. N'ai-je pas fait monter Israël du pays d'Egypte, les Philistins de Caphtor et les Syriens de Kir ?
\VS{8}Voici, les yeux du Seigneur, Yahweh, sont sur ce royaume pécheur. Je le détruirai de dessus la surface de la terre. Cependant, je ne détruirai pas entièrement la maison de Jacob, dit Yahweh.
\VS{9}Car voici, je donnerai mes ordres, et je secouerai la maison d'Israël parmi toutes les nations, comme on secoue le blé dans le crible, sans qu'il en tombe un grain à terre.
\VS{10}Tous les pécheurs de mon peuple mourront par l'épée, ceux qui disent : Le mal n'approchera pas, il ne nous atteindra pas.
\TextTitle{Yahweh relève la maison de David}
\VS{11}En ce temps-là, je relèverai le tabernacle de David qui est tombé, j’en réparerai les brèches, j’en redresserai les ruines, et je le rebâtirai comme il était autrefois,
\VS{12}afin qu'ils possèdent le reste d’Edom et toutes les nations sur lesquelles mon nom a été invoqué, dit Yahweh, qui accomplira cela.
\TextTitle{Restauration d'Isarël}
\VS{13}Voici, les jours viennent, dit Yahweh, où le laboureur suivra de près le moissonneur, et celui qui foule les raisins atteindra celui qui répand la semence ; et le moût ruissellera des montagnes et découlera de toutes les collines.
\VS{14}Je ramènerai les captifs de mon peuple d'Israël ; ils rebâtiront les villes dévastées, et y habiteront, ils planteront des vignes, et en boiront le vin ; ils feront des jardins et en mangeront les fruits.
\VS{15}Je les planterai sur leur terre, et ils ne seront plus arrachés du pays que je leur ai donné\FTNT{Cette prophétie annonce la restauration de la maison de David. Personne ne chassera Israël de sa terre, aucune nation n’a le pouvoir de le déloger, car c’est le Seigneur qui l’a établi.}, dit Yahweh, ton Dieu.
\PPE{}
\end{multicols}

%\clearpage\ShortTitle{Abdias}\BookTitle{Abdias}\BFont
\noindent\hrulefill
{\footnotesize
\textit{
\bigskip
{\centering{}
\\Auteur : Abdias
\\(Heb. : Obadyah)
\\Signification : Adorateur ou serviteur de Yahweh
\\Thème : Condamnation d'Edom
\\Date de rédaction : 6ème siècle av. J.-C.\\}
}
%\bigskip
\textit{
\\Prophète ayant exercé son ministère en Juda, Abdias reçut un message court, mais clair sur le jour de Yahweh, et plus particulièrement sur le jugement d’Edom à la suite de ses violences envers Israël.\bigskip
}
}
\par\nobreak\noindent\hrulefill
\begin{multicols}{2}
\TextTitle{[I. Malédiction d'Edom : Introduction]}
\Chap{1}
\VerseOne{}Vision d'Abdias. Ainsi parle le Seigneur, Yahweh, sur Edom : Nous l’avons entendu de la part de Yahweh, et un messager a été envoyé parmi les nations : Courage, levons-nous contre lui pour le combattre\FTNT{Jé. 49:14} !
\VS{2}Voici, je te rendrai petit parmi les nations, tu seras fort méprisé.
\VS{3}L'orgueil de ton coeur t'a égaré, toi qui habites dans le creux des rochers, qui sont ta haute demeure, et qui dis en toi-même : Qui me précipitera jusqu’à terre ?
\VS{4}Quand tu élèverais ton nid comme l'aigle, et quand bien même tu le mettrais entre les étoiles, je te précipiterai de là, dit Yahweh.
\VS{5}Si des voleurs entraient chez toi, ou des pillards de nuit, comme te voilà ruiné ! Mais ils ne prendraient que ce qui leur suffit. Si des vendangeurs entraient chez toi, ne laisseraient-ils pas des grappillages ?
\VS{6}Comme Esaü a été fouillé ! Comme ses trésors cachés ont été découverts !
\VS{7}Tous tes alliés t'ont chassé jusqu'à la frontière, ceux qui étaient en paix avec toi t'ont trompé et ont eu le dessus sur toi, ceux qui mangeaient ton pain t'ont tendu des pièges, et tu ne t’en es pas aperçu.
\VS{8}N’est-ce pas en ce jour-là, dit Yahweh, que je ferai périr les sages d'Edom, et l’intelligence de la montagne d'Esaü ?
\VS{9}Tes guerriers seront effrayés, ô Théman ! Afin qu’ils soient tous retranchés de la montagne d'Esaü par le carnage\FTNT{Ez. 25:13 ; Mi. 7:8 ; So. 2:8}.
\TextTitle{II. Causes de la malédiction}
\VS{10}A cause de la violence que tu as faite à ton frère Jacob, la honte te couvrira, et tu seras retranché à jamais.
\VS{11}Le jour où tu te tenais en face de lui, le jour où des étrangers emmenaient captive son armée, où des inconnus entraient dans ses portes et jetaient le sort sur Jérusalem, toi aussi, tu étais comme l'un d'eux.
\VS{12}Ne considère pas avec joie le jour de ton frère, le jour de son malheur, ne te réjouis pas sur les enfants de Juda au jour de leur ruine, et n’ouvre pas une grande bouche au jour de la détresse.
\VS{13}N’entre pas dans les portes de mon peuple au jour de sa ruine, ne considère pas avec joie son malheur, au jour de sa ruine, et que tes mains ne se portent pas sur ses richesses, au jour de sa ruine !
\VS{14}Ne te tiens pas aux carrefours pour exterminer ses fugitifs, et ne livre pas ses fuyards au jour de la détresse.
\TextTitle{III. Edom au jour de Yahweh}
\VS{15}Car le jour de Yahweh est proche pour toutes les nations ; on te fera comme tu as fait, tes actes retomberont sur ta tête\FTNT{Jé. 50:15-29 ; Ez. 35:15}.
\VS{16}Car comme vous avez bu sur ma montagne sainte, ainsi toutes les nations boiront continuellement ; elles boiront, elles avaleront, et elles seront comme si elles n'avaient jamais été\FTNT{Jé. 25:15-28}.
\TextTitle{Délivrance future de Jacob et jugement sur Edom}
\VS{17}Mais le salut\FTNT{Le salut sera sur la montagne de Sion. Cette prophétie fait allusion au Royaume messianique. Voir Ro. 11:26.} sera sur la montagne de Sion, elle sera sainte, et la maison de Jacob possédera ses possessions.
\VS{18}La maison de Jacob sera un feu, et la maison de Joseph une flamme, et la maison d'Esaü du chaume ; ils l'allumeront et la consumeront ; et il ne restera rien de la maison d'Esaü, car Yahweh a parlé.
\VS{19}Ceux du midi possèderont la montagne d'Esaü, ceux de la plaine le pays des Philistins ; ils posséderont le territoire d'Ephraïm et celui de Samarie ; et Benjamin possédera Galaad.
\VS{20}Les captifs de cette armée des enfants d'Israël posséderont le pays des Cananéens jusqu'à Sarepta, et ceux qui auront été transportés de Jérusalem, qui sont à Sepharad, posséderont les villes du midi.
\VS{21}Des libérateurs monteront sur la montagne de Sion, pour juger la montagne d'Esaü ; et la royauté sera à Yahweh.
\PPE{}
\end{multicols}

%\clearpage\ShortTitle{Jonas}\BookTitle{Jonas}\BFont
\begin{multicols}{2}
\TextTitle{[I. Désobéissance et fuite de Jonas
\\Introduction]}
\Chap{1}
\VerseOne{}La parole de Yahweh fut adressée à Jonas, fils d'Amitthaï, en ces mots :
\TextTitle{[Jonas fuit la face de Yahweh]}
\VS{2}Lève-toi, va à Ninive (1), la grande ville, et crie contre elle ! Car leur malice est montée jusqu'à moi.
\VS{3}Mais Jonas se leva pour s'enfuir à Tarsis, loin de la face de Yahweh. Il descendit à Japho, où il trouva un navire qui allait à Tarsis ; il paya le prix du transport et y entra pour aller à Tarsis, loin de la face de Yahweh.
\VS{4}Mais Yahweh fit lever un grand vent sur la mer, et il y eut une grande tempête sur la mer, de sorte que le navire semblait se briser.
\VS{5}Et les mariniers eurent peur, et ils crièrent chacun à son dieu, et jetèrent dans la mer les objets qui étaient dans le navire, pour l’alléger. Jonas descendit au fond du navire, se coucha et s’endormit profondément.
\VS{6}Le chef des marins s'approcha de lui, et lui dit : Qu’as-tu dormeur ? Lève-toi, invoque ton Dieu ! Peut-être ton Dieu pensera à nous et nous ne périrons pas.
\VS{7}Puis ils se dirent l'un à l'autre : Venez, tirons au sort pour savoir qui est la cause de ce malheur. Ils tirèrent au sort, et le sort tomba sur Jonas.
\VS{8}Alors ils lui dirent : Dis-nous quelle est la cause de ce malheur. Quel est ton métier, et d'où viens-tu ? Quel est ton pays, et de quel peuple es-tu ?
\VS{9}Il leur répondit : Je suis Hébreu, et je crains Yahweh, le Dieu des cieux, qui a fait la mer et la terre sèche.
\VS{10}Alors ces hommes furent saisis d'une grande crainte, et lui dirent : Pourquoi as-tu fait cela ? Car ces hommes savaient qu’il fuyait loin de la face de Yahweh, parce qu'il le leur avait déclaré.
\VS{11}Ils lui dirent : Que te ferons-nous pour que la mer se calme ? Car la mer était de plus en plus agitée.
\TextTitle{[II. Jonas et le poisson
\\Jonas englouti par le poisson]}
\VS{12}Il leur répondit : Prenez-moi, et jetez-moi dans la mer, et la mer se calmera ; car je sais que c’est moi la cause de cette grande tempête.
\VS{13}Ces hommes ramaient pour revenir sur la terre sèche, mais ils ne le purent, car la mer s'agitait toujours plus contre eux.
\VS{14}Alors ils invoquèrent Yahweh, et dirent : O Yahweh, ne nous fais pas périr à cause de la vie de cet homme, et ne mets pas sur nous le sang innocent ! Car toi, Yahweh, tu fais comme il te plait (2).
\VS{15}Alors ils prirent Jonas, et le jetèrent dans la mer. Et la fureur de la mer s'arrêta.
\VS{16}Ces hommes furent saisis d’une grande crainte envers Yahweh, et ils offrirent des sacrifices à Yahweh, et firent des vœux.
\VS{17}Yahweh ordonna à un grand poisson d’engloutir Jonas, et Jonas fut dans le ventre du poisson trois jours et trois nuits.
\Chap{2}
\VerseOne{}Jonas pria Yahweh, son Dieu, dans le ventre du poisson.
\TextTitle{[Prière de Jonas et l'exaucement de Yahweh]}
\VS{2}Il dit : Dans ma détresse j’ai invoqué Yahweh, et il m'a exaucé ; du sein du scheol j’ai crié, et tu as entendu ma voix (1).
\VS{3}Tu m'as jeté dans les profondeurs, au cœur de la mer, et le courant m'a environné ; tous tes flots et toutes tes vagues ont passé sur moi.
\VS{4}Je disais : Je suis chassé loin de tes yeux ! Cependant je verrai encore le temple de ta sainteté.
\VS{5}Les eaux m'ont environné jusqu'à l'âme. L'abîme m'a enveloppé, les roseaux ont lié ma tête.
\VS{6}Je suis descendu jusqu'aux bases des montagnes, la terre fermait sur moi ses barres pour toujours ; mais tu m’as fait remonter vivant de la fosse, Yahweh, mon Dieu !
\VS{7}Quand mon âme s’était affaiblie en moi, je me suis souvenu de Yahweh, et ma prière est parvenue jusqu’à toi, dans le temple de ta sainteté.
\VS{8}Ceux qui s’adonnent à des vanités mensongères abandonnent ta miséricorde.
\VS{9}Mais moi, je t’offrirai des sacrifices avec un cri de louange, j’accomplirai les vœux que j’ai faits : Car le salut vient de Yahweh (2).
\VS{10}Alors Yahweh parla au poisson, et le poisson vomit Jonas sur la terre sèche.
\TextTitle{[III. Le plus grand réveil de l'histoire
\\Ninive se repent et elle est épargnée]}
\Chap{3}
\VerseOne{}La parole de Yahweh fut adressée à Jonas une seconde fois, en disant :
\VS{2}Lève-toi, va à Ninive, la grande ville, et proclames-y à haute voix ce que je t'ordonne !
\VS{3}Jonas se leva, et alla à Ninive, suivant la parole de Yahweh. Or Ninive était une grande ville devant Dieu, de trois jours de marche.
\VS{4}Jonas commença dans la ville le chemin d'une journée de marche ; il criait et disait : Encore quarante jours, et Ninive sera renversée !
\VS{5}Les hommes de Ninive crurent à Dieu, ils publièrent un jeûne, et se vêtirent de sacs, depuis le plus grand d'entre eux jusqu'au plus petit.
\VS{6}Cette parole parvint au roi de Ninive ; il se leva de son trône, ôta de dessus lui son manteau, se couvrit d'un sac, et s'assit sur la cendre.
\VS{7}Puis il fit faire une proclamation, et publier dans Ninive par décret du roi et de ses grands : Que les hommes, les bêtes, les bœufs et les brebis, ne goûtent de rien, ne paissent point, et ne boivent point d'eau !
\VS{8}Que les hommes et les bêtes soient couverts de sacs, qu'ils crient à Dieu avec force, et que chacun revienne de sa mauvaise voie, et des actions violentes que ses mains ont commises !
\VS{9}Qui sait si Dieu ne reviendra pas et ne se repentira pas, et s'il ne se détournera pas de son ardente colère, en sorte que nous ne périssions point ?
\VS{10}Dieu vit ce qu’ils faisaient et comment ils revenaient de leur mauvaise voie. Alors Dieu se repentit du mal qu'il avait déclaré de leur faire, et il ne le fit point.
\TextTitle{[IV. Miséricorde infinie de Dieu
\\Mécontentement de Jonas]}
\Chap{4}
\VerseOne{}Mais cela déplut fortement à Jonas, et il fut furieux.
\VS{2}Il pria Yahweh, et dit : Oh ! Yahweh, n'est-ce pas là ce que je disais quand j'étais encore dans mon pays ? C'est pourquoi j'ai voulu m'enfuir à Tarsis. Car je savais que tu es un Dieu compatissant, miséricordieux, lent à la colère et riche en bonté, et qui te repens du mal (1).
\VS{3}Maintenant, Yahweh, prends-moi donc la vie, car la mort m'est meilleure que la vie.
\TextTitle{[Reproches de Yahweh à Jonas]}
\VS{4}Et Yahweh répondit : Fais-tu bien de te mettre en colère ?
\VS{5}Alors Jonas sortit de la ville, et s'assit à l'orient de la ville, là il se fit une cabane, et y resta à l'ombre, jusqu'à ce qu'il vît ce qui arriverait à la ville.
\VS{6}Yahweh Dieu ordonna à un ricin de croître au-dessus de Jonas, pour donner de l’ombre sur sa tête et pour le délivrer de son mal. Jonas éprouva une grande joie à cause de ce ricin.
\VS{7}Mais le lendemain, à l’aurore, Dieu ordonna à un ver d’attaquer le ricin, et le ricin sécha.
\VS{8}Au lever du soleil, Dieu ordonna à un vent chaud d’orient de souffler, et le soleil frappa la tête de Jonas, au point qu’il s’évanouit. Il demanda la mort, et dit : La mort m'est meilleure que la vie.
\VS{9}Dieu dit à Jonas : Fais-tu bien de te mettre en colère à cause du ricin ? Et il répondit : Je fais bien de m’irriter jusqu’à la mort.
\VS{10}Et Yahweh dit : Tu as pitié du ricin pour lequel tu n'as point travaillé et que tu n'as point fait croître, qui est né dans une nuit et qui a péri dans une nuit.
\VS{11}Et moi, je n’aurais pas pitié de Ninive, la grande ville, dans laquelle il y a plus de cent vingt mille personnes qui ne savent point distinguer leur main droite de leur main gauche, et des animaux en grand nombre !
\PPE{}
\end{multicols}

%\clearpage\ShortTitle{Michée}\BookTitle{Michée}\BFont
\noindent\hrulefill
{\footnotesize
\textit{
\bigskip
{\centering{}
\\(Mikha)
\\Signifie : Qui est semblable à Dieu ?
\\Thème : Le jugement et le royaume
\\Auteur : Michée 
\\Date de rédaction : 8ème siècle av. J.-C.\\}
}
%\bigskip
\textit{
\\Originaire de Moréscheth, Michée exerça son ministère dans le royaume du sud au temps d’Ezéchias, roi de Juda et fut contemporain d’Osée, Amos et Esaïe. Alors que la corruption et l’idolâtrie régnaient en Samarie et à Jérusalem, Michée appela le peuple à se détourner de ses iniquités et les prévint du danger qui les menaçait. Il prophétisa également le rétablissement final de la nation juive et mit en exergue la miséricorde divine.\bigskip
}
}
\par\nobreak\noindent\hrulefill
\begin{multicols}{2}
\TextTitle{[Jugement de Yahweh sur Israël infidèle]}
\Chap{1}
\VerseOne{}La Parole de Yahweh qui fut adressée à Michée, de Moréscheth, au temps de Jotham, Achaz, et Ezéchias, rois de Juda, laquelle lui fut adressée dans une vision contre Samarie et Jérusalem.
\VS{2}Vous tous, peuples, écoutez ! Et toi, terre, et tout ce qui est en elle, soyez attentifs ! Et que le Seigneur, Yahweh, soit témoin contre vous, le Seigneur, sortant du palais de sa sainteté.
\VS{3}Car voici, Yahweh sortira de son lieu, il descendra, et marchera sur les hauts lieux de la terre\FTNT{Cette prophétie annonce à la fois la destruction du royaume du nord par Salmanasar V (régna de 727-722 av. J.-C.) en 722 av. J.-C. (2 R. 17:1-23), l’invasion de Sanchérib (régna de 705 à 681 av. J.-C.), (2 R. 18:13 à 2 R. 19:37) ainsi que celle de Nebucadnetsar (régna de 604 à 562 av. J.-C.), (2 R. 24 et 25).} ;
\VS{4}et les montagnes se fondront sous lui, et les vallées se fendront, elles seront comme de la cire devant le feu et comme des eaux qui coulent sur une pente.
\VS{5}Tout ceci arrivera à cause du crime de Jacob, et à cause des péchés de la maison d'Israël. Or quel est le crime de Jacob ? N'est-ce pas Samarie ? Et quels sont les hauts lieux de Juda ? N'est-ce pas Jérusalem ?
\TextTitle{[Chutes futures de Samarie et Jérusalem]}
\VS{6}C'est pourquoi je réduirai Samarie en un monceau de pierres dans les champs, un lieu où l'on plante des vignes ; et je ferai rouler ses pierres dans la vallée, et je découvrirai ses fondements.
\VS{7}Toutes ses images taillées seront brisées, tous ses salaires de prostitution seront brûlés au feu, et je mettrai tous ses faux dieux en désolation ; parce qu'elle les a entassés par le moyen du salaire de sa prostitution, ils serviront de salaire à une prostituée.
\VS{8}C'est pourquoi je me plaindrai, et je hurlerai ; je m'en irai dépouillé et nu ; je ferai une lamentation comme celle des dragons, et je mènerai le deuil comme celui des autruches.
\VS{9}Car sa plaie est incurable, elle est venue jusqu'en Juda, et est parvenue jusqu'à la porte de mon peuple, jusqu'à Jérusalem.
\VS{10}Ne l'annoncez point dans Gath, et ne pleurez nullement ! Vautre-toi dans la poussière à Beth-Leaphra.
\VS{11}Passez habitants de Schaphir, dans la nudité et la honte ! Les habitants de Tsaanan ne sont point sortis ; le deuil de Beth-Haëtsel vous prive de son abri.
\VS{12}L’habitante de Maroth est dans l'angoisse à cause de son bien ; parce que le mal est descendu de par Yahweh sur la porte de Jérusalem.
\VS{13}Attelle le cheval au char, habitante de Lakisch ! Toi qui es le commencement du péché de la fille de Sion ; car en toi ont été trouvés les crimes d'Israël.
\VS{14}C'est pourquoi donne des présents à cause de Moréschet-Gath ; les maisons d'Aczib mentiront aux rois d'Israël.
\VS{15}Je t’amènerai un autre héritier, habitante de Maréscha ; et la gloire d'Israël s'en ira jusqu'à Adullam.
\VS{16}Arrache tes cheveux et fais-toi tondre, à cause de tes fils qui font tes délices ; arrache tout le poil de ton corps, comme un aigle qui mue, car ils sont emmenés prisonniers loin de toi.
\TextTitle{[Causes du jugement de Dieu sur Israël]}
\Chap{2}
\VerseOne{}Malheur à ceux qui pensent à faire outrage, qui forgent le mal sur leurs lits, et qui l'exécutent dès le point du jour, parce qu'ils ont le pouvoir en main.
\VS{2}S'ils convoitent des possessions, ils les ont aussitôt ravies, des maisons, ils les ont aussitôt prises ; ainsi ils oppriment l'homme et sa maison, l'homme et son héritage.
\VS{3}C'est pourquoi ainsi parle Yahweh : Voici, je médite contre cette famille-ci un mal duquel vous ne pourrez point préserver votre cou, et vous ne marcherez point la tête levée, car ce temps est mauvais.
\VS{4}En ce temps-là on fera de vous un proverbe lugubre, et l'on gémira d'un gémissement lamentable, en disant : Nous sommes entièrement détruits ; la part de mon peuple, il la change de mains ! Comment nous enlève-t-il et partage-t-il notre terre à l'infidèle ?
\VS{5}C'est pourquoi il n'y aura personne qui jettera le cordeau pour ton lot, dans l'assemblée de Yahweh.
\VS{6}Ne prophétisez point, disent-ils, ne prophétisez point de telles choses ; l'opprobre ne s'éloignera point.
\VS{7}Or toi qui es appelée maison de Jacob, l'Esprit de Yahweh est-il amoindri ? Sont-ce là ses actes ? Mes paroles ne sont-elles pas bonnes pour celui qui marche droitement ?
\VS{8}Mais celui qui était hier mon peuple, s'élève à la manière d'un ennemi ; vous dépouillez le manteau avec le vêtement à ceux qui passent en assurance, de retour de la guerre.
\VS{9}Vous chassez les femmes de mon peuple hors des maisons de leurs délices ; vous ôtez pour toujours ma gloire de dessus leurs enfants.
\VS{10}Levez-vous et marchez, car ce pays n'est plus pour vous un lieu de repos ; à cause de la souillure, il vous détruira d’une violente destruction.
\VS{11}S'il y a quelque homme qui court après le vent et le mensonge, et qui mente en disant : Je te prophétiserai sur du vin et sur les boissons fortes, ce sera le prophète de ce peuple.
\TextTitle{[Yahweh, le Dieu qui rassemble son peuple]}
\VS{12}Mais je t'assemblerai tout entier, ô Jacob ! Et je ramasserai entièrement le reste d'Israël, et le mettrai tout ensemble comme les brebis d'une bergerie, comme un troupeau au milieu de son pâturage ; il y aura un grand bruit pour la foule des hommes.
\VS{13}Celui qui fera la brèche\FTNT{«~Celui qui fera la brèche~» est Jean-Baptiste, qui fut suscité à une époque où la gloire de Dieu ne se manifestait plus depuis quatre cents ans. En effet, Dieu n’avait plus suscité de ministère prophétique ou de messagers pour parler à son peuple. Le ciel était comme fermé. Il est donc venu ouvrir une brèche, c’est-à-dire préparer le chemin du Seigneur selon Es. 40:3-5, Mal. 3:1, Mal. 4:5-6.} montera  devant eux, on brisera, et on passera outre, et ils sortiront par la porte ; et leur Roi marchera devant eux\FTNT{Le Roi qui marchera devant eux est bien Jésus-Christ. Le message de Jean-Baptiste était clair : «~Repentez-vous car le Royaume des cieux arrive~»,  or ce royaume est celui du Messie (Mt. 3:1-3).}, Yahweh sera à leur tête.
\TextTitle{[Corruption et méchanceté des chefs]}
\Chap{3}
\VerseOne{}C'est pourquoi je dis : Ecoutez, chefs de Jacob, et vous conducteurs de la maison d'Israël ! N'est-ce point à vous de connaître ce qui est juste ?
\VS{2}Ils haïssent le bien, et aiment le mal ; ils leur arrachent la peau et la chair de dessus les os.
\VS{3}Ils dévorent la chair de mon peuple, lui arrachent la peau, et lui brisent les os ; ils le mettent en pièces comme dans un pot, comme de la viande dans une chaudière.
\VS{4}Alors ils crieront à Yahweh, mais il ne les exaucera point, et il leur cachera sa face en ce temps-là, parce qu’ils se sont mal conduits dans leurs actions
\VS{5}Ainsi parle Yahweh contre les prophètes qui égarent mon peuple, qui annoncent la paix si leurs dents ont quelque chose à mordre, et qui publient la guerre si on ne leur met rien dans la bouche :
\VS{6}C'est pourquoi la nuit sera sur vous, afin que vous n'ayez plus de vision ; et elle s'obscurcira, afin que vous ne deviniez plus ; le soleil se couchera sur ces prophètes-là, et le jour leur sera ténébreux.
\VS{7}Les voyants seront honteux, et les devins seront confondus ; eux tous se couvriront la barbe, parce qu'il n'y aura aucune réponse de Dieu.
\VS{8}Mais moi, je suis rempli de force, de justice, et de courage, par l'Esprit de Yahweh, pour déclarer à Jacob son crime, et à Israël son péché.
\TextTitle{[Future destruction de Jérusalem]}
\VS{9}Ecoutez maintenant ceci, chefs de la maison de Jacob, et vous conducteurs de la maison d'Israël, qui avez la justice en abomination, et qui pervertissez tout ce qui est droit,
\VS{10}vous qui bâtissez Sion avec le sang, et Jérusalem avec l'injustice.
\VS{11}Ses chefs jugent pour des présents, ses sacrificateurs enseignent pour un salaire, et ses prophètes devinent pour de l'argent\FTNT{Ce passage est encore d’actualité de nos jours. En effet, nombreux sont les dirigeants chrétiens qui exigent un salaire, notamment par le moyen de la dîme pour la plupart, en échange de leurs prières, enseignements,  conseils, formations bibliques… Le même constat se fait avec les chantres qui font des concerts payants alors qu’ils ont reçu leurs grâces du Seigneur gratuitement. Voir commentaire en Mt. 10:8.}, puis ils s'appuient sur Yahweh, en disant : Yahweh n'est-il pas parmi nous ? Le mal ne nous atteindra pas.
\VS{12}C'est pourquoi, à cause de vous, Sion sera labourée comme un champ, et Jérusalem sera réduite en ruines, et la montagne du temple en hauts lieux de forêt.
\TextTitle{[Marcher au nom de Yahweh]}
\Chap{4}
\VerseOne{}Mais il arrivera dans les derniers jours\FTNT{Voir commentaire en Ge. 49:1-2.}, que  la montagne de la maison de Yahweh\FTNT{Dans les Ecritures, les montagnes symbolisent parfois une grande puissance terrestre, et les collines celles de moindre importance. Cette prophétie confirme l’établissement du Royaume messianique dont la capitale sera Jérusalem (2 S. 7:14-16). Esaïe avait reçu la même prophétie que l’on peut découvrir au chapitre 2 de son livre.} sera affermie au sommet des montagnes, et sera élevée par-dessus les collines ; les peuples y afflueront.
\VS{2}Et des nations nombreuses iront et diront : Venez, et montons à la montagne de Yahweh, à la maison du Dieu de Jacob ; il nous enseignera ses voies, et nous marcherons dans ses sentiers ; car la loi sortira de Sion, et la parole de Yahweh de Jérusalem.
\VS{3}Il exercera le jugement parmi des peuples nombreux, et il sera l'arbitre de nations puissantes et lointaines ; et de leurs épées elles forgeront des hoyaux ; et de leurs hallebardes, des serpes ; une nation ne lèvera plus l'épée contre une autre, et on n'apprendra plus la guerre.
\VS{4}Mais chacun demeurera sous sa vigne et sous son figuier, et il n'y aura personne qui les épouvante ; car la bouche de Yahweh des armées aura parlé.
\VS{5}Les peuples marchent chacun au nom de leur dieu ; mais nous, nous marchons au Nom de Yahweh, notre Dieu, à toujours et à perpétuité.
\TextTitle{[Future restauration d'Israël]}
\VS{6}En ce jour-là, dit Yahweh, j'assemblerai les boiteux, je recueillerai ceux que j'avais chassés et ceux que j'avais maltraités.
\VS{7}Je ferai de ceux qui boitent un reste, et de ceux qui étaient éloignés une nation robuste ; Yahweh régnera sur eux, à la montagne de Sion, dès lors et à toujours.
\VS{8}Et toi, tour du troupeau, citadelle de la fille de Sion, jusqu'à toi viendra, à toi arrivera la souveraineté première ; le royaume sera à la fille de Jérusalem.
\TextTitle{[Yahweh, le Dieu qui rachète son peuple]}
\VS{9}Pourquoi maintenant pousses-tu des cris ? N'y a-t-il point de roi au milieu de toi ? Ou ton conseiller est-il mort, que la douleur t'ait saisie comme celle qui enfante ?
\VS{10}Souffre et gémis, fille de Sion, comme celle qui enfante ; car tu sortiras bientôt de la ville, tu demeureras aux champs, et tu iras jusqu'à Babylone ; là tu seras délivrée ; là Yahweh te rachètera de la main de tes ennemis.
\TextTitle{[Nations rassemblés pour l'Harmaguedon]}
\VS{11}Maintenant plusieurs nations se sont rassemblées contre toi\FTNT{Il est ici question de la guerre d’Harmaguédon. Voir commentaire en Ap. 16:12-16.} et disent : Qu'elle soit profanée ! Et que notre œil voie en Sion ce qu’il y voudrait voir\FTNT{Jérusalem est une horloge de Dieu. Le Seigneur a fait en sorte que les nations aient les yeux tournés vers ce bout de terre, car c'est là que débutera la troisième guerre mondiale, l’Harmaguédon (Mt. 24:15-28 ; Ap. 16:12-16 ; Ap. 19:11-21). Là aura lieu le jugement des nations, dans la vallée de Josaphat (Joë. 3:2-12). Le Messie reviendra (Es. 59:20-21 ; Za. 14:1-8 ; Ac. 1:10-11) et gouvernera le monde depuis Jérusalem (Za. 14:9-21). L'actuel conflit israélo-palestinien nous confirme bien ces prophéties. Il ne se passe pas un jour sans que les informations nous rapportent des événements venant de cet endroit du monde. D'ailleurs le Seigneur lui-même nous invite à suivre de près ce qu'il s'y passe (Mt. 24:32-34).}.
\VS{12}Mais ils ne connaissent point les pensées de Yahweh, et ne comprennent pas ses desseins ; car il les a assemblées comme des gerbes dans l'aire.
\VS{13}Lève-toi, et foule, fille de Sion ! Car je te ferai une corne de fer, et te mettrai des ongles d'airain ; et tu écraseras des peuples nombreux, et tu consacreras par interdit leurs profits à Yahweh, et leurs richesses au Seigneur de toute la terre.
\VS{14}Maintenant, fille de troupes, rassemble tes troupes ; on a mis le siège contre nous, on frappera le juge d'Israël avec la verge sur la joue.
\TextTitle{[Naissance du roi: le Messie]
\\(cp. Mt. 2:1-6 ; 27:24-37)}
\Chap{5}
\VerseOne{}Mais toi, Bethléhem Ephrata, petite pour être entre les milliers de Juda, de toi sortira quelqu’un pour être dominateur en Israël, dont l'origine remonte aux temps anciens, aux jours de l'éternité\FTNT{Il est question ici de Jésus-Christ. Ce passage nous parle de sa préexistence éternelle. Les pharisiens, scribes et principaux sacrificateurs avaient la connaissance de cette prophétie concernant le Messie (Mt. 2:1-6).}.
\VS{2}C'est pourquoi il les livrera jusqu'au temps où enfantera celle qui doit enfanter ; et le reste de ses frères retournera avec les enfants d'Israël.
\VS{3}Et il se maintiendra et gouvernera par la force de Yahweh, avec la magnificence du Nom de Yahweh, son Dieu ; et ils auront une demeure assurée, car dès lors il sera élevé jusqu'aux extrémités de la terre.
\VS{4}C'est lui qui sera la paix. Lorsque l'Assyrien sera entré dans notre pays, et qu'il aura mis le pied dans nos palais, nous élèverons contre lui sept pasteurs et huit princes du peuple.
\VS{5}Ils ravageront le pays d'Assyrie avec l'épée, et le pays de Nimrod à ses portes. Il nous délivrera ainsi des Assyriens, quand ils seront entrés dans notre pays, et qu'ils auront mis le pied dans nos quartiers.
\VS{6}Et le reste de Jacob sera au milieu de peuples nombreux, comme une rosée qui vient de Yahweh, et comme une pluie qui tombe sur l'herbe, qui ne s’attend à aucun homme, et qui n'espère pas des enfants des hommes.
\VS{7}Aussi le reste de Jacob sera parmi les nations, et au milieu de peuples nombreux, comme un lion parmi les bêtes de la forêt, et comme un lionceau parmi les troupeaux de brebis ; qui, en passant, foule et déchire, sans que personne ne puisse les sauver.
\TextTitle{[Jugement de Dieu sur les ennemis d'Isrël]}
\VS{8}Ta main se lèvera sur tes adversaires, et tous tes ennemis seront exterminés.
\VS{9}Et il arrivera en ce temps-là, dit Yahweh, que j'exterminerai du milieu de toi tes chevaux, et ferai périr tes chars.
\VS{10}J'exterminerai les villes de ton pays et renverserai toutes tes forteresses.
\VS{11}J'exterminerai aussi de ta main les sorcelleries, et tu n'auras plus aucun devin.
\VS{12}Et j'exterminerai du milieu de toi tes idoles et tes statues, et tu ne te prosterneras plus devant l'ouvrage de tes mains.
\VS{13}J'arracherai aussi du milieu de toi les poteaux d'Asherah\FTNT{Ex. 34:13.}, et détruirai tes ennemis.
\VS{14}Et j'exercerai ma vengeance avec colère et avec fureur contre toutes les nations qui ne m'auront pas écouté.
\TextTitle{[Yahweh appelle son peuple à l'humilité]}
\Chap{6}
\VerseOne{}Ecoutez maintenant ce que dit Yahweh : Lève-toi, plaide devant les montagnes, et que les collines entendent ta voix !
\VS{2}Ecoutez, montagnes, le procès de Yahweh, vous solides fondements de la terre ! Car Yahweh a un procès avec son peuple, et il plaidera avec Israël.
\VS{3}Mon peuple, que t'ai-je fait, ou en quoi t'ai-je causé de la peine ? Réponds-moi !
\VS{4}Car je t'ai fait sortir du pays d'Égypte et t'ai délivré de la maison de servitude, et j'ai envoyé devant toi Moïse, Aaron et Marie.
\VS{5}Mon peuple, rappelle-toi quel conseil Balak, roi de Moab, avait pris contre toi, et de ce que Balaam, fils de Beor, lui répondit ; et de ce que j'ai fait depuis Sittim jusqu'à Guilgal, afin que tu connaisses les justices de Yahweh.
\TextTitle{[Pratiquer la justice]}
\VS{6}Avec quoi me présenterai-je devant Yahweh, et me prosternerai-je devant le Dieu Très-Haut ? Me présenterai-je avec des holocaustes, et avec des veaux d'un an ?
\VS{7}Yahweh prendra-t-il plaisir à des milliers de béliers ou à des myriades de torrents d'huile ? Donnerai-je pour mon crime mon premier-né\FTNT{Selon la loi (Ex. 13 : 2 ; Ex. 3 : 12), les premiers-nés de l’homme et des animaux appartenaient au Seigneur. Ceux des animaux étaient offerts en sacrifice alors que le sacrifice des enfants était formellement interdit sous peine de mort (Lé. 18:21 ; Lé. 20:2-5 ; De. 12:31 ; De. 18:10).}, le fruit de mes entrailles pour le péché de mon âme ?
\VS{8}Ô homme ! Il t'a fait connaître ce qui est bon, et ce que Yahweh exige de toi : Que tu fasses ce qui est juste, que tu aimes la miséricorde, et que tu marches en toute humilité avec ton Dieu.
\VS{9}La voix de Yahweh crie à la ville, et le sage reconnaît son Nom. Écoutez la verge, et celui qui la dirige !
\VS{10}Y a-t-il encore dans la maison du méchant des trésors iniques, et un épha court et détestable ?
\VS{11}Tiendrai-je pour pur celui qui a de fausses balances et de faux poids dans son sac ?
\VS{12}Ses riches sont pleins de violence, ses habitants usent de mensonge, et ils ont une langue trompeuse dans leur bouche.
\VS{13}C'est pourquoi je te rendrai languissante en te frappant, et te ravagerai à cause de tes péchés.
\VS{14}Tu mangeras, mais tu ne seras pas rassasiée, et la faim sera au-dedans de toi-même ; tu mettras de côté, mais tu ne sauveras point, et ce que tu auras sauvé je le livrerai à l'épée.
\VS{15}Tu sèmeras, mais tu ne moissonneras point ; tu presseras l'olive, mais tu ne feras pas d'onctions d'huile ; et tu presseras le moût, mais tu ne boiras pas le vin.
\VS{16}Car tu as gardé les ordonnances d'Omri, et toutes les œuvres de la maison d'Achab, et tu as marché dans leurs conseils. C'est pourquoi je te livrerai à la désolation, je ferai de tes habitants un objet de raillerie, et vous porterez l'opprobre de mon peuple.
\TextTitle{[Le mal appelé bien, et le bien appelé mal]}
\Chap{7}
\VerseOne{}Malheur à moi ! Car je suis comme quand on a cueilli les fruits d'été et les grappillages de la vendange : Il n'y a ni grappe pour manger ni les premiers fruits que mon âme désirait.
\VS{2}Le fidèle est exterminé du pays, et il n'y a plus de juste entre les hommes ; ils sont tous en embûche pour verser le sang, chacun chasse son frère avec des filets.
\VS{3}Leurs mains sont habiles à faire le mal : Le gouverneur exige, le juge demande un salaire, le grand déclare ce qu'il convoite, et ils s'unissent.
\VS{4}Le meilleur d'entre eux est comme une ronce, et le plus juste est pire qu'une haie d'épines. Le jour annoncé par tes sentinelles, ton châtiment arrive. C'est alors qu'ils seront dans la confusion.
\VS{5}Ne crois pas à ton ami intime, et ne te confie pas en tes conducteurs ; garde-toi d'ouvrir ta bouche devant la femme qui dort dans ton sein.
\VS{6}Car le fils déshonore le père, la fille s'élève contre sa mère, la belle-fille contre sa belle-mère, et chacun a pour ennemis les gens de sa maison.
\TextTitle{[Espérance en Yahweh, le Dieu de notre salut]}
\VS{7}Mais moi, je regarderai vers Yahweh, je m'attendrai au Dieu de mon salut ; mon Dieu m'exaucera.
\VS{8}Toi, mon ennemie, ne te réjouis pas sur moi ; si je suis tombée, je me relèverai ; si j'ai été gisante dans les ténèbres, Yahweh m'éclairera.
\VS{9}Je supporterai la colère de Yahweh, car j'ai péché contre lui, jusqu'à ce qu'il défende ma cause, et qu'il me fasse justice ; il me conduira à la lumière, je verrai sa justice.
\VS{10}Et mon ennemie le verra, et la honte la couvrira ; elle qui me disait : Où est Yahweh, ton Dieu ? Mes yeux la verront, et alors elle sera foulée aux pieds comme la boue des rues.
\VS{11}Le jour où il rebâtira tes murs, en ce jour-là tes limites seront reculées.
\VS{12}En ce jour-là on viendra jusqu'à toi d'Assyrie et des villes d'Égypte, et depuis les villes d'Égypte jusqu'au fleuve, et depuis une mer jusqu'à l'autre mer, et depuis une montagne jusqu'à l'autre montagne ;
\VS{13}après que le pays aura été en désolation à cause de ses habitants, et du fruit de leurs actions.
\VS{14}Pais ton peuple avec ta houlette, le troupeau de ton héritage, qui demeure seul dans les forêts au milieu de Carmel ! Et fais qu'ils paissent en Basan et en Galaad, comme aux temps anciens.
\VS{15}Je lui ferai voir des choses merveilleuses, comme au jour où tu sortis du pays d'Égypte.
\VS{16}Les nations le verront, et elles seront honteuses avec toute leur force ; elles mettront la main sur la bouche, et leurs oreilles seront sourdes.
\VS{17}Elles lécheront la poussière comme le serpent, comme les reptiles de la terre ; elles trembleront dans leurs forteresses et accourront toutes effrayées vers Yahweh, notre Dieu, et te craindront.
\VS{18}Quel dieu est semblable à toi, qui est un Dieu qui pardonne l'iniquité, et qui passe par-dessus les péchés du reste de son héritage ? Il ne garde pas à toujours sa colère, parce qu'il prend plaisir à la miséricorde.
\VS{19}Il aura encore compassion de nous ; il effacera nos iniquités, et jettera tous nos péchés au fond de la mer.
\VS{20}Tu feras voir ta fidélité à Jacob, et ta miséricorde à Abraham, comme tu l’as juré à nos pères dès les temps anciens\FTNT{Les versets 18 à 20 de Mi. 7 sont lus chaque année dans les synagogues le jour des expiations.}.
\PPE{}
\end{multicols}

%\clearpage\ShortTitle{Na.}\BookTitle{Nahum}\BFont
\noindent\hrulefill
{\footnotesize
\textit{
\bigskip
{\centering{}
\\Auteur~: Nahum
\\(Heb.~: Nachuwm)
\\Signification~: Consolation, qui a compassion
\\Thème~: La ruine de Ninive
\\Date de rédaction~: 7\up{ème} siècle av. J.-C.\\}
}
\textit{
\\Nahum d'Elkosch, contemporain d'Habakuk, exerça son service dans le royaume de Juda. Il fut chargé d'annoncer la chute de Ninive, qui s'était repentie quelques décennies plus tôt, suite à la prédication de Jonas, mais elle multiplia de nouveau ses actes d'injustice et de violences au point de terroriser tous les peuples des alentours.\bigskip
}
}
\par\nobreak\noindent\hrulefill
\begin{multicols}{2}
\Chap{1}
\VerseOne{}Prophétie sur Ninive\FTNT{Ninive était la capitale de l'ancien empire assyrien. Voir Jon. 1:1-2.}, qui est le livre de la vision de Nahum d'Elkosch.
\TextTitle{Jugement annoncé sur Ninive}
\VS{2}Yahweh est un Dieu jaloux, il se venge, Yahweh se venge, il est plein de fureur~; Yahweh se venge de ses adversaires, et il garde sa colère contre ses ennemis.
\VS{3}Yahweh est lent à la colère, et grand par sa force, mais il ne tient nullement le coupable pour innocent. Yahweh marche parmi les tourbillons et les tempêtes, les nuées sont la poussière de ses pieds.
\VS{4}Il réprimande la mer et la fait tarir, il dessèche tous les fleuves~; le Basan et le Carmel languissent, la fleur du Liban languit.
\VS{5}Les montagnes tremblent à cause de lui, et les collines se fondent~; la terre se soulève devant sa présence, dis-je, et tous ceux qui y habitent.
\VS{6}Qui subsistera devant son indignation~? Et qui demeurera ferme dans l'ardeur de sa colère~? Sa fureur se répand comme un feu, et les rochers se brisent devant lui.
\VS{7}Yahweh est bon, il est une forteresse au jour de la détresse, et il connaît ceux qui se confient en lui.
\VS{8}Il s'en va passer comme un débordement d'eaux~; il réduira son lieu à néant, et fera que les ténèbres poursuivront ses ennemis.
\TextTitle{Jugement contre les ennemis de Yahweh}
\VS{9}Que projetez-vous contre Yahweh~? C'est lui qui réduit à néant~; la détresse ne se lèvera pas deux fois~;
\VS{10}car entrelacés comme des épines, et ivres de leur vin, ils seront consumés entièrement comme la paille sèche.
\VS{11}De toi est sorti celui qui méditait du mal contre Yahweh, et qui avait de mauvais desseins.
\VS{12}Ainsi parle Yahweh~: Bien qu'ils soient en paix et en grand nombre, ils seront certainement retranchés, et on passera outre. Or je t'ai affligé, mais je ne t'affligerai plus.
\VS{13}Maintenant je briserai son joug de dessus toi, et je détacherai tes liens.
\VS{14}Voici ce qu'a ordonné Yahweh contre toi~: Tu n'auras plus de semence qui porte ton nom~; je retrancherai de la maison de tes dieux les images taillées et en fonte~; j'en ferai ta tombe parce que tu es insignifiant.
\Chap{2}
\TextTitle{Délivrance et réjouissance}
\VerseOne{}Voici sur les montagnes les pieds de celui qui apporte de bonnes nouvelles\FTNT{Es. 40:9~; 52:7~; Ro. 10:15.}, qui publie la paix~! Toi Juda, célèbre tes fêtes solennelles, accomplis tes vœux~; car à l'avenir les hommes violents ne passeront plus au milieu de toi, ils sont entièrement retranchés.
\TextTitle{Récit de la destruction de Ninive}
\VS{2}Le destructeur est monté contre toi~; garde la forteresse~! Veille sur la route~! Affermis tes reins~! Consolide toute ta force~!
\VS{3}Car Yahweh se détourne de la majesté de Jacob et d'Israël~; parce que les dévastateurs les ont vidés, et qu'ils ont détruit leurs sarments.
\VS{4}Le bouclier de ses hommes forts est rouge~; ses hommes puissants sont teints de pourpre~; le fer des chars étincelle au jour qu'il a fixé pour la bataille, et les lances sont agitées.
\VS{5}Les chars s'élancent avec rapidité dans les rues, ils se précipitent sur les places, ils sont comme des flambeaux, et courent comme des éclairs.
\VS{6}Il se souvient de ses hommes vaillants, mais ils chancellent dans leur marche. Ils se hâtent vers les murs, et ils se préparent à la défense.
\VS{7}Les portes des fleuves sont ouvertes, et le palais se fond.
\VS{8}C'est fixé~: Elle est découverte et emportée~; ses servantes gémissent de leur voix comme des colombes, frappant leurs poitrines comme un tambour.
\VS{9}Or Ninive, depuis qu'elle a été bâtie, a été comme un vivier d'eaux~; mais ils s'enfuient… Arrêtez-vous~! Arrêtez-vous~! Mais il n'y a personne qui tourne le visage…
\VS{10}Pillez l'argent~! Pillez l'or~! Il y a des dispositions sans fin, des richesses en objets précieux.
\VS{11}On pille, on dévaste, on ravage~! Et les cœurs se fondent, leurs genoux se heurtent l'un contre l'autre. Que le tourment soit dans les reins de tous~! Et que leurs visages deviennent noirs comme un pot qui a été mis sur le feu~!
\VS{12}Où est le repaire des lions, le pâturage des lionceaux, dans lequel se retiraient les lions, et où se tenaient le lion et la lionne, et le petit du lion, sans qu'aucun ne les effraie~?
\VS{13}Les lions ravissaient tout ce qu'il fallait pour leurs petits, et étranglaient les bêtes pour leurs lionnes, ils remplissaient leurs tanières de proies, et leurs repaires de dépouilles.
\VS{14}Voici, j'en veux à toi, dit Yahweh des armées, je brûlerai tes chars, et ils s'en iront en fumée, l'épée consumera tes lionceaux. Je retrancherai de la terre ta proie, et la voix de tes messagers ne sera plus entendue.
\Chap{3}
\TextTitle{Méchanceté de Ninive}
\VerseOne{}Malheur à la ville sanguinaire qui est pleine de mensonge, pleine de violence~; la rapine ne s'en retirera point,
\VS{2}ni le bruit du fouet, ni le bruit impétueux des roues, ni le galop des chevaux, ni le saut des chars~;
\VS{3}ni les cavaliers montant leurs chevaux, ni l'épée flamboyante, ni la lance étincelante, ni la multitude des blessés, ni le grand nombre de cadavres. Des corps morts à l'infini, on trébuche sur un grand nombre de corps morts~!
\VS{4}à cause de la multitude des prostitutions de cette prostituée, pleine de charmes, experte en sortilèges, qui vendait les nations par ses prostitutions, et les familles par ses enchantements.
\VS{5}Voici, j'en veux à toi, dit Yahweh des armées, je relèverai ta robe jusqu'à ton visage~; je manifesterai ta nudité aux nations, et ton ignominie aux royaumes.
\VS{6}Je ferai tomber sur ta tête la peine de tes abominations, je te consumerai et je te donnerai en spectacle.
\VS{7}Et il arrivera que quiconque te verra, s'éloignera de toi et dira~: Ninive est détruite~! Qui aura compassion d'elle~? D'où te chercherai-je des consolateurs~?
\VS{8}Vaux-tu mieux que No-Amon, qui est assise au milieu des fleuves, qui a des eaux autour d'elle, dont la mer est le rempart, et à qui la mer sert de murailles~?
\VS{9}L'Ethiopie et l'Egypte étaient sa force, et une infinité d'autres peuples~; Puth et les Lybiens sont allés à son secours.
\VS{10}Elle-même aussi est transportée hors de sa terre, elle s'en est allée en captivité~; ses enfants ont été écrasés aux carrefours de toutes les rues, et on a jeté le sort sur ses gens honorables, et tous ses grands ont été liés de chaînes.
\VS{11}Toi aussi, tu seras enivrée, tu te tiendras cachée, et tu chercheras un refuge contre l'ennemi.
\VS{12}Toutes tes forteresses seront comme des figues, et comme des premiers fruits qui étant secoués, tombent dans la bouche de celui qui veut les manger.
\VS{13}Voici, ton peuple sera comme autant de femmes au milieu de toi~; les portes de ton pays seront toutes ouvertes, elles seront ouvertes à tes ennemis~; le feu consumera tes verrous.
\VS{14}Puise-toi de l'eau pour le siège~! Fortifie tes remparts~! Enfonce le pied dans la boue, foule l'argile~! Et fortifie le four à brique~!
\VS{15}Là, le feu te consumera, l'épée te retranchera, elle te dévorera comme la sauterelle dévore les arbres. Multiplie-toi comme les sauterelles~! Multiplie-toi comme les sauterelles~!
\VS{16}Tu as multiplié le nombre de tes marchands plus que les étoiles des cieux~; les sauterelles s'étant répandues, ont tout ravagé, et puis se sont envolées.
\VS{17}Tes princes et leurs scribes sont comme des sauterelles qui campent dans les murs au temps de la froidure~: Le soleil paraît, elles s'envolent et on ne reconnaît plus le lieu où elles étaient.
\VS{18}Tes bergers se sont endormis, ô roi d'Assyrie~! Tes grands hommes se tiennent dans leurs tentes~; ton peuple est dispersé par les montagnes, et il n'y a personne qui le rassemble.
\VS{19}Il n'y a point de remède à ta blessure, ta plaie est douloureuse~; tous ceux qui entendront parler de toi battront des mains sur toi~; car qui n'a pas continuellement éprouvé les effets de ta méchanceté~?
\PPE{}
\end{multicols}

%\clearpage\ShortTitle{Habakuk}\BookTitle{Habakuk}\BFont
\noindent\hrulefill
{\footnotesize
\textit{
\bigskip
{\centering{}
\\Signifie : Embrasser, amour
\\Thème : Du doute à la foi
\\Auteur : Habakuk
\\Date de rédaction : 7ème siècle av. J.-C.\\}
}
%\bigskip
\textit{
\\Habakuk, contemporain de Nahum, Sophonie et Jérémie, exerça son ministère dans le royaume de Juda. Véritable sentinelle, il fut chargé d’annoncer le châtiment de Juda par les chaldéens. Ce récit, qui est en partie un dialogue entre Dieu et Habakuk, témoigne de la relation qui les liait. Il est aussi une invitation à la patience et la foi en Yahweh.\bigskip
}
}
\par\nobreak\noindent\hrulefill
\begin{multicols}{2}
\Chap{1}
\TextTitle{[Quand la méchanceté semble triompher de la justice]}
\VerseOne{}Oracle qu’Habakuk, le prophète a vu.
\VS{2}Ô Yahweh ! Jusqu’à quand crierai-je sans que tu m'écoutes ? Jusqu'à quand crierai-je vers toi ? On me traite avec violence sans que tu me délivres !
\VS{3}Pourquoi me fais-tu voir la méchanceté\FTNT{La perplexité d’Habakuk était la même que celle de Job (Job. 21:7), d’Asaph (Ps. 73) et de Jérémie (Jé. 12:1-2). Les méchants semblent prospérer tandis que les justes pleurent et sont persécutés (Mal. 3 : 12-15).}, et vois-tu la perversité ? Pourquoi y a-t-il de l’oppression et de la violence devant moi, et des gens qui excitent des procès et des querelles ?
\VS{4}Parce que la loi est sans force, et que la justice ne se fait jamais, à cause de cela le méchant environne le juste, et à cause de cela on rend des jugements corrompus\FTNT{Jé. 5:26 ; Am. 5:7.}.
\TextTitle{[La réponse de Yahweh]}
\VS{5}Regardez parmi les nations, et voyez, et soyez étonnés et stupéfaits ! Car je vais faire en vos jours une œuvre que vous ne croiriez pas si on vous la racontait\FTNT{Ac. 13:41.}.
\VS{6}Car voici, je vais susciter les Chaldéens, ce peuple cruel et impétueux, marchant sur l'étendue de la terre, pour posséder des demeures qui ne lui appartiennent pas.
\VS{7}Il est redoutable et terrible, son gouvernement et son autorité viennent de lui-même .
\VS{8}Ses chevaux sont plus légers que les léopards, et ils ont la vue plus aiguë que les loups du soir ; et ses cavaliers se répandront çà et là, même ses cavaliers viendront de loin ; ils voleront comme un aigle qui fond sur sa proie\FTNT{Jé. 5:6 ; So. 3:3.}.
\VS{9}Ils viendront tous pour la violence ; ce qu'ils engloutiront de leurs regards sera porté vers l'orient, et ils amasseront les prisonniers comme du sable.
\VS{10}Ce peuple se moque des rois, et les princes sont l’objet de ses railleries ; il se rit de toutes les forteresses ; il amoncelle de la terre, et il s’en empare.
\VS{11}Alors il traverse comme le vent, il passe outre et se rend coupable, car sa force est son dieu.
\TextTitle{[La souveraineté de Dieu]}
\VS{12}N'es-tu pas de toute éternité, ô Yahweh ! Mon Dieu, mon Saint ? Nous ne mourrons point ! Ô Yahweh, tu l'as établi pour exécuter tes jugements ; et toi, mon rocher\FTNT{Voir commentaire en  Es. 8:13-17.}, tu l'as fondé pour punir.
\VS{13}Tu as les yeux trop purs pour voir le mal, et tu ne saurais prendre plaisir à regarder le mal qu'on fait à autrui. Pourquoi regarderais-tu les perfides, et te tairais-tu quand le méchant dévore son prochain qui est plus juste que lui ?
\VS{14}Or tu as fait les hommes comme les poissons de la mer, et comme le reptile qui n'a point de maître.
\VS{15}Il a tout enlevé avec l'hameçon ; il l'a amassé avec son filet, et l'a assemblé dans son rets ; c'est pourquoi il se réjouira et s'égayera\FTNT{Am. 4:2.}.
\VS{16}A cause de cela, il sacrifie à son filet, et il offre de l’encens à ses rets, parce qu'il aura eu par leur moyen une grasse portion, et que sa viande est une chose moelleuse.
\VS{17}Videra-t-il à cause de cela son filet ? Et ne cessera-t-il jamais de faire le carnage des nations ?
\TextTitle{[S'attendre à Yahweh]}
\Chap{2}
\VerseOne{}Je me tenais en sentinelle, j'étais debout dans la forteresse et je faisais le guet, pour voir ce qu’il me dira, et ce que je répondrais après ma plainte\FTNT{Jé. 6:17 ; Es. 21:1-6 ; Ez. 33:1-19.}.
\TextTitle{[Le juste vivra par la foi]}
\VS{2}Et Yahweh m'a répondu et m'a dit : Ecris la vision, et grave-la sur des tablettes, afin qu'on la lise couramment.
\VS{3}Car la vision est encore différée jusqu'à un certain temps, et Yahweh parlera de ce qui arrivera à la fin, et il ne mentira point. S'il tarde, attends-le, car il ne manquera point de venir, et il ne tardera point\FTNT{Hé. 10:37.}.
\VS{4}Voici, l'âme de celui qui s'élève n'est point droite en lui ; mais le juste vivra de sa foi\FTNT{Ro. 1:17 ; Hé. 10:38.}.
\VS{5}Et combien plus l'homme adonné au vin est-il perfide, et l'homme puissant est-il orgueilleux, ne se tenant point tranquille chez lui ; il élargit son âme comme le scheol, et il est insatiable comme la mort, il rassemble vers lui toutes les nations, et réunit à lui tous les peuples.
\VS{6}Tous ceux-là ne feront-ils pas de lui un sujet de raillerie et d’énigmes ? Et ne dira-t-on pas : Malheur à celui qui accumule ce qui ne lui appartient point ; jusqu'à quand le fera-t-il, et entassera-t-il sur lui de la boue épaisse ?
\VS{7}Ne se lèveront-ils pas soudain, ceux qui le mordront ? Ne se réveilleront-ils pas pour te tourmenter ? Et tu deviendras leur proie.
\VS{8}Parce que tu as pillé beaucoup de nations, tout le reste des peuples te pillera, et à cause aussi des meurtres des hommes, et de la violence faite dans le pays, contre la ville, et contre tous ses habitants\FTNT{Es. 33:1 ; Na. 3:1.}.
\VS{9}Malheur à celui qui amasse pour sa maison des gains injustes, afin de placer son nid dans un lieu élevé, pour échapper à l’atteinte de la calamité !
\VS{10}C’est pour la confusion de ta maison que tu as pris conseil, en détruisant beaucoup de peuples, et c’est contre ton âme que tu as péché.
\VS{11}Car la pierre crie du milieu de la muraille, et de la charpente la poutre lui répond.
\VS{12}Malheur à celui qui bâtit des villes avec le sang et qui fonde des cités sur l'iniquité.
\VS{13}Voici, n'est-ce pas la volonté de Yahweh des armées que les peuples travaillent pour le feu, et que les peuples se lassent pour le néant ?
\VS{14}Car la terre sera remplie de la connaissance de la gloire de Yahweh\FTNT{Es. 11:9.}, comme le fond de la mer par les eaux qui le couvrent.
\VS{15}Malheur à celui qui fait boire son compagnon en lui approchant sa bouteille, et qui l’enivre afin qu'on voie sa nudité\FTNT{Es. 5:22 ; Ge. 9:21-24.}.
\VS{16}Tu seras rassasié de honte plutôt que de gloire ; toi aussi bois, et découvre-toi. La coupe de la droite de Yahweh fera le tour jusqu’à toi, et l'ignominie sera répandue sur ta gloire.
\VS{17}Car la violence faite au Liban retombera sur toi ; et les ravages des bêtes t’effrayeront, parce que tu as répandu le sang des hommes, et commis  des violences dans le pays, contre la ville et tous ses habitants.
\VS{18}A quoi sert l'image taillée  pour qu’un ouvrier la taille ? A quoi sert l’image de fonte, docteur de mensonge, a quoi sert-elle pour que l'ouvrier qui l’a faite place en elle sa confiance en fabriquant des idoles muettes ?
\VS{19}Malheur à ceux qui disent au bois : Réveille-toi ! Et à la pierre muette : Réveille-toi ! Enseignera-t-elle ? Voici, elle est couverte d'or et d'argent, et il n'y a aucun esprit au-dedans d’elle.
\VS{20}Mais Yahweh est dans le temple de sa sainteté. Que toute la terre fasse silence devant lui !
\TextTitle{[Habakuk reconnait et accepte la volonté de Dieu]}
\Chap{3}
\VerseOne{}Prière d'Habakuk, le prophète, sur le mode des chants lyriques.
\VS{2}Yahweh, j'ai entendu ce que tu m'as fait entendre, et j'ai été saisi de crainte, ô Yahweh ! Dans le cours des années, ravive ton œuvre ; dans le cours des années, fais-la connaître; dans ta colère souviens-toi de tes compassions.
\VS{3}Dieu vient de Théman, et le Saint vient du mont de Paran. Sélah. Sa majesté couvre les cieux, et la terre est remplie de sa louange.
\VS{4}Sa splendeur est comme la lumière même, et des rayons sortent de sa main ; c'est là où réside sa force.
\VS{5}La peste marche devant lui, et une  flamme ardente sort sous ses pieds.
\VS{6}Il s'arrête et mesure la terre ; il regarde et met en déroute les nations ; les montagnes antiques sont  brisées en éclats,  et les collines éternelles s’affaissent. Ses voies sont les voies anciennes.
\VS{7}Je vois les tentes de Cuschan accablées sous la punition ; les pavillons du pays de Madian sont ébranlés.
\VS{8}Est-ce contre les fleuves que s’irrite Yahweh ? Ta colère est-elle contre les fleuves, et ta fureur contre la mer, que tu sois monté sur tes chevaux et sur tes chars de délivrance ?
\VS{9}Ton arc est mis à nu et tire toutes les flèches, selon le serment fait aux tribus, à savoir ta parole. Sélah. Tu fends la terre et tu en fais sortir des fleuves\FTNT{Ps. 78:15-16 ; Ps. 105:41.}.
\VS{10}Les montagnes te voient et elles tremblent\FTNT{Ps. 114:4-7.}; des torrents d’eau se précipitent, l'abîme fait retentir sa voix de la profondeur, il élève ses mains en haut.
\VS{11}Le soleil et la lune s'arrêtent dans leur habitation\FTNT{Jos. 10:12 ; Ap. 22:5.}, ils marchent à la lueur de tes flèches, et à la splendeur de l'éclat de ta lance étincelante.
\VS{12}Tu marches sur la terre avec indignation, et foules les nations avec colère.
\VS{13}Tu sors pour la délivrance de ton peuple, tu sors avec ton Oint pour la délivrance ; tu transperces le chef, afin qu'il n'y en ait plus dans la maison du méchant, tu en découvres le fondement  jusqu’au fond. Sélah.
\VS{14}Tu perces avec ses flèches  la tête de ses chefs, quand ils viennent comme une tempête pour me dissiper ; ils s'égaient comme pour dévorer l'affligé dans sa retraite.
\VS{15}Tu marches avec tes chevaux par la mer, les grandes eaux ayant été amoncelées.
\VS{16}J'ai entendu ce que tu m'as déclaré, et mes entrailles en sont émues ; à ta voix le tremblement saisit mes lèvres ; la pourriture entre dans mes os, et je tremble en moi-même, car je serai en repos au jour de la détresse, lorsque montant vers le peuple, il le mettra en pièces.
\VS{17}Car le figuier ne fleurira pas, et il n'y aura point de fruit dans les vignes ; ce que l'olivier produit mentira, et aucun champ ne produira rien à manger ; les brebis seront retranchées du parc, et il n'y aura point de bœufs dans les étables.
\VS{18}Mais moi, je me réjouis en Yahweh, et je me réjouis dans le Dieu de ma délivrance.
\VS{19}Yahweh, le Seigneur, est ma force, et il rend mes pieds semblables à ceux des biches, et me fait marcher sur mes lieux élevés\FTNT{Ps. 18:33-34 ; De. 32:13.}. Au chef des chantres avec instruments à cordes.
\PPE{}
\end{multicols}

%\clearpage\ShortTitle{So.}\BookTitle{Sophonie}\BFont
\noindent\hrulefill
{\footnotesize
\textit{
\bigskip
{\centering{}
\\Auteur~: Sophonie
\\(Heb.~: Tsephanyah)
\\Signification~: Yahweh a caché, protégé
\\Thème~: Le jour de Yahweh
\\Date de rédaction~: 7\up{ème} siècle av. J.-C.\\}
}
\textit{
\\De lignée royale, Sophonie exerça son service dans le royaume de Juda au temps du roi Josias et fut contemporain de Jérémie, Habakuk, Ezéchiel et Abdias. A une époque où l'iniquité s'était accrue au point où les quelques personnes fidèles à Dieu étaient persécutées, Sophonie fut suscité par Yahweh pour annoncer le jugement de Juda, d'Israël et de quelques nations païennes.\bigskip
}
}
\par\nobreak\noindent\hrulefill
\begin{multicols}{2}
\Chap{1}
\TextTitle{Yahweh annonce son jugement sur Juda, conséquence de son idolâtrie}
\VerseOne{}C'est ici la parole de Yahweh qui fut adressée à Sophonie, fils de Cuschi, fils de Guedalia, fils d'Amaria, fils d'Ezéchias, du temps de Josias, fils d'Amon, roi de Juda.
\VS{2}Je ferai entièrement périr toutes choses de dessus cette terre, dit Yahweh.
\VS{3}Je ferai périr l'homme et le bétail~; je consumerai les oiseaux des cieux et les poissons de la mer~; et la ruine arrivera aux méchants, et je retrancherai les hommes de dessus cette terre, dit Yahweh.
\VS{4}J'étendrai ma main sur Juda, et sur tous les habitants de Jérusalem~; je retrancherai de ce lieu-ci le reste de Baal\FTNT{Voir commentaire en Jg. 2:13.}, les noms des prêtres des faux dieux, les prêtres,
\VS{5}ceux qui se prosternent sur les toits devant l'armée des cieux, ceux qui se prosternent devant Yahweh, qui jurent par lui, et qui jurent aussi par Malcom\FTNT{2 R. 17:33~; 2 R. 23:11-12~; Jé. 19:13.},
\VS{6}ceux qui se détournent de Yahweh, ceux qui n'ont point cherché Yahweh, qui ne l'ont point consulté.
\VS{7}Silence, à cause de la présence du Seigneur Yahweh, car le jour de Yahweh est proche\FTNT{Voir commentaire en Za. 14:1.}~; Yahweh a préparé le sacrifice, il a invité ses conviés.
\VS{8}Et il arrivera au jour du sacrifice de Yahweh que je punirai les chefs, et les enfants du roi, et tous ceux qui portent des vêtements étrangers.
\VS{9}Et je punirai, en ce jour-là, tous ceux qui sautent par-dessus le seuil, et ceux qui remplissent de violence et de fraude la maison de leurs maîtres.
\VS{10}Et en ce jour-là dit Yahweh, il y aura de grands cris vers la porte des poissons, et des hurlements vers la seconde partie de la ville, et une grande désolation sur les collines.
\VS{11}Vous qui habitez dans Macthesch\FTNT{Macthesch était un bas-quartier de Jérusalem où se trouvaient les marchés.}, hurlez~! Car tous ceux qui trafiquaient ont été détruits, et tous ceux qui apportaient de l'argent ont été retranchés.
\VS{12}Et il arrivera en ce temps-là que je fouillerai Jérusalem avec des lampes, que je punirai les hommes qui sont figés sur leurs lies, et qui disent dans leurs cœurs~: Yahweh ne nous fera ni bien ni mal.
\VS{13}Leurs biens seront au pillage et leurs maisons en désolation~; et ils auront bâti des maisons, mais ils ne les habiteront pas~; ils auront planté des vignes, mais ils n'en boiront pas le vin.
\VS{14}Le grand jour de Yahweh est proche, il est proche, et il se hâte beaucoup~; le jour de Yahweh n'est que bruit~; celui qui est dans l'amertume, crie de toute sa force. Là sont les hommes vaillants\FTNT{Jé. 30:7~; Joë. 2:11~; Am. 5:18.}.
\VS{15}Ce jour est un jour de fureur, un jour de détresse et d'angoisse, un jour de bruit éclatant et effrayant, un jour de ténèbres et d'obscurité, un jour de nuées et de brouillards~;
\VS{16}un jour de shofar et de cris de guerre contre les villes fortifiées, et contre les hautes tours.
\VS{17}Je mettrai les hommes dans la détresse, et ils marcheront comme des aveugles, parce qu'ils ont péché contre Yahweh~; et leur sang sera répandu comme de la poussière, et leur chair comme des ordures.
\VS{18}Ni leur argent ni leur or ne pourront les délivrer au jour de la fureur de Yahweh~; et tout ce pays sera dévoré par le feu de sa jalousie, car il se hâtera de consumer tous les habitants de ce pays\FTNT{Ez. 7:19~; Pr. 11:4.}.
\Chap{2}
\TextTitle{Yahweh invite Israël à la repentance}
\VerseOne{}Examinez-vous, examinez-vous avec soin ô nation non désirée\FTNT{1 Th. 5:21~; 2 Co. 13:5~; Ep. 5:10.}~!
\VS{2}Avant que le décret enfante, et que le jour passe comme la balle~; avant que l'ardeur de la colère de Yahweh vienne sur vous, avant que le jour de la colère de Yahweh vienne sur vous~!
\VS{3}Vous, tous les pauvres du pays, qui faites ce qu'il ordonne, cherchez Yahweh, cherchez la justice, cherchez l'humilité~; peut-être serez-vous protégés au jour de la colère de Yahweh\FTNT{Am. 5:15.}.
\VS{4}Mais Gaza sera abandonnée, et Askalon sera en désolation~; on chassera les habitants d'Asdod en plein midi, et Ekron sera arrachée\FTNT{Am. 8:9~; Za. 9:5.}.
\VS{5}Malheur aux habitants de la contrée maritime, à la nation des Kéréthiens~! La parole de Yahweh est contre vous~; Canaan, qui est le pays des Philistins, je te détruirai, si bien que, personne n'y habitera.
\VS{6}Et la contrée maritime sera des pâturages, des demeures pour les bergers, et des parcs pour les troupeaux.
\VS{7}Et cette contrée sera pour le reste de la maison de Juda~; ils paîtront dans ces lieux-là, et le soir ils feront leur gîte dans les maisons d'Askalon~; car Yahweh, leur Dieu, les visitera, et il ramènera leurs captifs.
\VS{8}J'ai entendu les insultes de Moab, et les outrages des fils d'Ammon, quand ils ont diffamé mon peuple, et l'ont bravé sur leur frontière\FTNT{Ez. 25:3-6.}.
\VS{9}C'est pourquoi, je suis vivant, dit Yahweh des armées, le Dieu d'Israël, Moab sera comme Sodome, et les fils d'Ammon comme Gomorrhe, un lieu couvert d'orties, et une carrière de sel et de désolation à jamais~; les restes de mon peuple les pilleront, et les restes de ma nation les posséderont.
\VS{10}Ceci leur arrivera en échange de leur orgueil, parce qu'ils ont usé d'insultes et d'arrogance, contre le peuple de Yahweh des armées\FTNT{Es. 16:6~; Jé. 48:29.}.
\VS{11}Yahweh sera terrible contre eux, car il anéantira tous les dieux du pays~; et on se prosternera devant lui, chacun de son lieu, même dans toutes les îles des nations\FTNT{Mal. 1:11~; Jn. 4:21.}.
\VS{12}Vous aussi, habitants de Cusch, vous serez blessés à mort par mon épée.
\VS{13}Il étendra aussi sa main sur le nord, et il détruira l'Assyrie, et il fera de Ninive une désolation, dans un lieu aride comme un désert.
\VS{14}Et les troupeaux feront leur gîte au milieu d'elle, et toutes les bêtes des nations, même le pélican et le hérisson, habiteront parmi les chapiteaux de ses colonnes~; la voix des oiseaux retentira à la fenêtre, la désolation sera au seuil, parce qu'il en aura abattu les cèdres\FTNT{Es. 14:23~; Es. 34:11.}.
\VS{15}C'est là cette ville remplie de joie, qui se tenait assurée, et qui disait en son cœur~: C'est moi, et il n'y en a point d'autre que moi~! Comment a-t-elle été réduite en désert, pour être le repère des bêtes~? Quiconque passera près d'elle sifflera et secouera sa main.
\Chap{3}
\TextTitle{Israël persiste dans l'immoralité}
\VerseOne{}Malheur à la ville immonde et souillée et qui ne fait qu'opprimer~!
\VS{2}Elle n'a point écouté la voix, elle n'a point reçu d'instruction, elle ne s'est point confiée en Yahweh, elle ne s'est point approchée de son Dieu.
\VS{3}Ses chefs au milieu d'elle sont des lions rugissants, et ses juges sont des loups du soir, qui ne gardent pas les os pour les ronger le matin\FTNT{Ez. 22:27~; Pr. 28:15.}.
\VS{4}Ses prophètes sont des téméraires, et des hommes infidèles~; ses prêtres ont souillé les choses saintes, ils ont fait violence à la loi\FTNT{Jé. 23:11-32.}.
\VS{5}Yahweh est juste au milieu d'elle, il ne commet point d'iniquité\FTNT{De. 32:4.}. Chaque matin il met en lumière son jugement, il n'y manque pas~; mais celui qui est inique ne sait ce que c'est que d'avoir honte.
\VS{6}J'ai exterminé les nations, et leurs forteresses ont été désolées~; j'ai rendu désertes leurs places, si bien que personne n'y passe~; leurs villes ont été détruites, sans qu'il y soit resté un seul homme, et sans qu'il y ait aucun habitant.
\VS{7}Et je disais~: Au moins tu me craindras, tu recevras instruction, et sa demeure ne sera pas retranchée, quelque soit la punition que je lui envoie. Mais ils se sont levés de bon matin, ils ont corrompu toutes leurs actions.
\VS{8}C'est pourquoi attendez-moi, dit Yahweh, au jour où je me lèverai pour le butin~; car j'ai résolu de rassembler les nations et de réunir les royaumes, pour répandre sur eux mon indignation, et toute l'ardeur de ma colère~; car tout le pays sera dévoré par le feu de ma jalousie.
\TextTitle{Un reste trouve refuge en Yahweh}
\VS{9}Alors je transformerai les langues\FTNT{Il est question ici de la conversion des peuples issus des nations (Ap. 7:9-17).} des nations en des langues pures, afin qu'elles invoquent toutes le Nom de Yahweh, pour qu'elles le servent d'un commun accord.
\VS{10}Mes adorateurs qui sont au-delà des fleuves de Cusch, à savoir la fille de mes dispersés, m'apporteront mes offrandes\FTNT{Es. 19:21~; Es. 27:13~; Ps. 68:31-32~; Ps. 72:10-11.}.
\VS{11}En ce jour-là, tu ne seras plus confuse à cause de toutes tes actions, par lesquelles tu as péché contre moi~; parce qu'alors j'aurai ôté du milieu de toi ceux qui se réjouissent de ton orgueil, et désormais tu ne t'enorgueilliras plus de la montagne de ma sainteté.
\VS{12}Et je laisserai au milieu de toi un peuple humble et faible, et il mettra sa confiance dans le Nom de Yahweh.
\VS{13}Les restes d'Israël ne commettront point d'iniquité, et ne proféreront point de mensonge, et il n'y aura point dans leur bouche de langue trompeuse~; aussi ils paîtront et se reposeront, et il n'y aura personne qui les épouvante.
\TextTitle{Israël délivré et restauré}
\VS{14}Réjouis-toi avec chant de triomphe, fille de Sion~! Pousse des cris de réjouissance, ô Israël~! Réjouis-toi et triomphe de tout ton cœur, fille de Jérusalem~!
\VS{15}Yahweh a aboli ta condamnation, il a éloigné ton ennemi. Le Roi d'Israël, Yahweh, est au milieu de toi~; tu ne verras plus de mal\FTNT{Ps. 46:5-6~; Col. 2:14.}.
\VS{16}En ce temps-là, on dira à Jérusalem~: Ne crains point Sion, que tes mains ne défaillent point~!
\VS{17}Yahweh, ton Dieu, est au milieu de toi comme le Puissant qui sauve~; il se réjouira à cause de toi d'une grande joie~; il se taira à cause de son amour, et se réjouira à cause de toi avec chant de triomphe.
\VS{18}Je rassemblerai ceux qui sont tristes à cause de l'assemblée solennelle, ils sont sortis de toi~; sur eux pèse l'opprobre.
\VS{19}Voici, je détruirai en ce temps-là tous ceux qui t'auront affligé~; je sauverai la boiteuse, je recueillerai celle qui avait été chassée, et je les ferai louer et devenir célèbres, dans tous les pays où ils auront été couverts de honte.
\VS{20}En ce temps-là, je vous ramènerai, et en ce temps-là je vous rassemblerai~; car je vous rendrai célèbres et un sujet de louange parmi tous les peuples de la terre, quand je ramènerai vos captifs sous vos yeux, dit Yahweh.
\PPE{}
\end{multicols}

%\clearpage\ShortTitle{Ag.}\BookTitle{Aggée}\BFont
\noindent\hrulefill
{\footnotesize
\textit{
\bigskip
{\centering{}
\\Auteur~: Aggée
\\(Heb.~: Chaggay)
\\Signification~: En fête, né un jour de fête
\\Thème~: Reconstruction du temple
\\Date de rédaction~: 6\up{ème} siècle av. J.-C.\\}
}
\textit{
\\Aggée, contemporain de Zacharie, exerça son service dans le royaume de Juda après le retour de l'exil. Alors que la reconstruction du temple était négligée, Aggée reçut un message rappelant au peuple quelles devaient être ses priorités et redéfinissant les exigences de Yahweh en matière de sainteté. Ce récit montre la bénédiction accompagnant celui qui oublie ses propres intérêts et qui prend véritablement à cœur l'œuvre de Dieu.\bigskip
}
}
\par\nobreak\noindent\hrulefill
\begin{multicols}{2}
\Chap{1}
\TextTitle{Israël coupable de négligence}
\VerseOne{}La seconde année du roi Darius, le premier jour du sixième mois, la parole de Yahweh vint par le moyen d'Aggée, le prophète, à Zorobabel, fils de Schealthiel, gouverneur de Juda, et à Josué, fils de Jotsadak, le grand-prêtre, en ces mots\FTNT{Esd. 4:24.}~:
\VS{2}Ainsi parle Yahweh des armées, en disant~:~Ce peuple dit : Le temps n'est pas encore venu, le temps de rebâtir la maison de Yahweh.
\VS{3}C'est pourquoi la parole de Yahweh a été adressée par le moyen d'Aggée, le prophète, en disant~:
\VS{4}Est-il temps pour vous d'habiter dans vos maisons lambrissées pendant que cette maison est en ruine~?
\VS{5}Maintenant donc, ainsi parle Yahweh des armées~: Considérez attentivement votre conduite~!
\VS{6}Vous avez semé beaucoup, mais vous avez récolté peu. Vous avez mangé, mais non pas jusqu'à être rassasiés. Vous avez bu, mais vous n'avez pas eu de quoi boire abondamment. Vous avez été vêtus, mais non pas jusqu'à en être échauffés. Et celui qui se loue, se loue pour mettre son salaire dans un sac percé\FTNT{Mi. 6:14-15.}.
\VS{7}Ainsi parle Yahweh des armées~: Considérez attentivement vos chemins~!
\VS{8}Montez à la montagne, apportez du bois, et bâtissez cette maison~; et j'y prendrai mon plaisir et je serai glorifié, a dit Yahweh.
\VS{9}Vous comptiez sur beaucoup, et voici, il y a eu peu~; vous l'avez apporté à la maison et j'ai soufflé dessus. Pourquoi~? A cause de ma maison, dit Yahweh des armées, parce qu'elle est en ruine pendant que vous vous empressez chacun pour sa maison.
\VS{10}A cause de cela, les cieux au-dessus de vous retiennent la rosée, et la terre a retenu ses fruits\FTNT{Lé. 26:19~; De. 28:23.}.
\VS{11}Et j'ai appelé la sécheresse sur la terre, et sur les montagnes, et sur le blé, et sur le moût, et sur l'huile, et sur tout ce que la terre produit, et sur les hommes et sur les bêtes, et sur tout le travail des mains\FTNT{Am. 4:7~; Ps. 105:16.}.
\TextTitle{Yahweh réveille son peuple}
\VS{12}Zorobabel donc, fils de Schealthiel, et Josué, fils de Jotsadak, le grand-prêtre, et tout le reste du peuple, entendirent la voix de Yahweh, leur Dieu, et les paroles d'Aggée, le prophète, ainsi que Yahweh, leur Dieu, l'avait envoyé~; et le peuple eut de la crainte devant Yahweh.
\VS{13}Et Aggée, messager de Yahweh, parla au peuple, suivant le message de Yahweh, en disant~: Je suis avec vous, dit Yahweh.
\VS{14}Et Yahweh réveilla l'esprit de Zorobabel, fils de Schealthiel, gouverneur de Juda, et l'esprit de Josué, fils de Jotsadak, le grand-prêtre, et l'esprit de tout le reste du peuple. Et ils vinrent et travaillèrent à la maison de Yahweh, leur Dieu,
\VS{15}le vingt-quatrième jour du sixième mois, de la seconde année du roi Darius.
\Chap{2}
\TextTitle{Encouragements à poursuivre la construction}
\VerseOne{}Le vingt et unième jour du septième mois, la parole de Yahweh vint par le moyen d'Aggée, le prophète, en disant~:
\VS{2}Parle maintenant à Zorobabel, fils de Schealthiel, gouverneur de Juda, et à Josué, fils de Jotsadak, le grand-prêtre, et au reste du peuple, en disant~:
\VS{3}Quel est parmi vous le survivant qui ait vu cette maison dans sa première gloire~? Et comment la voyez-vous maintenant~? N'est-elle pas comme un rien devant vos yeux, au prix de celle-là\FTNT{Esd. 3:12.}~?
\VS{4}Maintenant donc Zorobabel, fortifie-toi~! dit Yahweh. Toi aussi, Josué, fils de Jotsadak, grand-prêtre, fortifie-toi~! Vous aussi, tout le peuple du pays, fortifiez-vous~! dit Yahweh. Et travaillez, car je suis avec vous, dit Yahweh des armées.
\VS{5}La parole de l'Alliance que je traitai avec vous, quand vous sortîtes d'Egypte, et mon Esprit, demeurent au milieu de vous~; ne craignez point\FTNT{Za. 4:6.}~!
\VS{6}Car ainsi parle Yahweh des armées~: Encore un peu de temps, et j'ébranlerai les cieux et la terre, la mer et le sec\FTNT{Hé. 12:26.}~;
\VS{7}j'ébranlerai toutes les nations~; et les trésors de toutes les nations viendront, et je remplirai de gloire cette maison, dit Yahweh des armées.
\VS{8}L'argent est à moi, et l'or est à moi, dit Yahweh des armées.
\VS{9}La gloire de cette dernière maison sera plus grande que celle de la première, dit Yahweh des armées~; et je mettrai la paix en ce lieu, dit Yahweh des armées.
\TextTitle{Purification et sanctification du peuple}
\VS{10}Le vingt-quatrième jour du neuvième mois de la seconde année de Darius, la parole de Yahweh vint par le moyen d'Aggée le prophète, en disant~:
\VS{11}Ainsi parle Yahweh des armées~: Interroge maintenant les prêtres sur la loi en ces mots~:
\VS{12}Si quelqu'un porte de la chair consacrée dans le pan de son vêtement, et que ce vêtement touche du pain, ou un mets cuit, ou du vin, ou de l'huile, ou un aliment quelconque, cela devient-il sanctifié~? Et les prêtres répondirent et dirent~: Non~!
\VS{13}Alors Aggée dit~: Si celui qui est souillé pour un mort touche toutes ces choses-là, ne seront-elles pas souillées~? Et les prêtres répondirent et dirent~: Elles seront souillées\FTNT{Lé. 17:15~; No. 19:22~; Tit. 1:15.}.
\VS{14}Alors Aggée répondit et dit~: Tel est ce peuple et telle est cette nation devant ma face, dit Yahweh~; et telles sont toutes les œuvres de leurs mains~; même ce qu'ils offrent ici est souillé.
\VS{15}Maintenant donc mettez ceci, je vous prie, dans votre cœur, depuis ce jour et par la suite, avant qu'on ait mis pierre sur pierre au temple de Yahweh~!
\VS{16}Avant cela, dis-je, quand on venait à un monceau de blé, au lieu de vingt mesures, il n'y en avait que dix~; et quand on  venait au pressoir, au lieu de puiser de la cuve cinquante mesures, il n'y en avait que vingt.
\VS{17}Je vous ai frappés de brûlure, de rouille, de grêle, dans tout le travail de vos mains. Et vous n'êtes point revenus à moi, dit Yahweh\FTNT{De. 28:22~; 1 R. 8:37~; Am. 4:9~; 2 Ch. 6:28.}.
\VS{18}Mettez maintenant ceci dans votre cœur~; depuis ce jour-ci et dans la suite~; depuis, dis-je, le vingt-quatrième jour du neuvième mois, depuis le jour où les fondements du temple de Yahweh ont été posés, mettez ceci dans votre cœur~!
\VS{19}Ya-t-il encore de la semence dans les greniers~? Même jusqu'à la vigne, au figuier, au grenadier, et à l'olivier, rien n'a rapporté~; mais depuis ce jour-ci, je donnerai la bénédiction.
\TextTitle{Destruction des royaumes des nations}
\VS{20}Et la parole de Yahweh vint pour la seconde fois à Aggée, le vingt-quatrième jour du mois, en disant~:
\VS{21}Parle à Zorobabel, gouverneur de Juda, et dis-lui~: J'ébranlerai les cieux et la terre~;
\VS{22}je renverserai le trône des royaumes, je détruirai la force des royaumes des nations, je renverserai les chars et ceux qui les montent~; et les chevaux et ceux qui les montent seront abattus, chacun par l'épée de son frère.
\VS{23}En ce jour-là, dit Yahweh des armées, je te prendrai, ô Zorobabel, fils de Schealthiel, mon serviteur, dit Yahweh~; et je te mettrai comme un sceau\FTNT{Le sceau est un symbole d'autorité.}, car je t'ai choisi, dit Yahweh des armées.
\PPE{}
\end{multicols}

%\clearpage\ShortTitle{Zacharie}\BookTitle{Zacharie}\BFont
\noindent\hrulefill
{\footnotesize
\textit{
\bigskip
{\centering{}
\\Signifie : Yahweh se souvient
\\Thème : Les deux avènements du Messie
\\Auteur : Zacharie
\\Date de rédaction : 6ème siècle av. J.-C.\\}
}
%\bigskip
\textit{
\\Zacharie, contemporain d’Aggée, exerça son ministère en Juda au retour des exilés de Babylone, où il était né. Il annonça la venue du Messie et raconta de manière très précise différents épisodes de sa vie, également retrouvés dans le récit des évangiles. Il dévoila également quelques-uns des attributs du Sauveur, premièrement rejeté mais finalement accepté par le peuple juif pendant le millenium. On y découvre ainsi le Christ en tant que souverain sacrificateur, germe, serviteur, ange de l’Yahweh, roi de paix, fils de David…\bigskip
}
}
\par\nobreak\noindent\hrulefill
\begin{multicols}{2}
\Chap{1}
\TextTitle{Yahweh avertit son peuple}
\VerseOne{}Le huitième mois de la deuxième année de Darius, la parole de Yahweh fut adressée à Zacharie, le prophète, fils de Bérékia, fils d’Iddo, en ces mots :
\VS{2}Yahweh a été extrêmement irrité contre vos pères.
\VS{3}C'est pourquoi tu leur diras : Ainsi parle Yahweh des armées : Revenez à moi, dit Yahweh des armées, et je reviendrai à vous, dit Yahweh des armées\FTNT{Joë. 2:12 ; Es. 31:6 ; Jé. 3:12.}.
\VS{4}Ne soyez point comme vos pères, auxquels s’adressaient les premiers prophètes, en disant : Ainsi a dit Yahweh des armées : Détournez-vous maintenant de vos mauvaises voies et de vos mauvaises actions ! Mais ils n’écoutèrent pas, ils ne furent pas attentifs à ce que je leur disais, dit Yahweh\FTNT{2 Ch. 29:6 ; Esd. 9:7 ; Né. 9:16 ; La. 5:7.}.
\VS{5}Vos pères où sont-ils ? Et ces prophètes-là pouvaient-ils vivre éternellement ?
\VS{6}Cependant mes paroles et mes ordonnances que j'avais données aux prophètes, mes serviteurs, n'ont-elles pas atteint vos pères ? De sorte qu’étant revenus, ils ont dit : Yahweh des armées nous a traités comme il avait résolu de le faire, selon nos voies et nos actions.
\TextTitle{Le cavalier sur le cheval roux}
\VS{7}Le vingt-quatrième jour du onzième mois, qui est le mois de Schebat, la deuxième année de Darius, la parole de Yahweh fut adressée à Zacharie, le prophète, fils de Bérékia, fils d’Iddo, en ces mots :
\VS{8}Je voyais de nuit une vision, et voici, un homme était monté sur un cheval roux, et il se tenait parmi des myrtes qui étaient dans un lieu creux ; il y avait derrière lui des chevaux roux, fauves et blancs\FTNT{Ap. 6:2-4.}.
\VS{9}Je dis : Mon Seigneur ! Que signifient ces choses ? Et l’Ange qui me parlait me dit : Je te montrerai ce que signifient ces choses.
\VS{10}L’homme qui se tenait parmi les myrtes répondit et dit : Ce sont ceux que Yahweh a envoyés pour parcourir la terre.
\VS{11}Et ils répondirent à l'Ange de Yahweh\FTNT{Voir commentaire en Ge. 16:7.} qui se tenait parmi les myrtes, et dirent : Nous avons parcouru la terre ; et voici, toute la terre est en repos et tranquille.
\TextTitle{La compassion de Yahweh pour Jérusalem}
\VS{12}Alors l'Ange de Yahweh répondit et dit : Yahweh des armées, jusqu'à quand n'auras-tu pas compassion de Jérusalem et des villes de Juda, contre lesquelles tu es irrité depuis soixante-dix ans\FTNT{Jérémie prophétisa que la captivité babylonienne durerait soixante-dix ans (Jé. 25:11-12 ; Jé. 29:10). Les soixante-dix ans commencèrent à la déportation de la famille royale à Babylone en 605 av. J.-C. (2 R. 24 ; Da. 1) et se terminèrent avec la première vague de retours conduite par Zorobabel (Esd. 1). Les Israélites furent emmenés en captivité en plusieurs vagues. Le livre d’Esdras raconte les deux premières. En 538 av. J.-C., Zorobabel mena la première vague et fut nommé gouverneur (Ag. 1:1). Le sacrificateur Josué  (Esd. 3:2) et les prophètes Aggée et Zacharie (Es. 5:1-2) le secondaient. Leur plus grand défi fut de rebâtir le temple. Puisque la seule tribu à retourner en masse fut celle de Juda, dès lors, le reste du peuple fut appelé «~les Juifs~» (Esd. 4:23).} ?
\VS{13}Yahweh répondit à l'Ange qui me parlait, par de bonnes paroles, par des paroles de consolation.
\VS{14}Puis l'Ange qui me parlait me dit : Crie, en disant : Ainsi parle Yahweh des armées : Je suis ému d'une grande jalousie pour Jérusalem et pour Sion,
\VS{15}et je suis extrêmement irrité contre les nations qui sont à leur aise ; car je n’étais que peu irrité, mais elles ont contribué au mal.
\VS{16}C'est pourquoi ainsi parle Yahweh : Je reviens à Jérusalem avec compassion, et ma maison y sera rebâtie, dit Yahweh des armées ; et le cordeau sera étendu sur Jérusalem.
\VS{17}Crie encore, et dis : Ainsi parle Yahweh des armées : Mes villes regorgeront encore de biens, et Yahweh consolera encore Sion, il choisira encore Jérusalem.
\TextTitle{Les quatre cornes et les quatre forgerons}
\VS{18}Puis je levai les yeux et je regardai ; et voici, quatre cornes\FTNT{Da. 7:7-11 ; Da. 8:22 ; Ap. 13:1-11.}.
\VS{19}Alors je dis à l'Ange qui me parlait : Que veulent dire ces choses ? Et il me répondit : Ce sont les cornes qui ont dispersé Juda, Israël et Jérusalem.
\VS{20}Puis Yahweh me fit voir quatre forgerons.
\VS{21}Je dis : Que viennent-ils faire ? Et il répondit et dit : Ce sont les cornes qui ont dispersé Juda, au point que personne ne lève la tête ; et ces forgerons sont venus pour les effrayer, et pour abattre les cornes des nations qui ont levé la corne contre le pays de Juda, pour le disperser.
\Chap{2}
\TextTitle{L'homme tenant dans sa main le cordeau pour mésurer}
\VerseOne{}Je levai encore mes yeux et je regardai, et voici, il y avait un homme tenant dans la main un cordeau pour mesurer,
\VS{2}auquel je dis : Où vas-tu ? Et il me répondit : Je vais mesurer Jérusalem, pour voir quelle est sa largeur et quelle est sa longueur.
\VS{3}Et voici, l'Ange qui me parlait s’avança, et un autre ange sortit à sa rencontre.
\TextTitle{Yahweh, la gloire de Jérusalem}
\VS{4}Il lui dit : Cours, et parle à ce jeune homme, et dis : Jérusalem sera habitée comme les villes sans murailles, à cause de la multitude d'hommes et de bêtes qui seront au milieu d'elle\FTNT{Né. 1:3 ; Né. 2:13.}.
\VS{5}Mais je serai pour elle, dit Yahweh, une muraille de feu tout autour, et je serai sa gloire au milieu d'elle\FTNT{Es. 60:19.}.
\VS{6}Ha ! Fuyez, fuyez hors du pays du nord ! dit Yahweh. Car je vous ai dispersés aux quatre vents des cieux, dit Yahweh.
\VS{7}Ha ! Sauve-toi, Sion, toi qui habites chez la fille de Babylone\FTNT{Jé. 50:8 ; Es. 48:20 ; Es. 52:11 ; Jé. 51:6.} !
\VS{8}Car ainsi parle Yahweh des armées, lequel après la gloire, il m'a envoyé vers les nations qui ont fait de vous leur proie ; car celui qui vous touche, touche à la prunelle de son œil\FTNT{De. 32:10 ; Ps. 17:8.}.
\VS{9}Car voici, je vais lever ma main contre elles, et elles seront la proie de ceux qui leur étaient asservis. Et vous saurez que Yahweh des armées m'a envoyé.
\VS{10}Pousse des cris d’allégresse et réjouis-toi, fille de Sion ! Car voici, je viens\FTNT{Jésus-Christ est Yahweh qui vient ! C’est la seconde venue de Jésus-Christ qui est évoquée ici (Es. 40:10-11 ; Za. 12:10-14 ; Za. 14:1-10 ; Ac. 1:1-11 ; Ap. 1:7-8 ; Ap. 22:12-17).}, et j'habiterai au milieu de toi, dit Yahweh.
\VS{11}Beaucoup de nations se joindront à Yahweh en ce jour-là, et deviendront mon peuple ; et j'habiterai au milieu de toi ; et tu sauras que Yahweh des armées m'a envoyé vers toi.
\VS{12}Yahweh possédera Juda comme sa part dans la terre sainte, et il choisira encore Jérusalem.
\VS{13}Que toute chair fasse silence devant la face de Yahweh ! Car il s'est réveillé de sa demeure sainte.
\Chap{3}
\TextTitle{Yahweh enlève l'iniquité du pays}
\VerseOne{}Puis Yahweh me fit voir Josué, le souverain sacrificateur, se tenant debout devant l'Ange de Yahweh\FTNT{Voir commentaire en Ge. 16:7.}, et Satan qui se tenait debout à sa droite, pour l’accuser.
\VS{2}Yahweh dit à Satan : Que Yahweh te réprime, ô Satan ! Que Yahweh, dis-je, qui a choisi Jérusalem, te réprime ! N’est-ce pas là un tison qui a été retiré du feu\FTNT{Jud. 1:9 ; Am. 4:11.} ?
\VS{3}Or Josué était vêtu de vêtements sales, et il se tenait debout devant l'Ange.
\VS{4}L’Ange prit la parole et dit à ceux qui étaient debout devant lui : Otez-lui ces vêtements sales ! Et il dit à Josué : Regarde, je t’enlève ton iniquité, et je te revêts d’habits de fête.
\VS{5}Je dis : Qu'on mette sur sa tête un turban pur ! Et ils mirent un turban pur sur sa tête, puis ils lui mirent des vêtements\FTNT{Ap. 19:8.}. L’Ange de Yahweh était présent.
\VS{6}Alors l'Ange de Yahweh fit à Josué cette déclaration, en disant :
\VS{7}Ainsi parle Yahweh des armées : Si tu marches dans mes voies, et si tu observes mes commandements, tu jugeras ma maison, tu garderas mes parvis, et je te donnerai libre accès parmi ceux qui se tiennent devant moi.
\VS{8}Ecoute maintenant, Josué, souverain sacrificateur, toi, et tes compagnons qui sont assis devant toi ! Car ce sont des hommes qui serviront de signes. Certainement voici je ferai venir mon serviteur, le Germe\FTNT{Le Germe est un autre nom de Jésus-Christ, notre Seigneur (Es. 4:2).}.
\VS{9}Car voici, quant à la pierre\FTNT{Jésus-Christ est le Rocher des âges (Es. 8:13-17; Ap. 5:1-7).} que j'ai mise devant Josué, sur cette pierre, qui n'est qu'une\FTNT{Cette pierre est UNE (~E’had~) c’est-à-dire indivisible (De. 6:4).}, il y a sept yeux. Voici, je graverai moi-même ce qui doit y être gravé, dit Yahweh des armées ; et j'ôterai en un jour l'iniquité de ce pays.
\VS{10}En ce jour-là, dit Yahweh des armées, chacun de vous appellera son prochain sous la vigne et sous le figuier.
\Chap{4}
\TextTitle{Le peuple de Yahweh peut tout par son Esprit}
\VerseOne{}Puis l'Ange qui me parlait revint, et il me réveilla comme un homme que l’on réveille de son sommeil.
\VS{2}Il me dit : Que vois-tu ? Et je répondis : Je regarde, et voici, il y a un chandelier tout en or, surmonté d’un vase et portant ses sept lampes, avec sept conduits pour les sept lampes qui sont au sommet du chandelier\FTNT{Ap. 1:12-13.} ;
\VS{3}et il y a deux oliviers près de lui, l'un à la droite du vase, et l'autre à sa gauche.
\VS{4}Alors je pris la parole et je dis à l'Ange qui me parlait : Mon Seigneur, que signifient ces choses ?
\VS{5}L'Ange qui me parlait répondit et me dit : Ne sais-tu pas ce que signifient ces choses ? Je dis : Non, mon Seigneur !
\VS{6}Alors il reprit et me dit : C'est ici la parole que Yahweh adresse à Zorobabel : Ce n'est point par la puissance ni par la force, mais par mon Esprit, dit Yahweh des armées.
\VS{7}Qui es-tu, grande montagne, devant Zorobabel ? Tu seras aplanie. Il fera sortir la pierre principale ; il y aura des sons éclatants : Grâce, grâce pour elle !
\TextTitle{Yahweh encourage son peuple à achever l'oeuvre commencée}
\VS{8}Aussi la parole de Yahweh me fut adressée en ces mots :
\VS{9}Les mains de Zorobabel ont fondé cette maison, et ses mains l'achèveront ; et tu sauras que Yahweh des armées m'a envoyé vers vous.
\VS{10}Car qui est-ce qui a méprisé le jour des faibles commencements ? Ils se réjouiront en voyant le niveau dans la main de Zorobabel.  Ces sept\FTNT{Les sept yeux de Yahweh sont aussi les sept yeux de l’Agneau (Ap. 5 : 6). Ces yeux représentent l’omniscience et l’omniprésence de Jésus-Christ (Za. 3:9 ; Za 4:10. ; Za. 14:7 ; Jn. 16:30 ; Ac. 1:24 ; Ap. 21:17).} sont les yeux de Yahweh qui parcourent toute la terre.
\VS{11}Je pris la parole et je lui dis : Que signifient ces deux oliviers\FTNT{L’identité de ces deux individus est inconnue.  Selon Ap. 11:3, ces deux hommes recevront des pouvoirs incroyables pour les trois années et demi de la grande tribulation qui précéderont le retour du Christ. Si quiconque tente de leur faire du mal ou d’interférer dans leur ministère et leur témoignage, «~… du feu sortirait de leur bouche et dévorerait leurs ennemis~», (Ap. 11:5). Ils auront aussi le pouvoir de provoquer la sécheresse et la famine sur la terre, tout comme l’avait fait Elie (1 R. 17:1-7 ; 2 R. 1:9-15 ; Lu. 4:25). Ils auront également le pouvoir de frapper la Terre par des plaies diverses, semblables à celles provoquées par Moïse (Chapitres 7, 8, 9, 10, 11 d’Exode ; Ap. 11:6).}, à la droite et à la gauche du chandelier ?
\VS{12}Je pris la parole pour la seconde fois et je lui dis : Que signifient ces deux branches d'olivier qui sont près des deux conduits d'or, d’où l'or découle ?
\VS{13}Il me répondit et dit : Ne sais-tu pas ce que signifient ces choses ? Et je dis : Non, mon Seigneur.
\VS{14}Et il dit : Ce sont les  deux fils oints, qui se tiennent devant le Seigneur de toute la terre.
\Chap{5}
\TextTitle{La malédiction se répands sur Israël}
\VerseOne{}Puis je me retournai, et levai mes yeux pour regarder ; et voici, un rouleau qui volait.
\VS{2}Alors il me dit : Que vois-tu ? Je répondis : Je vois un rouleau qui vole, dont la longueur est de vingt coudées, et la largeur de dix coudées.
\VS{3}Et il me dit : C'est l’exécration du serment qui sort sur la face de tout le pays ; car selon elle, quiconque d'entre ce peuple-ci vole, sera puni comme elle ; et selon elle, quiconque d'entre ce peuple parjure, sera puni comme elle.
\VS{4}Je déploierai cette exécration dit Yahweh des armées, et elle entrera dans la maison du voleur, et dans la maison de celui qui jure faussement en mon Nom, et elle logera au milieu de leur maison, et la consumera avec son bois et ses pierres.
\TextTitle{L'épha au pays de Schinear}
\VS{5}L’Ange qui me parlait sortit, et me dit : Lève maintenant tes yeux, et regarde ce qui sort là.
\VS{6}Et je dis : Qu'est-ce ? Et il répondit : C'est l’épha\FTNT{L'épha était une unité de mesure utilisée dans le commerce des céréales, souvent à des fins frauduleuses (De. 25:14 ; Mi. 6:10 ; Am. 8:5).} qui sort dehors. Puis il dit : C'est ici leur aspect dans tout le pays.
\VS{7}Et voici, on portait une masse de plomb, et une femme était assise au milieu de l'épha\FTNT{Zacharie voit une femme assise au milieu de l'épha. L'ange déclare : «~C'est la méchanceté ou l’iniquité~». Elle représente la grande prostituée décrite en Ap. 17, avec sa coupe d'or pleine de ses abominations et des impuretés de sa fornication (v. 4). Cette femme est la figure du «~mystère de l'iniquité qui opère déjà~» (2 Th. 2:7).}.
\VS{8}Il dit : C'est là l’iniquité\FTNT{L’iniquité ou la méchanceté.} ; puis il la repoussa dans l'épha, et il jeta la masse de plomb sur l’ouverture.
\VS{9}Je levai les yeux et je regardai, et voici deux femmes sortirent\FTNT{Les deux femmes ayant «~des ailes de cigogne~» apparaissent portées par le vent. Sous Moïse, cet oiseau devait être considéré comme impur (Lé. 11 : 19). Dans les Ecritures, le vent est constamment en relation avec le jugement (Job. 27:20-22 ; Job. 30:22 ; Es. 7:2 ; Es. 26:6 ; Es. 41:16). Elles soulèvent l'épha et l'emportent dans son lieu d'origine, le pays de Schinear, c’est-à-dire Babylone, pour lui bâtir une maison, au siège même de l'idolâtrie et de la révolte contre Dieu. (Ge. 11:2-9 ; 2 R. 17:24).}. Le vent soufflait dans leurs ailes : Elles avaient des ailes comme les ailes de la cigogne. Et elles enlevèrent l'épha entre la terre et le ciel.
\VS{10}Je dis à l'Ange qui me  parlait : Où emportent-elles l'épha ?
\VS{11}Il me répondit : C'est pour lui bâtir une maison dans le pays de Schinear\FTNT{Schinear ou Babylone (Ge. 10:6-12).} ; et quand elle sera prête, il sera déposé là, sur sa base.
\Chap{6}
\TextTitle{Les quatre vents des cieux}
\VerseOne{}Je levai encore les yeux et je regardai, et voici quatre chars\FTNT{Dans les Ecritures, les chars et les chevaux représentent souvent la puissance de Dieu exerçant un jugement sur la terre (Jé. 46:9-10 ; Joë. 2:3-11). Ce jugement concerne le monde entier (Ap. 6:1-8).} sortaient d'entre deux montagnes ; et ces montagnes étaient des montagnes d'airain.
\VS{2}Au premier char, il y avait des chevaux roux ; au deuxième char, des chevaux noirs,
\VS{3}au troisième char, des chevaux blancs, et au quatrième char, des chevaux tachetés, rouges.
\VS{4}Je pris la parole et je dis à l'Ange qui me parlait : Mon Seigneur, que veulent dire ces choses ?
\VS{5}L’Ange répondit et me dit : Ce sont les quatre vents des cieux, qui sortent du lieu où ils se tenaient devant le Seigneur de toute la terre.
\VS{6}Quant au char où sont les chevaux noirs, ils se dirigent vers le pays du nord, et les blancs sortent après eux ; les tachetés se dirigent vers le pays du midi.
\VS{7}Ensuite les rouges sortirent et demandèrent à aller parcourir la terre. L’Ange leur dit : Allez, et parcourez la terre ! Et ils parcoururent la terre.
\VS{8}Puis il m'appela, et me parla, en disant : Voici, ceux qui se dirigent vers le pays du nord ont apaisé mon Esprit dans le pays du nord.
\TextTitle{Prophétie sur le règne du germe de Yahweh}
\VS{9}La parole de Yahweh me fut adressée en ces mots :
\VS{10}Tu recevras les dons de ceux qui sont de retour de la captivité : Heldaï, Tobija et Jedaeja. Et tu iras toi-même ce même jour-là, et tu iras dans la maison de Josias, fils de Sophonie, où ils se sont rendus en arrivant de Babylone.
\VS{11}Tu prendras de l'argent et de l'or, et tu en feras des couronnes que tu mettras sur la tête de Josué, fils de Jotsadak, le souverain sacrificateur.
\VS{12}Tu lui diras : Ainsi parle Yahweh des armées : Voici un homme, dont le nom est Germe\FTNT{Es. 4:2.}, germera dans son lieu, et bâtira le temple de Yahweh\FTNT{C’est Yahweh, c’est-à-dire Jésus-Christ lui-même, qui bâtit son temple (Ps. 127:1-2 ; Mt. 16:18).}.
\VS{13}Oui, lui-même bâtira le temple de Yahweh ; et lui-même sera rempli de majesté. Il s’assiéra et dominera sur son trône, il sera Sacrificateur\FTNT{Jésus-Christ est Souverain Sacrificateur (Hé. 6:20 ; Hé. 7:1-28).}, étant sur son trône ; et il y aura un conseil de paix entre les deux.
\VS{14}Les couronnes seront pour Hélem, Tobija et Jedaeja, et pour Hen, fils de Sophonie, un souvenir dans le temple de Yahweh.
\VS{15}Ceux qui sont éloignés viendront, et travailleront au temple de Yahweh ; et vous saurez que Yahweh des armées m'a envoyé vers vous. Cela arrivera, si vous écoutez attentivement la voix de Yahweh, votre Dieu.
\Chap{7}
\TextTitle{Yahweh dénonce le jeûne formaliste}
\VerseOne{}La quatrième année du roi Darius, la parole de Yahweh fut adressée à Zacharie, le quatrième jour du neuvième mois, qui est le mois de Kisleu.
\VS{2}On avait envoyé à Béthel Scharetser et Réguem-Mélec avec ses gens, pour supplier Yahweh,
\VS{3}et pour parler aux sacrificateurs de la maison de Yahweh des armées, et aux prophètes, en disant : Dois-je pleurer au cinquième mois, et faire abstinence, comme j'ai déjà fait pendant plusieurs années ?
\VS{4}La parole de Yahweh des armées me fut adressée en ces mots :
\VS{5}Parle à tout le peuple du pays et aux sacrificateurs, et dis-leur : Quand vous avez jeûné et pleuré au cinquième mois et au septième, et cela depuis soixante-dix ans, avez-vous célébré ce jeûne par amour pour moi ?
\VS{6}Et quand vous buvez et mangez, n'est-ce pas vous qui mangez et vous qui buvez\FTNT{Es. 58:3-4.} ?
\VS{7}Ne connaissez-vous pas les paroles qu’a proclamées Yahweh par les premiers prophètes, lorsque Jérusalem était habitée et paisible avec ses villes à l’entour, et que le midi et la plaine étaient habités ?
\TextTitle{Yahweh n'exauce pas les pécheurs}
\VS{8}Puis la parole de Yahweh fut adressée à Zacharie en ces mots :
\VS{9}Ainsi parlait Yahweh des armées, en disant : Rendez véritablement la justice, et exercez la miséricorde et la compassion chacun envers son frère.
\VS{10}N’opprimez pas la veuve et l'orphelin, l'étranger et le pauvre, et ne méditez aucun mal dans vos cœurs chacun contre son frère\FTNT{Ex. 22:21 ; Es. 1:23 ; Jé. 5:28 ; Pr. 22:22-23.}.
\VS{11}Mais ils refusèrent d’être attentifs, ils eurent l'épaule rebelle, et ils endurcirent leurs oreilles pour ne pas entendre.
\VS{12}Ils rendirent leur cœur dur comme le diamant, pour ne pas écouter la loi et les paroles que Yahweh des armées adressait par son Esprit, par les premiers prophètes. C’est pourquoi  Yahweh des armées s’enflamma d’une grande colère.
\VS{13}Quand il appelait, ils n'ont pas écouté. Aussi n’ai-je pas écouté, quand ils ont appelé, dit Yahweh des armées\FTNT{Pr. 1:28 ; Es. 1:15 ; Jé. 11:11.}.
\VS{14}Je les ai dispersés comme par un tourbillon parmi toutes les nations qu'ils ne connaissaient pas ; le pays a été dévasté derrière eux, il n’y a plus eu ni allants ni venants ; et d’un pays de délices ils ont fait un désert.
\Chap{8}
\TextTitle{Futur royaume d'Israël rétabli dans la justice}
\VerseOne{}La parole de Yahweh des armées me fut encore adressée en ces mots :
\VS{2}Ainsi parle Yahweh des armées : Je suis jaloux pour Sion d'une grande jalousie, et je suis jaloux pour elle d’une grande fureur.
\VS{3}Ainsi parle Yahweh : Je retourne à Sion, et j'habiterai au milieu de Jérusalem ; et Jérusalem sera appelée ville fidèle ; et la montagne de Yahweh des armées sera appelée montagne sainte\FTNT{Es. 1:26.}.
\VS{4}Ainsi parle Yahweh des armées : Il y aura encore des vieillards et des femmes âgées, assis dans les rues de Jérusalem, et chacun aura son bâton à la main, à cause du grand nombre de leurs jours.
\VS{5}Les rues de la ville seront remplies de fils et de filles, jouant dans les rues.
\VS{6}Ainsi parle Yahweh des armées : S’il semble difficile aux yeux du reste de ce peuple que cela arrive, en ces jours-là, sera-t-il de même difficile à mes yeux ? dit Yahweh des armées.
\VS{7}Ainsi parle Yahweh des armées : Voici, je délivre mon peuple du pays de l'orient et du pays du soleil couchant.
\VS{8}Je les ramènerai, et ils habiteront au milieu de Jérusalem ; ils seront mon peuple, et je serai leur Dieu avec vérité et droiture.
\TextTitle{Juger selon la vérité}
\VS{9}Ainsi parle Yahweh des armées : Que vos mains soient fortifiées, vous qui entendez aujourd’hui ces paroles de la bouche des prophètes qui parurent au jour où la maison de Yahweh fut fondée, et où le temple allait être bâti\FTNT{Ag. 2:4.}.
\VS{10}Car avant ces jours-là, il n'y avait pas de salaire pour l'homme ni de salaire pour la bête ; et il n'y avait pas de paix pour ceux qui entraient et sortaient, à cause de la détresse ; et je lâchais tous les hommes les uns contre les autres.
\VS{11}Mais maintenant je ne serai pas pour le reste de ce peuple comme les premiers jours,  dit Yahweh des armées.
\VS{12}Car les semailles prospéreront, la semence de paix sera là ; la vigne rendra son fruit, et la terre donnera ses produits ; les cieux donneront leur rosée, et je ferai hériter toutes ces choses au reste de ce peuple.
\VS{13}De même que vous avez été en malédiction parmi les nations, ô maison de Juda et maison d'Israël, de même je vous délivrerai, et vous serez en bénédiction. Ne craignez pas, mais que vos mains soient fortifiées\FTNT{Ge. 1:11 ; Ap. 22:2.}.
\VS{14}Car ainsi parle Yahweh des armées : Comme j'ai eu la pensée de vous affliger, lorsque vos pères ont provoqué ma colère, dit Yahweh des armées, et que je ne m'en suis point repenti,
\VS{15}ainsi je reviens en arrière et j’ai résolu en ces jours de faire du bien à Jérusalem, et à la maison de Juda. Ne craignez pas !
\VS{16}Voici les choses que vous devez faire : Que chacun dise la vérité à son prochain ; jugez selon la vérité et prononcez un jugement en vue de la paix dans vos portes\FTNT{Ep. 4:25 ; Ex. 20:16 ; Mt. 19:18 ; Lu. 18:20.} ;
\VS{17}que personne ne projette du mal dans son cœur contre son prochain ; et n'aimez point le faux serment, car ce sont là des choses que je hais, dit Yahweh\FTNT{Ps. 5:5 ; Ps. 11:5 ; Pr. 6:16-19.}.
\VS{18}Puis la parole de Yahweh des armées me fut adressée en ces mots :
\VS{19}Ainsi parle Yahweh des armées : Le jeûne du quatrième mois, le jeûne du cinquième, le jeûne du septième et le jeûne du dixième seront changés pour la maison de Juda en joie et en allégresse, et en fêtes solennelles de réjouissance. Aimez donc la vérité et la paix\FTNT{Ep. 4:15.}.
\TextTitle{Les nations reconnaissent que Yahweh est le seul Dieu}
\VS{20}Ainsi parle Yahweh des armées : Il viendra encore des peuples et des habitants de plusieurs villes.
\VS{21}Les habitants d’une ville  iront à l'autre, en disant : Allons, allons implorer Yahweh et chercher Yahweh des armées ! Nous irons aussi !
\VS{22}Et beaucoup de peuples et de puissantes nations viendront rechercher Yahweh des armées à Jérusalem\FTNT{Jérusalem est appelée à devenir le centre d’adoration de la terre et la capitale du monde à cause de la présence de Dieu (Es. 66:23 ; Za. 14:16-21).}, et implorer Yahweh.
\VS{23}Ainsi parle Yahweh des armées : En ce jour-là, dix hommes de toutes les langues des nations saisiront le pan de la robe d'un homme Juif, et diront : Nous irons avec vous, car nous avons entendu que Dieu est avec vous.
\Chap{9}
\TextTitle{Le jugement de Yahweh sur les nations}
\VerseOne{}Oracle, parole de Yahweh sur le pays de Hadrac. Elle s’arrête sur Damas, car Yahweh a l’œil sur les hommes et sur toutes les tribus d'Israël.
\VS{2}Il s’arrête aussi sur Hamath, à la frontière de Damas, sur Tyr, et Sidon, quoique chacune d'elles soit fort sage.
\VS{3}Car Tyr s'est bâti une forteresse ; elle a amassé l'argent comme la poussière, et l’or fin comme la boue des rues\FTNT{Ez. 28:3-17.}.
\VS{4}Voici, le Seigneur l'appauvrira, et en la frappant, il jettera sa puissance dans la mer, et elle sera consumée par le feu\FTNT{Ez. 26:3-4.}.
\VS{5}Askalon le verra, et elle sera dans la crainte ; Gaza aussi le verra, et un violent tremblement la saisira ; Ekron aussi, car son espoir sera confondu. Et il n'y aura plus de roi à Gaza, et Askelon ne sera plus habitée\FTNT{So. 2:4.}.
\VS{6}Et le bâtard habitera à Asdod ; et j’abattrai l'orgueil des Philistins.
\VS{7}J'ôterai le sang de la bouche de chacun d'eux, et leurs abominations d'entre leurs dents ; et lui aussi restera pour notre Dieu, il sera comme un chef en Juda, et Ekron sera comme le Jébusien.
\VS{8}Je camperai autour de ma maison, pour la défendre contre une armée, contre les allants et les venants, et l’oppresseur ne passera plus près d’eux ; car maintenant mes yeux sont fixés sur elle.
\TextTitle{Prophétie sur la première venue du Messie}
\VS{9}Sois transportée d’allégresse, fille de Sion ! Pousse des cris de joie, fille de Jérusalem ! Voici, ton Roi vient à toi ; il est juste et vainqueur, il est monté sur un âne, sur un âne, le petit d'une ânesse\FTNT{Cette prophétie s’est accomplie 500 ans après. Jésus est effectivement entré à Jérusalem monté sur un âne (Mt. 21:1-11 ; Lu. 19:28-40 ; Jn. 12:12-19).}.
\TextTitle{La vision du Messie pour Israël}
\VS{10}Je détruirai les chars d'Ephraïm, et les chevaux de Jérusalem ; et les arcs de guerre seront aussi retranchés.  Et le Roi parlera de paix aux nations ; et sa domination s'étendra d’une mer à l’autre, depuis le fleuve jusqu'aux extrémités de la terre\FTNT{Es. 57:19 ; Ps. 2:8 ; Ps. 72:8.}.
\VS{11}Quant à toi, à cause de ton alliance scellée par le sang, je retirerai tes captifs de la fosse où il n'y a pas d'eau.
\VS{12}Retournez à la forteresse, captifs pleins d’espérance ! Aujourd'hui même je le déclare, je te rendrai le double.
\VS{13}Car je bande Juda comme un arc, je m’arme d’Ephraïm comme d’un arc, et j’exciterai tes enfants, ô Sion, contre tes enfants, ô Javan ! Je te rendrai pareille à l’épée d’un vaillant homme.
\VS{14}Alors Yahweh au-dessus d’eux apparaîtra, et ses dards partiront comme l'éclair, et le Seigneur, Yahweh, sonnera du shofar, il s’avancera dans le tourbillon du midi.
\VS{15}Yahweh des armées sera leur protecteur ; ils dévoreront, après avoir subjugué ceux qui tirent les pierres de fronde ; ils boiront, ils seront bruyants comme des hommes ivres, ils se rempliront de vin comme un bassin, et comme les coins de l'autel.
\VS{16}Yahweh, leur Dieu, les sauvera en ce jour-là, comme le troupeau de son peuple ; car ils sont les pierres d’une couronne  qui brilleront dans son pays.
\VS{17}Car combien est grande sa bonté ! Quelle beauté ! Le froment fera croître les jeunes hommes, et le vin doux rendra ses vierges éloquentes.
\Chap{10}
\TextTitle{Yahweh rassemblera son peuple}
\VerseOne{}Demandez à Yahweh la pluie\FTNT{Les pluies en Israël : En Israël, la saison des pluies commence généralement vers la fin du mois d’octobre avec de légères pluies qui ramollissent la terre (Ps. 65:10), et se poursuit ensuite par de fortes précipitations intermittentes durant deux ou trois jours, tout au long des mois de novembre et de décembre. Ces fortes précipitations étaient appelées dans les écritures «~la pluie de la première saison~» (en hébreu «~yoreb~» ou «~moreh~). Les fermiers dépendaient de la pluie de la première saison pour que la terre dure comme le roc soit rendue apte au labour et à l’ensemencement. Quand ces fortes précipitations s’achèvent, des pluies plus fines continuent encore de façon intermittente. Toutefois, à l’approche de la moisson,  la forte pluie revenait gonfler le grain et le fruit en préparation. Celle-ci était connue comme étant «~la pluie de l’arrière-saison~» (Jé. 5:24; Joë. 2:23-24 ; Os. 6:3).}, la pluie au temps de la dernière saison ! Yahweh produira des éclairs, et il vous donnera une abondante pluie, il donnera à chacun de l'herbe dans son champ.
\VS{2}Car les théraphim ont des paroles vaines, et les devins prophétisent le mensonge, ils profèrent des songes vains et consolent par la vanité. C'est pourquoi ils sont errants comme des brebis, ils sont malheureux, parce qu'il n'y a point de pasteur\FTNT{Mt. 9:36 ; Ez. 34:2 ; Jé. 23:21-30.}.
\VS{3}Ma colère s'est enflammée contre ces pasteurs, et je châtierai ces boucs ; car Yahweh des armées visite son troupeau, la maison de Juda ; et il les a rangés en bataille comme son cheval d'honneur.
\VS{4}De lui sortira l’Angle\FTNT{De lui (Juda) sortira l’Angle ou la pierre angulaire (Jésus-Christ), (1 Pi. 2:7 ; Es. 8:13-17).}, de lui sortira le clou, de lui sortira l'arc de bataille, et de lui sortiront tous les chefs ensemble.
\VS{5}Ils seront comme des vaillants hommes foulant la boue des rues dans la bataille, et ils combattront, parce que Yahweh sera avec eux ; et les cavaliers seront confus.
\VS{6}Car je fortifierai la maison de Juda, et je sauverai la maison de Joseph ; je les ramènerai, et je les ferai habiter en repos, parce que j'aurai compassion d'eux, et ils seront comme si je ne les avais point rejetés ; car je suis Yahweh, leur Dieu, et je les exaucerai.
\VS{7}Et ceux d'Ephraïm seront comme un héros, et leur cœur se réjouira comme par le vin ; leurs fils le verront, et se réjouiront ; leur cœur se réjouira en Yahweh.
\VS{8}Je les sifflerai et les rassemblerai, car je les rachète ; et ils seront multipliés comme ils l'ont été auparavant.
\VS{9}Et après que je les aurai dispersés parmi les peuples, ils se souviendront de moi dans les pays éloignés, et ils vivront avec leurs enfants, et ils reviendront.
\VS{10}Ainsi je les ramènerai du pays d'Egypte, je les rassemblerai de l'Assyrie, je les ferai venir au pays de Galaad, et au Liban, et il n'y aura point assez d'espace pour eux.
\VS{11}Il passera la mer de détresse, et il frappera les flots de la mer ; et toutes les profondeurs du fleuve seront desséchées ; l'orgueil de l'Assyrie sera abattu, et le sceptre d'Egypte sera ôté.
\VS{12}Je les fortifierai en Yahweh, et ils marcheront en son Nom, dit Yahweh.
\Chap{11}
\TextTitle{Les houlettes du vrai berger}
\VerseOne{}Liban, ouvre tes portes, et que le feu dévore tes cèdres !
\VS{2}Cyprès, gémis, car le cèdre est tombé, parce que les choses magnifiques ont été ravagées ! Chênes de Basan, gémissez, car la forêt inaccessible est coupée !
\VS{3}Les pasteurs poussent des cris de lamentations, parce que leur magnificence est ravagée ; on entend le rugissement des lionceaux, parce que l'orgueil du Jourdain est abattu.
\VS{4}Ainsi parle Yahweh, mon Dieu : Pais les brebis exposées au carnage !
\VS{5}Leurs possesseurs les égorgent, sans qu'on les tienne pour coupables, et celui qui les vend dit : Béni soit Yahweh, car je m’enrichis !  Et leurs pasteurs ne les épargnent pas.
\VS{6}Car je n’ai plus de pitié pour les habitants du pays, dit Yahweh ; et voici, je livre les hommes aux mains les uns des autres et aux mains de leur roi ; ils ravageront le pays, et je ne le délivrerai pas de leur main.
\VS{7}Alors je me mis donc à paître les brebis exposées au carnage, qui sont véritablement les plus misérables du troupeau. Puis je pris deux verges : J'appelai l'une Grâce, et l'autre Cordon ; et je me mis à paître les brebis.
\VS{8}Et je supprimai les trois pasteurs en un mois ; car mon âme était impatiente à leur sujet, et leur âme aussi avait pour moi du dégoût.
\VS{9}Et je dis : Je ne vous paîtrai plus ; que celle qui va mourir meure, et que celle qui va périr périsse, et que celles qui restent se dévorent la chair les unes les autres.
\VS{10}Puis je pris ma verge, appelée Grâce, et je la brisai, pour rompre mon alliance que j'avais traitée avec tous ces peuples.
\VS{11}Elle fut rompue en ce jour-là ; et les plus malheureuses brebis\FTNT{Les plus malheureuses des brebis sont le reste d’Israël.}, qui prirent garde à moi, reconnurent ainsi que c'était la parole de Yahweh.
\VS{12}Je leur dis : S'il vous semble bon, donnez-moi mon salaire ; sinon, ne me le donnez pas.  Alors ils pesèrent\FTNT{Mt. 26:15 ; Mt. 27:9-10.}  pour mon salaire trente pièces d'argent\FTNT{Selon la loi de Moïse, pour racheter un mâle de 20 à 60 ans, ayant fait un vœu, il fallait payer cinquante sicles d'argent (Lé. 27:3). Pour dédommager un préjudice causé par un bœuf ayant frappé un esclave, on devait donner trente sicles d'argent au maître de l'esclave et lapider le bœuf (Ex. 21:32). Or le prix du Seigneur a été estimé à trente sicles d’argent, comme pour les esclaves.}.
\VS{13}Yahweh me dit : Jette-le au potier, ce prix honorable auquel ils m’ont estimé ! Alors je pris les trente pièces d'argent, et les jetai dans la maison de Yahweh, pour le potier.
\VS{14}Puis je brisai ma seconde verge, appelée Cordon, pour rompre la fraternité entre Juda et Israël.
\TextTitle{Caractéristiques du faux berger}
\VS{15}Yahweh me dit : Prends-toi encore l'équipage d'un berger insensé.
\VS{16}Car voici, je susciterai dans le pays un pasteur, qui ne visitera pas les brebis qui périssent ; il ne cherchera pas celles qui s’égarent, il ne guérira pas celles qui sont blessées, et il ne soutiendra pas celles qui sont saines, mais il dévorera la chair des plus grasses, et il déchirera jusqu’aux cornes de leurs pieds.
\VS{17}Malheur au pasteur inutile qui abandonne les brebis ! Que l’épée fonde sur son bras et sur son œil droit ! Que son bras se dessèche, et que son œil droit s’éteigne entièrement !
\Chap{12}
\TextTitle{Jérusalem, une coupe d'étourdissement pour les nations}
\VerseOne{}Oracle, parole de Yahweh, sur Israël. Ainsi parle Yahweh, qui a étendu les cieux et fondé la terre, et qui a formé l'esprit de l'homme au-dedans de lui :
\VS{2}Voici, je ferai de Jérusalem une coupe d'étourdissement pour tous les peuples d'alentour ; et aussi pour Juda dans le siège de Jérusalem\FTNT{Ap. 16:12-16.}.
\VS{3}En ce jour-là, je ferai de Jérusalem une pierre pesante pour tous les peuples ; tous ceux qui en porteront le poids seront entièrement écrasés, car toutes les nations de la terre s'assembleront contre elle.
\VS{4}En ce temps-là, dit Yahweh, je frapperai d'étourdissement tous les chevaux, et de folie ceux qui les monteront ; mais j’aurai les yeux ouverts sur la maison de Juda, et je frapperai d'aveuglement tous les chevaux des peuples.
\VS{5}Les chefs de Juda diront en leur cœur : Les habitants de Jérusalem sont notre force, par Yahweh des armées, leur Dieu.
\VS{6}En ce jour-là, je ferai des chefs de Juda comme un foyer de feu parmi du bois, et comme une torche enflammée parmi des gerbes ; ils dévoreront à droite et à gauche tous les peuples d'alentour ; et Jérusalem sera encore habitée à sa place, à Jérusalem.
\VS{7}Yahweh sauvera premièrement les tentes de Juda, afin que la gloire de la maison de David, la gloire des habitants de Jérusalem, ne s'élève point au-dessus de Juda.
\VS{8}En ce jour-là, Yahweh sera le protecteur des habitants de Jérusalem ; et le plus faible parmi eux sera en ce jour-là comme David ; la maison de David sera comme Dieu, comme l'Ange de Yahweh devant leur face.
\VS{9}En ce jour-là, je chercherai à détruire toutes les nations qui viendront contre Jérusalem.
\TextTitle{Repentance et délivrance d'Israël}
\VS{10}Et je répandrai sur la maison de David, et sur les habitants de Jérusalem, l'Esprit de grâce et de supplications\FTNT{Joë. 2:28-30.}, et ils regarderont vers moi\FTNT{Au retour du Messie, il y aura une repentance et une conversion nationale d’Israël (Ro. 11:26).}, celui qu’ils ont percé, et ils pleureront sur lui\FTNT{Celui qu’ils ont percé : Il est question ici du Seigneur Jésus, le Messie (Ap. 1:7).}, comme on pleure sur un fils unique, et ils pleureront amèrement sur lui, comme quand on pleure sur un premier-né.
\VS{11}En ce jour-là, il y aura un grand deuil à Jérusalem, comme le deuil d'Hadadrimmon dans la vallée de Meguiddon.
\VS{12}Le pays sera dans le deuil, chaque famille à part : La famille de la maison de David à part, et les femmes de cette maison-là à part ; la famille de la maison de Nathan à part, et les femmes de cette maison-là à part.
\VS{13}La famille de la maison de Lévi à part, et les femmes de cette maison-là à part ; la famille de Schimeï à part, et ses femmes à part.
\VS{14}Toutes les autres  familles, chaque famille à part, et leurs femmes à part.
\Chap{13}
\TextTitle{Dieu frappe les faux prophètes}
\VerseOne{}En ce jour-là, il y aura une source ouverte en faveur de la maison de David et des habitants de Jérusalem, pour le péché et pour la souillure.
\VS{2}En ce jour-là, dit Yahweh des armées, je retrancherai du pays les noms des faux dieux, et on n'en fera plus mention. J'ôterai aussi du pays les faux prophètes et l'esprit d'impureté.
\VS{3}Et il arrivera que si quelqu'un prophétise encore, son père et sa mère qui l’ont engendré, lui diront : Tu ne vivras plus ; car tu as prononcé des mensonges au Nom de Yahweh ; et son père et sa mère qui l’ont engendré, le transperceront quand il prophétisera.
\VS{4}En ce jour-là, les prophètes seront confus de leurs visions, quand ils prophétiseront ; et ils ne revêtiront plus un manteau de poil pour mentir.
\VS{5}Chacun d’eux dira : Je ne suis pas prophète, mais je suis laboureur, car on m'a appris à gouverner du bétail dès ma jeunesse.
\VS{6}Et si on lui demande : Que veulent donc dire ces blessures que tu as aux mains ? Et il répondra : C’est dans la maison de mes amis qu’on me les a faites.
\TextTitle{Prophétie sur le vrai berger, le Messie}
\VS{7}Epée, réveille-toi contre mon Berger\FTNT{Mon Berger : Il est question de Jésus-Christ, le Bon Berger (Ps. 23 ; Jn. 10:1-17).}, et sur l'homme qui est mon compagnon ! dit Yahweh des armées frappe le Berger, et les brebis seront dispersées\FTNT{Frappe le Berger : Cette prophétie fait référence à la crucifixion du Seigneur Jésus-Christ (Ge. 3:15 ; Mt. 26:31 ; Mc. 14:27 ; Mc. 14:50 ; Mc. 15:19).} ; et je tournerai ma main vers les faibles.
\TextTitle{Le reste de Yahweh épuré à travers l'épreuve}
\VS{8}Dans tout le pays, dit Yahweh, les deux tiers seront retranchées et périront, et l’autre tiers restera.
\VS{9}Je mettrai ce tiers dans le feu, et je le purifierai comme on purifie l'argent, je les éprouverai comme on éprouve l'or. Il invoquera mon Nom, et je l'exaucerai ; je dirai : C'est ici mon peuple ! Et il dira : Yahweh est mon Dieu\FTNT{1 Pi. 1:6-7 ; Ps. 50:15 ; Ps. 91:15 ; Ps. 144:15.} !
\Chap{14}
\TextTitle{Imminence du jour de Yahweh}
\VerseOne{}Voici, le jour de Yahweh\FTNT{L’expression «~le jour du Seigneur~» ou «~le jour de Yahweh~» est utilisée dix-neuf fois dans le Tanakh (Es. 2:12 ;  Es. 13:6 ; Es. 13:9 ; Ez. 13:5 ; Ez. 30:3 ; Joë. 1:15 ; Joë. 2:1 ; Joë. 2:11 ; Joë. 2:31 ;  Joë. 3:14 ; Am. 5:18-20 ; Ab. 1:15 ; So. 1:7 ; So. 1:14 ; Za. 14:1 ; Mal. 4:5) et quatre fois dans les textes de la nouvelle alliance (Ac. 2:20 ; 2 Th. 2:2 ; 2 Pi. 3:10 ; Ap. 6:17 ; Ap. 16:14). Cette expression désigne habituellement des événements qui se déroulent à la fin des temps (Es. 7:18-25). Elle désigne un espace de temps au cours duquel Dieu va intervenir personnellement dans l’histoire des hommes. Appelé  «~jour de colère~», «~jour de  visitation~», et «~grand jour du Dieu Tout-Puissant~» ; il se réfère ainsi à un accomplissement encore futur, quand la colère de Dieu viendra s’abattre sur l’Israël qui n’aura pas cru (Es. 22 ; Jé. 30:1-17 ; Joë. 1 et 2 ; Am. 5 ; So. 1) et sur tous les incrédules du monde (Ez. 38 et 39 ; Za. 14). Ce jour sera aussi un temps de salut puisque Dieu va délivrer «~le reste~» d’Israël, accomplissant ainsi sa promesse selon laquelle «~tout Israël sera sauvé~» (Ro. 11:26) : il pardonnera leurs péchés et restaurera le peuple qu’Il s’est choisi sur la terre promise à Abraham (Es. 10:27 ; Jé. 30:19-31 ; Mi. 4 ; Za. 13).} arrive, et tes dépouilles seront partagées au milieu de toi, Jérusalem.
\VS{2}Je rassemblerai toutes les nations à Jérusalem pour qu’elles lui fassent la guerre\FTNT{Joë. 3 ; Ap. 16:12-16.} ; la ville sera prise, les maisons pillées, et les femmes violées ; la moitié de la ville ira en captivité, mais le reste du peuple ne sera pas retranché de la ville.
\VS{3}Yahweh sortira, et il combattra contre ces nations, comme il a combattu au jour de la bataille.
\TextTitle{Retour visible et en gloire du Seigneur}
\VS{4}Ses pieds se poseront en ce jour sur la Montagne des Oliviers\FTNT{Ce sont les pieds de Jésus-Christ (Ac. 1:10-11).}, qui est vis-à-vis de Jérusalem, du côté de l’orient ; et la Montagne des Oliviers se fendra par le milieu, à l'orient et à l'occident, de sorte qu'il y aura une très grande vallée ; une moitié de la montagne reculera vers le nord, et l'autre moitié vers le midi.
\VS{5}Vous fuirez alors dans la vallée de mes montagnes ; car la vallée des montagnes s’étendra jusqu'à Atzel ; et vous fuirez comme vous avez fui devant le tremblement de terre, aux jours d’Ozias, roi de Juda. Alors Yahweh, mon Dieu, viendra, et tous les saints seront avec lui\FTNT{Ce passage confirme clairement que Jésus-Christ est Yahweh (1 Th. 3:13 ; Jud. 14-15 ; Es. 34:5 ; Es. 40:10-11 ; Es. 62:11-15).}.
\VS{6}Et il arrivera qu'en ce jour-là, la lumière précieuse ne sera pas mêlée de ténèbres.
\VS{7}Ce sera un jour unique, connu de Yahweh, et qui ne sera ni jour ni nuit ; mais au temps du soir il y aura de la lumière.
\VS{8}Et il arrivera qu'en ce jour-là, des eaux vives\FTNT{Ez. 47:1-12 ;  Ap. 22:1-2.} sortiront de Jérusalem, la moitié d'elles coulera vers la mer orientale, et l'autre moitié, vers la mer occidentale ; il en sera ainsi été et hiver.
\TextTitle{Le royaume messianique}
\VS{9}Yahweh sera Roi sur toute la terre ; en ce jour-là, Yahweh sera Un, et son nom sera Un\FTNT{Littéralement «~E’had~». Le jour du Seigneur est un comme le jour un de Ge. 1:5. Yahweh est Un et non trois (De. 6:4). Son Nom est Un (Ac. 4:12).}.
\VS{10}Toute la terre deviendra comme la plaine, depuis Guéba jusqu'à Rimmon, au midi de Jérusalem ; et Jérusalem sera exaltée et restera à sa place, depuis la porte de Benjamin, jusqu'à l'endroit de la première porte, jusqu'à la porte des angles, et depuis la tour de Hananeel, jusqu'aux pressoirs du roi.
\VS{11}On habitera dans son sein, et il n'y aura plus d'interdit, mais Jérusalem sera habitée en sûreté.
\VS{12}Voici la plaie dont Yahweh frappera tous les peuples qui auront fait la guerre contre Jérusalem ; il fera que la chair de chacun tombera en pourriture tandis qu’ils seront sur leurs pieds, leurs yeux tomberont en pourriture dans leurs orbites, et leur langue tombera en pourriture dans leur bouche.
\VS{13}Et il arrivera en ce jour-là que Yahweh produira un grand trouble parmi eux ; car chacun saisira la main de son prochain, et la main de l'un s'élèvera contre la main de l'autre.
\VS{14}Juda combattra aussi dans Jérusalem, et les richesses de toutes les nations d'alentour y seront amassées : L'or, l'argent, et des vêtements en très grand nombre.
\VS{15}Et la même plaie sera sur les chevaux, les mulets, les chameaux, les ânes et sur toutes les bêtes qui seront dans ces camps, cette plaie sera semblable à l’autre.
\TextTitle{Adoration de Yahweh des armées dans le royaume}
\VS{16}Et il arrivera que tous ceux qui resteront de toutes les nations venues contre Jérusalem, monteront en foule chaque année pour adorer le Roi, Yahweh des armées, et pour célébrer la fête des tabernacles.
\VS{17}S’il y a des familles de la terre qui ne montent pas à Jérusalem, pour adorer le Roi, Yahweh des armées, la pluie ne tombera pas sur elles.
\VS{18}Si la famille d'Egypte ne monte pas, si elle ne vient pas, la pluie ne tombera pas sur elle ; elle sera frappée de la plaie dont Yahweh frappera les nations qui ne monteront pas pour célébrer la fête des tabernacles.
\VS{19}Ce sera la peine du péché de l’Egypte, et du péché de toutes les nations qui ne monteront pas pour célébrer la fête des tabernacles.
\VS{20}En ce jour-là, il sera écrit sur les clochettes des chevaux : Sainteté à Yahweh ! Et les chaudières dans la maison de Yahweh seront comme les coupes devant l'autel.
\VS{21}Toute chaudière qui sera à Jérusalem et dans Juda, sera consacrée à Yahweh des armées ; et tous ceux qui offriront des sacrifices viendront, et s’en serviront pour cuire  les viandes ; et il n'y aura plus de marchands dans la maison de Yahweh des armées, en ce jour-là.
\PPE{}
\end{multicols}

%\clearpage\ShortTitle{Malachie}\BookTitle{Malachie}\BFont
\noindent\hrulefill
\textit{
\bigskip
{\centering{}
\\(Malakhi)
\\Signifie : Mon messager, mon ange
\\Thème : Message final de l'ancienne alliance à une nation désobéissante
\\Auteur : Malachie
\\Date de rédaction : Vème siècle av. J.-C.\\}
}
%\bigskip
\textit{
\\Dernier prophète de l’ancienne alliance, Malachie exerça son ministère en Juda après la reconstruction du temple et la reprise des cultes. Il annonça la venue du Messie et du messager qui devait le précéder, le nouvel Elie que Jésus-Christ reconnut en Jean-Baptiste. Ses écrits mettent en évidence l’importance de l’obéissance à la loi de Yahweh et la justice divine.
\bigskip
\\message. 
\bigskip
\\message.\bigskip
}
\par\nobreak\noindent\hrulefill
\begin{multicols}{2}
\TextTitle{[L'amour de Yahweh pour son peuple]}
\Chap{1}
\VerseOne{}Oracle, parole de Yahweh contre Israël, par le moyen de Malachie.
\VS{2}Je vous ai aimés, dit Yahweh ; et vous dites : En quoi nous as-tu aimés ? Esaü n'était-il pas frère de Jacob ? dit Yahweh. Or j'ai aimé Jacob,
\VS{3}Mais j'ai eu de la haine pour Esaü, et j'ai fait de ses montagnes une solitude, j’ai livré son héritage aux chacals du désert.
\VS{4}Si Edom dit : Nous sommes détruits, nous rebâtirons les lieux ruinés ! Ainsi parle Yahweh des armées : Ils rebâtiront, mais je détruirai, et on les appellera pays de méchanceté, peuple contre lequel Yahweh est irrité pour toujours.
\VS{5}Vos yeux le verront, et vous direz : Yahweh est grand par-delà les frontières d'Israël !
\TextTitle{[Le péché des sacrificateurs après le retour d'exil]}
\VS{6}Un fils honore son père, et un serviteur son maître. Si donc je suis Père, où est l'honneur qui m'appartient ? Si je suis maître, où est la crainte qu'on a de moi ? dit Yahweh des armées, à vous sacrificateurs, qui méprisez mon Nom, et qui dites : En quoi avons-nous méprisé ton Nom ?
\VS{7}Vous offrez sur mon autel des aliments souillés, et vous dites : En quoi t'avons-nous profané ? C'est en disant : La table de Yahweh est méprisable !
\VS{8}Et quand vous amenez une bête aveugle pour la sacrifier, n'y a-t-il point de mal en cela ? Quand vous en offrez une boiteuse ou malade, n’est-ce pas mal ? Offre-la à ton gouverneur ! T’agréera-t-il, te recevra-t-il favorablement ? dit Yahweh des armées.
\VS{9}Maintenant, donc suppliez Dieu, pour qu'il ait pitié de nous ! Cela vient de vos mains : Vous recevra-t-il favorablement ? dit Yahweh des armées.
\VS{10}Lequel de vous fermera les portes pour que vous n’allumiez pas en vain le feu sur mon autel ? Je ne prends aucun plaisir en vous, dit Yahweh des armées, et je n’agrée pas l'offrande de vos mains.
\VS{11}Car depuis le soleil levant jusqu'au soleil couchant, mon Nom est grand parmi les nations, et en tous lieux on brûle de l’encens en l'honneur de mon Nom, et des offrandes pures ; car mon Nom est grand parmi les nations, dit Yahweh des armées.
\VS{12}Mais vous, vous le profanez, en disant : La table de Yahweh est souillée, et ce qu’elle rapporte est un aliment méprisable.
\VS{13}Vous dites aussi : Quelle fatigue ! Et vous le dédaignez, dit Yahweh des armées ; vous amenez ce qui a été dérobé, ce qui est boiteux, et malade, ce sont là les offrandes que vous faites ! Accepterai-je cela de vos mains ? dit Yahweh.
\VS{14}C'est pourquoi, maudit soit l'homme trompeur, qui a dans son troupeau un mâle, et qui voue et sacrifie à Yahweh ce qui est corrompu ! Car je suis un grand roi, dit Yahweh des armées, et mon Nom est redoutable parmi les nations.
\TextTitle{[Mise en garde de Yahweh aux sacrificateurs]}
\Chap{2}
\VerseOne{}Maintenant, c’est à vous, sacrificateurs que s'adresse ce commandement :
\VS{2}Si vous n'écoutez pas, et que vous ne preniez point pas à cœur de donner gloire à mon Nom, dit Yahweh des armées, j'enverrai sur vous la malédiction, et je maudirai vos bénédictions ; et déjà même je les ai maudites, parce que vous ne prenez pas cela à cœur.
\VS{3}Voici, je vais détruire vos semences, et je répandrai les excréments de vos victimes sur vos visages, les excréments, dis-je, de vos solennités, et on vous emportera avec eux.
\VS{4}Alors vous saurez que je vous ai adressé ce commandement, afin que mon alliance avec Lévi subsiste, dit Yahweh des armées.
\VS{5}Mon alliance avec lui était la vie et la paix, c’est ce que je lui accordai pour qu’il me craigne ; il a eu pour moi de la crainte, et il a tremblé devant mon Nom.
\VS{6}La loi de la vérité était dans sa bouche, et il ne s'est point trouvé de perversité sur ses lèvres ; il a marché avec moi dans la paix et dans la droiture, et il en a détourné beaucoup de l'iniquité.
\VS{7}Car les lèvres du sacrificateur doivent garder la science, et c’est de sa bouche qu’on demande la loi, parce qu'il est un messager de Yahweh des armées.
\VS{8}Mais vous, vous vous êtes écartés de la voie, vous avez fait de la loi une occasion de chute pour beaucoup, et vous avez corrompu l'alliance de Lévi, dit Yahweh des armées.
\TextTitle{[Infidélités envers des frères et envers Yahweh]}
\VS{9}C'est pourquoi je vous rendrai méprisables et abjects aux yeux de tout le peuple. Parce que vous n’avez pas gardé mes voies, et vous avez égard à l'apparence des personnes quand vous enseignez la loi.
\VS{10}N'avons-nous pas tous un seul Père ? N’est-ce pas un seul Dieu qui nous a créés ? Pourquoi donc agissons-nous avec perfidie l’un avec l’autre, en violant l'alliance de nos pères ?
\VS{11}Juda s’est montré infidèle, et une abomination a été commise en Israël et à Jérusalem ; car Juda a profané ce qui est consacré à Yahweh, ce qu’il aime, il s'est marié à la fille d'un dieu étranger.
\VS{12}Yahweh retranchera l’homme qui fait cela, celui qui veille et qui répond, il le retranchera des tentes de Jacob, et il retranchera celui qui présente une offrande à Yahweh des armées.
\VS{13}Voici une autre chose que vous faites : Vous couvrez l'autel de Yahweh de larmes, de plaintes et de gémissements, en sorte qu’il n’a plus égard aux offrandes et qu’il ne peut rien agréer de vos mains.
\VS{14}Et vous dites : Pourquoi ?... C'est parce que Yahweh est intervenu comme témoin entre toi et la femme de ta jeunesse, envers laquelle tu as été infidèle, bien qu’elle soit ta compagne et la femme de ton alliance.
\VS{15}Nul n’a fait cela, avec un reste de bon esprit. Un seul l’a fait, et pourquoi ? Parce qu'il cherchait une postérité de Dieu. Prenez donc garde en votre esprit, et qu’aucun ne soit infidèle à la femme de sa jeunesse !
\VS{16}Car je hais la répudiation, dit Yahweh, le Dieu d'Israël, et celui qui couvre de violence son vêtement, dit Yahweh des armées. Prenez donc garde en votre esprit, et ne soyez pas infidèles !
\TextTitle{[Fausse profession religieuse]}
\VS{17}Vous fatiguez Yahweh par vos paroles, et vous dites : En quoi l'avons-nous fatigué ? C'est quand vous dites : Quiconque fait le mal plaît à Yahweh, et il prend plaisir à de tels gens ! Autrement : Où est le Dieu du jugement ?
\TextTitle{[Venue du précurseur du messie]}
\Chap{3}
\VerseOne{}Voici, j'enverrai mon messager\FTNT{Ce messager, ou Elie le prophète, est Jean-Baptiste (Es. 40:1-3 ; Mal. 3:1 ; Mt. 3:1-15 ; Mt. 11:14 ; Mt. 17:10-13 ; Mc. 1:1-11 ; Mc. 9:11-13 ; Lu. 1:17 ; Lu. 3:1-5).} ; il préparera le chemin devant moi. Et soudain entrera dans son temple le Seigneur que vous cherchez ; l’ange de l'alliance\FTNT{L’ange de l’alliance est le Seigneur Jésus-Christ. Voir aussi commentaire en Mt. 1:20}, que vous désirez, voici, il vient, dit Yahweh des armées.
\VS{2}Mais qui pourra soutenir le jour de sa venue ? Qui pourra subsister quand il paraîtra ? Car il sera comme le feu du fondeur et comme la potasse des foulons.
\VS{3}Et il sera assis comme celui qui raffine et purifie l'argent ; il nettoiera les fils de Lévi, il les épurera comme l’or et l'argent, et ils présenteront à Yahweh des offrandes avec justice.
\VS{4}Alors l’offrande de Juda et de Jérusalem sera agréable à Yahweh, comme aux anciens jours, comme aux années d'autrefois.
\VS{5}Je m'approcherai de vous pour le jugement, et je me hâterai de témoigner contre les enchanteurs et les adultères, contre ceux qui jurent faussement, et contre ceux qui retiennent le salaire du mercenaire, qui oppriment la veuve et l'orphelin, qui font tort à l'étranger, et qui ne me craignent point, dit Yahweh des armées.
\VS{6}Parce que je suis Yahweh et que je n'ai point changé ; à cause de cela, enfants de Jacob, vous n'avez point été consumés.
\TextTitle{[Le peuple infidèle qui vole Yahweh]}
\VS{7}Depuis le temps de vos pères, vous vous êtes écartés de mes ordonnances, vous ne les avez point observées. Revenez à moi, et je reviendrai à vous, dit Yahweh des armées. Et vous dites : En quoi nous convertirons-nous ?
\VS{8}L'homme pillera-t-il Dieu, que vous me pilliez ? Et vous dites : En quoi t'avons-nous pillé ? Vous l'avez fait dans les dîmes et dans les offrandes.
\VS{9}Vous êtes certainement maudits, parce que vous me pillez, vous, toute la nation !
\VS{10}Apportez toutes les dîmes\FTNT{Il est question ici de la dîme de la dîme que les levites donnaient aux sacrificateurs. Cette dîme était rapportée aux magasins, ou greniers (Né. 10:35-39), là aussi était stocké toute sorte de trésor. Pour les autres dîmes, voir le commentaire dans Dt. 14:22-29.} aux magasins, afin qu'il y ait provision dans ma maison ; et dès maintenant éprouvez moi en cela, a dit Yahweh des armées, si je ne vous ouvre pas les écluses des cieux, et si je ne répands pas en votre faveur la bénédiction, jusqu'à ce qu'il n'y ait plus assez de place.
\VS{11}Et je réprimerai pour l'amour de vous le dévorateur, et il ne vous ravagera pas les fruits de la terre, et vos vignes ne seront pas stériles dans vos campagnes, a dit Yahweh des armées.
\VS{12}Toutes les nations vous diront heureux, car vous serez un pays de délices, dit Yahweh des armées.
\VS{13}Vos paroles sont rudes contre moi, a dit Yahweh. Et vous dites : Qu'avons-nous donc dit contre toi ?
\VS{14}Vous avez dit : C'est en vain que l’on sert Dieu ; et qu'avons-nous gagné à observer ses ordonnances, et à marcher en pauvre état pour l'amour de Yahweh des armées ?
\VS{15}Et maintenant nous tenons pour heureux les orgueilleux ; et même ceux qui commettent la mechanceté, sont avancés ; et s'ils ont tenté Dieu, ils ont été délivrés !
\TextTitle{[Le "reste d'Israël" demeure fidèle à Yahweh"]}
\VS{16}Alors ceux qui craignent Yahweh se parlèrent l'un à l’autre ; et Yahweh fut attentif, et il écouta ; et un livre de souvenir fut écrit devant lui pour ceux qui craignent Yahweh et qui pensent à son Nom.
\VS{17}Ils seront à moi, a dit Yahweh des armées, le jour où je mettrai à part mes plus précieux joyaux, et je leur pardonnerai comme un homme pardonne à son fils qui le sert.
\VS{18}Convertissez-vous donc, et vous verrez la différence qu'il y a entre le juste et le méchant, entre celui qui sert Dieu et celui qui ne le sert pas.
\TextTitle{[Avènement du jour de Yahweh]}
\Chap{4}
\VerseOne{}Car voici, le jour vient, ardent comme une fournaise. Tous les orgueilleux et tous les méchants seront comme du chaume ; et ce jour qui vient, dit Yahweh des armées, les embrasera, il ne leur laissera ni racine ni rameau.
\VS{2}Mais pour vous qui craignez mon Nom, se lèvera le Soleil de justice\FTNT{Le Soleil de justice : Jésus-Christ est notre Soleil (Lu. 1:78-79). Cet aspect de Jésus-Christ nous parle de la grâce de Dieu : « il fait lever son soleil sur les méchants, et sur les gens de bien » (Mt. 5:45). Le soleil évoque aussi le jugement de Dieu. Ainsi, en plein midi, il est le feu de la justice et de la colère de Dieu. Son pardon et son amour pour nous sont alors comparés à une ombre fraîche qui nous sauve de sa chaleur ardente (Ps. 121 ; Es. 25:4). Dans Ps. 19:6, le soleil est comparé à un époux. Or Jésus-Christ est notre époux et le soleil qui nous apporte la guérison.}, et la guérison sera sous ses ailes ; vous sortirez, et bondirez comme les veaux d’une étable.
\VS{3}Et vous foulerez les méchants, car ils seront comme de la cendre sous les plantes de vos pieds, au jour où je ferai mon œuvre, dit Yahweh des armées.
\VS{4}Souvenez-vous de la loi de Moïse, mon serviteur, auquel j’ai prescrit en Horeb, pour tout Israël, des statuts et des ordonnances.
\TextTitle{[Retour d'Elie avant le jour de Yahweh]}
\VS{5}Voici, je vous enverrai Elie, le prophète\FTNT{Voir commentaire en Mal. 3:1.}, avant que le jour grand et redoutable de Yahweh vienne.
\VS{6}Il ramènera le cœur des pères à leurs enfants, et le cœur des enfants à leurs pères, de peur que je ne vienne et que je ne frappe la terre d’interdit.
\PPE{}
\end{multicols}

%\addcontentsline{toc}{section}{Ketouvim (Écrits)}\clearpage
%\clearpage\ShortTitle{Psaumes}\BookTitle{Psaumes}\BFont
\noindent\hrulefill
{\footnotesize
\textit{
\bigskip
{\centering{}
\\Auteurs : David essentiellement et d'autres écrivains
\\(Heb. : Tehilim)
\\Signification : Louanges 
\\Thème : La louange et l'adoration
\\Date de rédaction : A compter du 10\up{ème} siècle et au-delà\\}
}
%\bigskip
\textit{
\\Le terme « psaume » désigne un poème chanté avec l'accompagnement d'un instrument. C'est ainsi que furent initialement contés les récits de la création divine, la captivité ou encore la gloire de Jérusalem. Expressions de joie, de reconnaissance, de repentance, d'angoisse ou de vulnérabilité de l'homme, ces hymnes étaient des prières adressées à Dieu.
%\bigskip
\\Prophétiques, certains psaumes annoncent les événements de la fin des temps, notamment les souffrances de Christ. Utilisé comme recueil de chants, le livre des Psaumes exalte la grandeur de Dieu, sa souveraineté, sa miséricorde et son omniscience. Il est le fruit d'une grande variété d'expériences spirituelles du fait de la diversité de ses auteurs. De plus, il contient une richesse de styles considérable, ce qui en fait le chef-d'œuvre de la poésie hébraïque.\bigskip
}
}
\par\nobreak\noindent\hrulefill
\begin{multicols}{2}
\Chap{1}
\TextTitle{La voie du juste et du pécheur}
\VerseOne{}Heureux l'homme qui ne marche pas selon le conseil des méchants, et qui ne s'arrête pas sur la voie des pécheurs, qui ne s'assied pas dans l'assemblée des moqueurs\FTNT{Jé. 15:17 ; 1 Co. 15 : 33 ; Ep. 5:11.},
\VS{2}mais qui prend plaisir dans la loi de Yahweh, et qui médite sa loi jour et nuit\FTNT{De. 6:6 ; De. 17:19 ; Jos. 1:8.}.
\VS{3}Il est comme un arbre planté près des ruisseaux d'eaux, qui rend son fruit en sa saison, et dont le feuillage ne se flétrit point\FTNT{Jé. 17:7-8 ; Ez. 47:12 ; Jn. 15:8 ; Ap. 22:2.}. Et ainsi tout ce qu'il fera réussira.
\VS{4}Il n'en est pas ainsi des méchants : Ils sont comme la balle que le vent chasse au loin\FTNT{Job. 21:17-18 ; Os. 13:3.}.
\VS{5}C'est pourquoi les méchants ne résistent pas dans le jugement, ni les pécheurs dans l'assemblée des justes.
\VS{6}Car Yahweh connaît la voie des justes, mais la voie des méchants périra.
\Chap{2}
\TextTitle{Complot des nations contre le Messie}
\VerseOne{}Pourquoi cette agitation parmi les nations, et pourquoi les peuples projettent-ils des choses vaines ?
\VS{2}Pourquoi les rois de la terre se lèvent-ils en personne, et les princes se liguent-ils avec eux contre Yahweh, et contre son Messie\FTNT{Cette prophétie concerne le complot des Juifs, de Pilate et d'Hérode contre Jésus-Christ, notre Seigneur. Il est également question du gouvernement mondial dirigé par Satan. Mt. 12:14 ; Mt. 26:3-4 ; Mt. 26:59-66. ; Mt. 27:1-2 ; Mc. 3:6 ; Mc. 11:18 ; Ac. 4:23-29.}?
\VS{3}Rompons leurs liens et jetons loin de nous leurs cordes!
\VS{4}Celui qui habite dans les cieux se rit d'eux, le Seigneur se moque d'eux.
\VS{5}Il leur parle dans sa colère, et il les remplira de terreur par la grandeur de son courroux\FTNT{Pr. 1:26.}:
\VS{6}C'est moi qui ai consacré mon Roi sur Sion, la montagne de ma sainteté\FTNT{Mi. 4:7.}!
\VS{7}Je vous réciterai cette ordonnance ; Yahweh m'a dit : Tu es mon Fils ! Je t'ai engendré aujourd'hui\FTNT{Ac. 13:33 ; Hé. 1:5 ; Hé. 5:5.}.
\VS{8}Demande-moi, et je te donnerai les nations pour héritage, et les extrémités de la terre pour possession.
\VS{9}Tu les briseras avec un sceptre de fer et tu les mettras en pièces comme un vase de potier\FTNT{Da. 2:44 ; Ap. 2:27.}.
\VS{10}Maintenant donc, rois, ayez de l'intelligence ! Juges de la terre, recevez instruction !
\VS{11}Servez Yahweh avec crainte, et réjouissez-vous avec tremblement\FTNT{Ps. 19:10.}.
\VS{12}Embrassez le Fils, de peur qu'il ne s'irrite et que vous ne périssiez dans cette conduite, quand sa colère s'embrasera promptement. Heureux sont tous ceux qui se confient en lui !
\Chap{3}
\TextTitle{Yahweh, le véritable secours}
\VerseOne{}Psaume de David au sujet de sa fuite devant Absalom, son fils.
\VS{2}Ô Yahweh, que mes adversaires sont nombreux ! Beaucoup de gens se lèvent contre moi!
\VS{3}Plusieurs disent à mon âme : Plus de salut pour lui auprès de Dieu! Sélah\FTNT{Le mot hébreu « Sélah » signifie « élever, exalter ». Il peut aussi traduire une pause dans le cantique ou le texte. C'est sûrement un terme technique musical montrant probablement une accentuation, une pause, une interruption.}.
\VS{4}Mais toi, ô Yahweh ! Tu es un bouclier autour de moi, tu es ma gloire, et tu relèves ma tête.
\VS{5}De ma voix je crie à Yahweh, et il me répond de sa sainte montagne. Sélah.
\VS{6}Je me couche, je m'endors, je me réveille, car Yahweh me soutient\FTNT{Lé. 26:6.}.
\VS{7}Je ne crains pas les myriades de peuples quand ils se rangent contre moi de toutes parts.
\VS{8}Lève-toi, Yahweh, mon Dieu ! Délivre-moi ! Car tu frappes à la joue tous mes ennemis, tu brises les dents des méchants.
\VS{9}La délivrance vient de Yahweh\FTNT{Es. 43:11 ; Jé. 3:23 ; Pr. 21:31 ; Ap. 7:10.} ! Que ta bénédiction soit sur ton peuple! Sélah.
\Chap{4}
\TextTitle{Yahweh, la joie et la paix du juste}
\VerseOne{}Psaume de David, donné au chef des chantres pour le chanter sur Neguinoth.
\VS{2}Ô Dieu de ma justice, puisque je crie, réponds-moi ! Quand j'étais à l'étroit, tu m'as mis au large ! Aie pitié de moi, et exauce ma prière\FTNT{Ps. 28:1-2.} !
\VS{3}Fils des hommes, jusqu'à quand ma gloire sera-t-elle diffamée ? Jusqu'à quand aimerez-vous la vanité et chercherez-vous le mensonge ? Sélah.
\VS{4}Or sachez que Yahweh s'est choisi un bien-aimé. Yahweh m'exauce quand je crie à lui\FTNT{1 Jn. 5:14.}.
\VS{5}Tremblez et ne péchez point ; parlez en vos cœurs sur votre couche et taisez-vous. Sélah.
\VS{6}Offrez des sacrifices de justice\FTNT{Ps. 51:19.} et confiez-vous en Yahweh.
\VS{7}Plusieurs disent : Qui nous fera voir le bonheur ? Lève sur nous la lumière de ta face, ô Yahweh !
\VS{8}Tu mets plus de joie dans mon cœur qu'ils n'en ont, quand abondent leur froment et leur vin.
\VS{9}Je me couche et je m'endors en paix, car toi seul, ô Yahweh ! Tu me fais reposer en sécurité\FTNT{Pr. 3:24.}.
\Chap{5}
\TextTitle{Recours à la protection de Yahweh}
\VerseOne{}Psaume de David, donné au chef des chantres, pour le chanter sur Nehiloth.
\VS{2}Yahweh, prête l'oreille à mes paroles ! Ecoute ma méditation !
\VS{3}Mon Roi et mon Dieu ! Sois attentif à la voix de mon cri ; car c'est à toi que j'adresse ma requête.
\VS{4}Yahweh, le matin tu entends ma voix, dès le matin je me tourne vers toi, et je veille.
\VS{5}Car tu n'es point un Dieu qui prenne plaisir au mal ; le méchant n'a point sa demeure auprès de toi.
\VS{6}Les orgueilleux ne subsistent pas devant tes yeux ; tu hais tous ceux qui commettent l'iniquité\FTNT{Ps. 1:5 ; Ha. 1:13.}.
\VS{7}Tu fais périr les menteurs ; Yahweh a en abomination l'homme sanguinaire et le trompeur.
\VS{8}Mais moi, comblé de tes bienfaits, j'entrerai dans ta maison, je me prosternerai dans le palais de ta sainteté avec les sentiments d'une crainte respectueuse.
\VS{9}Yahweh, conduis-moi dans ta justice, à cause de mes ennemis, aplanis ta voie sous mes pas\FTNT{Ps. 25:4-5 ; Ps. 27:11.}.
\VS{10}Car il n'y a rien de droit dans leur bouche; leur cœur est rempli de malice, leur gosier est un sépulcre ouvert, ils flattent de leur langue\FTNT{Ps. 10:7 ; Ps. 12:3 ; Ro. 3:13.}.
\VS{11}Ô Dieu ! Fais-leur leur procès, qu'ils échouent dans leurs entreprises ! Chasse-les au loin, à cause du grand nombre de leurs transgressions ! Car ils se sont rebellés contre toi.
\VS{12}Mais que tous ceux qui se confient en toi se réjouiront, qu'ils soient dans la joie perpétuellement, et que tu sois leur protecteur ; et que ceux qui aiment ton Nom s'égayent en toi !
\VS{13}Car tu bénis le juste, ô Yahweh ! Et tu l'entoures de ta bienveillance comme d'un bouclier.
\Chap{6}
\TextTitle{La miséricorde de Yahweh}
\VerseOne{}Psaume de David, donné au chef des chantres, pour le chanter pour le chanter en Neguinoth, sur Sheminith.
\VS{2}Yahweh, ne me punis pas dans ta colère et ne me châtie pas dans ta fureur\FTNT{Jé. 10:24.}.
\VS{3}Yahweh, aie pitié de moi ! Car je suis sans aucune force. Guéris-moi, ô Yahweh ! Car mes os sont épouvantés.
\VS{4}Même mon âme est fort troublée ; et toi, ô Yahweh ! Jusqu'à quand ?
\VS{5}Reviens, Yahweh ! Délivre mon âme. Sauve-moi, à cause de ta miséricorde.
\VS{6}Car celui qui meurt n'a plus ton souvenir ; qui te célébrera dans le scheol\FTNT{Es. 38 : 18 ; Ps. 88 : 11 ; Ps. 115 : 17.} ?
\VS{7}Je m'épuise à force de gémir; chaque nuit ma couche est baignée de mes larmes\FTNT{Job 7 : 3-4.}, mon lit est arrosé de mes pleurs.
\VS{8}J'ai le visage usé par le chagrin\FTNT{Ps. 31 : 10.}; il vieillit à cause de tous ceux qui m'oppriment.
\VS{9}Retirez-vous loin de moi, vous tous ouvriers d'iniquité\FTNT{Mt. 7:23 ; Mt. 25 : 41 ; Lu. 13 : 27.}! Car Yahweh a entendu la voix de mes pleurs.
\VS{10}Yahweh a entendu ma supplication, Yahweh a reçu ma prière.
\VS{11}Tous mes ennemis sont confondus, saisis d'épouvante; ils reculent soudain, honteux.
\Chap{7}
\TextTitle{La délivrance se trouve auprès de Yahweh}
\VerseOne{}Shiggaïon de David, chantée à Yahweh, au sujet de Cusch, le Benjamite.
\VS{2}Yahweh, mon Dieu ! Je cherche en toi mon refuge. Sauve-moi de tous mes persécuteurs et délivre-moi,
\VS{3}afin qu'ils ne me déchirent pas, comme un lion qui dévore sans qu'il n'y ait personne qui me secoure.
\VS{4}Yahweh, mon Dieu ! Si j'ai commis une telle action, s'il y a de l'iniquité dans mes mains,
\VS{5}si j'ai rendu le mal à celui qui était paisible envers moi, si j'ai dépouillé celui qui m'opprimait sans cause,
\VS{6}que l'ennemi me poursuive et m'atteigne, qu'il foule à terre ma vie, et qu'il couche ma gloire dans la poussière ! Sélah.
\VS{7}Lève-toi, ô Yahweh ! Dans ta colère, lève-toi contre la fureur de mes adversaires. Réveille-toi pour me secourir, ordonne un jugement !
\VS{8}Que l'assemblée des peuples t'environne ! Monte au-dessus d'elle vers les lieux élevés !
\VS{9}Yahweh juge les peuples : Rends-moi justice, ô Yahweh\FTNT{Ps. 9 : 5.} ! Selon ma droiture et selon mon intégrité !
\VS{10}Que la malice des méchants prenne fin, et affermis le juste, toi qui sondes les cœurs et les reins\FTNT{Jé. 11:20 ; Jé. 17:10.}, ô Dieu juste !
\VS{11}Mon bouclier est en Dieu, qui délivre ceux qui sont droits de cœur.
\VS{12}Dieu est un juste juge, Dieu s'irrite en tout temps.
\VS{13}Si le méchant ne se convertit pas, Dieu aiguise son épée\FTNT{De. 32 : 41.}, il bande son arc, et vise.
\VS{14}Il dirige sur lui des traits meurtriers, il rend ses flèches brûlantes.
\VS{15}Voici, le méchant prépare le mal, il conçoit l'iniquité, et il enfante le mensonge\FTNT{Ja. 1:15.}.
\VS{16}Il fait une fosse, il la creuse, et il tombe dans la fosse qu'il a faite\FTNT{Ps. 9 : 16.}.
\VS{17}Son travail retourne sur sa tête, et sa violence redescend sur son front.
\VS{18}Je célébrerai Yahweh à cause de sa justice, je psalmodierai le Nom de Yahweh, du Très-Haut.
\Chap{8}
\TextTitle{Magnificence de Dieu et vanité de l'homme}
\VerseOne{}Psaume de David, donné au chef des chantres, pour le chanter sur guitthith.
\VS{2}Yahweh, notre Seigneur ! Que ton Nom est magnifique sur toute la terre ! Ta majesté s'élève au-dessus des cieux\FTNT{Es. 6 : 3.}.
\VS{3}Par la bouche des petits enfants et de ceux qui tètent\FTNT{Mt. 21 : 16.}, tu as fondé ta puissance, à cause de tes adversaires, afin de faire cesser l'ennemi et le vindicatif.
\VS{4}Quand je regarde tes cieux, l'ouvrage de tes doigts, la lune et les étoiles que tu as fixées:
\VS{5}Qu'est-ce que l'homme, pour que tu te souviennes de lui ? Et le fils de l'homme, pour que tu le visites\FTNT{Dans ce passage, il est question de l'incarnation de Yahweh afin de nous sauver (1 Co. 15:45-49 ; 1 Ti. 3:16 ; Hé. 2:14). Jésus-Christ s'est lui-même nommé « fils de l'homme » (Lu. 9:22-26), littéralement « fils d'Adam », d'ailleurs cette expression apparaît dans les évangiles plus de quatre-vingt fois.} ?
\VS{6}Tu l'as fait de peu inférieur aux anges, et tu l'as couronné de gloire et d'honneur.
\VS{7}Tu lui as donné la domination sur les œuvres de tes mains, tu as tout mis sous ses pieds\FTNT{1 Co. 15:27.},
\VS{8}les brebis comme les bœufs, les animaux des champs,
\VS{9}les oiseaux du ciel et les poissons de la mer, tout ce qui parcourt les sentiers des mers.
\VS{10}Yahweh, notre Seigneur ! Que ton Nom est magnifique sur toute la terre !
\Chap{9}
\TextTitle{Louange à Yahweh, l'auteur de nos victoires}
\VerseOne{}Psaume de David, donné au chef des chantres, pour le chanter sur Muth-Labben.
\VS{2}Je célébrerai de tout mon cœur Yahweh, je raconterai toutes tes merveilles.
\VS{3}Je me réjouirai et je m'égaierai en toi, je chanterai ton Nom, ô Très-Haut !
\VS{4}Mes ennemis reculent, ils trébuchent, ils périssent devant ta face.
\VS{5}Car tu soutiens mon droit et ma cause, tu sièges sur ton trône en juste juge.
\VS{6}Tu châties les nations, tu détruis le méchant, tu effaces leur nom pour toujours, et à perpétuité.
\VS{7}Plus d'ennemis ! Les désolations ont-elles pris fin ? As-tu aussi rasé les villes pour toujours ? Leur mémoire est perdue avec elles.
\VS{8}Mais Yahweh sera assis éternellement, il a établi son trône pour juger.
\VS{9}Il juge le monde avec justice, il juge les peuples avec droiture\FTNT{Ps. 96:13 ; Ps. 98:9.}.
\VS{10}Yahweh est un refuge pour l'opprimé, un refuge au temps de la détresse\FTNT{Ps. 37:39 ; Ps. 46:2 ; Ps. 91:2.}.
\VS{11}Ceux qui connaissent ton Nom se confient en toi\FTNT{Pr. 3:5.}. Car tu n'abandonnes point ceux qui te cherchent, ô Yahweh !
\VS{12}Chantez à Yahweh qui habite en Sion, annoncez ses exploits parmi les peuples !
\VS{13}Lorsqu'il recherche le sang versé, il se souvient des malheureux? il n'oublie pas le cris des affligés.
\VS{14}Aie pitié de moi, Yahweh ! Vois la misère où me réduisent mes ennemis, enlève-moi des portes de la mort,
\VS{15}afin que je raconte toutes tes louanges, dans les portes de la fille de Sion. Je me réjouirai de la délivrance\FTNT{Voir commentaire en Es. 26:1.} que tu m'auras donnée.
\VS{16}Les nations tombent dans la fosse qu'elles ont faite\FTNT{Ps. 10:2 ; Ps. 35:7.}, leur pied se prend au filet qu'elles ont caché.
\VS{17}Yahweh se fait connaître, il fait justice, le méchant est enlacé dans l'ouvrage de ses mains. Jeu d'instruments. Sélah.
\VS{18}Les méchants retournent dans le scheol, toutes les nations qui oublient Dieu.
\VS{19}Car le pauvre n'est point oublié à jamais, l'espérance des affligés ne périt pas à toujours.
\VS{20}Lève-toi, ô Yahweh ! Que l'homme mortel ne triomphe point ! Que les nations soient jugées devant ta face !
\VS{21}Frappe-les de terreur, ô Yahweh ! Que les peuples sachent qu'ils ne sont que des hommes mortels\FTNT{Es. 51:12.}! Sélah.
\Chap{10}
\TextTitle{Appel au jugement de Dieu sur les méchants}
\VerseOne{}Pourquoi, ô Yahweh ! Te tiens-tu éloigné ? Pourquoi te caches-tu au temps où nous sommes dans la détresse\FTNT{Ps. 13:2 ; Ps. 44:24.} ?
\VS{2}Le méchant par son orgueil poursuit ardemment les affligés, mais ils seront pris par les machinations qu'ils ont préméditées\FTNT{Ps. 7:15-16 ; Ps. 9:16 ; Ps. 35:8.}.
\VS{3}Car le méchant se glorifie du désir de son âme, il bénit l'avare et il méprise Yahweh.
\VS{4}Le méchant dit avec arrogance : Il ne fera pas d'enquête ! Il n'y a point de Dieu\FTNT{Ps. 14:1 ; Ps. 53:2.} ! Voilà toutes ses pensées.
\VS{5}Ses voies réussissent en tout temps; tes jugements sont éloignés de lui, il souffle contre tous ses adversaires.
\VS{6}Il dit en son cœur : Je ne chancelle pas, je suis pour toujours à l'abri du malheur !
\VS{7}Sa bouche est pleine de malédictions, de tromperies et de fraudes ; il n'y a sous sa langue qu'oppression et outrage\FTNT{Ps. 59:7-8 ; Ps. 64:3-4 ; Job. 20:12.}.
\VS{8}Il se tient aux embûches dans des villages, il tue l'innocent dans des lieux cachés, ses yeux épient le malheureux.
\VS{9}Il se tient aux aguets dans un lieu caché, comme un lion dans sa tanière, il se tient aux aguets pour attraper l'affligé ; il attrape l'affligé, l'attirant dans son filet.
\VS{10}Il se courbe, il se baisse, et les malheureux tombent dans ses griffes.
\VS{11}Il dit en son cœur : Dieu oublie ! Il cache sa face, il ne le verra jamais\FTNT{Ps 94:7.}!
\VS{12}Lève-toi, ô Yahweh ! Lève ta main ! N'oublie pas les malheureux !
\VS{13}Pourquoi le méchant méprise-t-il Dieu ? Il dit en son cœur que tu ne punis pas.
\VS{14}Tu regardes cependant, car tu vois la peine et la souffrance, pour prendre en main leur cause ; c'est toi qui viens en aide à l'orphelin.
\VS{15}Brise le bras du méchant, punis ses iniquités et qu'il disparaisse à tes yeux !
\VS{16}Yahweh est Roi à toujours et à perpétuité\FTNT{Ps. 29:10 ; Ps. 145:13 ; Ps. 146:10 ; La. 5:19.} ; les nations sont exterminées de sa terre.
\VS{17}Tu entends les vœux de ceux qui souffrent, ô Yahweh ! Tu affermis leur cœur; tu prêtes l'oreille
\VS{18}pour rendre justice à l'orphelin et à l'opprimé, afin que l'homme mortel tiré de la terre cesse d'inspirer l'effroi.
\Chap{11}
\TextTitle{Yahweh, le refuge des hommes droits}
\VerseOne{}Psaume de David, donné au chef des chantres. C'est en Yahweh que je cherche un refuge. Comment un homme peut-il dire à mon âme : Fuis dans vos montagnes, comme un oiseau ?
\VS{2}En effet, les méchants bandent l'arc\FTNT{Ps. 37:14.}, ils ajustent leur flèche sur la corde, pour tirer dans l'ombre sur ceux dont le cœur est droit.
\VS{3}Quand les fondements sont renversés, que fera le juste ?
\VS{4}Yahweh est dans son saint temple, Yahweh a son trône dans les cieux ; ses yeux contemplent, ses paupières sondent les fils des hommes.
\VS{5}Yahweh sonde le juste et le méchant ; et son âme hait celui qui aime la violence.
\VS{6}Il fait pleuvoir sur les méchants des charbons, du feu et du soufre\FTNT{Ez. 38:22.} ; un vent brûlant, c'est le calice qu'ils ont en partage.
\VS{7}Car Yahweh est juste, il aime la justice ; les hommes droits contemplent sa face.
\Chap{12}
\TextTitle{Le langage des lèvres arrogantes}
\VerseOne{}Psaume de David, donné au chef des chantres pour le chanter sur Sheminith.
\VS{2}Sauve, ô Yahweh ! Car les hommes pieux s'en vont, les fidèles disparaissent parmi les fils de l'homme.
\VS{3}Chacun dit des faussetés à son compagnon avec des lèvres flatteuses et ils parlent avec un cœur double.
\VS{4}Que Yahweh retranche toutes les lèvres flatteuses, la langue qui parle fièrement\FTNT{Ps. 17:10.},
\VS{5}parce qu'ils disent : Nous sommes puissants par nos langues, nous avons nos lèvres avec nous ; qui serait notre maître ?
\VS{6}A cause du mauvais traitement que l'on fait aux malheureux, à cause du gémissement des pauvres, je me lèverai maintenant, dit Yahweh, je mettrai en sûreté celui à qui l'on tend des pièges.
\VS{7}Les paroles de Yahweh sont des paroles pures, c'est un argent éprouvé sur terre au creuset\FTNT{Ps. 19:10 ; Ps. 119:140 ; Pr. 30:5.}, et sept fois épuré.
\VS{8}Toi, Yahweh ! Garde-les, préserve cette race à jamais.
\VS{9}Les méchants se promènent de toutes parts, tandis que des gens abjects sont élevés parmi les fils des hommes.
\Chap{13}
\TextTitle{Savoir attendre le secours de Dieu}
\VerseOne{}Psaume de David, donné au chef des chantres.
\VS{2}Yahweh, jusqu'à quand m'oublieras-tu ? Sera-ce pour toujours ? Jusqu'à quand me cacheras-tu ta face\FTNT{Ps. 10:1 ; Ps. 27:9.} ?
\VS{3}Jusqu'à quand consulterai-je mon âme, affligerai-je mon cœur tous les jours ? Jusqu'à quand mon ennemi s'élèvera-t-il contre moi ?
\VS{4}Yahweh, mon Dieu ! Regarde, exauce-moi, illumine mes yeux, de peur que je ne dorme du sommeil de la mort,
\VS{5}de peur que mon ennemi ne dise : J'ai eu le dessus! Que mes adversaires ne se réjouissent, si je venais à tomber\FTNT{Ps. 25:2.}.
\VS{6}Mais moi, je me confie en ta bonté, mon cœur se réjouira de la délivrance que tu m'auras donnée ; je chanterai à Yahweh, parce qu'il m'a fait du bien.
\Chap{14}
\TextTitle{L'insensé ne cherche pas Dieu}
\VerseOne{}Psaume de David, donné au chef des chantres. L'insensé dit en son cœur : Il n'y a point de Dieu\FTNT{Ceux qui ne croient pas en l'existence de Dieu sont appelés insensés. En effet, la création révèle l'existence du Créateur (Ro. 1:19-20).} ! Ils se sont corrompus, ils ont commis des actions abominables ; il n'y a personne qui fasse le bien.
\VS{2}Yahweh regarde des cieux les fils de l'homme, pour voir s'il y a quelqu'un qui soit intelligent, qui cherche Dieu\FTNT{Ps. 33:13 ; Job. 28:24.}.
\VS{3}Ils se sont tous égarés, ils se sont tous ensemble rendus odieux, il n'y a personne qui fasse le bien, pas même un seul\FTNT{Tous les hommes naissent pécheurs (Ro. 3:10-23).}.
\VS{4}Tous ces ouvriers d'iniquité n'ont-ils point de connaissance ? Ils dévorent mon peuple, ils le prennent pour nourriture ; ils n'invoquent point Yahweh.
\VS{5}Là, ils seront saisis d'une grande frayeur, car Dieu est avec la race des justes.
\VS{6}Jetez l'opprobre sur l'espérance du malheureux…Yahweh est son refuge.
\VS{7}Oh ! Qui fera partir de Sion la délivrance d'Israël\FTNT{C'est le Messie qui délivrera Israël (Ro. 11:25-27).} ? Quand Yahweh ramenera son peuple captif, Jacob se réjouira, Israël se réjouira.
\Chap{15}
\TextTitle{L'homme que Yahweh agrée}
\VerseOne{}Psaume de David. Yahweh, qui séjournera dans ta tente ? Qui demeurera sur ta montagne sainte\FTNT{Ps. 24:3-4.} ?
\VS{2}Celui qui marche dans l'intégrité, qui fait ce qui est juste, et qui profère la vérité telle qu'elle est dans son cœur,
\VS{3}qui ne calomnie point avec sa langue, qui ne fait point de mal à son ami, qui ne diffame point son prochain.
\VS{4}Il regarde avec dédain celui qui est méprisable, mais il honore ceux qui craignent Yahweh; il ne se rétracte point s'il fait un serment à son préjudice.
\VS{5}Il n'exige point d'intérêt de son argent, et il n'accepte point de présent contre l'innocent\FTNT{Lé. 25:36 ; De. 16:19 ; De. 27:25.}. Celui qui fait ces choses ne sera jamais ébranlé.
\Chap{16}
\TextTitle{Yahweh, la source de la vie}
\VerseOne{}Mictam de David. Garde-moi, ô Dieu ! Car je cherche en toi mon refuge.
\VS{2}Je dis à Yahweh : Tu es mon Seigneur, tu es mon bonheur!
\VS{3}Les saints qui sont dans le pays, les hommes pieux sont l'objet de toute mon affection.
\VS{4}On multiplie les peines, on court après les dieux étrangers : Je ne répands pas leurs libations de sang et je ne mets pas leurs noms sur mes lèvres.
\VS{5}Yahweh est la part de mon héritage et ma coupe ; tu maintiens mon lot;
\VS{6}un héritage délicieux m'est échu, une belle possession m'est accordée.
\VS{7}Je bénirai Yahweh qui me donne conseille; je le bénirai même durant les nuits dans lesquelles mes reins m'enseignent.
\VS{8}J'ai constamment Yahweh sous mes yeux ; quand il est à ma droite, je ne chancelle pas\FTNT{Ps. 109:31 ; Ps. 110:5 ; Ac. 2:25.}.
\VS{9}C'est pourquoi mon cœur se réjouit, mon esprit se réjouit et mon corps repose en sécurité.
\VS{10}Car tu n'abandonneras point mon âme au scheol, tu ne permettras point que ton bien-aimé voie la corruption\FTNT{Le roi David prophétise ici la résurrection du Messie.}.
\VS{11}Tu me feras connaître le chemin de la vie ; il y a d'abondantes joies devant ta face, des délices éternels à ta droite.
\Chap{17}
\TextTitle{L'assurance en Dieu}
\VerseOne{}Prière de David. Yahweh, écoute la droiture, sois attentif à mon cri, prête l'oreille à ma prière faite avec des lèvres sans tromperie !
\VS{2}Que ma justice paraisse devant ta face, que tes yeux contemplent mon intégrité !
\VS{3}Tu as sondé mon cœur\FTNT{Ps. 139:1 ; Jé. 12:3.}, tu l'as visité de nuit, tu m'as examiné, tu n'as rien trouvé : Ma pensée ne va point au-delà de ma parole.
\VS{4}Quant aux actions des hommes, selon la parole de tes lèvres, je me tiens en garde contre la voie du violent.
\VS{5}Mes pas sont fermes dans tes sentiers, mes pieds ne chancellent point.
\VS{6}Je t'invoque, car tu m'exauces, ô Dieu ! Incline ton oreille vers moi, écoute mes paroles !
\VS{7}Signale ta bonté, toi qui sauves ceux qui cherchent un refuge, et qui par ta droite les délivres de leurs adversaires !
\VS{8}Garde-moi comme la prunelle de l'œil, cache-moi à l'ombre de tes ailes\FTNT{Mt. 23:37.},
\VS{9}contre les méchants qui me traitent violemment, mes ardents ennemis qui m'entourent.
\VS{10}Ils sont enfermés dans leur propre graisse, leur bouche parle avec orgueil.
\VS{11}Maintenant, ils nous environnent à chaque pas que nous faisons ; ils jettent leur regard pour nous étendre par terre.
\VS{12}Ils ressemblent au lion qui ne demande qu'à déchirer, et au lionceau qui se tient dans les lieux cachés.
\VS{13}Lève-toi, ô Yahweh, devance-les, renverse-les ! Délivre mon âme du méchant par ton épée.
\VS{14}Yahweh, délivre-moi par ta main de ces gens, des gens de ce monde ! Leur part est dans cette vie et tu remplis leur ventre de tes biens ; leurs enfants sont rassasiés et ils laissent leurs restes à leurs petits-enfants.
\VS{15}Mais moi, dans mon innocence, je verrai ta face\FTNT{Job. 19:26-27 ; Ps. 16:10-11.}, et je me rassasierai de ton image, dès mon réveil.
\Chap{18}
\TextTitle{Louange à Dieu, le bouclier des saints}
\VerseOne{}Psaume de David, serviteur de Yahweh, qui adressa à Yahweh les paroles de ce cantique le jour où Yahweh l'eut délivré de la main de Saül. Au chef des chantres.
\VS{2}Il dit donc : Je t'aime, ô Yahweh, ma force !
\VS{3}Yahweh est mon rocher\FTNT{Yahweh est le rocher sur lequel s'appuyait David. Paul enseigne que ce rocher était Jésus-Christ (1 Co. 10:1-4). Voir commentaire en Es. 8:13-17.}, ma forteresse et mon libérateur ! Mon Dieu, mon rocher où je trouve un refuge ! Mon bouclier, la force qui me sauve, ma haute retraite !
\VS{4}Je crie : Loué soit Yahweh ! Et je suis délivré de mes ennemis.
\VS{5}Les liens de la mort m'avaient environné et des torrents de destruction m'avaient épouvanté.
\VS{6}Les liens du scheol m'avaient entouré, les filets de la mort m'avaient surpris\FTNT{Ps. 116:3.}.
\VS{7}Dans ma détresse, j'ai invoqué Yahweh, j'ai crié à mon Dieu ; il a entendu ma voix de son palais, mon cri est parvenu devant lui à ses oreilles.
\VS{8}La terre fut ébranlée et trembla, les fondements des montagnes croulèrent\FTNT{Es. 5:25 ; Es. 64:1-3 ; Jé. 4:24 ; Ps. 104:32.}, et ils furent ébranlés, parce qu'il était irrité.
\VS{9}Une fumée montait de ses narines, et de sa bouche sortait un feu dévorant, des charbons embrasés.
\VS{10}Il abaissa les cieux et descendit : Il y avait une épaisse nuée sous ses pieds.
\VS{11}Il était monté sur un chérubin, et il volait, il était porté sur les ailes du vent\FTNT{Ps. 104:3.}.
\VS{12}Il faisait des ténèbres sa demeure secrète, autour de lui était sa tente, il était enveloppé des eaux obscures et de sombres nuages.
\VS{13}De la splendeur qui le précédait s'échappaient les nuées, lançant de la grêle et des charbons de feu.
\VS{14}Yahweh tonna dans les cieux, le Très-Haut fit retentir sa voix avec de la grêle et des charbons de feu.
\VS{15}Il tira ses flèches, et écarta mes ennemis, il lança des éclairs et les mit en déroute\FTNT{Ps. 77:18.}.
\VS{16}Le fond des eaux parut, les fondements du monde furent découverts, par ta menace, ô Yahweh ! Par le souffle du vent de tes narines.
\VS{17}Il étendit la main d'en haut, il m'enleva et me retira des grandes eaux\FTNT{2 S. 22:17.} ;
\VS{18}il me délivra de mon puissant ennemi, et de ceux qui me haïssaient, car ils étaient plus forts que moi.
\VS{19}Ils m'avaient surpris au jour de ma détresse, mais Yahweh, fut mon appui.
\VS{20}Il m'a mis au large, il m'a délivré, parce qu'il m'aime.
\VS{21}Yahweh m'a rendu selon ma justice, il m'a traité selon la pureté de mes mains\FTNT{Ps. 18:25 ; Ps. 7:9.},
\VS{22}car j'ai observé les voies de Yahweh et je n'ai point été coupable envers mon Dieu.
\VS{23}Car j'ai eu devant moi toutes ses ordonnances et je ne me suis point écarté de ses lois.
\VS{24}J'ai été intègre envers lui, et je me suis tenu en garde contre mon iniquité.
\VS{25}Aussi Yahweh m'a rendu selon ma justice, selon la pureté de mes mains devant ses yeux,
\VS{26}avec celui qui est bon, tu te montres bon, avec l'homme droit tu agis selon la droiture.
\VS{27}Avec celui qui est pur, tu te montres pur, et avec le pervers tu agis selon sa perversité.
\VS{28}Car tu sauves le peuple affligé et tu abaisses les yeux hautains\FTNT{Es. 2:11 ; Es. 5:15.}.
\VS{29}Tu fais briller ma lumière ; Yahweh, mon Dieu, éclaire mes ténèbres.
\VS{30}Avec toi, je me précipite sur un corps d'armée, avec mon Dieu je franchis la muraille.
\VS{31}Les voies de Dieu sont sans défaut ; la parole de Yahweh est éprouvée\FTNT{De. 32:4 ; Ps. 19:8-9 ; Da. 4:37.} ; il est un bouclier pour tous ceux qui se confient en lui.
\VS{32}Car qui est Dieu, si ce n'est Yahweh ? Et qui est un rocher, si ce n'est notre Dieu ?\FTNT{1 S. 2:2 ; 2 S. 22:32.}
\VS{33}C'est le Dieu qui me ceint de force, et qui me conduit dans la voie droite.
\VS{34}Il rend mes pieds semblables à ceux des biches\FTNT{2 S. 2:18.}, et il me place sur mes lieux élevés.
\VS{35}Il exerce mes mains au combat, tellement qu'un arc d'airain a été rompu avec mes bras.\FTNT{Job. 20:24.}.
\VS{36}Tu me donnes le bouclier de ton salut, ta droite me soutient, et je deviens puissant par ta bonté.
\VS{37}Tu élargis le chemin sous mes pas, et mes pieds ne chancellent point.
\VS{38}Je poursuis mes ennemis, je les atteints, et je ne reviens pas avant de les avoir anéantis.
\VS{39}Je les brise et ils ne peuvent se relever ; ils tombent sous mes pieds.
\VS{40}Car tu m'as ceint de force pour le combat, tu fais plier sous moi ceux qui s'élevaient contre moi.
\VS{41}Tu fais tourner le dos à mes ennemis devant moi, et j'extermine ceux qui me haïssaient.
\VS{42}Ils crient, mais il n'y a point de libérateur! Ils crient à Yahweh, mais il ne leur répond point!
\VS{43}Je les brise comme la poussière qui est dispersée par le vent et je les foule comme la boue des rues.
\VS{44}Tu me délivres des séditions du peuple, tu m'établis chef des nations. Un peuple que je ne connais point m'est asservi.
\VS{45}Ils m'obéissent au premier ordre, les fils de l'étranger me flattent.
\VS{46}Les étrangers s'enfuient et ils tremblent de peur dans leurs forteresses.
\VS{47}Yahweh est vivant, et béni soit mon rocher ! Que le Dieu de mon salut soit exalté !
\VS{48}Le Dieu qui est mon vengeur et qui m'assujettit les peuples,
\VS{49}c'est lui qui me délivre de mes ennemis ! Tu m'élèves au-dessus de mes adversaires, tu me sauves de l'homme violent.
\VS{50}C'est pourquoi, ô Yahweh, je te célébrerai parmi les nations ! Et je chanterai des psaumes à ton Nom.
\VS{51}Il accorde de grandes délivrances à son roi, et il fait miséricorde à son oint, à David, et à sa postérité, pour toujours.
\Chap{19}
\TextTitle{La création exalte la grandeur de Dieu}
\VerseOne{}Psaume de David, donné au chef des chantres.
\VS{2}Les cieux racontent la gloire de Dieu, et l'étendue met en évidence l'oeuvre de ses mains.
\VS{3}Un jour en instruit un autre jour, et une nuit fait connaître sa science à l'autre nuit.
\VS{4}Ce n'est pas un langage, ce ne sont pas des paroles dont le cri ne soit point entendu :
\VS{5}Leur retentissement couvre toute la terre, et leur voix est allée jusqu'aux extrémités du monde\FTNT{Ro. 10:18.}. Il a dressé une tente pour le soleil.
\VS{6}Et le soleil est semblable à un époux sortant de sa chambre ; il s'élance sur le sentier avec la joie d'un homme vaillant;
\VS{7}il se lève à l'extrémité des cieux et achève sa course à l'autre extrémité\FTNT{Ec. 1:5.}: Rien ne se dérobe à sa chaleur.
\VS{8}La loi de Yahweh est parfaite, elle restaure l'âme ; le témoignage de Yahweh est fidèle, il donne la sagesse au simple\FTNT{2 S. 22:31 ; Ps. 18:31 ; Ps. 119:130.}.
\VS{9}Les ordonnances de Yahweh sont droites, elles réjouissent le cœur ; les commandements de Yahweh sont purs, ils éclairent les yeux.
\VS{10}La crainte de Yahweh est pure, elle subsiste à toujours ; les jugements de Yahweh sont vrais, et ils sont tous justes.
\VS{11}Ils sont plus précieux que l'or, que beaucoup d'or fin ; et plus doux que le miel, que celui qui coule des rayons de miel\FTNT{Ps. 119:103.}.
\VS{12}Ton serviteur aussi en reçoit l'éclairage ; pour qui les observe la récompense est grande.
\VS{13}Qui connaît ses fautes commises par erreur ? Purifie-moi de mes fautes cachées.
\VS{14}Eloigne aussi ton serviteur des actions commises par fierté, en sorte qu'elles ne dominent point sur moi, qu'elles cessent et que je sois nettoyé de mes grands péchés!
\VS{15}Que les propos de ma bouche et la méditation de mon cœur te soient agréables, ô Yahweh ! Mon rocher et mon rédempteur\FTNT{Voir commentaire en Es. 60:16.}!
\Chap{20}
\TextTitle{Recours à l'intervention de Dieu}
\VerseOne{}Psaume de David, donné au chef des chantres.
\VS{2}Que Yahweh te réponde au jour de la détresse, que le Nom du Dieu de Jacob te protège !
\VS{3}Qu'il envoie ton secours du saint lieu, et qu'il te soutienne de Sion !
\VS{4}Qu'il se souvienne de toutes tes offrandes, qu'il réduise en cendres ton holocauste ! Sélah.
\VS{5}Qu'il te donne ce que ton cœur désire, et qu'il fasse réussir tes desseins !
\VS{6}Nous triompherons dans ton salut, nous lèverons la bannière au Nom de notre Dieu ; Yahweh exaucera tous tes vœux.
\VS{7}Je sais déjà que Yahweh sauve son oint ; il l'exaucera des cieux, de sa sainte demeure, par le secours puissant de sa droite.
\VS{8}Les uns se vantent de leurs chars, et les autres de leurs chevaux ; mais nous, nous glorifierons le Nom de Yahweh, notre Dieu.
\VS{9}Eux ils plient, et ils tombent ; nous, nous tenons ferme, et restons debout.
\VS{10}Yahweh, sauve le roi ! Qu'il nous réponde quand nous crions à lui !
\Chap{21}
\TextTitle{La protection de Dieu sur le roi}
\VerseOne{}Psaume de David, donné au chef des chantres.
\VS{2}Yahweh, le roi se réjouit de ta puissance, ton secours le remplit d'allégresse !
\VS{3}Tu lui as donné ce que désirait son cœur et tu n'as point refusé ce que demandaient ses lèvres. Sélah.
\VS{4}Car tu l'as prévenu par les bénédictions de ta bonté, et tu as mis sur sa tête une couronne d'or pur.
\VS{5}Il t'avait demandé la vie, et tu la lui as donnée, une vie longue pour toujours et à perpétuité.
\VS{6}Sa gloire est grande à cause de ton salut, tu l'as couvert de majesté et d'honneur.
\VS{7}Tu le rends à jamais un objet de bénédictions, tu le combles de joie devant ta face\FTNT{Ps. 16:11.}.
\VS{8}Le roi se confie en Yahweh, et par la bonté du Très-Haut, il ne chancelle pas\FTNT{Ps. 16:8.}.
\VS{9}Ta main trouvera tous tes ennemis, ta droite trouvera tous ceux qui te haïssent.
\VS{10}Tu les rendras tels qu'une fournaise ardente le jour où l'on verra ta face ; Yahweh les engloutira dans sa colère, et le feu les consumera.
\VS{11}Tu feras périr leur fruit de la terre et leur race du milieu des fils des hommes.
\VS{12}Car ils ont projeté du mal contre toi et ils ont conçu de mauvais desseins dont ils ne pourront venir à bout.
\VS{13}Parce que tu leur feras tourner le dos, et avec ton arc tu tireras sur eux.
\VS{14}Elève-toi, Yahweh, par ta force ! Nous chanterons et célébrerons ta puissance.
\Chap{22}
\TextTitle{Les souffrances du Messie}
\VerseOne{}Psaume de David, donné au chef des chantres, pour le chanter sur Ajéleth-Hashakhar.
\VS{2}Mon Dieu ! Mon Dieu ! Pourquoi m'as-tu abandonné\FTNT{Le Ps. 22 est une description détaillée de la mort par crucifixion du Seigneur Jésus-Christ (Mt. 27:45-46).}, et t'éloignes-tu sans me secourir, sans écouter mes plaintes ?
\VS{3}Mon Dieu ! Je crie le jour, mais tu ne réponds point ; la nuit, et je n'ai point de repos.
\VS{4}Pourtant tu es le Saint, tu habites au milieu des louanges d'Israël.
\VS{5}Nos pères se sont confiés en toi ; ils se sont confiés, et tu les as délivrés.
\VS{6}Ils ont crié vers toi, et ils ont été délivrés ; ils se sont appuyés sur toi, et ils n'ont point été confus\FTNT{Es. 49:23 ; Ps. 25:3 ; Ps. 31:2.}.
\VS{7}Et moi, je suis un ver et non un homme, l'opprobre des hommes et le méprisé du peuple\FTNT{Es. 53:2-3.}.
\VS{8}Tous ceux qui me voient se moquent de moi, ils ouvrent les lèvres, secouent la tête\FTNT{Ps. 109:25 ; Mt. 27:39.} :
\VS{9}Recommande-toi à Yahweh ! Qu'il te délivre, et qu'il te sauve, puisqu'il prend plaisir en toi\FTNT{Mt. 27:43.} !
\VS{10}Cependant, c'est toi qui m'as tiré hors du ventre de ma mère, qui m'as mis en sûreté lorsque j'étais sur les mamelles de ma mère.
\VS{11}J'ai été sous ta garde, dès le sein maternel, tu as été mon Dieu dès le ventre de ma mère\FTNT{Es. 49:1.}.
\VS{12}Ne t'éloigne point de moi, car la détresse est près de moi, et il n'y a personne qui me secoure\FTNT{Ps. 69:21.} !
\VS{13}Plusieurs taureaux sont autour de moi, de puissants taureaux de Basan m'entourent.
\VS{14}Ils ouvrent leur gueule contre moi, comme un lion qui déchire et rugit.
\VS{15}Je suis comme de l'eau qui s'écoule, et tous mes os se séparent ; mon cœur est comme de la cire, il se fond dans mes entrailles.
\VS{16}Ma force se dessèche comme l'argile, et ma langue s'attache à mon palais ; tu me réduis à la poussière de la mort.
\VS{17}Car des chiens m'environnent, une assemblée de méchants m'entoure, ils ont percé mes mains et mes pieds.
\VS{18}Je pourrais compter tous mes os un par un. Eux, ils m'examinent, ils me regardent.
\VS{19}Ils se partagent mes vêtements et tirent au sort ma tunique\FTNT{Mt. 27:35 ; Mc. 15:24 ; Lu. 23:33.}.
\VS{20}Et toi, Yahweh, ne t'éloigne point ! Ma force, hâte-toi de me secourir !
\VS{21}Délivre ma vie de l'épée, ma vie contre le pouvoir des chiens !
\VS{22}Sauve-moi de la gueule du lion, délivre-moi des cornes du buffle !
\VS{23}Je déclarerai ton Nom à mes frères, je te louerai au milieu de l'assemblée\FTNT{Hé. 2:12.}.
\VS{24}Vous qui craignez Yahweh, louez-le ! Toute la race de Jacob, glorifiez-le ! Toute la race d'Israël, redoutez-le !
\VS{25}Car il n'a ni mépris ni dédain pour les peines du misérable, et il ne lui cache point sa face, mais il l'écoute quand il crie à lui.
\VS{26}Tu seras l'objet de mes louanges dans la grande assemblée ; j'accomplirai mes vœux en présence de ceux qui te craignent\FTNT{Ps. 56:13.}.
\VS{27}Les malheureux mangeront et seront rassasiés, ceux qui cherchent Yahweh le loueront. Votre cœur vivra à perpétuité !
\VS{28}Toutes les extrémités de la terre s'en souviendront, ils se convertiront à Yahweh, et toutes les familles des nations se prosterneront devant toi\FTNT{Ps. 72:8-11 ; Ps. 86:9.}.
\VS{29}Car le règne appartient à Yahweh : Il domine sur les nations.
\VS{30}Tous les gens de la terre mangeront et se prosterneront devant lui ; tous ceux qui descendent dans la poussière s'inclineront, même celui qui ne peut conserver sa vie.
\VS{31}La postérité le servira, on parlera du Seigneur de génération en génération\FTNT{Es. 59:21 ; Es. 65:23 ; Ps. 110:3.}.
\VS{32}Ils viendront et ils publieront sa justice au peuple qui naîtra, parce qu'il aura fait ces choses.
\Chap{23}
\TextTitle{Le bon Berger}
\VerseOne{}Psaume de David. Yahweh est mon berger\FTNT{Yahweh, le bon berger, est notre Seigneur Jésus-Christ. Es. 40:11 ; Jé. 23:4 ; Jn. 10:11.}: Je ne manquerai de rien.
\VS{2}Il me fait reposer dans de verts pâturages, il me dirige près des eaux paisibles.
\VS{3}Il restaure mon âme, et me conduit dans les sentiers de la justice, à cause de son Nom.
\VS{4}Quand je marche dans la vallée de l'ombre de la mort, je ne crains aucun mal\FTNT{Ps. 118:6.}, car tu es avec moi : Ton bâton et ta houlette me consolent.
\VS{5}Tu dresses devant moi une table, en face de mes adversaires; tu oins d'huile ma tête et ma coupe déborde.
\VS{6}Le bonheur et la grâce m'accompagneront tous les jours de ma vie, et j'habiterai dans la maison de Yahweh jusqu'à la fin de mes jours.
\Chap{24}
\TextTitle{Accueil de Yahweh, le Roi de gloire}
\VerseOne{}Psaume de David. La terre appartient à Yahweh, avec tout ce qui est en elle\FTNT{Ex. 19:5 ; De. 10:14 ; Ps. 50:12 ; Job. 41:2 ; 1 Co. 10:26.}, le monde et ceux qui y habitent!
\VS{2}Car il l'a fondée sur les mers, et affermie sur les fleuves.
\VS{3}Qui pourra monter à la montagne de Yahweh ? Qui s'élèvera jusqu'à son lieu saint\FTNT{Ps. 15:1-2 ; Ps. 118:19.} ?
\VS{4}Celui qui a les mains pures et le cœur pur, qui ne livre point son âme au mensonge, et qui ne jure pas pour tromper.
\VS{5}Il obtiendra la bénédiction de Yahweh et la justice du Dieu de son salut.
\VS{6}Voilà le partage de la génération qui l'invoque, de ceux qui cherchent ta face de Jacob ! Sélah.
\VS{7}Portes, élevez vos linteaux; élevez-vous portes éternelles ! Que le Roi de gloire fasse son entrée !
\VS{8}Qui est ce Roi de gloire ? C'est Yahweh fort et puissant, Yahweh puissant dans les combats.
\VS{9}Portes, élevez vos linteaux; élevez-les aussi, vous portes éternelles! Que le Roi de gloire fasse son entrée !
\VS{10}Qui est ce Roi de gloire ? Yahweh des armées : Voilà le Roi de gloire! Sélah.
\Chap{25}
\TextTitle{La crainte de Dieu mène à la voie de Yahweh}
\VerseOne{}Psaume de David. [Aleph.] Yahweh, j'élève mon âme à toi.
\VS{2}[Beth.] Mon Dieu ! Je me confie en toi : Que je ne sois point honteux\FTNT{Ps. 22:5 ; Ps. 31:2.} ! Que mes ennemis ne triomphent point de moi!
\VS{3}[Guimel.] Tous ceux qui espèrent en toi ne seront point confus\FTNT{Ro. 10:11.} ; ceux qui agissent avec tromperie sans cause seront honteux.
\VS{4}[Daleth.] Yahweh ! Fais-moi connaître tes voies, enseigne-moi tes sentiers\FTNT{Ps. 27:11 ; Ps. 86:11 ; Ps. 143:10.}.
\VS{5}[He. Vav.] Fais-moi marcher selon la vérité, et instruis-moi, car tu es le Dieu de ma délivrance, je m'attends à toi tous les jours.
\VS{6}[Zayin.] Yahweh ! Souviens-toi de ta miséricorde et de ta bonté, car elles sont éternelles\FTNT{Jé. 33:11 ; Ps. 103:17 ; Ps. 106:1 ; Ps. 107:1 ; Ps. 117:2 ; Ps. 136:1-2.}.
\VS{7}[Heth.] Ne te souviens point des péchés de ma jeunesse ni de mes transgressions ; souviens-toi de moi selon ta miséricorde, à cause de ta bonté, ô Yahweh !
\VS{8}[Teth.] Yahweh est bon et droit : C'est pourquoi il enseigne aux pécheurs la voie.
\VS{9}[Yod.] Il conduit les humbles dans la justice, et il leur enseigne sa voie.
\VS{10}[Kaf.] Tous les sentiers de Yahweh sont miséricorde et fidélité, pour ceux qui gardent son alliance et son témoignage.
\VS{11}[Lamed.] Pour l'amour de ton Nom, ô Yahweh ! Tu me pardonneras mon iniquité, car elle est grande\FTNT{2 S. 24:10.}.
\VS{12}[Mem.] Qui est l'homme qui craint Yahweh ? Yahweh lui enseignera la voie qu'il doit choisir.
\VS{13}[Nun.] Son âme demeurera dans le bonheur, et sa postérité possédera la terre en héritage.
\VS{14}[Samech.] Le secret de Yahweh est pour ceux qui le craignent, et son alliance leur donne le savoir.
\VS{15}[Ayin.] Mes yeux sont continuellement sur Yahweh, car c'est lui qui sortira mes pieds du filet.
\VS{16}[Pe.] Tourne ta face vers moi, et aie pitié de moi, car je suis seul et affligé.
\VS{17}[Tsade.] Les angoisses de mon cœur augmentent ; sors-moi de ma détresse.
\VS{18}[Resh.] Vois ma misère et ma peine, et pardonne tous mes péchés.
\VS{19}[Resh.] Vois combien mes ennemis sont nombreux, et me haïssent d'une haine pleine de violence\FTNT{Jn. 15:25.}.
\VS{20}[Shin.] Garde mon âme et délivre-moi ! Que je ne sois point confus, car je me suis réfugié en toi!
\VS{21}[Tav.] Que l'innocence et la droiture me protègent, car je m'attends à toi!
\VS{22}[Pe.] Ô Dieu ! Rachète Israël de toutes ses détresses !
\Chap{26}
\TextTitle{Demeurer dans l'intégrité}
\VerseOne{}Psaume de David. Yahweh, rends-moi justice\FTNT{Ps. 43:1 ; Ps. 54:3.} ! Car je marche dans l'intégrité, je me confie en Yahweh, je ne chancelle pas.
\VS{2}Sonde-moi et éprouve-moi\FTNT{Ps. 11:4-5 ; Ps. 17:3 ; Ps. 139:23.}, Yahweh ! Fais passer au creuset mes reins et mon cœur ;
\VS{3}car ta grâce est devant mes yeux, et je marche dans ta vérité.
\VS{4}Je ne m'assieds pas avec les hommes faux\FTNT{Ps. 1:1 ; 1 Co. 5:9-11 ; 1 Co. 15:33.}, et je ne vais point avec les gens dissimulés.
\VS{5}Je hais la compagnie de ceux qui font le mal\FTNT{Ps. 101:2-5 ; Ps. 119:113.}, et je ne m'assieds pas avec les méchants.
\VS{6}Je lave mes mains dans l'innocence et je fais le tour de ton autel\FTNT{Ps. 73:13.}, ô Yahweh !
\VS{7}Pour faire entendre le cri de reconnaissance, et pour raconter toutes tes merveilles.
\VS{8}Yahweh, j'aime la demeure de ta maison, le lieu dans lequel est le tabernacle de ta gloire.
\VS{9}N'enlève pas mon âme avec les pécheurs, ma vie avec les hommes de sang,
\VS{10}dont les mains sont criminelles, et la droite pleine de présents.
\VS{11}Moi, je marche dans l'intégrité ; délivre-moi et aie pitié de moi !
\VS{12}Mon pied se tient dans la droiture ; je bénirai Yahweh dans les assemblées.
\Chap{27}
\TextTitle{La foi qui triomphe des épreuves}
\VerseOne{}Psaume de David. Yahweh est ma lumière\FTNT{Es. 60:19-20 ; Mi. 7:8 ; Ps. 118:6 ; Jn. 8:12 ; Ap. 21:23.} et mon salut : De qui aurai-je peur ? Yahweh est le soutien de ma vie : De qui aurai-je peur ?
\VS{2}Lorsque les méchants s'avancent contre moi pour dévorer ma chair, ce sont mes adversaires et mes ennemis qui chancellent et tombent.
\VS{3}Si toute une armée campait contre moi, mon cœur ne craindrait point ; si une guerre s'élevait contre moi, je serai plein de confiance.
\VS{4}Je demande une chose à Yahweh, que je désire ardemment : C'est d'habiter dans la maison de Yahweh tous les jours de ma vie, pour contempler la beauté de Yahweh et pour admirer son temple.
\VS{5}Car il me cachera dans son tabernacle au jour du malheur, il me tiendra caché sous l'abri de sa tente; il m'élèvera sur un rocher.
\VS{6}Même maintenant ma tête s'élève par-dessus mes ennemis qui m'entourent ; et j'offrirai des sacrifices dans sa tente, au son de la trompette; je chanterai et célèbrerai Yahweh.
\VS{7}Yahweh ! Ecoute ma voix, je t'invoque : Aie pitié de moi et exauce-moi !
\VS{8}Mon cœur dit de ta part : Cherche ma face ! Je chercherai ta face, ô Yahweh !
\VS{9}Ne me cache point ta face, ne rejette point avec colère ton serviteur ! Tu es mon secours, ne me laisse pas, ne m'abandonne pas, Dieu de mon salut !
\VS{10}Car mon père et ma mère m'abandonnent, mais Yahweh me recueillera\FTNT{Es. 49:15.}.
\VS{11}Yahweh, enseigne-moi ta voie, et conduis-moi dans le sentier de la droiture, à cause de mes ennemis\FTNT{Ps. 5:9 ; Ps. 25:4-5.}.
\VS{12}Ne me livre pas au désir de mes adversaires, car s'élèvent contre moi de faux témoins et des gens qui ne respirent que la violence.
\VS{13}Oh ! Si je n'étais pas sûr de voir la bonté de Yahweh sur la terre des vivants…
\VS{14}Espère en Yahweh ! Fortifie-toi et que ton cœur s'affermisse\FTNT{Es. 33:2 ; Ps. 31:25.} ! Espère en Yahweh !
\Chap{28}
\TextTitle{Louange à Yahweh, le Rocher de son peuple}
\VerseOne{}Psaume de David. Je crie à toi, ô Yahweh ! Mon rocher! Ne te rends point sourd envers moi, de peur que si tu ne me réponds pas, je ne sois semblable à ceux qui descendent dans la fosse\FTNT{Ps. 4:2 ; Ps. 143:7. 
Voir commentaire en Es. 8:13-17.}.
\VS{2}Ecoute la voix de mes supplications, lorsque je crie à toi, quand j'élève mes mains vers ton saint sanctuaire.
\VS{3}Ne m'emporte pas avec les méchants ni avec les ouvriers d'iniquité, qui parlent de paix avec leur prochain pendant que la malice est dans leur cœur\FTNT{Jé. 9:8 ; Ps. 26:9.}.
\VS{4}Traite-les selon leurs œuvres et selon la malice de leurs actions, traite-les selon l'ouvrage de leurs mains, rends-leur ce qu'ils ont mérité\FTNT{2 Ti. 4:14.}.
\VS{5}Parce qu'ils ne prennent point garde aux œuvres de Yahweh, à l'œuvre de ses mains. Qu'il les renverse et ne les édifie point !
\VS{6}Béni soit Yahweh ! Car il exauce la voix de mes supplications.
\VS{7}Yahweh est ma force et mon bouclier ; mon cœur se confie en lui, et je suis secouru ; mon cœur se réjouit, c'est pourquoi je le loue par mes chants.
\VS{8}Yahweh est la force de son peuple, il est le refuge des délivrances de son oint.
\VS{9}Sauve ton peuple et bénis ton héritage ! Nourris-les et élève-les éternellement.
\Chap{29}
\TextTitle{La suprématie de Dieu}
\VerseOne{}Psaume de David. Fils de Dieu, rendez à Yahweh, rendez à Yahweh la gloire et la force\FTNT{Ps. 96:7-8.} !
\VS{2}Rendez à Yahweh la gloire due à son Nom ! Prosternez-vous devant Yahweh avec des ornements sacrés !
\VS{3}La voix de Yahweh est sur les eaux, le Dieu de gloire fait tonner ; Yahweh est sur les grandes eaux.
\VS{4}La voix de Yahweh est forte, la voix de Yahweh est majestueuse.
\VS{5}La voix de Yahweh brise les cèdres, Yahweh brise les cèdres du Liban,
\VS{6}il les fait sauter comme un veau, le Liban et le Sirion comme de jeunes buffles.
\VS{7}La voix de Yahweh fait jaillir des flammes de feu.
\VS{8}La voix de Yahweh fait trembler le désert; Yahweh fait trembler le désert de Kadès.
\VS{9}La voix de Yahweh fait naître les biches, et dépouille les forêts. Dans son palais tout s'écrie : Gloire !
\VS{10}Yahweh était assis lors du déluge ; Yahweh est assis comme roi éternellement\FTNT{Ps. 146:10.}.
\VS{11}Yahweh donne de la force à son peuple ; Yahweh bénit son peuple en paix.
\Chap{30}
\TextTitle{De la délivrance découle la louange}
\VerseOne{}Psaume. Cantique pour la dédicace de la maison de David.
\VS{2}Yahweh, je t'exalte parce que tu m'as relevé, tu n'as pas voulu que mes ennemis se réjouissent à mon sujet.
\VS{3}Yahweh, mon Dieu ! J'ai crié à toi, et tu m'as guéri.
\VS{4}Yahweh ! Tu as fait remonter mon âme du scheol, tu m'as rendu la vie, afin que je ne descende point dans la fosse.
\VS{5}Chantez à Yahweh, vous ses bien-aimés, et célébrez la mémoire de sa sainteté\FTNT{Ps. 97:12.} !
\VS{6}Car sa colère dure un instant, mais sa grâce toute la vie. Le soir arrivent les pleurs, et le matin les cris de louange.
\VS{7}Dans ma sécurité, je disais : Je ne serai jamais ébranlé\FTNT{Ps. 10:6.} !
\VS{8}Yahweh ! Par ta faveur tu avais affermi ma montagne… Tu cachas ta face, et je fus terrifié\FTNT{Ps. 13:2 ; Ps. 88:15 ; Ps. 102:3 ; Ps. 143:7.}.
\VS{9}Yahweh, j'ai crié à toi, j'ai présenté ma supplication à Yahweh :
\VS{10}Que gagnes-tu à verser mon sang si je descends dans la fosse ? La poussière te célébrera-t-elle ? Racontera-t-elle ta fidélité\FTNT{Es. 38:18.} ?
\VS{11}Yahweh, écoute, et aie pitié de moi ! Yahweh, secours-moi !
\VS{12}Tu as changé mon deuil en allégresse, tu as détaché mon sac, et tu m'as ceint de joie,
\VS{13}afin que ma langue te loue\FTNT{Ps. 57:10.} et ne se taise point. Yahweh, mon Dieu ! Je te célébrerai toujours.
\Chap{31}
\TextTitle{Recherche de la protection divine}
\VerseOne{}Psaume de David. Au chef des chantres.
\VS{2}Yahweh ! Tu es mon refuge : Que je ne sois jamais confus ! Délivre-moi par ta justice\FTNT{Ps. 25:2-20 ; Ps. 71:1-2.} !
\VS{3}Incline ton oreille vers moi, hâte-toi de me délivrer ! Sois pour moi un rocher protecteur, une forteresse, afin que je puisse m'y sauver !
\VS{4}Car tu es mon rocher, ma forteresse ; tu me dirigeras et tu me donneras du repos, à cause de ton Nom.
\VS{5}Tire-moi du filet qu'ils m'ont tendu en secret, car tu es ma vigueur.
\VS{6}Je remets mon esprit entre tes mains\FTNT{Lu. 23:46.} ; tu me rachèteras, Yahweh, Dieu de vérité !
\VS{7}Je hais ceux qui s'adonnent aux vanités trompeuses, et je me confie en Yahweh.
\VS{8}Je serai par ta bonté dans l'allégresse et dans la joie ; car tu vois mon affliction, tu sais les angoisses de mon âme,
\VS{9}tu ne m'as pas livré entre les mains de l'ennemi, mais tu feras tenir mes pieds au large.
\VS{10}Yahweh, aie pitié de moi! Car je suis dans la détresse. Mes yeux, mon âme et mon corps dépérissent de chagrin\FTNT{Ps. 6:8 ; Ps. 88:10.}.
\VS{11}Ma vie se consume dans la douleur, et mes années dans les soupirs ; ma force chancelle à cause de mon iniquité, et mes os sont consumés.
\VS{12}J'ai été un objet d'opprobre à cause de tous mes adversaires, de grand opprobre pour mes voisins, et de terreur pour ceux qui me connaissent ; ceux qui me voient dehors s'enfuient loin de moi\FTNT{Ps. 38:12 ; Job. 19:13-14.}.
\VS{13}Je suis oublié des cœurs comme un mort, je suis comme un vase détruit.
\VS{14}J'entends les calomnies de plusieurs, la crainte m'environne, quand ils se concertent unis contre moi : Ils projettent de m'ôter la vie\FTNT{Jé. 20:10.}.
\VS{15}Toutefois, je me confie en toi, ô Yahweh ! Je dis : Tu es mon Dieu !
\VS{16}Ma destinée est entre tes mains ; délivre-moi de la main de mes ennemis et de ceux qui me poursuivent !
\VS{17}Fais luire ta face sur ton serviteur\FTNT{Ps. 4:7 ; Ps. 67:2.}, délivre-moi par ta bonté !
\VS{18}Yahweh, que je ne sois point confus puisque je t'ai invoqué. Que les méchants soient confus, qu'ils soient couchés dans le scheol !
\VS{19}Que les lèvres menteuses soient muettes, elles profèrent des paroles dures contre le juste, avec orgueil et avec mépris!
\VS{20}Que ta bonté est grande\FTNT{Ps. 36:6.} ! Toi qui la réserves pour ceux qui te craignent, tu leur fais un refuge à la vue des fils de l'homme !
\VS{21}Tu les caches sous l'abri de ta face, loin du complot des hommes, tu les caches sous ton abri contre les langues querelleuses.
\VS{22}Béni soit Yahweh ! Car il a rendu merveilleuse sa bonté envers moi, comme si j'avais été dans une ville retranchée.
\VS{23}Je disais dans ma précipitation : Je suis retranché loin de ton regard ! Mais tu as entendu la voix de mes supplications quand j'ai crié vers toi.
\VS{24}Aimez Yahweh, vous tous, ses bien-aimés! Yahweh garde les fidèles, et il punit sévèrement les orgueilleux.
\VS{25}Fortifiez-vous et que votre esprit s'affermisse, espérez en Yahweh\FTNT{Ps. 27:14.} !
\Chap{32}
\TextTitle{La puissance du pardon}
\VerseOne{}Cantique de David. Heureux celui à qui la transgression est pardonnée, et dont le péché est couvert !
\VS{2}Heureux l'homme à qui Yahweh n'impute point son iniquité\FTNT{Ro. 4:6-8.}, et dans l'esprit duquel il n'y a point de fraude !
\VS{3}Quand je me suis tu, mes os se sont consumés, je n'ai fait que gémir tout le jour;
\VS{4}parce que jour et nuit ta main s'appesantissait sur moi\FTNT{Ps. 38:3.}, ma vigueur s'est changée en une sécheresse d'été. Sélah.
\VS{5}Je t'ai fait connaître mon péché et je n'ai point caché mon iniquité; j'ai dit : J'avouerai mes transgressions à Yahweh\FTNT{Pr. 28:13 ; 1 Jn 1:9.} ! Et tu as porté la peine de mon péché. Sélah.
\VS{6}Que tout fidèle te prie au temps convenable\FTNT{Es. 55:6 ; So. 2:3 ; Ps. 69:14.}! Si de grandes eaux débordent, elles ne l'atteindront point.
\VS{7}Tu es mon asile, tu me gardes de la détresse, tu m'environnes de chants de triomphe à cause de ta délivrance. Sélah.
\VS{8}Je te rendrai intelligent, je t'enseignerai la voie dans laquelle tu dois marcher ; je te guiderai, mon oeil sera sur toi.
\VS{9}Ne soyez pas comme le cheval ni comme le mulet qui sont sans intelligence ; il faut brider leur bouche avec un mors et un frein, de peur qu'ils ne s'approchent de toi\FTNT{Ja. 3:3.}.
\VS{10}Beaucoup de douleurs atteindront le méchant\FTNT{Pr. 19:29.}, mais la bonté environne l'homme qui se confie en Yahweh.
\VS{11}Vous justes, réjouissez-vous en Yahweh, soyez dans l'allégresse ! Criez de joie, vous tous qui êtes droits de cœur\FTNT{Ps. 33:1 ; Ps. 64:11.} !
\Chap{33}
\TextTitle{Louanges à Yahweh, le Dieu fidèle}
\VerseOne{}Vous justes, poussez un cri de joie à cause de Yahweh\FTNT{Ps. 32:11 ; Ps. 97:12 ; Ps. 147:1.} ! Sa louange sied aux hommes droits.
\VS{2}Célébrez Yahweh avec la harpe, chantez-le sur le luth à dix cordes!
\VS{3}Chantez-lui un cantique nouveau\FTNT{Ps. 40:4 ; Ps. 96:1 ; Ps. 98:1 ; Ps. 144:9 ; Ap. 5:9 ; Ap. 14:3.} ! Jouez de vos instruments avec un cri de réjouissance !
\VS{4}Car la parole de Yahweh est droite, et toutes ses œuvres s'accomplissent avec fidélité ;
\VS{5}il aime la justice et la droiture\FTNT{Ps. 45:8 ; Hé. 1:9.} ; la terre est remplie de la bonté de Yahweh.
\VS{6}Les cieux ont été faits par la parole de Yahweh, et toute leur armée par le souffle de sa bouche\FTNT{Ge. 2:1-2.}.
\VS{7}Il amoncelle en un tas les eaux de la mer, il met les abîmes dans des réservoirs.
\VS{8}Que toute la terre craigne Yahweh ! Que tous les habitants du monde le redoutent !
\VS{9}Car il dit, et la chose arrive ; il ordonne, et la chose se présente.
\VS{10}Yahweh rompt le conseil des nations, il anéantit les desseins des peuples ;
\VS{11}mais le conseil de Yahweh subsiste à toujours, les desseins de son cœur subsistent d'âge en âge\FTNT{Pr. 19:21.}.
\VS{12}Heureuse la nation dont Yahweh est le Dieu\FTNT{Ps. 144:15.} et le peuple qu'il s'est choisi pour héritage !
\VS{13}Yahweh regarde des cieux, il voit tous les fils des hommes\FTNT{Job. 28:24.};
\VS{14}du lieu de sa demeure, il observe tous les habitants de la terre.
\VS{15}C'est lui qui forme également leur cœur et qui prend garde à toutes leurs actions.
\VS{16}Le roi n'est point sauvé par une grande armée, l'homme puissant n'échappe point par sa grande force;
\VS{17}le cheval est impuissant pour sauver, et ne délivre point par la grandeur de sa force\FTNT{Ps. 147:10.}.
\VS{18}Voici, l'œil de Yahweh est sur ceux qui le craignent\FTNT{Ps. 34:16 ; 1 Pi. 3:12.}, sur ceux qui s'attendent à sa bonté,
\VS{19}afin qu'il les délivre de la mort, et les fasse vivre durant la famine.
\VS{20}Notre âme espère en Yahweh; il est notre aide et notre bouclier.
\VS{21}Notre cœur se réjouit en lui, car nous avons confiance en son saint Nom.
\VS{22}Que ta bonté soit sur nous, ô Yahweh ! Nous nous attendons à toi!
\Chap{34}
\TextTitle{Yahweh libère les siens}
\VerseOne{}Psaume de David, lorsqu'il contrefît l'insensé en présence d'Abimélec, qui s'en alla, chassé par lui.
\VS{2}[Aleph.] Je bénirai Yahweh en tout temps, sa louange sera continuellement dans ma bouche.
\VS{3}[Beth.] Mon âme se glorifie en Yahweh ! Que les pauvres écoutent et se réjouissent.
\VS{4}[Guimel.] Glorifiez Yahweh avec moi ! Elevons son Nom tous ensemble !
\VS{5}[Daleth.] J'ai cherché Yahweh et il m'a répondu ; il m'a délivré de toutes mes frayeurs.
\VS{6}[He. Vav.] Quand on le regarde, on est illuminé, et la face n'est point confuse.
\VS{7}[Zayin.] Cet affligé a crié et Yahweh l'a exaucé, et l'a délivré de toutes ses détresses.
\VS{8}[Heth.] L'ange de Yahweh campe tout autour de ceux qui le craignent, et les équipe.
\VS{9}[Teth.] Goûtez et voyez combien Yahweh est bon ! Heureux l'homme qui se confie en lui !
\VS{10}[Yod.] Craignez Yahweh vous ses saints ! Car rien ne manque à ceux qui le craignent.
\VS{11}[Kaf.] Les lionceaux éprouvent la disette et la faim, mais ceux qui cherchent Yahweh ne manquent d'aucun bien.
\VS{12}[Lamed.] Venez, mes fils, écoutez-moi ! Je vous enseignerai la crainte de Yahweh.
\VS{13}[Mem.] Qui est l'homme qui prend plaisir à la vie, qui aime la prolonger pour jouir du bonheur ?
\VS{14}[Nun.] Garde ta langue du mal et tes lèvres des paroles trompeuses\FTNT{1 Pi. 3:10.} ;
\VS{15}[Samech.] détourne-toi du mal et fais-le bien ; cherche la paix et poursuis-la\FTNT{Hé. 12:14.}.
\VS{16}[Ayin.] Les yeux de Yahweh sont sur les justes et ses oreilles sont attentives à leur cri.
\VS{17}[Pe.] La face de Yahweh est contre ceux qui font le mal, pour retrancher de la terre leur mémoire\FTNT{Jé. 44:11 ; Lé. 17:10.}.
\VS{18}[Tsade.] Quand les justes crient, Yahweh les exauce et il les délivre de toutes leurs détresses.
\VS{19}[Qof.] Yahweh est près de ceux qui ont le cœur déchiré par la douleur, et il délivre ceux qui ont l'esprit abattu.
\VS{20}[Resh.] Le juste a des maux en grand nombre, mais Yahweh le délivre de tous\FTNT{2 Ti. 3:11.}.
\VS{21}[Shin.] Il garde tous ses os, aucun d'eux n'est brisé.
\VS{22}[Tav.] Le mauvais tue le méchant, et ceux qui haïssent le juste sont détruits.
\VS{23}[Pe.] Yahweh rachète l'âme de ses serviteurs, et aucun de ceux qui se confient en lui ne sera détruit.
\Chap{35}
\TextTitle{Prière au juste Juge}
\VerseOne{}Psaume de David. Yahweh, défends-moi contre mes adversaires, combats ceux qui me combattent !
\VS{2}Prends le petit et le grand bouclier, et lève-toi pour me secourir !
\VS{3}Brandis la lance et le javelot contre mes persécuteurs ! Dis à mon âme : Je suis ta délivrance !
\VS{4}Que ceux qui en veulent à ma vie soient honteux et confus\FTNT{Jé. 17:18 ; Ps. 40:15 ; Ps. 70:3.} ! Que ceux qui méditent ma perte reculent et rougissent !
\VS{5}Qu'ils soient comme la balle emportée par le vent\FTNT{Es. 29:5 ; Os. 13:3.}, et que l'Ange de Yahweh les chasse !
\VS{6}Que leur chemin soit ténébreux et glissant, et que l'Ange de Yahweh les poursuive.
\VS{7}Car sans cause ils m'ont tendu leur filet sur une fosse, sans cause ils l'ont creusée pour m'ôter la vie\FTNT{Jé. 18:20 ; Ps. 57:7 ; Ps. 140:5 ; Ps. 141:9.}.
\VS{8}Que la ruine les atteigne sans qu'ils le sachent, qu'ils soient capturés dans le filet qu'ils ont caché. Qu'ils y tombent et soient ravagés !
\VS{9}Mon âme aura de la joie en Yahweh, de l'allégresse en sa délivrance.
\VS{10}Tous mes os diront : Yahweh ! Qui est semblable à toi ? Qui délivre l'affligé de la main de celui qui est plus fort que lui ? L'affligé et le pauvre de celui qui le pille ?
\VS{11}De faux témoins s'élèvent contre moi : On m'interroge sur ce que j'ignore.
\VS{12}Ils me rendent le mal pour le bien, tâchant de m'ôter la vie\FTNT{Ps. 38:21 ; Ps. 109:5.}.
\VS{13}Mais moi, quand ils étaient malades, je me couvrais d'un sac, j'affligeais mon âme par le jeûne, je priais dans mon sein,
\VS{14}comme pour un ami, pour un frère, j'étais abattu, en pleurs, comme pour le deuil d'une mère.
\VS{15}Mais quand je chancelle, ils se réjouissent et s'assemblent, ils s'assemblent contre moi sans que je le sache pour me frapper, ils me déchirent pour que je sois silencieux ;
\VS{16}avec les hypocrites d’entre les railleurs qui suivent les bonnes tables, et ils ont grincé les dents contre moi.
\VS{17}Seigneur ! Jusqu'à quand le verras-tu ? Détourne mon âme de leurs tempêtes, mon unique des lionceaux.
\VS{18}Je te célébrerai dans la grande assemblée, je te louerai parmi un peuple nombreux\FTNT{Ps. 111:1.}.
\VS{19}Que ceux qui sont mes ennemis par leur mensonge ne se réjouissent point de moi, que ceux qui me haïssent sans cause ne m'insultent point par leurs regards\FTNT{Jn. 15:25.}.
\VS{20}Car ils ne parlent point de paix, mais ils préméditent des choses pleines de fraudes contre les gens tranquilles de la terre.
\VS{21}Ils ont ouvert leur bouche autant qu'ils ont pu contre moi, et ont dit : Ah ! Ah ! Nos yeux l'ont vu !
\VS{22}Yahweh ! Tu le vois : Ne te tais point\FTNT{Ps. 83:2.} ! Seigneur, ne t'éloigne point de moi !
\VS{23}Réveille-toi, réveille-toi pour me rendre justice\FTNT{Ps. 44:24.} ! Mon Dieu et mon Seigneur, défends ma cause !
\VS{24}Juge-moi selon ta justice, Yahweh mon Dieu ! et qu'ils ne se réjouissent point de moi !
\VS{25}Qu'ils ne disent point dans leur cœur : Ah ! Notre âme ! Et qu'ils ne disent point : Nous l'avons englouti !
\VS{26}Que ceux qui se réjouissent de mon mal soient honteux et rougissent tous ensemble ! Que ceux qui s'élèvent contre moi soient couverts de honte et de confusion !
\VS{27}Mais que ceux qui prennent plaisir à ma justice se réjouissent avec des chants de triomphe, qu'ils disent sans cesse : Grand est Yahweh qui désire la paix de son serviteur !
\VS{28}Alors ma langue publiera ta justice et ta louange tous les jours.
\Chap{36}
\TextTitle{Opposition : Les justes et les méchants}
\VerseOne{}Psaume de David, serviteur de Yahweh, donné au chef des chantres.
\VS{2}La transgression du méchant me dit, au dedans de mon cœur, qu'il n'y a point de crainte de Dieu devant ses yeux.
\VS{3}Car il se flatte à ses propres yeux pour consumer, pour assouvir sa haine.
\VS{4}Les paroles de sa bouche ne sont que méchanceté et tromperie, il cesse d'être sage et de faire le bien.
\VS{5}Il projette le malheur sur sa couche, il se tient sur un chemin qui n'est pas bon, il ne rejette pas le mal.
\VS{6}Yahweh ! Ta bonté atteint jusqu'aux cieux, ta fidélité jusqu'aux nues\FTNT{Ps. 57:11 ; Ps. 108:5.}.
\VS{7}Ta justice est comme les montagnes de Dieu, tes jugements sont un grand abîme. Yahweh ! Tu sauves les hommes et les bêtes.
\VS{8}Ô Dieu ! Combien est précieuse ta bonté ! Aussi les fils des hommes se retirent à l'ombre de tes ailes\FTNT{Ps. 17:8 ; Ps. 57:2.}.
\VS{9}Ils seront abondamment rassasiés de la graisse de ta maison et tu les abreuveras au fleuve de tes délices.
\VS{10}Car la source de la vie est auprès de toi, et par ta lumière nous voyons la lumière.
\VS{11}Etends ta bonté sur ceux qui te connaissent, et ta justice sur ceux qui ont le cœur droit !
\VS{12}Que le pied de l'orgueilleux ne s'avance point sur moi, et que la main des méchants ne m'ébranle point !
\VS{13}Là sont tombés les ouvriers d'iniquité ; ils sont renversés et ne peuvent se relever.
\Chap{37}
\TextTitle{Se confier en la justice de Yahweh}
\VerseOne{}Psaume de David. [Aleph.] Ne t'irrite pas contre les méchants, ne jalouse pas ceux qui s'adonnent à la perversité\FTNT{Pr. 23:17 ; Pr. 24:19.}.
\VS{2}Car ils seront soudainement retranchés comme le foin, et ils se faneront comme l'herbe verte.
\VS{3}[Beth.] Confie-toi en Yahweh, et fais ce qui est bon ; aie le pays pour demeure et la fidélité pour pâture.
\VS{4}Fais de Yahweh tes délices et il t'accordera ce que ton cœur désire.
\VS{5}[Guimel.] Recommande tes voies à Yahweh, confie-toi en lui et il agira\FTNT{Ps. 22:9 ; Ps. 55:23 ; Pr. 16:3.}.
\VS{6}Il manifestera ta justice comme la lumière et ton droit comme le soleil à son midi\FTNT{Pr. 4:18.}.
\VS{7}[Daleth.] Garde le silence devant Yahweh et tremble devant lui ; ne t'irrite point contre celui qui réussit dans ses voies, contre celui qui vient à bout de ses mauvais desseins.
\VS{8}[He.] Laisse la colère et abandonne la rage\FTNT{Ep. 4:26.} ; ne t'irrite pas pour faire le mal.
\VS{9}Car les méchants seront retranchés, mais ceux qui se confient en Yahweh hériteront la terre.
\VS{10}[Vav.] Encore un peu de temps et le méchant ne sera plus ; tu regardes le lieu où il était et il n'y est plus\FTNT{Job. 7:10 ; Job. 20:9.}.
\VS{11}Les pauvres prennent possession du pays et jouissent abondamment de la paix.
\VS{12}[Zayin.] Le méchant complote contre le juste et grince ses dents contre lui.
\VS{13}Le Seigneur se rit de lui, car il voit que son jour approche.
\VS{14}[Heth.] Les méchants tirent leur épée et bandent leur arc pour faire tomber le malheureux et le pauvre, pour massacrer ceux qui marchent dans la droiture\FTNT{Ps. 11:2.}.
\VS{15}Mais leur épée entre dans leur propre cœur, et leurs arcs se brisent.
\VS{16}[Teth.] Mieux vaut au juste le peu qu'il a, que l'abondance de beaucoup de méchants\FTNT{Pr. 15:16-17 ; Ec. 4:6.} ;
\VS{17}car les bras des méchants seront brisés, mais Yahweh soutient les justes.
\VS{18}[Yod.] Yahweh connaît les jours de ceux qui sont intègres, et leur héritage demeure à jamais.
\VS{19}Ils ne sont pas honteux au jour du malheur, mais ils sont rassasiés au jour de la famine.
\VS{20}[Kaf.] Mais les méchants périssent, et les ennemis de Yahweh, comme les beaux pâturages, s'évanouissent, ils s'évanouissent en fumée.
\VS{21}[Lamed.] Le méchant emprunte et ne rend point ; mais le juste a compassion et donne.
\VS{22}Car les bénis de Yahweh hériteront la terre, mais ceux qu'il a maudits seront retranchés.
\VS{23}[Mem.] Yahweh affermit les pas de l'homme, et il prend plaisir à ses voies.
\VS{24}S'il tombe, il ne sera pas entièrement abattu, car Yahweh le soutient de sa main.
\VS{25}[Nun.] J'ai été jeune et j'ai vieilli ; et je n'ai point vu le juste abandonné, ni sa postérité mendiant son pain.
\VS{26}Il est compatissant tout le temps, et il prête ; et sa postérité est bénie.
\VS{27}[Samech.] Retire-toi du mal et fais le bien ; et tu auras une demeure éternelle.
\VS{28}Car Yahweh aime ce qui est juste, et il n'abandonne point ses fidèles ; c'est pourquoi ils sont sous sa garde pour toujours, mais la postérité des méchants est retranchée.
\VS{29}[Ayin.] Les justes hériteront la terre et y habiteront à perpétuité.
\VS{30}[Pe.] La bouche du juste prononce la sagesse et sa langue déclare la justice.
\VS{31}La loi de son Dieu est dans son cœur\FTNT{Ps. 40:8-9.}, aucun de ses pas ne chancellera.
\VS{32}[Tsade.] Le méchant épie le juste et cherche à le faire mourir.
\VS{33}Yahweh ne l'abandonne point entre ses mains et ne le laisse point condamner quand on le juge.
\VS{34}[Qof.] Espère en Yahweh et garde sa voie, et il t'élèvera pour que tu hérites la terre ; tu verras les méchants retranchés.
\VS{35}[Resh.] J'ai vu le méchant dans toute sa puissance, il s'étendait comme un arbre verdoyant.
\VS{36}Il a passé, et voici, il n'est plus ; je le cherche et il ne se trouve plus.
\VS{37}[Shin.] Observe l'homme intègre et considère l'homme droit, car il y a une issue pour l'homme de paix.
\VS{38}Mais les rebelles seront tous détruits et ce qui sera resté des méchants sera retranché.
\VS{39}[Tav.] Mais la délivrance des justes viendra de Yahweh, il sera leur force au temps de la détresse.
\VS{40}Yahweh les secourt et les délivre ; il les délivre des méchants et les sauve, parce qu'ils se confient en lui.
\Chap{38}
\TextTitle{La tristesse du péché mène à la repentance}
\VerseOne{}Psaume de David. Pour souvenir.
\VS{2}Yahweh ! Ne me juge pas dans ta colère et ne me châtie pas dans ta fureur.
\VS{3}Car tes flèches m'ont atteint et ta main s'est appesantie sur moi.
\VS{4}Il n'y a rien de sain dans ma chair, à cause de ta colère, ni de paix dans mes os, à cause de mon péché.
\VS{5}Car mes iniquités s'élèvent au-dessus de ma tête, elles se sont appesanties comme un pesant fardeau, au-delà de mes forces\FTNT{Ps. 40:13.}.
\VS{6}Mes plaies ont une mauvaise odeur et sont purulentes à cause de ma folie.
\VS{7}Je suis courbé et abattu outre mesure ; je marche en pleurs tout le jour.
\VS{8}Car un mal brûlant remplit mes reins, et dans ma chair il n'y a rien de sain.
\VS{9}Je suis affaibli et brisé, je rougis le cœur troublé.
\VS{10}Seigneur, tout mon désir est devant toi, et mon soupir ne t'est point caché.
\VS{11}Mon cœur est agité çà et là, ma force m'abandonne, et la lumière de mes yeux n'est plus avec moi.
\VS{12}Ceux qui m'aiment, et même mes amis intimes, se tiennent loin de ma plaie, et mes proches se tiennent loin de moi\FTNT{Job. 19:13-14.}.
\VS{13}Ceux qui en veulent à ma vie me tendent des pièges ; ceux qui cherchent ma perte parlent de calamités et méditent des tromperies tous les jours.
\VS{14}Mais moi je suis comme un sourd, comme un muet qui n'ouvre point sa bouche.
\VS{15}Je suis, dis-je, comme un homme qui n'entend pas et qui n'a point de réplique dans sa bouche.
\VS{16}Car je m'attends à toi, ô Yahweh ! Tu me répondras, Seigneur mon Dieu !
\VS{17}Je dis : Il faut prendre garde qu'ils ne triomphent de moi ; quand mon pied chancelle, ils s'élèvent contre moi\FTNT{Ps. 94:18.} !
\VS{18}Car je suis près de tomber et ma douleur est continuellement devant moi.
\VS{19}Car je reconnais mon iniquité et je suis dans la crainte à cause de mon péché.
\VS{20}Cependant mes ennemis qui sont vivants se renforcent, et ceux qui me haïssent à tort se multiplient.
\VS{21}Ceux qui me rendent le mal pour le bien sont mes adversaires, parce que je recherche le bien\FTNT{Ps. 109:5 ; Jé. 18:20.}.
\VS{22}Ne m'abandonne pas Yahweh ! Mon Dieu, ne t'éloigne pas de moi !
\VS{23}Hâte-toi de venir à mon secours, Seigneur, tu es ma délivrance !
\Chap{39}
\TextTitle{La faiblesse de l'homme}
\VerseOne{}Psaume de David, donné au chef des chantres, à Jeduthun.
\VS{2}J'ai dit : Je prends garde à mes voies, de peur de pécher par ma langue ; je mettrai un frein à ma bouche tant que le méchant sera devant moi.
\VS{3}Je suis resté muet, dans le silence ; je me suis tu, quoique malheureux ; et ma douleur n'était pas moins vive.
\VS{4}Mon cœur brûlait au-dedans de moi, un feu intérieur me consumait, et la parole est venue sur ma langue.
\VS{5}Yahweh ! Dis-moi quel est le terme de ma vie et quelle est la mesure de mes jours\FTNT{Ps. 119:84.} ; que je sache combien je suis fragile.
\VS{6}Voici, tu as réduit mes jours à la largeur de ma main, et ma vie est comme un rien devant toi. Oui, tout homme debout n'est qu'un souffle\FTNT{Ja. 4:14.}. Sélah.
\VS{7}Oui, l'homme se promène comme une ombre, il s'agite inutilement ; il amasse des biens et il ne sait pas qui les recueillera.
\VS{8}Maintenant que puis-je espérer, Seigneur ? Mon espérance est en toi.
\VS{9}Délivre-moi de toutes mes transgressions ! Ne permets pas l'opprobre des insensés.
\VS{10}Je me suis tu et je n'ai point ouvert ma bouche, parce que c'est toi qui agis.
\VS{11}Détourne de moi tes coups ! Je suis consumé par les attaques de ta main.
\VS{12}Aussitôt que tu châties quelqu'un, en le punissant à cause de son iniquité, tu détruis comme la teigne ce qu'il a de plus cher. Oui, tout homme est une vapeur. Sélah.
\VS{13}Yahweh, écoute ma prière et prête l'oreille à mon cri ! Ne sois point sourd à mes larmes ! Car je suis un voyageur et un étranger chez toi, comme tous mes pères\FTNT{Lé. 25:23 ; Ps. 119:19 ; 1 Pi. 2:11 ; Hé. 11:13.}.
\VS{14}Détourne ton regard de moi, afin que je reprenne mes forces, avant que je m'en aille et que je ne sois plus.
\Chap{40}
\TextTitle{Un cantique nouveau à Yahweh}
\VerseOne{}Psaume de David, donné au chef des chantres.
\VS{2}J'ai attendu patiemment Yahweh, et il s'est tourné vers moi et a entendu mon cri.
\VS{3}Il m'a retiré de la fosse de destruction, du fond de la boue ; il a mis mes pieds sur un roc et a assuré mes pas.
\VS{4}Il a mis dans ma bouche un cantique nouveau, qui est la louange de notre Dieu ; plusieurs verront cela et ils craindront et se confieront en Yahweh.
\VS{5}Heureux l'homme qui place sa confiance en Yahweh et qui ne se tourne pas vers les orgueilleux et les menteurs !
\VS{6}Yahweh, mon Dieu ! Tu as multiplié tes merveilles et tes desseins envers nous ; nul n'est comparable à toi ; je voudrais les annoncer et les déclarer, mais leur nombre est trop grand pour que je les raconte.
\VS{7}Tu ne désires ni sacrifice ni offrande. Tu m'as percé les oreilles ; tu ne demandes ni holocauste ni victime expiatoire pour le péché\FTNT{Hé. 10:5.}.
\VS{8}Alors je dis : Voici, je viens avec le rouleau du livre écrit pour moi.
\VS{9}Mon Dieu, je prends plaisir à faire ta volonté, et ta loi est au fond de mes entrailles\FTNT{Ps. 37:31 ; Es. 51:7.}.
\VS{10}J'annonce ta justice dans la grande assemblée ; voilà, je ne ferme pas mes lèvres, Yahweh, tu le sais !
\VS{11}Je ne cache pas ta justice, qui est dans mon cœur ; je déclare ta fidélité et ta délivrance ; je ne cache pas ta bonté ni ta vérité dans la grande assemblée.
\VS{12}Toi, Yahweh ! Ne m'épargne point tes compassions, que ta bonté et ta vérité me gardent continuellement.
\VS{13}Car des maux sans nombre m'environnent ; mes iniquités m'atteignent et je ne supporte pas leur vue ; elles surpassent en nombre les cheveux de ma tête, et mon cœur m'abandonne.
\VS{14}Yahweh, veuille me délivrer ! Yahweh, hâte-toi de venir à mon secours !
\VS{15}Que tous ensemble ils soient honteux et confus, ceux qui cherchent mon âme pour la perdre ; et que ceux qui prennent plaisir à mon malheur retournent en arrière et rougissent.
\VS{16}Que ceux qui disent de moi : Ah ! Ah ! Soient consumés, en récompense de la honte qu'ils m'ont faite.
\VS{17}Que tous ceux qui te cherchent soient dans l'allégresse et se réjouissent en toi\FTNT{Ps. 70:5.} ! Que ceux qui aiment ta délivrance disent continuellement : Grand est Yahweh !
\VS{18}Moi, je suis affligé et misérable, mais le Seigneur prend soin de moi. Tu es mon secours et mon libérateur : Mon Dieu ne tarde point\FTNT{Ps. 70:6.} !
\Chap{41}
\TextTitle{Secours de Yahweh dans le malheur}
\VerseOne{}Psaume de David, donné au chef des chantres.
\VS{2}Heureux celui qui s'intéresse au pauvre ! Yahweh le délivrera au jour du malheur ;
\VS{3}Yahweh le garde et lui conserve la vie. Il est heureux sur la terre, et tu ne le livres pas au bon plaisir de ses ennemis.
\VS{4}Yahweh le soutient sur son lit de douleur ; tu le soulages dans toutes ses maladies.
\VS{5}Je dis : Yahweh ! Aie pitié de moi, guéris mon âme, car j'ai péché contre toi.
\VS{6}Mes ennemis disent du mal de moi : Quand mourra-t-il ? Et quand périra son nom ?
\VS{7}Si quelqu'un vient me voir, il dit des mensonges, il recueille de mauvais desseins\FTNT{Ps. 5:10 ; Ps. 10:7 ; Ps. 12:3.}, il s'en va et il parle au-dehors.
\VS{8}Tous ceux qui m'ont en haine murmurent sourdement ensemble contre moi, et machinent du mal contre moi.
\VS{9}Quelque action criminelle\FTNT{Le mot « criminelle » donne en hébreu « Belial »} pèse sur lui ; le voilà couché, disent–ils, il ne se relèvera plus !
\VS{10}Même celui qui était en paix avec moi, qui avait ma confiance et qui mangeait mon pain, a levé le talon contre moi\FTNT{Il est question ici de la trahison du Messie par Judas (Jn. 13:18-19).}.
\VS{11}Mais toi, ô Yahweh ! Aie pitié de moi et relève-moi ! Et je leur rendrai ce qui leur est dû.
\VS{12}Je connaîtrai que tu prends plaisir en moi, si mon ennemi ne triomphe pas de moi.
\VS{13}Pour moi, tu m'as soutenu à cause de mon intégrité, et tu m'as établi pour toujours en ta présence.
\VS{14}Béni soit Yahweh, le Dieu d'Israël, d'éternité en éternité. Amen ! Amen !
\Chap{42}
\TextTitle{Avoir soif de Dieu}
\VerseOne{}Cantique des fils de Koré, donné au chef des chantres.
\VS{2}Comme une biche soupire après des courants d'eau, ainsi mon âme soupire ardemment après toi, ô Dieu !
\VS{3}Mon âme a soif de Dieu, du Dieu vivant\FTNT{Ps. 63:2 ; Ps. 84:3.} : Ô quand entrerai-je et me présenterai-je devant la face de Dieu ?
\VS{4}Mes larmes sont ma nourriture jour et nuit, quand on me dit chaque jour : Où est ton Dieu\FTNT{Ps. 80:6 ; Ps. 115:2.} ?
\VS{5}Je rappelais ces choses dans mon souvenir, en répandant mon âme au-dedans de moi, savoir que je marchais dans la foule, et que je m'en allais tout doucement en leur compagnie, avec une voix de triomphe et de louange, jusqu'à la maison de Dieu, et qu'une grande multitude de gens sautait alors de joie.
\VS{6}Mon âme, pourquoi t'abats-tu et murmures-tu au-dedans de moi ? Attends-toi à Dieu, car je le célébrerai encore ; sa face est la délivrance même.
\VS{7}Mon Dieu ! Mon âme est abattue au-dedans de moi, aussi je me souviens de toi depuis la terre du Jourdain, depuis l'Hermon, et depuis la montagne de Mitsear.
\VS{8}Un flot appelle un autre flot au bruit de tes ondées ; toutes tes vagues et tes flots passent sur moi.
\VS{9}Toutefois, Yahweh enverra sa bonté compatissante de jour, et de nuit son cantique sera avec moi, et ma prière sera au Dieu qui est ma vie.
\VS{10}Je dis à Dieu, mon rocher : Pourquoi m'oublies-tu ? Pourquoi marcherai-je dans la tristesse à cause de l'oppression de l'ennemi ?
\VS{11}Comme avec une épée dans mes os, mes ennemis m'outragent, tandis qu'ils me disent chaque jour : Où est ton Dieu ?
\VS{12}Mon âme, pourquoi t'abats-tu et pourquoi murmures-tu au-dedans de moi ? Attends-toi à Dieu, car je le célébrerai encore, il est ma délivrance et mon Dieu.
\Chap{43}
\TextTitle{Espérer dans la délivrance de Dieu}
\VerseOne{}Fais-moi justice, ô Dieu ! Et défends ma cause contre une nation infidèle\FTNT{Ps. 26:1 ; Ps. 35:1.} ! Délivre-moi de l'homme trompeur et pervers.
\VS{2}Toi, mon Dieu protecteur, pourquoi me repousses-tu ? Pourquoi marcherai-je dans la tristesse à cause de l'oppression de l'ennemi ?
\VS{3}Envoie ta lumière et ta vérité, afin qu'elles me conduisent et m'introduisent dans ta sainte montagne, et dans tes demeures.
\VS{4}Alors je viendrai à l'autel de Dieu, au Dieu de ma joie et mon allégresse, et je te célébrerai sur la harpe, ô Dieu ! Mon Dieu !
\VS{5}Mon âme, pourquoi t'abats-tu et pourquoi murmures-tu au-dedans de moi ? Attends-toi à Dieu, car je le célébrerai encore ; il est ma délivrance et mon Dieu\FTNT{Ps. 42:6.}.
\Chap{44}
\TextTitle{Prière des affligés}
\VerseOne{}Cantique des fils de Koré, donné au chef des chantres.
\VS{2}Ô Dieu ! Nous avons entendu de nos oreilles et nos pères nous ont raconté les exploits que tu as faits de leur temps, aux jours d'autrefois\FTNT{Jg. 6:13 ; Ps. 77:12.}.
\VS{3}Tu as de ta main chassé les nations, et tu as affermi nos pères, tu as affligé les peuples, et tu as fait prospérer nos pères.
\VS{4}Car ce n'est point par leur épée qu'ils ont conquis le pays, et ce n'est pas leur bras qui les a délivrés, mais c'est ta droite, c'est ton bras, c'est la lumière de ta face, parce que tu les aimais.
\VS{5}Ô Dieu ! Tu es mon Roi : Ordonne la délivrance de Jacob !
\VS{6}Avec toi nous battrons nos adversaires, par ton Nom nous foulerons ceux qui s'élèvent contre nous.
\VS{7}Car je ne me confie point en mon arc, et ce n'est pas mon épée qui me délivrera.
\VS{8}Mais tu nous délivreras de nos adversaires, et tu rendras confus ceux qui nous haïssent.
\VS{9}Nous nous glorifierons en Dieu chaque jour et nous célébrerons à jamais ton Nom. Sélah.
\VS{10}Mais tu nous rejettes, tu nous confonds, et tu ne sors plus avec nos armées.
\VS{11}Tu nous fais reculer devant l'adversaire, et ceux qui nous haïssent enlèvent nos dépouilles.
\VS{12}Tu nous livres comme des brebis destinées à être dévorées, et tu nous as dispersés parmi les nations.
\VS{13}Tu as vendu ton peuple pour rien, et tu ne l'estimes pas d'une grande valeur\FTNT{Es. 52:3 ; Jé. 15:13.}.
\VS{14}Tu nous as mis en opprobre chez nos voisins, en dérision, et en sujet de moquerie auprès de ceux qui habitent autour de nous\FTNT{Jé. 24:9 ; Ps. 79:4.}.
\VS{15}Tu fais de nous un objet de sarcasmes parmi les nations, et de hochement de tête parmi les peuples.
\VS{16}Ma confusion est tout le jour devant moi, et la honte couvre ma face,
\VS{17}à cause des discours de celui qui nous fait des reproches et qui nous injurie, et à cause de l'ennemi et du vindicatif.
\VS{18}Tout cela nous est arrivé, et cependant nous ne t'avons point oublié, et nous n'avons point violé ton alliance.
\VS{19}Notre cœur ne s'est point détourné, nos pas ne se sont point éloignés de tes sentiers,
\VS{20}pour que tu nous écrases dans le lieu du serpent, et que tu nous couvres de l'ombre de la mort\FTNT{Ps. 23:4.}.
\VS{21}Si nous avions oublié le Nom de notre Dieu et étendu nos mains vers un dieu étranger,
\VS{22}Dieu ne le saurait-il pas, lui qui connaît les secrets du cœur ?
\VS{23}Mais nous sommes tous les jours mis à mort pour l'amour de toi, nous sommes regardés comme des brebis destinées à la boucherie\FTNT{Es. 53:7.}.
\VS{24}Lève-toi, pourquoi dors-tu Seigneur ? Réveille-toi ! Ne nous rejette point à jamais !
\VS{25}Pourquoi caches-tu ta face, pourquoi oublies-tu notre affliction et notre oppression ?
\VS{26}Car notre âme est abattue dans la poussière et notre ventre est attaché à la terre.
\VS{27}Lève-toi pour nous secourir ! Délivre-nous à cause de ta bonté.
\Chap{45}
\TextTitle{La beauté du Roi}
\VerseOne{}Cantique des fils de Koré, qui est un chant nuptial donné au chef des chantres pour le chanter sur Shoshannim.
\VS{2}Des paroles agréables bouillonnent dans mon cœur, et j'ai dit : Mon œuvre est pour le roi ! Ma langue sera comme la plume d'un habile écrivain !\FTNT{Ca. 5:13 ; Ca. 5:16.}
\VS{3}Tu es le plus beau des fils de l'homme, la grâce est répandue sur tes lèvres : C'est pourquoi Dieu t'a béni éternellement.
\VS{4}Ô héros, ceins ton épée sur ta cuisse, ta majesté et ta magnificence,
\VS{5}et prospère dans ta magnificence. Sois porté sur la parole de vérité, de douceur, et de justice, et ta droite versera des choses terribles !
\VS{6}Tes flèches sont aiguës, les peuples tomberont sous toi, elles perceront le cœur des ennemis du roi.
\VS{7}Ô Dieu, ton trône est à toujours et à perpétuité ! Le sceptre de ton règne est un sceptre d'équité.
\VS{8}Tu aimes la justice et tu hais la méchanceté : C'est pourquoi, ô Dieu, ton Dieu t'a oint d'une huile de joie par privilège sur tes compagnons\FTNT{Hé. 1:8-9.}.
\VS{9}Tous tes vêtements sont parfumés de myrrhe, d'aloès et de casse. Dans les palais d'ivoire les instruments à cordes te réjouissent.
\VS{10}Des filles de rois sont parmi tes bien-aimées ; la reine est à ta droite, parée d'or d'Ophir.
\VS{11}Ecoute, jeune fille, vois et prête l'oreille ; oublie ton peuple et la maison de ton père.
\VS{12}Le roi porte ses désirs sur ta beauté ; puisqu'il est ton Seigneur, prosterne-toi devant lui.
\VS{13}La fille de Tyr et les plus riches des peuples te supplieront avec des présents.
\VS{14}La fille du roi est intérieurement pleine de gloire. Elle porte un vêtement tissé d'or.
\VS{15}Elle sera présentée au roi en vêtements de broderie, et les filles qui viennent après elle, et qui sont ses compagnes, seront amenées vers toi.
\VS{16}Elles te seront présentées avec réjouissance et allégresse, et elles entreront au palais du roi.
\VS{17}Tes fils seront au lieu de tes pères, tu les établiras pour princes sur toute la terre.
\VS{18}Je rendrai ton Nom mémorable dans tous les âges, et à cause de cela les peuples te célébreront pour toujours et à perpétuité\FTNT{Ps. 67:3-5.}.
\Chap{46}
\TextTitle{L'assurance du peuple de Dieu}
\VerseOne{}Cantique des fils de Koré, donné au chef des chantres pour le chanter sur Alamoth. Cantique.
\VS{2}Dieu est notre retraite, notre force, et notre secours qui ne manque jamais dans les détresses\FTNT{Ps. 9:10.}.
\VS{3}C'est pourquoi nous ne craindrons point quand la terre est bouleversée et que les montagnes chancellent au cœur des mers\FTNT{Es. 54:10.},
\VS{4}quand ses eaux mugissent et écument, se soulèvent jusqu'à faire trembler les montagnes. Sélah.
\VS{5}Il est un fleuve dont les courants réjouissent la cité de Dieu, le lieu saint des demeures du Très-Haut\FTNT{Ez. 47:1-2 ; Za. 14:8-9 ; Jn. 7:38 ; Ap. 22:1-2.}.
\VS{6}Dieu est au milieu d'elle : Elle n'est point ébranlée. Dieu la secourt dès le point du jour\FTNT{So. 3:16-17.}.
\VS{7}Les nations murmurent, les royaumes s'ébranlent ; il a fait entendre sa voix et la terre se fond.
\VS{8}Yahweh des armées est avec nous, le Dieu de Jacob est pour nous une haute retraite. Sélah.
\VS{9}Venez, contemplez les œuvres de Yahweh et voyez quels ravages il a faits sur la terre.
\VS{10}Il a fait cesser les guerres jusqu'au bout de la terre, il a brisé l'arc et rompu la lance, il a consumé par le feu les chars de guerre\FTNT{Es. 2:4.}.
\VS{11}Arrêtez, et sachez que je suis Dieu : Je suis élevé parmi les nations, je suis élevé sur toute la terre.
\VS{12}Yahweh des armées est avec nous, le Dieu de Jacob est pour nous une haute retraite. Sélah.
\Chap{47}
\TextTitle{Yahweh, le Dieu élevé}
\VerseOne{}Psaume des fils de Koré, donné au chef des chantres.
\VS{2}Peuples battez tous des mains ! Poussez vers Dieu des cris de joie avec une voix de triomphe !
\VS{3}Car Yahweh, le Très-Haut, est terrible. Il est un grand Roi sur toute la terre.
\VS{4}Il nous assujettit des peuples et des nations sous nos pieds.
\VS{5}Il nous choisit notre héritage, la gloire de Jacob qu'il aime. Sélah.
\VS{6}Dieu est monté avec un cri de réjouissance, Yahweh monte au son du shofar.
\VS{7}Chantez à Dieu, chantez ! Chantez à notre Roi, chantez !
\VS{8}Car Dieu est le Roi de toute la terre : Chantez un cantique !
\VS{9}Dieu règne sur les nations, Dieu est assis sur son saint trône.
\VS{10}Les princes des peuples se rassemblent vers le peuple du Dieu d'Abraham, car les boucliers de la terre sont à Dieu : Il est puissamment élevé.
\Chap{48}
\TextTitle{Sion, splendeur du grand Roi}
\VerseOne{}Cantique. Psaume des fils de Koré.
\VS{2}Yahweh est grand, il est l'objet de toutes les louanges dans la ville de notre Dieu, sur sa montagne sainte.
\VS{3}Belle est la colline, joie de toute la terre, la montagne de Sion, le côté nord, c'est la ville du grand Roi.
\VS{4}Dieu est connu dans ses palais pour une haute retraite.
\VS{5}Ils l'ont vue, et aussitôt ils ont été émerveillés ; ils ont été troublés et se sont enfuis à la hâte.
\VS{6}Ils ont regardé tout stupéfaits, ils ont eu peur et ont pris la fuite.
\VS{7}Là un tremblement les a saisis, une douleur comme celle de l'enfantement\FTNT{Es. 13:8.}.
\VS{8}Ils ont été chassés comme par le vent d'orient qui brise les navires de Tarsis.
\VS{9}Comme nous l'avions entendu, ainsi l'avons-nous vu dans la ville de Yahweh des armées, dans la ville de notre Dieu : Dieu l'établira à toujours. Sélah.
\VS{10}Ô Dieu ! Nous pensons à ta bonté au milieu de ton temple.
\VS{11}Ô Dieu ! Comme ton Nom, ta louange retentit jusqu'aux extrémités de la terre ; ta droite est pleine de justice.
\VS{12}La montagne de Sion se réjouit, et les filles de Juda sont dans la joie, à cause de tes jugements.
\VS{13}Entourez Sion, faites-en le tour, comptez ses tours.
\VS{14}Observez son rempart, examinez ses palais pour le raconter à la génération future.
\VS{15}Car ce Dieu-là est notre Dieu éternellement et à jamais ; il nous accompagnera jusqu'à la mort.
\Chap{49}
\TextTitle{Vanité des richesses terrestres}
\VerseOne{}Psaume des fils de Koré, au chef des chantres.
\VS{2}Vous tous peuples, entendez ceci, vous habitants du monde, prêtez l'oreille,
\VS{3}petits et grands, riches et pauvres !
\VS{4}Ma bouche prononcera des discours pleins de sagesse, et les pensées de mon cœur sont pleines de sens.
\VS{5}Je prête l'oreille aux sentences qui me sont inspirées, je poserai mes questions au son de la harpe.
\VS{6}Pourquoi craindrai-je au jour du malheur, quand l'iniquité de mes adversaires m'entoure ?
\VS{7}Ils mettent leur confiance dans leurs biens et se glorifient de l'abondance de leurs richesses.
\VS{8}Ils ne peuvent se racheter l'un l'autre ni donner à Dieu le prix du rachat\FTNT{Mt. 16:26 ; Mc. 8:36-37 ; Lu. 12:15-21.}.
\VS{9}Car le rachat de leur âme est trop considérable, et il ne se fera jamais ;
\VS{10}ils ne vivront pas toujours et n'éviteront pas la vue de la fosse.
\VS{11}Car on voit que les sages meurent, l'insensé et le stupide périssent également, et ils laissent à d'autres leurs biens\FTNT{Ec. 2:21 ; Ec. 6:2.}.
\VS{12}Leur intention est que leurs maisons durent éternellement, et que leurs habitations demeurent d'âge en âge, ils ont donné leurs noms à leurs terres.
\VS{13}Mais l'homme qui est en honneur n'a point de durée, il est semblable aux bêtes que l'on égorge.
\VS{14}Tel est leur chemin, leur folie, et ceux qui les suivent se plaisent à leurs discours. Sélah.
\VS{15}Ils seront mis dans le scheol comme des brebis, la mort en fait sa pâture, et au matin les hommes droits les foulent aux pieds, leur beau rocher s'use, le scheol est leur résidence\FTNT{Job. 24:19.}.
\VS{16}Mais Dieu rachètera mon âme du pouvoir du scheol, quand il m'enlèvera de sa captivité\FTNT{Ps. 68:19 ; Ep. 4:8-9.}. Sélah.
\VS{17}Ne crains point quand tu verras quelqu'un s'enrichir et quand les trésors de sa maison se multiplient.
\VS{18}Car lorsqu'il mourra, il n'emportera rien, ses trésors ne descendront point après lui\FTNT{Job. 27:16-19 ; 1 Ti. 6:7.}.
\VS{19}Il aura beau s'estimer heureux pendant sa vie, on aura beau te louer des jouissances que tu te donnes,
\VS{20}tu iras néanmoins au séjour de tes pères, qui jamais ne reverront la lumière.
\VS{21}L'homme qui est en honneur, qui n'a pas d'intelligence, est semblable aux bêtes que l'on égorge.
\Chap{50}
\TextTitle{Yahweh, le juste Juge}
\VerseOne{}Psaume d'Asaph. Le Dieu puissant, Dieu, Yahweh a parlé et il a appelé toute la terre, depuis le soleil levant jusqu'au soleil couchant.
\VS{2}De Sion, Dieu a fait luire sa splendeur qui est d'une beauté parfaite,
\VS{3}notre Dieu viendra, il ne se taira point : Il y aura devant lui un feu dévorant, et tout autour de lui une grosse tempête.
\VS{4}Il appellera les cieux d'en haut, et la terre pour juger son peuple :
\VS{5}Rassemblez-moi mes bien-aimés qui ont traité alliance avec moi par le sacrifice\FTNT{Mt. 24:29-31.}.
\VS{6}Les cieux aussi annonceront sa justice parce que Dieu est le juge. Sélah.
\VS{7}Ecoute, ô mon peuple ! Et je parlerai. Entends, Israël ! Et je t'avertirai. Moi je suis Dieu, ton Dieu.
\VS{8}Je ne te réprimande pas pour tes sacrifices, tes holocaustes sont continuellement devant moi.
\VS{9}Je ne prendrai point de taureaux de ta maison ni de boucs de tes bergeries\FTNT{Ps. 40:7.}.
\VS{10}Car tous les animaux des forêts sont à moi, toutes les bêtes qui paissent sur mille montagnes.
\VS{11}Je connais tous les oiseaux des montagnes, et tout ce qui se meut dans les champs m'appartient.
\VS{12}Si j'avais faim, je ne t'en dirais rien, car le monde est à moi et tout ce qu'il renferme.
\VS{13}Mangerais-je la chair des gros taureaux ? Et boirais-je le sang des boucs ?
\VS{14}Offre à Dieu la reconnaissance et accomplis tes vœux envers le Très-Haut.
\VS{15}Invoque-moi au jour de ta détresse, je te délivrerai, et tu me glorifieras\FTNT{Ps. 37:5.}.
\VS{16}Dieu dit au méchant : Quoi donc ? Tu énumères mes lois ! Et tu as mon alliance dans ta bouche !
\VS{17}Toi qui hais la correction, et qui jettes mes paroles derrière toi !
\VS{18}Si tu vois un voleur, tu te plais avec lui, et ta part est avec les adultères.
\VS{19}Tu livres ta bouche au mal, et ta langue est un tissu de tromperies.
\VS{20}Tu t'assieds et parles contre ton frère, tu couvres d'opprobre le fils de ta mère.
\VS{21}Tu as fait ces choses-là, et je me suis tu. Tu as estimé que je te ressemble, mais je vais te reprendre et tout mettre sous tes yeux.
\VS{22}Comprenez cela maintenant, vous qui oubliez Dieu, de peur que je ne déchire sans que personne ne vous délivre.
\VS{23}Celui qui offre la louange me glorifie, et à celui qui veille sur sa voie, je lui montrerai le salut de Dieu.
\Chap{51}
\TextTitle{Le cœur repentant, sacrifice agréable à Dieu}
\VerseOne{}Psaume de David, au chef des chantres.
\VS{2}Lorsque Nathan le prophète vint à lui, après que David fut allé vers Bath-Schéba\FTNT{2 S. 11 ; 2 S. 12.}.
\VS{3}Ô Dieu ! Aie pitié de moi dans ta bonté, selon ta grande miséricorde, efface mes transgressions ;
\VS{4}lave-moi parfaitement de mon iniquité et purifie-moi de mon péché.
\VS{5}Car je reconnais mes transgressions, et mon péché est continuellement devant moi\FTNT{Es. 59:12.}.
\VS{6}J'ai péché contre toi, contre toi seul, et j'ai fait ce qui déplaît à tes yeux : En sorte que tu seras juste dans ta sentence, sans reproche dans ton jugement.
\VS{7}Voici, je suis né dans l'iniquité, et ma mère m'a conçu dans le péché.
\VS{8}Mais tu prends plaisir à la vérité au fond du cœur, et tu me fais connaître la sagesse au-dedans de moi.
\VS{9}Purifie-moi de mon péché avec de l'hysope, et je serai pur ; lave-moi, et je serai plus blanc que la neige.
\VS{10}Fais-moi entendre la joie et l'allégresse, et les os que tu as brisés se réjouiront.
\VS{11}Détourne ta face de mes péchés, et efface toutes mes iniquités.
\VS{12}Ô Dieu ! Crée en moi un cœur pur et renouvelle en moi un esprit ferme\FTNT{Mt. 5:8.}.
\VS{13}Ne me rejette pas loin de ta face et ne m'ôte pas ton Esprit Saint.
\VS{14}Rends-moi la joie de ton salut et qu'un esprit bien disposé me soutienne.
\VS{15}J'enseignerai tes voies aux transgresseurs et les pécheurs reviendront à toi.
\VS{16}Ô Dieu, Dieu de mon salut ! Délivre-moi de tant de sang, et ma langue chantera hautement ta justice.
\VS{17}Seigneur, ouvre mes lèvres, et ma bouche annoncera ta louange.
\VS{18}Car tu ne prends point plaisir aux sacrifices, autrement je t'en donnerais ; l'holocauste ne t'est point agréable.
\VS{19}Les sacrifices à Dieu, c'est un esprit brisé. Ô Dieu ! Tu ne méprises point un cœur brisé et contrit.
\VS{20}Répands par ta grâce, tes bienfaits sur Sion, édifie les murs de Jérusalem.
\VS{21}Alors tu prendras plaisir aux sacrifices de justice, à l'holocauste, et aux sacrifices qui se consument entièrement par le feu ; alors on offrira des taureaux sur ton autel.
\Chap{52}
\TextTitle{Sort de l'homme qui se confie en ses richesses}
\VerseOne{}Cantique de David, donné au chef des chantres.
\VS{2}A l'occasion du rapport que Doëg, l'Edomite, vint faire à Saül, en lui disant : David s'est rendu dans la maison d'Achimélec.
\VS{3}Pourquoi te vantes-tu du mal, vaillant homme ? La bonté de Dieu dure à toujours.
\VS{4}Ta langue trame des méchancetés, elle est comme un rasoir affilé qui trompe.
\VS{5}Tu aimes plus le mal que le bien, le mensonge plutôt que de dire la vérité. Sélah.
\VS{6}Tu aimes tous les discours pernicieux, le langage trompeur.
\VS{7}Aussi Dieu te détruira pour toujours, il t'enlèvera et t'arrachera de ta tente ; il te déracinera de la terre des vivants. Sélah.
\VS{8}Les justes le verront et auront de la crainte, et ils se riront d'un tel homme, disant :
\VS{9}Voilà cet homme qui ne tenait point Dieu pour sa protection, mais qui se confiait en ses grandes richesses et qui mettait sa force dans ses mauvais désirs\FTNT{Es. 47:10 ; Lu. 12:15-21.}.
\VS{10}Mais moi, je serai dans la maison de Dieu comme un olivier verdoyant. Je me confie dans la bonté de Dieu pour toujours et à jamais.
\VS{11}Je te célébrerai à jamais, car tu agis ; et je mettrai mon espérance en ton Nom, parce qu'il est bon envers tes fidèles.
\Chap{53}
\TextTitle{Egarement des impies}
\VerseOne{}Cantique de David, donné au chef des chantres, pour le chanter sur la flûte.
\VS{2}L'insensé dit en son cœur : Il n'y a point de Dieu ! Ils se sont corrompus, ils ont commis des injustices abominables ; il n'y a personne qui fasse le bien\FTNT{Ps. 10:4 ; Ro. 1:20-21 ; Ro. 3:12.}.
\VS{3}Dieu a regardé des cieux les fils des hommes, pour voir s'il y a quelqu'un qui soit intelligent, qui cherche Dieu.
\VS{4}Ils se sont tous détournés et se sont tous rendus odieux. Il n'y a personne qui fasse le bien, pas même un seul.
\VS{5}Les ouvriers d'iniquité n'ont-ils point de connaissance ? Ils mangent mon peuple comme s'ils mangeaient du pain. Ils n'invoquent point Dieu.
\VS{6}Ils seront épouvantés sans qu'il y ait sujet d'épouvante, car Dieu a dispersé les os de celui qui campe contre toi. Tu les confondras, car Dieu les a rejetés.
\VS{7}Oh ! Qui fera partir de Sion les délivrances d'Israël ? Quand Dieu aura ramené son peuple captif, Jacob s'égayera, Israël se réjouira.
\Chap{54}
\TextTitle{La délivrance vient de Yahweh}
\VerseOne{}Cantique de David, donné au chef des chantres, pour le chanter avec instruments à cordes.
\VS{2}Lorsque les Ziphiens vinrent dire à Saül : David n'est-il pas caché parmi nous\FTNT{1 S. 23:19 ; 1 S. 26:1.} ?
\VS{3}Ô Dieu ! Délivre-moi par ton Nom et fais-moi justice par ta puissance.
\VS{4}Ô Dieu ! Ecoute ma prière, prête l'oreille aux paroles de ma bouche !
\VS{5}Car des étrangers se sont élevés contre moi, et des gens terribles qui ne mettent pas Dieu devant eux en veulent à ma vie. Sélah.
\VS{6}Voilà, Dieu m'accorde son secours, le Seigneur est de ceux qui soutiennent mon âme.
\VS{7}Il fera retourner le mal sur ceux qui m'épient ; détruis-les selon ta vérité.
\VS{8}Je t'offrirai de bon cœur des sacrifices ; Yahweh ! Je célébrerai ton Nom parce qu'il est bon.
\VS{9}Car il m'a délivré de toute détresse ; et mes yeux se réjouissent à la vue de mes ennemis.
\Chap{55}
\TextTitle{Se garder des méchants}
\VerseOne{}Cantique de David, donné au chef des chantres, pour le chanter avec instruments à cordes.
\VS{2}Ô Dieu ! Prête l'oreille à ma prière et ne te cache pas de mes supplications !
\VS{3}Ecoute-moi et réponds-moi ! J'erre çà et là dans ma méditation et je suis agité
\VS{4}à cause du bruit que fait l'ennemi, à cause de l'oppression du méchant ; car ils font tomber sur moi les outrages, et ils me haïssent jusqu'à la fureur.
\VS{5}Mon cœur tremble au-dedans de moi et les terreurs de la mort tombent sur moi.
\VS{6}La crainte et l'épouvante m'atteignent et le frisson m'habille.
\VS{7}Je dis : Qui me donnera des ailes de colombe ? Je m'envolerais et je trouverais ma demeure.
\VS{8}Voilà, je m'enfuirais bien loin et je me tiendrais au désert. Sélah.
\VS{9}Je m'échapperais en toute hâte, plus rapide que le vent impétueux, que la tempête.
\VS{10}Seigneur, réduis à néant, divise leur langue, car j'ai vu la violence et les querelles dans la ville.
\VS{11}Elles font jour et nuit le tour sur les murailles ; l'iniquité et la malice sont dans son sein.
\VS{12}Les calamités sont au milieu d'elle, et la tromperie et la fraude ne partent point de ses places.
\VS{13}Car ce n'est pas mon ennemi qui m'a diffamé, je le supporterais ; ce n'est point celui qui m'a en haine qui s'élève contre moi, je me cacherais de lui.
\VS{14}Mais c'est toi, ô homme ! Que j'estimais mon égal, mon confident et mon ami\FTNT{Ps. 41:10.} !
\VS{15}Nous prenions plaisir à communiquer nos secrets ensemble, nous allions avec la multitude dans la maison de Dieu.
\VS{16}Que la mort les séduise ! Qu'ils descendent vivants dans le scheol ! Car le mal est dans leur demeure, parmi eux dans leur assemblée.
\VS{17}Mais moi je crie à Dieu, et Yahweh me délivrera.
\VS{18}Le soir, le matin, et à midi je me plains et je gémis, et il entendra ma voix.
\VS{19}Il délivrera mon âme de la guerre et me rendra la paix ; car ils sont nombreux contre moi.
\VS{20}Dieu entendra et témoignera en ma faveur. Lui qui de toute éternité est assis sur son trône. Sélah. Car il n'y a point de changement en eux, et ils ne craignent point Dieu.
\VS{21}Chacun d'eux porte la main sur ceux qui vivaient en paix avec lui, et viole son alliance.
\VS{22}Les paroles de sa bouche sont plus douces que la crème, mais la guerre est dans son cœur ; ses paroles sont plus douces que l'huile, néanmoins elles sont tout autant d'épées nues.
\VS{23}Remets ton sort à Yahweh et il te soulagera, il ne permettra jamais que le juste tombe.
\VS{24}Mais toi, ô Dieu ! Tu les précipiteras au puits de la perdition ; les hommes sanguinaires et trompeurs ne parviendront point à la moitié de leurs jours. C'est en toi que je me confie.
\Chap{56}
\TextTitle{Se glorifier en la Parole de Yahweh}
\VerseOne{}Hymne de David, donné au chef des chantres, pour le chanter sur « Colombe des térébinthes lointains ». Lorsque les Philistins le saisirent à Gath\FTNT{1 S. 21:10-14.}.
\VS{2}Dieu ! Aie pitié de moi, car des hommes m'écrasent et m'oppriment, me faisant tout le jour la guerre, ils m'oppressent.
\VS{3}Mes adversaires me piétinent tout le jour ; car, ô Très-Haut, plusieurs me font la guerre comme des hautains.
\VS{4}Le jour où j'aurai peur, je me confierai en toi.
\VS{5}Je me glorifierai en Dieu, en sa parole ; je me confie en Dieu, je ne craindrai rien. Que peuvent me faire les hommes\FTNT{Ps. 118:6 ; Hé. 13:6.} ?
\VS{6}Tout le jour ils tordent mes propos, et toutes leurs pensées tendent à me nuire.
\VS{7}Ils s'assemblent, ils se tiennent cachés, ils observent mes pas, s'attendant à m'ôter la vie.
\VS{8}C'est par l'iniquité qu'ils espèrent échapper. Dans ta colère, ô Dieu, précipite les peuples !
\VS{9}Tu comptes mes allées et venues ; recueille mes larmes dans tes outres : Ne sont-elles pas écrites dans ton livre ?
\VS{10}Le jour où je crierai à toi, mes ennemis reculeront ; je sais que Dieu est pour moi.
\VS{11}Je me glorifierai en Dieu, en sa parole, je me glorifierai en Yahweh, en sa parole.
\VS{12}Je me confie en Dieu, je ne craindrai rien : Que me fera l'homme ?
\VS{13}Ô Dieu ! Les vœux que je t'ai fait s'accompliront, je te louerai.
\VS{14}Car tu as délivré mon âme de la mort, tu as garanti mes pieds de la chute, afin que je marche devant Dieu, à la lumière des vivants.
\Chap{57}
\TextTitle{Avoir confiance en Dieu dans les difficultés}
\VerseOne{}Hymne de David, donné au chef des chantres, pour le chanter sur Al-Thasheth\FTNT{Al-Thasheth signifie « Ne détruis pas »}. Lorsqu'il se réfugia dans la caverne, poursuivi par Saül\FTNT{1 S. 22:1.}.
\VS{2}Aie pitié de moi, ô Dieu, aie pitié de moi ! Car mon âme cherche un refuge ; je cherche un refuge à l'ombre de tes ailes, jusqu'à ce que les calamités soient passées\FTNT{Ps. 17:8.}.
\VS{3}Je crie au Dieu Très-Haut, au Dieu qui accomplit son œuvre pour moi.
\VS{4}Il m'enverra des cieux la délivrance, il rendra honteux celui qui veut me dévorer. Sélah. Dieu enverra sa bonté et sa vérité.
\VS{5}Mon âme est parmi des lions ; je suis couché au milieu de gens qui vomissent la flamme, parmi des hommes dont les dents sont des lances et des flèches, et dont la langue est une épée aiguë\FTNT{Ps. 59:8 ; Ps. 64:4 ; Ja. 3:5-12.}.
\VS{6}Ô Dieu, élève-toi sur les cieux ! Que ta gloire soit sur toute la terre !
\VS{7}Ils avaient tendu un filet sous mes pas : Mon âme se courbait. Ils avaient creusé une fosse devant moi, mais ils y sont tombés. Sélah.
\VS{8}Mon cœur est affermi, ô Dieu ! Mon cœur est affermi, je chanterai et je ferai retentir mes instruments.
\VS{9}Réveille-toi ma gloire ! Réveillez-vous mon luth et ma harpe ! Je me réveillerai à l'aube du jour.
\VS{10}Seigneur, je te célébrerai parmi les peuples, je te chanterai parmi les nations.
\VS{11}Car ta bonté est grande jusqu'aux cieux, et ta vérité jusqu'aux nues\FTNT{Ps. 118:4-5.}.
\VS{12}Ô Dieu ! Elève-toi sur les cieux ! Que ta gloire soit sur toute la terre !
\Chap{58}
\TextTitle{Yahweh rend justice sur la terre}
\VerseOne{}Hymne de David, donné au chef des chantres, pour le chanter sur Al-Thasheth « Ne détruis pas ».
\VS{2}En vérité, vous, gens de l'assemblée, prononcez-vous ce qui est juste ? Vous, fils des hommes, jugez-vous avec droiture ?
\VS{3}Au contraire, vous tramez des injustices dans votre cœur. Sur la terre, c'est la violence de vos mains que vous placez sur la balance.
\VS{4}Les méchants se sont égarés dès le sein maternel, ils ont erré dès le ventre de leur mère, en parlant faussement.
\VS{5}Ils ont un venin semblable au venin du serpent, ils sont comme l'aspic sourd, qui ferme son oreille,
\VS{6}qui n'entend pas la voix des enchanteurs, du magicien le plus sage.
\VS{7}Ô Dieu, brise-leur les dents dans leur bouche ! Yahweh, brise les mâchoires des lionceaux !
\VS{8}Qu'ils s'écoulent comme de l'eau, et qu'ils se fondent ! Que chacun d'eux bande son arc, mais que ses flèches soient comme si elles étaient rompues !
\VS{9}Qu'ils s'en aillent comme un limaçon qui se fond ! Qu'ils ne voient point le soleil comme l'avorton d'une femme !
\VS{10}Avant que vos chaudières aient senti le feu des épines, l'ardeur de la colère, semblable à un tourbillon, enlèvera chacun d'eux comme de la chair crue..
\VS{11}Le juste se réjouira quand il aura vu la vengeance, il lavera ses pieds avec le sang du méchant.
\VS{12}Et chacun dira : Quoi qu'il en soit, il y a une récompense pour le juste ; quoi qu'il en soit, il y a un Dieu qui juge sur la terre.
\Chap{59}
\TextTitle{Intervention divine}
\VerseOne{}Hymne de David, donné au chef des chantres, pour le chanter sur Al-Thasheth « Ne détruis pas ». Lorsque Saül envoya des gens qui épièrent sa maison afin de le tuer\FTNT{1 S. 19:11.}.
\VS{2}Mon Dieu ! Délivre-moi de mes ennemis, protège-moi de ceux qui s'élèvent contre moi !
\VS{3}Délivre-moi des ouvriers d'iniquité et garde-moi des hommes sanguinaires !
\VS{4}Car voici, ils m'ont dressé des embûches, et des gens robustes se sont assemblés contre moi, bien qu'il n'y ait point en moi de transgression ni de péché, ô Yahweh !
\VS{5}Ils courent çà et là, et se mettent en ordre, bien qu'il n'y ait point d'iniquité en moi. Réveille-toi pour venir au-devant de moi ! Et regarde !
\VS{6}Toi donc, ô Yahweh ! Dieu des armées, Dieu d'Israël, réveille-toi pour visiter toutes les nations ! Ne fais point de grâce à aucun de ceux qui me trahissent ! Sélah.
\VS{7}Ils reviennent chaque soir, ils hurlent comme des chiens, ils font le tour de la ville.
\VS{8}Voici, de leur bouche ils font jaillir le mal, il y a des épées sur leurs lèvres\FTNT{Ja. 3:5-12.} ; car, disent-ils, qui nous entend ?
\VS{9}Mais toi, Yahweh ! Tu te riras d'eux, tu te moqueras de toutes les nations\FTNT{Ps. 2:4.}.
\VS{10}Quelle que soit leur force, je m'attends à toi, car Dieu est ma haute retraite.
\VS{11}Dieu qui me favorise me préviendra, Dieu me fera voir mes adversaires\FTNT{Ps. 118:7.}.
\VS{12}Ne les tue pas, de peur que mon peuple ne l'oublie ; fais-les errer par ta puissance et abats-les ; Seigneur, notre bouclier !
\VS{13}Leur bouche pèche à chaque parole de leurs lèvres ; qu'ils soient pris par leur orgueil ! Ils ne tiennent que des discours de malédiction et de mensonge.
\VS{14}Consume-les avec fureur, consume-les de sorte qu'ils ne soient plus ! Qu'on sache que Dieu domine sur Jacob et jusqu'aux extrémités de la terre ! Sélah.
\VS{15}Qu'ils reviennent le soir, et qu'ils hurlent comme des chiens, et qu'ils fassent le tour de la ville.
\VS{16}Qu'ils errent çà et là cherchant leur nourriture, et qu'ils passent la nuit sans être rassasiés.
\VS{17}Mais moi je chanterai ta force, je louerai dès le matin à haute voix ta bonté\FTNT{Ps. 88:14.}. Car tu es pour moi une haute retraite, et mon asile au jour de ma détresse.
\VS{18}Ma force ! Je te chanterai ; car Dieu est ma haute retraite, le Dieu qui me favorise.
\Chap{60}
\TextTitle{Yahweh, le meilleur secours}
\VerseOne{}Hymne de David, pour enseigner, donné au chef des chantres, pour le chanter sur le lis lyrique,
\VS{2}lorsqu'il fit la guerre contre les Syriens de Mésopotamie, et contre les Syriens de Tsoba, et que Joab revint et défit douze mille Edomites dans la vallée du sel\FTNT{2 S. 8:3-13 ; 1 Ch. 18:3-12.}.
\VS{3}Ô Dieu ! Tu nous as rejetés, tu nous as dispersés, tu t'es irrité : Reviens vers nous !
\VS{4}Tu as ébranlé la terre et l'as mise en pièces ; répare ses brèches, car elle chancelle !
\VS{5}Tu as fait voir à ton peuple des choses dures, tu nous as abreuvés d'un vin d'étourdissement\FTNT{Es. 51:17-21 ; Ap. 14:10.}.
\VS{6}Mais tu as donné une bannière à ceux qui te craignent, afin de l'élever bien haut pour l'amour de ta vérité. Sélah.
\VS{7}Afin que ceux que tu aimes soient délivrés ; sauve-moi par ta droite et exauce-moi\FTNT{Ps. 108:6.}.
\VS{8}Dieu a parlé dans son lieu saint : Je me réjouirai, je partagerai Sichem, je mesurerai la vallée de Succoth ;
\VS{9}Galaad est à moi, Manassé aussi est à moi, et Ephraïm est la protection de ma tête et Juda mon sceptre.
\VS{10}Moab est le bassin où je me lave ; je jette mon soulier sur Edom ; pays des Philistins, pousse des cris de guerre à mon sujet\FTNT{2 S. 8:2 ; 1 Ch. 18:2.}.
\VS{11}Qui me conduira dans la ville forte ? Qui me conduira jusqu'en Edom ?
\VS{12}Ne sera-ce pas toi, ô Dieu, qui nous avais rejetés, et qui ne sortais plus, ô Dieu, avec nos armées ?
\VS{13}Donne-nous du secours pour sortir de la détresse ! Car la délivrance qu'on attend de l'homme est vanité\FTNT{Jé. 17:5 ; Ps. 118:8.}.
\VS{14}Avec le secours de Dieu, nous ferons des exploits, et il foulera nos ennemis.
\Chap{61}
\TextTitle{Dieu, le parfait Refuge}
\VerseOne{}Psaume de David, donné au chef des chantres, pour le chanter sur instruments à cordes.
\VS{2}Ô Dieu, je crie à toi, sois attentif à ma prière !
\VS{3}Je crie à toi du bout de la terre, le cœur abattu ; conduis-moi sur le rocher qui est trop haut pour moi !
\VS{4}Car tu es mon refuge, une tour forte au-devant de l'ennemi.
\VS{5}Je séjournerai éternellement dans ta tente, je me retirerai à l'ombre de tes ailes. Sélah.
\VS{6}Car, ô Dieu ! Tu exauces mes vœux, et tu me donnes l'héritage de ceux qui craignent ton Nom.
\VS{7}Tu ajoutes des jours aux jours du roi ; que ses années se prolongent à jamais !
\VS{8}Qu'il demeure toujours dans la présence de Dieu ! Que la bonté et la vérité le gardent !
\VS{9}Ainsi je chanterai ton Nom à perpétuité, en rendant mes vœux chaque jour.
\Chap{62}
\TextTitle{La confiance en Dieu}
\VerseOne{}Psaume de David, donné au chef des chantres, d'après Jeduthun.
\VS{2}Quoiqu'il en soit, mon âme se repose en Dieu ; c'est de lui que vient ma délivrance.
\VS{3}Quoiqu'il en soit, il est mon rocher, ma délivrance, et ma haute retraite ; je ne serai pas entièrement ébranlé.
\VS{4}Jusqu'à quand accablerez-vous de maux un homme ? Vous serez tous mis à mort, et vous serez comme le mur qui penche, comme une cloison qui a été ébranlée.
\VS{5}Ils ne font que consulter pour le faire déchoir de son élévation ; ils prennent plaisir au mensonge ; ils bénissent de leur bouche, mais au-dedans ils maudissent. Sélah.
\VS{6}Mais toi mon âme, demeure tranquille, regarde à Dieu, car mon espérance est en lui.
\VS{7}Quoiqu'il en soit, il est mon rocher, ma délivrance, et ma haute retraite ; je ne serai point ébranlé.
\VS{8}En Dieu est ma délivrance et ma gloire ; en Dieu est le rocher de ma force et ma retraite.
\VS{9}Peuples, confiez-vous en lui en tout temps, déchargez votre cœur sur lui ! Dieu est notre retraite. Sélah.
\VS{10}Oui, vanité, les fils de l'homme ! Mensonge, les fils de l'homme ! Dans une balance, ils monteraient tous ensemble, plus légers qu'un souffle.
\VS{11}Ne vous confiez pas dans la violence ni dans la rapine ; ne devenez point vains ; quand les richesses abonderont, n'y mettez point votre cœur.
\VS{12}Dieu a parlé une fois, j'ai entendu cela deux fois : C'est que la force est à Dieu.
\VS{13}Et c'est à toi, Seigneur, qu'appartient la bonté ; certainement tu rendras à chacun selon son œuvre\FTNT{Jé. 32:19 ; Pr. 24:12 ; Job. 34:11 ; Mt. 16:27 ; Ro. 2:6.}.
\Chap{63}
\TextTitle{Soif de la présence de Dieu}
\VerseOne{}Psaume de David, lorsqu'il était dans le désert de Juda\FTNT{1 S. 22:5 ; 1 S. 23:14-15.}.
\VS{2}Ô Dieu ! Tu es mon Dieu, je te cherche au point du jour ; mon âme a soif de toi, mon corps soupire après toi sur cette terre aride, desséchée, et sans eau\FTNT{Ps. 42:2 ; Ps. 84:3 ; Ps. 143:6.}.
\VS{3}Ainsi je te contemple dans ton lieu saint pour voir ta force et ta gloire.
\VS{4}Car ta bonté vaut mieux que la vie, mes lèvres te louent.
\VS{5}Et ainsi je te bénirai donc toute ma vie\FTNT{Ps. 104:33.}, j'élèverai mes mains en ton Nom.
\VS{6}Mon âme est rassasiée comme de mets gras et succulents, et ma bouche te loue avec un chant de réjouissance.
\VS{7}Quand je me souviens de toi dans mon lit, je médite sur toi durant les veilles de la nuit\FTNT{Ps. 16:7 ; Ps. 119:55.}.
\VS{8}Car tu m'as secouru, je me réjouirai à l'ombre de tes ailes.
\VS{9}Mon âme s'est attachée à toi pour te suivre, ta droite me soutient.
\VS{10}Mais ceux-ci qui demandent que mon âme tombe en ruine, entreront au plus bas de la terre.
\VS{11}On les détruira à coups d'épée, ils seront la proie des chacals.
\VS{12}Mais le roi se réjouira en Dieu ; quiconque jure par lui s'en glorifiera, car la bouche de ceux qui mentent sera fermée\FTNT{Ps. 107:42 ; Job. 5:16.}.
\Chap{64}
\TextTitle{Yahweh, le seul abri}
\VerseOne{}Psaume de David, donné au chef des chantres.
\VS{2}Ô Dieu ! Ecoute ma voix quand je m'écrie. Protège ma vie contre l'ennemi que je crains !
\VS{3}Cache-moi des complots des méchants, de l'assemblée tumultueuse des ouvriers d'iniquité !
\VS{4}Ils aiguisent leur langue comme une épée\FTNT{Jé. 9:3 ; Ps. 11:2 ; Ps. 59:8.}, ils tirent comme des flèches leurs paroles amères,
\VS{5}afin de tirer sur l'innocent dans sa cachette ; ils tirent soudainement sur lui et n'ont aucune crainte.
\VS{6}Ils se fortifient dans leur méchanceté, tiennent des discours pour tendre des pièges, ils disent : Qui les verra\FTNT{Job. 24:15.} ?
\VS{7}Ils cherchent curieusement des méchancetés ; ils ont sondé tout ce qui se peut sonder, même ce qui peut être au–dedans de l'homme, et au cœur le plus profond.
\VS{8}Mais Dieu lance contre eux ses traits, soudain les voilà frappés.
\VS{9}Leur langue a causé leur chute ; tous ceux qui les voient secouent leur tête.
\VS{10}Et tous les hommes craindront et raconteront l'œuvre de Dieu, et considéreront ce qu'il aura fait.
\VS{11}Le juste se réjouira en Yahweh, et se retirera vers lui, et tous ceux qui sont droits de cœur s'en glorifieront\FTNT{Ps. 63:12 ; Ps. 97:12.}.
\Chap{65}
\TextTitle{Le règne de Yahweh sur la nature}
\VerseOne{}Psaume de David. Cantique. Donné au chef des chantres.
\VS{2}Ô Dieu ! Dans le calme, on te louera dans Sion, et l'on accomplira nos vœux\FTNT{Ps. 50:14 ; Ps. 66:13.}.
\VS{3}Tu entends nos prières, toute chair viendra jusqu'à toi.
\VS{4}Les iniquités prévalent sur moi, mais tu feras la propitiation de nos transgressions.
\VS{5}Heureux celui que tu choisis et que tu admets dans ta présence pour qu'il habite dans tes parvis ! Nous serons rassasiés des biens de ta maison, des biens du saint lieu de ton temple.
\VS{6}Dans ta justice, tu nous réponds par des choses terribles, ô Dieu de notre salut, espoir de toutes les extrémités lointaines de la terre et de la mer.
\VS{7}Il affermit les montagnes par sa force, il est ceint de puissance.
\VS{8}Il apaise le mugissement de la mer, le mugissement de leurs flots, et le tumulte des peuples.
\VS{9}Ceux qui habitent aux extrémités de la terre ont peur de tes prodiges ; tu réjouis l'orient et l'occident.
\VS{10}Tu visites la terre, tu lui donnes l'abondance, tu la combles de richesses ; le ruisseau de Dieu est plein d'eau ; tu prépares le blé, quand tu l'établis ainsi.
\VS{11}Tu arroses ses sillons, et tu aplanis ses mottes ; tu l'amollis par la pluie, et tu bénis son germe\FTNT{Es. 55:10 ; Ps. 104:13-14.}.
\VS{12}Tu couronnes l'année de tes biens, et tes voies versent l'abondance.
\VS{13}Les plaines du désert sont abreuvées et les collines sont ceintes de joie.
\VS{14}Les pâturages se couvrent de brebis, et les vallées se revêtent de froments ; les cris de joie et les chants retentissent.
\Chap{66}
\TextTitle{Louange au Dieu de grâces}
\VerseOne{}Cantique. Psaume, donné au chef des chantres. Vous tous habitants de toute la terre, poussez des cris de triomphe à Dieu.
\VS{2}Chantez la gloire de son Nom, faites éclater sa gloire par vos louanges.
\VS{3}Dites à Dieu : Que tes œuvres sont redoutables ! Tes ennemis te mentiront à cause de la grandeur de ta force.
\VS{4}Toute la terre se prosterne devant toi et te chante ; elle chante ton Nom. Sélah.
\VS{5}Venez et voyez les œuvres de Dieu : Il est redoutable quand il agit sur les fils des hommes.
\VS{6}Il a fait de la mer une terre sèche ; on a passé le fleuve à pied sec ; là, nous nous sommes réjouis en lui\FTNT{Ex. 14:21 ; Jos. 3:14-17.}.
\VS{7}Il domine par sa puissance éternellement ; ses yeux prennent garde sur les nations\FTNT{Ps. 14:2 ; Ps. 33:13 ; Job. 28:24.} ; les rebelles ne pourront point s'élever. Sélah.
\VS{8}Peuples, bénissez notre Dieu, et faites retentir le son de sa louange.
\VS{9}C'est lui qui a remis notre âme en vie, et qui n'a point permis que nos pieds chancellent.
\VS{10}Car, ô Dieu, tu nous as éprouvés ! Tu nous a fait passer au creuset comme l'argent.
\VS{11}Tu nous as amenés dans le filet, tu as mis sur nos reins un pesant fardeau.
\VS{12}Tu as fait monter des hommes sur notre tête, et nous avons passé par le feu et par l'eau. Mais tu nous as fait entrer dans un lieu d'abondance.
\VS{13}J'entrerai dans ta maison avec des holocaustes, j'accomplirai mes vœux envers toi\FTNT{Ps. 22:26 ; Ps. 76:12 ; Ps. 116:14.}.
\VS{14}Pour eux, mes lèvres se sont ouvertes et ma bouche les a prononcés dans ma détresse.
\VS{15}Je t'offrirai en holocauste des brebis grasses, avec la graisse des béliers, je te sacrifierai des taureaux et des boucs. Sélah.
\VS{16}Vous tous qui craignez Dieu, venez, écoutez, et je raconterai ce qu'il a fait à mon âme.
\VS{17}Je l'ai invoqué de ma bouche, et la louange a été sur ma langue.
\VS{18}Si j'avais conçu l'iniquité dans mon cœur, le Seigneur ne m'aurait pas écouté\FTNT{Jn. 9:31.}.
\VS{19}Mais certainement Dieu m'a écouté, il a été attentif à la voix de ma prière.
\VS{20}Béni soit Dieu qui n'a point rejeté ma prière, et qui n'a point éloigné de moi sa bonté.
\Chap{67}
\TextTitle{Louange des peuples}
\VerseOne{}Psaume. Cantique donné au chef des chantres, pour le chanter avec instruments à cordes.
\VS{2}Que Dieu ait pitié de nous et qu'il nous bénisse, qu'il fasse luire sa face sur nous\FTNT{No. 6:25 ; Ps. 4:7 ; Ps. 31:17 ; Ps. 119:135.}. Sélah.
\VS{3}Afin que ta voie soit connue sur la terre et ta délivrance parmi toutes les nations.
\VS{4}Les peuples te célébreront, ô Dieu ! Tous les peuples te célébreront\FTNT{Ps. 22:27 ; Ps. 68:33.} !
\VS{5}Les peuples se réjouissent et chantent de joie, car tu juges les peuples avec droiture et tu conduis les nations sur la terre\FTNT{Ps. 96:10.}. Sélah.
\VS{6}Les peuples te célébreront, ô Dieu ! Tous les peuples te célébreront !
\VS{7}La terre produira son fruit ; Dieu, notre Dieu, nous bénira.
\VS{8}Dieu nous bénira, et toutes les extrémités de la terre le craindront.
\Chap{68}
\TextTitle{Yahweh, le Dieu glorieux}
\VerseOne{}Psaume. Cantique de David, donné au chef des chantres.
\VS{2}Que Dieu se lève, et ses ennemis seront dispersés, et ceux qui le haïssent s'enfuiront devant lui\FTNT{No. 10:35.}.
\VS{3}Tu les chasseras comme la fumée est chassée par le vent ; comme la cire se fond devant le feu, ainsi les méchants périront devant Dieu\FTNT{Ps. 37:20 ; Ps. 97:5.}.
\VS{4}Mais les justes se réjouiront et s'égayeront devant Dieu, et tressailliront de joie\FTNT{Ps. 67:4-5.}.
\VS{5}Chantez à Dieu, célébrez son Nom ! Exaltez celui qui est monté sur les cieux ! Son Nom est Yahweh ! Réjouissez-vous dans sa présence.
\VS{6}Il est le père des orphelins et le juge des veuves ; Dieu est dans sa demeure sainte\FTNT{Ps. 146:9}.
\VS{7}Dieu donne une famille à ceux qui étaient abandonnés, il délivre ceux qui étaient enchaînés, mais les rebelles habitent sur une terre déserte.
\VS{8}Ô Dieu ! Quand tu sortis devant ton peuple, quand tu marchais dans le désert ! Sélah.
\VS{9}La terre trembla et les cieux répandirent leurs eaux à cause de la présence de Dieu, le mont Sinaï trembla à cause de la présence de Dieu, du Dieu d'Israël\FTNT{Ex. 19:18 ; Jg. 5:5.}.
\VS{10}Ô Dieu ! Tu as fait tomber une pluie abondante sur ton héritage, et quand il était épuisé, tu l'as rétabli.
\VS{11}Ton troupeau établit sa demeure dans le pays, que par ta bonté tu avais préparé pour les malheureux, ô Dieu !
\VS{12}Le Seigneur donne une parole, et les messagères de bonnes nouvelles sont une grande armée.
\VS{13}Les rois des armées se sont enfuis, ils se sont enfuis, et celle qui se tenait à la maison a partagé le butin\FTNT{1 S. 30:16.}.
\VS{14}Tandis que vous vous couchez dans les étables, les ailes de la colombe sont couvertes d'argent, et son plumage est d'un jaune d'or.
\VS{15}Quand le Tout-Puissant dispersa les rois dans le pays, il devint blanc comme la neige du Tsalmon.
\VS{16}La montagne de Dieu est un mont de Basan ; une montagne élevée, un mont de Basan.
\VS{17}Pourquoi l'insultez-vous, montagnes dont le sommet est élevé ? Dieu a désiré cette montagne pour y habiter, et Yahweh y demeurera à jamais.
\VS{18}Les chars de Dieu se comptent par vingt-mille, par milliers et par milliers ; le Seigneur est au milieu d'eux ; le Sinaï est dans le sanctuaire.
\VS{19}Tu es monté dans les hauteurs, tu as emmené des captifs, tu as pris des dons pour les distribuer parmi les hommes, et même parmi les rebelles, afin qu'ils habitent dans le lieu de Yahweh Dieu\FTNT{Ep. 4:8-10. Cette prophétie concerne la résurrection du Seigneur Jésus-Christ.}.
\VS{20}Béni soit le Seigneur, qui tous les jours nous comble de ses biens ; Dieu est notre délivrance. Sélah.
\VS{21}Dieu est pour nous le Dieu de délivrance, et les issues de la mort sont à Yahweh le Seigneur.
\VS{22}Certainement, Dieu écrasera la tête de ses ennemis\FTNT{Ge. 3:15.}, le sommet de la tête chevelue de celui qui marche dans ses péchés.
\VS{23}Le Seigneur dit : Je les ramènerai de Basan\FTNT{No. 21:33-35.}, je les ramènerai du fond de la mer.
\VS{24}Afin que tu plonges ton pied dans le sang\FTNT{Ps. 58:11.}, et que la langue de tes chiens ait sa part de tes ennemis.
\VS{25}Ils voient ta marche, ô Dieu ! Ils ont vu ta marche dans le lieu saint, la marche de mon Dieu, mon Roi.
\VS{26}Les chantres allaient devant, ensuite les joueurs d'instruments, et au milieu les jeunes filles jouant du tambour\FTNT{Ex. 15:20 ; 1 S. 18:6.}.
\VS{27}Bénissez Dieu dans les assemblées, bénissez le Seigneur, vous qui êtes descendants d'Israël.
\VS{28}Là sont Benjamin, le plus jeune qui domine sur eux, les chefs de Juda et leur corps d'armée, les chefs de Zabulon, et les chefs de Nephthali.
\VS{29}Ton Dieu ordonne que tu sois puissant. Affermis, ô Dieu, ce que tu as fait pour nous.
\VS{30}Dans ton temple, à Jérusalem, les rois t'amèneront des présents\FTNT{1 R. 10:10 ; Ps. 72:10 ; 2 Ch. 32:23.}.
\VS{31}Epouvante les bêtes sauvages des roseaux, la troupe des taureaux, et les veaux des peuples, et ceux qui se prosternent avec des pièces d'argent. Disperse les peuples qui prennent plaisir à la guerre.
\VS{32}De grands seigneurs viendront d'Egypte ; l'Ethiopie se hâtera d'étendre ses mains vers Dieu.
\VS{33}Royaumes de la terre, chantez à Dieu, célébrez le Seigneur ! Sélah.
\VS{34}Chantez celui qui est monté dans les cieux des cieux, les cieux éternels ; voilà, il fait retentir de sa voix un son puissant.
\VS{35}Attribuez la force à Dieu ; sa majesté est sur Israël, et sa force est dans les nuées.
\VS{36}Dieu ! Tu es redouté à cause de ton lieu saint. Le Dieu d'Israël est celui qui donne la force et la puissance à son peuple. Béni soit Dieu !
\Chap{69}
\TextTitle{Dieu attentif à la prière de ceux qui s'humilient}
\VerseOne{}Psaume de David, donné au chef des chantres, pour le chanter sur les lis.
\VS{2}Délivre-moi, ô Dieu, car les eaux menacent ma vie\FTNT{Ps. 124:4 ; Ps. 144:7.}.
\VS{3}Je suis enfoncé dans un bourbier profond, sans appui ; je suis entré au plus profond des eaux, et les courants d'eau me submergent.
\VS{4}Je suis las de crier, mon gosier se dessèche, mes yeux se consument pendant que je m'attends à Dieu.
\VS{5}Ceux qui me haïssent sans cause\FTNT{Jn. 15:25.} dépassent en nombre les cheveux de ma tête ; ceux qui tâchent de me ruiner et qui sont mes ennemis à tort se sont renforcés ; je dois rendre ce que je n'avais point ravi.
\VS{6}Ô Dieu ! Tu connais ma folie et mes fautes ne te sont point cachées.
\VS{7}Ô Seigneur Yahweh des armées ! Que ceux qui se confient en toi ne soient point honteux à cause de moi ; et que ceux qui te cherchent ne soient point humiliés à cause de moi, ô Dieu d'Israël !
\VS{8}Car pour l'amour de toi j'ai souffert l'opprobre, la honte a couvert mon visage.
\VS{9}Je suis devenu un étranger pour mes frères, et un homme de dehors pour les fils de ma mère\FTNT{Ge. 31:14-15 ; Jn. 7:3-5}.
\VS{10}Car le zèle de ta maison me dévore\FTNT{Jn. 2:17 ; Ro. 15:3.}, et les outrages de ceux qui t'insultaient sont tombés sur moi.
\VS{11}Je pleure et je jeûne : C'est ce qui m'attire l'opprobre.
\VS{12}Je prends un sac pour vêtement, et je suis l'objet de leurs discours moqueurs.
\VS{13}Ceux qui sont assis à la porte parlent de moi, et les buveurs de boissons fortes me mettent en chanson\FTNT{Job. 30:9 ; La. 3:14.}.
\VS{14}Mais je t'adresse ma prière, ô Yahweh\FTNT{Ps. 102:2.} ! Que ce soit le temps favorable, ô Dieu ! Par ta grande bonté. Réponds-moi en m'assurant ta délivrance.
\VS{15}Délivre-moi de la boue, que je ne m'y enfonce point\FTNT{Ps. 40:3.}, et que je sois délivré de ceux qui me haïssent, et des eaux profondes.
\VS{16}Que les courants d'eau ne me submergent plus, que l'abîme ne m'engloutisse point, et que le puits ne ferme point sa bouche sur moi.
\VS{17}Yahweh ! Exauce-moi, car ta bonté est agréable ; dans tes grandes compassions, tourne ta face vers moi ;
\VS{18}et ne cache point ta face à ton serviteur, car je suis en détresse. Hâte-toi, exauce-moi !
\VS{19}Approche-toi de mon âme, rachète-la ; délivre-moi à cause de mes ennemis.
\VS{20}Tu connais toi-même mon opprobre, et ma honte, et mon ignominie ; tous mes ennemis sont devant toi.
\VS{21}L'opprobre m'a brisé le cœur, et je suis languissant ; j'ai attendu que quelqu'un ait compassion de moi, mais il n'y en a point eu. J'ai attendu des consolateurs, mais je n'en ai point trouvé.
\VS{22}Ils m'ont au contraire donné du fiel\FTNT{Mt. 27:34 ; Mt. 27:48.} pour mon repas ; et dans ma soif, ils m'ont abreuvé de vinaigre.
\VS{23}Que leur table soit pour eux un piège et un appât au sein de leur perfection.
\VS{24}Que leurs yeux soient tellement obscurcis, qu'ils ne puissent point voir ; et fais continuellement chanceler leurs reins.
\VS{25}Répands ton indignation sur eux, et que l'ardeur de ta colère les saisisse.
\VS{26}Que leur campement soit désolé, et qu'il n'y ait personne qui habite dans leurs tentes.
\VS{27}Car ils persécutent celui que tu avais frappé, et racontent les souffrances de ceux que tu blesses.
\VS{28}Mets des iniquités à leurs iniquités ; et qu'ils n'entrent point dans ta justice.
\VS{29}Qu'ils soient effacés du livre de vie, et qu'ils ne soient point inscrits avec les justes.
\VS{30}Mais pour moi, qui suis affligé et dans la douleur, ta délivrance, ô Dieu, m'élèvera en une haute retraite.
\VS{31}Je louerai le Nom de Dieu par des cantiques et je le glorifierai par des louanges.
\VS{32}Cela est agréable à Yahweh plus qu'un taureau avec des cornes et des sabots fendus.
\VS{33}Les malheureux le voient et ils se réjouissent ; que votre cœur vive, vous qui cherchez Dieu.
\VS{34}Car Yahweh exauce les misérables et ne méprise point ses prisonniers.
\VS{35}Que les cieux et la terre le louent ; que la mer et tout ce qui s'y meut le louent aussi\FTNT{Ps. 96:11.}.
\VS{36}Car Dieu délivrera Sion et bâtira les villes de Juda ; on y habitera et on la possèdera.
\VS{37}Et la postérité de ses serviteurs en fera son héritage, et ceux qui aiment son Nom y auront leur demeure.
\Chap{70}
\TextTitle{Le pauvre et l'indigent}
\VerseOne{}Psaume de David, pour souvenir, donné au chef des chantres.
\VS{2}Dieu ! Hâte-toi de me délivrer, ô Dieu ! Hâte-toi de venir à mon secours\FTNT{Ps. 40:14 ; Ps. 71:12.}.
\VS{3}Que ceux qui cherchent mon âme soient honteux et rougissent\FTNT{Ps. 35:4 ; Ps. 71:13.} ; et que ceux qui prennent plaisir à mon mal soient repoussés en arrière et soient confus.
\VS{4}Que ceux qui disent : Aha ! Aha ! Retournent en arrière par l'effet de leur honte.
\VS{5}Que tous ceux qui te cherchent exultent et se réjouissent en toi ; et que ceux qui aiment ta délivrance disent toujours : Glorifié soit Dieu !
\VS{6}Moi, je suis affligé et misérable, ô Dieu ! Hâte-toi de venir vers moi ; tu es mon secours et mon libérateur, ô Yahweh ! Ne tarde point.
\Chap{71}
\TextTitle{Demeurer en Dieu jusqu'au bout}
\VerseOne{}Yahweh ! Je cherche en toi mon refuge : Que je ne sois jamais confus !
\VS{2}Délivre-moi par ta justice et sauve-moi. Incline ton oreille vers moi, mets-moi en sûreté.
\VS{3}Sois pour moi le rocher de mon refuge, afin que je puisse toujours m'y retirer ; tu as donné l'ordre de me mettre en sûreté, car tu es mon rocher et ma forteresse.
\VS{4}Mon Dieu ! Délivre-moi de la main du méchant, de la main du pervers et de l'oppresseur.
\VS{5}Car tu es mon espérance, Seigneur Yahweh ! Ma confiance dès ma jeunesse.
\VS{6}Je m'appuie sur toi dès le ventre de ma mère ; c'est toi qui m'as tiré hors des entrailles de ma mère\FTNT{Ps. 22:10-11.} ; tu es le sujet continuel de mes louanges.
\VS{7}Je suis pour plusieurs comme un miracle, mais tu es mon puissant refuge.
\VS{8}Que ma bouche soit remplie de ta louange et de ta gloire chaque jour.
\VS{9}Ne me rejette point au temps de ma vieillesse ; ne m'abandonne point maintenant que ma force est consumée.
\VS{10}Car mes ennemis ont parlé de moi, et ceux qui épient mon âme ont pris conseil ensemble,
\VS{11}disant : Dieu l'a abandonné. Poursuivez-le et saisissez-le, car il n'y a personne qui le délivre.
\VS{12}Dieu, ne t'éloigne point de moi ! Mon Dieu hâte-toi de venir à mon secours !
\VS{13}Que ceux qui sont les ennemis de mon âme soient honteux et défaits ; et que ceux qui cherchent mon malheur soient enveloppés d'opprobre et de honte.
\VS{14}Mais moi, j'espèrerai toujours et je te louerai tous les jours davantage.
\VS{15}Ma bouche racontera chaque jour ta justice et ta délivrance, bien que je n'en sache point le nombre.
\VS{16}Je marcherai par la force du Seigneur Yahweh ; je raconterai ta seule justice.
\VS{17}Ô Dieu ! Tu m'as enseigné dès ma jeunesse et j'ai annoncé jusqu'à présent tes merveilles.
\VS{18}Ô Dieu ! Ne m'abandonne pas, même dans la blanche vieillesse. Afin que j'annonce ta force à cette génération présente, ta puissance à la génération à venir.
\VS{19}Car ta justice, ô Dieu, est haut élevée, car tu as fait de grandes choses. Ô Dieu, qui est semblable à toi ?
\VS{20}Tu m'as fait éprouver bien des détresses et des malheurs, mais tu me redonneras la vie et tu me feras remonter hors des abîmes de la terre.
\VS{21}Relève ma grandeur et console-moi encore.
\VS{22}Je te louerai au son du luth, je chanterai ta fidélité, mon Dieu, je te célèbrerai avec la harpe, Saint d'Israël !
\VS{23}Mes lèvres et mon âme, que tu as rachetée, pousseront des cris de joie quand je te chanterai.
\VS{24}Ma langue aussi publiera chaque jour ta justice, car ceux qui cherchent mon malheur seront honteux et rougiront.
\Chap{72}
\TextTitle{Le royaume messianique}
\VerseOne{}De Salomon. Ô Dieu, donne tes jugements au roi et ta justice au fils du roi.
\VS{2}Qu'il juge avec justice ton peuple, et tes malheureux avec équité.
\VS{3}Que les montagnes portent la paix pour le peuple, et que les collines la portent en justice.
\VS{4}Qu'il fasse droit aux malheureux du peuple, qu'il délivre les fils du misérable, et qu'il écrase l'oppresseur !
\VS{5}Ils te craindront tant que le soleil et la lune dureront d'âge en âge.
\VS{6}Il descendra comme la pluie sur l'herbe fauchée, comme les ondées qui arrosent la terre.
\VS{7}En son temps, le juste fleurira, et il y aura abondance de paix jusqu'à ce qu'il n'y ait plus de lune.
\VS{8}Il dominera depuis une mer jusqu'à l'autre, et depuis le fleuve jusqu'aux extrémités de la terre.
\VS{9}Les habitants des déserts se courberont devant lui, et ses ennemis lécheront la poussière.
\VS{10}Les rois de Tarsis et des îles lui rapporteront des dons ; les rois de Saba et de Séba lui apporteront des présents.
\VS{11}Tous les rois aussi se prosterneront devant lui, toutes les nations le serviront.
\VS{12}Car il délivrera le pauvre qui crie vers lui, l'affligé et celui qui n'a personne qui l'aide\FTNT{Ps. 34:18 ; Job. 29:12.}.
\VS{13}Il aura compassion du pauvre et du misérable, et il sauvera les âmes des misérables.
\VS{14}Il garantira leur âme de la fraude et de la violence, et leur sang sera précieux devant ses yeux.
\VS{15}Il vivra donc, et on lui donnera de l'or de Séba, et on fera des prières pour lui continuellement ; on le bénira chaque jour.
\VS{16}Les blés abonderont dans le pays, au sommet des montagnes, et leurs épis s'agiteront comme les arbres du Liban ; les hommes fleuriront dans les villes comme l'herbe de la terre.
\VS{17}Sa renommée durera à toujours ; sa renommée ira de père en fils tant que le soleil durera ; et on se bénira en lui ; toutes les nations le diront heureux.
\VS{18}Béni soit Yahweh Dieu, le Dieu d'Israël, qui seul fait des choses merveilleuses !
\VS{19}Béni soit éternellement son Nom glorieux, et que toute la terre soit remplie de sa gloire. Amen ! Oui, amen !
\VS{20}Fin des prières de David, fils d'Isaï.
\Chap{73}
\TextTitle{L'orgueil des méchants}
\VerseOne{}Psaume d'Asaph. Quoi qu'il en soit, Dieu est bon pour Israël, pour ceux qui ont le cœur pur\FTNT{Mt. 5:8.}.
\VS{2}Toutefois, mes pieds allaient fléchir, mes pas étaient sur le point de glisser.
\VS{3}Car j'ai porté envie aux insensés en voyant la prospérité des méchants.
\VS{4}Rien ne les tourmente jusqu'à leur mort, et leur corps est gras.
\VS{5}Ils n'ont point de part aux peines des humains, et ils ne sont point frappés avec les autres hommes.
\VS{6}C'est pourquoi l'orgueil les environne comme un collier, et un vêtement de violence les couvre.
\VS{7}Les yeux leur sortent dehors à force de graisse ; ils surpassent les desseins de leur cœur.
\VS{8}Ils sont pernicieux, et parlent méchamment d'opprimer ; ils parlent d'une manière hautaine.
\VS{9}Ils élèvent leur bouche jusqu'aux cieux et leur langue parcourt la terre.
\VS{10}C'est pourquoi son peuple se tourne de leur côté, il avale l'eau abondamment.
\VS{11}Ils disent : Comment Dieu saurait-il ? Comment le Très-Haut connaîtrait-il\FTNT{Es. 29:15 ; Ez. 8:12 ; Ps. 94:7 ; Job. 22:12-13.} ?
\VS{12}Voilà, ceux-ci sont méchants, ils prospèrent toujours dans ce monde et acquièrent de plus en plus de richesses.
\VS{13}Quoi qu'il en soit, c'est donc en vain que j'ai purifié mon cœur et que j'ai lavé mes mains dans l'innocence\FTNT{Mal. 3:14 ; Job. 35:3}.
\VS{14}Je suis frappé tous les jours, et tous les matins mon châtiment est là.
\VS{15}Si je disais : Je veux parler comme eux, voici je trahirais la génération de tes fils.
\VS{16}Toutefois, j'ai tâché de connaître cela, mais cela m'a paru fort difficile,
\VS{17}jusqu'à ce que je sois entré dans le sanctuaire de Dieu et que j'aie considéré la fin de telles gens.
\VS{18}Quoi qu'il en soit, tu les as mis sur des voies glissantes, tu les fais tomber dans des précipices.
\VS{19}Comment ont-ils été ainsi détruits en un instant ? Ont-ils défailli ? Ont-ils été consumés d'épouvante ?
\VS{20}Ils sont comme un songe lorsqu'on s'est réveillé. Seigneur, tu méprises leur image à ton réveil.
\VS{21}Quand mon cœur s'aigrissait et que je me sentais percé dans les entrailles,
\VS{22}j'étais alors stupide, et je n'avais aucune connaissance ; j'étais comme une bête dans ta présence.
\VS{23}Je serai donc toujours avec toi ; tu m'as pris par la main droite.
\VS{24}Tu me conduiras par ton conseil, et tu me recevras dans la gloire.
\VS{25}Quel autre ai-je au ciel ? Or sur la terre je ne prends plaisir qu'en toi seul.
\VS{26}Ma chair et mon cœur étaient consumés, mais Dieu est le rocher de mon cœur, et mon partage pour toujours.
\VS{27}Car voilà, ceux qui s'éloignent de toi périront ; tu retrancheras tous ceux qui se détournent de toi.
\VS{28}Mais pour moi, m'approcher de Dieu c'est mon bien ; j'ai mis toute mon espérance dans le Seigneur Yahweh, afin de raconter toutes tes œuvres.
\Chap{74}
\TextTitle{Appel au secours du peuple de Dieu}
\VerseOne{}Cantique d'Asaph. Ô Dieu, pourquoi nous as-tu rejetés pour toujours ? Et pourquoi ta colère fume-t-elle contre le troupeau de ton pâturage\FTNT{Ps. 79:5.} ?
\VS{2}Souviens-toi de ton assemblée que tu as acquise autrefois. Tu t'es approprié cette montagne de Sion, sur laquelle tu habitais, afin qu'elle soit la portion de ton héritage.
\VS{3}Elève tes pas vers les lieux constamment dévastés ; l'ennemi a tout renversé dans le lieu saint.
\VS{4}Tes adversaires ont rugi au milieu de ton assemblée ; ils ont mis leurs signes pour signes.
\VS{5}On les a vus pareils à celui qui lève la cognée dans une épaisse forêt.
\VS{6}Et maintenant, avec des haches et des marteaux, ils brisent les sculptures.
\VS{7}Ils ont mis le feu à ton lieu saint. Ils ont abattu à terre et profané la demeure dédiée à ton Nom\FTNT{2 R. 25:9.}.
\VS{8}Ils ont dit en leur cœur : Saccageons-les tous ensemble ! Ils ont brûlé dans le pays tous les lieux saints de Dieu.
\VS{9}Nous ne voyons plus nos signes ; il n'y a plus de prophètes ; et personne parmi nous qui sache jusqu'à quand\FTNT{La. 2:9-10.}.
\VS{10}Ô Dieu ! Jusqu'à quand l'adversaire te couvrira-t-il d'opprobres et l'ennemi méprisera-t-il ton Nom à jamais ?
\VS{11}Pourquoi retires-tu ta main, même ta droite ? Consume-les en la tirant du milieu de ton sein !
\VS{12}Or Dieu est mon Roi dès les temps anciens, faisant des délivrances au milieu de la terre.
\VS{13}Tu as fendu la mer par ta force ; tu as brisé les têtes des serpents sur les eaux.
\VS{14}Tu as brisé les têtes du léviathan, tu l'as donné pour nourriture au peuple du désert.
\VS{15}Tu as ouvert la fontaine et le torrent, tu as desséché les grosses rivières.
\VS{16}A toi est le jour, à toi aussi est la nuit ; tu as établi la lumière et le soleil.
\VS{17}Tu as posé toutes les limites de la terre ; tu as formé l'été et l'hiver.
\VS{18}Souviens-toi de ceci : Que l'ennemi a blasphémé Yahweh et qu'un peuple insensé a outragé ton Nom.
\VS{19}Ne livre pas aux vivants l'âme de la tourterelle, n'oublie pas à toujours la vie de tes affligés.
\VS{20}Regarde à ton alliance, car les lieux ténébreux de la terre sont remplis d'habitations de violence.
\VS{21}Ne permets pas que celui qui est foulé s'en retourne tout confus. Que l'affligé et le pauvre louent ton Nom !
\VS{22}Ô Dieu ! Lève-toi, défends ta cause, souviens-toi de l'opprobre qui t'est fait tous les jours par l'insensé !
\VS{23}N'oublie pas le cri de tes adversaires, le bruit de ceux qui s'élèvent contre toi monte continuellement !
\Chap{75}
\TextTitle{L'élevation vient de Yahweh}
\VerseOne{}Psaume d'Asaph. Cantique donné au chef des chantres, pour le chanter sur Al-Thasheth\FTNT{Voir Ps. 57:1.}.
\VS{2}Nous te célébrons, ô Dieu ! Nous te célébrons et ton Nom est près de nous ; nous racontons tes merveilles.
\VS{3}Au temps que j'aurai fixé, je jugerai avec droiture.
\VS{4}La terre se dissout avec tous ceux qui y habitent, mais j'affermis ses piliers. Sélah.
\VS{5}Je dis aux insensés : N'agissez point follement ; et aux méchants : N'élevez pas la tête.
\VS{6}N'élevez pas si haut votre tête, et ne parlez point avec fierté.
\VS{7}Car l'élévation ne vient point d'orient, ni d'occident ni du désert.
\VS{8}Car c'est Dieu qui gouverne ; il abaisse l'un, et élève l'autre\FTNT{1 S. 2:7.}.
\VS{9}Il y a une coupe dans la main de Yahweh\FTNT{Es. 51:17-22 ; Jé. 25:27-28 ; Ap. 14:10 ; Ap. 16:19.}, et le vin rougit dedans ; il est plein de mélange, et Dieu en verse ; certainement, tous les méchants de la terre en suceront et en boiront jusqu'à la lie.
\VS{10}Mais moi, je raconterai ces choses à jamais, je chanterai au Dieu de Jacob.
\VS{11}J'humilierai tous les méchants, mais les justes seront élevés.
\Chap{76}
\TextTitle{La Puissance du Dieu redoutable}
\VerseOne{}Psaume d'Asaph. Cantique donné au chef des chantres, pour le chanter avec instruments à cordes.
\VS{2}Dieu est connu en Judée, sa renommée est grande en Israël ;
\VS{3}sa tente est à Salem et sa demeure à Sion.
\VS{4}Là il a brisé les arcs étincelants, le bouclier, l'épée et les armes de guerre. Sélah.
\VS{5}Tu es resplendissant, plus magnifique que les montagnes des ravisseurs.
\VS{6}Les plus courageux sont étourdis, ils sont dans un profond assoupissement, et aucun de ces hommes vaillants n'a trouvé ses mains.
\VS{7}Ô Dieu de Jacob, les cavaliers et les chevaux se sont endormis quand tu les as menacés.
\VS{8}Tu es redoutable, toi. Qui peut se tenir devant toi quand ta colère éclate ?
\VS{9}Tu fais entendre des cieux le jugement ; la terre en a eu peur et s'est tenue dans le silence.
\VS{10}Quand tu te lèves, ô Dieu, pour faire jugement, pour délivrer tous les malheureux de la terre ! Sélah.
\VS{11}L'homme te célèbre, même dans sa fureur, quand tu te ceins de toute ta colère.
\VS{12}Faites vos vœux à Yahweh votre Dieu et accomplissez-les ! Que tous ceux qui l'environnent apportent des dons au Dieu terrible !
\VS{13}Il coupe le souffle des princes ; il est redoutable aux rois de la terre.
\Chap{77}
\TextTitle{Se souvenir des prodiges de Yahweh}
\VerseOne{}Psaume d'Asaph, donné au chef des chantres, d'après Jeduthun.
\VS{2}Ma voix s'élève à Dieu, et je crie ; ma voix s'adresse à Dieu, et il m'écoutera.
\VS{3}Je cherche le Seigneur au jour de ma détresse ; sans cesse mes mains s'étendent durant la nuit ; mon âme refuse d'être consolée.
\VS{4}Je me souviens de Dieu, et je gémis ; je médite, et mon esprit est affaibli. Sélah.
\VS{5}Tu empêches mes yeux de dormir ; je suis troublé, et ne peux parler.
\VS{6}Je pense aux jours d'autrefois et aux années des siècles passées\FTNT{Ps. 143:5.}.
\VS{7}Je me souviens de mes chants pendant la nuit, je médite en mon cœur, et mon esprit cherche diligemment.
\VS{8}Le Seigneur m'a-t-il rejeté pour toujours ? Ne me sera-t-il plus favorable ?
\VS{9}Sa bonté est-elle disparue pour toujours ? Sa parole a-t-elle pris fin pour l'éternité ?
\VS{10}Dieu a-t-il oublié d'avoir compassion ? A-t-il dans sa colère retiré sa miséricorde ? Sélah.
\VS{11}Je dis : Ce qui me fait devenir malade, je me souviendrai des années de la droite du Très–Haut.
\VS{12}Je me souviens des exploits de Yahweh ; je me suis, dis-je, souvenu de tes merveilles d'autrefois.
\VS{13}Je méditerai toutes tes œuvres, et je parlerai de tes œuvres.
\VS{14}Ô Dieu ! Tes voies sont saintes. Quel dieu est grand comme Dieu ?
\VS{15}Tu es le Dieu qui fait des merveilles ! Tu as fait connaître ta force parmi les peuples.
\VS{16}Tu as délivré par ton bras ton peuple, les fils de Jacob et de Joseph. Sélah.
\VS{17}Les eaux t'ont vu, ô Dieu ! Les eaux t'ont vu et ont tremblé, même les abîmes en ont été émus.
\VS{18}Les nuées ont versé un déluge d'eau, les nuées ont fait retentir leur son ; tes flèches ont volé de toutes parts.
\VS{19}La voix de ton tonnerre était dans le tourbillon, les éclairs ont éclairé le monde, la terre en a été émue et en a tremblé.
\VS{20}Tu te frayas un chemin par la mer, un sentier par les grosses eaux ; et tes traces ne furent plus reconnues.
\VS{21}Tu as mené ton peuple comme un corps d'armée sous la conduite de Moïse et d'Aaron\FTNT{Mi. 6:4.}.
\Chap{78}
\TextTitle{Les œuvres de Dieu dans l'histoire d'Israël}
\VerseOne{}Cantique d'Asaph. Mon peuple, écoute ma loi, prêtez vos oreilles aux paroles de ma bouche.
\VS{2}J'ouvrirai ma bouche en une parabole ; je proférerai les énigmes cachées des temps anciens\FTNT{Mt. 13:35.}.
\VS{3}Ce que nous avons entendu et connu, et que nos pères nous ont raconté\FTNT{Ps. 44:2.},
\VS{4}nous ne le cacherons point à leurs fils. Ils raconteront à la génération à venir les louanges de Yahweh, sa puissance et ses merveilles qu'il a faites.
\VS{5}Car il a établi le témoignage en Jacob, et il a mis la loi en Israël ; il a donné cet ordre à nos pères de la faire connaître à leurs fils\FTNT{De. 4:9.},
\VS{6}pour qu'elle soit connue de la génération future, des fils qui naîtraient, et pour que lorsqu'ils seront grands, ils la relatent à leurs fils,
\VS{7}afin qu'ils mettent leur confiance en Dieu, et qu'ils n'oublient point les œuvres de Dieu, et qu'ils gardent ses commandements.
\VS{8}Afin qu'ils ne soient point comme leurs pères, une génération revêche et rebelle, une génération insoumise de cœur, dont l'esprit est infidèle à Dieu\FTNT{Ex. 32:9 ; Ac. 7:51.}.
\VS{9}Les fils d'Ephraïm, armés et tirant de l'arc, tournèrent le dos le jour de la bataille.
\VS{10}Ils ne gardèrent point l'alliance de Dieu et refusèrent de marcher selon sa loi.
\VS{11}Ils oublièrent ses œuvres et ses merveilles qu'il leur avait fait voir.
\VS{12}Il avait fait des miracles en présence de leurs pères, dans le pays d'Egypte, dans le champ de Tsoan.
\VS{13}Il fendit la mer et les fit passer au travers ; et il fit arrêter les eaux comme un monceau de pierres.
\VS{14}Il les conduisit de jour par la nuée, et toute la nuit par une lumière de feu\FTNT{Ex. 13:21.}.
\VS{15}Il fendit les rochers au désert, et leur donna à boire d'abondantes eaux, comme s'il eût puisé des abîmes.
\VS{16}Il fit sortir des ruisseaux de la roche\FTNT{Ex. 17:6 ; No. 20:11 ; 1 Co. 10:4.} et fit couler des eaux comme des rivières.
\VS{17}Toutefois, ils continuèrent à pécher contre lui, irritant le Très-Haut dans le désert.
\VS{18}Ils tentèrent Dieu dans leurs cœurs, en demandant de la viande selon leur désir.
\VS{19}Ils parlèrent contre Dieu, disant : Dieu pourrait-il dresser une table dans ce désert\FTNT{No. 11:4.} ?
\VS{20}Voilà, dirent-ils, il a frappé le rocher, et les eaux ont coulé et des torrents ont débordé ; mais pourrait-il aussi nous donner du pain ? Fournirait-il de la viande à son peuple ?
\VS{21}C'est pourquoi, Yahweh les ayant entendus, se mit dans une grande colère, et le feu s'embrasa contre Jacob, et sa colère s'excita contre Israël.
\VS{22}Parce qu'ils n'avaient point cru en Dieu et ne s'étaient point confiés en sa délivrance.
\VS{23}Il ordonna aux nuées d'en haut et il ouvrit les portes des cieux ;
\VS{24}il fit pleuvoir la manne sur eux pour leur nourriture et il leur donna le blé du ciel\FTNT{Ex. 16:14 ; Jn. 6:31.}.
\VS{25}Ils mangèrent tous le pain des grands. Il leur envoya de la viande pour s'en rassasier.
\VS{26}Il excita dans les cieux le vent d'orient et il amena par sa puissance le vent du sud.
\VS{27}Il fit pleuvoir sur eux de la viande comme de la poussière, et comme le sable des mers des oiseaux ailées.
\VS{28}Il les fit tomber au milieu du camp, autour de leurs demeures.
\VS{29}Ils en mangèrent et en furent pleinement rassasiés, car il leur donna selon leur désir.
\VS{30}Mais ils ne furent pas encore dégoûtés de leur désir, et leur viande était encore dans leur bouche
\VS{31}quand la colère de Dieu s'excita contre eux, et qu'il mit à mort les plus gras d'entre eux, et abattit les gens d'élite d'Israël\FTNT{1 Co. 10:5.}.
\VS{32}Malgré cela, ils péchèrent encore et ne crurent point à ses prodiges\FTNT{No. 14:2.}.
\VS{33}C'est pourquoi il consuma leurs jours par la vanité et leurs années par une fin soudaine.
\VS{34}Quand il les mettait à mort, alors ils le recherchaient ; ils se repentaient et ils cherchaient Dieu dès le matin.
\VS{35}Ils se souvenaient que Dieu était leur rocher, et Dieu, le Très-Haut, était leur libérateur.
\VS{36}Mais ils le trompaient de leur bouche et ils lui mentaient de leur langue\FTNT{Es. 29:13 ; Jé. 12:2 ; Mt. 15:8.} ;
\VS{37}car leur cœur n'était point droit envers lui, et ils ne furent point fidèles à son alliance.
\VS{38}Toutefois, comme il est compatissant, il pardonna leur iniquité, au point qu'il ne les détruisit pas ; mais il détourna souvent sa colère et ne réveilla pas toute sa fureur.
\VS{39}Il se souvint qu'ils n'étaient que chair, qu'un vent qui passe et qui ne revient point.
\VS{40}Combien de fois l'ont–ils irrité au désert, et combien de fois l'ont–ils attristé dans ce lieu inhabitable ?
\VS{41}Ils ne cessèrent de tenter Dieu et de provoquer le Saint d'Israël.
\VS{42}Ils ne se souvinrent point de sa puissance, du jour où il les délivra de la main de l'ennemi,
\VS{43}des miracles qu'il accomplit en Egypte, et de ses merveilles dans les champs de Tsoan.
\VS{44}Il changea en sang leurs fleuves et leurs ruisseaux et ils ne purent en boire les eaux\FTNT{Ex. 7:20.}.
\VS{45}Il envoya contre eux des mouches qui les dévorèrent et des grenouilles qui les détruisirent\FTNT{Ex. 8:6-24.}.
\VS{46}Il livra leurs récoltes aux sauterelles, le produit de leur travail aux sauterelles\FTNT{Ex. 10:13.}.
\VS{47}Il détruisit leurs vignes par la grêle, et leurs sycomores par les orages\FTNT{Ex. 9:23.}.
\VS{48}Il livra leur bétail à la grêle, et leurs troupeaux aux foudres étincelantes.
\VS{49}Il envoya sur eux l'ardeur de sa colère, la fureur, la rage et la détresse, un corps d'armée de messagers de malheur.
\VS{50}Il donna libre cours à sa colère, et ne retira point leur âme de la mort ; il livra leur vie à la peste\FTNT{Ex. 9:6.}.
\VS{51}Il frappa tout premier-né en Egypte, les prémices de la vigueur dans les tentes de Cham\FTNT{Ex. 12:29.}.
\VS{52}Il fit partir son peuple comme des brebis, il les mena comme un corps d'armée dans le désert.
\VS{53}Il les conduisit sûrement, et sans qu'ils eussent aucune frayeur, là où la mer couvrit leurs ennemis.
\VS{54}Il les amena vers sa frontière sainte, vers cette montagne que sa droite a acquise\FTNT{Ex. 15:17.}.
\VS{55}Il chassa devant eux les nations, leur distribua le pays en héritage, et fit habiter les tribus d'Israël dans les tentes de ces nations.
\VS{56}Mais ils tentèrent et irritèrent le Dieu Très-Haut, et ne gardèrent point ses préceptes.
\VS{57}Et ils se retirèrent en arrière et furent infidèles comme leurs pères ; ils tournèrent comme un arc trompeur.
\VS{58}Ils le provoquèrent à la colère par leurs hauts lieux, et l'émurent à la jalousie par leurs images taillées\FTNT{De. 32:16-21.}.
\VS{59}Dieu l'entendit et se mit dans une grande colère, et il méprisa fortement Israël.
\VS{60}Il abandonna la demeure de Silo, la tente où il habitait parmi les hommes.
\VS{61}Il livra en captivité sa force et son ornement entre les mains de l'ennemi.
\VS{62}Il livra son peuple à l'épée et se mit dans une grande colère contre son héritage.
\VS{63}Le feu consuma leurs gens d'élite, et leurs vierges ne furent point louées.
\VS{64}Leurs sacrificateurs tombèrent par l'épée, et leurs veuves ne les pleurèrent point.
\VS{65}Puis le Seigneur se réveilla comme un homme qui se serait endormi, et comme un puissant homme qui s'écrie ayant encore le vin dans la tête.
\VS{66}Il frappa ses adversaires par derrière et les mit en opprobre perpétuel.
\VS{67}Mais il dédaigna la tente de Joseph, et ne choisit point la tribu d'Ephraïm.
\VS{68}Mais il choisit la tribu de Juda, la montagne de Sion, celle qu'il aime.
\VS{69}Il bâtit son lieu saint dans les lieux élevés, et l'établit comme la terre qu'il a fondée pour toujours.
\VS{70}Il choisit David, son serviteur, et le prit de la bergerie\FTNT{1 S. 16:11 ; 2 S. 7:8.} ;
\VS{71}il le prit derrière les brebis qui allaitent et l'amena pour paître Jacob, son peuple, et Israël, son héritage.
\VS{72}Aussi il les dirigea selon l'intégrité de son cœur, et les conduisit avec des mains intelligentes.
\Chap{79}
\TextTitle{Appel au jugement de Dieu}
\VerseOne{}Psaume d'Asaph. Ô Dieu ! Les nations sont entrées dans ton héritage ; on a profané ton saint temple, on a mis Jérusalem en monceaux de pierres.
\VS{2}On a livré les cadavres de tes serviteurs pour viande aux oiseaux du ciel, et la chair de tes fidèles aux bêtes de la terre.
\VS{3}On a répandu leur sang comme de l'eau autour de Jérusalem, et il n'y a eu personne pour les enterrer.
\VS{4}Nous sommes un sujet d'opprobre à nos voisins, de moquerie et de risée à ceux qui habitent autour de nous\FTNT{Ps. 44:14 ; Ps. 80:7.}.
\VS{5}Jusqu'à quand, ô Yahweh, t'irriteras-tu sans cesse et ta jalousie s'embrasera-t-elle comme un feu\FTNT{Ps. 89:47.} ?
\VS{6}Répands ta fureur sur les nations qui ne te connaissent point et sur les royaumes qui n'invoquent point ton Nom\FTNT{Jé. 10:25.}.
\VS{7}Car on a dévoré Jacob et on a ravagé ses demeures.
\VS{8}Ne rappelle point devant nous les iniquités passées. Que tes compassions viennent en hâte au-devant de nous, car nous sommes dans une extrême détresse.
\VS{9}Ô Dieu de notre délivrance ! Aide-nous pour l'amour de la gloire de ton Nom, et délivre-nous ! Pardonne-nous nos péchés pour l'amour de ton Nom !
\VS{10}Pourquoi les nations diraient-elles : Où est leur Dieu ? Que la vengeance du sang de tes serviteurs, qui a été répandu, soit manifestée parmi les nations en notre présence.
\VS{11}Que le gémissement des captifs parviennent jusqu'à toi. Par ton bras puissant sauve tes fils, ceux qui vont périr !
\VS{12}Et rends à nos voisins, dans leur sein, sept fois au double l'opprobre qu'ils t'ont fait, ô Yahweh !
\VS{13}Mais nous, ton peuple, et le troupeau de ton pâturage, nous te louerons pour toujours, et de génération en génération nous publierons tes louanges.
\Chap{80}
\TextTitle{Implorer Yahweh}
\VerseOne{}Psaume d'Asaph, donné au chef des chantres, pour le chanter Sosannim-héduth.
\VS{2}Toi qui pais Israël, prête l'oreille ! Toi qui mènes Joseph comme un troupeau, toi qui es assis entre les chérubins\FTNT{2 S. 6:2 ; Es. 37:16 ; Ps. 99:1.}, fais briller ta splendeur !
\VS{3}Réveille ta puissance au-devant d'Ephraïm, de Benjamin et de Manassé ; et viens pour notre délivrance !
\VS{4}Dieu, ramène-nous et fais briller ta face ! Et nous serons délivrés !
\VS{5}Ô Yahweh, Dieu des armées, jusqu'à quand seras-tu irrité contre la prière de ton peuple ?
\VS{6}Tu les nourris de pain de larmes et tu les abreuves de larmes à pleine mesure.
\VS{7}Tu fais de nous un sujet de dispute entre nos voisins, et nos ennemis se moquent de nous.
\VS{8}Ô Dieu des armées, ramène-nous et fais briller ta face ! Et nous serons délivrés.
\VS{9}Tu avais retiré une vigne hors d'Egypte, tu as chassé les nations, et tu l'as plantée\FTNT{Es. 5:1-7 ; Os. 10:1 ; Mt. 20:1 ; Mt. 21:28-33.}.
\VS{10}Tu as préparé une place devant elle, tu lui as fait prendre racine, et elle a rempli la terre.
\VS{11}Les montagnes étaient couvertes de son ombre, et ses rameaux étaient comme de hauts cèdres de Dieu.
\VS{12}Elle étendait ses branches jusqu'à la mer, et ses rejetons jusqu'au fleuve.
\VS{13}Pourquoi as-tu rompu ses clôtures, de sorte que tous les passants sur la route cueillent ses raisins ?
\VS{14}Les sangliers de la forêt l'ont détruite, et toutes les bêtes des champs en font leur pâture.
\VS{15}Ô Dieu des armées, reviens ! Regarde des cieux, vois, et visite cette vigne ;
\VS{16}et le plant que ta droite avait planté, et le fils que tu t'es choisi.
\VS{17}Elle est brûlée par le feu, elle est coupée ; ils périssent devant ta face menaçante.
\VS{18}Que ta main soit sur l'homme de ta droite, sur le fils de l'homme que tu t'es choisi.
\VS{19}Et nous ne nous éloignerons plus de toi. Rends-nous la vie, et nous invoquerons ton Nom.
\VS{20}Ô Yahweh ! Dieu des armées, ramène-nous, fais briller ta face, et nous serons délivrés !
\Chap{81}
\TextTitle{Se débarasser des dieux étrangers}
\VerseOne{}Psaume d'Asaph, donné au chef des chantres, pour le chanter sur la Guitthith.
\VS{2}Chantez avec allégresse à notre Dieu, notre force ! Poussez des cris de joie en l'honneur du Dieu de Jacob.
\VS{3}Sonnez du shofar, prenez le tambour, la harpe mélodieuse et le luth.
\VS{4}Sonnez du shofar à la nouvelle lune, à la pleine lune, au jour de notre fête\FTNT{No. 10:10.}.
\VS{5}Car c'est une loi pour Israël, une ordonnance du Dieu de Jacob.
\VS{6}Il établit un statut à Joseph, lorsqu'il marcha contre le pays d'Egypte, où j'entendis un langage que je ne connaissais pas.
\VS{7}J'ai retiré son épaule du fardeau, et ses mains ont lâché les corbeilles.
\VS{8}Tu as crié dans la détresse, et je t'ai sauvé ; je t'ai répondu dans le lieu caché du tonnerre ; je t'ai éprouvé auprès des eaux de Mériba. Sélah.
\VS{9}Ecoute mon peuple, je te relèverai. Israël, si tu m'écoutais !
\VS{10}Qu'il n'y ait point de dieu étranger au milieu de toi, et ne te prosterne point devant les dieux des étrangers.
\VS{11}Je suis Yahweh, ton Dieu, qui t'ai fait monter hors du pays d'Egypte. Ouvre ta bouche et je la remplirai.
\VS{12}Mais mon peuple n'a point écouté ma voix, et Israël ne m'a point obéi.
\VS{13}C'est pourquoi je les ai abandonnés aux penchants de leur cœur, et ils ont suivi leurs propres conseils\FTNT{Es. 63:17 ; Es. 65:2 ; 2 Pi. 3:3.}.
\VS{14}Ô si mon peuple m'écoutait ! Si Israël marchait dans mes voies !
\VS{15}J'abattrais en un instant leurs ennemis et je tournerais ma main contre leurs adversaires.
\VS{16}Ceux qui haïssent Yahweh le flatteraient, et le bonheur de mon peuple durerait toujours.
\VS{17}Dieu le nourrirait du meilleur froment ; et je le rassasierais du miel du rocher.
\Chap{82}
\TextTitle{Dieu dénonce l'injustice des hommes}
\VerseOne{}Psaume d'Asaph. Dieu se tient dans l'assemblée de Dieu, il juge au milieu des juges.
\VS{2}Jusqu'à quand jugerez-vous injustement et aurez-vous égard à l'apparence de la personne des méchants\FTNT{Ps. 58:2.} ? Sélah.
\VS{3}Faites droit à celui qu'on opprime et à l'orphelin ; faites justice à l'affligé et au pauvre ;
\VS{4}délivrez celui qu'on maltraite et le misérable, retirez-le de la main des méchants.
\VS{5}Ils ne connaissent ni n'entendent rien ; ils marchent dans les ténèbres, tous les fondements de la terre sont ébranlés.
\VS{6}J'ai dit : Vous êtes des dieux\FTNT{Jn. 10:34.}, et vous êtes tous fils du Très-Haut.
\VS{7}Toutefois, vous mourrez comme des hommes, et vous les princes vous tomberez comme les autres.
\VS{8}Ô Dieu ! Lève-toi, juge la terre ; car tu auras en héritage toutes les nations\FTNT{Ps. 2:8 ; Hé. 1:2.}.
\Chap{83}
\TextTitle{Dessein et confusion des ennemis d'Israël}
\VerseOne{}Cantique et psaume d'Asaph.
\VS{2}Ô Dieu ! Ne garde point le silence, ne te tais point, et ne te tiens point en repos, ô Dieu\FTNT{Ps. 35:22.} !
\VS{3}Car voici, tes ennemis s'agitent, et ceux qui te haïssent ont levé la tête.
\VS{4}Ils ont consulté finement en secret contre ton peuple, et ils ont tenu conseil contre ceux qui se sont retirés vers toi pour se cacher\FTNT{Ps. 2:2.}.
\VS{5}Ils disent : Venez et détruisons-les, en sorte qu'ils ne soient plus une nation, et qu'on ne fasse plus mention du nom d'Israël\FTNT{Ce passage fait allusion aux désirs qu'ont certaines nations de voir Israël détruite Mi. 4:11 ; Ap. 11:1-2.}.
\VS{6}Car ils consultent ensemble d'un même esprit ; ils font alliance contre toi.
\VS{7}Les tentes d'Edom et des Ismaélites, des Moabites et des Hagaréniens ;
\VS{8}de Guebal, d'Ammon, d'Amalek, les Philistins avec les habitants de Tyr.
\VS{9}L'Assyrie aussi se joint à eux ; ils ont servi de bras aux fils de Lot. Sélah.
\VS{10}Fais-leur comme tu fis à Madian\FTNT{Jg. 7:15.}, comme à Sisera\FTNT{Jg. 4:15.}, et comme à Jabin, auprès du torrent de Kison !
\VS{11}Ils furent détruits à En-Dor et servirent de fumier à la terre.
\VS{12}Que leurs chefs soient traités comme Oreb et comme Zeeb ; et que tous leurs princes soient comme Zébach et Tsalmunna\FTNT{Jg. 7:25.} ;
\VS{13}parce qu'ils ont dit : Prenons possession des habitations agréables de Dieu.
\VS{14}Mon Dieu ! Rends-les semblables au tourbillon et au chaume chassé par le vent,
\VS{15}comme le feu brûle une forêt, et comme la flamme embrase les montagnes.
\VS{16}Poursuis-les ainsi par ta tempête et épouvante-les par ton tourbillon !
\VS{17}Couvre leurs visages d'ignominie afin qu'on cherche ton Nom, ô Yahweh !
\VS{18}Qu'ils soient honteux et épouvantés à jamais, qu'ils rougissent, et qu'ils périssent ;
\VS{19}afin qu'on sache que toi seul, dont le nom est Yahweh, tu es le Très-Haut sur toute la terre.
\Chap{84}
\TextTitle{Délices pour ceux qui ont Yahweh comme appui}
\VerseOne{}Psaume des fils de Koré, donné au chef des chantres, pour le chanter sur la Guitthith.
\VS{2}Yahweh des armées, que tes demeures sont aimables !
\VS{3}Mon âme soupire et languit après les parvis de Yahweh ; mon cœur et ma chair poussent des cris de joie vers le Dieu vivant.
\VS{4}Le passereau même trouve sa maison, et l'hirondelle son nid où elle a mis ses petits… Tes autels, ô Yahweh des armées ! Mon Roi et mon Dieu !
\VS{5}Heureux ceux qui habitent ta maison et qui te louent sans cesse ! Sélah.
\VS{6}Heureux l'homme dont la force est en toi, ils trouvent dans leur cœur des chemins tout tracés !
\VS{7}Passant par la vallée de Baca, ils la réduisent en fontaine ; la pluie la couvre de bénédictions.
\VS{8}Ils marchent avec force pour se présenter devant Dieu à Sion.
\VS{9}Yahweh Dieu des armées, écoute ma prière, Dieu de Jacob, prête l'oreille. Sélah.
\VS{10}Ô Dieu, notre bouclier, vois et regarde la face de ton oint !
\VS{11}Car mieux vaut un jour dans tes parvis, que mille ailleurs. J'aimerais mieux me tenir à la porte dans la maison de mon Dieu, que de demeurer dans les tentes des méchants.
\VS{12}Car Yahweh Dieu est un soleil et un bouclier\FTNT{Ge. 15:1 ; Ps. 89:19 ; Ps. 144:2.} ; Yahweh donne la grâce et la gloire, et il ne refuse aucun bien à ceux qui marchent dans l'intégrité.
\VS{13}Yahweh des armées, heureux l'homme qui se confie en toi\FTNT{Ps. 2:12.} !
\Chap{85}
\TextTitle{Supplication des rescapés de l'exil}
\VerseOne{}Psaume des fils de Koré, donné au chef des chantres.
\VS{2}Yahweh, tu as été favorable à ta terre, tu as ramené et mis en repos les prisonniers de Jacob.
\VS{3}Tu as pardonné l'iniquité de ton peuple, tu as couvert tous leurs péchés. Sélah.
\VS{4}Tu as retiré toute ta colère, tu es revenu de l'ardeur de ton indignation.
\VS{5}Ô Dieu de notre délivrance, rétablis-nous et fais cesser la colère que tu as contre nous.
\VS{6}Seras-tu irrité à jamais contre nous ? Feras-tu durer ta colère de génération en génération ?
\VS{7}Ne reviendras–tu pas nous rendre la vie\FTNT{Ps. 71:20.}, afin que ton peuple se réjouisse en toi ?
\VS{8}Yahweh, fais-nous voir ta miséricorde et accorde-nous ta délivrance !
\VS{9}J'écouterai ce que dira Dieu, Yahweh ; car il parlera de paix à son peuple et à ses bien-aimés, pourvu que jamais ils ne retournent à leur folie.
\VS{10}Certainement sa délivrance est proche de ceux qui le craignent, la gloire habite dans notre pays.
\VS{11}La bonté et la vérité se rencontrent ; la justice et la paix s'embrassent\FTNT{Hé. 7:2.}.
\VS{12}La vérité germe de la terre et la justice regarde des cieux.
\VS{13}Yahweh aussi donne le bien et notre terre rendra son fruit.
\VS{14}La justice marchera devant lui, et il la mettra partout où il passera.
\Chap{86}
\TextTitle{Coeur disposé à la crainte de Dieu}
\VerseOne{}Prière de David. Yahweh, écoute, réponds-moi, car je suis affligé et misérable.
\VS{2}Garde mon âme, car je suis un de tes bien-aimés ; ô toi mon Dieu, délivre ton serviteur qui se confie en toi !
\VS{3}Seigneur, aie pitié de moi, car je crie à toi tout le jour.
\VS{4}Réjouis l'âme de ton serviteur, car j'élève mon âme à toi, Seigneur.
\VS{5}Yahweh ! Tu es bon et clément, et d'une grande bonté envers tous ceux qui t'invoquent\FTNT{Joë. 2:13.}.
\VS{6}Yahweh, prête l'oreille à ma prière, et sois attentif à la voix de mes supplications.
\VS{7}Je t'invoque au jour de ma détresse, car tu m'exauces\FTNT{Ps. 50:15.}.
\VS{8}Seigneur, nul n'est comme toi parmi les dieux, et rien ne ressemble à tes œuvres\FTNT{De. 3:24 ; Ps. 95:3.}.
\VS{9}Seigneur, toutes les nations que tu as faites viendront et se prosterneront devant toi, et glorifieront ton Nom,
\VS{10}car tu es grand, et tu fais des choses merveilleuses ; tu es Dieu, toi seul.
\VS{11}Yahweh ! Enseigne-moi tes voies et je marcherai dans ta vérité\FTNT{Ps. 25:4 ; Ps. 27:11.} ; lie mon cœur à la crainte de ton Nom.
\VS{12}Seigneur, mon Dieu, je te célébrerai de tout mon cœur, et je glorifierai ton Nom à toujours.
\VS{13}Car ta bonté est grande envers moi, et tu as retiré mon âme du profond scheol.
\VS{14}Ô Dieu ! Des gens orgueilleux se sont élevés contre moi, et un corps d'armée de méchants en veut à ma vie ; ils ne portent pas leurs pensées sur toi.
\VS{15}Mais toi, Seigneur, tu es le Dieu compatissant, miséricordieux, lent à la colère, riche en bonté et en vérité.
\VS{16}Tourne-toi vers moi, et aie pitié de moi ! Donne ta force à ton serviteur, délivre le fils de ta servante !
\VS{17}Accorde–moi un signe de ta faveur, et que ceux qui me haïssent le voient et soient honteux, parce que tu m'aideras, ô Yahweh ! Tu me consoleras !
\Chap{87}
\TextTitle{Sion, la cité de Dieu}
\VerseOne{}Psaume. Cantique des fils de Koré. Elle est fondée sur les montagnes saintes.
\VS{2}Yahweh aime les portes de Sion, plus que toutes les demeures de Jacob.
\VS{3}Ce qui se dit de toi, cité de Dieu, sont des choses glorieuses. Sélah.
\VS{4}Je ferai mention de Rahab et de Babylone parmi ceux qui me connaissent ; voici le pays des philistins, et Tyr avec l'Ethiopie. C'est dans Sion qu'ils sont nés.
\VS{5}Et de Sion il est dit : Un homme y est né ; le Très-Haut lui-même l'établira.
\VS{6}Yahweh compte en inscrivant les peuples : C'est là qu'ils sont nés. Sélah.
\VS{7}Et les chantres, de même que les joueurs de flûte, toutes mes sources sont en toi.
\Chap{88}
\TextTitle{Lamentation dans l'affliction}
\VerseOne{}Cantique. Psaume des fils de Koré, donné au chef des chantres. Pour chanter sur la flûte. Cantique d'Héman, l'Ezrahite.
\VS{2}Yahweh ! Dieu de ma délivrance, je crie jour et nuit devant toi\FTNT{Lu. 18:7.}.
\VS{3}Que ma prière parvienne en ta présence ; étends ton oreille à mon cri.
\VS{4}Car mon âme est rassasiée de maux, et ma vie atteint le scheol\FTNT{Lu. 16:23.}.
\VS{5}On m'a mis au rang de ceux qui descendent dans la fosse\FTNT{Ps. 28:1 ; Ps. 31:13.} ; je suis devenu comme un homme qui n'a plus de vigueur.
\VS{6}Je suis étendu parmi les morts, semblable à ceux qui sont tués et couchés dans la tombe, à ceux dont tu n'as plus le souvenir, et qui sont séparés par ta main.
\VS{7}Tu m'as jeté dans une fosse profonde, dans les ténèbres, dans les abîmes.
\VS{8}Ta fureur se pose sur moi, et tu m'as accablé de tous tes flots. Sélah.
\VS{9}Tu as éloigné de moi ceux de qui j'étais connu, tu m'as mis en abomination devant eux ; je suis enfermé et je ne peux sortir.
\VS{10}Mes yeux se consument dans la souffrance ; Yahweh ! Je crie à toi tout le jour ! J'étends mes mains vers toi\FTNT{Ex. 9:29 ; 1 R. 8:22. ; Job. 17:7} !
\VS{11}Est-ce pour les morts que tu fais des miracles ? Les morts se relèveront-ils pour te célébrer\FTNT{1 Th. 4:16 ; 1 Co. 15:12-13.} ? Sélah.
\VS{12}Parle-t-on de ta bonté dans le sépulcre, de ta fidélité dans le tombeau\FTNT{Ep. 4:9-10 ; 1 Pi. 3:18-20.} ?
\VS{13}Connaîtra-t-on tes merveilles dans les ténèbres et ta justice dans la terre de l'oubli ?
\VS{14}Mais moi, ô Yahweh ! J'implore ton secours, ma prière s'élève dès le matin.
\VS{15}Yahweh ! Pourquoi rejettes-tu mon âme, pourquoi me caches-tu ta face\FTNT{Mt. 27:46 ; Mc. 15:34.} ?
\VS{16}Je suis malheureux et moribond dès ma jeunesse ; j'ai été exposé à tes terreurs, et je ne sais pas où j'en suis.
\VS{17}Les ardeurs de ta colère sont passées sur moi et tes terreurs m'anéantissent\FTNT{Es. 53:5.}.
\VS{18}Elles m'environnent tout le jour comme des eaux, elles m'enveloppent toutes à la fois.
\VS{19}Tu as éloigné de moi mon ami et mon compagnon, mes connaissances ont disparu\FTNT{Mt. 26:56.}.
\Chap{89}
\TextTitle{«Heureux le peuple qui connaît le son de la trompette »}
\VerseOne{}Cantique d'Ethan, l'Ezrachite.
\VS{2}Je chanterai toujours les bontés de Yahweh ; je ferai connaître de ma bouche ta fidélité de génération en génération.
\VS{3}Car je dis : Ta bonté a des fondements éternels, tu établis ta fidélité dans les cieux quand tu dis :
\VS{4}J'ai traité alliance avec mon élu, j'ai fait serment à David mon serviteur :
\VS{5}J'affermirai ta postérité pour toujours, et j'établirai ton trône de génération en génération\FTNT{2 S. 7:8-16.}. Sélah.
\VS{6}Les cieux célèbrent tes merveilles, ô Yahweh ! Ta fidélité aussi est célébrée dans l'assemblée des saints.
\VS{7}Car qui dans le ciel peut se comparer à Yahweh ? Qui est semblable à Yahweh parmi les fils de Dieu ?
\VS{8}Dieu se rend extrêmement terrible dans le conseil secret des saints, il est plus redouté que tous ceux qui sont à l'entour de lui.
\VS{9}Ô Yahweh Dieu des armées ! Qui est semblable à toi, puissant Yahweh ? Aussi ta fidélité t'environne.
\VS{10}Tu domines l'élévation des flots de la mer ; quand ses vagues s'élèvent, tu les calmes\FTNT{Job. 26:12 ; Job. 38:8-12.}.
\VS{11}Tu écrasas Rahab\FTNT{Ce terme hébreu fait référence au nom emblèmatique de l'Egypte, il signifie « largeur », « arrogance ».} comme un homme blessé à mort ; tu dispersas tes ennemis par le bras de ta force.
\VS{12}A toi sont les cieux, à toi aussi est la terre ; tu as fondé le monde, et tout ce qui est en lui.
\VS{13}Tu as créé le nord et le sud ; le Thabor et l'Hermon se réjouissent en ton Nom.
\VS{14}Ton bras est puissant, ta main est forte, ta droite est haut élevée.
\VS{15}La justice et l'équité sont la base de ton trône ; la bonté et la vérité marchent devant ta face.
\VS{16}Heureux le peuple qui connaît le son de la trompette\FTNT{1 Co. 15:52 ; Ap. 10:7.} ! Il marche, ô Yahweh ! A la clarté de ta face.
\VS{17}Il se réjouit chaque jour en ton Nom, et il se glorifie de ta justice.
\VS{18}Parce que tu es la gloire de leur force ; et notre pouvoir est distingué par ta faveur.
\VS{19}Car notre bouclier est Yahweh, et notre Roi est le Saint d'Israël.
\VS{20}Tu as autrefois parlé en vision touchant ton bien-aimé, et tu as dit : J'ai ordonné mon secours en faveur d'un homme vaillant ; j'ai élevé l'élu du milieu du peuple.
\VS{21}J'ai trouvé David mon serviteur, je l'ai oint de ma sainte huile\FTNT{1 S. 16:13 ; Ac. 13:22.} ;
\VS{22}ma main sera ferme avec lui, et mon bras le renforcera.
\VS{23}L'ennemi ne le surprendra point, et l'inique ne l'affligera point ;
\VS{24}mais j'écraserai devant lui ses adversaires, et je détruirai ceux qui le haïssent.
\VS{25}Ma fidélité et ma bonté seront avec lui, et sa gloire sera élevée en mon Nom.
\VS{26}Je mettrai sa main sur la mer, et sa droite sur les fleuves.
\VS{27}Il m'invoquera : Tu es mon Père, mon Dieu, et le rocher\FTNT{Voir commentaire en Es. 8:14.} de ma délivrance.
\VS{28}Aussi je ferai de lui le premier-né\FTNT{Col. 1:15.}, le plus élevé des rois de la terre.
\VS{29}Je lui garderai ma bonté à toujours, et mon alliance lui sera assurée.
\VS{30}Je rendrai éternelle sa postérité, et son trône comme les jours des cieux.
\VS{31}Mais si ses fils abandonnent ma loi, et ne marchent point selon mes ordonnances ;
\VS{32}s'ils violent mes statuts, et qu'ils ne gardent point mes commandements ;
\VS{33}je punirai de la verge leur transgression, et de plaie leur iniquité.
\VS{34}Mais je ne retirerai point de lui ma bonté, et je ne lui trahirai point ma fidélité.
\VS{35}Je ne violerai point mon alliance, et je ne changerai point ce qui est sorti de mes lèvres.
\VS{36}J'ai une fois juré par ma sainteté : Mentirais-je à David\FTNT{Hé. 6:13.} ?
\VS{37}Sa postérité sera à toujours, et son trône sera devant moi comme le soleil.
\VS{38}Il aura une durée éternelle comme la lune ; le témoin qui est dans le ciel est fidèle. Sélah.
\VS{39}Néanmoins, tu l'as rejeté et dédaigné\FTNT{Es. 53:3.} ; tu t'es mis en grande colère contre ton oint.
\VS{40}Tu as rejeté l'alliance faite avec ton serviteur ; tu as souillé sa couronne en la jetant par terre.
\VS{41}Tu as rompu toutes ses murailles ; tu as mis en ruines ses forteresses.
\VS{42}Tous ceux qui passaient par le chemin l'ont pillé ; il a été mis en opprobre à ses voisins.
\VS{43}Tu as élevé la droite de ses adversaires, tu as réjoui tous ses ennemis.
\VS{44}Tu as fait reculer le tranchant de son épée, et tu ne l'as point élevé dans le combat.
\VS{45}Tu as fait cesser sa splendeur, et tu as jeté par terre son trône.
\VS{46}Tu as abrégé les jours de sa jeunesse et l'as couvert de honte. Sélah.
\VS{47}Jusqu'à quand, ô Yahweh ? Te cacheras-tu à jamais ? Ta fureur s'embrasera-t-elle comme un feu ?
\VS{48}Souviens-toi quelle est la durée de ma vie ; pourquoi aurais-tu créé en vain tous les fils des hommes ?
\VS{49}Qui est l'homme qui vivra et ne verra point la mort, et qui garantira son âme de la main du scheol\FTNT{1 Co. 15:54-57.} ? Sélah.
\VS{50}Seigneur, où sont tes bontés premières que tu juras à David dans ta fidélité ?
\VS{51}Seigneur ! Souviens-toi de l'opprobre de tes serviteurs, et comment je porte dans mon sein l'opprobre qui nous a été fait par tous les peuples nombreux.
\VS{52}Souviens-toi des outrages de tes ennemis, ô Yahweh ! Des outrages contre les pas de ton oint.
\VS{53}Béni soit à toujours Yahweh ; amen ! Oui, amen !
\Chap{90}
\TextTitle{Mortalité de l'homme}
\VerseOne{}Prière de Moïse, homme de Dieu\FTNT{De. 33:1.}. Seigneur ! Tu as été pour nous un refuge de génération en génération.
\VS{2}Avant que les montagnes soient nées et que tu aies formé la terre et le monde, d'éternité en éternité, tu es Dieu\FTNT{Ge. 17:1 ; Es. 40:28.}.
\VS{3}Tu fais revenir l'homme à la poussière, et tu dis : Fils des hommes, retournez\FTNT{Ge. 3:19 ; Ec. 12:7.} !
\VS{4}Car mille ans sont à tes yeux comme le jour d'hier qui est passé, et comme une veille de la nuit\FTNT{Ps. 39:5 ; 2 Pi. 3:8.}.
\VS{5}Tu les emportes semblables à un songe qui, le matin, passe comme l'herbe :
\VS{6}Elle fleurit au matin et reverdit ; le soir on la coupe et elle se fane\FTNT{1 Pi. 1:24.}.
\VS{7}Car nous sommes consumés par ta colère et nous sommes troublés par ta fureur.
\VS{8}Tu as mis devant toi nos iniquités, et à la lumière de ta face nos fautes cachées.
\VS{9}Car tous nos jours s'en vont par ta grande colère, et nos années se consument dans un soupir.
\VS{10}Les jours de nos années reviennent à soixante-dix ans, et pour les plus forts, à quatre-vingts ans ; l'orgueil qu'ils en tirent n'est que peine et misère ; car il passe vite, et nous nous envolons.
\VS{11}Qui connaît, selon ta crainte, la force de ton indignation et de ta grande colère ?
\VS{12}Enseigne-nous à compter nos jours, afin que nous puissions avoir un cœur rempli de sagesse.
\VS{13}Yahweh ! Reviens ! Jusqu'à quand ? Sois apaisé envers tes serviteurs.
\VS{14}Rassasie-nous chaque matin de ta bonté, afin que nous nous réjouissions et que nous soyons joyeux tout le long de nos jours.
\VS{15}Réjouis-nous autant de jours que tu nous as affligés, autant d'années que nous avons vu le malheur.
\VS{16}Que ton œuvre se voie sur tes serviteurs, et ta gloire sur leurs fils.
\VS{17}Que la grâce de Yahweh, notre Dieu, soit sur nous, et affermis l'œuvre de nos mains ; oui, affermis l'œuvre de nos mains !
\Chap{91}
\TextTitle{La sécurité et la fidélité de Yahweh}
\VerseOne{}Celui qui demeure sous la couverture\FTNT{Jésus est notre couverture spirituelle.} du Très-Haut, repose à l'ombre du Tout-Puissant.
\VS{2}Je dis à Yahweh : Tu es ma retraite et ma forteresse, tu es mon Dieu en qui je me confie.
\VS{3}Certes, il te délivre du filet de l'oiseleur, de la peste et de ses ravages.
\VS{4}Il te couvrira de ses plumes et tu trouveras un refuge sous ses ailes ; sa fidélité est un bouclier et une cuirasse.
\VS{5}Tu ne craindras ni les terreurs de la nuit, ni la flèche qui vole le jour\FTNT{Pr. 3:23-24.},
\VS{6}ni la peste qui marche dans les ténèbres, ni la destruction qui frappe en plein midi.
\VS{7}Que mille tombent à ton côté et dix mille à ta droite, tu ne seras pas atteint.
\VS{8}De tes yeux tu regarderas et tu verras la rétribution des méchants.
\VS{9}Car tu es mon refuge, ô Yahweh ! Tu fais du Très-Haut ta demeure.
\VS{10}Aucun malheur ne s'approchera de toi, aucun fléau n'approchera de ta tente\FTNT{Ex. 8:18-19 ; Ps. 121:6-8.}.
\VS{11}Car il ordonnera à ses anges de te garder dans toutes tes voies.
\VS{12}Ils te porteront sur les mains, de peur que ton pied ne heurte contre une pierre\FTNT{Mt. 4:5-6 ; Lu. 4:9-11.}.
\VS{13}Tu marcheras sur le lion et sur l'aspic, tu piétineras le lionceau et le dragon.
\VS{14}Puisqu'il m'aime, je le délivrerai ; je le mettrai sur les hauteurs, parce qu'il connaît mon Nom.
\VS{15}Il m'invoquera et je l'exaucerai ; je serai avec lui dans la détresse, je le délivrerai et le glorifierai.
\VS{16}Je le rassasierai de jours et je lui ferai voir ma délivrance.
\Chap{92}
\TextTitle{Proclamer la louange de Dieu}
\VerseOne{}Psaume. Cantique pour le jour du sabbat.
\VS{2}C'est une belle chose que de célébrer Yahweh, et de chanter ton Nom, ô Très-Haut\FTNT{Ps. 147:1.} !
\VS{3}Afin d'annoncer chaque matin ta bonté et ta fidélité toutes les nuits\FTNT{Ps. 59:17 ; Ps. 88:14 ; Ps. 89:2.}.
\VS{4}Sur l'instrument à dix cordes, sur le luth, et par un cantique prémédité sur la harpe.
\VS{5}Car, ô Yahweh ! Tu me réjouis par tes œuvres, je me réjouis des œuvres de tes mains.
\VS{6}Ô Yahweh ! Que tes œuvres sont magnifiques ! Tes pensées sont merveilleusement profondes\FTNT{Es. 55:8-9 ; Job. 5:9.}.
\VS{7}L'homme stupide n'y connaît rien et le fou n'y prend point garde\FTNT{Es. 5:12 ; Ro. 1:21.}.
\VS{8}Les méchants croissent comme l'herbe, et tous les ouvriers d'iniquité fleurissent pour être exterminés éternellement\FTNT{Jé. 12:1-2 ; Mal. 3:15 ; Ps. 37:2 ; Ps. 73:1-20.}.
\VS{9}Mais toi, ô Yahweh ! Tu es élevé à toujours.
\VS{10}Car voici tes ennemis, ô Yahweh ! Car voici, tes ennemis périssent, tous les ouvriers d'iniquité sont dispersés.
\VS{11}Mais tu élèveras ma corne comme celle d'un buffle, je serai oint d'une huile fraîche\FTNT{Ps. 23:5 ; Hé. 1:9.}.
\VS{12}Mes yeux se plaisent à regarder ceux qui m'épient, et mes oreilles à entendre les méchants qui s'élèvent contre moi.
\VS{13}Le juste fleurit comme le palmier, il croît comme le cèdre au Liban.
\VS{14}Etant plantés dans la maison de Yahweh, ils fleurissent dans les parvis de notre Dieu.
\VS{15}Ils portent encore des fruits dans la blanche vieillesse ; ils sont gras et verdoyants\FTNT{Os. 14:6 ; Ps. 1:3.},
\VS{16}afin d'annoncer que Yahweh est droit ; c'est mon rocher, et il n'y a point d'injustice en lui.
\Chap{93}
\TextTitle{Majesté et puissance de Yahweh}
\VerseOne{}Yahweh règne, il est revêtu de majesté ; Yahweh est revêtu de force, il s'en est ceint ; aussi le monde est ferme, tellement qu'il ne sera point ébranlé.
\VS{2}Ton trône est établi dès lors, tu es de toute éternité\FTNT{Ps. 9:8 ; Hé. 1:8.}.
\VS{3}Les fleuves élevés, ô Yahweh ! Les fleuves augmentent leur bruit, les fleuves élèvent leurs flots\FTNT{Ps. 46:4 ; Ps. 65:7-8.} ;
\VS{4}Yahweh, qui est dans les lieux élevés, est plus puissant que le bruit des grandes eaux, et que les fortes vagues de la mer\FTNT{Es. 57:15 ; Ac. 7:49.}.
\VS{5}Tes préceptes sont entièrement fidèles. Yahweh ! La sainteté orne ta maison pour une longue durée.
\Chap{94}
\TextTitle{A Dieu seul la vengeance}
\VerseOne{}Ô Yahweh ! Dieu des vengeances, Dieu des vengeances, fais briller ta splendeur !
\VS{2}Toi, juge de la terre, élève-toi ! Rends aux orgueilleux selon leurs œuvres.
\VS{3}Jusqu'à quand les méchants, ô Yahweh ! Jusqu'à quand les méchants se réjouiront-ils ?
\VS{4}Jusqu'à quand tous les ouvriers d'iniquité discourront-ils et diront-ils des paroles rudes et se vanteront-ils ?
\VS{5}Yahweh, ils écrasent ton peuple et affligent ton héritage.
\VS{6}Ils tuent la veuve et l'étranger, et ils mettent à mort les orphelins.
\VS{7}Ils disent : Yahweh ne le voit point, le Dieu de Jacob n'entend rien.
\VS{8}Vous les plus abrutis d'entre les peuples, prenez garde à ceci ; et vous insensés, quand serez-vous intelligents ?
\VS{9}Celui qui a planté l'oreille, n'entendrait-il point ? Celui qui a formé l'œil, ne verrait-t-il point\FTNT{Ex. 4:11 ; Pr. 20:12.} ?
\VS{10}Celui qui châtie les nations, celui qui enseigne la science aux hommes, ne réprimanderait-il point\FTNT{Ap. 19:15.} ?
\VS{11}Yahweh connaît les pensées des hommes qui ne sont que vanité.
\VS{12}Heureux l'homme que tu châties, ô Yahweh\FTNT{Hé. 12:6.} ! Que tu instruis par ta loi,
\VS{13}afin qu'il soit dans la paix aux jours du malheur, jusqu'à ce que la fosse soit creusée pour le méchant !
\VS{14}Car Yahweh ne délaisse point son peuple et n'abandonne point son héritage\FTNT{Es. 49:15 ; Ro. 11:2.}.
\VS{15}C'est pourquoi le jugement s'unira à la justice, et tous ceux qui sont droits de cœur le suivront.
\VS{16}Qui se lèvera pour moi contre les méchants\FTNT{Job. 19:25 ; Ro. 8:31.} ? Qui m'assistera contre les ouvriers d'iniquité ?
\VS{17}Si Yahweh n'était pas mon secours, mon âme serait bien vite dans la demeure du silence.
\VS{18}Quand je dis : Mon pied chancelle, ta bonté me soutient, ô Yahweh !
\VS{19}Quand j'ai beaucoup de pensées au-dedans de moi, tes consolations font les délices de mon âme.
\VS{20}Serais–tu l'allié du trône de méchanceté, qui forge des injustices contre les règles de la justice ?
\VS{21}Ils se rassemblent contre l'âme du juste et condamnent le sang innocent\FTNT{Mt. 27:1-4 ; Mt. 27:24.}.
\VS{22}Or Yahweh est pour moi une haute retraite ; mon Dieu est le rocher de mon refuge.
\VS{23}Il fera retourner sur eux leur iniquité et les détruira par leur propre méchanceté. Yahweh notre Dieu les détruira\FTNT{Mt. 13:30 ; Ap. 20:14-15.}.
\Chap{95}
\TextTitle{Adoration à Yahweh}
\VerseOne{}Venez, chantons à Yahweh, poussons des cris de réjouissance au rocher de notre salut.
\VS{2}Allons au-devant de lui en lui présentant nos louanges ; et poussons devant lui des cris de réjouissance en chantant des psaumes.
\VS{3}Car Yahweh est un grand Dieu, et il est un grand Roi au-dessus de tous les dieux.
\VS{4}Les lieux les plus profonds de la terre sont dans sa main, et les sommets des montagnes sont à lui.
\VS{5}C'est à lui qu'appartient la mer, car lui-même l'a faite, et ses mains ont formé la terre.
\VS{6}Venez, prosternons-nous, inclinons-nous, et mettons-nous à genoux devant Yahweh qui nous a faits\FTNT{Ps. 96:9 ; Ph. 2:10-11.}.
\VS{7}Car il est notre Dieu, et nous sommes le peuple de son pâturage, et les brebis que sa main conduit\FTNT{Ps. 23:1 ; Ps. 100:3 ; Jn. 10:11.}. Si vous entendez aujourd'hui sa voix,
\VS{8}n'endurcissez point votre cœur\FTNT{Hé. 3:8 ; Hé. 4:7.}, comme à Meriba, comme à la journée de Massa, au désert ;
\VS{9}là où vos pères m'ont tenté et éprouvé bien qu'ils virent mes œuvres\FTNT{Ex. 17:7.}.
\VS{10}J'ai eu cette génération en dégoût durant quarante ans, et j'ai dit : C'est un peuple dont le cœur s'égare ; et ils n'ont point connu mes voies ;
\VS{11}c'est pourquoi j'ai juré dans ma colère, ils n'entreront pas dans mon repos\FTNT{No. 14:22-23 ; Hé. 3:15-19 ; Hé. 4:3.}.
\Chap{96}
\TextTitle{La grandeur et la gloire de Dieu}
\VerseOne{}Chantez à Yahweh un cantique nouveau\FTNT{Es. 42:10 ; Ps. 98:1 ; Ap. 5:9 ; Ap. 14:3.} ! Vous tous habitants de la terre chantez à Yahweh !
\VS{2}Chantez à Yahweh, bénissez son Nom ! Prêchez de jour en jour sa délivrance !
\VS{3}Racontez sa gloire parmi les nations, ses merveilles parmi tous les peuples\FTNT{Ps. 67:5.}.
\VS{4}Car Yahweh est grand et digne d'être loué ; il est redoutable au-dessus de tous les dieux\FTNT{Ph. 2:9 ; Ap. 5:9.} ;
\VS{5}car tous les dieux des peuples ne sont que des idoles, mais Yahweh a fait les cieux.
\VS{6}La splendeur et la magnificence marchent devant lui, la force et la beauté sont dans son lieu saint.
\VS{7}Familles des peuples, rendez à Yahweh, rendez à Yahweh la gloire et la puissance !
\VS{8}Rendez à Yahweh la gloire due à son Nom ! Apportez des offrandes et entrez dans ses parvis !
\VS{9}Prosternez–vous devant Yahweh avec des ornements sacrés ; tremblez devant lui, vous toute la terre !
\VS{10}Dites parmi les nations : Yahweh règne ; même le monde est affermi, il ne sera point ébranlé ; il jugera les peuples avec équité.
\VS{11}Que les cieux se réjouissent et que la terre soit dans l'allégresse ! Que la mer tonne avec tout ce qui la remplit !
\VS{12}Que les champs s'égayent avec tout ce qui est en eux. Alors tous les arbres de la forêt chanteront de joie
\VS{13}devant Yahweh, car il vient ! Car il vient pour juger la terre ; il jugera avec justice le monde habitable et les peuples selon sa fidélité.
\Chap{97}
\TextTitle{Aimer Dieu, c'est haïr le mal}
\VerseOne{}Yahweh règne, que la terre soit dans l'allégresse et que les îles nombreuses s'en réjouissent\FTNT{Es. 42:10 ; Ps. 86:9 ; Ps. 93:1 ; Ps. 99:1} !
\VS{2}La nuée et l'obscurité sont autour de lui ; la justice et le jugement sont la base de son trône.
\VS{3}Le feu marche devant lui et embrase tout autour ses adversaires.
\VS{4}Ses éclairs éclairent le monde, et la terre le voit et tremble tout étonnée\FTNT{Job. 38:35 ; Ap. 4:5.}.
\VS{5}Les montagnes se fondent comme de la cire\FTNT{Mi. 1:4.}, à cause de la présence de Yahweh, à cause de la présence du Seigneur de toute la terre.
\VS{6}Les cieux annoncent sa justice et tous les peuples voient sa gloire.
\VS{7}Que tous ceux qui servent les images et qui se glorifient des idoles soient confus\FTNT{De. 4:25-26 ; 1 S. 5:1-5.} ; vous dieux, prosternez-vous tous devant lui.
\VS{8}Sion l'a entendu et s'en est réjouie ; et les filles de Juda se sont égayées pour l'amour de tes jugements, ô Yahweh !
\VS{9}Yahweh, tu es le Très-Haut sur toute la terre ; tu es élevé au-dessus de tous les dieux.
\VS{10}Vous qui aimez Yahweh, haïssez le mal\FTNT{Am. 5:14-15 ; Ro. 12:9.} ! Il garde les âmes de ses bien-aimés et les délivre de la main des méchants\FTNT{Ps. 34:8 ; Jn. 10:28-29.}.
\VS{11}La lumière est faite pour le juste\FTNT{Mt. 5:15-16.} et la joie pour ceux qui sont droits de cœur.
\VS{12}Justes, réjouissez-vous en Yahweh et célébrez la mémoire de sa sainteté.
\Chap{98}
\TextTitle{Invitation à la louange}
\VerseOne{}Psaume. Chantez à Yahweh un cantique nouveau, car il a fait des choses merveilleuses ; sa droite et le bras de sa sainteté l'ont délivré\FTNT{Es. 52:10 ; Es 53:1 ; Es. 63:3-5.}.
\VS{2}Yahweh a fait connaître son salut\FTNT{Il est question de la révélation de Jésus. Voir commentaire en Es. 26:1.}, il a révélé sa justice devant les yeux des nations.
\VS{3}Il s'est souvenu de sa bonté et de sa fidélité envers la maison d'Israël ; toutes les extrémités de la terre ont vu la délivrance de notre Dieu\FTNT{Es. 49:6 ; Lu. 1:72 ; Ac. 13:47.}.
\VS{4}Vous tous habitants de la terre, poussez des cris de réjouissance à Yahweh ! Faites retentir vos cris et chantez de joie !
\VS{5}Chantez à Yahweh avec la harpe, avec la harpe et avec une voix mélodieuse !
\VS{6}Poussez des cris de réjouissance avec le shofar au son du cor devant le Roi, Yahweh !
\VS{7}Que la mer tonne avec tout ce qu'elle contient, que la terre et ceux qui y habitent fassent éclater leurs cris !
\VS{8}Que les fleuves frappent des mains et que les montagnes chantent de joie
\VS{9}devant Yahweh ! Car il vient pour juger la terre\FTNT{Yahweh, qui vient pour juger la terre, est Jésus-Christ ( 
Za. 14:1-7 ; 2 Ti. 4:1 ; Ap. 19:15).} ; il jugera le monde habitable avec justice et les peuples avec équité.
\Chap{99}
\TextTitle{Grandeur, justice, et sainteté de Dieu}
\VerseOne{}Yahweh règne, que les peuples tremblent ; il est assis entre les chérubins, que la terre soit ébranlée\FTNT{Ex. 25:22 ; Es. 37:16.}.
\VS{2}Yahweh est grand en Sion, et il est élevé au-dessus de tous les peuples.
\VS{3}Ils célébreront ton Nom, grand et terrible, car il est saint.
\VS{4}Qu'on célèbre la force du roi qui aime la justice ! Tu as ordonné l'équité, tu as prononcé des jugements justes en Jacob.
\VS{5}Exaltez Yahweh notre Dieu et prosternez-vous devant son marchepied ! Il est saint !
\VS{6}Moïse et Aaron étaient parmi ses sacrificateurs\FTNT{Ex. 31:10 ; Lé. 2:2.} ; et Samuel parmi ceux qui invoquaient son Nom ; ils invoquaient Yahweh et il leur répondait\FTNT{1 S. 12:18-19.}.
\VS{7}Il leur parlait de la colonne de nuée ; ils ont gardé ses préceptes et l'ordonnance qu'il leur avait donnée.
\VS{8}Ô Yahweh, mon Dieu ! Tu les as exaucés, tu as été pour eux un Dieu qui pardonne\FTNT{Hé. 10:16-17.}, mais tu les as punis de leurs fautes.
\VS{9}Exaltez Yahweh notre Dieu ! Prosternez-vous sur la montagne de sa sainteté ! Car Yahweh, notre Dieu est saint !
\Chap{100}
\TextTitle{Célébrer et bénir le nom de Yahweh}
\VerseOne{}Psaume de louange. Vous tous habitants de la terre, poussez des cris de réjouissance à Yahweh !
\VS{2}Servez Yahweh avec allégresse, venez devant lui avec un chant de joie !
\VS{3}Sachez que Yahweh est Dieu. C'est lui qui nous a faits, ce n'est pas nous qui nous sommes faits ; nous sommes son peuple et le troupeau de son pâturage\FTNT{Ps. 79:13 ; Ps. 80:2 ; Ps. 95:6 ; Ps. 119:73.}.
\VS{4}Entrez dans ses portes avec des louanges ; et dans ses parvis, avec des cantiques. Célébrez-le, bénissez son Nom !
\VS{5}Car Yahweh est bon ; sa bonté demeure à toujours et sa fidélité de génération en génération.
\Chap{101}
\TextTitle{Appel à l'intégrité}
\VerseOne{}Psaume de David. Je chanterai la miséricorde et la justice ; Yahweh ! Je te chanterai.
\VS{2}Je me rendrai attentif à une conduite pure jusqu'à ce que tu viennes à moi ; je marcherai dans l'intégrité de mon cœur au milieu de ma maison.
\VS{3}Je ne mettrai point devant mes yeux des choses de Bélial\FTNT{Ps. 26:5 ; Ps. 119:115.}; j'ai en haine les actions de ceux qui se détournent ; elles ne s'attacheront pas à moi.
\VS{4}Le cœur mauvais s'éloignera de moi ; je ne connaîtrai pas le méchant.
\VS{5}Je retrancherai celui qui calomnie en secret son prochain ; je ne supporterai pas celui qui a les yeux élevés et le cœur enflé\FTNT{Pr. 6:16-17.}.
\VS{6}Je prendrai garde aux gens de bien du pays afin qu'ils demeurent avec moi ; celui qui marche dans la voie de l'intégrité me servira.
\VS{7}Celui qui usera de tromperie ne demeurera point dans ma maison ; celui qui profèrera des mensonges ne sera point affermi devant mes yeux.
\VS{8}Je retrancherai chaque matin tous les méchants du pays, afin d'exterminer de la cité de Yahweh tous les ouvriers d'iniquité.
\Chap{102}
\TextTitle{Yahweh, le Dieu immuable}
\VerseOne{}Prière de l'affligé étant dans l'angoisse et répandant sa plainte devant Yahweh.
\VS{2}Yahweh ! Ecoute ma prière, et que mon cri parvienne jusqu'à toi\FTNT{Ps. 69:14.}.
\VS{3}Ne cache pas ta face arrière de moi ; au jour où je suis en détresse, prête l'oreille à ma prière ; au jour où je t'invoque, hâte-toi de me répondre.
\VS{4}Car mes jours se sont évanouis comme la fumée et mes os brûlent comme dans un foyer.
\VS{5}Mon cœur est frappé et se dessèche comme l'herbe, car j'ai oublié de manger mon pain\FTNT{Mt. 4:4 ; Lu. 4:4.}.
\VS{6}Le gémissement de ma voix est tel que mes os s'attachent à ma chair\FTNT{Job. 19:20.}.
\VS{7}Je suis devenu semblable au pélican du désert ; et je suis comme la chouette des lieux sauvages.
\VS{8}Je veille et je suis semblable au passereau solitaire sur le toit.
\VS{9}Mes ennemis m'outragent tous les jours, et ceux qui sont furieux contre moi jurent contre moi.
\VS{10}Car j'ai mangé la cendre comme le pain et j'ai mêlé des larmes à ma boisson,
\VS{11}à cause de ta colère et de ta fureur ; car après m'avoir soulevé, tu m'as jeté par terre.
\VS{12}Mes jours sont comme l'ombre qui décline et je deviens sec comme l'herbe.
\VS{13}Mais toi, ô Yahweh ! Tu demeures éternellement, et ta mémoire est de génération en génération.
\VS{14}Tu te lèveras, tu auras compassion de Sion ; car il est temps d'en avoir pitié, parce que le temps assigné est échu.
\VS{15}Car tes serviteurs aiment ses pierres et chérissent sa poussière.
\VS{16}Alors les nations redouteront le Nom de Yahweh, et tous les rois de la terre ta gloire.
\VS{17}Quand Yahweh aura édifié Sion, quand il aura été vu dans sa gloire,
\VS{18}quand il aura eu égard à la prière du désolé et qu'il n'aura point méprisé leur supplication.
\VS{19}Cela sera enregistré pour la génération à venir, le peuple qui sera créé louera Yahweh !
\VS{20}Car il regarde du lieu élevé de sa sainteté. Du haut des cieux, Yahweh regarde la terre,
\VS{21}pour entendre le gémissement des prisonniers, pour délier ceux qui étaient voués à la mort\FTNT{Es. 42:6-7 ; Es. 61:1 ; Lu. 4:18-19.},
\VS{22}afin qu'on annonce le Nom de Yahweh dans Sion et sa louange dans Jérusalem,
\VS{23}quand les peuples se seront joints ensemble, et les royaumes aussi, pour servir Yahweh.
\VS{24}Il a brisé ma force en chemin, il a abrégé mes jours.
\VS{25}Je dis : Mon Dieu, ne m'enlève point au milieu de mes jours dont les années durent éternellement.
\VS{26}Tu as jadis fondé la terre, et les cieux sont l'ouvrage de tes mains.
\VS{27}Ils périront, mais tu subsisteras, ils s'useront tous comme un vêtement ; tu les changeras comme un habit, et ils seront changés.
\VS{28}Mais toi, tu es toujours le même et tes années ne seront jamais achevées.
\VS{29}Les fils de tes serviteurs habiteront près de toi et leur postérité sera établie devant toi.
\Chap{103}
\TextTitle{Yahweh, le Dieu miséricordieux et compatissant}
\VerseOne{}Psaume de David. Mon âme, bénis Yahweh, et que tout ce qui est en moi bénisse son saint Nom.
\VS{2}Mon âme, bénis Yahweh, et n'oublie pas un de ses bienfaits\FTNT{De. 6:12.}.
\VS{3}C'est lui qui pardonne toutes tes iniquités, qui guérit toutes tes infirmités\FTNT{Es. 33:24 ; Es. 53:5 ; Jé. 17:14 ; Ps. 130:3-4 ; Mt. 9:6 ; Lu. 7:47.} ;
\VS{4}qui garantit ta vie de la fosse\FTNT{Es. 59:20 ; Ps. 106:10.}, qui te couronne de bonté et de compassions ;
\VS{5}qui rassasie ta bouche de biens ; ta jeunesse est renouvelée comme celle de l'aigle\FTNT{Es. 40:31.}.
\VS{6}Yahweh fait justice et droit à tous les opprimés\FTNT{Ps. 146:7.}.
\VS{7}Il a fait connaître ses voies à Moïse, et ses exploits aux fils d'Israël\FTNT{Ex. 33:12-17.}.
\VS{8}Yahweh est compatissant, miséricordieux, lent à la colère, et riche en bonté.
\VS{9}Il ne conteste pas éternellement, et il ne garde point à toujours sa colère\FTNT{Es. 57:16 ; Jé. 3:5 ; Mi. 7:18.}.
\VS{10}Il ne nous traite pas selon nos péchés, et ne nous rend point selon nos iniquités\FTNT{Esd. 9:13.}.
\VS{11}Car autant les cieux sont élevés au-dessus de la terre, autant sa bonté est grande sur ceux qui le craignent.
\VS{12}Il éloigne de nous nos transgressions, autant que l'orient est éloigné de l'occident\FTNT{Es. 38:17.}.
\VS{13}Comme un père a compassion de ses fils, Yahweh a compassion de ceux qui le craignent\FTNT{Mal. 3:17 ; Lu. 11:11-13.}.
\VS{14}Car il sait bien de quoi nous sommes faits, se souvenant que nous ne sommes que poussière.
\VS{15}L'homme ! Ses jours sont comme l'herbe, il fleurit comme la fleur d'un champ.
\VS{16}Car le vent étant passé par-dessus, elle n'est plus, et son lieu ne la reconnaît plus.
\VS{17}Mais la miséricorde de Yahweh est de tout temps, et elle sera pour toujours en faveur de ceux qui le craignent ; et sa justice en faveur des fils de leurs fils ;
\VS{18}pour ceux qui gardent son alliance, et qui se souviennent de ses commandements pour les faire\FTNT{De. 7:9.}.
\VS{19}Yahweh a établi son trône dans les cieux, et son règne domine sur tout.
\VS{20}Bénissez Yahweh, vous ses anges puissants en force, qui faites ses affaires, en obéissant à la voix de sa parole.
\VS{21}Bénissez Yahweh, vous toutes ses armées, qui êtes ses serviteurs faisant sa volonté.
\VS{22}Bénissez Yahweh, vous toutes ses œuvres, par tous les lieux de sa domination. Mon âme, bénis Yahweh !
\Chap{104}
\TextTitle{Yahweh, le Dieu de toute la création}
\VerseOne{}Mon âme, bénis Yahweh. Ô Yahweh mon Dieu, tu es merveilleusement grand, tu es revêtu de majesté et de splendeur.
\VS{2}Il s'enveloppe de lumière comme d'un vêtement, il étend les cieux comme un voile\FTNT{Es. 40:22 ; Job. 9:8 ; 1 Ti. 6:16.}.
\VS{3}Avec les eaux, il va à la rencontre de sa demeure ; il fait des grosses nuées son char, il se promène sur les ailes du vent\FTNT{Es. 19:1 ; Ps. 18:10 ; Ap. 14:14.}.
\VS{4}Il fait des vents ses messagers, et des flammes de feu ses serviteurs\FTNT{Ps. 148:8 ; Hé. 1:7 ; Jn. 3:8.}.
\VS{5}Il a fondé la terre sur ses bases, elle ne sera jamais ébranlée\FTNT{Ps. 24:1-2 ; Ps. 78:69 ; Ps. 93:1 ; Job. 26:7 ; Job. 38:4-6.}.
\VS{6}Tu l'avais couverte de l'abîme comme d'un vêtement, les eaux se tenaient sur les montagnes\FTNT{Ge. 1:2.}.
\VS{7}Elles s'enfuirent à ta menace et se mirent promptement en fuite au son de ton tonnerre.
\VS{8}Les montagnes s'élevèrent et les vallées s'abaissèrent au même lieu que tu leur avais fixé.
\VS{9}Tu as posé une limite que les eaux ne doivent point franchir, afin qu'elles ne reviennent plus couvrir la terre\FTNT{Ge. 1:9 ; Jé. 5:2 ; Pr. 8:29 ; Job. 26:10.}.
\VS{10}C'est lui qui conduit les sources par les vallées, elles se promènent entre les monts.
\VS{11}Elles abreuvent toutes les bêtes des champs, les ânes sauvages y étanchent leur soif.
\VS{12}Les oiseaux des cieux se tiennent auprès d'elles, et font résonner leur voix parmi les rameaux.
\VS{13}Il abreuve les montagnes de ses chambres hautes ; la terre est rassasiée du fruit de tes œuvres.
\VS{14}Il fait germer l'herbe pour le bétail, et les plantes pour le besoin de l'homme, faisant sortir le pain de la terre,
\VS{15}et le vin qui réjouit le cœur de l'homme\FTNT{Jg. 9:11 ; Pr. 31:6-7.}, qui fait resplendir son visage avec l'huile, et qui soutient le cœur de l'homme avec le pain.
\VS{16}Les hauts arbres de Yahweh en sont rassasiés, ainsi que les cèdres du Liban qu'il a plantés,
\VS{17}afin que les oiseaux y fassent leurs nids. Quant à la cigogne, les sapins sont sa demeure.
\VS{18}Les hautes montagnes sont pour les chamois, et les rochers sont la retraite des lapins.
\VS{19}Il a fait la lune pour les saisons, et le soleil sait quand il doit se coucher\FTNT{Ge. 1:16.}.
\VS{20}Tu amènes les ténèbres, et il fait nuit ; alors toutes les bêtes de la forêt sont en mouvement.
\VS{21}Les lionceaux rugissent après la proie pour demander à Dieu leur nourriture.
\VS{22}Le soleil se lève-t-il ? Ils se retirent et se couchent dans leurs tanières.
\VS{23}Alors l'homme sort pour se rendre à son ouvrage, et à son travail jusqu'au soir.
\VS{24}Ô Yahweh, que tes œuvres sont en grand nombre ! Tu les as toutes faites avec sagesse ; la terre est pleine de tes richesses.
\VS{25}Cette mer grande et spacieuse, là où des animaux sans nombre se meuvent, des petites bêtes avec des grandes !
\VS{26}Là se promènent les navires, et ce léviathan que tu as formé pour jouer dans les flots.
\VS{27}Ils s'attendent tous à toi afin que tu leur donnes la nourriture en leur temps.
\VS{28}Quand tu la leur donnes, ils la recueillent, et quand tu ouvres ta main, ils sont rassasiés de biens.
\VS{29}Caches-tu ta face ? Ils sont troublés ; retires-tu leur souffle ? Ils défaillent et retournent dans leur poussière.
\VS{30}Tu envoies ton souffle, ils sont créés ; et tu renouvelles la face de la terre.
\VS{31}Que la gloire de Yahweh subsiste à toujours, que Yahweh se réjouisse dans ses œuvres !
\VS{32}Il jette son regard sur la terre et elle tremble ; il touche les montagnes et elles fument.
\VS{33}Je chanterai à Yahweh durant ma vie ; je chanterai à mon Dieu tant que j'existerai.
\VS{34}Ma méditation lui sera agréable, et je me réjouirai en Yahweh.
\VS{35}Que les pécheurs soient consumés de dessus la terre et qu'il n'y ait plus de méchants ! Mon âme, bénis Yahweh ! Louez Yahweh !
\Chap{105}
\TextTitle{Yahweh, le Dieu fidèle}
\VerseOne{}Célébrez Yahweh, invoquez son Nom, faites connaître parmi les peuples ses exploits.
\VS{2}Chantez-le, chantez-le, parlez de toutes ses merveilles !
\VS{3}Glorifiez-vous de son saint Nom, et que le cœur de ceux qui cherchent Yahweh se réjouisse.
\VS{4}Recherchez Yahweh et sa puissance ; cherchez continuellement sa face.
\VS{5}Souvenez-vous de ses merveilles qu'il a faites, de ses miracles, et des jugements de sa bouche.
\VS{6}La postérité d'Abraham sont ses serviteurs ; les enfants de Jacob sont ses élus !
\VS{7}Il est Yahweh notre Dieu, ses jugements sont sur toute la terre.
\VS{8}Il s'est souvenu pour toujours de son alliance, de la parole qu'il a ordonnée pour mille générations,
\VS{9}du traité qu'il a fait avec Abraham et du serment qu'il a fait à Isaac\FTNT{Ge. 17:2 ; Ge. 22:16 ; Ge. 26:3 ; Ge. 28:13 ; Ge. 33:11 ; Lu. 1:73.}.
\VS{10}Il l'a érigé pour être une ordonnance à Jacob, et à Israël pour être une alliance éternelle,
\VS{11}en disant : Je te donnerai le pays de Canaan, comme héritage qui vous est échu\FTNT{Ge. 13:15 ; Ge. 15:18.}.
\VS{12}Ils étaient alors un petit nombre de gens, très peu nombreux, et étrangers dans le pays.
\VS{13}Car ils allaient de nation en nation, et d'un royaume vers un autre peuple.
\VS{14}Il ne permit à personne de les opprimer, il châtia des rois à cause d'eux\FTNT{Ge. 35:5.},
\VS{15}disant : Ne touchez point à mes oints et ne faites point de mal à mes prophètes\FTNT{1 Ch. 16:22.} !
\VS{16}Il appela aussi la famine sur la terre, rompit le bâton du pain\FTNT{Lé. 26:26 ; Es. 3:1 ; Ez. 4:16.}.
\VS{17}Il envoya un homme devant eux ; Joseph fut vendu pour esclave\FTNT{Ge. 37:28-36.}.
\VS{18}On serra ses pieds dans des ceps, sa personne fut mise aux fers.
\VS{19}Jusqu'au temps où arriva ce qu'il avait annoncé, et où la parole de Yahweh l'éprouva.
\VS{20}Le roi le relâcha et le laissa aller ; le dominateur des peuples le délivra.
\VS{21}Il l'établit pour maître sur sa maison, et pour gouverneur sur tout son domaine\FTNT{Ge. 41:40.} ;
\VS{22}pour soumettre les princes à ses désirs, et pour instruire ses anciens.
\VS{23}Puis Israël entra en Egypte, et Jacob séjourna dans le pays de Cham\FTNT{Ge. 46:6 ; Ps. 78:51.}.
\VS{24}Yahweh rendit son peuple très fécond et le rendit plus puissant que ceux qui l'opprimaient.
\VS{25}Il changea leur cœur, au point qu'ils haïrent son peuple jusqu'à conspirer contre ses serviteurs\FTNT{Ex. 1:7-12.}.
\VS{26}Il envoya Moïse son serviteur, et Aaron, qu'il avait élu\FTNT{Ex. 4:14.}.
\VS{27}Ils accomplirent au milieu d'eux des prodiges et des miracles qu'ils avaient eu la charge de faire dans le pays de Cham.
\VS{28}Il envoya les ténèbres et fit venir l'obscurité ; et ils ne furent point rebelles à sa parole.
\VS{29}Il changea leurs eaux en sang et fit mourir leurs poissons.
\VS{30}Leur terre produisit en abondance des grenouilles jusque dans les chambres de leurs rois.
\VS{31}Il dit, et des mouches vinrent, des poux sur tout leur pays.
\VS{32}Il leur donna pour pluie de la grêle, et un feu flamboyant sur la terre.
\VS{33}Il frappa leurs vignes et leurs figuiers, et il brisa les arbres du pays.
\VS{34}Il ordonna et les sauterelles vinrent, des jeunes sauterelles sans nombre
\VS{35}qui dévorèrent toute l'herbe du pays, et qui dévorèrent le fruit de leur terroir.
\VS{36}Il frappa tous les premiers-nés du pays, les prémices de toute leur vigueur\FTNT{Lire Ex. 7 à 12.}.
\VS{37}Puis il les fit sortir avec de l'or et de l'argent, et nul ne chancela parmi ses tribus.
\VS{38}Les Egyptiens se réjouirent à leur départ, car la peur qu'ils avaient d'eux les avait saisis.
\VS{39}Il étendit la nuée pour couverture, et le feu pour éclairer la nuit.
\VS{40}Le peuple demanda et il fit venir des cailles ; et il les rassasia du pain des cieux\FTNT{Ex. 16:12-13.}.
\VS{41}Il ouvrit le rocher et les eaux en coulèrent ; elles se répandirent comme un fleuve dans les lieux arides\FTNT{Ex. 17:6.}.
\VS{42}Car il se souvint de sa parole sainte qu'il avait donnée à Abraham son serviteur\FTNT{Ge. 15:13-16.}.
\VS{43}Il fit sortir son peuple dans l'allégresse, ses élus au milieu des cris retentissants\FTNT{Ex. 15:1.}.
\VS{44}Il leur donna les terres des nations et ils possédèrent le fruit du travail des peuples,
\VS{45}afin qu'ils gardent ses statuts et qu'ils observent ses lois. Louez Yahweh !
\Chap{106}
\TextTitle{L'infidélité d'Israël}
\VerseOne{}Louez Yahweh ! Célébrez Yahweh car il est bon, car sa bonté demeure à toujours !
\VS{2}Qui pourrait réciter les exploits de Yahweh ? Qui pourrait faire retentir toute sa louange ?
\VS{3}Heureux ceux qui observent la justice, qui font en tout temps ce qui est juste !
\VS{4}Yahweh, souviens-toi de moi selon la bienveillance que tu portes à ton peuple, aie soin de moi selon ta délivrance !
\VS{5}Afin que je voie le bien de tes élus, que je me réjouisse dans la joie de ta nation, que je me glorifie avec ton héritage.
\VS{6}Nous avons péché avec nos pères, nous avons agi dans l'iniquité, nous avons fait le mal\FTNT{Da. 9:16 ; Esd. 9:7 ; Né. 1:6.}.
\VS{7}Nos pères n'ont point été attentifs à tes merveilles en Egypte ; ils ne se sont point souvenus de la multitude de tes faveurs ; mais ils furent rebelles près de la mer, vers la Mer Rouge\FTNT{Ex. 14:11.}.
\VS{8}Toutefois, il les délivra pour l'amour de son Nom, afin de faire connaître sa puissance.
\VS{9}Il menaça la Mer Rouge et elle se sécha ; et il les conduisit à travers les profondeurs de la mer comme un désert ;
\VS{10}il les délivra de la main de ceux qui les haïssaient et les racheta de la main de l'ennemi.
\VS{11}Les eaux couvrirent leurs oppresseurs, il n'en resta pas un seul\FTNT{Ex. 14:27.}.
\VS{12}Alors ils crurent à ses paroles et ils chantèrent sa louange.
\VS{13}Mais ils oublièrent vite ses œuvres et ne s'attendirent point à son conseil.
\VS{14}Ils furent épris de convoitise au désert et ils tentèrent Dieu dans le désert.
\VS{15}Alors il leur donna ce qu'ils avaient demandé, toutefois il leur envoya le dépérissement dans leur corps.
\VS{16}Ils jalousèrent dans le camp Moïse et Aaron, le saint de Yahweh.
\VS{17}La terre s'ouvrit et engloutit Dathan ; et recouvrit de terre Abiram\FTNT{No. 16.}.
\VS{18}Le feu s'alluma au milieu de leur assemblée, la flamme brûla les méchants.
\VS{19}Ils firent un veau en Horeb, et se prosternèrent devant une image de métal fondu\FTNT{Ex. 32.}.
\VS{20}Ils changèrent leur gloire contre la figure d'un bœuf qui mange l'herbe.
\VS{21}Ils oublièrent Dieu, leur libérateur, qui avait fait de grandes choses en Egypte,
\VS{22}des choses merveilleuses dans le pays de Cham, et des prodiges sur la Mer Rouge.
\VS{23}C'est pourquoi il dit qu'il les détruirait ; mais Moïse, son élu, se tint à la brèche devant lui pour détourner sa fureur, afin qu'il ne les détruisît point\FTNT{Ex. 32:11.}.
\VS{24}Ils méprisèrent le pays désirable et ne crurent point à sa parole.
\VS{25}Ils murmurèrent dans leurs tentes et n'obéirent point à la voix de Yahweh.
\VS{26}C'est pourquoi il leur jura la main levée de les faire tomber dans le désert,
\VS{27}d'accabler leur postérité parmi les nations, et de les disperser au milieu des pays\FTNT{No. 14:22.}.
\VS{28}Ils se joignirent aux adorateurs de Baal-Peor et mangèrent des victimes sacrifiées aux morts.
\VS{29}Ils irritèrent Dieu par leurs actions, au point qu'une plaie fit une brèche parmi eux.
\VS{30}Mais Phinées se présenta et fit justice ; et la plaie fut arrêtée.
\VS{31}Cela lui fut imputé à justice de génération en génération, pour toujours\FTNT{No. 25:3-8.}.
\VS{32}Ils excitèrent aussi sa colère près des eaux de Meriba, et Moïse fut puni à cause d'eux.
\VS{33}Car ils aigrirent son esprit et il parla avec légèreté de ses lèvres\FTNT{No. 20:12.}.
\VS{34}Ils ne détruisirent point les peuples que Yahweh leur avait dit de détruire,
\VS{35}mais ils se mêlèrent parmi ces nations et apprirent leurs manières de faire.
\VS{36}Ils servirent leurs faux dieux qui furent un piège pour eux.
\VS{37}Car ils sacrifièrent leurs fils et leurs filles aux démons\FTNT{Lé. 18:21 ; De. 12:31 ; 2 R. 16:3 ; Ez. 20:26.}.
\VS{38}Ils répandirent le sang innocent, le sang de leurs fils et de leurs filles, ils sacrifièrent aux faux dieux de Canaan ; et le pays fut souillé de sang\FTNT{No. 35:33.}.
\VS{39}Ils se souillèrent par leurs œuvres et se prostituèrent par leurs actions.
\VS{40}C'est pourquoi la colère de Yahweh s'embrasa contre son peuple et il eut en abomination son héritage.
\VS{41}Il les livra entre les mains des nations, et ceux qui les haïssaient dominèrent sur eux.
\VS{42}Leurs ennemis les opprimèrent et ils furent humiliés sous leur main.
\VS{43}Il les délivra souvent, mais ils se montrèrent rebelles dans leurs desseins et furent humiliés par leur iniquité.
\VS{44}Toutefois, il vit leur détresse lorsqu'il entendit leurs supplications.
\VS{45}Il se souvint en leur faveur de son alliance et se repentit selon la grandeur de ses compassions.
\VS{46}Il fit que ceux qui les avaient emmenés captifs eurent pitié d'eux.
\VS{47}Yahweh notre Dieu, délivre-nous et rassemble-nous du milieu des nations ! Afin que nous célébrions ton saint Nom et que nous mettions notre gloire à te louer !
\VS{48}Béni soit Yahweh, le Dieu d'Israël, d'éternité en éternité ! Et que tout le peuple dise amen ! Louez Yahweh.
\Chap{107}
\TextTitle{La grâce de Yahweh pour ses rachetés}
\VerseOne{}Célébrez Yahweh car il est bon, parce que sa bonté demeure à toujours.
\VS{2}Qu'ainsi disent les rachetés de Yahweh, ceux qu'il a rachetés de la main de l'oppresseur,
\VS{3}et qu'il a rassemblés de tous les pays, de l'orient et de l'occident, du nord et de la mer.
\VS{4}Ils erraient dans le désert, ils marchaient dans la solitude, sans trouver une ville où ils puissent habiter.
\VS{5}Ils étaient affamés et assoiffés, leur âme était languissante.
\VS{6}Alors ils crièrent vers Yahweh dans leur détresse et il les délivra de leurs angoisses ;
\VS{7}il les conduisit sur le droit chemin pour aller dans une ville habitée.
\VS{8}Qu'ils célèbrent Yahweh pour sa bonté et ses merveilles envers les fils des hommes !
\VS{9}Parce qu'il a désaltéré l'âme altérée et rassasié de ses biens l'âme affamée\FTNT{Ps. 146:7 ; Lu. 1:53.}.
\VS{10}Ceux qui avaient pour demeure les ténèbres et l'ombre de la mort, vivaient captifs dans l'affliction et dans les chaînes,
\VS{11}parce qu'ils furent rebelles aux paroles de Dieu, et parce qu'ils avaient rejeté le conseil du Très-Haut\FTNT{De. 31:20 ; La. 3:42.}.
\VS{12}Il humilia leur cœur par la souffrance, ils furent abattus ; et personne ne les secourut.
\VS{13}Alors ils crièrent vers Yahweh dans leur détresse, et il les délivra de leurs angoisses.
\VS{14}Il les fit sortir hors des ténèbres et de l'ombre de la mort ; et il rompit leurs liens\FTNT{Ps. 68:19 ; Ep. 4:8 ; Col. 1:12-13.}.
\VS{15}Qu'ils célèbrent Yahweh pour sa bonté et ses merveilles envers les fils des hommes !
\VS{16}Parce qu'il a brisé les portes d'airain et cassé les barreaux de fer.
\VS{17}Les insensés sont affligés à cause de leurs transgressions et à cause de leurs iniquités.
\VS{18}Leur âme avait en horreur toute nourriture, et ils touchaient aux portes de la mort.
\VS{19}Alors ils crièrent vers Yahweh dans leur détresse, et il les délivra de leurs angoisses\FTNT{Ps. 50:15 ; Os. 5:15.}.
\VS{20}Il envoya sa parole et les guérit ; et il les délivra de leurs tombeaux.
\VS{21}Qu'ils célèbrent Yahweh pour sa bonté et ses merveilles envers les fils des hommes !
\VS{22}Qu'ils offrent des sacrifices de remerciements, et qu'ils racontent ses œuvres avec des cris de joie.
\VS{23}Ceux qui descendaient sur la mer dans des navires, faisant commerce sur les grandes eaux,
\VS{24}ceux-là virent les œuvres de Yahweh et ses merveilles dans les lieux profonds,
\VS{25}car il dit, et il fit paraître la tempête qui souleva les vagues de la mer.
\VS{26}Ils montaient vers les cieux, ils descendaient dans l'abîme ; leur âme se fondait d'angoisse.
\VS{27}Saisis de vertiges, ils chancelaient comme un homme ivre ; et toute leur sagesse était anéantie\FTNT{Es. 51:17-21 ; Jé. 13:13.}.
\VS{28}Alors ils crièrent vers Yahweh dans leur détresse, et il les tira hors de leurs angoisses.
\VS{29}Il arrêta la tempête, la changeant en calme, et les ondes se turent.
\VS{30}Puis ils se réjouirent de ce qu'elles s'étaient apaisées, et il les conduisit au port qu'ils désiraient.
\VS{31}Qu'ils célèbrent Yahweh pour sa bonté et ses merveilles envers les fils des hommes !
\VS{32}Et qu'ils l'exaltent dans l'assemblée du peuple et le louent dans l'assemblée des anciens.
\VS{33}Il réduit les fleuves en désert, et les sources d'eaux en sécheresse ;
\VS{34}la terre fertile en terre salée, à cause de la méchanceté de ses habitants\FTNT{Jé. 12:4 ; Jé. 17:6.}.
\VS{35}Il transforme le désert en étangs d'eaux, et la terre sèche en des sources d'eaux\FTNT{Es. 41:18.} ;
\VS{36}il y établit ceux qui sont affamés, ils bâtissent des villes pour l'habiter.
\VS{37}Ils ensemencent des champs et plantent des vignes qui rendent du fruit tous les ans.
\VS{38}Il les bénit et ils se multiplient extrêmement ; et il ne laisse point diminuer leur bétail.
\VS{39}Puis ils sont amoindris et humiliés par l'oppression, le malheur et la souffrance.
\VS{40}Il répand le mépris sur les princes et les fait errer dans des lieux déserts sans chemin.
\VS{41}Mais il relève le pauvre et le délivre de la misère, il établit les familles comme des troupeaux\FTNT{1 S. 2:8 ; Ps. 113:7.}.
\VS{42}Les hommes droits le voient et se réjouissent, mais toute iniquité a la bouche fermée.
\VS{43}Quiconque est sage prendra garde à ces choses, afin qu'on considère les bontés de Yahweh.
\Chap{108}
\TextTitle{Yahweh, le secours}
\VerseOne{}Cantique. Psaume de David. Mon cœur est affermi, ô Dieu ! Je chante et je joue de mes instruments, c'est ma gloire !
\VS{2}Réveillez-vous, mon luth et ma harpe ! Je me réveillerai à l'aube du jour.
\VS{3}Yahweh, je te célébrerai parmi les peuples et je te chanterai parmi les nations.
\VS{4}Car ta bonté est grande par-dessus les cieux, et ta vérité atteint jusqu'aux nues.
\VS{5}Ô Dieu ! Elève-toi sur les cieux, et que ta gloire soit sur toute la terre !
\VS{6}Afin que ceux que tu aimes soient délivrés ; sauve-moi par ta droite et exauce-moi !
\VS{7}Dieu a dit dans sa sainteté : Je me réjouirai, je partagerai Sichem et mesurerai la vallée de Succoth.
\VS{8}Galaad sera à moi, Manassé sera à moi, et Ephraïm sera le sommet de ma forteresse, Juda mon législateur.
\VS{9}Moab sera le bassin où je me laverai, je jetterai mon soulier sur Edom, je triompherai des Philistins. 
\VS{10}Qui me conduira dans la ville forte ? Qui me conduira jusqu'en Edom ?
\VS{11}N'est-ce pas toi, ô Dieu, qui nous avais rejetés, et qui ne sortais plus, ô Dieu, avec nos armées ?
\VS{12}Donne–nous du secours pour sortir de la détresse ! Car la délivrance qu'on attend de l'homme est vaine.
\VS{13}Avec Dieu, nous ferons des exploits ; il foulera nos ennemis\FTNT{Ps. 60:5-14.}.
\Chap{109}
\TextTitle{La méchanceté de l'homme}
\VerseOne{}Psaume de David ; Donné au chef des chantres\FTNT{Les Psaumes d'imprécations (Ps. 35 ; 52 ; 55 ; 58, 59 ; 79 ; 109 ; 137) sont des demandes faites à Dieu pour qu'il punisse les méchants. Le Seigneur Jésus-Christ nous demande aujourd'hui de bénir nos ennemis (Lu. 6:27-37).}. Dieu de ma louange, ne te tais point !
\VS{2}Car la bouche du méchant et la bouche remplie de fraude se sont ouvertes contre moi ; ils parlent contre moi avec une langue mensongère !
\VS{3}Ils m'entourent de paroles pleines de haine et ils me font la guerre sans cause !
\VS{4}Tandis que je les aime, ils sont mes ennemis ; mais moi, je n'ai fait que prier en leur faveur !
\VS{5}Ils me rendent le mal pour le bien, et la haine pour l'amour que je leur porte.
\VS{6}Etablis le méchant sur lui, et que Satan se tienne à sa droite !
\VS{7}Quand il sera jugé, fais qu'il soit déclaré méchant, et que sa prière soit regardée comme un crime !
\VS{8}Que sa vie soit courte\FTNT{Il est question ici du suicide de Judas (Mt. 27:3-5).} et qu'un autre prenne sa charge\FTNT{Ce passage parle de Judas (Ac. 1:20).} !
\VS{9}Que ses enfants soient orphelins et sa femme veuve !
\VS{10}Que ses enfants soient entièrement vagabonds, et qu'ils mendient et quêtent en sortant de leurs maisons détruites\FTNT{Job. 20:10.} !
\VS{11}Que le créancier usant d'exaction attrape tout ce qui est à lui et que les étrangers butinent tout son travail !
\VS{12}Qu'il n'y ait personne qui étende sa compassion sur lui, et qu'il n'y ait personne qui ait pitié de ses orphelins !
\VS{13}Que sa postérité soit exposée à être retranchée ; que leur nom soit effacé dans la génération qui le suivra !
\VS{14}Que l'iniquité de ses pères revienne en mémoire à Yahweh, et que le péché de sa mère ne soit point effacé !
\VS{15}Qu'ils soient continuellement devant Yahweh, et qu'il retranche leur mémoire de la terre\FTNT{Ps. 34:17.},
\VS{16}parce qu'il ne s'est point souvenu d'user de miséricorde, mais il a persécuté l'homme affligé et misérable, dont le cœur est brisé, et cela pour le faire mourir !
\VS{17}Puisqu'il aime la malédiction, que la malédiction tombe sur lui ! Puisqu'il ne prend pas plaisir à la bénédiction, que la bénédiction aussi s'éloigne de lui !
\VS{18}Et qu'il soit revêtu de la malédiction comme de sa robe ; qu'elle entre dans son corps comme de l'eau, et dans ses os comme de l'huile !
\VS{19}Qu'elle lui soit comme un vêtement dont il se couvre, et comme une ceinture dont il se ceigne continuellement !
\VS{20}Telle soit, de la part de Yahweh, la récompense de mes adversaires, et de ceux qui parlent mal de moi !
\VS{21}Mais toi, Yahweh, Seigneur, agis avec moi pour l'amour de ton Nom ! Et parce que ta miséricorde est grande, délivre-moi !
\VS{22}Car je suis affligé et misérable, et mon cœur est blessé au-dedans de moi.
\VS{23}Je m'en vais comme l'ombre quand elle décline, et je suis chassé comme une sauterelle.
\VS{24}Mes genoux sont affaiblis par le jeûne, et mon corps est épuisé de maigreur au lieu d'être gras.
\VS{25}Je suis pour eux un objet d'opprobre ; quand ils me voient, ils secouent la tête.
\VS{26}Yahweh, mon Dieu ! Aide-moi, délivre-moi selon ta miséricorde.
\VS{27}Afin qu'on sache que c'est ta main, que c'est toi, ô Yahweh, qui l'as fait.
\VS{28}Ils maudiront, mais tu béniras ; ils s'élèveront, mais ils seront confus ; et ton serviteur se réjouira.
\VS{29}Que mes adversaires soient revêtus de confusion et couverts de leur honte comme d'un manteau.
\VS{30}Je célébrerai hautement de ma bouche Yahweh, et je le louerai au milieu de plusieurs nations.
\VS{31}De ce qu'il se tient à la droite du misérable pour le délivrer de ceux qui condamnent son âme.
\Chap{110}
\TextTitle{Yahweh, le Roi et le Sacrificateur}
\VerseOne{}Psaume de David. Yahweh a dit à mon Seigneur : Assieds-toi à ma droite, jusqu'à ce que je fasse de tes ennemis le marchepied de tes pieds\FTNT{Ce psaume affirme la divinité de Jésus-Christ (Mt. 22:41-46 ; Mc. 12:35-37 ; Lu. 20:41-44 ; Ac. 2:34-35 ; Hé. 1:13 ; Hé. 10:12-13).}.
\VS{2}Yahweh étendra de Sion le sceptre de ta puissance, en disant : Domine au milieu de tes ennemis\FTNT{Es. 2:2-3 ; Da. 7:14.} !
\VS{3}Ton peuple est plein d'ardeur quand tu rassembles ton armée ; avec des ornements sacrés, du sein de l'aurore, ta jeunesse vient à toi comme une rosée.
\VS{4}Yahweh l'a juré, et il ne s'en repentira point que tu es sacrificateur éternellement, à la manière de Melchisédek\FTNT{Hé. 5:6 ; Hé. 6:20 ; Hé. 7:17 ; Ge. 14:18.}.
\VS{5}Le Seigneur est à ta droite, il brisera les rois au jour de sa colère.
\VS{6}Il exercera le jugement sur les nations, il remplira tout de cadavres ; il brisera le chef qui domine sur un grand pays\FTNT{Ap. 14 ; Ap. 16.}.
\VS{7}Il boit au torrent pendant la marche : C'est pourquoi il lève haut la tête.
\Chap{111}
\TextTitle{Les oeuvres magnifiques de Dieu}
\VerseOne{}Louez Yahweh. [Aleph.] Je célébrerai Yahweh de tout mon cœur, [Beth.] dans la compagnie des hommes droits et dans l'assemblée.
\VS{2}[Guimel.] Les œuvres de Yahweh sont grandes, [Daleth.] elles sont recherchées par tous ceux qui y prennent plaisir.
\VS{3}[He.] Son œuvre n'est que majesté et magnificence, [Vav.] et sa justice demeure à perpétuité.
\VS{4}[Zayin.] Il a rendu ses merveilles mémorables. [Heth.] Yahweh est miséricordieux et compatissant.
\VS{5}[Teth.] Il a donné de la nourriture à ceux qui le craignent ; [Yod.] il s'est souvenu pour toujours de son alliance.
\VS{6}[Kaf.] Il a manifesté à son peuple la puissance de ses œuvres, [Lamed.] en leur donnant l'héritage des nations.
\VS{7}[Mem.] Les œuvres de ses mains ne sont que vérité et équité. [Nun.] Tous ses commandements sont véritables,
\VS{8}[Samech.] appuyés à perpétuité, éternellement, [Ayin.] faits avec fidélité et droiture.
\VS{9}[Pe.] Il a envoyé la rédemption à son peuple\FTNT{Ex. 6:6 ; Jn. 3:16.} ; [Tsade.] il lui a donné une alliance éternelle ; [Qof.] son nom est saint et redoutable.
\VS{10}[Resh.] Le commencement de la sagesse c'est la crainte de Yahweh : [Shin.] Tous ceux qui s'adonnent à faire ce qu'elle prescrit sont sages\FTNT{Pr. 1:7 ; Pr. 9:10 ; Pr. 8:13 ; De. 4:6.}. [Tav.] Sa louange demeure à perpétuité.
\Chap{112}
\TextTitle{La crainte de Yahweh enrichit et donne de l'assurance}
\VerseOne{}Louez Yahweh. [Aleph.] Heureux l'homme qui craint Yahweh [Beth.] et qui prend un grand plaisir à ses commandements !
\VS{2}[Guimel.] Sa postérité sera puissante sur la terre, [Daleth.] la génération des hommes droits sera bénie\FTNT{Pr. 20:7.}.
\VS{3}[He.] Il y aura des biens et des richesses dans sa maison ; [Vav.] et sa justice demeure à perpétuité.
\VS{4}[Zayin.] La lumière s'est levée dans les ténèbres sur ceux qui sont justes\FTNT{Pr. 4:18 ; Ps. 37:6.} ; [Heth.] il est compatissant, miséricordieux et juste.
\VS{5}[Teth.] Heureux l'homme de bien qui exerce la miséricorde et prête, [Yod.] qui règle ses actions avec justice.
\VS{6}[Kaf.] Il ne chancelle jamais. [Lamed.] La mémoire du juste dure toujours\FTNT{Pr. 10:7.}.
\VS{7}[Mem.] Il ne craint pas les mauvaises nouvelles ; [Nun.] son cœur est ferme, confiant en Yahweh.
\VS{8}[Samech.] Son cœur est bien affermi, il ne craint pas, [Ayin.] jusqu'à ce qu'il mette son plaisir à regarder ses adversaires.
\VS{9}[Pe.] Il fait des largesses, il donne aux pauvres ; [Tsade.] sa justice demeure à perpétuité ; [Qof.] sa corne s'élève en gloire.
\VS{10}[Resh.] Le méchant le voit et s'irrite. [Shin.] Il grince des dents et se consume ; [Tav.] les désirs des méchants périssent.
\Chap{113}
\TextTitle{Yahweh, le Dieu élevé au-dessus de tout}
\VerseOne{}Louez Yahweh\FTNT{Les psaumes disant « Alléluia » sont les Ps. 104 à 106, 111 à 113, 115 à 117, 135 à 136, 146 à 150. Parmi eux, les psaumes 135 et 146 à 150, étaient chantés durant le service quotidien d'adoration dans la synagogue. Les psaumes 115 à 118, appelés « le grand Hallel », étaient chantés lors des fêtes de Pâque. Alléluia veut dire « Louez Yahweh » (Ap. 19:1).} ! Louez, vous serviteurs de Yahweh, louez le Nom de Yahweh !
\VS{2}Que le Nom de Yahweh soit béni dès maintenant et à toujours !
\VS{3}Le Nom de Yahweh est digne de louanges depuis le soleil levant jusqu'au soleil couchant.
\VS{4}Yahweh est élevé par-dessus toutes les nations, sa gloire est au-dessus des cieux.
\VS{5}Qui est semblable à Yahweh notre Dieu, qui habite dans les lieux très hauts ?
\VS{6}Il s'abaisse pour regarder sur le ciel et sur la terre,
\VS{7}Il relève l'affligé de la poussière, et retire le pauvre\FTNT{1 S. 2:8 ; Ps. 107:41.} de dessus le fumier,
\VS{8}pour le faire asseoir avec les nobles, avec les nobles de son peuple\FTNT{Job. 36:7.}.
\VS{9}Il donne une maison à la femme stérile, il en fait une mère joyeuse au milieu de ses fils\FTNT{Ge. 17:17-21 ; 1 S. 2:5 ; Ps. 68:6.}. Louez Yahweh !
\Chap{114}
\TextTitle{La création tremble devant le Tout-Puissant}
\VerseOne{}Quand Israël sortit d'Egypte, quand la maison de Jacob s'éloigna d'un peuple barbare,
\VS{2}Juda devint son lieu saint, Israël son domaine\FTNT{Jé. 2:2-3.}.
\VS{3}La mer le vit et s'enfuit, le Jourdain retourna en arrière\FTNT{Jos. 3:13-16 ; Ps. 77:17.}.
\VS{4}Les montagnes sautèrent comme des béliers, les collines comme des agneaux\FTNT{Jg. 5:5 ; Ha. 3:10 ; Ps. 68:9.}.
\VS{5}Ô Mer ! Qu'avais-tu pour t'enfuir ? Jourdain, pour retourner en arrière ?
\VS{6}Et vous montagnes, pour sauter comme des béliers ? Et vous collines, comme des agneaux ?
\VS{7}Ô Terre ! Tremble devant la présence du Seigneur, devant la présence du Dieu de Jacob,
\VS{8}qui a changé le rocher en un étang d'eaux, la pierre très dure en une source d'eaux.
\Chap{115}
\TextTitle{Louange au Dieu de gloire}
\VerseOne{}Non point à nous, ô Yahweh ! Non point à nous, mais à ton Nom donne gloire, à cause de ta bonté, à cause de ta fidélité !
\VS{2}Pourquoi les nations diraient-elles : Où est maintenant leur Dieu ?
\VS{3}Certes notre Dieu est au ciel, il fait tout ce qu'il veut\FTNT{Ps. 135:6 ; Job. 23:13.}.
\VS{4}Leurs idoles sont des dieux d'or et d'argent, elles sont l'ouvrage de mains d'homme.
\VS{5}Elles ont une bouche, et ne parlent point ; elles ont des yeux, et ne voient point ;
\VS{6}elles ont des oreilles, et n'entendent point ; elles ont un nez, et ne sentent point ;
\VS{7}elles ont des mains, et elles ne touchent point ; elles ont des pieds, et elles ne marchent point ; et elles ne rendent aucun son de leur gosier\FTNT{Ex. 32:2-8 ; 1 R. 18:25-26 ; Es. 44:9 ; Ez. 8:8-12.}.
\VS{8}Ils leur ressemblent, ceux qui les fabriquent, tous ceux qui se confient en eux.
\VS{9}Israël confie-toi en Yahweh ; il est leur secours et leur bouclier de ceux qui se confient en lui.
\VS{10}Maison d'Aaron, confie-toi en Yahweh ; il est leur secours et leur bouclier.
\VS{11}Vous qui craignez Yahweh, confiez-vous en Yahweh ; il est leur secours et leur bouclier.
\VS{12}Yahweh s'est souvenu de nous, il bénira, il bénira la maison d'Israël, il bénira la maison d'Aaron.
\VS{13}Il bénira ceux qui craignent Yahweh, tant les petits que les grands.
\VS{14}Yahweh vous multipliera ses bénédictions, à vous et à vos fils.
\VS{15}Vous êtes bénis de Yahweh, qui a fait les cieux et la terre.
\VS{16}Les cieux, sont les cieux de Yahweh, mais il a donné la terre aux fils des hommes.
\VS{17}Ce ne sont pas les morts qui célèbrent Yahweh, ce n'est aucun de ceux qui descendent dans le lieu du silence\FTNT{Ps. 88:11 ; Es. 38:18-19 ; Ps. 6:6.}.
\VS{18}Mais nous, nous bénirons Yahweh dès maintenant et pour toujours. Louez Yahweh !
\Chap{116}
\TextTitle{Psaume des rachetés}
\VerseOne{}J'aime Yahweh, car il a entendu ma voix et mes supplications.
\VS{2}Car il a incliné son oreille vers moi, c'est pourquoi je l'invoquerai durant mes jours.
\VS{3}Les liens de la mort m'avaient environné, et les angoisses du scheol m'avaient trouvé\FTNT{2 S. 22:5 ; Ps. 18:5.} ; j'avais trouvé la détresse et la douleur.
\VS{4}Mais j'invoquai le Nom de Yahweh en disant : Je te prie, délivre mon âme, ô Yahweh !
\VS{5}Yahweh est compatissant et juste, et notre Dieu fait miséricorde.
\VS{6}Yahweh garde les simples ; j'étais devenu misérable et il m'a sauvé.
\VS{7}Mon âme, retourne dans ton repos, car Yahweh t'a fait du bien.
\VS{8}Parce que tu as retiré mon âme de la mort, mes yeux des larmes et mes pieds de la chute,
\VS{9}je marcherai dans la présence de Yahweh, sur la terre des vivants.
\VS{10}J'ai cru, c'est pourquoi j'ai parlé\FTNT{2 Co. 4:13.} ; j'ai été fort affligé.
\VS{11}Je disais dans ma précipitation : Tout homme est menteur\FTNT{Ro. 3:4.}.
\VS{12}Que rendrai-je à Yahweh ? Tous ses bienfaits envers moi ?
\VS{13}J'élèverai la coupe des délivrances et j'invoquerai le Nom de Yahweh.
\VS{14}J'accomplirai maintenant mes vœux à Yahweh, devant tout son peuple.
\VS{15}Elle a du prix aux yeux de Yahweh, la mort de ceux qu'il aime.
\VS{16}Ecoute-moi, ô Yahweh ! Car je suis ton serviteur, je suis ton serviteur, fils de ta servante. Tu as délié mes liens.
\VS{17}Je t'offrirai le sacrifice de remerciement et j'invoquerai le Nom de Yahweh.
\VS{18}J'accomplirai maintenant mes vœux à Yahweh, devant tout son peuple,
\VS{19}dans les parvis de la maison de Yahweh, au milieu de toi, Jérusalem ! Louez Yahweh !
\Chap{117}
\TextTitle{Toutes les nations louent Yahweh}
\VerseOne{}Toutes les nations, louez Yahweh ! Tous les peuples, célébrez-le !
\VS{2}Car sa miséricorde est grande envers nous, et sa fidélité dure à toujours. Louez Yahweh !
\Chap{118}
\TextTitle{Yahweh, le Dieu de mon secours}
\VerseOne{}Célébrez Yahweh, car il est bon, parce que sa bonté dure à toujours !
\VS{2}Qu'Israël dise maintenant : Car sa bonté dure à toujours !
\VS{3}Que la maison d'Aaron dise maintenant : Car sa bonté dure à toujours !
\VS{4}Que ceux qui craignent Yahweh disent maintenant : Car sa bonté dure à toujours !
\VS{5}Me trouvant dans la détresse, j'ai invoqué Yahweh\FTNT{Ps. 120:1.} ; et Yahweh m'a répondu et m'a mis au large.
\VS{6}Yahweh est pour moi, je ne craindrai point. Que me ferait l'homme ?
\VS{7}Yahweh est pour moi parmi ceux qui me secourent, c'est pourquoi je verrai en ceux qui me haïssent ce que je désire.
\VS{8}Mieux vaut se confier en Yahweh que se confier en l'homme\FTNT{Es. 2:22 ; Jé. 17:5 ; Ps. 62:9.}.
\VS{9}Mieux vaut se confier en Yahweh que se reposer sur les grands d'entre les peuples.
\VS{10}Toutes les nations m'avaient environné, mais au Nom de Yahweh je les taille en pièces.
\VS{11}Elles m'avaient environné, elles m'avaient, dis-je, environné ; mais au Nom de Yahweh je les taille en pièces.
\VS{12}Elles m'avaient environné comme des abeilles, elles s'éteignent comme un feu d'épines\FTNT{De. 1:44.}, car au Nom de Yahweh je les taille en pièces.
\VS{13}Tu me poussais violemment pour me faire tomber, mais Yahweh m'a secouru.
\VS{14}Yahweh est ma force et le sujet de mes louanges, et il a été ma délivrance\FTNT{Ex. 15:2 ; Es. 12:2.}.
\VS{15}Une voix de chant de triomphe et de délivrance retentit dans les tentes des justes : La droite de Yahweh exerce la puissance !
\VS{16}La droite de Yahweh est élevée, la droite de Yahweh exerce sa puissance.
\VS{17}Je ne mourrai pas, je vivrai et je raconterai les œuvres de Yahweh.
\VS{18}Yahweh m'a châtié sévèrement, mais il ne m'a point livré à la mort.
\VS{19}Ouvrez-moi les portes de la justice ; j'y entrerai et je célébrerai Yahweh.
\VS{20}C'est ici la porte de Yahweh, les justes y entreront.
\VS{21}Je te célébrerai parce que tu m'as exaucé et tu as été mon libérateur.
\VS{22}La Pierre que les architectes avaient rejetée, est devenue la principale de l'angle\FTNT{Le Messie est présenté comme la pierre ou le rocher. (cp. Es. 8:13-17 ; 1 Pi. 2:7).}.
\VS{23}Ceci a été fait par Yahweh, c'est un prodige à nos yeux.
\VS{24}C'est ici la journée que Yahweh a faite, qu'elle soit pour nous un sujet d'allégresse et de joie.
\VS{25}Yahweh, je te prie, délivre maintenant. Yahweh, je te prie, donne maintenant la prospérité !
\VS{26}Béni soit celui qui vient au Nom de Yahweh ! Nous vous bénissons de la maison de Yahweh.
\VS{27}Yahweh est Dieu, et il nous a éclairés. Liez avec des cordes la bête du sacrifice, et amenez-la jusqu'aux cornes de l'autel.
\VS{28}Tu es mon Dieu, c'est pourquoi je te célébrerai. Tu es mon Dieu, je t'exalterai.
\VS{29}Célébrez Yahweh car il est bon, parce que sa miséricorde demeure à toujours !
\Chap{119}
\TextTitle{La Parole de Yahweh éclaire}
\VerseOne{}[Aleph.] Heureux ceux qui sont intègres dans leur voie, qui marchent selon la loi de Yahweh.
\VS{2}Heureux sont ceux qui gardent ses préceptes et qui le cherchent de tout leur cœur\FTNT{Jos. 1:8.} ;
\VS{3}qui ne font point d'iniquité, qui marchent dans ses voies\FTNT{1 Jn. 3:9 ; 1 Jn. 5:18.}.
\VS{4}Tu as donné tes commandements afin qu'on les garde soigneusement.
\VS{5}Oh ! Que mes voies soient bien établies pour garder tes statuts !
\VS{6}Et je ne rougirai point de honte quand je regarderai à tous tes commandements.
\VS{7}Je te célébrerai avec droiture de cœur quand j'aurai appris les ordonnances de ta justice.
\VS{8}Je veux garder tes statuts, ne me délaisse point entièrement.
\VS{9}[Beth.] Par quel moyen le jeune homme rendra-t-il pure sa voie ? Ce sera en y prenant garde selon ta parole.
\VS{10}Je te recherche de tout mon cœur, ne me laisse pas m'égarer loin de tes commandements.
\VS{11}Je serre ta parole dans mon cœur afin de ne pas pécher contre toi.
\VS{12}Yahweh ! Tu es béni ; enseigne-moi tes statuts.
\VS{13}De mes lèvres je raconte toutes les ordonnances de ta bouche.
\VS{14}Je me réjouis dans le chemin de tes préceptes comme si je possédais toutes les richesses du monde.
\VS{15}Je médite tes commandements et j'observe tes voies.
\VS{16}Je prends plaisir à tes statuts et je n'oublie pas tes paroles.
\VS{17}[Guimel.] Fais du bien à ton serviteur afin que je vive, et je garderai ta parole\FTNT{Ps. 116:7.}.
\VS{18}Ouvre mes yeux afin que je regarde aux merveilles de ta loi\FTNT{Ep. 1:18.} !
\VS{19}Je suis voyageur sur la terre, ne me cache pas tes commandements.
\VS{20}Mon âme est brisée par le désir qui toujours me porte vers tes ordonnances.
\VS{21}Tu réprimandes les orgueilleux, ces maudits, qui se détournent de tes commandements.
\VS{22}Décharge-moi de l'opprobre et du mépris, car j'ai gardé tes préceptes\FTNT{Ps. 3:9.}.
\VS{23}Même les princes s'assoient et parlent contre moi pendant que ton serviteur médite tes statuts.
\VS{24}Tes préceptes font mes délices, ce sont mes conseillers.
\VS{25}[Daleth.] Mon âme est attachée à la poussière, fais-moi revivre selon ta parole\FTNT{Ps. 44:26 ; Ps. 143:11.}.
\VS{26}Je te raconte mes voies et tu me réponds ; enseigne-moi tes statuts.
\VS{27}Fais-moi entendre la voie de tes commandements, et je parlerai de tes merveilles\FTNT{Ps. 145:6.}.
\VS{28}Mon âme pleure de chagrin, relève-moi selon tes paroles.
\VS{29}Eloigne de moi la voie du mensonge et accorde–moi la grâce d'observer ta loi\FTNT{Le mot « loi » vient de l'hebreu « towrah » qui donne « Thora » en français}.
\VS{30}Je choisis la voie de la vérité et je place tes ordonnances sous mes yeux.
\VS{31}Je m'attache à tes préceptes, ô Yahweh ! Ne me fais point rougir de honte.
\VS{32}Je cours dans la voie de tes commandements car tu élargis mon cœur.
\VS{33}[He.] Yahweh, enseigne-moi la voie de tes statuts, et je la garderai jusqu'au bout.
\VS{34}Donne-moi de l'intelligence ; je garderai ta loi et je l'observerai de tout mon cœur\FTNT{Pr. 2:6 ; Ja. 1:5.}.
\VS{35}Fais-moi marcher sur le sentier de tes commandements car j'y prends plaisir.
\VS{36}Incline mon cœur à tes préceptes et non point au profit\FTNT{Ez. 33:31 ; Mc 7:21-22 ; Hé. 13:5.}.
\VS{37}Détourne mes yeux de la vue des choses vaines ; fais-moi vivre dans ta voie.
\VS{38}Accomplis ta parole envers ton serviteur, parole qui est pour ceux qui te craignent.
\VS{39}Eloigne de moi l'opprobre que je redoute, car tes ordonnances sont bonnes.
\VS{40}Voici, je désire pratiquer tes commandements, fais-moi vivre dans ta justice.
\VS{41}[Vav.] Que ta miséricorde vienne sur moi, ô Yahweh ! Et ta délivrance aussi, selon ta promesse !
\VS{42}Et je pourrai répondre à celui qui m'outrage, car je me confie en ta parole.
\VS{43}N'arrache pas de ma bouche la parole de vérité, car j'espère en tes jugements.
\VS{44}Je garderai continuellement ta loi, à toujours et à perpétuité.
\VS{45}Je marcherai au large parce que je recherche tes commandements.
\VS{46}Je parlerai de tes préceptes devant les rois et je ne rougirai pas de honte\FTNT{Ps. 138:1-4 ; Mt. 10:18-19 ; Ac. 26.}.
\VS{47}Je fais mes délices de tes commandements que j'aime ;
\VS{48}j'étends mes mains vers tes commandements que j'aime ; et je médite tes statuts.
\VS{49}[Zayin.] Souviens-toi de la parole donnée à ton serviteur, sur laquelle tu m'as fait espérer.
\VS{50}C'est ici ma consolation dans mon affliction, car ta parole me rend la vie.
\VS{51}Les orgueilleux se sont fort moqués de moi, mais je ne me suis pas détourné de ta loi.
\VS{52}Yahweh, je me souviens de tes jugements anciens et je me suis consolé en eux.
\VS{53}L'horreur me saisit à cause des méchants qui abandonnent ta loi.
\VS{54}Tes statuts sont le sujet de mes cantiques dans la maison où je suis étranger.
\VS{55}Yahweh, je me souviens de ton Nom pendant la nuit et je garde ta loi.
\VS{56}Cela m'arrive parce que je garde tes commandements.
\VS{57}[Heth.] Ô Yahweh ! J'en conclus que ma part est de garder tes paroles.
\VS{58}Je te supplie de tout mon cœur : Aie pitié de moi selon ta parole.
\VS{59}Je fais le compte de mes voies et je rebrousse chemin vers tes préceptes\FTNT{Os. 6:3 ; La. 3:40.}.
\VS{60}Je me hâte, je ne diffère point de garder tes commandements.
\VS{61}Une compagnie de méchants me pille, mais je n'oublie pas ta loi.
\VS{62}Je me lève au milieu de la nuit pour te célébrer à cause des ordonnances de ta justice.
\VS{63}Je suis l'ami de tous ceux qui te craignent et qui gardent tes commandements.
\VS{64}Yahweh, la terre est pleine de ta bonté ; enseigne-moi tes statuts.
\VS{65}[Teth.] Yahweh, tu fais du bien à ton serviteur selon ta parole.
\VS{66}Enseigne-moi le bon sens et la connaissance car je crois à tes commandements.
\VS{67}Avant d'avoir été humilié, je m'égarais, mais maintenant j'observe ta parole.
\VS{68}Tu es bon et bienfaisant, enseigne-moi tes statuts.
\VS{69}Les orgueilleux imaginent des faussetés contre moi, mais je garde de tout mon cœur tes commandements.
\VS{70}Leur cœur est insensible comme la graisse, mais moi, je prends plaisir dans ta loi\FTNT{De. 32:15 ; Jé. 5:28.}.
\VS{71}Il est bon que je sois humilié afin que j'apprenne tes statuts.
\VS{72}La loi que tu as prononcée de ta bouche m'est plus précieuse que mille pièces d'or ou d'argent\FTNT{Ps. 19:10-11 ; Job. 22:2.}.
\VS{73}[Yod.] Tes mains m'ont façonné, elles m'ont formé\FTNT{Jé. 1:5 ; Job. 10:9.} ; donne-moi l'intelligence afin que j'apprenne tes commandements.
\VS{74}Ceux qui te craignent me verront et se réjouiront, parce que j'espère en tes promesses.
\VS{75}Je reconnais, ô Yahweh, que tes jugements sont justes, et que tu m'as humilié par ta fidélité\FTNT{Hé. 12:10.}.
\VS{76}Que ta bonté soit ma consolation, comme tu l'as promis à ton serviteur.
\VS{77}Que tes compassions viennent sur moi et je vivrai ; car ta loi fait mes délices.
\VS{78}Que les orgueilleux rougissent de honte, de ce qu'ils m'oppriment sans cause ; mais moi, je médite sur tes ordonnances.
\VS{79}Que ceux qui te craignent et ceux qui connaissent tes préceptes reviennent vers moi.
\VS{80}Que mon cœur soit intègre dans tes statuts afin que je ne sois pas couvert de honte.
\VS{81}[Kaf.] Mon âme se consume en attendant ta délivrance ; j'espère en ta promesse.
\VS{82}Mes yeux s'épuisent en attendant ta promesse, lorsque je dis : Quand me consoleras-tu ?
\VS{83}Car je suis comme une outre dans la fumée, je n'oublie pas tes statuts.
\VS{84}Quel est le nombre de jours de ton serviteur ? Quand jugeras-tu ceux qui me poursuivent\FTNT{Ap. 6:10.} ?
\VS{85}Les orgueilleux me creusent des fosses, ils n'agissent pas selon ta loi.
\VS{86}Tous tes commandements ne sont que fidélité ; on me persécute sans cause, aide-moi\FTNT{Mt. 5:10.} !
\VS{87}On m'a presque réduit à rien et mis par terre ; mais je n'ai point abandonné tes commandements.
\VS{88}Fais-moi revivre selon ta miséricorde et je garderai les préceptes de ta bouche.
\VS{89}[Lamed.] Ô Yahweh ! Ta parole subsiste à toujours dans les cieux.
\VS{90}Ta fidélité dure d'âge en âge ; tu as établi la terre, et elle demeure ferme\FTNT{Pr. 1:4.}.
\VS{91}Ces choses subsistent aujourd'hui selon tes ordonnances, car toutes choses te servent.
\VS{92}Si ta loi n'avait pas fait mes délices, j'aurais déjà péri dans mon affliction.
\VS{93}Je n'oublierai jamais tes commandements car c'est par eux que tu m'as fait revivre.
\VS{94}Je suis à toi, sauve-moi ; car je recherche tes commandements.
\VS{95}Les méchants m'attendent pour me faire périr, mais je suis attentif à tes préceptes.
\VS{96}Je vois des bornes à tout ce qui est parfait, mais tes commandements n'ont point de limites.
\VS{97}[Mem.] Combien j'aime ta loi\FTNT{Ps. 1:2.} ! Elle est tout le jour l'objet de ma méditation.
\VS{98}Par tes commandements, tu m'as rendu plus sage que mes ennemis, parce que tes commandements sont toujours avec moi.
\VS{99}J'ai surpassé en prudence tous ceux qui m'avaient enseigné parce que tes préceptes sont l'objet de ma méditation.
\VS{100}Je suis devenu plus intelligent que les vieillards parce que j'observe tes commandements.
\VS{101}Je garde mes pieds de toute mauvaise voie afin d'observer ta parole.
\VS{102}Je ne me suis point détourné de tes ordonnances parce que tu me les enseignes.
\VS{103}Que ta parole est douce à mon palais ! Plus douce que le miel à ma bouche.
\VS{104}Je suis devenu intelligent par tes commandements, c'est pourquoi je hais toute voie de mensonge.
\VS{105}[Nun.] Ta parole est une lampe à mes pieds et une lumière sur mon sentier\FTNT{Pr. 6:23 ; 2 Pi. 1:19.}.
\VS{106}J'ai juré et je le tiendrai, d'observer les lois de ta justice\FTNT{Né. 10:29.}.
\VS{107}Yahweh, je suis extrêmement affligé, fais-moi revivre selon ta parole.
\VS{108}Yahweh, je te prie, agrée les sentiments que ma bouche exprime, et enseigne-moi tes ordonnances\FTNT{Os. 14:2 ; Hé. 13:15.}.
\VS{109}Ma vie est continuellement en danger, toutefois je n'oublie pas ta loi.
\VS{110}Les méchants m'ont tendu des pièges, toutefois je ne me suis point égaré de tes commandements.
\VS{111}J'ai pris pour héritage perpétuel tes préceptes car ils sont la joie de mon cœur.
\VS{112}J'ai incliné mon cœur à accomplir toujours tes statuts jusqu'au bout.
\VS{113}[Samech.] Je hais les hommes indécis\FTNT{1 R. 18:21 ; Ja. 1:6 ; Ja. 4:8.}, mais j'aime ta loi.
\VS{114}Tu es mon refuge et mon bouclier, je m'attends à ta parole.
\VS{115}Méchants, retirez-vous de moi\FTNT{Mt. 7:23 ; Ps. 6:9.} ! Et je garderai les commandements de mon Dieu.
\VS{116}Soutiens-moi suivant ta parole, et je vivrai ; et ne me fais point rougir de honte en me refusant ce que j'espérais.
\VS{117}Soutiens-moi, et je serai en sûreté ; et j'aurai continuellement les yeux sur tes statuts.
\VS{118}Tu as foulé aux pieds tous ceux qui se détournent de tes statuts, car le mensonge est le moyen dont ils se servent pour tromper.
\VS{119}Tu réduis à néant tous les méchants de la terre, comme de l'écume ; c'est pourquoi j'aime tes préceptes.
\VS{120}Ma chair frissonne de l'effroi que tu m'inspires et je crains tes jugements\FTNT{Ha. 3:16.}.
\VS{121}[Ayin.] J'ai exercé le jugement et la justice, ne m'abandonne pas à ceux qui me font tort.
\VS{122}Prends sous ta garantie le bien de ton serviteur et ne permets pas que je sois opprimé par les orgueilleux.
\VS{123}Mes yeux s'épuisent en attendant ta délivrance et la parole de ta justice.
\VS{124}Agis envers ton serviteur suivant ta miséricorde et enseigne-moi tes statuts.
\VS{125}Je suis ton serviteur, donne-moi l'intelligence, et je connaîtrai tes préceptes\FTNT{Pr. 1:4 ; Pr. 6:23.}.
\VS{126}Il est temps que Yahweh opère ; ils ont aboli ta loi.
\VS{127}C'est pourquoi j'aime tes commandements, plus que l'or et l'or fin.
\VS{128}C'est pourquoi je trouve justes tous tes commandements, je hais toute voie de mensonge.
\VS{129}[Pe.] Tes préceptes sont merveilleux, c'est pourquoi mon âme les garde.
\VS{130}La révélation de tes paroles éclaire, elle donne de l'intelligence aux simples.
\VS{131}J'ouvre ma bouche et je soupire, car je désire tes commandements.
\VS{132}Regarde-moi, et aie pitié de moi, selon tes jugements à l'égard de ceux qui aiment ton Nom.
\VS{133}Affermis mes pas sur ta parole, et que l'iniquité n'ait point d'emprise sur moi.
\VS{134}Délivre-moi de l'oppression des hommes afin que je garde tes commandements.
\VS{135}Fais luire ta face sur ton serviteur et enseigne-moi tes statuts.
\VS{136}Mes yeux répandent des torrents d'eau parce qu'on n'observe point ta loi.
\VS{137}[Tsade.] Tu es juste, ô Yahweh, et droit dans tes jugements.
\VS{138}Tu ordonnes tes préceptes avec justice et grande fidélité.
\VS{139}Mon zèle me consume parce que mes adversaires oublient tes paroles.
\VS{140}Ta parole est entièrement éprouvée, c'est pourquoi ton serviteur l'aime.
\VS{141}Je suis petit et méprisé, toutefois je n'oublie point tes commandements.
\VS{142}Ta justice est une justice éternelle, et ta loi est la vérité.
\VS{143}La détresse et l'angoisse m'atteignent, mais tes commandements font mes délices.
\VS{144}Tes préceptes ne sont que justice éternelle ; donne-moi l'intelligence afin que je vive.
\VS{145}[Qof.] Je crie de tout mon cœur, réponds-moi, ô Yahweh ! Je garde tes statuts.
\VS{146}Je crie vers toi, sauve-moi afin que j'observe tes préceptes.
\VS{147}Je devance l'aurore et je crie ; je m'attends à ta parole.
\VS{148}Mes yeux ont devancé les veilles de la nuit pour méditer ta parole.
\VS{149}Ecoute ma voix selon ta miséricorde, ô Yahweh ! Fais-moi revivre selon ton ordonnance.
\VS{150}Ceux qui poursuivent le crime s'approchent de moi, et ils s'éloignent de ta loi.
\VS{151}Yahweh, tu es près de moi ; et tous tes commandements ne sont que vérité.
\VS{152}Depuis longtemps, je sais par tes préceptes, que tu les as établis pour toujours.
\VS{153}[Resh.] Regarde mon affliction et sauve-moi, car je n'oublie pas ta loi.
\VS{154}Soutiens ma cause et rachète-moi ; fais-moi revivre selon ta parole.
\VS{155}La délivrance est loin des méchants parce qu'ils ne recherchent pas tes statuts.
\VS{156}Tes compassions sont en grand nombre, ô Yahweh ! Fais-moi revivre selon tes ordonnances.
\VS{157}Ceux qui me persécutent et qui me pressent sont en grand nombre, toutefois je ne me détourne pas de tes préceptes.
\VS{158}Je vois avec dégoût les traîtres et je suis rempli de tristesse car ils n'observent pas ta parole.
\VS{159}Regarde combien j'aime tes commandements, Yahweh ! Fais-moi revivre selon ta miséricorde !
\VS{160}Le fondement de ta parole est la vérité, et toutes les lois de ta justice sont éternelles.
\VS{161}[Shin.] Les princes du peuple me persécutent sans cause, mais mon cœur tremble à cause de ta parole.
\VS{162}Je me réjouis de ta parole comme ferait celui qui aurait trouvé un grand butin.
\VS{163}J'ai en haine et en abomination le mensonge ; j'aime ta loi.
\VS{164}Sept fois le jour je te loue à cause des ordonnances de ta justice.
\VS{165}Il y a une grande paix pour ceux qui aiment ta loi, et rien ne peut les renverser\FTNT{Es. 32:17 ; Ph. 4:7.}.
\VS{166}Yahweh, j'espère en ta délivrance et je pratique tes commandements.
\VS{167}Mon âme observe tes préceptes, je les aime beaucoup.
\VS{168}J'observe tes commandements et tes préceptes, car toutes mes voies sont devant toi.
\VS{169}[Tav.] Yahweh, que mon cri parvienne jusqu'à toi, donne-moi l'intelligence selon ta parole.
\VS{170}Que ma supplication vienne devant toi, délivre-moi selon ta parole.
\VS{171}Mes lèvres publieront ta louange quand tu m'auras enseigné tes statuts.
\VS{172}Ma langue ne s'entretiendra que de ta parole, parce que tous tes commandements ne sont que justice.
\VS{173}Que ta main me soit en aide, parce que j'ai choisi tes commandements.
\VS{174}Yahweh, je souhaite ta délivrance, et ta loi fait mes délices.
\VS{175}Que mon âme vive afin qu'elle te loue, et que tes ordonnances me soient en aide !
\VS{176}Je suis errant comme une brebis perdue\FTNT{Es. 53:6 ; Lu. 15:4 ; 1 Pi. 2:25.} ; cherche ton serviteur, car je n'oublie pas tes commandements.
\Chap{120}
\TextTitle{Cri de détresse}
\VerseOne{}Cantique des degrés\FTNT{Les psaumes 120 à 134 sont appelés « psaumes des degrés » ou de « l'ascension ». Ces psaumes furent chantés par les Israélites montant à Jérusalem au retour de la captivité de Babylone.}. J'ai invoqué Yahweh dans ma grande détresse, et il m'a exaucé.
\VS{2}Yahweh, délivre mon âme des lèvres mensongères et de la langue trompeuse.
\VS{3}Que te donne, que te rapporte la langue trompeuse ?
\VS{4}Ce sont des flèches aiguës tirées par un homme puissant et des charbons ardents du genêt\FTNT{Jé. 9:3 ; Ja. 3:5-6.}.
\VS{5}Malheureux que je suis de séjourner à Méschec et de demeurer aux tentes de Kédar !
\VS{6}Assez longtemps mon âme a demeuré auprès de ceux qui haïssent la paix !
\VS{7}Je ne cherche que la paix, mais lorsque j'en parle, ils sont pour la guerre.
\Chap{121}
\TextTitle{Yahweh ne dort ni ne sommeille}
\VerseOne{}Cantique des degrés. J'élève mes yeux vers les montagnes, d'où me viendra le secours.
\VS{2}Mon secours vient de Yahweh qui a fait les cieux et la terre\FTNT{Ps. 124:8.}.
\VS{3}Il ne permettra point que ton pied chancelle, celui qui te garde ne sommeillera point\FTNT{Es. 27:3 ; Pr. 3:23.}.
\VS{4}Voici, il ne sommeille ni ne dort celui qui garde Israël.
\VS{5}Yahweh est celui qui te garde, Yahweh est ton ombre à ta main droite\FTNT{Es. 25:4.}.
\VS{6}Pendant le jour, le soleil ne te frappera point, ni la lune pendant la nuit\FTNT{Es. 49:10. Ap. 7:16.}.
\VS{7}Yahweh te gardera de tout mal, il gardera ton âme.
\VS{8}Yahweh gardera ton départ et ton arrivée, dès maintenant et à jamais\FTNT{De. 28:6.}.
\Chap{122}
\TextTitle{Jérusalem, la ville de Yahweh}
\VerseOne{}Cantique des degrés de David. Je me réjouis à cause de ceux qui me disent : Allons à la maison de Yahweh\FTNT{Ps. 84:1-5.} !
\VS{2}Nos pieds s'arrêtent dans tes portes, ô Jérusalem !
\VS{3}Jérusalem, qui est bâtie comme une ville dont les édifices sont joints ensemble,
\VS{4}à laquelle montent les tribus, les tribus de Yahweh, selon le témoignage d'Israël, pour célébrer le Nom de Yahweh.
\VS{5}Car c'est là qu'ont été posés les trônes pour juger\FTNT{Mt. 19:28.}. Les trônes de la maison de David.
\VS{6}Demandez la paix de Jérusalem ; que ceux qui t'aiment jouissent du repos.
\VS{7}Que la paix soit dans tes murs et la tranquillité dans tes palais.
\VS{8}Pour l'amour de mes frères et de mes amis, je prie maintenant pour ta paix.
\VS{9}A cause de la maison de Yahweh notre Dieu, je fais une requête pour ton bonheur.
\Chap{123}
\TextTitle{Les regards fixés sur Yahweh}
\VerseOne{}Cantique des degrés. J'élève mes yeux vers toi qui habites dans les cieux.
\VS{2}Voici, comme les yeux des serviteurs regardent la main de leurs maîtres, comme les yeux de la servante regardent la main de sa maîtresse, ainsi nos yeux regardent à Yahweh notre Dieu, jusqu'à ce qu'il ait pitié de nous\FTNT{Ps. 25:15.}.
\VS{3}Aie pitié de nous, ô Yahweh ! Aie pitié de nous ! Car nous sommes assez rassasiés de mépris !
\VS{4}Notre âme est assez rassasiée des moqueries des orgueilleux, du mépris des hautains.
\Chap{124}
\TextTitle{Yahweh, le Dieu qui secours et protège son peuple}
\VerseOne{}Cantique des degrés, de David. Sans Yahweh, qui nous protégea, qu'Israël le dise !
\VS{2}Sans Yahweh, qui nous protégea, quand les hommes s'élevèrent contre nous ?
\VS{3}Ils nous auraient engloutis tous vivants quand leur colère s'enflamma contre nous.
\VS{4}Alors les eaux nous auraient submergés, les torrents auraient passé sur notre âme.
\VS{5}Alors les flots impétueux auraient passé sur notre âme.
\VS{6}Béni soit Yahweh qui ne nous a point livrés en proie à leurs dents !
\VS{7}Notre âme s'est échappée comme l'oiseau du filet des oiseleurs ; le filet a été rompu, et nous nous sommes échappés\FTNT{Pr. 6:5.}.
\VS{8}Notre secours est dans le Nom de Yahweh\FTNT{Ac. 4:11-12.} qui a fait les cieux et la terre.
\Chap{125}
\TextTitle{Yahweh entoure tous ceux qui se confient en lui}
\VerseOne{}Cantique des degrés. Ceux qui se confient en Yahweh sont comme la montagne de Sion : Elle ne chancelle point et est affermie pour toujours.
\VS{2}Quant à Jérusalem, il y a des montagnes autour d'elle, ainsi Yahweh entoure son peuple, dès maintenant et à jamais.
\VS{3}Car la verge de la méchanceté ne restera pas sur le lot des justes, de peur que les justes n'étendent leurs mains vers l'iniquité\FTNT{Es. 14:5.}.
\VS{4}Yahweh, répands tes bienfaits sur les bons et sur ceux dont le cœur est droit.
\VS{5}Mais ceux qui s'engagent dans des voies détournées, que Yahweh les fasse marcher avec les ouvriers d'iniquité\FTNT{Mt. 7:23.}. La paix sera sur Israël.
\Chap{126}
\TextTitle{Yahweh, le libérateur}
\VerseOne{}Cantique des degrés. Quand Yahweh ramena les captifs de Sion, nous étions comme ceux qui font un rêve.
\VS{2}Alors notre bouche était remplie de joie, et notre langue de chants de triomphe, alors on disait parmi les nations : Yahweh a fait de grandes choses pour eux !
\VS{3}Yahweh a fait de grandes choses pour nous ; nous sommes dans la joie.
\VS{4}Ô Yahweh ! Ramène nos captifs, comme des ruisseaux dans le midi\FTNT{Os. 6:11 ; Joë. 3:11.} !
\VS{5}Ceux qui sèment avec larmes moissonneront avec chants d'allégresse\FTNT{Ga. 6:9.}.
\VS{6}Celui qui marche en pleurant quand il porte la semence pour la mettre en terre, revient avec des chants d'allégresse quand il porte ses gerbes.
\Chap{127}
\TextTitle{Yahweh, le plus grand architecte}
\VerseOne{}Cantique des degrés, de Salomon. Si Yahweh ne bâtit la maison, ceux qui la bâtissent travaillent en vain ; si Yahweh ne garde la ville, celui qui la garde fait le guet en vain.
\VS{2}C'est en vain que vous vous levez de grand matin, que vous vous couchez tard, et que vous mangez le pain de douleurs ; certes c'est Dieu qui donne du repos à celui qu'il aime\FTNT{Ez. 20:20 ; Mc. 2:27.}.
\VS{3}Voici, les fils sont un héritage donné par Yahweh et le fruit du ventre est une récompense de Dieu\FTNT{Ps. 113:9 ; Ps. 128:3-6.}.
\VS{4}Telles sont les flèches dans la main d'un homme puissant, tels sont les fils de la jeunesse.
\VS{5}Heureux l'homme qui en a rempli son carquois ! Ils ne seront pas honteux quand ils parleront avec leurs ennemis à la porte.
\Chap{128}
\TextTitle{Yahweh assure la paix à celui qui le craint}
\VerseOne{}Cantique des degrés. Heureux tout homme qui craint Yahweh et marche dans ses voies !
\VS{2}Tu jouis du travail de tes mains ; tu es heureux et tu prospères\FTNT{Es. 3:10.}.
\VS{3}Ta femme est dans ta maison comme une vigne qui porte du fruit ; tes fils sont autour de ta table comme des plants d'oliviers.
\VS{4}Voici, certainement ainsi sera béni l'homme qui craint Yahweh.
\VS{5}Yahweh te bénira de Sion et tu verras le bien de Jérusalem tous les jours de ta vie.
\VS{6}Tu verras les fils de tes fils. La paix sera sur Israël.
\Chap{129}
\TextTitle{L'opprimé plus que vainqueur en Yahweh}
\VerseOne{}Cantique des degrés. Qu'Israël dise maintenant : Ils m'ont souvent tourmenté dès ma jeunesse.
\VS{2}Ils m'ont assez opprimé dès ma jeunesse, mais ils ne m'ont pas vaincu.
\VS{3}Des laboureurs ont labouré mon dos, ils y ont tracé de longs sillons.
\VS{4}Yahweh est juste, il a coupé les cordes des méchants.
\VS{5}Qu'ils soient honteux et qu'ils reculent, tous ceux qui haïssent Sion !
\VS{6}Qu'ils soient comme l'herbe des toits qui sèche avant qu'on l'arrache !
\VS{7}Le moissonneur n'en remplit point sa main, ni celui qui lie les gerbes n'en remplit point ses bras ;
\VS{8}et les passants ne disent pas : Que la bénédiction de Yahweh soit sur vous ! Nous vous bénissons au nom de Yahweh !
\Chap{130}
\TextTitle{La rédemption en abondance auprès de Yahweh}
\VerseOne{}Cantique des degrés. Ô Yahweh ! Je t'invoque du fond de l'abîme.
\VS{2}Seigneur, écoute ma voix ! Que tes oreilles soient attentives à la voix de mes supplications !
\VS{3}Yahweh ! si tu prends garde aux iniquités, Seigneur, qui subsistera ?
\VS{4}Mais le pardon se trouve auprès de toi, afin qu'on te craigne\FTNT{Mt. 26:28 ; Ro. 3:24 ; Col. 1:12-14.}.
\VS{5}J'espère en Yahweh, mon âme espère, et j'attends sa parole.
\VS{6}Mon âme attend le Seigneur plus que les sentinelles n'attendent le matin, plus que les sentinelles n'attendent le matin.
\VS{7}Israël, attends-toi à Yahweh, car Yahweh est miséricordieux et la rédemption est auprès de lui en abondance.
\VS{8}Lui-même rachètera Israël de toutes ses iniquités.
\Chap{131}
\TextTitle{Mettre son espoir en Yahweh seul}
\VerseOne{}Cantique des degrés, de David. Ô Yahweh ! Je n'ai ni un cœur qui s'élève ni un regard hautain\FTNT{Pr. 16:5 ; Pr. 6:17.} ; je ne m'occupe pas de choses trop grandes et trop extraordinaires pour moi.
\VS{2}J'ai l'âme calme et tranquille comme un enfant sevré de sa mère ; j'ai l'âme comme un enfant sevré.
\VS{3}Israël attends-toi à Yahweh dès maintenant et à jamais !
\Chap{132}
\TextTitle{Sion, le trône de Yahweh}
\VerseOne{}Cantique des degrés. Ô Yahweh ! Souviens-toi de David et de toute son affliction !
\VS{2}Il a juré à Yahweh et fait ce vœu au puissant de Jacob :
\VS{3}Je n'entrerai pas dans la tente où j'habite, je ne monterai pas sur le lit où je couche,
\VS{4}je ne donnerai pas du sommeil à mes yeux, je ne laisserai pas sommeiller mes paupières,
\VS{5}jusqu'à ce que j'aie trouvé un lieu pour Yahweh, une demeure pour le puissant de Jacob\FTNT{1 Ch. 15:1.}.
\VS{6}Voici, nous avons entendu parler d'elle à Ephrata, nous l'avons trouvée dans les champs de Jaar.
\VS{7}Entrons dans sa demeure, prosternons-nous devant son marchepied.
\VS{8}Lève-toi, ô Yahweh, pour venir à ton lieu de repos, toi et l'arche de ta force\FTNT{No. 10:35-36 ; 2 Ch. 6:41.}.
\VS{9}Que tes sacrificateurs soient revêtus de justice et que tes bien-aimés chantent de joie\FTNT{Es. 11:5 ; Ap. 19:8.} !
\VS{10}Pour l'amour de David, ton serviteur, ne permets pas que ton oint retourne en arrière !
\VS{11}Yahweh a juré la vérité à David, et il ne se rétractera pas, disant : Je mettrai le fruit de tes entrailles\FTNT{2 S. 7:12 ; 1 R. 8:25 ; 2 Ch. 6:16 ; Lu. 1:69 ; Ac. 2:30.} sur ton trône.
\VS{12}Si tes fils gardent mon alliance et mon témoignage que je leur enseignerai, leurs fils aussi seront assis à perpétuité sur ton trône.
\VS{13}Car Yahweh a choisi Sion, il l'a préférée pour être son trône :
\VS{14}Elle est mon lieu de repos à perpétuité, j'y habiterai parce que je l'ai désirée.
\VS{15}Je bénirai abondamment sa nourriture, je rassasierai de pain ses pauvres.
\VS{16}Je revêtirai de salut ses sacrificateurs, et ses bien-aimés chanteront avec des cris de joie.
\VS{17}Je ferai qu'en elle germera une corne à David ; je préparerai une lampe à mon oint,
\VS{18}je revêtirai de honte ses ennemis, et sur lui fleurira son diadème.
\Chap{133}
\TextTitle{La bénédiction dans la communion fraternelle}
\VerseOne{}Cantique des degrés. De David. Voici, oh ! Que c'est une chose bonne et que c'est une chose agréable que des frères demeurent unis ensemble\FTNT{Hé. 13:1 ; Ac. 2:46.} !
\VS{2}C'est comme cette huile précieuse, répandue sur la tête, qui coule sur la barbe d'Aaron\FTNT{Ex. 30:22-30.}, sur le bord de ses vêtements ;
\VS{3}comme la rosée de l'Hermon, celle qui descend sur les montagnes de Sion. Car c'est là que Yahweh a ordonné la bénédiction et la vie, pour l'éternité.
\Chap{134}
\TextTitle{Bénissez Yahweh, vous tous ses serviteurs}
\VerseOne{}Cantique des degrés. Voici, bénissez Yahweh ! Vous tous les serviteurs de Yahweh ! Qui vous tenez toutes les nuits dans la maison de Yahweh !
\VS{2}Elevez vos mains vers le lieu saint ! Et bénissez Yahweh !
\VS{3}Que Yahweh, qui a fait les cieux et la terre, te bénisse de Sion !
\Chap{135}
\TextTitle{La souveraineté de Dieu}
\VerseOne{}Louez le Nom de Yahweh ! Vous serviteurs de Yahweh ! Louez-le !
\VS{2}Vous qui vous tenez dans la maison de Yahweh, dans les parvis de la maison de notre Dieu,
\VS{3}louez Yahweh, car Yahweh est bon ! Chantez son Nom, car il est agréable !
\VS{4}Car Yahweh s'est choisi Jacob et Israël pour sa possession\FTNT{Ex. 19:5 ; De. 7:6 ; Tit. 2:14 ; 1 Pi. 2:9.}.
\VS{5}Certainement, je sais que Yahweh est grand et que notre Seigneur est au-dessus de tous les dieux.
\VS{6}Yahweh fait tout ce qu'il lui plaît, dans les cieux et sur la terre, dans la mer et dans tous les abîmes.
\VS{7}C'est lui qui fait monter les vapeurs des extrémités de la terre ; il fait les éclairs et la pluie ; il tire le vent hors de ses trésors.
\VS{8}C'est lui qui a frappé les premiers-nés d'Egypte, tant des hommes que des bêtes ;
\VS{9}qui a envoyé des prodiges et des miracles au milieu de toi, ô Egypte ! Contre Pharaon et contre tous ses serviteurs ;
\VS{10}qui a frappé plusieurs nations et tué les puissants rois ;
\VS{11}Sihon, roi des Amoréens, et Og, roi de Basan, et ceux de tous les royaumes de Canaan\FTNT{No. 21:33-35 ; De. 3:11.} ;
\VS{12}qui a donné leur pays en héritage, en héritage à Israël son peuple.
\VS{13}Yahweh, ton Nom est pour toujours ! Yahweh, ta mémoire de génération en génération !
\VS{14}Car Yahweh jugera son peuple et se repentira à l'égard de ses serviteurs.
\VS{15}Les dieux des nations ne sont que de l'or et de l'argent, un ouvrage de mains d'homme.
\VS{16}Ils ont une bouche, et ne parlent point ; ils ont des yeux, et ne voient point ;
\VS{17}ils ont des oreilles, et n'entendent point ; il n'y a point de souffle dans leur bouche.
\VS{18}Ils leur ressemblent ceux qui les font, et tous ceux qui s'y confient.
\VS{19}Maison d'Israël, bénissez Yahweh ! Maison d'Aaron, bénissez Yahweh !
\VS{20}Maison des Lévites, bénissez Yahweh ! Vous qui craignez Yahweh, bénissez Yahweh !
\VS{21}Béni soit de Sion Yahweh qui habite dans Jérusalem ! Louez Yahweh !
\Chap{136}
\TextTitle{La bonté de Yahweh demeure à toujours}
\VerseOne{}Célébrez Yahweh, car il est bon, car sa bonté demeure à toujours !
\VS{2}Célébrez le Dieu des dieux, car sa bonté demeure à toujours !
\VS{3}Célébrez le Seigneur des seigneurs, car sa bonté demeure à toujours !
\VS{4}Célébrez celui qui seul fait de grandes merveilles, car sa bonté demeure à toujours !
\VS{5}Celui qui a fait avec intelligence les cieux, car sa bonté demeure à toujours !
\VS{6}Celui qui a étendu la terre sur les eaux, car sa bonté demeure à toujours !
\VS{7}Celui qui a fait les grands luminaires, car sa bonté demeure à toujours !
\VS{8}Le soleil pour dominer sur le jour, car sa bonté demeure à toujours !
\VS{9}La lune et les étoiles pour dominer la nuit, car sa bonté demeure à toujours !
\VS{10}Celui qui a frappé l'Egypte dans leurs premiers-nés, car sa bonté demeure à toujours !
\VS{11}Qui a fait sortir Israël du milieu d'eux, car sa bonté demeure à toujours.
\VS{12}Et cela avec main forte et bras étendu, car sa bonté demeure à toujours !
\VS{13}Il a fendu la Mer Rouge en deux, car sa bonté demeure à toujours !
\VS{14}Il a fait passer Israël par le milieu d'elle, car sa bonté demeure à toujours !
\VS{15}Il a renversé Pharaon et son armée dans la Mer Rouge, car sa bonté demeure à toujours !
\VS{16}Il a conduit son peuple dans le désert, car sa bonté demeure à toujours !
\VS{17}Il a frappé les grands rois, car sa bonté demeure à toujours !
\VS{18}Qui a tué des grands rois, car sa bonté demeure à toujours !
\VS{19}Sihon, roi des Amoréens, car sa bonté demeure à toujours !
\VS{20}Og, roi de Basan, car sa bonté demeure à toujours !
\VS{21}Il a donné leur pays en héritage, car sa bonté demeure à toujours\FTNT{Jos. 12:7.} !
\VS{22}En héritage à Israël son serviteur, car sa bonté demeure à toujours !
\VS{23}Et qui, lorsque nous étions humiliés, s'est souvenu de nous, car sa bonté demeure à toujours !
\VS{24}Il nous a délivrés de la main de nos adversaires, car sa bonté demeure à toujours !
\VS{25}Il donne la nourriture à toute chair, car sa bonté demeure à toujours\FTNT{Ps. 104:21 ; Mt. 6:26 ; Ps. 147:9.} !
\VS{26}Célébrez le Dieu des cieux, car sa bonté demeure à toujours !
\Chap{137}
\TextTitle{Le coeur des captifs}
\VerseOne{}Sur les bords des fleuves de Babylone, nous étions assis et nous pleurions en nous souvenant de Sion.
\VS{2}Nous avions suspendu nos harpes au milieu des saules.
\VS{3}Là, ceux qui nous avaient emmenés en captivité, nous ont demandé des paroles de chants, et nos oppresseurs de la joie, en nous disant : Chantez-nous quelques cantiques de Sion ! Nous avons répondu :
\VS{4}Comment chanterions-nous les cantiques de Yahweh sur une terre étrangère ?
\VS{5}Si je t'oublie, Jérusalem, que ma droite s'oublie elle-même.
\VS{6}Que ma langue soit attachée à mon palais\FTNT{Ez. 3:26.}, si je ne me souviens pas de toi, si je ne fais pas de Jérusalem le sujet de ma réjouissance.
\VS{7}Ô Yahweh, souviens-toi des fils d'Edom, qui dans la journée de Jérusalem disaient : Rasez, rasez jusqu'à ses fondements\FTNT{Jé. 25:15-21 ; Jé. 49:7-8 ; Ez. 25:12 ; La. 4:21 ; Am. 1:11.} !
\VS{8}Fille de Babylone, qui va être détruite, heureux celui qui te rend la pareille de ce que tu nous as fait\FTNT{Jé. 50:15-29 ; Ap. 18:6.} !
\VS{9}Heureux celui qui saisit tes petits enfants et qui les écrase contre le rocher\FTNT{Es. 13:16.} !
\Chap{138}
\TextTitle{La renommée de Yahweh dans les nations}
\VerseOne{}Psaume de David. Je te célèbre de tout mon cœur, je te chante des louanges dans la présence de Dieu.
\VS{2}Je me prosterne dans ton saint temple, et je célèbre ton Nom à cause de ta bonté et de ta fidélité ; car ta renommée s'est accrue par l'accomplissement de ta promesse.
\VS{3}Le jour où je t'ai invoqué, tu m'as exaucé, tu m'as rassuré, tu m'as fortifié d'une nouvelle force en mon âme.
\VS{4}Yahweh ! Tous les rois de la terre te célèbrent, quand ils entendent les paroles de ta bouche.
\VS{5}Ils chantent les voies de Yahweh, car la gloire de Yahweh est grande.
\VS{6}Car Yahweh est haut élevé, il voit les humbles et il reconnaît de loin les orgueilleux.
\VS{7}Quand je marche au milieu de l'adversité, tu me rends la vie, tu avances ta main contre la colère de mes ennemis, et ta droite me délivre.
\VS{8}Yahweh achèvera ce qui me concerne. Yahweh, ta bonté demeure toujours ; tu n'abandonnes pas l'œuvre de tes mains\FTNT{Ph. 1:6.}.
\Chap{139}
\TextTitle{L'omniscience de Yahweh}
\VerseOne{}Psaume de David, donné au chef des chantres. Yahweh, tu me sondes et tu me connais\FTNT{Jé. 12:3 ; Ps. 17:3.}.
\VS{2}Tu sais quand je m'assieds et quand je me lève ; tu discernes de loin ma pensée.
\VS{3}Tu sais quand je marche et quand je me couche ; tu connais parfaitement toutes mes voies.
\VS{4}Avant que la parole soit sur ma langue, voici, ô Yahweh, tu la connais déjà !
\VS{5}Tu m'entoures par derrière et par devant, et tu mets ta main sur moi.
\VS{6}Ta science est trop merveilleuse pour moi, elle est si haut élevée que je ne saurais l'atteindre\FTNT{Job. 42:3 ; Ps. 92:6 ; Ro. 11:33.}.
\VS{7}Où irai-je loin de ton Esprit, et où fuirai-je loin de ta face\FTNT{Jé. 23:24 ; Am. 9:2-4 ; Jon. 1:3.} ?
\VS{8}Si je monte aux cieux, tu y es ; si je me couche dans le scheol, t'y voilà.
\VS{9}Si je prends les ailes de l'aurore et que je demeure à l'extrémité de la mer,
\VS{10}là aussi ta main me conduira et ta droite me saisira.
\VS{11}Si je dis : Au moins les ténèbres me couvriront, la nuit même sera une lumière tout autour de moi.
\VS{12}Même les ténèbres ne me cacheront point de toi, et la nuit resplendira comme le jour, et les ténèbres comme la lumière.
\VS{13}Tu as créé mes reins, tu me couvres du sein de ma mère.
\VS{14}Je te célèbre de ce que je suis une créature redoutée et merveilleuse ; tes œuvres sont merveilleuses, et mon âme le reconnaît très bien.
\VS{15}Mon corps n'était pas caché devant toi lorsque j'ai été fait dans un lieu secret et brodé dans les profondeurs de la terre\FTNT{Ps. 119:73 ; Ec. 11:5.}.
\VS{16}Tes yeux me voyaient quand je n'étais qu'un embryon, et sur ton livre étaient inscrits tous les jours qui m'étaient destinés\FTNT{Ph. 4:3 ; Ap. 3:5 ; Ap. 20:15.}.
\VS{17}Dieu ! Que tes pensées sont précieuses ! Que le nombre en est grand !
\VS{18}Si je les compte, elles sont plus nombreuses que les grains de sable. Je m'éveille et je suis encore avec toi.
\VS{19}Ô Dieu ! Ne tueras-tu pas le méchant ? C'est pourquoi, hommes sanguinaires, retirez-vous loin de moi !
\VS{20}Car ils ont parlé de toi en pensant à quelque méchanceté ; ils ont élevé tes ennemis en mentant.
\VS{21}Yahweh, n'aurais-je point en haine ceux qui te haïssent ; et ne serais-je point irrité contre ceux qui s'élèvent contre toi ?
\VS{22}Je les hais d'une parfaite haine ; ils sont pour moi des ennemis.
\VS{23}Ô Dieu ! Sonde-moi et considère mon cœur ! Eprouve-moi et considère mes discours !
\VS{24}Et regarde si je suis sur une mauvaise voie ; conduis-moi sur la voie de l'éternité.
\Chap{140}
\TextTitle{Yahweh, le protecteur}
\VerseOne{}Psaume de David, donné au chef des chantres. Yahweh, délivre-moi de l'homme méchant, garde-moi de l'homme violent.
\VS{2}Ils méditent des méchancetés dans leur cœur, tous les jours ils complotent des guerres.
\VS{3}Ils aiguisent leur langue comme un serpent, il y a du venin de vipère sous leurs lèvres. Sélah.
\VS{4}Yahweh, garde-moi de la main du méchant, préserve-moi de l'homme violent, de ceux qui méditent de me faire tomber.
\VS{5}Les orgueilleux me tendent un piège et des filets, et ils étendent des rets le long du chemin, ils me dressent des embûches. Sélah.
\VS{6}Je dis à Yahweh : Tu es mon Dieu, Yahweh ! Prête l'oreille à la voix de mes supplications !
\VS{7}Ô Yahweh ! Seigneur ! La force de mon salut ! Tu couvres ma tête au jour de la bataille.
\VS{8}Yahweh n'accorde point au méchant ses désirs ; qu'il n'apporte pas ses méchants desseins, ils s'élèveraient. Sélah.
\VS{9}Quant à la tête de ceux qui m'environnent, que la méchanceté de leurs lèvres les recouvre.
\VS{10}Que des charbons ardents soient jetés sur eux ! Qu'ils tombent sur eux ! Qu'il les fasse tomber dans le feu, et dans des fosses profondes, sans qu'ils se relèvent\FTNT{Pr. 25:21-22 ; Ro. 12:20.} !
\VS{11}Que l'homme à la langue méchante ne soit point affermi sur la terre ; quant à l'homme violent et mauvais, qu'on le chasse jusqu'à ce qu'il soit exterminé.
\VS{12}Je sais que Yahweh fera justice au malheureux et droit aux indigents.
\VS{13}Quoi qu'il en soit, les justes célébreront ton Nom, les hommes droits habiteront devant ta face.
\Chap{141}
\TextTitle{Yahweh, garde-moi du mal !}
\VerseOne{}Psaume de David. Yahweh, je t'invoque, hâte-toi de venir vers moi ; prête l'oreille à ma voix lorsque je crie à toi.
\VS{2}Que ma prière te soit agréable comme l'encens, et l'élévation de mes mains comme l'offrande du soir\FTNT{Ex. 30:1 ; Ap. 5:8 ; Ap. 8:3.}.
\VS{3}Yahweh, mets une garde à ma bouche, garde l'entrée de mes lèvres.
\VS{4}N'incline point mon cœur à des choses mauvaises, au point que je commette quelques méchantes actions par malice, avec les hommes qui font le mal ; et que je ne mange point de leurs délices.
\VS{5}Que le juste me frappe, ce me sera une faveur ; et qu'il me réprimande, ce sera pour moi un baume excellent\FTNT{Pr. 27:6 ; Ec. 7:5.} ; il ne blessera point ma tête ; car ma prière sera pour eux leur calamité.
\VS{6}Que leurs juges soient précipités le long des rochers, et l'on écoutera mes paroles, car elles sont agréables.
\VS{7}Nos os sont dispersés dans la bouche du scheol comme quand on laboure la terre et on fend le bois.
\VS{8}C'est pourquoi, ô Yahweh, Seigneur, mes yeux sont sur toi, je me suis retiré vers toi, n'abandonne point mon âme !
\VS{9}Garde-moi du piège qu'ils m'ont tendu et des filets de ceux qui font le mal.
\VS{10}Que tous les méchants tombent dans leurs filets, jusqu'à ce que je sois passé.
\Chap{142}
\TextTitle{Yahweh, mon refuge}
\VerseOne{}Cantique de David. Prière qu'il fit lorsqu'il était dans la caverne\FTNT{1 S. 24:4.}.
\VS{2}Je crie de ma voix à Yahweh, je supplie de ma voix Yahweh.
\VS{3}Je répands devant lui ma complainte, je déclare mon angoisse devant lui\FTNT{1 S. 1:15 ; La. 2:19.}.
\VS{4}Quand mon esprit est abattu en moi, toi, tu connais mon sentier. Ils me tendent un piège sur le chemin par lequel je marche.
\VS{5}Je contemple à ma droite et je regarde, et il n'y a personne qui me reconnaît ; tout refuge s'évanouit devant moi, il n'y a personne qui prend soin de mon âme.
\VS{6}Yahweh, je crie vers toi ; je dis : Tu es mon refuge, ma part sur la terre des vivants.
\VS{7}Sois attentif à mon cri car je suis devenu très affaibli. Délivre-moi de ceux qui me poursuivent car ils sont plus puissants que moi.
\VS{8}Retire mon âme de sa prison afin que je célèbre ton Nom ! Les justes viendront m'entourer quand tu m'auras fait du bien.
\Chap{143}
\TextTitle{Yahweh, enseigne-moi à faire ta volonté}
\VerseOne{}Psaume de David. Yahweh, écoute ma requête, prête l'oreille à mes supplications ! Exauce-moi dans ta fidélité, réponds-moi à cause de ta justice !
\VS{2}N'entre point en jugement avec ton serviteur, car aucun homme vivant n'est juste devant toi.
\VS{3}Car l'ennemi poursuit mon âme, il foule ma vie par terre ; il me fait habiter dans les ténèbres comme ceux qui sont morts depuis longtemps.
\VS{4}Et mon esprit est abattu au-dedans de moi, mon cœur est épouvanté en mon sein.
\VS{5}Je me souviens des jours anciens, je médite sur toutes tes œuvres, je médite sur l'ouvrage de tes mains\FTNT{Ps. 77:11-13.}.
\VS{6}J'étends mes mains vers toi ; mon âme s'adresse à toi comme une terre desséchée\FTNT{Ps. 28:1 ; Ps. 42:1-3.}. Sélah.
\VS{7}Ô Yahweh, hâte-toi, réponds-moi ! Mon esprit se consume ! Ne me cache point ta face au point que je devienne semblable à ceux qui descendent dans la fosse !
\VS{8}Fais-moi entendre dès le matin ta miséricorde, car je me confie en toi ; fais-moi connaître le chemin par lequel je dois marcher, car j'ai élevé mon cœur vers toi\FTNT{Ps. 25:1.}.
\VS{9}Yahweh, délivre-moi de mes ennemis, car je me suis réfugié auprès de toi !
\VS{10}Enseigne-moi à faire ta volonté, car tu es mon Dieu ! Que ton bon Esprit me conduise sur la voie de la droiture\FTNT{Jn. 16:13.} !
\VS{11}Yahweh, rends-moi la vie pour l'amour de ton Nom ! Retire mon âme de la détresse à cause de ta justice !
\VS{12}Et selon la bonté que tu as pour moi, retranche mes ennemis ! Détruis tous ceux qui tiennent mon âme oppressée, parce que je suis ton serviteur !
\Chap{144}
\TextTitle{Se confier en Yahweh, le Rocher}
\VerseOne{}Psaume de David. Béni soit Yahweh, mon rocher\FTNT{Voir commentaire en Es. 8:13-17.} qui exerce mes mains au combat et mes doigts à la bataille,
\VS{2}qui déploie sa bonté envers moi, qui est ma forteresse, ma haute retraite, mon libérateur\FTNT{Es. 59:20-21 ; Ro. 11:26.}, mon bouclier\FTNT{Ep. 6:16.}, mon refuge\FTNT{Ps. 91 ; Mt. 11:28-30.}, qui m'assujettit mon peuple.
\VS{3}Ô Yahweh ! Qu'est-ce que l'homme pour que tu aies soin de lui\FTNT{Ps. 8:5 ; Job. 7:17 ; Hé. 2:6-7.} ? Le fils de l'homme mortel pour que tu prennes garde à lui ?
\VS{4}L'homme est semblable à la vanité, ses jours sont comme une ombre qui passe\FTNT{Ps. 102:12 ; Job. 14:1-2 ; Ec. 6:12.}.
\VS{5}Yahweh abaisse tes cieux et descends ! Touche les montagnes et qu'elles soient fumantes\FTNT{Es. 63:19 ; Ps. 18:7-8.}.
\VS{6}Lance les éclairs et disperse mes ennemis ! Lance tes flèches et mets-les en déroute !
\VS{7}Etends tes mains d'en haut ; sauve-moi et délivre-moi des grandes eaux, de la main des fils de l'étranger,
\VS{8}dont la bouche profère le mensonge, et dont la droite est une droite trompeuse !
\VS{9}Ô Dieu ! Je chanterai un cantique nouveau ! Je te célèbrerai sur le luth à dix cordes !
\VS{10}Toi qui donnes la délivrance aux rois et qui délivres de l'épée meurtrière David, ton serviteur.
\VS{11}Retire-moi et délivre-moi de la main des fils de l'étranger, dont la bouche profère le mensonge et dont la droite est une droite trompeuse ;
\VS{12}afin que nos fils soient comme des plantes qui croissent dans leur jeunesse et nos filles comme des pierres angulaires taillées pour l'ornement d'un palais.
\VS{13}Que nos greniers soient pleins, fournissant toute espèce de provision ; que nos troupeaux multiplient par milliers, même par dix milliers dans nos rues.
\VS{14}Que nos bœufs soient chargés de graisse. Qu'il n'y ait ni brèche, ni sortie dans nos murailles, ni cri dans nos places.
\VS{15}Heureux le peuple pour qui il en est ainsi ! Heureux le peuple dont Yahweh est le Dieu !
\Chap{145}
\TextTitle{Louange à Yahweh pour tout ce qu'il est}
\VerseOne{}Psaume de louange, composé par David. [Aleph.] Mon Dieu, mon roi, je t'exalterai et je bénirai ton Nom à toujours, et à perpétuité !
\VS{2}[Beth.] Je te bénirai chaque jour, et je louerai ton Nom à toujours, et à perpétuité !
\VS{3}[Guimel.] Yahweh est grand et très digne de louanges, il n'est pas possible de sonder sa grandeur.
\VS{4}[Daleth.] Que chaque génération célèbre tes œuvres et publie tes hauts faits !
\VS{5}[He.] Je dirai la splendeur glorieuse de ta majesté et de tes faits merveilleux.
\VS{6}[Vav.] On parlera de ta puissance redoutable, et je raconterai ta grandeur.
\VS{7}[Zayin.] Ils proclameront le souvenir de ton immense bonté, et ils raconteront avec chants de triomphe ta justice.
\VS{8}[Heth.] Yahweh est miséricordieux et compatissant, lent à la colère et grand en bonté.
\VS{9}[Teth.] Yahweh est bon envers tous et ses compassions sont au-dessus de toutes ses œuvres.
\VS{10}[Yod.] Yahweh, toutes tes œuvres te célébreront, et tes fidèles te béniront.
\VS{11}[Kaf.] Ils diront la gloire de ton règne, et ils proclameront ta puissance
\VS{12}[Lamed.] pour faire connaître aux fils de l'homme ta puissance et la splendeur glorieuse de ton règne.
\VS{13}[Mem.] Ton règne est un règne de tous les siècles et ta domination subsiste dans tous les âges.
\VS{14}[Samech.] Yahweh soutient tous ceux qui tombent et redresse tous ceux qui sont courbés\FTNT{Ps. 146:8.}.
\VS{15}[Ayin.] Les yeux de tous les animaux s'attendent à toi et tu leur donnes leur nourriture en leur temps.
\VS{16}[Pe.] Tu ouvres ta main et tu rassasies à souhait toute créature vivante.
\VS{17}[Tsade.] Yahweh est juste dans toutes ses voies et plein de bonté dans toutes ses œuvres\FTNT{Da. 4:37.}.
\VS{18}[Qof.] Yahweh est près de tous ceux qui l'invoquent, de tous ceux qui l'invoquent avec vérité\FTNT{Ps. 34:18.}.
\VS{19}[Resh.] Il accomplit le désir de ceux qui le craignent, il entend leur cri et les délivre.
\VS{20}[Shin.] Yahweh garde tous ceux qui l'aiment, mais il exterminera tous les méchants.
\VS{21}[Tav.] Ma bouche racontera la louange de Yahweh, et toute chair bénira le Nom de sa sainteté à toujours, et à perpétuité\FTNT{Ps. 103:1.}.
\Chap{146}
\TextTitle{La fidélité de Yahweh dure à toujours}
\VerseOne{}Louez Yahweh ! Mon âme, loue Yahweh !
\VS{2}Je louerai Yahweh durant ma vie, je chanterai mon Dieu tant que je vivrai !
\VS{3}Ne vous confiez pas aux grands, ni en aucun fils de l'homme qui ne peuvent délivrer.
\VS{4}Son esprit s'en va et l'homme retourne dans sa terre, et ce même jour ses desseins périssent.
\VS{5}Heureux celui qui a pour secours le Dieu de Jacob, qui met son espoir en Yahweh, son Dieu !
\VS{6}Il a fait les cieux et la terre, la mer et tout ce qui s'y trouve. Il garde la vérité à toujours !
\VS{7}Il fait droit aux opprimés, il donne du pain aux affamés ; Yahweh délie ceux qui sont liés\FTNT{Jn. 11:43-44.}.
\VS{8}Yahweh ouvre les yeux des aveugles\FTNT{Les miracles de Jésus-Christ confirment sa divinité (Es. 35:4-6 ; Lu. 7:19-23).} ; Yahweh redresse ceux qui sont courbés\FTNT{Lu. 13:11-13.} ; Yahweh aime les justes.
\VS{9}Yahweh protège les étrangers, il soutient l'orphelin et la veuve, mais il renverse la voie des méchants.
\VS{10}Yahweh règne éternellement. Ô Sion ! Ton Dieu subsiste d'âge en âge. Louez Yahweh !
\Chap{147}
\TextTitle{Yahweh aime ceux qui le craignent et qui s'attendent à sa bonté}
\VerseOne{}Louez Yahweh ! Car il est beau de chanter à notre Dieu ! Car il est doux et bienséant de le louer !
\VS{2}Yahweh est celui qui bâtit Jérusalem ; il rassemblera ceux d'Israël qui sont dispersés çà et là.
\VS{3}Il guérit ceux qui ont le cœur brisé et il bande leurs plaies\FTNT{Ex. 15:26 ; De. 32:39 ; Job. 5:18.}.
\VS{4}Il compte le nombre des étoiles, il les appelle toutes par leur nom.
\VS{5}Notre Seigneur est grand, puissant par sa force, son intelligence n'a point de limites.
\VS{6}Yahweh soutient les malheureux, mais il abaisse les méchants jusqu'à terre.
\VS{7}Chantez à Yahweh avec reconnaissance ! Célébrez notre Dieu avec la harpe !
\VS{8}Il couvre les cieux de nuées, il prépare la pluie pour la terre ; il fait germer l'herbe sur les montagnes.
\VS{9}Il donne la nourriture au bétail et aux petits du corbeau qui crient.
\VS{10}Il ne prend point plaisir dans la force du cheval ; il ne fait point cas des jambes de l'homme.
\VS{11}Yahweh aime ceux qui le craignent, ceux qui s'attendent à sa bonté.
\VS{12}Jérusalem, loue Yahweh ! Sion, loue ton Dieu !
\VS{13}Car il a affermi les barres de tes portes, il a béni tes fils au milieu de toi.
\VS{14}Il rend la paix à son territoire et te rassasie du meilleur froment.
\VS{15}C'est lui qui envoie ses ordres sur la terre, sa parole court avec rapidité\FTNT{Es. 55:10-11.}.
\VS{16}C'est lui qui donne la neige comme des flocons de laine et qui répand la gelée blanche comme de la cendre.
\VS{17}C'est lui qui lance sa glace comme par morceaux, qui peut résister devant son froid ?
\VS{18}Il envoie sa parole, et il les fond ; il fait souffler son vent, et les eaux coulent\FTNT{Ps. 135:7.}.
\VS{19}Il déclare ses paroles à Jacob, ses statuts et ses ordonnances à Israël\FTNT{Ps. 78:5.}.
\VS{20}Il n'a pas agi de même pour toutes les nations, c'est pourquoi elles ne connaissent point ses ordonnances. Louez Yahweh !
\Chap{148}
\TextTitle{La création loue son Dieu}
\VerseOne{}Louez Yahweh ! Louez des cieux Yahweh ! Louez-le dans les lieux élevés !
\VS{2}Louez-le, vous tous anges ! Louez-le, vous toutes ses armées !
\VS{3}Louez-le, vous, soleil et lune ! Louez-le, vous toutes, étoiles lumineuses !
\VS{4}Louez-le, vous, cieux des cieux ! Et vous, eaux qui êtes au-dessus des cieux !
\VS{5}Qu'ils louent le Nom de Yahweh ! Car il a commandé et ils ont été créés\FTNT{Ge. 1:3-6 ; Jé. 31:35.}.
\VS{6}Il les a établis à perpétuité et à toujours ; il a donné des lois, et il ne les violera pas\FTNT{Ps. 104:5 ; Ps. 119:91 ; Job. 14:5.}.
\VS{7}De la terre, louez Yahweh ! Louez-le, monstres marins et tous les abîmes !
\VS{8}Feu et grêle, neige et brouillard, vent impétueux qui exécutez ses ordres,
\VS{9}montagnes et toutes les collines, arbres fruitiers et tous les cèdres,
\VS{10}bêtes sauvages et tout le bétail, reptiles et oiseaux ailés,
\VS{11}rois de la terre et tous les peuples, princes et tous les juges de la terre,
\VS{12}ceux qui sont à la fleur de leur âge, et les vierges aussi, les vieillards, et les jeunes gens !
\VS{13}Qu'ils louent le Nom de Yahweh ! Car son Nom seul est haut élevé ! Sa majesté est au-dessus de la terre et des cieux.
\VS{14}Il a relevé la force de son peuple, sujet de louange pour tous ses fidèles, pour les fils d'Israël, du peuple qui est près de lui. Louez Yahweh !
\Chap{149}
\TextTitle{Adorons Yahweh}
\VerseOne{}Louez Yahweh ! Chantez à Yahweh un cantique nouveau et louez-le dans l'assemblée de ses fidèles !
\VS{2}Qu'Israël se réjouisse en celui qui l'a fait ! Et que les fils de Sion soient dans l'allégresse à cause de leur Roi\FTNT{Ps. 100:3 ; Za. 9:9 ; Mt. 21:5.} !
\VS{3}Qu'ils louent son Nom avec des danses ! Qu'ils le chantent avec le tambourin et la harpe !
\VS{4}Car Yahweh prend plaisir à son peuple, il glorifie les pauvres en les délivrant.
\VS{5}Que les fidèles se réjouissent dans la gloire, qu'ils poussent des cris de joie sur leur couche.
\VS{6}Les louanges de Dieu sont dans leur bouche et les épées affilées à deux tranchants dans leur main,
\VS{7}pour se venger des nations, pour châtier les peuples,
\VS{8}pour lier leurs rois avec des chaînes, et les plus honorables parmi eux avec des ceps de fer,
\VS{9}pour exercer sur eux le jugement qui est écrit ! Cet honneur est pour tous ses fidèles. Louez Yahweh !
\Chap{150}
\TextTitle{Que tout ce qui respire loue Yahwe !}
\VerseOne{}Louez Yahweh ! Louez Dieu à cause de sa sainteté ! Louez-le dans l'étendue de toute sa puissance !
\VS{2}Louez-le pour ses hauts faits ! Louez-le selon la grandeur de sa magnificence !
\VS{3}Louez-le au son du shofar ! Louez-le avec le luth et la harpe !
\VS{4}Louez-le avec le tambour et avec des danses ! Louez-le avec des instruments à cordes et le chalumeau !
\VS{5}Louez-le avec les cymbales sonores ! Louez-le avec les cymbales de cri de joie !
\VS{6}Que tout ce qui respire loue Yahweh ! Louez Yahweh !
\PPE{}
\end{multicols}

%\clearpage\ShortTitle{Proverbes}\BookTitle{Proverbes}\BFont
\noindent\hrulefill
{\footnotesize
\textit{
\bigskip
{\centering{}
\\Auteurs : Salomon, Agur et Lemuel
\\(Heb. : Mishlei)
\\Signification : Paraboles
\\Thème : La sagesse
\\Date de rédaction : 10\up{ème} siècle av J.-C.\\}
}
%\bigskip
\textit{
\\Le mot « proverbe » désigne un genre littéraire appliqué à une sentence, une énigme, une comparaison, un oracle, une
parabole ou une parole de sagesse. Le livre des proverbes est donc un recueil de sentences dont la majeure partie
est attribuée à Salomon. Véritable collection de maximes morales et spirituelles, la sagesse, la crainte de Dieu et la
tempérance en sont les thèmes principaux.
%\bigskip
\\Ce livre met en évidence l'opposition entre la voie du méchant et celle du juste, entre la femme étrangère et la femme
vertueuse, entre l'orgueil et l'humilité, entre la sagesse et la folie, entre le chemin de la vie et celui de la mort. Comme il était coutume au Moyen-Orient, ces écrits s'adressaient particulièrement aux jeunes gens en vue de leur instruction.\bigskip
}
}
\par\nobreak\noindent\hrulefill
\begin{multicols}{2}
\Chap{1}
\TextTitle{[But du livre : connaître la sagesse}
\VerseOne{}Les Proverbes de Salomon, fils de David et roi d'Israël.
\VS{2}Pour connaître la sagesse et l'instruction, pour discerner les paroles d'intelligence ;
\VS{3}pour recevoir une leçon de bon sens, de justice, de jugement et d'équité.
\VS{4}Pour donner du discernement aux simples, aux jeunes gens de la connaissance et de la réflexion.
\VS{5}Le sage écoutera, et il augmentera son savoir, et l'homme intelligent acquerra de la prudence ;
\VS{6}afin d'entendre les paraboles et les énigmes ; les discours des sages et leurs énigmes.
\TextTitle{Le fondement de la sagesse : la crainte de Dieu}
\VS{7}La crainte de Yahweh\FTNT{Pr. 8:13.} est la principale de la science ; mais les fous méprisent la sagesse et l'instruction.
\VS{8}Mon fils, écoute l'instruction de ton père, et n'abandonne pas l'enseignement\FTNT{Loi vient de Torah (instruction, enseignement, direction etc.).} de ta mère.
\VS{9}Car se sont des grâces enfilées ensemble autour de ta tête, et des colliers autour de ton cou.
\VS{10}Mon fils, si les pécheurs veulent t'attirer, ne t'y accorde pas.
\VS{11}S'ils disent : Viens avec nous, dressons des embûches pour tuer; épions secrètement l'innocent, quoiqu'il ne nous en ai point donné de sujet aucune raison.
\VS{12}Engloutissons-les tout vifs, comme le scheol ; et tout entiers, comme ceux qui descendent dans la fosse ;
\VS{13}nous trouverons toutes sortes de biens précieux, nous remplirons nos maisons de butin ;
\VS{14}tu auras ta part avec nous, il n'y aura qu'une bourse pour nous tous.
\VS{15}Mon fils, ne te mets point en chemin avec eux ; retires ton pied de leur sentier ;
\VS{16}parce que leurs pieds courent au mal, et se hâtent pour répandre le sang\FTNT{Es. 59:7.}.
\VS{17}Car c'est en vain qu'on jette le filet devant les yeux de tout Baal ailé\FTNT{Ec. 10:20.} ;
\VS{18}ainsi ceux-ci dressent des embûches contre le sang de ceux-là et épient secrètement leurs vies.
\VS{19}Tel est le train de tout homme convoiteux de gain déshonnête, qui ôte la vie à ceux qui y sont adonnés.
\TextTitle{La sagesse crie}
\VS{20}La souveraine sagesse crie hautement au-dehors, elle fait retentir sa voix dans les rues.
\VS{21}Elle crie dans les carrefours, là où on fait le plus de bruit, aux entrées des portes, elle prononce ses paroles dans la ville :
\VS{22}Stupides, dit-elle, jusqu'à quand aimerez-vous la stupidité ? Et jusqu'à quand les moqueurs prendront-ils plaisir à la moquerie, et les haïront-ils la connaissance ?
\VS{23}Etant repris par moi, convertissez-vous ; voici, je vous donnerai de mon Esprit en abondance, et je vous ferai connaître mes paroles.
\VS{24}Parce que je crie, et que vous refusez d'entendre ; parce que j'étends ma main, et que personne n'y prend garde ;
\VS{25}et parce que vous rejetez tout mon conseil, et que vous n'avez accepté je vous reprenne ;
\VS{26}moi aussi je rirai quand vous serez dans le malheur, je me moquerai quand la terreur viendra sur vous.
\VS{27}Quand votre effroi surviendra comme une ruine, et que votre calamité viendra comme un tourbillon; vous enveloppera comme un tourbillon ; quand la détresse et l'angoisse viendront  sur vous ;
\VS{28}alors on criera vers moi, mais je ne répondrai point ; on me cherchera de grand matin, mais on ne me trouvera pas\FTNT{De. 31:18 ; Job. 35:12.}.
\VS{29}Parce qu'ils auront haï la connaissance, et qu'ils n'auront point choisi la crainte de Yahweh.
\VS{30}Ils n'ont point aimé mon conseil ; ils ont rejeté toutes mes réprimandes.
\VS{31}Qu'ils mangent donc le fruit de leur voie, et qu'ils se rassasient de leurs conseils.
\VS{32}Car l'égarement des sots les tue, et la prospérité des insensés les perd.
\VS{33}Mais celui qui m'écoute habitera en sécurité et sera tranquille, sans être effrayé d'aucun mal.
\Chap{2}
\TextTitle{La sagesse nous libère du mal}
\VerseOne{}Mon fils, si tu reçois mes paroles, et que tu gardes précieusement en toi mes commandements,
\VS{2}si tu rends ton oreille attentive à la sagesse, et que tu inclines ton cœur à l'intelligence ;
\VS{3}si tu appelles à toi la sagesse, et que tu adresses ta voix à l'intelligence,
\VS{4}si tu la cherches comme de l'argent, et si tu la recherches soigneusement comme des trésors,
\VS{5}alors tu connaîtras la crainte de Yahweh, et tu trouveras la connaissance de Dieu.
\VS{6}Car Yahweh donne la sagesse, et de sa bouche procède la connaissance et l'intelligence.
\VS{7}Il réserve le salut pour ceux qui sont droits, et il est le bouclier de ceux qui marchent dans l'intégrité,
\VS{8}pour garder les sentiers de la justice ; il gardera la voie de ses bien-aimés.
\VS{9}Alors tu comprendras la justice, le jugement, l'équité, et tout bon chemin.
\VS{10}Si la sagesse vient dans ton coeur, et si la connaissance est agréable à ton âme ;
\VS{11}la réflexion veillera sur toi, et l'intelligence te gardera,
\VS{12}pour te délivrer du mauvais chemin, et de l'homme qui tient de mauvais discours ,
\VS{13}de ceux qui abandonnent les voies de la droiture pour marcher dans les chemins ténébreux,
\VS{14}qui sont joyeux de mal faire, et qui se réjouissent dans la perversité des méchants.
\VS{15}Eux dont les sentiers sont tortueux, et qui dans leur conduite vont de travers.
\VS{16}Afin qu'il te délivre de la femme étrangère\FTNT{La femme étrangère est la prostituée ou l'esprit de Jézabel qui séduit les hommes. Voir Pr. 6:24 ; Pr. 7:5.}, et de la femme d'autrui, dont les paroles sont flatteuses ;
\VS{17}qui abandonne l'ami de sa jeunesse et qui oublie l'alliance de son Dieu.
\VS{18}Car sa maison penche vers la mort, et son chemin mène vers les morts.
\VS{19}Pas un de ceux qui vont vers elle n'en retourne, ni ne reprend les sentiers de la vie.
\VS{20}Ainsi tu marcheras dans la voie des gens de bien, et tu garderas les sentiers des justes.
\VS{21}Car ceux qui sont droits habiteront la terre, les hommes intègres y demeureront.
\VS{22}Mais les méchants seront retranchés de la terre, et ceux qui agissent perfidement seront arrachés.
\Chap{3}
\TextTitle{La sagesse bénit et protège}
\VerseOne{}Mon fils, ne mets pas en oubli mon enseignement, et que ton cœur garde mes commandements.
\VS{2}Car ils t'apportent de longs jours et des années de vie et de paix.
\VS{3}Que la bonté et la vérité ne t'abandonnent pas : Lie-les à ton cou, et écris-les sur la table de ton coeur ;
\VS{4}et tu trouveras la grâce et la prudence au yeux de Dieu et des hommes.
\VS{5}Confie-toi de tout ton coeur en Yahweh et ne t'appuie point sur ton intelligence.
\VS{6}Considère-le dans toutes tes voies et il dirigera tes sentiers.
\VS{7}Ne sois point sage à tes yeux ; crains Yahweh, et détourne-toi du mal.
\VS{8}Ce sera la guérison de ton nombril et un rafraîchissement pour tes os.
\VS{9}Honore Yahweh avec tes biens et les prémices de tout ton revenu\FTNT{De. 12:6.} :
\VS{10}Alors tes greniers seront remplis d'abondance, et tes cuves regorgeront de vin nouveau.
\VS{11}Mon fils, ne rebute pas l'instruction de Yahweh, et ne te fâche pas de ce qu'il te reprend.
\VS{12}Car Yahweh châtie\FTNT{Hé. 12:4-11.} celui qu'il aime, comme un père le fils auquel il prend plaisir.
\VS{13}Heureux l'homme qui a trouvé la sagesse, et l'homme qui possède l'intelligence !
\VS{14}Car le trafic qu'on peut faire d'elle est meilleur que le trafic l'argent; et le profit qu'on en tire est meilleur que l'or fin.
\VS{15}Elle est plus précieuse que les perles, et toutes tes choses désirables ne la valent point.
\VS{16}Il y a de longs jours dans sa main droite, des richesses et de la gloire en sa gauche.
\VS{17}Ses voies sont des voies agréables, et tous ses sentiers ne sont que paix.
\VS{18}Elle est l'arbre de vie pour ceux qui l'embrassent ; et tous ceux qui la tiennent sont heureux\FTNT{Ge. 2:9 ; Ap. 22:2.}.
\VS{19}Yahweh a fondé la terre par la sagesse, il a disposé les cieux par l'intelligence.
\VS{20}C'est par sa science que les abîmes se sont ouverts, et que les nuages distillent la rosée.
\VS{21}Mon fils, que ces enseignements ne s'écartent point de devant tes yeux ; garde la sagesse et la réflexion :
\VS{22}Elles seront la vie de ton âme et l'ornement de ton cou.
\VS{23}Alors tu marcheras avec assurance dans ton chemin, et ton pied ne bronchera pas.
\VS{24}Si tu te couches, tu seras sans crainte, et quand tu seras couché ton sommeil sera doux.
\VS{25}Ne crains ni une terreur soudaine, ni la ruine des méchants, quand elle arrivera.
\VS{26}Car Yahweh sera ton assurance, et il gardera ton pied de toute embûche.
\VS{27}Ne retiens pas le bien à ceux à qui il est dû, quand il est au pouvoir de ta main de le faire\FTNT{Ga. 6:10.}.
\VS{28}Ne dis pas à ton prochain : Va, et reviens, demain je te donnerai ! Quand tu as de quoi donner.
\VS{29}Ne médite pas le mal contre ton prochain, lorsqu'il demeure tranquillement près de toi.
\VS{30}Ne conteste pas sans motif avec quelqu'un, à moins qu'il ne t'ait causé quelque tort\FTNT{Ro. 12:18.}.
\VS{31}Ne porte pas envie à l'homme violent, et ne choisis aucune de ses voies.
\VS{32}Car celui qui va de travers est en abomination à Yahweh ; mais son intimité est pour ceux qui sont justes.
\VS{33}La malédiction de Yahweh est dans la maison du méchant ; mais il bénit la demeure des justes.
\VS{34}Certes il se moque des moqueurs, mais il fait grâce à ceux qui s'humilient.
\VS{35}Les sages hériteront la gloire ; mais la honte élève des insensés.
\Chap{4}
\TextTitle{Instructions et conseils d'un père}
\VerseOne{}Ecoutez, mes fils, l'instruction du père, et soyez attentifs pour connaître l'intelligence.
\VS{2}Car je vous donne une bonne doctrine, ne rejetez donc pas mon enseignement.
\VS{3}J'ai été un fils pour mon père. Un fils tendre et unique auprès de ma mère.
\VS{4}Il m'a enseigné, et m'a dit : Que ton coeur retienne mes paroles ; garde mes commandements et tu vivras.
\VS{5}Acquiers la sagesse, acquiers l'intelligence ; n'oublie pas les paroles de ma bouche, et ne t'en détourne pas.
\VS{6}Ne l'abandonne point, et elle te gardera ; aime-la, et elle te protégera.
\VS{7}La principale chose c'est la sagesse ; donc acquiers la sagesse ; et sur toutes tes acuisitions, acquiers la prudence.
\VS{8}Exalte-la, et elle t'élèvera ; elle te glorifiera quand tu l'auras embrassée.
\VS{9}Elle posera sur ta tête une couronne de grâce, et elle t'ornera d'un magnifique diadème.
\VS{10}Ecoute, mon fils, et reçois mes paroles, ainsi les années de ta vie te seront multipliées.
\VS{11}Je t'ai enseigné le chemin de la sagesse, et je t'ai conduit dans les sentiers de la droiture\FTNT{Ps. 23:3.}.
\VS{12}Quand tu y marcheras, ton pas ne sera pas gêné ; et si tu cours, tu ne chancelleras pas\FTNT{Ps. 121:3.}.
\VS{13}Embrasse l'instruction, ne la lâche pas ; garde-la ; car elle est ta vie.
\VS{14}N'entre pas dans le sentier des méchants, et ne marche pas dans la voie des hommes mauvais.
\VS{15}Détourne-t'en, ne passe pas par là, détourne-t'en, et passe outre.
\VS{16}Car ils ne dormiraient pas, s'ils n'avaient fait quelque mal, et le sommeil leur serait ôté, s'ils n'avaient fait tomber quelqu'un.
\VS{17}Parce qu'ils mangent le pain de méchanceté, et qu'ils boivent le vin de la violence.
\VS{18}Mais le sentier des justes est comme la lumière resplendissante, dont l'éclat augmente jusqu'à ce que le jour soit dans sa perfection.
\VS{19}La voie des méchants est comme l'obscurité ; ils n'aperçoivent pas ce qui les fera tomber.
\VS{20}Mon fils, sois attentif à mes paroles, incline ton oreille à mes discours.
\VS{21}Qu'ils ne s'écarte pas de tes yeux ; garde-les dans le fond de ton coeur.
\VS{22}Car ils sont la vie pour ceux qui les trouvent, et la santé de tout le corps de chacun d'eux.
\VS{23}Garde ton coeur de tout ce dont il faut se garder ; car de lui procèdent les sources de la vie\FTNT{Mt 12:35 ; Mt. 15:18-19.}.
\VS{24}Eloigne de toi la perversité de la bouche et la dépravation des lèvres.
\VS{25}Que tes yeux regardent droit et que tes paupières dirigent ton chemin devant toi.
\VS{26}Pèse le chemin de tes pieds, et que toutes tes voies soient bien stables.
\VS{27}Ne te tourne ni à droite ni à gauche ; détourne ton pied du mal.
\Chap{5}
\TextTitle{[Se garder de l'immoralité]}
\VerseOne{}Mon fils, sois attentif à ma sagesse, incline ton oreille à mon intelligence ;
\VS{2}afin que tu gardes mes avis, et que tes lèvres conservent la connaissance.
\VS{3}Car les lèvres de l'étrangère distillent des rayons de miel, et son palais est plus doux que l'huile.
\VS{4}Mais ce qui en provient est amère comme de l'absinthe, et aigu comme une épée à deux tranchants.
\VS{5}Ses pieds descendent à la mort, ses pas atteignent le scheol.
\VS{6}Afin que tu ne balance pas sur le chemin de la vie car, ses chemins en sont écartés; tu ne le connaîtras pas.
\VS{7}Maintenant donc, fils, écoutez-moi, et ne vous détournez pas des paroles de ma bouche.
\VS{8}Eloigne ton chemin de la femme étrangère et n'approche pas de l'entrée de sa maison.
\VS{9}De peur que tu ne donnes ton honneur à d'autres, et tes années à un homme cruel.
\VS{10}De peur que les étrangers ne se rassasient de tes biens, et que le fruit de ton travail ne soit dans la maison d'un étranger.
\VS{11}De peur que tu ne gémisses quand tu seras près de ta fin, quand ta chair et ton corps seront consumés ;
\VS{12}et que tu ne dises : Comment donc ai-je pu haïr la correction, et comment mon coeur a-t-il dédaigné les réprimandes ?
\VS{13}Et comment n'ai-je point obéi à la voix de ceux qui m'instruisaient, et n'ai-je point incliné mon oreille à ceux qui m'enseignaient ?
\VS{14}Peu s'en est fallu que je n'aie été dans toute sorte de mal, au milieu du peuple et de l'assemblée.
\VS{15}Bois des eaux de ta citerne et de ce qui coule du milieu de ton puits ;
\VS{16}que tes sources se répandent dehors, et les ruisseaux d'eau sur les rues ;
\VS{17}qu'elles soient à toi seul, et non aux étrangers avec toi.
\VS{18}Que ta source soit bénie, et réjouis-toi de la femme de ta jeunesse,
\VS{19}comme d'une biche des amours, et d'une chevrette gracieuse ; que ses mamelles te rassasient en tout temps, et sois continuellement épris de son amour.
\VS{20}Et pourquoi, mon fils, irais-tu errant après l'étrangère et embrasserais-tu le sein de l'inconnue ?
\VS{21}Vu que les voies de l'homme sont devant les yeux de Yahweh et qu'il pèse toutes ses voies\FTNT{Jé 16:17 ; Hé. 4:13.}.
\VS{22}Les iniquités du méchant l'attraperont, et il sera retenu par les cordes de son péché.
\VS{23}Il mourra faute d'instruction et il s'égarera par l'excès de sa folie.
\Chap{6}
\TextTitle{Recommandations diverses}
\VerseOne{}Mon fils, si tu t'es porté caution pour ton prochain, si tu as engagé ta main pour un étranger,
\VS{2}tu es enlacé par les paroles de ta bouche, tu es pris par les paroles de ta bouche.
\VS{3}Mon fils, fais maintenant ceci, et dégage-toi, puisque tu es tombé entre les mains de ton intime ami, va, prosterne-toi, et importune tes amis.
\VS{4}Ne donne point de sommeil à tes yeux et ne laisse point sommeiller tes paupières.
\VS{5}Dégage-toi comme la gazelle de la main du chasseur, et comme l'oiseau de la main de l'oiseleur.
\VS{6}Va, paresseux, vers la fourmi, regarde ses voies, et sois sage.
\VS{7}Elle n'a ni chef, ni directeur, ni gouverneur,
\VS{8}et cependant elle prépare en été son pain, et amasse durant la moisson de quoi manger.
\VS{9}Paresseux, jusqu'à quand resteras-tu couché ? Quand te lèveras-tu de ton sommeil ?
\VS{10}Un peu de sommeil, dis-tu, un peu d'assoupissement, un peu croiser les mains afin de rester couché ;
\VS{11}et ta pauvreté viendra comme un voyageur, et ta disette comme un soldat.
\VS{12}Celui qui marche, la fausseté dans sa bouche, est un homme de Bélial\FTNT{1 S. 2:12.}, un homme inique.
\VS{13}Il cligne des yeux, parle du pied, enseigne de ses doigts.
\VS{14}Il y a la perversité dans son cœur, il machine du mal en tout temps, il fait naître des querelles.
\VS{15}C'est pourquoi sa calamité viendra subitement, il sera subitement brisé, il n'y aura point de guérison.
\VS{16}Il y a six choses que Yahweh hait, et il y en a sept qui sont en abomination à son âme ;
\VS{17}savoir, les yeux hautains\FTNT{Ps. 101:5.}, la langue mensongère\FTNT{Ps. 120:2-3.}, les mains qui répandent le sang innocent\FTNT{Es. 1:15.},
\VS{18}le coeur qui médite des projets iniques\FTNT{Ps. 36:5.}, les pieds qui se hâtent de courir au mal\FTNT{Es. 59:7.},
\VS{19}le faux témoin qui profère des mensonges\FTNT{Ps. 27:12.}, et celui qui sème des querelles entre les frères\FTNT{Jud. 1:16-19.}.
\VS{20}Mon fils, garde le commandement de ton père, et n'abandonne pas l'enseignement de ta mère ;
\VS{21}attache-les continuellement à ton coeur, lie-les à ton cou.
\VS{22}Quand tu marcheras, il te conduira ; et quand tu te coucheras, il te gardera ; et quand tu te réveilleras, il s'entretiendra avec toi.
\VS{23}Car le commandement est une lampe ; et l'enseignement une lumière\FTNT{Ps. 119:105.} ; et les réprimandes propres à instruire sont le chemin de la vie.
\VS{24}Ils te préserveront de la mauvaise femme, de la langue doucereuse de l'étrangère.
\VS{25}Ne la convoite pas dans ton coeur pour sa beauté, et ne te laisse pas prendre par ses yeux\FTNT{Mt. 5:28.}.
\VS{26}Car pour l'amour de la femme prostituée on est réduit à un morceau de pain, et la femme adultère chasse après l'âme précieuse de l'homme.
\VS{27}Un homme peut-il prendre du feu dans son sein, sans que ses habits brûlent ?
\VS{28}Un homme marchera-t-il sur des charbons ardents, sans que ses pieds en soient brûlés ?
\VS{29}Il en est de même pour celui qui va vers la femme de son prochain ; quiconque la touchera ne restera pas impuni.
\VS{30}On ne méprise pas un voleur, s'il vole pour satisfaire son âme quand il a faim ;
\VS{31}si on le trouve, il rendra sept fois autant, il donnera tout ce qu'il a dans sa maison.
\VS{32}Mais celui qui commet un adultère avec une femme est dépourvu de sens ; et celui qui le fera, détruira son âme.
\VS{33}Il trouvera des plaies et de l'ignominie, et son opprobre ne sera pas effacé.
\VS{34}Car la jalousie d'un mari est une fureur, il n'épargnera pas l'adultère au jour de la vengeance.
\VS{35}Il n'aura égard à aucune rançon, et il n'acceptera rien, quand même tu multiplierais les présents.
\Chap{7}
\TextTitle{Mise en garde contre la femme prostituée}
\VerseOne{}Mon fils, observe mes paroles, et garde avec toi mes commandements.
\VS{2}Garde mes commandements, et tu vivras, garde mes enseignements comme la prunelle de tes yeux\FTNT{Lé. 18:5.}.
\VS{3}Lie-les sur tes doigts, écris-les sur la table de ton coeur.
\VS{4}Dis à la sagesse : Tu es ma soeur ; et appelle l'intelligence, ton amie.
\VS{5}Afin qu'elles te préservent de la femme étrangère, de l'étrangère qui emploie des paroles doucereuses.
\VS{6}Comme je regardais de la fenêtre de ma maison à travers mon treillis,
\VS{7}je vis parmi les stupides, et je remarquai parmi les jeunes gens un jeune homme dépourvu de sens.
\VS{8}Il passait dans la rue, près de l'angle où se tenait une de ces femmes, et qui suivait le chemin de sa maison,
\VS{9}au crépuscule, au soir du jour, au milieu de la nuit et de l'obscurité.
\VS{10}Et voici, il fut abordé par une femme, vêtue en tenue de prostituée, et pleine de ruse dans le cœur.
\VS{11}Elle était bruyante et rebelle, ses pieds ne restaient point dans sa maison ;
\VS{12}tantôt dehors, tantôt sur les places, elle était aux aguets à chaque coin de rue.
\VS{13}Elle le saisit, et l'embrassa ; et avec un visage effronté, lui dit :
\VS{14}J'ai chez moi des sacrifices d'offrande de paix ; j'ai aujourd'hui accompli mes voeux.
\VS{15}C'est pourquoi je suis sortie à ta rencontre pour chercher ton visage, et je t'ai trouvé.
\VS{16}J'ai orné mon lit de couvertures, d'étoffes de fil d'Egypte.
\VS{17}J'ai parfumé ma couche de myrrhe, d'aloès et de cinnamome.
\VS{18}Viens, enivrons-nous de plaisir jusqu'au matin, réjouissons-nous en amours.
\VS{19}Car mon mari n'est point à la maison, il est parti pour un voyage lointain.
\VS{20}Il a pris un sac d'argent dans sa main, il ne reviendra à la maison qu'à la nouvelle lune.
\VS{21}Elle l'a fait détourner par beaucoup de douces paroles, et l'a attiré par la flatterie de ses lèvres.
\VS{22}Il s'en alla aussitôt après elle, comme un boeuf qui va à la boucherie, comme le fou qu'on lie pour être châtié ;
\VS{23}jusqu'à ce que la flèche lui ait transpercé le foie ; comme l'oiseau qui se hâte vers le filet, sans savoir que c'est au prix de sa vie.
\VS{24}Maintenant donc, fils, écoutez-moi, et soyez attentifs aux paroles de ma bouche.
\VS{25}Que ton coeur ne se détourne pas vers les voies d'une telle femme, ne t' égare pas dans ses sentiers.
\VS{26}Car elle a fait tomber plusieurs blessés à mort, et tous ceux qu'elle a tués sont nombreux.
\VS{27}Sa maison est le chemin du scheol, qui descend vers les demeures de la mort.
\Chap{8}
\TextTitle{[La sagesse préférable aux richesses]}
\VerseOne{}La sagesse ne crie-t-elle pas ? Et l'intelligence ne fait-elle pas entendre sa voix ?
\VS{2}Elle s'est présentée sur le sommet des lieux élevés, sur le chemin, aux carrefours.
\VS{3}Elle crie près des portes, devant la ville, à l'entrée des portes,
\VS{4}ô vous ! Hommes de qualité, je vous appelle ; et ma voix s'adresse aussi aux fils des hommes.
\VS{5}Vous stupides, apprenez le discernement, et vous tous, devenez intelligents de coeur.
\VS{6}Écoutez, car je dirai des choses importantes : Et j'ouvrirai mes lèvres pour enseigner des choses droites.
\VS{7}Parce que ma bouche proclame la vérité, et mes lèvres ont en horreur le mensonge.
\VS{8}Tous les discours de ma bouche sont selon la justice, il n'y a rien en eux de faux, ni de déformé.
\VS{9}Ils sont tous clairs à l'homme intelligent, et droits pour ceux qui ont trouvé la connaissance.
\VS{10}Recevez mon instruction plutôt que de l'argent, la connaissance à l'or le plus précieux.
\VS{11}Car la sagesse vaut mieux que les perles, et tout ce qu'on pourrait souhaiter ne la vaut pas\FTNT{Ps. 19:11 ; Ps. 119:127 ; Job. 28:18.}.
\VS{12}Moi, la Sagesse, j'habite avec le discernement, et je possède la connaissance de la réflexion.
\VS{13}La crainte de Yahweh c'est la haine du mal. Je hais l'orgueil et l'arrogance, la voie du mal, et la bouche perverse.
\VS{14}A moi appartiennent le conseil et le succès ; je suis l'intelligence, à moi appartient la force.
\VS{15}Par moi règnent les rois, et par moi les princes décrètent ce qui est juste.
\VS{16}Par moi gouvernent les seigneurs, les princes, et tous les juges de la terre.
\VS{17}J'aime ceux qui m'aiment ; et ceux qui me cherchent soigneusement me trouveront\FTNT{Mt. 7:7 ; Lu. 11:9 ; Jn 14:23-24.}.
\VS{18}Avec moi sont la richesse et la gloire, les biens durables et la justice.
\VS{19}Mon fruit est meilleur que le fin or, même que l'or raffiné ; et mon revenu est meilleur que l'argent choisi.
\VS{20}Je marche dans le chemin de la justice, au milieu des sentiers de la droiture ;
\VS{21}pour donner des biens en héritage à ceux qui m'aiment, et pour remplir leurs trésors.
\VS{22}Yahweh m'a acquise dès le commencement de ses voies, avant ses œuvres les plus anciennes.
\VS{23}J'ai été déclarée princesse depuis l'éternité, dès le commencement, avant l'origine de la terre.
\VS{24}J'ai été engendrée lorsqu'il n'y avait point encore d'abîmes, ni de sources chargées d'eaux.
\VS{25}Avant que les montagnes soient affermies, avant que les collines existent, j'ai été engendrée.
\VS{26}Lorsqu'il n'avait pas encore fait la terre et les campagnes, et le commencement de la poussière du monde habitable.
\VS{27}Lorsqu'il disposa les cieux, j'étais là ; lorsqu'il traça un cercle à la surface de l'abîme ;
\VS{28}lorsqu'il fixa les nuages en haut ; et que les sources de l'abîme jaillirent avec force ;
\VS{29}lorsqu'il donna une limite à la mer, pour que les eaux ne franchissent pas les bords ; lorsqu'il posa les fondements de la terre,
\VS{30}j'étais à l'œuvre auprès de lui, je faisais ses délices tous les jours, et toujours j'étais en joie en sa présence.
\VS{31}Je me réjouissais dans la partie habitable de sa terre, trouvant mes délices avec les fils de l'homme.
\VS{32}Maintenant donc, mes fils, écoutez-moi : Heureux sont ceux qui observent mes voies.
\VS{33}Ecoutez l'instruction, et soyez sages, et ne la rejetez point.
\VS{34}Ô ! Heureux est l'homme qui m'écoute, qui veille chaque jour à mes portes. Et qui monte la garde aux montants de mes portes !
\VS{35}Car celui qui me trouve a trouvé la vie, et obtient la faveur de Yahweh.
\VS{36}Mais celui qui pèche contre moi nuit à son âme ; tous ceux qui me haïssent aiment la mort.
\Chap{9}
\TextTitle{[La sagesse, source de vie]}
\VerseOne{}La Souveraine Sagesse a bâti sa maison, elle a taillé ses sept colonnes.
\VS{2}Elle a apprêté sa viande, elle a mêlé son vin ; elle a aussi dressé sa table.
\VS{3}Elle a envoyé ses servantes, elle crie du haut des lieux les plus élevés de la ville, disant : 
\VS{4}Que celui qui est stupide, entre ici ; et elle dit à ceux qui sont dépourvus de sens :
\VS{5}Venez, mangez de mon pain, et buvez du vin que j'ai mêlé.
\VS{6}Abandonnez la stupidité, et vous vivrez ; et marchez droit dans la voie de l'intelligence.
\VS{7}Celui qui instruit le moqueur, en reçoit de l'ignominie ; et celui qui reprend le méchant en reçoit une tache.
\VS{8}Ne reprends point le moqueur, de crainte qu'il ne te haïsse ; reprends le sage, et il t'aimera\FTNT{Ps. 141:5.}.
\VS{9}Donne l'instruction au sage, et il deviendra encore plus sage ; enseigne le juste, et il croîtra en science.
\VS{10}Le commencement de la sagesse est la crainte de Yahweh\FTNT{Ps. 19:10.} ; et la connaissance des saints, c'est l'intelligence.
\VS{11}Car tes jours se multiplieront par moi, et les années de vie augmenteront.
\VS{12}Si tu es sage, tu es sage pour toi-même ; si tu es moqueur, tu en porteras seul la peine.
\VS{13}La femme folle est bruyante, stupide et elle ne connaît rien.
\VS{14}Et elle s'assied à la porte de sa maison sur un siège, dans les lieux élevés de la ville ;
\VS{15}pour appeler les passants qui vont droit leur chemin, disant :
\VS{16}Que celui qui est stupide entre ici ! Elle dit à celui qui est dépourvu de sens :
\VS{17}Les eaux dérobées sont douces, et le pain pris en secret est agréable.
\VS{18}Et il ne sait pas que là sont les défunts, et que ceux qu'elle a conviés sont dans le scheol.
\Chap{10}
\TextTitle{[La justice s'oppose à la méchanceté]}
\VerseOne{}Proverbes de Salomon. Le fils sage réjouit son père, mais le fils insensé est l'ennui de sa mère.
\VS{2}Les trésors de méchanceté ne profitent pas, mais la justice délivre de la mort.
\VS{3}Yahweh ne laisse pas l'âme du juste avoir faim, mais il repousse au loin l'avidité des méchants.
\VS{4}Celui qui agit d'une main nonchalante s'appauvrit, mais la main des diligents enrichit.
\VS{5}L'enfant prudent amasse en été, mais celui qui dort durant la moisson est un enfant qui fait honte.
\VS{6}Les bénédictions seront sur la tête du juste, mais la violence couvrira la bouche des méchants.
\VS{7}La mémoire du juste est en bénédiction\FTNT{Ps 112:6.}, mais la réputation des méchants tombe en pourriture.
\VS{8}Celui qui est sage de coeur reçoit les commandements, mais celui qui est insensé des lèvres, tombera.
\VS{9}Celui qui marche dans l'intégrité marche avec assurance, mais celui qui pervertit ses voies, sera connu.
\VS{10}Celui qui cligne de l'oeil cause du chagrin, et celui qui a les lèvres insensées sera renversé.
\VS{11}La bouche du juste est une source de vie, mais la cruauté couvre la bouche des méchants.
\VS{12}La haine excite les querelles, mais la charité couvre toutes les fautes\FTNT{1 Pi 4:8.}.
\VS{13}La sagesse se trouve sur les lèvres de l'homme intelligent, mais la verge est pour le dos de celui qui est dépourvu de sens.
\VS{14}Les sages tiennent la connaissance en réserve, mais la bouche de l'insensé est une ruine prochaine.
\VS{15}Les biens du riche sont la ville de sa force, mais la pauvreté des misérables est leur ruine.
\VS{16}L'oeuvre du juste est pour la vie, mais le revenu du méchant est pour le péché.
\VS{17}Celui qui garde l'instruction est dans le chemin de la vie, mais celui qui néglige la correction s'y égare.
\VS{18}Celui qui dissimule la haine a des lèvres menteuses, et celui qui répand la calomnie est un insensé.
\VS{19}Dans la multitude de paroles le péché ne manque pas, mais celui qui retient ses lèvres est prudent.
\VS{20}La langue du juste est un argent de choix, mais le coeur des méchants est bien peu de chose.
\VS{21}Les lèvres du juste en instruisent plusieurs, mais les insensés mourront faute de sens.
\VS{22}La bénédiction de Yahweh est celle qui enrichit, et il n'y ajoute aucune peine.
\VS{23}C'est comme un jeu à un insensé de pratiquer l'infamie, mais la sagesse appartient à l'homme intelligent.
\VS{24}Ce que redoute le méchant, c'est ce qui lui arrive ; mais Dieu accorde aux justes ce qu'ils désirent.
\VS{25}Comme le tourbillon passe, ainsi le méchant n'est plus ; mais le juste est un fondement perpétuel.
\VS{26}Ce qu'est le vinaigre aux dents et la fumée aux yeux, tel est le paresseux à ceux qui l'envoient.
\VS{27}La crainte de Yahweh augmente les jours, mais les années des méchants sont raccourcies.
\VS{28}L'espérance des justes n'est que joie, mais l'espérance des méchants périra.
\VS{29}La voie de Yahweh est le refuge de l'homme intègre, mais elle est la ruine pour ceux qui pratiquent l'iniquité.
\VS{30}Le juste ne sera jamais ébranlé, mais les méchants ne demeureront pas sur la terre.
\VS{31}La bouche du juste produit la sagesse, mais la langue perverse sera retranchée.
\VS{32}Les lèvres du juste connaissent ce qui est agréable ; mais la bouche des méchants n'est que perversité.
\Chap{11}
\TextTitle{[La justice s'oppose à la méchanceté (suite)]}
\VerseOne{}La fausse balance est une abomination à Yahweh, mais le poids juste lui est agréable\FTNT{Lé 19:35-36 ; De. 25:13-16.}.
\VS{2}Quand l'orgueil vient, la honte vient aussi ; mais la sagesse est avec ceux qui sont modestes.
\VS{3}L'intégrité des hommes droits les conduit, mais la perversité des perfides les détruit.
\VS{4}Les richesses ne servent à rien au jour de la colère, mais la justice délivre de la mort.
\VS{5}La justice de l'homme intègre rend droite sa voie, mais le méchant tombe par sa méchanceté.
\VS{6}La justice des hommes droits les délivre, mais les perfides sont pris par leur méchanceté.
\VS{7}Quand l'homme méchant meurt, son espoir périt ; et l'espérance des hommes iniques périt.
\VS{8}Le juste est délivré de la détresse, et le méchant y entre à sa place.
\VS{9}Par sa bouche l'impie corrompt son prochain, mais les justes en sont délivrés par la connaissance.
\VS{10}La ville se réjouit quand les justes sont heureux, et quand les méchants périssent, c'est un triomphe.
\VS{11}La ville est élevée par la bénédiction des hommes droits, mais elle est renversée par la bouche des méchants.
\VS{12}Celui qui méprise son prochain est dépourvu de sens, mais l'homme prudent se tait.
\VS{13}Celui qui va rapportant, révèle les secrets, mais celui qui a l'esprit qui supporte les paroles, les couvre.
\VS{14}Le peuple tombe par faute de prudence, mais la délivrance est dans la multitude de conseillers.
\VS{15}Celui qui se porte garant pour un étranger en souffrira, et celui qui hait le cautionnement est assuré.
\VS{16}La femme gracieuse obtient de l'honneur, et les hommes robustes obtiennent les richesses.
\VS{17}L'homme doux fait du bien à son âme, mais le cruel trouble sa chair.
\VS{18}Le méchant fait une oeuvre qui le trompe, mais la récompense est assurée à celui qui sème la justice\FTNT{Os. 10:12.}.
\VS{19}Ainsi la justice conduit à la vie, mais celui qui poursuit le mal aboutit à sa mort.
\VS{20}Ceux qui ont le cœur pervers sont en abomination à Yahweh, mais ceux qui sont intègres dans leurs voies lui sont agréables.
\VS{21}De main en main le méchant ne demeurera point impuni, mais la race des justes sera délivrée.
\VS{22}Une belle femme qui se détourne de la raison est comme un anneau d'or au nez d'un pourceau.
\VS{23}Le souhait des justes n'est que le bien, mais l'attente des méchants c'est l'indignation.
\VS{24}Tel, qui donne libéralement, devient plus riche ; et tel qui épargne à l'excès ne fait que s'appauvrir.
\VS{25}Celui qui bénit sera engraisssé ; et celui qui arrose abondamment sera lui-même arrosé.
\VS{26}Sera maudit du peuple, celui qui cache le froment, mais la bénédiction est sur la tête de celui qui le vend.
\VS{27}Qui recherche le bien cherche la faveur, mais le mal arrive à qui le recherche.
\VS{28}Celui qui se confie dans ses richesses tombera, mais les justes verdiront comme le feuillage\FTNT{Ps. 1:3 ; Jé 17: 8.}.
\VS{29}Celui qui ne gouverne pas sa maison avec ordre, aura le vent pour héritage, et le fou sera le serviteur de celui qui a le coeur sage.
\VS{30}Le fruit du juste est un arbre de vie, et celui qui gagne les âmes est sage.
\VS{31}Voici, le juste reçoit sur la terre sa rétribution, combien plus le méchant et le pécheur la recevront-ils ?
\Chap{12}
\TextTitle{[La justice s'oppose à la méchanceté (suite)]}
\VerseOne{}Celui qui aime la correction aime la connaissance, mais celui qui hait la réprimande est un stupide.
\VS{2}L'homme de bien obtient la faveur de Yahweh, mais Yahweh condamne l'homme qui a des mauvaises pensées.
\VS{3}L'homme ne sera point affermi par la méchanceté, mais la racine des justes ne sera point ébranlée.
\VS{4}La femme vertueuse est la couronne de son mari\FTNT{Pr. 31:10.}, mais celle qui fait honte est comme la pourriture dans ses os.
\VS{5}Les pensées des justes ne sont que jugement, mais les conseils des méchants ne sont que fraude.
\VS{6}Les paroles des méchants ne tendent qu'à dresser des embûches pour répandre le sang, mais la bouche des hommes droits les délivrera.
\VS{7}Les méchants sont renversés, et ils ne sont plus, mais la maison des justes se maintiendra.
\VS{8}L'homme est estimé en raison de sa prudence, mais celui qui a le coeur pervers est l'objet du mépris.
\VS{9}Mieux vaut l'homme qui ne fait pas cas de lui-même, bien qu'il ait des serviteurs, que celui qui se glorifie, et qui manque de pain.
\VS{10}Le juste a égard à la vie de sa bête, mais les entrailles des méchants sont cruelles.
\VS{11}Celui qui cultive son champ sera rassasié de pain, mais celui qui court après des futilités est dépourvu de sens.
\VS{12}Ce que le méchant désire, est un filet des hommes mauvais, mais la racine des justes donnera son fruit.
\VS{13}Il y a dans le péché des lèvres un piège pernicieux, mais le juste sortira de la détresse.
\VS{14}L'homme sera rassasié de biens par le fruit de sa bouche, et on rendra à l'homme la rétribution de ses mains.
\VS{15}La voie de l'insensé est droite à son opinion, mais celui qui écoute le conseil est sage.
\VS{16}Quand à l'insensé, sa colère est révélée le jour même, mais l'homme bien avisé couvre son ignominie.
\VS{17}Celui qui prononce des choses véritables rend un témoignage juste, mais le faux témoin fait des rapports trompeurs.
\VS{18}Il y a tel homme dont les paroles blessent comme des pointes d'épée, mais la langue des sages apporte la guérison.
\VS{19}La lèvre véridique est affermie pour toujours, mais la fausse langue n'est que pour un moment\FTNT{Ps. 52: 6-7.}.
\VS{20}Il y a de la tromperie dans le coeur de ceux qui méditent le mal, mais il y a de la joie pour ceux qui conseillent la paix.
\VS{21}Il n'arrivera aucun outrage aux justes, mais les méchants seront remplis de mal.
\VS{22}Les fausses lèvres sont une abomination à Yahweh\FTNT{Ap. 22:15.}, mais ceux qui agissent fidèlement lui sont agréables.
\VS{23}L'homme bien avisé cache sa connaissance, mais le coeur des insensés publie la folie.
\VS{24}La main des diligents dominera, mais la main paresseuse sera tributaire.
\VS{25}Le chagrin qui est au cœur de l'homme, l'accable ; mais la bonne parole le réjouit.
\VS{26}Le juste a plus de reste que son voisin, mais la voie des méchants les égare.
\VS{27}L'homme paresseux ne rôtit point son gibier ; mais les biens précieux de l'homme sont au diligent.
\VS{28}La vie est dans le chemin de la justice, et la voie de son sentier ne tend point à la mort.
\Chap{13}
\TextTitle{[La justice s'oppose à la méchanceté (suite)]}
\VerseOne{}Un fils sage écoute l'instruction de son père, mais le moqueur n'écoute pas la réprimande\FTNT{Ps. 1:1.}.
\VS{2}L'homme mange du bien par le fruit de sa bouche, mais l'âme de ceux qui agissent perfidement mangent l'injustice.
\VS{3}Celui qui garde sa bouche, garde son âme ; mais celui qui ouvre à tout propos ses lèvres, tombera en ruine\FTNT{Ps. 39:2.}.
\VS{4}L'âme du paresseux a des désirs qu'il ne peut satisfaire, mais l'âme des diligents sera engraissée.
\VS{5}Le juste hait la parole mensongère, mais elle rend le méchant odieux et le fait tomber dans la confusion.
\VS{6}La justice garde celui qui est intègre dans sa voie, mais la méchanceté renversera celui qui s'égare.
\VS{7}Tel fait le riche et n'a rien du tout, tel fait le pauvre et a de grandes fortunes.
\VS{8}Les richesses d'un homme servent de rançon pour sa vie, mais le pauvre n'entend pas des réprimandes.
\VS{9}La lumière des justes remplit de joie, mais la lampe des méchants s'éteint.
\VS{10}L'orgueil ne produit que querelle, mais la sagesse est avec ceux qui écoutent les conseils.
\VS{11}Les richesses provenues de la fraude seront diminuées, mais celui qui amasse peu à peu les augmentera.
\VS{12}Un espoir différé fait languir le cœur, mais un désir accompli est comme un arbre de vie.
\VS{13}Celui qui méprise la parole périra à cause d'elle, mais celui qui craint le commandement en sera récompensé.
\VS{14}L'enseignement du sage est une source de vie, pour se détourner des pièges de la mort.
\VS{15}Le bon sens donne de la grâce ; mais la voie de ceux qui agissent perfidement est raboteuse.
\VS{16}Tout homme bien avisé agira avec connaissance, mais l'insensé fera l'étalage de sa folie\FTNT{Da.11:32}.
\VS{17}Le méchant messager tombe dans le mal, mais l'ambassadeur fidèle apporte la guérison.
\VS{18}La pauvreté et l'ignominie arrivent à celui qui rejette l'instruction, mais celui qui garde la réprimande est honoré.
\VS{19}Le souhait accompli est une chose douce à l'âme, mais se détourner du mal est une abomination aux insensés.
\VS{20}Celui qui marche avec les sages deviendra sage, mais le compagnon des insensés sera accablé.
\VS{21}Le mal poursuit les pécheurs, mais le bien sera rendu aux justes.
\VS{22}L'homme de bien laissera de quoi hériter aux fils de ses fils, mais les richesses du pécheur sont réservées aux justes.
\VS{23}Il y a beaucoup à manger dans les terres défrichées des pauvres, mais il y a tel qui est consumé faute de règles.
\VS{24}Celui qui épargne sa verge hait son fils, mais celui qui l'aime se hâte de le châtier.
\VS{25}Le juste mangera jusqu'à être rassasié à son souhait, mais le ventre des méchants aura la disette.
\Chap{14}
\TextTitle{[La justice s'oppose à la méchanceté (suite)]}
\VerseOne{}Toute femme sage bâtit sa maison, mais la folle la ruine de ses mains.
\VS{2}Celui qui marche dans la droiture craint Yahweh, mais celui dont les voies sont perverses le méprise.
\VS{3}La verge d'orgueil est dans la bouche de l'insensé, mais les lèvres des sages les garderont.
\VS{4}Où il n'y a point de boeuf, la grange est vide ; et l'abondance du revenu provient de la force du boeuf.
\VS{5}Le témoin véritable ne ment jamais, mais le faux témoin avance volontiers des mensonges.
\VS{6}Le moqueur cherche la sagesse et ne la trouve pas, mais la connaissance est aisée à trouver pour l'homme intelligent.
\VS{7}Eloigne-toi de l'homme insensé, puisque tu n'as pas trouvé sur ses lèvres la connaissance.
\VS{8}La sagesse d'un homme avisé est de connaître les règles de sa voie, mais la folie des insensés est la tromperie.
\VS{9}Les insensés se moquent du péché, mais parmi les hommes droits se trouve la bienveillance.
\VS{10}Le cœur d'un chacun connaît l'amertume de son âme, et un autre ne saurait partager sa joie.
\VS{11}La maison des méchants sera abolie, mais la tente des hommes droits fleurira.
\VS{12}Il y a telle voie qui semble droite à l'homme, mais dont l'issue sont les voies de la mort.
\VS{13}Même en riant le coeur sera triste, et la joie finit par l'ennui.
\VS{14}Celui qui a un cœur hypocrite, sera rassasié de ses voies ; mais l'homme de bien de ce qui est en lui.
\VS{15}Le simple croit à toute parole ; mais l'homme bien avisé considère ses pas.
\VS{16}Le sage craint et se retire du mal, mais l'insensé se met en colère et est confiant.
\VS{17}Celui qui est prompt à la colère agit follement\FTNT{Ps. 37:8.}, et l'homme plein de ruse est haï.
\VS{18}Les naïfs hériteront la folie ; mais les prudents seront couronnés de connaissance.
\VS{19}Les malins seront humiliés devant les bons, et les méchants, devant les portes du juste.
\VS{20}Le pauvre est haï même de son ami, mais les amis du riche sont en grand nombre.
\VS{21}Celui qui méprise son prochain commet un péché, mais celui qui a pitié des pauvres affligés est heureux.
\VS{22}Ceux qui méditent le mal ne s'égarent-ils pas ? Mais la bonté et la vérité sont pour ceux qui méditent le bien.
\VS{23}En tout travail il y a quelque profit, mais les vains discours ne tournent qu'à la disette.
\VS{24}Les richesses des sages leur sont comme une couronne, mais la stupidité des insensés est toujours stupidité.
\VS{25}Le témoin fidèle délivre les âmes, mais celui qui prononce des mensonges est trompeur.
\VS{26}En la crainte de Yahweh il y a une ferme assurance, et une retraite pour ses fils.
\VS{27}La crainte de Yahweh est une source de vie pour se détourner des pièges de la mort.
\VS{28}La gloire d'un roi, c'est la multitude du peuple, mais quand le peuple manque, c'est la ruine du prince.
\VS{29}Celui qui est lent à la colère a une grande intelligence, mais celui qui est prompt à s'emporter excite la folie.
\VS{30}Un coeur sain est la vie de la chair, mais l'envie est la pourriture des os.
\VS{31}Celui qui fait tort au pauvre déshonore celui qui l'a fait, mais celui qui a pitié de l'indigent honore Yahweh\FTNT{De. 24:11 ; Ps. 107:41.}.
\VS{32}Le méchant est chassé par sa malice, mais le juste trouve un refuge même dans sa mort.
\VS{33}La sagesse repose au coeur de l'homme intelligent, et elle est même reconnue au milieu des insensés.
\VS{34}La justice élève une nation, mais le péché est l'ignominie des peuples.
\VS{35}Le roi prend plaisir au serviteur prudent, mais son indignation sera contre celui qui lui fait honte.
\Chap{15}
\TextTitle{[La justice s'oppose à la méchanceté (suite)]}
\VerseOne{}La réponse douce apaise la fureur ; mais la parole douloureuse excite la colère
\VS{2}La langue des sages se réjouit de la connaissance, mais la bouche des insensés profère la sottise.
\VS{3}Les yeux de Yahweh sont en tous lieux, observant les méchants et les bons.
\VS{4}La langue qui corrige le prochain est comme l'arbre de vie, mais celle où il y a de la perversité est comme une brèche dans l'esprit.
\VS{5}L'insensé méprise l'instruction de son père, mais celui qui prend garde à la réprimande agit avec prudence.
\VS{6}Il y a un grand trésor dans la maison du juste, mais il y a du trouble dans les revenus du méchant.
\VS{7}Les lèvres des sages répandent partout la connaissance, mais le coeur des insensés ne fait pas ainsi.
\VS{8}Le sacrifice des méchants est en abomination à Yahweh, mais la requête des hommes droits lui est agréable.
\VS{9}La voie du méchant est en abomination à Yahweh, mais il aime celui qui poursuit soigneusement la justice.
\VS{10}Le châtiment est fâcheux à celui qui quitte le droit chemin, mais celui qui hait d'être repris, mourra.
\VS{11}Le schéol et le gouffre sont devant Yahweh ; combien plus les coeurs des fils des hommes !
\VS{12}Le moqueur n'aime pas qu'on le reprenne, et il ne va pas vers les sages.
\VS{13}Le cœur joyeux rend le visage beau, mais l'esprit est abattu par l'ennui du cœur.
\VS{14}Le cœur de l'homme prudent cherche la science ; mais la bouche des insensés se repaît de folie.
\VS{15}Tous les jours de l'affligé sont mauvais, mais quand on a le coeur gai, c'est un festin perpétuel.
\VS{16}Un peu de bien vaut mieux avec la crainte de Yahweh, qu'un grand trésor avec lequel il y a du trouble\FTNT{Ps. 37:16.}.
\VS{17}Mieux vaut un repas d'herbes où il y a de l'amitié, qu'un repas de boeuf bien gras où il y a de la haine.
\VS{18}L'homme furieux excite la querelle, mais l'homme lent à la colère apaise la dispute.
\VS{19}La voie du paresseux est comme une haie d'épines, mais le chemin des hommes droits est aplani.
\VS{20}Un fils sage réjouit le père, et un homme insensé méprise sa mère.
\VS{21}La stupidité est la joie de celui qui est dépourvu de sens, mais un homme prudent dresse ses pas au chemin de la droiture.
\VS{22} Les résolutions deviennent inutiles où il n'y a point de conseil ; mais il y a de la fermeté dans la multitude des conseillers.
\VS{23}L'homme a de la joie dans les réponses de sa bouche ; et combien est bonne une parole dite en son temps !
\VS{24}Le chemin de la vie élève l'homme prudent, afin qu'il se détourne du scheol qui est en bas.
\VS{25}Yahweh renverse la maison des orgueilleux, mais il affermit la borne de la veuve.
\VS{26}Les pensées du malin sont en abomination à Yahweh, mais celles de ceux qui sont purs sont des paroles agréables à ses yeux.
\VS{27}Celui qui est entièrement adonné au gain déshonnête trouble sa maison, mais celui qui hait les présents vivra.
\VS{28}Le coeur du juste médite ce qu'il doit répondre, mais la bouche des méchants profère des choses mauvaises.
\VS{29}Yahweh est loin des méchants, mais il exauce la requête des justes.
\VS{30}La clarté des yeux réjouit le coeur ; et la bonne renommée fortifie les os.
\VS{31}L'oreille qui écoute la correction qui donne la vie habite parmi les sages.
\VS{32}Celui qui rejette l'instruction a en dédain son âme, mais celui qui écoute la réprimande s'acquiert du sens.
\VS{33}La crainte de Yahweh enseigne la sagesse, et l'humilité précède la gloire\FTNT{Ps. 19:10.}.
\Chap{16}
\TextTitle{[La justice s'oppose à la méchanceté (suite)]}
\VerseOne{}Les préparations du cœur sont à l'homme, mais le discours réponse de la langue est de par Yahweh.
\VS{2}Chacune des voies de l'homme lui semble pure à ses yeux; mais Yahweh pèse les esprits.
\VS{3}Recommande tes affaires à Yahweh, et tes pensées seront bien ordonnées.
\VS{4}Yahweh a fait toutes choses pour lui-même ; et même le méchant pour le jour de l'affliction.
\VS{5}Yahweh a en abomination tout homme hautain de coeur ; assurément, il ne demeurera pas impuni.
\VS{6}Il y aura propitiation de l'iniquité par la miséricorde et la vérité ; on se détourne du mal par la crainte de Yahweh. 
\VS{7}Quand Yahweh prend plaisir aux voies d'un homme, il apaise\FTNT{Apaiser vient de shalom qui signifie : être dans une alliance de paix, être en paix, apaiser, vivre dans la paix etc.} envers lui même ses ennemis.
\VS{8}Il vaut mieux un peu de bien avec justice, qu'un gros revenu là où on n'a pas de droit.
\VS{9}Le cœur de l'homme médite sur sa voie, mais Yahweh conduit ses pas.
\VS{10}La divination est sur les lèvres du roi : Sa bouche ne doit pas s'égarer du droit.
\VS{11}La balance et le poids justes sont à Yahweh, tous les poids du sachet sont aussi son oeuvre.
\VS{12}Commettre une injustice doit être en abomination aux rois, parce que le trône est affermi par la justice.
\VS{13}Les rois doivent prendre plaisir aux lèvres de justice, et aimer celui qui profère des paroles justes.
\VS{14}Ce sont autant de messagers de mort que la colère du roi, mais l'homme sage l'apaisera.
\VS{15}Le visage serein du roi c'est la vie, et sa faveur est comme la nuée portant la pluie de la dernière saison.
\VS{16}Combien est-il plus précieux que l'or fin, d'acquérir de la sagesse! Et combien est-il plus excellent que l'argent, d'acquérir de la prudence ! 
\VS{17}Le chemin aplani des hommes droits, c'est de se détourner du mal ; celui qui prend garde de sa voie garde son âme.
\VS{18}L'orgueil va devant l'écrasement, et la fierté d'esprit devant la ruine.
\VS{19}Mieux vaut être humilié d'esprit avec les débonnaires, que de partager le butin avec les orgueilleux.
\VS{20}Celui qui prend garde à la parole trouvera le bien, et celui qui se confie en Yahweh est heureux\FTNT{Ps. 2:12.}.
\VS{21}On appellera prudent le sage de cœur, et la douceur des lèvres augmente l'instruction.
\VS{22}La prudence est à ceux qui la possèdent une source de vie ; mais le l'instruction des fous c'est leur folie.
\VS{23}Celui qui est sage de coeur conduit prudemment sa bouche, et ajoute l'instruction sur ses lèvres.
\VS{24}Les paroles agréables sont des rayons de miel, douces à l'âme et santé pour les os.
\VS{25} II y a telle voie qui semble droite à l'homme, mais dont la fin sont les voies de la mort.
\VS{26}Celui qui travaille, travaille pour lui-même, parce que sa bouche se courbe devant lui\FTNT{Ec. 6:7.}.
\VS{27}L'homme méchant creuse le mal, et il y a comme un feu brûlant sur ses lèvres.
\VS{28}L'homme qui use de perversité sème des querelles, et le rapporteur divise les grands amis.
\VS{29}L'homme violent attire son compagnon et le fait marcher dans une voie qui n'est pas bonne.
\VS{30}Il fait signe des yeux pour méditer des choses perverses, et remuant ses lèvres il exécute le mal.
\VS{31}Les cheveux blancs sont une couronne d'honneur ; elle se trouvera dans la voie de la justice.
\VS{32}Celui qui est lent à la colère vaut mieux que l'homme fort, et celui qui est maître de son cœur, vaut mieux que celui qui prend des villes.
\VS{33}On jette le sort dans le pan de la robe, mais tout ce qui doit arriver est de part Yahweh.
\Chap{17}
\TextTitle{[La justice s'oppose à la méchanceté (suite)]}
\VerseOne{}Mieux vaut un morceau de pain sec là où il y a la paix, qu'une maison pleine de viandes, là où il y a des querelles.
\VS{2}Le serviteur prudent sera maître sur l'enfant qui fait honte, et il partagera l'héritage entre les frères.
\VS{3}Le creuset est pour éprouver l'argent, et le fourneau l'or ; mais Yahweh éprouve les coeurs\FTNT{Jé. 17:10 ; Mal. 3:3 ; Ps. 26:2.}.
\VS{4}L'homme mauvais est attentif à la lèvre trompeuse, et le menteur écoute la mauvaise langue.
\VS{5}Celui qui se moque du pauvre déshonore celui qui l'a fait ; et celui qui se réjouit de l'affliction ne demeurera pas impuni.
\VS{6}Les petits-fils sont la couronne des vieillards\FTNT{Ps. 127:3 ; Ps. 128:3.}, et les pères sont la gloire de leurs fils.
\VS{7}La parole distinguée ne convient pas à un fou ; combien moins aux principaux du peuple des paroles de mensonge!
\VS{8}Le présent est comme une pierre précieuse aux yeux de ceux qui y sont adonnés ; de quelque côté qu'ils se tournent, ils réussissent.
\VS{9}Celui qui couvre les fautes cherche l'amitié, mais celui qui rapporte la chose divise les plus grands amis.
\VS{1}La répréhension se fait mieux sentir sur l'homme prudent que cent coups au fou.
\VS{11}Le méchant ne cherche que rébellion, mais le messager cruel sera envoyé contre lui.
\VS{12}Que l'homme rencontre plutôt une ourse qui a perdu ses petits qu'un fou dans sa folie.
\VS{13}Le mal ne partira point de la maison de celui qui rend le mal pour le bien.
\VS{14}Le commencement d'une querelle est comme quand on lâche une l'eau; mais avant qu'on en vienne à la dispute, retire-toi.
\VS{15}Celui qui déclare juste le méchant et celui qui déclare méchant le juste, sont tous deux en abomination à Yahweh\FTNT{Ex. 23:7 ; Es. 5:23.}.
\VS{16}A quoi sert le prix dans la main du fou pour acheter la sagesse, vu qu'il n'a pas de sens?
\VS{17}L'ami intime aime en tout temps, et il naît comme un frère dans la détresse.
\VS{18}Celui là est dépourvu de sens qui touche à la main et se rend caution pour son ami.
\VS{19}Celui qui aime les querelles aime le péché ; celui qui élève sa porte cherche sa ruine.
\VS{20}Celui qui est pervers de coeur ne trouve pas le bien; et l'hypocrite tombe dans le malheur.
\VS{21}Celui qui engendre un sot en aura de l'ennui, et le père du sot ne se réjouira pas.
\VS{22}Le coeur joyeux est un remède, mais l'esprit abattu dessèche les os.
\VS{23}Le méchant rend les présents en secret, pour pervertir les voies du jugement.
\VS{24}La sagesse est en présence de l'homme prudent; mais les yeux du fou sont à l'extrémité de la terre.
\VS{25}Le fils fou est l'ennui de son père, et l'amertume de celle qui l'a enfanté.
\VS{26}Il n'est pas bon de condamner l'innocent à l'amende, ni que les principaux frappent quelqu'un pur avoir agi avec droiture.
\VS{27}L'homme retenu dans ses paroles sait ce qu'est la connaissance, et l'homme qui est d'un esprit calme est un homme intelligent.
\VS{28}Même le fou, quand il se tait, est réputé sage ; et celui qui ferme ses lèvres est réputé intelligent.
\Chap{18}
\TextTitle{[La justice s'oppose à la méchanceté (suite)]}
\VerseOne{}Celui qui se sépare cherche ce qui lui fait plaisir, et se mêle de savoir comment tout doit aller.
\VS{2}Le fou ne prend pas plaisir à l'intelligence, mais à ce que son cœur soit manifesté.
\VS{3}Quand le méchant vient, le mépris vient aussi, et le reproche avec l'ignominie.
\VS{4}Les paroles de la bouche d'un homme sont des eaux profondes ; et la source de la sagesse est un torrent qui bouillonne\FTNT{Jn. 4:14.}.
\VS{5}Il n'est pas bon d'avoir égard à l'apparence de la personne du méchant, pour renverser le juste en jugement.
\VS{6}La bouche du fou entrent en querelles, et sa bouche appelle les combats.
\VS{7}La bouche du fou lui est une ruine, et ses lèvres sont un piège à son âme.
\VS{8}Les paroles du flatteur sont de ceux qui font semblant d'y toucher ; mais elles pénètrent jusqu'au-dedans des entrailles.
\VS{9}Celui qui se relâche dans son ouvrage est frère de celui qui dissipe ce qu'il a.
\VS{10}Le Nom de Yahweh est une tour forte, le juste y court et y trouve une haute retraite.
\VS{11}Les biens du riche sont sa ville forte et comme une haute muraille de retraite, selon son imagination.
\VS{12}Le coeur de l'homme s'élève avant que la ruine arrive, mais l'humilité précède la gloire.
\VS{13}Celui qui répond à quelque propos avant de l'avoir entendu, agit en fou et s'attire le reproche.
\VS{14}L'esprit d'un homme fort soutiendra dans son infirmité ; mais l'esprit abattu, qui le relèvera ?
\VS{15}Le coeur de l'homme intelligent acquiert la connaissance, et l'oreille des sages cherche la connaissance.
\VS{16}Le présent d'un homme lui fait faire place, et le conduit devant les grands.
\VS{17}Celui qui plaide le premier paraît juste; mais sa partie adverse vient, et examine le tout.
\VS{18}Le sort fait cesser les procès et fait les partages entre les puissants.
\VS{19}Un frère offensé se rend plus difficile qu'une ville forte, et les discordes entre frères sont comme les verrous d'un palais.
\VS{20}Le ventre de chacun est rassasié du fruit de sa bouche, il se rassasie du revenu de ses lèvres.
\VS{21}La mort et la vie sont au pouvoir de la langue\FTNT{Mt. 12:37.}, et celui qui aime à parler mangera de ses fruits.
\VS{22}Celui qui trouve une femme vertueuse trouve le bonheur et il obtient une faveur de Yahweh.
\VS{23}Le pauvre ne prononce que des supplications, mais le riche ne répond que des paroles dures.
\VS{24}L'homme qui a des intimes amis se tiennent à leur amitié parce qu'il y a tel ami qui est plus attaché que le frère.
\Chap{19}
\TextTitle{[La justice s'oppose à la méchanceté (suite)]}
\VerseOne{}Le pauvre qui marche dans son intégrité, vaut mieux que celui qui pervertit ses lèvres et qui est fou.
\VS{2}La vie même sans connaissance n'est pas une bonne personne ; et celui qui hâte ses pas dans le péché, s'égare.
\VS{3}La folie de l'homme renverse son chemin ; et cependant, c'est contre Yahweh que son coeur s'irrite.
\VS{4}Les richesses attirent un grand nombre d'amis, mais celui qui est pauvre est abandonné même par son ami.
\VS{5}Le faux témoin ne restera pas impuni, et celui qui profère des mensonges n'échappera pas.
\VS{6}Plusieurs supplient celui qui est en état de faire du bien, et chacun est ami de celui qui donne.
\VS{7}Tous les frères du pauvre le haïssent ; combien plus ses amis se retirent-ils de lui ! Il les supplie, mais il n'y a que des paroles pour lui.
\VS{8}Celui qui acquiert du sens aime son âme, et celui qui prend garde à l'intelligence c'est pour trouve le bonheur.
\VS{9}Le faux témoin ne restera pas impuni, et celui qui profère des mensonges périra.
\VS{10}Il ne sied pas à un fou de vivre dans les délices ; combien moins sied-il à un esclave de dominer sur les personnes de distinction !
\VS{11}La prudence de l'homme retient à la colère ; c'est un honneur pour lui de passer par dessus le tort qu'on lui fait.
\VS{12}La colère du roi est comme le rugissement d'un jeune lion, mais sa faveur est comme la rosée sur l'herbe.
\VS{13}Un fils insensé est un grand malheur pour son père, et les querelles d'une femme sont une gouttière continuelle.
\VS{14}On peut hériter de ses pères une maison et des richesses, mais la femme prudente est un don de Yahweh.
\VS{15}La paresse fait venir le sommeil, et l'âme paresseuse a faim.
\VS{16}Celui qui garde le commandement garde son âme, mais celui qui méprise ses voies mourra.
\VS{17}Celui qui a pitié du pauvre prête à Yahweh, qui lui rendra son bienfait.
\VS{18}Châtie ton fils tandis qu'il y a de l'espérance, mais ne va pas jusqu'à le faire mourir.
\VS{19}Celui qui est de grande colère en porte la peine ; et si tu l'en retires, tu y ajoute davantage.
\VS{20}Ecoute le conseil et reçois l'instruction, afin que tu deviennes sage en ton dernier temps.
\VS{21}Il y a dans le cœur de l'homme plusieurs pensées, mais le conseil de Yahweh est\FTNT{Es. 46:10 ; Ps. 33:11.}.
\VS{22}Ce que l'homme doit désirer, c'est d'exercer la miséricorde ; et le pauvre vaut mieux qu'un menteur.
\VS{23}La crainte de Yahweh conduit à la vie, et celui qui l'a, passe la nuit étant rassasié, sans qu'il soit visité par aucun mal.
\VS{24}Le paresseux cache sa main dans le sein, et il ne daigne même pas la ramener à sa bouche.
\VS{25}Si tu bats le moqueur, le sot en rend garde ; et si tu reprends l'homme intelligent, il discernera ce qu'il faut savoir.
\VS{26}L'enfant qui fait honte et sème la confusion, détruit le père et met en fuite sa mère.
\VS{27}Mon fils, cesse d'écouter ce qui pourrait t'apprendre à te détourner des paroles de la connaissance.
\VS{28}Le témoin indigne\FTNT{Le mot « pervers » vient de l'hébreu « beliya'al » : « sans valeur », « vaurien » (Jg. 19:22 ; 1. S. 2:12). Bélial est aussi un autre nom de Satan (2 Co. 6:15).} se moque de la justice, et la bouche des méchants avale l'iniquité.
\VS{29}Les jugements sont préparés pour les moqueurs, et les grands coups pour le dos des fous.
\Chap{20}
\TextTitle{[La justice s'oppose à la méchanceté (suite)]}
\VerseOne{}Le vin est moqueur et les boissons fortes sont tumultueuses, quiconque en fait excès, n'est pas sage.
\VS{2}La terreur du roi est comme le rugissement d'un jeune lion, celui qui l'irrite pèche contre sa propre âme.
\VS{3}C'est une gloire à l'homme de s'abstenir des disputes, mais tout insensé s'y engage.
\VS{4}Le paresseux ne labourera pas à cause de l'hiver, lors de la moisson il mendiera et n'aura rien.
\VS{5}Les conseils dans le coeur d'un homme sage sont comme des eaux profondes, et l'homme intelligent sait y puiser.
\VS{6}Beaucoup de gens vantent leur bonté ; mais l'homme fidèle, qui le trouvera ?
\VS{7}Ô, que les fils du juste qui marchent dans son intégrité seront heureux après lui !
\VS{8}Le roi assis sur le trône de justice dissipe tout mal par son regard.
\VS{9}Qui est-ce qui peut dire : J'ai purifié mon cœur, je suis net de mon péché ?
\VS{10}Le double poids et la double mesure sont tous deux en abomination à Yahweh.
\VS{11}Un jeune enfant fait connaître par ses actions si son oeuvre sera pure et droite.
\VS{12}L'oreille qui entend et l'oeil qui voit, Yahweh les a faits tous les deux.
\VS{13}N'aime point le sommeil, de peur que tu ne deviennes pauvre ; ouvre tes yeux, et tu auras suffisamment de pain.
\VS{14}Il est mauvais, il est mauvais, dit l'acheteur ; puis il s'en va, et se vante.
\VS{15}Il y a de l'or et beaucoup de perles ; mais les lèvres qui gardent la connaissance sont un vase précieux.
\VS{16}Quand quelqu'un se porte garant pour l'étranger, prends son vêtement ; exige de lui des gages pour cet étranger.
\VS{17}Le pain acquis par la tromperie est doux à l'homme, mais ensuite sa bouche sera remplie de gravier.
\VS{18}Les projets s'affermissent par le conseil ; fais donc la guerre avec prudence.
\VS{19}Celui qui médit révèle les secrets ; ne te mêle donc pas avec celui qui séduit par ses lèvres.
\VS{20}Celui qui traite avec mépris son père ou sa mère, sa lampe s'éteindra au milieu des ténèbres les plus noires\FTNT{Ex. 21:17 ; Lé. 20:9 ; Mt. 15:4.}.
\VS{21}L'héritage pour lequel on s'est trop hâté dès l'origine, ne sera pas béni à la fin.
\VS{22}Ne dis point : Je rendrai le mal ; espère en Yahweh, et il te délivrera.
\VS{23}Le double poids est en horreur à Yahweh, et la balance fausse n'est pas une chose bonne.
\VS{24}Les pas de l'homme sont dirigés par Yahweh, comment donc l'homme comprendrait-il sa voie ?
\VS{25}C'est un piège à l'homme que prendre à la légère un engagement sacré, et de ne réfléchir qu'après avoir fait un vœu.
\VS{26}Un roi sage disperse les méchants et ramène la roue sur eux.
\VS{27}L'esprit de l'homme est une lampe de Yahweh, il pénètre jusqu'au fond des entrailles.
\VS{28}La bienveillance et la vérité protègent le roi, et il soutient son trône par la bienveillance.
\VS{29}La force est la gloire des jeunes gens, et les cheveux blancs sont l'honneur des vieillards.
\VS{30}Les meurtrissures et les plaies nettoient le mal, de même les coups qui pénètrent jusqu'au fond des entrailles.
\Chap{21}
\TextTitle{[La justice s'oppose à la méchanceté (suite)]}
\VerseOne{}Le coeur du roi est un courant d'eau dans la main de Yahweh ; il l'incline partout où il veut.
\VS{2}Toutes les voies de l'homme sont droites à ses yeux, mais c'est Yahweh qui pèse les coeurs.
\VS{3}Faire ce qui est juste et droit est une chose que Yahweh préfère aux sacrifices.
\VS{4}Des regards hautains et le coeur qui s'enfle sont la lampe des méchants, ce n'est que péché.
\VS{5}Les projets de l'homme diligent ne mènent qu'à l'abondance, mais celui qui agit avec précipitation ne court qu'à l'indigence.
\VS{6}Des trésors acquis par une langue mensongère, c'est une vanité qu'on ne peut retenir, un signe avant-coureur de la mort.
\VS{7}La violence des méchants les emporte, parce qu'ils refusent de faire ce qui est droit.
\VS{8}La voie d'un homme coupable est détournée, mais l'oeuvre de celui qui est innocent est droite.
\VS{9}Il vaut mieux habiter à l'angle d'un toit qu'avec une femme querelleuse dans une grande maison.
\VS{10}L'âme du méchant désire le mal, son prochain ne trouve pas de grâce à ses yeux.
\VS{11}Quand on punit le moqueur, le sot devient sage ; et quand on instruit le sage, il reçoit la connaissance.
\VS{12}Il y a un juste qui considère attentivement la maison du méchant, Yahweh renverse les méchants dans le malheur.
\VS{13}Celui qui bouche son oreille pour ne pas entendre le cri du pauvre, criera aussi lui-même, et on ne lui répondra point.
\VS{14}Un don fait en secret apaise la colère, et un présent fait en cachette calme une fureur violente.
\VS{15}C'est une joie pour le juste de pratiquer la justice, mais c'est la ruine pour les ouvriers d'iniquité.
\VS{16}L'homme qui s'écarte du chemin de la sagesse aura sa demeure dans l'assemblée des morts.
\VS{17}Celui qui aime les réjouissances reste dans l'indigence ; et celui qui aime le vin et l'huile ne s'enrichira pas.
\VS{18}Le méchant sert de rançon pour le juste, et le déloyal pour les hommes intègres.
\VS{19}Il vaut mieux habiter dans une terre déserte qu'avec une femme querelleuse et qui se dépite.
\VS{20}Des précieux trésors et l'huile sont dans la demeure du sage, mais l'homme insensé les engloutit.
\VS{21}Celui qui poursuit la justice et la bonté, trouve la vie, la justice et la gloire.
\VS{22}Le sage entre dans la ville des forts et il abat la force qui lui donnait de l'assurance.
\VS{23}Celui qui veille sur sa bouche et sur sa langue préserve son âme des angoisses.
\VS{24}On appelle moqueur un superbe arrogant, qui agit avec colère et orgueil.
\VS{25}Les désirs du paresseux le tuent, parce que ses mains refusent de travailler.
\VS{26}Tout le jour il désire avidement, mais le juste donne sans parcimonie.
\VS{27}Le sacrifice des méchants est une abomination ; combien plus quand ils l'apportent avec des mauvaises intentions\FTNT{1 S. 15:22.} ?
\VS{28}Le témoin menteur périra, mais l'homme qui écoute parlera avec gain de cause.
\VS{29}L'homme méchant prend un air effronté, mais l'homme droit règle sa conduite.
\VS{30}Il n'y a ni sagesse, ni intelligence, ni conseil, contre Yahweh.
\VS{31}Le cheval est équipé pour le jour de la bataille, mais la délivrance vient de Yahweh.
\Chap{22}
\TextTitle{[La justice s'oppose à la méchanceté (suite)]}
\VerseOne{}La renommée est préférable aux grandes richesses\FTNT{Ec.7:1.}, et la bonne grâce plus que l'argent et l'or.
\VS{2}Le riche et le pauvre se rencontrent ; celui qui les a faits l'un et l'autre, c'est Yahweh\FTNT{Lu. 16.}.
\VS{3}L'homme prudent voit le mal et se cache, mais les stupides passent et en portent la peine.
\VS{4}Les fruits de l'humilité et de la crainte de Yahweh sont les richesses, la gloire et la vie.
\VS{5}Il y a des épines et des pièges dans la voie de l'homme pervers ; celui qui aime son âme s'en retirera loin.
\VS{6}Instruis le jeune enfant selon la voie qu'il doit suivre, et quand il sera vieux, il ne s'en détournera pas.
\VS{7}Le riche domine sur les pauvres\FTNT{Ja. 2:6.}, et celui qui emprunte est l'esclave de celui qui prête.
\VS{8}Celui qui sème l'injustice moissonne le malheur\FTNT{Job. 4:8 ; Ga. 6:7.}, et la verge de sa fureur prendra fin.
\VS{9}Celui qui a l'oeil bienveillant sera béni, parce qu'il aura donné de son pain au pauvre.
\VS{10}Chasse le moqueur, et la querelle prendra fin ; les disputes et l'ignominie cesseront.
\VS{11}Le roi est ami de celui qui aime la pureté de coeur, et qui a de la grâce dans ses paroles.
\VS{12}Les yeux de Yahweh veillent sur la connaissance, mais il confond les paroles du perfide.
\VS{13}Le paresseux dit : Il y a un lion dehors ! Je serais tué dans les rues !
\VS{14}La bouche des courtisanes est une fosse profonde, celui contre qui Yahweh est irrité y tombera.
\VS{15}La folie est liée au coeur du jeune enfant, mais la verge de la correction l'éloignera de lui.
\VS{16}Celui qui fait tort au pauvre pour s'enrichir et pour donner au riche, ne peut manquer de tomber dans l'indigence.
\VS{17}Prête ton oreille et écoute les paroles des sages, et applique ton coeur à ma connaissance.
\VS{18}Car ce sera une chose agréable pour toi si tu les gardes au-dedans de toi, et qu'elles soient toutes présentes sur tes lèvres.
\VS{19}Je te les ai fait connaître à toi aujourd'hui, dis-je, afin que ta confiance soit en Yahweh.
\VS{20}N'ai-je pas déjà pour toi mis par écrit des choses qui conviennent à ceux qui gouvernent, des conseils et des réflexions,
\VS{21}pour te faire connaître la certitude des paroles vraies, afin que tu répondes par des paroles vraies à celui qui t'envoie ?
\VS{22}Ne dépouille pas le pauvre, parce qu'il est pauvre ; et n'opprime pas le malheureux à la porte.
\VS{23}Car Yahweh défendra leur cause et privera de la vie ceux qui les auront volés.
\VS{24}Ne fréquente pas quelqu'un de coléreux, ne va pas avec l'homme violent ;
\VS{25}de peur que tu n'apprennes ses manières, et qu'ils ne deviennent un piège pour ton âme.
\VS{26}Ne sois pas parmi ceux qui prennent des engagements ni de ceux qui cautionnent les dettes.
\VS{27}Si tu n'as pas de quoi payer, pourquoi prendrait-on ton lit de dessous toi ?
\VS{28}Ne déplace pas la borne ancienne, que tes pères ont posée.
\VS{29}As-tu vu un homme habile en son travail ? Il sera au service des rois, il ne se tiendra pas devant des gens obscurs.
\Chap{23}
\TextTitle{[La justice s'oppose à la méchanceté (suite)]}
\VerseOne{}Quand tu t'assieds pour manger avec un gouverneur, considère avec attention celui qui est devant toi.
\VS{2}Autrement tu te mettras le couteau à la gorge, si ton appétit te domine.
\VS{3}Ne convoite pas ses friandises, car c'est un pain trompeur.
\VS{4}Ne travaille pas en vue d'acquérir des richesses ; désiste-toi de la résolution que tu as prise.
\VS{5}Jetteras-tu tes yeux sur ce qui bientôt n'est plus ? Car certainement, il se fera des ailes, il s'envolera comme un aigle dans les cieux.
\VS{6}Ne mange pas le pain de celui dont le regard est envieux, et ne désire pas ses friandises.
\VS{7}Car il est tel qu'il pense dans son âme. Il te dira bien : Mange et bois, mais son coeur n'est pas avec toi.
\VS{8}Tu voudrais vomir le morceau que tu auras mangé, et tu auras perdu tes paroles agréables.
\VS{9}Ne parle pas aux oreilles de l'insensé, car il méprise le bon sens de ton discours.
\VS{10}Ne déplace pas la borne ancienne et n'entre pas dans les champs des orphelins :
\VS{11}Car leur vengeur est puissant, il défendra leur cause contre toi.
\VS{12}Applique ton coeur à l'instruction, et tes oreilles aux paroles de la connaissance.
\VS{13}Ne te retiens pas de corriger le jeune enfant ; quand tu l'auras frappé de la verge, il n'en mourra pas.
\VS{14}En le frappant de la verge, tu préserves son âme du scheol.
\VS{15}Mon fils, si ton coeur est sage, mon coeur s'en réjouira, oui, moi-même.
\VS{16}Certes, mes reins tressailliront de joie quand tes lèvres proféreront ce qui est droit.
\VS{17}Que ton coeur ne porte pas d'envie aux pécheurs, mais adonne-toi tout le jour à la crainte de Yahweh.
\VS{18}Car il y a véritablement un avenir, et ton espérance ne sera pas retranchée.
\VS{19}Toi, mon fils, écoute et sois sage, et dirige ton coeur dans la bonne voie.
\VS{20}Ne fréquente point les ivrognes ni les gourmands\FTNT{Ro. 13:13 ; Ep. 5:18 ; Ga. 5:18-21.}.
\VS{21}Car l'ivrogne et le gourmand s'appauvrissent ; et l'assoupissement fait porter des vêtements déchirés.
\VS{22}Ecoute ton père, c'est celui qui t'a engendré ; et ne méprise pas ta mère, quand elle est devenue vieille.
\VS{23}Acquiers la vérité, et ne la vends point, acquiers la sagesse, l'instruction et l'intelligence.
\VS{24}Le père du juste aura beaucoup de joie, et celui qui donne naissance à un sage se réjouira en lui.
\VS{25}Que ton père et ta mère se réjouissent, que celle qui t'a enfanté soit dans l'allégresse !
\VS{26}Mon fils, donne-moi ton coeur, et que tes yeux prennent plaisir à mes voies.
\VS{27}Car la femme prostituée est une fosse profonde, et la courtisane un puits de détresse.
\VS{28}Aussi se tient-elle en embûche comme un voleur, et elle augmente parmi les hommes le nombre des infidèles.
\VS{29}Pour qui les « ah » ? Pour qui les « malheur à moi ! » Pour qui les disputes ? Pour qui les plaintes ? Pour qui les blessures sans raison ? Pour qui les yeux rouges ?
\VS{30}Pour ceux qui s'attardent auprès du vin, pour ceux qui vont chercher des vins mélangés.
\VS{31}Ne regarde pas le vin parce qu'il est d'un beau rouge, qu'il donne son éclat dans la coupe, et qu'il coule aisément.
\VS{32}Il finit par mordre par derrière comme un serpent, et par piquer comme un basilic.
\VS{33}Ensuite tes yeux regarderont les femmes étrangères, et ton coeur parlera d'une manière perverse.
\VS{34}Tu seras comme un homme qui dort au milieu de la mer, et comme un homme couché sur le sommet d'un mât.
\VS{35}On m'a battu, diras-tu, et je n'en ai pas été malade, on m'a frappé, et je ne l'ai pas senti, quand me réveillerai-je ? Je me remettrai encore à chercher le vin.
\Chap{24}
\TextTitle{[La justice s'oppose à la méchanceté (suite)]}
\VerseOne{}N'envie pas les hommes qui font le mal, et ne désire pas être avec eux.
\VS{2}Car leur coeur médite la destruction, et leurs lèvres parlent d'iniquité.
\VS{3}C'est par la sagesse qu'une maison est bâtie, et par l'intelligence qu'elle s'affermit.
\VS{4}C'est par la connaissance que les chambres seront remplies de tous les biens précieux et agréables.
\VS{5}Un homme sage est accompagné de force, et celui qui a de la connaissance affermit sa vigueur.
\VS{6}Car avec de bonnes directives tu feras la guerre avantageusement, et le salut est dans le grand nombre des bons conseillers.
\VS{7}La sagesse est trop élevée pour l'insensé, il n'ouvrira pas sa bouche à la porte.
\VS{8}Celui qui médite de faire le mal s'appelle un homme plein de malice.
\VS{9}Le projet de la folie est un péché, et le moqueur est en abomination parmi les hommes.
\VS{10}Si tu perds courage au jour de la détresse, ta force n'est que détresse.
\VS{11}Ne te retiens pas de délivrer ceux qu'on traîne à la mort, ceux qu'on va égorger, agis pour qu'on les épargne !
\VS{12}Si tu dis : Ah ! nous n'en savions rien… Celui qui pèse les cœurs, lui, ne le considérera-t-il pas ? Celui qui garde ton âme, lui, le sait, et il rend à chacun selon son œuvre.
\VS{13}Mon fils, mange le miel, car il est bon ; un rayon de miel sera doux à ton palais.
\VS{14}Ainsi sera à ton âme la connaissance de la sagesse, quand tu l'auras trouvée ; il y a un avenir, et ton espérance ne sera pas anéantie.
\VS{15}Méchant, ne mets pas des embûches dans le domaine du juste, et ne dévaste pas le lieu où il se repose.
\VS{16}Car le juste tombera sept fois, et sera relevé\FTNT{Ps. 34:20. ; Job. 5:19.} ; mais les méchants trébuchent pour tomber dans le malheur.
\VS{17}Si ton ennemi tombe, ne t'en réjouis pas, et que ton coeur ne soit pas dans l'allégresse quand il chancelle,
\VS{18}de peur que Yahweh ne le voie et que cela ne lui déplaise, tellement qu'il détourne sa colère de dessus de lui sur toi.
\VS{19}Ne t'irrite pas à cause de ceux qui font le mal, n'envie pas les méchants,
\VS{20}car il n'y a pas d'avenir pour celui qui fait le mal, la lampe des méchants sera éteinte.
\VS{21}Mon fils, crains Yahweh et le roi ; et ne te mêle pas avec les gens agités.
\VS{22}Car leur ruine s'élèvera tout d'un coup ; et qui sait le malheur qui arrivera à l'un et à l'autre ?
\VS{23}Voici encore ce qui vient des sages : Il n'est pas bon d'avoir égard à l'apparence des personnes dans le jugement.
\VS{24}Celui qui dit au méchant : Tu es juste ! Les peuples le maudiront, et les nations seront indignées contre lui.
\VS{25}Mais pour ceux qui le reprennent, ils en retireront de la satisfaction. Et la bénédiction vient sur eux pour leur bonheur.
\VS{26}Celui qui répond avec justesse fait plaisir à celui qui l'écoute.
\VS{27}Prépare ton ouvrage au-dehors, et apprête ton champ, et après, tu bâtiras ta maison.
\VS{28}Ne témoigne pas sans cause contre ton prochain ; car voudrais-tu tromper de tes lèvres\FTNT{Ep. 4:25.} ?
\VS{29}Ne dis pas : Comme il m'a fait, ainsi je lui ferai ; je rendrai à cet homme selon ce qu'il m'a fait.
\VS{30}J'ai passé près du champ de l'homme de paresseux, et près de la vigne d'un homme dépourvu de sens ;
\VS{31}et voici, tout y était monté en chardons, et les ronces avaient couvert la surface, et le mur de pierres était écroulé.
\VS{32}Et j'ai regardé, j'y ai appliqué mon coeur, j'ai vu et j'en ai tiré instruction.
\VS{33}Un peu de sommeil, un peu d'assoupissement, un peu croiser les mains pour dormir ! …
\VS{34}Et la pauvreté te surprendra comme un rôdeur ; et la disette, comme un homme armé.
\Chap{25}
\TextTitle{[Avertissements et conseils]}
\VerseOne{}Ce sont ici aussi des proverbes de Salomon\FTNT{1 R. 4: 32.}, que les gens d'Ezéchias, roi de Juda, ont transcrits.
\VS{2}La gloire de Dieu est de cacher les choses, et la gloire des rois est de sonder les choses.
\VS{3}Les cieux dans leur hauteur, la terre dans sa profondeur, le coeur des rois sont impénétrables.
\VS{4}Ôte de l'argent les scories, et il en sortira un vase pour l'orfèvre.
\VS{5} De même, ôte le méchant de devant le roi, et son trône sera affermi par la justice.
\VS{6}Ne t'élève pas devant le roi et ne te tiens pas à la place des grands.
\VS{7}Car il vaut mieux qu'on te dise: Monte-ici ! Que si l'on t'abaisse devant un seigneur que tes yeux ont vu\FTNT{Lu. 14:8-11.}.
\VS{8}Ne te hâte pas d'entrer en contestation, de peur que tu ne saches que faire à la fin, lorsque ton prochain t'aura confondu.
\VS{9}Plaide ta cause contre ton prochain, mais ne révèle pas le secret d'autrui.
\VS{10}De peur que celui qui l'écoute ne te couvre de honte, et que ton opprobre ne s'efface pas.
\VS{11}Telles que sont des pommes d'or sur des ciselures d'argent, telle est une parole dite quand il faut.
\VS{12}Comme un anneau d'or ou comme un joyau d'or fin, ainsi est l'oreille obéissante pour le sage qui réprimande.
\VS{13}Le messager fidèle est à ceux qui l'envoient, comme la fraîcheur de la neige au temps de la moisson, il restaure l'âme de son maître.
\VS{14}Celui qui se glorifie faussement de ses libéralités, est comme les nuages et le vent sans pluie.
\VS{15}Un prince est fléchi par la patience, et la langue douce brise les os.
\VS{16}As-tu trouvé du miel, manges-en autant qu'il t'en faut, de peur que tu n'en sois rassasié, que tu ne le vomisses.
\VS{17}De même, mets rarement le pied dans la maison de ton prochain, de peur qu'étant rassasié de toi, il ne te haïsse.
\VS{18}Comme une massue, une épée et une flèche aiguë, ainsi est un homme qui porte un faux témoignage contre son prochain.
\VS{19}Comme une dent qui se rompt, un pied qui glisse, telle est la confiance qu'on met en un traître au jour de la détresse.
\VS{20}Celui qui chante des chansons à un coeur affligé est comme celui qui ôte sa robe dans un jour froid, et comme du vinaigre répandu sur du nitre.
\VS{21}Si celui qui te hait a faim, donne-lui à manger du pain ; et s'il a soif, donne-lui à boire de l'eau\FTNT{Mt 5:39-44.}.
\VS{22}Car ce sont des charbons ardents que tu lui mets sur sa tête, et Yahweh te le rendra.
\VS{23}Le vent du nord engendre les averses, et la langue qui médit en cachette un visage irrité.
\VS{24}Il vaut mieux habiter à l'angle d'un toit que de partager la demeure d'une femme querelleuse.
\VS{25}Comme de l'eau fraîche pour une personne fatiguée et lasse, ainsi est une bonne nouvelle venant d'une terre lointaine.
\VS{26}Le juste qui bronche devant le méchant est une fontaine troublée et une source gâtée.
\VS{27}Comme il n'est pas bon de manger beaucoup de miel, aussi n'y a-t-il pas de gloire pour les hommes de rechercher leur gloire avec ardeur.
\VS{28}L'homme qui n'est pas maître de son esprit est comme une ville où il y a une brèche, et qui est sans murailles.
\Chap{26}
\TextTitle{[Avertissements et conseils (suite)]}
\VerseOne{}Comme la neige en été et la pluie pendant la moisson, ainsi la gloire ne convient pas à un insensé.
\VS{2}Comme l'oiseau est prompt à s'échapper et l'hirondelle à s'envoler, ainsi la malédiction sans cause n'atteint pas.
\VS{3}Le fouet est pour le cheval, le mors pour l'âne, et la verge pour le dos des insensés.
\VS{4}Ne réponds pas à l'insensé selon sa folie, de peur que tu ne lui ressembles toi-même.
\VS{5}Réponds à l'insensé selon sa folie, de peur qu'il ne devienne sage à ses propres yeux.
\VS{6}Celui qui envoie des messages par l'intermédiaire d'un insensé, se coupe les pieds et boit la peine du tort qu'il s'est fait.
\VS{7}Faites marcher un homme boiteux, ainsi il en sera d'un proverbe dans la bouche des insensés.
\VS{8}Celui qui donne de la gloire à un insensé, c'est comme s'il jetait une pierre précieuse dans un monceau de pierres.
\VS{9}Comme une épine dans la main d'un homme ivre, ainsi est un proverbe dans la bouche des insensés.
\VS{10}Les puissants donnent de l'ennui à tout le monde, et prennent à leur service les insensés et les passants.
\VS{11}Comme le chien retourne à ce qu'il a vomi, ainsi l'insensé réitère sa folie\FTNT{2 Pi. 2:22.}.
\VS{12}As-tu vu un homme qui croit être sage ? Il y a plus à espérer d'un insensé que de lui.
\VS{13}Le paresseux dit : Il y a un lion rugissant sur le chemin, il y a un lion dans les rues.
\VS{14}Comme une porte tourne sur ses gonds, ainsi fait le paresseux sur son lit.
\VS{15}Le paresseux plonge sa main dans le plat, et il trouve fatigant de la ramener à sa bouche.
\VS{16}Le paresseux se croit plus sage que sept hommes qui répondent avec bon sens.
\VS{17}Celui qui en passant se met en colère pour une dispute qui ne le touche en rien, est comme celui qui prend un chien par les oreilles.
\VS{18}Tel est celui qui fait l'insensé, et qui cependant jette des feux, des flèches, et des choses propres à tuer,
\VS{19}tel est l'homme qui a trompé son ami, et qui après cela dit : N'était-ce pas pour plaisanter ?
\VS{20}Le feu s'éteint faute de bois, ainsi quand il n'y a pas de rapporteurs la querelle s'apaise.
\VS{21}Le charbon est pour faire de la braise, et le bois pour faire du feu, et l'homme querelleur pour exciter des querelles.
\VS{22}Les paroles d'un rapporteur sont comme des friandises, elles descendent jusqu'au fond des entrailles.
\VS{23}Les lèvres brûlantes de zèle et le coeur mauvais sont comme des scories d'argent appliquées sur un vase de terre.
\VS{24}Celui qui a de la haine se déguise par ses discours, mais au-dedans de lui il nourrit la trahison.
\VS{25}Lorsqu'il prend une voix douce, ne le crois pas, car il y a sept abominations dans son coeur.
\VS{26}S'il cache sa haine sous la dissimulation, sa méchanceté sera révélée dans l'assemblée.
\VS{27}Celui qui creuse la fosse y tombe ; et la pierre retourne sur celui qui la roule\FTNT{Ps. 7:16-17 ; Ps. 57:7 ; Ec. 10:8.}.
\VS{28}La fausse langue hait ceux qu'elle a écrasés ; et la bouche flatteuse prépare la ruine.
\Chap{27}
\TextTitle{[Avertissements et conseils (suite)]}
\VerseOne{}Ne te vante point du lendemain, car tu ne sais pas quelle chose le jour enfantera\FTNT{Ja. 4:13-15.}.
\VS{2}Qu'un autre te loue, et non pas ta propre bouche ; que ce soit l'étranger, et non pas tes lèvres.
\VS{3}La pierre est pesante, et le sable est lourd ; mais l'irritation de l'insensé est plus pesante que tous les deux.
\VS{4}Il y a de la cruauté dans la fureur, et du débordement dans la colère ; mais qui pourra subsister devant la jalousie ?
\VS{5}Mieux vaut une réprimande ouverte qu'une amitié cachée.
\VS{6}Les blessures d'un ami sont dignes de confiance, mais les baisers d'un ennemi sont à craindre\FTNT{Il est question ici de Judas}.
\VS{7}L'âme rassasiée foule aux pieds les rayons de miel ; mais l'âme qui a faim trouve doux même ce qui est amer.
\VS{8}Tel qu'est un oiseau errant loin de son nid, tel est l'homme qui s'écarte de son lieu.
\VS{9}L'huile et les parfums réjouissent le coeur, et il en est ainsi de la douceur d'un ami dont le fruit est un conseil qui vient du coeur.
\VS{10}Ne quitte point ton ami ni l'ami de ton père, et n'entre pas dans la maison de ton frère au jour de ta détresse ; car le voisin qui est proche vaut mieux que le frère qui est éloigné.
\VS{11}Mon fils sois sage, et réjouis mon coeur, afin que j'aie de quoi répondre à celui qui m'outrage.
\VS{12}L'homme prudent voit le malheur arriver et se cache ; mais les stupides passent outre et en payent l'amende.
\VS{13}Quand quelqu'un se portera garant pour l'étranger, prends son vêtement, exige de lui des gages, à cause des étrangers.
\VS{14}Celui qui bénit son ami à haute voix, se levant de grand matin, on le lui comptera comme une malédiction.
\VS{15}Une gouttière continuelle en un jour de grosse pluie, et une femme querelleuse, cela se ressemble.
\VS{16}Celui qui veut la retenir est comme s'il voulait arrêter le vent, et retenir dans sa main une huile qui s'écoule.
\VS{17}Comme le fer aiguise le fer, ainsi l'homme aiguise la personnalité de son prochain.
\VS{18}Comme celui qui garde le figuier mangera de son fruit, ainsi celui qui garde son maître sera honoré.
\VS{19}Comme dans l'eau le visage répond au visage, ainsi le coeur d'un homme répond à celui d'un autre homme.
\VS{20}Le scheol et le gouffre ne sont jamais rassasiés ; de même, les yeux des hommes sont insatiables\FTNT{Ec. 1:8 ; 2 Pi. 2:14.}.
\VS{21}Le fourneau est pour éprouver l'argent, et le creuset pour l'or ; mais un homme est jugé d'après sa renommée.
\VS{22}Quand tu pilerais un insensé dans un mortier, parmi du grain qu'on pile avec un pilon, sa stupidité ne se détacherait pas de lui.
\VS{23}Sois diligent à reconnaître l'état de chacune de tes brebis, et applique ton coeur aux troupeaux.
\VS{24}Car l'abondance ne dure pas à toujours, et une couronne dure-t-elle d'âge en âge ?
\VS{25}Le foin s'enlève et la verdure paraît, et les herbes des montagnes sont amassées.
\VS{26}Les agneaux sont pour te vêtir, et les boucs pour payer le champ ;
\VS{27}et l'abondance du lait des chèvres sera pour ta nourriture et celle de ta maison, et pour la subsistance de tes servantes.
\Chap{28}
\TextTitle{[Avertissements et conseils (suite)]}
\VerseOne{}Le méchant prend la fuite sans qu'on le poursuive, mais les justes seront assurés comme un jeune lion.
\VS{2}Il y a plusieurs chefs, à cause de la rébellion d'un pays, mais pour l'amour de l'homme avisé et intelligent, il y aura prolongation du même gouvernement.
\VS{3}L'homme qui est pauvre et qui opprime les pauvres, est comme une pluie violente qui cause la disette du pain.
\VS{4}Ceux qui abandonnent la loi louent le méchant, mais ceux qui gardent la loi leur font la guerre.
\VS{5}Les gens adonnés au mal n'entendent point ce qui est droit ; mais ceux qui cherchent Yahweh comprennent tout.
\VS{6}Le pauvre qui marche dans son intégrité vaut mieux que le pervers qui marche par deux chemins et qui est riche.
\VS{7}Celui qui garde la loi est un fils prudent, mais celui qui entretient les gourmands fait honte à son père.
\VS{8}Celui qui augmente ses biens par l'intérêt et l'usure, les amasse pour celui qui en fera des libéralités aux pauvres.
\VS{9}Celui qui détourne son oreille pour ne pas écouter la loi, sa prière même est une abomination\FTNT{La prière doit être faite selon la Parole de Dieu, en conformité avec sa volonté. Le Seigneur n'exauce que ceux qui obéissent à sa Parole (Mt. 6:9-10 ; Jn. 9:31 ; Jn. 15:7 ; 1 Jn. 5:14-15).}.
\VS{10}Celui qui égare les hommes droits dans le mauvais chemin tombera dans la fosse qu'il aura creusée, mais ceux qui sont intègres hériteront le bonheur.
\VS{11}L'homme riche pense être sage, mais le pauvre qui est intelligent le sondera.
\VS{12}Quand les justes se réjouissent, la gloire est grande, mais quand les méchants sont élevés, chacun se déguise.
\VS{13}Celui qui cache ses transgressions ne prospère point, mais celui qui les confesse et les délaisse, obtient miséricorde\FTNT{Ec. 1:8 ; 2 Pi. 2:14.}.
\VS{14}Heureux est l'homme qui est continuellement dans la crainte, mais celui qui endurcit son coeur tombera dans la calamité.
\VS{15}Le méchant qui domine sur un peuple pauvre est un lion rugissant et comme un ours quêtant sa proie.
\VS{16}Le conducteur qui manque d'intelligence fait beaucoup d'extorsions, mais celui qui hait le gain déshonnête prolonge ses jours.
\VS{17}L'homme chargé du sang d'une personne fuira jusqu'à la fosse sans qu'aucun ne le retienne.
\VS{18}Celui qui marche dans l'intégrité sera sauvé, mais le pervers qui suit deux chemins tombera tout à coup.
\VS{19}Celui qui laboure sa terre sera rassasié de pain, mais celui qui suit les fainéants sera accablé de misère.
\VS{20}L'homme fidèle abondera en bénédictions, mais celui qui se hâte de s'enrichir ne restera pas impuni.
\VS{21}Il n'est pas bon d'avoir égard à l'apparence des personnes, car pour un morceau de pain l'homme commet un crime.
\VS{22}L'homme qui a l'oeil malin se hâte pour avoir des richesses, et il ne sait pas que la disette lui arrivera.
\VS{23}Celui qui reprend les hommes obtient ensuite plus de faveur que celui qui flatte de sa langue.
\VS{24}Celui qui pille son père ou sa mère, et qui dit que ce n'est point un péché, est compagnon de l'homme dissipateur.
\VS{25}Celui qui a l'âme enflée excite les querelles, mais celui qui se confie en Yahweh sera rassasié.
\VS{26}Celui qui se confie dans son propre coeur est un fou, mais celui qui marche sagement sera délivré.
\VS{27}Celui qui donne au pauvre n'aura point de disette, mais celui qui en détourne ses yeux abondera en malédictions.
\VS{28}Quand les méchants s'élèvent, l'homme se cache ; mais quand ils périssent, les justes se multiplient.
\Chap{29}
\TextTitle{[Avertissements et conseils (suite)]}
\VerseOne{}L'homme qui étant repris, raidit son cou, sera subitement brisé et sans qu'il y ait de guérison.
\VS{2}Quand les justes sont nombreux, le peuple se réjouit ; mais quand le méchant domine, le peuple gémit.
\VS{3}L'homme qui aime la sagesse, réjouit son père, mais celui qui se plaît avec les femmes prostituées dissipe ses richesses.
\VS{4}Le roi affermit le pays par la justice, mais l'homme qui est adonné aux présents le ruinera.
\VS{5}L'homme qui flatte son prochain lui tend un piège sous ses pas.
\VS{6}Le péché de l'homme méchant lui tend un piège dangereux, mais le juste triomphe et se réjouit.
\VS{7}Le juste prend connaissance de la cause des pauvres, mais le méchant n'en prend pas connaissance.
\VS{8}Les hommes moqueurs troublent la ville, mais les sages apaisent la colère.
\VS{9}Un homme sage qui conteste avec un insensé, qu'il se fâche ou qu'il rie, la paix n'aura pas lieu.
\VS{10}Les hommes de sang ont en haine l'homme intègre, mais les hommes droits tiennent chère sa vie.
\VS{11}L'insensé pousse au-dehors tout ce qu'il a dans l'esprit, mais le sage le calme et le retient en arrière.
\VS{12}Tous les serviteurs d'un prince qui prêtent l'oreille à la parole de mensonge sont méchants.
\VS{13}Le pauvre et l'oppresseur se rencontrent, c'est Yahweh qui illumine les yeux de l'un et de l'autre.
\VS{14}Le trône du roi qui fait justice selon la vérité aux pauvres, sera établi à perpétuité.
\VS{15}La verge et la réprimande donnent la sagesse, mais l'enfant livré à lui-même fait honte à sa mère.
\VS{16}Quand les méchants se multiplient, les péchés s'accroissent, mais les justes verront leur ruine.
\VS{17}Corrige ton fils, et il te donnera du repos, et il procurera du plaisir à ton âme.
\VS{18}Lorsqu'il n'y a pas de vision\FTNT{Le manque de vision n'est bon pour personne. Dieu donne une vision aux personnes qu'il a appelées. La vision peut être un songe, une directive, une prophétie, etc. Il s'agit des objectifs à atteindre.}, le peuple est sans frein, mais heureux est celui qui garde la loi !
\VS{19}L'esclave ne se corrige pas par des paroles, même s'il comprend, il n'obéit pas.
\VS{20}As-tu vu un homme irréfléchi dans ses paroles ? Il y a plus à espérer d'un insensé que de lui.
\VS{21}Le serviteur qu'on a traité délicatement dès sa jeunesse finit par se croire un fils.
\VS{22}L'homme coléreux excite des querelles, et l'homme furieux commet beaucoup de péchés.
\VS{23}L'orgueil de l'homme l'abaisse, mais celui qui est humble d'esprit obtient la gloire\FTNT{Mt. 23:12 ; Lu. 14:11 ; 1 Pi. 5:5.}.
\VS{24}Celui qui partage avec un voleur hait son âme ; il entend la malédiction, et il ne révèle rien.
\VS{25}La crainte qu'on a des hommes tend un piège, mais celui qui se confie en Yahweh est élevé dans une haute retraite.
\VS{26}Plusieurs recherchent la face de celui qui domine, mais c'est de Yahweh que vient le jugement des hommes.
\VS{27}L'homme inique est en abomination aux justes, et celui dont la voie est droite est en abomination au méchant.
\Chap{30}
\TextTitle{[Proverbe d'Agur]}
\VerseOne{}Les paroles d'Agur, fils de Jaké, à savoir la sentence prononcée par cet homme pour Ithiel, pour Ithiel et Ucal.
\VS{2}Certainement je suis le plus stupide de tous les hommes, et il n'y a pas en moi l'intelligence humaine.
\VS{3}Et je n'ai pas appris la sagesse ; et je n'ai pas la connaissance des saints.
\VS{4}Qui est celui qui est monté aux cieux, ou qui en est descendu\FTNT{Jn. 3:13 ; Ro. 10:6-7.} ? Qui est celui qui a recueilli le vent dans le creux de sa main, qui a serré les eaux dans son manteau, qui a dressé toutes les bornes de la terre ? Quel est son nom, et quel est le nom de son fils, le sais-tu ?
\VS{5}Toute la parole de Dieu est éprouvée ; il est un bouclier pour ceux qui se réfugient en lui\FTNT{Ps. 18:31 ; Ps. 115:9-11.}.
\VS{6}N'ajoute rien à ses paroles, de peur qu'il ne te reprenne et que tu ne sois trouvé menteur.
\VS{7}Je te demande deux choses : Ne me les refuse pas durant ma vie.
\VS{8}Eloigne de moi la vanité et la parole mensongère ; ne me donne ni pauvreté ni richesse, nourris-moi du pain qui m'est nécessaire.
\VS{9}De peur que dans l'abondance je ne te renie, et que je ne dise : Qui est Yahweh ? Ou que dans la pauvreté, je ne dérobe et que je ne porte atteinte au Nom de mon Dieu.
\VS{10}N'accuse pas un serviteur devant son maître, de peur que ce serviteur ne te maudisse, et qu'il ne t'en arrive du mal.
\VS{11}Il est une race de gens qui maudit son père et qui ne bénit pas sa mère.
\VS{12}Il est une race de gens qui croit être pure, et qui toutefois n'est point lavée de son ordure.
\VS{13}Il est une race de gens dont les yeux sont fort hautains, et les paupières élevées.
\VS{14}Il est une race de gens dont les dents sont des épées, et les mâchoires sont des couteaux pour dévorer les malheureux sur la terre et les pauvres d'entre les hommes.
\VS{15}La sangsue a deux filles qui disent : Apporte ! Apporte ! Il y a trois choses qui sont insatiables, il y en a même quatre qui ne disent point : C'est assez !
\VS{16}Le scheol, la matrice stérile, la terre qui n'est pas rassasiée d'eau, et le feu qui ne dit jamais : C'est assez !
\VS{17}L'oeil de celui qui se moque de son père et qui méprise l'enseignement de sa mère, les corbeaux des torrents le crèveront, et les petits de l'aigle le mangeront.
\VS{18}Il y a trois choses qui sont trop merveilleuses pour moi, même quatre, que je ne connais point :
\VS{19}La trace de l'aigle dans le ciel, la trace du serpent sur un rocher, le chemin d'un navire au milieu de la mer, et la trace de l'homme chez la jeune femme.
\VS{20}Telle est la trace de la femme adultère : Elle mange, et s'essuie la bouche, puis elle dit : Je n'ai pas commis d'iniquité.
\VS{21}La terre tremble pour trois choses, même pour quatre, qu'elle ne peut supporter :
\VS{22}Pour l'esclave quand il vient à régner, pour l'insensé quand il est rassasié de pain,
\VS{23}pour la femme odieuse quand elle se marie, et pour la servante quand elle hérite de sa maîtresse.
\VS{24}Il y a quatre choses des plus petites de la terre qui toutefois sont bien sages entre les sages :
\VS{25}Les fourmis, qui sont un peuple sans force, et qui néanmoins préparent durant l'été leur nourriture ;
\VS{26}les damans, qui sont un peuple qui n'est pas puissant, et qui néanmoins font leurs maisons dans les rochers ;
\VS{27}les sauterelles, qui n'ont point de roi, et qui toutefois sortent toutes par divisions ;
\VS{28}les lézards que tu peux saisir avec les mains, et qui sont pourtant dans les palais des rois.
\VS{29}Il y a trois choses qui ont une belle allure, même quatre, qui ont une belle démarche :
\VS{30}Le lion, qui est le plus fort d'entre les animaux, et qui ne recule pas à la rencontre de qui que ce soit ;
\VS{31}le cheval, qui a les flancs bien troussés ; le bouc, et le roi devant qui personne ne résiste.
\VS{32}Si tu t'es conduit follement en t'emportant, et si tu as des mauvaises intentions, mets la main sur ta bouche.
\VS{33}Comme celui qui bat le lait en fait sortir le beurre, et comme celui qui presse le nez en fait sortir le sang, ainsi celui qui provoque la colère excite la querelle.
\Chap{31}
\TextTitle{Proverbe de Lemuel}
\VerseOne{}Les paroles du roi Lemuel et l'instruction que sa mère lui donna.
\VS{2}Quoi, mon fils ? Quoi, fils de mes entrailles ? Eh quoi, mon fils, pour lequel j'ai tant fait de voeux ?
\VS{3}Ne livre pas ta vigueur aux femmes et tes voies à celles qui perdent les rois.
\VS{4}Lemuel, ce n'est point aux rois, ce n'est point aux rois de boire le vin, ni aux princes de boire la cervoise\FTNT{La cervoise est une bière faite avec de l'orge ou d'autres céréales.}
\VS{5}de peur qu'ayant bu, ils n'oublient ce qui a été prescrit, et qu'ils n'altèrent le jugement de tous les pauvres affligés.
\VS{6}Donnez de la cervoise à celui qui va périr, et du vin à celui qui a l'amertume dans le coeur ;
\VS{7}afin qu'il en boive, et qu'il oublie sa pauvreté, et ne se souvienne plus de ses peines.
\VS{8}Ouvre ta bouche en faveur du muet, pour la cause de tous les fils délaissés qui vont périr.
\VS{9}Ouvre ta bouche, fais justice, et plaide pour le malheureux et l'indigent.
\TextTitle{[La femme vertueuse]}
\VS{10}[Aleph.] Qui est-ce qui trouvera une femme vertueuse ? Car son prix surpasse de beaucoup les perles.
\VS{11}[Beth.] Le coeur de son mari a confiance en elle, et il ne manquera point de dépouilles.
\VS{12}[Guimel.] Elle lui fera du bien tous les jours de sa vie, et jamais du mal.
\VS{13}[Daleth.] Elle cherche de la laine et du lin, et elle en fait ce qu'elle veut avec ses mains.
\VS{14}[He.] Elle est semblable aux navires d'un marchand, elle amène son pain de loin.
\VS{15}[Vav.] Elle se lève lorsqu'il est encore nuit, elle donne la nourriture nécessaire à sa maison et elle donne à ses servantes leur tâche.
\VS{16}[Zayin.] Elle pense à un champ, et l'acquiert ; et elle plante la vigne du fruit de ses mains.
\VS{17}[Heth.] Elle ceint ses reins de force, et affermit ses bras.
\VS{18}[Teth.] Elle sent que ce qu'elle gagne est bon ; sa lampe ne s'éteint pas pendant la nuit.
\VS{19}[Yod.] Elle met sa main à la quenouille, et ses doigts tiennent le fuseau.
\VS{20}[Kaf.] Elle tend sa main au malheureux, elle tend ses mains à l'indigent.
\VS{21}[Lamed.] Elle ne craint point la neige pour sa famille, car toute sa famille est vêtue de vêtements doubles.
\VS{22}[Mem.] Elle se fait des couvertures, le fin lin et l'écarlate sont ce dont elle s'habille.
\VS{23}[Nun.] Son mari est considéré aux portes, lorsqu'il siège avec les anciens du pays.
\VS{24}[Samech.] Elle fait des chemises et les vend, et elle livre des ceintures au marchand.
\VS{25}[Ayin.] Elle est revêtue de force et de gloire, elle se rit du jour à venir.
\VS{26}[Pe.] Elle ouvre sa bouche avec sagesse, et la loi de la charité est sur sa langue.
\VS{27}[Tsade.] Elle veille sur ce qui se passe dans sa maison, et elle ne mange pas le pain de la paresse.
\VS{28}[Qof.] Ses fils se lèvent et la disent bienheureuse ; son mari aussi, et il la loue, en disant :
\VS{29}[Resh.] Plusieurs filles se sont conduites vertueusement, mais toi, tu les surpasses toutes.
\VS{30}[Shin.] La grâce est trompeuse et la beauté vaine, mais la femme qui craint Yahweh est celle qui sera louée.
\VS{31}[Tav.] Récompensez-la du fruit de ses mains, et que ses oeuvres la louent aux portes.
\PPE{}
\end{multicols}

%\clearpage\ShortTitle{Job}\BookTitle{Job}\BFont
\noindent\hrulefill
{\footnotesize
\textit{
\bigskip
{\centering{}
\\Auteur : Inconnu
\\(Heb. : Iyov)
\\Signification : haï, ennemi et « je m'exclamerai »
\\Thème : La souffrance
\\Date de rédaction : Incertaine\\}
}
%\bigskip
\textit{
\\Job était un homme prospère et intègre auquel Dieu rendit témoignage. Il subit une succession de malheurs en très peu de temps en perdant tout ce qui lui était cher. Après avoir cherché à se justifier et subi les railleries de sa femme et les accusations de ses amis, Job s'humilia devant Dieu et comprit l'impuissance de sa propre justice. Cette histoire, dont on n'a aucune indication spatio-temporelle et qui pourtant parle à tous, est un encouragement pour le juste éprouvé.
%\bigskip
\\Rappelant que la souffrance peut être le moyen choisi par Dieu pour enseigner et se révéler, ce récit illustre la fidélité et la bonté de Yahweh envers ceux qui le craignent.\bigskip
}
}
\par\nobreak\noindent\hrulefill
\begin{multicols}{2}
\Chap{1}
\TextTitle{Job et sa famille}
\VerseOne{}Il y avait dans le pays d'Uts\FTNT{Ge. 36:28} un homme dont le nom était Job\FTNT{Ez. 14:14; Ja. 5:11.}. Cet homme était intègre\FTNT{1 R. 8:61.} et droit, craignant\FTNT{Ps. 19:10; Pr. 1:7.} Dieu et se détournant du mal.
\VS{2}Il eut sept fils et trois filles.
\VS{3}Et son bétail était de sept mille brebis, trois mille chameaux, cinq cents paires de bœufs, cinq cents ânesses, avec un très grand nombre de serviteurs\FTNT{Job 42:12-13.}; tellement que cet homme était le plus puissant de tous les Orientaux.
\VS{4}Or ses fils allaient et faisaient des festins les uns chez les autres chacun à son jour, et ils envoyaient appeler leurs trois sœurs pour manger et boire avec eux.
\VS{5}Quand les jours de festin étaient passés, Job envoyait chercher ses fils pour les sanctifier, et se levant de bon matin, il offrait un holocauste selon le nombre de ses enfants ; car Job disait : Peut-être mes fils ont-ils péché, et ont-ils blasphémé contre Dieu dans leurs cœurs. Job faisait toujours ainsi.\FTNT{Job 42:8.}
\VS{6}Or, il arriva un jour que les fils de Dieu\FTNT{Ps. 89:7 ; Job 38:7.} vinrent se présenter devant Yahweh, et Satan\FTNT{Es. 14:12; Ap. 12:9-10.} aussi vint au milieu d'eux.
\VS{7}Yahweh dit à Satan : D'où viens-tu ? Et Satan répondit à Yahweh : De courir çà et là sur la terre et de m'y promener\FTNT{1 Pi. 5:8.}.
\VS{8}Yahweh dit à Satan : N'as-tu point considéré mon serviteur Job, qui n'a point d'égal sur la terre ; homme intègre et droit, craignant Dieu, et se détournant du mal ?
\VS{9}Et Satan répondit à Yahweh : Est-ce en vain que Job craint Dieu ?
\VS{10}N'as-tu pas mis une haie \FTNT{La haie est une protection, une barrière végétale entretenue afin de protéger et de clôturer un terrain.} tout autour de lui, autour de sa maison, autour de tout ce qui lui appartient ? Tu as béni l'œuvre de ses mains, et ses troupeaux se répandent sur la terre.
\VS{11}Mais étends maintenant ta main, touche à tout ce qui lui appartient, et tu verras s'il ne te maudit pas en face.
\VS{12}Et Yahweh dit à Satan : Voilà, tout ce qui lui appartient est en ton pouvoir ; seulement ne porte pas la main sur lui. Et Satan sortit de devant la face de Yahweh\FTNT{1 R. 22:22.}.
\TextTitle{Première attaque de Satan}
\VS{13}Il arriva donc qu'un jour comme les fils et les filles de Job mangeaient et buvaient du vin dans la maison de leur frère aîné, un messager vint vers Job,
\VS{14}et lui dit : Les bœufs labouraient, et les ânesses paissaient à côté d'eux;
\VS{15}et ceux de Séba se sont jetés dessus, les ont pris, et ont frappé les serviteurs au fil de l'épée. Et je me suis échappé, moi seul, pour te l'annoncer.
\VS{16}Cet homme parlait encore, lorsqu'un autre vint et dit : Le feu de Dieu est tombé du ciel, il a brûlé les brebis et les serviteurs, et les a consumés\FTNT{2 R. 1:10-12.}. Et je me suis échappé moi seul, pour te l'annoncer.
\VS{17}Cet homme parlait encore, lorsqu'un autre vint et dit : Des Chaldéens\FTNT{Ge. 11:28.} ont fait trois bandes, se sont jetés sur les chameaux et les ont pris, ils ont frappé les serviteurs au fil de l'épée, et je me suis échappé moi seul, pour te l'annoncer.
\VS{18}Cet homme parlait encore, lorsqu'un autre vint et dit : Tes fils et tes filles mangeaient et buvaient du vin dans la maison de leur frère aîné ;
\VS{19}voici, un grand vent est venu de l'autre côté du désert et a frappé contre les quatre coins de la maison ; elle est tombée sur les jeunes gens, et ils sont morts. Et je me suis échappé, moi seul, pour te l'annoncer.
\VS{20}Alors Job se leva, déchira\FTNT{Job 2:12 ; Est. 4:1.} son manteau et  rasa la tête ; et se jetant par terre, se prosterna,
\VS{21}et dit : Je suis sorti nu du ventre de ma mère, et nu je retournerai dans le sein de la terre\FTNT{Ec. 5:14. ; 1 Ti. 6:7.} ; Yahweh a donné, Yahweh a enlevé\FTNT{1 S. 2:6.} ; que le nom de Yahweh soit béni !
\VS{22}En tout cela, Job ne pécha pas et n'attribua rien d'injuste à Dieu.
\Chap{2}
\TextTitle{Deuxième attaque de Satan}
\VerseOne{}Or il arriva un jour que les fils de Dieu vinrent un jour se présenter devant Yahweh, Satan\FTNT{Za. 3:1-2.} vint aussi au milieu d'eux se présenter devant Yahweh.
\VS{2}Yahweh dit à Satan : D'où viens-tu ? Satan répondit à Yahweh : De courir çà et là sur la terre et de m'y promener.
\VS{3}Yahweh dit à Satan: N'as-tu point considéré mon serviteur Job,qui n'a point d'égal sur la terre ; homme sincère et droit, craignant Dieu, et se détournant du mal ? Il demeure ferme dans son intégrité, quoique tu m'aies incité contre lui à le détruire sans cause\FTNT{Job 9:17.}.
\VS{4}Et Satan répondit à l’Eternel, en disant : Chacun donnera peau pour peau, et tout ce qu’il a, pour sa vie.
\VS{5}Mais étends maintenant ta main, et frappe à ses os et à sa chair\FTNT{Job 19:20.}, et tu verras s'il ne te maudit pas en face. 
\VS{6}Yahweh dit à Satan : Voici, il est en ta main : Seulement garde sa vie.
\VS{7}Ainsi Satan sortit de devant l’Eternel, et frappa Job d’un ulcère malin, depuis la plante de ses pieds jusqu’au sommet de la tête.
\VS{8}Job prit un tesson pour se gratter et s'assit au milieu de la cendre\FTNT{Jé. 6:26 ; Jon. 3:6.}.
\TextTitle{Réaction de Job et de sa femme}
\VS{9}Et sa femme lui dit : Conserveras-tu encore ton intégrité ?  Bénis\FTNT{Job 1:11.} Dieu, et meurs !
\VS{10}Et il lui dit : Tu parles comme une femme insensée ! Nous recevons le bien de la part de Dieu, et nous n'en recevrions pas le mal !\FTNT{Es. 45:7 ; Am. 3:6 ; La. 3:37.} En tout cela, Job ne pécha pas par ses lèvres.
\TextTitle{Job et ses trois amis}
\VS{11}Et trois des amis de Job, Eliphaz de Théman, Bildad de Schuach, et Tsophar de Naama, ayant appris tous les maux qui lui étaient arrivés, vinrent chacun du lieu de leur demeure, après s'être convenus ensemble d'un jour pour venir le plaindre et le consoler.
\VS{12}Ayant de loin levé les yeux sur lui, ils ne le reconnurent pas, alors ils élevèrent la voix et ils pleurèrent. Ils déchirèrent leurs manteaux, et jetèrent de la poussière vers le ciel au-dessus de leur tête.
\VS{13}Et ils s'assirent à terre avec lui, sept jours et sept nuits, et aucun d'eux ne lui dit une parole, car ils voyaient que sa douleur était fort grande.
\Chap{3}
\TextTitle{Lamentations de Job}
\VerseOne{}Après cela, Job ouvrit la bouche et maudit le jour de sa naissance.\FTNT{Jé. 20:14 ; Job 10:18.}
\VS{2}Car prenant la parole, il dit :
\VS{3}Périsse le jour où je suis né, et la nuit qui a dit : Un homme est conçu !
\VS{4}Que ce jour-là ne soit que ténèbres ; que Dieu ne le recherche point d'en haut, et qu'il ne soit point éclairé de la lumière ! 
\VS{5}Que les ténèbres et l'ombre de la mort\FTNT{Job 10:21-22.} s'en emparent, que les nuées demeurent sur lui, qu'il soit rendu terrible comme le jour de ceux à qui la vie est amère ! 
\VS{6}Que l'obscurité prenne cette nuit, qu'elle ne se réjouisse pas au milieu des jours de l'année, qu'elle n'entre pas dans le compte des mois !
\VS{7}Voici, que cette nuit soit stérile, et qu'aucun cri de joie n'y survienne !
\VS{8}Qu'ils la maudissent ceux qui maudissent les jours, ceux qui sont prêts à réveiller le Léviathan !
\VS{9}Que les étoiles de son crépuscule soient obscurcies ; qu'elle attende la lumière, mais qu'il n'y en ait point, et qu'elle ne voie point les rayons de l'aube du jour ! \FTNT{Job 41:9.} !
\VS{10}Parce qu'elle n'a pas fermé le sein qui me conçut ni caché la souffrance à mes yeux.
\VS{11}Pourquoi ne suis-je pas mort dans le sein de ma mère ? Pourquoi n'ai-je pas expiré aussitôt que je suis sorti de ses entrailles ?\FTNT{Job 10:18.}
\VS{12}Pourquoi des genoux m'ont-ils reçu? Pourquoi des mamelles m'ont-elles allaité ?
\VS{13}Je serais couché maintenant, je h serais tranquille, je dormirais, je me reposerais\FTNT{Job 17:16.},
\VS{14}avec les rois et les grands de la terre, qui se bâtirent des mausolées,
\VS{15}avec les princes qui possedèrent de l'or, et qui remplirent d'argent leurs maisons.
\VS{16}Ou comme l'avorton caché, je n'existerais pas\FTNT{Ps. 58:9.}, comme les petits enfants qui n'ont pas vu la lumière.
\VS{17}Là les méchants n'agitent plus personne, et là se reposent ceux qui sont fatigués. 
\VS{18}Pareillement ceux qui avaient été dans les liens, jouissent là du repos, et n'entendent plus la voix de l'oppresseur. 
\VS{19}le petit et le grand sont là, et l'esclave est délivré de son maître.
\VS{20}Pourquoi la lumière est-elle donnée au misérable, et la vie à ceux qui ont le cœur dans l'amertume ;
\VS{21}qui désirent en vain la mort, et qui la recherchent plus que le trésor,\FTNT{Ap. 9:6.}
\VS{22}qui seraient ravis de joie et seraient dans l'allégresse s'ils avaient trouvé le tombeau ?
\VS{23}Pourquoi, dis-je, la lumière est-elle donnée à l'homme à qui le chemin est caché, et que Dieu a enfermé de toutes parts\FTNT{Job 19:8 ; La. 3:7.} ?
\VS{24}Car avant que je mange, mon soupir vient, et mes cris se répandent comme de l'eau. 
\VS{25}Ce que je crains le plus, m'arrive, et ce que je redoute le plus, m'atteint. 
\VS{26}Je n'ai point eu de paix, je n'ai point eu de repos, ni de calme, depuis que ce trouble m'est arrivé. 
\Chap{4}
\TextTitle{Premier discours d'Eliphaz}
\VerseOne{}Alors Eliphaz de Théman prit la parole et dit :
\VS{2}Si l'on tente de te parler, en seras-tu peiné ? Mais qui pourrait retenir ses paroles ?
\VS{3}Voici, tu as souvent instruit les autres, et tu as fortifié les mains affaiblies\FTNT{Es. 35:3 ; Hé. 12:12.},
\VS{4}Tes paroles ont affermi ceux qui chancelaient, et tu as fortifié les genoux qui pliaient\FTNT{Job 16:5.}.
\VS{5}Et maintenant que le malheur t'arrive, tu faiblis ! Maintenant que tu es atteint, tu en es tout troublé !
\VS{6} Ta crainte de Yahweh n'a-t-elle pas été ton espérance ? Et l'intégrité de tes voies n'a-t-elle pas été ton attente ? 
\VS{7}Rappelle, je te prie, dans ton souvenir : Quel est l'innocent qui a péri ? Quels sont les justes qui ont été exterminés ?\FTNT{Job 8:20.}
\VS{8}Selon ce que j'ai vu, ceux qui labourent l'iniquité et qui sèment la peine en moissonnent les fruits ;\FTNT{Job 15:35 ; Ga. 6:7.}
\VS{9}ils périssent par le souffle de Dieu, et ils sont consumés par le vent de ses narines.\FTNT{Ex. 15:8 ; Es. 11:4 ; 30:33 ; Job 15:30 ; 2 Th. 2:8.}
\VS{10}Il étouffe le rugissement du lion, et le cri d'un grand lion, et il arrache les dents des lionceaux ;
\VS{11}le lion périt faute de proie, et les petits de la lionne sont dispersés.
\VS{12}Une parole m'est furtivement arrivée, et mon oreille en a saisi les sons légers.
\VS{13}Au moment où les visions de la nuit agitent la pensée, quand un profond sommeil tombe sur les hommes\FTNT{Job 33:15.},
\VS{14}une frayeur et un tremblement me saisirent, et tous mes os tremblèrent.
\VS{15}Un esprit passa devant moi, et mes cheveux en furent tout hérissés. 
\VS{16}Il se tint là et je ne reconnus pas son visage ; une figure était devant mes yeux. Et j'entendis un léger murmure et une voix :
\VS{17}L'homme serait-il juste devant Dieu ? L'homme serait-il pur devant celui qui l'a fait ?\FTNT{Job 25:4.}
\VS{18}Voici, il ne se fie pas à ses serviteurs, il trouve des erreurs à ses anges\FTNT{Job 15:15 ; Job 25:5 ; 2 Pi 2:4. }
\VS{19}combien plus chez ceux qui habitent des maisons d'argile, qui ont leurs fondements dans la poussière, qu'on écrase comme des vermisseaux !\FTNT{Job 25:6.}
\VS{20}Du matin au soir ils sont brisés, et, sans qu'on s'en aperçoive, ils périssent pour toujours. 
\VS{21}L'excellence qui était en eux, n'a-t-elle pas été emportée ? Ils meurent sans être sages. 
\Chap{5}
\VerseOne{}Crie maintenant ! Y aura-t-il quelqu'un qui te réponde ? Et vers quel saint te tourneras-tu ?\FTNT{Job 15:15.}
\VS{2}La colère tue l'insensé, et le fou meurt dans ses emportements.
\VS{3}J'ai vu l'insensé qui s'enracinait \FTNT{Jé. 12:1-2.}, mais j'ai aussitôt maudit sa demeure.
\VS{4}Ses fils sont loin de tout secours ; ils sont écrasés à la porte, et personne ne les délivre !\FTNT{Ps. 119:155.}
\VS{5} Sa moisson est dévorée par l'affamé, qui même la ravit d'entre les épines ; et le voleur convoite ses biens.
\VS{6}Le malheur ne sort pas de la poussière, et le travail ne germe pas de la terre ;
\VS{7}l'homme naît pour la peine\FTNT{Ge. 3:17-19 ; Job 14:1-5.}, comme l'étincelle pour voler et s'élever.
\VS{8}Mais moi, j'aurais recours à Dieu, et j'adresserais ma parole à Dieu.
\VS{9}Il fait de grandes choses qu'on ne peut sonder, de merveilleuses choses qu'on ne peut compter\FTNT{Ps. 72:18. Ps. 92:5 ; Job 9:10.}.
\VS{10}Il répand la pluie sur la face de la terre, et envoie les eaux sur les campagnes\FTNT{De. 28:12 ; Ps. 135:7 ; Job 28:26; Job 38:25-26 ; Ac. 14:17.};
\VS{11}il met en haut ceux qui sont abaissés, et délivre les affligés\FTNT{1 S. 2:7; Ez. 21:31 ; Ps. 113:7-8.} ;
\VS{12}il anéantit les projets des hommes rusés, de sorte qu'ils ne viennent pas à bout de leurs entreprises\FTNT{Es. 8:10 ; Ps. 33:10 ; Né. 4:15.} ;
\VS{13}il prend les sages dans leur propre ruse\FTNT{1 Co. 3:19.}, et les desseins des hommes pervers sont renversés :
\VS{14}De jour ils rencontrent les ténèbres, et ils marchent à tâtons en plein midi, comme dans la nuit.
\FTNT{De. 28:29.}.
\VS{15}Ainsi Dieu délivre le pauvre de l'épée de leur bouche, et le sauve de la main des puissants\FTNT{Ps. 12:3-4; Ps 52:2; Ps. 57:4.} ;
\VS{16}et l'espérance soutient le malheureux\FTNT{1S. 2:8.}, et la méchanceté a la bouche fermée\FTNT{Es. 52:15 ; Ps. 63: 11; Ps. 107: 42; Pr. 10:6.}.
\VS{17}Voici, heureux est celui que Yahweh châtie ! Ne rejette donc point le châtiment de Yahweh.
\FTNT{Ps 94:12 ; Pr.3:11-12 ; Hé. 12:5-6; Ap. 3:19.}.
\VS{18}Car c'est lui qui fait la plaie, et la bande ; il blesse et ses mains guérissent\FTNT{De. 32:39; 1S. 2: 6-7 ; Cp. Es. 30:26 ; Os. 6:1.}.
\VS{19}Six fois il te délivrera de l'angoisse, et sept fois le mal ne te touchera pas\FTNT{Ps 34:20; Ps. 91:3; Pr.24:16.}.
\VS{20}Il te sauvera de la mort pendant la famine, et du tranchant de l'épée pendant la guerre\FTNT{Ps. 33: 19; Ps. 37:19.}.
\VS{21}Tu seras à l'abri du fléau de la langue, et tu n’auras point peur de la dévastation, quand elle arrivera.
\FTNT{Ps. 31:21.}.
\VS{22}Tu riras de la dévastation et de la famine, et tu n'auras pas peur des bêtes de la terre\FTNT{Es. 65:25; Ez. 34:25; Os. 2:20.};
\VS{23}car tu feras une alliance avec les pierres des champs, et les bêtes des champs seront en paix avec toi\FTNT{Os. 2:20.}.
\VS{24}Tu jouiras en paix de la prospérité sous ta tente, tu pourvoiras à ta demeure et tu n'y seras point trompé ;
\VS{25}tu verras ta postérité s'accroître, et tes descendants se multiplier comme l'herbe de la terre\FTNT{Ps. 72:16; Ps. 127: 3-5; Ps. 128:6.}.
\VS{26}Tu entreras au tombeau dans ta vieillesse, comme une gerbe qu'on emporte en son temps\FTNT{Pr. 9:11; Pr. 10:27.}.
\VS{27}Voilà ce que nous avons examiné, voilà ce qui est ; à toi d'entendre et de choisir.
\Chap{6}
\TextTitle{Réponse de Job}
\VerseOne{}Job prit la parole et dit :
\VS{2}Oh ! si l’on pesait ma douleur, et si l’on mettait en même temps mes calamités dans la balance !
\VS{3}Car elle serait plus pesante que le sable de la mer ; c’est pourquoi mes paroles sont englouties.
\FTNT{Pr. 27:3.} !
\VS{4}Car les flèches du Tout-Puissant sont sur moi, mon âme en boit le venin ; les terreurs\FTNT{Job 30:15 ; Ps. 88:16-17.} de Dieu se rangent en bataille contre moi\FTNT{Job 19:12 ; Ps. 38:2-3.}.
\VS{5}L'âne sauvage\FTNT{Job 39:8.} brait-il auprès de l’herbe ? Le bœuf mugit-il auprès de son fourrage ?
\VS{6}Mange-t-on sans sel ce qui est fade ? Trouve-t-on du goût dans un blanc d’œuf ?
\VS{7}Ce que mon âme voudrait ne pas toucher, c'est là ma nourriture, si dégoûtante soit-elle !
\VS{8}Oh ! Puisse ma prière s'accomplir et Dieu me donner ce que j'attends !
\VS{9}Qu’il plaise à Dieu de me réduire en poussière, qu’il laisse aller sa main pour m’achever !
!\FTNT{Job 7:16; 9:21; 10:1; cp. No. 11:15; 1R. 19:4; Jon. 4:3, 8.}
\VS{10}Mais j’ai encore cette consolation, quoique la douleur me consume, et qu’elle ne m’épargne point, je n'ai pas transgressé les paroles du Saint.
\VS{11}Quelle est ma force pour que j’espère, et quelle est ma fin pour que je prenne patience ?
\VS{12}Ma force est-elle une force de pierre ? Ma chair est-elle d'airain ?
\VS{13}Ne suis-je pas sans secours, et le salut n'est-il pas loin de moi ?
\VS{14}A celui qui souffre, est due la compassion de son ami ; mais il a abandonné la crainte. \FTNT{Ps. 19:10.} du Tout-Puissant\FTNT{Pr. 17:17.}.
\VS{15}Mes frères m'ont trompé comme un torrent, comme le lit des torrents qui passent\FTNT{Ps. 38:12; Ps 41:10; Ps 69:9; Jé. 15:19.}.
\VS{16}Les glaçons en troublent le cours, la neige s'y cache ;
\VS{17}mais au temps de la sécheresse, ils tarissent, et dans les chaleurs, ils disparaissent de leur place.
\VS{18}Les caravanes se détournent de leur route, elles montent dans le désert et périssent.
\VS{19}Les caravanes de Théma\FTNT{Ge. 25:15.} fixent le regard, les voyageurs de Séba\FTNT{1R. 10:1; Ps. 72:10; Ez. 27:22-23.} s'attendent à eux;
\VS{20}ils sont honteux d'avoir eu cette confiance, ils restent confondus quand ils arrivent.
\VS{21}Certes, vous m'êtes devenus inutiles ; vous voyez mon angoisse, et vous en avez horreur !\FTNT{Job 19:13 ; Ps. 31:12.}
\VS{22}Mais vous ai-je dit : Donnez-moi quelque chose, et de vos biens, faites des présents en ma faveur ? 
\VS{23}délivrez-moi de la main de l'ennemi, et rachetez-moi de la main des violents ?
\VS{24}Instruisez-moi, et je me tairai ; faites-moi comprendre en quoi je me suis égaré.
\VS{25}Ô combien sont fortes les paroles de vérité ! Mais que veut censurer votre argumentation ?
\VS{26}Voulez-vous donc blâmer ce que j'ai dit, et ne voir que du vent dans les paroles d'un homme désespéré ?\FTNT{Ec. 9:16.}
\VS{27}Vous vous jetez même sur un orphelin, vous persécutez votre ami.
\VS{28}Regardez-moi, je vous prie ! Et voyez si je vous mens en face ?
\VS{29}Revenez\FTNT{Job 17:10.} donc, soyez sans injustice; revenez, et reconnaissez mon innocence\FTNT{Job 27:5-6 ; 34:5 ; cp. Job 23:10 ; 42:1-6.}.
\VS{30}Y a-t-il de l'injustice dans ma langue ? et mon palais ne sait-il pas discerner le mal ? 
\Chap{7}
\VerseOne{}N'y a-t-il pas un temps de guerre limité à l'homme sur la terre ? Et ses jours ne sont-ils pas comme les jours d'un mercenaire ?
\VS{2}Comme un esclave, il soupire après l'ombre, comme un mercenaire\FTNT{Es. 16:14.}, il attend son salaire\FTNT{Ps. 39:5.}.
\VS{3}Ainsi j'ai reçu en partage des mois en vain, et l'on m'a assigné des nuits de peine\FTNT{Ps. 6:6.}.
\VS{4}Si je suis couché, je dis : Quand me lèverai-je ? Quand finira la nuit ? Et je suis rassasié d'agitations jusqu'au point du jour\FTNT{De. 28:67.}.
\VS{5}Ma chair se couvre de vers et d'une croûte terreuse, ma peau se crevasse et coule.
\VS{6}Mes jours sont plus rapides que la navette du tisserand, ils se consument sans espoir !\FTNT{Es. 38:12 ; Job 9:25 ; 17:11; Ja. 4:14.}
\VS{7}Souviens-toi que ma vie est un souffle ! Et que mes yeux ne reverront plus le bonheur\FTNT{Es. 40:6 ; Ps. 78:39 ; Ps. 89:48 ; Ps. 102: 12 ; Ps. 103:15 ; Job 8:9 ; Job 14:1-2 ; 1P. 1:24.}.
\VS{8}L'œil de ceux qui me regarndent ne me verra plus ; tes yeux seront sur moi, et je ne serai plus.
\VS{9}La nuée se dissipe et s'en va, ainsi celui qui descend au scheol\FTNT{cp. Ha. 2:5 ; Lu. 16:23.} ne remontera pas\FTNT{Job 10:21-22 ; Job 14:7-14.};
\VS{10}il ne reviendra plus dans sa maison, et le lieu qu'il habitait ne le reconnaîtra plus\FTNT{Ps. 37:35-36 ; Ps. 103:16 ; Job 10:21.}.
\VS{11}C'est pourquoi, je ne retiendrai pas ma bouche, je parlerai dans l'angoisse de mon esprit, je me plaindrai dans l'amertume de mon âme\FTNT{Job 10:1.}.
\VS{12}Suis-je une mer ? Suis-je un monstre marin, pour que tu poses autour de moi des gardes ?
\VS{13}Quand je dis : Mon lit me consolera, ma couche calmera ma plainte,
\VS{14}alors tu me terrifies par des songes, et tu m'épouvantes par des visions.
\VS{15}C'est pourquoi je choisirais d'être étranglé, et de mourir, plutôt que de conserver mes os.
\VS{16}Je les méprise !… Je ne vivrai pas toujours… Laisse-moi car mes jours sont un souffle\FTNT{Job 10:20.}.
\VS{17}Qu'est-ce que l'homme pour que tu en fasses tant de cas, pour que tu poses ta main sur son cœur,\FTNT{Ps. 8:5 ; Ps. 144:3 ; Hé. 2:6.}
\VS{18}pour que tu le visites tous les matins, pour que tu l'éprouves\FTNT{Job 23:10.} à chaque instant ?
\VS{19}Quand finiras-tu de me regarder? Ne me lâcheras-tu pas, pour que j'avale ma salive ?\FTNT{Job 9:18.}
\VS{20}J'ai péché ; que te ferai-je, gardien des hommes ?  \FTNT{1 Ti. 4:10.}  Pourquoi m'as-tu mis en butte à tes coups, et pourquoi suis-je à charge à moi-même ?
\VS{21}Et pourquoi ne pardonnes-tu pas mon péché, et ne fais-tu pas passer mon iniquité ? Car je vais maintenant me coucher dans la poussière ; tu me chercheras, et je ne serai plus.
\Chap{8}
\TextTitle{Premier discours de Bildad}
\VerseOne{}Bildad de Schuach prit la parole et dit :
\VS{2}Jusqu'à quand parleras-tu ainsi, et les paroles de ta bouche seront-elles un vent impétueux ?\FTNT{Job 15:2.}
\VS{3}Dieu renverserait-il le droit, et le Tout-puissant renverserait-il la justice ? \FTNT{Cp. Ge. 18:25.} ?\FTNT{De. 32:4 ; Job 34:12 ; Da. 9:14 ; 2 Ch. 19:7.}
\VS{4}Si tes fils ont péché contre lui, il les a livrés à leur crime.
\VS{5}Mais toi si tu cherches Dieu, si tu demandes grâce au Tout-Puissant ;\FTNT{Cp. Job 5:17-27.}
\VS{6}si tu es pur et droit, il veillera certainement sur toi, il rendra le bonheur à la demeure de ta justice ;
\VS{7}tes commencements\FTNT{Za. 4:10.} auront été peu de chose, et ta fin sera bien plus grande.\FTNT{Job 42:12.}
\VS{8}Interroge ceux des générations précédentes, applique-toi à l'expérience de leurs pères.\FTNT{De. 4:32 ; De. 32:7.}
\VS{9}Car nous sommes d'hier, et nous ne savons rien, parce que nos jours sur la terre ne sont qu'une ombre.\FTNT{Ps. 102:12 ; Ps. 144:41 ; Ch. 29:15.}
\VS{10}Ils t'instruiront, ils te parleront, ils tireront de leur cœur ces discours :
\VS{11}Le roseau croît-il sans marais ? Le jonc pousse-t-il sans eau ?
\VS{12}Il est encore en sa verdure, sans qu'on le coupe, il sèche plus vite que toutes les herbes.\FTNT{Cp. Jé. 17:5-8 ; Ps. 129:6.}
\VS{13}Ainsi est la voie de tous ceux qui oublient Dieu\FTNT{Ps. 9:18.}, et l'espérance de l'impie périra\FTNT{Ps. 1:4 ; Ps. 112:10 ; Pr. 10:28 ; Job 11:20 ; Job 27:8.}.
\VS{14}Sa confiance est brisée, son soutien est une toile d'araignée.
\VS{15}Il s'appuie sur sa maison, et elle ne tient pas ; il s'y cramponne, et elle ne reste pas debout.
\VS{16}Dans toute sa vigueur, en plein soleil, il étend ses rameaux sur son jardin,
\VS{17}mais ses racines s'entrelacent parmi des monceaux de pierres, il pénètre dans les rochers.
\VS{18}S'Il l'ôte de sa place, celle-ci le renie, disant je ne t'ai pas connu ! 
\VS{19}Telle est la joie que ses voies lui procurent. Puis sur le même sol, d'autres s'élèvent après lui.
\VS{20}Dieu ne rejette pas l'homme intègre, il ne soutient pas la main des méchants.\FTNT{Job 4:7.}
\VS{21}Il remplira encore ta bouche de cris de joie, et tes lèvres de chants d'allégresse.\FTNT{Ps. 126:2.}
\VS{22}Ceux qui te haïssent seront revêtus de honte, et la tente des méchants ne sera plus. \FTNT{Ps. 35:26 ; Ps. 109:29.}
\Chap{9}
\TextTitle{Réponse de Job}
\VerseOne{}Job prit la parole et dit :
\VS{2}Certainement, je sais qu'il en est ainsi ; et comment l'homme mortel se justifierait-il devant Dieu ? \FTNT{Ha. 2:4 ; Ga. 3:11 ; Ro. 1:17 ; Hé. 10:38.} devant Dieu ?\FTNT{Ps. 25:4 ; Ps. 143:2 ; Job 15:14-16 ; Da. 9:11 ; Ro 3:19.}
\VS{3}S'il veut plaider avec lui, il ne lui répondra pas une fois sur mille. \FTNT{Es. 45:9-10.}
\VS{4} Dieu est sage de coeur, et puissant en force. Qui est-ce qui s'est opposé à lui, et s'en est bien trouvé ? \FTNT{Job 12:13 ; Job 36:5 ; Job 37:23.}
\VS{5}Il transporte les montagnes, et quand il les renverse dans sa fureur, elles n'en connaissent rien.\FTNT{Ps. 144:5.}
\VS{6}Il remue la terre de sa place, et ses piliers sont ébranlés.\FTNT{Ag. 2:6, 21 ; Hé. 12:26.}
\VS{7}Il commande au soleil, et le soleil ne se lève pas ; et il met un sceau sur les étoiles.\FTNT{Jos. 10:12.}
\VS{8} C'est lui seul qui étend les cieux\FTNT{Ge 1:6-8 ; Es. 44:24; Es. 51:13 ; Ps. 104:2.},qui marche sur les hauteurs de la mer\FTNT{Cp. Mt. 14:25.}.
\VS{9}Il a fait la grande ourse, l'orion, les pléiades, et les étoiles des régions australes.\FTNT{Ge. 1:16 ; Am. 5:8 ; Ps. 89:12 ; Job 38:31-32.}
\VS{10}Il fait de grandes choses qu'on ne peut sonder, des merveilles sans nombre.\FTNT{Ps. 86:10 ; Ps. 139:6, 17-18 ; Job 5:9 ; Job 37:5.}
\VS{11}Voici, il passe près de moi, et je ne le vois pas ; il passe encore, et je ne l'aperçois pas.\FTNT{Job 23:8-9 ; 35:14.}
\VS{12}S'il enlève, qui l'en détournera? Qui lui dira : Que fais-tu ?\FTNT{Es. 45: 9-10 ; Da. 4:35 ; Ro. 11:33-35.}
\VS{13}Dieu ne revient pas sur sa colère ; sous lui s'inclinent les appuis de l'orgueil.\FTNT{Job 26:12; Cp. Es. 30:7.}
\VS{14}Combien moins lui répondrais-je, moi et comment choisirais-je mes paroles contre lui ? 
\VS{15}Quand je serais juste, je ne répondrais pas ; je demanderais grâce à mon juge.\FTNT{Job 23:1-7.}
\VS{16}Si je l'invoque et qu'il me réponde, ne croirais-je pas qu'il ait écouté ma voix,
\VS{17}lui qui m'assaille comme par une tempête, qui multiplie mes plaies sans motif,\FTNT{Job 6:29.}
\VS{18}qui ne me permet pas de reprendre haleine ; qui me rassasie d'amertume.\FTNT{Job 7:19.}
\VS{19}S'il est question de savoir qui est le plus fort ; voilà, il est fort ; et s'il est question d'aller en justice, qui est-ce qui m'y fera comparaître ? 
\VS{20}Si je me justifie, ma propre bouche me condamnera ; si je me fais parfait, il me convaincra d'être coupable.
\VS{21}Je suis innocent ! Je ne me soucie pas de vivre, je méprise ma vie.\FTNT{Job 10:1.}
\VS{22}Tout se vaut! C'est pourquoi j'ai dit: Il détruit l'innocent comme l'impie. \FTNT{ Cp. Ez. 21:3 ; Ec. 9:2-3 ; Mt 5:45.}
\VS{23}Au moins si le fléau dont il frappe faisait mourir tout aussitôt ; mais il se rit de l'épreuve des innocents. 
\VS{24}[C'est par lui que] la terre est livrée entre les mains du méchant ; c'est lui qui couvre la face des juges de la [terre] ; et si ce n'est pas lui, qui est-ce donc ? 
\VS{25}Or mes jours vont plus vite qu'un courrier ; ils s'en fuient sans avoir vu le bonheur ;\FTNT{Job 7:6-7.}
\VS{26}ils passent comme les navires de roseaux, comme l'aigle qui fond sur sa proie.
\VS{27}i je dis : J'oublierai ma plainte, je renoncerai à ma colère, je me fortifierai; 
\VS{28}Je suis épouvanté de tous mes tourments. Je sais que tu ne me jugeras pas innocent. .\FTNT{Cp. Ps. 130:3.}
\VS{29}Je serai jugé coupable ; pourquoi travaillerais-je en vain ?
\VS{30}Quand je me laverais dans de l'eau de neige, et que je nettoierais mes mains dans la pureté, \FTNT{Jé. 2:22.}
\VS{31}tu me plongerais dans le fossé, et mes vêtements m'auraient en horreur.
\VS{32}Car il n'est pas comme moi un homme, pour que je lui réponde, [et] que nous allions ensemble en jugement. .\FTNT{Es. 45:9 ; Jé 49:19 ; Ec. 6:10; Ro. 9:20.}
\VS{33}Mais il n'y a personne qui prend connaissance de la cause qui serait entre nous, et qui pose la main sur nous deux. \FTNT{Cp. 1 S. 2:25.}
\VS{34}Qu'il ôte donc sa verge de dessus moi, et que la frayeur que j'ai de lui ne me trouble plus. ;
\VS{35}Je parlerai, et je ne le craindrai pas ; mais dans l'état où je suis je ne suis plus à moi-même. 
\Chap{10}
\VerseOne{}Mon âme a pris en dégoût la vie ! Je laisserai aller ma plainte, je parlerai dans l'amertume de mon âme.
\VS{2}Je dirai à Dieu : Ne me condamne pas ; montre-moi pourquoi tu plaides contre moi ?
\VS{3}Te plais-tu à m'opprimer, et à dédaigner l'ouvrage de tes mains, et à bénir les desseins des méchants\FTNT{Es. 64:7-8.} ?
\VS{4}As-tu des yeux de chair ? Vois-tu comme voit un homme mortel?
\VS{5}Tes jours sont-ils comme les jours de l'homme mortel ? Tes années sont-elles comme les jours de l'homme, 
\VS{6}Que tu recherches mon iniquité, et que tu t'informes de mon péché,
\VS{7}tu sais que je n'ai point commis de crime, et qu'il n'y a personne qui me délivre de ta main?
\VS{8}Tes mains m'ont formé, et elles ont rangé toutes les parties de mon corps; et tu me détruirais !\FTNT{Ge. 2:7 ; Ps. 119:73 ; Ps. 139:14-15.} !
\VS{9}Souviens-toi, je te prie, que tu m'as formé comme de la boue, et que tu me feras retourner en poudre?
\VS{10}Ne m'as-tu pas coulé comme du lait ? et ne m'as-tu pas fait cailler comme un fromage ?
\VS{11}Tu m'as revêtu de peau et de chair, et tu m'as composé d'os et de nerfs;
\VS{12}Tu m'as donné la vie, et tu as usé de miséricorde envers moi, et [par] tes soins continuels tu as gardé mon esprit.
\VS{13}Et cependant tu gardais ces choses en ton cœur ; mais je connais que cela était devant toi. 
\VS{14}Si j'ai pèche, tu m'observes, et tu ne me tiens pas pour innocent de mon iniquité.
\VS{15}Si j'agis méchamment, malheur à moi ! si je suis juste, je n'en lève pas la tête plus haut. Je suis rempli d'ignominie ; mais regarde mon affliction. 
\VS{16}Si je redresse la tête, tu me poursuis comme à un lion, et tu multiplie tes exploits contre moi; \FTNT{Zq. 38:13 ; La. 3:10.}.
\VS{17}Tu renouvelles tes témoins contre moi, et ton indignation augmente contre moi. De nouvelles troupes toutes fraîches viennent contre moi.
\VS{18}Mais pourquoi m'as-tu fait sortir du sein de ma mère? J'aurais expiré, et aucun œil ne m'aurait vu ;
\VS{19}et j'aurais été comme n'ayant jamais été, et j'aurais été porté du ventre de ma mère au tombeau.
\VS{20}Mes jours ne sont-ils pas en petit nombre ? Cesse donc et retire-toi de moi, et que je me renforce un peu. .
\VS{21}Avant que j'aille au lieu d'où je ne reviendrai plus, en la terre de ténèbres et de l'ombre de la mort,
\VS{22}terre d'une grande obscurité, comme les ténèbres de l'ombre de la mort, où il n'y a aucun ordre, et où rien ne luit que des ténèbres. 
\Chap{11}
\TextTitle{Première accusation de Tsophar}
\VerseOne{}Tsophar de Naama prit la parole et dit :
\VS{2}Ne répondra-t-on point à tant de discours, et suffira-t-il d'être un grand parleur pour être justifié ?
\VS{3}Tes vains feront-ils taire les gens ? Et quand tu te seras moqué, n'y aura-t-il personne qui te fasse honte ?
\VS{4}Car tu as dit : Ma doctrine est pure, et je suis sans tache devant tes yeux. 
\VS{5}Mais je voudrais que Dieu parle, et qu'il ouvre sa bouche pour te répondre; ,
\VS{6}Qu'il te montre les secrets de sa sagesse, de son immense sagesse; et que tu reconnaisse que Dieu oublie une partie de ton iniquité. .
\VS{7}Trouveras-tu Dieu en le sondant ? Connaîtras-tu parfaitement le Tout-puissant ? 
\VS{8}Ce sont les hauteurs des cieux : Qu'y feras-tu ? C'est plus profond que le scheol : Qu'y connaîtras-tu ?
\VS{9}Son étendue est plus longue que la terre, et plus large que la mer.
\VS{10}S'il remue, et qu'il resserre, ou qu'il rassemble, qui l'en détournera ?
\VS{11}Car il connaît les hommes vicieux, il discerne par le regard les coupables\FTNT{Ps. 10:11-14 ; Ps. 35:22.}.
\VS{12}Mais l'homme vide de sens devient intelligent, quoique l'homme naisse comme un ânon sauvage\FTNT{Ec. 3:18.}.
\VS{13}Si tu disposes ton cœur, et que tu étendes tes mains vers lui,
\VS{14}Si tu éloignes de toi l'iniquité qui est en ta main, et si tu ne permets pas que la méchanceté habite dans tes tentes ; 
\VS{15}Alors certainement tu pourras élever ton visage sans tache ; tu seras ferme et tu ne craindras rien;
\VS{16}tu oublieras tes peines, tu t'en souviendras comme des eaux écoulées.
\VS{17}La vie se lèvera pour toi plus brillante que le midi, et l'obscurité même sera comme le matin\FTNT{Ps. 37:6 ; Ps. 112:4.}.
\VS{18} Tu seras plein de confiance, parce qu'il y aura de l'espérance pour toi; tu creuseras, et tu reposeras sûrement. \FTNT{Lé. 26:6 ; Ps. 3:6 ; Pr. 3:24.}.
\VS{19}Tu te coucheras, et il n'y aura personne qui t'épouvante, et plusieurs te feront la cour. 
\VS{20}Mais les yeux des méchants seront consumés; tout refuge leur sera ôté et toute leur espérance sera de rendre l'âme !
\Chap{12}
\TextTitle{Réplique de Job}
\VerseOne{}Job reprit la parole, et dit :
\VS{2}On dirait vraiment que vous êtes tout un peuple, et qu'avec vous doit mourir la sagesse.
\VS{3}J'ai du bon sens aussi bien que vous, et je ne vous suis point inférieur ; et qui ne sait de telles choses ?
\VS{4}Je suis pour mes amis un objet de raillerie, quand je m'écrie à Dieu pour qu'il me réponde; on se moque d'un homme qui est juste et droit.
\VS{5} Mépris au malheur! telle est la pensée des heureux; le mépris est réservé à ceux dont le pied chancelle !
\VS{6}Elles sont en paix, les tentes des pillards, et toutes les sécurités sont pour ceux qui irritent Dieu, qui se font un dieu de leur bras. \FTNT{ Jé. 12:1 ; Ps. 73:12.}.
\VS{7}Mais interroge donc les bêtes, et elles t'instruiront, ou les oiseaux des cieux, et ils te l'annonceront ;
\VS{8}Ou parle à la terre, et elle t'enseignera ; même les poissons de la mer te le raconteront ; 
\VS{9}Qui est-ce qui ne sait toutes ces choses; que c'est la main de Yahweh qui a fait cela ?
\VS{10} Qu'il tient en sa main, l'âme de tout ce qui vit, et l'esprit de toute chair humaine,
\VS{11}L'oreille ne discerne-t-elle pas les discours, ainsi que le palais savoure les aliments ?
\VS{12}La sagesse est dans les vieillards, et l'intelligence est le fruit d'une longue vie.
\VS{13}Mais en Dieu est la sagesse et la force ; à lui appartient le conseil et l'intelligence. \FTNT{Da. 2:20.}.
\VS{14}Voici, il démolit, et on ne rebâtit pas ; il enferme un homme, et on ne lui ouvre pas\FTNT{Es. 22:22 ; Ap. 3:7.}.
\VS{15}Voilà, il retient les eaux, et tout devient sec ; il les lâche, et elles bouleversent la terre.
\VS{16} En lui résident la puissance et la sagesse; de lui dépendent celui qui s'égare et celui qui égare.
\VS{17} Il emmène dépouillés les conseillers, et il met hors de sens les juges. \FTNT{2 S. 15:31 ; 2 S. 17:14-23 ; Es. 19:12 ; Es. 29:14 ; 1 Co. 1:19.}.
\VS{18}Il rend impuissant le gouvernement des rois, et lie de chaînes leurs reins. 
\VS{19}Il fait marcher pieds nus les sacrificateurs ; et il renverse les puissants.
\VS{20}Il ôte la parole à ceux qui sont les plus assurés en leurs discours, et il prive de sens les anciens.
\VS{21}Il verse le mépris sur les nobles ; il relâche la ceinture des forts\FTNT{Es. 40:23.}.
\VS{22}Il met en évidence les choses qui étaient cachées dans les ténèbres, et il produit en lumière l'ombre de la mort. \FTNT{Ps. 139:11-12 ; Ec. 12:16 ; Mt. 10:26 ; 1 Co. 4:6.}.
\VS{23} Il multiplie les nations, et les fait périr ; il répand çà et là les nations, et puis il les ramène. 
\VS{24}Il ôte la raison aux Chefs des peuples de la terre, et les fait errer dans les déserts où il n'y a point de chemin;
\VS{25}ils tâtonnent dans les ténèbres, sans aucune clarté, et il les fait chanceler comme des gens ivres. 
\Chap{13}
\VerseOne{}Voici, mon œil a vu toutes ces choses, mon oreille l'a entendu et compris.
\VS{2}Comme vous les savez, je les sais aussi ; je ne vous suis pas inférieur. 
\VS{3}Mais je veux parler au Tout-Puissant, je veux plaider auprès de Dieu. .
\VS{4}Et certes vous inventez des mensonges ; vous êtes tous des médecins inutiles.
\VS{5}Plaît à Dieu que vous demeuriez entièrement dans le silence ; et cela vous sera réputé à sagesse. \FTNT{Pr. 17:28.}.
\VS{6}Ecoutez donc maintenant ma cause, et soyez attentifs à la défense de mes lèvres.
\VS{7}Tiendrez-vous des discours injustes en faveur de Dieu, et, pour le défendre, direz-vous des mensonges?
\VS{8}Ferez-vous acception de personnes en sa faveur? Prétendrez-vous plaider pour Dieu? 
\VS{9}S'il vous sonde, vous trouvera-t-il bon ? Comme on trompe un homme, le tromperez-vous? 
\VS{10}Certainement il vous reprendra, si même en secret vous faites acception de personnes.
\VS{11}Sa majesté ne vous épouvantera-t-elle pas ? Et sa frayeur ne tombera-t-elle pas sur vous ? 
\VS{12}Vos discours mémorables sont des sentences de cendre, et vos éminences sont des éminences de boue. 
\VS{13}Taisez-vous devant moi, et que je parle ; et il m'arrivera ce qui pourra. 
\VS{14}Pourquoi porterais-je ma chair entre mes dents, et tiendrais-je mon âme entre mes mains ? \FTNT{Jg. 12:3 ; 1 S. 19:5.}.
\VS{15}Voilà, qu'il me tue, je ne cesserai pas d'espérer en lui ; et je défendrai ma conduite en sa présence.
\VS{16}Et qui plus est, il sera lui-même mon salut ; mais l'hypocrite ne viendra point devant sa face. \FTNT{Ps. 1:5.}.
\VS{17}Ecoutez attentivement mes paroles, et prêtez l'oreille à ce que je vais vous déclarer. 
\VS{18}Voici, j'ai préparé ma cause. Je sais que je serai justifié.
\VS{19}Qui est-ce qui veut disputer contre moi ? car maintenant si je me tais, je mourrai. 
\VS{20}Seulement ne me fais pas ces deux choses, et alors je ne me cacherai point devant ta face :
\VS{21}Retire ta main de dessus moi, et que tes terreurs ne me troublent pas.
\VS{22}Puis appelle-moi, et je répondrai ; ou bien je parlerai, et tu me répondras. 
\VS{23}Combien ai-je d'iniquités et de péchés ? Montre-moi mon crime et mon péché. 
\VS{24}Pourquoi caches-tu ta face, et me tiens-tu pour ton ennemi ?
\VS{25}Déploieras-tu tes forces contre une feuille que le vent emporte ? Poursuivras-tu du chaume tout sec \FTNT{1 S. 24:15.} ?
\VS{26}Que tu écrives contre moi des choses amères, et que tu me fasses porter la peine des péchés de ma jeunesse? \FTNT{Ps. 25:7.} ?
\VS{27}Que tu mettes mes pieds aux ceps, et observes tous mes chemins, et que tu suives les traces de mes pieds,
\VS{28}Quand mon corps s'en va par pièces comme du bois vermoulu, et comme une robe que la teigne a rongée? 
\Chap{14}
\VerseOne{}L'homme né de la femme est de courte vie, et rassasié d'agitations. \FTNT{Ps. 102:12 ; Ps. 103:15 ; Ps. 144:4 ; Ja. 4:14.}.
\VS{2}Il sort comme une fleur, puis il est coupé, et il s'enfuit comme une ombre qui ne s'arrête pas. \FTNT{Es. 40:6 ; Ps. 90:61 ; 1 Pi. 1:24.}.
\VS{3}Cependant tu as ouvert tes yeux sur lui, et tu me conduis en justice avec toi.
\VS{4}Qui est-ce qui tirera le pur de l'impur ? Personne. \FTNT{Es. 48:8 ; Pr. 22:15.}.
\VS{5}Les jours de l'homme sont déterminés, le nombre de ses mois est entre tes mains, tu lui as prescrit ses limites, et il ne passera point au delà.
\VS{6}Retire-toi de lui, afin qu'il ait du relâche, jusqu'à ce que comme un mercenaire il ait achevé sa journée.
\VS{7}Car si un arbre est coupé, il y a de l'espérance, et il poussera encore, et ne manquera pas de rejetons ; 
\VS{8}quoique sa racine ait vieilli dans la terre, et que son tronc soit mort dans la poussière;
\VS{9}Dès qu'il sent l'eau il regerme, et produit des branches, comme un arbre nouvellement planté. 
\VS{10}Mais l'homme meurt et perd toute sa force; il expire et puis où est-il ?
\VS{11}Les eaux s'écoulent de la mer, et une rivière s'assèche, et tarit ;
\VS{12} Ainsi l'homme est couché par terre, et ne se relève plus ; jusqu'à ce qu'il n'y ait plus de cieux, ils ne se réveillera plus, et ne sera pas réveillé de son sommeil. 
\VS{13}Oh que tu me caches dans le scheol, que tu me gardes à l'abri jusqu'à ce que ta colère soit passée, que tu me donnes un temps arrêté, après lequel tu te souviendrais de moi !
\VS{14}Si un homme meurt, revivra-t-il ? Tous les jours de ma détresse, j'attendrais jusqu'à ce que mon état vînt à changer.
\VS{15} Tu appellerais, et moi je te répondrais, tu ne dédaignerais pas l'ouvrage de tes mains.
\VS{16}Mais maintenant tu comptes mes pas, et tu n'exceptes rien de mon péché. \FTNT{Ps. 56:9 ; Ps. 139:2-4 ; Pr. 5:21.} ;
\VS{17}Mes péchés sont scellés dans un sac, et tu as cousu ensemble mes iniquités. \FTNT{Os. 13:12.}.
\VS{18}Car comme une montagne s'éboule en tombant, et comme un rocher est transporté de sa place ; 
\VS{19} et comme les eaux minent les pierres, et entraînent par leur débordement la poussière de la terre, avec tout ce qu'elle a produit, tu fais ainsi périr l'attente de l'homme. 
\VS{20}Tu te montres toujours plus fort que lui, et il s'en va, et lui ayant défiguré le visage, tu le renvoies.
\VS{21}Quand ses fils sont honorés, il n'en sait rien ; et quand ils sont abaissés, il ne s'en aperçoit pas.
\VS{22}Seulement sa chair sur lui, a de la douleur, et son âme en lui s'afflige. 
\Chap{15}
\TextTitle{Deuxième discours d'Eliphaz}
\VerseOne{}Eliphaz de Théman prit la parole et dit :
\VS{2}Un homme sage profère-t-il dans ses réponses une science aussi légère que le vent, des opinions vaines ? Remplit-il son ventre du vent d'orient ?
\VS{3}Contestant avec des discours qui ne servent de rien, et avec des paroles dont on ne peut tirer aucun profit ?
\VS{4}Certainement tu abolis la crainte de Dieu, et tu anéantis peu à peu la prière qu'on doit présenter à Dieu. 
\VS{5} Car ta bouche fait connaître ton iniquité, et tu as choisi un langage trompeur. 
\VS{6}C'est ta bouche qui te condamne, et non pas moi ; et tes lèvres témoignent contre toi. 
\VS{7}Es-tu le premier homme né ? Ou as-tu été formé avant les montagnes ? \FTNT{Ps. 90:2 ; Pr. 8:25.} ?
\VS{8}As-tu été instruit dans le conseil secret de Dieu, et renfermes-tu seul la sagesse ? \FTNT{Es. 40:13 ; Jé. 23:18 ; Ro. 11:34.} ?
\VS{9}Que sais-tu que nous ne sachions pas ? Quelle connaissance as-tu que nous n'ayons pas ?
\VS{10}Parmi nous, il y a des hommes à cheveux blancs, et des gens d'une fort grande vieillesse, il y en a même de plus âgés que ton père. 
\VS{11}Les consolations du Dieu te semblent-elles trop petites ? As-tu quelque chose de caché par-devant toi ? …
\VS{12}Pourquoi ton coeur s'emporte-il et pourquoi tes yeux clignent-ils ?
\VS{13}C'est contre Dieu que tu tournes ta colère, et que tu fais sortir de ta bouche de tels discours! 
\VS{14}Qu'est-ce que de l'homme, pour qu'il soit pur, et celui qui est né de femme, pour qu'il soit juste ? \FTNT{Ps. 14:3 ; Pr. 20:9 ; Ec. 7:20.} ?
\VS{15}Si Voici, Dieu ne se fie pas à ses saints, et les cieux ne sont pas purs à ses yeux,
\VS{16}Combien plus est abominable et corrompu, l'homme qui boit l'iniquité comme l'eau !  
\VS{17}Je t'enseignerai, écoute-moi, et je te raconterai ce que j'ai vu ,
\VS{18} savoir ce que les sages ont déclaré, et qu'ils n'ont point caché ; ce qu'ils avaient [reçu] de leurs pères.
\VS{19}Eux à qui seuls la terre a été donnée, et parmi lesquels l'étranger n'est point passé.
\VS{20}Toute sa vie, le méchant est tourmenté, et un petit nombre d'années sont réservées au malfaiteur.\FTNT{Es. 48:22 ; Es. 57:21.}.
\VS{21}Un cri de frayeur est dans ses oreilles ; au milieu de la paix [il croit] que le destructeur se jette sur lui. \FTNT{1 Th. 5:3.} ;
\VS{22}Il ne croit pas pouvoir sortir des ténèbres, car il voit la menace de   l’épée;
\VS{23}il court çà et là pour chercher son pain, il sait que le jour des ténèbres lui est préparé \FTNT{Ps. 109:10.}.
\VS{24}La détresse et l'angoisse l'épouvantent, elles l'assaillent comme un roi prêt à combattre ;
\VS{25}Parce qu'il a élevé sa main contre Dieu, et qu'il s'est levé contre le Tout-puissant ;
\VS{26}Il lui a sauté au collet, et sur l'épaisseur de ses gros boucliers. 
\VS{27}Parce que la graisse a couvert son visage, et qu'elle a fait des replis sur son ventre;
\VS{28} il habite des villes détruites, des maisons désertes, tout près de n'être plus que des monceaux de pierres. 
\VS{29}Et il ne s'enrichira plus, car ses biens ne subsisteront pas, et ses richesses ne se répandront pas sur la terre. 
\VS{30}Il ne pourra pas se détourner des ténèbres, la flamme desséchera ses rejetons, et Dieu le fera disparaître par le souffle de sa bouche.
\VS{31}S'il a confiance dans la vanité, il se trompe, car la vanité sera sa récompense.
\VS{32}Ce sera fait de lui avant son temps, ses branches ne reverdiront plus. 
\VS{33}On arrachera ses fruits non mûrs, comme à une vigne; on jettera sa fleur, comme celle d'un olivier. 
\VS{34}Car la famille des hypocrites est stérile, et le feu dévore les tentes de l'homme corrompu.
\VS{35}Ils conçoivent le travail, et ils enfantent la misère, et machinent dans le cœur des fraudes. \FTNT{Es. 59:4 ; Os. 10:13.}.
\Chap{16}
\TextTitle{Réponse de Job}
\VerseOne{}Job répondit, et dit :
\VS{2}J'ai souvent entendu de pareils discours ; vous êtes tous des consolateurs fâcheux.
\VS{3}Y aura-t-il une fin à [ces] paroles de vent ? Qu'est-ce qui t'irrite, que tu répondes ?
\VS{4}Parlerais-je comme vous faites, si vous étiez en ma place ; accumulerais-je des paroles contre vous, ou secouerais-je ma tête contre vous ? 
\VS{5}Je vous fortifierais de ma bouche, et le mouvement de mes lèvres vous soulagerait.
\VS{6}Si je parle, ma douleur ne sera point soulagée. Si je me tais, en sera-t-elle diminuée?
\VS{7}Maintenant il m'a épuisé... Tu as dévasté toute ma famille, ;
\VS{8}Tu m'as tout couvert de rides, qui sont un témoignage des maux que je souffre ; et il s'est élevé en moi une maigreur qui en rend aussi témoignage sur mon visage. 
\VS{9}Sa fureur me déchire, il se déclare mon ennemi, il grince des dents contre moi, et étant devenu mon ennemi, il étincelle des yeux contre moi.
\VS{10}Ils ouvrent contre moi leur bouche; ils me frappent à la joue pour m'outrager; ils se réunissent tous ensemble contre moi. 
\VS{11}Dieu m'a livré à l'impie, et m'a jeté entre les mains des méchants. 
\VS{12}J'étais tranquille, et il m'a secoué, il m'a saisi par la nuque et m'a brisé, il m'a posé en butte à ses traits.
\VS{13}Ses archers m'ont environné, il me perce les reins, et ne m'épargne pas ; il répand mon fiel par terre. 
\VS{14}Il m'a brisé en me faisant plaie sur plaie, il a couru sur moi comme un homme fort.
\VS{15}J'ai cousu un sac sur ma peau, j'ai souillé ma tête dans la poussière\FTNT{Ps. 44 : 25 ; Ps. 119 : 25.},
\VS{16}J'ai le visage tout enflammé, à force de pleurer, et l'ombre de la mort est sur mes paupières, 
\VS{17}Quoiqu'il n'y ait point de violence dans mes mains, et que ma prière fut toujours pure.
\VS{18}Ô terre, ne cache pas mon sang, et qu'il n'y ait aucun lieu où s'arrête mon cri !
\VS{19}Mais maintenant voilà, mon témoin est aux cieux, mon témoin est dans les lieux élevés. \FTNT{Ap. 1:5 ; Ap. 3:14.}.
\VS{20}Mes amis se moquent de moi: c'est vers Dieu que mon œil se tourne en pleurant,
\VS{21}pour qu'il fasse justice entre l'homme et Dieu, entre le fils d'Adam et son semblable.
\VS{22}Car les années de mon compte arrivent à leur terme, et j'entre dans un sentier d'où je ne reviendrai plus. 
\Chap{17}
\VerseOne{}Mon souffle se perd, mes jours s'éteignent, le sépulcre m'attend.
\VS{2}Je suis environné de moqueurs, et mon œil veille toute la nuit au milieu de leurs insultes.
\VS{3}Dépose un gage, sois ma caution auprès de toi-même; car qui voudrait répondre pour moi? 
\VS{4}C'est pourquoi tu ne les élèveras pas\FTNT{De. 29:4 ; Mt. 11:25.}.
\VS{5}Celui qui trahit ses amis pour qu'ils soient pillés, les yeux de ses fils se consument.
\VS{6}On a fait de moi la fable des peuples, un être à qui l'on crache au visage.
\VS{7}Mon œil est obscurci par le chagrin, tous mes membres sont comme une ombre\FTNT{Ps. 6:7 ; Ps. 31:10.}.
\VS{8}Les hommes droits en sont consternés, et l'innocent est irrité contre l'impie. 
\VS{9}Toutefois le juste se tient ferme dans sa voie, et celui qui a les mains pures, se renforce.
\VS{10}Retournez donc vous tous, et revenez, je vous prie ; car je ne trouve pas de sages parmi vous. 
\VS{11}Mes jours sont passés; mes desseins, chers à mon cœur, sont renversés.…
\VS{12}On me change la nuit en jour, et on fait que la lumière se trouve proche des ténèbres !
\VS{13}Certes je n'ai plus à attendre que le sépulcre, qui va être ma maison ; j'ai dressé mon lit dans les ténèbres ;
\VS{14}J'ai crié à la fosse : tu es mon père ; et aux vers : vous êtes ma mère et ma Soeur. 
\VS{15}Où est donc mon espérance? Et mon espérance, qui pourrait la voir? \VS{16}Elle descendra au fond du sépulcre ; certes elle reposera avec moi dans la poussière. 
\Chap{18}
\TextTitle{Deuxième discours de Bildad}
\VerseOne{}Bildad de Schuach prit la parole et dit :
\VS{2}Quand finirez-vous ces discours ? Ecoutez, et puis nous parlerons.
\VS{3}Pourquoi sommes-nous regardés comme des bêtes, et sommes-nous stupides à vos yeux?
\VS{4}Ô toi qui déchires ton âme dans ta colère, la terre sera-t-elle abandonnée à cause de toi, et le rocher sera-t-il transporté de sa place ? 
\VS{5}Certainement, la lumière du méchant s'éteindra, et la flamme de son feu ne brillera pas\FTNT{Ps. 37:9-10.}.
\VS{6}La lumière sera ténèbres dans sa tente, et sa lampe sera éteinte au-dessus de lui. 
\VS{7}Les pas de sa force seront resserrés, et son propre conseil le renversera.
\VS{8}Car il est poussé dans le filet par ses propres pieds ; et il marche sur les mailles du filet.
\VS{9}Le piège le prend par le talon, et le filet s'empare de lui;
\VS{10}la corde est cachée dans la terre, et la trappe est sur son sentier.
\VS{11}Les terreurs l'assiègent de tous côtés, et le font courir ses pieds çà et là.\FTNT{Jé. 6:25 ; Jé. 46:5 ; Jé. 49:29.}.
\VS{12}Sa vigueur sera affamée, la détresse est à ses côtés.
\VS{13}Il dévorera les membres de son corps, il dévorera ses membres, le premier-né de la mort ! 
\VS{14} Les choses en quoi il mettait sa confiance seront arrachées de sa tente, et il sera conduit vers le Roi des épouvantements. 
\VS{15}On habitera dans sa tente, qui ne sera plus à lui; le soufre sera répandu sur sa demeure. 
\VS{16}Ses racines sèchent au dessous, et ses branches sont coupées en haut. 
\VS{17}Sa mémoire périt sur la terre, et on ne parle plus de son nom dans les places \FTNT{Ps. 109:13 ; Pr. 10:7.}.
\VS{18}Il est chassé de la lumière dans les ténèbres, et il est exterminé du monde. 
\VS{19}Il n'a ni lignée, ni descendance au milieu de son peuple, ni survivant dans ses habitations. \FTNT{Es. 14:20-22 ; Jé. 22:30 ; Ps. 37:28 ; Ps. 109:13.}. 
\VS{20}Ceux qui seront venus après lui, seront étonnés de sa ruine ; et ceux qui auront été avant lui en seront saisis d'horreur. 
\VS{21}Tel est le sort de l'injuste. Telle est la destinée de celui qui ne connaît pas Dieu. 
\Chap{19}
\TextTitle{Réponse de Job}
\VerseOne{}Job prit la parole et dit :
\VS{2}Jusqu'à quand affligerez-vous mon âme, et m'accablerez-vous de paroles ?
\VS{3} Voilà déjà dix fois que vous m'outragez: vous n'avez pas honte de me maltraiter? 
\VS{4}Vraiment si j'ai failli, ma faute demeure avec moi. 
\VS{5}Si réellement vous voulez vous élever contre moi et faire valoir mon opprobre contre moi, 
\VS{6}Sachez donc que c'est Dieu qui me renverse, et qui tend son filet autour de moi.
\VS{7}Voici je crie pour la violence qui m'est faite, et je ne suis pas exaucé ; je m'écrie, et il n'y a point de justice !
\VS{8}Il a fermé mon chemin, et je ne puis passer; il a mis des ténèbres sur mes sentiers. 
\VS{9}Il m'a dépouillé de ma gloire, il a ôté la couronne de ma tête.
\VS{10}Il m'a détruit de tous côtés, et je m'en vais ; il a arraché mon espérance comme un arbre.
\VS{11}Il s'est enflammé de colère contre moi, et m'a traité comme un de ses ennemis\FTNT{La. 2:5.}.
\VS{12}Ses troupes sont venues ensemble, et elles ont dressé leur chemin contre moi, et se sont campées autour de ma tente\FTNT{La. 2:22.}.
\VS{13}Il a éloigné de moi mes frères, et ceux qui me connaissaient se sont écartés comme des étrangers\FTNT{Ps. 88:9.} ;
\VS{14}mes proches m'ont abandonné, et ceux que je connaissais m'ont oublié.
\VS{15}Ceux qui séjournent dans ma maison et mes servantes m'ont traité comme un étranger; je suis devenu un inconnu pour eux. 
\VS{16}J'appelle mon serviteur, il ne me répond; de ma propre bouche, je le supplie en vain. 
\VS{17}Mon haleine est devenue dégoûtante à ma femme, et ma plainte aux fils de mes entrailles.
\VS{18}Je suis méprisé même par des enfants ; si je me lève, ils parlent contre moi.
\VS{19}Ceux que j'avais pour confidents m'ont en horreur, ceux que j'aimais se sont tournés contre moi\FTNT{Ps. 55:13-14.}.
\VS{20}Mes os sont attachés à ma peau et à ma chair ; et je me suis échappé avec la peau de mes dents\FTNT{La. 4:8.}.
\VS{21}Ayez pitié, ayez pitié de moi, vous, mes amis ! Car la main de Dieu m'a frappé.
\VS{22}Pourquoi, comme Dieu, me poursuivez-vous et n'êtes-vous pas rassasiés de ma chair \FTNT{Ps. 27:2.} ?
\VS{23}Oh! je voudrais que mes paroles fussent écrites quelque part, je voudrais qu'elles fussent inscrites dans un livre; 
\VS{24}Qu'avec un burin de fer et avec du plomb, elles fussent gravées sur le roc, pour toujours... 
\VS{25}Mais je sais que mon rédempteur est vivant, il demeurera le dernier sur la terre.
\VS{26}Et après que cette peau aura été détruite, hors de ma chair, je verrai Dieu \FTNT{Ps. 17:15.}.
\VS{27}Je le verrai moi-même, et mes yeux le verront, et non un autre. Mes reins se consument dans mon sein. 
\VS{28}Vous direz : Comment le poursuivrons-nous, et trouverons-nous en lui la cause de son malheur? 
\VS{29}Ayez peur de l'épée ; car la fureur [avec laquelle vous me persécutez], est [du nombre] des iniquités qui attirent l'épée ; c'est pourquoi sachez qu'il y a un jugement. 
\Chap{20}
\TextTitle{Dernier discours de Tsophar}
\VerseOne{}Tsophar de Naama prit la parole et dit :
\VS{2}C'est à cause de cela que mes pensées diverses me poussent à répondre, et que cette promptitude est en moi. 
\VS{3}J'ai entendu la correction dont tu veux me faire honte, mais mon esprit tirera de mon intelligence la réponse pour moi. 
\VS{4}Ne sais-tu pas que, de tout temps, depuis que Dieu a mis l'homme sur la terre, 
\VS{5}Le triomphe des méchants est de peu de durée, et la joie de l'hypocrite n'est que pour un moment \FTNT{Ps. 37:35-36.} ?
\VS{6}Quand son élévation monterait jusqu'aux cieux, et que sa tête atteindrait les nues,
\VS{7}il périra pour toujours comme ses ordures, et ceux qui le voyaient diront : Où est-il ?
\VS{8}Il s'envolera comme un songe, et on ne le trouvera plus ; il se  retirera comme une vision nocturne\FTNT{Ps. 73:19-20.} ;
\VS{9}l'œil qui le regardait ne le regardera plus, le lieu qu'il habitait ne le contemplera plus.
\VS{10}Ses fils rechercheront la faveur des pauvres, et ses mains restitueront ce que sa violence a ravi\FTNT{Ps. 109:10.}.
\VS{11}Ses os seront pleins de la punition, à  cause des péchés de sa jeunesse, et elle reposera avec lui dans la poussière.
\VS{12}Le mal était doux à sa bouche, il le cachait sous sa langue,
\VS{13}s'il l'épargne, et ne le rejette point, mais le retient dans son palais ; 
\VS{14}Ce qu'il mangera se changera dans ses entrailles en un fiel d'aspic.
\VS{15}Il a englouti des richesses, il les vomira ; Dieu les arrachera de son ventre.
\VS{16}Il a sucé du venin d'aspic, la langue de la vipère le tuera.
\VS{17}Il ne verra plus les ruisseaux, les fleuves, les torrents de miel et de lait.
\VS{18}Il rendra le fruit de son travail, et ne l'avalera pas; il restituera à proportion de ce qu'il aura amassé, et ne s'en réjouira pas\FTNT{So. 2:10.}.
\VS{19}Car il a opprimé, délaissé les pauvres; il a pillé des maisons et ne les a pas rebâties.
\VS{20}Certainement il ne sentira pas dans son ventre la satisfaction de son avidité, et il ne sauvera rien de ce qu'il aura tant convoité \FTNT{Ec. 5:12.}.
\VS{21}Rien n'échappait à sa voracité, mais son bonheur ne durera pas.
\VS{22}Après que la mesure de ses biens aura été remplie, il sera dans la misère ; toutes les mains de ceux qu'il aura opprimés se jetteront sur lui.
\VS{23}Il arrivera que pour lui remplir le ventre, Dieu enverra contre lui l'ardeur de sa colère; il la fera pleuvoir sur lui et entrer dans sa chair.
\VS{24}S’il s’enfuit de devant les armes de fer, l’arc d’airain le transpercera.
\VS{25}Il arrachera la flèche, et elle sortira de son corps, et le fer étincelant, de son foie; les frayeurs de la mort viendront sur lui.
\VS{26}Toutes les ténèbres sont renfermées dans ses demeures les plus secrètes ; un feu qu'on n'aura point soufflé, le consumera ; l'homme qui restera dans sa tente sera malheureux\FTNT{Ps. 12:6.}.
\VS{27}Les cieux découvriront son iniquité, et la terre s'élèvera contre lui. 
\VS{28}Le revenu de sa maison sera emporté. Tout s'écoulera au jour de la colère de Dieu.
\VS{29}C'est là la portion que Dieu réserve à l'homme méchant, et l'héritage qu'il aura de Dieu pour ses discours.
\Chap{21}
\TextTitle{Réponse de Job}
\VerseOne{}Job répondit, et dit :
\VS{2}Ecoutez, écoutez mes discours, donnez-moi seulement cette consolation.
\VS{3}Supportez-moi, et je parlerai ; et quand j'aurai parlé, tu pourras te moquer.
\VS{4}Mais est-ce contre un homme que s'adresse ma plainte ? Et pourquoi mon âme ne serait-elle pas impatiente ?
\VS{5}Regardez-moi, soyez étonnés, et mettez la main sur la bouche.
\VS{6}Quand j'y pense, cela m'épouvante, et un frisson saisit mon corps.
\VS{7}Pourquoi les méchants vivent-ils, vieillissent-ils, et croissent-ils en puissance\FTNT{Jé. 12:1 ; Ha. 1:3 ; Mal. 3:14-15.}?
\VS{8}Leur postérité s'établit avec eux et en leur présence, leurs rejetons prospèrent sous leurs yeux.
\VS{9}Dans leurs maisons règne la paix, loin de la crainte ; la verge de Dieu ne vient pas les frapper.
\VS{10}Leurs taureaux sont féconds, leurs génisses conçoivent et n'avortent pas\FTNT{Ps. 144:13-14.}.
\VS{11}Ils laissent courir leurs enfants comme un troupeau, et les enfants prennent leurs ébats.
\VS{12}Ils chantent au son du tambourin et de la harpe, ils se réjouissent au son du chalumeau.
\VS{13}Ils passent leurs jours dans le bonheur, et ils descendent en un instant au scheol.
\VS{14}Ils disaient pourtant à Dieu : Éloigne-toi de nous, nous ne voulons pas connaître tes voies.
\VS{15}Qu'est-ce que le Tout-Puissant pour que nous le servions ? Que gagnerions-nous à lui adresser nos prières\FTNT{Ex. 5:2.} ?
\VS{16}Quoi donc ! Ne sont-ils pas en possession du bonheur entre leurs mains ? Loin de moi le conseil des méchants\FTNT{Ps. 1:1-2.} !
\VS{17}Mais arrive-t-il que la lampe des méchants s'éteigne, que la ruine vienne sur eux, que Dieu leur distribue leur part dans sa colère\FTNT{Ps. 11:5-6 ; Pr. 13:9.},
\VS{18}qu'ils soient comme la paille face au vent, comme la balle enlevée par le tourbillon\FTNT{Ps. 1:4.}?
\VS{19}Dieu réservera-t-il aux enfants du méchant la punition de ses violences? Il la leur rendra, et il le connaîtra!
\VS{20}Il verra de ses propres yeux sa ruine, c'est lui qui devrait boire la colère du Tout-Puissant\FTNT{Es. 51:17-22 ; Jé. 25:15 ; Ez. 23:31-32 ; Ap. 14:10.}.
\VS{21}Car que lui importe sa maison après lui, quand le nombre de ses mois est achevé ?
\VS{22}Enseignerait-on la science à Dieu, lui qui juge les esprits élevés\FTNT{Ro. 11:34 ; 1 Co. 2:16.} ?
\VS{23}L'un meurt au sein du bien-être, tout à son aise et en joie,
\VS{24}les flancs chargés de graisse, et ses os comme abreuvés de mœlle ;
\VS{25}l'autre meurt l'amertume dans l'âme, n'ayant jamais mangé ce qui est bon.
\VS{26}Et tous deux se couchent dans la poussière, tous deux deviennent couverts de vers.
\VS{27}Je sais bien quelles sont vos pensées, quels jugements iniques vous portez sur moi.
\VS{28}Vous dites : Où est la maison de l'homme puissant ? Où est la tente, demeure des méchants ?
\VS{29}Mais quoi ! N'avez-vous pas interrogé les voyageurs, et n'avez-vous pas appris par les rapports qu'ils vous on faits ?
\VS{30}Au jour du malheur, le méchant est épargné ; au jour de la colère, il échappe\FTNT{Pr. 16:4 ; Ec. 9:12.}.
\VS{31}Qui lui dit en face sa conduite ? Qui lui rend ce qu'il a fait ?
\VS{32}Il est porté au tombeau, et il veille encore sur sa tombe.
\VS{33}Les mottes de la vallée lui sont légères ; et tous après lui suivront la même voie, comme une multitude l'a déjà suivie.
\VS{34}Pourquoi donc m'offrir de vaines consolations ? Ce qui reste de vos réponses n'est que transgression.
\Chap{22}
\TextTitle{Dernier discours d'Eliphaz}
\VerseOne{}Eliphaz de Théman prit la parole et dit :
\VS{2}Un homme peut-il être utile à Dieu ? Mais le sage n'est utile qu'à lui-même.
\VS{3}Si tu es juste, est-ce à l'avantage du Tout-Puissant ? Si tu es intègre dans tes voies, qu'y gagne-t-il ?
\VS{4}Est-ce par crainte de toi qu'il te reprend, qu'il entre en jugement avec toi?
\VS{5}Ta méchanceté n'est-elle pas grande ? Tes iniquités ne sont-elles pas sans fin ?
\VS{6}Tu a pris sans raison le gage de tes frères, tu privais de leurs vêtements ceux qui étaient nus\FTNT{Ex. 22:21.} ;
\VS{7}tu ne donnais pas d'eau à boire à l'homme altéré, tu refusais du pain à l'homme affamé.
\VS{8}Le pays était à l'homme le plus fort, et le puissant s'y établissait.
\VS{9}Tu renvoyais les veuves à vide, les bras des orphelins étaient brisés.
\VS{10}C'est pour cela que tu es entouré de pièges, et que la terreur t'a saisi tout à coup.
\VS{11}Ne vois-tu donc pas ces ténèbres, ces eaux débordées qui te couvrent ?
\VS{12}Dieu n'est-il pas là-haut dans les cieux ? Regarde le sommet des étoiles, comme il est élevé !
\VS{13}Et tu dis : Qu'est-ce que Dieu connaît ? Peut-il juger à travers l'obscurité\FTNT{So. 1:12 ; Ps. 10:11-13 ; Ps. 94:7.} ?
\VS{14}Les nuées l'enveloppent, et il ne voit rien ; il ne parcourt que la voûte des cieux.
\VS{15}Eh quoi ! N'as-tu pas pris garde à l'ancienne route qu'ont suivie les hommes d'iniquité ?
\VS{16}Ils ont été emportés avant le temps, ils ont eu la durée d'un torrent qui s'écoule.
\VS{17}Ils disaient à Dieu : Éloigne-toi de nous ; que peut faire pour nous le Tout-Puissant ?
\VS{18}Dieu cependant avait rempli leurs maisons de biens ! Loin de moi le conseil des méchants !
\VS{19}Les justes le verront, se réjouiront, et l'innocent se moquera d'eux\FTNT{Ps. 107:42.} :
\VS{20}Certainement, notre adversaire a été détruit, le feu a dévoré ce qui en restait\FTNT{Ps. 37:20 ; Ec. 8:12-13.} !
\VS{21}Attache-toi donc à Dieu, et tu auras la paix, tu atteindras ainsi le bonheur.
\VS{22}Reçois de sa bouche l'instruction, et mets ses paroles dans ton cœur\FTNT{Ps. 119:72.}.
\VS{23}Si tu reviens au Tout-Puissant, tu seras rétabli ; si tu éloignes l'iniquité de ta tente.
\VS{24}Jette l'or dans la poussière, l'or d'Ophir parmi les rochers des torrents ;
\VS{25}et le Tout-Puissant sera ton or, ton argent, ta richesse.
\VS{26}Alors tu feras du Tout-Puissant tes délices, tu élèveras vers Dieu ta face ;
\VS{27}tu le prieras, et il t'exaucera, et tu lui rendras tes vœux\FTNT{Ps. 50:14-15.}.
\VS{28}Quand tu prendras des résolutions elles s'accompliront, sur tes sentiers brillera la lumière\FTNT{Ps. 97:11.}.
\VS{29}Quand on aura abaissé quelqu'un et que tu auras dit qu'il soit élevé; alors Dieu délivrera celui qui tenait les yeux abaissés\FTNT{Pr. 29:23.}.
\VS{30}Il délivrera le coupable ; il sera délivré par la pureté de tes mains.
\Chap{23}
\TextTitle{Réponse de Job}
\VerseOne{}Job répondit, et dit :
\VS{2}Maintenant encore ma plainte est une révolte, et pourtant ma main appesantit mes soupirs.
\VS{3}Oh ! Si je savais où le trouver, j'irais jusqu'à son trône,
\VS{4}je disposerais en ordre ma cause devant lui, je remplirais ma bouche d'arguments,
\VS{5}je saurais ce qu'il peut avoir à répondre, je comprendrais ce qu'il peut avoir à me dire.
\VS{6}Contesterait-il avec moi dans la grandeur de sa force? Ne prendrait-il pas le temps de m'écouter ?
\VS{7}Ce serait un homme juste qui argumenterait avec lui, et je serais pour toujours absous par mon juge.
\VS{8}Mais, si je vais à l'orient, il n'y est pas ; si je vais à l'occident, je ne l'aperçois pas ;
\VS{9}est-il occupé au nord, je ne le vois pas ; se cache-t-il au midi, je ne l'aperçois pas.
\VS{10}Il connaît la voie que j'ai suivie ; et s'il m'éprouvait, j'en sortirai pur comme l'or\FTNT{1 Pi. 1:7.}.
\VS{11}Mon pied s'est attaché à ses pas ; j'ai gardé sa voie, et je ne m'en suis pas détourné.
\VS{12}Je n'ai pas abandonné les commandements de ses lèvres ; j'ai fait plier ma volonté aux paroles de sa bouche.
\VS{13}Mais il n'a qu'une pensée ; qui l'en fera revenir ? Ce que son âme désire, il le fait\FTNT{Ps. 115:3 ; Ps. 135:6.}.
\VS{14}Il achèvera donc ses desseins à mon égard, et il en concevra beaucoup d'autres encore.
\VS{15}C'est pourquoi je suis terrifié à cause de sa présence, et quand je le considère, je suis effrayé devant lui.
\VS{16}Dieu a brisé mon cœur, le Tout-Puissant m'a épouvanté.
\VS{17}Car ce n'est pas la présence des ténèbres qui m'anéantit, ce n'est pas l'obscurité dont ma face est couverte.
\Chap{24}
\VerseOne{}Pourquoi le Tout-Puissant ne met-il pas des temps en réserve, et pourquoi ceux qui le connaissent ne voient-ils pas ses jours ?
\VS{2}On déplace les bornes, on ravit des troupeaux, et on les fait paître\FTNT{De. 19:14 ; De. 27:17 ; Pr. 13:10 ; Pr. 22:28.} ;
\VS{3}on emmène l'âne de l'orphelin, on prend pour gage le bœuf de la veuve ;
\VS{4}on fait écarter les pauvres du chemin, on force tous les affligés du pays à se cacher.
\VS{5}Et voici, comme les ânes sauvages du désert, ils sortent le matin pour chercher de la nourriture, ils n'ont que le désert pour trouver le pain de leurs enfants ;
\VS{6}ils moissonnent le fourrage qui reste dans les champs, ils grappillent dans la vigne de l'impie ;
\VS{7}ils passent la nuit nus, sans vêtements, sans couverture contre le froid\FTNT{Lé. 19:13 ; De. 24:12-13.} ;
\VS{8}ils sont percés par la pluie des montagnes, et ils embrassent les rochers comme unique refuge.
\VS{9}On arrache l'orphelin à la mamelle, on prend des gages sur le pauvre.
\VS{10}Ils font aller sans habits l'homme qu'ils ont dépouillé; et ils enlèvent à ceux qui n'avaient pas de quoi manger, ce qu'ils avaient glâné.
\VS{11}Dans les enclos de l'impie, ils font de l'huile, ils foulent le pressoir à raisin et ils ont soif.
\VS{12}Ils font gémir les gens dans la ville, l'âme de ceux qu'ils ont fait mourir, crient; Dieu ne fait rien d'indigne de lui.
\VS{13}En voici d'autres qui se révoltent contre la lumière, ils n'en connaissent pas les voies, ils ne restent pas sur leurs sentiers.
\VS{14}Le meurtrier se lève au point du jour ; il tue le pauvre et l'indigent, et il dérobe pendant la nuit\FTNT{Ps. 10:8-9.}.
\VS{15}L'œil de l'adultère épie le crépuscule ; aucun œil ne me verra, dit-il, et il met un voile sur le visage\FTNT{Ps. 64:6 ; Pr. 7:7-10.}.
\VS{16}Ils percent durant les ténèbres les maisons, qu'ils avaient marquées le jour, ils haïssent la lumière.\FTNT{Jn. 3:20.}.
\VS{17}Pour eux, le matin c'est l'ombre de la mort ; si quelqu'un les reconnaît, ils ont des terreurs.
\VS{18}Eh quoi ! L'impie est d'un poids léger sur la surface de l'eau, il n'a sur la terre qu'un héritage maudit, il ne prend jamais le chemin des vignes !
\VS{19}Comme la sécheresse et la chaleur absorbent les eaux de la neige, ainsi le scheol engloutit ceux qui pèchent\FTNT{Ps. 49:15.} !
\VS{20}Quoi ! Le sein maternel l'oublie, les vers en font leurs délices, on ne se souvient plus de lui ! L'injuste est brisé comme du bois,
\VS{21}lui qui dépouille la femme stérile et sans enfants, lui qui ne répand aucun bien sur la veuve !…
\VS{22}Non ! Dieu par sa force prolonge les jours des violents, et les voilà s'élever quand ils ne croyaient plus en la vie.
\VS{23}Il leur donne de la sécurité et de la confiance, ses yeux sont sur leurs voies.
\VS{24}Ils se sont élevés ; et en un peu de temps ils ne sont plus, ils s'affaissent, ils meurent en chemin comme tous les hommes, ils sont coupés comme une tête d'épi.
\VS{25}S'il n'en est pas ainsi, qui me fera mentir, qui fera de mes paroles un rien ?
\Chap{25}
\TextTitle{Dernier discours de Bildad}
\VerseOne{}Bildad de Schuach prit la parole et dit :
\VS{2}La domination et la terreur appartiennent à Dieu ; il fait régner la paix dans ses hautes régions.
\VS{3}Ses armées peuvent-elles se compter ? Sur qui sa lumière ne se lève-t-elle pas\FTNT{Mt. 5:45.} ?
\VS{4}Comment l'homme serait-il juste devant Dieu ? Comment celui qui est né de la femme serait-il pur ?
\VS{5}Voici, la lune même n'est pas brillante, et les étoiles ne sont pas pures à ses yeux ;
\VS{6}combien moins l'homme qui n'est qu'un ver, le fils de l'homme qui n'est qu'un vermisseau\FTNT{Ps. 22:7.} !
\Chap{26}
\TextTitle{Réponse de Job}
\VerseOne{}Job répondit, et dit :
\VS{2}Comme tu as aidé celui qui était sans force ! Comme tu as secouru le bras sans force !
\VS{3}Quels bons conseils tu donnes à celui qui manque de sagesse ! Tu fais connaître l'abondance de ton intelligence !
\VS{4}A qui s'adressent tes paroles ? Et de qui est l'esprit qui est sorti de toi ?
\VS{5}Devant Dieu les ombres des morts tremblent au-dessous des eaux, et de leurs habitants ;
\VS{6}devant lui le scheol est nu, l'abîme est sans voile\FTNT{Ps. 139:8-12 ; Pr. 15:11 ; Hé. 4:13.}.
\VS{7}Il étend la direction nord sur le vide, il suspend la terre sur le néant.
\VS{8}Il renferme les eaux dans ses nuages, et la nuée n'éclate pas sous leur poids\FTNT{Ps. 104:2-3.}.
\VS{9}Il couvre la face de son trône, il répand sur lui sa nuée.
\VS{10}Il a tracé un cercle à la surface des eaux, comme limite entre la lumière et les ténèbres\FTNT{Ge. 1:9 ; Jé. 5:22 ; Ps. 33:7 ; Ps. 104:9 ; Pr. 8:29.}.
\VS{11}Les colonnes du ciel s'ébranlent et s'étonnent à sa menace.
\VS{12}Par sa force il soulève la mer, par son intelligence il en brise l'orgueil\FTNT{Ps. 89:10.}.
\VS{13}Il a orné les cieux par son Esprit, et  de sa main, il transperce le serpent fuyard.
\VS{14}Ce sont là les bords de ses voies, c'est le discours fait en chuchotant que nous entendons ; mais qui comprendra le tonnerre de sa puissance\FTNT{Ec. 3:10.} ?
\Chap{27}
\VerseOne{}Et Job continuant, reprit son discours sentencieux, et dit :
\VS{2}Dieu, qui met mon droit à l'écart, et le Tout-puissant qui remplit mon âme d'amertume, est vivant.
\VS{3}Aussi longtemps que j'aurai ma respiration et que l'esprit de Dieu sera dans mes narines,
\VS{4}mes lèvres ne prononceront rien d'injuste, et ma langue ne dira pas de chose fausse\FTNT{Es. 33:15 ; Ps. 15:2 ; Ps. 24:4.}.
\VS{5}Loin de moi la pensée de vous reconnaître pour justes ! Tant que je vivrai je n'abandonnerai pas mon intégrité.
\VS{6}Je conserve ma justice, et je ne l'abandonne pas ; et mon cœur ne me reproche rien en mes jours.
\VS{7}Qu'il en soit de mon ennemi comme du méchant ; et de celui qui se lève contre moi, comme de l'injuste !
\VS{8}Quelle espérance reste-t-il à l'hypocrite quand Dieu coupe le fil de sa vie, quand il lui retire son âme\FTNT{Mt. 16:26 ; Lu. 12:20.} ?
\VS{9}Est-ce que Dieu entend ses cris, quand l'angoisse vient sur lui\FTNT{ Es. 1:15 ; Jé. 14:12 ; Ez. 8:18 ; Mi. 3:4 ; Ps. 18:41 ; Pr. 1:28 ; Jn. 9:31 ; Ja. 4:3.} ?
\VS{10}Trouvera-t-il son plaisir dans le Tout-Puissant ? Invoque-t-il Dieu en tout temps ?
\VS{11}Je vous enseignerai comment la main de Dieu agit, je ne vous cacherai pas les desseins du Tout-Puissant.
\VS{12}Voilà, vous avez tous vu ces choses, et pourquoi vous laissez-vous aller à des pensées vaines ?
\VS{13}Voici la part que Dieu réserve à l'homme méchant, l'héritage que les violents reçoivent du Tout-Puissant.
\VS{14}S'il a des fils en grand nombre, c'est pour l'épée, et ses rejetons ne seront pas rassasiés de pain ;
\VS{15}ses survivants sont ensevelis par la peste, et leurs veuves ne les pleurent pas\FTNT{Ps. 78:64.}.
\VS{16}Parce qu’il entasse l'argent comme la poussière, et qu'il entasse des habits comme on amasse de la boue,
\VS{17}le riche tombe, et il n’est pas relevé ; il ouvre ses yeux, et il ne trouve rien.
\VS{18}Il se bâtit une maison comme celle de la teigne, comme la cabane que fait un gardien\FTNT{Ps. 49:18.}.
\VS{19}Il se couche riche, et il périt dépouillé ; il ouvre les yeux, et tout a disparu.
\VS{20}Les frayeurs l'atteignent comme des eaux ; le tourbillon l'enlève de nuit.
\VS{21}Le vent d'orient l'emporte, et il s'en va ; il l'arrache de sa demeure comme un tourbillon.
\VS{22}Dieu le précipite à terre et ne l'épargne pas, et le méchant voudrait fuir devant sa main.
\VS{23}On applaudit à sa chute, et on le siffle au lieu où il se tient.
\Chap{28}
\VerseOne{}Il y a pour l'argent une mine d'où on le fait sortir, et pour l'or un lieu d'où on le purifie pour l'affiner ;
\VS{2}le fer se tire de la poussière, et la pierre se fond pour produire l'airain.
\VS{3}L'homme fait cesser les ténèbres, il explore jusqu'aux extrêmes limites les pierres cachées dans l'obscurité et dans l'ombre de la mort.
\VS{4}Il creuse un puits, loin des lieux habités ; ne se souvenant plus de ses pieds, il est suspendu, balancé, loin des humains.
\VS{5}La terre, d'où sort le pain, est bouleversée dans ses entrailles comme par le feu.
\VS{6}Ses pierres sont la demeure du saphir, et l'on y trouve de la poudre d'or.
\VS{7}L'oiseau de proie n'en connaît pas le chemin, l'œil du vautour ne l'aperçoit pas ;
\VS{8}les plus jeunes et fiers animaux n'y ont pas marché, le lion n'y a jamais passé.
\VS{9}L'homme avance sa main sur le roc, il renverse les montagnes depuis la racine ;
\VS{10}il fend des tranchées dans les rochers, et son œil voit tout ce qu'il y a de précieux ;
\VS{11}il arrête l'écoulement des eaux, et il fait sortir ce qui est caché.
\VS{12}Mais la sagesse, où se trouve-t-elle ? Où est le lieu où se tient l'intelligence ?
\VS{13}L'homme n'en connaît pas le prix, elle ne se trouve pas dans la terre des vivants.
\VS{14}L'abîme dit : Elle n'est pas en moi ; et la mer dit : Elle n'est pas avec moi.
\VS{15}Elle ne se donne pas contre de l'or pur, elle ne s'achète pas au poids de l'argent\FTNT{Pr. 3:14 ; Pr. 8:11 ; Pr. 16:16.} ;
\VS{16}elle ne se pèse pas contre de l'or d'Ophir, ni contre le précieux onyx, ni contre le saphir.
\VS{17}Elle ne peut se comparer à l'or ni au verre, elle ne peut s'échanger pour un vase d'or fin.
\VS{18}On ne se souvient ni du corail ni du cristal auprès d'elle : La sagesse vaut plus que les perles.
\VS{19}On ne la compare pas avec la topaze d'Ethiopie ; on ne la met pas en balance avec l'or pur.
\VS{20}D'où vient donc la sagesse ? Où est la demeure de l'intelligence ?
\VS{21}Elle est cachée aux yeux de tous les vivants, elle est cachée aux oiseaux des cieux.
\VS{22}L'abîme et la mort disent : Nous en avons entendu parler de nos oreilles.
\VS{23}C'est Dieu qui en sait le chemin, c'est lui qui en connaît la demeure ;
\VS{24}car il regarde jusqu'aux extrémités de la terre, il voit tout sous les cieux\FTNT{Ps. 14:2 ; Ps. 33:13-14 ; Ps. 102:20.}.
\VS{25}Quand il façonna le poids du vent, et qu'il estima la mesure des eaux\FTNT{Pr. 8:29.},
\VS{26}quand il ordonna des lois à la pluie, et qu'il fit un chemin à l'éclair et au tonnerre,
\VS{27}alors il vit la sagesse et la manifesta ; il l'établit et la sonda.
\VS{28}Puis il dit à l'homme : Voici, la crainte du Seigneur, c'est la sagesse ; se détourner du mal, c'est l'intelligence\FTNT{De. 4:6 ; Jé. 9:24 ; Ps. 111:10 ; Pr. 1:7 ; Pr. 9:10 ; Ec. 12:15.}.
\Chap{29}
\TextTitle{La postérité passée de Job}
\VerseOne{}Job prit de nouveau la parole sous forme sentencieuse et dit :
\VS{2}Oh ! Que ne puis-je être comme aux mois du passé, comme aux jours où Dieu me gardait,
\VS{3}quand sa lampe brillait sur ma tête, quand je marchais à sa lumière dans les ténèbres !
\VS{4}Que ne suis-je comme aux jours de mon automne, où Dieu veillait en ami sur ma tente,
\VS{5}quand le Tout-Puissant était encore avec moi, et que mes serviteurs m'entouraient ;
\VS{6}quand je lavais mes pieds dans le lait, et que le rocher répandait près de moi des torrents d'huile\FTNT{De. 32:13.} !
\VS{7}Si je sortais pour aller à la porte de la ville, et si je me faisais préparer un siège dans la place,
\VS{8}les jeunes gens se retiraient en me voyant, les vieillards se levaient et se tenaient debout.
\VS{9}Les princes s'abstenaient de parler, et mettaient la main sur leur bouche ;
\VS{10}la voix des chefs se taisait, et leur langue s'attachait à leur palais.
\VS{11}L'oreille qui m'entendait me disait heureux, l'œil qui me voyait me rendait témoignage ;
\VS{12}car je délivrais l'affligé qui criait au secours, et l'orphelin qui n'avait personne pour le secourir\FTNT{Ps. 72:12 ; Pr. 21:13.}.
\VS{13}La bénédiction de celui qui allait périr venait sur moi ; je remplissais de joie le cœur de la veuve.
\VS{14}Je me revêtais de la justice et elle se revêtait de moi, j'avais ma droiture pour manteau et pour turban\FTNT{Es. 59:17 ; 1 Th. 5:8 ; Ep. 6:14-17.}.
\VS{15}J'étais les yeux de l'aveugle et les pieds du boiteux.
\VS{16}J'étais le père des pauvres, j'examinais la cause de l'inconnu\FTNT{Pr. 29:7.} ;
\VS{17}je brisais les mâchoires de l'injuste, et j'arrachais la proie d'entre ses dents\FTNT{Ps. 58:7.}.
\VS{18}Alors je disais : Je mourrai dans mon nid, mes jours seront aussi nombreux que le sable ;
\VS{19}l'eau pénétrera dans mes racines, la rosée passera la nuit sur mes branches\FTNT{Jé. 17:5-8 ; Ps. 1:3.} ;
\VS{20}ma gloire se renouvellera sans cesse en moi, et mon arc se renouvellera dans ma main.
\VS{21}On m'écoutait et l'on restait dans l'attente, on gardait le silence devant mes conseils.
\VS{22}Après mes discours, nul ne répliquait, et ma parole était pour tous une bienfaisante rosée ;
\VS{23}ils s'attendaient à moi comme à la pluie, ils ouvraient la bouche comme pour une pluie de printemps.
\VS{24}Je souriais quand ils perdaient confiance, et l'on ne pouvait faire tomber la sérénité de mon visage.
\VS{25}J'aimais à aller avec eux, et je m'asseyais à leur tête ; j'étais comme un roi au milieu de ses gardes, comme un consolateur auprès des affligés.
\Chap{30}
\TextTitle{Son humiliation}
\VerseOne{}Mais maintenant !… Chaque jour je suis la risée de plus jeunes que moi, de ceux dont je dédaignais de mettre les pères parmi les chiens de mon troupeau.
\VS{2}Mais à quoi me servirait la force de leurs mains ? En eux avait péri toute vigueur.
\VS{3}Desséchés par la disette et la faim, ils fuient dans les lieux arides, depuis longtemps abandonnés et déserts ;
\VS{4}ils arrachent près des buissons l'herbe sauvage, et la racine des genêts est leur nourriture.
\VS{5}On les chasse du milieu des hommes, on crie après eux comme après un voleur.
\VS{6}Ils habitent dans le creux des torrents, dans les trous de la terre et des rochers ;
\VS{7}ils hurlent parmi les buissons, ils se rassemblent sous les ronces.
\VS{8}Peuple insensé et sans nom, on les repousse du pays !
\VS{9}Et maintenant, je suis le sujet de leurs chansons, je suis en butte à leurs propos\FTNT{Ps. 69:12 ; La. 3:14.}.
\VS{10}Ils m'ont en horreur, ils s'éloignent de moi, ils ne se retiennent pas de me cracher leur salive au visage.
\VS{11}Ils n'ont aucune retenue et ils m'humilient, ils rejettent tout frein devant moi.
\VS{12}Ces misérables se lèvent à ma droite et me poussent les pieds, ils se fraient contre moi des routes pour ma ruine\FTNT{Ps. 35:15.} ;
\VS{13}ils détruisent mon propre sentier et travaillent à ma perte, eux à qui personne ne viendrait en aide ;
\VS{14}ils viennent contre moi comme par une brèche large, et ils se sont jetés sur moi à cause de ma désolation.
\VS{15}Toutes les terreurs se tournent contre moi ; ma gloire est emportée comme par le vent, mon bonheur a passé comme un nuage\FTNT{Os. 13:3.}.
\VS{16}Et maintenant, mon âme se répand en mon sein, les jours d'affliction m'ont saisi.
\VS{17}La nuit me perce et m'arrache les os, la douleur qui me ronge ne se donne aucun repos.
\VS{18}Par la violence du mal, mon vêtement se déforme, il se colle à mon corps comme ma tunique.
\VS{19}Dieu m'a jeté dans la boue, et je ressemble à la poussière et à la cendre.
\VS{20}Je crie vers toi, et tu ne me réponds pas ; je me tiens debout, et tu m'aperçois.
\VS{21}Tu deviens cruel contre moi, tu t'opposes à moi avec la force de ta main.
\VS{22}Tu me soulèves, tu me fais chevaucher sur le vent, et tu me fais fondre au bruit de la tempête.
\VS{23}Car, je le sais, tu me mènes à la mort, à la demeure fixée pour tous les vivants\FTNT{Hé. 9:27.}.
\VS{24}Mais celui qui va périr n'étend-il pas les mains ? Celui qui est dans le malheur n'implore-t-il pas du secours ?
\VS{25}Ne pleurais-je pas sur l'homme qui passait des jours difficiles ? Mon âme n'avait-elle pas pitié du pauvre\FTNT{Ro. 12:15.} ?
\VS{26}J'attendais le bonheur, et le malheur est arrivé ; j'espérais la lumière, et les ténèbres sont venues.
\VS{27}Mes entrailles bouillonnent sans repos, les jours d'affliction m'ont confronté.
\VS{28}Je marche noirci, mais non par le soleil ; je me lève en pleine assemblée, et je crie.
\VS{29}Je suis devenu le frère des serpents, le compagnon des autruches\FTNT{Ps. 102:7-8.}.
\VS{30}Ma peau noircit et tombe, mes os brûlent et se dessèchent\FTNT{La. 4:8 ; La. 5:10.}.
\VS{31}Ma harpe n'est plus qu'un instrument de deuil, et mon chalumeau ne peut rendre que des voix en pleurs.
\Chap{31}
\TextTitle{Job se justifie}
\VerseOne{}J'avais fait une alliance avec mes yeux, et je n'aurais pas regardé une vierge.
\VS{2}Quelle part Dieu m'eût-il réservée d'en haut ? Quel héritage le Tout-Puissant m'aurait-il envoyé des cieux ?
\VS{3}La ruine n'est-elle pas pour l'injuste, et le malheur pour ceux qui commettent l'iniquité ?
\VS{4}Dieu ne voit-il pas mes voies ? Ne compte-t-il pas tous mes pas\FTNT{Pr. 5:21 ; Pr. 15:3 ; 2 Ch. 16:9.} ?
\VS{5}Si j'ai marché dans le mensonge, si mon pied s'est hâté pour tromper,
\VS{6}que Dieu me pèse dans des balances justes, et il reconnaîtra mon intégrité !
\VS{7}Si mes pas se sont détournés du droit chemin, si mon cœur a suivi mes yeux, si quelque souillure s'est attachée à mes mains,
\VS{8}Que je sème et qu'un autre mange, et tout ce que j'aurais fait produire soit déraciné!
\VS{9}Si mon cœur a été séduit par une femme, si j'ai fait le guet à la porte de mon prochain\FTNT{Pr. 7.},
\VS{10}que ma femme broie le grain pour un autre, et que d'autres se penchent sur elle !
\VS{11}Car c'est un crime, une iniquité punie par les juges ;
\VS{12}c'est un feu qui dévore jusqu'à la destruction, et qui aurait détruit toutes mes récoltes dans leur racine.
\VS{13}Si j'ai méprisé le droit de mon serviteur ou de ma servante, lorsqu'ils étaient en contestation avec moi,
\VS{14}qu'ai-je à faire, quand Dieu se lève ? Qu'ai-je à répondre, quand il châtie ?
\VS{15}Celui qui m'a fait dans le ventre de ma mère ne l'a-t-il pas fait aussi ? Un même Dieu ne nous a-t-il pas formés dans le sein maternel\FTNT{Pr. 14:31 ; Pr. 17:5.} ?
\VS{16}Si j'ai refusé aux pauvres leur désir, si j'ai laissé se consumer les yeux de la veuve\FTNT{Es. 10:2 ; Lu. 18:2-3.},
\VS{17}si j'ai mangé seul mon morceau de pain, sans que l'orphelin en ait sa part,
\VS{18}moi qui l'ai dès ma jeunesse fait grandir près de moi comme un père, et qui dès le sein de ma mère, ai été le guide de la veuve ;
\VS{19}si j'ai vu le malheureux périr faute de vêtements, le pauvre manquer de couverture\FTNT{Mt. 25:41-45.},
\VS{20}sans que ses reins m'aient béni, sans qu'il ait été réchauffé par la toison de mes agneaux ;
\VS{21}si j'ai levé la main contre l'orphelin, parce que je me voyais comme un appui dans les portes\FTNT{Pr. 22:22.} ;
\VS{22}que mon épaule tombe de sa jointure, que mon bras tombe et qu'il se brise l'os !
\VS{23}Car les châtiments de Dieu m'épouvantent, et je ne pourrais pas prévaloir devant sa majesté.
\VS{24}Si j'ai mis dans l'or ma confiance, si j'ai dit à l'or fin : Tu es mon espoir\FTNT{Mc. 10:24 ; 1 Ti. 6:17.} ;
\VS{25}si je me suis réjoui de ma grande puissance, de la quantité des richesses que ma main a acquise\FTNT{Ps. 62:11.} ;
\VS{26}si j'ai regardé le soleil quand il brillait, la lune quand elle s'avançait de façon majestueuse,
\VS{27}et si mon cœur s'est laissé secrètement séduire, si ma main a envoyé des baisers de ma bouche ;
\VS{28}c'est encore une iniquité que doit punir le juge, et j'aurais renié le Dieu d'en haut !
\VS{29}Si je me suis réjoui du malheur de mon ennemi, si j'ai sauté d'allégresse quand le mal l'a atteint\FTNT{Mt. 5:43-44.},
\VS{30}moi qui n'ai pas permis à ma langue de pécher en demandant sa mort par des malédictions ;
\VS{31}si les gens de ma tente ne disaient pas : Où est celui qui n'a pas été rassasié de sa viande\FTNT{Ps. 27:2.} ?
\VS{32}Si l'étranger passait la nuit dehors, si je n'ouvrais pas ma porte au voyageur\FTNT{Ge. 19:1-2 ; De. 10:19 ; 1 Pi. 4:9 ; Hé. 13:2.} ;
\VS{33}si, comme les hommes, j'ai caché mes transgressions et mon crime dans mon sein\FTNT{Ge. 3:10-12 ; Pr. 28:13.},
\VS{34}parce que je craignais la multitude, et je craignais le mépris des familles, en sorte que je restais tranquille et n'osais franchir ma porte…
\VS{35}Oh ! Qui me fera trouver quelqu'un qui m'écoute ? Voilà ma défense toute signée : Que le Tout-Puissant me réponde ! Qui me donnera la plainte écrite par mon adversaire ?
\VS{36}Je porterai son écrit sur mon épaule, je l'attacherai sur mon front comme une couronne ;
\VS{37}je lui déclarerai le nombre de mes pas, je m'approcherai de lui comme un prince.
\VS{38}Si ma terre crie contre moi, et que ses sillons pleurent ;
\VS{39}si j'en ai mangé le produit sans l'avoir payée, et que j'aie attristé l'âme de ses anciens maîtres ;
\VS{40}Qu'elle en produise des épines au lieu du froment, et de l'ivraie au lieu de l'orge ! C'est ici la fin des paroles de Job.
\Chap{32}
\TextTitle{Discours d'Elihu : reproches à Job et à ses amis}
\VerseOne{}Ces trois hommes-là cessèrent de répondre à Job, parce qu'il se regardait comme juste.
\VS{2}Élihu, fils de Barakeel de Buz, de la famille de Ram, s'enflamma de colère contre Job, parce qu'il disait son âme juste devant Dieu.
\VS{3}Et sa colère s'enflamma contre ses trois amis, parce qu'ils ne trouvaient rien à répondre et que néanmoins ils condamnaient Job.
\VS{4}Comme ils étaient plus âgés que lui, Elihu avait attendu jusqu'à ce moment pour parler à Job.
\VS{5}Mais, voyant que ces trois hommes n'avaient plus aucune réponse à la bouche, Elihu se mit en colère.
\VS{6}Et Elihu, fils de Barakeel de Buz, prit la parole et dit : Je suis jeune et vous êtes des vieillards ; c'est pourquoi j'ai craint, j'ai eu peur de vous faire connaître mon sentiment.
\VS{7}Je disais: les jours parleront, et le grand nombre des années fera connaître la sagesse.
\VS{8}L'esprit dans l'homme, c'est l'esprit, le souffle du Tout-Puissant qui rend intelligent\FTNT{Da. 1:17 ; Da. 2:21 ; Pr. 2:6 ; Ec. 2:26.} ;
\VS{9}ce ne sont pas les aînés qui sont sages, ce ne sont pas les vieillards qui comprennent ce qui est juste.
\VS{10}C'est pourquoi je dis : Ecoute-moi ! Et je dirai aussi ma pensée.
\VS{11}J'ai attendu la fin de vos discours, j'ai écouté vos raisonnements, jusqu'à ce que vous ayez bien examiné les discours de Job.
\VS{12}J'ai pris le soin de vous écouter ; et voici, aucun de vous n'a convaincu Job, aucun n'a répondu à ses paroles.
\VS{13}Qu'il ne vous n'arrive pas de dire: nous avons trouvé la sagesse ; c'est Dieu qui le poursuit, et non pas l'homme !
\VS{14}Il n'a pas dirigé ses discours contre moi : Aussi je ne lui répondrai pas à votre manière.
\VS{15}Ils sont étonnés! Ils ne répondent plus rien! On leur a ôté la parole!
\VS{16}J'ai attendu jusqu'à ce qu'ils n'ont plus rien dit, car ils sont demeurés muets, et ils n'ont plus su que répondre.
\VS{17}A mon tour, je veux répondre pour moi, et je veux donner mon avis.
\VS{18}Car je suis rempli de discours, l'esprit qui est en mon sein me presse.
\VS{19}Mon sein est comme vin sans air, comme des outres neuves qui vont éclater\FTNT{Mt. 9:17 ; Mc. 2:22 ; Lu. 5:38.}.
\VS{20}Je parlerai pour respirer à l'aise, j'ouvrirai mes lèvres et je répondrai.
\VS{21}Je ne ferai pas acception de personnes, et je flatterai aucun homme.
\VS{22}Car je ne sais pas flatter : Mon Créateur m'enlèverait bien vite.
\Chap{33}
\TextTitle{Discours d'Elihu : la justice de Dieu}
\VerseOne{}C'est pourquoi Job, écoute mon discours, je te prie et prête l'oreille à toutes mes paroles !
\VS{2}Voici, j'ouvre la bouche, ma langue parle dans mon palais.
\VS{3}Mes paroles exprimeront la droiture de mon cœur, mes lèvres diront la vérité pure.
\VS{4}L'Esprit de Dieu m'a fait, et le souffle du Tout-Puissant me donne la vie\FTNT{Ge. 2:7.}.
\VS{5}Si tu peux, réponds-moi, dresse-toi contre moi, demeure ferme!
\VS{6}Voici, je suis pour le Dieu fort, selon que tu en as parlé; j'ai été formé de la terre tout comme toi.\FTNT{Ac. 14:15.} ;
\VS{7}Voici ma terreur ne te trouble pas, et ma main ne s'appesantit pas sur toi
\VS{8}Quoi qu'il en soit, tu as dit, moi l'entendant, j'ai entendu la voix de tes discours:
\VS{9}Je suis pur, sans péché, je suis net, il n'y a pas d'iniquité en moi.
\VS{10}Voici il cherche à rompre avec moi, il me considère comme son ennemi ;
\VS{11}il met mes pieds dans les ceps, il surveille tous mes chemins.
\VS{12}Je te répondrai qu'en cela tu n'as pas été juste, car Dieu sera toujours plus grand que l'homme.
\VS{13}Pourquoi as-tu donc plaidé contre lui ? Car il ne rend pas compte de toutes ses actions.
\VS{14}Car Dieu parle une première fois, et une seconde fois à celui qui n'aura pas pris garde à la première.
\VS{15}Par des songes, par des visions nocturnes, quand les hommes tombent dans un profond sommeil, quand ils dorment sur leur couche.
\VS{16}Alors il ouvre l'oreille de l'homme d'une mauvaise action et de rabaisser la fierté de l'homme.
\VS{17}afin de détourner l'homme de son œuvre et de le préserver de l'orgueil,
\VS{18}il garantit son âme de la fosse, et sa vie de l'épée.
\VS{19}L'homme est aussi châtié par des douleurs sur son lit, à cause d'une lutte perpétuelle en ses os\FTNT{Ps. 38:4.}.
\VS{20}Alors sa vie prend en horreur le pain et son âme les mets les plus désirés\FTNT{Ps. 107:18.} ;
\VS{21}Sa chair est tellement consumée qu'elle paraît plus, ses os sont tellement brisés, qu'on y connaît plus rien;
\VS{22}son âme s'approche de la fosse, et sa vie des messagers de la mort.
\VS{23}Mais s'il y a pour cet homme un messager qui interprète, un d'entre les mille, pour lui annoncer la voie de la droiture,
\VS{24}alors Dieu prend pitié de lui et dit : Garantis-le, afin qu'il ne descende pas dans la fosse ; j'ai trouvé la propitiation!
\VS{25}Sa chair devient plus délicate qu'elle n'était dan son enfance; il revient aux jours de sa jeunesse.
\VS{26}Il supplie Dieu par ses prières, et Dieu lui est favorable, il lui laisse voir sa face avec joie, et lui rend sa justice\FTNT{Es. 58:9.}.
\VS{27}Il regarde vers les hommes et dit : J'ai péché, j'ai violé la justice, et je n'ai pas été puni comme je le méritais ;
\VS{28}Dieu a racheté mon âme afin qu'elle ne passe pas dans la fosse, et ma vie voit encore la lumière !
\VS{29}Voilà ce que Dieu fait, deux fois, trois fois, envers l'homme\FTNT{Ps. 62:11.},
\VS{30}pour ramener son âme de la fosse, pour l'éclairer de la lumière des vivants\FTNT{Ps. 56:14.}.
\VS{31}Sois attentif, Job, écoute-moi ! Tais-toi, et je parlerai !
\VS{32}Si tu as quelque chose à dire, réponds-moi ! Parle, car je désire te justifier.
\VS{33}Sinon, écoute, tais-toi et je t'enseignerai la sagesse.
\Chap{34}
\TextTitle{Discours d'Elihu : il accuse Job de se révolter}
\VerseOne{}Elihu reprit la parole, et dit :
\VS{2}Sages, écoutez mes discours ! Vous qui avez la connaissance, prêtez-moi l'oreille !
\VS{3}Car l'oreille discerne les discours, comme le palais savoure ce qu'il mange.
\VS{4}Choisissons ce qui est juste, voyons entre nous ce qui est bon.
\VS{5}Job dit : Je suis juste, et Dieu a écarté ma justice;
\VS{6}mentirai-je à mon droit? Ma flèche est mortelle sans que j'aie commis de crime.
\VS{7}où y a-t-il un homme comme Job, qui boit le péché la moquerie comme l'eau,
\VS{8}qui marche en compagnie des ouvriers d'iniquité, et qui fréquente  avec les hommes marchant de pair avec les hommes méchants ?
\VS{9}Car il a dit : Il est inutile à l'homme de plaire à Dieu\FTNT{Mal. 3:14.}.
\VS{10} C'est pourquoi écoutez, vous qui avez de l'intelligence, écoutez-moi ! Loin de Dieu la méchanceté, loin du Tout-Puissant l'injustice\FTNT{De. 32:4 ; Ps. 92:16 ; Ro. 9:14.} !
\VS{11}Car il rend à l'homme selon son œuvre, il fait trouver à chacun selon sa voie\FTNT{Jé. 17:10 ; Jé. 32:19 ; Ez. 7:27 ; Pr. 24:12 ; Mt. 16:27 ; Ro. 2:6 ; 2 Co. 5:10 ; Ep. 6:8 ; Ap. 22:12.}.
\VS{12}Certes, Dieu ne commet pas l'injustice ; le Tout-Puissant ne renverse pas le droit.
\VS{13}Qui lui a donné la terre en charge ? Ou Qui a placé la terre habitable?
\VS{14}S'il ne pensait qu'à lui-même, s'il retirait à lui son esprit et son souffle\FTNT{Ps. 104:29.},
\VS{15}toute chair périrait ensemble, et l'homme retournerait dans la poussière\FTNT{Ge. 3:19 ; Ec. 3:20 ; Ec. 12:9.}.
\VS{16}Si donc tu as de l'intelligence, écoute ceci, prête l'oreille à ce que tu entendras de moi.
\VS{17}Comment celui qui n'aimerait pas à faire la justice jugerait-il le monde? Et condamneras-tu celui qui est souverainement juste ?
\VS{18}Dira-t-on à un roi, qu'il est un scélérat ? Et aux princes, qu'ils sont des méchants ?
\VS{19}Combien moins le dira-t-on à celui qui n'a point d'égard à la personne des grands, et qui ne connaît point les riches pour les préférer aux pauvres, parce qu'ils sont tous l'ouvrage de ses mains\FTNT{De. 10:17 ; 2 Ch. 19:7 ; Ac. 10:34 ; Ga. 2:6 ; Ro. 2:11 ; Ep. 6:9 ; Col. 3:25.} ?
\VS{20}En un moment, ils mourront ; au milieu de la nuit, un peuple est ébranlé et passe ; le puissant s'en va, sans la main d'aucun homme.
\VS{21}Car les yeux de Dieu sont sur les voies de l'homme, il regarde tous ses pas.
\VS{22}Il n'y a ni ténèbres ni ombre de la mort où puissent se cacher les ouvriers d'iniquité.
\VS{23}Dieu ne regarde pas à deux fois un homme, pour le faire aller en jugement avec lui.
\VS{24}Il brise les puissants par des voies incompréhensibles, et il n établit d'autres à leur place ;
\VS{25}car il connaît leurs œuvres. Il les renverse de nuit, et ils sont écrasés ;
\VS{26}il les frappe comme des impies, au lieu où se tiennent tous les regards.
\VS{27}Du fait qu'ils se sont détournés de lui, et qu'ils n'ont considéré aucune de ses voies.
\VS{28}ils ont fait monter à Dieu le cri du pauvre, et il a entendu le cri des affligés\FTNT{Ja. 5:4.}.
\VS{29}S'il donne le repos, qui est-ce qui causera du trouble? S'il cache sa face à quelqu'un, qui le regardera, qu'il s'agisse de toute une nation ou d'un seul un homme?
\VS{30}afin que l'hypocrite ne règne pas, de peur qu'il plus un piège pour le peuple.
\VS{31}Car a-t-il jamais dit à Dieu : J'ai été pardonné, je ne pécherai plus ;
\VS{32}montre-moi ce que je ne vois pas ; si j'ai fait le mal, je ne le ferai plus ?
\VS{33}Mais Dieu ne te le rendra-t-il pas, puisque tu as rejeté son châtiment, quand tu as fait le choix que tu as fait? Pour moi, je ne sais que dire à cela; mais toi, si tu as quelque chose à répondre, parle.
\VS{34}Les gens de bon sens diront avec moi, et tout homme sage en conviendra,
\VS{35}que Job ne parle pas avec connaissance, et ses paroles manquent d'intelligence.
\VS{36}Ah! Mon père, que Job soit éprouvé jusqu'à ce qu'il soit vaincu, puisqu'il vaincu, puisqu'il répond comme les impies.
\VS{37}Car il ajoute péché sur péché; il applaudit au milieu de nous ; il parle de plus en plus contre Dieu.
\Chap{35}
\TextTitle{Discours d'Elihu : il reproche à Job ses propos irréfléchis}
\VerseOne{}Elihu reprit la parole et dit :
\VS{2}Penses-tu avoir raison de dire : Je suis juste devant Dieu ?
\VS{3}Quand tu dis : Que me sert-il, et que gagnerais-je de plus sans pécher ?
\VS{4}Je te répondrai en ces termes, et à tes amis qui sont avec toi.
\VS{5}Regarde les cieux, et considère-les ! Vois les nuées, comme elles sont plus hautes que toi !
\VS{6}Si tu pèches, quel mal fais-tu à Dieu ? Et si tes péchés se multiplient, quel mal reçoit-il ?
\VS{7}Si tu es juste, que lui donnes-tu ? Que reçoit-il de ta main ?
\VS{8}C'est à un homme, comme tu es, que ta méchanceté peut seule nuire, et c'est au fils d'un homme que ta justice peut seule être utile.
\VS{9}On fait crier les opprimés par la grandeur des maux qu'on leur afflige; ils crient à cause de la violence des grands;
\VS{10}Et nul ne dit : Où est le Dieu qui m'a fait, qui donne de quoi chanter pendant la nuit ,
\VS{11}qui nous instruit plus que les animaux de la terre, et plus intelligent que les oiseaux des cieux ?
\VS{12}On crie donc à cause de la fierté des méchants; mais Dieu ne l'exauce pas.
\FTNT{Es. 1:15 ; Ez. 8:18 ; Mi. 3:4 ; Jn. 9:31.}.
\VS{13} Cependant que ce soit en vain; que Dieu n'écoute pas, et que le Tout-puissant n'y a pas égard.
\VS{14}Encore moins dois-tu lui dire, tu ne le vois pas; car le jugement est devant lui, attends-le donc!
\VS{15}Mais maintenant, ce n'est rien ce que sa colère exécute, et il n'a pas encore pris connaissance en profondeur toutes les choses que tu as faites.
\VS{16}Job ouvre donc sa bouche pour se plaindre, il multiplie les paroles sans intelligence.
\Chap{36}
\TextTitle{Discours d'Elihu : Dieu traite les hommes selon leurs oeuvres}
\VerseOne{}Elihu continua de parler, et dit :
\VS{2}Attends un peu, et je te montrerai qu'il y a encore d'autres raisons pour la cause de Dieu.
\VS{3}Je tirerai de loin mes raisons, et je défendrai la justice du Créateur.
\VS{4}Car certainement il n'y aura rien de faux en tout ce que je dirai, et celui qui est avec toi, est parfait dans sa connaissance.
\VS{5}Dieu est puissant, mais il ne méprise personne ; il est puissant par la force de son coeur.
\VS{6}Il ne laisse pas vivre le méchant, et il fait droit aux pauvres.
\VS{7}Il ne détourne pas ses yeux de dessus les justes, il les place sur le trône avec les rois, il les y fait asseoir pour toujours, afin qu'ils soient élevés\FTNT{Ps. 33:18 ; Ps. 34:16.}.
\VS{8}S'ils sont liés de chaînes, s'ils sont pris dans les liens de l'affliction,
\VS{9}il leur fait connaître leurs œuvres, leurs transgressions, leur orgueil.
\VS{10}Alors il ouvre leur oreille pour leur discipline, il leur dit de se détourner de l'iniquité.
\VS{11}S'ils écoutent, et s'ils le servent, ils achèvent leurs jours dans le bonheur, leurs années dans la joie.
\VS{12}S'ils n'écoutent pas, ils passent par l'épée, ils expirent dans leur aveuglement.
\VS{13}Ceux qui sont hypocrites dans leur cœur, ils ne crient pas à lui quand il les a liés ;
\VS{14}leur personne meurt dans sa jeunesse, leur vie s'éteint parmi les débauchés.
\VS{15}Mais Dieu sauve celui qui est affligé de son oppression, et c'est par la détresse qu'il lui ouvre les oreilles.
\VS{16}Il t'écartera aussi de la détresse, pour te mettre au large, loin de toute angoisse, et ta table sera chargée de viandes grasses\FTNT{Ps. 50:15 ; Ps. 63:6.}.
\VS{17}Or tu remplis le jugement du méchant, mais le jugement et le droit subsisteront.
\VS{18}Certainement Dieu et irrité; prends garde qu'il ne te  plonge dans l'affliction, car il n'y aura pas alors de rançon si grande pour te délivrer\FTNT{Ps. 49:8.} !
\VS{19}Tes cris valent-ils ton or, et même toutes les forces qui se trouvent dans tes richesses ?
\VS{20}Ne soupire pas après la nuit, qui enlève les peuples de leur place.
\VS{21}Garde-toi de te retourner vers l'iniquité, car la souffrance t'y dispose.
\VS{22}Dieu est élevé par sa puissance ; qui saurait enseigner comme lui ?
\VS{23}Qui lui prescrit le chemin qu'il devait tenir? Qui lui dit : Tu a fait une injustice ?
\VS{24}Souviens-toi de célébrer ses ouvrages, que tous les hommes voient.
\VS{25}Tout homme les voit, chacun les contemple de loin.
\VS{26}Dieu est grand, mais nous ne le connaissons pas, quant au nombre de ses années il est insondable\FTNT{Es. 63:16 ; Ps. 92:8 ; Ps. 93:2 ; Ps. 102:13 ; La. 5:19.}.
\VS{27}Parce qu'il met les eaux  en petites gouttes, elle répandent la pluie selon la vapeur d'eau qui la contient ;
\VS{28}les nuées la font dégoutter, elles coulent sur les hommes en abondance.
\VS{29}Et qui pourra comprendre l'étendue des nuages et le son éclatant de sa tente ?
\VS{30}Voici, il étend sa lumière sur elle, et il se cache jusque dans les profondeurs de la mer.
\VS{31} Or c'est par ces choses qu'il juge les peuples, qu' il donne la nourriture en abondance.
\VS{32} Il tient caché dans les paumes de ses mains le feu étincelant, et lui ordonne de frapper de ce qui se présente à sa rencontre.
\VS{33} Son bruit porte les nouvelles ; les troupeaux font connaître qu'il approche.
\Chap{37}
\TextTitle{Discours d'Elihu : conclusion}
\VerseOne{}Mon cœur même à cause de cela est tout tremblant, il sort de sa place.
\VS{2}Ecoutez attentivement et en tremblant le bruit de sa voix, le grondement qui sort de sa bouche\FTNT{Ps. 29:3-9.} !
\VS{3}Il le conduit dans toute l'étendue des cieux, et son éclair brille jusqu'aux extrémités de la terre\FTNT{Ps. 97:4.}.
\VS{4}Après lui de l'élève un grand bruit, il tonne de sa voix majestueuse; il ne tarde pas après que sa voix a été entendue\FTNT{Jé. 10:13.}.
\VS{5}Dieu tonne avec sa voix d'une manière étonnante ; il fait de grandes choses que nous ne comprenons pas.
\VS{6}Car il dit à la neige : Tombe sur la terre ! Il le dit à la pluie, même aux plus fortes pluies.
\VS{7}Il met un sceau à la main de tous les hommes pour reconnaître tous les hommes qui sont son ouvrage.
\VS{8}Les bêtes entrent dans leurs tanières, et elles demeurent dans leurs repaires.
\VS{9}L'ouragan vient du fond du sud, et le froid vient des vents du nord.
\VS{10}Par son souffle, Dieu donne la glace, et il réduit l'espace où se répondaient au large les eaux\FTNT{Ps. 147:17-18.}.
\VS{11}Il lasse les nuages à force d'arroser, il écarte les nuages par sa lumière.
\VS{12}Et ceux-ci font plusieurs tours pour faire ce qu'il a commandé, sur la face de la terre sur la face de la terre habitée ;
\VS{13}Il les fait venir pour s'en servir soit comme une verge pour la terre, soit pour répandre ses bienfaits\FTNT{Ex. 9:18-23 ; 1 S. 12:18-19.}.
\VS{14}Job, arrête-toi, prête l'oreille à ces choses ! Considère encore les merveilles de Dieu !
\VS{15}Sais-tu comment Dieu les dispose, et fait briller la lumière de ses nuages?
\VS{16}Connais-tu le balancement des nuages, les merveilles de celui dont la science est parfaite ?
\VS{17}Sais-tu pourquoi tes vêtements sont chauds quand la terre se repose par le vent du midi ?
\VS{18}Peux-tu étendre avec lui les cieux, aussi fermes qu'un miroir de fonte ?
\VS{19}Montre-nous ce que nous pouvons lui dire ; car nous ne saurions rien dire par ordre à cause de nos ténèbres. 
\VS{20}Lui racontera-t-on quand je parlerai ? S'il y a un homme qui en parle, certainement il en sera englouti ?
\VS{21}Et maintenant on ne voit pas la lumière du soleil qui resplendit dans les cieux, lorsque le vent passe et le nettoie ;
\VS{22}Le temps qui la reluit comme l'or vient du nord. Il y a en Dieu une majesté redoutable.
\VS{23}Nous ne saurions comprendre le Tout-Puissant, grand en puissance, en jugement et en abondante justice, il n'opprime personne !
\VS{24}C'est pourquoi les hommes le craignent ; mais il ne les voit pas tous sages de cœur\FTNT{Ps. 92:7 ; Ro. 1:21.}.
\Chap{38}
\TextTitle{Yahweh interroge Job}
\VerseOne{}Yahweh répondit à Job du milieu de le tourbillon et dit :
\VS{2}Qui est celui qui obscurcit mes décisions par des paroles sans connaissance ?
\VS{3}Ceins maintenant tes reins comme un vaillant homme ; je t'interrogerai, et tu me fera voir ta science.
\VS{4}Où étais-tu quand je fondais la terre ? Dis-le, si tu as de l'intelligence\FTNT{Pr. 8:29.}.
\VS{5}Qui en a réglé les mesures, le sais-tu ? Ou qui a appliqué sur elle le niveau ?
\VS{6}Sur quoi ses bases sont-elles plantées ? Ou qui en a posé la pierre angulaire pour la soutenir\FTNT{Ps. 104:5.}?
\VS{7}Quand les étoiles du matin se réjouissent ensemble, et que tous les fils de Dieu poussent des cris de joie \FTNT{Ps. 148:3.} ?
\VS{8}Qui a renfermé la mer dans ses bords, quand elle fut tirée de la matrice et qu'elle sortit? 
\VS{9}quand je lui donnai la nuée pour vêtement, et l'obscurité pour langes ;
\VS{10}que je lui imposai ma loi, et que je lui mis des barrières et des portes;
\VS{11} et quand je dis : Tu viendras jusqu'ici, tu n'iras pas plus loin ; ici s'arrêtera l'orgueil de tes flots ?
\VS{12}Depuis que tu es au monde, as-tu commandé au matin et as-tu montré à l'aube du jour le lieu où elle doit se lever,
\VS{13}pour qu'elle saisisse les extrémités de la terre, et que les méchants en soient chassés ;
\VS{14}pour que la terre prenne une forme comme l'argile qui reçoit un sceau, et qu'elle soit parée comme d'un vêtement nouveau ;
\VS{15}pour que la lumière soit ôtée aux méchants, et que le bras qui se lève soit brisé\FTNT{Ps. 10:15.} ?
\VS{16}As-tu pénétré jusqu'aux sources de la mer ? T'es-tu promené dans les profondeurs de l'abîme ?
\VS{17}Les portes de la mort se sont-elles découvertes à toi ? As-tu vu les portes de l'ombre de la mort ?
\VS{18}As-tu compris l'étendue de la terre ? Si tu sais tout cela, dis-le !
\VS{19}Où est la demeure de la lumière, et où est le lieu des ténèbres ?
\VS{20}Pour que tu les prennes à leur limite, et que tu connaisses le chemin de leur maison ?
\VS{21}Tu le sais, car alors tu étais né, et le nombre de tes jours est grand !
\VS{22}Es tu entré dans les trésors de la neige ? As-tu vu les trésors de grêle,
\VS{23}que je réserve pour les temps de détresse, pour les jours de guerre et de bataille\FTNT{Ex. 9:23 ; Jos. 10:11 ; Ap. 8 :7.} ?
\VS{24}Par quel chemin la lumière se partage la lumière, et le vent d'orient se répand-il sur la terre\FTNT{Jn. 3:8.} ?
\VS{25}Qui a ouvert un conduit aux inondations, et tracé la route de l'éclair et du tonnerre,
\VS{26}pour qu'elle pleuve sur une terre sans habitants, sur un désert sans hommes\FTNT{Ps. 104:13-14 ; PS. 147:8 ; Ac. 14:17.} ;
\VS{27}pour qu'elle abreuve les lieux solitaires et arides, et qu'elle fasse germer et sortir l'herbe ?
\VS{28}La pluie a-t-elle un père ? Qui enfante les gouttes de la rosée ?
\VS{29}De quel sein est sortie la glace ? Et qui enfante le givre du ciel,
\VS{30}pour que les eaux se cachent comme une pierre, et que le dessus de l'abîme soit enchaîné ?
\VS{31}Peux-tu resserer les liens des pléiades ou détacher les chaînes d'orient\FTNT{Am. 5:8.}?
\VS{32}Fais-tu sortir en leur temps les signes du zodiaque, et conduis-tu la Grande Ourse avec ses petits ?
\VS{33}Connais-tu les lois du ciel ? Disposes-tu de son pouvoir sur la terre\FTNT{Jé. 31:35-36 ; Ps. 104:4.} ?
\VS{34}Élèves-tu la voix jusqu'aux nuées, pour que des eaux abondantes te couvrent ?
\VS{35}Envoies-tu les éclairs ? Partent-ils ? Te disent-ils : Nous voici ?
\VS{36}Qui a mis la sagesse dans le cœur, ou qui a donné l'intelligence à l'esprit\FTNT{Ec. 2:26.} ?
\VS{37}Qui Est-ce qui peut avec intelligence compter les nuages, et pour placer les outres des cieux,
\VS{38}Quand la poussière est détrompée par les eaux qui l'arrosent, et que les mottes viennent à se joindre ?
\Chap{39}
\TextTitle{Yahweh démontre son omnipotence}
\VerseOne{}Chasses-tu de la proie pour la lionne, et apaises-tu la faim des lionceaux\FTNT{Ps. 104:21.},
\VS{2}Quand ils se tapissent dans leurs tanières et se tiennent aux aguets dans leur repaire ?
\VS{3}Qui est-ce qui apprête la nourriture au corbeau, quand ses petits crient à Dieu, et qu'ils vont errants, parce qu'ils n'ont point de quoi manger\FTNT{Ps. 104:27 ; Ps. 147:9 ; Mt. 6:26.} ?
\VS{4}Sais-tu quand les boucs de rochers mettent bas ? Observes-tu les biches de rochers quand elles font leurs petits\FTNT{Ps. 29:9.} ?
\VS{5}Comptes-tu les mois de leur gestation, et sais-tu le temps auquel elles font leurs petits ?
\VS{6}Et qu'elles se courbent pour mettre bas leurs petits et se délivrent de leurs douleurs ?
\VS{7}Leurs petits se fortifient, ils croissent en plein air, ils s’en vont et ne reviennent plus vers elles.
\VS{8}Qui a laissé aller libre l’âne sauvage ? et qui a délié les liens de l’âne farouche ?
\VS{9}Auquel j’ai donné le désert pour maison, et la terre inhabitée pour ses retraites\FTNT{Jé. 2:24.} ?
\VS{10}Il se rit du bruit des villes, il n'entend pas les cris d'un exacteur.
\VS{11}Les montagnes qu'il va épiant çà et là, sont ses pâturages, et il cherche toute sorte de verdure. 
\VS{12}Le buffle voudra-t-il te servir, ou demeurera-t-il à ta crèche ? 
\VS{13}Lies-tu le buffle avec son licou pour labourer ? ou rompra-t-il les mottes des vallées après toi ? 
\VS{14}Te fies-tu à lui parce que sa force est grande, et lui abandonnes-tu ton travail? 
\VS{15}Comptes-tu sur lui pour rentrer ta semence, et pour l'amasser sur ton aire ? 
\VS{16}As-tu donné aux paons ce plumage qui est si brillant, ou à l'autruche les ailes et les plumes ? 
\VS{17}Néanmoins elle abandonne ses oeufs à terre, et les fait échauffer sur la poussière ;
\VS{18}et elle oublie que le pied peut les écraser, ou que les bêtes des champs peuvent les fouler. 
\VS{19}Elle est dure envers ses petits, comme s’ils n’étaient pas siens. Son travail est vain, elle ne s’en inquiète pas.
\VS{20}Car Dieu l'a privée de sagesse et ne lui a pas donné l'intelligence.
\VS{21}A la première occasion elle se dresse en haut, et se moque du cheval et de celui qui le monte. 
\VS{22}As-tu donné la force au cheval ? et as-tu revêtu son cou d'un hennissement éclatant comme le tonnerre ? 
\VS{23}Fais-tu bondir le cheval comme la sauterelle ? le son magnifique de ses narines est effrayant.
\VS{24}Il creuse la terre de son pied, il s'égaie dans sa force, il va à la rencontre d'un homme armé ;
\VS{25}Il se rit de la frayeur, il ne s'épouvante de rien, et il ne se détourne point de devant l'épée.
\VS{26}Il n'a point peur des flèches qui sifflent tout autour de lui, ni du fer luisant de la lance et du javelot. 
\VS{27}Il creuse la terre, plein d'émotion et d'ardeur au son de la trompette, et il ne peut se retenir. 
\VS{28}Au son bruyant de la trompette, il dit : En avant ! En avant ! Il flaire de loin la bataille, le tonnerre des capitaines, et le cri de triomphe.
\VS{29}Est-ce par ta sagesse que l'épervier prend son vol, et qu'il étend ses ailes vers le midi ?
\VS{30}Est-ce par ton commandement que l'aigle s'élève, et qu'il place son nid sur les hauteurs\FTNT{Jé. 49:16 ; Abd. 1:4.} ?
\VS{31}Elle habite sur les rochers, et elle s'y tient ; même sur les sommets des rochers et dans des lieux forts. 
\VS{32}De là il découvre le gibier, ses yeux voient de loin.
\VS{33}Ses petits auss sucent le sang ; et là où sont des cadavres, il s'y trouve aussitôt\FTNT{Mt. 24:28 ; Lu. 17:37.}.
\TextTitle{Yahweh lui pose une question}
\VS{34}Yahweh prit encore la parole et dit à Job :
\VS{35}Celui qui conteste avec le Tout-puissant, lui apprendra-t-il quelque chose ? Que celui qui dispute avec Dieu réponde à ceci.
\TextTitle{Réponse de Job}
\VS{36}Alors Job répondit à Yahweh et dit :
\VS{37}Voici, je suis un homme vil ; que te répondrais-je ? Je mets ma main sur ma bouche\FTNT{Ps. 39:10.}.
\VS{38}J'ai parlé une fois, mais je ne répondrai plus ; j'ai même parlé deux fois mais je n'ajouterai plus.
\Chap{40}
\TextTitle{Yahweh questionne encore Job}
\VerseOne{}Et Yahweh répondit à Job du milieu d'un tourbillon, et lui dit :
\VS{2}Ceins maintenant tes reins comme un vaillant homme ; je t'interrogerai et tu m'enseigneras.
\VS{3}Anéantiras-tu mon jugement ? me condamneras-tu pour te justifier\FTNT{Ps. 51:6 ; Ro. 3:4.} ?
\VS{4}As-tu un bras comme celui de Dieu ; tonnes-tu de la voix, comme lui ?
\VS{5}Pare-toi maintenant de magnificence et de grandeur, et revêts-toi de majesté et de gloire.
\VS{6}Répands les fureurs de ta colère, d'un regard, humilie tous les orgueilleux
\VS{7}D'un regard humilie les orgueilleux, écrase sur place les méchants,
\VS{8}cache-les tous ensemble dans la poussière, enferme leur face dans les ténèbres !
\VS{9}Alors je rends hommage à mon sauveur qui me sauve par sa droite.
\VS{10}Voici le béhémoth, que j'ai façonné comme toi ! Il mange de l'herbe comme le bœuf.
\VS{11}Regarde donc, sa force est dans ses reins, et sa puissance dans les muscles de son ventre ;
\VS{12}il plie sa queue aussi ferme qu'un cèdre ; les tendons de ses cuisses sont entrelacés ;
\VS{13}ses os sont des tubes d'airain, ses membres sont comme des barres de fer.
\VS{14}C’est le chef-d’œuvre de Dieu ; celui qui l’a fait lui a donné son épée.
\VS{15}Car les montagnes lui apportent sa pâture, là où se jouent toutes les bêtes des champs.
\VS{16}Il se couche sous les lotus, caché dans les roseaux et les marécages ;
\VS{17}les lotus le couvrent de leur ombre, les saules du torrent l'enveloppent.
\VS{18}Voilà, il engloutit une rivière en buvant, et il ne s'en retire pas vite ; et il ne s'étonnerait pas quand le Jourdain se dégorgerait dans sa gueule. 
\VS{19}Il l'engloutit en le voyant, et son nez passe au travers des empêchements qu'il rencontre.  
\VS{20}Attireras-tu le léviathan à l'hameçon ? Saisiras-tu sa langue avec une corde ?
\VS{21}Mettras-tu un jonc dans ses narines ? Lui perceras-tu la mâchoire avec un crochet ?
\VS{22}Accumulera-t-il les supplications ? Te parlera-t-il d'une voix douce ?
\VS{23}Fera-t-il une alliance avec toi, pour te prendre pour toujours comme esclave ?
\VS{24}Joueras-tu avec lui comme avec un oiseau ? L'attacheras-tu pour amuser les jeunes filles ?
\VS{25}Les pêcheurs en trafiquent-ils ? Le partagent-ils entre les marchands ?
\VS{26}Couvriras-tu sa peau de dards, et sa tête de harpons ?
\VS{27}Mets ta main contre lui, et tu ne te souviendras plus de l'attaquer.
\VS{28}Voici, on est trompé dans son attente ; à sa vue n'est-on pas terrassé ?
\Chap{41}
\VerseOne{}Nul n'est assez féroce pour l'exciter ; qui donc me résisterait en face ?
\VS{2}De qui suis-je le débiteur ? Je le paierai. Sous le ciel tout m'appartient\FTNT{Ex. 19:5 ; De. 10:14 ; Ps. 24:1 ; Ps. 50:12 ; 1 Co. 10:26 ; Ro. 11:35.}.
\VS{3}Je veux encore parler de ses discours, et de sa force, et de la beauté de sa structure.
\VS{4}Qui découvrira son vêtement devant ma face ? Qui viendra freiner ses mâchoires par un mors ?
\VS{5}Qui ouvrira les portes devant sa face ? Autour du lion habite la terreur.
\VS{6}Ses magnifiques et puissants boucliers sont fermés comme un sceau ;
\VS{7}ils se serrent l'un contre l'autre, et l'air n'entrerait pas entre eux ;
\VS{8}ce sont des frères qui s'embrassent, se saisissent, demeurent inséparables.
\VS{9}Ses éternuements font briller la lumière, ses yeux sont comme les paupières de l'aurore.
\VS{10}Des flammes viennent de sa bouche, des étincelles de feu s'en échappent.
\VS{11}Une fumée sort de ses narines, comme d'un chaudron qui bout, d'une chaudière ardente.
\VS{12}Son souffle allume les charbons, de sa bouche sort la flamme.
\VS{13}La force a son cou pour demeure, et l'effroi bondit devant lui.
\VS{14}Ses parties charnues sont jointes ensemble, fondues sur lui, inébranlables.
\VS{15}Son cœur est dur comme la pierre, dur comme la meule inférieure.
\VS{16}Quand il se lève, les plus vaillants ont peur, et l'épouvante les fait quitter le droit chemin.
\VS{17}C'est en vain qu'on l'attaque avec l'épée ; la lance, le javelot, la cuirasse ne servent à rien.
\VS{18}Il regarde le fer comme de la paille, l'airain comme du bois pourri.
\VS{19}La flèche de l'arc ne le met pas en fuite, les pierres de la fronde sont pour lui changés en chaume.
\VS{20}Il ne voit dans la massue qu'un brin de paille, il rit au sifflement des dards.
\VS{21}Sous son ventre sont des pointes aiguës : On dirait une herse qu'il étend sur la boue.
\VS{22}Il fait bouillir les profondeurs de la mer comme une chaudière, il la traite comme un vase rempli de parfums.
\VS{23}Il laisse après lui un sentier lumineux ; l'abîme prend la chevelure d'un vieillard.
\VS{24}Sur la terre nul n'est son maître ; il a été façonné pour ne rien craindre.
\VS{25}Il regarde avec dédain tout ce qui est élevé, il est le roi des plus fiers animaux.
\Chap{42}
\TextTitle{Job reconnaît la souveraineté de Dieu et s'humilie}
\VerseOne{}Job répondit à Yahweh et dit :
\VS{2}Je sais que tu peux tout, et qu'on ne saurait t'empêcher de faire ce que tu penses.
\VS{3}Quel est celui qui a la folie d'obscurcir mes conseils ? Oui, j'ai parlé sans les comprendre, de merveilles qui me dépassent et que je ne connais pas\FTNT{Ps. 40:6 ; Ps. 131:1 ; Ps. 139:6.}.
\VS{4}Écoute-moi maintenant, et je parlerai ; je t'interrogerai et tu m'instruiras.
\VS{5}J'avais entendu parler de toi ; mais maintenant mon œil t'a vu.
\VS{6}C'est pourquoi je me condamne et je me repens d'avoir ainsi parlé et je m'en sur la poussière et sur la cendre.
\VS{7}Après que Yahweh eut ainsi parlé à Job, il dit à Eliphaz de Théman : Ma colère est embrasée contre toi et contre tes deux amis, parce que vous n'avez pas parlé de moi avec droiture comme Job, mon serviteur.
\VS{8} C'est pourquoi prenez maintenant sept taureaux et sept béliers, allez auprès de mon serviteur Job, et offrez un holocauste pour vous. Job, mon serviteur, priera pour vous, et certainement j'exaucerai sa prière, afin que je ne vous traite pas selon votre folie ; car vous n'avez pas parlé de moi avec droiture, comme mon serviteur Job.
\VS{9} Ainsi Eliphaz de Théman, Bildad de Schuach, et Tsophar de Naama allèrent et firent comme Yahweh leur avait commandé ; et Yahweh exauça la prière de Job.
\VS{10}Yahweh rétablit Job de sa captivité, quand il eut prié pour ses amis ; et Yahweh lui ajouta le double de tout ce qu'il avait possédé.
\VS{11}Ses frères, ses sœurs, et tous ceux qui l'avaient connu auparavant vinrent tous le visiter, et ils mangèrent avec lui dans sa maison. Ils compatirent et le consolèrent au sujet de tout le mal que Yahweh avait fait venir sur lui, et chacun lui donna une kesita et un anneau d'or.
\VS{12}Pendant ses dernières années, Job reçut de Yahweh plus de bénédictions qu'il n'en avait reçu dans les premières. Il posséda quatorze mille brebis, six mille chameaux, mille paires de bœufs, et mille ânesses.
\VS{13}Il eut aussi sept fils et trois filles :
\VS{14}Il donna à la première le nom de Jemima, à la seconde celui de Ketsia, à la troisième celui de Kéren-Happuc.
\VS{15}Et il ne se trouvait pas de femmes aussi belles que les filles de Job dans tout le pays. Leur père leur donna une part de l'héritage parmi leurs frères.
\VS{16}Job vécut, après ces choses, cent quarante ans, et il vit ses fils et les fils de ses fils jusqu'à la quatrième génération.
\VS{1} Job mourut âgé et rassasié de jours.
\PPE{}
\end{multicols}

%\clearpage\ShortTitle{Cantique des cantiques}\BookTitle{Cantique des cantiques}\BFont
\begin{multicols}{2}
\TextTitle{[La fiancée et le fiancé exprime leur amour mutuel.]}
\Chap{1}
\VerseOne{}Le Cantique des cantiques, de Salomon.
\VS{2}[La Sulamithe :] Qu'il me baise des baisers de sa bouche ! [Les filles de Jérusalem :] Car tes amours sont plus agréables que le vin,
\VS{3}à cause de l'odeur de tes excellents parfums, ton nom est comme un parfum qui se répand ; c'est pourquoi les filles t'aiment.
\VS{4}[La Sulamithe :] Entraine-moi après toi ! [Les filles de Jérusalem :] Nous courrons ! [La Sulamithe :] Le roi m'introduit dans ses appartements. [Les filles de Jérusalem :] Nous nous égaierons et nous nous réjouirons à cause de toi ; nous célébrerons ton amour plus que le vin. C’est avec raison que l’on t’aime.
\VS{5}[La Sulamithe :] Ô filles de Jérusalem, je suis noire, je suis belle. Je suis comme les tentes de Kédar, et comme les pavillons de Salomon.
\VS{6}Ne prenez pas garde à moi, de ce que je suis noire : Le soleil m'a brûlée. Les fils de ma mère se sont mis en colère contre moi, ils m'ont faite gardienne des vignes. Ma vigne à moi, je ne l’ai pas gardée.
\VS{7}Dis-moi, toi que mon âme aime, où tu fais paître ton troupeau, et où tu les fais reposer à midi ; car pourquoi serais-je comme une femme errante près des troupeaux de tes compagnons ?
\VS{8}[Salomon :] Si tu ne le sais pas, ô la plus belle des femmes, sors sur les traces du troupeau, et fais paître tes chevreaux près des demeures des bergers.
\VS{9}Ma grande amie, je te compare au plus beau couple de chevaux que j'ai aux chars de Pharaon.
\VS{10}Tes joues sont belles au milieu de tes colliers, et ton cou est beau au milieu des rangées de perles.
\VS{11}[Les filles de Jérusalem :] Nous te ferons des colliers d'or, avec des boutons d'argent.
\VS{12}[La Sulamithe :] Tandis que le roi est assis à table, mon nard répand son parfum.
\VS{13}Mon bien-aimé est pour moi comme un bouquet de myrrhe, il passe la nuit entre mes seins.
\VS{14}Mon bien-aimé m'est comme une grappe de troëne dans les vignes d'En-Guédi.
\VS{15}[Salomon :] Que tu es belle, ma grande amie, que tu es belle ! Tes yeux sont comme ceux des colombes.
\VS{16}[La Sulamithe :] Que tu es beau, mon bien-aimé, que tu es aimable ! Aussi, notre lit est verdoyant.
\VS{17}Les poutres de nos maisons sont faites de cèdre, et nos chevrons de cyprès.
\TextTitle{[l'amour entre la fiancée et le fiancé]}
\Chap{2}
\VerseOne{}[La Sulamithe :] Je suis la rose de Saron, et le lis des vallées.
\VS{2}[Salomon :] Comme un lis au milieu des épines (1), telle est ma grande amie parmi les filles.
\VS{3}[La Sulamithe :] Comme un pommier au milieu des arbres de la forêt, tel est mon bien-aimé parmi les jeunes hommes. J'ai désiré m’asseoir à son ombre, son fruit est doux à mon palais.
\VS{4}Il m'a fait entrer dans la salle du festin (2) ; et la bannière qu’il déploie sur moi c’est l’amour (3).
\VS{5}Soutenez-moi avec des gâteaux de raisins, fortifiez-moi avec des pommes ; car je suis malade d'amour.
\VS{6}Que sa main gauche soit sous ma tête et que sa droite m'embrasse !
\VS{7}[Salomon :] Filles de Jérusalem, je vous en conjure, par les gazelles et par les biches des champs, ne réveillez pas celle que j'aime, ne la réveillez pas jusqu'à ce qu'elle le veuille.
\TextTitle{[La Sulamithe parle de Salomon]}
\VS{8}[La Sulamithe :] C'est la voix de mon bien-aimé ! Le voici, il vient, sautant sur les montagnes et bondissant sur les collines.
\VS{9}Mon bien-aimé est semblable à la gazelle ou aux faons des biches. Le voici qui se tient derrière notre muraille, il regarde par les fenêtres, il se fait voir par les treillis.
\VS{10}[La Sulamithe rapporte les paroles de Salomon :] Mon bien-aimé parle et me dit : Lève-toi, ma grande amie, ma belle, et viens !
\VS{11}Car voici, l'hiver est passé ; la pluie a cessé, elle s'en est allée.
\VS{12}Les fleurs paraissent sur la terre, le temps des chansons est venu, et la voix de la tourterelle se fait déjà entendre dans notre contrée.
\VS{13}Le figuier produit ses premiers fruits, et les vignes en fleurs exhalent le parfum. Lève-toi, ma grande amie, ma belle, et viens !
\VS{14}Ma colombe qui te tiens dans les fentes du rocher (4), dans les lieux secrets (5) et escarpés, fais-moi voir ta figure, fais-moi entendre ta voix ; car ta voix est douce, et ta figure est belle.
\VS{15}Prenez-nous les renards, et les petits renards qui ravagent les vignes, car nos vignes produisent des grappes.
\VS{16}[La Sulamithe :] Mon bien-aimé est à moi, et je suis à lui ; il fait paître son troupeau parmi les lis.
\VS{17}Avant que le jour se rafraîchisse et que les ombres s'enfuient, reviens mon bien-aimé ! Sois comme la gazelle ou les faons des biches, sur les montagnes qui nous séparent.
\TextTitle{[La fiancée recherche son fiancé et le trouve.]}
\Chap{3}
\VerseOne{}[La Sulamithe :] J'ai cherché pendant les nuits, sur ma couche, celui que mon âme aime ; je l'ai cherché, mais je ne l'ai point trouvé.
\VS{2}Je me lèverai maintenant, et je ferai le tour de la ville, des carrefours et des places ; je chercherai celui que mon âme aime. Je l'ai cherché, mais je ne l'ai point trouvé.
\VS{3}Les gardes qui font la ronde dans la ville m'ont rencontrée : N'avez-vous pas vu, leur ai-je dit, celui que mon âme aime ?
\VS{4}A peine les avais-je passés, que j’ai trouvé celui que mon âme aime ; je l’ai saisi et je ne l’ai point lâché jusqu’à ce que je l'aie amené dans la maison de ma mère, dans la chambre de celle qui m'a conçue.
\VS{5}[Salomon :] Filles de Jérusalem, je vous en conjure par les gazelles et par les biches des champs, ne réveillez pas celle que j'aime, ne la réveillez pas, jusqu'à ce qu'elle le veuille.
\TextTitle{[Salomon entre avec sa fiancée dans Sion]}
\VS{6}[La Sulamithe :] Qui est celle qui monte du désert, comme des colonnes de fumée en forme de palmiers, parfumée de myrrhe et d'encens, et de tous les aromates du parfumeur ?
\VS{7}[Un officier des gardes du roi Salomon :] Voici la litière de Salomon, autour de laquelle il y a soixante vaillants hommes, des plus vaillants d'Israël.
\VS{8}Tous sont armés de l'épée et sont très bien exercés à la guerre ; chacun porte son épée sur sa hanche, à cause des frayeurs de la nuit.
\VS{9}Le roi Salomon s'est fait une litière de bois du Liban.
\VS{10}Il en a fait les piliers d’argent, le dossier d’or, le siège de pourpre ; et au milieu, il a placé un tissu que les filles de Jérusalem aiment.
\VS{11}[Les filles de Jérusalem :] Sortez, filles de Sion, et regardez le roi Salomon avec la couronne dont sa mère l'a couronné le jour de ses fiançailles, le jour de la joie de son cœur.
\TextTitle{[Le fiancé déclare son amour]}
\Chap{4}
\VerseOne{}[Salomon :] Que tu es belle, ma grande amie, que tu es belle ! Tes yeux sont comme ceux des colombes derrière ton voile. Tes cheveux sont comme le poil d'un troupeau de chèvres suspendu aux flancs de la montagne de Galaad.
\VS{2}Tes dents sont comme un troupeau de brebis tondues qui remontent de l’abreuvoir ; toutes sont des jumelles, il n’y en a pas une qui manque.
\VS{3}Tes lèvres sont comme un fil teint d’écarlate, et ta bouche est charmante ; ta joue est comme une moitié de grenade, derrière ton voile.
\VS{4}Ton cou est comme la tour de David, bâtie pour être un arsenal ; mille boucliers y sont suspendus, tous les grands boucliers des héros.
\VS{5}Tes deux seins sont comme deux faons, comme les jumeaux d'une gazelle qui paissent au milieu des lis.
\VS{6}Avant que le jour se rafraîchisse et que les ombres s'enfuient, je m'en irai à la montagne de myrrhe, et à la colline de l'encens.
\VS{7}Tu es toute belle, ma grande amie, et il n'y a point de défaut (1) en toi.
\VS{8}Viens du Liban avec moi, mon épouse, viens du Liban avec moi ! Regarde du sommet de l'Amana, du sommet du Senir et de l’Hermon, des repaires des lions et des montagnes des léopards.
\VS{9}Tu me ravis le cœur, ma sœur, mon épouse, tu me ravis le cœur, par l'un de tes regards et par l'un des colliers de ton cou.
\VS{10}Que de charmes dans ton amour, ma sœur, mon épouse ! Comme ton amour vaut mieux que le vin, et combien l'odeur de tes parfums exhale plus que tous les aromates !
\VS{11}Tes lèvres, mon épouse, distillent des rayons de miel ; le miel et le lait sont sous ta langue, et l'odeur de tes vêtements est comme l'odeur du Liban.
\VS{12}Ma sœur, mon épouse, tu es un jardin fermé, une source fermée, une fontaine scellée (1).
\VS{13}Tes jets forment un jardin, où sont des grenadiers, avec les fruits les plus excellents, les troënes avec le nard.
\VS{14}Le nard et le safran, le roseau aromatique et le cinnamome, avec tous les arbres qui donnent l'encens ; la myrrhe et l'aloès, avec tous les excellents aromates.
\VS{15}Ô fontaine des jardins ! Ô source d'eau vive ! Ruisseaux coulant du Liban !
\VS{16}Lève-toi, nord ! Et viens, vent du midi ! Soufflez sur mon jardin afin que ses parfums s’en exhalent ! [La Sulamithe :] Que mon bien-aimé entre dans son jardin et qu'il mange de ses fruits délicieux.
\Chap{5}
\VerseOne{}[Salomon :] J’entre dans mon jardin, ma sœur, mon épouse ; je cueille ma myrrhe avec mes aromates, je mange mes rayons de miel ; avec mon miel, je bois mon vin avec mon lait. Mes amis, mangez, buvez, enivrez-vous d’amour, mes bien-aimés !
\TextTitle{[La Sulamithe raconte son rêve]}
\VS{2}[La Sulamithe :] J'étais endormie, mais mon cœur veillait (1), c’est la voix de mon bien-aimé qui frappe, en disant : [La Sulamithe rapporte les propos de Salomon :] Ouvre-moi, ma sœur, ma grande amie, ma colombe, ma parfaite ! Car ma tête est pleine de rosée et mes cheveux des gouttes de la nuit.
\VS{3}[La Sulamithe :] J'ai ôté ma tunique, lui dis-je, comment la remettrais-je ? J'ai lavé mes pieds, comment les salirais-je ?
\VS{4}Mon bien-aimé a passé la main par la fenêtre, et mes entrailles se sont émues pour lui.
\VS{5}Je me suis levée pour ouvrir à mon bien-aimé, et la myrrhe a ruisselé de mes mains, et la myrrhe s’est répandue de mes doigts sur la poignée du verrou.
\VS{6}J'ai ouvert à mon bien-aimé, mais mon bien-aimé s'en était allé, il avait disparu ; mon âme était hors de moi quand il me parlait. Je l’ai cherché, mais je ne l’ai point trouvé ; je l'ai appelé, mais il ne m’a point répondu.
\VS{7}Les gardes qui font la ronde dans la ville m’ont rencontrée ; ils m’ont battue, ils m’ont blessée ; les gardes des murailles m'ont enlevé mon voile.
\VS{8}Filles de Jérusalem, je vous en conjure, si vous trouvez mon bien-aimé, que lui direz-vous ? Que je suis malade d’amour.
\VS{9}[Les filles de Jérusalem :] Qu'a ton bien-aimé de plus qu’un autre, ô la plus belle des femmes ? Qu'a ton bien-aimé de plus qu’un autre pour que tu nous conjures ainsi ?
\TextTitle{[La Sulamithe décrit Salomon]}
\VS{10}[La Sulamithe :] Mon bien-aimé est blanc et vermeil, il se distingue entre dix mille.
\VS{11}Sa tête est de l’or pur, ses cheveux sont bouclés et flottants, noirs comme le corbeau.
\VS{12}Ses yeux sont comme ceux des colombes au bord des ruisseaux, lavés dans du lait, reposant au sein de la plénitude.
\VS{13}Ses joues sont comme un parterre d’aromates, comme des fleurs parfumées ; ses lèvres sont comme des lis d’où découle la myrrhe.
\VS{14}Ses mains sont comme des anneaux d'or, garnis de chrysolithes ; son ventre est comme de l’ivoire bien poli, couvert de saphirs.
\VS{15}Ses jambes sont comme des piliers de marbre, posés sur des bases d’or fin. Son aspect est comme le Liban, il est précieux comme les cèdres.
\VS{16}Son palais n'est que douceur, et toute sa personne est pleine de charme (2). Tel est mon bien-aimé, tel est mon ami, filles de Jérusalem !
\TextTitle{[Les filles de Jérusalem aident la Sulamithe à chercher Salomon]}
\Chap{6}
\VerseOne{}[Les filles de Jérusalem :] Où est allé ton bien-aimé, ô la plus belle des femmes ? De quel côté est allé ton bien-aimé ? Nous le chercherons avec toi.
\VS{2}[La Sulamithe :] Mon bien-aimé est descendu à son jardin, au parterre d’aromates, pour faire paître son troupeau dans les jardins, et pour cueillir des lis.
\VS{3}Je suis à mon bien-aimé et mon bien-aimé est à moi ; il fait paître son troupeau parmi les lis.
\TextTitle{[Salomon à la Sulamithe]}
\VS{4}[Salomon :] Tu es belle, grande amie, comme Thirtsa ; agréable comme Jérusalem, redoutable comme des troupes sous leurs bannières.
\VS{5}Détourne de moi tes yeux, car ils me troublent. Tes cheveux sont comme un troupeau de chèvres suspendus aux flancs de Galaad.
\VS{6}Tes dents sont comme un troupeau de brebis qui remontent de l’abreuvoir ; toutes portent des jumeaux, et aucune d'elles n'est stérile.
\VS{7}Ta joue est comme une moitié de grenade, derrière ton voile.
\VS{8}Il y a soixante reines, quatre-vingts concubines, et des vierges sans nombre.
\VS{9}Une seule est ma colombe, ma parfaite ; elle est l’unique de sa mère, celle qui l'a enfantée. Les filles la voient et la disent heureuse ; les reines et les concubines la louent en disant :
\VS{10}Qui est celle qui apparaît comme l’aurore, belle comme la lune, brillante comme le soleil, redoutable comme des troupes sous leurs bannières ?
\VS{11}[La Sulamithe :] Je suis descendu au jardin des noyers pour voir les fruits de la vallée qui mûrissent, et pour voir si la vigne pousse, et si les grenadiers fleurissent.
\VS{12}Je ne me suis pas aperçue que mon affection m'a rendue semblable aux chars d’Amminadib (1).
\TextTitle{[Description de la beauté de la Sulamithe.]}
\Chap{7}
\VerseOne{}[Les filles de Jérusalem :] Reviens, reviens, ô Sulamithe ! Reviens, reviens, afin que nous te contemplions. [La Sulamithe :] Qu’avez-vous à contempler la Sulamithe comme une danse de deux armées ?
\VS{2}[Les filles de Jérusalem :] Que tes pieds sont beaux dans tes chaussures, fille de prince ! Les contours de ta hanche sont comme des colliers, œuvres des mains d'un excellent artisan.
\VS{3}Ton sein est comme une coupe arrondie, pleine d’un vin aromatisé, ton ventre est comme un tas de blé entouré de lis.
\VS{4}Tes deux seins sont comme deux faons, comme les jumeaux d'une gazelle.
\VS{5}Ton cou est comme une tour d'ivoire ; tes yeux sont comme les étangs de Hesbon, près de la porte de Bath-Rabbim ; ton nez est comme la tour du Liban qui regarde vers Damas.
\VS{6}Ta tête est élevée comme le Carmel, et les cheveux fins de ta tête sont comme le pourpre ; un roi est enchaîné par tes boucles pour te contempler.
\VS{7}[Salomon :] Que tu es belle, que tu es agréable, ô mon amour, au milieu des délices !
\VS{8}Ta taille est semblable à un palmier, et tes seins à des grappes.
\VS{9}Je me dis : Je monterai sur le palmier, je prendrai possession de ses rameaux ! Que tes seins soient comme les grappes de la vigne, l’odeur de tes narines comme celle des pommes,
\VS{10}et ta bouche comme un vin excellent… [La Sulamithe :] …qui coule droitement pour mon bien-aimé, et glisse sur les lèvres de ceux qui s’endorment.
\VS{11}Je suis à mon bien-aimé et ses désirs se portent vers moi.
\VS{12}Viens, mon bien-aimé, sortons dans les champs, passons la nuit dans les villages !
\VS{13}Levons-nous dès le matin pour aller aux vignes, nous verrons si la vigne pousse, si la fleur s’ouvre, si les grenadiers fleurissent. Là je te donnerai mes amours.
\VS{14}Les mandragores répandent leur parfum, et à nos portes il y a toutes sortes de fruits exquis, des fruits nouveaux, et des fruits anciens : Mon bien-aimé, je les ai gardés pour toi.
\Chap{8}
\VerseOne{}[La Sulamithe :] Oh ! Que n’es-tu pour moi comme un frère, allaité des seins de ma mère ! Je te rencontrerais dehors, je t’embrasserais, et on ne me mépriserait pas.
\VS{2}Je te conduirais, je t’introduirais dans la maison de ma mère ; tu m’instruiras, et je te ferai boire du vin parfumé d'aromates et du moût de mon grenadier.
\VS{3}Que sa main gauche soit sous ma tête, et que sa droite m'embrasse !
\VS{4}[La Sulamithe (citant Salomon) :] Je vous en conjure, filles de Jérusalem, ne réveillez pas celle que j'aime, ne la réveillez pas, jusqu'à ce qu'elle le veuille.
\VS{5}[Les frères de la Sulamithe :] Qui est celle qui monte du désert, mollement appuyée sur son bien-aimé ? [Salomon :] Je t'ai réveillée sous le pommier ; là où ta mère t'a enfantée, là où celle qui t'a conçue t'a donné le jour.
\VS{6}Mets-moi comme un sceau sur ton cœur (1), comme un sceau sur ton bras ; car l'amour est fort comme la mort, et la jalousie est cruelle comme le scheol ; leurs ardeurs sont des ardeurs de feu, une flamme de Yahweh.
\VS{7}[La Sulamithe (à Salomon) :] Les grandes eaux ne peuvent éteindre l’amour, même les fleuves ne pourraient le submerger ; quand un homme donnerait toutes les richesses de sa maison contre l’amour, il ne s’attirerait qu’un profond mépris.
\VS{8}[La Sulamithe (racontant ce que ses frères lui ont dit) :] Nous avons une petite sœur qui n'a pas encore de seins ; que ferons-nous à notre sœur le jour où on parlera d’elle ?
\VS{9}Si elle est comme une muraille, nous bâtirons sur elle un palais d'argent ; si elle est une porte, nous la renforcerons avec une planche de cèdre.
\VS{10}[La Sulamithe :] Je suis comme une muraille, et mes seins sont comme des tours ; j'ai été à ses yeux comme celle qui trouve la paix.
\VS{11}Salomon avait une vigne à Baal-Hamon ; il remit la vigne à des gardiens ; chacun apportait pour son fruit mille pièces d'argent.
\VS{12}Ma vigne, qui est à moi, je la garde. Ô Salomon, que les mille pièces d'argent soient à toi, et deux cents pour les gardiens du fruit de la vigne !
\VS{13}[Les frères de la Sulamithe :] Ô toi qui habites dans les jardins ! Les amis sont attentifs à ta voix. [Salomon :] Daigne me la faire entendre !
\VS{14}[La Sulamithe :] Fuis, mon bien-aimé ! Sois semblable à la gazelle ou au faon des biches, sur les montagnes des aromates !
\PPE{}
\end{multicols}

%\clearpage\ShortTitle{Ruth}\BookTitle{Ruth}\BFont
\noindent\hrulefill
{\footnotesize
\textit{
\bigskip
{\centering{}
\\(Routh)
\\Signifie : Amitié, une amie
\\Thème : Les origines de la famille messianique
\\Auteur : Inconnu
\\Date de rédaction : 11ème siècle av. J.-C.\\}
}
%\bigskip
\textit{
\\Au temps des juges, tout le pays fut frappé par une famine qui poussa Elimélec, sa femme Naomi, et ses deux fils à s’installer dans le pays de Moab. Ce pays tire son nom de son fondateur Moab, né de l’inceste entre Lot et sa fille ainée.
%\bigskip
\\Ils y rencontrèrent Ruth qui devint ensuite la belle fille d’Elimélec. Après la mort de son époux, cette moabite démontra  son attachement non seulement à cette famille mais également au Dieu de cette famille qui devint aussi le sien.
%\bigskip
\\Au prix de sa détermination, son obéissance et son humilité, la destinée de cette femme fut complètement bouleversée. Image du rachat des nations, elle entra dans la lignée de Jésus-Christ homme.\bigskip
}
}
\par\nobreak\noindent\hrulefill
\begin{multicols}{2}
\Chap{1}
\TextTitle{[Famine en Juda]}
\VerseOne{}Au temps où les juges gouvernaient, il y eut une famine dans le pays. Un homme de Bethléhem de Juda s'en alla, avec sa femme et ses deux fils, pour séjourner sur la terre de Moab.
\TextTitle{[Séjour en Moab]}
\VS{2}Le nom de cet homme était Elimélec, celui de sa femme Naomi, et les noms de ses deux fils Machlon et Kiljon ; ils étaient Ephratiens, de Bethléhem de Juda. Entrés sur la terre de Moab, ils s’y établirent.
\VS{3}Elimélec, mari de Naomi, mourut, et elle resta avec ses deux fils.
\VS{4}Ils prirent pour eux des femmes Moabites, dont l'une se nommait Orpa, et la seconde Ruth\FTNT{Ruth, la Moabite, dont l’ancêtre était issu d’une relation incestueuse (Ge. 19:36-37), est devenue l’ancêtre du Messie (Mt. 1:5-6).}, et ils demeurèrent là environ dix ans.
\VS{5}Machlon et Kiljon moururent aussi tous les deux, et cette femme resta privée de ses deux fils et de son mari.
\TextTitle{[Retour en Juda]}
\VS{6}Puis elle se leva avec ses belles-filles afin de retourner de la terre de Moab, car elle entendit, au pays de Moab, que Yahweh avait visité son peuple en lui donnant du pain.
\VS{7}Elle sortit du lieu où elle habitait, avec ses deux belles-filles, et elle marcha pour revenir sur la terre de Juda.
\VS{8}Naomi dit à ses deux belles-filles : Allez, retournez chacune dans la maison de sa mère ! Que Yahweh use de bonté envers vous, comme vous l'avez fait envers ceux qui sont morts et envers moi !
\VS{9}Que Yahweh vous donne de trouver chacune du repos dans la maison d'un mari ! Et elle les embrassa. Elles levèrent leur voix, et pleurèrent ;
\VS{10}et elles lui dirent : Non, nous retournerons avec toi vers ton peuple.
\TextTitle{[Décision loyale de Ruth]}
\VS{11}Naomi dit : Retournez, mes filles ! Pourquoi viendriez-vous avec moi ? Ai-je encore dans mon sein des fils qui puissent devenir vos maris ?
\VS{12}Retournez, mes filles, allez ! Je suis trop vieille pour me remarier. Et quand je dirais : J'ai de l'espérance, quand cette nuit même je serais avec un mari, et que j'enfanterais des fils,
\VS{13}Attendriez-vous donc qu'ils aient grandi, refuseriez-vous donc des maris ? Non, mes filles ! Je suis dans une plus grande amertume que vous, car la main de Yahweh s'est éloignée de moi.
\VS{14}Et elles levèrent leur voix, et pleurèrent encore. Orpa embrassa sa belle-mère, mais Ruth s’attacha à elle.
\VS{15}Naomi dit à Ruth : Voici, ta belle-sœur est retournée vers son peuple et vers ses dieux ; retourne, après ta belle-sœur.
\VS{16}Ruth répondit : Ne me prie pas de te laisser, de me retourner et de ne pas te suivre ! Où tu iras, j'irai, où tu demeureras je demeurerai ; ton peuple sera mon peuple, et ton Dieu sera mon Dieu ;
\VS{17}Où tu mourras je mourrai, et j'y serai enterrée. Que Yahweh me traite dans toute sa rigueur, si autre chose que la mort vient à me séparer de toi !
\VS{18}Naomi, la voyant déterminée à aller avec elle, arrêta de lui parler.
\TextTitle{[Arrivée à Bethléhem]}
\VS{19}Elles marchèrent toutes deux jusqu'à ce qu'elles entrent à Bethléhem. Lorsqu'elles entrèrent dans Bethléhem, toute la ville fut agitée à cause d'elles, et les femmes disaient : Est-ce là Naomi ?
\VS{20}Elle leur dit : Ne m'appelez pas Naomi ; appelez-moi Mara, car le Tout-Puissant m'a remplie de beaucoup d'amertume.
\VS{21}J’étais dans l'abondance à mon départ, et Yahweh me ramène à vide. Pourquoi m'appelleriez-vous Naomi, après que Yahweh s'est prononcé contre moi, et que le Tout-Puissant m'a affligée ?
\VS{22}Ainsi revinrent de la terre de Moab Naomi et sa belle-fille, Ruth, la Moabite. Elles entrèrent dans Bethléhem au commencement de la moisson des orges.
\Chap{2}
\TextTitle{[Boaz félicite Ruth des soins désintéressés dont elle entoure Naomi]}
\VerseOne{}Naomi avait un parent de son mari. C'était un homme puissant et riche, de la famille d'Elimélec, et qui s’appelait Boaz.
\VS{2}Ruth la Moabite dit à Naomi : Je te prie laisse-moi aller glaner des épis dans le champ de celui aux yeux duquel je trouverai grâce. Elle lui dit : Va, ma fille.
\VS{3}Elle s'en alla et entra dans un champ, pour glaner après les moissonneurs. Et elle arriva par hasard sur une parcelle de champ qui appartenait à Boaz, qui était de la famille d'Elimélec.
\VS{4}Or voici, Boaz vint de Bethléhem, et il dit aux moissonneurs : Que Yahweh soit avec vous ! Ils lui dirent : Que Yahweh te bénisse !
\VS{5}Et Boaz dit à son serviteur qui était établi sur les moissonneurs : A qui est cette jeune fille ?
\VS{6}Le serviteur qui était établi sur les moissonneurs répondit et dit : C'est une jeune femme Moabite, qui est revenue avec Naomi de la terre de Moab.
\VS{7}Elle nous a dit : Permettez-moi de glaner et de recueillir des épis entre les gerbes, après les moissonneurs. Depuis ce matin qu'elle est venue, elle est restée jusqu'à présent, et s'est à peine assise dans la maison.
\VS{8}Boaz dit à Ruth : Ecoute, ma fille, ne va pas glaner dans un autre champ ; ne pars pas au loin, et reste avec mes servantes.
\VS{9}Regarde où l'on moissonne dans le champ, et va après elles. J'ai défendu à mes serviteurs de te toucher. Et si tu as soif, va prendre des vases, et bois de ce que les serviteurs auront puisé.
\VS{10}Alors elle tomba sur sa face, et se prosterna contre terre, et elle lui dit : Comment ai-je trouvé grâce à tes yeux, pour que tu prêtes attention à moi, moi qui suis une étrangère ?
\VS{11}Boaz lui répondit et dit : On m'a raconté tout ce que tu as fait pour ta belle-mère depuis que ton mari est mort, comment tu as laissé ton père, ta mère, et le pays de ta naissance, pour aller vers un peuple que tu ne connaissais pas auparavant.
\VS{12}Que Yahweh te récompense pour ton œuvre, et que ton salaire soit entier de la part de Yahweh le Dieu d'Israël, sous les ailes duquel tu es venue te réfugier !
\VS{13}Et elle dit : Mon seigneur, que je trouve grâce à tes yeux ! Car tu m'as consolée, et tu as parlé au cœur de ta servante. Et pourtant je ne suis pas, moi, comme l'une de tes servantes.
\VS{14}Au moment du repas, Boaz dit à Ruth : Approche-toi ici, mange du pain, et trempe ton morceau dans le vinaigre. Elle s'assit à côté des moissonneurs. On lui donna du grain rôti ; elle mangea et se rassasia, et elle garda le reste.
\VS{15}Puis elle se leva pour glaner. Boaz ordonna à ses serviteurs : Qu'elle glane même entre les gerbes, et ne lui faites pas honte.
\VS{16}Et vous retirerez même pour elle quelques poignées de gerbes, que vous lui laisserez glaner, sans la réprimander.
\VS{17}Elle glana donc dans le champ jusqu'au soir, et elle battit ce qu'elle avait glané. Il y eut environ un épha d'orge.
\VS{18}Elle l'emporta, entra dans la ville, et sa belle-mère vit ce qu'elle avait glané. Elle sortit aussi les restes de son repas, et les lui donna.
\VS{19}Sa belle-mère lui dit : Où as-tu glané aujourd'hui, et où as-tu travaillé ? Béni soit celui qui t'a reconnue ! Et Ruth raconta à sa belle-mère chez qui elle avait travaillé : L'homme chez qui j'ai travaillé aujourd'hui s’appelle Boaz.
\VS{20}Naomi dit à sa belle-fille : Qu'il soit béni de Yahweh, puisqu'il a la même bonté pour les vivants, comme il en eut pour ceux qui sont morts ! Cet homme est un proche parent, lui dit encore Naomi, il est un de ceux qui ont sur nous le droit de rachat\FTNT{Le droit de rachat : Le rédempteur est celui qui rachète une personne moyennant le paiement d'une rançon. Sous la première alliance, le rachat se faisait soit par un frère, soit par un proche parent, pour la libération de celui qui s’était fait esclave ou qui avait aliéné sa propriété ou son bien (Lé. 25:25 et 48). Sous la nouvelle alliance, Jésus-Christ est désormais notre rédempteur. Romains 3:23-24 nous dit : «~ Car tous ont péché, et sont entièrement privés de la gloire de Dieu. Et ils sont gratuitement justifiés par sa grâce, par la rédemption qui est en Jésus-Christ «~». Christ nous a rachetés de la malédiction de la loi en se donnant lui-même pour nous afin de nous délivrer de toute iniquité (Ga. 3:13 ; Ti. 2:14). Dieu s’est fait homme (Hé. 2:14-17) afin de mieux nous libérer de l'esclavage du diable par sa mort à la croix de Golgotha (Es. 60:16).}.
\VS{21}Ruth la Moabite dit : Il m'a même dit : Reste avec mes serviteurs jusqu'à ce qu'ils aient achevé toute ma moisson.
\VS{22}Et Naomi dit à Ruth, sa belle-fille : Ma fille, il est bon que tu sortes avec ses servantes, et qu'on ne te rencontre pas dans un autre champ.
\VS{23}Elle resta donc avec les servantes de Boaz, pour glaner, jusqu'à la fin de la moisson des orges et la moisson des froments. Et elle demeurait avec sa belle-mère.
\Chap{3}
\TextTitle{[Ruth dans l'obéissance de la foi]}
\VerseOne{}Naomi, sa belle-mère, lui dit : Ma fille, je voudrais chercher ton repos, afin que tu sois heureuse.
\VS{2}Maintenant Boaz, avec les servantes duquel tu as été, n'est-il pas de notre parenté ? Voici, il doit vanner cette nuit les orges qui ont été foulées dans l'aire.
\VS{3}Lave-toi et oins-toi, puis mets tes habits, et descends dans l'aire. Ne te fais pas connaître à lui, jusqu'à ce qu'il ait achevé de manger et de boire.
\VS{4}Quand il se couchera, découvre le lieu où il se couche. Ensuite, entre, découvre ses pieds, et couche-toi. Il te dira ce que tu as à faire.
\VS{5}Elle lui répondit : Je ferai tout ce que tu as dit.
\VS{6}Elle descendit à l'aire, et fit tout ce que sa belle-mère lui avait ordonné.
\VS{7}Boaz mangea et but, et son cœur était joyeux. Il vint se coucher à l'extrémité d'un tas de gerbes. Ruth vint secrètement, découvrit ses pieds, et se coucha.
\VS{8}Au milieu de la nuit, cet homme eut peur ; il se retourna et retira ses pieds, car voici, une femme était couchée à ses pieds.
\VS{9}Il dit : Qui es-tu ? Elle répondit : Je suis Ruth, ta servante ; étends le pan de ta robe sur ta servante, car tu as droit de rachat.
\VS{10}Et il dit : Ma fille, que Yahweh te bénisse ! Ce dernier trait de bonté me réjouit plus que le premier, car tu n'es pas allée après des jeunes gens, pauvres ou riches.
\VS{11}Maintenant, ma fille, ne crains pas ; je te ferai tout ce que tu me diras ; car toute la porte de mon peuple sait que tu es une femme vertueuse.
\VS{12}Il est bien vrai que j'ai droit de rachat, mais il existe un autre plus proche que moi, qui a le droit de rachat.
\VS{13}Passe ici la nuit, et demain, si cet homme veut user envers toi du droit de rachat, à la bonne heure, qu'il te rachète ; mais s'il ne lui plaît pas de te racheter, moi je te rachèterai, Yahweh est vivant ! Couche-toi jusqu'au matin.
\VS{14}Elle se coucha à ses pieds jusqu'au matin, et elle se leva avant qu'on puisse se reconnaître l'un l'autre. Boaz dit : Qu'on ne sache pas qu'une femme est entrée dans l'aire.
\VS{15}Et il dit : Donne-moi le manteau qui est sur toi, et tiens-le. Elle le tint, et il mesura six mesures d'orge, qu'il posa sur elle. Puis il entra dans la ville.
\VS{16}Ruth revint auprès de sa belle-mère, et Naomi dit : Est-ce toi ma fille ? Ruth lui raconta tout ce que cet homme avait fait pour elle.
\VS{17}Elle dit : Il m'a donné ces six mesures d'orge, en disant : Tu n'iras pas à vide vers ta belle-mère.
\VS{18}Et Naomi dit : Ma fille, assieds-toi ici jusqu'à ce que tu saches ce que l'affaire deviendra, car cet homme ne se donnera pas de repos, qu'il n'ait achevé cette affaire aujourd'hui.
\Chap{4}
\TextTitle{[Ruth comblée par le mariage]}
\VerseOne{}Boaz monta à la porte, et s'y assit. Or voici, celui qui avait le droit de rachat, et dont Boaz avait parlé, passa. Boaz lui dit : Ah ! Détourne-toi, reste ici, toi un tel. Et il se détourna, et s'assit.
\VS{2}Boaz prit dix hommes d'entre les anciens de la ville, et leur dit : Asseyez-vous ici. Et ils s'assirent.
\VS{3}Puis il dit à celui qui avait le droit de rachat : Naomi qui est revenue de la terre de Moab, a vendu la parcelle du champ qui appartenait à notre frère Elimélec.
\VS{4}J'ai parlé à tes oreilles afin de te le faire savoir et te le dire : Acquiers-la en la présence de ceux qui sont assis ici et en présence des anciens de mon peuple. Si tu veux racheter par droit de rachat, rachète-la ; mais si tu ne veux pas la racheter, déclare-le-moi, afin que je le sache. Car il n'y a pas d'autre que toi qui ait le droit de rachat, et je l'ai après toi. Et il dit : je rachèterai.
\VS{5}Boaz dit : Le jour où tu acquerras le champ de la main de Naomi, tu l'acquerras aussi de Ruth la Moabite, femme du défunt, pour maintenir le nom du défunt dans son héritage.
\VS{6}Et celui qui avait le droit de rachat dit : Je ne puis pas racheter pour mon compte, de peur de détruire mon héritage ; prends pour toi le droit de rachat, car je ne puis pas le racheter.
\VS{7}Autrefois en Israël, pour confirmer une affaire quelconque relative à un rachat ou à un échange, l'homme ôtait son soulier et le donnait à son parent : C'était là, en Israël, un témoignage qu'on cédait son droit.
\VS{8}Celui qui avait le droit de rachat dit à Boaz : Acquiers-le pour toi ! Et il ôta son soulier.
\VS{9}Alors Boaz dit aux anciens et à tout le peuple : Vous êtes aujourd'hui témoins que j'ai acquis de la main de Naomi tout ce qui appartenait à Elimélec, à Kiljon et à Machlon.
\VS{10}Et que je me suis également acquis pour femme Ruth la Moabite, femme de Machlon, pour maintenir le nom du défunt dans son héritage, et afin que le nom du défunt ne soit pas retranché d'entre ses frères et de la porte de sa ville. Vous en êtes témoins aujourd'hui !
\VS{11}Tout le peuple qui était à la porte et les anciens dirent : Nous en sommes témoins ! Que Yahweh rende la femme qui entre dans ta maison semblable à Rachel et à Léa, qui ont bâti toutes deux, la maison d'Israël ! Montre ta puissance dans Ephrata et proclame ton nom dans Bethléhem !
\VS{12}Puisse la postérité que Yahweh te donnera de cette jeune femme, rendre ta maison semblable à la maison de Pérets, que Tamar enfanta à Juda !
\VS{13}Boaz prit Ruth, qui devint sa femme, et il alla vers elle. Yahweh lui fit la grâce de concevoir, et elle enfanta un fils.
\VS{14}Les femmes dirent à Naomi : Béni soit Yahweh qui ne t'a pas laissé manquer aujourd'hui d'un homme, ayant droit de rachat, et dont le nom sera proclamé en Israël !
\VS{15}Cet enfant restaurera ton âme, et sera le soutien de ta vieillesse ; car ta belle-fille, qui t'aime, l'a enfanté, et elle vaut mieux que sept fils.
\VS{16}Naomi prit l'enfant et le posa sur son sein, et elle fut sa nourrice.
\TextTitle{[Le fils de Ruth sera le grand-père de David]}
\VS{17}Les voisines lui donnèrent un nom, en disant : Un fils est né à Naomi ! Et elles l'appelèrent du nom de Obed. Ce fut le père d'Isaï, père de David.
\VS{18}Voici la généalogie de Pérets. Pérets engendra Hetsron ;
\VS{19}Hetsron engendra Ram ; Ram engendra Amminadab ;
\VS{20}Amminadab engendra Nachschon ; Nachschon engendra Salmon ;
\VS{21}Salmon engendra Boaz ; Boaz engendra Obed ;
\VS{22}Obed engendra Isaï, et Isaï engendra David.
\PPE{}
\end{multicols}

%\clearpage\ShortTitle{Lamentations de Jérémie}\BookTitle{Lamentations de Jérémie}\BFont
\noindent\hrulefill
{\footnotesize
\textit{
\bigskip
{\centering{}
\\Auteur : Jérémie
\\(Heb. : Eikha)
\\Signification : Où ?
\\Thème : Affliction pour Jérusalem
\\Date de rédaction : 6\up{ème} siècle av. J.-C\\}
}
%\bigskip
\textit{
\\Recueil de pièces poétiques, les lamentations de Jérémie furent composées selon un procédé visant à accentuer le caractère funèbre, de façon à ce qu'elles soient récitées avec gémissements. Ses complaintes exposent la profonde désolation du prophète face au fardeau du peuple qu'il portait dans ses entrailles tout comme la douleur et la tristesse de Yahweh face à Israël.
%\bigskip
\\Très différentes des prophéties retrouvées dans le livre de Jérémie, les Lamentations reflètent l'affliction convenant à la
gravité du châtiment subi : famine, pillage et ruine du temple, déportation, cessation du culte, diverses calamités… Jérémie rappelle ainsi les conséquences de l'endurcissement du cœur face aux appels à la repentance ; il présente aussi les bontés éternelles de Yahweh.\bigskip
}
}
\par\nobreak\noindent\hrulefill
\begin{multicols}{2}
\Chap{1}
\TextTitle{Pleurs et désolation de Jérusalem}
\VerseOne{}[Aleph.] Comment est-il arrivé que la ville si peuplée se trouve si solitaire ? Que celle qui était grande entre les nations est devenue comme une veuve ? Que celle qui était noble dame entre les provinces a été rendue tributaire ?
\VS{2}[Beth.] Elle ne cesse de pleurer pendant la nuit, et ses larmes sont sur ses joues ; il n'y a pas un de tous ses amis qui la console ; ses intimes amis ont agi perfidement contre elle, ils sont devenus ses ennemis.
\VS{3}[Guimel.] Juda a été emmenée captive tant elle est affligée, et tant est grande sa servitude ; elle demeure maintenant entre les nations, et ne trouve point de repos ; tous ses persécuteurs l'ont attrapée dans sa détresse\FTNT{Jé. 52:26.}.
\VS{4}[Daleth.] Les chemins de Sion mènent deuil de ce qu'il n'y a plus personne qui vienne aux fêtes solennelles ; toutes ses portes sont désolées, ses sacrificateurs sanglotent, ses vierges sont accablées de tristesse ; elle est remplie d'amertume. 
\VS{5}[He.] Ses adversaires sont établis pour chefs, ses ennemis prospèrent ; car Yahweh l'a humiliée à cause de la multitude de ses transgressions ; ses petits enfants ont marché captifs devant l'adversaire\FTNT{Jé. 30:14.}.
\VS{6}[Vav.] Et tout l'honneur de la fille de Sion s'est retiré d'elle ; ses chefs sont devenus semblables à des cerfs qui ne trouvent pas de pâture, et qui fuient sans force devant celui qui les poursuit.
\VS{7}[Zayin.] Jérusalem dans les jours de son affliction et de son pauvre état s'est souvenue de toutes ses choses précieuses qu'elle avait depuis si longtemps, lorsque son peuple est tombé par la main de l'ennemi, sans aucun secours ; les ennemis l'ont vue, et se sont moqués de ses sabbats.
\VS{8}[Heth.] Jérusalem a grièvement péché ; c'est pourquoi elle est devenue un objet de dégoût ; tous ceux qui l'honoraient l'ont méprisée parce qu'ils ont vu son ignominie ; elle en a aussi sangloté, et s'est retournée en arrière.
\VS{9}[Teth.] Sa souillure était dans les pans de sa robe, et elle ne s'est pas souvenue de sa dernière fin ; elle a été extraordinairement abaissée, et elle n'a pas de consolateur. Vois ma misère, ô Yahweh ! Car l'ennemi s'est élevé avec orgueil !
\VS{10}[Yod.] L'ennemi a étendu sa main sur toutes ses choses désirables ; car elle a vu entrer dans son sanctuaire les nations au sujet desquelles tu avais donné cet ordre : Elles n'entreront point dans ton assemblée\FTNT{De. 23:3.}.
\VS{11}[Kaf.] Tout son peuple gémit, cherchant du pain\FTNT{Jé. 52:6.} ; ils ont donné leurs choses désirables pour des aliments, afin ranimer leur vie. Vois, ô Yahweh ! Regarde combien je suis méprisée.
\VS{12}[Lamed.] Cela ne vous touche-t-il point ? Vous tous passants, contemplez, et voyez s'il est une douleur comme ma douleur, celle dont j'ai été frappée ! Moi que Yahweh a accablée de douleur au jour de l'ardeur de sa colère.
\VS{13}[Mem.] Il a envoyé d'en haut, dans mes os, un feu qui les domine ; il a tendu un filet sous mes pieds, et m'a fait revenir en arrière ; il m'a mise dans la désolation, dans une langueur de tous les jours.
\VS{14}[Nun.] Le joug de mes iniquités est lié par sa main ; elles sont entrelacées, et appliquées sur mon cou ; il a renversé ma force ; le Seigneur m'a livrée entre les mains de ceux contre qui je ne pourrai pas me lever.
\VS{15}[Samech.] Le Seigneur a abattu tous les hommes forts que j'avais au milieu de moi ; il a appelé contre moi, au temps fixé, une armée pour détruire mes jeunes hommes ; le Seigneur a foulé au pressoir la vierge, fille de Juda.
\VS{16}[Ayin.] À cause de ces choses, je pleure, mes yeux fondent en larmes ; car le consolateur qui restaurait ma vie est loin de moi. Mes fils sont dans la désolation parce que l'ennemi a été plus fort.
\VS{17}[Pe.] Sion a étendu les mains, et personne ne l'a consolée ; Yahweh a ordonné aux ennemis de Jacob de l'entourer de toutes parts. Jérusalem a été comme une impureté au milieu d'eux.
\VS{18}[Tsade.] Yahweh est juste car j'ai été rebelle à ses ordres. Ecoutez, vous tous, peuples, et voyez ma douleur ! Mes vierges et mes jeunes hommes sont allés en captivité.
\VS{19}[Qof.] J'ai appelé mes amis, mais ils m'ont trompé. Mes sacrificateurs et mes anciens sont morts dans la ville : Ils cherchaient de la nourriture afin de restaurer leur vie.
\VS{20}[Resh.] Regarde Yahweh ! car je suis dans la détresse ; mes entrailles bouillonnent, mon coeur palpite au dedans de moi, parce que je n'ai fait qu'être rebelle ; au dehors l’épée m’a privée d’enfants ; au dedans il y a comme la mort. 
\VS{21}[Shin.] On m'a entendu sangloter et je n'ai personne qui me console ; tous mes ennemis ont appris mon malheur, et s’en sont réjouis, parce que tu l’as fait ; tu amèneras le jour que tu as assigné, et ils seront dans mon état.
\VS{22}[Tav.]Que toute leur méchanceté vienne devant toi, et traite-leur comme tu m'as traitée à cause de tous mes péchés ; car mes sanglots sont en grand nombre et mon coeur est languissant. 
\Chap{2}
\TextTitle{Le jour de la colère de Yahweh}
\VerseOne{}[Aleph.] Comment est-il arrivé que le Seigneur a couvert de sa colère la fille de Sion tout à l'entour, comme d'une nuée, et qu'il a précipité du ciel sur la terre la beauté d'Israël, et ne s'est pas souvenu du marchepied de ses pieds\FTNT{Ez. 43:7.} au jour de sa colère ?
\VS{2}[Beth.] Le Seigneur a englouti sans épargner toutes les habitations de Jacob ; il a dans sa fureur renversé les forteresses de la fille de Juda, il les a jetées par terre ; il a profané le royaume et ses chefs.
\VS{3}[Guimel.] Il a retranché toute la force d'Israël par l'ardeur de sa colère ; il a retiré sa droite en arrière devant l'ennemi ; il s'est allumé dans Jacob comme un feu flamboyant qui le consume de toutes parts.
\VS{4}[Daleth.] Il a tendu son arc comme un ennemi ; sa droite s'est dressée comme celle d'un adversaire ; il a tué tout ce qui était agréable à l'œil dans la tente de la fille de Sion ; il a répandu sa fureur comme un feu.
\VS{5}[He.] Le Seigneur a été comme un ennemi ; il a englouti Israël, il a englouti tous ses palais, il a détruit toutes ses forteresses ; il a multiplié chez la fille de Juda le deuil et les afflictions.
\VS{6}[Vav.] Il a mis en pièces avec violence sa tente comme un jardin ; il a détruit le lieu de son assemblée ; Yahweh a fait oublier dans Sion la fête solennelle et le sabbat, et dans sa violente colère, il a rejeté le roi et le sacrificateur.
\VS{7}[Zayin.] Le Seigneur a rejeté au loin son autel, il a dédaigné son sanctuaire ; il a livré entre les mains de l'ennemi les murailles de ses palais ; ils ont poussé des cris dans la maison de Yahweh, comme aux jours des fêtes solennelles.
\VS{8}[Heth.] Yahweh avait projeté de détruire les murailles de la fille de Sion ; il a étendu le cordeau, il n'a pas fait revenir sa main sans les avoir engloutis ; il a plongé dans le deuil remparts et murailles, ils ont été ruinés tous ensemble.
\VS{9}[Teth.] Ses portes sont enfoncées dans la terre ; il en a détruit et brisé les barres. Son roi et ses chefs sont parmi les nations ; la loi n'est plus. Même les prophètes ne reçoivent plus aucune vision de Yahweh\FTNT{Ez. 7:26.}.
\VS{10}[Yod.] Les anciens de la fille de Sion sont assis à terre, ils sont muets ; ils ont couvert leur tête de poussière, ils se sont ceints de sacs ; les vierges de Jérusalem baissent leurs têtes vers la terre.
\VS{11}[Kaf.] Mes yeux se consument à force de larmes, mes entrailles bouillonnent, ma bile se répand sur la terre. À cause des ruines de la fille de mon peuple, des enfants et des nourrissons qui tombent en défaillance dans les rues de la ville.
\VS{12}[Lamed.] Ils disaient à leurs mères : Où y a-t-il du blé et du vin ? Et ils tombaient comme morts dans les rues de la ville, comme un homme blessé à mort, ils rendaient l'âme sur le sein de leurs mères.
\VS{13}[Mem] Qui dois-je prendre à témoin ? À qui te comparer, fille de Jérusalem ? Qui pourrait t'égaler, et quelle consolation te donner, vierge, fille de Sion ? Car ta ruine est grande comme une mer : Qui pourrait te guérir\FTNT{Es. 51:19-20.} ?
\VS{14}[Nun.] Tes prophètes ont eu pour toi des visions vaines et insensées ; ils n'ont pas découvert ton iniquité, afin de détourner ta captivité ; ils t'ont prophétisé des oracles mensongers et trompeurs\FTNT{Jé. 2:8 ; Jé. 5:31 ; Jé. 14:14.}.
\VS{15}[Samech.] Tous les passants applaudissent sur toi, ils sifflent, ils secouent leur tête contre la fille de Jérusalem : Est-ce ici la ville de laquelle on disait : La parfaite en beauté, la joie de toute la terre\FTNT{Na. 3:19.} ?
\VS{16}[Pe.] Tous tes ennemis ouvrent la bouche contre toi, ils sifflent, ils grincent des dents, ils disent : Nous l'avons engloutie ! C'est ici le jour que nous attendions, nous l'avons atteint, nous le voyons !
\VS{17}[Ayin.] Yahweh a fait ce qu'il avait projeté, il a accompli sa parole qu'il avait ordonnée depuis longtemps, il a détruit sans épargner, il a fait de toi la joie de l'ennemi, il a donné de la force à tes adversaires.
\VS{18}[Tsade.] Leur cœur crie au Seigneur… Muraille de la fille de Sion, fais couler des larmes jour et nuit, comme un torrent\FTNT{Jé. 14:17.} ! Ne te donne pas de repos ; et que la prunelle de tes yeux ne se repose pas !
\VS{19}[Qof.] Lève-toi, pousse des cris dès le commencement des veilles de la nuit ! Répands ton cœur comme de l'eau en présence du Seigneur ! Lève tes mains vers lui pour l'âme de tes enfants qui meurent de faim aux coins de toutes les rues !
\VS{20}[Resh.] Vois, ô Yahweh ! Regarde qui tu as traité avec sévérité ! Les femmes n'ont-elles pas mangé leur fruit : leurs petits enfants objets de leur tendresse ? Le sacrificateur et le prophète n'ont-ils pas été tués dans le sanctuaire du Seigneur\FTNT{Lé. 26:29 ; De. 28:53 ; Jé. 19:9.} ?
\VS{21}[Shin.] Les jeunes gens et les vieillards sont couchés par terre dans les rues ; mes vierges et mes jeunes hommes sont tombés par l'épée ; tu as tué au jour de ta colère, tu as massacré sans épargner.
\VS{22}[Tav.] Tu as convié comme pour un jour solennel mes frayeurs de toutes parts. Au jour de la colère de Yahweh, il n'y a eu ni réchappé ni survivant. Ceux que j'avais langés et élevés, mon ennemi les a consumés.
\Chap{3}
\TextTitle{Jérémie partage l'affliction des siens}
\VerseOne{}[Aleph.] Je suis l'homme qui a vu l'affliction par la verge de sa fureur\FTNT{Jé. 15:15-18.}.
\VS{2}Il m'a conduit, mené dans les ténèbres, et non dans la lumière.
\VS{3}Certes c'est contre moi qu'il a tout le jour tourné et retourné sa main.
\VS{4}[Beth.] Il a fait vieillir ma chair et ma peau, il a brisé mes os\FTNT{Es. 38:13.}.
\VS{5}Il a bâti autour de moi, il m'a environné de venin et de peine.
\VS{6}Il me fait habiter dans les lieux ténèbreux, comme ceux qui sont morts depuis longtemps.
\VS{7}[Guimel.] Il a fait une cloison autour de moi, afin que je ne sorte point ; il a appesanti mes chaînes.
\VS{8}Même quand je crie et que j'élève ma voix, il rejette ma prière.
\VS{9}Il a fait un mur de pierres de taille pour fermer mes chemins, il a renversé mes sentiers.
\VS{10}[Daleth.] Il a été pour moi un ours en embuscade, un lion qui se tient dans un lieu caché\FTNT{Os. 13:8.}.
\VS{11}Il a détourné mes chemins, il m'a mis en pièces, il m'a mis dans la désolation.
\VS{12}Il a tendu son arc, et il m'a placé comme une cible pour sa flèche.
\VS{13}[He.] Il a fait entrer dans mes reins les flèches de son carquois.
\VS{14}Je suis la risée pour tout mon peuple, et leur chanson\FTNT{Ps. 69:13 ; Job. 30:9.} tout le jour.
\VS{15}Il m'a rassasié d'amertume, il m'a enivré d'absinthe.
\VS{16}[Vav.] Il a brisé mes dents avec du gravier, il m'a couvert de cendres.
\VS{17}Tellement que la paix s'est éloignée de mon âme, j'ai oublié ce que c'est que d'être à son aise.
\VS{18}Et j'ai dit : Ma force est perdue, et mon espérance aussi que j'avais en Yahweh.
\VS{19}[Zayin.] Souviens-toi de mon affliction, et de mon pauvre état qui n'est qu'absinthe et que fiel ;
\VS{20}Mon âme s'en souvient sans cesse, et elle est abattue au-dedans de moi.
\VS{21}Mais je rappellerai ceci en mon coeur, et c'est pourquoi j'aurai de l'espérance :
\VS{22}[Heth.] C'est une grâce de Yahweh que nous n'avons point été consumés parce que ses compassions ne sont pas épuisées\FTNT{Ps. 103:10.} ;
\VS{23}elles se renouvellent chaque matin. C'est une chose grande que ta fidélité !
\VS{24}Yahweh est ma portion, dit mon âme ; c'est pourquoi j'aurai espérance en lui\FTNT{Ps. 16:5.}.
\VS{25}[Teth.] Yahweh est bon pour ceux qui s'attendent à lui, pour l'âme qui le cherche.
\VS{26}Il est bon d'espérer et d'attendre en silence la délivrance de Yahweh.
\VS{27}Il est bon pour l'homme de porter le joug dans sa jeunesse.
\VS{28}[Yod.] Il sera assis solitaire et silencieux parce qu'on le lui impose.
\VS{29}Il mettra sa bouche dans la poussière, peut-être y aura-t-il quelque espérance ?
\VS{30}Il présentera la joue à celui qui le frappe, il se rassasiera d'opprobres.
\VS{31}[Kaf.] Car le Seigneur ne rejette pas à toujours\FTNT{Es. 57:16 ; Ps. 77:8.}.
\VS{32}Mais s'il afflige quelqu'un, il a aussi compassion selon la grandeur de sa miséricorde.
\VS{33}Car ce n'est pas sa volonté d'affliger et d'humilier les fils des hommes.
\VS{34}[Lamed.] Lorsqu'on foule aux pieds tous les prisonniers de la terre,
\VS{35}lorsqu'on pervertit la justice humaine en la présence du Très-Haut,
\VS{36}lorsqu'on fait tort à quelqu'un dans son procès, le Seigneur ne le voit-il pas ?
\VS{37}[Mem.] Qui est-ce qui dit qu'une chose est arrivée sans que le Seigneur l'ait commandé ?
\VS{38}Les maux et les biens\FTNT{Es. 45:7 ; Am. 3:6 ; Job. 1:21.} ne procèdent-ils pas de la bouche du Très-Haut ?
\VS{39}Pourquoi un homme vivant se plaindrait-il, un homme, à cause de la peine de ses péchés ?
\TextTitle{Le peuple appelé à s'examiner pour revenir à Yahweh}
\VS{40}[Nun.] Recherchons nos voies, sondons-les, et retournons à Yahweh\FTNT{Ps. 119:59 ; 2 Co. 13:5.} ;
\VS{41}élevons nos cœurs et nos mains vers Dieu qui est au ciel :
\VS{42}Nous avons péché, nous avons été rebelles ! Tu n'as pas pardonné !
\VS{43}[Samech.] Tu nous as couverts de ta colère, et tu nous as poursuivis ; tu as tué sans épargner ;
\VS{44}tu t'es couvert d'une nuée pour que les prières ne te parviennent pas.
\VS{45}Tu nous as fait être la raclure et le rebut au milieu des peuples.
\VS{46}[Pe.] Tous nos ennemis ouvrent leur bouche contre nous.
\VS{47}La frayeur et la fosse, le dégât et la calamité nous sont arrivés\FTNT{Es. 24:18 ; Jé. 48:44.}.
\VS{48}De mes yeux coulent des torrents d'eau à cause de la ruine de la fille de mon peuple.
\VS{49}[Ayin.] Mon œil fond en larmes, sans repos, sans relâche,
\VS{50}jusqu'à ce que Yahweh regarde et voie des cieux\FTNT{Ps. 80:15 ; Ps. 102:20.} ;
\VS{51}mon œil fait souffrir mon âme à cause de toutes les filles de ma ville.
\TextTitle{Yahweh, le soutien de Jérémie dans la détresse}
\VS{52}[Tsade.] Ceux qui sont mes ennemis sans cause m'ont poursuivi à outrance, comme après un oiseau.
\VS{53}Ils ont voulu anéantir ma vie dans une fosse, et ils ont jeté une pierre sur moi.
\VS{54}Les eaux ont coulé par-dessus ma tête ; je disais : Je suis retranché !
\VS{55}[Qof.] J'ai invoqué ton nom, ô Yahweh, du fond de la fosse\FTNT{Jé. 38:6.}.
\VS{56}Tu as entendu ma voix : Ne ferme pas tes oreilles à mes soupirs, à mes cris !
\VS{57}Au jour où je t'ai invoqué, tu t'es approché, et tu as dit : Ne crains rien !
\VS{58}[Resh.] Ô Seigneur, tu as plaidé la cause de mon âme, tu as racheté ma vie.
\VS{59}Tu as vu, ô Yahweh ! le tort qu'on me fait, fais-moi justice !
\VS{60}Tu as vu toutes les vengeances dont ils ont usé, et toutes leurs machinations contre moi.
\VS{61}[Shin.] Yahweh, tu as entendu leurs outrages, toutes leurs machinations contre moi,
\VS{62}les discours de ceux qui se lèvent contre moi, et leur dessein qu'ils ont contre moi tout au long du jour.
\VS{63}Considère quand ils sont assis et quand ils se lèvent, car je suis leur chanson.
\VS{64}[Tav.] Rends-leur la pareille, ô Yahweh, selon l'œuvre de leurs mains ;
\VS{65}livre-les à l'endurcissement de leur cœur, à ta malédiction.
\VS{66}Poursuis-les dans ta colère, et extermine-les de dessous les cieux, ô Yahweh !
\Chap{4}
\TextTitle{Crimes et apostasie du peuple}
\VerseOne{}[Aleph.] Comment l'or est-il devenu obscur, et le fin or s'est-il altéré ? Comment les pierres du sanctuaire sont-elles répandues aux coins de toutes les rues ?
\VS{2}[Beth.] Comment les chers fils de Sion, qui étaient estimés à l'égal de l'or pur, sont-ils reputés comme des vases de terre, ouvrage des mains du potier !
\VS{3}[Guimel.] Il y a même des monstres marins qui présentent leurs mamelles et allaitent leurs petits ; mais la fille de mon peuple est devenue cruelle comme les autruches du désert.
\VS{4}[Daleth.] La langue de celui qui têtait s'est attachée à son palais dans sa soif ; les enfants demandent du pain, et personne ne leur en donne\FTNT{Jé. 52:6.}.
\VS{5}[He.] Ceux qui mangeaient des mets délicats sont en désolation dans les rues ; ceux qui étaient nourris sur l'étoffe écarlate embrassent le fumier.
\VS{6}[Vav.] L'iniquité de la fille de mon peuple est plus grande que le péché de Sodome, renversée en un instant, sans que personne n'ait tourné la main sur elle.
\VS{7}[Zayin.] Ses naziréens étaient plus purs que la neige, plus blancs que le lait ; leur teint était plus vermeil que les pierres précieuses ; ils étaient polis comme un saphir.
\VS{8}[Heth.] Leur apparence est plus sombre que le noir ; on ne les reconnaît pas dans les rues ; ils ont la peau collée sur les os ; elle est devenue sèche comme du bois\FTNT{Job. 30:30.}.
\VS{9}[Teth.]Ceux qui ont été mis à mort par l'épée, ont été plus heureux que ceux qui sont morts par la famine, qui eux sont consumés peu à peu, transpercés par le défaut du fruit des champs.
\VS{10}[Yod.] Les mains des femmes, naturellement tendres, font cuire leurs enfants ; ils leur servent de nourriture dans la ruine de la fille de mon peuple\FTNT{De. 28:57 ; 2 R. 6:29.}.
\VS{11}[Kaf.] Yahweh a accompli sa fureur, il a répandu l'ardeur de sa colère ; il a allumé dans Sion un feu qui en dévore les fondements.
\VS{12}[Lamed.] Les rois de la terre, et tous les habitants de la terre habitable n'auraient jamais cru que l'adversaire et l'ennemi entrerait dans les portes de Jérusalem.
\VS{13}[Mem.] Cela est arrivé à cause des péchés de ses prophètes, et des iniquités de ses sacrificateurs, qui répandaient le sang des justes au milieu d'elle\FTNT{Jé. 5:29-31. Le péché des conducteurs donne accès à l'ennemi pour les détruire ainsi que les biens qui leur ont été confiés (Lu. 11:21-22).}.
\VS{14}[Nun.] Ils erraient comme des aveugles dans les rues, souillés de sang, au point qu'on ne pouvait pas toucher leurs vêtements.
\VS{15}[Samech.] On leur criait : retirez-vous, souillés, retirez-vous, retirez-vous, ne nous touchez point. Quand ils se sont enfuis, ils ont erré ça et là ; on a dit parmi les nations : Ils n'auront plus leur demeure !
\VS{16}[Pe.] La face de Yahweh les a dispersés, il ne veut plus les regarder ; ils n'ont pas eu de respect pour les sacrificateurs, et n'ont pas été miséricordieux envers les vieillards.
\VS{17}[Ayin.] Pour nous, nos yeux se consumaient après un vain secours ; nous regardions du haut de nos lieux élevés vers une nation qui ne pouvait pas délivrer\FTNT{Jé. 18:15.}.
\VS{18}[Tsade.] Ils ont épié nos pas afin de nous empêcher d'aller sur nos places ; notre fin s'approchait, nos jours étaient accomplis… Notre fin est arrivée !
\VS{19}[Qof.] Nos persécuteurs étaient plus légers que les aigles des cieux ; ils nous ont poursuivis sur les montagnes, ils ont mis des embûches contre nous dans le désert.
\VS{20}[Resh.] Le souffle de nos narines, l'oint de Yahweh\FTNT{L'oint en question est le roi Josias (2 R. 21:24 ; 22 ; 23).}, a été pris dans leurs fosses, celui de qui nous disions : Nous vivrons sous son ombre parmi les nations.
\VS{21}[Shin.] Réjouis-toi, sois dans l'allégresse, fille d'Edom, habitante du pays d'Uts ! La coupe passera aussi vers toi ; tu en seras enivrée, et tu seras mise à nu\FTNT{Jé. 25:15-18 ; Ps. 137:7.}.
\VS{22}[Tav.] Fille de Sion, ton iniquité est expiée ; il ne t'enverra plus en exil. Fille d'Edom, il châtiera ton iniquité, il découvrira tes péchés.
\Chap{5}
\TextTitle{Supplications de Jérémie à Yahweh}
\VerseOne{}Souviens-toi, ô Yahweh, de ce qui nous est arrivé ! Regarde et vois notre opprobre !
\VS{2}Notre héritage a été renversé par des étrangers, nos maisons par des inconnus.
\VS{3}Nous sommes devenus comme des orphelins qui sont sans pères, et nos mères sont comme des veuves.
\VS{4}Nous buvons notre eau à prix d'argent, et notre bois nous est vendu.
\VS{5}Ceux qui nous poursuivent sont sur notre cou ; nous sommes épuisés, nous n'avons pas de repos.
\VS{6}Nous avons étendu la main vers l'Egypte, et vers l'Assyrie pour nous rassasier de pain.
\VS{7}Nos pères ont péché, ils ne sont plus, et c'est nous qui portons la peine de leurs iniquités\FTNT{Chaque homme naît pécheur et hérite de la nature pécheresse d'Adam (Ge. 3:20 ; Ac. 17:26 ; Ro. 5:12-21). De ce fait, les péchés commis par les parents ont des conséquences sur les enfants (Ex. 20:4-5 ). Jésus-Christ nous a délivrés du péché d'Adam et de celui de nos ancêtres à la croix (Col. 1:12). Lors de notre naissance d'en haut, les péchés de notre passé et de nos origines sont expiés (2 Co. 5:17 ; Ep. 1:7 ; Ep. 2:1-15 ; Col. 1:12-14 ; Col. 2:13-15 ; 1 Pi. 1:18-19 ; 1 Jn. 1:7 ; 1 Jn. 1:9). Les péchés des ancêtres et leurs conséquences touchent les personnes qui vivent dans les péchés de leurs ancêtres, c'est-à-dire ceux qui haïssent Dieu et ses commandements (De. 24:16 ; Jé. 31:29-34 ; Ez. 18:17-20).}.
\VS{8}Les esclaves dominent sur nous, et personne ne nous délivre de leurs mains.
\VS{9}Nous amenons notre pain au péril de notre vie, à cause de l'épée du désert.
\VS{10}Notre peau est brûlante comme un four, à cause l'ardeur de la faim.
\VS{11}Ils ont déshonoré les femmes dans Sion, les vierges dans les villes de Juda.
\VS{12}Des chefs ont été pendus par leurs mains ; et ils n'ont pas honoré la personne des vieillards.
\VS{13}Ils ont pris les jeunes gens pour moudre, et les enfants sont tombés sous le bois.
\VS{14}Les vieillards ont cessé de se trouver aux portes, et les jeunes gens de chanter.
\VS{15}La joie a disparu de notre cœur, et notre danse est changée en deuil.
\VS{16}La couronne de notre tête est tombée ! Malheur à nous, parce que nous avons péché !
\VS{17}C'est pourquoi notre coeur est languissant. À cause de ces choses, nos yeux sont obscurcis.
\VS{18}À cause de la montagne de Sion qui est désolée ; les renards s'y promènent.
\VS{19}Toi, ô Yahweh, tu demeures éternellement, et ton trône subsiste de génération en génération.
\VS{20}Pourquoi nous oublierais-tu à jamais ? pourquoi nous délaisserais-tu si longtemps ?
\VS{21}Convertis-nous à toi, ô Yahweh ! et nous serons convertis ; renouvelle nos jours comme ils étaient autrefois\FTNT{Jé. 30:20 ; Jé. 31:18 ; Ps. 80:3.}.
\VS{22}Ou bien, nous aurais-tu entièrement rejetés ? Serais-tu extrêmement courroucé contre nous ?
\PPE{}
\end{multicols}

%\clearpage\ShortTitle{Ec.}\BookTitle{Ecclésiaste}\BFont
\noindent\hrulefill
{\footnotesize
\textit{
\bigskip
{\centering{}
\\Auteur~: Salomon
\\(Heb.~: Qohelet)
\\Signification~: Prédicateur
\\Thème~: Les raisonnements humains
\\Date de rédaction~: 10\up{ème} siècle av. J.-C.\\}
}
\textit{
\\Ce livre, du fait de sa date de rédaction, est généralement attribué à Salomon en raison de l'allusion faite au premier chapitre et du style adopté. L'Ecclésiaste figure d'ailleurs dans le canon des livres reconnus d'inspiration divine.
\\La problématique centrale du livre est de savoir si la vie vaut la peine d'être vécue ou non. L'auteur y répondit en connaissance de cause car il avait obtenu tout ce que l'homme pouvait désirer~: les richesses, le luxe, la volupté, la sagesse… Sans pour autant incriminer Dieu, il dressa le constat de ce qu'est l'expérience humaine. Selon lui, l'homme vit dans un cycle d'éternels recommencements où tout n'est que poursuite du vent et vanité.\bigskip
}
}
\par\nobreak\noindent\hrulefill
\begin{multicols}{2}
\Chap{1}
\TextTitle{Tout est vanité\FTNTT{Ec. 12:8.}}
\VerseOne{}Les Paroles de l'Ecclésiaste, fils de David, roi de Jérusalem.
\VS{2}Vanité des vanités, dit l'Ecclésiaste, vanité des vanités, tout est vanité.
\TextTitle{Le cycle du temps}
\VS{3}Quel avantage a l'homme de tout son travail auquel il s'occupe sous le soleil~?\FTNT{Ec. 2:22~; Ec. 3:9.}
\VS{4}Une génération passe et une autre génération vient, mais la terre demeure toujours ferme.
\VS{5}Le soleil aussi se lève et le soleil se couche~; il soupire après le lieu d'où il se lève.
\VS{6}Le vent va vers le midi, tourne vers le nord~; il va tournoyant çà et là, et il retourne après ses circuits.
\VS{7}Tous les fleuves vont à la mer, et la mer n'en est point remplie~; les fleuves retournent au lieu d'où ils étaient partis, pour revenir\FTNT{Job 38:8-11~; Ps. 104:9-10.} à la mer. 
\VS{8}Toutes les choses sont lassantes et l'homme ne peut en parler~; l'œil n'est jamais rassasié de voir\FTNT{Pr. 27:20.} et l'oreille ne se lasse pas d'entendre. 
\VS{9}Ce qui a été, c'est ce qui sera, et ce qui s'est fait, c'est ce qui se fera, il n'y a rien de nouveau sous le soleil\FTNT{Ec. 3:15.}.
\VS{10}Y~a-t-il quelque chose dont on puisse dire~: Regarde cela, il est nouveau~? Il a déjà été dans les siècles qui ont été avant nous.
\VS{11}On ne se souvient plus des choses d'autrefois~; de même on ne se souviendra point des choses à venir et ceux qui viendront n'en auront aucun souvenir. 
\TextTitle{La sagesse des hommes ne comble pas}
\VS{12}Moi l'Ecclésiaste, j'ai été roi sur Israël à Jérusalem.
\VS{13}Et j'ai appliqué mon cœur à rechercher et à sonder par la sagesse tout ce qui se fait sous les cieux~: C'est une occupation désagréable que Dieu a donnée aux hommes, afin qu'ils s'y occupent\FTNT{1 R. 4:30-34~; Ec. 7:25.}.
\VS{14}J'ai vu toutes les œuvres qui se font sous le soleil~; et voici tout est vanité et tourment d'esprit.
\VS{15}Ce qui est courbé ne peut se redresser, et ce qui manque ne peut être compté.
\VS{16}J'ai parlé en mon cœur, disant~: Voici, je suis devenu grand et j'ai surpassé en sagesse tous ceux qui ont été avant moi sur Jérusalem, et mon cœur a vu beaucoup de sagesse et de science.
\VS{17}Et j'ai appliqué mon cœur à connaître la sagesse, et à connaître les sottises et la stupidité~; et j'ai reconnu que cela aussi était un tourment d'esprit.
\VS{18}Car là où il y a beaucoup de sagesse, il y a beaucoup de chagrin, et celui qui augmente sa connaissance, augmente son chagrin.
\Chap{2}
\TextTitle{Les richesses ne comblent pas}
\VerseOne{}J'ai dit en mon cœur~: Allons, que je t'éprouve maintenant par la joie et prends du bon temps. Et voici, c'est encore une vanité\FTNT{Lu. 12:19.}.
\VS{2}J'ai dit concernant le rire~: Il est insensé~! Et concernant la joie~: A quoi sert-elle~?
\VS{3}J'ai recherché en moi-même le moyen de me traiter délicatement, de faire que mon cœur s'accoutume cependant à la sagesse, et qu'il comprenne ce que c'est que la folie, jusqu'à ce que je voie ce qu'il est bon aux hommes de faire sous les cieux, pendant les jours de leur vie. 
\VS{4}Je me suis fait des choses magnifiques~; je me suis bâti des maisons~; je me suis planté des vignes.
\VS{5}Je me suis fait des jardins et des vergers, et j'y plantai des arbres fruitiers de toutes sortes~;
\VS{6}je me suis fait des réservoirs d'eaux pour arroser la forêt où poussent les arbres.
\VS{7}J'ai acquis des hommes et des femmes esclaves~; et j'ai eu des esclaves nés dans ma maison, et j'ai eu plus de gros et de menu bétail que tous ceux qui ont été avant moi dans Jérusalem. 
\VS{8}Je me suis aussi amassé de l'argent et de l'or, et des plus précieux joyaux qui se trouvent chez les rois et dans les provinces\FTNT{1 R. 9:28~; 1 R. 10:10~; 2 Ch. 1:15.}. Je me suis acquis des chanteurs et des chanteuses, et les délices des hommes~; une harmonie d'instruments de musique, même plusieurs harmonies de toutes sortes d'instruments\FTNT{La plupart des bibles ont traduit la deuxième partie de ce verset par le mot «~femme~», or le sens du terme hébreu «~shiddah~» est incertain. Toutefois, le contexte de ce verset montre clairement que Salomon parlait des instruments de musique et des chanteurs et chanteuses qu'il a acquis, et non de ses conquêtes féminines.}. 
\VS{9}Je me suis aggrandi et me suis accru plus que tous ceux qui ont été avant moi dans Jérusalem. Et ma sagesse est demeurée avec moi.
\VS{10}Enfin je n'ai rien refusé à mes yeux de tout ce qu'ils ont demandé~; et je n'ai épargné aucune joie à mon cœur~; car mon cœur s'est réjoui de tout mon travail et c'est là tout ce que j'ai eu de tout mon travail.
\VS{11}Mais ayant considéré toutes mes œuvres que mes mains avaient faites, et tout le travail auquel je m'étais occupé en les faisant, voilà tout était vanité et tourment d'esprit~; tellement que l'homme n'a aucun avantage de ce qui est sous le soleil.
\TextTitle{Le sage et l'insensé ont le même sort}
\VS{12}Puis je me suis mis à considérer tant la sagesse que les sottises et la folie, (or qui est l'homme qui pourrait suivre le roi dans ce qui a été déjà fait~?).
\VS{13}Et j'ai vu que la sagesse a beaucoup d'avantage sur la folie, comme la lumière a beaucoup d'avantage sur les ténèbres.
\VS{14}Le sage a ses yeux à sa tête, et l'insensé marche dans les ténèbres. Mais j'ai aussi reconnu qu'un même sort leur arrive à tous\FTNT{Ps. 49:11~; Ec. 3:17~; Ec. 9:2.}.
\VS{15}C'est pourquoi j'ai dit en mon cœur~: Il m'arrivera le même sort que l'insensé~; de quoi donc me servira-t-il alors d'avoir été plus sage~? C'est pourquoi j'ai dit en mon cœur que cela aussi est une vanité. 
\VS{16}Car le souvenir du sage n'est pas plus éternel que celui de l'insensé, parce que ce qui est maintenant va être oublié dans les jours qui suivent. Le sage meurt aussi bien que l'insensé\FTNT{Ec. 8:10~; Ec. 9:5.}~!
\VS{17}C'est pourquoi j'ai haï cette vie, car les choses qui se sont faites sous le soleil m'ont déplu~; car tout est vanité, et tourment d'esprit.
\VS{18}J'ai aussi haï tout mon travail, auquel je me suis occupé sous le soleil, parce que je le laisserai à l'homme qui sera après moi\FTNT{Ec. 4:8.}.
\VS{19}Et qui sait s'il sera sage ou insensé~? Cependant il sera maître de tout mon travail, auquel je me suis occupé et de ce en quoi j'ai été sage sous le soleil. Cela aussi est une vanité.
\VS{20}C'est pourquoi j'ai fait en sorte que mon cœur perde toute espérance de tout le travail auquel je m'étais occupé sous le soleil.
\VS{21}Car il y a tel homme, dont le travail a été avec sagesse, science, et adresse, qui néanmoins le laisse à celui qui n'y a point travaillé comme étant sa part~; cela aussi est une vanité et un grand mal. 
\VS{22}Car qu'est-ce que l'homme a de tout son travail et du désir de son cœur, dont il souffre sous le soleil~?
\VS{23}Puisque tous ses jours ne sont que douleur, et son occupation n'est que chagrin~; même la nuit son cœur ne repose point. Cela aussi est une vanité\FTNT{Ps. 90:9~; Job. 14:1.}.
\VS{24}N'est-ce donc pas un bien pour l'homme de manger, et de boire, et de faire que son âme jouisse du bien dans son travail~? J'ai vu aussi que cela vient de la main de Dieu\FTNT{Ec. 3:12~; Ec. 3:22~; Ec. 5:18~; Ec. 8:15.}.
\VS{25}Car qui en mangera, qui s'est réjoui plus que moi~?
\VS{26}Parce que Dieu donne à celui qui lui est agréable, de la sagesse, de la science et de la joie~; mais il donne au pécheur de l'occupation à recueillir et à assembler, afin que cela soit donné à celui qui est agréable à Dieu. Cela aussi est une vanité et un tourment d'esprit\FTNT{Pr. 13:22~; Pr. 28:8~; Job 27:17.}.
\Chap{3}
\TextTitle{Il y a un temps pour toute chose}
\VerseOne{}A toute chose sa saison, et à toute affaire sous les cieux son temps.
\VS{2}Un temps pour naître et un temps pour mourir~; un temps pour planter et un temps pour arracher ce qui est planté~;
\VS{3}un temps pour tuer et un temps pour guérir~; un temps pour démolir et un temps pour bâtir~;
\VS{4}un temps pour pleurer et un temps pour rire~; un temps pour se lamenter et un temps pour sauter de joie~;
\VS{5}un temps pour jeter des pierres et un temps pour ramasser des pierres~; un temps pour embrasser et un temps pour s'éloigner des embrassements~;
\VS{6}un temps pour chercher et un temps pour perdre~; un temps pour garder et un temps pour jeter~;
\VS{7}un temps pour déchirer et un temps pour coudre~; un temps pour être silencieux et un temps pour parler~;
\VS{8}un temps pour aimer et un temps pour haïr~; un temps pour la guerre et un temps pour la paix.
\TextTitle{Dieu fait toute chose belle en son temps}
\VS{9}Quel avantage celui qui travaille a-t-il de sa peine~?
\VS{10}J'ai considéré cette occupation que Dieu a donnée aux fils des hommes pour s'y appliquer.
\VS{11}Il a fait que toutes choses sont belles en leur temps~; aussi a-t-il mis l'éternité dans leur cœur, sans toutefois que l'homme puisse comprendre du commencement à la fin\FTNT{Ec. 8:17.}l'œuvre que Dieu a faite. 
\VS{12}C'est pourquoi j'ai reconnu qu'il n'y a rien de meilleur aux hommes, que de se réjouir et de se faire du bien pendant leur vie. 
\VS{13}Et même si un homme mange et boit et jouit du bien-être de tout son travail, c'est un don de Dieu\FTNT{Ec. 5:18~; Ec. 8:15~; Ec. 9:7.}.
\VS{14}J'ai reconnu que tout ce que Dieu fait subsiste à toujours, il n'y a rien à y ajouter et rien à en retrancher, et Dieu le fait afin que devant lui, on le craigne.
\VS{15}Ce qui a été, est maintenant~; et ce qui doit être, a déjà été~; et Dieu rappelle ce qui est passé.
\VS{16}J'ai encore vu sous le soleil qu'au lieu établi pour juger, il y a de la méchanceté~; et qu'au lieu établi pour la justice, il y a de la méchanceté.
\VS{17}J'ai dit en mon cœur~: Dieu jugera le juste et le méchant~; car il y a là un temps pour toute chose et pour toute œuvre.
\VS{18}J'ai dit en mon cœur sur l'état des fils de l'homme, que Dieu les éprouverait, et qu'ils verraient qu'ils ne sont que des bêtes.
\VS{19}Car le sort des fils d'Adam et le sort de la bête est un même sort~; telle qu'est la mort de l'un, telle est la mort de l'autre. Tous ont un même souffle et la supériorité de l'homme sur la bête est nulle. Car tout est vanité.
\VS{20}Tout va dans un même lieu~; tout a été fait de la poussière, et tout retourne à la poussière\FTNT{Ge. 3:19~; Job 34:15~; Ec. 6:6~; Ec. 12:9.}.
\VS{21}Qui sait si l'esprit des fils de l'homme monte en haut, et si l'esprit de la bête descend en bas dans la terre\FTNT{Les animaux comme les hommes ont une âme et un esprit (Ge. 1:20~; Ez. 1:1-28). A leur mort, leurs esprits quittent leurs corps (Ja. 2:26). L'homme régénéré reçoit le Saint-Esprit, ce qui n'est pas le cas des animaux (Ro. 8:16). Les animaux, tout comme la création tout entière, attendent leur rédemption, car ils ont été soumis à la corruption à cause du péché de l'homme (Ro. 8:19-22).Dans le royaume millénaire, il y aura des animaux (Es. 11:6-9).}~?
\VS{22}J'ai donc vu qu'il n'y a rien de meilleur pour l'homme que de se réjouir de ses œuvres~: C'est là sa part. Car qui le ramènera pour voir ce qui sera après lui~?
\Chap{4}
\TextTitle{Un monde injuste}
\VerseOne{}Puis je me suis mis à regarder toutes les injustices qui se font sous le soleil~; et voici les larmes de ceux à qui on fait tort, et ils n'ont point de consolation. Et la force est du côté de ceux qui leur font tort, et ils n'ont point de consolateur. 
\VS{2}C'est pourquoi j'estime plus les morts qui sont déjà morts, que les vivants qui sont encore vivants\FTNT{Ec. 7:1.}~;
\VS{3}même j'estime celui qui n'a pas encore été, plus heureux que les uns et les autres~; car il n'a pas vu les mauvaises actions qui se font sous le soleil.
\VS{4}Puis j'ai vu que tout travail et tout succès dans le travail n'est que jalousie de l'un à l'égard de l'autre. Cela aussi est une vanité et un tourment d'esprit. 
\VS{5}L'insensé se croise les mains et dévore sa propre chair\FTNT{Pr. 6:10~; Pr. 19:24~; Pr. 24:33~; Pr. 26:15.}.
\VS{6}Mieux vaut le creux de la main pleine avec repos, que les deux mains pleines avec travail et tourment d'esprit\FTNT{Ps. 37:16~; Pr. 15:16-17~; Pr. 16:8.}. 
\VS{7}Puis je me suis mis à regarder une autre vanité sous le soleil. 
\VS{8}C'est qu'il y a tel qui est seul, et qui n'a point de second, qui aussi n'a ni fils ni frère et qui cependant ne met nulle fin à son travail~; même son œil ne voit jamais assez de richesses, et il ne se dit point en lui-même~: Pour qui est-ce que je travaille, et que je prive mon âme du bien~? Cela aussi est une vanité et une fâcheuse occupation\FTNT{Ec. 2:26~; Ps. 39:7~; Lu. 12:20.}.
\VS{9}Deux valent mieux qu'un, car ils ont un meilleur salaire de leur travail.
\VS{10}Même si l'un des deux tombe, l'autre relèvera son compagnon~; mais malheur à celui qui est seul~; parce qu'étant tombé, il n'aura personne pour le relever. 
\VS{11}Si deux aussi couchent ensemble, ils en auront plus de chaleur~; mais celui qui est seul, comment aura-t-il chaud~? 
\VS{12}Et si quelqu'un a le dessus sur l'un ou sur l'autre, les deux peuvent lui résister~; et la corde à trois cordons ne se rompt pas rapidement.
\VS{13}Un enfant pauvre et sage vaut mieux qu'un roi vieux et insensé qui ne sait ce que c'est que d'être averti.
\VS{14}Car tel qui sort de prison pour régner, et de même tel étant né roi, devient pauvre dans son royaume.
\VS{15}J'ai vu tous les vivants qui marchent sous le soleil suivre le fils qui est la seconde personne après le roi, et qui doit être à sa place. 
\VS{16}Il n'y a pas de fin à tout le peuple, à tous ceux qui ont été devant eux~; cependant ceux qui viendront après ne se réjouiront point en lui. Certainement cela aussi est une vanité, et un tourment d'esprit. 
\TextTitle{Le sacrifice des insensés}
\VS{17}Quand tu entres dans la maison de Dieu, prends garde à ton pied, et approche-toi pour écouter, plutôt que pour donner ce que donnent les insensés, car ils ne savent pas qu'ils font mal.
\Chap{5}
\VerseOne{}Ne te précipite point à parler, et que ton cœur ne se hâte point de parler devant Dieu~; car Dieu est au ciel, et toi sur la terre~; c'est pourquoi use de peu de paroles.
\VS{2}Car comme le songe vient de la multitude des occupations~; ainsi la voix des insensés sort de la multitude des paroles\FTNT{Pr. 10:19.}.
\VS{3}Quand tu as fais quelque vœu à Dieu, ne diffère point de l'accomplir~; car il ne prend point de plaisir aux insensés~; accomplis donc le vœu que tu as fait\FTNT{No. 30:3. De. 23:21}.
\VS{4}Il vaut mieux que tu ne fasses point de vœux que d'en faire et de ne pas les accomplir\FTNT{De. 23:21-22.}.
\VS{5}Ne permets pas à ta bouche de faire pécher ta chair, et ne dis point devant le messager de Dieu que c'est un péché involontaire. Pourquoi Yahweh s'irriterait-il de tes paroles, et détruirait-il l'œuvre de tes mains~?
\VS{6}Car comme dans la multitude des songes il y a des vanités, aussi y en a-t-il beaucoup dans la multitude des paroles~; mais crains Dieu\FTNT{Ec. 10:14~; Pr. 10:19.}.
\VS{7}Si tu vois dans la Province qu'on fasse tort au pauvre, et que le droit et la justice y soient violés, ne t'étonne point de cela~; car celui qui est plus élevé que les plus hauts élevés y prend garde, et il y en a de plus élevés qu'eux\FTNT{Es. 3:14-15.}.
\TextTitle{Vanité des richesses}
\VS{8}C'est un avantage pour le pays, un roi qui travaille dans les champs.
\VS{9}Celui qui aime l'argent n'est point rassasié par l'argent\FTNT{Jé. 6:13~; Pr. 22:7~; Pr. 28:16~; Mt. 6:33~; Mt. 7:7-11~; Lu. 12:13-20~; Ac. 20:33~; 2 Co. 9:5~; Ep. 4:19~; Ep. 5:5~; Col. 3:5~; 1 Ti. 6:10~; Hé. 13:5.}, et celui qui aime les richesses n'en est pas nourri~; cela aussi est une vanité. 
\VS{10}Où il y a beaucoup de bien, là il y a beaucoup de gens qui le mangent~; et quel avantage en revient-il à son maître, sinon qu'il le voit de ses yeux~? 
\VS{11}Le sommeil de celui qui travaille est doux, qu'il mange peu ou beaucoup~; mais le rassasiement du riche ne le laisse point dormir. 
\VS{12}Il y a un mal fâcheux que j'ai vu sous le soleil, c'est que des richesses sont conservées à leurs maîtres afin qu'ils en aient du mal. 
\VS{13}Ces richesses périssent par quelque fâcheux accident~; de sorte qu'on aura engendré un fils et il n'aura rien entre ses mains.
\VS{14}Et comme il est sorti du ventre de sa mère, il s'en retournera nu, s'en allant comme il était venu, et il n'emportera rien de son travail auquel il a employé ses mains\FTNT{1 Ti. 6:7.}.
\VS{15}Et c'est aussi un mal fâcheux, que comme il est venu, il s'en va de même~; et quel avantage a-t-il d'avoir travaillé pour du vent~?
\VS{16}Il mange aussi tous les jours de sa vie dans les ténèbres et se chagrine beaucoup, et son mal va jusqu'à la fureur.
\VS{17}Voilà donc ce que j'ai vu~; que c'est une chose bonne et agréable à l'homme de manger, de boire et de jouir du bien-être de tout son travail qu'il fait sous le soleil, pendant le nombre des jours de vie que Dieu lui a donnés~; car c'est là sa part.
\VS{18}Aussi ce que Dieu donne de richesses et de biens à un homme, quel qu'il soit~; ce dont il le fait maître pour en manger, pour en prendre sa part et pour se réjouir de son travail~; c'est là un don de Dieu. 
\VS{19}Car il ne se souviendra pas beaucoup des jours de sa vie, parce que Dieu lui répond par la joie de son cœur. 
\Chap{6}
\TextTitle{Vanité de la vie de l'homme}
\VerseOne{}Il y a un mal que j'ai vu sous le soleil, et qui est fréquent parmi les hommes.
\VS{2}C'est qu'il y a tel homme à qui Dieu a donné des richesses, des biens et des honneurs, en sorte qu'il ne manque rien pour son âme de tout ce qu'il peut souhaiter. Mais Dieu ne l'en fait pas le maître pour en manger, car un étranger le mangera. Cela est une vanité et un mal fâcheux. 
\VS{3}Quand un homme engendrerait cent fils, qu'il vivrait plusieurs années, en sorte que les jours de ses années se soient fort multipliés, cependant si son âme ne s'est point rassasiée de bien, et même s'il n'a point eu de sépulture, je dis qu'un avorton vaut mieux que lui.
\VS{4}Car il est venu en vain, et s'en va dans les ténèbres, et son nom est couvert de ténèbres~;
\VS{5}Il n'a même point vu le soleil~; il n'a rien connu~; il a plus de repos que cet homme-là\FTNT{Job 3:16.}.
\VS{6}Et s'il vivait deux fois mille ans, et qu'il ne jouit d'aucun bien, tous ne vont-ils pas dans un même lieu\FTNT{Ec. 3:20~; Job 3:13-19~; Job 30:23~; Ps. 89:48~; Hé. 9:27.}~?
\VS{7}Tout le travail de l'homme est pour sa bouche, et cependant son âme n'est jamais satisfaite\FTNT{Les richesses de ce monde ne peuvent jamais combler le vide de l'âme. Seul l'amour de Dieu peut réellement inonder nos âmes (Pr. 13:4).}.
\VS{8}Car qu'est-ce que le sage a de plus que l'insensé~? Ou quel avantage a le malheureux qui sait se conduire devant les vivants~?
\VS{9}Mieux vaut ce qu'on voit de ses yeux, que les grandes recherches que fait l'âme. Cela aussi est une vanité, et un tourment d'esprit\FTNT{1 Ti. 6:9.}.
\VS{10}Ce qui existe a déjà été appelé par son nom\FTNT{Ec. 1:9~; Ec. 3:15.}~; et savait-on ce que devait être l'homme, et qu'il ne pourrait plaider avec celui qui est plus fort que lui~? 
\VS{11}Quand on a beaucoup de choses, on a beaucoup de vanités. Quel avantage en a l'homme~? 
\VS{12}Car qui est-ce qui connaît ce qui est bon à l'homme dans sa vie, pendant les jours de la vie de sa vanité, lesquels il passe comme une ombre~? Et qui est-ce qui déclarera à l'homme ce qui sera après lui sous le soleil\FTNT{Ps. 144:4~; Ec. 8:7~; Ec. 8:13~; Ec. 10:14~; Ja. 4:13-14.}~?
\Chap{7}
\TextTitle{La sagesse qu'enseigne la vie de l'homme}
\VerseOne{}Une bonne réputation vaut mieux que le bon parfum, et le jour de la mort que le jour de la naissance\FTNT{Pr. 22:1.}.
\VS{2}Il vaut mieux aller dans une maison de deuil que d'aller dans une maison de festin~; car c'est là la fin de tout homme, et le vivant met cela dans son cœur.
\VS{3}Il vaut mieux le chagrin que le rire~; car par la tristesse du visage le cœur devient joyeux\FTNT{Ec. 8:1~; 2 Co. 7:10.}.
\VS{4}Le cœur des sages est dans la maison du deuil, mais le cœur des insensés est dans la maison de joie.
\VS{5}Il vaut mieux entendre la réprimande du sage, que d'entendre la chanson des hommes insensés\FTNT{Ps. 141:5~; Pr. 13:18~; Pr. 15:31-32.}.
\VS{6}Car tel qu'est le bruit des épines sous la chaudière, tel est le rire de l'insensé. Cela aussi est une vanité. 
\VS{7}Certainement l'oppression fait perdre le sens au sage~; et le don fait perdre l'entendement. 
\VS{8}Mieux vaut la fin d'une chose que son commencement. Mieux vaut l'homme qui est d'un esprit patient que l'homme qui est d'un esprit hautain. 
\VS{9}Ne te précipite point en ton esprit de t'irriter, car l'irritation repose dans le sein des insensés.
\VS{10}Ne dis point~: D'où vient que les jours passés ont été meilleurs que ceux-ci~? Car ce n'est pas par sagesse que tu demandes cela. 
\VS{11}La sagesse est bonne avec un héritage, elle est un avantage pour ceux qui voient le soleil.
\VS{12}Car on est à couvert à l'ombre de la sagesse, de même qu'à l'ombre de l'argent~; mais la science a cet avantage, que la sagesse fait vivre celui qui en est doué. 
\VS{13}Regarde l'œuvre de Dieu~: Qui pourra redresser ce qu'il a renversé~?
\VS{14}Au jour du bonheur, sois heureux, et au jour de l'adversité, prends-y garde~; car Dieu a fait l'un exactement comme l'autre, afin que l'homme ne trouve rien à redire après lui. 
\VS{15}J'ai vu tout ceci pendant les jours de ma vanité. Il y a tel juste qui périt dans sa justice, et il y a tel méchant qui prolonge ses jours dans sa méchanceté\FTNT{Ec. 8:14~; Job 21:7-8.}.
\VS{16}Ne te crois pas trop juste, et ne te fais pas plus sage qu'il ne faut~: Pourquoi t'exposer à la ruine\FTNT{Pr. 3:7~; Ro. 12:16.}~?
\VS{17}Ne sois point méchant à l'excès, et ne sois point insensé~: Pourquoi mourrais-tu avant ton temps\FTNT{Ec. 9:16.}~?
\VS{18}Il est bon que tu retiennes ceci, et que tu ne retires point ta main de cela~; car celui qui craint Dieu sort de tout.
\VS{19}La sagesse donne plus de force au sage que dix gouverneurs qui sont dans une ville.
\VS{20}Certainement il n'y a point d'homme juste sur la terre qui agisse toujours bien, et qui ne pèche point\FTNT{Ps. 14:3~; Pr. 20:9~; 2 Ch. 6:36~; Ja. 3:2~; Ro. 3:12~; 1 Jn. 1:8.}.
\VS{21}Ne mets point aussi ton cœur à toutes les paroles qu'on dira, afin que tu n'entendes pas ton serviteur médire de toi. 
\VS{22}Car ton cœur aussi a reconnu plusieurs fois que tu as pareillement mal parlé des autres. 
\VS{23}J'ai essayé tout ceci avec sagesse, et j'ai dit~: J'acquerrai de la sagesse~; mais elle s'est éloignée de moi. 
\VS{24}Ce qui est loin et ce qui est profond, qui le trouvera~?
\VS{25}Je me suis appliqué dans mon cœur à connaître, à sonder, et à chercher la sagesse et la raison de tout~; et à connaître la méchanceté de la folie, de la bêtise et des sottises. 
\VS{26}Et j'ai trouvé plus amère que la mort, la femme dont le cœur est un piège et un filet, et dont les mains sont des liens~; celui qui est agréable à Dieu lui échappera~; mais le pécheur sera pris par elle\FTNT{Pr. 5:3-4~; Pr. 6:26~; Pr. 7:13-27~; Pr. 9:13-16~; Pr. 22:14.}.
\VS{27}Voici, dit l'Ecclésiaste, ce que j'ai trouvé en cherchant la raison de toutes choses, l'une après l'autre~;
\VS{28}C'est que jusqu'à présent, mon âme a cherché, mais que je n'ai point trouvé, c'est que j'ai bien trouvé un homme entre mille~; mais pas une femme entre elles toutes. 
\VS{29}Seulement voici ce que j'ai trouvé~; c'est que Dieu a créé l'homme juste~; mais ils ont cherché beaucoup d'inventions.
\Chap{8}
\TextTitle{L'obéissance aux autorités}
\VerseOne{}Qui est tel que le sage~? Et qui sait ce que veulent dire les choses~? La sagesse de l'homme fait briller son visage, et son regard farouche en est changé\FTNT{Ec. 7:3~; Pr. 15:13.}.
\VS{2}Je te le dis~: Prends garde aux ordres du roi, et cela à cause du serment fait à Dieu.
\VS{3}Ne te précipite point de te retirer de devant sa face~; et ne persévère point dans une chose mauvaise~; car il fera tout ce qu'il lui plaira. 
\VS{4}En quelque lieu qu'est la parole du roi, là est la puissance~; et qui lui dira~: Que fais-tu~? 
\VS{5}Celui qui garde le commandement ne sentira aucun mal~; et le cœur du sage discerne le temps et ce qui est juste. 
\VS{6}Car dans toute affaire il y a un temps et un jugement, autrement mal sur mal tombe sur l'homme. 
\VS{7}Car il ne sait pas ce qui arrivera~; et même qui est-ce qui lui déclarera quand cela arrivera~? 
\VS{8}L'homme n'est point maître de son souffle\FTNT{Le souffle ou l'esprit de l'homme quitte son corps le jour de sa mort (Ps. 39:5~; Ja. 2:26).} pour pouvoir le retenir, il n'a aucune puissance sur le jour de la mort~; il n'y a point de délivrance dans ce combat, et la méchanceté ne délivrera point son maître.
\VS{9}J'ai vu tout cela, et j'ai appliqué mon cœur à toute œuvre qui se fait sous le soleil. Il y a un temps où l'homme domine sur l'autre pour son malheur.
\VS{10}Alors j'ai vu les méchants ensevelis et s'en aller~; et ceux qui avaient agi avec droiture s'en aller loin du lieu saint et être oubliés dans la ville. Cela aussi est une vanité\FTNT{Ec. 2:16~; Ec. 9:5.}.
\VS{11}Parce que la sentence contre les mauvaises œuvres ne s'exécute point promptement, à cause de cela le cœur des fils de l'homme se remplit en eux de l'envie de faire le mal\FTNT{Ec. 12:1.}.
\VS{12}Car bien que le pécheur fasse le mal cent fois, et qu'il y persévère longtemps, je sais aussi qu'il y aura du bonheur pour ceux qui craignent Dieu et qui révèrent sa face\FTNT{Job 22:21~; Pr. 1:33~; Es. 3:10.}.
\VS{13}Mais le bonheur n'est pas pour le méchant, et il ne prolongera point ses jours plus que l'ombre, parce qu'il n'a pas de crainte devant Dieu.
\VS{14}Il y a une vanité qui arrive sur la terre~: C'est qu'il y a des justes auxquels il arrive selon l'œuvre des méchants~; et il y a aussi des méchants auxquels il arrive selon l'œuvre des justes. Je dis que cela aussi est une vanité.
\VS{15}C'est pourquoi j'ai loué la joie, parce qu'il n'y a rien sous le soleil de meilleur à l'homme, que de manger et de boire et de se réjouir~; c'est aussi ce qui lui restera de son travail durant les jours de sa vie, que Dieu lui donne sous le soleil. 
\VS{16}Après avoir appliqué mon cœur à connaître la sagesse, et à regarder les occupations qu'il y a sur la terre, (car l'homme ne donne, ni jour ni nuit, de repos à ses yeux), 
\VS{17}après avoir vu, dis-je, toute l'œuvre de Dieu, j'ai vu que l'homme ne peut pas trouver l'œuvre qui se fait sous le soleil~; il a beau se fatiguer à chercher, il n'est pas capable de trouver~; et même si le sage dit la connaître, il ne peut la trouver.
\Chap{9}
\TextTitle{L'impuissance de la sagesse face à la mort}
\VerseOne{}Certainement j'ai appliqué mon cœur à tout cela~; et pour l'éclaircir à savoir que les justes, les sages et leurs actions sont dans la main de Dieu~; mais les hommes ne connaissent ni l'amour ni la haine de tout ce qui est devant eux. 
\VS{2}Tout arrive également à tous~; un même sort arrive au juste et au méchant~; au bon, au pur et au souillé~; à celui qui sacrifie et à celui qui ne sacrifie point~; le pécheur est comme l'homme de bien~; celui qui jure, comme celui qui craint de jurer. 
\VS{3}C'est un mal parmi tout ce qui se fait sous le soleil, c'est qu'il y a pour tous un même sort~; aussi le cœur des fils de l'homme est-il plein de méchanceté, et la folie est dans leur cœur pendant leur vie~; après cela, ils vont chez les morts. Qui est celui qui voudrait leur être associé\FTNT{Ec. 2:16~; Job 9:22}~?
\VS{4}Il y a de l'espérance pour tous ceux qui sont encore vivants~; et même un chien vivant vaut mieux qu'un lion mort.
\VS{5}Certainement les vivants savent qu'ils mourront, mais les morts ne savent rien, et ne gagnent plus rien~; car leur mémoire est mise en oubli. 
\VS{6}Aussi leur amour, leur haine, et leur envie ont déjà péri, et ils n'auront plus aucune part à tout ce qui se fait sous le soleil.
\VS{7}Va donc, mange ton pain avec joie, et bois gaiement ton vin~; car depuis longtemps Dieu prend plaisir à tes œuvres.
\VS{8}Que tes vêtements soient blancs en tout temps, et que le parfum ne manque point sur ta tête. 
\VS{9}Vis joyeusement tous les jours de ta vie de vanité avec la femme que tu aimes, qui t'a été donnée sous le soleil, tous les jours de ta vanité~; car c'est là ta part dans la vie, au milieu de ton travail que tu fais sous le soleil.
\VS{10}Tout ce que ta main trouve à faire, fais-le selon ton pouvoir~; car dans le scheol, où tu vas, il n'y a ni œuvre, ni pensée, ni connaissance, ni sagesse.
\VS{11}Je me suis tourné ailleurs, et j'ai vu sous le soleil que la course n'est point aux légers, ni la guerre aux héros, ni le pain aux sages, ni la richesse à ceux qui sont intelligents, ni la grâce aux savants~; mais que le temps et les circonstances décident de ce qui arrive à tous.
\VS{12}Car l'homme ne connaît pas son heure, comme les poissons qui sont pris au filet de malheur et les oiseaux qui sont pris au piège~; comme eux, les fils de l'homme sont enlacés au temps du malheur, lorsqu'il tombe subitement sur eux.
\VS{13}J'ai aussi vu cette sagesse sous le soleil, et elle m'a semblé grande.
\VS{14}Il y avait une petite ville, avec peu d'hommes dans son sein~; un roi puissant marcha contre elle, l'investit et bâtit de grands forts contre elle.
\VS{15}Mais il s'y trouvait un homme pauvre et sage qui délivra la ville par sa sagesse. Et personne ne s'est souvenu de cet homme pauvre.
\VS{16}Alors j'ai dit~: La sagesse vaut mieux que la force. Cependant, la sagesse du pauvre est méprisée, et ses paroles ne sont point écoutées.
\VS{17}Les paroles des sages doivent être écoutées plus paisiblement que le cri de celui qui domine parmi les insensés. 
\VS{18}Mieux vaut la sagesse que tous les instruments de guerre~; et un seul homme pécheur détruit beaucoup de bien.
\Chap{10}
\TextTitle{La sagesse vaut mieux que la folie}
\VerseOne{} Les mouches mortes font puer et fermenter les parfums du parfumeur~; et un peu de folie produit le même effet à l'égard de celui qui est estimé pour sa sagesse, et pour sa gloire.
\VS{2}Le cœur du sage est à sa droite, et le cœur de l'insensé est à sa gauche.
\VS{3}Et même quand l'insensé se met en chemin, le sens lui manque~; et il dit de chacun~: Il est insensé. 
\VS{4}Si l'esprit de celui qui domine s'élève contre toi, ne sors point de ta condition~; car la douceur fait pardonner de grandes fautes. 
\VS{5}Il y a un mal que j'ai vu sous le soleil, comme une erreur qui procède du prince~:
\VS{6}C'est que la folie est mise aux plus hauts lieux, et que les riches sont assis dans un lieu bas. 
\VS{7}J'ai vu des serviteurs sur des chevaux, et des princes marchant sur terre comme des serviteurs.
\VS{8}Celui qui creuse la fosse y tombera, et celui qui coupe la haie, le serpent le mordra\FTNT{Ps. 7:15~; Pr. 26:27~; Pr. 28:10.}.
\VS{9}Celui qui remue des pierres hors de leur place, en sera blessé, et celui qui fend du bois se met en danger.
\VS{10}Si le fer est émoussé, et qu'on n'en ait point aiguisé le tranchant, il devra redoubler de force~; mais la sagesse a l'avantage du succès.
\VS{11}Si le serpent mord sans faire du bruit, le médisant ne vaut pas mieux. 
\VS{12}Les paroles de la bouche du sage ne sont que grâce, mais les lèvres de l'insensé le réduisent à néant\FTNT{Pr. 10:21.}.
\VS{13}Le commencement des paroles de sa bouche est folie, et la fin de son discours est une méchante folie.
\VS{14}Or l'insensé multiplie les paroles. L'homme ne sait point ce qui arrivera, et qui lui déclarera ce qui sera après lui~?
\VS{15}Le travail de l'insensé le fatigue, parce qu'il ne sait pas aller à la ville.
\VS{16}Malheur à toi, pays dont le roi est un enfant, et dont les princes mangent dès le matin\FTNT{Es. 3:4.}~!
\VS{17}Que tu es béni, ô pays~! Si ton roi est de race illustre, et si tes gouverneurs mangent au temps convenable, pour leur réfection et non pour se livrer à la débauche~! 
\VS{18}A cause des mains paresseuses, la charpente s'affaisse~; et à cause des mains lâches, la maison a des gouttières.
\VS{19}On fait des pains pour se réjouir et le vin réjouit les vivants et l'argent répond à tout.
\VS{20}Ne maudis point le roi, même dans ta pensée, et ne maudis pas le riche dans la chambre où tu couches~; car l'oiseau du ciel emporterait ta voix, le Baal ailé\FTNT{Baal ailé~: Le terme sémitique «~baal~» (en hébreu ba'al) signifie à l'origine «~possesseur~», «~maître~» ou «~seigneur~». Le Baal ailé était une créature ailée. Utilisé au pluriel, l'expression «~baalim de flèches~» désignait des archers. Les écritures nous parlent de Baal-Zebub (seigneur des mouches), un démon adoré à Ekron, l'une des villes des Philistins (2 R. 1:1-16). Baal-Zebud à donné «~Béelzébul~» dans les Evangiles (Mt. 10:25~; Mt. 12:24~; Mt. 12:27~; Lu. 11:15-19). Ce passage nous enseigne clairement que les démons épient les enfants de Dieu et vont ensuite faire leurs rapports à Satan afin de mieux les attaquer. Ils agissent comme des espions. Ces esprits sont comme des mouches et essayent de s'infiltrer partout.} rapporterait tes paroles.
\Chap{11}
\TextTitle{L'homme travaille en tâtonnant}
\VerseOne{}Jette ton pain à la face des eaux, car avec le temps tu le retrouveras.
\VS{2}Donnes-en une part à sept et même à huit, car tu ne sais point quel mal viendra sur la terre.
\VS{3}Quand les nuages sont pleins, ils répandent la pluie sur la terre~; et quand un arbre tombe, au sud ou au nord, il reste à la place où il est tombé.
\VS{4}Celui qui prend garde au vent, ne sèmera point~; et celui qui regarde les nuées, ne moissonnera point. 
\VS{5}Comme tu ne sais point quel est le chemin du vent, ni comment se forment les os dans le ventre de celle qui est enceinte, ainsi tu ne connais pas l'œuvre de Dieu qui fait tout\FTNT{Ceux qui sont nés d'en-haut sont insaisissables comme le vent (Jn. 3:8).}.
\VS{6} Sème ta semence dès le matin, et ne laisse pas reposer tes mains le soir~; car tu ne sais point lequel sera le meilleur, ceci ou cela~; et si tous deux seront pareillement bons.
\VS{7}Il est vrai que la lumière est douce, et qu'il est agréable aux yeux de voir le soleil.
\VS{8}Mais si un homme vit de nombreuses années, qu'il se réjouisse, et qu'il se souvienne des jours de ténèbres qui seront en grand nombre, tout ce qui lui arrivera est vanité.
\Chap{12}
\TextTitle{Message à la jeunesse}
\VerseOne{}Jeune homme, réjouis-toi dans ton jeune âge, et que ton cœur te rende gai aux jours de ta jeunesse, et marche comme ton cœur te mène, et selon le regard de tes yeux~; mais sache que pour toutes ces choses Dieu t'amènera en jugement. 
\VS{2}Ôte le chagrin de ton cœur, et éloigne de toi le mal~; car le jeune âge et l'adolescence ne sont que vanité. 
\VS{3}Mais souviens-toi de ton Créateur pendant les jours de ta jeunesse, avant que les jours mauvais arrivent et que viennent les années où tu diras~: Je n'y prends point de plaisir~;
\VS{4}avant que le soleil et la lumière, la lune et les étoiles s'obscurcissent, et que les nuages reviennent après la pluie.
\VS{5}Lorsque les gardes de la maison tremblent\FTNT{«~Ceux qui gardent la maison~» représentent les mains.}, et que les hommes forts\FTNT{«~Les hommes forts~» sont les jambes.} se courbent, et que celles qui moulent\FTNT{Les dents sont celles qui moulent.} cessent de travailler parce qu'elles sont diminuées, et quand ceux qui regardent par les fenêtres\FTNT{Les yeux sont ceux qui regardent par les fenêtres.} sont obscurcis.
\VS{6}Et quand les deux battants de la porte\FTNT{Les oreilles sont les deux battants de la porte.} se ferment sur la rue quand s'abaisse le bruit de la meule, quand on se lève au chant de l'oiseau, et que toutes les chanteuses s'affaiblissent.
\VS{7}Quand aussi on craint ce qui est élevé, et qu'on tremble en chemin, quand l'amandier fleurit, et quand les cigales deviennent pesantes, et que l'appétit s'en ira, car l'homme s'en va vers sa maison éternelle\FTNT{La maison éternelle c'est la Nouvelle Jérusalem pour les chrétiens (Ap. 21.) et pour les païens le lac de feu (Ap. 20:11-15).}, et ceux qui pleurent font le tour des rues.
\VS{8}Avant que la corde d'argent\FTNT{Cette corde est comme le cordon ombilical, elle lie l'âme au corps. Lors de la mort, la corde d'argent est coupée.} se détache, que le vase d'or\FTNT{Le corps humain est comme un vase ou une tente qui renferme son esprit. Comme l'argile dans la main du potier, ainsi est l'homme dans celle de Dieu. Avec cette argile, il décide souverainement de fabriquer de la même masse un vase d'honneur et un autre pour un usage vil (Jé. 18:4-6~; Ro. 9:21~; 2 Ti. 2:20-21).} se brise, que la cruche se rompe sur la source, que la roue s'écrase sur la citerne~;
\VS{9}avant que la poussière retourne dans la terre, comme elle y avait été, et que l'esprit retourne à Dieu qui l'a donné.
\TextTitle{Conclusion}
\VS{10}Vanité des vanités, dit l'Ecclésiaste, tout est vanité.
\VS{11}Plus l'Ecclésiaste a été sage, plus il a enseigné la science au peuple~; il a fait entendre, il a recherché et mis en ordre plusieurs graves sentences. 
\VS{12}L'Ecclésiaste a cherché pour trouver des discours agréables~; mais ce qui en a été écrit ici, est la droiture même~; ce sont des paroles de vérité. 
\VS{13}Les paroles des sages sont comme des aiguillons, et les maîtres qui en ont fait des recueils, sont comme des clous plantés, et ces choses ont été données par un maître.
\VS{14}Mon fils, garde-toi de ce qui est au-delà de ceci~; car il n'y a point de fin à faire plusieurs livres, et tant d'étude n'est que travail qu'on se donne. 
\VS{15}Voici la conclusion de tout le discours qui a été entendu~: Crains Dieu, et garde ses commandements~; car c'est là le tout de l'homme.
\VS{16}Parce que Dieu amènera toute œuvre en jugement, au sujet de tout ce qui est caché, soit bien, soit mal.
\PPE{}
\end{multicols}

%\clearpage\ShortTitle{Esther}\BookTitle{Esther}\BFont
\noindent\hrulefill
{\footnotesize
\textit{
\bigskip
{\centering{}
\\(Ecter)
\\Signifie : Etoile (persan) ; Myrthe (hébreu)
\\Thème : Délivrance des juifs de l’extermination
\\Auteur : Inconnu
\\Date de rédaction : 5ème siècle av. J.C.\\}
}
%\bigskip
\textit{
\\Dernier livre à caractère historique du Tanahk, l’histoire d’Esther se déroula à Suse, capitale du royaume de Perse. En ce temps, le peuple d’Israël était dispersé et le roi Assuérus régnait sur un large territoire allant de l’Inde à l'Ethiopie.
%\bigskip
\\Ce livre raconte la vie d’Esther, son ascension au trône royal où elle succéda à la reine Vasthi et la manière dont elle fut utilisée pour éviter le génocide du peuple juif.  
%\bigskip
\\Bien que ne comportant pas le nom de Dieu ni d’allusion à une œuvre spirituelle, hormis le jeûne, ce récit met en évidence le secours divin.\bigskip
}
}
\par\nobreak\noindent\hrulefill
\begin{multicols}{2}
\TextTitle{[Un festin de sept jour au palais de Suse]}
\Chap{1}
\VerseOne{}Or il arriva qu’au temps d’Assuérus, de cet Assuérus qui régnait depuis les Indes jusqu'en Ethiopie, sur cent vingt-sept provinces ;
\VS{2}[Il arriva, dis-je], en ce temps-là, que le roi Assuérus était assis sur le trône royal à Suse, dans la capitale.
\VS{3}La troisième année de son règne, il fit un festin à tous les principaux princes de ses pays ; à ses serviteurs, à l’armée des Perses et de Mèdes, aux nobles et aux chefs des provinces qui furent réunis devant lui,
\VS{4}pour leur montrer la gloire de la richesse de son royaume et la splendeur de sa grandeur, durant plusieurs jours, pendant cent quatre-vingts jours.
\VS{5}Lorsque ces jours furent achevés, le roi fit pour tout le peuple qui se trouvait à Suse, la capitale, depuis le plus grand jusqu'au plus petit, un festin pendant sept jours, dans la cour du jardin du palais royal.
\VS{6}Des étoffes blanches, vertes et violettes, étaient attachées par des cordons de byssus et de pourpre à des anneaux d'argent et à des colonnes de marbre. Les lits étaient d'or et d'argent sur un pavé de porphyre, de marbre, de nacre, et de pierres noires.
\VS{7}On servait à boire dans des vases d'or, de différentes espèces, et il y avait du vin royal en abondance, selon la libéralité du roi.
\VS{8}On ne forçait personne à boire, car le roi avait ordonné à tous les chefs de sa maison de se conformer à la volonté de chacun.
\VS{9}La reine Vasthi fit aussi un festin aux femmes dans la maison royale du roi Assuérus.
\TextTitle{[Destitution de la reine Vasthi]}
\VS{10}Or le septième jour, le cœur du roi était réjoui par le vin, il ordonna à Mehuman, Biztha, Harbona, Bigtha, Abagtha, Zéthar, et Carcas, les sept eunuques qui servaient devant le roi Assuérus,
\VS{11}d’amener en sa présence la reine Vasthi, portant la couronne royale, afin de montrer sa beauté aux peuples et aux princes, car elle était belle de figure.
\VS{12}Les eunuques transmirent l’ordre du roi à la reine Vasthi, mais elle refusa de venir. Et le roi fut très irrité, et il s’enflamma de colère.
\VS{13}Alors le roi dit aux sages qui avaient la connaissance des temps. Car le roi traitait ainsi les affaires en présence de tous ceux qui connaissaient les lois et le droit.
\VS{14}Il avait auprès de lui, Carschena, Schéthar, Admatha, Tarsis, Mérès, Marsena, Memucan, sept princes de Perse et de Médie, qui voyaient la face du roi et qui occupaient le premier rang dans le royaume.
\VS{15}Que faut-il faire dit-il, selon les lois, à la reine Vasthi, pour n'avoir pas observé l’ordre que le roi Assuérus lui a ordonné par les eunuques ?
\VS{16}Alors Memucan répondit en présence du roi et des princes : La reine Vasthi n'a pas seulement mal agi contre le roi, mais aussi contre tous les princes et tous les peuples qui sont dans toutes les provinces du roi Assuérus.
\VS{17}Car l'action de la reine parviendra à la connaissance de toutes les femmes, et les portera à mépriser leurs maris ; elles diront : Le roi Assuérus avait ordonné qu'on fasse venir en sa présence la reine, et elle n'y est pas allée.
\VS{18}Dès ce jour, les princesses de Perse et de Médie qui auront appris l’action de la reine répondront de même à tous les princes du roi ; ce sera une marque de mépris et un sujet de colère.
\VS{19}Si le roi le trouve bon, qu'un édit royal soit publié de sa part, et qu'il soit écrit parmi les lois de Perse et de Médie, avec défense de la transgresser, que Vasthi ne vienne plus devant le roi Assuérus et le roi donnera sa royauté à une compagne, qui sera meilleure qu'elle.
\VS{20}L’édit du roi sera présenté et connu dans tout son royaume, quelque grand qu'il soit, toutes les femmes honoreront leurs maris\FTNT{Respect ou soumission de la femme à l’égard de son mari : Ep. 5 : 22 ; Col. 3 : 18 ; Ti. 2 : 5 ; 1 Pi. 3 : 1-5.}, depuis le plus grand jusqu'au plus petit.
\VS{21}Cette parole plut au roi et aux princes, et le roi fit selon la parole de Memucan.
\VS{22}Il envoya des lettres à toutes les provinces du royaume, à chaque province selon son écriture et à chaque peuple selon sa langue ; elles portaient que tout homme devait être le maître de sa maison\FTNT{L’homme, chef de la femme et maître de la maison : 1 Co. 11 : 3 ; Ep. 5 : 23.}, et qu’il parlerait la langue de son peuple.
\TextTitle{[Le roi choisit une autre reine]}
\Chap{2}
\VerseOne{}Après ces choses, quand la colère du roi Assuérus fut calmée, il se souvint de Vasthi, de ce qu'elle avait fait, et de ce qui avait été décrété à son sujet.
\VS{2}Les serviteurs qui servaient le roi dirent : Qu'on cherche pour le roi des jeunes filles, vierges, et belles de figure.
\VS{3}Que le roi désigne des commissaires dans toutes les provinces de son royaume chargés de rassembler toutes les jeunes filles, vierges et belles de figure, dans Suse, la capitale, dans la maison des femmes sous la charge d'Hégué, eunuque du roi et gardien des femmes, qu'on leur donne les parfums nécessaires pour leur toilette ;
\VS{4}et la jeune fille qui plaira au roi régnera à la place de Vasthi. Ce discours plût au roi, et il fit ainsi.
\VS{5}Or, il y avait à Suse, la capitale, un juif nommé Mardochée, fils de Jaïr, fils de Schimeï, fils de Kis, Benjamite,
\VS{6}qui avait été emmené de Jérusalem\FTNT{La captivité babylonienne : Voir 2 R. 24.}, parmi les captifs déportés avec Jeconia, roi de Juda, par Nebucadnetsar, roi de Babylone.
\VS{7}Il élevait Hadassa, qui est Esther, fille de son oncle ; car elle n'avait ni père ni mère. La jeune fille était belle de taille et très belle de figure. Après la mort de son père et de sa mère, Mardochée l'avait prise pour fille.
\VS{8}Lorsqu’on eut publié l’ordre du roi et son édit, un grand nombre de jeunes filles furent rassemblées à Suse, la capitale, sous la charge d'Hégaï. Esther fut aussi amenée dans la maison du roi, sous la charge d'Hégaï, gardien des femmes.
\VS{9}La jeune fille lui plut, et trouva grâce à ses yeux, il s’empressa de lui fournir les parfums nécessaires pour sa toilette, et pour sa subsistance, lui donna sept jeunes filles choisies, et établies dans la maison du roi, il lui fit changer d'appartement, et la logea, elle et ses servantes, dans le meilleur des appartements de la maison des femmes.
\VS{10}Esther ne fit connaître ni son peuple ni sa parenté, car Mardochée lui avait ordonné de ne rien raconter.
\VS{11}Tous les jours Mardochée allait et venait devant la cour de la maison des femmes, pour savoir comment se portait Esther, et comment on s’occupait d'elle.
\VS{12}Chaque jeune fille allait à son tour vers le roi Assuérus, après s’être conformée au décret concernant les femmes pendant douze mois\FTNT{Esther se soumit à une toilette particulière avant de rencontrer le roi. Le mot « toilette » vient de l’hébreu « tam-rook », qui signifie « grattement ». La racine de ce mot signifie « nettoyer », « purifier », « polir » (voir Lé. 6 : 28 ; Jé. 46 : 4). Ce grattage symbolise le dépouillement du vieil homme et le renoncement aux œuvres de la chair (Ep. 4 : 22).
Douze mois étaient nécessaires pour préparer Esther aux noces : Six mois avec de l’huile de myrrhe et six mois avec des aromates et des parfums. La myrrhe était l’une des composantes de l’onction sainte dont on s’est servi pour oindre notamment la tente d’assignation, l’arche du témoignage ainsi qu’Aaron et ses fils (Ex. 30 : 23-30). Cet aspect de la toilette d’Esther nous parle de la sanctification sans laquelle nul ne peut voir le Seigneur (Hé. 12 : 14). La myrrhe est par ailleurs citée à sept reprises dans le livre du Cantique des cantiques, véritable hymne de l’amour parfait qui lie Christ à son Eglise. Le parfum quant à lui symbolise les prières que nous devons faire en tout temps afin de maintenir notre communion avec Jésus, notre époux (Ap. 5 : 8 ; Ap. 8 : 4 ; Ep. 6 : 18 ; 1 Th. 5 : 17).
Ainsi, à l’instar d’Esther qui se préparait à rencontrer le roi, l’Eglise se prépare depuis deux mille ans pour les noces de l’agneau (Ap. 19 : 7-9).}. C'est ainsi que s'accomplissaient les jours de leurs préparatifs, six mois avec de l'huile de myrrhe, et six autres mois avec des aromates et des parfums en usage parmi les femmes.
\VS{13}C'est ainsi que la jeune fille entrait vers le roi ; et, quand elle passait de la maison des femmes à la maison du roi, on lui laissait prendre ce qu’elle voulait.
\VS{14}Elle y entrait le soir, et le matin elle retournait dans la seconde maison des femmes sous la charge de Schaaschgaz, eunuque du roi et gardien des concubines. Elle ne retournait plus vers le roi, à moins que le roi n’en ait le désir et qu'elle soit appelée par son nom.
\TextTitle{[Esther, reine de Suse]}
\VS{15}Quand son tour d’aller vers le roi fut arrivé, Esther, fille d'Abichaïl, oncle de Mardochée qui l’avait prise pour sa fille, ne demanda rien sinon ce qui fut ordonné par Hégaï, eunuque du roi et gardien des femmes. Esther trouva grâce aux yeux de tous ceux qui la voyaient.
\VS{16}Ainsi Esther fut amenée auprès du roi Assuérus, dans sa maison royale, le dixième mois, qui est le mois de Tébeth, la septième année de son règne.
\VS{17}Le roi aima Esther plus que toutes les autres femmes, elle obtint sa grâce et sa bienveillance plus que toutes les vierges. Il mit la couronne royale sur sa tête, et l'établit reine à la place de Vasthi.
\VS{18}Le roi fit alors un grand festin à tous les princes de ses pays, et à ses serviteurs, un festin en l’honneur d'Esther ; il donna du repos aux provinces, et fit des présents selon la puissance du roi.
\VS{19}Or pendant qu'on assemblait les vierges pour la seconde fois, Mardochée s’assit à la porte du roi.
\VS{20}Esther n’avait fait connaître ni sa parenté ni son peuple, car Mardochée le lui avait défendu. Elle faisait tout ce que lui disait Mardochée, comme à l’époque où elle était élevée par lui.
\TextTitle{[Mardochée sauve la vie du roi]}
\VS{21}En ces jours-là, Mardochée s’assit à la porte du roi, Bigthan et Théresch, deux eunuques du roi, gardes du seuil, s’irritèrent et cherchèrent à mettre la main sur le roi Assuérus.
\VS{22}Mardochée ayant eu connaissance de l’affaire, informa la reine Esther, qui le redit au roi de la part de Mardochée.
\VS{23}On vérifia l’affaire et on trouva que cela était exact, les deux eunuques furent pendus à un bois, et cela fut écrit dans le livre des chroniques en présence du roi.
\TextTitle{[Conspiration de Haman contre les Juifs]}
\Chap{3}
\VerseOne{}Après ces choses, le roi Assuérus fit de grands honneurs à Haman, fils d'Hammedatha, l’Agaguite ; il l'éleva en dignité et plaça son siège au-dessus de tous les princes qui étaient auprès de lui.
\VS{2}Tous les serviteurs du roi qui étaient à la porte du roi s'inclinaient et se prosternaient devant Haman, car le roi l’avait ainsi ordonné. Mais Mardochée ne s'inclinait pas et ne se prosternait pas devant lui.
\VS{3}Les serviteurs du roi, qui étaient à la porte du roi, disaient à Mardochée : Pourquoi transgresses-tu l’ordre du roi ?
\VS{4}Comme ils le lui répétaient chaque jour et qu'il ne les écoutait pas, ils le rapportèrent à Haman, pour voir si Mardochée tiendrait ferme dans sa résolution ; car il leur avait déclaré qu'il était juif.
\VS{5}Haman vit que Mardochée ne s'inclinait pas et ne se prosternait pas devant lui et il fut rempli de colère.
\VS{6}Mais il dédaigna de porter la main sur Mardochée seul, car on lui avait rapporté de quel peuple était Mardochée. Haman chercha à exterminer tous les juifs, le peuple de Mardochée qui se trouvait dans tout le royaume d'Assuérus.
\VS{7}Au premier mois, qui est le mois de Nissan, la douzième année du roi Assuérus, on jeta le pur, c'est-à-dire le sort, devant Haman, pour chaque jour et pour chaque mois, jusqu’au douzième mois, qui est le mois d'Adar.
\VS{8}Haman dit au roi Assuérus : Il y a un peuple dispersé dans toutes les provinces de ton royaume, qui se tient à part parmi les peuples. Leurs lois sont différentes de celles de tous les autres peuples, ils n’observent pas les lois du roi. Il n'est pas dans l’intérêt du roi de le laisser en repos.
\VS{9}S'il plaît au roi, qu'on écrive l’ordre de les faire périr, et je pèserai dix mille talents d'argent entre les mains de ceux qui s’occupent des affaires, pour les porter dans le trésor du roi.
\VS{10}Le roi ôta son anneau de sa main, et le donna à Haman fils de Hammedatha, l’Agaguite, l’adversaire des Juifs.
\VS{11}Outre cela, le roi dit à Haman : Cet argent t'est donné avec ce peuple ; fais-en ce que tu voudras.
\VS{12}Le treizième jour du premier mois, les secrétaires du roi furent appelés, et on écrivit selon l’ordre d'Haman, aux satrapes du roi, aux gouverneurs de chaque province et aux princes de chaque peuple, à chaque province selon son écriture et à chaque peuple selon sa langue. Ce fut au nom du roi Assuérus que l’on écrivit, et on scella avec l'anneau du roi.
\VS{13}Les lettres furent envoyées par des coureurs dans toutes les provinces du roi, afin qu'on extermine, qu’on tue et qu’on fasse périr tous les juifs, jeunes et vieux, petits enfants et femmes, en un seul jour, le treizième du douzième mois, qui est le mois d'Adar, et pour que leurs biens soient livrés au pillage.
\VS{14}Ces lettres qui furent écrites portaient une copie de l’édit, qui devait être publié dans chaque province, et invitaient publiquement tous les peuples, à se tenir prêts pour ce jour-là.
\VS{15}Ainsi les coureurs partirent en toute hâte d’après l’ordre du roi. L'édit fut aussi publié dans Suse, la capitale. Or le roi et Haman étaient assis pour boire, pendant que la ville de Suse était dans la confusion.
\TextTitle{[Esther avertie du complot d'Haman]}
\Chap{4}
\VerseOne{}Mardochée, ayant appris ce qui se passait, déchira ses vêtements et se couvrit d'un sac et de la cendre. Puis il alla au milieu de la ville en poussant avec force des cris amers,
\VS{2}et se rendit jusqu'à la porte du roi, or il était interdit d'entrer dans le palais du roi revêtu d'un sac.
\VS{3}Dans chaque province, partout où arrivait l’ordre du roi et son édit, il y eut une grande désolation parmi les juifs ; ils jeûnaient, pleuraient, gémissaient, et beaucoup se couchaient sur le sac et la cendre.
\VS{4}Les servantes d'Esther et ses eunuques vinrent lui raconter ces choses, et la reine fut très effrayée. Elle envoya des vêtements à Mardochée pour le couvrir et lui faire ôter son sac, mais il ne les prit pas.
\VS{5}Alors Esther appela Hathac, l'un des eunuques que le roi avait établi pour la servir, et elle le chargea de demander à Mardochée ce qui s’était passé et pourquoi il agissait ainsi.
\VS{6}Hathac sortit donc vers Mardochée sur la place de la ville, devant la porte du roi.
\VS{7}Mardochée lui raconta tout ce qui lui était arrivé, et la somme d'argent qu'Haman avait promis de payer comptant au trésor du roi, pour la destruction des juifs.
\VS{8}Il lui donna aussi une copie de l'édit publié dans Suse en vue de leur extermination, afin qu’il le montre à Esther et lui fasse tout connaître ; et il ordonna qu’Esther se rende chez le roi pour implorer sa miséricorde, et faire une requête en faveur de son peuple.
\VS{9}Hathac vint rapporter à Esther les paroles de Mardochée.
\TextTitle{[Mardochée incite Esther à risquer sa vie pour ses frères]}
\VS{10}Esther chargea Hathac de dire à Mardochée :
\VS{11}Tous les serviteurs du roi et le peuple des provinces du roi savent qu'il existe une loi prescrivant la peine de mort contre quiconque, homme ou femme, entre chez le roi, dans la cour intérieure sans avoir été appelé ; à moins que le roi ne lui tende le sceptre d'or, celui-là a la vie sauve. Or il y a déjà trente jours que je n'ai pas été appelée pour entrer chez le roi.
\VS{12}On rapporta les paroles d'Esther à Mardochée.
\VS{13}Mardochée fit cette réponse à Esther : Ne t’imagine pas que tu échapperas seule d'entre tous les juifs parce que tu es dans la maison du roi.
\VS{14}Mais si tu te tais et gardes le silence en ce temps-ci, les juifs seront secourus et délivrés par un autre moyen, mais toi et la maison de ton père vous périrez. Et qui sait si tu n'es pas arrivée à la royauté pour un temps comme celui-ci ?
\TextTitle{[Esther demande un jeûne]}
\VS{15}Esther fit cette réponse à Mardochée :
\VS{16}Va, rassemble tous les juifs qui se trouvent à Suse, et jeûnez pour moi, sans manger ni boire pendant trois jours, ni la nuit ni le jour. Moi aussi et mes servantes nous jeûnerons de même, puis j'entrerai chez le roi, malgré la loi ; et si je dois périr, je périrai.
\VS{17}Mardochée s'en alla, et fit comme Esther lui avait ordonné.
\TextTitle{[Esther se présente devant le roi]}
\Chap{5}
\VerseOne{}Le troisième jour, Esther mit des vêtements royaux et se présenta dans la cour intérieure de la maison du roi, devant la maison du roi. Le roi était assis sur le trône dans la maison royale, en face de l’entrée de la maison.
\VS{2}Dès que le roi vit la reine Esther debout dans la cour, elle trouva grâce à ses yeux ; le roi tendit à Esther le sceptre d'or qui était dans sa main. Esther s'approcha, et toucha le bout du sceptre.
\VS{3}Le roi lui dit : Qu'as-tu, reine Esther, et que demandes-tu ? Quand ce serait la moitié du royaume, elle te serait donnée.
\VS{4}Esther répondit : Si le roi le trouve bon, que le roi vienne aujourd'hui avec Haman au festin que je lui ai préparé.
\VS{5}Alors le roi dit : Qu'on fasse venir en toute hâte, Haman, pour accomplir la parole d'Esther. Le roi vint donc avec Haman au festin qu'Esther avait préparé.
\VS{6}Le roi dit à Esther, pendant qu’on buvait le vin : Quelle est ta demande ? Elle te sera accordée. Quelle est ta requête ? Quand ce serait la moitié du royaume, tu l’obtiendras.
\VS{7}Esther répondit et dit : Voici ce que je demande et ce que je désire.
\VS{8}Si j'ai trouvé grâce aux yeux du roi, et si le roi trouve bon d'accorder ma requête, que le roi et Haman viennent au festin que je leur préparerai, et je donnerai demain une réponse au roi selon sa parole.
\VS{9}Haman sortit ce jour-là, joyeux et le cœur content. Mais aussitôt qu'il vit, à la porte du roi, Mardochée, qui ne se levait ni ne tremblait devant lui, il fut rempli de colère contre Mardochée.
\VS{10}Il sut toutefois se contenir, et il alla dans sa maison. Puis il envoya chercher ses amis et Zéresch, sa femme.
\VS{11}Haman leur parla de la magnificence de ses richesses, du nombre de ses fils, et tout ce qu’avait fait le roi pour le rendre puissant, et comment il l'avait élevé au-dessus des princes et des serviteurs du roi.
\VS{12}Puis Haman ajouta : Même la reine Esther n'a fait venir que moi et le roi au festin qu'elle a fait, et je suis encore invité demain chez elle avec le roi.
\VS{13}Mais tout cela n’est d’aucun intérêt, aussi longtemps que je verrai Mardochée, le juif, assis à la porte du roi.
\VS{14}Zéresch sa femme, et tous ses amis lui répondirent : Qu'on prépare un bois haut de cinquante coudées, et demain matin dis au roi qu'on y pende Mardochée ; et tu iras joyeux au festin avec le roi. Cette parole plut à Haman, et il fit préparer le bois.
\TextTitle{[Le roi Assuérus se souvient de Mardochée]}
\Chap{6}
\VerseOne{}Cette nuit-là, le roi ne put dormir, il fit apporter le livre des annales, les chroniques. On les lut devant le roi,
\VS{2}et l’on trouva écrit ce que Mardochée avait rapporté au sujet de la conspiration de Bigthan et de Théresch, les deux eunuques du roi, gardes du seuil, qui avaient cherché à mettre la main sur le roi Assuérus.
\VS{3}Le roi dit : Quel honneur et quelle distinction a-t-on accordé à Mardochée pour cela ? Il n’a rien reçu répondirent les serviteurs du roi.
\VS{4}Le roi dit : Qui est dans la cour ? Haman était venu dans la cour extérieure de la maison du roi, pour demander au roi de pendre Mardochée au bois qu'il avait préparé.
\VS{5}Les serviteurs du roi répondirent : C’est Haman qui se tient dans la cour. Et le roi dit : Qu'il entre.
\VS{6}Haman entra, et le roi lui dit : Que faudrait-il faire à un homme que le roi désire honorer ? Haman se dit en lui-même : A qui le roi voudrait-il faire plus honneur qu'à moi ?
\VS{7}Haman répondit au roi : Pour un homme que le roi désire honorer,
\VS{8}qu’on lui apporte le vêtement royal, dont le roi se revêt, et qu'on lui amène le cheval que le roi monte, et qu'on lui mette la couronne royale sur la tête.
\VS{9}Et qu'ensuite on donne ce vêtement et ce cheval à quelqu'un des principaux et des plus grands chefs qui sont auprès du roi, et qu'on revête l'homme que le roi prend plaisir d'honorer, et qu'on le fasse aller à cheval par les rues de la ville ; et qu'on crie devant lui : C'est ainsi qu'on doit faire à l'homme que le roi prend plaisir d'honorer.
\VS{10}Alors le roi dit à Haman : Prends tout de suite le vêtement, et le cheval, comme tu l'as dit, et fais ainsi à Mardochée, le juif qui est assis à la porte du roi ; ne néglige rien de tout ce que tu as déclaré.
\VS{11}Et Haman prit le vêtement et le cheval, il revêtit Mardochée, il le promena à cheval à travers les rues de la ville, et il criait devant lui : C'est ainsi que l’on fait à l'homme que le roi désire honorer.
\VS{12}Mardochée retourna à la porte du roi, et Haman se retira en hâte dans sa maison, pleurant et ayant la tête voilée.
\VS{13}Haman raconta à Zéresch, sa femme, et à tous ses amis, tout ce qui lui était arrivé. Ses sages, et Zéresch, sa femme, lui répondirent : Si Mardochée devant lequel tu as commencé à tomber, est de la race des juifs, tu n'auras pas le dessus sur lui, mais tu tomberas certainement devant lui.
\VS{14}Comme ils parlaient encore avec lui, les eunuques du roi vinrent, et se hâtèrent d'amener Haman au festin qu'Esther avait préparé.
\TextTitle{[Esther plaide sa cause et celle de son peuple]}
\Chap{7}
\VerseOne{}Le roi et Haman allèrent au festin chez la reine Esther.
\VS{2}Le roi dit encore à Esther, ce second jour, pendant qu’on buvait le vin : Quelle est ta demande, reine Esther ? Elle te sera donnée. Que désires-tu ? Quand ce serait la moitié du royaume, cela te sera accordé.
\VS{3}Alors la reine Esther répondit, et dit : Si j'ai trouvé grâce à tes yeux, ô roi ! et si le roi le trouve bon, que ma vie me soit donnée à ma demande, et que mon peuple me soit donné à ma prière.
\VS{4}Car nous avons été vendus, mon peuple et moi, pour être détruits, tués, exterminés. Si nous avions été vendus pour être esclaves et serviteurs, j’aurais gardé le silence, bien que l'oppresseur ne saurait compenser le dommage fait au roi.
\VS{5}Le roi Assuérus parla et dit à la reine Esther : Qui est-il et où est l’homme dont le cœur est consacré à faire cela ?
\VS{6}Esther répondit : L'oppresseur, l'ennemi, c’est Haman, ce méchant ! Alors Haman fut terrifié en présence du roi et de la reine.
\TextTitle{[Haman pendu au gibet qu'il avait dressé]}
\VS{7}Le roi, dans sa colère, se leva et quitta le festin, il entra dans le jardin du palais. Haman resta pour demander grâce pour sa vie à la reine Esther, car il voyait bien que sa perte était résolue par le roi.
\VS{8}Puis le roi revint du jardin du palais dans la salle du festin, il vit Haman qui s’était précipité sur le lit où était Esther, et il dit : Serait-ce encore pour faire violence sous mes yeux à la reine dans cette maison ? Dès que la parole fut sortie de la bouche du roi, on voila le visage d'Haman.
\VS{9}Et Harbona, l'un des eunuques, dit en présence du roi : Voici, le bois préparé par Haman pour Mardochée, qui a parlé pour le bien du roi, est dressé dans la maison d'Haman, à une hauteur de cinquante coudées. Le roi dit : Qu’on y pende Haman !
\VS{10}On pendit Haman au bois qu'il avait préparé pour Mardochée. Et la colère du roi fut apaisée.
\TextTitle{[Un décret royal fait échouer le complot d'Haman]}
\Chap{8}
\VerseOne{}Ce même jour, le roi Assuérus donna à la reine Esther la maison d'Haman, l'oppresseur des juifs ; et Mardochée fut introduit devant le roi, car Esther avait déclaré quel était son lien de parenté avec elle.
\VS{2}Le roi ôta son anneau, qu'il avait repris à Haman, et le donna à Mardochée ; Esther établit Mardochée sur la maison d'Haman.
\VS{3}Esther parla encore en présence du roi. Elle se jeta à ses pieds, elle pleura, elle l’implora d’empêcher les effets de la méchanceté d'Haman, l’Agaguite, et la réussite de ses projets contre les juifs.
\VS{4}Le roi tendit le sceptre d'or à Esther qui se releva et resta debout devant le roi.
\VS{5}Elle dit : Si le roi le trouve bon, et si j'ai trouvé grâce devant lui, si mes paroles semblent convenables au roi et si je suis agréable à ses yeux, qu'on écrive pour révoquer les lettres conçues par Haman, fils d'Hammedatha, l’Agaguite, qu'il écrivit afin de détruire les juifs qui sont dans toutes les provinces du roi.
\VS{6}Car comment pourrais-je voir le mal qui atteindrait mon peuple, et comment pourrais-je voir la destruction de ma race ?
\VS{7}Le roi Assuérus dit à la reine Esther et au juif Mardochée : Voici, j'ai donné la maison d'Haman à Esther, et il a été pendu au bois pour avoir étendu sa main contre les juifs.
\VS{8}Ecrivez donc, au nom du roi, en faveur des juifs comme il vous plaira, et scellez l'écrit de l'anneau du roi ; car un édit écrit au nom du roi et scellé de l'anneau du roi ne peut être révoqué.
\VS{9}En ce temps, le vingt-troisième jour du troisième mois, qui est le mois de Sivan, les secrétaires du roi furent appelés, et on écrivit, comme Mardochée l’ordonna, aux juifs, aux satrapes, aux gouverneurs, et aux princes des cent vingt-sept provinces, de l’Inde jusqu'en Ethiopie, à chaque province selon son écriture, à chaque peuple selon sa langue, et aux juifs selon leur écriture et selon leur langue.
\VS{10}On écrivit les lettres au nom du roi Assuérus, et on les scella de l'anneau du roi. On les envoya par des coureurs, ayant pour montures des chevaux et des mulets nés de juments.
\VS{11}Par ces lettres, le roi accordait aux juifs, qui étaient dans chaque ville la permission de se rassembler et de défendre leur vie, de détruire, de tuer, et d’exterminer toute force armée du peuple et de quelque province que ce soit, qui prendraient les armes pour les attaquer, ainsi que leurs petits enfants et leurs femmes, et de piller leurs biens ;
\VS{12}et cela en un seul jour, dans toutes les provinces du roi Assuérus, le treizième jour du douzième mois, qui est le mois d'Adar.
\VS{13}Ces lettres écrites portaient une copie de l’édit qui devait être publié dans chaque province, et informaient tous les peuples que les juifs seraient prêts en ce jour à se venger de leurs ennemis.
\VS{14}Les coureurs, montés sur des chevaux et des mulets, partirent aussitôt et en toute hâte, d’après l’ordre du roi. L'édit fut aussi publié dans Suse, la capitale.
\TextTitle{[Mardochée honoré]}
\VS{15}Mardochée sortit de chez le roi, en vêtement royal violet et blanc, avec une grande couronne d'or, et une robe de byssus et de pourpre. La ville de Suse poussait des cris, et elle fut dans la joie.
\VS{16}Il eut pour les juifs du bonheur et de la joie, des réjouissances et des honneurs.
\VS{17}Dans chaque province et dans chaque ville, partout où arrivaient l’ordre du roi et son décret, il y eut pour les juifs de la joie, des réjouissances, des festins, et des fêtes. Et beaucoup de gens d'entre les peuples du pays se faisaient juifs, parce que la crainte des juifs les avait saisis.
\TextTitle{[Les juifs triomphent de leurs ennemis]}
\Chap{9}
\VerseOne{}Le douzième mois, qui est le mois d'Adar, le treizième jour du mois, où l’ordre du roi et son décret devaient être exécutés, au jour où les ennemis des juifs espéraient dominer, ce fut le contraire qui arriva, les juifs dominèrent sur leurs ennemis.
\VS{2}Les juifs se rassemblèrent dans leurs villes, dans toutes les provinces du roi Assuérus, pour mettre la main sur ceux qui cherchaient leur perte ; et personne ne put leur résister, car la crainte qu'on avait d'eux avait saisi tous les peuples.
\VS{3}Et tous les princes des provinces, les satrapes, les gouverneurs, et ceux qui s’occupaient des affaires du roi, soutenaient les juifs, à cause de la terreur que leur inspirait Mardochée.
\VS{4}Car Mardochée était puissant dans la maison du roi, et sa renommée se répandait dans toutes les provinces, parce qu’il devenait de plus en plus puissant.
\VS{5}Les juifs frappèrent tous leurs ennemis à coups d'épée, ils les tuèrent et les détruisirent ; ils traitèrent selon leurs désirs ceux qui les haïssaient.
\VS{6}Dans Suse, la capitale, les juifs tuèrent et firent périr cinq cents hommes.
\VS{7}Ils tuèrent aussi Parschandatha, Dalphon, Aspatha,
\VS{8}Poratha, Adalia, Aridatha,
\VS{9}Parmaschtha, Arizaï, Aridaï, et Vajezatha,
\VS{10}les dix fils d'Haman, fils d'Hammedatha, l'oppresseur des juifs. Mais ils ne mirent pas leurs mains au pillage.
\VS{11}Ce jour-là, on rapporta au roi le nombre de ceux qui avaient été tués dans Suse, la capitale.
\VS{12}Le roi dit à la reine Esther : Dans Suse, la capitale, les juifs ont tué et détruit cinq cents hommes, et les dix fils d'Haman, qu'auront-ils fait dans le reste des provinces du roi ? Quelle est ta demande ? Et elle te sera accordée. Que désires-tu encore ? Tu l’obtiendras.
\VS{13}Esther répondit : Si le roi le trouve bon qu'il soit permis aux juifs, qui sont à Suse, d’agir encore demain selon le décret d’aujourd'hui, et que l'on pende au bois les dix fils d'Haman.
\VS{14}Et le roi ordonna de faire ainsi. L'édit fut publié dans Suse. On pendit les dix fils d'Haman ;
\VS{15}et les juifs qui étaient dans Suse se rassemblèrent encore le quatorzième jour du mois d'Adar et tuèrent dans Suse trois cents hommes. Mais ils ne mirent pas la main au pillage.
\VS{16}Les autres juifs qui étaient dans les provinces du roi se rassemblèrent, et défendirent leur vie ; ils eurent du repos et furent délivrés de leurs ennemis, et ils tuèrent soixante-quinze mille hommes de ceux qui les haïssaient. Mais ils ne mirent pas la main au pillage.
\VS{17}Ces choses arrivèrent le treizième jour du mois d'Adar, et le quatorzième du même mois ils se reposèrent, et ils en firent un jour de festin et de joie.
\VS{18}Les juifs qui étaient dans Suse, s'assemblèrent le treizième et le quatorzième jour du même mois, et ils se reposèrent le quinzième jour, et ils en firent un jour de festin et de joie.
\VS{19}C'est pourquoi les juifs des campagnes qui habitent dans des villes sans murailles, font le quatorzième jour du mois d'Adar, un jour de réjouissance, de festin et de fête, où l’on s’envoie des portions les uns aux autres.
\TextTitle{[Esther confirme l'instauration la fête des Purim]}
\VS{20}Mardochée écrivit ces choses, et il envoya les lettres à tous les juifs qui étaient dans toutes les provinces du roi Assuérus, auprès et au loin.
\VS{21}Il leur prescrivait de célébrer chaque année le quatorzième jour et le quinzième jour du mois d'Adar.
\VS{22}Comme les jours où les juifs avaient obtenu du repos en se délivrant de leurs ennemis, de célébrer le mois où leur angoisse fut changée en joie, et leur deuil en jour heureux, et de faire de ces jours des jours de festin et de joie, où l’on s’envoie des portions les uns aux autres, et des dons aux pauvres.
\VS{23}Les juifs s’engagèrent à faire ce qu’ils avaient déjà commencé et ce que Mardochée leur prescrivit.
\VS{24}Car Haman, fils d'Hammedatha, l’Agaguite, l'oppresseur de tous les juifs, avait projeté de détruire les juifs, et il avait jeté le pur, c'est-à-dire le sort, afin de les détruire et de les tuer ;
\VS{25}mais Esther s’étant présentée devant le roi, le roi ordonna par écrit que le méchant projet qu'Haman avait imaginé contre les juifs, retombe sur sa tête, et qu'on le pende au bois, lui et ses fils.
\VS{26}C'est pourquoi on appelle ces jours-là purim, du nom de pur\FTNT{Pur ou purim : Ce terme signifie sort (Est. 3 : 7). La fête de purim a été instituée pour célébrer leur délivrance de l’extermination planifiée par Haman, à la suite de l’intervention héroïque d’Esther. Les juifs l’observent désormais chaque année le 14 du mois d’Adar (février ou mars) depuis le temps d’Esther jusqu’à ce jour.}. D’après tout le contenu de cette lettre, et selon ce qu’ils avaient eux-mêmes vu et ce qui leur était arrivé,
\VS{27}Les juifs établirent et adoptèrent pour eux, pour leur postérité, et pour tous ceux qui s’attacheraient à eux, l’engagement de ne pas manquer de célébrer chaque année ces deux jours, selon le mode prescrit et au temps fixé.
\VS{28}Ces jours devaient être rappelés et observés de génération en génération, dans chaque famille, dans chaque province et dans chaque ville ; et ces jours de Purim ne devaient jamais être abolis au milieu des juifs, ni le souvenir s’en effacer parmi leurs descendants.
\VS{29}La reine Esther, fille d'Abichaïl, écrivit aussi avec le juif Mardochée, de manière pressante pour la seconde fois, pour confirmer la lettre sur les Purim.
\VS{30}On envoya des lettres à tous les juifs, dans les cent vingt-sept provinces du royaume d'Assuérus. Elles contenaient des paroles de paix et de vérité,
\VS{31}pour établir ces jours de Purim au temps fixé, comme Mardochée le juif et la reine Esther les avaient établis pour eux, et comme ils les avaient établis pour eux-mêmes et pour leur postérité, à l’occasion de leur jeûne et de leurs cris.
\VS{32}Ainsi l'édit d'Esther confirma l’institution des Purim, et cela fut écrit dans le livre.
\TextTitle{[Mardochée établi dans la cours du roi]}
\Chap{10}
\VerseOne{}Le roi Assuérus imposa un tribut au pays, et aux îles de la mer.
\VS{2}Tous les faits concernant ses exploits, et les détails sur la grandeur à laquelle le roi éleva Mardochée, ne sont-ils pas écrits dans le livre des chroniques des rois de Médie et de Perse ?
\VS{3}Car Mardochée le juif était le premier après le roi Assuérus ; grand parmi les juifs et agréable à la multitude de ses frères, il chercha le bien-être de son peuple, et parla pour la paix de toute sa race.
\PPE{}
\end{multicols}

%\clearpage\ShortTitle{Daniel}\BookTitle{Daniel}\BFont
\noindent\hrulefill
{\footnotesize
\textit{
\bigskip
{\centering{}
\\Auteur : Daniel
\\(Heb. : Daniye'l)
\\Signification : Dieu est mon juge
\\Thème : Ascension et chute des royaumes
\\Date de rédaction : 6\up{ème} siècle av. J.-C.\\}
}
%\bigskip
\textit{
\\Issu d'une famille princière de Juda, Daniel fut déporté de Jérusalem à Babylone pendant sa jeunesse, sous le règne de Nebucadnestar. Lui et trois de ses amis – eux aussi de noble lignée - furent choisis pour être instruits selon la sagesse babylonienne en vue de servir le roi. Fervent dans sa foi en Yahweh, Daniel - imité ensuite par ses compagnons – résolut de ne point se souiller et obtint ainsi la faveur de son Dieu. Son intégrité et sa crainte de Dieu lui valurent de miraculeuses victoires, de nombreuses distinctions et une grande sagesse. Daniel avait reçu du discernement pour expliquer songes et visions et délivra plusieurs prophéties dont certaines se sont déjà accomplies, d'autres se réaliseront à la fin du temps des nations, au moment du retour de Christ.
%\bigskip
\\Dieu témoigna de la justice de Daniel au prophète Ezéchiel dont il fut contemporain.\bigskip
}
}
\par\nobreak\noindent\hrulefill
\begin{multicols}{2}
\Chap{1}
\TextTitle{Juda livré à la captivité babylonienne}
\VerseOne{}La troisième année du règne de Jojakim, roi de Juda, Nebucadnetsar, roi de Babylone, vint contre Jérusalem et l'assiégea.
\VS{2}Le Seigneur livra entre ses mains Jojakim\FTNT{En 597 av. J-C., la ville de Jérusalem tomba entre les mains des Babyloniens qui déportèrent le roi Jojakim et nommèrent comme roi à sa place son oncle Sédécias. Une petite partie de la population fut déportée à cette occasion. Cette première déportation ne concernait que l'élite administrative et sacerdotale : prêtres, scribes, hauts fonctionnaires, membres de la famille royale et artisans métallurgistes. Pour de nombreux historiens, il s'agissait moins d'une déportation que d'une constitution d'un groupe d'otages. Le roi, quelques membres de sa famille, et de diverses familles de notables, furent tenus en résidence surveillée à la cour babylonienne pour s'assurer que le royaume de Juda resterait pacifié.}, roi de Juda et une partie des vases de la maison de Dieu. Nebucadnetsar emporta les vases au pays de Schinear\FTNT{Schinear : « le pays des deux fleuves ». C'est l'ancien nom du territoire qui est devenu Babylonie ou Chaldée. C'est le pays de Nimrod (Ge. 10:6-12). C'est à Schinear qu'on tenta de construire la tour de Babel et de mettre en place le premier gouvernement mondial.}, dans la maison de son dieu, il les mit dans la maison du trésor de son dieu.
\VS{3}Le roi dit à Aschpenaz, capitaine de ses eunuques, d'amener quelques-uns des enfants d'Israël de race royale\FTNT{Daniel était de la race royale (2 R. 20:16-19 ; Es. 39:1-8).} et des principaux seigneurs,
\VS{4}quelques jeunes enfants en qui il n'y avait aucun défaut corporel, beaux de figure, instruits en toute sagesse, connaissant les sciences, pleins d'intelligence, et capables de se tenir dans le palais du roi ; et à qui l'on enseignerait les lettres et la langue des Chaldéens.
\VS{5}Le roi leur assigna pour provision chaque jour une portion de la viande royale et du vin dont il buvait, afin qu'on les nourrisse ainsi pendant trois ans au bout desquels ils se tiendraient devant le roi.
\VS{6}Il y avait parmi eux, d'entre les fils de Juda, Daniel, Hanania, Mischaël et Azaria.
\VS{7}Mais le capitaine des eunuques leur donna d'autres noms, il donna à Daniel le nom de Beltschatsar, à Hanania celui de Schadrac, à Mischaël celui de Méschac et à Azaria celui d'Abed-Nego.
\TextTitle{La fermeté de Daniel à Babylone}
\VS{8}Daniel résolut dans son cœur de ne pas se souiller par la portion de la viande du roi et par le vin dont le roi buvait; c'est pourquoi il supplia le chef des eunuques afin qu'il ne l'oblige pas à se souiller.
\VS{9}Et Dieu fit trouver à Daniel faveur et grâce auprès du chef des eunuques. 
\VS{10}Et le chef des eunuques dit à Daniel : Je crains le roi, mon seigneur, qui a fixé ce que vous devez manger et boire ; car pourquoi verrait-il vos visages plus défaits que ceux des jeunes gens de votre âge ? Vous exposeriez ma tête auprès du roi.
\VS{11}Mais Daniel dit à Meltsar, l'intendant à qui le chef des eunuques avait remis la surveillance de Daniel, Hanania, Mischaël et Azaria :
\VS{12}Eprouve, je te prie, tes serviteurs pendant dix jours, et qu'on nous donne des légumes à manger et de l'eau à boire.
\VS{13}Après cela, tu regarderas nos visages et ceux des jeunes enfants qui mangent la portion de la viande royale; puis tu feras à tes serviteurs selon ce que tu auras vu.
\VS{14}Et il les écouta dans cette affaire et les éprouva pendant dix jours.
\VS{15}Au bout des dix jours, leurs visages parurent en meilleur état et plus d'embonpoint que tous les jeunes gens qui mangeaient la portion de la viande royale.
\VS{16} Ainsi Meltsar prenait la portion de leur viande et le vin qu'ils devaient boire, et leur donnait des légumes.
\VS{17}Et Dieu donna à ces quatre jeunes gens de la science et de l'intelligence dans toutes les lettres, et de la sagesse ; et Daniel comprenait toutes les visions et tous les songes.
\VS{18}Et à la fin des jours fixés par le roi pour qu'on les lui amène, le chef des eunuques les présenta à Nebucadnetsar.
\VS{19}Le roi s'entretint avec eux ; mais entre eux tous il ne s'en trouva pas de tels que Daniel, Hanania, Mischaël et Azaria ; et ils entrèrent au service du roi.
\VS{20}Sur toutes les questions savantes qui réclamaient de la sagesse et de l'intelligence, et sur lesquelles le roi les interrogeait, il les trouva dix fois supérieurs à tous les magiciens et les astrologues qui étaient dans tout son royaume.
\VS{21}Et Daniel fut là jusqu'à la première année du roi Cyrus.
\Chap{2}
\TextTitle{Les sages de Babylone tous condamnés à mort}
\VerseOne{}La deuxième année du règne de Nebucadnetsar, Nebucadnetsar eut des songes, et son esprit fut agité, et son sommeil fut interrompu.
\VS{2}Alors le roi fit appeler les magiciens, les astrologues, les enchanteurs et les Chaldéens, pour qu'ils lui expliquent ses songes ; ils vinrent donc et se présentèrent devant le roi.
\VS{3}Le roi leur dit : J'ai eu un songe, mon esprit est agité, tâchant de connaître ce songe\FTNT{Ce songe annonce la future mise en place d'un gouvernement mondial. Voir commentaire en Da. 7:3.}.
\VS{4}Et les Chaldéens répondirent au roi en langue araméenne\FTNT{L'araméen : De Daniel 2:5 à 7:28, le livre est écrit en araméen.} : Ô roi, vis éternellement ! Dis le songe à tes serviteurs et nous en donnerons l'interprétation.
\VS{5}Mais le roi répondit et dit aux Chaldéens : La chose m'a échappé ; si vous ne me faites connaître le songe et son interprétation, vous serez mis en pièces et vos maisons seront réduites en un tas d'immondices.
\VS{6}Mais si vous me faites connaître le songe et son interprétation, vous recevrez de moi, des dons, des présents et un grand honneur. Quoi qu'il en soit faites-moi connaître le songe et son interprétation.
\VS{7}Ils répondirent pour la seconde fois et dirent : Que le roi dise le songe à ses serviteurs et nous en donnerons l'interprétation.
\VS{8}Le roi répondit et dit : Je m'aperçois en vérité, que vous ne cherchez qu'à gagner du temps, parce que vous voyez que la chose m'a échappée.
\VS{9}Mais si vous ne me faites pas connaître le songe, il y a une même sentence contre vous tous ; car vous vous êtes préparés à dire devant moi des mensonges et des faussetés en attendant que le temps soit changé. Quoi qu'il en soit, dites-moi le songe et je saurai que vous pouvez m'en donner l'interprétation.
\VS{10}Les Chaldéens répondirent au roi et dirent : Il n'y a aucun homme sur la terre qui puisse exécuter ce que le roi demande. Et aussi il n'y a ni roi, ni seigneur, ni gouverneur qui ait jamais demandé une telle chose à quelque magicien, astrologue ou Chaldéen que ce soit.
\VS{11}Car la chose que le roi demande est extrêmement difficile et il n'y a personne qui puisse le faire connaître au roi, excepté les dieux dont la demeure n'est pas parmi les hommes.
\VS{12}A cause de cela, le roi s'irrita et se mit dans une très grande colère, et ordonna qu'on fasse périr tous les sages de Babylone.
\VS{13}La sentence fut donc publiée; on mettait à mort les sages et l'on cherchait Daniel et ses compagnons pour les faire périr.
\TextTitle{Daniel implore la miséricorde de Dieu}
\VS{14}Alors Daniel détourna l'exécution du conseil et l'arrêt donné à Arjoc, chef des gardes du roi, qui était sorti pour tuer les sages de Babylone.
\VS{15}Et il demanda et dit à Arjoc, commandant du roi : Pourquoi la sentence du roi est-elle si sévère ? Arjoc exposa la chose à Daniel.
\VS{16}Et Daniel entra et pria le roi de lui accorder du temps pour donner l'interprétation au roi.
\VS{17}Alors Daniel alla dans sa maison et informa de cette affaire Hanania, Mischaël et Azaria, ses compagnons,
\VS{18}pour implorer la miséricorde du Dieu des cieux sur ce secret, afin qu'on ne mette pas à mort Daniel et ses compagnons avec le reste des sages de Babylone. 
\TextTitle{Le songe de la grande statue révélé à Daniel}
\VS{19}Et le secret fut révélé à Daniel dans une vision pendant la nuit. Et Daniel bénit le Dieu des cieux.
\VS{20}Daniel prit donc la parole et dit : Béni soit le nom de Dieu, d'éternité en éternité ! A lui appartiennent la sagesse et la force\FTNT{Job. 12:13 ; Ap. 5:12 ; Ap. 7:12.}.
\VS{21}C'est lui qui change les temps et les saisons, qui ôte et qui établit les rois, qui donne la sagesse aux sages et la connaissance à ceux qui ont de l'intelligence.
\VS{22}C'est lui qui révèle les choses profondes et cachées, il connaît les choses qui sont dans les ténèbres et la lumière demeure avec lui\FTNT{De. 29:29 ; Es. 48:6 ; Jé. 33:3 ; Lu. 12:2-3.}.
\VS{23}Ô Dieu de nos pères ! Je te glorifie et te loue de ce que tu m'as donné de la sagesse et de la force, et de ce que tu m'as maintenant fait connaître ce que nous t'avons demandé, en nous ayant fait connaître le secret du roi.
\VS{24}Après cela, Daniel alla auprès d'Arjoc, à qui le roi avait ordonné de faire périr les sages de Babylone. Il alla et lui parla ainsi : Ne fais pas périr les sages de Babylone, mais fais-moi entrer devant le roi et je donnerai au roi l'interprétation qu'il souhaite.
\VS{25}Alors Arjoc conduisit promptement Daniel devant le roi et lui parla ainsi : J'ai trouvé parmi les captifs de Juda un homme qui donnera au roi l'interprétation de son songe.
\VS{26}Le roi prit la parole et dit à Daniel, qu'on nommait Beltschatsar : Es-tu capable de me faire connaître le songe que j'ai eu et son interprétation ?
\VS{27}Daniel répondit en présence du roi et dit : Ce que le roi demande est un secret que les sages, les astrologues, les magiciens et les devins ne sont pas capables de révéler au roi.
\VS{28}Mais il y a dans les cieux un Dieu qui révèle les secrets et qui a fait connaître au roi Nebucadnetsar ce qui doit arriver dans les derniers jours\FTNT{« Les derniers jours » voir commentaires dans Ge. 49:1-2.}. Voici ton songe et les visions de ta tête que tu as eues sur ta couche.
\VS{29}Sur ta couche, ô roi, il t'est monté des pensées touchant ce qui arriverait après ce temps-ci ; et celui qui révèle les secrets t'a fait connaître ce qui doit arriver.
\VS{30}Si ce secret m'a été révélé, ce n'est point qu'il y ait en moi une sagesse supérieure à celle de tous les vivants, mais c'est afin de donner au roi l'interprétation de son songe et afin que tu connaisses les pensées de ton cœur.
\VS{31}Ô roi, tu regardais et tu voyais une grande statue\FTNT{Voir annexe « La statue de Nebucadnetsar ».} ; cette grande statue, dont la splendeur était extraordinaire, était debout devant toi et son apparence était terrible.
\VS{32}La tête de cette statue était d'un or très fin, sa poitrine et ses bras étaient d'argent ; son ventre et ses cuisses étaient d'airain ;
\VS{33}ses jambes étaient de fer et ses pieds étaient en partie de fer et en partie de terre.
\VS{34}Tu regardais cela, jusqu'à ce qu'une pierre se détacha sans main, frappa les pieds de fer et d'argile de la statue et les brisa.
\VS{35}Alors le fer, l'argile, l'airain, l'argent et l'or furent brisés ensemble et devinrent comme la paille de l'aire en été que le vent transporte çà et là ; et nulle trace n'en fut retrouvée. Mais la pierre qui avait frappé la statue devint une grande montagne et remplit toute la terre.
\TextTitle{Premier empire universel : Babylone\FTNT{Cp. Da. 7:4.}}
\VS{36}C'est là le songe. Nous en donnerons maintenant l'interprétation devant le roi.
\VS{37}Ô roi, tu es le roi des rois, parce que le Dieu des cieux t'a donné le royaume, la puissance, la force et la gloire.
\VS{38}Il a remis entre tes mains, en quelque lieu qu'ils habitent, les enfants des hommes, les bêtes des champs et les oiseaux du ciel, et il t'a fait dominer sur eux tous : C'est toi qui es la tête d'or\FTNT{Jé. 27:6-7.}.
\TextTitle{Deuxième et troisième empires : les Mèdes et les Perses\FTNTT{cp. Da. 7:5 ; 8:20} et la Grèce\FTNTT{cp. Da. 7:6 ; 8:21}}
\VS{39}Mais après toi, il s'élèvera un autre royaume, moindre que le tien ; et ensuite un troisième royaume qui sera d'airain et qui dominera sur toute la terre.
\TextTitle{Quatrième empire : Rome\FTNTT{Cp. Da. 7:7 ; 9:26.}}
\VS{40}Puis il y aura un quatrième royaume, fort comme du fer ; de même que le fer brise et rompt tout ainsi il brisera et rompra tout, comme le fer qui met tout en pièces.
\VS{41}Et quant à ce que tu as vu, que les pieds et les orteils étaient en partie d'argile de potier et en partie de fer, c'est que ce royaume sera divisé, mais il y aura en lui de la force du fer, parce que tu as vu le fer mêlé avec l'argile de potier.
\VS{42}Et comme les doigts des pieds étaient en partie de fer et en partie d'argile, ce royaume sera en partie fort et en partie fragile.
\VS{43} Quant à ce que tu as vu, le fer mêlé avec l'argile de potier, c'est qu'ils se mêleront par des alliances humaines\FTNT{Le mot « alliances » vient de l'araméen « zera » qui signifie « semence » et « descendant ».} ; mais ils ne seront point unis l'un à l'autre de même que le fer ne s'allie point avec l'argile.
\TextTitle{Le royaume du Messie}
\VS{44}Dans le temps de ces rois, le Dieu des cieux suscitera un Royaume qui ne sera jamais détruit, et ce Royaume ne passera point à un autre peuple ; il brisera et anéantira tous ces royaumes-là, et lui-même sera établi éternellement.
\VS{45}Selon que tu as vu que de la montagne une pierre a été coupée sans main et qu'elle a brisé le fer, l'airain, la terre, l'argent et l'or. Le grand Dieu a fait connaître au Roi ce qui arrivera ci-après ; or le songe est véritable et son interprétation est certaine.
\TextTitle{Yahweh, le Dieu qui revèle les secrets}
\VS{46}Alors le roi Nebucadnetsar tomba sur sa face et se prosterna devant Daniel et il ordonna qu'on lui offre des offrandes de bonne odeur et des parfums.
\VS{47}Le roi parla à Daniel et lui dit : Certainement, votre Dieu est le Dieu des dieux, et le Seigneur des rois, et il révèle les secrets, puisque tu as pu découvrir ce secret.
\VS{48}Alors le roi éleva Daniel en dignité et lui fit de nombreux et riches présents ; il l'établit gouverneur sur toute la province de Babylone et chef suprême de tous les sages de Babylone.
\VS{49}Daniel pria le roi de remettre l'intendance de la province de Babylone à Schadrac, Méschac et Abed-Négo. Et Daniel se tenait à la porte du roi.
\Chap{3}
\TextTitle{La statue d'or de Nebucadnetsar}
\VerseOne{}Le roi Nebucadnetsar fit une statue d'or\FTNT{Nebucadnetsar est un type de l'antéchrist qui s'oppose aux plans de Dieu. Contrairement à la statue composée de plusieurs métaux qu'il avait vue en songe, et où il est représenté par la tête en or (Da. 2:38), il s'est fait construire une statue entièrement en or, se déclarant ainsi symboliquement invincible et immortel. En agissant de la sorte, Nebucadnetsar se fait Dieu et exige d'être adoré (2 Th. 2:3-4). Cette statue annonçait prophétiquement la mise en place d'une religion mondiale, fruit d'un mélange entre la politique et la religion. Ces choses sont déjà bien installées, il ne manque plus que la révélation de l'impie. Nous vivons dans une époque où on oblige les chrétiens à adhérer à des organisations politiques, religieuses, sous contrôle de l'état, et cela dans le but de contrôler les individus et le message qu'ils entendent et diffusent. Ainsi, les dirigeants actuels doivent passer au préalable par des études théologiques, dont l'enseignement contredit de plus en plus la vérité biblique pour se conformer aux préceptes de ce monde. Une fois ordonnés, ils doivent affilier leurs églises à des fédérations qui sont sous contrôle de l'état. En échange des subventions, beaucoup accepteront de diluer l'évangile, privant ainsi les âmes de la vérité.}, dont la hauteur était de soixante coudées, et la largeur de six coudées. Il la dressa dans la vallée de Dura, dans la province de Babylone.
\VS{2}Puis le roi Nebucadnetsar envoya pour rassembler les satrapes, les intendants, les gouverneurs, les conseillers, les trésoriers, les jurisconsultes, les juges, et tous les magistrats des provinces, afin qu'ils se rendent à la dédicace de la statue que le roi Nebucadnetsar avait dressée.
\VS{3}Ainsi furent assemblés les satrapes, les intendants, les gouverneurs, les conseillers, les trésoriers, les jurisconsultes, les juges, et tous les magistrats des provinces, pour la dédicace de la statue que le roi Nebucadnetsar avait dressée. Ils s'assemblèrent devant la statue que le roi Nebucadnetsar avait dressée.
\VS{4}Alors un héraut cria à haute voix, en disant : On vous fait savoir, ô peuples, nations, et langues !
\VS{5}Au moment où vous entendrez le son du cor, du chalumeau, de la guitare, de la sambuque, du psaltérion, de la cornemuse, et de toutes sortes d'instruments de musique, vous vous jetterez à terre et vous adorerez la statue d'or que le roi Nebucadnetsar a dressée.
\VS{6}Quiconque ne se jettera pas à terre et n'adorera pas sera jeté à l'instant même au milieu de la fournaise de feu ardent.
\VS{7}C'est pourquoi, au moment où tous les peuples entendirent le son du cor, du chalumeau, de la guitare, de la sambuque, du psaltérion, et de toutes sortes d'instruments de musique, tous les peuples, les nations, et les hommes de toutes les langues, se prosternèrent et adorèrent la statue d'or que le roi avait dressée.
\TextTitle{Le refus de l'idolâtrie}
\VS{8}Alors à ce même moment, certains chaldéens s'approchèrent et accusèrent les Juifs.
\VS{9}Et ils parlèrent et dirent au roi Nebucadnetsar : Roi, vis éternellement !
\VS{10}Toi, ô roi, tu as donné un ordre d'après lequel tout homme qui entendrait le son du cor, du chalumeau, de la guitare, de la sambuque, du psaltérion, de la cornemuse, et de toutes sortes d'instruments de musique, devrait se prosterner et adorer la statue d'or,
\VS{11}et que quiconque ne se prosternerait pas et ne l'adorerait pas, serait jeté au milieu d'une fournaise ardente.
\VS{12}Or, il y a certains Juifs que tu as établis sur les affaires de la province de Babylone, Schadrac, Méschac, et Abed-Négo ; ces hommes-là, ô roi, ne tiennent aucun compte de toi ; ils ne servent pas tes dieux, et ils n'adorent pas la statue d'or que tu as dressée.
\VS{13}Alors le roi Nebucadnetsar, saisi de colère et de fureur, ordonna qu'on amène Schadrac, Méschac, et Abed-Négo. Et ces hommes furent amenés devant le roi.
\VS{14}Et le roi Nebucadnetsar prit la parole et leur dit : Est-il vrai, Schadrac, Méschac, et Abed-Négo, que vous ne servez pas mes dieux, et que vous ne vous prosternez pas devant la statue d'or que j'ai dressée ?
\VS{15}Maintenant si vous êtes prêts, au moment où vous entendrez le son du cor, du chalumeau, de la guitare, de la sambuque, du psaltérion, de la cornemuse, et de toutes sortes d'instruments de musique, vous vous prosternerez, et vous adorerez la statue que j'ai faite ; si vous ne l'adorez pas, vous serez jetés à l'instant au milieu de la fournaise de feu ardent. Et qui est le dieu qui vous délivrera de mes mains ?
\VS{16}Schadrac, Méschac et Abed-Négo répondirent et dirent au roi Nebucadnetsar : Nous n'avons pas besoin de te répondre sur ce sujet.
\VS{17}Voici, notre Dieu, que nous servons, peut nous délivrer de la fournaise de feu ardent, et il nous délivrera de ta main, ô roi !
\VS{18}Sinon, sache, ô roi, que nous ne servirons pas tes dieux, et que nous n'adorerons pas la statue d'or que tu as dressée.
\TextTitle{L'épreuve de la fournaise de feu ardent}
\VS{19}Alors Nebucadnetsar fut rempli de fureur, et il changea de visage en tournant ses regards contre Schadrac, Méschac, et Abed-Négo. Il prit la parole et ordonna de chauffer la fournaise sept fois plus qu'on avait coutume de la chauffer.
\VS{20}Puis il commanda aux hommes les plus forts et les plus vaillants qui étaient dans son armée de lier Schadrac, Méschac, et Abed-Négo, et de les jeter dans la fournaise de feu ardent.
\VS{21}Et en même temps ces hommes furent liés avec leurs caleçons, leurs chaussures, leurs tiares, et leurs vêtements, et furent jetés au milieu de la fournaise de feu ardent.
\VS{22}Et parce que l'ordre du roi était sévère, et que la fournaise était extraordinairement chauffée, la flamme tua les hommes qui y avaient jetés, Schadrac, Méschac, et Abed-Négo.
\VS{23}Et ces trois hommes, Schadrac, Méschac, et Abed-Négo, tombèrent tous liés au milieu de la fournaise ardente.
\TextTitle{La grandeur de Yahweh, le Dieu qui délivre}
\VS{24}Alors le roi Nebucadnetsar fut effrayé, et se leva précipitamment. Il prit la parole et il dit à ses conseillers : N'avons-nous pas jeté trois hommes liés au milieu du feu ? Ils répondirent et dirent au roi : Certainement, ô roi !
\VS{25}Il reprit et dit : Voici, je vois quatre hommes sans liens qui marchent au milieu du feu, et qui n'ont point de mal ; et la figure du quatrième est semblable à celle d'un fils de Dieu.
\VS{26}Alors Nebucadnetsar s'approcha vers la porte de la fournaise de feu ardent ; et prenant la parole, il dit : Schadrac, Méschac, et Abed-Négo, serviteurs du Dieu Très-Haut, sortez et venez ! Alors Schadrac, Méschac, et Abed-Négo sortirent du milieu du feu.
\VS{27}Puis les satrapes, les intendants, les gouverneurs, et les conseillers du roi s'assemblèrent pour contempler ces hommes-là, et le feu n'avait eu aucun pouvoir sur leurs corps, et aucun cheveu de leur tête n'était brûlé, et leurs caleçons n'étaient point endommagés, et l'odeur du feu n'avait pas passé sur eux.
\VS{28}Alors Nebucadnetsar prit la parole et dit : Béni soit le Dieu de Schadrac, Méschac, et Abed-Négo, lequel a envoyé son ange et délivré ses serviteurs qui ont eu confiance en lui, et qui ont violé l'ordre du roi et livré leur corps plutôt que de servir et d'adorer aucun autre dieu que leur Dieu\FTNT{Mt. 4:10 ; Ac. 4:19 ; Ac. 5:29.}.
\TextTitle{Schadrac, Méschac et Abed-Nego élevé par le roi}
\VS{29}Voici maintenant l'ordre que je donne : Tout homme, à quelque nation ou langue qu'il appartienne, qui parlera mal du Dieu de Schadrac, Méschac, et Abed-Négo, sera mis en pièces, et sa maison sera réduite en un tas d'immondices, parce qu'il n'y a aucun autre dieu qui puisse délivrer comme lui.
\VS{30}Alors le roi fit réussir Schadrac, Méschac, et Abed-Négo dans la province de Babylone.
\Chap{4}
\TextTitle{La suprématie de Yahweh déclarée aux nations}
\VerseOne{}Le roi Nebucadnetsar, à tous les peuples, aux nations, aux hommes de toutes langues qui habitent sur toute la terre : Que votre paix soit multipliée !
\VS{2}Il m'a semblé bon de vous déclarer les signes et les merveilles que le Dieu Très-Haut a opérés à mon égard.
\VS{3}Ô que ses signes sont grands, et ses merveilles pleines de force ! Son règne est un règne éternel, et sa domination subsiste de génération en génération\FTNT{Ps 102:12 ; La. 5:19 ; Lu. 1:33.}.
\TextTitle{La vision du grand arbre}
\VS{4}Moi, Nebucadnetsar, j'étais tranquille dans ma maison, et heureux dans mon palais.
\VS{5}J'ai eu un songe qui m'épouvanta ; et les pensées sur ma couche et les visions de ma tête me troublèrent.
\VS{6}J'ordonnai qu'on fasse venir devant moi tous les sages de Babylone, afin qu'ils me donnent l'interprétation du songe.
\VS{7}Alors vinrent les magiciens, les astrologues, les Chaldéens et les devins. Je leur dis le songe, mais ils ne purent m'en donner l'interprétation.
\VS{8}En dernier lieu, se présenta devant moi Daniel, nommé Beltschatsar, selon le nom de mon Dieu, et qui a en lui l'Esprit des dieux saints. Je lui dis le songe :
\VS{9}Beltschatsar, chef des magiciens, comme je sais que l'Esprit des dieux saints est en toi, et que nul secret ne t'est difficile, écoute les visions que j'ai eues en songe, et donne-moi son interprétation.
\VS{10}Voici les visions de ma tête, pendant que j'étais sur ma couche. Je regardais, et voici, il y avait un arbre au milieu de la terre d'une grande hauteur.
\VS{11}Cet arbre était devenu grand et fort, sa cime s'élevait jusqu'aux cieux, et on le voyait des extrémités de toute la terre.
\VS{12}Son feuillage était beau, et son fruit abondant, et il portait de la nourriture pour tous ; les bêtes des champs s'abritaient sous son ombre, les oiseaux du ciel habitaient dans ses branches, et tout être vivant tirait de lui sa nourriture.
\VS{13}Dans les visions de ma tête que j'avais sur ma couche, je regardais, et voici, un de ceux qui veillent et qui sont saints descendit des cieux.
\VS{14}Il cria à haute voix et parla ainsi : Abattez l'arbre, et coupez ses branches ! Secouez son feuillage, et dispersez son fruit ; que les bêtes s'enfuient de dessous, et les oiseaux du milieu de ses branches !
\VS{15}Mais laissez en terre le tronc où se trouvent ses racines, et liez-le avec des chaînes de fer et d'airain, qu'il soit parmi l'herbe des champs. Qu'il soit trempé de la rosée des cieux, et qu'il ait, comme les bêtes, l'herbe de la terre pour partage.
\VS{16}Que son cœur d'homme soit changé, et qu'un cœur de bête lui soit donné ; et que sept temps passent sur lui.
\VS{17}Cette sentence est le décret de ceux qui veillent, cette résolution est un ordre des saints, afin que les vivants sachent que le Très-Haut domine sur le royaume des hommes, qu'il le donne à qui il lui plaît, et qu'il y élève le plus vil des hommes\FTNT{Cette vérité est confirmée par l'apôtre Paul dans Ro. 13:1. C'est Dieu qui choisit souverainement qui il établit à la tête d'un pays. Selon les Ecritures, toute autorité vient de Dieu.}.
\VS{18}Voilà le songe que j'ai eu, moi, le roi Nebucadnetsar. Toi donc Beltschatsar, donnes-en l'interprétation, puisque tous les sages de mon royaume ne peuvent me la donner ; mais toi, tu le peux, parce l'Esprit des dieux saints est en toi.
\TextTitle{Interprétation de la vision}
\VS{19}Alors Daniel, nommé Beltschatsar, fut stupéfait environ une heure, et ses pensées le troublaient. Le roi reprit et dit : Beltschatsar, que le songe et son interprétation ne te troublent pas ! Et Beltschatsar répondit : Mon seigneur, que le songe soit pour ceux qui te haïssent, et son interprétation pour tes ennemis !
\VS{20}L'arbre que tu as vu, qui était devenu grand et fort, dont la cime s'élevait jusqu'aux cieux, et qu'on voyait de tous les points de la terre ;
\VS{21}cet arbre, dont le feuillage était beau, et les fruits abondants, qui portait de la nourriture pour tous, sous lequel s'abritaient les bêtes des champs, et parmi les branches duquel les oiseaux du ciel faisaient leur demeure,
\VS{22}c'est toi, ô roi, qui es devenu grand et fort, dont la grandeur s'est accrue et s'est élevée jusqu'aux cieux, et dont la domination s'étend jusqu'aux extrémités de la terre.
\VS{23}Le roi a vu un de ceux qui veillent et qui sont saints descendre des cieux et dire : Abattez l'arbre, et détruisez-le ! Toutefois, laissez en terre le tronc où se trouvent ses racines, et liez-le avec des chaînes de fer et d'airain, parmi l'herbe des champs, qu'il soit trempé de la rosée du ciel, et que son partage soit avec les bêtes des champs, jusqu'à ce que sept temps soient passés sur lui.
\VS{24}Voici l'interprétation, ô roi, voici le décret du Très-Haut, qui s'accomplira sur mon seigneur, le roi :
\VS{25}On te chassera du milieu des hommes, tu auras ta demeure avec les bêtes des champs, et l'on te donnera de l'herbe comme aux bœufs, et tu seras trempé de la rosée du ciel ; et sept temps passeront sur toi, jusqu'à ce que tu reconnaisses que le Très-Haut domine sur le royaume des hommes, et qu'il le donne à qui il lui plaît.
\VS{26}L'ordre de laisser le tronc où se trouvent les racines de cet arbre signifie que ton royaume te sera rendu, dès que tu auras reconnu que les cieux dominent.
\VS{27}C'est pourquoi, ô roi, que mon conseil te soit agréable : Rachète tes péchés par la justice, et tes iniquités en faisant miséricorde aux pauvres, et ta paix pourra se prolonger.
\TextTitle{Le roi déchu à cause de son orgueil}
\VS{28}Toutes ces choses se sont accomplies sur le roi Nebucadnetsar.
\VS{29}Au bout de douze mois, comme il se promenait dans le palais royal de Babylone,
\VS{30}le roi prit la parole et dit : N'est-ce pas ici Babylone la grande, que j'ai bâtie pour être la maison royale, par la puissance de ma force et pour la gloire de ma magnificence ?
\VS{31}La parole était encore dans la bouche du roi, qu'une voix descendit du ciel, disant : Roi Nebucadnetsar, on t'annonce que ton royaume va t'être ôté.
\VS{32}On te chassera du milieu des hommes, tu auras ta demeure avec les bêtes des champs ; on te donnera de l'herbe à manger comme aux bœufs ; et sept temps passeront sur toi, jusqu'à ce que tu reconnaisses que le Très-Haut domine sur le royaume des hommes, et qu'il le donne à qui il lui plaît.
\VS{33}Au même instant, la parole s'accomplit sur Nebucadnetsar. Il fut chassé du milieu des hommes, il mangea de l'herbe comme les bœufs, et son corps fut trempé de la rosée du ciel jusqu'à ce que ses cheveux croissent comme les plumes des aigles, et ses ongles comme ceux des oiseaux.
\TextTitle{Le roi est rétabli ; il loue le Dieu Très-Haut}
\VS{34}Mais à la fin de ces jours-là, moi Nebucadnetsar, je levai mes yeux vers le ciel, et la raison me revint. J'ai béni le Très-Haut, j'ai loué et glorifié celui qui vit éternellement, celui dont la domination est une domination éternelle, et dont le règne subsiste de génération en génération.
\VS{35}Tous les habitants de la terre ne sont à ses yeux que néant ; il agit comme il lui plaît avec l'armée des cieux et avec les habitants de la terre, et il n'y a personne qui empêche sa main, et qui lui dise : Que fais-tu\FTNT{Es. 45:9 ; Jé. 23:18-22 ; ps.115:3 ; Job. 9:12.} ?
\VS{36}En ce temps, la raison me revint, et je retournai à la gloire de mon royaume, ma magnificence et ma splendeur me furent rendues ; mes conseillers et mes grands me redemandèrent ; je fus rétabli dans mon royaume, et ma gloire fut augmentée.
\VS{37}Maintenant, moi, Nebucadnetsar, je loue, j'exalte, et je glorifie le Roi des cieux, dont toutes les œuvres sont véritables et ses voies justes, et qui peut abaisser ceux qui marchent avec orgueil\FTNT{De. 32:4 ; Es. 13:11 ; Ez. 17:24 ; Ps. 145:17.}.
\Chap{5}
\TextTitle{Les vases du temple souillés}
\VerseOne{}Le roi Belschatsar donna un grand festin à ses grands au nombre de mille, et il buvait le vin devant ces mille courtisans.
\VS{2}Et ayant goûté au vin, Belschatsar ordonna qu'on apporte les vases d'or et d'argent que Nebucadnetsar, son père, avait enlevés du temple de Jérusalem\FTNT{Ex. 27 ; Ex. 30 ; 2 Ch. 36:10.}, afin que le roi et ses grands, ses femmes et ses concubines, s'en servent pour boire.
\VS{3}Alors furent apportés les vases d'or qui avaient été enlevés du temple, de la maison de Dieu qui était à Jérusalem ; et le roi, et ses grands, ses femmes et ses concubines, s'en servirent pour boire.
\VS{4}Ils burent du vin, et ils louèrent leurs dieux d'or, d'argent, d'airain, de fer, de bois et de pierre.
\TextTitle{L'écriture sur la muraille}
\VS{5}Et à cette même heure-là sortirent de la muraille des doigts d'une main d'homme, qui écrivaient à l'endroit du chandelier, sur l'enduit de la muraille du palais royal ; et le Roi voyait cette partie de main qui écrivait.
\VS{6}Alors l'aspect du roi changea, et ses pensées l'effrayèrent, si bien que les jointures de ses reins se desserrèrent, et ses genoux se cognèrent l'un contre l'autre.
\VS{7}Puis le roi cria avec force qu'on fasse venir les astrologues, les Chaldéens et les devins ; et le roi prit la parole et dit aux sages de Babylone : Quiconque lira cette écriture, et m'en donnera l'interprétation, sera revêtu de pourpre, il aura un collier d'or à son cou, et sera le troisième dans le gouvernement du royaume.
\VS{8}Alors tous les sages du roi entrèrent, mais ils ne purent pas lire l'écriture et en donner au roi l'interprétation.
\VS{9}Sur quoi le roi Belschatsar fut très effrayé, il changea de couleur, et ses grands furent consternés.
\TextTitle{Interprétation de l'écriture: Division de l'empire babylonien}
\VS{10}La reine entra dans la maison du festin, à cause de ce qui était arrivé au roi et à ses grands. La reine prit la parole et dit : Ô roi, vis éternellement ! Que tes pensées ne te troublent pas, et que ton visage ne change pas de couleur !
\VS{11}Il y a dans ton royaume un homme qui a en lui l'Esprit des dieux saints ; et du temps de ton père, on trouva en lui une lumière, une intelligence, et une sagesse semblable à la sagesse des dieux. Aussi, le roi Nebucadnetsar, ton père, et le roi, ton père\FTNT{Belschatsar était le petit-fils de Nebucadnetsar qui avait régné conjointement avec son père, Nabonide, à partir de 552 av. J.-C.}, ô roi, l'établirent chef des magiciens, des astrologues, des Chaldéens et des devins,
\VS{12}parce qu'on trouva chez lui, chez Daniel, que le roi avait nommé Beltschatsar, un esprit supérieur, de la connaissance et de l'intelligence, pour interpréter les songes, pour expliquer les énigmes et résoudre les questions difficiles. Que Daniel soit donc appelé et il donnera l'interprétation que tu souhaites.
\VS{13}Alors Daniel fut introduit devant le roi. Le roi prit la parole et dit à Daniel : Es-tu ce Daniel, l'un des captifs de Juda, que le roi, mon père, a amenés de Juda ?
\VS{14}J'ai appris sur ton compte que tu as en toi l'Esprit des dieux, et qu'on trouve en toi une lumière, une intelligence et une sagesse extraordinaires.
\VS{15}On vient d'amener devant moi les sages et les astrologues, afin qu'ils lisent cette écriture et m'en donnent l'interprétation, mais ils n'ont pas pu donner l'interprétation de la chose.
\VS{16}J'ai appris que tu peux interpréter et résoudre les choses difficiles ; maintenant donc si tu peux lire cette écriture, et m'en donner l'interprétation, tu seras revêtu de pourpre, tu porteras à ton cou un collier d'or, et tu seras le troisième dans le gouvernement du royaume.
\VS{17}Alors Daniel répondit et dit en présence du roi : Que tes dons restent à toi, et donne tes présents à un autre ; toute fois je lirai l'écriture au roi, et je lui en donnerai l'interprétation.
\VS{18}Ô roi ! Le Dieu Très-Haut avait donné à Nebucadnetsar, ton père, le royaume, la magnificence, la gloire et l'honneur.
\VS{19}Et à cause de la grandeur qu'il lui avait donnée, tous les peuples, les nations, et les hommes de toutes langues tremblaient devant lui et le redoutaient. Il faisait mourir ceux qu'il voulait, et il laissait la vie à ceux qu'il voulait ; il élevait ceux qu'il voulait, et il abaissait ceux qu'il voulait.
\VS{20}Mais lorsque son cœur s'éleva et que son esprit s'endurcit jusqu'à l'arrogance, il fut renversé de son trône royal et dépouillé de sa gloire ;
\VS{21}il fut chassé du milieu des fils des hommes, son cœur fut rendu semblable à celui des bêtes, et sa demeure fut avec les ânes sauvages ; on lui donna comme aux bœufs de l'herbe à manger, et son corps fut trempé de la rosée du ciel, jusqu'à ce qu'il reconnaisse que le Dieu Très-Haut domine sur les royaumes des hommes, et qu'il y établit ceux qu'il lui plaît.
\VS{22}Et toi aussi, Belschatsar, son fils, tu n'as pas humilié ton cœur, quoique tu saches toutes ces choses.
\VS{23}Mais tu t'es élevé contre le Seigneur des cieux ; les vases de sa maison ont été apportés devant toi, et vous vous en êtes servis pour boire du vin, toi et tes grands, tes femmes et tes concubines ; tu as loué les dieux d'argent, d'or, d'airain, de fer, de bois et de pierre, qui ne voient point, qui n'entendent point, et qui ne savent rien, et tu n'as pas glorifié le Dieu dans la main duquel est ton souffle, et toutes tes voies\FTNT{Job. 12:10 et 33:4.}.
\VS{24}Alors de sa part a été envoyée cette partie de main, et cette écriture a été gravée.
\VS{25}Voici l'écriture qui a été gravée : Compté, compté, pesé et divisé.
\VS{26}Et voici l'interprétation de ces paroles. Compté : Dieu a compté ton règne, et y a mis la fin.
\VS{27}Pesé : Tu as été pesé dans la balance, et tu as été trouvé léger.
\VS{28}Mesuré : Ton royaume a été divisé, et donné aux Mèdes et aux Perses.
\VS{29}Aussitôt, Belschatsar donna des ordres, et l'on revêtit Daniel de pourpre, on lui mit un collier d'or au cou, et on publia qu'il serait le troisième dans le gouvernement du royaume.
\VS{30}Cette même nuit, Belschatsar, roi des Chaldéens, fut tué.
\VS{31}Et Darius, le Mède, reçut le royaume, étant âgé d'environ soixante-deux ans\FTNT{Es. 13:17 ; Es. 21:2 ; Jé. 51:11.}.
\Chap{6}
\TextTitle{Règne de Darius, le Mède}
\VerseOne{}Darius trouva bon d'établir sur le royaume cent vingt satrapes, qui devaient être répartis dans tout le royaume.
\VS{2}Il mit à leur tête trois chefs, au nombre desquels était Daniel, afin que ces satrapes leur rendent compte, et que le roi ne souffre aucun préjudice.
\VS{3}Daniel surpassait les autres chefs et satrapes, parce qu'il y avait en lui un Esprit supérieur ; et le roi pensait à l'établir sur tout le royaume.
\TextTitle{Daniel refuse l'idolâtrie et persévère dans la prière}
\VS{4}Alors les chefs et les satrapes cherchèrent une occasion d'accuser Daniel en ce qui concerne les affaires du royaume. Mais ils ne purent trouver en lui aucune occasion, ni aucune fausseté, parce qu'il était fidèle, et il ne se trouvait en lui ni faute ni vice.
\VS{5}Et ces hommes dirent : Nous ne trouverons aucune occasion d'accuser ce Daniel, à moins que nous n'en trouvions une dans la loi de son Dieu.
\VS{6}Alors ces chefs et ces satrapes se rendirent tumultueusement auprès du roi, et lui parlèrent ainsi : Roi Darius, vis éternellement !
\VS{7}Tous les chefs de ton royaume, les intendants, les satrapes, les conseillers, et les gouverneurs, sont d'avis d'établir un édit royal et une défense sévère, portant que quiconque, dans l'espace de trente jours, adressera des prières à quelque dieu ou à quelque homme, excepté à toi, ô roi, sera jeté dans la fosse aux lions.
\VS{8}Maintenant donc, ô roi, établis cette défense, et écris le décret afin qu'il soit irrévocable, selon la loi des Mèdes et des Perses, qui est immuable.
\VS{9}Là-dessus, le roi Darius écrivit le décret et la défense.
\VS{10}Lorsque Daniel sut que le décret était écrit, il entra dans sa maison, où les fenêtres de sa chambre étaient ouvertes dans la direction de Jérusalem ; et trois fois par jour, il se mettait à genoux, il priait, et il louait son Dieu, comme il le faisait auparavant\FTNT{1 R. 8:44 ; Ps. 55:17-18.}.
\VS{11}Alors ces hommes entrèrent tumultueusement, et ils trouvèrent Daniel qui priait et invoquait son Dieu.
\VS{12}Puis ils s'approchèrent du roi, et lui dirent au sujet de la défense royale : N'as-tu pas écrit une défense portant que tout homme dans l'espace de trente jours qui adresserait des prières à quelque dieu ou à quelque homme, excepté à toi, ô roi, serait jeté dans la fosse aux lions ? Le roi répondit : La chose est certaine, selon la loi des Mèdes et des Perses, qui est irrévocable.
\VS{13}Ils prirent de nouveau la parole et dirent au roi : Daniel, l'un des captifs de Juda, n'a tenu aucun compte de toi, ô roi, ni de la défense que tu as écrite, et il fait sa prière trois fois par jour.
\VS{14}Le roi fut très affligé quand il entendit cela ; il prit à cœur de délivrer Daniel, et jusqu'au coucher du soleil il s'efforça de le sauver.
\VS{15}Mais ces hommes se rendirent tumultueusement auprès du roi, et lui dirent : Sache, ô roi, que la loi des Mèdes et des Perses exige que toute défense ou tout décret établi par le roi soit irrévocable.
\TextTitle{Daniel demeure fidèle à Dieu face à la mort}
\VS{16}Alors le roi commanda qu'on amène Daniel, et qu'on le jette dans la fosse aux lions. Et le roi prenant la parole et dit à Daniel : Ton Dieu, lequel tu sers constamment, sera celui qui te délivrera.
\VS{17}On apporta une pierre, et on la mit sur l'ouverture de la fosse ; le roi la scella de son anneau, et de l'anneau de ses grands, afin que rien ne soit changé à l'égard de Daniel.
\TextTitle{Yahweh fait justice à Daniel}
\VS{18}Le roi se rendit ensuite dans son palais ; il passa la nuit à jeun, il ne fit point venir des danseuses\FTNT{Le mot « danseuse » vient de l'araméen « dachavah » qui signifie « divertissement », « instrument de musique », « danseuse », « concubine », « musique ».} auprès de lui, et il ne put se livrer au sommeil.
\VS{19}Puis le roi se leva au point du jour, avec l'aurore, et il alla précipitamment à la fosse aux lions.
\VS{20}En s'approchant de la fosse, il cria d'une voix triste : Daniel ! Le roi prit la parole et dit à Daniel : Daniel, serviteur du Dieu vivant, ton Dieu, que tu sers avec persévérance, a-t-il pu te délivrer des lions ?
\VS{21}Alors Daniel dit au roi : Ô roi, vis éternellement !
\VS{22}Mon Dieu a envoyé son ange, et a tellement fermé la gueule des lions, qu'ils ne m'ont fait aucun mal, parce que j'ai été trouvé innocent devant lui ; et même à ton égard, ô Roi ! je n'ai commis aucune faute. 
\VS{23}Alors le roi fut extrêmement heureux pour lui et il ordonna qu'on fasse retirer Daniel de la fosse. Ainsi Daniel fut retiré de la fosse, et on ne trouva sur lui aucune blessure, parce qu'il avait cru en son Dieu.
\VS{24}Le roi ordonna que ces hommes qui avaient accusé Daniel, soient amenés et jetés dans la fosse aux lions, eux, leurs enfants et leurs femmes, et avant qu'ils soient parvenus au fond de la fosse, les lions se saisirent d'eux, et leur brisèrent tous les os.
\TextTitle{Les merveilles de Yahweh proclamées aux nations}
\VS{25}Après cela, le roi Darius écrivit à tous les peuples, à toutes les nations, aux hommes de toutes les langues, qui habitent sur toute la terre : Que votre paix soit multipliée !
\VS{26}J'ordonne que dans toute l'étendue de mon royaume on ait de la crainte et de la frayeur pour le Dieu de Daniel, car c'est le Dieu vivant, et il subsiste éternellement ; son Royaume ne sera jamais détruit, et sa domination durera jusqu'à la fin\FTNT{Lu. 1:33 ; Es. 11.}.
\VS{27}Il sauve et délivre, il fait des prodiges et des merveilles dans les cieux et sur la terre, et il a délivré Daniel de la puissance des lions.
\VS{28}Ainsi Daniel prospéra sous le règne de Darius, et sous le règne de Cyrus, roi de Perse.
\Chap{7}
\TextTitle{Songe des quatre animaux ; Explication des visions de Daniel}
\VerseOne{}La première année de Belschatsar, roi de Babylone, Daniel eut un songe et des visions de sa tête, étant dans sa couche. Ensuite il écrivit le songe, et il relata les principales choses.
\VS{2}Daniel donc parla et dit : Je regardais dans ma vision nocturne, et voici, les quatre vents des cieux se levèrent avec impétuosité sur la grande mer.
\VS{3}Puis quatre grandes bêtes\FTNT{Les quatre bêtes représentent les quatre empires historiques. 
Le lion :
V. 4 : Le premier animal est un lion, il représente l'Empire néo-babylonien (625 – 539 av. J.-C.). Les ailes suggèrent la rapidité de la conquête babylonienne (Ha.1:6-8 ; Jé. 4:13). En 30 ans, l'Arabie, la Judée, la Syrie et la Phénicie furent conquises.
Les ailes arrachées annonçant l'arrêt des grandes conquêtes avec la mort de Nebucadnetsar. 
Le cœur d'homme donné au lion symbolise la conversion de Nebucadnetsar et le changement dans l'attitude des rois babyloniens (Da.4:30-31 ; 2 R. 25:27-30).
L'ours :
V. 5 : Le deuxième animal est un ours. Il représente l'empire médo-perse (539–331 av. J.-C.) qui succéda à l'Empire Babylonien.
Le fait que l'ours se tienne sur un côté indique que les Mèdes sont soumis aux Perses qui sont les véritables maîtres de l'empire. Les trois côtes dans la gueule de l'ours symbolisent trois grandes conquêtes médo-perses : la Lydie (546 av. J.-C.), la Babylonie (539 av. J.-C.) et l'Egypte (524 av. J.-C.).
Le léopard :
V. 6 : Le troisième animal est un léopard, qui représente l'Empire gréco-macédonien (331 – 146 av. J.-C.). En 331 av. J.-C., le coup de grâce est donné aux Médo-Perses à la bataille d'Arbèles.
Les quatre ailes symbolisent la grande rapidité des conquêtes. Quand Alexandre le Grand mourut à l'âge de 33 ans, il avait le plus grand Empire jamais vu jusqu'à l'époque. Ses conquêtes s'étendaient jusqu'en Inde !
Les quatre têtes symbolisent quatre de ses généraux qui, à la mort d'Alexandre, se partagèrent l'immense empire : Cassandre en Grèce et en Macédoine, Lysimaque en Thrace et en Asie Mineure, Séleucus en Syrie et en orient, Ptolémée en Égypte.
Très rapidement, la Palestine, qui se trouvait au croisement des routes, fut l'objet de rivalités entre les généraux et leurs successeurs. Après quelques années de stabilité, les généraux luttèrent entre eux jusqu'au maintien de deux dynasties : les Séleucides, au nord, et les Lagides, au sud, en Égypte. Cela dura jusqu'à l'apparition de l'Empire romain.
La quatrième bête, est différente des autres
V. 7, 19, 24 : La quatrième bête est extraordinaire, terrible, effrayante, elle ne porte même pas de nom ! Elle représente l'Empire romain qui succéda à l'Empire gréco-macédonien (146 av. J.-C. – 476 ap. J.-C.). En 168 av. J.-C., la Macédoine passa sous le contrôle de Rome, puis, en 146 av. J.-C., c'est au tour de la Grèce de devenir une province romaine.
 Le quatrième empire ne peut être que celui de Rome comme l'enseigne l'histoire de l'antiquité. 
Dès le quatrième siècle, l'Empire romain fut assailli par les tribus barbares venues du nord (Alamans, Wisigoths, Goths, Vandales, Burgondes, Ostrogoths, etc.) et, en 476, le dernier empereur romain d'occident, Romulus Augustule, fut chassé par le roi barbare Odoacre (Goth). L'Empire romain n'est plus.
Les orteils en partie de fer et en partie d'argile représentent les nations européennes issues de la fragmentation de l'Empire romain qui a eu lieu le 4 septembre 476.} montèrent de la mer, différentes les unes des autres.
\TextTitle{Premier empire universel: Babylone\FTNTT{Cp. Da. 2:37-38.}}
\VS{4}La première était semblable à un lion, et avait des ailes d'aigle ; je la regardai jusqu'à ce que les plumes de ses ailes furent arrachées ; elle fut enlevée de terre et dressée sur ses pieds comme un homme, et un cœur d'homme lui fut donné.
\TextTitle{Deuxième empire: Les Mèdes et les Perses\FTNTT{Cp. Da. 2:39 ; 8:20.}}
\VS{5}Et voici, une deuxième bête était semblable à un ours, et se tenait sur un côté ; il avait trois côtes dans la gueule entre ses dents ; et on lui disait ainsi : Lève-toi, mange beaucoup de chair.
\TextTitle{Troisième empire: La Grèce\FTNTT{Cp. Da. 2:39 ; 8:21-22 ; 11:2-4.}}
\VS{6}Après cela je regardai, et voici une autre bête, semblable à un léopard, qui avait sur son dos quatre ailes d'oiseau, et cette bête avait quatre têtes, et la domination lui fut donnée.
\TextTitle{Quatrième empire: Rome\FTNTT{Cp. Da. 2:40-43 ; 7:23-24 ; 9:26.}}
\VS{7}Après cela, je regardai dans mes visions nocturnes, et voici, il y avait une quatrième bête, terrible, épouvantable et extraordinairement forte ; elle avait de grandes dents de fer, elle mangeait, brisait, et elle foulait à ses pieds ce qui restait ; elle était différente de toutes les bêtes qui avaient été avant elle, et elle avait dix cornes.
\TextTitle{Les dix cornes et la petite corne\FTNTT{Da. 7:24-27.}}
\VS{8}Je considérai ses cornes, et voici, une autre petite corne sortit du milieu d'elles, et trois des premières cornes furent arrachées par elle ; et voici, elle avait des yeux comme des yeux d'homme, et une bouche qui proférait de grandes choses.
\TextTitle{Le règne de Yahweh, l'Ancien des jours\FTNTT{Cp. Mt. 24:27-30 ; 25:31-34 ; Ap. 19:11-21.}}
\VS{9}Je regardai jusqu'à ce que les trônes soient placés. Et l'Ancien des jours s'assit. Son vêtement était blanc comme la neige, et les cheveux de sa tête étaient comme de la laine pure ; son trône était des flammes de feu, et ses roues un feu ardent.
\VS{10}Un fleuve de feu coulait et sortait de devant lui. Mille milliers le servaient, et dix mille millions se tenaient en sa présence. Le jugement se tint, et les livres furent ouverts\FTNT{Ap. 5:11 ; Ps. 68:18 ; 1 R. 22:19.}.
\VS{11}Je regardai alors, à cause du bruit des paroles arrogantes que proférait la corne ; et tandis que je regardais, la bête fut tuée, et son corps fut détruit et livré pour être brûlé au feu.
\VS{12}Les autres bêtes furent dépouillées de leur domination, mais une prolongation de vie leur fut accordée jusqu'à un temps déterminé.
\TextTitle{La domination du Fils de l'homme est éternelle\FTNTT{Cp. Ap. 5:1-14.}}
\VS{13}Je regardai encore dans les visions nocturnes, et je vis, comme le Fils de l'homme, qui venait avec les nuées des cieux, et il vint jusqu'à l'Ancien des jours, et se tint devant lui\FTNT{Ap. 19:14 ; Jud. 1:14.}.
\VS{14}Et il lui donna la domination, la gloire et le règne ; et tous les peuples, les nations et les langues le serviront. Sa domination est une domination éternelle qui ne passera point, et son règne ne sera jamais détruit.
\TextTitle{Interprétation de la vision du quatrième animal}
\VS{15}Moi, Daniel, j'eus l'esprit troublé au-dedans de moi, et les visions de ma tête m'effrayèrent.
\VS{16}Je m'approchai de l'un des assistants, et lui demandai ce qu'il y avait de vrai dans toutes ces choses. Il me parla, et me donna l'interprétation de ces choses, en disant :
\VS{17}Ces quatre grandes bêtes sont quatre rois, qui s'élèveront de la terre.
\VS{18}Mais les saints du Très-Haut recevront le Royaume, et ils posséderont le Royaume éternellement, d'éternité en éternité.
\VS{19}Alors, je désirai savoir la vérité sur la quatrième bête, qui était différente de toutes les autres, extraordinairement terrible, qui avait des dents de fer et des ongles d'airain, qui mangeait, brisait, et foulait à ses pieds ce qui restait ;
\VS{20}et sur les dix cornes qu'elle avait à la tête, et sur l'autre corne qui était sortie et devant laquelle trois étaient tombées, sur cette corne qui avait une bouche parlant avec arrogance, et une plus grande apparence que celle de ses associées.
\VS{21}Je regardai comment cette corne faisait la guerre aux saints et l'emportait sur eux\FTNT{Ap. 13:2-7.},
\VS{22}jusqu'au moment où l'Ancien des jours vint donner droit aux saints du Très-Haut, et que le temps arriva où les saints furent en possession du Royaume.
\VS{23}Il me parla donc ainsi : La quatrième bête est un quatrième royaume qui sera sur la terre, différent de tous les royaumes, et qui dévorera toute la terre, la foulera, et la brisera.
\TextTitle{Règne de l'homme impie et jugement de Dieu}
\VS{24}Mais les dix cornes sont dix rois qui s'élèveront de ce royaume. Un autre s'élèvera après eux, il sera différent des premiers, et il abattra trois rois.
\VS{25}Il proférera des paroles contre le Très-Haut, il harcelera les saints du Très-Haut, et il aura l'intention de changer les temps et la loi ; et les saints seront livrés entre ses mains pendant un temps, des temps, et la moitié d'un temps.
\VS{26}Mais le jugement se tiendra, et on lui ôtera sa domination, en la détruisant et la faisant périr, jusqu'à en voir la fin..
\VS{27}Afin que le règne, la domination, et la grandeur de tous les royaumes qui sont sous les cieux, soient donnés au peuple des saints du Très-Haut. Son royaume est un royaume éternel, et tous les royaumes lui seront assujettis et lui obéiront.
\VS{28}Jusqu'ici est la fin de cette affaire. Quant à moi, Daniel, mes pensées m'effrayèrent beaucoup, et ma splendeur changea en moi, toutefois je gardais cette affaire dans mon cœur.
\Chap{8}
\TextTitle{Vision du bélier et du bouc}
\VerseOne{}La troisième année du règne du roi Belschatsar, moi, Daniel, j'eus cette vision, en plus de celle que j'avais eue auparavant.
\VS{2}Je vis cette vision, et il arriva, comme je regardais, que j'étais à Suse, la capitale, dans la province d'Élam, et dans ma vision, je me trouvais près du fleuve d'Ulaï.
\VS{3}Et je levai mes yeux, je regardai, et voici, un bélier se tenait devant le fleuve, et il avait deux cornes ; et les deux cornes étaient hautes, mais l'une était plus haute que l'autre, et la plus haute s'éleva sur la dernière.
\VS{4}Je vis ce bélier qui frappait de ses cornes à l'occident, au nord, et au midi ; aucune bête ne pouvait subsister devant lui,et il n'y avait personne qui puisse délivrer de sa puissance ; et il agissait selon sa volonté et devenait grand.
\VS{5}Comme je regardais attentivement, voici, un bouc d'entre les chèvres venait de l'occident, et parcourait toute la terre à sa surface, sans la toucher ; ce bouc avait entre les yeux une corne considérable.
\VS{6}Il arriva jusqu'au bélier qui avait deux cornes et que j'avais vu se tenant devant le fleuve, et il courut sur lui dans la fureur de sa force.
\VS{7}Je le vis qui s'approchait du bélier et s'irritait contre lui ; il frappa le bélier et lui brisa les deux cornes, et le bélier n'avait aucune force pour tenir ferme contre lui ; et quand il l'eut jeté par terre, il le foula; et il n'y eut personne pour délivrer le bélier de sa puissance.
\VS{8}Alors le bouc d'entre les chèvres grandit extrêmement ; mais lorsqu'il fut puissant, sa grande corne se brisa. Quatre grandes cornes s'élevèrent pour la remplacer, aux quatre vents des cieux.
\TextTitle{La petite corne renverse la vérité}
\VS{9}De l'une d'elles sortit une petite corne\FTNT{Antiochus IV Épiphane est le fils d'Antiochos III le Grand, né vers 215 av. J.-C. Il gouverna le royaume séleucide de 175 av. J.-C. à 164 av. J.-C., date de sa mort. Ce dernier avait profané le temple de Jérusalem en sacrifiant des porcs sur l'autel (Voir commentaire en Mt. 24:15). Cette petite corne, qui fait tomber par terre une partie de l'armée des étoiles, agit de même que Satan au ciel qui avait fait chuter un tiers des étoiles, soit des anges (Ap. 12:3-4).}, qui s'agrandit beaucoup vers le midi, et vers l'orient, et vers le pays de noblesse.
\VS{10}Elle s'éleva même jusqu'à l'armée des cieux, elle fit tomber à terre une partie de l'armée et des étoiles, et elle les foula\FTNT{Es. 14:12-15 ; Ez. 28:12-19.}.
\VS{11}Et elle s'éleva même jusqu'au chef de l'armée, lui enleva le sacrifice perpétuel, et renversa la demeure de son sanctuaire.
\VS{12}L'armée fut livrée avec le sacrifice perpétuel, à cause du péché ; la corne jeta la vérité par terre, et fit de grands exploits, et prospéra.
\VS{13}Alors j'entendis un saint qui parlait ; et un autre saint disait à celui qui parlait : Jusqu'à quand durera cette vision sur le sacrifice perpétuel et sur le péché qui cause la désolation ? Jusqu'à quand le sanctuaire et l'armée seront-ils foulés ?
\VS{14}Et il me dit : Deux mille trois cents soirs et matins ; puis le sanctuaire sera purifié.
\TextTitle{La vision du bélier et du bouc interprétée}
\VS{15}Et quand à moi, Daniel, j'avais cette vision et que je désirais la comprendre, voici, quelqu'un qui avait l'apparence d'un homme se tenait devant moi.
\VS{16}Et j'entendis la voix d'un homme au milieu du fleuve Ulaï ; il cria et dit : Gabriel, explique-lui la vision.
\VS{17}Puis Gabriel vint alors près du lieu où je me tenais ; et à son approche, je fus effrayé, et je tombai sur ma face. Il me dit : Comprends, fils de l'homme, car la vision est pour le temps de la fin.
\VS{18}Comme il me parlait, je restai frappé d'étourdissement, la face contre terre. Il me toucha, et me fit tenir debout à la place où je me trouvais.
\VS{19}Et il dit : Voici, je vais t'apprendre ce qui arrivera à la fin de la colère, car il y a un temps marqué pour la fin.
\VS{20}Le bélier que tu as vu qui avait deux cornes, ce sont les rois des Mèdes et des Perses ;
\VS{21}et le bouc velu, c'est le roi de Javan\FTNT{Javan ou Grèce.} ; et la grande corne entre ses yeux, c'est le premier roi.
\VS{22}Les quatre cornes qui se sont élevées pour remplacer cette corne brisée, ce sont quatre royaumes qui s'élèveront de cette nation, mais qui n'auront pas autant de force.
\TextTitle{Le roi impie, adversaire de Dieu ; la vision scellée}
\VS{23}A la fin de leur règne, lorsque les pécheurs seront consumés, il se lèvera un roi cruel et artificieux.
\VS{24}Sa puissance s'accroîtra, mais non par sa propre force ; il fera d'incroyables ravages, il réussira dans ses entreprises, il détruira les puissants et le peuple des saints.
\VS{25}Et par la subtilité de son esprit, il fera prospérer la fraude dans sa main. Il aura de l'arrogance dans le cœur, et fera périr beaucoup d'hommes qui vivaient dans la paix, et il s'élèvera contre le Prince des princes ; mais il sera brisé, sans l'effort d'aucune main.
\VS{26}Et la vision du soir et du matin, dont il s'agit, est véritable. Mais toi, scelle la vision, car elle se rapporte à un temps éloigné.
\VS{27}Moi Daniel, je fus tout défait et malade pendant quelques jours ; puis je me levai, et je m'occupai des affaires du roi. J'étais étonné de la vision, et personne n'en eut connaissance.
\Chap{9}
\TextTitle{Supplications de Daniel à Yahweh}
\VerseOne{}La première année de Darius, fils d'Assuérus, de la race des Mèdes, lequel était établi roi sur le royaume des Chaldéens.
\VS{2}La première année, dis-je, de son règne, moi Daniel, je discernai par les livres, que le nombre des années dont Yahweh avait parlé au prophète Jérémie\FTNT{Jé. 25:11.} pour finir les désolations de Jérusalem, était de soixante et dix ans. 
\VS{3}Et je tournai ma face vers le Seigneur Dieu, pour le chercher par la prière et des supplications, avec jeûne, et le sac et la cendre.
\VS{4}Je priai Yahweh, mon Dieu, et je lui fis ma confession : Ah ! Seigneur, Dieu grand et redoutable, toi qui gardes ton alliance et qui fais miséricorde à ceux qui t'aiment et qui gardent tes commandements !
\VS{5}Nous avons péché, nous avons commis l'iniquité, nous avons agi méchamment, nous avons été rebelles, et nous nous sommes détournés de tes commandements et de tes ordonnances.
\VS{6}Nous n'avons pas écouté tes serviteurs, les prophètes, qui ont parlé en ton Nom à nos rois, à nos chefs, à nos pères, et à tout le peuple du pays.
\VS{7}Ô Seigneur ! A toi est la justice, et à nous la confusion de face, en ce jour, aux hommes de Juda, aux habitants de Jérusalem, et à tout Israël, à ceux qui sont près et à ceux qui sont loin, dans tous les pays où tu les as dispersés, à cause des infidélités dont ils se sont rendus coupables envers toi\FTNT{Né. 9:30 ; Ps. 106:6 ; La. 3:42.}.
\VS{8}Seigneur, à nous est la confusion de face, à nos rois, à nos chefs, et à nos pères, parce que nous avons péché contre toi.
\VS{9}Auprès du Seigneur, notre Dieu, la miséricorde et le pardon, car nous avons été rebelles envers lui.
\VS{10}Nous n'avons pas écouté la voix de Yahweh, notre Dieu, pour marcher dans ses lois, qu'il a mises devant nous par le moyen de ses serviteurs, les prophètes.
\VS{11}Tout Israël a transgressé ta loi, et s'est détourné pour ne pas écouter ta voix. Alors se sont répandues sur nous les malédictions et les imprécations qui sont écrites dans la loi de Moïse, serviteur de Dieu, parce que nous avons péché contre Dieu\FTNT{Lé. 26:14-33 ; De. 27:15-33.}.
\VS{12}Il a accompli les paroles qu'il avait prononcées contre nous, et contre nos chefs qui nous ont gouvernés, et il a fait venir sur nous un grand mal, et il n'en est jamais arrivé sous le ciel entier un semblable à celui qui est arrivé à Jérusalem.
\VS{13}Comme cela est écrit dans la loi de Moïse, ce mal est venu sur nous ; et nous n'avons pas imploré Yahweh, notre Dieu, pour nous détourner de nos iniquités, et pour nous rendre attentifs à ta vérité.
\VS{14}Yahweh a veillé sur le mal que nous avons fait et il l'a fait venir sur nous ; car Yahweh, notre Dieu, est juste dans toutes les œuvres qu'il a faites, vu que nous n'avons point obéi à sa voix.
\VS{15}Or maintenant, Seigneur, notre Dieu ! Toi qui as tiré ton peuple du pays d'Egypte par ta main puissante, et qui t'es acquis un Nom comme il l'est aujourd'hui, nous avons péché, nous avons été méchants.
\VS{16}Seigneur, je te prie que selon ta justice, que ta colère et ton indignation se détournent de ta ville de Jérusalem, de la montagne de ta sainteté ; car à cause de nos péchés et des iniquités de nos pères, Jérusalem et ton peuple sont en opprobre à tous ceux qui nous entourent.
\VS{17}Maintenant donc, ô notre Dieu, écoute la prière et les supplications de ton serviteur, et pour l'amour du Seigneur, fais briller ta face sur ton sanctuaire dévasté.
\VS{18}Mon Dieu ! Prête l'oreille, et écoute ; ouvre tes yeux, et regarde nos ruines, et la ville sur laquelle ton Nom a été invoqué ; car ce n'est pas à cause de notre justice que nous te présentons nos supplications, c'est à cause de tes grandes compassions.
\VS{19}Seigneur, exauce, Seigneur pardonne, Seigneur sois attentif, et opère ; ne tarde pas, par amour pour toi, ô mon Dieu ! Car ton Nom a été invoqué sur ta ville, et sur ton peuple.
\VS{20}Je parlais encore, je priais, je confessais mon péché, et le péché de mon peuple d'Israël, et je présentais ma supplication à Yahweh, mon Dieu, en faveur de la sainte montagne de mon Dieu.
\TextTitle{Les soixante-dix semaines}
\VS{21}Je parlais encore dans ma prière, quand l'homme Gabriel, que j'avais vu précédemment dans une vision, s'approcha de moi d'un vol rapide au moment de l'offrande du soir.
\VS{22}Il m'instruisit, et s'entretint avec moi. Il me dit : Daniel, je suis venu maintenant pour ouvrir ton intelligence.
\VS{23}La parole est sortie dès le commencement de tes supplications, et je suis venu pour te la déclarer, car tu es un bien-aimé. Sois attentif à la parole, et comprends la vision.
\VS{24}Il y a soixante-dix semaines\FTNT{Le verset 24 concerne la chronologie de l'accomplissement de la prophétie de Jérémie (25 : 11). Les soixante-dix semaines auxquelles elle fait allusion représentent une période de 490 ans, conformément au principe biblique prophétique selon lequel un jour prophétique équivaut à une année (No. 14:33-34 ; Ez. 4:4-6). Dans les versets 25 et 27, les soixante-dix semaines sont divisées en trois périodes : 7 semaines (49 ans), 62 semaines (434 ans), et une semaine (7 ans). Les soixante-dix semaines devaient débuter au moment où la parole a annoncé que Jérusalem serai rebâtie (v. 25). En 445 avant notre ère, dans la vingtième année de son règne, le roi Artaxerxès publia un décret permettant à Esdras de retourner à Jérusalem pour achever la reconstruction de la ville (Esd. 7:6-10 ; Esd. 9:9 ; Né 2:5). Il est attesté par l'histoire profane que cette date est le point de départ de la soixante-dixième semaine de Daniel. Les 69 premières semaines vont jusqu'au Messie conducteur. La semaine qui reste (7 ans) concerne la période du règne de l'Antéchrist, celle-ci est divisée en deux période : trois ans et demi de fausse paix (1 Th. 5:3), et trois ans et demi concernent la Grande Tribulation (Ap. 7:9-17 ; Ap. 11:1-3 ; Ap. 12:6 ; Ap. 13:5).} fixées sur ton peuple et sur ta ville sainte, pour abolir la transgression et mettre fin aux péchés, faire la propitiation pour l'iniquité, pour amener la justice éternelle, pour mettre le sceau à la vision et à la prophétie et pour oindre le Saint des saints.
\VS{25}Tu sauras donc et tu comprendras, que depuis le moment où la parole a annoncé que Jérusalem sera rebâtie jusqu'au Messie, le Conducteur, il y a sept semaines et soixante-deux semaines ; et les places et les brèches seront rebâties, mais en des temps d'angoisse.
\VS{26}Et après ces soixante-deux semaines, le Messie sera retranché, mais non pas pour lui. Le peuple du chef qui viendra, détruira la ville et le sanctuaire, et sa fin arrivera comme par une inondation ; il est déterminé que les dévastations dureront jusqu'à la fin de la guerre.
\VS{27}Et il confirmera l'alliance à plusieurs pour une semaine, et à la moitié de cette semaine il fera cesser le sacrifice, et l'offrande ; puis par le moyen des ailes abominables, qui causeront la désolation, même jusqu'à une consomption déterminée, la désolation fondra sur le désolé.
\Chap{10}
\TextTitle{Daniel voit la gloire du Messie}
\VerseOne{}La troisième année de Cyrus, roi de Perse, une parole fut révélée à Daniel, qu'on nommait Beltschatsar. Cette parole est véritable et annonce une grande guerre. Il fut attentif à cette parole, et il eut l'intelligence de la vision.
\VS{2}En ce temps-là, moi Daniel, je fus dans le deuil pendant trois semaines entières.
\VS{3}Je ne mangeai aucun mets délicat, il n'entra ni viande ni vin dans ma bouche, et je ne m'oignis point, jusqu'à ce que ces trois semaines entières soient accomplies.
\VS{4}Le vingt-quatrième jour du premier mois, j'étais au bord du grand fleuve qui est Hiddékel.
\VS{5}Je levai les yeux, et je regardai, et voici, il y avait un homme vêtu de lin, et ayant sur les reins une ceinture d'or fin d'Uphaz.
\VS{6}Son corps était comme de chrysolithe, et son visage brillait comme l'éclair, ses yeux étaient comme des flammes de feu, ses bras et ses pieds ressemblaient à de l'airain poli, et le son de sa voix était comme le bruit d'une multitude de gens\FTNT{Ap. 1:13-15.}.
\VS{7}Moi, Daniel, je vis seul la vision, et les hommes qui étaient avec moi ne la virent point ; mais ils furent saisis d'une grande frayeur, et ils s'enfuirent pour se cacher.
\VS{8}Je restai seul, et je vis cette grande vision ; les forces me manquèrent, mon visage changea de couleur et fut tout défait, et je ne conservai aucune vigueur.
\VS{9}J'entendis le son de ses paroles ; et comme j'entendais le son de ses paroles, je tombai frappé d'étourdissement, la face contre terre.
\TextTitle{Le combat du monde spirituel}
\VS{10}Et voici, une main me toucha et me fit mettre sur mes genoux, et sur les paumes de mes mains.
\VS{11}Puis il me dit : Daniel, homme aimé de Dieu, sois attentif aux paroles que je vais te dire, et tiens-toi debout à la place où tu es ; car je suis maintenant envoyé vers toi. Lorsqu'il m'eut ainsi parlé, je me tins debout en tremblant.
\VS{12}Il me dit : Ne crains rien, Daniel, car dès le premier jour où tu as appliqué ton cœur à comprendre, et à t'humilier devant ton Dieu, tes paroles ont été exaucées, et c'est à cause de tes paroles que je viens.
\VS{13}Mais le chef du royaume de Perse m'a résisté vingt et un jours ; mais voici, Micaël, l'un des principaux chefs, est venu à mon secours, et je suis demeuré là auprès des rois de Perse.
\VS{14}Je viens maintenant pour te faire connaître ce qui doit arriver à ton peuple dans les derniers jours, car la vision s'étend jusqu'à ces temps-là.
\VS{15}Pendant qu'il m'adressait ces paroles, je mis mon visage contre terre, et je gardai le silence.
\VS{16}Et voici, quelqu'un qui avait l'apparence des fils de l'homme toucha mes lèvres. J'ouvris la bouche, je parlai, et je dis à celui qui se tenait devant moi : Mon seigneur ! La vision m'a rempli d'effroi, et j'ai perdu toute vigueur.
\VS{17}Comment le serviteur de mon seigneur pourrait-il parler avec mon seigneur ? Maintenant les forces me manquent, et je n'ai plus de souffle.
\VS{18}Alors celui qui avait l'apparence d'un homme me toucha encore, et me fortifia.
\VS{19}Puis il me dit : Ne crains rien, homme bien-aimé, que la paix soit avec toi ! Fortifie-toi, fortifie-toi ! Et comme il me parlait, je repris des forces, et je dis : Que mon seigneur parle, car tu m'as fortifié.
\VS{20}Il me dit : Ne sais-tu pas pourquoi je suis venu vers toi ? Maintenant je m'en retournerai pour combattre le chef de Perse ; et quand je partirai, voici, le chef de Javan viendra.
\VS{21}Mais je veux te faire connaître ce qui est écrit dans le livre de vérité. Et il n'y a personne qui me soutienne contre ceux-là, excepté Micaël, votre chef.
\Chap{11}
\TextTitle{Succession des monarques jusqu'à l'homme impie\FTNTT{Da. 11:1 - 12:13.}}
\VerseOne{}Et moi, dans la première année de Darius, le Mède, je me tenais auprès de lui pour l'aider et le fortifier.
\VS{2}et maintenant, je vais te faire connaître la vérité : Voici, il y aura encore trois rois en Perse. Le quatrième amassera plus de richesses que les autres ; et quand il sera puissant par ses richesses, il soulèvera tout le monde contre le royaume de Javan.
\VS{3}Mais il s'élèvera un vaillant roi\FTNT{Ce vaillant roi est Alexandre le Grand qui règna de 336 à 323 av. J.-C.}, qui dominera avec une grande puissance, et fera ce qu'il voudra.
\VS{4}Et sitôt qu'il sera élevé, son royaume sera brisé et sera divisé\FTNT{A la mort d'Alexandre le Grand, ses quatre principaux généraux se partagèrent l'empire : 
-Lysimaque régna sur l'Asie mineure. 
-Cassandre régna sur la Grèce et la Macédoine.
-Seleucos régna en Syrie, en Babylonie et sur toutes les régions à l'est jusqu'aux Indes. 
-Ptolémée régna sur l'Égypte, la Judée et une partie de la Syrie.} vers les quatre vents des cieux ; il ne passera point à ses descendants, et n'aura pas la même puissance qu'il a exercée, car son royaume sera déchiré, et il passera à d'autres qu'à eux.
\VS{5}Le roi du midi\FTNT{Le roi du midi est Ptolémée 1er Soter (règne : 323-285 av. J.-C.), le chef plus fort que lui est Séleucus 1er Nicator (règne : 305-281 av. J.-C.).} deviendra fort et puissant. Mais un de ses chefs [roi de Javan] sera plus puissant que lui et dominera ; sa domination sera puissante.
\VS{6}Au bout de quelques années, ils s'allieront, et la fille du roi du midi viendra vers le roi du nord pour redresser les affaires. Mais elle ne conservera pas la force de son bras, et il ne résistera pas, ni lui ni son bras ; elle sera livrée avec ceux qui l'auront amenée, avec son père et avec celui qui aura été son soutien dans ce temps-là.
\VS{7}Mais un rejeton de ses racines s'élèvera pour le remplacer\FTNT{Ptolémée III Evergète (règne : 246-222 av. J.-C.)} ; il viendra à l'armée, il entrera dans les forteresses du roi du nord, il en disposera à son gré, et il se rendra puissant.
\VS{8}Et même il emmènera captifs en Egypte leurs dieux, avec leurs images de fonte et avec leurs vases précieux d'argent et d'or. Puis il restera quelques années éloigné du roi du nord.
\VS{9}Et celui-ci marchera contre le royaume du roi du midi, et retournera dans son pays.
\VS{10}Ses fils\FTNT{Ses fils : Ce sont les deux rois de Syrie, Seuleucus III Ceraunus (règne : 225-223 av. J.-C.), et Antiochus III le Grand (223-187 av. J.-C.).} entreront en guerre et rassembleront une multitude nombreuse de troupes ; l'un d'eux s'avancera et se répandra comme un torrent, débordera, puis reviendra ; et il poussera la guerre jusqu'à la forteresse du roi du midi.
\VS{11}Et le roi du midi sera irrité, il sortira et combattra contre lui, savoir contre le roi du nord ; il soulèvera une grande multitude, et les troupes du roi du nord seront livrées entre les mains du roi du midi.
\VS{12}Et après avoir défait cette multitude, le cœur du roi s'élèvera ; il fera tomber des milliers, mais il ne triomphera pas.
\VS{13}Car le roi du nord reviendra et rassemblera une plus grande multitude que la première ; au bout de quelque temps, de quelques années, il viendra avec une grande armée et de grandes richesses.
\VS{14}Et en ce temps-là, plusieurs s'élèveront contre le roi du midi ; et des hommes violents parmi ton peuple se révolteront pour accomplir la vision, mais ils succomberont.
\VS{15}Le\FTNT{Le roi du nord est Séleucos IV Philopator (règne :187-175 av. J.-C.)} roi du nord viendra, il élèvera des terrasses, et prendra les villes fortes. Les bras du midi et l'élite du roi ne résisteront pas, ils manqueront de force pour résister.
\VS{16}Celui qui marchera contre lui fera ce qu'il voudra, et personne ne lui résistera ; il s'arrêtera dans le pays de noblesse, exterminant ce qui tombera sous sa main.
\VS{17}Puis il tournera sa face pour entrer avec la force de tout son royaume, et fera un accord avec le roi du midi, et il lui donnera sa fille pour femme, pour ruiner le royaume ; mais cela ne tiendra pas, et elle ne sera pas pour lui. 
\VS{18}Puis il tournera ses vues vers les îles, et il en prendra plusieurs ; mais un chef mettra fin à l'opprobre qu'il voulait lui attirer, et le fera retomber sur lui.
\VS{19}Il se dirigera ensuite vers les forteresses de son pays ; et il chancellera, il tombera, et on ne le trouvera plus.
\VS{20}Et un autre sera établi à sa place, qui fera passer un exacteur dans l'ornement du royaume, et en peu de jours il sera brisé, et ce ne sera ni par la colère ni par la guerre.
\TextTitle{Usage de la tromperie pour régner}
\VS{21}Et sa place il en sera établi un autre qui sera méprisé, auquel on ne donnera pas l'honneur royal ; mais il viendra en paix, et il s'emparera du royaume par des flatteries.
\VS{22}Les troupes qui se répandront comme un torrent seront submergées devant lui, et brisées, de même qu'un chef de l'alliance.
\VS{23}Mais après les accords faits avec lui, il usera de tromperie, et il montera, et il aura le dessus avec peu de gens.
\VS{24} Il entrera tranquillement dans les lieux les plus riches de la province, et il fera ce que n'avaient pas fait ses pères, ni les pères de ses pères ; il distribuera le butin, le pillage et les richesses ; et il formera des desseins contre les places fortes, et cela jusqu'à un certain temps.
\VS{25}Puis il réveillera sa force et son coeur contre le roi du midi avec une grande armée. Et le roi du midi s'avancera en bataille avec une très grande et très forte armée ; mais il ne résistera pas, car on formera des complots contre lui.
\VS{26}Ceux qui mangent les mets de sa table le mettront en pièces ; son armée se répandra comme un torrent, et beaucoup de gens tomberont blessés à mort.
\VS{27}Et les deux rois chercheront en leur cœur à se nuire, et à la même table ils parleront avec fausseté. Mais cela ne réussira pas ; car la fin ne viendra qu'au temps marqué.
\VS{28}Après quoi il retournera dans son pays avec de grandes richesses ; et son cœur sera contre la sainte alliance, il agira contre elle, puis retournera dans son pays.
\VS{29}Ensuite il retournera au temps fixé, et il viendra contre le midi ; mais cette dernière expédition ne sera pas comme la précédente.
\VS{30}Car les navires de Kittim viendront contre lui ; affligé, il rebroussera chemin. Puis irrité contre la sainte alliance, il agira contre elle, il retournera et s'entendra avec les apostats de la sainte alliance.
\VS{31}Et les forces seront de son côté, et on profanera le sanctuaire qui est la forteresse, et on fera cesser le sacrifice perpétuel, et on y dressera l'abomination qui causera la désolation.
\VS{32}Et il corrompra par des flatteries ceux qui agissent méchamment à l'égard de l'alliance. Mais ceux du peuple qui connaîtront leur Dieu agiront avec courage.
\VS{33}Et les plus intelligents parmi le peuple donneront instruction à plusieurs. Il en est qui succomberont pour un temps à l'épée et à la flamme, à la captivité et au pillage.
\VS{34}Dans le temps où ils succomberont, ils seront un peu secourus, et plusieurs se joindront à eux par hypocrisie.
\VS{35}Et quelques-uns des hommes intelligents succomberont, afin qu'ils soient épurés, purifiés et blanchis, jusqu'au temps de la fin, car elle n'arrivera qu'au temps marqué.
\TextTitle{Blasphème du roi contre Yahweh, le Dieu des dieux}
\VS{36}Le roi fera ce qu'il voudra, il s'élèvera, il se glorifiera au-dessus de tous les dieux ; il proférera des choses étranges contre le Dieu des dieux, il prospérera jusqu'à ce que la colère soit consommée, car ce qui est décrété sera exécuté.
\VS{37}Il n'aura égard ni aux dieux de ses pères, ni à l'objet du désir des femmes ; il n'aura égard à aucun dieu ; car il s'élèvera au-dessus de tout.
\VS{38}Mais, à la place, il honorera le dieu Mahuzzim ; ce dieu que ses pères n'ont pas connu, il rendra des hommages avec de l'or et de l'argent, et des pierres précieuses, et des objets de prix.
\VS{39}C'est avec le dieu étranger qu'il agira contre les lieux les plus fortifiés ; et il comblera d'honneurs ceux qui le reconnaîtront, il les fera dominer sur plusieurs, il leur partagera des terres à prix d'argent.
\VS{40}Au temps de la fin, le roi du midi se heurtera contre lui de ses cornes. Et le roi du nord fondra sur lui comme une tempête, avec des chars et des cavaliers, et avec de nombreux navires ; il s'avancera dans les terres, se répandra comme un torrent et débordera.
\VS{41}Il entrera dans le pays de noblesse, et plusieurs pays succomberont ; mais Edom, Moab et les principaux des enfants d'Ammon seront délivrés de sa main.
\VS{42}Il étendra sa main sur ces pays-là, et le pays d'Egypte n'échappera point.
\VS{43}Il se rendra maître des trésors d'or et d'argent, et de toutes les choses précieuses de l'Egypte ; les Libyens et les Ethiopiens seront à sa suite.
\VS{44}Mais des nouvelles de l'orient et du nord viendront le troubler, et il partira avec une grande fureur, pour détruire et exterminer beaucoup de gens.
\VS{45}Il dressera les tentes de son palais entre les mers, vers la glorieuse et sainte montagne. Puis il arrivera à la fin, et personne ne lui donnera du secours.
\Chap{12}
\TextTitle{La résurrection pour le jugement éternel}
\VerseOne{}Or, en ce temps-là Michaël, ce grand Chef qui tient ferme pour les enfants de ton peuple, tiendra ferme ; et ce sera un temps de détresse, tel qu'il n'y en a point eu de semblable depuis que les nations existent jusqu'à ce temps-là. En ce temps-là, ceux de ton peuple qui seront trouvés inscrits dans le livre seront sauvés.
\TextTitle{Les deux résurrections}
\VS{2}Plusieurs de ceux qui dorment dans la poussière de la terre se réveilleront\FTNT{Il est question ici de la résurrection. Tout d'abord, il y aura la résurrection des morts en Christ, lors du retour de Jésus-Christ (1 Th. 4:12-17). Ensuite, il y aura celle de tous les saints lors du retour de Christ avec l'Eglise (Ap. 19 et 20). Enfin, la dernière résurrection interviendra à l'issue du millénium. Il s'agit de la résurrection des impies (Ap. 20:11-15). Voir également Jn. 5 : 24-29 ; Jn. 11:25.}, les uns pour la vie éternelle, et les autres pour l'opprobre, pour l'infamie éternelle.
\VS{3}Ceux qui auront été intelligents, brilleront comme la splendeur du ciel, et ceux qui auront amené plusieurs à la justice brilleront comme les étoiles, à toujours et à perpétuité\FTNT{Mt. 13:43.}.
\TextTitle{Dernières paroles de Yahweh à Daniel ; le livre scellé jusqu'au temps de la fin}
\VS{4}Mais toi, Daniel, tiens secrètes ces paroles, et scelle le livre jusqu'au temps de la fin. Plusieurs le liront et la connaissance augmentera\FTNT{Ap. 10:4 ; Ap. 5:2.}.
\VS{5}Et moi, Daniel, je regardai, et voici, deux autres hommes se tenaient debout, l'un en deçà du bord du fleuve, et l'autre au-delà du bord du fleuve.
\VS{6}L'un d'eux dit à l'homme vêtu de lin, qui se tenait au-dessus des eaux du fleuve : Quand sera la fin de ces merveilles ?
\VS{7}Et j'entendis l'homme vêtu de lin, qui se tenait au-dessus des eaux du fleuve ; il leva sa main droite et sa main gauche vers les cieux, et il jura par celui qui vit éternellement que ce sera dans un temps, des temps, et la moitié d'un temps, et que toutes ces choses finiront quand la force du peuple saint sera entièrement brisée.
\VS{8}J'entendis, mais je ne compris pas ; et je dis : Mon seigneur, quelle sera l'issue de ces choses ?
\VS{9}Il répondit : Va, Daniel, car ces paroles sont tenues secrètes et scellées jusqu'au temps de la fin.
\VS{10}Plusieurs seront purifiés, blanchis et éprouvés ; mais les méchants agiront avec méchanceté, et aucun des méchants ne comprendra, mais les sages comprendront.
\VS{11}Depuis le temps où cessera le sacrifice perpétuel et où sera dressée l'abomination de la désolation, il y aura mille deux cent quatre-vingt-dix jours\FTNT{Mt. 24:15 ; Mc. 13:14 ; Lu. 21:20.}.
\VS{12}Heureux celui qui attendra et qui parviendra jusqu'à mille trois cent trente-cinq jours.
\VS{13}Mais toi, marche vers ta fin ; néanmoins tu te reposeras, et tu seras debout pour ton héritage à la fin des jours.
\PPE{}
\end{multicols}

%\clearpage\ShortTitle{Esdras}\BookTitle{Esdras}\BFont
\noindent\hrulefill
{\footnotesize
\textit{
\bigskip
{\centering{}
\\Auteur : Esdras
\\(Heb : Ezrah)
\\Signification : Secours
\\Thème : Edit de Cyrus et reconstruction du temple 
\\Date de rédaction : 5ème siècle av. J.-C.\\}
}
%\bigskip
\textit{
\\Conformément aux prophéties reçues par Esaïe et Jérémie, Yahweh toucha le cœur du roi Cyrus afin de renvoyer les fils d’Israël sur leur terre avec la mission de reconstruire le temple détruit quelques décennies auparavant. Ce livre montre comment Dieu ramena glorieusement son peuple à Jérusalem et retrace la reconstruction du temple ainsi que les épreuves ayant accompagné ce projet. Il traite des réformes sociales et religieuses mises en place dans le cadre d’un retour total à Yahweh.\bigskip
}
}
\par\nobreak\noindent\hrulefill
\begin{multicols}{2}
\Chap{1}
\TextTitle{Publication de Cyrus}
\VerseOne{}La première année de Cyrus\FTNT{538 av. J.-C.}, roi de Perse, afin que la parole de Yahweh prononcée par la bouche de Jérémie\FTNT{Jé. 25:12 ; 29:10 ; 33:7-10.} soit accomplie,  Yahweh réveilla l'esprit de Cyrus, roi de Perse, qui fit publier par écrit et de vive voix dans tout son royaume, en disant :
\VS{2}Ainsi parle Cyrus, roi de Perse : Yahweh, le Dieu des cieux, m'a donné tous les royaumes de la terre, et il m'a ordonné de lui bâtir une maison à Jérusalem, en Juda.
\VS{3}Qui d'entre vous est de son peuple, qui veut s'y employer ? Que son Dieu soit avec lui, qu'il monte à Jérusalem, en Juda, et qu'il rebâtisse la maison de Yahweh, le Dieu d'Israël ! C'est le Dieu qui est à Jérusalem.
\VS{4}Dans tout lieu où séjournent des restes du peuple,  les gens du lieu leur donneront de l’argent, de l’or, des biens, et du bétail, avec des offrandes volontaires pour la maison du Dieu qui est à Jérusalem.
\TextTitle{Cyrus rend les ustensiles}
\VS{5}Alors les chefs des familles de Juda et de Benjamin, les sacrificateurs et les Lévites, tous ceux dont Dieu réveilla l'esprit, se levèrent afin de monter pour rebâtir la maison de Yahweh à Jérusalem.
\VS{6}Tous ceux qui étaient autour d'eux les encouragèrent, leur fournissant des objets d'argent, d'or, des biens, du bétail, et des choses précieuses, outre toutes les offrandes volontaires.
\VS{7}Le roi Cyrus prit les ustensiles de la maison de Yahweh, que Nebucadnetsar avait emportés\FTNT{2 R. 24:13 ; 2 Ch 36:7.} de Jérusalem et mis dans la maison de son dieu.
\VS{8}Cyrus, roi de Perse, les fit sortir par Mithredath, le trésorier, qui les remit à Scheschbatsar, prince de Juda.
\VS{9}Et voici leur nombre : Trente bassins d'or, mille bassins d'argent, vingt-neuf couteaux,
\VS{10}trente coupes d'or, quatre cent dix coupes d'argent de second ordre, et d'autres ustensiles par milliers.
\VS{11}Tous les ustensiles d'or et d'argent étaient de cinq mille quatre cents. Scheschbatsar emporta le tout, lorsqu’on fit remonter de Babylone à Jérusalem ceux de la captivité.
\Chap{2}
\TextTitle{Dénombrement des Israélites revenus de captivité}
\VerseOne{}Voici ceux de la province qui revinrent de la captivité, d'entre ceux que Nebucadnetsar, roi de Babylone, avait transportés en exil à Babylone, et qui retournèrent à Jérusalem, et en Juda ; chacun dans sa ville\FTNT{Esd. 5:8 ; Né. 1:3 ; Né. 7:6.}.
\VS{2}Ils vinrent avec Zorobabel, Josué, Néhémie, Seraja, Reélaja, Mardochée, Bilschan, Mispar, Bigvaï, Rehum, Baana. Nombre des hommes du peuple d'Israël :
\VS{3}Les fils de Pareosch, deux mille cent soixante-douze\FTNT{Né. 7:8.} ;
\VS{4}les fils de Schephathia, trois cent soixante-douze ;
\VS{5}les fils d'Arach, sept cent soixante-quinze ;
\VS{6}les fils de Pachath-Moab, des fils de Josué, de Joab, deux mille huit cent douze ;
\VS{7}les fils d'Elam, mille deux cent cinquante-quatre ;
\VS{8}les fils de Zatthu, neuf cent quarante-cinq ;
\VS{9}les fils de Zaccaï, sept cent soixante ;
\VS{10}les fils de Bani, six cent quarante-deux\FTNT{Né. 7:15.} ;
\VS{11}les fils de Bébaï, six cent vingt-trois ;
\VS{12}les fils d'Hazgad, mille deux cent vingt-deux ;
\VS{13}les fils d'Adonikam, six cent soixante-six ;
\VS{14}les fils de Bigvaï, deux mille cinquante-six ;
\VS{15}les fils d’Adin, quatre cent cinquante-quatre ;
\VS{16}les fils d'Ather, de la famille d'Ezéchias, quatre-vingt-dix-huit ;
\VS{17}les fils de Betsaï, trois cent vingt-trois ;
\VS{18}les fils de Jora, cent douze ;
\VS{19}les fils de Haschum, deux cent vingt-trois ;
\VS{20}les fils de Guibbar, quatre-vingt-quinze ;
\VS{21}les fils de Bethléhem, cent vingt-trois ;
\VS{22}les gens de Netopha, cinquante-six ;
\VS{23}les gens d'Anathoth, cent vingt-huit ;
\VS{24}les fils d'Azmaveth, quarante-deux ;
\VS{25}les fils de Kirjath-Arim, de Kephira, et de Beéroth, sept cent quarante-trois ;
\VS{26}les fils de Rama et de Guéba, six cent vingt et un ;
\VS{27}les gens de Micmas, cent vingt-deux ;
\VS{28}les gens de Béthel et d’Aï, deux cent vingt-trois ;
\VS{29}les fils de Nebo, cinquante-deux ;
\VS{30}les fils de Magbisch, cent cinquante-six.
\VS{31}les fils d'un autre Elam, mille deux cent cinquante-quatre ;
\VS{32}les fils de Harim, trois cent vingt ;
\VS{33}les fils de Lod, de Hadid, et d'Ono, sept cent vingt-cinq ;
\VS{34}les fils de Jéricho, trois cent quarante-cinq ;
\VS{35}les fils de Senaa, trois mille six cent trente.
\TextTitle{Dénombrement des sacrificateurs revenus de captivité}
\VS{36}Des sacrificateurs : Les fils de Jedaeja, de la maison de Josué, neuf cent soixante-treize ;
\VS{37}les fils d'Immer, mille cinquante-deux ;
\VS{38}les fils de Paschhur, mille deux cent quarante-sept ;
\VS{39}les fils de Harim, mille dix-sept.
\TextTitle{Dénombrement des Lévites revenus de captivité}
\VS{40}Des Lévites : Les fils de Josué et de Kadmiel, d'entre les fils d’Hodavia, soixante-quatorze.
\VS{41}Des chantres : Les fils d'Asaph, cent vingt-huit.
\VS{42}Des fils des portiers : Les fils de Schallum, les fils d'Ather, les fils de Thalmon, les fils d’Akkub, les fils de Hathitha, les fils de Schobaï, en tout cent trente-neuf.
\VS{43}Des Néthiniens : Les fils de Tsicha, les fils de Hasupha, les fils de Tabbahoth\FTNT{Esd.8:17 ; Jos. 9:23.},
\VS{44}les fils de Kéros, les fils de Siaha, les fils de Padon,
\VS{45}les fils de Lebana, les fils de Hagaba, les fils d'Akkub,
\VS{46}les fils de Hagab, les fils de Schamlaï, les fils de Hanan,
\VS{47}les fils de Guiddel, les fils de Gachar, les fils de Reaja,
\VS{48}les fils de Retsin, les fils de Nekoda, les fils de Gazzam,
\VS{49}les fils d'Uzza, les fils de Paséach, les fils de Bésaï,
\VS{50}les fils d'Asna, les fils de Mehunim, les fils de Nephusim,
\VS{51}les fils de Bakbuk, les fils de Hakupha, les fils de Harhur,
\VS{52}les fils de Batsluth, les fils de Mehida, les fils de Harscha,
\VS{53}les fils de Barkos, les fils de Sisera, les fils de Thamach,
\VS{54}les fils de Netsiach, les fils de Hathipha.
\TextTitle{Dénombrement des serviteurs de Salomon revenus de captivité}
\VS{55}Des fils des serviteurs de Salomon : Les fils de Sothaï, les fils de Sophéreth, les fils de Peruda,
\VS{56}les fils de Jaala, les fils de Darkon, les fils de Guiddel,
\VS{57}les fils de Schephathia, les fils de Hatthil, les fils de Pokéreth-Hatsebaïm, les fils d'Ami.
\VS{58}Total des Néthiniens et des fils des serviteurs de Salomon : Trois cent quatre-vingt-douze.
\VS{59}Voici ceux qui montèrent de Thel-Mélach, de Thel-Harscha, de Kerub-Addan et qui ne purent pas faire connaître leur maison paternelle et leur race, pour prouver qu’ils étaient d'Israël :
\VS{60}Les fils de Delaja, les fils de Tobija, les fils de Nekoda, six cent cinquante-deux.
\TextTitle{Certains sacrificateurs rejetés de la sacrificature}
\VS{61}Des fils des sacrificateurs : Les fils de Habaja, les fils d'Hakkots, les fils de Barzillaï, qui avait pris pour femme une des filles de Barzillaï, le Galaadite, fut appelé de leur nom.
\VS{62}Ils cherchèrent leurs registres généalogiques, mais ils ne les trouvèrent point. C'est pourquoi ils furent rejetés pour ne pas souiller le sacerdoce,
\VS{63}et le gouverneur leur dit de ne pas manger des choses très saintes, en attendant qu'un sacrificateur ait consulté l'urim et le thummim.
\TextTitle{Nombre total des Israélites revenus de captivité}
\VS{64}L’assemblée tout entière était de quarante-deux mille trois cent soixante,
\VS{65}sans leurs serviteurs et leurs servantes, qui étaient sept mille trois cent trente-sept. Ils avaient deux cents chantres ou chanteuses.
\VS{66}Ils avaient sept cent trente-six chevaux, et deux cent quarante-cinq mulets,
\VS{67}quatre cent trente-cinq chameaux, et six mille sept cent vingt ânes.
\VS{68}Quelques-uns d'entre les chefs des pères, quand ils vinrent à la maison de Yahweh à Jérusalem, firent des offrandes volontaires pour la maison de Dieu, afin qu'on la rétablît sur son emplacement.
\VS{69}Ils donnèrent au trésor de l'ouvrage, selon leurs moyens, soixante et un mille drachmes d'or, et cinq mille mines d'argent, et cent tuniques de sacrificateurs.
\VS{70}Ainsi, les sacrificateurs, les Lévites, quelques-uns du peuple, les chantres, les portiers et les Néthiniens habitèrent dans leurs villes. Et tous ceux d'Israël dans leurs villes aussi.
\Chap{3}
\TextTitle{Rétablissement de l'autel et des sacrifices}
\VerseOne{}Le septième mois approcha, et les fils d'Israël étaient dans leurs villes. Le peuple s'assembla alors comme un seul homme à Jérusalem.
\VS{2}Alors\FTNT{Ag. 1:1 ; De.12:5-6.} Josué, fils de Jotsadak, avec ses frères les sacrificateurs, et Zorobabel, fils de Schealthiel, avec ses frères, se levèrent et bâtirent l'autel du Dieu d'Israël, pour y offrir des holocaustes, comme il est écrit dans la loi de Moïse, homme de Dieu.
\VS{3}Ils rétablirent l'autel de Dieu sur ses fondements, parce qu'ils avaient peur en eux-mêmes des peuples du pays, et ils y offrirent des holocaustes à Yahweh, les holocaustes du matin et du soir\FTNT{No. 28:3.}.
\VS{4}Ils célébrèrent aussi la fête des tabernacles, comme il est écrit, et ils offrirent des holocaustes, autant qu'il en fallait  chaque jour\FTNT{Lé. 23:34 ; No. 29:12.}.
\VS{5}Après cela, ils offrirent l'holocauste perpétuel, ceux des nouvelles lunes, de toutes les fêtes solennelles consacrées à Yahweh, et ceux de quiconque faisait des offrandes volontaires à Yahweh\FTNT{No. 28:11 ; Né.10:33.}.
\VS{6}Dès le premier jour du septième mois, ils commencèrent à offrir des holocaustes à Yahweh. Cependant, les fondements du temple de Yahweh n'étaient pas encore posés.
\VS{7}Ils donnèrent de l'argent aux tailleurs de pierres et aux charpentiers, et aussi de la nourriture, des boissons, de l'huile aux Sidoniens et aux Tyriens, afin qu'ils amènent du bois de cèdre du Liban par la mer de Japho, selon la permission que Cyrus, roi de Perse, leur en avait donnée.
\TextTitle{Les fondements du temple posés}
\VS{8}Et la deuxième année depuis leur arrivée à la maison de Dieu à Jérusalem, au deuxième mois, Zorobabel, fils de Schealthiel,  Josué, fils de Jotsadak, et le reste de leurs frères les sacrificateurs et les Lévites, et tous ceux qui étaient revenus de la captivité à Jérusalem, débutèrent l’œuvre et désignèrent des Lévites, depuis l'âge de vingt ans et au-dessus pour surveiller l'ouvrage de la maison de Yahweh.
\VS{9}Et Josué, avec ses fils et ses frères, Kadmiel, avec ses fils, fils de Juda, les fils de Hénadad, avec leurs fils et leurs frères les Lévites, se tenaient debout pour surveiller ceux qui faisaient l'ouvrage de la maison de Dieu.
\VS{10}Et lorsque ceux qui bâtissaient posèrent les fondements du temple de Yahweh, on fit assister les sacrificateurs revêtus de leurs habits, avec leurs trompettes, et les Lévites, fils d'Asaph, avec les cymbales, pour qu’ils célèbrent Yahweh, selon l’institution de David, roi d'Israël.
\VS{11}Et en louant et célébrant Yahweh, ils s'entre-répondaient : Il est bon, parce que sa miséricorde demeure à toujours sur Israël ! Et tout le peuple poussait de grands cris de joie en louant Yahweh, parce qu'on posait les fondements de la maison de Yahweh.
\VS{12}Mais plusieurs des sacrificateurs et des Lévites, et des chefs de familles âgés, qui avaient vu la première maison, pleuraient à grand bruit pendant qu'on posait sous leurs yeux les fondements de cette maison. Et beaucoup élevaient leur voix avec des cris de joie,
\VS{13}et le peuple ne pouvait distinguer le bruit des cris de joie  d'avec le bruit des pleurs du peuple, car le peuple poussait de grands cris de joie dont le son s’entendait de très loin.
\Chap{4}
\TextTitle{Les ennemis de Juda et de Benjamin découragent le peuple de Juda}
\VerseOne{}Les ennemis de Juda et de Benjamin entendirent que les fils de la captivité rebâtissaient un temple à Yahweh, le Dieu d'Israël.
\VS{2}Ils vinrent vers Zorobabel et vers les chefs des familles, et leur dirent : Nous bâtirons\FTNT{On ne doit jamais s’associer avec les impies pour bâtir l’œuvre du Seigneur. Satan essaie toujours de s’infiltrer dans les assemblées afin de nous éloigner de la vérité, c’est pour cela que nous devons faire preuve de discernement (2 Co. 6:14-16).} avec vous ; car nous invoquons votre Dieu comme vous ; et nous lui avons sacrifié depuis le temps d'Esar-Haddon, roi d'Assyrie, qui nous a fait monter ici.
\VS{3}Mais Zorobabel, Josué, et les autres chefs des familles d'Israël, leur répondirent : Il ne convient pas à vous de bâtir la maison de notre Dieu ; mais nous, qui sommes ici ensemble, nous la bâtirons à Yahweh, le Dieu d'Israël, comme nous l'a ordonné le roi Cyrus, roi de Perse\FTNT{Esd. 1:1,2,5.}.
\VS{4}Alors les gens du pays rendirent paresseuses les mains du peuple de Juda ; ils l’intimidèrent pour l'empêcher de bâtir,
\VS{5}ils avaient même engagé à prix d’argent des conseillers pour faire échouer leur projet, pendant toute la durée de vie de Cyrus, roi de Perse, jusqu'au règne de Darius, roi de Perse.
\VS{6}Et sous le règne d'Assuérus, au commencement de son règne, ils écrivirent une accusation contre les habitants de Juda et de Jérusalem.
\TextTitle{Lettre envoyé à Artaxerxès}
\VS{7}Et du temps d'Artaxerxès, Bischlam, Mithredath, Thabeel, et le reste de leurs collègues, écrivirent à Artaxerxès, roi de Perse. La lettre était écrite en caractères araméens, et elle était traduite en araméen.
\VS{8}Rehum, le gouverneur, et Schimschaï, le secrétaire, écrivirent au roi Artaxerxès la lettre suivante concernant Jérusalem :
\VS{9}Rehum, gouverneur, Schimschaï, secrétaire, et le reste de leurs collègues, ceux de Din, d'Arpharsathac, de Tharpel, d'Apharas, d'Erec, de Babylone, de Suse, de Déha, d'Elam,
\VS{10}et les autres peuples que le grand et illustre Osnappar a transportés et fait habiter dans la ville de Samarie et les autres régions au-delà du fleuve, à cette date.
\VS{11}Voici donc ici la copie de la lettre qu'ils envoyèrent au roi Artaxerxès : Tes serviteurs, les gens de ce côté du fleuve, à cette date.
\VS{12}Que le roi sache que les Juifs qui sont montés de chez lui et arrivés vers nous à Jérusalem rebâtissent la ville rebelle et méchante, et achèvent en finissant de poser et de réparer les fondements des murs.
\VS{13}Que le roi sache donc que si cette ville est rebâtie et si ses murs sont réparés, ils ne paieront plus de tribut, ni d’impôt, ni de droit de passage, et elle causera une grande nuisance aux revenus du roi.
\VS{14}Et parce que nous mangeons le sel du palais, il ne nous parait pas convenable de voir le roi déshonoré ; c'est pourquoi nous envoyons au roi ces informations.
\VS{15}Qu'on recherche dans le livre des mémoires de tes pères, et tu trouveras et tu apprendras dans ce livre des mémoires que cette ville est une ville rebelle, nuisible aux rois et aux provinces ; et qu’on s’y est livré à la révolte depuis toujours. Donc cette ville a été détruite à cause de cela.
\VS{16}Nous faisons donc savoir au roi que si cette ville est rebâtie et si ses murs sont relevés, il n'aura plus de possession de ce côté du fleuve.
\TextTitle{Réponse du roi Artaxerxès}
\VS{17}Et c'est ici le décret envoyé par le roi à Rehum, le gouverneur, à Schimschaï, le secrétaire, et au reste de leurs collègues demeurant à Samarie, et aux autres de l'autre côté du fleuve : Paix sur vous, à cette date.
\VS{18}La lettre que vous nous avez envoyée a été lue exactement en ma présence.
\VS{19}J'ai donné ordre de faire des recherches et l’on a trouvé  que depuis toujours cette ville s'est soulevée contre les rois, et qu'on s’y est livré à la sédition et à la révolte.
\VS{20}Il y eut aussi à Jérusalem des rois puissants, maîtres de tout le pays de l'autre côté du fleuve, et auxquels on payait  tribut, impôt et droit de passage\FTNT{2 S. 8:2,6 ; 1 R 4:21 ; 2 Ch. 17:11 ; 32:23.}.
\VS{21}A présent, donnez l’ordre de ne pas laisser continuer ces gens-là, afin que cette ville ne se rebâtisse point, jusqu’à ce que je l'ordonne par décret.
\VS{22}Gardez-vous de mettre en cela de la négligence, de peur que le mal  n’augmente au préjudice des rois.
\VS{23}Aussitôt que la copie de la lettre du roi Artaxerxès eut été lue en présence de Rehum, de Schimschaï,  le secrétaire, et de leurs collègues, ils allèrent en hâte à Jérusalem vers les Juifs, et ils les firent cesser leurs travaux avec violence et force.
\VS{24}Alors l’ouvrage de la maison de Dieu, à Jérusalem, cessa, et elle demeura dans cet état, jusqu'à la deuxième année du règne de Darius, roi de Perse\FTNT{Esd. 5:2}.
\Chap{5}
\TextTitle{Aggée et Zacharie prophétisent}
\VerseOne{}Aggée, le prophète, et Zacharie, fils d'Iddo, le prophète, prophétisèrent aux Juifs qui étaient en Juda et à Jérusalem, au nom du Dieu d'Israël, qui s’adressait à eux\FTNT{Ag. 1:4 ; Za. 1:1.}.
\VS{2}Alors Zorobabel, fils de Schealthiel, et Josué, fils de Jotsadak, se levèrent et commencèrent à rebâtir la maison de Dieu à Jérusalem. Et ils avaient avec eux les prophètes de Dieu qui les soutenaient\FTNT{Ag. 1:14 ; Esd. 6:14.}.
\VS{3}En ce temps-là, Thathnaï, gouverneur de ce côté du fleuve, et Schethar-Boznaï, et leurs collègues, vinrent à eux et leur parlèrent ainsi : Qui vous a donné l’ordre de rebâtir cette maison et de relever ces murs ?\FTNT{Esd. 5:9.}
\VS{4}Ils leur dirent alors : Quels sont les noms des hommes qui construisent cet édifice ?
\VS{5}Mais l’œil de Dieu était sur les anciens des Juifs. Et on ne les laissa continuer les travaux, pendant l’envoi d’un rapport à Darius, et jusqu'à la réception d’une lettre sur cet objet.
\TextTitle{Thathnaï, Schethar-Boznaï et leurs collègues d'Apharsac écrivent à Darius}
\VS{6}Copie de la lettre envoyée au roi Darius par Thathnaï, gouverneur de ce côté du fleuve, Schethar-Boznaï, et leurs collègues d'Apharsac, de l’autre côté du fleuve.
\VS{7}Ils lui envoyèrent un rapport ainsi écrit : Paix parfaite soit au roi Darius !
\VS{8}Que le roi sache que nous sommes allés dans la province de Juda, vers la maison du grand Dieu. Elle se bâtit avec des pierres de taille, et le bois se pose dans les murs ; ce travail se réalise complètement et prospère entre leurs mains\FTNT{Esd. 2:1.}.
\VS{9}Nous avons interrogé ces anciens, et nous leur avons parlé ainsi : Qui vous a donné l'autorisation de rebâtir cette maison et de finir ces murs ?\FTNT{Esd. 5:3.}
\VS{10}Nous leur avons aussi demandé leurs noms pour te les faire connaître, et nous avons mis par écrit les noms des hommes à leur tête.
\VS{11}Et ils nous ont répondu de cette manière, disant : Nous sommes les serviteurs du Dieu des cieux et de la terre, et nous rebâtissons la maison qui avait été bâtie il y a de nombreuses années ; un grand roi d'Israël l’avait bâtie et finie.
\VS{12}Mais après que nos pères eurent provoqué la colère du Dieu des cieux, il les livra entre les mains de Nebucadnetsar\FTNT{Voir 2 R. 24 et 25.}, roi de Babylone, Chaldéen, qui détruisit cette maison et qui emmena le peuple en exil à Babylone\FTNT{2 Ch. 36:7}.
\VS{13}Mais la première année de Cyrus, roi de Babylone, le roi Cyrus prit un décret pour rebâtir cette maison de Dieu\FTNT{Esd. 1:1-2.}.
\VS{14}Et même le roi Cyrus ôta du temple de Babylone les ustensiles d'or et d'argent de la maison de Dieu, que Nebucadnetsar avait sortis du temple qui était à Jérusalem et transportés dans le temple de Babylone, et il les fit remettre au nommé Scheschbatsar, qu’il établit gouverneur\FTNT{Esd. 1:8.},
\VS{15}et il lui dit : Prends ces ustensiles, et va les déposer dans le temple de Jérusalem ; et que la maison de Dieu soit rebâtie sur sa place.
\VS{16}Alors ce Scheschbatsar est venu, et il a posé les fondements de la maison de Dieu à Jérusalem ; et depuis ce temps-là jusqu'à présent, on la bâtit, et elle n'est point encore achevée.
\VS{17}Maintenant, s'il semble bon au roi, que l’on fasse des recherches dans la maison des trésors du roi à Babylone, pour voir s'il est vrai qu'il y a eu un ordre donné par Cyrus de rebâtir cette  maison de Dieu à Jérusalem. Puis, que le roi nous transmette sa volonté sur cet objet.
\Chap{6}
\TextTitle{Darius confirme l'édit de Cyrus}
\VerseOne{}Alors le roi Darius donna un ordre de faire des recherches dans la maison des livres où l'on déposait les  trésors à Babylone.
\VS{2}Et l’on trouva à Achmetha, dans un coffre, capitale de la province de Médie, un rouleau à l’intérieur duquel était écrit  le mémoire suivant :
\VS{3}La première année du roi Cyrus, le roi Cyrus prit un décret quant à la maison de Dieu à Jérusalem : Que cette maison soit rebâtie, afin d’être un lieu où l'on offre des sacrifices, et que ses fondements soient solides pour porter sa charge. La hauteur sera de soixante coudées, et la longueur de soixante coudées,
\VS{4}trois rangées de pierres de taille et une rangée de bois neuf.  La dépense sera payée par la maison du roi.
\VS{5}Aussi, les ustensiles d'or et d'argent de la maison de Dieu, que Nebucadnetsar avait enlevés du temple de Jérusalem et apportés à Babylone, seront remis et apportés dans le temple de Jérusalem, à leur place, et déposés dans la maison de Dieu.
\VS{6}Maintenant, Thathnaï, gouverneur de l'autre côté du fleuve, Schethar-Boznaï, et vos collègues d'Apharsac de l'autre côté du fleuve, tenez-vous loin de ce lieu.
\VS{7}Laissez le travail de cette maison de Dieu ; que le gouverneur des Juifs et les anciens des Juifs rebâtissent cette maison de Dieu à sa place.
\VS{8}En raison de ce décret pris, ce que vous aurez à exécuter, avec les anciens de ces Juifs pour rebâtir cette maison de Dieu : Sur les finances du roi provenant du tribut de l’autre côté du fleuve, les frais seront complètement payés à ces hommes, afin qu'il n'y ait pas d'interruption.
\VS{9}Et ce qui sera nécessaire pour les holocaustes du Dieu des cieux, veaux, béliers et agneaux, blé, sel, vin et huile, seront livrés, sur leur demande, aux sacrificateurs de Jérusalem, jour après jour, sans négligence,
\VS{10}afin qu'ils offrent des sacrifices de bonne odeur au Dieu des cieux et qu'ils prient pour la vie du roi et de ses fils.
\VS{11}Et voici l’ordre que je donne touchant quiconque changera cette parole : On arrachera de sa maison une pièce de bois, on la dressera, afin qu'il y soit exterminé, et l’on fera de sa maison un tas de déchets\FTNT{2 R. 10:27 ; Ez. 6:11 ; Da. 3:29.}.
\VS{12}Et que Dieu, qui fait résider en ce lieu son nom, renverse tout roi et tout peuple qui étendrait sa main pour changer et détruire cette maison de Dieu à Jérusalem ! Moi, Darius, j’ai donné cet ordre. Qu'il soit donc exécuté complètement.
\TextTitle{Achèvement et dédicace de la maison de Dieu}
\VS{13}Alors Thathnaï, gouverneur de l'autre côté du fleuve,  Schethar-Boznaï, et leurs collègues, firent exécuter ainsi complètement ce que le roi Darius leur envoya.
\VS{14}Et les anciens des Juifs bâtirent avec succès, selon les prophéties d'Aggée, le prophète, et de Zacharie, fils d’Iddo ; ils bâtirent et finirent, d'après l'ordre du Dieu d'Israël, et d'après l'ordre de Cyrus, de Darius, et d'Artaxerxès, roi de Perse.
\VS{15}Cette maison fut achevée le troisième jour du mois d'Adar, dans la sixième année du règne du roi Darius.
\VS{16}Les fils d'Israël, les sacrificateurs, les Lévites, et le reste des fils de la captivité, célébrèrent la dédicace de cette maison de Dieu avec joie.
\VS{17}Ils offrirent pour la dédicace de cette maison de Dieu, cent taureaux, deux cents béliers, quatre cents agneaux, et douze boucs comme victimes expiatoires pour tout Israël, selon le nombre des tribus d'Israël.
\VS{18}Ils établirent les sacrificateurs selon leurs classes et les Lévites selon leurs divisions, pour le service de Dieu à Jérusalem,  selon ce qui est écrit dans le livre de Moïse\FTNT{No. 3:6,32 ; No. 8:11}.
\TextTitle{Rétablissement de la Pâque}
\VS{19}Puis les fils de la captivité célébrèrent la Pâque le quatorzième jour du premier mois\FTNT{Lé. 23:5 ; No. 28:16 ; De. 16:2.}.
\VS{20}Les sacrificateurs et les Lévites s'étaient purifiés comme un seul homme, tous étaient purs ; c'est pourquoi ils immolèrent la Pâque pour tous les fils de la captivité, pour leurs frères les sacrificateurs, et pour eux-mêmes\FTNT{2 Ch. 30:15,17,21}.
\VS{21}Les fils d'Israël revenus de la captivité mangèrent la Pâque, avec tous ceux qui s'étaient séparés de l’impureté des nations du pays pour chercher Yahweh, le Dieu d'Israël.
\VS{22}Ils célébrèrent avec joie la fête des pains sans levain pendant sept jours, car Yahweh les avait réjouis en disposant  le cœur du roi d'Assyrie à fortifier leurs mains dans l’œuvre de la maison de Dieu, du Dieu d'Israël.
\Chap{7}
\TextTitle{Voyage d'Esdras jusqu'à Jérusalem}
\VerseOne{}Après ces choses, sous le règne d'Artaxerxès, roi de Perse, Esdras, fils de Seraja, fils d'Azaria, fils de Hilkija\FTNT{Esd. 6:14.},
\VS{2}fils de Schallum, fils de Tsadok, fils d'Achithub,
\VS{3}fils d'Amaria, fils d'Azaria, fils de Merajoth,
\VS{4}fils de Zerachja, fils d'Uzzi, fils de Bukki,
\VS{5}fils d'Abischua, fils de Phinées, fils d'Eléazar, fils d'Aaron, souverain sacrificateur.
\VS{6}Esdras monta de Babylone : C’était un scribe bien exercé dans la loi de Moïse, donnée par Yahweh, le Dieu d'Israël. Et comme la main de Yahweh, son Dieu, était sur lui, le roi lui accorda toute sa requête\FTNT{Vers. 9,28.}.
\VS{7}Des fils d'Israël, des sacrificateurs, des Lévites, des chantres, des portiers, et des Néthiniens, montèrent à Jérusalem, la septième année du roi Artaxerxès.
\VS{8}Il entra à Jérusalem le cinquième mois de la septième année du roi ;
\VS{9}il était parti de Babylone au premier jour du premier mois, et il entra à Jérusalem au premier jour du cinquième mois, selon que la main de son Dieu était bonne sur lui.
\VS{10}Car Esdras avait disposé son cœur à étudier la loi de Yahweh, à l’observer et à enseigner les lois et les ordonnances parmi le peuple d'Israël.
\TextTitle{Lettre d'Artaxerxès à Esdras}
\VS{11}Voici la copie de la lettre que le roi Artaxerxès donna à Esdras, sacrificateur et scribe, enseignant les paroles des commandements de Yahweh et ses ordonnances concernant Israël :
\VS{12}Artaxerxès, roi des rois, à Esdras, sacrificateur et scribe de la loi du Dieu des cieux, à cette date.
\VS{13}J’ai donné ordre de laisser aller tous ceux de mon royaume qui sont du peuple d'Israël, de ses sacrificateurs et Lévites, qui se présenteront volontairement pour aller avec toi à Jérusalem.
\VS{14}Tu es envoyé de la part du roi, et de ses sept conseillers, pour inspecter Juda et Jérusalem touchant la loi de ton Dieu, laquelle est entre tes mains,
\VS{15}et pour porter l'argent et l'or que le roi et ses conseillers ont offert volontairement au Dieu d'Israël, dont la demeure est à Jérusalem\FTNT{Esd. 8:24.},
\VS{16}tout l'argent et l'or que tu trouveras dans toute la province de Babylone, avec les offrandes volontaires du peuple et des sacrificateurs, qu'ils feront volontairement à la maison de leur Dieu à Jérusalem.
\VS{17} C'est pourquoi tu achèteras avec cet argent des taureaux, des béliers, des agneaux, avec leurs offrandes et leurs libations, et tu les offriras sur l'autel de la maison de votre Dieu à Jérusalem.
\VS{18}Vous ferez, selon la volonté de votre Dieu, ce qu'il te semblera bon à toi et à tes frères de faire du reste de l'argent et de l'or.
\VS{19}Et pour ce qui est des ustensiles qui te sont remis pour le service de la maison de ton Dieu, déposes-les en présence du Dieu de Jérusalem.
\VS{20}Quand au reste de ce qui sera nécessaire pour la maison de ton Dieu, autant qu'il t'en faudra employer, tu le prendras de la maison des trésors du roi.
\VS{21}Moi, le roi Artaxerxès, je donne l’ordre à tous les trésoriers qui sont de l'autre côté du fleuve de livrer exactement à Esdras, sacrificateur et scribe de la loi du Dieu des cieux, tout ce qu’il vous demandera,
\VS{22}jusqu'à cent talents d'argent, cent cors de froment, cent baths de vin, cent baths d'huile, et du sel sans nombre.
\VS{23}Que tout ce qui est ordonné par le Dieu des cieux se fasse exactement pour la maison du Dieu des cieux, afin que sa colère ne soit pas sur le royaume, sur le roi et sur ses fils.
\VS{24}Nous vous faisons savoir qu'on ne pourra imposer ni tribut, ni impôt, ni droit de passage sur aucun des sacrificateurs, des Lévites, des chantres, des portiers, des  Néthiniens, et des serviteurs de cette maison de Dieu.
\VS{25}Et toi, Esdras, établis des magistrats et des juges selon la sagesse de ton Dieu que tu possèdes, afin qu'ils rendent justice à tout ce peuple de l'autre côté du fleuve, à tous ceux qui connaissent les lois de ton Dieu ; afin que vous enseigniez celui qui ne les connaît point.
\VS{26}Et tous ceux qui n'observeront point la loi de ton Dieu et la loi du roi seront aussitôt jugés, soit à la mort, soit au bannissement, soit à une amende pécuniaire, ou à l'emprisonnement.
\VS{27}Béni soit Yahweh, le Dieu de nos pères, qui a mis cela au cœur du roi, pour honorer la maison de Yahweh, qui est à Jérusalem ;
\VS{28}et qui a fait que j'ai trouvé grâce devant le  roi, devant ses conseillers, et devant tous les puissants chefs ! Fortifié par la main de Yahweh, mon Dieu, qui était sur moi, j'ai rassemblé les chefs d'Israël, afin qu'ils montent avec moi.
\Chap{8}
\TextTitle{Dénombrement de ceux qui montèrent avec Esdras}
\VerseOne{}Voici les chefs des pères, avec le dénombrement fait selon les généalogies de ceux qui montèrent avec moi de Babylone, pendant le règne du roi Artaxerxès\FTNT{1 Ch. 4:33.}.
\VS{2}Des fils de Phinées, Guerschom ; des fils d'Ithamar, Daniel ; des fils de David, Hattusch ;
\VS{3}des fils de Schecania ; des fils de Pareosch, Zacharie, et avec lui, en faisant le dénombrement par leur généalogie selon les hommes, cent cinquante hommes ;
\VS{4}des fils de Pachat Moab, Eljoénaï, fils de Zerachja, et avec lui deux cents hommes;
\VS{5}des fils de Schecania, le fils de Jachaziel, et avec lui trois cents hommes;
\VS{6}des fils d'Adin, Ebed, fils de Jonathan, et avec lui cinquante hommes ;
\VS{7}des fils d'Elam, Esaïe, fils d'Athalia, et avec lui soixante-dix hommes;
\VS{8}des fils de Schephathia, Zebadia, fils de Micaël, et avec lui quatre-vingts hommes ;
\VS{9}des fils de Joab, Abdias, fils de Jehiel, et avec lui deux cent dix-huit hommes ;
\VS{10}des fils de Schelomith, le fils de Josiphia, et avec lui cent soixante hommes ;
\VS{11}des fils de Bébaï, Zacharie, fils de Bébaï, et avec lui vingt-huit hommes ;
\VS{12}des fils d'Azgad, Jochanan, fils d'Hakkathan, et avec lui cent-dix hommes ;
\VS{13}des fils d'Adonikam, les derniers, dont voici les noms: Eliphélet, Jeïel, et Schemaeja, et avec eux soixante hommes ;
\VS{14}des fils de Bigvaï, Uthaï, Zabbud, et avec eux soixante-dix hommes.
\VS{15}Je les rassemblai près du fleuve qui coule vers Ahava, et nous campâmes là trois jours. Puis je portai mon attention sur  le peuple et les sacrificateurs, et je n'y trouvai aucun des fils de Lévi.
\VS{16}Alors j'envoyai d'entre les chefs Eliézer, Ariel, Schemaeja, Elnathan, Jarib, Elnathan, Nathan, Zacharie et Meschullam, avec les docteurs Jojarib et Elnathan.
\VS{17}Je leur donnai des ordres pour le chef Iddo, demeurant à Casiphia, et je mis dans leur bouche les paroles qu'ils devaient dire à Iddo et à ses frères les Néthiniens, qui étaient à Casiphia, afin qu'ils nous amènent des serviteurs pour la maison de notre Dieu\FTNT{Esd. 2:43.}.
\VS{18}Et comme la bonne main de notre Dieu était sur nous, ils nous amenèrent Schérébia, un homme intelligent, d'entre les fils de Machli, fils de Lévi, fils d'Israël, et avec ses fils et ses frères, au nombre dix-huit\FTNT{Esd. 7:6,9,28.} ;
\VS{19}Haschabia, et avec lui Esaïe, d'entre les fils de Merari, ses frères, et leurs fils, au nombre vingt ;
\VS{20}et des Néthiniens, que David et les chefs du peuple avaient assignés pour le service des Lévites, deux cent vingt Néthiniens, tous désignés par leurs noms\FTNT{Esd. 2:43,58.}.
\TextTitle{Esdras publie un jeûne pour obtenir la protection de Dieu}
\VS{21}Et je publiai là un jeûne près de la rivière d'Ahava, afin de nous humilier devant notre Dieu, le priant de nous donner un heureux voyage, pour nos enfants, et pour tous nos biens.
\VS{22}Car j'aurais eu honte de demander au roi une armée et des cavaliers pour nous soutenir contre des ennemis pendant le chemin ; car nous avions dit au roi : La main de notre Dieu est favorable sur tous ceux qui le cherchent ; mais sa force et sa colère sont contre ceux qui l'abandonnent.
\VS{23}Nous jeûnâmes donc, et nous cherchâmes notre Dieu à cause de cela. Et il se laissa fléchir par nos prières.
\TextTitle{Trésors remis par Esdras entre les mains de douze sacrificateurs}
\VS{24}Alors je mis à part douze chefs des sacrificateurs, Schérébia, Haschabia, et dix de leurs frères.
\VS{25}Je pesai l'argent, l'or et les ustensiles donnés en offrandes pour la maison de notre Dieu par le roi, ses conseillers, ses chefs, et tous ceux d'Israël qu'on avait trouvés\FTNT{Esd. 7:14,15.}.
\VS{26}Je pesai donc, et je remis entre leurs mains six cent cinquante talents d'argent, des ustensiles d'argent pesant cent talents, cent talents d'or,
\VS{27}vingt coupes d'or valant mille drachmes, et deux ustensiles d’un bel airain poli, aussi précieux que de l'or.
\VS{28}Et je leur dis : Vous êtes consacrés à Yahweh ; et les ustensiles sont sanctifiés, et cet argent et cet or sont une offrande volontaire faite à Yahweh, le Dieu de vos pères.
\VS{29}Soyez vigilants et gardez-les, jusqu'à ce que vous les pesiez devant les chefs des sacrificateurs et les Lévites, et devant les chefs des pères d'Israël, à Jérusalem, dans les chambres de la maison de Yahweh.
\VS{30}Les sacrificateurs et les Lévites reçurent le poids de l'argent, de l'or, et des ustensiles, pour les apporter à Jérusalem, dans la maison de notre Dieu.
\TextTitle{Esdras arrive à Jérusalem}
\VS{31}Nous partîmes du fleuve d'Ahava pour aller à Jérusalem, le douzième jour du premier mois. La main de notre Dieu fut sur nous et nous délivra de la main des ennemis et des  embûches sur le chemin.
\VS{32}Puis nous arrivâmes à Jérusalem, et nous nous y reposâmes trois jours.
\VS{33}Le quatrième jour, nous pesâmes l'argent, l'or, et les ustensiles dans la maison de notre Dieu, et nous les remîmes à Merémoth, fils d'Urie, le sacrificateur - il était avec Eléazar, fils de Phinées, et avec eux les Lévites Jozabad, fils de Josué, et Noadia, fils de Binnuï-
\VS{34} selon tout le nombre et le poids de toutes ces choses, et tout le poids fut mis alors par écrit.
\VS{35}Et les fils de la captivité revenus de l’exil offrirent en holocauste au Dieu d'Israël douze taureaux, quatre-vingt-seize béliers, soixante-dix-sept agneaux, et douze boucs comme victimes expiatoires pour tout Israël, le tout en holocauste à Yahweh.
\VS{36}Ils transmirent les ordres du roi entre les mains des satrapes du roi et des gouverneurs qui étaient de ce côté du fleuve, lesquels favorisèrent le peuple et la maison de Dieu.
\Chap{9}
\TextTitle{La désobéissance}
\VerseOne{}Après que ces choses furent terminées, les chefs du peuple s'approchèrent de moi, en disant : Le peuple d'Israël,  les sacrificateurs et les Lévites ne se sont point séparés des peuples de ces pays, quant à leurs abominations, celles des Cananéens, des Héthiens, des Phéréziens, des Jébusiens, des Ammonites, des Moabites, des Egyptiens, et des Amoréens.
\VS{2}Car ils ont pris de leurs filles pour eux et pour leurs fils, et ont mêlé la semence sainte avec les peuples de ces pays ; et des chefs et des magistrats ont été les premiers à commettre ce péché\FTNT{Né. 13:3.}.
\VS{3}Lorsque j'entendis cela, je déchirai mes vêtements et mon manteau, j'arrachai les cheveux de ma tête et ma barbe, et je m'assis tout épouvanté.
\VS{4}Et tous ceux qui tremblaient aux paroles du Dieu d'Israël, s'assemblèrent auprès de moi, à cause de l’infidélité de ceux de la captivité ; et je demeurai assis tout épouvanté jusqu'à l'offrande du soir.
\TextTitle{Prière et confession d'Esdras}
\VS{5}Et au temps de l'offrande du soir, je me levai du sein de mon affliction, et ayant mes vêtements et mon manteau déchirés, je me mis à genoux, et j'étendis mes mains vers Yahweh, mon Dieu,
\VS{6}et je dis : Mon Dieu ! J’ai honte, et je suis trop confus, ô mon Dieu, pour lever ma face vers toi ; car nos iniquités se sont multipliées au-dessus de nos têtes, et notre péché s'est élevé jusqu’aux cieux.
\VS{7}Depuis les jours de nos pères jusqu'à ce jour, nous sommes grandement coupables, et c’est à cause de nos iniquités que nous avons été livrés, nous, nos rois et nos sacrificateurs entre les mains des rois des pays, à l'épée, à la captivité, au pillage, et à la honte, comme il paraît aujourd'hui.
\VS{8}Et cependant Yahweh, notre Dieu, nous a maintenant fait grâce, en épargnant un reste, et il nous a donné un clou dans son saint lieu, afin d'éclaircir nos yeux et nous donner un peu de répit dans notre servitude\FTNT{Es. 22:23.}.
\VS{9}Car nous sommes esclaves, mais notre Dieu ne nous a point abandonnés dans notre servitude. Il a incliné la bienveillance des rois de Perse pour nous accorder de préserver nos vies afin que nous puissions relever la maison de notre Dieu, et rétablir ces lieux en ruines, et pour nous donner une clôture en Juda et à Jérusalem.
\VS{10}Mais maintenant, ô notre Dieu ! Que dirons-nous après ces choses ? Car nous avons abandonné tes commandements,
\VS{11}que tu as ordonnés par tes serviteurs les prophètes, en disant : Le pays dans lequel vous entrez pour le posséder est un pays souillé par les impuretés des peuples de ces pays, à cause des abominations dont ils l'ont rempli d’un bout à l'autre par leurs impuretés\FTNT{Lé. 18:25-27.};
\VS{12}maintenant donc, ne donnez point vos filles à leurs fils, et ne prenez point leurs filles pour vos fils, ne cherchez jamais ni leur bonheur, ni leur paix, ainsi vous deviendrez forts,  vous mangerez les meilleurs productions du pays, et vous le laisserez hériter à vos fils pour toujours\FTNT{De. 7:3.}.
\VS{13}Après toutes les choses qui nous sont arrivées à cause de nos mauvaises actions et des grandes offenses que nous avons commises - quoi que tu ne nous aies pas, ô notre Dieu, punis en proportion de nos péchés et maintenant que  tu nous as conservé ces réchappés ; 
\VS{14}retournerions-nous à violer tes commandements, et à faire alliance avec ces peuples abominables ? Ne serais-tu pas en colère contre nous, jusqu'à nous exterminer, sans aucun reste ni aucun réchappé ?
\VS{15}Yahweh, Dieu d'Israël ! Tu es juste, car nous sommes aujourd'hui un reste de réchappés. Voici, nous sommes devant toi avec nos fautes, ne pouvant subsister à cause d’elles devant ta face.
\Chap{10}
\TextTitle{Confession et séparation}
\VerseOne{}Pendant qu’Esdras priait et faisait cette confession, pleurant et étant prosterné à terre devant la maison de Dieu, une grande multitude d'hommes, de femmes, et d’enfants d'Israël, s'assembla auprès de lui ; et le peuple se lamenta abondamment par des pleurs.
\VS{2}Alors Schecania, fils de Jehiel, d'entre les fils d’Elam, prit la parole, et dit à Esdras : Nous avons péché contre notre Dieu, en nous mariant avec des femmes étrangères d'entre les peuples de ce pays. Mais Israël ne reste pas pour cela sans espérance\FTNT{De. 7:22,23.}.
\VS{3}Faisons maintenant une alliance avec notre Dieu pour le renvoi de toutes ces femmes et de leurs enfants, selon le conseil de mon seigneur et de ceux qui tremblent devant les commandements de notre Dieu. Et qu'il en soit fait selon la loi\FTNT{Esd. 9:4 ; Mal. 3:16.}.
\VS{4}Lève-toi, car cette affaire te regarde. Nous serons avec toi. Prends donc courage et agis.
\VS{5}Esdras se leva, et il fit jurer aux chefs des sacrificateurs, des Lévites, et de tout Israël, de faire selon cette parole. Et ils le jurèrent.
\VS{6}Puis Esdras se retira de devant la maison de Dieu, et s'en alla dans la chambre de Jochanan, fils d'Eliaschib ; et quand il y fut entré, il ne mangea point de pain, ne but point d'eau, parce qu'il se lamentait à cause du péché de ceux de la captivité.
\VS{7}Alors on publia dans le pays de Juda et à Jérusalem que tous ceux qui étaient retournés de la captivité aient à s'assembler à Jérusalem,
\VS{8}et que quiconque ne s'y rendrait pas dans trois jours, selon l'avis des chefs et des anciens, aurait tous ses biens complètement détruits, et que lui-même serait séparé de l'assemblée de ceux de la captivité.
\VS{9}Ainsi tous ceux de Juda et de Benjamin s'assemblèrent à Jérusalem dans les trois jours. C’était le vingtième jour du neuvième mois. Tout le peuple se tenait sur la place de la maison de Dieu, tremblant au sujet de cette affaire et à cause des pluies\FTNT{1 S. 12:18.}.
\VS{10}Esdras, le sacrificateur, se leva et leur dit : Vous avez péché en vous mariant avec des femmes étrangères, de sorte que vous avez augmenté la culpabilité d'Israël\FTNT{De. 7:3.}.
\VS{11}Prononcez maintenant votre confession à Yahweh, le Dieu de vos pères, et faites sa volonté ! Séparez-vous des peuples du pays et des femmes étrangères.
\VS{12}Et toute l'assemblée répondit à haute voix : A nous de faire ce que tu as dit !
\VS{13}Mais le peuple est nombreux, le temps est pluvieux, et il n'y a pas moyen de se tenir dehors ; d’ailleurs, ce n’est pas l’affaire d’un jour ou de deux, car il y en a beaucoup parmi nous qui ont péché dans cette affaire.
\VS{14}Que tous nos chefs se présentent donc devant toute l'assemblée, et que tous ceux qui sont dans nos villes, et qui se sont mariés avec des femmes étrangères, viennent à un temps fixé, et que les anciens de chaque ville et ses juges soient avec eux, jusqu'à ce que nous détournions de nous l'ardente colère de notre Dieu à ce sujet.
\VS{15}Il n'y eut que Jonathan, fils d'Asaël, et Jachzia, fils de Thikva, qui s'opposèrent a cet avis ; et Meschullam et Schabthaï, Lévites, les appuyèrent ;
\VS{16}mais ceux qui étaient retournés de la captivité s’y conformèrent. On choisit Esdras, le sacrificateur, et des chefs de famille selon leurs maisons paternelles, tous désignés par leurs noms ; ils siégèrent le premier jour du dixième mois, pour suivre cette affaire.
\VS{17}Le premier jour du premier mois, ils en finirent avec tous les hommes qui s’étaient mariés à des femmes étrangères.
\VS{18}Parmi les fils des sacrificateurs qui s’étaient mariés à des femmes étrangères, il se trouva d'entre les fils de Josué, fils de Jotsadak et de ses frères, Maaséja, Eliézer, Jarib et Guedalia,
\VS{19}qui, en donnant leurs mains, renvoyèrent leurs femmes ; et offrirent un bélier comme sacrifice de culpabilité ;
\VS{20}des fils d'Immer, Hanani et Zebadia ;
\VS{21}des fils de Harim, Maaséja, Elie, Schemaeja, Jehiel et Ozias ;
\VS{22}des fils de Paschhur, Eljoénaï, Maaséja, Ismaël, Nethaneel, Jozabad et Eleasa.
\VS{23}Parmi les Lévites : Jozabad, Schimeï, Kélaja (ou Kelitha) Pethachja, Juda et Eliézer.
\VS{24}Parmi les chantres : Eliaschib. Et des portiers : Schallum, Thélem et Uri.
\VS{25}Parmi ceux d'Israël : Des fils de Pareosch, Ramia, Jizzija, Malkija, Mijamin, Eléazar, Malkija et Benaja ;
\VS{26}des fils d’Elam, Matthania, Zacharie, Jehiel, Abdi, Jérémoth et Elie ;
\VS{27}des fils de Zatthu, Eljoénaï, Eliaschib, Matthania, Jérémoth, Zabad et Aziza ;
\VS{28}des fils de Bébaï, Jochanan, Hanania, Zabbaï et Athlaï ;
\VS{29}des fils de Bani, Meschullam, Malluc, Adaja, Jaschub, Scheal et Ramoth ;
\VS{30}des fils de Pachath-Moab, Adna, Kelal, Benaja, Maaséja, Matthania, Betsaleel, Binnuï et Manassé ;
\VS{31}des fils de Harim, Eliézer, Jischija, Malkija, Schemaeja, Siméon,
\VS{32}Benjamin, Malluc et Schemaria ;
\VS{33}des fils de Haschum, Matthnaï, Matthattha, Zabad, Eliphéleth, Jerémaï, Manassé et Schimeï ;
\VS{34}des fils de Bani, Maadaï, Amram, Uel,
\VS{35}Benaja, Bédia, Keluhu,
\VS{36}Vania, Merémoth, Eliaschib,
\VS{37}Matthania, Matthnaï, Jaasaï,
\VS{38}Bani, Binnuï, Schimeï,
\VS{39}Schélémia, Nathan, Adaja,
\VS{40}Macnadbaï, Schaschaï, Scharaï,
\VS{41}Azareel, Schélémia, Schemaria,
\VS{42}Schallum, Amaria et Joseph ;
\VS{43}des fils de Nebo, Jeïel, Matthithia, Zabad, Zebina, Jaddaï, Joël et Benaja.
\VS{44}Tous ceux-là avaient pris des femmes étrangères ; et  quelques-uns avaient eu des fils avec ces femmes-là.
\PPE{}
\end{multicols}

%\clearpage\ShortTitle{Néhémie}\BookTitle{Néhémie}\BFont
\noindent\hrulefill
{\footnotesize
\textit{
\bigskip
{\centering{}
\\Auteur : Néhémie
\\(Heb. : Nechemyah)
\\Signification : Yahweh a consolé
\\Thème : Reconstruction des murailles de Jérusalem
\\Date de rédaction : 5\up{ème} siècle av. J.-C.\\}
}
%\bigskip
\textit{
\\En apprenant l'état de ruine dans lequel se trouvait Jérusalem, Néhémie, échanson du roi perse Artaxerxés Ier, fut profondément affecté. Après plusieurs jours dans la désolation et l'humiliation, le Seigneur toucha le cœur du roi qui lui donna l'autorisation et le matériel nécessaire pour rebâtir la muraille de Jérusalem. Malgré les nombreuses oppositions dont il fit l'objet au cours de son entreprise, Néhémie acheva l'œuvre qui lui avait été confiée. Dans le même temps, il mit en place de profondes réformes dans le cadre du retour à la loi de Yahweh.
%\bigskip
\\Complément du livre d'Esdras avec lequel il ne formait initialement qu'un ouvrage, le livre de Néhémie présente un homme de prière, un serviteur œuvrant pour, avec, et au Nom de Yahweh.\bigskip
}
}
\par\nobreak\noindent\hrulefill
\begin{multicols}{2}
\Chap{1}
\TextTitle{La détresse du peuple resté à Jérusalem est racontée à Néhémie}
\VerseOne{}Paroles de Néhémie, fils de Hacalia. Il arriva au mois de Kisleu, la vingtième année, comme j'étais à Suse, la capitale,
\VS{2}Hanani, l'un de mes frères et quelques hommes arrivèrent de Juda. Je les questionnai au sujet des Juifs réchappés qui étaient restés de la captivité et au sujet de Jérusalem.
\VS{3}Et ils me dirent : Ceux qui sont restés de la captivité sont là dans la province, dans une grande misère et dans l'opprobre ; et la muraille de Jérusalem demeure renversée et ses portes ont été consumées par le feu.
\TextTitle{Néhémie prie Yahweh et implore sa grâce}
\VS{4}Or il arriva que, dès que j'entendis ces paroles, je m'assis, je pleurai et je fus dans le deuil plusieurs jours. Je jeûnai et je priai devant le Dieu des cieux,
\VS{5} et je dis : Je te prie, ô Yahweh ! Dieu des cieux, Dieu grand et redoutable, qui garde l'alliance et la miséricorde de ceux qui t'aiment et qui observent tes commandements !
\VS{6}Je te prie que ton oreille soit attentive et que tes yeux soient ouverts pour entendre la prière que ton serviteur te présente en ce temps-ci, jour et nuit, pour tes serviteurs les enfants d'Israël, en confessant les péchés des enfants d'Israël, que nous avons commis contre toi ; même moi et la maison de mon père, nous avons péché.
\VS{7}Certainement nous sommes coupables devant toi, nous n'avons pas gardé les commandements, les lois et les ordonnances que tu prescrivis à Moïse, ton serviteur.
\VS{8}Mais, je te prie, souviens-toi de la parole que tu chargeas Moïse, ton serviteur, de dire : Vous pécherez et je vous disperserai parmi les peuples\FTNT{De. 28:63-67.} ;
\VS{9}mais si vous revenez à moi, et si vous gardez mes commandements et les observez ; et s'il y en a d'entre vous qui ont été chassés jusqu'à l'extrémité du ciel, je vous rassemblerai de là, et je vous ramènerai au lieu que j'aurai choisi pour y faire habiter mon Nom\FTNT{De. 30:1-10.}.
\VS{10}Ils sont tes serviteurs et ton peuple, que tu as rachetés par ta grande puissance et par ta main forte.
\VS{11}Je te prie donc, Seigneur, que ton oreille soit maintenant attentive à la prière de ton serviteur, et à la prière de tes serviteurs qui prennent plaisir à craindre ton Nom ! Je te prie, donne aujourd'hui du succès à ton serviteur, et fais-lui trouver grâce devant cet homme ! J'étais alors échanson du roi.
\Chap{2}
\TextTitle{Yahweh exauce Néhémie et lui donne la faveur du roi}
\VerseOne{}Et il arriva, au mois de Nisan, la vingtième année du roi Artaxerxès, comme le vin était devant lui, je pris le vin et le présentai au roi. Je n'avais jamais été triste devant lui\FTNT{Pr. 15:13.}.
\VS{2}Et le roi me dit : Pourquoi as-tu mauvais visage, puisque tu n'es point malade ? Cela ne peut être qu'une tristesse de cœur. Je fus alors saisi d'une grande crainte,
\VS{3}et je répondis au roi : Que le roi vive éternellement ! Comment n'aurais-je pas mauvais visage, puisque la ville où sont les sépulcres de mes pères demeure désolée et que ses portes ont été consumées par le feu ?
\VS{4}Et le roi dit : Que me demandes-tu ? Alors je priai le Dieu des cieux,
\VS{5}et je dis au roi : Si le roi le trouve bon, et si ton serviteur lui est agréable, envoie-moi en Juda, vers la ville des sépulcres de mes pères, pour la rebâtir\FTNT{La reconstruction de la ville de Jérusalem sous Néhémie date, selon certains, de l'an 445 av. J.-C., suite au décret d'Artaxerxès. Cette date marquerait le point de départ des soixante-dix semaines d'années annoncées par Daniel (Da. 9:24-27).}.
\VS{6}Le roi me dit, et sa femme aussi qui était assise auprès de lui : Combien ton voyage durera-t-il, et quand seras-tu de retour ? Je lui précisai le temps, et le roi trouva bon de m'envoyer.
\VS{7}Puis je dis au roi : Si le roi le trouve bon, qu'on me donne des lettres pour les gouverneurs de l'autre côté du fleuve, afin qu'ils me laissent passer, jusqu'à ce que j'arrive en Juda ;
\VS{8}et des lettres pour Asaph, le garde de la forêt du roi, afin qu'il me donne du bois pour la charpente des portes de la forteresse près de la maison, pour les murailles de la ville, et pour la maison dans laquelle j'entrerai. Et le roi me l'accorda, car la main de mon Dieu était bonne sur moi.
\TextTitle{Arrivée à Jérusalem, constat des murailles en ruines}
\VS{9}J'allai donc vers les gouverneurs qui sont de l'autre côté du fleuve et je leur donnai les lettres du roi. Le roi avait aussi envoyé avec moi des chefs de l'armée et des cavaliers.
\VS{10}Quand Sanballat, le Horonite, et Tobija, le serviteur Ammonite, l'ayant appris, ils eurent un très grand déplaisir de ce qu'il venait un homme pour procurer du bien aux enfants d'Israël.
\VS{11}Ainsi j'arrivai à Jérusalem et j'y passai trois jours.
\VS{12}Puis je me levai de nuit, avec quelques hommes ; mais je ne dis à personne ce que Dieu avait mis dans mon cœur de faire pour Jérusalem. Il n'y avait point d'autre bête avec moi que celle sur laquelle j'étais monté.
\VS{13}Je sortis donc de nuit par la porte de la vallée et me dirigeai vers la source du dragon, vers la porte du fumier ; et je considérai les murailles de Jérusalem qui étaient en ruines\FTNT{Jé. 39:8.}, et ses portes consumées par le feu.
\VS{14}Je passai près de la porte de la source et vers l'étang du roi ; et il n'y avait point de place par où je puisse passer avec ma monture.
\VS{15}Je montai de nuit par le torrent et je considérai la muraille. Puis en revenant, je rentrai par la porte de la vallée ; et ainsi je fus de retour.
\VS{16}Or les magistrats ne savaient pas où j'étais allé, ni ce que je faisais ; car je n'avais rien dit jusqu'à ce moment, ni aux Juifs, ni aux sacrificateurs, ni aux chefs, ni aux magistrats, ni au reste de ceux qui s'occupaient des affaires.
\TextTitle{Néhémie partage sa vision de rebâtir la muraille}
\VS{17}Alors je leur dis : Vous voyez la misère dans laquelle nous sommes ! Comment Jérusalem demeure désolée et ses portes brûlées par le feu ! Venez et rebâtissons les murailles de Jérusalem et nous ne serons plus dans l'opprobre.
\VS{18}Et je leur déclarai comment la main de mon Dieu avait été bonne sur moi, et quelles paroles le roi m'avait dites. Alors ils dirent : Levons-nous et bâtissons ! Ils fortifièrent leurs mains pour bien faire.
\TextTitle{Premières oppositions}
\VS{19}Mais Sanballat, le Horonite, Tobija, le serviteur Ammonite, et Guéschem, l'Arabe, l'ayant appris, se moquèrent de nous et nous méprisèrent. Ils dirent : Qu'est-ce que vous faites ? Ne vous rebellez-vous pas contre le roi ?
\VS{20}Et je leur répondis cette parole : Le Dieu des cieux lui-même nous donnera le succès ! Nous donc, qui sommes ses serviteurs, nous nous lèverons et nous bâtirons ; mais vous, vous n'avez aucune part, ni droit, ni souvenir, à Jérusalem.
\Chap{3}
\TextTitle{Les participants à la reconstruction de la muraille}
\VerseOne{}Eliaschib, le souverain sacrificateur, se leva donc avec ses frères, les sacrificateurs et ils rebâtirent la porte des brebis\FTNT{La première porte qui fut reconstruite fut la porte des brebis. Cette porte est très proche du temple, c'est par elle que l'on faisait entrer les brebis destinées aux sacrifices dans la cour du temple. Cette porte est la préfiguration de Jésus-Christ qui s'est lui-même présenté comme étant la « porte des brebis » (Jn. 10:7).}. Ils la sanctifièrent, ils y posèrent ses battants. Ils la sanctifièrent depuis la tour de Méa jusqu'à la tour de Hananeel.
\VS{2}Et les gens de Jéricho rebâtirent à son côté ; et à côté d'eux Zaccur, fils d'Imri, rebâtit aussi.
\VS{3}Les fils de Senaa rebâtirent la porte des poissons. Ils en firent la charpente et y mirent ses portes, ses serrures et ses barres.
\VS{4}Et à leur côté travailla aux réparations Merémoth, fils d'Urie, fils d'Hakkots ; et à leur côté travailla Meschullam, fils de Bérékia, fils de Meschézabeel, et à leur côté travailla Tsadok, fils de Baana.
\VS{5}A leur côté travaillèrent les Tekoïtes ; mais les chefs d'entre eux ne vinrent point au service de leur Seigneur.
\VS{6}Et Jojada, fils de Paséach, et Meschullam, fils de Besodia, réparèrent la vieille porte. Ils en firent la charpente, y mirent ses battants, ses serrures et ses barres.
\VS{7}A leur côté travaillèrent Melatia, le Gabaonite, Jadon, le Méronothite, et les hommes de Gabaon et de Mitspa, vers le siège du gouverneur de ce côté du fleuve.
\VS{8}A côté d'eux travailla Uzziel, fils de Harhaja, d'entre les orfèvres, et à côté de lui travailla Hanania, d'entre les parfumeurs. Et ainsi ils relevèrent Jérusalem jusqu'à la muraille large.
\VS{9}Et à leur côté travailla Rephaja, fils de Hur, chef d'un demi-quartier de Jérusalem.
\VS{10}Puis à leur côté travailla Jedaja, fils de Harumaph, devant sa maison ; et à son côté travailla Hattusch, fils de Haschabnia.
\VS{11}Et Malkija, fils de Harim, et Haschub, fils de Pachath-Moab, en réparèrent une seconde section, et la tour des fours.
\VS{12}Et à leur côté travailla, avec ses filles, Schallum, fils de d'Hallochesch, chef de la moitié du quartier de Jérusalem.
\VS{13}Hanun et les habitants de Zanoach réparèrent la porte de la vallée. Ils la rebâtirent et mirent ses battants, ses serrures, et ses barres, et ils bâtirent mille coudées de muraille, jusqu'à la porte du fumier.
\VS{14}Et Malkija, fils de Récab, chef du quartier de Beth-Hakkérem, répara la porte du fumier. Il la rebâtit et mit ses battants, ses serrures et ses barres.
\VS{15}Schallum, fils de Col-Hozé, chef du quartier de Mitspa, répara la porte de la source. Il la rebâtit et la couvrit, et mit ses portes, ses serrures, et ses barres. Il répara aussi la muraille de l'étang de Siloé, vers le jardin du roi, et jusqu'aux marches qui descendent de la cité de David.
\VS{16}Après lui travailla Néhémie, fils d'Azbuk, chef de la moitié du quartier de Beth-Tsur, jusqu'à l'endroit des sépulcres de David, et jusqu'à l'étang qui avait été refait, et jusqu'à la maison des hommes vaillants.
\VS{17}Après lui travaillèrent les Lévites, Rehum, fils de Bani ; et à son côté travailla Haschabia, chef de la moitié du quartier de Keïla, pour ceux de son quartier.
\VS{18}Après lui travaillèrent leurs frères, Bavvaï, fils de Hénadad, chef de la moitié du quartier de Keïla.
\VS{19}A son côté, Ezer, fils de Josué, chef de Mitspa, en répara autant, à l'endroit où l'on monte à l'arsenal, à l'angle.
\VS{20}Après lui Baruc, fils de Zabbaï, répara avec ardeur une seconde section, depuis l'angle jusqu'à la porte de la maison d'Eliaschib, le souverain sacrificateur.
\VS{21}Après lui Merémoth, fils d'Urie, fils d'Hakkots, répara une seconde section, depuis l'entrée de la maison d'Eliaschib, jusqu'à l'extrémité de la maison d'Eliaschib.
\VS{22}Et après lui travaillèrent les sacrificateurs, habitants des environs.
\VS{23}Après eux, Benjamin et Haschub travaillèrent devant leur maison. Après eux, Azaria, fils de Maaséja, fils d'Anania, travailla auprès de sa maison.
\VS{24}Après lui, Binnuï, fils de Hénadad, répara une seconde section, depuis la maison d'Azaria jusqu'à l'angle et jusqu'au coin.
\VS{25}Palal, fils d'Uzaï, travailla vis-à-vis de l'angle, et de la tour qui sort de la tour supérieure du roi, qui est auprès de la cour de la prison. Après lui travailla Pedaja, fils de Pareosch.
\VS{26}Les Néthiniens, qui demeuraient sur la colline, réparèrent vers l'orient, jusqu'à l'endroit de la porte des eaux, et vers la tour qui sort.
\VS{27}Après eux, les Tekoïtes réparèrent une seconde section, depuis l'endroit de la grande tour qui sort en dehors, jusqu'à la muraille de la colline.
\VS{28}Au-dessus de la porte des chevaux, les sacrificateurs travaillèrent, chacun devant de sa maison.
\VS{29}Après eux, Tsadok, fils d'Immer, travailla devant sa maison. Après lui répara Schemaeja, fils de Schecania, gardien de la porte orientale.
\VS{30}Après lui, Hanania, fils de Schélémia et Hanun le sixième fils de Tsalaph, en réparèrent une seconde section. Après eux, Meschullam, fils de Bérékia, travailla vis-à-vis de sa chambre.
\VS{31}Après lui, Malkija, fils de l'orfèvre, répara jusqu'à la maison des Néthiniens et des marchands, vis-à-vis de la porte de Miphkad, et jusqu'à la chambre haute du coin.
\VS{32}Et les orfèvres et les marchands travaillèrent entre la chambre haute du coin et la porte des brebis.
\Chap{4}
\TextTitle{La prière, solution pour faire face aux attaques et moqueries}
\VerseOne{}Or il arriva que Sanballat apprit que nous rebâtissions la muraille, il devint furieux et très fâché. Il se moqua des Juifs.
\VS{2}Et il dit en présence de ses frères, et des gens de guerre de Samarie : Que font ces faibles Juifs ? Les laissera-t-on faire ? Sacrifieront-ils ? Et achèveront-ils tout en un jour ? Pourront-ils faire revenir à la vie les pierres des monceaux de poussière, puisqu'elles sont brûlées ?
\VS{3}Et Tobija, l'Ammonite, qui était auprès de lui, dit : Qu'ils bâtissent encore ! Si un renard monte, il rompra leur muraille de pierre !
\VS{4}Ô notre Dieu, écoute comment nous sommes méprisés ! Fais retourner leurs insultes sur leur tête, et donne-les en pillage dans un pays de captivité.
\VS{5}Ne couvre point leur iniquité, et que leur péché ne soit point effacé de devant ta face ; car ils ont irrité les bâtisseurs.
\VS{6}Nous rebâtîmes donc la muraille, et tout le mur fut achevé jusqu'à sa moitié ; et le peuple avait le cœur au travail.
\VS{7}Mais quand Sanballat et Tobija, les Arabes, les Ammonites et les Asdodiens eurent appris que la muraille de Jérusalem avait été refaite, et qu'on avait commencé à fermer les brèches, ils s'enflammèrent de colère.
\VS{8}Et ils se liguèrent tous ensemble pour venir faire la guerre contre Jérusalem, et pour les faire échouer.
\VS{9}Alors nous priâmes notre Dieu, et ayant peur d'eux, nous établîmes une garde jour et nuit pour nous défendre contre leurs attaques.
\TextTitle{Persévérance du peuple prêt à se battre à tout moment}
\VS{10}Et Juda disait : La force des ouvriers est affaiblie, et il y a beaucoup de débris, en sorte que nous ne pourrons pas bâtir la muraille.
\VS{11}Et nos ennemis disaient : Qu'ils n'en sachent rien et qu'ils ne voient rien, jusqu'à ce que nous entrions au milieu d'eux ; nous les tuerons et ferons ainsi cesser l'ouvrage.
\VS{12}Mais il arriva que les Juifs, qui habitaient près d'eux, vinrent dix fois nous avertir, de tous les lieux d'où ils se rendaient vers nous.
\VS{13}C'est pourquoi je plaçai le peuple depuis le bas, derrière la muraille, et sur des lieux élevés, secs et lumineux, selon leurs familles, avec leurs épées, leurs lances et leurs arcs.
\VS{14}Puis je regardai et m'étant levé, je dis aux chefs, aux magistrats et au reste du peuple : N'ayez point peur d'eux ! Souvenez-vous du Seigneur, qui est grand et terrible, et combattez pour vos frères, pour vos fils et pour vos filles, pour vos femmes et pour vos maisons !
\VS{15}Et quand nos ennemis entendirent que nous étions avertis, Dieu fit échouer leur projet, et nous retournâmes tous aux murailles, chacun à son travail.
\VS{16}Depuis ce jour-là, la moitié de mes serviteurs travaillait, et l'autre moitié avait des lances, des boucliers, des arcs et des cuirasses. Les gouverneurs suivaient chaque maison de Juda.
\VS{17}Ceux qui bâtissaient la muraille, et ceux qui portaient ou chargeaient les fardeaux, travaillaient chacun d'une main, et de l'autre ils tenaient une arme.
\VS{18}Car chacun de ceux qui bâtissaient avait son épée ceinte autour des reins. Et celui qui sonnait du shofar se tenait près de moi.
\VS{19}Et je dis aux chefs, aux magistrats et au reste du peuple : L'ouvrage est grand et étendu, et nous sommes séparés sur la muraille, éloignés les uns des autres.
\VS{20}En quelque lieu donc d'où vous entendrez le son du shofar, courez-y vers nous ; notre Dieu combattra pour nous\FTNT{Ex. 14:14 ; De. 1:30 ; 2 Ch. 20:29.}.
\VS{21}C'est donc ainsi que nous accomplissions le travail ; la moitié tenait des lances, depuis le lever du jour jusqu'à l'apparition des étoiles.
\VS{22}En ce temps-là, je dis aussi au peuple : Que chacun passe la nuit dans Jérusalem avec son serviteur, afin de faire la garde la nuit et de travailler le jour.
\VS{23}Et nous ne quittions point nos vêtements, ni moi, ni mes frères, ni mes serviteurs, ni les hommes de garde qui me suivaient; chacun n'avait que ses armes et de l'eau.
\Chap{5}
\TextTitle{Cupidité des chefs dévoilée ; rétablissement de la justice}
\VerseOne{}Or il y eut un grand cri du peuple et de leurs femmes, contre les Juifs, leurs frères.
\VS{2}Les uns disaient : Nous, nos fils et nos filles, nous sommes nombreux; qu'on nous donne du blé, afin que nous mangions et que nous vivions.
\VS{3}Et d'autres disaient : Nous engageons nos champs, nos vignes et nos maisons, pour avoir du blé pendant la famine.
\VS{4}D'autres disaient : Nous avons emprunté de l'argent sur nos champs et sur nos vignes pour le tribut du roi.
\VS{5}Toutefois notre chair est comme la chair de nos frères, et nos fils sont comme leurs fils ; et voici, nous soumettons à la servitude nos fils et nos filles ; et quelques-unes de nos filles sont déjà esclaves et ne sont plus en notre pouvoir ; et nos champs et nos vignes sont à d'autres.
\VS{6}Je fus très en colère quand j'entendis leur cri et ces paroles-là.
\VS{7}Je résolus dans mon cœur de réprimander les chefs et les magistrats, et je leur dis : Vous prêtez avec intérêt à vos frères\FTNT{Ex.22:25 ; Lé. 25:36.} ! Et je fis convoquer autour d'eux une grande foule.
\VS{8}Et je leur dis : Nous avons racheté selon notre pouvoir nos frères Juifs vendus aux nations, et vous vendriez vous-mêmes vos frères, ou c'est à nous qu'ils seraient vendus ? Ils se turent, ne trouvant rien à dire.
\VS{9}Et je dis : Ce que vous faites n'est pas bien. Ne voulez-vous pas marcher dans la crainte de notre Dieu, plutôt que d'être insultés par les nations qui sont nos ennemies ?
\VS{10}Moi aussi, mes frères et mes serviteurs, nous leur avons prêté de l'argent et du blé. Abandonnons je vous prie, cette dette !
\VS{11}Rendez-leur, je vous prie, aujourd'hui leurs champs, leurs vignes, leurs oliviers et leurs maisons ; et outre cela, le centième de l'argent, du blé, du vin, et de l'huile que vous exigez d'eux.
\VS{12}Et ils répondirent : Nous les rendrons et nous ne leur demanderons rien ; nous ferons ce que tu dis. Alors j'appelai les sacrificateurs et je les fis jurer de tenir parole.
\VS{13}Et je secouai mon bras et je dis : Que Dieu secoue ainsi de sa maison et de son travail tout homme qui n'aura pas tenu parole, et qu'il soit ainsi secoué et vidé ! Et toute l'assemblée répondit : Amen ! Et ils louèrent Yahweh. Et le peuple fit selon cette parole.
\TextTitle{Néhémie, modèle de dévouement}
\VS{14}Et même, depuis le jour où le roi m'établit comme gouverneur au pays de Juda, depuis la vingtième année jusqu'à la trente-deuxième année du roi Artaxerxès, pendant douze ans, moi et mes frères, nous n'avons pas pris ce qui était assigné au gouverneur comme revenu.
\VS{15}Quoique, les premiers gouverneurs qui avaient été avant moi, chargeaient le peuple, et prenaient de lui du pain et du vin, outre quarante sicles d'argent, et leurs serviteurs tyrannisaient le peuple. Mais je n'ai point fait ainsi, à cause de la crainte de mon Dieu.
\VS{16}Et même, j'ai travaillé à la réparation d'une partie de cette muraille, et nous n'avons acheté aucun champ, et tous mes serviteurs étaient tous ensemble à l'ouvrage.
\VS{17}Et outre cela, j'avais aussi à ma table les Juifs et les magistrats, au nombre de cent cinquante hommes, et ceux qui venaient vers nous des nations d'alentour.
\VS{18}On m'apprêtait chaque jour un bœuf, six moutons choisis et aussi des volailles ; et tous les dix jours on me présentait toutes sortes de vins en abondance. Malgré cela, je n'ai point demandé le revenu qui était assigné au gouverneur ; parce que les travaux étaient à la charge de ce peuple.
\VS{19}Ô mon Dieu ! Souviens-toi de moi en bien, à cause de tout ce que j'ai fait pour ce peuple.
\Chap{6}
\TextTitle{Complot et mensonge contre Néhémie ; fermeté et confiance en Dieu}
\VerseOne{}Or il arriva que quand Sanballat, Tobija, et Guéschem l'Arabe, et le reste de nos ennemis apprirent que j'avais rebâti la muraille, et qu'il n'y restait aucune brèche. (bien que jusqu'à ce temps-là, je n'avais pas encore mis les battants aux portes.)
\VS{2}Alors Sanballat et Guéschem envoyèrent vers moi, pour dire : Viens, et ayons ensemble une rencontre dans les villages qui sont dans la vallée d'Ono. Or ils avaient comploté de me faire du mal.
\VS{3}Mais j'envoyai des messagers vers eux pour leur dire : J'ai un grand ouvrage à faire, et je ne puis descendre. Le travail serait interrompu pendant que je le quitterais pour aller vers vous.
\VS{4}Ils m'adressèrent la même chose quatre fois ; et je leur répondis la même réponse.
\VS{5}Alors Sanballat m'envoya son serviteur pour me tenir le même discours une cinquième fois ; et il avait dans sa main une lettre ouverte.
\VS{6}Il y était écrit : On entend dire parmi les nations, et Gaschmu le dit, que vous pensez, toi et les Juifs, à vous révolter, et que c'est pour cela que tu rebâtis la muraille. Et tu vas, dit-on, devenir leur roi ;
\VS{7}Même que tu as ordonné des prophètes pour te louer dans Jérusalem, et pour dire : Il est roi de Juda. Et maintenant, on fera entendre au roi ces mêmes choses. Viens donc afin que nous consultions ensemble.
\VS{8}Et je renvoyai vers lui pour lui dire : Ce que tu dis là n'est point, mais c'est toi qui l'inventes dans ton propre coeur !
\VS{9}Car tous ces gens voulaient nous effrayer, en disant : Leurs mains relâcheront le travail, de sorte qu'il ne se fera point. Maintenant donc, ô Dieu, fortifie-moi !
\VS{10}Je me rendis à la maison de Schemaeja, fils de Delaja, fils de Mehétabeel. Il s'était enfermé et il me dit : Assemblons-nous dans la maison de Dieu, au milieu du temple et fermons les portes du temple ; car ils doivent venir pour te tuer, et ils viendront pendant la nuit pour te tuer.
\VS{11}Mais je répondis : Un homme tel que moi s'enfuirait-il ? Et quel homme tel que moi pourrait entrer dans le temple pour sauver sa vie ? Je n'y entrerai point.
\VS{12}Et voilà, je reconnus bien que Dieu ne l'avait point envoyé, mais qu'il avait prononcé cette prophétie contre moi parce que Sanballat et Tobija lui avaient donné de l'argent.
\VS{13}Car il était leur pensionnaire pour m'épouvanter, et pour m'obliger à agir de la sorte, et à commettre cette faute, afin qu'ils aient quelque mauvaise chose à me reprocher.
\VS{14}Ô mon Dieu ! Souviens-toi de Tobija et de Sanballat, et de leurs actions et aussi de Noadia, la prophétesse, et du reste des prophètes qui cherchaient à m'effrayer !
\TextTitle{Achèvement de la muraille}
\VS{15}Néanmoins, la muraille fut achevée le vingt-cinquième jour du mois d'Elul, en cinquante-deux jours.
\VS{16}Quand donc tous nos ennemis l'apprirent et qu'ils la virent, toutes les nations qui étaient autour de nous furent dans la crainte ; elles éprouvèrent une grande humiliation, et ils reconnurent que cet ouvrage s'était accompli par le secours de notre Dieu.
\VS{17}Mais aussi en ce temps-là, les chefs de Juda adressaient fréquemment des lettres à Tobija, et celles de Tobija venaient à eux.
\VS{18}Car il y en avait plusieurs en Juda qui s'étaient liés à lui par serment, parce qu'il était gendre de Schecania, fils d'Arach, et que son fils Jochanan avait pris la fille de Meschullam, fils de Bérékia.
\VS{19}Ils racontaient même du bien de lui en ma présence, et lui rapportaient mes paroles. Et Tobija envoyait des lettres pour m'effrayer.
\Chap{7}
\TextTitle{Instructions spécifiques à Hanani et Hanania}
\VerseOne{}Or après que la muraille fut rebâtie, et que j'aie mis les portes, et qu'on ait fait la revue des portiers, des chantres et des Lévites ; 
\VS{2}je donnai cet ordre à Hanani, mon frère, et à Hanania, chef de la forteresse de Jérusalem ; car il était tel qu'un homme fidèle doit être, et il craignait Dieu plus que plusieurs autres ;
\VS{3}et je leur dis : Que les portes de Jérusalem ne s'ouvrent point avant la chaleur du soleil ; et pendant que les gardes seront encore là, que l'on ferme les portes, et qu'on y mette les barres ; que l'on place comme gardes les habitants de Jérusalem, chacun à son poste, et chacun devant de sa maison.
\VS{4}Or la ville était spacieuse et grande, mais il y avait peu de gens, et ses maisons n'étaient point bâties\FTNT{De. 4:27.}.
\TextTitle{Liste des familles revenues de captivité avec Zorobabel}
\VS{5}Et mon Dieu me mit à coeur d'assembler les chefs, les magistrats et le peuple, pour en faire le dénombrement selon leurs généalogies. Je trouvai le registre du dénombrement selon les généalogies de ceux qui étaient montés la première fois. Et j'y trouvai ainsi écrit :
\VS{6}Ce sont ici ceux de la province qui remontèrent de la captivité, d'entre ceux que Nebucadnetsar, roi de Babylone, avait transportés en exil, et qui retournèrent à Jérusalem et en Juda, chacun dans sa ville.
\VS{7}Ils vinrent avec Zorobabel\FTNT{Esd. 5:2.}, Josué, Néhémie, Azaria, Raamia, Nachamani, Mardochée, Bilschan, Mispéreth, Bigvaï, Nehum, et Baana. Nombre des hommes du peuple d'Israël :
\VS{8}Les fils de Pareosch, deux mille cent soixante-douze.
\VS{9}Les fils de Schephathia, trois cent soixante-douze.
\VS{10}Les fils d'Arach, six cent cinquante-deux.
\VS{11}Les fils de Pachath-Moab, des fils de Josué et de Joab, deux mille huit cent dix-huit.
\VS{12}Les fils d'Elam, mille deux cent cinquante-quatre.
\VS{13}Les fils de Zatthu, huit cent quarante-cinq.
\VS{14}Les fils de Zaccaï, sept cent soixante.
\VS{15}Les fils de Binnuï, six cent quarante-huit.
\VS{16}Les fils de Bébaï, six cent vingt-huit.
\VS{17}Les fils d'Azgad, deux mille trois cent vingt-deux.
\VS{18}Les fils d'Adonikam, six cent soixante-sept.
\VS{19}Les fils de Bigvaï, deux mille soixante-sept.
\VS{20}Les fils d'Adin, six cent cinquante-cinq.
\VS{21}Les fils d'Ather, issu d'Ezéchias, quatre-vingt-dix-huit.
\VS{22}Les fils de Haschum, trois cent vingt-huit.
\VS{23}Les fils de Betsaï, trois cent vingt-quatre.
\VS{24}Les fils de Hariph, cent douze.
\VS{25}Les fils de Gabaon, quatre-vingt-quinze.
\VS{26}Les gens de Bethléhem et de Netopha, cent quatre-vingt-huit.
\VS{27}Les gens d'Anathoth, cent vingt-huit.
\VS{28}Les gens de Beth-Azmaveth, quarante-deux.
\VS{29}Les gens de Kirjath-Jearim, de Kephira et de Beéroth, sept cent quarante-trois.
\VS{30}Les gens de Rama et de Guéba, six cent vingt et un.
\VS{31}Les gens de Micmas, cent vingt-deux.
\VS{32}Les gens de Béthel et d'Aï, cent vingt-trois.
\VS{33}Les gens de l'autre Nebo, cinquante-deux.
\VS{34}Les fils d'un autre Elam, mille deux cent cinquante-quatre.
\VS{35}Les fils de Harim, trois cent vingt.
\VS{36}Les fils de Jéricho, trois cent quarante-cinq.
\VS{37}Les fils de Lod, de Hadid et d'Ono, sept cent vingt et un.
\VS{38}Les fils de Senaa, trois mille neuf cent trente.
\TextTitle{Liste des sacrificateurs revenus de captivité}
\VS{39}Sacrificateurs : Les fils de Jedaeja, de la maison de Josué, neuf cent soixante-treize.
\VS{40}Les fils d'Immer, mille cinquante-deux.
\VS{41}Les fils de Paschhur, mille deux cent quarante-sept.
\VS{42}Les fils de Harim, mille dix-sept.
\TextTitle{Liste des Lévites revenus de captivité}
\VS{43}Lévites : Les fils de Josué et de Kadmiel, d'entre les fils de Hodva, soixante quatorze.
\VS{44}Chantres : Les fils d'Asaph, cent quarante-huit.
\VS{45}Portiers : Les fils de Schallum, les fils d'Ather, les fils de Thalmon, les fils d'Akkub, les fils de Hathitha, les fils de Schobaï, cent trente-huit.
\TextTitle{Liste des Néthiniens revenus de captivité}
\VS{46}Néthiniens : Les fils de Tsicha, les fils de Hasupha, les fils de Thabbaoth,
\VS{47}les fils de Kéros, les fils de Sia, les fils de Padon,
\VS{48}les fils de Lebana, les fils de Hagaba, les fils de Salmaï,
\VS{49}les fils de Hanan, les fils de Guiddel, les fils de Gachar,
\VS{50}les fils de Reaja, les fils de Retsin, les fils de Nekoda,
\VS{51}les fils de Gazzam, les fils d'Uzza, les fils de Paséach,
\VS{52}les fils de Bésaï, les fils de Mehunim, les fils de Nephischsim,
\VS{53}les fils de Bakbuk, les fils de Hakupha, les fils de Harhur,
\VS{54}les fils de Batslith, les fils de Mehida, les fils de Harscha,
\VS{55}les fils de Barkos, les fils de Sisera, les fils de Thamach,
\VS{56}les fils de Netsiach, les fils de Hathipha.
\TextTitle{Liste des fils des serviteurs de Salomon revenus de captivité}
\VS{57}Fils des serviteurs de Salomon : Les fils de Sothaï, les fils de Sophéreth, les fils de Perida,
\VS{58}les fils de Jaala, les fils de Darkon, les fils de Guiddel,
\VS{59}les fils de Schephathia, les fils de Hatthil, les fils de Pokéreth-Hatsebaïm, les fils d'Amon.
\VS{60}Tous les Néthiniens, et les fils des serviteurs de Salomon, étaient trois cent quatre-vingt-douze.
\VS{61}Voici ceux qui montèrent de Thel-Mélach, de Thel-Harscha, de Kerub-Addon et d'Immer, lesquels ne purent montrer la maison de leurs pères, ni leur race, pour prouver qu'ils étaient d'Israël.
\VS{62}Les fils de Delaja, les fils de Tobija, les fils de Nekoda, six cent quarante-deux.
\TextTitle{Liste des sacrificateurs exclus de la sacrificature}
\VS{63}Et les sacrificateurs : Les fils de Hobaja, les fils d'Hakkots, les fils de Barzillaï, qui prit pour femme une des filles de Barzillaï, le Galaadite, et qui fut appelé de leur nom.
\VS{64}Ils cherchèrent leur registre généalogique, mais ils n'y furent point trouvés ; c'est pourquoi ils furent exclus de la sacrificature.
\VS{65}Et le gouverneur leur dit de ne pas manger des choses très saintes, jusqu'à ce que le sacrificateur eût consulté l'urim et le thummim\FTNT{Ex. 28:30.}.
\TextTitle{Somme des Israélites revenus de captivité}
\VS{66}Toute l'assemblée réunie était de quarante-deux mille trois cent soixante ;
\VS{67}sans leurs serviteurs et leurs servantes, qui étaient sept mille trois cent trente-sept ; et ils avaient deux cent quarante-cinq chantres ou chanteuses.
\TextTitle{Dons des fils d'Israël pour le trésor}
\VS{68}Ils avaient sept cent trente-six chevaux, deux cent quarante-cinq mulets ;
\VS{69}quatre cent trente-cinq chameaux et six mille sept cent vingt ânes.
\VS{70}Or quelques-uns des chefs des pères firent des dons pour l'ouvrage. Le gouverneur donna au trésor mille drachmes d'or, cinquante coupes, cinq cent trente tuniques de sacrificateurs.
\VS{71}Quelques autres d'entre les chefs des pères donnèrent pour le trésor de l'ouvrage vingt mille drachmes d'or et deux mille deux cent mines d'argent.
\VS{72}Le reste du peuple donna vingt mille drachmes d'or, deux mille mines d'argent et soixante-sept tuniques de sacrificateurs.
\VS{73}Et ainsi les sacrificateurs, les Lévites, les portiers, les chantres, quelques-uns du peuple, les Néthiniens, et tous ceux d'Israël habitèrent dans leurs villes. Ainsi, quand le septième mois approcha, les enfants d'Israël étaient dans leurs villes.
\Chap{8}
\TextTitle{Lecture du livre de la loi, conviction de péché du peuple}
\VerseOne{}Or tout le peuple s'assembla, comme un seul homme, sur la place qui est devant la porte des eaux. Et ils dirent à Esdras, le scribe, d'apporter le livre de la loi de Moïse, que Yahweh avait ordonnée à Israël.
\VS{2}Et ainsi le premier jour du septième mois, Esdras, le sacrificateur, apporta la loi devant l'assemblée, composée d'hommes et de femmes, et de tous ceux qui étaient capables de l'entendre.
\VS{3}Et il lut dans le livre, sur la place qui est devant la porte des eaux, depuis le matin jusqu'au milieu du jour, en présence des hommes et des femmes, et de ceux qui étaient capables d'entendre. Et les oreilles de tout le peuple étaient attentives à la lecture du livre de la loi.
\VS{4}Ainsi Esdras, le scribe, était debout sur une tour bâtie de bois, qu'on avait dressée pour cela. Il avait auprès de lui, à sa droite, Matthithia, Schéma, Anaja, Urie, Hilkija et Maaséja ; et à sa gauche étaient Pedaja, Mischaël, Malkija, Haschum, Haschbaddana, Zacharie, et Meschullam.
\VS{5}Esdras ouvrit le livre devant les yeux de tout le peuple ; car il était au-dessus de tout le peuple ; et sitôt qu'il l'eut ouvert, tout le peuple se tint debout.
\VS{6}Puis Esdras bénit Yahweh, le grand Dieu ; et tout le peuple répondit en élevant leurs mains: Amen ! Amen ! Et ils s'inclinèrent et se prosternèrent devant Yahweh, le visage contre terre.
\VS{7}Aussi Josué, Bani, Schérébia, Jamin, Akkub, Schabbethaï, Hodija, Maaséja, Kelitha, Azaria, Jozabad, Hanan, Pelaja, et les Lévites, faisaient comprendre la loi au peuple, et le peuple se tenait à sa place.
\VS{8}Et ils lisaient dans le livre de la loi de Dieu, ils l'expliquaient et en donnaient l'intelligence, la faisant comprendre par l'Ecriture elle-même.
\VS{9}Or Néhémie, qui est le gouverneur, Esdras, le sacrificateur et le scribe, et les Lévites qui instruisaient le peuple dirent à tout le peuple : Ce jour est consacré à Yahweh, notre Dieu ; ne soyez pas dans les lamentations, et ne pleurez point ! Car tout le peuple pleurait en entendant les paroles de la loi.
\VS{10}Puis on leur dit : Allez, mangez des viandes grasses, et buvez du vin doux ; et envoyez-en des portions à ceux qui n'ont rien de prêt ; car ce jour est consacré à notre Seigneur. Ne soyez donc point tristes, car la joie de Yahweh est votre force.
\VS{11}Et les Lévites faisaient faire silence parmi tout le peuple, en disant : Taisez-vous, car ce jour est saint, et ne vous affligez point.
\VS{12}Ainsi tout le peuple s'en alla pour manger et pour boire, pour envoyer des portions, et pour faire une grande réjouissance, parce qu'ils avaient bien compris les paroles qu'on leur avait fait connaître.
\TextTitle{Célébration de la fête des tabernacles}
\VS{13}Et le second jour, les chefs des pères de tout le peuple, les sacrificateurs et les Lévites, s'assemblèrent auprès d'Esdras, le scribe, pour sagement comprendre les paroles de la loi.
\VS{14}Et ils trouvèrent écrit dans la loi que Yahweh avait ordonnée par Moïse, que les enfants d'Israël devaient habiter sous des tentes\FTNT{Voir les sept fêtes de Yahweh en Lé. 23.} pendant la fête solennelle au septième mois.
\VS{15}Ce qu'ils firent savoir et qu'ils publièrent dans toutes leurs villes et à Jérusalem, en disant : Allez sur la montagne, et apportez des rameaux d'oliviers, et des rameaux d'autres arbres huileux, des rameaux de myrte, des rameaux de palmier, et des rameaux d'arbres touffus, afin de faire des tentes, selon ce qui est écrit.
\VS{16}Alors le peuple alla et apporta des rameaux. Ils se firent des tentes, chacun sur son toit, dans les cours de leurs maisons, et dans les parvis de la maison de Dieu, sur la place de la porte des eaux, et sur la place de la porte d'Ephraïm.
\VS{17}Ainsi toute l'assemblée de ceux qui étaient revenus de la captivité fit des tentes, et ils habitèrent sous ces tentes. Or les enfants d'Israël n'en avaient point fait de telles depuis les jours de Josué, fils de Nun, jusqu'à ce jour ; et il y eut une très grande joie.
\VS{18}On lut dans le livre de la loi de Dieu chaque jour, depuis le premier jour jusqu'au dernier. On célébra la fête pendant sept jours, et il y eut une assemblée solennelle au huitième jour, comme cela est ordonné.
\Chap{9}
\TextTitle{Confession, jeune et prière du peuple}
\VerseOne{}Et le vingt-quatrième jour du même mois, les enfants d'Israël s'assemblèrent, jeûnant, revêtus de sacs, et ayant de la terre sur eux.
\VS{2}Et la race d'Israël se sépara de tous les étrangers, et ils se présentèrent confessant leurs péchés et les iniquités de leurs pères.
\VS{3}Ils se levèrent donc à leur place, et on lut dans le livre de la loi de Yahweh, leur Dieu, pendant un quart de la journée, et pendant un autre quart, ils faisaient confession, et se prosternaient devant Yahweh, leur Dieu.
\TextTitle{Prière des Lévites, alliance avec Yahweh}
\VS{4}Josué, Bani, Kadmiel, Schebania, Bunni, Schérébia, Bani et Kenani se levèrent sur le lieu qu'on avait élevé pour les Lévites, et crièrent à haute voix à Yahweh, leur Dieu.
\VS{5}Et les Lévites Josué, Kadmiel, Bani, Haschabnia, Schérébia, Hodija, Schebania et Pethachja, dirent : Levez-vous, bénissez Yahweh, votre Dieu, d'éternité en éternité ! Que l'on bénisse ton Nom glorieux, qui est au-dessus de toute bénédiction et de toute louange !
\VS{6}Toi seul, Yahweh, tu as fait les cieux, les cieux des cieux, et toute leur armée ; la terre, et tout ce qui y est ; les mers, et toutes les choses qui y vivent. Tu donnes la vie à toutes ces choses, et l'armée des cieux se prosterne devant toi.
\VS{7}Tu es Yahweh, notre Dieu, qui as choisi Abram, et qui l'as fait sortir d'Ur en Chaldée, et qui lui as donné le nom d'Abraham\FTNT{Ge. 11:31 ; Ge. 17:5.}.
\VS{8}Tu trouvas son coeur fidèle devant toi, et tu traitas avec lui cette alliance que tu donneras à sa postérité le pays des Cananéens, des Héthiens, des Amoréens, des Phéréziens, des Jébusiens, et des Guirgasiens. Et tu as accompli ce que tu as promis, parce que tu es juste.
\VS{9}Car tu vis l'affliction de nos pères en Egypte et tu entendis leurs cris près de la Mer Rouge\FTNT{Ex. 2:23-25.}.
\VS{10}Tu fis des miracles et des prodiges sur Pharaon et sur tous ses serviteurs, et sur tout le peuple de son pays ; parce que tu connus qu'ils s'étaient orgueilleusement élevés contre eux, et tu t'es acquis un renom, tel qu'il paraît aujourd'hui.
\VS{11}Tu fendis aussi la mer devant eux, et ils passèrent à sec au milieu de la mer ; et tu jetas dans l'abîme ceux qui les poursuivaient, comme une pierre dans les eaux violentes.
\VS{12}Tu les fis marcher de jour par la colonne de nuée, et de nuit par la colonne de feu, pour les éclairer dans le chemin par où ils devaient aller\FTNT{Ex. 13:21.}.
\VS{13}Tu descendis sur la montagne de Sinaï, tu parlas avec eux du haut des cieux, tu leur donnas des ordonnances justes et des lois de vérité, des statuts et des commandements bons.
\VS{14}Tu leur fis connaître ton saint sabbat\FTNT{Ge 2:1-3 ; Ex. 20:8-11.} ; et tu leur donnas les commandements, les statuts, et la loi par Moïse, ton serviteur.
\VS{15}Tu leur donnas aussi, du haut des cieux, du pain quand ils avaient faim, et tu fis sortir de l'eau du rocher quand ils avaient soif\FTNT{Ex. 16:13-36 ; No. 20 : 8.}. Et tu leur dis d'entrer et de posséder le pays que tu avais juré de leur donner.
\VS{16}Mais nos pères s'élevèrent orgueilleusement et raidirent leur cou. Ils n'écoutèrent point tes commandements.
\VS{17}Ils refusèrent d'écouter et ne se souvinrent point des merveilles que tu avais faites en leur faveur. Mais ils raidirent leur cou, et par leur rébellion, ils s'attribuèrent un chef pour retourner à leur servitude. Mais toi, tu es un Dieu qui pardonne, miséricordieux, compatissant, lent à la colère et abondant en bonté, et tu ne les abandonnas pas.
\VS{18}Et quand ils se firent un veau en métal fondu et qu'ils dirent : Voici ton Dieu qui t'a fait sortir hors d'Egypte, et qu'ils te firent de grands outrages\FTNT{Ex. 32:1-14.} ;
\VS{19}dans ton immense miséricorde, tu ne les abandonnas pourtant pas dans le désert ; et la colonne de nuée ne se retira point pour les conduire le jour par le chemin, ni la colonne de feu la nuit, pour les éclairer dans le chemin par lequel ils devaient aller.
\VS{20}Tu leur donnas ton bon Esprit pour les rendre sages ; tu ne retiras point ta manne de leur bouche, et tu leur donnas de l'eau pour leur soif.
\VS{21}Tu les nourris ainsi quarante ans au désert, en sorte que rien ne leur manqua. Leurs vêtements ne s'usèrent point, et leurs pieds ne s'enflèrent point.
\VS{22}Tu leur donnas les royaumes et les peuples, dont tu partageas entre eux les contrées ; et ils possédèrent le pays de Sihon, le pays du roi de Hesbon, et le pays d'Og, roi de Basan.
\VS{23}Et tu multiplias leurs fils comme les étoiles des cieux, et les fis entrer au pays dont tu avais dit à leurs pères qu'ils y entreraient pour le posséder.
\VS{24}Ainsi leurs fils y entrèrent et possédèrent le pays ; tu humilias devant eux les habitants du pays, les Cananéens, et les livras entre leurs mains, eux et leurs rois, et les peuples du pays, afin qu'ils en fissent selon leur volonté.
\VS{25}Ils prirent les villes fortifiées et la terre grasse, ils possédèrent les maisons remplies de toutes sortes de biens, les puits qu'on avait creusés, les vignes, les oliviers, et les arbres fruitiers en abondance ; ils mangèrent, ils se rassasièrent ; ils s'engraissèrent et ils vécurent dans les délices de ta grande bonté.
\VS{26}Mais ils se rebellèrent et se révoltèrent contre toi. Ils jetèrent ta loi derrière leur dos, ils tuèrent tes prophètes qui les avertissaient pour les ramener à toi, et ils te firent de grands outrages.
\VS{27}C'est pourquoi tu les donnas aux mains de leurs ennemis, qui les opprimèrent. Mais au temps de leur détresse, ils crièrent à toi, et tu les entendis des cieux ; et selon ta grande miséricorde, tu leur donnas des libérateurs qui les délivrèrent de la main de leurs ennemis.
\VS{28}Mais dès qu'ils eurent du repos, ils recommencèrent à faire le mal devant toi. Alors tu les abandonnas entre les mains de leurs ennemis, qui dominèrent sur eux. Puis ils revinrent et crièrent vers toi, et tu les entendis des cieux. Ainsi tu les délivras selon tes miséricordes, plusieurs fois, et en divers temps.
\VS{29}Et tu les exhortas à revenir à ta loi, mais ils s'élevèrent orgueilleusement et n'écoutèrent pas tes commandements ; ils péchèrent contre tes ordonnances, qui font vivre l'homme qui les observe. Ils tirèrent l'épaule en arrière, raidirent leur cou et n'écoutèrent pas.
\VS{30}Tu les supportas patiemment plusieurs années, et tu les avertissais par ton Esprit, par la main de tes prophètes ; mais ils ne prêtèrent point l'oreille. C'est pourquoi tu les livras entre les mains des peuples des pays étrangers.
\VS{31}Néanmoins, dans ta grande miséricorde, tu ne les anéantis pas et tu ne les abandonnas pas ; car tu es un Dieu compatissant et miséricordieux.
\VS{32}Et maintenant donc, ô notre Dieu ! Grand, puissant et terrifiant, qui garde ton alliance et la miséricorde ; ne regarde pas comme peu de chose cette affliction qui nous est arrivée, à nous, à nos rois, à nos chefs, à nos sacrificateurs, à nos prophètes, à nos pères et à tout ton peuple, depuis le temps des rois d'Assyrie jusqu'à aujourd'hui.
\VS{33}Tu as été juste dans toutes les choses qui nous sont arrivées ; car tu as agi avec fidélité, mais nous, nous avons agi méchamment.
\VS{34}Nos rois, nos chefs, nos sacrificateurs et nos pères n'ont point pratiqué ta loi et n'ont point été attentifs à tes commandements ni à tes témoignages par lesquels tu les as avertis.
\VS{35}Ils ne t'ont point servi durant leur règne ni durant les grands biens que tu leur as faits, même dans le pays vaste et riche que tu leur avais donné pour être à leur disposition, et ils ne se sont point détournés de leurs mauvaises oeuvres.
\VS{36}Voici, nous sommes aujourd'hui esclaves ! Sur la terre que tu as donnée à nos pères pour en manger le fruit et les biens ; voici, nous y sommes esclaves !
\VS{37}Elle rapporte ses produits en abondance pour les rois que tu as établis sur nous à cause de nos péchés, et qui dominent sur nos corps et sur nos bêtes, à leur volonté, de sorte que nous sommes dans une grande angoisse !
\VS{38}C'est pourquoi, à cause de toutes ces choses, nous contractâmes une alliance et nous l'écrivîmes ; et les chefs d'entre nous, nos Lévites et nos sacrificateurs y apposèrent leur sceau.
\Chap{10}
\TextTitle{Liste des contractants et termes de l'alliance}
\VerseOne{}Voici ceux qui apposèrent leur sceau. Néhémie, qui est le gouverneur, fils de Hacalia, et Sédécias.
\VS{2}Seraja, Azaria, Jérémie,
\VS{3}Paschhur, Amaria, Malkija,
\VS{4}Hattusch, Schebania, Malluc,
\VS{5}Harim, Merémoth, Abdias,
\VS{6}Daniel, Guinnethon, Baruc,
\VS{7}Meschullam, Abija, Mijamin,
\VS{8}Maazia, Bilgaï et Schemaeja. Ce sont les sacrificateurs.
\VS{9}Des Lévites : Josué, fils d'Azania, Binnuï d'entre les fils de Hénadad, et Kadmiel.
\VS{10}Et leurs frères, Schebania, Hodija, Kelitha, Pelaja, Hanan,
\VS{11}Michée, Rehob, Haschabia.
\VS{12}Zaccur, Schérébia, Schebania,
\VS{13}Hodija, Bani et Beninu.
\VS{14}Des chefs du peuple : Pareosch, Pachath-Moab, Elam, Zatthu, Bani,
\VS{15}Bunni, Azgad, Bébaï,
\VS{16}Adonija, Bigvaï, Adin,
\VS{17}Ather, Ezéchias, Azzur,
\VS{18}Hodija, Haschum, Betsaï,
\VS{19}Hariph, Anathoth, Nébaï,
\VS{20}Magpiasch, Meschullam, Hézir,
\VS{21}Meschézabeel, Tsadok, Jaddua,
\VS{22}Pelathia, Hanan, Anaja,
\VS{23}Hosée, Hanania, Haschub,
\VS{24}Hallochesch, Pilcha, Schobek,
\VS{25}Rehum, Haschabna, Maaséja,
\VS{26}Achija, Hanan, Anan,
\VS{27}Malluc, Harim et Baana.
\VS{28}Quant au reste du peuple, les sacrificateurs, les Lévites, les portiers, les chantres, les Néthiniens et tous ceux qui s'étaient séparés des peuples de ces pays pour suivre la loi de Dieu, leurs femmes, leurs fils et leurs filles, tous ceux qui étaient capables de connaissance et d'intelligence,
\VS{29}se joignirent à leurs frères les plus considérables d'entre eux. Ils s'engagèrent par serment et jurèrent de marcher dans la loi de Dieu, qui avait été donnée par Moïse, serviteur de Dieu ; de garder et faire tous les commandements de Yahweh, notre Seigneur, ses jugements et ses ordonnances ;
\VS{30}de ne pas donner nos filles aux peuples du pays, et de ne pas prendre leurs filles pour nos fils ;
\VS{31}de ne rien prendre le jour du sabbat, ou tel autre jour consacré, des peuples du pays qui apporteraient des marchandises et toutes sortes de denrées, le jour du sabbat, pour les vendre, d'abandonner la septième année et de faire remise de toute dette.
\VS{32}Nous fîmes aussi des ordonnances, nous chargeant de donner chaque année le tiers d'un sicle, pour le service de la maison de notre Dieu,
\VS{33}pour les pains de proposition, pour l'offrande perpétuelle et pour l'holocauste perpétuel ; pour ceux des sabbats, des nouvelles lunes et des fêtes ; pour les choses consacrées, pour les sacrifices d'expiation afin de faire propitiation pour Israël ; et pour toute l'oeuvre de la maison de notre Dieu.
\VS{34}Nous tirâmes au sort, pour l'offrande du bois, tant les sacrificateurs et les Lévites, que le peuple, afin de l'amener dans la maison de notre Dieu, selon les maisons de nos pères, et dans les temps fixés, d'année en année, pour le brûler sur l'autel de Yahweh, notre Dieu, ainsi qu'il est écrit dans la loi.
\VS{35}Nous décidâmes aussi d'apporter dans la maison de Yahweh, d'année en année, les premiers fruits de notre terre, et les prémices de tous les fruits de tous les arbres ;
\VS{36}d'amener les premiers-nés de nos fils, et de nos bêtes, comme il est écrit dans la loi ; et d'amener dans la maison de notre Dieu, aux sacrificateurs qui font le service dans la maison de notre Dieu, les premiers-nés de nos boeufs et de notre menu bétail;
\VS{37}d'apporter les prémices de notre pâte, nos offrandes, les fruits de tous les arbres, le vin, et l'huile aux sacrificateurs, dans les chambres de la maison de notre Dieu, et la dîme de notre terre aux Lévites, et que les Lévites prendraient les dîmes dans toutes les villes agricoles.
\VS{38}Le sacrificateur, fils d'Aaron, sera avec les Lévites, lorsque les Lévites paieront la dîme\FTNT{Il est question de la dîme des Lévites (No. 18:24 ; De. 14:28-29).}; et les Lévites apporteront la dîme de la dîme\FTNT{Il s'agit ici de la dîme de la dîme que les Lévites donnaient aux sacificateurs. Elle était apportée aux magasins du temple. Voir commentaires en No. 18:21 et Mal. 3:10.} à la maison de notre Dieu, dans les chambres de la maison où sont les magasins\FTNT{Le mot hébreu « owtsar » (trésor) signifie aussi magasin (Né. 12:44 ; Né. 13:12. Né. 13:13).}.
\VS{39}Car les enfants d'Israël et les fils de Lévi apporteront dans ces chambres les offrandes du blé, du vin et de l'huile ; là sont les ustensiles du sanctuaire, et les sacrificateurs qui font le service, les portiers, et les chantres. Et nous n'abandonnâmes point la maison de notre Dieu.
\Chap{11}
\TextTitle{Les habitants de Jérusalem}
\VerseOne{}Les chefs du peuple demeurèrent à Jérusalem. Mais tout le reste du peuple tira au sort, afin qu'un sur dix vînt habiter à Jérusalem, la ville sainte, et que les neuf autres parties demeurassent dans les autres villes.
\VS{2}Et le peuple bénit tous ceux qui se présentèrent volontairement pour habiter à Jérusalem.
\VS{3}Voici les chefs de la province qui habitèrent à Jérusalem ; les autres s'étant établis dans les villes de Juda, chacun dans sa propriété, selon sa ville, Israélites, sacrificateurs, Lévites, Néthiniens, et les fils des serviteurs de Salomon.
\VS{4}A Jérusalem habitèrent donc des fils de Juda et des fils de Benjamin. Des fils de Juda : Athaja, fils d'Ozias, fils de Zacharie, fils d'Amaria, fils de Schephathia, fils de Mahalaleel, d'entre les fils de Pérets,
\VS{5}et Maaséja, fils de Baruc, fils de Col-Hozé, fils de Hazaja, fils d'Adaja, fils de Jojarib, fils de Zacharie, fils de Schiloni.
\VS{6}Total des fils de Pérets, qui s'établirent à Jérusalem : Quatre cent soixante-huit vaillants hommes.
\VS{7}Voici les fils de Benjamin : Sallu, fils de Meschullam, fils de Joëd, fils de Pedaja, fils de Kolaja, fils de Maaséja, fils d'Ithiel, fils d'Esaïe,
\VS{8}et après lui, Gabbaï et Sallaï : Neuf cent vingt-huit.
\VS{9}Joël, fils de Zicri, était leur chef ; et Juda, fils de Senua, était le second chef de la ville.
\VS{10}Des sacrificateurs : Jedaeja, fils de Jojarib, Jakin,
\VS{11}Seraja, fils de Hilkija, fils de Meschullam, fils de Tsadok, fils de Merajoth, fils d'Achithub, prince de la maison de Dieu,
\VS{12}et leurs frères, faisant le service de la maison : Huit cent vingt-deux. Adaja, fils de Jerocham, fils de Pelalia, fils d'Amtsi, fils de Zacharie, fils de Paschhur, fils de Malkija,
\VS{13}et ses frères, chefs des pères : Deux cent quarante-deux ; et Amaschsaï, fils d'Azareel, fils d'Achzaï, fils de Meschillémoth, fils d'Immer,
\VS{14}et leurs frères, forts et vaillants: Cent vingt-huit. Zabdiel, fils de Guedolim, était leur chef.
\VS{15}Des Lévites : Schemaeja, fils de Haschub, fils d'Azrikam, fils de Haschabia, fils de Bunni,
\VS{16}Schabbethaï et Jozabad chargés des travaux extérieurs pour la maison de Dieu, étant d'entre les chefs des Lévites ;
\VS{17}Matthania, fils de Michée, fils de Zabdi, fils d'Asaph, était le chef qui commençait le premier à chanter les louanges dans la prière, et Bakbukia, le second parmi ses frères, puis Abda, fils de Schammua, fils de Galal, fils de Jeduthun.
\VS{18}Total des Lévites dans la ville sainte : Deux cent quatre-vingt-quatre.
\VS{19}Et les portiers : Akkub, Thalmon, et leurs frères qui gardaient les portes : Cent soixante-douze.
\TextTitle{Les habitants des autres villes}
\VS{20}Le reste d'Israël, des sacrificateurs et des Lévites, fut dans toutes les villes de Juda, chacun dans son héritage.
\VS{21}Mais les Néthiniens habitèrent sur la colline ; et Tsicha et Guischpa étaient leurs chefs.
\VS{22}Celui qui avait la charge des Lévites à Jérusalem était Uzzi, fils de Bani, fils de Haschabia, fils de Matthania, fils de Michée, d'entre les fils d'Asaph, chantres, pour l'ouvrage de la maison de Dieu ;
\VS{23}car il y avait un commandement du roi à leur égard, et il y avait chaque jour un salaire assuré pour les chantres.
\VS{24}Pethachja, fils de Meschézabeel, d'entre les fils de Zérach, fils de Juda, était commissaire du roi pour toutes les affaires du peuple.
\VS{25}Dans les villages et leurs territoires, quelques-uns des fils de Juda habitèrent à Kirjath-Arba, et dans les lieux de son ressort ; à Dibon, et dans les lieux de son ressort ; à Jekabtseel, et dans les villages de son ressort,
\VS{26}à Jéschua, à Molada, à Beth-Paleth,
\VS{27}à Hatsar-Schual, à Beer-Schéba, et dans les lieux de son ressort,
\VS{28}à Tsiklag, à Mecona, et dans les lieux de son ressort,
\VS{29}à En-Rimmon, à Tsorea, à Jarmuth,
\VS{30}à Zanoach, à Adullam, et dans leurs villages, à Lakis et dans ses territoires, à Azéka et dans les lieux de son ressort. Ils habitèrent depuis Beer-Schéba jusqu'à la vallée de Hinnom.
\VS{31}Et les fils de Benjamin habitèrent depuis Guéba à Micmasch, à Ajja, à Béthel, et dans les lieux de son ressort,
\VS{32}à Anathoth, à Nob, à Hanania,
\VS{33}à Hatsor, à Rama, à Guitthaïm,
\VS{34}à Hadid, à Tseboïm, à Neballath,
\VS{35}à Lod, et à Ono, la vallée des ouvriers.
\VS{36}D'entre les Lévites, des classes de Juda se rattachèrent à Benjamin.
\Chap{12}
\TextTitle{Les sacrificateurs et les Lévites montés avec Zorobabel}
\VerseOne{}Voici les sacrificateurs et les Lévites qui montèrent avec Zorobabel, fils de Schealthiel, et avec Josué : Seraja, Jérémie, Esdras,
\VS{2}Amaria, Malluc, Hattusch,
\VS{3}Schecania, Rehum, Merémoth,
\VS{4}Iddo, Guinnethoï, Abija,
\VS{5}Mijamin, Maadia, Bilga,
\VS{6}Schemaeja, Jojarib, Jedaeja,
\VS{7}Sallu, Amok, Hilkija, Jedaeja. Ce furent là les chefs des sacrificateurs, et de leurs frères, du temps de Josué.
\VS{8}Lévites : Josué, Binnuï, Kadmiel, Schérébia, Juda, Matthania, qui dirigeait les louanges, lui et ses frères.
\VS{9}Bakbukia et Unni, leurs frères, étaient avec eux pour la surveillance.
\TextTitle{Les fils des sacrificateurs}
\VS{10}Josué engendra Jojakim, Jojakim engendra Eliaschib, Eliaschib engendra Jojada,
\VS{11}Jojada engendra Jonathan, et Jonathan engendra Jaddua.
\VS{12}Au temps de Jojakim, étaient sacrificateurs, chefs des pères : Pour Seraja, Meraja ; pour Jérémie, Hanania ;
\VS{13}pour Esdras, Meschullam ; pour Amaria, Jochanan ;
\VS{14}pour Meluki, Jonathan ; pour Schebania, Joseph ;
\VS{15}pour Harim, Adna ; pour Merajoth, Helkaï ;
\VS{16}pour Iddo, Zacharie ; pour Guinnethon, Meschullam ;
\VS{17}pour Abija, Zicri ; pour Minjamin et Moadia, Pilthaï ;
\VS{18}pour Bilga, Schammua ; pour Schemaeja, Jonathan ;
\VS{19}pour Jojarib, Matthnaï ; pour Jedaeja, Uzzi ;
\VS{20}pour Sallaï, Kallaï ; pour Amok, Eber ;
\VS{21}pour Hilkija, Haschabia ; pour Jedaeja, Nethaneel.
\TextTitle{Les chefs des fils de Lévi}
\VS{22}Au temps d'Eliaschib, de Jojada, de Jochanan et de Jaddua, les Lévites, chefs de famille, et les sacrificateurs, furent inscrits sous le règne de Darius, le Perse.
\VS{23}Les fils de Lévi, chefs des pères, furent enregistrés dans le livre des Chroniques jusqu'au temps de Jochanan, fils d'Eliaschib.
\VS{24}Les chefs des Lévites, Haschabia, Schérébia, et Josué, fils de Kadmiel, et leurs frères, étaient vis-à-vis d'eux, pour louer et célébrer, selon l'ordre de David, homme de Dieu.
\VS{25}Matthania, Bakbukia, Abdias, Meschullam, Thalmon, et Akkub, les portiers, faisaient la garde au seuil des portes.
\VS{26}Ce fut du temps de Jojakim, fils de Josué, fils de Jotsadak, et du temps de Néhémie, le gouverneur, et d'Esdras, sacrificateur et scribe.
\TextTitle{La dédicace de la muraille de Jérusalem}
\VS{27}Lors de la dédicace de la muraille de Jérusalem, on envoya chercher les Lévites de tous les lieux où ils étaient, pour les faire venir à Jérusalem, afin de célébrer la dédicace avec joie, par des louanges, et par des chants sur des cymbales, des luths et des harpes.
\VS{28}Les fils des chantres se rassemblèrent des plaines aux alentours de Jérusalem, des villages des Nethophatiens,
\VS{29}de Beth-Guilgal, et des territoires de Guéba et d'Azmaveth ; car les chantres s'étaient bâtis des villages aux alentours de Jérusalem.
\VS{30}Les sacrificateurs et les Lévites se purifièrent, et ils purifièrent le peuple, les portes et la muraille.
\VS{31}Puis je fis monter sur la muraille les chefs de Juda, et j'établis deux grands chœurs. Le premier se mit en marche du côté droit sur la muraille, vers la porte du fumier.
\VS{32}Et après eux marchait Hosée, avec la moitié des chefs de Juda,
\VS{33}Azaria, Esdras, Meschullam,
\VS{34}Juda, Benjamin, Schemaeja et Jérémie,
\VS{35}des fils des sacrificateurs avec les trompettes, Zacharie, fils de Jonathan, fils de Schemaeja, fils de Matthania, fils de Michée, fils de Zaccur, fils d'Asaph,
\VS{36}et ses frères, Schemaeja, Azareel, Milalaï, Guilalaï, Maaï, Nethaneel, Juda, et Hanani, avec les instruments des cantiques de David, homme de Dieu. Esdras, le scribe, marchait devant eux.
\VS{37}A la porte de la source, qui était vis-à-vis d'eux, ils montèrent aux marches de la cité de David, par la montée de la muraille, depuis la maison de David, jusqu'à la porte des eaux, vers l'orient.
\VS{38}Le second choeur de ceux qui chantaient les louanges allait à l'opposé. J'allais après lui, avec l'autre moitié du peuple, allant sur la muraille. Passant par-dessus la tour des fours, jusqu'à la muraille large ;
\VS{39}puis vers la porte d'Ephraïm, vers la vieille porte, vers la porte des poissons, la tour de Hananeel, et la tour de Méa, jusqu'à la porte des brebis. Et l'on s'arrêta à la porte de la prison.
\VS{40}Les deux choeurs s'arrêtèrent dans la maison de Dieu ; moi aussi, avec les magistrats qui étaient avec moi,
\VS{41}et les sacrificateurs Eliakim, Maaséja, Minjamin, Michée, Eljoénaï, Zacharie, Hanania, avec les trompettes,
\VS{42}et Maaséja, Schemaeja, Eléazar, Uzzi, Jochanan, Malkija, Elam et Ezer. Puis les chantres, desquels Jizrachja avait la charge, se firent entendre.
\VS{43}On offrit ce jour-là de nombreux sacrifices, et on se réjouit, parce que Dieu leur avait donné un grand sujet de joie. Les femmes et les enfants se réjouirent aussi ; et la joie de Jérusalem fut entendue au loin.
\TextTitle{Les sacrificateurs et les Lévites à leur poste}
\VS{44}En ce jour-là, on établit des hommes sur les chambres des trésors, des offrandes, des prémices et des dîmes ; pour rassembler du territoire des villes les portions ordonnées par la loi aux sacrificateurs et aux Lévites. Car Juda se réjouissait de ce que les sacrificateurs et de ce que les Lévites étaient à leur poste,
\VS{45}et parce qu'ils avaient gardé la charge qui leur avait été donnée de la part de leur Dieu, et la charge de la purification. Les chantres et les portiers remplissaient aussi leurs fonctions, selon le commandement de David, et de Salomon, son fils.
\VS{46}Car autrefois, du temps de David et d'Asaph, on avait établi des chefs de chantres et des cantiques de louange et de reconnaissance à Dieu.
\VS{47}Tout Israël, du temps de Zorobabel et de Néhémie, donna les portions des chantres et des portiers, jour par jour, et les consacraient aux Lévites, et les Lévites les consacraient aux fils d'Aaron.
\Chap{13}
\TextTitle{Lecture du livre de Moïse, séparation d'avec les étrangers}
\VerseOne{}Dans ce temps-là, on lut en présence du peuple dans le livre de Moïse, et l'on y trouva écrit que les Ammonites et les Moabites ne devaient jamais entrer dans l'assemblée de Dieu,
\VS{2}parce qu'ils n'étaient pas venus au-devant des enfants d'Israël avec du pain et de l'eau ; et qu'ils avaient engagé à prix d'argent Balaam\FTNT{Balaam : voir No. 22,23 et 24.} contre eux pour qu'il les maudisse ; mais notre Dieu changea la malédiction en bénédiction.
\VS{3}Dès qu'on eut entendu la loi, on sépara d'Israël tous les étrangers.
\TextTitle{Purification des chambres du temple}
\VS{4}Avant cela, le sacrificateur Eliaschib, établi sur les chambres de la maison de notre Dieu, et parent de Tobija,
\VS{5}avait disposé pour lui une grande chambre, où on mettait auparavant les offrandes, l'encens, les ustensiles, les dîmes du blé, du vin et de l'huile, qui étaient ordonnées pour les Lévites, pour les chantres et pour les portiers, avec les contributions pour les sacrificateurs.
\VS{6}Je n'étais point à Jérusalem pendant tout cela, car j'étais retourné vers le roi la trente-deuxième année d'Artaxerxès, roi de Babylone. Et à la fin de l'année, j'obtins du roi la permission
\VS{7}de revenir à Jérusalem, et je m'aperçus du mal qu'Eliaschib avait fait, en disposant une chambre pour Tobija dans le parvis de la maison de Dieu.
\VS{8}J'en éprouvai un vif déplaisir, et je jetai tous les objets de Tobija hors de la chambre ;
\VS{9}j'ordonnai qu'on purifie les chambres, et j'y ramenai les ustensiles de la maison de Dieu, les offrandes et l'encens.
\TextTitle{Sur les portions des Lévites}
\VS{10}J'appris aussi que les portions des Lévites ne leur avaient point été données ; et que les Lévites et les chantres qui faisaient le service s'étaient enfuis chacun sur sa terre.
\VS{11}Je fis des réprimandes aux magistrats, leur disant : Pourquoi a-t-on abandonné la maison de Dieu ? Je rassemblai les Lévites et les chantres, et les rétablis à leur place.
\VS{12}Alors tous ceux de Juda apportèrent dans le trésor les dîmes du blé, du vin et de l'huile.
\VS{13}Je confiai la surveillance du trésor à Schélémia, le sacrificateur, et Tsadok, le scribe, et Pedaja, l'un des Lévites ; et pour les aider, Hanan, fils de Zaccur, fils de Matthania, parce qu'ils étaient considérés comme très fidèles. Ils furent chargés de faire les distributions à leurs frères.
\VS{14}Souviens-toi de moi, ô mon Dieu, à cause de cela et n'efface point ce que j'ai fait avec fidélité pour la maison de mon Dieu, et pour ce qu'il est ordonné d'y faire !
\TextTitle{Avertissement pour le respect du sabbat}
\VS{15}En ces jours-là, je vis quelques-uns de Juda fouler aux pressoirs le jour du sabbat, et d'autres apporter des gerbes, et charger sur des ânes du vin, des raisins, des figues, et toutes autres sortes de fardeaux, et les apporter à Jérusalem le jour du sabbat ; et je les avertis le jour où ils vendaient leurs denrées.
\VS{16}Les Tyriens, qui demeuraient aussi à Jérusalem, apportaient du poisson, et plusieurs autres marchandises, et les vendaient aux fils de Juda dans Jérusalem le jour du sabbat.
\VS{17}Je fis des réprimandes aux chefs de Juda, et leur dis : Quel mal ne faites-vous pas, en violant le jour du sabbat ?
\VS{18}Vos pères n'ont-ils pas fait la même chose, et n'est-ce pas pour cela que notre Dieu a fait venir tout ce mal sur nous et sur cette ville ? Et vous amenez de nouveau son ardente colère contre Israël, en violant le sabbat !
\VS{19}C'est pourquoi, dès que le soleil s'était retiré des portes de Jérusalem, avant le sabbat, par mon commandement, on ferma les portes ; j'ordonnai aussi qu'on ne les ouvre point jusqu'après le sabbat. Et je plaçai quelques-uns de mes serviteurs aux portes, afin d'empêcher l'entrée des fardeaux le jour du sabbat.
\VS{20}Alors les marchands et les vendeurs de toutes sortes de denrées passèrent une ou deux fois la nuit hors de Jérusalem.
\VS{21}Je les avertis et je leur dis : Pourquoi passez-vous la nuit devant la muraille ? Si vous le faites encore, je mettrai la main sur vous. Ainsi, depuis ce temps-là, ils ne vinrent plus le jour du sabbat.
\VS{22}J'ordonnai aussi aux Lévites de se purifier, et de venir garder les portes pour sanctifier le jour du sabbat. Souviens-toi de moi, ô mon Dieu, à cause de cela, et ai compassion de moi selon la grandeur de ta miséricorde !
\TextTitle{Condamnation des unions mixtes ; rétablissement des fonctions des sacrificateurs et des Lévites}
\VS{23}En ces jours-là, je vis des Juifs qui avaient pris des femmes Asdodiennes, Ammonites et Moabites.
\VS{24}La moitié de leurs fils parlaient en partie asdodien et ne savaient point parler l'hébreu ; mais ils parlaient la langue de divers peuples.
\VS{25}Je leur fis des réprimandes et les maudis ; j'en frappai même quelques-uns, leur arrachai les cheveux et les fis jurer par Dieu, qu'ils ne donneraient point leurs filles à leurs fils, et qu'ils ne prendraient point leurs filles pour leurs fils, ou pour eux.
\VS{26}Salomon, le roi d'Israël, n'avait-il point péché par ce moyen ? Il n'y avait point de roi semblable à lui parmi un grand nombre de nations, il était aimé de son Dieu, et Dieu l'avait établi pour roi sur tout Israël ; toutefois, les femmes étrangères l'amenèrent à pécher.
\VS{27}Faut-il donc apprendre que vous fassiez tout ce grand mal, de commettre ce péché contre notre Dieu, en prenant des femmes étrangères ?
\VS{28}Or un des fils de Jojada, fils d'Eliaschib, grand sacrificateur, était gendre de Sanballat, le Horonite. Je le chassai loin de moi.
\VS{29}Souviens-toi d'eux, ô mon Dieu, car ils ont souillé le sacerdoce et l'alliance contractée par les sacrificateurs et les Lévites.
\VS{30}Ainsi je les nettoyai de tous les étrangers et je rétablis les fonctions des sacrificateurs et des Lévites, chacun selon ce qu'il avait à faire,
\VS{31}et ce qui concernait l'offrande du bois aux temps fixés, de même que les prémices. Souviens-toi de moi en bien, ô mon Dieu !
\PPE{}
\end{multicols}

%\clearpage\ShortTitle{1 Chroniques}\BookTitle{1 Chroniques}\BFont
\noindent\hrulefill
{\footnotesize
\textit{
\bigskip
{\centering{}
\\Auteur : Probablement Esdras
\\(Heb. : Hayyamim dibre)
\\Signification : Actes des journées
\\Thème : Généalogies et Histoire
\\Date de rédaction : 5ème siècle av. J.-C.\\}
}
%\bigskip
\textit{
\\Les deux livres des Chroniques constituent des compléments aux livres des Rois dans la mesure où ils confirment les récits de ceux-ci.
%\bigskip
\\Après avoir établi la généalogie d’Adam à Jacob, puis une généalogie plus détaillée de la descendance de Jacob jusqu’au retour de la captivité babylonienne, le premier livre des Chroniques reprend l’histoire du roi David et met un accent particulier sur certains combats qu’il eut à mener, les rapports avec ses serviteurs, ainsi que les préparatifs de la construction du temple. Il présente aussi l’organisation du travail des sacrificateurs et des Lévites au service de
Dieu et du peuple.\bigskip
}
}
\par\nobreak\noindent\hrulefill
\begin{multicols}{2}
\Chap{1}
\TextTitle{Généalogie d'Adam à Noé\FTNTT{Ge. 5:1-32}}
\VerseOne{}Adam, Seth, Enosch\FTNT{Les généalogies se faisaient par les premiers-nés de chaque famille.}.
\VS{2}Kénan, Mahalaleel, Jéred ;
\VS{3}Hénoc, Metuschélah, Lémec.
\VS{4}Noé, Sem, Cham et Japhet\FTNT{Ge. 5:1-32.}.
\TextTitle{Les fils de Japhet\FTNTT{Ge. 10:2-5.}}
\VS{5}Les fils de Japhet furent : Gomer, Magog, Madaï, Javan, Tubal, Méschec et Tiras.
\VS{6}Les fils de Gomer furent : Aschkenaz, Diphat et Togarma.
\VS{7}Les fils de Javan furent : Elischa, Tarsisa, Kittim et Rodanim.
\TextTitle{Les fils de Cham\FTNTT{Ge. 10:6-20.}}
\VS{8}Les fils de Cham furent : Cusch, Mitsraïm, Puth et Canaan.
\VS{9}Les fils de Cusch furent  : Saba, Havila, Sabta, Raema et Sabteca. Les fils de Raema furent : Séba et Dedan.
\VS{10}Cusch engendra aussi Nimrod qui commença à être puissant sur la terre.
\VS{11}Mitsraïm engendra les Ludim, les Anamim, les Lehabim, les Naphtuhim,
\VS{12}les Patrusim, les Casluhim, desquels sont issus les Philistins et les Caphtorim.
\VS{13}Canaan engendra Sidon, son fils aîné, et Heth;
\VS{14}les Jébusiens, les Amoréens, les Guirgasiens,
\VS{15}les Héviens, les Arkiens, les Siniens,
\VS{16}les Arvadiens, les Tsemariens et les Hamathiens.
\TextTitle{Les fils de Sem\FTNTT{Ge. 10:21-31.}}
\VS{17}Les fils de Sem furent : Elam, Assur, Arpacschad, Lud, Aram, Uts, Hul, Guéter et Méschec.
\VS{18}Arpacschad engendra Schélach, et Schélach engendra Héber.
\VS{19}A Héber naquirent deux fils : L'un s'appelait Péleg, car en son temps la terre fut partagée; et son frère s’appelait Jokthan.
\VS{20}Jokthan engendra Almodad, Schéleph, Hatsarmaveth, Jérach,
\VS{21}Hadoram, Uzal, Dikla,
\VS{22}Ebal, Abimaël, Séba,
\VS{23}Ophir, Havila et Jobab; tous ceux-là furent des fils de Jokthan\FTNT{Ge. 10:2-31.}.
\TextTitle{De Sem aux fils d'Abraham\FTNTT{Ge. 11:10-26.}}
\VS{24}Sem, Arpacschad, Schélach\FTNT{Ge. 11:10-26},
\VS{25}Héber, Péleg, Rehu,
\VS{26}Serug, Nachor, Térach,
\VS{27}et Abram, qui est Abraham.
\VS{28}Les fils d’Abraham furent  Isaac et Ismaël.
\TextTitle{Les fils d’Ismaël\FTNTT{Ge. 25:12-18.}}
\VS{29}Voici leur postérité\FTNT{Ge 25:12-18} : Le premier-né d'Ismaël fut Nebajoth, puis Kédar, Adbeel, Mibsam,
\VS{30}Mischma, Duma, Massa, Hadad, Téma,
\VS{31}Jethur, Naphisch et Kedma; ce sont là les fils d'Ismaël.
\TextTitle{Les fils de Ketura\FTNTT{Ge. 25:1-4.}}
\VS{32}Quant aux fils de Ketura, concubine d'Abraham, elle enfanta Zimran, Jokschan, Medan, Madian, Jischbak et Schuach; et les fils de Jokschan furent Séba et Dedan.
\VS{33}Les fils de Madian furent  Epha, Epher, Hénoc, Abida et Eldaa. Tous ceux-là furent les fils de Ketura.
\TextTitle{Les fils d’Isaac\FTNTT{Ge. 25:19-26.}}
\VS{34}Or Abraham engendra Isaac; et les fils d'Isaac furent  Esaü et Israël.
\TextTitle{Les descendants d’Esaü \FTNTT{Ge. 36:1-14.}}
\VS{35}Les fils d’Esaü furent  Eliphaz, Reuel, Jeusch, Jaelam et Koré\FTNT{Ge. 36:1-14.}.
\VS{36}Les fils d’Eliphaz furent Théman, Omar, Tsephi, Gaetham et Kenaz; Thimna lui enfanta Amalek.
\VS{37}Les fils de Reuel furent Nahath, Zérach, Schamma et Mizza.
\VS{38}Les fils de Séir furent Lothan, Schobal, Tsibeon, Ana, Dischon, Etser et Dischan.
\VS{39}Les fils de Lothan furent  Hori et Homam; et Thimna fut la soeur de Lothan.
\VS{40}Les fils de Schobal furent Aljan, Manahath, Ebal, Schephi et Onam. Les fils de Tsibeon furent Ajja et Ana.
\VS{41}Ana eut un fils : Dischon.  Les fils de Dischon furent Hamran, Eschban, Jithran et Keran.
\VS{42}Les fils d’Etser furent Bilhan, Zaavan et Jaakan. Les fils de Dischon furent Uts et Aran.
\TextTitle{Les rois et les chefs d’Edom\FTNTT{Ge. 36:15-19, 25-43}}
\VS{43}Voici les rois qui ont régné au pays d'Edom, avant qu’un roi ne règne sur les fils d’Israël : Béla, fils de Beor, et le nom de sa ville était Dinhaba.
\VS{44}Béla mourut, et Jobab, fils de Zérach de Botsra, régna à sa place.
\VS{45}Jobab mourut, et Huscham, du pays des Thémanites, régna à sa place.
\VS{46}Huscham mourut, et Hadad, fils de Bedad, régna à sa place. C’est lui qui frappa Madian dans les champs de Moab. Le nom de sa ville était Avith.
\VS{47}Hadad mourut, et Samla de Masréka, régna à sa place.
\VS{48}Samla mourut, et Saül de Rehoboth, sur le fleuve, régna à sa place.
\VS{49}Saül mourut, et Baal-Hanan, fils de Acbor, régna à sa place.
\VS{50}Baal-Hanan mourut, et Hadad régna à sa place. Le nom de sa ville était Pahi, et le nom de sa femme Mehéthabeel, qui était fille de Mathred, et petite- fille de Mézahab.
\VS{51}Enfin Hadad mourut. Ensuite vinrent les chefs d'Edom, le chef Thimna, le chef Alja, le chef Jetheth.
\VS{52}Le chef Oholibama, le chef Ela, le chef Pinon.
\VS{53}Le chef Kenaz, le chef Théman, le chef Mibtsar.
\VS{54}Le chef Magdiel, et le chef Iram. Ce sont là les chefs d'Edom.
\Chap{2}
\TextTitle{Les douze fils de Jacob (Israël)\FTNTT{Ge. 29:31-35 ; 30:6-24 ; 35:16-18}}
\VerseOne{}Voici les fils d'Israël : Ruben, Siméon, Lévi, Juda, Issacar, Zabulon,
\VS{2}Dan, Joseph, Benjamin, Nephthali, Gad et Aser.
\TextTitle{Les descendants de Juda jusqu’aux fils d’Hetsron\FTNTT{Ge. 46:12 ; No 26:19-22}}
\VS{3}Les fils de Juda furent  Er, Onan, et Schéla. Ces trois lui naquirent de la fille de Schua, la Cananéenne. Mais Er, premier-né de Juda, fut méchant aux yeux de Yahweh, qui le fit mourir.
\VS{4}Et Tamar, belle-fille de Juda, lui enfanta Pérets et Zérach. Tous les fils de Juda furent cinq.
\VS{5}Les fils de Pérets furent Hetsron et Hamul.
\VS{6}Et les fils de Zérach furent Zimri, Ethan, Héman, Calcol et Dara, cinq en tout.
\VS{7}Carmi n'eut point d’autre fils qu'Acar qui troubla Israël et qui pécha en prenant de l'interdit.
\VS{8}Ethan eut un seul fils : Azaria.
\VS{9}Les fils qui naquirent à Hetsron furent Jerachmeel, Ram et Kelubaï.
\TextTitle{Les descendants de Ram jusqu’à David\FTNTT{Ru. 4:17-22}}
\VS{10}Ram engendra Amminadab et Amminadab engendra Nachschon, chef des fils de Juda.
\VS{11}Nachschon engendra Salma et Salma engendra Boaz.
\VS{12}Boaz engendra Obed et Obed engendra Isaï.
\VS{13}Isaï engendra son premier-né Eliab, le second Abinadab, le troisième Schimea,
\VS{14}le quatrième Nethaneel, le cinquième Raddaï,
\VS{15}le sixième Otsem, et le septième, David.
\VS{16}Tseruja et Abigaïl furent leurs soeurs. Tseruja eut trois fils : Abischaï, Joab, et Asaël.
\VS{17}Abigaïl enfanta Amasa, dont le père fut Jéther l’Ismaélite.
\TextTitle{Les descendants de Caleb}
\VS{18}Or Caleb, fils de Hetsron, eut des enfants d’Azuba sa femme, et aussi de Jerioth; et ses fils furent Jéscher, Schobab et Ardon.
\VS{19}Azuba mourut, et Caleb prit pour femme Ephrath, qui lui enfanta Hur.
\VS{20}Hur engendra Uri, et Uri engendra Betsaleel.
\VS{21}Après cela, Hetsron vint vers la fille de Makir, père de Galaad, et la prit pour sa femme ; il était âgé de soixante ans, et elle lui enfanta Segub.
\VS{22}Segub engendra Jaïr, qui eut vingt-trois villes au pays de Galaad.
\VS{23}Il prit sur Gueschur et sur la Syrie les bourgades de Jaïr, et Kenath, avec les villes de son ressort, au nombre de soixante. Tous ceux-là furent fils de Makir, père de Galaad.
\VS{24}Après la mort de Hetsron, à Caleb-Ephratha, la femme de Hetsron, Abija, lui enfanta Aschchur, père de Tekoa.
\VS{25}Les fils de Jerachmeel, premier-né de Hetsron furent : Ram, son fils aîné, puis Buna, Oren et Otsem, nés d'Achija.
\VS{26}Jerachmeel eut aussi une autre femme, dont le nom était Athara, qui fut mère d'Onam.
\VS{27}Les fils de Ram, premier-né de Jerachmeel, furent Maats, Jamin et Eker.
\VS{28}Les fils  d'Onam furent  Schammaï et Jada; et les fils de Schammaï furent  Nadab et Abischur.
\VS{29}Le nom de la femme d'Abischur fut Abichaïl, qui lui enfanta Achban et Molid.
\VS{30}Les fils de Nadab furent Séled et Appaïm; mais Séled mourut sans fils.
\VS{31}Appaïm eut un seul fils : Jischeï. Jischeï eut un seul fils : Schéschan. Schéschan n'eut qu' Achlaï.
\VS{32}Les fils de Jada, frère de Schammaï, furent Jéther et Jonathan ; mais Jéther mourut sans fils.
\VS{33}Les fils de Jonathan furent Péleth et Zara ; ce furent là les fils de Jerachmeel.
\VS{34}Schéschan n'eut point de fils, mais des filles ; or il avait un serviteur Egyptien, dont le nom était Jarcha ;
\VS{35}Schéschan donna sa fille pour femme à Jarcha, son serviteur, et elle lui enfanta Attaï.
\VS{36}Attaï engendra Nathan, et Nathan engendra Zabad ;
\VS{37}Zabad engendra Ephlal; et Ephlal engendra Obed ;
\VS{38} Obed engendra Jéhu; Jéhu engendra Azaria;
\VS{39}Azaria engendra Halets;  Halets engendra Elasa;
\VS{40}Elasa engendra Sismaï; Sismaï engendra Schallum;
\VS{41}Schallum engendra Jekamja; Jekamja engendra Elischama.
\TextTitle{Les autres fils de Caleb}
\VS{42}Les fils de Caleb, frère de Jerachmeel, furent Méscha, son premier-né, qui fut le père de Ziph, et les fils de Maréscha, père d'Hébron.
\VS{43}Les fils d'Hébron furent  Koré, Thappuach, Rékem et Schéma.
\VS{44}Schéma engendra Racham, père de Jorkeam, et Rékem engendra Schammaï.
\VS{45}Le fils de Schammaï fut Maon. Maon fut père de Beth-Tsur.
\VS{46}Et Epha, concubine de Caleb, enfanta Haran, Motsa et Gazez; Haran aussi engendra Gazez.
\VS{47}Les fils de Jahdaï furent Réguem, Jotham, Guéschan, Péleth, Epha et Schaaph.
\VS{48}Maaca, la concubine de Caleb, enfanta Schéber et Tirchana.
\VS{49}La femme de Schaaph, père de Madmanna, enfanta Scheva, père de Macbéna, et le père de Guibea, et la fille de Caleb fut Acsa.
\TextTitle{Les descendants de Hur, fils de Caleb\FTNTT{v. 19 ; cp. 1 Ch. 4:1}}
\VS{50}Ceux-ci furent les fils de Caleb, fils de Hur, premier-né d'Ephrata : Schobal, père de Kirjath-Jearim.
\VS{51}Salma, père de Bethléhem, Hareph, père de Beth-Gader.
\VS{52}Schobal, père de Kirjath-Jearim, eut des fils : Haroé et Hatsi-Hammenuhoth.
\VS{53}Les familles de Kirjath-Jearim furent les Jéthriens, les Puthiens, les Schumathiens et les Mischraïens, desquels sont sortis les Tsoreathiens et les Eschthaoliens.
\VS{54}Les fils de Salma : Bethléhem et les Nethophatiens, Athroth-Beth-Joab, Hatsi-Hammanachthi et les Tsoreïens.
\VS{55}Et les familles des scribes, qui habitaient à Jaebets : Les Thireathiens, les Schimeathiens, les Sucathiens ; ce sont les Kéniens, qui sont sortis de Hamath père de Récab.
\Chap{3}
\TextTitle{Les fils de David\FTNTT{2 S. 3:2-5 ; 5:13-16}}
\VerseOne{}Voici les fils de David, qui lui naquirent à Hébron\FTNT{2 S. 3:2-5.}. Le premier-né fut Amnon, fils d' Achinoam de Jizreel; le second Daniel, d'Abigaïl de Carmel.
\VS{2}Le troisième, Absalom, fils de Maaca, fille de Talmaï, roi de Gueschur ; le quatrième, Adonija, fils de Haggith ;
\VS{3}le cinquième, Schephatia, d'Abithal; le sixième, Jithream, d'Egla sa femme.
\VS{4}Ces six lui naquirent à Hébron, où il régna sept ans et six mois ; puis il régna trente-trois ans à Jérusalem.
\VS{5}Ceux-ci lui naquirent à Jérusalem : Schimea, Schobab, Nathan et Salomon, tous quatre de Bath-Schua, fille d'Ammiel ;
\VS{6}et Jibhar, Elischama, Eliphéleth,
\VS{7}Noga, Népheg, Japhia,
\VS{8}Elischama, Eliada et Eliphéleth, qui sont neuf.
\VS{9}Ce sont tous des fils de David, outre les fils de ses concubines. Et Tamar était leur sœur.
\TextTitle{De Salomon à Sédécias}
\VS{10}Le fils de Salomon fut  Roboam. Abija, son fils ; Asa, son fils ; Josaphat, son fils ;
\VS{11}Joram, son fils ; Achazia, son fils ; Joas, son fils ;
\VS{12}Amatsia, son fils ; Azaria, son fils ; Jotham, son fils ;
\VS{13}Achaz, son fils ; Ezéchias, son fils ; Manassé, son fils ;
\VS{14}Amon, son fils ; Josias, son fils.
\VS{15}Les fils de Josias furent Jochanan, son premier-né ; le deuxième, Jojakim ; le troisième Sédécias ; le quatrième, Schallum.
\VS{16}Les fils de Jojakim furent  Jéconias, son fils, qui eut pour fils Sédécias.
\TextTitle{Les fils de Jéconias}
\VS{17}Quant aux fils de Jéconias, Assir qui fut emmené en captivité, Schealthiel fut son fils ;
\VS{18}dont les fils furent Malkiram, Pedaja, Schénatsar, Jekamia, Hoschama et Nedabia.
\VS{19}Les fils de Pedaja furent Zorobabel et Schimeï ; et les fils de Zorobabel furent Meschullam et Hanania ; et Schelomith était leur soeur.
\VS{20}De Meschullam, Haschuba, Ohel, Bérékia, Hasadia et Juschab-Hésed, en tout cinq.
\VS{21}Les fils de Hanania furent  Pelathia et Esaïe ; les fils de Rephaja, les fils d'Arnan, les fils d’Abdias et les fils de Schecania.
\VS{22}De Schecania naquit Schemaeja ; et les fils de Schemaeja, Hattusch, Jigueal, Bariach, Nearia, Schaphath, en tout six.
\VS{23}Les fils de Nearia furent trois : Eljoénaï, Ezéchias et Azrikam.
\VS{24}Et les fils d' Eljoénaï furent  sept : Hodavia, Eliaschib, Pelaja, Akkub, Jochanan, Delaja, et Anani.
\Chap{4}
\TextTitle{Les autres fils de Hur\FTNTT{1 Ch. 2:50}}
\VerseOne{}Les fils de Juda furent Pérets, Hetsron, Carmi, Hur et Schobal.
\VS{2}Reaja, fils de Schobal, engendra Jachath ; et Jachath engendra Achumaï et Lahad. Ce sont les familles des Tsoreathiens.
\VS{3}Voici les descendants du père d’Etham : Jizreel, Jischma, et Jidbasch ; le nom de leur soeur était Hatselelponi.
\VS{4}Penuel, père de Guedor, et Ezer, père de Huscha, sont les fils de Hur, premier-né d'Ephrata, père de Bethléhem.
\TextTitle{Les descendants d’Aschchur\FTNTT{1 Ch. 2:24}}
\VS{5}Aschchur, père de Tekoa, eut deux femmes : Hélea et Naara.
\VS{6}Naara lui enfanta Achuzzam, Hépher, Thémeni et Achaschthari. Ce sont là les fils de Naara.
\VS{7}Les fils de Hélea furent Tséreth, Tsochar et Ethnan.
\VS{8}Kots engendra Anub, Hatsobéba et les familles Acharchel, fils de Harum.
\TextTitle{Jaebets invoque Dieu}
\VS{9}Jaebets était plus honoré que ses frères ; sa mère lui avait donné le nom de Jaebets, parce que, dit-elle, je l'ai enfanté avec douleur.
\VS{10}Jaebets invoqua le Dieu d'Israël, en disant : Ô, si tu me bénis abondamment et que tu étends mes limites, si ta main est avec moi, et si tu me mets à l'abri du mal, en sorte que je ne sois pas dans l’affliction !... Et Dieu lui accorda ce qu'il avait demandé.
\TextTitle{Les fils de Juda et de Caleb}
\VS{11}Kelub, frère de Schucha, engendra Mechir, qui fut père d'Eschthon.
\VS{12}Et Eschthon engendra la maison de Rapha, Paséach et Thechinna, père de la ville de Nachasch ; ce sont là les gens de Réca.
\VS{13}Les fils de Kenaz furent Othniel et Seraja. Et le fils d’Othniel, Hathath.
\VS{14}Meonothaï engendra Ophra ; et Seraja engendra Joab, père de la vallée des ouvriers ; car ils étaient ouvriers.
\VS{15}Les fils de Caleb, fils de Jephunné, furent Iru, Ela et Naam, et les fils d'Ela, Kenaz.
\VS{16}Les fils de Jehalléleel furent Ziph, Zipha, Thirja, et Asareel.
\VS{17}Les fils d'Esdras furent Jéther, Méred, Epher, et Jalon ; et la femme de Méred enfanta Miriam, Schammaï, et Jischbach, père d' Eschthemoa.
\VS{18}Sa femme, la Juive, enfanta Jéred, père de Guedor ;  Héber, père de Soco ;  Jekuthiel, père de Zanoach. Ceux-là sont les fils de Bithja, fille de Pharaon, que Méred prit pour femme.
\VS{19}Les fils de la femme de Hodija, soeur de Nacham : Le père de Kehila, le Garmien, et Eschthemoa, le Maacathien.
\VS{20}Et les fils de Simon furent Amnon, Rinna, Ben-Hanan et Thilon. Les fils de Jischeï furent Zocheth et Ben-Zocheth.
\TextTitle{Les fils de Juda par Schéla\FTNTT{1 Ch. 2:3}}
\VS{21}Les fils de Schéla, fils de Juda, furent Er, père de Léca ; Laeda, père de Maréscha ; et les familles de la maison où l'on travaille le byssus, qui sont de la maison d'Aschbéa.
\VS{22}Jokim, et les gens de Cozéba, Joas et Saraph dominèrent sur Moab, avec Jaschubi-Léchem. Mais ce sont là des choses anciennes.
\VS{23}C’étaient les potiers et les habitants des plantations  et des parcs. Ils cohabitaient là chez le roi et oeuvraient pour lui.
\TextTitle{Les descendants de Siméon ; leurs terres et leurs conquêtes}
\VS{24}Les fils de Siméon furent Nemuel, Jamin, Jarib, Zérach et Saül.
\VS{25}Schallum son fils, Mibsam son fils, et Mischma son fils.
\VS{26}Les fils de Mischma furent Hammuel son fils, Zaccur son fils, et Schimeï son fils.
\VS{27}Schimeï eut seize fils et six filles ; mais ses frères n'eurent pas beaucoup de fils, et toute leur famille ne put être aussi nombreuse que celle des fils de Juda.
\VS{28}Ils habitèrent à Beer-Schéba, à Molada, à Hatsar-Schual,
\VS{29}à Bilha, à Etsem, à Tholad,
\VS{30}à Bethuel, à Horma, à Tsiklag,
\VS{31}à Beth-Marcaboth, à Hatsar-Susim, à Beth-Bireï, et à Schaaraïm. Ce furent là leurs villes jusqu'au temps où David devint roi.
\VS{32}Leurs villages furent Etham, Aïn, Rimmon, Thoken, et Aschan, cinq villes ;
\VS{33}et tous leurs villages, qui étaient autour de ces villes-là, jusqu'à Baal. Ce sont là leurs habitations et leur généalogie :
\VS{34}Meschobab, Jamlec, Joscha fils d'Amatsia ;
\VS{35}Joël, Jéhu fils de Joschibia, fils de Seraja, fils d'Asiel ;
\VS{36}Eljoénaï, Jaakoba, Jeschochaja, Asaja, Adiel, Jesimiel, Benaja,
\VS{37}Ziza, fils de Schipheï, fils d'Allon, fils de Jedaja, fils de Schimri, fils de Schemaeja.
\VS{38}Ceux-là furent désignés pour être des chefs dans leurs familles, et les maisons de leurs pères s’étendirent abondamment.
\VS{39}Et ils allèrent pour entrer dans Guedor, jusqu'à l'orient de la vallée, cherchant des pâturages pour leurs troupeaux.
\VS{40}Ils trouvèrent des pâturages gras et bons, et un pays spacieux, paisible et fertile ; car ceux qui habitaient là auparavant étaient descendus de Cham.
\VS{41}Ceux-ci, dont les noms sont inscrits, vinrent du temps d'Ezéchias, roi de Juda, et abattirent leurs tentes ; et quant aux Maonites qui s’y trouvaient, et les détruisirent à la façon de l'interdit jusqu'à ce jour, et y habitèrent à leur place, car il y avait là des pâturages pour leurs troupeaux.
\VS{42}Cinq cents hommes d'entre eux, c'est-à-dire des fils de Siméon, s'en allèrent à la montagne de Séir, et ils avaient à leur tête Pelathia, Nearia, Rephaja, et Uziel, fils de Jischeï ;
\VS{43}ils frappèrent le reste des réchappés d'Amalek, et ils demeurèrent là jusqu'à ce jour.
\Chap{5}
\TextTitle{Les descendants de Ruben jusqu’au temps des captivités}
\VerseOne{}Les fils de Ruben, le premier-né d'Israël. - Car il était le premier-né ; mais après qu'il eut souillé le lit de son père, son droit d'aînesse fut donné aux fils de Joseph fils d'Israël ; cependant, Joseph ne fut pas enregistré  dans la généalogie selon le droit d'aînesse.
\VS{2}Car Juda fut le plus puissant parmi ses frères, et de lui est issu un chef ; mais le droit d'aînesse est à Joseph.
\VS{3}Les fils de Ruben, premier-né d'Israël, furent donc Hénoc, Pallu, Hetsron, et Carmi.-
\VS{4}Les fils de Joël furent  Schemaeja, son fils ; Gog, son fils ; Schimeï, son fils ;
\VS{5}Michée, son fils ; Reaja, son fils ; Baal,  son fils ;
\VS{6}Beéra, son fils, qui fut emmené captif par Tilgath- Pilnéser, roi d’Assyrie ; c'est lui qui était le principal chef des Rubénites.
\VS{7}Ses frères, selon leurs familles, d’après le registre généalogique et selon leurs générations, avaient pour chefs Jeïel et Zacharie.
\VS{8}Béla, fils d’Azaz, fils de Schéma, fils de Joël, habitait depuis Aroër jusqu'à Nebo et Baal-Meon.
\VS{9}Ensuite, il habita du côté de l’orient jusqu'à l'entrée du désert, depuis le fleuve d'Euphrate; car son bétail s'était multiplié dans le pays de Galaad.
\VS{10}Du temps de Saül, ils firent la guerre contre les Hagaréniens, qui tombèrent par leurs mains, et ils habitèrent dans leurs tentes, dans toute la partie orientale de Galaad.
\TextTitle{Les descendants de Gad et leurs villes}
\VS{11}Les fils de Gad habitaient près d'eux, au pays de Basan, jusqu'à Salca.
\VS{12}Joël fut le premier chef, et Schapham le deuxième après lui, puis Jaenaï, puis Schaphath en Basan.
\VS{13}Et leurs frères, selon la maison de leurs pères, furent sept : Micaël, Meschullam, Schéba, Joraï, Jaecan, Zia, et Eber.
\VS{14}Ceux-ci furent les fils d'Abichaïl, fils de Huri, fils de Jaroach, fils de Galaad, fils de Micaël, fils de Jeschischaï, fils de Jachdo, fils de Buz.
\VS{15}Achi, fils d'Abdiel, fils de Guni, fut le chef de la maison de leurs pères.
\VS{16}Ils habitèrent en Galaad, et en Basan, dans les villes de son ressort, et dans toutes les banlieues de Saron, jusqu’à leurs limites.
\VS{17}Tous ceux-ci furent inscrits dans la généalogie du temps de Jotham, roi de Juda, et du temps de Jéroboam, roi d'Israël.
\TextTitle{Captivité de Ruben, Gad et la demi-tribu de Manassé}
\VS{18}Il y eut des fils de Ruben, et de ceux de Gad, et de la demi-tribu de Manassé, d'entre les vaillants hommes, portant le bouclier et l'épée, tirant de l'arc, et exercés à la guerre, quarante-quatre mille sept cent soixante, en état d’aller à l’armée.
\VS{19}Ils firent la guerre contre les Hagaréniens, contre Jethur, Naphisch, et Nodab.
\VS{20}Et ils reçurent du secours contre eux, de sorte que les Hagaréniens, et tous ceux qui étaient avec eux furent livrés entre leurs mains, parce qu'ils crièrent à Dieu dans la bataille, et il les exauça parce qu'ils avaient mis leur confiance en lui.
\VS{21}Ainsi ils prirent leurs troupeaux, consistant en cinquante mille chameaux, deux cent cinquante mille brebis, deux mille ânes, avec cent mille personnes ;
\VS{22}car il y eut beaucoup de morts, parce que la bataille venait de Dieu.  Ils habitèrent là, à leur place, jusqu'au temps de la déportation.
\VS{23}Les fils de la demi-tribu de Manassé habitèrent aussi dans ce pays-là, et s'étendirent depuis Basan jusqu'à Baal-Hermon et à Sénir, à la montagne d’Hermon ; ils étaient nombreux.
\VS{24}Et voici les chefs de la maison de leurs pères : Epher, Jischeï, Eliel, Azriel, Jérémie, Hodavia, et Jachdiel, hommes forts et vaillants, gens de réputation, et chefs des maisons de leurs pères.
\VS{25}Mais ils péchèrent contre le Dieu de leurs pères, et se prostituèrent après les dieux des peuples du pays, que Dieu avait détruits devant eux.
\VS{26}Le Dieu d'Israël excita l'esprit de Pul, roi d’Assyrie, et l'esprit de Thilgath-Pilnéser, roi d’Assyrie,  qui emmena en captivité les Rubénites, les Gadites et la demi-tribu de Manassé, et les emmena à Chalach, à Chabor, à Hara, et au fleuve de Gozan, où ils sont restés jusqu'à ce jour.
\Chap{6}
\TextTitle{Les fils de Kehath le Lévite, jusqu'à la captivité}
\VerseOne{}Les fils de Lévi furent Guerschon, Kehath et Merari.
\VS{2}Les fils de Kehath furent  Amram, Jitsehar, Hébron, et Uziel.
\VS{3}Et les fils d'Amram furent Aaron, Moïse et Marie. Les fils d'Aaron furent  Nadab, Abihu, Eléazar et Ithamar.
\VS{4}Eléazar engendra Phinées, et Phinées engendra Abischua.
\VS{5}Abischua engendra Bukki, et Bukki engendra Uzzi.
\VS{6}Uzzi engendra Zerachja, et Zerachja engendra Merajoth.
\VS{7}Merajoth engendra Amaria, et Amaria engendra Achithub.
\VS{8}Achithub engendra Tsadok, et Tsadok engendra Achimaats.
\VS{9}Achimaats engendra Azaria, et Azaria engendra Jochanan.
\VS{10}Jochanan engendra Azaria, qui exerça la sacrificature au temple que Salomon bâtit à Jérusalem.
\VS{11}Azaria engendra Amaria, et Amaria engendra Achithub.
\VS{12}Achithub engendra Tsadok, et Tsadok engendra Schallum.
\VS{13}Schallum engendra Hilkija, et Hilkija engendra Azaria.
\VS{14}Azaria engendra Seraja, et Seraja engendra Jehotsadak,
\VS{15}Jehotsadak s'en alla, quand Yahweh emmena en exil Juda et Jérusalem par le moyen de Nebucadnetsar.
\TextTitle{Les fils de Guerschon, Kehath et Mérari}
\VS{16}Les fils de Lévi  furent donc  Guerschon, Kehath et Merari.
\VS{17}Voici les noms des fils de Guerschon : Libni et Schimeï.
\VS{18}Les fils de Kehath furent Amram, Jitsehar, Hébron et Uziel.
\VS{19}Les fils de Merari furent  Machli et Muschi. Ce sont là les familles des Lévites, selon les maisons de leurs pères.
\VS{20}De Guerschon, Libni son fils, Jachath son fils, Zimma son fils,
\VS{21}Joach son fils, Iddo son fils, Zérach son fils, Jeathraï son fils.
\VS{22}Des fils de Kehath, Amminadab son fils, Koré son fils, Assir son fils,
\VS{23}Elkana son fils, Ebjasaph son fils, Assir son fils,
\VS{24}Thachath son fils, Uriel son fils, Ozias son fils, et Saül son fils.
\VS{25}Les fils d’Elkana furent  Amasaï, Achimoth ;
\VS{26}Elkana, son fils ; les fils d’Elkana furent Elkana-Tsophaï, son fils, Nachath son fils,
\VS{27}Eliab son fils, Jerocham son fils, Elkana son fils.
\VS{28}Quant aux fils de Samuel, fils d'Elkana, son fils aîné fut Vaschni, puis Abija.
\VS{29}Les fils de Merari furent Machli, Libni son fils, Schimeï son fils, Uzza son fils,
\VS{30}Schimea son fils, Hagguija son fils, Asaja son fils.
\TextTitle{Les chefs des chantres}
\VS{31}Or voici ceux que David établit pour la direction de la musique dans la maison de Yahweh, depuis que l’arche fut en lieu de repos.
\VS{32}Ils faisaient le service comme chantres devant le tabernacle, devant la tente d'assignation, jusqu'à ce que Salomon eût bâti la maison de Yahweh à Jérusalem ; ils continuèrent dans leur service selon l'ordonnance qui était prescrite. Voici ceux qui firent le service avec leurs fils : D'entre les fils des Kehathites, Héman le chantre, fils de Joël, fils de Samuel,
\VS{34}fils d'Elkana, fils de Jerocham, fils d’Eliel, fils de Thoach,
\VS{35}fils de Tsuph, fils d'Elkana, fils de Machath, fils de Amasaï,
\VS{36}fils d'Elkana, fils de Joël, fils d’Azaria, fils de Sophonie,
\VS{37}fils de Thachath, fils d'Assir, fils de Ebjasaph, fils de Koré,
\VS{38}fils de Jitsehar, fils de Kehath, fils de Lévi, fils d'Israël.
\VS{39}Son frère Asaph, qui se tenait à sa droite. Asaph était fils de Bérékia, fils de Schimea,
\VS{40}fils de Micaël, fils de Baaséja, fils de Malkija,
\VS{41}fils d’Ethni, fils de Zérach, fils d’Adaja,
\VS{42}fils d'Ethan, fils de Zimma, fils de Schimeï,
\VS{43}fils de Jachath, fils de Guerschon, fils de Lévi.
\VS{44}Les fils de Merari, leurs frères étaient à la gauche ; à savoir Ethan, fils de Kischi, fils d’Abdi, fils de Malluc,
\VS{45}fils de Haschabia, fils d'Amatsia, fils de Hilkija,
\VS{46}fils d'Amtsi, fils de Bani, fils de Schémer,
\VS{47}fils de Machli, fils de Muschi, fils de Merari, fils de Lévi.
\VS{48}Et leurs autres frères Lévites furent ordonnés pour tout le service du tabernacle de la maison de Dieu.
\VS{49}Mais Aaron et ses fils offraient les parfums sur l'autel de l'holocauste et sur l'autel des parfums ; pour tout ce qu'il fallait faire dans le Saint des saints, et pour faire propitiation pour Israël ; comme Moïse, serviteur de Dieu, l'avait commandé.
\TextTitle{Les sacrificateurs d’Aaron à Achimaats}
\VS{50}Voici les fils d'Aaron : Eléazar son fils, Phinées son fils, Abischua son fils,
\VS{51}Bukki son fils, Uzzi son fils, Zerachja son fils,
\VS{52}Merajoth son fils, Amaria son fils, Achithub son fils,
\VS{53}Tsadok son fils, Achimaats son fils.
\TextTitle{Villes des fils d’Aaron et des Lévites}
\VS{54}Voici leurs lieux d’habitation, selon leurs demeures et leurs limites. Aux fils d'Aaron, qui appartiennent à la famille des Kehathites, désignés par le sort,
\VS{55}on leur donna Hébron dans le pays de Juda, et sa banlieue tout autour.
\VS{56}Mais on donna à Caleb, fils de Jephunné, le territoire de la ville et ses villages.
\VS{57}On donna donc aux fils d'Aaron, d'entre les villes de refuge, Hébron, Libna et sa banlieue, Jatthir et Eschthemoa, avec leurs banlieues,
\VS{58}Hilen, avec sa banlieue, Debir avec sa banlieue,
\VS{59}Aschan avec sa banlieue, et Beth-Schémesch avec sa banlieue.
\VS{60}De la tribu de Benjamin, Guéba, avec sa banlieue, Allémeth avec sa banlieue, et Anathoth avec sa banlieue. Toutes leurs villes, selon leurs familles, étaient treize en nombre.
\VS{61}On donna au reste des fils de Kehath, par le sort, dix villes des familles de la demi-tribu, c'est-à-dire de la demi-tribu de Manassé.
\VS{62}Et aux fils de Guerschon, selon leurs familles, de la tribu d'Issacar, de la tribu d'Aser, de la tribu de Nephthali, et de la tribu de Manassé en Basan, treize villes.
\VS{63}Aux fils de Merari, selon leurs familles, par le sort, douze villes, de la tribu de Ruben, de la tribu de Gad, et de la tribu de Zabulon.
\VS{64}Ainsi, les fils d’Israël donnèrent aux Lévites ces villes-là, avec leurs banlieues.
\VS{65}Et ils donnèrent, par le sort, de la tribu des fils de Juda, de la tribu des fils de Siméon, et de la tribu des fils de Benjamin, ces villes qu’ils désignèrent par leurs noms.
\VS{66}Et pour les autres familles des fils de Kehath, ils eurent pour territoire des villes de la tribu d'Ephraïm.
\VS{67}Car on leur donna entre les villes de refuge, Sichem avec sa banlieue, dans la montagne d'Ephraïm, Guézer avec sa banlieue,
\VS{68}Jokmeam avec sa banlieue, Beth-Horon avec sa banlieue,
\VS{69}Ajalon avec sa banlieue, et Gath-Rimmon avec sa banlieue.
\VS{70}De la demi-tribu de Manassé, Aner avec sa banlieue, et Bileam avec sa banlieue, on donna ces villes-là aux familles qui restaient des fils de Kehath.
\VS{71}Aux fils de Guerschon, on donna, des familles de la demi-tribu de Manassé, Golan en Basan avec sa banlieue, et Aschtaroth, avec sa banlieue.
\VS{72}De la tribu d'Issacar, Kédesch avec sa banlieue, Dobrath avec sa banlieue,
\VS{73}Ramoth avec sa banlieue, et Anem avec sa banlieue.
\VS{74}Et de la tribu d'Aser, Maschal, avec sa banlieue, Abdon, avec sa banlieue,
\VS{75}Hukok avec sa banlieue, et Rehob avec sa banlieue.
\VS{76}De la tribu de Nephthali, Kédesch en Galilée avec sa banlieue, Hammon avec sa banlieue, et Kirjathaïm avec sa banlieue.
\VS{77}Aux fils de Merari, qui étaient le reste d'entre les Lévites, on donna, de la tribu de Zabulon, Rimmono avec sa banlieue, et Thabor avec sa banlieue.
\VS{78}Au-delà du Jourdain, vis-à-vis de Jéricho, vers l’orient du Jourdain, de la tribu de Ruben, Betser au désert avec sa banlieue, Jahtsa avec sa banlieue,
\VS{79}Kedémoth avec sa banlieue, et Méphaath avec sa banlieue.
\VS{80}De la tribu de Gad, Ramoth en Galaad avec sa banlieue, Mahanaïm avec sa banlieue,
\VS{81}Hesbon avec sa banlieue, et Jaezer avec sa banlieue.
\Chap{7}
\TextTitle{Les descendants d'Issacar}
\VerseOne{}Les fils d'Issacar furent  Thola, Pua, Jaschub et Schimron, quatre.
\VS{2}Les fils de Thola furent Uzzi, Rephaja, Jeriel, Jachmaï, Jibsam et Samuel, chefs des maisons de leurs pères qui étaient de Thola, gens forts et vaillants dans leurs générations ; leur nombre, aux jours de David, était de vingt-deux mille six cents.
\VS{3}Le fils d’Uzzi : Jizrachja. Et les fils de Jizrachja : Micaël, Abdias, Joël, et Jischija, en tout cinq chefs.
\VS{4}Ils avaient avec eux, selon leurs générations, et selon les familles de leurs pères, trente-six mille hommes de troupe, armés pour la guerre, car ils eurent plusieurs femmes et plusieurs fils.
\VS{5}Leurs frères selon toutes les familles d'Issacar, hommes forts et vaillants, étant comptés tous selon leur généalogie, furent quatre-vingt-sept mille.
\TextTitle{Les descendants de Benjamin}
\VS{6}Les fils de Benjamin furent Béla, Béker et Jediaël, trois\FTNT{Benjamin avait encore d’autres fils (Ge. 46:21 ; No. 26:38-41 ; 1 Ch. 8:1-2).}.
\VS{7}Les fils de Béla furent  Etsbon, Uzzi, Uziel, Jerimoth et Iri, cinq chefs des familles de leurs pères, hommes forts et vaillants, et enregistrés dans la généalogie au nombre de vingt-deux mille trente-quatre.
\VS{8}Les fils de Béker furent  Zemira, Joasch, Eliézer, Eljoénaï, Omri, Jerémoth, Abija, Anathoth, et Alameth, tous ceux-là furent fils de Béker,
\VS{9}et enregistrés dans les généalogies, selon leurs générations, comme chefs des familles de leurs pères, hommes forts et vaillants au nombre de vingt mille deux cents.
\VS{10}Jediaël eut pour fils Bilhan. Et les fils de Bilhan furent Jeusch, Benjamin, Ehud, Kenaana, Zéthan, Tarsis, et Achischachar.
\VS{11}Tous ceux-là furent fils de Jediaël, comme chefs des familles de leurs pères, dix-sept mille deux cents hommes forts et vaillants, en état de porter les armes et d’aller à la guerre.
\VS{12}Schuppim et Huppim furent  des fils d’Ir ; et Huschim fut fils d'Acher.
\TextTitle{Les descendants de Nephtali}
\VS{13}Les fils de Nephthali furent Jahtsiel, Guni, Jetser, et Schallum, fils de Bilha.
\TextTitle{Les descendants de Manassé}
\VS{14}Les fils de Manassé : Asriel, qu’enfanta sa concubine Araméenne. Elle enfanta Makir, père de Galaad.
\VS{15}Makir prit une femme de la parenté de Huppim et de Schuppim ; car ils avaient une sœur dont le nom était Maaca. Et le nom d'un des petits-fils de Galaad fut Tselophchad ; et Tselophchad eut des filles.
\VS{16}Maaca, femme de Makir, enfanta un fils et l'appela Péresch, et le nom de son frère Schéresch, dont les fils furent Ulam et Rékem.
\VS{17}Le fils d'Ulam fut Bedan. Ce sont là les fils de Galaad, fils de Makir, fils de Manassé.
\VS{18}Mais sa soeur Hammoléketh enfanta Ischhod, Abiézer et Machla.
\VS{19}Les fils de Schemida furent Achjan, Sichem, Likchi et Aniam.
\TextTitle{Les descendants d'Ephraïm et leurs villes}
\VS{20}Or les fils d'Ephraïm furent  Schutélach ; Béred son fils, Tachath son fils, Eleada son fils, Tachath son fils.
\VS{21}Zabad son fils, Schutélach son fils, Ezer, et Elead. Mais ceux de Gath, nés dans le pays, les mirent à mort, parce qu'ils étaient descendus pour prendre leur bétail.
\VS{22}Ephraïm, leur père, fut dans le deuil plusieurs jours, et ses frères vinrent pour le consoler.
\VS{23}Puis il alla vers sa femme, qui conçut et enfanta un fils ; et elle l'appela du nom de Beria, parce que le malheur était dans sa maison.
\VS{24}Il eut pour fille Schééra, qui bâtit la basse et la haute Beth-Horon, et Uzzen-Schééra.
\VS{25}Son fils fut  Réphach, puis Réscheph, et Thélach son fils, Thachan son fils,
\VS{26}Laedan son fils, Ammihud son fils, Elischama son fils,
\VS{27}Nun son fils, Josué son fils.
\VS{28}Ils possédaient et habitaient Béthel ainsi que les villes de son ressort ; à l’orient Naaran, à l’occident Guézer, avec les villes de son ressort, et Sichem avec les villes de son ressort, jusqu'à Gaza avec les villes de son ressort.
\VS{29}Les lieux qui étaient aux fils de Manassé furent  Beth-Schean avec les villes de son ressort, Thaanac avec les villes de son ressort, Meguiddo avec les villes de son ressort, et Dor avec les villes de son ressort. Les fils de Joseph, fils d'Israël, habitèrent dans ces villes.
\TextTitle{Les descendants d'Aser}
\VS{30}Les fils d’Aser furent Jimna, Jischva, Jischvi, Beria, et Sérach leur soeur.
\VS{31}Les fils de Beria furent Héber et Malkiel, qui fut père de Birzavith.
\VS{32}Héber engendra Japhleth, Schomer, Hotham, et Schua leur soeur.
\VS{33}Les fils de Japhleth furent Pasac, Bimhal, et Aschvath. Ce sont là les fils de Japhlet.
\VS{34}Et les fils de Schamer furent Achi, Rohega, Hubba et Aram.
\VS{35}Les fils d'Hélem, son frère, furent  Tsophach, Jimna, Schélesch et Amal.
\VS{36}Les fils de Tsophach furent Suach, Harnépher, Schual, Béri, Jimra,
\VS{37}Betser, Hod, Schamma, Schilscha, Jithran, et Beéra.
\VS{38}Les fils de Jéther furent Jephunné, Pispa et Ara.
\VS{39}Les fils d'Ulla furent Arach, Hanniel et Ritsja.
\VS{40}Tous ceux-là furent fils d'Aser, chefs des maisons de leurs pères, gens d'élite, forts et vaillants, chefs des princes, enregistrés au nombre de vingt-six mille hommes, en état de porter les armes et d’aller en guerre.
\Chap{8}
\TextTitle{Les descendants de Benjamin}
\VerseOne{}Benjamin engendra Béla, qui fut son premier-né, Aschbel le deuxième, Achrach le troisième,
\VS{2}Nocha le quatrième, et Rapha le cinquième.
\VS{3}Les fils de Béla furent Addar, Guéra, Abihud,
\VS{4}Abischua, Naaman, Achoach,
\VS{5}Guéra, Schephuphan et Huram.
\VS{6}Voici les fils d'Echud, qui étaient chefs des maisons des pères des habitants de Guéba, et qui les transportèrent à Manachath :
\VS{7}Naaman, Achija, et Guéra. Guéra, qui les transporta et qui après engendra Uzza et Achichud.
\VS{8}Or Schacharaïm eut des enfants au pays de Moab, après avoir renvoyé Huschim et Baara, ses femmes.
\VS{9}Il engendra, de Hodesch sa femme, Jobab, Tsibja, Méscha, Malcam,
\VS{10}Jeuts, Schocja et Mirma. Ce sont là ses fils, chefs des pères.
\VS{11}Mais de Huschim, il engendra Abithub, Elpaal.
\VS{12}Les fils d'Elpaal furent Eber, Mischeam, et Schémer, qui bâtit Ono, Lod et les villes de son ressort.
\VS{13}Et Beria et Schéma furent chefs des pères des habitants d'Ajalon ; ils mirent en fuite les habitants de Gath.
\VS{14}Achjo, Schaschak, Jerémoth,
\VS{15}Zebadja, Arad, Eder,
\VS{16}Micaël, Jischpha, et Jocha, fils de Beria.
\VS{17}Zebadja, Meschullam, Hizki, Héber,
\VS{18}Jischmeraï, Jizlia, et Jobab, fils d' Elpaal.
\VS{19}Jakim, Zicri, Zabdi,
\VS{20}Eliénaï, Tsilthaï, Eliel,
\VS{21}Adaja, Beraja, et Schimrath, fils de Schimeï.
\VS{22}Jischpan, Eber, Eliel,
\VS{23}Abdon, Zicri, Hanan,
\VS{24}Hanania, Elam, Anthothija,
\VS{25}Jiphdeja et Penuel, fils de Schaschak.
\VS{26}Schamscheraï, Schecharia, Athalia,
\VS{27}Jaaréschia, Elija, et Zicri, fils de Jerocham.
\VS{28}Ce sont là les chefs des pères, selon leurs générations ; et ils habitèrent à Jérusalem.
\TextTitle{Les fils du père de Gabaon, ascendant de Saül}
\VS{29}Le père de Gabaon habita à Gabaon, sa femme avait pour nom Maaca.
\VS{30}Son fils premier-né fut Abdon, puis Tsur, Kis, Baal, Nadab,
\VS{31}Guedor, Achjo, et Zéker.
\VS{32}Mikloth engendra Schimea. Ils habitèrent aussi vis-à-vis de leurs frères à Jérusalem, avec leurs frères.
\VS{33}Ner engendra Kis, et Kis engendra Saül, et Saül engendra Jonathan, Malki-Schua, Abinadab, et Eschbaal.
\VS{34}Le fils de Jonathan fut  Merib-Baal ; et Merib-Baal engendra Michée.
\VS{35}Les fils de Michée furent Pithon, Mélec, Thaeréa, et Achaz.
\VS{36}Achaz engendra Jehoadda ; et Jehoadda engendra Alémeth, Azmaveth et Zimri ; Zimri engendra Motsa.
\VS{37}Motsa engendra Binea, qui eut pour fils Rapha, qui eut pour fils Eleasa, qui eut pour fils Atsel.
\VS{38}Atsel eut six fils, dont les noms sont : Azrikam, Bocru, Ismaël, Schearia, Abdias, et Hanan ; tous ceux-là furent fils d'Atsel.
\VS{39}Les fils d'Eschek, son frère, furent  Ulam son premier-né, Jéusch le second, Eliphéleth le troisième.
\VS{40}Et les fils d'Ulam furent des hommes forts et vaillants, tirant bien de l'arc, et ils eurent beaucoup de fils et de petits-fils, jusqu'à cent cinquante ; tous des fils de Benjamin.
\Chap{9}
\TextTitle{Les habitants de Jérusalem}
\VerseOne{}Ainsi, tous ceux d'Israël furent enregistrés par généalogie et inscrits dans le livre des rois d'Israël. Et ceux de Juda furent emmenés en captivité à Babylone à cause de leurs péchés\FTNT{La captivité babylonienne voir 2 R. 24 et 25.}.
\VS{2}Mais ce sont ici les premiers qui habitèrent dans leurs possessions, et dans leurs villes, tant d'Israël que des sacrificateurs, des Lévites, et des Néthiniens.
\VS{3}A Jérusalem habitaient les fils de Juda, les fils de Benjamin, et les fils d'Ephraïm et de Manassé.
\VS{4}Uthaï, fils d'Ammihud, fils d'Omri, fils d'Imri, fils de Bani, des fils de Pérets, fils de Juda.
\VS{5}Des Schilonites, Asaja le premier-né, et ses fils.
\VS{6}Des fils de Zérach, Jeuel, et ses frères, six cent quatre-vingt-dix.
\VS{7}Des fils de Benjamin, Sallu fils de Meschullam, fils de Hodavia, fils d'Assenua.
\VS{8}Jibneja, fils de Jerocham, et Ela fils d’Uzzi, fils de Micri ; et Meschullam fils de Schephathia, fils de Reuel, fils de Jibnija.
\VS{9}Leurs frères, selon leurs générations, furent neuf cent cinquante-six. Tous ces hommes-là furent chefs des pères dans les maisons de leurs  pères.
\VS{10}Des sacrificateurs : Jedaeja, Jehojarib, et Jakin.
\VS{11}Azaria fils de Hilkija, fils de Meschullam, fils de Tsadok, fils de Merajoth, fils d' Achithub, intendant de la maison de Dieu.
\VS{12}Adaja, fils de Jerocham, fils de Paschhur, fils de Malkija; et Maesaï, fils d'Adiel, fils de Jachzéra, fils de Meschullam, fils de Meschillémith, fils d'Immer.
\VS{13}Leurs frères, chefs de la maison de leurs pères, mille sept cent soixante hommes, forts et vaillants, occupés au service de la maison de Dieu.
\VS{14}Des Lévites : Schemaeja, fils de Haschub, fils d'Azrikam, fils de Haschabia, des fils de Merari,
\VS{15}Bakbakkar, Héresch, et Galal ; et Matthania, fils de Michée, fils de Zicri, fils d'Asaph,
\VS{16}Abdias fils de Schemaeja, fils de Galal, fils de Jeduthun ; et Bérékia, fils d'Asa, fils d'Elkana, qui habita dans les villages des Nethophathiens.
\VS{17}Et les portiers : Schallum, Akkub, Thalmon, et Achiman, et leurs frères ; mais Schallum était le chef.
\VS{18}Il l'a été jusqu'à maintenant, ayant la charge de la porte du roi vers l’orient. Ceux-là furent portiers pour le camp des fils de Lévi.
\VS{19}Schallum, fils de Koré, fils d'Ebiasaph, fils de Koré, et ses frères Koréites, de la maison de son père, remplissaient les fonctions de gardiens, gardant les seuils de la tente, comme leurs pères en avaient gardé l'entrée au camp de Yahweh ;
\VS{20}Phinées, fils d'Eléazar, fut établi chef sur eux en présence de Yahweh qui était avec lui.
\VS{21}Zacharie, fils de Meschélémia, était le portier de l'entrée de la tente d'assignation.
\VS{22}Ils étaient en tout deux cent douze, choisis pour être les portiers des seuils, et enregistrés selon les familles dans la généalogie, selon leurs villages ; David et Samuel, le voyant, les avaient établis dans leurs fonctions.
\VS{23}Eux, dis-je, et leurs fils furent établis sur les portes de la maison de Yahweh, qui est la maison du tabernacle, pour y faire la garde.
\VS{24}Il y avait des portiers aux quatre vents, à l'orient, à l'occident, au nord et au midi.
\VS{25}Et leurs frères, qui étaient dans leurs villages, devaient de temps à autre venir auprès d’eux pendant sept jours.
\VS{26}Car selon cette fonction, il y avait toujours quatre chefs des portiers, des Lévites, qui avaient la surveillance des chambres et des trésors de la maison de Dieu.
\VS{27}Ils se tenaient la nuit tout autour de la maison de Dieu, dont ils avaient la garde, et qu’ils devaient ouvrir tous les matins.
\VS{28}Certains d’entre eux prenaient soin des ustensiles du service; car on en faisait le compte lorsqu'on les rentrait et qu'on les sortait.
\VS{29}D’autres veillaient sur les ustensiles, sur tous les ustensiles du sanctuaire, sur la fleur de farine, sur le vin, sur l'huile, sur l'encens et sur les aromates.
\VS{30}Mais ceux qui composaient les parfums aromatiques étaient des fils de sacrificateurs.
\VS{31}Matthithia, d'entre les Lévites, premier-né de Schallum, Koréite, s’occupait des gâteaux cuits sur les plaques.
\VS{32}Et quelques-uns de leurs frères, parmi les fils des Kehathites, avaient la charge du pain de proposition\FTNT{Il y avait douze gâteaux de pain qu’on plaçait sur une table dans le tabernacle ou dans le temple et qu’on remplaçait chaque sabbat (Ex. 35:13 ;  Ex. 39:36 ; 1 R. 7:48 ; 2 Ch. 13:11 ; Né. 10:32-33). En hébreu, le pain de proposition signifie littéralement le « pain de la face ». Le mot rendu par « face » se rapporte parfois à la « présence » (2 R. 13:23). Le pain de proposition est en réalité l’image du Seigneur Jésus-Christ, notre pain de vie (Jn. 6:48-59).}  pour l'apprêter chaque sabbat.
\VS{33}Certains étaient des chantres, chefs des pères des Lévites, qui demeuraient dans les chambres, sans avoir d’autres charges, parce qu'ils devaient être en fonction le jour et la nuit.
\VS{34}Ce sont là les chefs des pères des Lévites, selon leurs familles ; ils furent chefs, et ils habitèrent à Jérusalem.
\TextTitle{De Jeïel au roi Saül, de Jonathan à Arsel\FTNTT{1 Ch. 10 ; 1 S. 1 ; 30}}
\VS{35}Or Jeïel, le père de Gabaon, habita à Gabaon ; et le nom de sa femme était Maaca.
\VS{36}Son fils premier-né, Abdon, puis Tsur, Kis, Baal, Ner, Nadab,
\VS{37}Guedor, Achjo, Zacharie, et Mikloth.
\VS{38}Mikloth engendra Schimeam ; et ils habitèrent vis-à-vis de leurs frères à Jérusalem, avec leurs frères.
\VS{39}Ner engendra Kis, et Kis engendra Saül, et Saül engendra Jonathan, Malki-Schua, Abinadab et Eschbaal.
\VS{40}Le fils de Jonathan fut  Merib-Baal; et Merib-Baal engendra Michée.
\VS{41}Et les fils de Michée furent Pithon, Mélec, Thachréa et Achaz.
\VS{42}Achaz engendra Jaera ; et Jaera engendra Alémeth, Azmaveth, et Zimri ; et Zimri engendra Motsa.
\VS{43}Motsa engendra Binea, qui eut pour fils Rephaja, qui eut pour fils Eleasa, qui eut pour fils Atsel.
\VS{44}Atsel eut six fils, dont les noms sont Azrikam, Bocru, Ismaël, Schearia, Abdias et Hanan. Ce furent là les fils d'Atsel.
\Chap{10}
\TextTitle{Mort de Saül\FTNTT{1 S. 31:1-10 ; 2 S. 1}}
\VerseOne{}Les Philistins combattirent contre Israël, et les hommes d'Israël s'enfuirent devant les Philistins, et tombèrent blessés à mort sur la montagne de Guilboa\FTNT{1 S. 31:1-10}.
\VS{2}Les Philistins poursuivirent et atteignirent Saül et ses fils, et tuèrent Jonathan, Abinadab et Malki-Schua, les fils de Saül.
\VS{3}L’effort du combat se porta sur Saül ; de sorte que les archers l'atteignirent, et il eut peur de ces archers.
\VS{4}Alors Saül dit à celui qui portait ses armes : Tire ton épée, et transperce-moi, de peur que ces incirconcis ne viennent et ne fassent de moi selon leur volonté ; mais celui qui portait ses armes ne voulut pas, parce qu’il avait très peur. Saül prit donc son épée, et se jeta dessus.
\VS{5}Alors celui qui portait les armes de Saül, ayant vu que Saül était mort, se jeta aussi sur son épée, et il mourut.
\VS{6}Ainsi mourut Saül, et ses trois fils, et toute sa maison périt avec lui.
\VS{7}Tous ceux d'Israël, qui étaient dans la vallée, ayant vu qu'on avait fui, et que Saül et ses fils étaient morts, abandonnèrent leurs villes et s'enfuirent, de sorte que les Philistins y entrèrent et y habitèrent.
\VS{8}Or il arriva que dès le lendemain, les Philistins vinrent pour dépouiller les morts, et ils trouvèrent Saül et ses fils étendus sur la montagne de Guilboa.
\VS{9}Ils le dépouillèrent et emportèrent sa tête et ses armes. Puis ils firent annoncer ces bonnes nouvelles par tout le pays des Philistins, et aux environs, pour en faire savoir les nouvelles à leurs dieux et au peuple.
\VS{10}Ils mirent ses armes dans la maison de leur dieu, et ils attachèrent sa tête dans la maison de Dagon\FTNT{1 S. 5:1-11.}.
\VS{11}Tous ceux de Jabès de Galaad, ayant appris tout ce que les Philistins avaient fait à Saül,
\VS{12}tous les vaillants hommes d’entre eux se levèrent et enlevèrent le corps de Saül et les corps de ses fils ; ils les apportèrent à Jabès, et ils ensevelirent leurs os sous un chêne à Jabès, et jeûnèrent pendant sept jours.
\VS{13}Saül mourut pour le crime qu'il avait commis contre Yahweh, en ce qu'il n'avait point gardé la parole de Yahweh, et qu'il avait même consulté ceux qui évoquent les morts\FTNT{1 S. 28:7-20.} pour savoir ce qui devait lui arriver.
\VS{14}Il ne consulta point Yahweh ; c'est pourquoi Yahweh le fit mourir, et transféra la royauté à David, fils d'Isaï.
\Chap{11}
\TextTitle{David règne sur Israël\FTNTT{2 S. 5:1-3 ; 2 S. 2-4}}
\VerseOne{}Tous ceux d'Israël s'assemblèrent auprès de David à Hébron, et lui dirent : Voici, nous sommes tes os et ta chair.
\VS{2}Autrefois déjà, quand Saül était roi, tu étais celui qui faisais sortir et qui ramenais Israël. Yahweh, ton Dieu, t'a dit : Tu paîtras mon peuple d'Israël, et tu seras le chef de mon peuple d'Israël.
\VS{3}Ainsi, tous les anciens d'Israël vinrent auprès du roi à Hébron ; et David traita alliance avec eux à Hébron, devant Yahweh. Ils oignirent David pour roi sur Israël, selon la parole de Yahweh, prononcée par Samuel\FTNT{2 S. 2, 3, 4 ; 2 S. 5:1-3.}.
\TextTitle{Jérusalem devient la cité de David\FTNTT{2 S. 5:6-10}}
\VS{4}David et tous ceux d'Israël s'en allèrent à Jérusalem, qui est Jébus. Là étaient  les Jébusiens qui habitaient le pays.
\VS{5}Ceux qui habitaient à Jébus dirent à David : Tu n'entreras point ici. Mais David prit la forteresse de Sion, qui est la cité de David.
\VS{6}Car David avait dit: Quiconque battra le premier les Jébusiens sera chef et prince. Joab, fils de Tseruja, monta le premier, et fut fait chef.
\VS{7}David s’établit dans la forteresse ; c'est pourquoi on l'appela la cité de David\FTNT{2 S. 5:6-10.}.
\VS{8}Il bâtit aussi la ville tout autour, depuis Millo et ses environs ; et Joab répara le reste de la ville.
\VS{9}David devenait de plus en plus grand, car Yahweh des armées était avec lui.
\TextTitle{Les vaillants hommes de David\FTNTT{2 S. 23:8-39}}
\VS{10}Voici les chefs des hommes vaillants qui étaient au service de David, qui l’aidèrent avec tout Israël à assurer sa royauté, afin de le faire régner selon la parole de Yahweh au sujet d’Israël.
\VS{11}Ceux-ci sont du nombre des vaillants hommes que David avait. Jaschobeam, fils de Hacmoni, chef entre les trois principaux. Il brandit sa lance contre trois cents hommes et les blessa à mort en une seule fois\FTNT{2 S. 23:8-39. }.
\VS{12}Après lui était Eléazar, fils de Dodo, l'Achochite, qui fut l’un des trois vaillants hommes.
\VS{13}Il se trouvait avec David à Pas-Dammim, lorsque les Philistins s'étaient assemblés pour combattre. Il y avait là une parcelle de terre remplie d'orge ; et le peuple fuyait devant les Philistins.
\VS{14}Ils s'arrêtèrent au milieu de cette parcelle de champ, la défendirent, et battirent les Philistins. Ainsi, Yahweh accorda une grande délivrance.
\VS{15}Il en descendit encore trois des trente chefs près du rocher, auprès de David, dans la caverne d'Adullam, lorsque l'armée des Philistins campait dans la vallée des Rephaïm.
\VS{16}David était alors dans la forteresse, et la garnison des Philistins était en ce même temps-là à Bethléhem.
\VS{17}David eut un désir, et dit : Qui est-ce qui me fera boire de l'eau du puits qui est à la porte de Bethléhem ?
\VS{18}Alors ces trois hommes passèrent au travers du camp des Philistins, et puisèrent de l'eau du puits qui était à la porte de Bethléhem ; et l'ayant apportée, la présentèrent à David, qui ne voulut point la boire, mais la répandit en l'honneur de Yahweh.
\VS{19}Car il dit : Que mon Dieu me garde de faire une telle chose ! Boirais-je le sang de ces hommes qui ont fait un tel voyage au péril de leur vie ? Car ils m'ont apporté cette eau au péril de leur vie. Ainsi, il ne voulut point la boire. Voilà ce que firent ces trois vaillants hommes.
\VS{20}Abischaï, frère de Joab, était chef des trois. Il sortit sa lance sur trois cents hommes, les blessa à mort ; et il eut du renom entre les trois.
\VS{21}Entre les trois, il fut plus honoré que les deux autres, et il fut leur chef ; cependant, il n'égala point ces trois premiers.
\VS{22}Benaja aussi, fils de Jehojada, fils d'un vaillant homme de Kabtseel, avait fait de grands exploits. Il tua deux des plus puissants hommes de Moab. Il descendit et frappa un lion au milieu d'une fosse en un jour de neige.
\VS{23}Il tua aussi un homme Egyptien qui était haut de cinq coudées. Cet Egyptien avait à la main une lance grosse comme une ensouple de tisserand ; mais il descendit contre lui avec un bâton, et arracha la lance de la main de l'Egyptien, et le tua avec sa propre lance.
\VS{24}Benaja, fils de Jehojada, fit ces choses-là, et fut célèbre entre ces trois vaillants hommes.
\VS{25}Voilà, il était le plus honoré des trente ; cependant, il n'égala point les trois premiers. David l'établit dans son conseil privé.
\VS{26}Et les plus vaillants d'entre les gens de guerre furent Asaël, frère de Joab ; et Elchanan fils de Dodo, de Bethléhem,
\VS{27}Schammoth d'Haror, Hélets de Palon,
\VS{28}Ira, fils d'Ikkesch, de Tekoa, Abiézer d'Anathoth,
\VS{29}Sibbecaï le Huschatite, Ilaï d'Achoach,
\VS{30}Maharaï de Nethopha, Héled fils de Baana de Nethopha,
\VS{31}Ithaï fils de Ribaï, de Guibea des fils de Benjamin, Benaja de Pirathon,
\VS{32}Huraï de Nachalé-Gaasch, Abiel d'Araba,
\VS{33}Azmaveth de Bacharum, Eliachba de Schaalbon,
\VS{34}Bené-Haschem de Guizon, Jonathan fils de Schagué d'Harar,
\VS{35}Achiam fils de Sacar d'Harar, Eliphal fils d'Ur,
\VS{36}Hépher de Mekéra, Achija de Palon,
\VS{37}Hetsro de Carmel, Naaraï fils d'Ezbaï,
\VS{38}Joël frère de Nathan, Mibchar fils d'Hagri,
\VS{39}Tsélek l'Ammonite, Nachraï de Béroth, qui portait les armes de Joab fils de Tseruja,
\VS{40}Ira de Jéther, Gareb de Jéther,
\VS{41}Urie le Héthien, Zabad fils d' Achlaï,
\VS{42}Adina fils de Schiza le Rubénite, chef des Rubénites, et trente avec lui.
\VS{43}Hanan fils de Maaca, et Josaphat de Mithni,
\VS{44}Ozias d'Aschtharoth, Schama et Jehiel fils de Hotham d'Aroër,
\VS{45}Jediaël fils de Schimri, et Jocha son frère, le Thitsite,
\VS{46}Eliel de Machavim, Jeribaï, et Joschavia fils d'Elnaam, et Jithma le Moabite,
\VS{47}Eliel, et Obed, et Jaasie-Metsobaja.
\Chap{12}
\TextTitle{Les guerriers venus chez David à Tsiklag\FTNTT{2 S. 5:17 ; 1 Ch. 12:8-15 ; 1 Ch. 14:8}}
\VerseOne{}Voici ceux qui allèrent trouver David à Tsiklag, lorsqu'il était encore éloigné de la présence de Saül, fils de Kis. Ils étaient parmi les vaillants hommes qui lui prêtèrent leur secours pendant la guerre.
\VS{2}Ils étaient équipés d'arcs, se servant de la main droite et de la gauche pour jeter des pierres, et pour tirer des flèches avec l'arc. Ils étaient frères de Saül, de Benjamin,
\VS{3}Achiézer, le chef, et Joas, fils de Schemaa, qui était de Guibea, Jeziel, Péleth, fils d'Azmaveth, Beraca et Jéhu d'Anathoth ;
\VS{4}Jischmaeja de Gabaon, vaillant entre les trente, et même au-dessus des trente, et Jérémie, Jachaziel, Jochanan et Jozabad de Guedéra ;
\VS{5}Eluzaï, Jerimoth, Bealia, Schemaria et Schephathia de Haroph ;
\VS{6}Elkana, Jischija, Azareel, Joézer et Jaschobeam Koréites ;
\VS{7}Joéla et Zebadia, fils de Jerocham de Guedor.
\TextTitle{Les guerriers venus chez David dans la forteresse de Moab\FTNTT{1 S. 22:2-4}}
\VS{8}Quelques-uns aussi des Gadites se retirèrent auprès de David, dans la forteresse, au désert, hommes forts et vaillants, experts à la guerre et maniant le bouclier et la lance. Leurs visages étaient comme des faces de lion, et aussi prompts que des gazelles sur les montagnes.
\VS{9}Ezer le premier, Abdias le second, Eliab le troisième ;
\VS{10}Mischmanna le quatrième, Jérémie le cinquième ;
\VS{11}Attaï le sixième ; Eliel le septième ;
\VS{12}Jochanan le huitième, Elzabad le neuvième ;
\VS{13}Jérémie le dixième, Macbannaï le onzième.
\VS{14}C’étaient des fils de Gath, qui furent chefs de l'armée ; le plus petit avait la charge de cent hommes, et le plus grand de mille.
\VS{15}Ce sont ceux qui passèrent le Jourdain au premier mois, quand il déborde sur tous ses rivages ; et ils chassèrent ceux qui demeuraient dans les vallées, vers l'orient et l'occident.
\VS{16}Il vint aussi des fils de Benjamin et de Juda vers David à la forteresse.
\VS{17}David sortit au-devant d'eux, et prenant la parole, il leur dit : Si vous êtes venus en paix vers moi pour m'aider, mon cœur s’unira à vous ; mais si c'est pour me trahir et me livrer à mes ennemis, quoique je ne sois coupable d'aucune violence, que le Dieu de nos pères le voie, et qu'il fasse justice !
\VS{18}Alors Amasaï, l’un des principaux officiers, fut revêtu de l’Esprit, et dit : Que la paix soit avec toi, ô David ! Qu'elle soit avec toi, fils d'Isaï ! Que la paix soit à ceux qui t'aident, puisque ton Dieu t'aide ! Et David les reçut, et les établit parmi les chefs de ses troupes.
\VS{19}Des hommes de Manassé se joignirent à David, lorsqu’il alla combattre Saül avec les Philistins. Mais David et ses gens ne les aidèrent pas, parce que les princes des Philistins, après en avoir délibéré entre eux, le renvoyèrent, en disant : Il se tournera vers son maître Saül, au péril de nos têtes.
\VS{20}Quand donc il retournait à Tsiklag, Adnach, Jozabad, Jediaël, Micaël, Jozabad, Elihu et Tsilthaï, chefs des milliers qui étaient en Manassé, se tournèrent vers lui.
\VS{21}Et ils aidèrent David contre la troupe des Amalécites, car ils étaient tous forts et vaillants, et ils furent faits chefs dans l'armée.
\VS{22}De jour en jour, il venait des gens auprès de David pour l'aider, de sorte qu'il eut une grande armée, comme une armée de Dieu\FTNT{1 S. 22:2-4}.
\TextTitle{Les guerriers venus chez David à Hébron\FTNTT{2 S. 5:1-3}}
\VS{23}Voici le nombre des hommes équipés pour la guerre, qui vinrent auprès de David à Hébron, afin de lui transférer la royauté de Saül, selon le commandement de Yahweh\FTNT{2 S. 5:1-3}.
\VS{24}Des fils de Juda, qui portaient le bouclier et la lance, six mille huit cents, équipés pour la guerre.
\VS{25}Des fils de Siméon, forts et vaillants pour la guerre, sept mille cent.
\VS{26}Des fils de Lévi, quatre mille six cents.
\VS{27}Et Jehojada, prince de ceux d'Aaron, et avec lui trois mille sept cents ;
\VS{28}et Tsadok, jeune homme fort et vaillant, et vingt-deux chefs de la maison de son père.
\VS{29}Des fils de Benjamin, parents de Saül, trois mille ; car jusqu'alors la plus grande partie d’entre eux soutenaient la maison de Saül.
\VS{30}Des fils d'Ephraïm, vingt mille huit cents, forts et vaillants, et hommes de renom dans la maison de leurs pères.
\VS{31}De la demi-tribu de Manassé, dix-huit mille, qui furent désignés par leur nom pour aller établir David roi.
\VS{32}Des fils d'Issacar\FTNT{Les fils d’Issacar avaient la connaissance des temps. Discerner les temps dans lesquels nous sommes n’a rien à voir avec le fait de chercher à connaître la date du retour du Seigneur. Seul le Père connaît la date du retour du Messie (Za. 14:7 ; Mt. 24:36). Comprendre les caractéristiques de notre époque nous aide à nous réveiller afin d’accomplir les oeuvres que le Seigneur nous confie. Cette prise de conscience nous aidera à éviter les pièges de Satan et à mieux nous préparer aux noces de l’Agneau. Voir Mt. 16:3 ;  Ro. 13:11-14 ;  2 Pi. 1:19.}, fort intelligents dans la connaissance des temps, pour savoir ce que devait faire Israël, deux cents de leurs chefs, et tous leurs frères sous leurs ordres.
\VS{33}De Zabulon, cinquante mille combattants, rangés en bataille avec toutes sortes d'armes, et prêts à livrer bataille d’un cœur assuré.
\VS{34}De Nephthali, mille capitaines, et avec eux trente-sept mille, portant le bouclier et la lance.
\VS{35}Des Danites, vingt-huit mille six cents, équipés pour la guerre.
\VS{36}D'Aser, quarante mille combattants, et prêts à combattre.
\VS{37}De l’autre côté du Jourdain, à savoir des Rubénites, des Gadites, et de la demi-tribu de Manassé, cent vingt mille, avec tous les instruments de guerre pour combattre.
\VS{38}Tous ces hommes, gens de guerre, prêts a combattre, vinrent tous de bon cœur à Hébron, pour établir David roi sur tout Israël. Et tout le reste d'Israël était aussi d'un même sentiment pour établir David roi.
\VS{39}Et ils furent là avec David, mangeant et buvant pendant trois jours ; car leurs frères leur avaient préparé des vivres.
\VS{40}Et même ceux qui étaient les plus proches d'eux, jusqu'à Issacar, Zabulon et Nephthali, apportaient du pain sur des ânes, sur des chameaux, sur des mulets et sur des bœufs, de la farine, des figues sèches, des raisins secs, du vin, et de l'huile ; et ils amenaient des bœufs et des brebis en abondance, car il y avait une joie en Israël.
\Chap{13}
\TextTitle{Retour de l'arche, Uzza frappé par Yahweh\FTNTT{2 S. 6:1-11}}
\VerseOne{}Or David tint conseil avec les chefs de milliers et de centaines, avec tous les princes du peuple.
\VS{2}Et il dit à toute l'assemblée d'Israël : Si vous l'approuvez, et que cela vient de Yahweh, notre Dieu, envoyons partout vers nos autres frères, qui sont dans toutes les contrées d'Israël, et avec lesquels sont les sacrificateurs et les Lévites, dans leurs villes et dans leurs banlieues, afin qu'ils se réunissent à nous,
\VS{3}et que nous ramenions auprès de nous l’arche de notre Dieu ; car nous ne nous en sommes pas occupés du temps de Saül.
\VS{4}Et toute l'assemblée répondit qu'on le fasse ainsi ; car la chose fut approuvée par tout le peuple.
\VS{5}David donc assembla tout Israël, depuis Schichor, le torrent d'Egypte, jusqu'à l'entrée du pays de Hamath, pour ramener de Kirjath-Jearim l’arche de Dieu.
\VS{6}Et David monta avec tout Israël vers Baala à Kirjath-Jearim, qui appartient à Juda, pour faire amener de là l’arche de Dieu, devant laquelle est invoqué le Nom de Yahweh, qui habite entre les chérubins.
\VS{7}Ils mirent l’arche de Dieu sur un char neuf, et l'emmenèrent de la maison d'Abinadab ; et Uzza et Achjo conduisaient le char.
\VS{8}Et David et tout Israël dansaient en présence de Dieu de toute leur force, en chantant des cantiques et en jouant sur des violons, des luths, des tambourins, des cymbales, et des trompettes.
\VS{9}Quand ils furent arrivés à l'aire de Kidon, Uzza\FTNT{L’arche devait être transportée grâce à  des barres faites spécialement à cet effet, qui ne devaient pas être enlevées (Ex 27:6-7 ;  No. 1:51). Selon la Loi, seuls les Lévites devaient préparer et déplacer tout ce qui concernait le tabernacle.  Et même parmi les Lévites, chaque famille avait une fonction spécifique (No. 3 ; No. 4). Les Kehathites n’étaient pas autorisés à toucher l’arche, leur rôle se limitait seulement à la transporter à l’aide des barres (No. 4:15). Uzza a étendu sa main sur l’arche, alors qu’il n’était certainement pas Lévite. Il était devenu trop familier avec les choses saintes et avait pris à la légère les principes de Dieu. Il a voulu aider le Seigneur. Or, il ne faut jamais chercher à servir Dieu sans être appelé par lui.}  étendit sa main pour retenir l’arche, parce que les boeufs avaient glissé.
\VS{10}Et la colère de Yahweh s'enflamma contre Uzza, et il le frappa, parce qu'il avait étendu sa main sur l’arche. Uzza mourut en présence de Dieu.
\VS{11}David fut irrité de ce que Yahweh avait fait une brèche en la personne de Uzza. On a appelé jusqu'à ce jour ce lieu-là Pérets-Uzza, brèche d'Uzza.
\VS{12}David eut peur de Dieu en ce jour-là, et il dit: Comment ferais-je entrer chez moi l’arche de Dieu ?
\VS{13}C'est pourquoi David ne la retira point chez lui, dans la cité de David, mais il la fit conduire dans la maison d'Obed- Edom de Gath.
\VS{14}Et l’arche de Dieu demeura trois mois avec la famille d'Obed-Edom, dans sa maison. Yahweh bénit la maison d'Obed-Edom, et tout ce qui lui appartenait.
\Chap{14}
\TextTitle{Rayonnement du règne de David\FTNTT{2 S. 5:11-25 ; 23:13-17 ; 1 Ch. 3:5-9 ; 11:15-19 ; 12:8-15}}
\VerseOne{}Hiram, roi de Tyr, envoya des messagers à David, et du bois de cèdre, des tailleurs de pierres et des charpentiers, pour lui bâtir une maison.
\VS{2}Alors David reconnut que Yahweh l'affermissait comme roi sur Israël, et que son règne était fort élevé, à cause de son peuple d'Israël.
\VS{3}David prit encore des femmes à Jérusalem, et il engendra encore des fils et des filles.
\VS{4}Voici les noms des fils qu'il eut à Jérusalem : Schammua, Schobab, Nathan, Salomon,
\VS{5}Jibhar, Elischua, Elphéleth,
\VS{6}Noga, Népheg, Japhia,
\VS{7}Elischama, Beéliada et Eliphéleth.
\VS{8}Or quand les Philistins apprirent que David avait été oint pour roi sur tout Israël, ils montèrent tous à sa recherche. David l'ayant appris, sortit au-devant d'eux.
\VS{9}Les Philistins vinrent et se répandirent dans la vallée des Rephaïm.
\VS{10}David consulta Dieu, en disant : Monterai-je contre les Philistins, et les livreras-tu entre mes mains? Yahweh lui répondit : Monte, et je les livrerai entre tes mains.
\VS{11}Alors ils montèrent à Baal-Peratsim\FTNT{Baal-Peratsim signifie « seigneur des brèches »}, où David les battit. Puis il dit : Dieu a fait une brèche au milieu de mes ennemis par ma main, comme une brèche faite par les eaux. C'est pourquoi on donna à ce lieu-là le nom de Baal- Peratsim.
\VS{12}Et ils laissèrent là leurs dieux, et David ordonna qu'on les brûle au feu.
\VS{13}Les Philistins se répandirent encore une autre fois dans cette même vallée.
\VS{14}David consulta encore Dieu ; et Dieu lui répondit : Tu ne monteras point vers eux, mais tu te détourneras d'eux, et tu iras contre eux vis-à-vis des mûriers.
\VS{15}Dès que tu auras entendu au sommet des mûriers un bruit comme des gens qui marchent, tu sortiras alors pour combattre, car c’est Dieu qui marche devant toi pour frapper le camp des Philistins.
\VS{16}David fit selon ce que Dieu lui avait ordonné, et on frappa le camp des Philistins, depuis Gabaon jusqu'à Guézer.
\VS{17}Ainsi, la renommée de David se répandit par tous ces pays-là, et Yahweh remplit de frayeur toutes ces nations-là, au seul nom de David.
\Chap{15}
\TextTitle{David coordonne avec minutie l’arrivé de l’arche à Jérusalem\FTNTT{2 S. 6:12}}
\VerseOne{}David se bâtit des maisons dans la cité de David ; il prépara un lieu pour l’arche de Dieu, et dressa pour elle une tente.
\VS{2}Et David dit : L’arche de Dieu ne doit être portée que par les Lévites, car Yahweh les a choisis pour porter l’arche de Dieu, et pour faire le service à toujours\FTNT{No. 4:15.}.
\VS{3}David donc assembla tous ceux d'Israël à Jérusalem, pour faire monter l’arche de Yahweh dans le lieu qu'il lui avait préparé.
\VS{4}David assembla aussi les fils d'Aaron, et les Lévites.
\VS{5}Des fils de Kehath : Uriel, le chef, et ses frères, cent vingt.
\VS{6}Des fils de Merari : Asaja, le chef, et ses frères, deux cent vingt.
\VS{7}Des fils de Guerschon : Joël, le chef, et ses frères, cent trente.
\VS{8}Des fils d'Elitsaphan : Schemaeja, le chef, et ses frères, deux cents.
\VS{9}Des fils de Hébron : Eliel, le chef, et ses frères, quatre-vingts.
\VS{10}Des fils de Uziel : Amminadab, le chef, et ses frères, cent douze.
\VS{11}David appela les sacrificateurs Tsadok et Abiathar, et les Lévites, à savoir Uriel, Asaja, Joël, Schemaeja, Eliel, et Amminadab ;
\VS{12}et il leur dit: Vous qui êtes les chefs des familles des Lévites, sanctifiez-vous, vous et vos frères ; et transportez l’arche de Yahweh, le Dieu d'Israël, au lieu que je lui ai préparé.
\VS{13}Parce que vous n'y étiez pas la première fois, Yahweh, notre Dieu, a fait une brèche parmi nous ; car nous ne l'avons pas cherché selon la loi.
\VS{14}Les sacrificateurs donc et les Lévites se sanctifièrent pour faire monter l’arche de Yahweh, le Dieu d'Israël.
\VS{15}Et les fils des Lévites portèrent l’arche de Dieu sur leurs épaules, avec les barres qu'ils avaient sur eux, comme Moïse l'avait ordonné selon la parole de Yahweh.
\VS{16}David dit aux chefs des Lévites d'établir quelques-uns de leurs frères chantres, avec des instruments de musique, des luths, des violons, et des cymbales qui feraient retentir des sons éclatants, en signe de réjouissance.
\VS{17}Les Lévites donc établirent Héman, fils de Joël, et parmi ses frères, Asaph, fils de Bérékia; et des fils de Merari, qui étaient leurs frères, Ethan, fils de Kuschaja ;
\VS{18}avec eux leurs frères pour être du second ordre : Zacharie, Ben, Jaaziel, Schemiramoth, Jehiel, Unni, Eliab, Benaja, Maaséja, Matthithia, Eliphelé, Miknéja, Obed-Edom, et Jeïel, les portiers.
\VS{19}Quant aux chantres : Héman, Asaph et Ethan, ils avaient des cymbales d'airain pour les faire retentir.
\VS{20}Zacharie, Aziel, Schemiramoth, Jehiel, Unni, Eliab, Maaséja, et Benaja jouaient des luths sur alamoth ;
\VS{21}et Matthithia, Eliphelé, Miknéja, Obed-Edom, Jeïel et Azazia jouaient des harpes à huit cordes, pour conduire le chant.
\VS{22}Mais Kenania, le chef des Lévites, avait la charge de faire porter l’arche, enseignant comment il fallait la porter, car il était un homme très intelligent.
\VS{23}Bérékia et Elkana étaient portiers de l’arche.
\VS{24}Schebania, Josaphat, Nethaneel, Amasaï, Zacharie, Benaja, Eliézer, les sacrificateurs, sonnaient des trompettes devant l’arche de Dieu, et Obed-Edom et Jechija étaient portiers de l’arche.
\TextTitle{L'arche transportée au milieu des réjouissances\FTNTT{2 S. 6:12}}
\VS{25}David et les anciens d'Israël, avec les gouverneurs de milliers, marchaient, amenant avec joie l’arche de l'alliance de Yahweh, de la maison d' Obed-Edom.
\VS{26}Dieu aidait les Lévites qui portaient l’arche de l'alliance de Yahweh, et l’on sacrifia sept veaux et sept béliers.
\VS{27}David était vêtu d'un manteau de fin lin ; et tous les Lévites aussi qui portaient l’arche, les chantres ;  et Kenania, qui avait la principale charge de faire porter l’arche, était avec les chantres ; et David avait un éphod de lin.
\VS{28}Ainsi tout Israël amena l’arche de l'alliance de Yahweh, avec de grands cris de joie, et au son du cor, des shofars et des cymbales, faisant retentir leur voix avec des luths et des harpes.
\VS{29}Mais il arriva, comme l’arche de l'alliance de Yahweh entrait dans la cité de David, que Mical, fille de Saül, regardant par la fenêtre, vit le roi David sautant et dansant, et elle le méprisa dans son coeur.
\Chap{16}
\TextTitle{L’arche placée dans une tente à Jérusalem ; sacrifices et cantiques pour Yahweh\FTNTT{2 S. 6:17-19}}
\VerseOne{}Ils amenèrent donc l’arche de Dieu et la posèrent au milieu de la tente que David avait dressée pour elle ; et on offrit devant Dieu des holocaustes et des sacrifices d’offrande de paix.
\VS{2}Quand David eut achevé d'offrir les holocaustes et les sacrifices d’offrande de paix, il bénit le peuple au Nom de Yahweh.
\VS{3}Et il distribua à chacun, tant aux hommes qu'aux femmes, un pain,  un morceau de viande et un gâteau de raisin.
\VS{4}Et il établit quelques-uns des Lévites pour faire le service devant l’arche de Yahweh, pour célébrer, remercier, et louer le Dieu d'Israël.
\VS{5}Asaph était le premier et Zacharie le second ; Jeïel, Schemiramoth, Jehiel, Matthithia, Eliab, Benaja, Obed-Edom, et Jeïel, qui avaient des instruments de musique, à savoir des luths et des harpes ; et Asaph faisait retentir sa voix avec des cymbales.
\VS{6}Benaja et Jachaziel, les sacrificateurs, étaient continuellement avec des trompettes devant l’arche de l'alliance de Dieu.
\VS{7}Et en ce même jour, David remit entre les mains d'Asaph et de ses frères, les Psaumes suivants, pour commencer à célébrer Yahweh :
\VS{8}Célébrez Yahweh, invoquez son Nom ! Faites connaître parmi les peuples ses exploits !
\VS{9}Chantez-le,  célébrez-le !  Parlez de toutes ses merveilles !
\VS{10}Glorifiez-vous de son saint Nom !  Que le cœur de ceux qui cherchent Yahweh se réjouisse !
\VS{11}Recherchez Yahweh et sa force, cherchez continuellement sa face !
\VS{12}Souvenez-vous des merveilles qu'il a faites, de ses miracles et des jugements de sa bouche.
\VS{13}Postérité d'Israël, son serviteur, fils de Jacob, ses élus !
\VS{14}Yahweh est notre Dieu ; ses jugements s’exercent sur toute la terre.
\VS{15}Souvenez-vous toujours de son alliance, de ses promesses établies pour mille générations ;
\VS{16}du traité qu'il a fait avec Abraham et du serment qu’il a fait à Isaac,
\VS{17}et qu’il a confirmé à Jacob et à Israël, pour être une loi et une alliance éternelle,
\VS{18}en disant : Je te donnerai le pays de Canaan, comme l’héritage qui vous est échu.
\VS{19}Ils étaient alors une  poignée de gens, peu nombreux, et étrangers dans le pays,
\VS{20}car ils étaient errants de nation en nation, et d'un royaume vers un autre peuple.
\VS{21}Il ne permit à personne de les opprimer ; il a même châtié des rois à cause d'eux.
\VS{22}Et il a dit : Ne touchez point à mes oints, et ne faites point de mal à mes prophètes\FTNT{L’expression « ne touchez pas à mes oints » signifie qu'il ne faut pas leur porter physiquement atteinte. C’est une expression associée à des mauvais traitements physiques. Il est donc clair que ce verset, qu’on trouve également dans le  Ps. 105 : 15, ne peut absolument pas concerner la remise en question des enseignements d'un quelconque pasteur, prophète ou apôtre. Dans le contexte de ce passage, il est question des rois, des prophètes et des sacrificateurs, car c’est sur eux que reposait l’onction. Aujourd’hui, tous les chrétiens sont oints de Dieu (Ep. 1:13 ; Ep. 4:30).} !
\VS{23}Habitants de la terre, chantez à Yahweh ! Racontez chaque jour sa délivrance.
\VS{24}Racontez sa gloire parmi les nations, et ses merveilles parmi tous les peuples !
\VS{25}Car Yahweh est grand et très digne de louanges, il est plus redoutable que tous les dieux.
\VS{26}Car tous les dieux des peuples sont des idoles\FTNT{Jésus-Christ est le seul et le véritable Dieu (1 Co. 8:6 ; 1 Jn. 5:20).}, mais Yahweh a fait les cieux.
\VS{27}La majesté et la magnificence marchent devant lui; la force et la joie sont dans le lieu où il habite.
\VS{28}Familles des peuples, donnez à Yahweh, donnez à Yahweh gloire et force !
\VS{29}Donnez à Yahweh la gloire due à son Nom ! Apportez des offrandes, et présentez-vous devant lui. Prosternez-vous devant Yahweh avec des ornements saints !
\VS{30}Tremblez, vous tous habitants de la terre tout étonnés devant sa face ! Car la terre habitable est affermie par lui, et elle ne chancelle point.
\VS{31}Que les cieux se réjouissent, que la terre soit dans l’allégresse ! Et que l’on dise parmi les nations : Yahweh règne !
\VS{32}Que la mer retentisse avec tout ce qu'elle contient ! Que la campagne se réjouisse avec tout ce qu’elle renferme !
\VS{33}Que les arbres de la forêt poussent des cris de joie au devant de Yahweh, parce qu'il vient juger la terre\FTNT{Yahweh vient juger la terre. Cette prophétie confirme de façon incontestable la divinité de Jésus-Christ. Voir Za. 14:1-7.}.
\VS{34}Célébrez Yahweh, car il est bon, car sa miséricorde demeure à jamais !
\VS{35}Et dites : Ô Dieu de notre salut, sauve-nous, et rassemble-nous, et retire-nous d'entre les nations, pour célébrer ton saint Nom, et que nous nous glorifions de ta louange !
\VS{36}Béni soit Yahweh, le Dieu d'Israël, de siècle en siècle ! Et tout le peuple dit : Amen ! Louez Yahweh !
\VS{37}On laissa donc là, devant l’arche de l'alliance de Yahweh, Asaph et ses frères, pour faire le service continuellement, remplissant leur tâche jour par jour devant l’arche.
\VS{38}On laissa Obed-Edom, et ses frères, au nombre de soixante-huit, Obed-Edom, dis-je, fils de Jeduthun, et Hosa comme portiers.
\VS{39}On établit le sacrificateur Tsadok, et les sacrificateurs ses frères, devant le tabernacle de Yahweh, dans le haut lieu qui était à Gabaon,
\VS{40}pour offrir des holocaustes à Yahweh continuellement sur l'autel de l'holocauste, matin et soir, selon tout ce qui est écrit dans la loi de Yahweh, qu’il ordonna à Israël.
\VS{41}Auprès d’eux étaient Héman et Jeduthun, et les autres qui furent choisis et désignés par leur nom, pour célébrer Yahweh, parce que sa miséricorde demeure éternellement.
\VS{42}Et Héman et Jeduthun étaient avec ceux-là; il y avait aussi des trompettes et des cymbales pour ceux qui les faisaient retentir, et des instruments pour chanter les cantiques de Dieu. Les fils de Jeduthun étaient portiers.
\VS{43}Puis tout le peuple s'en alla chacun dans sa maison, et David aussi s'en retourna pour bénir sa maison.
\Chap{17}
\TextTitle{David veut construire un temple à Yahweh\FTNTT{2 S. 7:1-3}}
\VerseOne{}Or il arriva après que David fut établi dans sa maison, qu'il dit à Nathan, le prophète : Voici,  j’habite dans une maison de cèdres, et l’arche de l'alliance de Yahweh est sous une tente.
\VS{2}Nathan dit à David : Fais tout ce que tu as dans le cœur, car Dieu est avec toi.
\TextTitle{Réponse de Yahweh à David\FTNTT{2 S. 7:4-17}}
\VS{3}Mais il arriva cette nuit-là que la parole de Dieu fut adressée à Nathan, en disant :
\VS{4}Va, et dis à David, mon serviteur : Ainsi parle Yahweh :  Tu ne me bâtiras point de maison pour y habiter.
\VS{5}Puisque je n'ai point habité dans une maison depuis le jour où j'ai fait monter les fils d’Israël hors d'Egypte jusqu'à ce jour ; mais j'ai été de tente en tente, et de tabernacle en tabernacle.
\VS{6}Partout où j’ai marché avec tout Israël, ai-je dit un mot à un seul des juges d'Israël, auxquels j'ai ordonné de paître mon peuple, ai-je dit : Pourquoi ne m'avez-vous point bâti une maison de cèdres ?
\VS{7}Maintenant donc tu diras ainsi à David, mon serviteur : Ainsi parle Yahweh des armées : Je t'ai pris d'une cabane, d'auprès des brebis, afin que tu sois le conducteur de mon peuple d'Israël ;
\VS{8}j'ai été avec toi partout où tu as marché, j'ai exterminé devant toi tous tes ennemis, et j’ai rendu ton nom semblable au nom des grands qui sont sur la terre.
\VS{9}J’ai établi un lieu pour mon peuple d'Israël, et je l’ai planté afin qu’il habite chez lui et ne soit plus agité.  Les fils d'iniquité ne le détruiront plus comme ils l’ont fait auparavant,
\VS{10}et comme à l’époque où j'ai établi des juges sur mon peuple d'Israël. J’ai humilié tous tes ennemis. Je t’informe que Yahweh te bâtira une maison.
\VS{11}Quand tes jours seront accomplis pour t'en aller avec tes pères, je ferai lever ta postérité après toi, l’un de tes fils, et j'affermirai son règne\FTNT{Cette prophétie est relative au Messie. Voir 2 S. 7:12-17.}.
\VS{12}Il me bâtira une maison, et j'affermirai son trône éternellement.
\VS{13}Je serai pour lui un père, et il sera pour moi un fils ; et je ne retirerai point de lui ma grâce, comme je l'ai retirée de celui qui a été avant toi.
\VS{14}Mais je l'établirai dans ma maison et dans mon royaume éternellement, et son trône sera affermi pour toujours.
\VS{15}Nathan récita à David toutes ces paroles, et toute cette vision.
\TextTitle{Adoration et reconnaissance de David à Yahweh\FTNTT{2 S. 7:18-29}}
\VS{16}Alors le roi David entra, et se tint devant Yahweh, et dit : Ô Yahweh Dieu ! Qui suis-je, et quelle est ma maison, que tu m'aies fait parvenir au point où je suis ?
\VS{17}Mais cela t'a semblé être peu de chose, ô Dieu ! Et tu as parlé de la maison de ton serviteur pour le temps à venir, et tu as porté les regards sur moi à la manière de l'homme, toi qui es élevé, ô Yahweh Dieu !
\VS{18}Que pourrait te dire encore David de l'honneur que tu fais à ton serviteur ? Car tu connais ton serviteur.
\VS{19}Ô Yahweh ! Pour l'amour de ton serviteur, et selon ton cœur, tu as fait toutes ces grandes choses, pour lui révéler toutes ces grandeurs.
\VS{20}Ô Yahweh ! Nul n’est semblable à toi, et il n'y a point d'autre Dieu que toi selon tout ce que nous avons entendu de nos oreilles.
\VS{21}Et qui est comme ton peuple d'Israël, la seule nation sur la terre que Dieu lui-même est venu racheter pour lui, afin qu'elle soit son peuple, et pour te faire un Nom et pour accomplir des miracles et des prodiges, en chassant les nations devant ton peuple que tu as racheté d'Egypte ?
\VS{22}Et tu as établi ton peuple d'Israël afin qu’il soit ton peuple à toujours ; et toi, ô Yahweh ! Tu as été son Dieu.
\VS{23}Maintenant donc, ô Yahweh ! Que la parole que tu as prononcée sur ton serviteur et sur sa maison, soit ferme à jamais, et agis selon ta parole !
\VS{24}Et que ton Nom subsiste et soit magnifié éternellement, de sorte qu'on dise : Yahweh des armées, le Dieu d'Israël, est Dieu pour Israël ; et que la maison de David, ton serviteur, soit affermie devant toi.
\VS{25}Car, ô mon Dieu ! Tu as révélé à ton serviteur que tu lui bâtirais une maison. C'est pourquoi ton serviteur a pris la hardiesse de te faire cette prière.
\VS{26}Maintenant, ô Yahweh ! Tu es Dieu, et tu as parlé de ce bien à ton serviteur.
\VS{27}Veuille donc maintenant bénir la maison de ton serviteur, afin qu'elle soit éternellement devant toi ; car tu l'as bénie, ô Yahweh ! Et elle sera bénie à jamais !
\Chap{18}
\TextTitle{Le règne de David affermi\FTNTT{2 S. 8:1-18}}
\VerseOne{}Et il arriva que David battit les Philistins, et les humilia, et il enleva de la main des Philistins Gath et les villes de son ressort\FTNT{2 S. 8.  }.
\VS{2}Il battit aussi les Moabites, et les Moabites furent asservis à David et lui payèrent un tribut.
\VS{3}David battit aussi Hadarézer, roi de Tsoba, vers Hamath, lorsqu’il alla établir sa domination sur le fleuve de l'Euphrate.
\VS{4}David lui prit mille chars, sept mille cavaliers, et vingt mille hommes de pied ; et il coupa les jarrets des chevaux de tous les chars, mais il réserva cent chars.
\VS{5}Les Syriens de Damas vinrent au secours d’Hadarézer, roi de Tsoba, et David battit vingt-deux mille Syriens.
\VS{6}Puis David mit des garnisons dans la Syrie de Damas. Et les Syriens furent assujettis à David et lui payèrent un tribut. Yahweh sauvait David partout où il allait.
\VS{7}Et David prit les boucliers d'or qui étaient aux serviteurs de Hadarézer, et les apporta à Jérusalem.
\VS{8}Il emporta aussi de Thibchath, et de Cun, villes de Hadarézer, une grande quantité d'airain, dont Salomon fit la mer d'airain, les colonnes et les ustensiles d'airain.
\VS{9}Thohu, roi de Hamath, apprit que David avait défait toute l'armée de Hadarézer, roi de Tsoba.
\VS{10}Et il envoya Hadoram, son fils, vers le roi David pour le saluer et le féliciter de ce qu'il avait combattu Hadarézer, et qu'il l'avait défait. Car Hadarézer était dans une guerre continuelle contre Thohu. Quant à tous les vases d'or, d'argent, et d'airain,
\VS{11}le roi David les consacra aussi à Yahweh, avec l'argent et l'or qu'il avait emporté de toutes les nations, à savoir d'Edom, de Moab, des fils d'Ammon, des Philistins, et d’Amalek.
\VS{12}Et Abischaï, fils de Tseruja battit dix-huit mille Edomites dans la vallée du sel.
\VS{13}Il mit une garnison dans Edom, et tous les Edomites furent asservis à David ; et Yahweh gardait David partout où il allait.
\VS{14}Ainsi, David régna sur tout Israël, rendant jugement et justice à tout son peuple.
\VS{15}Joab, fils de Tseruja, avait la charge de l'armée, et Josaphat, fils d'Achilud, était archiviste.
\VS{16}Tsadok, fils d'Achithub, et Abimélec, fils d'Abiathar, étaient les sacrificateurs; et Schavscha était le secrétaire.
\VS{17}Benaja, fils de Jehojada, était sur les Kéréthiens et les Péléthiens ; mais les fils de David étaient les premiers auprès du roi.
\Chap{19}
\TextTitle{David monte contre les Ammonites et les Syriens\FTNTT{2 S. 10}}
\VerseOne{}Or il arriva après cela que Nachasch, roi des fils d’Ammon, mourut ; et son fils régna à sa place.
\VS{2}David dit : J'userai de bonté envers Hanun, fils de Nachasch, car son père a usé de bonté envers moi. Ainsi, David envoya des messagers pour le consoler de la mort de son père ; et les serviteurs de David vinrent au pays des fils d’Ammon vers Hanun pour le consoler.
\VS{3}Mais les chefs d'entre les fils d’Ammon dirent à Hanun : Penses-tu que ce soit pour honorer ton père que David t'a envoyé des consolateurs ? N'est-ce pas pour examiner et épier le pays, afin de le détruire, que ses serviteurs sont venus vers toi ?
\VS{4}Alors Hanun prit les serviteurs de David, les fit raser, et les fit couper leurs habits par le milieu jusqu'aux hanches. Puis il les renvoya.
\VS{5}Ils s'en allèrent, et le firent savoir par le moyen de quelques personnes à David, qui envoya des gens à leur rencontre ; car ces hommes-là étaient fort confus. Et le roi leur fit dire : Restez à Jéricho jusqu'à ce que votre barbe ait repoussé, et revenez ensuite.
\VS{6}Or les fils d’Ammon voyant qu'ils s'étaient rendus odieux à David, Hanun et les fils d'Ammon envoyèrent mille talents d'argent pour prendre à leur solde des chars et des cavaliers de Mésopotamie, de Syrie, de Maaca et de Tsoba.
\VS{7}Ils prirent à leur solde trente-deux mille hommes et des chars, et le roi de Maaca avec son peuple, lesquels vinrent camper devant Médeba. Les fils d'Ammon aussi s'assemblèrent de leurs villes et vinrent pour combattre.
\VS{8}David l’ayant appris, envoya Joab et ceux de toute l'armée qui étaient les plus vaillants.
\VS{9}Les fils d’Ammon sortirent et rangèrent leur armée en bataille à l'entrée de la ville ; et les rois qui étaient venus étaient à part dans la campagne.
\VS{10}Joab, voyant que l'armée était tournée contre lui devant et derrière, prit de tous les gens d'élite d'Israël, et les rangea contre les Syriens.
\VS{11}Et il donna la conduite du reste du peuple à Abischaï, son frère; et on les rangea contre les fils d’Ammon.
\VS{12}Et Joab lui dit: Si les Syriens sont plus forts que moi, tu viendras me délivrer ; et si les fils d'Ammon sont plus forts que toi, je te délivrerai.
\VS{13}Sois ferme, et montrons-nous vaillants pour notre peuple, et pour les villes de notre Dieu ; et que Yahweh fasse ce qui lui semblera bon.
\VS{14}Alors Joab et le peuple qui était avec lui s'approchèrent pour livrer bataille aux Syriens qui s'enfuirent de devant lui.
\VS{15}Et les fils d'Ammon voyant que les Syriens s'étaient enfuis, eux aussi s'enfuirent devant Abischaï, frère de Joab, et rentrèrent dans la ville, et Joab revint à Jérusalem.
\VS{16}Mais les Syriens, qui avaient été battus par ceux d'Israël, envoyèrent des messagers et firent venir les Syriens qui étaient au-delà du fleuve ; et Schophach, chef de l'armée d'Hadarézer, les conduisait.
\VS{17}On le rapporta à David, qui assembla tout Israël, passa le Jourdain, alla au-devant d'eux, et se rangea en bataille contre eux. David donc rangea la bataille contre les Syriens, et ils combattirent contre lui.
\VS{18}Mais les Syriens s'enfuirent de devant Israël ; et David défit sept mille chars des Syriens et quarante mille hommes de pied ; et il tua Schophach, le chef de l'armée.
\VS{19}Alors les serviteurs d'Hadarézer, voyant qu'ils avaient été battus par ceux d'Israël, firent la paix avec David, et lui furent asservis ; et les Syriens ne voulurent plus secourir les fils d’Ammon.
\Chap{20}
\TextTitle{Conquête de Rabba\FTNTT{2 S. 11:1-12:25 ; 2 S. 12:26-31}}
\VerseOne{}L’année suivante, au temps où les rois se mettaient en campagne, Joab conduisit une forte armée et ravagea le pays des fils d’Ammon ; puis il alla assiéger Rabba, tandis que David resta à Jérusalem. Joab battit Rabba, et la détruisit\FTNT{2 S 12:26-31.}.
\VS{2}David enleva la couronne de dessus la tête de son roi, et il trouva qu'elle pesait un talent d'or : Elle était garnie de pierres précieuses. On la mit sur la tête de David, qui emmena un très grand butin de la ville.
\VS{3}Il emmena aussi le peuple qui y était, et les mit aux scies, aux pics de fer et aux haches de fer ; David traita de la sorte toutes les villes des fils d’Ammon ; puis il s'en retourna avec tout le peuple à Jérusalem.
\TextTitle{Guerre contre les Philistins\FTNTT{2 S. 21:15-22}}
\VS{4}Il arriva après cela que la guerre continua à Guézer contre les Philistins. Alors Sibbecaï, le Huschatite, frappa Sippaï, qui était des fils de Rapha, et ils furent humiliés\FTNT{2 S 21:15-22.}.
\VS{5}Il y eut encore une autre guerre contre les Philistins. Et Elchanan, fils de Jaïr, frappa Lachmi, frère de Goliath de Gath, qui avait une lance dont le bois était comme une ensouple de tisserand.
\VS{6}Il y eut encore une autre guerre à Gath, où se trouva un homme de grande stature, qui avait six doigts à chaque main, et six orteils à chaque pied, de sorte qu'il en avait en tout vingt-quatre ; et il était aussi issu de Rapha.
\VS{7}Et il défia Israël ; mais Jonathan, fils de Schimea, frère de David, le tua.
\VS{8}Ceux-là naquirent à Gath ; ils étaient des enfants de Rapha, et ils moururent par les mains de David, et par les mains de ses serviteurs.
\Chap{21}
\TextTitle{David fait le dénombrement contre la volonté de Yahweh\FTNTT{2 S. 24:1-17}}
\VerseOne{}Mais Satan s'éleva contre Israël, et il incita David à faire le dénombrement d'Israël.
\VS{2}Et David dit à Joab et aux chefs du peuple : Allez et faites le dénombrement d'Israël, depuis Beer-Schéba jusqu'à Dan, et rapportez-le-moi, afin que j’en connaisse le nombre.
\VS{3}Mais Joab répondit : Que Yahweh veuille augmenter son peuple cent fois encore plus qu'il ne l’est, ô roi, mon seigneur. Tous ne sont-ils pas serviteurs de mon seigneur ? Pourquoi mon seigneur cherche-t-il cela ? Et pourquoi cela serait-il imputé comme un crime à Israël ?
\VS{4}Mais la parole du roi l'emporta sur Joab. Et Joab partit et parcourut tout Israël ; puis il revint à Jérusalem.
\VS{5}Et Joab donna à David le rôle du dénombrement du peuple, et il se trouva dans tout Israël onze cent mille hommes tirant l'épée ; et dans Juda quatre cent soixante-dix mille hommes tirant l'épée.
\VS{6}Bien qu'il n'eût pas compté entre eux ni Lévi ni Benjamin, parce que Joab exécutait la parole du roi en l’ayant en abomination,
\VS{7}cette chose déplut à Dieu, c'est pourquoi il frappa Israël.
\VS{8}Et David dit à Dieu : J'ai commis un très grand péché d'avoir fait une telle chose ; je te prie, pardonne maintenant l'iniquité de ton serviteur, car j'ai agi en insensé.
\VS{9}Et Yahweh parla à Gad, le voyant de David, en disant :
\VS{10}Va, parle à David, et dis-lui : Ainsi parle Yahweh, je te propose trois choses ; choisis l'une d'elles, afin que je te la fasse.
\VS{11}Et Gad vint à David, et lui dit : Ainsi parle Yahweh :
\VS{12}Choisis soit la famine durant l'espace de trois ans ; soit d'être consumé durant trois mois, étant poursuivi par tes ennemis, en sorte que l'épée de tes ennemis t'atteigne ; ou que l'épée de Yahweh, la peste, soit durant trois jours sur le pays, et que l'Ange de Yahweh porte la destruction dans toutes les contrées d'Israël. Vois maintenant ce que j'aurai à répondre à celui qui m'a envoyé.
\VS{13}Alors David répondit à Gad : Je suis dans une très grande angoisse ! Que je tombe, je te prie, entre les mains de Yahweh, parce que ses compassions sont immenses ; mais que je ne tombe point entre les mains des hommes !
\VS{14}Yahweh envoya donc la peste sur Israël, et il tomba soixante-dix mille hommes d'Israël.
\VS{15}Dieu envoya aussi un ange à Jérusalem pour la détruire ; et comme il la détruisait, Yahweh regarda et se repentit de ce mal. Et il dit à l'Ange qui détruisait : C'est assez ! Retire à présent ta main. Et l'Ange de Yahweh se tenait près de l'aire d'Ornan, le Jébusien.
\VS{16}Or David leva les yeux,  et vit l'Ange de Yahweh\FTNT{Ge. 16:7.} qui était entre la terre et le ciel, ayant dans sa main son épée nue, tournée contre Jérusalem. Et David et les anciens, couverts de sacs, tombèrent sur leurs faces.
\VS{17}Et David dit à Dieu : N'est-ce pas moi qui ai ordonné qu'on fasse le dénombrement du peuple ? C'est donc moi qui ai péché et qui ai très mal agi ; mais ces brebis qu'ont-elles fait ? Yahweh, mon Dieu ! Je te prie que ta main soit contre moi, et contre la maison de mon père, mais qu'elle ne soit pas contre ton peuple, pour le détruire.
\TextTitle{Fin de la plaie après l’offrande de David\FTNTT{2 S. 24:18-25}}
\VS{18}Alors l'Ange de Yahweh ordonna à Gad de dire à David, qu'il monte pour dresser un autel à Yahweh, dans l'aire d'Ornan, le Jébusien.
\VS{19}David donc monta selon la parole que Gad lui avait dite au Nom de Yahweh.
\VS{20}Ornan s'étant retourné, et ayant vu l'Ange,  ses quatre fils se cachèrent avec lui. Or Ornan foulait du blé.
\VS{21}David vint jusqu'à Ornan, et Ornan regarda, et ayant vu David, il sortit de l'aire et se prosterna devant lui, le visage à terre.
\VS{22}Et David dit à Ornan : Donne-moi la place de cette aire, et j'y bâtirai un autel à Yahweh ; donne-la-moi pour le prix qu'elle vaut, afin que cette plaie soit arrêtée de dessus le peuple.
\VS{23}Et Ornan dit à David : Prends-la, et que le roi, mon seigneur, fasse tout ce qui lui semblera bon. Voici, je donne ces bœufs pour les holocaustes, et ces instruments à fouler du blé pour le bois, et ce blé pour l'offrande ; je donne toutes ces choses.
\VS{24}Mais le roi David lui répondit : Non, mais certainement j'achèterai tout cela au prix qu'il vaut ; car je ne présenterai point à Yahweh ce qui est à toi, et je n'offrirai point un holocauste qui ne me coûte rien.
\VS{25}David donna donc à Ornan pour cette place, six cents sicles d'or de poids.
\VS{26}Puis il bâtit là un autel à Yahweh, et il offrit des holocaustes et des sacrifices d’offrande de paix, et il invoqua Yahweh, qui l'exauça par le feu envoyé des cieux sur l'autel de l'holocauste.
\VS{27}Alors Yahweh parla à l'ange, et l'ange remit son épée dans son fourreau.
\VS{28}En ce temps-là, David, voyant que Yahweh l'avait exaucé dans l'aire d'Ornan, le Jébusien, y offrait des sacrifices.
\VS{29}Or le tabernacle de Yahweh, que Moïse avait construit au désert, et l'autel des holocaustes, étaient en ce temps-là dans le haut lieu de Gabaon.
\VS{30}Mais David ne pouvait pas aller devant cet autel pour invoquer Dieu, parce qu'il avait été épouvanté à cause de l'épée de l'Ange de Yahweh.
\Chap{22}
\TextTitle{Préparatifs de David pour la construction du temple}
\VerseOne{}Et David dit : C'est ici la maison de Yahweh Dieu, et c'est ici l'autel pour les holocaustes d'Israël.
\VS{2}David ordonna de rassembler les étrangers qui étaient dans le pays d'Israël, et il établit des tailleurs de pierres pour tailler des pierres de taille, pour la construction de la maison de Dieu.
\VS{3}David prépara aussi du fer en abondance, afin d'en faire des clous pour les battants des portes et pour les crampons,  de l’airain en quantité telle qu’il n’était pas possible de le peser,
\VS{4}et du bois de cèdre sans nombre, parce que les Sidoniens et les Tyriens amenaient à David du bois de cèdre en abondance.
\VS{5}David dit : Salomon, mon fils, est jeune et délicat, et la maison qu'il faut bâtir à Yahweh doit être magnifique en excellence, en réputation, et en gloire, dans tous les pays. Je lui préparerai donc maintenant de quoi la bâtir. Ainsi, David prépara, avant sa mort, ces choses en abondance.
\TextTitle{Recommandation de David à Salomon }
\VS{6}Puis il appela Salomon, son fils, et lui ordonna de bâtir une maison à Yahweh, le Dieu d'Israël.
\VS{7}David donc dit à Salomon : Mon fils, j’avais à cœur de bâtir une maison au Nom de Yahweh, mon Dieu.
\VS{8}Mais la parole de Yahweh m'a été adressée, en disant : Tu as répandu beaucoup de sang, et tu as fait de grandes guerres ; tu ne bâtiras point de maison à mon Nom, parce que tu as répandu beaucoup de sang sur la terre devant moi.
\VS{9}Voici, il te naîtra un fils, qui sera un homme de repos,  et à qui je donnerai du repos par rapport à tous ses ennemis tout autour, c'est pourquoi son nom sera Salomon. Et en son temps, je donnerai la paix et le repos à Israël.
\VS{10}Ce sera lui qui bâtira une maison à mon Nom ; et il sera un fils pour moi, et je serai un père pour lui ; et j'affermirai le trône de son règne sur Israël à jamais.
\VS{11}Maintenant donc, mon fils, Yahweh sera avec toi, et tu prospéreras, et tu bâtiras la maison de Yahweh, ton Dieu, ainsi qu'il l’a déclaré à ton égard.
\VS{12}Seulement, que Yahweh te donne de la sagesse et de l'intelligence, et qu'il t'instruise touchant le gouvernement d'Israël, et comment tu dois garder la loi de Yahweh, ton Dieu.
\VS{13}Tu prospéreras si tu as soin de mettre en pratique les lois et les ordonnances que Yahweh a prescrites à Moïse pour Israël. Fortifie-toi et prends courage ; ne crains point et ne t'effraie de rien.
\VS{14}Voici, selon ma petitesse, j'ai préparé pour la maison de Yahweh cent mille talents d'or et un million de talents d'argent. Quant à l'airain et au fer, il est d'un poids incalculable, car il est en abondance. J'ai aussi préparé le bois et les pierres ; et tu y ajouteras ce qu'il faudra .
\VS{15}Tu as avec toi beaucoup d'ouvriers, de maçons, de tailleurs de pierres, de charpentiers, et toutes sortes de gens experts dans toute espèce d’ouvrage.
\VS{16}Il y a de l'or et de l'argent, de l'airain et du fer sans nombre. Lève-toi et agis, et Yahweh sera avec toi.
\VS{17}David ordonna aussi à tous les chefs d'Israël d'aider Salomon, son fils ; et il leur dit :
\VS{18}Yahweh,votre Dieu, n'est-il pas avec vous, et ne vous a-t-il pas donné du repos de tous côtés ? Car il a livré entre mes mains les habitants du pays, et le pays a été soumis devant Yahweh, et devant son peuple.
\VS{19}Maintenant donc, appliquez vos cœurs et vos âmes à rechercher Yahweh, votre Dieu ; levez-vous et bâtissez le sanctuaire de Yahweh Dieu, afin d’amener l’arche de l'alliance de Yahweh, et les ustensiles consacrés à Dieu dans la maison qui doit être bâtie au Nom de Yahweh.
\Chap{23}
\TextTitle{David désigne Salomon comme son successeur\FTNTT{1 Ch. 28:1}}
\VerseOne{}David étant vieux et rassasié de jours, établit Salomon, son fils, pour roi sur Israël.
\VS{2}Et il assembla tous les principaux d'Israël, les sacrificateurs et les Lévites.
\VS{3}On fit le dénombrement des Lévites, depuis l'âge de trente ans et au-dessus ; et les mâles d' entre-eux étant comptés, chacun par tête, il y eut trente-huit mille hommes\FTNT{No. 3:25-37}.
\VS{4}Et David dit : Qu’il y en ait  parmi eux vingt-quatre mille pour vaquer ordinairement à l'œuvre de la maison de Yahweh, et six mille comme  magistrats et juges,
\VS{5}quatre mille portiers, et quatre autres mille pour louer Yahweh avec des instruments que j'ai faits pour le louer.
\TextTitle{Dénombrement des Lévites\FTNTT{No. 3:25-37}}
\VS{6}David les divisa en classes d'après les fils de Lévi, à savoir Guerschon, Kehath et Merari.
\VS{7}Des Guerschonites il y eut Laedan et Schimeï.
\VS{8}Les fils de Laedan furent ces trois : Jehiel le premier, puis Zétham, puis Joël.
\VS{9}Les fils de Schimeï furent ces trois : Schelomith, Haziel et Haran. Ce sont là les chefs des maisons paternelles de la famille de Laedan.
\VS{10}Et les fils de Schimeï furent Jachath, Zina, Jeusch et Beria. Ce sont là les quatre fils de Schimeï.
\VS{11}Jachath était le premier et Zina le second; mais Jeusch et Beria n'eurent pas beaucoup de fils, c'est pourquoi ils furent comptés pour une seule maison paternelle dans le dénombrement.
\VS{12}Des fils de Kehath il y eut Amram, Jitsehar, Hébron et Uziel, en tout quatre.
\VS{13}Les fils d’Amram furent Aaron et Moïse. Aaron fut séparé lui et ses fils à toujours, pour sanctifier le Saint des saints, pour faire brûler des parfums en présence de Yahweh, pour le servir, et pour bénir en son Nom à toujours.
\VS{14}Et quant à Moïse, homme de Dieu, ses fils devaient être comptés de la tribu de Lévi.
\VS{15}Les fils de Moïse furent Guerschom et Eliézer.
\VS{16}Des fils de Guerschom, Schebuel le premier.
\VS{17}Quant aux fils d' Eliézer, Rechabia fut le premier ; et Eliézer n'eut point d'autres fils, mais les fils de Rechabia furent très nombreux.
\VS{18}Des fils de Jitsehar, Schelomith était le premier.
\VS{19}Les fils de Hébron furent Jerija le premier, Amaria le second, Jachaziel le troisième, Jekameam le quatrième.
\VS{20}Les fils d’Uziel furent Michée le premier, Jischija le second.
\VS{21}Des fils de Merari il y eut Machli et Muschi. Les fils de Machli furent Eléazar et Kis.
\VS{22}Eléazar mourut, et n'eut point de fils, mais des filles ; et les fils de Kis, leurs frères les prirent pour femmes.
\VS{23}Les fils de Muschi furent Machli, Eder et Jerémoth, eux trois.
\TextTitle{Fonctions des Lévites\FTNTT{No. 3:5-12}}
\VS{24}Ce sont là les fils de Lévi, selon les maisons de leurs pères, chefs des maisons paternelles, selon leurs dénombrements qui furent faits en comptant leurs noms, étant comptés chacun par tête ; et ils faisaient l’œuvre du service de la maison de Yahweh, depuis l'âge de vingt ans et au-dessus.
\VS{25}Car David dit : Yahweh, le Dieu d'Israël, a donné du repos à son peuple, et il établira sa demeure dans Jérusalem  à toujours.
\VS{26}Quant aux Lévites, ils n'auront plus à porter le tabernacle ni tous les ustensiles pour son service.
\VS{27}C'est pourquoi, dans les derniers registres de David, les fils de Lévi furent dénombrés depuis l'âge de vingt ans et au-dessus.
\VS{28}Car leur charge était d'assister les fils d'Aaron pour le service de la maison de Yahweh, étant établis sur le parvis et sur les chambres, pour la purification de toutes les choses saintes, pour l'œuvre du service de la maison de Dieu,
\VS{29}pour les pains de proposition, de la fleur de farine pour l'offrande, des galettes sans levain, pour tout ce qui se cuit sur la plaque, pour tout ce qui est rissolé, et pour la petite et grande mesure,
\VS{30}pour se présenter tous les matins et tous les soirs, afin de célébrer et louer Yahweh,
\VS{31}et offrir tous les holocaustes qu'il fallait offrir à Yahweh les jours de sabbat, aux nouvelles lunes, et aux fêtes solennelles, continuellement devant Yahweh, selon le nombre et les usages prescrits.
\VS{32}Ils donnaient leurs soins à la tente d’assignation, au lieu saint, et aux fils d’Aaron, leurs frères, pour le service de la maison de Yahweh.
\Chap{24}
\TextTitle{Vingt-quatre classes de sacrificateurs}
\VerseOne{}Quant aux fils d'Aaron, voici leurs classes\FTNT{Les vingt-quatre classes de sacrificateurs qui se tenaient devant Yahweh dans le temple de Jérusalem étaient une représentation des vingt-quatre vieillards qui se tiennent devant le trône de Dieu (Ap. 4:4).}. Les fils d’Aaron furent Nadab, Abihu, Eléazar et Ithamar.
\VS{2}Mais Nadab et Abihu\FTNT{Lé. 10:1-4.} moururent en présence de leur père, et n'eurent point de fils ; et Eléazar et Ithamar exercèrent la sacrificature.
\VS{3}Or David les sépara, à savoir Tsadok, qui était des fils d'Eléazar, et Achimélec, qui était des fils d'Ithamar, en fonction de leurs charges dans le service qu'ils avaient à faire.
\VS{4}Il se trouva parmi les fils d'Eléazar plus de chefs que parmi les fils d'Ithamar, et on en fit la division ; les fils d'Eléazar avaient seize chefs, selon leurs maisons paternelles, et les fils d'Ithamar huit chefs de maisons paternelles.
\VS{5}Et on les classa par le sort, les entremêlant les uns avec les autres, car les chefs du sanctuaire et les chefs de la maison de Dieu furent tirés tant des fils d'Eléazar que des fils d'Ithamar.
\VS{6}Schemaeja, fils de Nethaneel, le scribe, qui était de la tribu de Lévi, les mit par écrit devant le roi, les princes du peuple, devant Tsadok, le sacrificateur, et Achimélec, fils d'Abiathar, et devant les chefs de maisons paternelles des sacrificateurs et des Lévites. On tira au sort une maison paternelle pour Eléazar, et une autre fut tirée pour Ithamar.
\VS{7}Le premier sort échut à Jehojarib, le second à Jedaeja,
\VS{8}le troisième à Harim, le quatrième à Seorim,
\VS{9}le cinquième à Malkija, le sixième à Mijamin,
\VS{10}le septième à Hakkots, le huitième à Abija,
\VS{11}le neuvième à Josué, le dixième à Schecania,
\VS{12}le onzième à Eliaschib, le douzième à Jakim,
\VS{13}le treizième à Huppa, le quatorzième à Jeschébeab,
\VS{14}le quinzième à Bilga, le seizième à Immer,
\VS{15}le dix-septième à Hézir, le dix-huitième à Happitsets,
\VS{16}le dix-neuvième à Pethachja, le vingtième à Ezéchiel,
\VS{17}le vingt et unième à Jakim, et le vingt-deuxième à Gamul,
\VS{18}le vingt-troisième à Delaja, le vingt-quatrième à Maazia.
\VS{19}Tel fut leur classement pour le service qu'ils avaient à faire, lorsqu'ils entraient dans la maison de Yahweh, selon qu'il leur avait été ordonné par Aaron, leur père, comme Yahweh, le Dieu d'Israël, le lui avait ordonné.
\TextTitle{Les chefs des lévites ; les fils de Kehath et de Merari}
\VS{20}Voici les chefs du reste des Lévites. Des fils de Amram : Schubaël ; et des fils de Shubaël, Jechdia.
\VS{21}De Rechabia, des fils de Rechabia, Jischija était le premier.
\VS{22}Des Jitseharites, Schelomoth ; des fils de Schelomoth, Jachath.
\VS{23}Des fils d’Hébron, Jerija, Amaria le second ; Jachaziel le troisième, Jekameam le quatrième.
\VS{24}Des fils d'Uziel, Michée ; des fils de Michée, Schamir.
\VS{25}Le frère de Michée était Jischija ; des fils de Jischija, Zacharie.
\VS{26}Des fils de Merari, Machli et Muschi. Des fils de Jaazija, son fils.
\VS{27}Des fils de Merari, de Jaazija, son fils : Schoham, Zaccur et Ibri.
\VS{28}De Machli, Eléazar, qui n'eut point de fils.
\VS{29}De Kis, les fils de Kis, Jerachmeel.
\VS{30}Et des fils de Muschi, Machli, Eder et Jerimoth. Ce sont là les fils des Lévites, selon les maisons de leurs pères.
\VS{31}Eux aussi, comme leurs frères, les fils d'Aaron, ils tirèrent au sort, devant le  roi David, Tsadok et Ahimélec, et les chefs des pères des sacrificateurs et des Lévites. Il en fut ainsi pour chaque chef de maison comme pour le moindre de ses frères.
\Chap{25}
\TextTitle{Dénombrement des musiciens et des chantres}
\VerseOne{}David et les chefs de l'armée mirent à part pour le service ceux des fils d'Asaph, d'Héman et de Jeduthun qui prophétisaient avec des harpes, des luths et des cymbales. Et voici le nombre des hommes employés pour le service qu’ils avaient à faire.
\VS{2}Des fils d'Asaph : Zaccur, Joseph, Nethania et Aschareéla, fils d'Asaph, sous la conduite d'Asaph, qui prophétisait selon les ordres du roi.
\VS{3}De Jeduthun, les six fils de Jeduthun : Guedalia, Tseri, Esaïe, Haschabia, Matthithia et Schimeï, jouaient de la harpe, sous la conduite de leur père Jeduthun, qui prophétisait en célébrant et louant Yahweh.
\VS{4}D'Héman, les fils d'Héman : Bukkija, Matthania, Uziel, Schebuel, Jerimoth, Hanania, Hanani, Eliatha, Guiddalthi, Romamthi-Ezer, Joschbekascha, Mallothi, Hothir, Machazioth.
\VS{5}Tous ceux-là étaient fils d'Héman, le voyant du roi, qui révélait les paroles de Dieu pour en exalter la puissance. Dieu donna à Héman quatorze fils et trois filles.
\VS{6}Tous ceux-là étaient employés, sous la conduite de leurs pères, aux cantiques de la maison de Yahweh, avec des cymbales, des luths, et des harpes, dans le service de la maison de Dieu, selon les ordres du roi donnés à Asaph, à Jeduthun et à Héman.
\VS{7}Et leur nombre avec leurs frères, auxquels on avait enseigné les cantiques de Yahweh, était de deux cent quatre-vingt-huit, tous très habiles.
\TextTitle{Les musiciens et chantres répartis en vingt-quatre classes}
\VS{8}Et ils tirèrent au sort pour leurs fonctions, petits et grands, maîtres et disciples.
\VS{9}Et le premier sort échut à Asaph, à savoir à Joseph. Le second à Guedalia, lui, ses frères et ses fils étaient douze.
\VS{10}Le troisième à Zaccur, lui, ses fils et ses frères étaient douze.
\VS{11}Le quatrième à Jitseri, lui, ses fils et ses frères étaient douze.
\VS{12}Le cinquième à Nethania, lui, ses fils et ses frères étaient douze.
\VS{13}Le sixième à Bukkija, lui, ses fils et ses frères étaient douze.
\VS{14}Le septième à Jesareéla, lui, ses fils et ses frères étaient douze.
\VS{15}Le huitième à Esaïe, lui, ses fils et ses frères étaient douze.
\VS{16}Le neuvième à Matthania, lui, ses fils et ses frères étaient douze.
\VS{17}Le dixième à Schimeï, lui, ses fils et ses frères étaient douze.
\VS{18}L'onzième à Azareel, lui, ses fils et ses frères étaient douze.
\VS{19}Le douzième à Haschabia, lui, ses fils, et ses frères étaient douze.
\VS{20}Le treizième à Schubaël, lui, ses fils et ses frères étaient douze.
\VS{21}Le quatorzième à Matthithia, lui, ses fils et ses frères étaient douze.
\VS{22}Le quinzième à Jerémoth, lui, ses fils et ses frères étaient douze.
\VS{23}Le seizième à Hanania, lui, ses fils et ses frères étaient douze.
\VS{24}Le dix-septième à Joschbekascha, lui, ses fils et ses frères étaient douze.
\VS{25}Le dix-huitième à Hanani, lui, ses fils et ses frères étaient douze.
\VS{26}Le dix-neuvième à Mallothi, lui, ses fils et ses frères étaient douze.
\VS{27}Le vingtième à Elijatha, lui, ses fils et ses frères étaient douze.
\VS{28}Le vingt et unième à Hothir, lui, ses fils et ses frères étaient douze.
\VS{29}Le vingt-deuxième à Guiddalthi, lui, ses fils et ses frères étaient douze.
\VS{30}Le vingt-troisième à Machazioth, lui, ses fils et ses frères étaient douze.
\VS{31}Le vingt-quatrième à Romamthi-Ezer, lui, ses fils et ses frères étaient douze.
\Chap{26}
\TextTitle{Les classes des portiers}
\VerseOne{}Et quant aux classes des portiers, il y eut pour les Koréites : Meschélémia, fils de Koré, d'entre les fils d'Asaph.
\VS{2}Les fils de Meschélémia furent Zacharie, le premier-né, Jediaël le second, Zebadia le troisième, Jathniel le quatrième,
\VS{3}Elam le cinquième, Jochanan le sixième et Eljoénaï le septième.
\VS{4}Les fils d’Obed-Edom furent Schemaeja le premier-né, Jozabad le second, Joach le troisième, Sacar le quatrième, Nethaneel le cinquième,
\VS{5}Ammiel le sixième, Issacar le septième, Peulthaï le huitième ; car Dieu l'avait béni.
\VS{6}A Schemaeja, son fils, naquirent des fils qui eurent le commandement sur la maison de leur père, parce qu'ils étaient des hommes forts et vaillants.
\VS{7}Les fils donc de Schemaeja furent Othni, Rephaël, Obed, Elzabad et ses frères, hommes vaillants, Elihu et Semaeja.
\VS{8}Tous ceux-là étaient des fils d' Obed-Edom, eux, leurs fils et leurs frères, étaient des hommes pleins de vigueur et de force pour le service ; ils étaient soixante-deux d'Obed-Edom.
\VS{9}Les fils de Meschélémia avec ses frères, vaillants hommes étaient au nombre de dix-huit.
\VS{10}Les fils de Hosa, d'entre les fils de Merari, furent  Schimri le chef, quoiqu'il ne fût pas l'aîné, néanmoins son père l'établit pour chef ;
\VS{11}Hilkija était le second, Thebalia le troisième, Zacharie le quatrième; tous les fils et frères de Hosa furent treize.
\VS{12}A ces classes de portiers, aux chefs de ces hommes et à leurs frères, fut remise la garde pour le service de la maison de Yahweh.
\VS{13}Ils tirèrent au sort pour chaque porte, autant pour le plus petit que pour le plus grand, selon leurs familles.
\VS{14}Et ainsi, le sort pour la porte vers l'orient échut à Schélémia. Puis on tira au sort pour Zacharie, son fils, qui était un sage conseiller, et la porte du côté du nord lui fut échue par le sort.
\VS{15}Le sort d'Obed-Edom lui échut pour la porte du côté du sud, et la maison des magasins échut à ses fils.
\VS{16}A Schuppim et à Hosa pour la porte vers l'occident, auprès de la porte de Schalléketh, au chemin montant ; une garde étant vis-à-vis de l'autre.
\VS{17}Il y avait vers l'orient six Lévites ; vers le nord, quatre par jour  ; vers le sud, quatre aussi par jour ; et vers la maison des magasins, deux de chaque côté ;
\VS{18}du côté de la banlieue vers l'occident, il y en avait quatre au chemin, et deux vers la banlieue.
\VS{19}Ce sont là les classes des portiers pour les fils des Koréites, et pour les fils de Merari.
\TextTitle{Les Lévites commis sur les trésors du temple}
\VS{20}Ceux-ci aussi étaient Lévites : Achija commis sur les trésors de la maison de Dieu et les trésors des choses consacrées.
\VS{21}Des fils de Laedan, qui étaient d'entre les fils des Guerschonites, du côté de Laedan, d'entre les chefs des maisons paternelles appartenant à Laedan le Guerschonite, Jehiéli.
\VS{22}D'entre les fils de Jehiéli : Zétham et Joël, son frère, commis sur les trésors de la maison de Yahweh.
\VS{23}Pour les Amramites, les Jitseharites, les Hébronites et les  Uziélites,
\VS{24}Schebuel, fils de Guerschom, fils de Moïse, était commis sur les autres trésors.
\VS{25}Et quant à ses frères issus d'Eliézer, dont Rechabia fut fils, dont le fils fut Esaïe, dont le fils fut Joram, dont le fils fut  Zicri, dont le fils fut Schelomith,
\VS{26}c’étaient Schelomith et ses frères qui gardaient tous les trésors des choses saintes que le roi David, les chefs des familles paternelles, les chefs de milliers et de centaines, et les chefs de l'armée avaient consacrées.
\VS{27}C'était le butin de guerre qu'ils avaient consacré, pour l’entretien de la maison de Yahweh.
\VS{28}Tout ce qu'avait consacré Samuel, le voyant, Saül, fils de Kis, Abner, fils de Ner et Joab, fils de Tseruja, toutes les choses consacrées étaient mises sous la main de Schelomith et de ses frères.
\TextTitle{Les magistrats et juges en Israël}
\VS{29}Parmi les Jitseharites, Kenania et ses fils étaient employés aux affaires extérieures sur Israël pour être magistrats et juges.
\VS{30}Quant aux Hébronites, Haschabia et ses frères, hommes vaillants, au nombre de mille sept cents, avaient la surveillance d'Israël de l’autre côté du Jourdain, vers l'occident, pour toute œuvre qui concernait Yahweh, et pour le service du roi.
\VS{31}Quant aux Hébronites, selon leurs générations dans les maisons paternelles, Jerija fut le chef des Hébronites. On fit une recherche au sujet des Hébronites à la quarantième année du règne de David, et on trouva parmi eux à Jaezer de Galaad, des hommes forts et vaillants.
\VS{32}Les frères de Jerija, hommes vaillants, furent deux mille sept cents, issus des maisons paternelles ; et le roi David les établit sur les Rubénites, sur les Gadites, et sur la demi-tribu de Manassé, pour toute œuvre qui concernait Dieu, et pour les affaires du roi.
\Chap{27}
\TextTitle{Les douze chefs de guerre de David}
\VerseOne{}Quant aux fils d'Israël, selon leur dénombrement, il y avait des chefs de maisons paternelles, des chefs de milliers et de centaines, et leurs officiers, qui servaient le roi pour tout ce qui concernait les divisions, leur arrivée et leur départ, mois par mois, pendant tous les mois de l'année, et chaque division était de vingt-quatre mille hommes.
\VS{2}Et Jaschobeam, fils de Zabdiel, présidait sur la première division, pour le premier mois ; et dans sa division il y avait vingt-quatre mille hommes.
\VS{3}Il était des fils de Pérets, chef de tous les capitaines de l'armée du premier mois.
\VS{4}Dodaï, l'Achochite, présidait sur la division du deuxième mois, Mikloth, l’un des chefs de sa division ; et il avait une division de vingt-quatre mille hommes.
\VS{5}Le chef de la troisième armée pour le troisième mois était Benaja, fils de Jehojada, le sacrificateur et le capitaine en chef ; et dans sa division il y avait vingt-quatre mille hommes.
\VS{6}C'est ce Benaja qui était fort entre les trente, et par dessus les trente ; et Ammizadab, son fils, était dans sa division.
\VS{7}Le quatrième pour le quatrième mois était Asaël, frère de Joab, et Zébadia son fils, après lui ; et il y avait dans sa division vingt-quatre mille hommes.
\VS{8}Le cinquième pour le cinquième mois était le capitaine Schamehuth, le Jizrachite ; et dans sa division il y avait vingt-quatre mille hommes.
\VS{9}Le sixième pour le sixième mois était Ira, fils d' Ikkesch le Tekoïte ; et dans sa division il y avait vingt-quatre mille hommes.
\VS{10}Le septième pour le septième mois était Hélets le Pelonite, des fils d'Ephraïm ; et il y avait dans sa division vingt-quatre mille hommes.
\VS{11}Le huitième pour le huitième mois était Sibbecaï le Huschatite, de la famille des Zérachites ; et il y avait dans sa division vingt-quatre mille hommes.
\VS{12}Le neuvième pour le neuvième mois était Abiézer d'Anathoth, des Benjamites ; et il y avait dans sa division vingt-quatre mille hommes.
\VS{13}Le dixième pour le dixième mois était Maharaï de Nethopha, de la famille des Zérachites ; et il y avait dans sa division vingt-quatre mille hommes.
\VS{14}Le onzième pour le onzième mois était Benaja de Pirathon, des fils d'Ephraïm; et il y avait dans sa division vingt-quatre mille hommes.
\VS{15}Le douzième pour le douzième mois était Heldaï de Nethopha, appartenant à Othniel; et il y avait dans sa division vingt-quatre mille hommes.
\TextTitle{Les douze chefs des tribus d’Israël}
\VS{16}Et ceux-ci présidaient sur les tribus d'Israël : Eliézer, fils de Zicri, était le conducteur des Rubénites. Des Siméonites : Schephathia, fils de Maaca.
\VS{17}Des Lévites, Haschabia, fils de Kemuel. De ceux d'Aaron : Tsadok.
\VS{18}De Juda : Elihu, qui était des frères de David. De ceux d'Issacar : Omri, fils de Micaël.
\VS{19}De ceux de Zabulon : Jischemaeja, fils d'Abdias. De ceux de Nephthali : Jerimoth, fils d'Azriel.
\VS{20}Des fils d'Ephraïm : Hosée, fils d'Azazia. De la demi-tribu de Manassé : Joël, fils de Pedaja.
\VS{21}De l'autre demi-tribu de Manassé en Galaad : Jiddo, fils de Zacharie. De ceux de Benjamin : Jaasiel, fils d'Abner.
\VS{22}De ceux de Dan : Azareel, fils de Jerocham. Ce sont là les chefs des tribus d'Israël.
\TextTitle{Dénombrement arrêté par Yahweh}
\VS{23}Mais David ne fit point le dénombrement des Israélites, depuis l'âge de vingt ans et au-dessous ; parce que Yahweh avait dit qu'il multiplierait Israël comme les étoiles du ciel.
\VS{24}Joab, fils de Tseruja, avait bien commencé à en faire le dénombrement, mais il n'acheva pas parce que la colère de Dieu s'était répandue à cause de cela sur Israël ; c'est pourquoi ce dénombrement ne fut point mis parmi les dénombrements enregistrés dans les Chroniques du roi David.
\TextTitle{Les gestionnaires de David}
\VS{25}Or Azmaveth, fils d'Adiel, était commis sur les finances du roi ; mais Jonathan, fils d'Ozias, était commis sur les provisions dans les champs, dans les villes, les villages et les châteaux.
\VS{26}Et Ezri, fils de Kelub, était commis sur ceux qui travaillaient dans la campagne et cultivaient la terre.
\VS{27}Et Schimeï de Rama sur les vignes, et Zabdi de Schepham sur ce qui provenait des vignes, et sur les celliers du vin.
\VS{28}Et Baal-Hanan de Guéder sur les oliviers et sur les figuiers qui étaient à la campagne ; et Joasch sur les celliers à huile.
\VS{29}Schithraï de Saron était commis sur le gros bétail qui paissait en Saron ; Schaphath, fils d'Adlaï, sur le gros bétail qui paissait dans les vallées.
\VS{30}Obil, l'Ismaélite, sur les chameaux; Jechdia de Méronoth, sur les ânesses.
\VS{31}Jaziz, l'Hagarénien, sur les troupeaux du menu bétail. Tous ceux-là avaient la charge des biens qui appartenaient au roi David.
\TextTitle{Les conseillers de David}
\VS{32}Mais Jonathan, oncle de David, était conseiller, homme très intelligent et scribe ; et Jehiel, fils de Hacmoni, était avec les fils du roi.
\VS{33}Achitophel était le conseiller du roi ; et Huschaï, l'Arkien, était l'intime ami du roi.
\VS{34}Après Achitophel était Jehojada, fils de Benaja et Abiathar; et Joab était le chef de l'armée du roi.
\Chap{28}
\TextTitle{Dernières paroles de David, la royauté remise à Salomon\FTNTT{1 Ch. 23:2}}
\VerseOne{}David convoqua à Jérusalem tous les chefs d'Israël, les chefs des tribus, et les chefs des divisions qui servaient le roi ; et les chefs de milliers et de centaines, et ceux qui avaient la charge de tous les biens du roi, et de tout ce qu’il possédait, ses fils avec ses eunuques, et les hommes puissants, et tous les  héros et tous les hommes vaillants.
\VS{2}Puis le roi David se leva sur ses pieds, et dit : Mes frères et mon peuple, écoutez-moi ! J’avais à cœur de bâtir une maison de repos pour l’arche de l'alliance de Yahweh, et pour le marchepied de notre Dieu, et j'ai fait les préparatifs pour la bâtir.
\VS{3}Mais Dieu m'a dit : Tu ne bâtiras point de maison à mon Nom, parce que tu es un homme de guerre, et que tu as répandu beaucoup de sang.
\VS{4}Or comme Yahweh, le Dieu d'Israël, m'a choisi dans toute la maison de mon père pour être roi sur Israël à toujours ; car il a choisi Juda pour conducteur, et de la maison de Juda la maison de mon père, et d'entre les fils de mon père il a pris son plaisir en moi, pour me faire régner sur tout Israël.
\VS{5}Aussi, entre tous mes fils, car Yahweh m'a donné beaucoup de fils, il a choisi Salomon, mon fils, pour le faire asseoir sur le trône du royaume de Yahweh, sur Israël.
\VS{6}Et il m'a dit : Salomon, ton fils, est celui qui bâtira ma maison et mes parvis ; car je me le suis choisi pour fils et je serai pour lui un père.
\VS{7}Et j'affermirai son règne à toujours s'il s'applique à pratiquer mes commandements et à observer mes ordonnances, comme il le fait aujourd'hui.
\VS{8}Maintenant donc, je vous somme en présence de tout Israël, qui est l'assemblée de Yahweh, et devant notre Dieu qui l'entend, que vous ayez à garder et à rechercher diligemment tous les commandements de Yahweh, votre Dieu, afin que vous possédiez ce bon pays, et que vous le fassiez hériter à vos fils après vous, à jamais.
\VS{9}Et toi, Salomon, mon fils, connais le Dieu de ton père, et sers-le avec un cœur droit et une bonne volonté ; car Yahweh sonde tous les cœurs, et connaît toutes les dispositions des pensées. Si tu le cherches, il se laissera trouver par toi ; mais si tu l'abandonnes, il te rejettera pour toujours.
\VS{10}Considère maintenant que Yahweh t'a choisi pour bâtir une maison pour son sanctuaire. Fortifie-toi donc et applique-toi à y travailler.
\VS{11}David donna à Salomon, son fils, le modèle\FTNT{David donna le modèle du temple qu’il  avait reçu de Dieu à Salomon. Beaucoup veulent servir Dieu sans modèle, tandis que d’autres vont chercher des modèles dans le monde (2 R. 16:10-18). Nous devons faire l’œuvre de Dieu uniquement selon le modèle biblique.} du portique, de ses maisons, des chambres du trésor, des chambres hautes, des chambres intérieures et du lieu du propitiatoire.
\VS{12}Il lui donna le modèle de toutes les choses qui lui avaient été inspirées par l'Esprit qui était avec lui, pour les parvis de la maison de Yahweh, pour les chambres d'alentour, pour les trésors de la maison de Yahweh et pour les trésors des choses saintes ;
\VS{13}pour les divisions des sacrificateurs et des Lévites, pour toute l'œuvre du service de la maison de Yahweh, et pour tous les ustensiles du service de la maison de Yahweh.
\VS{14}Il lui donna aussi de l'or, un certain poids, pour les choses qui devaient être d'or, à savoir pour tous les ustensiles de chaque service ; et de l'argent, un certain poids, pour tous les ustensiles d'argent, à savoir pour tous les ustensiles de chaque service.
\VS{15}Le poids des chandeliers d'or, et de leurs lampes d'or, selon le poids de chaque chandelier et de ses lampes ; et le poids des chandeliers d'argent, selon le poids de chaque chandelier et de ses lampes, selon l’usage de chaque chandelier.
\VS{16}Et le poids de l'or pesant ce qu'il fallait pour chaque table des pains de proposition; et de l'argent pour les tables d'argent.
\VS{17}Il lui donna le modèle pour les fourchettes, pour les bassins et pour les calices d’or pur ; le modèle pour les coupes d'or, selon le poids de chaque coupe,  et de l'argent pour les coupes d'argent, selon le poids de chaque coupe ;
\VS{18}et le modèle pour l'autel des parfums en or épuré, avec le poids. Il lui donna encore le modèle du char, des chérubins d’or qui étendent les ailes et qui couvrent l’arche de l'alliance de Yahweh.
\VS{19}C’est par un écrit de sa main, dit-il, que Yahweh m’a donné l'intelligence de tout cela, de tous les ouvrages de ce modèle.
\TextTitle{David demande à Salomon de  bâtir le temple}
\VS{20}C'est pourquoi David dit à Salomon, son fils : Fortifie-toi, prends courage et travaille ; ne crains point et ne t'effraie point. Car Yahweh Dieu, mon Dieu, sera avec toi, il ne te délaissera point, et il ne t'abandonnera point, jusqu'à ce que tu aies achevé tout l'ouvrage du service de la maison de Yahweh.
\VS{21}Et voici, j'ai fait les divisions des sacrificateurs et des Lévites pour tout le service de la maison de Dieu ; et il y a avec toi pour tout cet ouvrage toutes sortes de gens prompts et experts, pour toutes sortes de services ; et les chefs avec tout le peuple seront prêts pour exécuter tout ce que tu diras.
\Chap{29}
\TextTitle{Offrandes volontaires de David et de tout le peuple}
\VerseOne{}Puis le roi David dit à toute l'assemblée : Dieu a choisi un seul de mes fils, à savoir Salomon, qui est encore jeune et délicat, et l'ouvrage est considérable, car ce palais n'est point pour un homme, mais pour Yahweh Dieu.
\VS{2}Et moi, j'ai préparé de toutes mes forces pour la maison de mon Dieu, de l'or pour les choses qui doivent être d'or, de l'argent pour celles qui doivent être d'argent, de l'airain pour celles d'airain, du fer pour celles de fer, du bois pour celles de bois, des pierres d'onyx, et des pierres pour être enchâssées, des pierres d'escarboucle, et des pierres de diverses couleurs, des pierres précieuses de toutes sortes, et du marbre en abondance.
\VS{3}Et outre cela, parce que j'ai une grande affection pour la maison de mon Dieu, je donne pour la maison de mon Dieu, outre toutes les choses que j'ai préparées pour la maison du sanctuaire, l'or et l'argent que j'ai entre mes plus précieux joyaux :
\VS{4}Trois mille talents d'or, de l'or d'Ophir, et sept mille talents d'argent affiné, pour revêtir les murailles de la maison ;
\VS{5}afin qu'il y ait de l'or partout où il faut de l'or, et de l'argent partout où il faut de l'argent ; et pour tout l'ouvrage qui se fera par la main des ouvriers. Or qui est celui d'entre vous qui se disposera volontairement à offrir aujourd'hui libéralement à Yahweh ?
\VS{6}Alors les chefs des maisons paternelles, les chefs des tribus d'Israël, les chefs de milliers et de centaines et les intendants du roi offrirent volontairement.
\VS{7}Ils donnèrent pour le service de la maison de Dieu cinq mille talents et dix mille drachmes d'or, dix mille talents d'argent, dix-huit mille talents d'airain, et cent mille talents de fer.
\VS{8}Ils mirent aussi les pierres que chacun avait, pour le trésor de la maison de Yahweh, entre les mains de Jehiel, le Guerschonite.
\VS{9}Et le peuple offrait avec joie volontairement, car ils offraient de tout leur cœur leurs offrandes volontaires à Yahweh; et David en eut une très grande joie.
\TextTitle{Prières de David}
\VS{10}Puis David bénit Yahweh en présence de toute l'assemblée, et dit : Ô Yahweh, Dieu d'Israël, notre père ! Tu es béni de tout temps et à toujours.
\VS{11}Ô Yahweh ! C’est à toi qu'appartient la magnificence, la puissance, la gloire, l'éternité, et la majesté; car tout ce qui est aux cieux et sur la terre est à toi, ô Yahweh ! Le règne est à toi, et tu t'élèves en souverain au-dessus de toutes choses !
\VS{12}Les richesses et les honneurs viennent de toi, et tu as la domination sur toutes choses ; la force et la puissance sont dans ta main, et il est aussi du pouvoir de ta main d'agrandir et de fortifier toutes choses.
\VS{13}Maintenant donc, ô notre Dieu ! Nous te célébrons et nous louons ton Nom glorieux.
\VS{14}Mais qui suis-je, et qui est mon peuple, que nous ayons assez pour pouvoir t’offrir ces choses volontairement ? Car toutes choses viennent de toi, et les ayant reçues de ta main, nous te les présentons.
\VS{15}Et même nous sommes devant toi des étrangers et des habitants, comme ont été tous nos pères ; et nos jours sont comme l'ombre sur la terre, et il n'y a point d’espérance.
\VS{16}Yahweh, notre Dieu,  toute cette abondance que nous avons préparée pour bâtir une maison à ton saint Nom, est de ta main, et toutes ces choses sont à toi.
\VS{17}Et je sais, ô mon Dieu, que c'est toi qui sondes les cœurs, et que tu prends plaisir à la droiture. C'est pourquoi j'ai volontairement offert d'un cœur droit toutes ces choses, et j'ai vu maintenant avec joie que ton peuple, qui se trouve ici, t'a fait son offrande volontairement.
\VS{18}Ô Yahweh ! Dieu d'Abraham, d'Isaac et d'Israël, nos pères, conserve à toujours dans le cœur de ton peuple, ces dispositions et ces pensées, et affermis leurs cœurs en toi.
\VS{19}Donne aussi un cœur droit à Salomon, mon fils, afin qu'il garde tes commandements, tes préceptes et tes lois, et qu'il fasse tout ce qui est nécessaire et qu'il bâtisse le palais que j'ai préparé.
\TextTitle{Sacrifices en l’honneur de Yahweh ; Salomon oint roi\FTNTT{1 Ch. 23:1 ; 1 R. 2:12 ; 1 R. 1:32-37}}
\VS{20}Après cela, David dit à toute l'assemblée : Bénissez maintenant Yahweh, votre Dieu ! Et toute l'assemblée bénit Yahweh, le Dieu de leurs pères. Ils s'inclinèrent et se prosternèrent devant Yahweh et devant le roi.
\VS{21}Et le lendemain, ils offrirent des sacrifices à Yahweh, et des holocaustes ; à savoir mille veaux, mille moutons, et mille agneaux, avec leurs libations ; et des sacrifices en grand nombre pour tous ceux d'Israël.
\VS{22}Et ils mangèrent et burent ce jour-là devant Yahweh avec une grande joie ; et ils établirent roi pour la seconde fois Salomon, fils de David, et l'oignirent en l'honneur de Yahweh pour être leur conducteur, et Tsadok pour sacrificateur.
\VS{23}Salomon s'assit donc sur le trône de Yahweh pour être roi à la place de David, son père. Il prospéra, car tout Israël lui obéit.
\VS{24}Et tous les chefs et les héros, et même tous les fils du roi David consentirent d'être les sujets du roi Salomon.
\VS{25}Ainsi, Yahweh éleva souverainement Salomon, à la vue de tout Israël, et lui donna une majesté royale telle qu'aucun roi avant lui n'en avait eue en Israël.
\TextTitle{Fin du règne de David ; sa mort\FTNTT{2 S. 5:4-5 ; 1 R. 2:10-12 ; 1 Ch. 3:4}}
\VS{26}David donc, fils d'Isaï, régna sur tout Israël.
\VS{27}Et les jours qu'il régna sur Israël furent quarante ans ; il régna sept ans à Hébron et trente-trois ans à Jérusalem.
\VS{28}Puis il mourut dans une heureuse vieillesse, rassasié de jours, de richesses, et de gloire. Et Salomon, son fils, régna à sa place.
\VS{29}Les actions du roi David, tant les premières que les dernières, sont écrites dans le livre de Samuel le voyant, dans le livre de Nathan le prophète, et dans le livre de Gad le prophète,
\VS{30}avec tout son règne, ses exploits et ce qui se passa de son temps, tant sur Israël que sur tous les royaumes du territoire.
\PPE{}
\end{multicols}

%\clearpage\ShortTitle{2 Chroniques}\BookTitle{2 Chroniques}\BFont
\noindent\hrulefill
{\footnotesize
\textit{
\bigskip
{\centering{}
\\Auteur : Inconnu
\\(Heb. : Hayyamim dibre)
\\Signification : Actes des journées
\\Thème : La grandeur de Juda
\\Date de rédaction : 5\up{ème} siècle av. J.-C.\\}
}
%\bigskip
\textit{
\\Initialement, 1 et 2 Chroniques ne constituaient qu'un seul ouvrage. Ce livre raconte le règne de Salomon, la construction de la maison de Dieu et du palais. Il reprend ensuite l'histoire des royaumes d'Israël et de Juda, du schisme à la captivité babylonienne, mettant en exergue l'instabilité du peuple dont le cœur balançait entre Yahweh et les idoles.\bigskip
}
}
\par\nobreak\noindent\hrulefill
\begin{multicols}{2}
\Chap{1}
\TextTitle{Yahweh élève Salomon qui demande la sagesse\FTNTT{1 R. 2:12 ; 3:4-9 ; 1 Ch. 29:23-25}}
\VerseOne{}Or Salomon, fils de David, se fortifia dans son royaume ; Yahweh, son Dieu, fut avec lui, et l'éleva au plus haut.
\VS{2}Salomon parla à tout Israël, aux chefs de milliers et de centaines, aux juges et à tous les principaux de tout Israël, chefs des pères.
\VS{3}Salomon et toute l'assemblée avec lui allèrent au haut lieu qui était à Gabaon ; car là était la tente d'assignation de Dieu, que Moïse, serviteur de Yahweh, avait faite dans le désert.
\VS{4}Mais David avait fait monter l'arche de Dieu de Kirjath-Jearim au lieu qu'il avait préparé ; car il lui avait dressé une tente à Jérusalem.
\VS{5}L'autel d'airain que Betsaleel, fils d'Uri, fils de Hur, avait fait, était là devant le tabernacle de Yahweh. Et Salomon et l'assemblée y cherchèrent Yahweh\FTNT{Ex. 27:1-8 ; Ex. 36:1-2.}.
\VS{6}Salomon offrit là, devant Yahweh, mille holocaustes, sur l'autel d'airain qui était devant la tente d'assignation.
\VS{7}En cette nuit-là, Dieu apparut à Salomon, et lui dit : Demande ce que tu veux que je te donne.
\VS{8}Et Salomon répondit à Dieu : Tu as usé d'une grande bienveillance envers David, mon père, et tu m'as établi roi à sa place.
\VS{9}Maintenant, ô Yahweh Dieu ! Que ta parole à David, mon père, se confirme ; car tu m'as établi roi sur un peuple nombreux comme la poussière de la terre.
\VS{10}Donne-moi donc maintenant de la sagesse et de l'intelligence, afin que je sache me conduire devant ce peuple ; car qui pourrait juger ton peuple, ce peuple si grand ?
\TextTitle{Yahweh agrée la prière de Salomon et l'exauce\FTNTT{1 R. 3:10-28}}
\VS{11}Et Dieu dit à Salomon : Puisque c'est là ce qui est dans ton cœur, et que tu n'as demandé ni des richesses, ni des biens, ni de la gloire, ni la mort de ceux qui te haïssent, ni même des jours nombreux, mais que tu as demandé pour toi de la sagesse et de l'intelligence, afin de pouvoir juger mon peuple, sur lequel je t'ai établi roi,
\VS{12}la sagesse et l'intelligence te sont données. Je te donnerai aussi des richesses, des biens et de la gloire, comme n'en ont pas eu les rois qui ont été avant toi, et comme il n'en aura aucun après toi.
\VS{13}Puis Salomon s'en retourna à Jérusalem, du haut lieu qui était à Gabaon devant la tente d'assignation ; et il régna sur Israël.
\VS{14}Salomon rassembla des chars et des cavaliers ; il avait quatorze cents chars et douze mille cavaliers ; et il les plaça dans les villes où il tenait ses chars, et auprès du roi, à Jérusalem.
\VS{15}Et le roi fit que l'argent et l'or étaient aussi communs à Jérusalem que les pierres, et les cèdres que les sycomores de la plaine.
\VS{16}Le lieu d'où étaient issus les chevaux de Salomon était l'Egypte ; une caravane de marchands du roi allait les prendre par troupe à un prix convenu.
\VS{17}On faisait monter et sortir d'Egypte un char pour six cents sicles d'argent, et un cheval pour cent cinquante. On en amenait de même par eux pour tous les rois des Héthiens, et pour les rois de Syrie.
\Chap{2}
\TextTitle{La prière de Salomon exaucée\FTNTT{1 R. 5:1-18 ; 7:13,14}}
\VerseOne{}Or Salomon ordonna de bâtir une maison au Nom de Yahweh, ainsi qu'une maison royale.
\VS{2}Et il fit un dénombrement de soixante et dix mille hommes qui portaient les fardeaux, et de quatre vingt mille qui coupaient le bois sur la montagne, et de trois mille six cents qui étaient commis sur eux.
\VS{3}Puis Salomon envoya vers Huram, roi de Tyr, pour lui dire : Fais pour moi comme tu as fait pour David, mon père, à qui tu as envoyé des cèdres, pour se bâtir une maison afin d'y habiter.
\VS{4}Voici, je vais bâtir une maison au Nom de Yahweh, mon Dieu, pour la lui consacrer, pour faire brûler devant lui le parfum des aromates, pour présenter continuellement devant lui les pains de proposition, et pour offrir les holocaustes du matin et du soir, des sabbats, des nouvelles lunes, et des fêtes de Yahweh, notre Dieu, ce qui est perpétuel en Israël.
\VS{5}La maison que je vais bâtir sera grande ; car notre Dieu est plus grand que tous les dieux.
\VS{6}Mais qui aurait le pouvoir de lui bâtir une maison, puisque les cieux et les cieux des cieux ne sauraient le contenir ? Et qui suis-je pour lui bâtir une maison, si ce n'est pour faire brûler des parfums devant sa face ?
\VS{7}Maintenant, envoie-moi un homme habile pour travailler l'or, l'argent, l'airain et le fer, en écarlate, en cramoisi et en pourpre, sachant faire des sculptures, pour travailler avec les hommes habiles que j'ai avec moi en Juda et à Jérusalem, et que David, mon père, a préparés.
\VS{8}Envoie-moi aussi du Liban du bois de cèdre, de cyprès et de santal ; car je sais que tes serviteurs savent couper les bois du Liban. Voici, mes serviteurs seront avec les tiens.
\VS{9}Qu'on me prépare du bois en grande quantité ; car la maison que je vais bâtir sera grande et magnifique.
\VS{10}Et je donnerai à tes serviteurs qui couperont, qui abattront les bois, vingt mille cors de froment foulé, vingt mille cors d'orge, vingt mille baths de vin, et vingt mille baths d'huile.
\VS{11}Huram, roi de Tyr, répondit dans un écrit qu'il envoya à Salomon : C'est parce que Yahweh aime son peuple qu'il t'a établi roi sur eux.
\VS{12}Et Huram dit : Béni soit Yahweh, le Dieu d'Israël, qui a fait les cieux et la terre, de ce qu'il a donné au roi David un fils sage, prudent et intelligent, qui va bâtir une maison à Yahweh, et une maison royale !
\VS{13}Je t'envoie donc un homme habile et intelligent, Huram-Abi,
\VS{14}fils d'une femme d'entre les filles de Dan, et d'un père tyrien. Il sait travailler l'or, l'argent, l'airain et le fer, les pierres et le bois, en écarlate, en pourpre, en fin lin et en cramoisi ; il sait faire toutes sortes de sculptures et imaginer toutes sortes d'objets d'art qu'on lui donne à faire. Il travaillera avec tes hommes habiles et avec les hommes habiles de mon seigneur David, ton père.
\VS{15}Et maintenant, que mon seigneur envoie à ses serviteurs le froment, l'orge, l'huile et le vin comme il l'a dit.
\VS{16}Et nous couperons des bois du Liban autant que tu en auras besoin, et nous te les amènerons en radeaux, par la mer, jusqu'à Japho, et tu les feras monter à Jérusalem.
\VS{17}Alors Salomon compta tous les hommes étrangers qui étaient au pays d'Israël, d'après le dénombrement que David, son père, en avait fait. On en trouva cent cinquante-trois mille six cents.
\VS{18}Et il en établit soixante-dix mille qui portaient des fardeaux, quatre-vingt mille qui taillaient les pierres dans la montagne, et trois mille six cents surveillants pour faire travailler le peuple.
\Chap{3}
\TextTitle{Salomon commence la construction du temple\FTNTT{1 R. 6:1}}
\VerseOne{}Salomon commença donc à bâtir la maison de Yahweh à Jérusalem, sur la montagne de Morija, qui avait été indiquée à David, son père, au lieu même que David avait préparé dans l'aire d'Ornan, le Jébusien.
\VS{2}Il commença à bâtir, le second jour du second mois, la quatrième année de son règne.
\TextTitle{Les matériaux du temple et les dimensions\FTNTT{1 R. 6:2-38 ; 7:13-22}}
\VS{3}Or voici les fondements fixés par Salomon pour bâtir la maison de Dieu : La longueur, en coudées de l'ancienne mesure, était de soixante coudées, et la largeur de vingt coudées.
\VS{4}Le portique qui était sur le devant, et dont la longueur répondait à la largeur de la maison, avait vingt coudées, et cent vingt de hauteur. Il le revêtit intérieurement d'or pur.
\VS{5}Et il recouvrit la grande maison de bois de cyprès ; il la revêtit d'or fin, et y fit mettre des palmes et des chaînettes.
\VS{6}Il revêtit la maison de pierres précieuses, pour l'ornement ; et l'or était de l'or de Parvaïm.
\VS{7}Il revêtit d'or la maison, les poutres, les seuils, les parois et les portes ; et il fit sculpter des chérubins sur les parois.
\VS{8}Il fit le Saint des saints, dont la longueur était de vingt coudées, selon la largeur de la maison, et la largeur de vingt coudées ; et il le couvrit d'or fin, pour une valeur de six cents talents.
\VS{9}Et le poids des clous montait à cinquante sicles d'or. Il revêtit aussi d'or les chambres hautes.
\VS{10}Il fit dans le Saint des saints deux chérubins sculptés, et on les couvrit d'or ;
\VS{11}La longueur des ailes des chérubins était de vingt coudées. L'aile du premier, longue de cinq coudées, touchait la paroi de la maison, et l'autre aile, longue de cinq coudées, touchait une aile de l'autre chérubin.
\VS{12}Et une aile de l'autre chérubin, longue de cinq coudées, touchait la paroi de la maison ; et l'autre aile longue de cinq coudées, joignait l'aile de l'autre chérubin.
\VS{13}Les ailes étendues de ces chérubins faisaient vingt coudées. Ils se tenaient debout sur leurs pieds, leurs faces tournées vers la maison.
\VS{14}Il fit le voile de pourpre, d'écarlate, de cramoisi et de fin lin: Et il y représenta par dessus des chérubins.
\VS{15}Devant la maison, il fit deux colonnes de trente-cinq coudées de hauteur, et le chapiteau sur leur sommet était de cinq coudées.
\VS{16}Il fit des chaînes dans le sanctuaire ; et il en mit sur le sommet des colonnes ; et il fit cent grenades qu'il mit aux chaînes.
\VS{17}Il dressa les colonnes sur le devant du temple, l'une à droite, et l'autre à gauche ; il appela celle de droite Jakin, et celle de gauche Boaz.
\Chap{4}
\TextTitle{L'autel d'airain, la mer de fonte et les ustensiles du temple\FTNTT{1 R. 7:23-50}}
\VerseOne{}Il fit aussi un autel d'airain\FTNT{Voir l'annexe « Le temple de Salomon - extérieur »} long de vingt coudées, large de vingt coudées, et haut de dix coudées.
\VS{2}Il fit la mer de fonte de dix coudées d'un bord à l'autre, ronde tout autour, et haute de cinq coudées, et une circonférence que mesurait un cordon de trente coudées.
\VS{3}Des figures de bœufs l'entouraient en dessous, dix par coudée, faisant tout le tour de la mer ; il y avait deux rangées de bœufs fondus avec elle en une seule pièce.
\VS{4}Elle était posée sur douze bœufs, dont trois tournés vers le nord, trois tournés vers l'occident, trois tournés vers le sud, et trois tournés vers l'orient. La mer était sur eux, et toute la partie postérieure de leur corps était en dedans.
\VS{5}Son épaisseur était d'une paume ; et son bord était comme le bord d'une coupe en fleur de lis. Elle avait une contenance de trois mille baths\FTNT{Ex 25 ; Ex 27.}.
\VS{6}Il fit aussi dix cuves, et en mit cinq à droite et cinq à gauche, pour servir à la purification. On y lavait ce qui appartenait aux holocaustes, et la mer servait aux sacrificateurs pour s'y laver.
\VS{7}Il fit dix chandeliers d'or, d'après l'ordonnance, et les mit dans le temple, cinq à droite et cinq à gauche.
\VS{8}Il fit aussi dix tables, et il les mit dans le temple, cinq à droite et cinq à gauche. Il fit cent coupes d'or.
\VS{9}Il fit encore le parvis des sacrificateurs, le grand parvis et des portes pour ce parvis, et couvrit d'airain ces portes.
\VS{10}Il mit la mer du côté droit, vers l'orient, face au sud-est.
\VS{11}Et Huram fit les cuves, les pelles et les bassins. Huram acheva de faire l'ouvrage qu'il faisait pour le roi Salomon dans la maison de Dieu :
\VS{12}Deux colonnes, les bourrelets et les deux chapiteaux sur le sommet des colonnes ; les deux maillages pour couvrir les deux bourrelets des chapiteaux sur le sommet des colonnes ;
\VS{13}et les quatre cents grenades pour les deux maillages, deux rangs de grenades à chaque maille, pour couvrir les deux bourrelets des chapiteaux sur le sommet des colonnes.
\VS{14}Il fit aussi les bases, et il fit les cuves sur les bases ;
\VS{15}la mer et les douze bœufs sous elle ;
\VS{16}les pots, les pelles et les fourchettes et tous leurs ustensiles ; Huram-Abi les fit au roi Salomon, pour la maison de Yahweh, en airain poli.
\VS{17}Le roi les fit fondre dans la plaine du Jourdain, dans une terre grasse, entre Succoth et Tseréda.
\VS{18}Et Salomon fit tous ces ustensiles en si grand nombre qu'on ne rechercha point le poids de l'airain.
\VS{19}Salomon fit encore tous les ustensiles\FTNT{Voir l'annexe « Le temple de Salomon - intérieur »} qui étaient dans la maison de Yahweh : L'autel d'or, et les tables sur lesquelles on mettait le pain de proposition ;
\VS{20}les chandeliers et leurs lampes d'or fin, qu'on devait allumer devant le sanctuaire, selon l'ordonnance ;
\VS{21}les fleurs, les lampes, et les mouchettes d'or, d'un or parfaitement pur ;
\VS{22}et les mouchettes, les bassins, les tasses et les encensoirs d'or fin. Quant à l'entrée de la maison, les portes intérieures conduisant dans le Saint des saints, et les portes de la maison pour entrer au temple étaient d'or.
\Chap{5}
\TextTitle{L'arche dans le sanctuaire, Yahweh manifeste sa gloire\FTNTT{1 R. 7:51-8:11}}
\VerseOne{}Ainsi fut achevé tout l'ouvrage que Salomon fit pour la maison de Yahweh. Puis Salomon fit apporter ce que David, son père, avait consacré : L'argent, l'or et tous les ustensiles ; et il les mit dans les trésors de la maison de Dieu.
\VS{2}Alors Salomon assembla à Jérusalem les anciens d'Israël, et tous les chefs des tribus, les chefs des pères des fils d'Israël, pour transporter de la ville de David, qui est Sion, l'arche de l'alliance de Yahweh.
\VS{3}Et tous les hommes d'Israël s'assemblèrent auprès du roi pour la fête ; c'était le septième mois.
\VS{4}Tous les anciens d'Israël vinrent, et les Lévites portèrent l'arche.
\VS{5}Ils transportèrent l'arche, la tente d'assignation, et tous les ustensiles sacrés qui étaient dans la tente ; les sacrificateurs et les Lévites les emportèrent.
\VS{6}Or le roi Salomon et toute l'assemblée d'Israël réunie auprès de lui étaient devant l'arche, sacrifiant du menu et du gros bétail en si grand nombre qu'on ne pouvait ni dénombrer ni compter.
\VS{7}Les sacrificateurs portèrent l'arche de l'alliance de Yahweh à sa place, dans le sanctuaire de la maison, dans le Saint des saints, sous les ailes des chérubins.
\VS{8}Les chérubins étendaient les ailes sur l'endroit où devait être l'arche, et les chérubins couvraient l'arche et ses barres par-dessus.
\VS{9}Les barres avaient une longueur telle que leurs extrémités se voyaient en avant de l'arche, devant le sanctuaire ; mais elles ne se voyaient point du dehors. Et l'arche a été là jusqu'à ce jour.
\VS{10}Il n'y avait dans l'arche que les deux tables que Moïse y avait mises en Horeb, quand Yahweh traita alliance avec les enfants d'Israël à leur sortie d'Egypte.
\VS{11}Or il arriva que comme les sacrificateurs sortaient du lieu saint (car tous les sacrificateurs présents s'étaient sanctifiés, sans observer l'ordre des classes),
\VS{12}etque tous les Lévites qui étaient chantres, Asaph, Héman, Jeduthun, leurs fils et leurs frères, vêtus de fin lin, avec des cymbales, des luths et des harpes, se tenaient à l'orient de l'autel ; et il y avait avec eux cent vingt sacrificateurs sonnant des trompettes.
\VS{13}Il arriva, dis-je, que comme un seul homme, ceux qui sonnaient des trompettes et ceux qui chantaient firent entendre leur voix d'un même accord, pour célébrer et pour louer Yahweh, et firent retentir le son des trompettes, des cymbales et d'autres instruments de musique, et ils célébrèrent Yahweh, en disant : Car il est bon, car sa miséricorde demeure à toujours\FTNT{Jé. 33:11 ; Ps. 118:29 ;  Ps. 136} ! Il arriva que la maison de Yahweh fut remplie d'une nuée.
\VS{14}Les sacrificateurs ne purent s'y tenir pour faire le service, à cause de la nuée ; car la gloire de Yahweh remplissait la maison de Dieu.
\Chap{6}
\TextTitle{Salomon s'adresse à l'assemblée d'Israël\FTNTT{1 R. 8:12-21}}
\VerseOne{}Alors Salomon dit : Yahweh a dit qu'il habiterait dans l'obscurité\FTNT{Nous avons ici une prophétie concernant la venue du Messie. Dieu, qui est lumière, a accepté d'habiter dans les ténèbres afin de nous sauver (Mt. 4:16 ; Jn. 1:5).}.
\VS{2}Et moi, j'ai bâti une maison qui sera ta demeure, et un domicile afin que tu y résides à toujours !
\VS{3}Puis le roi tourna son visage, et bénit toute l'assemblée d'Israël ; et toute l'assemblée d'Israël était debout.
\VS{4}Et il dit : Béni soit Yahweh, le Dieu d'Israël, qui de sa bouche a parlé à David, mon père, et qui par sa main puissante accomplit ce qu'il avait déclaré en disant :
\VS{5}Depuis le jour où j'ai fait sortir mon peuple du pays d'Egypte, je n'ai point choisi de ville entre toutes les tribus d'Israël pour y bâtir une maison afin que mon Nom y réside, et je n'ai point choisi d'homme pour être chef de mon peuple d'Israël.
\VS{6}Mais j'ai choisi Jérusalem pour que mon Nom y réside, et j'ai choisi David pour qu'il règne sur mon peuple d'Israël.
\VS{7}Or David, mon père, avait à cœur de bâtir une maison au Nom de Yahweh, le Dieu d'Israël.
\VS{8}Mais Yahweh parla à David, mon père : Puisque tu as eu à cœur de bâtir une maison à mon Nom, tu as bien fait d'avoir eu cette intention.
\VS{9}Seulement, ce n'est pas toi qui bâtiras cette maison ; mais ce sera ton fils, qui sortira de tes entrailles, qui bâtira cette maison à mon Nom.
\VS{10}Yahweh a accompli la parole qu'il avait déclarée ; j'ai succédé à David, mon père, et je me suis assis sur le trône d'Israël, comme Yahweh l'avait dit, et j'ai bâti cette maison au Nom de Yahweh, le Dieu d'Israël.
\VS{11}J'y ai mis l'arche où est l'alliance de Yahweh, qu'il traita avec les enfants d'Israël.
\TextTitle{Prière de Salomon\FTNTT{1 R. 8:22-61}}
\VS{12}Puis il se plaça devant l'autel de Yahweh, en face de toute l'assemblée d'Israël, et il étendit ses mains.
\VS{13}Car Salomon avait fait une tribune d'airain, et il l'avait mise au milieu du grand parvis ; elle était longue de cinq coudées, large de cinq coudées, et haute de trois coudées. Il s'y plaça, se mit à genoux en face de toute l'assemblée d'Israël, et étendant ses mains vers les cieux, il dit :
\VS{14}Ô Yahweh, Dieu d'Israël ! Il n'y a ni dans les cieux ni sur la terre de Dieu semblable à toi, qui gardes l'alliance et la miséricorde envers tes serviteurs qui marchent de tout leur cœur devant ta face.
\VS{15}Toi qui as tenu parole à ton serviteur David, mon père. Ce que tu lui avais promis, et ce que tu as déclaré de ta bouche, tu l'as accompli de ta main puissante, comme il paraît aujourd'hui.
\VS{16}Maintenant, ô Yahweh, Dieu d'Israël ! Tiens la parole que tu as faite à ton serviteur David, mon père, en disant : Tu ne manqueras jamais devant moi d'un successeur assis sur le trône d'Israël, pourvu que tes fils prennent garde à leur voie pour marcher dans ma loi, comme tu as marché devant ma face.
\VS{17}Et maintenant, ô Yahweh, Dieu d'Israël ! Que ta parole, que tu as déclarée à David, ton serviteur, soit confirmée !
\VS{18}Mais Dieu habiterait-il véritablement sur la terre avec les hommes ? Voici, les cieux, même les cieux des cieux, ne peuvent te contenir, combien moins cette maison que j'ai bâtie !
\VS{19}Toutefois, ô Yahweh, mon Dieu, aie égard à la prière de ton serviteur et à sa supplication, pour écouter le cri et la prière que ton serviteur t'adresse.
\VS{20}Que tes yeux soient ouverts jour et nuit sur cette maison, sur le lieu où tu as promis de mettre ton Nom ! Ecoute la prière que ton serviteur te fait en ce lieu.
\VS{21}Exauce les supplications de ton serviteur et de ton peuple d'Israël, quand ils prieront en ce lieu. Exauce des cieux, du lieu de ta demeure ; exauce et pardonne !
\VS{22}Si quelqu'un pèche contre son prochain, et qu'on lui impose un serment pour le faire jurer, et qu'il vient prêter serment devant ton autel, dans cette maison ;
\VS{23}écoute-le des cieux, agis et juge tes serviteurs, en donnant au méchant son salaire, et fais retomber sa conduite sur sa tête, en justifiant le juste, et lui rendant selon sa justice.
\VS{24}Quand ton peuple d'Israël sera battu par l'ennemi, pour avoir péché contre toi ; s'ils retournent à toi, s'ils donnent gloire à ton Nom, s'ils t'adressent dans cette maison des prières et des supplications ;
\VS{25}toi, exauce-les des cieux, et pardonne le péché de ton peuple d'Israël, et ramène-les dans la terre que tu leur as donnée à eux et à leurs pères.
\VS{26}Quand les cieux seront fermés, et qu'il n'y aura point de pluie, parce qu'ils auront péché contre toi ; s'ils prient en ce lieu, s'ils donnent gloire à ton Nom, et s'ils se détournent de leurs péchés, parce que tu les auras affligés ;
\VS{27}toi, exauce-les des cieux, et pardonne le péché de tes serviteurs et de ton peuple d'Israël, après que tu leur auras enseigné le bon chemin, par lequel ils doivent marcher ; et envoie de la pluie sur la terre que tu as donnée en héritage à ton peuple.
\VS{28}Quand il y aura dans le pays la famine ou la peste, quand il y aura la rouille, la nielle, les sauterelles d'une espèce ou d'une autre, quand les ennemis les assiégeront dans leur pays, dans leurs portes, ou qu'il y aura un fléau, une maladie quelconque ;
\VS{29}si un homme, si tout ton peuple d'Israël fait entendre des prières et des supplications, et que chacun reconnaît sa plaie et sa douleur, et étend ses mains vers cette maison ;
\VS{30}exauce-le des cieux, du lieu de ta demeure, et pardonne. Rends à chacun selon toutes ses voies, toi qui connais leur cœur ; car seul tu connais le cœur des fils des hommes ;
\VS{31}afin qu'ils te craignent, pour marcher dans tes voies, tout le temps qu'ils vivront sur la terre que tu as donnée à nos pères.
\VS{32}Et l'étranger, qui ne sera pas de ton peuple d'Israël, mais qui viendra d'un pays éloigné, à cause de ton grand Nom, de ta main puissante, et de ton bras étendu ; quand il viendra prier dans cette maison,
\VS{33}exauce-le des cieux, du lieu de ta demeure, et accorde tout ce que cet étranger réclamera de toi ; afin que tous les peuples de la terre connaissent ton Nom pour te craindre comme ton peuple d'Israël, et sachent que ton Nom est invoqué sur cette maison que j'ai bâtie.
\VS{34}Quand ton peuple sortira en guerre contre ses ennemis, par la voie par laquelle tu l'auras envoyé ; s'ils te prient, en regardant vers cette ville que tu as choisie, et vers cette maison que j'ai bâtie à ton Nom,
\VS{35}exauce des cieux leur prière et leur supplication, et fais-leur droit.
\VS{36}Quand ils pécheront contre toi, car il n'y a point d'homme qui ne pèche, et qu'irrité contre eux, tu les auras livrés à leurs ennemis, et que ceux qui les auront pris les auront emmenés captifs en quelque pays, soit éloigné soit proche ;
\VS{37}si dans le pays où ils seront captifs, ils rentrent en eux-mêmes et s'ils se repentent, s'ils t'adressent des supplications dans le pays de leur captivité, en disant : Nous avons péché, nous avons commis l'iniquité, nous avons agi méchamment !
\VS{38}S'ils retournent à toi de tout leur cœur et de toute leur âme, dans le pays de leur captivité où ils ont été emmenés captifs, et s'ils t'adressent des prières, les regards tournés vers leur pays que tu as donné à leurs pères, vers cette ville que tu as choisie, et vers cette maison que j'ai bâtie à ton Nom ;
\VS{39}exauce des cieux, du lieu de ta demeure, leurs prières et leurs supplications, et fais-leur droit ; pardonne à ton peuple qui aura péché contre toi !
\VS{40}Maintenant, ô mon Dieu, que tes yeux soient ouverts et que tes oreilles soient attentives à la prière qu'on te fera en ce lieu !
\VS{41}Et maintenant, Yahweh Dieu ! Lève-toi, viens au lieu de ton repos, toi et l'arche de ta puissance. Yahweh Dieu, que tes sacrificateurs soient revêtus du salut, et que tes bien-aimés se réjouissent du bien que tu leur fais !
\VS{42}Yahweh Dieu, ne repousse la face pas ton oint ; souviens-toi des grâces accordées à David, ton serviteur.
\Chap{7}
\TextTitle{Yahweh répond par le feu : Sa gloire remplit la maison}
\VerseOne{}Lorsque Salomon eut achevé de prier, le feu descendit du ciel et consuma l'holocauste et les sacrifices\FTNT{Lé 9:24 ; 1 R 18:38.} ; et la gloire de Yahweh remplit la maison.
\VS{2}Les sacrificateurs ne pouvaient entrer dans la maison de Yahweh, parce que la gloire de Yahweh avait rempli la maison de Yahweh.
\VS{3}Tous les enfants d'Israël virent descendre le feu et la gloire de Yahweh sur la maison ; et ils se courbèrent, le visage contre terre, sur le pavé, se prosternèrent et louèrent Yahweh, en disant : Car il est bon, car sa miséricorde demeure éternellement !
\TextTitle{Salomon et le peuple offrent des sacrifices à Yahweh\FTNTT{1 R. 8:62-66}}
\VS{4}Or le roi et tout le peuple offraient des sacrifices devant Yahweh.
\VS{5}Le roi Salomon offrit un sacrifice de vingt-deux mille bœufs, et cent vingt mille brebis. Ainsi, le roi et tout le peuple firent la dédicace de la maison de Dieu.
\VS{6}Les sacrificateurs se tenaient à leurs fonctions, ainsi que les Lévites, avec les instruments de musique de Yahweh, que le roi David avait faits pour louer Yahweh en disant : Car sa miséricorde demeure éternellement ; ayant les Psaumes de David entre leurs mains. Et les sacrificateurs sonnaient des trompettes vis-à-vis d'eux, et tout Israël se tenait debout.
\VS{7}Salomon consacra le milieu du parvis, qui est devant la maison de Yahweh ; car il offrit là les holocaustes et les graisses des sacrifices d'offrande de paix\FTNT{Voir commentaire en Lé. 3:1.}, parce que l'autel d'airain que Salomon avait fait ne pouvait contenir les holocaustes, les offrandes et les graisses.
\VS{8}Ainsi Salomon célébra, en ce temps-là, la fête pendant sept jours, avec tout Israël. Il y avait une grande multitude, venue depuis l'entrée d'Hamath jusqu'au torrent d'Egypte.
\VS{9}Le huitième jour, ils firent une assemblée solennelle ; car ils firent la dédicace de l'autel pendant sept jours, et la fête pendant sept jours.
\VS{10}Le vingt-troisième jour du septième mois, il laissa aller le peuple dans ses tentes, se réjouissant et ayant le cœur plein de joie, à cause du bien que Yahweh avait fait à David, à Salomon, et à Israël, son peuple.
\TextTitle{Yahweh apparaît à Salomon\FTNTT{1 R. 9:1-9}}
\VS{11}Salomon acheva donc la maison de Yahweh et la maison du roi ; et Salomon réussit dans tout ce qui lui vint à cœur de faire dans la maison de Yahweh et dans sa maison.
\VS{12}Yahweh apparut à Salomon pendant la nuit, et lui dit : J'exauce ta prière, et je choisis ce lieu comme une maison de sacrifices.
\VS{13}Quand je fermerai les cieux, et qu'il n'y aura point de pluie, et quand j'ordonnerai aux sauterelles de consumer le pays, et quand j'enverrai la peste parmi mon peuple ;
\VS{14}si mon peuple, sur lequel mon Nom est invoqué, s'humilie, prie, et cherche ma face, et s'il se détourne de ses mauvaises voies, alors je l'exaucerai des cieux, je pardonnerai ses péchés, et je guérirai son pays.
\VS{15}Mes yeux seront désormais ouverts, et mes oreilles seront attentives à la prière faite en ce lieu.
\VS{16}Maintenant je choisis et je sanctifie cette maison, afin que mon Nom y soit à toujours ; mes yeux et mon cœur seront toujours là.
\VS{17}Et toi, si tu marches devant moi comme David, ton père, a marché, faisant tout ce que je t'ai ordonné, et si tu gardes mes lois et mes ordonnances,
\VS{18}j'affermirai le trône de ton royaume, comme je l'ai déclaré à David, ton père, en disant : Il ne te manquera point de successeur qui règne en Israël.
\VS{19}Mais si vous vous détournez, et si vous abandonnez mes lois et mes commandements que je vous ai prescrits, et si vous allez servir d'autres dieux et vous prosterner devant eux,
\VS{20}je vous arracherai de mon pays que je vous ai donné, je rejetterai loin de moi cette maison que j'ai consacrée à mon Nom, et j'en ferai un sujet de sarcasmes et de moqueries parmi tous les peuples.
\VS{21}Et quiconque passera près de cette maison qui aura été élevée, sera dans l'étonnement et dira : Pourquoi Yahweh a-t-il ainsi traité ce pays et cette maison?
\VS{22}Et on répondra : Parce qu'ils ont abandonné Yahweh, le Dieu de leurs pères, qui les a fait sortir du pays d'Egypte, et qu'ils se sont attachés à d'autres dieux, et qu'ils se sont prosternés devant eux, et les ont servis ; à cause de cela, il a fait venir sur eux tous ces maux.
\Chap{8}
\TextTitle{Les réalisations de Salomon\FTNTT{1 R. 9:15-28 ; 10:26-29}}
\VerseOne{}Au bout de vingt ans, pendant lesquels Salomon bâtit la maison de Yahweh et sa propre maison,
\VS{2}il bâtit les villes que Huram lui avait données et y fit habiter les enfants d'Israël.
\VS{3}Puis Salomon marcha contre Hamath de Tsoba, et la conquit.
\VS{4}Il bâtit Thadmor au désert, et toutes les villes servant de magasins qu'il bâtit dans le pays de Hamath.
\VS{5}Il bâtit Beth-Horon la haute, et Beth-Horon la basse, villes fortes de murailles, de portes et de barres ;
\VS{6}Baalath, et toutes les villes servant de magasins qu'avait Salomon, toutes les villes pour les chars, les villes pour la cavalerie, et tout ce que Salomon prit plaisir à bâtir à Jérusalem, au Liban, et dans tout le pays de sa domination.
\VS{7}Tout le peuple qui était resté des Héthiens, des Amoréens, des Phéréziens, des Héviens et des Jébusiens, qui n'étaient point d'Israël ;
\VS{8}leurs descendants, qui étaient restés après eux dans le pays, et que les enfants d'Israël n'avaient pas détruits, Salomon les leva comme des gens de corvée jusqu'à ce jour.
\VS{9}Salomon n'employa comme esclave pour ses travaux aucun des fils d'Israël ; car ils étaient des hommes de guerre, les chefs de ses officiers, les chefs de ses chars et de ses hommes d'armes.
\VS{10}Voici le nombre des chefs de ceux qui étaient préposés aux travaux du roi Salomon : Ils étaient deux cent cinquante, ayant autorité sur le peuple.
\VS{11}Salomon fit monter la fille de Pharaon de la cité de David dans la maison qu'il lui avait bâtie ; car il dit : Ma femme n'habitera point dans la maison de David, roi d'Israël, parce que les lieux où l'arche de Yahweh est entrée sont saints.
\VS{12}Alors Salomon offrit des holocaustes à Yahweh, sur l'autel de Yahweh qu'il avait bâti devant le portique.
\VS{13}Il offrait chaque jour ce qui était prescrit par Moïse pour les sabbats, pour les nouvelles lunes, et pour les fêtes, trois fois l'année, à la fête des pains sans levain, à la fête des semaines, et à la fête des tabernacles\FTNT{Ex. 14:17 ; Lé. 23:1-44.}.
\VS{14}Il établit, selon l'ordonnance de David, son père, les classes des sacrificateurs selon leur fonction, et les Lévites selon leurs charges, pour célébrer Yahweh et pour faire, jour par jour, le service en présence des sacrificateurs ; et les portiers, selon leurs classes, à chaque porte ; car tel était le commandement de David, homme de Dieu.
\VS{15}Et on ne s'écarta pas du commandement du roi à l'égard des Sacrificateurs et des Lévites, en aucune chose, ni à l'égard les trésors.
\VS{16}Ainsi fut préparé tout l'ouvrage de Salomon, jusqu'au jour de la fondation de la maison de Yahweh et jusqu'à ce qu'elle fut terminée. La maison de Yahweh fut donc achevée.
\VS{17}Alors Salomon alla à Etsjon-Guéber et à Eloth, sur le rivage de la mer, dans le pays d'Edom.
\VS{18}Et Huram lui envoya, sous la conduite de ses serviteurs, des navires et des serviteurs connaissant la mer. Ils allèrent avec les serviteurs de Salomon à Ophir, et ils y prirent quatre cent cinquante talents d'or, qu'ils apportèrent au roi Salomon.
\Chap{9}
\TextTitle{La reine de Séba chez Salomon\FTNTT{1 R. 10:1-13}}
\VerseOne{}Or la reine de Séba, ayant appris la renommée de Salomon, vint à Jérusalem pour éprouver Salomon par des énigmes. Elle avait une suite très nombreuse, et des chameaux portant des aromates, de l'or en grande quantité et des pierres précieuses. Elle vint auprès de Salomon, et elle lui parla de tout ce qu'elle avait dans le cœur.
\VS{2}Salomon lui expliqua tout ce qu'elle lui proposa ; il n'y eut rien que Salomon n'entendît et qu'il ne sût lui expliquer.
\VS{3}Alors, la reine de Séba vit toute la sagesse de Salomon, et la maison qu'il avait bâtie,
\VS{4}les mets de sa table, la demeure de ses serviteurs, l'ordre de service et les vêtements de ceux qui le servaient, ses échansons et leurs vêtements, et les marches par où l'on montait à la maison de Yahweh, et elle fut toute ravie hors d'elle-même.
\VS{5}Elle parla ainsi au roi : Ce que j'ai entendu dire dans mon pays de tes actions et de ta sagesse était donc vrai !
\VS{6}Je ne croyais pas ce qu'on en disait avant d'être venue et que mes yeux ne l'aient vu ; et voici, on ne m'avait pas rapporté la moitié de la grandeur de ta sagesse ; tu surpasses la rumeur que j'avais entendue.
\VS{7}Heureux tes gens ! Heureux tes serviteurs qui se tiennent continuellement devant toi, et qui entendent ta sagesse !
\VS{8}Béni soit Yahweh, ton Dieu, qui a pris plaisir en toi pour te placer sur son trône comme roi pour Yahweh, ton Dieu ! C'est parce que ton Dieu aime Israël et veut le faire subsister à jamais, qu'il t'a établi roi sur eux pour faire droit et justice.
\VS{9}Puis elle donna au roi cent vingt talents d'or, une très grande quantité d'aromates, et des pierres précieuses ; et il n'y eut plus d'aromates tels que ceux que la reine de Séba donna au roi Salomon.
\VS{10}Les serviteurs de Huram et les serviteurs de Salomon, qui amenèrent de l'or d'Ophir, amenèrent aussi du bois de santal et des pierres précieuses.
\VS{11}Le roi fit de ce bois de santal les chemins qui allaient à la maison de Yahweh et à la maison du roi, et des harpes et des luths pour les chantres. On n'en avait point vu auparavant de semblable dans le pays de Juda.
\VS{12}Le roi Salomon donna à la reine de Séba tout ce qu'elle désira, ce qu'elle demanda, plus qu'elle n'avait apporté au roi ; et elle s'en retourna, revint dans son pays, elle et ses serviteurs.
\TextTitle{Les richesses de Salomon\FTNTT{cp. 1 R. 4:1-34}}
\VS{13}Le poids de l'or qui arrivait à Salomon chaque année était de six cent soixante-six talents d'or,
\VS{14}outre ce qu'il retirait des négociants et des marchands qui en apportaient, et de tous les rois d'Arabie et des gouverneurs de ces pays-là, qui apportaient de l'or et de l'argent à Salomon.
\VS{15}Le roi Salomon fit deux cents grands boucliers d'or battu, employant six cents sicles d'or battu pour chaque bouclier ;
\VS{16}et trois cents autres boucliers plus petits d'or battu, employant trois cents sicles d'or pour chaque bouclier ; et le roi les mit dans la maison de la forêt du Liban.
\VS{17}Le roi fit aussi un grand trône d'ivoire, qu'il couvrit d'or pur.
\VS{18}Ce trône avait six marches et un marchepied d'or qui était accolé au trône ; et il avait des accoudoirs de l'un et de l'autre côté du siège ; et deux lions se tenaient auprès des accoudoirs.
\VS{19}Douze lions se tenaient là sur les six marches de part et d'autre. Rien de pareil n'avait été fait pour aucun royaume.
\VS{20}Et toutes les coupes à boire du roi Salomon étaient d'or, et toute la vaisselle de la maison de la forêt du Liban était d'or pur ; rien n'était d'argent ; on n'en faisait aucun cas du temps de Salomon.
\VS{21}Car les navires du roi allaient à Tarsis avec les serviteurs de Huram ; et une fois tous les trois ans arrivaient les navires de Tarsis, apportant de l'or, de l'argent, des dents d'éléphants, des singes et des paons.
\VS{22}Le roi Salomon fut plus grand que tous les rois de la terre, tant en richesses qu'en sagesse.
\VS{23}Tous les rois de la terre cherchaient à voir la face de Salomon, pour écouter la sagesse que Dieu avait mise dans son cœur.
\VS{24}Et chacun d'eux apportait son présent : Des ustensiles d'argent, des ustensiles d'or, des vêtements, des armes, des aromates, des chevaux et des mulets, et il en était ainsi année après année.
\VS{25}Salomon avait quatre mille écuries pour ses chevaux, avec des chars ; et douze mille cavaliers qu'il plaça dans les villes où il avait des chars et auprès du roi à Jérusalem.
\VS{26}Il dominait sur tous les rois depuis le fleuve jusqu'au pays des Philistins, et jusqu'à la frontière d'Egypte.
\VS{27}Et le roi fit que l'argent était aussi commun à Jérusalem que les pierres, et les cèdres aussi nombreux que les sycomores qui sont dans les plaines.
\VS{28}On tirait des chevaux pour Salomon de l'Egypte et de tous les pays.
\TextTitle{Mort de Salomon\FTNTT{1 R. 11:1-40}}
\VS{29}Le reste des actions de Salomon, les premières et les dernières, cela n'est-il pas écrit dans le livre de Nathan le prophète, dans la prophétie d'Achija de Silo, et dans la vision de Jéedo le voyant, touchant Jéroboam, fils de Nebath ?
\VS{30}Salomon régna quarante ans à Jérusalem sur tout Israël.
\VS{31}Puis Salomon s'endormit avec ses pères, et on l'ensevelit dans la cité de David, son père ; et Roboam, son fils, régna à sa place.
\Chap{10}
\TextTitle{Roboam règne sur Israël\FTNTT{1 R. 12:1-15}}
\VerseOne{}Roboam se rendit à Sichem, car tout Israël était venu à Sichem pour l'établir roi.
\VS{2}Quand Jéroboam, fils de Nebath, qui était en Egypte, où il s'était enfui de devant le roi Salomon, l'eut appris, il revint d'Egypte.
\VS{3}Or on l'envoya appeler. Ainsi Jéroboam et tout Israël vinrent et parlèrent à Roboam, en disant :
\VS{4}Ton père a mis sur nous un joug pesant. Allège maintenant cette rude servitude de ton père, et ce joug pesant qu'il a mis sur nous, et nous te servirons.
\VS{5}Alors il leur dit : Revenez vers moi dans trois jours. Et le peuple s'en alla.
\VS{6}Le roi Roboam demanda conseil aux vieillards qui avaient été auprès de Salomon, son père, pendant sa vie, et il leur parla ainsi : Comment, et quelle chose me conseillez-vous de répondre à ce peuple ?
\VS{7} Et ils lui répondirent en ces termes : Si tu es bon envers ce peuple, si tu es bienveillant envers eux, et que tu leur dises de bonnes paroles, ils seront tes serviteurs à toujours.
\VS{8}Mais il laissa le conseil que les vieillards lui avaient donné, et il demanda conseil aux jeunes gens qui avaient grandi avec lui, et qui se tenaient auprès de lui.
\VS{9}Et il leur dit : Que me conseillez-vous de répondre à ce peuple qui m'a parlé en disant : Allège le joug que ton père a mis sur nous ?
\VS{10}Et les jeunes gens qui avaient grandi avec lui, lui parlèrent en disant : Tu répondras en disant à ce peuple qui t'a parlé et t'a dit : Ton père a mis sur nous un joug pesant, mais toi, allège-le ; tu leur répondras donc : Mon petit doigt est plus gros que les reins de mon père.
\VS{11}Or mon père a mis sur vous un joug pesant, mais moi, je rendrai votre joug encore plus pesant. Mon père vous a châtiés avec des fouets, mais moi, je vous châtierai avec des scorpions.
\TextTitle{Roboam délaisse le conseil des anciens}
\VS{12}Trois jours après, Jéroboam, avec tout le peuple, vint vers Roboam, suivant ce qu'avait dit le roi : Revenez vers moi dans trois jours.
\VS{13}Mais le roi leur répondit durement. Le roi Roboam délaissa le conseil des anciens,
\VS{14}et leur parla suivant le conseil des jeunes gens, en disant : Mon père a mis sur vous un joug pesant ; mais moi, j'y ajouterai encore. Mon père vous a châtiés avec des fouets ; mais moi, je vous châtierai avec des scorpions.
\VS{15}Le roi n'écouta donc point le peuple ; cela était conduit par Dieu, afin que Yahweh accomplisse la parole qu'il avait déclarée par Achija de Silo, à Jéroboam, fils de Nebath.
\TextTitle{Israël se détache de la maison de David\FTNTT{1 R. 12:16-19}}
\VS{16}Quand tout Israël vit que le roi ne les écoutait pas, le peuple répondit au roi, en disant : Quelle part avons-nous avec David ? Nous n'avons point d'héritage avec le fils d'Isaï. Israël, chacun à ses tentes ! Et toi David, pourvois maintenant à ta maison. Ainsi, tout Israël s'en alla dans ses tentes.
\VS{17}Mais quant aux enfants d'Israël qui habitaient les villes de Juda, Roboam régna sur eux.
\VS{18}Alors le roi Roboam envoya Hadoram, qui était préposé aux impôts ; mais les enfants d'Israël le lapidèrent à coups de pierres et il mourut. Et le roi Roboam se hâta de monter sur un char pour s'enfuir à Jérusalem.
\VS{19}C'est ainsi qu'Israël s'est rebellé contre la maison de David, jusqu'à ce jour\FTNT{1 R. 12:16-19.}.
\Chap{11} 
\TextTitle{Yahweh interdit la guerre entre Juda et Israël\FTNTT{1 R. 12:21-24}}
\VerseOne{}Roboam, étant arrivé à Jérusalem, assembla la maison de Juda et de Benjamin, cent quatre-vingt mille hommes d'élite et de guerre, afin de combattre contre Israël, pour le ramener sous le règne de Roboam.
\VS{2}Mais la parole de Yahweh fut adressée à Schemaeja, homme de Dieu, en ces termes :
\VS{3}Parle à Roboam, fils de Salomon, roi de Juda, et à ceux d'Israël qui sont en Juda et en Benjamin, et dis-leur :
\VS{4}Ainsi parle Yahweh : Ne montez point, et ne combattez point contre vos frères. Retournez chacun dans sa maison ; c'est par moi que cette chose est arrivée. Et ils obéirent aux paroles de Yahweh, et ils s'en retournèrent sans aller contre Jéroboam\FTNT{1 R. 12:21-24.}.
\VS{5}Roboam demeura donc à Jérusalem, et il bâtit des villes fortes en Juda.
\VS{6}Il bâtit Bethléhem, Etham, Tekoa,
\VS{7}Beth-Tsur, Soco, Adullam,
\VS{8}Gath, Maréscha, Ziph,
\VS{9}Adoraïm, Lakis, Azéka,
\VS{10}Tsorea, Ajalon et Hébron, qui étaient en Juda et en Benjamin, et en fit des villes fortes.
\VS{11}Il les fortifia et y mit des gouverneurs, des provisions de vivres, d'huile et de vin.
\VS{12}Dans chacune de ces villes, il mit des boucliers et des lances, et il les rendit puissantes. Ainsi Juda et Benjamin lui furent soumis.
\TextTitle{Les sacrificateurs et les Lévites soutiennent Roboam}
\VS{13}Les sacrificateurs et les Lévites, qui étaient dans tout Israël, vinrent de toutes leurs contrées se joindre à lui.
\TextTitle{Jéroboam abandonne Yahweh\FTNTT{1 R. 12:26-30 ; 14:7-8}}
\VS{14}Car les Lévites abandonnèrent leurs faubourgs et leurs possessions et vinrent en Juda et à Jérusalem, parce que Jéroboam et ses fils les avaient rejetés des fonctions de sacrificateurs pour Yahweh.
\VS{15}Car il s'était établi des sacrificateurs pour les hauts lieux, pour les boucs, et pour les veaux qu'il avait faits.
\VS{16}Et à leur suite, ceux d'entre toutes les tribus d'Israël qui avaient appliqué leur cœur à chercher Yahweh, le Dieu d'Israël, vinrent à Jérusalem pour sacrifier à Yahweh, le Dieu de leurs pères.
\VS{17}Ils fortifièrent le royaume de Juda et affermirent Roboam, fils de Salomon, pendant trois ans ; car on suivit les voies de David et de Salomon pendant trois ans.
\TextTitle{Les femmes et les enfants de Roboam}
\VS{18}Or Roboam prit pour femme : Mahalath, fille de Jerimoth, fils de David et d'Abichaïl, fille d'Eliab, fils d'Isaï.
\VS{19}Elle lui enfanta des fils : Jeusch, Schemaria et Zaham.
\VS{20}Après elle, il prit Maaca, fille d'Absalom, qui lui enfanta Abija, Attaï, Ziza et Schelomith.
\VS{21}Roboam aima Maaca, fille d'Absalom, plus que toutes ses femmes et ses concubines. Car il prit dix-huit femmes et soixante concubines, et il engendra vingt-huit fils et soixante filles.
\VS{22}Roboam établit pour chef Abija, fils de Maaca, comme prince entre ses frères ; car il voulait le faire roi.
\VS{23}Il agit prudemment et dispersa tous ses fils dans toutes les contrées de Juda et de Benjamin, dans toutes les villes fortes ; il leur donna de quoi vivre en abondance, et demanda pour eux une multitude de femmes.
\Chap{12}
\TextTitle{Roboam affermi, il abandonne Yahweh\FTNTT{1 R. 14:21-24}}
\VerseOne{}Lorsque la royauté de Roboam fut affermie et qu'il eut acquis de la force, il abandonna la loi de Yahweh, et tout Israël avec lui\FTNT{1 R. 14:21-29.}.
\TextTitle{Yahweh veut livrer Juda à Schischak\FTNTT{1 R. 14:25-28}}
\VS{2}C'est pourquoi il arriva que la cinquième année du Roi Roboam, Schischak, roi d'Egypte, monta contre Jérusalem, parce qu'ils avaient péché contre Yahweh.
\VS{3}Il avait mille deux cents chars et soixante mille cavaliers, et le peuple qui vint avec lui d'Egypte, des Libyens, des Sukkiens et des Ethiopiens, était innombrable.
\VS{4}Il prit les villes fortes qui appartenaient à Juda, et vint jusqu'à Jérusalem.
\VS{5}Alors Schemaeja, le prophète, vint vers Roboam et les chefs de Juda, qui s'étaient assemblés à Jérusalem à cause de Schischak, et leur dit : Ainsi parle Yahweh : Vous m'avez abandonné ; moi aussi je vous abandonne aux mains de Schischak.
\VS{6}Alors les chefs d'Israël et le roi s'humilièrent, et dirent : Yahweh est juste !
\VS{7}Et quand Yahweh vit qu'ils s'humiliaient, la parole de Yahweh fut adressée à Schemaeja, et il lui dit : Ils se sont humiliés ; je ne les détruirai pas, mais je leur donnerai dans peu de temps un moyen d'échapper, et ma fureur ne se répandra point sur Jérusalem par la main de Schischak.
\VS{8}Toutefois, ils lui seront asservis, afin qu'ils sachent ce que c'est que de me servir ou de servir les royaumes de la terre.
\VS{9}Schischak, roi d'Egypte, monta donc contre Jérusalem, et prit les trésors de la maison de Yahweh et les trésors de la maison du roi ; il prit tout. Il prit les boucliers d'or que Salomon avait faits.
\VS{10}Le roi Roboam fit des boucliers d'airain à leur place, et il les mit entre les mains des chefs des coureurs qui gardaient la porte de la maison du roi.
\VS{11}Et toutes les fois que le roi entrait dans la maison de Yahweh, les coureurs venaient et les portaient ; puis ils les rapportaient dans la chambre des coureurs.
\VS{12}Ainsi comme il s'était humilié, la colère de Yahweh se détourna de lui, et ne le détruisit pas entièrement ; car il y avait encore de bonnes choses en Juda.
\TextTitle{Mort de Roboam\FTNTT{1 R. 14:21,29,31}}
\VS{13}Le roi Roboam se fortifia donc dans Jérusalem, et régna. Il avait quarante et un ans quand il devint roi, et il régna dix-sept ans à Jérusalem, la ville que Yahweh avait choisie de toutes les tribus d'Israël, pour y mettre son Nom. Sa mère s'appelait Naama, l'Ammonite.
\VS{14}Il fit le mal, car il ne disposa point son cœur pour chercher Yahweh.
\VS{15}Or les actions de Roboam, les premières et les dernières, ne sont-elles pas écrites dans les livres de Schemaeja le prophète, et d'Iddo le voyant, parmi les registres généalogiques ? Les guerres entre Roboam et Jéroboam furent continuelles.
\VS{16}Roboam s'endormit avec ses pères, et il fut enseveli dans la cité de David ; et Abija, son fils, régna à sa place.
\Chap{13}
\TextTitle{Abija règne sur Juda ; guerre entre Israël et Juda\FTNTT{1 R. 15:1-8}}
\VerseOne{}La dix-huitième année du roi Jéroboam, Abija commença à régner sur Juda.
\VS{2}Il régna trois ans à Jérusalem. Sa mère s'appelait Micaja, fille d'Uriel, de Guibea. Or il y eut guerre entre Abija et Jéroboam.
\VS{3}Abija engagea la guerre avec une armée de vaillants guerriers, quatre cent mille hommes d'élite ; et Jéroboam se rangea en bataille contre lui avec huit cent mille hommes d'élite, forts et vaillants.
\VS{4}Et Abija se leva du haut de la montagne de Tsemaraïm, parmi les montagnes d'Ephraïm, et dit : Jéroboam et tout Israël, écoutez-moi!
\VS{5}Ne savez-vous pas que Yahweh, le Dieu d'Israël, a donné pour toujours la royauté sur Israël à David, à lui et à ses fils, par une alliance de sel\FTNT{Sel : Voir commentaire en Lé. 2:13} !
\VS{6}Mais Jéroboam, fils de Nebath, serviteur de Salomon, fils de David, s'est élevé et s'est rebellé contre son seigneur.
\VS{7}Et des gens sans valeur, des fils de Belial, se sont assemblés avec lui et se sont fortifiés contre Roboam, fils de Salomon. Or Roboam était un jeune homme craintif et sans force devant eux.
\VS{8}Et maintenant, vous vous dites être forts devant la royauté de Yahweh, qui est aux mains des fils de David ; vous êtes une multitude, et vous avez avec vous les veaux d'or que Jéroboam vous a faits pour dieux.
\VS{9}N'avez-vous pas rejeté les sacrificateurs de Yahweh, les fils d'Aaron, et les Lévites ? Et ne vous êtes-vous pas faits des sacrificateurs comme les peuples des autres pays ? Quiconque venait, avec un jeune taureau et sept béliers pour être consacré, devenait sacrificateur de ce qui n'est pas Dieu.
\VS{10}Mais quant à nous, Yahweh est notre Dieu, et nous ne l'avons pas abandonné ; les sacrificateurs qui font le service de Yahweh sont fils d'Aaron, et ce sont les Lévites qui tiennent cette fonction.
\VS{11}Nous faisons brûler pour Yahweh, chaque matin et chaque soir, les holocaustes et le parfum d'aromates. Les pains de proposition sont rangés sur la table pure, et on allume le chandelier d'or avec ses lampes, chaque soir. Car nous gardons ce que Yahweh, notre Dieu,veut qu'on garde ; mais vous, vous l'avez abandonné.
\VS{12}Voici, Dieu est avec nous pour être notre chef, avec ses sacrificateurs, et les trompettes retentissantes, pour les faire sonner contre vous. Fils d'Israël, ne combattez pas contre Yahweh, le Dieu de vos pères ; car cela ne vous réussira pas.
\VS{13}Mais Jéroboam fit une embuscade par un détour, et arriva derrière eux. De sorte que les Israélites étaient en face de Juda, qui avait l'embuscade par-derrière.
\VS{14}Ceux de Juda se retournèrent et voici ils avaient la bataille par-devant et par-derrière. Alors ils crièrent à Yahweh, et les sacrificateurs sonnèrent des trompettes.
\TextTitle{Victoire de Juda sur Israël}
\VS{15}Les hommes de Juda poussèrent un cri, et au cri de guerre des hommes de Juda, Yahweh frappa Jéroboam et tout Israël devant Abija et Juda.
\VS{16}Les fils d'Israël s'enfuirent devant ceux de Juda, parce que Dieu les livra entre leurs mains.
\VS{17}Abija et son peuple leur firent un grand un carnage, et il tomba d'Israël cinq cent mille hommes d'élite blessés à mort.
\VS{18}Ainsi, les enfants d'Israël furent humiliés en ce temps-là ; et les enfants de Juda devinrent plus forts, parce qu'ils s'étaient appuyés sur Yahweh, le Dieu de leurs pères.
\VS{19} Abija poursuivit Jéroboam, et lui prit ces villes : Béthel et les villes de son ressort, Jeschana et les villes de son ressort, Ephron et les villes de son ressort.
\TextTitle{Mort de Jéroboam\FTNTT{1 R. 14:19,20}}
\VS{20}Et Jéroboam n'eut plus de force durant le temps d'Abija ; et Yahweh le frappa, et il mourut.
\TextTitle{Les femmes et les fils d'Abija\FTNTT{1 R. 15:7-8}}
\VS{21}Mais Abija se fortifia ; il prit quatorze femmes, et engendra vingt-deux fils et seize filles.
\VS{22}Le reste des actions d'Abija, sa conduite et ses paroles sont écrites dans les mémoires du prophète Iddo.
\VS{23}Abija s'endormit avec ses pères, et on l'ensevelit dans la cité de David ; et Asa, son fils, régna à sa place. De son temps, le pays fut en repos pendant dix ans.
\Chap{14}
\TextTitle{Asa règne sur Juda, il rétablit l'ordre de Yahweh\FTNTT{1 R. 15:11}}
\VerseOne{}Asa fit ce qui est bon et droit aux yeux de Yahweh, son Dieu.
\VS{2}Il ôta les autels étrangers et les hauts lieux ; il brisa les statues et mit en pièces les idoles d'Asherah.
\VS{3}Et il recommanda à Juda de rechercher Yahweh, le Dieu de leurs pères, et de pratiquer la loi et les commandements.
\VS{4}Il ôta de toutes les villes de Juda les hauts lieux et les colonnes consacrées au soleil. Et le royaume fut en repos devant lui.
\VS{5}Il bâtit des villes fortes en Juda, car le pays fut en repos. Et pendant ces années-là, il n'y eut point de guerre contre lui, parce que Yahweh lui donna du repos.
\VS{6}Et il dit à Juda : Bâtissons ces villes, et entourons-les de murailles, de tours, de portes et de barres ; le pays est encore devant nous, parce que nous avons recherché Yahweh, notre Dieu. Nous l'avons recherché, et il nous a donné du repos de toutes parts. Ainsi, ils bâtirent et prospérèrent.
\TextTitle{Asa s'appuie sur Yahweh et triomphe de Zérach\FTNTT{2 Ch. 16:1-10}}
\VS{7}Or Asa avait dans son armée trois cent mille hommes de Juda, portant le grand bouclier et la lance, et deux cent quatre-vingt mille de Benjamin, portant le bouclier et tirant de l'arc, tous vaillants guerriers.
\VS{8}Mais Zérach, l'Ethiopien, sortit contre eux avec une armée d'un million d'hommes, et de trois cents chars ; et il vint jusqu'à Maréscha.
\VS{9}Asa alla au-devant de lui, et ils se rangèrent en bataille dans la vallée de Tsephata, près de Maréscha.
\VS{10}Alors Asa cria à Yahweh, son Dieu, et dit : Yahweh ! Toi seul peux nous secourir, que l'on soit nombreux ou sans force ! Aide-nous, Yahweh, notre Dieu ! Car nous nous appuyons sur toi, et nous sommes venus en ton Nom contre cette multitude. Tu es Yahweh, notre Dieu : Que l'homme ne prévale pas contre toi !
\VS{11}Et Yahweh frappa les Ethiopiens devant Asa et devant Juda ; et les Ethiopiens s'enfuirent.
\VS{12}Asa et le peuple qui était avec lui les poursuivirent jusqu'à Guérar, et tant d'Ethiopiens tombèrent sans pouvoir sauver leur vie ; car ils furent brisés devant Yahweh et son armée, et on emporta un très grand butin.
\VS{13}Ils frappèrent aussi toutes les villes autour de Guérar, car la terreur de Yahweh était sur eux ; et ils pillèrent toutes ces villes, car il s'y trouvait un grand butin.
\VS{14}Ils frappèrent aussi les tentes des troupeaux, et emmenèrent des brebis et des chameaux en abondance ; puis ils retournèrent à Jérusalem.
\Chap{15}
\TextTitle{Azaria le prophète avertit Asa}
\VerseOne{}Alors l'Esprit de Dieu fut sur Azaria, fils d'Oded.
\VS{2}Et il sortit au-devant d'Asa, et lui dit : Asa, et tout Juda et Benjamin, écoutez-moi ! Yahweh est avec vous quand vous êtes avec lui. Si vous le cherchez, vous le trouverez ; mais si vous l'abandonnez, il vous abandonnera.
\VS{3}Pendant longtemps Israël a été sans vrai Dieu, sans sacrificateur qui l'enseignait, et sans loi.
\VS{4}Mais dans leur détresse, ils sont revenus vers Yahweh, le Dieu d'Israël ; ils l'ont cherché, et ils l'ont trouvé\FTNT{Ps. 107:19-20.}.
\VS{5}Dans ces temps-là, il n'y avait point de sûreté pour ceux qui allaient et venaient, car il y avait de grands troubles parmi tous les habitants du pays.
\VS{6}Une nation était écrasée par une autre nation, et une ville par une autre ville ; car Dieu les agitait par toutes sortes d'angoisses.
\VS{7}Mais vous, fortifiez-vous, et que vos mains ne se relâchent pas ; car il y a une récompense pour vos œuvres.
\TextTitle{Asa écoute les paroles d'Azaria\FTNTT{1 R. 15:12-15}}
\VS{8}Or dès qu'Asa eut entendu ces paroles et la prophétie d'Oded le prophète, il se fortifia ; et fit disparaître les abominations de tout le pays de Juda et de Benjamin, et des villes qu'il avait prises dans les montagnes d'Ephraïm ; et il rétablit l'autel de Yahweh, qui était devant le portique de Yahweh.
\VS{9}Puis il assembla tout Juda et Benjamin, et ceux d'Ephraïm, de Manassé et de Siméon, qui habitaient avec eux ; car un grand nombre de gens d'Israël passaient à lui, voyant que Yahweh, son Dieu, était avec lui.
\VS{10}Ils s'assemblèrent donc à Jérusalem, le troisième mois de la quinzième année du règne d'Asa ;
\VS{11}et ils sacrifièrent ce jour-là à Yahweh sept cents bœufs et sept mille brebis, du butin qu'ils avaient amené.
\VS{12}Et ils rentrèrent dans l'alliance pour chercher Yahweh, le Dieu de leurs pères, de tout leur cœur et de toute leur âme ;
\VS{13}de sorte qu'on devait faire mourir quiconque ne rechercherait pas Yahweh, le Dieu d'Israël, petit ou grand, homme ou femme.
\VS{14}Et ils jurèrent à Yahweh, à haute voix, avec des cris de joie, et au son des shofars et des cors.
\VS{15}Tout Juda se réjouit de ce serment, parce qu'ils avaient juré de tout leur cœur et qu'ils avaient recherché Yahweh de leur plein gré, et qu'ils l'avaient trouvé. Et Yahweh leur donna du repos de toutes parts.
\VS{16}Le roi Asa destitua même sa mère, Maaca, de son rang de reine, parce qu'elle avait fait une idole pour Astarté. Asa abattit l'idole, l'écrasa et la brûla près du torrent de Cédron.
\VS{17}Mais les hauts lieux ne furent point ôtés du milieu d'Israël. Néanmoins, le cœur d'Asa fut intègre tout le long de ses jours.
\VS{18}Il remit dans la maison de Dieu les choses que son père avait consacrées, avec ce qu'il avait lui-même consacré, l'argent, l'or et les ustensiles.
\VS{19}Et il n'y eut point de guerre jusqu'à la trente-cinquième année du règne d'Asa.
\Chap{16}
\TextTitle{Alliance d'Asa et du roi de Syrie contre le roi d'Israël\FTNTT{ 1 R. 15:16-22 ; cp. 1 R. 15:27 ; 16:7}}
\VerseOne{}La trente-sixième année du règne d'Asa, Baescha, roi d'Israël, monta contre Juda, et il bâtit Rama, pour empêcher quiconque de sortir et d'entrer vers Asa, roi de Juda.
\VS{2}Alors Asa sortit de l'argent et de l'or des trésors de la maison de Yahweh et de la maison royale, et il envoya dire à Ben-Hadad, roi de Syrie, qui habitait à Damas :
\VS{3}Il y a alliance entre nous, et entre mon père et ton père ; voici, je t'envoie de l'argent et de l'or ; va, romps l'alliance que tu as avec Baescha, roi d'Israël, afin qu'il s'éloigne de moi.
\VS{4}Ben-Hadad écouta le roi Asa, et il envoya les chefs de son armée contre les villes d'Israël, et ils frappèrent Ijjon, Dan, Abel-Maïm, et tous les magasins des villes de Nephthali.
\VS{5}Et aussitôt que Baescha l'apprit, il cessa de bâtir Rama et suspendit ses travaux.
\VS{6}Alors le roi Asa prit avec lui tout Juda, et ils emportèrent les pierres et le bois de Rama, que Baescha faisait bâtir ; et il en bâtit Guéba et Mitspa.
\TextTitle{Hanani condamne l'alliance d'Asa}
\VS{7}En ce temps-là, Hanani le voyant, vint vers Asa, roi de Juda, et lui dit : Parce que tu t'es appuyé sur le roi de Syrie, et que tu ne t'es point appuyé sur Yahweh, ton Dieu, l'armée du roi de Syrie a échappé de ta main.
\VS{8}Les Ethiopiens et les Libyens n'étaient-ils pas une grande armée, ayant des chars et une multitude de cavaliers ? Mais parce que tu t'étais appuyé sur Yahweh, il les livra entre tes mains.
\VS{9}Car les yeux de Yahweh parcourent toute la terre, pour soutenir ceux dont le cœur est tout entier à lui. Tu as agi follement en cela ; car désormais tu auras des guerres.
\VS{10}Asa fut irrité contre le voyant, et le mit en prison, car il était indigné contre lui à ce sujet. Asa opprima aussi, en ce temps-là, quelques-uns du peuple.
\TextTitle{Mort d'Asa\FTNTT{1 R. 15:23-24}}
\VS{11}Or voici, les actions d'Asa, les premières et les dernières, sont écrites dans le livre des rois de Juda et d'Israël.
\VS{12}Asa fut malade des pieds la trente-neuvième année de son règne, et sa maladie fut très grave. Toutefois, il ne chercha point Yahweh dans sa maladie, mais les médecins.
\VS{13}Puis Asa s'endormit avec ses pères, et il mourut la quarante et unième année de son règne.
\VS{14}On l'ensevelit dans le sépulcre qu'il s'était creusé dans la cité de David. On le coucha dans un lit qui était rempli de parfums et d'aromates, composés par le travail d'un parfumeur ; et on lui en brûla une quantité considérable.
\Chap{17}
\TextTitle{Josaphat règne sur Juda, il recherche Yahweh\FTNTT{1 R. 15:24}}
\VerseOne{}Josaphat son fils régna à sa place et se fortifia contre Israël.
\VS{2}Il mit des troupes dans toutes les villes fortes de Juda, et des garnisons dans le pays de Juda, et dans les villes d'Ephraïm qu'Asa, son père, avait prises.
\VS{3}Yahweh fut avec Josaphat, parce qu'il suivit les premières voies de David, son père, et qu'il ne rechercha point les Baals ;
\VS{4}car il rechercha le Dieu de son père, et il marcha dans ses commandements, et non pas selon ce que faisait Israël.
\VS{5}Yahweh affermit donc le royaume entre ses mains ; et tout Juda apportait des présents à Josaphat, et il eut en abondance des richesses et de la gloire.
\VS{6}Son cœur grandit dans les voies de Yahweh, et il ôta encore de Juda les hauts lieux et les idoles d'Astarté.
\VS{7}Puis, la troisième année de son règne, il envoya ses chefs Ben-Haïl, Abdias, Zacharie, Nethaneel et Michée, pour enseigner dans les villes de Juda ;
\VS{8}et avec eux les Lévites Schemaeja, Nethania, Zebadia, Asaël, Schemiramoth, Jonathan, Adonija, Tobija et Tob-Adonija, Lévites, et avec eux Elischama et Joram, les sacrificateurs.
\VS{9}Ils enseignèrent dans Juda, ayant avec eux le livre de la loi de Yahweh. Ils firent le tour de toutes les villes de Juda, et enseignèrent parmi le peuple.
\TextTitle{Affermissement du règne de Josaphat}
\VS{10}La terreur de Yahweh fut sur tous les royaumes des pays qui entouraient Juda, et ils ne firent point la guerre à Josaphat.
\VS{11}On apporta aussi à Josaphat des présents de la part des Philistins, et un impôt en argent ; et les Arabes lui amenèrent aussi du bétail, sept mille sept cents béliers et sept mille sept cents boucs.
\VS{12}Ainsi Josaphat s'élevait jusqu'au plus haut degré de gloire. Et il bâtit en Juda des châteaux et des villes pour servir de magasins.
\VS{13}Il fit de grands travaux dans les villes de Juda ; et il avait à Jérusalem des gens de guerre puissants et vaillants.
\VS{14}Voici leur dénombrement, selon les maisons de leurs pères. Les chefs de milliers de Juda furent Adna le chef, avec trois cent mille vaillants guerriers.
\VS{15}Et après lui, Jochanan le chef, avec deux cent quatre-vingt mille hommes.
\VS{16}A ses côtés, Amasia, fils de Zicri, qui s'était volontairement offert à Yahweh, avec deux cent mille vaillants guerriers.
\VS{17}De Benjamin, Eliada, vaillant guerrier, avec deux cent mille hommes, armés d'arcs et de boucliers,
\VS{18}à côté de lui Zozabad, avec cent quatre-vingt mille hommes équipés pour le combat.
\VS{19}Tels sont ceux qui étaient au service du roi, outre ceux que le roi avait placés dans toutes les villes fortes de Juda.
\Chap{18}
\TextTitle{Josaphat s'allie à Achab contre les Syriens\FTNTT{1 R. 22:2-4}}
\VerseOne{}Or Josaphat, ayant beaucoup de richesses et de gloire, s'allia par mariage avec Achab.
\VS{2}Et au bout de quelques années, il descendit vers Achab, à Samarie. Achab tua pour lui, et pour le peuple qui était avec lui, un grand nombre de brebis et de bœufs, et l'incita à monter contre Ramoth de Galaad\FTNT{1 R. 22:2-40.}.
\VS{3}Achab, roi d'Israël, dit à Josaphat, roi de Juda : Viendras-tu avec moi contre Ramoth de Galaad ? Et il lui répondit : Compte sur moi comme sur toi, et sur mon peuple comme sur ton peuple, nous irons avec toi à la guerre.
\TextTitle{Les prophètes de mensonge encouragent Achab\FTNTT{1 R. 22:5-12}}
\VS{4}Puis Josaphat dit au roi d'Israël : Consulte aujourd'hui, je te prie, la parole de Yahweh.
\VS{5}Le roi d'Israël assembla les prophètes, au nombre de quatre cents, et leur dit : Irons-nous à la guerre contre Ramoth de Galaad, ou dois-je y renoncer ? Ils répondirent : Monte, et Dieu la livrera entre les mains du roi.
\VS{6}Mais Josaphat dit : N'y a-t-il point encore ici quelque prophète de Yahweh, afin que nous l'interrogions ?
\VS{7}Le roi d'Israël dit à Josaphat : Il y a encore un homme par qui on peut consulter Yahweh ; mais je le hais parce qu'il ne me prophétise rien de bon, mais du mal ; c'est Michée, fils de Jimla. Josaphat dit : Que le roi ne parle pas ainsi !
\VS{8}Alors le roi d'Israël appela un eunuque, et dit : Fais promptement venir Michée, fils de Jimla.
\VS{9}Or le roi d'Israël et Josaphat, roi de Juda, étaient assis, chacun sur son trône, revêtus de leurs habits, et ils étaient assis dans la place, à l'entrée de la porte de Samarie ; et tous les prophètes prophétisaient en leur présence.
\VS{10}Alors Sédécias, fils de Kenaana, s'étant fait des cornes de fer, dit : Ainsi parle Yahweh : Avec ces cornes tu heurteras les Syriens jusqu'à les détruire.
\VS{11}Tous les prophètes prophétisaient de même, en disant : Monte à Ramoth de Galaad, et tu prospèreras ; Yahweh la livrera entre les mains du roi.
\TextTitle{Michée annonce la défaite et la mort d'Achab\FTNTT{1 R. 22:13-28 ; 1 R. 22:29-40}}
\VS{12}Or le messager qui était allé appeler Michée, lui parla et lui dit : Voici, tous les prophètes disent d'une même bouche du bien au roi ; je te prie que ta parole soit semblable à celle de chacun d'eux ! Annonce du bien !
\VS{13}Mais Michée répondit : Yahweh est vivant ! Je dirai ce que mon Dieu dira.
\VS{14}Il vint donc vers le roi, et le roi lui dit : Michée, irons-nous à la guerre contre Ramoth de Galaad, devons-nous y renoncer ? Et il répondit : Montez, vous prospérerez, et ils seront livrés entre vos mains.
\VS{15}Et le roi lui dit : Combien de fois devrais-je te faire jurer de ne me dire que la vérité au Nom de Yahweh ?
\VS{16}Et il répondit : J'ai vu tout Israël dispersé par les montagnes, comme un troupeau de brebis qui n'a point de berger ; et Yahweh a dit : Ces gens n'ont point de seigneur ; que chacun retourne en paix dans sa maison !
\VS{17}Alors le roi d'Israël dit à Josaphat : Ne t'ai-je pas dit qu'il ne prophétise rien de bon quand il s'agit de moi, mais seulement du mal ?
\VS{18}Et Michée dit : Ecoute la parole de Yahweh ! J'ai vu Yahweh assis sur son trône, et toute l'armée des cieux se tenant à sa droite et à sa gauche.
\VS{19}Et Yahweh dit : Qui est-ce qui séduira Achab, roi d'Israël, afin qu'il monte et qu'il tombe à Ramoth de Galaad ? Et l'un répondait d'une façon et l'autre d'une autre.
\VS{20}Alors un esprit s'avança et se tint devant Yahweh, et dit : Moi, je le séduirai. Yahweh lui dit : Comment ?
\VS{21}Il répondit : Je sortirai, dit-il, et je serai un esprit de mensonge\FTNT{Achab a été frappé de l'esprit d'égarement (2 Th. 2:9-11). Voir commentaires en Ge. 6:3 ; Mt. 12:31.} dans la bouche de tous ses prophètes. Et Yahweh dit : Tu le séduiras, et même tu en viendras à bout. Sors, et fais ainsi.
\VS{22}Maintenant voici, Yahweh a mis un esprit de mensonge dans la bouche de tes prophètes que voilà ; et Yahweh a prononcé du mal contre toi.
\VS{23}Alors Sédécias, fils de Kenaana, s'étant approché, frappa Michée sur la joue, et dit : Par quel chemin l'Esprit de Yahweh s'est-il retiré de moi pour te parler ?
\VS{24}Et Michée répondit : Voici, tu le verras au jour où tu iras de chambre en chambre pour te cacher !
\VS{25}Alors le roi d'Israël dit : Prenez Michée, et emmenez-le vers Amon, chef de la ville, et vers Joas, fils du roi.
\VS{26}Et vous direz : Ainsi parle le roi : Mettez cet homme en prison, et nourrissez-le du pain et de l'eau de l'affliction, jusqu'à ce que je revienne en paix.
\VS{27}Et Michée dit : Si jamais tu retournes et reviens en paix, Yahweh n'aura point parlé par moi. Et il dit : Entendez cela peuples, vous tous qui êtes ici !
\VS{28}Le roi d'Israël monta donc avec Josaphat, roi de Juda, à Ramoth de Galaad.
\VS{29}Le roi d'Israël dit à Josaphat : Je vais me déguiser pour aller au combat ; mais toi, revêts-toi de tes habits. Ainsi le roi d'Israël se déguisa ; et ils allèrent au combat.
\VS{30}Or le roi des Syriens avait donné cet ordre aux chefs de ses chars, disant : Vous ne combattrez ni petit ni grand, mais seulement le roi d'Israël.
\VS{31}Les chefs des chars aperçurent Josaphat, et dirent : C'est le roi d'Israël ! Et ils se tournèrent vers lui pour le combattre ; mais Josaphat poussa un cri, et Yahweh le secourut, et Dieu les éloigna de lui.
\VS{32}Quand les chefs des chars virent que ce n'était pas le roi d'Israël, ils se détournèrent de lui.
\VS{33}Alors quelqu'un tira de son arc au hasard, et frappa le roi d'Israël entre les jointures de la cuirasse ; et le roi dit à son conducteur de char : Tourne-toi, et sors-moi du camp ; car je suis blessé.
\VS{34}Or en ce jour-là, le combat fut très rude. Le roi d'Israël se posa dans son char, en face des Syriens, jusqu'au soir ; et il mourut vers le coucher du soleil.
\Chap{19}
\TextTitle{Jéhu dénonce l'alliance de Josaphat avec Achab}
\VerseOne{}Josaphat roi de Juda, revint en paix dans sa maison, à Jérusalem.
\VS{2}Mais Jéhu, fils de Hanani, le voyant, sortit au-devant du roi Josaphat, et lui dit : Faut-il donner du secours au méchant, ou aimer ceux qui haïssent Yahweh ? A cause de cela, Yahweh est irrité contre toi.
\VS{3}Mais il s'est trouvé de bonnes choses en toi, puisque tu as ôté du pays les idoles d'Astarté, et tu as appliqué ton cœur à rechercher Dieu.
\VS{4}Josaphat demeura à Jérusalem. Puis, il ressortit de nouveau parmi le peuple, depuis Beer-Schéba jusqu'à la montagne d'Ephraïm, et il les ramena à Yahweh, le Dieu de leurs pères.
\TextTitle{Josaphat organise la justice}
\VS{5}Il établit aussi des juges dans le pays, dans toutes les villes fortes de Juda, de ville en ville.
\VS{6}Et il dit aux juges : Veillez sur ce que vous ferez ; car vous n'exercez pas la justice de la part d'un homme, mais de la part de Yahweh, qui sera avec vous quand vous prononcerez les jugements.
\VS{7}Maintenant, que la crainte de Yahweh soit sur vous ; prenez garde à ce que vous ferez ; car il n'y a point d'iniquité chez Yahweh, notre Dieu, ni d'acception de personnes, ni d'acceptation de présents. 
\VS{8}Josaphat établit aussi à Jérusalem des Lévites, des sacrificateurs, et des chefs des pères d'Israël, pour le jugement de Yahweh, et pour les contestations ; car on revenait à Jérusalem.
\VS{9}Il leur donna des ordres, en disant : Vous agirez ainsi dans la crainte de Yahweh, avec fidélité et avec intégrité de cœur.
\VS{10}Dans toute contestation qui viendra devant vous, de la part de vos frères qui habitent dans leurs villes, soit d'un meurtre, d'une loi, d'un commandement, d'un statut ou d'une ordonnance, vous les instruirez, afin qu'ils ne se rendent pas coupables envers Yahweh, et que sa colère ne vienne pas sur vous et sur vos frères. Vous agirez ainsi afin de ne pas être coupables.
\VS{11}Et voici, Amaria, le souverain sacrificateur, sera au-dessus de vous pour toutes les affaires de Yahweh ; et Zebadia, fils d'Ismaël, prince de la maison de Juda, pour toutes les affaires du roi ; et pour secrétaires, vous avez devant vous les Lévites. Fortifiez-vous et faites ainsi ; et que Yahweh soit avec l'homme de bien !
\Chap{20}
\TextTitle{Menaces des ennemis de Juda, prière de Josaphat}
\VerseOne{}Après ces choses, les fils de Moab et les fils d'Ammon, et avec eux les Maonites, vinrent contre Josaphat pour lui faire la guerre.
\VS{2}On vint le rapporter à Josaphat, en disant : Il vient contre toi une grande multitude depuis l'autre bord de la mer, de Syrie ; et les voici à Hatsatson-Thamar, qui est En-Guédi.
\VS{3}Alors Josaphat craignit ; mais il se disposa à rechercher Yahweh, et publia un jeûne pour tout Juda.
\VS{4}Juda s'assembla donc pour rechercher Yahweh ; on vint même de toutes les villes de Juda pour chercher Yahweh.
\VS{5}Et Josaphat se tint au milieu de l'assemblée de Juda et de Jérusalem, dans la maison de Yahweh, devant le nouveau parvis.
\VS{6}Il dit : Yahweh, Dieu de nos pères ! N'es-tu pas Dieu dans les cieux, toi qui domines sur tous les royaumes des nations ? Ne tiens-tu pas dans ta main la force et la puissance, et à qui nul ne peut résister ?
\VS{7}N'est-ce pas toi, ô notre Dieu, qui as dépossédé les habitants de ce pays devant ton peuple d'Israël, et qui l'as donné pour toujours à la postérité d'Abraham, qui t'aimait ?
\VS{8}Ils y ont habité et t'y ont bâti un sanctuaire pour ton Nom, en disant :
\VS{9}S'il nous arrive quelque malheur, l'épée, le jugement, la peste, ou la famine, nous nous tiendrons devant cette maison, et en ta présence ; car ton Nom est en cette maison ; et nous crierons à toi dans notre détresse, et tu exauceras et tu délivreras !
\VS{10}Maintenant, voici les enfants d'Ammon et de Moab, et ceux de la montagne de Séir, chez lesquels tu ne permis pas à Israël d'entrer quand il venait du pays d'Egypte, car il se détourna d'eux, et ne les détruisit pas.
\VS{11}Voici, pour nous récompenser, ils viennent nous chasser de ton héritage, que tu nous as fait posséder.
\VS{12}Ô notre Dieu ! Ne seras-tu pas juge contre eux ? Car nous sommes sans force devant cette grande multitude qui vient contre nous, et nous ne savons que faire ; mais nos yeux sont sur toi.
\VS{13}Or tout Juda se tenait devant Yahweh, même avec leurs petits enfants, leurs femmes et leurs fils.
\TextTitle{Yahweh répond à Josaphat}
\VS{14}Alors l'Esprit de Yahweh saisit au milieu de l'assemblée Jachaziel, fils de Zacharie, fils de Benaja, fils de Jeïel, fils de Matthania, Lévite, d'entre les fils d'Asaph,
\VS{15}et il dit : Soyez attentifs, tout Juda et habitants de Jérusalem, et toi, roi Josaphat ! Ainsi parle Yahweh : Ne craignez point, et ne soyez point effrayés en face de cette grande multitude ; car ce ne sera pas à vous de combattre, mais à Dieu.
\VS{16}Descendez demain vers eux ; les voici qui montent par la montée de Tsits, et vous les trouverez à l'extrémité de la vallée, en face du désert de Jeruel.
\VS{17}Ce ne sera point à vous de combattre en cette bataille ; présentez-vous, tenez-vous là, et voyez la délivrance que Yahweh va vous donner. Juda et Jérusalem, ne craignez point, et ne soyez point effrayés ! Demain, sortez au-devant d'eux, et Yahweh sera avec vous.
\VS{18}Alors Josaphat s'inclina le visage contre terre, et tout Juda et les habitants de Jérusalem se jetèrent devant Yahweh, se prosternant devant Yahweh.
\VS{19}Et les Lévites, d'entre les fils des Kehathites et d'entre les fils des Koréites, se levèrent pour célébrer Yahweh, le Dieu d'Israël, d'une voix haute et forte.
\TextTitle{Yahweh délivre Juda des armées ennemies}
\VS{20}Puis, le matin, ils se levèrent de bonne heure et sortirent vers le désert de Tekoa. Et comme ils sortaient, Josaphat se tint debout et dit : Ecoutez-moi Juda et vous, habitants de Jérusalem ! Croyez en Yahweh, votre Dieu, et vous serez en sûreté ; croyez en ses prophètes, et vous réussirez.
\VS{21}Puis, ayant consulté le peuple, il établit des chantres de Yahweh, qui célébraient sa sainte magnificence ; et marchant devant l'armée, ils disaient : Louez Yahweh, car sa miséricorde dure à toujours\FTNT{Ps. 136} !
\VS{22}Et au moment où ils commencèrent le chant et la louange, Yahweh mit des embuscades contre les fils d'Ammon, de Moab, et ceux de la montagne de Séir, qui venaient contre Juda. Et ils furent battus.
\VS{23}Les fils d'Ammon et de Moab se levèrent contre les habitants de la montagne de Séir pour les dévouer par interdit et les exterminer ; et quand ils en eurent fini avec les habitants de Séir, ils s'aidèrent l'un l'autre à se détruire mutuellement.
\VS{24}Et quand Juda fut arrivé sur la hauteur d'où l'on voit le désert, ils regardèrent vers cette multitude, et voici, c'étaient des cadavres gisant à terre, et personne n'avait échappé.
\VS{25}Ainsi Josaphat et son peuple vinrent pour piller leurs dépouilles, et ils trouvèrent parmi les cadavres des biens en abondance, et des objets précieux ; et ils en saisirent tant qu'ils ne pouvaient tout porter ; et ils pillèrent le butin pendant trois jours, car il était considérable.
\VS{26}Le quatrième jour, ils s'assemblèrent dans la vallée de Beraca ; car ils bénirent là Yahweh ; c'est pourquoi on a appelé ce lieu, jusqu'à ce jour, la vallée de Beraca.
\VS{27}Et tous les hommes de Juda et de Jérusalem, et Josaphat à leur tête, s'en retournèrent, revenant à Jérusalem avec joie ; car Yahweh les avait réjouis au sujet de leurs ennemis. 
\VS{28}Ils entrèrent donc à Jérusalem, dans la maison de Yahweh, avec des luths, des harpes et des trompettes.
\VS{29}Et la crainte de Dieu fut sur tous les royaumes des autres pays, lorsqu'ils apprirent que Yahweh avait combattu contre les ennemis d'Israël.
\VS{30}Ainsi le royaume de Josaphat fut tranquille, et son Dieu lui donna du repos de toutes parts.
\TextTitle{Règne de Josaphat, son alliance coupable\FTNTT{1 R. 22:41-49}}
\VS{31}Josaphat régna donc sur Juda. Il était âgé de trente-cinq ans quand il devint roi, et il régna vingt-cinq ans à Jérusalem. Sa mère s'appelait Azuba, fille de Schilchi.
\VS{32}Il suivit les traces d'Asa, son père, et il ne s'en détourna point, faisant ce qui est droit aux yeux de Yahweh.
\VS{33}Seulement les hauts lieux ne furent pas ôtés, et le peuple n'avait pas encore le cœur fermement attaché au Dieu de ses pères.
\VS{34}Or le reste des actions de Josaphat, les premières et les dernières, voici, elles sont écrites dans les mémoires de Jéhu, fils de Hanani, insérées dans le livre des rois d'Israël.
\VS{35}Après cela, Josaphat, roi de Juda, s'associa avec Achazia, roi d'Israël, dont la conduite était impie.
\VS{36}Il s'associa avec lui pour faire des navires, afin d'aller à Tarsis ; et ils firent des navires à Etsjon-Guéber.
\VS{37}Alors Eliézer, fils de Dodava, de Maréscha, prophétisa contre Josaphat, en disant : Parce que tu t'es associé avec Achazia, Yahweh a détruit ton œuvre. Et les navires furent brisés, et ne purent aller à Tarsis.
\Chap{21}
\TextTitle{Joram règne sur Juda\FTNTT{1 R. 22:50 ; 2 R. 8:16-19}}
\VerseOne{}Puis Josaphat s'endormit avec ses pères, et il fut enseveli avec eux dans la cité de David. Et Joram, son fils, régna à sa place\FTNT{1 R. 22:51.}.
\VS{2}Il avait des frères, fils de Josaphat : Azaria, Jehiel, Zacharie, Azaria, Micaël et Schephathia. Tous ceux-là étaient fils de Josaphat, roi d'Israël.
\VS{3}Leur père leur avait fait de grands dons d'argent, d'or et de choses précieuses, avec des villes fortes en Juda ; mais il avait donné le royaume à Joram, parce qu'il était le premier-né.
\VS{4}Quand Joram fut élevé sur le royaume de son père, et s'y fut fortifié, il tua avec l'épée tous ses frères, et quelques-uns aussi des chefs d'Israël.
\VS{5}Joram était âgé de trente-deux ans quand il devint roi, et il régna huit ans à Jérusalem.
\VS{6}Il marcha dans la voie des rois d'Israël, comme avait fait la maison d'Achab ; car la fille d'Achab était sa femme, et il fit ce qui est mal aux yeux de Yahweh.
\VS{7}Toutefois, Yahweh, à cause de l'alliance qu'il avait traitée avec David, ne voulut pas détruire la maison de David, selon qu'il avait dit qu'il lui donnerait une lampe, à lui et à ses fils, pour toujours.
\TextTitle{Rébellion d'Edom et de Libna\FTNTT{1 R. 8:20-23}}
\VS{8}De son temps, Edom se rebella de l'autorité de Juda, et établit un roi sur lui\FTNT{2 R. 8:20-23}.
\VS{9}Joram se mit donc en marche avec ses chefs et tous ses chars ; et s'étant levé de nuit, il battit les Edomites qui l'entouraient, et tous les chefs des chars.
\VS{10}Néanmoins, Edom se rebella contre l'autorité de Juda jusqu'à ce jour. En ce même temps, Libna se rebella aussi contre son autorité, parce qu'il avait abandonné Yahweh, le Dieu de ses pères.
\VS{11}Il fit aussi des hauts lieux dans les montagnes de Juda ; il fit que les habitants de Jérusalem se prostituèrent, et il y entraîna ceux de Juda.
\TextTitle{Elie prononce un jugement sur Joram}
\VS{12}Alors il lui vint un écrit de la part d'Elie, le prophète, disant : Ainsi parle Yahweh, le Dieu de David, ton père : Parce que tu n'as point suivi le chemin de Josaphat, ton père, ni celui d'Asa, roi de Juda,
\VS{13}mais que tu as suivi les voies des rois d'Israël, et que tu as poussé à la prostitution Juda et les habitants de Jérusalem, comme s'est prostituée la maison d'Achab, et que tu as tué tes frères, meilleurs que toi, la maison même de ton père ;
\VS{14}voici, Yahweh frappera d'une grande plaie ton peuple, tes fils, tes femmes et tous tes biens.
\VS{15}Et toi, tu auras une grosse maladie, une maladie d'entrailles ; jusqu'à ce que, de jour en jour, tes entrailles sortent par la force de la maladie.
\TextTitle{Yahweh excite les Philistins et les Arabes contre Joram}
\VS{16}Yahweh souleva contre Joram l'esprit des Philistins et des Arabes, qui habitent près des Ethiopiens.
\VS{17}Ils montèrent donc contre Juda, et firent une brèche pour piller toutes les richesses qui furent trouvées dans la maison du roi ; et même, ils emmenèrent captifs ses fils et ses femmes, de sorte qu'il ne lui resta d'autre fils que Joachaz, le plus jeune de ses fils.
\TextTitle{Mort de Joram}
\VS{18}Après tout cela, Yahweh frappa ses entrailles d'une maladie sans remède.
\VS{19}Elle s'avança chaque jour, et vers la fin de la seconde année, ses entrailles sortirent par la force de son mal, et il mourut dans de grandes souffrances. Son peuple ne brûla pas sur lui de parfums, comme il l'avait fait pour ses pères.
\VS{20}Il était âgé de trente-deux ans quand il devint roi, et il régna huit ans à Jérusalem. Il s'en alla sans être regretté, et on l'ensevelit dans la cité de David, mais non dans les sépulcres des rois.
\Chap{22}
\TextTitle{Achazia règne sur Juda\FTNTT{2 R. 8:24-29}}
\VerseOne{}Les habitants de Jérusalem firent régner à sa place Achazia, le plus jeune de ses fils, parce que les troupes qui étaient venues au camp avec les Arabes avaient tué tous les plus âgés ; et Achazia, fils de Joram, roi de Juda, régna\FTNT{2 R. 8:24-29 ; 2 R. 9:16.}.
\VS{2}Achazia était âgé de quarante-deux ans quand il devint roi, et il régna un an à Jérusalem. Sa mère avait pour nom Athalie, fille d'Omri.
\VS{3}Il suivit aussi les voies de la maison d'Achab, car sa mère lui donnait des conseils impies.
\VS{4}Il fit donc ce qui est mal aux yeux de Yahweh, comme la maison d'Achab ; parce qu'ils furent ses conseillers après la mort de son père, pour sa ruine.
\TextTitle{Achazia livré aux mains de Jéhu\FTNTT{2 R. 8:28-29 ; 2 R. 9:1-30}}
\VS{5}Conduit par leurs conseils, il alla avec Joram, fils d'Achab, roi d'Israël, à la guerre à Ramoth de Galaad, contre Hazaël, roi de Syrie. Et les Syriens frappèrent Joram,
\VS{6}qui s'en retourna à Jizreel, pour guérir des blessures que les Syriens lui avaient faites à Rama, lorsqu'il faisait la guerre contre Hazaël, roi de Syrie. Azaria, fils de Joram, roi de Juda, descendit pour voir Joram, le fils d'Achab, à Jizreel, parce qu'il était malade.
\VS{7}Dieu fit pour sa ruine qu'Achazia vint auprès de Joram. En effet, quand il fut arrivé, il sortit avec Joram pour aller au-devant de Jéhu, fils de Nimschi, que Yahweh avait oint pour retrancher la maison d'Achab.
\VS{8}Et comme Jéhu faisait justice de la maison d'Achab\FTNT{2 R. 10:12-30.}, il trouva les chefs de Juda et les fils des frères d'Achazia, qui servaient Achazia, et il les tua.
\VS{9}Il chercha ensuite Achazia, qui s'était caché en Samarie. On le prit, et on l'amena vers Jéhu qui le fit mourir. Puis on l'ensevelit, car on dit : C'est le fils de Josaphat, qui cherchait Yahweh de tout son cœur. Et il n'y eut plus personne dans la maison d'Achazia qui fut capable de régner.
\TextTitle{Joas échappe au massacre de sa famille\FTNTT{2 R. 11:1-3}}
\VS{10}Or Athalie, mère d'Achazia, voyant que son fils était mort, se leva et fit périr toute la race royale de la maison de Juda\FTNT{1 R. 11:1-3.}.
\VS{11}Mais Joschabeath, fille du roi Joram, prit Joas, fils d'Achazia, en le dérobant d'entre les fils du roi qu'on faisait mourir. Elle le mit avec sa nourrice dans la salle des lits. Ainsi Joschabeath, fille du roi Joram et femme de Jehojada le sacrificateur, étant la sœur d'Achazia, le cacha de la vue d'Athalie, qui ne put le faire mourir.
\VS{12}Il fut ainsi caché avec eux dans la maison de Dieu six ans ; et c'est Athalie qui régnait sur le pays.
\Chap{23}
\TextTitle{Joas devient roi grâce à Jehojada\FTNTT{2 R. 11:4-12}}
\VerseOne{}Mais la septième année, Jehojada prit courage et traita alliance avec les chefs de centaines, Azaria, fils de Jerocham, Ismaël, fils de Jochanan, Azaria, fils d'Obed, Maaséja, fils d'Adaja, et Elischaphath, fils de Zicri.
\VS{2}Ils firent le tour de Juda, pour rassembler de toutes les villes de Juda les Lévites et les chefs des pères d'Israël ; puis ils vinrent à Jérusalem.
\VS{3}Et toute cette assemblée traita alliance avec le roi dans la maison de Dieu. Jehojada leur dit : Voici, c'est le fils du roi qui régnera, selon la parole de Yahweh au sujet des fils de David.
\VS{4}Vous ferez donc ceci : Le tiers qui parmi vous entre en service au sabbat, sacrificateurs et Lévites, fera la garde des seuils.
\VS{5}Un autre tiers se tiendra dans la maison du roi, et un tiers à la porte de Jesod ; et tout le peuple sera dans les parvis de la maison de Yahweh.
\VS{6}Que personne n'entre dans la maison de Yahweh, sauf les sacrificateurs et les Lévites de service : Ils entreront car ils sont sanctifiés ; et tout le reste du peuple gardera les ordres de Yahweh.
\VS{7}Les Lévites environneront le roi de toutes parts, tenant chacun leurs armes à la main, et donneront la mort à quiconque voudra entrer dans la maison ; vous serez avec le roi quand il entrera et quand il sortira.
\VS{8}Les Lévites et tout Juda firent tout ce que Jehojada le sacrificateur, avait ordonné. Ils prirent chacun leurs gens, tant ceux qui entraient en service que ceux qui en sortaient au sabbat ; car Jehojada, le sacrificateur, n'avait exempté aucune classe.
\VS{9}Et Jehojada le sacrificateur, donna aux chefs de centaines les lances, les grands et les petits boucliers qui provenaient du roi David, et qui étaient dans la maison de Dieu.
\VS{10}Puis il rangea tout le peuple autour du roi, chacun tenant ses armes à la main, du côté droit du temple jusqu'au côté gauche de la maison, près de l'autel et de la maison.
\VS{11}Alors ils firent sortir le fils du roi, et mirent sur lui la couronne et le témoignage. Ils l'établirent roi, et Jehojada et ses fils l'oignirent et dirent : Vive le roi !
\TextTitle{Mort d'Athalie\FTNTT{2 R. 11:13-16}}
\VS{12}Mais Athalie, entendant le bruit du peuple qui courait et célébrait le roi, vint vers le peuple, dans la maison de Yahweh.
\VS{13}Elle regarda, et voici, le roi se tenait près de la colonne, à l'entrée ; les chefs et les trompettes étaient près du roi, et tout le peuple du pays était dans la joie, et l'on sonnait des trompettes ; les chantres, avec des instruments de musique, dirigeaient les chants de louanges. Alors Athalie déchira ses vêtements et dit : Conspiration ! Conspiration !
\VS{14}Le sacrificateur Jehojada fit sortir les chefs de centaines qui étaient à la tête de l'armée, et leur dit : Faites-la sortir hors des rangs, et que celui qui la suivra soit mis à mort par l'épée ! Car le sacrificateur avait dit : Ne la mettez pas à mort dans la maison de Yahweh.
\VS{15}Ils mirent donc la main sur elle pour la faire entrer dans la maison du roi, par l'entrée de la porte des chevaux ; et là ils la firent mourir.
\TextTitle{Jehojada fait asseoir Joas sur le trône de Juda\FTNTT{2 R. 11:17-20}}
\VS{16}Puis Jehojada traita, avec tout le peuple et le roi, une alliance pour être le peuple de Yahweh.
\VS{17}Et tout le peuple entra dans la maison de Baal pour la détruire. Ils brisèrent ses autels et ses images et ils tuèrent devant les autels Matthan, sacrificateur de Baal.
\VS{18}Jehojada remit aussi les fonctions de la maison de Yahweh entre les mains des sacrificateurs, des Lévites, comme David les avait répartis dans la maison de Yahweh, afin qu'ils élèvent des holocaustes à Yahweh, comme cela est écrit dans la loi de Moïse, avec joie et avec des chants, selon les ordonnances de David.
\VS{19}Il établit aussi les portiers aux portes de la maison de Yahweh, afin qu'il n'y entrât aucune personne souillée de quelque manière que ce fût.
\VS{20}Il prit les chefs de centaines, hommes considérés, qui avaient de l'autorité parmi le peuple, et tout le peuple du pays. Il fit descendre le roi, de la maison de Yahweh à la maison du roi, en entrant par la porte supérieure ; et ils firent asseoir le roi sur le trône royal.
\VS{21}Alors tout le peuple du pays se réjouit, et la ville fut tranquille, bien qu'on eût mis à mort Athalie par l'épée.
\Chap{24}
\TextTitle{Joas règne sur Juda ; ses travaux sur le temple\FTNTT{2 R. 11:21-12:8}}
\VerseOne{}Joas était âgé de sept ans quand il devint roi, et il régna quarante ans à Jérusalem. Sa mère avait pour nom Tsibja, de Beer-Schéba\FTNT{2 R. 11:21 ; 2 R. 12:1-3.}.
\VS{2}Joas fit ce qui est droit aux yeux de Yahweh, pendant toute la vie de Jehojada, le sacrificateur.
\VS{3}Et Jehojada prit pour lui deux femmes, dont il eut des fils et des filles.
\VS{4}Après cela Joas eut la pensée de renouveler la maison de Yahweh\FTNT{2 R. 12:4-16.}.
\VS{5}Il assembla donc les sacrificateurs et les Lévites, et leur dit : Allez vers les villes de Juda, et recueillez de l'argent par tout Israël, suffisamment pour réparer la maison de votre Dieu d'année en année, et hâtez-vous pour cette affaire. Mais les Lévites ne se hâtèrent point.
\VS{6}Alors le roi appela Jehojada, leur chef, et lui dit : Pourquoi n'as-tu pas veillé à ce que les Lévites aient apporté de Juda et de Jérusalem, l'impôt sur l'assemblée d'Israël, selon Moïse, serviteur de Yahweh, pour la tente du témoignage ?
\VS{7}Car l'impie Athalie et ses fils ont ravagé la maison de Dieu ; et même ils ont employé pour les Baals toutes les choses consacrées à la maison de Yahweh.
\TextTitle{Offrandes volontaires pour la réparation du temple\FTNTT{2 R. 12:9-16}}
\VS{8}Et le roi ordonna qu'on fasse un seul coffre, et qu'on le mette à la porte de la maison de Yahweh, à l'extérieur.
\VS{9}Puis on publia dans Juda et dans Jérusalem, pour qu'on qu'on apportât à Yahweh l'impôt mis par Moïse, serviteur de Dieu, sur Israël dans le désert.
\VS{10}Tous les chefs et tout le peuple s'en réjouirent, et l'on apporta et jeta le tribut dans le coffre, jusqu'à ce qu'il fût plein.
\VS{11}Au moment venu, les Lévites apportaient le coffre aux inspecteurs du roi, car ceux-ci voyaient qu'il y avait beaucoup d'argent. Un secrétaire du roi et un commissaire du souverain sacrificateur venaient et vidaient le coffre ; puis ils le rapportaient et le remettaient à sa place. Ils faisaient ainsi jour après jour, et ils recueillaient de l'argent en abondance.
\VS{12}Le roi et Jehojada le donnaient à ceux qui étaient chargés de l'ouvrage pour le service de la maison de Yahweh, et ceux-ci engageaient des tailleurs de pierres et des charpentiers pour réparer la maison de Yahweh, et aussi des ouvriers pour le fer et l'airain, afin de réparer la maison de Yahweh.
\VS{13}Ceux qui étaient chargés de l'ouvrage travaillèrent donc ; et par leurs mains, les travaux s'exécutèrent, de sorte qu'ils rétablirent la maison de Dieu en son état, et l'affermirent.
\VS{14}Lorsqu'ils eurent achevé, ils apportèrent devant le roi et devant Jehojada le reste de l'argent ; et on fit faire des ustensiles pour la maison de Yahweh, des ustensiles pour le service et pour les holocaustes, des coupes et d'autres ustensiles d'or et d'argent. Et on offrit continuellement des holocaustes dans la maison de Yahweh, tant que vécut Jehojada.
\TextTitle{Mort de Jehojada, Joas abandonne Yahweh\FTNTT{2 R. 12:9-16}}
\VS{15}Or Jehojada devint vieux et rassasié de jours, et il mourut ; il était âgé de cent trente ans quand il mourut.
\VS{16}On l'ensevelit dans la cité de David avec les rois ; car il avait fait du bien à Israël, et à l'égard de Dieu et de sa maison.
\VS{17}Mais, après la mort de Jehojada, les chefs de Juda vinrent et se prosternèrent devant le roi ; et le roi les écouta.
\VS{18}Ils abandonnèrent la maison de Yahweh, le Dieu de leurs pères, et ils servirent les idoles d'Astarté et les faux dieux ; et la colère de Yahweh fut sur Juda et sur Jérusalem, parce qu'ils s'étaient ainsi rendus coupables.
\VS{19}Yahweh envoya parmi eux des prophètes, pour les faire retourner à lui par leurs avertissements ; mais ils ne voulurent point les écouter.
\VS{20}Alors l'Esprit de Dieu revêtit Zacharie, fils de Jehojada, le sacrificateur, et se tenant devant le peuple, il leur dit : Dieu m'a parlé ainsi : Pourquoi transgressez-vous les commandements de Yahweh ? Vous ne prospérerez point ; car vous avez abandonné Yahweh, et il vous abandonnera aussi.
\VS{21}Mais ils se liguèrent contre lui et le lapidèrent, par ordre du roi, dans le parvis de la maison de Yahweh.
\VS{22}Ainsi le roi Joas ne se souvint point de la bonté dont Jehojada, père de Zacharie, avait usé envers lui ; et il tua son fils, qui dit en mourant : Yahweh le voit, et il en demandera compte !
\TextTitle{Invasion des Syriens, conspiration et mort de Joas\FTNTT{2 R. 12:17-21 ; cp. 2 R. 13:7}}
\VS{23}A la fin de cette année-là, l'armée de Syrie monta contre Joas, et vint en Juda et à Jérusalem. Ils détruisirent, d'entre le peuple, tous les chefs du peuple, et ils envoyèrent au roi de Damas tout leur butin.
\VS{24}Et quoique l'armée venue de Syrie fût peu nombreuse, Yahweh livra entre leurs mains une armée très nombreuse, parce qu'ils avaient abandonné Yahweh, le Dieu de leurs pères. Ainsi les Syriens furent le châtiment de Joas.
\VS{25}Quand ils s'éloignèrent de lui, après l'avoir laissé dans de grandes souffrances, ses serviteurs conspirèrent contre lui, à cause du sang des fils de Jehojada, le sacrificateur ; ils le tuèrent sur son lit, et il mourut. On l'ensevelit dans la cité de David, mais on ne l'ensevelit pas dans les sépulcres des rois.
\VS{26}Ce sont ici ceux qui conspirèrent contre lui : Zabad, fils de Schimeath, femme ammonite, et Jozabad, fils de Schimrith, femme Moabite.
\VS{27}Quant à ses fils et à la grande charge qui reposa sur lui, et à la réparation de la maison de Dieu, voici, ces choses sont écrites dans les mémoires du livre des rois. Amatsia, son fils, régna à sa place.
\Chap{25}
\TextTitle{Amatsia règne sur Juda\FTNTT{2 R. 12:21 ; 2 R. 14:1-6}}
\VerseOne{}Amatsia devint roi à l'âge de vingt-cinq ans, et il régna vingt-neuf ans à Jérusalem. Sa mère avait pour nom Joaddan, de Jérusalem\FTNT{2 R. 12:21 ; 2 R. 14:1-20}.
\VS{2}Il fit ce qui est droit aux yeux de Yahweh, mais non d'un cœur entier.
\VS{3}Après qu'il fut affermi dans son règne, il fit mourir ses serviteurs qui avaient tué le roi, son père.
\VS{4}Mais il ne fit pas mourir leurs fils ; car il fit selon ce qui est écrit dans la loi, dans le livre de Moïse, où Yahweh a donné ce commandement : Les pères ne mourront point pour les fils, et les fils ne mourront point pour les pères ; mais chacun mourra pour son péché.
\TextTitle{Amatsia en guerre contre les Edomites, sa victoire\FTNTT{2 R. 14:7}}
\VS{5}Puis Amatsia rassembla ceux de Juda, et il les rangea selon les maisons paternelles, par chefs de milliers et par chefs de centaines, pour tout Juda et Benjamin ; il en fit le dénombrement depuis l'âge de vingt ans et au-dessus ; et il trouva trois cent mille hommes d'élite, propres à l'armée, maniant la lance et le bouclier.
\VS{6}Il prit aussi à sa solde, pour cent talents d'argent, cent mille vaillants hommes de guerre d'Israël.
\VS{7}Mais un homme de Dieu vint à lui, et lui dit : Ô roi ! Que l'armée d'Israël ne marche point avec toi ; car Yahweh n'est point avec Israël ni avec tous ces fils d'Ephraïm.
\VS{8}Si tu vas avec eux, quand bien même tu ferais de vaillants combats, Dieu te fera tomber devant l'ennemi ; car Dieu a la puissance d'aider et de faire tomber.
\VS{9}Amatsia dit à l'homme de Dieu : Mais que faire des cent talents que j'ai donnés à la troupe d'Israël ? L'homme de Dieu dit : Yahweh peut t'en donner beaucoup plus.
\VS{10}Ainsi Amatsia sépara les troupes qui lui étaient venues d'Ephraïm, et les fit retourner chez elles ; mais leur colère s'enflamma très ardemment contre Juda, et ces gens retournèrent chez eux dans une grande colère.
\VS{11}Alors Amatsia prit courage, conduisit son peuple et s'en alla dans la vallée du sel, où il battit dix mille hommes des fils de Séir.
\VS{12}Les fils de Juda prirent dix mille hommes vivants, et les ayant amenés sur le sommet d'une roche, ils les jetèrent du haut de la roche, de sorte qu'ils furent tous brisés.
\VS{13}Mais les gens de la troupe qu'Amatsia avait renvoyée, afin qu'ils n'aillent pas avec lui à la guerre, firent une incursion dans les villes de Juda, depuis Samarie jusqu'à Beth-Horon. Ils y tuèrent trois mille personnes et emportèrent un gros butin.
\TextTitle{Idolâtrie d'Amatsia\FTNTT{2 R. 14:7}}
\VS{14}Lorsqu'Amatsia fut de retour de la défaite des Edomites, et ayant apporté les dieux des fils de Séir, il se les établit pour dieux ; il se prosterna devant eux et leur brûla de l'encens.
\VS{15}Et la colère de Yahweh s'enflamma contre Amatsia, et il envoya vers lui un prophète qui lui dit : Pourquoi as-tu recherché les dieux d'un peuple qui n'ont point délivré leur peuple de ta main ?
\VS{16}Et comme il parlait au roi, il lui répondit : T'a-t-on établi conseiller du roi ? Cesse maintenant ! Pourquoi veux-tu qu'on te tue ? Et le prophète se retira, mais en disant : Je sais que Dieu a résolu de te détruire, parce que tu as fait cela, et que tu n'as point écouté mon conseil.
\TextTitle{Défaite d'Amatsia contre Israël\FTNTT{2 R. 14:8-14}}
\VS{17}Puis Amatsia, roi de Juda, ayant tenu conseil, envoya vers Joas, fils de Joachaz, fils de Jéhu, roi d'Israël, pour lui dire : Viens, voyons-nous en face !
\VS{18}Mais Joas, roi d'Israël, envoya dire à Amatsia, roi de Juda : L'épine du Liban envoya dire au cèdre du Liban : Donne ta fille pour femme à mon fils ! Et les bêtes sauvages qui sont au Liban passèrent et foulèrent l'épine.
\VS{19}Voici, tu dis que tu as frappé les Edomites, et ton cœur s'est élevé pour te glorifier. Maintenant, reste dans ta maison ! Pourquoi t'engagerais-tu dans un combat où tu tomberais, et Juda avec toi ?
\VS{20}Mais Amatsia ne l'écouta point ; Dieu avait résolu de le livrer aux mains de Joas parce qu'il eût recours aux dieux d'Edom.
\VS{21}Joas, roi d'Israël, monta ; et ils se virent en face, lui et Amatsia, roi de Juda, à Beth-Schémesch, qui est de Juda.
\VS{22}Juda fut battu en face d'Israël, et chacun s'enfuit dans sa tente.
\VS{23}Joas, roi d'Israël, prit Amatsia, roi de Juda, fils de Joas, fils de Joachaz, à Beth-Schémesch. Il l'emmena à Jérusalem et fit une brèche de quatre cents coudées dans la muraille de Jérusalem, depuis la porte d'Ephraïm jusqu'à la porte de l'angle.
\VS{24}Il prit l'or, l'argent, tous les vases qui se trouvaient dans la maison de Dieu sous la garde d'Obed-Edom, les trésors de la maison du roi ; il fit des otages et il retourna à Samarie.
\TextTitle{Assassinat d'Amatsia\FTNTT{2 R. 14:17-20}}
\VS{25}Amatsia, fils de Joas, roi de Juda, vécut quinze ans, après que Joas, fils de Joachaz, roi d'Israël, mourut.
\VS{26}Le reste des actions d'Amatsia, les premières et les dernières, voici cela n'est-il pas écrit dans le livre des rois de Juda et d'Israël ?
\VS{27}Or depuis le moment où Amatsia se détourna de Yahweh, on fit une conspiration contre lui à Jérusalem, et il s'enfuit à Lakis ; mais on le poursuivit à Lakis, et on le fit mourir.
\VS{28}Puis on le transporta sur des chevaux, et on l'ensevelit avec ses pères dans la ville de Juda.
\Chap{26}
\TextTitle{Ozias règne sur Juda ; il est fidèle à Yahweh\FTNTT{2 R. 14:21-15:4}}
\VerseOne{}Alors, tout le peuple de Juda prit Ozias, âgé de seize ans, et l'établit roi à la place de son père Amatsia\FTNT{2 R. 14:21 ; 2 R. 15:1-4.}.
\VS{2}Ce fut lui qui bâtit Eloth, et la ramena sous la puissance de Juda, après que le roi se fut endormi avec ses pères.
\VS{3}Ozias était âgé de seize ans quand il devint roi, et il régna cinquante-deux ans à Jérusalem. Sa mère avait pour nom Jecolia, de Jérusalem.
\VS{4}Il fit ce qui est droit aux yeux de Yahweh, comme avait fait Amatsia, son père.
\VS{5}Il s'appliqua à rechercher Dieu pendant les jours de Zacharie, qui avait une intelligence dans les visions de Dieu et pendant les jours où il rechercha Yahweh, Dieu le fit prospérer.
\VS{6}Il sortit et fit la guerre contre les Philistins. Il brisa la muraille de Gath, la muraille de Jabné, et la muraille d'Asdod ; et il bâtit des villes dans le pays d'Asdod et chez les Philistins.
\VS{7}Dieu le secourut contre les Philistins et contre les Arabes qui habitaient à Gur-Baal, et contre les Maonites.
\VS{8}Même les Ammonites faisaient des présents à Ozias, et sa renommée parvint jusqu'à l'entrée de l'Egypte ; car il était devenu très puissant.
\VS{9}Ozias bâtit des tours à Jérusalem, sur la porte de l'angle, sur la porte de la vallée, sur l'angle, et il les fortifia.
\VS{10}Il bâtit des tours dans le désert, et il creusa de nombreux puits, parce qu'il avait de nombreux troupeaux dans la plaine et dans la campagne, des laboureurs et des vignerons sur les montagnes, et au Carmel ; car il aimait l'agriculture.
\VS{11}Ozias avait une armée de gens de guerre, allant à la guerre par bandes, selon le compte de leur dénombrement fait par Jeïel le scribe, et Maaséja le commissaire, et sous la conduite de Hanania l'un des chefs du roi.
\VS{12}Le nombre total des chefs des maisons paternelles, des vaillants guerriers, était de deux mille six cents.
\VS{13}Il y avait sous leur conduite une armée de trois cent sept mille cinq cents combattants, tous gens de guerre, puissants et vaillants, capables de soutenir le roi contre l'ennemi.
\VS{14}Ozias leur procura, pour toute l'armée, des boucliers, des lances, des casques, des cuirasses, des arcs et des pierres de fronde.
\VS{15}Il fit faire à Jérusalem des machines inventées par un ingénieur, pour être placées sur les tours et sur les angles, pour lancer des flèches et de grosses pierres. Et sa renommée se répandit au loin ; car il fut extraordinairement soutenu, jusqu'à ce qu'il devienne fort puissant.
\TextTitle{Ozias pèche et est frappé de lèpre\FTNTT{2 R. 15:5-7, 32}}
\VS{16}Mais dès qu'il fut puissant, son cœur s'éleva pour le corrompre. Et il pécha contre Yahweh, son Dieu : Il entra dans le temple de Yahweh pour brûler des parfums sur l'autel des parfums\FTNT{2 R. 15:5-7.}.
\VS{17}Mais Azaria le sacrificateur, entra après lui, et avec lui quatre-vingts sacrificateurs de Yahweh, hommes vaillants,
\VS{18}qui s'opposèrent au roi Ozias, et lui dirent : Ce n'est pas à toi, Ozias, d'offrir le parfum à Yahweh, mais aux sacrificateurs, fils d'Aaron, qui sont consacrés pour cela. Sors du sanctuaire, car tu as péché ! Et cela ne sera pas à ta gloire devant Yahweh Dieu.
\VS{19}Alors Ozias, qui avait à la main un encensoir pour faire brûler le parfum, se mit en colère. Et comme il s'irritait contre les sacrificateurs, la lèpre parut sur son front, en présence des sacrificateurs, dans la maison de Yahweh, près de l'autel des parfums.
\VS{20}Azaria, le principal sacrificateur, le regarda ainsi que tous les sacrificateurs. Et voici, il avait de la lèpre sur le front. Ils le pressèrent et lui-même se hâta de sortir, parce que Yahweh l'avait frappé.
\VS{21}Le roi Ozias fut ainsi lépreux jusqu'au jour de sa mort ; il habita seul comme lépreux dans une maison écartée, car il était exclu de la maison de Yahweh. Et Jotham, son fils, avait la charge de la maison du roi, jugeant le peuple du pays.
\VS{22}Esaïe, fils d'Amots, le prophète, a écrit le reste des actions d'Ozias, les premières et les dernières.
\VS{23}Ozias s'endormit avec ses pères, et on l'ensevelit avec ses pères dans le champ de la sépulture des rois ; car on dit : Il est lépreux. Et Jotham, son fils, régna à sa place.
\Chap{27}
\TextTitle{Jotham règne sur Juda ; sa mort\FTNTT{2 R. 15:7, 32-38}}
\VerseOne{}Jotham était âgé de vingt-cinq ans quand il devint roi, et il régna seize ans à Jérusalem\FTNT{1 R. 15:7.}. Sa mère avait le nom de Jeruscha, fille de Tsadok.
\VS{2}Il fit ce qui est droit aux yeux de Yahweh, tout comme Ozias, son père, avait fait ; mais il n'entra pas dans le temple de Yahweh. Néanmoins, le peuple se corrompait encore.
\VS{3}Ce fut lui qui bâtit la porte supérieure de la maison de Yahweh, et il fit beaucoup de constructions sur les murs de la colline.
\VS{4}Il bâtit des villes sur les montagnes de Juda, des châteaux et des tours dans les forêts.
\VS{5}Il fut en guerre avec le roi des fils d'Ammon, et fut le plus fort. Cette année-là, les fils d'Ammon lui donnèrent cent talents d'argent, dix mille cors de froment, et dix mille d'orge. Les fils d'Ammon lui en donnèrent autant la seconde et la troisième année.
\VS{6}Jotham devint donc très puissant, parce qu'il avait affermi ses voies devant Yahweh, son Dieu.
\VS{7}Le reste des actions de Jotham, tous ses combats et sa conduite, voici, toutes ces choses sont écrites dans le livre des rois d'Israël et de Juda.
\VS{8}Il était âgé de vingt-cinq ans quand il devint roi, et il régna seize ans à Jérusalem.
\VS{9}Puis Jotham s'endormit avec ses pères, et on l'ensevelit dans la cité de David. Et Achaz, son fils, régna à sa place.
\Chap{28}
\TextTitle{Achaz règne sur Juda\FTNTT{2 R. 15:38-16:4}}
\VerseOne{}Achaz était âgé de vingt ans quand il devint roi, et il régna seize ans à Jérusalem\FTNT{2 R. 15:38 ; 2 R. 16:1-4.}. Il ne fit point ce qui est droit aux yeux de Yahweh, comme David, son père.
\VS{2}Il suivit la voie des rois d'Israël ; et il fit même des images de fonte pour les Baals.
\VS{3}Il brûla des parfums dans la vallée du fils de Hinnom, et il brûla ses fils au feu, suivant les abominations des nations que Yahweh avait chassées devant les enfants d'Israël.
\VS{4}Il offrait aussi des sacrifices et brûlait des parfums dans les hauts lieux, sur les collines, et sous tout arbre vert.
\TextTitle{La Syrie et Israël envahissent Juda\FTNTT{2 R. 16:5-6}}
\VS{5}C'est pourquoi Yahweh, son Dieu, le livra entre les mains du roi de Syrie. Les Syriens le battirent et lui prirent un grand nombre de prisonniers, qu'ils emmenèrent à Damas. Il fut livré aussi entre les mains du roi d'Israël, qui lui fit endurer une grande défaite\FTNT{2 R . 16:5-20.}.
\VS{6}Car Pékach, fils de Remalia, tua en un seul jour en Juda cent vingt mille hommes, tous vaillants, parce qu'ils avaient abandonné Yahweh, le Dieu de leurs pères.
\VS{7}Zicri, homme vaillant d'Ephraïm, tua Maaséja, fils du roi, et Azrikam, chef de la maison, et Elkana, le second après le roi.
\VS{8}Les fils d'Israël emmenèrent prisonniers deux cent mille de leurs frères, tant femmes que fils et filles ; ils firent aussi sur eux un gros butin. Ils emmenèrent le butin à Samarie.
\TextTitle{Les captifs de Juda libérés grâce à Obed}
\VS{9}Or il y avait un prophète de Yahweh nommé Oded. Il sortit au-devant de cette armée qui revenait à Samarie, et leur dit : Voici, Yahweh, le Dieu de vos pères, étant indigné contre Juda, les a livrés entre vos mains, et vous les avez tués avec une colère telle qu'elle est parvenue aux cieux.
\VS{10}Et maintenant, vous pensez assujettir les fils de Juda et de Jérusalem pour serviteurs et pour servantes ! Mais n'êtes-vous pas également coupables envers Yahweh, votre Dieu ?
\VS{11}Maintenant écoutez-moi, et ramenez les prisonniers que vous vous êtes faits parmi vos frères ; car la colère ardente de Yahweh est sur vous.
\VS{12}Alors quelques-uns des chefs des fils d'Ephraïm, Azaria, fils de Jochanan, Bérékia, fils de Meschillémoth, Ezéchias, fils de Schallum, et Amasa, fils de Hadlaï, s'élevèrent contre ceux qui retournaient de la guerre,
\VS{13}et leur dirent : Vous ne ferez point entrer ici ces captifs. C'est pour nous rendre coupables devant Yahweh, voulez-vous en rajouter à nos péchés et à notre culpabilité ; car nous sommes déjà grandement coupables, et une colère ardente est sur Israël.
\VS{14}Alors les soldats abandonnèrent les captifs et le butin devant les chefs et toute l'assemblée.
\VS{15}Et des hommes, désignés par leurs noms, se levèrent, prirent les captifs, utilisèrent le butin pour revêtir tous ceux d'entre eux qui étaient nus avec des vêtements et des chaussures. Ils leur donnèrent à manger et à boire, les oignirent et ils conduisirent sur des ânes tous ceux qui étaient affaiblis pour les emmener à Jéricho, la ville des palmiers, auprès de leurs frères ; puis ils s'en retournèrent à Samarie.
\TextTitle{Achaz fait appel aux Assyriens\FTNTT{2 R. 15:29 ; 16:7-18}}
\VS{16}En ce temps-là, le roi Achaz envoya demander du secours aux rois d'Assyrie.
\VS{17}Les Edomites étaient revenus, avaient battu Juda et avaient emmené des prisonniers.
\VS{18}Les Philistins s'étaient aussi jetés sur les villes de la plaine et du sud de Juda ; et ils avaient pris Beth-Schémesch, Ajalon, Guedéroth, Soco et les villes de son ressort, Thimna et les villes de son ressort, Guimzo et les villes de son ressort, et ils y demeurèrent.
\VS{19}Car Yahweh humilia Juda, à cause d'Achaz, roi d'Israël, parce qu'il avait mis le désordre en Juda, et qu'il avait commis des transgressions contre Yahweh.
\VS{20}Tilgath-Pilnéser, roi d'Assyrie, vint vers lui ; mais il l'assiégea, et ne le fortifia pas.
\VS{21}Or Achaz dépouilla la maison de Yahweh, la maison du roi et celle des chefs, pour faire des dons au roi d'Assyrie, mais sans avoir du secours.
\TextTitle{Achaz irrite Yahweh par ses péchés}
\VS{22}Dans le temps de sa détresse, il continua à pécher contre Yahweh, lui, le roi Achaz.
\VS{23}Il sacrifia aux dieux de Damas qui l'avaient battu, et il dit : Puisque les dieux des rois de Syrie leur viennent en aide, je leur sacrifierai, afin qu'ils me viennent en aide. Mais ils furent la cause de sa chute et de celle de tout Israël.
\VS{24}Or Achaz rassembla les ustensiles de la maison de Dieu, et il mit en pièces les ustensiles de la maison de Dieu. Il ferma les portes de la maison de Yahweh, et se fit des autels dans tous les coins de Jérusalem.
\VS{25}Il fit des hauts lieux dans chaque ville de Juda, pour offrir des parfums à d'autres dieux ; et il irrita Yahweh, le Dieu de ses pères.
\TextTitle{Mort d'Achaz\FTNTT{2 R. 16:19-20}}
\VS{26}Quant au reste de ses actions et de toutes ses voies, les premières et les dernières, voici, elles sont écrites dans le livre des rois de Juda et d'Israël.
\VS{27}Puis Achaz s'endormit avec ses pères, et on l'ensevelit dans la ville de Jérusalem ; car on ne le mit point dans les sépulcres des rois d'Israël. Et Ezéchias, son fils, régna à sa place.
\Chap{29}
\TextTitle{Ezéchias règne sur Juda ; le réveil du peuple\FTNTT{2 R. 18:1-7 ; cp. Es. 36-39}}
\VerseOne{}Ezéchias devint roi à l'âge de vingt-cinq ans, et il régna vingt-neuf ans à Jérusalem\FTNT{ Es. 36 ; Es. 37 ; Es. 38, Es. 39 ; 2 R. 18:1-7 ;.}. Sa mère avait pour nom Abija, fille de Zacharie.
\VS{2}Il fit ce qui est droit aux yeux de Yahweh, tout comme avait fait David, son père.
\VS{3}La première année de son règne, au premier mois, il ouvrit les portes de la maison de Yahweh, et il les répara.
\VS{4}Il fit venir les sacrificateurs et les Lévites, et les rassembla dans la place orientale.
\VS{5}Et il leur dit : Ecoutez-moi, Lévites ! Sanctifiez-vous et sanctifiez la maison de Yahweh, le Dieu de vos pères, et ôtez du sanctuaire tout ce qui est impur.
\VS{6}Car nos pères ont péché, ils ont fait ce qui est mal aux yeux de Yahweh, notre Dieu. Ils l'ont abandonné, ils ont détourné leurs faces du tabernacle de Yahweh et lui ont tourné le dos.
\VS{7}Ils ont même fermé les portes du portique et ont éteint les lampes, ils n'ont fait ni monter d'offrande, ni brûler du parfum et des holocaustes au Dieu d'Israël dans le sanctuaire.
\VS{8}C'est pourquoi la colère de Yahweh a été sur Juda et sur Jérusalem ; et il les a livrés à de grands troubles, à la ruine et à la moquerie, comme vous le voyez de vos yeux.
\VS{9}Car voici, nos pères sont tombés par l'épée, et nos fils, nos filles et nos femmes sont en captivité.
\VS{10}Maintenant donc j'ai à cœur de traiter alliance avec Yahweh, le Dieu d'Israël, pour que son ardente colère se détourne de nous.
\VS{11}Or mes fils, cessez d'être négligents ; car Yahweh vous a choisis, afin que vous vous teniez devant lui à son service, comme ses serviteurs, pour lui brûler des parfums.
\VS{12}Les Lévites se levèrent : Machath, fils d'Amasaï, Joël, fils d'Azaria, des fils des Kehathites ; et des fils des Merarites, Kis, fils d'Abdi, Azaria, fils de Jehalléleel ; et des Guerschonites, Joach, fils de Zimma, et Eden, fils de Joach ;
\VS{13}et des fils d'Elitsaphan, Schimri et Jeïel ; et des fils d'Asaph, Zacharie et Matthania ;
\VS{14}et des fils d'Héman, Jehiel et Schimeï, et des fils de Jeduthun, Schemaeja et Uzziel.
\VS{15}Ils assemblèrent leurs frères, et ils se sanctifièrent ; puis ils entrèrent selon l'ordre du roi, et d'après la parole de Yahweh, pour purifier la maison de Yahweh.
\VS{16}Ainsi les sacrificateurs entrèrent à l'intérieur de la maison de Yahweh pour la purifier. Ils firent sortir dans le parvis de la maison de Yahweh toutes les impuretés qu'ils trouvèrent dans le temple de Yahweh. Les Lévites les prirent pour les emporter dehors, au torrent de Cédron.
\VS{17}Ils commencèrent à sanctifier le temple le premier jour du premier mois. Le huitième jour du mois, ils entrèrent au portique de Yahweh, et ils sanctifièrent la maison de Yahweh pendant huit jours. Le seizième jour du premier mois, ils avaient achevé.
\VS{18}Puis ils se rendirent chez le roi Ezéchias, et dirent : Nous avons purifié toute la maison de Yahweh, l'autel des holocaustes et ses ustensiles, la table des pains de proposition et ses ustensiles\FTNT{Ex. 29.}.
\VS{19}Nous avons remis en état et sanctifié tous les ustensiles que le roi Achaz avait rendus odieux pendant son règne, par ses transgressions ; ils sont maintenant devant l'autel de Yahweh.
\TextTitle{Nouvelle consécration du temple}
\VS{20}Alors le roi Ezéchias se leva de bonne heure, rassembla les chefs de la ville, et monta à la maison de Yahweh.
\VS{21}Ils amenèrent sept taureaux, sept béliers, sept agneaux et sept boucs sans défaut, en sacrifice pour le péché, pour le royaume, pour le sanctuaire et pour Juda\FTNT{Lé 4:3-26}. Puis le roi dit aux sacrificateurs, fils d'Aaron, de les faire monter en offrande sur l'autel de Yahweh.
\VS{22}Ils égorgèrent donc les bœufs, et les sacrificateurs recueillirent le sang et en aspergèrent l'autel ; ils égorgèrent les béliers et aspergèrent le sang sur l'autel ; ils égorgèrent les agneaux et aspergèrent le sang sur l'autel.
\VS{23}Puis on fit approcher les boucs pour le sacrifice du péché, devant le roi et devant l'assemblée, et ils posèrent leurs mains sur eux\FTNT{Lé 8:14.}.
\VS{24}Alors les sacrificateurs les égorgèrent, et offrirent en expiation leur sang vers l'autel, afin de faire propitiation pour tout Israël ; car le roi avait ordonné cet holocauste et ce sacrifice d'expiation pour tout Israël.
\VS{25}Il plaça aussi les Lévites dans la maison de Yahweh, avec des cymbales, des luths et des harpes, comme l'avait ordonné David, Gad, le voyant du roi, et Nathan le prophète ; car c'était un commandement de Yahweh, par ses prophètes.
\VS{26}Les Lévites se tinrent donc là avec les instruments de David, et les sacrificateurs avec les trompettes.
\VS{27}Alors Ezéchias ordonna de faire monter en offrande l'holocauste sur l'autel ; et au moment où commença l'holocauste, le cantique de Yahweh commença aussi, avec les trompettes et les instruments de David, roi d'Israël.
\VS{28}Toute l'assemblée se prosterna en chantant le cantique, et les trompettes sonnèrent ; et cela continua jusqu'à ce que l'holocauste fût achevé.
\VS{29}Et quand on eut achevé de faire monter l'holocauste, le roi et tous ceux qui se trouvaient avec lui fléchirent les genoux et se prosternèrent.
\VS{30}Puis le roi Ezéchias et les chefs dirent aux Lévites de célébrer Yahweh par les paroles de David et d'Asaph le voyant ; et ils le célébrèrent dans des réjouissances, et s'inclinèrent pour se prosterner.
\VS{31}Alors Ezéchias prit la parole, et dit : Vous avez maintenant consacré vos mains à Yahweh. Approchez-vous, amenez des sacrifices et faites des sacrifices de reconnaissances dans la maison de Yahweh. Et l'assemblée amena des sacrifices et firent des sacrifices de reconnaissances, et tous ceux qui étaient d'un cœur volontaire offrirent des holocaustes.
\VS{32}Le nombre des holocaustes que l'assemblée offrit fut de soixante-dix taureaux, cent béliers, deux cents agneaux, le tout en holocauste à Yahweh.
\VS{33}Et les autres choses consacrées furent, six cents bœufs, et trois brebis moutons.
\VS{34}Mais ils étaient peu de sacrificateurs et ne purent pas dépouiller tous les holocaustes ; les Lévites, leurs frères, les aidèrent jusqu'à ce que cette œuvre fut achevée, et jusqu'à ce que les autres sacrificateurs se fussent sanctifiés ; car les Lévites avaient eu plus à cœur de se sanctifier que les sacrificateurs.
\VS{35}Il y eut aussi un grand nombre d'holocaustes, avec les graisses des offrande de paix et avec les libations des holocaustes. Ainsi, le service de la maison de Yahweh fut rétabli.
\VS{36}Ezéchias et tout le peuple se réjouirent de ce que Dieu avait ainsi disposé le peuple ; car les choses se firent instantanément.
\Chap{30}
\TextTitle{Rétablissement de la Pâque}
\VerseOne{}Puis Ezéchias envoya dire à tout Israël et à Juda ; et il écrivit aussi des lettres à Ephraïm et à Manassé, pour les faire venir à la maison de Yahweh à Jérusalem, pour célébrer la Pâque en l'honneur de Yahweh, le Dieu d'Israël.
\VS{2}Le roi, ses chefs et toute l'assemblée avaient tenu un conseil à Jérusalem afin de célébrer la Pâque au second mois\FTNT{No. 9:10-11.} ;
\VS{3}car on ne pouvait la célébrer au temps ordinaire, parce qu'il n'y avait pas un nombre suffisant de sacrificateurs sanctifiés, et que le peuple n'était pas rassemblé à Jérusalem.
\VS{4}Le roi vit cela d'un bon œil ainsi que toute l'assemblée.
\VS{5}Ils décidèrent de faire une publication dans tout Israël, depuis Beer-Schéba jusqu'à Dan, pour que l'on vienne à Jérusalem célébrer la Pâque à Yahweh, le Dieu d'Israël. Car elle n'était pas célébrée par la multitude depuis longtemps conformément à ce qui était écrit.
\VS{6}Les coureurs allèrent donc avec des lettres de la part du roi et de ses chefs, partout en Israël et en Juda. Selon que le roi l'avait ordonné, ils disaient : Enfants d'Israël, retournez à Yahweh, le Dieu d'Abraham, d'Isaac et d'Israël, et afin qu'il revienne vers vous, qui êtes le reste échappé de la main des rois d'Assyrie.
\VS{7}Ne soyez pas comme vos pères ni comme vos frères, qui ont péché contre Yahweh, le Dieu de leurs pères, c'est pourquoi il les a livrés à la désolation, comme vous le voyez.
\VS{8}Maintenant, ne raidissez pas votre cou comme vos pères. Tendez les mains vers Yahweh, venez à son sanctuaire consacré pour toujours, servez Yahweh, votre Dieu, et son ardente colère se détournera de vous.
\VS{9}Car si vous revenez à Yahweh, vos frères et vos fils trouveront grâce auprès de ceux qui les ont emmenés captifs, et ils reviendront en ce pays parce que Yahweh, votre Dieu, est compatissant et miséricordieux ; et il ne détournera point sa face de vous, si vous revenez à lui.
\VS{10}Les coureurs passaient ainsi de ville en ville, par le pays d'Ephraïm et de Manassé jusqu'à Zabulon ; mais on riait et on se moquait d'eux.
\VS{11}Toutefois, quelques-uns d'Aser, de Manassé et de Zabulon s'humilièrent, et vinrent à Jérusalem.
\VS{12}La main de Dieu fut aussi sur Juda, pour leur donner un même cœur, afin d'exécuter l'ordre du roi et des chefs, selon la parole de Yahweh.
\VS{13}C'est pourquoi il s'assembla un grand peuple à Jérusalem pour célébrer la fête des pains sans levain\FTNT{Ex. 12:15 ; Lé. 23:6.}, au second mois. Ce fut une très grande assemblée.
\VS{14}Ils se levèrent et ôtèrent les autels qui étaient à Jérusalem ; ils ôtèrent aussi tous ceux où l'on brûlait de l'encens, et ils les jetèrent dans le torrent de Cédron.
\VS{15}Puis on immola la Pâque, au quatorzième jour du second mois ; car les sacrificateurs et les Lévites avaient eu honte et s'étaient sanctifiés, et ils amenèrent les holocaustes dans la maison de Yahweh.
\VS{16}Ils se tinrent à leur poste, selon leur charge, d'après la loi de Moïse, homme de Dieu. Et les sacrificateurs répandaient le sang qu'ils recevaient des mains des Lévites.
\VS{17}Car il y en avait un grand nombre dans cette assemblée qui ne s'étaient pas sanctifiés ; c'est pourquoi les Lévites eurent la charge d'immoler la Pâque pour tous ceux qui n'étaient pas purs, afin de les consacrer à Yahweh.
\VS{18}Car une grande partie du peuple, à savoir la plupart de ceux d'Ephraïm, de Manassé, d'Issacar et de Zabulon, ne s'étaient pas purifiés et mangèrent la Pâque contrairement à ce qui est écrit. Mais Ezéchias pria pour eux, en disant : Que Yahweh, qui est bon, tienne la propitiation pour faite,
\VS{19}pour quiconque a disposé son cœur à rechercher Dieu Yahweh, le Dieu de leurs pères, bien qu'il ne soit pas purifié conformément au sanctuaire !
\VS{20}Yahweh exauça Ezéchias, et guérit le peuple.
\VS{21}Les enfants d'Israël qui se trouvèrent à Jérusalem célébrèrent donc la fête des pains sans levain, pendant sept jours, dans une grande réjouissance ; les Lévites et les sacrificateurs célébraient Yahweh chaque jour, avec les instruments qui retentissaient à la louange de Yahweh.
\VS{22}Ezéchias parla au cœur de tous les Lévites, qui prêtaient une grande attention et de l'intelligence au service de Yahweh. Ils mangèrent pendant la fête, sept jours durant, offrant des sacrifices d'offrande de paix, et louant Yahweh, le Dieu de leurs pères.
\TextTitle{Sept jours supplémentaires pour la Pâque}
\VS{23}Puis toute l'assemblée fut d'avis de célébrer sept autres jours. Et ils célébrèrent ces sept jours dans la joie.
\VS{24}Car Ezéchias, roi de Juda, offrit à l'assemblée mille taureaux et sept mille brebis ; et les chefs donnèrent à l'assemblée mille taureaux et dix mille brebis ; et des sacrificateurs en grand nombre s'étaient sanctifiés.
\VS{25}Toute l'assemblée de Juda, avec les sacrificateurs et les Lévites, et toute l'assemblée venue d'Israël, ainsi que les étrangers venus du pays d'Israël, et ceux qui habitaient en Juda, se réjouirent.
\VS{26}Il y eut une grande joie à Jérusalem ; car depuis le temps de Salomon, fils de David, roi d'Israël, il ne s'était pas fait une telle chose dans Jérusalem.
\VS{27}Puis les sacrificateurs et les Lévites se levèrent et bénirent le peuple, et leur voix fut entendue, leur prière parvint jusqu'aux cieux, jusqu'à la sainte demeure de Yahweh.
\Chap{31}
\TextTitle{Destruction des idoles et organisation des services du temple}
\VerseOne{}Lorsque tout cela fut achevé, tous ceux d'Israël qui s'étaient retrouvés là, allèrent dans les villes de Juda, et brisèrent les statues, abattirent les idoles d'Astarté et renversèrent les hauts lieux et les autels, dans tout Juda et Benjamin, dans Ephraïm et Manassé, jusqu'à détruire tout\FTNT{2 R. 18:4.}. Puis tous les enfants d'Israël retournèrent dans leurs villes, chacun dans sa possession.
\VS{2}Et Ezéchias rétablit les classes des sacrificateurs et des Lévites, selon leur partage, chacun suivant sa charge, tant les sacrificateurs que les Lévites, pour les holocaustes et les offrandes de paix, pour faire le service, célébrer et chanter les louanges aux portes du camp de Yahweh.
\VS{3}Le roi donna une portion de ses biens pour les holocaustes, pour les holocaustes du matin et du soir, pour les holocaustes des sabbats, des nouvelles lunes et des fêtes, comme cela est écrit dans la loi de Yahweh.
\VS{4}Il dit au peuple, aux habitants de Jérusalem, de donner la portion des sacrificateurs et des Lévites, afin de s'appliquer à la loi de Yahweh.
\VS{5}Dès que la chose fut publiée, les enfants d'Israël amenèrent en abondance les prémices du blé, du moût, de l'huile, du miel et de tous les produits des champs ; ils apportèrent les dîmes de tout, en abondance.
\VS{6}Les enfants d'Israël et de Juda, qui demeuraient dans les villes de Juda, apportèrent aussi les dîmes du gros et du menu bétail, et les dîmes des choses saintes, qui étaient consacrées à Yahweh, leur Dieu ; et ils les mirent par tas.
\VS{7}Ils commencèrent à former les tas au troisième mois, et ils les achevèrent au septième mois.
\VS{8}Alors Ezéchias et les chefs vinrent voir les tas, et ils bénirent Yahweh et son peuple d'Israël.
\VS{9}Ezéchias interrogea les sacrificateurs et les Lévites au sujet de ces tas.
\VS{10}Le souverain sacrificateur Azaria, de la maison de Tsadok, lui répondit, et parla ainsi : Depuis qu'on a commencé à apporter des offrandes à la maison de Yahweh, nous avons mangé et avons été rassasiés, et il est resté cette grande quantité ; car Yahweh a béni son peuple, et cette grande quantité est le reste.
\VS{11}Alors Ezéchias leur dit de préparer des chambres dans la maison de Yahweh ; et ils les préparèrent.
\VS{12}On y apporta fidèlement les offrandes et les dîmes, les choses consacrées. Conania, le Lévite, en eut l'intendance, et Schimeï, son frère, était son second.
\VS{13}Jehiel, Azazia, Nachath, Asaël, Jerimoth, Jozabad, Eliel, Jismakia, Machath, et Benaja, étaient commis sous l'autorité de Conania et de Schimeï, son frère, d'après l'indication du roi Ezéchias, et d'Azaria, chef de la maison de Dieu.
\VS{14}Koré, le Lévite, fils de Jimna, portier de l'orient, avait la charge des offrandes volontaires offertes à Dieu, pour distribuer l'offrande élevée à Yahweh, et les choses consacrées et saintes.
\VS{15}Il avait sous sa direction Eden, Minjamin, Josué, Schemaeja, Amaria, et Schecania, dans les villes des sacrificateurs, pour distribuer fidèlement les portions à leurs frères, grands et petits, suivant leurs divisions,
\VS{16}à ceux qui étaient enregistrés comme mâles, depuis l'âge de trois ans et au-delà ; à tous ceux qui entraient dans la maison de Yahweh, pour le service quotidien, pour servir dans leurs charges et suivant leurs divisions ;
\VS{17}aux sacrificateurs et aux Lévites enregistrés selon la maison de leurs pères, depuis ceux de vingt ans et au-delà, selon leurs charges et selon leurs divisions ;
\VS{18}à ceux de toute l'assemblée enregistrés avec leurs petits enfants, leurs femmes, leurs fils et leurs filles ; car ils se consacraient avec fidélité aux choses saintes ;
\VS{19}et pour les enfants d'Aaron, les sacrificateurs, qui étaient à la campagne et dans les faubourgs de leurs villes, dans chaque ville, il y avait des gens désignés par leur nom, pour distribuer les portions à tous les mâles des sacrificateurs, et à tous les Lévites enregistrés.
\VS{20}Ezéchias en fit ainsi dans tout Juda ; et il fit ce qui est bon, droit et véritable, devant Yahweh, son Dieu.
\VS{21}Il travailla de tout son cœur et il réussit dans tout l'ouvrage qu'il entreprit pour le service de la maison de Dieu, et pour la loi, et pour les commandements, en recherchant son Dieu.
\Chap{32}
\TextTitle{Menaces de Sanchérib, roi d'Assyrie\FTNTT{2 R. 19:17-37 ; 19:8-13 ; Es. 36:2-20}}
\VerseOne{}Après que ces choses furent bien établies, Sanchérib, roi d'Assyrie, vint et entra en Juda, et campa contre les villes fortes, dans l'intention de faire une brèche\FTNT{Es. 36:2-21 ; 2 R. 18:13-37.}.
\VS{2}Ezéchias, voyant que Sanchérib était venu, et qu'il se tournait vers Jérusalem pour lui faire la guerre,
\VS{3}tint conseil avec ses chefs et ses vaillants hommes pour boucher les sources d'eau qui étaient hors de la ville, et ils l'aidèrent.
\VS{4}Un peuple nombreux s'assembla, et ils bouchèrent toutes les sources et le torrent qui coule par le milieu de la contrée, en disant : Pourquoi les rois d'Assyrie trouveraient-ils à leur venue de l'eau en abondance ?
\VS{5}Il se fortifia et rebâtit toute la muraille où il y avait une brèche, et l'éleva jusqu'aux tours ; il bâtit une autre muraille en dehors ; il répara Millo, dans la cité de David, et il fit faire beaucoup d'armes et de boucliers.
\VS{6}Il donna des chefs de guerre au peuple, les assembla auprès de lui sur la place de la porte de la ville, et parla à leur cœur, en disant :
\VS{7}Fortifiez-vous, soyez forts ! Ne craignez point et ne soyez pas effrayés devant le roi d'Assyrie et toute la multitude qui est avec lui ; car avec nous il y a quelqu'un de plus puissant.
\VS{8}Avec lui est le bras de la chair, mais avec nous est Yahweh, notre Dieu, pour nous aider et pour combattre dans nos combats. Et le peuple s'appuya sur les paroles d'Ezéchias, roi de Juda.
\VS{9}Après cela, Sanchérib, roi d'Assyrie, pendant qu'il était devant Lakis, ayant avec lui toutes les forces de son royaume, envoya ses serviteurs à Jérusalem vers Ezéchias, roi de Juda, et vers tous ceux de Juda qui étaient à Jérusalem, pour leur dire :
\VS{10}Ainsi parle Sanchérib, roi d'Assyrie : Sur qui vous confiez-vous pour que vous restiez à Jérusalem pour y être assiégés ?
\VS{11}Ezéchias ne vous incite-t-il pas pour vous livrer à la mort, par la famine et par la soif, en vous disant : Yahweh, notre Dieu, nous délivrera de la main du roi d'Assyrie ?
\VS{12}Cet Ezéchias n'a-t-il pas ôté les hauts lieux et les autels, et n'a-t-il pas ordonné à Juda et à Jérusalem : Vous vous prosternerez devant un seul autel pour y brûler le parfum ?
\VS{13}Ne savez-vous pas ce que nous avons fait, moi et mes pères, à tous les peuples des autres pays ? Les dieux des nations de ces pays ont-ils pu de quelque manière que ce soit délivrer leur pays de ma main ?
\VS{14}Quel est celui de tous les dieux de ces nations, que mes pères ont entièrement détruites, qui ait pu délivrer son peuple de ma main, pour que votre Dieu puisse vous délivrer de ma main ?
\VS{15}Maintenant donc, qu'Ezéchias ne vous abuse point, et qu'il ne vous incite plus de cette manière, et ne le croyez pas ; car aucun dieu d'aucune nation ni d'aucun royaume n'a pu délivrer son peuple de ma main ni de la main de mes pères ; combien moins votre Dieu vous délivrerait-il de ma main ?
\VS{16}Ses serviteurs parlèrent encore contre Yahweh Dieu, et contre Ezéchias, son serviteur.
\VS{17}Il écrivit aussi une lettre pour blasphémer contre Yahweh, le Dieu d'Israël, en parlant ainsi : Comme les dieux des nations des autres pays n'ont pu délivrer leur peuple de ma main, ainsi le Dieu d'Ezéchias ne pourra délivrer son peuple de ma main\FTNT{2 R. 19:14-37.}.
\VS{18}Et ses serviteurs crièrent à haute voix en langue judaïque, au peuple de Jérusalem qui était sur la muraille, pour les effrayer et les épouvanter, afin de prendre la ville.
\VS{19}Ils parlèrent du Dieu de Jérusalem comme des dieux des peuples de la terre, qui ne sont qu'un ouvrage de mains d'homme.
\TextTitle{Prière d'Ezéchias et exaucement de Yahweh\FTNTT{2 R. 19:14-37 ; Es. 36:21-37:35}}
\VS{20}Alors le roi Ezéchias, et Esaïe, le prophète, fils d'Amots, prièrent à ce sujet et crièrent vers les cieux.
\VS{21}Et Yahweh envoya un ange, dans le camp du roi d'Assyrie, qui extermina tous les vaillants hommes, les princes et les chefs, en sorte qu'il retourna dans son pays, dans la honte. Il entra dans la maison de son dieu ; et là, ceux qui étaient sortis de ses entrailles le firent tomber par l'épée.
\VS{22}C'est ainsi que Yahweh sauva Ezéchias et les habitants de Jérusalem de la main de Sanchérib, roi d'Assyrie, et de la main de tout homme, et il les protégea de toutes parts.
\VS{23}Plusieurs apportèrent des offrandes à Yahweh, à Jérusalem, et des choses précieuses à Ezéchias, roi de Juda, qui après cela fut élevé aux yeux de toutes les nations.
\TextTitle{Maladie et guérison d'Ezéchias\FTNTT{2 R. 20:1-11}}
\VS{24}En ces jours-là, Ezéchias fut malade à en mourir, et il pria Yahweh, qui l'exauça et lui accorda un prodige\FTNT{2 R. 20:1-11}.
\VS{25}Mais Ezéchias ne fut pas reconnaissant du bienfait qu'il avait reçu ; car son cœur s'éleva, et il y eut des maux contre lui, et contre Juda et Jérusalem.
\VS{26}Mais Ezéchias s'humilia de l'élévation de son cœur, lui et les habitants de Jérusalem, et la colère de Yahweh ne vint plus sur eux durant les jours d'Ezéchias.
\TextTitle{Fin du règne d'Ezéchias, sa mort\FTNTT{2 R. 20:12-21 ; cp. Es. 39}}
\VS{27}Ezéchias eut de très grandes richesses et de la gloire, et il se fit des trésors d'argent, d'or, de pierres précieuses, d'aromates, de boucliers, et de toutes sortes d'objets précieux ;
\VS{28}des magasins pour les récoltes de blé, de moût et d'huile, des étables pour toutes sortes de bétail, avec des rangées dans les étables.
\VS{29}Il se fit aussi des villes, et il acquit des troupeaux du gros et du menu bétail en abondance ; car Dieu lui avait donné de très grandes richesses.
\VS{30}Ce fut Ezéchias, qui boucha le canal du haut des eaux de Guihon, et les conduisit directement en bas, vers l'occident de la cité de David. Ainsi Ezéchias réussit dans tout ce qu'il fit.
\VS{31}Toutefois, lorsque les princes de Babylone envoyèrent des messagers vers lui pour s'informer du prodige qui s'était produit dans le pays, Dieu l'abandonna pour le mettre à l'épreuve, afin de connaître tout ce qui était dans son cœur\FTNT{Es. 29.}.
\VS{32}Le reste des actions d'Ezéchias, ses bonnes œuvres, voici, elles sont écrites dans la vision d'Esaïe, le prophète, fils d'Amots, dans le livre des rois de Juda et d'Israël.
\VS{33}Puis Ezéchias s'endormit avec ses pères, et on l'ensevelit au plus haut des sépulcres des fils de David ; et tout Juda, et Jérusalem lui firent honneur à sa mort, et Manassé, son fils régna à sa place.
\Chap{33}
\TextTitle{Manassé, roi impie de Juda\FTNTT{2 R. 21:1-9}}
\VerseOne{}Manassé était âgé de douze ans quand il devint roi, et il régna cinquante-cinq ans à Jérusalem.
\VS{2}Il fit ce qui est mal aux yeux de Yahweh, suivant les abominations des nations que Yahweh avait chassées devant les enfants d'Israël.
\VS{3}Il rebâtit les hauts lieux qu'Ezéchias, son père, avait démolis, il redressa les autels aux Baals, il fit des idoles d'Astarté, et se prosterna devant toute l'armée des cieux et la servit.
\VS{4}Il bâtit aussi des autels dans la maison de Yahweh, de laquelle Yahweh avait parlé ainsi : Mon Nom sera dans Jérusalem à jamais.
\VS{5}Il bâtit des autels à toute l'armée des cieux, dans les deux parvis de la maison de Yahweh.
\VS{6}Il fit passer ses fils par le feu dans la vallée du fils de Hinnom ; il pratiquait la magie, les sorcelleries et la voyance ; il établit des gens qui évoquaient les esprits et des devins. Il s'adonna à faire à l'extrême ce qui est mal aux yeux de Yahweh, pour l'irriter.
\VS{7}Il posa aussi une image taillée, une idole qu'il avait faite, dans la maison de Dieu, de laquelle Dieu avait dit à David, et à Salomon, son fils : Je mettrai à perpétuité mon Nom dans cette maison et dans Jérusalem, que j'ai choisie entre toutes les tribus d'Israël ;
\VS{8}et je ne ferai plus sortir Israël de la terre que j'ai assignée à leurs pères, pourvu seulement qu'ils prennent garde à faire tout ce que je leur ai ordonné, selon toute la loi, les préceptes et les ordonnances prescrites par Moïse.
\VS{9}Manassé donc fit s'égarer Juda et les habitants de Jérusalem, jusqu'à faire pire que les nations que Yahweh avait exterminées de devant les enfants d'Israël.
\TextTitle{Yahweh avertit Manassé\FTNTT{2 R. 21:10-16}}
\VS{10}Yahweh parla à Manassé et à son peuple ; mais ils ne furent pas attentifs.
\TextTitle{Manassé emmené captif se repent\FTNTT{2 R. 21:17-18}}
\VS{11}Alors Yahweh fit venir contre eux les chefs de l'armée du roi d'Assyrie, qui mirent Manassé dans les fers ; ils le lièrent d'une double chaîne d'airain, et l'emmenèrent à Babylone.
\VS{12}Et dès qu'il fut dans l'angoisse, il supplia Yahweh, son Dieu, et il s'humilia profondément devant le Dieu de ses pères.
\VS{13}Il lui adressa ses supplications, et Dieu se laissa fléchir par sa prière, et exauça sa supplication. Il le fit retourner à Jérusalem, dans son royaume. Manassé reconnut alors que c'est Yahweh qui est Dieu.
\VS{14}Après cela, il bâtit une muraille extérieure à la cité de David, vers l'occident de Guihon, dans la vallée, jusqu'à l'entrée de la porte des poissons ; il environna la colline et l'éleva à une grande hauteur ; il établit aussi des chefs d'armée dans toutes les villes fortes de Juda.
\VS{15}Il ôta de la maison de Yahweh l'idole, et les dieux étrangers, et tous les autels qu'il avait bâtis sur la montagne de la maison de Yahweh et à Jérusalem, et les jeta hors de la ville.
\VS{16}Puis il rebâtit l'autel de Yahweh et y offrit des sacrifices d'offrande de paix et de reconnaissance ; et il ordonna à Juda de servir Yahweh, le Dieu d'Israël.
\VS{17}Toutefois, le peuple sacrifiait encore dans les hauts lieux, mais seulement à Yahweh, son Dieu.
\VS{18}Le reste des actions de Manassé, et la prière qu'il fit à son Dieu, et les paroles des voyants qui lui parlaient, au Nom de Yahweh, le Dieu d'Israël, voilà, toutes ces choses sont écrites dans les actes des rois d'Israël.
\VS{19}Sa prière, et comment Dieu se laissa fléchir par sa prière, ses péchés et ses infidélités, les lieux sur lesquels il bâtit des hauts lieux, et dressa des idoles d'Astarté et des images taillées, avant de s'être humilié, voici cela est écrit dans le livre de Hozaï.
\VS{20}Puis Manassé s'endormit avec ses pères, et on l'ensevelit dans sa maison. Et Amon, son fils, régna à sa place.
\TextTitle{Amon règne brièvement sur Juda\FTNTT{2 R. 21:18-26}}
\VS{21}Amon était âgé de vingt-deux ans quand il devint roi, et il régna deux ans à Jérusalem.
\VS{22}Il fit ce qui est mal aux yeux de Yahweh, comme avait fait Manassé, son père. Il sacrifia à toutes les images taillées que Manassé, son père, avait faites, et il les servit.
\VS{23}Mais il ne s'humilia point devant Yahweh, comme s'était humilié Manassé, son père, mais se rendit de plus en plus coupable.
\VS{24}Et ses serviteurs ayant fait une conspiration contre lui, le firent mourir dans sa maison.
\VS{25}Mais le peuple du pays frappa tous ceux qui avaient conspiré contre le roi Amon. Et le peuple du pays établit pour roi, à sa place, Josias, son fils.
\Chap{34}
\TextTitle{Josias règne sur Juda ; ses réformes\FTNTT{2 R. 22:1-2}}
\VerseOne{}Josias était âgé de huit ans quand il devint roi, et il régna trente et un ans à Jérusalem.
\VS{2}Il fit ce qui est droit aux yeux de Yahweh. Il marcha dans les voies de David, son père ; et ne s'en détourna ni à droite ni à gauche.
\VS{3}La huitième année de son règne, lorsqu'il était jeune, il commença à rechercher le Dieu de David, son père ; et à la douzième année, il commença à purifier Juda et Jérusalem des hauts lieux, des idoles d'Astarté, et des images taillées, et des images de fonte.
\VS{4}On démolit dans sa présence les autels des Baals, et il abattit les tentes solaires\FTNT{Tentes solaires : lieux d’idolâtrie} qui étaient par-dessus. Il brisa les idoles d'Astarté, les images taillées et les images de fonte ; et les ayant réduites en poudre, il la répandit sur les sépulcres de ceux qui leur avaient sacrifié.
\VS{5}Puis il brûla les os des prêtres sur leurs autels, et il purifia ainsi Juda et Jérusalem.
\VS{6}Il fit la même chose dans les villes de Manassé, d'Ephraïm et de Siméon, et jusqu'à Nephthali, dans leurs ruines et tout autour.
\VS{7}Il démolit les autels et mit en pièces les idoles d'Astarté et les images taillées, et il les réduisit en poussière ; il abattit toutes les tentes solaires dans tout le pays d'Israël. Puis il revint à Jérusalem.
\TextTitle{Restauration du temple\FTNTT{2 R. 22:3-7}}
\VS{8}La dix-huitième année de son règne, après avoir purifié le pays et le temple, il envoya Schaphan, fils d'Atsalia, et Maaséja, chefs de la ville, et Joach, fils de Joachaz, commis sur les registres, pour réparer la maison de Yahweh, son Dieu.
\VS{9}Ils vinrent vers Hilkija, le souverain sacrificateur ; et on livra l'argent qui avait été apporté dans la maison de Dieu et que les Lévites, gardes du seuil, avaient amassé des mains de Manassé, d'Ephraïm et de tout le reste d'Israël, et aussi de tout Juda et Benjamin ; puis ils s'en retournèrent à Jérusalem.
\VS{10}On le remit entre les mains de ceux qui avaient la charge de l'ouvrage, qui étaient préposés sur la maison de Yahweh. Et ceux qui avaient la charge de l'ouvrage et qui travaillaient dans la maison de Yahweh le distribuèrent pour restaurer et réparer la maison de Yahweh.
\VS{11}Ils le donnèrent aux charpentiers et aux maçons, pour acheter des pierres de taille et du bois pour les poutres et pour la charpente des maisons que les rois de Juda avaient détruites.
\VS{12}Ces hommes s'employaient fidèlement à cet ouvrage. Jachath et Abdias, Lévites d'entre les fils de Merari, étaient préposés sur eux, et Zacharie et Meschullam, d'entre les fils des Kehathites, pour les diriger. Ces Lévites avaient tous de l'intelligence pour les instruments de musique.
\VS{13}Ils surveillaient ceux qui portaient les fardeaux, et dirigeaient tous ceux qui faisaient l'ouvrage, dans quelque service que ce soit ; les scribes, les administrateurs et les portiers, d'entre les Lévites.
\TextTitle{Le livre de la loi redécouvert\FTNTT{2 R. 22:8-10}}
\VS{14}Au moment où l'on sortit l'argent qui avait été apporté dans la maison de Yahweh, Hilkija, le sacrificateur, trouva le livre de la loi de Yahweh, donné par Moïse.
\VS{15}Alors Hilkija, prenant la parole, dit à Schaphan, le secrétaire : J'ai trouvé le livre de la loi dans la maison de Yahweh. Et Hilkija donna le livre à Schaphan.
\VS{16}Schaphan apporta le livre au roi, et rapporta tout au roi, en disant : Les mains de tes serviteurs ont fait tout ce qui leur a été donné à faire.
\VS{17}Ils ont amassé l'argent qui se trouvait dans la maison de Yahweh, et l'ont livré entre les mains des administrateurs, et entre les mains de ceux qui ont la charge de l'ouvrage.
\VS{18}Schaphan, le secrétaire, raconta en disant au roi : Hilkija, le sacrificateur, m'a donné un livre ; et Schaphan le lut devant le roi.
\TextTitle{Lecture du livre de la loi\FTNTT{2 R. 22:11-13}}
\VS{19}Lorsque le roi entendit les paroles de la loi, il déchira ses vêtements.
\VS{20}Il ordonna à Hilkija, à Achikam, fils de Schaphan, à Abdon, fils de Michée, à Schaphan, le secrétaire, et à Asaja, serviteur du roi, en disant :
\VS{21}Allez, consultez Yahweh pour moi et pour ce qui reste en Israël et en Juda, touchant les paroles du livre qui a été trouvé ; car la colère de Yahweh est grande, et elle s'est déversée sur nous, parce que nos pères n'ont point gardé la parole de Yahweh, pour faire selon tout ce qui est écrit dans ce livre.
\TextTitle{Instruction de la prophétesse Hulda\FTNTT{2 R. 22:14-20}}
\VS{22}Hilkija et les gens du roi allèrent vers Hulda, la prophétesse, femme de Schallum, fils de Thokehath, fils de Hasra, garde des vêtements, laquelle demeurait à Jérusalem, dans un autre quartier, et lui en parlèrent.
\VS{23}Alors elle leur répondit : Ainsi parle Yahweh, le Dieu d'Israël : Dites à l'homme qui vous a envoyés vers moi :
\VS{24}Ainsi parle Yahweh : Voici, je vais faire venir le malheur sur ce lieu et sur ses habitants, à savoir toutes les malédictions du serment qui sont écrites dans le livre qu'on a lu devant le roi de Juda.
\VS{25}Parce qu'ils m'ont abandonné, et qu'ils ont fait brûler des parfums aux autres dieux, pour m'irriter par toutes les œuvres de leurs mains, ma colère s'est déversée sur ce lieu, et elle ne sera point éteinte.
\VS{26}Mais quant au roi de Juda, qui vous a envoyés pour consulter Yahweh, vous lui direz : Ainsi parle Yahweh, le Dieu d'Israël, au sujet des paroles que tu as entendues :
\VS{27}Parce que ton cœur a été touché, et que tu t'es humilié devant Dieu, quand tu as entendu ses paroles contre ce lieu et contre ses habitants, et que t'étant humilié devant moi, tu as déchiré tes vêtements et pleuré devant moi, je t'ai aussi entendu, dit Yahweh.
\VS{28}Voici, je vais te recueillir avec tes pères, et tu seras recueilli dans tes sépulcres en paix, et tes yeux ne verront point tout ce mal que je vais faire venir sur ce lieu et sur ses habitants. Et ils rapportèrent cette parole au roi.
\TextTitle{Renouvellement de l'alliance avec Yahweh\FTNTT{2 R. 23:1-3}}
\VS{29}Alors le roi envoya assembler tous les anciens de Juda et de Jérusalem.
\VS{30}Le roi monta à la maison de Yahweh avec tous les hommes de Juda et les habitants de Jérusalem, les sacrificateurs et les Lévites, et tout le peuple, depuis le plus grand jusqu'au plus petit ; et on lut devant eux toutes les paroles du livre de l'alliance, qui avait été trouvé dans la maison de Yahweh.
\VS{31}Et le roi se tint debout à sa place ; et traita devant Yahweh cette alliance qu'ils suivraient Yahweh, et qu'ils garderaient ses commandements, ses témoignages et ses lois, chacun de tout son cœur et de toute son âme, en pratiquant les paroles de l'alliance écrites dans ce livre.
\VS{32}Et il fit tenir debout tous ceux qui se trouvèrent à Jérusalem et en Benjamin ; et les habitants de Jérusalem firent selon l'alliance de Dieu, le Dieu de leurs pères.
\VS{33}Josias ôta de tous les pays qui appartenaient aux enfants d'Israël, toutes les abominations ; et il obligea tous ceux qui se trouvaient en Israël à servir Yahweh, leur Dieu. Pendant toute sa vie, ils ne se détournèrent point de Yahweh, le Dieu de leurs pères.
\Chap{35}
\TextTitle{Josias rétablit la Pâque\FTNTT{2 R. 23:21-27}}
\VerseOne{}Or Josias célébra la Pâque à Yahweh à Jérusalem, et on immola la Pâque le quatorzième jour du premier mois.
\VS{2}Il rétablit les sacrificateurs dans leurs charges, et les encouragea au service de la maison de Yahweh.
\VS{3}Il dit aussi aux Lévites qui enseignaient tout Israël et qui étaient consacrés à Yahweh : Mettez l'arche sainte dans la maison que Salomon, fils de David, roi d'Israël, a bâtie. Qu'elle ne soit plus une charge sur vos épaules. Maintenant, servez Yahweh, votre Dieu, et son peuple d'Israël.
\VS{4}Préparez-vous, selon les maisons de vos pères, selon vos divisions suivant l'écrit de David, roi d'Israël, et suivant l'écrit de Salomon, son fils.
\VS{5}Tenez-vous dans le sanctuaire pour vos frères, les fils du peuple, selon les classes des maisons des pères, et selon que chaque famille des Lévites est partagée.
\VS{6}Immolez la Pâque, sanctifiez-vous, et préparez-la pour vos frères, afin qu'ils puissent la faire selon la parole que Yahweh a donnée par Moïse.
\VS{7}Josias éleva une offrande pour les gens du peuple et pour tous ceux qui se trouvaient là, des troupeaux d'agneaux et de chevreaux, au nombre de trente mille, et trois mille bœufs, le tout pour la Pâque ; cela fut pris sur les biens du roi.
\VS{8}Ses chefs élevèrent une offrande de bon gré pour le peuple, aux sacrificateurs et aux Lévites. Hilkija, Zacharie et Jehiel, princes de la maison de Dieu, donnèrent aux sacrificateurs, pour la Pâque, deux mille six cents agneaux, et trois cents bœufs.
\VS{9}Conania, Schemaeja et Nethaneel, ses frères, et Haschabia, Jeïel et Jozabad, qui étaient les princes des Lévites, élevèrent une offrande de cinq mille agneaux aux Lévites pour faire la Pâque, et cinq cents bœufs.
\VS{10}Le service étant préparé, les sacrificateurs se tinrent à leurs postes, et les Lévites suivant leurs divisions, selon l'ordre du roi.
\VS{11}Puis on immola la Pâque ; et les sacrificateurs répandaient le sang reçu de leurs mains, et les Lévites les dépouillaient.
\VS{12}Ils mirent à part les holocaustes, pour les donner aux gens du peuple, suivant les divisions des maisons de leurs pères, afin de les offrir à Yahweh, selon ce qui est écrit au livre de Moïse ; ils firent de même pour les bœufs.
\VS{13}Ils firent cuire la Pâque au feu, selon l'ordonnance ; et ils firent cuire dans des chaudières, des chaudrons et des poêles, les choses consacrées ; et ils les apportèrent rapidement à tous les gens du peuple.
\VS{14}Puis ils apprêtèrent ce qui était pour eux et pour les sacrificateurs, car les sacrificateurs, fils d'Aaron, furent occupés jusqu'à la nuit à élever en offrande les holocaustes et les graisses ; c'est pourquoi les Lévites apprêtèrent ce qui était pour eux et pour les sacrificateurs, fils d'Aaron.
\VS{15}Les chantres, fils d'Asaph, étaient à leur place, selon l'ordre de David, d'Asaph, d'Héman et de Jeduthun, le voyant du roi. Les portiers étaient à chaque porte, ils n'eurent pas à se détourner de leur service, car leurs frères les Lévites apprêtaient ce qui était pour eux.
\VS{16}Ainsi, tout le service de Yahweh, en ce jour-là, fut réglé pour faire la Pâque et pour élever en offrande les holocaustes sur l'autel de Yahweh, selon l'ordre du roi Josias.
\VS{17}Les fils d'Israël qui s'y trouvèrent célébrèrent donc la Pâque en ce temps-là, et la fête des pains sans levain pendant sept jours.
\VS{18}Or on n'avait point célébré en Israël de Pâque semblable à celle-là depuis les jours de Samuel le prophète ; et aucun des rois d'Israël n'avait célébré une Pâque pareille comme le fit Josias, avec les sacrificateurs et les Lévites, et tout Juda et Israël, qui s'y étaient trouvés avec les habitants de Jérusalem.
\VS{19}Cette Pâque fut célébrée la dix-huitième année du règne de Josias.
\TextTitle{Blessure et mort de Josias\FTNTT{2 R. 23:28-30}}
\VS{20}Après tout cela, quand Josias eut réparé la maison de Yahweh, Néco, roi d'Egypte, monta pour faire la guerre à Carkemisch, sur l'Euphrate. Josias sortit à sa rencontre.
\VS{21}Mais Néco envoya vers lui des messagers pour lui dire : Qu'y a-t-il entre nous, roi de Juda ? Ce n'est pas à toi que j'en veux aujourd'hui, mais à une maison qui me fait la guerre ; et Dieu m'a dit de me hâter. Désiste-toi donc de venir contre Dieu, qui est avec moi, de peur qu'il ne te détruise.
\VS{22}Cependant Josias ne se détourna point de lui, mais se déguisa pour combattre contre lui et il n'écouta pas les paroles de Néco, qui venaient de la bouche de Dieu. Il vint donc pour combattre dans la vallée de Meguiddo.
\VS{23}Les archers tirèrent sur le roi Josias ; et le roi dit à ses serviteurs : Emportez-moi, car je suis très blessé.
\VS{24}Ses serviteurs l'ôtèrent du char, le mirent sur un second char qu'il avait, et le menèrent à Jérusalem. Il mourut et il fut enseveli dans les sépulcres de ses pères, et tous ceux de Juda et de Jérusalem menèrent le deuil de Josias.
\VS{25}Jérémie fit aussi des lamentations sur Josias ; et tous les chanteurs et toutes les chanteuses parlèrent dans leurs complaintes sur Josias jusqu'à ce jour ; et on en a fait une coutume en Israël. Voici, ces choses sont écrites dans les lamentations.
\VS{26}Le reste des actions de Josias, et ses œuvres de piété, selon ce qui est écrit dans la loi de Yahweh,
\VS{27}ses premières et ses dernières actions, sont écrites dans le livre des rois d'Israël et de Juda.
\Chap{36}
\TextTitle{Joachaz règne brièvement sur Juda\FTNTT{2 R. 23:31-33}}
\VerseOne{}Alors le peuple du pays prit Joachaz, fils de Josias, et on l'établit roi à Jérusalem, à la place de son père.
\VS{2}Joachaz était âgé de vingt-trois ans quand il devint roi, et il régna trois mois à Jérusalem.
\VS{3}Le roi d'Egypte le destitua à Jérusalem, et condamna le pays à une amende de cent talents d'argent et d'un talent d'or.
\TextTitle{Règne de Jojakim, déportation à Babylone\FTNTT{2 R. 23:34-24:4-9}}
\VS{4}Le roi d'Egypte établit pour roi sur Juda et Jérusalem Eliakim, frère de Joachaz ; et changea son nom en celui de Jojakim. Puis Néco prit Joachaz, son frère, et l'emmena en Egypte.
\VS{5}Jojakim était âgé de vingt-cinq ans quand il devint roi, et il régna onze ans à Jérusalem. Il fit ce qui est mal aux yeux de Yahweh, son Dieu.
\VS{6}Nebucadnetsar, roi de Babylone, monta contre lui et le lia de doubles chaînes d'airain pour le mener à Babylone.
\VS{7}Nebucadnetsar emporta aussi à Babylone des ustensiles de la maison de Yahweh, et il les mit dans son temple à Babylone.
\VS{8}Le reste des actions de Jojakim, et les abominations qu'il commit, et ce qui fut trouvé en lui, cela est écrit dans le livre des rois d'Israël et de Juda. Et Jojakin, son fils, régna à sa place.
\VS{9}Jojakin était âgé de huit ans quand il devint roi, et il régna trois mois et dix jours à Jérusalem. Il fit ce qui est mal aux yeux de Yahweh.
\TextTitle{Sédécias le dernier roi de Juda, autres déportations à Babylone\FTNTT{2 R.24:10-20 ; cp. 2 R. 25:1-21 ; Jé. 39:8-10}}
\VS{10}Et l'année suivante, le roi Nebucadnetsar envoya, et le fit emmener à Babylone avec les ustensiles précieux de la maison de Yahweh ; et il établit roi sur Juda et Jérusalem, Sédécias, son frère.
\VS{11}Sédécias était âgé de vingt et un ans quand il devint roi, et il régna onze ans à Jérusalem.
\VS{12}Il fit ce qui est mal aux yeux de Yahweh, son Dieu ; et il ne s'humilia point devant Jérémie le prophète, qui lui parlait de la part de Yahweh.
\VS{13}Et même il se rebella contre le roi Nebucadnetsar, qui l'avait fait prêter serment par le Nom de Dieu. Il raidit son cou, et il obstina son cœur pour ne point retourner à Yahweh, le Dieu d'Israël.
\VS{14}Pareillement, tous les chefs des sacrificateurs et le peuple furent infidèles et continuèrent de plus en plus à pécher, selon toutes les abominations des nations ; et ils souillèrent la maison que Yahweh avait sanctifiée dans Jérusalem.
\VS{15}Or Yahweh, le Dieu de leurs pères, les avait sommés par ses messagers qu'il envoya de bonne heure, car il voulait épargner son peuple et sa propre demeure.
\VS{16}Mais ils se moquèrent des messagers de Dieu, ils méprisèrent ses paroles et traitèrent ses prophètes de séducteurs, jusqu'à ce que la fureur de Yahweh montat contre son peuple au point qu'il n'y eut plus de remède.
\VS{17}C'est pourquoi il fit monter contre eux le roi des Chaldéens, qui tua par l'épée leurs jeunes gens dans la maison de leur sanctuaire ; il n'épargna ni le jeune homme, ni la vierge, ni le vieillard, ni l'homme à cheveux blancs ; il les livra tous entre ses mains.
\VS{18}Il fit apporter à Babylone tous les ustensiles de la maison de Dieu, grands et petits, les trésors de la maison de Yahweh, et les trésors du roi et ceux de ses chefs.
\VS{19}Ils brûlèrent la maison de Dieu, ils démolirent les murailles de Jérusalem, ils livrèrent au feu tous ses palais et détruisirent tout ce qu'il y avait comme objets précieux.
\VS{20}Puis le roi de Babylone transporta à Babylone le reste qui échappa à l'épée, et ils furent ses esclaves et ceux de ses fils, jusqu'à la domination du royaume de Perse,
\VS{21}afin que la parole de Yahweh, prononcée par la bouche de Jérémie, fût accomplie ; jusqu'à ce que la terre eût pris plaisir à ses sabbats et durant tous les jours qu'elle demeura dévastée ; elle se reposa pour accomplir les soixante-dix années.
\TextTitle{L'édit de Cyrus autorise les juifs à retourner dans leurs villes}
\VS{22}Mais la première année de Cyrus, roi de Perse, afin que la parole de Yahweh prononcée par Jérémie fût accomplie, Yahweh réveilla l'esprit de Cyrus, roi de Perse, qui fit publier dans tout son royaume, et même par écrit, en disant :
\VS{23}Ainsi parle Cyrus, roi de Perse : Yahweh, le Dieu des cieux, m'a donné tous les royaumes de la terre, et lui-même m'a ordonné de lui bâtir une maison à Jérusalem, qui est en Juda. Qui d'entre vous est de son peuple ? Que Yahweh, son Dieu, soit avec lui, et qu'il monte !
\PPE{}
\end{multicols}

%\addcontentsline{toc}{chapter}{Évangiles}\clearpage
%\clearpage\ShortTitle{Matthieu}\BookTitle{Matthieu}\BFont
\noindent\hrulefill
{\footnotesize
\textit{
\bigskip
{\centering{}
\\Auteur : Matthieu
\\(Gr. : Matthaios)
\\Signifie : Don de Yahweh
\\Thème : Jésus le Roi
\\Date de rédaction : Env. 50 ap. J.-C.\\}
}
%\bigskip
\textit{
\\Matthieu, également connu sous le nom de Lévi, était un juif percepteur d'impôts au service des Romains. Appelé par
Jésus-Christ à Capernaüm et choisi pour être l'un des douze disciples, il offrit un banquet en l'honneur de Jésus dans
sa maison, ce qui lui valut l'hostilité des pharisiens. Rédigé à Antioche de Syrie, son évangile était destiné à des juifs
convertis comme en témoignent ses nombreuses allusions à l'Ancienne Alliance.
%\bigskip
\\Matthieu, démontre l'hégémonie de Jésus, fils de David, fils d'Abraham, roi d'Israël. Il évoque son règne qui se manifestera un jour physiquement, lorsque le Roi jugera tous les hommes du « trône de sa gloire ». Il fait concorder les paroles et les événements de la vie de Jésus avec les prophéties de l'Ancienne Alliance.
%\bigskip
\\Son récit exalte la royauté de Jésus et expose l'Evangile du Royaume.\bigskip
}
}
\par\nobreak\noindent\hrulefill
\begin{multicols}{2}
\Chap{1}
\TextTitle{Présentation du Roi : La généalogie de Jésus-Christ}
\VerseOne{}Livre de la généalogie de Jésus-Christ, fils de David, fils d'Abraham.
\VS{2}Abraham engendra Isaac ; Isaac engendra Jacob ; Jacob engendra Juda et ses frères ;
\VS{3}Juda engendra Pérets et Zara, de Thamar ; Pérets engendra Esrom ; Esrom engendra Aram ;
\VS{4}Aram engendra Aminadab ; Aminadab engendra Naasson ; et Naasson engendra Salmon ;
\VS{5}Salmon engendra Boaz, de Rahab\FTNT{Rahab était une prostituée cananéenne qui est devenue l'ancêtre du Messie (Jos. 6).} ; Boaz engendra Obed, de Ruth\FTNT{Ruth était une Moabite, son peuple était issu de la relation incestueuse de Lot et sa fille aînée (Ge. 19:36-37). Elle est devenue l'ancêtre du Messie (Ru. 4:17).} ; Obed engendra Isaï ;
\VS{6}Isaï engendra le roi David ; le roi David engendra Salomon, de la femme d'Urie ;
\VS{7}Salomon engendra Roboam ; Roboam engendra Abia ; Abia engendra Asa ;
\VS{8}Asa engendra Josaphat ; Josaphat engendra Joram ; Joram engendra Ozias ;
\VS{9}Ozias engendra Joatham ; Joatham engendra Achaz ; Achaz engendra Ezéchias ;
\VS{10}Ezéchias engendra Manassé ; Manassé engendra Amon ; Amon engendra Josias ;
\VS{11}Josias engendra Jéchonias et ses frères, au temps de la déportation à Babylone.
\VS{12}Après la déportation à Babylone, Jéchonias engendra Salathiel ; Salathiel engendra Zorobabel ;
\VS{13}Zorobabel engendra Abiud ; Abiud engendra Eliakim ; Eliakim engendra Azor ;
\VS{14}Azor engendra Sadok ; Sadok engendra Achim ; Achim engendra Eliud ;
\VS{15}Eliud engendra Eléazar ; Eléazar engendra Matthan ; Matthan engendra Jacob ;
\VS{16}Jacob engendra Joseph, l'époux de Marie, de laquelle est né Jésus, qui est appelé Christ\FTNT{Christ : Du grec « christos », ce qui signifie « oint », est l'équivalent grec du mot hébreu « mashiyach », traduit par « messie » en français (Da. 9:25-26). C'est le titre officiel du Seigneur Jésus.}.
\VS{17}Ainsi, il y a en tout quatorze générations depuis Abraham jusqu'à David, quatorze générations depuis David jusqu'à la déportation à Babylone, et quatorze générations depuis la déportation à Babylone jusqu'au Christ.
\TextTitle{Naissance miraculeuse de Jésus-Christ\FTNTT{Lu. 1:26-38 ; 2:1-7 ; Jn. 1:1-2,14}}
\VS{18}Voici de quelle manière arriva la naissance de Jésus-Christ. Marie, sa mère, ayant été fiancée à Joseph, se trouva enceinte par l'opération du Saint-Esprit, avant qu'ils aient habité ensemble.
\VS{19}Joseph, son époux, qui était un homme juste et qui ne voulait pas la diffamer, se proposa de la répudier secrètement.
\VS{20}Mais comme il y pensait, voici, l'Ange du Seigneur lui apparut en songe et lui dit : Joseph, fils de David, ne crains point de prendre avec toi Marie, ta femme, car l'enfant qu'elle a conçu est du Saint-Esprit.
\VS{21}Elle enfantera un fils, et tu lui donneras le nom de Jésus. C'est lui qui sauvera son peuple de ses péchés.
\VS{22}Tout cela arriva afin que s'accomplisse ce que le Seigneur avait annoncé par le prophète :
\VS{23}Voici, la vierge deviendra enceinte, elle enfantera un fils ; et on lui donnera le nom d'Emmanuel\FTNT{Es. 7:14.}, ce qui signifie, Dieu est avec nous.
\VS{24}Joseph s'étant donc réveillé de son sommeil, fit ce que l'Ange du Seigneur lui avait ordonné, et il prit sa femme.
\VS{25}Mais il ne la connut point jusqu'à ce qu'elle ait enfanté son fils premier-né, auquel il donna le nom de Jésus.
\Chap{2}
\TextTitle{Les mages adorent Jésus}
\VerseOne{}Jésus étant né à Bethléhem, ville de Juda, au temps du roi Hérode, voici des mages d'orient arrivèrent à Jérusalem.
\VS{2}En disant : Où est le Roi des Juifs qui vient de naître ? Car nous avons vu son étoile en orient, et nous sommes venus l'adorer.
\VS{3}Le roi Hérode ayant entendu, fut troublé et tout Jérusalem avec lui.
\VS{4}Et ayant assemblé tous les principaux sacrificateurs et les scribes du peuple, il s'informa auprès d'eux où le Christ devait naître.
\VS{5}Et ils lui dirent : A Bethléhem, ville de Judée ; car voici ce qui a été écrit par le prophète :
\VS{6}Et toi, Bethléhem, terre de Juda, tu n'es nullement la plus petite parmi les gouverneurs de Juda, car de toi sortira le Chef qui paîtra mon peuple d'Israël\FTNT{Mi. 5:1.}.
\VS{7}Alors Hérode ayant appelé en secret les mages, s'informa soigneusement auprès d'eux depuis combien de temps brillait l'étoile.
\VS{8}Puis il les envoya à Bethléhem, en leur disant : Allez, et prenez des informations exactes sur le petit enfant ; et quand vous l'aurez trouvé, faites-le-moi savoir, afin que j'aille aussi moi-même l'adorer.
\VS{9}Après avoir entendu le roi, ils partirent. Et voici, l'étoile\FTNT{Le Seigneur Jésus-Christ s'est révélé à Jean comme l'étoile brillante du matin (Ap. 22:16).} qu'ils avaient vue en orient allait devant eux, jusqu'au moment où, arrivée au-dessus du lieu où était le petit enfant, elle s'arrêta.
\VS{10}Quand ils virent l'étoile, ils furent saisis d'une très grande joie.
\VS{11}Ils entrèrent dans la maison, virent le petit enfant avec Marie sa mère, se prosternèrent et l'adorèrent. Ils ouvrirent ensuite leurs trésors et lui offrirent des présents : De l'or, de l'encens et de la myrrhe.
\VS{12}Puis, divinement avertis en songe de ne pas retourner vers Hérode, ils regagnèrent leur pays par un autre chemin.
\TextTitle{Fuite en Egypte}
\VS{13}Lorsqu'ils furent partis, voici, l'Ange du Seigneur apparut dans un songe à Joseph et lui dit : Lève-toi et prends le petit enfant et sa mère, fuis en Egypte, et demeure là, jusqu'à ce que je te le dise ; car Hérode cherchera le petit enfant pour le faire mourir.
\VS{14}Joseph donc étant réveillé, prit de nuit le petit enfant et sa mère, et se retira en Egypte.
\VS{15}Il y resta là jusqu'à la mort d'Hérode ; afin que s'accomplisse ce que le Seigneur avait annoncé par le prophète : J'ai appelé mon Fils hors d'Egypte\FTNT{Os. 11:1}.
\TextTitle{Hérode envoie tuer des enfants innocents}
\VS{16}Alors Hérode voyant que les mages s'étaient moqués de lui, se mit dans une grande colère, et il envoya tuer tous les enfants qui étaient à Bethléhem, et dans tout son territoire ; depuis l'âge de deux ans, et au-dessous, selon la date dont il s'était exactement enquis auprès des mages.
\VS{17}Alors s'accomplit ce qui avait été annoncé par Jérémie le prophète :
\VS{18}On a entendu à Rama des cris, des lamentations, des plaintes, et des grands gémissements : Rachel pleure ses enfants et n'a pas voulu être consolée, parce qu'ils ne sont plus\FTNT{Jé. 31:15}.
\TextTitle{Joseph revient en Israël et s'installe à Nazareth\FTNTT{Lu. 2:39-52}}
\VS{19}Mais après qu'Hérode fut mort, voici, l'Ange du Seigneur apparut dans un songe à Joseph en Egypte,
\VS{20}et lui dit : Lève-toi, et prends le petit enfant et sa mère, et va dans le pays d'Israël ; car ceux qui cherchaient à ôter la vie au petit enfant sont morts.
\VS{21}Joseph donc s'étant réveillé, prit le petit enfant et sa mère, et alla dans le pays d'Israël.
\VS{22}Mais quand il eut appris qu'Archélaüs régnait en Judée, à la place d'Hérode son père, il craignit d'y aller ; et étant divinement averti dans un songe, il se retira dans le territoire de la Galilée,
\VS{23}et vint habiter dans la ville appelée Nazareth ; afin que s'accomplisse ce qui avait été dit par les prophètes : Il sera appelé Nazaréen.
\Chap{3}
\TextTitle{Ministère de Jean-Baptiste\FTNTT{Mc. 1:1-8 ; Lu. 3:1-20 ; Jn. 1:6-8,15-37}}
\VerseOne{}Or, en ce temps-là arriva Jean-Baptiste, prêchant dans le désert de la Judée.
\VS{2}Il disait : Repentez-vous, car le Royaume des cieux est proche.
\VS{3}Car c'est celui dont Esaïe le prophète a parlé, en disant : C'est ici la voix de celui qui crie dans le désert : Préparez le chemin du Seigneur, aplanissez ses sentiers.
\VS{4}Jean avait un vêtement de poils de chameau et une ceinture de cuir autour de ses reins. Et il se nourrissait de sauterelles et de miel sauvage.
\VS{5}Alors les habitants de Jérusalem, et de toute la Judée, et de tout le pays des environs du Jourdain, vinrent à lui ;
\VS{6}et confessant leurs péchés, ils se faisaient baptiser par lui dans le Jourdain.
\VS{7}Mais, voyant venir à son baptême beaucoup de pharisiens et de sadducéens, il leur dit : Race de vipères, qui vous a appris à fuir la colère à venir ?
\VS{8}Produisez donc des fruits convenables à la repentance
\VS{9}et ne prétendez pas dire en vous-mêmes : Nous avons Abraham pour père ! Car je vous dis que Dieu peut faire naître de ces pierres mêmes des enfants à Abraham.
\VS{10}Et déjà la cognée est mise à la racine des arbres ; c'est pourquoi tout arbre qui ne produit pas de bons fruits sera coupé et jeté au feu.
\VS{11}Pour moi, je vous baptise d'eau en signe de repentance ; mais celui qui vient après moi est plus puissant que moi, et je ne suis pas digne de porter ses souliers ; celui-là vous baptisera du Saint-Esprit et de feu\FTNT{Le baptême du Saint-Esprit ne doit pas être confondu avec la plénitude du Saint-Esprit. Le baptême est un acte définitif qui nous greffe au corps du Christ lors de la conversion (1 Co. 12:13). La plénitude consiste quant à elle en un constant renouvellement que nous devons impérativement rechercher (Ep. 5:18). Certains courants chrétiens charismatiques enseignent que le parler en langues est le signe distinctif du baptême du Saint-Esprit. Cette doctrine est basée sur au moins trois passages : Ac. 2:4 ; Ac. 10:44-46 et Ac. 19:1-7. Si cela était vraiment le cas, plusieurs chrétiens seraient encore dans leurs péchés et n'appartiendraient pas au Seigneur Jésus-Christ. En effet, Ro. 8:9 déclare ceci : « Si quelqu'un n'a pas l'Esprit de Christ, il ne lui appartient pas ». Or il est manifeste que bon nombre de chrétiens nés d'en haut ne parlent pas en langues, ce qui est d'ailleurs attesté par l'apôtre Paul (1 Co. 12:30). Il n'y a aucun verset dans les Ecritures qui nous ordonne de chercher le baptême du Saint-Esprit pour la bonne et simple raison que nous le recevons à la conversion.}.
\VS{12}Il a son van à la main, et il nettoiera entièrement son aire, et il assemblera son froment dans le grenier ; mais il brûlera la paille dans un feu qui ne s'éteint point.
\TextTitle{Jean baptise Jésus-Christ\FTNTT{Mc. 1:9-11 ; Lu. 3:21-22 ; Jn. 1:31-34}}
\VS{13}Alors Jésus vint de Galilée au Jourdain vers Jean pour être baptisé par lui.
\VS{14}Mais Jean l'en empêchait avec force en lui disant : J'ai besoin d'être baptisé par toi, et tu viens vers moi ?
\VS{15}Et Jésus répondit en disant : Laisse-moi faire pour le moment, car il nous est ainsi convenable d'accomplir tout ce qui est juste. Et alors il le laissa faire.
\VS{16}Dès que Jésus eut été baptisé, il sortit aussitôt hors de l'eau. Et voici, les cieux lui furent ouverts, et Jean vit l'Esprit de Dieu descendant comme une colombe et venant sur lui.
\VS{17}Et voici une voix du ciel déclara : Celui-ci est mon Fils bien-aimé en qui j'ai mis toute mon affection.
\Chap{4}
\TextTitle{La tentation\FTNTT{Ge. 3:6 ; Mc. 1:12-13 ; Lu. 4:1-13 ; 1 Jn. 2:16.}}
\VerseOne{}Alors Jésus fut emmené par l'Esprit dans le désert, pour être tenté par le diable.
\VS{2}Après avoir jeûné quarante jours et quarante nuits, finalement il eut faim.
\VS{3}Et le tentateur s'étant approché, lui dit : Si tu es le Fils de Dieu, ordonne que ces pierres deviennent des pains.
\VS{4}Mais Jésus répondit et dit : Il est écrit : L'homme ne vivra point de pain seulement, mais de toute parole qui sort de la bouche de Dieu\FTNT{De. 8:3.}.
\VS{5}Alors le diable le transporta dans la sainte ville et le mit sur le haut du temple ;
\VS{6}et il lui dit : Si tu es le Fils de Dieu, jette-toi en bas ; car il est écrit : Il ordonnera à ses anges de te porter sur leurs mains de peur que ton pied ne heurte contre une pierre\FTNT{Ps. 91:12-13.}.
\VS{7}Jésus lui dit : Il est aussi écrit : Tu ne tenteras point le Seigneur ton Dieu\FTNT{De. 6:16.}.
\VS{8}Le diable le transporta encore sur une forte haute montagne, et lui montra tous les royaumes du monde et leur gloire ;
\VS{9}et il lui dit : Je te donnerai toutes ces choses, si tu te prosternes et m'adores.
\VS{10}Mais Jésus lui dit : Retire-toi Satan ! Car il est écrit : Tu adoreras le Seigneur ton Dieu, et tu le serviras lui seul\FTNT{De. 6:13 ; De. 10:20.}.
\VS{11}Alors le diable le laissa. Et voici, des anges s'approchèrent et le servirent.
\TextTitle{Etablissement de Jésus à Capernaüm\FTNTT{Mc. 1:14-15 ; Lu. 4:14-15}}
\VS{12}Jésus, ayant appris que Jean avait été mis en prison, se retira dans la Galilée.
\VS{13}Et ayant quitté Nazareth, il alla demeurer à Capernaüm, ville maritime, sur les confins de Zabulon et de Nephthali ;
\VS{14}afin que s'accomplisse ce qui avait été annoncé par Esaïe, le prophète, en disant :
\VS{15}Le pays de Zabulon et le pays de Nephthali, de la contrée voisine de la mer, au-delà du Jourdain, et la Galilée des Gentils ;
\VS{16}Ce peuple, assis dans les ténèbres, a vu une grande lumière ; et à ceux qui étaient assis dans la région et l'ombre de la mort, la lumière elle-même s'est levée\FTNT{Es. 9:1.}.
\VS{17}Dès lors, Jésus commença à prêcher et à dire : Repentez-vous, car le Royaume des cieux est proche.
\TextTitle{Appel de Pierre, André, Jacques et Jean\FTNTT{Mc. 1:16-20 ; Lu. 5:1-11 ; Jn. 1:35-51}}
\VS{18}Comme Jésus marchait le long de la mer de Galilée, il vit deux frères, Simon, appelé Pierre, et André, son frère, qui jetaient leurs filets dans la mer ; car ils étaient pêcheurs.
\VS{19}Et il leur dit : Suivez-moi et je vous ferai pêcheurs d'hommes.
\VS{20}Et ayant aussitôt quitté leurs filets, ils le suivirent.
\VS{21}Et de là étant allé plus en avant, il vit deux autres frères, Jacques, fils de Zébédée, et Jean, son frère, dans une barque, avec Zébédée leur père, qui réparaient leurs filets, et il les appela.
\VS{22}Et ayant aussitôt quitté leur barque et leur père, ils le suivirent.
\TextTitle{Ministère de Jésus-Christ en Galilée}
\VS{23}Jésus allait par toute la Galilée, enseignant dans leurs synagogues, prêchant l'Evangile du Royaume, et guérissant toutes sortes de maladies, et toutes sortes d'infirmités parmi le peuple.
\VS{24}Et sa renommée se répandit par toute la Syrie ; et on lui présentait tous ceux qui se portaient mal, tourmentés de diverses maladies, des démoniaques, des lunatiques, des paralytiques ; et il les guérissait.
\VS{25}Une grande foule le suivit, de Galilée, de la Décapole, de Jérusalem, de Judée et au-delà du Jourdain.
\Chap{5}
\TextTitle{L'enseignement de Jésus sur la montagne\FTNTT{Lu. 6:20-49 ; Mt. 5:1-48}}
\VerseOne{}Voyant la foule, Jésus monta sur la montagne ; puis s'étant assis, ses disciples s'approchèrent de lui.
\VS{2}Puis, ayant ouvert la bouche, il les enseigna de la sorte :
\VS{3}Heureux les pauvres en esprit, car le Royaume des cieux est à eux.
\VS{4}Heureux ceux qui pleurent, car ils seront consolés.
\VS{5}Heureux les humbles, car ils hériteront la terre.
\VS{6}Heureux ceux qui ont faim et soif de la justice, car ils seront rassasiés.
\VS{7}Heureux les miséricordieux, car ils obtiendront miséricorde.
\VS{8}Heureux sont ceux qui sont purs de cœur, car ils verront Dieu.
\VS{9}Heureux ceux qui procurent la paix, car ils seront appelés enfants de Dieu.
\VS{10}Heureux ceux qui sont persécutés pour la justice, car le Royaume des cieux est à eux.
\VS{11}Heureux serez-vous lorsqu'on vous outragera, qu'on vous persécutera et qu'on dira faussement de vous toute sorte de mal à cause de moi.
\VS{12}Réjouissez-vous et soyez dans l'allégresse, parce que votre récompense sera grande dans les cieux ; car c'est ainsi qu'on a persécuté les prophètes qui ont été avant vous.
\TextTitle{Le sel de la terre et la lumière du monde\FTNTT{Mc. 4:21-23 ; Lu. 8:16-18 ; 11:33-36}}
\VS{13}Vous êtes le sel de la terre mais si le sel perd sa saveur, avec quoi le salera-t-on ? Il ne sert plus qu'à être jeté dehors, et foulé aux pieds par les hommes.
\VS{14}Vous êtes la lumière du monde. Une ville située sur une montagne ne peut être cachée,
\VS{15}et on n'allume point la lampe pour la mettre sous un boisseau, mais sur un chandelier et elle éclaire tous ceux qui sont dans la maison.
\VS{16}Ainsi, que votre lumière luise devant les hommes, afin qu'ils voient vos bonnes œuvres et qu'ils glorifient votre Père qui est dans les cieux.
\TextTitle{Le Messie et la loi}
\VS{17}Ne croyez pas que je sois venu abolir la loi ou les prophètes ; je ne suis pas venu les abolir, mais les accomplir.
\VS{18}Car, je vous le dis en vérité, tant que le ciel et la terre ne passeront point, il ne disparaîtra pas de la loi un seul iota ou un seul trait de lettre jusqu'à ce que tout soit arrivé.
\VS{19}Celui donc qui aura violé l'un de ces petits commandements, et qui aura enseigné les hommes à faire de même, sera appelé le plus petit au Royaume des cieux ; mais celui qui les observera et qui enseignera à les observer, celui-là sera appelé grand au Royaume des cieux.
\VS{20}Car je vous dis que si votre justice ne surpasse celle des scribes et des pharisiens, vous n'entrerez point dans le Royaume des cieux.
\VS{21}Vous avez entendu qu'il a été dit aux anciens : Tu ne tueras point, et celui qui tuera, sera puni par les juges.
\VS{22}Mais moi je vous dis que quiconque se met en colère sans cause contre son frère sera puni par les juges ; et celui qui dira à son frère : Raca\FTNT{Raca : Expression de mépris utilisée parmis les Juifs au temps de Jésus signifiant vide, indigne ou encore vaurien.} ! sera puni par le conseil ; et celui qui lui dira : Insensé ! sera puni par le feu de la géhenne\FTNT{La géhenne ou le lac de feu : Voir commentaire en Ap. 20:14.}.
\VS{23}Si donc tu apportes ton offrande à l'autel, et que là tu te souviennes que ton frère a quelque chose contre toi,
\VS{24}laisse là ton offrande devant l'autel et va te réconcilier d'abord avec ton frère, puis viens et offre ton offrande.
\VS{25}Accorde-toi rapidement avec ta partie adverse, tandis que tu es en chemin avec elle ; de peur que ta partie adverse ne te livre au juge, et que le juge ne te livre à l'officier de justice, et que tu ne sois mis en prison.
\VS{26}En vérité, je te dis, tu ne sortiras point de là, jusqu'à ce que tu n'aies payé le dernier quart de sou.
\TextTitle{Convoitise, adultère et divorce\FTNTT{Mt. 19:3-11 ; Mc. 10:2-12 ; 1 Co. 7:1-16}}
\VS{27}Vous avez entendu qu'il a été dit aux anciens : Tu ne commettras point d'adultère.
\VS{28}Mais moi, je vous dis que quiconque regarde une femme pour la convoiter a déjà commis dans son cœur un adultère avec elle.
\VS{29}Si ton œil droit est pour toi une occasion de chute, arrache-le et jette-le loin de toi ; car il est avantageux pour toi qu'un seul de tes membres périsse et que ton corps entier ne soit pas jeté dans la géhenne.
\VS{30}Si ta main droite est pour toi une occasion de chute, coupe-la et jette-la loin de toi ; car il est avantageux pour toi qu'un seul de tes membres périsse et que ton corps entier ne soit pas jeté dans la géhenne.
\VS{31}Il a été dit encore : Si quelqu'un répudie sa femme, qu'il lui donne une lettre de divorce.
\VS{32}Mais moi, je vous dis que celui qui répudie sa femme, si ce n'est pour cause d'adultère, l'expose à devenir adultère ; et que celui qui épouse une femme répudiée commet un adultère.
\TextTitle{Acquittement des promesses faites au Seigneur ; attitude face à son prochain}
\VS{33}Vous avez aussi appris qu'il a été dit aux anciens : Tu ne te parjureras point, mais tu rendras au Seigneur ce que tu auras promis par serment.
\VS{34}Mais moi, je vous dis de ne jurer aucunement, ni par le ciel, parce que c'est le trône de Dieu ;
\VS{35}ni par la terre, parce que c'est le marchepied de ses pieds, ni par Jérusalem, parce que c'est la ville du grand Roi.
\VS{36}Ne jure pas non plus par ta tête, car tu ne peux pas rendre blanc ou noir un seul cheveu.
\VS{37}Mais que votre parole soit : Oui, oui ; non, non ; car ce qui est de plus, vient du malin.
\VS{38}Vous avez appris qu'il a été dit : Œil pour œil et dent pour dent.
\VS{39}Mais moi, je vous dis : Ne résistez point au méchant. Si quelqu'un te frappe sur ta joue droite, présente-lui aussi l'autre.
\VS{40}Si quelqu'un veut plaider contre toi, et prendre ta tunique, laisse-lui encore ton manteau.
\VS{41}Si quelqu'un te force à faire un mille, fais-en deux avec lui.
\VS{42}Donne à celui qui te demande et ne te détourne point de celui qui veut emprunter de toi.
\TextTitle{Le standard de l'Amour\FTNTT{Lu. 6:27-36}}
\VS{43}Vous avez appris qu'il a été dit : Tu aimeras ton prochain et tu haïras ton ennemi.
\VS{44}Mais moi, je vous dis : Aimez vos ennemis, bénissez ceux qui vous maudissent, faites du bien à ceux qui vous haïssent, et priez pour ceux qui vous maltraitent et vous persécutent,
\VS{45}afin que vous soyez les fils de votre Père qui est dans les cieux ; car il fait lever son soleil sur les méchants et sur les gens de bien, et il envoie sa pluie sur les justes et sur les injustes.
\VS{46}Car si vous aimez seulement ceux qui vous aiment, quelle récompense en aurez-vous ? Les publicains aussi n'en font-ils pas tout autant ?
\VS{47}Et si vous faites accueil seulement à vos frères, que faites-vous de plus que les autres ? Les publicains aussi ne le font-ils pas de même ?
\VS{48}Soyez donc parfaits, comme votre Père qui est dans les cieux est parfait.
\Chap{6}
\TextTitle{Jésus condamne l'hypocrisie}
\VerseOne{}Gardez-vous de pratiquer votre justice devant les hommes, pour en être vus ; autrement, vous ne recevrez point la récompense de votre Père qui est dans les cieux.
\VS{2}Donc, lorsque tu fais ton aumône, ne fais point sonner la trompette devant toi, comme font les hypocrites dans les synagogues et dans les rues, afin d'être glorifiés par les hommes. Je vous le dis en vérité, ils reçoivent leur récompense.
\VS{3}Mais quand tu fais ton aumône, que ta main gauche ne sache pas ce que fait ta droite ;
\VS{4}afin que ton aumône se fasse en secret, et ton Père, qui voit ce qui se fait dans le secret, te récompensera publiquement.
\VS{5}Et quand tu pries, ne sois point comme les hypocrites ; car ils aiment à prier en se tenant debout dans les synagogues et aux coins des rues, pour être vus des hommes. Je vous le dis en vérité, ils reçoivent leur récompense.
\VS{6}Mais toi, quand tu pries, entre dans ta chambre, et ayant fermé ta porte, prie ton Père, qui est là dans ce lieu secret ; et ton Père qui te voit dans ce lieu secret, te récompensera publiquement.
\VS{7}Quand vous priez, ne multipliez pas de vaines paroles, comme font les Gentils ; car ils s'imaginent qu'à force de paroles ils seront exaucés.
\VS{8}Ne leur ressemblez donc point ; car votre Père sait de quoi vous avez besoin, avant que vous le lui demandiez.
\TextTitle{Instructions de Jésus sur la prière}
\VS{9}Voici donc comment vous devez prier : Notre Père qui es aux cieux, que ton Nom soit sanctifié.
\VS{10}Que ton règne vienne. Que ta volonté soit faite sur la terre comme au ciel.
\VS{11}Donne-nous aujourd'hui notre pain quotidien.
\VS{12}Et remets nous nos dettes\FTNT{Du grec « opheilema » : « ce qui est légalement dû, une dette ». Les Ecritures considèrent le péché comme une dette. Voir Mat. 18:21-35.}, comme nous aussi nous remettons les dettes à nos débiteurs.
\VS{13}Ne nous induis pas en tentation ; mais délivre-nous du mal. Car c'est à toi qu'appartiennent, dans tous les siècles, le règne, et la puissance et la gloire. Amen !
\VS{14}Car si vous pardonnez aux hommes leurs offenses, votre Père céleste vous pardonnera aussi les vôtres.
\VS{15}Mais si vous ne pardonnez point aux hommes leurs offenses, votre Père ne vous pardonnera point non plus vos offenses.
\TextTitle{Attitude pendant le jeûne}
\VS{16}Et quand vous jeûnerez, ne prenez pas un air triste, comme font les hypocrites ; car ils se rendent le visage tout défait, afin de montrer aux hommes qu'ils jeûnent. Je vous le dis en vérité, ils reçoivent leur récompense.
\VS{17}Mais toi, quand tu jeûnes, oint ta tête et lave ton visage,
\VS{18}afin qu'il ne paraisse pas aux hommes que tu jeûnes, mais à ton Père qui est présent dans ton lieu secret ; et ton Père qui te voit dans ton lieu secret, te récompensera publiquement.
\TextTitle{Le trésor selon Dieu}
\VS{19}Ne vous amassez point des trésors sur la terre, où les vers et la rouille détruisent, et où les voleurs percent et dérobent.
\VS{20}Mais amassez-vous des trésors dans le ciel, où les vers et la rouille ne détruisent point, et où les voleurs ne percent ni ne dérobent.
\VS{21}Car là où est ton trésor, là aussi sera ton cœur.
\VS{22}L'œil est la lampe du corps. Si donc ton œil est en bon état, tout ton corps sera éclairé.
\VS{23}Mais si ton œil est mal disposé, tout ton corps sera ténébreux. Si donc la lumière qui est en toi n'est que ténèbres, combien seront grandes les ténèbres même ?
\VS{24}Nul ne peut servir deux maîtres. Car, ou il haïra l'un et aimera l'autre ; ou il s'attachera à l'un et méprisera l'autre. Vous ne pouvez pas servir Dieu et Mammon\FTNT{Mammon : Mot d'origine araméenne signifiant « riche ». Certains le rapprochent de l'hébreu « matmon » signifiant « trésor, argent ». D'autres le rapprochent du phénicien « mommon » signifiant « bénéfice ». Dans les évangiles, il signifie « possession » (matérielle), mais il est parfois personnifié.}.
\TextTitle{Rechercher le Royaume}
\VS{25}C'est pourquoi je vous dis : Ne vous inquiétez pas pour votre vie, de ce que vous mangerez, et de ce que vous boirez ; ni pour votre corps, de quoi vous serez vêtus. La vie n'est-elle pas plus que la nourriture et le corps plus que le vêtement ?
\VS{26}Considérez les oiseaux du ciel ; car ils ne sèment, ni ne moissonnent, ni n'assemblent dans des greniers, et cependant votre Père céleste les nourrit. N'êtes vous pas beaucoup plus excellents qu'eux ?
\VS{27}Et qui est celui d'entre vous qui puisse, par ses inquiétudes, ajouter une coudée à sa taille ?
\VS{28}Et pourquoi vous inquiéter au sujet du vêtement ? Apprenez comment croissent les lis des champs : Ils ne travaillent ni ne filent ;
\VS{29}cependant je vous dis que Salomon même, dans toute sa gloire, n'a pas été vêtu comme l'un d'eux.
\VS{30}Si donc Dieu revêt ainsi l'herbe des champs, qui est aujourd'hui sur pied, et qui demain sera jetée au four, ne vous vêtira-t-il pas à plus forte raison, ô gens de petite foi ?
\VS{31}Ne vous inquiétez donc point en disant : Que mangerons-nous ? Ou, que boirons-nous ? Ou de quoi serons-nous vêtus ?
\VS{32}Vu que les païens recherchent toutes ces choses; car votre Père céleste sait que vous avez besoin de toutes ces choses.
\VS{33}Mais cherchez premièrement le Royaume de Dieu et sa justice, et toutes ces choses vous seront données par-dessus.
\VS{34}Ne vous inquiétez donc pas pour le lendemain ; car le lendemain prendra soin de lui-même. A chaque jour suffit sa peine.
\Chap{7}
\TextTitle{Le jugement hypocrite\FTNTT{Lu. 6:37-42}}
\VerseOne{}Ne jugez point afin que vous ne soyez point jugés.
\VS{2}Car de tel jugement que vous jugez, vous serez jugés ; et de telle mesure que vous mesurerez, on vous mesurera réciproquement.
\VS{3}Et pourquoi vois-tu la paille qui est dans l'œil de ton frère, et n'aperçois-tu pas la poutre qui est dans ton œil ?
\VS{4}Ou comment peux-tu dire à ton frère : Permets que j'ôte de ton œil cette paille et n'aperçois-tu pas la poutre dans ton œil ?
\VS{5}Hypocrite, ôte premièrement de ton œil la poutre, et après cela tu verras comment tu ôteras la paille de l'œil de ton frère.
\VS{6}Ne donnez point les choses saintes aux chiens et ne jetez point vos perles devant les pourceaux, de peur qu'ils ne les foulent aux pieds, ne se retournent et ne vous déchirent.
\TextTitle{Exhortation à la prière}
\VS{7}Demandez et il vous sera donné. Cherchez et vous trouverez. Frappez et l'on vous ouvrira.
\VS{8}Car quiconque demande, reçoit ; et celui qui cherche trouve ; et l'on ouvre à celui qui frappe.
\VS{9}Lequel de vous donnera une pierre à son fils, s'il lui demande du pain ?
\VS{10}Ou, s'il lui demande un poisson, lui donnera-t-il un serpent ?
\VS{11}Si donc vous, méchants comme vous l'êtes, savez donner à vos enfants de bonnes choses, à combien plus forte raison votre Père qui est dans les cieux, donnera-t-il des bonnes choses à ceux qui les lui demandent ?
\TextTitle{La règle d'or de la loi et des prophètes\FTNTT{Lu. 6:31; Ep.4:32}}
\VS{12}Tout ce que vous voulez que les hommes fassent pour vous, faites-le de même pour eux, car c'est la loi et les prophètes.
\TextTitle{Les deux chemins\FTNTT{Ps. 1}}
\VS{13}Entrez par la porte étroite, car c'est la porte large et le chemin spacieux qui mènent à la perdition, et il y en a beaucoup qui entrent par elle.
\VS{14}Mais étroite est la porte, resserré le chemin qui mènent à la vie, et il y en a peu qui les trouvent.
\TextTitle{Les faux prophètes, reconnaissables à leurs fruits\FTNTT{Lu. 6:43-45}}
\VS{15}Gardez-vous des faux prophètes, ils viennent à vous en habits de brebis, mais au-dedans ce sont des loups ravisseurs.
\VS{16}Vous les reconnaîtrez à leurs fruits. Cueille-t-on des raisins sur des épines, ou des figues sur des chardons ?
\VS{17}Ainsi tout bon arbre porte de bons fruits ; mais le mauvais arbre porte de mauvais fruits.
\VS{18}Un bon arbre ne peut porter de mauvais fruits ni le mauvais arbre porter de bons fruits.
\VS{19}Tout arbre qui ne porte pas de bons fruits est coupé et jeté au feu.
\VS{20}Vous les reconnaîtrez donc à leurs fruits.
\TextTitle{Fausse confession\FTNTT{Lu. 6:46}}
\VS{21}Ceux qui me disent : Seigneur ! Seigneur ! N'entreront pas tous dans le Royaume des cieux ; mais celui qui fait la volonté de mon Père qui est dans les cieux.
\VS{22}Plusieurs me diront en ce jour-là : Seigneur ! Seigneur ! N'avons-nous pas prophétisé en ton Nom ? N'avons-nous pas chassé les démons en ton Nom ? N'avons-nous pas fait beaucoup de miracles en ton Nom ?
\VS{23}Alors je leur dirai ouvertement : Je ne vous ai jamais connus. Retirez-vous de moi, vous qui commettez l'iniquité.
\TextTitle{Parabole des deux bâtisseurs et des deux fondements\FTNTT{Lu. 6:47-49}}
\VS{24}Quiconque entend ces paroles que je dis, et les met en pratique, je le comparerai à un homme prudent qui a bâti sa maison sur le roc.
\VS{25}La pluie est tombée, les torrents sont venus, les vents ont soufflé contre cette maison : Elle n'est point tombée parce qu'elle était fondée sur le roc\FTNT{Jésus-Christ le Rocher : voir Es. 8:13-17.}.
\VS{26}Mais quiconque entend ces paroles que je dis et ne les met point en pratique, sera semblable à un homme insensé qui a bâti sa maison sur le sable.
\VS{27}La pluie est tombée, les torrents sont venus, les vents ont soufflé contre cette maison : Elle est tombée et sa ruine a été grande.
\TextTitle{Effet de l'enseignement}
\VS{28}Or il arriva que quand Jésus eut achevé ce discours, la foule fut frappée de sa doctrine ;
\VS{29}car il les enseignait comme ayant de l'autorité et non comme les scribes.
\Chap{8}
\TextTitle{Le lépreux guérit\FTNTT{Mc. 1:40-45}}
\VerseOne{}Et quand il fut descendu de la montagne, de grandes foules le suivirent.
\VS{2}Et voici, un lépreux vint et se prosterna devant lui, en lui disant : Seigneur\FTNT{Seigneur : Du grec « kurios ». C'est la première fois que ce terme est appliqué à Jésus. Notez que c'est un lépreux qui a eu la révélation que Jésus-Christ est YHWH.}, si tu veux, tu peux me rendre pur.
\VS{3}Et Jésus étendit la main, le toucha, en disant : Je le veux, sois pur. A l'instant même il fut purifié de sa lèpre.
\VS{4}Puis Jésus lui dit : Prends garde de ne le dire à personne ; mais va te montrer au sacrificateur et offre l'offrande que Moïse a prescrite afin que cela leur serve de témoignage.
\TextTitle{Guérison du serviteur d'un centenier\FTNTT{Lu. 7:1-10}}
\VS{5}Et quand Jésus fut entré dans Capernaüm, un centenier vint à lui, le priant
\VS{6}et disant : Seigneur, mon serviteur qui est paralytique est couché à la maison et il souffre extrêmement.
\VS{7}Jésus lui dit : J'irai et je le guérirai.
\VS{8}Mais le centenier lui répondit : Seigneur, je ne suis pas digne que tu entres sous mon toit ; mais dis seulement une parole et mon serviteur sera guéri.
\VS{9}Car moi-même qui suis un homme soumis à l'autorité d'un autre, j'ai des soldats sous mes ordres, et je dis à l'un : Va ! et il va ; et à un autre : Viens ! et il vient ; et à mon serviteur : Fais cela ! et il le fait.
\VS{10}Après l'avoir entendu, Jésus fut étonné et dit à ceux qui le suivaient : Je vous le dis en vérité, même en Israël je n'ai pas trouvé une aussi grande foi.
\VS{11}Or, je vous dis que plusieurs viendront de l'orient et de l'occident, et seront à table dans le Royaume des cieux, avec Abraham, Isaac et Jacob.
\VS{12}Et les enfants du Royaume seront jetés dans les ténèbres du dehors, où il y aura des pleurs et des grincements de dents.
\VS{13}Alors Jésus dit au centenier : Va, et qu'il te soit fait selon ta foi. Et à l'heure même, son serviteur fut guéri.
\TextTitle{Guérison de la belle-mère de Pierre\FTNTT{Mc. 1:29-34 ; Lu. 4:38-41}}
\VS{14}Puis Jésus alla à la maison de Pierre, dont il vit la belle-mère couchée et ayant la fièvre.
\VS{15}Il toucha sa main, et la fièvre la quitta ; puis elle se leva, et les servit.
\VS{16}Et le soir étant venu, on lui amena plusieurs démoniaques. Et il chassa par sa parole les esprits malins, et guérit tous ceux qui étaient malades,
\VS{17}afin que s'accomplisse ce qui avait été annoncé par Esaïe le prophète, en disant : Il a pris nos faiblesses et a porté nos maladies\FTNT{Es. 53:4.}.
\TextTitle{Les disciples éprouvés dans leur consécration\FTNTT{Lu. 9:57-62}}
\VS{18}Or Jésus voyant autour de lui de grandes foules, donna l'ordre de passer à l'autre rive.
\VS{19}Et un scribe s'approchant, lui dit : Maître, je te suivrai partout où tu iras.
\VS{20}Jésus lui dit : Les renards ont des tanières, et les oiseaux du ciel ont des nids ; mais le Fils de l'homme n'a pas de place pour reposer sa tête.
\VS{21}Puis un autre de ses disciples lui dit : Seigneur, permets-moi d'aller d'abord ensevelir mon père.
\VS{22}Et Jésus lui dit : Suis-moi et laisse les morts ensevelir leurs morts.
\TextTitle{Autorité de Jésus face à la tempête\FTNTT{Mc. 4:35-41 ; Lu. 8:22-25}}
\VS{23}Il monta dans la barque et ses disciples le suivirent.
\VS{24}Et voici, il s'éleva sur la mer une si grande tempête que la barque était couverte de flots ; et Jésus dormait.
\VS{25}Et ses disciples vinrent le réveiller en lui disant : Seigneur, sauve-nous, nous périssons !
\VS{26}Et il leur dit : Pourquoi avez-vous peur, gens de peu de foi ? Alors s'étant levé, il menaça les vents et la mer, et il se fit un grand calme.
\VS{27}Et les gens qui étaient là furent étonnés, et dirent : Qui est celui-ci à qui obéissent même les vents et la mer ?
\TextTitle{Deux aveugles et un démoniaque guéris\FTNTT{Mc. 5:1-20 ; Lu. 8:26-40}}
\VS{28}Et quand il fut passé de l'autre côté, dans le pays des Gadaréniens, deux démoniaques sortant des sépulcres, vinrent le rencontrer. Ils étaient si dangereux que personne ne pouvait passer par ce chemin-là.
\VS{29}Et voici, ils s'écrièrent : Qu'y a-t-il entre nous et toi, Jésus Fils de Dieu ? Es-tu venu ici nous tourmenter avant le temps ?
\VS{30}Et il y avait loin d'eux un grand troupeau de pourceaux qui paissaient.
\VS{31}Et les démons le priaient en disant : Si tu nous chasses dehors, permets-nous d'entrer dans ce troupeau de pourceaux.
\VS{32}Et il leur dit : Allez ! Et ils sortirent et entrèrent dans le troupeau de pourceaux. Et voici, tout le troupeau de pourceaux se précipita des pentes escarpées dans la mer et ils périrent dans les eaux.
\VS{33}Ceux qui les gardaient s'enfuirent et allèrent dans la ville, ils racontèrent toutes ces choses et ce qui était arrivé aux démoniaques.
\VS{34}Et voici, toute la ville alla à la rencontre de Jésus, et l'ayant vu, ils le prièrent de se retirer de leur pays.
\Chap{9}
\TextTitle{Un paralytique guéri\FTNTT{Mc. 2:3-12 ; Lu. 5:18-26}}
\VerseOne{}Alors, étant monté dans une barque, il traversa la mer et vint dans sa ville.
\VS{2}Et voici, on lui présenta un paralytique couché sur un lit. Et Jésus voyant leur foi, dit au paralytique : Prends courage, mon enfant ! Tes péchés te sont pardonnés.
\VS{3}Et voici, quelques-uns des scribes disaient au dedans d'eux : Cet homme blasphème.
\VS{4}Mais Jésus, connaissant leurs pensées, leur dit : Pourquoi avez-vous de mauvaises pensées dans vos cœurs ?
\VS{5}Car lequel est le plus aisé de dire : Tes péchés te sont pardonnés ; ou de dire : Lève-toi et marche ?
\VS{6}Or afin que vous sachiez que le Fils de l'homme a le pouvoir sur la terre de pardonner les péchés : Lève-toi, dit-il au paralytique, prends ton lit et va dans ta maison.
\VS{7}Et il se leva et s'en alla dans sa maison.
\VS{8}Quand la foule vit cela, elle fut saisie d'étonnement, et elle glorifiait Dieu qui a donné aux hommes un tel pouvoir.
\TextTitle{L'appel de Matthieu\FTNTT{Mc. 2:14 ; Lu. 5:27-28}}
\VS{9}De là, étant allé plus loin, Jésus vit un homme nommé Matthieu, assis au bureau du péage et il lui dit : Suis-moi ; et il se leva et le suivit.
\TextTitle{L'appel des pécheurs\FTNTT{Mc. 2:15-20 ; Lu. 5:29-35}}
\VS{10}Comme Jésus était à table dans la maison de Matthieu, beaucoup de publicains et des gens de mauvaise vie, qui étaient venus là, se mirent à table avec Jésus et avec ses disciples.
\VS{11}Les pharisiens virent cela et ils dirent à ses disciples : Pourquoi votre Maître mange-t-il avec les publicains et les gens de mauvaise vie ?
\VS{12}Jésus l'ayant entendu, leur dit : Ce ne sont pas ceux qui sont en bonne santé qui ont besoin de médecin, mais les malades.
\VS{13}Mais allez et apprenez ce que veulent dire ces paroles : Je prends plaisir à la miséricorde et non aux sacrifices\FTNT{Os. 6:6.}. Car je ne suis pas venu appeler à la repentance les justes, mais les pécheurs.
\VS{14}Alors les disciples de Jean vinrent auprès de lui et lui dirent : Pourquoi nous et les pharisiens jeûnons-nous souvent, tandis que tes disciples ne jeûnent point ?
\VS{15}Et Jésus leur répondit : Les amis de l'époux peuvent-ils s'affliger pendant que l'époux est avec eux ? Mais les jours viendront où l'époux leur sera enlevé, alors ils jeûneront.
\TextTitle{Parabole du drap neuf et des outres neuves\FTNTT{Mc. 2:21-22 ; Lu. 5:36-39}}
\VS{16}Aussi personne ne met une pièce de drap neuf à un vieil habit car la pièce emporterait une partie de l'habit et la déchirure serait pire.
\VS{17}On ne met pas non plus du vin nouveau dans de vieilles outres ; autrement les outres se rompent, et le vin se répand, et les outres sont perdues ; mais on met le vin nouveau dans des outres neuves, et l'un et l'autre se conservent.
\TextTitle{Résurrection de la fille de Jaïrus et guérison de la femme à la perte de sang\FTNTT{Mc. 5:21-43 ; Lu. 8:41-56}}
\VS{18}Tandis qu'il leur disait ces choses, voici, arriva un chef qui se prosterna devant lui, en lui disant : Ma fille est morte il y a un instant, mais viens, et impose-lui ta main et elle vivra.
\VS{19}Et Jésus s'étant levé le suivit avec ses disciples.
\VS{20}Et voici, une femme atteinte d'une perte de sang depuis douze ans s'approcha par-derrière et toucha le bord de son vêtement.
\VS{21}Car elle disait en elle-même : Si je puis seulement toucher son vêtement, je serai guérie.
\VS{22}Et Jésus se retourna, et dit en la voyant : Prends courage, ma fille ! Ta foi t'a sauvée. Et cette femme fut guérie à l'heure même.
\VS{23}Lorsque Jésus fut arrivé à la maison du chef et qu'il vit les joueurs de flûte et une foule bruyante,
\VS{24}il leur dit : Retirez-vous car la jeune fille n'est pas morte, mais elle dort ; et ils se moquaient de lui.
\VS{25}Quand la foule eut été renvoyée, il entra, prit la main de la jeune fille et elle se leva.
\VS{26}Et le bruit s'en répandit dans toute la contrée.
\TextTitle{Deux aveugles et un démoniaque guéris}
\VS{27}Etant parti de là, Jésus fut suivi par deux aveugles qui criaient : Fils de David, aie pitié de nous !
\VS{28}Et quand il fut arrivé dans la maison, les aveugles s'approchèrent de lui et Jésus leur dit : Croyez-vous que je puisse faire ce que vous me demandez ? Ils lui répondirent : Oui, Seigneur !
\VS{29}Alors il toucha leurs yeux en disant : Qu'il vous soit fait selon votre foi.
\VS{30}Et leurs yeux s'ouvrirent. Alors Jésus leur dit sévèrement : Prenez garde que personne ne le sache.
\VS{31}Mais, dès qu'ils furent sortis, ils répandirent sa renommée dans tout le pays.
\VS{32}Comme ils s'en allaient, voici, on présenta à Jésus un homme muet et démoniaque.
\VS{33}Et le démon ayant été chassé, le muet parla ; et les foules étonnées disaient : Jamais pareille chose ne s'est vue en Israël.
\VS{34}Mais les pharisiens disaient : Il chasse les démons par le prince des démons.
\VS{35}Jésus allait dans toutes les villes et les villages, enseignant dans leurs synagogues, et prêchant l'Evangile du Royaume, et guérissant toutes sortes de maladies et toutes sortes d'infirmités parmi le peuple.
\TextTitle{Jésus ému de compassion pour la foule\FTNTT{Mc. 6:34}}
\VS{36}Et voyant les foules, il fut ému de compassion, parce qu'elles étaient dispersées et errantes comme des brebis qui n'ont point de pasteur.
\VS{37}Et il dit à ses disciples : La moisson est grande, mais il y a peu d'ouvriers.
\VS{38}Priez donc le Maître de la moisson d'envoyer des ouvriers dans sa moisson.
\Chap{10}
\TextTitle{Appel et mission des douze apôtres\FTNTT{Mc. 6:7-13 ; Lu. 9:1-6}}
\VerseOne{}Alors Jésus ayant appelé ses douze disciples, leur donna le pouvoir de chasser les esprits impurs et de guérir toutes sortes de maladies et toutes sortes d'infirmités.
\VS{2}Et voici les noms des douze apôtres : Le premier est Simon, nommé Pierre, et André son frère ; Jacques, fils de Zébédée, et Jean, son frère ;
\VS{3}Philippe et Barthélemy ; Thomas, et Matthieu le péager ; Jacques, fils d'Alphée, et Lebbée, surnommé Thaddée.
\VS{4}Simon le Cananite, et Judas Iscariot, celui qui le livra.
\VS{5}Tels sont les douze que Jésus envoya, et leur donna ses ordres en disant : N'allez point vers les Gentils et n'entrez point dans aucune ville des Samaritains ;
\VS{6}Mais allez plutôt vers les brebis perdues de la maison d'Israël.
\VS{7}Et quand vous serez partis, prêchez, en disant : Le Royaume des cieux est proche.
\VS{8}Guérissez les malades, rendez purs les lépreux, ressuscitez les morts, chassez les démons hors des possédés. Vous l'avez reçu gratuitement, donnez-le gratuitement\FTNT{Vous avez reçu gratuitement : Aucun chrétien, quel que soit son appel ou son don ne peut prétendre qu'il a payé pour avoir les talents qu'il a reçus du Seigneur. Dans 1 Co. 4:7 Paul nous pose une question : « Qu'as-tu que tu n'aies reçu et si tu l'as reçu pourquoi te glorifies-tu ? » Dieu interroge également Job : « De qui suis-je le débiteur ? » (Job 41:2). Vendre quelque chose qu'on a reçu gratuitement n'est rien d'autre que du vol. Donnez gratuitement : C'est la suite logique des choses, on reçoit gratuitement et on donne gratuitement. Si nous aimons Dieu, nous devons garder sa Parole et marcher comme lui a marché (Jn. 14:15 ; 1 Jn. 2:6). Il a donné ses enseignements et nourri les gens gratuitement. Dans Ap. 21:6 et 22:17, le Seigneur invite toutes les personnes qui ont soif à venir s'abreuver gratuitement. Alors pourquoi vendre la Parole qu'on a reçue gratuitement ? Le Seigneur a envoyé les douze en mission et leur a demandé d'apporter l'évangile du Royaume, de guérir les malades et de délivrer les possédés gratuitement (Ac. 8:18-24 ; Ac. 20:33-35 ; Ap. 21:6 ; Ap. 22:17).}.
\VS{9}Ne prenez ni or, ni argent, ni monnaie dans vos ceintures ;
\VS{10}ni de sac pour le voyage, ni deux tuniques, ni souliers, ni bâton ; car l'ouvrier mérite sa nourriture.
\VS{11}Et dans quelque ville ou village que vous entriez, informez-vous qui y est digne de vous loger ; et demeurez chez lui jusqu'à ce que vous partiez de là.
\VS{12}Et quand vous entrerez dans quelque maison, saluez-la.
\VS{13}Et si cette maison en est digne, que votre paix vienne sur elle ; mais si elle n'en est pas digne, que votre paix retourne à vous.
\VS{14}Mais lorsque quelqu'un ne vous recevra point et n'écoutera point vos paroles, secouez, en partant de cette maison ou de cette ville, la poussière de vos pieds.
\VS{15}Je vous dis en vérité que ceux du pays de Sodome et de Gomorrhe seront traités moins rigoureusement au jour du jugement que cette ville-là.
\TextTitle{La proclamation du Royaume avant le retour du Messie}
\VS{16}Voici, je vous envoie comme des brebis au milieu des loups ; soyez donc prudents comme des serpents et simples comme des colombes.
\VS{17}Et mettez-vous en garde contre les hommes ; car ils vous livreront aux tribunaux et vous battront de verges dans leurs synagogues.
\VS{18}Et vous serez menés devant des gouverneurs et même devant des rois, à cause de moi, pour rendre témoignage de moi devant eux et aux nations.
\VS{19}Mais, quand ils vous livreront, ne vous inquiétez pas de ce que vous aurez à dire, ni comment vous parlerez. Ce que vous aurez à dire vous sera donné à l'heure même.
\VS{20}Car ce n'est pas vous qui parlez, mais c'est l'Esprit de votre Père qui parlera en vous.
\VS{21}Le frère livrera son frère à la mort, et le père son enfant ; et les enfants s'élèveront contre leurs pères et leurs mères, et les feront mourir.
\VS{22}Et vous serez haïs de tous à cause de mon Nom ; mais celui qui persévérera jusqu'à la fin sera sauvé.
\VS{23}Quand ils vous persécuteront dans une ville, fuyez dans une autre. Je vous le dis en vérité, vous n'aurez pas achevé de parcourir toutes les villes d'Israël, que le Fils de l'homme sera venu.
\TextTitle{La consécration du disciple et sa récompense}
\VS{24}Le disciple n'est point au-dessus du maître, ni le serviteur au-dessus de son seigneur.
\VS{25}Il suffit au disciple d'être traité comme son maître, et au serviteur comme son seigneur. S'ils ont appelé le père de famille Béelzébul, à combien plus forte raison appelleront-ils ainsi ses domestiques ?
\VS{26}Ne les craignez donc point. Car il n'y a rien de caché qui ne doive être découvert, ni rien de secret qui ne doive être connu.
\VS{27}Ce que je vous dis dans les ténèbres, dites-le dans la lumière ; et ce que je vous dis à l'oreille, prêchez-le sur les toits.
\VS{28}Et ne craignez point ceux qui tuent le corps et qui ne peuvent tuer l'âme ; mais craignez plutôt celui qui peut faire périr et l'âme et le corps en les jetant dans la géhenne.
\VS{29}Ne vend-on pas deux passereaux pour un sou ? Cependant, il n'en tombe pas un à terre sans la volonté de votre Père.
\VS{30}Et même les cheveux de votre tête sont tous comptés.
\VS{31}Ne craignez donc point : Vous valez plus que beaucoup de passereaux.
\VS{32}Quiconque donc me confessera devant les hommes, je le confesserai aussi devant mon Père qui est aux cieux.
\VS{33}Mais quiconque me reniera devant les hommes, je le renierai aussi devant mon Père qui est dans les cieux.
\VS{34}Ne croyez pas que je sois venu apporter la paix sur la terre. Je ne suis pas venu apporter la paix, mais l'épée.
\VS{35}Car je suis venu mettre en division le fils contre son père, et la fille contre sa mère, et la belle-fille contre sa belle-mère.
\VS{36}Et les propres domestiques d'un homme seront ses ennemis.
\VS{37}Celui qui aime son père ou sa mère plus que moi, n'est pas digne de moi ; et celui qui aime son fils ou sa fille plus que moi, n'est pas digne de moi.
\VS{38}Et quiconque ne prend pas sa croix et ne vient pas après moi, n'est pas digne de moi.
\VS{39}Celui qui aura conservé sa vie la perdra ; mais celui qui aura perdu sa vie pour l'amour de moi la retrouvera.
\VS{40}Celui qui vous reçoit me reçoit, et celui qui me reçoit, reçoit celui qui m'a envoyé.
\VS{41}Celui qui reçoit un prophète en qualité de prophète, recevra la récompense d'un prophète ; et celui qui reçoit un juste en qualité de juste recevra la récompense d'un juste.
\VS{42}Et quiconque aura donné à boire seulement un verre d'eau froide à l'un de ces petits parce qu'il est mon disciple, je vous le dis en vérité qu'il ne perdra point sa récompense.
\Chap{11}
\TextTitle{Jean-Baptiste le plus grand des hommes\FTNTT{Lu. 7:19-35}}
\VerseOne{}Et il arriva que quand Jésus eut achevé de donner ses ordres à ses douze disciples, il partit de là pour aller enseigner et prêcher dans leurs villes.
\VS{2}Jean, ayant entendu parler dans sa prison des œuvres du Christ, envoya deux de ses disciples pour lui dire :
\VS{3}Es-tu celui qui devait venir, ou devons-nous en attendre un autre ?
\VS{4}Et Jésus leur répondit : Allez, et rapportez à Jean les choses que vous entendez et que vous voyez.
\VS{5}Les aveugles recouvrent la vue, les boiteux marchent, les lépreux sont purifiés, les sourds entendent, les morts sont ressuscités, et l'Evangile est annoncé aux pauvres\FTNT{Jésus-Christ est le Dieu véritable dont la venue était annoncée par Esaïe (Es. 35:4-6).}.
\VS{6}Mais, heureux est celui qui n'aura point été scandalisé en moi ;
celui pour qui je ne serai pas une occasion de chute !
\VS{7}Et comme ils s'en allaient, Jésus se mit à dire à la foule au sujet de Jean : Mais qu'êtes-vous allés voir dans le désert ? Un roseau agité par le vent ?
\VS{8}Mais qu'êtes-vous allés voir ? Un homme vêtu de précieux vêtements ? Voici, ceux qui portent des habits précieux sont dans les maisons des rois.
\VS{9}Mais qu'êtes-vous allés voir ? Un prophète ? Oui, vous dis-je, et plus qu'un prophète.
\VS{10}Car c'est celui dont il est écrit : Voici, j'envoie mon messager\FTNT{Mal. 3:1.} devant ta face, pour préparer ton chemin devant toi.
\VS{11}En vérité, je vous le dis, parmi ceux qui sont nés de femmes, il n'en a point paru de plus grand que Jean-Baptiste. Toutefois, le plus petit dans le Royaume des cieux, est plus grand que lui.
\VS{12}Or depuis le temps de Jean-Baptiste jusqu'à maintenant, le Royaume des cieux est forcé et ce sont les violents qui s'en emparent.
\VS{13}Car tous les prophètes et la loi ont prophétisé jusqu'à Jean.
\VS{14}Et si vous voulez recevoir mes paroles, c'est lui qui est l'Elie\FTNT{Mal. 4:5-6.} qui devait venir.
\VS{15}Que celui qui a des oreilles pour entendre, entende.
\VS{16}Mais à qui comparerai-je cette génération ? Elle est semblable aux petits-enfants qui sont assis sur les places publiques, et qui crient à leurs compagnons
\VS{17}et leur disent : Nous vous avons joué de la flûte et vous n'avez point dansé ; nous vous avons chanté des complaintes et vous ne vous êtes point lamentés.
\VS{18}Car Jean est venu ne mangeant ni ne buvant et ils disent : Il a un démon.
\VS{19}Le Fils de l'homme est venu mangeant et buvant et ils disent : C'est un mangeur et un buveur, un ami des publicains et des gens de mauvaise vie. Mais la sagesse a été justifiée par ses enfants.
\TextTitle{Jésus dénonce les indifférents}
\VS{20}Alors il se mit à faire des reproches aux villes où il avait fait beaucoup de miracles, parce qu'elles ne s'étaient point repenties.
\VS{21}Malheur à toi, Chorazin ! Malheur à toi, Bethsaïda ! Car si les miracles qui ont été faits au milieu de vous, avaient été faits dans Tyr et dans Sidon, il y a longtemps qu'elles se seraient repenties, en prenant le sac et la cendre.
\VS{22}C'est pourquoi je vous dis que Tyr et Sidon seront traitées moins rigoureusement que vous, au jour du jugement.
\VS{23}Et toi Capernaüm, qui as été élevée jusqu'au ciel, tu seras précipitée jusqu'à Hadès\FTNT{Voir commentaire Mt. 16:18} ; car si les miracles qui ont été faits au milieu de toi, avaient été faits dans Sodome, elle subsisterait encore.
\VS{24}C'est pourquoi je vous dis que ceux de Sodome seront traités moins rigoureusement que toi, au jour du jugement.
\TextTitle{La relation personnelle du disciple avec son Seigneur}
\VS{25}En ce temps-là, Jésus prenant la parole dit : Je te loue, ô mon Père ! Seigneur du ciel et de la terre, de ce que tu as caché ces choses aux sages et aux intelligents, et que tu les as révélées aux petits enfants.
\VS{26}Oui, Père, je te loue parce que telle a été ta bonne volonté.
\VS{27}Toutes choses m'ont été données par mon Père ! Et personne ne connaît le Fils si ce n'est le Père ; et personne ne connaît le Père si ce n'est le Fils, et celui à qui le Fils veut le révéler.
\VS{28}Venez à moi vous tous qui êtes fatigués et chargés, et je vous donnerai du repos.
\VS{29}Prenez mon joug sur vous et recevez mes instructions, car je suis doux et humble de cœur ; et vous trouverez le repos pour vos âmes.
\VS{30}Car mon joug est doux et mon fardeau est léger.
\Chap{12}
\TextTitle{Jésus, le Maître du sabbat\FTNTT{Mc. 2:23-28 ; Lu. 6:1-5}}
\VerseOne{}En ce temps-là, Jésus traversa des champs de blé un jour de sabbat. Et ses disciples qui avaient faim se mirent à arracher des épis et à les manger.
\VS{2}Les pharisiens voyant cela, lui dirent : Voici, tes disciples font ce qu'il n'est pas permis de faire le jour du sabbat.
\VS{3}Mais il leur dit : N'avez-vous pas lu ce que fit David quand il eut faim, lui et ceux qui étaient avec lui ?
\VS{4}Comment il entra dans la maison de Dieu, et mangea les pains de proposition, qu'il ne lui était pas permis de manger ni à lui, ni à ceux qui étaient avec lui, mais aux sacrificateurs seulement ?
\VS{5}Ou n'avez-vous pas lu dans la loi, qu'aux jours du sabbat, les sacrificateurs violent le sabbat dans le temple, sans se rendre coupables ?
\VS{6}Or, je vous le dis, qu'il y a ici quelqu'un de plus grand que le temple.
\VS{7}Si vous saviez ce que signifient ces paroles : Je veux la miséricorde, et non pas le sacrifice, vous n'auriez pas condamné ceux qui ne sont pas coupables\FTNT{1 S. 15:22 ; Os. 6:6.}.
\VS{8}Car le Fils de l'homme est Maître même du sabbat.
\TextTitle{Jésus guérit l'homme à la main sèche le jour du sabbat\FTNTT{Mc. 3:1-5 ; Lu. 6:6-11}}
\VS{9}Puis étant parti de là, il entra dans leur synagogue.
\VS{10}Et voici, il s'y trouvait un homme qui avait la main sèche. Et pour avoir sujet de l'accuser, ils l'interrogèrent en disant : Est-il permis de guérir les jours du sabbat ?
\VS{11}Et il répondit : Lequel d'entre vous s'il n'a qu'une brebis, et qu'elle vienne à tomber dans une fosse le jour du sabbat, ne la saisira-t-il pas pour l'en retirer ?
\VS{12}Combien un homme ne vaut-il pas plus qu'une brebis ! Il est donc permis de faire du bien les jours du sabbat.
\VS{13}Alors il dit à cet homme : Etends ta main. Il l'étendit et elle devint saine comme l'autre.
\TextTitle{Jésus accomplit de nombreuses guérisons}
\VS{14}Les pharisiens sortirent et ils se consultèrent sur les moyens de le faire périr.
\VS{15}Mais Jésus, l'ayant su, partit de là, et de grandes foules le suivirent. Il les guérit tous.
\VS{16}Et il leur défendit avec menaces de le faire connaître,
\VS{17}afin que s'accomplît ce qui avait été annoncé par Esaïe le prophète, en disant :
\VS{18}Voici mon serviteur que j'ai élu, mon bien-aimé, qui est l'objet de mon amour, je mettrai mon Esprit en lui et il annoncera le jugement aux nations.
\VS{19}Il ne contestera point, il ne criera point et personne n'entendra sa voix dans les rues.
\VS{20}Il ne brisera point le roseau cassé et n'éteindra point le lumignon qui fume, jusqu'à ce qu'il ait fait triompher la justice.
\VS{21}Et les nations espéreront en son nom\FTNT{Es. 42:1-4.}.
\VS{22}Alors on lui amena un homme tourmenté d'un démon, aveugle et muet, et il le guérit ; de sorte que celui qui avait été aveugle et muet, parlait et voyait.
\VS{23}Et toutes les foules en furent étonnées, et elles disaient : Celui-ci n'est-il pas le Fils de David ?
\TextTitle{Le blasphème contre le Saint-Esprit\FTNTT{Mc. 3:22-30 ; Lu. 11:15-23}}
\VS{24}Mais les pharisiens ayant entendu cela, disaient : Celui-ci ne chasse les démons que par Béelzébul, prince des démons.
\VS{25}Mais Jésus connaissant leurs pensées, leur dit : Tout royaume divisé contre lui-même sera réduit en désert ; et toute ville, ou maison, divisée contre elle-même ne subsistera point.
\VS{26}Or si Satan chasse Satan, il est divisé contre lui-même ; comment donc son royaume subsistera-t-il ?
\VS{27}Et si je chasse les démons par Béelzébul, par qui vos fils les chassent-ils ? C'est pourquoi ils seront eux-mêmes vos juges.
\VS{28}Mais si je chasse les démons par l'Esprit de Dieu, certes le Royaume de Dieu est donc venu jusqu'à vous.
\VS{29}Ou, comment quelqu'un peut-il entrer dans la maison d'un homme fort et piller ses biens, sans avoir auparavant lié cet homme fort ? Alors il pillera sa maison.
\VS{30}Celui qui n'est pas avec moi, est contre moi, et celui qui n'assemble pas avec moi disperse.
\VS{31}C'est pourquoi je vous dis, que tout péché et tout blasphème sera pardonné aux hommes ; mais le blasphème contre l'Esprit ne leur sera point pardonné.
\VS{32}Quiconque parlera contre le Fils de l'homme, il lui sera pardonné ; mais quiconque parlera contre le Saint-Esprit, il ne lui sera pardonné ni dans ce siècle, ni dans le siècle à venir\FTNT{Le blasphème contre le Saint-Esprit : Le blasphème est un outrage, une calomnie à l'encontre de Dieu. En attribuant l'œuvre de Dieu à Satan, les pharisiens ont commis l'impardonnable. Beaucoup de personnes craignent d'avoir commis ce péché par inadvertance, en ayant par exemple un doute sur l'origine d'un miracle. La Parole nous recommande de ne pas ajouter foi à tout esprit, mais d'éprouver les esprits pour savoir s'ils sont de Dieu (1 Jn. 4:1). On ne pèche donc pas lorsqu'on exerce son discernement. De plus, si l'on a commis une erreur de jugement par ignorance, le Seigneur ne nous en tiendra pas rigueur (Ac. 17:30). Le blasphème contre le Saint-Esprit est commis par des personnes qui, bien qu'ayant la connaissance et la capacité de différencier le bien du mal, font preuve de mauvaise foi. Ainsi, les pharisiens avaient constaté les bons fruits portés par Jésus, mais ils ont hypocritement qualifié de mal le bien qu'il faisait (Es. 5:20). Ceux qui blasphèment contre le Saint-Esprit sont loin d'être ignorants. Comme nous l'atteste Hé. 6:4-6, parmi ces gens, certains « ont goûté le don céleste », « ont eu part au Saint-Esprit, et ont goûté la bonne parole de Dieu, et les puissances du siècle à venir ». En choisissant sciemment de pécher, alors qu'ils ont expérimenté la grâce de Dieu, ils retournent à ce qu'ils ont vomi et outragent ainsi le Seigneur (2 Pi. 2:18-22). Leur cœur endurci à l'extrême rejette volontairement la vérité pour s'attacher au mensonge. Constatant leur refus définitif de se repentir, le Saint-Esprit finit par se retirer pour laisser la place à l'esprit d'égarement qui les maintiendra dans l'erreur (2 Th. 2:9-12). Enfin, il est à noter qu'en Ap. 14:9-11, ceux qui ont reçu la marque de la bête sont condamnés d'office. Il ne faut nullement conclure que le Seigneur leur a refusé son pardon, mais plutôt que les personnes ayant reçu cette marque ont aussi blasphémé contre le Saint-Esprit. Voir commentaire en Ap. 13:16.}.
\TextTitle{Toute parole proclamée appelle un jugement}
\VS{33}Ou dites que l'arbre est bon et son fruit est bon ; ou dites que l'arbre est mauvais et son fruit est mauvais ; car on connaît l'arbre par le fruit.
\VS{34}Race de vipères, comment pourriez-vous dire de bonnes choses, méchants comme vous l'êtes ? Car c'est de l'abondance du cœur que la bouche parle.
\VS{35}L'homme de bien\FTNT{Le mot « bien » dans ce passage vient du grec « Agathos » qui signifie : « de bonne constitution ou nature », « utile », « salutaire », « bon », « agréable », plaisant », « joyeux », « heureux », « excellent », « distingué », « droit », « honorable ».} tire de bonnes choses du bon trésor de son cœur ; et l'homme méchant tire de mauvaises choses du mauvais trésor de son cœur.
\VS{36}Je vous le dis : Les hommes rendront compte au jour du jugement, de toute parole vaine qu'ils auront proférée.
\VS{37}Car tu seras justifié par tes paroles et tu seras condamné par tes paroles.
\TextTitle{Le miracle du prophète Jonas\FTNTT{Jon. 2:1 ; Lu. 11:29-32.}}
\VS{38}Alors quelques-uns des scribes et des pharisiens lui dirent : Maître, nous voudrions bien te voir faire quelque miracle.
\VS{39}Mais il leur répondit et dit : Une génération méchante et adultère demande un miracle, mais il ne lui sera point donné d'autre miracle que celui de Jonas le prophète.
\VS{40}Car, de même que Jonas fut trois jours et trois nuits dans le ventre d'un grand poisson, de même le Fils de l'homme sera trois jours et trois nuits dans le sein de la terre.
\VS{41}Les Ninivites se lèveront au jour du jugement contre cette génération et la condamneront, parce qu'ils se repentirent à la prédication de Jonas ; et voici, il y a ici plus que Jonas.
\TextTitle{La condamnation de cette génération par la reine de Séba\FTNTT{2 Ch. 9:1-12}}
\VS{42}La reine du Midi se lèvera au jour du jugement contre cette nation et la condamnera, parce qu'elle vint des extrémités de la terre pour entendre la sagesse de Salomon ; et voici, il y a ici plus que Salomon.
\TextTitle{Le retour de l'esprit impur\FTNTT{Lu. 11:24-26}}
\VS{43}Lorsque l'esprit impur est sorti d'un homme, il va par des lieux arides, cherchant du repos, mais il n'en trouve point.
\VS{44}Alors il dit : Je retournerai dans ma maison, d'où je suis sorti ; et quand il arrive, il la trouve vide, balayée et ornée.
\VS{45}Puis il s'en va et prend avec lui sept autres esprits plus méchants que lui ; qui y étant entrés, habitent là ; et ainsi la dernière condition de cet homme est pire que la première. Il en sera de même pour cette génération perverse.
\TextTitle{La famille spirituelle\FTNTT{Mc. 3:31-35 ; Lu. 8:19-21}}
\VS{46}Et comme il parlait encore aux foules, voici, sa mère et ses frères se tenaient dehors, cherchant à lui parler.
\VS{47}Et quelqu'un lui dit : Voici, ta mère et tes frères sont là dehors, qui cherchent à te parler.
\VS{48}Mais il répondit à celui qui lui avait dit cela : Qui est ma mère et qui sont mes frères ?
\VS{49}Et étendant sa main sur ses disciples, il dit : Voici ma mère et mes frères.
\VS{50}Car, quiconque fera la volonté de mon Père qui est dans les cieux, celui-là est mon frère, et ma sœur, et ma mère.
\Chap{13}
\TextTitle{1. Parabole des quatre terrains\FTNTT{Mc. 4:1-20 ; Lu. 8:4-15}}
\VerseOne{}Ce même jour, Jésus sortit de la maison et s'assit au bord de la mer.
\VS{2}Une grande foule s'assembla auprès de lui, c'est pourquoi il monta dans une barque et il s'assit. Aussi, toute la foule se tenait sur le rivage.
\VS{3}Et il leur parla en paraboles sur beaucoup de choses et il dit : Un semeur sortit pour semer.
\VS{4}Et comme il semait, une partie de la semence tomba le long du chemin, et les oiseaux vinrent, et la mangèrent toute.
\VS{5}Et une autre partie tomba dans les endroits pierreux où elle n'avait pas beaucoup de terre : Elle leva aussitôt parce qu'elle n'entrait pas profondément dans la terre ;
\VS{6}mais, quand le soleil parut, elle fut brûlée et sécha parce qu'elle n'avait point de racines.
\VS{7}Une autre partie tomba parmi les épines ; et les épines montèrent et l'étouffèrent.
\VS{8}Une autre partie tomba dans la bonne terre : Et elle donna du fruit, un grain en donna cent, un autre, soixante, et un autre, trente.
\VS{9}Que celui qui a des oreilles pour entendre, qu'il entende.
\TextTitle{Explication aux disciples}
\VS{10}Alors les disciples s'approchèrent et lui dirent : Pourquoi leur parles-tu en paraboles ?
\VS{11}Il leur répondit et dit : Parce qu'il vous a été donné de connaître les mystères du Royaume des cieux, et que cela ne leur a pas été donné de les connaître.
\VS{12}Car on donnera à celui qui a, et il sera dans l'abondance, mais à celui qui n'a pas, on ôtera même ce qu'il a.
\VS{13}C'est pourquoi je leur parle en paraboles, parce qu'en voyant, ils ne voient point, et qu'en entendant, ils n'entendent point et ne comprennent point.
\VS{14}Et ainsi s'accomplit pour eux la prophétie d'Esaïe qui dit : Vous entendrez de vos oreilles et vous ne comprendrez point ; et vous regarderez des yeux, et vous ne verrez point.
\VS{15}Car le cœur de ce peuple est engraissé, et ils ont endurci leurs oreilles, et ils ont fermé leurs yeux de peur qu'ils ne voient de leurs yeux, qu'ils n'entendent de leurs oreilles, qu'ils ne comprennent de leur cœur, qu'ils ne se convertissent et que je ne les guérisse\FTNT{Es. 6:9-10.}.
\VS{16}Mais heureux sont vos yeux, car ils voient ; et vos oreilles, parce qu'elles entendent.
\VS{17}Je vous le dis en vérité, beaucoup de prophètes et de justes ont désiré voir les choses que vous voyez, et ils ne les ont point vues, entendre les choses que vous entendez, et ils ne les ont point entendues.
\VS{18}Vous donc, écoutez la signification de la parabole du semeur.
\VS{19}Lorsqu'un homme écoute la parole du Royaume et ne la comprend pas, le malin vient et ravit ce qui est semé dans son cœur ; cet homme est celui qui a reçu la semence le long du chemin.
\VS{20}Et celui qui a reçu la semence dans les endroits pierreux, c'est celui qui entend la parole et la reçoit aussitôt avec joie ;
\VS{21}mais il n'a point de racine en lui-même, il croit pour un temps, et dès que survient une tribulation ou une persécution à cause de la parole, il y trouve une occasion de chute.
\VS{22}Et celui qui a reçu la semence parmi les épines, c'est celui qui entend la parole de Dieu, mais en qui les soucis du siècle et la séduction des richesses étouffent la parole et la rendent infructueuse.
\VS{23}Mais celui qui a reçu la semence dans la bonne terre, c'est celui qui entend la parole et la comprend. Il porte du fruit, et un grain donne cent, un autre soixante, et un autre trente.
\TextTitle{2. Parabole du blé et de l'ivraie}
\VS{24}Il leur proposa une autre parabole et il dit\FTNT{La parabole du blé et de l'ivraie. En méditant cette parabole, nous remarquons que lorsque le blé eut poussé et donné du fruit, l'ivraie parut aussi. Il est vrai que lorsqu'il y a un réveil spirituel divin dans une assemblée ou dans un pays, l'ennemi suscite aussi un faux réveil avec des faux ouvriers et des fausses manifestations spirituelles. Voilà pourquoi l'ivraie côtoiera le blé jusqu'à la fin du monde. Le mot « ivraie » se dit « ebriacus » en latin, ce qui donne « ébriété » en français. Nous comprenons donc que l'un des rôles de l'ivraie est d'enivrer le blé (les enfants de Dieu). Dans les Ecritures, l'ivresse est synonyme de la débauche spirituelle ou physique. En grec l'ivraie se dit « zizanion » qui donne en français « zizanie ». Voir Mt. 12:25. La division est l'œuvre de l'ivraie dans les églises qui cherche à créer des sectes et des partis pris.} : Le Royaume des cieux est semblable à un homme qui a semé de la bonne semence dans son champ.
\VS{25}Mais, pendant que les hommes dormaient, son ennemi vint, sema de l'ivraie parmi le blé, puis s'en alla.
\VS{26}Lorsque l'herbe eut poussé et donné du fruit, l'ivraie parut aussi.
\VS{27}Et les serviteurs du maître de la maison vinrent à lui et lui dirent : Seigneur, n'as-tu pas semé de la bonne semence dans ton champ ? D'où vient donc qu'il y a de l'ivraie ?
\VS{28}Mais il leur répondit : C'est un ennemi qui a fait cela. Et les serviteurs lui dirent : Veux-tu donc que nous allions l'arracher ?
\VS{29}Et il leur dit : Non, de peur qu'en arrachant l'ivraie, vous ne déraciniez le blé en même temps.
\VS{30}Laissez-les croître tous deux ensemble, jusqu'à la moisson ; et au temps de la moisson, je dirai aux moissonneurs : Arrachez premièrement l'ivraie, et liez-la en gerbes pour la brûler, mais amassez le blé dans mon grenier.
\TextTitle{3. Parabole du grain de sénevé\FTNTT{Mc. 4:30-32 ; Lu. 13:18-19}}
\VS{31}Il leur proposa une autre parabole et il dit : Le Royaume des cieux est semblable au grain de sénevé qu'un homme a pris et semé dans son champ.
\VS{32}C'est la plus petite de toutes les semences ; mais, quand il a poussé, il est plus grand que les autres plantes et devient un arbre, de sorte que les oiseaux du ciel viennent habiter et font leurs nids dans ses branches.
\TextTitle{4. Parabole du levain\FTNTT{Lu. 13:20-21}}
\VS{33}Il leur dit une autre parabole : Le Royaume des cieux est semblable à du levain qu'une femme a pris et mis dans trois mesures de farine, jusqu'à ce que toute la pâte soit levée.
\VS{34}Jésus dit à la foule toutes ces choses en paraboles, et il ne lui parlait point sans paraboles,
\VS{35}afin que s'accomplisse ce qui avait été annoncé par le prophète : J'ouvrirai ma bouche en paraboles, je déclarerai les choses qui ont été cachées dès la fondation du monde\FTNT{Ps. 78:2.}.
\TextTitle{Explication de la parabole du blé et de l'ivraie}
\VS{36}Alors Jésus renvoya la foule et entra dans la maison, et ses disciples s'approchèrent de lui et lui dirent : Explique-nous la parabole de l'ivraie du champ.
\VS{37}Et il leur répondit et dit : Celui qui sème la bonne semence, c'est le Fils de l'homme ;
\VS{38}le champ, c'est le monde ; la bonne semence ce sont les fils du Royaume, et l'ivraie ce sont les fils du malin ;
\VS{39}et l'ennemi qui l'a semée, c'est le diable ; la moisson, c'est la fin du monde, et les moissonneurs sont les anges.
\VS{40}Or, comme on arrache l'ivraie et qu'on la brûle au feu, il en sera de même à la fin de ce monde.
\VS{41}Le Fils de l'homme enverra ses anges qui arracheront de son Royaume tous les scandales et ceux qui commettent l'iniquité,
\VS{42}et les jetteront dans la fournaise ardente, où il y aura des pleurs et des grincements de dents.
\VS{43}Alors les justes resplendiront comme le soleil dans le Royaume de leur Père. Que celui qui a des oreilles pour entendre, qu'il entende.
\TextTitle{5. Parabole du trésor caché}
\VS{44}Le Royaume des cieux est encore semblable à un trésor caché dans un champ. L'homme qui l'a trouvé, le cache ; puis dans sa joie, il va vendre tout ce qu'il a, et achète ce champ.
\TextTitle{6. Parabole de la perle}
\VS{45}Le Royaume des cieux est encore semblable à un marchand qui cherche de bonnes perles.
\VS{46}Il a trouvé une perle de grand prix et il est allé vendre tout ce qu'il avait, et l'a achetée.
\TextTitle{7. Parabole du filet}
\VS{47}Le Royaume des cieux est encore semblable à un filet jeté dans la mer et ramassant toutes sortes de choses.
\VS{48}Quand il est rempli, les pêcheurs le tirent en haut sur le rivage, puis s'étant assis, ils mettent ce qu'il y a de bon à part dans leurs vases et jettent dehors ce qui est mauvais.
\VS{49}Il en sera de même à la fin du monde, les anges viendront séparer les méchants d'avec les justes,
\VS{50}et les jetteront dans la fournaise ardente, où il y aura des pleurs et des grincements de dents.
\VS{51}Jésus leur dit : Avez-vous compris toutes ces choses ? Ils lui répondirent : Oui, Seigneur.
\TextTitle{8. Le maître de la maison}
\VS{52}Et il leur dit : C'est pourquoi, tout scribe instruit de ce qui regarde le Royaume des cieux, est semblable à un père de famille qui tire de son trésor des choses nouvelles et des choses anciennes.
\TextTitle{Jésus à Nazareth\FTNTT{Mc. 6:1-6}}
\VS{53}Et quand Jésus eut achevé ces paraboles, il partit de là.
\VS{54}Et s'étant rendu dans sa patrie, il enseignait dans la synagogue, de telle sorte que ceux qui l'entendirent étaient étonnés et disaient : D'où lui viennent cette sagesse et ces miracles ?
\VS{55}Celui-ci n'est-il pas le fils du charpentier ? Sa mère ne s'appelle-t-elle pas Marie ? Et ses frères ne s'appellent-ils pas Jacques, Joseph, Simon et Jude ?
\VS{56}Et ses sœurs ne sont-elles pas toutes parmi nous ? D'où lui viennent donc toutes ces choses ?
\VS{57}Et il était pour eux une occasion de chute. Mais Jésus leur dit : Un prophète n'est méprisé que dans sa patrie et dans sa maison.
\VS{58}Et il ne fit là que peu de miracles, à cause de leur incrédulité.
\Chap{14}
\TextTitle{Mort de Jean-Baptiste\FTNTT{Mc. 6:14-29; Lu. 9:7-9}}
\VerseOne{}En ce temps-là, Hérode le tétrarque entendit parler de la renommée de Jésus, et il dit à ses serviteurs : C'est Jean-Baptiste !
\VS{2}Il est ressuscité des morts, c'est pourquoi la puissance de faire des miracles agit puissamment en lui.
\VS{3}Car Hérode avait fait arrêter Jean, et l'avait fait lier et mettre en prison, à cause d'Hérodias, femme de Philippe son frère.
\VS{4}Parce que Jean lui disait : Il ne t'est pas permis de l'avoir pour femme.
\VS{5}Et il voulait le faire mourir, mais il craignait la foule, parce qu'elle regardait Jean comme un prophète.
\VS{6}Or, le jour où l'on célébra la naissance d'Hérode, la fille d'Hérodias dansa au milieu de l'assemblée et plut à Hérode.
\VS{7}C'est pourquoi il lui promit avec serment de lui donner tout ce qu'elle demanderait.
\VS{8}A l'instigation de sa mère, elle dit : Donne-moi ici, sur un plat, la tête de Jean-Baptiste.
\VS{9}Le roi fut attristé ; mais à cause de ses serments et de ceux qui étaient à table avec lui, il commanda qu'on la lui donne.
\VS{10}Et il envoya décapiter Jean dans la prison.
\VS{11}Et sa tête fut apportée sur un plat et donnée à la fille qui la présenta à sa mère.
\VS{12}Puis ses disciples vinrent, et emportèrent son corps, et l'ensevelirent. Et ils allèrent l'annoncer à Jésus.
\VS{13}Et Jésus, ayant appris ce qu'Hérode avait fait, partit de là dans une barque, pour se retirer à l'écart dans un lieu désert ; et la foule l'ayant appris, sortit des villes voisines et le suivit à pied.
\VS{14}Et Jésus étant sorti, vit une grande foule, et il fut ému de compassion pour elle, et guérit les malades.
\TextTitle{Multiplication des pains pour les cinq mille hommes\FTNTT{Mc. 6:32-44 ; Lu. 9:12-17 ; Jn. 6:1-14}}
\VS{15}Et comme il se faisait tard, ses disciples vinrent à lui et lui dirent : Ce lieu est désert et l'heure est déjà avancée. Renvoie la foule, afin qu'elle aille dans les villages, pour s'acheter des vivres.
\VS{16}Mais Jésus leur dit : Ils n'ont pas besoin de s'en aller ; donnez-leur vous-mêmes à manger.
\VS{17}Et ils lui dirent : Nous n'avons ici que cinq pains et deux poissons.
\VS{18}Et il leur dit : Apportez-les-moi ici.
\VS{19}Et après avoir ordonné à la foule de s'asseoir sur l'herbe, il prit les cinq pains et les deux poissons, et levant les yeux au ciel, il rendit grâces à Dieu. Puis ayant rompu les pains, il les donna aux disciples qui les distribuèrent à la foule.
\VS{20}Tous en mangèrent et furent rassasiés, et l'on emporta douze paniers pleins des morceaux qui restaient.
\VS{21}Ceux qui avaient mangé étaient environ cinq mille hommes, sans compter les femmes et les petits enfants.
\TextTitle{Jésus marche sur les eaux, incrédulité de Pierre\FTNTT{Mc. 6:45-56 ; Jn. 6:15-21}}
\VS{22}Aussitôt après, Jésus obligea ses disciples à monter dans la barque et à passer avant lui de l'autre côté, pendant qu'il renverrait la foule.
\VS{23}Et quand il l'eut renvoyée, il monta sur une montagne pour être à part, afin de prier ; et le soir étant venu, il était là seul.
\VS{24}La barque, déjà au milieu de la mer, était battue par les flots ; car le vent était contraire.
\VS{25}Et vers la quatrième veille de la nuit, Jésus alla vers eux, marchant sur la mer.
\VS{26}Et ses disciples le voyant marcher sur la mer, ils furent troublés et ils dirent : C'est un fantôme ! Et, dans leur frayeur, ils poussèrent des cris.
\VS{27}Jésus leur dit aussitôt : Rassurez-vous, c'est moi, n'ayez pas de peur !
\VS{28}Et Pierre lui répondit : Seigneur, si c'est toi, ordonne que j'aille vers toi sur les eaux.
\VS{29}Et il lui dit : Viens ! Pierre sortit de la barque, marcha sur les eaux pour aller vers Jésus.
\VS{30}Mais voyant que le vent était fort, il eut peur ; et comme il commençait à enfoncer, il s'écria : Seigneur ! Sauve-moi !
\VS{31}Et aussitôt Jésus étendit sa main et le prit en lui disant : Homme de peu de foi, pourquoi as-tu douté ?
\VS{32}Et quand ils furent montés dans la barque, le vent s'apaisa.
\VS{33}Alors ceux qui étaient dans la barque, vinrent adorer Jésus et dirent : Certes, tu es le Fils de Dieu.
\TextTitle{Jésus guérit des malades à Génésareth\FTNTT{Mc. 6:53-56}}
\VS{34}Après avoir traversé la mer, ils vinrent dans le pays de Génézareth.
\VS{35}Les gens de ce lieu ayant reconnu Jésus, envoyèrent des messagers dans tous les environs et on lui amena tous les malades.
\VS{36}Et ils le prièrent de leur permettre de toucher seulement le bord de son vêtement. Et tous ceux qui le touchèrent furent guéris.
\Chap{15}
\TextTitle{Jésus-Christ condamne les traditions\FTNTT{Mc. 7:1-13}}
\VerseOne{}Alors des scribes et des pharisiens vinrent de Jérusalem auprès de Jésus et lui dirent :
\VS{2}Pourquoi tes disciples transgressent-ils la tradition des anciens ? Car ils ne se lavent point les mains quand ils prennent leur repas.
\VS{3}Il leur répondit : Et vous, pourquoi transgressez-vous le commandement de Dieu par votre tradition ?
\VS{4}Car Dieu a dit : Honore ton père et ta mère. Et il a dit aussi : Celui qui maudira son père ou sa mère finira à la mort.
\VS{5}Mais vous, vous dites : Celui qui dira à son père ou à sa mère : Tout ce dont j'aurais pu t'assister est une offrande à Dieu, n'est pas coupable, quoiqu'il n'honore pas son père ou sa mère.
\VS{6}Vous annulez ainsi le commandement de Dieu par votre tradition.
\VS{7}Hypocrites, Esaïe a bien prophétisé de vous, en disant :
\VS{8}Ce peuple s'approche de moi de sa bouche et m'honore des lèvres ; mais son cœur est très éloigné de moi.
\VS{9} Mais ils m'honorent en vain, en enseignant des doctrines qui ne sont que des commandements d'hommes\FTNT{Es. 29:13.}.
\TextTitle{Verdict sur le coeur humain\FTNTT{Mc. 7:14-23}}
\VS{10}Puis ayant appelé à lui la foule, il lui dit : Ecoutez, et comprenez ceci :
\VS{11}Ce n'est pas ce qui entre dans la bouche qui souille l'homme ; mais ce qui sort de la bouche c'est ce qui souille l'homme.
\VS{12}Sur cela les disciples s'approchant, lui dirent : Sais-tu que les pharisiens ont été scandalisés quand ils ont entendus ce discours ?
\VS{13}Et il répondit et dit : Toute plante que mon Père céleste n'a pas plantée sera déracinée.
\VS{14}Laissez-les, ce sont des aveugles, conducteurs d'aveugles ; si un aveugle conduit un autre aveugle, ils tomberont tous deux dans la fosse.
\VS{15}Alors Pierre prenant la parole, lui dit : Explique-nous cette parabole.
\VS{16}Et Jésus dit : Vous aussi, êtes-vous encore sans intelligence ?
\VS{17}Ne comprenez-vous pas encore que tout ce qui entre dans la bouche va dans le ventre, puis est jeté dans les lieux secrets ?
\VS{18}Mais les choses qui sortent de la bouche partent du cœur, et ces choses-là souillent l'homme.
\VS{19}Car c'est du cœur que sortent les mauvaises pensées, les meurtres, les adultères, les fornications, les vols, les faux témoignages, les médisances.
\VS{20}Ce sont ces choses-là qui souillent l'homme ; mais de manger sans avoir les mains lavées, cela ne souille point l'homme.
\TextTitle{Jésus et la femme cananéenne\FTNTT{Mc. 7:24-30}}
\VS{21}Alors Jésus, partant de là se retira dans le territoire de Tyr et de Sidon.
\VS{22}Et voici, une femme cananéenne, qui venait de ces contrées, lui cria : Seigneur ! Fils de David, aie pitié de moi ! Ma fille est cruellement tourmentée par le démon.
\VS{23}Mais il ne lui répondit pas un mot. Et ses disciples s'approchèrent et lui dirent : Renvoie-la, car elle crie derrière nous.
\VS{24}Et il répondit : Je n'ai été envoyé qu'aux brebis perdues de la maison d'Israël.
\VS{25}Mais elle vint et l'adora, disant : Seigneur, assiste-moi !
\VS{26}Et il lui répondit en disant : Il ne convient pas de prendre le pain des enfants et de le jeter aux petits chiens.
\VS{27}Mais elle dit : Cela est vrai, Seigneur ! Cependant les petits chiens mangent des miettes qui tombent de la table de leurs maîtres.
\VS{28}Alors Jésus répondant, lui dit : Ô femme ! Ta foi est grande. Qu'il te soit fait comme tu le souhaites. Et, à l'heure même, sa fille fut guérie.
\TextTitle{Nouvelles guérisons}
\VS{29}Et Jésus quitta ces lieux et vint près de la mer de Galilée. Puis il monta sur une montagne et s'y assit là.
\VS{30}Et une grande foule vint à lui, ayant avec elle des boiteux, des aveugles, des muets, des estropiés et beaucoup d'autres malades. On les mit aux pieds de Jésus et il les guérit ;
\VS{31}de sorte que la foule était dans l'admiration de voir que les muets parlaient, que les estropiés étaient guéris, que les boiteux marchaient, que les aveugles voyaient ; et elle glorifiait le Dieu d'Israël.
\TextTitle{Seconde multiplication des pains\FTNTT{Mc. 8:1-9}}
\VS{32}Alors Jésus, ayant appelé ses disciples, dit : Je suis ému de compassion pour cette foule de gens ; car voilà trois jours qu'ils sont près de moi et ils n'ont rien à manger. Je ne veux pas les renvoyer à jeun, de peur que les forces ne leur manquent en chemin.
\VS{33}Et ses disciples lui dirent : D'où pourrions-nous tirer dans ce désert assez de pains pour rassasier une si grande multitude ?
\VS{34}Et Jésus leur dit : Combien avez-vous de pains ? Ils lui dirent : Sept, et quelque peu de petits poissons.
\VS{35}Alors il commanda aux foules de s'asseoir par terre.
\VS{36}Et ayant prit les sept pains et les poissons, et après avoir béni Dieu, il les rompit et les donna à ses disciples, qui les distribuèrent à la foule.
\VS{37}Et tous mangèrent et furent rassasiés, et l'on emporta sept corbeilles pleines des morceaux qui restaient.
\VS{38}Or, ceux qui avaient mangé étaient quatre mille hommes, sans compter les femmes et les petits enfants.
\VS{39}Et Jésus renvoya la foule, monta sur une barque, et se rendit dans le territoire de Magdala.
\Chap{16}
\TextTitle{La cécité d'une génération méchante et adultère\FTNTT{Mc. 8:10-14}}
\VerseOne{}Alors les pharisiens et les sadducéens vinrent à lui, et pour l'éprouver, lui demandèrent de leur faire voir un signe venant du ciel.
\VS{2}Mais il leur répondit : Quand le soir est venu, vous dites : Il fera beau temps, car le ciel est rouge.
\VS{3}Et le matin vous dites : Il y aura de l'orage aujourd'hui, car le ciel est d'un rouge sombre. Hypocrites, vous savez bien discerner l'aspect du ciel, et vous ne pouvez discerner les signes des temps !
\VS{4}Une génération méchante et adultère demande un miracle ; mais il ne lui sera point donné d'autre miracle que celui de Jonas le prophète. Puis il les quitta et s'en alla.
\VS{5}Et ses disciples, en passant sur l'autre bord, avaient oublié de prendre des pains.
\TextTitle{Le levain des pharisiens et des sadducéens, une doctrine corrompue\FTNTT{Mc. 8:15-21 ; Lu. 12:1-15}}
\VS{6}Et Jésus leur dit : Gardez-vous avec soin du levain des pharisiens et des sadducéens.
\VS{7}Ils résonnaient en eux-mêmes et disaient : C'est parce que nous n'avons pas pris de pains.
\VS{8}Et Jésus connaissant leurs pensées leur dit : Gens de peu de foi, pourquoi raisonnez-vous en vous-mêmes sur le fait que vous n'avez pas pris de pains ?
\VS{9}Ne comprenez-vous point encore, et ne vous rappelez-vous plus les cinq pains des cinq mille hommes et combien de paniers vous avez emportés,
\VS{10}ni des sept pains des quatre mille hommes et combien de corbeilles vous avez emportées ?
\VS{11}Comment ne comprenez-vous pas que ce n'est pas au sujet du pain que je vous ai dit, de vous garder du levain des pharisiens et des sadducéens ?
\VS{12}Alors ils comprirent que ce n'était pas du levain du pain qu'il leur avait dit de se garder, mais de la doctrine des pharisiens et des sadducéens.
\TextTitle{Pierre reconnaît Jésus comme le Messie\FTNTT{Mc. 8:27-30 ; Lu. 9:18-21 ; Jn. 6:66-71}}
\VS{13}Jésus, étant arrivé dans le territoire de Césarée de Philippe, demanda à ses disciples : Qui disent les hommes que je suis, moi le Fils de l'homme ?
\VS{14}Et ils lui répondirent : Les uns disent que tu es Jean-Baptiste ; les autres, Elie ; et les autres, Jérémie, ou l'un des prophètes.
\VS{15}Il leur dit : et vous, qui dites-vous que je suis ?
\VS{16}Simon Pierre répondit et dit : Tu es le Christ, le Fils du Dieu vivant.
\TextTitle{Jésus bâtit son Eglise}
\VS{17}Et Jésus lui répondit et dit : Tu es heureux, Simon, fils de Jonas, car ce ne sont pas la chair et le sang qui t'ont révélé cela, mais mon Père qui est dans les cieux.
\VS{18}Et moi je te dis, que tu es Pierre, et que sur ce Roc\FTNT{Le Roc : Ce passage a été mal traduit dans beaucoup de Bibles comme suit : « Et moi, je te dis que tu es Pierre, et que sur cette pierre je bâtirai mon Eglise… ». Or pour une bonne compréhension des propos de Jésus, il est important d'insister sur la distinction que le grec fait entre « Petros » (pierre, caillou), l'apôtre Pierre, et « Petra » (roc, rocher), qui n'est autre que Jésus-Christ, le rocher des siècles (Es. 17:10 ; Es. 26:4 ; 1 Co. 10:4). De là en découle un enseignement fondamental : l'Eglise n'est bâtie ni par un homme ni sur l'homme, en l'occurrence Pierre et ses supposés successeurs (papes), mais par Jésus-Christ lui-même qui en est la Pierre Angulaire et le fondement inébranlable (Ac. 4:11 ; Ep. 2:20).} je bâtirai mon Eglise ; et les portes de l'enfer\FTNT{Enfer : du grec « Hadès ». Hadès chez les Grecs ou Pluton chez les Romains, était considéré comme le dieu des profondeurs souterraines et le maître des enfers. Ce terme est parfois traduit par « séjour des morts », équivalent hébreu de « Scheol ». Les Grecs utilisaient l'euphémisme Pylartes, signifiant « aux portes solidement closes », pour parler du très craint Hadès. En effet, Juifs, Grecs et Romains avaient conscience que les portes closes de l'enfer ne laissaient personne sortir du royaume de la mort. Tous les impies, et même les croyants d'avant Jésus-Christ, étaient retenus par les portes de l'enfer. Toutefois, les croyants allaient dans une partie de l'enfer que les juifs appelaient « sein d'Abraham » (1 Sam. 28:7-19 ; Lu. 16:22-25 ; Lu. 23:43) où ils ne subissaient pas les tourments infligés aux impies. Lorsque le Seigneur est mort, il est descendu « dans les régions inférieures de la terre » pour prendre les clés du Hadès, clés du séjour des morts (Col. 2:15 ; Ap. 1:17-18) et libérer les captifs pieux. Jésus affirme que les portes de l'enfer ne prévaudront jamais contre son Eglise puisque c'est lui qui l'a bâtie. Malgré tout, Hadès, bien que vaincu par le Seigneur, essaie d'attirer l'Eglise que le Seigneur a établie dans les lieux célestes (Ep. 2:4-9 ; Col. 3:1) vers le royaume des ténèbres, au travers des fausses doctrines et du péché. Au jour du jugement dernier, Hadès et la mort, qui sont deux démons, seront jetés dans l'étang de feu et de soufre (Ap. 20:11-15).} ne prévaudront point contre elle.
\VS{19}Je te donnerai les clefs du Royaume des cieux ; et tout ce que tu lieras sur la terre, sera lié dans les cieux ; et tout ce que tu délieras sur la terre, sera délié dans les cieux\FTNT{Une mauvaise compréhension de ce verset a contribué à propager l'idée erronée selon laquelle Pierre serait le médiateur entre Dieu et les hommes, puisque c'est lui qui détiendrait les clés du Royaume des cieux. Toutefois, Es. 22:22 affirme que seul Jésus-Christ détient ces clés qui symbolisent l'autorité et la domination. Or dans le cadre de l'héritage que le Seigneur nous a laissé, cette autorité est désormais exercée en son Nom par tous les membres du corps de Christ (Mt. 18:18).}.
\VS{20}Alors il commanda expressément à ses disciples de ne dire à personne qu'il était Jésus le Christ.
\TextTitle{Jésus parle de sa mort et de sa résurrection\FTNTT{Mc. 8:31-33 ; Lu. 9:22}}
\VS{21}Dès lors Jésus commença à déclarer à ses disciples qu'il fallait qu'il aille à Jérusalem, qu'il souffre beaucoup de la part des anciens, des principaux sacrificateurs et des scribes, qu'il soit mis à mort, et qu'il ressuscite le troisième jour.
\VS{22}Mais Pierre l'ayant pris à part, se mit à le reprendre en lui disant : Seigneur, aie pitié de toi, cela ne t'arrivera point !
\VS{23}Mais lui, s'étant retourné, dit à Pierre : Arrière de moi, Satan ! Tu m'es en scandale, car tu ne comprends pas les choses qui sont de Dieu, mais celles qui sont des hommes.
\TextTitle{La consécration du disciple\FTNTT{Mc. 8:34-38 ; Lu. 9:23-26}}
\VS{24}Alors Jésus dit à ses disciples : Si quelqu'un veut venir après moi, qu'il renonce à lui-même, et qu'il se charge de sa croix, et qu'il me suive.
\VS{25}Car quiconque voudra sauver son âme, la perdra ; mais quiconque perdra son âme à cause de son amour pour moi, la trouvera.
\VS{26}Et que servirait-il à un homme de gagner tout le monde, s'il perdait son âme ? Ou, que donnerait un homme en échange de son âme ?
\VS{27}Car le Fils de l'homme doit venir dans la gloire de son Père avec ses anges ; et alors il rendra à chacun selon ses œuvres.
\VS{28}Je vous le dis en vérité, quelques-uns de ceux qui sont ici présents, ne mourront point, qu'ils n'aient vu le Fils de l'homme venir dans son règne\FTNT{Ce passage doit être lu de concert avec Mt. 24:32-34. Jésus utilise un langage prophétique pour expliquer deux réalités. La première réalité est spirituelle et concerne ses contemporains qui allaient vivre l'effusion de l'Esprit pour rétablir le Royaume de Dieu dans le cœur des gens. En effet, le Seigneur ne les a pas laissés orphelins, mais il est revenu sous la forme de l'Esprit (Jn. 14:17-18 ; Ac. 2 ; Ac. 16:7). Aussi, les apôtres ont pu proclamer ce Royaume partout où ils allaient (Ac. 20:25). La deuxième réalité est matérielle et concerne le fleurissement du figuier, c'est-à-dire Israël. L'histoire atteste le fleurissement de ce figuier tant sur le plan géographique que sur le plan numérique. Depuis le 14 mai 1948, date de la naissance officielle de l'état hébreu, Israël ne cesse de s'étendre. Cette nation est l'horloge des temps car le Messie gouvernera le monde entier depuis Jérusalem (Mi. 4 ; Za. 14).}.
\Chap{17}
\TextTitle{Transfiguration de Jésus-Christ\FTNTT{Mc. 9:1-8 ; Lu. 9:27-36}}
\VerseOne{}Six jours après, Jésus prit Pierre, Jacques et Jean son frère, et les conduisit à l'écart sur une haute montagne.
\VS{2}Et il fut transfiguré en leur présence et son visage resplendit comme le soleil ; et ses vêtements devinrent blancs comme la lumière.
\VS{3}Et voici, ils virent Moïse et Elie qui s'entretenaient avec lui.
\VS{4}Alors Pierre prenant la parole, dit à Jésus : Seigneur, il est bon que nous soyons ici. Faisons-y, si tu le veux, trois tentes, une pour toi, une pour Moïse, et une pour Elie.
\VS{5}Et comme il parlait encore, voici une nuée resplendissante les couvrit de son ombre. Et voici, une voix fit entendre de la nuée ces paroles : Celui-ci est mon Fils bien-aimé, en qui j'ai pris mon bon plaisir : Ecoutez-le !
\VS{6}Lorsque les disciples entendirent cette voix, ils tombèrent le visage contre terre et furent saisis d'une très grande frayeur.
\VS{7}Mais Jésus, s'approchant, les toucha et leur dit : Levez-vous et n'ayez pas peur.
\VS{8}Ils levèrent les yeux, et ne virent personne, excepté Jésus tout seul.
\VS{9}Et comme ils descendaient de la montagne, Jésus leur donna cet ordre, en disant : Ne parlez à personne de cette vision, jusqu'à ce que le Fils de l'homme soit ressuscité des morts.
\VS{10}Et ses disciples l'interrogèrent, en disant : Pourquoi donc les scribes disent-ils qu'il faut qu'Elie vienne premièrement ?
\VS{11}Et Jésus répondant, leur dit : Il est vrai qu'Elie viendra premièrement et rétablira toutes choses.
\VS{12}Mais je vous dis qu'Elie est déjà venu, et ils ne l'ont pas reconnu et ils lui ont fait tout ce qu'ils ont voulu. De même, le Fils de l'homme doit souffrir aussi de leur part.
\VS{13}Alors les disciples comprirent que c'était de Jean-Baptiste qu'il leur parlait.
\TextTitle{Le manque de foi des disciples\FTNTT{Mc. 9:14-29 ; Lu. 9:37-43}}
\VS{14}Et quand ils furent arrivés près de la foule, un homme s'approcha et se mit à genoux devant lui,
\VS{15}et lui dit : Seigneur ! Aie pitié de mon fils qui est lunatique et misérablement affligé ; car il tombe souvent dans le feu et souvent dans l'eau.
\VS{16}Et je l'ai présenté à tes disciples, mais ils n'ont pas pu le guérir.
\VS{17}Et Jésus répondit et dit : Ô race incrédule et perverse, jusqu'à quand serai-je avec vous ? Jusqu'à quand vous supporterai-je ? Amenez-le-moi ici.
\VS{18}Et Jésus parla sévèrement au démon, qui sortit de lui, et à l'heure même l'enfant fut guéri.
\VS{19}Alors les disciples s'approchèrent de Jésus et lui dirent en particulier : Pourquoi n'avons-nous pas pu le chasser ?
\VS{20}Et Jésus leur répondit : C'est à cause de votre incrédulité. Je vous le dis en vérité, si vous aviez de la foi, comme un grain de sénevé, vous diriez à cette montagne : Transporte-toi d'ici là, et elle se transporterait ; et rien ne vous serait impossible.
\VS{21}Mais cette sorte de démon ne sort que par la prière et par le jeûne.
\TextTitle{Jésus évoque à nouveau sa mort et sa résurrection\FTNTT{Mc. 9:30-32 ; Lu. 9:44-45}}
\VS{22}Et comme ils se trouvaient en Galilée, Jésus leur dit : Il arrivera que le Fils de l'homme sera livré entre les mains des hommes ;
\VS{23}Et qu'ils le feront mourir, mais le troisième jour il ressuscitera. Et les disciples en furent fort attristés.
\TextTitle{La pièce d'argent dans la bouche d'un poisson\FTNTT{Mc. 12:13-17}}
\VS{24}Et lorsqu'ils arrivèrent à Capernaüm, ceux qui percevaient les deux drachmes s'adressèrent à Pierre et lui dirent : Votre Maître ne paye-t-il pas les deux drachmes ?
\VS{25}Oui dit-il. Et quand il fut entré dans la maison, Jésus le prévint en lui disant : Qu'est-ce qu'il t'en semble, Simon ? Les rois de la terre, de qui perçoivent-ils des tributs ou des impôts ? Est-ce de leurs enfants ou des étrangers ?
\VS{26}Pierre dit : Des étrangers. Jésus lui répondit : Les enfants en sont donc exempts.
\VS{27}Mais afin que nous ne les scandalisions point, va-t'en à la mer et jette l'hameçon, et prends le premier poisson qui viendra ; ouvre-lui la bouche, tu trouveras un statère. Prends-le et donne-le-leur pour moi et pour toi.
\Chap{18}
\TextTitle{L'humilité, secret de la vraie grandeur\FTNTT{Mc. 9:33-37; Lu. 9:46-48}}
\VerseOne{}En cette même heure-là, les disciples s'approchèrent de Jésus, en lui disant : Qui est le plus grand dans le Royaume des cieux ?
\VS{2}Et Jésus ayant appelé un petit enfant, le mit au milieu d'eux,
\VS{3}et leur dit : Je vous le dis en vérité, que si vous ne vous convertissez pas et si vous ne devenez pas comme les petits enfants, vous n'entrerez pas dans le Royaume des cieux.
\VS{4}C'est pourquoi quiconque deviendra humble, comme ce petit enfant, celui-là est le plus grand dans le Royaume des cieux.
\VS{5}Et quiconque reçoit en mon Nom un petit enfant comme celui-ci, il me reçoit.
\VS{6}Mais, quiconque scandalise un de ces petits qui croient en moi, il vaudrait mieux pour lui qu'on mette à son cou une meule d'âne, et qu'on le jette au fond de la mer.
\TextTitle{Les scandales et les occasions de chute}
\VS{7}Malheur au monde à cause des scandales ! Car il est nécessaire qu'il arrive des scandales ; mais malheur à l'homme par qui le scandale arrive !
\VS{8}Si ta main ou ton pied est pour toi une occasion de chute, coupe-les et jette-les loin de toi ; car il vaut mieux que tu entres boiteux ou manchot dans la vie, que d'avoir deux pieds ou deux mains, et d'être jeté dans le feu éternel.
\VS{9}Et si ton œil est pour toi une occasion de chute, arrache-le et jette-le loin de toi ; car il vaut mieux que tu entres dans la vie n'ayant qu'un œil, que d'avoir deux yeux, et d'être jeté dans le feu de la géhenne.
\VS{10}Gardez-vous de mépriser un seul de ces petits ; car je vous dis que dans les cieux leurs anges voient continuellement la face de mon Père qui est aux cieux.
\VS{11}Car le Fils de l'homme est venu pour sauver ce qui était perdu.
\TextTitle{Parabole de la brebis égarée\FTNTT{Lu. 15:3-7}}
\VS{12}Que vous en semble ? Si un homme a cent brebis, et que l'une d'elles s'égare, ne laisse-t-il pas les quatre-vingt-dix-neuf autres, pour aller dans les montagnes chercher celle qui s'est égarée ?
\VS{13}Et, s'il arrive qu'il la trouve, je vous le dis en vérité, qu'il en a plus de joie, que les quatre-vingt-dix-neuf qui ne se sont pas égarées.
\VS{14}Ainsi la volonté de votre Père qui est aux cieux n'est pas qu'un seul de ces petits périsse.
\TextTitle{Discipline dans les assemblées}
\VS{15}Que si ton frère a péché contre toi, va, et reprends-le entre toi et lui seul. S'il t'écoute, tu as gagné ton frère.
\VS{16}Mais s'il ne t'écoute pas, prends encore avec toi une ou deux personnes, afin que par la bouche de deux ou trois témoins toute parole soit ferme\FTNT{De. 19:15.}.
\VS{17}S'il refuse de les écouter, dis-le à l'Eglise ; et s'il refuse aussi d'écouter l'Eglise, qu'il soit pour toi comme un païen et comme un publicain.
\VS{18}En vérité je vous dis que tout ce que vous lierez sur la terre, sera lié dans le ciel ; et tout ce que vous délierez sur la terre sera délié dans le ciel\FTNT{Voir commentaire en Mt. 16:19.}.
\VS{19}Je vous dis aussi que si deux d'entre vous s'accordent sur la terre, tout ce qu'ils demanderont leur sera donné par mon Père qui est aux cieux.
\VS{20}Car là où deux ou trois sont assemblés en mon Nom, je suis là au milieu d'eux.
\TextTitle{Ne jamais se lasser de pardonner}
\VS{21}Alors Pierre s'approchant, lui dit : Seigneur, combien de fois mon frère péchera-il contre moi et lui pardonnerai-je ? Sera-ce jusqu'à sept fois ?
\VS{22}Jésus lui répondit : Je ne te dis pas jusqu'à sept fois, mais jusqu'à soixante-dix fois sept fois.
\TextTitle{Parabole du roi et du méchant serviteur}
\VS{23}C'est pourquoi le Royaume des cieux est semblable à un roi qui voulut faire rendre compte à ses serviteurs.
\VS{24}Et quand il se mit à compter, on lui en présenta un qui lui devait dix mille talents.
\VS{25}Et parce qu'il n'avait pas de quoi payer, son maître ordonna qu'il soit vendu, lui, sa femme, ses enfants et tout ce qu'il avait, et que la dette soit payée.
\VS{26}Mais ce serviteur se jetant à ses pieds, le suppliait en disant : Seigneur, aie patience envers moi et je te rendrai le tout.
\VS{27}Alors le maître de ce serviteur, ému de compassion, le relâcha et lui remit la dette.
\VS{28}Mais ce serviteur étant sorti, rencontra un de ses compagnons de service, qui lui devait cent deniers ; et l'ayant pris, il l'étranglait, en lui disant : paye-moi ce que tu me dois.
\VS{29}Mais son compagnon de service se jetant à ses pieds, le suppliait en disant : Aie patience et je te rendrai le tout.
\VS{30}Mais l'autre ne voulut pas et il alla le jeter en prison, jusqu'à ce qu'il ait payé la dette.
\VS{31}Or ses autres compagnons de service voyant ce qui était arrivé, en furent extrêmement attristés et ils allèrent raconter à leur maître tout ce qui s'était passé.
\VS{32}Alors son maître le fit venir et lui dit : Méchant serviteur, je t'avais remis en entier ta dette, parce que tu m'en avais supplié ;
\VS{33}Ne te fallait-il pas aussi avoir pitié de ton compagnon de service, comme j'avais eu pitié de toi ?
\VS{34}Et son maître étant en colère le livra aux bourreaux, jusqu'à ce qu'il lui ait payé tout ce qu'il devait.
\VS{35}C'est ainsi que vous fera mon Père céleste, si vous ne pardonnez de tout votre cœur, chacun à son frère, ses fautes.
\Chap{19}
\TextTitle{Enseignement de Jésus sur le mariage et le divorce\FTNTT{Mt. 5:31-32 ; Mc. 10:2-12 ; Lu. 16:18 ; Ro. 7:1-3 ; 1 Co. 7:10-16}}
\VerseOne{}Et il arriva que quand Jésus eut achevé ces discours, il quitta la Galilée, et alla dans le territoire de la Judée, au-delà du Jourdain.
\VS{2}Et de grandes foules le suivirent, et là il guérit leurs malades.
\VS{3}Alors les pharisiens vinrent à lui pour l'éprouver, et ils lui dirent : Est-il permis à un homme de répudier sa femme pour quelque cause que ce soit ?
\VS{4}Et il répondit et leur dit : N'avez-vous pas lu que le Créateur, au commencement, fit l'homme et la femme ?
\VS{5}Et il dit : A cause de cela, l'homme quittera son père et sa mère, et s'attachera à sa femme, et les deux ne seront qu'une seule chair ?
\VS{6}Ainsi ils ne sont plus deux, mais une seule chair\FTNT{Ge. 2:24.}. Que l'homme donc ne sépare pas ce que Dieu a mis ensemble sous un joug\FTNT{La majorité des traducteurs traduisent ce verset par « Que l'homme donc ne sépare pas ce que Dieu a joint. ». Or le terme grecque « suzeugnumi » qu'ils ont traduit par « joint » signifie plutôt « attacher un joug à quelqu'un, mettre ensemble sous un joug ». Un joug est une pièce de bois servant à atteler une paire d'animaux. De ce fait, les animaux sont contraints d'avancer dans la même direction, côte à côte (Ge. 2:22). En De. 22:10, Dieu interdit d'atteler un âne avec un bœuf ensemble. L'adverbe « ensemble » vient de l'hébreu « Yachad » et signifie « union d'une façon unitaire ». Ce verset fait référence symboliquement aux paroles de l'apôtre Paul en 2 Co. 6:14-16 qui nous mettent en garde contre le mariage avec des infidèles. Le mariage est donc semblable à un joug qui nous contraint à marcher à l'unisson dans la même direction. Ainsi, si on se lie à un inconverti, ce dernier risque de nous entraîner sur la voie de la perdition. En Mt 11:29, Christ nous invite à nous mettre sous son joug, qui est doux et léger. Quelle belle demande en mariage !}.
\VS{7}Ils lui dirent : Pourquoi donc Moïse a-t-il commandé de donner la lettre de divorce, et de répudier sa femme\FTNT{De. 24:1.} ?
\VS{8}Il leur répondit : C'est à cause de la dureté de votre cœur que Moïse vous a permis de répudier vos femmes, mais au commencement il n'en était pas ainsi.
\VS{9}Et moi je vous dis, que quiconque répudiera sa femme, si ce n'est pour cause d'adultère\FTNT{Adultère : Du grec « porneia » c'est-à-dire relation sexuelle illicite, impudicité.}, et se mariera à une autre, commet un adultère ; et que celui qui se sera marié à celle qui est répudiée, commet un adultère.
\VS{10}Ses disciples lui dirent : Si telle est la condition de l'homme à l'égard de sa femme, il ne convient pas de se marier.
\VS{11}Mais il leur répondit : Tous ne sont pas capables de cela, mais seulement ceux à qui il est donné.
\VS{12}Car il y a des eunuques, qui sont ainsi nés dés le ventre de leur mère ; et il y a des eunuques, qui ont été faits eunuques par les hommes ; et il y a des eunuques qui se sont faits eux-mêmes eunuques pour le Royaume des cieux. Que celui qui peut comprendre ceci, le comprenne.
\TextTitle{Le Royaume des cieux pour ceux qui ressemblent aux petits enfants\FTNTT{Mc. 10:13-16 ; Lu. 18:15-17}}
\VS{13}Alors on lui présenta des petits enfants, afin qu'il leur impose les mains et qu'il prie pour eux. Mais les disciples les en reprenaient.
\VS{14}Et Jésus leur dit : Laissez venir à moi les petits enfants et ne les empêchez pas ; car le Royaume des cieux est pour ceux qui leur ressemblent.
\VS{15}Puis il leur imposa les mains et il partit de là.
\TextTitle{Le jeune homme riche\FTNTT{Mc. 10:17-31 ; Lu. 10:25-37 ; Lu. 18:18-27.}}
\VS{16}Et voici, quelqu'un s'approchant lui dit : Maître qui est bon, quel bien ferai-je pour avoir la vie éternelle ?
\VS{17}Il lui répondit : Pourquoi m'appelles-tu bon ? Dieu est le seul être qui soit bon. Que si tu veux entrer dans la vie, garde les commandements.
\VS{18}Il lui dit : Lesquels ? Et Jésus lui répondit : Tu ne tueras point. Tu ne commettras point d'adultère. Tu ne déroberas point. Tu ne diras point de faux témoignage.
\VS{19}Honore ton père et ta mère ; et tu aimeras ton prochain comme toi-même\FTNT{Ex. 20:12-16 ; Lé. 19:18.}.
\VS{20}Le jeune homme lui dit : J'ai gardé toutes ces choses dès ma jeunesse. Que me manque-t-il encore ?
\VS{21}Jésus lui dit : Si tu veux être parfait, va, vends ce que tu as, et donne-le aux pauvres, et tu auras un trésor dans le ciel ; puis viens et suis-moi.
\VS{22}Mais quand ce jeune homme eut entendu cette parole, il s'en alla tout triste, parce qu'il avait de grands biens.
\VS{23}Alors Jésus dit à ses disciples : Je vous le dis en vérité, un riche entrera difficilement dans le Royaume des cieux.
\VS{24}Je vous le dis encore : Il est plus aisé à un chameau de passer par le trou d'une aiguille\FTNT{Le trou d'une aiguille : Jésus fait référence à une porte de la ville de Jérusalem qui était trop basse pour que les chameaux puissent y passer avec leurs chargements.}, qu'il ne l'est qu'un riche entre dans le Royaume de Dieu.
\VS{25}Ses disciples ayant entendu ces choses furent très étonnés et dirent : Qui peut donc être sauvé ?
\VS{26}Et Jésus les regarda et leur dit : Quant aux hommes, cela est impossible, mais quant à Dieu toutes choses sont possibles.
\TextTitle{Récompenses actuelles et dans le Royaume à venir\FTNTT{Mc. 10:28-31 ; Lu. 18:28-30}}
\VS{27}Alors Pierre prenant la parole, lui dit : Voici, nous avons tout quitté et nous t'avons suivi ; que nous en arrivera-t-il donc ?
\VS{28}Et Jésus leur dit : Je vous le dis en vérité, quand le Fils de l'homme, au renouvellement de toutes choses, sera assis sur le trône de sa gloire, vous qui m'avez suivi, vous serez assis sur douze trônes et vous jugerez les douze tribus d'Israël.
\VS{29}Et quiconque aura quitté ou maisons, ou frères, ou sœurs, ou père, ou mère, ou femme, ou enfants, ou champs, à cause de mon Nom, il en recevra cent fois autant et héritera la vie éternelle.
\VS{30}Mais plusieurs qui sont les premiers seront les derniers et les derniers seront les premiers.
\Chap{20}
\TextTitle{Parabole des ouvriers}
\VerseOne{} Car le Royaume des cieux est semblable à un père de famille, qui sortit dès le point du jour afin de louer des ouvriers pour sa vigne.
\VS{2}Et quand il eut accordé avec les ouvriers à un denier par jour, il les envoya à sa vigne.
\VS{3}Puis étant sorti vers la troisième heure, il en vit d'autres qui étaient sur la place publique, sans rien faire.
\VS{4}Il leur dit : Allez aussi à ma vigne, et je vous donnerai ce qui sera raisonnable.
\VS{5}Et ils y allèrent. Puis il sortit de nouveau vers la sixième heure et vers la neuvième, et il fit de même.
\VS{6}Et étant sorti vers la onzième heure, il en trouva d'autres qui étaient sur la place publique sans rien faire, et il leur dit : Pourquoi vous tenez-vous ici toute la journée sans rien faire ?
\VS{7}Ils lui répondirent : Parce que personne ne nous a loués. Et il leur dit : Allez-vous aussi à ma vigne et vous recevrez ce qui sera raisonnable.
\VS{8}Et le soir étant venu, le maître de la vigne dit à son intendant : Appelle les ouvriers et paye-leur le salaire, en commençant depuis les derniers jusqu'aux premiers.
\VS{9}Alors ceux qui avaient été loués vers la onzième heure vinrent et reçurent chacun un denier.
\VS{10}Or quand les premiers furent venus, ils croyaient recevoir davantage, mais ils reçurent aussi chacun un denier.
\VS{11}Et l'ayant reçu, ils murmuraient contre le père de famille,
\VS{12}en disant : Ces derniers n'ont travaillé qu'une heure, et tu les as faits égaux à nous, qui avons supporté le poids du jour et la chaleur.
\VS{13}Et il répondit à l'un d'eux et lui dit : Mon ami, je ne te fais pas de tort, n'es-tu pas tombé d'accord avec moi pour un denier ?
\VS{14}Prends ce qui est à toi et va-t'en. Mais je veux donner à ce dernier autant qu'à toi,
\VS{15}ne m'est-il pas permis de faire ce que je veux de mes biens ? Ou vois-tu d'un mauvais œil que je sois bon ?
\VS{16}Ainsi les derniers seront les premiers et les premiers seront les derniers, car il y a beaucoup d'appelés, mais peu d'élus.
\TextTitle{Jésus annonce à nouveau sa mort et sa résurrection\FTNTT{Mt. 12:38-42 ; 16:21-28 ; 17:22-23 ; Mc. 10:32-34 ; Lu. 18:31-34.}}
\VS{17}Pendant que Jésus montait à Jérusalem, il prit à part ses douze disciples et il leur dit en chemin :
\VS{18}Voici, nous montons à Jérusalem, et le Fils de l'homme sera livré aux principaux sacrificateurs et aux scribes, et ils le condamneront à la mort.
\VS{19}Ils le livreront aux nations pour qu'elles se moquent de lui, le battent de verges et le crucifient ; et le troisième jour il ressuscitera.
\TextTitle{Réponse de Jésus à la requête de la mère de Jacques et Jean\FTNTT{Mc. 10:35-45}}
\VS{20}Alors la mère des fils de Zébédée s'approcha de lui avec ses fils et se prosterna pour lui demander quelque chose.
\VS{21}Et il lui dit : Que veux-tu ? Elle lui dit : Ordonne que mes deux fils, qui sont ici, soient assis l'un à ta droite, et l'autre à ta gauche dans ton Royaume.
\VS{22}Et Jésus répondit et dit : Vous ne savez pas ce que vous demandez. Pouvez-vous boire la coupe que je dois boire, et être baptisés du baptême dont je dois être baptisé ? Ils lui répondirent : Nous le pouvons.
\VS{23}Et il leur dit : Il est vrai que vous boirez ma coupe et que vous serez baptisés du baptême dont je serai baptisé ; mais pour ce qui est d'être assis à ma droite ou à ma gauche, cela ne dépend pas de moi, et ne sera donné qu'à ceux à qui mon Père l'a réservé.
\VS{24}Les dix autres disciples ayant entendu cela, furent indignés contre les deux frères.
\VS{25}Mais Jésus les appela et leur dit : Vous savez que les princes des nations les dominent, et que les grands les asservissent.
\VS{26}Mais il n'en sera pas ainsi entre vous. Au contraire, quiconque veut être grand entre vous, qu'il soit votre serviteur.
\VS{27}Et quiconque veut être le premier parmi vous, qu'il soit votre serviteur.
\VS{28}De même que le Fils de l'homme n'est pas venu pour être servi, mais pour servir, et afin de donner sa vie en rançon pour plusieurs.
\TextTitle{Jésus guérit deux aveugles\FTNTT{Mc. 10:46-53 ; Lu. 18:35-43}}
\VS{29}Et comme ils partaient de Jéricho, une grande foule le suivit.
\VS{30}Et voici, deux aveugles qui étaient assis au bord du chemin, entendirent que Jésus passait, et crièrent en disant : Seigneur, Fils de David ! Aie pitié de nous !
\VS{31}Et la foule les reprenait pour les faire taire ; mais ils criaient encore plus fort : Seigneur, Fils de David ! Aie pitié de nous !
\VS{32}Jésus s'arrêta les appela et leur dit : Que voulez-vous que je vous fasse ?
\VS{33}Ils lui dirent : Seigneur, que nos yeux soient ouverts.
\VS{34}Et Jésus étant ému de compassion, toucha leurs yeux, et aussitôt ils recouvrèrent la vue et ils le suivirent.
\Chap{21}
\TextTitle{Jésus-Christ se présente publiquement comme Roi\FTNTT{Za. 9:9 ; Mc. 11:1-11 ; Lu. 19:28-40 ; Jn. 12:12-19.}}
\VerseOne{}Et quand ils furent près de Jérusalem, et qu'ils furent arrivés à Bethphagé vers le Mont des Oliviers, Jésus envoya alors deux disciples,
\VS{2}en leur disant : Allez au village qui est devant vous. Vous trouverez une ânesse attachée, et son ânon avec elle. Détachez-les et amenez-les-moi.
\VS{3}Et si quelqu'un vous dit quelque chose, vous direz que le Seigneur en a besoin ; et aussitôt il les laissera aller.
\VS{4}Or, tout cela arriva afin que s'accomplisse ce qui avait été annoncé par le prophète, en disant :
\VS{5}Dites à la fille de Sion : Voici, ton Roi vient à toi, plein de douceur, et monté sur un âne, sur un ânon, le petit d'une ânesse\FTNT{Za. 9:9.}.
\VS{6}Les disciples donc s'en allèrent et firent ce que Jésus leur avait ordonné.
\VS{7}Et ils amenèrent l'ânesse et l'ânon, et mirent leurs vêtements sur eux, et le firent asseoir dessus.
\VS{8}Alors de grandes foules étendirent leurs vêtements sur le chemin, et les autres coupaient des rameaux des arbres, et les étendaient sur le chemin.
\VS{9}Et les foules qui allaient devant, et celles qui suivaient, criaient en disant : Hosanna au Fils de David ! Béni soit celui qui vient au Nom du Seigneur ! Hosanna dans les lieux très hauts !
\VS{10}Lorsqu'il entra dans Jérusalem, toute la ville fut émue et l'on disait : Qui est celui-ci ?
\VS{11}Et les foules disaient : C'est Jésus, le prophète de Nazareth en Galilée.
\TextTitle{Jésus chasse les marchands du temple\FTNTT{Mc. 11:15-18 ; Lu. 19:45-46 ; Jn. 2:13-16}}
\VS{12}Jésus entra dans le temple de Dieu. Il chassa dehors tous ceux qui vendaient et qui achetaient dans le temple ; il renversa les tables des changeurs et les sièges de ceux qui vendaient des pigeons ;
\VS{13}et il leur dit : Il est écrit : Ma maison sera appelée une maison de prière, mais vous en avez fait une caverne de voleurs\FTNT{Es. 56:7 ; Jé. 7:11.}.
\VS{14}Alors des aveugles et des boiteux s'approchèrent de lui dans le temple et il les guérit.
\VS{15}Mais les principaux sacrificateurs et les scribes furent indignés à la vue des choses merveilleuses qu'il avait faites, et des enfants qui criaient dans le temple : Hosanna au Fils de David !
\VS{16}Et ils lui dirent : Entends-tu ce qu'ils disent ? Oui, leur répondit Jésus. N'avez-vous jamais lu ces paroles : Tu as tiré des louanges de la bouche des enfants, et de ceux qui sont à la mamelle\FTNT{Ps. 8:3.} ?
\VS{17}Et, les ayant laissés, il sortit de la ville, pour aller à Béthanie, où il passa la nuit.
\TextTitle{Le figuier stérile\FTNTT{Mc. 11:12-14,20-26}}
\VS{18}Le matin, comme il retournait à la ville, il eut faim.
\VS{19}Et voyant un figuier qui était sur le chemin, il s'en approcha, mais il n'y trouva que des feuilles ; et il lui dit : Qu'aucun fruit ne naisse plus jamais de toi ! Et aussitôt le figuier sécha.
\VS{20}Les disciples qui virent cela furent étonnés et dirent : Comment ce figuier est-il devenu sec en un instant ?
\VS{21}Jésus leur répondit : Je vous le dis en vérité, si vous aviez la foi, et que vous ne doutiez point, non seulement vous ferez ce qui a été fait à ce figuier, mais quand vous diriez à cette montagne : Ôte-toi de là et jette-toi dans la mer, cela se ferait.
\VS{22}Et quoi que vous demandiez en priant Dieu si vous croyez, vous le recevrez.
\TextTitle{L'incrédulité des principaux sacrificateurs et des anciens\FTNTT{Mc. 11:27-33 ; Lu. 20:1-8}}
\VS{23}Puis, s'étant rendu dans le temple, les principaux sacrificateurs et les anciens du peuple vinrent auprès de lui, pendant qu'il enseignait, et lui dirent : Par quelle autorité fais-tu ces choses ; et qui t'a donné cette autorité ?
\VS{24}Jésus répondant leur dit : Je vous interrogerai aussi sur une chose, et si vous me répondez, je vous dirai par quelle autorité je fais ces choses.
\VS{25}Le baptême de Jean d'où venait-il ? Du ciel ou des hommes ? Mais ils raisonnèrent ainsi entre eux : Si nous disons : Du ciel, il nous dira : Pourquoi n'avez-vous pas cru en lui ?
\VS{26}Et si nous disons : Des hommes, nous craignons la foule, car tous tiennent Jean pour un prophète.
\VS{27}Alors ils répondirent à Jésus : Nous ne savons pas. Et il leur dit : Moi non plus, je ne vous dirai pas par quelle autorité je fais ces choses.
\TextTitle{Parabole des deux fils}
\VS{28}Mais que vous en semble ? Un homme avait deux fils ; et s'adressant au premier, il lui dit : Mon fils, va travailler aujourd'hui dans ma vigne.
\VS{29}Il répondit : Je ne veux pas y aller. Ensuite il se repentit et y alla.
\VS{30}S'adressant à l'autre, il lui dit la même chose. Et ce fils répondit : Je veux bien, seigneur. Et il n'alla pas.
\VS{31}Lequel des deux a fait la volonté du père ? Ils lui répondirent : Le premier. Et Jésus leur dit : Je vous le dis en vérité, les publicains et les prostituées vous devanceront dans le Royaume de Dieu.
\VS{32}Car Jean est venu à vous dans la voie de la justice et vous ne l'avez pas cru ; mais les publicains et les femmes débauchées ont cru en lui. Et vous, qui avez vu cela, vous ne vous êtes pas ensuite repentis pour croire en lui.
\TextTitle{Parabole des vignerons\FTNTT{Es. 5:1-7 ; Mc. 12:1-12 ; Lu. 20:9-18.}}
\VS{33}Ecoutez une autre parabole : Il y avait un père de famille qui planta une vigne, et l'entoura d'une haie, et y creusa un pressoir, et bâtit une tour ; puis il l'afferma à des vignerons, et quitta le pays.
\VS{34}Lorsque la saison de la récolte fut arrivé, il envoya ses serviteurs vers les vignerons pour recevoir les fruits.
\VS{35}Mais les vignerons s'étant saisis de ses serviteurs, fouettèrent l'un, tuèrent l'autre et lapidèrent le troisième.
\VS{36}Il envoya encore d'autres serviteurs en plus grand nombre que les premiers, et ils leur firent de même.
\VS{37}Enfin, il envoya vers eux son propre fils, en disant : Ils auront du respect pour mon fils.
\VS{38}Mais quand les vignerons virent le fils, ils dirent entre eux : Voici l'héritier. Venez, tuons-le et emparons-nous de son héritage.
\VS{39}Et s'étant saisis de lui, il le jetèrent hors de la vigne et le tuèrent.
\VS{40}Quand donc le maître de la vigne viendra, que fera-t-il à ces vignerons ?
\VS{41}Ils lui dirent : Il les fera périr malheureusement comme des méchants et louera sa vigne à d'autres vignerons, qui lui en rendront les fruits en leur saison.
\VS{42}Et Jésus leur dit : N'avez-vous jamais lu dans les Ecritures : La pierre qu'ont rejetée ceux qui bâtissaient, est devenue la principale de l'angle. C'est du Seigneur que cela est venu, et c'est un prodige à nos yeux\FTNT{Es. 8:13-17 ; Es. 28:16.} ?
\VS{43}C'est pourquoi je vous dis que le Royaume de Dieu vous sera enlevé et il sera donné à une nation qui en rendra les fruits.
\VS{44}Celui qui tombera sur cette pierre s'y brisera, et celui sur qui elle tombera sera écrasé.
\VS{45}Après avoir entendu ses paraboles, les principaux sacrificateurs et les pharisiens comprirent qu'il parlait d'eux.
\VS{46}Et ils cherchaient à se saisir de lui, mais ils craignaient la foule, parce qu'elle le tenait pour un prophète.
\Chap{22}
\TextTitle{Parabole des noces\FTNTT{Lu. 14:16-24}}
\VerseOne{}Alors Jésus, prenant la parole, leur parla de nouveau en paraboles et il dit :
\VS{2}Le Royaume des cieux est semblable à un roi qui fit des noces pour son fils.
\VS{3}Il envoya ses serviteurs pour appeler ceux qui avaient été conviés aux noces ; mais ils ne voulurent pas venir.
\VS{4}Il envoya encore d'autres serviteurs, disant : Dites aux conviés : Voici, j'ai préparé mon festin ; mes bœufs et mes bêtes grasses sont tués, et tout est prêt ; venez aux noces.
\VS{5}Mais, sans tenir compte de l'invitation, ils s'en allèrent l'un à son champ, et l'autre à son trafic.
\VS{6}Et les autres se saisirent de ses serviteurs les outragèrent, et les tuèrent.
\VS{7}Quand le roi l'entendit, il se mit en colère ; il envoya ses troupes, fit périr ces meurtriers et brûla leur ville.
\VS{8}Puis il dit à ses serviteurs : Les noces sont prêtes, mais les conviés n'en étaient pas dignes.
\VS{9}Allez donc dans les carrefours des chemins, et autant de gens que vous trouverez, appelez-les aux noces.
\VS{10}Alors ces serviteurs allèrent dans les chemins et rassemblèrent tous ceux qu'ils trouvèrent, méchants et bons, et la salle des noces fut remplie de conviés qui étaient à table.
\VS{11}Et le roi étant entré pour voir ceux qui étaient à table, il aperçut là un homme qui n'avait pas revêtu un habit de noces\FTNT{Ap. 19:7-8.}.
\VS{12}Et il lui dit : Mon ami, comment es-tu entré ici sans avoir un habit de noces ? Cet homme eut la bouche fermée.
\VS{13}Alors le roi dit aux serviteurs : Liez-lui les pieds et les mains, emportez-le et jetez-le dans les ténèbres de dehors, où il y aura des pleurs et des grincements de dents.
\VS{14}Car il y a beaucoup d'appelés, mais peu d'élus.
\TextTitle{Le tribut dû à César\FTNTT{Mc. 12:13-17 ; Lu. 20:19-26}}
\VS{15}Alors les pharisiens allèrent se consulter ensemble sur les moyens de le surprendre par ses propres paroles.
\VS{16}Ils envoyèrent auprès de lui leurs disciples, avec des hérodiens, qui dirent : Maître, nous savons que tu es véritable, que tu enseignes la voie de Dieu selon la vérité, sans t'inquiéter de personne ; car tu ne regardes point à l'apparence des hommes.
\VS{17}Dis-nous donc ce qu'il t'en semble : Est-il permis de payer le tribut à César, ou non ?
\VS{18}Et Jésus connaissant leur malice, dit : Hypocrites, pourquoi me tentez-vous ?
\VS{19}Montrez-moi la monnaie avec laquelle on paie le tribut ; et ils lui présentèrent un denier.
\VS{20}Il leur demanda : De qui porte-t-il l'image et l'inscription ?
\VS{21}De César, lui répondirent-ils. Alors il leur dit : Rendez donc à César ce qui est à César, et à Dieu, ce qui est à Dieu.
\VS{22}Et ayant entendu cela, ils furent étonnés, ils le quittèrent et s'en allèrent.
\TextTitle{Enseignement de Jésus sur la résurrection\FTNTT{Mc. 12:18-27 ; Lu. 20:27-38}}
\VS{23}Le même jour, les sadducéens, qui disent qu'il n'y a pas de résurrection, vinrent auprès de lui et lui posèrent cette question,
\VS{24}en disant : Maître, Moïse a dit : Si quelqu'un meurt sans enfants, son frère épousera sa femme et suscitera une postérité à son frère.
\VS{25}Or, il y avait parmi nous sept frères. Le premier se maria et mourut ; et, n'ayant pas eu d'enfants, il laissa sa femme à son frère.
\VS{26}Il en fut de même du deuxième, puis du troisième, jusqu'au septième.
\VS{27}Après eux tous, la femme mourut aussi.
\VS{28}A la résurrection, duquel des sept sera-t-elle la femme ? Car tous l'ont eue.
\VS{29}Mais Jésus répondant leur dit : Vous êtes dans l'erreur, parce que vous ne connaissez ni les Ecritures, ni la puissance de Dieu.
\VS{30}Car à la résurrection on ne prendra ni on ne donnera de femmes en mariage, mais on sera comme les anges de Dieu dans le ciel.
\VS{31}Et quant à la résurrection des morts, n'avez-vous point lu ce dont Dieu vous a parlé, disant :
\VS{32}Je suis le Dieu d'Abraham, le Dieu d'Isaac et le Dieu de Jacob\FTNT{Ge. 17:7 ; Ge. 26:24 ; Ge. 28:21.}. Or Dieu n'est pas le Dieu des morts, mais des vivants.
\VS{33}Ce que les foules ayant entendu, elles admirèrent sa doctrine.
\TextTitle{Le plus grand commandement de la loi\FTNTT{Mc. 12:28-34 ; Lu. 10:25-28}}
\VS{34}Quand les pharisiens apprirent qu'il avait fermé la bouche aux sadducéens, ils se rassemblèrent dans un même lieu,
\VS{35}et l'un d'eux, qui était docteur de la loi, l'interrogea pour l'éprouver, en disant :
\VS{36}Maître, quel est le plus grand commandement de la loi ?
\VS{37}Jésus lui dit : Tu aimeras le Seigneur ton Dieu de tout ton cœur, de toute ton âme et de toute ta pensée.
\VS{38}Celui-ci est le premier et le grand commandement.
\VS{39}Et voici le deuxième qui lui est semblable : Tu aimeras ton prochain comme toi-même.
\VS{40}De ces deux commandements dépendent toute la loi et les prophètes.
\TextTitle{Jésus interroge les pharisiens au sujet du Messie\FTNTT{Mc. 12:35-37 ; Lu. 20:39-44}}
\VS{41}Et les Pharisiens étant assemblés, Jésus les interrogea,
\VS{42}Disant : que pensez-vous du Christ ? De qui est-il Fils ? Ils lui répondirent : de David.
\VS{43}Et il leur dit : Comment donc David, parlant par l'Esprit, l'appelle-t-il son Seigneur ? Disant :
\VS{44}Le Seigneur a dit à mon Seigneur, assieds-toi à ma droite, jusqu'à ce que j'aie mis tes ennemis pour le marchepied de tes pieds\FTNT{Ps. 110:1.}.
\VS{45}Si donc David l'appelle son Seigneur, comment est-il son Fils ?
\VS{46}Et personne ne pouvait lui répondre un seul mot. Et depuis ce jour, personne n'osa plus lui poser des questions.
\Chap{23}
\TextTitle{Caractéristiques des scribes et des pharisiens\FTNTT{Mc. 12:38-40 ; Lu. 11:39-54 ; Lu. 20:45-47}}
\VerseOne{}Alors Jésus parla à la foule et à ses disciples,
\VS{2}disant : Les scribes et les pharisiens sont assis dans la chaire de Moïse.
\VS{3}Toutes les choses donc qu'ils vous diront d'observer, observez-les et faites-les, mais non point leurs œuvres : parce qu'ils disent et ne font pas.
\VS{4}Car ils lient ensemble des fardeaux pesants et insupportables et les mettent sur les épaules des hommes ; mais ils ne veulent point les remuer de leur doigt.
\VS{5}Et ils font toutes leurs œuvres pour être vus des hommes. Ainsi, ils portent de larges phylactères et de longues franges à leurs vêtements.
\VS{6}Ils aiment les premières places dans les festins, et les premiers sièges dans les synagogues.
\VS{7}Ils aiment les salutations dans les places publiques, et à être appelés par les hommes : Notre maître ! Notre maître !
\VS{8}Mais vous, ne vous faites pas appeler, Notre maître ; car Christ seul est votre Docteur ; et vous êtes tous frères.
\VS{9}Et n'appelez personne sur la terre votre père ; car un seul est votre Père, celui qui est dans les cieux.
\VS{10}Et ne soyez point appelés Docteurs : car Christ seul est votre Docteur.
\VS{11}Mais que celui qui est le plus grand entre vous, soit votre serviteur.
\VS{12}Car quiconque s'élèvera sera abaissé ; et quiconque s'abaissera, sera élevé.
\VS{13}Mais malheur à vous, scribes et pharisiens hypocrites, qui fermez le Royaume des cieux aux hommes : car vous-mêmes n'y entrez point, et vous n'y laissez pas entrer ceux qui veulent y entrer.
\VS{14}Malheur à vous, scribes et pharisiens hypocrites, car vous dévorez les maisons des veuves, même sous le prétexte de faire de longues prières, c'est pourquoi vous en recevrez une plus grande condamnation.
\VS{15}Malheur à vous, scribes et pharisiens hypocrites ! Parce que vous courez la mer et la terre pour faire un prosélyte, et quand il l'est devenu, vous le rendez fils de la géhenne, deux fois plus que vous.
\VS{16}Malheur à vous conducteurs aveugles, qui dites : Si quelqu'un jure par le temple, ce n'est rien ; mais si quelqu'un jure par l'or du temple, il est engagé.
\VS{17}Insensés et aveugles ! Car lequel est le plus grand, l'or, ou le temple qui sanctifie l'or ?
\VS{18}Si quelqu'un, dites-vous encore, jure par l'autel, ce n'est rien ; mais si quelqu'un jure par l'offrande qui est sur l'autel, il est engagé.
\VS{19}Insensés et aveugles ! Car lequel est le plus grand, l'offrande, ou l'autel qui sanctifie l'offrande ?
\VS{20}Celui donc qui jure par l'autel, jure par l'autel et par toutes les choses qui sont dessus.
\VS{21}Celui qui jure par le temple, jure par le temple et par celui qui y habite ;
\VS{22}et celui qui jure par le ciel, jure par le trône de Dieu et par celui qui y est assis.
\VS{23}Malheur à vous, scribes et pharisiens hypocrites ! Parce que vous payez la dîme\FTNT{Voir commentaire en Mal. 3:10.} de la menthe, de l'aneth et du cumin ; et vous laissez les choses les plus importantes de la loi, c'est-à-dire la justice, la miséricorde et la fidélité. Il fallait pratiquer ces choses-là, sans négliger les autres choses.
\VS{24}Conducteurs aveugles ! Vous coulez le moucheron et vous engloutissez le chameau\FTNT{Les pharisiens filtraient leur eau par crainte d'avaler un moucheron.}.
\VS{25}Malheur à vous, scribes et pharisiens hypocrites ! Parce que vous nettoyez le dehors de la coupe et du plat ; alors qu'au-dedans ils sont pleins de rapine et d'intempérance.
\VS{26}Pharisien aveugle, nettoie premièrement l'intérieur de la coupe et du plat, afin que l'extérieur aussi devienne net.
\VS{27}Malheur à vous, scribes et pharisiens hypocrites ! Parce que vous êtes semblables aux sépulcres blanchis, qui paraissent beaux au-dehors, et qui au-dedans sont pleins d'ossements de morts, et de toutes espèces d'impuretés.
\VS{28}Ainsi, au-dehors vous paraissez justes aux hommes, mais au-dedans vous êtes pleins d'hypocrisie et d'iniquité.
\VS{29}Malheur à vous, scribes et pharisiens hypocrites ! Parce que vous bâtissez les tombeaux des prophètes et vous ornez les sépulcres des justes ;
\VS{30}et vous dites : Si nous avions vécu du temps de nos pères, nous n'aurions pas participé avec eux au meurtre des prophètes.
\VS{31}Ainsi vous êtes témoins contre vous-mêmes, que vous êtes les enfants de ceux qui ont fait mourir les prophètes.
\VS{32}Et vous achevez de remplir la mesure de vos pères.
\VS{33}Serpents, race de vipères ! Comment éviterez-vous le supplice de la géhenne ?
\VS{34}Car voici, je vous envoie des prophètes, des sages et des scribes. Vous tuerez et crucifierez les uns, vous battrez de verges les autres dans vos synagogues, et vous les persécuterez de ville en ville,
\VS{35}afin que vienne sur vous tout le sang innocent qui a été répandu sur la terre, depuis le sang d'Abel le juste, jusqu'au sang de Zacharie, fils de Barachie, que vous avez tué entre le temple et l'autel.
\VS{36}Je vous le dis en vérité, que toutes ces choses viendront sur cette génération.
\TextTitle{Lamentations de Jésus sur Jérusalem\FTNTT{Jé. 22:5 ; Lu. 13:34-35 ; 19:41-44.}}
\VS{37}Jérusalem, Jérusalem, qui tues les prophètes, et qui lapides ceux qui te sont envoyés, combien de fois ai-je voulu rassembler tes enfants, comme la poule rassemble ses poussins sous ses ailes, et vous ne l'avez point voulu !
\VS{38}Voici, votre maison va devenir déserte.
\VS{39}Car je vous dis, que désormais vous ne me verrez plus, jusqu'à ce que vous disiez : Béni soit celui qui vient au Nom du Seigneur\FTNT{Ps. 118:26.}!
\Chap{24}
\TextTitle{Prophétie sur la destruction du temple de Jérusalem\FTNTT{Mc. 13:1-2 ; Lu. 21:5-6}}
\VerseOne{}Comme Jésus sortait et s'en allait du temple, ses disciples s'approchèrent de lui pour lui faire remarquer les bâtiments du temple.
\VS{2}Mais Jésus leur dit : Voyez-vous bien toutes ces choses ? Je vous le dis en vérité, il ne restera pas ici pierre sur pierre qui ne soit démolie.
\TextTitle{Le signe de l'accomplissement\FTNTT{Mc. 13:3-4 ; Lu. 21:7}}
\VS{3}Puis s'étant assis sur la Montagne des Oliviers, ses disciples vinrent à lui en particulier et lui dirent : Dis-nous quand ces choses arriveront, et quel sera le signe de ton avènement, et de la fin du monde ?
\TextTitle{Les temps de la fin\FTNTT{Da. 9:27 ; Mc. 13:5-13 ; Lu. 21:8-11}}
\VS{4}Et Jésus répondant leur dit : Prenez garde que personne ne vous séduise.
\VS{5}Car plusieurs viendront sous mon Nom, disant : Je suis le Christ. Et ils en séduiront plusieurs.
\VS{6}Vous entendrez parler de guerres et de bruits de guerres ; gardez-vous d'être troublés ; car il faut que toutes ces choses arrivent ; mais ce ne sera pas encore la fin.
\VS{7}Car une nation s'élèvera contre une autre nation, et un royaume contre un autre royaume ; et il y aura des famines, des pestes, et des tremblements de terre en divers lieux.
\VS{8}Mais toutes ces choses ne seront que le commencement des douleurs.
\VS{9}Alors ils vous livreront aux tourments, et vous tueront ; et vous serez haïs de toutes les nations, à cause de mon Nom.
\VS{10}Alors aussi plusieurs seront scandalisés, se trahiront et se haïront les uns les autres.
\VS{11}Et il s'élèvera plusieurs faux prophètes, qui en séduiront plusieurs.
\VS{12}Et parce que l'iniquité sera multipliée, la charité de plusieurs se refroidira.
\VS{13}Mais celui qui persévérera jusqu'à la fin, sera sauvé.
\VS{14}Cet Evangile du Royaume sera prêché dans toute la terre habitable, pour servir de témoignage à toutes les nations, et alors viendra la fin.
\TextTitle{L'abomination de la désolation\FTNTT{Da. 9:27 ; Da.11:32-35 ; Mc. 13:14-18 ; Lu. 21:20-23}}
\VS{15}Or quand vous verrez l'abomination qui causera la désolation, qui a été prédite par Daniel le Prophète\FTNT{Daniel fut le premier à parler de l'abomination de la désolation (Da. 9:24-27). « Et les forces se présenteront de sa part, elles profaneront le sanctuaire, la forteresse, elles feront cesser le sacrifice perpétuel, et dresseront l'abomination qui causera la désolation. » Da. 11:31. Cette prophétie s'est partiellement accomplie en 168 av. J.-C. lorsqu'Antiochus Epiphane (215 av. J.-C. - 163 av. J.-C.), roi de Syrie, défenseur zélé de la culture grecque, finança la construction du temple de Zeus à Athènes. Sa tentative d'hellénisation forcée de la Judée, soutenue par les grands prêtres Jason et Ménélas, provoqua la colère des Juifs traditionalistes. Antiochus avait interdit le culte mosaïque et consacra le temple de Jérusalem aux dieux grecs. En effet, il le pilla et y installa un autel du dieu Baal Shamen, puis il détruisit les murailles de la ville. Dans un édit de décembre 167 av. J.-C., il ordonna d'offrir des porcs en holocauste, interdit la circoncision, la lecture de la Torah, l'observance des fêtes de Yahweh et pourchassa les adversaires de l'hellénisation. En agissant de la sorte, il y avait deux choses principales que cet archétype de l'antichrist espérait changer en Israël : Les temps (le calendrier juif) et la Torah (la loi selon Da. 7:25). La deuxième partie de cette prophétie s'est accomplie en l'an 70 lors de la destruction du temple de Jérusalem par Titus (39-81 ap. J.-C.). La dernière partie de cette prophétie est en train de s'accomplir actuellement dans les assemblées où Satan distille des enseignements erronés au travers des faux prophètes. La prédication d'un évangile expurgé de son caractère christocentrique, de la nécessité de porter sa croix, axé sur les choses de ce monde, maintient les chrétiens dans une vie de péché. Ainsi, alors qu'ils sont censés être des temples vivants du Saint-Esprit (1 Co. 6:19), Satan s'est établi dans leurs cœurs. Enfin, la prophétie de Daniel trouvera son parfait accomplissement pendant le règne de la bête. L'homme impie s'introduira alors dans le temple de Jérusalem qui sera rebâti, se fera passer pour Dieu et se fera adorer à sa place (2 Th. 2:4).}, être établie dans le lieu saint, que celui qui lit ce prophète y fasse attention !
\VS{16}Alors, que ceux qui seront en Judée fuient dans les montagnes ;
\VS{17}et que celui qui sera sur le toit ne descende pas pour emporter quoi que ce soit de sa maison ;
\VS{18}que celui qui sera dans les champs ne retourne pas en arrière pour prendre ses habits.
\VS{19}Malheur aux femmes enceintes, et à celles qui allaiteront en ces jours-là.
\VS{20}Priez pour que votre fuite n'arrive pas en hiver, ni un jour de sabbat\FTNT{Sous la loi mosaïque, il était interdit aux juifs de parcourir plus de 2 000 coudées du lieu où ils se trouvaient pendant le sabbat (Ex. 16:29).}.
\TextTitle{La grande tribulation\FTNTT{Ps. 2:5 ; Jé. 30:5-8 ; Da.12:1 ; Mc. 13:19-23 ; Lu. 21:23-24}}
\VS{21}Car alors, la détresse sera si grande qu'il n'y en a point eu de semblable depuis le commencement du monde jusqu'à présent, et qu'il n'y en aura jamais.
\VS{22}Et si ces jours n'étaient abrégés, personne ne serait sauvé ; mais à cause des élus, ces jours seront abrégés.
\VS{23}Alors si quelqu'un vous dit : Voici, le Christ est ici ; ou, il est là ; ne le croyez point.
\VS{24}Car il s'élèvera de faux christs et de faux prophètes, ils feront de grands prodiges et des miracles, pour séduire même les élus, s'il était possible.
\VS{25}Voici, je vous l'ai prédit.
\VS{26}Si on vous dit : Voici, il est dans le désert, ne sortez point ; voici, il est dans les chambres, ne le croyez point.
\VS{27}Car, comme l'éclair part de l'orient et se montre jusqu'en occident, il en sera de même de l'avènement du Fils de l'homme.
\VS{28}Car là où est le cadavre, là s'assembleront les vautours.
\TextTitle{Retour du Roi sur la terre\FTNTT{Mc. 13:24-27 ; Lu. 21:25-28}}
\VS{29}Aussitôt après ces jours de détresse, le soleil s'obscurcira, la lune ne donnera plus sa lumière, et les étoiles tomberont du ciel, et les puissances des cieux seront ébranlées.
\VS{30}Alors le signe du Fils de l'homme paraîtra dans le ciel, toutes les tribus de la terre se lamenteront en se frappant la poitrine, et verront le Fils de l'homme venant sur les nuées du ciel, avec une grande puissance et une grande gloire.
\VS{31}Il enverra ses anges avec un grand son de trompette et ils rassembleront ses élus, des quatre vents, d'une extrémité des cieux à l'autre.
\TextTitle{Parabole du figuier\FTNTT{Mc. 13:28-31 ; Lu. 21:29-33}}
\VS{32}Mais apprenez la leçon tirée de la parabole du figuier. Dès que ses jeunes branches deviennent tendres et que ses feuilles poussent, vous savez que l'été est proche.
\VS{33}De même, quand vous verrez toutes ces choses, sachez que le Fils de l'homme est proche, à la porte.
\VS{34}Je vous le dis en vérité, cette génération ne passera point, jusqu'à ce que tout cela n'arrive.
\VS{35}Le ciel et la terre passeront, mais mes paroles ne passeront point.
\TextTitle{Exhortation à la vigilance\FTNTT{Mc. 13:32-37 ; Lu. 21:34-38}}
\VS{36}Pour ce qui est du jour et de l'heure, personne ne le sait, ni les anges des cieux, mais mon Père seul.
\VS{37}Mais comme il en était aux jours de Noé, il en sera de même de l'avènement du fils de l'homme.
\VS{38}Car, comme dans les jours avant le déluge, les hommes mangeaient et buvaient, se mariaient, et donnaient en mariage, jusqu'au jour où Noé entra dans l'arche ;
\VS{39}et ils ne connurent point que le déluge viendrait, jusqu'à ce qu'il vint et les emporta tous ; il en sera de même de l'avènement du Fils de l'homme.
\VS{40}Alors, de deux hommes qui seront dans un champ ; l'un sera pris, et l'autre laissé ;
\VS{41}de deux femmes qui moudront au moulin, l'une sera prise et l'autre laissée.
\VS{42}Veillez donc, car vous ne savez point à quelle heure votre Seigneur doit venir.
\VS{43}Mais sachez ceci, que si un père de famille savait à quelle veille de la nuit le voleur doit venir, il veillerait et ne laisserait pas percer sa maison.
\VS{44}C'est pourquoi, vous aussi tenez-vous prêts ; car le Fils de l'homme viendra à l'heure où vous n'y penserez pas.
\VS{45}Quel est donc le serviteur fidèle et prudent, que son maître a établi sur tous ses serviteurs, pour leur donner la nourriture au temps opportun ?
\VS{46}Heureux est ce serviteur que son maître en arrivant trouvera agir de cette manière.
\VS{47}Je vous le dis en vérité, il l'établira sur tous ses biens.
\VS{48}Mais si c'est un méchant serviteur, qui dit en lui-même : Mon maître tarde à venir ;
\VS{49}et s'il se met à battre ses compagnons de service, s'il mange et boit avec les ivrognes,
\VS{50}le maître de ce serviteur viendra le jour où il ne s'y attend pas et à l'heure qu'il ne connaît pas.
\VS{51}Et il le séparera, et le mettra au rang des hypocrites ; là il y aura des pleurs et des grincements de dents.
\Chap{25}
\TextTitle{Parabole des dix vierges}
\VerseOne{}Alors le Royaume des cieux sera semblable à dix vierges qui, ayant pris leurs lampes, allèrent à la rencontre de l'époux.
\VS{2}Or il y en avait cinq sages et cinq folles.
\VS{3}Les folles, en prenant leurs lampes, ne prirent pas d'huile avec elles ;
\VS{4}mais les sages prirent de l'huile dans leurs vases avec leurs lampes.
\VS{5}Et comme l'époux tardait à venir, elles s'assoupirent et s'endormirent toutes.
\VS{6}Or à minuit il se fit un cri disant : Voici, l'époux vient, allez à sa rencontre !
\VS{7}Alors toutes ces vierges se réveillèrent\FTNT{Réveiller : Du grec « egeiro ». Ce terme signifie également ressusciter. Les saints qui attendent le retour du Seigneur connaîtront un réveil après un temps de sommeil spirituel (Ro. 13:11).} et préparèrent leurs lampes.
\VS{8}Et les folles dirent aux sages : Donnez-nous de votre huile, car nos lampes s'éteignent.
\VS{9}Mais les sages répondirent en disant : Nous ne pouvons pas vous en donner, de peur que nous n'en ayons pas assez pour nous et pour vous ; mais allez plutôt chez ceux qui en vendent et achetez-en pour vous-mêmes.
\VS{10}Or pendant qu'elles allaient en acheter, l'époux arriva. Celles qui étaient prêtes entrèrent avec lui dans la salle des noces, puis la porte fut fermée.
\VS{11}Après cela, les autres vierges vinrent aussi, et dirent : Seigneur ! Seigneur ! Ouvre-nous !
\VS{12}Mais il leur répondit, et dit : Je vous le dis en vérité, je ne vous connais point.
\VS{13}Veillez donc ; car vous ne savez ni le jour ni l'heure en laquelle le Fils de l'homme viendra.
\TextTitle{Parabole des talents}
\VS{14}Car il en sera comme d'un homme qui, partant pour un voyage, appela ses serviteurs et leur remit ses biens.
\VS{15}Il donna à l'un cinq talents, à l'autre deux, et au troisième un ; à chacun selon sa capacité ; et aussitôt après il partit.
\VS{16}Celui qui avait reçu les cinq talents, s'en alla, et les fit valoir, et gagna cinq autres talents.
\VS{17}De même, celui qui avait reçu les deux talents, en gagna aussi deux autres.
\VS{18}Mais celui qui n'en avait reçu qu'un, alla et creusa dans la terre, et y cacha l'argent de son maître.
\VS{19}Longtemps après, le maître de ces serviteurs revint et leur fit rendre compte.
\VS{20}Alors celui qui avait reçu les cinq talents, vint et présenta cinq autres talents, en disant : Seigneur, tu m'as confié cinq talents, voici, j'en ai gagné cinq autres par-dessus.
\VS{21}Et son Seigneur lui dit : Cela est bien, bon et fidèle serviteur ; tu as été fidèle en peu de choses, je t'établirai sur beaucoup ; viens participer à la joie de ton Seigneur.
\VS{22}Ensuite, celui qui avait reçu les deux talents, vint et dit : Seigneur, tu m'as confié deux talents ; voici, j'en ai gagné deux autres par-dessus.
\VS{23}Et son Seigneur lui dit : Cela est bien, bon et fidèle serviteur, tu as été fidèle en peu de choses, je t'établirai sur beaucoup ; viens prendre part à la joie de ton Seigneur.
\VS{24}Mais celui qui n'avait reçu qu'un talent, vint et dit : Seigneur, je savais que tu es un homme dur, qui moissonnes où tu n'as point semé, et qui amasses où tu n'as point vanné,
\VS{25}c'est pourquoi craignant de perdre ton talent, je suis allé le cacher dans la terre. Voici, tu as ici ce qui t'appartient.
\VS{26}Et son Seigneur répondant, lui dit : Méchant et lâche serviteur, tu savais que je moissonnais où je n'ai point semé, et que j'amassais où je n'ai point vanné,
\VS{27}il te fallait donc remettre mon argent aux banquiers et à mon retour, je l'aurais retiré avec l'intérêt.
\VS{28}Ôtez-lui donc le talent et donnez-le à celui qui a les dix talents.
\VS{29}Car à celui qui a, il sera donné et il en aura encore plus, mais à celui qui n'a rien, cela même qu'il a, lui sera ôté.
\VS{30}Jetez donc le serviteur inutile dans les ténèbres de dehors ; où il y aura des pleurs et des grincements de dents.
\TextTitle{Séparation et jugement des brebis et des boucs\FTNTT{1 Co. 6:2}}
\VS{31}Or quand le Fils de l'homme viendra environné de sa gloire et accompagné de tous les saints anges, alors il s'assiéra sur le trône de sa gloire.
\VS{32}Et toutes les nations seront assemblées devant lui ; et il séparera les uns d'avec les autres, comme le berger sépare les brebis d'avec les boucs.
\VS{33}Et il mettra les brebis à sa droite et les boucs à sa gauche.
\VS{34}Alors le Roi dira à ceux qui seront à sa droite : Venez, vous qui êtes bénis de mon Père, possédez en héritage le Royaume qui vous a été préparé dès la fondation du monde.
\VS{35}Car j'ai eu faim et vous m'avez donné à manger ; j'ai eu soif et vous m'avez donné à boire ; j'étais étranger et vous m'avez recueilli ;
\VS{36}j'étais nu et vous m'avez vêtu ; j'étais malade et vous m'avez visité ; j'étais en prison et vous êtes venus vers moi.
\VS{37}Alors les justes lui répondront : Seigneur, quand t'avons-nous vu avoir faim et t'avons-nous donné à manger ; ou avoir soif et t'avons-nous donné à boire ?
\VS{38}Quand t'avons-nous vu étranger et t'avons-nous recueilli ; ou nu, et t'avons-nous vêtu ?
\VS{39}Ou quand t'avons-nous vu malade, ou en prison, et sommes-nous allés vers toi ?
\VS{40}Et le Roi répondant, leur dira : Je vous le dis en vérité, toutes les fois que vous avez fait ces choses à l'un de ces plus petits de mes frères, c'est à moi que vous les avez faites.
\VS{41}Alors il dira aussi à ceux qui seront à sa gauche : Maudits, retirez-vous de moi et allez dans le feu éternel, qui a été préparé pour le diable et pour ses anges.
\VS{42}Car j'ai eu faim et vous ne m'avez point donné à manger ; j'ai eu soif et vous ne m'avez point donné à boire ;
\VS{43}j'étais étranger et vous ne m'avez point recueilli ; j'ai été nu, et vous ne m'avez point vêtu ; j'ai été malade et en prison, et vous ne m'avez point visité.
\VS{44}Alors ils répondront aussi en disant : Seigneur, quand t'avons-nous vu avoir faim, ou avoir soif, ou être étranger, ou nu, ou malade, ou en prison, et ne t'avons-nous point secouru ?
\VS{45}Alors il leur répondra, en disant : Je vous le dis en vérité, toutes les fois que vous n'avez pas fait ces choses à l'un de ces plus petits, c'est à moi que vous ne les avez pas faites.
\VS{46}Et ceux-ci iront au châtiment éternel, mais les justes à la vie éternelle.
\Chap{26}
\TextTitle{Le complot\FTNTT{Mc. 14:1-2 ; Lu. 22:1-2}}
\VerseOne{}Et il arriva que quand Jésus eut achevé tous ces discours, il dit à ses disciples :
\VS{2}Vous savez que la fête de Pâque a lieu dans deux jours ; et le Fils de l'homme sera livré pour être crucifié.
\VS{3}Alors les principaux sacrificateurs, les scribes et les anciens du peuple, se réunirent dans la cour du souverain sacrificateur, appelé Caïphe ;
\VS{4}et tinrent conseil ensemble pour se saisir de Jésus par finesse, afin de le faire mourir.
\VS{5}Mais ils dirent : Que ce ne soit pas pendant la fête, de peur qu'il ne se fasse quelque tumulte parmi le peuple.
\TextTitle{Geste prophétique de Marie de Béthanie\FTNTT{Mc. 14:3-9 ; Jn. 12:1-8}}
\VS{6}Comme Jésus était à Béthanie, dans la maison de Simon le lépreux,
\VS{7}une femme s'approcha de lui tenant un vase d'albâtre, plein d'un parfum de grand prix, et pendant qu'il était à table, elle répandit le parfum sur sa tête.
\VS{8}Mais ses disciples voyant cela, en furent indignés et dirent : À quoi sert cette perte ?
\VS{9}Car ce parfum pouvait être vendu bien cher et être donné aux pauvres.
\VS{10}Mais Jésus connaissant cela, leur dit : Pourquoi faites-vous de la peine à cette femme ? Car elle a fait une bonne action à mon égard ;
\VS{11}car vous aurez toujours des pauvres avec vous ; mais vous ne m'aurez pas toujours.
\VS{12}En répandant ce parfum sur mon corps, elle l'a fait pour ma sépulture.
\VS{13}Je vous le dis en vérité, partout où cet Evangile sera prêché, dans le monde entier, on racontera aussi en mémoire de cette femme ce qu'elle a fait.
\TextTitle{La trahison de Judas\FTNTT{Mc. 14:10-11 ; Lu. 22:3-6}}
\VS{14}Alors l'un des douze, appelé Judas Iscariot, alla vers les principaux sacrificateurs,
\VS{15}et leur dit : Que voulez-vous me donner, et je vous le livrerai ? Et ils lui comptèrent trente pièces d'argent\FTNT{Za. 11:12-13.}.
\VS{16}Et dès lors, il cherchait une occasion favorable pour le livrer.
\TextTitle{La dernière Pâque\FTNTT{Mc. 14:12-21 ; Lu. 22:7-20 ; Jn. 13:1-12}}
\VS{17}Or le premier jour des pains sans levain, les disciples s'approchèrent de Jésus pour lui dire : Où veux-tu que nous te préparions le repas de la Pâque ?
\VS{18}Il répondit : Allez à la ville chez un tel et dites-lui : Le Maître dit : Mon temps est proche ; je ferai la Pâque chez toi avec mes disciples.
\VS{19}Les disciples firent comme Jésus leur avait ordonné et préparèrent la Pâque.
\VS{20}Et quand le soir fut venu, il se mit à table avec les douze.
\VS{21}Et comme ils mangeaient, il dit : Je vous le dis en vérité, l'un de vous me trahira.
\VS{22}Ils furent profondément attristés, et chacun d'eux commença à lui dire : Seigneur, est-ce moi ?
\VS{23}Mais il leur répondit : Celui qui a mis avec moi la main dans le plat pour tremper, c'est celui qui me trahira.
\VS{24}Le Fils de l'homme s'en va, selon qu'il est écrit de lui ; mais malheur à cet homme par qui le Fils de l'homme est trahi ! Mieux vaudrait pour cet homme qu'il ne soit pas né.
\VS{25}Judas qui le trahissait, prit la parole et dit : Maître, est-ce moi ? Jésus lui dit : Tu l'as dit.
\TextTitle{Le repas de la Pâque\FTNTT{Mc. 14:22-25 ; Lu. 22:17-20 ; Jn. 13:12-30 ; 1 Co. 11:23-26}}
\VS{26}Pendant qu'ils mangeaient, Jésus prit le pain, et après avoir rendu grâces à Dieu, il le rompit et le donna à ses disciples et leur dit : Prenez, mangez, ceci est mon corps.
\VS{27}Puis ayant pris la coupe, et béni Dieu, il la leur donna, en leur disant : buvez-en tous.
\VS{28}car ceci est mon sang, le sang de la Nouvelle Alliance, qui est répandu pour beaucoup, pour la rémission des péchés.
\VS{29}Or je vous dis : que depuis cette heure je ne boirai point de ce fruit de vigne, jusqu'au jour que je le boirai de nouveau avec vous, dans le Royaume de mon Père.
\TextTitle{Jésus informe Pierre de son triple reniement\FTNTT{Mc. 14:26-31 ; Lu. 22:31-34 ; Jn. 13:36-38}}
\VS{30}Quand ils eurent chanté le cantique\FTNT{Les cantiques : Du grec « humneo », chants d'hymnes pascals. Il s'agit plus précisément des psaumes 113 à 118 et du psaume 136, que les Juifs appellent le « grand Hallel ». Le Hallel consiste en six psaumes (113 à 118). Cet ensemble de textes est généralement entonné à haute voix par toute la communauté de prière lors de l'office religieux du matin, à l'issue de la « Amidah » (prière récitée debout), à l'occasion de la Pâque (le premier soir), de la Pentecôte et des Tabernacles, ainsi que pour Hanoucca et Rosh Hodesh. Voir Mc. 14:26.}, ils se rendirent à la Montagne des Oliviers.
\VS{31}Alors Jésus leur dit : Vous serez tous cette nuit scandalisés à cause de moi ; car il est écrit : Je frapperai le Berger, et les brebis du troupeau seront dispersées\FTNT{Za. 13:7.}.
\VS{32}Mais, après que je serai ressuscité, je vous précéderai en Galilée.
\VS{33}Pierre, prenant la parole, lui dit : Quand même tous seraient scandalisés à cause de toi, je ne le serai jamais.
\VS{34}Jésus lui dit : En vérité je te dis, qu'en cette même nuit, avant que le coq ait chanté, tu me renieras trois fois.
\VS{35}Pierre lui répondit : Même s'il me fallait mourir avec toi, je ne te renierai pas. Et tous les disciples dirent la même chose.
\TextTitle{Jésus dans le jardin de Gethsémané\FTNTT{Mc. 14:32-42 ; Lu. 22:39-46 ; Jn. 18:1}}
\VS{36}Alors Jésus alla avec eux dans un lieu appelé Gethsémané et il dit à ses disciples : Asseyez-vous ici, jusqu'à ce que j'aie prié dans le lieu où je vais.
\VS{37}Il prit avec lui Pierre et les deux fils de Zébédée, et il commença à être attristé et fort angoissé.
\VS{38}Alors il leur dit : Mon âme est de toutes parts saisie de tristesse jusqu'à la mort ; demeurez ici et veillez avec moi.
\TextTitle{Première prière de Jésus\FTNTT{Mc. 14:35-38 ; Lu. 22:41-42}}
\VS{39}Puis, ayant fait quelques pas en avant, il se prosterna le visage contre terre, priant et disant : Mon Père, s'il est possible, fais que cette coupe passe loin de moi ; toutefois non point comme je le veux, mais comme tu le veux.
\TextTitle{Jésus trouve les disciples endormis\FTNTT{Mc. 14:37-40 ; Lu. 22:45-46}}
\VS{40}Puis il vint vers ses disciples, qu'il trouva endormis, et il dit à Pierre : Vous n'avez pas pu veiller une heure avec moi ?
\VS{41}Veillez et priez, afin que vous ne tombiez pas en tentation : car l'esprit est prompt, mais la chair est faible.
\TextTitle{Deuxième prière\FTNTT{Mc. 14:39 ; Lu. 22:44}}
\VS{42}Il s'éloigna encore pour la seconde fois, et il pria, disant : Mon Père, s'il n'est pas possible que cette coupe s'éloigne sans que je la boive, que ta volonté soit faite.
\VS{43}Il revint ensuite et les trouva encore endormis ; car leurs yeux étaient appesantis.
\TextTitle{Troisième prière\FTNTT{Mc. 14:41}}
\VS{44}Et les ayant laissés, il s'en alla encore, et pria pour la troisième fois, disant les mêmes paroles.
\VS{45}Puis il alla vers ses disciples et leur dit : Dormez maintenant et reposez-vous ; voici, l'heure est proche, et le Fils de l'homme va être livré entre les mains des méchants.
\VS{46}Levez-vous, allons. Voici, celui qui me trahit s'approche.
\TextTitle{Jésus trahi et arrêté\FTNTT{Mc. 14:43-50 ; Lu. 22:47-53 ; Jn. 18:2-11}}
\VS{47}Comme il parlait encore, voici, Judas, l'un des douze, vint, et avec lui une grande foule, avec des épées et des bâtons, envoyée par les principaux sacrificateurs et par les anciens du peuple.
\VS{48}Celui qui le trahissait leur avait donné ce signe : Celui à qui je donnerai un baiser, c'est lui, saisissez-le.
\VS{49}Aussitôt, s'approchant de Jésus, il lui dit : Maître, je te salue ; et il le baisa.
\VS{50}Et Jésus lui dit : Mon ami, pour quel sujet es-tu ici ? Alors s'étant approchés, ils mirent les mains sur Jésus et le saisirent.
\VS{51}Et voici, l'un de ceux qui étaient avec Jésus, étendit la main et tira son épée ; il frappa le serviteur du souverain sacrificateur, et lui emporta l'oreille.
\VS{52}Alors Jésus lui dit : Remets ton épée à sa place ; car tous ceux qui prendront l'épée, périront par l'épée.
\VS{53}Crois-tu que je ne puisse pas maintenant prier mon Père, qui me donnerait à l'instant plus de douze légions d'anges ?
\VS{54}Mais comment donc s'accompliraient les Ecritures qui disent qu'il faut que cela arrive ainsi ?
\VS{55}En ce même instant Jésus dit à la foule : Vous êtes venus avec des épées et des bâtons, comme après un brigand, pour me prendre ; j'étais tous les jours assis parmi vous, enseignant dans le temple, et vous ne m'avez pas saisi.
\VS{56}Mais tout ceci est arrivé afin que les Ecritures des prophètes soient accomplies. Alors tous les disciples l'abandonnèrent et s'enfuirent.
\TextTitle{Jésus devant Caïphe et le sanhédrin\FTNTT{Mc. 14:53-65 ; Jn. 18:12-14, 19-24.}}
\VS{57}Ceux qui avaient saisi Jésus l'amenèrent chez Caïphe, le souverain sacrificateur, où les scribes et les anciens étaient assemblés.
\VS{58}Pierre le suivit de loin, jusqu'à la cour du souverain sacrificateur, y entra et s'assit avec les officiers pour voir comment cela finirait.
\VS{59}Les principaux sacrificateurs, les anciens et tout le sanhédrin cherchaient des faux témoignages contre Jésus pour le faire mourir.
\VS{60}Mais ils n'en trouvèrent point, et bien que plusieurs faux témoins se soient présentés, ils n'en trouvèrent point de propres ; mais à la fin, deux faux témoins s'approchèrent
\VS{61}et dirent : Celui-ci a dit : Je puis détruire le temple de Dieu et le rebâtir en trois jours.
\VS{62}Alors le souverain sacrificateur se leva et lui dit : Ne réponds-tu rien ? Qu'est-ce que ces hommes déposent contre toi ?
\VS{63}Jésus garda le silence. Et le souverain sacrificateur prenant la parole, lui dit : Je te somme par le Dieu vivant, de nous dire si tu es le Christ, le Fils de Dieu.
\VS{64}Jésus lui dit : Tu l'as dit. De plus, je vous dis que désormais vous verrez le Fils de l'homme assis à la droite de la puissance de Dieu et venant sur les nuées du ciel.
\VS{65}Alors le souverain sacrificateur déchira ses vêtements, en disant : Il a blasphémé ! Qu'avons-nous encore besoin de témoins ? Voici, vous avez entendu maintenant son blasphème. Que vous en semble ?
\VS{66}Ils répondirent : Il est digne de mort.
\VS{67}Alors ils lui crachèrent au visage, et lui donnèrent des coups de poing et des soufflets, et les autres le frappaient avec leurs bâtons ;
\VS{68}en disant : Christ, prophétise-nous qui est celui qui t'a frappé.
\TextTitle{Le triple reniement de Pierre\FTNTT{Mc. 14:66-72 ; Lu. 22:55-62 ; Jn. 18:15-18, 25-27.}}
\VS{69}Or Pierre était assis dehors dans la cour. Une servante s'approcha de lui et lui dit : Toi aussi, tu étais aussi avec Jésus le Galiléen.
\VS{70}Mais il le nia devant tous, en disant : Je ne sais pas ce que tu dis.
\VS{71}Et comme il était sorti dans le vestibule, une autre servante le vit et elle dit à ceux qui étaient là : Celui-ci aussi était avec Jésus de Nazareth.
\VS{72}Et il le nia encore avec serment, disant : Je ne connais pas cet homme.
\VS{73}Peu après, ceux qui se trouvaient là s'approchèrent et dirent à Pierre : Certainement tu es aussi de ces gens-là, car ton langage te fait connaître.
\VS{74}Alors il commença à faire des imprécations et à jurer, en disant : Je ne connais pas cet homme. Et aussitôt le coq chanta.
\VS{75}Et Pierre se souvint de la parole de Jésus, qui lui avait dit : Avant que le coq chante, tu me renieras trois fois. Et étant sorti dehors, il pleura amèrement.
\Chap{27}
\TextTitle{Jésus devant le gouverneur Pilate ; suicide de Judas\FTNTT{Ac. 1:16-19.}}
\VerseOne{}Puis quand le matin fut venu, tous les principaux sacrificateurs et les anciens du peuple tinrent conseil contre Jésus pour le faire mourir.
\VS{2}Après l'avoir lié, ils l'amenèrent et le livrèrent à Ponce Pilate, qui était le gouverneur.
\VS{3}Alors Judas qui l'avait trahi, voyant qu'il était condamné, se repentit et rapporta les trente pièces d'argent aux principaux sacrificateurs et aux anciens,
\VS{4}en leur disant : J'ai péché en trahissant le sang innocent ; mais ils lui dirent : Que nous importe ? Cela te regarde.
\VS{5}Et après avoir jeté les pièces d'argent dans le temple, il se retira et alla se pendre.
\VS{6}Mais les principaux sacrificateurs prirent les pièces d'argent et dirent : Il n'est pas permis de les mettre dans le trésor ; car c'est le prix du sang.
\VS{7}Et, après en avoir délibéré, ils achetèrent avec cet argent le champ d'un potier pour la sépulture des étrangers.
\VS{8}C'est pourquoi ce champ-là a été appelé jusqu'à aujourd'hui, le champ du sang.
\VS{9}Alors s'accomplit ce qui avait été annoncé par Jérémie le prophète, disant : Ils ont pris les trente pièces d'argent, le prix de celui qui a été estimé, qu'on a estimé de la part des enfants d'Israël ;
\VS{10}et ils les ont données pour acheter le champ d'un potier, selon ce que le Seigneur m'avait ordonné\FTNT{Ce verset se réfère certainement à Za. 11:12-13, avec une allusion à Jé. 18:1-4.}.
\VS{11}Jésus comparut devant le gouverneur. Le gouverneur l'interrogea : Es-tu le Roi des Juifs ? Jésus lui répondit : Tu le dis.
\VS{12}Mais il ne répondit rien aux accusations des principaux sacrificateurs et des anciens.
\VS{13}Alors Pilate lui dit : N'entends-tu pas de combien de choses ils t'accusent ?
\VS{14}Mais il ne lui donna de réponse sur aucune parole, ce qui étonna beaucoup le gouverneur.
\TextTitle{Jésus ou Barabbas ?\FTNTT{Mc. 15:6-15 ; Lu. 23:17-25 ; Jn. 18:39-40}}
\VS{15}Or le gouverneur avait coutume de relâcher un prisonnier à chaque fête, celui que demandait la foule.
\VS{16}Et il y avait alors un prisonnier fameux, nommé Barabbas.
\VS{17}Comme ils étaient assemblés, Pilate leur dit : Lequel voulez-vous que je vous relâche ? Barabbas ou Jésus qu'on appelle Christ ?
\VS{18}Car il savait bien qu'ils l'avaient livré par envie.
\VS{19}Et pendant qu'il siégeait au tribunal, sa femme envoya lui dire : Ne te mêle point de l'affaire de ce juste, car j'ai beaucoup souffert aujourd'hui en songe à cause de lui.
\VS{20}Les principaux sacrificateurs et les anciens persuadèrent la multitude du peuple de demander Barabbas et de faire périr Jésus.
\VS{21}Et le gouverneur prenant la parole leur dit : Lequel des deux voulez-vous que je vous relâche ? Ils dirent : Barabbas.
\VS{22}Pilate leur dit : Que ferai-je donc de Jésus qu'on appelle Christ ? Ils lui dirent tous : Qu'il soit crucifié !
\VS{23}Et le gouverneur leur dit : Mais quel mal a-t-il fait ? Et ils crièrent encore plus fort, en disant : Qu'il soit crucifié !
\VS{24}Alors Pilate voyant qu'il ne gagnait rien, mais que le tumulte s'augmentait, prit de l'eau et lava ses mains devant le peuple, en disant : Je suis innocent du sang de ce juste. Cela vous regarde.
\VS{25}Et tout le peuple répondit : Que son sang retombe sur nous et sur nos enfants !
\VS{26}Alors il leur relâcha Barabbas ; et après avoir fait fouetter Jésus, il le leur livra pour être crucifié.
\TextTitle{Le Roi couronné d'épines\FTNTT{Mc. 15:16-23 ; Lu. 23:26-32 ; Jn. 19:16-17}}
\VS{27}Les soldats du gouverneur amenèrent Jésus dans le prétoire et assemblèrent devant lui toute la cohorte.
\VS{28}Et après l'avoir dépouillé, ils le revêtirent d'un manteau d'écarlate.
\VS{29}Puis, ayant fait une couronne d'épines entrelacées, ils la mirent sur sa tête et ils lui mirent un roseau dans sa main droite ; puis s'agenouillant devant lui, ils se moquaient de lui, en disant : Nous te saluons, Roi des Juifs !
\VS{30}Et ils crachaient contre lui, prenaient le roseau et frappaient sur sa tête.
\VS{31}Après s'être ainsi moqués de lui, ils lui ôtèrent le manteau, et lui remirent ses vêtements, et l'amenèrent pour le crucifier.
\VS{32}Comme ils sortaient, ils rencontrèrent un homme de Cyrène, appelé Simon et ils le forcèrent à porter la croix de Jésus.
\TextTitle{La crucifixion de Jésus\FTNTT{Mc. 15:24-32 ; Lu. 23:33-43 ; Jn. 19:17-24}}
\VS{33}Et étant arrivés au lieu appelé Golgotha, c'est-à-dire le lieu du crâne,
\VS{34}ils lui donnèrent à boire du vinaigre mêlé avec du fiel\FTNT{Le vinaigre mêlé au fiel (Ps. 69:22) : Ce breuvage, appelé « posca », était un vin amer composé qui se transformait en vinaigre à cause des mauvaises conditions de conservation. Allongé avec de l'eau et parfois adoucie avec de l'œuf, cette boisson bon marché et très rafraîchissante était consommée principalement par les légionnaires et les esclaves. Connue pour ses vertus antiseptiques, les soldats de l'Antiquité avaient coutume d'y ajouter des drogues comme la myrrhe et le fiel (opium) pour atténuer les souffrances. En refusant de le boire, le Seigneur Jésus-Christ a réellement pris sur lui la plénitude du châtiment que nous méritons à cause de nos péchés.} ; mais quand il l'eut goûté, il ne voulut pas boire.
\VS{35}Et après l'avoir crucifié, ils partagèrent ses vêtements, en tirant au sort, afin que s'accomplît ce qui avait été annoncé par le prophète : Ils se sont partagés mes vêtements, et ont jeté ma tunique au sort\FTNT{Ps. 22:19.}.
\VS{36}Puis s'étant assis, ils le gardaient là.
\VS{37}Ils mirent aussi au-dessus de sa tête un écriteau, où la cause de sa condamnation était marquée en ces mots : CELUI-CI EST JESUS, LE ROI DES JUIFS.
\VS{38}Avec lui furent crucifiés deux brigands, l'un à sa droite et l'autre à sa gauche.
\VS{39}Et Ceux qui passaient par là, l'injuriaient et secouaient la tête
\VS{40}en disant : Toi qui détruis le temple et qui le rebâtis en trois jours, sauve-toi toi-même ! Si tu es le Fils de Dieu, descends de la croix !
\VS{41}Pareillement aussi, les principaux sacrificateurs avec les scribes et les anciens, se moquant, disaient :
\VS{42}Il a sauvé les autres et il ne peut pas se sauver lui-même ! S'il est le Roi d'Israël, qu'il descende maintenant de la croix et nous croirons en lui.
\VS{43}Il se confie en Dieu ; mais si Dieu l'aime, qu'il le délivre maintenant, car il a dit : Je suis le Fils de Dieu.
\VS{44}Les brigands aussi qui étaient crucifiés avec lui, lui reprochaient la même chose.
\TextTitle{Jésus accomplit la loi par sa mort\FTNTT{Mc. 15:33-41 ; Lu. 23:44-49 ; Jn. 19:30-37 ; Hé. 9:3-8 ; 10:19-20}}
\VS{45}Depuis la sixième heure jusqu'à la neuvième, il y eut des ténèbres sur toute la terre.
\VS{46}Et vers la neuvième heure, Jésus s'écria d'une voix forte : Eli, Eli, lama sabachthani ? C'est-à-dire : Mon Dieu ! Mon Dieu ! Pourquoi m'as-tu abandonné ?
\VS{47}Quelques-uns de ceux qui étaient là présents, ayant entendu cela, disaient : Il appelle Elie.
\VS{48}Et aussitôt l'un d'entre eux courut prendre une éponge, qu'il remplit de vinaigre, et l'ayant fixée au bout d'un roseau, lui donna à boire.
\VS{49}Mais les autres disaient : Laisse, voyons si Elie viendra le sauver.
\VS{50}Alors Jésus, poussa de nouveau un grand cri, et rendit l'esprit.
\VS{51}Et voici, le voile du temple se déchira en deux, depuis le haut jusqu'en bas\FTNT{C'est ici que s'achève la Première Alliance. Cette dernière était relative à la loi de Moïse, c'est-à-dire aux ordonnances liées au culte, qui reposait sur le sacerdoce lévitique et les sacrifices d'animaux, et au sanctuaire terrestre, à savoir le temple de Jérusalem (Hé. 9:1). Le Seigneur ayant offert une fois pour toutes le sacrifice parfait, les exigences de la justice divine ont été pleinement satisfaites (Hé. 9:11-12 ; 25-26). Désormais, la Première Alliance n'a plus de raison d'être et peut donc disparaître (Hé. 8:13). Non seulement la déchirure du voile séparant le lieu saint du Saint des saints atteste la fin de la Première Alliance, mais invite aussi tout homme à s'approcher de Dieu en esprit, sans intermédiaires (Lévites, sacrificateurs, pasteurs, prophètes…) ni nécessité de se rendre dans un temple (Jn. 4:23). La Nouvelle Alliance est aussi un testament puisque Jésus-Christ, notre légataire, est passé par la mort (Hé. 9:16-18). Voir aussi commentaire en Ex. 19:5.} ; et la terre trembla, et les pierres se fendirent.
\TextTitle{Le voile déchiré : Fin de la loi mosaïque ou de la Première Alliance}
\VS{52}Et les sépulcres s'ouvrirent et plusieurs corps des saints qui étaient morts ressuscitèrent.
\VS{53}Et étant sortis des sépulcres après la résurrection de Jésus, ils entrèrent dans la ville sainte et se montrèrent à plusieurs.
\VS{54}Le centenier et ceux qui étaient avec lui pour garder Jésus, ayant vu le tremblement de terre et tout ce qui venait d'arriver, furent saisis d'une grande frayeur et dirent : Certainement cet homme était le Fils de Dieu.
\VS{55}Il y avait là aussi plusieurs femmes qui regardaient de loin, et qui avaient suivi Jésus depuis la Galilée, pour le servir.
\VS{56}Entre lesquelles étaient Marie de Magdala, Marie mère de Jacques et de Joseph, et la mère des fils de Zébédée.
\TextTitle{Jésus enseveli\FTNTT{Mc. 15:42-47 ; Lu. 23:50-56 ; Jn. 19:38-42}}
\VS{57}Le soir étant venu, un homme riche d'Arimathée, appelé Joseph, qui était aussi disciple de Jésus,
\VS{58}se rendit vers Pilate et demanda le corps de Jésus. En même temps Pilate ordonna que le corps soit rendu.
\VS{59}Joseph prit le corps et l'enveloppa d'un linceul pur ;
\VS{60}et le mit dans un sépulcre neuf, qu'il s'était fait tailler dans le roc. Puis il roula une grande pierre à l'entrée du sépulcre et il s'en alla.
\VS{61}Marie de Magdala et l'autre Marie étaient là, assises vis-à-vis du sépulcre.
\TextTitle{Le sépulcre scellé et gardé}
\VS{62}Le lendemain, qui était le jour de la préparation du sabbat, les principaux sacrificateurs et les pharisiens allèrent ensemble auprès de Pilate,
\VS{63}et lui dirent : Seigneur ! Nous nous souvenons que ce séducteur disait, quand il était encore en vie : Après trois jours je ressusciterai.
\VS{64}Ordonne donc que le sépulcre soit gardé sûrement jusqu'au troisième jour ; de peur que ses disciples ne viennent de nuit, et ne dérobent son corps, et qu'ils ne disent au peuple : Il est ressuscité des morts. Cette dernière imposture serait pire que la première.
\VS{65}Pilate leur dit : Vous avez une garde ; allez et faites-le garder comme vous l'entendez.
\VS{66}Ils s'en allèrent donc, et s'assurèrent du sépulcre, au moyen d'une garde, après avoir scellé la pierre.
\Chap{28}
\TextTitle{Résurrection et apparition de Jésus-Christ\FTNTT{Mc. 16:1-14 ; Lu. 24:1-49 ; Jn. 20:1-23}}
\VerseOne{}Après le sabbat, à l'aube du premier jour de la semaine, Marie de Magdala et l'autre Marie allèrent voir le sépulcre.
\VS{2}Et voici, il eut un grand tremblement de terre ; car un ange du Seigneur descendit du ciel, vint rouler la pierre à côté de l'entrée du sépulcre et s'assit dessus.
\VS{3}Son visage était comme un éclair, et son vêtement blanc comme de la neige.
\VS{4}Les gardes furent tellement saisis de frayeur, qu'ils devinrent comme morts.
\VS{5}Mais l'ange prit la parole et dit aux femmes : Pour vous, ne craignez pas ; car je sais que vous cherchez Jésus, qui a été crucifié.
\VS{6}Il n'est point ici car il est ressuscité comme il l'avait dit. Venez et voyez le lieu où le Seigneur était couché,
\VS{7}et allez-vous-en promptement, et dites à ses disciples qu'il est ressuscité des morts. Et voici, il vous précède en Galilée ; c'est là que vous le verrez. Voici, je vous l'ai dit.
\VS{8}Alors elles sortirent promptement du sépulcre avec crainte et grande joie ; et coururent l'annoncer à ses disciples.
\VS{9}Mais comme elles allaient pour l'annoncer à ses disciples, voici, Jésus se présenta devant elles et leur dit : Je vous salue. Et elles s'approchèrent, embrassèrent ses pieds et l'adorèrent.
\VS{10}Alors Jésus leur dit : Ne craignez point. Allez et dites à mes frères d'aller en Galilée, c'est là qu'ils me verront.
\TextTitle{Les soldats soudoyés par les sacrificateurs}
\VS{11} Or quand elles furent parties, voici, quelques-uns de la garde vinrent dans la ville et ils rapportèrent aux principaux sacrificateurs toutes les choses qui étaient arrivées.
\VS{12}Sur quoi les sacrificateurs s'assemblèrent avec les anciens, et après avoir tenu conseil, donnèrent une forte somme d'argent aux soldats,
\VS{13}en leur disant : Dites : Ses disciples sont venus de nuit le dérober, pendant que nous dormions.
\VS{14}Et si le gouverneur l'apprend, nous l'apaiserons et nous vous tirerons de peine.
\VS{15}Les soldats prirent l'argent et suivirent les instructions qui leur furent données. Et ce bruit s'est répandu parmi les juifs, jusqu'à aujourd'hui.
\TextTitle{Mission des apôtres\FTNTT{Mc. 16:15-18 ; Lu. 24:46-48 ; Jn. 17:18 ; 20:21 ; Ac. 1:8 ; 1 Co. 15:6}}
\VS{16}Mais les onze disciples allèrent en Galilée, sur la montagne, où Jésus leur avait ordonné de se rendre.
\VS{17}Quand ils le virent, ils l'adorèrent, mais quelques-uns doutèrent.
\VS{18}Jésus s'étant approché, leur parla, en disant : Tout puissance m'a été donnée dans le ciel et sur la terre.
\VS{19}Allez donc et enseignez toutes les nations, les baptisant au Nom du Père, du Fils et du Saint-Esprit ;
\VS{20}les enseignant à garder toutes les choses que je vous ai commandées ; et voici, je suis avec vous toujours jusqu'à la fin du monde. Amen.
\PPE{}
\end{multicols}

%\clearpage\ShortTitle{Marc}\BookTitle{Marc}\BFont
\noindent\hrulefill
\textit{
\bigskip
{\centering{}
\\Signifie : Qui brille, luisant
\\Thème : Jésus le serviteur
\\Auteur : Marc
\\Date de rédaction : Env. 68 apr. J.-C.\\}
}
%\bigskip
\textit{
\\Originaire de Jérusalem, Marc, aussi appelé Jean, fut l’auteur de l’évangile du même nom. Cousin de Barnabas et collaborateur de Paul, ce dernier l’éconduit lors d’un voyage car Marc l’avait abandonné lors d’une précédente mission. Ce fut d’ailleurs la cause de la séparation entre Barnabas et Paul.  Par la suite, il renoua le contact avec Paul et devint un de ses fidèles compagnons de ministère. Lié à l’apôtre Pierre tel un fils, ce fut probablement sous son autorité qu’il écrivit. En effet, l’évangile de Marc expose le témoignage de Pierre sur Christ.
\bigskip
\\Adressé aux gentils, cet évangile contient peu de références à l’ancienne alliance ; on y découvre Jésus l’inlassable serviteur de Dieu et des hommes. Marc y exposa la richesse de ses bonnes œuvres, son incomparable dévouement et révéla les sentiments intimes du maître. Même si Marc présenta principalement Jésus en tant que serviteur, son récit des miracles met en exergue toute la puissance du Christ.\bigskip
}
\par\nobreak\noindent\hrulefill
\begin{multicols}{2}
\TextTitle{[Ministère de Jean-Baptiste]
\\(Mt. 3:1-12 ; Lu. 3:1-20 ; Jn. 1:6-8,15-37)}
\Chap{1}
\VerseOne{}Commencement de l'Evangile de Jésus-Christ, Fils de Dieu ;
\VS{2}selon qu'il est écrit dans les prophètes : Voici, j'envoie mon messager devant ta face, lequel préparera ta voie devant toi.
\VS{3}C’est la voix de celui qui crie dans le désert : Préparez le chemin du Seigneur, aplanissez ses sentiers{\FTNT{Es. 40:3 ; Mal. 3:1.}}.
\VS{4}Jean baptisait dans le désert, et prêchait le baptême de repentance, pour obtenir la rémission des péchés.
\VS{5}Et tout le pays de Judée, et les habitants de Jérusalem allaient vers lui, et confessant leurs péchés, ils se faisaient tous baptiser par lui dans le fleuve du Jourdain.
\VS{6}Jean était vêtu de poils de chameau, il avait une ceinture de cuir autour de ses reins, et mangeait des sauterelles et du miel sauvage.
\VS{7}Et il prêchait, en disant : Il vient après moi, celui qui est plus puissant que moi, et je ne suis pas digne de délier en me baissant la courroie de ses souliers.
\VS{8}Moi, je vous ai baptisés d'eau ; mais lui, il vous baptisera du Saint-Esprit.
\TextTitle{[Baptême de Jésus-Christ]
\\(Mt. 3:13-17 ; Lu. 3:21-22 ; Jn. 1:31-34)}
\VS{9}En ce temps-là, Jésus vint de Nazareth, ville de Galilée, et il fut baptisé par Jean dans le Jourdain.
\VS{10}Au moment où il sortait de l'eau, Jean vit les cieux s’ouvrir, et le Saint-Esprit descendre sur lui comme une colombe.
\VS{11}Et une voix fit entendre des cieux ces paroles : Tu es mon Fils bien-aimé, en qui j'ai mis toute mon affection.
\TextTitle{[La tentation]
\\(Mt. 4:1-11 ; Lu. 4:1-13)}
\VS{12}Aussitôt l'Esprit le poussa à se rendre dans un désert,
\VS{13}où il passa quarante jours, tenté par Satan. Il était avec les bêtes sauvages, et les anges le servaient.
\TextTitle{[Jésus en Galilée]
\\(Mt. 4:12-17 ; Lu. 4:14-15)}
\VS{14}Après que Jean eut été mis en prison, Jésus alla dans la Galilée, prêchant l'Evangile du Royaume de Dieu.
\VS{15}Il disait : Le temps est accompli, et le Royaume de Dieu est proche. Repentez-vous, et croyez à 1'Evangile.
\TextTitle{[Appel de Simon (Pierre), André, Jacques et Jean]
\\(Lu. 5:1-11 ; Jn. 1:35-51)}
\VS{16}Comme il marchait près de la mer de Galilée, il vit Simon et André son frère, qui jetaient leurs filets dans la mer, car ils étaient pêcheurs.
\VS{17}Jésus leur dit : Suivez-moi, et je vous ferai pêcheurs d'hommes.
\VS{18}Aussitôt ils laissèrent leurs filets et ils le suivirent.
\VS{19}Etant allé un peu plus loin, il vit Jacques fils de Zébédée, et Jean son frère, qui raccommodaient leurs filets dans la barque.
\VS{20}Aussitôt, il les appela ; et laissant leur père Zébédée dans la barque avec les ouvriers, ils le suivirent.
\TextTitle{[Jésus chasse un démon dans la synagogue]
\\(Lu. 4:31-37)}
\VS{21}Ils entrèrent dans Capernaüm. Et le jour du sabbat, Jésus entra d’abord dans la synagogue, et il enseigna.
\VS{22}Ils étaient étonnés de sa doctrine ; car il les enseignait comme ayant autorité, et non pas comme les scribes.
\VS{23}Il se trouvait dans leur synagogue un homme qui avait un esprit impur, et qui s'écria,
\VS{24}en disant : Ha ! Qu’y a-t-il entre toi et nous, Jésus de Nazareth ? Es-tu venu pour nous perdre ? Je sais qui tu es : Tu es le Saint de Dieu.
\VS{25}Mais Jésus le menaça, disant : Tais-toi, et sors de cet homme.
\VS{26}Alors l'esprit impur sortit de cet homme, en l’agitant avec violence, et en poussant un grand cri.
\VS{27}Tous furent étonnés, de sorte qu'ils se demandaient les uns aux autres, et disaient : Qu'est-ce que ceci ? Quelle est cette nouvelle doctrine ? Il commande avec autorité même aux esprits impurs, et ils lui obéissent.
\VS{28}Et sa renommée se répandit aussitôt dans tout le pays des environs de la Galilée.
\TextTitle{[Jésus guérit la belle-mère de Pierre]
\\(Mt. 8:14-15 ; Lu. 4:38-39)}
\VS{29}En sortant de la synagogue, ils se rendirent avec Jacques et Jean à la maison de Simon et d'André.
\VS{30}La belle-mère de Simon était couchée, ayant la fièvre ; et aussitôt on parla d’elle à Jésus.
\VS{31}S’étant approché, il la fit lever en la prenant par la main ; et à l'instant la fièvre la quitta ; et elle les servit.
\TextTitle{[Jésus guérit les malades et chasse des démons ; prédications en Galilée]
\\(Mt. 8:16-17 ; Lu. 4:40-44)}
\VS{32}Le soir étant venu, comme le soleil se couchait, on lui amena tous les malades, et les démoniaques.
\VS{33}Et toute la ville était assemblée devant sa porte.
\VS{34}Il guérit beaucoup de malades qui avaient différentes maladies et chassa beaucoup de démons, et il ne permettait pas aux démons de parler, parce qu’ils le connaissaient.
\VS{35}Vers le matin, pendant qu’il faisait encore très sombre, il se leva, et sortit pour aller dans un lieu désert, où il pria.
\VS{36}Simon et ceux qui étaient avec lui se mirent à sa recherche,
\VS{37}et quand ils l’eurent trouvé, ils lui dirent : Tous te cherchent.
\VS{38}Et il leur dit : Allons aux bourgades voisines, afin que j'y prêche aussi ; car c’est pour cela que je suis venu.
\VS{39}Il prêchait donc dans leurs synagogues, par toute la Galilée, et chassait les démons.
\TextTitle{[Jésus guérit un lépreux]
\\(Mt. 8:2-4 ; Lu. 5:12-14)}
\VS{40}Un lépreux vint à lui, le priant et se mettant à genoux devant lui, et lui dit : Si tu veux, tu peux me rendre pur.
\VS{41}Jésus, ému de compassion, étendit sa main et le toucha, en lui disant : Je le veux, sois pur.
\VS{42}La lèpre quitta aussitôt cet homme, et il fut purifié.
\VS{43}Jésus le renvoya sur-le-champ, avec de sévères recommandations,
\VS{44}et lui dit : Garde-toi de ne rien dire à personne ; mais va te montrer au sacrificateur, et présente pour ta purification les choses que Moïse a commandées, pour leur servir de témoignage{\FTNT{Loi sur la purification de la lèpre~: Lé. 14:1-32. Avant sa mort et sa résurrection, Jésus-Christ observait la loi de Moïse (Mt. 23:1-2).}}.
\VS{45}Mais cet homme, s’en étant allé, commença à publier ouvertement la chose et à divulguer ce qui s'était passé ; de sorte que Jésus ne pouvait plus entrer publiquement dans la ville, mais il se tenait dehors, dans des lieux déserts, et l’on venait à lui de toutes parts.
\TextTitle{[Jésus guérit un paralytique]
\\(Mt. 9:2-8 ; Lu. 5:18-26)}
\Chap{2}
\VerseOne{}Quelques jours après, Jésus revint à Capernaüm. On apprit qu'il était à la maison,
\VS{2}et aussitôt il s’assembla un si grand nombre de personnes, que l'espace même devant la porte ne pouvait plus les contenir. Il leur annonçait la parole.
\VS{3}Et quelques-uns vinrent à lui, amenant un paralytique qui était porté par quatre personnes.
\VS{4}Comme ils ne pouvaient pas s’approcher de lui à cause de la foule, ils découvrirent le toit du lieu où il était, et l'ayant percé, ils descendirent le lit dans lequel le paralytique était couché.
\VS{5}Jésus, voyant leur foi, dit au paralytique : Mon enfant, tes péchés te sont pardonnés.
\VS{6}Et quelques scribes qui étaient assis là, raisonnaient ainsi en eux-mêmes :
\VS{7}Comment cet homme parle-t-il ainsi ? Il blasphème. Qui peut pardonner les péchés, si ce n’est Dieu seul ?
\VS{8}Jésus, ayant aussitôt connu par son esprit qu'ils raisonnaient ainsi en eux-mêmes, leur dit : Pourquoi avez-vous de telles pensées dans vos cœurs ?
\VS{9}Lequel est le plus aisé de dire au paralytique : Tes péchés te sont pardonnés, ou de dire : Lève-toi, prends ton lit, et marche ?
\VS{10}Mais afin que vous sachiez que le Fils de l'homme a le pouvoir sur la terre de pardonner les péchés, il dit au paralytique :
\VS{11}Je te dis : Lève-toi, prends ton lit, et va dans ta maison.
\VS{12}Et il se leva aussitôt, et ayant pris son lit, il sortit en présence de tous ; de sorte qu'ils furent tous étonnés, et ils glorifièrent Dieu, en disant : Nous n’avons jamais rien vu de pareil.
\TextTitle{[Appel de Matthieu]
\\(Mt. 9:9 ; Lu. 5:27-28)}
\VS{13}Jésus sortit de nouveau du côté de la mer, toute la foule venait à lui, et il les enseignait.
\VS{14}En passant, il vit Lévi, fils d'Alphée, assis au bureau des péages, et il lui dit : Suis-moi. Et Lévi s'étant levé, le suivit.
\TextTitle{[Jésus appelle des pêcheurs à la repentance, non des justes]
\\(Mt. 9:10-15 ; Lu. 5:29-35)}
\VS{15}Comme Jésus était à table dans la maison de Lévi, plusieurs publicains et des gens de mauvaise vie se mirent aussi à table avec lui et avec ses disciples ; car ils étaient nombreux, et l'avaient suivi.
\VS{16}Mais les scribes et les pharisiens voyant qu'il mangeait avec les publicains et les gens de mauvaise vie, disaient à ses disciples : Pourquoi mange-t-il et boit-il avec les publicains et les gens de mauvaise vie ?
\VS{17}Jésus ayant entendu cela, leur dit : Ce ne sont pas ceux qui se portent bien qui ont besoin de médecin, mais les malades. Je ne suis pas venu appeler à la repentance les justes, mais les pécheurs.
\TextTitle{[Les pharisiens et les disciples de Jean interrogent Jésus sur le jeûne]}
\VS{18}Les disciples de Jean et ceux des pharisiens jeûnaient ; ils vinrent à Jésus et lui dirent : Pourquoi les disciples de Jean, et ceux des pharisiens, jeûnent-ils, tandis que tes disciples ne jeûnent point ?
\VS{19}Jésus leur répondit : Les amis de l'Epoux peuvent-ils jeûner pendant que l'Epoux est avec eux ? Aussi longtemps qu’ils ont avec eux l'Epoux, ils ne peuvent jeûner.
\VS{20}Mais les jours viendront où l'Epoux leur sera ôté, alors ils jeûneront en ce jour-là.
\TextTitle{[Parabole du drap neuf et des outres neuves]
\\(Mt. 9:16-17 ; Lu. 5:36-39)}
\VS{21}Personne ne coud une pièce de drap neuf à un vieil habit ; autrement, la pièce du drap neuf emporterait une partie du vieux, et la déchirure serait pire.
\VS{22}Et personne ne met du vin nouveau dans de vieilles outres ; autrement, le vin nouveau fait rompre les outres, et le vin se répand, et les outres sont perdues ; mais le vin nouveau doit être mis dans des outres neuves.
\TextTitle{[Jésus, le Maître du sabbat]
\\(Mt. 12:1-8 ; Lu. 6:1-5)}
\VS{23}Il arriva, un jour de sabbat, que Jésus traversa des champs de blé. Ses disciples en marchant se mirent à arracher des épis.
\VS{24}Les pharisiens lui dirent : Regarde, pourquoi font-ils ce qui n'est pas permis les jours de sabbat ?
\VS{25}Mais il leur dit : N'avez-vous jamais lu ce que fit David quand il fut dans la nécessité, et qu'il eut faim, lui et ceux qui étaient avec lui ?
\VS{26}Comment il entra dans la maison de Dieu, au temps du souverain sacrificateur Abiathar, et mangea les pains de proposition{\FTNT{1 S. 21:1-7.}} , qu’il n'est permis qu'aux sacrificateurs de manger ; et il en donna même à ceux qui étaient avec lui!
\VS{27}Puis il leur dit : Le sabbat a été fait pour l'homme, et non pas l'homme pour le sabbat ;
\VS{28}de sorte que le Fils de l'homme est Maître même du sabbat.
\TextTitle{[Jésus-Christ guérit un homme à la main sèche le jour du sabbat]
\\(Mt. 12:9-13 ; Lu. 6:6-11)}
\Chap{3}
\VerseOne{}Jésus entra de nouveau dans la synagogue, et il y avait là un homme qui avait une main sèche.
\VS{2}Ils l'observaient, pour voir s'il le guérirait le jour du sabbat, afin de l'accuser.
\VS{3}Et Jésus dit à l'homme qui avait la main sèche : Lève-toi, et tiens-toi là au milieu.
\VS{4}Puis il leur dit : Est-il permis de faire du bien les jours de sabbat, ou de faire du mal, de sauver une personne, ou de la tuer ? Mais ils gardèrent le silence.
\VS{5}Alors, les regardant tous avec indignation, et étant affligé de l'endurcissement de leur cœur, il dit à cet homme : Etends ta main. Il l'étendit, et sa main fut rendue saine comme l'autre.
\TextTitle{[Nombreuses guérisons de Jésus]
\\(Mt. 12:15-16 ; Lu. 6:17-19}
\VS{6}Alors les pharisiens sortirent, et aussitôt, ils se consultèrent avec les hérodiens, sur les moyens de le faire périr.
\VS{7}Mais Jésus se retira vers la mer avec ses disciples. Une grande multitude le suivit de la Galilée,
\VS{8}de Judée, de Jérusalem, de l’Idumée, d’au-delà du Jourdain, et des environs de Tyr et de Sidon, une grande multitude, ayant entendu les grandes choses qu'il faisait, vint vers lui en grand nombre.
\VS{9}Et il dit à ses disciples de tenir toujours à sa disposition une petite barque, afin de ne pas être pressé par la foule.
\VS{10}Car, comme il guérissait beaucoup de gens, tous ceux qui avaient des maladies se jetaient sur lui pour le toucher.
\VS{11}Et les esprits impurs, quand ils le voyaient, se prosternaient devant lui, et s'écriaient en disant : Tu es le Fils de Dieu.
\VS{12}Mais il leur défendait avec de grandes menaces de le faire connaître.
\TextTitle{[L'appel des douze apôtres]
\\(Mt. 10:1-4 ; Lu. 6:13-16)}
\VS{13}Puis il monta sur une montagne, appela ceux qu'il voulut, et ils vinrent auprès de lui.
\VS{14}Il en établit douze pour être avec lui,
\VS{15}et pour les envoyer prêcher, avec la puissance de guérir les maladies, et de chasser les démons.
\VS{16}Voici les douze qu’il établit : Simon qu'il nomma Pierre ;
\VS{17}Jacques fils de Zébédée, et Jean, frère de Jacques, auxquels il donna le nom de Boanergès, ce qui veut dire fils de tonnerre.
\VS{18}André ; Philippe ; Barthélemy ; Matthieu ; Thomas ; Jacques, fils d'Alphée ; Thaddée ; Simon le Cananite ;
\VS{19}et Judas Iscariot, celui qui livra Jésus.
\VS{20}Ils se rendirent à la maison, et une grande multitude s’assembla de nouveau, en sorte qu’ils ne pouvaient même pas prendre leur repas.
\VS{21}Quand les parents de Jésus apprirent cela, ils sortirent pour se saisir de lui. Car ils disaient : Il est hors de sens.
\TextTitle{[Le blasphème contre le Saint-Esprit]
\\(Mt. 12:24-32 ; Lu. 11:15-23)}
\VS{22}Et les scribes, qui étaient descendus de Jérusalem, disaient : Il est possédé par Béelzébul ; c’est par le prince des démons qu’il chasse les démons.
\VS{23}Mais Jésus les appela, et leur dit sous forme de paraboles : Comment Satan peut-il chasser Satan ?
\VS{24}Si un royaume est divisé contre lui-même, ce royaume ne peut subsister ;
\VS{25}et si une maison est divisée contre elle-même, cette maison ne peut subsister.
\VS{26}Si donc Satan s'élève contre lui-même, il est divisé, il ne peut subsister, mais il tend vers sa fin.
\VS{27}Personne ne peut entrer dans la maison d'un homme fort et piller ses biens, sans avoir auparavant lié cet homme fort ; alors il pillera sa maison.
\VS{28}Je vous le dis en vérité, que toutes sortes de péchés seront pardonnés aux enfants des hommes, et aussi toutes sortes de blasphèmes par lesquels ils auront blasphémé ;
\VS{29}mais quiconque blasphémera contre le Saint-Esprit n’obtiendra jamais de pardon : Il est coupable et subira une condamnation éternelle {\FTNT{Voir commentaire Mt. 12:32.}}.
\VS{30}Jésus parla ainsi parce qu'ils disaient : Il est possédé d'un esprit impur.
\TextTitle{[La famille spirituelle]
\\(Mt. 12:46-50 ; Lu. 8:19-21)}
\VS{31}Survinrent ses frères et sa mère qui, se tenant dehors, l'envoyèrent appeler. La multitude était assise autour de lui,
\VS{32}et on lui dit : Voici, ta mère et tes frères sont dehors et te demandent.
\VS{33}Mais il leur répondit : Qui est ma mère, et qui sont mes frères ?
\VS{34}Et, jetant les regards sur ceux qui étaient assis tout autour de lui, il dit : Voici ma mère et mes frères.
\VS{35}Car quiconque fera la volonté de Dieu, celui-là est mon frère, ma sœur, et ma mère.
\TextTitle{[Parabole du semeur et des quatre terrains]
\\(Mt. 13:1-17 ; Lu. 8:4-10)}
\Chap{4}
\VerseOne{}Jésus se mit de nouveau à enseigner près de la mer, et une grande foule s’étant assemblée auprès de lui, il monta dans une barque et s’assit dans la barque, sur la mer. Toute la foule était à terre sur le rivage de la mer.
\VS{2}Il leur enseignait beaucoup de choses en paraboles, et il leur dit dans son enseignement :
\VS{3}Ecoutez. Un semeur sortit pour semer.
\VS{4}Comme il semait, une partie de la semence tomba le long du chemin, et les oiseaux du ciel vinrent, et la mangèrent toute.
\VS{5}Une autre partie tomba dans les endroits pierreux, où elle n'avait pas beaucoup de terre ; elle leva aussitôt, parce qu'elle n'entrait pas profondément dans la terre ;
\VS{6}mais, quand le soleil parut, elle fut brûlée, et parce qu'elle n'avait pas de racine, elle se sécha.
\VS{7}Une autre partie tomba parmi les épines ; et les épines montèrent, et l'étouffèrent, et elle ne donna pas de fruit.
\VS{8}Une autre partie tomba dans la bonne terre, et donna du fruit qui montait et croissait en sorte qu'un grain en rapporta trente, un autre soixante, et un autre cent.
\VS{9}Et il leur dit : Que celui qui a des oreilles pour entendre, qu'il entende !
\VS{10}Lorsqu’il fut à l’écart, ceux qui étaient autour de lui avec les douze, l'interrogèrent touchant cette parabole.
\VS{11}Et il leur dit : Il vous est donné de connaître le mystère du Royaume de Dieu ; mais pour ceux qui sont dehors, tout se passe en paraboles,
\VS{12}afin qu'en voyant ils voient et n'aperçoivent point, et qu'en entendant ils entendent et ne comprennent point, de peur qu'ils ne se convertissent, et que leurs péchés ne leur soient pardonnés.
\TextTitle{[Explication de la parabole]
\\(Mt. 13:18-23 ; Lu. 8:11-15)}
\VS{13}Puis il leur dit : Ne comprenez-vous pas cette parabole ? Et comment donc comprendrez-vous toutes les paraboles ?
\VS{14}Le semeur c'est celui qui sème la parole.
\VS{15}Ceux qui sont le long du chemin, ce sont ceux en qui la parole est semée. Quand ils l’ont entendue, aussitôt Satan vient et enlève la parole qui a été semée dans leurs cœurs.
\VS{16}De même, ceux qui reçoivent la semence dans les endroits pierreux, ce sont ceux qui entendent la parole, ils la reçoivent aussitôt avec joie ;
\VS{17}mais ils n'ont pas de racine en eux-mêmes, ils croient pour un temps, et dès que survient une tribulation ou une persécution à cause de la parole, ils y trouvent une occasion de chute.
\VS{18}D’autres reçoivent la semence parmi les épines ; ce sont ceux qui entendent la parole,
\VS{19}mais en qui les soucis de ce monde, et la séduction des richesses, et les convoitises des autres choses étant entrées dans leurs esprits, étouffent la parole, et elle devient infructueuse.
\VS{20}Mais ceux qui ont reçu la semence dans la bonne terre, ce sont ceux qui entendent la parole, la reçoivent, et portent du fruit : L'un trente, et l'autre soixante, et l'autre cent{\FTNT{Voir commentaire Mt. 13:8.}}.
\TextTitle{[Parabole de la lampe]
\\(Mt. 5:15-16 ; Lu. 8:16-18 ; 11:33-36)}
\VS{21}Il leur dit encore : Apporte-t-on la lampe pour la mettre sous un boisseau, ou sous un lit ? N'est-ce pas pour la mettre sur un chandelier ?
\VS{22}Car il n'y a rien de secret qui ne doive être découvert, rien de caché qui ne doive être mis à jour.
\VS{23}Si quelqu'un a des oreilles pour entendre, qu'il entende.
\VS{24}Il leur dit encore : Prenez garde à ce que vous entendez. On vous mesurera avec la mesure dont vous vous serez servis, et on y ajoutera pour vous.
\VS{25}Car on donnera à celui qui a ; mais à celui qui n’a pas, on ôtera même ce qu’il a.
\TextTitle{[Parabole de la semence et de la croissance spirituelle]}
\VS{26}Il dit encore : Il en est du Royaume de Dieu comme quand un homme jette la semence en terre ;
\VS{27}qu’il dorme ou qu’il veille, nuit et jour, la semence germe et croit, sans qu'il sache comment.
\VS{28}Car la terre produit d'elle-même, premièrement l'herbe, ensuite l'épi, et puis le grain formé dans l'épi ;
\VS{29}et quand le fruit est mûr, on y met aussitôt la faucille, parce que la moisson est prête.
\TextTitle{[Parabole du grain de moutarde]
\\(Mt. 13:31-33 ; Lu. 13:18-19)}
\VS{30}Il dit encore : A quoi comparerons-nous le Royaume de Dieu, ou par quelle parabole le représenterons-nous ?
\VS{31}Il en est comme du grain de moutarde, qui, lorsqu'on le sème dans la terre, est la plus petite de toutes les semences qui sont jetées dans la terre.
\VS{32}Mais après qu'il a été semé, il monte et devient plus grand que toutes les autres plantes, et pousse de grandes branches, en sorte que les oiseaux du ciel peuvent faire leurs nids sous son ombre.
\VS{33}C’est par beaucoup de paraboles de cette sorte qu’il leur annonçait la parole de Dieu, selon qu'ils pouvaient l'entendre.
\VS{34}Et il ne leur parlait point sans paraboles ; mais en particulier, il expliquait tout à ses disciples.
\TextTitle{[Jésus apaise la tempête]
\\(Mt. 8:23-27 ; Lu. 8:22-25)}
\VS{35}Ce même jour sur le soir, Jésus leur dit : Passons sur l’autre bord.
\VS{36}Après avoir renvoyé la foule, ils l'emmenèrent avec eux, dans la barque ; et il y avait aussi d'autres petites barques avec lui.
\VS{37}Et il se leva un grand tourbillon, et les flots se jetaient dans la barque, de sorte qu'elle se remplissait déjà.
\VS{38}Et lui, il dormait à la poupe sur un oreiller. Ils le réveillèrent, et lui dirent : Maître, ne t’inquiètes-tu pas de ce que nous périssions ?
\VS{39}S’étant réveillé, il menaça le vent, et dit à la mer : Silence ! Tais-toi ! Et le vent cessa, et il eut un grand calme.
\VS{40}Puis il leur dit : Pourquoi avez-vous si peur ? Comment n'avez-vous point de foi ?
\VS{41}Et ils furent saisis d'une grande crainte, et ils se dirent les uns les autres : Quel est donc celui-ci, à qui obéissent le vent et la mer ?
\TextTitle{[Jésus-Christ délivre un possédé à Gadara]
\\(Mt. 8:28-34 ; Lu. 8:26-40)}
\Chap{5}
\VerseOne{}Ils arrivèrent sur l’autre bord de la mer, dans le pays des Gadaréniens.
\VS{2}Aussitôt que Jésus fut descendu de la barque, un homme possédé d’un esprit impur, sortit des sépulcres, et vint le rencontrer.
\VS{3}Cet homme avait sa demeure dans les sépulcres, et personne ne pouvait plus le lier, pas même avec des chaînes.
\VS{4}Car souvent, il avait eu les fers aux pieds et avait été lié de chaînes, mais il avait rompu les chaînes et brisé les fers, et personne ne pouvait le dompter.
\VS{5}Il était continuellement, nuit et jour sur les montagnes, et dans les sépulcres, criant et se meurtrissant avec des pierres.
\VS{6}Ayant vu Jésus de loin, il courut et se prosterna devant lui.
\VS{7}Et s’écria d’une voix forte : Qu'y a-t-il entre toi et moi, Jésus, Fils du Dieu Très-Haut ? Je te conjure au Nom de Dieu de ne pas me tourmenter.
\VS{8}Car Jésus lui disait : Sors de cet homme, esprit impur.
\VS{9}Alors il lui demanda : Quel est ton nom ? Légion{\FTNT{Une légion romaine contenait entre trois et six mille soldats. C’est autant de démons dont l’homme était possédé.}} est mon nom, lui répondit-il, car nous sommes plusieurs.
\VS{10}Et il le priait instamment de ne pas les envoyer hors de cette contrée.
\VS{11}Il y avait là, vers les montagnes, un grand troupeau de pourceaux qui paissaient.
\VS{12}Et tous ces démons le priaient en disant : Envoie-nous dans les pourceaux, afin que nous entrions en eux ; et aussitôt Jésus le leur permit.
\VS{13}Alors ces esprits impurs étant sortis, entrèrent dans les pourceaux, qui était environ deux mille, et le troupeau se précipita des pentes escarpées dans la mer ; et ils se noyèrent dans la mer.
\VS{14}Ceux qui paissaient les pourceaux s'enfuirent, et répandirent la nouvelle dans la ville et dans les campagnes.
\VS{15}Ceux de la ville sortirent pour voir ce qui était arrivé. Ils vinrent à Jésus et ils virent le démoniaque, celui qui avait eu la légion, assis et vêtu, et dans son bon sens ; et ils furent saisis de crainte.
\VS{16}Et ceux qui avaient vu le miracle leur racontèrent ce qui était arrivé au démoniaque et aux pourceaux.
\VS{17}Alors ils se mirent à supplier Jésus de quitter leur territoire.
\VS{18}Comme il montait dans la barque, celui qui avait été démoniaque le pria de lui permettre de rester avec lui.
\VS{19}Mais Jésus ne le lui permit pas, mais il lui dit : Va dans ta maison, vers les tiens, et raconte-leur les grandes choses que le Seigneur t'a faites, et comment il a eu pitié de toi.
\VS{20}Il s'en alla donc, et se mit à publier dans la Décapole les grandes choses que Jésus lui avait faites. Et tous furent dans l’étonnement.
\TextTitle{[La résurrection de la fille de Jaïrus et la guérison de la femme atteinte d'une perte de sang]
\\(Mt. 9:18-26 ; Lu. 8:41-56)}
\VS{21}Jésus dans la barque regagna l’autre rive, où une grande foule s’assembla près de lui. Il était près de la mer.
\VS{22}Alors vint un des chefs de la synagogue, nommé Jaïrus, qui l’ayant aperçu, se jeta à ses pieds,
\VS{23}et le pria instamment, en disant : Ma petite fille est à l'extrémité. Je te prie de venir et de lui imposer les mains, afin qu'elle soit guérie et qu'elle vive.
\VS{24}Jésus s'en alla donc avec lui. Et de grandes foules de gens le suivaient et le pressaient.
\VS{25}Or, il y avait une femme qui avait une perte de sang depuis douze ans,
\VS{26}et qui avait beaucoup souffert entre les mains de plusieurs médecins. Elle avait dépensé tout ce qu’elle possédait, sans avoir éprouvé aucun soulagement, mais était allée plutôt en empirant.
\VS{27}Ayant entendu parler de Jésus, elle vint dans la foule par derrière et toucha son vêtement.
\VS{28}Car elle disait : Si je puis seulement toucher ses vêtements, je serai guérie.
\VS{29}Au même instant, la perte de sang s'arrêta ; et elle sentit en son corps qu'elle était guérie de son fléau.
\VS{30}Et aussitôt Jésus connut en lui-même qu’une force était sortie de lui, et, se retournant vers la foule, il dit : Qui a touché mes vêtements ?
\VS{31}Et ses disciples lui dirent : Tu vois que la foule te presse, et tu dis : Qui m'a touché ?
\VS{32}Mais il regardait tout autour pour voir celle qui avait fait cela.
\VS{33}Alors la femme saisie de crainte et toute tremblante, sachant ce qui s’était passé en elle, vint et se jeta à ses pieds, et lui déclara toute la vérité.
\VS{34}Mais Jésus lui dit : Ma fille ! Ta foi t'a sauvée. Va en paix, et sois guérie de ton fléau.
\VS{35}Comme il parlait encore, il vint des gens de chez le chef de la synagogue, qui lui dirent : Ta fille est morte, pourquoi importuner davantage le Maître ?
\VS{36}Mais aussitôt que Jésus eut entendu cela, il dit au chef de la synagogue : Ne crains pas, crois seulement.
\VS{37}Et il ne permit à personne de le suivre, si ce n’est à Pierre, à Jacques, et à Jean, frère de Jacques.
\VS{38}Ils arrivèrent à la maison du chef de la synagogue, où Jésus vit le tumulte, c'est-à-dire ceux qui pleuraient et qui poussaient de grands cris.
\VS{39}Il entra, et leur dit : Pourquoi faites-vous tout ce bruit, et pourquoi pleurez-vous ? L’enfant n'est pas morte, mais elle dort.
\VS{40}Et ils se moquèrent de lui. Mais Jésus les ayant tous fait sortir, prit le père et la mère de la petite fille, et ceux qui étaient avec lui, et entra là où la petite fille était couchée.
\VS{41}Il la saisit par la main, et lui dit : Talitha koumi, ce qui signifie : Jeune fille, je te dis lève-toi.
\VS{42}Aussitôt la petite fille se leva, et se mit à marcher ; car elle était âgée de douze ans. Et ils furent dans un grand étonnement.
\VS{43}Jésus leur recommanda fort expressément que personne ne le sache ; et il dit qu'on donne à manger à la jeune fille.
\TextTitle{[Jésus à Nazareth]}
\Chap{6}
\VerseOne{}Jésus partit de là, et se rendit dans sa patrie. Ses disciples le suivirent.
\VS{2}Quand le jour du sabbat fut venu, il se mit à enseigner dans la synagogue. Et beaucoup de ceux qui l'entendaient étaient dans l'étonnement, et ils disaient : D'où lui viennent ces choses ? Et quelle est cette sagesse qui lui a été donnée, et comment de tels prodiges se font-ils par ses mains ?
\VS{3}N’est-ce pas le charpentier, le Fils de Marie, frère de Jacques, de Joses, de Jude, et de Simon ? Et ses sœurs ne sont-elles pas ici parmi nous ? Et ils étaient scandalisés à cause de lui.
\VS{4}Mais Jésus leur dit : Un prophète n'est méprisé que dans sa patrie, parmi ses parents et dans sa famille.
\VS{5}Et il ne put faire là aucun miracle, si ce n’est qu'il guérit quelques malades en leur imposant les mains.
\VS{6}Et il s'étonnait de leur incrédulité. Jésus parcourait les villages d'alentour, en enseignant.
\TextTitle{[Mission des apôtres]
\\(Mt. 10:1-42 ; Lu. 9:1-6)}
\VS{7}Alors il appela les douze, et commença à les envoyer deux à deux, en leur donnant pouvoir sur les esprits impurs.
\VS{8}Il leur prescrit de ne rien prendre pour le chemin, si ce n’est un bâton, et de ne porter ni sac, ni pain, ni monnaie dans leur ceinture ;
\VS{9}de chausser des sandales, et de ne pas porter deux tuniques.
\VS{10}Il leur disait aussi : Dans quelque maison que vous entriez, demeurez-y jusqu'à ce que vous partiez de là.
\VS{11}Et tous ceux qui ne vous recevront pas, et ne vous écouteront pas, en partant de là, secouez la poussière de vos pieds, en témoignage contre eux. Je vous le dis en vérité que ceux de Sodome et de Gomorrhe seront traités moins rigoureusement au jour du jugement que cette ville-là.
\VS{12}Ils partirent, et ils prêchèrent la repentance.
\VS{13}Ils chassèrent beaucoup de démons hors des possédés, et ils oignirent d'huile beaucoup de malades et les guérirent.
\TextTitle{[Jean-Baptiste décapité]
\\(Mt. 14:1-14 ; Lu. 9:7-9)}
\VS{14}Le roi Hérode entendit parler de Jésus, dont le nom était devenu fort célèbre, et il dit : C’est Jean-Baptiste qui est ressuscité des morts ; c'est pourquoi la puissance de faire des miracles agit puissamment en lui.
\VS{15}D’autres disaient : C'est Elie. Et les autres disaient : C'est un prophète, comme l’un des prophètes.
\VS{16}Mais Hérode en apprenant cela, disait : C'est Jean que j'ai fait décapiter, il est ressuscité des morts.
\VS{17}Car Hérode avait fait arrêter Jean, et l'avait fait lier en prison, à cause d'Hérodias, femme de Philippe son frère, parce qu'il l'avait prise en mariage.
\VS{18}Et que Jean lui disait : Il ne t'est pas permis d'avoir la femme de ton frère.
\VS{19}C'est pourquoi Hérodias était irritée contre Jean, et voulait le faire mourir, mais elle ne le pouvait pas ;
\VS{20}parce qu’Hérode craignait Jean, sachant que c'était un homme juste et saint ; il le protégeait, et, après l’avoir entendu, il faisait beaucoup selon ses avis, et l’écoutait avec plaisir.
\VS{21}Cependant, un jour propice arriva, lorsque Hérode à l’occasion du jour de sa naissance, donna un festin aux grands de sa cour, aux chefs militaires et aux principaux de la Galilée.
\VS{22}La fille d'Hérodias entra dans la salle ; elle dansa et plut à Hérode, et à ceux qui étaient à table avec lui. Le roi dit à la jeune fille : Demande-moi ce que tu voudras, et je te le donnerai.
\VS{23}Il ajouta avec serment : Tout ce que tu me demanderas, je te le donnerai, serait-ce la moitié de mon royaume.
\VS{24}Etant sortie, elle dit à sa mère : Que demanderai-je ? Et sa mère lui dit : La tête de Jean-Baptiste.
\VS{25}Et étant revenue en toute hâte vers le roi, et lui fit cette demande : Je veux que tu me donnes à l’instant sur un plat, la tête de Jean-Baptiste.
\VS{26}Le roi fut attristé, mais à cause de son serment et des convives, il ne voulut pas refuser.
\VS{27}Il envoya sur-le-champ l’un de ses gardes, avec ordre d'apporter la tête de Jean.
\VS{28}Le garde alla décapiter Jean dans la prison, et apporta sa tête sur un plat, et la donna à la jeune fille. Et la jeune fille la donna à sa mère.
\VS{29}Les disciples de Jean ayant appris cela, vinrent et emportèrent son corps, et le mirent dans un sépulcre.
\TextTitle{[Les apôtres rendent compte de leur mission à Jésus]
\\(Lu. 9:10)}
\VS{30}Les apôtres se rassemblèrent auprès de Jésus, et lui racontèrent tout ce qu'ils avaient fait et enseigné.
\VS{31}Jésus leur dit : Venez à l'écart dans un lieu désert, et reposez-vous un peu ; car il y avait beaucoup de gens qui allaient et qui venaient, de sorte qu'ils n'avaient même pas le temps de manger.
\TextTitle{[Multiplication des pains pour les cinq mille hommes]
\\(Mt. 14:12-21 ; Lu. 9:15-17 ; Jn. 6:1-14)}
\VS{32}Ils s'en allèrent donc dans une barque, à l’écart, dans un lieu désert.
\VS{33}Beaucoup de gens les virent s’en aller et les reconnurent, et de toutes les villes on accourut à pied et on les devança au lieu où ils se rendaient.
\VS{34}Quand il sortit, Jésus vit une grande foule, et fut ému de compassion pour elle, parce qu’ils étaient comme des brebis qui n'ont pas de pasteur ; et il se mit à leur enseigner plusieurs choses.
\VS{35}Comme il était déjà tard, ses disciples s'approchèrent de lui, en disant : Ce lieu est désert, et il est déjà tard,
\VS{36}renvoie-les, afin qu’ils s'en aillent dans les campagnes et dans les villages des environs pour s’acheter des pains ; car ils n'ont rien à manger.
\VS{37}Jésus leur répondit : Donnez-leur vous-mêmes à manger. Et ils lui dirent : Irions-nous acheter des pains pour deux cents deniers, et leur donnerions-nous à manger ?
\VS{38}Et il leur dit : Combien avez-vous de pains ? Allez voir. Et quand ils le surent, ils répondirent : Cinq, et deux poissons.
\VS{39}Alors il leur commanda de les faire tous asseoir par groupes sur l'herbe verte.
\VS{40}Et ils s'assirent par rangées de cent et de cinquante personnes.
\VS{41}Il prit les cinq pains et les deux poissons, et, levant les yeux vers le ciel, il bénit Dieu et rompit les pains, puis il les donna à ses disciples, afin qu'ils les distribuent à la foule. Il partagea aussi les deux poissons entre tous.
\VS{42}Tous mangèrent et furent rassasiés.
\VS{43}Et l’on emporta douze paniers pleins de morceaux de pains et de ce qui restait des poissons.
\VS{44}Ceux qui avaient mangé les pains étaient environ cinq mille hommes.
\TextTitle{[Jésus marche sur la mer]
\\(Mt. 14:22-33 ; Jn. 6:15-21)}
\VS{45}Et aussitôt après, il obligea ses disciples à monter dans la barque, et à le devancer sur l’autre bord, vers Bethsaïda, pendant que lui-même renverrait la foule.
\VS{46}Quand il l’eut renvoyée, il s'en alla sur la montagne pour prier.
\VS{47}Le soir étant venu, la barque était au milieu de la mer, et Jésus était seul à terre.
\VS{48}Il vit qu'ils avaient beaucoup de peine à ramer, parce que le vent leur était contraire. Vers la quatrième veille de la nuit, il alla vers eux marchant sur la mer, et il voulait les devancer.
\VS{49}Quand ils le virent marcher sur la mer, ils crurent que c’était un fantôme, et ils poussèrent des cris ;
\VS{50}car ils le voyaient tous, et ils furent troublés. Mais il leur parla aussitôt, et leur dit : Rassurez-vous, c'est moi. N’ayez pas peur.
\VS{51}Et il monta vers eux dans la barque, et le vent cessa. Et ils furent en eux-mêmes excessivement étonnés et remplis d’admiration.
\VS{52}Car ils n'avaient pas compris le miracle des pains, parce que leur cœur était endurci.
\TextTitle{[Jésus guérit les malades à Génésareth]
\\(Mt. 14:34-36)}
\VS{53}Après avoir traversé la mer, ils arrivèrent dans la contrée de Génésareth, où ils abordèrent.
\VS{54}Et dès qu’ils furent sortis de la barque, les gens, ayant aussitôt reconnu Jésus,
\VS{55}parcoururent tous les environs, et se mirent à lui apporter de tous côtés les malades sur de petits lits, partout où ils apprenaient qu'il était.
\VS{56}Et partout où il entrait, dans les villages, dans les villes, ou dans les campagnes, ils mettaient les malades dans les places publiques, et ils le priaient de leur permettre seulement de toucher le bord de son vêtement. Et tous ceux qui le touchaient étaient guéris.
\TextTitle{[Jésus condamne les traditions]
\\(Mt. 15:1-9)}
\Chap{7}
\VerseOne{}Alors les pharisiens, et quelques scribes qui étaient venus de Jérusalem, s'assemblèrent auprès de Jésus.
\VS{2}Ils virent quelques-uns de ses disciples mangeant du pain avec des mains impures, c'est-à-dire non lavées, et ils les blâmèrent.
\VS{3}Or, les pharisiens et tous les Juifs ne mangent pas sans s’être lavé leurs mains jusqu’au coude, conformément à la tradition des anciens.
\VS{4}Et quand ils reviennent de la place publique, ils ne mangent qu’après s’être lavés{\FTNT{Le verbe laver vient du grec «~baptizo~»~: «~Plonger, immerger, submerger, purifier en plongeant ou en submergeant, laver, rendre pur avec de l'eau, se baigner~» (Mt. 3:6-16~; Mt. 28:19~; Ac. 1:5~; Ac. 2:38~; 1 Co. 12:13, etc.) Jésus évoque ici les rites de purification chez les Juifs au premier siècle. A cette époque, le souci de purification avait conduit des groupes comme les pharisiens et les esséniens à multiplier les rites d’eau. Les découvertes de Qumran ont montré que les esséniens vivaient dans la hantise de ce qui aurait pu les rendre impurs. Ainsi, les rituels de purification avec de l’eau rythmaient la vie des juifs. A titre d’exemple, les jarres de Cana étaient utilisés à cet effet (Jn. 2:6).}}. Il y a plusieurs autres observances dont ils se sont chargés, comme le lavage des coupes, de cruches, des vases d'airain, et des lits.
\VS{5}Et les pharisiens et les scribes l'interrogèrent, en disant : Pourquoi tes disciples ne se conduisent-ils pas selon la tradition des anciens, mais prennent-ils leur repas sans se laver les mains ?
\VS{6}Jésus leur répondit : Hypocrites, Esaïe a bien prophétisé de vous, ainsi qu’il est écrit : Ce peuple m'honore des lèvres, mais leur cœur est éloigné de moi{\FTNT{Es. 29:13.}}.
\VS{7}C’est en vain qu’ils m'honorent, en enseignant des doctrines qui sont des commandements d'hommes.
\VS{8}Vous abandonnez le commandement de Dieu, et vous retenez la tradition des hommes, à savoir le lavage des cruches et des coupes, et vous faites beaucoup d'autres choses semblables.
\VS{9}Il leur dit aussi : Vous rejetez bien le commandement de Dieu, afin de garder votre tradition.
\VS{10}Car Moïse a dit : Honore ton père et ta mère ; et : celui qui maudira son père ou sa mère, sera puni de mort.
\VS{11}Mais vous, vous dites : Si quelqu'un dit à son père ou à sa mère : Tout ce dont je pourrais t’assister est corban, c’est-à-dire une offrande à Dieu, il ne sera point coupable.
\VS{12}Et vous ne lui permettez plus de rien faire pour son père ou pour sa mère,
\VS{13}anéantissant ainsi la parole de Dieu par votre tradition que vous avez établie. Et vous faites encore beaucoup d’autres choses semblables.
\TextTitle{[Le coeur humain]
\\(Mt. 15:10-20)}
\VS{14}Ensuite, ayant appelé la foule, il leur dit : Ecoutez-moi vous tous, et comprenez.
\VS{15}Il n’est hors de l’homme rien qui, entrant en lui, puisse le souiller ; mais ce qui sort de l’homme, c’est ce qui le souille.
\VS{16}Si quelqu'un a des oreilles pour entendre, qu'il entende.
\VS{17}Lorsqu’il fut entré dans la maison, loin de la foule, ses disciples l'interrogèrent sur cette parabole.
\VS{18}Et il leur dit : Vous aussi, êtes-vous sans intelligence ? Ne comprenez-vous pas que rien de ce qui du dehors entre dans l’homme ne peut le souiller ?
\VS{19}Car cela n'entre pas dans son cœur, mais dans son ventre, puis s’en va dans les lieux secrets, qui purifient le corps de tous les aliments.
\VS{20}Mais il leur dit : Ce qui sort de l'homme, c'est ce qui souille l'homme.
\VS{21}Car c’est du dedans, c'est-à-dire du cœur des hommes, que sortent les mauvaises pensées, les adultères, les fornications, les meurtres,
\VS{22}les vols, les cupidités, les méchancetés, la fraude, l'impudicité, le regard envieux, la calomnie, l’orgueil, la folie.
\VS{23}Tous ces maux sortent du dedans, et souillent l'homme.
\TextTitle{[Jésus et la femme syro-phénicienne]
\\(Mt. 15:21-28)}
\VS{24}Jésus, étant parti de là, s'en alla dans le territoire de Tyr et de Sidon. Il entra dans une maison, désirant que personne ne le sache ; mais il ne put rester caché.
\VS{25}Car une femme, dont la fille était possédée d'un esprit impur, ayant entendu parler de lui, vint et se jeta à ses pieds.
\VS{26}Cette femme était Grecque, Syro-Phénicienne d’origine. Elle le pria de chasser le démon hors de sa fille. Jésus lui dit :
\VS{27}Laisse premièrement les enfants se rassasier ; car il n'est pas raisonnable de prendre le pain des enfants, et de le jeter aux petits chiens.
\VS{28}Et elle lui répondit : Cela est vrai, Seigneur ! Cependant les petits chiens mangent sous la table les miettes que les enfants laissent tomber.
\VS{29}Alors il lui dit : A cause de cette parole va, le démon est sorti de ta fille.
\VS{30}Et quand elle rentra dans sa maison, elle trouva l’enfant couchée sur le lit, le démon étant sorti.
\TextTitle{[Jésus guérit un sourd-muet]
\\(Mt. 15:29-31)}
\VS{31}Jésus quitta le territoire de Tyr et de Sidon, et revint vers la mer de Galilée en traversant le pays de la Décapole.
\VS{32}On lui amena un sourd qui avait la parole empêchée, et on le pria de lui imposer les mains.
\VS{33}Jésus le prit à part, hors de la foule, lui mit les doigts dans les oreilles, et lui toucha la langue avec sa propre salive.
\VS{34}Puis, levant les yeux vers le ciel, il soupira, et lui dit : Ephphatha, c'est-à-dire : Ouvre-toi.
\VS{35}Aussitôt ses oreilles s'ouvrirent, et le lien de sa langue se délia, et il parla aisément.
\VS{36}Jésus leur recommanda de ne le dire à personne ; mais plus il le leur recommanda, plus ils le publièrent.
\VS{37}Et ils en étaient extrêmement étonnés, et disaient : Il fait tout à merveille ; même il fait entendre les sourds, et parler les muets.
\TextTitle{[Seconde multiplication des pains]
\\(Mt. 15:32-39)}
\Chap{8}
\VerseOne{}En ces jours-là, une grande foule s’était de nouveau réunie et n’avait rien à manger. Jésus appela ses disciples, et leur dit :
\VS{2}Je suis ému de compassion pour cette foule, car il y a déjà trois jours qu'ils sont près de moi, et ils n'ont rien à manger.
\VS{3}Si je les renvoie chez eux à jeun, ils tomberont en défaillance en chemin, car quelques-uns d'eux sont venus de loin.
\VS{4}Ses disciples lui répondirent : Comment pourrait-t-on les rassasier de pains, ici, dans un désert ?
\VS{5}Jésus leur demanda : Combien avez-vous de pains ? Sept lui répondirent-ils.
\VS{6}Alors il ordonna à la foule de s'asseoir par terre, et il prit les sept pains, et après avoir béni Dieu, il les rompit, et les donna à ses disciples pour les distribuer ; et ils les distribuèrent à la foule.
\VS{7}Ils avaient aussi quelques petits poissons ; et après avoir béni Dieu, il les fit aussi distribuer.
\VS{8}Ils mangèrent, et furent rassasiés ; et l’on remporta sept corbeilles pleines des morceaux qui restaient.
\VS{9}Ceux qui avaient mangé étaient environ quatre mille. Ensuite Jésus les renvoya.
\TextTitle{[L'enseignement corrompu des pharisiens]
\\(Mt. 16:1-12)}
\VS{10}Aussitôt après, il monta dans la barque avec ses disciples, et se rendit dans la contrée de Dalmanutha.
\VS{11}Les pharisiens survinrent, se mirent à discuter avec lui, et pour l'éprouver, lui demandèrent un signe venant du ciel.
\VS{12}Alors, Jésus soupirant profondément en son esprit, dit : Pourquoi cette génération demande-t-elle un signe ? Je vous le dis en vérité, il ne sera point donné de signe à cette génération.
\VS{13}Puis il les quitta, et remonta dans la barque, pour passer à l'autre rivage.
\VS{14}Les disciples avaient oublié de prendre des pains ; et ils n'en avaient qu'un seul avec eux dans la barque.
\VS{15}Jésus leur fit cette recommandation : Gardez-vous avec soin du levain des pharisiens et du levain d'Hérode.
\VS{16}Ils raisonnaient entre eux, disant : C'est parce que nous n'avons pas de pains.
\VS{17}Jésus, le sachant, leur dit : Pourquoi discourez-vous sur ce que vous n'avez pas de pains ? N’entendez-vous pas encore, et ne comprenez-vous pas ?
\VS{18}Avez-vous encore votre cœur endurci ? Ayant des yeux, ne voyez-vous point ? Ayant des oreilles, n'entendez-vous point ? Et n'avez-vous point de mémoire ?
\VS{19}Quand j’ai rompu les cinq pains pour les cinq mille hommes, combien de paniers pleins de morceaux avez-vous emportés ? Douze, lui répondirent-ils.
\VS{20}Et quand j’ai rompu les sept pains pour quatre mille hommes, combien de corbeilles pleines de morceaux avez-vous emportées ? Sept, répondirent-ils.
\VS{21}Et il leur dit : Comment n'avez-vous pas d'intelligence ?
\TextTitle{[Jésus guérit un aveugle]}
\VS{22}Ils se rendirent à Bethsaïda, et on lui présenta un aveugle, qu’on le pria de toucher.
\VS{23}Alors il prit la main de l'aveugle, et le conduisit hors du village ; puis il lui mit de la salive sur les yeux, lui imposa les mains, et lui demanda s'il voyait quelque chose.
\VS{24}Et cet homme ayant regardé, dit : Je vois des hommes qui marchent, et qui me paraissent comme des arbres.
\VS{25}Jésus lui mit de nouveau les mains sur les yeux, et lui dit de regarder ; et il fut rétabli, et les voyait tous distinctement.
\VS{26}Puis il le renvoya dans sa maison, en lui disant : N'entre pas dans le village, et ne le dis à personne du village.
\TextTitle{[Pierre reconnaît Jésus comme le Messie]
\\(Mt. 16:13-16 ; Lu. 9:18-21 ; Jn. 6:67-71)}
\VS{27}Jésus s’en alla, avec ses disciples, dans les villages de Césarée de Philippe, et sur le chemin il interrogea ses disciples, leur disant : Qui dit-on que je suis ?
\VS{28}Ils répondirent : Les uns disent que tu es Jean-Baptiste ; les autres, Elie ; et les autres, l'un des prophètes.
\VS{29}Alors il leur dit : Et vous, qui dites-vous que je suis ? Pierre lui répondit : Tu es le Christ.
\VS{30}Et il leur défendit très sévèrement de ne dire cela de lui à personne.
\VS{31}Alors il commença à leur enseigner qu'il fallait que le Fils de l'homme souffre beaucoup, qu'il soit rejeté par les anciens, par les principaux sacrificateurs et par les scribes, qu'il soit mis à mort, et qu'il ressuscite trois jours après.
\VS{32}Il leur tenait ces discours ouvertement. Et Pierre l’ayant pris à part, se mit à le reprendre.
\VS{33}Mais Jésus, se retournant et regardant ses disciples, réprimanda Pierre en lui disant : Va arrière de moi, Satan ! Car tu ne comprends pas les choses de Dieu, mais celles des hommes.
\TextTitle{[La consécration du disciple]
\\(Mt. 16:24-28 ; Lu. 9:23-26)}
\VS{34}Puis, ayant appelé la foule et ses disciples, il leur dit : Si quelqu’un veut venir après moi, qu'il renonce à lui-même, qu'il se charge de sa croix, et qu’il me suive.
\VS{35}Car quiconque voudra sauver son âme, la perdra ; mais quiconque perdra son âme pour l'amour de moi et de l'Evangile, celui-là la sauvera.
\VS{36}Car que sert-il à un homme de gagner tout le monde, s'il perd son âme ?
\VS{37}Que donnerait un homme en échange de son âme ?
\VS{38}Car quiconque aura honte de moi et de mes paroles au milieu de cette génération adultère et pécheresse, le Fils de l'homme aura aussi honte de lui, quand il viendra dans la gloire de son Père avec les saints anges.
\TextTitle{[La transfiguration]
\\(Mt. 17:1-8 ; Lu. 9:27-36)}
\Chap{9}
\VerseOne{}Il leur disait aussi : Je vous le dis en vérité, quelques-uns de ceux qui sont ici présents, ne mourront point qu’ils n’aient vu le Royaume de Dieu venir avec puissance{\FTNT{Voir commentaire Mt. 16:28.}}.
\VS{2}Six jours après, Jésus prit avec lui Pierre, Jacques et Jean, et les conduisit seuls à l'écart sur une haute montagne. Il fut transfiguré devant eux,
\VS{3}ses vêtements devinrent resplendissants, et blancs comme de la neige, tels qu'il n’est pas de foulon sur la terre qui puisse blanchir ainsi.
\VS{4}Et en même temps leur apparurent Moïse et Elie, qui s’entretenaient avec Jésus.
\VS{5}Alors Pierre prenant la parole, dit à Jésus : Rabbi, il est bon que nous soyons ici ; faisons donc trois tentes, une pour toi, une pour Moïse, et une pour Elie.
\VS{6}Car il ne savait pas quoi dire, ils étaient épouvantés.
\VS{7}Une nuée vint les couvrir de son ombre, et de la nuée sortit une voix : Celui-ci est mon Fils bien-aimé, écoutez-le.
\VS{8}Aussitôt les disciples regardèrent tout autour, et ils ne virent que Jésus seul avec eux.
\VS{9}Comme ils descendaient de la montagne, Jésus leur recommanda expressément de ne raconter à personne ce qu'ils avaient vu, jusqu’à ce que le Fils de l'homme soit ressuscité des morts.
\VS{10}Ils retinrent cette parole, se demandant entre eux ce que c'était que ressusciter des morts.
\VS{11}Les disciples l'interrogèrent, disant : Pourquoi les scribes disent-ils qu'il faut qu'Elie vienne premièrement ?
\VS{12}Il leur répondit : Il est vrai, Elie viendra premièrement, et rétablira toutes choses. Et pourquoi est-il écrit du Fils de l'homme qu’il doit beaucoup souffrir et être méprisé ?
\VS{13}Mais je vous dis qu’Elie est venu, et qu'ils lui ont fait tout ce qu’ils ont voulu, selon qu’il est écrit de lui.
\TextTitle{[Incapacité des disciples et la toute-puissance de Jésus-Christ]
\\(Mt. 17:14-21 ; Lu. 9:37-43)}
\VS{14}Lorsqu’il fut arrivé près des disciples, il vit une grande foule autour d’eux, et des scribes qui discutaient avec eux.
\VS{15}Dès que la foule vit Jésus, elle fut saisie d'étonnement, et accourut pour le saluer.
\VS{16}Alors il demanda aux scribes : De quoi discutez-vous avec eux ?
\VS{17}Et un homme de la foule prenant la parole, dit : Maître, je t'ai amené mon fils qui est possédé d’un esprit muet.
\VS{18}En quelque lieu qu’il le saisisse, il le jette par terre ; l’enfant écume, grince des dents, et devient tout raide. J’ai prié tes disciples de le chasser, mais ils n'ont pas pu.
\VS{19}Alors Jésus leur répondit : Ô génération incrédule ! Jusqu’à quand serai-je avec vous ? Jusqu’à quand vous supporterai-je ? Amenez-le-moi. Ils le lui amenèrent.
\VS{20}Et aussitôt que l’enfant vit Jésus, l'esprit l'agita sur-le-champ avec violence ; il tomba par terre, et se roulait en écumant.
\VS{21}Jésus demanda au père de l'enfant : Combien y a-t-il de temps que cela lui arrive ? Et il dit : Dès son enfance.
\VS{22}Et souvent l’esprit l’a jeté dans le feu et dans l'eau pour le faire périr. Mais si tu peux quelque chose, secours-nous, aie compassion de nous.
\VS{23}Alors Jésus lui dit : Si tu peux croire, tout est possible à celui qui croit.
\VS{24}Et aussitôt le père de l'enfant s'écriant avec larmes : Je crois, Seigneur ! Secours-moi dans mon incrédulité.
\VS{25}Jésus voyant accourir la foule, reprit sévèrement l’esprit impur, et lui dit : Esprit muet et sourd, je te l’ordonne, sors de cet enfant, et n'y rentre plus !
\VS{26}Et le démon sortit, en poussant des cris, et en l’agitant avec une grande violence. L’enfant devint comme mort, de sorte que plusieurs disaient qu’il était mort.
\VS{27}Mais Jésus, l'ayant pris par la main, le fit lever. Et il se tint debout.
\VS{28}Quand Jésus fut entré dans la maison, ses disciples lui demandèrent en particulier : Pourquoi n’avons-nous pas pu chasser cet esprit ?
\VS{29}Il leur répondit : Cette sorte de démons ne peut sortir que par la prière et par le jeûne.
\TextTitle{[Jésus annonce sa mort et sa résurrection]
\\(Mt. 17:22-23 ; Lu. 9:44-45)}
\VS{30}Puis étant partis de là, ils traversèrent la Galilée. Jésus ne voulait pas qu’on le sache.
\VS{31}Car il enseignait ses disciples, et il leur dit : Le Fils de l'homme va être livré entre les mains des hommes, et ils le feront mourir, mais après qu'il aura été mis à mort, il ressuscitera le troisième jour.
\VS{32}Mais ils ne comprenaient point ce discours, et ils craignaient de l'interroger.
\TextTitle{[L'humilité, secret de la vraie grandeur]
\\(Mt. 18:1-6 ; Lu. 9:46-48)}
\VS{33}Après ces choses il vint à Capernaüm, et quand il fut arrivé à la maison, il leur demanda : De quoi discutiez-vous ensemble en chemin ?
\VS{34}Mais ils gardèrent le silence, car ils avaient discuté entre eux en chemin sur celui qui serait le plus grand.
\VS{35}Alors il s’assit, appela les douze, et leur dit : Si quelqu'un veut être le premier parmi vous, il sera le dernier de tous, et le serviteur de tous.
\VS{36}Et ayant pris un petit enfant, il le mit au milieu d'eux, et après l'avoir pris entre ses bras, il leur dit :
\VS{37}Quiconque reçoit en mon Nom un de ces petits enfants, me reçoit ; et quiconque me reçoit, ce n'est pas moi qu’il reçoit, mais celui qui m'a envoyé.
\TextTitle{[Jésus condamne l'esprit sectaire]
\\(Lu. 9:49-50)}
\VS{38}Alors Jean prit la parole, et dit : Maître, nous avons vu quelqu'un qui chasse les démons en ton Nom et qui ne nous suit pas, et nous l'en avons empêché, parce qu'il ne nous suit pas.
\VS{39}Mais Jésus leur dit : Ne l'en empêchez pas ; car il n’est personne qui, faisant un miracle en mon Nom, puisse aussitôt après parler mal de moi.
\VS{40}Qui n'est pas contre nous est pour nous.
\VS{41}Et quiconque vous donnera à boire un verre d'eau en mon Nom, parce que vous êtes à Christ, je vous le dis en vérité, il ne perdra point sa récompense.
\TextTitle{[Avertissement de Jésus concernant les occasions de chute]}
\VS{42}Mais quiconque scandalisera un de ces petits qui croient en moi, il vaudrait mieux pour lui qu'on lui mette une pierre de moulin au cou, et qu'on le jette dans la mer.
\VS{43}Si ta main est pour toi une occasion de chute, coupe-la ; mieux vaut pour toi entrer manchot dans la vie, que d'avoir les deux mains, et d’aller dans la géhenne, dans le feu qui ne s'éteint point ;
\VS{44}là où leur ver ne meurt point, et le feu ne s'éteint point.
\VS{45}Si ton pied est pour toi une occasion de chute, coupe-le ; mieux vaut pour toi entrer boiteux dans la vie, que d'avoir les deux pieds, et d’être jeté dans la géhenne, dans le feu qui ne s'éteint point ;
\VS{46}là où leur ver ne meurt point, et où le feu ne s'éteint point.
\VS{47}Si ton œil est pour toi une occasion de chute, arrache-le ; mieux vaut pour toi entrer dans le Royaume de Dieu n'ayant qu'un œil, que d'avoir les deux yeux, et d’être jeté dans le feu de la géhenne,
\VS{48}où leur ver ne meurt point, et où le feu ne s'éteint point.
\VS{49}Car chacun sera salé de feu ; et toute offrande sera salée de sel.
\VS{50}Le sel est une bonne chose ; mais si le sel devient sans saveur, avec quoi lui rendra-t-on sa saveur ?
\VS{51}Ayez du sel en vous-mêmes, et soyez en paix les uns avec les autres.
\TextTitle{[Enseignement de Jésus sur le mariage et le divorce]
\\(Mt. 5:31-32 ; 19:1-9 ; Lu. 16:18 ; Ro. 7:1-3 ; 1 Co. 7:10-16)}
\Chap{10}
\VerseOne{}Jésus, étant parti de là, se rendit dans le territoire de la Judée, au-delà du Jourdain. La foule s’assembla de nouveau auprès de lui, et selon sa coutume, il se mit à l’enseigner.
\VS{2}Alors les pharisiens vinrent à lui, et, pour l'éprouver, ils lui demandèrent s’il est permis à un homme de répudier sa femme.
\VS{3}Il répondit et leur dit : Qu'est-ce que Moïse vous a prescrit ?
\VS{4}Moïse, dirent-ils, a permis d'écrire une lettre de divorce, et de répudier ainsi sa femme{\FTNT{De. 24:1.}}.
\VS{5}Et Jésus leur répondit : C’est à cause de la dureté de votre cœur que Moïse vous a donné ce commandement.
\VS{6}Mais au commencement de la création, Dieu fit l’homme et la femme.
\VS{7}C'est pourquoi l'homme quittera son père et sa mère, et s'attachera à sa femme,
\VS{8}et les deux deviendront une seule chair. Ainsi, ils ne sont plus deux, mais ils sont une seule chair.
\VS{9}Que l'homme donc ne sépare pas ce que Dieu a mis ensemble sous un joug{\FTNT{Voir commentaire Mt. 19:6.}}.
\VS{10}Lorsqu’ils furent dans la maison, ses disciples l'interrogèrent encore là-dessus.
\VS{11}Il leur dit : Celui qui répudie sa femme et qui en épouse une autre, commet un adultère à son égard.
\VS{12}Pareillement si la femme répudie son mari, et se marie à un autre, elle commet un adultère.
\TextTitle{[Jésus bénit les petits enfants]
\\(Mt. 19:13-15 ; Lu. 18:15-17)}
\VS{13}On lui amena de petits enfants afin qu'il les touche. Mais les disciples reprirent ceux qui les amenaient.
\VS{14}Jésus, voyant cela, fut indigné, et leur dit : Laissez venir à moi les petits enfants et ne les en empêchez point, car le Royaume de Dieu appartient à ceux qui leur ressemblent.
\VS{15}Je vous le dis en vérité, quiconque ne recevra pas comme un petit enfant le Royaume de Dieu, il n'y entrera point.
\VS{16}Après les avoir pris dans ses bras, il les bénit, en leur imposant les mains.
\TextTitle{[Le jeune homme riche]
\\(Mt. 19:16-30 ; Lu. 18:18-30 ; Lu. 10:25-37)}
\VS{17}Comme Jésus se mettait en chemin, un homme accourut, et se jetant à genoux devant lui : Bon Maître, lui demanda-t-il, que dois-je faire pour hériter la vie éternelle ?
\VS{18}Jésus lui répondit : Pourquoi m'appelles-tu bon ? Il n'y a de bon que Dieu seul{\FTNT{La même histoire est racontée en Lu. 18:18 qui précise que c’était un chef qui avait interrogé Jésus. La réponse du Seigneur est ironique. Jésus aurait aussi pu lui poser la question comme suit : «~Puisque tu penses que je ne suis qu’un simple homme, pourquoi m’appelles-tu bon ?~».}}.
\VS{19}Tu connais les commandements : Ne commets point d’adultère ; ne tue point ; ne dérobe point ; ne dis point de faux témoignage ; ne fais aucun tort à personne ; honore ton père et ta mère.
\VS{20}Il lui répondit : Maître, j'ai observé toutes ces choses dès ma jeunesse.
\VS{21}Jésus, l’ayant regardé, l'aima, et lui dit : Il te manque une chose : Va, et vends tout ce que tu as, et donne-le aux pauvres, et tu auras un trésor dans le ciel. Puis, viens, et suis-moi en te chargeant de ta croix.
\VS{22}Mais, affligé de cette parole, il s'en alla tout triste, parce qu'il avait de grands biens.
\TextTitle{[Tout est possible à Dieu]}
\VS{23}Alors Jésus, ayant regardé autour de lui, dit à ses disciples : Qu’il est difficile à ceux qui ont des richesses d’entrer dans le Royaume de Dieu.
\VS{24}Ses disciples furent étonnés de ces paroles ; mais Jésus reprenant la parole, leur dit : Mes enfants, qu'il est difficile à ceux qui se confient dans les richesses d'entrer dans le Royaume de Dieu !
\VS{25}Il est plus facile à un chameau de passer par le trou d'une aiguille{\FTNT{Voir commentaire Mt. 19:24.}}, qu’à un riche d’entrer dans le Royaume de Dieu.
\VS{26}Les disciples furent encore plus étonnés, et ils se dire les uns les autres : Et qui peut être sauvé ?
\VS{27}Mais Jésus les ayant regardés, leur dit : Cela est impossible aux hommes, mais non à Dieu ; car tout est possible à Dieu.
\TextTitle{[La fidélité à Jésus-Christ sera récompensée]}
\VS{28}Alors Pierre se mit à lui dire : Voici, nous avons tout quitté et nous t'avons suivi.
\VS{29}Et Jésus répondit, disant : Je vous le dis en vérité, il n’est personne qui, ayant quitté pour l'amour de moi et de l’Evangile, sa maison, ou ses frères, ou ses sœurs, ou son père, ou sa mère, ou sa femme, ou ses enfants, ou ses terres,
\VS{30}ne reçoive au centuple, présentement dans ce temps-ci, des maisons, des frères, des sœurs, des mères, des enfants, et des terres, avec des persécutions ; et dans le siècle à venir, la vie éternelle.
\VS{31}Plusieurs des premiers seront les derniers ; et plusieurs des derniers seront les premiers.
\TextTitle{[Jésus annonce sa mort et sa résurrection]
\\(Mt. 20:17-19 ; Lu. 18:31-34)}
\VS{32}Ils étaient en chemin, pour monter à Jérusalem, et Jésus allait devant eux. Les disciples étaient troublés, et le suivaient avec crainte. Et Jésus prit de nouveau à l'écart les douze, et commença à leur déclarer ce qui devait lui arriver,
\VS{33}disant : Voici, nous montons à Jérusalem, et le Fils de l'homme sera livré aux principaux sacrificateurs et aux scribes. Ils le condamneront à mort, et le livreront aux gentils
\VS{34}qui se moqueront de lui, le battront de verges, cracheront sur lui, et le feront mourir ; et il ressuscitera trois jours après.
\TextTitle{[Jésus répond à la question de Jacques et Jean]}
\VS{35}Alors Jacques et Jean, fils de Zébédée, s’approchèrent de Jésus et lui dirent : Maître, nous voudrions que tu fasses pour nous ce que nous te demanderons.
\VS{36}Il leur dit : Que voulez-vous que je fasse pour vous ?
\VS{37}Et ils lui dirent : Accorde-nous, lui dirent-ils, d’être assis l’un à ta droite et l’autre à ta gauche, quand tu seras dans ta gloire.
\VS{38}Jésus leur dit : Vous ne savez pas ce que vous demandez. Pouvez-vous boire la coupe que je dois boire, et être baptisés du baptême dont je dois être baptisé ?
\VS{39}Ils lui répondirent : Nous le pouvons. Et Jésus leur répondit : Il est vrai que vous boirez la coupe que je dois boire, et que vous serez baptisés du baptême dont je dois être baptisé ;
\VS{40}mais pour ce qui est d'être assis à ma droite et à ma gauche, ce n'est pas à moi de l’accorder ; mais cela ne sera donné qu’à ceux à qui cela est préparé.
\VS{41}Les dix autres, ayant entendu cela, commencèrent à s’indigner contre Jacques et Jean.
\VS{42}Jésus les appela et leur dit : Vous savez que ceux qu’on regarde comme les chefs des nations les dominent, et que les grands les asservissent.
\VS{43}Il n'en sera pas de même parmi vous. Mais quiconque veut être le plus grand parmi vous, qu’il soit votre serviteur,
\VS{44}et quiconque veut être le premier parmi vous, qu’il soit l’esclave de tous.
\VS{45}Car le Fils de l'homme est venu, non pour être servi, mais pour servir et donner sa vie en rançon pour plusieurs.
\TextTitle{[Jésus guérit l'aveugle Bartimée]
\\(Mt. 20:29-34 ; Lu. 18:35-43)}
\VS{46}Ils arrivèrent à Jéricho. Et lorsque Jésus en sortit, avec ses disciples et une grande foule, un aveugle, appelé Bartimée, c'est-à-dire le fils de Timée, était assis au bord du chemin et mendiait.
\VS{47}Il entendit que c'était Jésus de Nazareth, et il se mit à crier et à dire : Jésus, Fils de David, aie pitié de moi !
\VS{48}Plusieurs le reprenaient pour le faire taire ; mais il criait beaucoup plus fort : Fils de David, aie pitié de moi !
\VS{49}Jésus s’arrêta, et dit : Appelez-le. Ils appelèrent l’aveugle en lui disant : Prends courage, lève-toi, il t'appelle.
\VS{50}L’aveugle jeta son manteau, il se leva et vint vers Jésus.
\VS{51}Jésus, prenant la parole, lui dit : Que veux-tu que je te fasse ? Et l'aveugle lui dit : Maître, que je recouvre la vue.
\VS{52}Et Jésus lui dit : Va, ta foi t'a sauvé.
\VS{53}Et aussitôt il recouvra la vue, et suivit Jésus dans le chemin.
\TextTitle{[Entrée de Jésus à Jérusalem]
\\(Mt. 21:1-11 ; Lu. 19:28-40 ; Jn. 12:12-19 ; Za. 9:9)}
\Chap{11}
\VerseOne{}Lorsqu’ils approchaient de Jérusalem, et qu’ils furent près de Bethphagé et de Béthanie, vers le Mont des oliviers, Jésus envoya deux de ses disciples,
\VS{2}en leur disant : Allez au village qui est devant vous. Dès que vous y serez entrés, vous trouverez un ânon attaché, sur lequel aucun homme ne s’est encore assis. Détachez-le, et amenez-le.
\VS{3}Si quelqu'un vous dit : Pourquoi faites-vous cela ? Dites que le Seigneur en a besoin ; et à l’instant, il le laissera venir ici.
\VS{4}Ils partirent donc, et trouvèrent l'ânon qui était attaché dehors, près d’une porte, au contour du chemin, et ils le détachèrent.
\VS{5}Quelques-uns de ceux qui étaient là leur dirent : Pourquoi détachez-vous cet ânon ?
\VS{6}Ils leur répondirent comme Jésus l’avait ordonné ; et on les laissa faire.
\VS{7}Ils amenèrent donc l'ânon à Jésus, sur lequel ils jetèrent leurs vêtements, et Jésus s’assit dessus.
\VS{8}Beaucoup étendirent leurs vêtements sur le chemin, et d'autres des branches qu’ils coupèrent dans les champs.
\VS{9}Ceux qui allaient devant, et ceux qui suivaient, criaient en disant : Hosanna ! Béni soit celui qui vient au Nom du Seigneur !
\VS{10}Béni soit le règne de David notre père, le règne qui vient au Nom du Seigneur ! Hosanna dans les lieux très hauts !
\VS{11}Jésus entra ainsi à Jérusalem, dans le temple. Quand il eut tout considéré, il était déjà tard, il sortit pour aller à Béthanie avec les douze.
\TextTitle{[Le figuier sans fruit]
\\(Mt. 21:18-22)}
\VS{12}Le lendemain, après qu’ils furent sortis de Béthanie, Jésus eut faim.
\VS{13}Apercevant de loin un figuier qui avait des feuilles, il alla voir s'il y trouverait quelque chose ; et s’en étant approché, il ne trouva que des feuilles, car ce n'était pas la saison des figues.
\VS{14}Jésus prenant la parole dit au figuier : Que jamais personne ne mange de ton fruit ! Et ses disciples l'entendirent.
\TextTitle{[Jésus chasse les marchands du temple]
\\(Mt. 21:12-13 ; Lu. 19:45-46 ; Jn. 2:13-16)}
\VS{15}Ils arrivèrent donc à Jérusalem, et Jésus entra dans le temple. Il se mit à chasser dehors ceux qui vendaient, et ceux qui achetaient dans le temple, et il renversa les tables des changeurs, et les sièges de ceux qui vendaient des pigeons.
\VS{16}Il ne laissait personne porter aucun objet à travers le temple.
\VS{17}Et il les enseignait, en leur disant : N'est-il pas écrit : Ma maison sera appelée une maison de prière par toutes les nations ? Mais vous, vous en avez fait une caverne de voleurs{\FTNT{Jé. 7:11.}}.
\VS{18}Les scribes et les principaux sacrificateurs l’ayant entendu, cherchèrent les moyens de le faire périr ; car ils le craignaient, parce que toute la foule était frappée de sa doctrine.
\VS{19}Le soir étant venu, Jésus sortit de la ville.
\TextTitle{[La prière de la foi]
\\(1 Jn. 5:14-15)}
\VS{20}Le matin, en passant, les disciples virent le figuier séché jusqu’aux racines.
\VS{21}Pierre s'étant souvenu de ce qui s'était passé, dit à Jésus : Maître, voici, le figuier que tu as maudit a séché.
\VS{22}Jésus répondant, leur dit : Ayez foi en Dieu.
\VS{23}Je vous le dis en vérité, si quelqu’un dit à cette montagne : Ôte-toi de là et jette-toi dans la mer, et s’il ne doute point en son cœur, mais croit que ce qu’il a dit arrive, il le verra s’accomplir.
\VS{24}C'est pourquoi je vous dis : Tout ce que vous demanderez en priant, croyez que vous l’avez reçu, et vous le verrez s’accomplir.
\TextTitle{[Le pardon]}
\VS{25}Mais quand vous vous présenterez pour faire votre prière, si vous avez quelque chose contre quelqu'un, pardonnez-lui, afin que votre Père qui est dans les cieux vous pardonne aussi vos fautes.
\VS{26}Mais si vous ne pardonnez pas, votre Père qui est dans les cieux ne vous pardonnera point aussi vos fautes.
\TextTitle{[L'autorité de Jésus-Christ mise en doute]
\\(Mt. 21:23-27 ; Lu. 20:1-8)}
\VS{27}Ils se rendirent de nouveau à Jérusalem, et pendant que Jésus marchait dans le temple, les principaux sacrificateurs, les scribes et les anciens vinrent à lui,
\VS{28}et lui dirent : Par quelle autorité fais-tu ces choses, et qui t'a donné cette autorité pour faire les choses que tu fais ?
\VS{29}Jésus leur répondit : Je vous demanderai aussi une chose, et répondez-moi ; puis je vous dirai par quelle autorité je fais ces choses.
\VS{30}Le baptême de Jean venait-il du ciel, ou des hommes ? Répondez-moi.
\VS{31}Et ils raisonnaient entre eux, disant : Si nous disons, du ciel : Il nous dira : Pourquoi donc n’avez-vous pas cru en lui ?
\VS{32}Et si nous disons : Des hommes, nous avons à craindre le peuple, car tous croyaient que Jean était un vrai prophète.
\VS{33}Alors ils répondirent à Jésus : Nous ne savons pas. Et Jésus leur dit : Moi non plus je ne vous dirai pas par quelle autorité je fais ces choses.
\TextTitle{[Parabole des vignerons]
\\(Mt. 21:33-46 ; Lu. 20:9-18 ; Es. 5:1-7)}
\Chap{12}
\VerseOne{}Jésus se mit à leur parler en paraboles : Quelqu'un, dit-il, planta une vigne, et l'environna d'une haie, creusa un pressoir, et bâtit une tour ; puis il la loua à des vignerons, et quitta le pays.
\VS{2}Au temps de la récolte, il envoya un serviteur vers les vignerons, pour recevoir d'eux le fruit de la vigne.
\VS{3}S’étant saisis de lui, ils le battirent, et le renvoyèrent à vide.
\VS{4}Il envoya de nouveau un autre serviteur vers eux. Ils lui jetèrent des pierres, le frappèrent à la tête, et le renvoyèrent après l'avoir outragé.
\VS{5}Il en envoya de nouveau un troisième, qu’ils tuèrent ; et plusieurs autres, et ils battirent les uns, et tuèrent les autres.
\VS{6}Il avait encore un fils, son bien-aimé, il le leur envoya le dernier, disant : Ils auront du respect pour mon fils.
\VS{7}Mais ces vignerons dirent entre eux : Voici l'héritier, venez, tuons-le, et l'héritage sera à nous.
\VS{8}Ils se saisirent de lui, le tuèrent, et le jetèrent hors de la vigne.
\VS{9}Que fera donc le maître de la vigne ? Il viendra, et fera périr ces vignerons, et donnera la vigne à d'autres.
\VS{10}N'avez-vous pas lu cette parole de l’Ecriture ? La pierre qu’ont rejetée ceux qui bâtissaient est devenue la principale de l’angle{\FTNT{Jésus-Christ, la pierre angulaire : Ps. 118:22-23 ; Es. 8:13-17.}} ?
\VS{11}Cela a été fait par le Seigneur, et c'est une chose merveilleuse à nos yeux.
\VS{12}Alors ils cherchaient à se saisir de lui, mais ils craignirent la foule. Ils avaient compris que c’était pour eux que Jésus avait dit cette parabole. Et ils le quittèrent, et s’en allèrent.
\TextTitle{[Le tribut dû à César]
\\(Mt. 22:15-22 ; Lu. 20:19-26)}
\VS{13}Mais ils envoyèrent quelques-uns des pharisiens et des hérodiens auprès de Jésus afin de le surprendre par ses discours.
\VS{14}Et ils vinrent lui dire : Maître, nous savons que tu es vrai, et que tu ne t’inquiètes de personne ; car tu ne regardes pas à l'apparence des hommes, et tu enseignes la voie de Dieu selon la vérité. Est-il permis ou non de payer le tribut à César ? Devons-nous payer, ou ne pas payer ?
\VS{15}Mais Jésus, connaissant leur hypocrisie, leur dit : Pourquoi me tentez-vous ? Apportez-moi un denier, afin que je le voie.
\VS{16}Ils lui en apportèrent un. Alors il leur dit : De qui porte-t-il l’image et l’inscription ? De César, lui répondirent-ils.
\VS{17}Alors Jésus leur dit : Rendez à César ce qui est à César, et à Dieu ce qui est à Dieu. Et ils furent remplis d’admiration pour lui.
\TextTitle{[Jésus répond aux sadducéens sur la résurrection]
\\(Mt. 22:23-33 ; Lu. 20:27-38)}
\VS{18}Alors les sadducéens, qui disent qu'il n'y a point de résurrection, vinrent à lui, et l'interrogèrent, disant :
\VS{19}Maître, voici ce que Moïse nous a prescrit : Si le frère de quelqu'un meurt, et laisse sa femme sans avoir d'enfants, son frère épousera sa veuve et suscitera une postérité à son frère.
\VS{20}Or, il y avait sept frères. Le premier prit une femme et mourut sans laisser d'enfants.
\VS{21}Le deuxième prit la veuve pour femme, et mourut sans laisser de postérité. Il en fut de même du troisième,
\VS{22}et les sept l’épousèrent sans laisser de postérité. Après eux tous, la femme mourut aussi.
\VS{23}A la résurrection, quand ils seront ressuscités, duquel d’entre eux sera-t-elle la femme ? Car les sept l’ont eue pour femme.
\VS{24}Jésus leur répondit : La raison pour laquelle vous tombez dans l'erreur, c'est que vous ne connaissez ni les Ecritures ni la puissance de Dieu.
\VS{25}Car, à la résurrection des morts, les hommes ne prendront point de femmes, ni les femmes de maris, mais ils seront comme les anges dans les cieux.
\VS{26}Et quant aux morts, pour vous montrer qu'ils ressuscitent, n'avez-vous point lu dans le livre de Moïse, comment Dieu lui parla dans le buisson, en disant : Je suis le Dieu d'Abraham, et le Dieu d'Isaac, et le Dieu de Jacob ?
\VS{27}Or il n'est pas le Dieu des morts, mais le Dieu des vivants. Vous êtes donc dans une grande erreur.
\TextTitle{[Jésus répond aux pharisiens concernant le plus grand commandement de la loi]
\\(Mt. 22:34-40 ; Lu. 10:25-28)}
\VS{28}Un des scribes, qui les avait entendus discuter, voyant qu'il leur avait bien répondu, s'approcha de lui, et lui demanda : Quel est le premier de tous les commandements ?
\VS{29}Jésus lui répondit : Le premier de tous les commandements est : Ecoute Israël{\FTNT{Ecoute Israël~: Jésus se réfère ici à De. 6:4~: «~Ecoute, Israël ! Yahweh, notre Dieu Yahweh est Un~». Le Shema Israël est le noyau central de la prière que le Juif adulte doit lire matin et soir. C’est la confession de foi juive. Jacob est le premier à l’avoir enseignée à ses enfants dans Ge. 49:1-2.}}, le Seigneur notre Dieu, le Seigneur est Un{\FTNT{Jésus-Christ, notre Seigneur et notre modèle, a confirmé le Shema Israël qui déclare haut et fort que Dieu est Un et non trois en un. Le scribe, homme versé dans les Ecritures, était satisfait de la réponse de Jésus car il croyait aussi en un seul Dieu. Or le monothéisme est le fondement de la foi juive et des premiers chrétiens.}}.
\VS{30}Tu aimeras le Seigneur ton Dieu de tout ton cœur, de toute ton âme, de toute ta pensée, et de toute ta force. C'est là le premier commandement.
\VS{31}Voici le second, qui est semblable au premier : Tu aimeras ton prochain comme toi-même. Il n'y a pas d'autre commandement plus grand que ceux-là.
\VS{32}Et le scribe lui dit : Maître, tu as bien dit selon la vérité, qu'il y a un seul Dieu, et qu'il n'y en a point d'autre que lui ;
\VS{33}et que de l'aimer de tout son cœur, de toute son intelligence, de toute son âme, et de toute sa force ; et d'aimer son prochain comme soi-même, c'est plus que tous les holocaustes et les sacrifices.
\VS{34}Jésus voyant que ce scribe avait répondu prudemment, lui dit : Tu n'es pas loin du Royaume de Dieu. Et personne n'osait plus l'interroger.
\TextTitle{[Jésus dénonce les scribes]
\\(Mt. 22:41-46 ; Lu. 20:39-44)}
\VS{35}Comme Jésus enseignait dans le temple, il prit la parole et dit : Comment les scribes disent-ils que le Christ est le Fils de David ?
\VS{36}Car David lui-même a dit par le Saint-Esprit : Le Seigneur a dit à mon Seigneur : Assieds-toi à ma droite, jusqu'a ce que je fasse de tes ennemis ton marchepied{\FTNT{Ps. 110:1.}}.
\VS{37}David lui-même l'appelle son Seigneur, comment est-il son fils ? Et une grande foule l’écoutait avec plaisir.
\VS{38}Il leur disait dans son enseignement : Gardez-vous des scribes qui prennent plaisir à se promener en robes longues, et qui aiment les salutations dans les places publiques,
\VS{39}qui recherchent les premiers sièges dans les synagogues, et les premières places dans les festins ;
\VS{40}qui dévorent entièrement les maisons des veuves, et qui font pour l’apparence de longues prières. Ils seront jugés plus sévèrement.
\TextTitle{[L'offrande de la pauvre veuve]
\\(Lu. 21:1-4)}
\VS{41}Jésus, s’étant assis vis-à-vis du tronc, regardait comment la foule y mettait de l'argent. Plusieurs riches y mettaient beaucoup.
\VS{42}Et une pauvre veuve vint, elle y mit deux petites pièces, faisant le quart d’un sou.
\VS{43}Et Jésus, ayant appelé ses disciples, leur dit : Je vous le dis en vérité, cette pauvre veuve a plus mis dans le tronc que tous ceux qui y ont mis.
\VS{44}Car tous ont mis de leur superflu ; mais elle a mis de son nécessaire, tout ce qu'elle possédait, tout ce qu’elle avait pour vivre.
\TextTitle{[Les deux questions des disciples et la prophétie sur la destruction du temple de Jérusalem]
\\Mt. 24:3 ; Lu. 21:7)}
\Chap{13}
\VerseOne{}Lorsque Jésus sortit du temple, un de ses disciples lui dit : Maître, regarde quelles pierres et quelles constructions !
\VS{2}Jésus lui répondit : Vois-tu ces grands bâtiments ? Il ne restera pas pierre sur pierre qui ne soit pas démolie.
\VS{3}Il s’assit sur le Mont des oliviers, en face du temple. Et Pierre, Jacques, Jean et André, lui posèrent en particulier cette question :
\VS{4}Dis-nous quand cela arrivera-t-il, et à quel signe connaîtra-t-on que ces choses vont s'accomplir ?
\TextTitle{[Les temps de la fin]}
\VS{5}Jésus se mit à leur dire : Prenez garde que personne ne vous séduise.
\VS{6}Car plusieurs viendront en mon Nom, disant : C'est moi qui suis le Christ. Et ils séduiront beaucoup de gens.
\VS{7}Quand vous entendrez parler de guerres et des bruits de guerres, ne soyez point troublés ; parce qu'il faut que ces choses arrivent ; mais ce ne sera pas encore la fin.
\VS{8}Car une nation s'élèvera contre une autre nation, et un royaume contre un autre royaume ; et il y aura des tremblements de terre en divers lieux, et il y aura des famines et des troubles. Ces choses seront le commencement des douleurs.
\VS{9}Mais prenez garde à vous-mêmes. Car ils vous livreront aux tribunaux, et aux synagogues, vous serez battus de verges ; vous serez présentés devant les gouverneurs et devant les rois, à cause de moi, pour leur servir de témoignage.
\VS{10}Mais il faut premièrement que l'Evangile soit prêché à toutes les nations.
\VS{11}Et quand ils vous emmèneront pour vous livrer, ne vous inquiétez pas d’avance de ce que vous aurez à dire, mais dites ce qui vous sera donné à l’instant ; car ce n’est pas vous qui parlerez, mais le Saint-Esprit.
\VS{12}Le frère livrera son frère à la mort, et le père son enfant ; et les enfants se soulèveront contre leurs parents, et les feront mourir.
\VS{13}Vous serez haïs de tous à cause de mon Nom ; mais celui qui persévérera jusqu’à la fin, sera sauvé.
\TextTitle{[L'abomination de la désolation]
\\(Mt. 24:15-28 ; Ps. 2.5 ; Lu. 21:20-24 ; Ap. 7:14)}
\VS{14}Lorsque vous verrez l'abomination qui cause la désolation{\FTNT{Voir commentaire Mt. 24:15.}} qui a été prédite par Daniel, le prophète, établie là où elle ne doit pas être, que celui qui lit ce prophète fasse attention ! Alors que ceux qui seront en Judée fuient dans les montagnes.
\VS{15}Que celui qui sera sur le toit, ne descende pas dans la maison, et n’entre pas pour emporter quoi que ce soit de sa maison,
\VS{16}et que celui qui sera dans les champs, ne retourne pas en arrière pour emporter son manteau.
\VS{17}Malheur aux femmes qui seront enceintes, et à celles qui allaiteront en ces jours-là.
\VS{18}Priez Dieu que votre fuite n'arrive pas en hiver.
\VS{19}Car la détresse, en ces jours, sera telle qu’il n’y en a point eu de semblable depuis le commencement du monde que Dieu a créé jusqu’à présent, et qu’il n’y en aura jamais.
\VS{20}Et si le Seigneur n’avait abrégé ces jours, personne ne serait sauvé ; mais il les a abrégés, à cause des élus qu'il a choisis.
\VS{21}Si quelqu'un vous dit : Voici, le Christ est ici ; ou voici, il est là, ne le croyez point.
\VS{22}Car il s'élèvera des faux christs et des faux prophètes, qui feront des prodiges et des miracles, pour séduire même les élus s'il était possible.
\VS{23}Soyez sur vos gardes ; voici, je vous ai tout annoncé d’avance.
\TextTitle{[Retour du Messie sur la terre]
\\(Mt. 24:29-31 ; Lu. 21:25-28)}
\VS{24}Mais dans ces jours, après cette détresse, le soleil s’obscurcira, et la lune ne donnera plus sa clarté ;
\VS{25}les étoiles du ciel tomberont, et les puissances qui sont dans les cieux seront ébranlées.
\VS{26}Alors ils verront le Fils de l'homme venant sur les nuées, avec une grande puissance et une grande gloire.
\VS{27}Alors il enverra ses anges, et il rassemblera ses élus des quatre vents, de l’extrémité de la terre jusqu’à l’extrémité du ciel.
\TextTitle{[Parabole du figuier]
\\(Mt. 24:32-35 ; Lu. 21:29-33)}
\VS{28}Instruisez-vous par une comparaison tirée du figuier. Dès que ses branches deviennent tendres, et que les feuilles poussent, vous savez que l'été est proche.
\VS{29}Ainsi, quand vous verrez ces choses arriver, sachez que le Fils de l’homme est proche, à la porte.
\VS{30}Je vous le dis en vérité, cette génération ne passera point, que toutes ces choses ne soient arrivées.
\VS{31}Le ciel et la terre passeront, mais mes paroles ne passeront point.
\TextTitle{[Exhortation de Jésus sur la vigilance]
\\(Mt. 24:36-51 ; Lu. 21:34-38)}
\VS{32}Pour ce qui est du jour ou de l’heure, personne ne le sait, ni les anges dans le ciel, ni le Fils{\FTNT{Comment expliquer l’ignorance du Fils quant à l’heure de son retour ? En prenant la condition d’un homme, Jésus s’est dépouillé de ses prérogatives divines et a connu des limites propres au genre humain (Ph. 2:7)~: la fatigue (Jn. 4:6 ; Mc. 4:38), la faim (Mc. 11:12), l’angoisse et la peur (Mc. 14:33), la mortalité physique… Ce dépouillement incluait le renoncement à l’omniscience, d’où le fait que Jésus-Christ homme ne connaissait pas le jour et l’heure de son retour.}}, mais mon Père seul.
\VS{33}Prenez garde, veillez et priez ; car vous ne savez quand ce temps viendra.
\VS{34}Il en sera comme d’un homme qui, partant pour un voyage, laisse sa maison, remet l’autorité à ses serviteurs, marquant à chacun sa tâche, et ordonne au portier de veiller.
\VS{35}Veillez donc, car vous ne savez quand le Maître de la maison viendra, ou le soir, ou à minuit, ou à l'heure où le coq chante, ou le matin ;
\VS{36}craignez qu’il ne vous trouve endormis, à son arrivée soudaine.
\VS{37}Ce que je vous dis, je le dis à tous : Veillez.
\TextTitle{[Le complot]
\\(Mt. 26:1-5 ; Lu. 22:1-2)}
\Chap{14}
\VerseOne{}La fête de Pâque et des pains sans levain devait avoir lieu deux jours après. Les principaux sacrificateurs et les scribes cherchaient les moyens de se saisir de Jésus par ruse, et de le faire mourir.
\VS{2}Mais ils disaient : Que ce ne soit pas pendant la fête, afin qu'il n’y ait pas de tumulte parmi le peuple.
\TextTitle{[Marie de Béthanie oint Jésus pour sa sépulture]
\\(Mt. 26:6-13 ; Jn. 12:1-8)}
\VS{3}Comme Jésus était à Béthanie, dans la maison de Simon le lépreux, et pendant qu’il était à table, une femme vint à lui avec un vase d'albâtre, rempli d'un parfum de nard pur et de grand prix ; et ayant rompu le vase, elle répandit le parfum sur la tête de Jésus.
\VS{4}Quelques-uns en furent indignés en eux-mêmes, et ils disaient : A quoi sert la perte de ce parfum ?
\VS{5}On aurait pu le vendre plus de trois cents deniers, et les donner aux pauvres. Ainsi ils murmuraient contre elle.
\VS{6}Mais Jésus dit : Laissez-la. Pourquoi lui faites-vous de la peine ? Elle a fait une bonne action à mon égard.
\VS{7}Parce que vous aurez toujours des pauvres avec vous, et vous pouvez leur faire du bien quand vous voulez ; mais vous ne m'aurez pas toujours.
\VS{8}Elle a fait ce qu’elle a pu ; elle a d’avance embaumé mon corps pour la sépulture.
\VS{9}Je vous le dis en vérité, partout où cet Evangile sera prêché, dans le monde entier, on racontera aussi en mémoire de cette femme ce qu’elle a fait.
\TextTitle{[La trahison de Judas]
\\(Mt. 26:14-16 ; Lu. 22:3-6)}
\VS{10}Alors Judas Iscariot, l'un des douze, alla vers les principaux sacrificateurs pour le livrer.
\VS{11}Après l’avoir entendu, ils furent dans la joie, et promirent de lui donner de l'argent. Et Judas cherchait une occasion favorable pour le livrer.
\TextTitle{[La dernière Pâque]
\\(Mt. 26:17-25 ; Lu. 22:7-20 ; Jn. 13:1-12)}
\VS{12}Le premier jour des pains sans levain, où l’on sacrifiait l'agneau de Pâque, ses disciples lui dirent : Où veux-tu que nous allions te préparer l'agneau de Pâque afin que tu manges ?
\VS{13}Et il envoya deux de ses disciples, et leur dit : Allez dans la ville, vous rencontrerez un homme portant une cruche d'eau, suivez-le.
\VS{14}Où qu’il entre, dites au maître de la maison : Le Maître dit : Où est le lieu où je mangerai l'agneau de Pâque avec mes disciples ?
\VS{15}Et il vous montrera une grande chambre haute, meublée et toute prête : C’est là que vous nous préparerez l'agneau de Pâque.
\VS{16}Ses disciples partirent, arrivèrent dans la ville, ils trouvèrent les choses comme il l’avait dit ; et ils apprêtèrent l'agneau de Pâque.
\VS{17}Le soir étant venu, il arriva avec les douze.
\VS{18}Pendant qu’ils étaient à table, et qu'ils mangeaient, Jésus leur dit : Je vous le dis en vérité, l'un de vous, qui mange avec moi, me trahira.
\VS{19}Ils commencèrent à s'attrister, et ils lui dirent l'un après l'autre : Est-ce moi ?
\VS{20}Mais il leur répondit : C'est l'un des douze qui trempe avec moi dans le plat.
\VS{21}Certes le Fils de l'homme s'en va, selon qu'il est écrit de lui. Mais malheur à l'homme par qui le Fils de l'homme est trahi ! Mieux vaudrait pour cet homme qu’il ne soit pas né.
\TextTitle{[Le repas de la Pâque]
\\(Mt. 26:26-29 ; Lu. 22:17-20 ; Jn. 13:12-30 ; 1 Co. 11:23-26)}
\VS{22}Pendant qu’ils mangeaient, Jésus prit du pain, et après avoir béni Dieu, il le rompit et le leur donna, et leur dit : Prenez, mangez, ceci est mon corps.
\VS{23}Il prit ensuite une coupe, et après avoir rendu grâces, il la leur donna, et ils en burent tous.
\VS{24}Et il leur dit : Ceci est mon sang{\FTNT{Nouvelle Alliance : Voir Jn. 19:30.}}, le sang de la nouvelle alliance, qui est répandu pour plusieurs.
\VS{25}Je vous le dis en vérité, je ne boirai plus du fruit de la vigne jusqu'au jour où j’en boirai du nouveau dans le Royaume de Dieu.
\TextTitle{[Jésus avertit Pierre de son triple reniement]
\\(Mt. 26:30-35 ; Lu. 22:31-34 ; Jn. 13:36-38)}
\VS{26}Après avoir chanté les cantiques{\FTNT{Cantiques~: Voir Mt. 26:30.}}, ils se rendirent à la montagne des oliviers.
\VS{27}Jésus leur dit : Vous serez tous cette nuit scandalisés en moi ; car il est écrit : Je frapperai le Berger, et les brebis seront dispersées{\FTNT{Za. 13:7.}}.
\VS{28}Mais, après que je serai ressuscité, je vous précéderai en Galilée.
\VS{29}Pierre lui dit : Quand même tous seraient scandalisés, je ne le serai pourtant pas moi.
\VS{30}Et Jésus lui dit : Je te le dis en vérité, qu'aujourd'hui, cette nuit même, avant que le coq chante deux fois, tu me renieras trois fois.
\VS{31}Mais Pierre disait encore plus fortement : Quand même il me faudrait mourir avec toi, je ne te renierai pas. Et tous lui dirent la même chose.
\TextTitle{[Jésus dans le jardin de Gethsémané]
\\(Mt. 26:36-46 ; Lu. 22:39-46 ; Jn. 18:1)}
\VS{32}Ils allèrent dans un lieu appelé Gethsémané, et Jésus dit à ses disciples : Asseyez-vous ici jusqu'à ce que j’aie prié.
\VS{33}Il prit avec lui Pierre, Jacques et Jean, et il commença à être effrayé et fort angoissé.
\VS{34}Il leur dit : Mon âme est saisie de tristesse jusqu’à la mort, restez ici, et veillez.
\TextTitle{[Première prière de Jésus]
\\(Mt. 26:36 ; Lu. 22:41-42)}
\VS{35}Puis s'en allant un peu plus en avant, il se jeta contre terre, et pria que s'il était possible, cette heure s’éloigne de lui.
\VS{36}Il disait : Abba, Père, toutes choses te sont possibles, éloigne de moi cette coupe ! Toutefois, non pas ce que je veux, mais ce que tu veux.
\VS{37}Puis il vint vers les disciples qu’il trouva endormis, et il dit à Pierre : Simon, tu dors ! Tu n’as pas pu veiller une heure !
\VS{38}Veillez et priez afin que vous ne tombiez pas en tentation, l'esprit est bien disposé, mais la chair est faible.
\TextTitle{[Deuxième prière]
\\(Mt. 26:42 ; Lu. 22:44)}
\VS{39}Il s’éloigna de nouveau, et fit la même prière, disant les mêmes paroles.
\VS{40}Il revint, et les trouva encore endormis, car leurs yeux étaient appesantis. Ils ne surent que lui répondre.
\TextTitle{[Troisième prière]
\\(Mt. 26:44)}
\VS{41}Il revint encore, pour la troisième fois, et leur dit : Dormez maintenant, et reposez-vous ! C’est assez ! L’heure est venue ; voici, le Fils de l'homme est livré entre les mains des méchants.
\VS{42}Levez-vous, allons ; voici, celui qui me trahit s'approche.
\TextTitle{[Jésus trahi, abandonné et arrêté]
\\(Mt. 26:47-56 ; Lu. 22:47-53 ; Jn. 18:2-11)}
\VS{43}Et aussitôt, comme il parlait encore, Judas, l'un des douze, vint, et avec lui une grande foule ayant des épées et des bâtons, envoyée par les principaux sacrificateurs, par les scribes et par les anciens.
\VS{44}Celui qui le trahissait leur avait donné ce signe : Celui que j'embrasserai, c’est lui ; saisissez-le, et emmenez-le sûrement.
\VS{45}Dès qu’il fut arrivé, il s'approcha aussitôt de Jésus, et lui dit : Rabbi, Rabbi ! Et il l’embrassa.
\VS{46}Alors ils mirent la main sur Jésus, et le saisirent.
\VS{47}Un de ceux qui étaient là présents, tirant son épée, frappa le serviteur du souverain sacrificateur et lui emporta l'oreille.
\VS{48}Alors Jésus prit la parole, et leur dit : Vous êtes venus comme après un brigand, avec des épées et des bâtons, pour m’arrêter.
\VS{49}J’étais tous les jours parmi vous, enseignant dans le temple, et vous ne m'avez point saisi ; mais tout ceci est arrivé afin que les Ecritures soient accomplies.
\VS{50}Alors tous ses disciples l'abandonnèrent et s'enfuirent.
\VS{51}Un jeune homme le suivait, n’ayant sur le corps qu’un drap. Et quelques jeunes gens le saisirent,
\VS{52}mais il abandonna son linceul, et se sauva tout nu.
\TextTitle{[Jésus devant Caïphe et le sanhédrin]
\\(Mt. 26:57-68 ; Jn. 18:12-14,19-24)}
\VS{53}Ils emmenèrent Jésus chez le souverain sacrificateur, où s'assemblèrent tous les principaux sacrificateurs, les anciens et les scribes.
\VS{54}Pierre le suivait de loin jusque dans la cour du souverain sacrificateur ; et il était assis avec les serviteurs, et se chauffait près du feu.
\VS{55}Les principaux sacrificateurs et tout le sanhédrin cherchaient quelque témoignage contre Jésus pour le faire mourir, mais ils n'en trouvaient point.
\VS{56}Car plusieurs rendaient de faux témoignages contre lui, mais les témoignages ne s’accordaient pas.
\VS{57}Alors quelques-uns s'élevèrent, et portèrent de faux témoignages contre lui, disant :
\VS{58}Nous l’avons entendu dire : Je détruirai ce temple qui est fait de main d’homme, et en trois jours j'en rebâtirai un autre qui ne sera pas fait de main d’homme.
\VS{59}Même sur ce point-là leurs témoignages ne s’accordaient pas.
\VS{60}Alors le souverain sacrificateur se levant au milieu, interrogea Jésus, disant : Ne réponds-tu rien ? Qu’est-ce que ces gens déposent contre toi ?
\VS{61}Mais Jésus garda le silence, et ne répondit rien. Le souverain sacrificateur l'interrogea de nouveau, et lui dit : Es-tu le Christ, le Fils du Dieu béni ?
\VS{62}Jésus lui répondit : Je le suis. Et vous verrez le Fils de l'homme assis à la droite de la puissance de Dieu, et venant sur les nuées du ciel.
\VS{63}Alors le souverain sacrificateur déchira ses vêtements et dit : Qu'avons-nous encore besoin de témoins ?
\VS{64}Vous avez entendu le blasphème. Que vous en semble ? Alors tous le condamnèrent comme méritant la mort.
\VS{65}Et quelques-uns se mirent à cracher sur lui, à lui voiler le visage, et à lui donner des soufflets, en lui disant : Prophétise ! Et les serviteurs lui donnaient des coups avec leurs verges.
\TextTitle{[Triple reniement de Pierre]
\\(Mt. 26:69-75 ; Lu. 22:55-62 ; Jn. 18:15-18,25-27)}
\VS{66}Pendant que Pierre était en bas dans la cour, une des servantes du souverain sacrificateur vint.
\VS{67}Apercevant Pierre qui se chauffait, elle le regarda en face, et lui dit : Toi aussi, tu étais avec Jésus de Nazareth.
\VS{68}Mais il le nia, disant : Je ne le connais pas, et je ne sais pas ce que tu dis ; puis il sortit dehors pour aller dans le vestibule. Et le coq chanta.
\VS{69}La servante l'ayant vu de nouveau, elle se mit à dire à ceux qui étaient là présents : Celui-ci est de ces gens-là. Et il le nia de nouveau.
\VS{70}Peu après, ceux qui étaient là présents, dirent à Pierre : Certainement tu es de ces gens-là, car tu es Galiléen, et ton langage s'y rapporte.
\VS{71}Alors il commença à faire des imprécations et à jurer : Je ne connais pas cet homme dont vous parlez.
\VS{72}Et le coq chanta pour la seconde fois. Et Pierre se souvint de la parole que Jésus lui avait dite : Avant que le coq chante deux fois, tu me renieras trois fois. Et étant sorti promptement, il pleura.
\TextTitle{[Jésus livré à Pilate]
\\(Mt. 27:1-2,11-15 ; Lu. 23:1-7,13-18 ; Jn. 18:28-38 ; 19:1-15)}
\Chap{15}
\VerseOne{}Dès le matin, les principaux sacrificateurs tinrent conseil avec les anciens et les scribes, et tout le sanhédrin. Après avoir lié Jésus, ils l'emmenèrent, et le livrèrent à Pilate.
\VS{2}Pilate l'interrogea : Es-tu le Roi des Juifs ? Et Jésus répondit : Tu le dis.
\VS{3}Les principaux sacrificateurs l'accusaient de plusieurs choses, mais il ne répondit rien.
\VS{4}Pilate l'interrogea de nouveau : Ne réponds-tu rien ? Vois de combien de choses ils t’accusent.
\VS{5}Mais Jésus ne donna plus aucune réponse, ce qui étonna Pilate.
\TextTitle{[Jésus ou Barabbas ?]
\\(Mt. 27:15-26 ; Lu. 23:17-25 ; Jn. 18:39)}
\VS{6}A chaque fête, il relâchait un prisonnier, celui que demandait la foule.
\VS{7}Il y avait en prison un nommé Barabbas avec ses complices pour une sédition, dans laquelle ils avaient commis un meurtre.
\VS{8}La foule se mit à demander à Pilate, avec de grands cris, ce qu’il avait coutume de leur accorder.
\VS{9}Pilate leur répondit : Voulez-vous que je vous relâche le Roi des Juifs ?
\VS{10}Car il savait bien que les principaux sacrificateurs l'avaient livré par envie.
\VS{11}Mais les principaux sacrificateurs excitèrent la foule, afin que Pilate leur relâche plutôt Barabbas.
\VS{12}Pilate reprenant la parole, leur dit encore : Que voulez-vous donc que je fasse de celui que vous appelez Roi des Juifs ?
\VS{13}Ils crièrent de nouveau : Crucifie-le !
\VS{14}Alors Pilate leur dit : Mais quel mal a-t-il fait ? Et ils crièrent encore plus fort : Crucifie-le !
\VS{15}Pilate, voulant satisfaire la foule, leur relâcha Barabbas ; et après avoir fait battre de verges Jésus, il le livra pour être crucifié.
\TextTitle{[Jésus couronné d'épines]
\\(Mt. 27:27-31 ; Jn. 19:16-17)}
\VS{16}Alors les soldats emmenèrent Jésus dans l’intérieur de la cour, c’est-à-dire dans le prétoire, et ils assemblèrent toute la cohorte.
\VS{17}Ils le revêtirent d'une robe de pourpre, et posèrent sur sa tête une couronne d'épines qu’ils avaient tressée.
\VS{18}Puis ils commencèrent à le saluer, en lui disant : Nous te saluons, Roi des Juifs !
\VS{19}Et ils lui frappaient la tête avec un roseau, et crachaient sur lui, et fléchissant les genoux, ils se prosternaient devant lui.
\VS{20}Et après s'être ainsi moqués de lui, ils le dépouillèrent de la robe de pourpre, lui remirent ses habits, et l'emmenèrent dehors pour le crucifier.
\VS{21}Et un certain homme de Cyrène, nommé Simon, père d’Alexandre et de Rufus, passant par là en revenant des champs, fut forcé à porter la croix de Jésus.
\VS{22}Et ils conduisirent Jésus au lieu appelé Golgotha{\FTNT{Golgotha~: Le Golgotha (crâne) était une colline située à l'extérieur de Jérusalem, sur laquelle les Romains crucifiaient les condamnés.}}, c'est-à-dire, le lieu du Crâne.
\VS{23}Ils lui donnèrent à boire du vin mêlé de myrrhe, mais il ne le prit pas.
\TextTitle{[Jésus crucifié]
\\(Mt. 27:33-56 ; Lu. 23:33-49 ; Jn. 19:17-37)}
\VS{24}Ils le crucifièrent, et se partagèrent ses vêtements, en tirant au sort pour savoir ce que chacun aurait.
\VS{25}C’était la troisième heure, quand ils le crucifièrent.
\VS{26}L’écriteau indiquant la cause de sa condamnation portait ces mots : Le Roi des Juifs.
\VS{27}Ils crucifièrent aussi avec lui deux brigands, l'un à sa droite, et l'autre à sa gauche.
\VS{28}Et ainsi fut accomplie l'Ecriture, qui dit : Et il a été mis au rang des malfaiteurs{\FTNT{Es. 53:12.}}.
\VS{29}Les passants l’injuriaient, et secouaient la tête, en disant : Hé ! Toi qui détruis le temple et qui le rebâtis en trois jours,
\VS{30}sauve-toi toi-même, et descends de la croix !
\VS{31}Les principaux sacrificateurs aussi avec les scribes se moquaient entre eux, et disaient : Il a sauvé les autres, et il ne peut se sauver lui-même.
\VS{32}Que le Christ, le Roi d’Israël descende maintenant de la croix, afin que nous le voyions et que nous croyions ! Ceux qui étaient crucifiés avec lui l’insultaient aussi.
\VS{33}La sixième heure étant venue, il y eut des ténèbres sur toute la terre jusqu'à la neuvième heure.
\VS{34}Et à la neuvième heure, Jésus s’écria d’une voix forte : Eloï, Eloï, lama sabachthani ? C’est-à-dire : Mon Dieu ! Mon Dieu ! Pourquoi m'as-tu abandonné ?
\VS{35}Quelques-uns de ceux qui étaient là présents, l’ayant entendu, dirent : Voici, il appelle Elie.
\VS{36}Et l’un d’eux courut remplir une éponge de vinaigre{\FTNT{Le vinaigre~: Voir Mt. 27:34.}}, et l'ayant fixée au bout d'un roseau, il lui donna à boire, en disant : Laissez, voyons si Elie viendra le descendre de la croix.
\VS{37}Mais Jésus, ayant poussé un grand cri, expira.
\VS{38}Et le voile du temple se déchira en deux, depuis le haut jusqu'en bas{\FTNT{Hé 10:19-20.}}.
\TextTitle{[Fin de la Première Alliance]
\\(Hé. 9:16-18)}
\VS{39}Le centenier, qui était en face de Jésus, voyant qu'il avait expiré en criant de la sorte, dit : Certainement cet homme était Fils de Dieu.
\VS{40}Il y avait là aussi des femmes qui regardaient de loin. Parmi elles étaient Marie de Madgala, Marie mère de Jacques le mineur et de Joses, et Salomé,
\VS{41}qui le suivaient et le servaient lorsqu'il était en Galilée, et plusieurs autres qui étaient montées avec lui à Jérusalem.
\TextTitle{[Jésus enseveli]}
\VS{42}Le soir étant venu, comme c'était la préparation, c’est-à-dire le sabbat,
\VS{43}arriva Joseph d'Arimathée, conseiller de distinction, qui attendait aussi le Royaume de Dieu. Il osa se rendre vers Pilate pour demander le corps de Jésus.
\VS{44}Pilate s'étonna qu'il soit mort si tôt ; il fit venir le centenier, et lui demanda s'il était mort depuis longtemps.
\VS{45}S’en étant assuré par le centenier, il donna le corps à Joseph.
\VS{46}Et Joseph ayant acheté un linceul, descendit Jésus de la croix, et l'enveloppa du linceul, et le déposa dans un sépulcre taillé dans le roc. Puis il roula une pierre sur l'entrée du sépulcre.
\VS{47}Marie de Magdala, et Marie mère de Joses regardaient où on le mettait.
\TextTitle{[Jésus, ressuscité, apparaît à plusieurs disciples]
\\(Mt. 28:1-15 ; Lu. 24:1-49 ; Jn. 20:1-23)}
\Chap{16}
\VerseOne{}Lorsque le sabbat fut passé, Marie de Magdala, Marie mère de Jacques, et Salomé, achetèrent des aromates pour embaumer Jésus.
\VS{2}Le premier jour de la semaine, de grand matin, elles se rendirent au sépulcre, comme le soleil venait de se lever.
\VS{3}Elles disaient entre elles : Qui nous roulera la pierre de l'entrée du sépulcre ?
\VS{4}Et levant les yeux, elles virent que la pierre, qui était très grande, avait été roulée.
\VS{5}Elles entrèrent dans le sépulcre, virent un jeune homme assis à droite, vêtu d'une robe blanche, et elles furent épouvantées.
\VS{6}Mais il leur dit : Ne vous épouvantez pas. Vous cherchez Jésus de Nazareth qui a été crucifié. Il est ressuscité, il n'est point ici ; voici le lieu où on l'avait mis.
\VS{7}Mais allez, et dites à ses disciples, et à Pierre, qu'il vous précède en Galilée. C’est là que vous le verrez, comme il vous l'a dit.
\VS{8}Elles partirent aussitôt et s'enfuirent du sépulcre. La peur et le trouble les avaient saisies ; et elles ne dirent rien à personne, à cause de la peur.
\VS{9}Jésus étant ressuscité, le matin du premier jour de la semaine{\FTNT{Jésus a-t-il été crucifié un vendredi ? Si c’est le cas, comment a-t-il pu séjourner trois jours dans le tombeau s’il est ressuscité le dimanche matin comme l’enseigne la tradition catholique et la majorité des églises protestantes et évangéliques ? Tout d’abord il convient de signaler que selon Ge. 1, le jour commence au coucher du soleil, aux environs de dix-huit heures, et s’achève le lendemain au coucher du soleil. Chez les Romains, le jour commence à minuit et se termine le lendemain à minuit. C’est de cette manière que l’évangile de Jean compte les heures. Dans les autres évangiles, les journées commencent avec le lever du soleil. Jésus a été crucifié à «~la troisième heure~» (Mc. 15:25), ce qui correspond à neuf heures du matin. Ensuite, les évangiles nous apprennent qu’il y a eu des ténèbres sur la terre de la sixième à la neuvième heure, donc de midi à quinze heures (Mt. 27:45-46 ; Mc. 15:33-34 ; Lu. 23:44). Jésus est donc mort avant dix-huit heures. Ainsi, il est évident qu’il n’a pas pu passer toute la journée du vendredi au tombeau. Les Ecritures ne déclarent pas spécifiquement quel jour de la semaine Jésus a été crucifié. Les deux opinions dominantes sont vendredi et mercredi. D’autres font la synthèse des deux et acceptent le jeudi comme étant le jour de la crucifixion.  Jésus dit dans Mt. 12:40~: «~Car, de même que Jonas fut trois jours et trois nuits dans le ventre d’une baleine, de même le Fils de l'homme sera trois jours et trois nuits dans le sein de la terre~». Ceux qui défendent la crucifixion un vendredi disent qu’il est possible de compter de telle manière qu’on puisse effectivement considérer qu’il a été dans la tombe pendant trois jours. L’argument principal pour le vendredi se trouve dans Mc. 15:42 qui précise que Jésus a été crucifié la «~veille du sabbat~». S’il s’agit bien du sabbat hebdomadaire, c’est à dire le samedi, alors la crucifixion a bien eu lieu un vendredi. Un autre argument en faveur du vendredi se fonde sur des versets tels que Mt. 16:21 et Lu. 9:22 où Jésus enseigne qu’il ressuscitera le troisième jour, ce qui suppose qu’il ne restera pas trois jours et trois nuits entiers dans la tombe. Plusieurs traducteurs utilisent l’expression «~le troisième jour~», mais pas tous. Cependant, aucun d’eux ne conteste la manière de traduire ces versets. Dans Mc. 8:31, il est bien dit que Jésus sera ressuscité «~après~» trois jours.  Le débat sur le jeudi se construit sur celui du vendredi en concluant qu’il y a trop d’événements, selon ses défenseurs, qui se passent entre l’ensevelissement du Christ et dimanche matin, pour que tout se soit déroulé entre vendredi et dimanche matin. Il faut signaler qu’il est particulièrement problématique que le seul jour plein entre vendredi et dimanche soit le samedi. Un jour de plus ou deux résolvent ce problème. L’hypothèse du mercredi avance qu’il y avait deux sabbats cette semaine-là. Après le premier sabbat (celui qui débute le soir de la crucifixion, Mc. 15:42 ; Lu. 23:52-54), les femmes sont allées acheter les aromates. Notez bien qu’elles les ont achetées après le sabbat (Mc. 16:1). Dans cette hypothèse du mercredi, ce premier sabbat est la Pâque (cf. Lé. 16:29-31 ; Lé. 23:24-32,39) car les jours très saints sont aussi appelés sabbats. Le second sabbat de cette semaine était le sabbat hebdomadaire classique, le samedi. Notez que dans Lu. 23:56, les femmes qui avaient acheté les aromates après le premier sabbat s’en retournèrent, préparèrent les aromates puis «~se reposèrent durant le sabbat~» (Lu. 23:56). On ne peut pas imaginer qu’elles ont acheté les aromates après le sabbat et qu’elles les ont préparées avant le sabbat que s’il y a eu deux sabbats cette semaine-là. Avec l’hypothèse des deux sabbats, si le Messie a été crucifié un jeudi, alors le jour très saint (la Pâque) aurait débuté au coucher du soleil le jeudi et pris fin le vendredi au coucher du soleil – juste au début du sabbat hebdomadaire – le samedi. Acheter les aromates après le premier sabbat signifierait alors en faire l’acquisition le samedi, en violation des lois du sabbat.  L’hypothèse du mercredi est la seule qui corrobore les récits bibliques des femmes et des aromates et confirme la prophétie du Seigneur en Mt. 12:40. Le Messie a été arrêté à Gethsémané le mardi soir selon le calendrier romain et le mercredi selon le calendrier hébraïque. Le premier sabbat était un jour très saint, celui de Pâque (Mt. 26 ; Mc. 14 ; Lu. 22), un jeudi selon le calendrier hébraïque. Les femmes achetèrent les aromates le vendredi et s’en retournèrent les préparer le jour même. Elles se sont reposées le samedi, qui était le sabbat hebdomadaire, et ont enfin apporté les aromates au tombeau tôt le dimanche matin. Jésus a été enseveli au moment du coucher du soleil le mercredi, ce qui est le début du jeudi selon le calendrier juif. En usant de ce calendrier, nous avons~: - le jeudi nuit (première nuit), jeudi jour (premier jour) - le vendredi nuit (deuxième nuit), vendredi jour (deuxième jour) - le samedi nuit (troisième nuit), samedi jour (troisième jour). Nous ne savons pas exactement à quelle heure Jésus est ressuscité, mais sous savons que ce fut avant le lever du soleil du dimanche. En effet, Jn. 20:1 nous apprend que Marie de Magdala vint au tombeau «~alors qu’il faisait encore sombre~». Ainsi, Jésus serait ressuscité juste après le coucher du soleil du samedi soir, ce qui correspond au premier jour de la semaine pour les juifs.}} apparut d’abord à Marie de Madgala, de laquelle il avait chassé sept démons.
\VS{10}Elle alla l’annoncer à ceux qui avaient été avec lui, et qui étaient dans le deuil et pleuraient.
\VS{11}Mais quand ils entendirent qu'il était vivant, et qu'elle l'avait vu, ils ne la crurent point.
\VS{12}Après cela, il se montra sous une autre forme à deux d'entre eux, qui étaient en chemin pour aller à la campagne.
\VS{13}Ils revinrent l’annoncer aux autres, mais ils ne les crurent pas non plus.
\VS{14}Enfin, il se montra aux onze, qui étaient assis ensemble, et il leur reprocha leur incrédulité et leur dureté de cœur, parce ce qu'ils n'avaient pas cru ceux qui l'avaient vu ressuscité.
\TextTitle{[Nouvelle mission aux onze apôtres]
\\(Mt. 28:16-20 ; Lu. 24:46-48 ; Jn. 17:18 ; 20:21 ; Ac. 1:8)}
\VS{15}Puis il leur dit : Allez par tout le monde, et prêchez l'Evangile à toute créature.
\VS{16}Celui qui croira et qui sera baptisé, sera sauvé ; mais celui qui ne croira pas sera condamné.
\VS{17}Voici les miracles qui accompagneront ceux qui auront cru : Ils chasseront les démons en mon Nom ; ils parleront de nouvelles langues ;
\VS{18}ils saisiront les serpents avec la main, et s’ils boivent quelque breuvage mortel, il ne leur fera point de mal ; ils imposeront les mains aux malades, et les malades seront guéris.
\TextTitle{[Jésus enlevé au ciel]
\\(Lu. 24:49-53 ; Ac. 1:9-11)}
\VS{19}Le Seigneur, après leur avoir parlé de la sorte, fut enlevé au ciel, et il s'assit à la droite de Dieu.
\VS{20}Et ils s’en allèrent prêcher partout. Le Seigneur travaillait avec eux, et confirmait la parole par les miracles qui l'accompagnaient.
\PPE{}
\end{multicols}

%\clearpage\ShortTitle{Luc}\BookTitle{Luc}\BFont
\noindent\hrulefill
{\footnotesize
\textit{
\bigskip
{\centering{}
\\Auteur : Luc
\\(Gr. : Loukas)
\\Signifie : Qui donne la lumière
\\Thème : Jésus le Fils de l'homme
\\Date de rédaction : Env. 60 ap. J.-C.\\}
}
%\bigskip
\textit{
\\D'origine grecque, Luc fut l'auteur de l'évangile éponyme et du livre des « Actes des apôtres ». Celui que Paul appelait le « médecin bien-aimé », et qui fut son compagnon d'œuvre, avait entrepris des investigations visant à narrer avec exactitude la vie terrestre de Jésus-Christ dont il était devenu le disciple, probablement à la suite d'une prédication de Paul. Adressés initialement à Théophile, Luc était loin de penser que ses écrits constitueraient avec le temps une véritable richesse pour l'Eglise et pour le monde.
%\bigskip
\\L'évangile de Luc présente l'humanité parfaite de Jésus, sa compassion et sa miséricorde à l'égard des plus faibles. Rédigé avec rigueur et soin, il retrace le parcours du Fils de l'homme, de sa naissance à son adolescence, puis de sa mort à sa résurrection, et enfin son ascension. Il souligne aussi sa vie de prière et son fardeau pour le salut de l'homme. Par ailleurs, il fait ressortir la manière dont les femmes ont assisté Jésus par leurs biens durant son ministère.
%\bigskip
\\Fruit de recherches minutieuses, le récit de Luc présente certaines similitudes avec ceux de Matthieu et Marc, mais il est le seul à relater la célèbre parabole du fils prodigue, profonde représentation de l'amour du Père.\bigskip
}
}
\par\nobreak\noindent\hrulefill
\begin{multicols}{2}
\Chap{1}
\TextTitle{Introduction}
\VerseOne{}Parce que plusieurs se sont appliqués à mettre par ordre un récit des évènements qui ont été pleinement certifiés parmi nous,
\VS{2}suivant ce que nous ont transmis ceux qui ont été des témoins oculaires dès le commencement et sont devenus des ministres de la parole,
\VS{3}Il m'a aussi semblé bon, après avoir examiné exactement toutes choses depuis le commencement jusqu'à la fin, très excellent Théophile, de te les mettre en ordre par écrit,
\VS{4}afin que tu connaisses la certitude des choses dont tu as été informé.
\TextTitle{Annonce de la naissance de Jean-Baptiste}
\VS{5}Au temps d'Hérode, roi de Judée, il y avait un sacrificateur nommé Zacharie, de la classe d'Abia ; et sa femme était d'entre les filles d'Aaron, et s'appelait Elisabeth.
\VS{6}Et ils étaient tous deux justes devant Dieu, marchant dans tous les commandements, et dans {toutes } les ordonnances du Seigneur, sans reproche.
\VS{7}Et ils n'avaient point d'enfants, parce qu'Elisabeth était stérile, et qu'ils étaient fort avancés en âge.
\VS{8}Or il arriva que comme Zacharie exerçait la sacrificature devant Dieu, selon le tour de sa classe, il fut appelé par le sort,
\VS{9}selon la coutume d'exercer le sacerdoce, à entrer dans le temple du Seigneur pour offrir le parfum.
\VS{10}Toute la multitude du peuple était dehors en prière, à l'heure du parfum.
\VS{11}Et l'ange du Seigneur lui apparut, et se tint debout à droite de l'autel des parfums.
\VS{12}Zacharie fut troublé quand il le vit, et il fut saisi de crainte.
\VS{13}Mais l'ange lui dit : Zacharie, ne crains point ; car ta prière est exaucée. Et Elisabeth, ta femme, t'enfantera un fils, et tu lui donneras le nom de Jean.
\VS{14}Et il sera pour toi le sujet d'une grande joie et d'allégresse, et plusieurs se réjouiront de sa naissance.
\VS{15}Car il sera grand devant le Seigneur. Et il ne boira ni vin, ni boisson forte et il sera rempli du Saint-Esprit dès le ventre de sa mère.
\VS{16}Et il ramènera plusieurs des enfants d'Israël au Seigneur, leur Dieu.
\VS{17}Car il marchera devant lui animé de l'esprit et de la puissance d'Elie, pour ramener les cœurs des pères vers les enfants\FTNT{Mal. 4:6.}, et les rebelles à la sagesse des justes, pour préparer au Seigneur un peuple bien disposé.
\VS{18}Alors Zacharie dit à l'ange : A quoi reconnaîtrai-je cela ? Car je suis vieux, et ma femme est fort âgée.
\VS{19}L'ange répondant lui dit : Je suis Gabriel, je me tiens devant Dieu, et j'ai été envoyé pour te parler, et pour t'annoncer cette bonne nouvelle.
\VS{20}Et voici, tu seras muet, et tu ne pourras point parler jusqu'au jour où ces choses arriveront, parce que tu n'as pas cru à mes paroles qui s'accompliront en leur temps.
\VS{21}Or le peuple attendait Zacharie, et on s'étonnait de ce qu'il tardait tant dans le temple.
\VS{22}Mais quand il fut sorti, il ne pouvait pas leur parler, et ils comprirent qu'il avait eu une vision dans le temple ; car il leur faisait des signes et il resta muet.
\VS{23}Et il arriva que quand les jours de son ministère furent achevés, il retourna dans sa maison.
\VS{24}Et après ces jours-là, Elisabeth sa femme conçut, et elle se cacha l’espace de cinq mois, en disant :
\VS{25}Certes, le Seigneur en a agi avec moi ainsi aux jours qu’il m’a regardée pour ôter mon opprobre d’entre les hommes.
\TextTitle{Annonce de la naissance de Jésus-Christ}
\VS{26}Or au sixième mois, l'ange Gabriel fut envoyé par Dieu dans une ville de Galilée, appelée Nazareth,
\VS{27}vers une vierge fiancée à un homme nommé Joseph, qui était de la maison de David. Et le nom de la vierge était Marie.
\VS{28}Et l'ange étant entré dans le lieu où elle était, lui dit : Je te salue, toi à qui une grâce a été faite. Le Seigneur est avec toi ; tu es bénie parmi les femmes.
\VS{29}Troublée par cette parole, Marie se demandait ce que pouvait signifier une telle salutation.
\VS{30}L'ange lui dit : Marie, ne crains point ; car tu as trouvé grâce devant Dieu.
\VS{31}Et voici, tu concevras en ton ventre, et tu enfanteras un fils, et tu lui donneras le Nom de JESUS.
\VS{32}Il sera grand, et sera appelé le Fils du Très-Haut, et le Seigneur Dieu lui donnera le trône de David, son père.
\VS{33}Il régnera sur la maison de Jacob éternellement, et son règne n'aura pas de fin.
\TextTitle{Naissance miraculeuse de Jésus-Christ}
\VS{34}Alors Marie dit à l'ange : Comment cela se fera-t-il, puisque je ne connais point d'homme ?
\VS{35}L'ange lui répondit et dit : Le Saint-Esprit viendra sur toi, et la puissance du Très-Haut te couvrira de son ombre. C'est pourquoi, le Saint qui naîtra de toi sera appelé Fils de Dieu.
\VS{36}Voici, Elizabeth, ta cousine, a conçu elle aussi un fils en sa vieillesse, celle qui était appelée stérile est dans son sixième mois de grossesse.
\VS{37}Car rien n'est impossible à Dieu.
\VS{38}Et Marie dit : Voici la servante du Seigneur, qu'il me soit fait selon ta parole ! Et l'ange la quitta.
\TextTitle{Marie se rend chez Elisabeth}
\VS{39}Dans ce même temps, Marie se leva, et s'en alla en hâte au pays des montagnes dans une ville de Juda.
\VS{40}Elle entra dans la maison de Zacharie, et salua Elisabeth.
\VS{41}Et il arriva, comme Elisabeth entendait la salutation de Marie, que le petit enfant tressaillit dans son ventre ; et Elisabeth fut remplie de l’Esprit Saint,
\VS{42}Elle s'écria d'une voix forte et dit : Tu es bénie entre les femmes, et béni est le fruit de ton ventre.
\VS{43}Comment m'est-il accordé que la mère de mon Seigneur vienne vers moi ?
\VS{44}Car voici, dès que la voix de ta salutation est parvenue à mes oreilles, le petit enfant a tressailli de joie dans mon ventre.
\VS{45}Heureuse celle qui a cru, parce que les choses qui lui ont été dites par le Seigneur auront leur accomplissement.
\TextTitle{Cantique de Marie\FTNTT{Cp. 1 S. 2:1-10}}
\VS{46}Alors Marie dit : Mon âme magnifie le Seigneur,
\VS{47}et mon esprit se réjouit en Dieu, mon Sauveur.
\VS{48}Car il a jeté les yeux sur la bassesse de sa servante. Voici, certes désormais toutes les générations me diront bienheureuse,
\VS{49}parce que le Tout-Puissant a fait pour moi de grandes choses, et son Nom est Saint.
\VS{50}Et sa miséricorde s'étend de génération en génération en faveur de ceux qui le craignent.
\VS{51}Il a puissamment opéré par son bras. Il a dissipé les desseins que les orgueilleux formaient dans leurs cœurs.
\VS{52}Il a renversé de dessus leurs trônes les puissants, et il a élevé les petits.
\VS{53}Il a rassasié de biens les affamé, il a renvoyé les riches à vide.
\VS{54}Il a pris sous sa protection Israël, son serviteur, et il s'est souvenu de sa miséricorde,
\VS{55}comme il l'avait dit à nos pères, envers Abraham et sa postérité à jamais.
\VS{56}Marie demeura avec elle environ trois mois. Puis elle retourna dans sa maison.
\TextTitle{Naissance de Jean}
\VS{57}Le temps où Elisabeth devait accoucher arriva, et elle enfanta un fils.
\VS{58}Ses voisins et ses parents ayant appris que le Seigneur avait fait éclater sa miséricorde envers elle, s'en réjouissaient avec elle.
\VS{59}Et il arriva qu'au huitième jour, ils vinrent pour circoncire le petit enfant, et ils l'appelaient Zacharie, du nom de son père.
\VS{60}Mais sa mère prit la parole, et dit : Non, mais il sera appelé Jean.
\VS{61}Et ils lui dirent : Il n'y a personne dans ta parenté qui soit appelé de ce nom.
\VS{62}Alors ils firent signe à son père pour savoir comment il voulait qu'on l'appelle.
\VS{63}Et Zacharie ayant demandé des tablettes, écrivit : Jean est son nom. Et tous furent dans l'étonnement.
\VS{64}Au même instant, sa bouche s'ouvrit et sa langue se délia, et il parlait, bénissant Dieu.
\VS{65}Tous ses voisins furent saisis de crainte et toutes ces choses furent divulguées dans tout le pays des montagnes de Judée.
\VS{66}Tous ceux qui les apprirent les gardèrent dans leur cœur, disant : Que sera donc cet enfant ? Et la main du Seigneur était avec lui.
\TextTitle{Cantique de Zacharie}
\VS{67}Alors Zacharie, son père, fut rempli du Saint-Esprit, et il prophétisa en ces mots :
\VS{68}Béni soit le Seigneur, le Dieu d'Israël, de ce qu'il a visité et délivré son peuple
\VS{69}et de ce qu'il nous a suscité un puissant Sauveur dans la maison de David, son serviteur,
\VS{70}selon ce qu'il avait dit par la bouche de ses saints prophètes des temps anciens :
\VS{71}Un Sauveur qui nous délivre de nos ennemis et de la main de tous ceux qui nous haïssent !
\VS{72}C'est ainsi qu'il manifeste sa miséricorde envers nos pères, et se souvient de sa sainte alliance.
\VS{73}Selon le serment par lequel il avait juré à Abraham notre père,
\VS{74}de nous permettre, après que nous serions délivrés de la main de nos ennemis, de le servir sans crainte,
\VS{75}en marchant devant lui dans la sainteté et dans la justice tous les jours de notre vie.
\VS{76}Et toi, petit enfant, tu seras appelé prophète du Très-Haut ; car tu marcheras devant la face du Seigneur, pour préparer ses voies,
\VS{77}afin de donner à son peuple la connaissance du salut, par la rémission de leurs péchés,
\VS{78}grâce aux entrailles de la miséricorde de notre Dieu, en vertu de laquelle le Soleil Levant nous a visités d'en haut,
\VS{79}pour éclairer ceux qui sont assis dans les ténèbres et dans l'ombre de la mort, et pour conduire nos pas dans le chemin de la paix.
\VS{80}Or, le petit enfant croissait, et se fortifiait en esprit. Et il demeura dans les déserts jusqu'au jour où il se présenta à Israël.
\TextTitle{[Naissance de Jésus à Bethléhem]
\\(Mt. 1:18-25 ; 2:1 ; cp. Jn. 1:14}
\Chap{2}
\VerseOne{}En ces jours-là fut publié un édit par César Auguste, ordonnant un recensement de toute la terre.
\VS{2}Ce premier recensement eut lieu pendant que Quirinius était gouverneur de Syrie.
\VS{3}Ainsi, tous allaient pour s'inscrire, chacun dans sa ville.
\VS{4}Joseph aussi monta de Galilée en Judée, de la ville de Nazareth, la ville de David, appelée Bethléhem, parce qu'il était de la maison et de la famille de David ;
\VS{5}afin de se faire inscrire avec Marie, sa fiancée, qui était enceinte.
\VS{6}Pendant qu'ils étaient là, le temps où Marie devait accoucher arriva,
\VS{7}et elle enfanta son fils, premier-né et elle l'emmaillota et le coucha dans une crèche, parce qu'il n'y avait point de place pour eux dans l'hôtellerie.
\TextTitle{L'Ange du Seigneur annonce la naissance de Jésus}
\VS{8}Et il y avait dans cette même contrée des bergers qui couchaient dans les champs, et qui gardaient leur troupeau pendant les veilles de la nuit.
\VS{9}Et voici, l'Ange du Seigneur survint vers eux, et la gloire du Seigneur resplendit autour d'eux, et ils furent saisis d'une grande peur.
\VS{10}Mais l'Ange leur dit : Ne craignez point ; car, voici je vous annonce une bonne nouvelle qui sera un sujet de joie pour tout le peuple :
\VS{11}C'est qu'aujourd'hui, dans la ville de David, vous est né le Sauveur, qui est le Christ, le Seigneur.
\VS{12}Et voici à quel signe vous le reconnaîtrez : Vous trouverez le petit enfant emmailloté, et couché dans une crèche.
\VS{13}Et aussitôt il se joignit à l'Ange une multitude de l'armée céleste, louant Dieu et disant :
\VS{14}Gloire soit à Dieu dans les lieux très-hauts, que la paix soit sur la terre et la bonne volonté dans les hommes !
\TextTitle{Les bergers de Bethléhem}
\VS{15}Et il arriva qu'après que les anges s’en furent allés d’avec eux au ciel, les bergers se dirent les uns aux autres : Allons donc jusqu'à Bethléhem, et voyons cette chose qui est arrivée, ce que le Seigneur nous a fait connaître.
\VS{16}Ils y allèrent donc en hâte, et ils trouvèrent Marie et Joseph, et le petit enfant couché dans une crèche.
\VS{17}Après l'avoir vu, ils divulguèrent ce qui leur avait été dit au sujet de ce petit enfant.
\VS{18}Tous ceux qui les entendirent furent dans l'étonnement de ce que leur disaient les bergers.
\VS{19}Et Marie gardait soigneusement toutes ces choses, et les repassait dans son esprit.
\VS{20}Puis les bergers s'en retournèrent, glorifiant et louant Dieu pour tout ce qu'ils avaient entendu et vu, et qui était conforme à ce qui leur avait été annoncé.
\TextTitle{Jésus circoncis et présenté au temple de Jérusalem\FTNTT{Cp. Ex. 13:12,15}}
\VS{21}Et quand les huit jours furent accomplis pour circoncire l’enfant, on lui donna le Nom de Jésus, nom qu'avait indiqué l'Ange avant qu'il soit conçu dans le sein de sa mère.
\VS{22}Et quand les jours de la purification\FTNT{Lé : 12:2-6.} de Marie furent accomplis selon la loi de Moïse, Joseph et Marie le portèrent à Jérusalem, pour le présenter au Seigneur,
\VS{23}selon ce qui est écrit dans la loi du Seigneur : Tout mâle premier-né sera appelé Saint au Seigneur\FTNT{Ex. 13:2 ; Ex. 13:12 ; No. 3:13 ; No. 8:17.} ;
\VS{24}et pour offrir en sacrifice deux tourterelles ou deux jeunes pigeons comme cela est prescrit dans la loi du Seigneur\FTNT{Lé. 12:8.}.
\TextTitle{Adoration de Siméon et sa prophétie}
\VS{25}Et voici, il y avait à Jérusalem un homme appelé Siméon. Et cet homme était juste et pieux, il attendait la consolation d'Israël, et le Saint-Esprit était sur lui.
\VS{26}Il avait été averti divinement par le Saint-Esprit qu'il ne mourrait point avant d'avoir vu le Christ du Seigneur.
\VS{27}Il vint au temple, poussé par l'Esprit. Et comme les parents apportaient dans le temple l'enfant Jésus, pour accomplir à son égard ce qu'ordonnait la loi,
\VS{28}il le prit dans ses bras, bénit Dieu, et dit :
\VS{29}Seigneur, tu laisses maintenant ton serviteur s'en aller en paix selon ta parole.
\VS{30}Car mes yeux ont vu ton salut.
\VS{31}Lequel tu as préparé devant la face de tous les peuples.
\VS{32}La lumière pour éclairer les nations ; et pour être la gloire de ton peuple d'Israël.
\VS{33}Joseph et sa mère s'étonnaient des choses qui étaient dites de lui.
\VS{34}Siméon le bénit, et dit à Marie, sa mère : Voici, cet enfant est destiné à être une occasion de chute et de relèvement de beaucoup en Israël, et à devenir un signe qui provoquera la contradiction,
\VS{35}en sorte que les pensées de beaucoup de cœurs seront découvertes. Et pour toi, une épée te transpercera l'âme.
\TextTitle{Anne témoigne du Messie}
\VS{36}Il y avait aussi Anne, la prophétesse, fille de Phanuel de la tribu d'Aser, qui était déjà avancée en âge, et qui avait vécu avec son mari sept ans depuis sa virginité.
\VS{37}Restée veuve, et âgée d'environ quatre-vingt-quatre ans, elle ne quittait pas le temple, et elle servait Dieu nuit et jour dans le jeûne et dans les prières.
\VS{38}Etant arrivée à cette heure, elle louait aussi Dieu, et parlait de lui à tous ceux qui attendaient la délivrance de Jérusalem.
\TextTitle{Retour à Nazareth\FTNTT{Suite aux évènements de Mt. 2.}}
\VS{39}Et quand ils eurent accompli tout ce qui est ordonné par la loi du Seigneur, ils s'en retournèrent en Galilée, à Nazareth, leur ville.
\VS{40}Et le petit enfant croissait et se fortifiait en esprit. Il était rempli de sagesse, et la grâce de Dieu était sur lui.
\TextTitle{Jésus assis dans le temple de Jérusalem au milieu des docteurs}
\VS{41}Ses parents allaient tous les ans à Jérusalem, à la fête de Pâque.
\VS{42}Lorsqu'il fut âgé de douze ans, ses parents montèrent à Jérusalem selon la coutume de la fête.
\VS{43}Puis, quand les jours furent écoulés, et qu'ils s'en retournèrent, l'enfant Jésus resta à Jérusalem. Et son père et sa mère ne s'en aperçurent point.
\VS{44}Mais croyant qu'il était avec leurs compagnons de voyage, ils marchèrent une journée, puis ils le cherchèrent parmi leurs parents et parmis leur connaissance.
\VS{45}Et ne le trouvant point, ils retournèrent à Jérusalem, pour le chercher.
\VS{46}Or il arriva que trois jours après, ils le trouvèrent dans le temple, assis au milieu des docteurs, les écoutant, et les interrogeant.
\VS{47}Tous ceux qui l'entendaient s'étonnaient de sa sagesse et de ses réponses.
\VS{48}Quand ses parents le virent, ils furent saisis d'étonnement, et sa mère lui dit : Mon enfant, pourquoi nous as-tu fait ainsi ? Voici, ton père et moi te cherchions avec angoisse.
\VS{49}Et il leur dit : Pourquoi me cherchiez-vous ? Ne saviez-vous pas qu'il faut que je m'occupe des affaires de mon Père ?
\VS{50}Mais ils ne comprirent point ce qu'il leur disait.
\VS{51}Alors il descendit avec eux, et vint à Nazareth ; et il leur était soumis. Et sa mère gardait toutes ces paroles dans son cœur.
\TextTitle{Jésus grandit en sagesse, en stature et en grâce}
\VS{52}Et Jésus croissait en sagesse, en stature, et en grâce, au près de Dieu et devant les hommes.
\Chap{3}
\TextTitle{Ministère de Jean-Baptiste\FTNTT{Mt. 3:1-12 ; Mc. 1:1-8 ; Jn. 1:6-8,15-37}}
\VerseOne{}La quinzième année du règne de Tibère César, lorsque Ponce Pilate était gouverneur de la Judée, Hérode tétrarque de la Galilée, et son frère Philippe tétrarque de l'Iturée et du territoire de la Trachonite, et Lysanias tétrarque de l'Abilène,
\VS{2}et du temps des souverains sacrificateurs Anne et Caïphe, la parole de Dieu fut adressée à Jean, fils de Zacharie, dans le désert.
\VS{3}Et il alla dans tout le pays des environs du Jourdain, prêchant le baptême de repentance, pour la rémission des péchés,
\VS{4}comme il est écrit dans le livre des paroles d'Esaïe, le prophète disant : C'est la voix de celui qui crie dans le désert : Préparez le chemin du Seigneur, aplanissez ses sentiers.
\VS{5}Toute vallée sera comblée, et toute montagne et toute colline seront abaissées, et ce qui est tortueux sera redressé, et les chemins raboteux seront aplanis.
\VS{6}Et toute chair verra le salut de Dieu\FTNT{Es. 40:3-5.}.
\VS{7}Il disait donc à ceux qui venaient en foule pour être baptisés par lui : Races de vipères, qui vous a appris à fuir la colère à venir ?
\VS{8}Produisez donc des fruits dignes de la repentance, et ne vous mettez point à dire en vous-mêmes : Nous avons Abraham pour père. Car je vous dis que Dieu peut faire naître, même de ces pierres, des enfants à Abraham.
\VS{9}Or la cognée est déjà mise à la racine des arbres ; tout arbre donc qui ne produit pas de bon fruit, sera coupé, et jeté au feu.
\VS{10}Alors la foule l'interrogeait, disant : Que ferons-nous donc ?
\VS{11}Et il répondit, et leur dit : Que celui qui a deux tuniques partage avec celui qui n'en a point ; et que celui qui a de quoi manger en fasse de même.
\VS{12}Il vint aussi à lui des publicains pour être baptisés, et ils lui dirent : Maître, que ferons-nous ?
\VS{13}Et il leur dit : N'exigez rien au-delà de ce qui vous a été ordonné.
\VS{14}Des soldats l'interrogèrent aussi, disant : Et nous, que ferons-nous ? Et Il leur répondit : Ne commettez ni extorsion ni fraude envers personne, mais contentez-vous de votre solde.
\VS{15}Et comme le peuple était dans l'attente, et que tous se demandaient dans leurs cœurs si Jean n'était pas le Christ,
\VS{16}Jean prit la parole, et dit à tous : Moi, je vous baptise d'eau ; mais il vient, celui qui est plus puissant que moi, et je ne suis pas digne de délier la courroie de ses souliers. Lui, il vous baptisera du Saint-Esprit et de feu.
\VS{17}Il a son van à la main ; il nettoiera entièrement son aire, et amassera le froment dans son grenier, mais il brûlera la paille dans un feu qui ne s'éteint point.
\VS{18}Et faisant aussi plusieurs autres exhortations, il évangélisait le peuple.
\VS{19}Mais Hérode le tétrarque, étant repris par Jean au sujet d'Hérodias, femme de Philippe son frère, et à cause de toutes les choses méchantes qu'il faisait,
\VS{20}ajouta encore à toutes les autres celle de mettre Jean en prison.
\TextTitle{Baptême de Jésus-Christ\FTNTT{Mc. 1:9-11; cp. Jn. 1:31-34}}
\VS{21}Tout le peuple se faisait baptiser, Jésus aussi fut baptisé, et pendant qu'il priait, le ciel s'ouvrit,
\VS{22}et le Saint-Esprit descendit sur lui sous une forme corporelle, comme celle d'une colombe. Et une voix fit entendre du ciel ces paroles : Tu es mon Fils bien-aimé, en toi j'ai trouvé mon plaisir.
\TextTitle{Généalogie de Jésus-Christ\FTNTT{v. 31 ; Mt. 1:1-16}}
\VS{23}Jésus avait environ trente ans, lorsqu'il commença son ministère, étant comme on l'estimait, fils de Joseph, fils d'Héli,
\VS{24}fils de Matthat, fils de Lévi, fils de Melchi, fils de Jannaï, fils de Joseph,
\VS{25}fils de Mattathias, fils d'Amos, fils de Nahum, fils d'Esli, fils de Naggaï,
\VS{26}fils de Maath, fils de Mattathias, fils de Sémeï, fils de Josech, fils de Joda,
\VS{27}fils de Joanan, fils de Rhésa, fils de Zorobabel, fils de Salathiel, fils de Néri,
\VS{28}fils de Melchi, fils d'Addi, fils de Kosam, fils d'Elmadam, fils d'Er,
\VS{29}fils de Jésus, fils d'Eliézer, fils de Jorim, fils de Matthat, fils de Lévi,
\VS{30}fils de Siméon, fils de Juda, fils de Joseph, fils de Jonam, fils d'Eliakim,
\VS{31}fils de Méléa, fils de Menna, fils de Matthata, fils de Nathan, fils de David,
\VS{32}fils d'Isaï, fils d'Obed, fils de Booz, fils de Salmon, fils de Naasson,
\VS{33}fils d'Aminadab, fils d’Admin, fils d'Aram, fils d'Esrom, fils de Pérets, fils de Juda,
\VS{34}fils de Jacob, fils d'Isaac, fils d'Abraham, fils de Thara, fils de Nachor,
\VS{35}fils de Seruch, fils de Ragau, fils de Phalek, fils d'Eber, fils de Sala,
\VS{36}fils de Kaïnam, fils d’Arphaxad, fils de Sem, fils de Noé, fils de Lamech,
\VS{37}fils de Mathusala, fils d'Hénoc, fils de Jared, fils de Maléléel, fils de Kaïnan,
\VS{38}fils d'Enos, fils de Seth, fils d'Adam, fils de Dieu.
\Chap{4}
\TextTitle{Tentation de Jésus-Christ\FTNTT{Mt. 4:1 ; Mc. 1:12-13 ; cp. Ge. 3:6 ; 1 Jn. 2:16}}
\VerseOne{}Jésus, rempli du Saint-Esprit, revint du Jourdain, et il fut conduit par l'Esprit dans le désert,
\VS{2}où il fut tenté par le diable quarante jours. Et il ne mangea rien durant ces jours-là, et après qu'ils furent écoulés, il eut faim.
\VS{3}Le diable lui dit : Si tu es le Fils de Dieu, ordonne à cette pierre qu'elle devienne du pain.
\VS{4}Jésus lui répondit, en disant : Il est écrit que l'homme ne vivra pas seulement de pain, mais de toute parole de Dieu\FTNT{De. 8:3.}.
\VS{5}Alors le diable l'emmena sur une haute montagne, et lui montra en un instant tous les royaumes de la terre,
\VS{6}et le diable lui dit : Je te donnerai toute cette puissance et leur gloire ; car elle m'a été donnée, et je la donne à qui je veux.
\VS{7}Si donc tu m'adores, elle sera à toi.
\VS{8}Jésus lui répondit : Va arrière de moi, Satan ! Car il est écrit : Tu adoreras le Seigneur ton Dieu, et tu le serviras lui seul\FTNT{De. 6:13.}.
\VS{9}Le diable le conduisit encore à Jérusalem, et le plaça sur le haut du temple, et lui dit : Si tu es le Fils de Dieu, jette-toi d'ici en bas.
\VS{10}Car il est écrit : Il ordonnera à ses anges à ton sujet, afin qu'ils te gardent;
\VS{11}et ils te porteront dans leurs mains, de peur que ton pied ne heurte contre une pierre\FTNT{Ps. 91:11-12.}.
\VS{12}Mais Jésus répondant, lui dit : Il est écrit : Tu ne tenteras pas le Seigneur, ton Dieu\FTNT{De. 6:16.}.
\VS{13}Après l'avoir tenté de toutes ces manières, le diable s'éloigna de lui pour un temps.
\TextTitle{Jésus-Christ retourne en Galilée\FTNTT{Mt. 4:12-17 ; Mc. 1:14-15}}
\VS{14}Jésus retourna en Galilée dans la puissance de l'Esprit, et sa renommée se répandit dans tout le pays d'alentour.
\VS{15}Il enseignait dans leurs synagogues, et il était glorifié par tous.
\TextTitle{Jésus dans la synagogue de Nazareth le jour du sabbat\FTNTT{cp. Mt. 13:54-58 ; Mc. 6:1-6}}
\VS{16}Il se rendit à Nazareth, où il avait été élevé, et selon sa coutume, il entra dans la synagogue le jour du sabbat et il se leva pour faire la lecture,
\VS{17}et on lui donna le livre du prophète Esaïe et l'ayant déroulé, il trouva le passage où il est écrit :
\VS{18}L'Esprit du Seigneur est sur moi, parce qu'il m'a oint pour évangéliser les pauvres ; il m'a envoyé pour guérir ceux qui ont le cœur brisé,
\VS{19}pour proclamer aux captifs la délivrance, et aux aveugles le recouvrement de la vue ; pour mettre en liberté les opprimés ; pour publier une année de grâce du Seigneur\FTNT{Es. 61:1-2.}.
\VS{20}Ensuite, il roula le livre, le rendit au serviteur, et s'assit. Les yeux de tous ceux qui étaient dans la synagogue étaient fixés sur lui.
\VS{21}Alors il commença à leur dire : Aujourd'hui, cette parole de l'Ecriture que vous venez d'entendre, est accomplie.
\VS{22}Et tous lui rendaient témoignage, et s'étonnaient des paroles pleines de grâce qui sortaient de sa bouche ; et ils disaient : Celui-ci n'est-il pas le fils de Joseph ?
\VS{23}Et il leur dit : Assurément vous me direz ce proverbe : Médecin, guéris-toi toi-même. Et fais ici, dans ton pays, tout ce que nous avons appris que tu as fait à Capernaüm.
\VS{24}Mais il leur dit : En vérité je vous dis qu’aucun prophète n'est reçu dans son pays.
\VS{25}Je vous le dis en vérité : Il y avait plusieurs veuves en Israël, du temps d'Elie, lorsque le ciel fut fermé trois ans et six mois et qu'il y eut une grande famine dans tout le pays ;
\VS{26}toutefois Elie ne fut envoyé vers aucune d'elles, mais seulement vers une femme veuve à Sarepta, dans le pays de Sidon.
\VS{27}Il y avait aussi plusieurs lépreux en Israël du temps d'Elisée, le prophète, toutefois aucun d'eux ne fut purifié, si ce n'est Naaman, le Syrien.
\VS{28}Ils furent tous remplis de colère dans la synagogue lorsqu'ils entendirent ces choses.
\VS{29}Et s'étant levés, ils le chassèrent hors de la ville, et le menèrent jusqu'au bord de la montagne sur laquelle leur ville était bâtie, pour le jeter du haut en bas.
\VS{30}Mais il passa au milieu d'eux, et s'en alla.
\TextTitle{Jésus guérit un possédé\FTNTT{Mc. 1:21-28}}
\VS{31}Il descendit à Capernaüm, ville de Galilée, et il les enseignait les jours de sabbat.
\VS{32}Ils étaient frappés de sa doctrine ; car il parlait avec autorité.
\VS{33}Il y avait dans la synagogue un homme qui avait un esprit de démon impur, et qui s'écria d'une voix forte,
\VS{34}en disant : Ah ! Qu'y a-t-il entre nous et toi, Jésus de Nazareth ? Es-tu venu pour nous détruire ? Je sais qui tu es, le Saint de Dieu.
\VS{35}Jésus le menaça, en lui disant : Tais-toi, et sors de cet homme. Et le démon, l'ayant jeté avec impétuosité au milieu de l'assemblée, sortit de cet homme, sans lui faire aucun mal.
\VS{36}Et tous furent saisis de stupeur, et ils parlaient entre eux, et disaient : Quelle est cette parole ? Il commande avec autorité et puissance aux esprits impurs, et ils sortent ?
\VS{37}Et sa renommée se répandit dans tous les lieux d'alentour.
\TextTitle{Guérison de la belle-mère de Pierre et de plusieurs malades\FTNTT{Mt. 8:14-17 ; Mc. 1:29-34}}
\VS{38}Et quand Jésus se fut levé de la synagogue, et il se rendit à la maison de Simon, et la belle-mère de Simon avait une violente fièvre, et ils le prièrent en sa faveur.
\VS{39}Et s'étant penché sur elle, il menaça la fièvre, et la fièvre la quitta. A l'instant elle se leva, et les servit.
\VS{40}Et après le coucher du soleil, tous ceux qui avaient des malades atteints de diverses maladies, les lui amenèrent. Il imposa les mains à chacun d'eux, et il les guérit.
\VS{41}Les démons aussi sortirent de beaucoup de personnes, en criant et en disant : Tu es le Christ, le Fils de Dieu. Mais il les menaçait fortement, et ne leur permettait pas de dire qu'ils savaient qu'il était le Christ.
\VS{42}Dès que le jour parut, il sortit et alla dans un lieu désert et une foule de gens se mirent à sa recherche, et arrivèrent jusqu'à lui et ils voulaient le retenir, afin qu'il ne les quittât point.
\VS{43}Mais il leur dit : Il faut que j'annonce aux autres villes l'Evangile du Royaume de Dieu, car c'est pour cela que j'ai été envoyé.
\VS{44}Et il prêchait dans les synagogues de la Galilée.
\Chap{5}
\TextTitle{Appel des premiers disciples\FTNTT{Mt. 4:18-22 ; Mc. 1:16-20 ; cp. Jn. 1:35-51 ; 21:1-8}}
\VerseOne{}Or il arriva, comme la foule se jetait toute sur lui pour entendre la parole de Dieu, qu’il se tenait sur le bord du lac de Génézareth.
\VS{2}Et voyant deux barques qui étaient au bord du lac, et dont les pêcheurs étaient descendus, et lavaient leurs rets, il monta dans l’une de ces barques, qui était à Simon.
\VS{3}Il monta dans l'une de ces barques, qui était à Simon, et il le pria de s'éloigner un peu de terre. Puis il s'assit, et de la barque il enseignait la foule.
\VS{4}Et quand il eut cessé de parler, il dit à Simon : Avance en pleine eau, et jetez vos filets pour pêcher.
\VS{5}Et Simon répondant, lui dit : Maître, nous avons travaillé toute la nuit, et nous n’avons rien pris ; toutefois à ta parole je jetterai les filets.
\VS{6}Et ayant fait cela, ils prirent une si grande quantité de poissons que leur filet se rompait.
\VS{7}Et ils firent signe à leurs compagnons qui étaient dans l’autre barque, de venir les aider ; et étant venus, ils remplirent les deux barques, tellement qu’elles s’enfonçaient.
\VS{8}Et quand Simon Pierre vit cela, il se jeta aux genoux de Jésus, en lui disant : Seigneur, retire-toi de moi ; car je suis un homme pécheur.
\VS{9}Parce que la frayeur l’avait saisi, lui et tous ceux qui étaient avec lui, à cause de la prise de poissons qu’ils venaient de faire ; de même que Jacques et Jean, fils de Zébédée, qui étaient compagnons de Simon.
\VS{10}Alors Jésus dit à Simon : Ne crains point ; désormais tu seras un pêcheur d'hommes vivants.
\VS{11}Et quand ils eurent ramené les barques à terre, ils abandonnèrent tout et le suivirent.
\TextTitle{Guérison d'un lépreux\FTNTT{Mt. 8:2-4 ; Mc. 1:40-45}}
\VS{12}Et il arriva, comme il était dans une des villes, voici un homme plein de lèpre, voyant Jésus, se jeta sur sa face et le supplia, disant : Seigneur, si tu veux, tu peux me rendre pur.
\VS{13}Jésus étendit la main, et le toucha, en disant : Je le veux, sois pur. Aussitôt la lèpre le quitta.
\VS{14}Et il lui commanda de ne le dire à personne, mais va, lui dit-il, et montre-toi sacrificateur, et offre pour ta purification ce que Moïse a commandé\FTNT{Lé. 13 et 14.}, pour leur servir de témoignage.
\VS{15}Et sa renommée se répandait de plus en plus, tellement que de grandes foules s’assemblaient pour l’entendre, et pour être guéries par lui de leurs maladies.
\VS{16}Mais il se tenait retiré dans les déserts, et priait.
\TextTitle{Guérison d'un paralytique\FTNTT{Mt. 9:2-8 ; Mc. 2:3-12}}
\VS{17}Un jour Jésus enseignait. Et des pharisiens et des docteurs de la loi étaient là assis, venus de tous les villages de la Galilée, et de la Judée et de Jérusalem ; et la puissance du Seigneur se manifestait par des guérisons.
\VS{18}Et voici des hommes qui portaient sur un lit un homme qui était paralytique, et ils cherchaient le moyen de le porter dans la maison, et de le mettre devant lui.
\VS{19}Comme ils ne savaient pas par où l'introduire, à cause de la foule, ils montèrent sur le toit, et ils le descendirent par une ouverture, avec son lit, au milieu de la foule, devant Jésus.
\VS{20}Voyant leur foi, il dit au paralytique : Homme, tes péchés te sont pardonnés.
\VS{21}Alors les scribes et les pharisiens commencèrent à raisonner en eux-mêmes, disant : Qui est celui-ci qui profère des blasphèmes ? Qui est-ce qui peut pardonner les péchés, si ce n'est Dieu seul ?
\VS{22}Mais Jésus, connaissant leurs pensées, prit la parole et leur dit : Pourquoi raisonnez-vous ainsi en vous-mêmes ?
\VS{23}Lequel est le plus aisé de dire : Tes péchés te sont pardonnés ; ou de dire : Lève-toi et marche ?
\VS{24}Or afin que vous sachiez que le Fils de l'homme a le pouvoir sur la terre de pardonner les péchés, il dit au paralytique : Je te l'ordonne, lève-toi, prends ton lit, et va dans ta maison.
\VS{25}Et à l'instant, le paralytique s'étant levé devant eux, prit le lit sur lequel il était couché, et s'en alla dans sa maison, glorifiant Dieu.
\VS{26}Ils furent tous saisis d'étonnement, et ils glorifiaient Dieu ; et étant remplis de crainte, ils disaient : certainement nous avons vu aujourd'hui des choses étranges.
\TextTitle{Appel de Lévi\FTNTT{Mt. 9:9 ; Mc. 2:13-14}}
\VS{27}Après cela, Jésus sortit, et il vit un publicain nommé Lévi, assis au bureau des péages, et il lui dit : Suis-moi.
\VS{28}Et abandonnant tout, il se leva, et le suivit.
\TextTitle{Appelle des pécheurs à la repentance\FTNTT{Mt. 9:10-15 ; Mc. 2:13-14}}
\VS{29}Et Lévi lui fit un grand festin dans sa maison ; et il y avait une grande foule de publicains et d’autres gens qui étaient avec eux à table.
\VS{30}Les scribes de ce lieu-là et les pharisiens, murmuraient contre ses disciples en disant : Pourquoi mangez-vous et buvez-vous avec les publicains et les gens de mauvaise vie ?
\VS{31}Mais Jésus, prenant la parole, leur dit : ceux qui sont en santé n’ont pas besoin de médecin, mais ceux qui se portent mal.
\VS{32}Je ne suis point venu appeler à la repentance les justes, mais les pécheurs.
\VS{33}Ils lui dirent aussi : pourquoi est-ce que les disciples de Jean jeûnent souvent, et font des prières et également ceux des Pharisiens, mais les tiens mangent et boivent ?
\VS{34}Il leur répondit : Pouvez-vous faire jeûner les amis de l'Epoux pendant que l'Epoux est avec eux ?
\VS{35}Mais les jours viendront où l'Epoux leur sera enlevé alors ils jeûneront en ces jours-là.
\TextTitle{Parabole du drap neuf et des outres neuves\FTNTT{Mt. 9:16-17 ; Mc. 2:21-22}}
\VS{36}Puis il leur dit cette parabole : Personne ne met une pièce d'un habit neuf à un vieil habit ; autrement le neuf déchire le vieux, et la pièce du neuf ne s'accorde pas avec le vieux.
\VS{37}Et personne ne met du vin nouveau dans de vieilles outres ; autrement le vin nouveau fait rompre les outres, et il se répand, et les outres sont perdues.
\VS{38}Mais le vin nouveau doit être mis dans des outres neuves ; et ainsi ils se conservent l'un et l'autre.
\VS{39}Et personne, après avoir bu du vin vieux, ne veut du nouveau, car il dit : Le vieux est meilleur.
\Chap{6}
\TextTitle{Jésus, le Maître du sabbat\FTNTT{Mt. 12:1-8 ; Mc. 2:23-28}}
\VerseOne{}Or il arriva un jour de sabbat appelé second-premier, qu'il passait par des blés ; et ses disciples arrachaient des épis et les froissant dans leurs mains, ils les mangeaient.
\VS{2}Et quelques pharisiens leur dirent : Pourquoi faites-vous ce qu'il n'est pas permis de faire les jours du sabbat ?
\VS{3}Et Jésus prenant la parole, leur dit : N'avez-vous pas lu ce que fit David quand il eut faim, lui et ceux qui étaient avec lui ;
\VS{4}comment il entra dans la maison de Dieu, et prit les pains de proposition, et en mangea, et en donna aussi à ceux qui étaient avec lui, bien qu'il ne soit permis qu'aux sacrificateurs d'en manger ?\FTNT{2 S. 21:1-7.}
\VS{5}Puis il leur dit : Le Fils de l'homme est Maître même du sabbat.
\TextTitle{Guérison d'un homme à la main sèche\FTNTT{Mt. 12:9-13 ; Mc. 3:1-5}}
\VS{6}Et il arriva, un autre jour de sabbat, qu'il entra dans la synagogue, et qu'il enseignait et il s'y trouvait là un homme dont la main droite était sèche.
\VS{7}Or les scribes et les pharisiens l'observaient pour voir s'il ferait une guérison le jour du sabbat ; c'était afin d'avoir sujet de l'accuser.
\VS{8}Mais il connaissait leurs pensées et il dit à l'homme qui avait la main sèche : Lève-toi, et tiens-toi debout au milieu. Et il se leva et se tint debout.
\VS{9}Puis Jésus leur dit : Je vous demande une chose : Est-il permis de faire du bien les jours de sabbat, ou de faire du mal ? De sauver une personne, ou de la laisser mourir ?
\VS{10}Et ayant regardé tous ceux qui étaient autour de lui, il dit à l’homme : étends ta main ; ce qu’il fit, et sa main fut rendue saine comme l’autre.
\VS{11}Et ils furent remplis de fureur, et ils s’entretenaient ensemble touchant ce qu’ils pourraient faire à Jésus.
\VS{12}Or il arriva en ces jours-là, qu’il s’en alla sur une montagne pour prier, et qu’il passa toute la nuit à prier Dieu.
\TextTitle{Choix des douze apôtres\FTNTT{cp. Mt. 10:2-4 ; Mc. 3:13-19}}
\VS{13}Et quand le jour fut venu, il appela ses disciples. Et en ayant choisi douze d’entre eux, il les nomma apôtres :
\VS{14}Simon, qu'il nomma Pierre, et André son frère, Jacques et Jean, Philippe et Barthélemy ;
\VS{15}Matthieu et Thomas, Jacques fils d'Alphée, et Simon surnommé zélote\FTNT{Zélote : « Celui qui est zélé ». Les zélotes faisaient partie d'un mouvement politique Juif du premier siècle ap. J.-C., qui cherchait à inciter les gens de la province de Judée à se rebeller contre l'Empire Romain, et à le chasser du pays par les armes, pendant la Grande Révolte Juive (66-70 ap. J.-C.). Lorsque les Romains introduisirent le culte impérial, les Juifs se rebellèrent et furent réprimés. Les zélotes considéraient qu'Israël appartenait seulement à un roi Juif de la descendance de David. De plus, reconnaître l'empereur équivalait, à leurs yeux, à renier Dieu. Le mouvement zélote se réclamait intentionnellement de modèles bibliques tels que Phinées, le fils zélé d'Eléazar, fils d'Aaron (No. 25:11). Ce dernier s'était illustré par l'assassinat d'un prince de tribu d'Israël qui s'était fourvoyé dans la luxure aux yeux de tous.} ;
\VS{16}Jude, frère de Jacques, et Judas Iscariot, qui devint traître.
\TextTitle{Enseignement sur la montagne\FTNTT{Mt. 5-7}}
\VS{17}Puis descendant avec eux, il s'arrêta sur une plaine avec la foule de ses disciples et une grande multitude de peuple de toute la Judée, et de Jérusalem, et de la contrée maritime de Tyr et de Sidon, qui étaient venus pour l'entendre, et pour être guéris de leurs maladies.
\VS{18}Ceux aussi qui étaient tourmentés par des esprits impurs furent guéris.
\VS{19}Et toute la foule cherchait à le toucher, parce qu'une force sortait de lui et les guérissait tous.
\TextTitle{Enseignement de Jésus\FTNTT{Mt. 5:3-12}}
\VS{20}Alors Jésus, levant les yeux vers ses disciples, leur dit : Heureux vous qui êtes pauvres, car le Royaume de Dieu vous appartient.
\VS{21}Heureux vous qui avez faim maintenant, car vous serez rassasiés ! Heureux vous qui pleurez maintenant, car vous serez dans la joie !
\VS{22}Heureux serez-vous quand les hommes vous haïront, vous chasseront, vous outrageront, et rejetteront votre nom comme infâme, à cause du Fils de l'homme.
\VS{23}Réjouissez-vous en ce jour-là, et tressaillez d'allégresse, parce que votre récompense sera grande dans le ciel ; car leurs pères en faisaient de même aux prophètes.
\VS{24}Mais malheur à vous riches, car vous avez votre consolation.
\VS{25}Malheur à vous qui êtes rassasiés, car vous aurez faim. Malheur à vous qui riez maintenant, car vous serez dans le deuil et dans les larmes.
\VS{26}Malheur à vous quand tous les hommes diront du bien de vous ; car leurs pères en faisaient de même aux faux prophètes.
\VS{27}Mais à vous qui m’entendez, je vous dis : aimez vos ennemis ; faites du bien à ceux qui vous haïssent.
\VS{28}Bénissez ceux qui vous maudissent, et priez pour ceux qui vous maltraitent.
\VS{29}Si quelqu'un te frappe sur une joue, présente-lui aussi l'autre. Si quelqu'un prend ton manteau, ne l'empêche pas de prendre aussi ta tunique.
\VS{30}Donne à quiconque te demande, et ne réclame pas ton bien à celui qui s'en empare.
\VS{31}Ce que vous voulez que les hommes fassent pour vous, faites-le de même pour eux.
\VS{32}Mais si vous aimez seulement ceux qui vous aiment, quel gré vous en saura-t-on ? Les pécheurs aussi aiment ceux qui les aiment.
\VS{33}Et si vous faites du bien à ceux qui vous font du bien, quel gré vous en saura-t-on ? Les pécheurs aussi font de même.
\VS{34}Et si vous prêtez à ceux de qui vous espérez recevoir, quel gré vous en saura-t-on ? Les pécheurs aussi prêtent aux pécheurs, afin de recevoir la pareille.
\VS{35}C'est pourquoi aimez vos ennemis et faites-leur du bien, et prêtez sans rien espérer, et votre récompense sera grande, et vous serez les fils du Très-Haut, car il est bon envers les ingrats et les méchants.
\VS{36}Soyez donc miséricordieux comme votre Père est miséricordieux.
\VS{37}Ne jugez point, et vous ne serez point jugés ; ne condamnez point, et vous ne serez point condamnés ; absolvez, et vous serez absous.
\VS{38}Donnez, et il vous sera donné : On versera dans votre sein une bonne mesure, serrée, secouée et qui déborde ; car on vous mesurera avec la mesure dont vous vous serez servis.
\VS{39}Il leur disait aussi cette parabole : Un aveugle peut-il conduire un aveugle ? Ne tomberont-ils pas tous deux dans la fosse ?
\VS{40}Le disciple n'est pas au-dessus de son maître ; mais tout disciple accompli sera comme son maître.
\VS{41}Pourquoi regardes-tu la paille qui est dans l'œil de ton frère, et n'aperçois-tu pas la poutre qui est dans ton propre œil ?
\VS{42}Ou comment peux-tu dire à ton frère : Mon frère, laisse-moi enlever la paille qui est dans ton œil, toi qui ne vois pas la poutre qui est dans ton œil ? Hypocrite, ôte premièrement la poutre de ton œil, et après cela tu verras comment ôter la paille qui est dans l'œil de ton frère.
\VS{43}Ce n'est pas un bon arbre qui porte du mauvais fruit, ni un mauvais arbre qui porte du bon fruit.
\VS{44}Car chaque arbre se reconnaît à son fruit. On ne cueille pas des figues sur des épines, et l'on ne vendange pas des raisins sur des ronces.
\VS{45}L'homme de bien tire de bonnes choses du bon trésor de son cœur, et l'homme méchant tire de mauvaises choses du mauvais trésor de son cœur ; car c'est de l'abondance du cœur que la bouche parle.
\TextTitle{Parabole des deux bâtisseurs et des deux fondements\FTNTT{Mt. 7:24-27}}
\VS{46}Mais pourquoi m'appelez-vous Seigneur, Seigneur, et ne faites-vous pas ce que je dis ?
\VS{47}Je vous montrerai à qui est semblable celui qui vient à moi, entend mes paroles, et les met en pratique.
\VS{48}Il est semblable à un homme qui bâtissant une maison, a creusé, creusé profondément, et a mis le fondement sur le roc. Une inondation est venue, et le torrent s'est jeté contre cette maison, sans pouvoir l'ébranler, parce qu'elle était bâtie sur le roc.
\VS{49}Mais celui qui entend mes paroles, et ne les met pas en pratique, est semblable à un homme qui a bâti sa maison sur la terre, sans fondement. Le torrent s'est jeté contre elle ; aussitôt elle est tombée, et la ruine de cette maison a été grande.
\Chap{7}
\TextTitle{Guérison du serviteur d'un centenier\FTNTT{Mt. 8:5-13}}
\VerseOne{}Et quand il eut achevé tout ce discours devant le peuple qui l'écoutait, il entra dans Capernaüm.
\VS{2}Un centenier avait un serviteur, auquel il était très attaché, et qui était malade, sur le point de mourir.
\VS{3}Ayant entendu parler de Jésus, il envoya vers lui quelques anciens des Juifs, pour le prier de venir guérir son serviteur.
\VS{4}Et étant venu à Jésus, ils lui prièrent instamment, disant : Il mérite que tu lui accordes cela.
\VS{5}Car, disaient-ils, il aime notre nation, et c'est lui qui a bâti notre synagogue.
\VS{6}Jésus s'en alla donc avec eux. Il n'était guère éloigné de la maison, quand le centenier envoya ses amis au-devant de lui, pour lui dire : Seigneur, ne te fatigue point ; car je ne suis pas digne que tu entres sous mon toit.
\VS{7}C'est pourquoi aussi je ne me suis pas cru digne d'aller moi-même vers toi ; mais dis seulement une parole, et mon serviteur sera guéri.
\VS{8}Car, moi qui suis un homme soumis à des supérieurs, j'ai des soldats sous mes ordres ; et je dis à l'un : Va, et il va ; et à un autre : Viens, et il vient ; et à mon serviteur : Fais cela, et il le fait.
\VS{9}Lorsque Jésus entendit ces paroles, il admira le centenier ; et se tournant vers la foule qui le suivait, il dit : Je vous le dis, je n'ai pas trouvé, même en Israël, une si grande foi.
\VS{10}Et quand ceux qui avaient été envoyés furent de retour à la maison, ils trouvèrent le serviteur qui avait été malade, se portant bien.
\TextTitle{Le fils de la veuve de Naïn ressuscite}
\VS{11}Le jour suivant, Jésus alla dans une ville appelée Naïn ; plusieurs de ses disciples et une grande foule allaient avec lui.
\VS{12}Et comme il approchait de la porte de la ville, voici, on portait en terre un mort, fils unique de sa mère, qui était veuve ; et il y avait avec elle un grand nombre de gens de la ville.
\VS{13}Le Seigneur l'ayant vue, fut ému de compassion pour elle ; et il lui dit : Ne pleure pas !
\VS{14}Il s'approcha, et toucha le cercueil. Ceux qui le portaient s'arrêtèrent. Et il dit : Jeune homme, je te dis, lève-toi !
\VS{15}Et le mort s'assit, et se mit à parler. Et Jésus le rendit à sa mère.
\VS{16}Et ils furent tous saisis de crainte, et ils glorifiaient Dieu, disant : Certainement un grand prophète a paru parmi nous ; et Dieu a visité son peuple.
\VS{17}Cette parole sur ce miracle se répandit dans toute la Judée, et dans tout le pays d'alentour.
\VS{18}Jean fut informé de toutes ces choses par ses disciples.
\TextTitle{Jean-Baptiste, le plus grand des hommes\FTNTT{Mt. 11:1-19}}
\VS{19}Il en appela deux, et les envoya vers Jésus pour lui dire : Es-tu celui qui devait venir, ou devons-nous en attendre un autre ?
\VS{20}Et étant venus à lui, ils lui dirent : Jean-Baptiste nous a envoyés auprès de toi pour te dire : Es-tu celui qui devait venir, ou devons-nous en attendre un autre ?
\VS{21}A l'heure même, Jésus guérit plusieurs personnes de maladies,  d'infirmités, et d'esprits malins ; et il rendit la vue à plusieurs aveugles.
\VS{22}Ensuite Jésus leur répondit, et leur dit : Allez, et rapportez à Jean ce que vous avez vu et entendu : Les aveugles recouvrent la vue, les boiteux marchent, les lépreux sont purifiés, les sourds entendent, les morts ressuscitent, l'Evangile est annoncé aux pauvres.
\VS{23}Heureux celui qui n'aura point été scandalisé à cause de moi !
\VS{24}Lorsque les messagers de Jean furent partis, Jésus se mit à dire à la foule au sujet de Jean : Qu'êtes-vous allés voir au désert ? Un roseau agité par le vent ?
\VS{25}Mais qu'êtes-vous allés voir ? Un homme vêtu d'habits précieux ? Voici, ceux qui portent des habits magnifiques, et qui vivent dans les délices, sont dans les maisons des rois.
\VS{26}Mais qu'êtes-vous donc allés voir ? Un prophète ? Oui, vous dis-je, et plus qu'un prophète.
\VS{27}C'est de lui qu'il est écrit : Voici, j'envoie mon messager devant ta face, et il préparera ta voie devant toi\FTNT{Mal. 3:1.}.
\VS{28}Car je vous dis, parmi ceux qui sont nés de femmes, il n'y a aucun prophète plus grand que Jean-Baptiste. Cependant, le plus petit dans le Royaume de Dieu est plus grand que lui.
\VS{29}Et tout le peuple qui entendait cela, et les publicains, justifiaient Dieu, ayant été baptisés du baptême de Jean.
\VS{30}Mais les pharisiens, et les docteurs de la loi, qui n'avaient point été baptisés par lui, rendirent le dessein de Dieu inutile à leur égard.
\VS{31}Alors le Seigneur dit : A qui donc comparerai-je les hommes de cette génération ; et à quoi ressemblent-ils ?
\VS{32}Ils sont semblables aux enfants qui sont assis sur la place publique, et qui se parlant les uns aux autres, disent : Nous vous avons joué de la flûte, et vous n'avez pas dansé ; nous vous avons chanté des complaintes, et vous n'avez pas pleuré.
\VS{33}Car Jean-Baptiste est venu ne mangeant point de pain, et ne buvant point de vin ; et vous dites : Il a un démon.
\VS{34}Le Fils de l'homme est venu mangeant et buvant ; et vous dites : Voici un mangeur et un buveur, un ami des publicains et des pécheurs.
\VS{35}Mais la sagesse a été justifiée par tous ses enfants.
\TextTitle{Une pécheresse pardonnée par Jésus}
\VS{36}Un des pharisiens pria Jésus de manger chez lui ; et Jésus entra dans la maison de ce pharisien, et se mit à table.
\VS{37}Et voici, il y avait dans la ville une femme pécheresse, qui ayant su que Jésus était à table dans la maison du pharisien, apporta un vase d'albâtre plein de parfum,
\VS{38}et se tenant derrière à ses pieds, et pleurant, elle les mouilla de ses larmes, elle les essuya avec ses propres cheveux, et lui baisa les pieds, et les oignit de cette huile odoriférante.
\VS{39}Mais le pharisien qui l'avait invité, voyant cela, dit en lui-même : Si cet homme était prophète, certes il saurait qui et de quelle espèce est la femme qui le touche, il saurait que c'est une pécheresse.
\TextTitle{Parabole des deux débiteurs}
\VS{40}Et Jésus prenant la parole, lui dit : Simon, j'ai quelque chose à te dire. Maître, parle, répondit-il.
\VS{41}Un créancier avait deux débiteurs : L'un lui devait cinq cents deniers, et l'autre cinquante.
\VS{42}Et comme ils n'avaient pas de quoi payer, il leur remit à tous deux leur dette. Lequel l'aimera le plus ?
\VS{43}Et Simon répondant lui dit : Celui, je pense, à qui il a le plus remis. Jésus lui dit : Tu as droitement jugé.
\VS{44}Alors se tournant vers la femme, il dit à Simon : Vois-tu cette femme ? Je suis entré dans ta maison, et tu ne m'as point donné d'eau pour laver mes pieds ; mais elle, elle les a mouillés de ses larmes, elle les a essuyés avec ses propres cheveux.
\VS{45}Tu ne m'as point donné un baiser, mais elle, depuis que je suis entré, n'a cessé d'embrasser mes pieds.
\VS{46}Tu n'as pas oint ma tête d'huile ; mais elle, elle a oint mes pieds d'une huile odoriférante.
\VS{47}C'est pourquoi je te le dis, ses nombreux péchés ont été pardonnés, car elle a beaucoup aimé. Or celui à qui on pardonne peu, aime peu.
\VS{48}Puis il dit à la femme : Tes péchés sont pardonnés.
\VS{49}Ceux qui étaient avec lui à table, se mirent à dire en eux-mêmes : Qui est celui-ci qui pardonne même les péchés ?
\VS{50}Mais il dit à la femme : Ta foi t'a sauvée. Va en paix.
\Chap{8}
\TextTitle{Les femmes au service de Jésus durant son ministère}
\VerseOne{}Or il arriva après cela qu'il allait de ville en ville, et de villages en villages, prêchant et annonçant l'Evangile du Royaume de Dieu.
\VS{2}Les douze disciples étaient auprès de lui avec quelques femmes aussi qu'il avait délivrées d'esprits malins et de maladies : Marie de Magdala, de laquelle étaient sortis sept démons,
\VS{3}Et Jeanne, femme de Chuza, intendant d'Hérode, Susanne, et plusieurs autres qui l'assistaient de leurs biens.
\TextTitle{Parabole du semeur\FTNTT{Mt. 13:1-23 ; Mc. 4:1-20}}
\VS{4}Et comme une grande foule s'étant assemblée, et des gens étant venus de diverses villes auprès de lui, il leur dit cette parabole :
\VS{5}Un semeur sortit pour semer sa semence ; et en semant, une partie de la semence tomba le long du chemin ; elle fut foulée aux pieds, et les oiseaux du ciel la mangèrent toute.
\VS{6}Une autre partie tomba dans un endroit pierreux ; et quand elle fut levée, elle sécha, parce qu'elle n'avait point d'humidité.
\VS{7}Une autre partie tomba au milieu des épines ; les épines crurent avec elle, et l'étouffèrent.
\VS{8}Une autre partie tomba dans une bonne terre ; quand elle fut levée, elle donna du fruit au centuple. En disant ces choses, Jésus dit à haute voix : Que celui qui a des oreilles pour entendre, qu'il entende.
\VS{9}Et ses disciples l'interrogèrent pour savoir ce que signifiait cette parabole.
\VS{10}Il répondit : Il vous a été donné de connaître les mystères du Royaume de Dieu, mais pour les autres, cela leur est dit en paraboles, afin qu'en voyant ils ne voient point, et qu'en entendant ils ne comprennent point.
\VS{11}Voici donc ce que signifie cette parabole : La semence, c'est la parole de Dieu.
\VS{12}Ceux qui ont reçu la semence le long du chemin, ce sont ceux qui entendent la parole ; mais ensuite le diable vient et ôte la parole de leur cœur, de peur qu'ils ne croient et soient sauvés.
\VS{13}Et ceux qui ont reçu la semence dans un endroit pierreux, ce sont ceux qui, lorsqu'ils entendent la parole, la reçoivent avec joie ; mais ils n'ont point de racine ; ils croient pour un temps, mais au moment de la tentation ils se retirent.
\VS{14}Et ce qui est tombé parmi les épines, ce sont ceux qui ayant entendu la parole, s'en vont, et la laissent étouffer par les soucis, les richesses, et les plaisirs de la vie, et ils ne portent point de fruit qui vienne à maturité.
\VS{15}Mais ce qui est tombé dans une bonne terre, ce sont ceux qui ayant entendu la parole, la retiennent dans un cœur honnête et bon, et portent du fruit avec persévérance.
\TextTitle{Parabole du chandelier\FTNTT{Mt. 5:15-16 ; Mc. 4:21-23 ; Lu. 11:33-36}}
\VS{16}Personne, après avoir allumé la lampe, ne la couvre d'un vase ni ne la met sous un lit, mais il la met sur un chandelier, afin que ceux qui entrent voient la lumière.
\VS{17}Car il n'est rien de secret qui ne doive être découvert ; rien de caché qui ne doive être connu et qui ne vienne en évidence.
\VS{18}Prenez donc garde à la manière dont vous écoutez ; car on donnera à celui qui a, mais à celui qui n'a pas, on ôtera même ce qu'il croit avoir.
\TextTitle{La famille spirituelle\FTNTT{Mt. 12:46-50 ; Mc. 3:31-35}}
\VS{19}Alors sa mère et ses frères vinrent vers lui, mais ils ne pouvaient l'aborder à cause de la foule.
\VS{20}Et on vint lui dire : Ta mère et tes frères sont dehors, et ils désirent te voir.
\VS{21}Mais il répondit : Ma mère et mes frères sont ceux qui écoutent la parole de Dieu, et qui la mettent en pratique.
\TextTitle{Jésus calme la tempête\FTNTT{Mt. 8:23-27 ; Mc. 4:35-41}}
\VS{22}Or il arriva qu'un jour, Jésus monta dans une barque avec ses disciples, et il leur dit : Passons de l'autre côté du lac ; et ils partirent.
\VS{23}Pendant qu'ils naviguaient, il s'endormit. Un vent impétueux se leva sur le lac, la barque se remplissait d'eau, et ils étaient en danger.
\VS{24}Ils s'approchèrent et le réveillèrent, en disant : Maître ! Maître ! Nous périssons ! S'étant réveillé, il menaça le vent et les flots qui s'apaisèrent, et le calme revint.
\VS{25}Alors il leur dit : Où est votre foi ? Saisis de frayeur et d'étonnement, ils se dirent les uns aux autres : Quel est donc celui-ci qui commande même aux vents et à l'eau, et à qui ils obéissent ?
\TextTitle{Le démoniaque de Gérasa (Gadara) délivré\FTNTT{Mt. 8:28-34 ; Mc. 5:1-20}}
\VS{26}Puis ils abordèrent dans le pays des Géraséniens qui est vis-à-vis de la Galilée.
\VS{27}Et quand il fut descendu à terre, il vint à sa rencontre un homme de cette ville, qui depuis longtemps était possédé de plusieurs démons. Il ne portait point de vêtements, avait sa demeure, non dans une maison, mais dans les sépulcres.
\VS{28}Ayant vu Jésus, il s'écria et se prosterna devant lui, disant à haute voix : Qu'y a-t-il entre moi et toi, Jésus, Fils du Dieu Très-Haut ? Je te prie, ne me tourmente point.
\VS{29}Car Jésus commandait à l'esprit impur de sortir de cet homme, dont il s'était emparé depuis longtemps. On le gardait lié de chaînes et les fers aux pieds, mais il rompait les liens, et il était entrainé par le démon dans les déserts.
\VS{30}Jésus lui demanda : Quel est ton nom ? Légion\FTNT{Voir commentaire en Mc. 5:9.} répondit-il. Car plusieurs démons étaient entrés en lui.
\VS{31}Et ils priaient Jésus de ne pas leur ordonner d'aller dans l'abîme.
\VS{32}Or il y avait là, dans la montagne, un grand troupeau de pourceaux qui paissaient. Les démons supplièrent Jésus de leur permettre d'entrer dans ces pourceaux. Il le leur permit.
\VS{33}Les démons sortirent de cet homme et entrèrent dans les pourceaux ; et le troupeau se précipita des pentes escarpées dans le lac, et se noya.
\VS{34}Ceux qui les faisaient paître, voyant ce qui était arrivé, s'enfuirent et allèrent le raconter dans la ville et dans les campagnes.
\VS{35}Les gens sortirent pour voir ce qui était arrivé. Ils vinrent auprès de Jésus, et ils trouvèrent l'homme de qui étaient sortis les démons, assis aux pieds de Jésus, vêtu et dans son bon sens ; et ils furent saisis de frayeur.
\VS{36}Et ceux qui avaient vu ce qui s'était passé leur racontèrent comment le démoniaque avait été délivré.
\VS{37}Alors toute cette multitude venue de divers endroits voisins des Géraséniens, le prièrent de se retirer de chez eux ; car ils étaient saisis d'une grande crainte. Jésus monta donc dans la barque, et s'en retourna.
\VS{38}L'homme de qui étaient sortis les démons lui demanda la permission de rester avec lui ; mais Jésus le renvoya, en lui disant :
\VS{39}Retourne dans ta maison, et raconte tout ce que Dieu t'a fait. Il s'en alla donc, et publia par toute la ville tout ce que Jésus avait fait pour lui.
\VS{40}Quand Jésus fut de retour, la foule le reçut avec joie ; car tous l'attendaient.
\TextTitle{Les deux guérisons\FTNTT{Mt. 9:18-26 ; Mc. 5:21-43}}
\VS{41}Et voici, un homme appelé Jaïrus, qui était chef de la synagogue, vint et se jetant aux pieds de Jésus, le pria d'entrer dans sa maison
\VS{42}parce qu'il avait une fille unique, âgée d'environ douze ans, qui se mourait. Pendant que Jésus y allait, il était pressé par la foule.
\VS{43}Or, il y avait une femme atteinte d'une perte de sang depuis douze ans, et qui avait dépensé tout son bien pour les médecins, sans qu'aucun n'ait pu la guérir.
\VS{44}S'approchant de lui par derrière, elle toucha le bord de son vêtement. Au même instant la perte de sang s'arrêta.
\VS{45}Jésus dit : Qui m'a touché ? Comme tous le niaient, Pierre et ceux qui étaient avec lui, dirent : Maître, la foule qui t'entoure te presse et tu dis : Qui m'a touché ?
\VS{46}Mais Jésus dit : Quelqu'un m'a touché, car j'ai connu qu'une force est sortie de moi.
\VS{47}Alors la femme, se voyant découverte, vint toute tremblante se jeter à ses pieds, lui déclara devant tout le peuple pour quelle raison elle l'avait touché, et comment elle avait été guérie à l'instant.
\VS{48}Jésus lui dit : Ma fille, rassure-toi. Ta foi t'a guérie. Va en paix.
\VS{49}Et comme il parlait encore, quelqu'un vint de chez le chef de la synagogue, qui lui dit : Ta fille est morte, n'importune pas le Maître.
\VS{50}Mais Jésus ayant entendu cela, dit au père de la fille : Ne crains point ; crois seulement, et elle sera guérie.
\VS{51}Et quand il fut arrivé à la maison, il ne permit à personne d'entrer avec lui, si ce n'est à Pierre, à Jacques et à Jean, et au père et à la mère de la fille.
\VS{52}Or il la pleuraient tous et de douleur, ils se frappaient la poitrine ; mais il leur dit : Ne pleurez point, elle n'est pas morte, mais elle dort.
\VS{53}Et ils se moquaient de lui, sachant bien qu'elle était morte.
\VS{54}Mais les ayant tous fait sortir, il prit la main de la fille, et dit d'une voix forte : Enfant, lève-toi !
\VS{55}Et son esprit revint en elle, et à l'instant elle se leva ; et Jésus ordonna qu'on lui donne à manger.
\VS{56}Les parents de la fille furent dans l'étonnement, et il leur commanda de ne dire à personne ce qui était arrivé.
\Chap{9}
\TextTitle{Mission des douze apôtres\FTNTT{Mt. 10:1-42 ; cp. Mc. 6:7-13}}
\VerseOne{}Puis, Jésus ayant assemblé ses douze disciples, leur donna puissance et autorité sur tous les démons, avec le pouvoir de guérir les malades.
\VS{2}Il les envoya prêcher le Royaume de Dieu et guérir les malades.
\VS{3}Il leur dit : Ne prenez rien pour le voyage, ni bâton, ni sac, ni pain, ni argent ; et n'ayez pas chacun deux tuniques.
\VS{4}Dans quelque maison que vous entriez, demeurez-y jusqu'à ce que vous partiez de là.
\VS{5}Et partout où l'on ne vous recevra pas, en partant de cette ville secouez la poussière de vos pieds, en témoignage contre eux.
\VS{6}Ils partirent, et ils allèrent de village en village, évangélisant et opérant des guérisons partout.
\VS{7}Or Hérode le Tétrarque entendit parler de toutes les choses que Jésus faisait ; et il ne savait que penser. Car quelques-uns disaient que Jean était ressuscité des morts ;
\VS{8}d'autres, qu'Elie était apparu ; et d'autres, que quelqu'un des anciens prophètes était ressuscité.
\VS{9}Mais Hérode dit : J'ai fait décapiter Jean. Qui est donc celui-ci de qui j'entends dire de telles choses ? Et il cherchait à le voir.
\VS{10}Puis les apôtres étant de retour, lui racontèrent toutes les choses qu'ils avaient faites. Jésus les prit avec lui, et se retira dans un lieu désert, près de la ville appelée Bethsaïda.
\VS{11}Les foules l'ayant su, le suivirent. Jésus les accueillit, et il leur parlait du Royaume de Dieu ; il guérit aussi ceux qui avaient besoin d'être guéris.
\TextTitle{Multiplication des pains pour cinq mille hommes\FTNTT{Mt. 14:15-21 ; Mc. 6:32-44 ; Jn. 6:1-14}}
\VS{12}Comme le jour commençait à baisser, les douze disciples s'approchèrent, et lui dirent : Renvoie la foule, afin qu'elle aille dans les villages et dans les campagnes des environs, pour se loger et pour trouver à manger ; car nous sommes ici dans un pays désert.
\VS{13}Et il leur dit : Donnez-leur vous-mêmes à manger. Et ils dirent : Nous n'avons que cinq pains et deux poissons ; à moins que nous n'allions nous-mêmes acheter des vivres pour tout ce peuple.
\VS{14}Or, il y avait environ cinq mille hommes. Jésus dit aux disciples : Faites-les asseoir par rangées de cinquante chacune.
\VS{15}Ils le firent ainsi, et les firent tous asseoir.
\VS{16}Jésus prit les cinq pains et les deux poissons, et levant les yeux au ciel, il les bénit. Puis, il les rompit, et il les donna à ses disciples afin qu'ils les distribuent à la foule.
\VS{17}Tous mangèrent et furent rassasiés, et l'on remporta douze paniers pleins de morceaux de pain qui restaient.
\TextTitle{Pierre reconnaît Jésus comme le Messie\FTNTT{Mt. 16:13-16 ; Mc. 8:27-30 ; Jn. 6:66-71}}
\VS{18}Or il arriva que comme il était dans un lieu retiré pour prier, et que les disciples étaient avec lui, il les interrogea, disant : Qui disent les foules que je suis ?
\VS{19}Ils lui répondirent : Les uns disent que tu es Jean-Baptiste ; les autres, Elie ; et les autres, qu'un des anciens prophètes est ressuscité.
\VS{20}Il leur dit alors : Et vous, qui dites-vous que je suis ? Et Pierre répondit : Tu es le Christ de Dieu.
\VS{21}Jésus leur défendit sévèrement de ne le dire à personne.
\TextTitle{Jésus annonce sa mort et sa résurrection\FTNTT{Mt. 16:21-23 ; Mc. 8:31-33}}
\VS{22}Et il leur dit : Il faut que le Fils de l'homme souffre beaucoup, et qu'il soit rejeté par les anciens, par les principaux sacrificateurs et par les scribes, et qu'il soit mis à mort, et qu'il ressuscite le troisième jour.
\TextTitle{La consécration du disciple\FTNTT{Mt. 16:24-28 ; Mc. 8:34-38}}
\VS{23}Puis il dit à tous : Si quelqu'un veut venir après moi, qu'il renonce à lui-même, qu'il se charge chaque jour de sa croix, et qu'il me suive.
\VS{24}Car celui qui voudra sauver sa vie, la perdra ; mais celui qui perdra sa vie à cause de son amour pour moi, la sauvera.
\VS{25}Et que servirait-il à un homme de gagner tout le monde, s'il se détruisait ou se perdait lui-même ?
\VS{26}Car quiconque aura honte de moi et de mes paroles, le Fils de l'homme aura honte de lui quand il viendra dans sa gloire, et dans celle du Père et des saints anges.
\TextTitle{La transfiguration\FTNTT{Mt. 17:1-8 ; Mc. 9:1-8}}
\VS{27}Je vous le dis, en vérité, quelques-uns de ceux qui sont ici présents, ne mourront point qu'ils n'aient vu le Royaume de Dieu\FTNT{Voir commentaire en  Mt. 16:28.}.
\VS{28}Or il arriva environ huit jours après ces paroles, qu'il prit avec lui Pierre, et Jean, et Jacques, et qu'il monta sur une montagne pour prier.
\VS{29}Et comme il priait, l'aspect de son visage changea, et son vêtement devint blanc et resplendissant comme un éclair.
\VS{30}Et voici, deux hommes savoir Moïse et Elie, parlaient avec lui,
\VS{31}et ils apparurent environnés de gloire, et ils parlaient de sa mort\FTNT{Départ : Du grec « exodos », ce qui signifie « départ », « mort », « sortie », « hors de ». Jésus est le prophète de l'Exode dont Moïse a parlé dans De. 18:15.} qu'il allait accomplir à Jérusalem.
\VS{32}Or Pierre et ceux qui étaient avec lui étaient accablés de sommeil ; et quand ils furent réveillés, ils virent sa gloire, et les deux hommes qui étaient avec lui.
\VS{33}Et il arriva qu'au moment où ces hommes se séparaient de Jésus, Pierre dit : Maître, il est bon que nous soyons ici, dressons trois tentes, une pour toi, une pour Moïse, et une pour Elie. Il ne savait pas ce qu'il disait.
\VS{34}Et comme il parlait ainsi, une nuée vint les couvrir de son ombre ; et les disciples furent saisis de frayeur en les voyant entrer dans la nuée.
\VS{35}Et une voix vint de la nuée, disant : Celui-ci est mon Fils bien-aimé ; écoutez-le.
\VS{36}Quand la voix se fit entendre, Jésus se trouva seul. Les disciples gardèrent le silence, et ils ne rapportèrent rien à personne en ce temps-là de ce qu'ils avaient vu.
\TextTitle{Les disciples de Jésus montrent leur limite}
\VS{37}Or il arriva le jour suivant, lorsqu'ils furent descendus de la montagne, une grande foule vint à sa rencontre.
\VS{38}Et voici, du milieu de la foule un homme s'écria : Maître, je t'en prie, porte les regards sur mon fils, car c'est mon fils unique.
\VS{39}Et voici un esprit le saisit, et aussitôt le fait crier, et l'agite avec violence en le faisant écumer, et c'est à peine s'il se retire de lui après l'avoir broyé.
\VS{40}J'ai prié tes disciples de le chasser, mais ils n'ont pas pu.
\VS{41}Jésus répondit : Ô génération incrédule et perverse, jusqu'à quand serai-je avec vous, et vous supporterai-je ? Amène ici ton fils.
\VS{42}Comme il approchait, le démon l'agita violemment comme s'il voulait le déchirer ; mais Jésus menaça fortement l'esprit impur, guérit l'enfant, et le rendit à son père.
\VS{43}Et tous furent étonnés de la puissance magnifique de Dieu. Et comme ils étaient tous dans l'admiration de tout ce que Jésus faisait, il dit à ses disciples :
\TextTitle{Jésus annonce de nouveau sa mort et sa résurrection\FTNTT{Mt. 17:22-23 ; Mc. 9:30-32}}
\VS{44}Vous, écoutez bien ces discours : Le Fils de l'homme sera livré entre les mains des hommes.
\VS{45}Mais les disciples ne comprirent pas cette parole, elle était voilée pour eux, afin qu'ils n'en aient pas le sens ; et ils craignaient de l'interroger à ce sujet.
\TextTitle{L'humilité, le secret de la véritable grandeur\FTNTT{Mt. 18:1-6 ; Mc. 9:33-37}}
\VS{46}Or, une pensée leur vint à l'esprit, savoir lequel d'entre eux était le plus grand.
\VS{47}Mais Jésus voyant la pensée de leur cœur, prit un petit enfant et le mit auprès de lui.
\VS{48}Puis il leur dit : Quiconque reçoit ce petit enfant en mon Nom, me reçoit ; et quiconque me reçoit, reçoit celui qui m'a envoyé. Car celui qui est le plus petit d'entre vous tous, c'est celui-là qui est grand.
\TextTitle{Jésus condamne l'esprit sectaire de Jacques et Jean\FTNTT{Mc. 9:38-40}}
\VS{49}Et Jean prit la parole et dit : Maître, nous avons vu quelqu'un qui chassait les démons en ton Nom, et nous l'en avons empêché parce qu'il ne nous suit pas.
\VS{50}Mais Jésus lui dit : Ne l'en empêchez pas ; car celui qui n'est pas contre nous, est pour nous.
\TextTitle{Mission de Jésus : sauver les âmes}
\VS{51}Lorsque le temps où il devait être enlevé du monde approcha, Jésus prit la résolution d'aller à Jérusalem.
\VS{52}Il envoya devant lui des messagers, qui se mirent en route, et entrèrent dans un bourg des Samaritains, pour lui préparer un logement.
\VS{53}Mais les Samaritains ne le reçurent pas, parce qu'il se dirigeait sur Jérusalem.
\VS{54}Et quand Jacques et Jean, ses disciples virent cela, ils dirent : Seigneur ! Veux-tu que nous commandions que le feu descende du ciel, et les consume, comme fit Elie ?
\VS{55}Mais Jésus se tourna vers eux et les réprimanda fortement, en leur disant : Vous ne savez pas de quel esprit vous êtes animés.
\VS{56}Car le Fils de l'homme n'est pas venu pour perdre les âmes des hommes, mais pour les sauver. Ainsi, ils allèrent dans un autre bourg.
\TextTitle{Epreuves de l'engagement du disciple pour suivre Jésus\FTNTT{Mt. 8:19-22}}
\VS{57}Pendant qu'ils étaient en chemin, un homme lui dit : Seigneur, je te suivrai partout où tu iras.
\VS{58}Mais Jésus lui répondit : Les renards ont des tanières, et les oiseaux du ciel ont des nids, mais le Fils de l'homme n'a pas où reposer sa tête.
\VS{59}Puis il dit à un autre : Suis-moi. Et il répondit : Permets-moi d'aller d'abord ensevelir mon père.
\VS{60}Mais Jésus lui dit : Laisse les morts ensevelir leurs morts ; mais toi, va, et annonce le Royaume de Dieu.
\VS{61}Un autre aussi lui dit : Seigneur, je te suivrai ; mais permets-moi de prendre d'abord congé de ceux de ma maison.
\VS{62}Mais Jésus lui répondit : Quiconque met la main à la charrue, et regarde en arrière, n'est pas bien disposé pour le Royaume de Dieu.
\Chap{10}
\TextTitle{Soixante-dix disciples envoyés en mission}
\VerseOne{}Or après ces choses, le Seigneur désigna soixante-dix autres disciples, et il les envoya deux à deux devant lui, dans toutes les villes et dans tous les lieux où il devait aller.
\VS{2}Il leur dit : La moisson est grande, mais il y a peu d'ouvriers ; priez donc le seigneur de la moisson qu'il pousse des ouvriers dans sa moisson.
\VS{3}Allez, voici, je vous envoie comme des agneaux au milieu des loups.
\VS{4}Ne portez ni bourse, ni sac, ni souliers, et ne saluez personne en chemin.
\VS{5}En quelque maison que vous entriez, dites premièrement : Que la paix soit sur cette maison !
\VS{6}Et s'il y a là quelqu'un qui soit digne de paix, votre paix reposera sur lui ; sinon elle retournera à vous.
\VS{7}Et demeurez dans cette maison, mangeant et buvant de ce qui sera mis devant vous ; car l'ouvrier mérite son salaire. N'allez pas de maison en maison.
\VS{8}Dans quelque ville que vous entriez, et où l'on vous recevra, mangez ce qui sera mis devant vous,
\VS{9}guérissez les malades qui s'y trouveront, et dites-leur : Le Royaume de Dieu s'est approché de vous.
\VS{10}Mais dans quelque ville que vous entriez, et où l'on ne vous recevra pas, sortez dans ses rues et dites :
\VS{11}Nous secouons contre vous-mêmes la poussière de votre ville qui s'est attachée à nous ; toutefois sachez que le Royaume de Dieu s'est approché de vous.
\VS{12}Je vous le dis qu'en ce jour Sodome sera traitée moins rigoureusement que cette ville-là.
\TextTitle{Jésus dénonce les indifférents\FTNTT{Mt. 11:20-24}}
\VS{13}Malheur à toi Chorazin, malheur à toi Bethsaïda ! Car si les miracles qui ont été faits au milieu de vous avaient été faits dans Tyr et dans Sidon, il y a longtemps qu'elles se seraient repenties, couvertes d'un sac, et assises sur la cendre.
\VS{14}C'est pourquoi Tyr et Sidon seront traitées moins rigoureusement que vous au jour du jugement.
\VS{15}Et toi, Capernaüm, qui as été élevée jusqu'au ciel, tu seras précipitée jusque dans le Hadès\FTNT{Voir commentaire en Mt. 16:18}.
\VS{16}Celui qui vous écoute, m'écoute ; et celui qui vous rejette, me rejette. Or celui qui me rejette, rejette celui qui m'a envoyé.
\VS{17}Or les soixante-dix revinrent avec joie, disant : Seigneur, les démons mêmes nous sont soumis en ton Nom.
\VS{18}Jésus leur dit : Je voyais Satan tomber du ciel comme un éclair.
\VS{19}Voici, je vous ai donné le pouvoir de marcher sur les serpents et sur les scorpions, et sur toute la force de l'ennemi ; et rien ne pourra vous nuire.
\VS{20}Toutefois, ne vous réjouissez pas de ce que les esprits vous sont soumis, mais réjouissez-vous plutôt de ce que vos noms sont écrits dans les cieux.
\VS{21}En ce moment même, Jésus se réjouit en esprit, et dit : Je te loue, ô Père ! Seigneur du ciel et de la terre, de ce que tu as caché ces choses aux sages et aux intelligents, et que tu les as révélées aux petits enfants. Oui, Père, parce que telle a été ta bonne volonté.
\VS{22}Toutes choses m'ont été données en main par mon Père ; et personne ne connaît qui est le Fils, si ce n'est le Père ; ni qui est le Père, si ce n'est le Fils et celui à qui le Fils veut le révéler.
\VS{23}Puis, se tournant vers ses disciples, il leur dit en particulier : Heureux sont les yeux qui voient ce que vous voyez.
\VS{24}Car je vous dis que beaucoup de prophètes et de rois ont désiré voir ce que vous voyez, et ne l'ont pas vu, et entendre ce que vous entendez, et ne l'ont pas entendu.
\TextTitle{Un docteur de la loi tente d'éprouver Jésus\FTNTT{cp. Mt. 22:34-40 ; Mc. 12:28-34}}
\VS{25}Alors voici, un docteur de la loi s'étant levé pour l'éprouver lui dit : Maître, que dois-je faire pour avoir la vie éternelle ?
\VS{26}Et il lui dit : Qu'est-il écrit dans la loi ? Qu'y lis-tu ?
\VS{27}Il répondit : Tu aimeras le Seigneur ton Dieu de tout ton cœur, de toute ton âme, de toute ta force, et de toute ta pensée ; et ton prochain comme toi-même.
\VS{28}Jésus lui dit : Tu as bien répondu. Fais cela, et tu vivras.
\VS{29}Mais lui, voulant se justifier, dit à Jésus : Et qui est mon prochain ?
\TextTitle{Parabole du Samaritain}
\VS{30}Jésus reprit la parole et dit : Un homme descendait de Jérusalem à Jéricho. Il tomba entre les mains des brigands, qui le dépouillèrent, le chargèrent de plusieurs coups, et s'en allèrent, le laissant à demi mort.
\VS{31}Un sacrificateur, qui par hasard descendait par le même chemin, ayant vu cet homme, passa outre.
\VS{32}Un Lévite, qui arriva aussi dans ce lieu, l'ayant vu, passa outre.
\VS{33}Mais un Samaritain, qui voyageait, étant venu là, fut ému de compassion lorsqu'il le vit.
\VS{34}Il s'approcha, banda ses plaies, en y versant de l'huile et du vin ; puis le mit sur sa propre monture, et le conduisit à une hôtellerie, et prit soin de lui.
\VS{35}Et le lendemain, en partant il tira de sa bourse deux deniers, et les donna à l'hôte, en lui disant : Aie soin de lui ; et tout ce que tu dépenseras de plus, je te le rendrai à mon retour.
\VS{36}Lequel donc de ces trois te semble-t-il avoir été le prochain de celui qui était tombé entre les mains des brigands ?
\VS{37}Il répondit : C'est celui qui a usé de miséricorde envers lui. Jésus donc lui dit : Va, et toi aussi fais de même.
\TextTitle{Marthe et Marie}
\VS{38}Et il arriva comme ils s'en allaient, qu'il entra dans une bourgade ; et une femme nommée Marthe le reçut dans sa maison.
\VS{39}Elle avait une sœur nommée Marie, qui se tenant assise aux pieds de Jésus, écoutait sa parole.
\VS{40}Mais Marthe était distraite par divers soins domestiques ; et étant venue à Jésus, elle dit : Seigneur, ne te soucies-tu point que ma sœur me laisse servir toute seule, dis-lui donc de m'aider de son côté.
\VS{41}Jésus lui répondit : Marthe, Marthe, tu t'inquiètes et tu t'agites pour beaucoup de choses.
\VS{42}Mais une chose est nécessaire ; et Marie a choisi la bonne part, qui ne lui sera point ôtée.
\Chap{11}
\TextTitle{Enseignement de Jésus sur la prière\FTNTT{cp. Mt. 6:9-15}}
\VerseOne{}Et il arriva, comme il était en prière en un certain lieu, qu'après qu'il eut cessé de prier, un de ses disciples lui dit : Seigneur, enseigne-nous à prier, comme Jean l'a enseigné à ses disciples.
\VS{2}Il leur dit : Quand vous prierez, dites : Notre Père qui es aux cieux ! Que ton Nom soit sanctifié, que ton règne vienne ; que ta volonté soit faite sur la terre comme au ciel.
\VS{3}Donne-nous chaque jour notre pain quotidien.
\VS{4}Et pardonne-nous nos péchés ; car nous aussi, nous remettons les dettes à tous ceux qui nous doivent ; et ne nous induis point en tentation, mais délivre-nous du mal.
\TextTitle{Parabole des trois amis et de la prière importune}
\VS{5}Puis il leur dit : Si l'un de vous a un ami, et qu'il aille le trouver à minuit pour lui dire : Mon ami, prête-moi trois pains,
\VS{6}car un de mes amis est arrivé de voyage chez moi, et je n'ai rien à lui offrir,
\VS{7}et si, de l'intérieur de sa maison, cet ami lui répond : Ne m'importune pas ; car ma porte est déjà fermée, mes enfants et moi nous sommes au lit ; je ne puis me lever pour t'en donner.
\VS{8}Je vous le dis, même s'il ne se levait pas pour les lui donner parce que c'est son ami, il se lèverait à cause de son importunité, et lui donnerait tout ce dont il a besoin.
\VS{9}Ainsi je vous dis : Demandez, et il vous sera donné ; cherchez, et vous trouverez ; frappez, et l'on vous ouvrira.
\VS{10}Car quiconque demande, reçoit ; et celui qui cherche, trouve ; et l'on ouvre à celui qui frappe.
\TextTitle{Parabole du père}
\VS{11}Quel est parmi vous le père qui donnera une pierre à son fils, s'il lui demande du pain ? Ou, s'il lui demande un poisson, lui donnera-t-il un serpent au lieu d'un poisson ?
\VS{12}Ou, s'il demande un œuf, lui donnera-t-il un scorpion ?
\VS{13}Si donc vous qui êtes méchants, vous savez donner à vos enfants des choses bonnes, combien plus le Père qui est du ciel donnera-t-il l'Esprit Saint à ceux qui le lui demandent ,
\TextTitle{Jésus guérit un démoniaque}
\VS{14}Alors il chassa un démon qui était muet. Lorsque le démon fut sorti, le muet parla ; et la foule fut dans l'admiration.
\TextTitle{Le blasphème contre le Saint-Esprit\FTNTT{Mt. 12:24-32 ; Mc. 3:22-30}}
\VS{15}Mais quelques-uns d'entre eux dirent : C'est par Béelzebul\FTNT{Béelzebul : Dans 2 R. 1:2, il est fait mention de « Baal Zebub, dieu d'Eqrôn ». Littéralement, la formule signifie « maître (Baal) des mouches ». Ce mot a une autre signification que le grec de la Septante a adoptée en traduisant par Baal-myia, « Baal-mouche ». Le prince des démons.}, prince des démons, qu'il chasse les démons.
\VS{16}Mais les autres pour l'éprouver, lui demandaient un miracle venant du ciel.
\VS{17}Mais lui, connaissant leurs pensées, leur dit : Tout royaume divisé contre lui-même sera réduit en désert ; et toute maison divisée contre elle-même tombe en ruine.
\VS{18}Si donc Satan est divisé contre lui-même, comment son royaume subsistera-t-il ? Car vous dites que je chasse les démons par Béelzebul.
\VS{19}Et si moi, je chasse les démons par Béelzebul, vos fils par qui les chassent-ils ? C'est pourquoi ils seront eux-mêmes vos juges.
\VS{20}Mais si je chasse les démons par le doigt de Dieu, alors le royaume de Dieu est parvenu jusqu'à vous.
\VS{21}Lorsqu'un homme fort et bien armé garde sa bergerie\FTNT{Bergerie : Chez les Grecs, du temps d'Homère, c'était un espace découvert autour de la maison, fermé par un mur, tandis que chez les Orientaux, il s'agissait d'un espace dans la campagne, entouré d'un mur, où les troupeaux passaient la nuit. La bergerie désigne aussi la partie non couverte d'une maison. Dans la première alliance, il s'agit particulièrement du « parvis » du tabernacle et du temple à Jérusalem. Les demeures des gens de la haute société possédaient généralement deux de ces « cours » : une entre la porte et la rue, l'autre entourée par l'immeuble lui-même. C'est ce qui est mentionné en Mt. 26:69. Enfin, ce terme fait allusion à la maison elle-même, un palais.}, les biens qu'il a sont en sûreté.
\VS{22}Mais si un plus fort que lui survient et le vainque, il lui enlève toutes ses armes dans lesquelles il se confiait, et il partage ses dépouilles.
\VS{23}Celui qui n'est point avec moi est contre moi ; et celui qui n'assemble pas avec moi, il disperse.
\TextTitle{Le retour de l'esprit impur\FTNTT{Mt. 12:43-45}}
\VS{24}Quand l'esprit impur est sorti d'un homme, il va par des lieux secs, cherchant du repos. N'en trouvant point, il dit : Je retournerai dans ma maison, d'où je suis sorti,
\VS{25}et quand il arrive, il la trouve balayée et parée.
\VS{26}Alors il s'en va, et prend avec lui sept autres esprits plus méchants que lui, et ils entrent et demeurent là ; de sorte que la dernière condition de cet homme-là est pire que la première.
\VS{27}Or il arriva comme il disait ces choses, qu'une femme élevant sa voix du milieu de la foule, lui dit : Heureux est le ventre qui t'a porté, et les mamelles que tu as tétées !
\VS{28}Et il répondit : Heureux plutôt ceux qui écoutent la parole de Dieu, et qui la gardent !
\TextTitle{Le signe du prophète Jonas\FTNTT{Mt. 12:38-41}}
\VS{29}Et comme les foules s'amassaient ensemble, il se mit à dire : Cette génération est méchante ; elle demande un miracle, mais il ne lui sera donné d'autre miracle que celui de Jonas le prophète.
\VS{30}Car, de même que Jonas fut un miracle pour les Ninivites, de même le Fils de l'homme en sera un pour cette génération.
\VS{31}La reine du Midi se lèvera au jour du jugement contre les hommes de cette génération et les condamnera, parce qu'elle vint des extrémités de la terre pour entendre la sagesse de Salomon ; et voici, il y a ici plus que Salomon.
\VS{32}Les gens de Ninive se lèveront au jour du jugement contre cette génération et la condamneront, parce qu'ils se sont repentis à la prédication de Jonas ; et voici, il y a ici plus que Jonas.
\TextTitle{Parabole de la lampe\FTNTT{Mt. 5:14-16 ; Mc. 4:21-23 ; cp. Lu. 8:16-18}}
\VS{33}Or personne n'allume une lampe pour la mettre dans un lieu caché ou sous le boisseau, mais sur un chandelier, afin que ceux qui entrent voient la lumière.
\VS{34}La lumière du corps c'est l'œil. Si donc ton œil est sain, tout ton corps aussi sera éclairé ; mais s'il est mauvais, ton corps aussi sera ténébreux.
\VS{35}Prends donc garde que la lumière qui est en toi ne soit pas ténèbres.
\VS{36}Si donc ton corps est éclairé, n'ayant aucune partie dans les ténèbres, il sera entièrement éclairé, comme lorsque la lampe t'éclaire de sa lumière.
\VS{37}Comme il parlait, un pharisien le pria de dîner chez lui. Il entra, et se mit à table.
\VS{38}Mais le pharisien vit avec étonnement qu'il ne s'était pas premièrement lavé avant le dîner.
\TextTitle{Malheurs sur les pharisiens et les docteurs de la loi\FTNTT{cp. Mt. 12:38-41}}
\VS{39}Mais le Seigneur lui dit : Vous autres pharisiens, vous nettoyez le dehors de la coupe et du plat ; et à l'intérieur vous êtes pleins de rapine et de méchanceté.
\VS{40}Insensés, celui qui a fait le dehors, n'a-t-il pas fait aussi le dedans ?
\VS{41}Donnez plutôt en aumône ce qui est dedans, et voici, toutes choses seront pures pour vous.
\VS{42}Mais malheur à vous, pharisiens ! Car vous payez la dîme de la menthe, de la rue\FTNT{La rue : Il s'agit d'un arbuste ayant des propriétés médicinales. Les pharisiens poussaient leur zèle jusqu'à payer la dîme sur certaines herbes. Toutefois, en négligeant la justice et l'amour de Dieu, ils passaient à côté de l'essentiel. Toutes leurs œuvres étaient par conséquent vaines.}, et de toutes sortes d'herbes, et vous négligez la justice et l'amour de Dieu. C'est là ce qu'il fallait pratiquer, sans négliger les autres choses.
\VS{43}Malheur à vous, pharisiens, qui aimez les premières places dans les synagogues, et les salutations sur les places publiques.
\VS{44}Malheur à vous, scribes et pharisiens hypocrites, car vous êtes comme les sépulcres qui ne paraissent pas, et sur lesquels on marche sans le voir.
\VS{45}Alors un des docteurs de la loi prit la parole, et lui dit : Maître, en disant ces choses, tu nous outrages aussi.
\VS{46}Et il dit : À vous aussi, malheur, docteurs de la loi ! Car vous chargez les hommes de fardeaux difficiles à porter, et vous-mêmes vous ne touchez pas ces fardeaux d'un seul de vos doigts.
\VS{47}Malheur à vous, car vous bâtissez les sépulcres des prophètes, que vos pères ont tués.
\VS{48}Vous rendez donc témoignage aux œuvres de vos pères et vous y prenez plaisir ; car eux, ils les ont tués, et vous, vous bâtissez leurs sépulcres.
\VS{49}C'est pourquoi aussi la sagesse de Dieu a dit : Je leur enverrai des prophètes et des apôtres, et ils tueront les uns, et persécuteront les autres,
\VS{50}afin que le sang de tous les prophètes qui a été répandu dès la fondation du monde, soit redemandé à cette nation.
\VS{51}Depuis le sang d'Abel, jusqu'au sang de Zacharie, qui fut tué entre l'autel et le temple. Oui, je vous le dis qu'il sera redemandé à cette nation.
\VS{52}Malheur à vous, docteurs de la loi ! Parce que vous avez enlevé la clef de la science. Vous n'êtes pas entrés vous-mêmes, et vous avez empêché ceux qui entraient.
\VS{53}Et comme il leur disait ces choses, les scribes et les pharisiens commencèrent à le presser violemment, et à le faire parler sur beaucoup de choses,
\VS{54}lui dressant des pièges, et cherchant à tirer quelque chose de sa bouche, afin de l'accuser.
\TextTitle{[Enseignements divers de Jésus]
\\(cp. Mt. 16:6-12 ; Mc. 8:14-21}
\Chap{12}
\VerseOne{}Cependant les gens s'étaient rassemblés par milliers, au point de s'écraser les uns les autres. Jésus se mit à dire à ses disciples : Avant tout, gardez-vous surtout du levain des pharisiens qui est l'hypocrisie.
\VS{2}Car il n'y a rien de caché, qui ne doive être révélé, ni de secret, qui ne doive être connu.
\VS{3}C'est pourquoi tout ce que vous aurez dit dans les ténèbres, sera entendu dans la lumière ; et ce que vous aurez dit à l'oreille dans les chambres, sera prêché sur les toits.
\VS{4}Je vous dis à vous qui êtes mes amis : Ne craignez pas ceux qui tuent le corps, et qui après cela ne peuvent rien faire de plus.
\VS{5}Je vous montrerai qui vous devez craindre. Craignez celui qui, après avoir tué, a le pouvoir de jeter dans la géhenne ; oui, vous dis-je, craignez celui-là.
\VS{6}Ne vend-on pas cinq petits passereaux pour deux sous ? Cependant, aucun d'eux n'est oublié devant Dieu.
\VS{7}Et même les cheveux de votre tête sont tous comptés. Ne craignez donc point ; vous valez plus que beaucoup de passereaux.
\VS{8}Or, je vous dis, quiconque me confessera devant les hommes, le Fils de l'homme le confessera aussi devant les anges de Dieu.
\VS{9}Mais quiconque me reniera devant les hommes, il sera renié devant les anges de Dieu.
\VS{10}Et quiconque parlera contre le Fils de l'homme, il lui sera pardonné ; mais celui qui aura blasphémé contre le Saint-Esprit\FTNT{Voir commentaire en Mt. 12:32.}, il ne lui sera point pardonné.
\VS{11}Quand ils vous mèneront devant les synagogues, les magistrats et les autorités, ne vous inquiétez pas de la manière dont vous vous défendrez ni de ce que vous aurez à dire.
\VS{12}Car le Saint-Esprit vous enseignera à l'heure même ce qu'il faudra dire.
\TextTitle{Parabole du riche insensé}
\VS{13}Et quelqu'un de la foule lui dit : Maître, dis à mon frère qu'il partage avec moi notre héritage.
\VS{14}Mais il lui répondit : Ô homme ! Qui m'a établi sur vous pour être votre juge, et pour faire vos partages ?
\VS{15}Puis il leur dit : Gardez-vous avec soin de toute avarice ; car quoique les biens de quelqu'un abondent, il n'a pas la vie par ses biens.
\VS{16}Et il leur dit cette parabole : Les champs d'un homme riche avaient beaucoup rapporté.
\VS{17}Et il raisonnait en lui-même, disant : Que ferai-je, car je n'ai pas assez de place pour recueillir mes fruits ?
\VS{18}Puis il dit : Voici ce que je ferai : J'abattrai mes greniers et j'en bâtirai de plus grands, et j'y amasserai toute ma récolte et tous mes biens.
\VS{19}Puis je dirai à mon âme : Mon âme, tu as beaucoup de biens assemblés pour beaucoup d'années, repose-toi, mange, bois, et réjouis-toi.
\VS{20}Mais Dieu lui dit : Insensé ! Cette même nuit ton âme te sera redemandée ; et ces choses que tu as préparées, à qui seront-elles ?
\VS{21}Il en est ainsi de celui qui amasse des biens pour lui-même, et qui n'est pas riche en Dieu.
\TextTitle{Exhortation à se confier en Dieu}
\VS{22}Jésus dit à ses disciples : C'est pourquoi je vous dis : Ne vous inquiétez pas pour votre vie, de ce que vous mangerez, ni pour votre corps, de quoi vous serez vêtus.
\VS{23}La vie est plus que la nourriture, et le corps est plus que le vêtement.
\VS{24}Considérez les corbeaux, ils ne sèment, ni ne moissonnent, et ils n'ont point de cellier, ni de grenier, et cependant Dieu les nourrit. Combien ne valez-vous pas plus que les oiseaux ?
\VS{25}Qui de vous qui par ses inquiétudes peut ajouter une coudée à la durée de sa vie ?
\VS{26}Si donc vous ne pouvez pas même la moindre chose, pourquoi êtes-vous inquiets du reste ?
\VS{27}Considérez comment croissent les lis, ils ne travaillent, ni ne filent, et cependant je vous dis que Salomon même, dans toute sa gloire, n'a pas été vêtu comme l'un d'eux.
\VS{28}Si Dieu revêt ainsi l'herbe qui est aujourd'hui au champ, et qui demain sera jetée au four, à combien plus forte raison vous vêtira-t-il, ô gens de petite foi ?
\VS{29}Ne dites donc point : Que mangerons-nous, ou que boirons-nous ? Et ne soyez pas inquiets,
\VS{30}car toutes ces choses, ce sont les païens du monde qui les recherchent. Votre Père sait que vous en avez besoin.
\VS{31}Mais cherchez plutôt le Royaume de Dieu, et toutes ces choses vous seront données par-dessus.
\VS{32}Ne crains point petit troupeau, car il a plu à votre Père de vous donner le Royaume.
\VS{33}Vendez ce que vous avez, et donnez-le en aumône. Faites-vous des bourses qui ne s'usent point, un trésor dans les cieux qui ne défaille jamais, et où le voleur n'approche point, et où la teigne ne gâte rien.
\VS{34}Car là où est votre trésor, là sera aussi votre cœur.
\TextTitle{Importance de veiller en attendant le Maître\FTNTT{Mt. 24:36-25:30}}
\VS{35}Que vos reins soient ceints, et vos lampes allumées.
\VS{36}Et soyez semblables aux serviteurs qui attendent que leur maître revienne des noces, afin de lui ouvrir dès qu'il frappera.
\VS{37}Heureux ces serviteurs que le maître à son arrivée, trouvera veillant ! En vérité je vous le dis, il se ceindra, les fera mettre à table, et s'approchera pour les servir.
\VS{38}Qu'il arrive à la seconde veille ou à la troisième veille, heureux ces serviteurs, s'il les trouve veillant !
\VS{39}Or sachez ceci, si le père de famille savait à quelle heure le voleur doit venir, il veillerait, et ne laisserait pas percer sa maison.
\VS{40}Vous donc aussi tenez-vous prêts, car le Fils de l'homme viendra à l'heure où vous n'y penserez pas.
\TextTitle{Parabole des deux serviteurs}
\VS{41}Pierre lui dit : Seigneur, dis-tu cette parabole pour nous, ou aussi pour tous ?
\VS{42}Et le Seigneur dit : Quel est donc l'économe fidèle et prudent, que le maître établira sur les domestiques de sa maison pour leur donner la nourriture au temps convenable ?
\VS{43}Heureux ce serviteur, que son maître, à son arrivée, trouvera faisant ainsi.
\VS{44}Je vous le dis en vérité, il l'établira sur tous ses biens.
\VS{45}Mais si ce serviteur dit en son cœur : Mon maître tarde longtemps à venir, s'il se met à battre les serviteurs et les servantes, à manger, à boire et à s'enivrer,
\VS{46}le maître de cet esclave-là viendra en un jour qu'il n'attend pas, et à une heure qu'il ne sait pas, et il le coupera en deux\FTNT{Le mot grec « dichotomeo » signifie « couper en deux parts », « de couper quelqu'un en deux », « châtiant en coupant », « fléau sévère ». Certains peuples, dont les Hébreux, employaient cette méthode cruelle comme châtiment corporel.}, et lui donnera sa part avec les infidèles.
\VS{47}Or le serviteur qui a connu la volonté de son maître, et qui ne s'est pas tenu prêt, et n'a point fait selon sa volonté, sera battu de plusieurs coups.
\VS{48} Mais celui qui ne l'a point connue, et qui a fait des choses dignes de châtiment, sera battu de peu de coups. Et il sera beaucoup redemandé à quiconque il aura été beaucoup donné; et on exigera plus de celui à qui on aura beaucoup confié.
\TextTitle{Jésus objet de divisions}
\VS{49}Je suis venu jeter un feu sur la terre, et qu'ai-je à désirer, s'il est déjà allumé ?
\VS{50}Il est un baptême dont je dois être baptisé, et combien suis-je pressé jusqu'à ce qu'il soit accompli.
\VS{51}Pensez-vous que je sois venu apporter la paix sur la terre ? Non, vous dis-je ; mais plutôt la division.
\VS{52}Car désormais cinq dans une maison, seront divisés, trois contre deux, et deux contre trois.
\VS{53}Le père sera divisé contre le fils, et le fils contre le père ; la mère contre la fille, et la fille contre la mère ; la belle-mère contre sa belle-fille, et la belle-fille contre sa belle-mère.
\VS{54}Puis il dit encore aux foules : Quand vous voyez un nuage se lever à l'occident, vous dites aussitôt : La pluie vient, et cela arrive ainsi.
\VS{55}Et quand vous voyez souffler le vent du midi, vous dites qu'il fera chaud ; et cela arrive.
\VS{56}Hypocrites, vous savez bien discerner l'aspect du ciel et de la terre ; et comment ne discernez-vous point cette saison ?
\VS{57}Et pourquoi aussi ne reconnaissez-vous pas de vous-mêmes ce qui est juste ?
\VS{58}Or quand tu vas avec ton adversaire devant le magistrat, tâche en chemin de t'en délivrer, de peur qu'il ne te traîne devant le juge, et que le juge ne te livre à l'officier de justice et que celui-ci ne te mette en prison.
\VS{59}Je te le dis, tu ne sortiras pas de là que tu n'aies payé jusqu'au dernier pite\FTNT{Petite pièce de monnaie en laiton. Voir en annexe le « tableau des monnaies au temps de Jésus-Christ ».}.
\Chap{13}
\TextTitle{Exhortation à la repentance}
\VerseOne{}En ce même temps, quelques-uns qui se trouvaient là présents racontèrent à Jésus ce qui était arrivé à des Galiléens, dont Pilate avait mêlé le sang avec celui de leurs sacrifices.
\VS{2}Et Jésus répondant leur dit : Croyez-vous que ces Galiléens étaient de plus grands pécheurs que tous les autres Galiléens, parce qu'ils ont souffert de la sorte ?
\VS{3}Non, vous dis-je ; mais si vous ne vous repentez pas, vous périrez tous de la même manière.
\VS{4}Ou bien, ces dix-huit personnes sur qui est tombée la tour de Siloé et qu'elle a tuées, croyez-vous qu'elles étaient plus coupables que tous les habitants de Jérusalem ?
\VS{5}Non, vous dis-je ; mais si vous ne vous repentez pas, vous périrez tous de la même manière.
\TextTitle{Parabole du figuier stérile et le jugement différé d'Israël\FTNTT{cp. Mt. 21:18-21}}
\VS{6}Il disait aussi cette parabole : Un homme avait un figuier planté dans sa vigne. Il vint pour y chercher du fruit, mais il n'en trouva point.
\VS{7}Et il dit au vigneron : Voilà trois ans que je viens chercher du fruit à ce figuier, et je n'en trouve point. Coupe-le ; pourquoi occupe-t-il inutilement la terre ?
\VS{8}Et le vigneron lui répondit : Seigneur, laisse-le encore pour cette année, je creuserai tout autour, et j'y mettrai du fumier.
\VS{9}Peut-être portera-t-il du fruit ; sinon, tu le couperas après cela.
\TextTitle{Guérison de la femme courbée le jour du sabbat}
\VS{10}Or comme il enseignait dans une de leurs synagogues un jour de sabbat,
\VS{11}voici, il y avait là une femme qui était possédée d'un démon qui la rendait infirme depuis dix-huit ans, et elle était courbée, et ne pouvait nullement se redresser.
\VS{12}Et quand Jésus la vit, il l'appela, et lui dit : Femme, tu es délivrée de ton infirmité.
\VS{13}Et il lui imposa les mains ; et à l'instant elle se redressa, et glorifia Dieu.
\VS{14}Mais le chef de la synagogue, indigné de ce que Jésus avait opéré cette guérison un jour du sabbat, prenant la parole dit à l'assemblée : Il y a six jours pour travailler ; venez donc vous faire guérir ces jours-là, et non pas le jour du sabbat.
\VS{15}Hypocrites ! lui répondit le Seigneur, chacun de vous ne détache-t-il pas son bœuf ou son âne de la crèche le jour du sabbat, et ne les mène-t-il pas boire ?
\VS{16}Et ne fallait-il pas délier de ce lien le jour du sabbat cette femme qui est fille d'Abraham, et que Satan tenait liée depuis dix-huit ans ?
\VS{17}Comme il disait ces choses, tous ses adversaires étaient confus ; mais toutes les foules se réjouissaient de toutes les choses glorieuses qu'il opérait.
\TextTitle{Parabole du grain de moutarde et du levain\FTNTT{voir Mt. 13:31,33}}
\VS{18}Il disait aussi : A quoi est semblable le Royaume de Dieu, et à quoi le comparerai-je ?
\VS{19}Il est semblable au grain de semence de moutarde qu'un homme a pris et jeté dans son jardin ; il pousse, devient un grand arbre, et les oiseaux du ciel font leurs nids dans ses branches.
\VS{20}Il dit encore : A quoi comparerai-je le Royaume de Dieu ?
\VS{21}Il est semblable au levain qu'une femme a pris et mis dans trois mesures de farine, pour faire lever toute la pâte.
\TextTitle{Enseignements de Jésus sur le chemin de Jérusalem}
\VS{22}Puis il s'en allait par les villes et les villages, enseignant, et faisant route vers Jérusalem.
\VS{23}Quelqu'un lui dit : Seigneur, n'y a-t-il que peu de gens qui soient sauvés ? Il leur répondit :
\VS{24}Efforcez-vous d'entrer par la porte étroite. Car je vous le dis que beaucoup chercheront à entrer, et ne le pourront pas.
\VS{25}Quand le père de famille se sera levé, et aura fermé la porte, et que vous, étant dehors, vous vous mettrez à frapper à la porte, en disant : Seigneur ! Seigneur ! Ouvre-nous ! Il vous répondra : Je ne sais pas d'où vous êtes.
\VS{26}Alors vous vous mettrez à dire : Nous avons mangé et bu en ta présence, et tu as enseigné dans nos rues.
\VS{27}Mais il dira : Je vous le dis, je ne sais pas d'où vous êtes. Retirez-vous de moi, vous tous qui faites le métier d'iniquité.
\VS{28}C'est là qu'il y aura des pleurs et des grincements de dents, quand vous verrez Abraham, Isaac, et Jacob, et tous les prophètes dans le Royaume de Dieu, et que vous serez jetés dehors.
\VS{29}Il en viendra aussi d'orient et d'occident, du nord et du sud, et ils se mettront à table dans le Royaume de Dieu.
\VS{30}Et voici, ceux qui sont les derniers seront les premiers, et ceux qui sont les premiers seront les derniers.
\VS{31}En ce même jour, quelques pharisiens vinrent à lui et lui dirent : Retire-toi et va-t'en d'ici, car Hérode veut te tuer.
\VS{32}Il leur répondit : Allez, et dites à ce renard : Voici, je chasse les démons et j'achève de faire des guérisons aujourd'hui et demain, et le troisième jour je prends fin.
\VS{33}C'est pourquoi il me faut marcher aujourd'hui et demain, et le jour suivant ; car il ne convient pas qu'un prophète meure hors de Jérusalem.
\TextTitle{Lamentations de Jésus sur Jérusalem\FTNTT{Mt. 23:37-39 ; Lu. 19:41-44 ; cp. Jé. 22:5}}
\VS{34}Jérusalem, Jérusalem, qui tues les prophètes et qui lapides ceux qui te sont envoyés ; combien de fois ai-je voulu rassembler tes enfants, comme la poule rassemble ses poussins sous ses ailes, et vous ne l'avez pas voulu !
\VS{30}Voici, votre maison va être déserte ; et je vous le dis en vérité, que vous ne me verrez plus, jusqu'à ce que vous disiez : Béni soit celui qui vient au nom du Seigneur.
\Chap{14}
\TextTitle{Jésus guérit un hydropique le jour du sabbat\FTNTT{cp. Mt. 12:9-13}}
\VerseOne{}Jésus entra un jour de sabbat dans la maison d'un des chefs des pharisiens pour prendre un repas, les pharisiens l'observaient.
\VS{2}Et voici, un homme hydropique était là devant lui.
\VS{3}Jésus prit la parole, et dit aux docteurs de la loi et aux pharisiens : Est-il permis, ou non, de faire une guérison le jour du sabbat ?
\VS{4}Ils gardèrent le silence. Alors Jésus prit le malade, le guérit, et le renvoya.
\VS{5}Puis s'adressant à lui, il leur dit : Lequel de vous, si son fils ou son bœuf tombe dans un puits, ne l'en retirera pas aussitôt, le jour du sabbat ?
\VS{6}Et ils ne pouvaient répliquer à ces choses.
\TextTitle{Parabole de l'invité}
\VS{7}Il proposa aussi aux conviés une parabole, en voyant qu'ils choisissaient les premières places ; et il leur dit :
\VS{8}Quand tu seras convié par quelqu'un à des noces, ne te mets pas à la première place à table, de peur qu'il ne se trouve parmi les conviés une personne plus honorable que toi,
\VS{9}et que celui qui vous a conviés l'un et l'autre ne vienne te dire : Cède ta place à cette personne-là. Tu aurais alors honte d'aller occuper la dernière place.
\VS{10}Mais lorsque tu seras convié, va te mettre à la dernière place, afin que quand celui qui t'a convié viendra, il te dise : Mon ami, monte plus haut. Alors cela te fera honneur devant tous ceux qui seront à table avec toi.
\VS{11}Car quiconque s'élève, sera abaissé ; et quiconque s'abaisse, sera élevé.
\VS{12}Il dit aussi à celui qui l'avait convié : Lorsque tu fais un dîner ou un souper, n'invite pas tes amis, ni tes frères, ni tes parents, ni tes riches voisins ; de peur qu'ils ne te convient à leur tour, et qu'on ne te rende la pareille.
\VS{13}Mais, lorsque tu donneras un festin, convie les pauvres, les impotents, les boiteux et les aveugles.
\VS{14}Et tu seras heureux de ce qu'ils n'ont pas de quoi te rendre la pareille ; car elle te sera rendue à la résurrection des justes.
\TextTitle{Parabole du grand festin\FTNTT{Mt. 22:1-14}}
\VS{15}Un de ceux qui étaient à table, ayant entendu ces paroles, lui dit : Heureux celui qui mangera du pain dans le Royaume de Dieu.
\VS{16}Et Jésus lui répondit : Un homme fit un grand festin, et il convia beaucoup de gens.
\VS{17}Et à l'heure du souper, il envoya son serviteur pour dire aux conviés : Venez, car tout est déjà prêt.
\VS{18}Mais ils commencèrent tous unanimement à s'excuser. Le premier lui dit : J'ai acheté un champ, et il me faut nécessairement partir pour aller le voir ; je te prie, excuse-moi.
\VS{19}Un autre dit : J'ai acheté cinq paires de bœufs, et je vais les essayer ; je te prie, excuse-moi.
\VS{20}Et un autre dit : J'ai épousé une femme, c'est pourquoi je ne puis aller.
\VS{21}Le serviteur, de retour, rapporta ces choses à son maître. Alors le père de famille irrité, dit à son serviteur : Va promptement dans les places et dans les rues de la ville, et amène ici les pauvres, les impotents, les boiteux et les aveugles.
\VS{22}Puis le serviteur dit : Maître, ce que tu as commandé a été fait, et il y a encore de la place.
\VS{23}Et le maître dit au serviteur : Va dans les chemins et le long des haies, et ceux que tu trouveras, contrains-les d'entrer, afin que ma maison soit remplie.
\VS{24}Car je vous dis, qu'aucun de ces hommes qui avaient été conviés ne goûtera de mon souper.
\TextTitle{Test de la consécration du disciple}
\VS{25}Or de grandes foules faisaient route avec Jésus. Il se retourna et leur dit :
\VS{26}Si quelqu'un vient à moi, et ne hait pas son père et sa mère, sa femme et ses enfants, ses frères et ses sœurs, et même sa propre vie, il ne peut être mon disciple.
\VS{27}Et quiconque ne porte pas sa croix, et ne me suit pas, ne peut être mon disciple.
\TextTitle{Parabole de la tour}
\VS{28}Car lequel de vous, s'il veut bâtir une tour, ne s'assied pas premièrement pour calculer la dépense et voir s'il a de quoi l'achever ?
\VS{29}De peur qu'après avoir posé les fondements, il ne puisse pas l'achever, et que tous ceux qui le verront ne commencent à se moquer de lui,
\VS{30}en disant : Cet homme a commencé à bâtir, et il n'a pas pu achever.
\TextTitle{Parabole du roi qui se prépare à la guerre}
\VS{31}Ou, quel roi, s'il va faire la guerre à un autre roi, ne s'assied pas premièrement pour examiner s'il peut, avec dix mille hommes, aller à la rencontre de celui qui vient contre lui avec vingt mille ?
\VS{32}Autrement, pendant que cet autre roi est encore loin, il lui envoie une ambassade pour demander la paix.
\VS{33}Ainsi donc, quiconque d'entre vous ne renonce pas à tout ce qu'il possède ne peut être mon disciple.
\TextTitle{Parabole du sel}
\VS{34}Le sel est bon ; mais si le sel perd sa saveur, avec quoi l'assaisonnera-t-on ?
\VS{35}Il n'est bon ni pour la terre, ni pour le fumier ; mais on le jette dehors. Que celui qui a des oreilles pour entendre, qu'il entende !
\Chap{15}
\TextTitle{Trois paraboles sur la repentance}
\VerseOne{}Or tous les publicains et les pécheurs s'approchaient de Jésus pour l'entendre.
\VS{2}Mais les pharisiens et les scribes murmuraient, disant : Cet homme reçoit les pécheurs, et mange avec eux.
\TextTitle{Parabole de la brebis perdue\FTNTT{Mt. 18:12-14}}
\VS{3}Mais il leur proposa cette parabole, disant :
\VS{4}Lequel d'entre vous, s'il a cent brebis, et qu'il en perd une, ne laisse pas les quatre-vingt-dix-neuf dans le désert, pour aller à la recherche de celle qui est perdue, jusqu'à ce qu'il la trouve ?
\VS{5}Et l'ayant retrouvée, il la met avec joie sur ses épaules,
\VS{6}et, de retour à la maison, il appelle ses amis et ses voisins, et il leur dit : Réjouissez-vous avec moi ; car j'ai trouvé ma brebis qui était perdue.
\VS{7}De même, je vous le dis il y aura plus de joie dans le ciel pour un seul pécheur qui se repent, que pour les quatre-vingt-dix-neuf justes qui n'ont pas besoin de repentance.
\TextTitle{Parabole de la drachme perdue}
\VS{8}Ou quelle femme, si elle a dix drachmes, et qu'elle en perde une, n'allume pas une lampe, ne balaie la maison, et ne cherche avec soin, jusqu'à ce qu'elle la trouve ?
\VS{9}Lorsqu'elle l'a trouvée, elle appelle ses amies et ses voisines, en leur disant : Réjouissez-vous avec moi ; car j'ai trouvé la drachme que j'avais perdue.
\VS{10}Ainsi je vous le dis, il y a de la joie devant les anges de Dieu pour un seul pécheur qui vient à se repentir.
\TextTitle{Parabole du fils perdu}
\VS{11}Il leur dit aussi : Un homme avait deux fils ;
\VS{12}et le plus jeune dit à son père : Mon père, donne-moi la part de bien qui m'appartient ; et il leur partagea ses biens.
\VS{13}Et peu de jours après, le plus jeune fils, ayant tout ramassé, partit pour un pays éloigné, où il dissipa son bien en vivant dans la débauche.
\VS{14}Et après qu'il eut tout dépensé, une grande famine survint dans ce pays-là, et il commença à se trouver dans la disette.
\VS{15}Alors il alla se mettre au service d'un des habitants du pays, qui l'envoya dans ses possessions pour paître les pourceaux.
\VS{16}Il aurait bien voulu se rassasier des carouges que les pourceaux mangeaient ; mais personne ne lui en donnait.
\VS{17}Or étant revenu à lui-même, il dit : Combien d'ouvriers chez mon père ont du pain en abondance, et moi je meurs de faim !
\VS{18}Je me lèverai, j'irai vers mon père, et je lui dirai : Mon père, j'ai péché contre le ciel et devant toi ;
\VS{19}et je ne suis plus digne d'être appelé ton fils ; traite-moi comme l'un de tes ouvriers.
\VS{20}Il se leva donc, et alla vers son père. Et comme il était encore loin, son père le vit et fut ému de compassion, il courut se jeter à son cou et le baisa.
\VS{21}Mais le fils lui dit : Mon père, j'ai péché contre le ciel et devant toi ; et je ne suis plus digne d'être appelé ton fils.
\VS{22}Et le père dit à ses serviteurs : Apportez la plus belle robe et revêtez-le, mettez-lui un anneau au doigt, et des souliers aux pieds.
\VS{23}Amenez-moi le veau gras, et tuez-le. Mangeons et réjouissons-nous.
\VS{24}Car mon fils que voici, était mort, mais il est ressuscité ; il était perdu, mais il est retrouvé. Et ils commencèrent à se réjouir.
\VS{25}Or son fils aîné était dans les champs. Lorsqu'il revint et approcha de la maison, il entendit la musique et les danses.
\VS{26}Il appela un des serviteurs, et il lui demanda ce que c'était.
\VS{27}Ce serviteur lui dit : Ton frère est de retour, et ton père a tué le veau gras, parce qu'il l'a recouvré sain et sauf.
\VS{28}Mais il se mit en colère, et ne voulut pas entrer. Son père sortit et le pria d'entrer.
\VS{29}Mais il répondit, et dit à son père : voici, il y a tant d'années que je te sers, et jamais je n'ai transgressé ton commandement, et cependant tu ne m'as jamais donné un chevreau pour que je me réjouisse avec mes amis.
\VS{30}Mais quand ton fils est arrivé, celui qui a mangé ton bien avec des prostituées, c'est pour lui que tu as tué le veau gras.
\VS{31}Et le père lui dit : Mon enfant, tu es toujours avec moi, et tous mes biens sont à toi.
\VS{32}Or il fallait bien s'égayer et se réjouir, parce que ton frère que voici était mort et qu'il est ressuscité, parce qu'il était perdu et qu'il est retrouvé.
\Chap{16}
\TextTitle{Parabole de l'économe infidèle}
\VerseOne{}Il disait aussi à ses disciples : Il y avait un homme riche qui avait un économe, qui fut accusé devant lui comme dissipant ses biens.
\VS{2}Il l'appela et lui dit : Qu'est-ce que j'entends dire de toi ? Rends compte de ton administration ; car tu n'auras plus le pouvoir d'administrer mes biens.
\VS{3}Alors l'économe dit en lui-même : Que ferai-je, puisque mon maître m'ôte l'administration ? Travailler à la terre ? Je ne le puis. Mendier ? J'en ai honte.
\VS{4}Je sais ce que je ferai, afin que les gens me reçoivent dans leurs maisons quand mon administration me sera ôtée.
\VS{5}Alors il appela chacun des débiteurs de son maître, et il dit au premier : Combien dois-tu à mon maître ?
\VS{6}Il dit : Cent mesures d'huile. Et il lui dit : Prends ton billet, et assied-toi vite, et écris cinquante.
\VS{7}Puis il dit à un autre : Et toi, combien dois-tu ? Il dit : Cent mesures de froment. Et il lui dit : Prends ton billet, et écris quatre-vingts.
\VS{8}Et le maître loua l'économe infidèle de ce qu'il avait agi prudemment. Ainsi les enfants de ce siècle sont plus prudents dans leur génération, que les enfants de lumière.
\VS{9}Et moi aussi je vous dis : Faites-vous des amis avec les richesses injustes ; afin que quand vous viendrez à manquer, ils vous reçoivent dans les tabernacles éternels.
\VS{10}Celui qui est fidèle en très peu de chose, est fidèle aussi dans les grandes choses ; et celui qui est injuste en très peu de chose, est injuste aussi dans les grandes choses.
\VS{11}Si donc vous n'avez pas été fidèles dans les richesses injustes, qui vous confiera les véritables richesses ?
\VS{12}Et si en ce qui est à autrui vous n’avez pas été fidèles, qui vous donnera ce qui est vôtre ?
\VS{13}Nul serviteur ne peut servir deux maîtres. Car, ou il haïra l'un, et aimera l'autre ; ou il s'attachera à l'un, et méprisera l'autre. Vous ne pouvez pas servir Dieu et Mamon\FTNT{Voir commentaire en Mt. 6:24.}.
\TextTitle{L'avarice condamnée par Jésus}
\VS{14}Or les pharisiens aussi, qui étaient avares, entendaient toutes ces choses, et ils se moquaient de lui.
\VS{15}Et il leur dit : Vous, vous cherchez à paraître justes devant les hommes ; mais Dieu connaît vos cœurs ; c'est pourquoi ce qui est élevé parmi les hommes est une abomination devant Dieu.
\VS{16}La loi et les prophètes ont duré jusqu'à Jean ; depuis lors, le Royaume de Dieu est prêché, et chacun y fait violence.
\VS{17}Or il est plus aisé que le ciel et la terre passent, qu'il ne l'est qu'un trait de la lettre de la loi vienne à tomber.
\TextTitle{Enseignement de Jésus sur le divorce\FTNTT{Mt. 5:31-32 ; 19:1-9 ; Mc. 10:2-12}}
\VS{18}Quiconque répudie sa femme, et se marie à une autre, commet un adultère, et quiconque prend celle qui a été répudiée par son mari, commet un adultère.
\TextTitle{Histoire de l'homme riche et de Lazare}
\VS{19}Il y avait un homme riche, qui était vêtu de pourpre et de fin lin, et qui tous les jours se réjouissait d'une vie somptueuse.
\VS{20}Il y avait un pauvre, nommé Lazare, couché à la porte du riche, tout couvert d'ulcères,
\VS{21}et qui désirait se rassasier des miettes qui tombaient de la table du riche ; et même les chiens venaient encore lécher ses ulcères.
\VS{22}Et il arriva que le pauvre mourut, et il fut porté par les anges dans le sein d'Abraham\FTNT{Contrairement aux idées reçues, le sein d'Abraham ne se trouvait pas au ciel. En effet, le Seigneur a dit que personne n'était monté au ciel si ce n'est lui-même (Jn. 3:13). Avant l'ère de la grâce, tous les morts allaient dans le séjour des morts où ils étaient retenus prisonniers par le dieu Hadès (voir commentaire sur l'enfer en Mt. 16:18). Toutefois, ce lieu était séparé en deux parties distinctes, l'une réservée aux impies, où ils y subissaient des tourments, et l'autre réservée aux personnes pieuses qui se tenaient en repos, sans souffrir. En effet, lorsque Saül fit appel à une voyante pour faire remonter Samuel du séjour des morts afin de le consulter, Samuel lui annonça qu'il le rejoindrait dès le lendemain à l'endroit où il se trouvait (1 S. 28:19). De plus, dans le récit de la mort du pauvre Lazare et du riche, deux points importants sont à noter. D'une part, bien qu'étant séparés l'un de l'autre, ils pouvaient se voir et communiquer ensemble (Lu. 16:23-26). D'autre part, il est évident que le riche souffrait tandis que le pauvre Lazare était consolé (Lu. 16:25). Lorsque le Seigneur est mort, il est descendu dans « les régions inférieures de la terre » pour délivrer les captifs pieux qui avaient vécu avant Jésus-Christ (Ep. 4:8-9 ; 1 S. 2:6). Par la même occasion, il confirma la condamnation des impies (1 Pi. 3:19). Maintenant que Jésus-Christ est mort et ressuscité, tous ceux qui meurent dans le Seigneur vont au ciel (2 Co. 5:1-3. ; Ph. 1:22-23).}. Le riche mourut aussi, et il fut enseveli.
\VS{23}Etant en enfer\FTNT{Le mot traduit par « enfer » vient du grec « Hades ». Voir commentaire Mt. 16:18}, il leva ses yeux ; et, tandis qu'il était dans les tourments, il vit de loin Abraham et Lazare dans son sein.
\VS{24}Il s'écria : Père Abraham aie pitié de moi, et envoie Lazare, pour qu'il trempe le bout de son doigt dans l'eau et me rafraichisse la langue ; car je suis grièvement tourmenté dans cette flamme.
\VS{25}Abraham répondit : Mon enfant, souviens-toi que tu as reçu tes biens pendant ta vie, et que Lazare a eu ses maux pendant la sienne ; maintenant il est ici consolé, et toi, tu es grièvement tourmenté.
\VS{26}D'ailleurs, il y a entre nous et vous un grand abîme ; en sorte que ceux qui veulent passer d'ici vers vous ne le peuvent, et que ceux qui veulent passer de là ne traversent pas non plus vers nous.
\VS{27}Et il dit : Je te prie donc, père, de l'envoyer dans la maison de mon père ; car j'ai cinq frères.
\VS{28}Afin qu'il leur rende témoignage de l'état où je suis ; de peur qu'eux aussi ne viennent dans ce lieu de tourment.
\VS{29}Abraham lui répondit : Ils ont Moïse et les prophètes ; qu'ils les écoutent.
\VS{30}Mais il dit : Non, père Abraham, mais si quelqu'un des morts va vers eux, ils se repentiront.
\VS{31}Et Abraham lui dit : S'ils n'écoutent pas Moïse et les prophètes, ils ne seront pas non plus persuadés quand quelqu'un des morts ressusciterait.
\Chap{17}
\TextTitle{Instructions de Jésus au sujet des scandales, du pardon et de la foi\FTNTT{Mt. 5:31-32 ; 19:1-9 ; Mc. 10:2-12}}
\VerseOne{}Or il dit à ses disciples : Il est impossible qu'il n'arrive pas des scandales ; mais malheur à celui par qui ils arrivent.
\VS{2}Il vaudrait mieux pour lui qu'on lui mette une pierre de moulin autour de son cou, et qu'on le jette dans la mer, que de scandaliser un seul de ces petits.
\VS{3}Prenez garde à vous-mêmes. Si donc ton frère a péché contre toi, reprends-le ; et s'il se repent, pardonne-lui.
\VS{4}Et s'il a péché contre toi sept fois dans un jour et que sept fois il revienne à toi, disant : Je me repens, tu lui pardonneras.
\VS{5}Alors les apôtres dirent au Seigneur : Augmente-nous la foi.
\VS{6}Et le Seigneur dit : Si vous aviez de la foi aussi gros qu'un grain de semence de moutarde, vous diriez à ce sycomore : Déracine-toi, et plante-toi dans la mer ; et il vous obéirait.
\TextTitle{Les serviteurs inutiles}
\VS{7}Mais qui de vous, ayant un serviteur qui laboure ou paît les troupeaux, lui dira, quand il revient des champs : Approche-toi vite, et mets-toi à table.
\VS{8}Ne lui dira-t-il pas plutôt : Prépare-moi à souper, ceins-toi, et sers-moi jusqu'à ce que j'aie mangé et bu ; et après cela tu mangeras et tu boiras ?
\VS{9}Doit-il de la reconnaissance à ce serviteur parce qu'il a fait ce qui lui était ordonné ? Je ne le pense pas.
\VS{10}Vous de même, quand vous aurez fait tout ce qui vous a été ordonné, dites : Nous sommes des serviteurs inutiles ;  ce que nous étions obligés de faire, nous l'avons fait.
\TextTitle{Guérison de dix lépreux}
\VS{11}Et il arriva qu’en allant à Jérusalem, il passait par le milieu de la Samarie, et de la Galilée.
\VS{12}Et comme il entrait dans un village, dix hommes lépreux vinrent à sa rencontre. Se tenant à distance, ils élevèrent la voix, et dirent :
\VS{13}Jésus, Maître, aie pitié de nous !
\VS{14}Et quand il les eut vus, il leur dit : Allez, montrez-vous aux sacrificateurs\FTNT{Lé. 13.}. Et, pendant qu'ils y allaient, ils furent purifiés.
\VS{15}L'un d'eux se voyant guéri, revint sur ses pas, glorifiant Dieu à haute voix.
\VS{16}Et il se jeta en terre sur sa face aux pieds de Jésus, lui rendant grâces. Or c’était un Samaritain.
\VS{17}Alors Jésus prenant la parole, dit : Les dix n'ont-ils pas été rendus purs ? Et les neuf autres, où sont-ils ?
\VS{18}Il n'y a eu que cet étranger qui soit revenu pour rendre gloire à Dieu.
\VS{19}Alors il lui dit : Lève-toi. Va, ta foi t'a sauvé.
\TextTitle{Les pharisiens demandent à voir le Royaume de Dieu\FTNTT{cp. Lu. 19:11-27}}
\VS{20}Or les pharisiens demandèrent à Jésus quand viendrait le Royaume de Dieu. Il leur répondit, et leur dit : Le Royaume de Dieu ne vient pas de manière à attirer l'attention.
\VS{21}Et on ne dira point : Il est ici ; ou : Il est là. Car voici, le Royaume de Dieu est au milieu de vous.
\TextTitle{Jésus annonce sa seconde venue\FTNTT{voir De. 30:3}}
\VS{22}Il dit aussi à ses disciples : Des jours viendront où vous désirerez voir un des jours du Fils de l'homme, mais vous ne le verrez point. On vous dira :
\VS{23}Il est ici, ou : Il est là. N'allez pas, et ne les suivez point.
\VS{24}Car, comme l'éclair brille et resplendit d'une extrémité du ciel à l'autre, ainsi sera le Fils de l'homme en son jour.
\VS{25}Mais il faut premièrement qu'il souffre beaucoup, et qu'il soit rejeté par cette génération.
\VS{26}Ce qui arriva aux jours de Noé, arrivera de même aux jours du Fils de l'homme.
\VS{27}On mangeait et on buvait ; on prenait et on donnait des femmes en mariage jusqu'au jour où Noé entra dans l'arche ; le déluge vint, et les fit tous périr.
\VS{28}C’est encore ce qui arriva aux jours de Lot : On mangeait, on buvait, on achetait, on vendait, on plantait et on bâtissait.
\VS{29}Mais le jour où Lot sortit de Sodome, une pluie de feu et de soufre tomba du ciel, et les fit tous périr.
\VS{30}Il en sera de même au jour où le Fils de l'homme paraîtra.
\VS{31}En ce jour-là, que celui qui sera sur le toit, et qui aura ses effets dans la maison, ne descende point pour les prendre ; et que celui qui sera dans les champs, ne retourne pas non plus à ce qui est resté en arrière.
\VS{32}Souvenez-vous de la femme de Lot.
\VS{33}Quiconque cherchera à sauver sa vie, la perdra ; et quiconque la perdra, la retrouvera.
\VS{34}Je vous dis, qu’en cette nuit-là deux seront dans un même lit : l’un sera pris, et l’autre laissé ;
\VS{35}deux femmes moudront ensemble, l’une sera prise et l’autre laissée ;
\VS{36}deux seront aux champs, l’un sera pris et l’autre laissé.
\VS{37}Les disciples lui dirent : Où, Seigneur ? Et il leur dit, Là où est le corps, là aussi s’assembleront les aigles.
\Chap{18}
\TextTitle{Parabole du juge inique}
\VerseOne{}Et il leur proposa une parabole, pour montrer qu'il faut toujours prier, et ne point se relâcher,
\VS{2}disant : Il y avait dans une ville un juge qui ne craignait point Dieu et qui ne respectait personne.
\VS{3}Et dans la même ville, il y avait une veuve, qui venait souvent lui dire : Fais-moi justice de ma partie adverse.
\VS{4}Pendant longtemps il refusa. Mais après cela il dit en lui-même : Quoique je ne craigne point Dieu, et que je ne respecte personne,
\VS{5}néanmoins, parce que cette veuve me donne de la peine, je lui ferai justice, de peur qu'elle ne vienne sans cesse me casser la tête.
\VS{6}Et le Seigneur dit : Ecoutez ce que dit le juge inique.
\VS{7}Et Dieu ne ferait-il point justice à ses élus, qui crient à lui jour et nuit, quoiqu’il use de patience avant d’intervenir pour eux ?
\VS{8}Je vous le dis que bientôt il les vengera. Mais quand le Fils de l'homme viendra, pensez-vous qu'il trouvera la foi sur la terre ?
\TextTitle{Parabole du pharisien et du publicain}
\VS{9}Il dit aussi cette parabole au sujet de certaines personnes se persuadant qu'elles étaient justes, et ne faisant aucun cas des autres :
\VS{10}Deux hommes montèrent au temple pour prier, l'un était pharisien, et l'autre, publicain.
\VS{11}Le pharisien, se tenant debout, priait en lui-même en ces termes : Ô Dieu ! Je te rends grâces de ce que je ne suis pas comme le reste des hommes, qui sont ravisseurs, injustes, adultères, ni même comme ce publicain.
\VS{12}Je jeûne deux fois la semaine, et je donne la dîme de tout ce que je possède.
\VS{13}Mais le publicain se tenant loin, n'osait même pas lever les yeux vers le ciel, mais il se frappait la poitrine, en disant : Ô Dieu ! Sois apaisé envers moi qui suis pécheur !
\VS{14}Je vous dis que celui-ci descendit dans sa maison justifié, plutôt que l'autre ; car quiconque s'élève, sera abaissé, et quiconque s'abaisse, sera élevé.
\TextTitle{Le Royaume des cieux, pour ceux qui ressemblent aux petits enfants\FTNTT{Mt. 19:13-15 ; Mc. 10:13-16}}
\VS{15}Et quelques-uns lui présentèrent aussi de petits enfants, afin qu'il les touchât, mais les disciples voyant cela, reprenaient ceux qui les présentaient.
\VS{16}Mais Jésus les appela, et dit : Laissez venir à moi les petits enfants, et ne les en empêchez pas ; car le Royaume de Dieu est pour ceux qui leur ressemblent.
\VS{17}Je vous le dis en vérité, quiconque ne recevra point comme un enfant le Royaume de Dieu, n’y entrera point.
\TextTitle{Jésus dénonce l'attachement aux richesses\FTNTT{Mt. 19:16-30 ; Mc. 10:17-31 ; cp. Lu. 10:25-37}}
\VS{18}Un chef interrogea Jésus et dit : Bon Maître, que dois-je faire pour hériter la vie éternelle ?
\VS{19}Jésus lui dit : Pourquoi m'appelles-tu bon ? Il n'y a de bon que Dieu seul\FTNT{Voir commentaire Mc. 10:18.}.
\VS{20}Tu connais les commandements : Tu ne commettras point d'adultère. Tu ne tueras point. Tu ne déroberas point. Tu ne diras point de faux témoignage. Honore ton père et ta mère.
\VS{21}Et il lui dit : J'ai observé toutes ces choses dès ma jeunesse.
\VS{22}Et quand Jésus eut entendu cela, lui dit : Il te manque encore une chose : Vends tout ce que tu as, et distribue-le aux pauvres, et tu auras un trésor dans les cieux. Puis viens, et suis-moi.
\VS{23}Lorsqu'il entendit ces choses, il devint tout triste, car il était extrêmement riche.
\VS{24}Jésus voyant qu'il était devenu tout triste, dit : Qu'il est difficile à ceux qui ont des richesses d'entrer dans le Royaume de Dieu !
\VS{25}Car il est plus facile à un chameau de passer par le trou d'une aiguille, qu'à un riche d'entrer dans le Royaume de Dieu\FTNT{Voir commentaire Mt. 19:24.}.
\VS{26}Ceux qui entendirent cela, dirent : Et qui peut donc être sauvé ?
\VS{27}Jésus leur répondit : Ce qui est impossible aux hommes est possible à Dieu.
\TextTitle{Récompense pour un vrai disciple de Jésus}
\VS{28}Pierre dit : Voici, nous avons tout quitté, et nous t'avons suivi.
\VS{29}Et il leur dit : Je vous le dis en vérité, il n'est personne qui, ayant quitté pour l'amour du Royaume de Dieu, sa maison, ou ses parents, ou ses frères, ou sa femme, ou ses enfants,
\VS{30}ne reçoive beaucoup plus dans ce siècle-ci, et dans le siècle à venir la vie éternelle.
\TextTitle{Jésus annonce à nouveau sa mort et sa résurrection\FTNTT{Mt. 20:17-19 ; Mc. 10:32-34}}
\VS{31}Jésus prit à part les douze, et il leur dit : Voici, nous montons à Jérusalem, et tout ce qui est écrit par les prophètes au sujet du Fils de l'homme, s'accomplira.
\VS{32}Car il sera livré aux Gentils ; on se moquera de lui, on l'outragera, et on lui crachera au visage,
\VS{33}et après l'avoir battu de verges, on le fera mourir ; mais il ressuscitera le troisième jour.
\VS{34}Mais ils ne comprirent rien à cela, et ce discours était si obscur pour eux qu'ils ne comprirent point ce qu'il leur disait.
\TextTitle{Bartimée voit !\FTNTT{cp. Mt. 20:29-34 ; Mc. 10:46-53}}
\VS{35}Or comme il approchait de Jéricho, un aveugle était assis au bord du chemin, et mendiait.
\VS{36}Et entendant la foule qui passait, il demanda ce que c'était.
\VS{37}Et on lui dit : C'est Jésus de Nazareth qui passe.
\VS{38}Alors il cria, disant : Jésus, Fils de David, aie pitié de moi !
\VS{39}Ceux qui marchaient devant le reprenaient, pour le faire taire ; mais il criait beaucoup plus fort : Fils de David, aie pitié de moi !
\VS{40}Et Jésus s'étant arrêté ordonna qu'on le lui amène ; et, quand il se fut approché,
\VS{41}il lui demanda : Que veux-tu que je te fasse ? Il répondit : Seigneur, que je recouvre la vue.
\VS{42}Jésus lui dit : Recouvre la vue ; ta foi t'a sauvé.
\VS{43}Et à l'instant il recouvra la vue et suivit Jésus, glorifiant Dieu. Et tout le peuple voyant cela, loua Dieu.
\Chap{19}
\TextTitle{Conversion de Zachée}
\VerseOne{}Jésus, étant entré dans Jéricho, traversait la ville.
\VS{2}Et voici, un homme riche, appelé Zachée, chef des publicains, cherchait à voir qui était Jésus,
\VS{3}mais il ne le pouvait pas à cause de la foule, car il était de petite taille.
\VS{4}C'est pourquoi il accourut devant, et monta sur un sycomore pour le voir ; car il devait passer par là.
\VS{5}Et quand Jésus fut arrivé à cet endroit-là, il leva les yeux, le vit, et lui dit : Zachée, hâte-toi de descendre ; car il faut que je demeure aujourd'hui dans ta maison.
\VS{6}Zachée se hâta de descendre, et le reçut avec joie.
\VS{7}Et tous voyant cela murmuraient, et disaient : Il est entré chez un homme pécheur pour y loger.
\VS{8}Et Zachée, se présentant devant le Seigneur, lui dit : Voici, Seigneur, je donne la moitié de mes biens aux pauvres ; et si j'ai fait tort de quelque chose à quelqu'un, je lui rends le quadruple\FTNT{Lé. 5:20-24.}.
\VS{9}Et Jésus lui dit : Aujourd'hui le salut est entré dans cette maison ; parce que celui-ci aussi est fils d'Abraham.
\VS{10}Car le Fils de l'homme est venu chercher et sauver ce qui était perdu.
\TextTitle{Parabole des dix mines\FTNTT{Lu. 17:21}}
\VS{11}Et comme ils entendaient ces choses, Jésus poursuivit son discours, et proposa une parabole, parce qu'il était près de Jérusalem, et qu'ils pensaient que le royaume de Dieu allait immédiatement paraître.
\VS{12}Il dit donc : Un homme noble s'en alla dans un pays éloigné, pour prendre possession d'un Royaume, et revenir ensuite.
\VS{13}Il appela dix de ses serviteurs, il leur donna dix mines et leur dit : Faites-les valoir jusqu'à ce que je revienne.
\VS{14}Or ses concitoyens le haïssaient, c'est pourquoi ils envoyèrent après lui une ambassade, pour dire : Nous ne voulons pas que cet homme règne sur nous.
\VS{15} Il arriva donc après qu'il fut de retour, et après avoir pris possession du Royaume, qu'il fit appeler auprès de lui les serviteurs auxquels il avait confié son argent, afin de connaître comment chacun l'avait fait valoir.
\VS{16}Alors le premier vint, et dit : Seigneur, ta mine a produit dix autres mines.
\VS{17}Il lui dit : C'est bien, bon serviteur ; parce que tu as été fidèle en peu de choses, reçois le gouvernement de dix villes.
\VS{18}Et le second vint, et dit : Seigneur, ta mine a produit cinq autres mines.
\VS{19}Il dit aussi à celui-ci : Toi aussi, sois établi sur cinq villes.
\VS{20}Un autre vint, et dit : Seigneur, voici ta mine que j'ai gardée enveloppée dans un linge ;
\VS{21}car j'avais peur de toi, parce que tu es un homme sévère ; tu prends ce que tu n'as point déposé, et tu moissonnes ce que tu n'as pas semé.
\VS{22}Il lui dit : Méchant serviteur, je te jugerai sur tes propres paroles : Tu savais que je suis un homme sévère, prenant ce que je n'ai point déposé, et moissonnant ce que je n'ai point semé.
\VS{23}Pourquoi donc n'as-tu pas mis mon argent dans une banque, afin qu'à mon retour je le retire avec un intérêt ?
\VS{24}Alors il dit à ceux qui étaient présents : Ôtez-lui la mine, et donnez-la à celui qui a les dix.
\VS{25}Ils lui dirent : Seigneur, il a dix mines.
\VS{26}Ainsi je vous le dis, on donnera à celui qui a, mais à celui qui n'a pas, on ôtera ce qu'il a.
\VS{27}Au reste, amenez ici mes ennemis qui n'ont pas voulu que je règne sur eux, et tuez-les devant moi.
\TextTitle{Jésus fait son entrée à Jérusalem\FTNTT{Za. 9:9 ; Mt. 21:1-11 ; Mc. 11:1-11 ; Jn. 12:12-19}}
\VS{28}Après avoir ainsi parlé, Jésus marcha devant la foule, pour monter à Jérusalem.
\VS{29}Lorsqu'il approcha de Bethphagé et de Béthanie, vers la montagne appelée Montagne des Oliviers, Jésus envoya deux de ses disciples,
\VS{30}en leur disant : Allez au village qui est en face ; quand vous y serez entrés, vous trouverez un ânon attaché, sur lequel aucun homme n'est monté ; détachez-le, et amenez-le-moi.
\VS{31}Si quelqu'un vous demande pourquoi le détachez-vous, vous lui répondrez : Le Seigneur en a besoin.
\VS{32}Et ceux qui étaient envoyés s'en allèrent, et trouvèrent l'ânon comme il le leur avait dit.
\VS{33}Comme ils le détachaient, ses maîtres leur dirent : Pourquoi détachez-vous cet ânon ?
\VS{34}Ils répondirent : Le Seigneur en a besoin.
\VS{35}Ils emmenèrent à Jésus l'ânon, sur lequel ils jetèrent leurs vêtements, et firent monter Jésus dessus.
\VS{36}Quand il fut en marche, les gens étendirent leurs vêtements sur le chemin.
\VS{37}Et lorsque déjà il approchait de Jérusalem, vers la descente de la Montagne des Oliviers, toute la multitude des disciples saisie de joie, se mit à louer Dieu à haute voix, pour tous les miracles qu'ils avaient vus.
\VS{38}Ils disaient : Béni soit le Roi qui vient au Nom du Seigneur\FTNT{Ps. 118:26.} ! Paix dans le ciel, et gloire dans les lieux très hauts.
\VS{39}Quelques pharisiens, du milieu de la foule, lui dirent : Maître, reprends tes disciples.
\VS{40}Et Jésus répondit : Je vous le dis, s'ils se taisent, les pierres crieront.
\TextTitle{Nouvelles lamentations de Jésus sur Jérusalem\FTNTT{cp. Mt. 23:37-39 ; Lu. 13:34-35}}
\VS{41}Comme il approchait de la ville, Jésus, en la voyant, pleura sur elle, et dit :
\VS{42}Ô ! Si toi aussi, au moins en ce jour qui t'est donné, tu connaissais les choses qui appartiennent à ta paix ! Mais maintenant elles sont cachées à tes yeux.
\VS{43}Il viendra sur toi des jours où tes ennemis t'environneront de tranchées, t'enfermeront, et te serreront de tous côtés ;
\VS{44}ils te raseront, toi et tes enfants qui sont au milieu de toi, et ils ne laisseront pas en toi pierre sur pierre, parce que tu n'as pas connu le temps de ta visitation.
\TextTitle{Jésus chasse les marchands du temple}
\VS{45}Il entra dans le temple, et il se mit à chasser dehors ceux qui vendaient et qui achetaient.
\VS{46}Leur disant : Il est écrit : Ma maison sera appelée la maison de prière ; mais vous, vous en avez fait une caverne de voleurs\FTNT{Es. 56:7 ; Jé. 7:11.}.
\VS{47}Il enseignait tous les jours dans le temple. Et les principaux sacrificateurs et les scribes cherchaient à le faire mourir.
\VS{48}Mais ils ne savaient comment s'y prendre ; car tout le peuple s'attachait à ses paroles.
\Chap{20}
\TextTitle{L'autorité de Jésus et celle de Jean-Baptiste\FTNTT{Mt. 21:23-27 ; Mc. 11:27-33}}
\VerseOne{}Et il arriva un de ces jours-là, comme Jésus enseignait le peuple dans le temple, et qu'il évangélisait, les principaux sacrificateurs, les scribes et les anciens survinrent,
\VS{2}et lui parlèrent en disant : Dis-nous par quelle autorité fais-tu ces choses, ou qui est celui qui t'a donné cette autorité ?
\VS{3}Jésus leur répondit : Je vous adresserai aussi une question, et répondez-moi.
\VS{4}Le baptême de Jean venait-il du ciel ou des hommes ?
\VS{5}Ils raisonnaient entre eux, disant : Si nous répondons : Du ciel ; il dira : Pourquoi n'avez-vous pas cru en lui ?
\VS{6}Et si nous répondons : Des hommes, tout le peuple nous lapidera ; car il est persuadé que Jean était un prophète ;
\VS{7}Alors, ils répondirent qu'ils ne savaient d'où il était.
\VS{8}Et Jésus leur dit : Moi non plus, je ne vous dirai pas par quelle autorité je fais ces choses.
\TextTitle{Parabole des vignerons\FTNTT{Es. 5:1-7 ; Mt. 21:33-46 ; Mc. 12:1-12}}
\VS{9}Alors il se mit à dire au peuple cette parabole : Un homme planta une vigne, et la loua à des vignerons, et fut longtemps absent.
\VS{10}Et à la saison de la récolte, il envoya un serviteur vers les vignerons, afin qu'ils lui donnent du fruit de la vigne. Les vignerons le battirent, et le renvoyèrent à vide.
\VS{11}Il leur envoya encore un autre serviteur ; mais ils le battirent aussi, et après l'avoir traité indignement, ils le renvoyèrent à vide.
\VS{12}Il en envoya encore un troisième, mais ils le blessèrent aussi, et le jetèrent dehors.
\VS{13}Alors le maître de la vigne dit : Que ferai-je ? J'enverrai mon fils bien-aimé ; peut-être que quand ils le verront, ils le respecteront.
\VS{14}Mais quand les vignerons le virent, ils raisonnèrent entre eux, et dirent : Voici l'héritier ; venez, tuons-le, afin que l'héritage soit à nous.
\VS{15}Et ils le jetèrent hors de la vigne, et le tuèrent. Que leur fera donc le maître de la vigne ?
\VS{16}Il viendra, et fera périr ces vignerons-là, et il donnera la vigne à d'autres. Lorsqu'ils entendirent cela, ils dirent : A Dieu ne plaise !
\VS{17}Alors il les regarda, et dit : Que signifie donc ce qui est écrit : La pierre qu'on rejetée ceux qui bâtissaient est devenue la principale de l'angle\FTNT{Ps. 118:22.} ?
\VS{18}Quiconque tombera sur cette pierre, sera brisé ; et elle écrasera celui sur qui elle tombera.
\TextTitle{Le tribut à César\FTNTT{Mt. 22:15-22 ; Mc. 12:13-17}}
\VS{19}Les principaux sacrificateurs et les scribes cherchèrent à mettre la main sur lui à l'heure même, mais ils craignirent le peuple. Ils avaient compris que c'était pour eux que Jésus avait dit cette parabole.
\VS{20}Ils se mirent à observer Jésus ; et ils envoyèrent des agents secrets, qui feignaient d'être justes, pour lui tendre des pièges et saisir de lui quelque parole afin de le livrer au magistrat et à l'autorité du gouverneur.
\VS{21}Ils l'interrogèrent, en disant : Maître, nous savons que tu parles et enseignes conformément à la justice, et que tu ne regardes pas à l'apparence des personnes, mais que tu enseignes la voie de Dieu selon la vérité.
\VS{22}Nous est-il permis de payer le tribut à César, ou non ?
\VS{23}Jésus, apercevant leur ruse, leur dit : Pourquoi me tentez-vous ?
\VS{24}Montrez-moi un denier. De qui a-t-il l'image et l'inscription ? Ils lui répondirent : De César.
\VS{25}Alors il leur dit : Rendez donc à César ce qui est à César ; et à Dieu ce qui est à Dieu.
\VS{26}Ainsi ils ne purent le surprendre dans ses paroles devant le peuple ; mais, étonnés de sa réponse, ils gardèrent le silence.
\TextTitle{Les preuves de la résurrection\FTNTT{Mt. 22:23-33 ; Mc.12:18-27}}
\VS{27}Alors quelques-uns des sadducéens, qui nient formellement la résurrection, s'approchèrent et l'interrogèrent,
\VS{28}disant : Maître, voici ce que Moïse nous a prescrit : Si le frère de quelqu'un meurt, ayant une femme et pas d'enfants, son frère épousera la femme, et suscitera une postérité à son frère.
\VS{29}Or, il y avait sept frères. Le premier se maria, et mourut sans enfants.
\VS{30}Le deuxième épousa la femme et mourut sans enfants.
\VS{31}Puis le troisième l'épousa aussi, et tous les sept de même ; et ils moururent sans laisser d'enfants.
\VS{32}Enfin, la femme mourut aussi.
\VS{33}Duquel d'entre eux donc sera-t-elle la femme à la résurrection ? Car les sept l'ont eue pour femme.
\VS{34}Jésus leur répondit : Les enfants de ce siècle prennent des femmes et des maris ;
\VS{35}mais ceux qui seront trouvés dignes d'avoir part au siècle à venir et à la résurrection des morts, ne prendront ni femmes ni maris.
\VS{36}Car ils ne pourront plus mourir, parce qu'ils seront semblables aux anges, et qu'ils seront fils de Dieu, étant fils de la résurrection.
\VS{37}Que les morts ressuscitent, c'est ce que Moïse a fait connaître quand, à propos du buisson, il appelle le Seigneur le Dieu d'Abraham, le Dieu d'Isaac, et le Dieu de Jacob.
\VS{38}Or, Dieu n'est pas le Dieu des morts, mais des vivants ; car tous vivent en lui.
\TextTitle{Jésus dénonce l'attitude des scribes\FTNTT{cp. Mt. 22:41-23:36 ; Mc. 12:35-40}}
\VS{39}Quelques-uns des scribes prenant la parole, dirent : Maître, tu as bien parlé.
\VS{40}Et ils n'osaient plus lui poser aucune question.
\VS{41}Jésus leur dit : Comment dit-on que le Christ est Fils de David ?
\VS{42}Car David lui-même dit au livre des psaumes : Le Seigneur a dit à mon Seigneur : Assieds-toi à ma droite,
\VS{43}jusqu'à ce que j'aie mis tes ennemis pour le marchepied de tes pieds\FTNT{Ps. 110:1.}.
\VS{44}David donc l'appelle son Seigneur, comment est-il son Fils ?
\VS{45}Comme tout le peuple l'écoutait, il dit à ses disciples :
\VS{46}Gardez-vous des scribes, qui aiment à se promener en robes longues, et qui aiment les salutations sur les places publiques ; qui recherchent les premiers sièges dans les synagogues, et les premières places dans les festins ;
\VS{47}qui dévorent entièrement les maisons des veuves, et qui font pour l'apparence de longues prières. Ils seront jugés plus sévèrement.
\Chap{21}
\TextTitle{Offrande de la pauvre veuve\FTNTT{Mc. 12:41-44}}
\VerseOne{}Comme Jésus regardait, il vit des riches qui mettaient leurs offrandes dans le tronc.
\VS{2}Il vit aussi une pauvre veuve qui y mettait deux petites pièces de monnaie.
\VS{3}Et il dit : Je vous le dis en vérité, cette pauvre veuve a mis plus que tous les autres.
\VS{4}Car tous ceux-ci ont mis aux offrandes de Dieu, de leur superflu ; mais elle a mis de son nécessaire, tout ce qu'elle avait pour vivre.
\TextTitle{Enseignement sur le Mont des Oliviers\FTNTT{Mt. 24-25 ; Mc. 13}}
\VS{5}Comme quelques-uns disaient que le temple était orné de belles pierres et d'offrandes, il dit :
\VS{6}Vous contemplez ces choses ! Les jours viendront où, il ne restera pas pierre sur pierre qui ne soit démolie.
\TextTitle{Les disciples posent deux questions à Jésus\FTNTT{Mt. 24:3 ; Mc.13:3-4}}
\VS{7}Ils lui demandèrent : Maître, quand donc cela arrivera-t-il, et à quel signe connaîtra-t-on que ces choses vont arriver ?
\TextTitle{Les temps de la fin\FTNTT{Mt. 24:4-14 ; Mc. 13:5-13}}
\VS{8}Jésus répondit : Prenez garde que vous ne soyez point séduits. Car plusieurs viendront en mon Nom, disant : C'est moi qui suis le Christ et le temps approche. Ne les suivez pas.
\VS{9}Quand vous entendrez parler des guerres et des soulèvements, ne soyez pas effrayés ; car il faut que ces choses arrivent premièrement. Mais ce ne sera pas encore la fin.
\VS{10}Alors il leur dit : Une nation s'élèvera contre une autre nation, et un royaume contre un autre royaume.
\VS{11}Il y aura de grands tremblements de terre en divers lieux, des famines et des pestes ; il y aura des choses terribles, et de grands signes dans le ciel.
\TextTitle{Souffrance des croyants}
\VS{12}Mais, avant toutes ces choses, ils mettront la main sur vous, et l'on vous persécutera ; on vous livrera aux synagogues, on vous jettera en prison, on vous mènera devant des rois et devant des gouverneurs, à cause de mon Nom.
\VS{13}Cela vous arrivera pour que vous serviez de témoignage.
\VS{14}Mettez-vous donc dans vos cœurs de ne pas préméditer votre défense.
\VS{15}Car je vous donnerai une bouche et une sagesse à laquelle vos adversaires ne pourront résister ou contredire.
\VS{16}Vous serez livrés même par vos parents, par vos frères, par vos proches et par vos amis, et ils feront mourir plusieurs d'entre vous.
\VS{17}Vous serez haïs de tous à cause de mon Nom.
\VS{18}Mais il ne se perdra pas un cheveu de votre tête.
\VS{19}Vous sauverez vos âmes par votre persévérance.
\TextTitle{La destruction de Jérusalem prophétisée}
\VS{20}Lorsque vous verrez Jérusalem environnée par les armées, sachez alors que sa désolation est proche.
\VS{21}Alors, que ceux qui seront en Judée, fuient dans les montagnes ; et que ceux qui seront au milieu de Jérusalem, en sortent, et que ceux qui seront dans les champs, n'entrent pas dans la ville.
\VS{22}Car ce seront des jours de vengeance, afin que toutes les choses qui sont écrites soient accomplies.
\VS{23}Malheur aux femmes qui seront enceintes, et à celles qui allaiteront en ces jours-là ; car il y aura une grande calamité sur le pays, et une grande colère contre ce peuple.
\VS{24}Ils tomberont sous le tranchant de l'épée, ils seront emmenés captifs\FTNT{Les Juifs se révoltèrent plusieurs fois contre le joug des Romains installés en Palestine depuis l'an 65 av. J.-C. En 70, Titus s'empara de Jérusalem après une guerre de plusieurs années et un siège meurtrier de sept mois. Cette même année, le temple fut détruit. A la suite d'une dernière révolte, la ville fut prise de nouveau sous Hadrien. En l'an 135, les Juifs furent en grande partie exterminés, et les survivants furent à jamais chassés de Jérusalem. Ces événements marquèrent symboliquement les débuts de la dispersion des Juifs à travers le monde.} parmi toutes les nations ; et Jérusalem sera foulée par les nations, jusqu'à ce que les temps des nations soient accomplis.
\TextTitle{Retour du Messie sur la terre\FTNTT{Mt. 24:29-31 ; Mc. 13:24-27}}
\VS{25}Il y aura des signes dans le soleil, dans la lune, et dans les étoiles. Et sur terre, il y aura de la détresse chez les nations qui ne sauront que faire, au bruit de la mer et des flots,
\VS{26}les hommes seront comme rendant l'âme de frayeur, dans l'attente des choses qui surviendront dans le monde ; car les puissances des cieux seront ébranlées.
\VS{27}Alors on verra le Fils de l'homme venant sur une nuée avec puissance et grande gloire.
\VS{28}Quand ces choses commenceront à arriver, regardez en haut et levez vos têtes, parce que votre délivrance approche.
\TextTitle{Parabole du figuier\FTNTT{Mt. 24:29-31 ; Mc. 13:24-27}}
\VS{29}Et il leur proposa cette comparaison : Voyez le figuier, et tous les autres arbres.
\VS{30}Dès qu'ils ont poussé, vous savez de vous-mêmes, en regardant, que déjà l'été est proche.
\VS{31}Vous aussi de même, quand vous verrez arriver ces choses, sachez que le Royaume de Dieu est proche.
\VS{32}En vérité je vous le dis, que cette génération ne passera point, que toutes ces choses ne soient arrivées.
\VS{33}Le ciel et la terre passeront, mais mes paroles ne passeront point.
\TextTitle{Exhortation à veiller\FTNTT{Mt. 24:36-51 ; Mc. 13:32-37}}
\VS{34}Prenez donc garde à vous-mêmes, de peur que vos cœurs ne soient appesantis par la gourmandise et l'ivrognerie, et par les soucis de cette vie ; et que ce jour-là ne vous surprenne subitement.
\VS{35}Car il viendra comme un filet sur tous ceux qui habitent sur la surface de toute la terre.
\VS{36}Veillez donc, et priez en tout temps, afin que vous soyez trouvés dignes d'échapper à toutes ces choses qui arriveront, et de paraître devant le Fils de l'homme.
\VS{37}Pendant le jour, Jésus enseignait dans le temple, et il allait passer la nuit à la montagne appelée Montagne des Oliviers.
\VS{38}Et dès le point du jour, tout le peuple venait vers lui au temple pour l'entendre.
\Chap{22}
\TextTitle{Trahison de Judas\FTNTT{Mt. 26:14-16 ; Mc. 14:1-2,10-11}}
\VerseOne{}La fête des pains sans levain, qu'on appelle Pâque, approchait.
\VS{2}Les principaux sacrificateurs et les scribes cherchaient les moyens de faire mourir Jésus ; car ils craignaient le peuple.
\VS{3}Or, Satan entra dans Judas, surnommé Iscariot, qui était du nombre des douze.
\VS{4}Et Judas alla, et parla avec les principaux sacrificateurs et les chefs de gardes, sur la manière de le leur livrer.
\VS{5}Ils furent dans la joie, et convinrent de lui donner de l'argent.
\VS{6}Après s'être engagé, il cherchait une occasion favorable pour leur livrer Jésus à l'insu de la foule.
\TextTitle{La dernière Pâque\FTNTT{Mt. 26:17-25 ; Mc. 14:12-21 ; Jn. 13:1-12}}
\VS{7}Le jour des pains sans levain, où l'on devait immoler la Pâque, arriva.
\VS{8}Et Jésus envoya Pierre et Jean, en leur disant : Allez, et apprêtez-nous l'agneau de Pâque, afin que nous le mangions.
\VS{9}Et ils lui dirent : Où veux-tu que nous l'apprêtions ?
\VS{10}Il leur dit : Voici, quand vous serez entrés dans la ville vous rencontrerez un homme portant une cruche d'eau, suivez-le dans la maison où il entrera.
\VS{11}Et dites au maître de la maison : Le Maître te dit : Où est le lieu où je mangerai l'agneau de Pâque avec mes disciples ?
\VS{12}Et il vous montrera une grande chambre haute, meublée ; c'est là que vous apprêterez l'agneau de Pâque.
\VS{13}Ils partirent, et trouvèrent les choses comme il leur avait dit ; et ils apprêtèrent l'agneau de Pâque.
\VS{14}Et quand l'heure fut venue, il se mit à table, et les douze apôtres avec lui.
\VS{15}Il leur dit : J'ai désiré vivement manger cet agneau de Pâque avec vous avant de souffrir.
\VS{16}Car, je vous dis, que je ne le mangerai plus jusqu'à ce qu'il soit accompli dans le Royaume de Dieu.
\VS{17}Et, ayant pris la coupe, il rendit grâces, et il dit : Prenez cette coupe, et distribuez-la entre vous.
\VS{18}Car, je vous dis, que je ne boirai plus du fruit de la vigne, jusqu'à ce que le Royaume de Dieu soit venu.
\TextTitle{Institution du repas de la Pâque\FTNTT{Mt. 26:26-29 ; Mc. 14:22-25 ; cp. Jn. 13:12-30 ; 1 Co. 11:23-26}}
\VS{19}Ensuite il prit du pain, et après avoir rendu grâces, il le rompit et le leur donna, en disant : Ceci est mon corps, qui est donné pour vous ; faites ceci en mémoire de moi.
\VS{20}Il prit de même la coupe, après le souper, et la leur donna, en disant : Cette coupe est la Nouvelle Alliance en mon sang, qui est répandu pour vous.
\TextTitle{Jésus annonce qu'il sera livré\FTNTT{Mt. 26:21-25 ; Mc. 14:18-21 ; Jn. 13:18-30}}
\VS{21}Cependant voici, la main de celui qui me trahit est avec moi à table.
\VS{22}Le Fils de l'homme s'en va ; selon ce qui est déterminé. Mais malheur à cet homme par qui il est trahi.
\VS{23}Et ils commencèrent à se demander les uns aux autres, qui était celui d'entre eux qui ferait cela.
\TextTitle{Leçon d'humilité\FTNTT{Mt. 20:20-28 ; Mc. 9.33-37 ; 10:35-45 ; Jn. 13:1-17}}
\VS{24}Il s'éleva une contestation parmi les apôtres, pour savoir lequel d'entre eux devait être estimé le plus grand.
\VS{25}Jésus leur dit : Les rois des nations les maîtrisent ; et ceux qui les dominent sont appelés bienfaiteurs.
\VS{26}Mais il n'en sera pas ainsi de vous : Au contraire, que le plus grand parmi vous soit comme le plus petit ; et celui qui gouverne, comme celui qui sert.
\VS{27}Car lequel est le plus grand, celui qui est à table, ou celui qui sert ? N'est-ce pas celui qui est à table ? Or je suis au milieu de vous comme celui qui sert.
\TextTitle{Le Royaume, une récompense}
\VS{28}Vous, vous êtes ceux qui avez persévéré avec moi dans mes épreuves ;
\VS{29}c'est pourquoi je vous confie le Royaume comme mon Père me l'a confié,
\VS{30}afin que vous mangiez et buviez à ma table dans mon Royaume, et que vous soyez assis sur des trônes, pour juger les douze tribus d'Israël.
\TextTitle{Jésus prophétise le triple reniement de Pierre\FTNTT{Mt. 26:30-35 ; Mc. 14:26-31 ; Jn. 13:36-38}}
\VS{31}Le Seigneur dit aussi : Simon, Simon, voici, Satan vous a réclamés pour vous cribler comme le froment ;
\VS{32}mais j'ai prié pour toi afin que ta foi ne défaille point ; et toi donc, quand tu seras un jour converti, affermis tes frères.
\VS{33}Pierre lui dit : Seigneur, je suis prêt à aller avec toi en prison et à la mort.
\VS{34}Mais Jésus lui dit : Pierre, je te dis que le coq ne chantera pas aujourd'hui, que tu n'aies nié trois fois de me connaître.
\TextTitle{Recommandation aux disciples\FTNTT{cp. Jn. 14-16 ; contraste Mt. 10:9-13}}
\VS{35}Puis il leur dit : Quand je vous ai envoyés sans bourse, sans sac, et sans souliers, avez-vous manqué de quelque chose ? Ils répondirent : De rien.
\VS{36}Et il leur dit : Maintenant au contraire, que celui qui a une bourse la prenne, et de même celui qui a un sac ; et que celui qui n'a point d'épée vende son vêtement, et achète une épée.
\VS{37}Car je vous le dis, il faut que cette parole qui est écrite s'accomplisse en moi : Il a été mis au nombre des malfaiteurs\FTNT{Es. 53:12.}. Parce qu'en effet, ce qui me concerne est sur le point d'arriver.
\VS{38}Ils dirent : Seigneur, voici ici deux épées. Et il leur dit : Cela suffit.
\TextTitle{Gethsémané\FTNTT{Mt. 26:36-46 ; Mc. 14:32-42 ; Jn. 18:1 ; cp. Hé. 5:7-8}}
\VS{39}Après être sorti, il alla, selon sa coutume, au Mont des Oliviers ; et ses disciples le suivirent.
\VS{40}Lorsqu'ils arrivèrent dans ce lieu, il leur dit : Priez afin que vous ne tombiez pas en tentation.
\VS{41}Puis s'étant éloigné d'eux à la distance d'environ un jet de pierre, et s'étant mis à genoux, il pria,
\VS{42}disant : Père, si tu voulais éloigner cette coupe loin de moi ; toutefois que ma volonté ne soit point faite, mais la tienne.
\VS{43}Et un ange lui apparut du ciel, pour le fortifier.
\VS{44}Etant en agonie, il priait plus instamment, et sa sueur devint comme des grumeaux de sang qui tombaient à terre.
\VS{45}Après avoir prié, il revint vers ses disciples, qu'il trouva endormis de tristesse ;
\VS{46}et il leur dit : Pourquoi dormez-vous ? Levez-vous, et priez, afin que vous ne tombiez pas en tentation.
\TextTitle{Trahison de Judas\FTNTT{Mt. 26:47-54 ; Mc. 14:43-47 ; Jn. 18:2-11}}
\VS{47}Et comme il parlait encore, voici une foule arriva ; et celui qui s'appelait Judas, l'un des douze, marchait devant elle. Il s'approcha de Jésus pour l'embrasser.
\VS{48}Et Jésus lui dit : Judas, c'est par un baiser que tu trahis le Fils de l'homme ?
\VS{49}Alors ceux qui étaient autour de lui, voyant ce qui allait arriver, lui dirent : Seigneur, frapperons-nous de l'épée ?
\VS{50}Et l'un d'eux frappa le serviteur du souverain sacrificateur, et lui emporta l'oreille droite.
\VS{51}Mais Jésus prenant la parole dit : Laissez-les faire jusqu'ici. Et, ayant touché son oreille, il le guérit.
\VS{52}Puis Jésus dit aux principaux sacrificateurs, aux chefs des gardes du temple, et aux anciens qui étaient venus contre lui : Etes-vous venus comme après un brigand avec des épées et des bâtons ?
\VS{53}J'étais tous les jours avec vous dans le temple, et vous n'avez pas mis la main sur moi. Mais c'est ici votre heure, et la puissance des ténèbres.
\TextTitle{Triple reniement de Pierre\FTNTT{Mt. 26:55-58,69-75 ; Mc. 14:48-54,66-72 ; Jn. 18:15-18,25-27}}
\VS{54}Après avoir saisi Jésus, ils l'emmenèrent, et le conduisirent dans la maison du souverain sacrificateur. Pierre suivait de loin.
\VS{55}Ils allumèrent du feu au milieu de la cour, et ils s'assirent ensemble. Pierre s'assit aussi parmi eux.
\VS{56}Une servante le voyant assis auprès du feu, fixa sur lui les regards, et dit : Celui-ci aussi était avec lui.
\VS{57}Mais il le nia, disant : Femme, je ne le connais point.
\VS{58}Peu après, un autre le voyant, dit : Tu es aussi de ces gens-là, mais Pierre dit : Ô homme ! Je n'en suis point.
\VS{59}Environ une heure plus tard, un autre affirmait et disait : Certainement celui-ci aussi était avec lui car il est Galiléen.
\VS{60}Pierre dit : Ô homme ! Je ne sais pas ce que tu dis. Au même instant, comme il parlait encore, le coq chanta.
\VS{61}Et le Seigneur, s'étant retourné, regarda Pierre. Et Pierre se souvint de la parole que le Seigneur lui avait dite : Avant que le coq chante, tu me renieras trois fois.
\VS{62}Alors Pierre étant sorti dehors, pleura amèrement.
\TextTitle{Jésus est outragé\FTNTT{Mt. 26:67-68 ; Mc. 14:65 ; Jn. 18:22-23}}
\VS{63}Les hommes qui tenaient Jésus se moquaient de lui, et le frappaient.
\VS{64}Ils lui bandèrent les yeux, ils lui donnaient des coups sur le visage, et l'interrogeaient, disant : Devine qui est celui qui t'a frappé ?
\VS{65}Et ils proféraient contre lui beaucoup d'autres injures.
\TextTitle{Jésus déclare qu'il est fils de Dieu\FTNTT{Mt. 26:59-68 ; 27:1 ; Mc. 14:55-65 ; 15:1 ; Jn. 18:19-24}}
\VS{66}Quand le jour fut venu, les anciens du peuple, les principaux sacrificateurs, et les scribes, s'assemblèrent, et firent amener Jésus dans le sanhédrin.
\VS{67}Ils dirent : Si tu es le Christ, dis-le-nous. Et il leur répondit : Si je vous le dis, vous ne le croirez point ;
\VS{68}et si je vous interroge, vous ne me répondrez pas, et vous ne me laisserez pas aller.
\VS{69}Désormais le Fils de l'homme sera assis à la droite de la puissance de Dieu.
\VS{70}Alors ils dirent tous : Tu es donc le Fils de Dieu ? Et il leur répondit : Vous le dites vous-mêmes, je le suis.
\VS{71}Alors ils dirent : Qu'avons-nous besoin encore de témoignage ? Nous l'avons entendu nous-mêmes de sa bouche.
\Chap{23}
\TextTitle{Jésus devant Pilate\FTNTT{Mt. 27:2,11-14 ; Mc. 15:1-5 ; Jn. 18:28-38}}
\VerseOne{}Puis ils se levèrent tous, et ils conduisirent Jésus devant Pilate.
\VS{2}Et ils se mirent à l'accuser, disant : Nous avons trouvé cet homme excitant notre nation à la révolte, et empêchant de payer le tribut à César, et se disant lui-même Christ, Roi.
\VS{3}Pilate l'interrogea, disant : Es-tu le Roi des Juifs ? Et Jésus lui répondit : Tu le dis.
\VS{4}Alors Pilate dit aux principaux sacrificateurs et à la foule : Je ne trouve aucun crime en cet homme.
\VS{5}Mais ils insistèrent, et dirent : Il soulève le peuple, enseignant par toute la Judée, depuis la Galilée où il a commencé, jusqu'ici.
\TextTitle{Jésus envoyé devant Hérode par Pilate}
\VS{6}Quand Pilate entendit parler de la Galilée, il demanda si cet homme était Galiléen,
\VS{7}et, ayant appris qu'il était de la juridiction d'Hérode, il le renvoya à Hérode, qui se trouvait aussi à Jérusalem.
\VS{8}Lorsque Hérode vit Jésus, il en eut une grande joie ; car depuis longtemps il désirait le voir, à cause de ce qu'il avait entendu dire de lui, et il espérait qu'il le verrait faire quelque miracle.
\VS{9}Il lui adressa beaucoup de questions ; mais Jésus ne lui répondit rien.
\VS{10}Les principaux sacrificateurs et les scribes étaient là, et l'accusaient avec violence.
\VS{11}Mais Hérode, avec ses gardes, le traita avec mépris ; et, après s'être moqué de lui et l'avoir revêtu d'un vêtement éclatant, il le renvoya à Pilate.
\VS{12}Ce même jour, Pilate et Hérode devinrent amis ; car auparavant ils étaient ennemis.
\TextTitle{Hérode renvoie Jésus à Pilate\FTNTT{Mt. 27:15-26 ; Mc. 15:6-15 ; Jn. 18:39-19:15}}
\VS{13}Pilate, ayant assemblé les principaux sacrificateurs, les magistrats, et le peuple, leur dit :
\VS{14}Vous m'avez présenté cet homme comme soulevant le peuple. Et voici, je l'ai interrogé devant vous, et je ne l'ai trouvé coupable d'aucun des crimes dont vous l'accusez.
\VS{15}Hérode non plus ; car il nous l'a renvoyé, et voici, cet homme n'a rien fait qui soit digne de mort.
\VS{16}Je le relâcherai donc, après l'avoir châtié.
\VS{17}A chaque fête, il était obligé de leur relâcher un prisonnier.
\VS{18}Toutes les foules s'écrièrent ensemble, disant : Ôte celui-ci, et relâche-nous Barabbas.
\VS{19}Cet homme avait été mis en prison pour une sédition qui avait eu lieu dans la ville, et pour un meurtre.
\VS{20}Pilate leur parla de nouveau, ayant envie de relâcher Jésus.
\VS{21}Et ils crièrent : Crucifie, crucifie-le !
\VS{22}Pilate leur dit pour la troisième fois : Mais quel mal a fait cet homme ? Je ne trouve rien en lui qui soit digne de mort. Après l'avoir fait battre de verges, je le relâcherai.
\VS{23}Mais ils insistèrent à grands cris, demandant qu'il soit crucifié ; et leurs cris et ceux des principaux sacrificateurs l'emportèrent.
\VS{24}Alors Pilate prononça que ce qu'ils demandaient, serait fait.
\VS{25}Il leur relâcha celui qui avait été mis en prison pour sédition et pour meurtre, et qu'ils demandaient ; et il abandonna Jésus à leur volonté.
\TextTitle{Sur le chemin de Golgotha\FTNTT{Mt. 27:31-32 ; Mc. 15:20-21 ; Jn. 19:16-17}}
\VS{26}Comme ils l'emmenaient, ils prirent un certain Simon, de Cyrène, qui revenait des champs, et le chargèrent de la croix pour qu'il la porte derrière Jésus.
\VS{27}Il était suivi d'une grande multitude des gens du peuple et de femmes, qui se frappaient la poitrine, et se lamentaient sur lui.
\VS{28}Mais Jésus se tourna vers elles, leur dit : Filles de Jérusalem, ne pleurez point sur moi, mais pleurez sur vous-mêmes, et sur vos enfants.
\VS{29}Car voici, des jours viendront où l'on dira : Heureuses les stériles, les entrailles qui n'ont point enfanté, et les mamelles qui n'ont point allaité !
\VS{30}Alors ils se mettront à dire aux montagnes : Tombez sur nous ; et aux collines : Couvrez-nous !
\VS{31}Car s'ils font ces choses au bois vert, que sera-t-il fait au bois sec ?
\VS{32}On conduisait en même temps deux malfaiteurs, qui devaient être mis à mort avec Jésus.
\TextTitle{Crucifixion de Jésus\FTNTT{Mt. 27:33-43 ; Mc. 15:24-32 ; Jn. 19:17-37}}
\VS{33}Lorsqu'ils furent arrivés au lieu qui est appelé Calvaire (le Crâne), ils le crucifièrent là, et les malfaiteurs aussi, l'un à la droite, et l'autre à la gauche.
\VS{34}Jésus dit : Père, pardonne-leur, car ils ne savent pas ce qu'ils font. Ils se partagèrent ensuite ses vêtements, en tirant au sort.
\VS{35}Le peuple se tenait là, et regardait. Les magistrats se moquaient de Jésus disant : Il a sauvé les autres, qu'il se sauve lui-même, s'il est le Christ, l'élu de Dieu.
\VS{36}Les soldats aussi se moquaient de lui ; s'approchant et lui présentant du vinaigre,
\VS{37}ils disaient : Si tu es le Roi des Juifs, sauve-toi toi-même !
\VS{38}Or il y avait au-dessus de lui un écriteau en lettres Grecques, Romaines et Hébraïques, en ces mots : Celui-ci est le roi des juifs.
\TextTitle{Repentance du malfaiteur crucifié\FTNTT{cp. Mt. 27:44 ; Mc. 15:32}}
\VS{39}L'un des malfaiteurs qui étaient crucifiés, l'outrageait, disant : Si tu es le Christ, sauve-toi toi-même, et sauve-nous !
\VS{40}Mais l'autre le reprenait, et disait : Ne crains-tu pas Dieu, car tu es condamné au même supplice ?
\VS{41}Pour nous, c'est juste, car nous recevons ce qu'ont mérité nos crimes ; mais celui-ci n'a fait aucun mal.
\VS{42}Et il dit à Jésus : Seigneur ! Souviens-toi de moi quand tu viendras dans ton règne.
\VS{43}Jésus lui dit : Je te le dis en vérité, aujourd'hui tu seras avec moi dans le paradis.
\TextTitle{Jésus remet son esprit\FTNTT{Mt. 27:45-56 ; Mc. 15:33-41 ; Jn. 19:30-37}}
\VS{44}Il était déjà environ la sixième heure, et il eut des ténèbres sur toute la terre jusqu'à la neuvième heure.
\VS{45}Le soleil s'obscurcit, et le voile du temple se déchira par le milieu.
\VS{46}Et Jésus criant à haute voix, dit : Père, je remets mon esprit entre tes mains ! Et, en disant cela, il expira\FTNT{C'est la fin de la Première Alliance. Voir commentaire Jn. 19:30.}.
\TextTitle{Fin de la loi mosaïque ou de la Première Alliance}
\VS{47}Le centenier, voyant ce qui était arrivé, glorifia Dieu, et dit : Certes, cet homme était juste.
\VS{48}Et tous ceux qui assistaient en foule à ce spectacle, après avoir vu ce qui était arrivé, s'en retournèrent, se frappant la poitrine.
\VS{49}Tous ceux qui connaissaient Jésus, et les femmes qui l'avaient suivi de Galilée, se tenaient dans l'éloignement et regardaient ces choses.
\TextTitle{Sépulture de Jésus\FTNTT{Mt. 27:57-61 ; Mc. 15:42-47 ; Jn. 19:38-42}}
\VS{50}Il y avait un conseiller, nommé Joseph, homme bon et juste,
\VS{51}qui n'avait point participé au conseil et aux actes des autres ; il était d'Arimathée, ville des Juifs, et il attendait le Royaume de Dieu.
\VS{52}Cet homme se rendit vers Pilate et lui demanda le corps de Jésus.
\VS{53}Il le descendit de la croix, l'enveloppa d'un linceul, et le déposa dans un sépulcre taillé dans le roc, où personne n'avait encore été mis.
\VS{54}C'était le jour de la préparation, et le sabbat allait commencer.
\VS{55}Les femmes qui étaient venues de Galilée avec Jésus, accompagnèrent Joseph, virent le sépulcre, et la manière dont le corps de Jésus y fut déposé.
\VS{56}Et s'en étant retournées, elles préparèrent des aromates et des parfums ; et le jour du sabbat elles se reposèrent selon la loi.
\Chap{24}
\TextTitle{Résurrection du Messie\FTNTT{Mt. 28:1-15 ; Mc. 16:1-11 ; Jn. 20:1-18}}
\VerseOne{}Le premier jour de la semaine, elles se rendirent au sépulcre de grand matin, apportant les aromates qu'elles avaient préparés.
\VS{2}Elles trouvèrent la pierre roulée à côté du sépulcre.
\VS{3}Et, étant entrées, elles ne trouvèrent point le corps du Seigneur Jésus.
\VS{4}Comme elles ne savaient que penser de cela, voici, deux hommes leur apparurent en habits resplendissants.
\VS{5}Saisies de frayeur, elles baissèrent le visage contre terre, mais ils leur dirent : Pourquoi cherchez-vous parmi les morts celui qui est vivant ?
\VS{6}Il n'est point ici, mais il est ressuscité. Souvenez-vous comment il vous a parlé quand il était encore en Galilée,
\VS{7}et qu'il disait : Il faut que le Fils de l'homme soit livré entre les mains des pécheurs, et qu'il soit crucifié, et qu'il ressuscite le troisième jour.
\VS{8}Et elles se souvinrent de ses paroles.
\VS{9}A leur retour du sépulcre, elles annoncèrent toutes ces choses aux onze disciples, et à tous les autres.
\VS{10}Or c'étaient Marie de Magdala, Jeanne, Marie, mère de Jacques, et les autres qui étaient avec elles, qui dirent ces choses aux apôtres.
\VS{11}Mais les paroles de ces femmes leur semblèrent comme des paroles futiles, et ils ne les crurent point.
\VS{12}Mais Pierre s'étant levé, courut au sépulcre et s'étant courbé pour regarder, il ne vit que les linges là tout seuls, puis il s'en alla chez lui, dans l'étonnement de ce qui était arrivé.
\TextTitle{Jésus et les deux disciples sur le chemin d'Emmaüs\FTNTT{Mc. 16:12-13}}
\VS{13}Or voici, deux d'entre eux étaient ce jour-là en chemin, pour aller à un village nommée Emmaüs, éloigné de Jérusalem de soixante stades.
\VS{14}Et ils s'entretenaient ensemble de toutes ces choses qui étaient arrivées.
\VS{15}Et il arriva que, comme ils s'entretenaient et discutaient entre eux, Jésus lui-même s'approcha et se mit à marcher avec eux.
\VS{16}Mais leurs yeux étaient retenus de sorte qu'ils ne le reconnaissaient pas.
\VS{17}Et il leur dit : Quels sont ces discours que vous tenez ensemble en marchant ? Et pourquoi êtes-vous tout tristes ?
\VS{18}Et l'un d'eux, nommé Cléopas, lui répondit, et lui dit : Es-tu le seul étranger dans Jérusalem qui ne sache point les choses qui s'y sont passées ces jours-ci ?
\VS{19}Et il leur dit : Quelles ? Ils répondirent : Celles concernant Jésus de Nazareth, qui était un prophète puissant en œuvres et en paroles devant Dieu, et devant tout le peuple.
\VS{20}Et comment les principaux sacrificateurs et nos magistrats l'ont livré pour être condamné à mort, et l'ont crucifié.
\VS{21}Or nous espérions que ce serait lui qui délivrerait Israël ; mais avec tout cela, c'est aujourd'hui le troisième jour que ces choses sont arrivées.
\VS{22}Toutefois quelques femmes d'entre nous nous ont fort étonnés, car elles ont été de grand matin au sépulcre
\VS{23}et n'ayant point trouvé son corps, elles sont venues dire que même elles avaient vu une apparition d'anges, qui disaient qu'il est vivant.
\VS{24}Et quelques-uns des nôtres sont allés au sépulcre, et ont trouvé les choses comme les femmes l'avaient dit ; mais lui, ils ne l'ont point vu.
\VS{25}Alors Jésus leur dit : Ô gens sans intelligence, et dont le cœur est lent à croire tout ce que les prophètes ont annoncé !
\VS{26}Ne fallait-il pas que le Christ souffrît ces choses, et qu'il entra dans sa gloire ?
\VS{27}Puis commençant par Moïse, et continuant par tous les prophètes, il leur expliquait dans toutes les Ecritures ce qui le concernait.
\VS{28}Et comme ils furent près du village où ils allaient, il faisait comme s'il voulait aller plus loin.
\VS{29}Mais ils le forcèrent, en lui disant : Reste avec nous, car le soir approche et le jour commence à baisser. Et il entra donc pour rester avec eux.
\VS{30}Et il arriva comme il était à table avec eux, il prit le pain, et il le bénit ; et l'ayant rompu, il leur distribua.
\VS{31}Alors leurs yeux s'ouvrirent, et ils le reconnurent ; mais il disparut de devant eux.
\VS{32}Et ils se dirent l'un à l'autre : Notre cœur ne brûlait-il pas au-dedans de nous lorsqu'il nous parlait en chemin, et qu'il nous ouvrait les Ecritures ?
\TextTitle{Nouvelles apparitions du réssuscité\FTNTT{Mc. 16:14 ; Jn. 20:19-25 ; cp. Jn. 20:26-21:25}}
\VS{33}Et se levant à l'heure même, ils retournèrent à Jérusalem, et ils trouvèrent assemblés les onze et ceux qui étaient avec eux,
\VS{34}qui disaient : Le Seigneur est véritablement ressuscité, et il est apparu à Simon.
\VS{35}A leur tour, ils racontèrent ce qui leur était arrivé en chemin, et comment il avait été reconnu d'eux en rompant le pain.
\VS{36}Comme ils tenaient ces discours, Jésus se présenta lui-même au milieu d'eux, et leur dit : Que la paix soit avec vous !
\VS{37}Mais eux tout terrifiés et effrayés croyaient voir un esprit.
\VS{38}Et il leur dit : Pourquoi êtes-vous troublés, et pourquoi monte-t-il des pensées dans vos coeurs ?
\VS{39}Voyez mes mains et mes pieds, c'est bien moi. Touchez-moi, et voyez : Car un esprit n'a ni chair ni os, comme vous voyez que j'ai.
\VS{40}Et en disant cela, il leur montra ses mains et ses pieds.
\VS{41}Mais comme de joie, ils ne croyaient point encore, et qu'ils s'étonnaient, il leur dit : Avez-vous ici quelque chose à manger ?
\VS{42}Et ils lui présentèrent un morceau de poisson rôti, et un rayon de miel.
\VS{43}Et l'ayant pris, il mangea devant eux.
\TextTitle{La nouvelle mission des onze\FTNTT{Mt. 28:18-20 ; Mc. 16:15-18 ; Jn. 17:18 ; 20:21 ; Ac. 1:8}}
\VS{44}Puis il leur dit : Ce sont ici les paroles que je vous disais lorsque j'étais encore avec vous, qu'il fallait que s'accomplisse tout ce qui est écrit de moi dans la loi de Moïse, dans les prophètes, et dans les psaumes.
\VS{45}Alors il leur ouvrit l'esprit\FTNT{Pour comprendre les Ecritures, nous avons besoin de l'aide de l'Esprit de Dieu. La vraie connaissance ne vient pas des hommes, mais de Dieu (Da. 9:22).} afin qu'ils comprennent les Ecritures.
\VS{46}Et il leur dit : Il est ainsi écrit, et ainsi il fallait que le Christ souffre, et qu'il ressuscite des morts le troisième jour,
\VS{47}et que la repentance et le pardon des péchés seraient prêchés en son Nom à toutes les nations, à commencer par Jérusalem\FTNT{Es. 53.}.
\VS{48}Et vous êtes témoins de ces choses. 
\VS{49}Et voici, j'enverrai sur vous la promesse de mon Père, mais vous donc restez dans la ville de Jérusalem, jusqu'à ce que vous soyez revêtus de la puissance d'en haut.
\TextTitle{Jésus enlevé au ciel\FTNTT{Mc. 16:19-20 ; Ac. 1:9-11}}
\VS{50}Après quoi il les conduisit dehors jusqu'en Béthanie, et levant ses mains en haut, il les bénit.
\VS{51}Et pendant qu'il les bénissait, il se sépara d'eux, et fut élevé au ciel.
\VS{52}Pour eux, après l'avoir adoré, ils retournèrent à Jérusalem avec une grande joie.
\VS{53}Et ils étaient toujours dans le temple, louant et bénissant Dieu. Amen !
\PPE{}
\end{multicols}

%\clearpage\ShortTitle{Jean}\BookTitle{Jean}\BFont
\noindent\hrulefill
\textit{
\bigskip
{\centering{}
\\Signifie : Dieu pardonne, don de Dieu
\\Thème : Christ, Dieu
\\Auteur : Jean
\\Date de rédaction : Env. 85-90 apr. J.-C.\\}
}
%\bigskip
\textit{
\\Auteur d’un des quatre évangiles, des trois épîtres éponymes et de l’Apocalypse, Jean, fils de Zébédée, fut l’un des douze. Témoin oculaire du ministère terrestre de Jésus-Christ, il attesta par l’essence de ses écrits le caractère divin de ce dernier.
\bigskip
\\Fidèle au livre d’Exode où Yahweh se révéla comme étant « Je suis », Jean reprit les propos de Jésus et le présenta comme la Parole incarnée, le Pain de vie, la Lumière du monde, la Porte des brebis, le Bon berger, la Résurrection, la Vie… Proche du maître, Jean fut à même de relater les évènements marquants de sa vie comme la gloire de la Transfiguration, l’angoisse de la passion exprimée à Gethsémané, ou encore les déclarations solennelles précédées de l’expression «  En vérité, en vérité »… Il mit également en évidence la controverse suscitée par le Christ et l’opposition dont il fit l’objet de la part de certains pharisiens qui souhaitaient sa mort.
\bigskip
\\L’évangile de Jean exprime la nécessité de la nouvelle naissance et dévoile les attributs du Fils de Dieu, le Messie tant attendu.\bigskip
}
\par\nobreak\noindent\hrulefill
\begin{multicols}{2}
\TextTitle{[La divinité de Jésus-Christ]
\\(Jn. 10:30 ; Hé. 1:5-13)}
\Chap{1}
\VerseOne{}Au commencement était la Parole, et la Parole était avec Dieu, et la Parole était Dieu.
\VS{2}Elle était au commencement avec Dieu.
\TextTitle{[L'oeuvre de Jésus avant son incarnation]}
\VS{3}Toutes choses ont été faites par elle, et rien de ce qui a été fait, n'a été fait sans elle.
\VS{4}En elle était la vie, et la vie était la Lumière des hommes\FTNT{Jésus-Christ notre Lumière : Es. 60:19-20.}.
\VS{5}Et la Lumière luit dans les ténèbres, mais les ténèbres ne l'ont point reçue.
\TextTitle{[Ministère de Jean-Baptiste]}
\VS{6}Il y eut un homme appelé Jean, qui fut envoyé de Dieu.
\VS{7}Il vint pour servir de témoin, pour rendre témoignage à la Lumière, afin que tous croient par lui.
\VS{8}Il n'était pas la Lumière, mais il était envoyé pour rendre témoignage à la Lumière.
\TextTitle{[Jésus-Christ, la véritable lumière]
\\(Jn. 3:17-21 ; 8:12 ; 9:5 ; 12:46)}
\VS{9}Cette Lumière était la véritable Lumière, qui en venant dans le monde éclaire tout homme.
\VS{10}Elle était dans le monde, et le monde a été fait par elle ; mais le monde ne l'a point connue.
\VS{11}Elle est venue chez les siens ; et les siens ne l'ont point reçue.
\VS{12}Mais à tous ceux qui l'ont reçue, à ceux qui croient en son Nom, elle leur a donné le pouvoir de devenir enfants de Dieu.
\VS{13}Lesquels sont nés, non du sang, ni de la volonté de la chair, ni de la volonté de l'homme ; mais ils sont nés de Dieu.
\TextTitle{[La Parole faite chair]
\\(Jn. 14:9 ; Mt. 1:18-23 ; Lu. 1:30-35 ; 2:11 ; 1Tim. 3:16)}
\VS{14}Et la Parole a été faite chair, elle a habité parmi nous, pleine de grâce et de vérité, et nous avons contemplé sa gloire, une gloire, comme la gloire du Fils unique du Père.
\TextTitle{[Premier témoignage de Jean-Baptiste]
\\(Mt. 3:1-12 ; Mc. 1:1-11 ; Lu. 3:1-22)}
\VS{15}Jean a donc rendu témoignage de lui, et s’est écrié, disant : C'est celui dont j’ai dit : Celui qui vient après moi m’a précédé, car il était avant moi.
\VS{16}Et nous avons tous reçu de sa plénitude, et grâce pour grâce.
\VS{17}Car la loi\FTNT{La loi a été promulguée par Moïse.} a été donnée par Moïse, la grâce et la vérité sont venues par Jésus-Christ.
\VS{18}Personne n’a jamais vu Dieu, le Fils unique qui est dans le sein du Père, est celui qui nous l'a révélé.
\VS{19}Et c'est ici le témoignage de Jean, lorsque les Juifs envoyèrent de Jérusalem des sacrificateurs et des lévites pour l'interroger, et lui dire : Toi qui es-tu ?
\VS{20}Il confessa, et ne le nia point, il déclara, en disant : Ce n'est pas moi qui suis le Christ.
\VS{21}Et ils lui demandèrent : Quoi donc ? Es-tu Elie ? Et il dit : Je ne le suis point\FTNT{En Mt. 11:14 Jésus confirme pourtant que Jean-Baptiste est bien l’Elie qui devait venir. Comment expliquer qu’il nia l’être lorsqu’il fut interrogé par les pharisiens ? La seule explication plausible c’est qu’il l’ignorait. Toutefois, il avait conscience qu’il était ~la voix~ prophétisée par Esaïe. Remarquez que lorsqu’il fut emprisonné, il avait envoyé quelques-uns de ses disciples pour demander à Jésus s’il était bien le Messie (Mt. 11:13 ; Lu. 7:19-20) alors qu’il fut le premier à rendre témoignage du Seigneur. Ces éléments ne sont pas contradictoires, ils ne font que révéler les failles liées à la nature humaine de Jean.}. Es-tu le Prophète ? Et il répondit : Non.
\VS{22}Ils lui dirent donc : Qui es-tu, afin que nous donnions une réponse à ceux qui nous ont envoyés. Que dis-tu de toi-même ?
\VS{23}Il dit : Je suis la voix de celui qui crie dans le désert : Aplanissez le chemin du Seigneur, comme a dit Esaïe le prophète\FTNT{Es. 40:3.}.
\VS{24}Or ceux qui avaient été envoyés vers lui étaient des pharisiens.
\VS{25}Ils l'interrogèrent encore, et lui dirent : Pourquoi donc baptises-tu si tu n'es point le Christ, ni Elie, ni le Prophète ?
\VS{26}Jean leur répondit : Pour moi, je baptise d'eau ; mais il y a quelqu’un au milieu de vous que vous ne connaissez point.
\VS{27}C'est celui qui vient après moi, il m’a précédé, et je ne suis pas digne de délier la courroie de ses souliers.
\VS{28}Ces choses se passèrent à Béthanie, au-delà du Jourdain, où Jean baptisait.
\VS{29}Le lendemain Jean vit Jésus venir à lui, et il dit : Voici l'Agneau de Dieu, qui ôte le péché du monde.
\VS{30}C'est celui dont j’ai dit : Après moi vient un homme qui m’a précédé ; car il était avant moi.
\VS{31}Et pour moi, je ne le connaissais pas ; mais c'est afin qu'il soit manifesté à Israël que je suis venu baptiser d'eau.
\VS{32}Jean rendit aussi témoignage, en disant : J'ai vu l'Esprit descendre du ciel comme une colombe, et s'arrêter sur lui.
\VS{33}Et pour moi, je ne le connaissais point ; mais celui qui m'a envoyé baptiser d'eau, m'avait dit : Celui sur qui tu verras l'Esprit descendre et s’arrêter, c'est celui qui baptise du Saint-Esprit.
\VS{34}Et je l'ai vu, et j'ai rendu témoignage, que c'est lui qui est le Fils de Dieu.
\TextTitle{[Premiers disciples de Jésus-Christ]
\\(Mt. 4:18-22 ; Mc. 1:16-20 ; Lu. 5:1-11)}
\VS{35}Le lendemain Jean était encore là, avec deux de ses disciples ;
\VS{36}et regardant Jésus qui marchait, il dit : Voici l'Agneau de Dieu.
\VS{37}Les deux disciples l'entendirent prononcer ces paroles, et ils suivirent Jésus.
\VS{38}Et Jésus se retournant, et voyant qu'ils le suivaient, il leur dit : Que cherchez-vous ? Ils lui répondirent : Rabbi, c'est-à-dire Maître, où demeures-tu ?
\VS{39}Il leur dit : Venez, et voyez. Ils y allèrent, et ils virent où il demeurait ; et ils demeurèrent avec lui ce jour-là ; car il était environ dix heures.
\VS{40}André, frère de Simon Pierre, était l'un des deux qui avaient entendu les paroles de Jean et qui avaient suivi Jésus.
\VS{41}Ce fut lui qui rencontra le premier Simon son frère, et il lui dit : Nous avons trouvé le Messie, c'est-à-dire le Christ.
\VS{42}Et il le conduisit vers Jésus, et Jésus l’ayant regardé, dit : Tu es Simon, fils de Jonas, tu seras appelé Céphas ; c'est-à-dire, Pierre.
\VS{43}Le lendemain Jésus voulut se rendre en Galilée, et il trouva Philippe. Et il lui dit : Suis-moi.
\VS{44}Philippe était de Bethsaïda, la ville d'André et de Pierre.
\VS{45}Philippe rencontra Nathanaël, et lui dit : Nous avons trouvé celui de qui Moïse a écrit dans la loi, et dont les prophètes ont parlé, Jésus, qui est de Nazareth, fils de Joseph.
\VS{46}Et Nathanaël lui dit : Peut-il venir quelque chose de bon de Nazareth ? Philippe lui dit : Viens, et vois.
\VS{47}Jésus aperçut Nathanaël venir vers lui, et il dit de lui : Voici vraiment un Israëlite dans lequel il n'y a point de fraude.
\VS{48}Nathanaël lui dit : D'où me connais-tu ? Jésus répondit et lui dit : Avant que Philippe t’appelle, quand tu étais sous le figuier, je t’ai vu.
\VS{49}Nathanaël répondit et lui dit : Maître, tu es le Fils de Dieu. Tu es le Roi d'Israël.
\VS{50}Jésus lui répondit et dit : Parce que je t'ai dit que je t’ai vu sous le figuier, tu crois. Tu verras des choses plus grandes encore.
\VS{51}Il lui dit aussi : En vérité, en vérité je vous dis : Désormais vous verrez le ciel ouvert, et les anges de Dieu monter et descendre sur le Fils de l'homme.
\TextTitle{[Premier miracle, à Cana]}
\Chap{2}
\VerseOne{}Trois jours après, il y eut des noces à Cana en Galilée, et la mère de Jésus était là.
\VS{2}Et Jésus fut aussi convié aux noces avec ses disciples.
\VS{3}Et le vin ayant manqué, la mère de Jésus lui dit : Ils n'ont plus de vin.
\VS{4}Jésus lui répondit : Qu'y a-t-il entre moi et toi, femme ? Mon heure n'est point encore venue.
\VS{5}Sa mère dit aux serviteurs : Faites tout ce qu'il vous dira.
\VS{6}Or il y avait là six vases de pierre, destinés aux purifications des Juifs, et contenant chacun deux ou trois mesures.
\VS{7}Et Jésus leur dit : Remplissez d'eau ces vases. Et ils les remplirent jusqu’au bord.
\VS{8}Puis il leur dit : Puisez maintenant, et apportez-en au maître d'hôtel. Et ils lui en apportèrent.
\VS{9}Quand le maître d'hôtel eut goûté l'eau changée en vin, ne sachant d'où venait ce vin, tandis que les serviteurs qui avaient puisé l'eau le savaient bien, il s'adressa à l'époux
\VS{10}et lui dit : Tout homme sert d’abord le bon vin, et ensuite le moins bon, après qu’on s’est enivré ; mais toi, tu as gardé le bon vin jusqu'à maintenant.
\VS{11}Jésus fit ce premier miracle à Cana en Galilée, et il manifesta sa gloire, et ses disciples crurent en lui.
\VS{12}Après cela, il descendit à Capernaüm avec sa mère, et ses frères, et ses disciples ; mais ils y demeurèrent peu de jours.
\TextTitle{[La première Pâque]
\\(Jn. 6:4 ; 11:55)}
\VS{13}La Pâque des Juifs était proche ; c'est pourquoi Jésus monta à Jérusalem.
\VS{14}Et il trouva dans le temple des vendeurs de bœufs, de brebis, et de pigeons ; et les changeurs qui y étaient assis.
\VS{15}Et ayant fait un fouet avec des petites cordes, il les chassa tous du temple, avec les brebis, et les bœufs ; et il dispersa la monnaie des changeurs, et renversa les tables.
\VS{16}Et il dit aux vendeurs des pigeons : Otez ces choses d'ici, et ne faites pas de la maison de mon Père une maison de marché.
\VS{17}Alors ses disciples se souvinrent qu'il était écrit : Le zèle de ta maison me dévore\FTNT{Ps. 69:10.}.
\VS{18}Mais les Juifs prenant la parole, lui dirent : Quel signe nous montres-tu pour agir de la sorte ?
\VS{19}Jésus répondit et leur dit : Détruisez ce temple, et en trois jours je le relèverai.
\VS{20}Et les Juifs dirent : Il a fallu quarante-six ans pour bâtir ce temple, et toi, tu le relèveras en trois jours !
\VS{21}Mais il parlait du temple de son corps.
\VS{22}C'est pourquoi lorsqu'il fut ressuscité des morts, ses disciples se souvinrent qu'il leur avait dit cela, et ils crurent à l'Ecriture et à la parole que Jésus avait dite.
\VS{23}Et comme il était à Jérusalem le jour de la fête de Pâque, plusieurs crurent en son Nom, voyant les miracles qu'il faisait.
\VS{24}Mais Jésus ne se fiait point à eux, parce qu'il les connaissait tous ;
\VS{25}et parce qu'il n'avait pas besoin qu’on lui rende témoignage d'aucun homme ; car il savait lui-même ce qui était dans l'homme.
\TextTitle{[Jésus et Nicodème : la naissance d'en haut]}
\Chap{3}
\VerseOne{}Mais il y eut un homme d'entre les pharisiens, nommé Nicodème, qui était un des chefs des Juifs,
\VS{2}qui vint de nuit auprès de Jésus, et lui dit : Rabbi, nous savons que tu es un Docteur venu de Dieu, car personne ne peut faire les miracles que tu fais, si Dieu n'est avec lui.
\VS{3}Jésus lui répondit et dit : En vérité, en vérité je te le dis : Si quelqu'un ne naît d’en haut\FTNT{Naître d’en haut : Dans la plupart des Bibles modernes, on trouve l’expression naître « de nouveau », or cette traduction n’est pas correcte puisque le texte grec utilise l'expression naître « d’en haut ». L’adverbe « d’en haut » vient du mot grec « ahothen » qui signifie : depuis le haut, depuis un endroit plus élevé, ce qui vient des cieux ou de Dieu, depuis le début, l'origine. Ce mot se retrouve dans Mt. 27:51 ; Mc. 15:38 ; Lu. 1:3 ; Jn. 3:31 ; Jn. 19:11 ; Jn. 19:23 ; Ja. 1:17 ; Ja. 3:15 ; Ja. 3:17. ~Anothen~ vient de ~ano~ : choses d’en haut. En Ga. 4:26 ~ano~ peut se référer au lieu ou au temps. Le lieu : La Jérusalem qui est au-dessus, dans les cieux. Le temps : La Jérusalem éternelle qui a précédé la terrestre. Le mot ~ano~ a été traduit par ~en haut~ dans Jn. 8:23 ; Jn. 11:41 ; Ac. 2:19 ; Ga. 4:26 ; Col. 3:1-2 ; et par ~céleste~ dans Ph. 3:14. Jésus nous enseigne donc que la nouvelle naissance est en réalité la naissance d’en haut, une naissance qui a eu lieu dans la Nouvelle Jérusalem.}, il ne peut voir le Royaume de Dieu.
\VS{4}Nicodème lui dit : Comment un homme peut-il naître quand il est vieux ? Peut-il rentrer dans le sein de sa mère et naître une seconde fois ?
\VS{5}Jésus répondit : En vérité, en vérité je te dis : Si quelqu’un ne naît d'eau et d'Esprit, il ne peut entrer dans le Royaume de Dieu.
\VS{6}Ce qui est né de la chair est chair ; et ce qui est né de l'Esprit est esprit.
\VS{7}Ne t'étonne pas de ce que je t'ai dit : Il faut que vous naissiez d’en haut.
\VS{8}Le vent souffle où il veut, et tu en entends le bruit ; mais tu ne sais pas d'où il vient ni où il va : Il en est ainsi de tout homme qui est né de l'Esprit.
\VS{9}Nicodème lui dit : Comment cela peut-il se faire ?
\VS{10}Jésus répondit et lui dit : Tu es le docteur d'Israël, et tu ne connais point ces choses !
\VS{11}En vérité, en vérité je te le dis, nous disons ce que nous savons, et nous rendons témoignage de ce que nous avons vu ; et vous ne recevez pas notre témoignage.
\VS{12}Si vous ne croyez pas quand je vous ai parlé des choses terrestres, comment croirez-vous quand je vous parlerai des choses célestes ?
\VS{13}Personne n'est monté au ciel, si ce n’est celui qui est descendu du ciel, le Fils de l'homme qui est dans le ciel.
\VS{14}Et comme Moïse éleva le serpent\FTNT{Le serpent d’airain : No. 21:9}dans le désert, il faut de même que le Fils de l'homme soit élevé,
\VS{15}afin que quiconque croit en lui ne périsse point, mais qu'il ait la vie éternelle.
\VS{16}Car Dieu a tant aimé le monde, qu'il a donné son Fils unique, afin que quiconque croit en lui ne périsse point, mais qu'il ait la vie éternelle.
\VS{17}Car Dieu n'a point envoyé son Fils dans le monde pour condamner le monde, mais afin que le monde soit sauvé par lui.
\VS{18}Celui qui croit en lui ne sera point jugé ; mais celui qui ne croit point est déjà jugé ; parce qu'il n'a point cru au Nom du Fils unique de Dieu.
\VS{19}Et ce jugement c’est que la lumière est venue dans le monde et que les hommes ont préféré les ténèbres à la lumière, parce que leurs œuvres étaient mauvaises.
\VS{20}Car quiconque fait le mal, hait la lumière, et ne vient point à la lumière, de peur que ses œuvres ne soient condamnées.
\VS{21}Mais celui qui agit selon la vérité, vient à la lumière, afin que ses œuvres soient manifestées, parce qu'elles sont faites selon Dieu.
\TextTitle{[Nouveau témoignage de Jean-Baptiste]}
\VS{22}Après ces choses, Jésus s’en alla avec ses disciples dans la terre de Judée ; et là, il demeurait avec eux et il baptisait.
\VS{23}Jean aussi baptisait à Enon, près de Salim, parce qu'il y avait là beaucoup d'eau, et on y venait pour être baptisé.
\VS{24}Car Jean n'avait pas encore été mis en prison.
\VS{25}Or, il y eut une dispute entre les disciples de Jean et les Juifs touchant la purification.
\VS{26}Ils vinrent trouver Jean, et lui dirent : Maître, celui qui était avec toi au-delà du Jourdain, et à qui tu as rendu témoignage, voilà, il baptise, et tous vont à lui.
\VS{27}Jean répondit et dit : Un homme ne peut recevoir que ce qui lui a été donné du ciel.
\VS{28}Vous-mêmes m'êtes témoins que j'ai dit : Ce n'est pas moi qui suis le Christ, mais j’ai été envoyé devant lui.
\VS{29}Celui à qui appartient l'Epouse c’est l'Epoux ; mais l'ami de l'Epoux qui se tient là et qui l’entend, éprouve une grande joie à cause de la voix de l’Epoux ; c'est pourquoi cette joie qui est la mienne est parfaite.
\VS{30}Il faut qu'il croisse, et que je diminue.
\TextTitle{[Conclusion apportée par Jean]}
\VS{31}Celui qui vient d'en haut est au-dessus de tous ; celui qui est venu de la terre est de la terre, et il parle comme venant de la terre. Celui qui est venu du ciel est au-dessus de tous.
\VS{32}Et ce qu'il a vu et entendu, il le témoigne ; mais personne ne reçoit son témoignage.
\VS{33}Celui qui a reçu son témoignage a certifié que Dieu est véritable.
\VS{34}Car celui que Dieu a envoyé annonce les paroles de Dieu ; car Dieu ne lui donne point l'Esprit avec mesure.
\VS{35}Le Père aime le Fils, et il a remis toutes choses entre ses mains.
\VS{36}Celui qui croit au Fils a la vie éternelle, mais celui qui désobéit au Fils ne verra point la vie, mais la colère de Dieu demeure sur lui.
\TextTitle{[Jésus se rend en Galilée]}
\Chap{4}
\VerseOne{}Le Seigneur sut que les pharisiens avaient appris qu'il faisait et baptisait plus de disciples que Jean.
\VS{2}Toutefois Jésus ne baptisait point lui-même, mais c'étaient ses disciples.
\VS{3}Il quitta la Judée, et retourna encore en Galilée.
\TextTitle{[Jésus et la femme samaritaine]}
\VS{4}Comme il fallait qu'il passe par la Samarie,
\VS{5}il arriva dans une ville de Samarie nommée Sychar, près du champ que Jacob avait donné à Joseph son fils\FTNT{Ge. 48:22.}.
\VS{6}Or il y avait là le puits de Jacob ; et Jésus, fatigué du voyage, se tenait là, assis au bord du puits. C'était environ la sixième heure\FTNT{Sixième heure ou midi.}.
\VS{7}Une femme Samaritaine vint puiser de l'eau, Jésus lui dit : Donne-moi à boire.
\VS{8}Car ses disciples étaient allés à la ville pour acheter des vivres.
\VS{9}La femme Samaritaine lui dit : Comment toi qui es Juif, me demandes-tu à boire, à moi qui suis une femme Samaritaine ? Les Juifs, en effet, n’ont pas de relations avec les Samaritains.
\VS{10}Jésus lui répondit et dit : Si tu connaissais le don de Dieu, et qui est celui qui te dit : Donne-moi à boire, tu lui aurais toi-même demandé à boire, et il t’aurait donné de l'eau vive.
\VS{11}La femme lui dit : Seigneur, tu n'as rien pour puiser, et le puits est profond ; d'où aurais-tu donc cette eau vive ?
\VS{12}Es-tu plus grand que Jacob, notre père, qui nous a donné ce puits, et qui en a bu lui-même, ainsi que ses enfants et son bétail ?
\VS{13}Jésus répondit et lui dit : Quiconque boit de cette eau aura encore soif ;
\VS{14}mais celui qui boira de l'eau que je lui donnerai, n'aura jamais soif ; mais l'eau que je lui donnerai deviendra en lui une source d'eau qui jaillira jusque dans la vie éternelle.
\VS{15}La femme lui dit : Seigneur, donne-moi de cette eau, afin que je n'aie plus soif, et que je ne vienne plus ici puiser de l'eau.
\VS{16}Jésus lui dit : Va, appelle ton mari, et viens ici.
\VS{17}La femme lui répondit et dit : Je n'ai point de mari. Jésus lui dit : Tu as bien dit : Je n'ai point de mari.
\VS{18}Car tu as eu cinq maris, et celui que tu as maintenant n'est point ton mari ; en cela tu as dit la vérité.
\VS{19}La femme lui dit : Seigneur, je vois que tu es un prophète.
\VS{20}Nos pères ont adoré sur cette montagne\FTNT{Cette montagne, dont parle la Samaritaine, c'est le Mont Garizim (ou montagne de Sichem) sur lequel les samaritains construisirent leur temple et établirent leur culte, au temps de Néhémie.}, et vous, vous dites que le lieu où il faut adorer est à Jérusalem.
\VS{21}Jésus lui dit : Femme, crois-moi, l'heure vient où ce ne sera ni sur cette montagne ni à Jérusalem que vous adorerez le Père.
\VS{22}Vous adorez ce que vous ne connaissez pas ; nous, nous adorons ce que nous connaissons ; car le salut vient des Juifs.
\VS{23}Mais l'heure vient, et elle est déjà venue, où les vrais adorateurs adoreront le Père en esprit et en vérité ; car ce sont là les adorateurs que le Père demande.
\VS{24}Dieu est Esprit, et il faut que ceux qui l'adorent, l'adorent en esprit et en vérité.
\VS{25}La femme lui répondit : Je sais que le Messie, c'est-à-dire le Christ, doit venir ; quand il sera venu, il nous annoncera toutes choses.
\VS{26}Jésus lui dit : Je le suis, moi qui te parle.
\VS{27}Là-dessus arrivèrent ses disciples, et ils s'étonnèrent de ce qu'il parlait avec une femme. Toutefois aucun ne dit : Que demandes-tu ? Ou : Pourquoi parles-tu avec elle ?
\VS{28}La femme, ayant laissé sa cruche, s'en alla dans la ville, et elle dit aux habitants :
\VS{29}Venez voir un homme qui m'a dit tout ce que j'ai fait, ne serait-ce point le Christ ?
\VS{30}Ils sortirent donc de la ville, et vinrent vers lui.
\VS{31}Cependant les disciples le pressaient, disant : Maître, mange.
\VS{32}Mais il leur dit : J'ai à manger une nourriture que vous ne connaissez point.
\VS{33}Sur quoi les disciples se demandaient entre eux : Quelqu'un lui aurait-il apporté à manger ?
\VS{34}Jésus leur dit : Ma nourriture est de faire la volonté de celui qui m'a envoyé, et d’accomplir son œuvre.
\VS{35}Ne dites-vous pas qu'il y a encore quatre mois jusqu’à la moisson ? Voici, je vous dis, levez vos yeux, et regardez les champs qui déjà blanchissent pour la moisson.
\VS{36}Celui qui moissonne reçoit un salaire, et amasse des fruits pour la vie éternelle ; afin que celui qui sème et celui qui moissonne se réjouissent ensemble.
\VS{37}Car en ceci ce qu’on dit d’ordinaire est vrai : L’un sème et l'autre moissonne.
\VS{38}Je vous ai envoyés moissonner où vous n'avez point travaillé ; d'autres ont travaillé, et vous êtes entrés dans leur travail.
\TextTitle{[Jésus et les samaritains]}
\VS{39}Plusieurs Samaritains de cette ville crurent en lui, à cause de la parole de la femme qui avait rendu ce témoignage : Il m'a dit tout ce que j'ai fait.
\VS{40}Quand donc les Samaritains vinrent le trouver, ils le prièrent de demeurer avec eux ; et il demeura là deux jours.
\VS{41}Et beaucoup plus de gens crurent à cause de sa parole ;
\VS{42}et ils disaient à la femme : Ce n'est plus à cause de ta parole que nous croyons ; car nous l'avons entendu nous-mêmes, et nous savons qu’il est véritablement le Christ, le Sauveur du monde.
\VS{43}Après ces deux jours, Jésus partit de là, et s'en alla en Galilée.
\VS{44}Car il avait rendu témoignage qu'un prophète n'est pas honoré dans son pays.
\VS{45}Lorsqu’il arriva en Galilée, les Galiléens le reçurent, ayant vu toutes les choses qu'il avait faites à Jérusalem le jour de la Fête, car eux aussi étaient allés à la Fête.
\TextTitle{[Jésus guérit le fils d'un officier]}
\VS{46}Jésus retourna encore à Cana de Galilée, où il avait changé l'eau en vin. Or il y avait à Capernaüm un officier du roi, dont le fils était malade.
\VS{47}Ayant appris que Jésus était venu de Judée en Galilée, il alla vers lui, et le pria de descendre pour guérir son fils qui était près de mourir.
\VS{48}Mais Jésus lui dit : Si vous ne voyez pas des prodiges et des miracles, vous ne croyez point.
\VS{49}L’officier du roi lui dit : Seigneur, descends avant que mon fils meure.
\VS{50}Jésus lui dit : Va, ton fils vit. Cet homme crut à la parole que Jésus lui avait dite, et il s'en alla.
\VS{51}Et comme il descendait déjà, ses serviteurs vinrent au-devant de lui, et lui apportèrent des nouvelles, disant : Ton fils vit.
\VS{52}Et il leur demanda à quelle heure il s'était trouvé mieux ; et ils lui dirent : Hier, à la septième heure, la fièvre l’a quitté.
\VS{53}Le père reconnut que c'était à cette même heure-là que Jésus lui avait dit : Ton fils vit. Et il crut, avec toute sa maison.
\VS{54}Jésus fit encore ce second miracle quand il fut venu de Judée en Galilée.
\TextTitle{[Nouvelle fête des juifs et guérison d'un paralytique à la piscine de Béthesda]}
\Chap{5}
\VerseOne{}Après ces choses, il y eut une fête des Juifs, et Jésus monta à Jérusalem.
\VS{2}Or à Jérusalem, près de la porte des brebis, il y avait une piscine appelée en hébreu Béthesda, et qui avait cinq portiques.
\VS{3}Sous ces portiques étaient couchés un grand nombre de malades, des aveugles, des boiteux, des paralytiques, attendant le mouvement de l'eau.
\VS{4}Car un ange descendait de temps en temps dans la piscine, et agitait l'eau ; et alors le premier qui y descendait après que l'eau avait été agitée, était guéri, quelle que fût sa maladie.
\VS{5}Or il y avait là un homme malade depuis trente-huit ans.
\VS{6}Jésus, le voyant couché par terre, et sachant qu'il était déjà malade depuis longtemps, lui dit : Veux-tu être guéri ?
\VS{7}Le malade lui répondit : Seigneur, je n'ai personne pour me jeter dans la piscine quand l'eau est agitée, et pendant que j'y vais, un autre descend avant moi.
\VS{8}Jésus lui dit : Lève-toi, prends ton lit, et marche.
\VS{9}Et aussitôt cet homme fut guéri, il prit son lit, et marcha. Or c'était un jour de sabbat.
\VS{10}Les Juifs dirent donc à celui qui avait été guéri : C'est un jour de sabbat, il ne t'est pas permis de prendre ton lit.
\VS{11}Il leur répondit : Celui qui m'a guéri m'a dit : Prends ton lit et marche.
\VS{12}Alors ils lui demandèrent : Qui est celui qui t'a dit : Prends ton lit et marche ?
\VS{13}Mais celui qui avait été guéri ne savait pas qui c'était, car Jésus s'était éclipsé du milieu de la foule qui était en ce lieu-là.
\VS{14}Depuis, Jésus le trouva dans le temple, et lui dit : Voici, tu as été guéri ; ne pèche plus désormais, de peur qu’il ne t'arrive quelque chose de pire.
\VS{15}Cet homme s'en alla, et rapporta aux Juifs que c'était Jésus qui l'avait guéri.
\VS{16}C'est pourquoi les Juifs poursuivaient Jésus et cherchaient à le faire mourir, parce qu'il avait fait ces choses le jour du sabbat.
\TextTitle{[Jésus déclare son égalité avec le Père]}
\VS{17}Mais Jésus leur répondit : Mon Père agit jusqu'à présent ; moi aussi, j’agis.
\VS{18}A cause de cela, les Juifs cherchaient encore plus à le faire mourir, parce que non seulement il avait violé le sabbat, mais aussi parce qu'il disait que Dieu était son propre Père, se faisant égal à Dieu.
\VS{19}Mais Jésus répondit et leur dit : En vérité, en vérité je vous le dis, le Fils ne peut rien faire de lui-même, il ne fait que ce qu'il voit faire au Père ; et tout ce que le Père fait, le Fils le fait pareillement.
\VS{20}Car le Père aime le Fils, et lui montre toutes les choses qu'il fait ; et il lui montrera de plus grandes œuvres que celles-ci, afin que vous soyez dans l'admiration.
\VS{21}Car comme le Père ressuscite les morts et donne la vie, de même aussi le Fils donne la vie à ceux qu'il veut.
\VS{22}Car le Père ne juge personne ; mais il a donné tout jugement au Fils,
\VS{23}afin que tous honorent le Fils, comme ils honorent le Père ; celui qui n'honore point le Fils, n'honore point le Père qui l'a envoyé.
\VS{24}En vérité, en vérité je vous le dis, celui qui entend ma parole, et croit à celui qui m'a envoyé, a la vie éternelle et ne vient pas en jugement, mais il est passé de la mort à la vie.
\TextTitle{[Les deux résurrections]}
\VS{25}En vérité, en vérité je vous le dis, l'heure vient, et elle est déjà venue, où les morts entendront la voix du Fils de Dieu, et ceux qui l'auront entendue vivront.
\VS{26}Car comme le Père a la vie en lui-même, ainsi il a donné au Fils d'avoir la vie en lui-même.
\VS{27}Et il lui a donné le pouvoir de juger parce qu'il est le Fils de l'homme.
\VS{28}Ne soyez point étonnés de cela ; car l'heure vient où tous ceux qui sont dans les sépulcres entendront sa voix, et en sortiront.
\VS{29}Ceux qui auront fait le bien, ressusciteront pour la vie, mais ceux qui auront fait le mal, ressusciteront pour le jugement.
\TextTitle{[Témoignages confirmant celui de Jésus]}
\VS{30}Je ne puis rien faire de moi-même : Je juge conformément à ce que j'entends, et mon jugement est juste ; car je ne cherche point ma volonté, mais la volonté du Père qui m'a envoyé.
\VS{31}Si je rends témoignage de moi-même, mon témoignage n'est pas digne de foi.
\VS{32}C'est un autre qui rend témoignage de moi, et je sais que le témoignage qu'il rend de moi est digne de foi.
\TextTitle{[a. Le témoignage de Jean-Baptiste]}
\VS{33}Vous avez envoyé une délégation vers Jean, et il a rendu témoignage à la vérité.
\VS{34}Or je ne cherche point le témoignage des hommes ; mais je dis ces choses afin que vous soyez sauvés.
\VS{35}Jean était une lampe ardente et brillante ; et vous avez voulu vous réjouir pour un peu de temps à sa lumière.
\TextTitle{[b. Le témoignage des oeuvres de Jésus]}
\VS{36}Mais moi, j'ai un témoignage plus grand que celui de Jean ; car les œuvres que mon Père m'a donné d’accomplir, ces œuvres mêmes que je fais, témoignent de moi que c’est mon Père qui m'a envoyé.
\TextTitle{[c. Le témoignage du Père]
\\(Mt. 3:17)}
\VS{37}Et le Père qui m'a envoyé, a lui-même rendu témoignage de moi. Vous n’avez jamais entendu sa voix, vous n’avez jamais vu sa face.
\VS{38}Et sa parole ne demeure point en vous, puisque vous ne croyez pas à celui qu'il a envoyé.
\TextTitle{[d. Le témoignage de l'Ecriture]
\\(Lu. 24:27,44)}
\VS{39}Vous sondez les Ecritures, car vous pensez avoir en elles la vie éternelle, et ce sont elles qui rendent témoignage de moi.
\VS{40}Et vous ne voulez pas venir à moi, pour avoir la vie.
\VS{41}Je ne tire pas ma gloire des hommes.
\VS{42}Mais je sais que vous n'avez point l'amour de Dieu en vous.
\VS{43}Je suis venu au Nom de mon Père, et vous ne me recevez pas, si un autre vient en son propre nom, vous le recevrez.
\VS{44}Comment pouvez-vous croire, puisque vous recevez la gloire les uns des autres, et ne cherchez point la gloire qui vient de Dieu seul ?
\VS{45}Ne croyez point que je vous accuserai devant mon Père ; Moïse sur qui vous vous fondez, est celui qui vous accusera.
\VS{46}Car si vous croyiez Moïse, vous me croiriez aussi ; parce qu’il a écrit à mon sujet.
\VS{47}Mais si vous ne croyez pas à ses écrits, comment croirez-vous à mes paroles ?
\TextTitle{[Une autre Pâque et la multiplication des pains pour les cinq mille hommes]
\\(Mt. 14:15-21 ; Mc. 6:32-44 ; Lu. 9:12-17)}
\Chap{6}
\VerseOne{}Après ces choses, Jésus s'en alla au-delà de la mer de Galilée, qui est la mer de Tibériade.
\VS{2}Une grande foule le suivait, parce qu’elle voyait les miracles qu'il opérait sur les malades.
\VS{3}Jésus monta sur une montagne, et il s'assit là avec ses disciples.
\VS{4}Or, la Pâque, la fête des Juifs, était proche.
\VS{5}Et Jésus ayant levé ses yeux, et voyant qu’une grande foule venait à lui, dit à Philippe : Où achèterons-nous des pains, afin que ces gens aient à manger ?
\VS{6}Il disait cela pour l'éprouver, car il savait bien ce qu'il allait faire.
\VS{7}Philippe lui répondit : Les pains qu’on aurait pour deux cents deniers ne suffiraient pas pour que chacun en reçoive un peu.
\VS{8}Un de ses disciples, André, frère de Simon Pierre, lui dit :
\VS{9}Il y a ici un petit garçon qui a cinq pains d'orge et deux poissons ; mais qu'est-ce que cela pour tant de gens ?
\VS{10}Alors Jésus dit : Faites asseoir les gens. Il y avait beaucoup d'herbe dans ce lieu. Ils s'assirent au nombre d'environ cinq mille.
\VS{11}Et Jésus prit les pains ; et après avoir rendu grâces, il les distribua aux disciples, et les disciples à ceux qui étaient assis, et de même des poissons, autant qu'ils en voulaient.
\VS{12}Et après qu'ils furent rassasiés, il dit à ses disciples : Ramassez les morceaux qui restent, afin que rien ne soit perdu.
\VS{13}Ils les ramassèrent donc, et ils remplirent douze paniers avec les morceaux qui restèrent des cinq pains d'orge, après que tous eurent mangé.
\VS{14}Ces gens, ayant vu le miracle que Jésus avait fait, disaient : Celui-ci est véritablement le Prophète qui devait venir dans le monde.
\TextTitle{[Jésus marche sur les eaux]
\\(Mt. 14:22-33 ; Mc. 6:45-52)}
\VS{15}Mais Jésus, sachant qu'ils allaient venir l'enlever pour le faire Roi, se retira encore, lui seul, sur la montagne.
\VS{16}Et quand le soir fut venu, ses disciples descendirent à la mer.
\VS{17}Etant montés dans la barque, ils traversaient la mer pour se rendre à Capernaüm. Il faisait déjà nuit, et Jésus ne les avait pas encore rejoints.
\VS{18}Il soufflait un grand vent, et la mer était agitée.
\VS{19}Après avoir ramé environ vingt-cinq ou trente stades, ils virent Jésus marchant sur la mer, et s'approchant de la barque. Et ils eurent peur.
\VS{20}Mais il leur dit : C’est moi, ne craignez point.
\VS{21}Ils le reçurent donc avec plaisir dans la barque, et aussitôt la barque aborda au lieu où ils allaient.
\TextTitle{[Jésus, le pain de vie]}
\VS{22}Le lendemain, la foule qui était restée de l'autre côté de la mer, vit qu’il ne se trouvait là qu’une seule barque, et que Jésus n’était pas monté avec ses disciples dans la barque, mais qu’ils étaient partis seuls.
\VS{23}Cependant, d’autres barques étaient arrivées de Tibériade près du lieu où ils avaient mangé le pain, après que le Seigneur eut rendu grâces.
\VS{24}Quand la foule vit que ni Jésus ni ses disciples n’étaient là, les gens montèrent eux-mêmes dans ces barques, et allèrent à Capernaüm chercher Jésus.
\VS{25}Et l'ayant trouvé au-delà de la mer, ils lui dirent : Rabbi, quand es-tu arrivé ici ?
\VS{26}Jésus leur répondit et leur dit : En vérité, en vérité je vous le dis : Vous me cherchez, non parce que vous avez vu des miracles, mais parce que vous avez mangé des pains et que vous avez été rassasiés.
\VS{27}Travaillez, non pour la nourriture qui périt, mais pour celle qui est permanente jusqu’à la vie éternelle, et que le Fils de l'homme vous donnera ; car c’est lui que le Père, que Dieu, a marqué de son sceau.
\VS{28}Ils lui dirent donc : Que devons-nous faire pour accomplir les œuvres de Dieu ?
\VS{29}Jésus répondit et leur dit : C’est ici l’œuvre de Dieu, que vous croyiez en celui qu'il a envoyé.
\TextTitle{[Jésus envoyé du ciel]}
\VS{30}Alors ils lui dirent : Quel miracle fais-tu donc, afin que nous le voyions, et que nous croyions en toi ? Quelle œuvre fais-tu ?
\VS{31}Nos pères ont mangé la manne dans le désert ; selon ce qui est écrit : Il leur a donné à manger le pain du ciel\FTNT{} (1).
\VS{32}Mais Jésus leur dit : En vérité, en vérité je vous le dis : Moïse ne vous a pas donné le pain du ciel ; mais mon Père vous donne le vrai pain du ciel.
\VS{33}Car le pain de Dieu c'est celui qui est descendu du ciel et qui donne la vie au monde.
\VS{34}Ils lui dirent donc : Seigneur, donne-nous toujours ce pain-là.
\VS{35}Et Jésus leur dit : Je suis le pain de vie. Celui qui vient à moi, n'aura jamais faim ; et celui qui croit en moi, n'aura jamais soif.
\VS{36}Mais, je vous ai dit que vous m'avez vu, et cependant vous ne croyez point.
\VS{37}Tous ceux que mon Père me donne viendront à moi ; et je ne mettrai point dehors celui qui viendra à moi.
\VS{38}Car je suis descendu du ciel, non point pour faire ma volonté, mais la volonté de celui qui m'a envoyé.
\VS{39}Or, la volonté du Père qui m'a envoyé, c’est que je ne perde aucun de tous ceux qu'il m'a donnés, mais que je les ressuscite au dernier jour.
\VS{40}La volonté de celui qui m'a envoyé, c’est que quiconque contemple le Fils, et croit en lui, ait la vie éternelle ; et je le ressusciterai au dernier jour.
\VS{41}Les Juifs murmuraient contre lui de ce qu'il avait dit : Je suis le pain qui est descendu du ciel.
\VS{42}Et ils disaient : N’est-ce pas là Jésus, le fils de Joseph, celui dont nous connaissons le père et la mère ? Comment donc dit-il : Je suis descendu du ciel ?
\VS{43}Jésus leur répondit et leur dit : Ne murmurez pas entre vous.
\VS{44}Nul ne peut venir à moi, si le Père qui m'a envoyé ne l’attire ; et je le ressusciterai au dernier jour.
\VS{45}Il est écrit dans les prophètes : Ils seront tous enseignés de Dieu. Ainsi, quiconque a entendu le Père et a été instruit de ses intentions, vient à moi.
\VS{46}C’est que nul n’a vu le Père, sinon celui qui vient de Dieu, celui-là a vu le Père.
\VS{47}En vérité, en vérité je vous le dis : Celui qui croit en moi a la vie éternelle.
\VS{48}Je suis le pain de vie.
\VS{49}Vos pères ont mangé la manne dans le désert, et ils sont morts.
\VS{50}C'est ici le pain qui est descendu du ciel, afin que celui qui en mange, ne meure point.
\VS{51}Je suis le pain vivant qui est descendu du ciel. Si quelqu'un mange de ce pain, il vivra éternellement ; et le pain que je donnerai, c'est ma chair, que je donnerai pour la vie du monde.
\VS{52}Les Juifs donc discutaient entre eux, et disaient : Comment peut-il nous donner sa chair à manger ?
\VS{53}Et Jésus leur dit : En vérité, en vérité je vous le dis : Si vous ne mangez pas la chair du Fils de l'homme, et ne buvez pas son sang, vous n'aurez point la vie en vous-mêmes.
\VS{54}Celui qui mange ma chair, et qui boit mon sang, a la vie éternelle ; et je le ressusciterai au dernier jour.
\VS{55}Car ma chair est une véritable nourriture, et mon sang est un véritable breuvage.
\VS{56}Celui qui mange ma chair, et qui boit mon sang, demeure en moi, et moi en lui.
\VS{57}Comme le Père qui est vivant m'a envoyé, et que je suis vivant par le Père ; ainsi celui qui me mangera, vivra aussi par moi.
\VS{58}C'est ici le pain qui est descendu du ciel. Il n’en est pas comme de vos pères qui ont mangé la manne, et qui sont morts ; celui qui mangera ce pain, vivra éternellement.
\VS{59}Il dit ces choses dans la synagogue, enseignant à Capernaüm.
\TextTitle{[Epreuve de la consécration des disciples]
\\(Mt. 8:19-22 ; 10:36 ; Lu. 9:23-26)}
\VS{60}Plusieurs de ses disciples l'ayant entendu, dirent : Cette parole est dure, qui peut l’écouter ?
\VS{61}Mais Jésus sachant en lui-même que ses disciples murmuraient à ce sujet, leur dit : Cela vous scandalise-t-il ?
\VS{62}Que sera-ce donc si vous voyez le Fils de l'homme monter où il était auparavant ?
\VS{63}C'est l'Esprit qui vivifie ; la chair ne sert à rien. Les paroles que je vous ai dites, sont Esprit et vie.
\VS{64}Mais il en est parmi vous qui ne croient point. En effet, Jésus savait dès le commencement qui étaient ceux qui ne croiraient point, et qui était celui qui le trahirait.
\VS{65}Il leur dit donc : C’est pour cela que je vous ai dit, que nul ne peut venir à moi, si cela ne lui a pas été donné par mon Père.
\TextTitle{[Pierre reconnaît Jésus comme le Christ]
\\(Mt. 16:13-16 ; Mc. 8:27-30 ; Lu. 9:18-21}
\VS{66}Dès ce moment, plusieurs de ses disciples l'abandonnèrent, et ils ne marchèrent plus avec lui.
\VS{67}Et Jésus dit aux douze : Et vous, ne voulez-vous pas aussi vous en aller ?
\VS{68}Mais Simon Pierre lui répondit : Seigneur ! Auprès de qui irions-nous ? Tu as les paroles de la vie éternelle.
\VS{69}Et nous avons cru, et nous avons connu que tu es le Christ, le Fils du Dieu vivant.
\VS{70}Jésus leur répondit : Ne vous ai-je pas choisis, vous les douze ? Et toutefois l'un de vous est un démon.
\VS{71}Il parlait de Judas Iscariot, fils de Simon ; car c'était lui qui devait le trahir, quoiqu'il fût l'un des douze.
\TextTitle{[Jésus engagé par ses frères incrédules à se rendre à Jérusalem]}
\Chap{7}
\VerseOne{}Après ces choses, Jésus parcourait la Galilée, car il ne voulait pas parcourir la Judée, parce que les Juifs cherchaient à le faire mourir.
\VS{2}Or la fête des Juifs, appelée la fête des tabernacles, était proche.
\VS{3}Et ses frères lui dirent : Pars d'ici, et va en Judée, afin que tes disciples aussi contemplent les œuvres que tu fais.
\VS{4}Personne n’agit en secret, lorsqu'il cherche à être connu ; si tu fais ces choses, montre-toi toi-même au monde.
\VS{5}Car ses frères non plus ne croyaient pas en lui.
\VS{6}Et Jésus leur dit : Mon temps n'est pas encore venu, mais votre temps est toujours prêt.
\VS{7}Le monde ne peut pas vous haïr, mais il me hait parce que je rends témoignage contre lui que ses œuvres sont mauvaises.
\VS{8}Montez, vous, à cette fête ; pour moi, je n’y monte pas encore, parce que mon temps n'est pas encore accompli.
\VS{9}Après leur avoir dit ces choses, il resta en Galilée.
\TextTitle{[Jésus à la fête des tabernacles]}
\VS{10}Lorsque ses frères furent montés, alors il y monta aussi lui-même, non publiquement, mais comme en secret.
\VS{11}Les Juifs le cherchaient pendant la fête, et ils disaient : Où est-il ?
\VS{12}Et il y avait un grand murmure à son sujet parmi la foule. Les uns disaient : C’est un homme de bien ; et les autres disaient : Non, il séduit le peuple.
\VS{13}Toutefois personne ne parlait franchement de lui, à cause de la crainte qu'on avait des Juifs.
\VS{14}Vers le milieu de la fête, Jésus monta au temple. Et il enseignait.
\VS{15}Les Juifs s’étonnaient, disant : Comment connaît-il les Ecritures, lui qui n’a point étudié ?
\VS{16}Jésus leur répondit et dit : Ma doctrine n'est pas de moi, mais de celui qui m'a envoyé.
\VS{17}Si quelqu'un veut faire sa volonté, il connaîtra si ma doctrine est de Dieu, ou si je parle de moi-même.
\VS{18}Celui qui parle de son propre chef cherche sa propre gloire ; mais celui qui cherche la gloire de celui qui l'a envoyé, est véritable, et il n'y a point d'injustice en lui.
\VS{19}Moïse ne vous a-t-il pas donné la loi ? Cependant, nul de vous n'observe la loi. Pourquoi cherchez-vous à me faire mourir ?
\VS{20}La foule répondit : Tu as un démon ; qui est-ce qui cherche à te faire mourir ?
\VS{21}Jésus répondit et leur dit : J’ai fait une œuvre, et vous en êtes tous étonnés.
\VS{22}Moïse vous a donné la circoncision, non qu’elle vienne de Moïse, mais des pères, vous circoncisez bien un homme le jour du sabbat.
\VS{23}Si un homme reçoit la circoncision le jour du sabbat, afin que la loi de Moïse ne soit pas violée, pourquoi êtes-vous irrités contre moi de ce que j'ai guéri un homme tout entier le jour du sabbat ?
\VS{24}Ne jugez pas selon les apparences, mais jugez selon la justice.
\VS{25}Alors quelques-uns de ceux de Jérusalem disaient : N'est-ce pas celui qu'ils cherchent à faire mourir ?
\VS{26}Et cependant voici, il parle librement, et ils ne lui disent rien ! Est-ce que vraiment les chefs auraient reconnu qu’il est véritablement le Christ ?
\VS{27}Cependant celui-ci, nous savons d'où il est ; mais quand le Christ viendra, personne ne saura d'où il est.
\VS{28}Jésus, enseignant dans le temple, s’écria : Vous me connaissez, et vous savez d'où je suis ! Je ne suis pas venu de moi-même, mais celui qui m'a envoyé est véritable, et vous ne le connaissez pas.
\VS{29}Mais moi, je le connais ; car je viens de lui, et c'est lui qui m'a envoyé.
\VS{30}Ils cherchaient donc à se saisir de lui, mais personne ne mit la main sur lui, parce que son heure n'était pas encore venue.
\VS{31}Cependant, plusieurs parmi la foule crurent en lui, et ils disaient : Quand le Christ sera venu, fera-t-il plus de miracles que celui-ci n'a fait ?
\VS{32}Les pharisiens entendirent la foule murmurant ces choses de lui. Alors les principaux sacrificateurs et les pharisiens envoyèrent des huissiers pour le prendre.
\VS{33}Et Jésus leur dit : Je suis encore pour un peu de temps avec vous, puis je m'en vais vers celui qui m'a envoyé.
\VS{34}Vous me chercherez, mais vous ne me trouverez pas, et vous ne pouvez pas venir où je serai.
\VS{35}Les Juifs dirent donc entre eux : Où ira-t-il, pour que nous ne le trouvions pas ? Ira-t-il parmi ceux qui sont dispersés chez les Grecs, et enseignera-t-il les Grecs ?
\VS{36}Quel est ce discours qu'il a tenu : Vous me chercherez, mais vous ne me trouverez pas, vous ne pouvez pas venir où je serai ?
\TextTitle{[La grande prophétie sur le secret de la puissance du Saint-Esprit]
\\(Ac. 2:2-4 ; Jn. 4:14)}
\VS{37}Le dernier jour, le grand jour de la fête, Jésus, se tenant debout, s’écria : Si quelqu'un a soif, qu'il vienne à moi, et qu'il boive.
\VS{38}Celui qui croit en moi, des fleuves d'eau vive couleront de son sein, comme dit l'Ecriture.
\VS{39}Il dit cela de l'Esprit que devaient recevoir ceux qui croiraient en lui ; car le Saint-Esprit n'était pas encore donné, parce que Jésus n'était pas encore glorifié.
\TextTitle{[Diversité d'opinions au sujet de Jésus]}
\VS{40}Plusieurs de la foule ayant entendu ce discours, disaient : Celui-ci est véritablement le Prophète.
\VS{41}Les autres disaient : Celui-ci est le Christ. Et les autres disaient : Est-ce bien de la Galilée que doit venir le Christ ?
\VS{42}L'Ecriture ne dit-elle pas que le Christ doit venir de la postérité de David, et du village de Bethléem, où était David ?
\VS{43}Il y eut donc division parmi la foule à cause de lui.
\VS{44}Et quelques-uns d'entre eux voulaient le saisir, mais personne ne mit la main sur lui.
\VS{45}Ainsi les huissiers retournèrent vers les principaux sacrificateurs et les pharisiens, qui leur dirent : Pourquoi ne l'avez-vous pas amené ?
\VS{46}Les huissiers répondirent : Jamais homme n’a parlé comme cet homme.
\VS{47}Mais les pharisiens leur répondirent : Est-ce que vous aussi, vous avez été séduits ?
\VS{48}Y a-t-il quelqu’un des chefs ou des pharisiens qui ait cru en lui ?
\VS{49}Mais cette foule, qui ne connaît pas la loi, ce sont des maudits.
\VS{50}Nicodème, qui était venu vers Jésus de nuit, et qui était l'un d'entre eux, leur dit :
\VS{51}Notre loi condamne-t-elle un homme avant qu’on l’entende et qu’on ne sache ce qu’il a fait ?
\VS{52}Ils lui répondirent : Es-tu aussi Galiléen ? Examine, et tu verras qu'aucun prophète n’est sorti de la Galilée.
\VS{53}Et chacun s'en alla dans sa maison.
\TextTitle{[Les scribes et les pharisiens accusent une femme surprise en flagrant délit d'adultère]}
\Chap{8}
\VerseOne{}Jésus se rendit à la montagne des oliviers.
\VS{2}Et, dès le matin, il alla de nouveau dans le temple, et tout le peuple vint à lui ; et s'étant assis, il les enseignait.
\VS{3}Alors les scribes et les pharisiens lui amenèrent une femme surprise en adultère ;
\VS{4}et l'ayant placée au milieu du peuple, ils dirent à Jésus : Maître, cette femme a été surprise en flagrant délit d’adultère.
\VS{5}Moïse nous a ordonné dans la loi de lapider celles qui sont dans son cas ; toi donc qu'en dis-tu ?
\VS{6}Or ils disaient cela pour l'éprouver, afin de pouvoir l'accuser. Mais Jésus s'étant penché en bas, écrivait avec son doigt sur la terre.
\VS{7}Et comme ils continuaient à l'interroger, s'étant relevé, il leur dit : Que celui de vous qui est sans péché, jette le premier la pierre contre elle.
\VS{8}Et s'étant encore baissé, il écrivait sur la terre.
\VS{9}Quand ils entendirent cela, accusés par leur conscience, ils se retirèrent un à un, depuis les plus âgés jusqu’aux derniers ; et Jésus resta seul avec la femme qui était là au milieu.
\VS{10}Alors Jésus s'étant relevé, et ne voyant plus que la femme, il lui dit : Femme, où sont ceux qui t'accusaient ? Personne ne t’a-t-il condamnée ?
\VS{11}Elle dit : Non, Seigneur. Et Jésus lui dit : Je ne te condamne pas non plus ; va, et ne pèche plus.
\TextTitle{[Point crucial du conflit entre Jésus et les pharisiens : l'origine de Christ, Lumière du monde]
\\(Jn. 1:9)}
\VS{12}Et Jésus leur parla encore, en disant : Je suis la Lumière du monde ; celui qui me suit ne marchera pas dans les ténèbres, mais il aura la lumière de la vie.
\VS{13}Alors les pharisiens lui dirent : Tu rends témoignage de toi-même, ton témoignage n'est pas digne de foi.
\VS{14}Jésus répondit et leur dit : Quoique je rende témoignage de moi-même, mon témoignage est digne de foi ; car je sais d'où je suis venu et où je vais ; mais vous ne savez pas d'où je viens ni où je vais.
\VS{15}Vous jugez selon la chair, mais moi, je ne juge personne.
\VS{16}Et si je juge, mon jugement est digne de foi ; car je ne suis pas seul, mais le Père qui m'a envoyé est avec moi.
\VS{17}Il est même écrit dans votre loi que le témoignage de deux hommes est digne de foi\FTNT{De. 19:15.}.
\VS{18}Je rends témoignage de moi-même, et le Père qui m'a envoyé rend aussi témoignage de moi.
\VS{19}Alors ils lui dirent : Où est ton Père ? Jésus répondit : Vous ne connaissez ni moi ni mon Père. Si vous me connaissiez, vous connaîtriez aussi mon Père.
\VS{20}Jésus dit ces paroles au lieu où était le trésor, enseignant dans le temple ; mais personne ne le saisit, parce que son heure n'était pas encore venue.
\VS{21}Et Jésus leur dit encore : Je m'en vais, et vous me chercherez, et vous mourrez dans vos péchés ; vous ne pouvez pas venir où je vais.
\VS{22}Les Juifs disaient donc : Se tuera-t-il lui-même, puisqu’il dit : Vous ne pouvez pas venir où je vais ?
\VS{23}Alors il leur dit : Vous êtes d'en bas, mais moi, je suis d'en haut ; vous êtes de ce monde, mais moi, je ne suis pas de ce monde.
\VS{24}C'est pourquoi je vous ai dit que vous mourrez dans vos péchés ; car si vous ne croyez pas que je suis l'envoyé de Dieu, vous mourrez dans vos péchés.
\VS{25}Alors ils lui dirent : Toi, qui es-tu ? Et Jésus leur dit : Ce que je vous dis dès le commencement.
\VS{26}J'ai beaucoup de choses à dire de vous et à juger en vous, mais celui qui m'a envoyé est véritable, et les choses que j'ai entendues de lui, je les dis au monde.
\VS{27}Ils ne comprirent point qu'il leur parlait du Père.
\VS{28}Jésus leur dit donc : Quand vous aurez élevé le Fils de l'homme, vous connaîtrez alors que je suis l'envoyé de Dieu, et que je ne fais rien de moi-même, mais que je dis ces choses selon ce que mon Père m'a enseigné.
\VS{29}Celui qui m'a envoyé est avec moi ; le Père ne m'a pas laissé seul, parce que je fais toujours les choses qui lui plaisent.
\VS{30}Comme il disait ces choses, plusieurs crurent en lui.
\VS{31}Et Jésus disait aux Juifs qui avaient cru en lui : Si vous demeurez dans ma parole, vous serez vraiment mes disciples.
\VS{32}Vous connaîtrez la vérité, et la vérité vous rendra libres.
\VS{33}Ils lui répondirent : Nous sommes la postérité d'Abraham, et nous ne fûmes jamais esclaves de personne ; comment donc dis-tu : Vous deviendrez libres ?
\VS{34}Jésus leur répondit : En vérité, en vérité je vous le dis : Quiconque se livre au péché, est esclave du péché.
\VS{35}Or l'esclave ne demeure pas toujours dans la maison ; le fils y demeure toujours.
\VS{36}Si donc le Fils vous affranchit, vous serez véritablement libres.
\VS{37}Je sais que vous êtes la postérité d'Abraham, pourtant vous cherchez à me faire mourir, parce que ma parole n'est pas reçue dans vos cœurs.
\VS{38}Je vous dis ce que j'ai vu chez mon Père ; et vous aussi vous faites les choses que vous avez vues chez votre père.
\VS{39}Ils répondirent et lui dirent : Notre père c'est Abraham. Jésus leur dit : Si vous étiez enfants d'Abraham, vous feriez les œuvres d'Abraham.
\VS{40}Mais maintenant vous cherchez à me faire mourir, moi, un homme qui vous ai dit la vérité que j'ai entendue de Dieu. Cela, Abraham ne l’a point fait.
\VS{41}Vous faites les œuvres de votre père. Et ils lui dirent : Nous ne sommes pas des enfants illégitimes ; nous avons un seul père, Dieu.
\VS{42}Mais Jésus leur dit : Si Dieu était votre Père, certes vous m'aimeriez, car c’est de Dieu que je suis sorti et que je viens ; je ne suis pas venu de moi-même, mais c'est lui qui m'a envoyé.
\VS{43}Pourquoi ne comprenez-vous pas mon langage ? C’est parce que vous ne pouvez pas écouter ma parole.
\VS{44}Vous avez pour père le diable, et vous voulez accomplir les désirs de votre père. Il a été meurtrier dès le commencement, et il n'a pas persévéré dans la vérité, car la vérité n'est pas en lui. Toutes les fois qu'il profère le mensonge, il parle de son propre fond ; car il est menteur et le père du mensonge.
\VS{45}Et moi, parce que je dis la vérité, vous ne me croyez pas.
\VS{46}Qui de vous me convaincra de péché ? Si je dis la vérité, pourquoi ne me croyez-vous pas ?
\VS{47}Celui qui est de Dieu écoute les paroles de Dieu ; vous n’écoutez pas, parce que vous n'êtes pas de Dieu.
\VS{48}Alors les Juifs répondirent : N’avons-nous pas raison de dire que tu es un Samaritain, et que tu as un démon ?
\VS{49}Jésus répondit : Je n'ai point un démon, mais j'honore mon Père, et vous m’outragez.
\VS{50}Je ne cherche point ma gloire ; il y en a un qui la cherche, et qui juge.
\VS{51}En vérité, en vérité je vous le dis : Si quelqu'un garde ma parole, il ne verra jamais la mort.
\VS{52}Les Juifs lui dirent donc : Maintenant nous savons que tu as un démon. Abraham est mort, et les prophètes aussi, et tu dis : Si quelqu'un garde ma parole, il ne verra jamais la mort.
\VS{53}Es-tu plus grand que notre père Abraham qui est mort ? Les prophètes aussi sont morts. Qui prétends-tu être ?
\VS{54}Jésus répondit : Si je me glorifie moi-même, ma gloire n'est rien ; mon Père est celui qui me glorifie, celui que vous dites être votre Dieu.
\VS{55}Toutefois vous ne l'avez point connu, mais moi je le connais ; et si je disais que je ne le connais point, je serais un menteur, semblable à vous ; mais je le connais, et je garde sa parole.
\VS{56}Abraham votre père a tressailli de joie de ce qu’il verrait mon jour ; et il l'a vu, et il s’est réjoui.
\VS{57}Les Juifs lui dirent : Tu n'as pas encore cinquante ans, et tu as vu Abraham !
\VS{58}Jésus leur dit : En vérité, en vérité je vous le dis : Avant qu'Abraham fût, Je suis\FTNT{Je suis : L'évangile de Jean rapporte plusieurs déclarations incroyables que Jésus a faites à son sujet : Je suis le pain de vie (6:35), Je suis la Lumière du monde (8:12), Je suis le bon berger (10:11), Je suis la porte (10:7), Je suis la résurrection (11:25), Je suis le chemin, la vérité et la vie (14:6), Je suis la vraie vigne (15:1). Toutefois, dans ce verset, en déclarant être ~Je suis~, il s’identifie clairement au Nom que YHWH avait révélé à Moïse dans Ex. 3:14. C'est précisément pour cette raison que les juifs ont voulu le lapider.}.
\VS{59}Alors ils prirent des pierres pour les jeter contre lui, mais Jésus se cacha et sortit du temple, passant au milieu d'eux ; et ainsi il s'en alla.
\TextTitle{[Jésus guérit un aveugle-né]}
\Chap{9}
\VerseOne{}Comme Jésus passait, il vit un homme aveugle de naissance.
\VS{2}Ses disciples lui posèrent cette question : Rabbi, qui a péché ? Cet homme ou ses parents pour qu’il soit né aveugle ?
\VS{3}Jésus répondit : Ce n’est pas que lui ou ses parents aient péché ; mais c'est afin que les œuvres de Dieu soient manifestées en lui.
\VS{4}Il faut que je fasse, tandis qu’il est jour, les œuvres de celui qui m'a envoyé. La nuit vient, où personne ne peut travailler.
\VS{5}Pendant que je suis dans le monde, je suis la Lumière du monde.
\VS{6}Ayant dit ces paroles, il cracha à terre et fit de la boue avec sa salive, et mit de cette boue sur les yeux de l'aveugle.
\VS{7}Et il lui dit : Va, et lave-toi au réservoir de Siloé (nom qui veut dire envoyé). Il y alla donc, se lava, et s’en retourna voyant clair.
\VS{8}Ses voisins et ceux qui auparavant l’avaient connu comme mendiant disaient : N'est-ce pas celui qui était assis et qui mendiait ?
\VS{9}Les uns disaient : C’est lui. Et les autres disaient : Il lui ressemble. Mais lui-même disait : C'est moi.
\VS{10}Ils lui dirent donc : Comment tes yeux ont-ils été ouverts ?
\VS{11}Il répondit et dit : Cet homme, qu'on appelle Jésus, a fait de la boue et il l'a mise sur mes yeux, et m'a dit : Va au réservoir de Siloé et lave-toi. J’y suis allé, je me suis lavé, et j’ai recouvert la vue.
\VS{12}Alors ils lui dirent : Où est cet homme ? Il répondit : Je ne sais pas.
\VS{13}Ils amenèrent vers les pharisiens celui qui auparavant avait été aveugle.
\VS{14}Or c'était en un jour de sabbat que Jésus avait fait de la boue et lui avait ouvert les yeux.
\VS{15}C'est pourquoi les pharisiens l'interrogèrent encore, comment il avait pu voir ; et il leur dit : Il a mis de la boue sur mes yeux, et je me suis lavé, et je vois.
\VS{16}Sur quoi quelques-uns des pharisiens dirent : Cet homme n'est pas un envoyé de Dieu ; car il n’observe pas le sabbat ; mais d'autres disaient : Comment un homme pécheur peut-il faire de tels prodiges ? Et il y avait de la division entre eux.
\VS{17}Ils dirent encore à l'aveugle : Toi, que dis-tu de lui, sur ce qu'il t'a ouvert les yeux ? Il répondit : C’est un Prophète.
\VS{18}Mais les Juifs ne crurent point que cet homme avait été aveugle, et qu'il avait pu voir, jusqu'à ce qu'ils aient fait venir ses parents.
\VS{19}Et ils les interrogèrent, disant : Est-ce là votre fils, que vous dites être né aveugle ? Comment donc voit-il maintenant ?
\VS{20}Ses parents leur répondirent : Nous savons que c'est notre fils et qu'il est né aveugle.
\VS{21}Mais comment il voit maintenant, ou qui lui a ouvert les yeux, nous ne le savons pas ; il a de l'âge, interrogez-le, il parlera de ce qui le regarde.
\VS{22}Ses parents dirent ces choses parce qu'ils craignaient les Juifs ; car les Juifs avaient déjà convenu que si quelqu'un reconnaissait Jésus pour le Christ, il serait exclu de la synagogue.
\VS{23}C’est pourquoi ses parents dirent : Il a de l'âge, interrogez-le lui-même.
\VS{24}Ils appelèrent donc pour la seconde fois l'homme qui avait été aveugle et ils lui dirent : Donne gloire à Dieu ; nous savons que cet homme est un pécheur.
\VS{25}Il répondit : Je ne sais pas si c’est un pécheur ; je sais une chose, c’est que j’étais aveugle et que maintenant je vois.
\VS{26}Ils lui dirent donc encore : Que t'a-t-il fait ? Comment a-t-il ouvert tes yeux ?
\VS{27}Il leur répondit : Je vous l'ai déjà dit, et vous ne l'avez point écouté, pourquoi voulez-vous l’entendre encore ? Voulez-vous aussi devenir ses disciples ?
\VS{28}Alors ils l'injurièrent et lui dirent : C’est toi son disciple ; nous, nous sommes disciples de Moïse.
\VS{29}Nous savons que Dieu a parlé à Moïse ; mais celui-ci, nous ne savons pas d'où il est.
\VS{30}Cet homme répondit : Certes, c'est une chose étrange que vous ne sachiez point d'où il est ; et toutefois il a ouvert mes yeux.
\VS{31}Nous savons que Dieu n'exauce point les méchants, mais si quelqu'un est pieux envers Dieu, et fait sa volonté, il l'exauce.
\VS{32}Jamais on n’a entendu dire que quelqu’un ait ouvert les yeux d’un aveugle-né.
\VS{33}Si cet homme n'était pas un envoyé de Dieu, il ne pourrait rien faire de semblable.
\VS{34}Ils répondirent : Tu es entièrement né dans le péché, et tu nous enseignes ! Et ils le chassèrent dehors.
\TextTitle{[Jésus affirme sa divinité]}
\VS{35}Jésus apprit qu'ils l'avaient chassé dehors ; et l'ayant rencontré, il lui dit : Crois-tu au Fils de Dieu ?
\VS{36}Cet homme lui répondit : Qui est-il Seigneur, afin que je croie en lui ?
\VS{37}Jésus lui dit : Tu l'as vu, et c'est celui qui te parle.
\VS{38}Alors il dit : Je crois, Seigneur ; et il l'adora\FTNT{Au travers de la lecture de la Bible, on constate que les anges refusent l’adoration (Ap. 19:9-10) de même que les apôtres (Ac. 10:25-26 ; Ac. 14:5-18). Seul Dieu accepte l’adoration puisqu’il en est le seul digne. Jésus n’a jamais refusé l’adoration des hommes, car il est Dieu.}.
\VS{39}Et Jésus dit : Je suis venu dans ce monde pour exercer le jugement, afin que ceux qui ne voient point voient ; et que ceux qui voient deviennent aveugles.
\VS{40}Quelques pharisiens qui étaient avec lui, ayant entendu ces paroles, dirent : Et nous, sommes-nous aussi aveugles ?
\VS{41}Jésus leur répondit : Si vous étiez aveugles, vous n'auriez point de péché ; mais maintenant vous dites : Nous voyons. C’est à cause de cela que votre péché demeure.
\TextTitle{[Jésus, le Bon Berger]
\\(Ps. 23 ; Hé. 13:20 ; 1Pi. 5:4)}
\Chap{10}
\VerseOne{}En vérité, en vérité je vous le dis : Celui qui n'entre point par la porte dans la bergerie des brebis, mais y monte par ailleurs, est un voleur et un brigand.
\VS{2}Mais celui qui entre par la porte est le berger des brebis.
\VS{3}Le portier lui ouvre, et les brebis entendent sa voix, et il appelle les brebis qui lui appartiennent par leur nom, et il les conduit dehors.
\VS{4}Lorsqu’il a fait sortir toutes ses brebis dehors, il marche devant elles, et les brebis le suivent, parce qu'elles connaissent sa voix.
\VS{5}Mais elles ne suivront point un étranger, au contraire, elles fuiront loin de lui ; parce qu'elles ne connaissent point la voix des étrangers.
\VS{6}Jésus leur dit cette parabole, mais ils ne comprirent pas ce qu'il leur disait.
\VS{7}Jésus leur dit encore : En vérité, en vérité je vous le dis : Je suis la Porte par où entrent les brebis\FTNT{La porte des brebis était située près du temple et avait été bâtie du temps de Néhémie (Né. 3:1). Les animaux que l’on sacrifiait à Dieu franchissaient probablement cette porte.}.
\VS{8}Tout ceux qui sont venus avant moi sont des brigands et des voleurs ; mais les brebis ne les ont point écoutés.
\VS{9}Je suis la Porte : Si quelqu'un entre par moi, il sera sauvé ; il entrera et il sortira, et il trouvera des pâturages.
\VS{10}Le voleur ne vient que pour dérober, tuer et détruire ; moi, je suis venu afin que mes brebis aient la vie, et qu'elles l'aient même en abondance.
\VS{11}Je suis le bon berger : Le bon berger donne sa vie pour ses brebis.
\VS{12}Mais le mercenaire, qui n’est pas le berger, à qui n'appartiennent pas les brebis, voit venir le loup, abandonne les brebis, et s'enfuit ; et le loup ravit et disperse les brebis.
\VS{13}Ainsi le mercenaire s'enfuit, parce qu'il est mercenaire, et qu'il ne se soucie pas des brebis. Je suis le bon berger.
\VS{14}Je connais mes brebis, et mes brebis me connaissent.
\VS{15}Comme le Père me connaît, et comme je connais le Père ; et je donne ma vie pour mes brebis.
\VS{16}J'ai encore d'autres brebis qui ne sont pas de cette bergerie ; celles-là, il faut aussi que je les amène ; elles entendront ma voix, et il y aura un seul troupeau, et un seul berger.
\VS{17}Le Père m'aime, parce que je donne ma vie, afin de la reprendre.
\VS{18}Personne ne me l'ôte, mais je la donne de moi-même. J'ai le pouvoir de la donner, et j’ai le pouvoir de la reprendre ; j'ai reçu cet ordre de mon Père.
\VS{19}Il y eut de nouveau division parmi les Juifs à cause de ces discours.
\VS{20}Car plusieurs disaient : Il a un démon, il est fou ! Pourquoi l'écoutez-vous ?
\VS{21}Et les autres disaient : Ce ne sont pas les paroles d'un démoniaque ; un démon peut-il ouvrir les yeux des aveugles ?
\TextTitle{[Jésus réaffirme sa divinité]
\\(Jn. 5:26-27 ; 14:9 ; 20:28-29)}
\VS{22}On célébrait la fête de la dédicace\FTNT{Le terme ~dédicace~ est la traduction du mot hébreu ~Hanoukka~ qui sert à désigner la consécration ou l'inauguration de l'autel servant à offrir des sacrifices à Dieu (No. 7:10 ; 2 Ch. 7:9). La Bible l’utilise aussi pour parler de l'inauguration des murailles de Jérusalem après leur reconstruction au temps de Néhémie (Né. 12:27). La fête d’Hanoukka a été instituée par Judas Maccabé en 164 av. J.-C. en mémoire de la purification du temple qui avait été profané par Antiochus Epiphane. Elle débute le 25 du mois de chisleu (mi décembre) de chaque année et dure huit jours.} à Jérusalem. Et c'était l’hiver.
\VS{23}Et Jésus se promenait dans le temple, au portique de Salomon.
\VS{24}Et les Juifs l’entourèrent et lui dirent : Jusqu’à quand tiendras-tu notre âme en suspens ? Si tu es le Christ, dis-le-nous franchement.
\VS{25}Jésus leur répondit : Je vous l'ai dit, et vous ne le croyez point. Les œuvres que je fais au Nom de mon Père rendent témoignage de moi.
\VS{26}Mais vous ne croyez point, parce que vous n'êtes point de mes brebis, comme je vous l'ai dit.
\VS{27}Mes brebis entendent ma voix ; je les connais, et elles me suivent.
\VS{28}Et moi, je leur donne la vie éternelle, et elles ne périront jamais ; et personne ne les ravira de ma main.
\VS{29}Mon Père, qui me les a données, est plus grand que tous ; et personne ne peut les ravir des mains de mon Père.
\VS{30}Moi et le Père nous sommes un.
\VS{31}Alors les Juifs prirent de nouveau des pierres pour le lapider.
\VS{32}Jésus leur dit : Je vous ai fait voir plusieurs bonnes œuvres de la part de mon Père : Pour laquelle me lapidez-vous ?
\VS{33}Les Juifs répondirent : Ce n’est pas pour une bonne œuvre que nous te lapidons, mais pour un blasphème, parce que toi qui es un homme, tu te fais Dieu.
\VS{34}Jésus leur répondit : N’est-il pas écrit dans votre loi : J'ai dit : Vous êtes des dieux\FTNT{Ps. 82:6 : Le sens du mot ~dieu~ peut désigner des personnes ayant un certain pouvoir. D'ailleurs, le mot hébreu utilisé dans Ps. 82:6 est ~Elohim~, or ce mot signifie aussi ~juge~. De plus, dans le contexte du psaume, ~vous êtes des dieux~ ne s'applique pas à tous, mais seulement à une certaine catégorie de personnes qui exerçaient un pouvoir en Israël : rois, scribes, souverains sacrificateurs… Rappelons-nous aussi que Dieu a fait de Moïse un dieu pour Aaron (Ex. 7:1-2), mais cela n’a pas fait de lui le Dieu Créateur pour autant. En Jn. 17:3, Jésus atteste qu'il n’y a qu’un seul vrai Dieu. Satan veut nous faire croire que nous sommes des dieux et nous amener ainsi à pécher par l’orgueil (Ge. 3:5). Toutefois, comme le souligne si bien l’apôtre Paul, même s’il existe des créatures qu’on appelle dieux ou déesses, il ne reste pas moins vrai qu’il n’y a qu’un seul Dieu (1 Co. 8:5-7).} ?
\VS{35}Si elle a appelé dieux ceux à qui la parole de Dieu est adressée, et cependant l'Ecriture ne peut être anéantie,
\VS{36}celui que le Père a sanctifié et envoyé dans le monde, vous lui dites : Tu blasphèmes ! Et cela parce que j’ai dit : Je suis le Fils de Dieu ?
\VS{37}Si je ne faisais pas les œuvres de mon Père, ne me croyez pas.
\VS{38}Mais si je les fais, même si vous ne me croyez pas, croyez à ces œuvres, afin que vous sachiez que le Père est en moi et que je suis dans le Père.
\VS{39}Là-dessus, ils cherchaient encore à le saisir ; mais il s’échappa de leurs mains.
\TextTitle{[Jésus se retire de Jérusalem]}
\VS{40}Il s'en alla de nouveau au-delà du Jourdain, à l'endroit où Jean avait baptisé au commencement, et il demeura là.
\VS{41}Beaucoup de gens vinrent à lui, et ils disaient : Jean n’a fait aucun miracle ; mais tout ce que Jean a dit de cet homme, était vrai.
\VS{42}Et dans ce lieu-là, plusieurs crurent en lui.
\TextTitle{[Jésus ressuscite Lazare de Béthanie]}
\Chap{11}
\VerseOne{}Il y avait un homme malade, Lazare, de Béthanie, village de Marie et de Marthe, sa sœur.
\VS{2}C’était cette Marie qui oignit de parfum le Seigneur, et qui essuya ses pieds avec ses cheveux ; et c’était son frère Lazare qui était malade.
\VS{3}Ses sœurs envoyèrent donc dire à Jésus : Seigneur, voici, celui que tu aimes est malade.
\VS{4}Après avoir entendu cela, Jésus dit : Cette maladie n'est point à la mort, mais elle est pour la gloire de Dieu, afin que le Fils de Dieu soit glorifié par elle.
\VS{5}Or Jésus aimait Marthe, sa sœur, et Lazare.
\VS{6}Après qu'il eut appris que Lazare était malade, il resta deux jours encore dans le lieu où il était,
\VS{7}et il dit à ses disciples : Retournons en Judée.
\VS{8}Les disciples lui dirent : Rabbi, les Juifs tout récemment cherchaient à te lapider, et tu retournes en Judée !
\VS{9}Jésus répondit : N'y a-t-il pas douze heures au jour ? Si quelqu'un marche pendant le jour, il ne bronche point ; car il voit la lumière de ce monde.
\VS{10}Mais si quelqu'un marche pendant la nuit, il bronche ; car il n'y a point de lumière avec lui.
\VS{11}Après ces paroles, il leur dit : Notre ami Lazare dort, mais je vais le réveiller.
\VS{12}Ses disciples lui dirent : Seigneur, s'il dort, il sera guéri.
\VS{13}Jésus avait parlé de sa mort, mais ils pensaient qu'il parlait de l’assoupissement.
\VS{14}Alors Jésus leur dit ouvertement : Lazare est mort.
\VS{15}Et je me réjouis, à cause de vous, de ce que je n’étais pas là, afin que vous croyiez. Mais allons vers lui.
\VS{16}Alors Thomas, appelé Didyme, dit aux autres disciples : Allons-y aussi, afin que nous mourions avec lui.
\VS{17}Jésus, étant arrivé, trouva que Lazare était déjà depuis quatre jours dans le sépulcre.
\VS{18}Et comme Béthanie était près de Jérusalem à quinze stades environ,
\VS{19}beaucoup de Juifs étaient venus vers Marthe et Marie pour les consoler au sujet de leur frère.
\VS{20}Lorsque Marthe apprit que Jésus arrivait, elle alla au-devant de lui ; mais Marie se tenait assise à la maison.
\VS{21}Marthe dit à Jésus : Seigneur, si tu avais été ici mon frère ne serait pas mort.
\VS{22}Mais maintenant je sais que tout ce que tu demanderas à Dieu, Dieu te le donnera.
\VS{23}Jésus lui dit : Ton frère ressuscitera.
\VS{24}Marthe lui dit : Je sais qu'il ressuscitera à la résurrection, au dernier jour.
\VS{25}Jésus lui dit : Je suis la résurrection et la vie : Celui qui croit en moi vivra même s’il meurt.
\VS{26}Et quiconque vit et croit en moi ne mourra jamais ; crois-tu cela ?
\VS{27}Elle lui dit : Oui, Seigneur, je crois que tu es le Christ, le Fils de Dieu, qui devait venir dans le monde.
\VS{28}Ayant ainsi parlé, elle alla appeler secrètement Marie sa sœur, en lui disant : Le Maître est ici, et il t'appelle.
\VS{29}Aussitôt que Marie eut entendu, elle se leva rapidement, et alla vers lui.
\VS{30}Or Jésus n'était pas encore entré dans le village, mais il était au lieu où Marthe l'avait rencontré.
\VS{31}Alors les Juifs qui étaient avec Marie à la maison, et qui la consolaient, ayant vu qu'elle s'était levée si promptement, et qu'elle était sortie, la suivirent en disant : Elle va au sépulcre pour y pleurer.
\VS{32}Lorsque Marie fut arrivée où était Jésus, et qu’elle le vit, elle se jeta à ses pieds, en lui disant : Seigneur, si tu avais été ici, mon frère ne serait pas mort.
\VS{33}Jésus, la voyant pleurer, elle et les Juifs qui étaient venus avec elle, frémit en son esprit et fut tout ému.
\VS{34}Et il dit : Où l'avez-vous mis ? Ils lui répondirent : Seigneur, viens et vois.
\VS{35}Jésus pleura.
\VS{36}Sur quoi les Juifs dirent : Voyez comme il l'aimait.
\VS{37}Et quelques-uns d'entre eux disaient : Lui qui a ouvert les yeux de l'aveugle, ne pouvait-il pas faire aussi que cet homme ne meure point ?
\VS{38}Alors Jésus frémissant de nouveau en lui-même, se rendit au sépulcre. C'était une grotte, et il y avait une pierre placée devant.
\VS{39}Jésus dit : Ôtez la pierre. Mais Marthe, la sœur du mort, lui dit : Seigneur, il sent déjà, car il est là depuis quatre jours.
\VS{40}Jésus lui dit : Ne t'ai-je pas dit que si tu crois tu verras la gloire de Dieu ?
\VS{41}Ils ôtèrent donc la pierre de dessus le lieu où le mort était couché. Et Jésus levant ses yeux au ciel, dit : Père, je te rends grâces de ce que tu m'as exaucé.
\VS{42}Pour moi, je savais que tu m'exauces toujours ; mais j’ai parlé à cause de la foule qui m’entoure, afin qu’ils croient que c’est toi qui m'as envoyé.
\VS{43}Ayant dit ces choses, il cria à haute voix : Lazare sors dehors !
\VS{44}Alors le mort sortit, ayant les mains et les pieds liés de bandes ; et son visage était enveloppé d'un linge. Jésus leur dit : Déliez-le, et laissez-le aller.
\TextTitle{[Nombreuses conversions]
\\(Jn. 12:10-11)}
\TextTitle{[Conspiration des Pharisiens]}
\VS{45}Plusieurs des Juifs qui étaient venus vers Marie, et qui avaient vu ce que Jésus avait fait, crurent en lui.
\VS{46}Mais quelques-uns d'entre eux allèrent trouver les pharisiens et leur dirent les choses que Jésus avait faites.
\VS{47}Alors les principaux sacrificateurs et les pharisiens assemblèrent le sanhédrin, et ils dirent : Que ferons-nous ? Car cet homme fait beaucoup de miracles.
\VS{48}Si nous le laissons faire, tout le monde croira en lui, et les Romains viendront et ils détruiront et ce lieu et notre nation.
\VS{49}Alors l'un d'eux appelé Caïphe, qui était le souverain sacrificateur cette année-là, leur dit : Vous n’y comprenez rien.
\VS{50}Et vous ne réfléchissez pas qu'il est de notre intérêt qu'un homme meure pour le peuple, et que toute la nation ne périsse point.
\VS{51}Or il ne dit pas cela de lui-même, mais étant souverain sacrificateur de cette année-là, il prophétisa que Jésus devait mourir pour la nation.
\VS{52}Et non pas seulement pour la nation, mais aussi pour rassembler en un seul corps les enfants de Dieu dispersés.
\VS{53}Depuis ce jour, ils se concertèrent ensemble pour le faire mourir.
\VS{54}C'est pourquoi Jésus ne se montrait plus ouvertement parmi les Juifs, mais il se retira dans la contrée voisine du désert, dans une ville appelée Ephraïm, et il demeura là avec ses disciples.
\VS{55}La Pâque des Juifs était proche. Et beaucoup de gens du pays montèrent à Jérusalem avant Pâque, afin de se purifier.
\VS{56}Et ils cherchaient Jésus, et se disaient les uns les autres dans le temple : Que vous en semble ? Ne viendra-t-il pas à la Fête ?
\VS{57}Or, les principaux sacrificateurs et les pharisiens avaient donné l’ordre que si quelqu'un savait où il était, il le déclare, afin qu’on se saisisse de lui.
\TextTitle{[Marie de Béthanie oint les pieds de Jésus]
\\(Mt. 26:6-13 ; Mc. 14:3-9)}
\Chap{12}
\VerseOne{}Six jours avant la Pâque, Jésus arriva à Béthanie, où était Lazare qui avait été mort, et qu'il avait ressuscité des morts.
\VS{2}Là, on lui fit un souper ; Marthe servait, et Lazare était un de ceux qui étaient à table avec lui.
\VS{3}Alors Marie ayant pris une livre de nard pur de grand prix, oignit les pieds de Jésus, et les essuya avec ses cheveux ; et la maison fut remplie de l'odeur du parfum.
\VS{4}Alors Judas Iscariot, fils de Simon, l'un de ses disciples, celui qui devait le trahir, dit :
\VS{5}Pourquoi ce parfum n'a-t-il pas été vendu trois cents deniers, pour donner cet argent aux pauvres ?
\VS{6}Il dit cela, non parce qu’il se mettait en peine des pauvres, mais parce qu'il était voleur, et que tenant la bourse, il prenait ce qu’on y mettait.
\VS{7}Mais Jésus lui dit : Laisse-la faire ; elle l'a gardé pour le jour de ma sépulture.
\VS{8}Car vous aurez toujours des pauvres avec vous ; mais vous ne m'aurez pas toujours.
\VS{9}Une grande multitude des Juifs apprirent que Jésus était à Béthanie, et ils y vinrent, non seulement à cause de lui, mais aussi pour voir Lazare qu'il avait ressuscité des morts.
\VS{10}Sur quoi les principaux sacrificateurs résolurent de faire mourir aussi Lazare.
\VS{11}Car plusieurs des Juifs se retiraient d'avec eux à cause de lui, et croyaient en Jésus.
\TextTitle{[Entrée triomphante de Jésus à Jérusalem]
\\(Mt. 21:1-11 ; Mc. 11:1-11 ; Lu. 19:28-40 ; Za. 9:9 ; Ap. 19:11-16}
\VS{12}Le lendemain, une grande quantité de foules qui étaient venues à la fête, ayant entendu dire que Jésus se rendait à Jérusalem,
\VS{13}prit des branches de palmes, et sortit au-devant de lui en criant : Hosanna ! Béni soit le Roi d'Israël qui vient au Nom du Seigneur !
\VS{14}Jésus trouva un ânon, s'assit dessus, selon ce qui est écrit : 15 Ne crains point, fille de Sion ; voici, ton Roi vient, assis sur le petit d'une ânesse\FTNT{Za. 9:9.}.
\VS{16}Ses disciples ne comprirent pas d'abord ces choses ; mais quand Jésus eut été glorifié, ils se souvinrent alors qu’elles étaient écrites de lui, et qu’elles avaient été accomplies à son égard.
\VS{17}Tous ceux qui avaient été avec Jésus, quand il appela Lazare du sépulcre et le ressuscita des morts, lui rendaient témoignage ;
\VS{18}et la foule alla au-devant de lui, parce qu’elle avait appris qu'il avait fait ce miracle.
\VS{19}Sur quoi les pharisiens dirent entre eux : Vous ne voyez pas que vous ne gagnez rien ? Voici, le monde va après lui.
\TextTitle{[Quelques Grecs cherchent à voir Jésus]}
\VS{20}Quelques Grecs du nombre de ceux qui étaient montés pour adorer Dieu pendant la fête,
\VS{21}s’adressèrent à Philippe, qui était de Bethsaïda de Galilée, et lui dirent avec instances : Seigneur ! Nous voudrions voir Jésus.
\VS{22}Philippe alla le dire à André, et André et Philippe le dirent à Jésus.
\TextTitle{[Jésus annonce sa crucifixion]}
\VS{23}Jésus leur répondit, disant : L'heure est venue où le Fils de l'homme doit être glorifié.
\VS{24}En vérité, en vérité je vous le dis : Si le grain de blé qui est tombé en la terre ne meurt, il reste seul ; mais s'il meurt, il porte beaucoup de fruits.
\VS{25}Celui qui aime sa vie la perdra ; et celui qui hait sa vie dans ce monde, la conservera pour la vie éternelle.
\VS{26}Si quelqu'un me sert, qu'il me suive ; et là où je serai, là aussi sera celui qui me sert ; et si quelqu'un me sert, mon Père l'honorera.
\VS{27}Maintenant mon âme est troublée. Et que dirai-je ? Ô Père, délivre-moi de cette heure ? Mais c'est pour cela que je suis venu jusqu’à cette heure.
\VS{28}Père glorifie ton Nom ! Alors une voix vint du ciel et dit : Je l'ai glorifié, et je le glorifierai encore.
\VS{29}Et la foule qui était là, et qui avait entendu cette voix, disait que c'était un coup de tonnerre ; les autres disaient : Un ange lui a parlé.
\VS{30}Jésus prit la parole et dit : Ce n’est pas à cause de moi que cette voix s’est fait entendre ; c’est à cause de vous.
\VS{31}Maintenant est venu le jugement de ce monde ; maintenant le prince de ce monde sera jeté dehors.
\VS{32}Et moi, quand je serai élevé de la terre, j’attirerai tous les hommes à moi.
\VS{33}En parlant ainsi, il indiquait de quelle mort il devait mourir.
\VS{34}La foule lui répondit : Nous avons appris par la loi que le Christ demeure éternellement, comment donc dis-tu qu'il faut que le Fils de l'homme soit élevé ? Qui est ce Fils de l'homme ?
\VS{35}Alors Jésus leur dit : La Lumière est encore avec vous pour un peu de temps : Marchez pendant que vous avez la Lumière, de peur que les ténèbres ne vous surprennent ; car celui qui marche dans les ténèbres ne sait pas où il va.
\VS{36}Pendant que vous avez la Lumière, croyez en la Lumière, afin que vous soyez enfants de lumière. Jésus dit ces choses, puis il s'en alla, et se cacha de devant eux.
\VS{37}Malgré tant de miracles qu’il avait faits en leur présence, ils ne croyaient point en lui,
\VS{38}afin que s’accomplisse cette parole qui a été dite par Esaïe le prophète : Seigneur, qui a cru à notre parole, et à qui a été révélé le bras du Seigneur\FTNT{Es. 53:1.} ?
\VS{39}C'est pourquoi ils ne pouvaient pas croire, parce qu'Esaïe a dit encore :
\VS{40}Il a aveuglé leurs yeux, et il a endurci leur cœur, de peur qu'ils ne voient de leurs yeux, qu'ils ne comprennent du cœur, qu'ils ne se convertissent, et que je ne les guérisse\FTNT{Es. 6:9-10.}.
\VS{41}Esaïe dit ces choses quand il vit sa gloire, et qu'il parla de lui.
\VS{42}Cependant, même parmi les chefs, plusieurs crurent en lui ; mais ils ne le confessaient pas à cause des pharisiens, de peur d'être exclus de la synagogue.
\VS{43}Car ils aimèrent la gloire des hommes, plus que la gloire de Dieu.
\VS{44}Or Jésus s'écria et dit : Celui qui croit en moi, ne croit pas seulement en moi, mais en celui qui m'a envoyé.
\VS{45}Et celui qui me voit, voit celui qui m'a envoyé.
\VS{46}Je suis venu dans le monde pour en être la Lumière, afin que quiconque croit en moi ne demeure point dans les ténèbres.
\VS{47}Si quelqu'un entend mes paroles, et ne les garde point, ce n’est pas moi qui le juge ; car je ne suis point venu pour juger le monde, mais pour sauver le monde.
\VS{48}Celui qui me rejette et qui ne reçoit pas mes paroles, a son juge : La parole que j'ai annoncée sera celle qui le jugera au dernier jour.
\VS{49}Car je n'ai point parlé de moi-même, mais le Père qui m'a envoyé, m'a prescrit ce que je dois dire et annoncer.
\VS{50}Et je sais que son commandement est la vie éternelle ; les choses donc que je dis, je les dis comme mon Père me les a dites.
\TextTitle{[L'entretien de Jn. 13-14 eut lieu dans la chambre haute ; Mc. 14:14-16]}
\Chap{13}
\VerseOne{}Avant la fête de Pâque, Jésus sachant que son heure était venue de passer de ce monde au Père, et ayant aimé les siens, qui étaient dans le monde, il les aima jusqu'à la fin.
\TextTitle{[La dernière Pâque ; Jésus lave les pieds de ses disciples]
\\(Mt. 26:20-24 ; Mc. 14:17 ; Lu. 22:14,21-23)}
\VS{2}Pendant le souper, alors que le diable avait déjà mis dans le cœur de Judas Iscariot, fils de Simon, de le trahir,
\VS{3}Jésus sachant que le Père avait remis toutes choses entre ses mains, qu'il était venu de Dieu, et qu’il s'en allait à Dieu,
\VS{4}se leva de table, ôta ses vêtements, et prit un linge, dont il se ceignit.
\VS{5}Puis il mit de l'eau dans un bassin, et se mit à laver les pieds de ses disciples, et à les essuyer avec le linge dont il se ceignit.
\VS{6}Alors il vint à Simon Pierre, mais Pierre lui dit : Toi, Seigneur, tu me laves les pieds ?
\VS{7}Jésus répondit et lui dit : Tu ne comprends pas maintenant ce que je fais, mais tu le sauras dans la suite.
\VS{8}Pierre lui dit : Tu ne me laveras jamais les pieds ! Jésus lui répondit : Si je ne te lave pas, tu n'auras point de part avec moi.
\VS{9}Simon Pierre lui dit : Seigneur, non seulement mes pieds, mais aussi les mains et la tête.
\VS{10}Jésus lui dit : Celui qui est baigné n’a besoin que de se laver les pieds pour être entièrement pur ; vous êtes purs, mais non pas tous.
\VS{11}Car il savait qui était celui qui le trahirait ; c'est pourquoi il dit : Vous n'êtes pas tous purs.
\VS{12}Après qu'il leur eut lavé les pieds, il reprit ses vêtements, et s'étant remis à table, il leur dit : Comprenez-vous ce que je vous ai fait ?
\VS{13}Vous m'appelez Maître et Seigneur ; et vous dites bien, car je le suis.
\VS{14}Si donc moi, qui suis le Seigneur et le Maître, j'ai lavé vos pieds, vous devez aussi vous laver les pieds les uns des autres.
\VS{15}Car je vous ai donné un exemple, afin que vous fassiez comme je vous ai fait.
\VS{16}En vérité, en vérité je vous le dis : Le serviteur n'est pas plus grand que son maître ni l’apôtre plus grand que celui qui l'a envoyé.
\VS{17}Si vous savez ces choses, vous êtes heureux, pourvu que vous les pratiquiez.
\VS{18}Je ne parle pas de vous tous, je connais ceux que j'ai choisis. Mais il faut que l’Ecriture s’accomplisse : Celui qui mange le pain avec moi, a levé son talon contre moi\FTNT{Ps. 41:10.}.
\VS{19}Je vous dis ceci dès maintenant, avant que la chose arrive, afin que lorsqu’elle arrivera, vous croyiez que c'est moi que le Père a envoyé.
\VS{20}En vérité, en vérité je vous le dis : Celui qui reçoit celui que j’aurai envoyé, me reçoit ; et celui qui me reçoit, reçoit celui qui m'a envoyé.
\TextTitle{[Jésus annonce la trahison de Judas]
\\(Mt. 26:21-25 ; Mc. 14:18-21 ; Lu. 22:21-23)}
\VS{21}Ayant ainsi parlé, Jésus fut ému dans son esprit, et il déclara : En vérité, en vérité je vous le dis, l'un de vous me trahira.
\VS{22}Alors les disciples se regardaient les uns les autres, ne sachant de qui il parlait.
\VS{23}Un des disciples, celui que Jésus aimait, était à table couché sur le sein de Jésus.
\VS{24}Simon Pierre lui fit signe de demander qui était celui dont Jésus parlait.
\VS{25}Et ce disciple, s’étant penché sur la poitrine de Jésus, lui dit : Seigneur, qui est-ce ?
\VS{26}Jésus répondit : C'est celui à qui je donnerai le morceau trempé ; et ayant trempé le morceau, il le donna à Judas Iscariot, fils de Simon.
\VS{27}Après que Judas eut pris le morceau, Satan entra en lui. Jésus lui dit : Ce que tu fais, fais-le promptement.
\VS{28}Mais aucun de ceux qui étaient à table ne comprit pourquoi il lui avait dit cela.
\VS{29}Car quelques-uns pensaient que, comme Judas avait la bourse, Jésus voulait lui dire : Achète ce qui nous est nécessaire pour la Fête ; ou qu'il lui commandait de donner quelque chose aux pauvres.
\VS{30}Judas, ayant pris le morceau, sortit aussitôt. Il faisait nuit.
\VS{31}Lorsque Judas fut sorti, Jésus dit : Maintenant le Fils de l'homme est glorifié ; et Dieu est glorifié en lui.
\VS{32}Si Dieu est glorifié en lui, Dieu aussi le glorifiera en lui-même, et il le glorifiera bientôt.
\VS{33}Mes petits enfants, je suis encore pour un peu de temps avec vous ; vous me chercherez, mais comme j'ai dit aux Juifs : Vous ne pouvez pas venir où je vais, je vous le dis aussi maintenant.
\VS{34}Je vous donne un nouveau commandement : Aimez-vous les uns les autres. Comme je vous ai aimés, vous aussi, aimez-vous les uns les autres.
\VS{35}A ceci tous connaîtront que vous êtes mes disciples, si vous avez de l'amour les uns pour les autres.
\TextTitle{[Jésus annonce le reniement de Pierre]
\\(Mt. 26:30-35 ; Mc. 14:26-31 ; Lu. 22:31-34)}
\VS{36}Simon Pierre lui dit : Seigneur ! Où vas-tu ? Jésus lui répondit : Là où je vais, tu ne peux pas me suivre maintenant, mais tu me suivras plus tard.
\VS{37}Pierre lui dit : Seigneur ! Pourquoi ne puis-je pas te suivre maintenant ? J’exposerai ma vie pour toi.
\VS{38}Jésus lui répondit : Tu exposeras ta vie pour moi ? En vérité, en vérité je te le dis, le coq ne chantera pas, que tu ne m'aies renié trois fois.
\TextTitle{[Jésus réconforte les apôtres : Il reviendra vers eux]}
\Chap{14}
\VerseOne{}Que votre cœur ne se trouble point ; vous croyez en Dieu, croyez aussi en moi.
\VS{2}Il y a plusieurs demeures dans la maison de mon Père. Si cela n’était pas, je vous l’aurais dit ; je vais vous préparer une place.
\VS{3}Et quand je m'en serai allé, et que je vous aurai préparé une place, je reviendrai, et je vous prendrai avec moi ; afin que là où je suis, vous y soyez aussi.
\VS{4}Et vous savez où je vais, et vous en savez le chemin.
\VS{5}Thomas lui dit : Seigneur ! Nous ne savons point où tu vas, comment donc pouvons-nous en savoir le chemin ?
\VS{6}Jésus lui dit : Je suis le chemin, la vérité, et la vie ; nul ne vient au Père que par moi.
\TextTitle{[Le Père et le Fils sont un]}
\VS{7}Si vous me connaissiez, vous connaîtriez aussi mon Père ; mais dès maintenant vous le connaissez, et vous l'avez vu.
\VS{8}Philippe lui dit : Seigneur ! Montre-nous le Père, et cela nous suffit.
\VS{9}Jésus lui répondit : Je suis depuis si longtemps avec vous, et tu ne m'as pas connu ? Philippe ! Celui qui m'a vu a vu mon Père. Et comment dis-tu : Montre-nous le Père ?
\VS{10}Ne crois-tu pas que je suis dans le Père, et que le Père est en moi ? Les paroles que je vous dis, je ne les dis pas de moi-même ; mais le Père qui demeure en moi est celui qui fait les œuvres.
\VS{11}Croyez-moi, je suis dans le Père, et le Père est en moi, sinon croyez-moi à cause de ces œuvres.
\VS{12}En vérité, en vérité je vous le dis : Celui qui croit en moi fera les œuvres que je fais, et il en fera même de plus grandes que celles-ci, parce que je m'en vais vers mon Père.
\TextTitle{[Nouveau privilège par la prière]}
\VS{13}Et tout ce que vous demanderez en mon Nom, je le ferai ; afin que le Père soit glorifié dans le Fils.
\VS{14}Si vous demandez en mon Nom quelque chose, je le ferai.
\TextTitle{[Promesse quant à l'habitation de l'Esprit dans le coeur du croyant]}
\VS{15}Si vous m'aimez, gardez mes commandements.
\VS{16}Et moi, je prierai le Père, et il vous donnera un autre Consolateur, pour demeurer avec vous éternellement,
\VS{17}l'Esprit de vérité que le monde ne peut recevoir, parce qu'il ne le voit point, et qu'il ne le connaît point ; mais vous le connaissez, car il demeure avec vous, et il sera en vous.
\VS{18}Je ne vous laisserai pas orphelins, je viendrai vers vous.
\VS{19}Encore un peu de temps, et le monde ne me verra plus ; mais vous me verrez, parce que je vis, et vous aussi vous vivrez.
\VS{20}En ce jour-là vous connaîtrez que je suis en mon Père, que vous êtes en moi, et moi en vous.
\VS{21}Celui qui a mes commandements et qui les garde, c'est celui qui m'aime ; et celui qui m'aime sera aimé de mon Père ; je l'aimerai, et je me ferai connaître à lui.
\VS{22}Jude, non pas Iscariot, lui dit : Seigneur ! D’où vient que tu te feras connaître à nous, et non au monde ?
\VS{23}Jésus répondit et lui dit : Si quelqu'un m'aime, il gardera ma parole, et mon Père l'aimera, et nous viendrons à lui, et nous ferons notre demeure chez lui.
\VS{24}Celui qui ne m'aime point ne garde point mes paroles. Et la parole que vous entendez n'est point ma parole, mais c'est celle du Père qui m'a envoyé.
\VS{25}Je vous ai dit ces choses pendant que je demeure avec vous.
\VS{26}Mais le Consolateur, le Saint-Esprit, que le Père enverra en mon Nom, vous enseignera toutes choses, et il vous rappellera tout ce que je vous ai dit.
\TextTitle{[Christ donne sa paix]}
\VS{27}Je vous laisse la paix, je vous donne ma paix ; je ne vous la donne pas comme le monde la donne ; que votre cœur ne se trouble point, et ne s’alarme point.
\VS{28}Vous avez entendu que je vous ai dit : Je m'en vais, et je reviens à vous ; si vous m'aimiez, vous seriez certes joyeux de ce que j'ai dit : Je m'en vais au Père, car le Père est plus grand que moi.
\VS{29}Et maintenant je vous l'ai dit avant que cela soit arrivé, afin que quand il sera arrivé, vous croyiez.
\VS{30}Je ne parlerai plus guère avec vous ; car le prince de ce monde vient ; mais il n'a rien en moi.
\VS{31}Mais afin que le monde sache que j'aime le Père, et que je fais ce que le Père m'a commandé : Levez-vous, partons d'ici.
\TextTitle{[Le cep et les sarments]}
\Chap{15}
\VerseOne{}Je suis le vrai Cep\FTNT{Jésus est l’arbre de vie qui produit de bons fruits en nous, à condition que nous nous tenions loin de l’arbre de la connaissance du bien et du mal. Jésus, le vrai Cep, est la source de vie. La viabilité du sarment dépend de son attachement au Cep. Jésus a été pendu au bois (Ac 5:30), s’est chargé de nos malédictions (Ga. 3:13) et a été retranché à notre place.}, et mon Père est le Vigneron.
\VS{2}Il retranche tout sarment qui est en moi et qui ne porte pas de fruits ; et tout sarment qui porte du fruit, il l'émonde afin qu'il porte encore plus de fruits.
\VS{3}Vous êtes déjà purs à cause de la parole que je vous ai enseignée.
\VS{4}Demeurez en moi, et je demeurerai en vous ; comme le sarment ne peut de lui-même porter du fruit s'il ne demeure pas attaché au cep ; ainsi vous ne le pouvez pas non plus si vous ne demeurez pas en moi.
\VS{5}Je suis le Cep, et vous en êtes les sarments ; celui qui demeure en moi, et en qui je demeure porte beaucoup de fruits ; car hors de moi, vous ne pouvez rien produire.
\VS{6}Si quelqu'un ne demeure point en moi, il est jeté dehors comme le sarment, et il se sèche ; puis on l'amasse, on le met au feu, et il brûle.
\VS{7}Si vous demeurez en moi, et que mes paroles demeurent en vous, demandez tout ce que vous voudrez, et cela vous sera fait.
\VS{8}Si vous portez beaucoup de fruits, mon Père sera glorifié et vous serez alors mes disciples.
\VS{9}Comme le Père m'a aimé, ainsi je vous ai aimés, demeurez dans mon amour.
\VS{10}Si vous gardez mes commandements, vous demeurerez dans mon amour ; comme j'ai gardé les commandements de mon Père, et je demeure dans son amour.
\VS{11}Je vous ai dit ces choses afin que ma joie demeure en vous, et que votre joie soit parfaite.
\VS{12}C'est ici mon commandement : Aimez-vous les uns les autres comme je vous ai aimés.
\VS{13}Il n’y a pas de plus grand amour que de donner sa vie pour ses amis.
\VS{14}Vous serez mes amis, si vous faites tout ce que je vous commande.
\TextTitle{[Nouvelle intimité entre le Seigneur et les siens]}
\VS{15}Je ne vous appelle plus serviteurs, car le serviteur ne sait pas ce que fait son maître; mais je vous ai appelés mes amis, parce que je vous ai fait connaître tout ce que j'ai appris de mon Père.
\VS{16}Ce n'est pas vous qui m'avez choisi ; mais moi, je vous ai choisis, et je vous ai établis afin que vous alliez partout et que vous produisiez du fruit, et que votre fruit demeure ; afin que tout ce que vous demanderez au Père en mon Nom, il vous le donne.
\VS{17}Ce que je vous commande, c’est de vous aimer les uns les autres.
\TextTitle{[L'attitude du monde à l'égard des croyants en Christ]}
\VS{18}Si le monde vous hait, sachez qu’il m’a haï avant vous.
\VS{19}Si vous étiez du monde, le monde aimerait ce qui est à lui ; mais parce que vous n'êtes pas du monde, et que je vous ai choisis du milieu du monde, à cause de cela le monde vous hait.
\VS{20}Souvenez-vous de la parole que je vous ai dite : Le serviteur n'est pas plus grand que son maître ; s'ils m'ont persécuté, ils vous persécuteront aussi ; s'ils ont gardé ma parole, ils garderont aussi la vôtre.
\VS{21}Mais ils vous feront toutes ces choses à cause de mon Nom, parce qu'ils ne connaissent point celui qui m'a envoyé.
\VS{22}Si je n’étais pas venu, et que je ne leur avais point parlé, ils n'auraient point de péché, mais maintenant ils n'ont point d'excuse de leur péché.
\VS{23}Celui qui me hait, hait aussi mon Père.
\VS{24}Si je n’avais pas fait parmi eux les œuvres qu'aucun autre n'a faites, ils n'auraient point de péché ; mais maintenant ils les ont vues, et ils ont haï et moi et mon Père.
\VS{25}Mais cela est arrivé afin que s’accomplisse la parole qui est écrite dans leur loi : Ils m'ont haï sans cause\FTNT{Ps. 35:19 ; Ps. 69:5.}.
\VS{26}Mais quand le Consolateur sera venu, que je vous enverrai de la part de mon Père, l'Esprit de vérité qui procède de mon Père, il rendra témoignage de moi.
\VS{27}Et vous aussi, vous rendrez témoignage, car vous êtes dès le commencement avec moi.
\TextTitle{[Jésus avertit les siens de la persécution]
\\(Mt. 24:9-10 ; Lu. 21:16-19)}
\Chap{16}
\VerseOne{}Je vous ai dit ces choses, afin que vous ne soyez pas scandalisés.
\VS{2}Ils vous chasseront des synagogues ; et même l’heure vient où quiconque vous fera mourir, croira rendre un culte à Dieu.
\VS{3}Et ils vous feront ces choses, parce qu'ils n’ont connu ni le Père ni moi.
\VS{4}Je vous ai dit ces choses, afin que, lorsque l'heure sera venue, vous vous souveniez que je vous les ai dites ; je ne vous en ai pas parlé dès le commencement, parce que j'étais avec vous.
\VS{5}Mais maintenant je m'en vais vers celui qui m'a envoyé, et aucun de vous ne me demande : Où vas-tu ?
\VS{6}Mais parce que je vous ai dit ces choses, la tristesse a rempli votre cœur.
\TextTitle{[La triple activité de l'Esprit agit en faveur du monde]}
\VS{7}Toutefois je vous dis la vérité, il vous est avantageux que je m'en aille, car si je ne m'en vais pas, le Consolateur ne viendra pas vers vous ; mais si je m'en vais, je vous l'enverrai.
\VS{8}Et quand il sera venu, il convaincra le monde de péché, de justice, et de jugement :
\VS{9}du péché, parce qu'ils ne croient point en moi,
\VS{10}de justice, parce que je m'en vais à mon Père, et que vous ne me verrez plus ;
\VS{11}de jugement, parce que le prince de ce monde est déjà jugé.
\TextTitle{[Après son ascension, Christ continuera de révéler la verité par l'Esprit]}
\VS{12}J'ai encore beaucoup de choses à vous dire, mais vous ne pouvez pas les porter maintenant.
\VS{13}Mais quand le Consolateur sera venu, l'Esprit de vérité, il vous conduira dans toute la vérité ; car il ne parlera pas de lui-même, mais il dira tout ce qu'il aura entendu, et il vous annoncera les choses à venir.
\VS{14}Il me glorifiera, car il prendra ce qui est à moi, et vous l'annoncera.
\VS{15}Tout ce que mon Père a, est à moi ; c'est pourquoi j'ai dit qu'il prendra ce qui est à moi et qu’il vous l'annoncera.
\TextTitle{[Jésus parle de sa mort, de sa grandeur]}
\VS{16}Encore un peu de temps, et vous ne me verrez plus ; et après un peu de temps, vous me verrez, car je m'en vais à mon Père.
\VS{17}Quelques-uns de ses disciples dirent entre eux : Qu'est-ce qu'il nous dit : Encore un peu de temps, et vous ne me verrez plus ; et un peu de temps après, vous me verrez, car je m'en vais à mon Père ?
\VS{18}Ils disaient donc : Que signifient ces mots : Encore un peu de temps ? Nous ne comprenons pas ce qu'il dit.
\VS{19}Jésus sachant qu'ils voulaient l’interroger, leur dit : Vous vous demandez entre vous sur ce que j'ai dit : Encore un peu de temps, et vous ne me verrez plus, et un peu de temps après, vous me verrez.
\VS{20}En vérité, en vérité je vous le dis : Vous pleurerez et vous vous lamenterez, et le monde se réjouira ; vous serez, dis-je, attristés ; mais votre tristesse sera changée en joie.
\VS{21}La femme, lorsqu’elle enfante, éprouve de la tristesse, parce que son heure est venue ; mais, lorsqu’elle a donné le jour à l’enfant, elle ne se souvient plus de la souffrance, à cause de ce qu’un homme est né dans le monde.
\VS{22}Vous donc aussi, vous êtes maintenant dans la tristesse ; mais je vous reverrai encore, et votre cœur se réjouira, et personne ne vous ôtera votre joie.
\VS{23}En ce jour-là, vous ne m'interrogerez plus sur rien. En vérité, en vérité je vous le dis : Tout ce que vous demanderez au Père en mon Nom, il vous le donnera.
\VS{24}Jusqu'à présent vous n'avez rien demandé en mon Nom ; demandez, et vous recevrez, afin que votre joie soit parfaite.
\VS{25}Je vous ai dit ces choses en paraboles. Mais l'heure vient où je ne vous parlerai plus en paraboles ; mais je vous parlerai ouvertement de mon Père.
\VS{26}En ce jour-là, vous demanderez des grâces en mon Nom, et je ne vous dis pas que je prierai le Père pour vous ;
\VS{27}car le Père lui-même vous aime, parce que vous m'avez aimé, et que vous avez cru que je suis sorti de Dieu.
\VS{28}Je suis sorti du Père, et je suis venu dans le monde ; maintenant je quitte le monde, et je m'en vais au Père.
\VS{29}Ses disciples lui dirent : Voici, maintenant tu parles ouvertement, et tu n'uses plus de paraboles.
\VS{30}Maintenant nous savons que tu sais toutes choses\FTNT{Jésus est omniscient. Il s’est lui-même présenté à l’apôtre Jean comme celui qui est, qui était et qui sera (Ap. 1:7-8).}, et que tu n'as pas besoin que quelqu’un t'interroge ; à cause de cela nous croyons que tu es sorti de Dieu.
\VS{31}Jésus leur répondit : Croyez-vous maintenant ?
\VS{32}Voici, l'heure vient, et elle est déjà venue, où vous serez dispersés chacun de son côté, et vous me laisserez seul ; mais je ne suis pas seul, car le Père est avec moi.
\VS{33}Je vous ai dit ces choses afin que vous ayez la paix en moi. Vous aurez des tribulations dans le monde, mais prenez courage, j'ai vaincu le monde.
\TextTitle{[La prière d'intercession de Christ, le souverain sacrificateur]}
\Chap{17}
\VerseOne{}Après avoir ainsi parlé, Jésus leva ses yeux au ciel, et dit : Père, l'heure est venue, glorifie ton Fils, afin que ton Fils te glorifie ;
\VS{2}selon que tu lui as donné pouvoir sur tous les hommes ; afin qu'il donne la vie éternelle à tous ceux que tu lui as donnés.
\VS{3}Or, la vie éternelle, ce qu’ils te connaissent, toi, le seul vrai Dieu, et celui que tu as envoyé, Jésus-Christ.
\VS{4}Je t'ai glorifié sur la terre, j'ai achevé l’œuvre que tu m'avais donnée à faire.
\VS{5}Et maintenant glorifie-moi, toi Père, auprès de toi, de la gloire que j’avais auprès de toi avant que le monde soit.
\VS{6}J'ai fait connaître ton Nom aux hommes que tu m'as donnés du milieu du monde ; ils étaient à toi, et tu me les as donnés ; et ils ont gardé ta parole.
\VS{7}Maintenant ils ont connu que tout ce que tu m'as donné vient de toi.
\VS{8}Car je leur ai donné les paroles que tu m'as données, et ils les ont reçues, et ils ont vraiment connu que je suis sorti de toi, et ils ont cru que tu m'as envoyé.
\VS{9}C’est pour eux que je prie ; je ne prie pas pour le monde, mais pour ceux que tu m'as donnés, parce qu'ils sont à toi.
\VS{10}Et tout ce qui est à moi est à toi, et ce qui est à toi est à moi ; et je suis glorifié en eux.
\VS{11}Et maintenant je ne suis plus dans le monde, et ils sont dans le monde ; et moi je vais à toi. Père saint, garde en ton Nom ceux que tu m'as donnés, afin qu'ils soient un comme nous sommes un.
\VS{12}Quand j'étais avec eux dans le monde, je les gardais en ton Nom ; j'ai gardé ceux que tu m'as donnés, et aucun d'eux ne s’est perdu, sinon le fils de perdition, afin que l'Ecriture soit accomplie.
\VS{13}Et maintenant je vais à toi, et je dis ces choses étant encore dans le monde, afin qu'ils aient ma joie parfaite en eux-mêmes.
\VS{14}Je leur ai donné ta parole, et le monde les a haïs, parce qu'ils ne sont pas du monde, comme moi je ne suis pas du monde.
\VS{15}Je ne te prie pas de les ôter du monde, mais de les préserver du mal.
\VS{16}Ils ne sont pas du monde, comme moi je ne suis pas du monde.
\VS{17}Sanctifie-les par ta vérité ; ta parole est la vérité.
\VS{18}Comme tu m'as envoyé dans le monde, ainsi je les ai envoyés dans le monde.
\VS{19}Et je me sanctifie moi-même pour eux, afin qu'eux aussi soient sanctifiés par la vérité.
\VS{20}Je ne prie pas seulement pour eux, mais aussi pour ceux qui croiront en moi par leur parole.
\VS{21}Afin que tous soient un, ainsi que toi, Père, tu es en moi, et moi en toi ; afin qu'eux aussi soient un en nous ; et que le monde croie que c'est toi qui m'as envoyé.
\VS{22}Je leur ai donné la gloire que tu m'as donnée, afin qu'ils soient un comme nous sommes un.
\VS{23}Je suis en eux, et toi en moi, afin qu'ils soient parfaitement un, et que le monde connaisse que c'est toi qui m'as envoyé, et que tu les aimes, comme tu m'as aimé.
\VS{24}Père, mon désir est que ceux que tu m'as donnés soient avec moi là où je suis, afin qu'ils contemplent la gloire que tu m'as donnée ; parce que tu m'as aimé avant la fondation du monde.
\VS{25}Père juste, le monde ne t'a point connu ; mais moi je t'ai connu, et ceux-ci ont connu que c'est toi qui m'as envoyé.
\VS{26}Et je leur ai fait connaître ton Nom, et je le leur ferai connaître, afin que l'amour dont tu m'as aimé soit en eux, et que je sois en eux.
\TextTitle{[Jésus à Gethsémané]
\\(Mt. 26:36-46 ; Mc. 14:32-42 ; Lu. 22:39-46)}
\Chap{18}
\VerseOne{}Après que Jésus eut dit ces choses, il s'en alla avec ses disciples au-delà du torrent de Cédron, où il y avait un jardin dans lequel il entra avec ses disciples.
\TextTitle{[Jésus trahi et arrêté]
\\Mt. 26:47-56 ; Mc. 14:43-50 ; Lu. 22:47-54)}
\VS{2}Or Judas, qui le trahissait, connaissait aussi ce lieu-là, car Jésus s'y était souvent assemblé avec ses disciples.
\VS{3}Judas donc, ayant pris la cohorte, et des huissiers qu’envoyèrent les principaux sacrificateurs et les pharisiens, s'en vint là avec des lanternes, des flambeaux, et des armes.
\VS{4}Jésus, sachant tout ce qui devait lui arriver, s'avança et leur dit : Qui cherchez-vous ?
\VS{5}Ils lui répondirent : Jésus de Nazareth. Jésus leur dit : Moi, Je suis\FTNT{~Moi, Je suis~ (~ego eimi~), ce qui fait écho au nom sous lequel Dieu s’était révélé à Moïse en Ex. 3:14.}. Et Judas qui le trahissait était aussi avec eux.
\VS{6}Or après que Jésus leur eut dit : Moi Je suis, ils reculèrent, et tombèrent par terre.
\VS{7}Il leur demanda une seconde fois : Qui cherchez-vous ? Et ils répondirent : Jésus de Nazareth.
\VS{8}Jésus répondit : Je vous ai dit que moi, Je suis ; si donc vous me cherchez, laissez aller ceux-ci.
\VS{9}Il dit cela afin que s’accomplisse la parole qu'il avait dite : Je n'ai perdu aucun de ceux que tu m'as donnés.
\TextTitle{[Malchus frappé par Pierre]}
\VS{10}Simon Pierre, qui avait une épée, la tira, frappa le serviteur du souverain sacrificateur, et lui coupa l'oreille droite. Ce serviteur s’appelait Malchus.
\VS{11}Mais Jésus dit à Pierre : Remets ton épée au fourreau : Ne boirai-je pas la coupe que le Père m'a donnée ?
\TextTitle{[Jésus conduit auprès du souverain sacrificateur]
\\(v. 27 ; Mt. 26:57-68 ; Mc. 14:53-65 ; Lu. 22:63-71)}
\VS{12}La cohorte, le tribun, et les huissiers des Juifs se saisirent alors de Jésus et le lièrent.
\VS{13}Et ils l'emmenèrent premièrement chez Anne, car il était le beau-père de Caïphe, qui était le souverain sacrificateur de cette année-là.
\VS{14}Et Caïphe était celui qui avait donné ce conseil aux Juifs, qu'il était avantageux qu'un seul homme meure pour le peuple.
\TextTitle{[Le triple reniement de Pierre]
\\(v. 25-27 ; Mt. 26:69-75 ; Mc. 14:66-72 ; Lu. 22:54-62)}
\VS{15}Simon Pierre, avec un autre disciple, suivait Jésus ; et ce disciple était connu du souverain sacrificateur, et il entra avec Jésus dans la cour du souverain sacrificateur.
\VS{16}Mais Pierre était dehors à la porte, et l'autre disciple, qui était connu du souverain sacrificateur, sortit dehors et parla à la portière, et il fit entrer Pierre.
\VS{17}Et la servante qui était la portière dit à Pierre : N'es-tu pas aussi des disciples de cet homme ? Il dit : Je n'en suis point.
\VS{18}Les serviteurs et les huissiers qui étaient là avaient allumé un feu, parce qu'il faisait froid, et ils se chauffaient ; Pierre aussi était avec eux, et se chauffait.
\VS{19}Et le souverain sacrificateur interrogea Jésus sur ses disciples et sur sa doctrine.
\VS{20}Jésus lui répondit : J’ai ouvertement parlé au monde ; j'ai toujours enseigné dans la synagogue et dans le temple, où les Juifs s'assemblent toujours, et je n'ai rien dit en secret.
\VS{21}Pourquoi m'interroges-tu ? Interroge ceux qui ont entendu ce que je leur ai dit ; voici, ils savent ce que j'ai dit.
\VS{22}Quand il eut dit ces choses, un des huissiers qui se tenait là, donna un coup de sa verge à Jésus en lui disant : Est-ce ainsi que tu réponds au souverain sacrificateur ?
\VS{23}Jésus lui répondit : Si j'ai mal parlé, explique-moi ce que j’ai dit de mal ; et si j'ai bien parlé, pourquoi me frappes-tu ?
\VS{24}Anne l’envoya lié à Caïphe, le souverain sacrificateur.
\VS{25}Simon Pierre était là, et se chauffait. On lui dit : N’es-tu pas aussi de ses disciples ? Il le nia et dit : Je n'en suis point.
\VS{26}Un des serviteurs du souverain sacrificateur, parent de celui à qui Pierre avait coupé l'oreille, dit : Ne t'ai-je pas vu dans le jardin avec lui ?
\VS{27}Mais Pierre le nia de nouveau, et aussitôt le coq chanta.
\TextTitle{[Jésus devant Pilate]
\\(Mt. 27:2,11-14 ; Mc. 15:1-5 ; Lu. 23:1-7,13-15)}
\VS{28}Ils conduisirent Jésus de chez Caïphe au Prétoire\FTNT{Le Prétoire était à l'origine le nom du quartier général de la légion romaine. Il s’agissait plus particulièrement de la tente du général en chef d'une armée.} ; c'était le matin. Mais ils n'entrèrent point eux-mêmes dans le Prétoire, afin de ne pas se souiller, et de pouvoir manger l'agneau de Pâque.
\VS{29}C'est pourquoi Pilate\FTNT{Ponce Pilate était le préfet procurateur de la province romaine de Judée au Ier siècle (de 26 à 36).} sortit vers eux, et leur dit : Quelle accusation portez-vous contre cet homme ?
\VS{30}Ils lui répondirent : Si ce n'était pas un malfaiteur, nous ne te l’aurions pas livré.
\VS{31}Alors Pilate leur dit : Prenez-le vous-mêmes, et jugez-le selon votre loi. Mais les Juifs lui dirent : Il ne nous est pas permis de mettre quelqu’un à mort.
\VS{32}C’était afin que s’accomplisse la parole que Jésus avait dite, lorsqu’il indiquait de quelle mort il devait mourir.
\VS{33}Pilate entra de nouveau dans le Prétoire, et ayant appelé Jésus, il lui dit : Es-tu le Roi des Juifs ?
\VS{34}Jésus lui répondit : Est-ce de toi-même que tu dis cela, ou d’autres te l’ont dit de moi ?
\VS{35}Pilate répondit : Suis-je Juif ? Ta nation et les principaux sacrificateurs t'ont livré à moi ; qu'as-tu fait ?
\VS{36}Jésus répondit : Mon Royaume n'est pas de ce monde ; si mon Royaume était de ce monde, mes serviteurs auraient combattu pour moi afin que je ne sois pas livré aux Juifs ; mais maintenant mon règne n'est point d'ici-bas.
\VS{37}Alors Pilate lui dit : Es-tu donc Roi ? Jésus répondit : Tu le dis, que je suis Roi ; je suis né pour cela, et c'est pour cela que je suis venu dans le monde, pour rendre témoignage à la vérité. Quiconque est de la vérité entend ma voix.
\VS{38}Pilate lui dit : Qu'est-ce que la vérité ? Et quand il eut dit cela, il sortit de nouveau vers les Juifs, et il leur dit : Je ne trouve aucun crime en lui.
\TextTitle{[Barabbas libéré et Jésus condamné]
\\(Mt. 27:15-21 ; Mc. 15:6-11 ; Lu. 23:18-19)}
\VS{39}Or, comme c’est parmi vous une coutume que je vous relâche un prisonnier à la fête de Pâque ; voulez-vous donc que je vous relâche le Roi des Juifs ?
\VS{40}Et tous s'écrièrent, disant : Non pas celui-ci, mais Barrabas ; or Barrabas était un brigand.
\TextTitle{[Jésus couronné d'épines]
\\(Mt. 27:-30 ; Mc. 15:16-18)}
\Chap{19}
\VerseOne{}Alors Pilate prit Jésus, et le fit battre de verges.
\VS{2}Les soldats tressèrent une couronne d'épines qu'ils posèrent sur sa tête, et le vêtirent d'un vêtement de pourpre.
\VS{3}Puis ils lui disaient : Roi des Juifs, nous te saluons ; et ils lui donnaient des coups avec leurs verges.
\TextTitle{[Pilate fait un ultime effort pour relâcher Jésus]
\\(Mt. 27:22-26 ; Mc. 15:12-15 ; Lu. 23:20-25)}
\VS{4}Pilate sortit de nouveau dehors, et leur dit : Voici, je vous l'amène dehors, afin que vous sachiez que je ne trouve aucun crime en lui.
\VS{5}Jésus donc sortit portant la couronne d'épines et le manteau de pourpre ; et Pilate leur dit : Voici l'homme.
\VS{6}Mais quand les principaux sacrificateurs et leurs huissiers le virent, ils s'écrièrent, en disant : Crucifie ! Crucifie ! Pilate leur dit : Prenez-le vous-mêmes et crucifiez-le, car je ne trouve point de crime en lui.
\VS{7}Les Juifs lui répondirent : Nous avons une loi, et selon notre loi il doit mourir, car il s'est fait Fils de Dieu.
\VS{8}Quand Pilate entendit cette parole, sa frayeur augmenta.
\VS{9}Et il rentra dans le Prétoire et dit à Jésus : D'où es-tu ? Mais Jésus ne lui donna point de réponse.
\VS{10}Et Pilate lui dit : Est-ce à moi que tu ne parles pas ? Ne sais-tu pas que j'ai le pouvoir de te crucifier, et que j’ai le pouvoir de te délivrer ?
\VS{11}Jésus lui répondit : Tu n'aurais aucun pouvoir sur moi s'il ne t‘avait été donné d'en haut ; c'est pourquoi celui qui m'a livré à toi, commet un plus grand péché.
\VS{12}Dès ce moment, Pilate cherchait à le délivrer ; mais les Juifs criaient en disant : Si tu le délivres, tu n'es pas ami de César ; car quiconque se fait Roi est contre César.
\VS{13}Pilate, ayant entendu ces paroles, amena Jésus dehors, et il siégea au tribunal, au lieu appelé le Pavé, et en hébreu Gabbatha.
\VS{14}C’était la préparation de la Pâque, et environ la sixième heure ; et Pilate dit aux Juifs : Voici votre Roi.
\VS{15}Mais ils criaient : Ôte, ôte, crucifie-le ! Pilate leur dit : Crucifierai-je votre Roi ? Les principaux sacrificateurs répondirent : Nous n'avons pas d'autre roi que César.
\TextTitle{[Jésus crucifié]
\\(Mt. 27:31-50 ; Mc. 15:19-37 ; Lu. 23:26-46)}
\VS{16}Alors il le leur livra pour être crucifié. Ils prirent donc Jésus et l'emmenèrent.
\VS{17}Jésus, portant sa croix, arriva au lieu appelé le Crâne, et en hébreu Golgotha,
\VS{18}où ils le crucifièrent, et deux autres avec lui, un de chaque côté, et Jésus au milieu.
\VS{19}Pilate fit un écriteau, qu'il mit sur la croix, où étaient écrits ces mots : Jésus de Nazareth, le Roi des juifs.
\VS{20}Beaucoup des Juifs lurent cet écriteau, parce que le lieu où Jésus était crucifié, était près de la ville ; et cet écriteau était en hébreu, en grec et en latin.
\VS{21}C'est pourquoi les principaux sacrificateurs des Juifs dirent à Pilate : N'écris pas le Roi des Juifs, mais que celui-ci a dit : Je suis le Roi des Juifs.
\VS{22}Pilate répondit : Ce que j'ai écrit, je l'ai écrit.
\VS{23}Les soldats, après avoir crucifié Jésus, prirent ses vêtements, et ils en firent quatre parts, une part pour chaque soldat. Ils prirent aussi sa tunique, qui était sans couture, d’un seul tissu depuis le haut jusqu'en bas.
\VS{24}Ils se dirent entre eux : Ne la déchirons pas, mais tirons au sort, pour savoir à qui elle sera. Et cela arriva ainsi, afin que s’accomplisse cette parole de l’Ecriture : Ils ont partagé entre eux mes vêtements, et ils ont tiré au sort ma tunique\FTNT{Ps. 22:19.} ; ainsi firent les soldats.
\VS{25}Près de la croix de Jésus se tenaient sa mère, et la sœur de sa mère, Marie femme de Cléopas, et Marie de Magdala.
\VS{26}Jésus voyant sa mère, et auprès d'elle le disciple qu'il aimait, il dit à sa mère : Femme, voilà ton Fils.
\VS{27}Puis il dit au disciple : Voilà ta mère ; et dès ce moment, ce disciple la prit chez lui.
\VS{28}Après cela, Jésus sachant que toutes choses étaient déjà accomplies, il dit, afin que l'Ecriture soit accomplie : J'ai soif.
\VS{29}Et il y avait là un vase plein de vinaigre. Les soldats en remplirent une éponge et la mirent au bout d'une branche d'hysope, et la lui présentèrent à la bouche.
\VS{30}Quand Jésus eut pris le vinaigre, il dit : Tout est accompli\FTNT{La fin de la période de la première alliance n’a pas eu lieu à la naissance du Seigneur. En effet, Ga. 4:4 nous dit que Jésus est né sous la loi de Moïse et le récit des quatre évangiles atteste que depuis sa naissance jusqu’à sa mort, Jésus a scrupuleusement respecté et accompli toute la loi. En effet, il a lui-même dit : ~Ne croyez pas que je sois venu abolir la loi ou les prophètes ; je ne suis pas venu les abolir, mais les accomplir.~ (Mt. 5:17). Ainsi, durant son ministère terrestre, le Seigneur demandait à ce qu’on applique la loi (Mt. 8:4 ; Mt. 23:23 ; Lu. 17:11-14) tout en préparant ses disciples à la nouvelle alliance. L’évangile de Matthieu nous relate un événement capital qui a eu lieu juste après la mort du Seigneur : ~Alors Jésus, poussa de nouveau un grand cri, et rendit l'esprit. Et voici, le voile du temple se déchira en deux, depuis le haut jusqu'en bas ; et la terre trembla, et les pierres se fendirent.~ (Mt. 27:50-51). Il convient de rappeler que le temple était divisé en trois parties : le Parvis, le lieu Saint et le Saint des saints. Le Parvis était accessible à tout le monde, y compris aux non-Juifs. Le lieu Saint n’était accessible qu’aux lévites. La troisième partie, le Saint des saints, n’était accessible qu’au souverain sacrificateur. Le lieu Saint était séparé du Saint des saints par un voile qui symbolisait le mur d’inimitié (Es. 59:2 ; Ro. 3:23) qui sépare l’homme pécheur de la présence de Dieu, représentée dans le temple par l’arche de l’alliance. Ce voile n’avait rien d’un tissu léger et vaporeux, mais il ressemblait davantage à un épais tapis, opaque et surtout très résistant, et donc très difficile à déchirer. Le souverain sacrificateur rentrait seulement une fois par an dans le Saint des saints pour y offrir le sacrifice d’expiation pour le peuple ainsi que pour lui-même (Lé. 16 ; Hé. 9:7). Toutefois, la nécessité de répéter ce sacrifice chaque année prouvait que les exigences de la justice divine n’étaient pas pleinement satisfaites (Hé. 10:3-4). L’auteur de l’épître aux Hébreux nous apprend que le voile symbolisait également le corps physique de Christ (Hé. 10:19-20). Ainsi, lorsque le Seigneur a succombé à ses meurtrissures, le fameux voile s’est déchiré du haut jusqu’au bas. Or tant que le voile subsistait, l’accès à la présence de Dieu était fermé (Hé. 9:8). La déchirure atteste donc qu’en Christ, nous pouvons désormais nous approcher avec assurance du trône de Dieu, sans autre médiateur que le Seigneur lui-même (1 Ti. 2:5). ~Or là où les péchés sont pardonnés, il n'y a plus d’offrande pour le péché. Ainsi donc, mes frères, nous avons la liberté d'entrer dans le Saint des saints au moyen du sang de Jésus, qui est le chemin nouveau et vivant qu'il nous a frayé au travers du voile, c’est-à-dire de sa chair. Et ayant un Souverain Sacrificateur établi sur la maison de Dieu, approchons-nous de lui avec un cœur sincère et une foi inébranlable, ayant les cœurs purifiés d’une mauvaise conscience, et le corps lavé d’une eau pure. Retenons sans fléchir la profession de notre espérance, car celui qui nous a fait la promesse est fidèle.~ Hé. 10:18-23. Jésus-Christ est notre Pâque (1 Co. 5:5-8), il est le sacrifice parfait qui a expié nos péchés une fois pour toutes (Hé. 10:10). Par conséquent, il est celui à qui nous devons nous adresser pour recevoir pardon, miséricorde et compassion. ~Tout est accompli~. En s’écriant de la sorte, Jésus-Christ a proclamé la fin de l'ancienne alliance. En effet, la loi a été promulguée par Moïse, mais la grâce et la vérité sont venues par Jésus-Christ (Jn. 1:17). Toutefois, la nouvelle alliance n’a réellement débuté qu’à la Pentecôte avec l’effusion du Saint-Esprit. Voir commentaire en Actes 2.} ; et ayant baissé la tête, il rendit l'esprit.
\TextTitle{[Fin de la première Alliance]
\\(Mt.27:50-51 ; Mc. 15:37-38 ; Lu. 23:45-46)}
\VS{31}De peur que les corps ne restent sur la croix pendant le sabbat, car c’était la préparation, et ce jour de sabbat était un grand jour, les Juifs demandèrent à Pilate qu’on rompe les jambes aux crucifiés, et qu’on les enlève.
\VS{32}Les soldats vinrent donc, et ils rompirent les jambes au premier, et de même à l'autre qui était crucifié avec lui.
\VS{33}Puis étant venus à Jésus, et voyant qu'il était déjà mort, ils ne lui rompirent point les jambes ;
\VS{34}mais un des soldats lui perça le côté avec une lance, et aussitôt il sortit du sang et de l'eau.
\VS{35}Celui qui l'a vu en a rendu témoignage, et son témoignage est digne de foi ; et il sait qu'il dit vrai, afin que vous le croyiez.
\VS{36}Ces choses sont arrivées, afin que l’Ecriture soit accomplie : Aucun de ses os ne sera brisé\FTNT{Ps. 34:21 ; Ex. 12:46 ; No 9:12.}.
\VS{37}Et encore une autre Ecriture, qui dit : Ils verront celui qu'ils ont percé\FTNT{Za. 12:10.}.
\TextTitle{[Jésus enseveli]
\\(Mt. 27:57-66 ; Mc. 15:42-47 ; Lu. 23:50-56)}
\VS{38}Après ces choses, Joseph d'Arimathée, qui était disciple de Jésus, mais en secret parce qu'il craignait les Juifs, demanda à Pilate la permission d’enlever le corps de Jésus ; et Pilate le lui ayant permis, il vint et prit le corps de Jésus.
\VS{39}Nicodème, qui auparavant était allé de nuit vers Jésus, vint aussi, apportant un mélange de myrrhe et d'aloès d'environ cent livres.
\VS{40}Et ils prirent le corps de Jésus, et l'enveloppèrent de linges avec des aromates, comme les Juifs ont coutume d'ensevelir.
\VS{41}Or il y avait un jardin dans le lieu où Jésus fut crucifié, et dans le jardin un sépulcre neuf, où personne n'avait encore été mis.
\VS{42}Ce fut là qu’ils déposèrent Jésus, à cause de la préparation des Juifs, parce que le sépulcre était proche.
\TextTitle{[Déroulement des événements du jour de la résurection]
\\(Mt. 28:1-15 ; Mc. 16:1-14 ; Lu. 24:1-32)}
\Chap{20}
\VerseOne{}Le premier jour de la semaine, Marie de Magdala se rendit dès le matin au sépulcre, comme il faisait encore obscur ; et elle vit que la pierre était ôtée du sépulcre.
\VS{2}Elle courut vers Simon Pierre et vers l'autre disciple que Jésus aimait, et elle leur dit : Ils ont enlevé le Seigneur du sépulcre, et nous ne savons pas où ils l’ont mis.
\VS{3}Alors Pierre partit avec l'autre disciple, et ils s'en allèrent au sépulcre.
\VS{4}Ils couraient tous deux ensemble, mais l'autre disciple courait plus vite que Pierre, et il arriva le premier au sépulcre.
\VS{5}Et s'étant baissé, il vit les linges à terre ; mais il n'y entra point.
\VS{6}Alors Simon Pierre qui le suivait, arriva, et entra dans le sépulcre, et vit les linges à terre,
\VS{7}et le linge qu’on avait mis sur la tête de Jésus, non pas avec les bandes, mais plié dans un lieu à part.
\VS{8}Alors l'autre disciple, qui était arrivé le premier au sépulcre, entra aussi, il vit, et il crut.
\VS{9}Car ils ne comprenaient pas encore que, selon l'Ecriture, Jésus devait ressusciter des morts.
\VS{10}Et les disciples s'en retournèrent chez eux.
\TextTitle{[Jésus apparait aux disciples, Thomas étant absent]
\\(Mc. 16:14 ; Lu. 24:33-49)}
\VS{11}Mais Marie se tenait près du sépulcre dehors, et pleurait. Comme elle pleurait, elle se baissa dans le sépulcre,
\VS{12}et elle vit deux anges vêtus de blanc, assis à la place où avait été couché le corps de Jésus, l'un à la tête et l'autre aux pieds.
\VS{13}Ils lui dirent : Femme, pourquoi pleures-tu ? Elle leur dit : Parce qu'on a enlevé mon Seigneur, et je ne sais point où on l'a mis.
\VS{14}En disant cela, elle se retourna, et elle vit Jésus qui était là, mais elle ne savait pas que c’était Jésus.
\VS{15}Jésus lui dit : Femme, pourquoi pleures-tu ? Qui cherches-tu ? Elle, pensant que c’était le jardinier, lui dit : Seigneur, si c’est toi qui l'as emporté, dis-moi où tu l'as mis, et je le prendrai.
\VS{16}Jésus lui dit : Marie ! Et elle se retourna et lui dit : Rabbouni ! C’est-à-dire, mon Maître !
\VS{17}Jésus lui dit : Ne me touche pas ; car je ne suis point encore monté vers mon Père. Mais va trouver mes frères, et dis-leur que je monte vers mon Père et votre Père, vers mon Dieu et votre Dieu.
\VS{18}Marie de Magdala alla annoncer aux disciples qu'elle avait vu le Seigneur, et qu'il lui avait dit ces choses.
\VS{19}Le soir de ce jour, qui était le premier de la semaine, les portes du lieu où les disciples étaient assemblés, à cause de la crainte qu'ils avaient des Juifs, étaient fermées. Jésus vint, se présenta au milieu d'eux, et il leur dit : Que la paix soit avec vous !
\VS{20}Et quand il leur eut dit cela, il leur montra ses mains et son côté. Les disciples furent dans la joie en voyant le Seigneur.
\VS{21}Jésus leur dit de nouveau : Que la paix soit avec vous ! Comme mon Père m'a envoyé, ainsi je vous envoie.
\VS{22}Après ces paroles, il souffla sur eux, et leur dit : Recevez le Saint-Esprit.
\VS{23}Ceux à qui vous pardonnerez les péchés, ils leur seront pardonnés ; et ceux à qui vous les retiendrez, ils leur seront retenus.
\TextTitle{[Jésus apparait aux disciples, Thomas étant présent]}
\VS{24}Thomas, appelé Didyme, l'un des douze, n'était pas avec eux quand Jésus vint.
\VS{25}Les autres disciples lui dirent : Nous avons vu le Seigneur. Mais il leur dit : Si je ne vois pas les marques des clous dans ses mains, et si je ne mets pas mon doigt où étaient les clous, et si je ne mets pas ma main dans son côté, je ne le croirai point.
\VS{26}Huit jours après, les disciples étaient de nouveau dans la maison, et Thomas se trouvait avec eux. Jésus vint, les portes étant fermées, se présenta au milieu d'eux, et il leur dit : Que la paix soit avec vous !
\VS{27}Puis il dit à Thomas : Mets ton doigt ici, et regarde mes mains, avance aussi ta main, et mets-la dans mon côté ; et ne sois point incrédule, mais crois.
\VS{28}Et Thomas répondit et lui dit : Mon Seigneur, et mon Dieu !
\VS{29}Jésus lui dit : Parce que tu m'as vu, Thomas, tu as cru. Heureux sont ceux qui n'ont pas vu et qui ont cru.
\TextTitle{[But de l'Evangile selon Jean]}
\VS{30}Jésus fit encore, en présence de ses disciples, beaucoup d’autres miracles qui ne sont pas écrits dans ce livre.
\VS{31}Mais ces choses sont écrites afin que vous croyiez que Jésus est le Christ, le Fils de Dieu, et qu'en croyant vous ayez la vie par son Nom.
\TextTitle{[Jésus apparait à sept apôtres au bord de la mer de Galilée]}
\Chap{21}
\VerseOne{}Après cela, Jésus se montra de nouveau à ses disciples, près de la mer de Tibériade. Et voici de quelle manière il se montra.
\VS{2}Simon Pierre, Thomas, appelé Didyme, Nathanaël, de Cana en Galilée, les fils de Zébédée, et deux autres disciples de Jésus étaient ensemble.
\TextTitle{[Christ et notre service :
\\a. Le service de la volonté propre, sous des directives humaines]}
\VS{3}Simon Pierre leur dit : Je vais pêcher. Ils lui dirent : Nous allons aussi avec toi. Ils partirent et montèrent dans une barque ; mais ils ne prirent rien cette nuit-là.
\VS{4}Le matin étant venu, Jésus se trouva sur le rivage ; mais les disciples ne savaient pas que c’était Jésus.
\TextTitle{[b. Inutilité du service de la volonté propre]}
\VS{5}Jésus leur dit : Mes enfants, avez-vous quelque petit poisson à manger ? Ils lui répondirent : Non.
\TextTitle{[c. Résultat du service sous les directives de Christ]}
\VS{6}Et il leur dit : Jetez le filet du côté droit de la barque, et vous trouverez. Ils le jetèrent donc, et ils ne pouvaient plus le retirer à cause de la grande quantité de poissons.
\VS{7}Alors le disciple que Jésus aimait dit à Pierre : C’est le Seigneur. Et quand Simon Pierre eut entendu que c'était le Seigneur, il mit sa tunique et sa ceinture parce qu'il était nu, et il se jeta dans la mer.
\VS{8}Les autres disciples vinrent dans la barque, car ils n'étaient pas loin de terre, mais seulement à environ deux cents coudées, traînant le filet de poissons.
\VS{9}Lorsqu’ils furent descendus à terre, ils virent de la braise, et du poisson dessus, et du pain.
\VS{10}Jésus leur dit : Apportez des poissons que vous venez maintenant de prendre.
\VS{11}Simon Pierre monta et tira le filet à terre, plein de cent cinquante-trois grands poissons ; et quoiqu'il y en eût tant, le filet ne se rompit point.
\TextTitle{[d) Les ressources de Christ pour ses serviteurs]
\\(Lu. 22:35 ; Ph. 4:19)}
\VS{12}Jésus leur dit : Venez et mangez. Et aucun de ses disciples n'osait lui demander : Qui es-tu ? Sachant que c'était le Seigneur.
\VS{13}Jésus donc vint, et prit du pain, et leur en donna ; il fit de même du poisson aussi.
\VS{14}C’était déjà la troisième fois que Jésus se montrait à ses disciples depuis qu’il était ressuscité des morts.
\TextTitle{[e) La charité, seul motif valable pour le vrai service]
\\(1 Co. 13 ; 2 Co. 5:14 ; Ap. 2:4-5)}
\VS{15}Après qu'ils eurent mangé, Jésus dit à Simon Pierre : Simon fils de Jonas, m'aimes-tu plus que ne m’aiment ceux-ci ? Il lui répondit : Oui, Seigneur ! Tu sais que je t'aime. Il lui dit : Pais mes agneaux.
\VS{16}Il lui dit encore : Simon fils de Jonas, m'aimes-tu ? Il lui répondit : Oui, Seigneur ! Tu sais que je t'aime. Il lui dit : Pais mes brebis.
\VS{17}Il lui dit pour la troisième fois : Simon fils de Jonas, m'aimes-tu ? Pierre fut attristé de ce qu'il lui avait dit pour la troisième fois : M'aimes-tu ? Et il lui répondit : Seigneur, tu sais toutes choses, tu sais que je t'aime. Jésus lui dit : Pais mes brebis.
\TextTitle{[f) Le Maître révèle à Pierre qu'il Lui appartient de fixer le temps et la forme de sa mort]}
\VS{18}En vérité, en vérité je te le dis : Quand tu étais plus jeune, tu te ceignais toi-même, et tu allais où tu voulais ; mais quand tu seras vieux, tu étendras tes mains, et un autre te ceindra, et te mènera où tu ne voudras pas.
\VS{19}Il dit cela pour indiquer par quelle mort Pierre glorifierait Dieu. Et ayant ainsi parlé, il lui dit : Suis-moi.
\TextTitle{[g) Tous ses serviteurs ne mourront pas]
\\(1 Co. 15:51-52 ; 1 Th. 4:14-18)}
\VS{20}Pierre se retournant, vit venir après eux le disciple que Jésus aimait, celui qui pendant le souper s'était penché sur la poitrine de Jésus et avait dit : Seigneur, qui est celui qui te trahit ?
\VS{21}Quand donc Pierre le vit, il dit à Jésus : Seigneur, et celui-ci, que lui arrivera-t-il ?
\VS{22}Jésus lui dit : Si je veux qu'il demeure jusqu'à ce que je vienne, que t'importe ? Toi, suis-moi.
\VS{23}Là-dessus, le bruit courut parmi les frères que ce disciple ne mourrait point. Cependant Jésus ne lui avait pas dit : Il ne mourra point ; mais : Si je veux qu'il demeure jusqu'à ce que je vienne, que t'importe ?
\VS{24}C'est ce disciple qui rend témoignage de ces choses, et qui les a écrites. Et nous savons que son témoignage est digne de foi.
\VS{25}Jésus a fait encore beaucoup d’autres choses. Si on les écrivait en détail, je ne pense pas que le monde même pourrait contenir les livres qu’on écrirait. AMEN !
\PPE{}
\end{multicols}

%\addcontentsline{toc}{chapter}{Testament de Jésus}\clearpage
%\clearpage\ShortTitle{Ac.}\BookTitle{Actes}\BFont
\noindent\hrulefill
{\footnotesize
\textit{
\bigskip
{\centering{}
\\Auteur~: Luc
\\Thème~: Les missions du 1er siècle
\\Date de rédaction~: Env. 60 ap. J.-C.\\}
}
\textit{
\\D'origine grecque, Luc fut l'auteur du livre communément appelé «~actes des apôtres~» et de l'évangile éponyme tous deux adressés à Théophile. Ce livre retrace la genèse de l'Eglise, de l'ascension de Jésus à la Pentecôte, de la prédication vivante et fructueuse de Pierre à la conversion de Paul, jusqu'au voyage de celui-ci à Rome en tant que prisonnier. On y découvre des apôtres déterminés, des ouvriers de Christ qui acceptèrent de subir l'humiliation et la persécution par amour de la vérité. Sont également présentés des hommes et des femmes qui - touchés par la simplicité de l'Evangile du Royaume - se convertirent puis se firent baptiser.
\\Bien plus qu'un recueil relatant de banales manifestations, ce livre est avant tout celui des actes du Saint-Esprit. Il témoigne de la résurrection et de la puissance de Jésus-Christ manifestée au travers de son Corps. Il retrace l'origine et le développement du premier réveil après Jésus-Christ, qui fut un véritable bouleversement au sein d'un empire en proie à l'impiété et à l'idolâtrie.\bigskip
}
}
\par\nobreak\noindent\hrulefill
\begin{multicols}{2}
\Chap{1}
\TextTitle{Introduction~: le Messie ressuscité parle des choses qui concernent le Royaume de Dieu pendant quarante jours}
\VerseOne{}Nous avons rempli le premier traité, ô Théophile~! De toutes les choses que Jésus a faites et enseignées, 
\VS{2}jusqu'au jour où il fut élevé au ciel, après avoir donné par le Saint-Esprit, ses ordres aux apôtres qu'il avait élus.
\VS{3}A qui aussi, après avoir souffert, il se présenta lui-même vivant, avec plusieurs preuves assurées, étant vu par eux pendant quarante jours, et leur parlant des choses qui concernent le Royaume de Dieu.
\VS{4}Et les ayant assemblés, il leur ordonna de ne pas partir de Jérusalem, mais d'attendre la promesse du Père, ce que vous avez entendu de moi~;
\VS{5}car Jean a baptisé d'eau, mais vous serez baptisés du Saint-Esprit dans peu de jours.
\VS{6}Eux donc étant assemblés, l'interrogèrent, disant~: Seigneur, est-ce en ce temps-ci que tu rétabliras le royaume d'Israël~?
\VS{7}Mais il leur dit~: Ce n'est pas à vous de connaître les temps et les moments que le Père a fixés de sa propre autorité.
\TextTitle{La puissance du Saint-Esprit pour évangéliser les nations\FTNTT{\vref{Mt. 28:18-20}~; \vref{Mc. 16:15-18}~; \vref{Lu. 24:47-48}~; \vref{Jn. 20:21-22}.}}
\VS{8}Mais vous recevrez la puissance du Saint-Esprit qui viendra sur vous, et vous serez mes témoins, tant à Jérusalem que dans toute la Judée, et la Samarie, et jusqu'aux extrémités de la terre.
\VS{9}Et ayant dit ces choses, il fut élevé, comme ils le regardaient, une nuée le prit et l'emporta de devant leurs yeux.
\TextTitle{Promesse du retour de Jésus}
\VS{10}Et comme ils avaient les yeux fixés vers le ciel, à mesure qu'il s'en allait, voici, deux hommes en vêtements blancs se présentèrent devant eux,
\VS{11}et leur dirent~: Hommes Galiléens, pourquoi vous arrêtez-vous à regarder au ciel~? Ce Jésus qui a été élevé du milieu de vous au ciel, en descendra de la même manière que vous l'avez contemplé montant au ciel\FTNT{Jésus-Christ est monté au ciel depuis la Montagne des Oliviers et lors de son retour, ses pieds se poseront sur cette montagne. Voir \vref{Za. 14}.}.
\TextTitle{Attente du Saint-Esprit promis}
\VS{12}Alors ils s'en retournèrent à Jérusalem de la montagne appelée la Montagne des Oliviers, qui est près de Jérusalem, le chemin d'un sabbat\FTNT{Chemin de sabbat~: C'est la distance qu'il est permis à un juif de parcourir le jour de sabbat (\vref{Ex. 16:29}). Elle correspond à deux mille coudées ou 1100 m.}.
\VS{13}Et quand ils furent entrés dans la ville, ils montèrent dans une chambre haute où demeuraient Pierre et Jacques, Jean et André, Philippe et Thomas, Barthélemy et Matthieu, Jacques, fils d'Alphée, et Simon le zélote, et Jude, frère de Jacques.
\VS{14}Tous ceux-ci, d'un commun accord, persévéraient dans la prière et dans la supplication avec les femmes, avec Marie, mère de Jésus, et avec ses frères.
\TextTitle{Matthias désigné apôtre pour remplacer Judas}
\VS{15}Et en ces jours-là, Pierre se leva au milieu des disciples, qui étaient là assemblés au nombre d'environ cent vingt personnes, et il leur dit~:
\VS{16}Hommes frères, il fallait que s'accomplisse ce qui a été écrit, ce que le Saint-Esprit a annoncé d'avance par la bouche de David, au sujet de Judas, qui a été le guide de ceux qui ont saisi Jésus.
\VS{17}Car il était compté parmi nous, et il avait reçu en partage ce même service.
\VS{18}Mais après avoir acquis un champ avec le salaire du crime qui lui avait été donné, il est tombé, s'est rompu par le milieu, et toutes ses entrailles ont été répandues.
\VS{19}Et ceci a été connu de tous les habitants de Jérusalem, de sorte que ce champ a été appelé dans leur propre langue Hakeldama, c'est-à-dire le champ du sang.
\VS{20}Car il est écrit dans le livre des Psaumes~: Que sa demeure soit déserte, que personne ne l'habite\FTNT{\vref{Ps. 69:26}.}, et qu'un autre prenne sa charge\FTNT{Charge~: Du grec «~episkope~», il s'agit de la fonction d'un ancien. \vref{Ps. 109:8}.}.
\VS{21}Il faut donc que d'entre ces hommes qui se sont assemblés avec nous pendant tout le temps que le Seigneur Jésus a vécu entre nous,
\VS{22}en commençant depuis le baptême de Jean jusqu'au jour où il a été enlevé du milieu de nous, qu'il y en ait un qui soit témoin avec nous de sa résurrection.
\VS{23}Et ils en présentèrent deux~: Joseph, appelé Barsabbas, surnommé Justus, et Matthias.
\VS{24}Et en priant, ils dirent~: Toi, Seigneur, qui connais les cœurs de tous, désigne lequel de ces deux tu as choisi,
\VS{25}afin qu'il prenne part à ce service et à cet apostolat que Judas a abandonné pour aller en son lieu.
\VS{26}Puis ils les tirèrent au sort, et le sort tomba sur Matthias, qui, d'une commune voix, fut mis au rang des onze apôtres.
\Chap{2}
\TextTitle{Effusion de l'Esprit à la Pentecôte~; naissance de l'Eglise\FTNT{\vref{Joë. 2:32}.}}
\VerseOne{}Et comme le jour de la Pentecôte s'accomplissait, ils étaient tous ensemble dans un même lieu.
\VS{2}Et il se fit tout à coup un bruit du ciel, comme est le bruit d'un vent qui souffle avec véhémence, et il remplit toute la maison où ils étaient assis.
\VS{3}Et il leur apparut des langues divisées, comme de feu, qui se posèrent sur chacun d'eux.
\VS{4}Et ils furent tous remplis du Saint-Esprit, et commencèrent à parler des langues étrangères selon que l'Esprit leur donnait de parler.
\VS{5}Or il y avait à Jérusalem des Juifs qui y séjournaient, hommes pieux, de toute nation qui est sous le ciel.
\VS{6}Et ce bruit s'étant répandu, une multitude vint ensemble, et fut confondue de ce que chacun les entendait parler dans sa propre langue. 
\VS{7}Ils en étaient donc tout surpris, et s'en étonnaient, disant l'un à l'autre~: Voici, tous ceux-ci qui parlent ne sont-ils pas Galiléens~?
\VS{8}Comment donc chacun de nous les entendons-nous parler la propre langue du pays où nous sommes nés~? 
\VS{9}Parthes, Mèdes, Elamites, et ceux qui habitent la Mésopotamie, la Judée, la Cappadoce, le Pont, l'Asie,
\VS{10}la Phrygie, la Pamphylie, l'Egypte, le territoire de la Libye qui est près de Cyrène, et ceux qui sont venus de Rome~? Juifs et Prosélytes,
\VS{11} Crétois et Arabes, comment les entendons-nous parler chacun dans notre langue des merveilles de Dieu~?
\VS{12}Ils étaient donc tout étonnés, et ils ne savaient que penser, disant l'un à l'autre~: Que veut dire ceci~?
\VS{13}Mais les autres se moquaient, et disaient~: C'est qu'ils sont pleins de vin doux.
\TextTitle{Prédication de Pierre}
\VS{14}Alors Pierre, se présentant avec les onze, éleva sa voix, et leur dit~: Hommes Juifs, et vous tous qui habitez à Jérusalem, apprenez ceci, et faites attention à mes paroles~!
\VS{15}Ces gens ne sont pas ivres, comme vous le pensez, car c'est la troisième heure\FTNT{Neuf heures du matin.} du jour.
\VS{16}Mais c'est ici ce qui a été dit par le prophète Joël~:
\VS{17}Et il arrivera dans les derniers jours, dit Dieu, que je répandrai de mon Esprit sur toute chair~; vos fils et vos filles prophétiseront, vos jeunes gens auront des visions, et vos vieillards songeront des songes.
\VS{18}Et dans ces jours-là je répandrai de mon Esprit sur mes serviteurs et sur mes servantes, et ils prophétiseront.
\VS{19}Et je ferai des choses merveilleuses en haut dans le ciel, et des prodiges en bas sur la terre, du sang, du feu, et une vapeur de fumée.
\VS{20}Le soleil se changera en ténèbres, et la lune en sang, avant que ce grand et notable jour du Seigneur vienne.
\VS{21}Mais il arrivera que quiconque invoquera le Nom du Seigneur sera sauvé\FTNT{\vref{Joë. 2:28-32}.}.
\TextTitle{Proclamation de la résurrection du Messie}
\VS{22}Hommes Israélites, écoutez ces paroles~! Jésus de Nazareth, homme approuvé de Dieu parmi vous par les miracles et les prodiges et les signes que Dieu a faits par lui au milieu de vous, comme vous-mêmes vous le savez,
\VS{23}ayant été livré selon le dessein arrêté et selon la prescience de Dieu, vous l'avez pris et mis à la croix, vous l'avez fait mourir par les mains des impies.
\VS{24}Mais Dieu l'a ressuscité, ayant brisé les liens de la mort, parce qu'il n'était pas possible qu'il soit retenu par elle.
\VS{25}Car David dit de lui~: Je contemplais constamment le Seigneur devant moi, parce qu'il est à ma droite, afin que je ne sois point ébranlé\FTNT{\vref{Ps. 16:8-11}.}.
\VS{26}C'est pourquoi mon cœur est dans la joie, et ma langue dans l'allégresse~; et de plus, ma chair reposera avec espérance.
\VS{27}Car tu ne laisseras point mon âme en enfer\FTNT{Voir commentaire en \vref{Mt. 16:18}.} et tu ne permettras point que ton Saint voie la corruption.
\VS{28}Tu m'as fait connaître le chemin de la vie, tu me rempliras de joie dans ta présence\FTNT{\vref{Ps. 16:11}.}.
\VS{29}Hommes frères, qu'il me soit permis de vous dire librement, au sujet du patriarche David, qu'il est mort, qu'il a été enseveli, et que son sépulcre existe encore parmi nous jusqu'à ce jour.
\VS{30}Mais comme il était prophète, et qu'il savait que Dieu lui avait promis avec serment, que du fruit de ses reins il ferait naître selon la chair le Christ, pour le faire asseoir sur son trône~;
\VS{31}c'est la résurrection du Christ qu'il a prévue et annoncée, en disant qu'il ne serait pas abandonné en enfer et que sa chair ne verrait pas la corruption.
\VS{32}Dieu a ressuscité ce Jésus~; nous en sommes tous témoins.
\VS{33}Après donc qu'il a été élevé au ciel par la puissance de Dieu, et qu'il a reçu de son Père la promesse du Saint-Esprit, il a répandu ce que maintenant vous voyez et ce que vous entendez.
\VS{34}Car David n'est pas monté au ciel~; mais lui-même dit~: Le Seigneur a dit à mon Seigneur~: Assieds-toi à ma droite,
\VS{35}jusqu'à ce que j'aie mis tes ennemis pour le marchepied de tes pieds\FTNT{\vref{Ps. 110:1}.}. 
\VS{36}Que toute la maison d'Israël sache donc avec certitude que Dieu a fait Seigneur et Christ, ce Jésus, dis-je, que vous avez crucifié.
\TextTitle{Exhortation à la repentance}
\VS{37}Après avoir entendu ces choses, ils eurent le cœur touché de componction\FTNT{Componction~: Tristesse produite par les effets du repentir, le regret d'avoir offensé Dieu.}, et ils dirent à Pierre et aux autres apôtres~: Hommes frères, que ferons-nous~?
\VS{38}Et Pierre leur dit~: Repentez-vous, et que chacun de vous soit baptisé au Nom de Jésus-Christ, pour obtenir le pardon de vos péchés, et vous recevrez le don du Saint-Esprit.
\VS{39}Car à vous et à vos enfants est faite la promesse, et à tous ceux qui sont loin, autant que le Seigneur, notre Dieu en appellera à lui.
\VS{40}Et par plusieurs autres paroles, il les conjurait et les exhortait, en disant~: Sauvez-vous de cette génération perverse.
\TextTitle{Conversion et baptême de trois mille personnes~; les débuts de l'Eglise}
\VS{41}Ceux donc qui reçurent de bon cœur sa parole, furent baptisés~; et en ce jour-là furent ajoutées à l'Eglise environ trois mille âmes.
\VS{42}Et ils persévéraient tous dans la doctrine des apôtres, dans la communion fraternelle, dans la fraction du pain, et dans les prières.
\VS{43}Et tout le monde avait de la crainte, et beaucoup de miracles et de prodiges se faisaient par les apôtres.
\VS{44}Tous ceux qui croyaient étaient ensemble dans le même lieu, et ils avaient tout en commun~;
\VS{45}et ils vendaient leurs possessions et leurs biens, et les distribuaient à tous, selon les besoins de chacun.
\VS{46}Et tous les jours, ils persévéraient tous d'un commun accord dans le temple~; et rompant le pain de maison en maison, ils prenaient leur repas avec joie et simplicité de cœur~;
\VS{47}louant Dieu et se rendant agréables à tout le peuple. Et le Seigneur ajoutait tous les jours à l'Eglise des gens pour être sauvés.
\Chap{3}
\TextTitle{Guérison d'un homme boiteux de naissance}
\VerseOne{}Et comme Pierre et Jean montaient ensemble au temple à l'heure de la prière~; c'était la neuvième heure.
\VS{2}Et il y avait un homme boiteux de naissance, qu'on portait, et qu'on mettait tous les jours à la porte du temple, appelée la Belle, pour demander l'aumône à ceux qui entraient dans le temple. 
\VS{3}Cet homme voyant Pierre et Jean qui allaient entrer au temple, les pria de lui donner l'aumône.
\VS{4}Alors Pierre, de même que Jean, fixa les yeux sur lui, et lui dit~: Regarde-nous.
\VS{5}Et il les regardait attentivement, s'attendant de recevoir quelque chose d'eux.
\VS{6}Mais Pierre lui dit~: Je n'ai ni argent, ni or~; mais ce que j'ai, je te le donne~: Au Nom de Jésus-Christ de Nazareth, lève-toi et marche.
\VS{7}Et l'ayant pris par la main droite, il le fit lever~; et aussitôt les plantes et les chevilles de ses pieds devinrent fermes.
\VS{8}Et faisant un saut, il se tint debout, et marcha~; et il entra avec eux au temple, marchant, sautant, et louant Dieu.
\VS{9}Et tout le peuple le vit marchant et louant Dieu.
\VS{10}Et reconnaissant que c'était celui-là même qui était assis à la Belle, porte du temple, pour avoir l'aumône, ils furent remplis d'admiration et d'étonnement de ce qui lui était arrivé.
\VS{11}Et comme le boiteux, qui avait été guéri, tenait par la main Pierre et Jean, tout le peuple étonné accourut vers eux, au portique qu'on appelle de Salomon.
\TextTitle{Christ, le Messie annoncé par les prophètes}
\VS{12}Mais Pierre voyant cela, dit au peuple~: Hommes Israélites, pourquoi vous étonnez-vous de ceci~? Ou pourquoi avez-vous les regards fixés sur nous, comme si par notre puissance ou par notre piété, nous avions fait marcher cet homme~?
\VS{13}Le Dieu d'Abraham, d'Isaac, et de Jacob, le Dieu de nos pères, a glorifié son Fils Jésus, que vous avez livré et renié devant Pilate, quoiqu'il jugeât qu'il devait être relâché.
\VS{14}Mais vous avez renié le Saint et le Juste, et vous avez demandé qu'on vous relâche un meurtrier.
\VS{15}Vous avez fait mourir le Prince de la vie, que Dieu a ressuscité des morts~; nous en sommes témoins.
\VS{16}C'est par la foi en son Nom, que son Nom a raffermi les pieds de cet homme que vous voyez et connaissez. La foi, dis-je, que nous avons en lui, a donné à cet homme cette entière guérison de tous ses membres, en présence de vous tous.
\VS{17}Et maintenant, mes frères, je sais que vous avez agi par ignorance, de même que vos chefs.
\VS{18}Mais Dieu a ainsi accompli les choses qu'il avait prédites par la bouche de tous ses prophètes, que le Christ devait souffrir\FTNT{\vref{Es. 53}.}.
\VS{19}Repentez-vous donc, et convertissez-vous, afin que vos péchés soient effacés~;
\VS{20}afin que des temps de rafraîchissement viennent par la présence du Seigneur, et qu'il envoie celui qui vous a été auparavant annoncé, Jésus-Christ,
\VS{21}lequel il faut que le ciel reçoive, jusqu'au temps du rétablissement de toutes les choses que Dieu a prononcées par la bouche de tous ses saints prophètes, dès le commencement du monde.
\VS{22}Car Moïse lui-même a dit à nos pères~: Le Seigneur votre Dieu, vous suscitera d'entre vos frères un Prophète comme moi~; vous l'écouterez dans tout ce qu'il vous dira,
\VS{23}et il arrivera que toute personne qui n'aura pas écouté ce Prophète, sera exterminé du milieu du peuple\FTNT{\vref{De. 18:15-19}.}.
\VS{24}Et même tous les Prophètes depuis Samuel, et ceux qui l'ont suivi, tout autant qu'il y en a eu qui ont parlé, ont aussi prédit ces jours.
\VS{25}Vous êtes les enfants des prophètes et de l'Alliance que Dieu a traitée avec nos pères, en disant à Abraham~: Toutes les familles de la terre seront bénies en ta postérité\FTNT{\vref{Ge. 12:2}.}.
\VS{26}C'est à vous premièrement que Dieu, ayant suscité son Fils Jésus, l'a envoyé pour vous bénir, en détournant chacun de vous de vos iniquités.
\Chap{4}
\TextTitle{Première persécution de l'Eglise~: Pierre et Jean jetés en prison}
\VerseOne{}Mais comme ils parlaient au peuple, survinrent les prêtres, le commandant du temple et les sadducéens,
\VS{2}étant offensés de ce qu'ils enseignaient le peuple, et qu'ils annonçaient la résurrection des morts au Nom de Jésus.
\VS{3}Et les ayant fait arrêter, ils les mirent en prison jusqu'au lendemain, parce qu'il était déjà tard.
\VS{4}Et plusieurs de ceux qui avaient entendu la parole crurent~; et le nombre des personnes fut d'environ cinq mille.
\TextTitle{Pierre et Jean convoqués au sanhédrin}
\VS{5}Or il arriva que le lendemain, les chefs, les anciens et les scribes s'assemblèrent à Jérusalem~;
\VS{6}avec Anne, le grand-prêtre, Caïphe, Jean, Alexandre, et tous ceux qui étaient de la race des principaux prêtres.
\VS{7}Et ayant fait comparaître devant eux Pierre et Jean, ils leur demandèrent~: Par quelle puissance, ou au nom de qui avez-vous fait cette guérison~?
\VS{8}Alors Pierre étant rempli du Saint-Esprit, leur dit~: Chefs du peuple, et vous anciens d'Israël~:
\VS{9}Puisque nous sommes jugés aujourd'hui sur un bienfait accordé à un homme impotent, afin que nous disions comment il a été guéri,
\VS{10}sachez, vous tous et tout le peuple d'Israël, que c'est au Nom de Jésus-Christ de Nazareth, que vous avez crucifié, et que Dieu a ressuscité des morts~; c'est en son Nom, que cet homme qui parait ici devant vous, a été guéri.
\VS{11}C'est cette pierre rejetée, par vous qui bâtissez, qui est devenue la pierre principale de l'angle\FTNT{\vref{Ps. 118:22}.}.
\VS{12}Il n'y a de salut en aucun autre~: Car il n'y a sous le ciel aucun autre Nom qui ait été donné aux hommes par lequel nous devions être sauvés.
\TextTitle{Le sanhédrin interdit aux apôtres de prêcher au Nom de Jésus}
\VS{13}Eux, voyant la hardiesse de Pierre et de Jean, et sachant aussi qu'ils étaient des hommes sans instruction et du commun peuple~; s'en étonnaient, et ils reconnaissaient bien qu'ils avaient été avec Jésus.
\VS{14}Et voyant que l'homme qui avait été guéri, était présent avec eux, ils ne pouvaient contredire en rien.
\VS{15}Alors ils leur ordonnèrent de sortir hors du sanhédrin, et ils délibérèrent entre eux, disant~: Que ferons-nous à ces gens~?
\VS{16}Car il est manifeste pour tous les habitants de Jérusalem, qu'un miracle a été fait par eux, et cela est si évident que nous ne pouvons le nier.
\VS{17}Mais afin qu'il ne soit plus divulgué parmi le peuple, défendons-leur avec menaces expresses, qu'ils n'aient plus à parler à qui que ce soit en ce Nom.
\VS{18}Et les ayant donc appelés, ils leur ordonnèrent de ne plus parler ni d'enseigner en aucune manière au Nom de Jésus. 
\VS{19}Mais Pierre et Jean leur répondirent~: Jugez s'il est juste devant Dieu de vous obéir plutôt qu'à Dieu.
\VS{20}Car nous ne pouvons pas ne pas parler de ce que nous avons vu et entendu.
\VS{21}Alors ils les relâchèrent avec menaces, ne trouvant point comment ils pourraient les punir, à cause du peuple, parce que tous glorifiaient Dieu de ce qui avait été fait.
\VS{22}Car l'homme en qui cette miraculeuse guérison avait été faite, avait plus de quarante ans.
\TextTitle{L'Eglise demande l'assistance de Dieu}
\VS{23}Après avoir été relâchés, ils allèrent vers les leurs, et leur racontèrent tout ce que les principaux prêtres et les anciens leur avaient dit.
\VS{24}Eux l'ayant entendu, élevèrent tous ensemble la voix à Dieu, et dirent~: Seigneur, tu es le Dieu qui as fait le ciel et la terre, la mer, et toutes les choses qui y sont~;
\VS{25}et qui as dit par la bouche de David ton serviteur~: Pourquoi ce tumulte parmi les nations et ces vaines pensées parmi les peuples~?
\VS{26}Les rois de la terre se sont soulevés en personne, et les princes se sont ligués ensemble contre le Seigneur, et contre son Christ\FTNT{\vref{Ps. 2:1-2}.}.
\VS{27}En effet, contre ton Saint Fils Jésus, que tu as oint, se sont assemblés Hérode et Ponce Pilate, avec les Gentils, et le peuple d'Israël,
\VS{28}pour faire toutes les choses que ta main et ton conseil avaient auparavant déterminé qui seraient faites. 
\VS{29}Maintenant donc, Seigneur, regarde à leurs menaces, et donne à tes serviteurs d'annoncer ta parole avec toute hardiesse~;
\VS{30}en étendant ta main afin qu'il se fasse des guérisons, des prodiges, et des merveilles, par le Nom de ton Saint Fils Jésus.
\VS{31}Et quand ils eurent prié, le lieu où ils étaient assemblés trembla~; et ils furent tous remplis du Saint-Esprit, et ils annonçaient la parole de Dieu avec hardiesse.
\TextTitle{La multitude unie comme un seul corps\FTNTT{\vref{Ac. 2:42-47}.}}
\VS{32}Or la multitude de ceux qui croyaient n'était qu'un cœur et qu'une âme et nul ne disait d'aucune des choses qu'il possédait, qu'elle fût à lui, mais toutes choses étaient communes entre eux.
\VS{33}Aussi les apôtres rendaient témoignage avec une grande force à la résurrection du Seigneur Jésus~; et une grande grâce était sur eux tous.
\VS{34}Car il n'y avait parmi eux aucun indigent~; parce que tous ceux qui possédaient des champs ou des maisons, les vendaient, et ils apportaient le prix des choses vendues,
\VS{35}et le mettaient aux pieds des apôtres~; et il était distribué à chacun selon qu'il en avait besoin.
\VS{36}Or Joseph, surnommé par les apôtres Barnabas, c'est-à-dire, fils de consolation, Lévite, originaire de Chypre,
\VS{37}ayant une possession, la vendit, et en apporta le prix, et le mit aux pieds des apôtres.
\Chap{5}
\TextTitle{Mensonge d'Ananias et Saphira~: Leur mort}
\VerseOne{}Mais un homme appelé Ananias, et Saphira sa femme, vendit une possession,
\VS{2}et retint une partie du prix, sa femme le sachant~; puis il apporta le reste, et le déposa aux pieds des apôtres.
\VS{3}Mais Pierre lui dit~: Ananias comment Satan s'est-il emparé de ton cœur jusqu'à t'inciter à mentir au Saint-Esprit, et à soustraire une partie du prix de la possession~?
\VS{4}Si tu l'avais gardée, ne te restait-elle pas~? Et après qu'elle ait été vendue, le prix n'était-il pas à ta disposition~? Comment as-tu pu mettre en ton cœur un pareil dessein~? Tu n'as pas menti aux hommes mais à Dieu.
\VS{5}Et Ananias, entendant ces paroles, tomba et rendit l'âme~; ce qui causa une grande crainte à tous ceux qui en entendirent parler.
\VS{6}Et quelques jeunes hommes se levant le prirent, et l'emportèrent dehors, et l'ensevelirent.
\VS{7}Et il arriva environ trois heures après que sa femme entra, sans savoir ce qui était arrivé.
\VS{8}Et Pierre prenant la parole, lui dit~: Dis-moi, avez-vous autant vendu le champ~? Et elle dit~: Oui, autant.
\VS{9}Alors Pierre lui dit~: Pourquoi avez-vous fait un complot entre vous pour tenter l'Esprit du Seigneur~? Voici, à la porte, les pieds de ceux qui ont enterré ton mari, et ils t'emporteront.
\VS{10}Et au même instant, elle tomba à ses pieds et rendit l'esprit. Et quand les jeunes hommes furent entrés, ils la trouvèrent morte, et ils l'emportèrent dehors, et l'ensevelirent auprès de son mari.
\VS{11}Et cela donna une grande crainte à toute l'Eglise, et à tous ceux qui entendaient ces choses.
\TextTitle{Miracles à Jérusalem}
\VS{12}Beaucoup de prodiges et de miracles se faisaient parmi le peuple par les mains des apôtres~; et ils étaient tous d'un commun accord au portique de Salomon.
\VS{13}Cependant aucun des autres n'osait se joindre à eux, mais le peuple les louait hautement.
\VS{14}Et le nombre de ceux qui croyaient au Seigneur, tant d'hommes que de femmes, se multipliait de plus en plus.
\VS{15}Et on apportait les malades dans les rues, et on les mettait sur de petits lits et sur des couchettes, afin que quand Pierre viendrait, au moins son ombre passe sur quelqu'un d'eux.
\VS{16}La multitude accourait aussi des villes voisines à Jérusalem, amenant des malades, et ceux qui étaient tourmentés des esprits impurs~; et tous étaient guéris.
\TextTitle{Deuxième persécution de l'Eglise~: Les apôtres en prison puis devant le sanhédrin}
\VS{17}Alors le grand-prêtre se leva, lui et tous ceux qui étaient avec lui, à savoir la secte des sadducéens, et ils furent remplis de jalousie~;
\VS{18}et mettant la main sur les apôtres, ils les jetèrent dans la prison publique.
\VS{19}Mais l'Ange du Seigneur ouvrit pendant la nuit les portes de la prison, les fit sortir, et leur dit~:
\VS{20}Allez, et présentez-vous dans le temple, annoncez au peuple toutes les paroles de cette vie.
\VS{21}Ayant entendu cela, ils entrèrent dès le matin dans le temple, et se mirent à enseigner. Mais le grand-prêtre et ceux qui étaient avec lui étant arrivés, ils convoquèrent le sanhédrin et tous les anciens des fils d'Israël, et ils envoyèrent chercher les apôtres à la prison.
\VS{22}Mais, les huissiers à leur arrivée, ne les trouvèrent point dans la prison. Ils retournèrent, et firent leur rapport,
\VS{23}en disant~: Nous avons trouvé la prison fermée avec toute sûreté, et les gardes aussi qui étaient devant les portes~; mais après l'avoir ouverte, nous n'avons trouvé personne dedans.
\VS{24}Lorsque le grand-prêtre, le commandant du temple, et les principaux prêtres, eurent entendu ces paroles, ils ne savaient que penser au sujet des apôtres, ne sachant ce qui arriverait de tout cela.
\VS{25}Mais quelqu'un vint leur dire~: Voici, les hommes que vous avez mis en prison sont dans le temple, et ils enseignent le peuple.
\VS{26}Alors le commandant du temple partit avec les huissiers, et il les conduisit sans violence, car ils avaient peur d'être lapidés par le peuple.
\VS{27}Après qu'ils les eurent amenés, ils les présentèrent au sanhédrin. Et le grand-prêtre les interrogea, disant~: 
\VS{28}Ne vous avons-nous pas défendu expressément d'enseigner en ce Nom-là~? Et cependant voici, vous avez rempli Jérusalem de votre doctrine, et vous voulez faire retomber sur nous le sang de cet homme.
\VS{29}Alors Pierre et les autres apôtres répondant, dirent~: Il faut plutôt obéir à Dieu qu'aux hommes.
\VS{30}Le Dieu de nos pères a ressuscité Jésus, que vous avez fait mourir en le pendant au bois.
\VS{31}Dieu l'a élevé par sa puissance pour être Prince et Sauveur, afin de donner à Israël la repentance et la rémission des péchés.
\VS{32}Nous sommes témoins de ce que nous disons, de même que le Saint-Esprit que Dieu a donné à ceux qui lui obéissent, en est aussi témoin.
\TextTitle{Parole de sagesse de Gamaliel}
\VS{33}Mais eux, ayant entendu ces choses, grinçaient les dents, et consultaient pour les faire mourir.
\VS{34}Mais un pharisien nommé Gamaliel, docteur de la loi, honoré de tout le peuple, se leva dans le sanhédrin, et ordonna de faire sortir un instant les apôtres.
\VS{35}Puis il leur dit~: Hommes Israélites, prenez garde à ce que vous allez faire à l'égard de ces gens.
\VS{36}Car il n'y a pas longtemps que Theudas s'éleva, se disant être quelque chose, et auquel se joignit un nombre d'environ quatre cents hommes~; mais il fut tué, et tous ceux qui s'étaient joints à lui ont été dissipés et réduits à rien.
\VS{37}Après lui parut Judas le Galiléen au temps du recensement, et il attira à lui un grand peuple~; il périt aussi, et tous ceux qui s'étaient joints à lui ont été dispersés.
\VS{38}Maintenant donc je vous dis~: Ne continuez plus vos poursuites contre ces hommes, et laissez-les. Car si cette entreprise ou cette œuvre vient des hommes, elle sera détruite~;
\VS{39}mais si elle vient de Dieu, vous ne pourrez pas la détruire. Et prenez garde qu'il ne se trouve que vous combattiez contre Dieu.
\VS{40}Et ils furent de son avis. Et ayant appelé les apôtres, ils les firent battre de verges, ils leur défendirent de parler au Nom de Jésus, et ils les relâchèrent.
\TextTitle{Frappés, les apôtres continuent de prêcher le Nom de Jésus}
\VS{41}Et les apôtres se retirèrent de devant le sanhédrin, joyeux d'avoir été jugés dignes de subir des outrages pour le Nom de Jésus.
\VS{42}Et tous les jours, ils ne cessaient d'enseigner, et d'annoncer l'Evangile de Jésus-Christ dans le temple, et de maison en maison.
\Chap{6}
\TextTitle{Sept hommes choisis pour le service}
\VerseOne{}En ces jours-là, comme les disciples se multipliaient, il s'éleva un murmure des Hellénistes\FTNT{Les Hellénistes étaient des juifs issus de la diaspora ayant adopté la culture et la langue grecque.} contre les Hébreux, parce que leurs veuves étaient méprisées dans le service ordinaire.
\VS{2}C'est pourquoi les douze, ayant convoqué la multitude des disciples, leur dirent~: Il n'est pas raisonnable que nous laissions la parole de Dieu pour servir aux tables.
\VS{3}Regardez donc, mes frères, pour choisir sept hommes d'entre vous, de qui on ait bon témoignage, pleins du Saint-Esprit et de sagesse, auxquels nous confierons ce devoir.
\VS{4}Et nous, nous continuerons à vaquer à la prière et au service de la parole.
\VS{5}Et ce discours plut à toute l'assemblée qui était là présente~; et ils élurent Etienne, homme plein de foi et du Saint-Esprit, Philippe, Prochore, Nicanor, Timon, Parménas, et Nicolas, prosélyte d'Antioche.
\VS{6}Ils les présentèrent aux apôtres~; qui, après avoir prié, leur imposèrent les mains.
\VS{7}Et la parole de Dieu croissait, et le nombre des disciples se multipliait beaucoup dans Jérusalem~; un grand nombre aussi de prêtres obéissait à la foi.
\VS{8}Or Etienne, plein de foi et de puissance, faisait de grands miracles et de grands prodiges parmi le peuple.
\TextTitle{Troisième persécution de l'Eglise~; Etienne convoqué au sanhédrin}
\VS{9}Quelques-uns de la synagogue appelée la synagogue des affranchis\FTNT{Affranchis~: Du grec «~libertinos~», c'est-à-dire «~libertins~»~: Hommes libres. Fraction de la communauté Juive qui avait sa propre synagogue à Jérusalem. Probablement des Juifs qui avaient été faits prisonniers par Pompée et d'autres généraux romains, qui avaient été déportés à Rome, puis libérés.}, de celle des Cyrénéens et de celle des Alexandrins, avec ceux de Cilicie et d'Asie, se levèrent pour disputer contre Etienne.
\VS{10}Mais ils ne pouvaient pas résister à la sagesse et à l'Esprit par lequel il parlait.
\VS{11}Alors ils soudoyèrent des hommes qui dirent~: Nous l'avons entendu proférer des paroles blasphématoires contre Moïse et contre Dieu.
\VS{12}Et ils soulevèrent le peuple, les anciens, et les scribes, et se jetant sur lui, ils l'enlevèrent et l'amenèrent au sanhédrin.
\VS{13}Et ils présentèrent de faux témoins qui dirent~: Cet homme ne cesse de proférer des paroles blasphématoires contre ce saint lieu et contre la loi.
\VS{14}Car nous l'avons entendu dire que Jésus, ce Nazaréen, détruira ce lieu-ci, et changera les coutumes que Moïse nous a données.
\VS{15}Tous ceux qui siégeaient au sanhédrin avaient les yeux fixés sur lui, son visage leur parut comme celui d'un ange.
\Chap{7}
\TextTitle{Discours d'Etienne devant le sanhédrin}
\VerseOne{}Alors le grand-prêtre lui dit~: Ces choses sont-elles ainsi~?
\VS{2}Etienne répondit~: Hommes frères et pères, écoutez-moi~! Le Dieu de gloire apparut à notre père Abraham, lorsqu'il était en Mésopotamie, avant qu'il s'établisse à Charran, et lui dit~:
\VS{3}Sors de ton pays et de ta famille, et va dans le pays que je te montrerai.
\VS{4}Il sortit donc du pays des Chaldéens, et alla demeurer à Charran. De là, après la mort de son père, Dieu le fit passer dans ce pays que vous habitez maintenant.
\VS{5}Et il ne lui donna aucun héritage dans ce pays, non pas même d'un pied de terre, quoiqu'il lui ait promis de le lui donner en possession, et à sa postérité après lui, dans un temps où il n'avait point encore d'enfant.
\VS{6}Dieu lui parla ainsi~: Ta postérité séjournera dans une terre étrangère pendant quatre cents ans~; et on la réduira à la servitude et on la maltraitera.
\VS{7}Mais je jugerai la nation à laquelle ils auront été asservis, dit Dieu~; et après cela ils sortiront, et me serviront en ce lieu-ci\FTNT{\vref{Ge. 15:13-14}.}.
\VS{8}Puis il donna à Abraham l'alliance de la circoncision~; et après cela Abraham engendra Isaac qu'il circoncit le huitième jour. Isaac engendra Jacob, et Jacob les douze patriarches.
\VS{9}Les patriarches, jaloux de Joseph, le vendirent pour être emmené en Egypte.
\VS{10}Mais Dieu était avec lui, et le délivra de toutes ses afflictions~; et l'ayant rempli de sagesse il le rendit agréable à Pharaon, roi d'Egypte, qui l'établit gouverneur sur l'Egypte, et sur toute sa maison.
\VS{11}Or il survint dans tout le pays d'Egypte, et dans celui de Canaan, une famine et une grande détresse, en sorte que nos pères ne pouvaient trouver des vivres.
\VS{12}Mais Jacob apprit qu'il y avait du blé en Egypte, il y envoya une première fois nos pères.
\VS{13}Et la seconde fois, Joseph fut reconnu par ses frères, et la famille de Joseph fut déclarée à Pharaon.
\VS{14}Alors Joseph envoya chercher Jacob, son père, et toute sa famille, composée de soixante-quinze personnes.
\VS{15}Jacob descendit en Egypte, et il y mourut, lui et nos pères~;
\VS{16}qui furent transportés à Sichem, et mis dans le sépulcre qu'Abraham avait acheté à prix d'argent des fils d'Hamor, fils de Sichem.
\VS{17}Mais comme le temps de la promesse, pour laquelle Dieu avait juré à Abraham, s'approchait, le peuple s'augmenta et se multiplia en Egypte~;
\VS{18}jusqu'à ce que parut en Egypte un autre roi, qui n'avait pas connu Joseph.
\VS{19}Ce roi, usant d'artifice contre notre race, maltraita nos pères jusqu'à leur faire exposer leurs enfants à l'abandon, afin d'en faire périr la race.
\VS{20}En ce temps-là naquit Moïse, qui fut divinement beau. Et il fut nourri trois mois dans la maison de son père.
\VS{21}Mais ayant été exposé à l'abandon, la fille de Pharaon le recueillit et l'éleva comme son fils.
\VS{22}Moïse fut instruit dans toute la sagesse des Egyptiens~; et il était puissant en paroles et en œuvres.
\VS{23}Mais quand il fut parvenu à l'âge de quarante ans, il forma le dessein d'aller visiter ses frères, les enfants d'Israël.
\VS{24}Et voyant l'un d'eux à qui l'on faisait tort, il le défendit, et vengea celui qui était outragé en tuant l'Egyptien.
\VS{25}Il croyait que ses frères comprendraient par là que Dieu les délivrerait par son moyen~; mais ils ne le comprirent point.
\VS{26}Et le jour suivant, il parut au milieu d'eux comme ils se querellaient, et il tâcha de les mettre d'accord en leur disant~: Hommes, vous êtes frères, pourquoi vous faites-vous tort l'un à l'autre~?
\VS{27}Mais celui qui maltraitait son prochain le repoussa, en disant~: Qui t'a établi prince et juge sur nous~?
\VS{28}Veux-tu me tuer, comme tu as tué hier l'Egyptien~?
\VS{29}Alors Moïse s'enfuit sur un tel discours, et fut étranger dans le pays de Madian, où il eut deux fils.
\VS{30}Et quarante ans étant accomplis, l'Ange du Seigneur lui apparut au désert de la montagne de Sinaï, dans la flamme d'un buisson en feu.
\VS{31}Et quand Moïse le vit, il fut étonné de la vision, et comme il approchait pour considérer ce que c'était, la voix du Seigneur lui fut adressée, disant~:
\VS{32} Je suis le Dieu de tes pères, le Dieu d'Abraham, le Dieu d'Isaac, et le Dieu de Jacob. Et Moïse tout tremblant n'osait pas regarder.
\VS{33}Le Seigneur lui dit~: Ôte tes souliers de tes pieds, car le lieu sur lequel tu te tiens est une terre sainte.
\VS{34}J'ai vu, j'ai vu l'affliction de mon peuple qui est en Egypte, et j'ai entendu leur gémissement, et je suis descendu pour les délivrer. Maintenant donc, va, je t'enverrai en Egypte.
\VS{35}Ce Moïse, qu'ils avaient rejeté en disant~: Qui t'a établi prince et juge~? C'est lui que Dieu envoya comme prince et comme libérateur par le moyen de l'Ange qui lui était apparu dans le buisson.
\VS{36}C'est celui qui les tira dehors, en opérant des miracles et des prodiges au pays d'Egypte, au sein de la Mer Rouge, et au désert pendant quarante ans.
\VS{37}C'est ce Moïse qui a dit aux enfants d'Israël~: Le Seigneur votre Dieu vous suscitera d'entre vos frères un Prophète comme moi~; écoutez-le\FTNT{\vref{De. 18:15}.}~!
\VS{38}C'est lui, qui, lors de l'assemblée au désert, étant avec l'Ange qui lui parlait sur la montagne de Sinaï et avec nos pères, reçut les paroles de vie pour nous les donner.
\VS{39}Nos pères ne voulurent pas lui obéir, mais ils le rejetèrent, et ils tournèrent leur cœur vers l'Egypte,
\VS{40}en disant à Aaron~: Fais-nous des dieux qui marchent devant nous~; car nous ne savons point ce qui est arrivé à ce Moïse qui nous a amenés hors du pays d'Egypte.
\VS{41}Ils firent donc en ces jours-là un veau, et ils offrirent des sacrifices à l'idole, et se réjouirent de l'œuvre de leurs mains.
\VS{42}C'est pourquoi aussi Dieu se détourna d'eux, et les livra au culte de l'armée du ciel, ainsi qu'il est écrit dans le livre des prophètes~: Maison d'Israël, m'avez-vous offert des sacrifices et des victimes pendant quarante ans au désert~?
\VS{43}Mais vous avez porté la tente de Moloc\FTNT{\vref{Lé. 18:21}.}, et l'étoile de votre dieu Remphan~; qui sont des figures que vous avez faites pour les adorer. C'est pourquoi je vous transporterai au-delà de Babylone.
\VS{44}Nos pères avaient au désert le tabernacle du témoignage, comme l'avait ordonné celui qui avait dit à Moïse de le faire selon le modèle qu'il avait vu.
\VS{45}Et nos pères avaient reçu ce tabernacle, ils le portèrent sous la conduite de Josué dans le pays qui était possédé par les nations que Dieu chassa de devant eux, et il y resta jusqu'aux jours de David.
\VS{46}David trouva grâce devant Dieu, et demanda de pouvoir dresser une tente pour le Dieu de Jacob.
\VS{47}Et ce fut Salomon qui lui bâtit une maison.
\VS{48}Mais le Très-Haut n'habite pas dans des temples faits de main d'homme, selon ces paroles du prophète~:
\VS{49}Le ciel est mon trône, et la terre est le marchepied de mes pieds~: Quelle maison me bâtirez-vous, dit le Seigneur, ou quel pourrait être le lieu de mon repos~?
\VS{50}Ma main n'a-t-elle pas fait toutes ces choses\FTNT{\vref{Es. 66:1}.}~?
\VS{51}Hommes au cou raide, et incirconcis de cœur et d'oreilles, vous vous obstinez toujours contre le Saint-Esprit~; vous faites comme vos pères ont fait.
\VS{52}Lequel des prophètes vos pères n'ont-ils pas persécuté~? Ils ont même tué ceux qui annonçaient d'avance l'avènement du Juste, dont vous avez été les traîtres et les meurtriers,
\VS{53}vous qui avez reçu la loi par une ordonnance des anges, et qui ne l'avez point gardée.
\TextTitle{Etienne~: Premier martyr}
\VS{54}En entendant ces choses, leur cœur s'enflamma de colère, et ils grinçaient des dents contre lui.
\VS{55}Mais Etienne, rempli du Saint-Esprit, et fixant les yeux vers le ciel, vit la gloire de Dieu, et Jésus qui était à la droite de Dieu.
\VS{56}Et il dit~: Voici, je vois les cieux ouverts, et le Fils de l'homme étant à la droite de Dieu.
\VS{57}Alors ils s'écrièrent à haute voix, et bouchèrent leurs oreilles, et tous d'un commun accord se jetèrent sur lui.
\VS{58}Et l'ayant tiré hors de la ville, ils le lapidèrent~; et les témoins déposèrent leurs vêtements aux pieds d'un jeune homme nommé Saul.
\VS{59}Et ils lapidaient Etienne qui priait et disait~: Seigneur Jésus, reçois mon esprit\FTNT{Dans \vref{Ec. 12:9}, il est dit qu'à la mort, l'esprit retourne à Dieu qui l'a donné. Jésus est donc Dieu puisqu'il a reçu l'esprit d'Etienne.}~!
\VS{60}Et s'étant mis à genoux, il cria à haute voix~: Seigneur, ne leur impute point ce péché~! Et quand il eut dit cela, il s'endormit.
\Chap{8}
\TextTitle{Quatrième persécution de l'Eglise~: Saul opprime les saints}
\VerseOne{}Or Saul consentait à la mort d'Etienne, et en ce temps-là, il y eut une grande persécution contre l'Eglise de Jérusalem. Et tous, excepté les apôtres, se dispersèrent dans les contrées de la Judée et de la Samarie.
\VS{2}Et quelques hommes pieux emportèrent Etienne pour l'ensevelir, et le pleurèrent à grand bruit.
\VS{3}Mais Saul ravageait l'église, entrant dans toutes les maisons, et traînant par force hommes et femmes, il les mettait en prison.
\TextTitle{Le déploiement des chrétiens\FTNTT{\vref{Ac. 11:19-21}}}
\VS{4}Ceux qui avaient été dispersés allaient de lieu en lieu, annonçant la parole de Dieu.
\TextTitle{Philippe en Samarie~; Simon le magicien}
\VS{5}Philippe, étant descendu dans la ville de Samarie, leur prêcha Christ.
\VS{6}Et les foules tout entières étaient attentives à ce que Philippe disait, l'écoutant, lorsqu'elles virent les miracles qu'il faisait,
\VS{7}car les esprits impurs sortaient, en criant à haute voix, hors de plusieurs qui en étaient possédés, et beaucoup de paralytiques et de boiteux furent guéris.
\VS{8}Ce qui causa une grande joie dans cette ville-là.
\VS{9}Or il y avait auparavant dans la ville un homme nommé Simon qui exerçait l'art d'enchanteur, et ensorcelait le peuple de Samarie, se disant être quelque grand personnage.
\VS{10}Tous, depuis le plus petit jusqu'au plus grand étaient attachés à lui, et disaient~: Celui-ci est la grande puissance de Dieu.
\VS{11}Et ils étaient attachés à lui, parce que depuis longtemps il les avait éblouis par sa magie.
\VS{12}Mais quand ils eurent cru ce que Philippe leur annonçait, touchant l'Evangile du Royaume de Dieu, et le Nom de Jésus-Christ, tant les hommes que les femmes furent baptisés.
\VS{13}Et Simon crut aussi lui-même, et après avoir été baptisé, il ne quittait plus Philippe~; et voyant les prodiges et les grands miracles qui se faisaient, il était comme ravi hors de lui même.
\VS{14}Or quand les apôtres, qui étaient à Jérusalem eurent entendu que la Samarie avait reçu la parole de Dieu, ils leur envoyèrent Pierre et Jean~;
\VS{15}qui y étant descendus prièrent pour eux, afin qu'ils reçoivent le Saint-Esprit,
\VS{16}car il n'était pas encore descendu sur aucun d'eux, mais seulement ils étaient baptisés au Nom du Seigneur Jésus.
\VS{17}Puis ils leur imposèrent les mains, et ils reçurent le Saint-Esprit.
\VS{18}Lorsque Simon vit que le Saint-Esprit était donné par l'imposition des mains des apôtres, il leur présenta de l'argent,
\VS{19}en leur disant~: Donnez-moi aussi ce pouvoir, afin que tous ceux à qui j'imposerai les mains reçoivent le Saint-Esprit.
\VS{20}Mais Pierre lui dit~: Que ton argent périsse avec toi, puisque tu as estimé que le don de Dieu s'acquérait avec de l'argent.
\VS{21}Tu n'as point de part ni d'héritage en cette affaire~; car ton cœur n'est point droit devant Dieu.
\VS{22}Repens-toi donc de cette méchanceté, et prie Dieu, afin que, s'il est possible, la pensée de ton cœur te soit pardonnée.
\VS{23}Car je vois que tu es dans un fiel très amer et dans un lien d'iniquité.
\VS{24}Alors Simon répondit, et dit~: Vous, priez le Seigneur pour moi, afin qu'il ne m'arrive rien de ce que vous avez dit. 
\VS{25}Eux donc après avoir prêché et annoncé la parole du Seigneur, retournèrent à Jérusalem et annoncèrent l'Evangile dans plusieurs villages des Samaritains.
\TextTitle{Conversion et baptême de l'eunuque éthiopien}
\VS{26}Puis l'Ange du Seigneur parla à Philippe, en disant~: Lève-toi et va vers le Midi, sur le chemin qui descend de Jérusalem à Gaza, celui qui est désert.
\VS{27}Il se leva donc, et s'en alla. Et voici, un homme éthiopien, un eunuque, qui était un des principaux seigneurs de la cour de Candace, reine des Ethiopiens, et surintendant de toutes ses richesses, venu à Jérusalem pour adorer,
\VS{28}s'en retournait, assis dans son char, et lisait le prophète Esaïe.
\VS{29}L'Esprit dit à Philippe~: Avance, et approche-toi de ce char.
\VS{30}Philippe accourut et entendit l'Ethiopien qui lisait le prophète Esaïe~; et il lui dit~: Comprends-tu ce que tu lis~?
\VS{31}Et il lui dit~: Comment pourrais-je le comprendre, si quelqu'un ne me guide pas~? Et il pria Philippe de monter et s'asseoir avec lui.
\VS{32}Le passage de l'Ecriture qu'il lisait était celui-ci~: Il a été mené comme une brebis à la boucherie, et comme un agneau muet devant celui qui le tond~; en sorte qu'il n'a point ouvert sa bouche.
\VS{33}Dans son humiliation, son jugement a été levé~; mais qui racontera sa durée~? Car sa vie est retranchée de la terre\FTNT{\vref{Es. 53:7-8}.}.
\VS{34}Et l'eunuque prenant la parole, dit à Philippe~: Je te prie, de qui est-ce que le prophète dit cela~? Est-ce de lui-même, ou de quelque autre~?
\VS{35}Alors Philippe, ouvrant sa bouche, et commençant par cette Ecriture, lui annonça l'Evangile de Jésus.
\VS{36}Comme ils continuaient leur chemin, ils arrivèrent à un endroit où il y avait de l'eau. Et l'eunuque dit~: Voici de l'eau, qu'est-ce qui empêche que je ne sois baptisé~?
\VS{37}Philippe dit~: Si tu crois de tout ton cœur, cela t'est permis~; et l'eunuque répondit~: Je crois que Jésus-Christ est le Fils de Dieu.
\VS{38}Il fit arrêter le char~; Philippe et l'eunuque descendirent tous deux dans l'eau, et Philippe le baptisa.
\VS{39}Quand ils furent sortis de l'eau, l'Esprit du Seigneur enleva Philippe, et l'eunuque ne le vit plus. Tandis que tout joyeux il continua son chemin,
\VS{40}Philippe se trouva dans Azot, d'où il alla jusqu'à Césarée, en évangélisant toutes les villes par lesquelles il passait.
\Chap{9}
\TextTitle{Jésus se révèle à Saul\FTNTT{\vref{Ac. 22:1-16}~; \vref{26:9-18}}}
\VerseOne{}Or Saul, respirant encore la menace et le carnage contre les disciples du Seigneur, s'adressa au grand-prêtre,
\VS{2}et lui demanda des lettres de sa part pour les porter aux synagogues de Damas, afin que, s'il trouvait quelques-uns de cette secte, hommes ou femmes, il les amène liés à Jérusalem.
\VS{3}Or il arriva qu'en marchant, il approcha de Damas et tout à coup une lumière resplendit du ciel comme un éclair autour de lui.
\VS{4}Il tomba par terre et il entendit une voix qui lui disait~: Saul, Saul, pourquoi me persécutes-tu~?
\VS{5}Et il répondit~: Qui es-tu, Seigneur~? Et le Seigneur lui dit~: Je suis Jésus, que tu persécutes. Il te serait dur de regimber contre les aiguillons.
\VS{6}Alors, tout tremblant et tout effrayé, il dit~: Seigneur, que veux-tu que je fasse~? Et le Seigneur lui dit~: Lève-toi, et entre dans la ville, et on te dira ce que tu dois faire.
\VS{7}Les hommes qui l'accompagnaient s'arrêtèrent tout épouvantés, entendant bien la voix, mais ne voyant personne.
\VS{8}Et Saul se leva de terre, et ouvrant ses yeux, il ne voyait personne~; c'est pourquoi ils le conduisirent par la main, et le menèrent à Damas,
\VS{9}où il fut trois jours sans voir, sans manger ni boire.
\VS{10}Or il y avait à Damas un disciple, nommé Ananias, à qui le Seigneur dit en vision~: Ananias~! Et il répondit~: Me voici Seigneur~!
\VS{11}Et le Seigneur lui dit~: Lève-toi, va dans la rue appelée la droite, et cherche dans la maison de Judas un homme appelé Saul, de Tarse.
\VS{12}Car il prie. Or Saul avait vu en vision un homme appelé Ananias, entrant et lui imposant les mains, afin qu'il recouvre la vue. Et Ananias répondit~:
\VS{13} Seigneur, j'ai entendu parler plusieurs fois de cet homme-là~; et combien de maux il a faits à tes saints dans Jérusalem.
\VS{14}Il a même ici le pouvoir de la part des principaux prêtres, de lier tous ceux qui invoquent ton Nom.
\VS{15}Mais le Seigneur lui dit~: Va~; car il m'est un vase\FTNTT{Le mot «~vase~» vient du grec «~skeuos~». «~Vase~» était une métaphore grecque commune pour «~le corps~» car les Grecs pensaient que l'âme vivait temporairement dans les corps. \vref{2 Co. 4:7}~; \vref{Ro. 9:21-23}~; \vref{2 Ti. 2:20-21}.} que j'ai choisi, pour porter mon Nom devant les Gentils, et les rois, et les enfants d'Israël.
\VS{16}Car je lui montrerai combien il aura à souffrir pour mon Nom.
\TextTitle{Saul rempli du Saint-Esprit}
\VS{17}Ananias sortit~; et lorsqu'il fut arrivé dans la maison, il imposa les mains à Saul, et lui dit~: Saul mon frère, le Seigneur Jésus, qui t'est apparu sur le chemin par lequel tu venais, m'a envoyé, afin que tu recouvres la vue, et que tu sois rempli du Saint-Esprit.
\TextTitle{Saul est baptisé et évangélise Damas}
\VS{18}Et aussitôt il tomba de ses yeux comme des écailles~; et à l'instant il recouvra la vue. Puis il se leva, et fut baptisé.
\VS{19}Et ayant mangé, il reprit ses forces. Et Saul fut quelques jours avec les disciples qui étaient à Damas.
\VS{20}Et aussitôt il prêcha dans les synagogues que Jésus était le Fils de Dieu.
\VS{21}Et tous ceux qui l'entendaient, étaient comme ravis hors d'eux-mêmes, et ils disaient~: N'est-ce pas celui-là qui a détruit à Jérusalem ceux qui invoquaient ce Nom, et qui est venu ici exprès pour les amener liés aux principaux prêtres~?
\VS{22}Mais Saul se fortifiait de plus en plus, et confondait les Juifs qui habitaient à Damas, prouvant que Jésus était le Christ.
\TextTitle{Complot contre Saul}
\VS{23}Longtemps après, les Juifs conspirèrent ensemble pour le faire mourir~;
\VS{24}et leur complot parvint à la connaissance de Saul. Or ils gardaient les portes jour et nuit, afin de le faire mourir.
\VS{25}Mais pendant une nuit, les disciples le prirent, et le descendirent par la muraille dans une corbeille.
\TextTitle{Saul rencontre Barnaba et les apôtres à Jérusalem}
\VS{26}Lorsqu'il se rendit à Jérusalem, Saul tâcha de se joindre aux disciples~; mais tous le craignaient, ne croyant pas qu'il fût un disciple.
\VS{27}Alors Barnabas, l'ayant pris avec lui, le conduisit vers les apôtres, et leur raconta comment sur le chemin, Saul avait vu le Seigneur, qui lui avait parlé, et comment à Damas il parlait librement au Nom de Jésus.
\VS{28}Et il allait et venait avec eux dans Jérusalem, il parlait franchement au Nom du Seigneur, se montrant publiquement.
\VS{29}Et parlant sans déguisement au Nom du Seigneur Jésus, il disputait contre les Hellénistes, mais ils tentaient de le faire mourir.
\TextTitle{Retour à Tarse}
\VS{30}Les frères, l'ayant découvert, l'emmenèrent à Césarée, et le firent partir à Tarse.
\VS{31}Les églises étaient en paix dans toute la Judée, la Galilée, et la Samarie, étant édifiées et marchant dans la crainte du Seigneur~; et elles s'accroissaient par le rafraîchissement du Saint-Esprit.
\TextTitle{Guérison d'Enée, le paralytique}
\VS{32}Or il arriva que comme Pierre les visitait tous, il descendit aussi vers les saints qui demeuraient à Lydde.
\VS{33}Il y vint aussi un homme appelé Enée, qui était couché dans un petit lit depuis huit ans, car il était paralytique.
\VS{34}Et Pierre lui dit~: Enée, Jésus-Christ te guérit~! Lève-toi et arrange ton lit. Et aussitôt il se leva.
\VS{35}Tous ceux qui habitaient à Lydde et à Saron le virent, et ils se convertirent au Seigneur.
\TextTitle{Résurrection de Tabitha}
\VS{36}Il y avait à Joppé une femme disciple, appelée Tabitha, qui signifie en grec Dorcas~; elle faisait beaucoup de bonnes œuvres et d'aumônes.
\VS{37}Elle tomba malade en ce temps-là, et mourut. Après l'avoir lavée, on la déposa dans une chambre haute.
\VS{38}Comme Lydde était près de Joppé, les disciples ayant appris que Pierre était à Lydde, ils envoyèrent vers lui deux hommes, pour le prier de venir chez eux sans tarder.
\VS{39}Pierre se leva, et partit avec ces hommes. Lorsqu'il fut arrivé, on le conduisit dans la chambre haute. Toutes les veuves l'entourèrent en pleurant, et lui montrèrent les tuniques et les vêtements que faisait Dorcas quand elle était avec elles.
\VS{40}Pierre fit sortir tout le monde, se mit à genoux, et pria~; puis se tournant vers le corps, il dit~: Tabitha, lève-toi~! Et elle ouvrit ses yeux, et voyant Pierre, elle s'assit.
\VS{41}Il lui donna la main, et la fit lever. Puis ayant appelé les saints et les veuves, il la leur présenta vivante.
\VS{42}Cela fut connu dans tout Joppé~; et plusieurs crurent au Seigneur.
\VS{43}Et il arriva qu'il demeura plusieurs jours à Joppé, chez un corroyeur nommé Simon.
\Chap{10}
\TextTitle{Un ange de Dieu apparait à Corneille}
\VerseOne{}Il y avait à Césarée un homme nommé Corneille, centenier d'une cohorte de la légion appelée Italienne.
\VS{2}Cet homme était pieux et craignait Dieu avec toute sa famille. Il faisait aussi beaucoup d'aumônes au peuple, et priait Dieu continuellement.
\VS{3}Vers la neuvième heure du jour, il vit clairement dans une vision un ange de Dieu qui entra chez lui, et qui lui dit~: Corneille~!
\VS{4}Corneille ayant les yeux fixés sur lui, et tout effrayé, lui dit~: Qu'y a-t-il Seigneur~? Et il lui dit~: Tes prières et tes aumônes sont montées devant Dieu, et il s'en est souvenu.
\VS{5}Maintenant donc envoie des gens à Joppé, et fais venir Simon, surnommé Pierre.
\VS{6}Il est logé chez un certain Simon, corroyeur, qui a sa maison près de la mer~; c'est lui qui te dira ce qu'il faut que tu fasses.
\VS{7}Dès que l'ange qui lui parlait fut parti, Corneille appela deux de ses serviteurs, et un soldat craignant Dieu, d'entre ceux qui se tenaient près de lui.
\VS{8}Et après leur avoir tout raconté, il les envoya à Joppé.
\TextTitle{Vision de Pierre~: Une nappe descend du ciel}
\VS{9}Le lendemain, comme ils marchaient et qu'ils approchaient de la ville, Pierre monta sur le toit, vers la sixième heure, pour prier.
\VS{10}Et il arriva qu'ayant faim, il voulut prendre son repas. Pendant qu'on lui préparait à manger, il tomba en extase.
\VS{11}Il vit le ciel ouvert, et un vase descendant sur lui semblable à une grande nappe, attachée par les quatre coins, qui descendait vers la terre,
\VS{12}où se trouvaient tous les quadrupèdes, les bêtes sauvages, les reptiles et les oiseaux du ciel.
\VS{13}Et une voix lui dit~: Pierre, lève-toi, tue, et mange.
\VS{14}Mais Pierre répondit~: Non, Seigneur, car je n'ai jamais rien mangé de souillé ni d'impur.
\VS{15}Et la voix lui dit encore pour la seconde fois~: Les choses que Dieu a purifiées, ne les tiens point pour souillées.
\VS{16}Et cela arriva jusqu'à trois fois, et puis le vase fut retiré au ciel.
\VS{17}Comme Pierre ne savait pas en lui-même que penser du sens de la vision qu'il avait eue, voici, les hommes envoyés par Corneille s'étant mis en quête de la maison de Simon, se présentèrent à la porte,
\VS{18}et demandèrent à haute voix si c'était là que logeait Simon, surnommé Pierre.
\VS{19}Et comme Pierre pensait à la vision, l'Esprit lui dit~: Voici trois hommes qui te demandent.
\VS{20}Lève-toi donc et descends, et pars avec eux sans hésiter, car c'est moi qui les ai envoyés.
\VS{21}Pierre donc, descendit vers les gens qui lui avaient été envoyés par Corneille et leur dit~: Voici, je suis celui que vous cherchez~; pour quel sujet êtes-vous venus~?
\VS{22}Et ils dirent~: Corneille, centenier, homme juste et craignant Dieu, et à qui toute la nation des Juifs rend un bon témoignage, a été averti de Dieu par un saint ange de te faire venir dans sa maison et d'entendre tes paroles.
\TextTitle{Pierre chez Corneille}
\VS{23}Alors Pierre les fit entrer, et les logea. Le lendemain il s'en alla avec eux, et quelques-uns des frères de Joppé l'accompagnèrent.
\VS{24}Ils arrivèrent à Césarée le jour suivant. Corneille les attendait, et avait invité ses parents et ses amis.
\VS{25}Lorsque Pierre entra, Corneille qui était allé au-devant de lui, se jeta à ses pieds, et se prosterna.
\VS{26}Mais Pierre le releva en lui disant~: Lève-toi, moi aussi je suis un homme.
\VS{27}Et s'entretenant avec lui, il entra et trouva plusieurs personnes réunies.
\VS{28}Et il leur dit~: Vous savez qu'il n'est pas permis à un homme Juif de se lier avec un étranger, ou d'aller chez lui, mais Dieu m'a montré que je ne devais estimer aucun homme être impur ou souillé.
\VS{29}C'est pourquoi, ayant été appelé, je suis venu sans difficulté. Je vous demande donc pour quel sujet vous m'avez fait venir.
\VS{30}Corneille lui dit~: Il y a quatre jours, à cette heure-ci, j'étais en jeûne et en prière dans ma maison, et tout à coup, un homme, vêtu d'un habit resplendissant, se présenta devant moi et me dit~:
\VS{31}Corneille, ta prière est exaucée, et Dieu s'est souvenu de tes aumônes.
\VS{32}Envoie donc quelqu'un à Joppé, et fais venir Simon, surnommé Pierre, qui est logé dans la maison de Simon, le corroyeur, près de la mer. Quand il sera venu, il te parlera.
\VS{33}Aussitôt j'ai envoyé quelqu'un vers toi, et tu as bien fait de venir. Maintenant donc nous sommes tous présents devant Dieu pour entendre tout ce que Dieu t'a ordonné de nous dire.
\TextTitle{Pierre évangélise les Gentils\FTNTT{\vref{Ac. 2:14-41}}}
\VS{34}Alors Pierre prenant la parole, dit~: En vérité, je reconnais que Dieu n'a point égard à l'apparence des personnes,
\VS{35}mais qu'en toute nation celui qui le craint et qui pratique la justice, lui est agréable.
\VS{36}C'est ce qu'il a fait entendre aux enfants d'Israël, en leur annonçant la paix par Jésus-Christ, qui est le Seigneur de tous.
\VS{37}Vous savez ce qui est arrivé dans toute la Judée, après avoir commencé en Galilée, à la suite du baptême que Jean a prêché~;
\VS{38}vous savez comment Dieu a oint du Saint-Esprit et de force Jésus de Nazareth, qui allait de lieu en lieu, faisant du bien et guérissant tous ceux qui étaient sous l'empire du diable, car Dieu était avec lui.
\VS{39}Nous sommes témoins de toutes les choses qu'il a faites, dans le pays des Juifs et à Jérusalem. Cependant ils l'ont fait mourir en le pendant au bois.
\VS{40}Dieu l'a ressuscité le troisième jour, et il a permis qu'il apparaisse,
\VS{41}non à tout le peuple, mais aux témoins choisis d'avance par Dieu, à nous, qui avons mangé et bu avec lui après qu'il fut ressuscité des morts.
\VS{42}Et il nous a ordonné de prêcher au peuple, et d'attester que c'est lui qui a été établi par Dieu, juge des vivants et des morts.
\VS{43}Tous les prophètes rendent de lui le témoignage que quiconque croit en lui, reçoit la rémission de ses péchés par son Nom.
\TextTitle{Le Saint-Esprit descend sur les Gentils}
\VS{44}Comme Pierre prononçait encore ce discours, le Saint-Esprit descendit sur tous ceux qui écoutaient la parole.
\VS{45}Tous les fidèles circoncis qui étaient venus avec Pierre, furent étonnés de ce que le don du Saint-Esprit était aussi répandu sur les Gentils.
\VS{46}Car ils les entendaient parler diverses langues et glorifier Dieu.
\VS{47}Alors Pierre prenant la parole, dit~: Quelqu'un pourrait-il empêcher qu'on baptise dans l'eau ceux qui ont reçu le Saint-Esprit aussi bien que nous~?
\VS{48}Et il ordonna qu'ils soient baptisés au Nom du Seigneur. Après cela, ils le prièrent de rester quelques jours auprès d'eux.
\Chap{11}
\TextTitle{Dieu accorde la repentance aux Gentils}
\VerseOne{}Or les apôtres et les frères qui étaient en Judée apprirent que les Gentils aussi avaient reçu la parole de Dieu.
\VS{2}Et quand Pierre fut monté à Jérusalem, ceux de la circoncision disputaient contre lui,
\VS{3}disant~: Tu es entré chez des hommes incirconcis, et tu as mangé avec eux.
\VS{4}Alors Pierre commençant, leur exposa le tout par ordre, disant~:
\VS{5}J'étais dans la ville de Joppé, et pendant que je priais, je tombai en extase et j'eus une vision. Un vase semblable à une grande nappe, attachée par les quatre coins, descendit du ciel, et vint jusqu'à moi.
\VS{6}Les regards fixés sur cette nappe, j'examinai, et je vis les quadrupèdes, les bêtes sauvages, les reptiles, et les oiseaux du ciel.
\VS{7}Et j'entendis une voix qui me disait~: Pierre, lève-toi, tue, et mange.
\VS{8}Et je répondis~: Non Seigneur, car jamais rien de souillé ni d'impur n'est entré dans ma bouche.
\VS{9}La voix me parla du ciel une seconde fois~: Ce que Dieu a déclaré pur, ne le regarde pas comme souillé.
\VS{10}Cela arriva jusqu'à trois fois, puis toutes ces choses furent retirées dans le ciel.
\VS{11}Et voici, aussitôt trois hommes qui avaient été envoyés de Césarée vers moi, se présentèrent à la maison où j'étais.
\VS{12}L'Esprit me dit de partir avec eux sans hésiter. Les six frères que voici m'accompagnèrent, et nous entrâmes dans la maison de Corneille.
\VS{13}Cet homme nous raconta comment il avait vu dans sa maison un ange qui s'était présenté à lui, et lui avait dit~: Envoie des gens à Joppé, et fais venir Simon, surnommé Pierre,
\VS{14}qui te dira des choses par lesquelles tu seras sauvé, toi et toute ta maison.
\VS{15}Lorsque je me fus mis à parler, le Saint-Esprit descendit sur eux, comme il était descendu sur nous au commencement.
\VS{16}Et je me souvins de cette parole du Seigneur, et comment il avait dit~: Jean a baptisé d'eau, mais vous, vous serez baptisés du Saint-Esprit.
\VS{17}Or puisque Dieu leur a accordé le même don qu'à nous qui avons cru au Seigneur Jésus-Christ, pouvais-je, moi, m'opposer à Dieu~?
\VS{18}Après avoir entendu ces choses, ils s'apaisèrent, et ils glorifièrent Dieu en disant~: Dieu a donc accordé la repentance aussi aux Gentils, afin qu'ils aient la vie.
\TextTitle{Les disciples appelés «~chrétiens~» pour la première fois à Antioche}
\VS{19}Ceux qui avaient été dispersés par la persécution survenue à cause d'Etienne, allèrent jusqu'en Phénicie, dans l'île de Chypre, et à Antioche\FTNT{Antioche~: Capitale de la Syrie située sur le fleuve Oronte, fondée en 300 av. J.-C., et ainsi nommée en l'honneur de son fondateur Antiochus. De nombreux Juifs grecs y vivaient et c'est là que les disciples de Christ furent appelés pour la première fois, chrétiens.}, n'annonçant la parole à personne, seulement aux Juifs.
\VS{20}Mais il y eut parmi eux quelques hommes de Chypre et de Cyrène qui, étant venus à Antioche, parlèrent aussi aux Grecs, et leur annoncèrent l'Evangile du Seigneur Jésus.
\VS{21}La main du Seigneur était avec eux, et un grand nombre de personnes crurent et se convertirent au Seigneur.
\VS{22}Le bruit en parvint aux oreilles de l'Eglise qui était à Jérusalem, et ils envoyèrent Barnabas jusqu'à Antioche.
\VS{23}Lorsqu'il fut arrivé, et qu'il eut vu la grâce de Dieu, il s'en réjouit, et il les exhortait tous à demeurer attachés au Seigneur de tout leur cœur.
\VS{24}Car c'était un homme de bien, plein du Saint-Esprit et de foi. Et un grand nombre de personnes se joignirent au Seigneur.
\VS{25}Barnabas s'en alla à Tarse pour chercher Saul~;
\VS{26}et l'ayant trouvé, il l'amena à Antioche. Pendant toute une année, ils se réunirent aux assemblées de l'Eglise, et ils enseignèrent beaucoup de personnes. Ce fut à Antioche que, pour la première fois, les disciples furent appelés chrétiens.
\TextTitle{Prophétie d'Agabus}
\VS{27}En ce temps-là, quelques prophètes descendirent de Jérusalem à Antioche.
\VS{28}L'un d'eux, nommé Agabus, se leva et déclara par l'Esprit qu'une grande famine devait arriver sur toute la terre. Elle arriva, en effet, sous Claude César.
\VS{29}Les disciples résolurent d'envoyer, chacun selon ses moyens, quelque secours pour subvenir aux besoins des frères qui habitaient la Judée.
\VS{30}Ils le firent parvenir aux anciens par les mains de Barnabas et de Saul.
\Chap{12}
\TextTitle{Cinquième persécution de l'Eglise~: Meurtre de Jacques et arrestation de Pierre}
\VerseOne{}En ce même temps, le roi Hérode se mit à maltraiter quelques membres de l'Eglise~;
\VS{2}et il fit mourir par l'épée Jacques, frère de Jean.
\VS{3}Voyant que cela était agréable aux Juifs, il fit aussi arrêter Pierre. C'était pendant les jours des pains sans levain.
\VS{4}Après l'avoir saisi et jeté en prison, il le mit sous la garde de quatre bandes de quatre soldats chacune, avec l'intention de le faire comparaître devant le peuple après la fête de Pâque.
\TextTitle{L'Ange du Seigneur délivre Pierre de la prison}
\VS{5}Pierre était donc gardé dans la prison~; mais l'Eglise faisait sans cesse des prières à Dieu pour lui.
\VS{6}La nuit qui précéda le jour où Hérode devait l'envoyer au supplice, Pierre dormait entre deux soldats, lié de deux chaînes~; et les gardes qui étaient devant la porte gardaient la prison.
\VS{7}Et voici, l'Ange du Seigneur survint, et une lumière resplendit dans la prison. L'Ange réveilla Pierre en le frappant au côté, et en disant~: Lève-toi promptement~! Et les chaînes tombèrent de ses mains.
\VS{8}Et l'Ange lui dit~: Mets ta ceinture et tes sandales. Et il fit ainsi. L'Ange lui dit encore~: Enveloppe-toi de ton manteau et suis-moi.
\VS{9}Pierre sortit et le suivit, ne sachant pas que ce qui se faisait par l'Ange était réel, car il croyait qu'il avait une vision.
\VS{10}Lorsqu'ils eurent passé la première et la seconde garde, ils arrivèrent à la porte de fer qui mène à la ville, et qui s'ouvrit d'elle-même devant eux~; et ils sortirent et s'avancèrent dans une rue. Et subitement, l'Ange quitta Pierre.
\VS{11}Revenu à lui-même, Pierre dit~: Je vois à présent d'une manière certaine que le Seigneur a envoyé son Ange, et qu'il m'a délivré de la main d'Hérode, et de toute l'attente du peuple Juif.
\VS{12}Après avoir réfléchi, il alla à la maison de Marie, mère de Jean, surnommé Marc, où plusieurs personnes étaient assemblées et priaient.
\VS{13}Il frappa à la porte du vestibule, une servante, appelée Rhode, vint pour écouter.
\VS{14}Elle reconnut la voix de Pierre, et dans sa joie elle n'ouvrit pas la porte du vestibule, mais elle courut dans la maison et annonça que Pierre était devant la porte.
\VS{15}Ils lui dirent~: Tu es folle. Mais elle affirma que ce qu'elle disait était vrai.
\VS{16}Et ils dirent~: C'est son ange. Cependant Pierre continuait à frapper. Et quand ils eurent ouvert, ils le virent, et furent étonnés de le voir.
\VS{17}Mais leur ayant fait signe de la main de se taire, il leur raconta comment le Seigneur l'avait fait sortir de la prison, et il leur dit~: Annoncez ces choses à Jacques et aux frères. Puis sortant de là il s'en alla dans un autre lieu.
\VS{18}Quand il fit jour, les soldats furent dans une grande agitation, pour savoir ce que Pierre était devenu.
\VS{19}Et Hérode l'ayant cherché, et ne le trouvant point, après en avoir fait le procès aux gardes, il commanda qu'ils fussent menés au supplice.
\TextTitle{Mort d'Hérode}
\VS{20}Hérode avait le dessein de faire la guerre aux Tyriens et aux Sidoniens~; mais ils vinrent le trouver d'un commun accord~; et ayant gagné Blaste, son Chambellan, ils demandèrent la paix, parce que leur pays tirait sa subsistance de celui du roi.
\VS{21}A un jour marqué, Hérode, revêtu de ses habits royaux, s'assit sur son trône et les harangua publiquement.
\VS{22}Le peuple s'écria~: Voix d'un dieu et non point d'un homme~!
\VS{23}Et à l'instant l'Ange du Seigneur le frappa, parce qu'il n'avait pas donné gloire à Dieu. Et il expira, rongé des vers.
\VS{24}Cependant la parole de Dieu se répandait de plus en plus, et le nombre des disciples augmentait.
\VS{25}Barnabas et Saul, après s'être acquittés de leur service, s'en retournèrent de Jérusalem, ayant aussi pris avec eux Jean, surnommé Marc.
\Chap{13}
\TextTitle{Saul et Barnabas mis à part par le Saint-Esprit}
\VerseOne{}Or il y avait dans l'église qui était à Antioche des prophètes et des docteurs, Barnabas, Siméon, appelé Niger, Lucius, le Cyrénien, Manahen, qui avait été élevé avec Hérode, le tétrarque, et Saul.
\VS{2}Et tandis qu'ils servaient\FTNT{Certains traducteurs ont rajouté la phrase «~dans leur ministère~» alors que les textes originaux ne la mentionne pas.} le Seigneur et jeûnaient, le Saint-Esprit dit~: Séparez-moi maintenant Barnabas et Saul pour l'œuvre à laquelle je les ai appelés.
\VS{3}Alors après avoir jeûné et prié, ils leur imposèrent les mains, et les laissèrent partir\FTNT{Voir annexe «~Les voyages missionnaires de Paul~».}.
\TextTitle{Saul, Barnabas et Jean sur l'île de Chypre}
\VS{4}Barnabas et Saul, envoyés par le Saint-Esprit, descendirent à Séleucie, et de là ils s'embarquèrent pour l'île de Chypre.
\VS{5}Et lorsqu'ils furent à Salamine, ils annoncèrent la parole de Dieu dans les synagogues des Juifs~; ils avaient Jean avec eux pour les aider.
\TextTitle{Bar-Jésus aveuglé et conversion du proconsul Sergius Paulus}
\VS{6}Ayant ensuite traversé l'île jusqu'à Paphos, ils trouvèrent là un certain magicien, faux prophète Juif, nommé Bar-Jésus,
\VS{7}qui était avec le proconsul Sergius Paulus, homme intelligent qui fit appeler Barnabas et Saul, désirant entendre la parole de Dieu.
\VS{8}Mais Elymas, le magicien, car c'est ce que signifie ce nom, leur résistait, cherchant à détourner de la foi le proconsul.
\VS{9}Alors Saul, appelé aussi Paul, rempli du Saint-Esprit, fixa les yeux sur lui et dit~:
\VS{10}Ô homme plein de toute fraude et de toute ruse, fils du diable, ennemi de toute justice, ne cesseras-tu point de renverser les voies droites du Seigneur~?
\VS{11}C'est pourquoi, voici la main du Seigneur est sur toi, tu seras aveugle, et pour un temps tu ne verras pas le soleil. Aussitôt l'obscurité et les ténèbres tombèrent sur lui, et il cherchait, en tâtonnant, des personnes pour le guider.
\VS{12}Alors le proconsul voyant ce qui était arrivé, crut, étant rempli d'admiration pour la doctrine du Seigneur.
\VS{13}Et quand Paul et ceux qui étaient avec lui furent partis de Paphos, ils vinrent à Perge, ville de Pamphylie. Jean se sépara d'eux et retourna à Jérusalem.
\TextTitle{Paul à Antioche de Pisidie~: Le salut par la foi en Jésus}
\VS{14}De Perge, ils poursuivirent leur route, et arrivèrent à Antioche, ville de Pisidie\FTNT{Antioche de Pisidie~: Ville de Pisidie (en Turquie), à la frontière de Phrygie, fondée par Seleucus Nicanor. Elle devint une colonie romaine et fut aussi appelée Césarée.}, et étant entrés dans la synagogue le jour du sabbat, ils s'assirent.
\VS{15}Après la lecture de la loi et des prophètes, les chefs de la synagogue leur envoyèrent dire~: Hommes frères, si vous avez quelque parole d'exhortation pour le peuple, dites-la.
\VS{16}Alors Paul s'étant levé, et ayant fait signe de la main qu'on fasse silence, dit~: Hommes Israélites, et vous qui craignez Dieu, écoutez.
\VS{17}Le Dieu de ce peuple d'Israël a choisi nos pères. Il a distingué glorieusement ce peuple pendant son séjour au pays d'Egypte, et il l'en fit sortir par son bras élevé.
\VS{18}Il les supporta\FTNT{Le verbe supporter vient du grec «~tropophoreo~» qui signifie supporter les manières, endurer le caractère de quelqu'un.} au désert environ quarante ans.
\VS{19}Et ayant détruit sept nations au pays de Canaan, il leur distribua le pays par le sort.
\VS{20}Après cela, durant quatre cent cinquante ans, il leur donna des juges, jusqu'à Samuel le prophète.
\VS{21}Puis ils demandèrent un roi, et Dieu leur donna Saül fils de Kis, homme de la tribu de Benjamin~; et ainsi se passèrent quarante ans.
\VS{22}Et Dieu l'ayant rejeté, il leur suscita pour roi David, auquel il a rendu ce témoignage~: J'ai trouvé David, fils d'Isaï, homme selon mon cœur, qui exécutera toute ma volonté.
\VS{23}C'est de la postérité de David que Dieu, selon sa promesse, a suscité Jésus pour être le Sauveur d'Israël.
\VS{24}Avant la venue de Jésus, Jean avait prêché le baptême de repentance à tout le peuple d'Israël.
\VS{25}Et comme Jean achevait sa course, il disait~: Qui pensez-vous que je sois~? Je ne suis point le Christ~; mais voici, il en vient un après moi, dont je ne suis pas digne de délier le soulier de ses pieds.
\VS{26}Hommes frères, fils de la race d'Abraham, et vous qui craignez Dieu, c'est à vous que la parole de ce salut a été envoyée.
\VS{27}Car les habitants de Jérusalem et leurs chefs ont méconnu Jésus, et en le condamnant, ils ont accompli les paroles des prophètes qui se lisent chaque sabbat.
\VS{28}Quoiqu'ils n'aient rien trouvé en lui qui soit digne de mort, ils demandèrent à Pilate de le faire mourir.
\VS{29}Et après qu'ils eurent accompli toutes les choses qui avaient été écrites de lui, ils le descendirent du bois, et le déposèrent dans un sépulcre.
\VS{30}Mais Dieu l'a ressuscité des morts.
\VS{31}Il est apparu pendant plusieurs jours à ceux qui étaient montés avec lui de Galilée à Jérusalem, et qui sont ses témoins devant le peuple.
\VS{32}Et nous, nous vous annonçons cette bonne nouvelle que la promesse faite à nos pères,
\VS{33}Dieu l'a accomplie pour nous, leurs enfants, en ressuscitant Jésus, selon qu'il est écrit dans le deuxième psaume~: Tu es mon Fils, je t'ai aujourd'hui engendré\FTNT{\vref{Ps. 2:7}.}.
\VS{34}Et pour montrer qu'il l'a ressuscité des morts, pour ne plus devoir retourner au sépulcre, il a dit ainsi~: Je vous donnerai les grâces saintes promises à David, ces grâces qui sont assurées.
\VS{35}C'est pourquoi il a dit aussi dans un autre endroit~: Tu ne permettras point que ton Saint voie la corruption\FTNT{\vref{Ps. 16:10}.}.
\VS{36}Or David, après avoir servi en son temps au dessein de Dieu, est mort, a été réuni à ses pères, et a vu la corruption.
\VS{37}Mais celui que Dieu a ressuscité n'a pas vu la corruption.
\VS{38}Sachez donc, hommes frères, que c'est par lui que la rémission des péchés vous est annoncée,
\VS{39}et que quiconque croit est justifié par lui, de tout ce dont vous n'avez pas pu être justifiés par la loi de Moïse.
\VS{40}Prenez donc garde qu'il ne vous arrive ce qui est dit dans les prophètes~:
\VS{41}Voyez, vous mépriseurs, soyez étonnés et disparaissez~: Car je vais faire une œuvre en votre temps, une œuvre que vous ne croiriez pas si quelqu'un vous la racontait.
\VS{42}Lorsqu'ils sortirent de la synagogue des Juifs, les Gentils les prièrent de parler le sabbat suivant sur les mêmes choses.
\VS{43}Et quand l'assemblée fut séparée, beaucoup de Juifs et de prosélytes craignant Dieu, suivirent Paul et Barnabas qui les exhortèrent à persévérer dans la grâce de Dieu.
\TextTitle{Les juifs d'Antioche rejettent la Parole~; l'Évangile annoncé aux Gentils\FTNTT{\vref{Ac. 18:6}~; \vref{28:25-28}.}}
\VS{44}Le sabbat suivant, presque toute la ville s'assembla pour entendre la parole de Dieu.
\VS{45}Mais les Juifs voyant toute cette foule, furent remplis de jalousie, et ils s'opposaient à ce que Paul disait, en le contredisant et en blasphémant.
\VS{46}Alors Paul et Barnabas leur dirent avec assurance~: C'est à vous premièrement qu'il fallait annoncer la parole de Dieu, mais puisque vous la rejetez, et que vous vous jugez vous-mêmes indignes de la vie éternelle, voici, nous nous tournons vers les Gentils.
\VS{47}Car ainsi nous l'a ordonné le Seigneur~: Je t'ai établi pour être la lumière des Gentils, pour porter le salut jusqu'aux extrémités de la terre.
\VS{48}Les Gentils en entendant cela, se réjouissaient et ils glorifiaient la parole du Seigneur~; et tous ceux qui étaient destinés à la vie éternelle crurent.
\VS{49}Ainsi la parole du Seigneur se répandait dans tout le pays.
\VS{50}Mais les Juifs excitèrent quelques femmes dévotes et distinguées, et les principaux de la ville, et ils provoquèrent une persécution contre Paul et Barnabas, et les chassèrent de leur territoire.
\VS{51}Paul et Barnabas secouèrent contre eux la poussière de leurs pieds et allèrent à Icone,
\VS{52}tandis que les disciples étaient remplis de joie et du Saint-Esprit.
\Chap{14}
\TextTitle{Paul et Barnabas à Icone}
\VerseOne{}A Icone, Paul et Barnabas entrèrent ensemble dans la synagogue des Juifs, et ils parlèrent d'une telle manière qu'une grande multitude de Juifs et de Grecs crurent.
\VS{2}Mais ceux des Juifs qui furent rebelles, émurent et irritèrent les esprits des Gentils contre les frères.
\VS{3}Ils restèrent cependant assez longtemps à Icone, parlant avec assurance du Seigneur, qui rendait témoignage à la parole de sa grâce, en faisant par leurs mains des prodiges et des miracles.
\VS{4}La population de la ville fut partagée en deux, et les uns étaient du côté des Juifs, et les autres du côté des apôtres.
\TextTitle{Paul prêche à Derbe et à Lystre~; guérison d'un boiteux de naissance}
\VS{5}Et comme il se faisait une émeute des Gentils et des Juifs, avec leurs principaux chefs, pour outrager et lapider les apôtres,
\VS{6}Paul et Barnabas en ayant eu connaissance, se réfugièrent dans les villes de Lycaonie, à Lystre, à Derbe, et dans les contrées d'alentour.
\VS{7}Et ils y annoncèrent l'Evangile.
\VS{8}A Lystre, se tenait assis un homme impotent des pieds, boiteux dès sa naissance, et qui n'avait jamais marché.
\VS{9}Cet homme écoutait parler Paul. Et Paul fixant ses yeux sur lui, et voyant qu'il avait la foi pour être guéri,
\VS{10}lui dit à haute voix~: Lève-toi droit sur tes pieds. Et il se leva en sautant, et marcha.
\VS{11}Et les gens qui étaient là assemblés, ayant vu ce que Paul avait fait, élevèrent leur voix, disant en langue lycaonienne~: Les dieux sous une forme humaine, sont descendus vers nous.
\VS{12}Et ils appelaient Barnabas Jupiter, et Paul Mercure, parce que c'était lui qui portait la parole.
\VS{13}Le prêtre de Jupiter, qui était à l'entrée de leur ville, ayant amené des taureaux et des couronnes jusqu'à l'entrée de la porte voulait, de même que la foule, offrir un sacrifice.
\VS{14}Mais les apôtres Barnabas et Paul ayant appris cela, déchirèrent leurs vêtements et se précipitèrent au milieu de la foule,
\VS{15}et disant~: Ô hommes, pourquoi faites-vous cela~? Nous aussi, nous sommes des hommes, sujets aux mêmes passions que vous, et vous apportant l'Evangile, nous vous exhortons à renoncer à ces choses vaines, pour vous convertir au Dieu vivant, qui a fait le ciel et la terre, la mer, et tout ce qui s'y trouve.
\VS{16}Ce Dieu, dans les siècles passés, a laissé toutes les nations marcher dans leurs voies,
\VS{17}quoiqu'il n'ait cessé de rendre témoignage de ce qu'il est, en faisant du bien, en nous dispensant du ciel les pluies et les saisons fertiles, en nous donnant la nourriture avec abondance, et en remplissant nos cœurs de joie.
\VS{18}A peine purent-ils, par ces paroles, empêcher la foule de leur offrir un sacrifice.
\TextTitle{Paul lapidé à Lystre}
\VS{19}Alors survinrent quelques Juifs d'Antioche et d'Icone qui gagnèrent la foule, et qui après avoir lapidé Paul, le traînèrent hors de la ville, croyant qu'il était mort.
\VS{20}Mais les disciples s'étant assemblés autour de lui, il se leva et entra dans la ville~; et le lendemain il s'en alla avec Barnabas à Derbe.
\TextTitle{Vote et établissement des anciens dans les églises}
\VS{21}Quand ils eurent évangélisé cette ville, et fait un certain nombre de disciples, ils retournèrent à Lystre, à Icone, et à Antioche~;
\VS{22}fortifiant l'esprit des disciples, et les exhortant à persévérer dans la foi, disant que c'est par beaucoup de tribulations qu'il nous faut entrer dans le Royaume de Dieu.
\VS{23}Après le vote à main levée des assemblées, ils établirent des anciens dans chaque église, et après avoir prié et jeûné, ils les recommandèrent au Seigneur, en qui ils avaient cru.
\VS{24}Traversant ensuite la Pisidie, ils allèrent en Pamphylie,
\VS{25}annoncèrent la parole à Perge, et descendirent à Attalie.
\TextTitle{Retour à Antioche}
\VS{26}De là, ils s'embarquèrent pour Antioche, d'où ils avaient été recommandés à la grâce de Dieu, pour l'œuvre qu'ils venaient d'accomplir.
\VS{27}Et quand ils furent arrivés, ils convoquèrent l'église, et ils racontèrent toutes les choses que Dieu avait faites par eux, et comment il avait ouvert aux Gentils la porte de la foi.
\VS{28}Et ils demeurèrent assez longtemps avec les disciples.
\Chap{15}
\TextTitle{Des hommes venus de Judée veulent imposer la circoncision}
\VerseOne{}Quelques hommes qui étaient descendus de Judée, enseignaient les frères en disant~: Si vous n'êtes pas circoncis selon le rite de Moïse, vous ne pouvez pas être sauvés.
\VS{2}Paul et Barnabas eurent avec eux un débat et une vive discussion~; et les frères décidèrent que Paul et Barnabas, avec quelques-uns des leurs, monteraient à Jérusalem vers les apôtres et les anciens, pour traiter cette question.
\VS{3}Après avoir été accompagnés par l'assemblée, ils traversèrent la Phénicie et la Samarie, racontant la conversion des Gentils~; et ils causèrent une grande joie à tous les frères.
\VS{4}Arrivés à Jérusalem, ils furent reçus par l'église, les apôtres et les anciens, et ils racontèrent toutes les choses que Dieu avait faites par leur moyen.
\VS{5}Mais quelques-uns, de la secte des pharisiens qui avaient cru, se levèrent, en disant qu'il fallait circoncire les Gentils et leur ordonner de garder la loi de Moïse.
\TextTitle{Opposition de Pierre~: Les Gentils n'ont pas à être sous le joug de la loi}
\VS{6}Alors les apôtres et les anciens se réunirent pour examiner cette affaire.
\VS{7}Et après une grande discussion, Pierre se leva et leur dit~: Hommes frères, vous savez que depuis longtemps Dieu m'a choisi parmi nous, afin que par ma bouche, les Gentils entendent la parole de l'Evangile, et qu'ils croient.
\VS{8}Et Dieu, qui connaît les cœurs, leur a rendu témoignage en leur donnant le Saint-Esprit, de même qu'à nous.
\VS{9}Il n'a fait aucune différence entre nous et eux, ayant purifié leurs cœurs par la foi.
\VS{10}Maintenant donc pourquoi tentez-vous Dieu en voulant imposer aux disciples un joug que ni nos pères ni nous n'avons pu porter~?
\VS{11}Mais nous croyons que nous serons sauvés par la grâce du Seigneur Jésus-Christ, comme eux aussi.
\VS{12}Alors toute l'assemblée garda le silence, et l'on écouta Barnabas et Paul qui racontèrent tous les miracles et les prodiges que Dieu avait faits par leur moyen au milieu des Gentils.
\TextTitle{Discours de Jacques~: Les prophètes ont annoncé le salut pour les Gentils} 
\VS{13}Lorsqu'ils eurent cessé de parler, Jacques prit la parole et dit~: Hommes frères, écoutez-moi~!
\VS{14}Simon a raconté comment Dieu a premièrement jeté les regards sur les nations pour choisir du milieu d'elles un peuple consacré à son Nom. 
\VS{15}Et avec cela s'accordent les paroles des prophètes, selon qu'il est écrit~:
\VS{16}Après cela, je reviendrai, et je rebâtirai le tabernacle de David qui est tombé, je réparerai ses ruines et je le relèverai\FTNT{\vref{Am. 9:11}.}.
\VS{17}Afin que le reste des hommes recherche le Seigneur, et aussi toutes les nations sur lesquelles mon Nom est invoqué, dit le Seigneur, qui fait toutes ces choses.
\VS{18}Toutes les œuvres de Dieu lui sont connues de toute éternité.
\TextTitle{Les chrétiens issus des nations ne sont pas soumis à la loi mosaïque}
\VS{19}C'est pourquoi je suis d'avis qu'on ne crée pas des difficultés à ceux des Gentils qui se convertissent à Dieu~;
\VS{20}mais qu'on leur écrive de s'abstenir des souillures des idoles et de la fornication, des animaux étouffés et du sang.
\VS{21}Car depuis bien des générations, Moïse a, dans chaque ville, des gens qui le prêchent, puisqu'on le lit tous les jours de sabbat dans les synagogues.
\VS{22}Alors il parut bon aux apôtres et aux anciens avec toute l'Eglise, de choisir parmi eux et d'envoyer à Antioche avec Paul et Barnabas, Jude, appelé Barsabas, et Silas, hommes considérés entre les frères.
\VS{23}Ils écrivirent par eux en ces termes~: Les apôtres, les anciens, et les frères, aux frères d'entre les Gentils qui sont à Antioche, en Syrie, et en Cilicie, salut~!
\VS{24}Ayant appris que quelques hommes partis de chez nous, et auxquels nous n'avons donné aucun ordre, vous ont troublés par leurs discours et ont ébranlé vos âmes, en vous disant qu'il faut être circoncis et garder la loi,
\VS{25}nous avons été d'avis, étant assemblés tous d'un commun accord, d'envoyer vers vous, avec nos très chers Barnabas et Paul, des hommes que nous avons choisis. 
\VS{26}Ce sont des hommes qui ont abandonné leurs vies pour le Nom de notre Seigneur Jésus-Christ.
\VS{27}Nous avons donc envoyé Jude et Silas, qui vous feront entendre les mêmes choses de vive voix.
\VS{28}Car il a paru bon au Saint-Esprit et à nous, de ne vous imposer d'autre charge que ce qui est nécessaire,
\VS{29}savoir, de vous abstenir des viandes sacrifiées aux idoles, du sang, des animaux étouffés, et de la fornication~; choses contre lesquelles vous vous trouverez bien de vous tenir en garde. Adieu~!
\TextTitle{Mission de Jude et Silas à Antioche}
\VS{30}Après avoir donc pris congé de l'église, ils allèrent à Antioche, et ayant assemblé l'église, ils remirent la lettre.
\VS{31}Après l'avoir lue, les frères d'Antioche furent réjouis de la consolation qu'elle leur apportait.
\VS{32}Jude et Silas, qui étaient eux-mêmes prophètes, exhortèrent les frères par plusieurs discours, et les fortifièrent.
\VS{33}Au bout de quelque temps, ils furent renvoyés en paix par les frères vers les apôtres.
\VS{34}Toutefois Silas trouva bon de rester.
\VS{35}Et Paul et Barnabas demeurèrent aussi à Antioche, enseignant et annonçant, avec plusieurs autres, la parole du Seigneur.
\TextTitle{Paul et Barnabas se séparent}
\VS{36}Quelques jours après, Paul dit à Barnabas~: Retournons visiter nos frères dans toutes les villes où nous avons annoncé la parole du Seigneur, pour voir quel est leur état\FTNT{Voir annexe «~Les voyages missionaires de Paul~».}.
\VS{37}Barnabas voulait emmener avec eux Jean, surnommé Marc.
\VS{38}Mais Paul jugea plus convenable de ne pas prendre avec eux celui qui les avait quittés depuis la Pamphylie, et qui ne les avait point accompagnés dans leur œuvre.
\VS{39}Il y eut donc entre eux une contestation, en sorte qu'ils se séparèrent l'un de l'autre. Barnabas, prenant Marc avec lui, s'embarqua pour l'île de Chypre.
\VS{40}Mais Paul, ayant choisi Silas, pour l'accompagner, partit après avoir été recommandé à la grâce de Dieu par les frères.
\VS{41}Il traversa la Syrie et la Cilicie, fortifiant les églises.
\Chap{16}
\TextTitle{Circoncis, Timothée rejoint Paul dans la mission}
\VerseOne{}Il se rendit à Derbe et à Lystre, et voici, il y avait là un disciple, nommé Timothée, fils d'une femme juive fidèle et d'un père grec.
\VS{2}Les frères de Lystre et d'Icone rendaient de lui un bon témoignage.
\VS{3}C'est pourquoi Paul voulut l'emmener avec lui~; et l'ayant pris, il le circoncit, à cause des Juifs qui étaient dans ces lieux-là, car ils savaient tous que son père était grec.
\VS{4}En passant par les villes, ils recommandaient aux frères d'observer les ordonnances établies par les apôtres et les anciens de Jérusalem.
\VS{5}Ainsi les églises étaient affermies dans la foi et augmentaient en nombre chaque jour.
\TextTitle{Vision de Paul}
\VS{6}Ayant traversé la Phrygie et le pays de Galatie, le Saint-Esprit leur défendit d'annoncer la parole dans l'Asie.
\VS{7}Arrivés près de la Mysie, ils se disposaient à entrer en Bithynie~; mais l'Esprit de Jésus\FTNT{Notons que le Saint-Esprit est appelé Esprit de Jésus. Ainsi, de la même manière qu'on ne peut dissocier un homme de son esprit pour en faire deux entités distinctes, on ne peut dissocier Jésus de son Esprit. Dieu est un.} ne le leur permit pas.
\VS{8}Ils traversèrent ensuite la Mysie, et descendirent à Troas.
\VS{9}Pendant la nuit, Paul eut une vision d'un homme macédonien qui se présenta devant lui, et le pria, disant~: Passe en Macédoine et secours-nous~!
\VS{10}Après cette vision de Paul, nous cherchâmes aussitôt à nous rendre en Macédoine, concluant que le Seigneur nous appelait à les évangéliser.
\TextTitle{Paul à Philippes}
\VS{11}Ainsi étant partis de Troas, nous fîmes voile directement vers la Samothrace, et le lendemain à Néapolis.
\VS{12}De là nous allâmes à Philippes, qui est la première ville d'un district de Macédoine, et une colonie romaine. Nous séjournâmes quelque temps dans la ville.
\VS{13}Et le jour du sabbat nous sortîmes de la ville, et allâmes au lieu où on avait accoutumé de faire la prière, près du fleuve, et nous étant là assis nous parlâmes aux femmes qui y étaient assemblées.
\TextTitle{Conversion de Lydie}
\VS{14}L'une d'elles, appelée Lydie, marchande de pourpre, de la ville de Thyatire, était une femme craignant Dieu, et elle nous écoutait. Le Seigneur lui ouvrit le cœur, afin qu'elle soit attentive à ce que disait Paul.
\VS{15}Lorsqu'elle eut été baptisée, avec sa famille, elle nous fit cette demande~: Si vous me jugez fidèle au Seigneur, entrez dans ma maison, et demeurez-y. Et elle nous pressa par ses instances.
\TextTitle{Paul et Silas battus de verges et mis en prison}
\VS{16}Or il arriva que comme nous allions à la prière, une servante qui avait un esprit de python, et qui, en devinant, apportait un grand profit à ses maîtres, nous rencontra,
\VS{17}et elle se mit à nous suivre, Paul et nous, en criant et disant~: Ces hommes sont les serviteurs du Dieu Très-Haut, et ils vous annoncent la voie du salut~!
\VS{18}Elle fit cela pendant plusieurs jours. Mais Paul, fatigué, se retourna et dit à l'esprit~: Je t'ordonne au Nom de Jésus-Christ de sortir de cette fille. Et il sortit au même instant.
\VS{19}Mais les maîtres de la servante voyant disparaître l'espoir de leur gain, se saisirent de Paul et de Silas, et les traînèrent sur la place publique devant les magistrats.
\VS{20}Ils les présentèrent aux préteurs, en disant~: Ces hommes, qui sont Juifs, troublent notre ville.
\VS{21}Car ils annoncent des coutumes qu'il ne nous est pas permis de recevoir ni de suivre, à nous qui sommes Romains.
\VS{22}La foule se souleva aussi contre eux, et les préteurs, ayant fait déchirer leurs vêtements, ordonnèrent qu'ils soient battus de verges.
\VS{23}Après qu'on les eut chargés de coups de fouet, ils les mirent en prison, en recommandant au geôlier de les garder sûrement.
\VS{24}Le geôlier ayant reçu cet ordre, les mit au fond de la prison, et leur serra les pieds dans des ceps.
\TextTitle{Libération miraculeuse de Paul et Silas}
\VS{25}Vers minuit, Paul et Silas priaient et chantaient les louanges de Dieu, et les prisonniers les entendaient.
\VS{26}Tout à coup, il se fit un grand tremblement de terre, en sorte que les fondements de la prison furent ébranlés~; au même instant, toutes les portes s'ouvrirent et les liens de tous furent rompus.
\VS{27}Le geôlier se réveilla, et voyant les portes de la prison ouvertes, il tira son épée et allait se tuer, croyant que les prisonniers s'étaient enfuis.
\VS{28}Mais Paul cria d'une voix forte~: Ne te fais point de mal, nous sommes tous ici.
\TextTitle{Conversion et baptême du geôlier et de sa famille}
\VS{29}Alors le geôlier, ayant demandé de la lumière, entra précipitamment dans le cachot, et se jeta tout tremblant aux pieds de Paul et de Silas.
\VS{30}Il les fit sortir, et dit~: Seigneur, que faut-il que je fasse pour être sauvé~?
\VS{31}Paul et Silas répondirent~: Crois au Seigneur Jésus-Christ et tu seras sauvé, toi et ta famille.
\VS{32}Et ils lui annoncèrent la parole du Seigneur, et à tous ceux qui étaient dans sa maison.
\VS{33}Après cela, les prenant en cette même heure de la nuit, il lava leurs plaies, et aussitôt après il fut baptisé, avec tous ceux de sa maison.
\VS{34}Les ayant amenés dans sa maison, il leur servit à manger, et il se réjouit avec toute sa famille de ce qu'il avait cru en Dieu.
\TextTitle{Paul et Silas relâchés}
\VS{35}Quand il fit jour, les préteurs envoyèrent des huissiers pour dire au geôlier~: Relâche ces hommes.
\VS{36}Et le geôlier rapporta ces paroles à Paul, disant~: Les préteurs ont envoyé dire qu'on vous relâche~; maintenant donc sortez, et allez en paix.
\VS{37}Mais Paul dit aux huissiers~: Après nous avoir battus de verges publiquement et sans jugement, nous qui sommes Romains, ils nous ont jetés en prison, et maintenant ils nous font sortir secrètement~! Il n'en sera pas ainsi. Qu'ils viennent eux-mêmes nous mettre en liberté.
\VS{38}Les licteurs rapportèrent ces paroles aux préteurs qui furent effrayés en apprenant qu'ils étaient Romains.
\VS{39}Ils vinrent vers eux et leur firent des excuses, et ils les mirent en liberté en les priant de quitter la ville.
\VS{40}Quand ils furent sortis de la prison, ils entrèrent chez Lydie, et après avoir vu et consolé les frères, ils partirent.
\Chap{17}
\TextTitle{Paul et Silas à Thessalonique}
\VerseOne{}Paul et Silas passèrent par Amphipolis et par Apollonie, et ils arrivèrent à Thessalonique, où les Juifs avaient une synagogue.
\VS{2}Paul y entra, selon sa coutume. Pendant trois sabbats, il discuta avec eux d'après les Ecritures~;
\VS{3}expliquant et établissant que le Christ devait souffrir et ressusciter des morts. Et ce Jésus, que je vous annonce, disait-il, c'est lui qui est le Christ.
\VS{4}Quelques-uns d'entre eux crurent, et se joignirent à Paul et à Silas, ainsi qu'une grande multitude de Grecs craignant Dieu, et beaucoup de femmes de qualité.
\TextTitle{Emeute à Thessalonique}
\VS{5}Mais les Juifs rebelles et jaloux, prirent avec eux quelques hommes méchants et fainéants de la populace, provoquèrent des attroupements, et répandirent l'agitation dans la ville. Ils se rendirent à la maison de Jason, et ils cherchèrent Paul et Silas, pour les amener vers le peuple.
\VS{6}Ne les ayant pas trouvés, ils traînèrent Jason et quelques frères devant les magistrats de la ville, en criant~: Ces gens, qui ont bouleversé le monde, sont aussi venus ici, et Jason les a reçus chez lui.
\VS{7}Ils sont tous rebelles aux édits de César, disant qu'il y a un autre Roi, qu'ils nomment Jésus.
\VS{8}Ils soulevèrent donc le peuple et les magistrats de la ville, qui, entendant ces choses,
\VS{9}ne laissèrent aller Jason et les autres qu'après avoir obtenu d'eux une caution. 
\TextTitle{Paul et Silas fuient à Bérée}
\VS{10}Aussitôt les frères firent partir de nuit Paul et Silas pour Bérée. Lorsqu'ils furent arrivés, ils entrèrent dans la synagogue des Juifs.
\VS{11}Ces Juifs avaient des sentiments plus nobles que ceux de Thessalonique~; ils reçurent la parole avec beaucoup de promptitude, et ils examinaient tous les jours les Ecritures, pour voir si ce qu'on leur disait était exact.
\VS{12}Plusieurs d'entre eux crurent, ainsi que des femmes grecques de distinction, et des hommes en assez grand nombre.
\VS{13}Mais quand les Juifs de Thessalonique surent que Paul annonçait aussi à Bérée la parole de Dieu, ils vinrent y agiter la foule.
\VS{14}Alors les frères firent aussitôt partir Paul du côté de la mer~; Silas et Timothée restèrent à Bérée.
\VS{15}Ceux qui avaient pris la charge de mettre Paul en sûreté, le conduisirent jusqu'à Athènes. Puis ils s'en retournèrent, après avoir reçu l'ordre de Paul de dire à Silas et à Timothée de le rejoindre au plus tôt.
\TextTitle{Paul à Athènes}
\VS{16}Comme Paul les attendait à Athènes, il sentit au-dedans son esprit s'irriter à la vue de cette ville entièrement adonnée à l'idolâtrie.
\VS{17}Il s'entretenait donc dans la synagogue avec les Juifs et les hommes craignant Dieu, et tous les jours sur la place publique avec ceux qui s'y rencontraient.
\VS{18}Quelques philosophes épicuriens\FTNT{L'épicurisme a été fondé par Epicure (341 av. J.-C. - 270 av. J.-C.). Cette philosophie est axée sur la recherche du bonheur par l'évitement de la souffrance et des inquiétudes (ataraxie).} et stoïciens\FTNT{Les stoïciens étaient disciples de Zénon (336-264 av. J.-C.). Leur philosophie se fondait sur la conception d'un homme se suffisant à lui-même, sur une discipline rigoureuse, et sur la solidarité du genre humain.} se mirent à parler avec lui. Et les uns disaient~: Que veut dire ce discoureur~? Les autres disaient~: Il semble qu'il annonce des divinités étrangères. Parce qu'il leur annonçait Jésus et la résurrection.
\VS{19}Alors ils le prirent et le menèrent à l'Aréopage\FTNT{A l'origine, l'aréopage désignait le tribunal d'Athènes qui siégeait sur la colline d'Arès. Le sens figuré est le suivant~: Assemblée de juges, de savants, d'hommes de lettres très compétents.}, et lui dirent~: Pourrions-nous savoir quelle est cette nouvelle doctrine que tu enseignes~?
\VS{20}Car tu nous remplis les oreilles de certaines choses étranges~; nous voudrions donc savoir ce que veulent dire ces choses.
\VS{21}Or tous les Athéniens et les étrangers qui demeuraient à Athènes, ne passaient leur temps qu'à dire ou à écouter des nouvelles.
\TextTitle{Prédication de Paul à l'Aréopage}
\VS{22}Paul, debout au milieu de l'Aréopage, leur dit~: Hommes Athéniens, je vous trouve à tous égards extrêmement religieux.
\VS{23}Car en passant et en regardant vos divinités, j'ai même trouvé un autel sur lequel était écrit~: Au Dieu inconnu~! Celui que vous révérez sans le connaître, c'est celui que je vous annonce.
\VS{24}Le Dieu qui a fait le monde et tout ce qui s'y trouve, étant le Seigneur du ciel et de la terre, n'habite point dans des temples faits de main d'homme.
\VS{25}Il n'est point servi par les mains des hommes, comme s'il avait besoin de quoi que ce soit, lui qui donne à tous la vie, la respiration, et toutes choses.
\VS{26}Il a fait que tous les hommes, sortis d'un seul sang, habitent sur toute l'étendue de la terre, ayant déterminé la durée des temps et les bornes de leur habitation.
\VS{27}Il a voulu qu'ils cherchent le Seigneur, et qu'ils s'efforcent de le trouver en tâtonnant, quoiqu'il ne soit pas loin de chacun de nous,
\VS{28}car c'est par lui que nous avons la vie, le mouvement et l'être. C'est ce qu'ont dit quelques-uns même de vos poètes~: De lui nous sommes la race.
\VS{29}Ainsi donc, étant de la race de Dieu, nous ne devons pas croire que la divinité soit semblable à de l'or, ou à de l'argent, ou à de la pierre taillée par l'art et l'industrie des hommes.
\VS{30}Mais Dieu, sans tenir compte des temps d'ignorance, annonce maintenant à tous les hommes en tous lieux qu'ils se repentent,
\VS{31}parce qu'il a arrêté un jour où il jugera le monde selon la justice, par l'homme qu'il a établi pour cela, ce dont il a donné à tous une preuve certaine, en le ressuscitant des morts.
\VS{32}Lorsqu'ils entendirent parler de la résurrection des morts, les uns se moquèrent, et les autres dirent~: Nous t'entendrons là-dessus une autre fois.
\VS{33}Ainsi Paul se retira du milieu d'eux.
\VS{34}Quelques-uns néanmoins se joignirent à lui et crurent~: Denys, juge de l'Aéropage, une femme nommée Damaris, et d'autres avec eux.
\Chap{18}
\TextTitle{Paul enseigne à Corinthe pendant un an et demi}
\VerseOne{}Après cela, Paul partit d'Athènes, et se rendit à Corinthe.
\VS{2}Il y trouva un Juif, nommé Aquilas, originaire du Pont, récemment arrivé d'Italie, avec Priscille, sa femme, parce que Claude avait ordonné à tous les Juifs de sortir de Rome. Il s'approcha d'eux,
\VS{3}et comme il était du même métier qu'eux, il demeura chez eux et y travailla. Et leur métier était de faire des tentes.
\VS{4}Paul discourait dans la synagogue chaque sabbat, et il persuadait des Juifs et des Grecs.
\VS{5}Quand Silas et Timothée furent arrivés de Macédoine, Paul étant poussé par l'Esprit, rendait témoignage aux Juifs que Jésus était le Christ.
\VS{6}Mais comme ils s'opposaient à lui et qu'ils blasphémaient, il secoua ses vêtements, et leur dit~: Que votre sang retombe sur votre tête~! J'en suis pur~! Dès maintenant, j'irai vers les Gentils.
\VS{7}Et sortant de là, il entra dans la maison d'un homme appelé Justus, homme craignant Dieu, et dont la maison était contiguë à la synagogue.
\VS{8}Cependant Crispus, le chef de la synagogue, crut au Seigneur avec toute sa famille. Et plusieurs Corinthiens qui avaient entendu Paul, crurent aussi, et ils furent baptisés.
\VS{9}Le Seigneur dit à Paul dans une vision pendant la nuit~: Ne crains point, mais parle et ne te tais point,
\VS{10}parce que je suis avec toi, et personne ne mettra la main sur toi pour te faire du mal. Parle, car j'ai un peuple nombreux dans cette ville.
\VS{11}Il y demeura un an et six mois, enseignant parmi eux la parole de Dieu.
\TextTitle{Soulèvement des Juifs contre Paul}
\VS{12}Pendant que Gallion était proconsul de l'Achaïe, les Juifs se soulevèrent d'un commun accord contre Paul, et le menèrent devant le tribunal,
\VS{13}en disant~: Cet homme incite les gens à servir Dieu d'une manière contraire à la loi.
\VS{14}Et comme Paul voulait ouvrir la bouche pour parler, Gallion dit aux Juifs~: Ô Juifs~! S'il s'agissait de quelque injustice, ou de quelque crime, je vous écouterais patiemment, autant qu'il serait raisonnable.
\VS{15}Mais il s'agit de discussions sur une parole, sur des noms, et sur votre loi, vous y mettrez de l'ordre vous-mêmes, car je ne veux pas être juge de ces choses.
\VS{16}Et il les renvoya du tribunal.
\VS{17}Alors tous les Grecs se saisirent de Sosthène, le chef de la synagogue, le battirent devant le tribunal, sans que Gallion s'en mît en peine.
\TextTitle{Paul fait un vœu\FTNTT{\vref{Ga. 3:23-28}~; \vref{2 Co. 3:7-14}~; \vref{Ro. 6:14}}}
\VS{18}Paul resta encore assez longtemps à Corinthe. Ensuite il prit congé des frères et s'embarqua pour la Syrie, avec Priscille et Aquilas, après s'être fait raser la tête à Cenchrées, car il avait fait un vœu.
\VS{19}Ils arrivèrent à Ephèse, et Paul y laissa ses compagnons. Etant entré dans la synagogue, il s'entretint avec les Juifs,
\VS{20}qui le prièrent de rester encore plus longtemps avec eux.
\VS{21}Mais il n'y consentit point, et il prit congé d'eux en leur disant~: Il faut absolument que je célèbre la fête prochaine à Jérusalem. Je reviendrai vers vous, s'il plaît à Dieu. Ainsi il partit d'Ephèse.
\VS{22}Etant débarqué à Césarée, il monta à Jérusalem, et après avoir salué l'église, il descendit à Antioche.
\VS{23}Et ayant séjourné là quelque temps, il s'en alla, et traversa tout de suite la contrée de Galatie et de Phrygie, fortifiant tous les disciples\FTNT{Voir annexe «~Les voyages missionnaires de Paul~»}.
\TextTitle{Apollos annonce l'Evangile à Ephèse et à Corinthe}
\VS{24}En ce temps-là, un Juif, nommé Apollos, originaire d'Alexandrie, homme éloquent et puissant dans les Ecritures, vint à Ephèse.
\VS{25}Il était en quelque sorte instruit dans la voie du Seigneur, et fervent d'esprit~; il expliquait et enseignait avec exactitude ce qui concerne Jésus, bien qu'il ne connaisse que le baptême de Jean.
\VS{26}Il commança donc à parler avec hardiesse dans la synagogue~; et quand Aquilas et Priscille l'eurent entendu, ils le prirent avec eux, et lui exposèrent plus exactement la voie de Dieu.
\VS{27}Et comme il voulut passer en Achaïe, les frères, qui l'y encouragèrent écrivirent aux disciples de bien le recevoir. Quand il fut arrivé, il aida beaucoup ceux qui avaient cru par la grâce.
\VS{28}Car il réfutait publiquement les Juifs avec une grande véhémence, démontrant par les Ecritures que Jésus était le Christ.
\Chap{19}
\TextTitle{Paul enseigne à Ephèse\FTNTT{v. \vref{9-10}~; \vref{Ac. 20:31}}}
\VerseOne{}Pendant qu'Apollos était à Corinthe, Paul, après avoir parcouru toutes les hautes provinces de l'Asie, arriva à Ephèse. Ayant rencontré quelques disciples, il leur dit~:
\VS{2}Avez-vous reçu le Saint-Esprit quand vous avez cru~? Ils lui répondirent~: Nous n'avons même pas entendu dire qu'il y ait un Saint-Esprit.
\VS{3}Et il leur dit~: De quel baptême donc avez-vous été baptisés~? Ils répondirent~: Du baptême de Jean.
\VS{4}Alors Paul dit~: Il est vrai que Jean a baptisé du baptême de repentance, disant au peuple de croire en celui qui venait après lui, c'est-à-dire en Jésus-Christ.
\VS{5}Après avoir entendu ces choses, ils furent baptisés au Nom du Seigneur Jésus.
\VS{6}Lorsque Paul leur eut imposé les mains, le Saint-Esprit descendit sur eux, et ils parlaient diverses langues et prophétisaient.
\VS{7}Ils étaient en tout environ douze hommes.
\VS{8}Ensuite, Paul entra dans la synagogue où il parla librement. Pendant trois mois, il discourut sur les choses qui concernent le Royaume de Dieu avec persuasion.
\VS{9}Mais comme quelques-uns restaient endurcis et rebelles, décriant devant la multitude la voie du Seigneur, il se retira d'eux, sépara les disciples, et enseigna tous les jours dans l'école d'un nommé Tyrannus.
\VS{10}Cela dura deux ans, de sorte que tous ceux qui habitaient l'Asie, Juifs et Grecs, entendirent la parole du Seigneur Jésus.
\TextTitle{Réveil et prodiges à Ephèse}
\VS{11}Et Dieu faisait des prodiges extraordinaires par les mains de Paul,
\VS{12}au point qu'on appliquait sur les malades des mouchoirs ou des linges qui avaient touché son corps, et ils étaient guéris de leurs maladies, et les esprits malins sortaient.
\VS{13}Alors quelques exorcistes Juifs ambulants essayèrent d'invoquer le Nom du Seigneur Jésus sur ceux qui étaient possédés d'esprits malins, en disant~: Nous vous conjurons par ce Jésus que Paul prêche~!
\VS{14}Ceux qui faisaient cela étaient sept fils de Scéva, un homme Juif, l'un des principaux prêtres.
\VS{15}Mais l'esprit malin leur répondit~: Je connais Jésus, et je sais qui est Paul~; mais vous, qui êtes-vous~?
\VS{16}Et l'homme dans lequel était l'esprit malin se jeta sur eux, se rendit maître de deux d'entre eux, et les maltraita de telle sorte qu'ils s'enfuirent de cette maison nus et blessés.
\VS{17}Cela fut connu de tous les Juifs et de tous les Grecs qui demeuraient à Ephèse~; et ils furent tous saisis de crainte, et le Nom du Seigneur Jésus était glorifié.
\VS{18}Plusieurs de ceux qui avaient cru venaient, confessant et déclarant ce qu'ils avaient fait.
\VS{19}Et un grand nombre de ceux qui s'étaient adonnés à des pratiques magiques, apportèrent leurs livres et les brûlèrent devant tous. On en estima la valeur à cinquante mille pièces d'argent.
\VS{20}Ainsi la parole du Seigneur se répandait sensiblement, et produisait de grands effets.
\VS{21}Après que ces choses se furent passées, Paul se proposa par un mouvement de l'Esprit\FTNT{Paul fut conduit par le Saint-Esprit (\vref{Jn. 3:8}).} d'aller à Jérusalem, en traversant la Macédoine et l'Achaïe. Quand j'y serai allé, se disait-il, il faut aussi que je voie Rome.
\VS{22}Il envoya en Macédoine deux de ceux qui l'assistaient, Timothée et Eraste, et il resta lui-même quelque temps en Asie.
\TextTitle{Emeute suscitée par Démétrius}
\VS{23}Mais en ce temps-là il arriva un grand trouble, à cause de la doctrine.
\VS{24}Car un certain homme, nommé Démétrius, orfèvre, fabriquait de petits temples d'argent de Diane, et apportait beaucoup de profit aux ouvriers du métier.
\VS{25}Il les rassembla, avec ceux du même métier, et dit~: Ô hommes, vous savez que tout notre gain vient de cet ouvrage,
\VS{26}et vous voyez et entendez que, non seulement à Ephèse, mais dans presque toute l'Asie, ce Paul par ses persuasions a détourné beaucoup de monde, en disant que les dieux faits de main d'homme ne sont pas des dieux.
\VS{27}Et il n'y a pas seulement à craindre pour nous que notre métier ne soit décrié, mais même que le temple de la grande Diane ne tombe dans le mépris, et que sa majesté, que toute l'Asie et que le monde entier révère, ne soit anéantie.
\VS{28}Ayant entendu ces choses, ils furent tous remplis de colère, et s'écrièrent, disant~: Grande est la Diane des Ephésiens~!
\VS{29}Et toute la ville fut remplie de confusion~; et ils se jetèrent en foule dans le théâtre, et enlevèrent Gaïus et Aristarque Macédoniens, compagnons de voyage de Paul.
\VS{30}Et comme Paul voulait entrer vers le peuple, les disciples ne le lui permirent point.
\VS{31}Quelques-uns même des Asiarques, qui étaient ses amis, envoyèrent quelqu'un vers lui pour le prier de ne pas se présenter au théâtre.
\VS{32}Les uns criaient d'une manière, les autres d'une autre, car l'assemblée était confuse, et la plupart ne savaient pas pourquoi ils s'étaient assemblés.
\VS{33}Alors Alexandre fut contraint de sortir hors de la foule, les Juifs le poussant en avant~; et Alexandre, faisant signe de la main, voulait présenter quelque excuse au peuple.
\VS{34}Mais quand ils reconnurent qu'il était Juif, tous d'une seule voix crièrent pendant deux heures~: Grande est la Diane des Ephésiens~!
\VS{35}Cependant, le secrétaire de la ville, ayant apaisé la foule, dit~: Hommes éphésiens, quel est celui des hommes qui ignore que la ville d'Ephèse est la gardienne de la grande déesse Diane et de son image tombée de Jupiter\FTNT{Tombée de Jupiter~: C'est-à-dire du ciel.}~?
\VS{36}Cela étant donc incontestable, vous devez vous apaiser et ne rien faire avec précipitation.
\VS{37}Car ces gens que vous avez amenés ne sont ni sacrilèges ni blasphémateurs de votre déesse.
\VS{38}Mais si Démétrius et ses ouvriers ont à se plaindre de quelqu'un, il y a des jours d'audience et des proconsuls~; qu'ils s'appellent en justice les uns les autres~!
\VS{39}Et si vous avez quelque autre chose à réclamer, on pourra en décider dans une assemblée légale.
\VS{40}Car nous risquons d'être accusés de sédition pour ce qui s'est passé aujourd'hui, n'ayant aucune raison pour justifier ce rassemblement. Après ces paroles, il congédia l'assemblée.
\Chap{20}
\TextTitle{Paul annonce l'Evangile en Macédoine et en Grèce}
\VerseOne{}Lorsque le tumulte eut cessé, Paul fit venir les disciples, et après les avoir embrassés, il partit pour aller en Macédoine.
\VS{2}Il parcourut cette contrée, en adressant aux disciples de nombreuses exhortations.
\VS{3}Puis il se rendit en Grèce où il séjourna trois mois. Il était sur le point de s'embarquer pour la Syrie, quand les Juifs lui dressèrent des embûches. Alors il se décida à reprendre la route de la Macédoine.
\VS{4}Il avait pour l'accompagner jusqu'en Asie~: Sopater de Bérée, Aristarque et Second de Thessalonique, Gaïus de Derbe, Timothée, ainsi que Tychique et Trophime, originaires d'Asie.
\VS{5}Ceux-ci prirent les devants et nous attendirent à Troas.
\TextTitle{Paul ressuscite un jeune homme à Troas}
\VS{6}Et nous, ayant levé l'ancre à Philippes, après les jours des pains sans levain, nous arrivâmes au bout de cinq jours auprès d'eux à Troas, et nous y séjournâmes sept jours.
\VS{7}Le premier jour de la semaine, les disciples étant assemblés pour rompre le pain, Paul, qui devait partir le lendemain, leur fit un discours qu'il étendit jusqu'à minuit.
\VS{8}Or il y avait beaucoup de lampes dans la chambre haute où ils étaient assemblés.
\VS{9}Et un jeune homme nommé Eutychus, qui était assis sur une fenêtre, s'endormit profondément pendant le long discours de Paul~; entraîné par le sommeil, il tomba du troisième étage en bas, et quand on voulut le relever, il était mort.
\VS{10}Mais Paul, étant descendu, se pencha sur lui, le prit dans ses bras, et dit~: Ne vous troublez pas, car son âme est en lui.
\VS{11}Quand il fut remonté, il rompit le pain et mangea, et il parla longtemps encore jusqu'au jour. Après quoi il partit.
\VS{12}Ils ramenèrent le jeune homme vivant, et ce fut le sujet d'une grande consolation.
\TextTitle{Passage à Milet}
\VS{13}Pour nous, étant montés sur un navire, nous fîmes voile vers Assos, où nous avions convenu de reprendre Paul, parce qu'il devait faire la route à pied.
\VS{14}Lorsqu'il nous eut rejoints à Assos, nous le prîmes avec nous, et nous allâmes à Mytilène.
\VS{15}Puis étant partis de là, le jour suivant nous abordâmes vis-à-vis de Chios. Le lendemain, nous arrivâmes vers Samos, et nous nous arrêtâmes à Trogyle~; le jour d'après, nous vînmes à Milet.
\VS{16}Car Paul avait résolu de passer devant Ephèse sans s'y arrêter, afin de ne pas perdre de temps en Asie~; parce qu'il se hâtait pour être, si cela lui était possible, à Jérusalem le jour de la Pentecôte.
\TextTitle{Paul exhorte et prend congé des anciens d'Ephèse}
\VS{17}Cependant de Milet, il envoya chercher à Ephèse les anciens de l'Eglise.
\VS{18}Lorsqu'ils furent arrivés vers lui, il leur dit~: Vous savez de quelle manière je me suis toujours conduit avec vous dès le premier jour où je suis entré en Asie~;
\VS{19}servant le Seigneur en toute humilité, avec beaucoup de larmes, et au milieu des épreuves que me suscitaient les embûches des Juifs.
\VS{20}Vous savez que je n'ai rien caché de ce qui vous était utile, et que je n'ai pas craint de vous prêcher et de vous enseigner publiquement et dans les maisons,
\VS{21}prêchant tant aux Juifs qu'aux Grecs la repentance envers Dieu, et la foi en Jésus-Christ, notre Seigneur.
\VS{22}Et maintenant voici, étant lié par l'Esprit, je vais à Jérusalem, ignorant ce qui m'y arrivera~;
\VS{23}seulement, de ville en ville le Saint-Esprit m'avertit que des liens et des tribulations m'attendent.
\VS{24}Mais je ne fais pour moi-même aucun cas de ma vie, comme si elle m'était précieuse, pourvu que j'achève ma course avec joie, et le service que j'ai reçu du Seigneur Jésus, pour rendre témoignage à l'Evangile de la grâce de Dieu.
\VS{25}Et maintenant voici, je sais que vous ne verrez plus mon visage, vous tous au milieu desquels j'ai passé en prêchant le Royaume de Dieu.
\VS{26}C'est pourquoi je vous prends aujourd'hui à témoin que je suis net du sang de tous.
\VS{27}Car je vous ai annoncé tout le conseil de Dieu, sans en rien cacher.
\VS{28}Prenez donc garde à vous-mêmes, et à tout le troupeau sur lequel le Saint-Esprit vous a établis évêques\FTNT{Evêque, «~episcopos~» en grec~: surveillant, gardien. Ce terme désigne la fonction des anciens. Dans la Nouvelle Alliance, les évêques (ou anciens) sont des personnes dont la mission est de veiller au bon fonctionnement des assemblées locales. Jésus-Christ, notre Dieu, est l'Evêque par excellence (\vref{1 Pi. 2:25}).}, pour paître l'Eglise de Dieu, qu'il a acquise par son propre sang.
\VS{29}Car je sais qu'après mon départ, il s'introduira parmi vous des loups très dangereux, qui n'épargneront pas le troupeau,
\VS{30}et qu'il se lèvera du milieu de vous des hommes qui enseigneront des doctrines corrompues dans le but d'attirer les disciples après eux.
\VS{31}C'est pourquoi veillez, vous souvenant que durant l'espace de trois ans, je n'ai cessé nuit et jour d'avertir chacun de vous avec larmes.
\VS{32}Et maintenant, mes frères, je vous recommande à Dieu, et à la parole de sa grâce, à celui qui est puissant pour achever de vous édifier, et pour vous donner l'héritage avec tous les saints.
\VS{33}Je n'ai désiré ni l'argent, ni l'or, ni les vêtements de personne.
\VS{34}Et vous savez vous-mêmes que ces mains ont pourvu à mes besoins et à ceux des personnes qui étaient avec moi.
\VS{35}Je vous ai montré de toutes manières que c'est en travaillant ainsi qu'il faut soutenir les faibles, et se rappeler les paroles du Seigneur Jésus, qui a dit lui-même~: Il y a plus de bénédiction à donner qu'à recevoir\FTNT{\vref{Lu. 14:12}.}.
\VS{36}Après avoir ainsi parlé, il se mit à genoux et il pria avec eux tous.
\VS{37}Alors tous fondirent en larmes, et se jetant au cou de Paul,
\VS{38}ils l'embrassèrent, étant principalement affligés de ce qu'il avait dit qu'ils ne verraient plus son visage. Et ils l'accompagnèrent jusqu'au navire.
\Chap{21}
\TextTitle{L'équipe missionnaire à Tyr~; avertissement de l'Esprit}
\VerseOne{}Nous nous embarquâmes, après nous être séparés d'eux, et nous allâmes directement à Cos, et le jour suivant à Rhodes, et de là à Patara.
\VS{2}Et ayant trouvé un navire qui faisait la traversée vers la Phénicie, nous montâmes et partîmes.
\VS{3}Puis ayant découvert l'île de Chypre, nous la laissâmes à gauche, nous fîmes route vers la Syrie, nous arrivâmes à Tyr, car le navire devait y décharger sa cargaison.
\VS{4}Nous trouvâmes les disciples et nous restâmes là sept jours. Les disciples, poussés par l'Esprit, disaient à Paul de ne pas monter à Jérusalem.
\VS{5}Mais ces jours étant passés, nous partîmes et nous nous acheminâmes pour partir de Tyr, et tous nous accompagnèrent avec leurs femmes et leurs enfants, jusqu'à l'extérieur de la ville. Nous nous mîmes à genoux sur le rivage et nous fîmes la prière.
\VS{6}Et après nous être embrassés les uns les autres, nous montâmes sur le navire, et les autres retournèrent chez eux.
\TextTitle{Escales à Ptolémaïs puis à Césarée~; prophétie d'Agabus}
\VS{7}Et ainsi achevant notre navigation, nous allâmes de Tyr à Ptolémaïs~; et après avoir salué les frères, nous passâmes un jour avec eux.
\VS{8}Nous partîmes le lendemain, et nous arrivâmes à Césarée. Etant entrés dans la maison de Philippe, l'évangéliste, qui était l'un des sept, nous restâmes chez lui.
\VS{9}Il avait quatre filles vierges qui prophétisaient.
\VS{10}Comme nous étions là depuis plusieurs jours, un prophète, nommé Agabus, arriva de Judée
\VS{11}et vint nous trouver. Il prit la ceinture de Paul, se lia les mains et les pieds, et il dit~: Voici ce que déclare le Saint-Esprit~: L'homme à qui appartient cette ceinture, les Juifs le lieront de la même manière à Jérusalem, et le livreront entre les mains des Gentils.
\VS{12}Quand nous entendîmes ces choses, nous et ceux de l'endroit, nous priâmes Paul de ne pas monter à Jérusalem.
\VS{13}Mais Paul répondit~: Que faites-vous en pleurant et en affligeant mon cœur~? Je suis prêt, non seulement à être lié, mais aussi à mourir à Jérusalem pour le Nom du Seigneur Jésus.
\VS{14}Comme il ne se laissait pas persuader, nous n'insistâmes pas, et nous dîmes~: Que la volonté du Seigneur soit faite~!
\TextTitle{QUATRIEME VOYAGE~: DE JERUSALEM A ROME}
\TextTitle{Arrivée à Jérusalem~; accueil des anciens}
\VS{15}Quelques jours après, nous fîmes nos préparatifs et nous montâmes à Jérusalem \FTNT{Voir annexe «~Les voyages missionnaires de Paul~»}.
\VS{16}Quelques disciples de Césarée vinrent avec nous, amenant avec eux un homme appelé Mnason, de l'île de Chypre, disciple de longue date, chez qui nous devions loger.
\VS{17}Lorsque nous arrivâmes à Jérusalem, les frères nous reçurent avec joie.
\VS{18}Et le jour suivant, Paul se rendit avec nous chez Jacques, et tous les anciens s'y réunirent.
\VS{19}Après les avoir embrassés, il raconta en détail les choses que Dieu avait faites au milieu des Gentils par son service.
\VS{20}Quand ils l'eurent entendu, ils glorifièrent le Seigneur. Puis ils dirent à Paul~: Tu vois frère, combien de milliers de Juifs ont cru~; mais ils sont tous zélés pour la loi.
\VS{21}Or ils ont appris que tu enseignes à tous les Juifs qui sont parmi les Gentils, à renoncer à Moïse, en leur disant qu'ils ne doivent pas circoncire leurs enfants et de ne pas vivre selon les ordonnances de la loi.
\VS{22}Que faut-il donc faire~? Il faut absolument rassembler la multitude des fidèles, car ils apprendront que tu es venu.
\VS{23}C'est pourquoi fais ce que nous allons te dire~: Nous avons quatre hommes qui ont fait un vœu,
\VS{24}prends-les avec toi, purifie-toi avec eux, et pourvois à leurs besoins afin qu'ils se rasent la tête. Et ainsi tous sauront que ce qu'ils ont entendu sur ton compte est faux, mais que toi aussi tu te conduis en observateur de la loi.
\VS{25}A l'égard des Gentils qui ont cru, nous avons décidé et nous leur avons écrit qu'ils doivent s'abstenir des viandes sacrifiées aux idoles, du sang, des animaux étouffés, et de la débauche.
\VS{26}Alors Paul prit ces hommes, se purifia, et entra le lendemain dans le temple avec eux, pour annoncer quel jour leur purification devait s'achever, et quand l'offrande devait être présentée pour chacun d'eux.
\TextTitle{Paul chassé du temple et brutalisé par les Juifs}
\VS{27}A la fin des sept jours, les Juifs d'Asie ayant vu Paul dans le temple, soulevèrent tout le peuple, et mirent la main sur lui,
\VS{28}en criant~: Hommes israélites, au secours~! Voici l'homme qui prêche partout et à tout le monde contre le peuple, contre la loi, et contre ce lieu. Il a même introduit des Grecs dans le temple, et a profané ce saint lieu.
\VS{29}Car ils avaient vu auparavant Trophime d'Ephèse avec lui dans la ville, et ils croyaient que Paul l'avait fait entrer dans le temple.
\VS{30}Toute la ville fut émue, et le peuple accourut de toutes parts. Ils se saisirent de Paul et le traînèrent hors du temple, dont les portes furent aussitôt fermées.
\TextTitle{Intervention des soldats et des centeniers}
\VS{31}Comme ils cherchaient à le tuer, le bruit vint au tribun de la cohorte que tout Jérusalem était en trouble.
\VS{32}A l'instant, il prit des soldats et des centeniers, et courut vers eux. Voyant le tribun et les soldats, ils cessèrent de frapper Paul.
\VS{33}Alors le tribun s'approcha, se saisit de Paul, et le fit lier de deux chaînes. Puis il demanda qui il était et ce qu'il avait fait.
\VS{34}Les uns criaient d'une manière, et les autres d'une autre, dans la foule. Ne pouvant donc rien apprendre de certain à cause du tumulte, il ordonna de mener Paul dans la forteresse.
\VS{35}Lorsque Paul fut sur les degrés, il dut être porté par les soldats, à cause de la violence de la foule~;
\VS{36}car la multitude du peuple le suivait, en criant~: Fais-le mourir~!
\VS{37}Comme on allait faire entrer Paul dans la forteresse, il dit au tribun~: M'est-il permis de te dire quelque chose~? Et le tribun répondit~: Tu sais parler le grec~?
\VS{38}Tu n'es donc pas cet Egyptien qui a excité une sédition dernièrement, et qui a emmené dans le désert quatre mille brigands~?
\VS{39}Paul lui dit~: Je suis Juif de Tarse, citoyen de la ville renommée de la Cilicie. Permets-moi, je te prie, de parler au peuple.
\VS{40}Et quand il le lui permit, Paul se tenant sur les degrés fit signe de la main au peuple, et s'étant fait un grand silence, il leur parla en langue hébraïque, disant~:
\Chap{22}
\TextTitle{Paul raconte son témoignage de conversion\FTNTT{\vref{Ac. 9:1-18}~; \vref{26:9-18}.}}
\VerseOne{}Hommes frères et pères, écoutez ce que j'ai maintenant à vous dire pour ma défense~!
\VS{2}Lorsqu'ils entendirent qu'il leur parlait en langue hébraïque, ils redoublèrent de silence. Et Paul leur dit~:
\VS{3}Je suis Juif, né à Tarse en Cilicie~; mais j'ai été élevé dans cette ville-ci aux pieds de Gamaliel et instruit dans la connaissance exacte de la loi de nos pères, étant plein de zèle pour la loi de Dieu, comme vous l'êtes tous aujourd'hui.
\VS{4}J'ai persécuté à mort cette doctrine, liant et mettant en prison hommes et femmes.
\VS{5}Le grand-prêtre lui-même et toute l'assemblée des anciens m'en sont témoins. J'ai même reçu d'eux des lettres pour les frères de Damas, où je me rendis afin d'amener liés à Jérusalem ceux qui se trouvaient là et de les faire punir.
\VS{6}Or il arriva comme j'étais en chemin, que j'approchais de Damas, tout à coup, vers midi, une grande lumière venant du ciel resplendit comme un éclair autour de moi.
\VS{7}Je tombai par terre, et j'entendis une voix qui me dit~: Saul, Saul, pourquoi me persécutes-tu~?
\VS{8}Je répondis~: Qui es-tu Seigneur~? Et il me dit~: Je suis Jésus de Nazareth, que tu persécutes.
\VS{9}Ceux qui étaient avec moi furent tout effrayés, ils virent bien la lumière, mais ils ne comprirent pas la voix de celui qui me parlait. Alors je dis~: Que ferai-je Seigneur~?
\VS{10}Et le Seigneur me dit~: Lève-toi, va à Damas, et là on te dira tout ce que tu dois faire.
\VS{11}Comme je ne voyais rien, à cause de l'éclat de cette lumière, ceux qui étaient avec moi me prirent par la main, et j'arrivai à Damas.
\VS{12}Or un nommé Ananias, homme pieux selon la loi, et de qui tous les Juifs demeurant à Damas rendaient un bon témoignage, vint me trouver
\VS{13}et me dit~: Saul, mon frère, recouvre la vue. Au même instant, je recouvrai la vue et je le regardai.
\VS{14}Et il me dit~: Le Dieu de nos pères t'a destiné à connaître sa volonté, à voir le Juste, et à entendre les paroles de sa bouche.
\VS{15}Car tu lui serviras de témoin auprès de tous les hommes, des choses que tu as vues et entendues.
\VS{16}Et maintenant, pourquoi tardes-tu~? Lève-toi, et sois baptisé et purifié de tes péchés, en invoquant le Nom du Seigneur.
\TextTitle{Le Seigneur appelle Paul à quitter Jérusalem et l'envoie dans les nations}
\VS{17}Or il arriva qu'après que je sois retourné à Jérusalem, comme je priais dans le temple, je fus ravi en extase,
\VS{18}et je vis le Seigneur qui me disait~: Hâte-toi, et sors promptement de Jérusalem, parce qu'ils ne recevront pas le témoignage que tu leur rendras de moi.
\VS{19}Et je dis~: Seigneur, ils savent eux-mêmes que je faisais mettre en prison et battre de verges dans les synagogues ceux qui croyaient en toi~;
\VS{20}et que lorsque le sang d'Etienne, ton martyr, fut répandu, j'étais moi-même présent, je consentais à sa mort, et je gardais les vêtements de ceux qui le faisaient mourir.
\VS{21}Alors il me dit~: Va, car je t'enverrai au loin vers les Gentils.
\TextTitle{Les Juifs demandent la mort de Paul}
\VS{22}Et ils l'écoutèrent jusqu'à cette parole~; mais alors ils élevèrent leur voix, en disant~: Ote de la terre un tel homme~! Car il n'est pas concevable qu'il vive.
\VS{23}Et comme ils criaient à haute voix, secouaient leurs vêtements, et jetaient de la poussière en l'air,
\VS{24}le tribun commanda de faire entrer Paul dans la forteresse, et de lui donner la question par le fouet, afin de savoir pour quel sujet ils criaient ainsi contre lui.
\TextTitle{Paul revendique ses droits de citoyen romain}
\VS{25}Comme on l'attachait pour le frapper, Paul dit au centenier qui était près de lui~: Vous est-il permis de fouetter un homme romain, et qui n'est même pas condamné~?
\VS{26}A ces mots, le centenier alla vers le tribun pour l'avertir, disant~: Prends garde à ce que tu feras, car cet homme est Romain.
\VS{27}Et le tribun, étant venu, dit à Paul~: Dis-moi, es-tu Romain~? Et il répondit~: Oui, je le suis.
\VS{28}Le tribun lui dit~: J'ai acquis ce droit de citoyen pour une grande somme d'argent. Et moi, dit Paul, je l'ai par ma naissance.
\VS{29}Aussitôt, ceux qui devaient lui donner la question se retirèrent, et le tribun, voyant que Paul était Romain, fut dans la crainte parce qu'il l'avait fait lier.
\TextTitle{Paul devant le sanhédrin}
\VS{30}Le lendemain, voulant savoir avec certitude de quoi les Juifs l'accusaient, le tribun lui fit ôter ses liens, et donna l'ordre aux principaux prêtres et à tout le sanhédrin de se réunir~; puis, il fit descendre Paul, et il le présenta devant eux.
\Chap{23}
\VerseOne{}Paul regardant fixement le sanhédrin, dit~: Hommes frères~! Je me suis conduit en toute bonne conscience devant Dieu jusqu'à ce jour.
\VS{2}Le grand-prêtre Ananias ordonna à ceux qui étaient près de lui de le frapper sur la bouche.
\VS{3}Alors Paul lui dit~: Dieu te frappera, muraille blanchie~! Tu es assis pour me juger selon la loi, et tu violes la loi en ordonnant qu'on me frappe~!
\VS{4}Ceux qui étaient présents lui dirent~: Tu insultes le grand-prêtre de Dieu~?
\VS{5}Et Paul dit~: Je ne savais pas, mes frères, que c'était le grand-prêtre~; car il est écrit~: Tu ne parleras pas mal du chef de ton peuple.
\TextTitle{Dissensions entre pharisiens et sadducéens}
\VS{6}Paul, sachant qu'une partie de l'assemblée était composée de sadducéens et l'autre de pharisiens, s'écria dans le sanhédrin~: Hommes frères~! Je suis pharisien, fils de pharisien, c'est à cause de l'espérance et de la résurrection des morts que je suis mis en jugement.
\VS{7}Quand il eut dit cela, il s'éleva un débat entre les pharisiens et les sadducéens~; et l'assemblée se divisa.
\VS{8}Car les sadducéens disent qu'il n'y a point de résurrection, ni d'ange, ni d'esprit, mais les pharisiens soutiennent les deux choses.
\VS{9}Il y eut une grande clameur. Alors les scribes du parti des pharisiens se levèrent et contestèrent, disant~: Nous ne trouvons aucun mal en cet homme~; peut-être un esprit ou un ange lui a parlé, ne combattons point contre Dieu.
\VS{10}Et comme il se faisait une grande division, le tribun craignant que Paul ne soit mis en pièces par eux, ordonna que les soldats descendent, et qu'ils l'enlèvent du milieu d'eux, et l'amènent dans la forteresse.
\TextTitle{Le Seigneur fortifie Paul}
\VS{11}La nuit suivante, le Seigneur apparut à Paul et lui dit~: Prends courage~; car, de même que tu as rendu témoignage de moi dans Jérusalem, il faut aussi que tu rendes témoignage à Rome.
\TextTitle{Complot des Juifs pour tuer Paul}
\VS{12}Quand le jour fut venu, les Juifs formèrent un complot, et firent des imprécations contre eux-mêmes, en disant qu'ils ne mangeraient pas ni ne boiraient jusqu'à ce qu'ils aient tué Paul.
\VS{13}Ceux qui formèrent ce complot étaient plus de quarante,
\VS{14}et ils s'adressèrent aux principaux prêtres et aux anciens, et leur dirent~: Nous nous sommes engagés, avec des imprécations contre nous-mêmes, à ne rien manger jusqu'à ce que nous ayons tué Paul.
\VS{15}Vous donc, maintenant, adressez-vous, avec le sanhédrin, au tribun pour le faire descendre demain au milieu de vous, comme si vous vouliez examiner sa cause plus exactement~; et nous, avant qu'il approche, nous sommes tous prêts à le tuer.
\VS{16}Le fils de la sœur de Paul, ayant eu connaissance de ce complot, alla dans la forteresse et le rapporta à Paul.
\VS{17}Paul appela l'un des centeniers et lui dit~: Mène ce jeune homme au tribun, car il a quelque chose à lui rapporter.
\VS{18}Il le prit donc et le mena au tribun, et il lui dit~: Le prisonnier Paul m'a appelé et m'a prié de t'amener ce jeune homme qui a quelque chose à te dire.
\VS{19}Et le tribun le prenant par la main, se retira à part, et lui demanda~: Qu'est-ce que tu as à me rapporter~?
\VS{20}Et il lui dit~: Les Juifs ont conspiré de te prier que demain tu envoies Paul au sanhédrin, comme s'ils voulaient s'enquérir de lui plus exactement de quelque chose.
\VS{21}Mais n'y consens point, car plus de quarante hommes d'entre eux sont en embûches contre lui, qui ont fait un vœu avec exécration de serment, de ne manger ni boire jusqu'à ce qu'ils l'aient tué~; et ils sont maintenant tous prêts, attendant ce que tu leur permettras.
\VS{22}Le tribun donc renvoya le jeune homme, en lui recommandant de ne parler à personne de ce rapport qu'il lui avait fait.
\TextTitle{Paul transféré à Césarée}
\VS{23}Ensuite, il appela deux des centeniers, et il leur dit~: Tenez prêts, dès la troisième heure de la nuit, deux cents soldats, soixante-dix cavaliers, et deux cents archers, pour aller jusqu'à Césarée.
\VS{24}Et ayez soin qu'il y ait des montures prêtes, afin qu'ayant fait monter Paul, ils le mènent sûrement au gouverneur Félix. \FTNT{Marcus Antonuis Félix était procurateur de la province romaine de la Judée de 52 à 60 ap. J.-C.}.
\VS{25}Et il lui écrivit une lettre en ces termes~:
\VS{26}Claude Lysias au très excellent gouverneur Félix, salut~!
\VS{27}Les Juifs s'étaient saisis de cet homme et allaient le tuer, lorsque je survins avec des soldats et le leur enlevai, ayant appris qu'il était Romain.
\VS{28}Voulant connaître le motif pour lequel ils l'accusaient, je l'amenai devant leur sanhédrin.
\VS{29}J'ai trouvé qu'il était accusé au sujet de questions relatives à leur loi, mais qu'il n'avait commis aucun crime qui mérite la mort ou la prison.
\VS{30}Ayant été averti des embûches que les Juifs avaient dressées contre lui, je te l'ai aussitôt envoyé, en ordonnant à ses accusateurs de te dire eux-mêmes ce qu'ils ont contre lui. Adieu~!
\TextTitle{Paul arrive à Césarée}
\VS{31}Les soldats prirent Paul, selon l'ordre qu'ils avaient reçu, et le conduisirent pendant la nuit jusqu'à Antipatris.
\VS{32}Le lendemain, laissant les cavaliers poursuivre la route avec Paul, ils retournèrent à la forteresse.
\VS{33}Arrivés à Césarée, les cavaliers remirent la lettre au gouverneur, et lui présentèrent aussi Paul.
\VS{34}Le gouverneur, après avoir lu la lettre, demanda à Paul de quelle province il était. Ayant appris qu'il était de Cilicie~:
\VS{35}Je t'entendrai, lui dit-il, plus amplement quand tes accusateurs seront venus. Et il ordonna qu'il soit gardé dans le Prétoire d'Hérode.
\Chap{24}
\TextTitle{Paul devant le gouverneur Félix~; accusation des Juifs}
\VerseOne{}Or cinq jours après, Ananias le grand-prêtre descendit avec les anciens, et un certain orateur, nommé Tertulle, qui comparurent devant le gouverneur contre Paul. 
\VS{2}Et Paul étant appelé, Tertulle commença à l'accuser, en disant~:
\VS{3}Très excellent Félix, nous reconnaissons en toutes choses partout et avec une entière reconnaissance, que nous avons obtenu une grande tranquillité par ton moyen, et par les bons règlements que tu as faits pour ce peuple, selon ta prudence.
\VS{4}Mais afin de ne pas te retenir plus longtemps, je te prie de nous entendre, selon ton équité, dans ce que nous allons te dire en peu de paroles.
\VS{5}Nous avons trouvé cet homme, qui est une peste, qui sème des divisions parmi tous les Juifs du monde entier, et qui est le chef de la secte des Nazaréens.
\VS{6}Il a même tenté de profaner le temple~; et nous l'avons saisi, et avons voulu le juger selon notre loi.
\VS{7}Mais le tribun Lysias étant survenu, il nous l'a arraché de nos mains avec une grande violence,
\VS{8}en ordonnant à ses accusateurs de venir vers toi. Tu pourras toi-même, en l'interrogeant, apprendre de lui tout ce dont nous l'accusons.
\VS{9}Les Juifs consentirent à cela, en disant que les choses étaient ainsi.
\TextTitle{Paul défend sa cause devant Félix}
\VS{10}Et après que le gouverneur eut fait signe à Paul de parler, il répondit~: Sachant qu'il y a déjà plusieurs années que tu es le juge de cette nation je réponds pour moi avec plus de courage:
\VS{11}Puisque tu peux comprendre qu'il n'y a pas plus de douze jours que je suis monté à Jérusalem pour adorer Dieu.
\VS{12}Mais ils ne m'ont point trouvé dans le temple disputant avec personne, ni faisant un amas de peuple, soit dans les synagogues, soit dans la ville. 
\VS{13}Et ils ne sauraient soutenir les choses dont ils m'accusent présentement.
\VS{14}Or je te confesse bien ce point, que selon la voie qu'ils appellent secte, je sers ainsi le Dieu de mes pères, croyant toutes les choses qui sont écrites dans la loi et dans les prophètes,
\VS{15}et ayant en Dieu cette espérance, comme ils l'ont eux-mêmes, qu'il y aura une résurrection des justes et des injustes.
\VS{16}C'est pourquoi aussi je travaille pour avoir toujours une conscience pure devant Dieu, et devant les hommes.
\VS{17}Or après plusieurs années, je suis venu pour faire des aumônes et des offrandes dans ma nation.
\VS{18}Et comme je m'occupais de ces choses, quelques Juifs d'Asie m'ont trouvé purifié dans le temple, sans attroupement ni tumulte.
\VS{19}Ils auraient dû eux-mêmes comparaître devant toi et m'accuser, s'ils avaient eu quelque chose contre moi.
\VS{20}Ou bien, que ceux-ci eux-mêmes disent, s'ils ont trouvé en moi quelque injustice, quand j'ai été présenté au sanhédrin~;
\VS{21}à moins que ce ne soit uniquement cette parole que j'ai fait entendre au milieu d'eux~; c'est à cause de la résurrection des morts que je suis aujourd'hui mis en jugement devant vous.
\VS{22}Félix, qui était parfaitement au courant de ce qui concerne cette secte, les ajourna, en disant~: Quand le tribun Lysias sera venu, j'examinerai votre affaire.
\VS{23}Et il donna l'ordre au centenier de garder Paul, en lui laissant une certaine liberté, et n'empêchant aucun des siens de le servir, ou de venir vers lui.
\TextTitle{Paul prêche Christ au gouverneur et à sa femme}
\VS{24}Quelques jours après, Félix vint avec Drusille, sa femme, qui était Juive, et il envoya chercher Paul. Il l'entendit sur la foi en Christ.
\VS{25}Et comme il parlait de la justice, de la tempérance, et du jugement à venir, Félix tout effrayé répondit~: Pour le moment retire-toi~; et quand j'aurai la commodité, je te rappellerai.
\VS{26}Il espérait en même temps que Paul lui donnerait de l'argent afin de le délivrer, c'est pourquoi il l'envoyait chercher souvent, et s'entretenait avec lui.
\TextTitle{Paul emprisonné deux ans à Césarée}
\VS{27}Deux ans s'écoulèrent ainsi, et Félix eut pour successeur Porcius Festus\FTNT{Porcius Festus était procurateur de Judée d'environ 60 à 62, succédant à Antonius Félix.}, qui voulant faire plaisir aux Juifs, laissa Paul en prison.
\Chap{25}
\TextTitle{Paul devant le gouverneur Festus}
\VerseOne{}Festus, étant arrivé dans la province, monta trois jours après de Césarée à Jérusalem.
\VS{2}Le grand-prêtre, et les principaux d'entre les Juifs portèrent plainte contre Paul devant lui. Ils firent des instances auprès de Festus, et dans des vues hostiles,
\VS{3}lui demandèrent une faveur contre lui~: Qu'il le fasse venir à Jérusalem. Ils avaient dressé des embûches pour le tuer en chemin.
\VS{4}Mais Festus leur répondit que Paul était bien gardé à Césarée, et que lui-même devait partir sous peu.
\VS{5}Et il ajouta~: Que les principaux d'entre vous descendent avec moi, et s'il y a quelque chose de coupable contre cet homme, qu'ils l'accusent.
\VS{6}Festus ne passa que dix jours parmi eux, puis il descendit à Césarée. Le lendemain, siégeant au tribunal, il ordonna que Paul soit amené.
\VS{7}Quand il fut amené, les Juifs qui étaient descendus de Jérusalem l'entourèrent et portèrent contre lui de nombreuses et graves accusations, qu'ils ne pouvaient pas prouver.
\VS{8}Tandis que Paul parlait pour sa défense~: Je n'ai rien fait de coupable, ni contre la loi des Juifs, ni contre le temple, ni contre César.
\VS{9}Mais Festus voulant faire plaisir aux Juifs, répondit à Paul, et dit~: Veux-tu monter à Jérusalem et y être jugé sur ces choses devant moi~?
\TextTitle{Paul en appelle à César}
\VS{10}Paul dit~: Je comparais devant le tribunal de César, où il faut que je sois jugé. Je n'ai fait aucun tort aux Juifs, comme tu le sais très bien.
\VS{11}Si j'ai commis quelque injustice, ou un crime digne de mort, je ne refuse pas de mourir~; mais si les choses dont ils m'accusent sont fausses, personne n'a le droit de me livrer à eux. J'en appelle à César.
\VS{12}Alors Festus ayant conféré avec le conseil, lui répondit~: En as-tu appelé à César~? Tu iras à César.
\TextTitle{Le roi Agrippa informé du cas de Paul}
\VS{13}Quelques jours après, le roi Agrippa\FTNT{Agrippa II (27-28 ap. J.-C. – 93-101 ap. J.-C.) était le fils d'Agrippa I(10 av. J.-C. – 44 ap. J.-C.), qui était lui-même le petit-fils d'Hérode le Grand (73 av. J.-C. – 4 av. J.-C.).} et Bérénice\FTNT{Bérénice (née vers 28 ap. J.-C.) était la fille d'Agrippa I et donc la sœur d'Agrippa II. Pendant tout le règne de son frère, elle fut présentée comme reine à ses cotés, raison pour laquelle on soupçonna une liaison incestueuse entre eux.} arrivèrent à Césarée pour saluer Festus.
\VS{14}Comme ils passèrent là plusieurs jours, Festus fit mention au roi de l'affaire de Paul, en disant~: Félix a laissé prisonnier un homme
\VS{15}contre lequel, lorsque j'étais à Jérusalem, les principaux prêtres et les anciens des Juifs ont porté plainte, en demandant sa condamnation.
\VS{16}Mais je leur ai répondu que ce n'est pas la coutume des Romains de livrer quelqu'un à la mort, avant que l'inculpé ait été mis en présence de ses accusateurs, et qu'il ait eu la liberté de se défendre sur le crime dont on l'accuse.
\VS{17}Ils sont donc venus ici, et sans différer, je siégeai le lendemain, et je donnai l'ordre qu'on amène cet homme.
\VS{18}Ses accusateurs s'étant présentés, ne lui imputèrent aucun des crimes dont je pensais qu'ils l'accuseraient.
\VS{19}Mais ils avaient avec lui des discussions relatives à leurs superstitions, et à un certain Jésus qui est mort, que Paul affirmait être vivant.
\VS{20}Ne sachant quel parti prendre dans ce débat, je demandai à cet homme s'il voulait aller à Jérusalem et y être jugé sur ces choses.
\VS{21}Mais Paul en ayant appelé, pour que sa cause soit réservée à la connaissance de l'empereur, j'ai ordonné qu'on le garde jusqu'à ce que je l'envoie à César.
\VS{22}Alors Agrippa dit à Festus~: Je voudrais bien aussi entendre cet homme. Demain, dit-il, tu l'entendras.
\TextTitle{Paul est amené dans la salle d'audience}
\VS{23}Le lendemain donc, Agrippa et Bérénice étant venus en grande pompe, et étant entrés dans la salle d'audience avec les tribuns et les principaux de la ville, Paul fut amené sur l'ordre de Festus.
\VS{24}Et Festus dit~: Roi Agrippa, et vous tous qui êtes ici avec nous, vous voyez cet homme au sujet duquel toute la multitude des Juifs s'est adressée à moi, soit à Jérusalem soit ici, en s'écriant qu'il ne devait plus vivre.
\VS{25}Pour moi, ayant trouvé qu'il n'avait rien fait qui mérite la mort, et lui-même en ayant appelé à Auguste, j'ai résolu de le faire partir.
\VS{26}Comme je n'ai rien de certain à écrire à l'empereur sur son compte, je vous l'ai présenté, et principalement à toi, roi Agrippa, afin qu'après en avoir fait l'examen, j'aie de quoi écrire.
\VS{27}Car il me semble qu'il n'est pas raisonnable d'envoyer un prisonnier sans marquer les faits dont on l'accuse.
\Chap{26}
\TextTitle{Discours de Paul devant Agrippa\FTNTT{\vref{Ac. 9:1-18}~; \vref{22:1-16}}}
\VerseOne{}Agrippa dit à Paul~: Il t'est permis de parler pour toi-même. Alors Paul ayant étendu la main, parla ainsi pour sa défense~:
\VS{2}Roi Agrippa~! Je m'estime béni de ce que je dois me défendre aujourd'hui devant toi, de toutes les choses dont les Juifs m'accusent~;
\VS{3}car tu connais parfaitement leurs coutumes et leurs discussions. Je te prie donc de m'écouter avec patience.
\VS{4}Ma vie, dès les premiers temps de ma jeunesse, est connue de tous les Juifs, puisqu'elle s'est passée à Jérusalem, au milieu de ma nation.
\VS{5}Car ils savent depuis longtemps, s'ils veulent en rendre témoignage, que j'ai vécu en pharisien, selon la secte la plus rigide de notre religion.
\VS{6}Et maintenant, je suis mis en jugement pour l'espérance de la promesse que Dieu a faite à nos pères,
\VS{7}et à laquelle nos douze tribus, qui servent Dieu continuellement nuit et jour, espèrent parvenir~; et c'est pour cette espérance, ô roi Agrippa, que je suis accusé par les Juifs.
\VS{8}Quoi~? Jugez-vous incroyable que Dieu ressuscite les morts~?
\VS{9}Pour moi, j'avais cru devoir agir vigoureusement contre le Nom de Jésus de Nazareth.
\VS{10}C'est ce que j'ai fait à Jérusalem. J'ai mis en prison plusieurs des saints, après en avoir reçu le pouvoir des principaux prêtres, et quand on les faisait mourir, je joignais mon suffrage à celui des autres.
\VS{11}Je les ai souvent châtiés dans toutes les synagogues, et les forçais à blasphémer. Dans mes excès de fureur contre eux, je les persécutais même jusque dans les villes étrangères.
\VS{12}Comme j'allais aussi à Damas dans ce dessein, avec l'autorisation et la permission des principaux prêtres,
\VS{13}en plein midi, ô roi, je vis en chemin resplendir autour de moi et de mes compagnons, une lumière venant du ciel et dont l'éclat surpassait celui du soleil.
\VS{14}Nous tombâmes tous par terre, et j'entendis une voix qui me parlait en langue hébraïque~: Saul, Saul, pourquoi me persécutes-tu~? Il te serait dur de regimber contre les aiguillons.
\VS{15}Je répondis~: Qui es-tu Seigneur~? Et il répondit~: Je suis Jésus que tu persécutes.
\VS{16}Mais lève-toi, et tiens-toi sur tes pieds~; car je te suis apparu pour t'établir serviteur et témoin des choses que tu as vues et de celles pour lesquelles je t'apparaîtrai.
\VS{17}Je t'ai arraché du milieu du peuple et des Gentils, vers qui je t'envoie maintenant,
\VS{18}pour ouvrir leurs yeux afin qu'ils passent des ténèbres à la lumière, et de la puissance de Satan à Dieu~; afin que par la foi qu'ils auront en moi, ils reçoivent la rémission de leurs péchés et qu'ils aient part à l'héritage des saints.
\VS{19}Ainsi, ô roi Agrippa, je n'ai pas été désobéissant à la vision céleste.
\VS{20}A ceux de Damas d'abord, puis à Jérusalem, dans toute la Judée, et chez les Gentils, j'ai prêché la repentance et la conversion à Dieu, avec la pratique d'œuvres dignes de la repentance.
\VS{21}C'est pour cela que les Juifs se sont saisis de moi dans le temple, et ont tâché de me tuer.
\VS{22}Mais ayant été secouru par l'aide de Dieu, je suis vivant jusqu'à ce jour, rendant témoignage aux petits et aux grands, sans m'écarter en rien de ce que les prophètes et Moïse ont prédit devoir arriver,
\VS{23}à savoir que le Christ souffrirait, et que ressuscité le premier d'entre les morts, il annoncerait la lumière au peuple et aux nations.
\TextTitle{Paul exhorte Agrippa}
\VS{24}Comme il parlait ainsi pour sa défense, Festus dit à haute voix~: Tu es fou Paul~! Ton grand savoir dans les lettres te fait déraisonner.
\VS{25}Et Paul dit~: Je ne suis point fou, très excellent Festus, mais je dis des paroles de vérité et de bon sens.
\VS{26}Car le roi est bien informé de ces choses~; et je lui en parle librement, parce que je suis persuadé qu'il n'en ignore aucune, puisque ce n'est pas en cachette qu'elles se sont passées.
\VS{27}Ô Roi Agrippa~! Crois-tu aux prophètes~? Je sais que tu y crois.
\VS{28}Et Agrippa répondit à Paul~: Tu vas bientôt me persuader de devenir chrétien~!
\VS{29}Et Paul lui dit~: Je souhaiterais devant Dieu que non seulement toi, mais aussi tous ceux qui m'écoutent aujourd'hui, vous deveniez tels que je suis à l'exception de ces liens~!
\VS{30}Paul ayant dit ces choses, le roi se leva, avec le gouverneur et Bérénice, et ceux qui étaient assis avec eux.
\VS{31}Et s'étant retirés à part, ils se disaient les uns les autres~: Cet homme n'a rien fait qui mérite la mort ou la prison.
\VS{32}Et Agrippa dit à Festus~: Cet homme aurait pu être relâché s'il n'avait pas appelé à César.
\Chap{27}
\TextTitle{Toujours prisonnier, Paul embarque pour Rome}
\VerseOne{}Lorsqu'il fut décidé que nous embarquerions pour l'Italie, on remit Paul avec quelques autres prisonniers à un nommé Julius, centenier d'une cohorte de la légion appelée Auguste.
\VS{2}Nous montâmes sur un navire d'Adramytte, nous partîmes prenant notre route vers les côtes de l'Asie, ayant avec nous Aristarque, un Macédonien de la ville de Thessalonique.
\VS{3}Le jour suivant, nous arrivâmes à Sidon~; et Julius, qui traitait Paul avec bienveillance, lui permit d'aller vers ses amis afin de recevoir leurs soins.
\VS{4}Puis étant partis de là, nous longeâmes l'île de Chypre, parce que les vents étaient contraires.
\VS{5}Après avoir traversé la mer de Cilicie et de Pamphylie, nous arrivâmes à Myra, ville de Lycie.
\VS{6}Et là, le centenier trouva un navire d'Alexandrie qui allait en Italie, dans lequel il nous fit monter.
\VS{7}Et comme nous naviguions lentement pendant plusieurs jours, et que nous étions arrivés avec peine vis-à-vis de Cnide, parce que le vent ne nous permettait pas d'avancer, nous naviguâmes en dessous de la Crète, vers Salmone.
\VS{8}Nous la côtoyâmes avec peine, nous arrivâmes à un lieu qui est appelé Beaux-Ports, près duquel était la ville de Lasée.
\VS{9}Il s'était écoulé beaucoup de temps, et la navigation devenait dangereuse, car le temps du jeûne était déjà passé\FTNT{Ce jeûne correspondait au jour de l'expiation célébré le dixième jour du septième mois. \vref{Lé. 23:27}.}.
\VS{10}C'est pourquoi Paul les avertit, en disant~: Ô hommes, je vois que la navigation ne se fera pas sans péril et sans dommage, non seulement pour la cargaison et pour le navire, mais aussi pour nos propres vies.
\VS{11}Mais le centenier écouta plus le pilote et le maître du navire, plutôt que les paroles de Paul.
\VS{12}Et comme le port n'était pas bon pour y passer l'hiver, la plupart furent d'avis de partir de là, pour tâcher de gagner Phénix, qui est un port de Crète, qui regarde le vent d'Afrique et le couchant septentrional, afin d'y passer l'hiver.
\VS{13}Un vent du midi commença à souffler doucement, et se croyant maîtres de leur dessein, ils levèrent l'ancre et côtoyèrent de près l'île de Crète.
\TextTitle{Une tempête de plusieurs jours}
\VS{14}Mais bientôt un vent impétueux, du nord-est, qu'on appelle Euraquilon\FTNT{Euraquilon~: Vagues et vent d'Est}, se leva du côté de l'île.
\VS{15}Le navire fut emporté par la violence de la tempête, et ne pouvant résister, nous nous laissâmes aller au gré du vent.
\VS{16}Nous passâmes au-dessous d'une petite île nommée Clauda, et nous eûmes de la peine à nous rendre maîtres de la chaloupe~;
\VS{17}après l'avoir hissée, les matelots se servirent des moyens de secours pour ceindre le navire, et dans la crainte de tomber sur la Syrte\FTNT{Syrte~: Il s'agit de la Grande Syrte et de la Petite Syrte~: deux bancs de sables mouvants très redoutés.}, ils abaissèrent les voiles. C'est ainsi qu'on se laissa emporter par le vent.
\VS{18}Comme nous étions violemment battus par la tempête, le jour suivant, ils jetèrent la cargaison à la mer~;
\VS{19}et le troisième jour, nous jetâmes de nos propres mains les agrès du navire.
\VS{20}Le soleil et les étoiles ne parurent pas pendant plusieurs jours, et la tempête nous agitait si violemment que nous perdîmes enfin toute espérance de nous sauver.
\TextTitle{Paul rassure les membres du navire}
\VS{21}On n'avait pas mangé depuis longtemps. Paul se tenant alors debout au milieu d'eux, leur dit~: Ô hommes, il fallait m'écouter et ne pas partir de Crète, afin d'éviter cette tempête et ce dommage.
\VS{22}Maintenant je vous exhorte à prendre courage~; car aucun de vous ne perdra la vie, et il n'y aura de perte que celle du navire.
\VS{23}Car un ange du Dieu à qui j'appartiens et que je sers m'est apparu cette nuit,
\VS{24}et m'a dit~: Paul, ne crains point~; il faut que tu comparaisses devant César~; et voici, Dieu t'a donné tous ceux qui naviguent avec toi.
\VS{25}C'est pourquoi, ô hommes, prenez courage, car j'ai cette confiance en Dieu que la chose arrivera comme elle m'a été dite.
\VS{26}Mais nous devons échouer sur une île.
\VS{27}La quatorzième nuit, vers minuit, tandis que nous étions ballotés sur l'Adriatique, les matelots soupçonnèrent qu'on approchait de quelque terre.
\VS{28}Ayant jeté la sonde, ils trouvèrent vingt brasses~; puis étant passés un peu plus loin, et ayant encore jeté la sonde, ils trouvèrent quinze brasses.
\VS{29}Mais craignant de heurter contre des écueils, ils jetèrent quatre ancres de la poupe, et attendirent le jour avec impatience.
\VS{30}Mais comme les matelots cherchaient à s'échapper du navire, et mettaient la chaloupe à la mer, sous prétexte de jeter les ancres de la proue,
\VS{31}Paul dit au centenier et aux soldats~: Si ces hommes ne restent pas dans le navire, vous ne pouvez pas être sauvés.
\VS{32}Alors les soldats coupèrent les cordes de la chaloupe, et la laissèrent tomber.
\VS{33}Avant que le jour paraisse, Paul les exhorta tous à prendre de la nourriture, en leur disant~: C'est aujourd'hui le quatorzième jour que vous êtes en attente et que vous persistez à vous abstenir de manger.
\VS{34}Je vous exhorte donc à prendre quelque nourriture, vu que cela est nécessaire pour votre conservation, et aucun de vos cheveux ne se perdra.
\VS{35}Ayant ainsi parlé, il prit du pain, et rendit grâces à Dieu en présence de tous~; il le rompit et se mit à manger.
\VS{36}Et tous, reprenant courage, mangèrent aussi.
\VS{37}Nous étions dans le navire deux cent soixante-seize personnes.
\VS{38}Quand ils eurent mangé jusqu'à être rassasiés, ils allégèrent le navire en jetant le blé dans la mer.
\TextTitle{Naufrage du navire}
\VS{39}Lorsque le jour fut venu, ils ne reconnurent point la terre~; mais ayant aperçu un golfe avec un rivage, ils résolurent d'y faire échouer le navire, s'ils le pouvaient.
\VS{40}Ayant donc retiré les ancres, ils abandonnèrent le navire à la mer, lâchant en même temps les attaches des gouvernails~; et ayant tendu la voile de l'artimon, ils tâchaient de se diriger vers le rivage.
\VS{41}Mais ils rencontrèrent une langue de terre, où ils firent échouer le navire~; et la proue, s'étant engagée, resta immobile, tandis que la poupe se brisait par la violence des vagues.
\VS{42}Les soldats furent d'avis de tuer les prisonniers, de peur que quelqu'un d'eux ne s'échappe à la nage.
\VS{43}Mais le centenier, voulant sauver Paul, les empêcha d'exécuter ce conseil. Il ordonna à ceux qui savaient nager de se jeter les premiers dans l'eau pour gagner la terre,
\VS{44}et aux autres de se mettre sur des planches ou sur des débris du navire. Et ainsi tous parvinrent à terre sains et saufs.
\Chap{28}
\TextTitle{Paul mordu par une vipère sur l'île de Malte}
\VerseOne{}Une fois hors de danger, ils reconnurent alors que l'île s'appelait Malte.
\VS{2}Les barbares nous traitèrent avec beaucoup d'humanité~; ils nous recueillirent tous auprès d'un grand feu, qu'ils avaient allumé parce que la pluie tombait et qu'il faisait très froid.
\VS{3}Paul ayant ramassé un tas de broussailles et l'ayant mis au feu, une vipère en sortit à cause de la chaleur et s'attacha à sa main.
\VS{4}Et quand les barbares virent cette bête suspendue à sa main, ils se dirent les uns les autres~: Certainement cet homme est un meurtrier~; puisque après être échappé de la mer, la justice ne permet pas qu'il vive. 
\VS{5}Mais Paul ayant secoué la bête dans le feu, ne ressentit aucun mal.
\VS{6}Les barbares s'attendaient à le voir enfler ou tomber subitement mort~; mais après avoir longtemps attendu, voyant qu'il ne lui arrivait aucun mal, ils changèrent de langage et dirent que c'était un dieu.
\TextTitle{Guérison du père de Publius}
\VS{7}Or en cet endroit-là étaient des terres qui appartenaient au principal de l'île, nommé Publius, qui nous reçut et nous logea pendant trois jours avec beaucoup de bonté.
\VS{8}Et il arriva que le père de Publius était au lit, malade de la fièvre et de la dysenterie~; Paul s'étant rendu vers lui, pria, lui imposa les mains, et le guérit.
\VS{9}Là-dessus, vinrent tous les autres malades de l'île, et ils furent guéris.
\VS{10}Ils nous rendirent de grands honneurs, et à notre départ, on nous fournit ce qui nous était nécessaire.
\TextTitle{Paul arrive à Rome}
\VS{11}Trois mois après, nous partîmes sur un navire d'Alexandrie qui avait passé l'hiver dans l'île, et qui avait pour enseigne Castor et Pollux.
\VS{12}Ayant abordé à Syracuse, nous y restâmes trois jours.
\VS{13}De là, en suivant la côte, nous arrivâmes à Reggio~; et un jour après, le vent du Midi s'étant levé, nous fîmes en deux jours le trajet jusqu'à Pouzzoles,
\VS{14}où nous trouvâmes des frères qui nous prièrent de passer sept jours avec eux. Et ensuite, nous arrivâmes à Rome.
\VS{15}Et les frères qui y étaient, ayant appris de nos nouvelles, vinrent à notre rencontre jusqu'au Forum d'Appius et aux Trois-Tavernes~; en les voyant, Paul rendit grâces à Dieu et prit courage.
\TextTitle{Paul annonce Christ aux Juifs de Rome}
\VS{16}Lorsque nous fûmes arrivés à Rome, le centenier mit les prisonniers entre les mains du préfet du prétoire~; mais quant à Paul, il lui permit de demeurer dans un domicile particulier avec un soldat qui le gardait.
\VS{17}Or il arriva que trois jours après que Paul convoqua les principaux des Juifs~; et quand ils furent réunis, il leur dit~: Hommes frères~! Sans avoir rien fait contre le peuple ni contre les coutumes des pères, j'ai été mis en prison à Jérusalem, et livré entre les mains des Romains,
\VS{18}qui après m'avoir examiné, voulaient me relâcher parce qu'il n'y avait en moi aucun crime qui mérite la mort.
\VS{19}Mais les Juifs s'y opposèrent, j'ai été contraint d'en appeler à César~; n'ayant du reste aucun dessein d'accuser ma nation.
\VS{20}C'est pour ce sujet que je vous ai appelés, afin de vous voir et vous parler~; car c'est pour l'espérance d'Israël que je porte cette chaîne.
\VS{21}Mais ils lui répondirent~: Nous n'avons reçu de Judée aucune lettre à ton sujet, et il n'est venu aucun frère qui ait rapporté ou dit quelque mal de toi.
\VS{22}Cependant nous entendrons volontiers de toi quel est ton sentiment~; car quant à cette secte, il nous est connu qu'on la contredit partout.
\VS{23}Et après lui avoir assigné un jour, plusieurs vinrent auprès de lui dans son logis~; et il leur expliquait par plusieurs témoignages le Royaume de Dieu, et depuis le matin jusqu'au soir, il cherchait à les persuader de ce qui concerne Jésus, tant par la loi de Moïse que par les prophètes.
\VS{24}Et les uns furent persuadés par les choses qu'il disait~; et les autres n'y crurent point.
\TextTitle{Incrédulité des Juifs~: Paul se tourne vers les Gentils\FTNTT{\vref{Ap. 13:14}~; \vref{18:6}.}}
\VS{25}C'est pourquoi n'étant pas d'accord entre eux, ils se retirèrent après que Paul leur eut dit ces paroles~: Le Saint-Esprit a bien parlé à nos pères par le prophète Esaïe en disant~:
\VS{26}Va vers ce peuple et dis-lui~: Vous entendrez de vos oreilles, et vous ne comprendrez point~; vous regarderez de vos yeux, et vous ne verrez point.
\VS{27}Car le cœur de ce peuple est devenu insensible~; ils ont endurci leurs oreilles, et ils ont fermé leurs yeux~; de peur qu'ils ne voient des yeux, qu'ils n'entendent des oreilles, qu'ils ne comprennent de leur cœur, qu'ils ne se convertissent, et que je ne les guérisse\FTNT{\vref{Es. 6:10}.}.
\VS{28}Sachez donc que ce salut de Dieu est envoyé aux Gentils, et ils l'écouteront.
\VS{29}Lorsqu'il eut dit cela, les Juifs s'en allèrent, discutant vivement entre eux.
\VS{30}Paul demeura deux ans entiers dans une maison qu'il avait louée. Il recevait tous ceux qui venaient le voir,
\VS{31}prêchant le Royaume de Dieu, et enseignant les choses qui concernent le Seigneur Jésus-Christ en toute liberté dans les paroles et sans aucun empêchement.
\PPE{}
\end{multicols}

%\clearpage\ShortTitle{Jacques}\BookTitle{Jacques}\BFont
\noindent\hrulefill
{\footnotesize
\textit{
\bigskip
{\centering{}
\\Auteur : Jacques
\\Signification : Qui supplante
\\Thème : La vie chrétienne sous son aspect pratique
\\Date de rédaction : Env. 45-50 ap. J.-C.\\}
}
%\bigskip
\textit{
\\Jacques, frère de Jésus-Christ homme, et ancien au sein de la première église chrétienne située à Jérusalem, écrit aux chrétiens d'origine juive, dispersés dans l'empire romain. Il les console suite à l'adversité qu'ils rencontraient et les exhorte à tenir ferme, leur expliquant que la foi authentique doit être accompagnée d'œuvres. Il les met en garde contre la convoitise, source de toutes les tentations, et les prévient également quant à l'amour du monde et la confiance que certains peuvent mettre dans l'argent. Pour terminer, il leur recommande d'être patients dans l'épreuve et de prier sans cesse jusqu'au retour du Seigneur.\bigskip
}
}
\par\nobreak\noindent\hrulefill
\begin{multicols}{2}
\Chap{1}
\TextTitle{Introduction}
\VerseOne{}Jacques, serviteur de Dieu, et du Seigneur Jésus-Christ, aux douze tribus qui sont dispersées, salut !
\TextTitle{La nécessité de l'épreuve de la foi}
\VS{2}Mes frères, regardez comme un sujet d'une parfaite joie quand vous êtes exposés à diverses épreuves,
\VS{3}sachant que l'épreuve de votre foi produit la patience.
\VS{4}Mais il faut que la patience accomplisse parfaitement son œuvre, afin que vous soyez parfaits et accomplis, en sorte qu'il ne vous manque rien.
\VS{5}Et si quelqu'un de vous manque de sagesse, qu'il la demande à Dieu, qui la donne à tous libéralement, et sans reproche, et elle lui sera donnée.
\VS{6}Mais qu'il la demande avec foi, ne doutant nullement ; car celui qui doute est semblable au flot de la mer, agité et poussé çà et là par le vent.
\VS{7}Qu'un tel homme ne s'attende pas à recevoir quelque chose du Seigneur.
\VS{8}L'homme double de coeur est inconstant dans toutes ses voies.
\VS{9}Que le frère de basse condition se glorifie dans son élévation.
\VS{10}Que le riche, au contraire, se glorifie dans sa basse condition ; car il passera comme la fleur de l'herbe.
\VS{11}En effet, le soleil s'est levé avec sa chaleur ardente, et l'herbe a séché, et sa fleur est tombée, et son éclat a péri, ainsi le riche se flétrira dans ses entreprises.
\TextTitle{Dieu ne tente personne ; la justice de Dieu}
\VS{12}Heureux l'homme qui endure la tentation\FTNT{Le terme grec « peirasmos » utilisé dans ce verset veut aussi dire « épreuve ».} ; car après avoir été éprouvé, il recevra la couronne de vie, que le Seigneur a promise à ceux qui l'aiment.
\VS{13}Quand quelqu'un est tenté, qu'il ne dise pas : Je suis tenté par Dieu. Car Dieu ne peut être tenté par le mal, et aussi ne tente-t-il personne.
\VS{14}Mais chacun est tenté quand il est attiré et amorcé par sa propre convoitise.
\VS{15}Puis quand la convoitise a conçu, elle enfante le péché ; et le péché, étant consommé, produit la mort.
\VS{16}Mes frères bien-aimés, ne vous y trompez pas :
\VS{17}Tout ce qui nous est donné d'excellent et tout don parfait viennent d'en haut et descendent du Père des lumières, en qui il n'y a ni changement ni ombre de variation.
\VS{18}Il nous a engendrés de sa propre volonté, par la parole de la vérité, afin que nous soyons comme les prémices de ses créatures.
\VS{19}Ainsi, mes frères bien-aimés, que tout homme soit prompt à écouter, lent à parler et lent à la colère ;
\VS{20}car la colère de l'homme n'accomplit pas la justice de Dieu.
\TextTitle{Importance de la mise en pratique de la parole}
\VS{21}C'est pourquoi, rejetant toute souillure et tout résidu\FTNT{Le mot « résidu » vient du grec « perisseia », ce mot signifie « abondance », « surabondamment », « tout excès », « reste ». Les Grecs utilisaient ce terme pour décrire l'excès de cire dans leurs oreilles. Il est question de la méchanceté qui reste dans un Chrétien et qui provient de son état antérieur à sa conversion.} de méchanceté, recevez avec douceur la parole qui a été plantée en vous et qui peut sauver vos âmes.
\VS{22}Et mettez en pratique la parole, et ne l'écoutez pas seulement, en vous trompant vous-mêmes par de vains discours.
\VS{23}Car, si quelqu'un écoute la parole et ne la met pas en pratique, il est semblable à un homme qui regarde dans un miroir son visage naturel,
\VS{24}et qui, après s'être regardé, s'en va, et oublie aussitôt comment il était.
\VS{25}Mais celui qui aura plongé les regards dans la loi parfaite, la loi de la liberté, et qui aura persévéré, n'étant point un auditeur oublieux, mais pratiquant les œuvres qui lui sont prescrites, celui-là sera heureux dans son oeuvre.
\TextTitle{La religion pure et sans tache}
\VS{26}Si quelqu'un parmi vous croit être religieux alors qu'il ne tient pas sa langue en bride, mais séduit son cœur, la religion d'un tel homme est vaine.
\VS{27}La religion pure et sans tache envers notre Dieu et notre Père, c'est de visiter les orphelins et les veuves dans leurs afflictions, et de se conserver pur des souillures de ce monde.
\Chap{2}
\TextTitle{L'amour pour son prochain en pratique}
\VerseOne{}Mes frères, n'ayez point la foi en notre Seigneur Jésus-Christ glorieux, en ayant égard à l'apparence des personnes.
\VS{2}En effet, s'il entre dans votre assemblée un homme qui porte un anneau d'or et un habit magnifique, et qu'il y entre aussi un pauvre misérablement vêtu ;
\VS{3}et que vous ayez égard à celui qui porte l'habit magnifique et lui disiez : Toi, assieds-toi ici honorablement ! Et que vous disiez au pauvre : Toi, tiens-toi là debout ! Ou, assieds-toi ici sur mon marchepied !
\VS{4}n'avez-vous pas fait de différence en vous-mêmes, et n'êtes-vous pas des juges qui avez des pensées injustes ?
\VS{5}Ecoutez, mes frères bien-aimés : Dieu n'a-t-il pas choisi les pauvres de ce monde, qui sont riches en la foi, et héritiers du Royaume qu'il a promis à ceux qui l'aiment ?
\VS{6}Mais vous avez déshonoré le pauvre ! Et cependant les riches ne vous oppriment-ils pas, et ne vous traînent-ils pas devant les tribunaux ?
\VS{7}N'est-ce pas eux qui blasphèment le beau Nom qui a été invoqué sur vous ?
\VS{8}Si, en effet, vous accomplissez la loi royale, qui est selon l'Ecriture : Tu aimeras ton prochain comme toi-même\FTNT{Lé. 19:18.}, vous faites bien.
\VS{9}Mais si vous avez égard à l'apparence des personnes, vous commettez un péché, et vous êtes convaincus par la loi comme des transgresseurs.
\VS{10}Car quiconque observe toute la loi, mais pèche contre un seul commandement, devient coupable de tous.
\VS{11}En effet, celui qui a dit : Tu ne commettras point d'adultère, a dit aussi : Tu ne tueras point. Or, si donc tu ne commets point d'adultère\FTNT{Ex. 20:13-14.}, mais que tu tues, tu deviens transgresseur de la loi.
\VS{12}Ainsi parlez et ainsi agissez comme devant être jugés par la loi de la liberté,
\VS{13}car il y aura un jugement sans miséricorde sur celui qui n'aura point usé de miséricorde\FTNT{Mt. 7:2.} ; mais la miséricorde triomphe du jugement.
\TextTitle{Les œuvres de la foi}
\VS{14}Mes frères, que servira-t-il à quelqu'un de dire qu'il a la foi, s'il n'a pas les œuvres ? Cette foi peut-elle le sauver ?
\VS{15}Et si un frère ou une sœur sont nus et manquent de ce qui leur est nécessaire chaque jour pour vivre,
\VS{16}et que l'un d'entre vous leur dise : Allez en paix, chauffez-vous, et rassasiez-vous ! Et que vous ne leur donniez pas les choses nécessaires pour le corps, que leur servira cela ?
\VS{17}De même aussi la foi, si elle n'a pas les œuvres, elle est morte en elle-même.
\VS{18}Mais quelqu'un dira : Tu as la foi ; et moi, j'ai les œuvres. Montre-moi donc ta foi sans les œuvres, et moi, je te montrerai ma foi par mes œuvres.
\VS{19}Tu crois qu'il n'y a qu'un Dieu, tu fais bien ; les démons le croient aussi, et ils tremblent.
\VS{20}Mais, ô homme vain, veux-tu savoir que la foi qui est sans les œuvres est morte ?
\TextTitle{La foi d'Abraham et de Rahab manifestée dans leurs œuvres\FTNT{Ro. 4:1-25}}
\VS{21}Abraham, notre père, ne fut-il pas justifié par les œuvres, quand il offrit son fils Isaac sur l'autel ?
\VS{22}Ne vois-tu donc pas que sa foi agissait avec ses œuvres, et que ce fut par ses œuvres que sa foi fut rendue parfaite ?
\VS{23}Ainsi s'accomplit ce que dit l'Ecriture : Abraham crut en Dieu, et cela lui fut imputé à justice\FTNT{Ge. 15:6.}; et il fut appelé ami de Dieu.
\VS{24}Vous voyez donc que l'homme est justifié par les œuvres, et non par la foi seulement.
\VS{25}Pareillement, Rahab, la prostituée, ne fut-elle pas également justifiée par les œuvres, lorsqu'elle reçut les messagers, et qu'elle les fit partir par un autre chemin\FTNT{Jos. 2:1-21.} ?
\VS{26}Car, comme le corps sans esprit est mort, de même la foi sans les œuvres est morte.
\Chap{3}
\TextTitle{Les enseignants jugés plus sévèrement}
\VerseOne{}Ne soyez pas nombreux, mes frères, à devenir des enseignants\FTNT{Du grec « didaskalos » : « maître », « professeur », « docteur chargé d'instruire, d'enseigner la parole ». Mt. 8:19 ; Mt. 22:16 ; 1 Co. 12:28.}, sachant que nous en recevrons un plus grand jugement.
\TextTitle{Enseignements sur la langue}
\VS{2}Car nous péchons tous en plusieurs choses. Si quelqu'un ne pèche pas en paroles, c'est un homme parfait, et il peut même tenir en bride tout le corps.
\VS{3}Voici, nous mettons le mors dans la bouche des chevaux, afin qu'ils nous obéissent, et nous menons çà et là tout le corps.
\VS{4}Voici, aussi les navires, quoiqu'ils soient si grands et qu'ils soient agités par la tempête, ils sont dirigés partout çà et là par un petit gouvernail, selon qu'il plaît à celui qui les gouverne.
\VS{5}Il en est ainsi de la langue, c'est un petit membre, et cependant elle peut se vanter de grandes choses. Voici, un petit feu, combien de bois allume-t-il?
\VS{6}La langue aussi est un feu ; c'est le monde de l'iniquité. La langue est placée parmi nos membres, souillant tout le corps, et enflammant tout le cours de la vie, étant elle-même enflammée par le feu de la géhenne.
\VS{7}Car toutes les espèces d'animaux sauvages, d'oiseaux, de reptiles, et d'animaux marins, se domptent et ont été domptés par la nature humaine ;
\VS{8}mais nul homme ne peut dompter la langue ; c'est un mal qu'on ne peut réprimer ; elle est pleine d'un venin mortel.
\VS{9}Par elle nous bénissons Dieu notre Père, et par elle nous maudissons les hommes faits à la ressemblance de Dieu.
\VS{10}De la même bouche sortent la bénédiction et la malédiction. Il ne faut pas qu'il en soit ainsi, mes frères.
\VS{11}Une fontaine fait-elle jaillir par la même ouverture l'eau douce et l'eau amère ?
\VS{12}Mes frères, un figuier peut-il produire des olives, ou une vigne des figues ? De même, aucune fontaine ne peut produire de l'eau salée et de l'eau douce.
\TextTitle{La sagesse humaine et la sagesse d'en haut}
\VS{13}Y a-t-il parmi vous quelque homme sage et intelligent ? Qu'il fasse voir ses oeuvres par une bonne conduite avec douceur et sagesse.
\VS{14}Mais si vous avez une envie amère et un esprit de querelle dans vos cœurs, ne vous glorifiez pas, et ne mentez pas en déshonorant la vérité de l'Evangile.
\VS{15}Car ce n'est pas là la sagesse qui descend d'en haut ; mais c'est une sagesse terrestre, animale\FTNT{Animale ou charnelle.} et diabolique.
\VS{16}Car là où il y a de l'envie et un esprit de querelle, là est le désordre, et toute sorte de mal.
\VS{17}Mais la sagesse d'en haut est premièrement pure, ensuite pacifique, modérée, conciliante, pleine de miséricorde et de bons fruits, sans partialité, et sans hypocrisie.
\VS{18}Or le fruit de la justice est semé dans la paix pour ceux qui s'adonnent à la paix.
\Chap{4}
\TextTitle{Condamnation des mauvais désirs}
\VerseOne{}D'où viennent parmi vous les disputes et les querelles ? N'est-ce pas de vos voluptés qui combattent dans vos membres ?
\VS{2}Vous convoitez, et vous n'obtenez pas ce que vous désirez ; vous avez une envie mortelle, vous êtes jaloux, et vous ne pouvez obtenir ce que vous enviez ; vous vous querellez, vous vous disputez, et vous n'avez pas ce que vous désirez, parce que vous ne demandez pas.
\VS{3}Vous demandez, et vous ne recevez point, parce que vous demandez mal, dans le but de satisfaire vos voluptés.
\VS{4}Hommes et femmes adultères ! Ne savez-vous pas que l'amitié du monde est inimitié contre Dieu ? Celui donc qui veut être ami du monde, se rend ennemi de Dieu.
\VS{5}Pensez-vous que l'Ecriture parle en vain ? L'Esprit qui habite en nous, vous inspire-t-il l'envie ?
\TextTitle{S'humilier devant Yahweh, le juste Juge}
\VS{6}Il vous accorde, au contraire, une plus grande grâce ; c'est pourquoi l'Ecriture dit : Dieu résiste aux orgueilleux, mais il fait grâce aux humbles\FTNT{Pr. 3:34.}.
\VS{7}Soumettez-vous donc à Dieu ; résistez au diable, et il s'enfuira de vous.
\VS{8}Approchez-vous de Dieu, et il s'approchera de vous. Pécheurs, nettoyez vos mains ; et vous qui êtes doubles de cœur, purifiez vos cœurs.
\VS{9}Sentez vos misères ; et soyez dans le deuil et dans les larmes ; que votre rire se change en pleurs, et votre joie en tristesse.
\VS{10}Humiliez-vous dans la présence du Seigneur, et il vous élèvera.
\VS{11}Mes frères, ne médisez point les uns des autres. Celui qui médit de son frère, et qui condamne son frère, médit de la loi et juge la loi. Or, si tu juges la loi, tu n'es pas observateur de la loi, mais le juge.
\VS{12}Il n'y a qu'un seul Législateur, qui peut sauver et qui peut perdre ; mais toi, qui es-tu, qui juges les autres ?
\TextTitle{Abandonner ses désirs au profit de la volonté de Dieu}
\VS{13}A vous, maintenant, qui dites : Aujourd'hui ou demain nous irons dans telle ou telle ville, et nous y passerons une année, et nous trafiquerons et nous gagnerons !
\VS{14}Qui toutefois ne savez pas ce qui arrivera le lendemain ! Car qu'est-ce que votre vie ? Ce n'est certes qu'une vapeur qui parait pour un peu de temps, et qui ensuite s'évanouit.
\VS{15}Au lieu de dire : Si le Seigneur le veut, et si nous vivons, nous ferons aussi ceci ou cela.
\VS{16}Mais maintenant vous vous glorifiez dans vos pensées orgueilleuses. Une telle gloire est mauvaise.
\VS{17}Il y a donc du péché en celui qui sait faire le bien, et qui ne le fait pas.
\Chap{5}
\TextTitle{Avertissement aux riches}
\VerseOne{}A vous maintenant, riches ! Pleurez et gémissez à cause des malheurs qui vont tomber sur vous.
\VS{2}Vos richesses sont pourries et vos vêtements sont rongés par les vers.
\VS{3}Votre or et votre argent sont rouillés ; et leur rouille s'élèvera en témoignage contre vous et dévorera vos chairs comme un feu. Vous avez amassé des trésors pour les derniers jours.
\VS{4}Voici, le salaire des ouvriers qui ont moissonné vos champs, et dont vous les avez frustrés, crie ; et les cris des moissonneurs sont parvenus aux oreilles du Seigneur des armées.
\VS{5}Vous avez vécu dans les délices sur la terre, vous vous êtes livrés aux voluptés, et vous avez rassasié vos cœurs comme en un  jour de sacrifices.
\VS{6}Vous avez condamné et mis à mort le juste qui ne vous a pas résisté.
\TextTitle{Se préparer à l'avènement du Seigneur}
\VS{7}Mais vous, mes frères, attendez patiemment jusqu'à l'avènement du Seigneur. Voici, le laboureur attend le précieux fruit de la terre, prenant patience à son égard, jusqu'à ce qu'il ait reçu les pluies de la première et de la dernière saison.
\VS{8}Vous aussi, attendez patiemment, et affermissez vos cœurs, car l'avènement du Seigneur est proche.
\VS{9}Mes frères, ne vous plaignez pas les uns des autres, afin que vous ne soyez pas condamnés. Voici, le Juge se tient à la porte.
\VS{10}Mes frères, prenez pour exemple de patience dans les afflictions les prophètes qui ont parlé au Nom du Seigneur.
\VS{11}Voici, nous tenons pour bienheureux ceux qui ont enduré l'épreuve avec patience. Vous avez appris quelle a été la patience de Job, et vous avez vu la fin du Seigneur, car le Seigneur est plein de compassion et de miséricorde.
\VS{12}Avant toutes choses, mes frères, ne jurez ni par le ciel, ni par la terre, ni par aucun autre serment. Mais que votre oui soit oui, et que votre non soit non, afin que vous ne tombiez pas sous le jugement\FTNT{Mt. 5:37 ; Mt. 12:36.}.
\VS{13}Quelqu'un parmi vous est-il dans la souffrance ? Qu'il prie. Quelqu'un est-il dans la joie ? Qu'il chante.
\VS{14}Quelqu'un parmi vous est-il malade ? Qu'il appelle les anciens de l'église, et qu'ils prient pour lui en l'oignant d'huile au Nom du Seigneur.
\VS{15}Et la prière faite avec foi sauvera le malade, et le Seigneur le relèvera ; et s'il a commis des péchés, ils lui seront pardonnés.
\VS{16}Confessez donc vos péchés les uns les autres, et priez les uns pour les autres afin que vous soyez guéris. Car la prière du juste faite avec ferveur est de grande efficacité.
\VS{17}Elie était un homme sujet aux mêmes infirmités que nous, et cependant il pria avec instance pour qu'il ne pleuve point, et il ne tomba point de pluie sur la terre pendant trois ans et six mois\FTNT{1 R. 17:1.}.
\VS{18}Puis il pria de nouveau, et le ciel donna de la pluie, et la terre produisit son fruit.
\TextTitle{Conclusion}
\VS{19}Mes frères, si quelqu'un parmi vous s'est égaré loin de la vérité, et qu'un autre l'y ramène,
\VS{20}qu'il sache que celui qui ramènera un pécheur de son égarement, sauvera une âme de la mort et couvrira une multitude de péchés.
\PPE{}
\end{multicols}

%\clearpage\ShortTitle{Ga.}\BookTitle{Galates}\BFont
\noindent\hrulefill
{\footnotesize
\textit{
\bigskip
{\centering{}
\\Auteur~: Paul
\\Thème~: Le salut par la grâce
\\Date de rédaction~: Env. 50 ap. J.-C.\\}
}
\textit{
\\Province antique de l'Asie Mineure, la Galatie se situait en Anatolie. Elle devait son nom aux Galates, Celtes provenant des Balkans.
\\La lettre de Paul aux Galates est la seule épître dont le début ne contient pas de témoignage d'affection. Paul commence par justifier l'origine de son appel, en employant un ton sec et sévère. Les Galates, qu'il avait lui-même évangélisés lors de son premier voyage, s'étaient promptement détournés de l'Evangile qu'ils avaient reçu. Ils ne l'avaient pas totalement abandonné, mais y avaient ajouté ce qui ne leur avait point été prescrit. Troublés par les enseignements des judaïsants - des juifs ayant cru en Jésus-Christ, mais persistant toujours dans la pratique de la loi - les Galates avaient repris à leur compte leurs traditions, annihilant ainsi l'œuvre de la croix. Par cette lettre, Paul les exhorte d'une part à revenir à l'Evangile véritable et d'autre part à marcher par l'Esprit afin d'en porter le fruit.\bigskip
}
}
\par\nobreak\noindent\hrulefill
\begin{multicols}{2}
\Chap{1}
\TextTitle{Introduction}
\VerseOne{}Paul, apôtre, non de la part des hommes, ni de la part d'aucun homme, mais de la part de Jésus-Christ, et de la part de Dieu le Père, qui l'a ressuscité des morts,
\VS{2}et tous les frères qui sont avec moi, aux églises de Galatie\FTNT{La Galatie, ou Gallo-Grèce, était une province de l'Asie Mineure (région de la Turquie actuelle). Au nord, elle était délimitée par la Bithynie et la
Paphlagonie, à l'est par le Pont et la Cappadoce, au sud par la Cappadoce, la Lycaonie et la Phrygie, et à l'ouest par la Phrygie et la Bithynie. Son nom vient des Gaulois qui s'étaient installés dans la région en 279 av. J.-C. Conquise par les Romains en 189 av. J.-C., elle devint une province de l'Empire en 25 av. J.-C.} :
\VS{3}Que la grâce et la paix vous soient données de la part de Dieu le Père, et de la part de notre Seigneur Jésus-Christ,
\VS{4}qui s'est donné lui-même pour nos péchés, afin de nous arracher du présent siècle mauvais, selon la volonté de Dieu notre Père.
\VS{5}A lui soit la gloire aux siècles des siècles. Amen~!
\TextTitle{Les Galates se détournent de l'Evangile véritable}
\VS{6}Je m'étonne que vous abandonniez si promptement celui qui vous avait appelés à la grâce de Christ, pour passer à un autre évangile. 
\VS{7}Non qu'il y ait un autre évangile, mais il y a des gens qui vous troublent, et qui veulent renverser l'Evangile de Christ.
\VS{8}Mais quand nous-mêmes, ou quand un ange venu du ciel vous évangéliserait, outre\FTNT{Voir 1 R. 13:11-34.} ce que nous vous avons évangélisé, qu'il soit anathème~!
\VS{9}Comme nous l'avons déjà dit, je le dis encore maintenant~: Si quelqu'un vous évangélise outre ce que vous avez reçu, qu'il soit anathème~!
\TextTitle{Paul reçoit la révélation de l'Evangile}
\VS{10}Car est-ce les hommes que je prêche ou Dieu~? Ou est-ce que je cherche à plaire aux hommes~? Certes si je plaisais encore aux hommes, je ne serais pas le serviteur de Christ.
\VS{11}Je vous le déclare donc, mes frères, que l'Evangile que j'ai annoncé n'est pas selon l'homme,
\VS{12}parce que je ne l'ai ni reçu ni appris d'aucun homme, mais par la révélation de Jésus-Christ.
\VS{13}Car vous avez appris quelle a été autrefois ma conduite dans le judaïsme, et comment je persécutais à outrance l'Eglise de Dieu et la ravageais,
\VS{14}et comment j'étais plus avancé dans le judaïsme que beaucoup de ceux de mon âge et de ma nation, étant le plus ardent zélateur des traditions de mes pères.
\VS{15}Mais quand il a plu à Dieu, qui m'avait choisi dès le ventre de ma mère, et qui m'a appelé par sa grâce,
\VS{16}de révéler en moi son Fils, afin que je le prêche parmi les Gentils, aussitôt, je ne consultai ni la chair ni le sang,
\VS{17}et je ne montai point à Jérusalem vers ceux qui furent apôtres avant moi, mais je partis pour l'Arabie, puis je revins encore à Damas.
\VS{18}Ensuite, trois ans après, je montai à Jérusalem pour visiter Pierre, et je demeurai chez lui quinze jours.
\VS{19}Et je ne vis aucun des autres apôtres, sinon Jacques, le frère du Seigneur.
\VS{20}Or, dans les choses que je vous écris, voici, devant Dieu je vous dis que je ne mens point.
\VS{21}J'allai ensuite dans les pays de Syrie et de Cilicie.
\VS{22}Or j'étais inconnu de visage aux églises de Judée qui sont en Christ,
\VS{23}mais elles avaient seulement entendu dire~: Celui qui autrefois nous persécutait, annonce maintenant la foi qu'il détruisait autrefois.
\VS{24}Et elles glorifiaient Dieu à cause de moi.
\Chap{2}
\TextTitle{Paul et Barnabas se rendent à Jérusalem\FTNTT{Ac. 15.}}
\VerseOne{}Quatorze ans après, je montai de nouveau à Jérusalem\FTNT{La grande assemblée de Jérusalem. Voir Ac. 15.}, avec Barnabas, et je pris aussi avec moi Tite.
\VS{2}Et ce fut d'après une révélation que j'y montai. J'exposai l'Evangile que je prêche parmi les Gentils à ceux de Jérusalem, en particulier à ceux qui sont les plus considérés, afin de ne pas courir ou avoir couru en vain.
\VS{3}Et même on n'obligea pas Tite, qui était avec moi, de se faire circoncire quoiqu'il fût Grec.
\VS{4}Et cela à cause des faux frères qui s'étaient furtivement introduits et glissés dans l'église pour épier la liberté que nous avons en Jésus-Christ, afin de nous ramener dans la servitude.
\VS{5}Nous ne leur cédâmes pas un instant et nous résistâmes à leurs exigences, afin que la vérité de l'Evangile soit maintenue parmi vous.
\VS{6}Et je ne suis différent en rien de ceux qui sont les plus estimés, quels qu'ils aient été autrefois, Dieu n'ayant point d'égard à l'apparence extérieure de l'homme car ceux qui sont en estime ne m'ont rien communiqué de plus.
\VS{7}Au contraire, quand ils virent que la prédication de l'Evangile pour les incirconcis m'avait été confiée, comme à Pierre pour les circoncis,
\VS{8}car celui qui a opéré avec efficacité par Pierre dans la charge d'apôtre pour les circoncis, a aussi opéré avec efficacité par moi envers les Gentils.
\VS{9}Jacques, dis-je, Céphas, et Jean, qui sont estimés comme des colonnes, ayant reconnu la grâce que j'avais reçue, me donnèrent, à moi et à Barnabas, la main d'association, afin que nous allions, nous vers les Gentils, et qu'ils aillent eux vers les circoncis.
\VS{10}Ils nous recommandèrent seulement de nous souvenir des pauvres, ce que j'ai eu bien soin de faire.
\TextTitle{Paul reprend Pierre à Antioche}
\VS{11}Mais lorsque Pierre vint à Antioche, je lui résistai en face parce qu'il méritait d'être repris.
\VS{12}Car avant l'arrivée de quelques personnes envoyées par Jacques, il mangeait avec les Gentils, mais quand elles furent venues, il s'esquiva et se sépara des Gentils, craignant les circoncis.
\VS{13}Les autres Juifs aussi usèrent de dissimulation comme lui, de sorte que Barnabas même se laissait entraîner par leur hypocrisie.
\VS{14}Mais quand je vis qu'ils ne marchaient pas droit selon la vérité de l'Evangile, je dis à Pierre devant tous~: Si toi qui es Juif, tu vis comme les Gentils, et non pas comme les Juifs, pourquoi contrains-tu les Gentils à judaïser~?
\TextTitle{Le chrétien est mort à la loi mosaïque}
\VS{15}Nous qui sommes Juifs de naissance, et non point pécheurs d'entre les Gentils,
\VS{16}sachant que l'homme n'est pas justifié par les œuvres de la loi, mais seulement par la foi en Jésus-Christ\FTNT{La justification. Voir Ro. 5:1.}, nous, dis-je, nous avons cru en Jésus-Christ, afin que nous soyons justifiés par la foi en Christ, et non point par les œuvres de la loi~; parce que personne ne sera justifié par les œuvres de la loi.
\VS{17}Or si en cherchant à être justifiés par Christ, nous sommes aussi trouvés pécheurs, Christ est-il pourtant serviteur du péché~? A Dieu ne plaise~!
\VS{18}Car si je rebâtis les choses que j'ai renversées, je montre que je suis moi-même un transgresseur.
\VS{19}Car c'est par la loi que je suis mort à la loi, afin de vivre pour Dieu.
\TextTitle{La vie chrétienne doit refléter la vie de Jésus-Christ\FTNTT{Ga. 5:15-23.}}
\VS{20}Je suis crucifié avec Christ~; et si je vis, ce n'est plus moi qui vis, c'est Christ qui vit en moi~; si je vis maintenant dans la chair, je vis dans la foi au Fils de Dieu, qui m'a aimé et qui s'est livré lui-même pour moi.
\VS{21}Je n'anéantis point la grâce de Dieu, car si la justice vient de la loi, Christ est donc mort inutilement.
\Chap{3}
\TextTitle{L'Esprit s'acquiert par la foi}
\VerseOne{}Ô Galates insensés~! Qui vous a ensorcelés pour faire que vous n'obéissiez point à la vérité, vous, aux yeux de qui Jésus-Christ a été auparavant dépeint crucifié, au milieu de vous~?
\VS{2}Je voudrais seulement entendre ceci de vous~: Avez-vous reçu l'Esprit par les œuvres de la loi, ou par la prédication de la foi~?
\VS{3}Etes-vous si insensés, qu'après avoir commencé par l'Esprit, voulez-vous maintenant finir par la chair~?
\VS{4}Avez-vous tant souffert en vain~? Si toutefois c'est en vain.
\VS{5}Celui donc qui vous donne l'Esprit, et qui produit en vous les dons miraculeux, le fait-il par les œuvres de la loi ou par la prédication de la foi~?
\TextTitle{L'alliance avec Abraham, une promesse fondée sur la foi\FTNTT{Ro. 4.}}
\VS{6}Comme Abraham crut à Dieu, et cela lui fut imputé à justice,
\VS{7}sachez donc que ce sont ceux qui ont la foi qui sont fils d'Abraham.
\VS{8}Aussi, l'Ecriture prévoyant que Dieu justifierait les Gentils par la foi, a auparavant évangélisé à Abraham, en lui disant~: Toutes les nations seront bénies en toi\FTNT{Ge. 12:3.}.
\VS{9}C'est pourquoi ceux qui ont la foi sont bénis avec Abraham, le croyant.
\TextTitle{L'attachement aux œuvres de la loi produit la malédiction}
\VS{10}Car tous ceux qui s'attachent aux œuvres de la loi sont sous la malédiction~; car il est écrit~: Maudit est quiconque ne persévère pas dans toutes les choses qui sont écrites dans le livre de la loi et ne les met pas en pratique\FTNT{De. 27:26.}.
\VS{11}Et que nul ne soit justifié devant Dieu par la loi, cela est évident, puisqu'il est dit~: Le juste vivra de la foi\FTNT{Ha. 2:4.}.
\VS{12}Or la loi ne procède pas de la foi, mais elle dit~: L'homme qui mettra ces choses en pratique vivra par elles\FTNT{Lé. 18:5.}.
\TextTitle{Le Messie a racheté les chrétiens de la malédiction de la loi}
\VS{13}Christ nous a rachetés de la malédiction de la loi quand il a été fait malédiction pour nous~; car il est écrit~: Maudit est quiconque est pendu au bois\FTNT{De. 21:23.},
\VS{14}afin que la bénédiction d'Abraham ait son accomplissement pour les Gentils en Jésus-Christ, et que nous recevions par la foi l'Esprit qui avait été promis.
\VS{15}Mes frères, je parle à la manière des hommes, un testament en bonne forme, bien que fait par un homme, n'est annulé par personne, et personne n'y ajoute.
\VS{16}Or les promesses ont été faites à Abraham et à sa postérité. Il n'est pas dit~: Et aux postérités, comme s'il avait parlé de plusieurs, mais comme parlant d'une seule, et à sa postérité, c'est-à-dire Christ.
\VS{17}Voici ce que j'entends~: Une alliance, que Dieu a confirmée antérieurement, ne peut pas être annulée, et ainsi la promesse rendue vaine, par la loi survenue quatre cent trente ans plus tard.
\VS{18}Car si l'héritage venait de la loi, il ne viendrait plus de la promesse. Or c'est par la promesse que Dieu a fait à Abraham ce don de sa grâce.
\TextTitle{La loi~: Pédagogue révélant le péché et conduisant à Christ}
\VS{19}A quoi donc sert la loi~? Elle a été donnée ensuite à cause des transgressions, jusqu'à ce que vienne la postérité à qui la promesse avait été faite~; et elle a été promulguée par des anges, au moyen d'un médiateur.
\VS{20}Or le médiateur n'est pas médiateur d'un seul, mais Dieu est un seul.
\VS{21}La loi a-t-elle donc été ajoutée contre les promesses de Dieu~? Nullement ! Car s'il avait été donné une loi qui puisse procurer la vie, la justice viendrait réellement de la loi.
\VS{22}Mais l'Ecriture a renfermé tous les hommes sous le péché, afin que ce qui avait été promis soit donné par la foi en Jésus-Christ à ceux qui croient.
\VS{23}Or avant que la foi vienne, nous étions renfermés sous la garde de la loi, en vue de la foi qui devait être révélée.
\VS{24}Ainsi la loi a donc été notre pédagogue\FTNT{Le mot «~pédagogue~» du grec «~paidagogos~»~: «~celui qui dirige un garçon~». Un pédagogue était un tuteur, un gardien et un guide de garçons. Parmi les Grecs et les Romains, le mot était appliqué aux esclaves dignes de confiance qui étaient chargés de veiller à la vie et à la moralité des garçons appartenant aux classes supérieures. Les garçons ne pouvaient faire le moindre pas hors de la maison sans ces tuteurs tant qu'ils n'avaient pas atteint leur majorité.} pour nous amener à Christ, afin que nous soyons justifiés par la foi.
\VS{25}Mais la foi étant venue, nous ne sommes plus sous ce pédagogue.
\TextTitle{Ceux qui croient au Messie sont justifiés}
\VS{26}Parce que vous êtes tous fils de Dieu par la foi en Jésus-Christ,
\VS{27}car vous tous qui avez été baptisés en Christ, vous avez revêtu Christ.
\VS{28}Il n'y a plus ni Juif ni Grec, il n'y a plus ni esclave ni libre, il n'y a plus ni homme ni femme~; car vous êtes tous un en Jésus-Christ\FTNT{Ro. 10:12~; Col. 3:11.}.
\VS{29}Et si vous êtes de Christ, vous êtes donc la postérité d'Abraham, et héritiers selon la promesse.
\Chap{4}
\VerseOne{}Or aussi longtemps que l'héritier est enfant\FTNT{«~Enfant~», du grec «~nepios~», signifie aussi «~ignorant~».}, je dis qu'il ne diffère en rien d'un esclave, quoiqu'il soit le maître de tout.
\VS{2}Mais il est sous des tuteurs et des administrateurs jusqu'au temps déterminé par le Père.
\VS{3}Nous aussi, lorsque nous étions enfants, nous étions sous l'esclavage des rudiments du monde.
\VS{4}Mais lorsque les temps ont été accomplis, Dieu a envoyé son Fils, né d'une femme, né sous la loi,
\VS{5}afin qu'il rachète ceux qui étaient sous la loi, afin que nous recevions l'adoption.
\VS{6}Et parce que vous êtes fils, Dieu a envoyé l'Esprit de son Fils dans vos cœurs, lequel crie~: Abba ! C'est-à-dire Père.
\VS{7}Maintenant donc tu n'es plus esclave, mais fils~; or si tu es fils, tu es aussi héritier de Dieu par Christ.
\TextTitle{Le légalisme et la religiosité privent de la grâce}
\VS{8}Autrefois, ne connaissant pas Dieu, vous serviez des dieux qui ne le sont pas de leur nature.
\VS{9}Et maintenant que vous avez connu Dieu, ou plutôt que vous avez été connus de Dieu, comment retournez-vous encore à ces faibles et misérables éléments, auxquels vous voulez encore vous asservir comme auparavant~?
\VS{10}Vous observez les jours, les mois, les temps et les années.
\VS{11}Je crains d'avoir travaillé inutilement pour vous.
\VS{12}Soyez comme moi~; car je suis aussi comme vous~; je vous en prie mes frères.
\VS{13}Vous ne m'avez fait aucun tort. Et vous savez que ce fut à cause d'une infirmité de la chair\FTNT{Les Ecritures ne donnent pas de précisions au sujet de l'infirmité de la chair dont souffrait Paul. On suppose toutefois qu'il avait un handicap au niveau de ses yeux. Quatre arguments viennent renforcer cette hypothèse. Tout d'abord, l'allusion de Paul aux Galates qui étaient prêts à «~s'arracher les yeux~» pour les lui donner (Ga. 4:15) et le fait qu'il ait lui-même écrit cette épître avec de «~grandes lettres~» (Ga. 6:11). Ensuite, lors de sa comparution devant le sanhédrin à Jérusalem, Paul n'a pas reconnu le grand-prêtre pourtant facilement identifiable par sa tenue vestimentaire (Ac. 23:5). Enfin, l'apôtre avait l'habitude de dicter ses lettres, ce qui constitue un argument majeur. L'épître aux Galates était une exception parce qu'il n'avait sans doute pas de secrétaire à disposition.} que je vous ai pour la première fois évangélisés.
\VS{14}Et vous ne m'avez point méprisé ni rejeté à cause de ces épreuves que j'ai dans ma chair~; mais vous m'avez reçu comme un ange de Dieu, et comme Jésus-Christ.
\VS{15}Où donc est l'expression de votre bonheur~? Car je vous atteste que, si cela avait été possible, vous vous seriez arrachés les yeux pour me les donner.
\VS{16}Suis-je donc devenu votre ennemi en vous disant la vérité~?
\VS{17}Ils ont du zèle pour vous, mais non loyalement. Au contraire, ils veulent vous détacher de nous afin que vous soyez zélés pour eux.
\VS{18}Il est bon d'être zélé pour le bien en tout temps, et non pas seulement quand je suis présent parmi vous.
\TextTitle{La loi et la grâce ne peuvent cohabiter~: Agar et Sara représentent deux alliances}
\VS{19}Mes petits enfants, pour qui j'éprouve de nouveau les douleurs de l'enfantement, jusqu'à ce que Christ soit formé en vous,
\VS{20}je voudrais être maintenant avec vous, et changer de langage, car je suis dans une grande inquiétude à votre sujet.
\VS{21}Dites-moi, vous qui voulez être sous la loi, ne comprenez-vous point la loi~?
\VS{22}Car il est écrit qu'Abraham eut deux fils, un de l'esclave, et un de la femme libre.
\VS{23}Mais celui de l'esclave naquit selon la chair~; et celui de la femme libre naquit en vertu de la promesse.
\VS{24}Ces faits ont une valeur allégorique, car ces deux femmes sont deux alliances~: L'une du Mont Sinaï, qui n'enfante que des esclaves, et c'est Agar.
\VS{25}Car le nom d'Agar veut dire Sinaï, qui est une montagne en Arabie correspondant à la Jérusalem actuelle qui est dans la servitude avec ses enfants.
\VS{26}Mais la Jérusalem d'en haut est la femme libre, et c'est notre mère à nous tous.
\VS{27}Car il est écrit~: Réjouis-toi, stérile, toi qui n'enfantes point~! Eclate et pousse des cris, toi qui n'as pas éprouvé les douleurs de l'enfantement~! Car les enfants de la délaissée seront plus nombreux que les enfants de celle qui était mariée\FTNT{Es. 54:1.}.
\VS{28}Or pour nous, mes frères, nous sommes enfants de la promesse comme Isaac.
\VS{29}Et de même qu'alors, celui qui était né selon la chair persécutait celui qui était né selon l'Esprit, il en est de même maintenant.
\VS{30}Mais que dit l'Ecriture~? Chasse l'esclave et son fils, car le fils de l'esclave n'héritera pas avec le fils de la femme libre\FTNT{Ge. 21:10.}.
\VS{31}C'est pourquoi, mes frères, nous ne sommes pas enfants de l'esclave, mais de la femme libre.
\Chap{5}
\TextTitle{Le Messie nous a libérés de la servitude}
\VerseOne{}Demeurez donc fermes dans la liberté pour laquelle Christ nous a affranchis, et ne vous mettez plus sous le joug de la servitude.
\VS{2}Moi, Paul, je vous dis que si vous vous faites circoncire, Christ ne vous servira à rien.
\VS{3}Et j'affirme encore une fois à tout homme qui se fait circoncire qu'il est tenu de pratiquer la loi tout entière.
\VS{4}Vous êtes séparés de Christ, vous tous qui cherchez la justification dans la loi~; vous êtes déchus de la grâce.
\VS{5}Mais pour nous, nous attendons par l'Esprit l'espérance d'être justifiés par la foi.
\VS{6}Car en Jésus-Christ ni la circoncision ni le prépuce\FTNT{Voir le commentaire en 1 Co. 7:18.} n'ont de valeur, mais seulement la foi qui opère par la charité.
\VS{7}Vous couriez bien~: Qui vous a arrêtés pour vous empêcher d'obéir à la vérité~?
\VS{8}Cette influence ne vient pas de celui qui vous appelle.
\VS{9}Un peu de levain fait lever toute la pâte\FTNT{1 Co. 5:6.}.
\VS{10}J'ai cette confiance en vous dans le Seigneur que vous n'aurez pas d'autre sentiment~; mais celui qui vous trouble, quel qu'il soit, en portera la condamnation.
\VS{11}Quant à moi, mes frères, si je prêche encore la circoncision, pourquoi suis-je encore persécuté~? Le scandale de la croix est donc aboli.
\VS{12}Plaise à Dieu que ceux qui vous troublent soient retranchés~!
\VS{13}Car, mes frères, vous avez été appelés à la liberté, seulement ne faites pas de cette liberté une occasion de vivre selon la chair, mais servez-vous les uns les autres avec charité.
\VS{14}Car toute la loi est accomplie dans cette seule parole~: Tu aimeras ton prochain comme toi-même\FTNT{Lé. 19:18~; Mt. 22:39.}.
\VS{15}Mais si vous vous mordez et vous dévorez les uns les autres, prenez garde que vous ne soyez détruits les uns par les autres.
\VS{16}Je vous dis donc~: Marchez selon l'Esprit, et vous n'accomplirez point les désirs de la chair.
\TextTitle{La chair et ses œuvres s'opposent à l'Esprit de Dieu\FTNTT{Ro. 8:2.}}
\VS{17}Car la chair a des désirs contraires à ceux de l'Esprit, et l'Esprit en a des contraires à ceux de la chair~; et ils sont opposés entre eux afin que vous ne fassiez point ce que vous voudriez.
\VS{18}Or si vous êtes conduits par l'Esprit, vous n'êtes point sous la loi.
\VS{19}Car les œuvres de la chair sont évidentes~: Ce sont l'adultère, la fornication, l'impureté, l'impudicité,
\VS{20}l'idolâtrie, la sorcellerie\FTNT{La sorcellerie~: du grec «~pharmakeia~»~: «~usage~» ou «~administration~» de drogues, «~empoisonnement~», «~sorcellerie~», «~arts magiques~», souvent trouvés en liaison avec l'idolâtrie et nourrie par celle-ci.}, les inimitiés, les querelles, les jalousies, les animosités, les disputes, les divisions, les sectes,
\VS{21}les envies, les meurtres, l'ivrognerie, les excès de table, et les choses semblables à celles-là, au sujet desquelles je vous prédis, comme je vous l'ai déjà dit, que ceux qui commettent de telles choses n'hériteront point le Royaume de Dieu.
\TextTitle{Le fruit de l'Esprit\FTNTT{Jn. 15:1-5~; Ga. 2:20.}}
\VS{22}Mais le fruit de l'Esprit c'est la charité\FTNT{Il est question ici de l'amour «~agape~»~: l'amour fraternel, la charité désintéressée.}, la joie, la paix, la patience, la bonté, la bienveillance, la foi, la douceur, la tempérance.
\VS{23}La loi n'est pas contre ces choses.
\VS{24}Ceux qui sont à Christ ont crucifié la chair avec ses passions et ses désirs.
\VS{25}Si nous vivons par l'Esprit, marchons aussi par l'Esprit.
\VS{26}Ne cherchons pas une vaine gloire, en nous provoquant les uns les autres et en nous portant envie les uns aux autres.
\Chap{6}
\TextTitle{La mise en pratique de la vie nouvelle en Jésus-Christ}
\VerseOne{}Mes frères, lorsqu'un homme est surpris en quelque faute, vous qui êtes spirituels, redressez-le avec un esprit de douceur. Prends garde à toi-même, de peur que tu ne sois aussi tenté.
\VS{2}Portez les fardeaux les uns des autres, et vous accomplirez ainsi la loi de Christ.
\VS{3}Car si quelqu'un pense être quelque chose, quoiqu'il ne soit rien, il s'abuse lui-même.
\VS{4}Que chacun examine ses propres œuvres, et alors il aura de quoi se glorifier pour lui-même seulement, et non par rapport aux autres.
\VS{5}Car chacun portera son propre fardeau.
\VS{6}Que celui à qui l'on enseigne la parole fasse part en tous biens à celui qui l'enseigne\FTNT{Le mot «~bien~» vient du grec «~agathos~» qui donne en français~: «~de bonne constitution ou nature~», «~utile~», «~salutaire~», «~bon~», «~agréable~», «~plaisant~», «~joyeux~», «~heureux~», «~excellent~», «~distingué~», «~droit~», «~honorable~» et n'a rien à voir avec les biens matériels (Voir Ga. 6:10). Il ne doit en aucun cas servir de prétexte à ceux qui enseignent la Parole de Dieu pour exiger l'argent et les biens matériels des chrétiens. Ces derniers doivent donner sans contrainte, s'ils le veulent et comme ils le veulent (2 Co. 9:7). Le salaire de l'ouvrier du Seigneur c'est avant tout le gîte et le couvert (Mt. 10:10~; Lu. 10:8~; 1 Ti. 6:8). Ainsi, malgré le droit qu'il avait de moissonner les biens matériels pour avoir semé des biens spirituels (1 Co. 9:11-12), Paul «~n'a désiré ni l'or ni l'argent~» mais a travaillé de ses propres mains afin de pourvoir à ses besoins et de n'être à la charge de personne (Ac. 20:33-35~; 1 Th. 2:9~; 2 Th. 3:8~; 2 Co. 12:14 ).}.
\VS{7}Ne vous séduisez pas, on ne se moque pas de Dieu. Ce qu'un homme aura semé, il le moissonnera aussi.
\VS{8}C'est pourquoi celui qui sème pour sa chair moissonnera de la chair la corruption~; mais celui qui sème pour l'Esprit moissonnera de l'Esprit la vie éternelle.
\VS{9}Ne nous lassons pas de faire le bien~; car nous moissonnerons au temps convenable, si nous ne nous relâchons pas.
\VS{10}C'est pourquoi, pendant que nous en avons le temps, faisons du bien envers tous, mais principalement envers ceux qui sont de la famille de la foi.
\VS{11}Vous voyez avec quelles grandes lettres je vous ai écrit de ma propre main.
\VS{12}Tous ceux qui veulent se rendre agréables selon la chair vous contraignent à vous faire circoncire, uniquement afin de ne pas être persécutés pour la croix de Christ.
\VS{13}Car les circoncis eux-mêmes n'observent pas la loi~; mais ils veulent que vous soyez circoncis pour se glorifier dans votre chair.
\VS{14}Pour ce qui me concerne, loin de moi la pensée de me glorifier d'autre chose que de la croix de notre Seigneur Jésus-Christ, par qui le monde est crucifié pour moi, comme je le suis pour le monde~!
\VS{15}Car ce n'est rien que d'être circoncis ou incirconcis~; ce qui est quelque chose c'est d'être une nouvelle créature.
\VS{16}Que la paix et la miséricorde soient sur tous ceux qui suivront cette règle, et sur l'Israël de Dieu !
\TextTitle{Conclusion}
\VS{17}Au reste, que personne ne me fasse de la peine, car je porte sur mon corps les marques du Seigneur Jésus.
\VS{18}Mes frères, que la grâce de notre Seigneur Jésus-Christ soit avec votre esprit~! Amen~!
\PPE{}
\end{multicols}

%\clearpage\ShortTitle{1 Th.}\BookTitle{1 Thessaloniciens}\BFont
\noindent\hrulefill
{\footnotesize
\textit{
\bigskip
{\centering{}
\\Auteur~: Paul
\\Thème~: Le retour de Christ
\\Date de rédaction~: Env. 51 ap. J.-C.\\}
}
\textit{
\\Autrefois appelée Therme ou Therma, qui signifie «~source chaude~», Thessalonique reçut son nouveau nom de Cassandre, en l'honneur de sa femme Thessalonike, qui était aussi la sœur d'Alexandre le Grand (356 av. J.-C. - 323 av. J.-C.), à qui il succéda. Cette ville est située au nord de la Grèce actuelle, sur la côte de la mer Egée. Du temps de Paul, ce pays était divisé en deux parties. Dans la région du nord, la Macédoine, se trouvaient les villes de Philippes, Thessalonique et Bérée. Quant à la région du sud, l'Achaïe, elle comportait les villes d'Athènes et de Corinthe. Aujourd'hui, la ville s'appelle Salonique.
\\En ce temps-là, Thessalonique comptait environ 200 000 habitants (Grecs, Romains et Juifs) et jouissait d'une importante fréquentation puisqu'elle figurait parmi les trois ports principaux de la Méditerranée et se situait sur l'une des plus grandes routes commerciales de l'époque~: La Voie Egnatienne reliant Rome à Byzance.
\\Sur le plan religieux, les habitants étaient polythéistes et pratiquaient une variété de cultes, dont le culte impérial. Durant trois semaines, Paul enseigna dans une synagogue à Thessalonique et réussit à constituer un groupe de croyants composé de Juifs, de Gentils, de pauvres et de plusieurs femmes de la haute société. Toutefois, une violente persécution l'obligea à quitter promptement la ville, laissant la communauté nouvellement formée vulnérable et fragile.
\\La première épître adressée par Paul aux Thessaloniciens leur parvint quelques mois après le passage de l'équipe apostolique et après la visite de Timothée. Cette lettre avait pour but d'affermir les Thessaloniciens dans les vérités fondamentales qui leur avaient été enseignées, de les exhorter à vivre une vie de sainteté pour être agréables à Dieu, de les éclairer quant au devenir des défunts et de les assurer du retour certain du Seigneur.\bigskip
}
}
\par\nobreak\noindent\hrulefill
\begin{multicols}{2}
\Chap{1}
\TextTitle{Introduction}
\VerseOne{}Paul, et Silvain, et Timothée, à l'église des Thessaloniciens qui est en Dieu le Père, et en Jésus-Christ, notre Seigneur~: Que la grâce et la paix vous soient données de la part de Dieu notre Père, et du Seigneur Jésus-Christ~!
\VS{2}Nous rendons toujours grâces à Dieu pour vous tous, faisant mention de vous dans nos prières,
\VS{3}en nous rappelant sans cesse l'œuvre de votre foi, le travail de votre charité, et l'immuabilité de votre espérance en notre Seigneur Jésus-Christ devant notre Dieu et Père,
\VS{4}sachant, mes frères bien-aimés de Dieu, votre élection.
\TextTitle{Proclamation de l'Evangile avec puissance et avec l'Esprit Saint}
\VS{5}Car notre Evangile ne vous a pas été prêché en paroles seulement, mais aussi en puissance, avec l'Esprit Saint, et avec une pleine persuasion~; car vous n'ignorez pas que nous nous sommes montrés ainsi parmi vous, à cause de vous.
\VS{6}Aussi avez-vous été nos imitateurs et ceux du Seigneur, ayant reçu avec la joie du Saint-Esprit, la parole au milieu de grandes afflictions,
\VS{7}de sorte que vous avez été des modèles à tous les fidèles de la Macédoine\FTNT{La Macédoine était le pays natal d'Alexandre le Grand. Elle fut conquise par les Romains et devint une province romaine, dont la capitale était Thessalonique.} et de l'Achaïe\FTNT{L'Achaïe était une province romaine placée sous l'autorité d'un proconsul résidant dans la capitale qui était Corinthe (2 Co. 1:1).}.
\VS{8}Car la parole du Seigneur a retenti de chez vous, non seulement dans la Macédoine et dans l'Achaïe mais aussi en tous lieux, et votre foi envers Dieu est si répandue, que nous n'avons pas besoin d'en parler.
\VS{9}Car eux-mêmes racontent de nous quel accès nous avons eu auprès de vous, et comment vous vous êtes convertis à Dieu en vous séparant des idoles, pour servir le Dieu vivant et vrai,
\VS{10}et pour attendre des cieux son Fils Jésus, qu'il a ressuscité des morts, et qui nous délivre de la colère à venir\FTNT{La colère à venir. Voir les sept coupes de la colère de Dieu (Ap. 15:5-8~; 16:1-21).}.
\Chap{2}
\TextTitle{Annoncer l'Evangile en recherchant l'approbation de Dieu et non celle des hommes}
\VerseOne{}Car, mes frères, vous savez vous-mêmes que notre entrée au milieu de vous n'a point été vaine. 
\VS{2}Après avoir souffert et reçu des outrages à Philippes\FTNT{Philippes était une ville de Macédoine située en Thrace, près de la côte nord de la mer Egée. Voir Ac. 16:12-40 et l'épître de Paul aux Philippiens.}, comme vous le savez, nous avons pris de l'assurance en notre Dieu, pour vous annoncer l'Evangile de Dieu au milieu de beaucoup de combats.
\VS{3}Car il n'y a eu dans notre prédication ni séduction, ni motif impur, ni fraude.
\VS{4}Mais comme Dieu nous a considérés dignes de nous confier la prédication de l'Evangile, ainsi nous parlons non comme pour plaire aux hommes, mais à Dieu qui éprouve nos cœurs.
\VS{5}Car, en effet, nous n'avons jamais été surpris avec des paroles flatteuses, comme vous le savez~; jamais nous n'avons eu pour prétexte la cupidité, Dieu en est témoin.
\VS{6}Et nous n'avons point cherché la gloire qui vient des hommes, ni de vous, ni des autres~; nous aurions pu nous imposer comme apôtres de Christ,
\VS{7}mais nous avons été doux au milieu de vous, de même qu'une nourrice chérit ses enfants.
\VS{8}Nous aurions voulu dans notre affection envers vous, non seulement vous donner l'Evangile de Dieu, mais encore notre propre vie, tant vous nous étiez devenus chers.
\VS{9}Car, mes frères, vous vous souvenez de notre peine et de notre travail~; vu que nous vous avons prêché l'Evangile de Dieu, en travaillant nuit et jour, pour n'être point à charge à aucun de vous.
\VS{10}Vous êtes témoins et Dieu aussi, combien notre conduite envers vous qui croyez a été sainte, juste, et irréprochable.
\VS{11}Et vous savez que nous avons exhorté chacun de vous, comme un père exhorte ses enfants,
\VS{12}en vous exhortant, vous encourageant et vous conjurant de vous conduire d'une manière digne de Dieu, qui vous appelle à son Royaume et à sa gloire.
\VS{13}C'est pourquoi nous rendons sans cesse grâces à Dieu, de ce que, quand vous avez reçu de nous la parole de la prédication de Dieu, vous l'avez reçue non comme une parole des hommes, mais ainsi qu'elle est véritablement, comme la parole de Dieu, laquelle aussi agit avec efficacité en vous qui croyez.
\VS{14}En effet, mes frères, vous êtes devenus les imitateurs des églises de Dieu qui sont en Jésus-Christ dans la Judée, parce que vous aussi, vous avez souffert de la part de ceux de votre propre nation les mêmes choses qu'elles ont souffertes de la part des Juifs,
\VS{15}qui ont même mis à mort le Seigneur Jésus, et leurs propres prophètes, qui nous ont persécutés, qui ne plaisent point à Dieu, et qui sont ennemis de tous les hommes,
\VS{16}nous empêchant de parler aux Gentils afin qu'ils soient sauvés, comblant ainsi toujours plus la mesure de leurs péchés. Mais la colère de Dieu est venue sur eux jusqu'au plus haut degré.
\VS{17}Pour nous, mes frères, après avoir été quelque temps séparés de vous de corps et non de cœur, nous avons eu d'autant plus d'ardeur et d'empressement de vous revoir.
\VS{18}Nous avons donc voulu, une et même deux fois, aller chez vous, au moins, moi Paul~; mais Satan nous en a empêchés.
\VS{19}Car quelle est notre espérance, ou notre joie, ou notre couronne de gloire~? N'est-ce pas vous qui l'êtes, devant notre Seigneur Jésus-Christ lors de son avènement~?
\VS{20}Certes, vous êtes notre gloire et notre joie.
\Chap{3}
\TextTitle{La persévérance des Thessaloniciens dans l'affliction}
\VerseOne{}C'est pourquoi ne pouvant plus soutenir la privation de vos nouvelles, nous avons trouvé bon de demeurer seuls à Athènes.
\VS{2}Et nous avons envoyé Timothée, notre frère, serviteur de Dieu, et notre compagnon d'œuvre dans l'Evangile de Christ, pour vous affermir et vous exhorter au sujet de votre foi,
\VS{3}afin que nul ne soit troublé dans ces afflictions, puisque vous savez vous-mêmes que nous sommes destinés à cela.
\VS{4}Et lorsque nous étions avec vous, nous vous annoncions d'avance que nous aurions à souffrir des afflictions, comme cela est aussi arrivé, et vous le savez.
\VS{5}C'est pourquoi, dis-je, ne pouvant plus soutenir cette inquiétude, j'ai envoyé Timothée pour connaître l'état de votre foi, de peur que le tentateur ne vous ait tentés en quelque sorte, et que nous n'ayons travaillé en vain.
\VS{6}Mais Timothée étant revenu depuis peu de chez vous, nous a apporté d'agréables nouvelles de votre foi et de votre charité, et nous a dit que vous conservez toujours un bon souvenir de nous, désirant nous voir, comme nous désirons aussi vous voir.
\VS{7}C'est pourquoi, mes frères, nous avons été consolés par votre foi, dans toutes nos afflictions et dans toutes nos détresses.
\VS{8}Car maintenant nous vivons puisque vous demeurez fermes dans le Seigneur.
\VS{9}Et quelles actions de grâces ne pouvons-nous pas rendre à Dieu à votre sujet, pour toute la joie que nous éprouvons devant notre Dieu, à cause de vous.
\VS{10}Nuit et jour, nous le prions avec une extrême ardeur de nous permettre de vous voir, et de compléter\FTNT{Compléter~: du grec «~katartizo~» qui signifie «~redresser~», «~ajuster~», «~compléter~», «~raccommoder~» (ce qui a été abîmé), «~réparer~». Ce verbe est également utilisé dans Mt. 4:21 lorsque Jacques et Jean réparaient leurs filets. Le terme «~katartismos~» traduit par «~perfectionnement~» dans Ep. 4:11 vient de ce verbe. Ainsi, l'un des rôles de ces services est le perfectionnement des saints et non leur destruction.} ce qui manque à votre foi.
\VS{11}Que Dieu lui-même, notre Père, et notre Seigneur Jésus-Christ, aplanisse\FTNT{Le verbe «~aplanir~» vient du grec «~kateuthuno~». On constate que ce verbe est conjugué au singulier, y compris dans le texte original grec, ce qui atteste l'unité entre le Père et le Fils (voir 2 Th. 2:16-17).} notre chemin pour que nous allions vers vous.
\VS{12}Et que le Seigneur vous fasse croître et abonder de plus en plus en charité les uns envers les autres, et envers tous, comme nous abondons aussi en charité envers vous~;
\VS{13}qu'il affermisse vos cœurs pour qu'ils soient irréprochables dans la sainteté, devant Dieu qui est notre Père, lors de l'avènement de notre Seigneur Jésus-Christ, accompagné de tous ses saints.
\Chap{4}
\TextTitle{Appel à la sanctification et à l'amour fraternel}
\VerseOne{}Au reste, mes frères, nous vous prions donc, et nous vous conjurons par le Seigneur Jésus, que comme vous avez appris de nous de quelle manière on doit se conduire, et plaire à Dieu, vous y fassiez tous les jours de nouveaux progrès.
\VS{2}Car vous savez quels préceptes nous vous avons donnés de la part du Seigneur Jésus.
\VS{3}Parce que c'est ici la volonté de Dieu~; savoir votre sanctification\FTNT{La sanctification personnelle (1 Pi. 1:15-18~; Hé. 12:14~; Ap. 22:11). Chaque chrétien doit fournir un effort, en se servant quotidiennement de la Parole de Dieu et de la prière, pour se maintenir dans la sanctification. Cela implique la séparation d'avec le mal et des mauvaises compagnies (2 Co. 6:14-18). Elle se développe au prix de nombreuses souffrances et de multiples sacrifices (Ro. 12:1-3).}, et que vous vous absteniez de la fornication,
\VS{4}c'est que chacun de vous sache posséder son corps dans la sanctification et dans l'honneur,
\VS{5}et sans se laisser aller aux désirs de la convoitise, comme les Gentils qui ne connaissent point Dieu.
\VS{6}Que personne n'use de fraude envers son frère et de cupidité dans les affaires, parce que le Seigneur tire vengeance de toutes ces choses, comme nous vous l'avons dit et attesté.
\VS{7}Car Dieu ne nous a pas appelés à l'impureté, mais à la sanctification.
\VS{8}C'est pourquoi celui qui rejette ceci ne rejette pas un homme, mais Dieu qui a aussi donné son Saint-Esprit.
\VS{9}Quant à la charité fraternelle\FTNT{Le mot grec employé ici est «~philadelphia~». Ce terme désigne l'amour fraternel, l'amour que les chrétiens se portent entre eux.}, vous n'avez pas besoin que je vous en écrive~; car vous-mêmes vous êtes enseignés de Dieu à vous aimer les uns les autres,
\VS{10}et c'est aussi ce que vous faites à l'égard de tous les frères qui sont dans toute la Macédoine. Mais, mes frères, nous vous prions de vous perfectionner tous les jours davantage,
\VS{11}et de tâcher de vivre paisiblement~; de faire vos propres affaires, et de travailler de vos propres mains, ainsi que nous vous l'avons ordonné.
\VS{12}En sorte que vous vous conduisiez honnêtement envers ceux du dehors, et que vous n'ayez besoin de rien.
\TextTitle{L'enlèvement de l'Eglise}
\VS{13}Or, mes frères, je ne veux pas que vous soyez dans l'ignorance au sujet de ceux qui dorment, afin que vous ne soyez point attristés comme les autres qui n'ont point d'espérance. 
\VS{14}Car si nous croyons que Jésus est mort, et qu'il est ressuscité~; de même aussi ceux qui dorment en Jésus, Dieu les ramènera avec lui.
\VS{15}Car nous vous disons ceci par la parole du Seigneur, que nous qui vivrons et resterons pour l'avènement du Seigneur, ne précéderons point ceux qui dorment.
\VS{16}Car le Seigneur lui-même, avec un cri de commandement\FTNT{L'expression «~cri de commandement~» vient du grec «~keleuma~», ce mot signifie un ordre, et en particulier un cri stimulant, comme celui que reçoit un animal pressé par un homme, tels les chevaux par les conducteurs de chariots, les chiens de chasse par les chasseurs, etc.~; ou par lequel un ordre est donné par le capitaine d'un navire, aux soldats par un chef, un appel de trompette. La sagesse de Dieu crie (Pr. 8). Esaïe devait crier à plein gosier (Es. 58:1). Le cri du Seigneur ne sera entendu que par l'Eglise véritable qui est son épouse (Mt. 25:6).}, et une voix d'archange, et avec la trompette de Dieu, descendra du ciel, et les morts en Christ ressusciteront premièrement.
\VS{17}Puis nous qui vivrons et qui resterons, serons enlevés ensemble avec eux dans les nuées, à la rencontre du Seigneur, dans les airs et ainsi nous serons toujours avec le Seigneur. 
\VS{18}C'est pourquoi consolez-vous les uns les autres par ces paroles.
\Chap{5}
\TextTitle{Veiller en attendant le jour du Seigneur~; encouragements divers\FTNTT{Joë. 1:15.}}
\VerseOne{}Pour ce qui est des temps et des moments, mes frères, vous n'avez pas besoin qu'on vous en écrive,
\VS{2}puisque vous savez vous-mêmes très bien que le jour du Seigneur viendra comme un voleur dans la nuit\FTNT{Mt. 25:6~; 2 Pi. 3:10~; Ap. 3:3~; 16:15.}.
\VS{3}Quand ils diront~: Nous sommes en paix et en sûreté. Alors une destruction soudaine les surprendra, comme les douleurs de l'enfantement surprennent la femme enceinte, et ils n'échapperont point.
\VS{4}Mais quant à vous, mes frères, vous n'êtes pas dans les ténèbres pour que ce jour-là vous surprenne comme un voleur.
\VS{5}Vous êtes tous des enfants de la lumière, et des enfants du jour. Nous ne sommes point de la nuit ni des ténèbres.
\VS{6}Ne dormons donc point comme les autres, mais veillons et soyons sobres.
\VS{7}Car ceux qui dorment, dorment la nuit, et ceux qui s'enivrent, s'enivrent la nuit.
\VS{8}Mais nous qui sommes enfants du jour, soyons sobres, ayant revêtu la cuirasse de la foi et de la charité, et ayant pour casque l'espérance du salut\FTNT{Ro. 13:12~; Ep. 6:14~; 6:17.}.
\VS{9}Car Dieu ne nous a pas destinés à la colère\FTNT{La colère à venir. Voir 1 Th. 1:9-10.}, mais à l'acquisition du salut par notre Seigneur Jésus-Christ,
\VS{10}qui est mort pour nous, afin que soit que nous veillons, soit que nous dormions, nous vivions avec lui.
\VS{11}C'est pourquoi exhortez-vous réciproquement, et édifiez-vous tous, les uns les autres, comme aussi vous le faites.
\VS{12}Nous vous prions, mes frères, d'avoir de la considération pour ceux qui travaillent parmi vous, qui dirigent dans le Seigneur, et qui vous exhortent.
\VS{13}Ayez pour eux beaucoup d'affection\FTNT{Littéralement «~agape~»~: amour, charité, affection.} à cause de l'œuvre qu'ils font. Soyez en paix entre vous.
\VS{14}Nous vous en prions aussi, mes frères, avertissez ceux qui vivent dans le désordre\FTNT{Mt. 18:15~; Ga. 6:1.}, consolez ceux qui ont l'esprit abattu, supportez les faibles, et soyez patients envers tous.
\VS{15}Prenez garde que personne ne rende à autrui le mal pour le mal\FTNT{Mt. 5:44~; Ro. 12:21.}~; mais recherchez toujours ce qui est bon, soit entre vous, soit envers tous les hommes.
\VS{16}Soyez toujours joyeux.
\VS{17}Priez sans cesse.
\VS{18}Rendez grâces pour toutes choses, car c'est la volonté de Dieu par Jésus-Christ.
\VS{19}N'éteignez point l'Esprit.
\VS{20}Ne méprisez point les prophéties.
\VS{21}Eprouvez toutes choses~; retenez ce qui est bon.
\VS{22}Abstenez-vous de toute apparence de mal.
\VS{23}Que le Dieu de paix veuille vous sanctifier entièrement, et faire que votre être entier, l'esprit, l'âme et le corps soient conservés sans reproche lors de la venue de notre Seigneur Jésus-Christ\FTNT{L'avènement du Seigneur. Voir Mt. 24:1-3.}.
\VS{24}Celui qui vous appelle est fidèle, c'est pourquoi il fera ces choses en vous.
\TextTitle{Salutations}
\VS{25}Mes frères, priez pour nous.
\VS{26}Saluez tous les frères par un saint baiser.
\VS{27}Je vous en conjure par le Seigneur que cette épître soit lue à tous les saints frères.
\VS{28}Que la grâce de notre Seigneur Jésus-Christ soit avec vous~! Amen~!
\PPE{}
\end{multicols}

%\clearpage\ShortTitle{2 Th.}\BookTitle{2 Thessaloniciens}\BFont
\noindent\hrulefill
{\footnotesize
\textit{
\bigskip
{\centering{}
\\Auteur~: Paul
\\Thème~: Le jour de Christ
\\Date de rédaction~: Env. 51 ap. J.-C.\\}
}
\textit{
\\Autrefois appelée Therme ou Therma, qui signifie «~source chaude~», Thessalonique reçut son nouveau nom de Cassandre, en l'honneur de sa femme Thessalonike, qui était aussi la sœur d'Alexandre le Grand (356 av. J.-C. - 323 av. J.-C.), à qui il succéda.
\\Cette ville est située au nord de la Grèce actuelle, sur la côte de la mer Egée. Du temps de Paul, ce pays était divisé en deux parties. Dans la région du nord, la Macédoine, se trouvaient les villes de Philippes, Thessalonique et Bérée. Quant à la région du sud, l'Achaïe, comportait les villes d'Athènes et de Corinthe. Aujourd'hui, la ville s'appelle Salonique. La seconde épître de Paul aux Thessaloniciens fut rédigée peu de temps après la première. Elle fut motivée par des troubles survenus dans la communauté à la suite d'une annonce basée sur une lettre faussement attribuée à Paul prétendant que le «~jour du Seigneur~» était arrivé. Dans cette seconde épître, l'apôtre exhorte les chrétiens de Thessalonique à tenir ferme dans leur foi malgré la persécution, leur expliquant que le «~jour de Christ~» devait être précédé par l'apostasie et la venue de l'homme impie. Il conclut sa lettre en demandant aux chrétiens de s'éloigner de ceux qui vivent dans le désordre.\bigskip
}
}
\par\nobreak\noindent\hrulefill
\begin{multicols}{2}
\Chap{1}
\TextTitle{Introduction}
\VerseOne{}Paul, Silvain, et Timothée à l'église des Thessaloniciens\FTNT{Thessalonique. Voir Ac. 17:1-9.} qui est en Dieu notre Père, et en notre Seigneur Jésus-Christ~:
\VS{2}Que la grâce et la paix vous soient données de la part de Dieu notre Père, et de la part du Seigneur Jésus-Christ~!
\TextTitle{La persévérance dans l'affliction~; Dieu, le juste Juge}
\VS{3}Mes frères, nous devons toujours rendre grâces à Dieu à cause de vous, comme il est bien raisonnable, parce que votre foi augmente beaucoup, et que votre charité mutuelle fait des progrès.
\VS{4}De sorte que nous-mêmes nous nous glorifions de vous dans les églises de Dieu, à cause de votre persévérance et de votre foi au milieu de toutes vos persécutions, et des afflictions que vous avez à supporter,
\VS{5}qui sont une manifeste démonstration du juste jugement de Dieu, afin que vous soyez jugés dignes du Royaume de Dieu, pour lequel aussi vous souffrez.
\TextTitle{La fin de ceux qui ne connaissent pas Dieu et qui n'obéissent pas à l'Evangile}
\VS{6}Car il est juste devant Dieu qu'il rende l'affliction à ceux qui vous affligent,
\VS{7}et qu'il vous donne du repos à vous qui êtes affligés, de même qu'à nous, lorsque le Seigneur Jésus se révélera\FTNT{Révélation, du grec «~apokalupsis~», signifie «~mettre à nu, révélation d'une vérité~». Le fait de rendre visible ce qui était caché.} du ciel avec les anges de sa puissance,
\VS{8}avec des flammes de feu, pour exercer la vengeance contre ceux qui ne connaissent pas Dieu, contre ceux qui n'obéissent pas à l'Evangile de notre Seigneur Jésus-Christ.
\VS{9}Ils auront pour châtiment une ruine éternelle, loin de la face du Seigneur, et de la gloire de sa force,
\VS{10}quand il viendra pour être glorifié en ce jour-là dans ses saints, et pour être admiré dans tous ceux qui croient, parce le témoignage que avons rendu auprès de vous à été cru.
\VS{11}C'est pourquoi nous prions toujours pour vous, afin que notre Dieu vous juge dignes de la vocation, et qu'il accomplisse puissamment en vous tout le bon plaisir de sa bonté, et l'œuvre de la foi,
\VS{12}afin que le Nom de notre Seigneur Jésus-Christ soit glorifié en vous, et vous en lui, selon la grâce de notre Dieu et Seigneur Jésus-Christ.
\Chap{2}
\TextTitle{Le jour du Seigneur et l'apparition de l'homme impie}
\VerseOne{}Or pour ce qui concerne l'avènement\FTNT{L'avènement du Seigneur Jésus-Christ. Voir Mt. 24:1-3.} de notre Seigneur Jésus-Christ et notre réunion en lui, mes frères, nous vous prions
\VS{2}de ne pas vous laisser subitement ébranler dans votre entendement, ni troubler par une inspiration, ni par une parole, ou par quelque lettre qu'on dirait venir de nous, comme si le jour de Christ était déjà là.
\VS{3}Que personne donc ne vous séduise d'aucune manière~; car il faut que l'apostasie soit arrivée auparavant et que l'homme de péché, le fils de la perdition\FTNT{Il est question ici de l'homme impie, de l'antichrist, qui est la bête qui monte de la mer décrite par Jean (Ap. 13:11-18). Voir aussi Da. 11:36-38.}, soit révélé,
\VS{4}lequel s'oppose et s'élève contre tout ce qui est appelé Dieu, ou qu'on adore, jusqu'à être assis comme Dieu dans le temple de Dieu\FTNT{Selon les chapitres 40 à 42 d'Ezéchiel, le culte lévitique sera restauré à la fin des temps, ce qui suppose nécessairement la reconstruction du temple de Jérusalem. Cette prophétie est actuellement (2014-2015) en train de s'accomplir puisque des juifs religieux militent activement pour la réalisation de ce projet. L'organisation la plus connue œuvrant en ce sens est l'Institut du temple (fondé en 1987), qui a déjà restauré un grand nombre d'objets servant au culte. Toutefois, il ne faut pas sous-estimer la ruse de Satan, car au-delà du temple physique, il cherche prioritairement à s'asseoir dans les temples spirituels que sont les chrétiens (1 Co. 6:19). Pour parvenir à ses fins, Satan a envoyé plusieurs de ses émissaires pour prêcher un autre évangile et un autre christ. C'est ainsi que de nombreuses assemblées, séduites et captivées par de faux docteurs, n'ont plus Jésus-Christ comme Seigneur, mais Satan en personne. L'apostasie étant installée premièrement dans les cœurs, l'antichrist n'aura donc aucun mal à se faire passer pour le Christ et à s'asseoir dans le temple physique, où il usurpera l'adoration qui revient au Dieu véritable.} se proclamant lui-même être Dieu.
\VS{5}Ne vous souvenez-vous pas que je vous disais ces choses, lorsque j'étais encore chez vous~?
\VS{6}Et maintenant vous savez ce qui le retient, afin qu'il soit révélé en son temps.
\VS{7}Car le mystère de l'iniquité\FTNT{Le mystère de l'iniquité. Paul nous enseigne que ce mystère était déjà à l'œuvre au sein des églises primitives. Le prophète Zacharie, au chapitre 5 de son livre, l'avait personnifié en relatant une vision dans laquelle il avait vu «~deux femmes avec des ailes de cigogne~» emportant l'épha de l'iniquité des enfants d'Israël. Sur cet épha était assise une femme personnifiant l'iniquité, c'est-à-dire la femme de l'homme impie, la Babylone religieuse. Ces deux femmes aux ailes de cigogne allaient lui bâtir une maison au pays de Schinéar (Babylone selon Ge. 10:6-14).} opère déjà, seulement celui qui le retient en ce moment le fera jusqu’à ce qu’il soit hors du chemin.
\VS{8}Et alors sera révélé le méchant\FTNT{Es. 11:4.}, que le Seigneur détruira par le souffle de sa bouche et qu'il anéantira par l'éclat de son avènement.
\VS{9}L'avènement\FTNT{Il y aura un autre avènement, celui de l'homme impie.} de cet impie, se fera par la puissance de Satan, avec toutes sortes de miracles, de signes, et de prodiges mensongers,
\VS{10}et avec toutes les séductions de l'iniquité, pour ceux qui périssent parce qu'ils n'ont pas reçu l'amour de la vérité pour être sauvés.
\VS{11}C'est pourquoi Dieu leur envoie une puissance d'égarement\FTNT{L'esprit d'égarement. Voir Ro. 1:26,28~; 1 R. 22.}, pour qu'ils croient au mensonge,
\VS{12}afin que tous ceux qui n'ont pas cru à la vérité, mais qui ont pris plaisir à l'iniquité soient condamnés.
\TextTitle{Encouragements}
\VS{13}Mais nous, mes frères bien-aimés du Seigneur, nous devons toujours rendre grâces à Dieu pour vous, de ce que Dieu vous a élus dès le commencement pour le salut par la sanctification de l'Esprit, et par la foi en la vérité.
\VS{14}C'est à quoi il vous a appelés par notre Evangile, afin que vous possédiez la gloire qui nous a été acquise par notre Seigneur Jésus-Christ.
\VS{15}C'est pourquoi, mes frères, demeurez fermes, et retenez les enseignements que vous avez appris, soit par notre parole, soit par notre lettre.
\VS{16}Et que notre Seigneur Jésus-Christ lui-même, et notre Dieu et Père, qui nous a aimés, et qui nous a donné une consolation éternelle, et une bonne espérance par sa grâce,
\VS{17}console vos cœurs, et vous affermisse en toute bonne parole, et en toute bonne œuvre.
\Chap{3}
\VerseOne{}Au reste, mes frères, priez pour nous, afin que la parole du Seigneur poursuive sa course, et qu'elle soit glorifiée comme elle l'est parmi vous,
\VS{2}et que nous soyons délivrés des hommes méchants et pervers, car tous n'ont pas la foi.
\VS{3}Le Seigneur est fidèle, il vous affermira et vous gardera du mal.
\VS{4}Nous avons à votre égard cette confiance dans le Seigneur, que vous faites et que vous ferez les choses que nous recommandons.
\VS{5}Que le Seigneur veuille diriger vos cœurs vers l'amour de Dieu et vers l'attente de Christ~!
\TextTitle{Se séparer des mauvaises compagnies~; être un modèle~; subvenir à ses besoins}
\VS{6}Nous vous recommandons aussi, mes frères, au Nom de notre Seigneur Jésus-Christ, de vous éloigner\FTNT{La séparation d'avec la mauvaise compagnie. Voir 1 Co. 5:9-13, 15:33~; 2 Co. 6:14-18~; Ro. 16:17-18~; Tit. 3:10-11~; 2 Jn. 2-11.} de tout homme qui se dit frère, et qui vit d'une manière déréglée, et non selon les enseignements qu'il a reçus de nous.
\VS{7}Car vous savez vous-mêmes comment il faut nous imiter, puisque nous n'avons pas marché dans le désordre parmi vous,
\VS{8}et nous n'avons mangé gratuitement le pain de personne. Mais dans le labeur et dans la peine, nous avons travaillé nuit et jour, pour n'être à la charge\FTNT{Les véritables ouvriers de Dieu ne s'attendent pas aux hommes pour avoir leur salaire. Ils mettent leur confiance en Dieu qui est leur rémunérateur. Voir Ac. 20:33-35.} d'aucun de vous.
\VS{9}Ce n'est pas que nous n'en ayons pas le droit, mais c'est pour donner en nous-mêmes un modèle à imiter.
\VS{10}Car lorsque nous étions avec vous, nous vous déclarions expressément que si quelqu'un ne veut pas travailler, qu'il ne mange pas non plus.
\VS{11}Car nous apprenons qu'il y en a quelques-uns parmi vous qui marchent dans le désordre, qui ne travaillent pas, mais qui s'occupent de futilités.
\VS{12}C'est pourquoi nous recommandons donc à ces gens-là et nous les exhortons par notre Seigneur Jésus-Christ, à manger leur propre pain en travaillant paisiblement.
\VS{13}Mais pour vous, mes frères, ne vous lassez pas de faire le bien.
\VS{14}Et si quelqu'un n'obéit pas à ce que nous vous disons par cette épître, faites-le connaître, et n'ayez pas de relation avec lui, afin qu'il éprouve de la honte.
\VS{15}Toutefois, ne le regardez pas comme un ennemi, mais avertissez-le comme un frère.
\TextTitle{Conclusion}
\VS{16}Que le Seigneur de paix vous donne toujours la paix en tout temps~! Que le Seigneur soit avec vous tous~!
\VS{17}La salutation est de ma propre main, de moi Paul, c'est là ma signature dans toutes mes épîtres, c'est ainsi que j'écris.
\VS{18}Que la grâce de notre Seigneur Jésus-Christ soit avec vous tous~! Amen~!
\PPE{}
\end{multicols}

%\clearpage\ShortTitle{1 Corinthiens}\BookTitle{1 Corinthiens}\BFont
\noindent\hrulefill
{\footnotesize
\textit{
\bigskip
{\centering{}
\\Auteur : Paul
\\Thème : Le comportement du chrétien
\\Date de rédaction : Env. 56 ap. J.-C.\\}
}
%\bigskip
\textit{
\\Dans l’antiquité, Corinthe, capitale de l’Achaïe, était la ville la plus prospère et la plus puissante de Grèce. Située sur
un isthme séparant la mer Egée de la mer Ionienne, Corinthe était au carrefour de l’Asie et de l’Italie et constituait un  véritable centre commercial où les produits orientaux et occidentaux se croisaient.
%\bigskip
\\L’apôtre Paul arriva à Corinthe en 51, sous le règne de l’empereur romain Claude (10 av. J.-C. – 54 apr. J.-C.), et y demeura 18 mois. Il trouva une ville riche en pleine expansion, une population parlant diverses langues et rendant des cultes à une multitude de divinités. Rédigée au terme des trois ans passés à Ephèse, la première épître de Paul aux Corinthiens répond à une lettre dans laquelle ceux-ci s’interrogeaient sur le mariage et sur les aliments consacrés aux idoles. Ce fut aussi l’occasion pour lui de procéder à la correction de cette jeune église dont l’état charnel constituait un frein à l’avancée spirituelle. Les Corinthiens avaient en effet confondu le culte raisonnable et les pratiques liées aux cultes à mystères.\bigskip
}
}
\par\nobreak\noindent\hrulefill
\begin{multicols}{2}
\Chap{1}
\TextTitle{La grâce de Christ manifeste dans la vie des saints\FTNTT{Ro. 5:1-2 ; Ep. 1:3-14}}
\VerseOne{}Paul, appelé à être apôtre de Jésus-Christ, par la volonté de Dieu, et le frère Sosthène,
\VS{2}à l'église de Dieu qui est à Corinthe, aux sanctifiés en Jésus-Christ, appelés à être saints, et à tous ceux qui en quelque lieu que ce soit invoquent le Nom de notre Seigneur Jésus-Christ, leur Seigneur et le nôtre.
\VS{3}Que la grâce et la paix vous soient données de la part de Dieu notre Père et du Seigneur Jésus-Christ.
\VS{4}Je rends toujours grâces à mon Dieu à votre sujet, pour la grâce de Dieu qui vous a été donnée en Jésus-Christ.
\VS{5}Car en lui vous avez été enrichis de toutes les richesses qui concernent la parole et la connaissance,
\VS{6}selon que le témoignage de Jésus-Christ a été confirmé en vous,
\VS{7}de sorte qu'il ne vous manque aucun don, pendant que vous attendez la manifestation de notre Seigneur Jésus-Christ.
\VS{8}Qui vous affermira aussi jusqu’à la fin pour que vous soyez irrépréhensibles au jour de notre Seigneur Jésus-Christ.
\VS{9}Dieu qui vous a appelés à la communion de son Fils Jésus-Christ notre Seigneur est fidèle.
\TextTitle{Les rivalités, causes de divisions}
\VS{10}Je vous prie, mes frères, par le Nom de notre Seigneur Jésus-Christ, à tenir tous un même langage, et à ne point avoir de divisions parmi vous, mais à être parfaitement unis dans une même pensée et dans un même jugement.
\VS{11}Car mes frères, j’ai été informé par ceux de la maison de Chloé qu'il y a des dissensions parmi vous.
\VS{12}Je veux dire que chacun de vous parle ainsi : Moi je suis de Paul ! Et moi d'Apollos ! Et moi de Céphas ! Et moi de Christ !
\VS{13}Christ est-il divisé ? Paul a-t-il été crucifié pour vous ? Ou avez-vous été baptisés au nom de Paul ?
\VS{14}Je rends grâces à Dieu de ce que je n'ai baptisé aucun de vous, sinon Crispus et Gaïus,
\VS{15}afin que personne ne dise que j'ai baptisé en mon nom.
\VS{16}J'ai bien aussi baptisé la famille de Stéphanas ; du reste, je ne sais pas si j'ai baptisé quelque autre.
\VS{17}Car Christ ne m'a pas envoyé pour baptiser, mais pour évangéliser, non pas avec des discours de la sagesse humaine, afin que la croix de Christ ne soit pas anéantie.
\TextTitle{La sagesse de Dieu à la croix, dépasse l'entendement humain}
\VS{18}Car la prédication de la croix est une folie pour ceux qui périssent, mais pour nous qui sommes sauvés, elle est la puissance de Dieu.
\VS{19}Car il est écrit : Je détruirai la sagesse des sages et j'anéantirai l'intelligence des hommes intelligents\FTNT{Es. 29:14.}.
\VS{20}Où est le sage ? Où est le scribe ? Où est le disputeur de ce siècle ? Dieu n'a-t-il pas convaincu de folie la sagesse de ce monde ?
\VS{21}Puisque le monde, avec sa sagesse, n’a pas connu Dieu, dans la sagesse de Dieu, il a plu à Dieu de sauver les croyants par la folie de la prédication.
\VS{22}Les Juifs demandent des miracles et les Grecs cherchent la sagesse,
\VS{23}mais pour nous, nous prêchons Christ crucifié, scandale pour les Juifs, et folie pour les Grecs,
\VS{24}à ceux qui sont appelés, tant Juifs que Grecs, nous leur prêchons Christ, la puissance de Dieu et la sagesse de Dieu.
\VS{25}Parce que la folie de Dieu est plus sage que les hommes, et la faiblesse de Dieu est plus forte que les hommes.
\TextTitle{Dieu se sert des choses viles pour confondre le monde et sa sagesse}
\VS{26}Considérez, mes frères, que parmi vous qui avez été appelés, il n’y a pas beaucoup de sages selon la chair, ni beaucoup de puissants, ni beaucoup de nobles.
\VS{27}Mais Dieu a choisi les choses folles de ce monde pour confondre les sages ; et Dieu a choisi les choses faibles de ce monde pour confondre les fortes ;
\VS{28}et Dieu a choisi les choses viles de ce monde et les méprisées, même celles qui ne sont point, pour réduire à néant celles qui sont,
\VS{29}afin que nulle chair ne se glorifie devant lui.
\VS{30}Or c'est par lui que vous êtes en Jésus-Christ, lequel, de par Dieu, a été fait pour nous sagesse, justice, sanctification et rédemption ;
\VS{31}afin que comme il est écrit, celui qui se glorifie se glorifie dans le Seigneur\FTNT{Jé. 9:24.}.
\Chap{2}
\TextTitle{La foi en Dieu ne se base pas sur la sagesse humaine}
\VerseOne{}Pour moi donc, mes frères, lorsque je suis allé chez vous, ce n’est pas avec des discours pompeux, remplis de la sagesse humaine, que je suis allé vous annoncer le témoignage de Dieu.
\VS{2}Car je n’ai pas eu la pensée de savoir parmi vous autre chose que Jésus-Christ et Jésus-Christ crucifié.
\VS{3}Et j'ai même été parmi vous dans la faiblesse, dans la crainte, et dans un grand tremblement.
\VS{4}Et ma parole et ma prédication ne reposaient pas sur les discours persuasifs de la sagesse humaine, mais sur une démonstration d'Esprit et de puissance ;
\VS{5}afin que votre foi ne soit pas fondée sur la sagesse des hommes, mais sur la puissance de Dieu.
\VS{6}Cependant, nous prêchons une sagesse parmi les parfaits, une sagesse, dis-je, qui n'est pas de ce monde, ni des chefs de ce siècle, qui vont être anéantis.
\VS{7}Mais nous prêchons la sagesse de Dieu, qui est un mystère, c'est-à-dire cachée, que Dieu avant les siècles, avait prédestinée pour notre gloire,
\VS{8}sagesse qu’aucun des chefs de ce siècle n'a connue, car s'ils l’avaient connue, ils n’auraient pas crucifié le Seigneur de gloire.
\TextTitle{C'est l'Esprit de Dieu qui revèle les profondeurs de Dieu}
\VS{9}Mais comme il est écrit : Ce sont des choses que l’œil n'a point vues, que l'oreille n'a point entendues, et qui ne sont point montées au cœur de l'homme, des choses que Dieu a préparées pour ceux qui l’aiment\FTNT{Es. 64:4.}.
\VS{10}Mais Dieu nous les a révélées par son Esprit. Car l'Esprit sonde toutes choses, même les choses profondes de Dieu.
\VS{11}Qui donc, parmi les hommes, connaît les choses de l'homme, sinon l’esprit de l'homme qui est en lui ? De même aussi, personne ne connaît les choses de Dieu, si ce n’est l'Esprit de Dieu.
\VS{12}Or nous, nous n’avons pas reçu l'esprit de ce monde, mais l'Esprit qui vient de Dieu, afin que nous connaissions les choses qui nous ont été données de Dieu.
\TextTitle{La sagesse humaine n'accepte pas les choses de l'Esprit}
\VS{13}Et nous en parlons, non avec des discours que la sagesse humaine enseigne, mais avec celle qu'enseigne le Saint-Esprit, communiquant des choses spirituelles à ceux qui sont spirituels.
\VS{14}Mais l'homme animal\FTNT{L’homme animal (ou naturel) est un homme incrédule. C’est un homme  non-régénéré, ayant le principe de la vie animale, c’est-à-dire ce que les hommes ont en commun avec les brutes. Sa nature sensuelle est sujette aux appétits et aux passions (Jud. 1:19).} ne comprend pas les choses de l'Esprit de Dieu, car elles sont une folie pour lui ; et il ne peut même pas les entendre, parce c’est spirituellement qu’on en juge.
\VS{15}Mais l'homme spirituel\FTNT{L’homme spirituel est un homme dont l’esprit est régénéré et qui marche par l’Esprit. Il a la pensée de Christ et porte les fruits de l’Esprit.} juge de tout et il n'est jugé par personne.
\VS{16}Car qui a connu la pensée du Seigneur pour pouvoir l’instruire\FTNT{Es. 40:13.} ? Mais nous, nous avons la pensée de Christ.
\Chap{3}
\TextTitle{Les œuvres de la chair nuisent à la croissance chrétienne}
\VerseOne{}Pour moi, mes frères, je n'ai pas pu vous parler comme à des hommes spirituels, mais comme à des hommes charnels\FTNT{L’homme charnel est gouverné par la nature humaine et non par l'Esprit de Dieu (Ga. 5:16-21). L’homme charnel est un enfant en Christ, littéralement «~ignorant~» (Ga. 4:1). Il est comparé à un esclave.}, c'est-à-dire comme à des enfants en Christ.
\VS{2}Je vous ai donné du lait à boire, et non pas de la viande, parce que vous ne pouviez pas la supporter ; et même maintenant vous ne le pouvez pas encore, parce que vous êtes encore charnels.
\VS{3}Car puisqu'il y a parmi vous de la jalousie, des disputes, et des divisions, n'êtes-vous pas charnels, et ne vous conduisez-vous pas à la manière des hommes ?
\VS{4}Car quand l'un dit : Moi je suis de Paul ; et l'autre : Moi je suis d'Apollos, n'êtes-vous pas charnels ?
\TextTitle{Dieu est le maître de tout}
\VS{5}Qu’est-ce donc Paul, et qui est Apollos ? Des ministres, par le moyen desquels vous avez cru, selon que le Seigneur l’a donné à chacun.
\VS{6}J'ai planté, Apollos a arrosé, mais c'est Dieu qui a donné l'accroissement,
\VS{7}en sorte que ce n’est pas celui qui plante qui est quelque chose, ni celui qui arrose, mais Dieu qui donne l'accroissement.
\VS{8}Celui qui plante et celui qui arrose sont égaux, et chacun recevra sa récompense selon son propre travail.
\VS{9}Car nous sommes ouvriers avec Dieu. Vous êtes le champ de Dieu et l'édifice de Dieu.
\VS{10}Selon la grâce de Dieu qui m'a été donnée, j'ai posé le fondement comme un sage architecte, et un autre édifie dessus. Mais que chacun prenne garde comment il édifie dessus.
\TextTitle{Le seul fondement : Jésus-Christ}
\VS{11}Car personne ne peut poser un autre fondement que celui qui a été posé, à savoir Jésus-Christ.
\TextTitle{Deux types de construction}
\VS{12}Si quelqu'un édifie sur ce fondement avec de l'or, de l'argent, des pierres précieuses, du bois, du foin, du chaume, l’œuvre de chacun sera manifestée ;
\VS{13}car le jour la fera connaître, parce qu'elle sera manifestée par le feu ; et le feu éprouvera ce qu’est l’œuvre de chacun.
\VS{14}Si l’œuvre édifiée par quelqu’un sur le fondement subsiste, il recevra la récompense.
\VS{15}Si l’œuvre de quelqu'un est consumée, il perdra sa récompense ; mais pour lui, il sera sauvé, toutefois comme au travers du feu.
\VS{16}Ne savez-vous pas que vous êtes le temple\FTNT{Le temple de Dieu. Beaucoup veulent construire des bâtiments qu’ils appellent «~temples ou maisons de Dieu~» alors que chaque chrétien est le temple de Dieu. Voir Es. 66:1 ; Ac. 17:24 ; 1 Co. 6:19.} de Dieu et que l’Esprit de Dieu habite en vous ?
\VS{17}Si quelqu'un détruit le temple de Dieu, Dieu le détruira ; car le temple de Dieu est saint, et vous êtes ce temple.
\VS{18}Que personne ne s'abuse lui-même : Si quelqu'un d'entre vous croit être sage selon ce monde, qu'il devienne fou, afin de devenir sage.
\VS{19}Parce que la sagesse de ce monde est une folie devant Dieu ; car il est écrit : Il surprend les sages dans leur ruse\FTNT{Job 5:13.}.
\VS{20}Et encore : Le Seigneur connaît les pensées des sages, il sait qu’elles sont vaines\FTNT{Ps. 94:11.}.
\VS{21}Que personne donc ne mette sa gloire dans les hommes, car toutes choses sont à vous,
\VS{22}soit Paul, soit Apollos, soit Céphas, soit le monde, soit la vie, soit la mort, soit les choses présentes, soit les choses à venir, toutes choses sont à vous,
\VS{23}et vous à Christ, et Christ à Dieu.
\Chap{4}
\TextTitle{Le Seigneur est le seul véritable juge}
\VerseOne{}Que chacun nous regarde comme des serviteurs de Christ et des dispensateurs des mystères de Dieu.
\VS{2}Du reste, il est exigé des dispensateurs que chacun soit trouvé fidèle.
\VS{3}Pour moi, il m’importe fort peu d'être jugé par vous, ou par un jugement d'homme. Je ne me juge pas non plus moi-même, car je ne me sens coupable de rien,
\VS{4}mais ce n’est pas pour cela que je suis justifié. Celui qui me juge, c'est le Seigneur.
\VS{5}C'est pourquoi ne jugez de rien avant le temps, jusqu'à ce que le Seigneur vienne, alors il mettra en lumière les choses cachées dans les ténèbres et manifestera les desseins des cœurs. Alors chacun recevra de Dieu la louange qui lui sera due.
\VS{6}Or mes frères, j’ai fait de ces choses une application à ma personne et à celle d’Apollos, à cause de vous ; afin que vous appreniez de nous à ne point aller au-delà de ce qui est écrit, et que nul de vous ne conçoive de l’orgueil en faveur de l’un contre l’autre.
\VS{7}Car qui est-ce qui met de la différence entre toi et un autre ? Qu’as-tu que tu n’aies reçu ? Et si tu l'as reçu, pourquoi te glorifies-tu comme si tu ne l'avais pas reçu\FTNT{Les diverses grâces que Dieu accorde à ses enfants doivent les amener à l’humilité.} ?
\VS{8}Vous êtes déjà rassasiés, vous êtes déjà enrichis, vous êtes devenus rois sans nous. Plaise à Dieu que vous régniez en effet, afin que nous aussi nous régnions avec vous !
\TextTitle{L'humilité et la patience}
\VS{9}Car je pense que Dieu nous a exposés publiquement, nous qui sommes les derniers des apôtres, comme des gens condamnés à la mort, puisque nous avons été en spectacle au monde, aux anges et aux hommes.
\VS{10}Nous sommes fous pour l'amour de Christ, mais vous êtes sages en Christ ; nous sommes faibles, et vous êtes forts ; vous êtes dans l'estime, et nous sommes dans le mépris.
\VS{11}Jusqu'à cette heure, nous souffrons la faim, la soif, la nudité ; on nous frappe au visage, et nous sommes errants çà et là ;
\VS{12}nous nous fatiguons à travailler de nos propres mains ; on dit du mal de nous, et nous bénissons ; nous sommes persécutés, et nous le supportons.
\VS{13}Nous sommes calomniés, et nous prions ; nous sommes devenus comme les balayures du monde, comme le rebut de tous, jusqu'à maintenant.
\VS{14}Je n'écris pas ces choses pour vous faire honte, mais je vous avertis comme mes chers enfants.
\VS{15}Car même si vous aviez dix mille maîtres en Christ, vous n'avez pourtant pas plusieurs pères, car c'est moi qui vous ai engendrés en Jésus-Christ par l'Evangile.
\VS{16}Je vous prie donc d'être mes imitateurs.
\VS{17}C'est pour cela que je vous ai envoyé Timothée, qui est mon fils bien-aimé, et qui est fidèle dans le Seigneur, afin qu'il vous rappelle quelles sont mes voies en Christ et comment j'enseigne partout dans toutes les églises.
\TextTitle{L'autorité de Paul}
\VS{18}Quelques-uns se sont enflés d’orgueil comme si je ne devais pas aller chez vous.
\VS{19}Mais j'irai bientôt chez vous, si le Seigneur le veut ; et je connaîtrai non les paroles, mais la puissance de ceux qui se sont glorifiés.
\VS{20}Car le Royaume de Dieu ne consiste pas en paroles, mais en puissance.
\VS{21}Que voulez-vous ? Que j’aille chez vous avec la verge, ou avec charité et dans un esprit de douceur ?
\Chap{5}
\TextTitle{L'inceste à Corinthe}
\VerseOne{}On entend dire de toutes parts qu'il y a parmi vous de l’impudicité, et une impudicité telle qu’elle ne se rencontre même pas chez les gentils ; c'est au point où l’un de vous a la femme de son père\FTNT{L’inceste est interdit par la loi (Lé. 18:6-8).}.
\TextTitle{Oter le mal dans l'Eglise}
\VS{2}Et vous êtes enflés d'orgueil ! Et vous n'avez pas été plutôt dans le deuil, afin que celui qui a commis cette action soit retranché du milieu de vous.
\VS{3}Pour moi, étant absent de corps, mais présent en esprit, j'ai déjà jugé comme si j'étais présent, celui qui a commis une telle action.
\VS{4}Vous et mon esprit étant assemblés au nom de notre Seigneur Jésus-Christ, j'ai ordonné, avec la puissance de notre Seigneur Jésus-Christ,
\VS{5}qu'un tel homme soit livré à Satan\FTNT{Cette déclaration de Paul peut paraître choquante pour certains, mais elle nous rappelle l'histoire de Job, qui fut mis à l'épreuve par Yahweh qui l'avait livré à Satan (Job. 1:12). Paul espérait ainsi amener cet homme à la repentance en l’excluant de l’assemblée.} pour la destruction de la chair, afin que l'esprit soit sauvé au jour du Seigneur Jésus.
\VS{6}Votre vanité est mal fondée. Ne savez-vous pas qu'un peu de levain\FTNT{Le levain fait gonfler ou enfler. Il symbolise la cause principale de nombreux péchés : l'orgueil. Dans la Bible, le levain représente aussi des péchés spirituellement destructeurs comme la malice, la méchanceté, l'hypocrisie et les faux enseignements (Lu. 12:1, Mt. 16:11-12).} fait lever toute la pâte ?
\VS{7}Otez donc le vieux levain, afin que vous soyez une nouvelle pâte, puisque vous êtes sans levain ; car Christ, notre Pâque\FTNT{Ex. 12.}, a été sacrifié pour nous.
\VS{8}C'est pourquoi célébrons donc la fête, non avec du vieux levain, non avec un levain de méchanceté et de malice, mais avec les pains sans levain de la sincérité et de la vérité.
\TextTitle{Le disciple du Seigneur ne doit pas fréquenter les faux frères}
\VS{9}Je vous ai écrit dans ma lettre de ne pas vous mêlez\FTNT{Mêler vient du grec «~sunanamignumi~» qui signifie : «~mêler ensemble, se tenir en compagnie avec, être intime avec quelqu'un. Avoir des relations, être en communication~» (Ps. 1:1 ; Ro. 16:17-18 ; 1 Co. 15:33 ; Tit. 3:10).} avec les fornicateurs,
\VS{10}non pas d’une manière absolue avec les fornicateurs de ce monde, ou avec les cupides, ou les ravisseurs, ou les idolâtres ; autrement, il vous faudrait sortir du monde.
\VS{11}Maintenant, ce que je vous ai écrit, c’est de ne pas avoir de relations avec quelqu’un qui, se nommant frère, est fornicateur, ou cupide, ou idolâtre, ou médisant, ou ivrogne, ou ravisseur, de ne même pas manger avec un tel homme.
\VS{12}Car qu'ai-je à juger ceux qui sont dehors ? N’est-ce pas ceux du dedans que vous avez à juger ?
\VS{13}Mais Dieu juge ceux qui sont du dehors. Otez donc le méchant du milieu de vous.
\Chap{6}
\TextTitle{Procès entre chrétiens ou face aux non croyants}
\VerseOne{}Quand quelqu'un d'entre vous a une affaire contre un autre, ose-t-il bien aller en jugement devant les injustes, et il ne va pas devant les saints ?
\VS{2}Ne savez-vous pas que les saints jugeront le monde\FTNT{L’Eglise jugera les nations. Les douze apôtres jugeront Israël (Mt. 19:28 ; Lu. 22:30).} ? Or si le monde doit être jugé par vous, êtes-vous indignes de rendre les moindres jugements ?
\VS{3}Ne savez-vous pas que nous jugerons les anges\FTNT{Le mot ange vient du grec «~aggelos~» et veut dire «~messager, envoyé, ange~». Ce terme s’applique donc aussi bien aux hommes qu’aux créatures spirituelles.} ? Et à plus forte raison les choses de cette vie ?
\VS{4}Si donc vous avez des procès pour les affaires de cette vie, prenez pour juge ceux qui sont des moins estimés dans l'Eglise !
\VS{5}Je le dis à votre honte. Ainsi il n’y a parmi vous pas un seul homme sage qui puisse prononcer un jugement entre frères.
\VS{6}Mais un frère a des procès contre son frère, et cela devant les infidèles.
\VS{7}C'est déjà un grand défaut chez vous que vous ayez des procès entre vous. Pourquoi ne souffrez-vous pas plutôt quelque injustice ? Pourquoi ne vous laissez-vous pas plutôt dépouiller ?
\VS{8}Mais c’est vous qui commettez l’injustice et qui dépouillez, et c’est envers des frères que vous agissez de la sorte !
\TextTitle{Le chrétien est sanctifié, lavé et justifié}
\VS{9}Ne savez-vous pas que les injustes n'hériteront point le Royaume de Dieu ? Ne vous y trompez pas : Ni les fornicateurs, ni les idolâtres, ni les adultères,
\VS{10}ni les efféminés, ni les homosexuels, ni les voleurs, ni les avares, ni les ivrognes, ni les médisants, ni les ravisseurs, n'hériteront le Royaume de Dieu.
\VS{11}Et c’est là ce que vous étiez ; mais vous avez été lavés, mais vous avez été sanctifiés, mais vous avez été justifiés au nom du Seigneur Jésus, et par l'Esprit de notre Dieu.
\VS{12}Tout m’est permis, mais tout n’est pas utile ; tout m’est permis, mais je ne me rendrai esclave d’aucune chose.
\TextTitle{Le chrétien appartient au Seigneur}
\VS{13}Les aliments sont pour le ventre, et le ventre pour les aliments ; et Dieu détruira l'un comme les autres. Or le corps n'est point pour la fornication, mais pour le Seigneur, et le Seigneur pour le corps.
\VS{14}Et Dieu qui a ressuscité le Seigneur, nous ressuscitera aussi par sa puissance.
\VS{15}Ne savez-vous pas que vos corps sont les membres de Christ ? Prendrai-je donc les membres de Christ pour en faire les membres d'une prostituée ? Loin de là !
\VS{16}Ne savez-vous pas que celui qui s'unit à la prostituée devient un même corps avec elle ? Car il est dit : Les deux deviendront une même chair\FTNT{Ge. 2:24.}.
\VS{17}Mais celui qui s’unit au Seigneur est avec lui un seul esprit.
\VS{18}Fuyez la fornication. Quelque autre péché qu’un homme commette, ce péché est hors du corps ; mais le fornicateur pèche contre son propre corps.
\TextTitle{Le chrétien est le temple Saint-Esprit}
\VS{19}Ne savez-vous pas que votre corps est le temple du Saint-Esprit qui est en vous, et que vous avez reçu de Dieu, et que vous ne vous appartenez point à vous-mêmes ?
\VS{20}Car vous avez été achetés à un prix ; glorifiez donc Dieu dans votre corps et dans votre esprit, qui appartiennent à Dieu.
\Chap{7}
\TextTitle{La sainteté dans le mariage}
\VerseOne{}Pour ce qui concerne les choses au sujet desquelles vous m'avez écrit : Je vous dis qu'il est bon à l'homme de ne pas se marier.
\VS{2}Toutefois, pour éviter la fornication, que chacun ait sa femme, et que chaque femme ait son mari.
\VS{3}Que le mari rende à sa femme la bienveillance qui lui est due ; et que la femme de même la rende à son mari.
\VS{4}Car la femme n'a pas de pouvoir sur son propre corps, mais c’est son mari. De même, le mari n'a pas de pouvoir sur son propre corps, mais c’est sa femme.
\VS{5}Ne vous privez point l'un de l'autre, si ce n'est par un consentement mutuel, pour un temps, afin que vous vaquiez au jeûne et à la prière, mais après cela retournez ensemble, de peur que Satan ne vous tente par votre manque de contrôle.
\VS{6}Or je dis ceci par conseil, et non par commandement.
\VS{7}Car je voudrais que tous les hommes soient comme moi ; mais chacun a reçu de Dieu un don particulier, l'un d’une manière, l’autre d’une autre.
\VS{8}A ceux qui ne sont pas mariés, et aux veuves, je dis qu'il leur est bon de demeurer comme moi.
\VS{9}Mais s'ils manquent de maîtrise, qu'ils se marient ; car il vaut mieux se marier que de brûler.
\TextTitle{Recommandations à ceux qui sont mariés}
\VS{10}Et quant à ceux qui sont mariés, je leur ordonne, non pas moi, mais le Seigneur, que la femme ne se sépare point de son mari.
\VS{11}Et si elle s'en sépare, qu'elle demeure sans être mariée, ou qu'elle se réconcilie avec son mari ; que le mari aussi ne quitte point sa femme.
\VS{12}Mais aux autres je leur dis, et non pas le Seigneur : Si un frère a une femme incrédule et qu'elle consente d'habiter avec lui, qu'il ne la quitte point.
\VS{13}Et si une femme a un mari incrédule et qu'il consente d'habiter avec elle, qu'elle ne le quitte point.
\VS{14}Car le mari incrédule est sanctifié par la femme, et la femme incrédule est sanctifiée par le mari ; autrement vos enfants seraient impurs, or maintenant ils sont saints.
\VS{15}Que si l'incrédule se sépare, qu'il se sépare ; le frère ou la sœur ne sont point liés dans ce cas-là, car Dieu nous a appelés à la paix.
\VS{16}Car sais-tu, femme, si tu sauveras ton mari ? Ou que sais-tu, mari, si tu sauveras ta femme ?
\TextTitle{La circoncision et l'incirconcision ne sont rien, Dieu est tout}
\VS{17}Toutefois, que chacun marche selon le don qu'il a reçu de Dieu, chacun selon l’appel qu’il a reçu du Seigneur. C’est ainsi que je l’ordonne dans toutes les églises.
\VS{18}Quelqu'un a-t-il été appelé étant circoncis ? Qu’il ne redevienne pas incirconcis\FTNT{Vient du grec «~Epispaomai~» qui a pour définition : Ne pas devenir incirconcis. Aux jours  d'Antiochus IV, dit aussi Antioche Epiphane (voir commentaire en Da.8:9), certains Juifs, voulant échapper aux persécutions, cachaient le signe de leur nationalité, la circoncision, en se faisant reproduire artificiellement le prépuce par une opération chirurgicale qui étendait la peau restante.}. Quelqu'un a-t-il été appelé incirconcis ? Qu’il ne se fasse pas circoncire.
\VS{19}La circoncision n'est rien, et l’incirconcision aussi n'est rien, mais l'observation des commandements de Dieu est tout.
\VS{20}Que chacun demeure dans la condition où il était quand il a été appelé.
\VS{21}As-tu été appelé étant esclave ? Ne t'en inquiète pas ; mais si tu peux être mis en liberté, profites-en plutôt.
\VS{22}Car l’esclave qui a été appelé par notre Seigneur est un affranchi du Seigneur ; de même, celui qui est appelé étant libre, est un esclave de Christ.
\VS{23}Vous avez été rachetés à un prix, ne devenez pas les esclaves des hommes.
\VS{24}Mes frères, que chacun demeure devant Dieu dans l'état où il était quand il a été appelé.
\TextTitle{Conseils de Paul aux célibataires}
\VS{25}Pour ce qui concerne les vierges, je n'ai point de commandement du Seigneur, mais je donne un avis comme ayant obtenu miséricorde du Seigneur pour être fidèle.
\VS{26}Voici donc ce que j'estime bon, à cause des afflictions présentes : Il est avantageux à chacun de demeurer comme il est.
\VS{27}Es-tu lié à une femme ? Ne cherche pas à rompre ce lien. N’es-tu pas lié à une femme ? Ne cherche point de femme.
\VS{28}Si tu te maries, tu ne pèches point ; et si la vierge se marie, elle ne pèche point aussi ; mais ceux qui sont mariés auront des afflictions dans la chair ; or je voudrais vous les épargner.
\VS{29}Mais je vous dis ceci, mes frères : Le temps est court, que désormais ceux qui ont une femme soient comme n’en ayant pas ;
\VS{30}ceux qui pleurent comme ne pleurant pas, ceux qui se réjouissent comme ne se réjouissant pas, ceux qui achètent comme ne possédant pas,
\VS{31}et ceux qui usent de ce monde comme n'en usant pas, car la figure de ce monde passe.
\VS{32}Or je voudrais que vous soyez sans inquiétude. Celui qui n'est pas marié s’occupe des choses du Seigneur, cherchant à plaire au Seigneur.
\VS{33}Mais celui qui est marié s’occupe des choses de ce monde, cherchant à plaire à sa femme, et ainsi il est divisé.
\VS{34}Il y a de même une différence entre la femme mariée et la vierge : Celle qui n’est pas mariée s’occupe des choses du Seigneur, afin d’être sainte de corps et d'esprit ; mais celle qui est mariée s’occupe des choses du monde pour plaire à son mari.
\VS{35}Je dis cela dans votre intérêt, ce n’est pas pour vous tendre un piège, mais pour vous porter à ce qui est bienséant et propre à vous unir au Seigneur sans aucune distraction.
\VS{36}Mais si quelqu'un croit qu’il n’est pas honorable que sa fille dépasse la fleur de l’âge sans être mariée, et qu’il faille la marier, qu'il fasse ce qu'il veut, il ne pèche point ; qu'elle soit mariée.
\VS{37}Mais celui qui a pris une ferme résolution, sans contrainte, et avec l’exercice de sa propre volonté en son cœur, de garder sa fille vierge, celui-là fait bien.
\VS{38}Celui donc qui la marie fait bien, mais celui qui ne la marie pas fait mieux.
\VS{39}La femme est liée par la loi pendant tout le temps que son mari est en vie\FTNT{Dieu est contre le divorce. Pour le Seigneur, le mariage doit être un engagement à vie (Mal. 2:16 ; Ro. 7:1-3).}, mais si son mari meurt, elle est libre de se marier à qui elle veut ; seulement, que ce soit dans le Seigneur.
\VS{40}Elle est néanmoins plus heureuse si elle demeure ainsi, selon mon avis ; or j'estime que j'ai aussi l'Esprit de Dieu.
\Chap{8}
\TextTitle{Viandes sacrifiées aux idoles et les limites de la liberté chrétienne}
\VerseOne{}Pour ce qui concerne les choses qui sont sacrifiées aux idoles\FTNT{A Corinthe, on offrait rituellement des viandes sacrifiées aux idoles. A ces occasions, certaines parties des animaux sacrifiés étaient déposées sur l’autel de l’idole, d’autres étaient données aux prêtres et aux adorateurs, qui les mangeaient lors d’un repas ou d’un festin, soit dans le temple, soit dans une maison particulière. Certains morceaux de la chair offerte aux idoles étaient ensuite apportés au marché pour être vendus (Da. 1).}, nous savons que nous avons tous de la connaissance. La connaissance enfle, mais la charité édifie.
\VS{2}Et si quelqu'un croit savoir quelque chose, il n'a encore rien connu comme il faut connaître.
\VS{3}Mais si quelqu'un aime Dieu, il est connu de lui.
\VS{4}Pour ce qui est donc de manger des choses sacrifiées aux idoles, nous savons que l'idole n'est rien dans le monde et qu'il n'y a aucun autre Dieu qu’un seul\FTNT{Paul affirme avec force que le Dieu Créateur n’est pas mélangé avec d’autres divinités. Voir Dt. 6:4.}.
\VS{5}Car s’il est des êtres qui sont appelés dieux, soit dans le ciel, soit sur la terre, comme il existe réellement plusieurs dieux, et plusieurs seigneurs,
\VS{6}nous n’avons pourtant qu'un seul Dieu, qui est le Père, de qui viennent toutes choses, et pour qui nous sommes ; et un seul Seigneur : Jésus-Christ, par qui sont toutes choses, et par qui nous sommes.
\VS{7}Mais tous n’ont pas cette connaissance. Car quelques-uns, d’après la manière dont ils envisagent encore l'idole, mangent de ces choses comme étant sacrifiées aux idoles, et leur conscience qui est faible en est souillée.
\VS{8}Ce n’est pas une viande qui nous rend agréables à Dieu ; car si nous en mangeons, nous n'avons rien de plus ; si nous n’en mangeons pas, nous n’avons rien de moins.
\VS{9}Mais prenez garde que cette liberté que vous avez ne soit en quelque sorte un scandale pour les faibles.
\VS{10}Car si quelqu'un te voit, toi qui as de la connaissance, être à table dans le temple des idoles, sa conscience, à lui qui est faible, ne le portera-t-elle pas à manger des choses sacrifiées aux idoles ?
\VS{11}Et ainsi ton frère, qui est faible, et pour lequel Christ est mort, périra par ta connaissance.
\VS{12}Or quand vous péchez ainsi contre vos frères, et que vous blessez leur conscience qui est faible, vous péchez contre Christ.
\VS{13}C'est pourquoi, si la viande scandalise mon frère, je ne mangerai jamais de chair pour ne point scandaliser mon frère.
\Chap{9}
\TextTitle{Paul défend son apostolat\FTNTT{Ga. 1:11 ; 2:21}}
\VerseOne{}Ne suis-je pas apôtre ? Ne suis-je pas libre ? N’ai-je pas vu notre Seigneur Jésus-Christ ? N’êtes-vous pas mon ouvrage dans le Seigneur ?
\VS{2}Si je ne suis pas apôtre pour les autres, je le suis au moins pour vous, car vous êtes le sceau de mon apostolat dans le Seigneur.
\VS{3}C'est là ma défense contre ceux qui me condamnent.
\VS{4}N'avons-nous pas le droit de manger et de boire ?
\VS{5}N'avons-nous pas le droit de mener avec nous une sœur qui soit notre femme, comme font les autres apôtres, et les frères du Seigneur, et Céphas ?
\VS{6}N'y a-t-il que Barnabas et moi qui n'ayons pas le droit de ne pas travailler ?
\TextTitle{Dieu prend soin de ses serviteurs}
\VS{7}Qui est-ce qui va à la guerre à ses propres frais ? Qui est-ce qui plante une vigne et n’en mange pas le fruit ? Qui est-ce qui fait paître un troupeau et ne se nourrit pas du lait du troupeau ?
\VS{8}Ces choses que je dis n’existent-elles que dans la coutume des hommes ? La loi ne dit-elle pas aussi la même chose ?
\VS{9}Car il est écrit dans la Loi de Moïse : Tu ne muselleras pas le bœuf qui foule le grain\FTNT{De. 25:4.}. Dieu se met-il en peine des bœufs ?
\VS{10}Ou parle-t-il uniquement à cause de nous ? Oui, c’est à cause de nous qu’il a été écrit que celui qui laboure doit labourer avec espérance, et celui qui foule le blé, le foule avec l’espérance d’y avoir part.
\VS{11}Si nous avons semé parmi vous des biens spirituels, est-ce une grosse affaire si nous moissonnons vos biens temporels ?
\VS{12}Si d'autres usent de ce droit à votre égard, pourquoi n'en userions-nous pas plutôt qu'eux ? Cependant nous n'avons point usé de ce droit, mais au contraire, nous supportons toutes sortes d'incommodités, afin de ne pas créer d’obstacle à l'Evangile de Christ.
\VS{13}Ne savez-vous pas que ceux qui font le service sacré mangent des choses sacrées ; et que ceux qui servent à l'autel participent à l'autel\FTNT{No. 18:8-31.} ?
\VS{14}Le Seigneur a ordonné que ceux qui annoncent l'Evangile vivent de l'Evangile.
\VS{15}Pour moi, je n’ai usé d’aucun de ces droits, et ce n’est pas afin de les réclamer en ma faveur que j’écris ainsi ; car j’aimerais mieux mourir que de me laisser enlever cette gloire.
\VS{16}Car si j'évangélise, ce n’est pas pour moi un sujet de gloire, c’est parce que la nécessité m'en est imposée ; et malheur à moi si je n'évangélise pas !
\VS{17}Si je le fais de bon cœur, j’en aurai la récompense ; mais si je le fais malgré moi, c’est une charge qui m’est confiée.
\VS{18}Quelle récompense en ai-je donc ? C’est qu'en prêchant l'Evangile, je prêche l'Evangile de Christ sans qu'il en coûte rien\FTNT{Paul annonçait l’Evangile gratuitement. Donnez gratuitement : C’est la suite logique des choses, on reçoit gratuitement et on donne gratuitement. Si nous sommes comme Christ (car là est le sens du mot disciple), nous devons agir comme lui. Il a donné ses enseignements et nourrit les gens gratuitement. Dans Ap. 21:6 et 22:17, le Seigneur invite toutes les personnes qui ont soif à venir s’abreuver gratuitement. Alors pourquoi vendre la parole c’est-à-dire l’eau qu’on a reçue gratuitement ? Nous devons donner gratuitement.}, afin que je n'abuse pas de mon autorité dans l'Evangile.
\TextTitle{L'attitude d'un vrai serviteur de Dieu}
\VS{19}Car bien que je sois libre à l'égard de tous, je me suis pourtant rendu le serviteur de tous, afin de gagner plus de personnes.
\VS{20}Avec les Juifs, j’ai été comme Juif, afin de gagner les Juifs ; avec ceux qui sont sous la loi, comme si j'étais sous la loi, afin de gagner ceux qui sont sous la loi ;
\VS{21}avec ceux qui sont sans loi, comme si j'étais sans loi (quoique je ne sois point sans la Loi de Dieu, étant sous la Loi de Christ), afin de gagner ceux qui sont sans loi.
\VS{22}J’ai été faible avec les faibles, afin de gagner les faibles ; je me suis fait tout à tous, afin d’en sauver au moins quelques-uns.
\VS{23}Je fais cela à cause de l'Evangile, afin que j'en sois fait aussi participant avec les autres.
\VS{24}Ne savez-vous pas que ceux qui courent dans le stade, courent tous, mais qu’un seul remporte le prix ? Courez de manière à le remporter.
\VS{25}Tout homme qui combat, vit entièrement de régime ; et ces gens-là le font pour obtenir une couronne corruptible\FTNT{Couronne corruptible : Aux Jeux panhelléniques, il n’y avait qu’un seul vainqueur qui remportait pour prix une couronne de feuillage. Sur chacun des sites, les couronnes étaient fabriquées avec des feuillages différents : – A Olympie, c’était une couronne d’olivier sauvage – A Delphes, une couronne de laurier – A l’Isthme (Corinthe), une couronne de pin – A Némée, une couronne de céleri. En plus de sa couronne, l’athlète victorieux recevait un ruban de laine rouge. Des amphores remplies d’huile d’olive étaient également remises au vainqueur. A cette époque, l’huile d’olive était extrêmement précieuse et valait beaucoup d’argent. D’autres prix, comme des trépieds en bronze (grands vases munis de trois pieds), des boucliers en bronze ou des coupes en argent, pouvaient aussi faire partie des lots. La modeste couronne de feuillage était cependant la plus haute récompense attribuée alors dans le monde grec, car elle garantissait l’honneur et le respect de tous à celui qui la recevait.} ; mais nous, faisons-le pour une couronne incorruptible.
\VS{26}Moi donc je cours, non pas comme à l’aventure ; je combats, mais non pas comme battant l'air.
\VS{27}Mais je traite durement mon corps et je le tiens assujetti, de peur d’être moi-même désapprouvé après avoir prêché aux autres.
\Chap{10}
\TextTitle{Paul donne l'exemple d'Israël dans le désert}
\VerseOne{}Mes frères, je ne veux pas que vous ignoriez que nos pères ont tous été sous la nuée, et qu'ils ont tous passé au travers de la mer,
\VS{2}et qu'ils ont tous été baptisés en Moïse dans la nuée et dans la mer ;
\VS{3}et qu'ils ont tous mangé la même viande spirituelle ;
\VS{4}et qu'ils ont tous bu le même breuvage spirituel : Car ils buvaient de l'eau du rocher spirituel qui les suivait, et ce rocher\FTNT{Jésus-Christ, le Rocher des âges. Voir Es. 8:13-17.} était Christ.
\VS{5}Mais la plupart d’entre eux ne furent point agréables à Dieu puisqu’ils périrent dans le désert.
\VS{6}Or ces choses ont été des exemples pour nous, afin que nous ne convoitions point des choses mauvaises, comme eux-mêmes les ont convoitées.
\VS{7}Ne devenez point idolâtres, comme quelques-uns d’entre eux, selon qu'il est écrit : Le peuple s’assit pour manger et pour boire, puis ils se levèrent pour jouer\FTNT{Ex. 32:6.}.
\VS{8}Ne nous livrons pas à la fornication, comme quelques-uns d’entre eux s’y livrèrent, de sorte qu’il en tomba vingt-trois mille en un jour\FTNT{No. 25:9.}.
\VS{9}Ne tentons\FTNT{Tenter : Du grec «~ekpeirazo~» : mettre à l’épreuve ; éprouver le caractère de Dieu et son pouvoir.} point Christ, comme le tentèrent\FTNT{Tenter : Du grec «~peirazo~» : essayer si une chose peut être faite ; éprouver malicieusement, astucieusement, pour prouver ses sentiments et ses jugements ; essayer ou éprouver la foi, la vertu, le caractère par la séduction du péché ; solliciter à pécher ; infliger des maux dans le but d’éprouver. Ce terme est aussi utilisé lorsque les hommes veulent tenter Dieu en montrant leur méfiance, par une conduite impie ou méchante, pour éprouver la justice et la patience de Dieu, et le défier, pour le pousser à donner une preuve de ses perfections.} quelques-uns d’entre eux qui périrent par les serpents\FTNT{No. 21:6-9.}.
\VS{10}Ne murmurez point, comme quelques-uns d’entre eux qui périrent par le destructeur\FTNT{No. 14:2-29 ; No. 26:63-65.}.
\TextTitle{L'Eglise doit s'instruire par l'expérience d'Israël}
\VS{11}Or toutes ces choses leur sont arrivées pour servir d’exemples, et elles ont été écrites pour notre instruction, comme étant ceux auxquels les derniers temps sont parvenus.
\VS{12}Que celui donc qui pense demeurer debout prenne garde qu'il ne tombe.
\VS{13}Aucune tentation ne vous a éprouvés, qui n’ait été une tentation humaine, et Dieu qui est fidèle ne permettra pas que vous soyez tentés au-delà de vos forces, mais avec la tentation il préparera aussi le moyen d’en sortir, afin que vous puissiez la supporter.
\VS{14}C'est pourquoi, mes bien-aimés, fuyez l'idolâtrie.
\VS{15}Je vous parle comme à des personnes intelligentes, jugez vous-mêmes de ce que je dis.
\TextTitle{Distinction entre le repas et l'idolâtrie}
\VS{16}La coupe de bénédiction, que nous bénissons, n'est-elle pas la communion du sang de Christ ? Et le pain que nous rompons, n'est-il pas la communion au corps de Christ ?
\VS{17}Parce qu'il n'y a qu'un seul pain, nous qui sommes plusieurs sommes un seul corps ; car nous sommes tous participants du même pain.
\VS{18}Voyez l'Israël selon la chair, ceux qui mangent les sacrifices ne sont-ils pas en communion avec l'autel ?
\VS{19}Que dis-je donc ? Que l'idole soit quelque chose ? Ou que ce qui est sacrifié à l'idole soit quelque chose ? Nullement.
\VS{20}Mais je dis que les choses que les Gentils sacrifient, ils les sacrifient aux démons, et non à Dieu ; or je ne veux pas que vous soyez en communion avec des démons.
\VS{21}Vous ne pouvez pas boire la coupe du Seigneur et la coupe des démons ; vous ne pouvez pas participer à la table du Seigneur et à la table des démons\FTNT{L’apôtre Paul nous parle de deux sortes de tables : la table de Jézabel (ou des démons) et la table du Seigneur. La table du Seigneur à été révélée à Moïse (Ex. 25:23-30 ; Lé. 24:5-9). Il y avait dessus 12 pains destinés à la consommation des sacrificateurs. Ces pains étaient renouvelés chaque sabbat et représentaient Christ, le Pain de Dieu, qui est l’aliment du croyant-sacrificateur (Jn. 6:33-58). La table de Jézabel nous est présentée dans 1 R. 18:19 : «~Fais maintenant rassembler tout Israël auprès de moi, à la montagne du Carmel, et aussi les quatre cent cinquante prophètes de Baal et les quatre cents prophètes d’Astarté qui mangent à la table de Jézabel~». Jézabel avait à sa table 850 faux prophètes qui partageaient son repas. Voir Ap. 17. Satan est maître en matière de déguisement et d’imitation (2 Co. 11:13-15). Il a donc imité la table du Seigneur et propose aux hommes les mets du roi et le vin de la débauche (Da. 1). Il invite ceux qui cherchent Dieu à sa table afin de les détourner de la vision du ciel. Voir Mt. 6:24 ; Lu. 16:13.}.
\VS{22}Voulons-nous provoquer la jalousie du Seigneur ? Sommes-nous plus forts que lui ?
\TextTitle{La loi de l'amour s'applique dans le manger et le boire\FTNTT{Ro. 14:1-23}}
\VS{23}Toutes choses me sont permises, mais toutes ne sont pas utiles ; toutes choses me sont permises, mais toutes n'édifient pas.
\VS{24}Que personne ne cherche son propre intérêt, mais que chacun cherche celui d’autrui.
\VS{25}Mangez de tout ce qui se vend au marché, sans vous enquérir de rien par motif de conscience\FTNT{1 Ti. 4:3-5.}.
\VS{26}Car la terre avec tout ce qu'elle contient est au Seigneur.
\VS{27}Si un incrédule vous invite et que vous vouliez aller, mangez de tout ce qui sera mis devant vous, sans vous enquérir par motif de conscience.
\VS{28}Mais si quelqu'un vous dit : Ceci a été sacrifié aux idoles, n'en mangez pas, à cause de celui qui vous a avertis, et à cause de la conscience ; car la terre avec tout ce qu'elle contient est au Seigneur.
\VS{29}Je parle ici, non de votre conscience, mais de celle de l'autre. Pourquoi ma liberté serait-elle condamnée par la conscience d'un autre ?
\VS{30}Et si par la grâce j'en suis participant, pourquoi suis-je blâmé pour une chose dont je rends grâces ?
\VS{31}Soit donc que vous mangiez, soit que vous buviez, ou que vous fassiez quelque autre chose, faites tout à la gloire de Dieu.
\VS{32}Soyez tels que vous ne donniez aucun scandale ni aux Juifs, ni aux Grecs, ni à l'Eglise de Dieu,
\VS{33}de la même manière que moi aussi, je m’efforce en toutes choses de complaire à tous, cherchant, non pas mon avantage, mais celui du plus grand nombre, afin qu’ils soient sauvés.
\Chap{11}
\VerseOne{}Soyez mes imitateurs comme je le suis moi-même de Christ.
\TextTitle{Homme et femme devant Dieu}
\VS{2}Or mes frères, je vous loue de ce que vous vous souvenez de tout ce qui me concerne, et de ce que vous retenez mes instructions comme je vous les ai données.
\VS{3}Mais je veux que vous sachiez que Christ est le chef\FTNT{Le mot «~chef~» vient du grec «~kephal~» qui signifie tête. Jésus-Christ est la seule tête et l’unique chef de l’Eglise (Ep. 1:22-23 ; Col. 1:18). Toute personne qui se proclame la tête de l’église devient naturellement antéchrist.} de tout homme, que l’homme est le chef de la femme, et que Dieu est le chef de Christ.
\VS{4}Tout homme qui prie ou qui prophétise, ayant quelque chose sur la tête, déshonore son chef.
\VS{5}Toute femme au contraire qui prie, ou qui prophétise sans avoir la tête couverte, déshonore son chef, c'est comme si elle était rasée.
\VS{6}Car si une femme n'est pas couverte, qu’on lui coupe aussi les cheveux. Or, s'il est honteux pour une femme d'avoir les cheveux coupés, ou d'être rasée, qu'elle se voile.
\VS{7}Car pour ce qui est de l'homme, il ne doit point couvrir sa tête, vu qu'il est l'image et la gloire de Dieu ; mais la femme est la gloire de l'homme.
\VS{8}Parce que l'homme n'a point été tiré de la femme, mais la femme a été tirée de l'homme.
\VS{9}Et aussi l'homme n'a pas été créé pour la femme, mais la femme pour l'homme.
\VS{10}C'est pourquoi la femme à cause des anges doit avoir sur la tête une marque de l’autorité de son mari dont elle dépend.
\VS{11}Toutefois, dans le Seigneur, l'homme n'est point sans la femme ni la femme sans l'homme.
\VS{12}Car comme la femme est par l'homme, de même l'homme est par la femme, et tout cela procède de Dieu.
\VS{13}Jugez-en vous-mêmes : Est-il convenable que la femme prie Dieu sans être couverte ?
\VS{14}La nature elle-même ne vous enseigne-t-elle pas que c’est une honte pour l'homme d’avoir de longs cheveux,
\VS{15}mais que c’est une gloire pour la femme de porter des longs cheveux, parce que la chevelure lui a été donnée pour lui servir de voile ?
\VS{16}Si quelqu'un aime à contester, nous n'avons pas une telle coutume, ni les églises de Dieu.
\TextTitle{Le repas du Seigneur et les abus dénoncés par Paul}
\VS{17}Or en ce que je vais vous dire, je ne vous loue point : C’est que vous vous assemblez, non pour devenir meilleurs, mais pour empirer.
\VS{18}Car premièrement, lorsque vous vous réunissez en assemblée, j'apprends qu'il y a des divisions parmi vous et j'en crois une partie,
\VS{19}car il faut qu'il y ait même des hérésies parmi vous, afin que ceux qui sont dignes d’être approuvés soient reconnus parmi vous.
\VS{20}Quand donc vous vous assemblez ainsi tous ensemble, ce n'est pas pour manger le repas du Seigneur ;
\VS{21}car, quand on se met à table, chacun commence par prendre son souper particulier, et l'un a faim tandis que l'autre est ivre.
\VS{22}N'avez-vous donc pas de maisons pour manger et pour boire ? Ou méprisez-vous l'Eglise de Dieu et faites-vous honte à ceux qui n'ont rien ? Que vous dirai-je ? Vous louerai-je ? Je ne vous loue point en cela.
\TextTitle{Le repas du Seigneur}
\VS{23}Car j'ai reçu du Seigneur ce qu'aussi je vous ai donné ; c’est que le Seigneur Jésus, la nuit où il fut trahi, prit du pain,
\VS{24}et après avoir rendu grâces, le rompit et dit : Prenez, mangez : Ceci est mon corps qui est rompu pour vous ; faites ceci en mémoire de moi.
\VS{25}De même aussi après le souper, il prit la coupe, en disant : Cette coupe est la nouvelle alliance en mon sang ; faites ceci toutes les fois que vous en boirez, en mémoire de moi\FTNT{Mt. 26:26-28 ; Mc. 14:22-24 ; Lu. 22:19-20.}.
\VS{26}Car toutes les fois que vous mangerez de ce pain, et que vous boirez de cette coupe, vous annoncerez la mort du Seigneur, jusqu’à ce qu'il vienne.
\VS{27}C'est pourquoi quiconque mangera de ce pain ou boira de la coupe du Seigneur indignement, sera coupable envers le corps et le sang du Seigneur.
\VS{28}Que chacun donc s'éprouve soi-même, et ainsi qu'il mange de ce pain, et qu'il boive de cette coupe.
\VS{29} Car celui qui en mange et qui en boit indignement, mange et boit sa condamnation, ne distinguant point le corps du Seigneur.
\VS{30}C’est pour cela qu’il y a parmi vous beaucoup d’infirmes et de malades, et que plusieurs dorment.
\VS{31}Car si nous nous jugions nous-mêmes, nous ne serions point jugés.
\VS{32}Mais quand nous sommes jugés, nous sommes enseignés par le Seigneur, afin que nous ne soyons point condamnés avec le monde.
\VS{33}C'est pourquoi, mes frères, quand vous vous assemblez pour manger, attendez-vous les uns les autres.
\VS{34}Et si quelqu'un a faim, qu'il mange dans sa maison, afin que vous ne vous assembliez pas pour votre condamnation.  Touchant les autres points, je les réglerai quand je serai arrivé.
\Chap{12}
\TextTitle{L'Esprit révèle Christ}
\VerseOne{}Pour ce qui concerne les dons spirituels, je ne veux point, mes frères, que vous soyez ignorants.
\VS{2}Vous savez que lorsque vous étiez des gentils, vous vous laissiez entraîner vers les idoles muettes, selon que vous étiez conduits.
\VS{3}C'est pourquoi je vous fais savoir que personne, s’il parle par l'Esprit de Dieu, ne dit : Jésus est anathème ! Et personne ne peut dire : Jésus est le Seigneur ! Si ce n’est par le Saint-Esprit.
\TextTitle{La diversité des dons de l'Esprit\FTNTT{Ep. 4:7-16}}
\VS{4} Or il y a diversité de dons, mais il n'y a qu'un même Esprit.
\VS{5}Il y a aussi diversité de ministères, mais il n'y a qu'un même Seigneur.
\VS{6}Il y a aussi diversité d'opérations, mais il n'y a qu'un même Dieu qui opère toutes choses en tous.
\VS{7}Or à chacun est donnée la manifestation de l'Esprit pour l'utilité commune.
\VS{8}Car à l'un est donnée par l'Esprit, la parole de sagesse ; et à l'autre par le même Esprit, la parole de connaissance ;
\VS{9}et à un autre, la foi par ce même Esprit ; à un autre, les dons de guérison par ce même Esprit ;
\VS{10}et à un autre, les opérations des miracles ; à un autre, la prophétie ; à un autre, le don de discerner les esprits ; à un autre, la diversité de langues ; et à un autre, le don d'interpréter les langues.
\VS{11}Un seul et même Esprit opère toutes ces choses, distribuant à chacun ses dons en particulier comme il lui plaît.
\TextTitle{Chaque membre à son utilité dans le corps de Christ}
\VS{12}Car comme le corps est un, et cependant a plusieurs membres, et comme tous les membres du corps, malgré leur nombre, ne forment qu’un seul corps, il en est de même de Christ.
\VS{13}Nous avons tous, en effet, été baptisés d'un même Esprit\FTNT{Le baptême du Saint-Esprit : Les signes du baptême du Saint-Esprit (la conversion) sont les fruits de l’Esprit et sont abordés en Ga. 5:22. A aucun endroit, les écritures stipulent que le parler en langues, qui est un don gratuit (Mt. 7:16-20), est en soi le signe du baptême du Saint-Esprit. Ainsi, il nous est dit que chaque croyant en Christ a le Saint-Esprit (1 Co. 12:13 ; Ro. 8:9 ; Ep. 1:13-14) mais que tous les croyants ne parlent pas forcément en langues (1 Co. 12:29-31).}, pour être un même corps, soit Juifs, soit Grecs, soit esclaves, soit libres, nous avons tous, dis-je, été abreuvés d'un seul Esprit.
\VS{14}Ainsi, le corps n’est pas un seul membre, mais il est formé de plusieurs membres.
\VS{15}Si le pied dit : Parce que je ne suis pas la main, je ne suis point du corps ; ne serait-il pas pourtant du corps ?
\VS{16}Et si l'oreille dit : Parce que je ne suis pas l’œil, je ne suis point du corps ; ne serait-elle pas pourtant du corps ?
\VS{17}Si tout le corps est l’œil, où serait l'ouïe ? Si tout est l'ouïe, où serait l'odorat ?
\VS{18}Mais maintenant Dieu a placé chaque membre dans le corps comme il a voulu.
\VS{19}Et si tous étaient un seul membre, où serait le corps ?
\VS{20}Maintenant donc, il y a plusieurs membres et un seul corps.
\VS{21}L’œil ne peut pas dire à la main : Je n'ai pas besoin de toi ; ni la tête dire aux pieds : Je n'ai pas besoin de vous.
\VS{22}Et qui plus est, les membres du corps qui semblent être les plus faibles sont beaucoup plus nécessaires ;
\VS{23}et ceux que nous estimons être les moins honorables au corps, nous les entourons d’un plus grand honneur. Ainsi, nos membres les moins décents reçoivent le plus d’honneur,
\VS{24}Car les parties qui sont belles en nous, n’en ont pas besoin. Mais Dieu a disposé le corps de manière à donner plus d’honneur à ce qui en manquait,
\VS{25}afin qu'il n'y ait pas de division dans le corps, mais que les membres aient un soin mutuel les uns des autres.
\VS{26}Et si l'un des membres souffre quelque chose, tous les membres souffrent avec lui ; si l'un des membres est honoré, tous les membres ensemble se réjouissent avec lui.
\VS{27}Vous êtes le corps de Christ, et vous êtes chacun l’un de ses membres.
\VS{28}Et Dieu a établi dans l'Eglise premièrement des apôtres, deuxièmement des prophètes, troisièmement des docteurs, ensuite ceux qui ont le don des miracles, puis ceux qui ont les dons de guérir, de secourir, de gouverner, de parler diverses langues.
\VS{29}Tous sont-ils apôtres ? Tous sont-ils prophètes ? Tous sont-ils docteurs ? Tous ont-ils le don des miracles ?
\VS{30}Tous ont-ils les dons de guérisons ? Tous parlent-ils diverses langues ? Tous interprètent-ils ?
\VS{31}Désirez avec ardeur des dons plus excellents, et je vais vous montrer la voie la plus excellente.
\Chap{13}
\TextTitle{L'amour est la base de tout}
\VerseOne{}Quand je parlerais toutes les langues des hommes\FTNT{Les langues des hommes. Les 120 Galiléens ont été rendus capables de s’exprimer dans diverses langues afin de pouvoir annoncer la vérité aux personnes en voyage à Jérusalem dans leurs propres langues. Voir Es. 28:11-12 ; Ac. 2:1-13.}, et même des anges\FTNT{La langue des anges ou langue inconnue est incompréhensible à notre intelligence, elle est un des moyens par lequel nous disons des mystères à Dieu. Voir Ro. 8:25-26 ; 1 Co. 14:2 et 28. Il faut une interprétation si l’on veut parler cette langue dans l’assemblée à cause des non croyants qui nous visitent (1 Co. 14:23). Voir Mc. 16:17.}, si je n'ai pas la charité\FTNT{Il est question ici de l’amour «~agape~» : l’amour divin et désintéressé, l’amour fraternel.}, je suis un airain qui résonne ou une cymbale qui retentit.
\VS{2}Et quand j'aurais le don de prophétie et que je connaîtrais tous les mystères et la science de toutes choses ; et quand j'aurais même toute la foi qu'on puisse avoir, jusqu’à transporter les montagnes, si je n'ai pas la charité, je ne suis rien.
\VS{3}Et quand je distribuerais tous mes biens pour la nourriture des pauvres, quand je livrerais mon corps pour être brûlé, si je n'ai pas la charité, cela ne me sert à rien.
\VS{4}La charité est patiente, la charité est douce, la charité n'est point envieuse, la charité n'use point d'insolence, elle ne s’enfle point d’orgueil,
\VS{5}elle ne fait rien de malhonnête, elle ne cherche point son intérêt, elle ne s’irrite point, elle n’impute pas le mal,
\VS{6}elle ne se réjouit point de l'injustice, mais elle se réjouit de la vérité.
\VS{7}Elle couvre\FTNT{Dans ce passage, le grec utilisé, «~stego~», signifie «~toit, couverture, protéger ou garder en recouvrant, préserver~» (Pr. 10:12 ; Pr. 17:9). La charité ne rappelle pas sans cesses les erreurs des uns et des autres, mais sait préserver en gardant secret les fautes est expiées. Par contre, en aucun cas elle ne permet la compromission du péché en ne dénonçant pas les oeuvres des ténèbres (Mt. 18:15-18 ; Ja. 5:19-20).} tout, elle croit tout, elle espère tout, elle supporte tout.
\VS{8}La charité ne périt jamais. Les prophéties seront abolies et les langues cesseront, la connaissance sera abolie.
\VS{9}Car nous connaissons en partie et nous prophétisons en partie.
\VS{10}Mais quand la perfection sera venue, alors ce qui est en partie sera aboli.
\VS{11}Quand j'étais enfant, je parlais comme un enfant, je jugeais comme un enfant, je pensais comme un enfant ; mais quand je suis devenu homme, j'ai aboli ce qui était de l'enfance.
\VS{12}Car aujourd’hui nous voyons au moyen d’un miroir, de manière obscure, mais alors nous verrons face à face. Aujourd’hui je connais en partie, mais alors je connaîtrai comme j'ai été connu.
\VS{13}Maintenant ces trois choses demeurent : La foi, l'espérance et la charité ; mais la plus excellente de ces trois vertus c'est la charité.
\Chap{14}
\TextTitle{Importance du don de prophétie}
\VerseOne{}Recherchez la charité. Désirez avec ardeur les dons spirituels, mais surtout celui de prophétiser.
\VS{2}Parce que celui qui parle une langue inconnue ne parle point aux hommes, mais à Dieu, car personne ne le comprend, et c’est en esprit qu’il dit des mystères.
\VS{3}Mais celui qui prophétise, édifie, exhorte et console les hommes qui l'entendent.
\VS{4}Celui qui parle une langue inconnue s'édifie lui-même, mais celui qui prophétise édifie l'Eglise.
\VS{5}Je désire que vous parliez tous diverses langues, mais encore plus que vous prophétisiez. Celui qui prophétise est plus grand que celui qui parle diverses langues, à moins que ce dernier n’interprète, afin que l'Eglise en reçoive de l'édification.
\VS{6}Maintenant donc, mes frères, si je viens à vous et que je parle des langues inconnues, que vous servira cela si je ne vous parle pas par révélation, ou par science, ou par prophétie, ou par doctrine ?
\VS{7}De même, si les choses inanimées qui rendent un son, comme une flûte ou une harpe, ne rendent pas des sons distincts, comment reconnaîtra-t-on ce qui est joué sur la flûte ou sur la harpe ?
\VS{8}Et si la trompette rend un son confus, qui se préparera à la bataille ?
\VS{9}De même vous, si vous ne prononcez dans votre langue une parole distincte, comment saura-t-on ce que vous dites ? Car vous parlerez en l'air.
\VS{10}Et il y a, selon qu'il se rencontre, tant de divers sons dans le monde, et cependant aucun de ces sons n'est muet ;
\VS{11}mais si je ne sais point ce qu'on veut signifier par la parole, je serai un barbare pour celui qui parle, et celui qui parle sera un barbare pour moi.
\VS{12}Ainsi, puisque vous désirez avec ardeur les dons spirituels, que ce soit pour l’édification de l'Eglise que vous cherchiez à en posséder abondamment.
\VS{13}C'est pourquoi que celui qui parle une langue inconnue prie pour avoir le don d’interpréter.
\VS{14}Car si je prie dans une langue inconnue mon esprit est en prière, mais l'intelligence que j'en ai, est sans fruit.
\VS{15}Que faire donc ? Je prierai par l’esprit, mais je prierai aussi d'une manière à être entendu ; je chanterai par l’esprit, mais je chanterai aussi d'une manière à être entendu.
\VS{16}Autrement, si tu rends grâces par l’esprit, comment celui qui est du simple peuple dira-t-il Amen ! à ton action de grâces\FTNT{L’expression «~actions de grâces~» vient du grec «~eucharisteo~» ce qui signifie être reconnaissant, rendre grâces, remercier. Contrairement à ce que l’on enseigne dans beaucoup d’églises, il n’est pas question ici de faire une offrande d’argent mais de se montrer reconnaissant envers le Seigneur. Voir aussi commentaires en Lé. 3 et Lé. 7.}, puisqu'il ne sait pas ce que tu dis ?
\VS{17}Il est vrai que tu rends grâces, mais l’autre n’est pas édifié.
\VS{18}Je rends grâces à mon Dieu de ce que je parle plus de langues que vous tous.
\VS{19}Mais j'aime mieux prononcer dans l'Eglise cinq paroles d'une manière à être entendu, afin d’instruire aussi les autres, que dix mille paroles dans une langue inconnue.
\VS{20}Mes frères, ne soyez point des enfants sous le rapport du jugement, mais soyez des enfants à l’égard de la malice ; et à l'égard du jugement, soyez des hommes faits.
\VS{21}Il est écrit dans la loi : je parlerai à ce peuple par des gens d'une autre langue, et par des lèvres étrangères, et ils ne m’écouteront pas même ainsi, dit le Seigneur\FTNT{Es. 28:11.}.
\VS{22}C’est pourquoi les langues sont un signe, non pour les croyants, mais pour les non-croyants ; la prophétie, au contraire, est un signe, non pour les non-croyants, mais pour les croyants.
\TextTitle{L'exercice des dons spirituels dans les églises locales}
\VS{23}Si donc, l’Eglise entière s'assemble en un corps, et que tous parlent des langues étrangères et qu'il entre des gens du commun peuple ou des non-croyants, ne diront-ils pas que vous êtes hors de sens ?
\VS{24}Mais si tous prophétisent, et qu'il entre un non-croyant ou quelqu'un du commun peuple, il est convaincu par tous et il est jugé de tous,
\VS{25}ainsi les secrets de son cœur sont manifestés, de telle sorte qu'il tombera sur sa face, il adorera Dieu et publiera que Dieu est véritablement parmi vous.
\VS{26}Que faire donc mes frères ? Lorsque vous vous assemblez, les uns ou les autres parmi vous ont-ils un cantique, une instruction, une langue étrangère, une révélation, une interprétation, que tout se fasse pour l'édification.
\VS{27}Et si quelqu'un parle une langue inconnue, que cela se fasse par deux, ou tout au plus par trois, chacun à son tour, et que quelqu’un interprète ;
\VS{28}s'il n'y a point d'interprète, que cet homme se taise dans l'Eglise, et qu'il parle à lui-même et à Dieu.
\VS{29}Et que deux ou trois prophètes parlent, et que les autres en jugent ;
\VS{30}et si quelque chose est révélé à un autre qui est assis, que le premier se taise.
\VS{31}Car vous pouvez tous prophétiser l'un après l'autre, afin que tous soient instruits et que tous soient consolés.
\VS{32}Et les esprits des prophètes sont soumis aux prophètes.
\VS{33}Car Dieu n'est point un Dieu de confusion, mais de paix, comme on le voit dans toutes les églises des saints.
\VS{34}Que les femmes qui sont parmi vous se taisent dans les églises ; car il ne leur est point permis d’y parler, mais elles doivent être soumises, comme le dit aussi la loi.
\VS{35}Et si elles veulent s’instruire sur quelque chose, qu'elles interrogent leurs maris à la maison ; car il est honteux à une femme de parler dans l'église.
\VS{36}Est-ce de chez vous que la parole de Dieu est sortie ? Ou est-elle parvenue seulement à vous ?
\VS{37}Si quelqu'un croit être prophète, ou spirituel, qu'il reconnaisse que les choses que je vous écris sont des commandements du Seigneur.
\VS{38}Et si quelqu'un l’ignore, qu'il l’ignore.
\VS{39}C'est pourquoi, mes frères, désirez avec ardeur de prophétiser, et n'empêchez point de parler diverses langues.
\VS{40}Que toutes choses se fassent avec bienséance, et avec ordre.
\Chap{15}
\TextTitle{L'Evangile basé sur la résurrection de Christ}
\VerseOne{}Or, mes frères, je vous rappelle l'Evangile que je vous ai annoncé, que vous avez reçu, et auquel vous vous tenez ferme,
\VS{2}et par lequel vous êtes sauvés, si vous le retenez tel je vous l'ai annoncé ; à moins que vous n'ayez cru en vain. 
\VS{3}Car avant toutes choses, je vous ai donné ce que j'avais aussi reçu, à savoir que Christ est mort pour nos péchés, selon les Ecritures,
\VS{4}et qu'il a été enseveli, et qu'il est ressuscité\FTNT{La Résurrection du Messie. La résurrection de Jésus est un espoir pour tous les êtres humains. Elle est un principe fondamental de la foi chrétienne. Contrairement à toutes les autres religions, la foi chrétienne est la seule qui apporte l’espérance face à la mort. Toutes les autres religions ont été fondées par des hommes, leurs prophètes ou fondateurs sont morts et aucun n’est revenu à la vie. En tant que disciples de Jésus, nous sommes réconfortés par le fait que notre Dieu s'est fait homme, afin de mourir pour nos péchés, et est ressuscité le troisième jour. L’Enfer ne pouvait pas le retenir, et il tient les clés de la mort et de l’Enfer (Ap. 1:18). Voir Jn. 11:25-26. Jésus-Christ est la Résurrection.} le troisième jour, selon les Ecritures ;
\VS{5}et qu'il a été vu de Céphas, et ensuite des douze.
\VS{6}Depuis, il a été vu de plus de cinq cents frères à la fois, dont plusieurs sont encore vivants, et quelques-uns sont morts.
\VS{7}Depuis, il est apparu à Jacques, puis à tous les apôtres.
\VS{8}Après eux tous, il a été vu aussi de moi, comme d'un avorton.
\VS{9}Car je suis le moindre des apôtres, je ne suis pas digne d'être appelé apôtre, parce que j'ai persécuté l'Eglise de Dieu.
\VS{10}Mais par la grâce de Dieu, je suis ce que je suis ; et sa grâce envers moi n'a pas été vaine, mais j'ai travaillé plus qu'eux tous, toutefois non pas moi, mais la grâce de Dieu qui est avec moi.
\VS{11}Soit donc moi, soit eux, nous prêchons ainsi et vous l'avez cru ainsi.
\TextTitle{Importance de la résurrection de Christ}
\VS{12}Or si on prêche que Christ est ressuscité des morts, comment disent quelques-uns d'entre vous qu'il n'y a point de résurrection des morts ?
\VS{13}Car s'il n'y a point de résurrection des morts, Christ aussi n'est point ressuscité.
\VS{14}Et si Christ n'est pas ressuscité, notre prédication est donc vaine, et votre foi aussi est vaine.
\VS{15}Et même nous sommes de faux témoins de la part de Dieu, car nous avons rendu témoignage à l’égard de Dieu qu'il a ressuscité Christ, tandis qu’il ne l’aurait pas ressuscité, si les morts ne ressuscitent point.
\VS{16}Car si les morts ne ressuscitent point, Christ non plus n'est point ressuscité.
\VS{17}Et si Christ n'est pas ressuscité, votre foi est vaine, et vous êtes encore dans vos péchés,
\VS{18}et par conséquent aussi ceux qui dorment en Christ sont perdus.
\VS{19}Si nous n'avons d'espérance en Christ que pour cette vie seulement, nous sommes les plus misérables de tous les hommes.
\TextTitle{Détails sur les résurrections}
\VS{20}Mais maintenant Christ est ressuscité des morts, il est les prémices de ceux qui dorment.
\VS{21}Car puisque la mort est venue par un seul homme, c’est aussi par un homme qu’est venue la résurrection des morts.
\VS{22}Car comme tous meurent en Adam, de même aussi tous seront vivifiés en Christ.
\VS{23}Mais chacun en son rang, Christ comme prémices, puis ceux qui sont à Christ seront vivifiés lors de son avènement.
\VS{24}Ensuite viendra la fin, quand il aura remis le Royaume à Dieu le Père, après avoir aboli tout empire, toute puissance, et toute force.
\VS{25}Car il faut qu'il règne jusqu'à ce qu'il ait mis tous ses ennemis sous ses pieds\FTNT{Ps. 110:1.}.
\VS{26}L'ennemi qui sera détruit le dernier c'est la mort.
\VS{27}Car Dieu a tout mis sous ses pieds. Mais lorsqu’il dit que tout lui a été soumis, il est évident que celui qui lui a soumis toutes choses est excepté.
\VS{28}Et lorsque toutes choses lui auront été soumises, alors le Fils lui-même sera soumis à celui qui lui a soumis toutes choses, afin que Dieu soit tout en tous.
\VS{29}Autrement que feraient ceux qui se font baptiser pour les morts ? Si les morts ne ressuscitent absolument pas, pourquoi se font-ils baptiser pour les morts ?
\VS{30}Et nous, pourquoi sommes-nous en danger à toute heure ?
\VS{31}Tous les jours je suis exposé à la mort, je l’atteste, par la gloire de notre Seigneur Jésus-Christ.
\VS{32}Si j'ai combattu contre les bêtes à Ephèse dans des vues humaines, quel profit m’en revient-il ? Si les morts ne ressuscitent pas, mangeons et buvons, car demain nous mourrons.
\VS{33}Ne soyez point séduits : Les mauvaises compagnies corrompent les bonnes mœurs.
\VS{34}Réveillez-vous pour vivre justement, et ne péchez point ; car quelques-uns ne connaissent pas Dieu, je le dis à votre honte.
\TextTitle{Corps de résurrection}
\VS{35}Mais quelqu'un dira : Comment les morts ressuscitent-ils, et avec quel corps viennent-ils ?
\VS{36}Insensé ! Ce que tu sèmes ne reprend point vie s'il ne meurt pas\FTNT{Jn. 12:24.}.
\VS{37}Et ce que tu sèmes, tu ne sèmes point le corps qui naîtra, c’est un simple grain, de blé peut-être, ou d’une autre semence.
\VS{38}Mais Dieu lui donne le corps comme il veut, et à chacune des semences son propre corps.
\VS{39}Toute chair n'est pas de la même chair, mais autre est la chair des hommes, autre la chair des bêtes, autre celle des poissons, autre celle des oiseaux.
\VS{40}Il y a aussi des corps célestes, et des corps terrestres ; mais autre est l’éclat des corps célestes, et autre celui des corps terrestres.
\VS{41}Autre est l’éclat du soleil, autre l’éclat de la lune, autre l’éclat des étoiles ; même une étoile diffère d'une autre étoile en éclat.
\VS{42}Il en sera aussi de même à la résurrection des morts : Le corps est semé corruptible, il ressuscitera incorruptible.
\VS{43}Il est semé en déshonneur, il ressuscite glorieux ; il est semé en faiblesse, il ressuscite plein de force.
\VS{44}Il est semé corps animal, il ressuscitera corps spirituel. S’il y a un corps animal, il y a aussi un corps spirituel.
\VS{45}Comme aussi il est écrit : Le premier homme, Adam, devint une âme vivante\FTNT{Ge. 2:7.}. Le dernier Adam est devenu un Esprit vivifiant\FTNT{Jn. 5 : 21 ; Ro. 8 : 11. Jésus-Christ est le dernier Adam. Voir Ph. 2:7 ; 1 T. 3:16.}.
\VS{46}Or ce qui est spirituel n'est pas le premier, mais ce qui est animal ; et puis vient ce qui est spirituel.
\VS{47}Le premier homme, étant de la terre, est tiré de la poussière, mais le second homme, à savoir le Seigneur, est du ciel.
\VS{48}Tel qu'est celui qui est tiré de la poussière, tels aussi sont ceux qui sont tirés de la poussière ; et tel qu'est le céleste, tels aussi sont les célestes.
\VS{49}Et comme nous avons porté l'image de celui qui est tiré de la poussière, nous porterons aussi l'image du céleste.
\VS{50}Voici donc ce que je dis, mes frères, c'est que la chair et le sang ne peuvent hériter le Royaume de Dieu, et que la corruption n'hérite pas l'incorruptibilité.
\TextTitle{Mystère de la résurrection\FTNTT{1 Th. 4:14-17}}
\VS{51}Voici, je vous dis un mystère : Nous ne mourrons pas tous, mais tous nous serons changés,
\VS{52}en un instant, en un clin d’œil, à la dernière trompette\FTNT{Le mot dernier dans ce passage est «~eschatos~»  qui signifie «~dernier en temps ou en lieu, dernier dans des séries de lieux, dernier dans une succession dans le temps~». Paul associe le mystère de la résurrection à la dernière trompette. Or, dans le livre d'Apocalypse il n'y a que sept trompettes (Ap. 8 ; 9 et 11:15-19), et c'est à la dernière, c'est-à-dire la septième que le mystère de Dieu s'accomplit.}. La trompette sonnera, et les morts ressusciteront incorruptibles, et nous, nous serons changés.
\VS{53}Car il faut que ce corps corruptible revête l'incorruptibilité, et que ce corps mortel revête l'immortalité.
\TextTitle{La mort engloutie}
\VS{54}Lorsque ce corps corruptible aura revêtu l'incorruptibilité, et que ce corps mortel aura revêtu l'immortalité, alors cette parole de l'Ecriture sera accomplie : La mort a été engloutie dans la victoire\FTNT{Es. 25:8.}.
\VS{55}Ô mort, où est ta victoire ? Ô mort, où est ton aiguillon\FTNT{Os. 13:14.} ?
\VS{56}L'aiguillon de la mort c'est le péché ; et la puissance du péché c'est la loi.
\VS{57}Mais grâces soient rendues à Dieu qui nous a donné la victoire par notre Seigneur Jésus-Christ !
\VS{58}C'est pourquoi, mes frères bien-aimés, soyez fermes, inébranlables, vous appliquant toujours avec un nouveau zèle à l’œuvre du Seigneur, sachant que votre travail ne sera pas vain dans le Seigneur.
\Chap{16}
\TextTitle{Instructions et salutations de Paul}
\VerseOne{}Pour ce qui concerne la collecte en faveur des saints, faites comme je l’ai ordonné aux églises de Galatie.
\VS{2}C’est que chaque premier jour de la semaine, chacun de vous mette à part chez lui ce qu’il pourra assembler, selon la prospérité que Dieu lui accordera, afin qu’on n’attende pas mon arrivée pour recueillir les dons.
\VS{3}Puis quand je serai arrivé, j'enverrai les personnes que vous aurez approuvées avec des lettres pour porter votre libéralité à Jérusalem.
\VS{4}Et s’il convient que j’y aille moi-même, ils viendront aussi avec moi.
\VS{5}J'irai donc chez vous quand j’aurai traversé la Macédoine, car je traverserai par la Macédoine.
\VS{6}Et peut-être que je séjournerai parmi vous, ou même que j'y passerai l'hiver, afin que vous me conduisiez partout où j’irai.
\VS{7}Car je ne veux pas cette fois vous voir en passant, mais j’espère demeurer quelque temps auprès de vous, si le Seigneur le permet.
\VS{8}Toutefois, je resterai à Ephèse jusqu'à la Pentecôte.
\VS{9}Car une grande porte et un accès efficace m'y est ouverte, et les adversaires sont nombreux.
\VS{10}Si Timothée arrive, faites en sorte qu'il soit en sûreté parmi vous, car il travaille à l’œuvre du Seigneur comme moi-même.
\VS{11}Que personne donc ne le méprise. Accompagnez-le en paix, afin qu'il vienne vers moi, car je l'attends avec les frères.
\VS{12}Quant à Apollos, notre frère, je l'ai beaucoup exhorté à se rendre chez vous avec les frères, mais ce n’était décidément pas sa volonté de le faire maintenant ; il partira quand il en aura l’occasion.
\VS{13}Veillez, soyez fermes dans la foi, agissez courageusement, fortifiez-vous.
\VS{14}Que tout ce que vous faites se fasse avec charité.
\VS{15}Or, mes frères, vous connaissez la famille de Stéphanas, et vous savez qu'elle est les prémices de l'Achaïe, et qu’elle s’est entièrement appliquée au service des saints.
\VS{16}Je vous prie de vous soumettre à de tels hommes, et à tous de ceux qui s’emploient à l’œuvre du Seigneur, et qui travaillent avec nous.
\VS{17}Je me réjouis de l’arrivée de Stéphanas, de Fortunatus, et d’Achaïcus, parce qu'ils ont suppléé à votre absence.
\VS{18}Car ils ont tranquillisé mon esprit et le vôtre. Ayez donc de la considération pour de telles personnes.
\VS{19}Les églises d'Asie vous saluent. Aquilas et Priscille, avec l'église qui est dans leur maison, vous saluent affectueusement dans le Seigneur.
\VS{20}Tous les frères vous saluent. Saluez-vous les uns les autres par un saint baiser.
\VS{21}Je vous salue, moi Paul, de ma propre main.
\VS{22}Si quelqu'un n'aime pas le Seigneur Jésus-Christ, qu'il soit anathème ! Maranatha\FTNT{Maranatha signifie littéralement «~Le Seigneur vient !~}.
\VS{23}Que la grâce de notre Seigneur Jésus-Christ soit avec vous !
\VS{24}Mon amour est avec vous tous en Jésus-Christ. Amen.
\PPE{}
\end{multicols}

%\clearpage\ShortTitle{2 Corinthiens}\BookTitle{2 Corinthiens}\BFont
\noindent\hrulefill
{\footnotesize
\textit{
\bigskip
{\centering{}
\\Auteur : Paul avec Tite et Luc
\\Thème : L'autorité de Paul
\\Date de rédaction : Env. 57 ap. J.-C.\\}
}
%\bigskip
\textit{
\\Dans l’antiquité, Corinthe, capitale de l’Achaïe, était la ville la plus prospère et la plus puissante de Grèce. Située sur
un isthme séparant la mer Egée de la mer Ionienne, Corinthe était au carrefour de l’Asie et de l’Italie et constituait un
véritable centre commercial où les produits orientaux et occidentaux se croisaient.
%\bigskip
\\Rédigée quelques mois après la première, la seconde lettre de Paul aux Corinthiens fait état d’une vague de méfiance à l’égard de Paul et exprime les souffrances qui furent les siennes et qui somme toute authentifient son apostolat.\bigskip
}
}
\par\nobreak\noindent\hrulefill
\begin{multicols}{2}
\Chap{1}
\TextTitle{Introduction}
\VerseOne{}Paul, apôtre de Jésus-Christ par la volonté de Dieu, et le frère Timothée, à l'église de Dieu qui est à Corinthe, et à tous les saints qui sont dans toute l'Achaïe.
\VS{2}Que la grâce et la paix vous soient données de la part de Dieu notre Père et du Seigneur Jésus-Christ.
\TextTitle{Consolation de Paul dans ses afflictions}
\VS{3}Béni soit Dieu, le Père de notre Seigneur Jésus-Christ, le Père des miséricordes et le Dieu de toute consolation,
\VS{4}qui nous console dans toutes nos afflictions, afin que par la consolation dont nous sommes l’objet de la part de Dieu, nous puissions consoler ceux qui se trouvent dans l’affliction.
\VS{5}Car de même que les souffrances de Christ abondent en nous, de même notre consolation abonde par Christ.
\VS{6}Et si nous sommes affligés, c'est pour votre consolation et pour votre salut ; si nous sommes consolés, c'est pour votre consolation et pour votre salut qui se réalise par la patience à supporter les mêmes souffrances que nous endurons aussi.
\VS{7}Et l'espérance que nous avons de vous est ferme, sachant que comme vous êtes participants des souffrances, de même aussi vous le serez de la consolation.
\VS{8}Car mes frères, nous ne voulons pas que vous ignoriez l’affliction qui nous est survenue en Asie, que nous avons été excessivement accablés, au-delà de nos forces, de telle sorte que nous avions perdu l'espérance de conserver notre vie.
\VS{9}Et nous regardions comme certain notre arrêt de mort, afin de ne pas placer notre confiance en nous-mêmes, mais en Dieu qui ressuscite les morts.
\VS{10}C’est lui qui nous a délivrés et qui nous délivrera d'une si grande mort, et en qui nous espérons qu'il nous délivrera aussi à l'avenir.
\VS{11}Etant aussi aidés par la prière que vous faites pour nous, afin que la grâce obtenue pour nous par plusieurs soit pour plusieurs une occasion de rendre grâces à notre sujet.
\TextTitle{La sincérité de Paul dans son ministère}
\VS{12}Car ce qui fait notre gloire c’est le témoignage de notre conscience, que nous nous sommes conduits dans le monde, et surtout à votre égard, avec simplicité et sincérité de Dieu, non point avec une sagesse charnelle, mais avec la grâce de Dieu.
\VS{13}Nous ne vous écrivons pas autre chose que ce que vous lisez, et vous-mêmes le reconnaissez. Et j'espère que vous les reconnaîtrez aussi jusqu'à la fin,
\VS{14}de même que vous avez reconnu en partie que nous sommes votre gloire, comme vous serez aussi la nôtre au jour du Seigneur Jésus.
\TextTitle{Sa manière d'agir}
\VS{15}C’est dans une telle confiance que je voulais premièrement aller vers vous, afin que vous ayez une seconde grâce ;
\VS{16}et passer de chez vous en Macédoine, puis de Macédoine revenir vers vous, et être accompagné par vous en Judée.
\VS{17}Or quand je me proposais cela, ai-je usé de légèreté ? ou les choses  que je propose, sont-elles proposées selon la chair, de sorte qu'il y ait eu en moi le oui et le non ?
\VS{18}Mais Dieu est fidèle, la parole que nous vous avons adressée n’a pas été oui et non.
\VS{19}Car le Fils de Dieu, Jésus-Christ, qui a été prêché par nous au milieu de vous, par moi, par Silvain, et par Timothée, n'a pas été oui et non, mais il a été oui en lui.
\VS{20}Car autant qu’il y a de promesses de Dieu, elles sont oui en lui, et amen en lui, afin que Dieu soit glorifié par nous.
\VS{21}Or celui qui nous affermit avec vous en Christ, et qui nous a oints, c'est Dieu,
\VS{22}lequel nous a aussi marqués d’un sceau, et a mis dans nos cœurs les arrhes\FTNT{Du grec «~arrhabon~»~: arrhes~; monnaie donnée en gage d’un futur paiement, en attendant que le solde soit payé.} de l'Esprit.
\VS{23}Or j'appelle Dieu à témoin sur mon âme, que c’est pour vous épargner que je ne suis plus allé à Corinthe.
\VS{24}Non que nous dominions sur votre foi, mais nous contribuons à votre joie, puisque vous demeurez fermes dans la foi.
\Chap{2}
\TextTitle{Les fruits de la repentance}
\VerseOne{}Je résolus en moi-même de ne pas retourner chez vous avec tristesse.
\VS{2}Car si je vous attriste, qui peut me réjouir, sinon celui que j'aurai moi-même affligé ?
\VS{3}Je vous ai écrit ceci, pour ne pas éprouver à mon arrivée de la tristesse de la part de ceux de qui je devais recevoir de la joie, ayant en vous tous cette confiance que ma joie est la vôtre à tous.
\VS{4}Je vous ai écrit dans une grande affliction et angoisse de cœur, avec beaucoup de larmes, non pas afin que vous soyez attristés, mais afin que vous connaissiez la charité\FTNT{Littéralement «~agape~»~: amour fraternel.} toute particulière que j'ai pour vous.
\VS{5}Si quelqu'un a été la cause de cette tristesse, ce n'est pas moi seul qu'il a attristé, afin que je ne le surcharge point, mais en quelque sorte c'est vous tous.
\VS{6}C'est assez pour cet homme, de la correction qui lui a été faite par plusieurs,
\VS{7}en sorte que vous devez bien plutôt lui pardonner et le consoler, de peur qu’il ne soit accablé par une trop grande tristesse.
\VS{8}C'est pourquoi je vous prie de confirmer envers lui votre charité.
\VS{9}C’est aussi pour cela que je vous ai écrit, afin de vous éprouver, et de connaître si vous êtes obéissants en toutes choses.
\VS{10}Or à celui à qui vous pardonnez quelque chose, je pardonne aussi ; si j’ai pardonné à celui à qui j'ai pardonné, c’est à cause de vous, en présence de Christ,
\VS{11}afin que Satan n'ait pas le dessus sur nous, car nous n'ignorons pas ses machinations.
\VS{12}Au reste, lorsque je fus arrivé à Troas pour l'Evangile de Christ, quoique la porte m'y fût ouverte par le Seigneur, je n’eus point de repos en mon esprit, parce que je ne trouvai pas Tite, mon frère ;
\VS{13}mais ayant pris congé d'eux, je partis pour la Macédoine.
\VS{14}Grâces soient rendues à Dieu, qui nous fait toujours triompher en Christ, et qui manifeste par nous l'odeur de sa connaissance en tout lieu.
\VS{15}Nous sommes, en effet, pour Dieu le parfum de Christ, parmi ceux qui sont sauvés, et parmi ceux qui périssent :
\VS{16}Aux uns, une odeur mortelle, pour la mort ; aux autres, une odeur vivifiante, pour la vie. Mais qui est suffisant pour ces choses ?
\VS{17}Car nous ne falsifions pas la parole de Dieu, comme font plusieurs, mais nous parlons de Christ avec sincérité, comme de la part de Dieu, et devant Dieu.
\Chap{3}
\TextTitle{Les corinthiens : lettre de Christ écrite avec l'Esprit du Dieu vivant}
\VerseOne{}Commençons-nous de nouveau à nous recommander nous-mêmes ? Ou avons-nous besoin, comme quelques-uns, de lettres de recommandation auprès de vous, ou de lettres de recommandation de votre part ?
\VS{2}Vous êtes vous-mêmes notre lettre, écrite dans nos cœurs, connue et lue de tous les hommes.
\VS{3}Car il est manifeste que vous êtes la lettre de Christ, écrite par notre ministère, non avec de l'encre, mais avec l'Esprit du Dieu vivant, non sur des tables de pierre, mais sur les tables de chair, qui sont vos cœurs.
\VS{4}Or nous avons une telle confiance en Dieu par Christ.
\VS{5}Non que nous soyons capables de nous-mêmes de penser quelque chose, comme de nous-mêmes, mais notre capacité vient de Dieu.
\TextTitle{Paul ministre de la nouvelle alliance}
\VS{6}Lequel nous a rendus capables d'être ministres de la nouvelle alliance\FTNT{Dans la plupart des versions, le mot grec «~diatheke~» a été traduit par «~testament~» alors que ce mot signifie aussi «~alliance~». On le retrouve notamment dans les passages suivants~: Mt. 26:28~; Mc. 14:24~; Lu. 1:72~; 22:20~; Ac. 3:25~; 7:8~; Ro. 9:4~; 11:27~; 1 Co. 11:25~; Ga. 3:15,17~; 4:24~; Ep. 2:12~; Hé. 7:22~; 8:6,8-9~; 9:4,15-17,20~; 10:16,29~; 12:24~; 13:20~; Ap. 11:19. Le fait d’avoir regroupé les écrits de Genèse à Malachie sous l’appellatif «~Ancien Testament~» a induit beaucoup de chrétiens en erreur. L’Ancienne Alliance correspond uniquement à la loi cérémonielle de Moïse qui a été accomplie par Christ à la croix (Jn. 19:30). Ainsi, avant la mort du Seigneur, on ne peut pas parler de testament puisqu’il faut qu’il y ait au préalable la mort du testateur. Or il est évident que les animaux sacrifiés sous la loi ne nous ont rien légué (Hé. 9:1-16).}, non de la lettre, mais de l'Esprit ; car la lettre tue, mais l'Esprit vivifie.
\VS{7}Or si le ministère de la mort, écrit sur des lettres, et gravé avec des pierres, a été glorieux au point que les enfants d'Israël ne pouvaient regarder fixement le visage de Moïse, à cause de la gloire de son visage, bien que cette gloire devait disparaître,
\VS{8}comment le ministère de l'Esprit ne sera-t-il pas plus glorieux ?
\VS{9}Car si le ministère de la condamnation a été glorieux, le ministère de la justice le surpasse de beaucoup en gloire.
\VS{10}Et même ce premier ministère qui a été si glorieux, ne l'a pas été en comparaison du second qui le surpasse de beaucoup en gloire.
\VS{11}Car si ce qui devait disparaître a été glorieux, ce qui est permanent est beaucoup plus glorieux.
\VS{12}Ayant donc une telle espérance, nous usons d'une grande liberté,
\VS{13}et pas comme Moïse qui mettait un voile sur son visage, afin que les enfants d'Israël ne fixassent pas les yeux sur la fin de ce qui devait disparaître.
\VS{14}Mais ils sont devenus durs d’entendement. Car jusqu'à aujourd'hui, ce même voile qui n’est ôté que par Christ, demeure quand ils font la lecture de l'ancienne alliance.
\VS{15}Jusqu'à ce jour, quand on lit Moïse, un voile est jeté sur leur cœur.
\VS{16}Mais lorsque les cœurs se convertissent au Seigneur, le voile est ôté.
\VS{17}Or le Seigneur c'est l'Esprit ; et là où est l'Esprit du Seigneur, là est la liberté.
\VS{18}Nous tous qui contemplons comme dans un miroir la gloire du Seigneur à visage découvert, nous sommes transformés en la même image, de gloire en gloire, comme par l'Esprit du Seigneur.
\Chap{4}
\TextTitle{La vérité pratique de son ministère}
\VerseOne{}C'est pourquoi, ayant ce ministère selon la miséricorde que nous avons reçue, nous ne nous relâchons point.
\VS{2}Mais nous avons entièrement rejeté les choses honteuses que l'on cache, ne marchant point avec ruse, et ne falsifiant point la parole de Dieu, mais nous rendant approuvés à toute conscience des hommes devant Dieu, par la manifestation de la vérité.
\VS{3}Que si notre Evangile est encore voilé, il ne l'est que pour ceux qui périssent ;
\VS{4}pour les incrédules dont le dieu de ce siècle a aveuglé l’esprit, afin qu’ils ne soient pas éclairés par la lumière de l'Evangile de la gloire de Christ, lequel est l'image de Dieu.
\VS{5}Car nous ne nous prêchons pas nous-mêmes mais nous prêchons Jésus-Christ le Seigneur, et nous déclarons que nous sommes vos serviteurs pour l'amour de Jésus.
\VS{6}Car Dieu qui a dit que la lumière resplendisse des ténèbres\FTNT{Ge. 1:3.}, est celui qui a resplendi dans nos coeurs, pour manifester la connaissance de la gloire de Dieu en la présence de Jésus-Christ.
\VS{7}Mais nous avons ce trésor dans des vases de terre, afin que l'excellence de cette puissance soit de Dieu, et non pas de nous.
\TextTitle{Les souffrances de Paul}
\VS{8}Etant affligés à tous égards, mais non réduits entièrement à l’extrémité ; étant en perplexité, mais non sans secours ;
\VS{9}étant persécutés, mais non abandonnés ; étant abattus, mais non perdus ;
\VS{10}portant toujours partout dans notre corps la mort du Seigneur Jésus, afin que la vie de Jésus soit aussi manifestée dans notre corps.
\VS{11}Car nous qui vivons, nous sommes sans cesse livrés à la mort pour l'amour de Jésus, afin que la vie de Jésus soit aussi manifestée dans notre chair mortelle.
\VS{12}De sorte que la mort agit en nous, et la vie agit en vous.
\VS{13}Or ayant un même esprit de foi, selon qu'il est écrit : J'ai cru, c'est pourquoi j'ai parlé\FTNT{Ps. 116:10.} ! Nous croyons aussi , et c'est aussi pourquoi nous parlons,
\VS{14}sachant que celui qui a ressuscité le Seigneur Jésus nous ressuscitera aussi par Jésus, et nous fera comparaître en sa présence avec vous.
\VS{15}Car toutes ces choses sont pour vous, afin que cette grâce surabonde, à la gloire de Dieu, les actions de grâces d’un très grand nombre.
\VS{16}C'est pourquoi nous ne nous relâchons pas. Mais quoique notre homme extérieur se détruit, toutefois l'intérieur est renouvelé de jour en jour.
\VS{17}Car nos légères afflictions du moment, produisent pour nous, au-delà de toute mesure, un poids éternel d'une gloire souverainement excellente,
\VS{18}quand nous ne regardons point aux choses visibles, mais aux invisibles ; car les choses visibles ne sont que pour un temps, mais les invisibles sont éternelles.
\Chap{5}
\TextTitle{Ses ambitions}
\VerseOne{} Car nous savons que si notre habitation terrestre, qui n'est qu'une tente, est détruite, nous avons un édifice de Dieu qui n’a pas été fait de main d’homme, une maison éternelle dans les cieux.
\VS{2}Car c'est aussi pour cela que nous gémissons, désirant avec ardeur d'être revêtus de notre domicile qui est du ciel,
\VS{3}si toutefois nous sommes trouvés vêtus, et non pas nus.
\VS{4}Car nous qui sommes dans cette tente, nous gémissons, accablés, parce que nous désirons, non pas d'être dépouillés, mais d'être revêtus, afin que ce qui est mortel soit englouti par la vie.
\VS{5}Et celui qui nous a formés pour cela c'est Dieu qui nous a donné les arrhes de l'Esprit.
\VS{6}Nous avons donc toujours confiance ; et nous savons que logeant dans ce corps, nous demeurons loin du Seigneur,
\VS{7}car nous marchons par la foi, et non par la vue.
\VS{8}Nous avons, dis-je, de la confiance, et nous aimons mieux être absents de ce corps, et être avec le Seigneur.
\VS{9}C'est pourquoi aussi nous nous efforçons de lui être agréables, et présents, et absents.
\VS{10}Car il nous faut tous comparaître devant le tribunal\FTNT{Le Tribunal de Christ n’a pas vocation à déterminer le salut des enfants de Dieu. Les chrétiens y seront jugés en fonction des œuvres produites sur la terre. En effet, chacun devra rendre compte de ce qu’il aura fait et de la gestion des dons et ministères reçus. Voir Ro. 14:10~; 1 Co. 4:4~; 2 Co. 3:10-14~; 2 Tim. 4:8.} de Christ, afin que chacun reçoive en son corps selon ce qu'il aura fait, soit bien, soit mal.
\TextTitle{Ses motifs d'action}
\VS{11}Connaissant donc combien le Seigneur doit être craint, nous persuadons les hommes; et nous sommes connus de Dieu, et j’espère que dans vos consciences vous nous connaissez aussi.
\VS{12}Car nous ne nous recommandons pas de nouveau à vous, mais nous vous donnons l'occasion de vous glorifier à notre sujet, afin que vous ayez de quoi répondre à ceux qui se glorifient de l'apparence, et non pas de ce qui est dans le cœur.
\VS{13}Car soit que nous soyons hors de sens, c’est pour Dieu ; soit que nous soyons de bon sens c’est pour vous.
\VS{14}Car la charité de Christ nous lie, parce que nous jugeons que, si un est mort pour tous, tous aussi sont morts ;
\VS{15}et qu'il est mort pour tous, afin que ceux qui vivent, ne vivent plus pour eux-mêmes, mais pour celui qui est mort et ressuscité pour eux.
\VS{16}C'est pourquoi dès maintenant nous ne connaissons personne selon la chair ; et si nous avons connu Christ selon la chair, maintenant nous ne le connaissons plus ainsi.
\VS{17}Si donc quelqu'un est en Christ, il est une nouvelle créature ; les choses anciennes sont passées ; voici, toutes choses sont faites nouvelles.
\VS{18}Et tout cela vient de Dieu, qui nous a réconciliés avec lui par Jésus-Christ, et qui nous a donné le ministère de la réconciliation.
\VS{19}Car Dieu était en Christ, réconciliant le monde avec lui-même, en ne leur imputant point leurs péchés, et il a mis en nous la parole de la réconciliation.
\VS{20}Nous sommes donc ambassadeurs pour Christ, c'est comme si Dieu vous exhortait par notre ministère ; nous vous supplions donc pour l’amour de Christ : Réconciliez-vous avec Dieu !
\VS{21}Car il a fait celui qui n'a point connu de péché, être péché pour nous, afin que nous soyons en lui justice de Dieu.
\Chap{6}
\TextTitle{Son humilité}
\VerseOne{}Puisque nous travaillons avec le Seigneur, nous vous prions de ne pas recevoir la grâce de Dieu en vain.
\VS{2}Car il dit : Je t'ai exaucé au temps favorable et t'ai secouru au jour du salut\FTNT{Es. 49:8.} ; voici maintenant le temps favorable, voici maintenant le jour du salut.
\VS{3}Ne donnant aucun scandale en quoi que ce soit, afin que notre ministère ne soit point blâmé.
\VS{4}Mais nous rendant recommandables, en toutes choses, comme ministres de Dieu, en grande patience, en afflictions, en nécessités, en détresses,
\VS{5}sous les coups, dans les prisons, dans les troubles, dans les travaux, dans les veilles, dans les jeûnes,
\VS{6}par la pureté, par la connaissance, par la persévérance, par la douceur, par le Saint-Esprit, par une charité sincère,
\VS{7}par la parole de vérité, par la puissance de Dieu, par les armes de justice que l'on porte à la main droite et à la main gauche ;
\VS{8}au milieu de la gloire et de l'ignominie, au milieu de la mauvaise et de la bonne réputation ; étant regardés comme des séducteurs, quoique véridiques,
\VS{9}comme inconnus, quoique bien connus, comme mourants, et voici nous vivons, comme châtiés, quoique non mis à mort,
\VS{10}comme attristés, et nous sommes toujours joyeux, comme pauvres, et nous en enrichissons plusieurs, comme n'ayant rien, et nous possédons toutes choses.
\TextTitle{Appel à la séparation et à la purification}
\VS{11}Ô Corinthiens ! Notre bouche s’est ouverte pour vous, notre cœur s'est élargi.
\VS{12}Vous n’y êtes point à l'étroit, mais c’est votre cœur qui s’est rétréci pour nous.
\VS{13}Rendez-nous la pareille (je vous parle comme à mes enfants) élargissez aussi votre cœur !
\VS{14}Ne portez pas un même joug avec les infidèles ; car quelle communion y a-t-il entre justice et l'iniquité ? Ou qu’y a-t-il de commun entre la lumière et les ténèbres ?
\VS{15}Et quel accord y a-t-il entre Christ et Bélial\FTNT{Bélial~: De l’hébreu «~beliya’al~»~: méchants, pervers, pervertis, vil, destruction, dangereusement. L’un des noms de Satan qui signifie «~indignité, méchanceté, impiété~».} ? Ou quelle part a le fidèle avec l'infidèle ?
\VS{16}Et quel rapport y a-t-il entre le temple de Dieu et les idoles ? Car vous êtes le temple du Dieu vivant, selon ce que Dieu a dit : J'habiterai au milieu d'eux et j'y marcherai ; je serai leur Dieu, et ils seront mon peuple\FTNT{Lé. 26:12~; Ez. 37:26.}.
\VS{17}C'est pourquoi sortez du milieu d'eux, et séparez-vous, dit le Seigneur ; ne touchez pas à ce qui est impur, et je vous accueillerai\FTNT{Es. 52:11~; Ap. 18:4.}.
\VS{18}Je serai pour vous un Père, et vous serez pour moi des fils et des filles, dit le Seigneur Tout-Puissant\FTNT{Jn. 1:12~; Ap. 21:7.}.
\Chap{7}
\VerseOne{}Or donc mes bien-aimés, puisque nous avons de telles promesses, nettoyons-nous de toute souillure de la chair et de l'esprit, perfectionnant la sanctification dans la crainte de Dieu.
\TextTitle{Paul ouvre son coeur aux Corinthiens}
\VS{2}Recevez-nous, nous n'avons fait tort à personne, nous n'avons corrompu personne, nous n'avons pillé personne.
\VS{3}Je ne dis pas ceci pour vous condamner, car je vous ai déjà dit que vous êtes dans nos cœurs à la vie et à la mort.
\VS{4}J'ai une grande liberté envers vous, j'ai grand sujet de me glorifier de vous ; je suis rempli de consolation, je suis comblé de joie au milieu de toutes nos afflictions.
\VS{5}Car depuis notre arrivée en Macédoine, notre chair n’eut aucun repos, mais nous avons été affligés de toute manière, ayant eu des combats au dehors, et des craintes au dedans.
\VS{6}Mais Dieu qui console les abattus nous a consolés par l’arrivée de Tite,
\VS{7}et non seulement par son arrivée, mais aussi par la consolation qu'il a reçue de vous ; car il nous a raconté votre grand désir, vos larmes, votre affection ardente pour moi, en sorte que je m'en suis extrêmement réjoui.
\VS{8}Quoique je vous aie attristés par ma lettre, je ne m'en repens pas. Et si je m'en suis repenti, car je vois que cette lettre vous a affligés, bien que momentanément,
\VS{9}je me réjouis à présent, non de ce que vous avez été affligés, mais de ce que votre tristesse vous a portés à la repentance ; car vous avez été attristés selon Dieu, de sorte que vous n'avez reçu aucun dommage de notre part.
\VS{10}En effet, la tristesse selon Dieu produit une repentance à salut dont on ne se repent jamais, mais que la tristesse du monde produit la mort.
\VS{11}En effet, cette tristesse qui est selon Dieu, quel empressement n’a-t-elle pas produit en vous ! quelle justification, quelle indignation, quelle crainte, quel grand désir, quel zèle, quelle vengeance! Vous  vous êtes montrés de toutes manières purs dans cette affaire.
\VS{12}Quoi que je vous aie écrit, ce n'était ni à cause de celui qui a commis la faute, ni à cause de celui envers qui elle a été commise, mais pour faire voir parmi vous l'empressement que j'ai de vous devant Dieu.
\VS{13}C'est pourquoi nous avons été consolés de ce que vous avez fait pour notre consolation. Mais nous nous sommes encore plus réjouis de la joie qu'a eu Tite, en ce que son esprit a été tranquillisé par vous tous.
\VS{14}Et si en quelque chose je me suis glorifié de vous devant lui, je n’en ai point eu de confusion ; mais, comme nous avons toujours parlé selon la vérité, ce dont nous nous sommes glorifiés auprès de Tite s’est trouvé être aussi la vérité.
\VS{15}C'est pourquoi quand il se souvient de l'obéissance de vous tous, et comment vous l'avez reçu avec crainte et tremblement, son affection pour vous en est beaucoup plus grande.
\VS{16}Je me réjouis d'avoir confiance en vous en toutes choses.
\Chap{8}
\TextTitle{Exemple des Macédoniens concernant la collecte en faveur des pauvres de Jérusalem}
\VerseOne{}Au reste, mes frères, nous voulons vous faire connaître la grâce que Dieu a faite aux églises de la Macédoine.
\VS{2}Au travers de leur grande épreuve d'affliction, leur joie a été augmentée, et leur profonde pauvreté s'est répandue en richesses par leur prompte libéralité.
\VS{3}Car je suis témoin qu'ils ont donné volontairement selon leurs moyens, et même au-delà de leurs moyens,
\VS{4}Nous pressant avec de grandes prières de recevoir la grâce et de prendre part à cette contribution en faveur des Saints.
\VS{5}Et ils n'ont pas fait seulement comme nous l'espérions, mais ils se sont donnés premièrement eux-mêmes au Seigneur, et puis à nous, par la volonté de Dieu.
\VS{6}Nous avons exhorté Tite, comme il avait auparavant commencé, d'achever aussi cette grâce envers vous.
\TextTitle{Exemple du Messie}
\VS{7}C'est pourquoi, comme vous excellez en toutes choses, en foi, en parole, en connaissance, en toute diligence, et dans la charité que vous avez pour nous, faites en sorte d’exceller aussi dans cette œuvre de charité.
\VS{8}Je ne dis pas cela pour vous donner un ordre, mais pour éprouver par l’empressement des autres la sincérité de votre charité.
\VS{9}Car vous connaissez la grâce de notre Seigneur Jésus-Christ qui, étant riche, s'est fait pauvre pour vous, afin que par sa pauvreté vous soyez enrichis.
\VS{10}C’est un avis que je donne là-dessus, parce qu'il vous est convenable, à vous qui non seulement avez commencé à agir, mais en ayant même eu la volonté dès l'année passée.
\VS{11}Achevez donc maintenant d'agir, afin que comme vous avez été prompts à en avoir la volonté ; vous l'accomplissiez aussi selon vos moyens.
\VS{12}Car si la promptitude de la volonté existe,  on est agréable selon ce qu'on a, et non point selon ce qu'on n'a pas.
\VS{13}Je ne veux pas vous exposer à la détresse pour soulager les autres, mais suivre une règle d’égalité. Dans la circonstance présente, votre superflu pourvoira à leurs besoins,
\VS{14}afin que leur superflu pourvoie pareillement aux vôtres, en sorte qu’il y ait égalité,
\VS{15}selon ce qui est écrit : Celui qui avait beaucoup n'a rien eu de superflu, et celui qui avait peu n'en a pas eu moins.\FTNT{Ex. 16:18.}.
\TextTitle{Exemple des églises}
\VS{16}Grâces soient rendues à Dieu qui a mis dans le cœur de Tite le même empressement pour vous ;
\VS{17}lequel a bien reçu mon exhortation, et c'est avec un nouveau zèle et de son plein gré qu'il part pour aller chez vous.
\VS{18}Et nous avons aussi envoyé avec lui le frère dont la louange dans l'Evangile est répandue par toutes les églises ;
\VS{19}de plus, il a été choisi par élection des églises pour être notre compagnon de voyage et pour cette grâce\FTNT{ également traduit par «~les aumônes~»} qui est administrée par nous à la gloire du Seigneur même, et afin de répondre à l’ardeur de votre zèle. %et pour servir à la promptitude de votre zèle.
\VS{20}Evitant ainsi que personne ne nous blâme dans cette abondante collecte, qui est administrée par nous;
\VS{21}ayant soin de faire ce qui est bon, non seulement devant le Seigneur, mais aussi devant les hommes.
\VS{22}Nous avons envoyé aussi avec eux notre autre frère, dont nous avons souvent éprouvé le zèle à plusieurs occasions, qui est maintenant encore plus zélé, à cause de la grande confiance qu'il a en vous.
\VS{23}Ainsi donc, quant à Tite, il est mon associé et mon compagnon d’œuvre auprès de vous ; et quant à nos frères, ils sont les envoyés des églises, la gloire de Christ.
\VS{24}Montrez donc envers eux et devant les églises une preuve de votre charité, et du sujet que nous avons de nous glorifier de vous.
\Chap{9}
\TextTitle{Encouragement par rapport aux dons}
\VerseOne{}Il est superflu que je vous écrive touchant la collecte destinée aux saints.
\VS{2}Car je vois la promptitude de votre zèle, dont je me glorifie de vous devant ceux de Macédoine, leur disant que l’Achaïe est prête dès l'année dernière ; et votre zèle en a excité plusieurs.
\VS{3}J’ai envoyé ces frères, afin que ce en quoi je me suis glorifié de vous, ne soit pas vain en cette occasion, et que vous soyez prêts, comme j'ai dit.
\VS{4}De peur que ceux de Macédoine venant avec moi, et ne vous trouvant pas prêts, nous, pour ne pas dire vous, n'ayons de honte de l'assurance dont nous nous sommes glorifiés.
\VS{5}C'est pourquoi j'ai estimé nécessaire de prier les frères à se rendre premièrement vers vous, et d'achever de préparer votre bienfait déjà promis, afin qu'il soit prêt, comme un bienfait, et non comme de l'avarice.
\VS{6}Au reste, je vous avertis que celui qui sème peu moissonnera peu, et celui qui sème abondamment moissonnera abondamment.
\VS{7}Mais que chacun contribue selon qu'il se l'est proposé en son cœur, non à regret, ou par contrainte; car Dieu aime celui qui donne avec joie.
\VS{8}Et Dieu est Tout-Puissant pour vous combler de toutes sortes de grâces, afin qu'ayant toujours tout ce qui suffit en toute chose, vous soyez abondants en toute bonne œuvre,
\VS{9}selon ce qui est écrit : Il a fait des largesses, il a donné aux pauvres ; sa justice demeure éternellement\FTNT{Ps. 112:9.}.
\VS{10}Que celui qui fournit de la semence au semeur veuille aussi vous donner du pain à manger, et multiplier votre semence, et augmenter les revenus de votre justice ;
\VS{11}afin que vous soyez pleinement enrichis pour exercer une parfaite libéralité, laquelle fait que nous en rendons grâces à Dieu.
\VS{12}Car le service de cette assistance est non seulement suffisant pour subvenir aux nécessités des Saints, mais il abonde aussi de telle sorte, que plusieurs ont de quoi en rendre grâces à Dieu.
\VS{13}Glorifiant Dieu pour l’épreuve qu’ils font de cette assistance, en ce que vous vous soumettez à l’Évangile de Christ ; et de votre prompte et libérale communication envers eux, et envers tous.
\VS{14}ils prient Dieu pour vous, et ils vous aiment\FTNT{Aimer~: Du grec «~epipotheo~»~: désirer, chérir.} très affectueusement à cause de la grâce excellente que Dieu vous a accordée.
\VS{15}Grâces soient rendues à Dieu pour son don inexprimable.
\Chap{10}
\TextTitle{Paul défend son autorité apostolique}
\VerseOne{}Au reste, je vous prie, moi Paul, par la douceur et la bonté de Christ - moi qui parais méprisable lorsque je suis en votre présence, et plein de hardiesse quand je suis éloigné,
\VS{2}je vous prie, dis-je, que lorsque je serai présent, il ne faille point que j'use de hardiesse, laquelle je me propose d’user contre quelques-uns qui nous regardent comme marchant selon la chair.
\VS{3}Mais en marchant dans la chair, nous ne combattons pas selon la chair.
\VS{4}Car les armes de notre guerre ne sont pas charnelles, mais elles sont puissantes par la vertu de Dieu, pour la destruction des forteresses.
\VS{5}Détruisant les raisonnements et toute hauteur qui s'élèvent contre la connaissance de Dieu, et amenant toute pensée captive à l'obéissance de Christ.
\VS{6}Et étant prêts à tirer vengeance de toute désobéissance, lorsque votre obéissance sera complète.
\VS{7}Considérez-vous les choses selon l'apparence ? Si quelqu'un se persuade qu’il est de Christ, qu'il se dise bien en lui-même que, comme il est de Christ, nous aussi nous sommes de Christ.
\VS{8}Car si même je veux me glorifier davantage de l’autorité que le Seigneur nous a donnée pour votre édification et non votre destruction, je ne saurais en avoir honte,
\VS{9}afin que je ne paraisse pas vouloir vous effrayer par mes lettres.
\VS{10}Car mes lettres, disent-ils, sont graves et fortes, mais la présence de corps est faible, et la parole est méprisable.
\VS{11}Que celui qui est tel, considère que tels nous sommes en paroles dans nos lettres, étant absents, tels aussi nous sommes dans nos actes, étant présents.
\VS{12}Car nous n'osons pas nous joindre ni nous comparer à quelques-uns de ceux qui se recommandent eux-mêmes. Mais en se mesurant à leur propre mesure et en se comparant à eux-mêmes, ils manquent d’intelligence.
\VS{13}Mais pour nous, nous ne voulons pas nous glorifier outre mesure, mais seulement dans la limite du champ d’action que Dieu nous assigné en nous amenant jusqu’à vous.
\VS{14}Car nous ne nous étendons pas nous même au delà des limites prescrites, comme si nous n'étions pas parvenus jusqu'à vous ; vu que nous sommes parvenus même jusqu’à vous par la prédication de l'Evangile de Christ.
\VS{15}Nous ne nous glorifions pas des travaux d’autrui qui sont hors de nos limites. Mais nous avons l’espérance, si votre foi augmente, de devenir encore plus grands parmi vous, selon les limites qui nous sont assignées,
\VS{16}jusqu'à évangéliser dans les lieux qui sont au-delà de chez vous, sans nous glorifier de ce qui a déjà été fait dans le domaine des autres.
\VS{17}Que celui qui se glorifie se glorifie dans le Seigneur.
\VS{18}Car ce n'est pas celui qui se recommande lui-même qui est approuvé, c'est celui que le Seigneur recommande\FTNT{Le Seigneur recommande ses serviteurs, il témoigne d’eux auprès des autres (Ac. 10:1-48). Un véritable serviteur de Dieu laisse au Seigneur le soin de témoigner de lui auprès des autres alors que les faux ouvriers se recommandent eux-mêmes (2 Co. 3:1).}.
\Chap{11}
\VerseOne{}Oh ! Si vous pouviez supportez de ma part un peu de folie ! Mais vous me supportez !
\VS{2}Car je suis jaloux de vous d'une jalousie de Dieu, parce que je vous ai fiancés à un seul Epoux, pour vous présenter à Christ comme une vierge pure.
\TextTitle{Les faux docteurs}
\VS{3}Mais je crains que comme le serpent séduisit Eve\FTNT{Ge. 3:1-6.} par sa ruse, vos pensées aussi ne se corrompent en se détournant de la simplicité à l’égard de Christ.
\VS{4}Car si quelqu'un vient vous prêcher un autre Jésus que nous n'avons pas prêché, ou si vous recevez un autre esprit que celui que vous avez reçu, ou un autre évangile que celui que vous avez embrassé, vous le supportez fort bien.
\VS{5}Or j'estime que je n'ai été en rien moindre que les plus excellents apôtres.
\VS{6}Si je suis un ignorant sous le rapport du langage, je ne le suis pourtant point sous celui de la connaissance, et nous l’avons montré parmi vous à tous égards et en toutes choses.
\VS{7}Ai-je commis une faute, en m’abaissant moi-même afin que vous soyez élevés, quand je vous ai annoncé gratuitement l’Evangile de Dieu ?
\VS{8}J'ai dépouillé les autres églises prenant de quoi m'entretenir pour vous . Et lorsque j’étais chez vous et que je me suis trouvé dans le besoin, je n’ai été à la charge de personne,
\VS{9}car les frères venus de la Macédoine ont pourvu à ce qui me manquait. Et en toutes choses, je me suis gardé d’être à votre charge, et je m'en garderai encore.
\VS{10}Par la vérité de Christ qui est en moi, j’atteste que ce sujet de gloire ne me sera point ravi dans les contrées de l'Achaïe.
\VS{11}Pourquoi ? Est-ce parce que je ne vous aime point ? Dieu le sait !
\VS{12}Mais ce que je fais, je le ferai encore, pour ôter ce prétexte à ceux qui cherchent un prétexte, afin qu’ils soient trouvés tels que nous dans les choses dont ils se glorifient.
\VS{13}Car ces hommes-là sont de faux apôtres, des ouvriers trompeurs qui se déguisent en apôtres de Christ.
\VS{14}Et cela n'est pas étonnant puisque Satan lui-même se déguise en ange de lumière\FTNT{Satan est maître en matière de déguisement et d’imitation.}.
\VS{15}Ce n'est donc pas un grand sujet d'étonnement si ses ministres aussi se déguisent en ministres de justice ; mais leur fin sera conforme à leurs œuvres.
\TextTitle{Sujets de gloire de Paul\FTNTT{2 Co. 11:16-12:18}}
\VS{16}Je le dis encore, afin que personne ne me regarde comme un insensé ; sinon, supportez-moi comme un insensé, afin que je me glorifie aussi un peu.
\VS{17}Ce que je dis, avec l’assurance d’avoir sujet de me glorifier, je ne le dis pas selon le Seigneur, mais comme par folie.
\VS{18}Puisqu’il en est plusieurs qui se glorifient selon la chair, je me glorifierai aussi.
\VS{19}Car vous supportez bien volontiers les insensés, vous qui êtes sages.
\VS{20}Si quelqu'un vous asservit, si quelqu'un vous dévore, si quelqu'un prend votre bien, si quelqu'un est arrogant, si quelqu'un vous frappe au visage, vous le supportez.
\VS{21}Je le dis avec honte, nous avons montré de la faiblesse. Mais si en quelque chose quelqu'un ose se glorifier, je parle en insensé, j'ai la même hardiesse !
\VS{22}Sont-ils Hébreux ? Moi aussi. Sont-ils Israélites ? Moi aussi. Sont-ils de la postérité d'Abraham ? Moi aussi.
\VS{23}Sont-ils ministres de Christ ? - je parle comme un insensé - je le suis plus qu'eux ; par les travaux, bien plus ; par les blessures, bien plus ; par les emprisonnements, bien plus. Plusieurs fois en danger de mort,
\VS{24}cinq fois j’ai reçu des Juifs quarante coups moins un,
\VS{25}j'ai été battu de verges trois fois, j'ai été lapidé une fois, j'ai fait naufrage trois fois, j'ai passé un jour et une nuit dans l’abîme.
\VS{26}Fréquemment en voyage, j’ai été en péril sur les fleuves, en péril de la part des brigands, en péril de la part de ceux de ma nation, en péril de la part des gentils, en péril dans les villes, en péril dans les déserts, en péril sur la mer, en péril parmi de faux frères.
\VS{27}J’ai été dans le travail et dans la peine, exposé à de nombreuses veilles, à la faim et à la soif, à des jeûnes multipliés, au froid et à la nudité.
\VS{28}Outre les choses de dehors, ce qui me tient assiégé tous les jours, c'est le soucis que j'ai de toutes les églises.
\VS{29}Qui est affaibli, que je ne sois faible ? Qui est scandalisé, que je n'en sois aussi brûlé ?
\VS{30}S'il faut se glorifier, je me glorifierai des choses qui sont de mon infirmité.
\VS{31}Le Dieu et Père de notre Seigneur Jésus-Christ, lui qui est béni éternellement, sait que je ne mens point.
\VS{32}A Damas, le gouverneur du roi Arétas avait fait garder la ville des Damascéniens pour me prendre,
\VS{33}mais on me descendit par une fenêtre, dans une corbeille, le long de la muraille, et ainsi j'échappai de ses mains.
\Chap{12}
\VerseOne{}Certes, il ne me convient pas de me glorifier, car j’en viendrai jusqu’aux visions et aux révélations du Seigneur.
\VS{2}Je connais un homme en Christ, qui fut ravi jusqu’au troisième ciel, il y a quatorze ans passés, (si ce fut dans son corps, je ne sais pas ; si ce fut hors du corps, je ne sais pas ; Dieu le sait).
\VS{3}Et je sais que cet homme (si ce fut dans son corps, ou si ce fut hors du corps, je ne sais pas ; Dieu le sait),
\VS{4}fut ravi dans le paradis, et qu’il entendit des paroles inexprimables qu'il n'est pas permis à l'homme de révéler.
\TextTitle{Paul et son écharde}
\VS{5}Je me glorifierai d'un tel homme, mais je ne me glorifierai point de moi-même, sinon de mes infirmités.
\VS{6}Si je voulais me glorifier, je ne serais pas un insensé, car je dirais la vérité ; mais je m'en abstiens, afin que personne ne m'estime au-dessus de ce qu'il me voit être, ou de ce qu'il entend dire de moi.
\VS{7}Mais pour que je ne sois pas enflé d’orgueil, à cause de l'excellence de ces révélations, il m'a été mis une écharde\FTNT{La nature exacte de l'écharde de Paul ne nous est pas détaillée. Elle lui avait été infligée par un «~ange de Satan~», par la volonté de Dieu. Nous constatons une chose qui est commune à tous les enfants de Dieu~: Paul avait un adversaire constamment aux aguets pour essayer de le décourager, le détruire ou l’intimider en s'opposant par tous les moyens à la mission que le Seigneur lui avait confiée. Cette écharde était aussi un moyen utilisé par Dieu pour garder Paul dans l’humilité.} dans la chair, un ange de Satan pour me souffleter et m’empêcher de m’enorgueillir.
\VS{8}Trois fois j'ai prié le Seigneur de faire que cet ange de Satan se retire de moi.
\VS{9}Mais le Seigneur m'a dit : Ma grâce te suffit, car ma puissance s’accomplit dans la faiblesse. Je me glorifierai donc bien volontiers de mes faiblesses, afin que la puissance de Christ habite en moi.
\VS{10}C’est pourquoi je me plais dans les faiblesses, dans les outrages, dans les calamités, dans les persécutions, et dans les angoisses pour Christ ; car quand je suis faible, c'est alors que je suis fort.
\TextTitle{Avertissements}
\VS{11}J'ai été insensé en me glorifiant, mais vous m'y avez contraint ; c’est par vous que je devais être recommandé, car je n'ai été inférieur en rien aux apôtres par excellence, quoique je ne sois rien.
\VS{12}Certainement les preuves de mon apostolat ont éclaté au milieu de vous par une patience à toute épreuve, par des signes, des prodiges et des miracles.
\VS{13}Car en quoi avez-vous été inférieurs aux autres églises, sinon en ce que je n’ai point été à votre charge ? Pardonnez-moi ce tort.
\VS{14}Voici pour la troisième fois que je suis prêt à aller vers vous, et je ne serai point à votre charge ; car ce ne sont pas vos biens que je cherche, c’est vous-mêmes. Ce n’est pas, en effet, aux enfants d’amasser pour les parents, mais aux parents pour les enfants\FTNT{Un bon père amasse dans le but de préparer l’avenir de ses enfants et non l’inverse.}.
\VS{15}Pour moi, je dépenserai très volontiers pour vous tout ce que j’ai, et je me donnerai encore moi-même pour vos âmes. En vous aimant davantage, serais-je moins aimé de vous ?
\VS{16}Soit ! Dira-t-on, que je ne vous ai point été à charge, c'est qu'étant un homme intelligent, je vous ai pris par ruse !
\VS{17}Ai-je donc tiré profit de vous par quelqu’un de ceux que je vous ai envoyés ?
\VS{18}J'ai engagé Tite à aller chez vous, et avec lui j’ai envoyé le frère. Tite a-t-il tiré profit de vous ? Et n'avons-nous pas lui et moi marché dans le même esprit ? N'avons-nous pas marché sur les mêmes traces ?
\VS{19}Pensez-vous encore que nous voulions nous justifier auprès de vous ? Nous parlons devant Dieu en Christ, et tout cela, mes très chers frères, pour votre édification.
\VS{20}Car je crains de ne pas vous trouver, à mon arrivée, tels que je voudrais, et d’être moi-même trouvé par vous tel que vous ne voudriez pas. Je crains de trouver des querelles, de la jalousie, des animosités, des rivalités, des médisances, des calomnies, de l’orgueil, des troubles.
\VS{21}Je crains qu’à mon arrivée, mon Dieu ne m’humilie de nouveau à votre sujet, et que je n’aie à pleurer sur plusieurs de ceux qui ont péché précédemment, et qui ne se sont pas repentis de l’impureté, de la débauche et des dérèglements dont ils se sont rendus coupables.
\Chap{13}
\TextTitle{S'examiner}
\VerseOne{}Je vais chez vous pour la troisième fois. Toute affaire se réglera sur la déclaration de deux ou de trois témoins\FTNT{De. 19:15.}.
\VS{2}Lorsque j’étais présent pour la deuxième fois, j’ai déjà dit, et aujourd’hui que je suis absent je dis encore d’avance à ceux qui ont péché précédemment et à tous les autres, que si je retourne chez vous, je n'épargnerai personne,
\VS{3}puisque vous cherchez la preuve que Christ parle par moi, lui qui n'est point faible envers vous, mais qui est puissant parmi vous.
\VS{4}Car il a été crucifié à cause de sa faiblesse, mais il vit par la puissance de Dieu ; et nous de même, nous sommes aussi faibles comme lui, mais nous vivrons avec lui par la puissance que Dieu a déployée envers vous.
\VS{5}Examinez-vous vous-mêmes pour savoir si vous êtes dans la foi ; éprouvez-vous vous-mêmes. Ne reconnaissez-vous pas que Jésus-Christ est en vous ? A moins peut-être que vous ne soyez désapprouvés.
\VS{6}Mais j'espère que vous reconnaîtrez que nous, nous ne sommes pas désapprouvés.
\VS{7}Et je prie Dieu que vous ne fassiez rien de mal, non pour paraître nous-mêmes approuvés, mais afin que vous pratiquiez ce qui est bien et que nous, nous soyons comme désapprouvés.
\VS{8}Car nous n’avons pas de pouvoir contre la vérité, nous n’en avons que pour la vérité.
\VS{9}Nous nous réjouissons lorsque nous sommes faibles, tandis que vous êtes forts ; et ce que nous demandons à Dieu, c’est votre perfectionnement.
\VS{10}C'est pourquoi j'écris ces choses étant absent, afin que présent, je n’aie pas à user de rigueur, selon l’autorité que le Seigneur m'a donnée pour l'édification et non point pour la destruction.
\TextTitle{Conclusion}
\VS{11}Au reste, mes frères, réjouissez-vous, perfectionnez-vous, consolez-vous, ayez un même sentiment, vivez en paix ; et le Dieu de charité et de paix sera avec vous.
\VS{12}Saluez-vous les uns les autres par un saint baiser. Tous les saints vous saluent.
\VS{13}Que la grâce du Seigneur Jésus-Christ, la charité de Dieu, et la communion du Saint-Esprit soient avec vous tous. Amen !
\PPE{}
\end{multicols}

%\clearpage\ShortTitle{Romains}\BookTitle{Romains}\BFont
\noindent\hrulefill
{\footnotesize
\textit{
\bigskip
{\centering{}
\\Thème : L'Evangile de Dieu
\\Auteur : Paul
\\Date de rédaction : Env. 56\\}
}
%\bigskip
\textit{
\\Rome est une ville située dans la région du Latium, au centre de l’Italie, à la confluence de l’Aniene et du Tibre. Centre de l’Empire romain,  elle domina  l’Europe, l’Afrique du Nord et le Moyen-Orient du 1er siècle avant J.-C. au 5ème siècle après J.-C.
%\bigskip
\\La lettre était destinée à l’Eglise de Rome, fondée sans doute par des chrétiens convertis au  travers du ministère de Paul et d’autres apôtres itinérants. Cette Eglise comptait quelques juifs mais surtout des membres d’origine païenne. Cette épître fut rédigée au cours du 3ème voyage missionnaire de Paul,  pendant les trois mois que l’apôtre passa à Corinthe. En attendant de leur rendre visite physiquement, Paul avait le désir de communiquer aux chrétiens de Rome les grandes lignes du principe de la grâce, dont il avait eu la révélation. Il y aborda plusieurs doctrines majeures comme le salut par la foi et la grâce ainsi que des enseignements pratiques sur l’amour, le devoir du chrétien et la sainteté.\bigskip
}
}
\par\nobreak\noindent\hrulefill
\begin{multicols}{2}
\Chap{1}
\TextTitle{Introduction : L’Evangile de Christ, puissance de Dieu pour le salut de tous}
\VerseOne{}Paul, serviteur de Jésus-Christ, appelé à être apôtre, mis à part pour annoncer l'Evangile de Dieu,
\VS{2}qu’il avait auparavant promis par ses prophètes dans les saintes Ecritures,
\VS{3}et qui concerne son Fils, qui est né de la postérité de David, selon la chair,
\VS{4}et qui a été pleinement déclaré Fils de Dieu avec puissance, selon l'Esprit de sainteté, par sa résurrection d'entre les morts, c'est-à-dire, Jésus-Christ notre Seigneur,
\VS{5}par qui nous avons reçu la grâce et l’apostolat, pour amener en son Nom tous les gentils à l’obéissance de la foi,
\VS{6}parmi lesquels aussi vous êtes, vous qui êtes appelés par Jésus-Christ.
\VS{7}A vous tous qui êtes à Rome, bien-aimés de Dieu, appelés à être saints\FTNT{Le terme «~saint~» est tiré du grec «~hagios~» qui signifie «~consacré à Dieu~», «~saint~», «~sacré~», «~pieux~». Ce mot est souvent utilisé au pluriel dans le testament de Jésus (Ac. 9:13 ;  Ac. 26:10). Il n’y a aucun rapport entre la compréhension catholique romaine du terme «~saint~» et l’enseignement biblique. Dans l’enseignement catholique romain, une personne ne devient pas sainte tant qu’elle n’a pas été béatifiée ou canonisée par le pape ou l’éminent évêque. Dans la Bible, tous ceux qui reçoivent Jésus-Christ par la foi sont appelés saints car ils sont mis à part pour Dieu. Dans le culte de l’église catholique romaine, les saints sont vénérés, priés, et parfois adorés. Dans la Bible, les saints sont appelés à adorer et à prier Dieu seul par Jésus-Christ homme, le seul Médiateur entre Dieu et les hommes (1 Ti. 2:5). Dans le Tanakh, les mots «~sanctifié~», «~saint~» et leurs dérivés viennent du mot hébreu «~Quodesch~» dont le sens général est : «~Mis à part pour Dieu~». Dans la Bible, ces mots sont appliqués à des objets et à des personnes. Le terme «~sanctification~» appliqué à des objets sous-entend l'idée qu'ils sont réservés uniquement pour le service de Dieu ; ils sont sanctifiés, mis à part pour Dieu. Dans le Testament de Jésus, il est appliqué aux personnes et comprend plusieurs sens : - Les croyants, par leur position, sont éternellement mis à part pour Dieu par la rédemption (Hé. 10:10-14). Ils sont donc considérés comme «~saints~», «~sanctifiés~» dès leur conversion (Ph. 1:1 ; Hé. 3:1). - Les croyants sont amenés à la sanctification par l'action du Saint-Esprit au moyen des Ecritures (Jn. 15:3 ; Jn. 17:17 ; 2 Co. 3:18 ; Ep. 5:25-26 ; 1Th. 5:23-24). - Les croyants attendent la venue du Seigneur pour la réalisation complète de leur sanctification (1 Co. 15:29-50 ; Ph. 3:20-21 ; Ep. 5:27 ; 1 Jn. 3:2 ; Ap. 22:12).} ; que la grâce et la paix vous soient données de la part de Dieu notre Père et du Seigneur Jésus-Christ.
\VS{8}Je rends premièrement grâces à mon Dieu par Jésus-Christ, au sujet de vous tous, de ce que votre foi est renommée dans le monde entier.
\VS{9}Car Dieu, que je sers en mon esprit dans l'Evangile de son Fils, m'est témoin que je fais sans cesse mention de vous,
\VS{10}demandant continuellement dans mes prières que je puisse enfin trouver, par la volonté de Dieu, quelque moyen favorable pour aller vers vous.
\VS{11}Car je désire extrêmement vous voir, pour vous communiquer quelque don spirituel, afin que vous soyez affermis ;
\VS{12}et aussi, afin qu'étant parmi vous, nous nous consolions ensemble par la foi qui nous est commune.
\VS{13}Or mes frères, je ne veux pas que vous ignoriez que j’ai souvent formé le dessein d'aller vers vous, afin de recueillir quelque fruit parmi vous, comme parmi les autres nations ; mais j'en ai été empêché jusqu'à présent.
\VS{14}Je me dois aux Grecs et aux barbares, aux sages et aux ignorants.
\VS{15}Ainsi, autant qu’il dépend de moi, je suis prêt à vous annoncer aussi l'Evangile à vous qui êtes à Rome.
\VS{16}Car je n'ai point honte de l'Evangile de Christ, vu qu'il est la puissance de Dieu pour le salut de tous ceux qui croient : Du Juif premièrement, puis du Grec,
\VS{17}parce qu’en lui est révélée la justice de Dieu pleinement de foi en foi, selon qu'il est écrit : Le juste vivra par la foi\FTNT{Ha. 2:4.}.
\TextTitle{Jugement sur ceux qui retiennent la vérité captive}
\VS{18}Car la colère de Dieu se révèle pleinement du ciel contre toute impiété et injustice des hommes qui retiennent injustement la vérité captive.
\VS{19}Car ce qu’on peut connaître de Dieu est manifesté parmi eux ; car Dieu le leur a fait connaître.
\VS{20}En effet, les perfections invisibles de Dieu, à savoir sa puissance éternelle et sa divinité, se voient comme à l’œil nu, depuis la création du monde, quand on les considère dans ses ouvrages, de sorte qu'ils sont inexcusables.
\VS{21}Parce qu'ayant connu Dieu, ils ne l'ont point glorifié comme Dieu, et ils ne lui ont point rendu grâces, mais ils se sont égarés dans leurs pensées, et leur cœur sans intelligence a été plongé dans les ténèbres.
\VS{22}Se vantant d’être sages, ils sont devenus fous.
\VS{23}Et ils ont changé la gloire du Dieu incorruptible en images\FTNT{Ex. 20:4-5 ; Mt. 22:20 ; Mc. 12:16 ; Lu. 20:24. Ap.13:14-15 ; Ap. 14:9-11 ; Ap. 15:2 ; Ap. 16:2 ; 19:20 ;  Ap. 20:4.} représentant l'homme corruptible, des oiseaux, des quadrupèdes, et des reptiles.
\TextTitle{Les conséquences de l'endurcissement des hommes}
\VS{24}C'est pourquoi aussi Dieu les a livrés aux convoitises de leurs cœurs et à l’impureté\FTNT{Dieu les a livrés à l’esprit d’égarement (1 R. 22 ; 2 Th. 2:10-13).}, ainsi ils déshonorent eux-mêmes leurs propres corps ;
\VS{25}eux qui ont changé la vérité de Dieu en mensonge, et qui ont adoré et servi la créature, au lieu du Créateur, qui est béni éternellement. Amen !
\VS{26}C'est pourquoi Dieu les a livrés à des passions infâmes, car leurs femmes ont changé l'usage naturel en celui qui est contre la nature.
\VS{27}Et de même les hommes, abandonnant l'usage naturel de la femme, se sont enflammés dans leurs désirs les uns envers les autres, commettant homme avec homme des choses infâmes, et recevant en eux-mêmes le salaire que méritait leur égarement.
\VS{28}Car comme ils ne se sont pas souciés de connaître Dieu, aussi Dieu les a livrés à leur sens réprouvé, pour commettre des choses indignes.
\VS{29}Etant remplis de toute espèce d’injustice, d'impureté, de méchanceté, d'avarice, de malignité, pleins d'envie, de meurtre, de querelles, de fraude, de mauvaises mœurs,
\VS{30}rapporteurs, médisants, haïssant Dieu, outrageux, orgueilleux, vains, ingénieux au mal, rebelles à leurs parents,
\VS{31}dépourvus d’intelligence, de loyauté, d’affection naturelle, de miséricorde.
\VS{32}Et bien qu'ils connaissent le jugement de Dieu, déclarant dignes de mort ceux qui commettent de telles choses, non seulement ils les font, mais encore ils approuvent ceux qui les font.
\Chap{2}
\TextTitle{Condamnation du moralisme}
\VerseOne{}C'est pourquoi, ô homme, qui que tu sois, toi qui juges les autres, tu es donc inexcusable ; car en jugeant les autres, tu te condamnes toi-même, puisque toi qui juges, tu commets les mêmes choses.
\VS{2}Or nous savons que le jugement de Dieu est selon la vérité pour ceux qui commettent de telles choses.
\VS{3}Et penses-tu, ô homme, qui juges ceux qui commettent de telles choses, et qui les commets, que tu échapperas au jugement de Dieu ?
\VS{4}Ou méprises-tu les richesses de sa douceur, et de sa patience, et de sa bonté ; ne reconnaissant pas que la bonté de Dieu te convie à la repentance ?
\VS{5}Mais par ta dureté, et par ton cœur qui est sans repentance, tu t'amasses la colère pour le jour de la colère, et de la manifestation du juste jugement de Dieu,
\VS{6}qui rendra à chacun selon ses œuvres ;
\VS{7}à savoir la vie éternelle à ceux qui, en persévérant dans les bonnes œuvres, cherchent la gloire, l'honneur et l'immortalité.
\VS{8}Mais il y aura de l'indignation et de la colère contre ceux qui ont un esprit de dispute, et qui se rebellent contre la vérité, et obéissent à l'injustice.
\VS{9}Il y aura tribulation et angoisse sur toute âme d'homme qui fait le mal, pour le Juif premièrement, puis pour le Grec.
\VS{10}Mais gloire, honneur, et paix pour quiconque fait le bien ; pour le Juif premièrement, puis pour le Grec.
\VS{11}Car Dieu n'a point d'égard à l'apparence des personnes.
\VS{12}Tous ceux qui auront péché sans la loi, périront aussi sans la loi ; et tous ceux qui auront péché ayant la loi, seront jugés par la loi.
\VS{13}Car ce ne sont pas, en effet, ceux qui écoutent la loi qui sont justes devant Dieu, mais ce sont ceux qui la mettent en pratique qui seront justifiés.
\VS{14}Or quand les gentils, qui n'ont point la loi, font naturellement ce que prescrit la loi, n'ayant point la loi, ils sont une loi pour eux-mêmes.
\VS{15}Et ils montrent par-là que l’œuvre de la loi est écrite dans leurs cœurs, puisque leur conscience leur rend témoignage, et que leurs pensées les accusent ou les défendent.
\VS{16}Tous, dis-je, donc seront jugés le jour où Dieu jugera les secrets des hommes par Jésus-Christ, selon mon Evangile.
\TextTitle{Les Juifs, connaissant la loi, sont condamnés par leur transgression de la loi}
\VS{17}Voici, tu portes le nom de Juif, tu te reposes entièrement sur la loi, et tu te glorifies de Dieu ;
\VS{18}tu connais sa volonté, et tu sais discerner ce qui est contraire, étant instruit par la loi ; 
\VS{19}et tu te crois être le conducteur des aveugles, la lumière de ceux qui sont dans les ténèbres,
\VS{20}le docteur des insensés, le maître des ignorants, ayant le modèle de la science et de la vérité dans la loi.
\VS{21}Toi donc qui enseignes les autres, tu ne t’enseignes pas toi-même ! Toi qui prêches de ne pas dérober, tu dérobes !
\VS{22}Toi qui dis de ne pas commettre d’adultère, tu commets l’adultère ! Toi qui as en abomination les idoles, tu commets des sacrilèges !
\VS{23}Toi qui te glorifies de la loi, tu déshonores Dieu par la transgression de la loi.
\VS{24}Car le nom de Dieu est blasphémé parmi les gentils à cause de vous comme cela est écrit.
\VS{25}Il est vrai que la circoncision est profitable, si tu gardes la loi ; mais si tu es transgresseur de la loi, ta circoncision devient incirconcision.
\VS{26}Si donc l’incirconcis observe les ordonnances de la loi, son incirconcision ne sera-t-elle pas tenue pour circoncision ?
\VS{27}L’incirconcis de nature, qui accomplit la loi, ne te condamnera-t-il pas, toi qui la transgresses, tout en ayant la lettre de la loi et la circoncision ?
\VS{28}Le Juif, ce n’est pas celui qui en a les apparences\FTNT{Le formalisme (2 Ti. 3:5). L’apparence de la piété correspond aux vêtements des brebis : «~Gardez-vous des faux prophètes, ils viennent à vous en habits de brebis, mais au-dedans ce sont des loups ravisseurs.~» (Mt 7:15). «~Puis je vis une autre bête qui montait de la terre, et qui avait deux cornes semblables à celles de l'Agneau ; mais elle parlait comme le dragon.~» Ap. 13:11. Il a l’apparence d’un agneau, mais sa voix est celle du dragon, c’est-à-dire Satan.} ; et la circoncision, ce n’est pas celle qui est visible dans la chair.
\VS{29}Mais le Juif, c’est celui qui l’est intérieurement ; et la circoncision, c’est celle du cœur, selon l’Esprit et non selon la lettre. La louange de ce Juif ne vient pas des hommes, mais de Dieu.
\Chap{3}
\TextTitle{L'avantage du Juif peut devenir une condamnation}
\VerseOne{}Quel est donc l'avantage du Juif, ou quelle est l’utilité de la circoncision ?
\VS{2}Cet avantage est grand de toute manière, et tout d’abord en ce que les oracles de Dieu leur ont été confiés.
\VS{3}Eh quoi ! Si quelques-uns n'ont pas cru, leur incrédulité anéantira-t-elle la fidélité de Dieu ?
\VS{4}Nullement ! Que Dieu au contraire soit reconnu pour vrai, et tout homme pour menteur ; selon ce qui est écrit : Afin que tu sois trouvé juste dans tes paroles, et que tu triomphes lorsqu’on te juge\FTNT{Ps. 51:6.}.
\VS{5}Mais si notre injustice établit la justice de Dieu, que dirons-nous ? Dieu est-il injuste quand il déchaine sa colère ? (Je parle à la manière des hommes.)
\VS{6}Nullement ! Autrement, comment Dieu jugera-t-il le monde ?
\VS{7}Et si par mon mensonge la vérité de Dieu est plus abondante pour sa gloire, pourquoi suis-je encore condamné comme pécheur ?
\VS{8}Et pourquoi ne ferions-nous pas le mal, afin qu'il en arrive du bien, comme quelques-uns, qui nous calomnient, prétendent que nous le disons ? La condamnation de ces gens est juste.
\TextTitle{Juifs et Grecs coupables devant Dieu}
\VS{9}Quoi donc ! Sommes-nous plus excellents ? Nullement. Car nous avons déjà prouvé que tous, tant Juifs que Grecs, sont assujettis au péché.
\VS{10}Selon qu'il est écrit : Il n'y a point de juste, pas même un seul\FTNT{Ps. 14:3.}.
\VS{11}Il n'y a personne qui ait de l'intelligence, il n'y a personne qui recherche Dieu.
\VS{12}Ils se sont tous égarés, ils se sont tous corrompus : Il n'y en a aucun qui fasse le bien, pas même un seul.
\VS{13}Leur gosier est un sépulcre ouvert ; ils se servent de leur langue pour tromper ; il y a du venin d'aspic sous leurs lèvres.
\VS{14}Leur bouche est pleine de malédictions et d'amertume.
\VS{15}Leurs pieds sont légers pour répandre le sang.
\VS{16}La destruction et la misère sont sur leurs voies.
\VS{17}Et ils n'ont point connu la voie de la paix.
\VS{18}La crainte de Dieu n'est pas devant leurs yeux\FTNT{Ps. 14.}.
\VS{19}Or nous savons que tout ce que la loi dit, elle le dit à ceux qui sont sous la loi, afin que toute bouche soit fermée, et que tout le monde soit reconnu coupable devant Dieu.
\VS{20}C'est pourquoi personne ne sera justifié devant lui par les œuvres de la loi, puisque c’est par la loi que vient la connaissance du péché.
\TextTitle{La justification par la foi}
\VS{21}Mais maintenant, sans la loi, la justice de Dieu est manifestée, à laquelle rendent témoignage la loi et les prophètes.
\VS{22}La justice, dis-je, de Dieu par la foi en Jésus-Christ, envers tous et sur tous ceux qui croient. Car il n'y a point de distinction.
\VS{23}Car tous ont péché\FTNT{Le mot péché vient du terme grec «~hamartano~» : «~manquer la marque, manquer le chemin de la droiture et de l’honneur, s’éloigner de la loi de Dieu~». Le péché est la violation délibérée de la loi divine et l’absence de la droiture.} et sont entièrement privés de la gloire de Dieu.
\VS{24}Et ils sont gratuitement justifiés par sa grâce, par la rédemption\FTNT{La rédemption est la délivrance par le paiement d’un prix. Trois termes grecs sont utilisés pour parler de la rédemption : - Agorazo : acheter un objet au marché (agora signifiant marché). Les pécheurs sont considérés comme des esclaves vendus au marché (Ro. 7:14). - Exagorazo : acheter et amener un objet hors du marché (Ga. 3:13 ; Ga. 4:5). L’esclave acheté et amené hors du marché est définitivement délivré. - Lutroo : détacher, rendre libre (Lu. 24:21 ; Tit. 2:14 ; 1 P. 1:18.) Jésus-Christ nous a délivrés du péché, de la puissance de Satan et de la loi mosaïque. (Col. 1:12-14 ; Col. 2:14-17 ; 1 Jn. 3:5).} qui est en Jésus-Christ.
\VS{25}C’est lui que Dieu a destiné à être, par son sang, la victime propitiatoire\FTNT{Le terme propitiation vient du grec «~hilastérion~» qui signifie «~ce qui est expié, ce qui rend propice ou le don qui assure la propitiation~». C’est aussi le lieu où s’accomplit la propitiation (Hé. 9:5), c’est-à-dire le couvercle de l’arche. Lors du grand jour des expiations (Yom Kippour en hébreu), l’aspersion du sang était faite sur le propitiatoire (Lé. 16:14). Le Seigneur Jésus-Christ est notre victime expiatoire (1 Jn. 2:2 ; 1 Jn. 4:10).} pour ceux qui croiraient, afin de montrer sa justice, parce qu’il avait laissé impunis les péchés commis auparavant, au temps de sa patience.
\VS{26}Il montre, dis-je, sa justice dans le temps présent, de manière à être trouvé juste tout en justifiant celui qui a la foi en Jésus.
\VS{27}Où est donc le sujet de se glorifier ? Il est exclu. Par quelle loi ? Est-ce par la loi des œuvres ? Non, mais par la loi de la foi.
\VS{28}Nous concluons donc que l'homme est justifié par la foi, sans les œuvres de la loi.
\TextTitle{Circoncis et incirconcis, justifiés par la foi}
\VS{29}Dieu est-il seulement le Dieu des Juifs ? Ne l'est-il pas aussi des gentils ? Certes, il l'est aussi des gentils,
\VS{30}puisqu’il y a un seul Dieu qui justifiera par la foi les circoncis, et aussi les incirconcis par la foi.
\VS{31}Anéantissons-nous donc la loi par la foi ? Nullement ! Mais au contraire, nous affermissons la loi.
\Chap{4}
\TextTitle{Abraham et David justifiés par la foi\FTNTT{cp. v. 18-25}}
\VerseOne{}Que dirons-nous donc, qu'Abraham, notre père, a obtenu selon la chair ?
\VS{2}Certes, si Abraham a été justifié par les œuvres, il a de quoi se glorifier, mais non pas envers Dieu.
\VS{3}Car que dit l'Ecriture ? Qu’Abraham a cru en Dieu, et que cela lui a été imputé à justice\FTNT{Ge. 15:6.}.
\VS{4}Or à celui qui fait les œuvres, le salaire ne lui est pas imputé comme une grâce, mais comme une chose due.
\VS{5}Mais à celui qui ne fait pas les œuvres, mais qui croit en celui qui justifie le méchant, sa foi lui est imputée à justice.
\VS{6}De même, David exprime le bonheur de l'homme à qui Dieu impute la justice sans les œuvres, en disant :
\VS{7}Heureux sont ceux à qui les iniquités sont pardonnées, et dont les péchés sont couverts.
\VS{8}Heureux l'homme à qui le Seigneur n’impute pas son péché\FTNT{Ps. 32:1-2.}.
\TextTitle{Abraham obtient la justification par la foi avant sa circoncision}
\VS{9}Cette déclaration de bénédiction, est-elle seulement pour les circoncis, ou également pour les incirconcis ? Car nous disons que la foi a été imputée à Abraham à justice.
\VS{10}Comment donc lui a-t-elle été imputée ? Etait-ce après, ou avant sa circoncision ? Il n’était pas encore circoncis, il était incirconcis.
\VS{11}Et il reçut le signe de la circoncision comme sceau de la justice, qu’il avait obtenue par la foi, quand il était incirconcis, afin d’être le père de tous les incirconcis qui croient, pour que la justice leur soit aussi imputée ;
\VS{12}et le père des circoncis, qui ne sont pas seulement circoncis, mais encore qui marchent sur les traces de la foi de notre père Abraham, quand il était incirconcis.
\TextTitle{La justification s'accomplit sans la loi}
\VS{13}En effet, ce n’est pas par loi que la promesse d'être héritier du monde a été faite à Abraham, ou à sa postérité, mais par la justice de la foi.
\VS{14}Car, si les héritiers le sont par la loi, la promesse est annulée, et la foi est vaine
\VS{15}car la loi produit la colère ; car là où il n'y a point de loi, il n'y a point non plus de transgression.
\VS{16}C'est pourquoi les héritiers le sont par la foi, pour que ce soit par la grâce, et afin que la promesse soit assurée à toute la postérité ; non seulement à celle qui est de la loi, mais aussi à celle qui est de la foi d'Abraham, qui est le père de nous tous,
\VS{17}selon qu'il est écrit : Je t'ai établi père de plusieurs nations\FTNT{Ge 17:4-5.}. Il est notre père devant celui auquel il a cru, Dieu qui donne la vie aux morts, et qui appelle les choses qui ne sont point, comme si elles étaient.
\VS{18}Et Abraham ayant espéré contre toute espérance, crut qu'il deviendrait le père de plusieurs nations, selon ce qui lui avait été dit : Ainsi sera ta postérité.
\VS{19}Et sans faiblir dans la foi, il ne considéra point que son corps était déjà usé ; puisqu’il avait environ cent ans, et que Sara n’était plus en âge d'avoir des enfants.
\VS{20}Et il ne douta point de la promesse de Dieu par incrédulité, mais il fut fortifié par la foi, donnant gloire à Dieu,
\VS{21}étant pleinement persuadé que celui qui lui avait fait la promesse était aussi puissant pour l'accomplir.
\VS{22}C'est pourquoi cela lui fut imputé à justice.
\VS{23}Mais ce n’est pas à cause de lui seul qu’il est écrit que cela lui fut imputé à justice ;
\VS{24}c’est encore à cause de nous, à qui cela sera imputé, à nous, dis-je, qui croyons en celui qui a ressuscité des morts, Jésus notre Seigneur,
\VS{25}qui a été livré pour nos offenses, et est ressuscité pour notre justification.
\Chap{5}
\TextTitle{La justification permet la réconciliation avec Dieu}
\VerseOne{}Etant donc justifiés\FTNT{La justification est l’œuvre de Dieu par laquelle la justice de Jésus est comptée en faveur du pécheur, de sorte que le pécheur est déclaré juste par Dieu (Ro. 4 : 3 ; Ro. 5:1-9 ; Ga. 2:16 ; Ga. 3:11). Cette justice n’est pas obtenue par les efforts de la personne sauvée. La justification est une action instantanée qui a pour résultat la vie éternelle. Elle repose totalement et exclusivement sur le sacrifice de Jésus à la croix (1 Pi. 2:24). Elle ne peut être reçue que par la foi en Jésus-Christ (Ep. 2:8-9). La justification est un acte d’imputation divine et non une reconnaissance personnelle de l’homme. Elle provient de la grâce (Ro. 3:24 ; Tit. 3:7).} par la foi, nous avons la paix avec Dieu, par notre Seigneur Jésus-Christ.
\VS{2}Par lequel aussi nous avons été amenés par la foi à cette grâce, dans laquelle nous tenons ferme ; et nous nous glorifions dans l'espérance de la gloire de Dieu.
\VS{3}Bien plus, nous nous glorifions même dans les afflictions ; sachant que l'affliction produit la persévérance ;
\VS{4}et la persévérance l'épreuve ; et l'épreuve l'espérance.
\VS{5}Or l'espérance ne trompe point, parce que l'amour de Dieu est répandu dans nos cœurs par le Saint-Esprit qui nous a été donné.
\VS{6}Car lorsque nous étions encore sans force, Christ est mort en son temps pour nous qui étions des impies.
\VS{7}A peine mourrait-on pour un juste ; quelqu’un peut-être mourrait pour un homme de bien.
\VS{8}Mais Dieu prouve son amour envers nous, en ce que lorsque nous étions encore pécheurs, Christ est mort pour nous.
\VS{9}Etant donc maintenant justifiés par son sang, à plus forte raison serons-nous sauvés par lui de la colère.
\VS{10}Car si, lorsque nous étions ennemis, nous avons été réconciliés avec Dieu par la mort de son Fils, à plus forte raison, étant réconciliés, serons-nous sauvés par sa vie.
\VS{11}Et non seulement cela, mais encore nous nous glorifions même en Dieu par notre Seigneur Jésus-Christ, par qui nous avons maintenant obtenu la réconciliation.
\TextTitle{Parallèle entre l'œuvre de Jésus-Christ et celle d'Adam}
\VS{12}C'est pourquoi comme par un seul homme le péché est entré dans le monde, et par le péché la mort, et qu’ainsi la mort s’est étendue sur tous les hommes, parce que tous ont péché…
\VS{13}Car jusqu'à la loi le péché était dans le monde ; or le péché n'est point imputé quand il n'y a point de loi.
\VS{14}Mais la mort a régné depuis Adam jusqu'à Moïse, même sur ceux qui n'avaient pas péché par une transgression semblable à celle d’Adam, lequel est la figure de celui qui devait venir.
\VS{15}Mais il n'en est pas du don gratuit comme de l'offense ; car si par l'offense d'un seul il en est beaucoup qui sont morts, à plus forte raison la grâce de Dieu, et le don de la grâce, venant d'un seul homme, à savoir de Jésus-Christ, ont-ils été abondamment répandus sur plusieurs.
\VS{16}Et il n'en est pas du don comme de ce qui est arrivé par un seul qui a péché ; car c’est après une seule offense que le jugement est devenu condamnation, mais le don gratuit devient justification après plusieurs offenses.
\VS{17}Si par l'offense d'un seul la mort a régné par lui seul, à plus forte raison ceux qui reçoivent l'abondance de la grâce, et du don de la justice, régneront-ils dans la vie par Jésus-Christ lui seul.
\VS{18}Ainsi donc, comme par une seule offense la condamnation est venue sur tous les hommes, de même par un acte de justice la justification qui donne la vie s’étend à tous les hommes.
\VS{19}Car, comme par la désobéissance d'un seul homme plusieurs ont été rendus pécheurs, de même par l'obéissance d'un seul plusieurs seront rendus justes.
\VS{20}Or la loi est intervenue afin que l'offense abonde, mais là où le péché a abondé, la grâce a surabondé,
\VS{21}afin que, comme le péché a régné par la mort, ainsi la grâce règne par la justice pour donner la vie éternelle, par Jésus-Christ notre Seigneur.
\Chap{6}
\TextTitle{Délivré de la puissance du péché lié au cœur de l’homme }
\VerseOne{}Que dirons-nous donc ? Demeurerions-nous dans le péché, afin que la grâce abonde ?
\VS{2}A Dieu ne plaise ! Car nous qui sommes morts au péché, comment vivrions-nous encore dans le péché ?
\VS{3}Ignoriez-vous que nous tous qui avons été baptisés en Jésus-Christ, c’est en sa mort que nous avons été baptisés ?
\VS{4}Nous avons donc été ensevelis avec lui par le baptême en sa mort ; afin que comme Christ est ressuscité des morts par la gloire du Père, de même nous aussi nous marchions en nouveauté de vie.
\VS{5}Car, si nous sommes devenus une même plante avec lui par la conformité à sa mort, nous le serons aussi par la conformité à sa résurrection.
\VS{6}Sachant que notre vieil homme a été crucifié avec lui, afin que le corps du péché soit détruit, pour que nous ne soyons plus esclaves du péché.
\VS{7}Car celui qui est mort est libre du péché.
\VS{8}Or si nous sommes morts avec Christ, nous croyons que nous vivrons aussi avec lui,
\VS{9}sachant que Christ ressuscité des morts ne meurt plus, et que la mort n'a plus de pouvoir sur lui.
\VS{10}Car il est mort, et c’est pour le péché qu’il est mort une fois pour toutes ; il est revenu à la vie, et c’est pour Dieu qu’il est vivant.
\TextTitle{Mort au péché pour une vie nouvelle en Dieu}
\VS{11}Ainsi vous-mêmes, considérez-vous comme morts au péché, et comme vivants pour Dieu en Jésus-Christ notre Seigneur.
\VS{12}Que le péché ne règne donc point dans votre corps mortel, et n’obéissez pas à ses convoitises.
\VS{13}Et ne livrez pas vos membres au péché comme des instruments d'iniquité ; mais donnez-vous vous-mêmes à Dieu comme de morts étant devenus vivants, et offrez vos membres à Dieu pour être des instruments de justice.
\VS{14}Car le péché n'aura pas de domination sur vous, parce que vous n'êtes point sous la loi, mais sous la grâce.
\VS{15}Quoi donc ? Pécherions-nous parce que nous ne sommes point sous la loi, mais sous la grâce ? A Dieu ne plaise !
\VS{16}Ne savez-vous pas qu’en vous livrant à quelqu’un comme esclaves pour lui obéir, vous êtes esclaves de celui à qui vous obéissez, soit du péché qui conduit à la mort, soit de l'obéissance qui conduit à la justice ?
\VS{17}Mais grâces à Dieu de ce qu'ayant été les esclaves du péché, vous avez obéi de cœur à la forme expresse de la doctrine dans laquelle vous avez été élevés.
\VS{18}Ayant donc été affranchis du péché, vous avez été asservis à la justice.
\VS{19}Je parle à la façon des hommes, à cause de l'infirmité de votre chair. Comme donc vous avez appliqué vos membres pour servir à la souillure et à l’iniquité, ainsi appliquez vos membres pour servir à la justice en sainteté.
\VS{20}Car lorsque vous étiez esclaves du péché, vous étiez libres à l'égard de la justice.
\VS{21}Quel fruit portiez-vous alors ? Des fruits dont vous avez honte maintenant. Car la fin de ces choses c’est la mort.
\VS{22}Mais maintenant que vous êtes affranchis du péché, et asservis à Dieu, vous avez pour fruit la sanctification, et pour fin la vie éternelle.
\VS{23}Car le salaire du péché, c'est la mort ; mais le don gratuit de Dieu, c'est la vie éternelle par Jésus-Christ notre Seigneur.
\Chap{7}
\TextTitle{Le chrétien lié à Christ comme à un époux}
\VerseOne{}Ignorez-vous, frères, car je parle à des gens qui connaissent la loi, que la loi exerce son pouvoir sur l’homme aussi longtemps qu’il vit ?
\VS{2}Car la femme qui est sous la puissance d'un mari, est liée à son mari par la loi tandis qu'il est en vie ; mais si son mari meurt, elle est délivrée de la loi du mari.
\VS{3}Si donc, du vivant de son mari, elle épouse un autre homme, elle sera appelée adultère ; mais si son mari meurt, elle est délivrée de la loi, de sorte qu'elle ne sera point adultère si elle épouse un autre homme.
\VS{4}Ainsi donc, vous aussi, mes frères, vous avez été, par le corps de Christ, mis à mort en ce qui concerne la loi, pour que vous apparteniez à un autre, à savoir, à celui qui est ressuscité des morts, afin que nous portions des fruits pour Dieu.
\VS{5}Car lorsque nous étions dans la chair, les passions des péchés excitées par la loi, agissaient dans nos membres de manière à produire des fruits pour la mort.
\VS{6}Mais maintenant nous sommes délivrés de la loi, étant morts à cette loi sous laquelle nous étions retenus ; afin que nous servions Dieu dans un esprit nouveau, et non selon la lettre qui a vieilli.
\TextTitle{La loi a révélé le péché mais la délivrance vient par Jésus-Christ}
\VS{7}Que dirons-nous donc ? La loi est-elle péché ? Nullement ! Au contraire, je n'ai connu le péché que par la loi ; car je n’aurais pas connu la convoitise, si la loi n’avait pas dit : Tu ne convoiteras point\FTNT{Ex. 20:17.}.
\VS{8}Et le péché, saisissant l’occasion, produisit en moi, par le commandement, toutes sortes de convoitises ; parce que sans la loi le péché est mort.
\VS{9}Pour moi, étant autrefois sans loi, je vivais. Mais quand le commandement vint, le péché reprit vie, et moi je mourus.
\VS{10}Ainsi, le commandement qui conduit à la vie se trouva pour moi conduire à la mort.
\VS{11}Car le péché, saisissant l’occasion, me séduisit par le commandement, et par lui me fit mourir.
\VS{12}La loi donc est sainte, et le commandement est saint, juste, et bon.
\VS{13}Ce qui est bon a-t-il donc été pour moi une cause de mort ? Nullement ! Mais c’est le péché, afin qu'il se manifeste comme péché, en me donnant la mort par ce qui est bon, et que par le commandement, il devienne condamnable au plus haut point.
\VS{14}Car nous savons, en effet, que la loi est spirituelle ; mais moi, je suis charnel, vendu au péché.
\TextTitle{[la connaissance du bien incapable de délivrer l'homme du péché}
\VS{15}Car je n'approuve pas ce que je fais, puisque je ne fais point ce que je veux, mais je fais ce que je hais.
\VS{16}Or si je fais ce que je ne veux pas, je reconnais par cela même que la loi est bonne.
\VS{17}Et maintenant donc ce n'est plus moi qui fais cela, mais c'est le péché qui habite en moi.
\VS{18}Ce qui est bon, je le sais, n’habite pas en moi, c’est-à-dire dans ma chair. J’ai la volonté, mais non le pouvoir de faire le bien.
\VS{19}Car je ne fais pas le bien que je veux, mais je fais le mal que je ne veux point.
\VS{20}Or si je fais ce que je ne veux point, ce n'est plus moi qui le fais, mais c'est le péché qui habite en moi.
\VS{21}Je trouve donc cette loi au-dedans de moi : Quand je veux faire le bien, le mal est attaché à moi.
\VS{22}Car je prends bien plaisir à la loi de Dieu quant à l'homme intérieur,
\VS{23}mais je vois dans mes membres une autre loi, qui combat contre la loi de mon entendement\FTNT{Entendement : Du grec «~nous~», c’est-à-dire l’esprit, l’intelligence, le bon sens, la raison.}, et qui me rend prisonnier à la loi du péché qui est dans mes membres.
\VS{24}Ah, misérable que je suis ! Qui me délivrera du corps de cette mort ?
\TextTitle{Seul l'Esprit de Christ libère de la loi du péché}
\VS{25}Je rends grâces à Dieu par Jésus-Christ notre Seigneur !… Ainsi donc, moi-même, je suis par l’entendement esclave de la loi de Dieu, et je suis par la chair esclave de la loi du péché.
\Chap{8}
\VerseOne{}Il n'y a donc maintenant aucune condamnation pour ceux qui sont en Jésus-Christ, qui marchent, non selon la chair, mais selon l'Esprit.
\VS{2}Parce que la loi de l'Esprit de vie qui est en Jésus-Christ m'a affranchi de la loi du péché et de la mort.
\VS{3}Car chose impossible à la loi, parce que la chair la rendait impuissante, Dieu a condamné le péché dans la chair, en envoyant, à cause du péché, son propre Fils dans une chair semblable à celle du péché.
\VS{4}Afin que la justice de la loi soit accomplie en nous, qui ne marchons point selon la chair, mais selon l'Esprit.
\TextTitle{L'affection de l'Esprit opposée à celle de la chair\FTNTT{cp. Ga. 5:15-18}}
\VS{5}Car ceux, en effet, qui vivent selon la chair, s’affectionnent aux choses de la chair, tandis que ceux qui vivent selon l'Esprit, s’affectionnent aux choses de l'Esprit.
\VS{6}Or l'affection de la chair c’est la mort, tandis que l'affection de l'Esprit c’est la vie et la paix.
\VS{7}Car l'affection de la chair est inimitié contre Dieu, parce qu’elle ne se soumet pas à la loi de Dieu, et qu’elle ne le peut même pas.
\VS{8}C'est pourquoi ceux qui vivent selon la chair ne sauraient plaire à Dieu.
\VS{9}Pour vous, vous ne vivez pas selon la chair, mais selon l'Esprit, si du moins l'Esprit de Dieu habite en vous. Si quelqu'un n'a pas l'Esprit de Christ\FTNT{Notez que le Saint-Esprit est aussi appelé l’Esprit de Jésus (Ac. 16:7).}, il ne lui appartient pas.
\VS{10}Et si Christ est en vous, le corps est bien mort à cause du péché, mais l'Esprit est vie à cause de la justice.
\VS{11}Et si l'Esprit de celui qui a ressuscité Jésus d’entre les morts habite en vous, celui qui a ressuscité Christ d’entre les morts rendra aussi la vie à vos corps mortels par son Esprit qui habite en vous.
\VS{12}Ainsi donc, mes frères, nous ne sommes point redevables à la chair, pour vivre selon la chair.
\VS{13}Car si vous vivez selon la chair, vous mourrez ; mais si par l'Esprit vous faites mourir les actions du corps, vous vivrez.
\TextTitle{L'Esprit d'adoption\FTNTT{Ga. 4:7}}
\VS{14}Car tous ceux qui sont conduits par l'Esprit de Dieu sont enfants de Dieu.
\VS{15}Et vous n'avez point reçu un esprit de servitude pour être encore dans la crainte ; mais vous avez reçu l'Esprit d'adoption, par lequel nous crions Abba, c'est-à-dire Père.
\VS{16}L’Esprit lui-même rend témoignage à notre esprit que nous sommes enfants de Dieu.
\VS{17}Et si nous sommes enfants, nous sommes aussi héritiers : Héritiers, dis-je, de Dieu, et cohéritiers de Christ ; si toutefois nous souffrons avec lui, afin d’être glorifiés avec lui.
\TextTitle{La gloire à venir\FTNTT{cp. Ge. 3:18-19}}
\VS{18}Car tout bien compté, j'estime que les souffrances du temps présent ne sauraient être comparables à la gloire à venir qui doit être révélée pour nous.
\VS{19}Aussi, la création attend-elle avec un ardent désir la révélation des fils de Dieu.
\VS{20}Car la création a été soumise à la vanité, non de son gré, mais à cause de celui qui l’y a soumise ;
\VS{21}avec l’espérance qu’elle aussi sera affranchie de la servitude de la corruption, pour avoir part à la liberté de la gloire des enfants de Dieu.
\VS{22}Or, nous savons que jusqu’à ce jour, toute la création soupire et souffre les douleurs de l’enfantement.
\VS{23}Et non seulement elle, mais nous aussi, qui avons les prémices de l'Esprit ; nous-mêmes, dis-je, soupirons en nous-mêmes, en attendant l'adoption, c'est-à-dire la rédemption de notre corps\FTNT{1 Co. 15:35-43 ; 1 Co. 15:51-54.}.
\VS{24}Car c’est en espérance que nous sommes sauvés. Or l’espérance qu’on voit n’est plus espérance : Ce qu’on voit, peut-on l’espérer encore ?
\VS{25}Mais si nous espérons ce que nous ne voyons pas, c'est que nous l'attendons avec patience.
\TextTitle{L'Esprit intercède pour les saints\FTNTT{Hé. 7:25}}
\VS{26}De même aussi l’Esprit nous aide dans notre faiblesse, car nous ne savons pas ce qu’il nous convient de demander dans nos prières. Mais l’Esprit lui-même intercède par des soupirs inexprimables.
\VS{27}Et celui qui sonde les cœurs connaît quelle est la pensée de l'Esprit, car il intercède en faveur des saints, selon Dieu.
\TextTitle{Le plan de Dieu s'accomplit par l'Evangile}
\VS{28}Or nous savons aussi que toutes choses concourent au bien de ceux qui aiment Dieu, c'est-à-dire de ceux qui sont appelés selon son dessein.
\VS{29}Car ceux qu'il a connus d’avance, il les a aussi prédestinés à être semblables à l'image de son Fils, afin qu'il soit le premier-né de beaucoup de frères.
\VS{30}Et ceux qu'il a prédestinés, il les a aussi appelés ; et ceux qu'il a appelés, il les a aussi justifiés ; et ceux qu'il a justifiés, il les a aussi glorifiés.
\VS{31}Que dirons-nous donc à l’égard de ces choses ? Si Dieu est pour nous, qui sera contre nous ?
\VS{32}Lui qui n'a point épargné son propre Fils, mais qui l'a livré pour nous tous, comment ne nous donnera-t-il point aussi toutes choses avec lui ?
\VS{33}Qui accusera les élus de Dieu ? Dieu est celui qui justifie.
\VS{34}Qui les condamnera ? Christ est mort ; et bien plus, il est ressuscité, il est à la droite de Dieu, et il intercède pour nous.
\TextTitle{L'amour de Christ résiste contre tout}
\VS{35}Qui nous séparera de l'amour de Christ ? Sera-ce l'oppression, ou l'angoisse, ou la persécution, ou la famine, ou la nudité, ou le péril, ou l'épée ?
\VS{36}Ainsi qu'il est écrit : C’est à cause de toi que nous sommes livrés à la mort tous les jours, et qu’on nous regarde comme des brebis destinées à la boucherie\FTNT{Ps. 44:23.}.
\VS{37}Mais dans toutes ces choses nous sommes plus que vainqueurs par celui qui nous a aimés.
\VS{38}Car j’ai l’assurance que ni la mort, ni la vie, ni les anges, ni les principautés, ni les puissances, ni les choses présentes, ni les choses à venir,
\VS{39}ni la hauteur, ni la profondeur, ni aucune autre créature, ne pourra nous séparer de l'amour de Dieu manifesté en Jésus-Christ notre Seigneur.
\Chap{9}
\TextTitle{Le chagrin de Paul pour Israël son peuple}
\VerseOne{}Je dis la vérité en Christ, je ne mens point, ma conscience m’en rend témoignage par le Saint-Esprit :
\VS{2}J’éprouve une grande tristesse et un chagrin continuel dans mon cœur.
\VS{3}Car moi-même je souhaiterais être anathème et séparé de Christ pour mes frères, mes parents selon la chair,
\TextTitle{Les enfants de la chair et ceux de la promesse}
\VS{4}qui sont Israélites, à qui appartiennent l'adoption, la gloire, les alliances, l'ordonnance de la loi, le culte,
\VS{5}les promesses, les patriarches, et de qui est issu selon la chair Christ, qui est Dieu au-dessus de toutes choses, béni éternellement, Amen !
\VS{6}Toutefois il ne peut pas se faire que la parole de Dieu soit anéantie. Car tous ceux qui descendent d’Israël ne sont pas Israël.
\VS{7}Et bien qu’ils soient de la postérité d'Abraham, ils ne sont pas tous ses enfants, car il est dit : C'est en Isaac que tu auras une postérité appelée de ton nom ;
\VS{8}c'est-à-dire que ce ne sont pas ceux qui sont enfants de la chair qui sont enfants de Dieu, mais que ce sont les enfants de la promesse qui sont regardés comme la postérité.
\VS{9}Car voici la parole de la promesse : Je viendrai à cette même époque, et Sara aura un fils\FTNT{Ge. 18:10.}.
\VS{10}Et de plus, il en fut ainsi de Rébecca, qui conçut du seul Isaac notre père ;
\VS{11}car les enfants n’étaient pas encore nés et ils n’avaient fait ni bien ni mal, afin que le dessein arrêté selon l'élection de Dieu subsiste, sans dépendre des œuvres, mais par la volonté de celui qui appelle,
\VS{12}il lui fut dit : L’aîné sera assujetti au plus petit\FTNT{Ge. 25:23.}, selon qu’il est écrit :
\VS{13}J'ai aimé Jacob, et j'ai haï Esaü\FTNT{Mal. 1:2-3.}.
\TextTitle{La volonté souveraine de Dieu}
\VS{14}Que dirons-nous donc : Y a-t-il de l’injustice en Dieu ? A Dieu ne plaise !
\VS{15}Car il dit à Moïse : J'aurai compassion de celui de qui j’aurai compassion et je ferai miséricorde à celui à qui je ferai miséricorde\FTNT{Ex. 33:19.}.
\VS{16}Ainsi donc, cela ne vient pas de celui qui veut, ni de celui qui court, mais de Dieu qui fait miséricorde.
\VS{17}Car l'Ecriture dit à Pharaon : Je t'ai suscité dans le but de démontrer en toi ma puissance, et afin que mon Nom soit publié par toute la terre\FTNT{Ex. 9:16.}.
\VS{18}Ainsi, il fait miséricorde à qui il veut, et il endurcit qui il veut.
\VS{19}Tu me diras : Pourquoi se plaint-il encore ? Car qui est celui qui peut résister à sa volonté ?
\VS{20}Mais plutôt, ô homme, qui es-tu, toi qui contestes contre Dieu ? Le vase d’argile dira-t-il à celui qui l'a formé : Pourquoi m'as-tu ainsi fait ?
\VS{21}Le potier n'a-t-il pas le pouvoir de faire avec la même masse de terre un vase d’honneur et un vase d’un usage vil ?
\VS{22}Et que dire, si Dieu, en voulant montrer sa colère, et faire connaître sa puissance, a supporté avec une grande patience les vases de colère, préparés pour la perdition ?
\VS{23}Et s’il a voulu faire connaître les richesses de sa gloire envers les vases de miséricorde, qu'il a préparés d’avance pour la gloire ?
\VS{24}Ainsi il nous a appelés, non seulement d'entre les juifs, mais aussi d'entre les gentils,
\TextTitle{Les prophéties concernant l'aveuglement d'Israël et la grâce sur les gentils}
\VS{25}selon ce qu'il dit dans Osée : J'appellerai mon peuple celui qui n'était point mon peuple ; et la bien-aimée, celle qui n'était point la bien-aimée ;
\VS{26}et il arrivera qu'au lieu où il leur a été dit : Vous n’êtes pas mon peuple, là ils seront appelés les fils du Dieu vivant\FTNT{Os. 2:1.}.
\VS{27}Aussi Esaïe s'écrie au sujet d'Israël : Quand le nombre des enfants d'Israël serait comme le sable de la mer, un petit reste seulement sera sauvé.
\VS{28}Car le Seigneur exécutera pleinement et promptement sa parole sur la terre ce qu’il a résolu\FTNT{Es. 10:22-23.}.
\VS{29}Et comme Esaïe avait dit auparavant : Si le Seigneur des armées ne nous avait laissé une postérité, nous serions devenus comme Sodome, et nous aurions été semblables à Gomorrhe\FTNT{Es. 1:9.}.
\VS{30}Que dirons-nous donc ? Que les gentils, qui ne cherchaient pas la justice, ont obtenu la justice, la justice qui vient de la foi,
\VS{31}tandis qu’Israël qui cherchait la loi de la justice, n'est pas parvenu à cette loi.
\VS{32}Pourquoi ? Parce qu’Israël l’a cherchée non par la foi, mais comme provenant des œuvres de la loi. Ils se sont heurtés contre la pierre d'achoppement,
\VS{33}selon qu’il est écrit : Voici, je mets en Sion la pierre d'achoppement ; et un rocher de scandale, et quiconque croit en lui ne sera point confus\FTNT{Es. 28:16.}.
\Chap{10}
\TextTitle{La foi, seule condition du salut}
\VerseOne{}Mes frères, le souhait de mon cœur, et la prière que je fais à Dieu pour les Israélites, c'est qu'ils soient sauvés.
\VS{2}Car je leur rends témoignage qu'ils ont du zèle pour Dieu, mais sans connaissance.
\VS{3}Parce que ne connaissant point la justice de Dieu, et cherchant à établir leur propre justice, ils ne se sont point soumis à la justice de Dieu.
\VS{4}Car Christ est la fin de la loi\FTNT{Il est question de la loi cérémonielle relative au culte mosaïque. Avant sa mort, Jésus qui était né sous la loi (Ga. 4:4), demandait aux gens de l’appliquer. Ainsi, il demanda au lépreux qu’il avait guéri de présenter une offrande pour sa purification au temple (Mt. 8:1-4) et à ses disciples d'observer l'enseignement des scribes (Mt. 23:1-2). En effet, il fallait que les lois cérémonielles soient respectées jusqu’à sa résurrection. Une fois que Jésus eut dit «~tout est accompli~» (Jn. 19:30), toutes ces lois n’avaient plus aucune raison d’être (Col. 2:14-17 ; Hé. 7:11-22 ; Hé. 10:1-2).} pour la justification de tous ceux qui croient.
\VS{5}En effet, Moïse décrit ainsi la justice qui vient de la loi : L'homme qui fera ces choses vivra par elles\FTNT{Lé. 18:5.}.
\VS{6}Mais voici comment s'exprime la justice qui vient de la foi : Ne dis pas en ton cœur : Qui montera au ciel ? C’est en faire descendre Christ.
\VS{7}Ou : Qui descendra dans l'abîme ? C’est faire remonter Christ d’entre les morts.
\VS{8}Mais que dit-elle ? La parole est près de toi, dans ta bouche, et dans ton cœur. Or voilà la parole de foi que nous prêchons.
\VS{9}C'est pourquoi, si tu confesses de ta bouche le Seigneur Jésus, et si tu crois dans ton cœur que Dieu l'a ressuscité des morts, tu seras sauvé.
\VS{10}Car c’est en croyant du cœur qu’on parvient à la justice, et c’est en confessant de la bouche qu’on parvient au salut, selon ce que dit l’Ecriture :
\VS{11}Quiconque croit en lui ne sera point confus\FTNT{Es. 49:23.}.
\VS{12}Parce qu'il n'y a point de différence, en effet, entre le Juif et le Grec, puisqu’ils ont un même Seigneur, qui est riche pour tous ceux qui l'invoquent.
\VS{13}Car quiconque invoquera le nom du Seigneur sera sauvé\FTNT{Joë. 2:32.}.
\TextTitle{La proclamation de l'Evangile dans les nations}
\VS{14}Mais comment invoqueront-ils celui en qui ils n'ont point cru ? Et comment croiront-ils en celui dont ils n'ont point entendu parler ? Et comment en entendront-ils parler s'il n'y a personne qui leur prêche ?
\VS{15}Et comment y aura-t-il des prédicateurs, s’ils ne sont pas envoyés ? Selon qu'il est écrit : Qu’ils sont beaux les pieds de ceux qui annoncent la paix, de ceux qui annoncent de bonnes nouvelles\FTNT{Es. 52:7.} !
\VS{16}Mais tous n'ont pas obéi à l'Evangile ; car Esaïe dit : Seigneur, qui a cru à notre prédication\FTNT{Es. 53:1.} ?
\VS{17}Ainsi la foi vient de ce qu’on entend, et ce qu’on entend vient de la parole de Christ.
\VS{18}Mais je dis : Ne l'ont-ils point entendue ? Au contraire, leur voix est allée par toute la terre, et leur parole jusqu’aux extrémités du monde.
\VS{19}Mais je dis : Israël ne l'a-t-il point su ? Moïse le premier dit : J’exciterai votre jalousie par ce qui n'est point une nation, je provoquerai votre colère par une nation sans intelligence\FTNT{De. 32:21.}.
\VS{20}Et Esaïe pousse la hardiesse jusqu’à dire : J'ai été trouvé par ceux qui ne me cherchaient point, et je me suis clairement manifesté à ceux qui ne me demandaient pas\FTNT{Es. 65:1.}.
\VS{21}Mais au sujet d’Israël, il dit : J'ai tout le jour tendu mes mains vers un peuple rebelle et contredisant\FTNT{Es. 65:2.}.
\Chap{11}
\TextTitle{Un reste d'Israël participe à la grâce}
\VerseOne{}Je dis donc : Dieu a-t-il rejeté son peuple ? A Dieu ne plaise ! Car je suis aussi Israélite, de la postérité d'Abraham, de la tribu de Benjamin.
\VS{2}Dieu n'a point rejeté son peuple, qu’il a connu d’avance. Et ne savez-vous pas ce que l'Ecriture dit d'Elie, comment il a fait requête à Dieu contre Israël, disant :
\VS{3}Seigneur, ils ont tué tes prophètes, et ils ont démoli tes autels, et je suis resté moi seul ; et ils cherchent à m'ôter la vie\FTNT{1 R. 19:10.}.
\VS{4}Mais quelle réponse Dieu lui donna-t-il ? Je me suis réservé sept mille hommes, qui n'ont point fléchi le genou devant Baal\FTNT{1 R.19:18.}.
\VS{5}De même aussi dans le temps présent, il y a un reste selon l'élection de la grâce.
\VS{6}Or si c'est par la grâce, ce n'est plus par les œuvres ; autrement la grâce n'est plus la grâce. Mais si c'est par les œuvres, ce n'est plus par une grâce ; autrement l’œuvre n'est plus une œuvre.
\TextTitle{La nation d'Israël est temporairement mise à l'écart mais non rejetée}
\VS{7}Quoi donc ? Ce qu'Israël cherche, il ne l'a point obtenu ; mais les élus l’ont obtenu, tandis que les autres ont été endurcis,
\VS{8}selon qu'il est écrit : Dieu leur a donné un esprit d’assoupissement, des yeux pour ne point voir, et des oreilles pour ne point entendre\FTNT{Es. 29:10.}, jusqu’à ce jour. Et David dit :
\VS{9}Que leur table soit pour eux un filet, un piège, une occasion de chute, et cela pour leur récompense.
\VS{10}Que leurs yeux soient obscurcis pour ne point voir\FTNT{Ps. 69:23-24.} ; et tiens continuellement leur dos courbé !
\VS{11}Mais je dis : Est-ce pour tomber qu’ils ont bronché ? Nullement ! Mais par leur chute, le salut est accordé aux Gentils, afin qu’ils soient excités à la jalousie.
\VS{12}Or si leur chute est la richesse du monde, et leur amoindrissement la richesse des Gentils, combien plus en sera-t-il quand ils se convertiront tous ?
\TextTitle{Avertissement aux gentils}
\VS{13}Car je vous parle à vous, Gentils, en tant qu’apôtre des Gentils, je glorifie mon ministère,
\VS{14}afin, s’il est possible, d’exciter la jalousie de ceux de ma race et d’en sauver quelques-uns.
\VS{15}Car si leur mise à l’écart a été la réconciliation du monde, quelle sera leur réintégration, sinon le passage de la mort à la vie ?
\VS{16}Or si les prémices sont saintes, la masse l'est aussi ; et si la racine est sainte, les branches le sont aussi.
\VS{17}Mais si quelques-unes des branches ont été retranchées, et si toi qui étais un olivier sauvage, tu as été greffé à leur place et rendu participant de la racine et de la graisse de l'olivier,
\VS{18}ne te glorifie pas contre ces branches ; car si tu te glorifies, ce n'est pas toi qui portes la racine, mais c'est la racine qui te porte.
\VS{19}Mais tu diras : Les branches ont été retranchées, afin que moi je sois greffé.
\VS{20}Cela est vrai, elles ont été retranchées à cause de leur incrédulité, et tu es debout par la foi ; ne t'élève donc point par orgueil, mais crains.
\VS{21}Car si Dieu n'a point épargné les branches naturelles, prends garde qu'il ne t'épargne pas non plus.
\VS{22}Considère donc la bonté et la sévérité de Dieu ; la sévérité envers ceux qui sont tombés ; et la bonté envers toi, si tu persévères dans cette bonté : Car autrement tu seras aussi retranché.
\VS{23}Eux de même, s'ils ne persistent pas dans leur incrédulité, ils seront greffés ; car Dieu est puissant pour les greffer de nouveau.
\VS{24}Car si toi tu as été coupé de l'olivier sauvage selon sa nature, et greffé contrairement à ta nature sur l'olivier franc, à plus forte raison eux seront-ils greffés selon leur nature sur leur propre olivier.
\VS{25}Car mes frères, je ne veux pas que vous ignoriez ce mystère, afin que vous ne vous regardiez point comme sages : Une partie d’Israël est tombée dans l’endurcissement, jusqu’à ce que la totalité des gentils soit entrée.
\TextTitle{Yahweh prédit le salut futur d'Israël\FTNTT{Es. 66:8}}
\VS{26}Et ainsi tout Israël sera sauvé, selon qu’il est écrit : Le Libérateur viendra de Sion, et il détournera de Jacob les infidélités ;
\VS{27}et c'est là l'alliance que je ferai avec eux, lorsque j'ôterai leurs péchés\FTNT{Es. 59:20-21.}.
\VS{28}Ils sont certes ennemis par rapport à l'Evangile, à cause de vous ; mais en ce qui concerne l’élection, ils sont aimés à cause de leurs pères.
\VS{29}Car Dieu ne se repent pas de ses dons et de sa vocation.
\VS{30}De même que vous avez autrefois désobéi à Dieu et que par leur désobéissance vous avez maintenant obtenu miséricorde,
\VS{31}de même ils ont maintenant désobéi, afin que par la miséricorde qui vous a été faite, ils obtiennent aussi miséricorde.
\VS{32}Car Dieu les a tous renfermés sous la rébellion afin de faire miséricorde à tous.
\TextTitle{Les voies incompréhensibles de Dieu}
\VS{33}Ô profondeur de la richesse, de la sagesse et de la connaissance de Dieu ! Que ses jugements sont insondables et ses voies incompréhensibles !
\VS{34}Car qui a connu la pensée du Seigneur ? Ou qui a été son conseiller ?
\VS{35}Qui lui a donné le premier, pour qu’il ait à recevoir en retour ?
\VS{36}Car c’est de lui, par lui, et pour lui que sont toutes choses. A lui soit la gloire éternellement. Amen !
\Chap{12}
\TextTitle{Le culte raisonnable}
\VerseOne{}Je vous exhorte donc, mes frères, par les compassions de Dieu, à offrir vos corps comme un sacrifice vivant, saint, agréable à Dieu, ce qui est votre culte raisonnable.
\VS{2}Et ne vous conformez pas au siècle présent, mais soyez transformés\FTNT{Le verbe «~transformer~» est la traduction du terme grec «~metamorphoo~» qui a donné en français «~transfigurer~». C’est le même terme qui a été utilisé en Mt. 17:2 pour parler de la transfiguration du Seigneur. Si Paul recommandait cela à des personnes déjà converties, c’est parce que Dieu les appelait à aller plus loin. La transformation d’une chenille en papillon est un très bel exemple pour illustrer le changement radical qui doit s’opérer en nous. Pour atteindre ce stade, cet insecte passe par plusieurs étapes. La transformation nous permet de croître spirituellement. En effet, tout enfant de Dieu est appelé à devenir mature, à passer du stade de petit enfant à celui de jeune homme, et de celui de jeune homme à celui de père (1 Jn. 2:12-14).} par le renouvellement de votre entendement, afin que vous discerniez quelle est la volonté de Dieu, ce qui est bon, agréable et parfait.
\TextTitle{Exhortation à l'humilité et au service selon les dons de l'Esprit}
\VS{3}Par la grâce qui m’a été donnée, je dis à chacun de vous que nul ne présume d'être plus sage qu'il ne faut, mais d’avoir des sentiments modestes, selon la mesure de foi que Dieu a départie à chacun.
\VS{4}Car comme nous avons plusieurs membres dans un seul corps, et que tous les membres n'ont pas la même fonction,
\VS{5}ainsi, nous qui sommes plusieurs, nous formons un seul corps en Christ, et nous sommes tous membres les uns des autres.
\VS{6}Puisque nous avons des dons différents, selon la grâce qui nous est donnée, que celui qui a le don de prophétie l’exerce en analogie de la foi ;
\VS{7}que celui qui est appelé au ministère, s’attache à son ministère ; que celui qui enseigne s’attache à son enseignement,
\VS{8}et celui qui exhorte, à l’exhortation ; que celui qui donne, le fasse avec simplicité ; que celui qui préside, le fasse avec zèle ; que celui qui exerce la miséricorde, le fasse avec joie.
\TextTitle{Les relations mutuelles entre chrétien}
\VS{9}Que la charité soit sincère. Ayez en horreur le mal, attachez-vous fortement au bien.
\VS{10}Par charité fraternelle, soyez pleins d’affection les uns pour les autres ; par honneur, usez de prévenances réciproques.
\VS{11}Ne soyez point paresseux à vous employer pour autrui. Soyez fervents d'esprit. Servez le Seigneur.
\VS{12}Soyez joyeux dans l'espérance. Soyez patients dans la tribulation. Persévérez dans la prière.
\VS{13}Pourvoyez aux besoins des saints. Exercez l'hospitalité.
\VS{14}Bénissez ceux qui vous persécutent ; bénissez-les, et ne les maudissez point.
\VS{15}Réjouissez-vous avec ceux qui se réjouissent. Pleurez avec ceux qui pleurent.
\VS{16}Ayez les mêmes sentiments les uns envers les autres. N’aspirez pas à ce qui est élevé, mais laissez-vous attirer par ce qui est humble. Ne soyez point sages à votre propre jugement.
\TextTitle{Les relations du chrétiens avec ceux du dehors}
\VS{17}Ne rendez à personne le mal pour le mal. Recherchez les choses honnêtes devant tous les hommes.
\VS{18}S’il est possible, autant que cela dépend de vous, soyez en paix avec tous les hommes.
\VS{19}Ne vous vengez point vous-mêmes, mes bien-aimés, mais laissez agir la colère de Dieu, car il est écrit : A moi appartient la vengeance, à moi la rétribution, dit le Seigneur\FTNT{De. 32:35.}.
\VS{20}Si donc ton ennemi a faim, donne-lui à manger ; s'il a soif, donne-lui à boire, car en faisant cela, tu amasseras des charbons ardents sur sa tête.
\VS{21}Ne te laisse pas vaincre par le mal, mais surmonte le mal par le bien.
\Chap{13}
\TextTitle{Le chrétien et les autorités}
\VerseOne{}Que toute personne soit soumise aux autorités supérieures, car il n'y a point d’autorité qui ne vienne pas de Dieu, et les autorités qui existent ont été instituées de Dieu.
\VS{2}C'est pourquoi celui qui s’oppose à l’autorité résiste à l’ordre de Dieu ; et ceux qui y résistent attireront la condamnation sur eux-mêmes.
\VS{3}Car ce n’est pas pour une bonne action, c’est pour une mauvaise que les magistrats sont à craindre. Veux-tu ne pas craindre l’autorité ? Fais le bien, et tu auras sa louange.
\VS{4}Car le magistrat est un serviteur de Dieu pour ton bien. Mais si tu fais le mal, crains, car ce n’est pas en vain qu’il porte l'épée, étant serviteur de Dieu, ordonné pour faire justice en punissant celui qui fait le mal.
\VS{5}C'est pourquoi il faut être soumis, non seulement à cause de la punition, mais aussi à cause de la conscience.
\VS{6}Car c'est aussi pour cela que vous payez les impôts, parce que les magistrats sont les ministres de Dieu, s'employant à rendre la justice.
\VS{7}Rendez donc à tous ce qui leur est dû : L’impôt à qui vous devez l’impôt, le tribut à qui vous devez le tribut, le péage à qui vous devez le péage, la crainte à qui vous devez la crainte, l’honneur à qui vous devez l'honneur.
\TextTitle{L'amour de son prochain : accomplissement de la loi\FTNTT{cp. Lu. 10:29-37}}
\VS{8}Ne devez rien à personne, si ce n’est de vous aimer les uns les autres ; car celui qui aime les autres a accompli la loi.
\VS{9}En effet, les commandements : Tu ne commettras point d’adultère, tu ne tueras point, tu ne déroberas point, tu ne convoiteras point, et ceux qu’il peut encore y avoir, se résument dans cette parole : Tu aimeras ton prochain comme toi-même\FTNT{Ex. 20:12-17 ; Mt. 22:39.}.
\VS{10}La charité ne fait point de mal au prochain ; la charité est donc l'accomplissement de la loi.
\VS{11}Cela importe d’autant plus que vous savez en quelle saison nous sommes ; parce qu'il est déjà l’heure de nous réveiller du sommeil ; car maintenant le salut est plus près de nous que lorsque nous avons cru.
\VS{12}La nuit est avancée\FTNT{Mt. 25:1-13.} et le jour approche. Rejetons donc les œuvres des ténèbres, et soyons revêtus des armes de lumière.
\VS{13}Marchons honnêtement, comme en plein jour, loin des orgies\FTNT{Orgies : Du grec «~komos~». Ce terme désigne la procession nocturne et rituelle, qui avait lieu après un souper, de gens à moitié ivres, à l'esprit folâtre, qui défilaient à travers les rues avec torches et musique en l'honneur de Bacchus ou quelque autre divinité, et chantaient et jouaient devant les maisons de leurs amis, hommes ou femmes. Ce mot est aussi utilisé pour les fêtes et beuveries de nuit qui se terminaient en orgies.} et de l’ivrognerie, de la luxure et de la débauche, des querelles et des jalousies.
\VS{14}Mais revêtez-vous du Seigneur Jésus-Christ, et n'ayez point soin de la chair pour en satisfaire les convoitises.
\TextTitle{[L'attitude du chrétien face aux opinions différentes]
\\(cp. 1 Co. 8:1-10:33}
\Chap{14}
\VerseOne{}Or quant à celui qui est faible dans la foi, recevez-le, et n'ayez point avec lui des discussions sur les opinions.
\VS{2}L'un croit qu'on peut manger de tout, et l'autre, qui est faible, mange des légumes.
\VS{3}Que celui qui mange de tout ne méprise pas celui qui n'en mange point ; et que celui qui n'en mange point, ne juge point celui qui en mange, car Dieu l'a accueilli.
\VS{4}Qui es-tu, toi qui juges le serviteur d'autrui ? S’il se tient ferme ou s'il tombe, c’est à son maître de le juger ; mais il sera affermi, car Dieu est Puissant pour l'affermir.
\VS{5}Tel fait une distinction entre les jours, tel autre les estime tous égaux. Que chacun ait en son esprit une pleine conviction.
\VS{6}Celui qui distingue entre les jours agit ainsi pour le Seigneur. Celui qui mange, c’est pour le Seigneur qu’il mange, car il rend grâces à Dieu ; celui qui ne mange pas, c’est pour le Seigneur qu’il ne mange pas, et il rend grâces à Dieu.
\VS{7}Car nul de nous ne vit pour lui-même, et nul ne meurt pour lui-même.
\VS{8}Car si nous vivons, nous vivons pour le Seigneur ; et si nous mourons, nous mourons pour le Seigneur. Soit donc que nous vivions, soit que nous mourions, nous sommes au Seigneur.
\VS{9}Car c'est pour cela que Christ est mort, qu'il est ressuscité, et qu'il a repris la vie, afin de dominer sur les morts et sur les vivants.
\VS{10}Mais toi, pourquoi juges-tu ton frère ? Ou toi, pourquoi méprises-tu ton frère ? Puisque nous comparaîtrons tous devant le tribunal de Christ.
\VS{11}Car il est écrit : Je suis vivant, dit le Seigneur, tout genou fléchira devant moi, et toute langue donnera gloire à Dieu\FTNT{Es. 45:23 ; Ph. 2:10-11.}.
\VS{12} Ainsi, chacun de nous rendra compte à Dieu pour lui-même.
\TextTitle{Se garder d'être une occasion de chute}
\VS{13}Ne nous jugeons donc plus les uns les autres ; mais pensez plutôt à ne rien faire qui soit pour votre frère une pierre d’achoppement ou une occasion de chute.
\VS{14}Je sais, et je suis persuadé par le Seigneur Jésus, que rien n'est souillé en soi, et qu’une chose n’est souillée que par celui qui la croit souillée.
\VS{15}Mais si ton frère est attristé au sujet d’un aliment, tu ne marches plus selon la charité ; ne détruis point, par ton aliment, celui pour qui Christ est mort.
\VS{16}Que votre privilège ne soit pas un sujet de calomnie.
\VS{17}Car le Royaume de Dieu ne consiste ni dans le manger ni dans le boire, mais dans la justice, la paix et la joie par le Saint-Esprit.
\VS{18}Celui qui sert Christ de cette manière est agréable à Dieu et approuvé des hommes.
\VS{19}Recherchons donc ce qui contribue à la paix et à l’édification mutuelle.
\VS{20}Ne détruis pas l’œuvre de Dieu pour un aliment. Il est vrai que toutes choses sont pures, mais il est mal à l’homme, quand il mange, de devenir une pierre d’achoppement.
\VS{21}Il est bien de ne pas manger de viande, de ne pas boire de vin, et de s’abstenir de ce qui peut être pour ton frère une occasion de chute, de scandale ou de faiblesse.
\VS{22}As-tu la foi ? Garde-la devant Dieu. Heureux est celui qui ne se condamne pas lui-même dans ce qu'il approuve.
\VS{23}Mais celui qui a des doutes au sujet de ce qu’il mange est condamné, parce qu’il n’agit pas avec foi. Tout ce que l’on ne fait pas avec foi est un péché.
\Chap{15}
\VerseOne{}Nous devons, nous qui sommes forts, supporter les infirmités des faibles, et ne pas nous complaire en nous-mêmes.
\VS{2}Que chacun de nous plaise au prochain pour ce qui est bien, en vue de l’édification.
\VS{3}Car même Jésus-Christ n'a pas cherché ce qui lui plaisait, mais, selon qu’il est écrit : Les outrages de ceux qui t’insultent sont tombés sur moi\FTNT{Ps. 69:10.}.
\TextTitle{Les Juifs et gentils rachetés par un même salut}
\VS{4}Or tout ce qui a été écrit autrefois, a été écrit pour notre instruction, afin que par la patience et la consolation que donnent les Ecritures, nous possédions l’espérance.
\VS{5}Que le Dieu de patience et de consolation vous donne d’avoir les mêmes sentiments les uns envers les autres, selon Jésus-Christ,
\VS{6}afin que tous d'un même cœur et d'une même bouche, vous glorifiiez Dieu, qui est le Père de notre Seigneur Jésus-Christ.
\VS{7}C'est pourquoi, accueillez-vous les uns les autres, comme Christ nous a accueillis, pour la gloire de Dieu.
\VS{8}Je dis donc que Jésus-Christ a été Ministre des circoncis, pour prouver la vérité de Dieu, afin de confirmer les promesses faites aux pères,
\VS{9}afin que les gentils glorifient Dieu pour sa miséricorde, selon ce qui est écrit : C’est pourquoi je te louerai parmi les nations, et je chanterai à la gloire de ton Nom\FTNT{Ps. 18:50.}. Et il est dit encore :
\VS{10}Nations, réjouissez-vous avec son peuple\FTNT{De. 32:43.} !
\VS{11}Et encore : Louez le Seigneur, vous toutes les nations, et célébrez-le, vous tous les peuples\FTNT{Ps. 117:1.}. Esaïe dit aussi :
\VS{12}Il sortira d’Isaï un rejeton, qui se lèvera pour régner sur les nations ; les nations espéreront en lui\FTNT{Es. 11:1 ; Es. 11:10.}.
\VS{13}Que le Dieu de l’espérance vous remplisse de toute joie et de toute paix, dans la foi, afin que vous abondiez en espérance par la puissance du Saint-Esprit.
\TextTitle{Paul envisage d'aller à Jérusalem, à Rome et en Espagne}
\VS{14}Pour moi, mes frères, je suis persuadé que vous êtes pleins de bonté, remplis de toute connaissance, et capables de vous exhorter les uns les autres.
\VS{15}Cependant, mes frères, je vous ai écrit en quelque sorte plus librement, comme pour réveiller vos souvenirs, à cause de la grâce que Dieu m’a faite,
\VS{16}d’être ministre de Jésus Christ parmi les gentils ; je m’acquitte du divin service de l'Evangile de Dieu, afin que les gentils lui soient une offrande agréable, étant sanctifiée par le Saint-Esprit.
\VS{17}J'ai donc sujet de me glorifier en Jésus-Christ pour ce qui regarde les choses de Dieu.
\VS{18}Car je n’oserais parler de quoi que ce soit que Christ n’ait opéré par moi, pour amener les gentils à son obéissance, par la parole et par les œuvres,
\VS{19}par la puissance des prodiges et des miracles, par la puissance de l'Esprit de Dieu. Ainsi, depuis Jérusalem et les pays voisins jusqu’en Illyrie, j’ai abondamment répandu l’Evangile de Christ.
\VS{20}M'attachant ainsi avec affection à annoncer l’Evangile là où Christ n’avait point encore été prêché, afin que je ne bâtisse pas sur le fondement qu'un autre a déjà posé. 
\VS{21}Mais selon qu'il est écrit : Ceux à qui il n'a point été annoncé le verront ; et ceux qui n'en avaient point entendu parler l’entendront\FTNT{Es. 52:15.}.
\VS{22}Et c’est ce qui m'a souvent empêché d’aller vous voir.
\VS{23}Mais maintenant, n’ayant plus rien qui me retienne dans ces contrées, et ayant depuis plusieurs années le désir d’aller vers vous,
\VS{24}j’espère vous voir en passant, quand je me rendrai en Espagne, et y être accompagné par vous, après que j’aurai satisfait en partie mon désir de me trouver chez vous.
\VS{25}Maintenant je vais à Jérusalem pour assister les saints.
\VS{26}Car il a semblé bon à ceux de Macédoine et d’Achaïe de s’imposer une contribution pour les pauvres parmi les saints de Jérusalem.
\VS{27}Ils l’ont bien voulu, et ils le leur devaient, car si les gentils ont eu part à leurs avantages spirituels, ils doivent aussi les assister dans les choses temporelles.
\VS{28}Dès que j'aurai achevé cette affaire, et que je leur aurai remis ce fruit, j'irai en Espagne en passant par vos quartiers.
\VS{29}Et je sais qu’en allant vers vous, j’irai avec une pleine bénédiction de l'Evangile de Christ.
\VS{30}Je vous exhorte, mes frères, par notre Seigneur Jésus-Christ, et par la charité de l'Esprit, à combattre avec moi en adressant des prières à Dieu en ma faveur,
\VS{31}afin que je sois délivré des incrédules de Judée, et que mon ministère\FTNT{Ministère : Du grec «~diakonia~», terme qui désigne le service ou le ministère de ceux qui répondent aux besoins des autres. Ce vocable fait aussi allusion à l’office des diacres.} à Jérusalem soit agréable aux saints,
\VS{32}en sorte que, par la volonté de Dieu, j’arrive chez vous avec joie, et que je me repose avec vous.
\VS{33}Que le Dieu de paix soit avec vous tous. Amen !
\Chap{16}
\TextTitle{Salutations personnelles de Paul}
\VerseOne{}Je vous recommande notre sœur Phœbé, qui est diaconesse de l'église de Cenchrées,
\VS{2}afin que vous la receviez selon le Seigneur, comme il faut recevoir les saints, et que vous l'assistiez dans tout ce dont elle aura besoin ; car elle a exercé l'hospitalité à l'égard de plusieurs, et même à mon égard.
\VS{3}Saluez Priscille et Aquilas, mes compagnons d’œuvre en Jésus-Christ,
\VS{4}qui ont exposé leur cou pour ma vie ; ce n’est pas moi seul qui leur rends grâces, mais aussi toutes les églises des gentils.
\VS{5}Saluez aussi l'église qui est dans leur maison. Saluez Epaïnète, mon bien-aimé, qui a été pour Christ les prémices d'Achaïe.
\VS{6}Saluez Marie, qui a beaucoup travaillé pour nous.
\VS{7}Saluez Andronicus et Junias, mes parents, qui ont été prisonniers avec moi, et qui sont distingués parmi les apôtres, et qui ont même été en Christ avant moi.
\VS{8}Saluez Amplias, mon bien-aimé dans le Seigneur.
\VS{9}Saluez Urbain, notre compagnon d’œuvre en Christ, et Stachys, mon bien-aimé.
\VS{10}Saluez Apellès, qui est éprouvé en Christ. Saluez ceux de chez Aristobule.
\VS{11}Saluez Hérodion, mon parent. Saluez ceux de chez Narcisse qui sont dans le Seigneur.
\VS{12}Saluez Tryphène et Tryphose, qui travaillent pour le Seigneur. Saluez Perside, la bien-aimée qui a beaucoup travaillé pour le Seigneur.
\VS{13}Saluez Rufus, l’élu du Seigneur, et sa mère, qui est aussi la mienne.
\VS{14}Saluez Asyncrite, Phlégon, Hermas, Patrobas, Hermès, et les frères qui sont avec eux.
\VS{15}Saluez Philologue et Julie, Nérée et sa sœur, et Olympe, et tous les saints qui sont avec eux.
\VS{16}Saluez-vous les uns les autres par un saint baiser. Les églises de Christ vous saluent.
\TextTitle{Se garder de ceux qui causent des divisions et des scandales}
\VS{17}Je vous exhorte, mes frères, à prendre garde à ceux qui causent des divisions et des scandales contre la doctrine que vous avez apprise. Eloignez-vous d'eux.
\VS{18}Car de tels hommes ne servent point notre Seigneur Jésus-Christ, mais leur propre ventre, et par des paroles douces et flatteuses, ils séduisent les cœurs des simples.
\VS{19}Pour vous, votre obéissance est connue de tous ; je me réjouis donc à votre sujet, et je désire que vous soyez sages à l’égard du bien, et purs à l’égard du mal.
\VS{20}Le Dieu de paix brisera bientôt Satan sous vos pieds. Que la grâce de notre Seigneur Jésus-Christ soit avec vous. Amen !
\VS{21}Timothée, mon compagnon d’œuvre, vous salue, ainsi que Lucius, et Jason et Sosipater, mes parents.
\VS{22}Je vous salue dans le Seigneur, moi Tertius, qui ai écrit cette lettre.
\VS{23}Gaïus, mon hôte, et celui de toute l'église, vous salue. Eraste, l’économe de la ville, vous salue, et Quartus, notre frère.
\TextTitle{Bénédiction}
\VS{24}Que la grâce de notre Seigneur Jésus-Christ soit avec vous tous. Amen !
\VS{25}or à celui qui est puissant pour vous affermir selon mon Evangile, et selon la prédication de Jésus-Christ, conformément à la révélation du mystère qui a été caché dans les temps passés.
\VS{26}mais manifesté maintenant par les écrits des prophètes, d’après l’ordre du Dieu éternel, et porté à la connaissance de toutes les nations, afin qu’elles obéissent à la foi.
\VS{27}A Dieu, seul sage, soit la gloire éternellement, par Jésus-Christ. Amen !
\PPE{}
\end{multicols}

%\clearpage\ShortTitle{Ep.}\BookTitle{Ephésiens}\BFont
\noindent\hrulefill
{\footnotesize
\textit{
\bigskip
{\centering{}
\\Auteur~: Paul
\\Thème~: L'Eglise, corps de Christ
\\Date de rédaction~: Env. 60 ap. J.-C.\\}
}
\textit{
\\Ephèse figurait parmi les principales villes de l'Empire romain sous le règne de l'empereur Claude Ier (10 av. J.-C. – 54 ap. J.-C.). Bien que Pergame était considérée comme la capitale de l'Asie Mineure, en raison de sa position géographique et grâce à ses affluents, Ephèse possédait le plus grand port de la région, ce qui lui a valu le contrôle du trafic commercial. Richissime et prospère, elle était renommée pour son faste et sa liberté de parole, et constituait donc un endroit privilégié pour les philosophes. C'était une ville où l'activité culturelle tenait une grande place (Jeux olympiques, théâtres, cirques, etc.) et où chacun pouvait y pratiquer la religion de son choix (croyances gréco-romaines, égyptiennes, judaïque etc.).
\\Ephèse, dont le nom signifie «~désirable~», était la gardienne de l'Artémision, temple dédié à la déesse grecque Artémis, la Diane des Ephésiens.
\\L'église d'Ephèse vit le jour lors du second voyage missionnaire de Paul (50-52). Quand il repartit, il laissa à Aquilas et Priscille la charge de la toute jeune assemblée. Paul s'installa à Ephèse lors de son troisième voyage (53-57) et y demeura presque trois ans. Il discourut pendant trois mois dans la synagogue sur le Royaume de Dieu, mais se retrouva confronté à l'endurcissement de certains. C'est alors qu'il se retira pour enseigner dans l'école d'un certain Tyrannus durant deux ans, de sorte que tous ceux qui habitaient l'Asie, Juifs et Grecs, entendirent parler de Jésus-Christ.
\\Plusieurs de ceux qui avaient cru confessèrent leurs péchés et un certain nombre de ceux qui avaient pratiqué la magie allèrent même jusqu'à brûler leurs livres publiquement. C'est ainsi que l'église d'Ephèse croissait en puissance et en force. La prédication de Paul vint troubler le marché fructueux des fabricants d'idoles au point que Démétrius (orfèvre tirant un grand profit de cette industrie) entraina une émeute contre lui. Paul était cependant soutenu par des amis influents~: Les asiarques.
\\Rédigée en prison, cette épître a pour vocation d'enseigner les chrétiens d'Ephèse sur la manière dont il convient de vivre les uns avec les autres au sein de l'Eglise, corps du Christ.\bigskip
}
}
\par\nobreak\noindent\hrulefill
\begin{multicols}{2}
\Chap{1}
\TextTitle{Introduction}
\VerseOne{}Paul, apôtre de Jésus-Christ par la volonté de Dieu, aux saints et fidèles en Jésus-Christ qui sont à Ephèse~:
\VS{2}Que la grâce et la paix vous soient données par Dieu notre Père, et par le Seigneur Jésus-Christ~!
\TextTitle{La position des élus dans le Royaume de Dieu}
\VS{3}Béni soit Dieu, qui est le Père de notre Seigneur Jésus-Christ, qui nous a bénis de toutes bénédictions spirituelles dans les lieux célestes en Christ~!
\VS{4}Selon qu'il nous a élus en lui avant la fondation du monde, afin que nous soyons saints et irrépréhensibles devant lui dans la charité, 
\VS{5}nous ayant prédestinés pour nous adopter pour lui par Jésus-Christ, selon le bon plaisir de sa volonté,
\VS{6}à la louange de la gloire de sa grâce, par laquelle il nous a rendus agréables en son bien-aimé.
\VS{7}En lui nous avons la rédemption par son sang, à savoir la rémission des offenses, selon les richesses de sa grâce,
\VS{8}qu'il a fait abonder sur nous en toute sagesse et intelligence,
\VS{9}nous ayant donné à connaître le mystère de sa volonté, qu'il avait premièrement arrêté en lui-même,
\VS{10}afin que dans l'accomplissement des temps qu'il avait réglés, il réunit tout en Christ, tant ce qui est dans les cieux, que ce qui est sur la terre, en lui-même. 
\VS{11}En qui nous sommes aussi devenus héritiers, ayant été prédestinés, suivant la résolution de celui qui accomplit toutes choses avec efficacité selon le conseil de sa volonté,
\VS{12}afin que nous soyons à la louange de sa gloire, nous qui avons les premiers espéré en Christ.
\VS{13}En qui vous êtes aussi, ayant entendu la parole de la vérité, qui est l'Evangile de votre salut, et auquel ayant cru, vous avez été scellés du Saint-Esprit qui avait été promis,
\VS{14}lequel est le gage de notre héritage jusqu'à la rédemption de ceux qu'il s'est acquis à la louange de sa gloire.
\VS{15}C'est pourquoi, ayant aussi entendu parler de la foi que vous avez en notre Seigneur Jésus, et de la charité que vous avez envers tous les saints,
\VS{16}je ne cesse de rendre grâces pour vous dans mes prières,
\VS{17}afin que le Dieu de notre Seigneur Jésus-Christ, le Père de gloire, vous donne l'Esprit de sagesse et de révélation, dans ce qui regarde sa connaissance.
\VS{18}Qu'il illumine les yeux de votre esprit, afin que vous sachiez quelle est l'espérance de sa vocation, et quelles sont les richesses de la gloire de son héritage qu'il réserve aux saints,
\VS{19}et quelle est l'excellente grandeur de sa puissance envers nous qui croyons selon l'efficacité de la puissance de sa force, 
\VS{20}qu'il a déployée avec efficacité en Christ, quand il l'a ressuscité des morts et qu'il l'a fait asseoir à sa droite dans les lieux célestes,
\VS{21}au-dessus de toute principauté, de toute puissance, de toute dignité et de toute domination, et au-dessus de tout nom qui se nomme, non seulement dans le siècle présent, mais aussi dans celui qui est à venir.
\TextTitle{Le Messie est le Chef suprême de l'Eglise}
\VS{22}Et il a assujetti toutes choses sous ses pieds, et l'a établi sur toutes choses pour être le Chef de l'Eglise,
\VS{23}qui est son corps, et la plénitude de celui qui remplit tout en tous.
\Chap{2}
\TextTitle{Le salut par la grâce}
\VerseOne{}Et vous étiez morts par vos offenses et par vos péchés,
\VS{2}dans lesquels vous marchiez autrefois, suivant le train de ce monde, selon le prince de la puissance de l'air, qui est l'esprit qui agit maintenant avec efficacité dans les fils rebelles à Dieu,
\VS{3}parmi lesquels nous vivions tous autrefois, selon les convoitises de notre chair, accomplissant les désirs de la chair et de nos pensées. Et nous étions par nature des enfants de colère comme les autres.
\VS{4}Mais Dieu, qui est riche en miséricorde, à cause de sa grande charité dont il nous a aimés,
\VS{5}lorsque nous étions morts dans nos offenses, il nous a vivifiés ensemble avec Christ~; c'est par grâce que vous êtes sauvés.
\VS{6}Et il nous a ressuscités ensemble, et nous a fait asseoir ensemble dans les lieux célestes en Jésus-Christ,
\VS{7}afin qu'il montre dans les siècles à venir les immenses richesses de sa grâce par sa bonté envers nous, en Jésus-Christ.
\VS{8}Car vous êtes sauvés par la grâce, par la foi~; et cela ne vient pas de vous, c'est le don de Dieu~;
\VS{9}non pas par les œuvres, afin que personne ne se glorifie.
\VS{10}Car nous sommes son ouvrage, ayant été créés en Jésus-Christ pour les bonnes œuvres que Dieu a préparées d'avance, afin que nous marchions en elles.
\VS{11}C'est pourquoi, souvenez-vous que vous qui étiez autrefois Gentils dans la chair, et qui étiez appelés incirconcis par ceux qu'on appelle circoncis, et qui le sont dans la chair par la main des hommes, 
\VS{12}vous étiez en ce temps-là sans Christ, privés du droit de cité en Israël, étant étrangers des alliances de la promesse, n'ayant pas d'espérance, et étant sans Dieu dans le monde.
\VS{13}Mais maintenant, par Jésus-Christ, vous qui étiez autrefois éloignés, vous avez été rapprochés par le sang de Christ.
\TextTitle{Juifs et Gentils forment un seul corps}
\VS{14}Car il est notre paix, lui qui des deux n'en a fait qu'un en détruisant le mur de séparation,
\VS{15}ayant aboli dans sa chair l'inimitié, à savoir la loi des commandements qui consiste en ordonnances, afin de créer les deux en lui-même pour être un homme nouveau, en faisant la paix~;
\VS{16}et de réconcilier les uns et les autres avec Dieu pour former un seul corps par sa croix, ayant détruit par elle l'inimitié.
\VS{17}Et il est venu prêcher la paix à vous qui étiez loin, et à ceux qui étaient près,
\VS{18}car nous avons par lui les uns et les autres accès auprès du Père dans un même Esprit.
\TextTitle{L'Eglise véritable}
\VS{19}C'est pourquoi vous n'êtes plus des étrangers ni des gens de dehors, mais concitoyens des saints et gens de la maison de Dieu~;
\VS{20}étant édifiés sur le fondement\FTNT{Le fondement a été posé une fois pour toutes par les apôtres et les prophètes. Et ce fondement est notre Seigneur Jésus-Christ (1 Co. 3:11).} des apôtres et des prophètes, et Jésus-Christ lui-même étant la pierre angulaire~;
\VS{21}en qui tout l'édifice, bien ajusté ensemble, s'élève pour être un temple saint dans le Seigneur,
\VS{22}en qui vous êtes édifiés ensemble, pour être une habitation de Dieu en Esprit.
\Chap{3}
\TextTitle{Le mystère caché de tout temps\FTNTT{Col. 1:24-27.}}
\VerseOne{}C'est pour cela que moi, Paul, je suis prisonnier de Jésus-Christ pour vous Gentils.
\VS{2}Si toutefois vous avez entendu quelle est la gestion de la grâce de Dieu qui m'a été donnée pour vous,
\VS{3}comment par révélation ce mystère m'a été manifesté, ainsi que je l'ai écrit ci-dessus en peu de mots~;
\VS{4}d'où vous pouvez voir en le lisant, quelle est l'intelligence que j'ai du mystère de Christ,
\VS{5}lequel n'a pas été manifesté aux fils des hommes dans les autres générations, comme il a été révélé maintenant par l'Esprit à ses saints apôtres et à ses prophètes,
\VS{6}à savoir que les Gentils sont cohéritiers et d'un même corps, et qu'ils participent ensemble à sa promesse en Christ par l'Evangile,
\VS{7}dont j'ai été fait serviteur, selon le don de la grâce de Dieu qui m'a été donnée selon l'efficacité de sa puissance.
\VS{8}Cette grâce, dis-je, m'a été donnée à moi, qui suis le moindre de tous les saints, pour annoncer parmi les Gentils les richesses incompréhensibles de Christ,
\VS{9}et pour mettre en évidence devant tous quelle est la communion qui nous a été accordée du mystère qui était caché de tout temps en Dieu, lequel a créé toutes choses par Jésus-Christ,
\VS{10}afin que les principautés et les puissances dans les lieux célestes connaissent aujourd'hui par l'Eglise la sagesse infiniment variée de Dieu,
\VS{11}suivant le dessein arrêté dès les siècles, qu'il a établi en Jésus-Christ, notre Seigneur,
\VS{12}par lequel nous avons hardiesse et accès avec confiance, par la foi que nous avons en lui.
\VS{13}C'est pourquoi, je vous prie de ne pas vous relâcher à cause de mes afflictions que je souffre pour l'amour de vous, ce qui est votre gloire.
\VS{14}A cause de cela, je fléchis mes genoux devant le Père de notre Seigneur Jésus-Christ,
\VS{15}duquel toute parenté est nommée dans les cieux et sur la terre,
\VS{16}afin que selon les richesses de sa gloire, il vous donne d'être puissamment fortifiés par son Esprit dans l'homme intérieur,
\VS{17}en sorte que Christ habite dans vos cœurs par la foi~; afin qu'étant enracinés et fondés dans la charité,
\VS{18}vous puissiez comprendre avec tous les saints quelle est la largeur et la longueur, la profondeur et la hauteur,
\VS{19}et connaître la charité de Christ qui surpasse toute connaissance, afin que vous soyez remplis de toute la plénitude de Dieu.
\VS{20}Or à celui qui par la puissance qui agit en nous avec efficacité, peut faire infiniment au-delà de tout ce que nous demandons et pensons,
\VS{21}à lui soit la gloire dans l'Eglise, en Jésus-Christ, dans toutes les générations, aux siècles des siècles~! Amen~!
\Chap{4}
\TextTitle{L'unité}
\VerseOne{}Je vous prie donc, moi, le prisonnier dans le Seigneur, à marcher d'une manière digne de la vocation à laquelle vous êtes appelés,
\VS{2}avec toute humilité et douceur, avec patience, vous supportant les uns les autres dans la charité,
\VS{3}vous efforçant de garder l'unité de l'Esprit par le lien de la paix.
\VS{4}Il y a un seul corps, un seul Esprit, comme aussi vous êtes appelés à une seule espérance par votre vocation~;
\VS{5}il y a un seul Seigneur, une seule foi, un seul baptême,
\VS{6}un seul Dieu et Père de tous, qui est au-dessus de tous, parmi tous, et en vous tous.
\TextTitle{Les dons de Christ pour le perfectionnement et l'édification de son Corps\FTNTT{1 Co. 12:4-11.}}
\VS{7}Mais la grâce est donnée à chacun de nous selon la mesure du don de Christ.
\VS{8}C'est pourquoi il est dit~: Etant monté en haut, il a emmené captive une grande multitude de captifs, et il a donné des dons aux hommes\FTNT{Ps. 68:19.}.
\VS{9}Or que signifie~: Il est monté, sinon qu'il est premièrement descendu dans les parties les plus basses de la terre~?
\VS{10}Celui qui est descendu, c'est le même qui est monté au-dessus de tous les cieux, afin de remplir toutes choses.
\VS{11}Lui-même donc a donné les uns pour être apôtres, les autres pour être prophètes, les autres pour être évangélistes, les autres pour être pasteurs et docteurs,
\VS{12}pour travailler au perfectionnement\FTNT{Le mot «~perfectionnement~» vient du grec «~katartismos~», qui tire son origine du terme «~katartizo~»~: «~redresser, ajuster, compléter, raccommoder (ce qui a été abîmé), réparer~». Ainsi, les divers services ont vocation, d'une part, à réparer les dégâts causés par le péché dans les âmes, et d'autre part, à préparer les disciples à rentrer à leur tour dans leur propre service.} des saints, pour l'œuvre du service\FTNT{Le mot «~service~» vient du grec «~diakonos~», il signifie «~service, ministère de ceux qui répondent aux besoins des autres~». Jésus-Christ lui-même a pris la forme d'un serviteur pour nous servir et non pour être servi (Mt. 20:28).}, pour l'édification du corps de Christ,
\VS{13}jusqu'à ce que nous soyons tous parvenus à l'unité de la foi et de la connaissance du Fils de Dieu, à l'état d'homme parfait, à la mesure de la parfaite stature de Christ,
\VS{14}afin que nous ne soyons plus des enfants flottants et emportés çà et là à tous vents de doctrine, par la tromperie des hommes et par leur ruse à séduire artificieusement.
\VS{15}Mais afin que, suivant la vérité avec la charité, nous croissions en toutes choses en celui qui est le Chef, c'est-à-dire Christ,
\VS{16}dont tout le corps bien ajusté et lié ensemble par toutes les jointures de son assistance, tire son accroissement selon la force qu'il distribue à chaque membre, afin qu'il soit édifié dans la charité.
\TextTitle{Se dépouiller du vieil homme}
\VS{17}Je vous dis donc, et je vous conjure de la part du Seigneur, de ne plus vous conduire comme le reste des Gentils qui suivent la vanité de leurs pensées.
\VS{18}Ils ont l'intelligence obscurcie par les ténèbres et sont étrangers à la vie de Dieu, à cause de l'ignorance qui est en eux, par l'endurcissement de leur cœur.
\VS{19}Ils ont perdu tout sentiment, et se sont abandonnés à la dissolution pour commettre toute sorte d'impureté avec cupidité.
\VS{20}Mais vous n'avez pas ainsi appris Christ,
\VS{21}si toutefois vous l'avez entendu, et si vous avez été enseignés par lui~, selon que la vérité est en Jésus~; 
\VS{22}à savoir que vous dépouilliez le vieil homme, pour ce qui est de votre conduite précédente, qui se corrompt par les convoitises qui séduisent~; 
\VS{23}et que vous soyez renouvelés dans l'esprit de votre entendement,
\VS{24}et que vous soyez revêtus du nouvel homme, créé selon Dieu dans une justice et une sainteté véritables.
\VS{25}C'est pourquoi, ayant dépouillé le mensonge, parlez en vérité chacun avec son prochain~; car nous sommes membres les uns des autres.
\VS{26}Si vous vous mettez en colère, ne péchez pas, que le soleil ne se couche pas sur votre colère.
\VS{27}Ne donnez pas lieu au diable de vous perdre.
\VS{28}Que celui qui dérobait ne dérobe plus~; mais plutôt qu'il travaille en faisant de ses mains ce qui est bon, pour avoir de quoi donner à celui qui est dans le besoin.
\VS{29}Qu'aucun discours malhonnête ne sorte de votre bouche, mais seulement celui qui est propre à édifier, afin qu'il soit agréable à ceux qui l'écoutent.
\VS{30}Et n'attristez pas le Saint-Esprit de Dieu, par lequel vous avez été scellés pour le jour de la rédemption.
\VS{31}Que toute amertume, toute colère, toute irritation, toute clameur, toute médisance, et toute malice soient bannies du milieu de vous.
\VS{32}Mais soyez doux les uns envers les autres, pleins de compassion, et vous pardonnant les uns aux autres, ainsi que Dieu vous a pardonné par Christ.
\Chap{5}
\VerseOne{}Soyez donc les imitateurs de Dieu, comme ses enfants bien-aimés~;
\VS{2}et marchez dans la charité, ainsi que Christ nous a aimés et s'est livré lui-même pour nous comme une offrande et un sacrifice de bonne odeur à Dieu.
\VS{3}Que la fornication, ni aucune impureté, ni la cupidité, ne soient pas même nommées parmi vous, ainsi qu'il est convenable à des saints.
\VS{4}Qu'on n'entende ni parole grossière, ni propos insensés, ni plaisanterie, choses qui sont contraires à la bienséance, mais plutôt des actions de grâces.
\VS{5}Car sachez-le bien qu'aucun fornicateur, ni impur, ni cupide, qui est un idolâtre, n'a d'héritage dans le Royaume de Christ et de Dieu.
\VS{6}Que personne ne vous séduise par de vains discours~; car à cause de ces choses la colère de Dieu vient sur les fils de la rébellion.
\VS{7}Ne soyez donc pas leurs associés.
\VS{8}Car vous étiez autrefois ténèbres, mais maintenant vous êtes lumière dans le Seigneur. Conduisez-vous donc comme des enfants de la lumière~!
\VS{9}car le fruit de l'Esprit consiste en toute bonté, justice et vérité,
\VS{10}éprouvant ce qui est agréable au Seigneur~;
\VS{11}et ne participez pas aux œuvres infructueuses des ténèbres, mais au contraire condamnez-les~!
\VS{12}Car il est honteux de dire les choses qu'ils font en secret~;
\VS{13}mais toutes choses, étant mises en évidence par la lumière, sont rendues manifestes, car la lumière est celle qui manifeste tout.
\VS{14}C'est pourquoi il est dit~: Réveille-toi, toi qui dors, et relève-toi d'entre les morts, et Christ t'éclairera\FTNT{Es. 60:1.}.
\VS{15}Prenez donc garde de vous conduire soigneusement, non pas comme étant dépourvus de sagesse, mais comme étant sages,
\VS{16}rachetant le temps, car les jours sont mauvais.
\VS{17}C'est pourquoi ne soyez pas sans intelligence, mais comprenez bien quelle est la volonté du Seigneur.
\VS{18}Et ne vous enivrez pas du vin dans lequel il y a de la dissolution, mais soyez remplis de l'Esprit.
\VS{19}Entretenez-vous par des psaumes, des hymnes et des cantiques spirituels, chantant et psalmodiant de votre cœur au Seigneur~;
\VS{20}rendez toujours grâces pour toutes choses à Dieu notre Père, au Nom de notre Seigneur Jésus-Christ~;
\VS{21}soumettez-vous les uns aux autres dans la crainte de Christ.
\TextTitle{Le mariage selon Dieu}
\VS{22}Femmes, soyez soumises à vos maris comme au Seigneur~;
\VS{23}car le mari est le chef de la femme, comme Christ est le Chef de l'Eglise, qui est son corps, et dont il est le Sauveur.
\VS{24}Or de même que l'Eglise est soumise à Christ, les femmes aussi doivent l'être à leurs maris en toutes choses.
\VS{25}Et vous maris, aimez vos femmes, comme Christ a aimé l'Eglise, et s'est livré lui-même pour elle,
\VS{26}afin de la sanctifier en la purifiant et en la lavant par l'eau de la parole~;
\VS{27}afin de faire paraître devant lui cette Eglise glorieuse, sans tache, ni ride, ni rien de semblable, mais sainte et irréprochable.
\VS{28}C'est ainsi que les maris doivent aimer leurs femmes comme leurs propres corps. Celui qui aime sa femme s'aime lui-même,
\VS{29}car personne n'a jamais eu en haine sa propre chair, mais il la nourrit et l'entretient, comme le Seigneur entretient l'Eglise,
\VS{30}car nous sommes membres de son corps étant de sa chair et de ses os.
\VS{31}C'est pourquoi l'homme quittera son père et sa mère et s'attachera à sa femme, et les deux deviendront une seule chair.
\VS{32}Ce mystère est grand, or je parle de Christ et de l'Eglise.
\VS{33}Que chacun de vous donc aime sa femme comme lui-même, et que la femme respecte son mari.
\Chap{6}
\TextTitle{La famille selon Dieu}
\VerseOne{}Enfants, obéissez à vos pères et à vos mères, dans ce qui est selon le Seigneur, car cela est juste.
\VS{2}Honore ton père et ta mère, c'est le premier commandement avec une promesse,
\VS{3}afin que tout aille bien pour toi et que tu vives longtemps sur la terre.
\VS{4}Et vous, pères, n'irritez pas vos enfants, mais élevez-les\FTNT{Le verbe «~élever~» vient du grec «~ektrepho~» qui signifie nourrir jusqu'à maturité.} en les instruisant et les avertissant selon le Seigneur.
\TextTitle{Les rapports entre les maîtres et les serviteurs selon Dieu}
\VS{5}Serviteurs, obéissez à vos maîtres selon la chair avec crainte et tremblement, dans la simplicité de votre cœur, comme à Christ,
\VS{6}ne les servant pas seulement sous leurs yeux, comme cherchant à plaire aux hommes, mais comme serviteurs de Christ, faisant de bon cœur la volonté de Dieu,
\VS{7}servant avec bienveillance, comme servant le Seigneur et non pas les hommes~;
\VS{8}sachant que chacun, soit esclave, soit libre, recevra du Seigneur le bien qu'il aura fait.
\VS{9}Et vous maîtres, faites envers eux la même chose et renoncez aux menaces, sachant que leur Seigneur et le vôtre est dans les cieux, et qu'il n'y a pas en lui acception de personnes.
\TextTitle{Le combat spirituel}
\VS{10}Au reste, mes frères, fortifiez-vous dans le Seigneur, et dans la puissance de sa force.
\VS{11}Revêtez-vous de toutes les armes de Dieu, afin de pouvoir résister aux embûches du diable.
\VS{12}Car nous n'avons pas à lutter\FTNT{Le mot grec utilisé ici est «~pale~», il était employé pour parler de la lutte entre deux combattants où chacun essaye de renverser l’autre~; la victoire étant acquise par le maintien de l’adversaire au sol, et en lui mettant la main sur la nuque.} contre la chair et le sang, mais contre les principautés, contre les puissances, contre les seigneurs du monde des ténèbres de ce siècle, contre les méchancetés spirituelles qui sont dans les lieux célestes.
\VS{13}C'est pourquoi prenez toutes les armes de Dieu\FTNT{Voir en annexe «~Les armes du chrétien~».}, afin de pouvoir résister dans le mauvais jour, et tenir ferme après avoir tout surmonté.
\VS{14}Soyez donc fermes, ayant à vos reins la vérité pour ceinture, ayant revêtu la cuirasse de la justice~;
\VS{15}et ayant vos pieds chaussés, prêts pour l'Evangile de paix~;
\VS{16}par-dessus tout, prenez le bouclier de la foi, avec lequel vous pourrez éteindre tous les dards enflammés du malin~;
\VS{17}prenez aussi le casque du salut, et l'épée de l'Esprit, qui est la parole de Dieu~;
\VS{18}priant en votre esprit par toutes sortes de prières et de supplications en tout temps, veillant à cela avec une entière persévérance, et priant pour tous les saints,
\VS{19}et pour moi aussi, afin qu'il me soit donné de parler en toute liberté et avec hardiesse, pour faire connaître le mystère de l'Evangile,
\VS{20}pour lequel je suis ambassadeur quoique chargé de chaînes, afin, dis-je, que je parle librement, ainsi qu'il faut que je parle.
\TextTitle{Salutations}
\VS{21}Or afin que vous aussi vous sachiez ce qui me concerne et ce que je fais, Tychique, notre frère bien-aimé et fidèle serviteur du Seigneur, vous fera tout savoir.
\VS{22}Je l'envoie exprès vers vous, afin que vous connaissiez notre situation, et pour qu'il console vos cœurs.
\VS{23}Que la paix soit avec les frères, et la charité avec la foi, de la part de Dieu le Père, et du Seigneur Jésus-Christ~!
\VS{24}Que la grâce soit avec tous ceux qui aiment notre Seigneur Jésus-Christ dans l'incorruptibilité~! Amen~!
\PPE{}
\end{multicols}

%\clearpage\ShortTitle{Ph.}\BookTitle{Philippiens}\BFont
\noindent\hrulefill
{\footnotesize
\textit{
\bigskip
{\centering{}
\\Auteur~: Paul
\\Thème~: Expérience chrétienne
\\Date de rédaction~: Env. 60 ap. J.-C.\\}
}
\textit{
\\Fondée par Philippe II (382 av. J.-C. – 336 av. J.-C.) en 356 av. J.-C., Philippes est une ville grecque de Macédoine orientale. Située sur une voie romaine qui traversait les Balkans (Via Egnatia), elle est restée de taille modeste en dépit de son fort taux de fréquentation.
\\La première mention de l'assemblée de Philippes se trouve dans Actes 16, lors de la rencontre de Paul avec des femmes réunies à l'extérieur de la ville pour la prière. Au travers des paroles de Paul, le Seigneur toucha particulièrement Lydie qui, après avoir été baptisée avec sa famille, reçut Paul et ses compagnons dans sa maison.
\\C'est à Rome, sous le règne de Néron (37-68), que Paul, alors captif, rédigea cette lettre. Cet écrit de l'apôtre accusait réception d'un don monétaire que l'église de Philippes lui avait fait parvenir par le biais d'Epaphrodite. Paul y exprimait sa joie en dépit des souffrances et invitait les Philippiens à faire de même. Loin des erreurs doctrinales reprochées à d'autres, ces chrétiens recevaient ainsi l'expression de l'affection de Paul et ses encouragements à persévérer dans la foi en Christ en toutes circonstances.\bigskip
}
}
\par\nobreak\noindent\hrulefill
\begin{multicols}{2}
\Chap{1}
\TextTitle{Introduction}
\VerseOne{}Paul et Timothée, serviteurs de Jésus-Christ, à tous les saints en Jésus-Christ qui sont à Philippes, avec les évêques et les diacres~:
\VS{2}Que la grâce et la paix vous soient données de la part de Dieu, notre Père et du Seigneur Jésus-Christ~!
\VS{3}Je rends grâces à mon Dieu toutes les fois que je fais mention de vous,
\VS{4}en priant toujours pour vous tous avec joie dans toutes mes prières,
\VS{5}à cause de votre attachement à l'Evangile, depuis le premier jour jusqu'à maintenant.
\VS{6}Etant persuadé de cela même, que celui qui a commencé cette bonne œuvre en vous, l'achèvera jusqu'au jour de Jésus-Christ.
\VS{7}Comme il est juste que je pense ainsi de vous tous, parce que je retiens dans mon cœur, que vous avez tous été participants de la grâce avec moi dans mes liens, et dans la défense et la confirmation de l'Evangile.
\VS{8}Car Dieu m'est témoin que je vous aime tous tendrement, conformément à la charité de Jésus-Christ.
\VS{9}Et je lui demande cette grâce~: Que votre charité abonde encore de plus en plus avec connaissance et toute intelligence,
\VS{10}pour le discernement des choses contraires, afin que vous soyez purs et irréprochables pour le jour de Christ,
\VS{11}étant remplis de fruits de justice, qui sont par Jésus-Christ, à la gloire et à la louange de Dieu.
\TextTitle{Les chrétiens encouragés par la souffrance de Paul}
\VS{12}Or, mes frères, je veux bien que vous sachiez que les choses qui me sont arrivées, sont arrivées pour un plus grand avancement de l'Evangile.
\VS{13}De sorte que mes liens en Christ ont été rendus célèbres dans tout le Prétoire, et partout ailleurs. 
\VS{14}Et que plusieurs de nos frères en notre Seigneur, étant rassurés par mes liens, osent annoncer la parole plus hardiment, et sans crainte. 
\VS{15}Il est vrai que quelques-uns prêchent Christ par envie et par un esprit de dispute~; et que les autres le font, au contraire, par une bonne volonté. 
\VS{16}Les uns, dis-je, annoncent Christ par un esprit de dispute, et non pas purement, croyant ajouter de l'affliction à mes liens.
\VS{17}Mais les autres le font par charité, sachant que je suis établi pour la défense de l'Evangile\FTNT{Les versets 16 et 17 sont inversés dans les versions Segond, Darby et TOB notamment. Ces bibles sont basées sur les textes minoritaires, moins précis. Les versions Martin, Ostervald et King James, basées sur le texte majoritaire (byzantin), utilisent bien cet ordre des versets que l'on retrouvent ainsi dans les écrits grecs.}.
\VS{18}Quoi donc~? Toutefois, de toute manière, que ce soit par ostentation, ou par amour de la vérité, Christ n'est pas moins annoncé. Je m'en réjouis, et je m'en réjouirai encore.
\VS{19}Car je sais que cela tournera à mon salut par vos prières et par le secours de l'Esprit de Jésus-Christ,
\VS{20}selon ma ferme attente et mon espérance, je ne serai confus en rien, mais qu'en toute assurance, Christ sera maintenant, comme il l'a toujours été glorifié dans mon corps, soit par ma vie, soit par ma mort.
\VS{21}Car Christ est ma vie, et la mort m'est un gain.
\VS{22}Mais s'il est utile pour mon œuvre de vivre dans la chair, ce que je dois choisir, je n'en sais rien.
\VS{23}Car je suis pressé des deux côtés~: Mon désir tendant bien à déloger, et à être avec Christ, ce qui me serait beaucoup meilleur. 
\VS{24}Mais il est plus nécessaire pour vous que je demeure dans la chair.
\VS{25}Et je suis persuadé, je sais que je demeurerai et que je resterai avec vous tous, pour votre avancement et pour votre joie dans la foi.
\VS{26}Afin que vous ayez en moi un sujet de vous glorifier de plus en plus en Jésus-Christ, par mon retour au milieu de vous. 
\VS{27}Seulement, conduisez-vous dignement comme il est séant selon l'Evangile de Christ~; afin que, soit que je vienne, et que je vous voie~; soit que je sois absent, j'entende quant à votre état, que vous persistez dans un même esprit, combattant ensemble d'un même courage par la foi de l'Evangile, et n'étant en rien épouvantés par les adversaires.
\VS{28}Ce qui est pour eux une preuve de perdition, mais pour vous de salut~; et cela de la part de Dieu.
\TextTitle{Souffrir pour Christ: Une grâce}
\VS{29}Parce qu'il vous a été gratuitement donné dans ce qui a du rapport à Christ, non seulement de croire en lui, mais aussi de souffrir pour lui,
\VS{30}en soutenant le même combat que vous m'avez vu soutenir, et que vous apprenez maintenant que je soutiens encore.
\Chap{2}
\TextTitle{Exhortation à l'unité}
\VerseOne{}Si donc il y a quelque consolation en Christ, s'il y a quelque soulagement dans la charité, s'il y a quelque communion d'esprit, s'il y a quelques cordiales affections et quelques compassions,
\VS{2}rendez ma joie parfaite, ayant un même sentiment, un même amour, une même âme, et consentant tous à une même chose.
\VS{3}Ne faites rien par esprit de parti\FTNT{Parti~: «~eritheia~» en grec. Avant la Nouvelle Alliance, ce mot ne se trouve que dans les écrits d'Aristote (philosophe grec, disciple de Platon, né en 384 et mort en 322 av. J.-C) où il dénote une «~recherche personnelle, la poursuite d'une fonction politique par des moyens injustes~». Ce mot signifie aussi «~faire une campagne électorale~» ou «~intriguer pour une fonction dans un esprit partisan, querelleur~». On retrouve le mot grec dans Ja. 3:14,16~; Ga. 5:20~; Ro. 2:8~; Ph. 1:16}, ou par vaine gloire~; mais que l'humilité de cœur vous fasse regarder les autres comme étant au-dessus de vous-mêmes.
\VS{4}Ne regardez point chacun, à votre intérêt particulier, mais que chacun ait égard aussi à ce qui concerne les autres.
\TextTitle{L'humilité de Christ}
\VS{5}Qu'il y ait donc en vous un même sentiment qui a été en Jésus-Christ. 
\VS{6}Lequel étant en forme de Dieu, n'a point regardé son égalité avec Dieu comme une usurpation.
\VS{7}Cependant il s'est vidé lui-même, ayant pris la forme de serviteur, fait à la ressemblance des hommes.
\VS{8}Et, étant trouvé en apparence comme un homme, il s'est abaissé lui-même, en se rendant obéissant jusqu'à la mort, même jusqu'à la mort de la croix. 
\VS{9}C'est pourquoi aussi Dieu l'a souverainement élevé, et lui a donné le Nom qui est au-dessus de tout nom~;
\VS{10}afin qu'au Nom de Jésus, tout genou fléchisse, tant de ceux qui sont dans les cieux, que de ceux qui sont sur la terre, et sous la terre,
\VS{11}et que toute langue confesse que Jésus-Christ est le Seigneur, à la gloire de Dieu le Père.
\VS{12}C'est pourquoi, mes bien-aimés, comme vous avez toujours obéi, mettez en œuvre votre propre salut avec crainte et tremblement, non seulement comme en ma présence, mais beaucoup plus maintenant que je suis absent.
\VS{13}Car c'est Dieu qui produit en vous avec efficacité le vouloir et le faire, selon son bon plaisir.
\VS{14}Faites toutes choses sans murmures et sans disputes,
\VS{15}afin que vous soyez sans reproche, et purs, des enfants de Dieu, irrépréhensibles au milieu de la génération corrompue et perverse, parmi lesquels vous brillez comme des flambeaux dans le monde, qui portent au devant d'eux la parole de la vie. 
\VS{16}Pour me glorifier au jour de Christ de n'avoir point couru en vain, ni travaillé en vain. 
\VS{17}Et même si je sers de libation sur le sacrifice et sur le service de votre foi, je m'en réjouis, et je m'en réjouis avec vous tous.
\VS{18}Vous aussi pareillement, réjouissez-vous, et réjouissez-vous avec moi.
\TextTitle{Paul témoigne de Timothée et d'Epaphrodite}
\VS{19}Or j'espère avec la grâce du Seigneur Jésus, vous envoyer bientôt Timothée afin que j'aie aussi plus de courage quand j'aurai connu votre état. 
\VS{20}Car je n'ai personne d'un pareil courage, et qui soit vraiment soigneux de ce qui vous concerne,
\VS{21}parce que tous cherchent leur intérêt particulier, et non les intérêts de Jésus-Christ. 
\VS{22}Mais vous savez l'épreuve que j'ai faite de lui, puisqu'il a servi avec moi en l'Evangile, comme l'enfant sert son père.
\VS{23}J'espère donc vous l'envoyer dès que j'aurai pourvu à mes affaires.
\VS{24}Et j'ai cette confiance en notre Seigneur que moi-même aussi j'irai bientôt.
\VS{25}Mais j'ai cru nécessaire de vous envoyer Epaphrodite, mon frère, mon compagnon d'œuvre et mon compagnon d'armes, par qui vous m'aviez envoyé de quoi pourvoir à mes besoins.
\VS{26}Car aussi il désirait ardemment vous voir tous, et il était fort affligé de ce que vous aviez appris qu'il avait été malade.
\VS{27}En effet, il a été malade et tout près de la mort~; mais Dieu a eu pitié de lui, et non seulement de lui, mais aussi de moi, afin que je n'aie pas tristesse sur tristesse.
\VS{28}Je l'ai donc envoyé à cause de cela avec plus de soin, afin qu'en le revoyant vous ayez de la joie, et que j'aie moins de tristesse.
\VS{29}Recevez-le donc en notre Seigneur, avec toute sorte de joie~; et ayez de l'estime pour ceux qui sont tels que lui. 
\VS{30}Car il a été proche de la mort pour l'œuvre de Christ, n'ayant eu aucun égard à sa propre vie, afin de suppléer au défaut de votre service envers moi.
\Chap{3}
\TextTitle{Le légalisme et la justice de la loi mosaïque}
\VerseOne{}Au reste, mes frères, réjouissez-vous dans le Seigneur. Je ne me lasse point de vous écrire les mêmes choses, mais pour vous c'est une sécurité.
\VS{2}Prenez garde aux chiens~; prenez garde aux mauvais ouvriers~; prenez garde aux faux circoncis.
\VS{3}Car c'est nous qui sommes les circoncis, qui rendons à Dieu notre culte en Esprit, et qui nous glorifions en Jésus-Christ, et qui n'avons point de confiance en la chair.
\VS{4}Moi aussi, cependant, j'aurais sujet de mettre ma confiance en la chair. Si quelqu'un estime qu'il a de quoi se confier en la chair, je le puis bien davantage~:
\VS{5}Moi, circoncis le huitième jour, de la race d'Israël, de la tribu de Benjamin, Hébreu né d'Hébreux, pharisien en ce qui concerne la loi~;
\VS{6}quant au zèle, persécutant l'Eglise~; et quant à la justice à l'égard de la loi, étant sans reproche.
\TextTitle{Le Messie, objet de notre foi} 
\VS{7}Mais ces choses qui étaient pour moi un gain, je les ai regardées comme une perte à cause de l'amour de Christ.
\VS{8}Et certes, je regarde toutes les autres choses comme m'étant nuisibles en comparaison de l'excellence de la connaissance de Jésus-Christ, mon Seigneur, pour l'amour duquel je me suis privé de toutes ces choses, et je les estime comme du fumier, afin de gagner Christ,
\VS{9}et que je sois trouvé en lui, ayant non pas ma justice qui est de la loi, mais celle qui est par la foi en Christ, c'est-à-dire, la justice qui est de Dieu par la foi.
\VS{10}Ainsi, je connaîtrai Jésus-Christ et la puissance de sa résurrection, et la communion de ses souffrances, en devenant conforme à lui dans sa mort, pour parvenir,
\VS{11}si je puis, à la résurrection d'entre les morts.
\VS{12}Non que j'aie déjà atteint le but, ou que je sois déjà rendu parfait, mais je poursuis ce but pour tâcher d'y parvenir~; c'est pourquoi aussi j'ai été pris par Jésus-Christ.
\VS{13}Mes frères, pour moi, je ne me persuade pas d'avoir atteint le but~;
\VS{14}mais je fais une chose~: Oubliant les choses qui sont en arrière, et me portant vers celles qui sont en avant, je cours vers le but, pour remporter le prix de la vocation céleste de Dieu en Jésus-Christ.
\TextTitle{Paul exhorte les croyants à l'unité}
\VS{15}C'est pourquoi, nous tous qui sommes parfaits, ayons ce même sentiment~; et si vous êtes en quelque point d'un autre avis, Dieu vous le révélera aussi.
\VS{16}Cependant, marchons suivant une même règle pour les choses auxquelles nous sommes parvenus, et ayons un même sentiment.
\VS{17}Soyez tous ensemble mes imitateurs, mes frères, et portez les regards sur ceux qui marchent selon le modèle que vous avez en nous.
\VS{18}Car il y en a plusieurs qui marchent d'une telle manière, que je vous ai souvent dit, et maintenant je vous le dis encore en pleurant, qu'ils sont ennemis de la croix de Christ. 
\VS{19}Eux dont la fin est la perdition, qui ont pour dieu leur ventre, et dont la gloire est dans leur confusion, n'ayant d'affection que pour les choses de la terre.
\TextTitle{Le Messie: Notre espérance}
\VS{20}Mais pour nous, notre cité est dans les cieux, d'où nous aussi nous attendons le Sauveur, le Seigneur Jésus-Christ,
\VS{21}qui transformera notre corps vil, afin qu'il soit rendu conforme à son corps glorieux, selon cette efficacité\FTNT{Ep. 3:7} par laquelle il peut même s'assujettir toutes choses. 
\Chap{4}
\TextTitle{Avoir le même sentiment}
\VerseOne{}C'est pourquoi, mes très chers frères bien-aimés, vous qui êtes ma joie et ma couronne, demeurez ainsi fermes dans le Seigneur, mes bien-aimés.
\VS{2}J'exhorte Evodie, et j'exhorte aussi Syntyche, à être d'un même sentiment dans le Seigneur.
\VS{3}Et toi aussi, mon vrai compagnon\FTNT{Le mot compagnon est la traduction du grec «~suzugos~» qui signifie littéralement: Ensemble sous le joug, compagnon de peine, de joug. Cette expression renvoie à 2 Co. 6:14.}, oui je te prie de les aider, elles qui ont combattu avec moi pour l'Evangile, avec Clément, et mes autres compagnons d'œuvre, dont les noms sont écrits dans le livre de vie.
\VS{4}Réjouissez-vous toujours dans le Seigneur~; je vous le répète, réjouissez-vous~!
\VS{5}Que votre douceur soit connue de tous les hommes. Le Seigneur est proche.
\VS{6}Ne vous inquiétez de rien, mais en toutes choses présentez vos demandes à Dieu par des prières et des supplications, avec des actions de grâces.
\VS{7}Et la paix de Dieu, qui surpasse toute intelligence, gardera vos cœurs et vos sentiments en Jésus-Christ.
\TextTitle{L'objet de nos pensées}
\VS{8} Au reste, mes frères, que toutes les choses qui sont véritables, toutes les choses qui sont vénérables, toutes les choses qui sont justes, toutes les choses qui sont pures, toutes les choses qui sont aimables, toutes les choses qui sont de bonne renommée, toutes celles où il y a quelque vertu et quelque louange~; pensez à ces choses.
\VS{9}Car vous les avez aussi apprises, reçues, entendues et vues en moi. Faites ces choses, et le Dieu de paix sera avec vous. 
\TextTitle{Dieu soutient ses serviteurs}
\VS{10}Or je me suis fort réjoui en notre Seigneur, de ce qu'à la fin vous avez fait revivre le soin que vous aviez pour moi~; à quoi aussi vous pensiez, mais vous n'en aviez pas l'occasion. 
\VS{11}Je ne dis pas ceci à cause de mes besoins, car, moi, j'ai appris à être content en moi-même dans les circonstances où je me trouve.
\VS{12}Je sais être abaissé, je sais aussi être dans l'abondance~; partout et en toutes choses je suis instruit tant à être rassasié, qu'à avoir faim~; tant à être dans l'abondance, que dans la disette.
\VS{13}Je puis toutes choses en Christ qui me fortifie.
\VS{14}Néanmoins, vous avez bien fait de prendre part à mon affliction.
\VS{15}Vous savez aussi, vous Philippiens, qu'au commencement de la prédication de l'Evangile, quand je partis de Macédoine, aucune église ne me communiqua rien en matière de donner et de recevoir, excepté vous seuls. 
\VS{16}Et même lorsque j'étais à Thessalonique, vous m'avez envoyé une fois, et même deux fois, ce dont j'avais besoin. 
\VS{17}Ce n'est pas que je recherche des présents, mais je cherche le fruit qui abonde pour votre compte.
\VS{18}J'ai tout reçu, et je suis dans l'abondance, et j'ai été comblé de biens en recevant d'Epaphrodite ce qui vient de vous, comme un parfum de bonne odeur, comme un sacrifice que Dieu accepte et qui lui est agréable.
\VS{19}Aussi mon Dieu pourvoira à tout ce dont vous aurez besoin selon ses richesses, avec gloire en Jésus-Christ. 
\TextTitle{Salutations}
\VS{20}Or à notre Dieu et Père soit la gloire aux siècles des siècles~! Amen~!
\VS{21}Saluez tous les saints en Jésus-Christ. Les frères qui sont avec moi vous saluent.
\VS{22}Tous les saints vous saluent, et principalement ceux qui sont de la maison de César.
\VS{23}Que la grâce de notre Seigneur Jésus-Christ soit avec vous tous~! Amen~!
\PPE{}
\end{multicols}

%\clearpage\ShortTitle{Col.}\BookTitle{Colossiens}\BFont
\noindent\hrulefill
{\footnotesize
\textit{
\bigskip
{\centering{}
\\Auteur~: Paul
\\Thème~: La prééminence de Christ
\\Date de rédaction~: Env. 60 ap. J.-C.\\}
}
\textit{
\\Située en Asie Mineure, Colosses était une ville de Phrygie qui se trouvait à environ deux cents kilomètres d'Ephèse.
\\Rédigée lors de la première captivité romaine de Paul, la lettre aux Colossiens a pour but de rétablir la suprématie de Christ. En effet, cette église - dont Epaphras, le probable fondateur, s'était converti à Ephèse au cours des trois années que Paul y passa - était sous l'influence d'enseignements séducteurs basés sur le gnosticisme. Cette philosophie à la fois attrayante et très dangereuse prônait entre autres le salut par la connaissance et le dualisme.\bigskip
}
}
\par\nobreak\noindent\hrulefill
\begin{multicols}{2}
\Chap{1}
\TextTitle{Introduction}
\VerseOne{}Paul, apôtre de Jésus-Christ, par la volonté de Dieu, et le frère Timothée~:
\VS{2}Aux saints et frères, fidèles en Christ, qui sont à Colosses, que la grâce et la paix vous soient données de la part de Dieu notre Père et de la part du Seigneur Jésus-Christ~!
\VS{3}Nous rendons grâces à Dieu, qui est le Père de notre Seigneur Jésus-Christ, et nous prions toujours pour vous,
\VS{4}ayant entendu parler de votre foi en Jésus-Christ, et de votre charité envers tous les saints,
\VS{5}à cause de l'espérance des biens qui vous sont réservés dans les cieux, et dont vous avez eu précédemment connaissance par la parole de la vérité, c'est-à-dire par l'Evangile,
\VS{6}qui est parvenu jusqu'à vous, comme il l'est aussi dans le monde entier. Et il porte des fruits, comme aussi parmi vous, depuis le jour où vous avez entendu et connu la grâce de Dieu dans la vérité,
\VS{7}ainsi que vous en avez aussi été instruits par Epaphras, notre cher compagnon de service, qui est pour vous un fidèle serviteur de Christ,
\VS{8}et qui nous a fait connaître votre charité par le Saint-Esprit.
\TextTitle{Prière de Paul pour les Colossiens}
\VS{9}C'est pourquoi depuis le jour où nous l'avons appris, nous ne cessons point de prier pour vous, et de demander à Dieu que vous soyez remplis de la connaissance de sa volonté, en toute sagesse et intelligence spirituelle,
\VS{10}afin que vous vous conduisiez d'une manière digne du Seigneur, pour lui plaire en toutes choses, portant des fruits en toutes sortes de bonnes œuvres, et croissant dans la connaissance de Dieu,
\VS{11}étant fortifiés en toute force, selon la puissance de sa gloire, pour toute patience, et constance, avec joie.
\TextTitle{Le salut de Dieu}
\VS{12}Rendant grâces au Père, qui nous a rendus capables d'avoir part à l'héritage des saints dans la lumière,
\VS{13}qui nous a délivrés de la puissance des ténèbres, et nous a transportés dans le Royaume du Fils de son amour,
\VS{14}en qui nous avons la rédemption par son sang, à savoir la rémission des péchés.
\VS{15}Lequel est l'image de Dieu invisible, le premier-né\FTNT{Dans les Ecritures, l'expression «~premier-né~» est appliquée au Seigneur pour exprimer trois réalités. Tout d'abord, on parle de Jésus en tant que premier-né de Marie, c'est-à-dire son fils aîné (Lu. 2:6-7). Ensuite, on trouve cette expression au sens figuré, pour marquer une distinction (par exemple concernant Israël~; Ex. 4:2) ou désigner la particularité et la suprématie d'une personne. Ainsi, bien que David était le dernier-né de son père Isaï (Ps. 89:28), Dieu en fit un premier-né, «~le plus élevé des rois de la terre~» (Ps. 89:28). Il en va de même pour Jésus-Christ. Il n'est pas le premier-né de la création dans le sens de rang de naissance ou de création, autrement Paul aurait employé le terme grec «~prôtoktisis~» qui signifie «~premier-créé~», au lieu de «~prôtotokos~», c'est-à-dire «~premier-né~». Il faut donc voir dans cette expression un titre de supériorité et d'hiérarchie, pour marquer sa prééminence. En effet, la Parole de Dieu déclare clairement que le Seigneur Jésus-Christ est l'Alpha et le Commencement de toutes choses (Ap. 1:8~; Ap. 21:6~; Ap. 22:13), le Créateur suprême (Ge. 1:1~; Ge. 2:7~; Es. 45:11-18~; Ps. 104:30~; Job. 33:4~; Jn. 1:3~; 1 Co. 8:6~; Col. 1:12-16~; Ap. 22:3~; Ap. 14:6). D'ailleurs il l'a lui-même affirmé sans ambigüité~: «~Avant qu'Abraham fût, Je suis~» (Jn. 8:58). Enfin, Jésus-Christ est aussi appelé le premier-né d'entre les morts (Col. 1:18). Cela ne signifie pas qu'il a été le premier à ressusciter, car il y a eu plusieurs résurrections avant la sienne, mais il fut le premier à ressusciter avec un corps glorieux. Sa résurrection est donc le gage de la promesse de la résurrection de tous ceux qui ont foi en lui (Jn. 3:16).} de toute la création.
\VS{16}Car par lui ont été créées toutes les choses qui sont dans les cieux et sur la terre, les visibles et les invisibles, soit les trônes, ou les dominations, ou les principautés, ou les puissances, toutes choses ont été créées par lui, et pour lui.
\VS{17}Et il est avant toutes choses, et toutes choses subsistent par lui.
\VS{18}Et c'est lui qui est le Chef du corps de l'Eglise, et qui est le commencement et le premier-né d'entre les morts, afin qu'il tienne le premier rang en toutes choses,
\VS{19}car le bon plaisir du Père a été que toute plénitude habitât en lui.
\VS{20}Et de réconcilier par lui toutes choses avec lui même, ayant fait la paix par le sang de sa croix, à savoir, tant les choses qui sont dans les cieux que celles qui sont sur la terre.
\VS{21}Et vous, qui étiez autrefois étrangers, et qui étiez ses ennemis dans votre entendement, et dans les mauvaises œuvres, il vous a maintenant réconciliés 
\VS{22}par le corps de sa chair, par sa mort, pour vous présenter saints, et sans tache, et irrépréhensibles devant lui.
\VS{23}Si toutefois vous demeurez dans la foi, étant fondés et fermes, et n'étant point transportés hors de l'espérance de l'Evangile que vous avez entendu, lequel est prêché à toute créature qui est sous le ciel, dont moi Paul, j'ai été fait le serviteur.
\VS{24}Je me réjouis donc maintenant dans mes souffrances pour vous~; et j'accomplis le reste des afflictions de Christ dans ma chair, pour son corps, qui est l'Eglise.
\VS{25}C'est d'elle que j'ai été fait le serviteur, selon la gestion que Dieu m'a donnée auprès de vous, afin que j'exécute pleinement la parole de Dieu,
\VS{26}à savoir le mystère qui avait été caché dans tous les siècles et dans tous les âges, mais qui est maintenant manifesté à ses saints~;
\VS{27}auxquels Dieu a voulu donner à connaître quelles sont les richesses de la gloire de ce mystère parmi les Gentils, c'est à savoir Christ, qui a été prêché parmi vous, et qui est l'espérance de la gloire~; 
\VS{28}lequel nous annonçons, en exhortant tout homme, et en enseignant tout homme en toute sagesse, afin que nous présentions tout homme parfait en Jésus-Christ.
\TextTitle{Le combat de Paul}
\VS{29}A quoi aussi je travaille, en combattant selon son efficacité\FTNTT{le terme «~efficacité~» vient du grec «~energeia~» qui signifie «~action~», «~fonctionnement~», «~compétence~», ou encore «~force à l'œuvre dans~». Ce mot est utilisé seulement pour parler du pouvoir surhumain que ce soit celui de Dieu ou celui du diable. (Ep. 1:19~; Ph. 3:21~; 2 Ti. 2:9).}, qui agit puissamment en moi.
\Chap{2}
\VerseOne{}Or je veux, en effet, que vous sachiez combien est grand le combat que j'ai pour vous, et pour ceux qui sont à Laodicée, et pour tous ceux qui n'ont pas vu mon visage dans la chair,
\VS{2}afin que leurs cœurs soient consolés, étant unis ensemble dans la charité, et enrichis d'une pleine intelligence, pour la connaissance du mystère de notre Dieu et Père, et de Christ,
\VS{3}en qui sont cachés tous les trésors de la sagesse et de la connaissance.
\TextTitle{Mise en garde contre les discours séduisants et la philosophie\FTNTT{1 Co. 2:4~; Ro. 16:17-18~; 2 Pi. 2:3.}}
\VS{4}Or je dis ceci afin que personne ne vous trompe par des discours séduisants.
\VS{5}Car, quoique je sois absent de corps, toutefois je suis avec vous en esprit, me réjouissant, et voyant votre ordre et la fermeté de votre foi, que vous avez en Christ.
\VS{6}Ainsi, comme vous avez reçu le Seigneur Jésus-Christ, marchez en lui,
\VS{7}étant enracinés et édifiés en lui, et fortifiés en la foi, selon que vous avez été enseignés, abondant en elle avec action de grâces.
\VS{8}Prenez garde que personne ne fasse de vous sa proie par la philosophie, et par de vaines tromperies conformes à la tradition des hommes et aux rudiments du monde, et non point à la doctrine de Christ.
\TextTitle{La divinité du Christ}
\VS{9}Car en lui habite corporellement toute la plénitude de la divinité\FTNT{En Jésus-Christ habite toute la plénitude de la divinité. Il est le Dieu Tout-Puissant.}.
\VS{10}Et vous êtes rendus accomplis en lui, qui est le Chef de toute principauté et puissance.
\TextTitle{L'oeuvre de la croix}
\VS{11}En qui aussi vous êtes circoncis d'une circoncision faite sans main, qui consiste à dépouiller le corps des péchés de la chair, ce qui est la circoncision de Christ.
\VS{12}Etant ensevelis avec lui par le baptême, en qui aussi vous êtes ensemble ressuscités par la foi de l'efficacité de Dieu, qui l'a ressuscité des morts.
\VS{13}Et lorsque vous étiez morts dans vos offenses, et dans l'incirconcision de votre chair, il vous a vivifiés ensemble avec lui, vous ayant gratuitement pardonné toutes vos offenses.
\VS{14}Il a effacé l'acte qui était contre nous, qui consistait en des ordonnances, et qui nous était contraire, et il l'a entièrement aboli en le clouant à la croix.
\VS{15}Il a dépouillé les principautés et les puissances, et les a exposées publiquement en spectacle, en triomphant d'elles par la croix.
\TextTitle{Mise en garde contre les commandements et les doctrines des hommes}
\VS{16}Que personne donc ne vous juge au sujet du manger ou du boire, ou au sujet d'un jour de fête, ou d'un jour de nouvelle lune, ou de sabbat,
\VS{17}qui sont l'ombre des choses qui devaient venir, mais le corps est en Christ.
\VS{18}Que personne ne vous enlève à son gré le prix de la course, sous l'apparence d'humilité d'esprit et par un culte des anges, s'ingérant dans les choses qu'il n'a pas vues, étant témérairement enflé par ses pensées charnelles,
\VS{19}sans s'attacher au Chef, dont tout le corps étant joint et ajusté ensemble par des jointures et des liens, s'accroît d'un accroissement de Dieu.
\VS{20}Si donc vous êtes morts avec Christ quant aux rudiments du monde, pourquoi vous impose-t-on ces ordonnances, comme si vous viviez dans le monde~?
\VS{21}A savoir: Ne prends pas~! Ne goûte pas~! Ne touche pas~!
\VS{22}Lesquelles sont toutes périssables par l'usage, et établies suivant les commandements et les doctrines des hommes~; 
\VS{23}et qui ont pourtant quelque apparence de sagesse en dévotion volontaire, et en humilité d'esprit, et en ce qu'elles n'épargnent pas le corps, et n'ont aucun égard à la satisfaction de la chair.
\Chap{3}
\TextTitle{Rechercher les choses d'en haut}
\VerseOne{}Si donc vous êtes ressuscités avec Christ, cherchez les choses qui sont en haut, où Christ est assis à la droite de Dieu.
\VS{2}Pensez aux choses d'en haut, et non à celles qui sont sur la terre.
\VS{3}Car vous êtes morts, et votre vie est cachée avec Christ en Dieu.
\VS{4}Quand Christ, qui est votre vie, apparaîtra, vous paraîtrez aussi alors avec lui dans la gloire.
\TextTitle{La mort à soi en pratique}
\VS{5}Mortifiez donc vos membres qui sont sur la terre~: La fornication, l'impureté, les passions, les mauvais désirs, et la cupidité, qui est une idolâtrie.
\VS{6}C'est à cause de ces choses que la colère de Dieu vient sur les fils de la rébellion,
\VS{7}parmi lesquels vous marchiez autrefois, quand vous viviez dans ces choses.
\VS{8}Mais maintenant, vous aussi, rejetez toutes ces choses~: La colère, l'animosité, la médisance, et les paroles déshonnêtes qui pourraient sortir de votre bouche.
\VS{9}Ne mentez point les uns aux autres, vous étant dépouillés du vieil homme et de ses œuvres,
\VS{10}et ayant revêtu le nouvel homme, qui se renouvelle dans la connaissance, selon l'image de celui qui l'a créé.
\VS{11}En qui il n'y a ni Grec ni Juif, ni circoncis ni incirconcis, ni barbare ni Scythe, ni esclave ni libre~; mais Christ y est tout et en tous.
\VS{12}Ainsi donc, comme des élus de Dieu, saints et bien-aimés, revêtez-vous des entrailles de miséricorde, de bonté, d'humilité, de douceur, de patience.
\VS{13}Vous supportant les uns les autres, et vous pardonnant les uns aux autres~; et si l'un a querelle contre l'autre, comme Christ vous a pardonné, vous aussi faites-en de même.
\VS{14}Mais par-dessus toutes ces choses, revêtez-vous de la charité, qui est le lien de la perfection.
\VS{15}Et que la paix de Dieu, à laquelle aussi vous êtes appelés pour être un seul corps, tienne le principal lieu dans vos cœurs. Et soyez reconnaissants.
\VS{16}Que la parole de Christ habite en vous abondamment en toute sagesse~; vous enseignant et vous exhortant l'un l'autre par des psaumes, et des hymnes et des cantiques spirituels, avec grâce, chantant de votre cœur au Seigneur.
\VS{17}Et quoi que vous fassiez, en parole ou en œuvre, faites tout au Nom du Seigneur Jésus, rendant grâces par lui à notre Dieu et Père.
\TextTitle{La famille selon Dieu}
\VS{18}Femmes, soyez soumises à vos maris, comme il convient dans le Seigneur\FTNT{Ep. 5:22.}.
\VS{19}Maris, aimez vos femmes, et ne vous aigrissez pas contre elles\FTNT{Ep. 5:25.}.
\VS{20}Enfants, obéissez à vos pères et à vos mères en toutes choses, car cela est agréable au Seigneur\FTNT{Ep. 6:1-2.}.
\VS{21}Pères, n'irritez pas vos enfants\FTNT{Ep. 6:4.}, afin qu'ils ne se découragent pas.
\TextTitle{Les rapports entre serviteurs et maîtres selon Dieu}
\VS{22}Serviteurs, obéissez en toutes choses à ceux qui sont vos maîtres selon la chair, ne servant point seulement sous leurs yeux, comme voulant complaire aux hommes, mais en simplicité de cœur, craignant Dieu\FTNT{Ep. 6:5-6.}.
\VS{23}Et quoi que vous fassiez, faites tout de bon cœur, comme le faisant pour le Seigneur, et non pas pour les hommes,
\VS{24}sachant que vous recevrez du Seigneur l'héritage pour récompense. Car vous servez Christ, le Seigneur.
\VS{25}Mais celui qui agit injustement recevra ce qu'il aura fait injustement, car en Dieu il n'y a point d'égard à l'apparence des personnes.
\Chap{4}
\VerseOne{}Maîtres, accordez à vos serviteurs ce qui est juste et équitable, sachant que vous avez, vous aussi, un Maître dans les cieux.
\TextTitle{La persévérance dans la prière}
\VS{2}Persévérez dans la prière, veillant dans cet exercice avec des actions de grâces.
\VS{3}Priez aussi tous ensemble pour nous, afin que Dieu nous ouvre la porte de la parole, pour annoncer le mystère de Christ pour lequel aussi je suis prisonnier, 
\VS{4}afin que je le fasse connaître comme je dois en parler.
\VS{5}Conduisez-vous sagement envers ceux du dehors, et rachetez le temps.
\VS{6}Que votre parole soit toujours assaisonnée de sel, avec grâce, afin que vous sachiez comment vous avez à répondre à chacun.
\TextTitle{Salutations}
\VS{7}Tychique, notre frère bien-aimé, et fidèle serviteur, et mon compagnon de service en notre Seigneur, vous fera savoir tout mon état.
\VS{8}Je l'envoie vers vous expressément, afin qu'il connaisse quel est votre état, et qu'il console vos cœurs~;
\VS{9}avec Onésime, notre fidèle et bien-aimé frère, qui est des vôtres. Ils vous feront connaitre toutes les choses d'ici.
\VS{10}Aristarque, qui est prisonnier avec moi, vous salue aussi, et Marc qui est le cousin de Barnabas, au sujet duquel vous avez reçu un ordre, s'il vient à vous, recevez-le.
\VS{11}Et Jésus, appelé Justus, vous salue aussi. Ils sont du nombre des circoncis, et les seuls qui travaillent avec moi pour le Royaume de Dieu, et qui ont été pour moi une consolation.
\VS{12}Epaphras, qui est des vôtres, et serviteur de Jésus-Christ, vous salue~; il ne cesse de combattre pour vous dans ses prières afin que vous demeuriez parfaits et accomplis en toute la volonté de Dieu.
\VS{13}Car je lui rends témoignage qu'il a un grand zèle pour vous, et pour ceux de Laodicée, et pour ceux d'Hiérapolis.
\VS{14}Luc, le médecin bien-aimé, vous salue, et Démas aussi.
\VS{15}Saluez les frères qui sont à Laodicée, et Nymphas, avec l'église qui est dans sa maison.
\VS{16}Et quand cette lettre aura été lue entre vous, faites en sorte qu'elle soit aussi lue dans l'église des Laodicéens, et que vous lisiez aussi celle qui viendra de Laodicée.
\VS{17}Et dites à Archippe~: Prends garde au service que tu as reçu dans le Seigneur afin de bien le remplir.
\VS{18}Je vous salue, moi Paul, de ma propre main. Souvenez-vous de mes liens. Que la grâce soit avec vous~! Amen~!
\PPE{}
\end{multicols}

%\clearpage\ShortTitle{Philémon}\BookTitle{Philémon}\BFont
\noindent\hrulefill
{\footnotesize
\textit{
\bigskip
{\centering{}
\\Auteur : Paul
\\(Gr. : Philemon)
\\Signification : Attentionné, qui embrasse
\\Thème : Un exemple d'amour
\\Date de rédaction : Env. 60 ap. J.-C.\\}
}
%\bigskip
\textit{
\\Paul écrivit cette lettre en prison, lors de sa deuxième captivité à Rome vers l'été 62, en même temps que l'épître aux Colossiens. Il s'adresse à Philémon, chrétien fortuné de Colosses ainsi qu'à sa femme Apphia, son fils Archippe et à l'église qui se réunissait dans leur maison. Paul demande à Philémon de pardonner à Onésime, son esclave, de s'être échappé d'auprès de lui. Il assure à Philémon que désormais une nouvelle relation le lierait à Onésime qui avait accepté Jésus-Christ dans sa vie. Il va même jusqu'à proposer de payer personnellement ce qu'Onésime lui devait tout en exprimant l'espoir que Philémon ferait plus que ce qu'il lui demande. Ainsi, Paul plaide pour Onésime comme Christ le fit en notre faveur.\bigskip
}
}
\par\nobreak\noindent\hrulefill
\begin{multicols}{2}
\Chap{1}
\TextTitle{Introduction}
\VerseOne{}Paul, prisonnier de Jésus-Christ, et le frère Timothée, à Philémon notre bien-aimé et compagnon d'œuvre ;
\VS{2}à Apphia, notre bien-aimée, à Archippe, notre compagnon de combat, et à l'église qui est dans ta maison.
\VS{3}Que la grâce et la paix vous soient données de la part de Dieu notre Père, et de la part du Seigneur Jésus-Christ.
\VS{4}Je rends grâces à mon Dieu, faisant toujours mention de toi dans mes prières ;
\VS{5}apprenant la foi que tu as au Seigneur Jésus, et ta charité envers tous les saints.
\VS{6}Afin que la communication de ta foi devienne efficace, en se faisant connaître par tout le bien qui est en vous, par Jésus-Christ.
\VS{7}Car, mon frère, nous avons une grande joie et une grande consolation de ta charité, en ce que tu as réjoui les entrailles des saints.
\TextTitle{Paul plaide en faveur d'Onésime}
\VS{8}C'est pourquoi, bien que j'aie une grande liberté en Christ de t'ordonner ce qui est convenable,
\VS{9}cependant je te prie plutôt par la charité, bien que je suis ce que je suis, à savoir Paul, un vieillard, et même maintenant prisonnier de Jésus-Christ ;
\VS{10}je te prie donc pour mon fils Onésime, que j'ai engendré dans mes liens ;
\VS{11}qui t'a été autrefois inutile, mais qui maintenant est bien utile à toi et à moi, et que je te renvoie.
\VS{12}Reçois-le donc comme mes propres entrailles.
\VS{13}Je voulais le retenir auprès de moi, afin qu'il me serve à ta place, dans les liens de l'Evangile.
\VS{14}Mais je n'ai rien voulu faire sans ton avis, afin que ce ne soit point comme par contrainte, mais volontairement, que tu me laisses un bien qui est à toi.
\VS{15}Car peut-être n'a-t-il été séparé de toi que pour un temps, afin que tu le recouvres\FTNT{Synonymes : retrouver, reconquérir, regagner.} pour toujours ;
\VS{16}non plus comme un esclave, mais comme étant au-dessus d'un esclave, à savoir comme un frère bien-aimé, principalement de moi ; et combien plus de toi, soit selon la chair, soit selon le Seigneur ?
\VS{17}Si donc tu me tiens pour ton compagnon, reçois-le comme moi-même.
\VS{18}Que s'il t'a fait quelque tort, ou s'il te doit quelque chose, mets-le sur mon compte.
\VS{19}Moi Paul, j'ai écrit ceci de ma propre main, je te le payerai ; pour ne pas te dire que tu te dois toi-même à moi.
\VS{20}Oui, mon frère, que je reçoive ce plaisir de toi en notre Seigneur ; réjouis mes entrailles en notre Seigneur.
\VS{21}Je t'ai écrit m'assurant de ton obéissance, et sachant que tu feras même plus que ce que je te dis.
\TextTitle{Conclusion}
\VS{22}Mais aussi, en même temps, prépare-moi un logement ; car j'espère que je vous serai rendu par vos prières.
\VS{23}Epaphras, qui est prisonnier avec moi en Jésus-Christ, te salue ;
\VS{24}Marc aussi, Aristarque, Démas, et Luc, mes compagnons d'œuvre.
\VS{25}Que la grâce de notre Seigneur Jésus-Christ soit avec votre esprit, Amen !
\PPE{}
\end{multicols}

%\clearpage\ShortTitle{1 Ti.}\BookTitle{1 Timothée}\BFont
\noindent\hrulefill
{\footnotesize
\textit{
\bigskip
{\centering{}
\\Auteur~: Paul
\\(Gr.~: Timotheos)
\\Signification~: Qui adore ou honore Dieu
\\Thème~: Comment se conduire dans l'église
\\Date de rédaction~: Env. 64 ap. J.-C.\\}
}
\textit{
\\Cette lettre s'adresse à Timothée dont le père était grec et la mère juive. Le jeune homme se convertit à Christ avec sa mère et sa grand-mère dès le premier voyage missionnaire de Paul au cours duquel il passa à Lystre.
\\Cette épître fut rédigée après la première captivité de Paul à Rome. Alors que les Eglises connaissaient une certaine expansion, Paul s'adresse à Timothée, jeune et fidèle compagnon d'œuvre qu'il a lui-même formé, sur des questions d'ordre disciplinaire et sur la pureté de la foi. Dans cette épître, dite pastorale, Paul donne des instructions précises à Timothée pour enseigner, exhorter, diriger le culte public et choisir ses collaborateurs.\bigskip
}
}
\par\nobreak\noindent\hrulefill
\begin{multicols}{2}
\Chap{1}
\TextTitle{Introduction}
\VerseOne{}Paul, apôtre de Jésus-Christ par l'ordre de Dieu, notre Sauveur, et du Seigneur Jésus-Christ, notre espérance,
\VS{2}à Timothée mon véritable fils dans la foi~: Que la grâce, la miséricorde et la paix te soient données de la part de Dieu notre Père, et de Jésus-Christ, notre Seigneur.
\TextTitle{Mise en garde contre les erreurs doctrinales~; le but de la loi} 
\VS{3}Suivant la prière que je te fis de demeurer à Ephèse, lorsque j'allais en Macédoine, je te prie encore d'ordonner à certaines personnes de ne pas enseigner une autre doctrine,
\VS{4}et de ne pas s'adonner aux fables et aux généalogies sans fin, qui produisent des disputes plutôt que l'édification en Dieu qui consiste dans la foi.
\VS{5}Or le but du commandement c'est la charité qui procède d'un cœur pur, d'une bonne conscience, et d'une foi sincère.
\VS{6}Quelques-uns, s'étant détournés de ces choses, se sont écartés dans de vains discours,
\VS{7}voulant être docteurs de la loi~; mais ils ne comprennent ni ce qu'ils disent ni ce qu'ils affirment.
\VS{8}Or nous savons que la loi est bonne pour celui qui en fait un usage légitime,
\VS{9}sachant ceci, que ce n'est pas pour le juste que la loi a été établie, mais pour les méchants et les rebelles, pour les impies et les pécheurs, pour les irréligieux et les profanes, pour les parricides, les meurtriers,
\VS{10}pour les fornicateurs, pour les homosexuels, pour les voleurs d'hommes, pour les menteurs, pour les parjures, et contre telle autre chose qui est contraire à la saine doctrine,
\VS{11}selon l'Evangile de la gloire du Dieu béni, Evangile qui m'a été confié.
\TextTitle{Témoignage de Paul}
\VS{12}Je rends grâces à celui qui m'a fortifié, c'est-à-dire à Jésus-Christ, notre Seigneur, de ce qu'il m'a estimé fidèle en m'établissant dans le service,
\VS{13}moi qui auparavant étais un blasphémateur, un persécuteur, et un homme violent~; mais j'ai obtenu miséricorde parce que j'agissais par ignorance, étant dans l'incrédulité.
\VS{14}Or la grâce de notre Seigneur a surabondé en moi, avec la foi et l'amour qui est en Jésus-Christ.
\VS{15}Cette parole est certaine et entièrement digne d'être reçue, que Jésus-Christ est venu dans le monde pour sauver les pécheurs, dont je suis le premier.
\VS{16}Mais j'ai obtenu miséricorde, afin que Jésus-Christ fasse voir en moi le premier, toute sa clémence, pour que je serve d'exemple à ceux qui croiraient en lui pour la vie éternelle.
\VS{17}Or au Roi des siècles, immortel, invisible, à Dieu seul sage, soient honneur et gloire aux siècles des siècles~! Amen~!
\TextTitle{Recommandations à Timothée}
\VS{18}Mon fils Timothée, je te recommande ce commandement que conformément aux prophéties qui auparavant ont été faites sur toi, tu t'acquittes, selon elles, du devoir de combattre dans cette bonne guerre,
\VS{19}en gardant la foi et une bonne conscience, laquelle quelques-uns ayant rejetée, ont fait naufrage quant à la foi.
\VS{20}De ce nombre sont Hyménée et Alexandre, que j'ai livrés à Satan, afin qu'ils apprennent par ce châtiment à ne plus blasphémer.
\Chap{2}
\TextTitle{Instructions sur la prière}
\VerseOne{}J'exhorte donc, avant toutes choses, à faire des requêtes, des prières, des supplications, et des actions de grâces pour tous les hommes,
\VS{2}pour les rois et pour tous ceux qui sont constitués en dignité, afin que nous menions une vie paisible et tranquille, en toute piété et honnêteté.
\VS{3}Car cela est bon et agréable devant Dieu, notre Sauveur,
\VS{4}qui veut que tous les hommes soient sauvés et qu'ils viennent à la connaissance de la vérité.
\VS{5}Car il y a un seul Dieu, et aussi un seul Médiateur entre Dieu et les hommes, à savoir Jésus-Christ homme,
\VS{6}qui s'est donné lui-même en rançon pour tous. C'est le témoignage qui a été rendu en son propre temps.
\VS{7}C'est dans cette vue que j'ai été établi prédicateur, apôtre (je dis la vérité en Christ, je ne mens point) et docteur des Gentils dans la foi et dans la vérité.
\VS{8}Je veux donc que les hommes prient en tout lieu, levant leurs mains pures, sans colère, et sans dispute.
\TextTitle{Le tenue de la femme}
\VS{9}Et de même, que les femmes, vêtues d'une manière décente, avec pudeur et modestie, ne se parent ni de tresses, ni d'or, ni de perles, ni d'habits somptueux,
\VS{10}mais qu'elles se parent de bonnes œuvres, comme il convient à des femmes qui font profession de servir Dieu.
\TextTitle{Le comportement de la femme envers son mari}
\VS{11}Que la femme apprenne dans le silence, en toute soumission.
\VS{12}Car je ne permets pas à la femme d'enseigner ni d'user d'autorité\FTNT{Le mot «~autorité~» vient du grec «~authenteo~» et signifie «~celui qui tue de ses propres mains un autre ou lui-même~; celui qui agit de sa propre autorité, autocrate~; un maître absolu~; gouverneur, exercer une domination~».} sur le mari~; mais elle doit demeurer dans le silence\FTNT{Le mot grec traduit par «~silence~» est «~hesuchia~» qui signifie «~en silence~; paisiblement~». La racine de ce terme est «~hesuchios~»~: tranquille, paisible.}.
\VS{13}Car Adam a été formé le premier, Eve ensuite.
\VS{14}Et ce n'est pas Adam qui a été séduit, mais la femme, ayant été séduite, a été la cause de la transgression.
\VS{15}Elle sera néanmoins sauvée en mettant des enfants au monde\FTNT{Il est évident que le salut ne dépend pas du fait d'enfanter puisque nous sommes sauvés par grâce et non par les œuvres. Ce verset fait référence à Eve, la mère de tous les vivants. Par elle le péché et la mort sont entrés dans le monde (Ro. 5:12) mais c'est aussi par sa postérité, à savoir Christ (Ge. 3:15), qu'elle, ainsi que tout le genre humain (hommes et femmes), sera sauvé.}, pourvu qu'elle persévère dans la foi, dans la charité, et dans la sanctification, avec modestie.
\Chap{3}
\TextTitle{Les évêques et les diacres doivent manifester le caractère de Christ}
\VerseOne{}Cette parole est certaine, si quelqu'un désire la charge d'évêque\FTNT{Evêque, du grec «~episkope~», signifie «~investigation, inspection, visite d'inspection~». C'est un acte par lequel Dieu visite les hommes, observe leurs voies, leurs caractères, pour leur accorder en partage joie ou tristesse. Ce terme signifie également surveillance, charge, contrôle, fonction, la fonction d'un ancien. Voir Ac. 1:20.}, il désire une œuvre excellente.
\VS{2}Mais il faut que l'évêque soit irrépréhensible, mari\FTNT{Paul ne dit pas que les évêques ne peuvent pas être célibataires. Il y a en effet une différence entre mari et marié. L'apôtre met l'accent sur la monogamie. Un homme célibataire peut en effet être évêque s'il remplit les caractéristiques décrites dans ce passage.} d'une seule femme, vigilant, modéré, honorable, hospitalier, propre à enseigner.
\VS{3}Il faut qu'il ne soit ni adonné au vin, ni violent, ni porté au gain déshonnête, mais modéré, éloigné des querelles, exempt d'avarice.
\VS{4}Il faut qu'il dirige honnêtement sa propre maison, et qu'il tienne ses enfants dans la soumission et dans une parfaite honnêteté~;
\VS{5}car si quelqu'un ne sait pas diriger sa propre maison, comment pourra-t-il gouverner l'église de Dieu~?
\VS{6}Il ne faut pas qu'il soit un nouveau converti, de peur qu'enflé d'orgueil, il ne tombe sous le jugement du diable.
\VS{7}Il faut aussi qu'il reçoive un bon témoignage de ceux du dehors, afin de ne pas tomber dans l'opprobre et dans les pièges du diable.
\VS{8}Que les diacres aussi soient honnêtes, éloignés de la duplicité, des excès du vin, d'un gain sordide,
\VS{9}conservant le mystère de la foi dans une conscience pure.
\VS{10}Que ceux-ci aussi soient premièrement éprouvés, et qu'ensuite ils servent, après avoir été trouvés sans reproche.
\VS{11}Leurs femmes, de même, doivent être honnêtes, non médisantes, sobres, fidèles en toutes choses.
\VS{12}Les diacres doivent être maris d'une seule femme, dirigeant honnêtement leurs enfants, et leurs propres maisons.
\VS{13}Car ceux qui auront bien servi s'acquièrent un rang honorable, et une grande liberté dans la foi qui est en Jésus-Christ.
\VS{14}Je t'écris ces choses espérant que j'irai bientôt vers toi~;
\VS{15}mais si je tarde, je t'écris ces choses afin que tu saches comment il faut se conduire dans la maison de Dieu, qui est l'Eglise du Dieu vivant, la colonne et l'appui de la vérité.
\VS{16}Et sans contredit, le mystère de la piété\FTNT{Le mystère de la piété. Il s'agit de la connaissance de Dieu manifestée en chair dans la personne de Jésus-Christ, 100\% homme et 100\% Dieu. C'est l'incarnation du Dieu Tout-Puissant dans le seul but de sauver les hommes et de produire dans leurs cœurs la véritable piété.} est grand~: Dieu a été manifesté en chair, justifié par l'Esprit, vu des anges, prêché aux Gentils, cru dans le monde, et élevé dans la gloire.
\Chap{4}
\TextTitle{L'apostasie et la séduction: Signes des derniers temps}
\VerseOne{}Mais l'Esprit dit expressément que dans les derniers temps, quelques-uns se détourneront de la foi pour s'attacher à des esprits séducteurs et à des doctrines de démons\FTNT{Il est indéniable que nous vivons les dernières minutes avant le retour glorieux de Jésus-Christ. Toutes les conditions sont pratiquement réunies pour que le Seigneur revienne, c'est pourquoi chaque enfant de Dieu doit se préparer à la rencontre avec l'Epoux. Les prophètes, notamment Paul, ont annoncé que la fin des temps serait caractérisée par la séduction et l'abandon de la foi de beaucoup de chrétiens.},
\VS{2}par l'hypocrisie de faux docteurs, ayant leur propre conscience marquée au fer rouge\FTNT{L'expression «~marqué au fer~» ou «~marque de la flétrissure~» se dit «~kauteriazo~» en grec et veut dire «~ceux dont l'âme est stigmatisée par les marques du péché~». Dans un sens médical, ce mot signifie «~cautériser~». Ce passage fait allusion à la marque de la bête qui sera imprimée dans la conscience des hommes~; voilà pourquoi Dieu nous demande de garder sa parole dans nos cœurs (Ps. 119:11). Les Juifs devaient avoir sur leurs mains et sur leurs fronts la marque de Dieu qui est sa parole (De. 6:6-8). La main se dit «~yad~» en hébreu, ce qui signifie «~pouvoir~», «~force~» ou encore «~autorité~»~; elle symbolise donc l'action. Le front se dit «~towphaphah~» en hébreu, ce qui signifie «~marque~»~; il s'agit de la pensée.}~;
\VS{3}défendant de se marier et commandant de s'abstenir des viandes que Dieu a créées afin que les fidèles, et ceux qui ont connu la vérité, en usent avec actions de grâces.
\VS{4}Car tout ce que Dieu a créé est bon, et rien ne doit être rejeté, pourvu qu'on le prenne avec actions de grâces,
\VS{5}parce que tout est sanctifié par la parole de Dieu et par la prière.
\TextTitle{S'exercer à la piété}
\VS{6}En exposant ces choses aux frères, tu seras un bon serviteur de Jésus-Christ, nourri des paroles de la foi et de la bonne doctrine que tu as exactement suivie.
\VS{7}Mais rejette les fables profanes, et semblables aux récits de vieilles femmes.
\VS{8}Exerce-toi à la piété~; car l'exercice corporel est utile à peu de chose, tandis que la piété est utile à toutes choses, ayant les promesses de la vie présente et de celle qui est à venir.
\VS{9}C'est là une parole certaine et digne d'être entièrement reçue.
\VS{10}Car c'est aussi à cause de cela que nous endurons des travaux et des opprobres, parce que nous espérons dans le Dieu vivant, qui est le Sauveur de tous les hommes, mais principalement des fidèles.
\VS{11}Déclare ces choses et enseigne-les.
\VS{12}Que personne ne méprise ta jeunesse~; mais sois le modèle pour les fidèles en paroles, en conduite, en charité, en esprit, en foi, en pureté.
\VS{13}Applique-toi à la lecture, à l'exhortation et à l'enseignement, jusqu'à ce que je vienne.
\VS{14}Ne néglige pas le don qui est en toi, et qui t'a été donné par prophétie, par l'imposition des mains de l'assemblée des anciens.
\VS{15}Pratique ces choses et donne-toi tout entier à elles, afin que tes progrès soient évidents pour tous.
\VS{16}Veille sur toi-même et sur la doctrine~; persévère dans ces choses, car en agissant ainsi, tu te sauveras toi-même et tu sauveras ceux qui t'écoutent.
\Chap{5}
\TextTitle{Recommandations concernant les veuves}
\VerseOne{}Ne reprends pas rudement le vieillard, mais exhorte-le comme un père~; les jeunes gens comme des frères,
\VS{2}les femmes âgées comme des mères, celles qui sont jeunes comme des sœurs, en toute pureté.
\VS{3}Honore les veuves qui sont véritablement veuves.
\VS{4}Mais si une veuve a des enfants, ou des petits enfants, qu'ils apprennent avant tout à exercer la piété envers leur propre famille, et à rendre à leurs parents ce qu'ils ont reçu d'eux~; car cela est bon et agréable à Dieu.
\VS{5}Or celle qui est véritablement veuve, et qui est laissée seule, espère en Dieu, et persévère nuit et jour dans les supplications et les prières.
\VS{6}Mais celle qui vit dans les plaisirs est morte quoique vivante.
\VS{7}Avertis-les donc de ces choses, afin qu'elles soient irrépréhensibles.
\VS{8}Que si quelqu'un n'a pas soin des siens, et principalement de ceux de sa famille, il a renié la foi, et il est pire qu'un infidèle.
\VS{9}Qu'une veuve, pour être enregistrée sur le rôle\FTNT{Inscription sur le rôle~: Expression qui s'apparente à l'enrôlement des soldats. Il est question des veuves ayant une place importante dans l'église, du fait qu'elles exercent une certaine responsabilité sur le reste des femmes, et ayant en charge les veuves et les orphelins pris en compte pour la dépense publique.}, n'ait pas moins de soixante ans, qu'elle ait été la femme d'un seul mari,
\VS{10}ayant le témoignage d'avoir fait de bonnes œuvres, comme d'avoir bien élevé ses propres enfants, d'avoir exercé l'hospitalité envers les étrangers, d'avoir lavé les pieds des saints, d'avoir secouru les affligés, et de s'être ainsi constamment appliquée à toutes sortes de bonnes œuvres.
\VS{11}Mais refuse les veuves qui sont plus jeunes~; car quand elles sont devenues lascives\FTNT{Ce mot vient du grec «~katastreniao~»~: «~ressentir les pulsions du désir sexuel~».} contre Christ, elles veulent se marier,
\VS{12}ayant leur condamnation, en ce qu'elles ont violé leur première foi.
\VS{13}Et avec cela aussi, étant oisives, elles apprennent à aller de maison en maison~; et non seulement elles sont oisives, mais encore causeuses, et curieuses, et parlant de choses qui ne sont pas bienséantes.
\VS{14}Je veux donc que les jeunes veuves se marient, qu'elles aient des enfants, qu'elles gouvernent leur ménage, et qu'elles ne donnent à l'adversaire aucune occasion de médire.
\VS{15}Car quelques-unes se sont déjà détournées pour suivre Satan.
\VS{16}Si quelque fidèle, homme ou quelque femme, a des veuves, qu'ils les assistent, et que l'église n'en soit point chargée, afin qu'elle puisse assister celles qui sont véritablement veuves.
\TextTitle{Recommandations concernant les anciens}
\VS{17}Que les anciens qui dirigent\FTNT{Du grec «~proistemi~»~: «~disposer~» ou «~placer devant~», «~diriger~», «~présider~» (1 Th. 5:12~; Ro. 12:8~; \vref{1 Ti. 3:4-5,12}.} convenablement soient jugés dignes d'un double honneur, spécialement ceux qui travaillent à la prédication et à l'enseignement.
\VS{18}Car l'Ecriture dit~: Tu n'emmuselleras point le bœuf quand il foule le grain\FTNT{De. 25:4.}. Et l'ouvrier mérite son salaire\FTNT{Lu. 10:7.}.
\VS{19}Ne reçois point d'accusation contre un ancien, si ce n'est sur la déposition de deux ou de trois témoins\FTNT{De. 19:15~; Mt. 18:16~; 2 Co. 13:1}.
\VS{20}Reprends publiquement ceux qui pèchent, afin que les autres aussi en aient de la crainte.
\VS{21}Je te conjure devant Dieu, et devant le Seigneur Jésus-Christ, et devant les anges élus, d'observer ces choses sans préférer l'un à l'autre, et de ne rien faire avec partialité.
\VS{22}N'impose les mains à personne avec précipitation, et ne participe pas aux péchés d'autrui~; toi-même, conserve-toi pur.
\VS{23}Ne bois plus uniquement de l'eau~; mais use d'un peu de vin, à cause de ton estomac et de tes fréquentes maladies.
\VS{24}Les péchés de certains hommes sont manifestes, même avant tout jugement, alors que chez d'autres, ils ne se découvrent qu'après.
\VS{25}De même, les bonnes œuvres sont manifestes, et celles qui ne les sont pas ne peuvent pas rester cachées\FTNT{Mt. 10:26~; Mc. 4:22~; Lu. 8:17~; Lu. 12:2.}.
\Chap{6}
\TextTitle{L'attitude du serviteur envers son maître}
\VerseOne{}Que tous les esclaves qui sont sous le joug sachent qu'ils doivent à leurs maîtres toute sorte d'honneur, afin qu'on ne blasphème pas le Nom de Dieu et sa doctrine.
\VS{2}Et que ceux qui ont des fidèles pour maîtres ne les méprisent point sous prétexte qu'ils sont leurs frères, mais qu'ils les servent d'autant mieux que ce sont des fidèles et des bien-aimés de Dieu, étant participants de la grâce. Enseigne ces choses et recommande-les.
\VS{3}Si quelqu'un enseigne des fausses doctrines, et ne se soumet pas aux saines paroles de notre Seigneur Jésus-Christ, et à la doctrine qui est selon la piété,
\VS{4}il est enflé d'orgueil, il ne sait rien~; mais il a la maladie des questions et des disputes de mots, d'où naissent l'envie, les querelles, les médisances et les mauvais soupçons,
\VS{5}les vaines disputes d'hommes corrompus d'entendement et privés de la vérité, qui estiment que la piété est un moyen de gagner. Sépare-toi de ces sortes de gens.
\TextTitle{L'amour de l'argent~: La racine de tous les maux}
\VS{6}Or la piété avec le contentement d'esprit est un grand gain.
\VS{7}Car nous n'avons rien apporté dans le monde, et aussi il est évident que nous n'en pouvons rien emporter.
\VS{8}Si nous avons la nourriture et le vêtement, cela nous suffira.
\VS{9}Mais ceux qui veulent devenir riches tombent dans la tentation\FTNT{La tentation se rapporte à l'envie de toujours posséder, de s'enrichir et de gagner plus d'argent. Cela finit par faire tomber les gens dans l'orgueil, le mensonge, la duplicité, dans la fornication, etc.}, dans le piège\FTNT{Le mot «~piège~» vient du grec «~pagis~» qui donne en français «~trappe~», «~filet~». «~Car il surprendra comme un filet tous ceux qui habitent sur la surface de toute la terre.~» Lu. 21:35. Ce mot suggère l'inattendu, l'improviste, la surprise, car les oiseaux et autres animaux pris dans le filet sont attrapés par surprise. Les conséquences de la cupidité sont nombreuses, notamment le mensonge et l'adultère. En effet, une personne cupide finit en général par tromper son conjoint.}, et dans beaucoup de désirs insensés et pernicieux\FTNT{Les désirs insensés et pernicieux sont multiples~: l'envie de toujours posséder plus que les autres, la convoitise, les rivalités, la concurrence, la folie des grandeurs. Ces choses sortent les gens de la vision du Seigneur (Mc. 4:19).} qui plongent les hommes dans la ruine et la perdition\FTNT{La ruine et la perdition. Une personne cupide se perd en s'éloignant du Seigneur (2 Pi. 2). Selon Salomon, l'argent ne rassasie personne. «~Celui qui aime l'argent n'est point rassasié par l'argent, et celui qui aime un grand train, n'en est pas nourri~; cela aussi est une vanité.~» (Ec. 5:9). Selon les Ecritures, le système bancaire mondial s'écroulera dans les prochaines années (Ap. 18).}.
\VS{10}Car l'amour de l'argent est la racine de tous les maux\FTNT{L'amour de l'argent est la racine de tous les maux. Ceux qui espèrent en une sécurité divine doivent renoncer à la sécurité matérielle et financière que la chair désire.}~; et quelques-uns en étant possédés, se sont détournés de la foi et se sont jetés eux-mêmes dans bien des tourments.
\VS{11}Mais toi, homme de Dieu~! Fuis ces choses, et recherche la justice, la piété, la foi, la charité, la patience, la douceur.
\VS{12}Combats le bon combat de la foi, saisis la vie éternelle, à laquelle aussi tu as été appelé, et pour laquelle tu as fait une belle confession en présence de plusieurs témoins.
\VS{13}Je t'ordonne, devant Dieu qui donne la vie à toutes choses, et devant Jésus-Christ qui a fait cette belle confession devant Ponce Pilate,
\VS{14}de garder ce commandement, en te conservant sans tache et irrépréhensible, jusqu'à l'apparition de notre Seigneur Jésus-Christ,
\VS{15}qui sera manifesté en son temps, qui est le Béni et seul Prince, le Roi des rois, et le Seigneur des seigneurs,
\VS{16}qui seul possède l'immortalité, et qui habite une lumière inaccessible, que nul homme n'a vu ni ne peut voir, à qui appartiennent l'honneur et la puissance éternelle. Amen~!
\VS{17}Ordonne à ceux qui sont riches dans ce monde, qu'ils ne soient pas hautains, et qu'ils ne mettent pas leur confiance dans l'incertitude des richesses, mais dans le Dieu vivant, qui nous donne toutes choses abondamment pour en jouir.
\VS{18}Qu'ils fassent du bien, qu'ils soient riches en bonnes œuvres, qu'ils soient prompts à donner, avec libéralité,
\VS{19}s'amassant ainsi pour l'avenir un trésor placé sur un fondement solide, afin qu'ils obtiennent la vie éternelle.
\TextTitle{Conclusion}
\VS{20}Timothée, garde le dépôt, en fuyant les discours vains et profanes, et les contradictions d'une science faussement ainsi nommée,
\VS{21}dont font profession quelques-uns qui se sont détournés de la foi. Que la grâce soit avec toi~! Amen~!
\PPE{}
\end{multicols}

%\clearpage\ShortTitle{Tit.}\BookTitle{Tite}\BFont
\noindent\hrulefill
{\footnotesize
\textit{
\bigskip
{\centering{}
\\Auteur : Paul
\\(Gr. : Titos)
\\Signifie : Nourrice, honorable
\\Thème : L'ordre dans les églises
\\Date de rédaction : Env. 65 ap. J.-C.\\}
}
%\bigskip
\textit{
\\Cette épître pastorale fut écrite après la libération de Paul de sa première captivité romaine, peut-être dans la ville de Philippes. Tite, disciple d'origine païenne et collaborateur de Paul, se trouvait alors en Crète où Paul l'avait laissé afin qu'il organise les églises. Dans cette lettre, l'apôtre traite des conditions requises pour assumer la charge d'ancien en mettant l'accent sur la saine doctrine. Mentionnant également les obligations morales des jeunes, des personnes âgées, ainsi que des serviteurs, il exhorte Tite à veiller et à s'éloigner des apostats.\bigskip
}
}
\par\nobreak\noindent\hrulefill
\begin{multicols}{2}
\Chap{1}
\TextTitle{Introduction}
\VerseOne{}Paul, serviteur de Dieu, et apôtre de Jésus-Christ, selon la foi des élus de Dieu et la connaissance de la vérité qui est selon la piété,
\VS{2}dans l'espérance de la vie éternelle, que Dieu, qui ne peut mentir, avait promise avant les temps éternels,
\VS{3}mais qu'il a manifestée en son propre temps par sa parole, dans la prédication qui m'a été confiée, par le commandement de Dieu notre Sauveur,
\VS{4}à Tite mon vrai fils, selon la foi qui nous est commune: Que la grâce, la miséricorde, et la paix te soient données de la part de Dieu notre Père, et de la part du Seigneur Jésus-Christ, notre Sauveur !
\TextTitle{Les caractéristiques d'un ancien}
\VS{5}La raison pour laquelle je t'ai laissé en Crète, c'est afin que tu achèves de mettre en bon ordre les choses qui restent à régler, et que tu établisses des anciens de ville en ville, suivant ce que je t'ai ordonné,
\VS{6}s'il s'y trouve un homme qui soit irrépréhensible, mari d'une seule femme, ayant des enfants fidèles, qui ne soient ni accusés de dissolution, ni rebelles.
\VS{7}Car il faut que l'évêque soit irrépréhensible, comme étant économe dans la maison de Dieu ; qu'il ne soit ni arrogant, ni coléreux, ni adonné au vin, ni violent, non convoiteux d'un gain déshonnête ;
\VS{8}mais hospitalier, aimant les gens de bien, sage, juste, saint, tempérant,
\VS{9}attaché à la parole de la vérité comme elle lui a été enseignée, afin qu'il soit capable tant d'exhorter par la saine doctrine, que de réfuter les contredisants\FTNT{Lu. 2:34 ; Jn. 19:12 ; Ac. 13:45 ; 28:19 ; 28:22 ; Ro. 10:21.}.
\VS{10}Car il y en a plusieurs qui ne veulent pas se soumettre, vains discoureurs, et séducteurs d'esprits, principalement ceux qui sont de la circoncision,
\VS{11}auxquels il faut fermer la bouche, et qui renversent les maisons tout entières enseignant pour un gain déshonnête des choses qu'on ne doit point enseigner.
\VS{12}Quelqu'un d'entre eux, qui était leur propre prophète, a dit : Les Crétois sont toujours menteurs, de mauvaises bêtes, des ventres paresseux.
\VS{13}Ce témoignage est véritable. C'est pourquoi reprends-les vivement, afin qu'ils soient sains dans la foi,
\VS{14}et qu'ils ne s'attachent point aux fables judaïques et aux commandements d'hommes qui se détournent de la vérité.
\VS{15}Toutes choses sont bien pures pour ceux qui sont purs, mais rien n'est pur pour les impurs et les infidèles ; mais leur entendement et leur conscience sont souillés.
\VS{16}Ils font profession de connaître Dieu, mais ils le renient par leurs œuvres, car ils sont abominables, et rebelles, et réprouvés pour toute bonne œuvre.
\Chap{2}
\TextTitle{Recommandations de Paul à Tite}
\VerseOne{}Mais toi, annonce les choses qui conviennent à la saine doctrine.
\VS{2}Que les vieillards soient sobres, honnêtes, prudents, sains dans la foi, dans la charité, et dans la patience.
\VS{3}De même, que les femmes âgées règlent leur extérieur d'une manière convenable à la sainteté ; qu'elles ne soient ni médisantes, ni sujettes à beaucoup de vin, mais qu'elles enseignent de bonnes choses,
\VS{4}afin qu'elles instruisent les jeunes femmes à être modestes, à aimer leurs maris, à aimer leurs enfants,
\VS{5}à être modérées, pures, occupées aux soins domestiques, bonnes, soumises à leurs maris, afin que la Parole de Dieu ne soit point blasphémée.
\VS{6}Exhorte aussi les jeunes hommes à être modérés,
\VS{7}te montrant toi-même un modèle de bonnes œuvres en toutes choses, en une doctrine exempte de toute altération, en pureté, en intégrité,
\VS{8}en paroles saines, que l'on ne puisse point condamner, afin que celui qui vous est contraire, soit rendu confus, n'ayant aucun mal à dire de vous.
\VS{9}Que les serviteurs soient soumis à leurs maîtres, leur complaisant en toutes choses, n'étant point contredisants,
\VS{10}ne dérobant rien de ce qui appartient à leurs maîtres, mais faisant toujours paraître une grande fidélité, afin de rendre honorable en toutes choses la doctrine de Dieu, notre Sauveur.
\VS{11}Car la grâce de Dieu, salutaire à tous les hommes, a été manifestée.
\VS{12}Et elle nous enseigne à renoncer à l'impiété et aux passions mondaines, et à vivre dans le présent siècle, selon la sagesse, la justice et la piété,
\VS{13}en attendant la bienheureuse espérance, et l'apparition de la gloire du grand Dieu et notre Sauveur Jésus-Christ,
\VS{14}qui s'est donné lui-même pour nous, afin de nous racheter de toute iniquité, et de nous purifier, pour lui être un peuple qui lui appartienne en propre, et qui soit zélé pour les bonnes œuvres.
\VS{15}Enseigne ces choses, exhorte, et reprends avec une pleine autorité. Et que personne ne te méprise.
\Chap{3}
\TextTitle{Conseils pratiques de Paul}
\VerseOne{}Rappelle-leur d'être soumis aux magistrats et aux autorités, d'obéir aux gouverneurs, d'être prêts à faire toutes sortes de bonnes actions,
\VS{2}de ne médire de personne, de n'être point querelleurs, mais doux, et montrant une parfaite douceur envers tous les hommes.
\VS{3}Car nous aussi, nous étions autrefois insensés, désobéissants, égarés, asservis à toute espèce de convoitises et de voluptés, vivant dans la méchanceté et dans l'envie, dignes d'être haïs, et nous haïssant les uns les autres.
\VS{4}Mais, quand la bonté de Dieu notre Sauveur et son amour envers les hommes ont été manifestés, il nous a sauvés,
\VS{5}non par des œuvres de justice que nous aurions faites, mais selon la miséricorde, par le bain de la régénération et le renouvellement du Saint-Esprit,
\VS{6}qu'il a répandu abondamment sur nous par Jésus-Christ notre Sauveur,
\VS{7}afin qu'ayant été justifiés par sa grâce, nous soyons les héritiers de la vie éternelle selon notre espérance.
\VS{8}Cette parole est certaine, et je veux que tu affirmes ces choses, afin que ceux qui ont cru en Dieu aient soin principalement de s'appliquer à pratiquer les bonnes œuvres. Voilà les choses qui sont bonnes et utiles aux hommes.
\VS{9}Mais évite les discussions folles, les généalogies, les querelles et les disputes de la loi ; car elles sont inutiles et vaines.
\VS{10}Rejette l'homme hérétique, après le premier et le second avertissement,
\VS{11}sachant qu'un tel homme est perverti, et qu'il pèche en se condamnant lui-même.
\TextTitle{Salutations}
\VS{12}Quand je t'enverrai Artémas ou Tychique, hâte-toi de venir vers moi à Nicopolis ; car j'ai résolu d'y passer l'hiver.
\VS{13}Accompagne soigneusement Zénas, docteur de la loi, et Apollos, afin que rien ne leur manque.
\VS{14}Que les nôtres aussi apprennent à être les premiers à s'appliquer aux bonnes œuvres, pour les usages nécessaires, afin qu'ils ne soient point sans fruits.
\VS{15}Tous ceux qui sont avec moi te saluent. Salue ceux qui nous aiment dans la foi. Grâce soit avec vous tous ! Amen !
\PPE{}
\end{multicols}

%\clearpage\ShortTitle{1 Pierre}\BookTitle{1 Pierre}\BFont
\noindent\hrulefill
{\footnotesize
\textit{
\bigskip
{\centering{}
\\Auteur : Pierre
\\(Gr. : Petro)
\\Signification : Roc, pierre
\\Thème : La victoire sur la souffrance
\\Date de rédaction : Env. 65 ap. J.-C.\\}
}
%\bigskip
\textit{
\\Cette lettre semble avoir été écrite à Rome même si Pierre y parlait de « Babylone ». En ces temps de persécutions, les chrétiens devaient être prudents quant à la manière dont ils parlaient du pouvoir en place, c'est pourquoi ils utilisaient souvent des codes. C'est donc durant une période difficile que fut rédigée cette épître qui s'adressait à des églises d'Asie Mineure dont la plupart furent fondées par Paul. Au travers de ces quelques lignes, Pierre exhorte les frères et sœurs à tenir ferme dans la foi malgré les souffrances liées aux épreuves, et les encourage à espérer en Jésus-Christ, leur salut. Il finit cette épître en donnant des conseils quant à l'attitude à avoir au sein de l'église.\bigskip
}
}
\par\nobreak\noindent\hrulefill
\begin{multicols}{2}
\Chap{1}
\TextTitle{Introduction}
\VerseOne{}Pierre, apôtre de Jésus-Christ, à ceux qui sont étrangers et dispersés dans le Pont\FTNT{Le Pont : province formant presque la totalité de l'Asie Mineure.}, la Galatie, la Cappadoce, l'Asie et la Bithynie,
\VS{2}élus selon la prescience de Dieu le Père, par la sanctification de l'Esprit, afin d'obéir à Jésus-Christ, et qu'ils participent à l'aspersion de son sang : Que la grâce et la paix vous soient multipliées !
\TextTitle{Les souffrances du chrétien et sa conduite à la lumière d'un salut parfait}
\VS{3}Béni soit Dieu, et le Père de notre Seigneur Jésus-Christ, qui, par sa grande miséricorde, nous a régénérés, pour une espérance vivante, par la résurrection de Jésus-Christ d'entre les morts,
\VS{4}pour un héritage incorruptible, et qui ne peut ni se souiller, ni se flétrir, qui est conservé dans les cieux pour nous,
\VS{5}qui sommes gardés par la puissance de Dieu, par la foi, afin que nous obtenions le salut, qui est prêt à être révélé dans les derniers temps !
\VS{6}En quoi vous vous réjouissez, quoique vous soyez maintenant affligés pour un peu de temps par diverses épreuves, vu que cela est convenable,
\VS{7}afin que l'épreuve de votre foi, beaucoup plus précieuse que l'or périssable, et qui toutefois est éprouvé par le feu, ait pour résultat la louange, l'honneur et la gloire, lorsque Jésus-Christ sera révélé.
\VS{8}Lequel vous aimez quoique vous ne l'ayez point vu, en qui vous croyez, quoique maintenant vous ne le voyiez pas, et vous vous réjouissez d'une joie ineffable et glorieuse,
\VS{9}remportant la fin de votre foi, savoir le salut de vos âmes.
\VS{10}C'est au sujet de ce salut que les prophètes, qui ont prophétisé touchant la grâce qui vous était destinée, ont fait leurs recherches et leurs investigations.
\VS{11}Ils voulaient sonder l'époque et les circonstances marquées par l'Esprit prophétique de Christ qui était en eux, et qui rendait à l'avance témoignage, leur faisant connaître les souffrances de Christ et la gloire dont elles seraient suivies.
\VS{12}Mais il leur fut révélé que ce n'était pas pour eux-mêmes, mais pour nous, qu'ils administraient ces choses, que vous ont annoncées maintenant ceux qui vous ont prêché l'Evangile par le Saint-Esprit envoyé du ciel, et dans lesquelles les anges désirent plonger leurs regards.
\VS{13}C'est pourquoi, ceignez les reins de votre entendement, soyez sobres, et ayez une entière espérance dans la grâce qui vous est présentée, jusqu'à ce que Jésus-Christ soit révélé\FTNT{Révélé, voir commentaire en 2 Th. 1 :7.}.
\VS{14}Comme des enfants obéissants, ne vous conformez pas à vos convoitises d'autrefois, pendant votre ignorance. 
\VS{15}Mais, comme celui qui vous a appelés est saint, vous aussi de même soyez saints dans toute votre conduite,
\VS{16}selon ce qu'il est écrit : Soyez saints, car je suis saint\FTNT{Lé. 11:44.}.
\VS{17}Et si vous invoquez comme votre Père celui qui juge selon l'œuvre de chacun, sans favoritisme, conduisez-vous avec crainte pendant le temps de votre séjour sur la terre,
\VS{18}sachant que vous avez été rachetés de votre vaine conduite, qui vous avait été enseignée par vos pères, non point par des choses corruptibles, comme par argent, ou par or,
\VS{19}mais par le sang précieux de Christ, comme d'un agneau sans défaut et sans tache,
\VS{20}prédestiné avant la fondation du monde, et manifesté dans les derniers temps, pour vous.
\VS{21}Par lui, vous croyez en Dieu, qui l'a ressuscité des morts et lui a donné la gloire, afin que votre foi et votre espérance reposent sur Dieu.
\VS{22}Ayant donc purifié vos âmes en obéissant à la vérité par le Saint-Esprit, afin que vous ayez un amour fraternel et sans hypocrisie, aimez-vous ardemment les uns les autres d'un cœur pur,
\VS{23}puisque vous avez été régénérés, non par une semence corruptible, mais par une semence incorruptible, par la parole de Dieu qui vit et demeure éternellement.
\VS{24}Car toute chair est comme l'herbe, et toute la gloire de l'homme comme la fleur de l'herbe. L'herbe sèche, et sa fleur tombe ;
\VS{25}mais la parole du Seigneur demeure éternellement\FTNT{Es. 40:6-8.}. Et cette parole est celle qui vous a été annoncée par l'Evangile.
\Chap{2}
\VerseOne{}Ayant donc renoncé à toute sorte de malice, et de toute fraude, et de dissimulation, et d'envie et de toutes médisances,
\VS{2}désirez ardemment, comme des enfants nouveau-nés, le lait spirituel et pur, afin que vous croissiez par lui,
\VS{3}si toutefois vous avez goûté combien le Seigneur est bon.
\VS{4}Et vous approchant de lui, pierre vivante, rejetée par les hommes, mais choisie et précieuse devant Dieu ;
\VS{5}et vous aussi, comme des pierres vivantes, vous êtes édifiés pour être une maison spirituelle, et une sainte sacrificature, afin d'offrir des sacrifices spirituels, agréables à Dieu par Jésus-Christ. 
\VS{6}C'est pourquoi aussi, il est dit dans l'Ecriture : Voici, je mets en Sion la principale pierre\FTNT{Jésus-Christ est la Pierre rejetée par les bâtisseurs. Voir Es. 28:16 ; Ps. 118:22.} de l'angle, choisie et précieuse ; et celui qui croit en elle ne sera point confus.
\VS{7}Elle est donc précieuse pour vous qui croyez. Mais, par rapport aux rebelles, il est dit : La pierre que ceux qui bâtissaient ont rejetée, est devenue la principale de l'angle, 
\VS{8}et une pierre d'achoppement, et un rocher de scandale ; ils se heurtent contre la parole, et sont rebelles et c'est à cela qu'ils sont destinés.
\TextTitle{La position du croyant}
\VS{9}Mais vous, vous êtes la race élue, vous êtes la sacrificature royale, la nation sainte, le peuple acquis, afin que vous annonciez les vertus de celui qui vous a appelés des ténèbres à sa merveilleuse lumière,
\VS{10}vous qui autrefois n'étiez pas son peuple, mais qui maintenant êtes le peuple de Dieu, vous qui n'aviez point obtenu miséricorde, mais qui maintenant avez obtenu miséricorde.
\VS{11}Mes bien-aimés, je vous exhorte, comme étrangers et voyageurs, à vous abstenir des convoitises charnelles qui font la guerre à l'âme.
\VS{12}Ayant une conduite honnête avec les Gentils, afin que, là même où ils vous calomnient comme si vous étiez des malfaiteurs, ils remarquent vos bonnes œuvres, et glorifient Dieu, au jour où il les visitera.
\VS{13}Soyez donc soumis à tout établissement humain, pour l'amour de Dieu : Soit au roi, comme à celui qui est au-dessus des autres,
\VS{14}soit aux gouverneurs, comme à ceux qui sont envoyés de sa part pour punir les méchants et pour honorer les gens de bien.
\VS{15}Car c'est là la volonté de Dieu, qu'en faisant bien vous fermiez la bouche à l'ignorance des hommes insensés.
\VS{16}Comme libres, et non pas comme ayant la liberté pour servir de voile à la méchanceté, mais agissant comme des serviteurs de Dieu.
\VS{17}Honorez tout le monde ; aimez tous vos frères ; craignez Dieu ; honorez le roi.
\VS{18}Serviteurs, soyez soumis en toute crainte à vos maîtres, non seulement à ceux qui sont bons et équitables, mais aussi à ceux qui sont méchants.
\VS{19}Car c'est une chose agréable à Dieu si quelqu'un à cause de la conscience qu'il a envers Dieu, endure des afflictions en souffrant injustement. 
\VS{20}Autrement, quelle gloire en aurez-vous, si lorsque vous péchez et qu’on vous frappe, vous le supportez  patiemment ? Mais si quand vous faites le bien et que vous souffrez, vous le supportez patiemment, voilà où Dieu prend plaisir. 
\TextTitle{Les souffrances de Christ, le Substitut des hommes}
\VS{21}Car vous êtes appelés à cela, vu même que Christ a souffert pour nous, nous laissant un modèle, afin que vous suiviez ses traces, 
\VS{22}lui qui n'a point commis de péché, et dans la bouche duquel il ne s'est point trouvé de fraude ;
\VS{23}qui, lorsqu'on lui disait des outrages, n'en rendait point, et quand on lui faisait du mal, n'usait point de menaces, mais il se remettait à celui qui juge justement ; 
\VS{24}lui qui a porté lui-même nos péchés en son corps sur le bois, afin qu'étant morts aux péchés nous vivions pour la justice ; lui par la meurtrissure\FTNT{Es. 53:5.} duquel même vous avez été guéris.
\VS{25}Car vous étiez comme des brebis errantes, mais maintenant vous êtes convertis au Pasteur et à l'Evêque de vos âmes. 
\Chap{3}
\TextTitle{La conduite chrétienne à la maison et à l'église}
\VerseOne{}Femmes, soyez de même soumises à vos maris, afin que, si quelques-uns n'obéissent point à la parole, ils soient gagnés sans paroles par la conduite de leurs femmes,
\VS{2}lorsqu'ils verront la pureté de votre conduite, accompagnée de crainte.
\VS{3}Et que votre ornement ne soit point celui de dehors, qui consiste dans la frisure des cheveux, et dans une parure d'or, et dans la magnificence des habits,
\VS{4}mais que votre parure consiste dans l’homme caché dans le cœur, c’est-à-dire dans l’incorruptibilité d’un esprit doux et paisible, qui est d’un grand prix devant Dieu.
\VS{5}Car c'est ainsi que se paraient aussi autrefois les saintes femmes qui espéraient en Dieu, étant soumises à leurs maris,
\VS{6}comme Sara, qui obéissait à Abraham et l'appelait son seigneur. C'est d'elle que vous êtes devenues les filles, en faisant ce qui est bien et sans vous laisser troubler par aucune crainte.
\VS{7}Et vous, maris, de même comportez-vous selon la sagesse avec vos femmes, comme un vase\FTNT{Paul utilise une métaphore connu des grecs pour parler du corps : le vase.} plus fragile, c'est-à-dire, féminin ; leur portant honneur comme étant aussi ensemble héritiers de la grâce de la vie, afin que vos prières ne soient pas interrompues. 
\VS{8}Enfin, soyez tous d'un même sentiment, remplis de compassion les uns envers les autres, d'amour fraternel, miséricordieux et doux.
\VS{9}Ne rendez point mal pour mal, ou injure pour injure\FTNT{Mt. 5:44.} ; mais, au contraire, bénissez ; sachant que c'est à cela que vous êtes appelés, afin d'hériter la bénédiction.
\VS{10}Car celui qui veut aimer sa vie et voir des jours heureux, qu'il préserve sa langue du mal, et ses lèvres de prononcer aucune fraude,
\VS{11}qu'il se détourne du mal, et fasse le bien, qu'il recherche la paix, et qu'il tâche de se la procurer ;
\VS{12}car les yeux du Seigneur sont sur les justes, et ses oreilles sont attentives à leurs prières, mais la face du Seigneur est contre ceux qui se conduisent mal.
\TextTitle{La conduite chrétienne aux yeux du monde}
\VS{13}Et qui vous maltraitera, si vous êtes les imitateurs de celui qui est bon ?
\VS{14}Que si toutefois vous souffrez quelque chose pour la justice, vous êtes bienheureux. Mais ne craignez point les maux dont ils veulent vous faire peur, et n'en soyez point troublés ;
\VS{15}mais sanctifiez le Seigneur dans vos coeurs, et soyez toujours prêts à répondre, avec douceur et avec respect, à chacun qui vous demande raison de l'espérance qui est en vous, 
\VS{16}et ayant une bonne conscience, afin que, ceux qui blâment votre bonne conduite en Christ, soient confus de ce qu'ils médisent de vous, comme si vous étiez des malfaiteurs.
\VS{17}Car il vaut mieux, si telle est la volonté de Dieu, que vous souffriez en faisant le bien qu’en faisant le mal.
\TextTitle{Les souffrances de Christ}
\VS{18}Car aussi Christ a souffert une fois pour les péchés, lui juste pour les injustes, afin de nous amener à Dieu, étant mort en la chair, mais vivifié par l'Esprit,
\VS{19}par lequel aussi étant allé, il a  prêché aux esprits qui sont en prison\FTNT{La possibilité du salut après la mort n’a aucun fondement biblique (Hé. 9 :27). Dans ce passage, il est fait mention des pécheurs qui ont vécu du temps de Noé et auxquels le Seigneur Jésus a confirmé la condamnation lorsqu’il est descendu dans l’Hadès ( l'enfer; Ep. 4 :9). Voir aussi commentaire en Mt 16 :18.},
\VS{20}et qui avaient été autrefois incrédules, quand la patience de Dieu les attendait, durant les jours de Noé, tandis que l'arche se préparait dans laquelle un petit nombre, à savoir huit personnes furent sauvées par l'eau.
\VS{21}A quoi aussi maintenant répond la figure qui nous sauve, c'est-à-dire, le baptême ; non point celui par lequel les ordures de la chair sont nettoyées, mais la promesse faite à Dieu d'une conscience pure, par la résurrection de Jésus-Christ,
\VS{22}qui est à la droite de Dieu, étant allé au Ciel, et auquel sont assujettis les anges, et les dominations et les puissances.
\Chap{4}
\TextTitle{Souffrir dans la chair}
\VerseOne{}Puisque Christ a souffert pour nous dans la chair, vous aussi armez-vous de la même pensée. Car celui qui a souffert dans la chair a cessé de pécher,
\VS{2}afin de vivre, non plus selon les convoitises des hommes, mais selon la volonté de Dieu, pendant le temps qui lui reste à vivre dans la chair.
\VS{3}Car il nous suffit d'avoir accompli la volonté des Gentils, pendant le temps de notre vie passée, quand nous nous abandonnions aux impudicités, aux convoitises, à l'ivrognerie, aux excès dans le manger et dans le boire, et aux idolâtries abominables.
\VS{4}Ce que ces Gentils trouvent fort étrange, ils vous calomnient de ce que vous ne courez pas avec eux dans un même débordement de dissolution. 
\VS{5}Mais ils rendront compte à celui qui est prêt à juger les vivants et les morts.
\VS{6}Car c'est aussi pour cela que les morts ont été évangélisés, afin qu'ils soient jugés selon les hommes dans la chair, et qu'ils vivent selon Dieu dans l'esprit. 
\TextTitle{La conduite chrétienne dans le temps présent}
\VS{7}Or la fin de toutes choses est proche : Soyez donc sobres et vigilants pour prier. 
\VS{8}Mais surtout, ayez les uns pour les autres une ardente charité, car la charité couvre une multitude de péchés.
\VS{9}Soyez hospitaliers les uns envers les autres, sans murmures. 
\VS{10}Que chacun selon le don qu'il a reçu, l'emploie pour le service des autres, comme de bons gestionnaires des diverses grâces de Dieu. 
\VS{11}Si quelqu'un parle, qu'il parle comme annonçant les paroles de Dieu ; si quelqu'un administre, qu'il administre comme par la puissance que Dieu lui en a fournie, afin qu'en toutes choses Dieu soit glorifié par Jésus-Christ, auquel appartient la gloire et la force, aux siècles des siècles. Amen ! 
\VS{12}Mes bien-aimés, ne trouvez point étrange quand vous êtes comme dans une fournaise pour votre épreuve, comme s'il vous arrivait quelque chose d'extraordinaire. 
\VS{13}Mais réjouissez-vous, de ce que vous participez aux souffrances de Christ, afin qu'aussi à la révélation de sa gloire, vous vous réjouissiez avec allégresse.
\VS{14}Si on vous dit des injures pour le Nom de Christ, vous êtes heureux, car l'Esprit de gloire et de Dieu repose sur vous, lequel est blasphémé par ceux qui vous noircissent mais pour vous, vous le glorifiez.
\VS{15}Que nul de vous, ne souffre comme meurtrier, ou voleur, ou malfaiteur, ou curieux des affaires d'autrui. 
\VS{16}Mais si quelqu'un souffre comme chrétien, qu'il n'en ait point de honte, mais qu'il glorifie Dieu en cela.
\VS{17}Car il est temps que le jugement commence par la maison de Dieu\FTNT{Le jugement commence par la maison de Dieu. Ez. 9:1-11.}. Or s'il commence premièrement par nous, quelle sera la fin de ceux qui n'obéissent pas à l'Evangile de Dieu ?
\VS{18}Et si le juste est difficilement sauvé, où comparaîtra le méchant et le pécheur ? 
\VS{19}Que ceux-là donc aussi, qui souffrent par la volonté de Dieu, puisqu'ils font ce qui est bon lui recommandent leurs âmes, comme au fidèle Créateur. 
\Chap{5}
\TextTitle{Servir sans rien attendre en retour}
\VerseOne{}Je prie les anciens qui sont parmi vous, moi qui suis ancien avec eux, et témoin des souffrances de Christ, et participant de la gloire qui doit être révélée et je leur dis : 
\VS{2}Paissez le troupeau de Dieu qui vous est commis, en prenant garde sur lui, non point par contrainte, mais volontairement ; non point pour un gain déshonnête, mais par un principe d'affection. 
\VS{3}Et non pas comme ayant domination sur les héritages du Seigneur, mais de telle manière que vous soyez les modèles du troupeau. 
\VS{4}Et quand le souverain Pasteur\FTNT{Jésus est notre Souverain Pasteur. Voir Ps. 23 ; Jn. 10.} apparaîtra, vous obtiendrez la couronne incorruptible de la gloire.
\VS{5}De même, vous jeunes gens, soyez soumis aux anciens. Et ayant tous de la soumission les uns pour les autres, soyez parés par-dedans d'humilité; parce que Dieu résiste aux orgueilleux, mais il fait grâce aux humbles. 
\VS{6}Humiliez-vous donc sous la puissante main de Dieu, afin qu'il vous élève quand le temps sera venu ;
\VS{7}remettez-lui tout ce qui peut vous inquiéter, car il prend soin de vous.
\VS{8}Soyez sobres et veillez : Car le diable, votre adversaire, tourne autour de vous comme un lion rugissant, cherchant qui il pourra dévorer. 
\VS{9}Résistez-lui donc en demeurant fermes dans la foi, sachant que les mêmes souffrances s'accomplissent dans la compagnie de vos frères qui sont dans le monde. 
\TextTitle{Salutations}
\VS{10}Or que le Dieu de toute grâce, qui nous a appelés à sa gloire éternelle en Jésus-Christ, après que vous aurez souffert un peu de temps, vous rende parfaits, vous affermisse, vous fortifie et vous établisse. 
\VS{11}A lui soient la gloire et la force, aux siècles des siècles ! Amen !
\VS{12}Je vous ai écrit brièvement par Silvain, notre frère, que je crois vous être fidèle, vous déclarant et vous protestant que la grâce de Dieu dans laquelle vous êtes est la véritable. 
\VS{13}L'Eglise qui est à Babylone, élue avec vous, et Marc, mon fils, vous saluent. 
\VS{14}Saluez-vous les uns les autres par un baiser de charité. Que la paix soit avec vous tous qui êtes en Jésus-Christ ! Amen !
\PPE{}
\end{multicols}

%\clearpage\ShortTitle{2 Pierre}\BookTitle{2 Pierre}\BFont
\begin{multicols}{2}
\TextTitle{[Introduction]}
\Chap{1}
\VerseOne{}Simon Pierre, serviteur et apôtre de Jésus-Christ, à vous qui avez reçu en partage une foi du même prix que la nôtre, par la justice de notre Dieu et Sauveur Jésus-Christ (1).
\VS{2}Que la grâce et la paix vous soient multipliées, par la connaissance de Dieu, et de notre Seigneur Jésus.
\TextTitle{[Les grandes vertus chrétiennes]}
\VS{3}Puisque sa divine puissance nous a donné tout ce qui appartient à la vie et à la piété, par la connaissance de celui qui nous a appelés par sa gloire et par sa vertu,
\VS{4}par lesquelles nous sont données les grandes et précieuses promesses, afin que par elles vous soyez faits participants de la nature divine, en fuyant la corruption qui règne dans le monde par la convoitise.
\VS{5}A cause de cela même, faites tous vos efforts pour ajouter la vertu à votre foi ; à la vertu, la connaissance,
\VS{6}à la connaissance, la tempérance, à la tempérance, la patience, à la patience, la piété,
\VS{7}à la piété, l'amour fraternel, et à l'amour fraternel, la charité.
\VS{8}Car si ces choses sont en vous, et y abondent, elles ne vous laisseront point oisifs ni stériles pour la connaissance de notre Seigneur Jésus-Christ.
\VS{9}Mais celui en qui ces choses ne se trouvent point est aveugle, et ne voit point de loin, ayant oublié la purification de ses anciens péchés.
\VS{10}C'est pourquoi, mes frères, efforcez-vous plutôt à affermir votre vocation, et votre élection ; car en faisant cela vous ne broncherez jamais.
\VS{11}Car par ce moyen, l'entrée au Royaume éternel de notre Seigneur et Sauveur Jésus-Christ vous sera abondamment accordée.
\TextTitle{[Sollicitude de l'Apôtre pour ses lecteurs ; autorité de son témoignage et de la parole des prophètes]}
\VS{12}C'est pourquoi je ne négligerai pas de vous rappeler sans cesse ces choses, quoique vous ayez de la connaissance, et que vous soyez fondés dans la vérité présente.
\VS{13}Car je crois qu'il est juste que je vous réveille par des avertissements, pendant que je suis dans cette tente (2),
\VS{14}sachant que dans peu de temps je dois la quitter, comme notre Seigneur Jésus-Christ lui-même me l'a déclaré.
\TextTitle{[Souvenir de la transfiguration]}
\VS{15}Mais j'aurai soin qu’après mon départ vous puissiez toujours vous souvenir de ces choses.
\VS{16}Car ce n’est pas en suivant des fables composées avec artifice, que nous vous avons fait connaître la puissance et l’avènement (3) de notre Seigneur Jésus-Christ, mais comme ayant vu sa majesté de nos propres yeux.
\VS{17}Car il reçut de Dieu le Père, honneur et gloire, lorsque cette voix lui fut adressée du milieu de la gloire magnifique : Celui-ci est mon Fils bien-aimé, en qui j'ai mis toute mon affection (4).
\VS{18}Et nous entendîmes cette voix envoyée du ciel, lorsque nous étions avec lui sur la sainte montagne.
\TextTitle{[Témoignage à la véracité des Ecritures prophétiques]}
\VS{19}Nous avons aussi la parole des prophètes qui est très ferme, à laquelle vous faites bien d'être attentifs, comme à une lampe qui brille dans un lieu obscur, jusqu'à ce que le jour vienne à paraître et que l'étoile du matin (5) se lève dans vos cœurs.
\VS{20}Sachant premièrement ceci, qu'aucune prophétie de l'Ecriture ne procède d’une interprétation particulière.
\VS{21}Car la prophétie n'a jamais été autrefois apportée par la volonté humaine, mais les saints hommes de Dieu étant poussés par le Saint-Esprit, ont parlé.
\TextTitle{[Avertissement contre les faux docteurs]}
\Chap{2}
\VerseOne{}Mais comme il y a eu de faux prophètes parmi le peuple, il y aura aussi parmi vous de faux docteurs, qui introduiront secrètement des sectes pernicieuses, et qui reniant le Seigneur qui les a rachetés, attireront sur eux-mêmes une ruine soudaine.
\VS{2}Et plusieurs suivront leurs sectes de perdition, et à cause d'eux, la voie de la vérité sera blasphémée.
\VS{3}Par cupidité, ils trafiqueront (1) de vous au moyen de paroles déguisées, mais la condamnation qui leur est destinée depuis longtemps ne tarde point, et leur perdition ne sommeille point.
\VS{4}Car si Dieu n'a pas épargné les anges qui ont péché, s’il les a précipités dans l'abîme (2), les a liés avec des chaînes d'obscurité, les a livrés pour y être gardés jusqu’au jugement ;
\VS{5}et s'il n'a point épargné l’ancien monde, mais a gardé Noé (3), lui huitième, qui était le prédicateur de la justice ; et a fait venir le déluge sur le monde des impies ;
\VS{6}et s'il a condamné à la destruction totale les villes de Sodome et de Gomorrhe, les réduisant en cendres, et les mettant pour être un exemple à ceux qui vivraient dans l'impiété ;
\VS{7}et s'il a délivré le juste Lot (4), qui cruellement affligé de la conduite de ces hommes sans frein, eut beaucoup à souffrir de ces abominables par leur infâme conduite ;
\VS{8}car cet homme juste, qui habitait au milieu d’eux, affligeait chaque jour son âme juste, à cause de ce qu’il voyait et entendait dire de leurs méchantes actions.
\VS{9}Le Seigneur sait ainsi délivrer de l’épreuve les hommes pieux, et réserver les injustes pour être punis au jour du jugement ;
\VS{10}principalement ceux qui vont après la chair, dans la passion de l'impureté, et qui méprisent l’autorité. Gens audacieux et arrogants, ils ne craignent point d’injurier les gloires ;
\VS{11}alors que les anges qui sont supérieurs en force et en puissance, ne prononcent point contre elles de jugement blasphématoire devant le Seigneur ;
\VS{12}Mais eux, semblables à des bêtes brutes, qui s’abandonnent à leurs penchants naturels, et qui sont nées pour être prises et détruites, ils parlent d’une manière blasphématoire de ce qu’ils ignorent, et ils périront par leur propre corruption.
\VS{13}Et ils recevront la récompense de leur iniquité. Ils aiment à être tous les jours dans les délices. Ce sont des taches et des souillures, et ils font leurs délices de leurs tromperies dans les repas qu'ils font avec vous.
\VS{14}Ils ont les yeux pleins d'adultère, ils ne cessent jamais de pécher, ils attirent les âmes mal affermies ; ils ont le cœur exercé à la cupidité, ce sont des enfants de malédiction,
\TextTitle{[Caractéristiques des faux docteurs
\\a. ils ressemblent à Balaam]}
\VS{15}qui ayant laissé le droit chemin, se sont égarés, et ont suivi la voie de Balaam (5), fils de Bosor, qui aima le salaire de l'iniquité ; mais il fut repris pour sa transgression.
\VS{16}Car une ânesse muette parlant d'une voix humaine, arrêta la folie du prophète.
\TextTitle{[b. ils sont dépourvus dIntroduction]}
\VS{17}Ce sont des fontaines sans eau, des nuées agitées par le tourbillon, et des gens à qui l'obscurité des ténèbres est réservée éternellement.
\TextTitle{[c. leurs discours sont savants et prétentieux]
\\cp. 1 Co. 2:1-5)}
\VS{18}Car en prononçant des discours fort enflés de vanité, ils amorcent par les convoitises de la chair, et par leurs impudicités, ceux qui s'étaient véritablement retirés de ceux qui vivent dans l’égarement ;
\TextTitle{[d. ils corrompent la liberté chrétienne]}
\VS{19}ils leur promettent la liberté, quand ils sont eux-mêmes esclaves de la corruption ; car chacun est esclave de ce qui a triomphé de lui.
\VS{20}En effet, si après s’être retirés des souillures du monde, par la connaissance du Seigneur et Sauveur Jésus-Christ, ils s’y engagent de nouveau et sont vaincus, leur dernière condition est pire que la première (6).
\VS{21}Car mieux valait pour eux n'avoir pas connu la voie de la justice, que de l'avoir connue et se détourner du saint commandement qui leur avait été donné.
\TextTitle{[e. ils retournent à leurs premiers péchés]}
\VS{22}Mais ce qu'on dit par un proverbe véritable leur est arrivé : Le chien est retourné à ce qu'il avait vomi ; et la truie lavée est retournée se vautrer dans le bourbier.
\TextTitle{[Le but de l'Epitre]}
\Chap{3}
\VerseOne{}Mes bien-aimés, c'est ici la seconde lettre que je vous écris, afin de réveiller, dans l'une et dans l'autre, par mes avertissements, les sentiments purs que vous avez.
\VS{2}Et afin que vous vous souveniez des paroles qui ont été dites auparavant par les saints prophètes, et du commandement que vous avez reçu de nous, qui sommes apôtres du Seigneur et Sauveur.
\VS{3}Sur toutes choses, sachez qu'aux derniers jours (1) il viendra des moqueurs, se conduisant selon leurs propres convoitises,
\TextTitle{[La seconde venue de Christ et le jour du Seigneur
\\a. l'incrédulité sera générale quant au retour de Christ]}
\VS{4}et disant : Où est la promesse de son avènement ? Car depuis que les pères sont morts, toutes choses demeurent comme elles ont été dès le commencement de la création.
\VS{5}Car ils ignorent volontairement ceci, c’est que les cieux furent autrefois créés par la parole de Dieu, et que la terre est sortie de l'eau, et qu'elle subsiste parmi l'eau ;
\VS{6}et que par ces choses-là, le monde d'alors périt, étant submergé par les eaux du déluge (2).
\VS{7}Mais les cieux et la terre d’à présent sont gardés par la même parole, étant réservés pour le feu au jour du jugement, et de la destruction des hommes impies.
\VS{8}Mais vous mes bien-aimés, n'ignorez pas ceci, qu'un jour est devant le Seigneur comme mille ans, et mille ans comme un jour (3).
\VS{9}Le Seigneur ne retarde point l'exécution de sa promesse, comme quelques-uns croient qu'il y ait du retard, mais il est patient envers nous, ne voulant qu'aucun ne périsse, mais que tous se repentent.
\TextTitle{[b) la purification des cieux et de la terre]}
\VS{10}Or le jour (4) du Seigneur viendra comme un voleur dans la nuit, et en ce jour-là, les cieux passeront avec le bruit d’une effroyable tempête, et les éléments seront dissous par l'ardeur du feu ; et la terre avec toutes les œuvres qu’elle renferme sera brûlée entièrement.
\VS{11}Puisque toutes ces choses doivent se dissoudre, quelles ne doivent pas être la sainteté de votre conduite et votre piété.
\VS{12}En attendant, et en hâtant par vos désirs la venue du jour de Dieu, par lequel les cieux étant enflammés seront dissous, et les éléments se fondront par l'ardeur du feu.
\VS{13}Mais nous attendons, selon sa promesse, de nouveaux cieux et une nouvelle terre (5), où la justice habitera.
\VS{14}C'est pourquoi, mes bien-aimés, en attendant ces choses, appliquez-vous à être trouvés par lui sans tache et sans reproche dans la paix.
\VS{15}Et croyez que la longue patience de notre Seigneur est la preuve qu'il veut votre salut ; comme Paul, notre frère bien-aimé, vous l’a aussi écrit selon la sagesse qui lui a été donnée ;
\VS{16}comme il le fait aussi dans toutes ses lettres, où il parle de ces points, dans lesquels il y a des choses difficiles à comprendre, dont les personnes ignorantes et mal affermies tordent le sens (6), comme celui des autres Ecritures, pour leur propre perdition.
\TextTitle{[Conclusion]}
\VS{17}Vous donc mes bien-aimés, puisque vous êtes déjà avertis, prenez garde qu'étant emportés avec les autres par la séduction des abominables, vous ne veniez à déchoir de votre fermeté.
\VS{18}Mais croissez dans la grâce et dans la connaissance de notre Seigneur et Sauveur Jésus-Christ. A lui soit la gloire maintenant, et jusqu'au jour d'éternité ! Amen !
\PPE{}
\end{multicols}

%\clearpage\ShortTitle{2 Timothée}\BookTitle{2 Timothée}\BFont
\noindent\hrulefill
{\footnotesize
\textit{
\bigskip
{\centering{}
\\Signifie : Qui adore ou honore Dieu
\\Thème : Le maintien de la vérité
\\Auteur : Paul
\\Date de rédaction : Env. 67\\}
}
%\bigskip
\textit{
\\Cette lettre s’adresse à Timothée dont le père était grec et la mère juive. Le jeune homme se convertit à Christ avec sa mère et sa grand-mère dès le premier voyage missionnaire de Paul au cours duquel il passa à Lystre.
%\bigskip
\\Paul écrit cette épître pastorale en prison à Rome, après avoir été arrêté dans une province orientale à Ephèse ou Troas.  Ses conditions de détentions étant plus rudes que la première fois, Paul restait dubitatif quant à sa mise en liberté. Il demanda donc à Timothée, son fils dans la foi et fidèle compagnon d’œuvre, de le rejoindre à Rome afin semble-t-il de recevoir ses dernières volontés. Après avoir exposé à Timothée les qualités et les devoirs d’un bon serviteur de l’évangile, il l’encouragea à lutter contre les faux docteurs et l’apostasie en prêchant la Parole en toutes circonstances.\bigskip
}
}
\par\nobreak\noindent\hrulefill
\begin{multicols}{2}
\TextTitle{[Introduction]}
\Chap{1}
\VerseOne{}Paul, apôtre de Jésus-Christ, par la volonté de Dieu, selon la promesse de la vie qui est en Jésus-Christ.
\VS{2}A Timothée, mon fils bien-aimé, que la grâce, la miséricorde et la paix te soient données de la part de Dieu le Père, et de la part de Jésus-Christ notre Seigneur.
\TextTitle{[Paul encourage Timothée]}
\VS{3}Je rends grâces à Dieu, que mes ancêtres ont servi et que je sers avec une conscience pure, faisant sans cesse mention de toi dans mes prières nuit et jour,
\VS{4}me souvenant de tes larmes, je désire fort te voir afin que je sois rempli de joie.
\VS{5}Et me souvenant de la foi sincère qui est en toi, et qui a premièrement habité en Loïs, ta grand-mère, et en Eunice, ta mère, et qui, je suis persuadé qu'elle habite aussi en toi.
\VS{6}C'est pourquoi je t'exhorte de ranimer le don de Dieu qui est en toi par l'imposition de mes mains.
\VS{7}Car Dieu ne nous a pas donné un esprit de timidité, mais de force, de charité\FTNT{Il est question ici de l’amour «~agape~», c’est-à-dire divin.} et de sagesse.
\VS{8}N’aie donc point honte du témoignage à rendre à notre Seigneur ni de moi, qui suis son prisonnier ; mais souffre avec moi les afflictions de l'Evangile, selon la puissance de Dieu,
\VS{9}qui nous a sauvés et qui nous a appelés par une sainte vocation, non selon nos œuvres, mais selon son propre dessein, et selon la grâce qui nous a été donnée en Jésus-Christ avant les temps éternels,
\VS{10}et qui maintenant a été manifestée par l'apparition de notre Sauveur Jésus-Christ, qui a détruit la mort et qui a mis en lumière la vie et l'immortalité par l'Evangile,
\VS{11}pour lequel j'ai été établi prédicateur, apôtre et docteur des Gentils.
\VS{12}C'est pourquoi aussi je souffre ces choses, mais je n'en ai point de honte ; car je connais celui en qui j'ai cru, et je suis persuadé qu'il est Puissant pour garder mon dépôt\FTNT{Dépôt : Il est question ici de la connaissance correcte et de la pure doctrine de l’Evangile qui doit être fermement et fidèlement gardée, et qui doit être consciencieusement délivrée aux autres.} jusqu'à ce jour-là.
\VS{13}Retiens dans la foi et dans la charité qui est en Jésus-Christ le modèle des saines paroles que tu as apprises de moi.
\VS{14}Garde le bon dépôt par le Saint-Esprit qui habite en nous.
\VS{15}Tu sais que tous ceux qui sont en Asie se sont éloignés de moi ; entre lesquels sont Phygelle et Hermogène.
\VS{16}Que le Seigneur accorde sa miséricorde à la maison d'Onésiphore, car souvent il m'a consolé, et il n'a point eu honte de mes chaînes.
\VS{17}Au contraire, quand il a été à Rome, il m'a cherché avec beaucoup d’empressement, et il m'a trouvé.
\VS{18}Que le Seigneur lui fasse trouver miséricorde envers le Seigneur en ce jour-là ; et tu sais mieux que personne combien il m'a rendu de services à Ephèse.
\TextTitle{[La conduite d'un disciple de Christ dans les jours d'apostasie]}
\Chap{2}
\VerseOne{}Toi donc, mon fils, sois fortifié dans la grâce qui est en Jésus-Christ.
\VS{2}Et les choses que tu as entendues de moi devant plusieurs témoins, confie-les à des personnes fidèles qui soient capables de les enseigner aussi à d'autres.
\VS{3}Toi donc, souffre avec moi comme un bon soldat de Jésus-Christ.
\VS{4}Il n’est pas de soldat qui s'embarrasse des affaires de cette vie s’il veut plaire à celui qui l'a enrôlé pour la guerre.
\VS{5}De même, l’athlète qui combat n'est point couronné s'il n'a pas combattu selon les règles.
\VS{6}Il faut aussi que le laboureur travaille premièrement, et ensuite il recueille les fruits.
\VS{7}Considère ce que je dis, car le Seigneur te donne de l’intelligence en toutes choses.
\VS{8}Souviens-toi que Jésus-Christ, qui est de la semence de David, est ressuscité des morts, selon mon Evangile,
\VS{9}pour lequel je souffre beaucoup de maux, jusqu'à être mis dans les chaînes comme un malfaiteur, mais cependant, la parole de Dieu n'est point liée.
\VS{10}C'est pourquoi je souffre tout pour l'amour des élus, afin qu'eux aussi obtiennent le salut qui est en Jésus-Christ, avec la gloire éternelle.
\VS{11}Cette parole est certaine, que si nous mourons avec lui, nous vivrons aussi avec lui.
\VS{12}Si nous souffrons avec lui, nous régnerons aussi avec lui. Si nous le renions, il nous reniera aussi\FTNT{Lu. 9:26.}.
\VS{13}Si nous sommes infidèles, il demeure fidèle, car il ne peut pas se renier lui-même.
\VS{14}Remets ces choses en mémoire, protestant devant Dieu qu'on ait pas de disputes de mots, qui est une chose dont il ne revient aucun profit, mais elle est la ruine des auditeurs.
\VS{15}Efforce-toi de te rendre approuvé\FTNT{Approuvé vient du grec ~dokimos~. Du temps de l’apôtre Paul, les systèmes bancaires actuels n’existaient pas, toute la monnaie était en métal. Pour obtenir les pièces de monnaie, le métal était fondu et versé dans des moules et après le démoulage, il était nécessaire d’enlever les bavures. Or de nombreuses personnes les grattaient pour récupérer le surplus de métal et même davantage, ce qui faussait le poids de la monnaie. Face à ce problème, de nombreuses lois furent promulguées à Athènes pour éradiquer la pratique du rognage des pièces en circulation. Il existait toutefois quelques changeurs intègres qui ne mettaient en circulation que des pièces au bon poids. On appelait ces personnes des ~dokimos~, ce qui signifie ~éprouvés~ ou ~approuvés~.} devant Dieu, comme un ouvrier sans reproche, enseignant purement la parole de la vérité.
\VS{16}Mais évite les discours vains et profanes ; car ceux qui les tiennent avanceront toujours plus dans l'impiété,
\VS{17}et leur parole rongera comme une gangrène. Et parmi ceux-là sont Hyménée et Philète,
\VS{18}qui se sont écartés\FTNT{Ecarter, dévier, s'écarter de, manquer le but. A l’époque des apôtres, il y avait plusieurs faux frères qui semaient la zizanie au milieu des enfants de Dieu. Parmi eux étaient Alexandre le forgeron (1 Ti. 1:18-20), Hyménée (1 Ti. 1:18-20), Philète (2 Ti. 2:16-18), les judaïsants (Ac. 15 ; Ga. 2), Diotrèphe (3 Jn.). Les faux frères sont des séducteurs.} de la vérité, en disant que la résurrection est déjà arrivée, et qui renversent la foi de quelques-uns.
\VS{19}Toutefois, le fondement de Dieu demeure ferme, ayant ce sceau : Le Seigneur connaît ceux qui lui appartiennent\FTNT{Le Seigneur connaît ses brebis. Voir No. 16:5 ; Jn. 10:14.} ; et : Quiconque invoque le nom du Seigneur, qu'il s’éloigne de l'iniquité.
\VS{20}Or dans une grande maison, il n'y a pas seulement des vases d'or et d'argent, mais il y en a aussi de bois et de terre. Les uns sont des vases d’honneur et les autres sont d’un usage vil.
\VS{21}Si quelqu'un donc se purifie de ces choses, il sera un vase d’honneur, sanctifié et utile au Seigneur, et préparé pour toute bonne œuvre.
\VS{22}Fuis aussi les désirs de la jeunesse, et recherche la justice, la foi, la charité, et la paix avec ceux qui invoquent le Seigneur d'un cœur pur.
\VS{23}Et rejette les questions\FTNT{Questions folles. Il est question ici de disputes, débats, discussions ou questions oiseuses.} folles, et qui sont sans instruction, sachant qu'elles ne font que produire des querelles.
\VS{24}Or, il ne faut pas que le serviteur du Seigneur soit querelleur, il doit au contraire avoir de la douceur envers tout le monde, propre à enseigner, supportant patiemment les mauvais,
\VS{25}enseignant avec douceur ceux qui ont un sentiment contraire, dans l'espérance qu'un jour Dieu leur donnera la repentance pour reconnaître la vérité,
\VS{26}et afin qu'ils se réveillent pour sortir des pièges du diable, par lesquels ils ont été pris pour faire sa volonté.
\TextTitle{[L'Ecriture, l'arme du chrétien face à l'apostasie]}
\Chap{3}
\VerseOne{}Or sache ceci, que dans les derniers jours\FTNT{Les derniers jours. Voir Ge. 49:1-2.} il surviendra des temps difficiles.
\VS{2}Car les hommes seront idolâtres d’eux-mêmes, amis de l’argent, fanfarons, orgueilleux, blasphémateurs, rebelles à leurs parents, ingrats, irréligieux,
\VS{3}sans affection naturelle, sans fidélité, calomniateurs, intempérants, cruels, haïssant les gens de bien,
\VS{4}traîtres, emportés, enflés d'orgueil, amis des voluptés plûtot qu'amis de Dieu\FTNT{(Le mot grec «~philotheos~» (amour de Dieu) du préfixe «~philos~» qui signifie «~amis, être lié d'amitié avec quelqu'un~» (Matt. 11:19 ; Lu. 7:6 ;Jn. 15:13-15 etc...) et de «~theos~» qui signifie «~Dieu~».}
\VS{5}ayant l'apparence\FTNT{L’apparence de la piété. Le mot ~apparnec~ vient du grec ~morphosis~ et du latin ~forma~ qui donnent ~forme~ en français. Il est question du formalisme, de l’attachement excessif aux règles, aux rites, aux coutumes et aux traditions. Dans l’église de Laodicée, l’accent est plutôt mis sur les règles à observer et les apparences que sur la vie spirituelle et intérieure. Les manifestations extérieures du formalisme sont : les lieux «sacrés» pour adorer (temples, cathédrales, pèlerinages, etc.) ; l’observation des jours sacrés (dimanche et sabbat) ; les rituels censés permettre au croyant d’expérimenter Dieu et de rentrer dans une vie bénie (circoncision, ordination, bénédiction nuptiale, paiement de la dîme, présentation des enfants à Dieu par le pasteur...) ; une manière spéciale de s’habiller (toge, soutane, collet clérical, kippa, voile, costume/cravate, un régime alimentaire spécial, etc.). Voir Mt. 6:1-8.} de la piété, mais en ayant renié la force. Eloigne-toi donc de telles gens.
\VS{6}Il en est parmi eux qui se glissent dans les maisons et qui tiennent captives les femmes chargées de péchés et agitées de diverses convoitises,
\VS{7}qui apprennent toujours, mais qui ne peuvent jamais parvenir à la pleine connaissance de la vérité.
\VS{8}Et comme Jannès et Jambrès ont résisté à Moïse, ceux-ci de même résistent à la vérité, étant des gens qui ont l'esprit corrompu, et qui sont réprouvés quant à la foi.
\VS{9}Mais ils ne feront pas de plus grands progrès, car leur folie sera manifestée à tous, comme le fut celle de ceux-là.
\VS{10}Mais pour toi, tu as pleinement compris ma doctrine, ma conduite, mon intention, ma foi, ma douceur, ma charité, ma persévérance.
\VS{11}Et tu sais les persécutions et les afflictions qui me sont arrivées à Antioche, à Iconie, et à Lystre. Quelles persécutions n’ai-je pas supportées ? Et comment le Seigneur m'a délivré de toutes.
\VS{12}Or tous ceux aussi qui veulent vivre pieusement en Jésus-Christ seront persécutés.
\VS{13}Mais les hommes méchants et imposteurs iront en empirant, séduisant les autres, et étant séduits.
\VS{14}Mais toi, demeure ferme dans les choses que tu as apprises et qui t'ont été confiées, sachant de qui tu les as apprises,
\VS{15}vu même que dès ton enfance tu as la connaissance des saintes lettres, qui peuvent te rendre sage pour le salut par la foi en Jésus-Christ.
\VS{16}Toute l'Ecriture est inspirée de Dieu, et utile pour enseigner, pour convaincre, pour corriger, et pour instruire selon la justice,
\VS{17}afin que l'homme de Dieu soit accompli et parfaitement instruit pour toute bonne œuvre.
\TextTitle{[Paul encourage solennellement Timothée à prêcher la parole]}
\Chap{4}
\VerseOne{}Je te somme devant Dieu, et devant le Seigneur Jésus-Christ, qui doit juger les vivants et les morts, lors de son apparition et de son règne.
\VS{2}Prêche la parole, insiste en toute occasion, favorable ou non. Reprends, censure, exhorte avec toute douceur d'esprit, et avec doctrine.
\VS{3}Car il viendra un temps où les hommes ne supporteront pas la saine doctrine, mais aimant qu'on leur chatouille les oreilles par des discours agréables, ils chercheront des docteurs qui répondent à leurs désirs\FTNT{Beaucoup refusent la saine doctrine et acceptent un évangile basé sur les biens matériels.}.
\VS{4}Et ils détourneront leurs oreilles de la vérité, et se tourneront vers les fables.
\VS{5}Mais toi, veille en toutes choses, souffre les afflictions, fais l'œuvre d'un évangéliste, rends ton ministère pleinement approuvé.
\VS{6}Car pour moi, je m'en vais maintenant servir de libation, et le temps de mon départ est proche.
\VS{7}J'ai combattu le bon combat, j'ai achevé la course, j'ai gardé la foi.
\VS{8}Au reste, la couronne de justice m'est réservée, et le Seigneur, juste Juge, me la rendra en ce jour-là, et non seulement à moi, mais aussi à tous ceux qui auront aimé son apparition.
\VS{9}Hâte-toi de venir bientôt vers moi.
\VS{10}Car Démas m'a abandonné, ayant aimé le présent siècle, et il s'en est allé à Thessalonique ; Crescens est allé en Galatie ; et Tite en Dalmatie.
\VS{11}Luc est seul avec moi ; prends Marc, et amène-le avec toi, car il m'est fort utile pour le ministère.
\VS{12}J'ai aussi envoyé Tychique à Ephèse.
\VS{13}Quand tu viendras, apporte avec toi le manteau que j'ai laissé à Troas, chez Carpus, et les livres aussi ; mais principalement mes parchemins.
\VS{14}Alexandre le forgeron m'a fait beaucoup de mal. Le Seigneur lui rendra selon ses œuvres.
\VS{15}Garde-toi donc de lui, car il s'est fortement opposé à nos paroles.
\VS{16}Personne ne m'a assisté dans ma première défense, mais tous m'ont abandonné ; toutefois que cela ne leur soit point imputé !
\VS{17}Mais le Seigneur m'a assisté et fortifié, afin que ma prédication soit pleinement approuvée, et que tous les Gentils l’entendent ; et j'ai été délivré de la gueule du lion.
\VS{18}Le Seigneur aussi me délivrera de toute mauvaise œuvre, et me sauvera dans son Royaume céleste. A lui soit la gloire aux siècles des siècles. Amen !
\TextTitle{[Conclusion]}
\VS{19}Salue Priscille et Aquilas, et la famille d'Onésiphore.
\VS{20}Eraste est resté à Corinthe, et j'ai laissé Trophime malade à Milet.
\VS{21}Hâte-toi de venir avant l'hiver. Eubulus et Pudens, et Linus, et Claudia, et tous les frères te saluent.
\VS{22}Que le Seigneur Jésus-Christ soit avec ton esprit. Que la grâce soit avec vous. Amen !
\PPE{}
\end{multicols}

%\clearpage\ShortTitle{Jude}\BookTitle{Jude}\BFont
\begin{multicols}{2}
\TextTitle{[Introduction]}
\Chap{1}
\VerseOne{}Jude serviteur de Jésus-Christ, et frère de Jacques, à ceux qui ont été appelés par l'Evangile, que Dieu a sanctifiés et gardés pour Jésus-Christ :
\VS{2}Que la miséricorde, la paix et l'amour vous soient multipliés.
\TextTitle{[Mise en garde contre l'apostasie]}
\VS{3}Mes bien-aimés, comme je désirais vous écrire avec empressement au sujet de notre salut commun, j’ai jugé nécessaire de le faire pour vous exhorter à combattre pour la foi qui a été transmise aux saints une fois pour toutes.
\VS{4}Car il s’est glissé parmi vous, certains hommes dont la condamnation est écrite depuis longtemps, des impies qui changent la grâce de notre Dieu en dissolution, et qui renient le seul Dominateur Jésus-Christ, notre Dieu et Seigneur.
\TextTitle{[Exemples historiques d'incrédulité et de révolte]}
\VS{5}Je veux vous rappeler une chose que vous savez déjà : C'est que le Seigneur après avoir délivré le peuple du pays d'Egypte, fit ensuite périr les incrédules,
\VS{6}qu’il a réservés pour le jugement du grand jour, enchainés éternellement par les ténèbres, les anges qui n'ont pas gardé leur origine, mais qui ont abandonné leur propre demeure ;
\VS{7}que Sodome et Gomorrhe, et les villes voisines qui s'étaient abandonnées comme eux à l'impureté et à des vices contre nature, sont données en exemples, subissant la peine d’un feu éternel.
\TextTitle{[Description des faux docteurs]}
\VS{8}Malgré cela, ces hommes aussi, plongés dans leurs rêveries, souillent leur chair, méprisent l’autorité, et blasphèment contre les dignités.
\VS{9}Or, l'archange Michel, lorsqu’il contestait avec le diable et lui disputait le corps de Moïse, n'osa pas prononcer contre lui un jugement blasphématoire, mais il dit seulement : Que le Seigneur te réprime !
\VS{10}Eux, au contraire, ils blasphèment contre tout ce qu'ils ignorent, et ils se corrompent dans tout ce qu'ils savent naturellement, comme font les bêtes brutes.
\VS{11}Malheur à eux ! Car ils ont suivi la voie de Caïn, et ils se sont jetés dans l’égarement de Balaam, pour l’amour du gain, ils se sont perdus par la rébellion de Koré (1).
\VS{12}Ce sont des écueils dans vos agapes, lorsqu’ils prennent leurs repas avec vous sans aucune retenue, et se repaissant eux-mêmes ; ce sont des nuées sans eau, emportées par des vents çà et là ; des arbres d’automne dont le fruit se pourrit, et sans fruits, deux fois morts, et déracinés ;
\VS{13}des vagues impétueuses de la mer, jetant l'écume de leurs impuretés ; des étoiles errantes, à qui l'obscurité des ténèbres est réservée éternellement.
\VS{14}C’est aussi pour eux qu’Hénoc, le septième homme après Adam, a prophétisé en disant :
\VS{15}Voici, le Seigneur est venu avec ses saintes myriades, pour exercer un jugement contre tous les hommes, et pour convaincre tous les impies parmi eux de tous les actes d'impiété qu’ils ont commis et de toutes les paroles blasphématoires qu’ont proférées contre lui des pécheurs impies.
\VS{16}Ce sont des gens qui murmurent, qui se plaignent toujours, qui marchent selon leurs convoitises, qui ont à la bouche des discours hautains, qui admirent les personnes pour le profit qui leur en revient.
\VS{17}Mais vous, mes bien-aimés, souvenez-vous des choses qui ont été prédites par les apôtres de notre Seigneur Jésus-Christ.
\VS{18}Ils vous disaient que dans les derniers temps il y aurait des moqueurs, qui marcheraient selon leurs convoitises impies.
\VS{19}Ce sont ceux qui provoquent des divisions, des gens sensuels, n'ayant pas l'Esprit.
\TextTitle{[Exhortation aux chrétiens]}
\VS{20}Mais vous, mes bien-aimés, vous édifiant vous-mêmes sur votre très sainte foi, et priant par le Saint-Esprit,
\VS{21}maintenez-vous les uns les autres dans l'amour de Dieu, en attendant la miséricorde de notre Seigneur Jésus-Christ, pour obtenir la vie éternelle.
\VS{22}Et ayez pitié des uns en usant de discernement ;
\VS{23}sauvez-en d’autres avec crainte, en les arrachant hors du feu, haïssant jusqu’à la tunique souillée par la chair.
\TextTitle{[Conclusion]}
\VS{24}Or, à celui qui est puissant pour vous préserver de toute chute et vous faire paraître devant sa gloire irréprochables et dans l’allégresse,
\VS{25}à Dieu, seul sage, notre Sauveur, par Jésus-Christ notre Seigneur, soient gloire et magnificence, force et puissance, dès maintenant et dans tous les siècles, Amen !
\PPE{}
\end{multicols}

%\clearpage\ShortTitle{Hé.}\BookTitle{Hébreux}\BFont
\noindent\hrulefill
{\footnotesize
\textit{
\bigskip
{\centering{}
\\Auteur~: Inconnu
\\Thème~: La prêtrise du Messie
\\Date de rédaction~: Env. 68 ap. J.-C.\\}
}
\textit{
\\Cette épître fut rédigée avant la destruction de Jérusalem, car le temple y subsistait encore. Elle s'adressait à des juifs convertis connaissant bien l'auteur. Parmi eux, certains étaient tentés de retourner au judaïsme à cause des persécutions. L'auteur désire affermir ces chrétiens en leur montrant que l'objectif de la loi avait été réalisé par Christ qui est supérieur aux anges, aux prophètes et à Moïse. Il leur montre combien son œuvre rédemptrice est parfaite et les invite à suivre le Seigneur avec une foi indéfectible en persévérant dans l'amour fraternel.\bigskip
}
}
\par\nobreak\noindent\hrulefill
\begin{multicols}{2}
\Chap{1}
\TextTitle{Dieu parle par le Fils}
\VerseOne{}Dieu ayant anciennement parlé à nos pères par les prophètes, à plusieurs reprises et de plusieurs manières,
\VS{2}nous a parlé dans ces derniers jours\FTNT{Les derniers jours ont commencé avec la naissance de l'Eglise. Voir Joë. 2:28~; Ac. 2:14-17.} par son Fils, qu'il a établi héritier de toutes choses, et par lequel il a aussi créé l'univers~;
\VS{3}et qui étant la splendeur de sa gloire, et l'empreinte de sa substance, et soutenant toutes choses par sa parole puissante, ayant fait par lui-même la purification de nos péchés, s'est assis à la droite de la Majesté divine dans les lieux très hauts.
\TextTitle{Le Fils, supérieur aux anges}
\VS{4}Etant devenu d'autant supérieur aux anges, il a hérité d'un nom plus excellent que le leur.
\VS{5}Car auquel des anges a-t-il jamais dit~: Tu es mon Fils, je t'ai engendré aujourd'hui\FTNT{Ps. 2:7.}~? Et encore~: Je serai pour lui un Père, et il sera pour moi un Fils\FTNT{2 S. 7:14.}~?
\VS{6}Et quand il introduit de nouveau dans le monde son Fils premier-né\FTNT{Voir commentaire en Col. 1:15.}, il est dit~: Et que tous les anges de Dieu l'adorent\FTNT{Ps. 97:7.}~!
\VS{7}Car quant aux anges, il est dit~: Il fait de ses anges des vents, et de ses serviteurs des flammes de feu\FTNT{Ps. 104:4.}.
\VS{8}Mais à l'égard du Fils, il dit~: Ô Dieu, ton trône demeure aux siècles des siècles~; et le sceptre de ton Royaume est un sceptre d'équité~;
\VS{9}tu as aimé la justice, et tu as haï l'iniquité~; c'est pourquoi, ô Dieu, ton Dieu t'a oint d'une huile de joie par-dessus tous tes semblables\FTNT{Ps. 45:7-8.}~!
\VS{10}Et dans un autre endroit~: Toi, Seigneur, tu as fondé la terre dès le commencement, et les cieux sont les ouvrages de tes mains~;
\VS{11}ils périront, mais tu es permanent~; et ils vieilliront tous comme un vêtement,
\VS{12}et tu les rouleras comme un manteau et ils seront changés~; mais toi, tu restes le même, et tes années ne finiront point\FTNT{Es. 50:9~; Es. 51:6~; Ps. 102:27-28.}.
\VS{13}Et auquel des anges a-t-il jamais dit~: Assieds-toi à ma droite, jusqu'à ce que j'aie mis tes ennemis pour le marchepied de tes pieds\FTNT{Ps. 110:1.}~?
\VS{14}Ne sont-ils pas tous des esprits administrateurs, envoyés pour servir en faveur de ceux qui doivent recevoir l'héritage du salut~?
\Chap{2}
\TextTitle{Ne pas négliger le salut}
\VerseOne{}C'est pourquoi il nous faut prendre garde de plus près aux choses que nous avons entendues, de peur que nous les laissions s'échapper.
\VS{2}Car, si la parole prononcée par les anges a été ferme, et si toute transgression et toute désobéissance a reçu une juste rétribution,
\VS{3}comment échapperons-nous, si nous négligeons un si grand salut, qui, ayant été premièrement annoncé par le Seigneur, nous a été confirmé par ceux qui l'avaient entendu~?
\VS{4}Dieu confirmant aussi leur témoignage par des prodiges, et des miracles, et par plusieurs autres différents effets de sa puissance, et par les dons du Saint-Esprit, selon sa volonté.
\TextTitle{Toutes choses doivent être soumises à Christ}
\VS{5}Car, ce n'est pas aux anges qu'il a soumis le monde à venir dont nous parlons.
\VS{6}Et quelqu'un a rendu ce témoignage en quelque autre endroit, disant~: Qu'est-ce que l'homme, pour que tu te souviennes de lui, ou le fils de l'homme, pour que tu le visites~?
\VS{7}Tu l'as fait un peu moindre que les anges, tu l'as couronné de gloire et d'honneur, et l'as établi sur les œuvres de tes mains.
\VS{8}Tu as assujetti toutes choses sous ses pieds\FTNT{Ps. 8:5-7.}. En effet, en lui assujettissant toutes choses, il n'a rien laissé qui ne lui soit assujetti. Mais, nous ne voyons pourtant pas encore que toutes choses lui soient assujetties.
\TextTitle{Jésus abaissé un peu de temps pour sauver l'homme}
\VS{9}Mais celui qui a été fait un peu moindre que les anges, Jésus, nous le voyons couronné de gloire et d'honneur par la passion de sa mort, afin que par la grâce de Dieu, il souffrît la mort pour tous.
\VS{10}Car il était convenable, que celui pour qui sont toutes choses et par qui sont toutes choses, puisqu'il a amené plusieurs enfants à la gloire, consacre le Prince de leur salut par les afflictions.
\VS{11}Car, et celui qui sanctifie et ceux qui sont sanctifiés descendent tous d'un même père. C'est pourquoi il n'a pas honte de les appeler ses frères,
\VS{12}disant~: J'annoncerai ton Nom à mes frères, et je te louerai au milieu de l'assemblée\FTNT{Ps. 22:23.}.
\VS{13}Et encore~: Je me confierai en lui. Et encore~: Me voici, moi et les enfants que Dieu m'a donnés\FTNT{Es. 8:17-18.}.
\VS{14}Ainsi donc, puisque les enfants participent à la chair et au sang, lui aussi de même a participé aux mêmes choses, afin que, par la mort, il rende impuissant celui qui avait le pouvoir de la mort, c'est-à-dire le diable,
\VS{15}et qu'il délivre tous ceux qui, par crainte de la mort, étaient assujettis toute leur vie à la servitude.
\VS{16}Car, certes, il n'a nullement secouru les anges, mais il a secouru la postérité d'Abraham.
\VS{17}C'est pourquoi il a fallu qu'il soit semblable en toutes choses à ses frères, afin qu'il soit un Grand-Prêtre miséricordieux et fidèle dans les choses qui doivent être faites envers Dieu, pour faire la propitiation pour les péchés du peuple~;
\VS{18}car, parce qu'il a souffert lui-même, étant tenté, il est puissant pour secourir ceux qui sont tentés.
\Chap{3}
\TextTitle{Christ, supérieur à Moïse}
\VerseOne{}C'est pourquoi, mes frères saints, qui avez part à la vocation céleste, considérez attentivement Jésus-Christ, l'Apôtre et le Grand-Prêtre de notre profession,
\VS{2}qui a été fidèle à celui qui l'a établi, comme le fut Moïse dans toute sa maison.
\VS{3}Car Jésus-Christ a été jugé digne d'une gloire d'autant supérieure à celle de Moïse, que celui qui a construit une maison, a plus d'honneur que la maison même.
\VS{4}Car chaque maison est construite par quelqu'un, mais celui qui a construit toutes choses, c'est Dieu.
\VS{5}Et quant à Moïse, il a été fidèle dans toute sa maison, comme serviteur, pour témoigner des choses qui devaient être dites~;
\VS{6}mais Christ l'est comme Fils sur sa maison~; et nous sommes sa maison\FTNT{L'Eglise véritable est la maison de Dieu. Voir Es. 66:1~; 1 Co. 3:16~; 1 Co. 6:19~; Ep. 2:21-22. Les bâtiments ne sont pas la maison de Dieu. Le premier bâtiment d'église avait été édifié par des fidèles sous le règne d'Alexandre Sévère en 222-235. L'Eglise véritable est composée de pierres vivantes qui ont pour fondement le Roc (Jésus), parce qu'elle est bâtie par Jésus-Christ lui-même et qu'elle est sa propriété~; les démons ne peuvent pas la détruire. L'Eglise véritable ne peut donc être confondue avec un bâtiment ou une maison physique.}, pourvu que nous retenions fermement jusqu'à la fin l'assurance et la gloire de l'espérance.
\TextTitle{Résultat de l'incrédulité de la génération qui sortit d'Egypte}
\VS{7}C'est pourquoi, comme dit le Saint-Esprit~: Aujourd'hui, si vous entendez sa voix,
\VS{8}n'endurcissez point vos cœurs, comme il arriva dans le lieu de la rébellion, au jour de la tentation dans le désert,
\VS{9}où vos pères me tentèrent et m'éprouvèrent, et ils virent mes œuvres pendant quarante ans\FTNT{Ps. 95:8-11.}.
\VS{10}C'est pourquoi je fus irrité contre cette génération, et je dis~: Leur cœur s'égare toujours. Et ils n'ont pas connu mes voies.
\VS{11}Aussi, je jurai dans ma colère~: Ils n'entreront pas dans mon repos~!
\VS{12}Mes frères, prenez garde que quelqu'un de vous n'ait un cœur mauvais et incrédule, au point de se révolter contre le Dieu vivant,
\VS{13}mais exhortez-vous les uns les autres chaque jour, aussi longtemps qu'on peut dire~: Aujoud'hui~! De peur que quelqu'un d'entre vous ne s'endurcisse par la séduction du péché.
\VS{14}Car nous sommes devenus participants de Christ, pourvu que nous gardions ferme jusqu'à la fin notre première assurance,
\VS{15}pendant qu'il est dit~: Aujourd'hui, si vous entendez sa voix, n'endurcissez pas vos cœurs, comme il arriva dans le lieu de la rébellion.
\VS{16}Car, quelques-uns l'ayant entendue, le provoquèrent à la colère~; mais ce ne furent pas tous ceux qui étaient sortis d'Egypte par Moïse. 
\VS{17}Et contre qui Dieu fut-il irrité pendant quarante ans~? Ne fut-ce pas contre ceux qui péchèrent, et dont les cadavres tombèrent dans le désert~?
\VS{18}Et à qui jura-t-il qu'ils n'entreraient point dans son repos, sinon à ceux qui furent rebelles~?
\VS{19}Aussi, nous voyons qu'ils ne purent y entrer à cause de leur incrédulité.
\Chap{4}
\TextTitle{Le repos}
\VerseOne{}Craignons donc, que quelqu'un d'entre vous, venant à négliger la promesse d'entrer dans son repos, ne s'en trouve privé.
\VS{2}Car il nous a été évangélisé, aussi bien qu'à eux~; mais la parole qu'ils entendirent ne leur servit de rien, parce qu'elle n'était pas mêlée avec la foi dans ceux qui l'entendirent.
\VS{3}Pour nous qui avons cru, nous entrons dans le repos, suivant ce qui a été dit~: C'est pourquoi je jurai dans ma colère, ils n'entreront pas dans mon repos\FTNT{Hé. 3:11.}~! Il dit cela, quoique ses œuvres aient été achevées depuis la fondation du monde.
\VS{4}Car il a parlé quelque part ainsi du septième jour~: Et Dieu se reposa de toutes ses œuvres le septième jour\FTNT{Ge. 2:2.}.
\VS{5}Et encore dans ce passage~: Ils n'entreront pas dans mon repos~!
\VS{6}Puisqu'il reste donc à quelques-uns d'y entrer, et que ceux à qui d'abord il a été évangélisé n'y sont pas entrés à cause de leur désobéissance,
\VS{7}Dieu détermine de nouveau un certain jour, qu'il appelle aujourd'hui, en disant par David si longtemps après, selon ce qui a été dit~: Aujourd'hui, si vous entendez sa voix, n'endurcissez point vos cœurs\FTNT{Ps. 95:8-11.}.
\VS{8}Car, si Josué les avait introduits dans le repos, jamais après cela il n'aurait parlé d'un autre jour.
\TextTitle{Entrer dans le repos de Dieu}
\VS{9}Il reste donc encore un repos réservé au peuple de Dieu.
\VS{10}Car celui qui est entré dans son repos, se repose aussi de ses œuvres, comme Dieu s'est reposé des siennes.
\VS{11}Efforçons-nous donc d'entrer dans ce repos-là, de peur que quelqu'un ne tombe en imitant une semblable désobéissance.
\VS{12}Car la Parole de Dieu est vivante et efficace, et plus pénétrante qu'une épée quelconque à deux tranchants, et atteignant jusqu'à la division de l'âme et de l'esprit, et des jointures et des mœlles~; et elle juge les pensées et les intentions du cœur.
\VS{13}Et il n'y a aucune créature qui soit cachée devant lui, mais toutes choses sont nues et entièrement découvertes aux yeux de celui devant lequel nous devons rendre compte.
\VS{14}Ainsi, puisque nous avons un Souverain Grand-Prêtre, Jésus, le Fils de Dieu, qui a traversé les cieux, tenons ferme notre profession.
\VS{15}Car nous n'avons pas un Grand-Prêtre qui ne puisse avoir compassion de nos infirmités~; mais, nous avons celui qui a été tenté comme nous en toutes choses, mais sans pécher.
\VS{16}Approchons donc avec assurance du trône de la grâce, afin d'obtenir miséricorde et de trouver grâce, pour être secourus dans le temps convenable.
\Chap{5}
\TextTitle{Le service du grand-prêtre}
\VerseOne{}Or tout grand-prêtre pris d'entre les hommes est établi pour les hommes dans les choses qui concernent Dieu, afin qu'il offre des dons et des sacrifices pour les péchés.
\VS{2}Etant capable d'avoir de l'indulgence pour les ignorants et les égarés, puisqu'il est aussi lui-même enveloppé d'infirmité.
\VS{3}Et à cause de cette infirmité, il doit offrir pour les péchés, non seulement pour le peuple, mais aussi pour lui-même.
\VS{4}Et nul ne s'attribue cet honneur, si ce n'est celui qui est appelé de Dieu, comme Aaron.
\TextTitle{Christ, Grand-Prêtre selon l'ordre de Melchisédek}
\VS{5}De même, aussi Christ ne s'est point glorifié lui-même d'être fait Grand-Prêtre, mais celui qui lui a dit~: C'est toi qui es mon Fils, je t'ai engendré aujourd'hui\FTNT{Ps. 2:7.}~!
\VS{6}Comme il dit encore ailleurs~: Tu es prêtre éternellement, selon l'ordre de Melchisédek\FTNT{Ps. 110:4.}.
\VS{7}C'est lui qui, pendant les jours de sa chair, a offert avec de grands cris et avec larmes des prières et des supplications à celui qui pouvait le sauver de la mort, et il a été exaucé à cause de sa piété.
\VS{8}Quoiqu'il soit le Fils de Dieu, il a pourtant appris l'obéissance par les choses qu'il a souffertes.
\VS{9}Après avoir été consacré, il est devenu l'auteur du salut éternel pour tous ceux qui lui obéissent,
\VS{10}étant appelé de Dieu à être Grand-Prêtre selon l'ordre de Melchisédek~;
\VS{11}de qui nous avons beaucoup de choses à dire, mais elles sont difficiles à expliquer, parce que vous êtes devenus lents à comprendre.
\TextTitle{Du lait à la nourriture solide\FTNTT{jusqu'à Hé. 6:12}}
\VS{12}En effet, tandis que vous devriez être maîtres depuis longtemps, vous avez encore besoin qu'on vous enseigne quels sont les premiers rudiments des oracles de Dieu, et vous êtes devenus tels, que vous avez encore besoin de lait et non d'une nourriture solide.
\VS{13}Or quiconque use de lait, ne sait point ce que c'est que la parole de la justice, parce qu'il est un enfant\FTNT{Le mot enfant dans ce passage vient du grec «~nepios~» qui signifie «~ignorant~».}.
\VS{14}Mais la viande solide est pour ceux qui sont déjà hommes faits, {c'est-à-dire}, pour ceux qui, pour y être habitués, ont les sens exercés à discerner le bien et le mal.
\Chap{6}
\TextTitle{Tendre à la perfection}
\VerseOne{}C'est pourquoi, laissant la parole qui n'enseigne que les premiers principes de Christ, tendons à la perfection, ne posant pas de nouveau le fondement de la repentance des œuvres mortes, et de la foi en Dieu,
\VS{2}de la doctrine des baptêmes, et de l'imposition des mains, et de la résurrection des morts, et du jugement éternel.
\VS{3}Et c'est ce que nous ferons, si Dieu le permet.
\VS{4}Or il est impossible que ceux qui ont été une fois illuminés, et qui ont goûté le don céleste, et qui ont été fait participants au Saint-Esprit,
\VS{5}qui ont goûté la bonne parole de Dieu, et les puissances du siècle à venir,
\VS{6}s'ils retombent, soient changés de nouveau par la repentance, vu que, quant à eux, ils crucifient de nouveau le Fils de Dieu, et l'exposent à l'opprobre.
\VS{7}Car la terre qui est abreuvée par la pluie qui tombe souvent sur elle, et qui produit des herbes propres à ceux par qui elle est labourée, reçoit la bénédiction de Dieu~;
\VS{8}mais, celle qui produit des épines et des chardons, est rejetée et proche de malédiction, et sa fin est d'être brûlée.
\VS{9}Mais nous sommes persuadés, quoique nous parlions ainsi, en ce qui vous concerne, mes bien-aimés, des choses meilleures et qui tiennent au salut.
\VS{10}Car Dieu n'est pas injuste, pour oublier votre œuvre, et le travail de la charité que vous avez témoigné pour son Nom, en ce que vous avez secouru les saints, et que vous les secourez encore.
\VS{11}Or nous souhaitons que chacun de vous montre jusqu'à la fin le même empressement pour la pleine certitude de l'espérance,
\VS{12}afin que vous ne vous relâchiez point, mais que vous imitiez ceux qui, par la foi et par la patience, héritent ce qui leur a été promis.
\TextTitle{Christ entré au-delà du voile}
\VS{13}Car, lorsque Dieu fit la promesse à Abraham, ne pouvant jurer par un plus grand, il jura par lui-même,
\VS{14}en disant~: Certainement, je te bénirai abondement et je te multiplierai merveilleusement\FTNT{Ge. 22:16-17.}.
\VS{15}Et ainsi, Abraham ayant attendu patiemment, obtint ce qui lui avait été promis.
\VS{16}Or les hommes jurent par celui qui est plus grand qu'eux, et le serment qu'ils font pour confirmer leur parole met fin à tous leurs différends.
\VS{17}C'est pourquoi Dieu, voulant faire mieux connaître aux héritiers de la promesse la fermeté immuable de sa résolution, il y a fait intervenir le serment,
\VS{18}afin que, par deux choses immuables, dans lesquelles il est impossible que Dieu mente, nous ayons une ferme consolation, nous qui avons notre refuge à obtenir l'espérance qui nous est proposée.
\VS{19}Laquelle nous tenons comme une ancre sûre et ferme de l'âme, et qui pénètre jusqu'au-delà du voile,
\VS{20}où Jésus est entré comme notre précurseur, ayant été fait Grand-Prêtre éternellement, selon l'ordre de Melchisédek\FTNT{Voir Ge. 14.}.
\Chap{7}
\TextTitle{Melchisédek, type de Christ\FTNTT{Ge. 14}}
\VerseOne{}En effet, ce Melchisédek était Roi de Salem et Prêtre du Dieu Très-Haut\FTNT{Ge. 14:18.}. Il alla au-devant d'Abraham lorsqu'il revenait de la défaite des rois, et il le bénit,
\VS{2}et auquel Abraham donna pour sa part la dîme de tout\FTNT{Ge. 14:20. Pour en savoir plus sur la dîme, voir les commentaires en De. 14:22, No. 18:21 et Mal. 3:10.}. Son nom signifie premièrement Roi de justice, et puis il a été Roi de Salem, c'est-à-dire, Roi de paix.
\VS{3}Il est sans père, sans mère, sans généalogie, n'ayant ni commencement de jours ni fin de vie, mais il est rendu semblable au Fils de Dieu. Il demeure Prêtre continuellement.
\TextTitle{La prêtrise de Melchisédek, supérieure à celle d'Aaron}
\VS{4}Considérez donc combien est grand celui à qui même Abraham, le patriarche, donna la dîme du butin.
\VS{5}Car, quant à ceux d'entre les fils de Lévi qui reçoivent la prêtrise, ils ont bien une ordonnance de dîmer le peuple selon la loi, c'est-à-dire, de dîmer leurs frères, bien qu'ils soient sortis des reins d'Abraham.
\VS{6}Mais celui qui n'était pas de la même famille qu'eux reçut d'Abraham la dîme, et bénit celui qui avait les promesses.
\VS{7}Or sans contredit, celui qui est le moindre est béni par celui qui est le plus grand.
\VS{8}Et ici, ce sont les hommes mortels qui prennent les dîmes~; mais là, c'est celui de qui il est rendu témoignage qu'il est vivant.
\VS{9}Et pour ainsi dire, Lévi même qui prend des dîmes, les a payées en Abraham~;
\VS{10}car il était encore dans les reins de son père, quand Melchisédek alla au-devant de lui.
\TextTitle{La prêtrise selon l'ordre d'Aaron n'a rien amené à la perfection}
\VS{11}Si donc la perfection s'était trouvée dans la prêtrise lévitique, (car c'est sous elle que le peuple a reçu la loi) quel besoin était-il après cela qu'un autre prêtre se lève selon l'ordre de Melchisédek, et qui ne soit point nommé selon l'ordre d'Aaron~?
\VS{12}Or la prêtrise étant changée, il est nécessaire qu'il y ait aussi un changement de loi.
\VS{13}Car, celui à l'égard duquel ces choses sont dites, appartient à une autre tribu, de laquelle nul n'a assisté à l'autel~;
\VS{14}car il est évident que notre Seigneur est descendu de la tribu de Juda\FTNT{Mt. 1:2.}, à l'égard de laquelle Moïse n'a rien dit de la prêtrise.
\VS{15}Et cela est encore plus incontestable, en ce qu'un autre prêtre, à la ressemblance de Melchisédek, est suscité~;
\VS{16}qui n'a point été fait prêtre selon la loi du commandement charnel, mais selon la puissance de la vie impérissable.
\VS{17}Car Dieu lui rend ce témoignage~: Tu es prêtre éternellement, selon l'ordre de Melchisédek.
\VS{18}Or il se fait une abolition du commandement qui a précédé, à cause de sa faiblesse, et parce qu'il ne pouvait point profiter.
\VS{19}Car la loi n'a rien amené à la perfection, mais ce qui a amené à la perfection, c'est ce qui a été introduit par-dessus, à savoir une meilleure espérance, par laquelle nous approchons de Dieu.
\VS{20}D'autant plus, même que cela n'a pas été sans serment,
\VS{21}car les Lévites sont devenus prêtres sans serment, mais celui-ci l'est devenu avec serment par celui qui lui a dit~: Le Seigneur l'a juré, et il ne s'en repentira pas\FTNT{Voir Ps. 110:4}~: Tu es prêtre éternellement, selon l'ordre de Melchisédek.
\VS{22}C'est donc d'une alliance d'autant plus excellente que Jésus a été fait le garant.
\TextTitle{Les prêtres sont mortels, seul Christ est éternel}
\VS{23}Et quant aux prêtres, il y en a eu plusieurs qui se sont succédés parce que la mort les empêchait d'être perpétuels.
\VS{24}Mais lui, parce qu'il demeure éternellement, possède une prêtrise qui n'est pas transmissible.
\VS{25}C'est pourquoi aussi il peut sauver parfaitement ceux qui s'approchent de Dieu par lui, étant toujours vivant pour intercéder\FTNT{Le Seigneur Jésus-Christ est le modèle parfait en ce qui concerne la prière d'intercession. Il se tient devant le Père pour nous. En tant qu'homme (1 Ti. 2:5) et Grand-Prêtre, il se tient entre le Père et l'homme pécheur, comme le faisaient les prêtres sous la loi mosaïque. Voir Lu. 22:31-32~; Ro. 8:34~; 1 Jn. 2:1-2.} pour eux.
\VS{26}Or il nous était convenable d'avoir un tel Grand-Prêtre, saint, innocent, sans tache, séparé des pécheurs, et élevé au-dessus des cieux,
\VS{27}qui n'avait pas besoin, comme les grands-prêtres, d'offrir tous les jours des sacrifices, premièrement pour ses péchés, et ensuite pour ceux du peuple, vu qu'il a fait cela une fois, s'étant offert lui-même.
\VS{28}Car, la loi établit grands-prêtres des hommes faibles~; mais la parole du serment qui a été fait après la loi, établit le Fils, qui est parfait pour toujours.
\Chap{8}
\TextTitle{L'ancienne prêtrise~: L'ombre des choses célestes}
\VerseOne{}La chose principale de notre discours, c'est que nous avons un tel Grand-Prêtre, qui est assis à la droite du trône de la majesté de Dieu dans les cieux,
\VS{2}serviteur du sanctuaire, et du véritable tabernacle, que le Seigneur a dressé et non pas les hommes.
\VS{3}Car tout grand-prêtre est établi pour offrir des offrandes et des sacrifices~; c'est pourquoi il est nécessaire que celui-ci ait aussi quelque chose à offrir.
\VS{4}Vu même que s'il était sur la terre, il ne serait pas prêtre, pendant qu'il y aurait encore des prêtres qui offrent les offrandes selon la loi~;
\VS{5}lesquels font le service dans le lieu qui n'est que l'image et l'ombre des choses célestes, selon que Dieu le dit à Moïse, quand il devait achever le tabernacle~: Or prends garde, lui dit-il, de faire toutes choses selon le modèle qui t'a été montré sur la montagne\FTNT{Ex. 25:40.}.
\TextTitle{Christ, le Médiateur d'une alliance plus excellente}
\VS{6}Mais maintenant, notre Grand-Prêtre a obtenu un service d'autant supérieur qu'il est le Médiateur d'une alliance plus excellente, qui a été établie sur de meilleures promesses.
\TextTitle{Les prophètes ont annoncé la Première Alliance}
\VS{7}En effet, si la Première Alliance avait été irréprochable, il n'y aurait pas eu lieu d'en chercher une seconde.
\VS{8}Car en censurant les Juifs, Dieu leur dit~: Voici, les jours viendront, dit le Seigneur, où je traiterai avec la maison d'Israël et avec la maison de Juda une Alliance Nouvelle,
\VS{9}non selon l'alliance que je traitai avec leurs pères, le jour où je les saisis par la main pour les tirer du pays d'Egypte~; car ils n'ont pas persévéré dans mon alliance, c'est pourquoi je les ai méprisés, dit le Seigneur.
\VS{10}Mais voici l'alliance que je traiterai, après ces jours-là, avec la maison d'Israël, dit le Seigneur~: Je mettrai mes lois dans leur esprit, et je les écrirai dans leur cœur, je serai leur Dieu, et ils seront mon peuple.
\VS{11}Personne n'enseignera plus son prochain, ni personne son frère, en disant~: Connais le Seigneur~! Parce que tous me connaîtront, depuis le plus petit jusqu'au plus grand d'entre eux~;
\VS{12}car je serai miséricordieux par rapport à leurs injustices, et je ne me souviendrai plus de leurs péchés, ni de leurs iniquités\FTNT{Jé. 31:31-34.}.
\VS{13}En disant une Nouvelle Alliance, il a déclaré vieille la première~; or, ce qui devient vieux et ancien, est près d'être aboli.
\Chap{9}
\TextTitle{Les ordonnances et le sanctuaire de la Première Alliance~: Des symboles}
\VerseOne{}En vérité, la Première Alliance avait aussi des ordonnances touchant le service divin, et un sanctuaire terrestre.\FTNT{Ex. 25:1-9.}.
\VS{2}Car il fut construit un premier tabernacle, appelé le lieu saint, dans lequel étaient le chandelier, et la table, et les pains de proposition\FTNT{Ex. 25:30.}.
\VS{3}Et après le second voile\FTNT{Ex. 26:31-35.} était le tabernacle, qui était appelé le Saint des saints,
\VS{4}ayant un encensoir d'or\FTNT{Encensoir ou autel d'or pour les parfums~: Lé. 16:12.}, et l'arche de l'alliance\FTNT{Ex. 25:10.}, entièrement couverte d'or tout autour, dans laquelle était le vase d'or\FTNT{Ex. 16:33.} où était la manne, et la verge d'Aaron\FTNT{No. 17:1-10.} qui avait fleuri, et les tables de l'alliance\FTNT{Les tables de l'alliance ou tables du témoignage~: Ex. 34:29~; De. 10:2-5.}.
\VS{5}Et au-dessus de l'arche étaient les chérubins de la gloire, couvrant de leur ombre le propitiatoire\FTNT{Propitiatoire ou couvercle de l'arche de l'alliance~: Lé. 9:7~; Lé. 16:15-17.}. Ce n'est pas le moment de parler en détail là-dessus.
\VS{6}Or ces choses étant ainsi disposées, les prêtres qui font le service entrent en tout temps dans le premier tabernacle\FTNT{No. 28:3.}~;
\VS{7}mais seul le grand-prêtre entre dans le second une fois par an, non sans y porter du sang, qu'il offre pour lui-même et pour les péchés du peuple\FTNT{Lé. 16:34.}.
\VS{8}Le Saint-Esprit faisant connaitre par là que le chemin du Saint des saints n'était pas encore manifesté, tandis que le premier tabernacle était encore debout,
\VS{9}lequel était une figure destinée pour le temps présent, durant lequel étaient offerts des offrandes et des sacrifices qui ne pouvaient point sanctifier la conscience de celui qui faisait le service,
\VS{10}ordonnés seulement en aliments, et en breuvages, en diverses ablutions, et en des cérémonies charnelles, jusqu'au temps de la réforme.
\TextTitle{La réalité du sacrifice s'accomplit en Christ}
\VS{11}Mais Christ est venu comme Grand-Prêtre des biens à venir~; il a traversé un tabernacle plus excellent et plus parfait, qui n'est pas un tabernacle construit de main d'homme, c'est-à-dire, qui n'est pas de cette création~;
\VS{12}et il est entré une fois pour toutes dans le Saint des saints, non avec le sang des veaux ou des boucs, mais avec son propre sang, après avoir obtenu une rédemption éternelle.
\VS{13}Car si le sang des taureaux et des boucs, et la cendre de la génisse\FTNT{No. 19:1-12.}, répandue sur ceux qui sont souillés, sanctifient et procurent la pureté de la chair,
\VS{14}combien plus le sang de Christ, qui, par l'Esprit éternel, s'est offert lui-même à Dieu sans nulle tache, purifiera-t-il votre conscience des œuvres mortes, pour servir le Dieu vivant~?
\VS{15}C'est pourquoi il est le Médiateur de la Nouvelle Alliance, afin que, la mort étant intervenue pour la rançon des transgressions commises sous la Première Alliance, ceux qui ont été appelés reçoivent l'héritage éternel qui leur a été promis.
\TextTitle{Les clauses du testament du Messie}
\VS{16}Car là où il y a un testament, il est nécessaire que la mort du testateur intervienne,
\VS{17}parce que c'est par la mort du testateur qu'un testament est rendu ferme, puisqu'il n'a aucune force tant que le testateur est en vie.
\VS{18}C'est pourquoi la Première Alliance elle-même n'a point été confirmée sans le sang.
\VS{19}Car Moïse, après avoir prononcé devant tout le peuple tous les commandements de la loi, prit le sang des veaux et des boucs, avec de l'eau, et de la laine écarlate, et de l'hysope~; et il en fit l'aspersion sur le livre et sur tout le peuple, en disant~:
\VS{20}Ceci est le sang de l'Alliance que Dieu vous a ordonné d'observer\FTNT{Ex. 24:3-8.}.
\VS{21}Puis il fit aussi aspersion avec du sang sur le tabernacle et sur tous les ustensiles du service\FTNT{Ex. 29:12~; Ex. 29:36.}.
\VS{22}Et presque toutes choses, selon la loi, sont purifiées par le sang, et sans effusion de sang il n'y a pas de rémission des péchés.
\TextTitle{Un sacrifice plus excellent\FTNTT{Lé. 16:33}}
\VS{23}Il a donc fallu que les choses qui représentaient celles qui sont aux cieux, soient purifiées par de telles choses, mais que les célestes le soient par des sacrifices plus excellents que ceux-là.
\VS{24}Car Christ n'est pas entré dans un sanctuaire fait de main d'homme, et qui n'était que la figure du véritable, mais il est entré dans le ciel même, afin de comparaître maintenant pour nous devant la face de Dieu.
\VS{25}Et ce n'est pas pour s'offrir lui-même plusieurs fois qu'il y est entré, ainsi que le grand-prêtre entre dans le Saint des saints, chaque année, avec un autre sang~;
\VS{26}autrement, il aurait fallu qu'il ait souffert plusieurs fois depuis la création du monde~; mais maintenant, à la fin des siècles, il a paru une seule fois pour l'abolition du péché par son sacrifice.
\VS{27}Et comme il est réservé aux hommes de mourir une seule fois\FTNT{Ce passage réfute la doctrine de la réincarnation.}, et après cela suit le jugement,
\VS{28}de même aussi Christ, qui s'est offert une seule fois pour ôter les péchés de plusieurs, apparaîtra sans péché une seconde fois à ceux qui l'attendent pour le salut.
\Chap{10}
\TextTitle{Le sacrifice unique de Christ est supérieur à tous les sacrifices}
\VerseOne{}Car la loi qui possède l'ombre des biens à venir, et non l'image exacte des choses, ne peut jamais, par les mêmes sacrifices que l'on offre continuellement chaque année, sanctifier ceux qui s'y attachent.
\VS{2}Autrement, n'auraient-ils pas cessé d'être offerts~? Parce que les adorateurs, une fois expurgés, n'auraient plus eu conscience des péchés.
\VS{3}Or le souvenir des péchés est réitéré dans ces sacrifices chaque année~;
\VS{4}car il est impossible que le sang des taureaux et des boucs ôte les péchés.
\VS{5}C'est pourquoi Jésus-Christ, en entrant dans le monde, a dit~: Tu n'as pas voulu de sacrifice, ni d'offrande, mais tu m'as formé un corps~;
\VS{6}tu n'as pas pris plaisir aux holocaustes, ni aux sacrifices pour le péché\FTNT{Ps. 40:7-9.}.
\VS{7}Alors j'ai dit~: Me voici, je viens, il est écrit de moi au commencement du livre~: Que je fasse, ô Dieu, ta volonté~!
\VS{8}Après avoir dit d'abord~: Tu n'as pas voulu de sacrifice, ni d'offrande, ni d'holocauste, ni d'offrande pour le péché et tu n'y as point pris plaisir, lesquelles choses sont pourtant offertes selon la loi, alors il dit~: Me voici, je viens afin de faire, ô Dieu, ta volonté~!
\VS{9}Il abolit ainsi le premier afin d'établir le second.
\VS{10}Or c'est par cette volonté que nous sommes sanctifiés, à savoir par l'offrande du corps de Jésus-Christ qui a été faite une fois pour toutes.
\VS{11}De plus, tout prêtre fait chaque jour le service et offre souvent les mêmes sacrifices, qui ne peuvent jamais ôter les péchés,
\VS{12}mais lui, après avoir offert un seul sacrifice pour les péchés, s'est assis pour toujours à la droite de Dieu,
\VS{13}attendant désormais que ses ennemis soient mis pour le marchepied de ses pieds.
\VS{14}Car, par une seule offrande, il a rendu parfaits pour toujours ceux qui sont sanctifiés.
\VS{15}Et c'est aussi ce que le Saint-Esprit nous témoigne~; car, après avoir dit premièrement~:
\VS{16}Voici l'alliance que je ferai avec eux, après ces jours-là, dit le Seigneur\FTNT{Voir Jé. 31:31-34.}~: C'est que je mettrai mes lois dans leur cœur, et je les écrirai dans leur esprit~;
\VS{17}et je ne me souviendrai plus de leurs péchés, ni de leurs iniquités.
\VS{18}Or, là où les péchés sont pardonnés, il n'y a plus d'offrande pour le péché.
\TextTitle{Exhortation à s'approcher de Dieu avec foi}
\VS{19}Ainsi donc, mes frères, nous avons la liberté d'entrer dans le Saint des saints au moyen du sang de Jésus,
\VS{20}qui est le chemin\FTNT{Jésus est le chemin qui conduit au Saint des saints, à la vie (Voir Jn. 14:6), et ce chemin n'était pas encore manifesté avant sa naissance. Hé. 9:8.} nouveau et vivant qu'il a inauguré pour nous à travers le voile, c'est-à-dire sa propre chair,
\VS{21}et ayant un Grand-Prêtre établi sur la maison de Dieu,
\VS{22}approchons-nous de lui avec un cœur sincère, et une foi inébranlable, ayant les cœurs purifiés d'une mauvaise conscience, et le corps lavé d'une eau pure.
\VS{23}Retenons fermement la profession de notre espérance, car celui qui nous a fait la promesse est fidèle.
\VS{24}Veillons les uns sur les autres pour nous exciter à la charité et aux bonnes œuvres.
\VS{25}N'abandonnons pas notre assemblée\FTNT{Assemblée~: Du grec «~episunagoge~» qui veut dire «~être assemblé en un lieu, assemblée religieuse des chrétiens~». Il est question de ne pas abandonner la communion fraternelle et non une église locale. En effet, il est du devoir du chrétien de se séparer des faux frères de peur d'être entraîné dans leur égarement (Mt. 18:15-17~; 1 Co. 5:11~; 1 Co. 15:33).}, comme c'est la coutume de quelques-uns~; mais exhortons-nous les uns les autres, et cela d'autant plus que vous voyez approcher le jour.
\TextTitle{Ne pas mépriser le sacrifice de Christ}
\VS{26}Car, si nous péchons volontairement après avoir reçu la connaissance de la vérité, il ne reste plus de sacrifice pour les péchés,
\VS{27}mais une attente terrible du jugement et l'ardeur d'un feu qui doit dévorer les adversaires.
\VS{28}Si quelqu'un avait méprisé la loi de Moïse, il mourait sans miséricorde, sur la déposition de deux ou de trois témoins\FTNT{De. 17:6.}~;
\VS{29}de combien pires tourments pensez-vous donc que sera jugé digne celui qui aura foulé aux pieds le Fils de Dieu, et qui aura tenu pour une chose profane le sang de l'Alliance, par lequel il avait été sanctifié, et qui aura outragé l'Esprit de grâce~?
\VS{30}Car nous connaissons celui qui a dit~: C'est à moi que la vengeance appartient, et je le rendrai~! Dit le Seigneur. Et encore~: Le Seigneur jugera son peuple.\FTNT{De. 32:35-36.}
\VS{31}C'est une chose terrible que de tomber entre les mains du Dieu vivant.
\VS{32}Or rappelez-vous des premiers jours, où, après avoir été éclairés, vous avez soutenu un grand combat de souffrances,
\VS{33}ayant été, d'une part, exposés à la vue de tout le monde par des opprobres et des afflictions, et de l'autre, ayant participé aux maux de ceux qui ont souffert de semblables indignités.
\VS{34}Car vous avez aussi été participants de l'affliction de mes liens, et vous avez reçu avec joie l'enlèvement de vos biens, sachant en vous-mêmes que vous avez dans les cieux des biens meilleurs et permanents.
\VS{35}N'abandonnez donc pas cette fermeté que vous avez fait paraître, et qui sera bien récompensée.
\VS{36}Parce que vous avez besoin de patience, afin qu'après avoir fait la volonté de Dieu, vous receviez l'effet de sa promesse.
\TextTitle{La marche par la foi~: Exemples d'hommes et de femmes de foi}
\VS{37}Car, encore un peu de temps, et celui qui doit venir, viendra, et il ne tardera point.
\VS{38}Or le juste vivra de la foi~; mais si quelqu'un se retire, mon âme ne prend point de plaisir en lui\FTNT{Ha. 2:4.}.
\VS{39}Mais pour nous, nous ne sommes pas de ceux qui se retirent~; ce serait notre perdition~; mais nous persévérons dans la foi, pour le salut de l'âme.
\Chap{11}
\VerseOne{}Or la foi rend présentes les choses qu'on espère, et elle est une démonstration de celles qu'on ne voit point.
\VS{2}Car c'est par elle que les anciens ont obtenu un bon témoignage.
\VS{3}Par la foi, nous comprenons que l'univers a été fait par la parole de Dieu, de sorte que les choses qui se voient, n'ont pas été faites des choses visibles.
\VS{4}Par la foi, Abel\FTNT{Ge. 4:3-5.} offrit à Dieu un sacrifice plus excellent que Caïn~; et par elle il obtînt le témoignage d'être juste, parce que Dieu rendait témoignage de ses offrandes~; et c'est par elle qu'il parle encore, quoique mort.
\VS{5}Par la foi, Hénoc\FTNT{Ge. 5:22-24.} fut enlevé pour ne pas voir la mort, et il ne parut plus parce que Dieu l'avait enlevé~; car, avant qu'il soit enlevé, il avait obtenu le témoignage d'avoir été agréable à Dieu.
\VS{6}Or il est impossible de lui être agréable sans la foi~; car il faut que celui qui vient à Dieu, croie que Dieu est, et qu'il est le rémunérateur de ceux qui le cherchent.
\VS{7}Par la foi, Noé\FTNT{Ge. 6:14-22.}, ayant été divinement averti des choses qui ne se voyaient point encore, craignit, et bâtit l'arche pour la conservation de sa famille~; et par cette arche, il condamna le monde, et devint héritier de la justice qui est selon la foi.
\VS{8}Par la foi, Abraham\FTNT{Ge 12:1-4.}, étant appelé, obéit, pour aller sur la terre, qu'il devait recevoir en héritage, et il partit sans savoir où il allait.
\VS{9}Par la foi, il demeura comme étranger sur la terre, qui lui avait été promise, comme si elle ne lui avait point appartenu, demeurant sous des tentes avec Isaac et Jacob, qui étaient héritiers avec lui de la même promesse.
\VS{10}Car il attendait la cité qui a des fondements, celle dont Dieu est l'architecte et le constructeur.
\VS{11}Par la foi, aussi Sara\FTNT{Ge. 21:1-2.} reçut la force de concevoir un enfant, et elle enfanta hors d'âge, parce qu'elle fut persuadée que celui qui le lui avait promis, était fidèle.
\VS{12}C'est pourquoi d'un seul homme, et qui était déjà affaibli, il est né une multitude aussi nombreuse que les étoiles du ciel, et que le sable du bord de la mer, qui ne peut se compter\FTNT{Ge. 22:17.}.
\VS{13}Tous ceux-ci sont morts dans la foi, sans avoir reçu les choses dont ils avaient eu les promesses, mais ils les ont vues de loin, crues, et saluées, et ils ont fait profession qu'ils étaient étrangers et voyageurs sur la terre\FTNT{1 Pi. 2:11.}.
\VS{14}Car ceux qui tiennent ces discours montrent clairement qu'ils cherchent encore leur patrie.
\VS{15}Et certes, s'ils avaient eu en vue celle d'où ils étaient sortis, ils auraient eu le temps d'y retourner.
\VS{16}Mais maintenant, ils en désirent une meilleure, c'est-à-dire une céleste. C'est pourquoi Dieu n'a pas honte d'être appelé leur Dieu, parce qu'il leur a préparé une cité\FTNT{Jn. 14:2~; Ap. 21:2.}.
\VS{17}Par la foi, Abraham étant éprouvé, offrit Isaac~; celui qui avait reçu les promesses offrit même son fils unique\FTNT{Ge. 22:1.},
\VS{18}à l'égard duquel il lui avait été dit~: Les descendants d'Isaac seront ta véritable postérité\FTNT{Ge. 21:12.}.
\VS{19}Ayant estimé que Dieu pouvait même le ressusciter d'entre les morts~; c'est pourquoi aussi il le recouvra par une espèce de résurrection.
\VS{20}Par la foi, Isaac bénit Jacob et Esaü, en vue des choses à venir\FTNT{Ge. 27:26-40.}.
\VS{21}Par la foi, Jacob, mourant, bénit chacun des fils de Joseph\FTNT{Ge. 48:1-22.}, et adora Dieu, appuyé sur l'extrémité de son bâton\FTNT{Ge. 47:31.}.
\VS{22}Par la foi, Joseph mourant fit mention de la sortie des enfants d'Israël, et il donna des ordres au sujet de ses os\FTNT{Ge. 50:24-25.}.
\VS{23}Par la foi, Moïse\FTNT{Ex. 2:1-3.}, à sa naissance, fut caché pendant trois mois par son père et sa mère, parce qu'ils virent que l'enfant était beau, et ils ne craignirent pas l'ordre du roi.
\VS{24}Par la foi, Moïse, devenu grand, refusa d'être nommé fils de la fille de Pharaon,
\VS{25}choisissant plutôt d'être affligé avec le peuple de Dieu, que de jouir pour un peu de temps des délices du péché.
\VS{26}Et ayant estimé que l'opprobre de Christ était un plus grand trésor que les richesses de l'Egypte, parce qu'il avait égard à la rémunération.
\VS{27}Par la foi, il quitta l'Egypte, sans craindre la fureur du roi~; car il demeura ferme, comme voyant celui qui est invisible.
\VS{28}Par la foi, il fit la Pâque et l'aspersion du sang, afin que le destructeur qui tuait les premiers-nés, ne touche pas aux premiers-nés des Israélites\FTNT{Ex. 12:1-51.}.
\VS{29}Par la foi, ils traversèrent la Mer Rouge, comme un lieu sec, ce que les Egyptiens essayèrent de tenter, ils furent engloutis dans les eaux\FTNT{Ex. 14:13-31.}.
\VS{30}Par la foi, les murs de Jéricho tombèrent, après qu'on en eut fait le tour pendant sept jours\FTNT{Jos. 6:1-20.}.
\VS{31}Par la foi, Rahab, la prostituée, ne périt pas avec les incrédules, parce qu'elle avait reçu les espions et les avait renvoyés en paix\FTNT{Jos. 2:1-21~; Jos. 6:23.}.
\VS{32}Et que dirai-je encore~? Car le temps me manquerait si je voulais parler de Gédéon\FTNT{Jg. 6:11.}, et de Barak\FTNT{Jg. 4:6.}, et de Samson\FTNT{Jg. 13:24.}, et de Jephté\FTNT{Jg. 11:1.}, et de David\FTNT{1 S. 16-17.}, et de Samuel\FTNT{1 S. et 2 S.}, et des prophètes,
\VS{33}qui par la foi combattirent des royaumes, exercèrent la justice, obtinrent des promesses, fermèrent la gueule des lions,
\VS{34}éteignirent la force du feu, échappèrent au tranchant des épées, des malades devinrent vigoureux, se montrèrent fort dans la bataille, et mirent en fuite des armées étrangères.
\VS{35}Des femmes recouvrèrent leurs morts par le moyen de la résurrection~; et d'autres furent livrés aux tourments et n'acceptèrent point d'être délivrés, afin d'obtenir une meilleure résurrection.
\VS{36}Et d'autres subirent les moqueries et le fouet, les chaînes et la prison~;
\VS{37}ils furent lapidés, sciés, subirent de rudes épreuves, ils furent mis à mort par le tranchant de l'épée, ils errèrent çà et là, vêtus de peaux de brebis et de chèvres, réduits à la misère, affligés, tourmentés,
\VS{38}eux dont le monde n'était pas digne, errant dans les déserts et dans les montagnes, et dans les cavernes et dans les trous de la terre.
\VS{39} Et quoiqu'ils aient tous été recommandables par leur foi, ils n'ont pourtant point reçu l'effet de la promesse,
\VS{40}Dieu ayant pourvu quelque chose de meilleur pour nous, en sorte qu'ils ne parviennent pas à la perfection sans nous.
\Chap{12}
\TextTitle{Fixer les regards sur Jésus}
\VerseOne{}Nous donc aussi, puisque nous sommes environnés d'une si grande nuée de témoins\FTNT{Témoin~: du grec «~martus~», terme qui dans un sens légal et historique signifie «~celui qui est spectateur d'une chose~». Dans un sens éthique, il est question de «~ceux qui ont prouvé la force et l'authenticité de leur foi en Christ en supportant une mort violente~». «~Martus~» a donné le mot «~martyr~» en français.}, rejetons tout fardeau, et le péché qui nous enveloppe si aisément, et poursuivons constamment la course qui nous est proposée,
\VS{2}portant les yeux sur Jésus, le chef et le consommateur de la foi qui en échange de la joie qui lui était réservée, il a souffert la croix, ayant méprisé la honte, et s'est assis à la droite du trône de Dieu.
\VS{3}C'est pourquoi, considérez soigneusement celui qui a supporté contre sa personne une telle opposition de la part des pécheurs, afin que vous ne succombiez point, en perdant courage.
\VS{4}Vous n'avez pas encore résisté jusqu'à répandre votre sang en combattant contre le péché.
\TextTitle{La correction du Père}
\VS{5}Et cependant vous avez oublié l'exhortation qui vous est adressée comme à ses fils, disant~: Mon fils, ne méprise pas le châtiment du Seigneur, et ne perds point courage lorsqu'il te reprend~;
\VS{6}car le Seigneur châtie celui qu'il aime, et il frappe de la verge tous ceux qu'il reconnaît pour ses fils\FTNT{Pr. 3:11-12.}.
\VS{7}Si vous endurez le châtiment, Dieu se présente à vous comme à ses fils~; car qui est le fils que le père ne châtie point~?
\VS{8}Mais si vous êtes sans châtiment auquel tous participent, vous êtes donc des enfants illégitimes, et non pas des fils.
\VS{9}Et puisque nos pères selon la chair nous ont châtiés, et que malgré cela nous les avons respectés, ne serons-nous pas beaucoup plus soumis au Père des esprits, pour avoir la vie~?
\VS{10}Car par rapport à ceux-là, ils nous châtiaient pour un peu de temps, suivant leur volonté, mais celui-ci nous châtie pour notre profit, afin que nous soyons participants de sa sainteté.
\VS{11}Or tout châtiment ne semble pas sur l'heure être un sujet de joie, mais de tristesse~; mais ensuite il produit un fruit paisible de justice à ceux qui sont exercés par ce moyen.
\VS{12}Fortifiez donc vos mains languissantes et vos genoux affaiblis~;
\VS{13}et suivez avec vos pieds des chemins droits, afin que ce qui est boiteux ne dévie pas, mais plutôt se consolide.
\VS{14}Recherchez la paix avec tous, et la sanctification, sans laquelle nul ne verra le Seigneur.
\TextTitle{Que nul ne se prive de la grâce de Dieu !}
\VS{15}Veillez à ce que personne ne se prive de la grâce de Dieu~; à ce qu'aucune racine d'amertume, poussant des rejetons, ne vous trouble, et que plusieurs n'en soient souillés par elles~;
\VS{16}que nul de vous ne soit fornicateur, ou profane comme Esaü, qui pour un aliment vendit son droit d'aînesse\FTNT{Ge. 25:33}.
\VS{17}Car vous savez que plus tard, désirant hériter la bénédiction, il fut rejeté, car il ne trouva point de lieu à la repentance, quoiqu'il l'ait demandée avec larmes.
\TextTitle{L'Eglise véritable s'est approchée de Sion}
\VS{18}Vous ne vous êtes pas approchés d'une montagne qu'on pouvait toucher avec la main\FTNT{Ex. 19:12.}, ni du feu brûlant, ni de la nuée épaisse, ni des ténèbres, ni de la tempête,
\VS{19}ni du retentissement de la trompette, ni du son des paroles, au sujet duquel ceux qui l'entendirent prièrent que la parole ne leur soit plus adressée\FTNT{Ex. 20:18-26.},
\VS{20}car ils ne pouvaient pas supporter ce qui était ordonné, que si même une bête touche la montagne, elle sera lapidée ou percée d'un dard\FTNT{Ex. 19:13.}.
\VS{21}Et ce spectacle était si terrible que Moïse dit~: Je suis épouvanté et tout tremblant~!
\VS{22}Mais vous vous êtes approchés de la montagne de Sion, de la Cité du Dieu vivant, la Jérusalem céleste, d'une multitude innombrable d'anges,
\VS{23}et de l'assemblée et de l'Eglise des premiers-nés qui sont inscrits dans les cieux, du Dieu qui est le juge de tous, et des esprits des justes qui ont été rendus parfaits,
\VS{24}de Jésus, qui est le Médiateur de la Nouvelle Alliance, et du sang de l'aspersion, qui prononce des meilleurs choses que celui d'Abel.
\TextTitle{Exhortation à la crainte de Dieu}
\VS{25}Prenez garde de ne pas mépriser celui qui vous parle~; car si ceux qui méprisèrent celui qui leur parlait sur la terre, n'ont pas échappé, nous serons punis beaucoup plus, si nous nous détournons de celui qui parle des cieux,
\VS{26}lui, dont la voix ébranla alors la terre, mais à l'égard du temps présent, il a fait cette promesse, disant~: J'ébranlerai encore une fois non seulement la terre, mais aussi le ciel\FTNT{Ag. 2:6.}.
\VS{27}Or ces mots~: Une fois encore, marquent le changement des choses ébranlées, comme étant faites pour un temps, afin que celles qui sont inébranlables demeurent.
\VS{28}C'est pourquoi, saisissant le Royaume qui ne peut point être ébranlé, retenons la grâce par laquelle nous servions Dieu, en sorte que nous lui soyons agréables avec respect et avec crainte,
\VS{29}car notre Dieu est aussi un feu dévorant\FTNT{De. 4:24.}.
\Chap{13}
\TextTitle{Exhortations~; invariabilité de Christ}
\VerseOne{}Que la charité fraternelle demeure dans vos cœurs.
\VS{2}N'oubliez pas l'hospitalité~; car, par elle, quelques-uns ont logé des anges sans le savoir.
\VS{3}Souvenez-vous des prisonniers, comme si vous étiez emprisonnés avec eux~; et de ceux qui sont maltraités, comme étant aussi vous-mêmes du même corps.
\VS{4}Le mariage est honorable entre tous, et le lit sans souillure~; mais Dieu jugera les fornicateurs et les adultères.
\VS{5}Que votre conduite soit sans avarice, étant contents de ce que vous avez présentement~; car lui-même a dit~: Je ne te délaisserai point, et je ne t'abandonnerai point\FTNT{De. 31:6.}.
\VS{6}De sorte que nous pouvons dire avec assurance~: Le Seigneur est mon aide, et je ne craindrai point ce que l'homme pourrait me faire\FTNT{Ps. 118:6.}.
\VS{7}Souvenez-vous de vos conducteurs qui vous ont annoncé la parole de Dieu~; considérez quelle a été la fin de leur vie, et imitez leur foi.
\VS{8}Jésus-Christ est le même hier, aujourd'hui, et il l'est aussi éternellement.
\VS{9}Ne soyez point emportés çà et là par des doctrines diverses et étrangères~; car il est bon que le cœur soit affermi par la grâce, et non point par les aliments, lesquelles n'ont en rien profité à ceux qui s'y sont attachés.
\TextTitle{Porter ses regards sur la cité céleste}
\VS{10}Nous avons un autel dont ceux qui servent dans le tabernacle n'ont pas le droit de manger.
\VS{11}Car les corps des animaux, dont le sang est porté dans le sanctuaire par le grand-prêtre pour le péché, sont brûlés hors du camp.
\VS{12}C'est pourquoi aussi Jésus, afin de sanctifier le peuple par son propre sang, a souffert hors de la porte\FTNT{Ex. 29:14. Jésus a souffert hors de Jérusalem (Jn. 19:17-18).}.
\VS{13}Sortons donc vers lui, hors du camp\FTNT{Le mot «~camp~» dans ce passage vient du grec «~parambole~», terme faisant référence au judaïsme antique dans lequel s'étaient embourbés les chrétiens d'origine hébraïque. Aujourd'hui, il représente plutôt le christianisme paganisé, essentiellement basé sur la loi de Moïse et constituant une prison qui empêche certains enfants de Dieu de vivre pleinement leur liberté en Christ.}, en portant son opprobre.
\VS{14}Car nous n'avons point ici-bas de cité permanente, mais nous recherchons celle qui est à venir.
\TextTitle{Le sacrifice de louange et du serviteur de Dieu}
\VS{15}Offrons donc par lui sans cesse à Dieu un sacrifice de louange, c'est-à-dire, le fruit des lèvres, en confessant son Nom.
\VS{16}Or n'oubliez pas la bienfaisance et de faire part de vos biens, car Dieu prend plaisir à de tels sacrifices.
\TextTitle{L'obéissance aux conducteurs}
\VS{17}Obéissez\FTNT{Le terme «~obéissez~», en grec «~peitho~», veut dire «~se laisser persuader par des mots~». Il signifie aussi «~donner avec persuasion l'envie à quelqu'un de faire quelque chose en le rassurant~». Par conséquent, les conducteurs doivent comprendre que la soumission et l'obéissance des chrétiens n'a rien à voir avec la dictature et l'autoritarisme. Ils doivent les rassurer et les convaincre - car tout ce qui n'est pas fait avec foi est péché (Ro. 14:23) - et ne pas tyranniser leurs frères en les obligeant à leur obéir (Mt. 20:25~; 1 Pi. 5:2-3).} à vos conducteurs, et soyez-leur soumis, car ils veillent pour vos âmes, comme devant en rendre compte~; afin que ce qu'ils en font, ils le fassent avec joie, et non en gémissant, car cela ne vous serait pas profitable.
\VS{18} Priez pour nous, car nous nous assurons que nous avons une bonne conscience, désirant nous conduire honnêtement parmi tous.
\VS{19}C'est avec instance que je vous demande de le faire, afin que je vous sois rendu plus tôt.
\TextTitle{Bénédictions et salutations}
\VS{20}Que le Dieu de paix, qui a ramené d'entre les morts le grand Pasteur des brebis, par le sang de l'Alliance éternelle, notre Seigneur Jésus-Christ,
\VS{21}vous rende capables de toute bonne œuvre pour faire sa volonté~; qu'il fasse en vous ce qui lui est agréable par Jésus-Christ~; auquel soit la gloire aux siècles des siècles~! Amen~!
\VS{22}Aussi, mes frères, je vous prie de supporter la parole d'exhortation, car je vous ai écrit en peu de mots.
\VS{23}Sachez que notre frère Timothée a été relâché~; s'il vient bientôt, je vous verrai avec lui.
\VS{24}Saluez tous vos conducteurs, et tous les saints. Ceux d'Italie vous saluent.
\VS{25}Que la grâce soit avec vous tous~! Amen~!
\PPE{}
\end{multicols}

%\clearpage\ShortTitle{1 Jean}\BookTitle{1 Jean}\BFont
\noindent\hrulefill
{\footnotesize
\textit{
\bigskip
{\centering{}
\\Auteur : Jean
\\Signification : Yahweh a fait grâce
\\Thème : La communion fraternelle, la connaissance et l'amour
\\Date de rédaction : Env. 85 ap. J.-C.\\}
}
%\bigskip
\textit{
\\Cette épître, écrite par Jean à Ephèse, était destinée aux églises de la province d’Asie qu’il connaissait bien. Il souhaite rendre leur joie parfaite en fortifiant leur foi en Christ et en leur donnant l’assurance de la vie éternelle ;  tout en les mettant en garde contre les faux docteurs.\bigskip
}
}
\par\nobreak\noindent\hrulefill
\begin{multicols}{2}
\Chap{1}
\TextTitle{La Parole incarnée}
\VerseOne{}Ce qui était dès le commencement, ce que nous avons entendu, ce que nous avons vu de nos propres yeux, ce que nous avons contemplé, et que nos propres mains ont touché concernant la Parole de vie,
\VS{2}car la vie a été manifestée, et nous l'avons vue et nous lui rendons témoignage, et nous vous annonçons la vie éternelle, qui était avec le Père, et qui nous a été manifestée.
\TextTitle{Communion avec le Père et le Fils}
\VS{3}Ce que nous avons vu dis-je, et ce que nous avons entendu, nous vous l'annonçons, afin que vous soyez en communion avec nous, et que notre communion soit avec le Père et avec son Fils Jésus-Christ.
\VS{4}Et nous vous écrivons ces choses, afin que votre joie soit parfaite.
\TextTitle{De la communion avec Dieu, qui est Lumière, et de la confession des péchés}
\TextTitle{[Conditions de la communion avec Dieu
\\a. Position de l'enfant de Dieu dans la Lumière]}
\VS{5}Or c'est ici la déclaration que nous avons entendue de lui et que nous vous annonçons, à savoir que Dieu est Lumière et qu'il n'y a point en lui de ténèbres.
\VS{6}Si nous disons que nous sommes en communion avec lui, et que nous marchions dans les ténèbres, nous mentons, et nous n'agissons pas selon la vérité.
\VS{7}Mais si nous marchons dans la Lumière, comme Dieu est dans la Lumière, nous sommes en communion les uns avec les autres, et le sang de son Fils Jésus-Christ nous purifie de tout péché.
\TextTitle{b. Reconnaissance de la présence du péché en nous}
\VS{8}Si nous disons que nous n'avons point de péché, nous nous séduisons nous-mêmes, et la vérité n'est point en nous.
\TextTitle{c. la confession des péchés, le pardon et la purification}
\VS{9}Si nous confessons nos péchés, il est fidèle et juste pour nous les pardonner, et pour nous purifier de toute iniquité.
\VS{10}Si nous disons que nous n'avons point de péché, nous le faisons menteur, et sa Parole n'est point en nous.
\TextTitle{Celui qui connaît Jésus-Christ garde ses commandements}
\Chap{2}
\TextTitle{d. Christ, notre avocat pour nos péchés}
\VerseOne{}Mes petits-enfants, je vous écris ces choses afin que vous ne péchiez point. Et si quelqu'un a péché, nous avons un avocat\FTNT{Jésus, notre Avocat. Le mot grec «~parakletos~», traduit ici par «~avocat~», se trouve également en Jean 14 et 16, où il est traduit par «~Consolateur~» et s’applique au Saint-Esprit. Le Seigneur exerce la fonction d'avocat actuellement pour nous dans le ciel. Voir Ro. 8:33 ; Hé. 7:25.} auprès du Père, Jésus-Christ, le Juste.
\VS{2}Car c'est lui qui est la victime de propitiation pour nos péchés, et non seulement pour les nôtres, mais aussi pour ceux de tout le monde.
\TextTitle{e. reconnaissance de la sainteté de Dieu}
\VS{3}Et nous savons que nous l'avons connu, si nous gardons ses commandements.
\VS{4}Celui qui dit : Je l'ai connu, et qui ne garde point ses commandements, est un menteur, et il n'y a point de vérité en lui.
\VS{5}Mais celui qui garde sa Parole, l'amour de Dieu est véritablement parfait en lui : et c'est par cela que nous savons que nous sommes en lui.
\VS{6}Celui qui dit qu'il demeure en lui doit aussi vivre comme Jésus-Christ lui-même a vécu.
\VS{7}Mes frères, je ne vous écris point un commandement nouveau, mais un commandement ancien, que vous avez eu dès le commencement ; et ce commandement ancien c'est la Parole que vous avez entendue dès le commencement.
\VS{8}Cependant, le commandement que je vous écris est un commandement nouveau, c’est une chose véritable en lui et en vous, parce que les ténèbres sont passées, et que la véritable Lumière paraît déjà.
\VS{9}Celui qui dit qu'il est dans la Lumière, et qui hait son frère, est dans les ténèbres jusqu'à présent.
\VS{10}Celui qui aime son frère demeure dans la Lumière, et il n'y a rien en lui qui puisse le faire tomber.
\VS{11}Mais celui qui hait son frère est dans les ténèbres, et il marche dans les ténèbres, et il ne sait pas où il va car les ténèbres ont aveuglé ses yeux.
\TextTitle{Exhortations à la famille spirituelle}
\VS{12}Mes petits enfants, je vous écris parce que vos péchés vous sont pardonnés à cause de son Nom.
\VS{13}Pères, je vous écris parce que vous avez connu celui qui est dès le commencement. Jeunes gens, je vous écris parce que vous avez vaincu l’esprit du malin.
\VS{14}Jeunes enfants, je vous écris parce que vous avez connu le Père. Pères, je vous ai écrit parce que vous avez connu celui qui est dès le commencement. Jeunes gens, je vous ai écrit parce que vous êtes forts et que la Parole de Dieu demeure en vous, et que vous avez vaincu l’esprit du malin.
\TextTitle{Les enfants de Dieu ne doivent pas aimer le monde}
\VS{15}N'aimez point le monde ni les choses qui sont dans le monde ; si quelqu'un aime le monde, l'amour du Père n'est point en lui.
\VS{16}Car tout ce qui est dans le monde, c'est-à-dire la convoitise de la chair, la convoitise des yeux et l'orgueil de la vie, ne vient point du Père, mais vient du monde.
\VS{17}Et le monde passe, avec sa convoitise ; mais celui qui fait la volonté de Dieu demeure éternellement.
\TextTitle{Les enfants de Dieu mis en garde contre les apostats}
\VS{18}Petits enfants, c'est ici la dernière\FTNT{Dernière, du grec «~eschatos~», signifie «~dernier dans une succession dans le temps~». Voir Ge. 49:1-2.} heure ; et comme vous avez entendu que l'Antéchrist viendra, il y a maintenant plusieurs antéchrists ; et par là nous connaissons que c'est la dernière heure.
\VS{19}Ils sont sortis du milieu de nous, mais ils n'étaient pas des nôtres ; car s'ils avaient été des nôtres, ils seraient demeurés avec nous, mais c'est afin qu'il soit manifeste que tous ne sont point des nôtres.
\VS{20}Mais vous avez été oints par le Saint-Esprit, et vous connaissez toutes choses.
\VS{21}Je ne vous ai pas écrit comme si vous ne connaissiez point la vérité, mais parce que vous la connaissez, et qu'aucun mensonge ne vient de la vérité.
\VS{22}Qui est le menteur, sinon celui qui nie que Jésus est le Christ ? Celui-là est l'Antéchrist qui nie le Père et le Fils.
\VS{23}Quiconque nie le Fils, n'a point non plus le Père ; quiconque confesse le Fils, a aussi le Père.
\VS{24}Que ce que vous avez entendu dès le commencement demeure en vous, car si ce que vous avez entendu dès le commencement demeure en vous, vous demeurerez aussi dans le Fils et dans le Père.
\VS{25}Et c'est ici la promesse qu'il nous a faite, à savoir la vie éternelle.
\VS{26}Je vous ai écrit ces choses au sujet de ceux qui vous séduisent.
\VS{27}Mais l'onction que vous avez reçue de lui demeure en vous, et vous n'avez pas besoin qu'on vous enseigne ; mais comme la même onction vous enseigne toutes choses, qu'elle est véritable et n'est pas un mensonge, demeurez en lui selon les enseignements qu’elle vous a donnés.
\TextTitle{Exhortations à deumeurer en Christ}
\VS{28}Maintenant donc, mes petits enfants, demeurez en lui ; afin que quand il apparaîtra, nous ayons de l'assurance, et que nous ne soyons point confus devant lui lors de son avènement\FTNT{Avènement, du grec «~parousia~», veut dire «~l’arrivée~» ou «~la présence~». Lors de cette seconde venue, le Messie prendra son Epouse pour les noces, ensuite il posera ses pieds sur le Mont des Oliviers, détruira les armées de l'Antichrist, puis commencera son règne de mille ans. Voir Za. 14.}.
\VS{29}Si vous savez qu'il est juste, sachez que quiconque fait ce qui est juste est né de lui.
\Chap{3}
\VerseOne{}Voyez quelle charité le Père nous a témoignée, pour que nous soyons appelés enfants de Dieu ! Mais le monde ne nous connaît point, parce qu'il ne l'a point connu.
\VS{2}Mes bien-aimés, nous sommes maintenant enfants de Dieu, et ce que nous serons n'est pas encore manifesté ; or nous savons que lorsque le Fils de Dieu apparaîtra, nous serons semblables à lui, car nous le verrons tel qu'il est.
\VS{3}Et quiconque a cette espérance en lui se purifie, comme lui aussi est pur.
\TextTitle{Caractéristiques des enfants de Dieu et des enfants du diable}
\VS{4}Quiconque pèche, transgresse la loi, car le péché est la transgression de la loi.
\VS{5}Or vous savez qu'il est apparu pour ôter nos péchés ; et il n'y a point de péché en lui.
\VS{6}Quiconque demeure en lui ne pèche point ; quiconque pèche, ne l'a pas vu, et ne l'a pas connu.
\VS{7}Mes petits-enfants, que personne ne vous séduise. Celui qui fait ce qui est juste est une personne juste, comme Jésus-Christ est juste.
\VS{8}Celui qui vit dans le péché est du diable, car le diable pèche dès le commencement. Or le Fils de Dieu est apparu afin de détruire les œuvres du diable.
\VS{9}Quiconque est né de Dieu ne vit pas dans le péché, car la semence de Dieu demeure en lui ; et il ne peut pécher, parce qu'il est né de Dieu.
\VS{10}Et c'est par là que nous connaissons les enfants de Dieu et les enfants du diable. Quiconque ne fait pas ce qui est juste et qui n'aime pas son frère n'est point de Dieu.
\VS{11}Car ce qui vous a été annoncé et ce que vous avez entendu dès le commencement c’est que nous nous aimions les uns les autres.
\VS{12}Et que nous ne soyons pas comme Caïn\FTNT{La doctrine de la semence du serpent est présentée par certains comme une explication du sens caché de la chute de l’homme dans le jardin en Eden et du péché originel. Selon cette doctrine, l’acte sexuel serait le fruit de l’arbre de la connaissance du bien et du mal. Cependant, cette doctrine n’est pas biblique, Eve n’a jamais eu de relations sexuelles avec le serpent. Dans Jean 8:44, lorsque Jésus dit aux pharisiens «~vous avez pour père le diable…~» suppose-t-il que le diable engendre des enfants physiquement ? Bien sûr que non !}, qui était de l'esprit malin et qui tua son frère. Et pourquoi le tua-t-il ? C’est parce que ses œuvres étaient mauvaises, et que celles de son frère étaient justes.
\VS{13}Mes frères, ne vous étonnez point si le monde vous hait.
\VS{14}Nous savons que nous sommes passés de la mort à la vie parce que nous aimons nos frères. Celui qui n'aime pas son frère demeure dans la mort.
\VS{15}Quiconque hait son frère est un meurtrier, et vous savez qu'aucun meurtrier ne possède la vie éternelle.
\VS{16}Nous avons connu la charité en ce qu'il a donné sa vie pour nous ; nous aussi, nous devons donner nos vies pour nos frères\FTNT{Jn. 15:13.}.
\VS{17}Si quelqu’un possède les biens du monde, et que voyant son frère dans la nécessité, il lui ferme ses entrailles, comment la charité de Dieu demeure-t-elle en lui ?
\VS{18}Mes petits-enfants, n'aimons pas en paroles et avec la langue, mais par des œuvres et en vérité.
\VS{19}Car c'est par là que nous connaissons que nous sommes de la vérité ; et nous rassurerons ainsi nos cœurs devant lui.
\VS{20}Si notre cœur nous condamne, certes Dieu est plus grand que notre cœur, et il connaît toutes choses.
\VS{21}Mes bien-aimés, si notre cœur ne nous condamne point, nous avons de l’assurance devant Dieu.
\VS{22}Et quoi que nous demandions, nous le recevons de lui, parce que nous gardons ses commandements, et que nous faisons les choses qui lui sont agréables.
\VS{23}Et c'est ici son commandement, que nous croyions au Nom de son Fils Jésus-Christ, et que nous nous aimions les uns les autres, selon le commandement qu’il nous a donné.
\VS{24}Celui qui garde ses commandements demeure en Jésus-Christ, et Jésus-Christ demeure en lui ; et par là nous connaissons qu'il demeure en nous, par l'Esprit qu'il nous a donné.
\Chap{4}
\TextTitle{Il faut éprouver les esprits}
\VerseOne{}Mes bien-aimés, ne croyez pas à tout esprit, mais éprouvez les esprits pour savoir s'ils sont de Dieu, car plusieurs faux prophètes sont venus dans le monde.
\TextTitle{[Caractéristiques des faux prophètes
\\a. leur confession sur Jésus-Christ]}
\VS{2}Reconnaissez à cette marque l'Esprit de Dieu : Tout esprit qui confesse que Jésus-Christ est venu en chair est de Dieu.
\VS{3}Et tout esprit qui ne confesse point que Jésus-Christ est venu en chair n'est point de Dieu ; c’est l'esprit de l'Antéchrist, dont vous avez appris la venue, et qui maintenant est déjà dans le monde.
\VS{4}Mes petits-enfants, vous êtes de Dieu, et vous les avez vaincus, parce que celui qui est en vous est plus grand que celui qui est dans le monde.
\TextTitle{b. leur appartenance au monde}
\VS{5}Eux, ils sont du monde, c'est pourquoi ils parlent comme étant du monde, et le monde les écoute.
\VS{6}Nous sommes de Dieu ; celui qui connaît Dieu nous écoute ; mais celui qui n'est pas de Dieu ne nous écoute point ; c’est par là que nous connaissons l'esprit de vérité et l'esprit de l’erreur.
\TextTitle{La charité de Dieu}
\VS{7}Mes bien-aimés, aimons-nous les uns les autres, car la charité est de Dieu ; et quiconque aime son prochain est né de Dieu et connaît Dieu.
\VS{8}Celui qui n'aime point son prochain n'a pas connu Dieu, car Dieu est Charité\FTNT{« Agapé » en grec.}.
\VS{9}La charité de Dieu a été manifestée envers nous en ce que Dieu a envoyé son Fils unique dans le monde, afin que nous vivions par lui.
\VS{10}Et cette charité consiste, non point en ce que nous avons aimé Dieu, mais en ce qu'il nous a aimés, et qu'il a envoyé son Fils pour être la propitiation\FTNT{Du grec «~hilasmos~» qui signifie «~apaisement~». Les écritures nous parlent aussi du «~propitiatoire~», c'est-à-dire « le siège de la misericorde » ou « Lieu de l'expiation ». Le propitiatoire était une plaque en or du sommet de l'Arche de l'Alliance. Le souverain sacrificateur l'aspergeait sept fois, le jour de l'expiation afin de reconcilier symboliquement Yahweh et son peuple. Voir Ex. 25:17-22} pour nos péchés.
\VS{11}Mes bien-aimés, si Dieu nous a ainsi aimés, nous devons aussi nous aimer les uns les autres.
\VS{12}Personne n'a jamais vu Dieu ; si nous nous aimons les uns les autres, Dieu demeure en nous et sa charité est parfaite en nous.
\VS{13}A ceci nous connaissons que nous demeurons en lui, et lui en nous, c'est qu'il nous a donné de son Esprit.
\VS{14}Et nous l'avons vu, et nous témoignons que le Père a envoyé le Fils pour être le sauveur du monde.
\VS{15}Quiconque confessera que Jésus est le Fils de Dieu, Dieu demeure en lui, et lui en Dieu.
\VS{16}Et nous, nous avons connu et cru en la charité que Dieu a pour nous. Dieu est charité ; et celui qui demeure dans la charité, demeure en Dieu, et Dieu en lui.
\VS{17}Tel il est, tels aussi nous sommes dans ce monde : C’est en cela que la charité est parfaite en nous, afin que nous ayons de l’assurance au jour du jugement.
\VS{18}Il n'y a point de crainte dans la charité, mais la parfaite charité bannit la crainte, car la crainte suppose un châtiment ; or celui qui craint n'est pas accompli dans la charité.
\VS{19}Nous l'aimons, parce qu'il nous a aimés le premier.
\VS{20}Si quelqu'un dit : J'aime Dieu, et qu’il haïsse son frère, c’est un menteur ; car comment celui qui n'aime point son frère, qu'il voit, peut-il aimer Dieu, qu’il ne voit pas ?
\VS{21}Et nous avons ce commandement de sa part, que celui qui aime Dieu, aime aussi son frère.
\Chap{5}
\TextTitle{La foi, principe qui triomphe des conflits avec le monde}
\VerseOne{}Quiconque croit que Jésus est le Christ, est né de Dieu, et quiconque aime celui qui l'a engendré, aime aussi celui qui est né de lui.
\VS{2}Nous connaissons à ceci que nous aimons les enfants de Dieu, lorsque nous aimons Dieu et que nous gardons ses commandements.
\VS{3}Car c'est en ceci que consiste notre amour pour Dieu : Que nous gardions ses commandements. Et ses commandements ne sont point pénibles.
\VS{4}Parce que tout ce qui est né de Dieu est victorieux du monde ; et ce qui nous fait remporter la victoire sur le monde, c'est notre foi.
\VS{5}Qui est celui qui a remporté la victoire sur le monde, sinon celui qui croit que Jésus est le Fils de Dieu ?
\VS{6}C'est ce Jésus, le Christ, qui est venu avec l’eau et le sang, et pas seulement avec l'eau, mais avec l'eau et le sang ; et c'est l'Esprit qui rend témoignage, or l'Esprit est la vérité.
\VS{7}Car il y en a trois dans le ciel qui rendent témoignage, le Père, la Parole, et le Saint-Esprit ; et ces trois-là ne sont qu'un\FTNT{Dieu est UN. Voir De. 6:4.}.
\VS{8}Il y en a aussi trois qui rendent témoignage sur la terre, à savoir l'Esprit, l'eau, et le sang, et ces trois-là se rapportent à un.
\TextTitle{Une assurance bénie}
\VS{9}Si nous recevons le témoignage des hommes, le témoignage de Dieu est plus grand, car le témoignage de Dieu consiste en ce qu’il a rendu témoignage à son Fils.
\VS{10}Celui qui croit au Fils de Dieu a le témoignage de Dieu en lui-même ; mais celui qui ne croit pas Dieu, le fait menteur, car il ne croit pas au témoignage que Dieu a rendu de son Fils.
\VS{11}Et c'est ici le témoignage, à savoir que Dieu nous a donné la vie éternelle, et cette vie est dans son Fils.
\VS{12}Celui qui a le Fils a la vie, celui qui n'a pas le Fils de Dieu n'a pas la vie.
\VS{13}Je vous ai écrit ces choses, à vous qui croyez au Nom du Fils de Dieu, afin que vous sachiez que vous avez la vie éternelle, et afin que vous croyiez au Nom du Fils de Dieu.
\VS{14}Et c'est ici l’assurance que nous avons en Dieu, que si nous demandons quelque chose selon sa volonté, il nous exauce.
\VS{15}Et si nous savons qu'il nous exauce, quelque chose que nous demandions, nous savons que nous possédons la chose que nous lui avons demandée.
\VS{16}Si quelqu'un voit son frère commettre un péché qui ne mène point à la mort\FTNT{Le péché qui mène à la mort c’est le blasphème contre le Saint-Esprit. Voir commentaire en Mt. 12:32.}, qu’il prie pour lui, et Dieu donnera la vie à ce frère. Il la donnera à ceux qui commettent un péché qui ne mène point à la mort. Il y a un péché qui mène à la mort ; je ne te dis point de prier pour ce péché-là.
\VS{17}Toute iniquité est un péché, mais il y a quelque péché qui ne mène pas à la mort.
\VS{18}Nous savons que quiconque est né de Dieu ne pèche point ; mais celui qui est engendré de Dieu se garde lui-même, et le malin ne le touche point.
\VS{19}Nous savons que nous sommes nés de Dieu, mais le monde entier est plongé dans le mal.
\TextTitle{Conclusion}
\VS{20}Or nous savons que le Fils de Dieu est venu, et il nous a donné l'intelligence pour connaître le Véritable ; et nous sommes dans le Véritable, en son Fils Jésus-Christ. Il est le vrai\FTNT{Dans Jean 17:3, le Père est présenté comme le Vrai Dieu ; le terme grec traduit par «~vrai~» dans Jean est aussi appliqué à Jésus dans ce passage. Jésus est donc le Vrai Dieu.} Dieu, et la vie éternelle.
\VS{21}Mes petits enfants, gardez-vous des idoles. Amen.
\PPE{}
\end{multicols}

%\clearpage\ShortTitle{2 Jean}\BookTitle{2 Jean}\BFont
\noindent\hrulefill
{\footnotesize
\textit{
\bigskip
{\centering{}
\\Auteur : Jean
\\(Gr. : Ioannes)
\\Signification : Yahweh a fait grâce
\\Thème : Amour et vérité
\\Date de rédaction : Env. 85 ap. J.-C.\\}
}
%\bigskip
\textit{
\\Il semblerait que cette épître était adressée à une église se réunissant chez une personne du nom de Kyria. Jean les invite à demeurer dans la communion avec Dieu et les met en garde contre les hérésies et la fréquentation des faux docteurs.\bigskip
}
}
\par\nobreak\noindent\hrulefill
\begin{multicols}{2}
\Chap{1}
\TextTitle{Introduction}
\VerseOne{}L'ancien à Kyria l’élue, et à ses enfants, que j'aime dans la vérité, et ce n’est pas moi seul qui les aime, mais aussi tous ceux qui ont connu la vérité.
\VS{2}A cause de la vérité qui demeure en nous, et qui sera avec nous éternellement,
\VS{3}que la grâce, la miséricorde, et la paix de la part de Dieu le Père, et de la part du Seigneur Jésus-Christ, le Fils du Père, soient avec vous dans la vérité et dans la charité.
\TextTitle{La marche dans la vérité et dans la charité}
\VS{4}Je me suis fort réjoui d'avoir trouvé quelques-uns de tes enfants qui marchent dans la vérité selon le commandement que nous avons reçu du Père.
\VS{5}Et maintenant, ô Kyria ! Je te prie, non comme t'écrivant un nouveau commandement, mais celui que nous avons eu dès le commencement, que nous ayons de la charité les uns pour les autres.
\VS{6}Et c'est ici la charité, que nous marchions selon ses commandements. Et c'est là son commandement, comme vous l'avez entendu dès le commencement, afin que vous l'observiez.
\TextTitle{Le signe du séducteur et de l'antéchrist}
\VS{7}Car plusieurs séducteurs sont venus dans le monde, qui ne confessent point que Jésus-Christ est venu en chair ; un tel homme est un séducteur et un Antéchrist.
\VS{8}Prenez garde à vous-mêmes, afin que vous ne perdiez point le fruit du travail que vous avez fait, mais que vous en receviez une pleine récompense.
\VS{9}Quiconque transgresse la doctrine de Jésus-Christ et ne lui demeure point fidèle n'a point Dieu ; celui qui demeure dans la doctrine de Christ a le Père et le Fils.
\VS{10}Si quelqu'un vient à vous, et qu'il n'apporte point cette doctrine, ne le recevez point dans votre maison, et ne le saluez pas ;
\VS{11}car celui qui le salue participe à ses mauvaises œuvres.
\TextTitle{Conclusion}
\VS{12}Quoique j’aie plusieurs choses à vous écrire, je n’ai pas voulu les écrire avec du papier et de l'encre, mais j'espère aller vers vous, et vous parler bouche à bouche, afin que notre joie soit parfaite.
\VS{13}Les enfants de ta sœur élue te saluent. Amen !
\PPE{}
\end{multicols}

%\clearpage\ShortTitle{3 Jean}\BookTitle{3 Jean}\BFont
\noindent\hrulefill
{\footnotesize
\textit{
\bigskip
{\centering{}
\\Signifie : Yahweh a fait grâce
\\Thème : Sincérité, hospitalité et caractère chrétien
\\Auteur : Jean
\\Date de rédaction : Env. 85\\}
}
%\bigskip
\textit{
\\Cette épître fut destinée à Gaïus, l’un des responsables d’une église d’Asie Mineure dont Jean loua la piété et la générosité. Il l’avertit de l’orgueil et des agissements de Diotrèphe qui étaient contraires à la Parole mais souligna le bon témoignage de Démétrius.\bigskip
}
}
\par\nobreak\noindent\hrulefill
\begin{multicols}{2}
\TextTitle{[Introduction]}
\Chap{1}
\VerseOne{}L'ancien à Gaïus, le bien-aimé, que j'aime dans la vérité.
\VS{2}Bien-aimé, je souhaite que tu prospères{\FTNT{La prospérité dont il est question dans ce passage n’a rien à voir avec l’Evangile de prospérité qui met l’accent sur la richesse matérielle. Le mot grec «~euodoo~» signifie «~concevoir~» un voyage prospère et diligent~», «~mener par une voie directe et facile~», «~prospérer~», «~être heureux~».}} en toutes choses, et que tu sois en bonne santé, comme ton âme est en prospérité.
\VS{3}Car j’ai été fort réjoui quand les frères sont venus et ont rendu témoignage de ta sincérité, et comment tu marches dans la vérité.
\VS{4}Je n'ai pas de plus grande joie que d’apprendre que mes enfants marchent dans la vérité.
\TextTitle{[L'hospitalité]}
\VS{5}Bien-aimé, tu agis fidèlement dans tout ce que tu fais envers les frères et envers les étrangers,
\VS{6}qui en présence de l'église ont rendu témoignage de ta charité. Et tu feras bien de les accompagner dignement, comme il est séant selon Dieu.
\VS{7}Car ils sont partis pour son Nom, ne prenant rien des gentils.
\VS{8}Nous devons donc recevoir de tels hommes, afin d’être ouvriers avec eux pour la vérité.
\TextTitle{[Les mauvais actes de Diotrèphe et son caractère dominateur]}
\VS{9}J'ai écrit à l'Eglise, mais Diotrèphe, qui aime être le premier parmi eux, ne nous reçoit point.
\VS{10}C'est pourquoi, si je viens, je rappellerai les actions qu'il commet, en tenant contre nous de mauvais discours ; et n'étant pas content de cela, non seulement il ne reçoit pas les frères, mais il empêche même ceux qui veulent les recevoir et les chasse de l'église.
\VS{11}Bien-aimé, n'imite point le mal, mais le bien. Celui qui fait le bien est de Dieu ; mais celui qui fait le mal n'a point vu Dieu.
\TextTitle{[Témoignage de Démétrius]}
\VS{12}Tous rendent témoignage à Démétrius, et la vérité même le lui rend, et nous aussi nous lui rendons témoignage, et vous savez que notre témoignage est véritable.
\TextTitle{[Conclusion]}
\VS{13}J'avais plusieurs choses à écrire, mais je ne veux pas t'écrire avec l'encre et avec la plume.
\VS{14}Mais j'espère te voir bientôt, et nous parlerons de bouche à bouche.
\VS{15}Que la paix soit avec toi ! Les amis te saluent. Salue les amis, chacun par son nom.
\PPE{}
\end{multicols}

%\clearpage\ShortTitle{Ap.}\BookTitle{Apocalypse}\BFont
\noindent\hrulefill
{\footnotesize
\textit{
\bigskip
{\centering{}
\\Auteur~: Jean
\\Thème~: L'aboutissement de toutes choses
\\(Gr.~: Apokalupsis)
\\Signification~: Mettre à nu, révélation d'une vérité, action de révéler
\\Date de rédaction~: Env. 95 ap. J.-C.\\}
}
\textit{
\\Le terme apocalypse, du grec «~apokalupsis~», évoque «~l'action de révéler ce qui était caché ou inconnu~». Ce mot a pour racine «~apokalupto~» qui signifie aussi «~découvrir, dévoiler ce qui est voilé ou recouvert~».
\\C'est à Patmos, île grecque de la mer Egée - où il s'exila en raison de la persécution de l'empereur Domitien (51 - 96 ap. J.C.) - que Jean reçut une révélation de Jésus-Christ ainsi qu'un message s'adressant aux «~sept églises~» qui constituaient certainement les villes de l'Asie Mineure où se trouvaient les principales concentrations de chrétiens. Si Ephèse figure dans les écrits de la Nouvelle Alliance et que Thyatire et Laodicée y sont brièvement mentionnées, les quatre autres églises - qu'on ne retrouve nulle part ailleurs dans les Ecritures - étaient sans doute le fruit du travail missionnaire de Paul. Les sept lettres s'adressent à l'ange de chacune de ces assemblées locales, autrement dit aux messagers de celles-ci (probablement un ancien ou un responsable).
\\Ce livre, qui arrive en conclusion des Ecritures, annonce les événements qui doivent précéder la fin de l'histoire de l'humanité.\bigskip
}
} 
\par\nobreak\noindent\hrulefill
\begin{multicols}{2}
\Chap{1}
\TextTitle{Introduction}
\VerseOne{}La révélation\FTNT{«~Apokalupsis~» en grec. Voir l'introduction du livre.} de Jésus-Christ, que Dieu lui a donnée pour montrer à ses serviteurs les choses qui doivent arriver bientôt, et qui les a fait connaître en les envoyant par son ange à Jean, son serviteur,
\VS{2}qui a annoncé la parole de Dieu, et le témoignage de Jésus-Christ, et toutes les choses qu'il a vues.
\VS{3}Béni est celui qui lit et ceux qui écoutent les paroles de cette prophétie, et qui gardent les choses qui y sont écrites~! Car le temps est proche.
\TextTitle{Jésus-Christ}
\VS{4}Jean aux sept églises qui sont en Asie~: Que la grâce et la paix vous soient données de la part de celui QUI EST, QUI ETAIT, et QUI VIENT\FTNT{Les prophètes ont prophétisé la venue de Yahweh en personne~: Es. 35:4~; Es. 40:10-11~; Es. 60:1-2~; Za. 14:1-21~; Jn. 14:1-3. Jésus-Christ est bien Yahweh qui vient.}, et de la part des sept Esprits qui sont devant son trône,
\VS{5}et de la part de Jésus-Christ, qui est le témoin fidèle, le premier-né d'entre les morts\FTNT{Voir commentaire Col. 1:15.}, et le Prince des rois de la terre.
\VS{6}A lui, dis-je, qui nous a aimés, et qui nous a lavés de nos péchés dans son sang, et qui a fait de nous des rois et des prêtres pour Dieu, son Père, à lui soient la gloire et la force aux siècles des siècles. Amen~!
\TextTitle{La venue du Christ}
\VS{7}Voici, il vient avec les nuées, et tout œil le verra, et même ceux qui l'ont percé~; et toutes les tribus de la terre se lamenteront devant lui. Oui, amen~!
\VS{8}Je suis l'Alpha et l'Oméga, le commencement et la fin, dit le Seigneur, QUI EST, QUI ETAIT, et QUI VIENT, le Tout-Puissant.
\TextTitle{La vision doit être écrite}
\VS{9}Moi Jean, qui suis aussi votre frère et qui participe à la tribulation, au règne, et à la patience de Jésus-Christ, j'étais sur l'île appelée Patmos à cause de la parole de Dieu, et du témoignage de Jésus-Christ.
\VS{10}Je fus ravi en esprit au jour du Seigneur, et j'entendis derrière moi une voix forte, comme le son d'une trompette,
\VS{11}qui disait~: Je suis l'Alpha et l'Oméga, le premier et le dernier. Ecris dans un livre ce que tu vois, et envoie-le aux sept églises qui sont en Asie, à savoir à Ephèse, à Smyrne, à Pergame, à Thyatire, à Sardes, à Philadelphie, et à Laodicée.
\VS{12}Alors je me retournai pour voir celui dont la voix m'avait parlé, et après m'être retourné, je vis sept chandeliers d'or,
\VS{13}et au milieu des sept chandeliers d'or, quelqu'un qui ressemblait à un fils d'homme, vêtu d'une longue robe, et ayant une ceinture d'or sur la poitrine.
\VS{14}Sa tête et ses cheveux étaient blancs comme de la laine blanche, et comme de la neige, et ses yeux étaient comme une flamme de feu.
\VS{15}Ses pieds étaient semblables à de l'airain ardent, comme s'ils avaient été embrasés dans une fournaise~; et sa voix était comme le bruit des grandes eaux.
\VS{16}Et il avait dans sa main droite sept étoiles, et de sa bouche sortait une épée aiguë à deux tranchants, et son visage était semblable au soleil lorsqu'il brille dans sa force.
\VS{17}Quand je le vis, je tombai à ses pieds comme mort, et il mit sa main droite sur moi, en me disant~: Ne crains pas~!
\VS{18}Je suis le premier et le dernier, et je vis~; j'étais mort, et voici, je suis vivant aux siècles des siècles. Amen~! Et je tiens les clefs de Hadès\FTNT{Voir commentaire Mt. 16:18.} et de la mort.
\VS{19}Ecris les choses que tu as vues, celles qui sont présentement, et celles qui doivent arriver ensuite.
\VS{20}Le mystère des sept étoiles que tu as vues dans ma main droite, et les sept chandeliers d'or. Les sept étoiles sont les anges des sept églises~; et les sept chandeliers que tu as vus sont les sept églises.
\Chap{2}
\TextTitle{Ephèse~: L'église qui a perdu le premier amour}
\VerseOne{}Ecris à l'ange\FTNT{Ange, du grec «~aggelos~»~: Envoyé, messager, un ange. Un messager de Dieu. Ce terme sert à désigner aussi bien les créatures spirituelles que les êtres humains.} de l'église d'Ephèse~: Voici ce que dit celui qui tient les sept étoiles dans sa main droite, et qui marche au milieu des sept chandeliers d'or~:
\VS{2}Je connais tes œuvres, et ton travail, et ta patience, et je sais que tu ne peux pas supporter les méchants, et que tu as éprouvé ceux qui se disent être apôtres et qui ne le sont pas, et que tu les as trouvés menteurs~;
\VS{3}et que tu as souffert, et que tu as eu de la patience, et que tu as travaillé pour mon Nom, et que tu ne t'es pas lassé.
\VS{4}Mais j'ai quelque chose contre toi, c'est que tu as abandonné ta première charité\FTNT{Il est question ici de l'amour «~agape~»~: L'amour fraternel, l'amour divin.}.
\VS{5}C'est pourquoi souviens-toi donc d'où tu es tombé, repens-toi, et fais tes premières œuvres. Autrement, je viendrai à toi à toute vitesse, et j'ôterai ton chandelier de sa place si tu ne te repens pas.
\VS{6}Mais pourtant tu as ceci de bon, c'est que tu hais les œuvres des Nicolaïtes\FTNT{Nicolaïtes~: Tiré du nom Nicolas, qui signifie littéralement «~victorieux du peuple~». Il s'agit d'une secte dont les membres furent peut-être des disciples d'un certain Nicolas, l'un des diacres de l'église d'Antioche qui aurait dévié (Ac. 6:5). Ces derniers suivaient la doctrine de Balaam, enseignant aux chrétiens qu'à cause du principe de liberté, ils pouvaient manger des viandes sacrifiées aux idoles et commettre des actes immoraux comme les Gentils.}, œuvres que je hais moi aussi.
\VS{7}Que celui qui a des oreilles entende ce que l'Esprit dit aux églises~! A celui qui vaincra, je lui donnerai à manger de l'arbre de vie, qui est au milieu du paradis de Dieu.
\TextTitle{Smyrne~: L'église sous la persécution}
\VS{8}Ecris aussi à l'ange de l'église de Smyrne~: Voici ce que dit celui qui est le premier et le dernier, qui a été mort, et qui est revenu à la vie~:
\VS{9}Je connais tes œuvres, ton affliction et ta pauvreté, quoique tu sois riche, et le blasphème de ceux qui se disent être Juifs et qui ne le sont pas, mais qui sont la synagogue de Satan.
\VS{10}Ne crains rien des choses que tu as à souffrir. Voici, il arrivera que le diable mettra quelques-uns d'entre vous en prison, afin que vous soyez éprouvés~; et vous aurez une affliction de dix jours. Sois fidèle jusqu'à la mort, et je te donnerai la couronne de vie.
\VS{11}Que celui qui a des oreilles, entende ce que l'Esprit dit aux églises~! Celui qui vaincra n'aura pas à souffrir la seconde mort.
\TextTitle{Pergame~: L'église établie dans le monde}
\VS{12}Ecris aussi à l'ange de l'église de Pergame~: Voici ce que dit celui qui a l'épée aiguë à deux tranchants\FTNT{Hé. 4:12.}~: 
\VS{13}Je connais tes œuvres, et le lieu où tu habites, à savoir là où est le trône de Satan. Et que cependant tu retiens mon Nom, et tu n'as pas renié ma foi, même aux jours d'Antipas, mon fidèle martyr\FTNT{Du grec «~martus~» qui signifie «~témoin~».}, qui a été mis à mort chez vous, là où Satan habite.
\VS{14}Mais j'ai quelque chose contre toi, c'est que tu as là des gens attachés à la doctrine de Balaam, qui enseignait à Balak à mettre un scandale devant les enfants d'Israël, afin qu'ils mangent des viandes sacrifiées aux idoles, et qu'ils se livrent à la fornication\FTNT{No. 25:1-2~; No. 31:16.}.
\VS{15}De même, toi aussi tu as des gens attachés à la doctrine des Nicolaïtes~; ce que je hais~!
\VS{16}Repens-toi donc, autrement je viendrai à toi à toute vitesse, et je les combattrai avec l'épée de ma bouche.
\VS{17}Que celui qui a des oreilles, entende ce que l'Esprit dit aux églises~! A celui qui vaincra, je lui donnerai à manger de la manne qui est cachée, et je lui donnerai un caillou blanc, et sur ce caillou sera écrit un nouveau nom, que nul ne connaît, sinon celui qui le reçoit.
\TextTitle{Thyatire~: L'église en temps d'idolâtrie}
\VS{18}Ecris aussi à l'ange de l'église de Thyatire~: Voici ce que dit le Fils de Dieu, qui a ses yeux comme une flamme de feu, et dont les pieds sont semblables à de l'airain ardent.
\VS{19}Je connais tes œuvres, ta charité, ton service, ta foi, ta patience, et que tes dernières œuvres surpassent les premières.
\VS{20}Mais j'ai quelque peu de chose contre toi, c'est que tu laisses cette femme Jézabel\FTNT{1 R. 16:31~; 1 R. 21:25~; 2 R. 9:7~; 2 R. 9:22.}, qui se dit prophétesse, enseigner et séduire mes serviteurs pour les porter à la fornication, et leur faire manger des choses sacrifiées aux idoles.
\VS{21}Et je lui ai donné du temps, afin qu'elle se repente de sa prostitution, mais elle ne s'est pas repentie.
\VS{22}Voici, je vais la jeter sur un lit, et mettre dans une grande affliction ceux qui commettent l'adultère avec elle, s'ils ne se repentent pas de leurs œuvres.
\VS{23}Et je ferai mourir de mort ses enfants~; et toutes les églises connaîtront que je suis celui qui sonde les reins et les cœurs, et je rendrai à chacun de vous selon ses œuvres.
\VS{24}Mais je vous dis à vous et aux autres qui sont à Thyatire, à tous ceux qui n'ont pas cette doctrine, et qui n'ont pas connu les profondeurs de Satan, comme ils disent, je vous dis~: Je ne mettrai pas sur vous d'autre charge.
\VS{25}Mais retenez ce que vous avez, jusqu'à ce que je vienne.
\VS{26}Car à celui qui aura vaincu, et qui aura gardé mes œuvres jusqu'à la fin, je lui donnerai autorité sur les nations.
\VS{27}Et il les gouvernera avec un sceptre de fer, et elles seront brisées comme les vases d'un potier, ainsi que j'en ai moi-même reçu le pouvoir de mon Père.
\VS{28}Et je lui donnerai l'étoile du matin.
\VS{29}Que celui qui a des oreilles entende ce que l'Esprit dit aux églises~!
\Chap{3}
\TextTitle{Sardes~: L'église morte}
\VerseOne{}Ecris aussi à l'ange de l'église de Sardes~: Voici ce que dit celui qui a les sept Esprits de Dieu, et les sept étoiles~: Je connais tes œuvres. Tu as la réputation d'être vivant, mais tu es mort.
\VS{2}Sois vigilant, et affermis le reste qui va mourir~; car je n'ai pas trouvé tes œuvres parfaites devant Dieu.
\VS{3}Souviens-toi donc des choses que tu as reçues et entendues, garde-les, et repens-toi. Si tu ne veilles pas, je viendrai contre toi comme un voleur, et tu ne sauras pas à quelle heure je viendrai contre toi\FTNT{Mt. 24:43~; Lu. 12:39~; 1 Th. 5:2~; 2 Pi. 3:10.}.
\VS{4}Toutefois, tu as quelque peu de personnes à Sardes qui n'ont pas souillé leurs vêtements, et qui marcheront avec moi en vêtements blancs, car ils en sont dignes.
\VS{5}Celui qui vaincra sera vêtu de vêtements blancs, et je n'effacerai pas son nom du Livre de vie, mais je confesserai son nom devant mon Père, et devant ses anges.
\VS{6}Que celui qui a des oreilles, entende ce que l'Esprit dit aux églises~!
\TextTitle{Philadelphie~: L'église réveillée et fidèle}
\VS{7}Ecris aussi à l'ange de l'église de Philadelphie~: Voici ce que dit le Saint et le Véritable, qui a la clef de David, qui ouvre et nul ne ferme, qui ferme et nul n'ouvre.
\VS{8}Je connais tes œuvres. Voici, j'ai ouvert une porte devant toi, et personne ne peut la fermer~; parce que tu as peu de puissance, que tu as gardé ma parole, et que tu n'as pas renié mon Nom.
\VS{9}Voici, je ferai venir ceux de la synagogue de Satan qui se disent Juifs, et ne le sont pas, mais qui mentent~; voici, dis-je, je les ferai venir et se prosterner à tes pieds, et ils connaîtront que je t'aime.
\VS{10}Parce que tu as gardé la parole de ma persévérance, je te garderai aussi de l'heure de la tentation qui doit arriver dans le monde entier, pour éprouver les habitants de la terre.
\VS{11}Voici, je viens à toute vitesse. Tiens ferme ce que tu as, afin que personne ne t'enlève ta couronne.
\VS{12}Celui qui vaincra, je ferai de lui une colonne dans le temple de mon Dieu, et il n'en sortira plus~; et j'écrirai sur lui le Nom de mon Dieu, et le nom de la cité de mon Dieu, qui est la nouvelle Jérusalem qui descend du ciel d'auprès de mon Dieu, et mon nouveau Nom.
\VS{13}Que celui qui a des oreilles entende ce que l'Esprit dit aux églises~!
\TextTitle{Laodicée~: L'église apostate}
\VS{14}Ecris aussi à l'ange de l'église de Laodicée~: Voici ce que dit l'Amen, le témoin fidèle et véritable, le commencement de la création de Dieu~:
\VS{15}Je connais tes œuvres. Je sais que tu n'es ni froid ni bouillant~; puisses-tu être ou froid ou bouillant~!
\VS{16}Parce que tu es tiède, et que tu n'es ni froid ni bouillant, je te vomirai de ma bouche.
\VS{17}Car tu dis~: Je suis riche, je suis dans l'abondance, et je n'ai besoin de rien~; mais tu ne sais pas que tu es malheureux, misérable, pauvre, aveugle et nu.
\VS{18}Je te conseille d'acheter de moi de l'or éprouvé par le feu, afin que tu deviennes riche; et des vêtements blancs, afin que tu sois vêtu et que la honte de ta nudité ne paraisse pas~; et d'oindre tes yeux de collyre, afin que tu voies.
\VS{19}Moi, je reprends et je châtie\FTNT{De. 8:5~; 2 S. 7:14~; Pr. 13:24~; Hé. 12:7.} tous ceux que j'aime. Aie donc du zèle et repens-toi.
\TextTitle{Le Messie se retrouve hors des églises apostates}
\VS{20}Voici, je me tiens à la porte, et je frappe. Si quelqu'un entend ma voix et m'ouvre la porte, j'entrerai chez lui, et je souperai avec lui, et lui avec moi.
\VS{21}Celui qui vaincra, je le ferai asseoir avec moi sur mon trône, ainsi que j'ai vaincu et me suis assis avec mon Père sur son trône.
\VS{22}Que celui qui a des oreilles entende ce que l'Esprit dit aux églises~!
\Chap{4}
\TextTitle{Vision avant l'ouverture des sceaux}
\VerseOne{}Après ces choses, je regardai, et voici une porte était ouverte dans le ciel. Et la première voix que j'avais entendue, comme le son d'une trompette, et qui parlait avec moi, me dit~: Monte ici, et je te montrerai les choses qui doivent arriver à l'avenir.
\VS{2}Aussitôt, je fus ravi en esprit. Et voici, un trône était dressé dans le ciel, et sur ce trône, quelqu'un était assis.
\VS{3}Et celui qui y était assis était semblable à une pierre de jaspe et de sardoine~; et le trône était environné d'un arc-en-ciel semblable à de l'émeraude.
\TextTitle{Les trônes des vingt-quatre anciens}
\VS{4}Et il y avait autour du trône vingt-quatre trônes et je vis sur ces trônes vingt-quatre anciens assis, vêtus de vêtements blancs, et ayant sur leurs têtes des couronnes d'or.
\VS{5}Et du trône sortaient des éclairs, des tonnerres, et des voix~; et il y avait devant le trône sept lampes de feu ardentes, qui sont les sept Esprits de Dieu.
\TextTitle{Le Messie est digne de recevoir la louange et la gloire}
\VS{6}Et devant le trône, il y avait une mer de verre semblable à du cristal~; et au milieu du trône et autour du trône quatre animaux, pleins d'yeux devant et derrière.
\VS{7}Et le premier animal était semblable à un lion~; le second animal était semblable à un veau~; le troisième animal avait la face comme un homme~; et le quatrième animal était semblable à un aigle qui vole.
\VS{8}Et les quatre animaux avaient chacun six ailes, et tout autour et au-dedans ils étaient pleins d'yeux~; et ils ne cessent pas de dire jour et nuit~: Saint~! Saint~! Saint est le Seigneur Dieu Tout-puissant, QUI ETAIT, QUI EST, et QUI VIENT.
\VS{9}Et quand ces animaux rendaient gloire et honneur et des actions de grâces à celui qui était assis sur le trône, à celui qui est vivant aux siècles des siècles,
\VS{10}les vingt-quatre anciens se prosternaient devant celui qui était assis sur le trône, et adoraient celui qui est vivant aux siècles des siècles, et ils jetaient leurs couronnes devant le trône, en disant~:
\VS{11}Seigneur, tu es digne de recevoir gloire, honneur et puissance~; car tu as créé toutes choses, et c'est par ta volonté qu'elles existent et qu'elles ont été créées.
\Chap{5}
\TextTitle{Le Messie est le seul digne d'ouvrir le livre}
\VerseOne{}Puis je vis dans la main droite de celui qui était assis sur le trône, un livre écrit en dedans et en dehors, scellé de sept sceaux.
\VS{2}Et je vis aussi un ange remarquable par sa force, qui proclamait d'une voix forte~: Qui est digne d'ouvrir le livre, et d'en rompre les sceaux~?
\VS{3}Et il n'y avait personne, ni dans le ciel, ni sur la terre, ni sous la terre qui pouvait ouvrir le livre, ni le regarder.
\VS{4}Et je pleurais beaucoup parce que personne n'était trouvé digne d'ouvrir le livre, ni de le lire, ni de le regarder.
\VS{5}Et l'un des anciens me dit~: Ne pleure pas, voici le Lion qui vient de la tribu de Juda, de la racine de David, a vaincu pour ouvrir le livre et pour en rompre les sept sceaux.
\VS{6}Et je regardai, et voici il y avait au milieu du trône et des quatre animaux, et au milieu des anciens, un Agneau qui se tenait là comme immolé, ayant sept cornes, et sept yeux, qui sont les sept Esprits de Dieu envoyés par toute la terre.
\VS{7}Et il vint et prit le livre de la main droite de celui qui était assis sur le trône.
\TextTitle{L'Agneau est adoré\FTNTT{Ph. 2:9-11.}}
\VS{8}Et quand il eut pris le livre, les quatre animaux et les vingt-quatre anciens se prosternèrent devant l'Agneau, ayant chacun des harpes et des coupes d'or pleines de parfums, qui sont les prières des saints.
\VS{9}Et ils chantaient un cantique nouveau, en disant~: Tu es digne de prendre le livre, et d'en ouvrir les sceaux~; car tu as été mis à mort, et tu nous as rachetés pour Dieu par ton sang, de toute tribu, de toute langue, de tout peuple, et de toute nation~;
\VS{10}et tu as fait de nous des rois et des prêtres pour notre Dieu~; et nous régnerons sur la terre.
\VS{11}Puis je regardai, et j'entendis la voix de plusieurs anges autour du trône, et des anciens; et leur nombre était de plusieurs millions.
\VS{12}Et ils disaient à haute voix~: L'Agneau qui a été mis à mort est digne de recevoir puissance, richesses, sagesse, force, honneur, gloire et louange.
\VS{13}J'entendis aussi toutes les créatures qui sont dans le ciel, sur la terre, et sous la terre, et dans la mer, et toutes les choses qui y sont, disant~: A celui qui est assis sur le trône et à l'Agneau, soient louange, honneur, gloire, et force, aux siècles des siècles~!
\VS{14}Et les quatre animaux disaient~: Amen~! Et les vingt-quatre anciens se prosternèrent et adorèrent celui qui est vivant aux siècles des siècles.
\Chap{6}
\TextTitle{Premier sceau~: Le cavalier qui part pour vaincre}
\VerseOne{}Et quand l'Agneau eut ouvert l'un des sceaux, je regardai, et j'entendis l'un des quatre animaux qui disait comme avec une voix de tonnerre~: Viens, et vois.
\VS{2}Je regardai, et je vis un cheval blanc~; celui qui était monté dessus avait un arc, et il lui fut donné une couronne~; et il est sortit en vainqueur pour vaincre\FTNT{Contrairement aux apparences, ce cavalier couronné d'un diadème et qui monte un cheval blanc n'est pas Jésus-Christ, mais l'Antichrist qui singe le retour glorieux du Seigneur~: Da. 7:21~; Mt. 24:4-5~; 2 Th. 2:9-12~; Ap. 13:7. Le vrai Christ revenant triomphalement avec son Eglise est décrit en Ap. 19:11-16.}.
\TextTitle{Deuxième sceau~: La guerre}
\VS{3}Et quand il eut ouvert le second sceau, j'entendis le second animal qui disait~: Viens, et vois.
\VS{4}Et il sortit un autre cheval qui était roux~; il fut donné à celui qui était monté dessus de pouvoir ôter la paix de la terre, afin que les hommes se tuent les uns les autres~; et il lui fut donné une grande épée.
\TextTitle{Troisième sceau~: La famine}
\VS{5}Et quand il eut ouvert le troisième sceau, j'entendis le troisième animal qui disait~: Viens, et vois. Je regardai, et je vis un cheval noir, et celui qui était monté dessus avait une balance dans sa main.
\VS{6}Et j'entendis au milieu des quatre animaux une voix qui disait~: Une mesure de blé pour un denier, et les trois mesures d'orge pour un denier~; mais ne fais pas de mal au vin et à l'huile.
\TextTitle{Quatrième sceau~: La mort}
\VS{7}Et quand il eut ouvert le quatrième sceau, j'entendis la voix du quatrième animal qui disait~: Viens, et vois.
\VS{8}Je regardai, et je vis un cheval verdâtre~; et celui qui était monté dessus se nommait la Mort, et le Hadès l'accompagnait. Il leur fut donné le pouvoir sur le quart de la terre pour tuer par l'épée, par la famine, par la mortalité, et par les bêtes sauvages de la terre.
\TextTitle{Cinquième sceau~: Les martyrs}
\VS{9}Et quand il eut ouvert le cinquième sceau, je vis sous l'autel les âmes de ceux qui avaient été tués pour la parole de Dieu, et pour le témoignage qu'ils avaient gardé.
\VS{10}Et elles criaient à haute voix, disant~: Jusqu'à quand, Seigneur qui es saint et véritable, ne jugeras-tu pas et ne vengeras-tu pas notre sang de ceux qui habitent sur la terre~?
\VS{11}Et il leur fut donné à chacun des robes blanches, et il leur fut dit de se tenir en repos encore un peu de temps, jusqu'à ce que le nombre de leurs compagnons de service, et de leurs frères qui doivent être mis à mort comme eux, soit complet.
\TextTitle{Sixième sceau~: L'anarchie}
\VS{12}Et je regardai quand il eut ouvert le sixième sceau, et voici, il se fit un grand tremblement de terre, et le soleil devint noir comme un sac de crin, et la lune entière devint comme du sang.
\VS{13}Et les étoiles du ciel tombèrent sur la terre\FTNT{Mt. 24:29~; Mc. 13:25.}, comme lorsque le figuier est agité par un grand vent et laisse tomber ses figues encore vertes.
\VS{14}Et le ciel se retira comme un livre qu'on roule~; et toutes les montagnes et les îles furent remuées de leurs places.
\VS{15}Et les rois de la terre, et les princes, et les riches, et les capitaines, et les puissants, et tout esclave, et tout homme libre se cachèrent dans les cavernes et entre les rochers des montagnes.
\VS{16}Et ils disaient aux montagnes et aux rochers~: Tombez sur nous\FTNT{Lu. 23:30.}, et cachez-nous devant la face de celui qui est assis sur le trône, et devant la colère de l'Agneau~;
\VS{17}car le grand jour de sa colère est venu, et qui peut subsister~?
\Chap{7}
\TextTitle{Les 144 000 marqués du sceau de Dieu}
\VerseOne{}Après cela, je vis quatre anges qui se tenaient aux quatre coins de la terre, et qui retenaient les quatre vents de la terre, afin qu'ils ne soufflent pas sur la terre, ni sur la mer, ni sur aucun arbre.
\VS{2}Puis je vis un autre ange qui montait du côté de l'orient, tenant le sceau du Dieu vivant, et il cria d'une voix forte aux quatre anges à qui il avait été donné de faire du mal à la terre et à la mer,
\VS{3}et leur dit~: Ne faites pas de mal à la terre, ni à la mer, ni aux arbres, jusqu'à ce que nous ayons marqué du sceau les serviteurs de notre Dieu sur leurs fronts.
\VS{4}Et j'entendis que le nombre de ceux qui avaient été marqués du sceau était de cent quarante-quatre mille, de toutes les tribus des enfants d'Israël.
\VS{5}de la tribu de Juda, douze mille marqués du sceau~; de la tribu de Ruben, douze mille marqués du sceau~; de la tribu de Gad, douze mille marqués du sceau~;
\VS{6}de la tribu d'Aser, douze mille marqués du sceau~; de la tribu de Nephthali, douze mille marqués du sceau~; de la tribu de Manassé, douze mille marqués du sceau~;
\VS{7}de la tribu de Siméon, douze mille marqués du sceau~; de la tribu de Lévi, douze mille marqués du sceau~; de la tribu d'Issacar, douze mille marqués du sceau~;
\VS{8}de la tribu de Zabulon, douze mille marqués du sceau~; de la tribu de Joseph, douze mille marqués du sceau~; de la tribu de Benjamin, douze mille marqués du sceau.
\TextTitle{Multitude de sauvés pendant la grande tribulation}
\VS{9}Après cela, je regardai, et voici une grande multitude de gens, que personne ne pouvait compter, de toute nation, de toute tribu, de tout peuple et de toute langue, se tenaient devant le trône, et devant l'Agneau, vêtus de longues robes blanches, et ils avaient des palmes dans leurs mains.
\VS{10}Et ils criaient d'une voix forte, en disant~: Le salut est à notre Dieu, qui est assis sur le trône, et à l'Agneau.
\VS{11}Et tous les anges se tenaient autour du trône, et des anciens, et des quatre animaux, et ils se prosternèrent devant le trône sur leurs faces et adorèrent Dieu,
\VS{12}en disant~: Amen~! La louange, la gloire, la sagesse, les actions de grâces, l'honneur, la puissance et la force soient à notre Dieu, aux siècles des siècles. Amen~!
\VS{13}Et l'un des anciens prit la parole et me dit~: Ceux qui sont revêtus de longues robes blanches, qui sont-ils et d'où sont-ils venus~?
\VS{14}Et je lui dis~: Seigneur, tu le sais. Et il me dit~: Ce sont ceux qui sont venus de la grande tribulation\FTNT{Les saints ont toujours été persécutés. Cela a débuté dès la Genèse avec Caïn qui tua son frère Abel (Ge. 4:5-10). La grande tribulation correspond néanmoins à une période de persécutions particulièrement cruelles qui seront orchestrées par l'homme impie (à la tête de plusieurs nations) principalement contre les Juifs (Jé. 30:7~; Da. 9:24~; Lu. 21:20-24) et sans doute contre les personnes converties à Christ issues des nations (Ap. 7:9-17~; Ap. 12:17). Le Seigneur Jésus a prédit la grande tribulation à ses disciples (Mt. 24:15-29~; Mc. 13:14-19) en précisant qu'en ce temps là on verrait «~l'abomination de la désolation~» établie en lieu saint prophétisée par Daniel (Da. 11:31). La grande tribulation durera trois ans et demi, c'est ce que Daniel appelle «~un temps, des temps et la moitié d'un temps~» (Da. 7:25~; Ap. 11:3.) L'ère de paix factice instaurée par l'impie cédera alors soudainement la place à un temps d'angoisse sans précédent (1 Th. 5:3).}, et qui ont lavé et blanchi leurs longues robes dans le sang de l'Agneau.
\VS{15}C'est pourquoi ils sont devant le trône de Dieu, et ils le servent jour et nuit dans son temple~; et celui qui est assis sur le trône habitera avec eux.
\VS{16}Ils n'auront plus faim ni soif, et le soleil ne les frappera plus, ni aucune chaleur.
\VS{17}Car l'Agneau qui est au milieu du trône les paîtra, et les conduira aux sources des eaux de la vie, et Dieu essuiera toutes les larmes de leurs yeux.
\Chap{8}
\TextTitle{Septième sceau~: Annonce des sept trompettes\FTNTT{Ap. 4:1.}}
\VerseOne{}Et quand il eut ouvert le septième sceau, il y eut un silence dans le ciel d'environ une demi-heure.
\VS{2}Et je vis les sept anges qui se tiennent devant Dieu, et sept trompettes leur furent données.
\VS{3}Et un autre ange vint et se tint devant l'autel, ayant un encensoir d'or, et plusieurs parfums lui furent donnés pour les offrir, avec les prières de tous les saints, sur l'autel d'or qui est devant le trône.
\VS{4}Et la fumée des parfums monta avec les prières des saints de la main de l'ange devant Dieu.
\VS{5}Puis l'ange prit l'encensoir, et l'ayant rempli du feu de l'autel, il le jeta sur la terre~; et il y eut des tonnerres, des voix, des éclairs, et un tremblement de terre.
\VS{6}Alors les sept anges qui avaient les sept trompettes se préparèrent à en sonner.
\TextTitle{Première trompette~: Grêle et feu mêlés de sang}
\VS{7}Et le premier ange sonna de la trompette. Et il y eut de la grêle et du feu mêlés de sang, qui furent jetés sur la terre~; et le tiers des arbres fut brûlé, et toute herbe verte aussi fut brûlée.
\TextTitle{Deuxième trompette~: La montagne embrasée}
\VS{8}Et le second ange sonna de la trompette, et je vis comme une grande montagne embrasée de feu, qui fut jetée dans la mer~; et le tiers de la mer devint du sang,
\VS{9}et le tiers des créatures vivantes qui étaient dans la mer mourut, et le tiers des navires périt.
\TextTitle{Troisième trompette~: Absinthe, l'étoile tombée du ciel}
\VS{10}Et le troisième ange sonna de la trompette, et il tomba du ciel une grande étoile ardente comme un flambeau, et elle tomba sur le tiers des fleuves et sur les sources des eaux.
\VS{11}Le nom de l'étoile est Absinthe~; et le tiers des eaux fut changé en absinthe, et beaucoup d'hommes moururent par les eaux, parce qu'elles étaient devenues amères.
\TextTitle{Quatrième trompette~: Des signes dans le ciel}
\VS{12}Puis le quatrième ange sonna de la trompette, et le tiers du soleil fut frappé, ainsi que le tiers de la lune, et le tiers des étoiles, afin que le tiers en soit obscurci~; le jour fut privé d'un tiers de sa clarté, et la nuit de même.
\VS{13}Je regardai, et j'entendis un ange qui volait au milieu du ciel et qui disait à haute voix~: Malheur~! Malheur~! Malheur aux habitants de la terre à cause des autres sons de trompettes que les trois autres anges vont faire retentir.
\Chap{9}
\TextTitle{Cinquième trompette~: Ouverture du puits de l'abîme}
\VerseOne{}Le cinquième ange sonna de la trompette, et je vis une étoile qui tomba du ciel sur la terre, et la clef du puits de l'abîme fut donnée à cet ange.
\VS{2}Et il ouvrit le puits de l'abîme, et une fumée monta du puits comme la fumée d'une grande fournaise~; et le soleil et l'air furent obscurcis par la fumée du puits.
\VS{3}Des sauterelles sortirent de la fumée du puits et se répandirent sur la terre, et il leur fut donné un pouvoir comme le pouvoir qu'ont les scorpions de la terre.
\VS{4}Et il leur fut dit de ne pas faire de mal à l'herbe de la terre, ni à aucune verdure, ni à aucun arbre, mais seulement aux hommes qui n'avaient pas la marque de Dieu sur leurs fronts.
\VS{5}Et il leur fut donné, non de les tuer, mais de les tourmenter pendant cinq mois~; et le tourment qu'elles causaient était comme le tourment que cause le scorpion quand il pique un homme.
\VS{6}Et en ces jours-là, les hommes chercheront la mort, mais ils ne la trouveront pas~; et ils désireront mourir, mais la mort fuira loin d'eux.
\VS{7}Ces sauterelles ressemblaient à des chevaux préparés pour la guerre, et sur leurs têtes il y avait comme des couronnes semblables à de l'or, et leurs faces étaient comme des faces d'hommes.
\VS{8}Elles avaient les cheveux comme des cheveux de femmes~; et leurs dents étaient comme des dents de lions.
\VS{9}Elles avaient des cuirasses comme des cuirasses de fer~; et le bruit de leurs ailes était comme le bruit des chars à plusieurs chevaux qui courent à la guerre.
\VS{10}Elles avaient des queues armées d'aiguillons, comme les scorpions, et c'est dans leurs queues qu'était le pouvoir de faire du mal aux hommes pendant cinq mois.
\VS{11}Elles avaient sur elles comme roi l'ange de l'abîme, dont le nom en hébreu est Abaddon, mais en grec son nom est Apollyon\FTNT{Abaddon ou Apollyon~: Le nom de ce démon signifie «~Le destructeur~».}.
\VS{12}Le premier malheur est passé, et voici venir encore deux malheurs après celui-ci.
\TextTitle{Sixième trompette~: Les quatre anges de l'Euphrate déliés\FTNTT{Ap. 16:12.}}
\VS{13}Alors le sixième ange sonna de sa trompette, et j'entendis une voix sortant des quatre cornes de l'autel d'or qui est devant Dieu,
\VS{14}et disant au sixième ange qui avait la trompette~: Délie les quatre anges qui sont liés sur le grand fleuve, l'Euphrate.
\VS{15}On délia donc les quatre anges qui étaient prêts pour l'heure, le jour, le mois et l'année, afin de tuer le tiers des hommes.
\VS{16}Le nombre des cavaliers de l'armée était de deux cents millions, car j'en entendis le nombre.
\VS{17}Et je vis aussi dans la vision les chevaux et ceux qui étaient montés dessus, ayant des cuirasses de feu, d'hyacinthe et de soufre~; et les têtes des chevaux étaient comme des têtes de lions~; et de leurs bouches sortaient du feu, de la fumée et du soufre.
\VS{18}Le tiers des hommes fut tué par ces trois fléaux, par le feu, et par la fumée et par le soufre qui sortaient de leur bouche.
\VS{19}Car le pouvoir des chevaux était dans leurs bouches et dans leurs queues~; et leurs queues étaient semblables à des serpents ayant des têtes, et c'est avec elles qu'ils faisaient du mal.
\VS{20}Mais les autres hommes qui ne furent pas tués par ces fléaux, ne se repentirent pas des œuvres de leurs mains, ils ne cessèrent pas d'adorer les démons, les idoles d'or, d'argent, de cuivre, de pierre, et de bois, qui ne peuvent ni voir, ni entendre, ni marcher.
\VS{21}Et ils ne se repentirent pas aussi de leurs meurtres, ni de leurs enchantements, ni de leur impudicité, ni de leurs vols.
\Chap{10}
\TextTitle{Un ange puissant descend du ciel}
\VerseOne{}Je vis un autre ange puissant qui descendait du ciel, environné d'une nuée, au-dessus de sa tête était l'arc-en-ciel, son visage était comme le soleil, et ses pieds comme des colonnes de feu.
\VS{2}Et il avait dans sa main un petit livre ouvert, et il posa son pied droit sur la mer, et le pied gauche sur la terre~;
\VS{3}et il cria d'une voix forte, comme lorsqu'un lion rugit. Et quand il eut crié, les sept tonnerres firent entendre leurs voix.
\VS{4}Et après que les sept tonnerres eurent fait entendre leurs voix, j'allais écrire, mais j'entendis une voix du ciel qui me disait~: Scelle les choses que les sept tonnerres ont fait entendre, et ne les écris pas.
\VS{5}Et l'ange que j'avais vu se tenant sur la mer et sur la terre, leva sa main vers le ciel,
\VS{6}et jura par celui qui est vivant aux siècles des siècles, qui a créé le ciel avec les choses qui y sont, et la terre avec les choses qui y sont, et la mer avec les choses qui y sont, qu'il n'y aurait plus de temps~;
\VS{7}mais qu'aux jours de la voix du septième ange, quand il commencera à sonner de la trompette, le mystère de Dieu sera accompli, comme il l'a déclaré à ses serviteurs les prophètes.
\TextTitle{Nouvelle mission de Jean}
\VS{8}Et la voix que j'avais entendue du ciel me parla encore et me dit~: Va, et prends le petit livre ouvert qui est dans la main de l'ange qui se tient sur la mer et sur la terre.
\VS{9}Et j'allai vers l'ange, en lui disant~: Donne-moi le petit livre~; et il me dit~: Prends-le et mange-le~; il remplira tes entrailles d'amertume, mais il sera doux dans ta bouche comme du miel\FTNT{Ez. 3:1-3.}.
\VS{10}Je pris donc le petit livre de la main de l'ange, et je le mangeai~; il fut doux dans ma bouche comme du miel, mais quand je l'eus mangé, mes entrailles furent remplies d'amertume.
\VS{11}Alors il me dit~: Il faut que tu prophétises de nouveau sur beaucoup de peuples, et sur plusieurs nations, sur plusieurs langues et plusieurs rois.
\Chap{11}
\TextTitle{Le temps des nations}
\VerseOne{}On me donna un roseau semblable à une verge, et l'ange se présenta et me dit~: Lève-toi et mesure le temple de Dieu et l'autel, et ceux qui y adorent.
\VS{2}Mais laisse de côté le parvis extérieur du temple, et ne le mesure pas~; car il est donné aux Gentils, et ils fouleront aux pieds la ville sainte pendant quarante-deux mois\FTNT{C'est le temps que durera la grande tribulation, soit trois ans et demi. Daniel parle d'une semaine, un jour comptant pour une année (Da. 9:27). La grande tribulation débutera à la moitié de cette semaine, ce qui correspond bien à quarante-deux mois (Ap. 13:5) et à mille deux cent soixante jours (Ap. 11:3~; Ap. 12:6).}.
\TextTitle{Les deux témoins ressuscitent}
\VS{3}Mais je donnerai à mes deux témoins de prophétiser pendant mille deux cent soixante jours, revêtus de sacs.
\VS{4}Ce sont les deux oliviers\FTNT{Za. 4:14.} et les deux chandeliers qui se tiennent devant le Dieu de la terre. 
\VS{5}Et si quelqu'un veut leur faire du mal, du feu sort de leurs bouches et dévore leurs ennemis~; car si quelqu'un veut leur faire du mal, il faut qu'il soit tué de cette manière.
\VS{6}Ils ont le pouvoir de fermer le ciel, afin qu'il ne pleuve pas pendant les jours de leur prophétie~; ils ont aussi le pouvoir de changer les eaux en sang, et de frapper la terre de toutes sortes de plaies, toutes les fois qu'ils le voudront.
\VS{7}Et quand ils auront achevé de rendre leur témoignage, la bête qui monte de l'abîme\FTNT{L'homme impie, l'Antichrist, ou encore le fils de la perdition dont il est question dans 2 Th. 2:3~; 2 Th. 2:8-9.} leur fera la guerre, les vaincra, et les tuera.
\VS{8}Et leurs cadavres seront étendus sur les places de la grande ville, qui est appelée spirituellement Sodome et Egypte, où aussi notre Seigneur a été crucifié.
\VS{9}Et ceux des tribus, des peuples, des langues, et des nations verront leurs cadavres pendant trois jours et demi, et ils ne permettront pas que leurs cadavres soient mis dans des sépulcres.
\VS{10}Et les habitants de la terre se réjouiront, ils seront dans l'allégresse, ils s'enverront des présents les uns aux autres, parce que ces deux prophètes ont tourmenté les habitants de la terre.
\VS{11}Mais après ces trois jours et demi, l'Esprit de vie venant de Dieu entra en eux, et ils se tinrent sur leurs pieds, et une grande crainte saisit ceux qui les virent.
\VS{12}Après cela, ils entendirent une forte voix du ciel, leur disant~: Montez ici~! Et ils montèrent au ciel sur une nuée, et leurs ennemis les virent.
\VS{13}Et à cette même heure-là, il eut un grand tremblement de terre, et la dixième partie de la ville tomba, et sept mille hommes furent tués par ce tremblement de terre~; et les autres furent épouvantés et donnèrent gloire au Dieu du ciel.
\VS{14}Le second malheur est passé. Voici, le troisième malheur vient bientôt.
\TextTitle{Septième trompette~: Le règne du Messie annoncé, cantique des vingt-quatre vieillards\FTNTT{Ap. 8:2.}}
\VS{15}Le septième ange sonna de la trompette, et il se fit entendre au ciel de grandes voix qui disaient~: Les royaumes du monde sont soumis à notre Seigneur et à son Christ, et il régnera aux siècles des siècles.
\VS{16}Alors les vingt-quatre anciens qui étaient assis devant Dieu sur leurs trônes, se prosternèrent sur leurs faces et adorèrent Dieu,
\VS{17}en disant~: Nous te rendons grâces, Seigneur Dieu Tout-Puissant, QUI ES, QUI ETAIS, et QUI VIENS, de ce que tu as fait éclater ta grande puissance, et de ce que tu as agi en Roi.
\VS{18}Les nations se sont irritées, mais ta colère est venue, et le temps est venu de juger les morts, et de donner la récompense à tes serviteurs les prophètes et aux saints, et à ceux qui craignent ton Nom, petits et grands, et de détruire ceux qui corrompent la terre.
\VS{19}Et le temple de Dieu fut ouvert dans le ciel, et l'arche de son alliance apparut dans son temple. Et il y eut des éclairs, des voix, des tonnerres, un tremblement de terre, et une grosse grêle.
\Chap{12}
\TextTitle{Vision de la femme et du dragon}
\VerseOne{}Et un grand signe parut dans le ciel~: Une femme revêtue du soleil, la lune sous ses pieds, et sur sa tête une couronne de douze étoiles\FTNT{En Ge. 37:9-10, Joseph raconte à ses parents et à ses frères un songe particulier où il voyait le soleil, la lune et onze étoiles se prosterner devant lui. Jacob comprit que les onze étoiles représentaient ses enfants, la lune sa femme Rachel, qui était la mère de Joseph, et que le soleil c'était lui-même. Il est donc question ici d'Israël, qui a toujours été identifié à une femme (Ez. 16) de qui est issu le Messie selon la chair (Ro. 9:5).}.
\VS{2}Elle était enceinte, et elle criait, étant en travail d'enfant, souffrant les grandes douleurs de l'enfantement.
\VS{3}Il parut aussi un autre signe dans le ciel, et voici un grand dragon rouge feu ayant sept têtes et dix cornes, et sur ses têtes sept diadèmes.
\VS{4}Sa queue entraînait le tiers des étoiles du ciel et les jeta sur la terre\FTNT{Da. 8:10.}. Puis le dragon s'arrêta devant la femme qui devait accoucher, afin de dévorer son enfant\FTNT{Cet enfant est évidemment Jésus-Christ (Mt. 2:16.)}, dès qu'elle l'aurait mis au monde.
\TextTitle{La naissance du Messie}
\VS{5}Et elle accoucha d'un fils, qui doit gouverner toutes les nations avec un sceptre de fer\FTNT{Ps. 2:8-9.}. Et son enfant fut enlevé vers Dieu et vers son trône\FTNT{Lu. 24:51~; Ac. 1:9-11.}.
\VS{6}Et la femme s'enfuit dans un désert, où elle avait un lieu préparé par Dieu, afin d'y être nourrie pendant mille deux cent soixante jours.
\TextTitle{Guerre entre l'archange Michel et le dragon}
\VS{7}Et il y eut une guerre dans le ciel. Michel et ses anges combattirent contre le dragon. Et le dragon et ses anges combattirent contre Michel,
\VS{8}mais ils ne furent pas les plus forts, et leur place ne fut plus trouvée dans le ciel.
\VS{9}Et il fut précipité le grand dragon, le serpent ancien, appelé le diable et Satan, celui qui séduit toute la terre, il fut précipité sur la terre, et ses anges furent précipités avec lui\FTNT{Es. 14:12-15~; Ez. 28~; Lu. 10:18.}.
\VS{10}Et j'entendis une voix forte dans le ciel qui disait~: Maintenant le salut est arrivé, ainsi que la force, le règne de notre Dieu, et la puissance de son Christ~; car l'accusateur de nos frères, qui les accusait devant notre Dieu jour et nuit, a été précipité.
\VS{11}Et ils l'ont vaincu à cause du sang de l'Agneau, et à cause de la parole de leur témoignage, et ils n'ont pas aimé leurs vies, mais les ont exposées à la mort.
\VS{12}C'est pourquoi réjouissez-vous cieux, et vous qui y habitez. Mais malheur à vous habitants de la terre et de la mer~! Car le diable est descendu vers vous animé d'une grande fureur, sachant qu'il a peu de temps.
\TextTitle{Le dragon persécute la femme, sa postérité et les témoins du Messie}
\VS{13}Quand le dragon vit qu'il avait été précipité sur la terre, il persécuta la femme qui avait enfanté le fils.
\VS{14}Mais deux ailes d'un grand aigle furent données à la femme, afin qu'elle s'envole de devant le serpent au désert, où elle est nourrie un temps, des temps, et la moitié d'un temps.
\VS{15}Et de sa gueule, le serpent lança de l'eau comme un fleuve derrière la femme, afin de l'entraîner par le fleuve.
\VS{16}Mais la terre secourut la femme, elle ouvrit sa bouche, et elle engloutit le fleuve que le dragon avait lancé de sa gueule.
\VS{17}Alors le dragon fut irrité contre la femme, et s'en alla faire la guerre contre les autres qui sont de la semence de la femme, qui gardent les commandements de Dieu, et qui ont le témoignage de Jésus-Christ.
\VS{18}Et je me tins sur le sable qui borde la mer.
\Chap{13}
\TextTitle{La bête qui monte de la mer, l'antichrist}
\VerseOne{}Et je vis monter de la mer une bête\FTNT{Cette bête représente deux entités. Tout d'abord l'homme impie, l'Antichrist, et ensuite un système politique. Les dix cornes sur sa tête symbolisent les dix nations les plus puissantes de la terre avec lesquelles il imposera sa dictature mondiale (Da. 7:16-25). L'alliage des quatre métaux dans la statue de Nébucadnetsar en Da. 2 et la vision des quatre animaux en Da. 7, annoncent l'instauration d'un quatrième empire ou encore le système politique à la tête duquel sera la bête.} qui avait sept têtes et dix cornes, et sur ses cornes dix diadèmes, et sur ses têtes des noms de blasphème\FTNT{Voir annexe «~La bête d'apocalypse~».}.
\VS{2}Et la bête que je vis était semblable à un léopard, ses pieds étaient comme ceux d'un ours~; sa gueule était comme la gueule d'un lion\FTNT{Da. 7:7.}. Et le dragon lui donna sa puissance, son trône, et une grande autorité.
\VS{3}Et je vis l'une de ses têtes comme blessée à mort, mais sa blessure mortelle fut guérie. Remplie d'admiration, la terre entière suivit la bête.
\VS{4}Et ils adorèrent le dragon, parce qu'il avait donné l'autorité à la bête, et ils adorèrent aussi la bête, en disant~: Qui est semblable à la bête, et qui peut combattre contre elle~?
\VS{5}Et il lui fut donné une bouche qui proférait des discours pleins d'orgueil, et des blasphèmes~; et il lui fut aussi donné le pouvoir d'agir pendant quarante-deux mois.
\VS{6}Elle ouvrit sa bouche pour blasphémer contre Dieu, pour blasphémer son Nom et son tabernacle, et ceux qui habitent dans le ciel.
\VS{7}Et il lui fut donné de faire la guerre aux saints et de les vaincre. Il lui fut aussi donné autorité sur toute tribu, toute langue et toute nation.
\VS{8}Et tous les habitants de la terre l'adoreront, ceux dont les noms n'ont pas été écrits dans le livre de vie de l'Agneau immolé dès la fondation du monde.
\VS{9}Si quelqu'un a des oreilles qu'il entende.
\VS{10}Si quelqu'un est destiné à la captivité, il ira en captivité~; si quelqu'un tue avec l'épée, il faut qu'il soit lui-même tué avec l'épée. C'est ici la persévérance et la foi des saints.
\TextTitle{La bête qui monte de la terre, le faux prophète}
\VS{11}Puis je vis une autre bête qui montait de la terre\FTNT{Cette bête est identifiée au faux-prophète car son rôle consiste à amener les habitants de la terre à adorer la première bête, tout comme les vrais prophètes invitent les gens à l'adoration du Dieu véritable (Mt. 7:15).}, et qui avait deux cornes semblables à celles de l'Agneau~; mais elle parlait comme le dragon.
\VS{12}Et elle exerçait toute l'autorité de la première bête en sa présence, et elle obligeait la terre et ses habitants à adorer la première bête, dont la blessure mortelle avait été guérie\FTNT{Cette bête a existé par le passé sous la forme de l'empire romain qui s'est écroulé le 4 septembre 476. Ce régime a marqué l'histoire par son caractère universel et brutal. Le fait que cette bête blessée à mort reprenne vie, annonce l'instauration d'un empire universel qui aura les caractéristiques combinées de l'empire babylonien, médo-perse, gréco-macédonien et romain, ceux-ci correspondant aux quatre animaux de la vision de Da. 7:1-8~: Le lion, l'ours, le léopard et le quatrième animal.}.
\VS{13}Elle opérait de grands prodiges, même jusqu'à faire descendre le feu du ciel sur la terre devant les hommes.
\VS{14}Et elle séduisait les habitants de la terre, à cause des prodiges qu'il lui était donné d'opérer en présence de la bête, disant aux habitants de la terre de faire une image\FTNT{Dieu interdit la vénération des images (Ex. 20:4-5) La particularité de l'image de la bête est qu'elle possède un esprit (démon).} de la bête qui avait reçu le coup mortel de l'épée, et qui était bien vivante.
\VS{15}Et il lui fut donné de mettre un esprit à l'image de la bête, afin que même l'image de la bête parle, et qu'elle fasse que tous ceux qui n'adoreraient pas l'image de la bête soient mis à mort.
\VS{16}Elle fit que tous, petits et grands, riches et pauvres, libres et esclaves, reçoivent une marque sur leur main droite, ou sur leur front\FTNT{Il s'agit d'une marque qui est avant tout spirituelle. Car de la même façon que nous sommes scellés et marqués par l'Esprit de Dieu qui produit en nous la sainteté (Ga. 5:22~; Ro. 6:20-22~; Ep. 1:13~; Ep. 4:30) Satan marque les siens par le péché (1 Ti. 4:1-2~; 2 Ti. 3:1-5).}~;
\VS{17}et que personne ne puisse acheter ni vendre, sans avoir la marque ou le nom de la bête, ou le nombre de son nom.
\VS{18}Ici est la sagesse~: Que celui qui a de l'intelligence compte le nombre de la bête, car c'est un nombre d'homme, et son nombre est six cent soixante-six.
\Chap{14}
\TextTitle{L'Agneau et les 144 000}
\VerseOne{}Puis je regardai, et voici, l'Agneau se tenait sur la montagne de Sion, et il y avait avec lui cent quarante-quatre mille personnes qui avaient le Nom de son Père écrit sur leurs fronts.
\VS{2}Et j'entendis une voix du ciel comme le bruit des grandes eaux, et comme le bruit d'un grand tonnerre~; et j'entendis une voix de joueurs de harpe jouant de leurs harpes.
\VS{3}Et ils chantaient comme un cantique nouveau devant le trône, et devant les quatre animaux, et devant les anciens. Et personne ne pouvait apprendre le cantique, si ce n'est les cent quarante-quatre mille qui avaient été rachetés de la terre.
\VS{4}Ce sont ceux qui ne se sont pas souillés avec les femmes, car ils sont vierges~; ce sont ceux qui suivent l'Agneau partout où il va. Ils ont été rachetés d'entre les hommes pour être des prémices pour Dieu et pour l'Agneau.
\VS{5}Et dans leur bouche il ne s'est pas trouvé de fraude, car ils sont sans tache devant le trône de Dieu\FTNT{Ps. 32:2.}.
\TextTitle{l'Evangile éternel et la chute de Babylone}
\VS{6}Puis je vis un autre ange qui volait au milieu du ciel, il avait l'Evangile éternel pour évangéliser les habitants de la terre, de toute nation, de toute tribu, de toute langue et de tout peuple.
\VS{7}Il disait d'une voix forte~: Craignez Dieu, et donnez-lui gloire, car l'heure de son jugement est venue~; et adorez celui qui a fait le ciel et la terre, la mer et les sources des eaux.
\VS{8}Et un autre ange le suivit, disant~: Elle est tombée, elle est tombée Babylone, la grande ville, parce qu'elle a abreuvé toutes les nations du vin de la fureur de son impudicité~!
\TextTitle{Le jugement des adorateurs de la bête}
\VS{9}Et un troisième ange les suivit, disant d'une voix forte~: Si quelqu'un adore la bête et son image, et reçoit la marque sur son front ou sur sa main,
\VS{10}il boira, lui aussi, du vin de la colère de Dieu, du vin pur versé dans la coupe de sa colère, et il sera tourmenté dans le feu et le soufre devant les saints anges et devant l'Agneau.
\VS{11}Et la fumée de leur tourment montera aux siècles des siècles, et ils n'auront de repos ni jour ni nuit, ceux qui adorent la bête et son image, et quiconque reçoit la marque de son nom.
\VS{12}Ici est la persévérance des saints~; ici sont ceux qui gardent les commandements de Dieu, et la foi de Jésus.
\TextTitle{Bénédiction de ceux qui meurent en Christ}
\VS{13}Alors j'entendis une voix du ciel qui me disait~: Ecris~: Bénis sont dès à présent les morts qui meurent dans le Seigneur~! Oui, c'est vrai~! dit l'Esprit, afin qu'ils se reposent de leurs travaux, car leurs œuvres les suivent.
\TextTitle{Prophétie sur Harmaguédon}
\VS{14}Et je regardai, et voici, il y avait une nuée blanche, et sur la nuée était assis quelqu'un qui ressemblait à un homme\FTNT{Ez. 1:26~; Da. 7:13~; Mt. 24:30~; Mt. 26:64~; Ap. 1:13.}, ayant sur sa tête une couronne d'or, et dans sa main une faucille tranchante.
\VS{15}Et un autre ange sortit du temple, criant à haute voix à celui qui était assis sur la nuée~: Jette ta faucille, et moissonne~; car c'est ton heure de moissonner, parce que la moisson de la terre est mûre\FTNT{Jé. 51:33~; Mt. 13:30-39.}.
\VS{16}Alors celui qui était assis sur la nuée jeta sa faucille sur la terre, et la terre fut moissonnée.
\VS{17}Et un autre ange sortit du temple qui est dans le ciel, ayant lui aussi une faucille tranchante.
\VS{18}Et un autre ange, qui avait autorité sur le feu, sortit de l'autel, et s'adressant d'une voix forte à celui qui avait la faucille tranchante, dit~: Jette ta faucille tranchante, et vendange les grappes de la vigne de la terre, car ses raisins sont mûrs.
\VS{19}Et l'ange jeta sa faucille tranchante sur la terre et vendangea la vigne de la terre, et il jeta la vendange dans la grande cuve de la colère de Dieu.
\VS{20}Et la cuve fut foulée hors de la ville~; et du sang sortit de la cuve, jusqu'aux mors des chevaux, sur une étendue de mille six cents stades\FTNT{Es. 63:1-6.}.
\Chap{15}
\TextTitle{Une scène glorieuse au ciel}
\VerseOne{}Puis je vis dans le ciel un autre signe, grand et admirable~: Sept anges qui tenaient les sept derniers fléaux, car c'est par eux que s'accomplit la colère de Dieu.
\VS{2}Et je vis aussi comme une mer de verre mêlée de feu, et ceux qui avaient vaincu la bête et son image, et sa marque, et le nombre de son nom, étaient debout sur la mer qui était comme de verre, et ayant les harpes de Dieu.
\VS{3}Ils chantaient le cantique de Moïse, serviteur de Dieu, et le cantique de l'Agneau, en disant~: Tes œuvres sont grandes et merveilleuses, ô Seigneur Dieu Tout-Puissant~! Tes voies sont justes et véritables, ô Roi des saints~!
\VS{4}Seigneur, qui ne te craindrait, et qui ne glorifierait ton Nom~? Car toi seul tu es Saint, c'est pourquoi toutes les nations viendront et se prosterneront devant toi~; car tes jugements sont pleinement manifestés.
\VS{5}Et après ces choses, je regardai, et voici le temple du tabernacle du témoignage fut ouvert dans le ciel.
\VS{6}Et les sept anges qui avaient les sept fléaux sortirent du temple, revêtus d'un lin pur et blanc, et ayant des ceintures d'or autour de leurs poitrines.
\VS{7}Et l'un des quatre animaux donna aux sept anges sept coupes d'or, pleines de la colère du Dieu qui vit aux siècles des siècles.
\VS{8}Et le temple fut rempli de la fumée à cause de la gloire de Dieu et de sa puissance~; et personne ne pouvait entrer dans le temple jusqu'à ce que les sept fléaux des sept anges soient accomplis.
\Chap{16}
\TextTitle{Première coupe~: Les ulcères}
\VerseOne{}Et j'entendis du temple une voix éclatante qui disait aux sept anges~: Allez, et versez sur la terre les coupes de la colère de Dieu.
\VS{2}Et le premier ange s'en alla, et versa sa coupe sur la terre. Et un ulcère malin et dangereux frappa les hommes qui avaient la marque de la bête, et ceux qui adoraient son image.
\TextTitle{Deuxième coupe~: La mer changée en sang}
\VS{3}Et le second ange versa sa coupe sur la mer, et elle devint comme le sang d'un corps mort, et tout être qui vivait dans la mer mourut.
\TextTitle{Troisième coupe~: Les sources changées en sang}
\VS{4}Et le troisième ange versa sa coupe sur les fleuves et sur les sources des eaux, et elles devinrent du sang.
\VS{5}Et j'entendis l'ange des eaux qui disait~: Seigneur, QUI ES, QUI ETAIS, et QUI VIENS, tu es juste, parce que tu as exercé ce jugement.
\VS{6}Parce qu'ils ont répandu le sang des saints et des prophètes, tu leur as aussi donné du sang à boire, car ils le méritent.
\VS{7}Et j'entendis un autre de l'autel, qui disait~: Certainement, Seigneur Dieu Tout-Puissant, tes jugements sont véritables et justes.
\TextTitle{Quatrième coupe~: Une chaleur extrême}
\VS{8}Ensuite, le quatrième ange versa sa coupe sur le soleil, et le pouvoir lui fut donné de brûler les hommes par le feu,
\VS{9}de sorte que les hommes furent brûlés par de grandes chaleurs, et ils blasphémèrent le Nom de Dieu qui a puissance sur ces fléaux~; et ils ne se repentirent pas pour lui donner gloire.
\TextTitle{Cinquième coupe~: Les ténèbres sur le trône de la bête}
\VS{10}Après cela, le cinquième ange versa sa coupe sur le trône de la bête. Et son royaume fut couvert de ténèbres, et les hommes se mordaient la langue à cause de la douleur qu'ils ressentaient.
\VS{11}Et ils blasphémèrent le Dieu du ciel à cause de leurs douleurs et de leurs ulcères~; et ils ne se repentirent pas de leurs œuvres.
\TextTitle{Sixième coupe~: L'Euphrate asséché}
\VS{12}Puis le sixième ange versa sa coupe sur le grand fleuve, l'Euphrate. Et son eau tarit, afin de préparer la voie des rois venant du côté où le soleil se lève.
\VS{13}Et je vis sortir de la gueule du dragon, et de la gueule de la bête, et de la bouche du faux prophète, trois esprits impurs semblables à des grenouilles.
\VS{14}Car ce sont des esprits de démons, qui font des prodiges, et qui vont vers les rois de la terre et du monde entier, afin de les assembler pour le combat de ce grand jour du Dieu Tout-Puissant.
\VS{15}Voici, je viens comme un voleur. Béni est celui qui veille et qui garde ses vêtements, afin de ne pas marcher nu, et qu'on ne voie pas sa honte~!
\VS{16}Et ils les assemblèrent dans le lieu qui est appelé en hébreu Harmaguédon\FTNT{Le terme «~Harmaguédon~», mentionné uniquement dans ce passage, vient du mot hébreu «~Har-Magidown~», ce qui signifie «~Montagne de Megiddo~». Bien qu'il n'existe pas de montagne portant spécifiquement ce nom, l'emplacement probable de cet endroit est la plaine de Meggido se trouvant à proximité de Jérusalem. Par le passé, elle fut le théâtre de la victoire de Barak sur les Cananéens (Jg. 4:15) et de celle de Gédéon sur les Madianites (Jg. 7). C'est aussi à cet endroit que Saül et ses fils (1 Sa. 31~:8) ainsi que le roi Josias (2 R. 23:29-30~; 2 Ch. 35:22) trouvèrent la mort. Pour toutes ces raisons, elle devint au fil du temps le symbole de l'affrontement entre Dieu et la puissance des ténèbres. Selon les prophéties bibliques, la plaine de Meggido et la vallée de Jizréel constitueront le site de l'ultime guerre mondiale, celle opposant l'Antichrist et ses alliés (dirigeants des nations) contre Israël. Le Seigneur interviendra alors ouvertement dans les affaires humaines pour déverser la coupe de sa colère (Ap. 16:1) et anéantir l'homme impie et toute son armée (Ez. 38-39~; Joë. 3~; Mi. 4:11~; So. 1~; Za. 14~; Mt. 24:29-30~; Ap. 20:1-3~; Ap. 20:7-10).}.
\TextTitle{Septième coupe~: Une grosse grêle tombe du ciel}
\VS{17}Puis le septième ange versa sa coupe dans l'air~; et il sortit du temple du ciel une voix forte qui venait du trône, disant~: C'en est fait.
\VS{18}Et il y eut des éclairs, et des voix, et des tonnerres, et il se fit un grand tremblement de terre, dis-je, tel qu'il n'y en avait jamais eu depuis que les hommes sont sur la terre.
\VS{19}La grande ville fut divisée en trois parties, et les villes des nations tombèrent, et Dieu se souvint de Babylone la grande, pour lui donner la coupe du vin de son ardente colère.
\VS{20}Toutes les îles s'enfuirent et les montagnes ne furent plus retrouvées.
\VS{21}Une grosse grêle, dont les grêlons pesaient un talent\FTNT{Un talent d'argent pesait 45 kg, un talent d'or pesait 90 kg.}, tomba du ciel sur les hommes~; et les hommes blasphémèrent Dieu, à cause du fléau de la grêle, car le fléau qu'elle causa fut très grand.
\Chap{17}
\TextTitle{La prostituée}
\VerseOne{}Puis l'un des sept anges qui tenaient les sept coupes vint, et il m'adressa la parole, en disant~: Viens, je te montrerai le jugement de la grande prostituée, qui est assise sur les grandes eaux.
\VS{2}Avec elle, les rois de la terre ont commis la fornication, et les habitants de la terre ont été enivrés du vin de sa prostitution.
\VS{3}Il me transporta en esprit dans un désert~; et je vis une femme assise sur une bête écarlate, pleine de noms de blasphème, ayant sept têtes et dix cornes.
\VS{4}Et la femme était vêtue de pourpre et d'écarlate, et parée d'or, de pierres précieuses, et de perles~; et elle tenait à la main une coupe d'or, pleine des abominations de l'impureté de sa prostitution.
\VS{5}Et il y avait sur son front un nom écrit, un mystère~: Babylone la grande, la mère des impudicités et des abominations de la terre\FTNT{Symboliquement, Babylone la grande incarne l'Eglise apostate. Elle est soutenue par la bête qu'elle chevauche, c'est-à-dire l'homme impie. Ces deux entités forment un système impie où la politique et la religion se mélangent (Da. 2:43).}.
\VS{6}Et je vis cette femme ivre du sang des saints, et du sang des martyrs de Jésus. Et quand je la vis, je fus saisi d'un grand étonnement.
\TextTitle{Alliance entre la prostituée et la bête}
\VS{7}Et l'ange me dit~: Pourquoi t'étonnes-tu~? Je te dirai le mystère de la femme et de la bête qui la porte, qui a les sept têtes et les dix cornes.
\VS{8}La bête que tu as vue, était, et elle n'est plus. Elle doit monter de l'abîme, et aller à la perdition. Et les habitants de la terre, ceux dont les noms ne sont pas écrits dans le Livre de vie dès la fondation du monde, s'étonneront en voyant la bête parce qu'elle était, et qu'elle n'est plus, et qui toutefois est.
\VS{9}C'est ici qu'il faut un esprit intelligent et qui ait de la sagesse. Les sept têtes sont sept montagnes sur lesquelles la femme est assise.
\VS{10}Ce sont aussi sept rois, les cinq sont tombés~; l'un est, et l'autre n'est pas encore venu~; et quand il sera venu, il faut qu'il demeure pour un peu de temps.
\VS{11}Et la bête qui était, et qui n'est plus, est elle-même un huitième roi, et elle est du nombre des sept, mais elle tend à sa ruine.
\VS{12}Et les dix cornes que tu as vues sont dix rois, qui n'ont pas encore commencé à régner, mais ils recevront autorité comme rois en même temps avec la bête, pour une heure.
\VS{13}Ils ont un même dessein, et ils donneront leur puissance et leur autorité à la bête.
\TextTitle{Victoire de l'Agneau sur la prostituée}
\VS{14}Ils combattront contre l'Agneau et l'Agneau les vaincra, parce qu'il est le Seigneur des seigneurs, et le Roi des rois~; et les appelés, les élus et les fidèles qui sont avec lui, les vaincront aussi.
\VS{15}Puis il me dit~: Les eaux que tu as vues, et sur lesquelles la prostituée est assise, sont des peuples, des nations et des langues.
\VS{16}Les dix cornes que tu as vues sur la bête haïront la prostituée, la rendront désolée et nue, la dépouilleront, et mangeront sa chair, et la brûleront au feu.
\VS{17}Car Dieu a mis dans leurs cœurs de faire ce qu'il lui plaît, et de former un même dessein, et de donner leur royaume à la bête, jusqu'à ce que les paroles de Dieu soient accomplies.
\VS{18}Et la femme que tu as vue, c'est la grande ville, qui règne sur les rois de la terre.
\Chap{18}
\TextTitle{Babylone détruite}
\VerseOne{}Après ces choses, je vis descendre du ciel un autre ange, qui avait une grande autorité, et la terre fut illuminée de sa gloire.
\VS{2}Il cria avec force à haute voix, et il dit~: Elle est tombée, elle est tombée Babylone la grande, et elle est devenue la demeure de démons, et la retraite de tout esprit impur, et le repaire de tout oiseau impur et exécrable.
\VS{3}Car toutes les nations ont bu du vin de sa prostitution effrénée, et les rois de la terre ont commis la fornication avec elle, et les marchands de la terre se sont enrichis par l'excès de son luxe.
\VS{4}Puis j'entendis une autre voix du ciel, qui disait~: Sortez de Babylone, mon peuple, afin que vous ne participiez pas à ses péchés, et que vous n'ayez pas de part à ses fléaux.
\VS{5}Car ses péchés sont montés jusqu'au ciel, et Dieu s'est souvenu de ses iniquités.
\VS{6}Rendez-lui selon ce qu'elle vous a fait, et payez-lui au double selon ses œuvres~; et dans la même coupe où elle vous a versé à boire versez-lui au double.
\VS{7}Autant elle s'est glorifiée et plongée dans le luxe, autant donnez-lui de tourment et de deuil~; car elle dit en son cœur~: Je siège en reine, je ne suis pas veuve, et je ne verrai pas de deuil.
\VS{8}C'est pourquoi ses plaies, qui sont la mort, le deuil, et la famine, viendront en un même jour, et elle sera entièrement brûlée au feu~; car le Seigneur Dieu qui la jugera est puissant.
\TextTitle{Conséquence de la chute de Babylone~: Gémissements des habitants de la terre}
\VS{9}Et les rois de la terre, qui ont commis la fornication avec elle, et qui ont vécu dans le luxe, la pleureront, et mèneront deuil sur elle en se frappant la poitrine, quand ils verront la fumée de son embrasement~;
\VS{10}et ils se tiendront éloignés dans la crainte de son tourment, et diront~: Malheur~! Malheur~! Babylone la grande, cette ville si puissante, comment ta condamnation est-elle venue en une seule heure~?
\VS{11}Les marchands de la terre aussi pleureront, et seront dans le deuil à cause d'elle, parce que personne n'achète plus de leurs marchandises,
\VS{12}qui sont des marchandises d'or, d'argent, de pierres précieuses, de perles, de fin lin, de pourpre, de soie, d'écarlate, de toute sorte de bois odoriférant, de toute espèce de bois de senteur, d'ivoire, et de toute espèce de vaisseaux de bois très précieux, d'airain, de fer, et de marbre,
\VS{13}du cinnamome, des parfums, des essences, de l'encens, du vin, de l'huile, de la fine fleur de farine, du blé, des bœufs, des brebis, des chevaux, des chars, des esclaves, et des âmes d'hommes.
\VS{14}Car les fruits du désir de ton âme se sont éloignés de toi, et toutes les choses délicates et excellentes sont perdues pour toi, et dorénavant tu ne les trouveras plus.
\VS{15}Les marchands, dis-je, de ces choses, qui se sont enrichis par elle, se tiendront éloignés, dans la crainte de son tourment~; ils pleureront et seront dans le deuil,
\VS{16}et diront~: Malheur~! Malheur~! La grande ville qui était vêtue de fin lin, de pourpre, d'écarlate, qui était parée d'or, ornée de pierres précieuses, et de perles, comment en une seule heure tant de richesses ont été détruites~?
\VS{17}Et tous les pilotes aussi, tous ceux qui naviguent vers ce lieu, tous les marins, et tous ceux qui exploitent la mer, se tiendront éloignés,
\VS{18}et, en voyant la fumée de son embrasement, ils s'écrieront, en disant~: Quelle ville était semblable à cette grande ville~?
\VS{19}Ils jetteront de la poussière sur leurs têtes, pleurant et menant deuil, ils crieront, en disant~: Malheur~! Malheur~! La grande ville, où se sont enrichis par son opulence tous ceux qui ont des navires sur la mer, comment a-t-elle été réduite en désert en une seule heure~?
\TextTitle{Réjouissance des anges suite à la chute de Babylone}
\VS{20}Ô ciel~! Réjouis-toi à cause d'elle~; et vous aussi saints apôtres et prophètes, réjouissez-vous~! Car Dieu l'a punie à cause de vous.
\VS{21}Alors un ange d'une grande force prit une pierre semblable à une grande meule, et la jeta dans la mer, en disant~: Ainsi sera précipitée avec impétuosité Babylone, cette grande ville, et elle ne sera plus retrouvée\FTNT{Jé. 51:63-64.}.
\VS{22}Et l'on entendra plus chez toi les sons des joueurs de harpe, des musiciens, des joueurs de flûte, et de ceux qui sonnent de la trompette~; et on ne trouvera plus chez toi aucun artisan d'un métier quelconque, on n'entendra plus chez toi le bruit de la meule,
\VS{23}et la lumière de la lampe ne brillera plus chez toi, et la voix de l'époux et de l'épouse ne sera plus entendue chez toi~; car tes marchands étaient des princes de la terre, et parce que par tes enchantements toutes les nations ont été séduites,
\VS{24}et l'on a trouvé chez elle le sang des prophètes et des saints, et de tous ceux qui ont été mis à mort sur la terre.
\Chap{19}
\TextTitle{Allégresse dans les cieux suite au jugement de la grande prostituée\FTNTT{Ap. 17:16-17~; 18:8.}}
\VerseOne{}Après cela, j'entendis dans le ciel une voix forte d'une foule nombreuse, disant~: Alléluia~! Le salut, la gloire, l'honneur et la puissance appartiennent au Seigneur, notre Dieu,
\VS{2}car ses jugements sont véritables et justes~; car il a jugé la grande prostituée qui a corrompu la terre par son impudicité, et parce qu'il a vengé le sang de ses serviteurs versé de la main de la prostituée.
\VS{3}Et ils dirent encore~: Alléluia~! Et sa fumée monte aux siècles des siècles.
\VS{4}Et les vingt-quatre anciens et les quatre animaux se prosternèrent sur leurs faces, et adorèrent Dieu, qui était assis sur le trône, en disant~: Amen~! Alléluia~!
\VS{5}Et il sortit du trône une voix qui disait~: Louez notre Dieu, vous tous ses serviteurs, et vous qui le craignez, tant les petits que les grands\FTNT{Ps. 134.}.
\VS{6}J'entendis ensuite comme la voix d'une grande assemblée, et comme le bruit de grandes eaux, et comme l'éclat de grands tonnerres, disant~: Alléluia~! Car le Seigneur notre Dieu Tout-Puissant a pris possession de son Royaume.
\TextTitle{Festin des noces de l'Agneau}
\VS{7}Réjouissons-nous et tressaillons de joie, et donnons-lui gloire, car les noces de l'Agneau sont venues, et son Epouse s'est préparée.
\VS{8}Et il lui a été donné de se revêtir d'un fin lin pur et éclatant. Car le fin lin désigne la justice des saints.
\VS{9}Alors il me dit~: Ecris~: Bénis sont ceux qui sont appelés au festin des noces de l'Agneau\FTNT{Mt. 22:1-13~; Lu. 14:15-24.}~! Il me dit aussi~: Ces paroles de Dieu sont véritables.
\VS{10}Alors je tombai à ses pieds pour l'adorer, mais il me dit~: Garde-toi de le faire~! Je suis ton compagnon de service, et celui de tes frères qui ont le témoignage de Jésus. Adore Dieu~! Car le témoignage de Jésus est l'Esprit de la prophétie.
\TextTitle{Seconde venue du Messie dans la gloire\FTNTT{Mt. 24:16-30.}}
\VS{11}Puis je vis le ciel ouvert, et voici parut un cheval blanc. Et celui qui était monté dessus s'appelle FIDELE et VERITABLE, et il juge et combat avec justice.
\VS{12}Et ses yeux étaient comme une flamme de feu~; il y avait sur sa tête plusieurs diadèmes, et il avait un nom écrit que personne ne connaît, si ce n'est lui-même.
\VS{13}Il était revêtu d'un vêtement teint de sang, et son Nom s'appelle LA PAROLE DE DIEU.
\VS{14}Les armées qui sont dans le ciel le suivaient sur des chevaux blancs, revêtues de fin lin blanc et pur.
\VS{15}De sa bouche sortait une épée tranchante\FTNT{Es. 11:4~; 2 Th. 2:8~; Hé. 4:12.}, pour frapper les nations~; il les gouvernera avec un sceptre de fer\FTNT{Ps. 2:8-9.}, et il foulera la cuve du vin de l'indignation et de la colère du Dieu Tout-Puissant.
\VS{16}Et sur son vêtement et sur sa cuisse étaient écrits ces mots~: LE ROI DES ROIS ET LE SEIGNEUR DES SEIGNEURS.
\TextTitle{Bataille d'Harmaguédon\FTNTT{Ap. 16:16.}}
\VS{17}Puis je vis un ange qui se tenait dans le soleil. Il cria d'une voix forte, et dit à tous les oiseaux qui volaient au milieu du ciel~: Venez et rassemblez-vous pour le grand festin de Dieu,
\VS{18}afin de manger la chair des rois, la chair des chefs militaires, la chair des puissants, la chair des chevaux et de ceux qui les montent, et la chair de toute sorte de personnes libres, esclaves, petits et grands.
\VS{19}Alors je vis la bête et les rois de la terre, et leurs armées rassemblées pour faire la guerre\FTNT{Guerre d' Harmaguédon~: Voir commentaire Ap. 16:16.} contre celui qui était monté sur le cheval et contre son armée.
\TextTitle{Condamnation de la bête et du faux prophète}
\VS{20}Et la bête fut prise, et avec elle le faux prophète qui avait fait devant elle les prodiges par lesquels il avait séduit ceux qui avaient pris la marque de la bête, et adoré son image. Et ils furent tous deux jetés vivants dans l'étang ardent de feu et de soufre.
\TextTitle{Condamnation des rois et des armées}
\VS{21}Et le reste fut tué par l'épée qui sortait de la bouche de celui qui était monté sur le cheval, et tous les oiseaux furent rassasiés de leur chair.
\Chap{20}
\TextTitle{Satan lié pour mille ans et règne du Messie}
\VerseOne{}Après cela, je vis descendre du ciel un ange, qui avait la clef de l'abîme et une grande chaîne dans sa main.
\VS{2}Il saisit le dragon, le serpent ancien, qui est le diable et Satan, et le lia pour mille ans.
\VS{3}Il le jeta dans l'abîme, et il l'enferma et mit le sceau sur lui, afin qu'il ne séduise plus les nations, jusqu'à ce que les mille ans soient accomplis. Après quoi, il faut qu'il soit délié pour un peu de temps.
\TextTitle{Dernière phase de la première résurrection}
\VS{4}Je vis des trônes, sur lesquels des gens s'assirent, à qui l'autorité de juger fut donnée\FTNT{1 Co. 6:2.}. Et je vis les âmes de ceux qui avaient été décapités pour le témoignage de Jésus, et pour la parole de Dieu, et de ceux qui n'avaient pas adoré la bête ni son image, et qui n'avaient pas pris sa marque sur leurs fronts, ou sur leurs mains. Et ils vécurent\FTNT{Jn. 14:19.} et régnèrent avec Christ mille ans.
\VS{5}Les autres morts ne revinrent pas à la vie jusqu'à ce que les mille ans soient accomplis. C'est la première résurrection.
\VS{6}Bénis et saints sont ceux qui ont part à la première résurrection~! La seconde mort n'a pas de puissance sur eux, mais ils seront prêtres de Dieu, et de Christ, et ils régneront avec lui mille ans.
\TextTitle{Satan délié~; sa chute finale}
\VS{7}Et quand les mille ans seront accomplis, Satan sera délié de sa prison.
\VS{8}Et il sortira pour séduire les nations qui sont aux quatre coins de la terre, Gog et Magog, afin de les rassembler pour la guerre, et leur nombre est comme le sable de la mer.
\VS{9}Ils montèrent et se répandirent à la surface de la terre, et ils environnèrent le camp des saints, et la ville bien-aimée. Mais Dieu fit descendre un feu du ciel qui les dévora.
\TextTitle{Satan jeté dans l'étang de feu}
\VS{10}Et le diable qui les séduisait fut jeté dans l'étang de feu et de soufre, où sont la bête et le faux prophète. Et ils seront tourmentés jour et nuit, aux siècles des siècles.
\TextTitle{Résurrection des impies et jugement dernier~; l'Hadès (ou enfer) et la mort jetés dans l'étang de feu}
\VS{11}Puis je vis un grand trône blanc, et celui qui était assis dessus. La terre et le ciel s'enfuirent devant sa face, et il ne fut plus trouvé de place pour eux.
\VS{12}Et je vis les morts, les grands et les petits, qui se tenaient devant Dieu. Des livres furent ouverts. Et un autre livre fut ouvert, celui qui est le Livre de vie. Et les morts furent jugés selon les choses qui étaient écrites dans les livres, c'est-à-dire selon leurs œuvres.
\VS{13}Et la mer rendit les morts qui étaient en elle, et la mort et l'enfer\FTNT{le mot «~enfer~» vient de l'hébreu «~Hadès~». Voir commentaire dans Mt. 16:18.} rendirent les morts qui étaient en eux~; et ils furent jugés chacun selon ses œuvres.
\VS{14}Et la mort et l'enfer furent jetés dans l'étang de feu\FTNT{L'étang de feu est aussi appelé «~seconde mort~», c'est la destination finale de tous les impies, des démons et de Satan. On l'appelle «~la seconde mort~» parce qu'elle a été précédée de la mort physique. Cette mort n'est pas un anéantissement, mais une condition de souffrances éternelles. C'est la séparation définitive d'avec Dieu. A l'issue du jugement dernier, le séjour des morts (le dieu Hadès ou l'enfer) sera jeté dans le lac de feu (voir commentaire en Mt. 16:18). La Bible utilise également le mot «~géhenne~» pour décrire l'endroit où les impies passeront l'éternité. Ce terme vient de l'hébreu «~ge-hinnom~», autrement dit vallée de Ben Hinnom (littéralement «~le lieu du feu~») qui se trouve en Israël, en contrebas du mont Sion sur lequel est bâtie la ville de Jérusalem (Mt. 5:22~; Mt. 5:29-30~; Mt. 10:28~; Mt. 18:9~; Mt. 23:15~; Mt. 23:33~; Mc. 9:47~; Lu. 12:5~; Ja. 3:6). Autrefois, on y brûlait des enfants en l'honneur de Moloc, une divinité ammonite (2 R. 23:10~; Jé. 32:35), puis des immondices. Ce lieu est devenu avec le temps le symbole du péché et de l'affliction et c'est ainsi qu'il finit par désigner le lieu du châtiment éternel.}. C'est la seconde mort.
\VS{15}Et quiconque ne fut pas trouvé écrit dans le Livre de vie fut jeté dans l'étang de feu.
\Chap{21}
\TextTitle{Nouveaux cieux et une nouvelle terre~; la nouvelle Jérusalem}
\VerseOne{}Puis je vis un nouveau ciel et une nouvelle terre~; car le premier ciel et la première terre avaient disparu, et la mer n'était plus.
\VS{2}Et moi, Jean, je vis la ville sainte, la nouvelle Jérusalem, qui descendait du ciel, d'auprès de Dieu, parée comme une épouse qui s'est ornée pour son mari.
\VS{3}Et j'entendis du trône une voix forte qui disait~: Voici le tabernacle de Dieu avec les hommes~! Il habitera avec eux, et ils seront son peuple, et Dieu lui-même sera leur Dieu, et il sera avec eux.
\VS{4}Et Dieu essuiera toute larme de leurs yeux, et la mort ne sera plus~; et il n'y aura plus ni deuil, ni cri, ni douleur, car les premières choses sont passées.
\VS{5}Et celui qui était assis sur le trône dit~: Voici, je fais toutes choses nouvelles. Puis il me dit~: Ecris, car ces paroles sont véritables et certaines.
\VS{6}Il me dit aussi~: Tout est accompli. Je suis l'Alpha et l'Oméga, le commencement et la fin. A celui qui a soif, je lui donnerai de la source d'eau vive gratuitement\FTNT{Es. 55:1-2~; Mt. 10:8~; Ap. 22:17. Voir commentaire Mt. 10:8.}.
\VS{7}Celui qui vaincra héritera toutes choses~; je serai son Dieu, et il sera mon fils.
\VS{8}Mais pour les timides, les incrédules, les abominables, les meurtriers, les fornicateurs, les sorciers, les idolâtres et tous les menteurs, leur part sera dans l'étang ardent de feu et de soufre, qui est la seconde mort.
\TextTitle{L'Epouse de l'Agneau et la nouvelle Jérusalem}
\VS{9}Puis l'un des sept anges qui tenaient les sept coupes pleines des sept derniers fléaux s'approcha de moi et me parla, en disant~: Viens, et je te montrerai l'Epouse, la femme de l'Agneau.
\VS{10}Et il me transporta en esprit sur une grande et haute montagne, et il me montra la grande ville, la sainte Jérusalem, qui descendait du ciel d'auprès de Dieu,
\VS{11}ayant la gloire de Dieu. Son éclat était semblable à une pierre très précieuse, comme à une pierre de jaspe transparente comme du cristal.
\VS{12}Et elle avait une grande et haute muraille, avec douze portes, et aux portes douze anges, et des noms écrits sur elles, qui sont les noms des douze tribus des fils d'Israël\FTNT{Ez. 48:31-34.}.
\VS{13}A l'orient, trois portes, au nord, trois portes, du côté du sud, trois portes et du côté de l'occident, trois portes.
\VS{14}Et la muraille de la ville avait douze fondements, et les noms des douze apôtres de l'Agneau étaient écrits dessus\FTNT{Lu. 22:29-30~; Ep. 2:20.}.
\VS{15}Et celui qui parlait avec moi avait un roseau d'or pour mesurer la ville, ses portes et sa muraille.
\VS{16}Et la ville était bâtie en carré, et sa longueur était aussi grande que sa largeur. Il mesura donc la ville avec le roseau d'or, jusqu'à douze mille stades~; la longueur, la largeur et la hauteur étaient égales.
\VS{17}Puis il mesura la muraille qui fut de cent quarante-quatre coudées, de la mesure du personnage, c'est-à-dire de l'ange.
\VS{18}Et le bâtiment de la muraille était de jaspe, mais la ville était d'or pur, semblable à du verre fort transparent.
\VS{19}Et les fondements de la muraille de la ville étaient ornés de toutes sortes de pierres précieuses\FTNT{Es. 54:11-12.}~: Le premier fondement était de jaspe, le second de saphir, le troisième de calcédoine, le quatrième d'émeraude,
\VS{20}le cinquième de sardonyx, le sixième de sardoine, le septième de chrysolithe, le huitième de béryl, le neuvième de topaze, le dixième de chrysoprase, le onzième d'hyacinthe, le douzième d'améthyste.
\VS{21}Et les douze portes étaient douze perles~; chacune des portes était d'une seule perle. Et la place de la ville était d'or pur, comme du verre transparent.
\VS{22}Et je ne vis pas de temple dans la ville, parce que le Seigneur Dieu Tout-Puissant et l'Agneau en sont le Temple.
\VS{23}Et la ville n'a pas besoin du soleil ni de la lune pour l'éclairer, car la gloire de Dieu l'éclaire, et l'Agneau est son flambeau\FTNT{Es. 60:19.}.
\VS{24}Et les nations qui auront été sauvées marcheront à la faveur de sa lumière, et les rois de la terre y apporteront ce qu'ils ont de plus magnifique et de plus précieux.
\VS{25}Et ses portes ne se fermeront pas le jour, car il n'y aura pas de nuit\FTNT{Es. 60:11.}.
\VS{26}Et on y apportera la gloire et l'honneur des nations.
\VS{27}Il n'entrera chez elle rien de souillé, ni personne qui s'abandonne à l'abomination et au mensonge~; mais seulement ceux qui sont écrits dans le Livre de vie de l'Agneau.
\Chap{22}
\TextTitle{Règne éternel des saints avec l'Agneau}
\VerseOne{}Puis il me montra un fleuve d'eau de la vie\FTNT{Ce fleuve représente le Saint-Esprit~: Ez. 47:1-12~; Ps. 46:5~; Da. 7:9-10~; Jn. 7:38-39.}, transparent comme du cristal, qui sortait du trône de Dieu et de l'Agneau.
\VS{2}Et au milieu de la place de la ville, et des deux côtés du fleuve, était l'arbre de vie, portant douze fruits, et rendant son fruit chaque mois et les feuilles de l'arbre servaient à la guérison des nations\FTNT{Ge. 2:9~; Ge. 3:22~; Ez. 47:12.}.
\VS{3}Et il n'y aura plus d'anathème. Le trône de Dieu et de l'Agneau sera dans la ville, et ses serviteurs le serviront,
\VS{4}et ils verront sa face, et son Nom sera sur leurs fronts.
\VS{5}Et il n'y aura plus de nuit~; et ils n'auront besoin ni de lumière, ni de lampe, ni du soleil, parce que le Seigneur Dieu les éclairera, et ils régneront aux siècles des siècles.
\TextTitle{Certitude des prophéties de ce livre}
\VS{6}Puis il me dit~: Ces paroles sont certaines et véritables~; et le Seigneur, le Dieu des saints prophètes, a envoyé son ange pour manifester à ses serviteurs les choses qui doivent arriver bientôt.
\VS{7}Voici, je viens à toute vitesse\FTNT{Dans la plupart des traductions, ce passage a été traduit par «~Je viens bientôt~». Or le texte grec utilise le mot «~tachu~» qui signifie «~rapidement, à toute vitesse (sans tarder)~». Beaucoup doutent de cette promesse du Seigneur en faisant la même réflexion évoquée par Pierre~: «~Où est la promesse de son avènement~? Car depuis que les pères sont morts, toutes choses demeurent comme elles ont été dès le commencement de la création.~» (2 Pi. 3:4). Or le Seigneur ne tarde pas dans l'accomplissement de sa promesse, car il a fixé de sa propre autorité une date pour son retour, que lui seul connaît (Za. 14:7~; Mt. 24:36~; Mc. 13:32~; Ac. 1:6-7). Il sera donc fidèle à son calendrier et ne tardera pas (2 Pi. 3:9.~; Hé. 10:37).}. Béni est celui qui garde les paroles de la prophétie de ce livre~!
\VS{8}C'est moi, Jean, qui ai entendu et vu ces choses. Et après les avoir entendues et vues, je tombai à terre aux pieds de l'ange qui me les montrait pour l'adorer.
\VS{9}Mais il me dit~: Garde-toi de le faire~! Car je suis ton compagnon de service\FTNT{Hé. 1:14.} et celui de tes frères les prophètes, et de ceux qui gardent les paroles de ce livre. Adore Dieu~!
\VS{10}Il me dit aussi~: Ne scelle pas les paroles de la prophétie de ce livre. Car le temps est proche.
\VS{11}Que celui qui est injuste soit encore injuste, et que celui qui est souillé se souille encore~; et que celui qui est juste pratique encore la justice~; et que celui qui est saint se sanctifie encore~!
\VS{12}Voici, je viens à toute vitesse, et ma rétribution est avec moi\FTNT{Jésus affirme de nouveau ici sa divinité et confirme les prophéties d'Es. 35:4~; Es. 40:10~; Es. 62:11, où il est dit que Yahweh lui-même viendra avec ses rétributions.} pour rendre à chacun selon son œuvre.
\VS{13}Je suis l'Alpha et l'Oméga, le premier et le dernier, le commencement et la fin.
\VS{14}Bénis sont ceux qui lavent leurs robes afin d'avoir droit à l'arbre de vie, et d'entrer par les portes dans la ville.
\VS{15}Mais seront laissés dehors les chiens, les empoisonneurs, les fornicateurs, les meurtriers, les idolâtres et quiconque aime et pratique le mensonge.
\VS{16}Moi, Jésus, j'ai envoyé mon ange\FTNT{Cette déclaration de Jésus fait écho au verset 6 où il est dit que le Seigneur, le Dieu des esprits des prophètes, a envoyé son ange. Jésus confirme donc qu'il est Seigneur et Dieu.} pour vous confirmer ces choses dans les églises. Je suis le rejeton et la postérité de David, l'étoile brillante du matin.
\VS{17}Et l'Esprit et l'Epouse disent~: Viens~! Et que celui qui entend dise~: Viens~! Et que celui qui a soif vienne~; que celui qui veut, prenne gratuitement de l'eau de la vie.
\TextTitle{Nul ne doit y ajouter ou y retrancher}
\VS{18}Je le déclare à quiconque entend les paroles de la prophétie de ce livre~: Si quelqu'un y ajoute quelque chose, Dieu le frappera des fléaux décrits dans ce livre,
\VS{19}et si quelqu'un retranche quelque chose des paroles du livre de cette prophétie, Dieu retranchera la part qu'il a dans le livre de vie, dans la ville sainte et dans les choses qui sont écrites dans ce livre.
\VS{20}Celui qui rend témoignage de ces choses, dit~: Certainement, je viens à toute vitesse. Amen~! Oui, Seigneur Jésus, viens~!
\VS{21}Que la grâce de notre Seigneur Jésus-Christ soit avec vous tous~! Amen~!
\PPE{}
\end{multicols}

% inclusion des annexes
%\addcontentsline{toc}{chapter}{Aide}\clearpage
%\addcontentsline{toc}{section}{Dictionnaire}\clearpage
%% annexe dictionnaire
%\makeatletter
%    % mise en forme dictionnaire
%    \def\@oddhead{{\small{Dictionnaire\hfil\thepage\hfil\rightmark---\leftmark}}}
%    \def\@evenhead{{\small{\rightmark---\leftmark\hfil\thepage\hfil Dictionnaire}}}\clearpage
%    \makeatother
%        % inclusion dictionnaire
%        \small{\parindent=0mm{\begin{center}\Large\bfseries{Dictionnaire}\end{center}}\par\begin{multicols}{2}

\DicoEntry{AARON}\textit{, de l'hébreu «~Aharown~»~: «~haut placé~» ou «~éclairé~»}\newline
Issu de la tribu de Lévi, frère aîné de Moïse dont il fut le porte-parole. Premier grand prêtre* en Israël. Voir \vref{Ex. 4:14}~; \vref{Ex. 6:16-20} et \vref{Ex. 28}.

\DicoEntry{ABSALOM}\textit{, de l'hébreu «~'Abiyshalowm~»~: «~père de la paix~»}\newline
Fils du roi David et de Maaca, né à Hébron. Il tua Amnon son demi-frère aîné, car ce dernier avait déshonoré sa sœur Tamar. Quelques années plus tard, il conspira contre son père et se fit proclamer roi à Hébron. Il fut finalement tué par Joab, chef de l'armée de David. Voir \vref{2 S. 3:3}~; \vref{2 S. 13}~; \vref{2 S. 15-19}.

\DicoEntry{ABDIAS}\textit{, de l'hébreu «~Obadyah~»~: «~adorateur~» ou «~serviteur de Yahweh~»}\newline
Prophète de Yahweh dont le livre éponyme figure dans le Tanakh.

\DicoEntry{ABEL}\textit{, de l'hébreu «~Hebel~»~: «~souffle, vapeur~»}\newline
Deuxième fils d'Adam et Eve et première victime d'homicide de l'histoire, il fut assassiné par son frère Caïn et déclaré juste par Yahweh. Voir \vref{Ge. 4:2,8} et \vref{Mt. 23:35}.

\DicoEntry{ABIRAM}\textit{, de l'hébreu «~'Abyiram~»~: «~mon père est exalté~»}\newline
Issu de la tribu de Ruben, fils d'Eliab et frère de Dathan, il conspira avec Koré contre Moïse et Aaron. Voir \vref{No. 16:1-35}.

\DicoEntry{ABLUTION}\textit{, de l'hébreu «~rachats~»~: «~laver, baigner, nettoyer~»}\newline
Lavage de purification prescrit par la loi mosaïque et effectué avec de l'eau. Voir \vref{Ex. 29:4} et \vref{Hé. 9:10}.

\DicoEntry{ABOMINATION}\textit{, de l'hébreu «~tow'ebah~»~: «~une chose dégoûtante, abominable~» et du grec «~bdelugma~»~: «~chose folle, détestable~»}\newline
Pratique violant la loi de Yahweh et manifestant l'infidélité à Dieu comme l'idolâtrie* sous toutes ses formes, la magie ou l'homosexualité*. Voir \vref{Lé. 18:6-29}~; \vref{De. 29:17-18} et \vref{Ap. 21:27}.

\DicoEntry{ABRAM}\textit{, de l'hébreu «~Abryram~»~: «~père élevé~»}\newline
Voir ABRAHAM.

\DicoEntry{ABRAHAM}\textit{, de l'hébreu «~'Abraham~»~: «~père d'une multitude~»}\newline
Hébreu, fils de Térach, et originaire d'Ur en Chaldée. Dieu lui demanda de quitter sa terre et sa famille pour Canaan, lui promettant que sa postérité hériterait de cette terre. De sa servante Agar, lui naquit un premier fils, Ismaël, ancêtre du peuple arabe. De sa femme Sara, lui naquit Isaac qui hérita des promesses. Il mourut à cent soixante-quinze ans. Voir \vref{Ge. 12:1-7}~; \vref{Ge. 17:4-13}~; \vref{Ge. 16}~; \vref{Ge. 21:1-8} et \vref{Ge. 25:7}.

\DicoEntry{ACACIA}\textit{, de l'hébreu «~shittah~»~: «~acacia, bois d'acacia~»}\newline
Arbre épineux poussant en abondance dans la péninsule du Sinaï et dans la vallée du Jourdain, il est aussi appelé bois de Sittim. Il fut l'un des matériaux utilisés pour la fabrication des objets du culte lévitique, dont l'arche*. Voir \vref{Ex. 25:10,13,23,28}.

\DicoEntry{ACHAB}\textit{, de l'hébreu «~Ach'ab~»~: «~un frère du père~»}\newline
Fils d'Omri, il fut roi d'Israël pendant vingt-deux ans. Marié à Jézabel*, fille du roi des Sidoniens, Achab et sa femme commirent de grandes abominations* et s'opposèrent au prophète Elie*. Voir \vref{1 R. 16:29-31}~; \vref{1 R. 18:1-40} et \vref{1 R. 22:29-40}.

\DicoEntry{ADAM}\textit{, de l'hébreu «~'Adam~»~: «~être humain~» ou «~de la terre~»}\newline
Premier homme, il vécut la première partie de sa vie dans le jardin d'Eden* avec sa femme Eve. Après avoir désobéi à Dieu en goûtant le fruit de l'arbre de la connaissance du bien et du mal, ils furent chassés du jardin. Adam fut le père de Caïn, Abel et Seth. Il mourut à neuf cent trente ans. Voir \vref{Ge. 2:7-8}~; \vref{Ge. 3}~; \vref{Ge. 4:1-2}, \vref{25-26} et \vref{Ge. 5:5}.

\DicoEntry{ADONIJA}\textit{, de l'hébreu «~'Adoniyah~»~: «~Yahweh est Seigneur~»}\newline
Quatrième fils de David et de Haggith. Peu avant la mort de son père, il s'autoproclama roi tentant en vain de prendre la place qui devait revenir à Salomon. Ce dernier lui laissa la vie sauve, mais le fit tuer plus tard alors qu'il semblait encore convoiter le trône d'Israël. Voir \vref{2 S. 3:4} et \vref{1 R. 1,2}.

\DicoEntry{ADOPTION}\textit{, du grec «~huiothesia~»~: «~adoption, adoption comme fils~»}\newline
Manifestation de l'amour éternel de Dieu, l'adoption permet à tout homme de devenir par la foi enfant de Dieu. Ce privilège, autrefois réservé au peuple d'Israël, fut étendu à toutes les nations par le sacrifice de Jésus. Cette adoption est manifestée par l'Esprit de Dieu qui témoigne à l'esprit du chrétien son appartenance à Dieu~; elle inclut les avantages du fils, dont l'héritage. Voir \vref{Jn. 1:12}~; \vref{Ga. 4:7}~; \vref{Ro. 8:15-17}~; \vref{Ro. 9:4}~; \vref{Ep. 1:5,11} et \vref{1 Jn. 3:1}.

\DicoEntry{AGAR}\textit{, de l'hébreu «~Hagar~»~: «~fuite~»}\newline
Servante égyptienne de Sara que cette dernière donna à Abraham comme concubine. Elle enfanta Ismaël, fils premier-né d'Abraham. Après la naissance d'Isaac, Abraham la chassa avec son fils. Voir \vref{Ge. 16} et \vref{Ge. 21:1-18}.

\DicoEntry{AGABUS}\textit{, de l'hébreu «~Chagab~» et du grec «~Agabos~»~: «~sauterelle~»}\newline
Prophète de Yahweh suscité au temps de l'Eglise primitive. Il prophétisa une famine qui se réalisa sous le règne de l'empereur Claude. Il annonça aussi l'arrestation de Paul à Jérusalem. Voir \vref{Ac. 11:27-28} et \vref{Ac. 21:10-33}.

\DicoEntry{AGGÉE}\textit{, de l'hébreu «~Chaggay~»~: «~en fête~» ou «~né un jour de fête~»}\newline
Prophète de Yahweh d'après la captivité, dont le livre éponyme figure dans le Tanakh.

\DicoEntry{AGNEAU}\textit{, de l'hébreu «~kebes~»~: «~agneau, brebis, jeune bélier~»}\newline
Animal sacrifié et consommé lors de la Pâque des juifs. Il préfigurait Christ, l'Agneau de Dieu qui ôte le péché du monde. Voir \vref{Ex. 12:1-28} et \vref{Jn. 1:29}.

\DicoEntry{AÏ}\textit{, de l'hébreu «~'Ay~»~: «~tas de ruines~»}\newline
Ville située au sud-est de Béthel, à proximité de laquelle Abraham dressa sa tente à deux reprises. Il s'agit également de la deuxième ville que Dieu livra entre les mains de Josué après la prise de Jéricho. Voir \vref{Ge. 12:8}~; \vref{Ge. 13:3} et \vref{Jos. 8}.

\DicoEntry{ALLÉLUIA}\textit{, de l'hébreu «~allelouia~»~: «~Louez Yahweh~»}\newline
Retrouvé à maintes reprises dans les Psaumes sous la forme «~Louez Yahweh~», cette exclamation encourage à célébrer Dieu et à se réjouir en lui. Voir \vref{Ap. 19:1-6}.

\DicoEntry{ALLIANCE}\textit{, de l'hébreu «~beriyth~»~: «~pacte, alliance, engagement~»}\newline
Dieu a conclu plusieurs alliances avec les hommes (ex~: Noé, Abraham, David). On distingue communément deux alliances majeures dans les Ecritures~: l'Ancienne Alliance - conclue avec Israël au travers de Moïse - et la Nouvelle Alliance inaugurée par Jésus-Christ. Voir \vref{Ge. 9:8-17}~; \vref{Ge. 17}~; \vref{Ex. 19-34}~; \vref{2 S. 7:12-16} et \vref{Hé. 9-13}.

\DicoEntry{ALPHA ET OMEGA}\textit{}\newline
Première et dernière lettre de l'alphabet grec, la combinaison de ces deux lettres mentionnées ensemble se rapporte à l'idée que Dieu est le premier et le dernier. Jésus fut présenté plusieurs fois comme étant «~l'alpha et l'oméga~» soulignant ainsi son caractère éternel. Voir \vref{Ap. 1:8}~; \vref{Ap. 21:6} et \vref{Ap. 22:13}.

\DicoEntry{ÂME}\textit{, de l'hébreu «~nephesh~»~: «~âme, une personne, la vie, être vivant~», «~ce qui respire~», «~ce qui a une vie par le sang~» et du grec «~psuche~»~: «~le souffle, la vie, l'âme~»}\newline
L'âme correspond au sang~; elle est le siège des émotions, de la volonté humaine et de l'intelligence. Avec l'esprit et le corps, l'âme constitue l'être humain. Voir \vref{Ge. 19:20}~; \vref{Ge. 44:30}~; \vref{Lé. 17:11}~; \vref{Mt. 10:28}~; \vref{Ac. 20:10} et \vref{1 Th. 5:23}.

\DicoEntry{AMEN}\textit{, de l'hébreu «~'amen~»~: «~assuré, établi~» ou «~ainsi soit-il~!~»}\newline
Se rapportant exclusivement à ce qui est sûr, avéré et certain, ce terme est souvent utilisé comme interjection. Christ est appelé «~l'Amen~», faisant référence à la vérité qu'il incarne. Voir \vref{Jé. 28:6}~; \vref{1 Ch. 16:36}~; \vref{2 Co. 1:20} et \vref{Ap. 3:14}.

\DicoEntry{AMOUR}\textit{}\newline
Il existe plusieurs traductions et définitions du mot «~amour~» en hébreu et en grec, elles varient selon le contexte.
\\- Les termes hébreux désignant l'amour~:
\\1. «~'Ahab~»~: «~amours~»
\\Amours, amis. Voir \vref{Os. 8:9} et \vref{Pr. 5:19}.
\\2. «~'Ahabah~»~: «~amour humain, amour de Dieu pour son peuple~»
\\Amour, affection, aimer. Voir \vref{De. 7:8}~; \vref{1 S. 20:17} et \vref{Pr. 10:12}.
\\3. «~Checed~»~: «~bonté, miséricorde, fidélité~»
\\Grâce, miséricorde, compassion, affection. Voir \vref{Ge. 40:14}~; \vref{Ex. 34:7} et Nb. \vref{14:19}.
\\4. «~Yediyd~»~: «~bien-aimé~»
\\Bien-aimé, amour. Voir \vref{De. 33:12} et \vref{Es. 5:1}.
\\- Les termes grecs désignant l'amour~:
\\1. «~Agape~»~: «~amour, charité, affection, bienveillance~»
\\Amour de Dieu, amour désintéressé que doit manifester l'homme né d'en haut. Voir \vref{Jn. 15:13}~; \vref{Jn. 17:26}~; \vref{1 Co. 8:1}~; \vref{1 Co. 13:3}~; \vref{Ro. 5:5} et \vref{1 Jn. 4:8}.
\\2. «~Eros~»~: «~l'amour qui prend~»
\\Amour dans la dimension sexuelle.
\\3. «~Phileo~»~: «~aimer, montrer des signes d'amour~»
\\Amour filial. Voir \vref{Jn. 21:17}~; \vref{1 Co. 16:22}.
\\4. «~Philadelphia~»~: «~amour fraternel~»
\\Amour des frères et sœurs d'une même famille, amour des chrétiens les uns pour les autres. Voir \vref{1 Th. 4:9}~; \vref{Ro. 12:10} et \vref{Hé. 13:1}.
\\5. «~Storge~»~: «~amour filial~»
\\Amour familial, affection naturelle. Voir \vref{Ro. 1:31}.

\DicoEntry{AMOS}\textit{, de l'hébreu «~'Amowc~»~: «~fardeau, porteur de fardeaux~»}\newline
Originaire de Tekoa en Juda, prophète de Yahweh dont le livre éponyme figure dans le Tanakh.

\DicoEntry{AMMONITES}\textit{, de l'hébreu «~'Ammown~»~: «~appartenant à la nation~»}\newline
Peuple issu de Ben-Ammi, né de l'inceste entre Lot et sa fille cadette. Ils furent ennemis d'Israël. Voir \vref{Ge. 19:30-38} et \vref{Ez. 25:1-7}.

\DicoEntry{ANAKIM}\textit{, de l'hébreu «~'Anaqiy~»~: «~au long cou~»}\newline
Descendants d'Anak, race de géants habitant Canaan avant sa conquête par le peuple d'Israël. Ils furent vaincus par Josué et Caleb qui hérita d'une partie de leur territoire. Voir \vref{No. 13:28-33}~; \vref{De. 9:1-3}~; \vref{Jos. 11:21-22} et \vref{Jos. 14:6-15}.

\DicoEntry{ANANIAS}\textit{, de l'hébreu «~Chananyah~»~: «~Dieu a été miséricordieux~»}\newline
1. Chrétien ayant vendu un champ avec sa femme Saphira et ayant fait croire qu'ils avaient donné la totalité du prix rapporté pour l'Eglise alors qu'ils en avaient secrètement gardé une partie. Ce mensonge les conduisit tous deux à la mort. Voir \vref{Ac. 5:1-10}.
\\2. Homme pieux vivant à Damas que le Seigneur envoya imposer les mains à Saul qui venait de se convertir afin qu'il recouvre la vue. C'est également lui qui le baptisa. Voir \vref{Ac. 9:10-18} et \vref{Ac. 22:12-16}.

\DicoEntry{ANATHÈME}\textit{, du grec «~anathema~»~: «~tout ce qui est livré au malheur~»}\newline
Terme désignant une personne ou une chose maudite, vouée à la destruction. Voir \vref{Ga. 1:8} et \vref{1 Co. 12:3}.

\DicoEntry{ANCIENS}\textit{, de l'hébreu «~zaqen~»~: «~vieux, aîné, de ceux qui ont de l'autorité~» et du grec «~presbuteros~»~: «~ayant de l'âge~»}\newline
Chez les juifs, il s'agissait des chefs de famille ou de clan qui représentaient le peuple dans les affaires religieuses et civiles. Voir \vref{Ex. 3:16}~; \vref{Lé. 4:15} et \vref{De. 31:28}. Sous la Nouvelle Alliance, les églises de la Galatie avaient élu des anciens («~presbuteros~») pour prendre soin des frères et sœurs. Il s'agit d'un terme relatif aux personnes ayant de l'âge et non à la fonction d'évêque*. Voir \vref{Ac. 14:23}~; \vref{1 Ti. 5:17}~; \vref{Tit. 1:5-9} et \vref{1 Pi. 5:1-5}.

\DicoEntry{ANDRÉ}\textit{, du grec «~Andreas~»~: «~virilité~»}\newline
Frère de Simon Pierre, originaire de Bethsaïda en Galilée, et pêcheur de métier. Il devint l'un des douze apôtres de Jésus-Christ. Voir \vref{Mt. 10:2}~; \vref{Mc. 1:16-17} et \vref{Jn. 1:40}.

\DicoEntry{ANGE}\textit{, de l'hébreu «~mal'ak~» et du grec «~aggelos~»~: «~messager, envoyé~»}\newline
Etre spirituel au service de Dieu pouvant prendre une forme humaine. Les anges sont au service de Yahweh pour des missions spécifiques au ciel ou sur la terre. Ils peuvent avoir une fonction de messager, protecteur ou combattant. Voir \vref{Da. 10:10-13}~; \vref{Lu. 1:26-38} et \vref{Ap. 12:7}.

\DicoEntry{ANNE}\textit{, de l'hébreu «~Channah~»~: «~grâce, faveur~»}\newline
1. Une des deux femmes d'Elkana. Stérile, elle pria Yahweh de lui accorder un fils qu'elle lui consacrerait. Elle enfanta ainsi Samuel qui entra au service de Yahweh dès son plus jeune âge. Voir \vref{1 S. 1,2}.
\\2. Fille de Phanuel de la tribu d'Aser, prophétesse. Veuve, elle servait le Seigneur nuit et jour dans le temple. Elle rencontra Jésus nourrisson, lorsqu'il fut amené au temple pour y être présenté à Dieu. Voir \vref{Lu. 2:36-38}.
\\3. Grand prêtre, beau-père de Caïphe*. Il participa à la conspiration qui mena Jésus à la croix. Voir \vref{Lu. 3:2} et \vref{Jn. 18:13}.

\DicoEntry{ANTICHRIST}\textit{, du grec «~antichristos~»~: «~l'adversaire du Messie~»}\newline
Aussi appelé «~homme impie~» et «~fils de la perdition~», personnage dont l'apparition se fera avant le retour glorieux du Seigneur. Il dominera le monde avant d'être vaincu par Christ. Voir \vref{2 Th. 2:1-4}~; \vref{2 Jn. 1:7} et \vref{Ap. 19:19-21}.

\DicoEntry{ANTIOCHE}\textit{, du grec «~Antiocheia~»~: «~rapide comme un char~»}\newline
Capitale de la Syrie, elle fut fondée en 300 av. J.-C. par Séleucus Nicator (358-281 av. J.-C.) qui la baptisa du nom de son père Antiochus. Cette ville accueillit des chrétiens en exil~; l'évangile y fut ainsi annoncé aux juifs puis aux Grecs et un grand nombre de personnes se convertirent. Barnabas et Paul y demeurèrent une année durant laquelle ils enseignèrent la parole. C'est à Antioche que les disciples furent appelés chrétiens pour la première fois. Voir \vref{Ac. 11:19-26}.

\DicoEntry{APIS}\textit{}\newline
Divinité égyptienne symbolisant la force et la fertilité. Il est représenté sous la forme d'un veau d'or ou d'un homme à tête de taureau dont les cornes entourent un disque solaire. Les Hébreux se corrompirent plusieurs fois en le vénérant. Voir \vref{Ex. 32:1-6} et \vref{1 R. 12:28-30}.

\DicoEntry{APOCALYPSE}\textit{, du grec «~apokalupsis~»~: «~mettre à nu, révélation d'une vérité, action de révéler~»}\newline
Dernier livre de la Bible écrit par Jean, ce récit comporte une révélation de la gloire de Jésus-Christ et raconte les derniers événements de l'histoire de l'humanité jusqu'à l'avènement de la Nouvelle Jérusalem.

\DicoEntry{APOLLOS}\textit{, du grec «~Apollos~»~: «~donné par Apollon~»}\newline
Juif érudit d'Alexandrie ayant une très bonne connaissance des Ecritures et enseignant avec exactitude au sujet de Jésus. Sa rencontre avec Aquilas et Priscille lui permit d'aller plus en profondeur dans la Parole et d'annoncer avec plus de force, notamment aux Juifs, que Jésus est le Messie en se basant sur les écrits du Tanakh. Il réalisa plusieurs voyages missionnaires notamment à Corinthe. Voir \vref{Ac. 18:24-28}, \vref{1 Co. 3:5-6} et \vref{1 Co. 16:12}.

\DicoEntry{APOSTASIE}\textit{, du grec «~apostasia~»~: «~action de s'éloigner de, désertion, défection~»}\newline
Abandon de la foi en Jésus-Christ et de la saine doctrine* se manifestant sous deux formes principales. Certaines personnes abandonnent ouvertement la foi, la communion avec Dieu et l'assemblée des saints. D'autres continuent de fréquenter les assemblées chrétiennes, mais ont laissé la saine doctrine pour s'attacher à des doctrines séductrices. Voir \vref{Mt. 24:11-12}~; \vref{2 Th. 2:3}~; \vref{1 Ti. 4:1-3}~; \vref{2 Pi. 2:1-3}~; \vref{2 Ti. 3:1-8}~; \vref{Jud. 1:17-19} et \vref{1 Jn. 4:1}.

\DicoEntry{APÔTRE}\textit{, du grec «~apostolos~»~: «~envoyé en avant, messager, ambassadeur~»}\newline
Lors de son service terrestre, Jésus choisit douze apôtres qu'il forma pour continuer l'œuvre après lui. Plusieurs autres apôtres furent suscités au temps de l'Eglise primitive, notamment Paul et Jacques, frère du Seigneur, qui avec Jean et Pierre furent les principaux instruments utilisés pour poser les fondements de la doctrine de l'Eglise. Le service apostolique existe encore aujourd'hui, mais la mission des apôtres actuels n'est pas d'écrire des épîtres, car la fondation a déjà été posée. Leur travail aujourd'hui consiste davantage à enseigner et veiller à ce que le fondement demeure. Voir \vref{Mc. 3:14}~; \vref{Ac. 15}~; \vref{Ga. 2:19}~; \vref{Ro. 1:1}~; \vref{Ep. 2:20} et \vref{Ep. 4:11}.

\DicoEntry{AQUILAS}\textit{, du latin «~Aquilas~»~: «~un aigle~» et PRISCILLE, du latin «~Priscilla~»~: «~petite vieille~»}\newline
Couple de Juifs ayant accepté l'évangile. Après avoir été chassés de Rome, ils s'installèrent à Corinthe où ils hébergèrent Paul à son arrivée et devinrent par la suite compagnons d'œuvre de ce dernier. Ils participèrent à plusieurs voyages missionnaires, notamment à Ephèse où ils enseignèrent Apollos*. Voir \vref{Ac. 18:1-3,18,24-26} et \vref{Ro. 16:3-5}.

\DicoEntry{ARBRE}\textit{, de l'hébreu «~'ets~»~: «~arbre, bois~»}\newline
Organisme vivant porteur de semence produisant des feuilles et des fruits selon les espèces. Lors de la création, Dieu créa différents arbres dont les fruits furent donnés pour nourrir l'homme et également deux arbres spécifiques placés au milieu du jardin d'Eden.

\DicoEntry{ARBRE DE LA CONNAISSANCE DU BIEN ET DU MAL}\textit{}\newline
Arbre dont le fruit contenait la connaissance du bien et du mal. Dieu interdit la consommation de ce dernier à l'homme sous peine de mort, mais Adam et Eve transgressèrent le commandement. C'est ainsi que le péché et la mort régnèrent sur l'humanité. Voir \vref{Ge. 2:17}~; \vref{Ge. 3:1-6} et \vref{Ro. 5:12}.

\DicoEntry{ARBRE DE VIE}\textit{}\newline
Arbre dont la consommation donne la vie éternelle. Après la chute d'Adam et Eve, Dieu les chassa du jardin pour les empêcher d'y accéder. L'arbre de vie se trouve dans la ville sainte, la Nouvelle Jérusalem~; ses feuilles servent à la guérison des nations. Voir \vref{Ge. 3:22-24} et \vref{Ap. 22:2,14,19}.

\DicoEntry{ARC-EN-CIEL}\textit{, de l'hébreu «~qesheth~»~: «~arc~»}\newline
Signe de l'alliance* que Dieu conclut avec Noé et les générations qui le suivraient suite au déluge*. Cette alliance stipulait que Yahweh ne détruirait plus les hommes par les eaux. Voir \vref{Ge. 9:12-17}.

\DicoEntry{ARCHANGES}\textit{, du grec «~archaggelos~»~: «~chef des anges~»}\newline
Catégorie d'anges* ayant un rang et une dignité plus élevés que les autres. Voir \vref{1 Th. 4:16} et \vref{Jud. 1:9}.

\DicoEntry{ARCHE DE NOÉ}\textit{, de l'hébreu «~tebah~»~: «~arche, vaisseau, coffre~»}\newline
Embarcation construite par Noé pour le sauver lui, sa famille ainsi que les animaux, du déluge* qui allait s'abattre sur la terre. Voir \vref{Ge. 6:8-16}~; \vref{Mt. 24:37-39} et \vref{Lu. 17:26-27}.

\DicoEntry{ARCHE DU TEMOIGNAGE ou DE L'ALLIANCE}\textit{, de l'hébreu «~'arown~»~: «~arche, coffre, cercueil~»}\newline
Coffre rectangulaire en bois d'acacia* recouvert d'or pur, contenant les tables de l'alliance, la verge d'Aaron et une urne contenant un échantillon de la manne. Construite selon le modèle que Moïse avait reçu au mont Sinaï, elle était couverte par le propitiatoire*. L'arche fut placée dans le Saint des saints du tabernacle*, puis du temple*. Voir \vref{Ex. 25:10-22}~; \vref{1 R. 8:6}~; \vref{2 R. 25:8-9} et \vref{Hé. 9:4}.

\DicoEntry{ARTAXERXÈS}\textit{, (règne~: 465 av. J.-C.- 424 av. J.-C.), du persan «~Artachshashta~»~: «~celui qui fait régner la loi sacrée~»}\newline
Fils d'Assuérus*, roi de Perse. Il autorisa Esdras à retourner à Jérusalem avec des prêtres et des Lévites pour effectuer le sacerdoce dans le temple et faire respecter la loi de Yahweh. Voir \vref{Esd. 7:11-28}.

\DicoEntry{ASAPH}\textit{, de l'hébreu «~'Acaph~»~: «~celui qui rassemble, collecteur~»}\newline
Lévite et chef des chantres sous David, il participa au transfert de l'arche* à Jérusalem et écrivit certains psaumes. Voir \vref{1 Ch. 15:16-19} et \vref{1 Ch. 16:4-7}.

\DicoEntry{ASER}\textit{, de l'hébreu «~'Asher~»~: «~heureux~»}\newline
Fils de Jacob et de Zilpa, servante de Léa, il est le père de la tribu d'Aser. Voir \vref{Ge. 30:13}.

\DicoEntry{ASHERAH}\textit{, de l'hébreu «~'Asherah~»~: «~pieu sacré~».}\newline
Voir commentaire en \vref{Ex. 34:13}.

\DicoEntry{ASSUÉRUS ou XERXÈS Ier}\textit{, (485 av. J.-C. – 465 av. J.-C.), du persan «~'Achashverowsh~»~: «~je serai silencieux et pauvre~».}\newline
Père d'Artaxerxès, roi de Perse et époux d'Esther*. Voir le livre d'Esther.

\DicoEntry{ASTARTE}\\textit{, de l'hébreu «~Ashtoreth~»~: «~étoile~».}\newline
Voir commentaire en \vref{Jg. 2:13}.

\DicoEntry{AUTEL}\textit{, de l'hébreu «~mizbeach~»~: «~autel~»}\newline
Table généralement façonnée avec des monticules de pierres ou en terre et élevée spécialement pour offrir des holocaustes* et des sacrifices en l'honneur de Dieu. Voir \vref{Ge. 12:7}~; \vref{Ge. 35:7}~; \vref{Ex. 20:24-26} et \vref{Ex. 30:1-8}.

\DicoEntry{BAAL}\textit{, de l'hébreu «~Ba'al~»~: «~maître, possesseur, seigneur~»}\newline
Dieu primaire des Phéniciens et des Cananéens auquel les Israélites s'attachèrent à plusieurs reprises pour l'adorer. Voir \vref{No. 25:3}~; \vref{Jg. 2:11} et \vref{1 R. 18:21}. Voir aussi commentaire en \vref{Jg. 2:11}.

\DicoEntry{BABEL ou BABYLONE}\textit{, de l'hébreu «~Babel~»~: «~confusion (par le mélange)~»}\newline
Ville de Mésopotamie* située sur l'Euphrate, capitale de la Babylonie. Les hommes y entreprirent la construction de la tour de Babel. Cependant, Yahweh confondit leur langage et les dispersa sur toute la terre. Voir \vref{Ge. 10:8-10} et \vref{Ge. 11:1-9}.

\DicoEntry{BALAAM}\textit{, de l'hébreu «~Bil'am~»~: «~sans peuple~», «~dévorant~»}\newline
Prophète de Yahweh ayant vécu pendant la marche d'Israël dans le désert, il fut séduit par Balak, roi de Moab, qui lui proposa de maudire Israël contre de généreux présents. Son témoignage a été utilisé plusieurs fois pour avertir les enfants de Dieu des scandales* dont ils pourraient être la cause en suivant la voie de la cupidité. Voir \vref{No. 22-24}~; \vref{No. 31:8}~; \vref{Jud. 1:11} et \vref{Ap. 2:14}.

\DicoEntry{BALAK}\textit{, de l'hébreu «~Balaq~»~: «~gaspilleur, dévastateur~»}\newline
Roi de Moab, il essaya de convaincre Balaam* de maudire Israël qu'il redoutait. Voir \vref{No. 22-24}.

\DicoEntry{BANNIÈRE}\textit{, de l'hébreu «~nec~»~: «~quelque chose de levé, étendard, signal, enseigne~»}\newline
Drapeau, étendard élevé en signe d'appartenance à ce qu'il représente. Moïse bâtit un autel du nom de Yahweh-Nissi~: «~Yahweh ma bannière~». Voir \vref{Ex. 17:15}~; \vref{Es. 11:10,12} et \vref{Ps. 60:4}.

\DicoEntry{BAPTÊME}\textit{, du grec «~baptizo~»~: «~plonger, immerger, purifier en plongeant~»}\newline
On distingue trois types de baptêmes dans les Ecritures (\vref{Mt. 3:11})~:
\\1. le baptême d'eau~: acte suivant la conversion par lequel une personne est immergée dans l'eau - symbolisant la mort et la résurrection en Jésus-Christ. Il s'agit selon Pierre de l'«~engagement d'une bonne conscience envers Dieu~». Voir \vref{Ac. 2:38}~; \vref{Ac. 16:30-33}~; \vref{Col. 2:12-13} et \vref{1 Pi. 3:21}.
\\2. le baptême du Saint-Esprit~: lors de la naissance d'en haut, gage que le Seigneur donne au nouveau converti par l'envoi du Saint-Esprit. Voir \vref{Jn. 3:5-6}~; \vref{Tit. 3:4-7} et \vref{Ep. 1:13}.
\\3. le baptême de feu~: symbole des souffrances que Christ a endurées à la croix et par lesquelles tous les chrétiens sont appelés à passer pour être purifiés. Voir \vref{Mc. 10:35-39}~; \vref{Lu. 12:50}~; \vref{1 Pi. 1:6-9} et \vref{1 Pi. 4:12-13}.

\DicoEntry{BARAK}\textit{, de l'hébreu «~Baraq~»~: «~éclairs, foudre~»}\newline
Fils d'Abinoam, issu de la tribu de Nephtali, il vécut en Israël au temps des juges. Encouragé et accompagné par Débora, il battit l'armée de Jabin, roi de Canaan. Voir \vref{Jg. 4}.

\DicoEntry{BARTHÉLÉMY}\textit{, du grec «~Bartholomaios~»~: «~fils de Tolmaï~»}\newline
Un des douze apôtres de Jésus. Voir \vref{Mt. 10:3}.

\DicoEntry{BARTIMÉE}\textit{, du grec «~Bartimaios~»~: «~fils de Timée~»}\newline
Fils de Timée, mendiant aveugle que Jésus guérit suite à ses cris de supplications sur la route de Jéricho. Voir \vref{Mc. 10:46-52}.

\DicoEntry{BATH-SCHEBA}\textit{, de l'hébreu «~Bath-Sheba`~»~: «~fille d'un serment~»}\newline
Fille d'Eliam, femme d'Urie* le Héthien que David fit mourir après l'avoir mise enceinte. Elle devint la femme de David et fut la mère de Salomon. Voir \vref{2 S. 11:3-5,26-27} et \vref{2 S. 12:24-25}.

\DicoEntry{BEELZÉBUL}\textit{, de l'hébreu «~Ba'al-Zebuwb~»~: «~seigneur des mouches~»}\newline
Divinité adorée par les Philistins et considérée comme le prince des démons. Voir \vref{2 R. 1:2-6} et \vref{Mc. 3:22-26}.

\DicoEntry{BÉLIAL}\textit{, de l'hébreu «~Beliya'al~»~: «~indignité~»}\newline
Symbolisant l'infidélité, la méchanceté et la perversité, il s'agit d'un autre nom de Satan. Voir \vref{De. 15:9}~; \vref{1 S. 1:16}~; \vref{2 Co. 6:15}.

\DicoEntry{BÉNÉDICTION}\textit{, de l'hébreu «~barak~», «~berakah~», et du grec «~eulogia~»~: «~louange~»}\newline
Parole au travers de laquelle le Seigneur annonce sa grâce sur la vie d'une personne ou d'un peuple~; les bontés liées à la bénédiction sont cependant conditionnées par l'obéissance du bénéficiaire. Sous l'Ancienne Alliance, les pères avaient coutume de bénir leurs enfants~; la bénédiction se manifestait souvent par la prospérité matérielle, la fécondité et la santé. La bénédiction est la marque du chrétien qui voit avec un œil spirituel la faveur de Dieu dans sa vie et qui bénit Dieu dans toutes les circonstances. Voir \vref{Ge. 49:1-28}~; \vref{De. 28:1-14}~; \vref{Ps. 103:1-2} et \vref{Ep. 1:3}.

\DicoEntry{BENJAMIN}\textit{, de l'hébreu «~Binyamiyn~»~: «~fils de ma main droite~»}\newline
Dernier fils de Jacob et Rachel~; sa mère mourut en lui donnant naissance. Il est l'ancêtre de la tribu de Benjamin. Voir \vref{Ge. 35:16-18} et \vref{Ge. 49:27}.

\DicoEntry{BÊTE}\textit{, de l'araméen «~cheyva´~» et du grec «~therion~»~: «~bête, animal~»}\newline
Dans les récits à caractère apocalyptique, les bêtes sont des animaux symbolisant les puissances politiques. Voir \vref{Da. 7} et \vref{Ap. 13,17}.

\DicoEntry{BÉTHANIE}\textit{, du grec «~Bethania~»~: «~maison des dattes non mûres~», «~maison de l'affligé~»}\newline
Village proche de Jérusalem, près de la Montagne des Oliviers, où vivaient Simon le lépreux, Marthe*, Marie* et Lazare* que Jésus ressuscita des morts. Voir \vref{Mc. 11:1}~; \vref{Mc. 14:3} et \vref{Jn. 11:1}.

\DicoEntry{BÉTHEL}\textit{, de l'hébreu «~Beyth-'El~»~: «~maison de Dieu~»}\newline
Ville cananéenne située à l'occident de Aï. Autrefois appelé Luz - mais renommé par Jacob quand il y eut la visitation de Yahweh - Béthel devient la possession de la tribu d'Ephraïm lors de la conquête de Canaan conduite par Josué. Elle était connue pour être un lieu d'adoration où on y rendait un culte à Yahweh. Malheureusement suite au schisme d'Israël - et notamment sous le règne de Jéroboam, roi de Juda - elle devient un lieu d'abomination. C'est Josias son successeur qui, désirant marcher avec Yahweh, y ôta les faux dieux, rétablissant ainsi le culte en l'honneur du Dieu d'Israël. (\vref{Ge. 28:10-22}~; \vref{Ge. 31:13}~; \vref{1 R. 12:26-32}~; \vref{2 R. 23:1-15})

\DicoEntry{BETHLEHEM}\textit{, de l'hébreu «~Beyth Lechem~»~: «~maison du pain~»}\newline
Ville de Juda, lieu de naissance de David et de Jésus-Christ. Voir \vref{1 S. 16}~; \vref{Mt. 2:16} et \vref{Lu. 2:4-7}.

\DicoEntry{BIBLE}\textit{, du grec «~biblia~»~: «~livres~»}\newline
Aussi appelée «~Parole de Dieu~», recueil de livres inspirés de Dieu et utiles pour enseigner, convaincre, corriger et instruire dans la justice. Voir \vref{2 Ti. 3:16}.

\DicoEntry{BLASPHÈME}\textit{, de l'hébreu «~na'ats~»~: «~repousser, mépriser, rejeter~» et du grec «~blasphemia~»~: «~discours impie et injurieux envers Dieu~»}\newline
Parole outrageante ou insultante envers Dieu. Voir \vref{2 S. 12:14} et \vref{Ap. 16:9}.

\DicoEntry{BLASPHÈME CONTRE LE SAINT-ESPRIT}\textit{}\newline
Voir commentaire \vref{Mt. 12:22-32}.

\DicoEntry{BOAZ}\textit{, de l'hébreu «~Bo`az~»~: «~en lui est la force~»}\newline
Fils de Salmon et arrière-grand-père du roi David, il épousa Ruth la Moabite. Voir \vref{Ru. 4:13} et \vref{Mt. 1:1-6}.

\DicoEntry{BREBIS}\textit{}\newline
Femelle du bélier, c'est l'animal pour qui le berger donne sa vie. Elle est le symbole du véritable disciple qui n'obéit qu'à la voix de son Maître et qui se laisse conduire et choyer par Jésus, le bon berger. Voir \vref{Jn. 10:1-16}.

\DicoEntry{CAIN}\textit{, de l'hébreu «~Qayin~»~: «~possession~», «~artisan, forgeron~»}\newline
Fils aîné d'Adam et Eve, il fut l'auteur du premier homicide en tuant son frère Abel. Il engendra Lémec, premier polygame de l'histoire. Voir \vref{Ge. 4:1-8,16-19}.

\DicoEntry{CAÏPHE}\textit{, du grec «~Kaiaphas~»~: «~avenant, pierre~»}\newline
Grand prêtre nommé par Valerius Gratus, gouverneur de Judée de 15 à 26 ap. J.-C. Caïphe exerça sa fonction de 18 à 36. N'ayant pas reconnu en Christ le Messie, il déclara néanmoins qu'il était avantageux qu'un seul homme meure pour le peuple et participa à la condamnation à mort de Jésus. Voir \vref{Mt. 26:3,57-66}~; \vref{Jn. 11:47-53} et \vref{Jn. 18:12-14}.

\DicoEntry{CALEB}\textit{, de l'hébreu «~Kaleb~»~: «~chien~»}\newline
Fils de Jephunné, issu de la tribu de Juda, il fut l'un des espions envoyés pour explorer le pays de Canaan. Avec Josué, il fut le seul, parmi la génération sortie d'Egypte, à entrer dans la terre promise. Voir \vref{No. 13:1-6} et \vref{No. 14:22-30}.

\DicoEntry{CALENDRIER HEBRAÏQUE}\textit{}\newline
Nisan (ou Abib) = Mars~; Iyyar (ou Ziv) = Avril~; Sivan = Mai~; Thammuz = Juin~; Ab = Juillet~; Elul = Août~; Tisri (ou Ethanim) = Septembre~; Marchesvan (ou Bul) = Octobre~; Chislev (ou Kisleu) = Novembre~; Tébeth = Décembre~; Schebat = Janvier~; Adar = Février

\DicoEntry{CAMP}\textit{, de l'hébreu «~machaneh~»~: «~campement, camp~»}\newline
Lieu de stationnement temporaire d'un groupement civil ou militaire. Voir \vref{Ge. 32:2} et \vref{Ex. 14:19}.

\DicoEntry{CANAAN}\textit{, de l'hébreu «~Kena'an~»~: «~terre basse~», «~marchand~»}\newline
Fils de Cham. Ses descendants occupèrent la région éponyme qui correspond plus ou moins aujourd'hui aux territoires réunissant la Palestine, l'État d'Israël, l'ouest de la Jordanie, le sud du Liban et l'ouest de la Syrie. Ce territoire correspondait également à la terre promise par Dieu aux Israélites dont ils prirent possession sous la conduite de Josué. Voir \vref{Ge. 9:18}~; \vref{Jos. 6-21} et \vref{Ac. 13:19}.

\DicoEntry{CÉSAR, Jules}\textit{, (100 av. J.-C - 44 av. J.-C.) du latin «~kaisar~»~: «~séparé~», «~chef~»}\newline
Général romain. Son nom devint par la suite celui de certains empereurs romains. Dans les Ecritures, César symbolise également les autorités séculières. Voir \vref{Mt. 22:21}.

\DicoEntry{CESARÉE de Philippes}\textit{, du grec «~Kaisereia~»~: «~appartenant à César~»}\newline
Située près des sources du Jourdain, territoire qui doit son nom à l'empereur Tibère. C'est dans cette contrée que Pierre reconnut en Jésus le Messie, le Fils du Dieu vivant. Voir \vref{Mt. 16:13-17}.

\DicoEntry{CHAIR}\textit{, du grec «~sarx~»~: «~la chair, le corps, la nature sensuelle de l'homme, la nature animale~»}\newline
Selon le contexte, désigne le corps humain, l'être humain ou la nature humaine conduite par le péché*. Voir \vref{Lu. 3:6}~; \vref{Lu. 24:39}~; \vref{Jn. 17:2}~; \vref{Ga. 5:16-21}~; \vref{Ro. 8:5-9} et \vref{Ep. 2:3}.

\DicoEntry{CHALDÉE}\textit{, de l'hébreu «~Kasdiy~»~: «~briseurs de mottes~», «~comme des démons~»}\newline
Région située au sud de la Mésopotamie dont Abraham est originaire. Voir \vref{Ge. 11:28}.

\DicoEntry{CHAM}\textit{, de l'hébreu «~Cham~»~: «~chaud, bouillant~»}\newline
Fils de Noé et père de Canaan qui fut maudit par Noé. Voir \vref{Ge. 9:18-27}.

\DicoEntry{CHARAN}\textit{, de l'hébreu «~Charan~»~: «~montagnard~», «~route, caravane~»}\newline
Région proche d'Ur en Chaldée où Abraham séjourna jusqu'à la mort de son père Térach. Voir \vref{Ge. 11:31} et \vref{Ge. 12:4}.

\DicoEntry{CHEMIN DE SABBAT}\textit{}\newline
Selon la loi de Moïse, distance maximum que les juifs peuvent parcourir de leur demeure le jour du sabbat* (cf. tableau des mesures et.distances). Voir \vref{Ac. 1:12}.

\DicoEntry{CHÉRUBINS}\textit{, de l'hébreu «~keruwb~»~: «~être angélique, chérubin~»}\newline
Catégorie d'anges portant et.ou gardant la gloire de Dieu. Yahweh en avait placé à l'entrée du jardin d'Eden pour empêcher l'homme d'y accéder. Deux chérubins sur lesquels Dieu siégeait étaient représentés sur le propitiatoire*. Avant sa chute, Satan était un chérubin protecteur. Voir \vref{Ge. 3:24}~; \vref{Ex. 25:17-20}~; \vref{Es. 37:16} et \vref{Ez. 28:14}.

\DicoEntry{CHRÉTIEN}\textit{, du grec «~christanos~»~: «~de Christ~», «~petit christ~», «~comme Christ~»}\newline
Comme son étymologie le suggère, le chrétien appartient à Christ, dont il a la nature et à qui il ressemble. Il est donc un disciple* de Jésus-Christ qui suit son enseignement et le met en pratique. Ce terme fut employé pour la première fois à Antioche. Voir \vref{Ac. 11:26}.

\DicoEntry{CHRIST}\textit{, du grec «~christos~» et de l'hébreu «~mashiyach~»~: «~oint~»}\newline
Souvent accolé au nom de Jésus*, ce terme suggère que ce dernier est l'oint* de Dieu, le Messie tant attendu. Jésus annonça l'émergence de faux christs (= faux ouvriers de Christ) à la fin des temps. Voir \vref{Ro. 1:1}~; \vref{Mt. 16:15-16}~; \vref{Mt. 24:24} et \vref{Mc. 13:22-23} et \vref{Hé. 1:9}.

\DicoEntry{CIRCONCISION}\textit{, de l'hébreu «~muwlah~»~: «~circoncision~: couper autour~»}\newline
Section et ablation du prépuce. En signe d'alliance, Dieu ordonna à Abraham de circoncire tous les mâles de sa maison~; les enfants d'Israël ont perpétré cette pratique. Sous la Nouvelle Alliance, la circoncision requise est celle du cœur. Voir \vref{Ge. 17:9-14}~; \vref{Lu. 1:59}~; \vref{1 Co. 7:19} et \vref{Ro. 2:25-29}.

\DicoEntry{CLAUDE}\textit{, (10 av. J.-C. – 54 ap. J.-C.), du grec «~Klaudios~»~: «~boiteux~»}\newline
Fils de Nero Claudius Drusus (38 av. J.-C. – 9 av. J.-C.). Empereur romain qui régna de 41 à 54 ap. J.-C.~; il chassa les Juifs de Rome, parmi lesquels Aquilas et Priscille. Voir \vref{Ac. 18:2}.

\DicoEntry{CLERGÉ}\textit{, du grec «~klêrikos~»~: «~homme d'église~»}\newline
Au sein de l'Eglise catholique, corps séparé des fidèles ayant une fonction gouvernante~; ses membres sont appelés les clercs ou les ecclésiastiques. Ils accèdent à leur position par le sacrement de l'ordre (ou ordination*) qui comporte trois classes~: les diacres, les prêtres et les évêques.

\DicoEntry{CLÉRICALISME}\textit{, dérivé de clérical~: «~dévoué aux intérêts du clergé~»}\newline
Tendance en vertu de laquelle le clergé sort du domaine religieux pour se mêler des affaires publiques et politiques afin d'y exercer une influence et faire prédominer ses idées.

\DicoEntry{CŒUR}\textit{, de l'hébreu «~lebab~»~: «~homme intérieur, volonté, cœur, partie interne, pensée~»}\newline
Organe permettant la circulation du sang, les Ecritures définissent le cœur comme un grand abîme. Siège des émotions et des pensées intimes, il peut être une bonne ou une mauvaise source. Voir \vref{Ge. 20:6}~; \vref{Lé. 19:17}~; \vref{De. 4:29}~; \vref{1 S. 12:24} et \vref{Mc. 7:21}.

\DicoEntry{COLOSSES}\textit{, du grec «~Kolossai~»~: «~monstruosités~»}\newline
Située en Asie Mineure, ville de Phrygie se trouvant à environ deux cents kilomètres d'Ephèse. Il s'y trouvait une église à qui Paul écrivit une lettre qui figure dans le canon biblique.

\DicoEntry{COMMUNION}\textit{, du grec «~koinonia~»~: «~ce qui est commun à plusieurs personnes, association, union~»}\newline
Le disciple* de Christ est appelé à vivre deux types de communion. Il doit tout d'abord être en communion intime avec Dieu puis avec d'autres membres du corps de Christ pour vivre la communion fraternelle. Voir \vref{Ps. 133}~; \vref{Ac. 2:42}~; \vref{2 Co. 13:11-13} et \vref{1 Jn. 1:3}.

\DicoEntry{CONCILE}\textit{, du latin «~concilium~»~: «~assemblée~»}\newline
Assemblée d'évêques de l'Eglise catholique (également connue sous l'appellation «~pères de l'Eglise catholique~») réunis dans le but de définir les règles de la foi chrétienne. Cette pratique va à l'encontre du message de Christ puisqu'il a strictement condamné la modification du message qu'il a lui-même prêché et confié aux apôtres*. Voir \vref{Mt. 5:18} et \vref{Ga. 1:8-9}.

\DicoEntry{CONFESSION}\textit{, du grec «~exomologeo~»~: «~confesser, professer, reconnaître ouvertement~»}\newline
On peut confesser des péchés pour exposer les ténèbres ou le nom du Seigneur pour le louer et annoncer la vérité. Voir \vref{Mc. 1:5}~; \vref{Ac. 19:18} et \vref{Ph. 2:11}.

\DicoEntry{CONVERSION}\textit{, du grec «~epistrepho~»~: «~action de se retourner, de se tourner vers~»}\newline
Fruit d'une sincère repentance, la conversion est la décision de se tourner vers Christ et de se détourner des œuvres des ténèbres. Voir \vref{Ac. 26:20}~; \vref{Ga. 4:9}~; \vref{2 Co. 3:16} et \vref{1 Pi. 2:25}.

\DicoEntry{CONVOITISE}\textit{, du grec «~epithumia~»~: «~désir, convoitise, luxure~»}\newline
Précédant l'acte du péché, désir amorcé par les sens humains et lié à la soif de posséder ce qui est défendu et ce que le monde offre. Voir \vref{Ja. 1:14-15} et \vref{1 Jn. 2:15-17}.

\DicoEntry{CORINTHE}\textit{, du grec «~Korinthos~»~: «~rassasié~»}\newline
Dans l'Antiquité, Corinthe, capitale de l'Achaïe, était la ville la plus prospère et la plus puissante de Grèce. Située sur un isthme séparant la mer Egée de la mer Ionienne, Corinthe était au carrefour de l'Asie et de l'Italie et constituait un véritable centre commercial où les produits orientaux et occidentaux se croisaient. Paul demeura au moins un an et six mois à Corinthe, durée pendant laquelle il enseigna la parole de Dieu. Il écrivit par la suite deux lettres aux saints de cette ville qu'on retrouve dans le canon biblique.

\DicoEntry{CORNEILLE}\textit{, du grec «~Kornelios~»~: «~d'une corne~»}\newline
Centenier romain juste et craignant Dieu. Il vivait à Césarée où Simon Pierre fut envoyé pour lui annoncer la Parole. Au travers de l'expérience de Corneille, Dieu confirma que le salut était pour toutes les nations. Voir \vref{Ac. 10}.

\DicoEntry{COURONNE}\textit{, du grec «~stephanos~»~: «~couronne, une marque de rang royal, récompense de la justice, ornement~»}\newline
Jésus-Christ reçut une couronne d'épines lors de la crucifixion pour rappeler ironiquement son titre de «~roi des Juifs~». Après la résurrection, les chrétiens recevront une couronne en récompense de leur intégrité. Devant le trône de Dieu, les vingt-quatre vieillards jettent leurs couronnes pour rendre gloire à Dieu. Voir \vref{Mt. 27:29}~; \vref{Ja. 1:12}~; \vref{1 Co. 9:25}~; \vref{1 Pi. 5:4}~; \vref{2 Ti. 4:8}~; \vref{Ap. 2:10} et \vref{Ap. 4:4,10}.

\DicoEntry{CROIX}\textit{, du grec «~stauros~»~: «~pieu, croix~»}\newline
Châtiment romain consistant à clouer les mains et les pieds des condamnés sur des poteaux en bois en forme de croix. Symbole du sacrifice de Jésus pour le pardon des péchés, la croix est aussi l'image de la vie de souffrance et de consécration totale à laquelle est appelé tout disciple du Seigneur. Voir \vref{Es. 53}~; \vref{Mt. 16:24} et \vref{Lu. 9:23}.

\DicoEntry{CUPIDITÉ}\textit{, du grec «~pleonexia~»~: «~désir avide d'avoir plus, avarice~»}\newline
Forme d'idolâtrie*, péché consistant à désirer de manière excessive les biens de ce monde (argent, richesses, etc.) et menant à la perdition. Voir \vref{Ep. 5:3}~; \vref{Col. 3:5} et \vref{2 Pi. 2:14}.

\DicoEntry{CYRÈNE}\textit{, du grec «~Kurene~»~: «~suprématie de la bride~», «~qui gouverne, froid~»}\newline
Ville prospère située dans la région fertile d'Afrique du Nord (actuelle Libye) où vivait une importante communauté juive et de laquelle était originaire Simon à qui l'on demanda de porter la croix de Jésus. Voir \vref{Mc. 15:20-22} et \vref{Ac. 2:10}.

\DicoEntry{CYRUS II LE GRAND}\textit{(règne~: 559-530 av. J.-C.), du persan «~Kowresh~»~: «~possède la puissance, puissance suprême~»}\newline
Fils de Cambyse, il régna sur l'Empire perse. Réveillé par Yahweh, il publia un édit en faveur du retour des Juifs à Jérusalem pour la reconstruction du temple. Voir \vref{Esd. 1:1-2} et \vref{2 Ch. 36:22-23}.

\DicoEntry{DAGON}\textit{, «~Dagown~»~: «~un poisson~»}\newline
Divinité païenne adorée par les Philistins, il était représenté par un personnage avec des mains et une face humaine et le corps d'un poisson. Voir \vref{1 S. 5:1-5}.

\DicoEntry{DAN}\textit{, de l'hébreu «~dan~»~: «~un juge~»}\newline
Fils de Jacob et de Bilha, servante de Rachel, il est le père de la tribu des Danites. Voir \vref{Ge. 30:1-6} et \vref{Ge. 49:16-18}.

\DicoEntry{DANIEL}\textit{, de l'hébreu «~Daniye'l~»~: «~Dieu est mon juge~»}\newline
Issu d'une famille princière de Juda, il fut déporté pendant sa jeunesse de Jérusalem à Babylone où il reçut le nom de Beltshatsar. Son histoire est racontée dans le livre éponyme.

\DicoEntry{DARIQUE}\textit{, de l'hébreu «~darkemown~»~: «~darique, drachme, unité de mesure~»}\newline
Utilisée après le retour de l'exil babylonien, monnaie d'or mise en place par le roi Darius et circulant dans l'Empire perse. Voir \vref{Esd. 8:26-27} et \vref{Né. 7:71-72}.

\DicoEntry{DARIUS Ier}\textit{, (règne~: 522 av. J.-C. – 486 av. J.-C.), de l'hébreu «~Dar`yavesh~»~: «~seigneur~» (origine~: perse)}\newline
Fils d'Assuérus, d'origine mède, roi des Chaldéens. Il encouragea la reconstruction du temple de Jérusalem après la découverte des instructions laissées par Cyrus sur un rouleau retrouvé dans la province de Médie. Voir \vref{Esd. 6}.

\DicoEntry{DATHAN}\textit{, de l'hébreu «~Dathan~»~: «~appartenant à une fontaine~»}\newline
Issu de la tribu de Ruben, fils d'Eliab et frère d'Abiram, il participa avec Koré à la révolte contre Moïse et Aaron. Voir \vref{No. 16:1-35}.

\DicoEntry{DAVID}\textit{, de l'hébreu «~David~»~: «~bien aimé~»}\newline
Issu de la tribu de Juda et dernier fils d'Isaï, il entra dès son plus jeune âge au service du roi Saül avant de devenir roi d'Israël. Homme selon le cœur de Dieu, il connut de grands succès sur les champs de bataille et fut l'auteur de nombreux psaumes. Il régna quarante-quatre ans sur Israël puis son fils Salomon* lui succéda. Voir \vref{1 S. 13:14}~; \vref{1 S. 16:14-23}~; \vref{1 S. 17}~; \vref{1 R. 2:10-11} et \vref{Ac. 13:22}.

\DicoEntry{DÉBORA}\textit{, de l'hébreu «~Debowrah~»~: «~abeille~»}\newline
Femme de Lapiddoth, elle exerça les fonctions de prophétesse et juge en Israël. Elle fut utilisée par Dieu pour prophétiser la victoire d'Israël sur Canaan par Barak qu'elle accompagna sur le champ de bataille. Voir \vref{Jg. 4-5}.

\DicoEntry{DÉLUGE}\textit{, de l'hébreu «~mabbuwl~»~: «~inondation, déluge~»}\newline
Pluie torrentielle s'étant abattue sur la terre pendant quarante jours et quarante nuits au temps de Noé. Le déluge symbolisait le jugement de Dieu sur une génération dont la méchanceté avait atteint un niveau sans précédent. Tous les habitants et les animaux de la terre furent emportés par les eaux du déluge hormis Noé, sa famille et les animaux qui étaient avec eux dans l'arche*. Voir \vref{Ge. 6-8}.

\DicoEntry{DEMAS}\textit{, du grec «~Demas~»~: «~gouverneur du peuple~»}\newline
Compagnon d'œuvre de Paul qui le délaissa «~par amour pour le siècle présent~». Voir \vref{Col. 4:14} et \vref{2 Ti. 4:10}.

\DicoEntry{DEMETRIUS}\textit{, du grec «~Demetrios~»~: «~qui appartient à Déméter (déesse grecque de l'agriculture)~»}\newline
Orfèvre qui fabriquait des statues de la déesse Diane à Ephèse. Voyant son commerce mis en danger par les prédications de Paul, il déclencha une émeute contre ce dernier. Voir \vref{Ac. 19:23-41}.

\DicoEntry{DÉMONS}\textit{, du grec «~daimonion~»~: «~divinité inférieure, mauvais esprit, ministres du diable~»}\newline
Egalement appelés «~esprits impurs~», anges* déchus ayant pris part à la révolte et à la chute de Satan*. Ils peuvent posséder le corps d'une personne, mais sont soumis à la puissance de Jésus, au nom duquel les chrétiens peuvent les chasser. Voir \vref{Mt. 10:8}~; \vref{Mc. 7:26}~; \vref{Mc. 16:17}~; \vref{Lu. 4:33}~; \vref{Lu. 10:17}~; \vref{Jud. 1:6} et \vref{Ap. 12:4}.

\DicoEntry{DIABLE}\textit{}\newline
Voir SATAN.

\DicoEntry{DIACRE}\textit{, du grec «~diakonos~»~: «~domestique, subordonné, messager~»}\newline
Les premiers diacres étaient des hommes remplis de l'Esprit Saint et de sagesse~; ils furent nommés pour faire un travail complémentaire aux ministres de la Parole au sein de l'église de Jérusalem. Etienne* était l'un d'eux. Il existait aussi des femmes diaconesses comme Phœbe, de l'église de Cenchrées. Voir \vref{Ac. 6:1-8}~; \vref{Ro. 16:1-2} et \vref{1 Ti. 3:8-13}.

\DicoEntry{DIANE}\textit{, du grec «~Artemis~»~: «~de la lumière~»}\newline
Aussi appelée «~Artemis d'Ephèse~», divinité révérée dans toute l'Asie. Il existait un temple en son honneur à Ephèse. Voir \vref{Ac. 19:24-37}.

\DicoEntry{DIEU}\textit{}\newline
Dieu des dieux et Seigneur des seigneurs, il est le Créateur de l'univers, du ciel, de la terre et de tout ce qui s'y trouve. Architecte d'excellence, il forma l'homme à son image et lui manifesta un amour inconditionnel par son incarnation en Jésus-Christ*. Dieu se présenta à Moïse sous le nom YHWH* (=Je suis celui qui suis) montrant son caractère éternel. Il s'est révélé à différentes personnes sous divers noms et aspects, en fonction des situations traversées montrant qu'il est celui qui remplit tout en tous et qu'il est et a tout ce dont l'homme a besoin. Ainsi, on le découvre dans les Ecritures comme étant grand, unique et indivisible, omniprésent, omniscient, souverain, incorruptible, sage, patient, saint, parfait, merveilleux, tout-puissant, fidèle, juste et bon. Bien évidemment, Dieu ne peut en aucun cas être défini dans tout ce qu'il est, dans la mesure où sa nature même échappe à toute possibilité de frontière ou de limite. Toutefois, les saints auront l'éternité pour découvrir ce Père incomparable. Voir \vref{Ge. 1,2}~; \vref{Ge. 17:1}~; \vref{Ex. 3:14}~; \vref{De. 6:14}~; \vref{De. 10:17}~; \vref{Es. 6:3}~; \vref{Mal. 3:6}~; \vref{Ps. 11:7}~; \vref{Ps. 139:7-10}~; \vref{La. 3:22-23}~; \vref{Lu. 1:49}~; \vref{Ja. 1:17}~; \vref{1 Th. 4:17}~; \vref{1 Co. 1:9}~; \vref{Ro. 1:23}~; \vref{Ro. 2:4}~; \vref{Ro. 11:33-36}~; \vref{2 Ti. 4:8}~; \vref{Hé. 4:13} et \vref{1 Jn. 4:8}.

\DicoEntry{DÎME}\textit{, de l'hébreu «~ma'aser~»~: «~dîme, dixième partie~»}\newline
Abraham donna à Melchisédek la dîme du butin d'une bataille remportée (\vref{Ge. 14:17-20} et \vref{Hé. 7:1-2}). Yahweh instaura, au travers de Moïse, la dîme comme une loi à respecter par les enfants d'Israël. Il en existait quatre sortes~:
\\1. la dîme que les Lévites prélevaient sur le peuple (\vref{No. 18:21-24})
\\2. la dîme de la dîme, que les prêtres prélevaient sur les Lévites (\vref{No. 18:25-31}~; \vref{Né. 10:38})
\\3. la dîme consommée par les Juifs eux-mêmes lors des fêtes de Yahweh (\vref{De. 14:22-26})
\\4. la dîme pour l'étranger, la veuve, l'orphelin et le Lévite, donnée tous les trois ans (\vref{De. 14:28-29}).
\\Cette loi concernait exclusivement Israël et non l'Eglise – Jésus-Christ ayant accompli la loi (\vref{Mt. 5:17}). Sous la grâce, les chrétiens sont invités à faire des offrandes * librement et sans contrainte.

\DicoEntry{DINA}\textit{, de l'hébreu «~Diynah~»~: «~jugement, justice~»}\newline
Fille de Jacob et Léa. Elle fut enlevée et déshonorée par Sichem, fils de Hamor, prince du pays de Canaan. Sichem et tous les hommes de la ville furent ensuite tués par les frères de la jeune fille, Siméon et Lévi. Voir \vref{Ge. 34}.

\DicoEntry{DIOTRÈPHE}\textit{, du grec «~Diotrephes~»~: «~nourri par Zeus~»}\newline
Chrétien dont Jean dénonça l'arrogance et les mauvais agissements. Voir \vref{3 Jn. 1:9-11}.

\DicoEntry{DISCIPLE}\textit{, du grec «~mathetes~»~: «~un étudiant, un élève, un disciple~»}\newline
Personne qui écoute les enseignements de son maître et les met en pratique en vue de devenir comme lui. Jésus en choisit douze qu'il forma pendant son service. Le disciple de Christ doit manifester le caractère de son maître, lui être pleinement consacré et être prêt à souffrir en son nom. Voir \vref{Mt. 10}~; \vref{Lu. 6:12-16}~; \vref{Lu. 14:26-33}.

\DicoEntry{DIVORCE}\textit{, du grec «~apostasion~»~: «~divorce, répudiation, lettre de divorce~»}\newline
Brisement des liens du mariage*. Il fut autorisé sous la loi de Moïse à cause de la dureté des cœurs, mais Christ rappela l'indissolubilité du mariage au commencement. Voir \vref{De. 24:1-3} et \vref{Mt. 19:3-8}.

\DicoEntry{DOCTEUR}\textit{, du grec «~didaskalos~»~: «~professeur~», «~maître~».}\newline
Sous la loi de Moïse, les docteurs de la loi étaient chargés d'expliquer la Torah. Certains d'entre eux s'opposèrent à Jésus. Sous la Nouvelle Alliance, le docteur est un des cinq services liés à la Parole évoqués en \vref{Ep. 4:11}. Il enseigne la Parole de Dieu qui guérit les blessures de l'âme. Selon \vref{Ja. 3:1}, nous ne sommes pas tous appelés à être des docteurs. Voir \vref{Lu. 2:46}~; \vref{Lu. 5:17}~; \vref{1 Co. 12:28} et \vref{Ep. 4:11}.

\DicoEntry{DONS SPIRITUELS}\textit{, du grec «~charisma~»~: «~faveur que reçoit quelqu'un sans aucun mérite de sa part~», «~dons provenant du pouvoir de la grâce divine~»}\newline
Capacités distribuées par le Saint-Esprit aux chrétiens en vue de la formation et de l'édification des saints. Voir \vref{1 Co. 12:1-11}~; \vref{1 Co. 14:12}~; \vref{Ro. 12:6} et \vref{1 Pi. 4:10}.

\DicoEntry{EDEN}\textit{, de l'hébreu «~'Eden~»~: «~plaisir, délices~»}\newline
Appelé aussi jardin de Dieu, premier lieu de résidence d'Adam et Eve. Yahweh y avait fait pousser des arbres de toutes espèces~; il y avait également placé au milieu l'arbre de vie* ainsi que l'arbre de la connaissance du bien et du mal* dont la consommation des fruits conduirait à la mort. L'homme fut établi en tant que gardien et cultivateur de ce jardin. Cependant, il pêcha avec la femme et ils furent chassés de ce lieu des délices. Voir \vref{Ge. 2}~; \vref{Ge. 3:23-24} et \vref{Ez. 28:13}.

\DicoEntry{ÉGLISE}\textit{, du grec «~ekklesia~»~: «~appel hors de~»}\newline
Peuple mis à part dont Christ est le chef. L'Eglise est la sainte habitation de Dieu en esprit, le corps de Christ, l'épouse de l'Agneau. On distingue l'Eglise universelle - qui regroupe tous les saints du monde entier - de l'église locale - qui est composée de tous les chrétiens d'une ville. Voir \vref{Ac. 2:47}~; \vref{1 Th. 1:1}~; \vref{1 Co. 1:2}~; \vref{1 Co. 3:16}~; \vref{1 Co. 12:27}~; \vref{Ep. 2:20-22}~; \vref{Ep. 5:22-32}~; \vref{Ph. 1:1} et \vref{1 Ti. 2:4}.

\DicoEntry{ÉLÉAZAR}\textit{, de l'hébreu «~'El'azar~»~: «~Dieu a secouru~»}\newline
Fils d'Aaron, il était chef des chefs des Lévites avant de devenir le second grand prêtre d'Israël. Voir \vref{No. 3:32} et \vref{No. 20:25-28}.

\DicoEntry{ÉLECTION, ELU}\textit{, de l'hébreu «~bachiyr~» et du grec «~eklektos~»~: «~choisi, élu de Dieu~»}\newline
Dans le Tanakh, Israël fut présenté comme le peuple élu de Yahweh, appelé à être un exemple pour toutes les nations de la terre. Cette élection n'est pas synonyme de préférence, car la volonté de Dieu est de sauver tous les hommes. Au travers de l'œuvre de la croix, Dieu a effectivement montré que son choix se porte vers l'humanité tout entière en payant le prix des péchés de tous. Dans son omniscience, il sait toutefois d'avance qui croira en lui ou pas. Selon la parole, même après la conversion, le chrétien doit travailler son élection, c'est-à-dire se sanctifier et obéir aux commandements de Yahweh pour entrer dans son royaume. Voir \vref{Es. 45:4}~; \vref{Es. 49:6}~; \vref{Mt. 22:14}~; \vref{Ep. 1:4-6} et \vref{2 Pi. 1:10-11}.

\DicoEntry{ÉLIE}\textit{, de l'hébreu «~Eliyah~»~: «~Yahweh est mon Dieu~»}\newline
Prophète d'origine tschibite que Dieu suscita en Israël au temps du roi Achab*. Il ne connut point la mort, mais fut enlevé par le Seigneur. Son histoire, ses combats et ses exploits sont racontés dans les livres des Rois.

\DicoEntry{ÉLISÉE}\textit{, de l'hébreu «~'Eliysha'~»~: «~Dieu est sauveur~»}\newline
Prophète du royaume d'Israël, il succéda à Elie après avoir reçu la double portion de l'esprit qui était sur ce dernier. Après sa mort, ses os rendirent la vie à un défunt. Voir \vref{1 R. 19:16-21}~; \vref{2 R. 2:9-11} et \vref{2 R. 13:20-21}.

\DicoEntry{ENFER}\textit{}\newline
Voir SEJOUR DES MORTS.

\DicoEntry{ENLÈVEMENT}\textit{, du grec «~metathesis~»~: «~transfert d'un lieu à un autre, changement~»}\newline
Ravissement d'hommes au ciel sans que ces derniers ne connaissent la mort*. Dans le Tanakh, se trouvent deux cas d'enlèvement~: Hénoc (\vref{Ge. 5:24}~; \vref{Hé. 11:5}) et Elie (\vref{2 R. 2:11}). L'Eglise sera de même enlevée par le Seigneur au son de la dernière trompette. Voir \vref{1 Th. 4:17} et \vref{1 Co. 15:51-57}.

\DicoEntry{ÉPHÈSE}\textit{, du grec «~Ephesos~»~: «~permis~»}\newline
Une des principales villes de l'empire romain sous le règne de l'empereur Claude 1er (10 av. J.-C. – 54 ap. J.-C.) Ephèse possédait le plus grand port de l'Asie Mineure, ce qui lui attribuait le contrôle du trafic commercial. Richissime et prospère, elle était renommée pour son faste, sa liberté de parole et constituait donc un endroit privilégié pour les philosophes. L'église d'Ephèse naquit du ministère de Paul, qui y enseigna pendant au moins deux ans lors de son troisième voyage missionnaire. Cette église - figurant parmi les sept du livre d'Apocalypse - fit preuve de discernement et pratiquait de bonnes œuvres, mais le Seigneur avait néanmoins un reproche à lui adresser. Elle représente l'église apostate. Voir Epître aux Ephésiens et \vref{Ap. 2:1-7}.

\DicoEntry{ÉPHOD}\textit{, de l'hébreu «~ephowd~»~: «~couverture~»}\newline
Vêtement que les prêtres portaient par-dessus leur tunique lorsqu'ils étaient en service. L'éphod du grand prêtre était de broderie~; le pectoral était posé sur son devant. Voir \vref{Ex. 28} et \vref{Lé. 8:7}.

\DicoEntry{ÉPHRAIM}\textit{, de l'hébreu «~'Ephrayim~»: «~double fertilité~»}\newline
Second fils de Joseph né en Egypte, il fut adopté par Jacob avant sa mort et devint ainsi l'ancêtre d'une des douze tribus d'Israël. Voir \vref{Ge. 41:52}~; \vref{Ge. 48:5} et \vref{Jos. 14:4}.

\DicoEntry{ÉPICURIENS}\textit{, du grec «~epikoureios~»~: «~celui qui aide, le défenseur~»}\newline
Fondé à Athènes en 306 av. J.-C., groupe de philosophes se réclamant de la doctrine d'Epicure (341 av. J.-C. à 270 av. J.-C.). Ce dernier fonda une des plus importantes écoles philosophiques de l'Antiquité. Il développa une théorie athée selon laquelle l'homme est encouragé à rechercher les plaisirs matériels et sensuels. Rejetant la pensée d'une vie après la mort, les épicuriens renient l'existence d'un créateur qui se préoccuperait des hommes. Des adeptes de cette philosophie se confrontèrent à la doctrine de Christ annoncée par Paul et cherchèrent à l'entendre. Voir \vref{Ac. 17:18-20}.

\DicoEntry{ÉSAÏE}\textit{, de l'hébreu «~Yesha'yah~»~: «~Yahweh a sauvé~»}\newline
Fils d'Amotz, un prophète de Yahweh contemporain des rois Ozias, Jotham, Achaz et Ezéchias, il annonça la venue du Messie. L'ensemble de ses prophéties est contenu dans le livre portant son nom.

\DicoEntry{ÉSAÜ}\textit{, de l'hébreu «~'Esav~»~: «~velu, poilu, chevelu~»}\newline
Fils d'Isaac et Rebecca et frère jumeau de Jacob, qui lui soutira son droit d'aînesse et sa bénédiction. Il prit pour femmes Judith et Basmath, toutes deux originaires de Canaan. Egalement connu sous le nom d'Edom, il devint l'ancêtre des Edomites. Voir \vref{Ge. 25:25-34}~; \vref{Ge. 27} et \vref{Ge. 36}.

\DicoEntry{ESDRAS}\textit{, de l'hébreu «~`Ezra'~»~: «~secours~»}\newline
Fils de Sereja et descendant du grand prêtre Aaron, Esdras était scribe et prêtre. Il enseigna le peuple de Dieu dans la loi et mit en place des réformes après la reconstruction du temple. Son histoire se trouve dans le livre éponyme.

\DicoEntry{ESPRIT}\textit{, de l'hébreu «~ruwach~»~: «~vent, souffle, esprit~» et du grec «~pneuma~»~: «~vérité, inspiration, souffle, vent~»}\newline
L'esprit humain est aussi appelé homme intérieur, il constitue la partie spirituelle de l'homme lui permettant d'agir, de prendre des décisions et d'être en contact avec Dieu ou tout autre esprit. Principe vital, il amène l'âme à la vie. Avec l'âme* et le corps, l'esprit constitue l'être humain. Voir \vref{Ge. 6:3}~; \vref{Ex. 31:3}~; \vref{Job 27:3}~; \vref{Job 32:8}~; \vref{Mt. 12:28} et \vref{1 Th. 5:23}.

\DicoEntry{ESPRIT IMPUR}\textit{}\newline
Voir DEMONS.

\DicoEntry{ESTHER}\textit{, dérivation du perse «~'Ecter~»~: «~étoile~»}\newline
Cousine de Mardochée, Juif d'origine benjaminite, reine de Perse, épouse du roi Assuérus. Son nom juif était Hadassa~: «~myrte~». Son histoire, qui se déroula à Suse, est racontée dans le livre portant son nom.

\DicoEntry{ÉTANG DE FEU}\textit{}\newline
Lieu de douleur et de damnation éternelle créé initialement pour le diable et ses anges. Y seront jetés la bête et le faux prophète, le diable, la mort et le séjour des morts* ainsi que tous ceux dont le nom ne sera pas trouvé dans le livre de vie. Voir \vref{Mt. 25:41}~; \vref{Ap. 19:20} et \vref{Ap. 20:7-15}.

\DicoEntry{ÉTIENNE}\textit{, du grec «~stephanos~»~: «~couronne~»}\newline
Diacre* de l'église de Jérusalem rempli de sagesse et d'Esprit Saint. Premier martyr chrétien, sa mort marqua le début d'une grande persécution contre l'Eglise. Voir \vref{Ac. 6:1-6}~; \vref{Ac. 7} et \vref{Ac. 8:1-3}.

\DicoEntry{EUNUQUE}\textit{, du grec «~cariyc~»~: «~eunuque, chambellan castré~»}\newline
Homme dans l'incapacité de procréer ou émasculé. Dans l'Antiquité, les rois se choisissaient des eunuques pour les servir. En les castrant, ils s'assuraient de la fidélité et l'intégrité de ces derniers. En outre, Jésus distingua trois types d'eunuques. Voir \vref{2 R. 20:18}~; \vref{Da. 1:7}~; \vref{1 Ch. 28:1} et \vref{Mt. 19:12}).

\DicoEntry{ÉVANGELISTE}\textit{}\newline
Un des cinq ministères d'\vref{Ep. 4:11} dont la mission est de prêcher la repentance et la conversion à Jésus-Christ. Comme les ministères évoqués en \vref{Ep. 4:11}, il travaille également à la perfection des saints. Philippe exerça ce ministère, Timothée fut de même encouragé à faire l'œuvre d'un évangéliste. Tous les chrétiens doivent également évangéliser. Voir \vref{Ac. 21:8}~; \vref{Ep. 4:11} et \vref{2 Ti. 4:5}.

\DicoEntry{ÉVANGILE}\textit{}\newline
Enseignement donné par Jésus-Christ, la prédication de la croix (la mort et la résurrection de Jésus-Christ) et du Royaume de Dieu qui s'est approché des hommes (voir ROYAUME DE DIEU). Ce message annonce le salut, la guérison du cœur, la joie en Jésus-Christ, la justice, la paix, la grâce et la vie éternelle accordée à l'homme repentant, mais aussi le jugement à venir. Les apôtres propagèrent l'Evangile~; de même, tous les chrétiens sont appelés à le faire. Voir \vref{Es. 61}~; \vref{Mt. 10:7}~; \vref{Mt. 28:19-20}~; \vref{1 Co. 15:1-4}~; \vref{Ro. 1:16} et \vref{2 Ti. 4:1}.

\DicoEntry{EVE}\textit{, de l'hébreu «~Chavvah~»~: «~vie~»}\newline
Première femme et épouse d'Adam, elle fut formée à partir de la côte de son mari dans le but d'être l'aide de ce dernier. Séduite par Satan déguisé en serpent, elle mangea le fruit de la connaissance du bien et du mal et fut avec Adam chassée du jardin. Elle donna naissance à Caïn, Abel et Seth. Voir \vref{Ge. 2:18-24}~; \vref{Ge. 3:1-13} et \vref{Ge. 4:1-2,25}.

\DicoEntry{ÉVÊQUE}\textit{, du grec «~episcopos~»~: «~investigation, inspection, visite d'inspection~», «~acte par lequel Dieu visite les hommes, observe leurs voies, leurs caractères, pour leur accorder en partage joie ou tristesse~», «~surveillance, contrôle, fonction d'un ancien~», «~la charge d'une église chrétienne~».}\newline
Il est question ici d'une fonction consistant à visiter les assemblées, les inspecter afin de s'assurer du bon ordre. Voir \vref{Lu. 19:44}~; \vref{Ac. 1:20}~; \vref{1 Ti. 3:1}~; \vref{1 Pi. 2:12}.

\DicoEntry{EXPIATION}\textit{, de l'hébreu «~kaphar~»~: «~couvrir, purger, faire une expiation~»}\newline
Action de couvrir les fautes et les souillures de l'homme afin qu'il soit réconcilié avec Dieu. Sous l'Ancienne Alliance, le grand prêtre faisait tous les ans un sacrifice d'expiation en entrant dans le Saint des saints pour ses péchés et les péchés du peuple. Par son sacrifice, Christ est devenu la victime expiatoire pour les péchés de tous les hommes en les prenant sur lui à la croix~; il est l'Agneau de Dieu qui ôte les péchés du monde. Voir \vref{Lé. 16}~; \vref{Jn. 1:29}~; \vref{1 Jn. 2:2} et \vref{1 Jn. 4:10}.

\DicoEntry{ÉZÉCHIAS}\textit{, de l'hébreu «~Yechizqiyah~»~: «~Yahweh est ma force~»}\newline
Fils d'Osée, roi de Juda sur qui il régna vingt-neuf ans. Figurant parmi les rois les plus intègres, son règne fut caractérisé par la droiture et la fidélité à Yahweh. Voir \vref{2 R. 18-19}.

\DicoEntry{ÉZÉCHIEL}\textit{, de l'hébreu «~Yechezqe'l~»~: «~Dieu fortifie~»}\newline
Fils de Buri, prêtre et prophète de Yahweh ayant été déporté à Babylone. Il reçut de nombreuses visions - sur son temps et les temps de la fin - racontées dans le livre qui porte son nom.

\DicoEntry{FÉLIX}\textit{, du grec «~Phestos~»~: «~joyeux, en fête~»}\newline
Gouverneur de Judée de 52 à 60 ap. J.-C., il emprisonna Paul à la suite des plaintes des Juifs. S'entretenant avec lui de temps en temps et lui octroyant certaines libertés, Felix garda Paul en prison deux ans pour plaire aux Juifs. Voir \vref{Ac. 24}.

\DicoEntry{FESTUS}\textit{, du grec «~Phestos~»~: «~en fête, joyeux~»}\newline
Gouverneur de Judée qui succéda à Félix* de 60 à 62 ap. J.-C. Il poursuivit l'instruction du procès de Paul que les Juifs accusaient. Il permit à Paul de s'exprimer devant le roi Agrippa* et l'envoya à Rome afin qu'il comparaisse devant César*. Voir \vref{Ac. 24:27} et \vref{Ac. 25,26}.

\DicoEntry{FÊTES DE YAHWEH}\textit{}\newline
Selon la loi juive, sept fêtes étaient célébrées en l'honneur de Yahweh~: la Pâque de Yahweh, la fête des pains sans levain~; la fête des prémices~; la Pentecôte~; la fête des trompettes~; le jour des expiations et la fête des tabernacles. Voir \vref{Lé. 23:6-43}.

\DicoEntry{FIGUIER}\textit{}\newline
Arbre fruitier sous lequel il était coutume d'étudier la Torah en Israël. Ses fruits excellents et doux servaient en médecine. Le figuier est retrouvé dans de nombreuses histoires et paraboles des Ecritures. Il symbolise la douceur et l'humilité. Voir \vref{Jg. 9:11}~; \vref{2 R. 20:1-7}~; \vref{Lu. 13:6-9} et \vref{Jn. 1:43-51}.

\DicoEntry{FILS DE DIEU}\textit{}\newline
Expression désignant selon le contexte~:
\\1. les anges. Voir \vref{Ge. 6:2-4}~; \vref{Job 38:7} et \vref{Da. 3:25}.
\\2. Adam. Voir \vref{Lu. 3:38}.
\\3. les chrétiens. Voir \vref{Ga. 3:26} et \vref{Ro. 8:14}.
\\4. Jésus-Christ, le Fils unique de Dieu, en qui habite la plénitude de la divinité. Voir \vref{Mc. 15:39}~; \vref{Lu. 22:70}~; \vref{Jn. 1:14,34,49}~; \vref{Ro. 1:4}~; \vref{Col. 2:9} et \vref{1 Jn. 4:9,15}.

\DicoEntry{FILS DE L'HOMME}\textit{}\newline
Expression désignant un être humain, elle fut attribuée au prophète Ezéchiel près de cent fois. A de nombreuses reprises, Jésus-Christ se nomma lui-même «~Fils de l'homme~» afin de souligner sa nature humaine. Voir \vref{Ez. 2:1}~; \vref{Ez. 3:10}~; \vref{Ez. 4:1}~; \vref{Mc. 14:62}~; \vref{Jn. 5:27}~; \vref{Ro. 8:3} et \vref{Ph. 2:5-7}.

\DicoEntry{FIN DES TEMPS}\textit{, du grec «~eschatos~»~: «~extrême, dernier, fin~» et «~chronos~»~: «~temps, date, siècles~»}\newline
Appelée aussi derniers jours, période précédant la fin du monde*. Elle a commencé à l'effusion du Saint-Esprit selon la prophétie de Joël. La fin des temps est caractérisée d'un côté par des manifestations extraordinaires de l'Esprit de Dieu et l'annonce de l'Evangile à tous les peuples~; de l'autre par la séduction, l'apostasie* et le péché dans des dimensions jamais atteintes auparavant. Voir \vref{Joë. 2:28-29}~; \vref{Mt. 24:3-14}~; \vref{Ac. 2:16-18}~; \vref{1 Ti. 4:1} et \vref{2 Ti. 3:1-5}.

\DicoEntry{FIN DU MONDE}\textit{, du grec «~eschatos~»~: «~extrême, dernier, fin~» et «~aion~»~: «~monde, univers, période de temps~»}\newline
Cet événement correspond à la fin de notre ère. Après le jugement dernier, les impies iront dans l'étang de feu*, tandis que la Nouvelle Jérusalem accueillera les saints~; la terre sera détruite. Voir \vref{Mt. 13:36-43}~; \vref{2 Pi. 3:10-13}~; \vref{Ap. 20:11-15} et \vref{Ap. 21}.

\DicoEntry{FOI}\textit{, du grec «~pistis~»~: «~conviction de la vérité~»}\newline
Confiance en la véracité de Dieu, ses paroles et l'accomplissement de ses promesses. Bien qu'il n'existe qu'une seule foi, elle est présentée sous trois formes principales sous la Nouvelle Alliance~:
\\1. en tant que fruit de l'esprit*, c'est la foi qui sauve (\vref{Ga. 5:22} et \vref{Ro. 10:9})
\\2. en tant que don de l'Esprit,* c'est la foi accordée pour accomplir une tâche particulière (\vref{1 Co. 12:9})
\\3. en tant que Parole, c'est la foi liée à la saine doctrine, la vérité (\vref{Ro. 10:17} et \vref{2 Ti. 4:7})
\\Condition essentielle pour être agréable à Dieu~; la foi est éprouvée tout au long de la vie du croyant. Voir \vref{Lu. 7:50}~; \vref{Hé. 11} et \vref{1 Pi. 1:7}.

\DicoEntry{FORNICATION ou IMPUDICITÉ}\textit{, du grec «~pœrneia~»~: «~relation sexuelle illicite~»}\newline
Tous les rapports sexuels condamnés par la Parole, voir \vref{Lé. 18}~; \vref{1 Co. 6:13,16-18} et \vref{1 Co. 7:2}.

\DicoEntry{FRUIT DE L'ESPRIT}\textit{}\newline
Résultat de l'action de l'Esprit Saint dans l'homme intérieur dans le but de communiquer le caractère de Yahweh au chrétien né d'en haut. Voir \vref{Ga. 5:22}.

\DicoEntry{GABRIEL}\textit{, de l'hébreu «~Gabriy'el~»~: «~héros de Dieu~» ou «~homme de Dieu~»}\newline
Archange* que Dieu envoya pour délivrer des messages, notamment à Daniel, Zacharie et Marie. Voir \vref{Da. 9:21-27}~; \vref{Lu. 1:11-20} et \vref{Lu. 1:26-38}.

\DicoEntry{GAD}\textit{, de l'hébreu «~Gad~»~: «~bonheur~», «~heureux~», «~troupe~»}\newline
Fils de Jacob et Zilpa, servante de Léa, il devint l'ancêtre de la tribu de Gad. Voir \vref{Ge. 30:11} et \vref{Ge. 49:16}.

\DicoEntry{GALATIE}\textit{, du grec «~Galatia~»~: «~territoire des Gaulois, Gaule~»}\newline
Province antique de l'Asie Mineure, la Galatie se situait en Anatolie, dans l'actuelle Turquie autour d'Ankara. Elle devait son nom aux Galates, Celtes provenant des Balkans. Lors de son premier voyage missionnaire, Paul avait traversé cette région où plusieurs assemblées émergèrent. Il y revint plus tard pour fortifier les disciples et leur écrivit une lettre suite au trouble apporté par les judaïsants. Voir \vref{Ac. 16:6}~; \vref{Ac. 18:23} et \vref{Ga. 1-5}.

\DicoEntry{GALILÉE}\textit{, de l'hébreu «~Galiyl~»~: «~cercle, région, district~»}\newline
Région située au nord de la Palestine dans laquelle se trouve la localité de Nazareth où Jésus grandit. Il y commença son ministère, c'est aussi là qu'il se montra vivant à ses disciples après sa résurrection. Les disciples de Jésus étaient originaires de Galilée. Voir \vref{Mt. 2:19-23}~; \vref{Mc. 16:7}~; \vref{Jn. 2}~; \vref{Ac. 1:11} et \vref{Ac. 2:7}.

\DicoEntry{GARIZIM}\textit{, de l'hébreu «~Geriziym~»~: «~lieux arides~»}\newline
Montagne située au sud de Sichem, en face du mont Ebal, de laquelle les enfants d'Israël devaient prononcer la bénédiction* une fois entrés en Canaan. Voir \vref{De. 11:29}~; \vref{Jg. 9:7} et \vref{Jos. 8:33}.

\DicoEntry{GÉDÉON}\textit{, de l'hébreu «~Gid'own~»~: «~coupant, abattant~»}\newline
Issu de la tribu de Manassé et fils de Joas. Il fut mandaté pour délivrer Israël de la main des Madianites et fut juge en Israël pendant quarante ans. Voir \vref{Jg. 6-8}.

\DicoEntry{GÉHENNE}\textit{, du grec «~geena~»~: «~vallée de Hinnom~»}\newline
Initialement, vallée située au sud de Jérusalem où des enfants étaient jetés dans le feu en sacrifice à Moloc. Le terme «~géhenne~» représente la destruction future des méchants et se rapporte à l'étang de feu*. Voir \vref{2 R. 23:10} et \vref{Mt. 10:28}.

\DicoEntry{GENTILS}\textit{, du grec «~ethnos~»~: «~nations~», «~peuples~»}\newline
Dans les Ecritures, ce terme se rapportait initialement à tous ceux n'appartenant pas au peuple juif. Paul fut mandaté pour évangéliser les Gentils. A partir du IIIème siècle, le terme «~païen~» fut introduit dans le jargon chrétien pour désigner le «~non-chrétien~». Voir \vref{Mt. 18:17} et \vref{Ac. 26:17}.

\DicoEntry{GERME}\textit{, de l'hébreu «~tsemach~»~: «~pousse, croissance, branche~»}\newline
Terme désignant le Messie dans certains écrits prophétiques. Voir \vref{Es. 4:2}~; \vref{Jé. 23:5} et \vref{Za. 3:8}.

\DicoEntry{GLOIRE}\textit{, de l'hébreu «~kabhod~»~: «~poids~» ou «~kabowd~»~: «~gloire, honneur, richesse~»}\newline
La gloire se rapporte à ce qui a du poids, ce qui est lourd et écrasant – il est en effet difficile pour l'homme de supporter la splendeur et la magnificence de Yahweh. Image de sa sainteté, elle s'est manifestée dans un feu dévorant sur le mont Sinaï et fut révélée à Moïse au travers de la bonté et du nom de Dieu. Cette gloire sanctifie et génère de grands miracles~; elle est racontée par les cieux et toute la création. La gloire de Yahweh sera le luminaire de la Nouvelle Jérusalem. Elle invite à la crainte, la révérence, l'humilité, la louange~; lui seul mérite la gloire. Voir \vref{Ex. 16:10}~; \vref{Ex. 24:17}~; \vref{Ex. 29:43}~; \vref{Ex. 33:18-23}~; \vref{Es. 42:8}~; \vref{Es. 48:11}~; \vref{Ez. 44:4}~; \vref{Ps. 19:1}~; \vref{Pr. 15:33}~; \vref{2 Ch. 5:14}~; \vref{1 Th. 2:12} et \vref{Ap. 21:23}.

\DicoEntry{GOG}\textit{, de l'hébreu «~Gowg~»~: «~montagne~»}\newline
Très certainement le chef du pays de Magog. Voir \vref{Ez. 38} et \vref{Ap. 20:8}.

\DicoEntry{GOLGOTHA}\textit{, de l'araméen «~gulgoleth~»~: «~tête, crâne~»}\newline
Lieu de la crucifixion de Jésus-Christ, situé non loin de Jérusalem. Voir \vref{Jn. 19:17-20}.

\DicoEntry{GOMORRHE}\textit{, de l'hébreu «~Amorah~»~: «~submersion~»}\newline
Ville située dans la plaine du Jourdain. Après avoir atteint un haut degré de perversion et de débauche, elle fut détruite par Yahweh avec sa ville voisine, Sodome. Voir \vref{Ge. 13:10}~; \vref{Ge. 18:20-21} et \vref{Ge. 19:24}.

\DicoEntry{GRÂCE}\textit{, du grec «~charis~»~: «~bonne volonté~», «~bonté~», «~faveur~»}\newline
Don immérité de Dieu, elle est la source du salut* de tous les hommes et invite à la crainte de Dieu. La grâce est venue par Jésus-Christ et fut révélée au travers de l'œuvre parfaite de la croix*. Voir \vref{Jn. 1:17}~; \vref{Ro. 3:23-24}~; \vref{Tit. 2:11-12}.

\DicoEntry{GRAND PRÊTRE}\textit{}\newline
Voir PREMIER PRÊTRE.

\DicoEntry{GRANDE TRIBULATION}\textit{}\newline
Voir commentaire \vref{Ap. 7:14}.

\DicoEntry{GUILGAL}\textit{, de l'hébreu «~Gilgal~»~: «~action de rouler~»}\newline
Territoire situé à l'ouest du Jourdain et à l'est de Jéricho~; il fut le lieu de campement des Israélites après avoir passé le Jourdain à sec. Voir \vref{Jos. 4-5}.

\DicoEntry{HABAKUK}\textit{, de l'hébreu «~Chabaqquwq~»~: «~embrasser~», «~amour~»}\newline
Prophète de Yahweh qui exerça son ministère dans le royaume de Juda. L'ensemble de ses prophéties se trouve dans le livre éponyme.

\DicoEntry{HARMAGUEDON}\textit{, de l'hébreu «~Armageddon~»~: «~montagne de Méguiddo~»}\newline
Lieu situé au nord d'Israël dans la tribu de Zabulon. A la fin des temps, les rois et puissants de la terre s'y rassembleront pour combattre Yahweh et son armée. Voir \vref{2 R. 23:29} et \vref{Ap. 16:13-16}.

\DicoEntry{HÉBREU}\textit{, de l'hébreu «~`Ibriy~»~: «~qui provient de l'autre côté, qui traverse~»}\newline
Terme désignant les descendants d'Héber, fils de Schélach, de la postérité de Sem, dont est issu Abraham. Voir \vref{Ge. 11:10-32}~; \vref{Ge. 14:13-14} et \vref{Ex. 1:15-22}.

\DicoEntry{HELLÉNISTE}\textit{, du grec «~hellenistes~»~: «~celui qui adopte les manières et coutumes des Grecs~»}\newline
Israélites nés hors de la terre promise ayant adopté le mode de vie grec et parlant la langue grecque. Voir \vref{Ac. 6:1}.

\DicoEntry{HÉNOC}\textit{, de l'hébreu «~Chanowk~»~: «~consacré, dédié~»}\newline
Fils de Jéred et père de Metuschéla. Homme pieux ayant vécu trois cent soixante-cinq ans avant d'être enlevé au ciel sans connaître la mort. Voir \vref{Ge. 5:21-24} et \vref{Hé. 11:5}.

\DicoEntry{HÉRODE LE GRAND}\textit{, (73 av. J.-C.à 4 av. J.-C), du grec «~Herodes~»: «~héroïque~»}\newline
Roi de Judée, il fut l'instigateur du massacre des enfants de la région de Bethléhem au moment de la naissance de Jésus. Il mourut quand Jésus était encore enfant. Voir \vref{Mt. 2}.

\DicoEntry{HÉRODE ANTIPAS}\textit{, (ou le Tétrarque) (\ 21 av. J.-C. à 39 ap. J.-C.)}\newline
Fils d'Hérode le Grand*, il exerça la fonction de tétrarque* de Galilée et fut contemporain à Jésus-Christ pendant presque toute la vie de ce dernier. Hérode épousa sa belle-sœur Hérodias* et fit décapiter Jean-Baptiste. Il fut qualifié de «~renard~» par Jésus et s'accorda avec son ennemi Pilate lors de la crucifixion du Seigneur. Voir \vref{Mc. 6:14-28}~; \vref{Lu. 3:1}~; \vref{Lu. 13:31-32} et \vref{Lu. 23:8-12}.

\DicoEntry{HÉRODE AGRIPPA Ier}\textit{, (\ 10 av. J.-C. à 44 ap. J.-C)}\newline
Roi et tétrarque de Judée et petit fils du roi Hérode le Grand, il accéda au pouvoir à la genèse de l'Eglise primitive. Pour plaire aux Juifs, il fit mourir Jacques, fils de Zébédée, et emprisonna Pierre. Il mourut brusquement après avoir reçu du peuple la gloire qui devait revenir à Dieu. Voir \vref{Ac. 12}.

\DicoEntry{HÉRODE AGRIPPA II}\textit{, (\ 27 ap. J.-C. à 93 ap. J.-C.)}\newline
Fils d'Agrippa Ier, il est appelé «~roi Agrippa~» dans les Ecritures. Il fut inspecteur du temple de Jérusalem et avait le pouvoir de choisir les grands prêtres. Il rencontra Paul à Césarée lors d'une visite au gouverneur Festus*. Voir \vref{Ac. 25-26}.

\DicoEntry{HÉRODIAS}\textit{, du grec «~Herodias~»~: «~héroïque~»}\newline
Femme de Philippe I puis de son frère, Hérode le tétrarque. Elle commanda la décapitation de Jean-Baptiste. Voir \vref{Mc. 6:17-28}.

\DicoEntry{HOLOCAUSTE}\textit{, de l'hébreu «~'olah~»~: «~offrande entièrement consumée~»}\newline
Prescrit par la loi de Moïse, sacrifice consumé par le feu d'une agréable odeur à Yahweh. Il préfigurait le sacrifice à la croix de Jésus-Christ, l'Agneau de Dieu. Voir \vref{Lé. 1:1-17}~; \vref{Hé. 9:11-22} et \vref{Hé. 10:1-19}.

\DicoEntry{HOMOSEXUALITÉ}\textit{}\newline
Pratique abominable et fermement réprouvée par Dieu consistant en l'union de deux personnes du même sexe. Voir \vref{Lé. 18}~; \vref{1 Co. 6:9-10} et \vref{Ro. 1:24-32}.

\DicoEntry{HOSANNA}\textit{, de l'hébreu «~yasha'~»~: «~sauve~» et «~na´~»~: «~je te prie, maintenant~» et du grec «~hosanna~»~: «~sauve maintenant~!~»}\newline
Cri par lequel Jésus fut accueilli par la foule quand il entra à Jérusalem. Voir \vref{Mt. 21:9,15}~; \vref{Mc. 11:9-10} et \vref{Jn. 12:13}.

\DicoEntry{HULDA}\textit{, de l'hébreu «~chuldah~»~: «~belette, taupe~»}\newline
Femme de Schallum, prophétesse habitant à Jérusalem du temps de Josias, roi de Juda. Le roi chercha à consulter Yahweh au travers d'elle quand il découvrit le livre de la loi et les malheurs qui devaient suivre la désobéissance d'Israël. Voir \vref{2 R. 22:14-20} et \vref{2 Ch. 34:21-33}.

\DicoEntry{HYSOPE}\textit{, du grec «~hussopos~»~: «~hysope, branche d'hysope~»}\newline
Plante aromatique utilisée pour faire l'aspersion du sang ou d'eau sous l'Ancienne Alliance. C'est à l'aide d'une branche d'hysope qu'on présenta à Jésus une éponge trempée de vinaigre lors de sa crucifixion. Voir \vref{Ex. 12:22}~; \vref{Lé. 14:1-7}~; \vref{No. 19:18-19}~; \vref{Jn. 19:29} et \vref{Hé. 9:19}.

\DicoEntry{IDOLE, IDOLÂTRIE}\textit{, de l'hébreu «~gilluwl~»~: «~image~» et du grec «~eidolon~»~: «~image pour adorer~»}\newline
Une idole peut être l'image d'un faux dieu, l'image faussée de Yawheh ou encore une personne, un objet, une activité à qui l'on donne le rang de Dieu. L'idolâtrie - culte rendu à ces idoles – est fermement réprouvée dans la Parole. Voir \vref{Ex. 20:3-5}~; \vref{Ex. 32}~; \vref{1 R. 15:11-13}~; \vref{1 Co. 6:9}~; \vref{Ep. 5:5} et \vref{Col. 3:5}.

\DicoEntry{IMPOSITION DES MAINS}\textit{}\newline
Avant leur mort, les patriarches imposaient les mains à leurs enfants pour les bénir (\vref{Ge. 48:14}). Moïse imposa également les mains à Josué qui devait lui succéder (\vref{De. 34:9}). Sous la Nouvelle Alliance, on peut imposer les mains à quelqu'un en vue de lui transmettre la guérison divine, l'autorité liée à une fonction particulière, les dons spirituels et même le Saint-Esprit dans certains cas. Ce geste ne doit cependant pas être fait dans la précipitation. Voir \vref{Lu. 4:40}~; \vref{Mc. 16:18}~; \vref{Ac. 6:6}~; \vref{Ac. 8:17}~; \vref{1 Ti. 4:14} et \vref{1 Ti. 5:22}.

\DicoEntry{INCORRUPTIBILITÉ}\textit{, du grec «~aphtharsia~»~: «~perpétuité, pureté, sincérité~»}\newline
Terme désignant ce qui ne peut ni se corrompre, ni se flétrir, ni se détruire. A l'enlèvement de l'Eglise, les morts en Christ ressusciteront incorruptibles et les chrétiens revêtiront de même des corps incorruptibles. Voir \vref{Mt. 24:35} et \vref{1 Co. 15:40-57}.

\DicoEntry{INCREDULITÉ}\textit{, du grec «~apistia~»~: «~infidélité, sans foi, faiblesse dans la foi~»}\newline
Rejet, doute par rapport à la véracité de Dieu et de sa parole. Thomas fit preuve d'incrédulité quant à la résurrection de Christ avant de le voir vivant. Les incrédules ne peuvent pas hériter le Royaume de Dieu. Voir \vref{Jn. 1:1-14}~; \vref{Jn. 14:6}~; \vref{Jn. 20:24-29} et \vref{Ap. 21:8}.

\DicoEntry{INIQUITÉ}\textit{, du grec «~adikia~»~: «~injustice, tortuosité d'un cœur, violation volontaire de la loi~»}\newline
Tout ce qui constitue une violation de la loi* et la justice de Dieu. Voir \vref{Ro. 6:13}~; \vref{2 Pi. 2:13} et \vref{1 Jn. 5:17}.

\DicoEntry{MYSTEÈRE DE L'INIQUITÉ}\textit{}\newline
Voir commentaire \vref{2 Th. 2:7}.

\DicoEntry{INTERCESSION}\textit{, de l'hébreu «~palal~»~: «~intervenir, s'interposer, prier, agir en médiateur~»}\newline
Sous l'Ancienne Alliance, le grand prêtre avait la mission d'intercéder pour les péchés du peuple en offrant des sacrifices. A présent, Jésus-Christ le grand prêtre à perpétuité et l'avocat intercède pour ses enfants après s'être offert en sacrifice pour les péchés de l'humanité. Les hommes peuvent aussi faire des prières d'intercession comme Abraham pour Lot, Moïse pour Marie et l'Eglise pour tous les hommes. Voir \vref{Ge. 18:16-33}~; \vref{Lé. 16}~; \vref{No. 12:10-15}~; \vref{1 Ti. 2:1}~; \vref{Hé. 9:11-15} et \vref{1 Jn. 2:1}.

\DicoEntry{ISAAC}\textit{, de l'hébreu «~Yitschaq~»~: «~il rit~»}\newline
Fils de la promesse qui naquit à Abraham et Sara dans leur vieillesse. Il fut épargné quand Yahweh demanda à Abraham de lui offrir son fils en sacrifice. Isaac épousa Rébecca avec qui il eut deux fils~: Esaü et Jacob. Voir \vref{Ge. 17:17-21}~; \vref{Ge. 22:1-13} et \vref{Ge. 25:19-26}.

\DicoEntry{ISAÏ}\textit{, de l'hébreu «~Yishay~»~: «~je possède~»}\newline
Bethléhémite, petit-fils de Boaz et de Ruth, fils d'Obed et père de David. Voir \vref{Ru. 4:13-22}.

\DicoEntry{ISMAËL}\textit{, de l'hébreu «~Yishma'e'l~»~: «~Dieu entend~»}\newline
Fils d'Abraham et d'Agar, servante de Sara. Béni par Yahweh même après avoir été chassé avec sa mère par Sara, il devint le père des douze tribus ismaélites. Voir \vref{Ge. 16} et \vref{Ge. 25:12-16}.

\DicoEntry{ISSACAR}\textit{, de l'hébreu «~Yissaskar~»~: «~il donnera un salaire~»}\newline
Fils de Jacob et Léa, il devint l'ancêtre de la tribu d'Isaacar. Voir \vref{Ge. 30:18} et \vref{Ge. 49:14}.

\DicoEntry{ISRAËL}\textit{, de l'hébreu~: «~Yisra'el~»~: «~Dieu prévaut~»}\newline
Nom que Dieu donna à Jacob* après avoir lutté avec lui. Il s'agit également du nom désignant le peuple issu des douze fils de Jacob et le territoire que Dieu leur donna en héritage dont Jérusalem était la capitale. Après le schisme*, Israël se rapportait au royaume du nord composé de dix tribus. Voir \vref{Ge. 32:28}~; \vref{De. 33:5} et \vref{1 R. 12:1-24}.

\DicoEntry{IVRAIE}\textit{, du grec «~zizanion~»~: «~ivraie, ressemblant au blé, mais avec des grains noirs~»}\newline
Comme le blé, l'ivraie est une plante de la famille des graminées, mais c'est une mauvaise semence qui étouffe le blé. Elle représente les enfants du diable qui s'introduisent discrètement parmi les enfants de Dieu et qui en seront séparés uniquement à la fin du monde* pour aller vers la damnation éternelle. Voir \vref{Mt. 13:24-30,36-42}.

\DicoEntry{JACOB}\textit{, de l'hébreu «~Ya`aqob~»~: «~celui qui prend par le talon~» ou «~qui supplante~»}\newline
Fils d'Isaac et de Rebecca et frère jumeau d'Esaü. Il usa de stratagèmes pour ravir le droit d'aînesse ainsi que la bénédiction qui devaient revenir à son frère Esaü. Après avoir fui ce dernier, il se réfugia chez son oncle Laban dont il épousa les deux filles~: Léa et Rachel. De retour en Canaan après plusieurs années, Yahweh le rencontra en chemin et changea son nom en Israël. Jacob eut douze fils qui formèrent par la suite la nation d'Israël. Voir \vref{Ge. 25:21-34}~; \vref{Ge. 27-28}~; \vref{Ge. 29:1-30} et \vref{Ge. 49:1-28}.

\DicoEntry{JACQUES}\textit{, de l'hébreu~: «~Iakob~»~: «~qui supplante~» (variante de Jacob)}\newline
1. Fils de Zébédée et frère de Jean. Un des douze apôtres. Le roi Hérode Agrippa Ier* le fit mourir par l'épée. Voir \vref{Mt. 4:21-22}~; \vref{Lu. 6:12-16}~; \vref{Mc. 9:2-8}~; \vref{Mc. 14:32-33} et \vref{Ac. 12:1-2}.
\\2. Fils d'Alphée, un des douze apôtres~; il était aussi appelé Jacques le mineur. Voir \vref{Mt. 10:1-4}~; \vref{Mc. 15:40} et \vref{Lu. 6:12-16}.
\\3. Frère du Seigneur et apôtre, auteur de l'épître de Jacques. Voir \vref{Ac. 15:13-21}~; \vref{Ga. 1:19} et \vref{Mc. 6:3}.
\\4. Père de Jude, l'apôtre. Voir \vref{Lu. 6:16} et \vref{Ac. 1:13}.

\DicoEntry{JAPHET}\textit{, de l'hébreu «~Yepheth~»~: «~ouvert~», «~qui s'étend~»}\newline
Dernier des trois fils de Noé. Voir \vref{Ge. 10:1}.

\DicoEntry{JEAN}\textit{, de l'hébreu «~Yowchanan~» et du grec «~Ioannes~»~: «~Yahweh a fait grace~»}\newline
1. Fils de Zébédée, frère de Jacques et disciple aimé du Seigneur. Jean fut l'auteur de l'évangile éponyme, des trois épîtres qui portent son nom et de l'Apocalypse. Voir \vref{Mt. 10:2} et \vref{Jn. 13:23}.
\\2. Fils de Zacharie et Elisabeth, cousin de Jésus. Plus connu sous le nom de Jean-Baptiste, il fut envoyé pour préparer le chemin du Seigneur. Il fut décapité par Hérode Antipas*. Voir \vref{Lu. 1}~; \vref{Mal. 3:1-6}~; \vref{Mt. 1:12}~; \vref{Lu. 7:28} et \vref{Mt. 14:1-12}.

\DicoEntry{JELEK}\textit{, de l'hébreu «~yekeq~»~: «~jeune sauterelle~»}\newline
Désignant les sauterelles, il est souvent employé dans les Ecritures pour symboliser un grand nombre ou le dévoreur que Dieu envoie. Voir \vref{Joë. 1:4} et \vref{Na. 3:15-16}.

\DicoEntry{JÉRÉMIE}\textit{, de l'hébreu «~Yirmeyah~»~: «~celui que Yahweh a désigné~»}\newline
Fils de Hilkija, issu d'une famille de prêtres. Prophète de Yahweh, Jérémie fut appelé dès son plus jeune âge et exerça un ministère prophétique avant et pendant les premières années de déportation. Appelé à être eunuque*, il ne se maria jamais et n'eut point d'enfant. Il fut l'auteur des livres Jérémie et Lamentations de Jérémie.

\DicoEntry{JÉRICHO}\textit{, de l'hébreu «~Yeriychow~»~: «~ville de la lune~» ou «~ville des palmiers~»}\newline
Ville située à l'est de la tribu de Benjamin, près des rives du Jourdain. A la sortie du désert, les espions hébreux y furent cachés par Rahab la prostituée~; Jéricho fut ensuite détruite et livrée miraculeusement entre les mains d'Israël. C'est à Jéricho que Jésus guérit l'aveugle Bartimée et fut reçu par Zachée. Voir \vref{Jos. 2,6}~; \vref{Mc. 10:46-53} et \vref{Lu. 19:1-10}.

\DicoEntry{JÉROBOAM}\textit{, de l'hébreu «~Yarob'am~»~: «~le peuple devient nombreux~»}\newline
Fils de Nebath et serviteur de Salomon, il devint plus tard son ennemi. Après le schisme, il fut le premier roi du royaume du nord sur lequel il régna vingt-deux ans. Il fut une occasion de chute pour le peuple qu'il plongea dans l'idolâtrie*. Voir \vref{1 R. 11:26-40}~; \vref{1 R. 12-13}.

\DicoEntry{JÉRUSALEM}\textit{, de l'hébreu «~Yeruwshalem~»~: «~fondement de la paix~»}\newline
Ville située en Palestine, au nord de la Judée. Lors de la conquête de Canaan, la ville fut sous le contrôle des Jébusiens. Aux environs du Xe siècle av. J.-C., David reprit la ville alors devenue forteresse jébusienne. Il en fit la capitale politique et religieuse du royaume en y faisant établir l'arche de l'alliance. Salomon y construisit le temple* sur le mont Morija. En 586 av. J.-C., bien après le schisme*, les Babyloniens la détruisirent. Elle fut rebâtie par Néhémie après le retour de la captivité babylonienne. Jésus-Christ se lamenta sur la ville à cause de son incrédulité* et y annonça sa future destruction. Jérusalem fut en effet détruite par le général romain Titus en 70 ap. J.-C puis de nouveau rebâtie. Lors de son retour glorieux, le Seigneur Jésus posera ses pieds sur le Mont des Oliviers* qui est situé à Jérusalem. Le livre d'Apocalypse annonce après la fin du monde l'apparition de la Nouvelle Jérusalem, cité céleste. Voir \vref{2 S. 5:6-9}~; \vref{2 S. 6}~; \vref{Za. 14:1-4}~; \vref{2 Ch. 3:1}~; \vref{Lu. 19:41-44}~; et \vref{Ap. 21:2}.

\DicoEntry{JÉSUS}\textit{, de l'hébreu «~Yehowshuwa~»~: «~Yahweh est salut~»}\newline
Fils de l'homme* et fils de Dieu*, Jésus est le Dieu vivant manifesté en chair. Il fut conçu dans le ventre de Marie par la puissance du Saint-Esprit alors que cette dernière n'avait point connu d'homme. Selon les recherches de l'historien Flavius Josèphe, sa date de naissance se situerait autour de l'an 6 av. J.-C. - l'an zéro n'étant qu'une indication approximative. Fils adoptif de Joseph le charpentier et cousin de Jean Baptiste, il vécut la plus grande partie de sa vie en Galilée, dans la ville de Nazareth. Vers l'âge de 30 ans, il se fit baptiser dans le Jourdain et commença par la suite son ministère public. Grâce à sa vie exemplaire sans péché, il put se présenter comme une offrande agréable à Dieu répondant aux exigences de la justice divine pour sauver le monde. Remplissant toutes les prophéties relatives au Messie*, il fut trahi par un de ses disciples, Judas Iscariot*. Arrêté, maltraité puis crucifié, il mourut portant le poids des péchés de l'humanité, mais il ressuscita le troisième jour. Le salut réside dans la foi en son nom. Vivant de toute éternité, Jésus est le Dieu véritable et la vie éternelle. Sa justice ne tardera pas à se manifester~: il revient à toute vitesse. Voir \vref{Es. 53}~; \vref{Mt. 1:18-25}~; \vref{Mt. 2:23}~; \vref{Mc. 2:28}~; \vref{Lu. 1:36}~; \vref{Lu. 6:16}~; \vref{Lu. 24:46}~; \vref{Jn. 1:34}~; \vref{Ac. 4:12}~; \vref{2 Co. 5:21}~; \vref{1 Ti. 3:16}~; \vref{1 Pi. 2:21-25}~; \vref{1 Jn. 5:20} et \vref{Ap. 22:20}.

\DicoEntry{JETHRO}\textit{, de l'hébreu «~Yithrow~»~: «~son abondance, excellence~»}\newline
Prêtre de Madian chez qui Moise se réfugia après avoir fui l'Egypte. Il donna sa fille Séphora* pour femme à Moïse*. Voir \vref{Ex. 2:15-21}.

\DicoEntry{JEÛNE}\textit{, de l'hébreu «~tsuwn~»~: «~s'abstenir de nourriture~» et du grec «~nesteia~»~: «~le jeûne, un exercice volontaire et religieux~»}\newline
Privation totale ou partielle de nourriture dans le but d'humilier sa chair et d'adresser à Dieu des prières spécifiques. Le jeûne doit être exempt de toute hypocrisie et accompagné d'actes de justice pour être agréé de Dieu. Voir \vref{Es. 58}~; \vref{Est. 4:16}~; \vref{Da. 10}~; \vref{Mt. 6:16-18}~; \vref{Lu. 2:37}.

\DicoEntry{JÉZABEL}\textit{, de l'hébreu «~'Iyzebel~»~: «~Baal est l'époux~» ou «~l'impudique~»}\newline
Fille d'Ethbaal, roi de Sidon, et femme d'Achab roi d'Israël, elle extermina les prophètes de Yahweh et accueillait huit cent cinquante faux prophètes à sa table. Elle conduisit le peuple d'Israël dans l'idolâtrie au temps d'Elie*. Jézabel est associée à l'esprit du même nom qui prolifère de faux enseignements et entraîne le peuple de Dieu dans l'impudicité. Voir \vref{1 R. 16:31}, \vref{1 R. 18:4,19} et \vref{Ap. 2:20}.

\DicoEntry{JOB}\textit{, de l'hébreu «~'Iyowb~»~: «~haï, ennemi~» ou «~Je m'exclamerai~»}\newline
Originaire du pays d'Uts, homme prospère dont Yahweh témoigna l'intégrité et la droiture. Il subit en très peu de temps une succession de malheurs que Dieu permit pour se révéler à lui. Son histoire est racontée dans le livre portant son nom.

\DicoEntry{JOËL}\textit{, de l'hébreu «~Yow'el~»~: «~Yahweh est Dieu~»}\newline
Fils de Pethuel, il exerça la fonction de prophète dans le royaume de Juda. Il annonça la venue du Saint-Esprit sur toute chair à la fin des temps. Le contenu de son message se trouve dans le livre éponyme.

\DicoEntry{JONAS}\textit{, de l'hébreu «~Yonah~»~: «~colombe~»}\newline
Prophète de Yahweh envoyé à Ninive pour leur annoncer la destruction de leur ville. Son refus d'obéir à Dieu le conduisit dans le ventre d'un grand poisson. Son histoire est racontée dans le livre portant son nom.

\DicoEntry{JONATHAN}\textit{, de l'hébreu «~Yehownathan~»~: «~Yahweh a donné~»}\newline
Fils du roi Saül, homme de guerre reconnu. Lié à David par une très forte amitié, il protégea plusieurs fois ce dernier des relents meurtriers de son père. Il mourut à la bataille de Guilboa avec son père et ses frères. Voir \vref{1 S. 14:1-15}, \vref{1 S. 18:1-4}~; \vref{1 S. 19:1-8}~; \vref{1 S. 20} et \vref{1 S. 31:1-2}.

\DicoEntry{JOSAPHAT}\textit{, de l'hébreu «~Yehowshaphat~»~: «~Yahweh a jugé~»}\newline
Fils d'Asa et d'Azuba, il fut roi de Juda pendant vingt-cinq ans. Il eut un règne prospère et fit ce qui est droit aux yeux de Yahweh. Voir \vref{1 R. 15:24}~; \vref{1 R. 22:41-46} et \vref{2 Ch. 17}.

\DicoEntry{JOSEPH}\textit{, de l'hébreu «~Yowceph~»~: «~que Yahweh ajoute~» ou «~il enlève~»}\newline
1. Fils de Jacob et Rachel. Vendu comme esclave par ses frères, il devint, après plusieurs années de prison, gouverneur d'Egypte. Ses fils, Ephraïm et Manasée, furent adoptés par son père Jacob et furent les pères de deux des douze tribus d'Israël. Voir \vref{Ge. 30:22-24}~; \vref{Ge. 37,39,40,45,46}~; \vref{Ge. 48:5} et \vref{Jos. 14:4}.
\\2. Fils d'Héli, charpentier originaire de la tribu de Juda. Epoux de Marie, la mère de Jésus. Voir \vref{Mt. 1:18-25} et \vref{Mt. 13:55}.

\DicoEntry{JOSIAS}\textit{, de l'hébreu «~Yo'shiyah~»~: «~Yahweh guérit~»}\newline
Fils d'Amon, il devint roi de Juda à huit ans et y régna durant trente et un ans. Grand réformateur, à l'origine d'un grand réveil spirituel, il répara le temple, purifia le royaume des idoles et conclut une alliance de fidélité envers Yahweh. Voir \vref{2 R. 22-23}.

\DicoEntry{JOSUÉ}\textit{, de l'hébreu «~Yehowshuwa`~»~: «~Yahweh est salut~»}\newline
Fils de Nun de la tribu d'Ephraïm, choisi par Dieu pour succéder à Moïse. Accompagné de la puissante main de Dieu, il conduisit Israël à entrer en possession de Canaan. Son histoire se trouve dans le livre portant son nom.

\DicoEntry{JOURDAIN}\textit{, de l'hébreu «~Yarden~»~: «~celui qui descend~»}\newline
Très certainement le fleuve le plus connu des Ecritures, il est situé aux limites est de l'actuel territoire d'Israël. Josué et le peuple d'Israël passèrent le fleuve à sec. De même, Elie, puis Elisée, partagèrent les eaux du fleuve en deux. Après s'y être baigné sept fois sur les conseils d'Elisée, Naaman fut guéri de la lèpre. Jésus se fit baptiser par Jean dans le Jourdain. Voir \vref{Jos. 3}~; \vref{2 R. 2:8,12-14}~; \vref{2 R. 5:10-14} et \vref{Mt. 3:13-17}.

\DicoEntry{JOUR DU SEIGNEUR}\textit{}\newline
Jour où Yahweh manifestera sa justice et frappera les nations à cause de leurs péchés. Ce jour arrivera comme un voleur et surprendra beaucoup. Voir \vref{Es. 13:6-16}~; \vref{So. 1}~; \vref{2 Pi. 3:10}.

\DicoEntry{JUDA}\textit{, de l'hébreu «~Yehuwdah~»~: «~qu'il (Dieu) soit loué~»}\newline
Fils de Jacob et Léa, il est le père de la tribu du même nom installée au sud de Canaan. Sa descendance reçut la prédominance et la royauté~; David et Jésus-Christ étaient issus de cette tribu. Après le schisme*, Juda désigna aussi le nom du royaume du sud composé des tribus de Juda et Benjamin. Voir \vref{Ge. 29:35}~; \vref{Ge. 49:8-12}~; \vref{Jos. 15:1-12}~; \vref{1 R. 12:16-24}~; \vref{Mt. 1:1-16}.

\DicoEntry{JUDAS ISCARIOT}\textit{, de l'hébreu «~Yehuwdah~»~: «~qu'il (Dieu) soit loué~»}\newline
Fils de Simon Iscariot, il fut un des douze disciples de Jésus-Christ et était chargé de la trésorerie. Il trahit le Seigneur, ce qu'il regretta amèrement et le poussa à se suicider. Voir \vref{Mt. 26:14-16}~; \vref{Mt. 27:3-5}~; \vref{Lu. 6:16} et \vref{Jn. 12:4-6}.

\DicoEntry{JUDE}\textit{, de l'hébreu «~Yehuwdah~»~: «~qu'il (Dieu) soit loué~»}\newline
1. Fils de Jacques, un des douze apôtres, connu également sous le nom de Thadée. Voir \vref{Mc. 3:18} et \vref{Lu. 6:16}.
\\2. Prophète également appelé Barsabas, compagnon d'œuvre de Silas. Voir \vref{Ac. 15:22,32}.
\\3. Frère du Seigneur, auteur d'une épître qui porte son nom. Voir \vref{Mt. 13:55}~; \vref{Mc. 6:3} et \vref{Jud. 1:1}.

\DicoEntry{JUDÉE}\textit{, de l'hébreu «~Yehuwdah~»~: «~qu'il (Dieu) soit loué~»}\newline
Région située au sud de la Palestine où se trouvent notamment Jérusalem et Bethléhem. Elle correspondrait approximativement au territoire de l'ancien royaume de Juda. Ce terme n'est pas utilisé dans le Tanakh. Voir \vref{Mt. 2:1}~; \vref{Mc. 1:5} et \vref{Ga. 1:22}.

\DicoEntry{JUGE, JUGEMENT}\textit{, de l'hébreu «~shaphat~»~: «~juger, gouverner, défendre, punir~», «~agir comme un législateur, juge ou gouverneur~», «~exécuter un jugement~»}\newline
Dans toute la Parole, Yahweh est présenté comme le juge droit et incorruptible. Après la sortie d'Egypte, des juges ont été suscités par Dieu au milieu d'Israël pour délivrer le peuple de ses ennemis et le ramener vers lui (voir livre des Juges). Le Seigneur a toujours envoyé des prophètes pour annoncer ses jugements et ses décisions ainsi que des juges pour faire respecter sa loi. Sous la Nouvelle Alliance, l'homme spirituel est appelé à juger (discerner selon la Parole), mais condamner et décider du sort final d'une personne demeure la prérogative de Dieu. Yahweh est en effet le juste juge qui siège et tranche non seulement au tribunal de Christ, mais également au jugement dernier. Voir \vref{Ge. 18:25}~; \vref{Jé. 11:20}~; \vref{2 Co. 5:10} et \vref{Ap. 20:11-15}.

\DicoEntry{JUPITER}\textit{, du grec «~Zeus~»~: «~un père des secours~»}\newline
Divinité romaine assimilée à Zeus chez les Grecs. Lors d'une guérison miraculeuse à Lystres, la foule pensa voir en Paul la réincarnation de Mercure et en Barnabas celle de Jupiter. Pour cela, on voulut les adorer, ce qu'ils refusèrent avec véhémence. Voir \vref{Ac. 14:8-15}.

\DicoEntry{JUSTE, JUSTICE}\textit{, de l'hébreu «~tsedeq~»~: «~droiture, exactitude, conforme~» ou encore «~tsadiq~»~: «~juste, exact, innocent~» et du grec «~dikaiosune~»~: «~la condition acceptable par Dieu~» ou «~intégrité, vertu, pureté de vie, droiture~»}\newline
En qualité de juste juge, Yahweh a toujours recherché cette qualité chez l'homme, mais il ne l'a pas trouvé déclarant que nul n'est juste. Au travers de l'œuvre de la croix et avec l'aide du Saint-Esprit, le chrétien peut à présent marcher dans la justice de Dieu. Il est appelé à la rechercher plus que tout et à devenir esclave de la justice. Voir \vref{Mt. 5-7}~; \vref{Lu. 1:75}~; \vref{Ro. 3:10}~; \vref{Ro. 6:18} et \vref{2 Ti. 4:8}.

\DicoEntry{JUSTIFICATION}\textit{, du grec «~dikaiosis~»~: «~état du juste~»}\newline
Au travers de l'œuvre de la croix, Jésus-Christ est devenu la justification de tous ceux qui croient en lui, les rendant acceptables et libres de toute culpabilité. Voir \vref{Ro. 3:23-28}~; \vref{Ro. 4:25} et \vref{Ro. 5:18}.

\DicoEntry{KORÉ}\textit{, de l'hébreu «~Qorach~»~: «~chauve~»}\newline
Fils de Jitsehar, originaire de la tribu de Lévi, il se révolta avec Dathan* et Abiram* contre Moïse* et Aaron*. Suite à sa rébellion, il périt avec les gens de sa maison. Voir \vref{No. 16:1-35}.

\DicoEntry{LAÏC}\textit{, du grec «~laos~»~: «~peuple~»}\newline
Notion propre à l'Eglise catholique romaine. Opposé au clergé*, les laïcs sont les autres membres de l'église, ceux qui n'ont pas de fonction dirigeante, mais qui sont tout de même appelés à honorer Dieu dans leur vie et faire connaître leur foi au milieu du monde.

\DicoEntry{LANGUES}\textit{, de l'hébreu «~lashown~»~: «~langue, langage~» et du grec «~glossa~»~: «~la langue~» ou «~le langage d'un peuple particulier~»}\newline
Selon les Ecritures, les langues sont nées à Babylone* lorsque les hommes se sont rebellés contre la volonté de Yahweh et que ce dernier a confondu leur langage dans le but de les disperser. Dans la Parole sont cités différents types de langues, chacune liée à un don ou une manifestation particulière de l'Esprit de Dieu. Lors de l'effusion du Saint-Esprit à la Pentecôte, les disciples reçurent la capacité de parler des merveilles de Dieu dans des langues étrangères. Il s'agit du don spirituel* appelé la diversité des langues et concerne uniquement les langues usuelles. Il existe également des langues angéliques ou dites inconnues que le croyant peut utiliser pour s'adresser à Dieu. Les langues étrangères tout comme les langues des anges peuvent donner lieu à une interprétation, c'est ce qu'on appelle le don d'interpréter les langues. Voir \vref{Ge. 11}~; \vref{Ac. 2:1-11}~; \vref{1 Co. 12:10}~; \vref{1 Co. 13:1} et \vref{1 Co. 14:1-14,26-27}.

\DicoEntry{LAODICÉE}\textit{, du grec «~Laodikeia~»~: «~justice du peuple~»}\newline
Capitale de la Phrygie, l'une des provinces de l'Asie Mineure, réputée dans le domaine du commerce, notamment dans l'industrie textile. Ses vêtements et sa tapisserie principalement de couleur noire, firent sa renommée. Elle possédait une grande école de médecine qui fabriquait des remèdes réputés pour les yeux, notamment le fameux collyre. L'église de Laodicée est la dernière à qui fut adressée une lettre dans l'Apocalypse. Caractérisée par la tiédeur, l'affection aux choses terrestres et l'aveuglement spirituel, le Seigneur l'appela à la repentance*. Elle est l'image de l'église matérialiste. Voir \vref{Ap. 3:14-22}.

\DicoEntry{LAZARE}\textit{, du grec «~Lazaros~»~: «~Yahweh a secouru~»}\newline
1. Homme pauvre qui fut recueilli dans le sein d'Abraham après sa mort. Voir \vref{Lu. 16:19-21}.
\\2. Frère de Marthe et de Marie de Béthanie, et ami de Jésus-Christ qui le ressuscita des morts. Voir \vref{Jn. 11}.

\DicoEntry{LÉA}\textit{, de l'hébreu «~Le'ah~»~: «~lasse~»}\newline
Fille aînée de Laban et première femme de Jacob. Elle enfanta six fils, pères de six des douze tribus d'Israël (Ruben, Siméon, Lévi, Juda, Issacar et Zabulon) ainsi qu'une fille nommée Dina. Voir \vref{Ge. 29:16-23}~; \vref{Ge. 30:21} et \vref{Ge. 35:23}.

\DicoEntry{LÉMEC}\textit{, de l'hébreu «~Lemek~»~: «~puissant~»}\newline
Fils de Metuschaël et descendant de Caïn, il fut le premier polygame de l'histoire en prenant deux femmes~: Ada et Tsilla. Voir \vref{Ge. 4:16-24}.

\DicoEntry{LÈPRE}\textit{, de l'hébreu «~tsara'~»~: «~être morbide de peau~»}\newline
Commune en Egypte et en orient, maladie de la peau dont le virus peut se développer dans tout le corps. Contagieuse, elle peut même souiller les vêtements et les habitations. Sous la loi mosaïque, les personnes atteintes de cette maladie étaient considérées comme impures et devaient se tenir à l'écart. Durant son ministère, Jésus guérit plusieurs lépreux. Voir \vref{Lé. 13-14} et \vref{Lu. 17:11-14}.

\DicoEntry{LEVAIN}\textit{, de l'hébreu «~chametz~»~: «~ce qui est levé~»}\newline
Symbole du mal et de la corruption, le levain était interdit dans la quasi-totalité des offrandes. Jésus a assimilé le levain des pharisiens à l'hypocrisie, à la doctrine erronée. Les chrétiens sont appelés à faire disparaître le vieux levain et à devenir le levain du monde en y faisant progresser l'évangile du royaume. Voir \vref{Lé. 2:11}~; \vref{Mt. 16:6-12}~; \vref{Mt. 13:33} et \vref{1 Co. 5:6-7}.

\DicoEntry{LÉVI, LÉVITES}\textit{, de l'hébreu «~Leviy~»~: «~attachement~»}\newline
Fils de Jacob et Léa, Lévi participa avec son frère Siméon au massacre des hommes de la ville de Sichem après le viol de leur sœur Dina. Consacrés au service de Yahweh, ses descendants, les Lévites, n'eurent point d'héritage en Canaan, mais habitèrent différentes villes qui leur furent spécifiquement attribuées en Israël. Voir \vref{Ge. 29:34}~; \vref{Ge. 34}~; \vref{No. 18:20-24} et \vref{Jos. 13:14}.

\DicoEntry{LOI}\textit{, de l'hébreu «~towrah~»~: «~loi, direction, commandement~», «~loi mosaïque~»}\newline
L'ensemble des préceptes et ordonnances relatifs à l'alliance conclue entre Yahweh et le peuple hébreu, par l'intermédiaire de Moïse, est contenu dans les cinq premiers livres de la Bible appelée aussi «~le Pentateuque~». Selon la tradition juive, il existerait 613 commandements relatifs à la moralité, la vie en société et le culte rendu à Yahweh. L'homme en étant incapable, Jésus-Christ a accompli les exigences de la loi. Il est donc possible aux hommes d'obtenir le salut par la foi et non plus par les œuvres. La loi est maintenant gravée dans les cœurs des enfants de Dieu à qui le Saint-Esprit rappelle les paroles de Jésus. Voir \vref{Ex. 18:20}~; \vref{Ex. 24:12}~; \vref{Jn. 14:26} et \vref{Ro. 3:19-31}.

\DicoEntry{LOI DU PÉCHÉ}\textit{, du grec «~nomos~»~: «~toute chose établie, une coutume, un commandement~»}\newline
Loi spirituelle inscrite dans la chair qui pousse l'homme charnel à se révolter contre Dieu en commettant le péché. Voir \vref{Ro. 7:13-25}.

\DicoEntry{LOT}\textit{, de l'hébreu~: «~Lowt~»~: «~voile, couverture~»}\newline
Fils de Haran et neveu d'Abraham, Lot quitta Ur avec ce dernier avant de s'en séparer. Grâce à l'intercession d'Abraham, il fut sauvé de la destruction de Sodome avec ses deux filles. Ces dernières enivrèrent leur père et eurent des relations incestueuses avec lui de qui naquirent Moab, père des Moabites, et Amon, père des Ammonites. Voir \vref{Ge. 11:31}~; \vref{Ge. 13:1-13}~; et \vref{Ge. 19}.

\DicoEntry{LUC}\textit{, du grec «~Loukas~»~: «~qui donne la lumière~»}\newline
Médecin de métier, il fut un des compagnons d'œuvre de Paul et l'auteur de l'évangile qui porte son nom et du livre Actes des Apôtres. Voir \vref{Col. 4:14} et \vref{Phm. 1:24}.

\DicoEntry{MACÉDOINE}\textit{, du grec «~Makedonia~»~: «~terre étendue~»}\newline
Province romaine située au nord de la Grèce. Paul y effectua quelques voyages missionnaires et y implanta plusieurs assemblées. Voir \vref{Ac. 16:9-12}~; \vref{Ac. 20:1-3}~; \vref{1 Co. 8:1} et \vref{2 Co. 11:9} et \vref{Ro. 15:23}.

\DicoEntry{MADIAN}\textit{, de l'hébreu «~Midyan~»~: «~lutte, dispute~»}\newline
Un des fils issu de l'union d'Abraham* et Ketura, il devint l'ancêtre des Madianites, peuple qui habita à l'est de Canaan et au nord du désert d'Arabie. Voir \vref{Ge. 25:1-2}~; \vref{No. 31:1-12} et \vref{Jg. 6:2}.

\DicoEntry{MAGOG}\textit{, de l'hébreu «~Magowg~»~: «~territoire de montagne, qui domine~»}\newline
Fils de Japhet. Associé à Gog, il correspond aussi à la nation d'où vient le roi Gog qui fera la guerre à Dieu et à son peuple juste avant le jugement dernier. Voir \vref{Ge. 10:2,9} et \vref{Ap. 20:8}.

\DicoEntry{MAIN}\textit{, de l'hébreu «~yad~»~: «~main, force, pouvoir~»}\newline
Partie du corps permettant de toucher, saisir ou posséder, elle représente aussi l'action, la provision, la protection ou le joug. Tout au long des Ecritures, la main de Yahweh révèle sa puissance et sa bienveillance. Voir \vref{Es. 40:2}~; \vref{Jé. 18:6}~; \vref{Ps. 71:4}~; \vref{Pr. 10:4}~; \vref{Mc. 14:58}~; \vref{Lu. 11:20} et \vref{Ac. 11:21}.

\DicoEntry{MALACHIE}\textit{, de l'hébreu «~Mal`akiy~»~: «~mon messager~»}\newline
Dernier prophète du Tanakh, il condamna les péchés et l'hypocrisie des enfants d'Israël et annonça la venue de Jean-Baptiste. L'ensemble de ses prophéties est contenu dans le livre portant son nom.

\DicoEntry{MALÉDICTION}\textit{, de l'hébreu «~arar, meerah, qelalah~» et du grec «~ara, katara~»}\newline
Parole attirant le malheur sur un bien, une personne ou un peuple. Dieu a le pouvoir de maudire et aussi d'écarter toute malédiction. La malédiction de Dieu, contraire de la bénédiction*, fait suite à la désobéissance. A la nouvelle naissance, toutes les chaînes de malédiction qui liaient le chrétien sont brisées. Le chrétien ne doit pas maudire, mais bénir en tout temps, même ses ennemis. Voir \vref{De. 28:15-68}~; \vref{Mt. 5:44}~; \vref{2 Co. 5:17} et \vref{Ro. 8:1}.

\DicoEntry{MALFAITEUR REPENTANT}\textit{}\newline
Un des hommes coupables qui fut crucifié à côté de Jésus. Son humilité, sa sincérité et sa repentance lui permirent d'accéder au salut, Jésus-Christ lui ayant garanti l'accès au paradis. Voir \vref{Lu. 23:33-43}.

\DicoEntry{MAMON}\textit{, du grec «~Mammonas~»~: «~richesses~»}\newline
Dieu de l'argent. Jésus utilisa ce terme pour personnifier la richesse que beaucoup idolâtrent et qui est par conséquent en concurrence avec Yahweh dans le cœur de certains. Voir \vref{Mt. 6:24}.

\DicoEntry{MANASSÉ}\textit{, de l'hébreu «~Menashsheh~»~: «~oublieux~»}\newline
1. Fils aîné de Joseph* et d'Asnath, adopté par Jacob avant sa mort, ancêtre de la tribu de Manassé. Voir \vref{Ge. 41:51}~; \vref{Ge. 48:5} et \vref{Jos. 14:4}.
\\2. Fils d'Ezéchias et de Hephtsiba, il fut l'un des pires rois du royaume de Juda qui régna 55 ans. Malgré le réveil impulsé par son père, il se détourna entièrement de Yahweh et servit des dieux étrangers. Voir \vref{2 R. 21:1-18}.

\DicoEntry{MANNE}\textit{, de l'hébreu «~man~»~: «~qu'est-ce que cela~?~»}\newline
Nourriture céleste - à l'aspect de la graine de coriandre et au goût de gâteau de miel - que Dieu donna quotidiennement aux Israélites durant toute leur marche dans le désert. \vref{Ex. 16:15,31-35}.

\DicoEntry{MARANATHA}\textit{, de l'araméen «~maran atha~»~: «~le Seigneur vient~» ou «~Seigneur, viens~»}\newline
Expression prononcée par Paul quand il s'adressa aux Corinthiens et qui doit également être le cri du cœur de tout enfant de Dieu. Voir \vref{1 Co. 16:22} et \vref{Ap. 22:17,20}.

\DicoEntry{MARC}\textit{, du grec «~Markos~»~: «~une défense~» ou «~grand marteau~»}\newline
Appelé aussi Jean, cousin de Barnabas, il fut la cause de la séparation de Paul et Barnabas. Il partit avec ce dernier à Chypre et devint par la suite un fidèle compagnon d'œuvre de Paul. Il écrivit l'évangile portant son nom. Voir \vref{Ac. 12:12}~; \vref{Ac. 15:36-39}~; \vref{Col. 4:10} et \vref{Phm. 24}.

\DicoEntry{MARDOCHÉE}\textit{, de l'hébreu «~Mordekay~»~: «~petit homme~»}\newline
Fils de Jaïr de la tribu de Benjamin*, il adopta Esther*, fille de son oncle. Il sauva la vie du roi Assuérus* en déjouant les plans de Bigthan et Théresch et préserva le peuple juif des desseins meurtriers d'Haman. Il devint puissant dans la maison du roi et instaura la fête du Purim. Voir le livre d'Esther.

\DicoEntry{MARIAGE}\textit{, de l'hébreu «~chathan~»~: «~devenir un gendre, s'allier~»}\newline
Bénédiction de Dieu, le mariage est une alliance en principe indissoluble entre un homme et une femme dans le but d'accomplir le plan de Dieu. Il doit être célébré dans le respect des autorités du pays dans lequel le couple se trouve et honoré de tous, particulièrement des parents dont la bénédiction est essentielle. Voir \vref{Ge. 2:22-24}~; \vref{Ge. 24:60}~; \vref{Pr. 18:22}~; \vref{1 Co. 7} et \vref{Hé. 13:4}. Voir commentaire en \vref{Mt. 19:6}.

\DicoEntry{MARIE}\textit{, de l'hébreu «~Miryam~»~: «~rébellion, obstination~»}\newline
1. Sœur de Moïse et d'Aaron, prophètesse. Elle se rebella contre Moïse et fut frappée par la lèpre, mais en guérit grâce à l'intercession de Moïse. Voir \vref{Ex. 15:20} et \vref{No. 12}.
\\2. Mère de Jésus~: Elle conçut, par la vertu du Saint-Esprit, Jésus homme. Elle devint une de ses disciples et se trouvait parmi ceux qui persévéraient dans la prière dans la chambre haute lors de l'effusion du Saint-Esprit promis. Voir \vref{Es. 7:14}~; \vref{Mt. 1:18-25}~; \vref{Mc. 15:40-41}~; \vref{Lu. 1:26-38} et \vref{Ac. 1:13-14}.
\\3. Marie de Magdala~: Elle fut délivrée de sept démons par Jésus qu'elle suivit pendant son ministère terrestre, et ce jusqu'à la croix. Elle fut mandatée par le Seigneur pour annoncer sa résurrection aux apôtres. Voir \vref{Mt. 27:55-56}~; \vref{Mc. 16:1-11}~; \vref{Lu. 8:2} et \vref{Jn. 20:1-18}.
\\4. Marie de Béthanie~: Sœur de Marthe* et de Lazare*, que Jésus ressuscita des morts. Contrairement à sa sœur, elle choisit la bonne part en restant aux pieds du Maître. Elle toucha le cœur de ce dernier en l'oignant d'un parfum de grand prix. Voir \vref{Lu. 10:38-42}~; \vref{Jn. 11:1-44} et \vref{Jn. 12:1-7}.

\DicoEntry{MARTHE}\textit{, du grec «~Martha~»~: «~maîtresse, dame~»}\newline
Sœur de Lazare* - dont elle fut témoin de la résurrection - et de Marie* de Béthanie, elle reçut Christ dans sa maison, mais ce dernier lui reprocha son activisme au détriment de l'écoute de sa Parole. Voir \vref{Lu. 10:38-42} et \vref{Jn. 11:1-44}.

\DicoEntry{MATTHIAS}\textit{, de l'hébreu «~Mattithyah~»~: «~don de Yahweh~»}\newline
Disciple de Jésus et témoin oculaire de son ministère, il fut désigné pour devenir l'un des douze apôtres en remplacement de Judas Iscariot* qui avait trahi le Seigneur pour ensuite se suicider. Voir \vref{Ac. 1:15-26}.

\DicoEntry{MATTHIEU}\textit{, du grec «~Matthaios~»~: «~don de Yahweh~»}\newline
Collecteur d'impôts, il fut l'un des douze apôtres de Jésus et l'auteur de l'évangile qui porte son nom. Voir \vref{Mt. 9:9} et \vref{Mt. 10:3}.

\DicoEntry{MEÉDIATEUR}\textit{, du grec «~mesites~»~: «~celui qui intervient entre deux parties~», «~intermédiaire de communication~»}\newline
Moïse a exercé cette fonction auprès du peuple d'Israël qui avait expressément demandé que Dieu ne leur parle pas directement. Christ, garant d'une Nouvelle Alliance, est à présent l'unique intermédiaire et médiateur entre Dieu et les hommes. Voir \vref{Ex. 20:19}~; \vref{1 Ti. 2:5}~; \vref{Hé. 8:6} et \vref{Hé. 9:15}.

\DicoEntry{MELCHISÉDEK}\textit{, de l'hébreu «~Malkiy-Tsedeq~»~: «~roi de justice~»}\newline
Roi de Salem et prêtre du Dieu Très-Haut, il était une apparition de Jésus-Christ avant son apparition. Il bénit Abraham après sa victoire contre Kedorlaomer. Jésus-Christ est grand prêtre selon l'ordre de Melchisédek. Voir \vref{Ge. 14:14-20}~; \vref{Hé. 5:5-10} et \vref{Hé. 6:20}.

\DicoEntry{MENSONGE}\textit{, de l'hébreu «~sheqer~»~: «~mensonge, déception, fausseté, tromperie, fraude~» et du grec «~pseudos~»~: «~fausseté consciente et intentionnelle~»}\newline
Modification de la vérité. Satan est appelé père du mensonge et les menteurs auront droit à la même sentence que lui. Voir \vref{Ex. 20:16}~; \vref{Jn. 8:44} et \vref{Ap. 21:8}.

\DicoEntry{MÉSOPOTAMIE}\textit{, de l'hébreu «~'Aram Naharayim~»~: «~pays entre deux fleuves~»}\newline
Située entre le Tigre et l'Euphrate, région correspondant à l'actuel Irak. Avant son appel, Abraham vivait à Ur en Chaldée qui se trouvait au sud de la Mésopotamie. Voir \vref{Ge. 11:31}.

\DicoEntry{MESSIE}\textit{, de l'hébreu «~mashiyach~»~: «~oint, celui qui est l'oint~»}\newline
Voir CHRIST.

\DicoEntry{MICHÉE}\textit{, de l'hébreu «~Miykayehuw~»~: «~qui est comme Dieu~?~»}\newline
Originaire de Moréscheth, Michée exerça la fonction de prophète dans le royaume du sud au temps d'Ezéchias, roi de Juda. L'ensemble de ses prophéties se trouve dans le livre éponyme.

\DicoEntry{MICHEL ou MICHAËL}\textit{, de l'hébreu «~Miyka'el~»~: «~qui est semblable à Dieu~?~»}\newline
Archange* de Dieu, il est un des principaux chefs des anges*. Souvent présent dans les grandes batailles, il lutta notamment contre le roi de Perse et contre le diable. Voir \vref{Da. 10:13-21}~; \vref{Jud. 1:9} et \vref{Ap. 12:7}.

\DicoEntry{MILLE}\textit{, du grec «~million~»~: «~distance de mille pas~»}\newline
Unité de mesure romaine correspondant à 1480m environ. Voir \vref{Mt. 5:41}.

\DicoEntry{MILLÉNIUM}\textit{}\newline
Période de paix de mille ans durant laquelle le Seigneur régnera sur la terre. Voir \vref{Es. 11,12} et \vref{Ap. 20:2-7}.

\DicoEntry{MINISTÈRE}\textit{, du grec «~diakonia~»~: «~service~», dérivé du mot grec «~diakonos~»~: «~domestique~»}\newline
voir SERVICE. 

\DicoEntry{MISÉRICORDE}\textit{, de l'hébreu «~checed~»~: «~bonté, miséricorde, fidélité~» et du grec «~eleos~»~: «~bonne volonté envers le misérable associée à un désir de l'aider~»}\newline
Comme en témoigne le plan du salut* qu'il a déployé, Dieu est riche en miséricorde. Le disciple de Christ doit comme son maître se revêtir d'entrailles de miséricorde afin de représenter le Royaume de Dieu. \vref{Ge. 24:7}~; \vref{No. 24:18}~; \vref{Mt. 9:13}~; \vref{Lu. 1:78}~; \vref{Ro. 11:31} et \vref{2 Jn. 1:3}.

\DicoEntry{MOAB}\textit{, de l'hébreu «~Mow'ab~»~: «~issu d'un père~»}\newline
Fils de Lot*, né de sa relation incestueuse avec sa fille aînée, il donna naissance au peuple des moabites. Ils s'établirent au sud-est de la mer morte et s'opposèrent plusieurs fois aux enfants d'Israël. Voir \vref{Ge. 19:37}~; \vref{Jg. 3:12}~; \vref{2 S. 8:2}~; \vref{Ez. 25:8-11}.

\DicoEntry{MODALISME}\textit{}\newline
Doctrine enseignée à Rome au début du troisième siècle par Sabellius selon laquelle le Père, le Fils et le Saint-Esprit sont différents aspects au travers desquels Dieu se révèle et non trois personnes distinctes. Réfutant ainsi la doctrine de la trinité* largement acceptée par les catholiques, Sabellius fut condamné par le pape Callixte à cause de son enseignement pourtant biblique. Voir \vref{1 Th. 3:11}~; \vref{2 Th. 2:16-17} et \vref{1 Jn. 5:20}.

\DicoEntry{MOÏSE}\textit{, de l'hébreu «~Mosheh~»~: «~tiré de~»}\newline
Issu de la tribu de Lévi, il fut miraculeusement sauvé du massacre des enfants de sa génération pendant la servitude d'Israël en Egypte. Il vécut les quarante premières années de sa vie dans la maison de Pharaon puis les quarante suivantes dans le désert auprès de Madian*. A l'issue de cette deuxième période, Yahweh se révéla à lui et le mandata pour délivrer le peuple d'Israël de la captivité égyptienne afin de le faire entrer dans la terre promise. Après l'avoir fait sortir au milieu des miracles et des prodiges, Moïse conduisit le peuple dans le désert pendant quarante années au cours desquelles il leur communiqua l'intégralité de la loi*. Il mourut à la porte de la terre promise à l'âge de cent vingt ans. On lui attribue l'écriture des cinq premiers livres du Tanakh. Voir \vref{Ex. 1-2}~; \vref{Ex. 12:40-41}~; \vref{Ex. 14:21-31}~; \vref{Ex. 24:12}~; \vref{De. 8:2}~; \vref{De. 34:5-7}~; \vref{Ac. 7:20-43} et \vref{Hé. 11:23-29}.

\DicoEntry{MOISSON}\textit{, de l'hébreu «~qatsiyr~»~: «~moisson, travail de la moisson, récolte~»}\newline
Sous la loi, la fête des prémices avait lieu lors de la moisson. Jésus utilise ce terme pour parler du champ missionnaire, les personnes à qui l'évangile doit être annoncé. Dans le cadre de la fin du monde, la moisson se rapporte au jugement de Dieu qui va apporter la séparation entre ses fils et les fils du diable. Voir \vref{Lé. 23:10-14}~; \vref{Mt. 9:37-38}~; \vref{Mt. 13:33-43}.

\DicoEntry{MOLOC}\textit{, de l'hébreu «~Molek~»~: «~roi, conseiller~»}\newline
Divinité vénérée par les Ammonites à qui il était coutume de sacrifier des enfants brûlés vifs. Les Israélites se prostituèrent plusieurs fois à Moloc. Voir \vref{1 R. 11:5-7} et \vref{2 R. 23:10}.

\DicoEntry{MONT DES OLIVIERS}\textit{}\newline
Colline située à l'est de Jérusalem près de la vallée du Cédron. C'est du Mont des Oliviers que Jésus fut enlevé au ciel après avoir donné ses dernières recommandations aux apôtres~; c'est à ce même endroit qu'il posera les pieds lors de son glorieux retour. Voir \vref{Za. 14:1-4} et \vref{Ac. 1:4-12}.

\DicoEntry{MORT}\textit{, de l'hébreu «~muwth~»~: «~mourir, tuer, être exécuté~» et du grec «~thanatos~»~: «~mort du corps~»}\newline
La Bible distingue deux morts. La première entra dans le monde suite à la désobéissance de l'homme et correspond à la séparation d'avec Dieu et à la mort physique. La deuxième mort concerne uniquement ceux dont le nom n'est pas écrit dans le livre de vie et correspond à la souffrance éternelle dans l'étang de feu. Voir \vref{Ge. 3}~; \vref{Ro. 5:12}~; \vref{Ro. 6:23} et \vref{Ap. 20:11-15}.

\DicoEntry{MYRRHE}\textit{, de l'hébreu «~more~»~: «~myrrhe~»}\newline
Résine provenant de certains arbres d'Asie et d'Afrique, réputée pour son arôme de grand prix. Elle était utilisée sous forme d'huile pour l'onction sainte et pouvait atténuer les douleurs quand elle était mélangée au vin. Les mages offrirent de la myrrhe à Jésus lors de sa naissance. Voir \vref{Ex. 30:22-30}~; \vref{Mt. 2:11}~; \vref{Mt. 27:34} et \vref{Mc. 15:23}.

\DicoEntry{NAHUM}\textit{, de l'hébreu «~Nachuwm~»~: «~consolation, qui a compassion~»}\newline
Prophète de Yahweh né à Elkosch, il annonça la destruction de Ninive. L'ensemble de ses prophéties se trouve dans le livre portant son nom.

\DicoEntry{NAISSANCE D'EN HAUT}\textit{, du grec «~anothen~»~: «~depuis le haut, depuis un endroit plus élevé~»}\newline
Naissance d'eau et d'esprit symbolisant respectivement la Parole qui purifie et le Saint-Esprit* qui est le gage de l'appartenance à Dieu. La naissance d'en haut est l'œuvre du Saint-Esprit qui délivre une personne du royaume des ténèbres et la transporte dans le Royaume de Dieu. L'homme charnel devient alors spirituel, le cœur de pierre est ôté pour accueillir un cœur de chair, le citoyen terrestre se transforme en citoyen céleste et le vieil homme laisse place à une nouvelle créature. Voir \vref{Ez. 36:25-27}~; \vref{Jn. 3:1-8}~; \vref{Ja. 1:18}~; \vref{1 Co. 12:13} et \vref{2 Co. 5:17}~; \vref{Ep. 2:6}~; \vref{Ep. 5:26}~; \vref{1 Jn. 3:9}.

\DicoEntry{NATHAN}\textit{, de l'hébreu «~Nathan~»~: «~il (Yahweh) a donné~»}\newline
Prophète de Yahweh au temps du roi David. Il prophétisa le règne éternel de la postérité de David et la construction du temple par son fils. Il reprit David lorsque ce dernier fit assassiner Urie* pour prendre sa femme. Voir \vref{2 S. 7,12}.

\DicoEntry{NAZARÉEN, NAZIRÉEN}\textit{, de l'hébreu «~naziyr~»~: «~consacré ou voué~»}\newline
Terme pouvant désigner soit un habitant de la ville de Nazareth, soit une personne qui s'est consacrée à Yahweh dans le cadre d'un vœu de naziréat. Voir \vref{No. 6}.

\DicoEntry{NAZARETH}\textit{, du grec «~Nazareth~»~: «~verdoyant, germe, rejeton~»}\newline
Ville située dans la région de Galilée où Jésus passa la majeure partie de sa vie. Voir \vref{Mt. 2:22-23}.

\DicoEntry{NEBUCADNETSAR}\textit{, (règne~: 605 av. J.-C. – 562 av. J.-C.), «~Nebuwkadne'tstsar~»~: «~que Nebo protège la couronne, les frontières~» (origine inconnue)}\newline
Roi de Babylone, il mit fin au royaume de Juda en emmenant le peuple en captivité~; il détruisit le temple de Jérusalem. Il reçut l'interprétation de plusieurs songes au travers de Daniel* et reconnut le règne dominant et éternel de Yahweh. Voir \vref{2 R. 25}, \vref{Da. 1:1} et \vref{Da. 2,4}.

\DicoEntry{NÉHÉMIE}\textit{, de l'hébreu «~Nechemyah~»~: «~Yahweh a consolé~»}\newline
Fils d'Hacalia, il fut échanson du roi Artaxerxès* à Suse, pendant la captivité de Juda. Il entreprit la réparation des murailles de Jérusalem et initia une réforme en son temps. Il devint ensuite gouverneur de Juda. Son histoire est racontée dans le livre éponyme.

\DicoEntry{NÉPHILIM}\textit{, de l'hébreu «~nephiyl~»~: «~géant~», racine~: «~naphal~»~: «~tomber, chuter~»}\newline
Etres de grande taille nés de l'union des fils de Dieu et des filles des hommes avant le déluge. On en retrouve aussi en Canaan lorsque les douze espions hébreux étaient allés observer la terre promise. Voir \vref{Ge. 6:4} et \vref{No. 13:32-33}.

\DicoEntry{NEPHTHALI}\textit{, de l'hébreu «~Naphtaliy~»~: «~lutte, mon combat~»}\newline
Fils de Jacob* et de Bilha, servante de Rachel*, il est l'ancêtre de la tribu de Nephthali. Voir \vref{Ge. 30:8} et \vref{Ge. 49:21}.

\DicoEntry{NICODÈME}\textit{, du grec «~Nikodemos~»~: «~victorieux du peuple~»}\newline
Docteur de la loi, il s'approcha de Jésus de nuit, qui l'enseigna sur la naissance d'en haut. Après la crucifixion, il aida Joseph d'Arimathée pour embaumer le corps du Seigneur et pour le mettre dans un sépulcre. Voir \vref{Jn. 3:1-21} et \vref{Jn. 19:38-42}.

\DicoEntry{NICOLAÏTES}\textit{, du grec «~Nikolaites~»~: «~destruction du peuple~»}\newline
Secte suivant la doctrine de Nicolas, liée à la doctrine de Balaam, qui poussait à la consommation de viandes sacrifiées aux idoles et à l'impudicité. Voir \vref{Ap. 2:6,14-15}.

\DicoEntry{NIL}\textit{, de l'hébreu «~Shiychowr~»~: «~sombre, noir, boueux~»}\newline
Principal fleuve d'Egypte situé à l'est du pays. Voir \vref{Es. 23:3}~; \vref{Jé. 2:18}.

\DicoEntry{NIMROD}\textit{, de l'hébreu «~Nimrowd~»~: «~rebelle~»}\newline
Fils de Cush et descendant de Noé. Chasseur, il fut le premier homme puissant sur la terre et régna sur plusieurs villes dont Babel*. Voir \vref{Ge. 10:8-11} et \vref{Ge. 11:1-9}.

\DicoEntry{NINIVE}\textit{, de l'hébreu «~Niyneveh~»~: «~habitation de Ninus~»}\newline
Grande ville située sur la rive est du Tigre. Ses habitants se repentirent de leurs mauvaises voies suite à la prédication de Jonas, mais ils retombèrent dans le péché quelques années plus tard. Ninive fut finalement détruite sous le jugement de Dieu. Voir livres de Jonas et de Nahum.

\DicoEntry{NOCES}\textit{, du grec «~gamos~»~: «~fête du mariage~»}\newline
Festivités célébrant le mariage. Dans la tradition juive, les noces duraient sept jours même si une longue période pouvait parfois s'écouler entre la conclusion du mariage (accord des familles) et la consommation du mariage (nuit de noces). Ainsi, la fiancée devait se tenir prête pour les noces à tout moment. De même, l'Eglise se prépare à être enlevée par Jésus à tout moment pour les noces de l'Agneau qui seront célébrées au ciel pendant sept ans. Voir \vref{Ge. 29:27}~; \vref{Jg. 14:12}~; \vref{1 Th. 4:16-17} et \vref{Ap. 19:7}.

\DicoEntry{NOÉ}\textit{, de l'hébreu «~Noach~»~: «~repos, tranquillité~»}\newline
Fils de Lamech, il fut le père de trois fils~: Sem, Cham et Japhet. Qualifié d'homme juste et intègre en son temps, il trouva grâce devant Yahweh qui lui ordonna de construire une arche* pour le sauver lui, sa famille et une partie des animaux de la terre du déluge qui arrivait. Son obéissance sauva la race humaine. Il vécut 950 ans. Voir \vref{Ge. 6-9}.

\DicoEntry{NOUVELLE NAISSANCE}\textit{}\newline
Voir NAISSANCE D'EN HAUT.

\DicoEntry{OFFRANDE}\textit{, de l'hébreu «~minchah~»~: «~don, tribut, présent, oblation, sacrifice~»}\newline
Sous la loi, le peuple d'Israël avait reçu des prescriptions relatives aux offrandes agréables à Yahweh~; elles consistaient essentiellement en bétail et produits naturels et étaient offertes dans le cadre de cérémonies spécifiques. Des offrandes en argent pouvaient aussi être données, notamment pour soutenir l'entretien du temple. Sous la Nouvelle Alliance, les offrandes monétaires doivent être libres et volontaires~; l'offrande la plus importante aux yeux de Dieu reste la vie consacrée de ses enfants. Voir \vref{Lé. 1-7}~; \vref{Mc. 12:41-42}~; \vref{2 Co. 8:10-12}~; \vref{2 Co. 9:7}~; \vref{Ro. 12:1} et \vref{Ro. 15:15-16}.

\DicoEntry{OLIVIER}\textit{}\newline
Arbre fruitier donnant des olives avec lesquelles on produit de l'huile. Sous la loi de Moïse, elle était notamment utilisée pour alimenter les lampes qui devaient brûler continuellement dans le temple et pour oindre les personnes désignées par Dieu pour une tâche spécifique. L'olivier symbolise en outre le témoignage et la paix. Voir \vref{Ex. 27:20-21}~; \vref{Ex. 30:22-25}~; \vref{Jg. 9:8-9}~; \vref{1 S. 16:3} et \vref{1 R. 19:16}.

\DicoEntry{OMEGA}\textit{}\newline
Dernière lettre de l'alphabet grec désignant aussi la fin d'une chose (voir ALPHA et OMEGA).

\DicoEntry{ONCTION}\textit{, de l'hébreu «~mishchah~»~: «~portion consacrée, huile d'onction, oindre~» et du grec «~chrisma~»~: «~toute chose qui sert à enduire~» de la racine «~chrio~»~: «~oindre, imprégner les chrétiens des dons du Saint-Esprit~»}\newline
Sous l'Ancienne Alliance, l'onction était souvent accordée par l'action de verser de l'huile sur la tête de la personne ou de l'objet à consacrer. On oignait ainsi les prêtres, les rois et les prophètes selon leur mandat. Sous la Nouvelle Alliance, l'onction demeure en celui qui a reçu en lui le Seigneur Jésus. Toutefois, l'onction d'huile peut être pratiquée dans le cadre de la prière pour les malades. Voir \vref{Ex. 30:22-31}~; \vref{1 S. 16:3} et \vref{1 R. 19:16}~; \vref{Ac. 1:8}~; \vref{Ja. 5:14} et \vref{1 Jn. 2:20-27}.

\DicoEntry{ORDINATION}\textit{, du latin «~ordinatio~»~: «~action de disposer, de mettre en œuvre~»}\newline
Rite initiatique mis en place par l'Eglise catholique qu'on ne retrouve pas dans les Ecritures. Elle confère, par l'imposition des mains accompagnée d'une prière, la capacité d'exercer une fonction dirigeante au sein de l'église locale.

\DicoEntry{OTHNIEL}\textit{, de l'hébreu «~`Othniy'el~»~: «~Dieu est puissant~»}\newline
Fils de Kenaz et frère cadet de Caleb, il fut le premier juge en Israël, fonction qu'il exerça pendant 40 ans. Il délivra les enfants d'Israël du joug du roi de Mésopotamie, Cuschan-Rischeathaïm. Voir \vref{Jg. 3:8-11}.

\DicoEntry{OSÉE}\textit{, de l'hébreu «~Howshea`~»~: «~salut, sauve~»}\newline
Fils de Beéri, prophète qui, sous les ordres de Yahweh, épousa une prostituée pour illustrer l'infidélité des enfants d'Israël envers leur Dieu. L'histoire d'Osée et l'ensemble de ses prophéties se trouvent dans le livre portant son nom.

\DicoEntry{PAÏEN}\textit{, du latin «~paganus~» qui signifie «~paysan~», qui provient lui-même du mot «~pagus~» qui signifie «~campagne~».}\newline
Personne qui pratiquait une des religions polythéistes de l'Antiquité.

\DicoEntry{PAIX}\textit{, de l'hébreu «~shalowm~»~: «~état complet, perfection, bien-être, paix~» et du grec «~eirene~»~: «~état de tranquillité, paix entre les individus, harmonie, sécurité~»}\newline
Sous l'Ancienne Alliance, la paix était matérialisée par la prospérité, l'absence de guerre et de toutes sortes de malheurs. Sous la Nouvelle Alliance, la paix est un fruit de l'Esprit*, une promesse acquise en Jésus qui est lui-même le Prince de Paix. Différente de celle que le monde offre, la paix de Christ permet de rester confiant en toutes circonstances. Voir \vref{Lé. 26:6}~; \vref{Es. 26:12}~; \vref{Jn. 14:27}~; \vref{Jn. 16:33} et \vref{Ga. 5:22}.

\DicoEntry{PALMIER}\textit{, de l'hébreu «~tamar~»~: «~palmier, dattier~»}\newline
Arbre à tronc peu ou pas ramifié, on le retrouve essentiellement dans le désert. L'image du palmier fut utilisée en décoration dans le temple. Ses branches étaient utilisées pendant la fête des tentes. Symbole de la justice et de la victoire, on le retrouve lors de l'entrée royale de Jésus à Jérusalem et devant le trône de Dieu. Voir \vref{Lé. 23:40}~; \vref{1 R. 6:29}~; \vref{Jn. 12:12-13} et \vref{Ap. 7:9}.

\DicoEntry{PÂQUE}\textit{, de l'hébreu «~pecach~»~: «~passer outre, épargner~», «~sacrifice de la Pâque~» ou «~fête de la Pâque~»}\newline
Première fête du calendrier hébraïque, elle fut instituée par ordonnance perpétuelle dès la sortie d'Egypte. Cette fête commémore le salut de Yahweh accordé par le sacrifice de l'agneau~; elle préfigurait Christ, l'Agneau de Dieu qui est «~notre Pâque~». Voir \vref{Ex. 12}~; \vref{Lé. 23:5}~; \vref{Jn. 1:29} et \vref{1 Co. 5:7-8}.

\DicoEntry{PARADIS}\textit{, du grec «~paradeisos~»~: «~jardin~»}\newline
Lieu de repos et de félicité, le paradis fut ouvert par Jésus lors de sa résurrection. Il y emmena les justes décédés qui étaient jusque-là captifs dans le séjour des morts*. Les chrétiens rejoignent ce lieu céleste à leur décès, en attendant la résurrection*. A la croix, Christ garantit l'accès à ce lieu au malfaiteur repentant. Paul fut ravi à cet endroit où il entendit des paroles merveilleuses. Voir \vref{Lu. 23:43}~; \vref{2 Co. 12:2-4}~; \vref{Ep. 4:8-10} et \vref{Hé. 10:19-20}.

\DicoEntry{PARDON}\textit{, de l'hébreu «~nas a´~»~: «~action de lever, supporter, prendre~» et du grec «~aphesis~»~: «~libérer de l'esclavage~» ou «~oubli des péchés, rémission des peines~»}\newline
Sous l'Ancienne Alliance, le pardon était conditionné par les sacrifices d'animaux, mais Jésus-Christ a accompli cette prérogative en devenant la victime expiatoire pour nos péchés. En lui, l'homme repentant est pardonné de ses fautes et trouve également la force de pardonner à ceux qui l'offensent. Voir \vref{Lé. 4-6}~; \vref{Mt. 6:12,14-15}~; \vref{Jn. 1:29}~; \vref{Ac. 10:43} et \vref{1 Jn. 1:9}.

\DicoEntry{PARVIS}\textit{, de l'hébreu «~chatser~»~: «~cour, enclos, colonie, ville, village~» (voir illustration du temple)}\newline
Première des trois parties du tabernacle et du temple~; il s'agissait d'une cour dans laquelle se trouvait l'autel d'airain où se faisaient des sacrifices et la cuve d'airain contenant de l'eau pour la purification. Voir \vref{Ex. 27:9-19}.

\DicoEntry{PASTEUR}\textit{, de l'hébreu «~ra`ah~»~: «~berger~»}\newline
Un des cinq services d'\vref{Ep. 4:11} travaillant en collège, établi pour veiller pour le troupeau, le nourrir de la Parole et encourager les chrétiens à exercer pleinement et librement leur ministère. Toutefois, Jésus-Christ demeure le pasteur par excellence, le bon berger qui donne sa vie pour ses brebis et le gardien des âmes qui ne sommeille ni ne dort. Voir \vref{Ep. 4:11}~; \vref{Jn. 10:11-16}~; \vref{Ps. 23} et \vref{1 Pi. 2:25}.

\DicoEntry{PATMOS}\textit{, du grec «~Patmos~»~: «~mortel, fascinant~»}\newline
Petite île grecque de la mer Egée sur laquelle Jean fut exilé à la fin de sa vie. Il y reçut la révélation de l'Apocalypse. Voir \vref{Ap. 1:9}.

\DicoEntry{PAUL}\textit{, du grec «~Paulos~»~: «~petit~»}\newline
Issu de la tribu de Benjamin et né dans la ville de Tarse, son nom était initialement Saul*. Pharisien, son zèle excessif le poussa à persécuter violemment les chrétiens à la naissance de l'Eglise. Il rencontra Christ sur la route de Damas et devint par la suite l'apôtre des Gentils annonçant l'Evangile de villes en villes et de pays en pays au cours de nombreux voyages. Même en prison, il continua l'œuvre de Dieu en écrivant plusieurs lettres riches en enseignements que l'on peut retrouver dans le canon biblique. Voir \vref{Ac. 9-28} et les épîtres de Paul.

\DicoEntry{PÉAGER ou PUBLICAIN}\textit{, du grec «~telones~»~: «~un loueur, un collecteur de taxes~»}\newline
Les péagers d'origine juive étaient dépréciés de leurs compatriotes et assimilés à des pécheurs, car on les considérait comme des collaborateurs au service des romains. De plus, certains profitaient de leur fonction pour s'enrichir. Voir \vref{Mt. 9:10}~; \vref{Mt. 21:31}~; \vref{Lu. 3:12-13} et \vref{Lu. 19:2-8}.

\DicoEntry{PÉCHÉ}\textit{, de l'hébreu «~chatta'ah~»~: «~ce qui manque le but~» et du grec «~hamartano~»~: «~erreur, faux état d'esprit~»}\newline
Le péché entra dans le monde par la transgression d'Adam et Eve et tous les hommes en furent infectés. Origine de la séparation entre Dieu et les hommes, le péché conduit à la mort*. Voir \vref{Ge. 3}~; \vref{1 Co. 15:3}~; \vref{Ro. 5:12}~; \vref{Ro. 6:23}~; \vref{Ro. 8:1-4} et \vref{1 Pi. 2:21-24}.

\DicoEntry{PENTECÔTE}\textit{, du grec «~pentekoste~»~: «~le cinquantième jour~»}\newline
Fête annuelle juive célébrant la moisson des blés. La venue du Saint-Esprit* promis par Jésus eut lieu pendant la célébration de la Pentecôte. Voir \vref{Lé. 23:15-22}~; \vref{Jn. 16:7-11}~; \vref{Ac. 1:5} et \vref{Ac. 2:1-21}.

\DicoEntry{PHARAON}\textit{, de l'hébreu «~Par`oh~»~: «~grand palais~»}\newline
Titre donné aux rois égyptiens durant l'Antiquité. Voir \vref{Ge. 37:36} et \vref{Ge. 41}.

\DicoEntry{PHARISIEN}\textit{, du grec «~Pharisaios~»~: «~séparé~»}\newline
Secte juive dont les membres manifestaient un attachement excessif aux coutumes et traditions religieuses. Certains d'entre eux combattirent Jésus qui dénonça ouvertement leur fausse piété et leur dévouement hypocrite envers Dieu. Désirant la mort du Seigneur, ils participèrent à la conspiration qui précéda sa crucifixion. Voir \vref{Mt. 23:23-39}~; \vref{Mc. 7:1-13} et \vref{Jn. 18:2-3}.

\DicoEntry{PHILADELPHIE}\textit{, du grec «~Philadelpheia~»~: «~amour fraternel~»}\newline
Ville de Lydie en Asie Mineure. Irriguée par le fleuve Hermus, Philadelphie était une contrée très fertile, propice à l'agriculture et surtout à la culture de la vigne. Elle fut construite par le roi de Pergame, et plusieurs fois sujette à des tremblements de terre. Une des sept lettres d'Apocalypse s'adressait à l'église de Philadelphie. Cette dernière - contrairement aux autres qui cumulèrent des reproches – fut très encouragée par le Seigneur. Bien que située à 45 km de Sardes à laquelle elle était rattachée, l'Eglise de Philadelphie resta ferme en retenant la Parole de Dieu et ne se laissa pas influencer par les séductions du péché. Elle incarne ainsi l'Eglise que Jésus revient chercher, l'Eglise réveillée.

\DicoEntry{PHILÉMON}\textit{, du grec «~Philemon~»~: «~attentionné, qui embrasse~»}\newline
Disciple de Colosses qui recevait une église dans sa maison. Il avait un esclave nommé Onésime au sujet duquel Paul lui écrivit une lettre. Voir épître de Paul à Philémon.

\DicoEntry{PHILIPPE}\textit{, du grec «~Philippos~»~: «~aimant les chevaux~»}\newline
1. Homme de Bethsaïda, il fut l'un des douze apôtres* choisis par Jésus. Voir \vref{Mt. 10:3}~; \vref{Mc. 3:18} et \vref{Lu. 6:14}.
\\2. Un des sept diacres élus au sein de l'église de Jérusalem. Evangéliste*, il prêcha le Christ dans la ville de Samarie, à l'eunuque éthiopien qu'il baptisa et dans différentes villes. Voir \vref{Ac. 6:5}~; \vref{Ac. 8:4-8,26-40} et \vref{Ac. 21:8}.

\DicoEntry{PHILIPPES}\textit{, du grec «~Philippoi~»~: «~appartenant à Philippe~»}\newline
Fondée par Philippe II (382 av. J.-C. – 336 av. J.-C.) en 356 av. J.-C., ville grecque de Macédoine orientale. Située sur une voie romaine qui traversait les Balkans, elle est restée de taille modeste en dépit de son fort taux de fréquentation. Une église y naquit après la rencontre de Paul avec des femmes qui priaient à l'extérieur de la ville. L'apôtre leur écrivit une lettre qui figure dans le canon biblique. Voir \vref{Ac. 16:9} et l'épître aux Philippiens.

\DicoEntry{PHILISTINS}\textit{, de l'hébreu~: «~Pelesheth~»~: «~immigrants~»}\newline
Peuple qui habitait à l'extrême ouest de Canaan, le long de la mer Méditerranée. Ils furent plusieurs fois en conflit avec les Israélites~; Goliath était philistin. Voir \vref{Jg. 13-16} et \vref{1 S. 17}.

\DicoEntry{PHILOSOPHIE}\textit{, du grec «~philosophia~»~: «~amour de la sagesse~»}\newline
Discipline existant depuis l'Antiquité et ayant plusieurs courants de pensée en son sein comme les épicuriens* et les stoïciens*. Elle pousse ses adeptes à rechercher la sagesse par l'intelligence humaine. Paul invita les chrétiens à se garder de ces doctrines. Voir \vref{Ac. 17:16-20} et \vref{Col. 2:8}.

\DicoEntry{PHINÉES}\textit{, de l'hébreu «~Piynechac~»~: «~bouche de cuivre~»}\newline
Fils d'Eléazar, petit-fils d'Aaron et prêtre. Il se démarqua par son zèle pour Dieu pour arrêter un fléau sur Israël. A cette occasion, Yahweh fit alliance perpétuelle avec Phinées et sa descendance. Voir \vref{No. 25}.

\DicoEntry{PIERRE}\textit{, de l'hébreu «~Cephas~» et du grec «~Petros~»~: «~un roc ou une pierre~»}\newline
Fils de Jonas et frère d'André*, son nom était initialement Simon*. Pêcheur de métier originaire de la ville de Bethsaïda, il fut choisi comme apôtre pour les circoncis. Il écrivit deux épîtres portant son nom. Il aurait été crucifié à Rome. Voir \vref{Mt. 10:2}~; \vref{Jn. 1:42-44}~; \vref{Ga. 2:7-8}~; 1 Pi. et 2 Pi.

\DicoEntry{PILATE}\textit{, du grec «~Pilatos~»~: «~armé d'une lance~»}\newline
Gouverneur romain de la Judée en fonction pendant le ministère de Jésus. Il s'accorda avec son ennemi Hérode lorsqu'il fallut crucifier le Seigneur. N'ayant pas trouvé de crime en Jésus, il permit finalement sa crucifixion et fit mettre l'inscription suivante sur sa croix~: Jésus de Nazareth, roi des Juifs. Voir \vref{Lu. 3:1}~; \vref{Lu. 23:11} et \vref{Jn. 19:1-19}.

\DicoEntry{PRÉDESTINATION}\textit{, du grec «~proginosko~»~: «~avoir la connaissance avant~»}\newline
Révélant l'omniscience de Dieu qui connaît toutes choses à l'avance, la prédestination concerne l'œuvre de la croix prévue de toute éternité – l'Agneau ayant été immolé avant la fondation du monde. La prédestination est non pas la décision de Dieu d'envoyer certaines personnes en enfer, mais plutôt la capacité de Yahweh à connaître à l'avance ceux qui allaient devenir ses enfants d'adoption, transformés à l'image du Fils, en acceptant sa parole. Voir \vref{Jn. 1:12}~; \vref{Ro. 8:29-30}~; \vref{Ep. 1:5} et \vref{1 Pi. 1:19-20}.

\DicoEntry{PREMIER PRÊTRE}\textit{, de l'hébreu «~rosh~»~: «~tête, dessus, sommet, partie supérieure, chef, principal, premier, total, somme, hauteur, front, le devant, commencement~» et de «~kohen~»~: «~prêtre, intendant principal, ministre d'état~»}\newline
Sous la loi, le premier prêtre descendait d'Aaron. Il servait le Seigneur dans le sanctuaire* et enseignait la loi*. Tel un médiateur* entre Yahwhe et le peuple, il portait constamment le jugement de ce dernier pour qui il consultait Dieu au moyen de l'urim et du thummim. Il devait, une fois par an, entrer dans le Saint des saints et offrir des sacrifices d'animaux pour ses propres péchés et pour ceux du peuple. Par la suite, Jésus-Christ est devenu premier prêtre à perpétuité en s'offrant comme victime expiatoire et en présentant son sang une fois pour toutes dans le Saint des saints du temple céleste. Voir \vref{Ex. 28:30}~; \vref{Ex. 29:9}~; \vref{Esd. 2:63}~; \vref{No. 35:25}~; \vref{Hé. 4:14-16}~; \vref{Hé. 7:25-28} et \vref{Hé. 9:6-12,24-28}.

\DicoEntry{PRÉTOIRE}\textit{, du grec «~praitorion~»~: «~quartier général dans un camp romain, la tente du commandant en chef~»}\newline
Dans les évangiles, lieu de résidence des gouverneurs dans lequel se trouvaient notamment un tribunal et une prison. Voir \vref{Mt. 27:27}~; \vref{Jn. 18:28-29} et \vref{Ac. 23:35}.

\DicoEntry{PRIÈRE}\textit{, de l'hébreu «~palal~»~: «~intervenir, s'interposer, prier~» ou «~'athar~»~: «~prier, supplier, implorer~» et du grec «~proseuche~»~: «~prière adressée à Dieu~» ou «~parakaleo~»~: «~appeler à, convoquer, supplier, exhorter~»}\newline
Acte par lequel on s'approche de Dieu et on instaure un dialogue avec lui, en ayant foi dans sa présence et son action. Invité à prier constamment, le chrétien peut le faire pour se repentir, intercéder en faveur d'une situation particulière, demander quelque chose à Dieu, le remercier, le louer ou tout simplement lui exprimer son amour. La prière garde les enfants de Dieu dans la paix. Dieu connaissant toutes les pensées de l'homme, le plus important dans la prière reste l'écoute de la voix de Yahweh. Voir \vref{Ge. 20:17}~; \vref{1 S. 2:1}~; \vref{Job 22:27}~; \vref{Mt. 14:36}~; \vref{Ac. 16:9}~; \vref{1 Th. 5:17}~; \vref{Ph. 4:6-7} et \vref{1 Pi. 4:7}.

\DicoEntry{PROPHÈTE}\textit{, de l'hébreu «~nabiy'~»~: «~l'homme qui parle, celui qui est appelé, qui a reçu une inspiration~» et du grec «~prophetes~»~: «~celui qui interprète des oracles~», «~quelqu'un qui déclare ce qu'il a reçu par inspiration~»}\newline
Sous l'Ancienne Alliance, Dieu suscita de nombreux prophètes oints de l'Esprit afin qu'ils annoncent des messages particuliers et conduisent le peuple à l'obéissance et à la crainte de Yahweh. Sous la Nouvelle Alliance, il existe au moins trois types de prophètes. Le premier concerne ceux et celles qui prophétisent au sein des assemblées locales (\vref{Ac. 21:8-9}~; \vref{1 Co. 14:29-32}), ils exhortent, édifient et consolent le peuple (\vref{1 Co. 14:1-3}). Le deuxième concerne les personnes qui ont reçu la charge d'enseigner, poser les fondements, implanter des assemblées selon \vref{Ep. 4:11}. Parmi ces prophètes, on compte Barnabas, Siméon, Lucius de Cyrène, Manahen, Saul (\vref{Ac. 13:1-5}), Jude et Silas (\vref{Ac. 15:32-33}). Le troisième concerne tous les chrétiens qui sont des potentiels prophètes puisqu'ils ont l'Esprit de Christ en eux (\vref{1 Co. 14:23-25}~; \vref{1 Co. 14:31}). Dieu peut se servir n'importe quel chrétien pour prophétiser, c'est-à-dire communiquer une parole inspirée. Voir \vref{Ep. 2:20} et \vref{Ep. 4:11}.

\DicoEntry{PROPHÉTIE}\textit{, du grec «~propheteia~»~: «~discours émanant de l'inspiration divine et déclarant les desseins de Dieu~»}\newline
Depuis l'effusion du Saint-Esprit, tous les chrétiens nés d'en haut peuvent prophétiser sans pour autant avoir le ministère de prophète. La prophétie est en effet un don spirituel* auquel il faut aspirer et qui est attribué par le Saint-Esprit selon la volonté de Dieu. Voir \vref{Ac. 2:16-18}~; \vref{1 Co. 12:4-10} et \vref{1 Co. 14:1}.

\DicoEntry{PROPITIATOIRE}\textit{, de l'hébreu «~kapporeth~»~: «~siège de miséricorde, lieu d'expiation~»}\newline
Couvercle de l'arche* composé d'or pur, il était surmonté de deux chérubins* d'or se faisant face au milieu desquels Yahweh siégeait et se manifestait pour donner des instructions à Israël. Une fois par an, le grand prêtre entrait dans le Saint des saints et aspergeait le propitiatoire du sang des animaux sacrifiés pour la purification des péchés d'Israël. Voir \vref{Ex. 25:17-22} et \vref{Lé. 16}.

\DicoEntry{PROSÉLYTE}\textit{, du grec «~proselutos~»~: «~un nouveau venu, un étranger~»}\newline
Personne issue d'une nation païenne s'étant agrégée au peuple d'Israël par le rite de la circoncision* et la pratique de la loi mosaïque. Voir \vref{Mt. 23:15}~; \vref{Ac. 2:10}~; \vref{Ac. 6:5} et \vref{Ac. 13:43}.

\DicoEntry{PYTHON}\textit{, du grec «~Puthon~»~: «~un serpent ou un dragon}\newline
Esprit de divination auquel Paul fut confronté en Macédoine. Voir \vref{Ac. 16:16-18}.

\DicoEntry{RABBI}\textit{, de l'hébreu «~rab~»~: «~capitaine, chef~» et du grec «~rhabbi~»~: «~maître~», «~un grand monsieur, honorable~» ou «~un enseignant~»}\newline
Les disciples appelaient Jésus «~Rabbi~». Cependant, il a exhorté la foule et les conducteurs religieux à ne pas attribuer une telle marque de distinction aux hommes rappelant que seul Yahweh est maître. Voir \vref{Mt. 23:8}~; \vref{Mc. 11:21}~; \vref{Jn. 9:2}.

\DicoEntry{RACHEL}\textit{, de l'hébreu «~Rachel~»~: «~agnelle, brebis~»}\newline
Fille de Laban, deuxième femme de Jacob pour laquelle il travailla quatorze ans. Longtemps stérile, Yahweh lui donna finalement deux garçons~: Joseph et Benjamin. Elle mourut à l'accouchement du deuxième. Voir \vref{Ge. 29:10-31}~; \vref{Ge. 30:22-24} et \vref{Ge. 35:16-19}.

\DicoEntry{RAHAB}\textit{, de l'hébreu «~Rachab~»~: «~large, spacieux, tumultueux~»}\newline
Prostituée habitant Jéricho, elle cacha les deux espions juifs chez elle. Grâce à son acte, Josué* lui laissa la vie sauve ainsi qu'à sa famille lorsqu'il détruisit la ville et tous ceux qui s'y trouvaient. Rahab habita ensuite au milieu d'Israël~; elle figure non seulement parmi les héros de la foi, mais aussi dans la généalogie de Jésus-Christ. Voir \vref{Jos. 2:1}~; \vref{Jos. 6:17-25}~; \vref{Mt. 1:5-16} et \vref{Hé. 11:31}.

\DicoEntry{REBECCA}\textit{, de l'hébreu «~Ribqah~»~: «~ensorcelante, qui prend au piège~»}\newline
Fille de Bethuel et sœur de Laban, elle fut l'épouse d'Isaac*. Yahweh mit fin à sa stérilité et elle donna naissance à des jumeaux, Esaü et Jacob, qui devinrent deux grandes nations. Voir \vref{Ge. 24} et \vref{Ge. 25:21-26}.

\DicoEntry{RÉCONCILIATION}\textit{, du grec «~katallage~»~: «~échange, change, ajustement d'une différence~»}\newline
Jésus-Christ mourut à la croix pour réconcilier l'homme avec Dieu, c'est-à-dire le faire passer de l'état de séparation (causée par le péché) à l'état d'intimité avec Dieu. L'Eglise a le ministère de réconciliation et doit en ce sens présenter à l'homme pécheur la voie de la réconciliation avec Dieu au travers de la prédication de l'Evangile*. Voir \vref{Ro. 5:11}~; \vref{Hé. 10:18-20} et \vref{2 Co. 5:18-20}.

\DicoEntry{RÉDEMPTION}\textit{, de l'hébreu~: «~peduwth~»~: «~rachat~» et du grec: «~apolutrosis~»~: «~libération effectuée suite au paiement d'une rançon~»}\newline
Jésus-Christ a payé le prix nécessaire au rachat des péchés de tous les hommes par son sacrifice à la croix, leur permettant d'échapper à la mort éternelle au moyen de la foi*. Voir \vref{Ro. 3:23-24}~; \vref{Col. 1:14}~; \vref{Ep. 1:7} et \vref{Hé. 9:12}.

\DicoEntry{RÉFORME}\textit{, de l'hébreu «~yatab~»~: «~agir bien~» et du grec «~diorthosis~»~: «~remettre droit~»}\newline
La plupart des prophètes du Tanakh sont des réformateurs dans la mesure où ils prônent un retour à Dieu~; le roi Josias a institué une profonde réforme pendant son règne en déployant des efforts pour revenir à l'obéissance de la Parole. Jésus-Christ est le plus grand réformateur en ce qu'il marchait à contre-courant et vint restaurer l'homme à sa condition originelle, celle d'avant la chute. Ainsi, l'homme qui reçoit Jésus entre dans un processus où il est continuellement réformé par le Saint-Esprit au travers de la Parole. Voir \vref{2 R. 22}~; \vref{Jé. 7:5}~; \vref{Jé. 26:13}~; \vref{Os. 6:1}~; \vref{Mt. 19:8} et \vref{Jn. 16:7-15}.

\DicoEntry{REPENTANCE}\textit{, du grec «~metanoia~»~: «~changement de mentalité, d'intention~», «~tristesse qu'on éprouve de ses péchés~»}\newline
Un des points majeurs de la prédication de Jean-Baptiste* puis des apôtres*. La repentance est essentielle pour obtenir la rémission des péchés et doit être accompagnée de fruits. La repentance ne concerne pas uniquement le nouveau converti, mais tout disciple de Christ qui, jusqu'à la fin de sa vie, est dans un processus de perfectionnement. Voir \vref{Mc. 1:4}~; \vref{Lu. 3:8}~; \vref{2 Co. 7:9-10}~; \vref{Ro. 2:4}~; \vref{Ac. 2:38}~; \vref{Ac. 13:24}~; \vref{Ac. 17:30} et \vref{Ac. 26:20}.

\DicoEntry{RÉSURRECTION}\textit{, du grec «~anastasis~»~: «~se lever, ressusciter de la mort~».}\newline
Christ fut le premier à expérimenter la résurrection d'entre les morts. Au son de la dernière trompette*, les chrétiens décédés ressusciteront de même avec des corps incorruptibles pour les noces* de l'Agneau. Voir \vref{Mt. 28:6}~; \vref{1 Pi. 1:3}~; \vref{Ap. 1:5}~; \vref{1 Co. 15:52} et \vref{1 Th. 4:16}.

\DicoEntry{RÉVEIL}\textit{, du grec «~egeiro~»~: «~réveiller du sommeil, revenir à la vie, se lever~»}\newline
Prise de conscience personnelle ou collective sur sa condition de péché et.ou l'imminence du jugement de Dieu. Il en résulte la repentance*, la véritable conversion*, la crainte de Dieu, la préparation à la rencontre de Yahweh. Une personne réveillée a les yeux focalisés sur Christ et peut accomplir la volonté du Seigneur. Voir Jon.~; \vref{Ep. 5:14} et \vref{Ro. 13:11-14}.

\DicoEntry{ROBOAM}\textit{, de l'hébreu «~Rhoboam~»~: «~qui affranchit le peuple~»}\newline
Fils et successeur du roi Salomon. C'est sous son règne que se produit le schisme* entre les royaumes du nord et celui du sud. Il régna sur Juda dix-sept années pendant lesquelles il fut en guerre avec le royaume du nord et fit ce qui est mal aux yeux de Yahweh. Voir \vref{1 R. 12:1-24} et \vref{1 R. 14:21-31}.

\DicoEntry{ROMAIN}\textit{, du grec «~rhome~»~: «~force~»}\newline
Pendant la vie de Jésus et pendant l'époque de l'Eglise primitive, Israël était sous la domination de l'Empire romain qui l'oppressait et lui soutirait des impôts. Paul, né à Tarse - ville romaine - put bénéficier des privilèges liés à la nationalité romaine quand il fut livré aux tribunaux. Ce dernier écrivit une lettre aux chrétiens romains - figurant dans le canon biblique - avant de les rencontrer physiquement. Voir \vref{Mt. 22:17}~; \vref{Jn. 11:48}~; \vref{Ac. 16:35-39}~; \vref{Ac. 22:25-29}~; \vref{Ac. 23:27} et \vref{Ac. 25:16}.

\DicoEntry{ROME}\textit{, du grec «~rhome~»~: «~force~»}\newline
Capitale de l'Empire romain située en Italie, Rome jouissait d'une grande notoriété à l'époque de l'Eglise primitive. Bien que l'empereur Claude* ait ordonné aux Juifs de quitter la ville, Paul manifesta le désir de s'y rendre pour y annoncer l'Evangile. Il y arriva après bien des difficultés quelques années plus tard en tant que prisonnier. Voir \vref{Ac. 18:1-2}~; \vref{Ac. 19:21}~; \vref{Ac. 23:11} et \vref{Ac. 28:14-31}.

\DicoEntry{ROYAUME DE DIEU}\textit{, du grec «~basileia~»~: «~pouvoir royal, royauté, domination, autorité~»}\newline
Lors de son service terrestre, Jésus a annoncé que le Royaume de Dieu était proche. Il parlait de son autorité sur toutes choses et de son règne. Ne consistant pas dans les choses terrestres, ce royaume se manifeste par la puissance de Dieu, la justice, la paix et la joie par le Saint-Esprit. Voir \vref{Lu. 9:1-2}~; \vref{Lu. 11:17-20}~; \vref{Lu. 17:20-21} et \vref{Ro. 14:17}.

\DicoEntry{RUBEN}\textit{, de l'hébreu «~Re'uwben~»~: «~voici un fils~»}\newline
Premier fils de Jacob et Léa, il devint le père de la tribu des Rubénites qui s'installa à l'est de la terre promise. Il perdit son droit d'aînesse après avoir eu des rapports intimes avec Bilha, concubine de son père. Voir \vref{Ge. 29:32}~; \vref{Ge. 35:22} et \vref{Ge. 49:3-4}.

\DicoEntry{RUTH}\textit{, de l'hébreu «~Ruwth~»~: «~amitié, une amie~»}\newline
Originaire de Moab, belle-fille de Naomi avec qui elle s'installa à Bethléhem. Elle y épousa Boaz avec qui elle eut un fils, Obed, grand-père du roi David*. Son histoire est racontée dans le livre portant son nom.

\DicoEntry{SABBAT}\textit{, de l'hébreu «~shabbath~»~: «~repos, cessation d'activité~»}\newline
Septième et dernier jour de la semaine consacré à Yahweh pendant lequel aucune activité ne devait être pratiquée selon la loi. Le sabbat figure dans les dix commandements, son infraction devait être punie de mort. Suscitant de vives critiques de la part des religieux, Jésus a plusieurs fois enfreint le sabbat dont il s'est déclaré le maître. Sous la Nouvelle Alliance, le sabbat se trouve en Jésus-Christ, le chrétien n'est donc pas tenu de le respecter comme ce fut le cas sous la loi de Moïse. \vref{Ex. 20:8-11}~; \vref{Ex. 31:14-15}~; \vref{De. 5:12-15}~; \vref{Mt. 11:28-30}~; \vref{Mc. 2:23-28} et \vref{Mc. 3:1-6}.

\DicoEntry{SACERDOTALISME}\textit{}\newline
Doctrine d'origine catholique reconnaissant le prêtre ou le pasteur comme l'intermédiaire entre Dieu et les hommes. Voir \vref{1 Ti. 2:5}.

\DicoEntry{SACRIFICATURE, SACERDOCE}\textit{, de l'hébreu «~kahan~»~: «~service~»}\newline
Sous la loi mosaïque, il était exercé par les Lévites descendants d'Aaron dans le tabernacle puis le temple et consistait notamment à accomplir les différents rituels relatifs aux sacrifices d'animaux et aux offrandes de toutes sortes. Depuis le sacrifice de Jésus à la croix, le sacerdoce concerne tous les enfants de Dieu qui sont non seulement les prêtres, mais aussi les sacrifices auxquels le Seigneur prend plaisir. Voir Lé.~; \vref{Ro. 12:1}~; \vref{1 Pi. 2:9} et \vref{Ap. 1:6}.

\DicoEntry{SADDUCÉENS}\textit{, du grec «~saddoukaios~»~: «~les justes~»}\newline
Parti religieux juif attaché au Pentateuque de manière stricte, ils ne croyaient ni en la résurrection des morts ni aux anges. Ils s'opposèrent au service de Jésus qui les reprit sévèrement et échappa à leurs pièges. Ils combattirent de même les apôtres qu'ils jetèrent en prison. Voir \vref{Mt. 16:6-12}~; \vref{Mt. 22:23-33}~; \vref{Ac. 5:17-19} et \vref{Ac. 23:1-10}.

\DicoEntry{SAINT}\textit{, de l'hébreu «~qodesh~»~: «~consacré, mis à part~» et du grec «~hagios~»~: «~chose très sainte, consacré, un saint~»}\newline
Dieu appela Israël à la sainteté, c'est-à-dire à ne pas se mélanger avec les autres peuples de peur d'être contaminés par leurs pratiques méchantes et idolâtres. Yahweh est le Saint d'Israël. Sous la Nouvelle Alliance, les chrétiens sont appelés saints, car le Saint-Esprit qui est en eux leur communique sa nature, les purifie et leur enseigne la haine du péché. Voir \vref{De. 7:6}~; \vref{Es. 49:7}~; \vref{1 Co. 6:11,19}~; \vref{1 Th. 4:1-8} et \vref{Hé. 12:14}.

\DicoEntry{SAINT-ESPRIT}\textit{, (voir étymologie des mots «~saint~» et «~esprit~»)}\newline
Le Saint-Esprit est l'esprit de Dieu, l'Esprit de Jésus~; il est Dieu. Lors de son service terrestre, le Seigneur déclara qu'un consolateur viendrait habiter dans les corps des croyants. Cette parole s'accomplit lors de la Pentecôte. Le Saint-Esprit a pour mission de convaincre le monde en ce qui concerne le péché, la justice et le jugement. A la naissance d'en haut, il régénère l'esprit du chrétien sur qui il dépose son sceau, gage de l'adoption. Il enseigne et guide le chrétien tout au long de sa marche avec Dieu. Il transforme son caractère et distribue les dons spirituels pour l'édification de l'Eglise. Voir \vref{1 S. 10:10}~; \vref{2 Ch. 15:1}~; \vref{Jn. 14:16-17,26}~; \vref{Jn. 16:7-15}~; \vref{Ac. 2}~; \vref{1 Co. 6:11}~; \vref{Ro. 8:9}~; \vref{1 Co. 3:16}~; \vref{1 Co. 12:4-13}~; \vref{Ep. 1:13} et \vref{Ga. 5:16,22}.

\DicoEntry{SALOMON}\textit{, de l'hébreu «~Shelomoh~»~: «~paix, pacifique~»}\newline
Fils de David, il succéda à son père et fut roi d'Israël pendant quarante ans. Il construisit le premier temple de Yahweh sur le mont Morija à Jérusalem puis un palais royal. Outre ses importantes richesses, c'est la grande sagesse que Dieu lui donna qui fit sa renommée parmi tous les peuples. Il eut sept cents femmes et trois cent concubines - dont un grand nombre de femmes étrangères - ce qui détourna son cœur de son Dieu. On lui attribue la rédaction des livres Cantique des cantiques et Ecclésiaste~; il a aussi écrit certains psaumes et plusieurs proverbes. Voir \vref{1 R. 4:29-34}~; \vref{1 R. 5-7}~; \vref{1 R. 9:15-28}~; \vref{1 R. 11:1-10,42}~; \vref{Ps. 72,127} et \vref{Pr. 25-29}.

\DicoEntry{SALUT}\textit{, de l'hébreu «~yesha'~» et du grec «~soteria~»~: «~délivrance, sûreté, sécurité~»}\newline
Libération des chaînes du péché, de la condamnation et de tout type d'asservissement spirituel, le salut est un don gratuit de Dieu qui s'obtient par la grâce, au moyen de la foi. C'est la manifestation de l'amour éternel de Dieu qui - ne voulant pas que l'homme périsse dans le feu de la géhenne - a payé le prix pour lui offrir la vie éternelle. Le salut réside dans le seul nom de Jésus-Christ. Voir \vref{Jn. 3:16}~; \vref{Ac. 4:12}~; \vref{Ro. 8:1}~; \vref{1 Th. 5:9}~; \vref{Tit. 3:4-6} et \vref{Ep. 2:4-8}.

\DicoEntry{SAMARIE}\textit{, de l'hébreu «~Shomerown~» et du grec «~Samareia~»~: «~montagne de guet~»}\newline
Située dans l'actuelle Cisjordanie, ville fondée par Omri, roi d'Israël, et qui devint la capitale du royaume du nord. La ville fut prise par Salmanasar, roi d'Assyrie, sous le règne d'Osée, roi d'Israël. Au temps de Jésus, la Samarie n'était qu'une simple circonscription romaine dont la population était issue du métissage entre Israélites et des colons assyriens. Suite aux persécutions subies par l'Eglise primitive à Jérusalem, des chrétiens s'y réfugièrent et l'Evangile s'y propagea. Voir \vref{1 R. 16:23-24}~; \vref{2 R. 3:1}~; \vref{2 R. 18:9}~; \vref{Os. 7} et \vref{Ac. 8:1-17}.

\DicoEntry{SAMARITAINS}\textit{, du grec «~samareites~»~: «~un habitant de Samarie~»}\newline
Après l'assujettissement de la Samarie par Salmanasar, roi d'Assyrie, des peuples étrangers s'y établirent et s'assemblèrent avec les Israélites. Au IVe siècle av J.-C., les samaritains construisirent un temple sur le mont Garizim, qui devint le centre religieux de Samarie, entraînant une séparation avec le reste des Juifs qui adoraient à Jérusalem. Les samaritains étaient considérés comme des étrangers et non comme de véritables juifs du fait de la mixité de leur religion. Jésus-Christ ouvrit la voie de la réconciliation avec ce peuple en racontant la parabole du bon samaritain et en annonçant la bonne nouvelle à la femme samaritaine. Voir \vref{2 R. 17:3,24-29} et \vref{2 R. 18:9}~; \vref{Jn. 4:4-26} et \vref{Lu. 10:30-37}.

\DicoEntry{SAMSON}\textit{, de l'hébreu «~Shimshown~»~: «~petit soleil~»}\newline
Fils de Manoach, de la tribu de Dan, il fut juge en Israël pendant vingt-ans. Consacré à Dieu dès le sein maternel et doté d'une force extraordinaire, il réalisa des prouesses qui suscitèrent la crainte de ses ennemis. Choisi pour être le libérateur d'Israël, il fut incompris par les siens qui ne le soutinrent pas. Il mourut suite à la trahison de Delila, une femme d'origine philistine. Voir \vref{Jg. 13-16}.

\DicoEntry{SAMUEL}\textit{, de l'hébreu «~Shemuw'el~»~: «~entendu ou exaucé de Dieu~»}\newline
Fils d'Elkana, de la tribu d'Ephraïm, et d'Anne, il fut consacré au service de Yahweh dès son plus jeune âge. Il exerça les fonctions de juge, prêtre et prophète sur Israël. Il oignit les deux premiers rois d'Israël~: Saül et David. Son histoire est racontée dans les deux livres du Tanakh portant son nom.

\DicoEntry{SANCTIFICATION}\textit{, de l'hébreu «~Qadash~» et du grec «~hagiasmos~»~: «~consécration, purification, sainteté~» ou «~l'effet de la purification~»}\newline
Fruit de l'action conjointe de la Parole et l'Esprit de Dieu dans la vie du croyant, la sanctification doit être recherchée par le chrétien tout au long de sa vie. Sans elle, nul ne verra Dieu. \vref{Jn. 17:17}~; \vref{1 Th. 4:3-8}~; \vref{Hé. 12:14} et \vref{Ap. 22:11}.

\DicoEntry{SANCTUAIRE}\textit{, de l'hébreu «~miqdash~»~: «~lieu sacré, lieu saint, sanctuaire de Yahweh~»}\newline
Le sanctuaire terrestre, dont Moïse avait reçu le modèle, était une représentation de celui qui se trouve au ciel et où Jésus alla présenter son sang. Voir \vref{Ex. 25:8-9} et \vref{Hé. 9:1-24}.

\DicoEntry{SANG}\textit{, de l'hébreu «~dam~» et du grec «~haima~»~: «~sang~»}\newline
Déterminant le lien de famille et la lignée, le sang est, selon les Ecritures, l'âme*, la vie. Ainsi, l'effusion de sang fut nécessaire pour le pardon des péchés et le sang de Christ, qui ôte définitivement le péché, donne la vie. Voir \vref{Lé. 17:11}~; \vref{Ac. 17:26}~; \vref{Ro. 5:9}~; \vref{Hé. 9:22-28} et \vref{Ap. 5:9}.

\DicoEntry{SANHÉDRIN}\textit{, du grec «~sunedrion~»~: «~conseil, tribunal~»}\newline
Désignait d'une part, les petits tribunaux se tenant dans chaque ville pour régler les affaires locales et d'autre part, le grand conseil de Jérusalem où étaient traitées les affaires plus importantes. Ce dernier était composé de soixante et onze membres sélectionnés parmi l'élite religieuse et les anciens d'Israël~; le grand prêtre en était le président. Sous la domination romaine, ce tribunal fonctionnait de manière quasi autonome~; la sentence de la peine de mort devait néanmoins être validée par le gouverneur romain. Jésus fut jugé coupable de blasphème par le sanhédrin qui le condamna à mort. Voir \vref{No. 11:16-17}~; \vref{Mt. 5:22}~; \vref{Mt. 26:59-66}~; \vref{Jn. 11:47}~; \vref{Ac. 5:21-41} et \vref{Ac. 6:12-15}.

\DicoEntry{SARA}\textit{, de l'hébreu «~Sarah~»~: «~princesse, femme noble~»}\newline
Femme d'Abraham, elle enfanta Isaac à l'âge de 90 ans selon la promesse de Yahweh. Sara figure parmi les héros de la foi~; elle mourut à cent vingt-sept ans. Voir \vref{Ge. 12:5}~; 17~; \vref{Ge. 15-16}~; \vref{Ge. 21:1-7}~; \vref{Ge. 23:1} et \vref{Hé. 11:11}.

\DicoEntry{SARDES}\textit{, du grec «~sardeis~»~: «~les rouges~», «~prince de joie~»}\newline
Capitale antique de la Lydie, Sardes se situait sur la rivière Pactole, à environ 50 km au sud de Thyatire et 75 km à l'est de Smyrne. Réputée riche et puissante en raison de ses ressources en or, ses épithètes étaient sournois car sa forteresse reposait sur un sol boueux. En effet, au VIème siècle av. J.-C., Cyrus Le Grand - vainqueur de Crésus alors roi de Lydie - s'empara de Sardes par une attaque nocturne. Par la suite, la ville subit plusieurs invasions puis un tremblement de terre en 17 ap. J.-C. L'église de Sardes fut probablement fondée par Paul au cours d'un voyage à Ephèse. Au moment où ils reçurent le message de l'ange de l'Apocalypse, il semblerait que certains chrétiens de Sardes étaient retournés au culte licencieux de Cybèle, déesse-mère et gardienne des savoirs. Ceux qui s'étaient gardés purs devaient ainsi revivifier les autres membres. Cette église symbolise l'église morte. Voir \vref{Ap. 3:1-6}.

\DicoEntry{SATAN}\textit{, de l'hébreu «~Satan~»~: «~adversaire, ennemi~»}\newline
Autrefois chérubin protecteur, il a péché en voulant s'approprier la gloire qui ne revient qu'à Dieu. Dans sa rébellion, il entraîna un tiers des anges qui furent précipités avec lui sur la terre. Connu également sous les noms «~Prince de ce monde~», «~Prince des ténèbres~», «~Belzébul~», «~le malin~», «~l'accusateur~» ou «~le diable~», il est l'adversaire des enfants de Dieu à qui il fait la guerre. Il a cependant été vaincu à la croix par Jésus-Christ, au nom duquel les chrétiens peuvent le chasser. Satan sera enchaîné pendant le millenium puis libéré pour un peu de temps. Il sera finalement jeté dans l'étang de feu pour l'éternité. Voir \vref{Ez. 28:14-19}~; \vref{Es. 14:12-17}~; \vref{Ap. 12:4}~; \vref{Lu. 10:18-19}~; \vref{Ja. 4:7}~; \vref{Jn. 16:11}~; \vref{Ap. 12:4} et \vref{Ap. 20:1-15}.

\DicoEntry{SAÜL}\textit{, de l'hébreu «~Sha'uwl~»~: «~désiré, demandé (à Dieu)~»}\newline
1. Fils de Kis, israélite de la tribu de Benjamin, il fut choisi par Dieu pour être le premier roi d'Israël sur qui il régna pendant quarante ans. Il désobéit à la loi de Yahweh et tenta plusieurs fois d'assassiner David, choisi par Dieu pour lui succéder sur le trône. Saül mourut avec ses trois fils pendant la bataille de Guilboa. Voir \vref{1 S. 10}~; \vref{1 S. 13:1-14}~; \vref{1 S. 15:10-11}~; \vref{1 S. 18:8-16}~; \vref{1 S. 19:8-17} et \vref{1 S. 31}.
\\2. Nom initial de Paul*.

\DicoEntry{SCANDALE}\textit{, de l'hébreu «~mikshowl~»~: «~trébucher~» et du grec «~skandalon~»~: «~obstacle, piège~»}\newline
Pierre qu'on rencontre et qui peut faire glisser sur le chemin ou encore situation ou comportement qui provoque un trouble emmenant quelqu'un à fauter. Le scandale n'est pas forcément une mauvaise action en soi~; Christ lui-même fut un scandale pour les Juifs. Toutefois, il reste souvent lié aux œuvres de la chair et peut être provoqué par un manque de discernement. Le chrétien doit veiller par rapport aux scandales. Voir \vref{Ps. 106:36}~; \vref{Mt. 13:41}~; \vref{Mt. 18:7}~; \vref{1 Co. 1:23} et \vref{1 Pi. 2:7-8}.

\DicoEntry{SCEAU}\textit{, du grec «~sphragizo~»~: «~mettre un sceau dessus, poser une marque par l'impression d'un sceau~»}\newline
Sous l'Ancienne Alliance, la circoncision était une marque de l'alliance établie entre Yahweh et son peuple. A la naissance d'en haut, le chrétien est scellé du Saint-Esprit, témoignant de son appartenance à Christ. Voir \vref{Ge. 17:10-11}~; \vref{Ep. 1:13}~; \vref{Ap. 7:3} et \vref{Ap. 9:4}.

\DicoEntry{SCHEOL}\textit{}\newline
Voir SÉJOUR DES MORTS.

\DicoEntry{SCHISME D'ISRAËL}\textit{}\newline
Le schisme est la séparation d'Israël en deux royaumes suite à la dérive de Salomon. En 931 av. J.-C., Roboam succéda à son père Salomon sur le trône royal et n'accepta pas d'alléger le joug que son père avait mis sur eux, cela entraîna la séparation du royaume en deux. On retrouva d'une part, le royaume d'Israël dirigé par Jéroboam - appelé aussi royaume du nord -, composé des dix tribus du nord et d'autre part, le royaume de Juda gouverné par le roi Roboam composé des deux tribus du sud (Benjamin et Juda). Voir \vref{1 R. 12:1-24}.

\DicoEntry{SCRIBE}\textit{, de l'hébreu «~caphar~»~: «~secrétaire, scribe~», «~homme instruit, qui a le savoir~»}\newline
Les scribes occupaient une position importante auprès du peuple juif, ayant non seulement une mission d'enseignement de la loi, mais également une fonction au sein de la justice juive en prenant part au sanhédrin*. Voir \vref{Esd. 7:6-10} et \vref{Mt. 16:21}.

\DicoEntry{SECTE}\textit{, du grec «~hairesis~»~: «~action de prendre, capturer~»}\newline
Groupement de personnes adhérant à une doctrine particulière et vivant marginalement, comme les sadducéens ou les pharisiens. Les premiers disciples furent qualifiés de «~secte des Nazaréens~». Pierre met en garde contre les faux prophètes qui introduisent des sectes pernicieuses pour ravir la foi des chrétiens afin de les entraîner dans la dissolution. Voir \vref{Ac. 5:17}~; \vref{Ac. 26:5}~; \vref{Ac. 24:5} et \vref{2 Pi. 2:1}.

\DicoEntry{SÉDÉCIAS}\textit{, de l'hébreu «~Tsidqiyah~»~: «~Yahweh est justice~»}\newline
Fils d'Hamoutal et oncle de Jojakin, il fut le dernier roi de Juda sur qui il régna onze ans. Son nom initial, Matthania, fut changé en Sédécias par Nebucadnetsar, roi de Babylone. Il fit ce qui est mal aux yeux de Yahweh et connut un destin tragique~: ses fils furent égorgés devant lui, Nebucadnestar lui creva ensuite les yeux, Jérusalem et le temple* furent détruits et il fut emmené captif avec le peuple à Babylone. Voir \vref{2 R. 24:17-19}~; \vref{2 R. 25:1-21}~; \vref{Jé. 21}~; \vref{Jé. 22:1-9}~; \vref{Jé. 37,38,39:6-7}.

\DicoEntry{SÉJOUR DES MORTS}\textit{de l'hébreu «~she'owl~»~: «~monde souterrain, tombe, enfer, fosse~» et du grec «~hades~»~: «~dieu des profondeurs de la terre~»}\newline
Lieu de captivité où allaient les âmes de tous les défunts avant le sacrifice de Christ. Il était scindé en deux parties séparées par un grand abîme. D'un côté se trouvait un lieu de tourments et de souffrances extrêmes accueillant tous les méchants qui ont vécu dans le péché durant leur vie terrestre et qui n'y ont pas renoncé. D'un autre côté, il y avait le sein d'Abraham où reposaient et séjournaient les âmes des justes qui avaient foi en Yahweh. Après la résurrection de Jésus, ces derniers ont été arrachés du séjour des morts par le Seigneur qui les a emmenés au paradis*. Le ciel, en tant que destination des personnes décédées, fut en effet ouvert par Christ après sa résurrection. Par conséquent, le sein d'Abraham n'a jamais accueilli de chrétiens. Le séjour des morts est à présent composé uniquement d'impies~; à la fin du monde, il sera jeté avec tous ses habitants dans l'étang de feu*. Voir \vref{Lu. 16:19-31}~; \vref{1 S. 28:6-20}~; \vref{Mt. 11:23}~; \vref{Ac. 2:27}~; \vref{Jn. 3:13}~; \vref{Ep. 4:8} et \vref{Ap. 20:14}.
\\Note~: L'histoire de Lazare et de l'homme riche racontée dans Luc \vref{16:19-31} n'est pas une parabole. A la différence de tous les récits à caractère parabolique contés dans les Ecritures, cette histoire mentionne un nom.

\DicoEntry{SEM}\textit{, de l'hébreu «~Shem~»~: «~nom, renommée~»}\newline
Fils aîné de Noé et ancêtre d'Abraham. Voir \vref{Ge. 10:1} et \vref{Ge. 11:10-27}.

\DicoEntry{SÉNEVÉ}\textit{, du grec «~sinapi~»~: «~graine de moutarde~»}\newline
Plante des régions orientales ayant la forme d'une petite semence pouvant grandir de manière exponentielle et atteignant jusqu'à trois mètres. Elle symbolise spirituellement la puissance de la foi* capable de déplacer les montagnes. Voir \vref{Mt. 13:32-33} et \vref{Mt. 17:20}.

\DicoEntry{SEPHORA}\textit{, de l'hébreu «~Tsipporah~»~: «~petit oiseau, moineau~»}\newline
Fille de Jéthro, femme de Moïse et mère d'Eliézer et de Guerschom. Elle partit dans le pays d'Egypte avec Moïse quand il répondit à l'appel de Yahweh pour aller libérer Israël. \vref{Ex. 2:15-21} et \vref{Ex. 4:18-20}.

\DicoEntry{SÉRAPHINS}\textit{, de l'hébreu «~saraph~»~: «~être majestueux avec six ailes au service de Dieu~»}\newline
Catégorie d'anges* proclamant la sainteté de Dieu. Voir \vref{Es. 6:1-7}.

\DicoEntry{SERPENT}\textit{, de l'hébreu «~nachash~»~: «~serpent, reptile~»}\newline
C'est sous la forme du serpent que Satan vint séduire Eve dans le jardin d'Eden. Le serpent fut maudit d'entre tous les animaux pour son action. Le serpent ancien ou rusé désigne le diable*~; il s'oppose au serpent d'airain, Jésus, qui a donné à ses enfants le pouvoir de marcher sur les serpents. Voir \vref{Ge. 3:1-14}~; \vref{No. 21:4-9}~; \vref{2 Co. 11:3}~; \vref{Jn. 3:14-15}~; \vref{Lu. 10:19} et \vref{Ap. 12:9,14-15}.

\DicoEntry{SERVICE}\textit{, du grec «~diakonia~»~: «~service~», dérivé du mot grec «~diakonos~»~: «~domestique~»}\newline
Tâche que le chrétien exerce au service de Dieu et des hommes selon l'onction* et le mandat que Dieu lui donne. Le serviteur de Dieu est donc un serviteur inutile, un simple instrument utilisé pour la gloire de Yahweh. Voir \vref{Lu. 17:10}~; \vref{1 Co. 12}~; \vref{2 Co. 3:5}~; \vref{Ro. 12} et \vref{1 Pi. 4:10-11}.

\DicoEntry{SERVITEUR}\textit{, du grec «~diakonos~»~: «~domestique, subordonné, messager~» ou «~doulos~»~: «~esclave~»}\newline
Christ a renoncé à sa gloire et a pris la forme d'un simple serviteur. De même, le chrétien n'est pas uniquement serviteur de Dieu, il doit comme le maître servir son prochain. Voir \vref{Mc. 10:45}~; \vref{Ph. 1:1}~; \vref{Ph. 2:5-8} et \vref{2 Co. 6:4}.

\DicoEntry{SETH}\textit{, de l'hébreu «~Sheth~»~: «~compensation, mis à la place~»}\newline
Troisième fils d'Adam et Eve~; il naquit après le meurtre de son frère Abel que Caïn avait tué. Seth fut l'ancêtre de Noé et de Jésus-Christ. Voir \vref{Ge. 4:25}~; \vref{Ge. 5:6-29} et \vref{Lu. 3:38}.

\DicoEntry{SHOFAR}\textit{, de l'hébreu «~showphar~»~: «~corne, corne de bélier~».}\newline
Instrument de musique à vent fait à partir de la corne de bélier. Voir TROMPETTE.

\DicoEntry{SIDON}\textit{, de l'hébreu «~Tsiydown~»~: «~abondance de poisson, pêche~»}\newline
Ville de l'antique Phénicie (actuel Liban) située non loin de Tyr~; on y vénérait les Baals et les Astartés. La reine Jézabel était originaire de Sidon. Voir \vref{Jg. 10:6} et \vref{1 R. 16:31}.

\DicoEntry{SILAS}\textit{, du grec «~Silas~»~: «~de la forêt, demandé~»}\newline
Prophète, compagnon d'œuvre de Paul avec qui il effectua plusieurs voyages missionnaires. Voir \vref{Ac. 15-18}.

\DicoEntry{SILO}\textit{, de l'hébreu «~Shiyloh~»~: «~lieu de repos~»}\newline
Ville située au nord-est de la tribu d'Ephraïm où les enfants d'Israël se répartirent les territoires avant la conquête de Canaan. Avant d'être placée à Jérusalem, l'arche de l'alliance se trouvait à Silo. Voir \vref{Jos. 18:10}~; \vref{Jos. 19:51}~; \vref{1 S. 3:19-21} et \vref{1 S. 4:3}.

\DicoEntry{SILOÉ}\textit{, de l'hébreu «~Shiloach~»~: «~envoyé~»}\newline
Source d'eau se trouvant au sud-est de Jérusalem. Voir \vref{Né. 3:15} et \vref{Jn. 9:6-7}.

\DicoEntry{SIMÉON}\textit{, de l'hébreu «~Shim`own~»~: «~qui écoute, qui a été entendu~»}\newline
1. Fils de Jacob et Léa. Avec Lévi, son frère, il vengea le déshonneur de sa sœur Dina, en tuant Sichem, prince de Canaan, son père Hamor, et tous leurs hommes. Il fut gardé comme otage en Egypte, lorsque Joseph voulut éprouver la sincérité de ses frères. Il devint le père de la tribu des Siméonites qui s'installèrent au sud de Canaan. Voir \vref{Ge. 29:33}~; \vref{Ge. 34}~; \vref{Ge. 42:21-38} et \vref{Jos. 19:1-9}.
\\2. Homme de foi à qui le Saint-Esprit avait promis qu'il ne mourrait pas sans avoir vu le Messie. Il rencontra Jésus lorsqu'il était enfant, à Jérusalem. Voir \vref{Lu. 2:25-35}.

\DicoEntry{SIMON}\textit{, de l'hébreu «~Shiymown~»~: «~désert~» ou «~qui entend~»}\newline
1. Simon Pierre, le nom originel de Pierre* était Simon. Voir \vref{Jn. 1:40-42}.
\\2. Simon le zélote, il faisait partie du groupuscule des zélotes avant de devenir apôtre de Christ. Voir \vref{Lu. 6:13-16}.
\\3. Simon de Cyrène, il fut contraint d'aider Jésus à porter la croix jusqu'à Golgotha. Voir \vref{Mt. 27:32}.
\\4. Simon le magicien, originaire de la ville de Samarie, il fut baptisé par Philippe et crut pouvoir acheter à prix d'argent la puissance du Saint-Esprit. Voir \vref{Ac. 8:9-24}.

\DicoEntry{SION}\textit{, de l'hébreu «~Tsiyown~»~: «~lieu desséché~»}\newline
Autre nom pour parler de Jérusalem. Sous la Nouvelle Alliance, la montagne de Sion est l'image de la Jérusalem céleste. Voir \vref{De. 4:48}~; \vref{1 R. 8:1}~; \vref{Es. 2:3}~; \vref{2 S. 5:6-7} et \vref{Hé. 12:22}.

\DicoEntry{SISERA}\textit{, de l'hébreu «~Ciycera'~»~: «~déploiement, champ de bataille~»}\newline
Chef de l'armée du roi cananéen Jabin, son armée fut vaincue par Barak et Sisera fut tué par Jaël, femme de Héber, le Kénien. Voir \vref{Jg. 4}.

\DicoEntry{SMYRNE}\textit{, du grec «~Smurna~»~: «~myrrhe~»}\newline
Cité de la côte occidentale de l'Asie Mineure, Smyrne (aujourd'hui Izmir) était située au nord d'Ephèse et réputée pour sa splendeur et ses richesses. Ses forteresses et ses tours de l'acropole évoquaient une couronne. Très unie à Rome, des cultes en l'honneur du dieu Zeus, de la déesse Cybèle, ou encore de l'empereur Tibère et sa mère Julie y étaient célébrés. Proche d'Ephèse, l'église de Smyrne fut probablement le fruit du travail apostolique de Paul. En proie à ces doctrines impies, l'église de Smyrne était fortement persécutée aussi bien par les Romains que par «~les faux Juifs~» membres «~d'une synagogue de Satan~». Sa persévérance face aux afflictions lui permit de recevoir un bon témoignage du Seigneur. Elle incarne l'église persécutée. Voir \vref{Ap. 2:8-11}.

\DicoEntry{SODOME}\textit{, de l'hébreu «~Cedom~»~: «~qui brûle~»}\newline
Ville cananéenne située dans la plaine du Jourdain à proximité de laquelle Lot s'installa après s'être séparé d'Abraham. Ses habitants étaient de grands pêcheurs devant Yahweh à un tel point qu'il détruisit la ville - avec Gomorrhe* - en faisant tomber du ciel une pluie de feu et de soufre. Lot et ses deux filles furent épargnés grâce à l'intercession* d'Abraham. Voir \vref{Ge. 13:10-13} et \vref{Ge. 19:1-29}.

\DicoEntry{SOPHONIE}\textit{, de l'hébreu «~Tsephanyah~»~: «~Yahweh a caché, protégé~»}\newline
Fils de Cuschi, descendant du roi Ezéchias, prophète de Yahweh ayant vécu au temps du roi Josias. L'ensemble de ses prophéties se trouve dans le livre portant son nom.

\DicoEntry{STOÏCIENS}\textit{, du grec «~stoikos~»~: «~appartenant au portique~»}\newline
Adeptes de la doctrine de Zénon de Kition (336 av. J.-C. – 264 av. J.-C.) qui fonda le stoïcisme à Chypre en 301 av. J.-C. Le stoïcisme était l'une des principales doctrines philosophiques de la Grèce antique avec l'épicurisme. Elle reposait sur la morale et la maîtrise de ses sentiments par une vie en conformité avec la nature. A Athènes, quelques stoïciens, accompagnés d'épicuriens, se confrontèrent à Paul, le menant à l'aréopage afin de l'interroger. Voir \vref{Ac. 17:18-20}.

\DicoEntry{SYNAGOGUE}\textit{, du grec «~sunagogue~»~: «~assemblée, lieu de réunion~»}\newline
Assemblée de Juifs réunis pour prier et écouter la lecture des Ecritures. Jésus y enseigna régulièrement pendant son service. Les apôtres annoncèrent également l'Evangile dans des synagogues. Voir \vref{Mt. 4:23}~; \vref{Mt. 9:35}~; \vref{Mc. 6:2} et \vref{Ac. 14:1}.

\DicoEntry{TABERNACLE}\textit{, de l'hébreu «~mishkan~»~: «~sanctuaire, demeure, lieu d'habitation~»}\newline
Appelée aussi tente d'assignation, habitation mobile de Yahweh construite selon le modèle que Dieu donna à Moïse dans le désert. Les Lévites en assuraient le service avec tous les ustensiles qui lui étaient dédiés. Une nuée s'élevait au-dessus du tabernacle pour signifier aux Israélites qu'ils devaient lever le camp* et poursuivre leur marche. Voir \vref{Ex. 25:8-9}~; \vref{Ex. 39:32}~; \vref{No. 1:50-51}~; \vref{Ex. 40:36-38} et \vref{1 Ch. 6:48}.

\DicoEntry{TANAKH}\textit{}\newline
Voir Introduction.

\DicoEntry{TEMPLE}\textit{, de l'hébreu «~heykal~»~: «~palais, temple, sanctuaire~» (voir illustration)}\newline
David projeta de construire un temple pour Yahweh~; son fils Salomon fut mandaté pour l'ériger en remplacement du tabernacle. Il fut détruit une première fois par les Babyloniens au VIème siècle av. J.-C. Reconstruit lors du retour d'exil des Juifs, il fut de nouveau détruit en 70 par les Romains~; il n'en reste qu'un mur aujourd'hui appelé «~mur des lamentations~». Sous la Nouvelle Alliance, Yahweh a choisi pour temple l'Eglise*, le corps de chaque chrétien en qui il vient résider à la naissance d'en haut. Voir \vref{2 S. 7}~; \vref{1 R. 6}~; \vref{2 R. 25:8-9}~; \vref{Esd. 6:15}~; \vref{Ep. 2:21-22} et \vref{1 Co. 6:19}.

\DicoEntry{TÉNÈBRES}\textit{, de l'hébreu «~chosnek~»~: «~obscurité, ténèbres, nuit, lieu caché~»}\newline
Dès la Genèse, la lumière est séparée des ténèbres qui peuvent symboliser le péché, l'ignorance et l'absence de la vie de Dieu. Véritable prison rendant les hommes captifs, les ténèbres éternelles du séjour des morts* seront pour les anges déchus, le diable et tous les méchants. Voir \vref{Ge. 1:2-5}~; \vref{2 S. 22:29}~; \vref{Ps. 107:10}~; \vref{Job 17:13}~; \vref{Ro. 13:12}~; \vref{1 Th. 5:5}~; \vref{2 Pi. 2:4}~; \vref{1 Jn. 2:11} et \vref{Jud. 1:6-13}.

\DicoEntry{TÉRÉBINTHE}\textit{, de l'hébreu «~'elah~»~: «~térébinthe ou chêne~»}\newline
Grand arbre robuste dont l'ombrage est agréable, il est répandu en Israël. Jacob enterra les dieux étrangers de sa maison sous un térébinthe. L'ange de Yahweh apparut sous un térébinthe à Gédéon. Des cultes idolâtres étaient célébrés à l'ombre de ces arbres. C'est à la vallée des térébinthes, située au sud-ouest de Jérusalem, que David tua Goliath. Voir \vref{Ge. 35:4}~; \vref{Jg. 6:11,19}~; \vref{2 S. 18:9}~; \vref{1 Ch. 10:12}~; \vref{Os. 4:13}~; \vref{Es. 57:5} et \vref{1 S. 17:1-50}.

\DicoEntry{TÉTRARQUE}\textit{, du grec «~tetrarches~»~: «~tétrarque~»}\newline
Titre donné au gouverneur d'un territoire sous domination romaine. Hérode Antipas* était le tétrarque de Galilée. Voir \vref{Mt. 14:1} et \vref{Lu. 3:1}.

\DicoEntry{THADÉE}\textit{}\newline
Voir JUDE.

\DicoEntry{THÉRAPHIM}\textit{, de l'hébreu «~teraphiym~»~: «~idolâtries, idoles~»}\newline
Amulette utilisée dans les cultes idolâtres. Rachel déroba les théraphim de son père Laban avant de quitter sa maison. Voir \vref{Ge. 31:19,34-35}.

\DicoEntry{THESSALONIQUE}\textit{, du grec «~Thessalonike~»~: «~victoire de ce qui est faux~»}\newline
Ville située au Nord de la Grèce actuelle, sur la côte de la mer Egée, elle jouissait d'une importante fréquentation puisqu'elle figurait parmi les trois ports principaux de la Méditerranée et se situait sur l'une des plus grandes routes commerciales de l'époque~: la Voie Egnatienne reliant Rome à Byzance. Sur le plan religieux, les habitants étaient polythéistes et pratiquaient une variété de cultes, dont le culte impérial. Durant trois semaines, Paul enseigna dans une synagogue* à Thessalonique~; de là, il réussit à constituer un groupe de croyants. Toutefois, une violente persécution l'obligea à quitter promptement la ville, laissant la communauté nouvellement formée vulnérable et fragile. Il écrivit deux lettres aux saints de Thessalonique qui figurent dans le canon biblique.

\DicoEntry{THOMAS}\textit{, de l'hébreu «~Ta'own~»~: «~jumeau~»}\newline
Surnommé Didyme, il était l'un des douze apôtres*. Dans un premier temps incrédule quant à la résurrection de Jésus, il confessa la Seigneurie de ce dernier lorsqu'il le vit ressuscité. Voir \vref{Lu. 6:12}~; \vref{Jn. 11:16} et \vref{Jn. 20:24-29}.

\DicoEntry{TIMOTHÉE}\textit{, du grec «~Timotheos~»~: «~qui adore, ou honore Dieu~»}\newline
Fils d'une femme juive croyante et d'un père grec. Lié à Paul comme un fils à son père, il devint l'un de ses plus fidèles collaborateurs et l'accompagna à plusieurs reprises dans ses voyages missionnaires. Malgré sa jeunesse, il lui fut confié des tâches liées à la direction des églises, notamment à Ephèse. Timothée reçut de Paul deux lettres regorgeant de conseils et d'instructions pour être un bon serviteur de l'Evangile*. Voir \vref{Ac. 16:1-3}~; \vref{Ac. 18:5}~; \vref{1 Co. 16:10} et les deux épîtres de Paul à Timothée.

\DicoEntry{TITE}\textit{, du grec «~Titos~»~: «~nourrice, honorable~»}\newline
D'origine grecque, Tite fut un fidèle compagnon d'œuvre de l'apôtre Paul. Il l'accompagna à Jérusalem, œuvra à Corinthe et en Dalmatie et s'occupa plus particulièrement de l'église de Crète. Il reçut une lettre de Paul qui figure dans le canon biblique. Voir \vref{2 Co. 8:6,23}~; \vref{Ga. 2:1}~; \vref{2 Ti. 4:10} et l'épître de Paul à Tite.

\DicoEntry{TRIBULATION}\textit{, du grec «~thlipsis~»~: «~une pression, une oppression~»}\newline
Persécution, tourment provoqué par l'annonce de l'Evangile. Inévitables pour entrer dans le royaume de Dieu*, les tribulations ont pour but de rendre le chrétien patient, joyeux et persévérant en toutes circonstances. Voir \vref{Mc. 4:17}~; \vref{Jn. 16:33}~; \vref{Ac. 14:22}~; \vref{2 Co. 6:4}~; \vref{2 Co. 8:2}~; \vref{Ph. 1:29}~; \vref{1 Th. 3:3} et \vref{2 Th. 1:4}.

\DicoEntry{TRIBUNAL}\textit{, de l'hébreu «~qahal~»~: «~assembler, convoquer~» et du grec «~bema~»~: «~tribune~»}\newline
Lieu où les hommes sont jugés afin de recevoir une sentence en fonction des actes qu'ils ont posés. Chaque être humain comparaîtra devant le tribunal de Christ afin de rendre compte pour lui-même. Voir \vref{Ro. 14:10-12} et \vref{2 Co. 5:10}.

\DicoEntry{TRINITÉ}\textit{}\newline
Doctrine selon laquelle le Dieu unique se manifesterait en trois personnes distinctes~: Père, Fils et Saint-Esprit. Inspirée des triades païennes (babylonienne, égyptienne…), cette fausse doctrine d'origine catholique apparut au IIème siècle et fut fixée aux Conciles de Nicée en 325 et de Constantinople I en 381. Elle fut largement reprise par les protestants et la plupart des mouvements chrétiens alors que ni le mot ni le concept de trinité n'apparaissent dans les Ecritures. Voir \vref{De. 6:4}~; \vref{Es. 9:5}~; \vref{Jn. 4:23-24}~; \vref{Col. 2:8-10}~; \vref{2 Th. 1:12} et \vref{1 Jn. 5:20}.

\DicoEntry{TROMPETTE}\textit{, plusieurs mots hébreux ont été traduits par trompette, les plus utilisés sont~: «~chatsotserah~»~: «~trompette, clairon~»~; «~yobel~»~: «~bélier, corne de bélier~», «~retentissant~», «~jubilé~», et «~showphar~»~: «~corne de bélier~». Plusieurs mots grecs ont aussi été utilisés, notamment «~salpigx~»~: «~une trompette~» et «~salpizo~»~: «~sonner de la trompette~».}\newline
Sous l'Ancienne Alliance, on l'utilisait pour donner un signal, publier une sainte convocation, fêter des moments de joie, signifier une victoire, chanter des cantiques en l'honneur de Yahweh, avertir et rassembler le peuple. Le son de la trompette est aussi l'image des voix prophétiques qui crient et appellent le peuple à revenir totalement à Dieu. Selon les Ecritures, lorsque la dernière trompette retentira, l'Eglise sera enlevée pour les noces. Dans le livre d'Apocalypse, la voix du Seigneur est comparée au son d'une trompette. Voir \vref{Ex. 19:13}~; \vref{Lé. 23:24}~; \vref{No. 10:9-10}~; \vref{1 Ch. 16:42}~; \vref{Ez. 33:3}~; \vref{Mt. 24:31}~; \vref{1 Co. 15:52}~; \vref{1 Th. 4:16} et \vref{Ap. 1:10}.

\DicoEntry{TYR}\textit{, de l'hébreu «~Tsor~»~: «~un rocher~»}\newline
Ville de l'antique Phénicie (actuel Liban). Hiram, roi de Tyr, donna- en échange de vivres - du bois de cèdre et du bois de cyprès à Salomon pour la construction du temple. Le roi de Tyr est une image de Satan* dans une prophétie d'Ezéchiel. Voir \vref{1 R. 5:1-12} et \vref{Ez. 28}.

\DicoEntry{UR}\textit{, de l'hébreu «~'Uwr~»~: «~flamme, éclat, feu~»}\newline
Ville de Chaldée située au sud de la Babylonie et d'où Abraham était originaire. Voir \vref{Ge. 11:27-31}.

\DicoEntry{URIE}\textit{, de l'hébreu «~Uwriyah~»~: «~Yahweh est ma lumière~»}\newline
Héthien, mari de Bath-Schéba. Il mourut sur le champ de bataille suite à une conspiration de David qui avait connu sa femme et l'avait mise enceinte. Voir \vref{2 S. 11}.

\DicoEntry{VIE ÉTERNELLE}\textit{, de l'hébreu «~aionios~»~: «~sans commencement ni fin~»}\newline
La vie éternelle est un don gratuit de Dieu, un héritage, une promesse qui commence dès la conversion au travers de la connaissance de Dieu. La vie éternelle est Christ lui-même. Voir \vref{Jn. 3:16,36}~; \vref{Ro. 2:7}~; \vref{Ro. 6:23}~; \vref{Tit. 3:7}~; \vref{1 Jn. 2:25} et \vref{1 Jn. 5:20}.

\DicoEntry{VIGNE}\textit{}\newline
Arbre cultivé pour son fruit, la vigne est assimilée à la joie à cause du vin produit par le raisin et consommé dans le cadre de festivités. Le peuple d'Israël était la première vigne de Yahweh, mais elle ne porta pas de fruits. Le royaume de Dieu est aussi associé à la vigne~: Dieu est le vigneron, Jésus-Christ est le cep et tous les enfants de Dieu sont les sarments. Tout sarment qui ne porte pas de fruits est jeté au feu, c'est-à-dire en enfer. Voir \vref{Es. 5:1-7}~; \vref{Mt. 21:33-43} et \vref{Jn. 15:1-8}.

\DicoEntry{VOILE}\textit{, de l'hébreu «~porokhet~»~: «~rideau, voile~» et du grec «~peribolaion~»~: «~une couverture, une enveloppe~»}\newline
1. Etoffe de fin lin retors qui servait de séparation entre le lieu saint et le Saint des saints. Lorsque Jésus-Christ fut crucifié, ce voile se déchira en deux, de haut en bas, ouvrant ainsi l'accès au Saint des saints. Cet événement symbolisait que, par Jésus, tout homme pouvait accéder librement à la présence du Père. Voir \vref{Ex. 26:31-33}~; \vref{Lé. 16:11-19}~; \vref{Mt. 27:50-51}~; \vref{Hé. 9:7-8} et \vref{Hé. 10:19-20}.
\\2. Pan de tissu utilisé pour se couvrir la tête dans certaines cultures. Paul expliqua que les longs cheveux étaient une gloire pour la femme et qu'ils faisaient office de voile naturel. Voir \vref{Ge. 24:65} et \vref{1 Co. 11:15}.
\\3. Au sens figuré, le voile symbolise l'intelligence obscurcie, le cœur non converti et le manque de révélation de la parole qui sont des barrières à la compréhension de la loi. Voir \vref{2 Co. 3:14-16}.

\DicoEntry{YHWH}\textit{}\newline
Aussi appelé tétragramme (mot de quatre lettres), nom avec lequel Dieu se révéla à Moïse lorsque ce dernier le rencontra pour la première fois à Horeb. Ce nom, prononcé Yahweh, signifie «~Je suis celui qui suis~» et souligne le caractère éternel de Dieu. Voir \vref{Ex. 3:1-14}.

\DicoEntry{ZABULON}\textit{, de l'hébreu «~Zebuwluwn~»~: «~habitation~»}\newline
Fils de Jacob et Léa, il devint l'ancêtre de la tribu de Zabulon. Voir \vref{Ge. 30:19-20} et \vref{No. 2:7}.

\DicoEntry{ZACHARIE}\textit{, de l'hébreu «~Zekaryah~»~: «~Yahweh se souvient~»}\newline
1. Fils de Jéroboam, roi d'Israël sur qui il régna uniquement six mois. Il fit ce qui est mal devant Yahweh et fut tué suite à une conspiration contre lui. Voir \vref{2 R. 15:8-11}.
\\2. Prophète et prêtre, fils de Bérékia et petit-fils d'Iddo. Avec le prophète Aggée, il assista Zorobabel, gouverneur de Juda, et Josué, grand prêtre, dans la restauration du temple de Yahweh au retour de la captivité des Juifs. L'ensemble de ses prophéties se trouve dans le livre portant son nom. Voir \vref{Esd. 5:1-2} et \vref{Esd. 6:14-5}.
\\3. Prêtre et père de Jean-Baptiste qu'il eut avec sa femme Elisabeth à un âge avancé. Voir \vref{Lu. 1:5}.

\DicoEntry{ZÉLOTE}\textit{, du grec «~zelotes~»~: «~celui qui est zélé~»}\newline
Patriotes juifs fervents défenseurs de la loi et des traditions ayant pour objectif de résister à l'invasion romaine. Simon, l'un des douze apôtres, en faisait partie. Voir \vref{Lu. 6:15} et \vref{Ac. 1:13}.

\DicoEntry{ZOROBABEL}\textit{, de l'hébreu «~Zerubbabel~»~: «~rejeton de Babylone~»}\newline
Fils de Schealthiel, gouverneur de Juda, il participa à la restauration du temple* de Yahweh après le retour de la captivité du peuple juif. Il figure dans la généalogie de Jésus. Voir \vref{Esd. 3:2}~; \vref{Esd. 5:2}~; \vref{Ag. 1:14}~; \vref{Mt. 1:13} et \vref{Lu. 3:27}.

\end{multicols}
\clearpage}\markboth{}{}
%    \makeatletter
%    % réinitialiser mise en forme
%    \def\@oddhead{\hfil\thepage\hfil}
%    \def\@evenhead{\hfil}
%\makeatother
%% tableaux
%\addcontentsline{toc}{section}{Histoire de la Bible}\clearpage
%\begin{center}Histoire de la Bible 1\end{center}\clearpage
%\begin{center}Histoire de la Bible 2\end{center}\clearpage
%\addcontentsline{toc}{section}{Dénominations}\clearpage
%\begin{center}Dénominations 1\end{center}\clearpage
%\begin{center}Dénominations 2\end{center}\clearpage
%\addcontentsline{toc}{section}{Doctrines}\clearpage
%\begin{center}Doctrines 1\end{center}\clearpage
%\begin{center}Doctrines 2\end{center}\clearpage
%\begin{center}Doctrines 3\end{center}\clearpage
%\begin{center}Doctrines 4\end{center}\clearpage
%\begin{center}Doctrines 5\end{center}\clearpage
%\begin{center}Doctrines 6\end{center}\clearpage
%\begin{center}Doctrines 7\end{center}\clearpage
%\begin{center}Doctrines 8\end{center}\clearpage
%\begin{center}Doctrines 9\end{center}\clearpage
%\begin{center}Doctrines 10\end{center}\clearpage
%\begin{center}Doctrines 11\end{center}\clearpage
%\addcontentsline{toc}{section}{Monnaies}\clearpage
%\begin{center}Monnaies\end{center}\clearpage
%\addcontentsline{toc}{section}{Longueurs / Liquides}\clearpage
%\begin{center}Longueurs / Liquides\end{center}\clearpage
%\addcontentsline{toc}{section}{Poids}\clearpage
%\begin{center}Poids\end{center}\clearpage
%\addcontentsline{toc}{section}{Fêtes de Yahweh}\clearpage
%\begin{center}Fêtes de Yahweh\end{center}\clearpage
%\addcontentsline{toc}{section}{Alphabet hébreu}\clearpage
%\begin{center}Alphabet hébreu\end{center}\clearpage
\end{document}

\clearpage\ShortTitle{1 Rois}\BookTitle{1 Rois}\BFont
\noindent\hrulefill
{\footnotesize
\textit{
\bigskip
{\centering{}
\\Auteur : Inconnu
\\(Heb. : Melakhim)
\\Signification : Roi, Règne
\\Thème : Unité du royaume après le schisme
\\Date de rédaction : 6ème siècle av. J.-C.\\}
}
%\bigskip
\textit{
\\Ce livre relate la vie de Salomon : son accession à la royauté après la mort de son père David, son alliance avec Dieu qui lui accorda une sagesse exceptionnelle ainsi que la construction du temple de Yahweh et du palais royal.
%\bigskip
\\Les premières années du règne de Salomon furent exemplaires. Malheureusement, il ne fit pas preuve de la même piété
que son père et développa une affection particulière pour les femmes étrangères qui l’entrainèrent dans l’idolâtrie. A sa
mort, son fils Roboam accéda au pouvoir et provoqua la division du royaume en deux : d’un côté les dix tribus du nord qui gardèrent le nom d’Israël, gouvernées par Jéroboam, et de l’autre côté les deux tribus du sud, Juda et Benjamin, qui demeurèrent sous l’autorité de Roboam.
%\bigskip
\\Ce livre raconte également le règne et la conduite parfois abominable des rois d’Israël et de Juda jusqu’à Achab et Josaphat.
Il présente la puissance de l’appel prophétique d’Elie, le Tischbite, que Dieu suscita pour ramener son peuple à lui et
montrer sa souveraineté.\bigskip
}
}
\par\nobreak\noindent\hrulefill
\begin{multicols}{2}
\Chap{1}
\TextTitle{Fin de la vie de David}
\VerseOne{}Le roi David était vieux et avancé en âge ; on le couvrait de vêtements parce qu’il ne parvenait point à se réchauffer.
\VS{2}Ses serviteurs lui dirent : Que l'on cherche pour le roi notre seigneur, une jeune fille vierge ; qu’elle se tienne devant le roi, qu’elle le soigne et qu'elle dorme en son sein, afin que le roi notre seigneur, se réchauffe.
\VS{3}On chercha donc dans toutes les contrées d'Israël une jeune et belle femme, et on trouva Abischag la Sunamite, que l’on amena auprès du roi.
\VS{4}Cette jeune femme était fort belle. Elle prit soin du roi et le servit, mais le roi ne la connut point.
\TextTitle{Conspiration d’Adonija pour régner sur Israël}
\VS{5}Alors Adonija, fils de Haggith, se laissa emporter par l’orgueil en disant : Je suis le roi ! Il se procura un char, des cavaliers et cinquante hommes qui couraient devant lui.
\VS{6}Son père ne lui avait jamais fait un reproche jusqu’à ce jour-là, en disant : Pourquoi agis-tu ainsi ? Adonija était très beau de figure, il était né après Absalom.
\VS{7}Il s’entendit avec Joab, fils de Tseruja, et avec le sacrificateur Abiathar, qui embrassèrent son parti.
\VS{8}Mais le sacrificateur Tsadok, Benaja fils de Jehojada, Nathan le prophète, Schimeï, Reï et les vaillants hommes de David ne furent point du parti d'Adonija.
\VS{9}Or, Adonija fit tuer des brebis, des bœufs et des veaux gras près de la pierre de Zohéleth, qui est auprès d’En-Roguel ; il invita tous ses frères, fils du roi, et tous les hommes de Juda qui étaient au service du roi.
\VS{10}Mais il ne convia point Nathan le prophète, ni Benaja, ni les vaillants hommes, ni Salomon, son frère.
\TextTitle{Opposition de Nathan et Bath-Schéba}
\VS{11}Alors Nathan parla à Bath-Schéba, mère de Salomon, en disant : N'as-tu pas entendu qu'Adonija, fils de Haggith, a été fait roi ? Et David notre Seigneur n'en sait rien.
\VS{12}Maintenant donc viens, je t’en donne le conseil afin que tu sauves ta vie et la vie de ton fils Salomon.
\VS{13}Va, entre chez le roi David et dis-lui : Ô roi, mon seigneur, n'as-tu pas fait serment à ta servante, en disant : ton fils Salomon régnera après moi et sera assis sur mon trône ? Pourquoi donc Adonija règne-t-il ?
\VS{14}Et voici, lorsque tu seras encore là et que tu parleras avec le roi, je viendrai après toi et je confirmerai tes dires.
\VS{15}Bath-Schéba se rendit dans la chambre du roi. Or, le roi était très vieux et Abischag, la Sunamite, le servait.
\VS{16}Bath-Schéba s'inclina et se prosterna devant le roi. Et le roi lui dit : Qu'as-tu ?
\VS{17}Et elle lui répondit : Mon seigneur, tu as juré par Yahweh, ton Dieu à ta servante, en lui disant : Ton fils Salomon régnera après moi et s’assiéra sur mon trône.
\VS{18}Mais maintenant voici, Adonija est proclamé roi ! Et tu ne le sais pas, ô roi, mon seigneur !
\VS{19}Il a fait tuer des bœufs, des veaux gras et des brebis en grand nombre, il a convié tous les fils du roi, avec Abiathar, le sacrificateur, et Joab, chef de l'armée, mais il n'a point convié ton serviteur Salomon.
\VS{20}Ô roi mon seigneur ! Les yeux de tout Israël sont sur toi, afin que tu lui fasses connaître qui s’assiéra sur le trône du roi mon seigneur après lui.
\VS{21}Aussi, lorsque le roi mon seigneur sera endormi avec ses pères, nous serons traités comme des coupables, moi et mon fils Salomon.
\VS{22}Tandis qu’elle parlait encore avec le roi, Nathan le prophète se présenta.
\VS{23}On l’annonça au roi en disant : Voici Nathan le prophète ! Il se présenta devant le roi et se prosterna devant lui, le visage contre terre.
\VS{24}Et Nathan dit : Ô roi mon seigneur ! Tu as dit : Adonija régnera après moi et sera assis sur mon trône !
\VS{25}Car il est descendu aujourd'hui, il a sacrifié des bœufs, des veaux gras et des brebis en grand nombre. Il a convié tous les fils du roi, les chefs de l'armée et le sacrificateur Abiathar. Et voici, ils mangent et boivent devant lui ; ils disent : Vive le roi Adonija !
\VS{26}Mais il n'a convié ni moi, ton serviteur, ni le sacrificateur Tsadok, ni Benaja, fils de Jehojada, ni Salomon ton serviteur.
\VS{27}Est-ce bien par ordre de mon seigneur le roi que cette chose a lieu et sans que tu aies fait connaître à ton serviteur quel est celui qui doit s'asseoir sur le trône du roi mon seigneur après lui ?
\VS{28}Et le roi David répondit, en disant : Appelez-moi Bath-Schéba ; elle entra et se présenta devant le roi.
\VS{29}Alors le roi jura et dit : Yahweh, qui m'a délivré de toute détresse, est vivant !
\VS{30}Comme je te l'ai juré par Yahweh, le Dieu d'Israël, en disant : Ton fils Salomon régnera après moi et sera assis sur mon trône à ma place ; ainsi ferai-je aujourd'hui.
\VS{31}Alors Bath-Schéba s'inclina le visage contre terre et se prosterna devant le roi en disant : Que le roi David mon seigneur vive éternellement !
\VS{32}Et le roi David dit : Appelez-moi le sacrificateur Tsadok, le prophète Nathan et Benaja, fils de Jehojada ; et ils se présentèrent devant le roi.
\VS{33}Le roi leur dit : Prenez avec vous les serviteurs de votre seigneur, faites monter mon fils Salomon sur ma mule, et faites-le descendre à Guihon.
\VS{34}Que Tsadok le sacrificateur et Nathan le prophète, l'oignent en ce lieu-là pour roi sur Israël, puis vous sonnerez du shofar et vous direz : Vive le roi Salomon !
\VS{35}Vous monterez après lui et il viendra, il s'assiéra sur mon trône et il régnera à ma place ; car j'ai ordonné qu'il soit le chef d'Israël et de Juda.
\VS{36}Et Benaja fils de Jehojada répondit au roi : Amen ! Ainsi parle Yahweh, le Dieu de mon seigneur le roi !
\VS{37}Comme Yahweh a été avec mon seigneur le roi, qu'il soit aussi avec Salomon, et qu'il élève son trône encore plus que le trône du roi David mon seigneur !
\TextTitle{Salomon oint roi d’Israël par Tsadok\FTNTT{cp. 1 Ch. 29:22}}
\VS{38}Puis Tsadok le sacrificateur descendit avec Nathan le prophète et Benaja, fils de Jehojada, les Kéréthiens et les Péléthiens ; ils firent monter Salomon sur la mule du roi David et le menèrent à Guihon.
\VS{39}Tsadok le sacrificateur prit du tabernacle une corne d'huile dont il oignit Salomon. On sonna du shofar et tout le peuple dit : Vive le roi Salomon !
\VS{40}Et tout le monde monta après lui et le peuple jouait de la flûte, en se livrant à une grande joie, au point que la terre se fendait par leurs cris.
\VS{41}Ce bruit fut entendu d’Adonija et de tous les conviés qui étaient avec lui comme ils achevaient de manger ; et Joab entendant le son du shofar, dit : Pourquoi ce bruit de la ville en tumulte ?
\VS{42}Et comme il parlait encore, voici Jonathan, fils du sacrificateur Abiathar, arriva et Adonija lui dit : Entre, car tu es un vaillant homme et tu apportes de bonnes nouvelles.
\VS{43}Oui ! répondit Jonathan à Adonija : Le roi David, notre seigneur, a établi Salomon roi.
\VS{44}Et le roi a envoyé avec lui Tsadok le sacrificateur, Nathan le prophète, Benaja, fils de Jehojada, les Kéréthiens, et les Péléthiens, et ils l'ont fait monter sur la mule du roi.
\VS{45}Tsadok le sacrificateur, et Nathan le prophète l'ont oint pour roi à Guihon, d'où ils sont remontés avec joie, et la ville est ainsi émue ; c'est là le bruit que vous avez entendu.
\VS{46}Salomon s'est même assis sur le trône royal.
\VS{47}Et les serviteurs du roi sont venus pour bénir le roi David notre seigneur, en disant : Que ton Dieu rende le nom de Salomon encore plus grand que ton nom, et qu'il élève son trône encore plus que ton trône ! Et le roi s'est prosterné sur son lit.
\VS{48}Le roi a ainsi parlé : Béni soit Yahweh, le Dieu d'Israël, qui a aujourd’hui établi sur mon trône un successeur, et qui m’a permis de le voir !
\VS{49}Alors tous les conviés d’Adonija furent saisis de frayeur, ils se levèrent et s'en allèrent chacun son chemin.
\VS{50}Adonija eut peur de Salomon ; il se leva aussi et s'en alla empoigner les cornes de l'autel.
\VS{51}On vint l’apprendre à Salomon, en disant : Voici Adonija a peur du roi Salomon et il a saisi les cornes de l'autel, en disant : Que le roi Salomon me jure aujourd'hui qu'il ne fera point mourir son serviteur par l'épée.
\VS{52}Et Salomon dit : A l’avenir, s’il se comporte en homme de bien il ne tombera pas un seul de ses cheveux à terre ; mais s'il se trouve du mal en lui, il mourra.
\VS{53}Alors le roi Salomon envoya des personnes qui le firent descendre de l'autel. Il vint et se prosterna devant le roi Salomon, et Salomon lui dit : Va dans ta maison.
\Chap{2}
\TextTitle{Dernières paroles de David à Salomon}
\VerseOne{}David approchait du moment de sa mort, et il donna ses ordres à Salomon, son fils, en disant :
\VS{2}Je m'en vais par le chemin de toute la terre, fortifie-toi et comporte-toi en homme.
\VS{3}Observe les commandements de Yahweh, ton Dieu, en marchant dans ses voies, en gardant ses statuts, ses commandements, ses ordonnances et ses préceptes, selon ce qui est écrit dans la loi de Moïse, afin que tu réussisses dans tout ce que tu feras et dans tout ce que tu entreprendras ;
\VS{4}et afin que s’accomplisse cette parole de Yahweh déclarée sur moi : Si tes fils prennent garde à leur voie, pour marcher devant moi dans la vérité, de tout leur cœur et de toute leur âme, tu ne manqueras jamais de successeur sur le trône d'Israël.
\VS{5}Tu sais ce que m'a fait Joab, fils de Tseruja et ce qu'il a fait aux deux chefs des armées d'Israël, Abner, fils de Ner, et à Amasa, fils de Jéther, qu'il a tués, en versant pendant la paix le sang de la guerre ; il a mis de ce sang sur la ceinture qu'il avait sur ses reins et sur les chaussures qu'il avait aux pieds.
\VS{6}Tu agiras selon ta sagesse, en sorte que tu ne laisseras point ses cheveux blancs descendre en paix dans le scheol.
\VS{7}Tu traiteras avec bienveillance les fils de Barzillaï, le Galaadite, et ils seront du nombre de ceux qui mangent à ta table ; car ils se sont approchés de moi quand je fuyais Absalom, ton frère.
\VS{8}Voici, tu as avec toi Schimeï, fils de Guéra, le benjamite de Bachurim, qui proféra contre moi des malédictions violentes le jour où je m'en allais à Mahanaïm. Mais il descendit au-devant de moi vers le Jourdain et je lui jurai par Yahweh, en disant : Je ne te ferai point mourir par l'épée.
\VS{9}Maintenant donc tu ne le laisseras point impuni, car tu es sage, pour savoir comment tu dois le traiter ; et tu feras descendre ses cheveux blancs ensanglantés au scheol.
\TextTitle{Mort de David ; début du règne de Salomon\FTNTT{1 Ch. 29:23-30}}
\VS{10}Ainsi David se coucha avec ses pères, il fut enseveli dans la cité de David.
\VS{11}Et le temps que David régna sur Israël fut quarante ans. Il régna sept ans à Hébron et il régna trente-trois ans à Jérusalem.
\VS{12}Et Salomon s'assit sur le trône de David, son père, et son règne fut très affermi.
\TextTitle{Mort d'Adonija}
\VS{13}Alors Adonija, fils de Haggith, vint vers Bath-Schéba, mère de Salomon et elle dit : Amènes-tu la paix ? Et il répondit : Je viens en paix.
\VS{14}Il ajouta : J'ai un mot à te dire. Elle répondit : Parle !
\VS{15}Et il dit : Tu sais bien que le royaume m'appartenait et que tout Israël s'attendait à ce que je règne. Mais la royauté s’est détournée de moi, elle est échue à mon frère parce que Yahweh la lui a donnée.
\VS{16}Maintenant donc je te demande une chose, ne me la refuse point. Elle lui répondit : Parle !
\VS{17}Et il dit : Je te prie, dis au roi Salomon, car il ne te refusera rien, qu'il me donne Abischag, la Sunamite, pour femme.
\VS{18}Bath-Schéba répondit : Et bien, je parlerai pour toi au roi.
\VS{19}Bath-Schéba se rendit auprès du roi Salomon pour lui parler en faveur d’Adonija ; et le roi se leva pour aller au-devant d’elle, il se prosterna devant elle, puis il s'assit sur son trône. On plaça un siège pour la mère du roi, et elle s'assit à sa droite.
\VS{20}Elle dit alors : J'ai une petite demande à te faire : ne me la refuse pas ! Et le roi lui répondit : Demande, ma mère, car je ne te la refuserai point.
\VS{21}Et elle dit : Qu'on donne Abischag, la Sunamite, pour femme à Adonija, ton frère.
\VS{22}Mais le roi Salomon répondit à sa mère et dit : Et pourquoi demandes-tu Abischag, la Sunamite, pour Adonija ? Demande plutôt le royaume pour lui, parce qu'il est mon frère aîné ; demande-le pour lui, pour Abiathar, le sacrificateur, et pour Joab, fils de Tseruja !
\VS{23}Alors le roi Salomon jura par Yahweh, en disant : Que Dieu me traite dans toute sa rigueur, si Adonija n'a dit cette parole contre sa propre vie !
\VS{24}Maintenant Yahweh est vivant, lui qui m'a établi, qui m'a fait asseoir sur le trône de David, mon père, et qui m'a donné une maison, selon sa promesse ! Aujourd’hui Adonija mourra.
\VS{25}Et le roi Salomon envoya Benaja, fils de Jehojada, qui le frappa, et Adonija mourut.
\TextTitle{Abiathar dépouillé de ses fonctions au temple}
\VS{26}Puis le roi dit à Abiathar, le sacrificateur : Va-t'en à Anathoth sur tes terres, car tu mérites la mort ; toutefois je ne te ferai point mourir aujourd'hui, parce que tu as porté l'arche du Seigneur Yahweh devant David, mon père ; et parce que tu as eu part à toutes les afflictions de mon père.
\VS{27}Ainsi Salomon dépouilla Abiathar de ses fonctions, afin qu'il ne fût plus sacrificateur de Yahweh pour accomplir la parole de Yahweh, qu'il avait prononcée à Silo contre la maison d'Eli.
\TextTitle{Mort de Joab ; Benaja à la tête de l’armé}
\VS{28}Le bruit en parvint à Joab, qui avait suivi le parti d’Adonija, quoiqu'il n’eût pas suivi le parti d’Absalom. Joab s'enfuit au tabernacle de Yahweh et empoigna les cornes de l'autel.
\VS{29}On alla l’apprendre au roi Salomon, en disant : Joab s'en est enfui dans la tente de Yahweh et il est auprès de l'autel. Salomon envoya Benaja, fils de Jehojada, et lui dit : Va et frappe-le.
\VS{30}Benaja entra dans la tente de Yahweh et dit à Joab : Ainsi a parlé le roi : Sors de là ! Mais il répondit : Non ! Je veux mourir ici. Et Benaja rapporta la chose au roi en disant : Joab m'a parlé ainsi et c’est ainsi qu’il m’a répondu.
\VS{31}Et le roi dit à Benaja : Fais comme il t'a dit, frappe-le et enterre-le ; tu ôteras ainsi de dessus moi et de dessus la maison de mon père le sang que Joab a répandu sans cause.
\VS{32}Et Yahweh fera retomber son sang sur sa tête, car il a frappé deux hommes plus justes et meilleurs que lui et les a tués par l'épée, sans que mon père David n’en sût rien : Abner, fils de Ner, chef de l'armée d'Israël, et Amassa, fils de Jéther, chef de l'armée de Juda.
\VS{33}Leur sang retombera sur la tête de Joab et sur la tête de sa postérité à perpétuité ; mais il y aura paix à toujours de par Yahweh, pour David, pour sa postérité, pour sa maison et pour son trône.
\VS{34}Donc Benaja, fils de Jehojada, monta, et il frappa Joab à mort. On l'ensevelit dans sa maison, dans le désert.
\VS{35}Alors le roi établit Benaja, fils de Jehojada, sur l'armée à la place de Joab ; le roi établit aussi Tsadok sacrificateur à la place d'Abiathar.
\TextTitle{Mort de Schimeï}
\VS{36}Puis le roi fit appeler Schimeï et lui dit : Bâtis-toi une maison à Jérusalem, et demeures-y, et n'en sors point pour aller de côté ou d'autre.
\VS{37}Car sache que le jour où tu en sortiras et que tu passeras le torrent de Cédron, tu mourras certainement ; ton sang sera sur ta tête.
\VS{38}Schimeï répondit au roi : Cette parole est bonne ! Ton serviteur fera tout ce que le roi mon Seigneur a dit. Ainsi Schimeï demeura à Jérusalem plusieurs jours.
\VS{39}Mais il arriva qu'au bout de trois ans, deux serviteurs de Schimeï s'enfuirent vers Akisch, fils de Maaca, roi de Gath, et on le rapporta à Schimeï en disant : Voilà tes serviteurs sont à Gath.
\VS{40}Alors Schimeï se leva, sella son âne, et s'en alla à Gath vers Akisch pour chercher ses serviteurs. Schimeï s'en alla donc et ramena de Gath ses serviteurs.
\VS{41}On rapporta à Salomon que Schimeï était allé de Jérusalem à Gath, et qu'il était de retour.
\VS{42}Et le roi envoya appeler Schimeï, et lui dit : Ne t'avais-je pas fait jurer par Yahweh, et ne t'avais-je pas fait cette déclaration formelle : Sache-le, sache bien que le jour que tu sortiras pour aller de côté ou d’autre, tu mourras ? Et ne me répondis-tu pas : La parole que j'ai entendue est bonne ?
\VS{43}Pourquoi donc n'as-tu pas observé le serment que tu as fait par Yahweh et le commandement que je t'avais donné ?
\VS{44}Le roi dit aussi à Schimeï : Tu sais en ton cœur tout le mal que tu as fait à David, mon père ; c'est pourquoi Yahweh a fait retomber ta méchanceté sur ta tête.
\VS{45}Mais le roi Salomon sera béni, et le trône de David sera affermi devant Yahweh à jamais.
\VS{46}Et le roi donna commission à Benaja fils de Jehojada, qui sortit, et frappa Schimeï et Schimeï mourut. La royauté fut ainsi affermie entre les mains de Salomon.
\Chap{3}
\TextTitle{Salomon s'allie à Pharaon}
\VerseOne{}Or, Salomon s'allia avec Pharaon roi d'Egypte. Il prit pour femme la fille de Pharaon, et l'amena en la cité de David, jusqu'à ce qu'il eût achevé de bâtir sa maison, la maison de Yahweh, et la muraille de Jérusalem tout alentour.
\VS{2}Seulement le peuple sacrifiait dans les hauts lieux, parce que jusqu’alors on n’avait pas bâti de maison au nom de Yahweh.
\Chap{3}
\TextTitle{Salomon demande la sagesse à Yahweh\FTNTT{2 Ch. 1:2-10}}
\VS{3}Salomon aimait Yahweh, il marchait selon les ordonnances de David son père. Seulement, c’était sur les hauts lieux qu’il offrait des sacrifices et des parfums.
\VS{4}Le roi se rendit à Gabaon pour y sacrifier, car c'était le plus grand des hauts lieux. Et Salomon offrit mille holocaustes sur cet autel.
\VS{5}Et Yahweh apparut de nuit à Salomon à Gabaon dans un songe, et Dieu lui dit : Demande ce que tu veux que je te donne.
\VS{6}Et Salomon répondit : Tu as usé d'une grande bienveillance envers ton serviteur David, mon père, parce qu’il a marché devant toi fidèlement, dans la justice, et dans la droiture de cœur envers toi. Tu as gardé cette grande bienveillance envers lui en lui donnant un fils qui est assis sur son trône, comme on le voit aujourd'hui.
\VS{7}Or, maintenant, ô Yahweh mon Dieu ! Tu as fait régner ton serviteur à la place de David, mon père, et je ne suis qu'un jeune homme, je ne sais comment me conduire.
\VS{8}Ton serviteur est parmi ce peuple que tu as choisi, un peuple nombreux qui ne peut être compté ni dénombré à cause de sa multitude.
\VS{9}Accorde donc à ton serviteur un cœur intelligent pour juger ton peuple, pour discerner le bien du mal ! Car qui pourrait juger ce peuple si grand ?
\TextTitle{Yahweh exauce Salomon\FTNTT{2 Ch. 1:11-13}}
\VS{10}Cette demande de Salomon plut à Yahweh.
\VS{11}Et Dieu lui dit : Puisque c’est là ta demande et que tu n'as point demandé une longue vie, ni les richesses, ni la mort de tes ennemis, mais que tu as demandé de l'intelligence pour rendre justice,
\VS{12}voici, je fais selon ta parole. Voici, je te donne un cœur sage et intelligent, de sorte qu'il n'y aura eu personne de semblable avant toi et qu’il n'y en aura jamais de semblable après toi.
\VS{13}Et même, je te donne ce que tu n'as point demandé, les richesses et la gloire, de sorte qu'il n'y aura point de roi semblable à toi entre les rois, tant que tu vivras.
\VS{14}Et si tu marches dans mes voies pour garder mes ordonnances et mes commandements, comme David, ton père, je prolongerai tes jours.
\VS{15}Salomon s’éveilla. Et voilà le songe. Puis il s'en retourna à Jérusalem et se tint devant l'Arche de l'alliance de Yahweh. Là, il offrit des holocaustes et des offrandes de paix et fit un festin à tous ses serviteurs.
\VS{16}Alors deux femmes prostituées vinrent au roi et se présentèrent devant lui.
\VS{17}Et l'une de ces femmes dit : Hélas, mon Seigneur ! Nous demeurions cette femme-ci et moi dans une même maison et j'ai accouché près d’elle dans cette maison-là.
\VS{18}Trois jours après, cette femme a aussi accouché. Et nous étions ensemble, il n'y avait aucun étranger avec nous dans cette maison, il n’y avait que nous deux.
\VS{19}Or, l'enfant de cette femme est mort la nuit, parce qu'elle s'était couchée sur lui.
\VS{20}Elle s'est levée au milieu de la nuit, et a pris mon fils à mes côtés pendant que ta servante dormait, et l'a couché dans son sein. Et son fils mort, elle l’a couché dans mon sein.
\VS{21}Le matin, je me suis levée pour allaiter mon fils. Et voici, il était mort. Je l’ai regardé attentivement ce matin-là ; et voici, ce n'était point mon fils que j'avais enfanté.
\VS{22}L’autre femme dit : Non, c’est mon fils qui est vivant, et c’est ton fils qui est mort. Mais la première répliqua : Nullement ! Celui qui est mort est ton fils, et c’est mon fils qui vit. Elles parlaient ainsi devant le roi.
\VS{23}Et le roi dit : L’une dit : C’est mon fils qui est vivant, et c’est ton fils qui est mort ; l’autre dit : Nullement ! C’est ton fils qui est mort, et c’est mon fils qui est vivant.
\VS{24}Alors le roi dit : Apportez-moi une épée ! Et on apporta une épée devant le roi.
\VS{25}Puis le roi dit : Partagez en deux l'enfant qui vit, et donnez-en la moitié à l'une et la moitié à l'autre.
\VS{26}Alors la femme dont le fils était vivant sentit ses entrailles s’émouvoir pour son fils, et elle dit au roi : Ah ! Mon seigneur, qu'on donne à celle-ci l'enfant qui vit et qu'on ne le fasse pas mourir ! Mais l'autre dit : Il ne sera ni à moi ni à toi ; qu'on le partage.
\VS{27}Alors le roi répondit et dit : Donnez à la première l'enfant qui vit, et ne le faites pas mourir. C’est elle qui est sa mère.
\VS{28}Tout Israël entendit parler du jugement que le roi avait prononcé. Et l’on craignit le roi, car l’on reconnut que la sagesse divine était en lui pour rendre justice.
\Chap{4}
\TextTitle{Salomon établit onze chefs et douze intendants}
\VerseOne{}Le roi Salomon était roi sur tout Israël.
\VS{2}Voici les chefs qu’il avait à son service. Azaria, fils du sacrificateur Tsadok,
\VS{3}Elihoreph et Achija, enfants de Schischa, secrétaires ; Josaphat, fils d'Achilud, archiviste ;
\VS{4}Benaja, fils de Jehojada, commandait l'armée ; Tsadok et Abiathar étaient sacrificateurs ;
\VS{5}Azaria, fils de Nathan était chef des intendants ; Zabud, fils de Nathan, était le ministre d’état, favori du roi ;
\VS{6}Achischar, chef de la maison du roi ; et Adoniram, fils d’Abda, préposé sur les impôts.
\VS{7}Or, Salomon avait douze intendants sur tout Israël, qui veillaient à l’entretien du roi et de sa maison ; et chacun pendant un mois de l'année.
\VS{8}Voici leurs noms : Le fils de Hur, sur la montagne d'Ephraïm.
\VS{9}Le fils de Déker, sur Makats, sur Saalbim, sur Beth-Schémesch, à Elon de Beth-Hanan.
\VS{10}Le fils de Hésed, à Arubboth ; il avait Soco et tout le pays de Hépher.
\VS{11}Le fils d'Abinadab avait toute la contrée de Dor ; il avait Thaphath, fille de Salomon, pour femme.
\VS{12}Baana, fils d'Achilud, avait Thaanac et Meguiddo, et tout le pays de Beth-Schean qui est près de Tsarthan au-dessous de Jizreel, depuis Beth-Schean jusqu'à Abel-Mehola et jusqu'au-delà de Jokmeam.
\VS{13}Le fils de Guéber, à Ramoth en Galaad ; il avait les bourgs de Jaïr, fils de Manassé, en Galaad ; il avait aussi toute la contrée d'Argob en Basan, soixante grandes villes à murailles et garnies de barres d'airain.
\VS{14}Achinadab, fils d’Iddo, à Mahanaïm.
\VS{15}Achimaats, qui avait pour femme Basmath, fille de Salomon, en Nephthali.
\VS{16}Baana, fils de Huschaï, en Aser et sur Bealoth.
\VS{17}Josaphat, fils de Paruach, à Issacar.
\VS{18}Schimeï, fils d'Ela, en Benjamin.
\VS{19}Guéber, fils d'Uri, dans le pays de Galaad, le pays de Sihon, roi des Amoréens, et d’Og, roi de Basan ; et il était seul intendant de ce pays-là.
\TextTitle{L'étendue de la domination du royaume}
\VS{20}Juda et Israël étaient en grand nombre, semblable au sable sur le bord de la mer ; ils mangeaient, buvaient et se réjouissaient.
\VS{21}Et Salomon dominait sur tous les royaumes depuis le fleuve jusqu'au pays des Philistins et jusqu'à la frontière d'Egypte ; ils apportaient des présents, et lui furent assujettis pendant toute sa vie.
\VS{22}Or, les vivres de Salomon pour chaque jour étaient de trente cors de fine farine et soixante d'autre farine,
\VS{23}dix bœufs gras, vingt bœufs de pâturages, et cent moutons, outre les cerfs, les daims et les volailles engraissées.
\VS{24}Il dominait sur toutes les contrées de l’autre côté du fleuve, depuis Thiphsach jusqu'à Gaza, sur tous les rois qui étaient de l’autre côté du fleuve. Il était en paix avec tous les pays alentour.
\VS{25}Juda et Israël habitèrent en sécurité chacun sous sa vigne et sous son figuier, depuis Dan jusqu'à Beer-Schéba, durant toute la vie de Salomon.
\VS{26}Salomon avait aussi quarante mille crèches pour les chevaux destinés à ses chars et douze mille hommes de cheval.
\VS{27}Or, les intendants pourvoyaient à l’entretien du roi Salomon et de tous ceux qui s'approchaient de sa table, chacun en son mois ; ils ne les laissaient manquer de rien.
\VS{28}Ils faisaient aussi venir de l'orge et de la paille pour les chevaux et les coursiers dans le lieu où se trouvait le roi, chacun selon les ordres qu'il avait reçus.
\TextTitle{La sagesse de Salomon connue de toute la terre}
\VS{29}Dieu donna à Salomon de la sagesse, une très grande intelligence, et des connaissances multipliées comme le sable qui est sur le bord de la mer.
\VS{30}La sagesse de Salomon surpassait la sagesse de tous les fils de l’orient et toute la sagesse des égyptiens.
\VS{31}Il était plus sage qu’aucun homme, plus qu'Ethan, l’Ezrachite, plus qu'Héman, Calcol et Darda, les fils de Machol ; et sa renommée était répandue parmi toutes les nations d'alentour.
\VS{32}Il a prononcé trois mille paraboles et composa cinq mille cantiques.
\VS{33}Il a aussi parlé des arbres, depuis le cèdre du Liban jusqu'à l'hysope qui sort de la muraille ; il a aussi parlé sur les animaux, sur les oiseaux, sur les reptiles et sur les poissons.
\VS{34}Il venait des gens d'entre tous les peuples pour entendre la sagesse de Salomon, de la part de tous les rois de la terre qui avaient entendu parler de sa sagesse.
\Chap{5}
\TextTitle{Salomon prépare la construction du temple\FTNTT{2 Ch. 2:1 ; 13:16}}
\VerseOne{}Hiram, roi de Tyr, envoya ses serviteurs vers Salomon, car il apprit qu'on l'avait oint pour roi à la place de son père, car Hiram avait toujours aimé David.
\VS{2}Et Salomon fit dire à Hiram :
\VS{3}Tu sais que David, mon père, n'a pu bâtir une maison à Yahweh, son Dieu, à cause des guerriers qui l'ont encerclé, jusqu'à ce que Yahweh les ait mis sous la plante de ses pieds.
\VS{4}Maintenant Yahweh, mon Dieu, m'a donné du repos de toutes parts, et je n'ai plus d’adversaires, plus de calamités !
\VS{5}Voici donc j’ai l’intention de bâtir une maison au nom de Yahweh, mon Dieu, comme Yahweh l’a promis à David, mon père, en disant : Ton fils que je mettrai à ta place sur ton trône sera celui qui bâtira une maison à mon nom.
\VS{6}Ordonne maintenant que l’on coupe des cèdres du Liban pour moi. Mes serviteurs seront avec les tiens, et je donnerai pour tes serviteurs le salaire que tu auras fixé ; car tu sais qu'il n'y a personne parmi nous qui sache couper le bois comme les Sidoniens.
\VS{7}Lorsque Hiram eut entendu les paroles de Salomon, il eut une grande joie et il dit : Béni soit aujourd'hui Yahweh, qui a donné à David un fils sage pour chef de ce grand peuple !
\VS{8}Hiram fit répondre à Salomon : J'ai entendu ce que tu m'as envoyé dire et je ferai tout ce qui te plaira au sujet des bois de cèdre et des bois de cyprès.
\VS{9}Mes serviteurs les descendront du Liban à la mer, puis je les expédierai sur la mer par radeaux jusqu'au lieu que tu m'auras indiqué ; là je les ferai délier, et tu les prendras. Ce que je désire en retour, c’est que tu fournisses des vivres à ma maison.
\VS{10}Hiram donna du bois de cèdre et du bois de cyprès à Salomon autant qu'il en voulait.
\VS{11}Et Salomon donna à Hiram vingt mille cors de froment pour la nourriture de sa maison et vingt cors d'huile d’olives concassées ; Salomon en donna autant à Hiram chaque année.
\VS{12}Et Yahweh donna de la sagesse à Salomon, comme il le lui avait promis ; et il y eut paix entre Hiram et Salomon, et ils firent alliance ensemble.
\TextTitle{Les hommes de corvée\FTNTT{2 Ch. 2:2 ; 17:18}}
\VS{13}Le roi Salomon leva sur tout Israël des hommes de corvée ; ils étaient au nombre de trente mille hommes.
\VS{14}Il en envoya dix mille au Liban chaque mois, tour à tour, ils étaient un mois au Liban, et deux mois chez eux. Adoniram était préposé sur les hommes de corvée.
\VS{15}Salomon avait aussi soixante-dix mille hommes qui portaient les fardeaux et quatre-vingt mille qui taillaient les pierres dans la montagne,
\VS{16}sans compter les chefs au nombre de trois mille trois cents, préposés par Salomon sur le suivi des travaux, et chargés de surveiller les ouvriers.
\VS{17}Le roi ordonna d’extraire de grandes et précieuses pierres, pour faire le fondement de la maison, qui soient toutes taillées,
\VS{18}de sorte que les maçons de Salomon et ceux d'Hiram, taillèrent les pierres et préparèrent le bois et les pierres pour bâtir la maison.
\Chap{6}
\TextTitle{Construction du temple de Yahweh\FTNTT{2 Ch. 3:1-14}}
\VerseOne{}Ce fut la quatre cent quatre-vingtième année après la sortie des enfants d'Israël du pays d'Egypte que Salomon bâtit la maison de Yahweh\FTNTT{Voir les annexes «~Le temple de Salomon~»}, la quatrième année du règne de Salomon sur Israël, au mois de Ziv, qui est le second mois.
\VS{2}La maison que le roi Salomon bâtit à Yahweh avait soixante coudées de long, vingt de large, et trente de haut.
\VS{3}Le portique devant le temple de la maison avait vingt coudées de longueur, répondant à la largeur de la maison, et il avait dix coudées de profondeur sur le devant de la maison.
\VS{4}Il fit placer des fenêtres à la maison, fenêtres solidement grillées.
\VS{5}Il bâtit contre la muraille de la maison, à l’entour, des étages qui entouraient les murs de la maison, le temple et le sanctuaire ainsi il fit des chambres latérales tout autour.
\VS{6}L’étage inférieur était large de cinq coudées, celui du milieu de six coudées et le troisième de sept coudées ; car il avait aménagé des retraites à la maison tout autour en dehors, afin que la charpente n'entrât pas dans les murailles de la maison.
\VS{7}Pour bâtir la maison, on se servit de pierres déjà taillées, de sorte qu'en bâtissant la maison on n'entendit ni marteau, ni hache, ni aucun outil de fer.
\VS{8}L'entrée des chambres de l’étage inférieur était au côté droit de la maison, et on montait à l’étage du milieu par un escalier tournant, et de l’étage du milieu au troisième.
\VS{9}Après avoir achevé de bâtir la maison, Salomon couvrit la maison de planches et de poutres de cèdre.
\VS{10}Et il bâtit les étages joignant toute la maison, avec chacun cinq coudées de haut, et il les lia à la maison par des bois de cèdre.
\VS{11}Alors la parole de Yahweh fut adressée à Salomon, en ces termes :
\VS{12}Quant à cette maison que tu bâtis, si tu marches dans mes statuts, si tu pratiques mes ordonnances et que tu gardes tous mes commandements pour y marcher, j’accomplirai en ta faveur la parole que j'ai dite à David, ton père.
\VS{13}Et j'habiterai au milieu des enfants d'Israël, et je n'abandonnerai point mon peuple d'Israël.
\VS{14}Ainsi Salomon bâtit la maison et l'acheva.
\VS{15}Il revêtit de cèdre les murs de la maison, depuis le sol jusqu'au plafond ; il revêtit ainsi de bois l’intérieur, et il couvrit le sol de la maison de planches de cyprès.
\VS{16}Il revêtit aussi l'espace de vingt coudées de planches de cèdre à partir du fond de la maison, depuis le sol jusqu'au haut des murailles, et il bâtit cet espace au dedans pour en faire le sanctuaire, le saint des saints.
\VS{17}Les quarante coudées sur le devant formaient la maison, c’est-à-dire le temple.
\VS{18}Le bois de cèdre à l’intérieur de la maison était sculpté en coloquintes et en fleurs épanouies ; tout l’intérieur était de cèdre, on ne voyait aucune pierre.
\VS{19}Salomon disposa aussi le sanctuaire, au dedans de la maison vers le fond, pour y mettre l'arche de l'alliance de Yahweh.
\VS{20}Le sanctuaire avait par devant vingt coudées de long, vingt coudées de large, et vingt coudées de haut, et on le couvrit d’or pur ; on en couvrit aussi l'autel, fait de planches de bois de cèdre.
\VS{21}Salomon couvrit d’or pur l’intérieur de la maison, et fit passer un voile avec des chaînes d'or au-devant du sanctuaire, qu’il couvrit également d'or.
\VS{22}Ainsi il couvrit d'or la maison tout entière. Il couvrit aussi d'or tout l'autel qui était devant le sanctuaire.
\VS{23}Et il fit dans le sanctuaire deux chérubins de bois d'olivier sauvage, qui avaient chacun dix coudées de haut.
\VS{24}Chacune des ailes de l'un des chérubins avait cinq coudées et les ailes de l’autre chérubin avaient aussi cinq coudées ; depuis le bout d'une aile jusqu'au bout de l'autre aile il y avait donc dix coudées.
\VS{25}Le second chérubin était aussi de dix coudées. Les deux chérubins étaient d'une même mesure et taillés l'un comme l'autre.
\VS{26}La hauteur de chacun des deux chérubins était de dix coudées.
\VS{27}Salomon plaça les chérubins à l’intérieur, au milieu de la maison. Les ailes des chérubins étaient déployées : l'aile de l'un touchait à l’un des murs, l'aile de l'autre chérubin touchait à l'autre mur ; et leurs autres ailes se rencontraient par l’extrémité au milieu de la maison.
\VS{28}Salomon couvrit d'or les chérubins.
\VS{29}Il fit sculpter sur tout le pourtour des murs de la maison, à l’intérieur et à l’extérieur, des sculptures en relief de chérubins, des palmes et des fleurs épanouies.
\VS{30}Il couvrit aussi d'or le sol de la maison, tant à l’intérieur qu’au-dehors.
\VS{31}A l'entrée du sanctuaire, il fit une porte à deux battants de bois d'olivier sauvage, dont les linteaux avec les poteaux équivalaient à un cinquième du mur.
\VS{32}Les deux battants étaient de bois d'olivier sauvage. Il y fit sculpter des chérubins, des palmes et des fleurs épanouies qu’il couvrit d'or, étendant également l'or sur les chérubins et sur les palmes.
\VS{33}Il fit aussi, à l'entrée du temple, des poteaux de bois d'olivier sauvage, du quart de la dimension du mur.
\VS{34}Les deux battants étaient de bois de sapin ; chacun des battants était formé de deux planches brisées.
\VS{35}Il y fit sculpter des chérubins, des palmes et des fleurs épanouies, et les couvrit d'or, proprement posé sur la sculpture.
\VS{36}Il bâtit aussi le parvis de l’intérieur de trois rangées de pierres de taille et d'une rangée de poutres de cèdre.
\VS{37}La quatrième année, au mois de Ziv, les fondements de la maison de Yahweh furent posés.
\VS{38}Et la onzième année, au mois de Bul, qui est le huitième mois, la maison fut achevée dans toutes ses parties et telle qu’elle devait être. Salomon la construisit en l’espace de sept années.
\Chap{7}
\TextTitle{Construction du palais royal}
\VerseOne{}Salomon bâtit aussi sa maison, et l'acheva complètement en treize ans.
\VS{2}Il bâtit d’abord la maison de la forêt du Liban, de cent coudées de long, de cinquante coudées de large, et de trente coudées de haut, sur quatre rangées de colonnes de cèdre ; et sur les colonnes il y avait des poutres de cèdre.
\VS{3}On couvrit de bois de cèdre les chambres qui portaient sur les colonnes qui étaient au nombre de quarante-cinq, quinze par étages.
\VS{4}Et il y avait trois rangées de fenêtrages ; et une fenêtre répondait à l'autre en trois endroits.
\VS{5}Toutes les portes et tous les poteaux étaient formés de poutres carrées, avec les fenêtres ; et à chacun des trois étages, les ouvertures étaient en vis-à-vis les unes des autres.
\VS{6}Il fit aussi le portique de colonnes, long de cinquante coudées, et large de trente coudées ; et un autre portique en avant avec des colonnes et des degrés sur leur front.
\VS{7}Il fit aussi le portique du trône sur lequel il rendait ses jugements, appelé le portique du jugement ; on le couvrit de cèdre depuis un bout du sol jusqu'à l'autre.
\VS{8}La maison où il demeurait fut construite de la même manière, dans une autre cour, derrière le portique. Salomon fit une maison bâtie comme ce portique à la fille de Pharaon, qu'il avait prise pour femme.
\VS{9}Toutes ces constructions étaient de pierres de prix, taillées d’après des mesures, sciées à la scie, en dedans et en dehors, depuis les fondements jusqu'aux corniches, et par dehors jusqu'au grand parvis.
\VS{10}Le fondement était en pierres magnifiques et de grand prix, de grandes pierres, des pierres de dix coudées et des pierres de huit coudées.
\VS{11}Et par-dessus il y avait des pierres de prix, taillées d’après des mesures, et du bois de cèdre.
\VS{12}Et le grand parvis avait aussi tout alentour trois rangées de pierres de taille et une rangée de poutres de cèdre, comme le parvis intérieur de la maison de Yahweh, et le portique de la maison.
\TextTitle{Hiram, artisan spécialiste en airain\FTNTT{2 Ch. 2:12-13}}
\VS{13}Or, le roi Salomon fit venir de Tyr Hiram ;
\VS{14}fils d'une femme veuve de la tribu de Nephthali, et d’un père tyrien, Hiram travaillait le cuivre ; fort expert, intelligent et savant pour faire toutes sortes d'ouvrages d'airain ; il arriva auprès du roi Salomon, et il fit tout son ouvrage.
\TextTitle{Les colonnes du temple\FTNTT{2 Ch. 3:15-17}}
\VS{15}Il fit les deux colonnes d'airain, la première avait dix-huit coudées de hauteur ; et un cordon de douze coudées mesurait le tour de la seconde.
\VS{16}Il fit aussi deux chapiteaux d'airain fondu pour mettre sur les sommets des colonnes ; le premier chapiteau était de cinq coudées de hauteur, le second était aussi de cinq coudées.
\VS{17}Il fit des treillis en forme de maillages, des festons façonnés en forme de chaînes, pour les chapiteaux qui étaient sur le sommet des colonnes, sept pour le premier des chapiteaux, et sept pour le second.
\VS{18}Il fit deux rangs de grenades autour de l’un des treillis, pour couvrir le chapiteau qui était sur le sommet d'une des colonnes ; et il fit de même pour l'autre chapiteau.
\VS{19}Dans le portique, les chapiteaux qui étaient sur le sommet des colonnes figuraient des fleurs de lis hautes de quatre coudées au porche.
\VS{20}Ces chapiteaux placés sur les deux colonnes étaient entourés de deux cents grenades, en haut, depuis le renflement qui était au-delà du treillis ; il y avait aussi deux cents grenades, disposées par rangs, autour du second chapiteau.
\VS{21}Il dressa donc les colonnes au portique du temple. Il dressa la colonne de droite qu’il nomma Jakin ; puis il dressa la colonne de gauche qu’il nomma Boaz.
\VS{22}Et l’on mit sur le chapiteau des colonnes l'ouvrage figurant des fleurs de lis ; ainsi l'ouvrage des colonnes fut achevé.
\TextTitle{La mer de fonte\FTNTT{2 Ch. 4:2-5}}
\VS{23}Il fit aussi la mer de fonte. Elle avait dix coudées d'un bord à l'autre, ronde tout autour, avec cinq coudées de haut ; et un cordon de trente coudées en mesurait le tour.
\VS{24}Au-dessous de son bord, des coloquintes l'environnaient, dix à chaque coudée, lesquelles faisaient tout le tour de la mer. Il y avait deux rangées de coloquintes, jetées en fonte.
\VS{25}Et elle était posée sur douze bœufs, dont trois regardaient le nord et trois regardaient l'occident, trois regardaient le sud et trois regardaient l'orient. La mer était sur eux et toute la partie postérieure de leur corps était tournée en dedans.
\VS{26}Son épaisseur était d'une paume, et son bord était comme le bord d'une coupe en fleur de lis ; elle contenait deux mille baths.
\TextTitle{Les dix socles d'airain}
\VS{27}Il fit aussi dix socles d'airain, ayant chacun quatre coudées de long, quatre coudées de large et trois coudées de haut.
\VS{28}Ces socles étaient réalisés de telle manière qu'il y avait des panneaux enchâssés entre leurs bordures.
\VS{29}Sur les panneaux qui étaient entre les bordures, il y avait des lions, des bœufs et des chérubins. Et sur les bordures, au-dessus et en dessous des lions et des bœufs, il y avait des ornements qui pendaient en festons.
\VS{30}Chaque socle avait quatre roues d'airain avec des essieux d'airain. Ses quatre pieds leur servaient d’appuis. Ces appuis étaient fondus au-dessous de la cuve, et au-dessus étaient les festons.
\VS{31}Le couronnement offrait à son intérieur une ouverture avec un prolongement d'une coudée vers le haut ; cette ouverture était arrondie comme pour les ouvrages de ce genre et elle avait une coudée et demie de largeur. Il s’y trouvait aussi des sculptures ; les panneaux étaient carrés, et non arrondis.
\VS{32}Les quatre roues étaient sous les panneaux, et les essieux des roues fixés à la base ; chaque roue était haute d'une coudée et demie.
\VS{33}Les roues étaient faites comme les roues de chars ; leurs essieux, leurs jantes, leurs rais et leurs moyeux étaient tous de fonte.
\VS{34}Il y avait aux quatre angles de chaque socle quatre consoles d’une même pièce que la base.
\VS{35}La partie supérieure de la base se terminait par un cercle d’une demi-coudée de hauteur, et elle avait ses appuis et ses panneaux de la même pièce.
\VS{36}Puis, on sculpta sur la surface de ses appuis et sur ses panneaux, des chérubins, des lions et des palmes, selon les espaces libres, et des ornements tout autour.
\VS{37}Ainsi les dix socles étaient tous d’une même fonte, d’une même mesure et d’une même forme.
\TextTitle{Les dix cuves d'airain\FTNTT{2 Ch. 4:6}}
\VS{38}Il fit aussi dix cuves d'airain, dont chacune contenait quarante baths, et chaque cuve était de quatre coudées, chaque cuve était sur l’un des dix socles.
\VS{39}Il mit cinq socles au côté droit de la maison, et cinq au côté gauche de la maison ; quant à la mer, il l’a mis au côté droit de la maison, vers l'orient du côté sud.
\TextTitle{Totalité de l’œuvre d’Hiram}
\VS{40}Ainsi Hiram fit les cuves, les pelles et les bassins, et il acheva tout l'ouvrage qu'il faisait au roi Salomon pour la maison de Yahweh.
\VS{41}Savoir, deux colonnes avec les deux chapiteaux qui étaient sur le sommet des colonnes ; et deux maillages pour couvrir les deux bourrelets des chapiteaux qui étaient sur le sommet des colonnes ;
\VS{42}les quatre cents grenades pour les deux maillages, deux rangs de grenades pour chaque réseau, pour couvrir les deux renflements des chapiteaux, qui étaient sur les colonnes ;
\VS{43}les dix socles ; et les dix cuves pour mettre sur les socles ;
\VS{44}la mer avec les douze bœufs sous la mer ;
\VS{45}les pots, les pelles et les bassins. Tous ces ustensiles que Hiram fit au roi Salomon pour la maison de Yahweh étaient d'airain poli.
\VS{46}Le roi les fit fondre dans la plaine du Jourdain, dans un sol argileux, entre Succoth et Tsarthan.
\VS{47}Et Salomon ne pesa aucun de ces ustensiles, parce qu'ils étaient en trop grand nombre, de sorte qu'on ne rechercha point le poids de l’airain.
\TextTitle{Divers ustensiles d’or pour la maison de Yahweh}
\VS{48}Salomon fit aussi tous les ustensiles pour la maison de Yahweh, savoir l'autel d'or, et les tables d'or, sur lesquelles étaient les pains de proposition ;
\VS{49}les chandeliers d’or pur, cinq à droite et cinq à gauche devant le sanctuaire, avec les fleurs, les lampes et les mouchettes d'or ;
\VS{50}les coupes, les couteaux, les bassins, les tasses et les brasiers d’or pur. Les gonds, même des portes de la maison, à l’entrée du saint des saints, à la porte de la maison et à l’entrée du temple, étaient d'or.
\VS{51}Ainsi fut achevé tout l'ouvrage que le roi Salomon fit pour la maison de Yahweh ; puis il y fit apporter l'or, l’argent et les ustensiles que David, son père, avait consacrés ; il les mit dans les trésors de la maison de Yahweh.
\Chap{8}
\TextTitle{L’arche de l’alliance placée dans le saint des saints ; la gloire de Yahweh remplit le temple \FTNTT{2 Ch. 5:2-14}}
\VerseOne{}Alors le roi Salomon convoqua près de lui à Jérusalem les anciens d'Israël, tous les chefs des tribus et les chefs de famille des fils d'Israël, pour transporter l'arche de l'alliance de Yahweh de la cité de David, qui est Sion.
\VS{2}Tous les hommes d'Israël s’assemblèrent auprès du roi Salomon, au mois d'Ethanim, qui est le septième mois, pendant la fête.
\VS{3}Une fois tous les anciens d'Israël arrivés, les sacrificateurs portèrent l'arche.
\VS{4}Ils transportèrent l'arche de Yahweh, la tente d'assignation, et tous les ustensiles qui étaient dans le tabernacle ; les sacrificateurs et les Lévites les emportèrent.
\VS{5}Le roi Salomon et toute l'assemblée d'Israël convoquée auprès de lui se tinrent devant l'arche. Ils sacrifièrent du gros et du menu bétail en si grand nombre, qu'on ne pouvait ni nombrer ni compter.
\VS{6}Et les sacrificateurs portèrent l'arche de l'alliance de Yahweh à sa place, dans le sanctuaire de la maison, dans le saint des saints, sous les ailes des chérubins.
\VS{7}Car les chérubins avaient les ailes étendues sur l’emplacement de l'arche, et ils couvraient l'arche et ses barres par-dessus.
\VS{8}On avait donné aux barres une longueur telle que leurs extrémités se voyaient du lieu saint devant le sanctuaire, mais elles ne se voyaient point du dehors. Elles sont demeurées là jusqu'à ce jour.
\VS{9}Il n'y avait rien dans l'arche que les deux tables de pierre que Moïse y déposa en Horeb, lorsque Yahweh fit alliance avec les enfants d'Israël à leur sortie du pays d'Egypte.
\VS{10}Au moment où les sacrificateurs sortirent du lieu saint, la nuée remplit la maison de Yahweh.
\VS{11}Les sacrificateurs ne purent pas y rester pour faire le service, à cause de la nuée ; car la gloire de Yahweh remplissait la maison de Yahweh.
\TextTitle{Discours de Salomon\FTNTT{2 Ch. 6:1-11}}
\VS{12}Alors Salomon dit : Yahweh veut habiter dans l'obscurité !
\VS{13}J'ai achevé de bâtir une maison pour ta demeure ô Yahweh ! Ce sera une demeure, un lieu où tu résideras éternellement.
\VS{14}Le roi tourna son visage, et bénit toute l'assemblée d'Israël ; car toute l'assemblée d'Israël se tenait là debout.
\VS{15}Et il dit : Béni soit Yahweh, le Dieu d'Israël, qui a parlé de sa propre bouche à David, mon père, et qui a accompli par sa puissance ce qu’il avait déclaré en disant :
\VS{16}Depuis le jour où je fis sortir mon peuple d'Israël hors d'Egypte, je n'ai choisi aucune ville d'entre toutes les tribus d'Israël pour y bâtir une maison afin que mon nom y fût, mais j'ai choisi David pour qu’il règne sur mon peuple d'Israël.
\VS{17}David, mon père, avait à cœur de bâtir une maison au nom de Yahweh, le Dieu d'Israël.
\VS{18}Et Yahweh dit à David, mon père : Puisque tu as eu à cœur de bâtir une maison à mon nom, tu as bien fait d’avoir eu cette intention.
\VS{19}Néanmoins, tu ne bâtiras point cette maison, mais ton fils qui sortira de tes entrailles sera celui qui bâtira cette maison à mon Nom.
\VS{20}Yahweh a donc accompli la parole qu'il avait prononcée. Je me suis élevé à la place de David, mon père, et me suis assis sur le trône d'Israël, comme Yahweh l’avait annoncé, et j'ai bâti cette maison au Nom de Yahweh, le Dieu d'Israël.
\VS{21}J'y ai établi ici un lieu pour l'arche, dans lequel est l'alliance de Yahweh, qu'il traita avec nos pères quand il les fit sortir hors du pays d'Egypte.
\TextTitle{Prière de Salomon\FTNTT{2 Ch. 6:12-42}}
\VS{22}Ensuite Salomon se tint devant l'autel de Yahweh en la présence de toute l'assemblée d'Israël, et étendant ses mains vers les cieux,
\VS{23}il dit : Ô Yahweh, Dieu d'Israël ! Il n'y a point de Dieu semblable à toi en haut dans les cieux, ni en bas sur la terre ; tu gardes l'alliance et la miséricorde envers tes serviteurs qui marchent devant ta face de tout cœur !
\VS{24}Ainsi tu as tenu parole à ton serviteur David, mon père, car ce que tu as déclaré de ta bouche, tu l'as accompli en ce jour par ta main puissante.
\VS{25}Maintenant donc, ô Yahweh, Dieu d'Israël, prête attention à la promesse faite à ton serviteur David, mon père, en lui disant : Tu ne manqueras jamais devant moi d’un successeur assis sur le trône d'Israël, pourvu seulement que tes fils prennent garde à leur voie et qu’ils marchent devant ma face, comme tu y as marché.
\VS{26}Et maintenant, ô Dieu d'Israël ! Je te prie, que s’accomplisse la promesse que tu as faite à ton serviteur David, mon père.
\VS{27}Mais Dieu habiterait-il véritablement sur la terre ? Voilà, les cieux, même les cieux des cieux ne peuvent te contenir ; combien moins cette maison que j'ai bâtie !
\VS{28}Toutefois, ô Yahweh, mon Dieu, sois attentif à la prière que t’adresse ton serviteur et à sa supplication, pour entendre le cri et la prière que ton serviteur t’adresse aujourd'hui.
\VS{29}Que tes yeux soient ouverts jour et nuit sur cette maison, sur le lieu dont tu as dit : Là sera mon Nom ! Ecoute la prière que ton serviteur fait en ce lieu.
\VS{30}Daigne exaucer la supplication de ton serviteur et de ton peuple d'Israël lorsqu’ils te prieront en ce lieu ; exauce du lieu de ta demeure. Des cieux, exauce, et pardonne !
\VS{31}Si quelqu'un pèche contre son prochain et qu’on lui impose un serment pour le faire jurer, et que le serment aura été fait devant ton autel dans cette maison ;
\VS{32}écoute-le des cieux, et agis. Juge tes serviteurs, condamne le coupable en lui rendant selon sa conduite ; rends justice à l’innocent, et traite-le selon son innocence !
\VS{33}Quand ton peuple d'Israël sera battu par l'ennemi, pour avoir péché contre toi, s’il revient à toi et rend gloire à ton Nom, en t’adressant des prières et des supplications dans cette maison,
\VS{34}exauce-le des cieux, et pardonne le péché de ton peuple d'Israël, et ramène-le dans la terre que tu as donnée à leurs pères.
\VS{35}Quand les cieux seront fermés et qu'il n'y aura point de pluie, à cause de ses péchés contre toi, s'il te fait une prière en ce lieu-ci, qu’il loue ton Nom, et s’il se détourne de ses péchés, parce que tu les auras affligés,
\VS{36}exauce-le des cieux, pardonne le péché de tes serviteurs et de ton peuple d'Israël, à qui tu enseigneras quel est le chemin par lequel ils doivent marcher et envoie-leur la pluie sur la terre que tu as donnée à ton peuple pour héritage !
\VS{37}Quand il y aura dans le pays, famine, peste, jaunisse, nielle, sauterelles d’une espèce ou d’une autre, même quand les ennemis assiégeront ton peuple dans son propre pays, quand il y aura un fléau ou une maladie quelconque ;
\VS{38}si un homme, si tout ton peuple d'Israël fait entendre des prières et des supplications, que chacun reconnaisse la plaie de son cœur et étende les mains vers cette maison,
\VS{39}exauce-le des cieux, du lieu de ta demeure, pardonne, et agis. Rends à chacun selon toutes ses voies, parce que tu auras connu leurs cœurs ; car toi seul connais le cœur de tous les fils des hommes ;
\VS{40}et ils te craindront toute leur vie dans le pays que tu as donné à nos pères !
\VS{41}Et même lorsque l'étranger, qui n’est pas de ton peuple d'Israël, viendra d'un pays éloigné à cause de ton Nom,
\VS{42}car on saura que ton Nom est grand, ta main puissante et ton bras étendu, quand il viendra prier dans cette maison,
\VS{43}exauce-le des cieux, du lieu de ta demeure, et fais à cet étranger selon ce qu’il t’aura demandé, afin que tous les peuples de la terre connaissent ton Nom pour te craindre, comme ton peuple d'Israël ; et pour connaître que ton Nom est invoqué sur cette maison que j'ai bâtie !
\VS{44}Quand ton peuple sortira pour combattre son ennemi, par la voie par laquelle tu l’auras envoyé, s'ils prient Yahweh en regardant vers cette ville que tu as choisie et vers cette maison que j'ai bâtie à ton Nom !
\VS{45}Exauce des cieux leurs prières et leurs supplications, et fais leur justice !
\VS{46}Quand ils pécheront contre toi, car il n'y a point d'homme qui ne pèche, et que tu seras irrité contre eux et que tu les auras livrés à leurs ennemis, qui les emmènera captifs dans un pays ennemi, lointain ou proche ;
\VS{47}si dans le pays où ils auront été menés captifs, ils reviennent à toi et t’adressent des supplications, se repentent et te prient au pays de ceux qui les auront emmenés captifs, en disant : Nous avons péché, nous avons commis l’iniquité, nous avons fait le mal !
\VS{48}S'ils reviennent à toi de tout leur cœur et de toute leur âme, dans le pays de leurs ennemis, qui les auront emmenés captifs, et s'ils t'adressent leurs prières, les regards tournés vers le pays que tu as donné à leurs pères, vers la ville que tu as choisie, vers la maison que j'ai bâtie à ton Nom,
\VS{49}exauce des cieux, du lieu de ta demeure, leurs prières et leurs supplications, et fais-leur justice.
\VS{50}Pardonne à ton peuple ses offenses et ses péchés envers toi, et fais que ceux qui les auront emmenés captifs aient pitié d'eux et leur fassent grâce,
\VS{51}car ils sont ton peuple et ton héritage, et tu les as fait sortir hors d'Egypte, du milieu d'une fournaise de fer !
\VS{52}Que tes yeux donc soient ouverts sur la supplication de ton serviteur et celle de ton peuple d'Israël, pour les exaucer dans tout ce pourquoi ils crieront à toi !
\VS{53}Car tu les as séparés de tous les autres peuples de la terre pour être ton héritage, comme tu l’as déclaré par Moïse, ton serviteur, quand tu fis sortir nos pères hors d'Egypte, ô Seigneur Yahweh !
\TextTitle{Bénédictions et réjouissances\FTNTT{2 Ch. 7:4-10}}
\VS{54}Lorsque Salomon eut achevé de faire cette prière et cette supplication à Yahweh, il se leva de devant l'autel de Yahweh où il était agenouillé et les mains étendues vers les cieux.
\VS{55}Il se tint debout, et bénit toute l'assemblée d'Israël à haute voix, en disant :
\VS{56}Béni soit Yahweh, qui a donné du repos à son peuple d'Israël, comme il l’avait annoncé ! De toutes les paroles qu'il avait prononcées par le moyen de Moïse, son serviteur, aucune n’est restée sans effet.
\VS{57}Que Yahweh, notre Dieu, soit avec nous, comme il a été avec nos pères ; qu'il ne nous abandonne point et qu'il ne nous délaisse point,
\VS{58}mais qu'il incline nos cœurs vers lui, afin que nous marchions dans toutes ses voies, et que nous observions ses commandements, ses statuts et ses ordonnances, qu'il a prescrits à nos pères !
\VS{59}Que ces paroles, par lesquelles j'ai fait supplication à Yahweh, soient présentes devant Yahweh, notre Dieu, jour et nuit ; afin qu'il fasse justice à son serviteur et à son peuple d’Israël en tout temps,
\VS{60}afin que tous les peuples de la terre reconnaissent que c'est Yahweh qui est Dieu et qu'il n'y en a point d'autre !
\VS{61}Que votre cœur soit intègre envers Yahweh, notre Dieu, comme aujourd'hui, pour marcher dans ses statuts et pour garder ses commandements.
\VS{62}Le roi et tout Israël avec lui offrirent des sacrifices devant Yahweh.
\VS{63}Salomon offrit un sacrifice d'offrande de paix à Yahweh, savoir vingt-deux mille bœufs et cent vingt mille brebis. Ainsi le roi et tous les enfants d'Israël firent la dédicace de la maison de Yahweh.
\VS{64}En ce jour-là, le roi consacra le milieu du parvis, qui est devant la maison de Yahweh ; car il offrit là les holocaustes, les offrandes et les graisses des sacrifices d’offrandes de paix, parce que l'autel d'airain qui est devant Yahweh, était trop petit pour contenir les holocaustes, les offrandes et les graisses des offrandes de paix.
\VS{65}Et en ce temps-là, Salomon célébra une fête solennelle ; et tout Israël avec lui, venu en grande multitude depuis les environs de Hamath jusqu'au torrent d'Egypte, devant Yahweh, notre Dieu, pendant sept jours, et sept autres jours, soit quatorze jours.
\VS{66}Le huitième jour, il renvoya le peuple. Et ils bénirent le roi, et s'en allèrent dans leurs demeures, en se réjouissant, et le cœur heureux pour tout le bien que Yahweh avait fait à David, son serviteur, et à Israël, son peuple.
\Chap{9}
\TextTitle{Yahweh apparaît à Salomon une seconde fois\FTNTT{2 Ch. 7:11-22}}
\VerseOne{}Lorsque Salomon eut achevé de bâtir la maison de Yahweh, la maison royale, et tout ce que Salomon prit plaisir à faire,
\VS{2}Yahweh apparut à Salomon une seconde fois, comme il lui était apparu à Gabaon.
\VS{3}Et Yahweh lui dit : J'exauce ta prière, et la supplication que tu as faite devant moi, j'ai sanctifié cette maison que tu as bâtie pour y mettre mon Nom à jamais, et mes yeux et mon cœur seront toujours là.
\VS{4}Quant à toi, si tu marches devant moi comme David, ton père, a marché, avec intégrité et de cœur et avec droiture, en faisant tout ce que je t'ai commandé, et si tu gardes mes statuts et mes ordonnances,
\VS{5}j’affermirai le trône de ton royaume sur Israël à jamais, comme je l’ai déclaré à David, ton père, en disant : Tu ne manqueras jamais d’un successeur sur le trône d'Israël.
\VS{6}Mais si vous et vos fils, vous vous détournez de moi et que vous ne gardiez pas mes commandements, mes lois que je vous ai prescrites, et si vous allez servir d'autres dieux et vous prosterner devant eux,
\VS{7}je retrancherai Israël de la terre que je lui ai donnée, je rejetterai loin de moi cette maison que j'ai consacrée à mon Nom et Israël sera un sujet de sarcasme et de moquerie parmi tous les peuples.
\VS{8}Et si haut placée qu’ait été cette maison, quiconque passera auprès d'elle sera étonné et sifflera. Et on dira : Pourquoi Yahweh a-t-il ainsi traité ce pays et cette maison ?
\VS{9}Et on répondra : Parce qu'ils ont abandonné Yahweh, leur Dieu, qui avait tiré leurs pères hors du pays d'Egypte, qu'ils se sont attachés à d'autres dieux, se sont prosternés devant eux et les ont servis, voilà pourquoi Yahweh a fait venir sur eux tous ces maux.
\TextTitle{Les réalisations de Salomon\FTNTT{2 Ch. 8:1-18}}
\VS{10}Au bout de vingt ans, Salomon avait bâti les deux maisons, la maison de Yahweh et la maison royale.
\VS{11}Hiram, roi de Tyr, avait fourni à Salomon du bois de cèdre, du bois de sapin et de l'or, autant qu'il en avait voulu, le roi Salomon donna à Hiram vingt villes dans le pays de Galilée.
\VS{12}Hiram sortit de Tyr, pour voir les villes que Salomon lui avait données. Mais elles ne lui plurent point,
\VS{13}et il dit : Quelles villes m'as-tu assignées, mon frère ? Et il les appela, pays de Cabul, nom qu’elles ont conservé jusqu'à ce jour.
\VS{14}Hiram avait aussi envoyé au roi cent vingt talents d'or.
\VS{15}Voici ce qui concerne les hommes de corvée que le roi Salomon leva pour bâtir la maison de Yahweh, sa maison, Millo, la muraille de Jérusalem, Hatsor, Meguiddo et Guézer.
\VS{16}Pharaon, roi d'Egypte, était venu s’emparer de Guézer et l'avait incendiée, il avait tué les Cananéens qui habitaient dans la ville. Puis il la donna pour dot à sa fille, femme de Salomon.
\VS{17}Salomon donc bâtit Guézer, et Beth-Horon la basse,
\VS{18}Baalath et Thadmor, dans le désert qui est au pays,
\VS{19}toutes les villes servant de magasins et lui appartenant, les villes pour les chars et les villes pour la cavalerie, et tout ce qu’il plut à Salomon de bâtir à Jérusalem, au Liban, et dans tout le pays dont il était le souverain.
\VS{20}Tout le peuple qui était resté des Amoréens, des Héthiens, des Phéréziens, des Héviens et des Jébusiens ne faisaient point partie des fils d'Israël,
\VS{21}leurs descendants qui étaient demeurés après eux dans le pays et que les fils d'Israël n'avaient pu dévouer par le moyen de l'interdit, Salomon les fit placer à son service comme gens de corvée à toujours.
\VS{22}Mais Salomon n’employa aucun des fils d'Israël comme esclaves ; car ils étaient ses hommes de guerre, ses serviteurs, ses chefs, ses officiers, les chefs de ses chars et ses hommes d'armes.
\VS{23}Les chefs préposés aux travaux par Salomon étaient au nombre de cinq cent cinquante, lesquels géraient l'intendance des ouvriers.
\VS{24}La fille de Pharaon monta de la cité de David dans la maison que Salomon lui avait bâtie. Ce fut alors qu’il bâtit Millo.
\VS{25}Trois fois par an, Salomon offrait des holocaustes et des offrandes de paix sur l'autel qu'il avait bâti à Yahweh, et il brûlait des parfums sur celui qui était devant Yahweh. Et il acheva la maison.
\VS{26}Le roi Salomon construisit des navires à Etsjon-Guéber, près d'Eloth, sur le rivage de la Mer Rouge, au pays d'Edom.
\VS{27}Et Hiram envoya sur ces navires, auprès des serviteurs de Salomon, ses propres serviteurs, des hommes connaissant la mer.
\VS{28}Ils allèrent en Ophir, et ils prirent de là quatre cent vingt talents d'or qu’ils apportèrent au roi Salomon.
\Chap{10}
\TextTitle{La reine de Séba chez Salomon\FTNTT{2 Ch. 7:1-12}}
\VerseOne{}Or, la reine de Séba ayant appris la renommée de Salomon, à cause du Nom de Yahweh, vint l’éprouver par des énigmes.
\VS{2}Elle entra dans Jérusalem avec une suite fort nombreuse, et avec des chameaux qui portaient des aromates, une grande quantité d'or, et des pierres précieuses. Elle se rendit auprès de Salomon, et lui parla de tout ce qu'elle avait dans le cœur.
\VS{3}Salomon répondit à toutes ses questions, et il n’y eut aucune parole à laquelle le roi ne put fournir une explication.
\VS{4}La reine de Séba vit toute la sagesse de Salomon et la maison qu'il avait bâtie,
\VS{5}les mets de sa table, la demeure de ses serviteurs, l’ordre de service, leurs vêtements, ses échansons, et les holocaustes qu'il offrait dans la maison de Yahweh.
\VS{6}Elle fut toute ravie en elle-même, elle parla ainsi au roi : Ce que j'ai entendu dire dans mon pays au sujet de ta sagesse était donc vrai !
\VS{7}Je ne croyais pas ce qu’on en disait avant d’être venue et que mes yeux ne l'aient vu. Et voici, on ne m'en avait point rapporté la moitié. Ta sagesse et ta prospérité surpassent tout ce que j'en avais entendu.
\VS{8}Heureux sont tes gens ! Heureux tes serviteurs qui se tiennent continuellement devant toi, et qui entendent ta sagesse !
\VS{9}Béni soit Yahweh, ton Dieu, qui t’a accordé la faveur de t’établir sur le trône d'Israël ! Car Yahweh a aimé Israël à toujours ; et t'a établi roi pour faire droit et justice.
\VS{10}Puis elle donna au roi cent vingt talents d'or, une très grande quantité d’aromates et des pierres précieuses. Il ne vint jamais depuis une aussi grande abondance d’aromates que la reine de Séba en donna au roi Salomon.
\VS{11}Et les navires de Hiram, qui amenèrent de l'or d'Ophir, amenèrent aussi d’Ophir une grande quantité de bois de santal et de pierres précieuses.
\VS{12}Le roi fit des supports de ce bois de santal pour la maison de Yahweh et pour la maison royale ; il en fit aussi des harpes et des luths pour les chantres ; il ne vint plus de ce bois de santal et on n’en a plus vu jusqu'à ce jour-là.
\VS{13}Le roi Salomon donna à la reine de Séba tout ce qu'elle désira et répondit à tout et ce qu'elle lui demanda. Il lui fit en outre des présents dignes d'un roi tel que Salomon. Puis elle s'en retourna et alla dans son pays, elle et ses serviteurs.
\TextTitle{Les richesses de Salomon\FTNTT{2 Ch. 9:13-28}}
\VS{14}Le poids de l'or qui revenait à Salomon chaque année, était de six cent soixante-six talents d'or,
\VS{15}outre ce qui lui revenait des négociants, du trafic des marchands, de tous les rois d'Arabie, et des gouverneurs de ce pays-là.
\VS{16}Le roi Salomon fit aussi deux cents grands boucliers d'or battu au marteau, employant six cents sicles d'or pour chaque bouclier,
\VS{17}et trois cents autres boucliers d'or battu au marteau, pour chacun desquels il employa trois mines d'or ; et le roi les mit dans la maison de la forêt du Liban.
\VS{18}Le roi fit aussi un grand trône d'ivoire, qu'il couvrit d’or pur.
\VS{19}Ce trône avait six degrés, et la partie supérieure, le haut du trône était arrondi par derrière. Il y avait des accoudoirs de chaque côté du siège et deux lions se tenaient auprès des accoudoirs.
\VS{20}Il y avait aussi douze lions sur les six degrés du trône, de part et d'autre. Il ne s'est rien fait de tel dans aucun royaume.
\VS{21}Toute la vaisselle du buffet du roi Salomon était d'or, et toutes les coupes de la maison de la forêt du Liban étaient d’or pur. Il n'y en avait point en argent ; on n’en faisait aucun cas du temps de Salomon.
\VS{22}Car le roi avait en mer des navires de Tarsis avec la flotte d'Hiram ; et tous les trois ans la flotte de Tarsis revenait, apportant de l'or, de l'argent, de l'ivoire, des singes et des paons.
\VS{23}Le roi Salomon fut plus grand que tous les rois de la terre, tant en richesses qu'en sagesse.
\VS{24}Tous les habitants de la terre cherchaient à voir la face de Salomon, pour écouter la sagesse que Dieu avait mise en son cœur.
\VS{25}Et chacun d'eux lui apportait son présent, des vases d’or et d'argent, des vêtements, des armes, des aromates, des chevaux et des mulets, tous les ans.
\VS{26}Salomon rassembla ses chars et sa cavalerie ; il y avait mille quatre cents chars et douze mille chevaliers, qu'il plaça dans les villes où il tenait ses chars et à Jérusalem près du roi.
\VS{27}Le roi rendit l'argent aussi commun à Jérusalem que les pierres ; et les cèdres que les sycomores qui croissent dans les plaines, tant il y en avait.
\VS{28}C’est d’Egypte que provenaient les chevaux de Salomon ; une caravane de marchands du roi allait les chercher par troupes, à un prix fixe :
\VS{29}Un char montait et sortait d'Egypte pour six cents sicles d'argent et chaque cheval pour cent cinquante sicles ; ils en amenaient de même avec eux pour tous les rois des Héthiens et pour les rois de Syrie.
\Chap{11}
\TextTitle{Salomon détourne son cœur de Yahweh}
\VerseOne{}Le roi Salomon aima plusieurs femmes étrangères, outre la fille de Pharaon ; savoir des Moabites, des Ammonites, des Edomites, des Sidoniennes et des Héthiennes.
\VS{2}Elles étaient d'entre les nations dont Yahweh avait dit aux enfants d'Israël : Vous n'irez point vers elles, et elles ne viendront point vers vous ; car certainement elles feraient détourner vos cœurs pour suivre leurs dieux. Salomon s'attacha à elles et les aima.
\VS{3}Il eut donc pour femmes sept cents princesses et trois cents concubines ; et ses femmes détournèrent son cœur.
\VS{4}Au temps de la vieillesse de Salomon, ses femmes firent détourner son cœur vers d'autres dieux ; et son cœur ne fut point intègre devant Yahweh, son Dieu, comme David, son père.
\VS{5}Salomon alla après Astarté, la divinité des Sidoniens, et après Milcom, l'abomination des Ammonites.
\VS{6}Ainsi Salomon fit ce qui est mal aux yeux de Yahweh, et il ne persévéra point à suivre Yahweh, comme David, son père.
\VS{7}Et Salomon bâtit un haut lieu à Kemosch, l'abomination des Moabites, sur la montagne qui est vis-à-vis de Jérusalem ; et à Moloc, l'abomination des fils d’Ammon.
\VS{8}Il en fit de même pour toutes ses femmes étrangères, qui offraient des parfums et des sacrifices à leurs dieux.
\VS{9}C'est pourquoi Yahweh fut irrité contre Salomon, parce qu'il avait détourné son cœur de Yahweh, le Dieu d'Israël, qui lui était apparu deux fois.
\VS{10}Il lui avait donné cet ordre de ne point aller après d'autres dieux ; mais il ne garda point ce que Yahweh lui avait ordonné.
\VS{11}Et Yahweh dit à Salomon : Puisque tu as agi de la sorte, et que tu n'as pas observé l’alliance et les ordonnances que je t'avais prescrites, je déchirerai le royaume afin qu'il ne soit plus à toi et je le donnerai à ton serviteur.
\VS{12}Toutefois je ne le ferai point en ton temps, pour l’amour de David, ton père. Ce sera d'entre les mains de ton fils que je déchirerai le royaume.
\VS{13}Néanmoins je ne déchirerai pas tout le royaume, j'en donnerai une tribu à ton fils, pour l'amour de David, mon serviteur, et pour l'amour de Jérusalem, que j'ai choisie.
\TextTitle{Dieu suscite des ennemis à Salomon}
\VS{14}Yahweh donc suscita un ennemi à Salomon, savoir Hadad, l’Edomite, qui était de la race royale d'Edom.
\VS{15}Car il était arrivé qu'au temps que David était en Edom, Joab, chef de l'armée, étant monté pour ensevelir les morts, tua tous les mâles qui étaient en Edom ;
\VS{16}Joab demeura là six mois avec tout Israël, jusqu'à ce qu'il eût exterminé tous les mâles d'Edom.
\VS{17}Ce fut alors qu’Hadad prit la fuite avec des Edomites d'entre les serviteurs de son père, pour se retirer en Egypte. Hadad était alors un jeune garçon.
\VS{18}Une fois partis de Madian, ils allèrent à Paran, prirent avec eux des hommes de Paran, et arrivèrent en Egypte auprès de Pharaon, roi d'Egypte, qui lui donna une maison, pourvut à sa subsistance et lui donna aussi une terre.
\VS{19}Et Hadad trouva grâce aux yeux de Pharaon, de sorte que Pharaon lui donna pour femme la sœur de sa propre femme, la sœur de la reine Thachpenès.
\VS{20}Et la sœur de Thachpenès lui enfanta son fils Guenubath. Thachpenès le sevra dans la maison de Pharaon. Ainsi Guenubath fut dans la maison de Pharaon, parmi les fils de Pharaon.
\VS{21}Lorsque Hadad apprit en Egypte que David s'était endormi avec ses pères, et que Joab, chef de l'armée, était mort, il dit à Pharaon : Laisse-moi partir dans mon pays.
\VS{22}Et Pharaon lui répondit : Que te manque-t-il auprès de moi, pour désirer ainsi t'en aller dans ton pays ? Et il répondit : Je n’ai besoin de rien, mais cependant laisse-moi partir.
\VS{23}Dieu suscita aussi un autre ennemi à Salomon, savoir Rezon, fils d'Eliada, qui s'était enfui de chez son maître Hadadézer, roi de Tsoba,
\VS{24}Il avait rassemblé des gens auprès de lui, et était devenu chef de bandes, lorsque David les fit périr ; et ils s'en allèrent à Damas, s’y établirent et y régnèrent.
\VS{25}Rezon fut ennemi d'Israël au temps de Salomon, en même temps qu’Hadad le mettait à mal, il avait en aversion Israël et il régna sur la Syrie.
\VS{26}Jéroboam aussi, serviteur de Salomon, s'éleva également contre le roi. Il était fils de Nebath, Ephratien, de Tseréda, dont la mère s’appelait Tserua, femme veuve.
\VS{27}Voici à quelle occasion il s'éleva contre le roi. Salomon bâtissait Millo, et fermait la brèche de la cité de David, son père.
\VS{28}Jéroboam était un homme fort et vaillant ; et Salomon, voyant ce jeune homme à l’ouvrage, lui assigna la charge de toute la maison de Joseph.
\VS{29}Dans ce même temps, Jéroboam, étant sorti de Jérusalem, rencontra en chemin le prophète Achija de Silo, revêtu d'un manteau neuf, et ils étaient eux deux tout seuls dans les champs.
\VS{30}Et Achija prit le manteau neuf qu'il avait sur lui et le déchira en douze morceaux,
\VS{31}et il dit à Jéroboam : Prends-en pour toi dix morceaux ! Car ainsi parle Yahweh, le Dieu d'Israël : Voici, je vais arracher le royaume d'entre les mains de Salomon, et je t'en donnerai dix tribus.
\VS{32}Mais il aura une tribu, pour l'amour de David, mon serviteur, et pour l'amour de Jérusalem, qui est la ville que j'ai choisie d'entre toutes les tribus d'Israël.
\VS{33}Parce qu'ils m'ont abandonné, et se sont prosternés devant Astarté, la déesse des Sidoniens, devant Kemosch, dieu de Moab, et devant Milcom, le dieu des fils d’Ammon, et qu'ils n'ont point marché dans mes voies, pour faire ce qui est droit à mes yeux et garder mes statuts, et mes ordonnances, comme l’a fait David, père de Salomon.
\VS{34}Toutefois, je n'ôterai pas de sa main tout le royaume, car pendant toute sa vie je le maintiendrai prince, pour l'amour de David, mon serviteur, que j'ai choisi et qui a observé mes commandements et mes lois.
\VS{35}Mais j'ôterai le royaume d'entre les mains de son fils, je t'en donnerai dix tribus ;
\VS{36}j'en donnerai une tribu à son fils, afin que David, mon serviteur, ait une lampe à toujours devant moi dans Jérusalem, qui est la ville que j'ai choisie pour y mettre mon Nom.
\VS{37}Je te prendrai donc, tu régneras sur tout ce que ton âme désirera, tu seras roi sur Israël.
\VS{38}Et il arrivera que si tu m'obéis en tout ce que je te commanderai, que tu marches dans mes voies, en faisant tout ce qui est droit à mes yeux, en gardant mes statuts et mes commandements, comme l’a fait David, mon serviteur, je serai avec toi, je te bâtirai une maison qui sera stable, comme j'en ai bâti une à David, et je te donnerai Israël.
\VS{39}Ainsi j’humilierai la postérité de David à cause de cela, mais non pas à toujours.
\VS{40}Salomon chercha à faire mourir Jéroboam, mais Jéroboam se leva et s'enfuit en Egypte vers Schischak, roi d'Egypte ; et il demeura en Egypte jusqu'à la mort de Salomon.
\TextTitle{Mort de Salomon\FTNTT{2 Ch. 9:29-31}}
\VS{41}Or, le reste des faits de Salomon, tout ce qu'il a fait et sa sagesse, cela n'est-il pas écrit dans le livre des actes de Salomon ?
\VS{42}Salomon régna à Jérusalem sur tout Israël pendant quarante ans.
\VS{43}Ainsi Salomon s'endormit avec ses pères, il fut enseveli dans la cité de David, son père. Et Roboam, son fils, régna en sa place.
\Chap{12}
\TextTitle{Règne de Roboam\FTNTT{2 Ch. 10:1 ; cp. Ec. 2:18-19}}
\VerseOne{}Roboam se rendit à Sichem, parce que tout Israël était venu à Sichem pour l'établir roi.
\VS{2}Or, Jéroboam, fils de Nebath, était encore en Egypte, où il s'était enfui de devant le roi Salomon, quand il l'apprit, et c’était en Egypte qu’il habitait.
\VS{3}On l'envoya appeler. Ainsi Jéroboam et toute l'assemblée d'Israël vinrent, ils parlèrent à Roboam, en disant :
\VS{4}Ton père a mis sur nous un pesant joug ; mais toi allège maintenant cette rude servitude de ton père et ce pesant joug qu'il a mis sur nous ; et nous te servirons.
\VS{5}Il leur répondit : Allez, et dans trois jours revenez vers moi. Et le peuple s'en alla.
\VS{6}Le roi Roboam consulta les vieillards qui avaient été auprès de Salomon, son père, pendant sa vie et leur dit : Que me conseillez-vous de répondre à ce peuple ?
\VS{7}Et ils lui répondirent, en disant : Si aujourd'hui tu rends service à ce peuple et que tu leur cèdes, et si tu leur réponds avec des paroles bienveillantes, ils seront tes serviteurs à toujours.
\VS{8}Mais Roboam laissa le conseil que les vieillards lui avaient donné et consulta les jeunes gens qui avaient grandi avec lui et qui se tenaient près de lui.
\VS{9}Il leur dit : Que me conseillez-vous de répondre à ce peuple qui m'a parlé, en disant : Allège le joug que ton père a mis sur nous ?
\VS{10}Alors les jeunes gens qui avaient grandi avec lui, lui dirent : Tu parleras ainsi à ce peuple qui t'est venu dire : Ton père a mis sur nous un pesant joug, mais toi allège-le-nous ! Tu leur parleras ainsi : Mon petit doigt est plus gros que les reins de mon père.
\VS{11}Or, mon père a mis sur vous un pesant joug, mais moi je rendrai votre joug encore plus pesant ; mon père vous a châtiés avec des fouets, mais moi je vous châtierai avec des scorpions.
\VS{12}Or, trois jours après, Jéroboam avec tout le peuple vint vers Roboam, selon que le roi leur avait dit : Retournez vers moi dans trois jours.
\VS{13}Mais le roi répondit durement au peuple, laissant le conseil que les anciens lui avaient donné.
\VS{14}Il leur parla selon le conseil des jeunes gens, en leur disant : Mon père a mis sur vous un pesant joug, mais moi, je rendrai votre joug plus pesant encore ; mon père vous a châtiés avec des fouets, mais moi, je vous châtierai avec des scorpions.
\VS{15}Le roi donc n'écouta point le peuple ; car cela était ainsi conduit par Yahweh, en vue d’accomplir la parole qu'il avait prononcée par le ministère d'Achija de Silo, à Jéroboam, fils de Nebath.
\TextTitle{Schisme du royaume ; Jéroboam devient roi d’Israël\FTNTT{2 Ch. 10:12-19 ; 11:1-4}}
\VS{16}Et quand tout Israël vit que le roi ne les avait point écoutés, le peuple fit cette réponse au roi, en disant : Quelle part avons-nous avec David ? Nous n'avons point de propriété avec le fils d'Isaï ! A tes tentes, Israël ! Et toi David, pourvois maintenant à ta maison ! Ainsi Israël s'en alla dans ses tentes.
\VS{17}Les fils d'Israël qui habitaient dans les villes de Juda furent les seuls sur qui Roboam régna.
\VS{18}Or, le roi Roboam envoya Adoram, qui était préposé aux impôts, mais tout Israël le lapida, et il mourut. Alors le roi Roboam se hâta de monter sur un char pour s'enfuir à Jérusalem.
\VS{19}C’est ainsi qu’Israël s’est détaché de la maison de David jusqu'à ce jour.
\VS{20}Tout Israël apprit que Jéroboam était de retour, ils l'envoyèrent appeler dans l'assemblée, et l'établirent roi sur tout Israël. La tribu de Juda fut la seule qui suivit la maison de David.
\VS{21}Roboam arriva à Jérusalem, il rassembla toute la maison de Juda et la tribu de Benjamin, savoir cent quatre-vingt mille hommes d’élite choisis et disposés à faire la guerre, pour combattre contre la maison d'Israël, et ramener la domination à Roboam, fils de Salomon.
\VS{22}Mais la parole de Dieu fut ainsi adressée à Schemaeja, homme de Dieu, disant :
\VS{23}Parle à Roboam, fils de Salomon, roi de Juda, et à toute la maison de Juda, et de Benjamin, et au reste du peuple, en disant :
\VS{24}Ainsi parle Yahweh : Vous ne monterez point et vous ne combattrez point contre vos frères, les fils d'Israël ! Que chacun de vous retourne dans sa maison, car ceci a été fait de par moi. Ils obéirent à la parole de Yahweh, et s'en retournèrent, selon la parole de Yahweh.
\TextTitle{Idolâtrie de Jéroboam}
\VS{25}Or, Jéroboam bâtit Sichem sur la montagne d'Ephraïm, et y demeura, puis il en sortit et bâtit Penuel.
\VS{26}Et Jéroboam dit en son cœur : Maintenant le royaume pourrait bien retourner à la maison de David.
\VS{27}Si ce peuple monte à Jérusalem pour faire des sacrifices dans la maison de Yahweh, le cœur de ce peuple se tournera vers son seigneur, Roboam, roi de Juda, et ils me tueront, et ils retourneront à Roboam, roi de Juda.
\VS{28}Sur quoi le roi ayant pris conseil, fit deux veaux d'or et dit au peuple : Vous êtes longtemps montés à Jérusalem ! Voici ton dieu, ô Israël, qui t'a fait sortir hors du pays d'Egypte.
\VS{29}Il plaça un de ces veaux à Béthel, et il mit l'autre à Dan.
\VS{30}Et cela fut une occasion de péché, car le peuple allait jusqu'à Dan, pour se prosterner devant l'un des veaux.
\VS{31}Il fit aussi des maisons dans les hauts lieux, et établit des sacrificateurs pris parmi tout le peuple, qui n'étaient point des enfants de Lévi.
\VS{32}Jéroboam ordonna aussi une fête solennelle au huitième mois, le quinzième jour du mois, à l'imitation de la fête solennelle qu'on célébrait en Juda, et il offrait des sacrifices sur un autel. Il fit ainsi à Béthel, sacrifiant aux veaux qu'il avait faits, et il établit à Béthel des sacrificateurs des hauts lieux qu'il avait élevés.
\VS{33}Or, le quinzième jour du huitième mois, savoir au mois qu'il avait choisi lui-même, il monta sur l'autel qu'il avait fait à Béthel, et célébra cette fête solennelle pour les enfants d'Israël ; et fit brûler des parfums sur l'autel.
\Chap{13}
\TextTitle{Un homme de Dieu envoyé vers Jéroboam}
\VerseOne{}Et voici, un homme de Dieu vint de Juda à Béthel avec la parole de Yahweh, pendant que Jéroboam se tenait près de l'autel pour brûler des parfums.
\VS{2}Et il cria contre l'autel selon la parole de Yahweh, et dit : Autel ! Autel ! Ainsi parle Yahweh : Voici, un fils naîtra à la maison de David, qui aura pour nom Josias ; il immolera sur toi les sacrificateurs des hauts lieux qui brûlent des parfums sur toi, et on brûlera sur toi des ossements d’hommes !
\VS{3}Le même jour il donna un signe, en disant : C'est ici le signe dont Yahweh a parlé : Voici, l'autel se fendra, et la cendre qui est dessus sera répandue.
\VS{4}Lorsque le roi entendit la parole que l'homme de Dieu avait criée contre l'autel de Béthel, Jéroboam étendit sa main de l'autel, en disant : Saisissez-le ! Et la main qu'il étendit contre lui devint sèche, et il ne put la ramener à lui.
\VS{5}L'autel aussi se fendit, et la cendre qui était sur l'autel fut répandue, selon le signe que l'homme de Dieu avait donné par la parole de Yahweh.
\VS{6}Alors le roi prit la parole et dit à l'homme de Dieu : Implore Yahweh, ton Dieu, et prie pour moi, afin que ma main revienne à moi. L'homme de Dieu implora Yahweh, et la main du roi put revenir à lui et elle fut comme auparavant.
\VS{7}Alors le roi dit à l'homme de Dieu : Entre avec moi dans la maison, tu prendras quelque nourriture et je te donnerai un présent.
\VS{8}Mais l'homme de Dieu répondit au roi : Quand tu me donnerais la moitié de ta maison, je n'entrerais point chez toi, je ne mangerais point de pain, ni ne boirais d'eau en ce lieu.
\VS{9}Car cela m'a été ordonné par Yahweh, qui m'a dit : Tu ne mangeras point de pain, tu ne boiras point d'eau et tu ne t'en retourneras point par le chemin par lequel tu y seras allé.
\VS{10}Il s'en alla donc par un autre chemin, et ne s'en retourna point par le chemin par lequel il était venu à Béthel.
\TextTitle{L’homme de Dieu séduit par un vieux prophète}
\VS{11}Or, il y avait un vieux prophète qui demeurait à Béthel. Ses fils vinrent raconter toutes les choses que l'homme de Dieu avait faites ce jour-là à Béthel, et les paroles qu'il avait dites au roi ; et comme les fils de ce prophète les rapportaient à leur père,
\VS{12}il leur demanda : Par quel chemin s'en est-il allé ? Or, ses fils avaient vu le chemin par lequel l'homme de Dieu qui était venu de Juda s'en était allé.
\VS{13}Et il dit à ses fils : Sellez-moi un âne. Ils lui sellèrent, puis il monta dessus.
\VS{14}Et il s'en alla après l'homme de Dieu, et le trouva assis sous un chêne. Et il lui dit : Es-tu l'homme de Dieu qui est venu de Juda ? Et il lui répondit : C'est moi.
\VS{15}Alors il lui dit : Viens avec moi dans la maison, et tu prendras de quoi te nourrir.
\VS{16}Mais il répondit : Je ne puis retourner avec toi, ni entrer chez toi et je ne mangerai point de pain, ni ne boirai d'eau avec toi en ce lieu ;
\VS{17}Car il m'a été dit de la part de Yahweh : Tu ne mangeras point de pain, tu ne boiras point d'eau, et tu ne t'en retourneras point par le chemin par lequel tu seras allé.
\VS{18}Et il lui dit : Et moi aussi je suis prophète comme toi ; et un ange m'a parlé de la part de Yahweh, en disant : Ramène-le avec toi dans ta maison, qu'il mange du pain, et qu'il boive de l'eau ; mais il lui mentait.
\VS{19}Il s'en retourna donc avec lui, il mangea du pain et but de l'eau dans sa maison.
\VS{20}Et il arriva que comme ils étaient assis à table, la parole de Yahweh fut adressée au prophète qui l'avait ramené.
\VS{21}Et il cria à l'homme de Dieu qui était venu de Juda, en disant : Ainsi a parlé Yahweh : Parce que tu as été rebelle au commandement de Yahweh et que tu n'as point gardé l’ordre que Yahweh, ton Dieu, t'avait donné ;
\VS{22}mais tu t'en es retourné, tu as mangé du pain et bu de l'eau dans le lieu dont Yahweh t'avait dit : N'y mange point de pain et n'y bois point d'eau, ton cadavre n'entrera point au sépulcre de tes pères.
\VS{23}Et quand le prophète qu'il avait ramené eut mangé du pain et bu de l’eau, il sella l’âne pour lui.
\VS{24}L’homme de Dieu s'en alla, et un lion le rencontra dans le chemin, et le tua. Son corps était étendu dans le chemin, l'âne resta auprès du corps, et le lion aussi resta à côté du cadavre.
\VS{25}Et voici des passants virent le corps étendu dans le chemin et le lion qui se tenait auprès du corps ; et ils vinrent le dire dans la ville où le vieux prophète demeurait.
\VS{26}Et le prophète qui avait ramené du chemin l'homme de Dieu, l'ayant appris, dit : C'est l'homme de Dieu qui a été rebelle au commandement de Yahweh, c'est pourquoi Yahweh l'a livré au lion, qui l'aura déchiré après l'avoir tué, selon la parole que Yahweh avait dite à ce prophète.
\VS{27}Et il parla à ses fils, en disant : Sellez-moi un âne. Ils le lui sellèrent,
\VS{28}Et il s'en alla et trouva le corps de l'homme de Dieu étendu dans le chemin, l'âne et le lion qui se tenaient auprès du corps. Le lion n'avait pas dévoré le cadavre, ni déchiré l'âne.
\VS{29}Alors le prophète leva le corps de l'homme de Dieu, le plaça sur l'âne et le ramena ; et ce vieux prophète revint dans la ville pour le pleurer et l'enterrer.
\VS{30}Il mit le corps de ce prophète dans le sépulcre, et il pleura sur lui, en disant : Hélas, mon frère !
\VS{31}Après l’avoir enterré, il parla à ses fils, en disant : Quand je serai mort, enterrez-moi au sépulcre où est enterré l'homme de Dieu, et vous déposerez mes os à côté de ses os.
\VS{32}Car elle s’accomplira, la parole qu’il a criée de la part de Yahweh, contre l'autel qui est à Béthel et contre toutes les maisons des hauts lieux qui sont dans les villes de Samarie.
\TextTitle{Jéroboam continue dans le mal}
\VS{33}Néanmoins, Jéroboam ne se détourna point de sa mauvaise voie, mais il établit de nouveau des sacrificateurs de hauts lieux pris parmi tout le peuple ; quiconque le voulait, Jéroboam le consacrait sacrificateur des hauts lieux.
\VS{34}Cela fut une occasion de péché pour la maison de Jéroboam, qui fut effacée et exterminée de dessus la terre.
\Chap{14}
\TextTitle{Maladie et mort du fils de Jéroboam}
\VerseOne{}En ce temps-là, Abija, fils de Jéroboam, devint malade.
\VS{2}Et Jéroboam dit à sa femme : Lève-toi maintenant et déguise-toi, en sorte qu'on ne reconnaisse point que tu es la femme de Jéroboam, et va à Silo. Voici, là est Achija, le prophète, qui m'a dit que je serais roi sur ce peuple.
\VS{3}Emmène avec toi dix pains, des gâteaux et un vase de miel, et entre chez lui ; il te dira ce qui arrivera à l’enfant.
\VS{4}La femme de Jéroboam fit donc ainsi ; elle se leva et s'en alla à Silo puis elle entra dans la maison d'Achija. Or, Achija ne pouvait plus voir, parce qu’il avait les yeux figés à cause de sa vieillesse.
\VS{5}Et Yahweh dit à Achija : Voici, la femme de Jéroboam, qui vient te consulter concernant l’état de son fils, parce qu'il est malade. Tu lui parleras de telle et de telle manière. Quand elle arrivera, elle se sera déguisée.
\VS{6}Lorsque Achija eut entendu le bruit de ses pas, comme elle franchissait la porte, il dit : Entre, femme de Jéroboam. Pourquoi fais-tu semblant d'être quelqu’un d’autre ? Je suis chargé de t’annoncer des choses dures.
\VS{7}Va, dis à Jéroboam : Ainsi parle Yahweh, le Dieu d'Israël : Parce que je t'ai élevé du milieu du peuple et que je t'ai établi pour chef sur mon peuple d'Israël,
\VS{8}j'ai arraché le royaume de la maison de David et je te l'ai donné ; mais parce que tu n'as point été comme David, mon serviteur, qui a gardé mes commandements et qui a marché après moi de tout son cœur, ne faisant que ce qui est droit à mes yeux.
\VS{9}Tu as fait pire que tous ceux qui ont été devant toi, tu es allé te faire d'autres dieux et des images de fonte, pour m'irriter, et tu m'as rejeté derrière ton dos !
\VS{10}A cause de cela, voici, je vais faire venir le malheur sur la maison de Jéroboam ; je retrancherai ce qui appartient à Jéroboam, ce qu’il détient et ce qu’il néglige en Israël, et je brûlerai la maison de Jéroboam, comme on brûle les ordures, jusqu'à ce qu'il n'en reste plus.
\VS{11}Celui de la maison de Jéroboam qui mourra dans la ville, les chiens le mangeront, et celui qui mourra aux champs, les oiseaux du ciel le mangeront. Car Yahweh a parlé.
\VS{12}Toi donc lève-toi, va dans ta maison. Dès que tes pieds entreront dans la ville, l'enfant mourra.
\VS{13}Tout Israël le pleurera et on l’enterrera ; car lui seul de la famille de Jéroboam entrera au sépulcre, parce que Yahweh, le Dieu d'Israël, a trouvé quelque chose de bon en lui seul dans toute la maison de Jéroboam.
\VS{14}Yahweh s'établira un roi sur Israël qui retranchera la maison de Jéroboam. Ce jour-là, n’est-ce pas déjà ce qui arrive ?
\VS{15}Yahweh frappera Israël, l'agitant comme le roseau est agité dans l'eau ; et il arrachera Israël de ce bon pays qu'il a donné à leurs pères, et les dispersera au-delà du fleuve, parce qu'ils se sont fait des idoles, irritant Yahweh.
\VS{16}Il livrera Israël à cause des péchés que Jéroboam a commis et qu’il a fait commettre à Israël.
\VS{17}Alors la femme de Jéroboam se leva et s'en alla, elle vint à Thirtsa : et comme elle franchit le seuil de la maison, le jeune garçon mourut.
\VS{18}Il fut enseveli et tout Israël le pleura, selon la parole de Yahweh, proférée par son serviteur Achija, le prophète.
\TextTitle{Règne de Nadab sur Israël\FTNTT{cp. 2 Ch. 13:20}}
\VS{19}Quant au reste des faits de Jéroboam, comment il a fait la guerre et comment il a régné, cela est écrit dans le livre des Chroniques des rois d'Israël.
\VS{20}Jéroboam régna vingt-deux ans, puis il s'endormit avec ses pères. Et Nadab, son fils, régna à sa place.
\TextTitle{Juda dans l'apostasie\FTNTT{2 Ch. 12:1}}
\VS{21}Roboam, fils de Salomon, régna en Juda. Il avait quarante et un ans quand il devint roi, et il régna dix-sept ans à Jérusalem, la ville que Yahweh avait choisie d'entre toutes les tribus d'Israël pour y mettre son nom. Sa mère s’appelait Naama, l’Ammonite.
\VS{22}Juda fit ce qui est mal aux yeux de Yahweh ; et par les péchés qu'ils commirent, ils excitèrent sa jalousie plus que leurs pères ne l'avaient jamais fait.
\VS{23}Ils se bâtirent, eux aussi, des hauts lieux avec des statues et des idoles sur toute colline élevée, et sous tout arbre verdoyant.
\VS{24}Il y avait dans le pays des prostitués. Et ils firent selon toutes les abominations des nations que Yahweh avait chassées devant les enfants d'Israël.
\TextTitle{Le roi d’Egypte emporte les trésors de Juda ; mort de Roboam\FTNTT{2 Ch. 12:2-16}}
\VS{25}La cinquième année du roi Roboam, Schischak, roi d'Egypte, monta contre Jérusalem.
\VS{26}Il prit les trésors de la maison de Yahweh et les trésors de la maison royale, et il emporta tout. Il prit aussi tous les boucliers d'or que Salomon avait faits.
\VS{27}Le roi Roboam fit des boucliers d'airain au lieu de ceux-là, et les mit entre les mains des chefs des coureurs, qui gardaient l’entrée de la maison du roi.
\VS{28}Toutes les fois où le roi entrait dans la maison de Yahweh, les coureurs les portaient, et ensuite ils les rapportaient dans la chambre des coureurs.
\VS{29}Le reste des actions de Roboam, et tout ce qu'il a fait, n'est-il pas écrit au livre des Chroniques des rois de Juda ?
\VS{30}Il y eut toujours guerre entre Roboam et Jéroboam.
\VS{31}Roboam s'endormit avec ses pères et fut enseveli avec eux dans la cité de David. Sa mère avait pour nom Naama, l’Ammonite. Et Abijam, son fils, régna à sa place.
\Chap{15}
\TextTitle{Règne d'Abijam (ou Abija) sur Juda\FTNTT{2 Ch. 13:1-2}}
\VerseOne{}La dix-huitième année du roi Jéroboam, fils de Nebath, Abijam commença à régner sur Juda.
\VS{2}Il régna trois ans à Jérusalem. Sa mère s’appelait Maaca et était fille d'Abisalom.
\VS{3}Il marcha dans tous les péchés que son père avait commis avant lui ; son cœur ne fut point intègre envers Yahweh, son Dieu, comme l'avait été le cœur de David, son père.
\VS{4}Mais pour l'amour de David, Yahweh, son Dieu, lui donna une lampe dans Jérusalem, lui suscitant son fils après lui et laissant subsister Jérusalem ;
\VS{5}Parce que David avait fait ce qui est droit devant Yahweh, et que pendant toute sa vie il ne s'était point détourné d’aucun de ses commandements, hormis dans l'affaire d'Urie, le Héthien.
\VS{6}Or, il y eut toujours guerre entre Roboam et Jéroboam, pendant toute la vie de Roboam.
\VS{7}Le reste des actions d'Abijam, et même tout ce qu'il fit, n'est-il pas écrit au livre des Chroniques des rois de Juda ? Il y eut aussi guerre entre Abijam et Jéroboam.
\VS{8}Ainsi Abijam s'endormit avec ses pères, et on l'enterra dans la cité de David. Et Asa, son fils, régna à sa place.
\TextTitle{Règne d’Asa sur Juda\FTNTT{2 Ch. 14:1-5 ; 15:1-19}}
\VS{9}La vingtième année de Jéroboam, roi d'Israël, Asa commença à régner sur Juda.
\VS{10}Il régna quarante et un ans à Jérusalem. Sa mère avait pour nom Maaca, elle était fille d'Abisalom.
\VS{11}Asa fit ce qui est droit devant Yahweh, comme David, son père.
\VS{12}Il ôta du pays les prostitués, et ôta toutes les idoles que ses pères avaient faites.
\VS{13}Et même il ôta la dignité de reine à sa mère Maaca, parce qu'elle avait fait une idole pour Astarté. Asa mit en pièces l’idole qu'elle avait faite, et la brûla au torrent de Cédron.
\VS{14}Mais les hauts lieux ne furent point ôtés. Néanmoins, le cœur d'Asa fut intègre envers Yahweh pendant toute sa vie.
\TextTitle{Guerre entre Juda et Israël ; Asa s’allie avec la Syrie\FTNTT{1 Ch. 14:6-15 ; 16:1-10}}
\VS{15}Il remit dans la maison de Yahweh les choses qui avaient été consacrées par son père et par lui-même, de l'argent, de l'or et les ustensiles.
\VS{16}Or, il y eut guerre entre Asa et Baescha, roi d'Israël, pendant toute leur vie.
\VS{17}Baescha, roi d'Israël, monta contre Juda, et bâtit Rama, pour empêcher quiconque de sortir et entrer vers Asa, roi de Juda.
\VS{18}Asa prit tout l'argent et l'or qui était resté dans les trésors de Yahweh et dans les trésors de la maison royale, et les donna à ses serviteurs ; le roi Asa les envoya vers Ben-Hadad, fils de Thabrimmon, fils de Hezjon, roi de Syrie, qui demeurait à Damas, pour lui dire :
\VS{19}Qu’il y ait alliance entre moi et toi, comme entre mon père et le tien. Voici, je t'envoie un présent en argent et en or. Va, romps l'alliance que tu as avec Baescha, roi d'Israël, afin qu'il se retire de moi.
\VS{20}Et Ben-Hadad écouta le roi Asa ; il envoya les chefs de son armée contre les villes d'Israël, et il battit Ijjon, Dan, Abel-Beth-Maaca, tout Kinneroth, et tout le pays de Nephthali.
\VS{21}Lorsque Baescha l’apprit, il cessa de bâtir Rama et demeura à Thirtsa.
\VS{22}Alors le roi Asa fit publier par tout Juda que tous, sans en excepter aucun, eussent à emporter les pierres et le bois de Rama, que Baescha faisait bâtir, et le roi Asa s’en servit pour bâtir Guéba de Benjamin, et Mitspa.
\TextTitle{Mort d’Asa ; Josaphat règne sur Juda\FTNTT{1 Ch. 16:11-17:1}}
\VS{23}Le reste de toutes les actions d'Asa, tous ses exploits, tout ce qu'il fit, et les villes qu'il a bâties, cela n'est-il pas écrit au livre des Chroniques des rois de Juda ? Au reste, il fut malade de ses pieds au temps de sa vieillesse.
\VS{24}Et Asa s'endormit avec ses pères, avec lesquels il fut enseveli en la cité de David, son père. Et, son fils, Josaphat, régna à sa place.
\TextTitle{Baescha tue Nadab et devient roi d'Israël}
\VS{25}Or, Nadab, fils de Jéroboam, régna sur Israël la seconde année d'Asa, roi de Juda, et il régna deux ans sur Israël.
\VS{26}Il fit ce qui est mal aux yeux de Yahweh ; et il marcha dans la voie de son père, se livrant aux péchés que son père avait fait commettre à Israël.
\VS{27}Et Baescha, fils d'Achija, de la maison d'Issacar, fit une conspiration contre lui. Il le tua devant Guibbethon, qui était aux Philistins, lorsque Nadab et tout Israël assiégeaient Guibbethon.
\VS{28}Baescha le fit donc mourir la troisième année d'Asa, roi de Juda, et il régna à sa place.
\VS{29}Une fois proclamé roi, il frappa toute la maison de Jéroboam et ne laissa échapper aucune âme vivante, il détruisit tout ce qui respirait, selon la parole de Yahweh qu'il avait proférée par son serviteur Achija, Silonite,
\VS{30}A cause des péchés que Jéroboam avait commis et fait commettre à Israël, irritant ainsi Yahweh, le Dieu d'Israël.
\VS{31}Le reste des faits de Nadab, et même tout ce qu'il a fait, n'est-il pas écrit au livre des Chroniques des rois d'Israël ?
\VS{32}Or, il y eut guerre entre Asa et Baescha, roi d'Israël, pendant toute leur vie.
\VS{33}La troisième année d'Asa, roi de Juda, Baescha, fils d'Achija, commença à régner sur tout Israël à Thirtsa, il régna vingt-quatre ans.
\VS{34}Et il fit ce qui est mal aux yeux de Yahweh, et marcha dans la voie de Jéroboam, en se livrant aux péchés que Jéroboam avait fait commettre à Israël.
\Chap{16}
\TextTitle{Yahweh avertit Baescha avant sa mort}
\VerseOne{}Alors la parole de Yahweh fut adressée à Jéhu, fils de Hanani, contre Baescha, en ces mots :
\VS{2}Je t'ai élevé de la poussière et je t'ai établi chef de mon peuple d'Israël ; malgré cela tu as suivi la voie de Jéroboam et fait pécher mon peuple d'Israël, pour m'irriter par leurs péchés.
\VS{3}Voici, je m'en vais entièrement consumer Baescha et sa maison, et je rendrai ta maison semblable à la maison de Jéroboam, fils de Nebath.
\VS{4}Celui de la maison de Baescha qui mourra dans la ville, les chiens le mangeront, et celui des siens qui mourra aux champs, les oiseaux du ciel le mangeront.
\VS{5}Le reste des faits de Baescha, ce qu'il a fait et ses exploits, n'est-il pas écrit au livre des Chroniques des rois d'Israël ?
\VS{6}Ainsi Baescha s'endormit avec ses pères et fut enseveli à Thirtsa. Ela, son fils, régna à sa place.
\VS{7}La parole de Yahweh fut aussi adressée par le moyen de Jéhu, fils d'Hanani, le prophète, contre Baescha et contre sa maison, tant à cause de tout le mal qu'il avait fait devant Yahweh, en l'irritant par l'œuvre de ses mains et en devenant comme la maison de Jéroboam, que parce qu'il l'avait détruite.
\TextTitle{Ela puis Zimri règnent sur Israël}
\VS{8}La vingt-sixième année d'Asa, roi de Juda, Ela, fils de Baescha, commença à régner sur Israël et il régna deux ans à Thirtsa.
\VS{9}Son serviteur, Zimri, capitaine de la moitié des chars, fit une conspiration contre Ela, lorsqu'il était à Thirtsa, buvant et s'enivrant dans la maison d'Artsa, chef de la maison du roi à Thirtsa.
\VS{10}Alors, Zimri vint, le frappa et le tua, la vingt-septième année d'Asa, roi de Juda, et il régna à sa place.
\VS{11}Dès qu’il fut roi et qu'il fut assis sur son trône, il frappa toute la maison de Baescha, il n'en laissa échapper personne qui lui appartint, ni parent, ni ami.
\VS{12}Ainsi Zimri extermina toute la maison de Baescha, selon la parole que Yahweh avait proférée contre Baescha, par Jéhu, le prophète,
\VS{13}A cause de tous les péchés de Baescha, et des péchés d'Ela, son fils, qu’ils avaient commis et qu’ils avaient fait commettre à Israël, irritant Yahweh, le Dieu d'Israël, par leurs idoles.
\VS{14}Le reste des faits d'Ela, et même tout ce qu'il a fait, n'est-il pas écrit au livre des Chroniques des rois d'Israël ?
\VS{15}La vingt-septième année d'Asa, roi de Juda, Zimri régna sept jours à Thirtsa. Or, le peuple était campé contre Guibbethon qui appartenait aux Philistins.
\VS{16}Et le peuple qui était campé là entendit que l'on disait : Zimri a fait une conspiration, et il a même tué le roi ! En ce même jour, tout Israël établit dans le camp pour roi d’Israël Omri, chef de l'armée d'Israël.
\VS{17}Omri et tout Israël avec lui partirent de Guibbethon, et assiégèrent Thirtsa.
\VS{18}Mais dès que Zimri vit que la ville était prise, il entra au palais de la maison royale et brûla sur lui la maison royale, il mourut ainsi,
\VS{19}A cause des péchés qu’il avait commis, faisant ce qui est mal aux yeux de Yahweh, en suivant la voie de Jéroboam et le péché qu'il avait fait commettre à Israël.
\VS{20}Le reste des actions de Zimri et la conspiration qu'il forma, cela n’est-il pas écrit dans le livre des Chroniques des rois d'Israël ?
\TextTitle{Omri règne sur Israël}
\VS{21}Alors le peuple d'Israël se divisa en deux partis : la moitié du peuple voulait faire roi Thibni, fils de Guinath ; et l'autre moitié suivait Omri.
\VS{22}Mais le peuple qui suivait Omri, fut plus fort que le peuple qui suivait Thibni, fils de Guinath. Thibni mourut et Omri régna.
\VS{23}La trente et unième année d'Asa, roi de Juda, Omri commença à régner sur Israël et il régna douze ans après avoir régné six ans à Thirtsa.
\VS{24}Puis il acheta de Schémer la montagne de Samarie, deux talents d'argent ; il bâtit une ville sur cette montagne et nomma la ville qu'il bâtit, du nom de Schémer, seigneur de la montagne.
\VS{25}Omri fit ce qui est mal aux yeux de Yahweh ; il agit même plus mal que tous ceux qui avaient été avant lui.
\VS{26}Il marcha dans la voie de Jéroboam, fils de Nebath, et se livra aux péchés que Jéroboam avait fait commettre à Israël, irritant Yahweh, le Dieu d'Israël, par leurs idoles.
\VS{27}Le reste des actions d’Omri, tout ce qu'il a fait et ses exploits, cela n’est-il pas écrit au livre des Chroniques des rois d'Israël ?
\VS{28}Ainsi Omri s'endormit avec ses pères et fut enseveli à Samarie. Achab, son fils, régna à sa place.
\TextTitle{Achab règne sur Israël et épouse Jézabel}
\VS{29}Achab, fils d’Omri, régna sur Israël la trente-huitième année d'Asa, roi de Juda. Et Achab, fils d’Omri, régna sur Israël à Samarie vingt-deux ans.
\VS{30}Et Achab, fils d’Omri, fit ce qui est mal aux yeux de Yahweh, plus que tous ceux qui avaient été avant lui.
\VS{31}Et il arriva que, comme si ce lui eût été peu de chose de marcher dans les péchés de Jéroboam, fils de Nebath, il prit pour femme Jézabel, fille d'Ethbaal, roi des Sidoniens, puis il alla servir Baal et se prosterna devant lui.
\VS{32}Il dressa un autel à Baal, dans la maison de Baal, qu'il bâtit à Samarie.
\VS{33}Et Achab fit une idole d’Astarté. De sorte qu'Achab fit plus encore que tous les rois d'Israël qui avaient été avant lui, pour irriter Yahweh, le Dieu d'Israël.
\VS{34}En son temps, Hiel de Béthel bâtit Jéricho ; il en jeta les fondements au prix d’Abiram, son premier-né, et posa ses portes sur Segub, son plus jeune fils, selon la parole que Yahweh avait proférée par le moyen de Josué, fils de Nun.
\Chap{17}
\TextTitle{Elie annonce trois ans de sécheresse\FTNTT{1 R. 17-2 R. 1}}
\VerseOne{}Alors Elie, Thischbite, l’un des habitants de Galaad, dit à Achab : Yahweh, le Dieu d'Israël, en la présence duquel je me tiens, est vivant ! Il n'y aura ces années-ci ni rosée ni pluie, sinon à ma parole.
\TextTitle{Elie au torrent de Kerith}
\VS{2}Puis la parole de Yahweh fut adressée à Elie, en disant :
\VS{3}Va-t'en d'ici et tourne-toi vers l'orient ; cache-toi près du torrent de Kerith, qui est en face du Jourdain.
\VS{4}Tu boiras de l’eau du torrent, et j'ai commandé aux corbeaux de t'y nourrir.
\VS{5}Il partit donc et fit selon la parole de Yahweh, il s'en alla et demeura au torrent de Kerith, vis-à-vis du Jourdain.
\VS{6}Les corbeaux lui apportaient du pain et de la viande le matin, et du pain et de la viande le soir, et il buvait de l’eau du torrent.
\VS{7}Mais il arriva qu'au bout d’un certain temps le torrent tarit, parce qu'il n'y avait point eu de pluie dans le pays.
\TextTitle{Elie chez la veuve de Sarepta}
\VS{8}Alors la parole de Yahweh lui fut adressée, en ces mots :
\VS{9}Lève-toi, va à Sarepta, qui appartient à Sidon, et demeure-là. Voici, j'ai commandé là à une femme veuve de t'y nourrir.
\VS{10}Il se leva donc et s'en alla à Sarepta. Et comme il fut arrivé à l’entrée de la ville, voici, une femme veuve était là, qui ramassait du bois. Et il l'appela et lui dit : Apporte-moi, je te prie, un peu d'eau dans un vase et que je boive.
\VS{11}Elle alla en chercher. Il l’appela de nouveau et dit : Apporte-moi, je te prie, un morceau de pain de ta main.
\VS{12}Mais elle répondit : Yahweh, ton Dieu, est vivant ! Je n'ai rien de cuit, je n'ai qu’une poignée de farine dans un pot et un peu d'huile dans une cruche. Et voici, j'amasse deux morceaux de bois, puis je rentrerai, je l'apprêterai pour moi et pour mon fils, nous le mangerons, après quoi nous mourrons.
\VS{13}Et Elie lui dit : Ne crains point, va, fais comme tu dis. Seulement, fais-moi d’abord avec cela un petit gâteau et tu me l’apporteras, tu en feras ensuite pour toi et pour ton fils.
\VS{14}Car ainsi parle Yahweh, le Dieu d'Israël : La farine qui est dans le pot ne finira point et l'huile qui est dans la cruche ne diminuera point, jusqu'à ce que Yahweh donne de la pluie sur la terre.
\VS{15}Elle s'en alla donc, et fit selon la parole d'Elie. Et elle eut à manger, elle et sa famille, ainsi qu’Elie pendant plusieurs jours.
\VS{16}La farine du pot ne finit point, et l'huile de la cruche ne diminua point, selon la parole que Yahweh avait prononcée par le moyen d'Elie.
\TextTitle{Résurrection du fils de la veuve de Sarepta}
\VS{17}Après ces choses, il arriva que le fils de la femme, maîtresse de la maison, devint malade ; et la maladie fut si forte, qu'il expira.
\VS{18}Et elle dit à Elie : Qu'y a-t-il entre moi et toi, homme de Dieu ? Es-tu venu chez moi pour rappeler le souvenir de mon iniquité, et pour faire mourir mon fils ?
\VS{19}Et il lui dit : Donne-moi ton fils. Et il le prit du sein de cette femme, le porta dans la chambre haute où il demeurait, et le coucha sur son lit.
\VS{20}Puis il cria à Yahweh, et dit : Yahweh, mon Dieu ! Affligeras-tu cette veuve au point de faire mourir son fils, elle qui a m’a reçu comme un hôte ?
\VS{21}Et il s'étendit sur l'enfant par trois fois, et cria à Yahweh, en disant : Yahweh, mon Dieu ! Je te prie que l'âme de cet enfant revienne au-dedans de lui.
\VS{22}Et Yahweh écouta la voix d'Elie, l'âme de l'enfant revint au-dedans de lui, et il fut rendu à la vie.
\VS{23}Elie prit l'enfant, le descendit de la chambre haute dans la maison et le donna à sa mère, en lui disant : Regarde, ton fils est vivant.
\VS{24}Et la femme dit à Elie : Je reconnais maintenant, que tu es un homme de Dieu et que la parole de Yahweh, qui est dans ta bouche, est vérité.
\Chap{18}
\TextTitle{Elie à la rencontre d’Abdias puis d’Achab}
\VerseOne{}Et il arriva, après bien des jours, que la parole de Yahweh fut adressée à Elie, dans la troisième année, en disant : Va, montre-toi à Achab et je ferai tomber de la pluie sur la terre.
\VS{2}Et Elie s'en alla pour se présenter devant Achab. Il y avait alors une grande famine en Samarie.
\VS{3}Achab avait appelé Abdias, chef de sa maison ; or, Abdias craignait beaucoup Yahweh ;
\VS{4}quand Jézabel exterminait les prophètes de Yahweh, Abdias prit cent prophètes et les cacha, cinquante dans une caverne et cinquante dans une autre, et il les y nourrit de pain et d'eau.
\VS{5}Achab dit alors à Abdias : Va par le pays vers toutes les sources d'eaux et vers tous les torrents ; peut-être que nous trouverons de l'herbe, nous garderons ainsi en vie les chevaux et les mulets, et nous n’aurons pas besoin d’abattre du bétail.
\VS{6}Ils se partagèrent donc entre eux le pays pour le parcourir ; Achab allait seul par un chemin et Abdias allait seul par un autre chemin.
\VS{7}Comme Abdias était en chemin, voici, Elie le rencontra. Abdias reconnut Elie, il tomba sur son visage et lui dit : N'es-tu pas mon seigneur Elie ?
\VS{8}Il lui répondit : C'est moi ; va et dis à ton seigneur : Voici Elie !
\VS{9}Et Abdias dit : Quel péché ai-je commis, pour que tu livres ton serviteur entre les mains d'Achab pour me faire mourir ?
\VS{10}Yahweh, ton Dieu, est vivant ! Il n'y a ni nation, ni royaume, où mon seigneur n'ait envoyé pour te chercher ; et quand on répondait que tu n'y étais pas, il faisait jurer aux rois et au peuple que l'on ne t’avait pas trouvé.
\VS{11}Et maintenant tu dis : Va, dis à ton seigneur, voici Elie !
\VS{12}Puis, lorsque je t’aurai quitté, l'Esprit de Yahweh te transportera je ne sais où et j’irai informer Achab qui ne te trouvera pas et qui me tuera. Or, ton serviteur craint Yahweh dès sa jeunesse.
\VS{13}N'a-t-on point dit à mon seigneur ce que je fis quand Jézabel tuait les prophètes de Yahweh, comment j'en cachai cent, cinquante dans une caverne et cinquante dans une autre et les y ai nourris de pain et d'eau ?
\VS{14}Et maintenant tu dis : Va, dis à ton seigneur : Voici Elie ! Il me tuera !
\VS{15}Mais Elie lui répondit : Yahweh des armées, devant lequel je me tiens, est vivant ! Aujourd'hui, je me montrerai à Achab.
\VS{16}Abdias étant allé à la rencontre d'Achab, l’informa de la chose ; puis Achab alla au-devant d'Elie.
\VS{17}Et aussitôt qu'Achab eut vu Elie, il lui dit : Est-ce toi qui jettes le trouble en Israël ?
\VS{18}Et Elie lui répondit : Je n'ai point troublé Israël ; c'est toi et la maison de ton père, puisque vous avez abandonné les commandements de Yahweh et que vous êtes allés après les Baals.
\VS{19}Fais maintenant se rassembler tout Israël auprès de moi, sur le mont Carmel, les quatre cent cinquante prophètes de Baal et les quatre cents prophètes d’Astarté qui mangent à la table de Jézabel.
\TextTitle{Confrontation entre Elie et les prophètes de Baal sur le mont Carmel}
\VS{20}Ainsi Achab envoya des messagers vers tous les fils d'Israël et, il rassembla les prophètes sur le mont Carmel.
\VS{21}Alors Elie s'approcha de tout le peuple et dit : Jusqu'à quand clocherez-vous des deux côtés ? Si Yahweh est Dieu, suivez-le ; mais si Baal est dieu, suivez-le. Et le peuple ne lui répondit pas un seul mot.
\VS{22}Alors Elie dit au peuple : Je suis demeuré seul prophète de Yahweh ; et voici quatre cent cinquante prophètes de Baal.
\VS{23}Que l’on nous donne deux veaux, qu'ils en choisissent l'un pour eux, qu'ils le coupent en pièces et qu'ils le mettent sur du bois ; mais qu'ils n'y mettent point de feu ; et je préparerai l'autre veau, je le mettrai sur du bois, sans y mettre le feu.
\VS{24}Puis invoquez le nom de vos dieux, et moi j'invoquerai le nom de Yahweh ; que le dieu qui répondra par le feu, soit reconnu pour être Dieu. Et tout le peuple répondit et dit : C'est bien !
\VS{25}Et Elie dit aux prophètes de Baal : Choisissez un veau et préparez-le les premiers, car vous êtes en plus grand nombre et invoquez le nom de vos dieux ; mais n'y mettez point de feu.
\VS{26}Ils prirent donc un veau qu'on leur donna, ils l'apprêtèrent et ils invoquèrent le nom de Baal depuis le matin jusqu'à midi, en disant : Baal exauce-nous ! Mais il n'y avait ni voix ni réponse et ils sautaient devant l'autel qu'ils avaient fait.
\VS{27}A midi, Elie se moqua d'eux et dit : Criez à haute voix, puisqu’il est dieu ; mais il pense à quelque chose, ou il est occupé, ou il est en voyage ; peut-être qu'il dort et il se réveillera.
\VS{28}Ils criaient donc à haute voix ; ils se faisaient des incisions avec des couteaux et des lances, selon leur coutume, en sorte que le sang coulait sur eux.
\VS{29}Lorsque midi fut passé et qu'ils eurent fait les prophètes jusqu'au temps où l’on offre l'oblation, sans qu'il y eût ni voix, ni réponse, ni signe d’attention.
\VS{30}Elie dit alors à tout le peuple : Approchez-vous de moi ! Et tout le peuple s'approcha de lui et il répara l'autel de Yahweh, qui avait été renversé.
\VS{31}Puis Elie prit douze pierres, selon le nombre des tribus des fils de Jacob, auquel la parole de Yahweh avait été adressée, en disant : Israël sera ton nom.
\VS{32}Et il rebâtit de ces pierres l'autel au nom de Yahweh. Puis il fit un fossé de la capacité de deux mesures de semence autour de l'autel.
\VS{33}Il rangea le bois, il coupa le veau en pièces, et il le plaça sur le bois.
\VS{34}Puis il dit : Remplissez quatre cruches d'eau, puis versez-les sur l'holocauste et sur le bois. Puis il dit : Faites-le encore une seconde fois. Et ils le firent une seconde fois. Il dit : Faites-le une troisième fois. Et ils le firent pour la troisième fois ;
\VS{35}de sorte que les eaux allaient à l'entour de l'autel ; et il remplit aussi d’eau le fossé.
\VS{36}Et au moment de la présentation de l’offrande, Elie, le prophète, s'approcha et dit : Ô Yahweh ! Dieu d'Abraham, d'Isaac et d'Israël ! Que l’on sache aujourd'hui que tu es Dieu en Israël et que je suis ton serviteur ; et que j'ai fait toutes ces choses par ta parole !
\VS{37}Réponds-moi, Ô Yahweh ! Réponds-moi, afin que ce peuple connaisse que c’est toi, Yahweh, qui es Dieu et que c'est toi qui ramènes leur cœur.
\VS{38}Alors le feu de Yahweh tomba et consuma l'holocauste, le bois, les pierres et la terre, et il absorba toute l'eau qui était dans le fossé.
\VS{39}Quand tout le peuple vit cela, ils tombèrent sur leur visage et dirent : C'est Yahweh qui est Dieu ! C'est Yahweh qui est Dieu !
\VS{40}Et Elie leur dit : Saisissez les prophètes de Baal et qu'il n'en échappe aucun ! Ils les saisirent. Elie les fit descendre au torrent de Kison, où il les fit égorger là.
\TextTitle{Retour de la pluie selon la parole d’Elie\FTNTT{Ja. 5:17-18}}
\VS{41}Puis Elie dit à Achab : Monte, mange et bois ; car il se fait un bruit qui annonce la pluie.
\VS{42}Ainsi Achab monta pour manger et pour boire tandis qu’Elie monta au sommet du Carmel ; et, se penchant contre terre, il mit son visage entre ses genoux ;
\VS{43}Et il dit à son serviteur : Monte maintenant et regarde vers la mer. Le serviteur monta, il regarda et dit : Il n'y a rien. Elie dit par sept fois : Retournes-y.
\VS{44}A la septième fois, il dit : Voici un petit nuage qui s’élève de la mer et qui est comme la paume de la main d'un homme, laquelle monte de la mer. Elie dit : Monte et dis à Achab : Attelle ton char et descends de peur que la pluie ne t’arrête.
\VS{45}Ici et là, les cieux s'obscurcirent de nuages accompagnés de vent et il y eut une forte pluie. Achab monta sur son char et partit pour Jizreel.
\VS{46}Et la main de Yahweh fut sur Elie, qui se ceignit les reins et courut devant Achab, jusqu'à l'entrée de Jizreel.
\Chap{19}
\TextTitle{Fuite d’Elie devant les menaces de Jézabel}
\VerseOne{}Achab rapporta à Jézabel tout ce qu'Elie avait fait, et comment il avait tué par l'épée tous les prophètes.
\VS{2}Et Jézabel envoya un messager vers Elie, pour lui dire : Que les dieux me traitent dans toute leur rigueur, si demain, à cette heure-ci, je ne fais de ta vie ce que tu as fait de la vie de chacun d'eux !
\VS{3}Elie, voyant cela, se leva et s'en alla pour sauver sa vie. Il arriva à Beer-Schéba, qui appartient à Juda ; et il laissa là son serviteur.
\TextTitle{L’ange de Yahweh fortifie Elie}
\VS{4}Mais lui s'en alla dans le désert où, après une journée de marche, il s'assit sous un genêt et demanda la mort, en disant : C'en est assez, Ô Yahweh ! Prends mon âme, car je ne suis pas meilleur que mes pères.
\VS{5}Puis il se coucha et s'endormit sous un genêt. Voici un ange le toucha et lui dit : Lève-toi, mange.
\VS{6}Et il regarda, et voici à son chevet, un gâteau cuit sur des pierres chauffées et une cruche d'eau. Il mangea et but, puis se recoucha.
\VS{7}Et l'ange de Yahweh vint une seconde fois, le toucha et lui dit : Lève-toi, mange, car le chemin est trop long pour toi.
\TextTitle{Elie à Horeb, visitation et instructions de Yahweh}
\VS{8}Il se leva donc, mangea et but ; puis avec la force que lui donna cette nourriture, il marcha quarante jours et quarante nuits jusqu'à Horeb, la montagne de Dieu.
\VS{9}Et là, il entra dans une caverne et y passa la nuit. Et voici, la parole de Yahweh lui fut adressée en ces mots : Que fais-tu ici, Elie ?
\VS{10}Et il répondit : J'ai déployé mon zèle pour Yahweh, le Dieu des armées, parce que les enfants d'Israël ont abandonné ton alliance, ils ont renversé tes autels, ils ont tué tes prophètes par l'épée ; je suis resté, moi seul et ils me cherchent pour m'ôter la vie.
\VS{11}Yahweh lui dit : Sors et tiens-toi sur la montagne devant Yahweh. Et voici, Yahweh passa. Et devant Yahweh, il y eut un grand vent impétueux qui déchirait les montagnes et brisait les rochers, mais Yahweh n'était point dans ce vent. Après le vent, ce fut un tremblement de terre ; mais Yahweh n'était point dans ce tremblement de terre.
\VS{12}Après le tremblement de terre, un feu ; mais Yahweh n'était pas dans le feu. Et après le feu vint un murmure doux et léger.
\VS{13}Quand Elie l'entendit, il s’enveloppa le visage de son manteau, il sortit et se tint à l'entrée de la caverne. Et voici, une voix lui fit entendre ces paroles : Que fais-tu ici Elie ?
\VS{14}Et il répondit : J'ai déployé mon zèle pour Yahweh, le Dieu des armées, parce que les enfants d'Israël ont abandonné ton alliance, ils ont renversé tes autels, ils ont tué par l'épée tes prophètes ; je suis resté moi seul, et ils cherchent ma vie pour me l'ôter.
\VS{15}Yahweh lui dit : Va, retourne-t'en par ton chemin vers le désert de Damas ; et quand tu seras arrivé, tu oindras Hazaël pour roi de Syrie.
\VS{16}Tu oindras aussi Jéhu, fils de Nimschi, pour roi d’Israël ; et tu oindras Elisée, fils de Schaphath, d'Abel-Mehola, pour prophète à ta place.
\VS{17}Et il arrivera que quiconque échappera de l'épée de Hazaël, Jéhu le fera mourir ; et quiconque échappera de l'épée de Jéhu, Elisée le fera mourir.
\VS{18}Mais je me suis réservé sept mille hommes de reste en Israël, tous ceux qui n'ont point fléchi les genoux devant Baal, et dont la bouche ne l'a point baisé.
\TextTitle{Elisée devient disciple d’Elie}
\VS{19}Elie partit donc de là, et il trouva Elisée, fils de Schaphath, qui labourait. Il y avait douze paires de bœufs devant soi et il était avec la douzième. Quand Elie passa près de lui, il jeta sur lui son manteau.
\VS{20}Elisée laissa ses bœufs et courut après Elie, en disant : Je t’en prie, laisse-moi embrasser mon père et ma mère, et je te suivrai. Elie lui répondit : Va, et reviens ; car pense à ce que je t'ai fait.
\VS{21}Après s’être éloigné d’Elie, il revint prendre une paire de bœufs qu’il offrit en sacrifice ; et avec l'attelage des bœufs, il en fit bouillir la chair, et la donna au peuple ; ils mangèrent ; puis il se leva et suivit Elie. Dès lors, il fut à son service.
\Chap{20}
\TextTitle{Achab monte contre Ben-Hadad}
\VerseOne{}Alors Ben-Hadad, roi de Syrie rassembla toute son armée ; il avait avec lui trente-deux rois, des chevaux et des chars. Puis il monta, assiégea Samarie et il lui fit la guerre.
\VS{2}Il envoya des messagers à Achab, roi d'Israël, dans la ville ;
\VS{3}Et il lui fit dire : Ainsi parle Ben-Hadad : Ton argent et ton or sont à moi, tes femmes aussi et tes beaux enfants sont à moi.
\VS{4}Et le roi d'Israël répondit, et dit : Mon seigneur, je suis à toi, comme tu le dis, avec tout ce que j'ai.
\VS{5}Ensuite les messagers retournèrent, et dirent : Ainsi parle Ben-Hadad : Puisque je t'ai envoyé dire : Donne-moi ton argent et ton or, ta femme et tes enfants ;
\VS{6}A la même heure demain, j'enverrai chez toi mes serviteurs, ils fouilleront ta maison et les maisons de tes serviteurs, et se saisiront de tout ce que tu as de précieux, et ils l'emporteront.
\VS{7}Alors le roi d'Israël appela tous les anciens du pays, et il dit : Sachez et considérez, je vous prie, combien cet homme nous veut du mal ; car il m’a envoyé demander mes femmes, mes enfants, mon argent et mon or, et je ne lui avais rien refusé.
\VS{8}Et tous les anciens et tout le peuple lui dirent : Ne l'écoute point et ne consens pas.
\VS{9}Il répondit donc aux messagers de Ben-Hadad : Dites au roi, mon seigneur : Je ferai tout ce que tu as envoyé demander la première fois à ton serviteur, mais je ne pourrai faire ceci. Les messagers s'en allèrent et lui rapportèrent cette réponse.
\VS{10}Et Ben-Hadad envoya dire à Achab : Que les dieux me traitent dans toute leur rigueur, si la poudre de Samarie suffit pour remplir le creux de la main de tout le peuple qui me suit.
\VS{11}Mais le roi d'Israël répondit, et dit : Dites-lui : Que celui qui revêt une armure ne se glorifie point comme celui qui la dépose.
\VS{12}Lorsque Ben-Hadad entendit cette réponse, il était à boire avec les rois sous les tentes et il dit à ses serviteurs : Rangez-vous en bataille ! Et ils se rangèrent en bataille contre la ville.
\TextTitle{Victoire Achab}
\VS{13}Alors voici, un prophète s’approcha d’Achab, roi d'Israël et lui dit : Ainsi parle Yahweh : N'as-tu pas vu cette grande multitude ? Voilà, je m'en vais la livrer aujourd'hui entre tes mains, et tu sauras que je suis Yahweh.
\VS{14}Et Achab dit : Par qui ? Et il lui répondit : Ainsi parle Yahweh : Ce sera par les serviteurs des chefs des provinces. Et Achab dit : Qui engagera le combat ? Et il lui répondit : Toi.
\VS{15}Alors il passa en revue les serviteurs des chefs des provinces, qui furent deux cent trente-deux ; et après eux, il dénombra tout le peuple de tous les enfants d'Israël qui furent sept mille.
\VS{16}Ils firent une sortie en plein midi, lorsque Ben-Hadad buvait et s'enivrait dans les tentes, lui et les trente-deux rois qui étaient ses auxiliaires.
\VS{17}Les serviteurs des chefs des provinces sortirent les premiers et Ben-Hadad envoya quelques-uns qui le lui rapportèrent en disant : Des hommes sont sortis de Samarie.
\VS{18}Et il dit : Qu’ils soient sortis pour la paix, ou qu'ils soient sortis pour faire la guerre, saisissez-les tous vivants.
\VS{19}Les serviteurs des chefs de province sortirent de la ville puis l'armée qui était après eux.
\VS{20}Chacun d'eux frappa son homme, de sorte que les Syriens s'enfuirent et Israël les poursuivit. Ben-Hadad, roi de Syrie, se sauva sur un cheval, avec des cavaliers.
\VS{21}Et le roi d'Israël sortit et frappa les chevaux et les chars, en sorte qu'il fit éprouver une grande défaite aux Syriens.
\TextTitle{Achab monte de nouveau contre les Syriens}
\VS{22}Alors le prophète s’approcha du roi d'Israël, et lui dit : Va, fortifie-toi ; considère et vois ce que tu auras à faire ; car l’année révolue, le roi de Syrie montera contre toi.
\VS{23}Or, les serviteurs du roi de Syrie lui dirent : Leur dieu est un dieu de montagnes, c'est pourquoi ils ont été plus forts que nous. Mais combattons contre eux dans la plaine, et certainement, nous serons plus forts qu'eux.
\VS{24}Fais donc ceci : Ote chacun de ces rois de sa place, et remplace-les par des chefs ;
\VS{25}Puis lève une armée pareille à celle que tu as perdue, avec autant de chevaux et de chars, puis nous les combattrons dans la plaine et l’on verra si nous ne sommes pas plus forts qu'eux. Il les écouta, et fit ainsi.
\VS{26}L’année suivante, Ben-Hadad dénombra les Syriens et monta à Aphek pour combattre contre Israël.
\VS{27}On fit aussi le dénombrement des enfants d'Israël ; ils reçurent des vivres, et ils marchèrent à la rencontre des Syriens. Les enfants d'Israël campèrent vis-à-vis d'eux ; semblables à deux petits troupeaux de chèvres, tandis que les Syriens remplissaient le pays.
\VS{28}Alors l'homme de Dieu vint, et dit au roi d'Israël : Ainsi parle Yahweh : Parce que les Syriens ont dit : Yahweh est un dieu des montagnes et non un dieu des vallées, je livrerai entre tes mains toute cette grande multitude, et vous saurez que je suis Yahweh.
\VS{29}Sept jours durant ils campèrent vis-à-vis les uns des autres. Le septième jour, ils entrèrent en bataille, et les enfants d'Israël tuèrent en un seul jour cent mille hommes de pied des Syriens.
\VS{30}Le reste s'enfuit à la ville d'Aphek, où la muraille tomba sur vingt-sept mille hommes demeurés de reste. Ben-Hadad s'était réfugié dans la ville où il allait de chambre en chambre.
\TextTitle{Faute d'Achab qui épargne Ben-Hadad}
\VS{31}Ses serviteurs lui dirent : Voici maintenant, nous avons appris que les rois de la maison d'Israël sont des rois miséricordieux ; maintenant donc mettons des sacs sur nos reins et des cordes à nos têtes, sortons vers le roi d'Israël, peut-être qu'il te laissera la vie sauve.
\VS{32}Ils se mirent donc des sacs autour des reins et des cordes autour de leurs têtes. Ils allèrent auprès du roi d'Israël. Ils lui dirent : Ton serviteur Ben-Hadad dit : Laisse-moi la vie ! Achab répondit : Est-il encore vivant ? Il est mon frère.
\VS{33}Ces hommes tirèrent de là un bon augure, ils se hâtèrent de le prendre au mot et ils dirent : Ben-Hadad est-il ton frère ! Et il répondit : Allez, amenez-le. Ben-Hadad vint vers lui, et il le fit monter sur son char.
\VS{34}Et Ben-Hadad lui dit : Je te rendrai les villes que mon père avait prises à ton père ; et tu te feras des rues en Damas comme mon père avait fait en Samarie. Et moi, répondit Achab, je te laisserai aller en faisant alliance. Il traita donc alliance avec lui, et le laissa aller.
\VS{35}Alors un homme d'entre les fils des prophètes dit à son compagnon, sur l’ordre de Yahweh : Frappe-moi, je te prie ! Mais celui-là refusa de le frapper.
\VS{36}Et il lui dit : Parce que tu n'as point obéi à la parole de Yahweh, voilà, quand tu m’auras quitté, un lion te frappera. Quand il se fut séparé de lui, un lion survint et le frappa.
\VS{37}Puis il trouva un autre homme, et lui dit : Frappe-moi, je te prie. Cet homme-là le frappa et il le blessa.
\VS{38}Après cela le prophète s'en alla, et se plaça sur le chemin du roi ; il se déguisa avec un bandeau sur ses yeux.
\VS{39}Lorsque le roi passa, il cria vers lui, et dit : Ton serviteur était allé au milieu de la bataille ; et voici quelqu'un s'étant retiré, m'a amené un homme, en disant : Garde cet homme, s'il vient à s'échapper, ta vie en répondra, ou tu paieras un talent d'argent.
\VS{40}Et pendant que ton serviteur faisait quelques affaires çà et là, cet homme a disparu. Et le roi d'Israël lui répondit : Telle est ta condamnation, tu l’as toi-même prononcée.
\VS{41}Alors le prophète ôta promptement le bandeau de dessus ses yeux et le roi d'Israël reconnut que c'était l’un des prophètes.
\VS{42}Et il dit : Ainsi parle Yahweh : Parce que tu as laissé échapper de tes mains l'homme que j'avais dévoué par la voie de l'interdit, ta vie répondra de sa vie, et ton peuple de son peuple.
\VS{43}Mais le roi d'Israël se retira en sa maison, triste et irrité ; et il arriva en Samarie.
\Chap{21}
\TextTitle{Achab convoite la vigne de Naboth}
\VerseOne{}Après ces choses, voici ce qui arriva. Naboth de Jizreel, avait une vigne à Jizreel, près du palais d'Achab, roi de Samarie.
\VS{2}Achab parla à Naboth et lui dit : Cède-moi ta vigne, afin que j'en fasse un jardin potager, car elle est proche de ma maison et je te donnerai à la place une vigne meilleure ; ou, si cela te semble bon, je te paierai l'argent qu'elle vaut.
\VS{3}Mais Naboth répondit à Achab : Que Yahweh me garde de te donner l'héritage de mes pères !
\VS{4}Et Achab vint en sa maison tout triste et irrité, à cause de cette parole que lui avait dite Naboth de Jizreel, en disant : Je ne te donnerai point l'héritage de mes pères ! Il se coucha sur son lit, détourna son visage, et ne mangea rien.
\TextTitle{Manigance meurtrière de Jézabel}
\VS{5}Alors Jézabel, sa femme, vint auprès de lui, et lui dit : D'où vient que ton esprit est si triste ? Et pourquoi ne manges-tu point ?
\VS{6}Et il lui répondit : J’ai parlé à Naboth de Jizreel, et je lui ai dit : Donne-moi ta vigne pour de l'argent, ou si tu le désires, je te donnerai une autre vigne pour celle-là, mais il m'a dit : Je ne te céderai point ma vigne !
\VS{7}Alors Jézabel, sa femme, lui dit : Est-ce bien toi maintenant qui exerces la royauté sur Israël ? Lève-toi, prends un repas et que ton cœur se réjouisse ; je te ferai avoir la vigne de Naboth de Jizreel.
\VS{8}Et elle écrivit au nom d'Achab des lettres qu’elle scella du sceau du roi, et elle envoya aux anciens et magistrats qui habitaient avec Naboth, dans sa ville.
\VS{9}Voici ce qu’elle écrivit dans ces lettres : Publiez un jeûne et placez Naboth à la tête du peuple.
\VS{10}Mettez face à lui deux méchants hommes et qu'ils témoignent contre lui, en disant : Tu as maudit Dieu et le roi ! Puis vous le mènerez dehors et le lapiderez afin qu'il meure.
\VS{11}Les gens donc de la ville de Naboth, les anciens et les magistrats qui habitaient dans sa ville, agirent comme Jézabel le leur avait dit, et d’après ce qui était écrit dans les lettres qu'elle leur avait envoyées.
\VS{12}Ils publièrent un jeûne et ils placèrent Naboth à la tête du peuple.
\VS{13}Les deux méchants hommes vinrent et se mirent face à lui, et ces méchants hommes déclarèrent contre Naboth en la présence du peuple : Naboth a maudit Dieu et le roi ! Puis ils le menèrent hors de la ville, ils le lapidèrent, et il mourut.
\VS{14}Après cela, ils envoyèrent dire à Jézabel : Naboth a été lapidé, et il est mort.
\VS{15}Lorsque Jézabel apprit que Naboth avait été lapidé et qu'il était mort, elle dit à Achab : Lève-toi, mets-toi en possession de la vigne de Naboth de Jizreel, qu’il avait refusé de te donner pour de l'argent ; car Naboth n'est plus en vie, il est mort.
\VS{16}Ainsi dès qu'Achab eut entendu que Naboth était mort, il se leva pour descendre à la vigne de Jizreel et pour s'en mettre en possession.
\TextTitle{Jugement d’Achab et de Jézabel ; Achab s’humilie devant Dieu}
\VS{17}Alors la parole de Yahweh fut adressée à Elie, le Thischbite, en ces mots :
\VS{18}Lève-toi, descends au-devant d'Achab, roi d'Israël, lorsqu'il sera à Samarie. Le voilà dans la vigne de Naboth, où il est descendu pour en prendre possession.
\VS{19}Et tu lui diras : Ainsi parle Yahweh : N’es-tu pas un meurtrier et un voleur ? Puis tu lui diras : Ainsi parle Yahweh : Comme les chiens ont léché le sang de Naboth, les chiens lécheront aussi ton propre sang.
\VS{20}Et Achab dit à Elie : M'as-tu trouvé mon ennemi ? Mais il lui répondit : Oui, je t'ai trouvé, parce que tu t'es vendu pour faire ce qui est mal aux yeux de Yahweh.
\VS{21}Voici je vais faire venir le malheur sur toi, et je te consumerai, j’exterminerai quiconque appartient à Achab, tant celui qui est esclave, que celui qui est libre en Israël.
\VS{22}Je rendrai ta maison semblable à la maison de Jéroboam, fils de Nebath, et la maison de Baescha, fils d'Achija, parce que tu m'as irrité et fait pécher Israël.
\VS{23}Yahweh parla aussi contre Jézabel en disant : Les chiens mangeront Jézabel près du rempart de Jizreel.
\VS{24}Celui de la maison d’Achab qui mourra dans la ville, les chiens le mangeront, et celui qui mourra aux champs, les oiseaux des cieux le mangeront.
\VS{25}En effet, il n'y en avait point eu de personne comme Achab, qui se soit vendu pour faire ce qui est mal aux yeux de Yahweh, et sa femme Jézabel l’y excitait ;
\VS{26}de sorte qu'il se rendit fort abominable, allant après les idoles, comme l'avaient fait les Amoréens, que Yahweh avait chassés de devant les enfants d'Israël.
\VS{27}Après avoir entendu les paroles d’Elie, Achab déchira ses vêtements, il mit un sac sur son corps, et jeûna. Il se tenait couché avec ce sac, et il marchait lentement.
\VS{28}Et la parole de Yahweh fut adressée à Elie, le Thischbite, en disant :
\VS{29}As-tu vu comment Achab s'est humilié devant moi ? Parce qu'il s'est humilié devant moi, je ne ferai pas venir le malheur pendant sa vie, ce sera aux jours de son fils que je ferai venir le malheur sur sa maison.
\Chap{22}
\TextTitle{Josaphat aide Achab contre les Syriens}
\VerseOne{}Et on resta trois ans sans qu'il y eût guerre entre la Syrie et Israël.
\VS{2}Puis il arriva, dans la troisième année, que Josaphat, roi de Juda, descendit vers le roi d'Israël.
\VS{3}Le roi d'Israël dit à ses serviteurs : Ne savez-vous pas que Ramoth de Galaad nous appartient ? Et nous ne nous inquiétons pas de la reprendre des mains du roi de Syrie !
\VS{4}Puis il dit à Josaphat : Viendras-tu avec moi à la guerre contre Ramoth de Galaad ? Et Josaphat répondit au roi d'Israël : Nous irons, moi comme toi, mon peuple comme ton peuple, et mes chevaux comme tes chevaux.
\TextTitle{Les prophètes de mensonge\FTNTT{2 Ch. 18:4-5, 9-11}}
\VS{5}Josaphat dit encore au roi d'Israël : Consulte aujourd'hui, je te prie, la parole de Yahweh.
\VS{6}Et le roi d'Israël assembla les prophètes, au nombre de quatre cents environ, auxquels il dit : Irai-je à la guerre contre Ramoth de Galaad, ou dois-je y renoncer ? Et ils répondirent : Monte, car le Seigneur la livrera entre les mains du roi.
\VS{7}Mais Josaphat dit : N'y a-t-il point ici encore quelque prophète de Yahweh, afin que nous le consultions ?
\VS{8}Et le roi d'Israël dit à Josaphat : Il y a encore un homme par qui l’on puisse consulter Yahweh, mais je le hais, car il ne prophétise rien de bon, mais seulement du mal, c'est Michée, fils de Jimla. Josaphat dit : Que le roi ne parle point ainsi !
\VS{9}Alors le roi d'Israël appela un eunuque auquel il dit : Fais venir promptement Michée, fils de Jimla.
\VS{10}Or, le roi d'Israël et Josaphat, roi de Juda, étaient assis chacun sur son trône, revêtus de leurs habits, dans la place, vers l'entrée de la porte de Samarie ; et tous les prophètes prophétisaient en leur présence.
\VS{11}Sédécias, fils de Kenaana, s'était fait des cornes de fer et il dit : Ainsi parle Yahweh : De ces cornes-ci tu heurteras les Syriens, jusqu'à les détruire.
\VS{12}Et tous les prophètes prophétisaient de même, en disant : Monte à Ramoth de Galaad et tu réussiras ; et Yahweh la livrera entre les mains du roi.
\TextTitle{Michée annonce la défaite et la mort d'Achab\FTNTT{2 Ch. 18:6-8, 12-27, 28-34}}
\VS{13}Le messager qui était allé appeler Michée, lui parla ainsi : Voici, les prophètes parlent d'un commun accord au sujet du roi ; je te prie que ta parole soit semblable à celle de chacun d’eux ! Annonce du bien !
\VS{14}Mais Michée lui répondit : Yahweh est vivant ! J’annoncerai ce que Yahweh me dira.
\VS{15}Il vint donc vers le roi, et le roi lui dit : Michée, irons-nous à la guerre contre Ramoth de Galaad, ou devons-nous y renoncer ? Et il lui dit : Monte et tu réussiras, et Yahweh la livrera entre les mains du roi.
\VS{16}Et le roi lui dit : Jusqu'à combien de fois te conjurerai-je de ne me dire que la vérité au nom de Yahweh ?
\VS{17}Et il répondit : J'ai vu tout Israël dispersé par les montagnes, comme un troupeau de brebis qui n'a point de berger ; et Yahweh a dit : Ces gens n’ont point de maître, que chacun retourne en paix dans sa maison !
\VS{18}Alors le roi d'Israël dit à Josaphat : Ne t'ai-je pas bien dit que quand il est question de moi il ne prophétise rien de bon, mais seulement du mal ?
\VS{19}Et Michée lui dit : Ecoute néanmoins la parole de Yahweh ! J'ai vu Yahweh assis sur son trône, et toute l'armée des cieux se tenant devant lui, à sa droite et à sa gauche.
\VS{20}Et Yahweh a dit : Quel est celui qui séduira Achab, afin qu'il monte et qu'il périsse en Ramoth de Galaad ? Et ils répondaient, l'un parlait d'une manière et l'autre d'une autre.
\VS{21}Alors un esprit s'avança et se tint devant Yahweh, il déclara : Je le séduirai. Et Yahweh lui dit : Comment ?
\VS{22}Et il répondit : Je sortirai et je serai un esprit de mensonge dans la bouche de tous ses prophètes. Et Yahweh dit : Tu le séduiras et même tu en viendras à bout ; sors et fais ainsi !
\VS{23}Et maintenant, voici, Yahweh a mis un esprit de mensonge dans la bouche de tous tes prophètes que voilà et Yahweh a prononcé du mal contre toi.
\VS{24}Alors Sédécias, fils de Kenaana, s'approcha et frappa Michée sur la joue et dit : Par où l'Esprit de Yahweh est-il sorti de moi pour s'adresser à toi ?
\VS{25}Et Michée répondit : Voici, tu le verras le jour où tu iras de chambre en chambre pour te cacher.
\VS{26}Alors le roi d'Israël dit : Qu'on prenne Michée et qu'on le mène vers Amon, capitaine de la ville et vers Joas, le fils du roi.
\VS{27}Et tu diras : Ainsi a parlé le roi : Mettez cet homme en prison, nourrissez-le de pain et de l’eau d’affliction, jusqu'à ce que je revienne en paix.
\VS{28}Et Michée répondit : Si tu reviens en paix, Yahweh n'a point parlé par moi. Il dit aussi : Vous tous, peuples, entendez !
\VS{29}Le roi d'Israël monta avec Josaphat, roi de Juda, contre Ramoth de Galaad.
\VS{30}Et le roi d'Israël dit à Josaphat : Que je me déguise et que j'aille à la bataille ; mais toi, revêts-toi de tes habits. Le roi d'Israël donc se déguisa et alla au combat.
\VS{31}Or, le roi de Syrie avait donné un ordre aux trente-deux chefs de ses chars, en disant : Vous n’attaquerez ni petits ni grands, mais seulement contre le roi d'Israël.
\VS{32}Quand les chefs des chars aperçurent Josaphat, ils dirent : C'est certainement le roi d'Israël. Et ils s’approchèrent de lui pour le combattre, mais Josaphat s'écria.
\VS{33}Et quand les chefs des chars virent que ce n'était pas le roi d'Israël, ils se détournèrent de lui.
\TextTitle{Mort d’Achab}
\VS{34}Alors un homme tira de son arc au hasard, et frappa le roi d'Israël entre les jointures de la cuirasse. Et le roi dit à son conducteur de char : Tourne et fais-moi sortir du champ de bataille, car je suis blessé.
\VS{35}Or, le combat devint acharné ce jour-là. Le roi d'Israël fut arrêté dans son char en face des Syriens et il mourut sur le soir. Le sang de sa blessure coulait à l’intérieur du char.
\VS{36}Au coucher du soleil, on cria par tout le camp, en disant : Que chacun se retire en sa ville et chacun en son pays !
\VS{37}Ainsi mourut le roi, qui fut ramené à Samarie ; et l’on enterra le roi à Samarie.
\VS{38}Lorsqu’on lava le char à l’étang de Samarie, les chiens léchèrent le sang d’Achab, et les prostituées s’y baignèrent, selon la parole que Yahweh avait prononcée.
\VS{39}Le reste des actions d'Achab, tout ce qu’il a fait, la maison d'ivoire qu'il construisit et toutes les villes qu'il a bâties, toutes ces choses ne sont-elles pas écrites au livre des Chroniques des rois d'Israël ?
\TextTitle{Règne de Josaphat sur Juda\FTNTT{2 Ch. 17:19-20}}
\VS{40}Ainsi Achab se coucha avec ses pères. Et Achazia, son fils, régna à sa place.
\VS{41}Josaphat, fils d'Asa, régna sur Juda, la quatrième année d'Achab, roi d'Israël.
\VS{42}Josaphat avait trente-cinq ans lorsqu’il devint roi, et il régna vingt-cinq ans à Jérusalem. Sa mère s’appelait Azuba, fille de Schilchi.
\VS{43}Il suivit entièrement la voie d'Asa, son père, et ne s'en détourna point, faisant tout ce qui est droit aux yeux de Yahweh.
\VS{44}Toutefois les hauts lieux ne disparurent pas ; le peuple offrait encore des sacrifices et offrait encore des parfums sur les hauts lieux.
\VS{45}Josaphat fit aussi la paix avec le roi d'Israël.
\VS{46}Le reste des actions de Josaphat, ses exploits et les guerres qu'il mena ne sont-elles pas écrites au livre des Chroniques des rois de Juda ?
\VS{47}Il extermina du pays le reste des prostitués, qui étaient demeurés là depuis le temps d'Asa, son père.
\VS{48}Il n'y avait point alors de roi en Edom : c’était un intendant qui gouvernait.
\VS{49}Josaphat construisit des navires de Tarsis pour aller chercher de l'or à Ophir ; mais il n'y alla point, parce que les navires se brisèrent à Etsjon-Guéber.
\VS{50}Alors Achazia, fils d'Achab, dit à Josaphat : Que mes serviteurs aillent sur les navires avec les tiens, mais Josaphat ne le voulut point.
\TextTitle{Joram règne sur Juda\FTNTT{2 Ch. 21:1}}
\VS{51}Et Josaphat s’endormit avec ses pères et fut enterré avec eux en la cité de David, son père. Et Joram, son fils, régna à sa place.
\TextTitle{Achazia règne sur Israël}
\VS{52}Achazia, fils d'Achab, régna sur Israël à Samarie, la dix-septième année de Josaphat, roi de Juda. Et il régna deux ans sur Israël.
\VS{53}Il fit ce qui est mal aux yeux de Yahweh : il marcha dans la voie de son père, de sa mère et celle de Jéroboam, fils de Nebath, qui avait fait pécher Israël.
\VS{54}Il servit Baal, il se prosterna devant lui et il irrita Yahweh, le Dieu d'Israël, comme l’avait fait son père.
\PPE{}
\end{multicols}

\clearpage\ShortTitle{2 R.}\BookTitle{2 Rois}\BFont
\noindent\hrulefill
{\footnotesize
\textit{
\bigskip
{\centering{}
\\Auteur~: Inconnu
\\(Heb.~: Melakhim)
\\Signification~: Roi, Règne
\\Thème~: Suite de l'histoire d'Israël et de Juda
\\Date de rédaction~: 6\up{ème} siècle av. J.-C.\\}
}
\textit{
\\Le second livre des rois s'articule autour de la vie d'Elisée, serviteur d'Elie, devenu dorénavant son successeur. On y découvre le service prophétique au travers duquel Dieu se révéla comme le Tout-Puissant, le Dieu compatissant, le Maître des temps et des circonstances, le Libérateur, le Dieu de la résurrection, le Puissant Guerrier et aussi le Juge.
\\Ce livre relate l'histoire des derniers rois, la chute d'Israël et sa captivité, la destruction de Jérusalem par Nebudcanetsar, roi de Babylone, en 586 av. J.-C., et la captivité de Juda.\bigskip
}
}
\par\nobreak\noindent\hrulefill
\begin{multicols}{2}
\Chap{1}
\TextTitle{Jugement de Yahweh sur Achazia, roi d'Israël}
\VerseOne{}Or après la mort d'Achab, Moab se révolta contre Israël.
\VS{2}Or Achazia tomba par le treillis de sa chambre haute qui était à Samarie, et il en fut malade. Il envoya des messagers et leur dit~: Allez, consultez Baal-Zebub\FTNT{Baal-Zebub était une divinité des Philistins adorée à Ekron qui se nommait aussi Béelzébul (Mt. 10:25).}, dieu d'Ekron, pour savoir si je guérirai de cette maladie.
\VS{3}Mais l'Ange de Yahweh dit à Elie\FTNT{Elie~: Voir 1 R. 17.}, le Thischbite~: Lève-toi, monte à la rencontre des messagers du roi de Samarie, et dis-leur~: N'y a-t-il point de Dieu en Israël pour que vous alliez consulter Baal-Zebub, dieu d'Ekron~?
\VS{4}C'est pourquoi ainsi parle Yahweh~: Tu ne descendras pas du lit sur lequel tu es monté, mais tu mourras, tu mourras\FTNT{Voir en Gn. 2:16.}. Et Elie s'en alla.
\VS{5}Les messagers retournèrent vers Achazia. Et il leur dit~: Pourquoi revenez-vous~?
\VS{6}Ils lui répondirent~: Un homme est monté à notre rencontre et nous a dit~: Allez, retournez vers le roi qui vous a envoyés et dites-lui~: Ainsi parle Yahweh~: N'y a-t-il point de Dieu en Israël, pour que tu envoies consulter Baal-Zebub, dieu d'Ekron~? A cause de cela, tu ne descendras pas du lit sur lequel tu es monté, mais certainement tu mourras.
\VS{7}Achazia leur dit~: Comment était cet homme qui est monté à votre rencontre et qui vous a dit ces paroles~?
\VS{8}Ils lui répondirent~: C'était un homme vêtu de poil, ayant une ceinture de cuir, ceinte sur ses reins. Et Achazia dit~: C'est Elie, le Thischbite.
\TextTitle{Affirmation de l'autorité d'Elie}
\VS{9}Alors il envoya vers lui un chef de cinquante avec ses cinquante hommes. Ce chef monta auprès d'Elie, qui demeurait au sommet d'une montagne, et il lui dit~: Homme de Dieu, le roi a dit~: Descends~!
\VS{10}Mais Elie répondit et dit au chef de cinquante~: Si je suis un homme de Dieu, que le feu descende du ciel et te consume, toi et tes cinquante hommes~! Et le feu descendit du ciel et le consuma, lui et ses cinquante hommes.
\VS{11}Achazia envoya encore un autre chef de cinquante avec ses cinquante hommes. Ce chef prit la parole et dit à Elie~: Homme de Dieu, ainsi parle le roi~: Hâte-toi de descendre~!
\VS{12}Mais Elie répondit, et leur dit~: Si je suis un homme de Dieu, que le feu descende du ciel et te consume, toi et tes cinquante hommes~! Et le feu de Dieu descendit du ciel et le consuma, lui et ses cinquante hommes.
\VS{13}Achazia envoya encore un troisième chef de cinquante avec ses cinquante hommes. Ce troisième chef de cinquante hommes monta, et vint se mettre à genoux devant Elie, le suppliant, en disant~: Homme de Dieu, je te prie, que ma vie et la vie de ces cinquante hommes, tes serviteurs, soit précieuse à tes yeux~!
\VS{14}Voici, le feu est descendu du ciel et a consumé les deux premiers chefs de cinquante, avec leurs cinquante hommes~; mais maintenant, je te prie, que ma vie soit précieuse à tes yeux~!
\VS{15}Et l'Ange de Yahweh dit à Elie~: Descends avec lui, n'aie pas peur de lui. Elie se leva donc et descendit avec lui vers le roi.
\VS{16}Il lui dit~: Ainsi parle Yahweh~: Parce que tu as envoyé des messagers pour consulter Baal-Zebub, dieu d'Ekron, comme s'il n'y avait point de Dieu en Israël, pour consulter sa parole, tu ne descendras pas du lit sur lequel tu es monté, mais certainement tu mourras.
\TextTitle{Mort d'Achazia~; Joram règne sur Israël}
\VS{17}Achazia mourut, selon la parole de Yahweh prononcée par Elie. Et Joram régna à sa place, la seconde année de Joram, fils de Josaphat, roi de Juda, parce qu'Achazia n'avait point de fils.
\VS{18}Le reste des actions d'Achazia et ce qu'il a fait, cela n'est-il pas écrit dans le livre des Chroniques des rois d'Israël~?
\Chap{2}
\TextTitle{Enlèvement d'Elie au ciel}
\VerseOne{}Or il arriva lorsque Yahweh enleva Elie au ciel dans un tourbillon, Elie et Elisée partaient de Guilgal.
\VS{2}Elie dit à Elisée~: Je te prie, reste ici, car Yahweh m'envoie jusqu'à Béthel. Mais Elisée répondit Yahweh est vivant et ton âme est vivante~! Je ne te quitterai pas~! Ainsi ils descendirent à Béthel.
\VS{3}Les fils des prophètes qui étaient à Béthel sortirent vers Elisée, et lui dirent~: Ne sais-tu pas qu'aujourd'hui Yahweh va enlever ton maître au-dessus de ta tête~? Et il répondit~: Je le sais aussi~; taisez-vous~!
\VS{4}Elie lui dit~: Elisée, je te prie, reste ici, car Yahweh m'envoie à Jéricho. Mais Elisée lui répondit~: Yahweh est vivant et ton âme est vivante~! Je ne te quitterai pas~! Ainsi, ils arrivèrent à Jéricho.
\VS{5}Les fils des prophètes qui étaient à Jéricho s'approchèrent d'Elisée, et lui dirent~: Ne sais-tu pas qu'aujourd'hui Yahweh va enlever ton maître au-dessus de ta tête~? Et il répondit~: Je le sais aussi~; taisez-vous~!
\VS{6}Elie lui dit~: Elisée, je te prie demeure ici, car Yahweh m'envoie jusqu'au Jourdain. Mais Elisée répondit~: Yahweh est vivant et ton âme est vivante~! Je ne te quitterai pas~! Ainsi, ils s'en allèrent tous les deux.
\VS{7}Cinquante hommes d'entre les fils des prophètes arrivèrent et s'arrêtèrent à distance vis-à-vis d'eux, et eux deux s'arrêtèrent au bord du Jourdain.
\VS{8}Alors Elie prit son manteau, le roula et en frappa les eaux, qui se divisèrent çà et là, et ils passèrent tous deux à sec.
\VS{9}Quand ils furent passés, Elie dit à Elisée~: Demande ce que tu veux que je fasse pour toi, avant que je sois enlevé d'avec toi. Elisée répondit~: Je te prie, que j'aie, une double portion\FTNT{Le fils aîné recevait une double portion par rapport aux autres fils (De. 21:15-17).} de ton esprit~!
\VS{10}Elie lui dit~: Tu demandes une chose difficile. Mais si tu me vois pendant que je serai enlevé d'avec toi, cela te sera accordé~; mais si tu ne me vois pas, cela ne te sera pas accordé.
\VS{11}Comme ils continuaient à marcher en parlant, voici, un char de feu et des chevaux de feu les séparèrent l'un de l'autre, et Elie monta au ciel dans un tourbillon.
\TextTitle{La double portion de l'esprit d'Elie sur Elisée}
\VS{12}Elisée le regardait et criait~: Mon père~! Mon père~! Char d'Israël et sa cavalerie~! Et il ne le vit plus. Puis saisissant ses vêtements, il les déchira en deux morceaux.
\VS{13}Il releva le manteau qu'Elie avait laissé tomber. Puis il retourna et s'arrêta sur le bord du Jourdain.
\VS{14}Ensuite il prit le manteau qu'Elie avait laissé tomber et il en frappa les eaux, et dit~: Où est Yahweh, le Dieu d'Elie, Yahweh lui-même~? Lui aussi frappa les eaux qui se divisèrent en deux~; et Elisée passa.
\TextTitle{Le service d'Elisée est reconnu par les hommes}
\VS{15}Quand les fils des prophètes qui étaient à Jéricho, vis-à-vis, l'eurent vu, ils dirent~: L'esprit d'Elie repose sur Elisée~! Ils vinrent à sa rencontre et se prosternèrent contre terre devant lui.
\VS{16}Ils lui dirent~: Voici, il y a parmi tes serviteurs cinquante hommes vaillants~; veux-tu qu'ils aillent chercher ton maître, de peur que l'Esprit de Yahweh ne l'ait enlevé et ne l'ait jeté sur quelque montagne ou dans quelque vallée~? Elisée répondit~: Ne les envoyez pas.
\VS{17}Mais ils le pressèrent tant par leurs paroles, qu'il en était embarrassé. Il leur dit donc~: Envoyez-les. Ils envoyèrent cinquante hommes, qui pendant trois jours cherchèrent Elie, mais ils ne le trouvèrent point.
\VS{18}Puis ils retournèrent vers Elisée, qui était à Jéricho, et il leur dit~: Ne vous avais-je pas dit~: N'y allez pas~?
\VS{19}Les gens de la ville dirent à Elisée~: Voici, le séjour dans cette ville est bon, comme mon seigneur le voit~; mais les eaux sont mauvaises et le pays est stérile.
\VS{20}Il dit~: Apportez-moi un vase neuf et mettez-y du sel. Et ils le lui apportèrent.
\VS{21}Puis il alla vers la source des eaux, et il y jeta le sel, et dit~: Ainsi parle Yahweh~: J'assainis ces eaux~; elles ne causeront plus ni mort ni stérilité.
\VS{22}Les eaux furent assainies, jusqu'à ce jour, selon la parole qu'Elisée avait prononcée.
\TextTitle{Jugement des moqueurs}
\VS{23}Elisée monta de là à Béthel~; et comme il montait par le chemin, des petits garçons sortirent de la ville et se moquèrent de lui. Ils lui disaient~: Monte chauve~! Monte chauve~!
\VS{24}Il se retourna pour les regarder, et il les maudit au nom de Yahweh. Alors deux ours sortirent de la forêt et déchirèrent quarante-deux de ces enfants.
\VS{25}De là il alla sur la montagne de Carmel, d'où il retourna à Samarie.
\Chap{3}
\TextTitle{Joram règne sur Israël}
\VerseOne{}La dix-huitième année de Josaphat, roi de Juda, Joram, fils d'Achab, régna sur Israël à Samarie. Il régna douze ans.
\VS{2}Il fit ce qui est mal aux yeux de Yahweh, non pas toutefois comme son père et sa mère, car il ôta la statue de Baal que son père avait faite~;
\VS{3}mais il s'attacha aux péchés de Jéroboam, fils de Nebath, qui avait fait pécher Israël et il ne s'en détourna point.
\TextTitle{Rébellion de Moab~; Israël et Juda s'allient pour combattre}
\VS{4}Or Méscha, roi de Moab, possédait des troupeaux, et il payait au roi d'Israël un tribut de cent mille agneaux et cent mille béliers avec leur laine.
\VS{5}Mais aussitôt qu'Achab mourut, le roi de Moab se révolta contre le roi d'Israël.
\VS{6}C'est pourquoi le roi Joram sortit ce jour-là de Samarie et passa en revue tout Israël.
\VS{7}Il se mit en marche et fit dire à Josaphat, roi de Juda~: Le roi de Moab s'est rebellé contre moi~; veux-tu venir avec moi faire la guerre à Moab~? Josaphat répondit~: Je monterai, moi comme toi, mon peuple comme ton peuple, mes chevaux comme tes chevaux.
\VS{8}Ensuite il dit~: Par quel chemin monterons-nous~? Joram répondit~: Par le chemin du désert d'Edom.
\TextTitle{Les rois d'Israël, de Juda et d'Edom en marche~; ils consultent Elisée}
\VS{9}Ainsi, le roi d'Israël, le roi de Juda et le roi d'Edom, partirent~; ils firent un détour, et après une marche de sept jours, ils manquèrent d'eau pour l'armée et pour les bêtes qui la suivaient.
\VS{10}Alors le roi d'Israël dit~: Hélas~! Yahweh a appelé ces trois rois pour les livrer entre les mains de Moab.
\VS{11}Et Josaphat dit~: N'y a-t-il ici aucun prophète de Yahweh, par qui nous puissions consulter Yahweh~? Et un des serviteurs du roi d'Israël répondit, et dit~: Il y a ici Elisée, fils de Schaphath, qui versait de l'eau sur les mains d'Elie.
\VS{12}Alors Josaphat dit~: La parole de Yahweh est avec lui. Le roi d'Israël, Josaphat et le roi d'Edom descendirent vers lui.
\VS{13}Mais Elisée dit au roi d'Israël~: Qu'y a-t-il entre moi et toi~? Va-t'en vers les prophètes de ton père et vers les prophètes de ta mère. Et le roi d'Israël lui répondit~: Non~! Car Yahweh a appelé ces trois rois pour les livrer entre les mains de Moab.
\VS{14}Elisée dit~: Yahweh des armées, devant lequel je me tiens, est vivant~! Si je n'avais de la considération pour Josaphat, roi de Juda, je ne ferais aucune attention à toi et je ne te regarderais même pas.
\VS{15}Mais maintenant, amenez-moi un joueur d'instruments à cordes. Et comme le joueur jouait des instruments à cordes, la main de Yahweh fut sur Elisée.
\TextTitle{Prophétie sur la défaite de Moab}
\VS{16}Et il dit~: Ainsi parle Yahweh~: Faites des tranchées dans toute cette vallée.
\VS{17}Car ainsi parle Yahweh~: Vous ne verrez ni vent, ni pluie, et néanmoins cette vallée sera remplie d'eaux, et vous boirez, vous et vos bêtes.
\VS{18}Mais cela est peu de chose aux yeux de Yahweh. Il livrera Moab entre vos mains~;
\VS{19}Vous frapperez toutes les villes fortes et toutes les villes d'élite, vous abattrez tous les bons arbres, vous boucherez toutes les sources d'eau et vous ruinerez avec des pierres tous les meilleurs champs.
\VS{20}Il arriva donc au matin, environ à l'heure de l'offrande, que l'eau arriva du chemin d'Edom, en sorte que ce pays fut rempli d'eau.
\VS{21}Cependant, tous les Moabites ayant appris que ces rois étaient montés pour leur faire la guerre s'étaient assemblés. On convoqua tous ceux qui étaient en âge de porter les armes, et même au-dessus, et ils se tinrent sur la frontière.
\VS{22}Et le lendemain, ils se levèrent de bon matin, et comme le soleil se levait sur les eaux, les Moabites virent en face d'eux les eaux rouges comme du sang.
\VS{23}Ils dirent~: C'est du sang~! Certainement, ces rois-là se sont entretués, et chacun a frappé son compagnon~; maintenant, Moabites, au butin~!
\VS{24}Ainsi ils marchèrent contre le camp d'Israël. Mais Israël se leva et frappa Moab, qui prit la fuite devant eux. Puis ils pénétrèrent dans le pays et frappèrent Moab.
\VS{25}Ils détruisirent les villes, et chacun jetait des pierres dans les meilleurs champs, de sorte qu'ils les en remplirent, ils bouchèrent toutes les sources d'eaux et abattirent tous les bons arbres~; et les frondeurs entourèrent et frappèrent Kir-Haréseth, dont on ne laissa que les pierres.
\VS{26}Le roi de Moab, voyant qu'il n'était pas le plus fort dans la bataille, prit avec lui sept cents hommes tirant l'épée pour se frayer un passage jusqu'au roi d'Edom~; mais ils ne purent pas.
\VS{27}Alors il prit son fils premier-né, qui devait régner à sa place, et l'offrit en holocauste sur la muraille. Et il y eut une grande indignation en Israël~; ainsi ils se retirèrent du roi de Moab et retournèrent dans leur pays.
\Chap{4}
\TextTitle{Miracle~: Le vase d'huile de la veuve}
\VerseOne{}Or une femme d'un des fils des prophètes cria à Elisée, en disant~: Ton serviteur mon mari est mort, et tu sais que ton serviteur craignait Yahweh~; or son créancier est venu pour prendre mes deux enfants, afin qu'ils soient ses esclaves.
\VS{2}Elisée lui répondit~: Que puis-je faire pour toi~? Dis-moi ce que tu as à la maison. Et elle dit~: Ta servante n'a rien dans toute la maison qu'un vase d'huile.
\VS{3}Alors il lui dit~: Va, demande des vases dans la rue à tous tes voisins, des vases vides, et n'en demande pas un petit nombre.
\VS{4}Puis rentre et ferme la porte sur toi et sur tes enfants, et verse dans tous ces vases, et tu mettras de côté ceux qui seront pleins.
\VS{5}Alors elle le quitta. Ayant fermé la porte sur elle et sur ses enfants~; ils lui présentaient les vases, et elle versait.
\VS{6}Lorsqu'elle eut rempli les vases, elle dit à son fils~: Présente-moi encore un vase. Mais il répondit~: Il n'y a plus de vase. Et l'huile s'arrêta.
\VS{7}Elle alla le raconter à l'homme de Dieu, qui lui dit~: Va, vends l'huile, et paye ta dette~; et vous vivrez, toi et tes fils, de ce qui restera.
\TextTitle{Yahweh se souvient de la Sunamite}
\VS{8}Et il arriva un jour qu'Elisée passait par Sunem, où il y avait une femme importante~; elle le retint avec grande instance à manger du pain chez elle. Et toutes les fois qu'il passait, il s'y retirait pour manger du pain.
\VS{9}Elle dit à son mari~: Voilà, je sais que cet homme qui passe souvent chez nous est un saint homme de Dieu.
\VS{10}Faisons-lui, je te prie, une petite chambre haute avec des murs, et mettons-y pour lui un lit, une table, un siège et un chandelier, afin que quand il viendra chez nous, il s'y retire.
\VS{11}Un jour, Elisée étant revenu à Sunem, il se retira dans cette chambre haute et s'y coucha.
\VS{12}Puis il dit à Guéhazi, son serviteur~: Appelle cette Sunamite. Guéhazi l'appela, et elle se présenta devant lui.
\VS{13}Et Elisée dit à Guéhazi~: Dis maintenant à cette femme~: Voici, tu nous as montré tout cet empressement~; que pourrait-on faire pour toi~? Faut-il parler pour toi au roi ou au chef de l'armée~? Elle répondit~: J'habite au milieu de mon peuple.
\VS{14}Et il dit~: Que faudrait-il faire pour elle~? Guéhazi répondit~: Mais elle n'a point de fils et son mari est vieux.
\VS{15}Et il dit~: Appelle-la. Guéhazi l'appela, et elle se présenta à la porte.
\VS{16}Elisée lui dit~: L'année prochaine, à cette même époque, tu embrasseras un fils. Elle répondit~: Mon seigneur, homme de Dieu, ne trompe pas, ne trompe pas ta servante~!
\VS{17}Cette femme devint enceinte et enfanta un fils un an après, à la même époque, comme Elisée lui avait dit.
\TextTitle{Foi de la Sunamite, résurrection de son fils}
\VS{18}L'enfant grandit. Il sortit un jour pour aller trouver son père vers les moissonneurs.
\VS{19}Et il dit à son père~: Ma tête~! Ma tête~! Et le père dit au serviteur~: Porte-le à sa mère.
\VS{20}Il le porta donc et l'amena à sa mère. Et l'enfant resta sur les genoux de sa mère jusqu'à midi, puis il mourut.
\VS{21}Elle monta et le coucha sur le lit de l'homme de Dieu~; et ayant fermé la porte sur lui, elle sortit.
\VS{22}Elle appela son mari et dit~: Je te prie envoie-moi un des serviteurs et une ânesse~; j'irai chez l'homme de Dieu et je reviendrai.
\VS{23}Et il dit~: Pourquoi vas-tu vers lui aujourd'hui~? Ce n'est point la nouvelle lune ni le sabbat. Elle répondit~: Tout va bien~!
\VS{24}Elle fit donc seller l'ânesse, et dit à son serviteur~: Conduis-moi et ne m'arrête pas en route sans que je te le dise.
\VS{25}Ainsi elle s'en alla et se rendit vers l'homme de Dieu sur la montagne de Carmel. L'homme de Dieu, l'ayant aperçue, dit à Guéhazi son serviteur~: Voilà la Sunamite~!
\VS{26}Va, cours à sa rencontre et dis-lui~: Te portes-tu bien~? Ton mari se porte-t-il bien~? L'enfant se porte-t-il bien~? Et elle répondit~: Nous nous portons bien.
\VS{27}Dès qu'elle fut arrivée auprès de l'homme de Dieu sur la montagne, elle embrassa ses pieds. Guéhazi s'approcha pour la repousser, mais l'homme de Dieu lui dit~: Laisse-la, car son âme est dans l'amertume, et Yahweh me l'a caché, et ne me l'a pas révélé.
\VS{28}Alors elle dit~: Ai-je demandé un fils à mon seigneur~? N'ai-je pas dit~: Ne me trompe pas~?
\VS{29}Et Elisée dit à Guéhazi~: Ceins tes reins, prends mon bâton dans ta main, et pars. Si tu rencontres quelqu'un, ne le salue pas~; et si quelqu'un te salue, ne lui réponds pas. Tu mettras mon bâton sur le visage de l'enfant.
\VS{30}Mais la mère de l'enfant dit~: Yahweh est vivant, et ton âme est vivante~! Je ne te quitterai point. Il se leva donc et la suivit.
\VS{31}Or Guéhazi les avait devancés et il avait mis le bâton sur le visage de l'enfant~; mais il n'y eut ni voix ni signe d'attention. Guéhazi retourna à la rencontre d'Elisée et l'en informa en disant~: L'enfant ne s'est pas réveillé.
\VS{32}Lorsqu'Elisée entra dans la maison, l'enfant, mort, était couché sur son lit.
\VS{33}Il ferma la porte sur eux deux et pria Yahweh.
\VS{34}Puis, il monta et se coucha sur l'enfant~; il mit sa bouche sur la bouche de l'enfant, ses yeux sur ses yeux, ses mains sur ses mains, et il s'étendit sur lui. La chair de l'enfant se réchauffa.
\VS{35}Puis il s'éloigna et marcha dans la maison, tantôt dans un lieu, tantôt dans un autre, et il remonta et s'étendit encore sur lui. L'enfant éternua sept fois et ouvrit ses yeux.
\VS{36}Alors, Elisée appela Guéhazi, et lui dit~: Appelle cette Sunamite. Guéhazi l'appela, et elle vint vers Elisée qui lui dit~: Prends ton fils~!
\VS{37}Elle se jeta à ses pieds et se prosterna contre terre. Puis elle prit son fils et sortit.
\TextTitle{Les coloquintes sauvages}
\VS{38}Après cela, Elisée revint à Guilgal. Or il y avait une famine\FTNT{Par le passé, Israël a connu plusieurs famines, dont celle relatée en 2 R. 4:38-41.. Dans ce passage, l'un des fils des prophètes trouva une vigne sauvage dans un champ et y cueillit des coloquintes sauvages. Il les ajouta au potage qui mijotait dans un pot, ne sachant pas que c'était du poison. Le pot est l'image des églises de Laodicée dans lesquelles il y a un mélange mortel de fausses doctrines et de préceptes mondains qui viennent altérer la vérité de la parole de Dieu. Ce mélange impur est absorbé par des millions de personnes ignorantes à travers le monde. Celles-ci se rendent compte qu'elles ont été empoisonnées spirituellement, et une fois le mélange ingéré, elles constatent les effets pervers et dévastateurs souvent tardivement. Le champ tout comme la vigne sauvage, selon Mt. 13:38 et Ro. 11:17, symbolise le monde. Il est par ailleurs intéressant de noter que le mot herbe, «~owrah~» en hébreu, signifie aussi lumière (Ps. 139:12). Cette histoire n'est pas sans nous rappeler le feu étranger introduit par les fils d'Aaron dans le tabernacle, et ce, malgré l'interdiction formelle de Yahweh (Ex. 30:9~; Lé. 10:1-5). C'est exactement ce qui se passe de nos jours. Les églises importent de plus en plus en leur sein la lumière luciférienne du monde (musique, marketing, philosophie, etc.). Beaucoup de pasteurs et de musiciens cherchent malheureusement leur inspiration dans le monde à cause de la famine qui sévit dans les églises. Ce feu étranger représente la plupart des doctrines et pratiques promues par l'église de Laodicée.} dans le pays, et les fils des prophètes étaient assis devant lui~; et il dit à son serviteur~: Mets le grand pot et fais cuire du potage pour les fils des prophètes.
\VS{39}Mais quelqu'un étant sorti dans les champs pour cueillir des herbes, trouva de la vigne sauvage, et cueillit des coloquintes sauvages plein sa robe, et étant revenu, il les coupa en morceaux dans le pot où était le potage, car on ne savait pas ce que c'était.
\VS{40}Et on servit à manger de ce potage à quelques-uns~; mais aussitôt qu'ils eurent mangé de ce potage, ils s'écrièrent et dirent~: Homme de Dieu, la mort est dans le pot~! Et ils ne purent en manger.
\VS{41}Et il dit~: Apportez-moi de la farine~; et il en jeta dans le pot, puis il dit~: Qu'on en verse à ce peuple, afin qu'il mange~; et il n'y avait plus rien de mauvais dans le pot.
\TextTitle{Multiplication de pains}
\VS{42}Un homme venant de Baal-Schalischa apporta à l'homme de Dieu du pain des prémices, à savoir vingt pains d'orge et des épis nouveaux. Elisée dit~: Donne cela à ces gens, et qu'ils mangent.
\VS{43}Son serviteur répondit~: Comment pourrais-je en donner à cent hommes~? Mais Elisée lui répondit~: Donne-les à ces gens, et qu'ils mangent~; car ainsi parle Yahweh~: Ils mangeront et il en restera encore.
\VS{44}Il mit donc les pains devant eux. Ils mangèrent et en eurent de reste, selon la parole de Yahweh.
\Chap{5}
\TextTitle{Guérison miraculeuse de Naaman}
\VerseOne{}Or Naaman, chef de l'armée du roi de Syrie, était un homme puissant et très considéré aux yeux de son maître~; car c'était par lui que Yahweh avait délivré les Syriens. Mais cet homme fort et vaillant était lépreux.
\VS{2}Et les Syriens étaient sortis par troupes, et ils avaient emmené prisonnière une petite fille du pays d'Israël, qui était au service de la femme de Naaman.
\VS{3}Elle dit à sa maîtresse~: Oh~! Si mon seigneur se présentait devant le prophète qui est à Samarie, il le guérirait de sa lèpre~!
\VS{4}Naaman le rapporta à son maître, en disant~: La fille qui est du pays d'Israël a dit telle et telle chose.
\VS{5}Et le roi de Syrie dit à Naaman~: Va, rends-toi à Samarie et j'enverrai une lettre au roi d'Israël. Naaman donc s'en alla et prit avec lui dix talents d'argent et six mille pièces d'or, et dix vêtements de rechange.
\VS{6}Il porta au roi d'Israël la lettre, où il était dit~: Dès que cette lettre te sera parvenue, sache que je t'ai envoyé Naaman, mon serviteur, afin que tu le guérisses de sa lèpre.
\VS{7}Et dès que le roi d'Israël eut lu la lettre, il déchira ses vêtements et dit~: Suis-je Dieu pour faire mourir et pour rendre la vie, pour qu'il s'adresse à moi afin que je guérisse un homme de sa lèpre~? Voyez et comprenez qu'il cherche certainement une occasion de dispute avec moi.
\VS{8}Et il arriva qu'aussitôt qu'Elisée, homme de Dieu, apprit que le roi d'Israël avait déchiré ses vêtements, il envoya dire au roi~: Pourquoi as-tu déchiré tes vêtements~? Laisse-le venir vers moi et il saura qu'il y a un prophète en Israël.
\VS{9}Naaman vint avec ses chevaux et son char, et il s'arrêta à la porte de la maison d'Elisée.
\VS{10}Elisée envoya un messager vers lui, pour lui dire~: Va, et lave-toi sept fois dans le Jourdain, et ta chair redeviendra saine, et tu seras pur.
\VS{11}Mais Naaman se mit dans une grande colère, et s'en alla en disant~: Voilà, je me disais~: Il sortira et viendra vers moi, il se présentera lui-même, il invoquera le Nom de Yahweh, son Dieu, puis, il agitera sa main sur la plaie, et guérira le lépreux.
\VS{12}Les fleuves de Damas, l'Abana et le Parpar ne sont-ils pas meilleurs que toutes les eaux d'Israël~? Ne pourrais-je pas m'y laver et devenir pur~? Ainsi donc, il s'en retourna et s'en alla furieux.
\VS{13}Mais ses serviteurs s'approchèrent et lui parlèrent en disant~: Mon père, si le prophète t'avait imposé quelque chose de difficile, ne l'aurais-tu pas fait~? Combien plus dois-tu faire ce qu'il t'a dit~: Lave-toi, et tu deviendras pur~!
\VS{14}Alors il descendit et se plongea sept fois dans le Jourdain, selon la parole de l'homme de Dieu~; et sa chair redevint comme la chair d'un petit enfant~; et il fut pur.
\VS{15}Il retourna vers l'homme de Dieu, lui et tout son camp, et il vint se présenter devant lui et dit~: Voici, maintenant je sais qu'il n'y a point d'autre Dieu sur toute la terre, si ce n'est en Israël. Maintenant donc, je te prie, accepte ce présent de ton serviteur.
\VS{16}Elisée répondit~: Yahweh, devant lequel je me tiens, est vivant~! Je ne l'accepterai pas~! Naaman le pressa fort de l'accepter, mais Elisée refusa~!
\VS{17}Alors, Naaman dit~: Je te prie, permets que l'on donne de la terre à ton serviteur, une charge de deux mulets~; car ton serviteur ne fera plus d'holocauste ni de sacrifice à d'autres dieux, mais seulement à Yahweh.
\VS{18}Voici toutefois, que Yahweh pardonne ceci à ton serviteur. Quand mon maître entre dans la maison de Rimmon pour s'y prosterner et qu'il s'appuie sur ma main, je me prosterne aussi dans la maison de Rimmon~: Que Yahweh me pardonne, quand je me prosternerai dans la maison de Rimmon.
\TextTitle{Convoitise et mensonge de Guéhazi~; jugement de Dieu}
\VS{19}Elisée lui dit~: Va en paix. Lorsque Naaman eut quitté Elisée et qu'il fut à une certaine distance,
\VS{20}Guéhazi\FTNT{Guéhazi, dont le nom hébreu signifie «~vallée de la vision~».}, le serviteur d'Elisée, homme de Dieu, se dit en lui-même~: Voici, mon maître a ménagé Naaman, ce Syrien, et n'a pas accepté de sa main ce qu'il avait apporté~; Yahweh est vivant~! Je vais courir après lui et j'en obtiendrai quelque chose.
\VS{21}Et Guéhazi courut après Naaman. Naaman, le voyant courir après lui, descendit de son char pour aller à sa rencontre. Il dit~: Tout va bien~?
\VS{22}Guéhazi répondit~: Tout va bien. Mon maître m'envoie te dire~: Voici, il vient d'arriver chez moi deux jeunes hommes de la montagne d'Ephraïm, d'entre les fils des prophètes. Je te prie donne-leur un talent d'argent et deux vêtements de rechange.
\VS{23}Et Naaman dit~: Consens à prendre deux talents. Il insista, puis il serra deux talents d'argent dans deux sacs avec deux vêtements de rechange et les fit porter devant Guéhazi par deux de ses serviteurs.
\VS{24}Et quand il fut arrivé dans un lieu secret, il les prit de leurs mains, et les déposa dans la maison, et il renvoya ces gens qui s'en allèrent.
\VS{25}Puis il entra et se présenta devant son maître. Elisée lui dit~: D'où viens-tu, Guéhazi~? Et il répondit~: Ton serviteur n'est allé nulle part.
\VS{26}Mais Elisée lui dit~: Mon cœur n'est-il pas allé là, lorsque cet homme a quitté son char pour venir à ta rencontre~? Est-ce le temps de prendre de l'argent, de prendre des vêtements, des oliviers, des vignes, du menu et du gros bétail, des serviteurs et des servantes~?
\VS{27}C'est pourquoi la lèpre de Naaman s'attachera à toi et à ta postérité à jamais. Et Guéhazi sortit de la présence d'Elisée avec une lèpre comme de la neige.
\Chap{6}
\TextTitle{Miracle du fer de hache}
\VerseOne{}Les fils des prophètes dirent à Elisée~: Voici, le lieu où nous sommes assis devant toi est trop étroit pour nous.
\VS{2}Allons jusqu'au Jourdain~; nous prendrons là chacun une poutre et nous y ferons un lieu d'habitation. Elisée répondit~: Allez~!
\VS{3}Et l'un d'eux dit~: Veuille, je te prie, venir avec tes serviteurs. Il répondit~: J'irai.
\VS{4}Il partit donc avec eux. Arrivés au Jourdain, ils coupèrent du bois.
\VS{5}Mais il arriva que comme l'un d'eux abattait une poutre, le fer de sa cognée tomba dans l'eau. Il s'écria et dit~: Ah~! Mon seigneur~! Je l'avais emprunté~!
\VS{6}L'homme de Dieu dit~: Où est-il tombé~? Et il lui montra l'endroit. Alors Elisée coupa un morceau de bois, le jeta au même endroit, et fit surnager le fer.
\VS{7}Et il dit~: Retire-le~! Et cet homme étendit sa main et le prit.
\TextTitle{Yahweh révèle à Elisée les plans militaires des Syriens}
\VS{8}Le roi de Syrie était en guerre avec Israël, et, dans un conseil qu'il tint avec ses serviteurs, il dit~: Mon camp sera dans un tel lieu.
\VS{9}L'homme de Dieu envoya dire au roi d'Israël~: Garde-toi de passer dans ce lieu, car les Syriens y descendent.
\VS{10}Et le roi d'Israël envoya des gens, pour s'y tenir en observation, vers le lieu que l'homme de Dieu lui avait mentionné et signalé. Et il y était sur ses gardes. Et cela n'arriva pas seulement une fois ni deux fois.
\VS{11}Le roi de Syrie en eut le cœur troublé~; et il appela ses serviteurs et leur dit~: Ne voulez-vous pas me déclarer lequel de vous est pour le roi d'Israël~?
\VS{12}Et l'un de ses serviteurs répondit~: Personne~! Ô roi, mon seigneur~! Mais Elisée, le prophète qui est en Israël, révèle au roi d'Israël les paroles même que tu déclares dans ta chambre à coucher.
\VS{13}Et il dit~: Allez et voyez où il est, et je le ferai prendre. On vint lui dire~: Voici, il est à Dothan.
\VS{14}Il envoya là des chevaux et des chars, et une grande armée, qui arrivèrent de nuit, et qui entourèrent la ville.
\TextTitle{L'armée de Yahweh plus grande que celle des Syriens}
\VS{15}Le serviteur de l'homme de Dieu se leva de grand matin et sortit~; et voici, une armée, entourait la ville, avec des chevaux et des chars. Le serviteur dit à l'homme de Dieu~: Ah~! Mon seigneur, comment ferons-nous~?
\VS{16}Il lui répondit~: Ne crains point, car ceux qui sont avec nous sont en plus grand nombre que ceux qui sont avec eux.
\VS{17}Elisée pria et dit~: Je te prie, ô Yahweh~! Ouvre ses yeux, afin qu'il voie. Et Yahweh ouvrit les yeux du serviteur et il vit. Et voici la montagne était pleine de chevaux et de chars de feu autour d'Elisée.
\TextTitle{Dieu aveugle les Syriens à la prière d'Elisée}
\VS{18}Les Syriens descendirent vers Elisée. Il adressa alors cette prière à Yahweh~: Je te prie, frappe ces gens d'aveuglement~! Et Dieu les frappa d'aveuglement, selon la parole d'Elisée.
\VS{19}Elisée leur dit~: Ce n'est pas ici le chemin, et ce n'est pas ici la ville~; suivez-moi et je vous conduirai vers l'homme que vous cherchez. Et il les conduisit à Samarie.
\VS{20}Et il arriva qu'aussitôt qu'ils furent entrés dans Samarie, Elisée dit~: Ô Yahweh ouvre leurs yeux afin qu'ils voient. Et Yahweh ouvrit leurs yeux et ils virent qu'ils étaient au milieu de Samarie.
\VS{21}Et dès que le roi d'Israël le vit, il dit à Elisée~: Frapperai-je, frapperai-je, mon père~?
\VS{22}Et Elisée répondit~: Tu ne frapperas point~; frapperais-tu de ton épée et de ton arc ceux que tu as fait prisonniers~? Sers-leur du pain et de l'eau afin qu'ils mangent et boivent~; et après cela, qu'ils s'en aillent vers leur maître.
\VS{23}Le roi d'Israël leur fit servir un grand repas et ils mangèrent et burent~; puis il les renvoya et ils s'en allèrent vers leur maître. Alors, les armées de Syrie ne revinrent plus au pays d'Israël.
\TextTitle{Siège des Syriens et famine en Samarie}
\VS{24}Et il arriva après cela que Ben-Hadad, roi de Syrie, rassembla toute son armée, monta et assiégea Samarie.
\VS{25}Il y eut une grande famine\FTNT{Cette histoire est riche en enseignements pour notre génération. Le siège de la Samarie par les étrangers, la famine qui frappait les Hébreux, le cannibalisme de certaines femmes, la cherté des produits alimentaires, la consommation d'excréments d'animaux à cause de la famine, sont des conséquences du péché. Aujourd'hui, beaucoup d'églises sont assiégées par les choses du monde, les démons, les fausses doctrines, etc.} dans Samarie~; ils l'assiégèrent tellement qu'une tête d'âne se vendait quatre-vingts pièces d'argent, et le quart d'un kab de fiente de pigeon cinq pièces d'argent.
\VS{26}Et comme le roi d'Israël passait sur la muraille, une femme lui cria~: Ô roi, mon seigneur~! Sauve-moi.
\VS{27}Il répondit~: Si Yahweh ne te sauve pas, comment pourrais-je te sauver~? Serait-ce avec le produit de l'aire ou de la cuve~?
\VS{28}Il lui dit encore~: Qu'as-tu~? Elle répondit~: Cette femme-là m'a dit~: Donne ton fils, et mangeons-le aujourd'hui, et nous mangerons mon fils demain\FTNT{Lé. 26:29~; De. 28:53-57.}.
\VS{29}Ainsi nous avons fait bouillir mon fils et l'avons mangé. Et le jour suivant, je lui ai dit~: Donne ton fils et nous le mangerons. Mais elle a caché son fils.
\VS{30}Dès que le roi entendit les paroles de cette femme, il déchira ses vêtements et passa sur la muraille. Le peuple vit qu'il avait en dessous un sac sur son corps.
\VS{31}C'est pourquoi le roi dit~: Que Dieu me traite dans toute sa rigueur, si aujourd'hui la tête d'Elisée, fils de Schaphath, reste sur lui.
\VS{32}Or Elisée était assis dans sa maison, et les anciens étaient assis avec lui. Le roi envoya un homme devant lui. Mais avant que le messager soit arrivé, Elisée dit aux anciens~: Ne voyez-vous pas que le fils de ce meurtrier envoie quelqu'un pour m'ôter la tête~? Lorsque le messager viendra, fermez la porte et repoussez-le avec la porte. N'entendez-vous pas le bruit des pas de son maître derrière lui~?
\VS{33}Et comme il parlait encore avec eux, voici le messager descendit vers lui et dit~: Voici, ce mal vient de Yahweh~; qu'ai-je à espérer encore de Yahweh~?
\Chap{7}
\TextTitle{Prophétie d'Elisée~; les lépreux dans le camp des Syriens}
\VerseOne{}Alors Elisée dit~: Ecoutez la parole de Yahweh~! Ainsi parle Yahweh~: Demain, à cette heure, on aura une mesure de fleur de farine pour un sicle, et deux mesures d'orge pour un sicle, à la porte de Samarie.
\VS{2}Mais l'officier sur la main duquel le roi s'appuyait répondit à l'homme de Dieu et dit~: Quand Yahweh ferait des fenêtres au ciel, cela arriverait-il~? Et Elisée dit~: Tu le verras de tes yeux, mais tu n'en mangeras pas.
\VS{3}Or il y avait à l'entrée de la porte quatre hommes lépreux\FTNT{Dieu s'est servi de ces quatre lépreux comme messagers de bonnes nouvelles. Le Seigneur utilise souvent les personnes rejetées et déconsidérées (1 Co. 1:26-31).}, et ils se dirent l'un à l'autre~: Pourquoi resterions-nous ici jusqu'à ce que nous mourions~?
\VS{4}Si nous pensons à entrer dans la ville, la famine est dans la ville et nous y mourrons~; et si nous restons ici, nous mourrons également. Allons-nous jeter dans le camp des Syriens~; s'ils nous laissent vivre, nous vivrons, et s'ils nous font mourir, nous mourrons.
\VS{5}Ils se levèrent donc au crépuscule pour entrer au camp des Syriens. Lorsqu'ils furent arrivés à l'extrémité du camp, voici, il n'y avait personne.
\VS{6}Car le Seigneur avait fait entendre dans le camp des Syriens un bruit de chars, et un bruit de chevaux, et un bruit d'une grande armée~; de sorte qu'ils s'étaient dit l'un à l'autre~: Voici, le roi d'Israël a payé les rois des Héthiens et les rois des Egyptiens pour venir contre nous.
\VS{7}C'est pourquoi ils s'étaient levés au crépuscule et s'étaient enfuis. Ils avaient abandonné leurs tentes, leurs chevaux, leurs ânes, et le camp tel qu'il était, et ils s'étaient enfuis pour sauver leur vie.
\VS{8}Les lépreux donc arrivèrent jusqu'à l'extrémité du camp. Ils entrèrent dans une tente, mangèrent, burent, emportèrent de l'argent, de l'or, des vêtements, et ils s'en allèrent et les cachèrent. Ils revinrent et entrèrent dans une autre tente et emportèrent de là aussi des objets, s'en allèrent et les cachèrent.
\VS{9}Alors ils se dirent l'un à l'autre~: Nous n'agissons pas bien~! Ce jour est un jour de bonnes nouvelles~; si nous gardons le silence et si nous attendons jusqu'à lumière du matin, le châtiment nous atteindra. Venez maintenant et allons informer la maison du roi.
\VS{10}Ils partirent et appelèrent les portiers de la ville, et leur racontèrent, en disant~: Nous sommes entrés dans le camp des Syriens, et voici, il n'y a personne. On n'y entend aucune voix d'homme~; il n'y a que des chevaux attachés, des ânes attachés et les tentes sont comme elles étaient.
\VS{11}Alors les portiers crièrent et transmirent ce rapport à la maison du roi.
\TextTitle{Accomplissement de la prophétie d'Elisée}
\VS{12}Le roi se leva de nuit et dit à ses serviteurs~: Je veux vous dire ce que les Syriens ont préparé contre nous. Ils savent que nous sommes affamés et ils sont sortis du camp pour se cacher dans les champs, disant~: Quand ils sortiront hors de la ville, nous les saisirons vivants et nous entrerons dans la ville.
\VS{13}L'un des serviteurs du roi répondit et dit~: Qu'on prenne cinq des chevaux qui restent encore dans la ville~; c'est presque tout ce qui est resté du grand nombre des chevaux d'Israël~; ils sont comme toute la multitude d'Israël, qui est consumée. Envoyons voir ce qui se passe.
\VS{14}Ils prirent donc deux chars avec les chevaux, et le roi envoya des messagers après l'armée des Syriens, en disant~: Allez et voyez.
\VS{15}Et ils allèrent après eux jusqu'au Jourdain~; et voici, le chemin était plein de vêtements et d'objets que les Syriens avaient jetés dans leur précipitation. Les messagers revinrent et le rapportèrent au roi.
\VS{16}Alors le peuple sortit et pilla le camp des Syriens, de sorte qu'il eut une mesure de fleur de farine pour un sicle, et deux mesures d'orge pour un sicle, selon la parole de Yahweh.
\VS{17}Le roi donna à l'officier, sur la main duquel il s'appuyait, la charge de garder la porte. Mais cet officier fut écrasé à la porte par le peuple et il en mourut selon la parole qu'avait prononcée l'homme de Dieu, quand le roi était descendu vers lui.
\VS{18}Car lorsque l'homme de Dieu avait parlé au roi, en disant~: Demain matin, à cette heure-ci, on donnera à la porte de Samarie deux mesures d'orge pour un sicle et une mesure de fleur de farine pour un sicle~;
\VS{19}cet officier avait répondu à l'homme de Dieu~: Quand Yahweh ferait des fenêtres au ciel, ce que tu dis pourrait-il arriver~? Et l'homme de Dieu avait dit~: Voici, tu le verras de tes yeux, mais tu n'en mangeras pas.
\VS{20}C'est en effet ce qui lui arriva~; car le peuple l'écrasa à la porte et il mourut.
\Chap{8}
\TextTitle{Elisée annonce une famine de sept ans}
\VerseOne{}Elisée parla à la femme dont il avait fait revivre le fils, en disant~: Lève-toi et va-t'en, toi et ta famille, et séjourne où tu pourras~; car Yahweh a appelé la famine, et même elle vient sur le pays pour sept ans.
\VS{2}La femme se leva et elle fit selon la parole de l'homme de Dieu. Elle s'en alla, elle et sa famille, et séjourna sept ans au pays des Philistins.
\TextTitle{La Sunamite retrouve ses terres}
\VS{3}Mais il arriva qu'au bout des sept ans, la femme revint du pays des Philistins, et alla implorer le roi au sujet de sa maison et de ses champs.
\VS{4}Le roi parlait à Guéhazi\FTNT{Voir 2 R. 5.}, serviteur de l'homme de Dieu, en disant~: Je te prie raconte-moi toutes les grandes choses qu'Elisée a faites.
\VS{5}Et il arriva que comme il racontait au roi comment Elisée avait rendu la vie à un mort, la femme dont Elisée avait fait revivre le fils vint implorer le roi au sujet de sa maison et de ses champs. Guéhazi dit~: Ô roi, mon seigneur, voici la femme et voici son fils, à qui Elisée a rendu la vie.
\VS{6}Alors le roi interrogea la femme, et elle lui raconta ce qui s'était passé. Le roi lui donna un eunuque, auquel il dit~: Fais restituer tout ce qui lui appartenait, même tous les revenus de ses champs, depuis le jour où elle a quitté le pays jusqu'à maintenant.
\TextTitle{Prophétie sur le règne d'Hazaël sur la Syrie}
\VS{7}Elisée se rendit à Damas. Ben-Hadad, roi de Syrie, était malade et on lui fit ce rapport~: L'homme de Dieu est venu ici.
\VS{8}Le roi dit à Hazaël~: Prends avec toi un présent et va au-devant de l'homme de Dieu, et consulte par lui Yahweh, en disant~: Guérirai-je de cette maladie~?
\VS{9}Et Hazaël s'en alla au-devant d'Elisée, ayant pris avec lui un présent, à savoir quarante chameaux chargés de tout ce qu'il y avait de meilleur à Damas. Il vint se présenter devant Elisée et dit~: Ton fils, Ben-Hadad, roi de Syrie, m'a envoyé vers toi, pour te dire~: Guérirai-je de cette maladie~?
\VS{10}Et Elisée lui répondit~: Va, dis-lui~: Tu guériras~! Tu guériras~! Toutefois, Yahweh m'a révélé qu'il mourra, qu'il mourra.
\VS{11}L'homme de Dieu arrêta son regard sur Hazaël et le fixa longtemps, puis il pleura.
\VS{12}Hazaël dit~: Pourquoi mon seigneur pleure-t-il~? Et il répondit~: Parce que je sais le mal que tu feras aux enfants d'Israël~; tu mettras le feu à leurs villes fortes, tu tueras avec l'épée leurs jeunes gens, tu écraseras leurs petits-enfants et tu fendras le ventre de leurs femmes enceintes.
\VS{13}Hazaël dit~: Mais qu'est-ce que ton serviteur, ce chien, pour faire de si grandes choses~? Et Elisée répondit~: Yahweh m'a révélé que tu seras roi de Syrie.
\VS{14}Alors Hazaël quitta Elisée et revint vers son maître, qui lui demanda~: Que t'a dit Elisée~? Et il répondit~: Il m'a dit que tu guériras~! Tu guériras~!
\VS{15}Mais le lendemain, Hazaël prit une couverture et l'ayant plongé dans l'eau, il l'étendit sur le visage de Ben-Hadad, qui mourut. Et Hazaël régna à sa place.
\TextTitle{Joram règne sur Juda\FTNTT{2 Ch. 21:1-7.}}
\VS{16}La cinquième année de Joram, fils d'Achab, roi d'Israël, Josaphat était encore roi de Juda et Joram, fils de Josaphat, roi de Juda, commença à régner sur Juda.
\VS{17}Il était âgé de trente-deux ans lorsqu'il commença à régner. Il régna huit ans à Jérusalem.
\VS{18}Il marcha dans la voie des rois d'Israël comme avait fait la maison d'Achab, car il avait pour femme la fille d'Achab\FTNT{Le mariage de Joram, fils de Josaphat, avec Athalie, fille d'Achab, était une grande erreur. Cette union qui était contractée dans le but de favoriser la paix entre les deux royaumes, entraîna le déclin de Juda~; or Dieu est contre les alliances contre nature. Voir Es. 30-31.}, et il fit ce qui est mal aux yeux de Yahweh.
\VS{19}Mais Yahweh ne voulut point détruire Juda, par amour pour David, son serviteur, selon la promesse qu'il lui avait faite de lui donner toujours une lampe parmi ses fils.
\TextTitle{Révoltes contre l'autorité de Juda}
\VS{20}De son temps, Edom se révolta contre l'autorité de Juda et se donna un roi.
\VS{21}Joram passa à Tsaïr, avec tous ses chars~; il se leva de nuit, et frappa les Edomites qui l'entouraient, et les chefs des chars, mais le peuple s'enfuit dans ses tentes.
\VS{22}Néanmoins, les Edomites ont été rebelles à Juda jusqu'à ce jour. En ce même temps, Libna aussi se révolta.
\TextTitle{Achazia règne sur Juda\FTNTT{2 Ch. 21:18-22:4.}}
\VS{23}Le reste des actions de Joram et tout ce qu'il a fait, cela n'est-il pas écrit dans le livre des Chroniques des rois de Juda~?
\VS{24}Joram se coucha avec ses pères et il fut enterré avec ses pères dans la cité de David. Et Achazia, son fils, régna à sa place.
\VS{25}La douzième année de Joram, fils d'Achab, roi d'Israël, Achazia, fils de Joram, roi de Juda, commença à régner.
\VS{26}Achazia était âgé de vingt-deux ans lorsqu'il commença à régner. Il régna un an à Jérusalem. Sa mère s'appelait Athalie, fille d'Omri, roi d'Israël.
\VS{27}Il marcha dans la voie de la maison d'Achab et il fit ce qui est mal aux yeux de Yahweh, comme avait fait la maison d'Achab, car il était gendre de la maison d'Achab.
\VS{28}Il alla avec Joram, fils d'Achab, à la guerre contre Hazaël, roi de Syrie, à Ramoth en Galaad. Et les Syriens blessèrent Joram.
\VS{29}Le roi Joram s'en retourna pour se faire guérir à Jizreel des blessures que les Syriens lui avaient faites à Rama, lorsqu'il se battait contre Hazaël, roi de Syrie. Achazia, fils de Joram, roi de Juda, descendit pour voir Joram, fils d'Achab, à Jizreel, parce qu'il était malade.
\Chap{9}
\TextTitle{Jéhu oint roi d'Israël}
\VerseOne{}Alors Elisée, le prophète, appela l'un des fils des prophètes et lui dit~: Ceins tes reins, prends cette fiole d'huile dans ta main, et va à Ramoth en Galaad.
\VS{2}Quand tu y seras entré, vois Jéhu, fils de Josaphat, fils de Nimschi. Tu iras le faire lever du milieu de ses frères et tu le conduiras dans une chambre secrète.
\VS{3}Tu prendras la fiole d'huile, tu la verseras sur sa tête et tu diras~: Ainsi parle Yahweh~: Je t'ai oint pour être roi sur Israël. Après quoi tu ouvriras la porte, tu t'enfuiras et tu ne t'arrêteras pas.
\VS{4}Le jeune homme, serviteur du prophète, s'en alla à Ramoth en Galaad.
\VS{5}Quand il arriva, voici, les chefs de l'armée étaient là assis. Il dit~: Chef, j'ai à te parler. Et Jéhu répondit~: Auquel de nous parles-tu~? Et il répondit~: A toi, chef.
\VS{6}Alors Jéhu se leva, et entra dans la maison, et le jeune homme répandit l'huile sur la tête, et lui dit~: Ainsi parle Yahweh, le Dieu d'Israël~: Je t'ai oint pour être roi sur Israël, le peuple de Yahweh.
\VS{7}Tu frapperas la maison d'Achab, ton maître, et je vengerai sur Jézabel\FTNT{1 R. 16:31~; 1 R. 17,18,19.} le sang de mes serviteurs les prophètes, et le sang de tous les serviteurs de Yahweh.
\VS{8}Et toute la maison d'Achab périra, et je retrancherai quiconque appartient à Achab, celui qui est esclave et celui qui est libre en Israël.
\VS{9}Je rendrai la maison d'Achab semblable à la maison de Jéroboam, fils de Nebath, et à la maison de Baescha, fils d'Achija.
\VS{10}Les chiens mangeront Jézabel dans le champ de Jizreel, et il n'y aura personne pour l'enterrer. Puis il ouvrit la porte et s'enfuit.
\VS{11}Jéhu sortit pour rejoindre les serviteurs de son maître et on lui dit~: Tout va bien~? Pourquoi ce fou est-il venu vers toi~? Jéhu leur répondit~: Vous connaissez l'homme et ses rêveries.
\VS{12}Mais ils répliquèrent~: Mensonge~! Réponds-nous donc. Et il dit~: Il m'a parlé de telle et telle manière, disant~: Ainsi parle Yahweh, je t'ai oint pour être roi sur Israël.
\VS{13}Alors ils se hâtèrent, et prirent chacun leurs vêtements, et les mirent sous lui au plus haut des degrés. Ils sonnèrent du shofar et dirent~: Jéhu a été fait roi~!
\TextTitle{Mort de Joram}
\VS{14}Ainsi Jéhu, fils de Josaphat, fils de Nimschi, forma une conspiration contre Joram. Or Joram et tout Israël défendaient Ramoth en Galaad contre Hazaël, roi de Syrie.
\VS{15}Le roi Joram s'en était retourné pour se faire guérir à Jizreel des blessures que les Syriens lui avaient faites, lorsqu'il se battait contre Hazaël, roi de Syrie. Jéhu dit~: Si vous le trouvez bon, que personne ne sorte ni ne s'échappe de la ville pour aller porter cette nouvelle à Jizreel.
\VS{16}Alors, Jéhu monta à cheval et s'en alla à Jizreel, car Joram était là, malade, et Achazia, roi de Juda, y était descendu pour le visiter.
\VS{17}Or il y avait une sentinelle sur une tour à Jizreel, qui voyant venir la troupe de Jéhu dit~: Je vois une troupe de gens. Et Joram dit~: Prends un cavalier et envoie-le à leur rencontre, et qu'il dise~: Est-ce la paix~?
\VS{18}Le cavalier s'en alla à sa rencontre, et dit~: Ainsi parle le roi~: Est-ce la paix~? Et Jéhu répondit~: Qu'as-tu à faire de la paix~? Mets-toi derrière moi. La sentinelle le rapporta, en disant~: Le messager est allé jusqu'à eux et il ne revient pas.
\VS{19}Joram envoya un second cavalier, qui arriva jusqu'à eux et dit~: Ainsi parle le roi~: Est-ce la paix~? Et Jéhu répondit~: Qu'as-tu à faire de la paix~? Mets-toi derrière moi.
\VS{20}La sentinelle le rapporta et dit~: Il est arrivé jusqu'à eux et il ne revient pas~; mais la manière de conduire le char est comme celle de Jéhu, fils de Nimschi~; car il le conduit avec furie.
\VS{21}Alors Joram dit~: Attelle~! Et on attela son char. Ainsi Joram, roi d'Israël, sortit avec Achazia, roi de Juda, chacun dans son char, et ils allèrent à la rencontre de Jéhu, et ils le trouvèrent dans le champ de Naboth de Jizreel\FTNT{1 R. 21.}.
\VS{22}Dès que Joram vit Jéhu, il dit~: Est-ce la paix, Jéhu~? Jéhu répondit~: Quelle paix~! Tant que durent les prostitutions de Jézabel, ta mère, et la multitude de ses enchantements~!
\VS{23}Alors Joram tourna sa main et s'enfuit, et il dit à Achazia~: Trahison, Achazia~!
\VS{24}Mais Jéhu saisit l'arc de sa main, et il frappa Joram entre ses épaules, de sorte que la flèche transperça son cœur, et il tomba sur ses genoux dans son char.
\VS{25}Jéhu dit à Bidkar, son officier~: Prends-le et jette-le dans le champ de Naboth de Jizreel~; car souviens-toi, lorsque nous étions à cheval moi et toi, ensemble, derrière Achab, son père, Yahweh prononça cette sentence contre lui~:
\VS{26}N'ai-je pas vu hier le sang de Naboth et le sang de ses fils, dit Yahweh~? Et je te le rendrai dans ce champ-ci, dit Yahweh~! C'est pourquoi prends-le donc, et jette-le dans ce champ, selon la parole de Yahweh.
\TextTitle{Mort d'Achazia\FTNTT{2 Ch. 22:7,9.}}
\VS{27}Achazia, roi de Juda, ayant vu cela, s'enfuit par le chemin de la maison du jardin~; mais Jéhu le poursuivit et dit~: Frappez-le sur le char~! Et on le frappa à la montée de Gur, près de Jibleam. Puis il se réfugia à Meguiddo, et il y mourut.
\VS{28}Ses serviteurs le transportèrent sur un char à Jérusalem, et ils l'enterrèrent dans son sépulcre avec ses pères, dans la cité de David.
\VS{29}Achazia avait commencé à régner sur Juda la onzième année de Joram, fils d'Achab.
\TextTitle{Mort de Jézabel}
\VS{30}Jéhu entra dans Jizreel. Jézabel, l'ayant appris, mit du fard à ses yeux, orna sa tête et regarda par la fenêtre.
\VS{31}Comme Jéhu franchissait la porte, elle dit~: Est-ce la paix, Zimri, assassin de son maître~?
\VS{32}Il leva sa tête vers la fenêtre et dit~: Qui est avec moi~? Qui~? Alors deux ou trois des eunuques regardèrent vers lui.
\VS{33}Et il leur dit~: Jetez-la en bas~! Et ils la jetèrent, de sorte qu'il rejaillit de son sang sur la muraille et sur les chevaux. Jéhu la foula aux pieds~;
\VS{34}puis il entra, mangea et but, et il dit~: Allez voir maintenant cette maudite et enterrez-la, car elle est fille de roi.
\VS{35}Ils allèrent donc pour l'enterrer~; mais ils ne trouvèrent d'elle que le crâne, les pieds et les paumes des mains.
\VS{36}Ils retournèrent l'annoncer à Jéhu, qui dit~: C'est la parole que Yahweh avait déclarée par son serviteur Elie\FTNT{1 R. 21:23.}, le Thischbite, en disant~: Dans le champ de Jizreel les chiens mangeront la chair de Jézabel~;
\VS{37}et le cadavre de Jézabel sera comme du fumier sur la face des champs, dans le champ de Jizreel, de sorte qu'on ne pourra dire~: C'est Jézabel.
\Chap{10}
\TextTitle{Accomplissement du jugement de Dieu sur la maison d'Achab}
\VerseOne{}Achab avait soixante-dix fils dans Samarie. Jéhu écrivit des lettres qu'il envoya à Samarie aux chefs de Jizreel, aux anciens et aux gouverneurs d'Achab. Il y était dit~:
\VS{2}Dès que cette lettre vous sera parvenue, puisque vous avez avec vous les fils de votre maître, avec vous les chars et les chevaux, la ville forte et les armes,
\VS{3}choisissez qui est le plus considérable et le plus sincère parmi les fils de votre maître, mettez-le sur le trône de son père et combattez pour la maison de votre maître.
\VS{4}Ils eurent une très grande peur et ils dirent~: Voici, deux rois n'ont point pu tenir contre lui, comment donc résisterions-nous~?
\VS{5}Et le chef de la maison, le chef de la ville, les anciens et les gouverneurs envoyèrent dire à Jéhu~: Nous sommes tes serviteurs, nous ferons tout ce que tu nous diras~; nous n'établirons personne roi, fais ce qui te semblera bon.
\VS{6}Jéhu leur écrivit une seconde lettre, où il était dit~: Si vous êtes pour moi et si vous obéissez à ma voix, prenez les têtes des fils de votre maître et venez auprès de moi demain à cette heure-ci, à Jizreel. Or les soixante-dix hommes, fils du roi, étaient avec les plus grands de la ville qui les élevaient.
\VS{7}Aussitôt que la lettre leur fut parvenue, ils prirent les fils du roi et ils égorgèrent ces soixante-dix hommes~; et ayant mis leurs têtes dans des corbeilles, ils les envoyèrent à Jéhu, à Jizreel.
\VS{8}Un messager vint l'en informer, en disant~: Ils ont apporté les têtes des fils du roi. Et il répondit~: Mettez-les en deux tas à l'entrée de la porte, jusqu'au matin.
\VS{9}Le matin, il sortit~; et se présentant à tout le peuple, il dit~: Vous êtes justes~! Voici, j'ai conspiré contre mon maître et je l'ai tué~; mais qui a frappé tous ceux-ci~?
\VS{10}Sachez maintenant qu'il ne tombera rien à terre de la parole de Yahweh\FTNT{1 R. 21:19-24.}, de la parole que Yahweh a prononcée contre la maison d'Achab~; Yahweh accomplit ce qu'il avait déclaré par son serviteur Elie.
\VS{11}Jéhu tua aussi tous ceux qui restaient de la maison d'Achab à Jizreel, tous ses grands, ses familiers et ses prêtres, sans en laisser échapper un seul.
\TextTitle{Mise à mort des frères d'Achazia et de la lignée d'Achab\FTNTT{2 Ch. 22:8.}}
\VS{12}Puis il se leva et partit pour aller à Samarie. Et comme il était près d'une maison de bergers sur le chemin,
\VS{13}Jéhu trouva les frères d'Achazia, roi de Juda, et leur dit~: Qui êtes-vous~? Ils répondirent~: Nous sommes les frères d'Achazia et nous sommes descendus pour saluer les fils du roi et les fils de la reine.
\VS{14}Jéhu dit~: Saisissez-les vivants. Ils les saisirent vivants et les égorgèrent, à savoir quarante-deux hommes, auprès du puits de la maison des bergers, sans en laisser échapper un seul.
\VS{15}Jéhu étant parti de là, il rencontra Jonadab, fils de Récab, qui venait au-devant de lui. Il le salua, et lui dit~: Ton cœur est-il aussi droit envers moi comme mon cœur l'est à ton égard~? Et Jonadab répondit~: Il l'est. Donne-moi ta main répliqua Jéhu. Et Jonadab lui donna sa main, et Jéhu le fit monter auprès de lui dans son char.
\VS{16}Puis il dit~: Viens avec moi et tu verras le zèle que j'ai pour Yahweh. Il l'emmena ainsi dans son char.
\VS{17}Et quand Jéhu fut arrivé à Samarie, il tua tous ceux qui restaient de la maison d'Achab à Samarie, et il les extermina entièrement, selon la parole que Yahweh avait dite à Elie.
\TextTitle{Mise à mort de tous les prophètes de Baal}
\VS{18}Puis Jéhu assembla tout le peuple, et leur dit~: Achab a peu servi Baal\FTNT{Jg. 2:13.}, mais Jéhu le servira beaucoup.
\VS{19}Maintenant donc, convoquez-moi tous les prophètes de Baal, tous ses serviteurs, et tous ses prêtres, sans qu'il en manque un seul, car je veux offrir un grand sacrifice à Baal~: Quiconque manquera ne vivra pas. Jéhu agissait avec ruse, pour faire périr les serviteurs de Baal.
\VS{20}Jéhu dit~: Publiez une fête solennelle en l'honneur de Baal. Et ils la publièrent.
\VS{21}Jéhu envoya des messagers dans tout Israël~; et tous les serviteurs de Baal arrivèrent, il n'y en eut pas un qui ne vînt~; et ils entrèrent dans le temple de Baal, qui fut rempli d'un bout à l'autre.
\VS{22}Alors Jéhu dit à celui qui avait la charge du vestiaire~: Sors des vêtements pour tous les serviteurs de Baal. Et cet homme sortit des vêtements.
\VS{23}Alors Jéhu, et Jonadab, fils de Récab, entrèrent dans le temple de Baal, et Jéhu dit aux serviteurs de Baal~: Cherchez et regardez afin qu'il n'y ait pas ici de serviteurs de Yahweh. Prenez garde qu'il n'y ait seulement que les serviteurs de Baal.
\VS{24}Ils entrèrent donc pour offrir des sacrifices et des holocaustes. Or Jéhu avait placé dehors quatre-vingts hommes, et leur avait dit~: Celui qui laissera échapper un de ces hommes que je remets entre vos mains, sa vie répondra de la sienne.
\VS{25}Et il arriva que dès qu'on eut achevé d'offrir l'holocauste, Jéhu dit aux gardes et aux officiers~: Entrez, tuez-les, et que nul n'échappe. Les gardes et les officiers les frappèrent du tranchant de l'épée, et les jetèrent là~; puis ils allèrent jusqu'à la ville du temple de Baal.
\VS{26}Ils tirèrent dehors les statues de la maison de Baal, et les brûlèrent.
\VS{27}Et ils démolirent la statue de Baal. Ils démolirent aussi la maison de Baal, et ils en firent un cloaque qui subsiste jusqu'à ce jour.
\VS{28}Ainsi Jéhu extermina Baal d'Israël.
\TextTitle{L'idolâtrie dans la vie de Jéhu}
\VS{29}Toutefois, Jéhu ne se détourna point des péchés que Jéroboam, fils de Nebath, avait fait commettre à Israël, à savoir les veaux d'or\FTNT{1 R. 12:28-29.} qui étaient à Béthel et à Dan.
\VS{30}Yahweh dit à Jéhu~: Parce que tu as fort bien exécuté ce qui était droit à mes yeux, et que tu as fait à la maison d'Achab tout ce qui était conforme à ma volonté, tes fils seront assis sur le trône d'Israël jusqu'à la quatrième génération.
\VS{31}Mais Jéhu ne prit point garde à marcher de tout son cœur dans la loi de Yahweh, le Dieu d'Israël~; il ne se détourna point des péchés que Jéroboam avait fait commettre à Israël.
\TextTitle{Hazaël règne sur la Syrie}
\VS{32}Dans ce temps-là, Yahweh commença à entamer le territoire d'Israël, et Hazaël battit les Israélites sur toutes les frontières.
\VS{33}Depuis le Jourdain, jusqu'au soleil levant, il battit tout le pays de Galaad, les Gadites, les Rubénites et ceux de Manassé, depuis Aroër sur le torrent de l'Arnon, jusqu'à Galaad et à Basan.
\TextTitle{Joachaz règne sur Israël}
\VS{34}Le reste des actions de Jéhu, tout ce qu'il a fait, et tous ses exploits, ne sont-ils pas écrits dans le livre des Chroniques des rois d'Israël~?
\VS{35}Jéhu se coucha avec ses pères, et on l'enterra à Samarie. Et Joachaz, son fils, régna à sa place.
\VS{36}Jéhu avait régné vingt-huit ans sur Israël à Samarie.
\Chap{11}
\TextTitle{Athalie fait périr la race royale de Juda\FTNTT{2 Ch. 22:9-12.}}
\VerseOne{}Athalie, mère d'Achazia, ayant vu que son fils était mort, se leva et extermina toute la race royale.
\VS{2}Mais Joschéba, fille du roi Joram, sœur d'Achazia, prit Joas, fils d'Achazia, et l'enleva du milieu des fils du roi, quand on les fit mourir~: Elle le mit avec sa nourrice dans la chambre des lits. Il fut ainsi dérobé aux regards d'Athalie, de sorte qu'on ne le fit point mourir.
\VS{3}Il resta caché six ans avec Joschéba dans la maison de Yahweh. Cependant Athalie régnait sur le pays.
\TextTitle{Joas devient roi de Juda\FTNTT{2 Ch. 23:1-11.}}
\VS{4}La septième année, Jehojada envoya chercher les chefs de centaines des Kéréthiens et des archers, et il les fit venir auprès de lui dans la maison de Yahweh. Il traita alliance avec eux, les fit jurer dans la maison de Yahweh, et leur montra le fils du roi.
\VS{5}Puis il leur donna cet ordre, en disant~: Voici ce que vous ferez. Parmi ceux d'entre vous qui entrent en service le jour du sabbat, un tiers doit monter la garde à la maison du roi,
\VS{6}un tiers sera à la porte de Sur, et un tiers à la porte derrière les archers~; ainsi vous veillerez à la garde de la maison, afin que personne n'y entre par force.
\VS{7}Vos deux autres compagnies, tous ceux qui sortent de service le jour du sabbat feront la garde de la maison de Yahweh, auprès du roi~:
\VS{8}Et vous entourerez le roi de toutes parts, chacun ayant ses armes à la main, et l'on mettra à mort quiconque s'avancera dans les rangs~; vous serez avec le roi quand il sortira et quand il entrera.
\VS{9}Les chefs de centaines firent donc tout ce que Jehojada, le prêtre, avait ordonné. Ils prirent chacun leurs gens, ceux qui entraient en service et ceux qui sortaient de service le jour du sabbat, et ils se rendirent vers le prêtre Jehojada.
\VS{10}Le prêtre donna aux chefs de centaine les lances et les boucliers qui provenaient du roi David, et qui étaient dans la maison de Yahweh.
\VS{11}Les archers, chacun les armes à la main, entourèrent le roi, en se plaçant depuis le côté droit de la maison, jusqu'au côté gauche, près de l'autel et près de la maison.
\VS{12}Jehojada fit amener le fils du roi, et il mit sur lui la couronne\FTNT{Couronne ou consacrer.} et le témoignage. Ils l'établirent roi et l'oignirent, et frappant des mains, ils dirent~: Vive le roi~!
\TextTitle{Mort d'Athalie\FTNTT{2 Ch. 23:12-15,21.}}
\VS{13}Athalie entendit le bruit des archers et du peuple, et elle vint vers le peuple à la maison de Yahweh.
\VS{14}Elle regarda. Et voici, le roi se tenait sur l'estrade, selon la coutume des rois. Les chefs et les trompettes étaient près du roi~: Tout le peuple du pays éclatait de joie, et on sonnait des trompettes. Alors Athalie déchira ses vêtements, et cria~: Conspiration~! Conspiration~!
\VS{15}Alors le prêtre Jehojada donna cet ordre aux chefs de centaines, qui avaient la charge de l'armée~: Faites-la sortir hors des rangs, et que celui qui la suivra soit mis à mort par l'épée. Car le prêtre avait dit~: Qu'elle ne soit pas mise à mort dans la maison de Yahweh~!
\VS{16}Ils lui firent donc place, et elle retourna dans la maison du roi par le chemin de l'entrée des chevaux~: C'est là qu'elle fut tuée.
\TextTitle{Alliance entre Jehojada, Yahweh et le peuple~; réveil sous le règne de Joas\FTNTT{2 Ch. 23:16-21.}}
\VS{17}Jehojada traita entre Yahweh, le roi et le peuple l'alliance par laquelle ils devaient être le peuple de Yahweh~; il traita aussi l'alliance entre le roi et le peuple.
\VS{18}Alors tout le peuple du pays entra dans la maison de Baal, et ils la démolirent avec ses autels~; et ils brisèrent entièrement ses images~; ils tuèrent aussi Matthan, prêtre de Baal, devant les autels. Le prêtre Jehojada établit des gardes dans la maison de Yahweh.
\VS{19}Il prit les chefs de centaines, les Kéréthiens et les archers, et tout le peuple du pays~; et ils firent descendre le roi de la maison de Yahweh, et ils entrèrent dans la maison du roi par le chemin de la porte des archers, et Joas s'assit sur le trône des rois.
\VS{20}Tout le peuple du pays fut dans la joie, et la ville fut en repos, après qu'on eût mis à mort Athalie par l'épée dans la maison du roi.
\VS{21}Joas était âgé de sept ans lorsqu'il commença à régner.
\Chap{12}
\TextTitle{Joas ordonne des réparations dans le temple\FTNTT{2 Ch. 24:2.}}
\VerseOne{}La septième année de Jéhu, Joas, commença à régner. Il régna quarante ans à Jérusalem. Sa mère s'appelait Tsibja, elle était de Beer-Schéba.
\VS{2}Joas fit ce qui est droit aux yeux de Yahweh pendant tout le temps qu'il suivit les instructions de Jehojada, le prêtre.
\VS{3}Toutefois, les hauts lieux ne disparurent point~; le peuple offrait encore des sacrifices et des parfums sur les hauts lieux.
\VS{4}Joas dit aux prêtres~: Tout l'argent consacré qu'on apporte dans la maison de Yahweh, l'argent ayant cours, à savoir l'argent pour l'évaluation des personnes d'après l'estimation qui en est faite, et tout l'argent que chacun apporte volontairement à la maison de Yahweh,
\VS{5}que les prêtres le prennent, chacun de la part des gens de sa connaissance, et qu'ils l'emploient à réparer ce qui est à réparer dans la maison, partout où l'on trouvera quelque chose à réparer.
\VS{6}Mais il arriva que, la vingt-troisième année du roi Joas, les prêtres n'avaient point encore réparé les brèches de la maison.
\VS{7}Le roi Joas appela le prêtre Jehojada et les autres prêtres, et il leur dit~: Pourquoi n'avez-vous pas réparé ce qui était à réparer dans la maison~? Maintenant, vous ne prendrez plus l'argent de vos connaissances, mais vous le livrerez pour les réparations de la maison.
\VS{8}Les prêtres convinrent de ne plus prendre l'argent du peuple et de ne pas être chargés des réparations de la maison.
\TextTitle{Offrandes volontaires pour réparer le temple\FTNTT{2 Ch. 24:8-14.}}
\VS{9}Alors le prêtre Jehojada prit un coffre, et le perça dans son couvercle, et le plaça à côté de l'autel, à droite, à l'endroit par lequel on entrait à la maison de Yahweh. Les prêtres qui avaient la garde du seuil y mettaient tout l'argent qu'on apportait à la maison de Yahweh.
\VS{10}Et dès qu'ils voyaient qu'il y avait beaucoup d'argent dans le coffre, le secrétaire du roi montait avec le grand-prêtre, et ils mettaient dans des sacs l'argent qui se trouvait dans la maison de Yahweh, puis ils le comptaient.
\VS{11}Ils remettaient cet argent bien compté entre les mains de ceux qui étaient chargés de faire exécuter l'ouvrage dans la maison de Yahweh. Et l'on employait cet argent pour les charpentiers et pour les architectes qui travaillaient à la maison de Yahweh,
\VS{12}pour les maçons et les tailleurs de pierres, pour acheter du bois et des pierres de taille, afin de réparer les brèches de la maison de Yahweh, et pour acheter tout ce qu'il fallait pour la réparation de la maison.
\VS{13}Mais, avec l'argent qu'on apportait dans la maison de Yahweh, on ne fit pour la maison de Yahweh ni bassins d'argent ni de couteaux, ni coupes, ni trompettes, ni aucun autre ustensile d'or, ou ustensile d'argent~;
\VS{14}on le distribuait à ceux qui avaient la charge de l'ouvrage et qui réparaient la maison de Yahweh.
\VS{15}On ne demandait pas de comptes aux hommes entre les mains desquels on remettait l'argent pour qu'ils le donnent à ceux qui faisaient l'ouvrage, car ils le faisaient fidèlement.
\VS{16}L'argent des sacrifices pour la culpabilité et l'argent des sacrifices pour les expiations n'était point apporté dans la maison de Yahweh~: Car il était pour les prêtres.
\TextTitle{Invasion syrienne évitée~; mort de Joas}
\VS{17}Alors Hazaël\FTNT{Hazaël envahit Juda à deux reprises. Ce passage fait mention de la première invasion~; la deuxième invasion est relatée en 2 Ch. 24:23.}, roi de Syrie, monta et fit la guerre à Gath, dont il s'empara. Hazaël avait l'intention de monter contre Jérusalem.
\VS{18}Mais Joas, roi de Juda, prit tout ce qui était consacré, que Josaphat, Joram, et Achazia, ses pères, rois de Juda, avaient consacré, tout ce que lui-même avait consacré, tout l'or qui se trouva dans les trésors de la maison de Yahweh et de la maison du roi~; et il envoya le tout à Hazaël, roi de Syrie, qui ne monta pas contre Jérusalem.
\VS{19}Le reste des actions de Joas, tout ce qu'il a fait, cela n'est-il pas écrit dans le livre des Chroniques des rois de Juda~?
\VS{20}Ses serviteurs se soulevèrent et se liguèrent~; ils frappèrent Joas dans la maison de Millo, qui est à la descente de Silla.
\VS{21}Jozacar, fils de Schimeath, et Jozabad fils de Schomer, ses serviteurs, le frappèrent, et il mourut. On l'enterra avec ses pères dans la cité de David. Et Amatsia, son fils, régna à sa place.
\Chap{13}
\TextTitle{Joachaz règne sur Israël}
\VerseOne{}La vingt-troisième année de Joas, fils d'Achazia, roi de Juda, Joachaz, fils de Jéhu, commença à régner sur Israël à Samarie. Il régna dix-sept ans.
\VS{2}Il fit ce qui est mal aux yeux de Yahweh~; car il suivit les péchés de Jéroboam, fils de Nebath, par lesquels il avait fait pécher Israël, et il ne s'en détourna point.
\TextTitle{L'idolâtrie perdure dans le pays}
\VS{3}La colère de Yahweh s'enflamma contre Israël, et il les livra entre les mains de Hazaël, roi de Syrie, et entre les mains de Ben-Hadad, fils de Hazaël, tout le temps que ces rois vécurent.
\VS{4}Mais Joachaz implora Yahweh. Et Yahweh l'exauça, parce qu'il vit l'oppression sous laquelle le roi de Syrie tenait Israël.
\VS{5}Yahweh donna donc un libérateur à Israël, et ils échappèrent aux mains des Syriens~; ainsi les enfants d'Israël habitèrent dans leurs tentes comme auparavant.
\VS{6}Mais ils ne se détournèrent point des péchés de la maison de Jéroboam, par lesquels il avait fait pécher Israël~; ils s'y livrèrent, et même l'idole d'Asherah\FTNT{Voir commentaire Jg. 2:13.} resta debout à Samarie.
\VS{7}De tout le peuple de Joachaz, Dieu ne lui avait laissé que cinquante cavaliers, dix chars, et dix mille hommes de pied~; car le roi de Syrie les avait fait périr et les avait rendus semblables à la poussière qu'on foule aux pieds.
\TextTitle{Mort de Joachaz~; Joas règne sur Israël}
\VS{8}Le reste des actions de Joachaz, tout ce qu'il a fait, et ses exploits, cela n'est-il pas écrit dans le livre des Chroniques des rois d'Israël~?
\VS{9}Ainsi Joachaz se coucha avec ses pères, et on l'ensevelit à Samarie. Et Joas, son fils, régna à sa place.
\VS{10}La trente-septième année de Joas, roi de Juda, Joas, fils de Joachaz, commença à régner sur Israël à Samarie. Il régna seize ans.
\VS{11}Et il fit ce qui est mal aux yeux de Yahweh~; il ne se détourna d'aucun des péchés de Jéroboam, fils de Nebath, par lesquels il avait fait pécher Israël, il s'y livra comme lui.
\TextTitle{Mort de Joas}
\VS{12}Le reste des actions de Joas, tout ce qu'il a fait, ses exploits, et la guerre qu'il eut avec Amatsia, roi de Juda, tout cela n'est-il pas écrit dans le livre des Chroniques des rois d'Israël~?
\VS{13}Joas se coucha avec ses pères, et Jéroboam s'assit sur son trône. Joas fut enterré à Samarie avec les rois d'Israël.
\TextTitle{Fin de la vie d'Elisée~; récit de la visite de Joas roi d'Israël}
\VS{14}Elisée était atteint de la maladie dont il mourut~; et Joas, roi d'Israël, descendit vers lui, pleura sur son visage, en disant~: Mon père~! Mon père~! Char d'Israël et sa cavalerie~!
\VS{15}Elisée lui dit~: Prends un arc et des flèches. Il prit donc un arc et des flèches.
\VS{16}Puis Elisée dit au roi d'Israël~: Bande l'arc avec ta main. Mets ta main sur l'arc. Et quand il y eut mis sa main, Elisée mit ses mains sur les mains du roi,
\VS{17}et il lui dit~: Ouvre la fenêtre à l'orient. Et il l'ouvrit. Elisée lui dit~: Tire. Après qu'il eut tiré, il lui dit~: C'est la flèche de la délivrance de la part de Yahweh, la flèche de la délivrance contre les Syriens~; tu frapperas les Syriens à Aphek, jusqu'à leur extermination.
\VS{18}Elisée lui dit encore~: Prends les flèches. Et il les prit. Elisée dit au roi d'Israël~: Frappe contre terre. Et le roi frappa trois fois, puis il s'arrêta.
\VS{19}Et l'homme de Dieu se mit dans une très grande colère contre lui, et lui dit~: Il fallait frapper cinq ou six fois~; alors tu aurais battu les Syriens jusqu'à leur extermination~; mais maintenant tu ne les frapperas que trois fois.
\TextTitle{Mort d'Elisée~; ses os rendent la vie à un mort}
\VS{20}Elisée mourut, et on l'ensevelit. L'année suivante, quelques troupes de Moabites entrèrent dans le pays.
\VS{21}Et comme on enterrait un homme, voici, on aperçut l'une des troupes de soldats, et l'on jeta l'homme dans le sépulcre d'Elisée. L'homme alla toucher les os d'Elisée, il reprit vie et se leva sur ses pieds.
\TextTitle{Fin de l'oppression syrienne}
\VS{22}Pendant toute la vie de Joachaz, Hazaël, roi de Syrie, avait opprimé Israël.
\VS{23}Mais Yahweh eut compassion d'eux, leur fit miséricorde, il tourna sa face vers eux par amour pour son alliance avec Abraham, Isaac et Jacob, de sorte qu'il ne voulut point les exterminer, et il ne les rejeta pas de sa face, jusqu'à maintenant.
\VS{24}Puis Hazaël, roi de Syrie, mourut, et Ben-Hadad, son fils, régna à sa place.
\VS{25}Joas, fils de Joachaz, reprit des mains de Ben-Hadad, fils d'Hazaël, les villes enlevées par Hazaël, à Joachaz, son père, pendant la guerre. Joas le battit trois fois et recouvra les villes d'Israël.
\Chap{14}
\TextTitle{Amatsia règne sur Juda\FTNTT{2 Ch. 25:1-4.}}
\VerseOne{}La deuxième année de Joas, fils de Joachaz, roi d'Israël, Amatsia, fils de Joas, roi de Juda, commença à régner.
\VS{2}Il était âgé de vingt-cinq ans lorsqu'il commença à régner, et il régna vingt-neuf ans à Jérusalem. Sa mère s'appelait Joaddan, elle était de Jérusalem.
\VS{3}Il fit ce qui est droit aux yeux de Yahweh, non pas toutefois comme David, son père~; il agit entièrement comme avait agi Joas, son père.
\VS{4}Seulement, les hauts lieux ne furent point ôtés~; le peuple offrait encore des sacrifices et des parfums sur les hauts lieux.
\VS{5}Et il arriva que dès que le royaume fut affermi entre ses mains, il frappa ses serviteurs qui avaient tué le roi, son père.
\VS{6}Mais il ne fit point mourir les fils des meurtriers, suivant ce qui est écrit dans le livre de la loi de Moïse, où Yahweh donne ce commandement~: On ne fera point mourir les pères pour les enfants, et l'on ne fera pas mourir les enfants pour les pères~; mais on fera mourir chacun pour son péché\FTNT{De. 24:16~; Ez. 18:4,20.}.
\VS{7}Il frappa dix mille hommes d'Edom dans la vallée du sel~; et il prit Séla durant la guerre, et l'appela Joktheel, nom qu'elle a conservé jusqu'à ce jour.
\VS{8}Alors Amatsia envoya des messagers vers Joas, fils de Joachaz, fils de Jéhu, roi d'Israël, pour lui dire~: Viens, voyons-nous en face~!
\VS{9}Et Joas, roi d'Israël, envoya dire à Amatsia, roi de Juda~: L'épine du Liban envoya dire au cèdre du Liban~: Donne ta fille en mariage à mon fils~! Et les bêtes sauvages qui sont au Liban passèrent et foulèrent l'épine.
\VS{10}Parce que tu as frappé et ravagé Edom, ton cœur s'est élevé. Contente-toi de ta gloire et reste dans ta maison. Pourquoi exciterais-tu le mal par lequel tu tomberas, toi et Juda avec toi~?
\VS{11}Mais Amatsia ne l'écouta pas. Et Joas, roi d'Israël, monta~: Et ils s'affrontèrent, lui et Amatsia, roi de Juda, à Beth-Schémesch, qui est à Juda.
\VS{12}Juda fut battu par Israël, et ils s'enfuirent chacun dans leurs tentes.
\VS{13}Joas, roi d'Israël, prit Amatsia, roi de Juda, fils de Joas, fils d'Achazia, à Beth-Schémesch. Puis il vint à Jérusalem et fit une brèche de quatre cents coudées dans la muraille de Jérusalem, depuis la porte d'Ephraïm, jusqu'à la porte de l'angle.
\VS{14}Il prit tout l'or et tout l'argent et tous les vases qui se trouvaient dans la maison de Yahweh et dans les trésors de la maison royale~; il prit aussi des enfants en otages, et il retourna à Samarie.
\TextTitle{Jéroboam II règne sur Israël}
\VS{15}Le reste des actions de Joas, ses exploits, et comment il combattit contre Amatsia, tout cela n'est-il pas écrit dans le livre des Chroniques des rois d'Israël~?
\VS{16}Et Joas se coucha avec ses pères et fut enseveli à Samarie avec les rois d'Israël. Et Jéroboam, son fils, régna à sa place.
\TextTitle{Mort d'Amatsia~; Azaria (Ozias) règne sur Juda (2 Ch. 25:26-28)}
\VS{17}Amatsia, fils de Joas, roi de Juda, vécut quinze ans après la mort de Joas, fils de Joachaz, roi d'Israël.
\VS{18}Le reste des actions d'Amatsia n'est-il pas écrit dans le livre des Chroniques des rois de Juda~?
\VS{19}On forma une conspiration contre lui à Jérusalem, et il s'enfuit à Lakis~; mais on le poursuivit à Lakis, où on le fit mourir.
\VS{20}On le transporta sur des chevaux, et il fut enseveli à Jérusalem avec ses pères, dans la cité de David.
\VS{21}Alors tout le peuple de Juda prit Azaria, âgé de seize ans, et ils l'établirent roi à la place d'Amatsia, son père.
\VS{22}Azaria bâtit Elath et la fit rentrer sous la puissance de Juda, après que le roi se coucha avec ses pères.
\TextTitle{Prophétie de Jonas accomplie par Jéroboam II}
\VS{23}La quinzième année d'Amatsia, fils de Joas, roi de Juda, Jéroboam, fils de Joas, commença à régner sur Israël à Samarie, et il régna quarante et un ans.
\VS{24}Il fit ce qui est mal aux yeux de Yahweh, et ne se détourna d'aucun des péchés de Jéroboam, fils de Nebath, par lesquels il avait fait pécher Israël.
\VS{25}Il rétablit les frontières d'Israël depuis l'entrée de Hamath, jusqu'à la mer de la plaine, selon la parole de Yahweh, le Dieu d'Israël, qu'il avait prononcée par son serviteur Jonas\FTNT{Jon. 1:1.}, fils d'Amitthaï, le prophète, de Gath-Hépher.
\VS{26}Car Yahweh vit que l'affliction d'Israël était à son comble, et l'extrémité à laquelle se trouvaient réduits esclaves et hommes libres, sans qu'il n'y ait personne pour venir au secours d'Israël.
\VS{27}Or Yahweh n'avait point résolu d'effacer le nom d'Israël de dessous les cieux, à cause de cela, il les délivra par les mains de Jéroboam, fils de Joas.
\TextTitle{Zacharie règne sur Israël}
\VS{28}Le reste des actions de Jéroboam, tout ce qu'il a fait, ses exploits de guerre, et comment il reconquit pour Israël, Damas et Hamath qui avaient appartenu à Juda, cela n'est-il pas écrit dans le livre des Chroniques des rois d'Israël~?
\VS{29}Puis Jéroboam se coucha avec ses pères, avec les rois d'Israël. Et Zacharie, son fils, régna à sa place.
\Chap{15}
\TextTitle{Juda demeure dans l'idolâtrie sous le règne d'Azaria (Ozias)\FTNTT{2 R. 14:21-22~; 2 Ch. 26:1-15.}}
\VerseOne{}La vingt-septième année de Jéroboam, roi d'Israël, Azaria\FTNT{Azaria (Ozias, selon 2 Ch. 26:1-15~; à ne pas confondre avec le prophète du même nom que son grand-père avait fait assassiner) fut couronné à l'âge de seize ans et mourut à l'âge de soixante-huit ans.}, fils d'Amatsia, roi de Juda, régna.
\VS{2}Il était âgé de seize ans lorsqu'il commença à régner, et il régna cinquante-deux ans à Jérusalem. Sa mère s'appelait Jecolia, elle était de Jérusalem.
\VS{3}Il fit ce qui est droit aux yeux de Yahweh, entièrement comme avait fait Amatsia, son père.
\VS{4}Seulement, les hauts lieux ne disparurent pas~; le peuple offrait encore des sacrifices et des parfums sur les hauts lieux.
\TextTitle{Jugement de Yahweh sur Ozias par la lèpre\FTNTT{2 Ch. 26:16-21.}}
\VS{5}Alors Yahweh frappa le roi, qui fut lépreux jusqu'au jour de sa mort, et il demeura dans une maison à l'écart. Et Jotham, fils du roi, avait la charge de la maison, jugeant le peuple du pays.
\VS{6}Le reste des actions d'Azaria, tout ce qu'il a fait, cela n'est-il pas écrit dans le livre des Chroniques des rois de Juda~?
\VS{7}Azaria se coucha avec ses pères, et fut enseveli avec ses pères dans la cité de David, et Jotham, son fils, régna à sa place.
\TextTitle{Conspiration de Schallum contre Zacharie, roi d'Israël}
\VS{8}La trente-huitième année d'Azaria, roi de Juda, Zacharie, fils de Jéroboam, commença à régner sur Israël à Samarie, et il régna six mois.
\VS{9}Il fit ce qui est mal aux yeux de Yahweh, comme avaient fait ses pères~; il ne se détourna point des péchés de Jéroboam, fils de Nebath, par lesquels il avait fait pécher Israël.
\VS{10}Schallum, fils de Jabesch, fit une conspiration contre lui, et le frappa devant le peuple. Il le tua, et régna à sa place.
\VS{11}Quant au reste des actions de Zacharie, voilà, elles sont écrites dans le livre des Chroniques des rois d'Israël.
\VS{12}Ainsi s'accomplit la parole que Yahweh avait déclarée à Jéhu, en disant~: Tes fils seront assis sur le trône d'Israël jusqu'à la quatrième génération, et il en fut ainsi\FTNT{2 R. 10:30.}
\TextTitle{Schallum règne sur Israël~; sa mort}
\VS{13}Schallum, fils de Jabesch, commença à régner la trente-neuvième année d'Ozias, roi de Juda. Il régna pendant un mois à Samarie.
\VS{14}Menahem, fils de Gadi, monta de Thirtsa et vint dans Samarie, et frappa à Samarie, Schallum, fils de Jabesch, et le fit mourir~; et il régna à sa place.
\VS{15}Le reste des actions de Schallum, et la conspiration qu'il forma, cela est écrit dans le livre des Chroniques des rois d'Israël.
\TextTitle{Menahem règne sur Israël}
\VS{16}Alors Menahem frappa Thiphsach et tous ceux qui y étaient, avec son territoire depuis Thirtsa~; il la frappa parce qu'elle ne lui avait point ouvert ses portes. Il fendit le ventre de toutes les femmes enceintes.
\VS{17}La trente-neuvième année d'Azaria, roi de Juda, Menahem, fils de Gadi, commença à régner sur Israël. Il régna dix ans à Samarie.
\VS{18}Il fit ce qui est mal aux yeux de Yahweh~; il ne se détourna point des péchés de Jéroboam, fils de Nebath, par lesquels il avait fait pécher Israël.
\TextTitle{Invasion d'Israël par le roi d'Assyrie\FTNTT{1 Ch. 5:26.}}
\VS{19}Alors Pul, roi d'Assyrie, vint contre le pays~; et Menahem donna mille talents d'argent à Pul, afin qu'il l'aide à affermir son royaume entre ses mains.
\VS{20}Menahem leva cet argent sur tous ceux d'Israël qui avaient de la richesse pour le donner au roi d'Assyrie~; chacun cinquante sicles d'argent. Ainsi, le roi d'Assyrie s'en retourna, et ne s'arrêta point dans le pays.
\VS{21}Le reste des actions de Menahem, tout ce qu'il a fait, cela n'est-il pas écrit dans le livre des Chroniques des rois d'Israël~?
\TextTitle{Mort de Menahem~; Pekachia règne sur Israël}
\VS{22}Menahem se coucha avec ses pères, et Pekachia, son fils, régna à sa place.
\VS{23}La cinquantième année d'Azaria, roi de Juda, Pekachia, fils de Menahem, commença à régner sur Israël à Samarie. Il régna deux ans.
\VS{24}Il fit ce qui est mal aux yeux de Yahweh~; il ne se détourna point des péchés de Jéroboam, fils de Nebath, par lesquels il avait fait pécher Israël.
\TextTitle{Pékach tue Pekachia et devient roi d'Israël}
\VS{25}Pékach, fils de Remalia, son officier, conspira contre lui~; il le frappa à Samarie, dans le palais de la maison royale, de même qu'Argob et Arié~; il avait avec lui cinquante hommes d'entre les fils des Galaadites. Il fit ainsi mourir Pekachia, et il régna à sa place.
\VS{26}Le reste des actions de Pekachia tout ce qu'il a fait, cela est écrit dans le livre des Chroniques des rois d'Israël.
\VS{27}La cinquante-deuxième année d'Azaria, roi de Juda, Pékach, fils de Remalia, commença à régner sur Israël à Samarie. Il régna vingt ans.
\VS{28}Il fit ce qui est mal aux yeux de Yahweh et ne se détourna point des péchés de Jéroboam, fils de Nebath, par lesquels il avait fait pécher Israël.
\VS{29}Du temps de Pékach, roi d'Israël, Tiglath-Piléser, roi d'Assyrie, vint et prit Ijjon, Abel-Beth-Maaca, Janoach, Kédesch, Hatsor, Galaad et la Galilée, et même tout le pays de Nephthali, et il emmena captifs les habitants en Assyrie.
\TextTitle{Osée conspire contre Pékach et règne sur Israël}
\VS{30}Osée, fils d'Ela, forma une conspiration contre Pékach, fils de Remalia, le frappa et le fit mourir. Il régna à sa place la vingtième année de Jotham, fils d'Ozias.
\VS{31}Le reste des actions de Pékach, tout ce qu'il a fait, cela est écrit dans le livre des Chroniques des rois d'Israël.
\TextTitle{Jotham règne sur Juda~; sa mort\FTNTT{2 R. 15:2~; 2 Ch. 26:23~; 27:1-9.}}
\VS{32}La seconde année de Pékach, fils de Remalia, roi d'Israël, Jotham, fils d'Ozias, roi de Juda, commença à régner.
\VS{33}Il était âgé de vingt-cinq ans lorsqu'il commença à régner. Il régna seize ans à Jérusalem. Sa mère s'appelait Jeruscha, fille de Tsadok.
\VS{34}Il fit ce qui est droit aux yeux de Yahweh~; il agit entièrement comme avait agi Ozias, son père.
\VS{35}Seulement, les hauts lieux ne disparurent point~; et le peuple offrait encore des sacrifices et des parfums sur les hauts lieux. Jotham bâtit la porte supérieure de la maison de Yahweh.
\VS{36}Le reste des actions de Jotham, tout ce qu'il a fait, cela n'est-il pas écrit dans le livre des Chroniques des rois de Juda~?
\VS{37}Dans ce temps-là, Yahweh commença à envoyer contre Juda, Retsin, roi de Syrie, et Pékach, fils de Remalia.
\VS{38}Jotham se coucha avec ses pères, et il fut enseveli dans la cité de David, son père. Et Achaz, son fils, régna à sa place.
\Chap{16}
\TextTitle{Achaz règne sur Juda\FTNTT{2 R. 15:38~; 2 Ch. 28:1-4.}}
\VerseOne{}La dix-septième année de Pékach, fils de Remalia, Achaz, fils de Jotham, roi de Juda, commença à régner.
\VS{2}Achaz était âgé de vingt ans lorsqu'il commença à régner. Il régna seize ans à Jérusalem. Il ne fit point ce qui est droit aux yeux de Yahweh, son Dieu, comme avait fait David, son père.
\VS{3}Mais il suivit la voie des rois d'Israël et il fit même passer son fils par le feu, selon les abominations des nations que Yahweh avait chassées devant les enfants d'Israël.
\VS{4}Il offrait aussi des sacrifices et des parfums sur les hauts lieux, sur les coteaux et sous tout arbre vert.
\TextTitle{Juda envahi par les rois d'Assyrie et d'Israël\FTNTT{2 Ch. 28:5-19.}}
\VS{5}Alors Retsin, roi de Syrie, et Pékach, fils de Remalia, roi d'Israël, montèrent contre Jérusalem pour lui faire la guerre. Ils assiégèrent Achaz~; mais ne purent en venir à bout par les armes.
\VS{6}Dans ce même temps, Retsin, roi de Syrie, fit rentrer Elath au pouvoir des Syriens~; il expulsa les Juifs d'Elath, et les Syriens vinrent à Elath, où ils ont demeuré jusqu'à ce jour.
\TextTitle{Le roi d'Assyrie vient en aide à Achaz et s'empare de Damas\FTNTT{2 Ch. 28:16-25.}}
\VS{7}Achaz envoya des messagers à Tiglath-Piléser, roi d'Assyrie, pour lui dire~: Je suis ton serviteur et ton fils~; monte et délivre-moi de la main du roi des Syriens, et de la main du roi d'Israël, qui s'élèvent contre moi.
\VS{8}Alors Achaz prit l'argent et l'or qui se trouvaient dans la maison de Yahweh, et dans les trésors de la maison royale, et il les envoya en présent au roi d'Assyrie.
\VS{9}Le roi d'Assyrie l'écouta~; il monta contre Damas, la prit, emmena les habitants en captivité à Kir et fit mourir Retsin.
\VS{10}Alors le roi Achaz s'en alla à la rencontre de Tiglath-Piléser, roi d'Assyrie, à Damas. Et ayant vu l'autel\FTNT{Achaz, roi de Juda, se rendit chez le roi d'Assyrie et il fut fasciné par l'autel de son dieu au point de le convoiter. Il demanda au prêtre Urie de fabriquer un autel identique, dont le modèle n'était pas celui que Yahweh avait décrit à Moïse. Il introduisit un objet de culte d'origine païenne dans le temple de Jérusalem, sous prétexte d'honorer Yahweh. Certains «~Pères de l'Eglise~», comme les empereurs Constantin I (285-337) et Théodose I (347-395), se sont comportés exactement comme Achaz en adoptant les pratiques païennes. Les historiens s'accordent pour dire que la diffusion de la Parole de Dieu sous l'empire de Constantin I (285-337), empereur de Rome, avait des fins strictement politiques. Cette politique a eu deux conséquences essentielles concernant l'influence de l'Eglise chrétienne et son fonctionnement de plus en plus éloigné de la Parole de Dieu~:\\- Les peuples païens ont introduit leurs rites idolâtres au sein de l'Eglise. En effet, les dogmes de l'institution devaient plaire à la majorité.\\- L'Eglise chrétienne cessant d'être persécutée, son fonctionnement intimiste fondé sur l'implication de chaque croyant et l'exercice de la prêtrise universelle des chrétiens, a changé à cause de l'effet de masse. Devenant numériquement très importante, il a fallu imposer une autorité capable de contenir un nombre de fidèles de plus en plus élevé. Mais à cause de cette augmentation numérique et de la présence de «~faux convertis~» lié au fait que l'adhésion au christianisme (religion chrétienne fondée par les hommes) devenait une obligation, l'étude de la Parole, la fraction du pain et la prière ne pouvaient plus perdurer. C'est ainsi que beaucoup d'églises ont commencé à subir l'influence du monde.} qui était à Damas, le roi Achaz envoya au prêtre Urie, la forme et le modèle exact de cet autel.
\VS{11}Le prêtre Urie construisit un autel entièrement d'après le modèle envoyé de Damas par le roi Achaz, et le prêtre Urie le fit avant que le roi Achaz soit de retour de Damas.
\VS{12}Quand le roi Achaz revint de Damas et vit l'autel, il s'en approcha et y monta~;
\VS{13}Il fit brûler son holocauste et son sacrifice, versa ses libations et répandit sur l'autel le sang de ses sacrifices d'offrande de paix\FTNT{Voir commentaire en Lé. 3:1.}.
\VS{14}Il éloigna de la face de la maison l'autel d'airain qui était devant Yahweh, afin qu'il ne soit pas entre le nouvel autel et la maison de Yahweh~; et il le plaça à côté du nouvel autel, vers le nord.
\VS{15}Et le roi Achaz donna cet ordre au prêtre Urie~: Fais brûler l'holocauste du matin et l'offrande du soir, l'holocauste du roi et son offrande, les holocaustes de tout le peuple du pays et leurs offrandes, verses-y leurs libations, et répands-y tout le sang des holocaustes et tout le sang des sacrifices~; mais pour ce qui concerne l'autel d'airain, je m'en occuperai.
\VS{16}Le prêtre Urie, exécuta tout ce que le roi Achaz lui avait ordonné.
\VS{17}Le roi Achaz brisa les panneaux des bases et en ôta les cuves qui étaient dessus. Il descendit la mer de dessus les bœufs d'airain qui étaient sous elle et il la posa sur un pavé de pierre.
\VS{18}Il changea aussi dans la maison de Yahweh, à cause du roi d'Assyrie, le portique du sabbat qu'on y avait bâti et l'entrée extérieure du roi.
\TextTitle{Mort d'Achaz~; Ezéchias devient roi de Juda\FTNTT{2 Ch. 28:26-27.}}
\VS{19}Le reste des actions d'Achaz et tout ce qu'il a fait, cela n'est-il pas écrit dans le livre des Chroniques des rois de Juda~?
\VS{20}Achaz se coucha avec ses pères, et il fut enseveli avec ses pères dans la cité de David. Et Ezéchias, son fils, régna à sa place.
\Chap{17}
\TextTitle{Osée devient le dernier roi d'Israël}
\VerseOne{}La douzième année d'Achaz, roi de Juda, Osée, fils d'Ela, régna à Samarie sur Israël. Il régna neuf ans.
\VS{2}Il fit ce qui est mal aux yeux de Yahweh, non pas toutefois comme les rois d'Israël qui avaient été avant lui.
\TextTitle{Osée tente de s'affranchir du joug de l'Assyrie}
\VS{3}Salmanasar\FTNT{Le royaume d'Israël a été détruit en 722 av. J.-C., par l'empereur assyrien Salmanasar V (règne~: 727-722 av. J.-C.), après avoir assiégé trois ans le roi Osée (règne~: 732-722 av. J.-C.) dans sa capitale Samarie. Celui-ci ne payait plus le tribut et essayait d'obtenir l'appui de l'Egypte pour retrouver l'indépendance. Le royaume d'Israël a disparu au début du 8ème siècle av. J.-C., provoquant la dispersion dans le monde de plusieurs juifs issus des dix tribus. L'origine des Samaritains remonte à cette déportation, après que le royaume du Nord soit tombé aux mains de Salmanasar, roi d'Assyrie. Malgré les déportations, les Assyriens n'avaient pas laissé déserte cette région appelée «~Samarie~»~; plusieurs Israélites y étaient restés et des colons d'autres provinces assyriennes vinrent s'y établir. Les Samaritains sont issus du mélange de ces populations, et leur religion est un mélange entre le culte à Yahweh avec celui des dieux étrangers.}, roi d'Assyrie, monta contre lui~; et Osée lui fut assujetti et lui paya un tribut.
\VS{4}Mais le roi d'Assyrie découvrit une conspiration chez Osée, qui avait envoyé des messagers vers So, roi d'Egypte, et qui ne payait plus le tribut tous les ans au roi d'Assyrie. C'est pourquoi le roi d'Assyrie le fit enfermer et enchaîner dans une prison.
\TextTitle{Siège de Samarie par le roi d'Assyrie}
\VS{5}Le roi d'Assyrie parcourut tout le pays et monta contre Samarie qu'il assiégea pendant trois ans.
\TextTitle{Les causes de la captivité d’Israël par l'Assyrie}
\VS{6}La neuvième année d'Osée, le roi d'Assyrie prit Samarie et emmena captifs les Israélites en Assyrie. Il les fit habiter à Chalach, et sur le Chabor, fleuve de Gozan, et dans les villes des Mèdes.
\VS{7}Cela arriva parce que les enfants d'Israël péchèrent contre Yahweh, leur Dieu, qui les avait fait monter hors du pays d'Egypte, de dessous la main de Pharaon, roi d'Egypte, et parce qu'ils craignirent d'autres dieux.
\VS{8}Ils suivirent les coutumes des nations que Yahweh avait chassées devant les enfants d'Israël, et celles des rois d'Israël qu'ils avaient établis.
\VS{9}Les enfants d'Israël firent en secret des choses qui n'étaient point droites, contre Yahweh, leur Dieu. Ils se bâtirent des hauts lieux dans toutes leurs villes, depuis la tour des gardes jusqu'aux villes fortes.
\VS{10}Ils se dressèrent des statues et des Asherah sur toutes les hautes collines et sous tout arbre vert.
\VS{11}Et là, ils brûlèrent des parfums sur tous les hauts lieux, comme les nations que Yahweh avait chassées devant eux, et ils firent des choses mauvaises pour irriter Yahweh.
\VS{12}Ils servirent les idoles, au sujet desquelles Yahweh leur avait dit~: Vous ne ferez pas cela\FTNT{1 R. 12:28.}.
\VS{13}Yahweh fit avertir Israël et Juda par tous ses prophètes, tous les voyants, en disant~: Détournez-vous de toutes vos mauvaises voies, revenez, et gardez mes commandements et mes ordonnances, en suivant entièrement la loi que j'ai prescrite à vos pères et que je vous ai envoyée par mes serviteurs les prophètes.
\VS{14}Mais ils n'écoutèrent point et raidirent leur cou, comme leurs pères avaient raidi leur cou, et n'avaient pas cru en Yahweh, leur Dieu.
\VS{15}Ils rejetèrent ses lois, et son alliance qu'il avait traitée avec leurs pères, et ses avertissements, qu'il leur avait adressés. Ils allèrent après des choses de néant et ne furent eux-mêmes que néant, après les nations qui les entouraient et que Yahweh leur avait défendu d'imiter.
\VS{16}Ils abandonnèrent tous les commandements de Yahweh, leur Dieu, ils firent deux veaux en métal fondu, ils fabriquèrent des idoles d'Asherah\FTNT{Voir commentaire en Jg. 2:13.}, ils se prosternèrent devant toute l'armée des cieux et ils servirent Baal.
\VS{17}Ils firent aussi passer leurs fils et leurs filles par le feu, ils s'adonnèrent à la divination et aux enchantements, et ils se vendirent pour faire ce qui est mal aux yeux de Yahweh afin de l'irriter.
\VS{18}C'est pourquoi, Yahweh fut très irrité contre Israël et il les rejeta~; il n'est resté que la seule tribu de Juda.
\VS{19}Même Juda n'avait pas gardé les commandements de Yahweh, son Dieu, mais ils avaient suivi les ordonnances qu'Israël avait établies.
\VS{20}C'est pourquoi Yahweh rejeta toute la race d'Israël~; il les a humiliés, il les a livrés entre les mains des pillards, il a fini par les chasser loin de sa face.
\VS{21}Car Israël s'était détaché de la maison de David, et avait établi roi Jéroboam, fils de Nebath. Jéroboam avait détourné Israël de Yahweh, afin qu'il ne le suive plus, et lui avait fait commettre un grand péché.
\VS{22}C'est pourquoi les enfants d'Israël s'étaient livrés à tous les péchés que Jéroboam avait commis~; ils ne s'en détournèrent pas,
\VS{23}jusqu'à ce que Yahweh ait chassé Israël de devant sa face, comme il l'avait annoncé par tous ses serviteurs les prophètes. Et Israël fut emmené captif loin de son pays en Assyrie, jusqu'à ce jour.
\TextTitle{Jugement sur les étrangers occupant les villes d'Israël}
\VS{24}Le roi d'Assyrie fit venir des gens de Babylone, de Cutha, d'Avva, de Hamath et de Sepharvaïm. Il les fit habiter dans les villes de Samarie, à la place des enfants d'Israël. Ils prirent possession de la Samarie et habitèrent dans ses villes.
\VS{25}Lorsqu'ils commencèrent à y habiter, ils ne craignirent point Yahweh, et Yahweh envoya contre eux des lions, qui les tuaient.
\VS{26}Et on dit au roi d'Assyrie~: Les nations que tu as transportées et fait habiter dans les villes de Samarie ne connaissent pas la manière de servir le dieu du pays, c'est pourquoi il a envoyé contre eux des lions, et voilà, ces lions les tuent, parce qu'ils ne connaissent pas la manière de servir le dieu du pays.
\TextTitle{L'idolâtrie dans les villes occupées}
\VS{27}Alors le roi d'Assyrie donna cet ordre, en disant~: Faites-y aller quelqu'un des prêtres que vous avez emmenés de là en captivité~; qu'il parte pour s'y établir et qu'il leur enseigne la manière de servir le dieu du pays.
\VS{28}Alors l'un des prêtres, qui avaient été emmenés captifs de Samarie vint s'établir à Béthel et leur enseigna comment ils devaient craindre Yahweh.
\VS{29}Mais les nations firent chacune leurs dieux dans les villes qu'elles habitaient et les placèrent dans les maisons des hauts lieux bâties par les Samaritains.
\VS{30}Les gens de Babylone firent Succoth-Benoth, les gens de Cuth firent Nergal, et les gens de Hamath firent Aschima.
\VS{31}Ceux d'Avva firent Nibchaz et Thartak~; ceux de Sepharvaïm brûlaient leurs enfants par le feu à Adrammélec et Anammélec, les dieux de Sepharvaïm.
\VS{32}Toutefois, ils redoutaient Yahweh et ils établirent des prêtres des hauts lieux pris parmi tout le peuple~; ces prêtres offraient pour eux des sacrifices dans les maisons des hauts lieux.
\VS{33}Ils redoutaient Yahweh et en même temps, ils servaient leurs dieux à la manière des nations d'où on les avait transportés.
\VS{34}Et jusqu'à ce jour, ils font encore selon leurs premières coutumes~: Ils ne craignirent point Yahweh, et ils ne se conforment ni à leurs lois et à leurs ordonnances, ni à la loi et aux commandements prescrits par Yahweh Dieu aux enfants de Jacob, qu'il appela du nom d'Israël.
\VS{35}Yahweh avait traité alliance avec eux et leur avait donné cet ordre, en disant~: Vous ne craindrez point d'autres dieux~; et ne vous prosternerez point devant eux~; vous ne les servirez point et vous ne leur offrirez point de sacrifices.
\VS{36}Mais vous craindrez Yahweh, qui vous a fait monter hors du pays d'Egypte avec une grande puissance et à bras étendu~; et vous vous prosternerez devant lui, et vous lui offrirez des sacrifices.
\VS{37}Vous observerez et mettrez toujours en pratique les statuts, les ordonnances, la loi et les commandements, qu'il a écrits pour vous, et vous ne craindrez pas d'autres dieux.
\VS{38}Vous n'oublierez pas l'alliance que j'ai traitée avec vous et vous ne craindrez point d'autres dieux.
\VS{39}Mais vous craindrez Yahweh, votre Dieu, et il vous délivrera de la main de tous vos ennemis.
\VS{40}Ils n'écoutèrent pas et ils firent selon leurs premières coutumes.
\VS{41}Ainsi ces nations-là redoutaient Yahweh et servaient leurs images~; leurs enfants et les enfants de leurs enfants font jusqu'à ce jour ce que firent leurs pères.
\Chap{18}
\TextTitle{Ezéchias règne sur Juda\FTNTT{2 R. 16:20~; 2 Ch. 29:1-31:21.}}
\VerseOne{}La troisième année d'Osée, fils d'Ela, roi d'Israël, Ezéchias, fils d'Achaz, roi de Juda, commença à régner.
\VS{2}Il était âgé de vingt-cinq ans lorsqu'il commença à régner, il régna vingt-neuf ans à Jérusalem. Sa mère s'appelait Abi, fille de Zacharie.
\VS{3}Il fit ce qui est droit aux yeux de Yahweh, entièrement comme avait fait David, son père.
\TextTitle{Mouvement de réveil sous Ezéchias\FTNTT{2 Ch. 29:3-31:21.}}
\VS{4}Il fit disparaître les hauts lieux, mit en pièces les statues, abattit les idoles d'Asherah, et il brisa le serpent d'airain que Moïse avait fait, car les enfants d'Israël avaient jusqu'alors brûlé des parfums devant lui~; ils l'appelaient Nehuschtan.
\VS{5}Il se confia en Yahweh, le Dieu d'Israël~; et parmi tous les rois de Juda qui vinrent après ou qui le précédèrent, il n'y en eut point de semblable à lui.
\VS{6}Il s'attacha à Yahweh, il ne se détourna point de lui et il observa les commandements que Yahweh avait prescrits à Moïse.
\TextTitle{Révolte contre l'Assyrie~; victoire sur les Philistins}
\VS{7}Et Yahweh fut avec Ezéchias, qui réussit dans toutes ses entreprises. Il se révolta contre le roi d'Assyrie et ne lui fut plus assujetti.
\VS{8}Il frappa les Philistins jusqu'à Gaza et ravagea leur territoire depuis les tours des gardes jusqu'aux villes fortes.
\TextTitle{Captivité d'Israël par l'Assyrie\FTNTT{2 R. 17:4-6.}}
\VS{9}La quatrième année du roi Ezéchias, qui était la septième du règne d'Osée, fils d'Ela, roi d'Israël, Salmanasa, roi d'Assyrie, monta contre Samarie et l'assiégea.
\VS{10}Il la prit au bout de trois ans~; la sixième année du règne d'Ezéchias, qui était la neuvième d'Osée, roi d'Israël, Samarie fut prise.
\VS{11}Le roi d'Assyrie emmena Israël en Assyrie et il les établit à Chalach, sur le Chabor, fleuve de Gozan, et dans les villes des Mèdes,
\VS{12}parce qu'ils n'avaient point obéi à la voix de Yahweh, leur Dieu, et qu'ils avaient transgressé son alliance, parce qu'ils n'avaient ni écouté ni mis en pratique tout ce qu'avait ordonné Moïse, serviteur de Yahweh.
\TextTitle{Invasion de Juda par Sanchérib\FTNTT{2 Ch. 32:1-15,30~; Es. 36:1-10.}}
\VS{13}La quatorzième année du roi Ezéchias, Sanchérib, roi d'Assyrie, monta contre toutes les villes fortes de Juda et les prit.
\VS{14}Ezéchias, roi de Juda, envoya dire au roi d'Assyrie à Lakis~: J'ai commis une faute~! Eloigne-toi de moi. Je payerai tout ce que tu m'imposeras. Et le roi d'Assyrie imposa à Ezéchias, roi de Juda, trois cents talents d'argent et trente talents d'or.
\VS{15}Ezéchias donna tout l'argent qui se trouvait dans la maison de Yahweh et dans les trésors de la maison royale.
\VS{16}En ce temps-là, Ezéchias enleva les lames d'or dont il avait couvert les portes et les linteaux du temple de Yahweh, pour les livrer au roi d'Assyrie.
\VS{17}Puis le roi d'Assyrie envoya de Lakis à Jérusalem, vers le roi Ezéchias, Tharthan, Rab-Saris et Rabschaké avec une puissante armée. Ils montèrent et arrivèrent à Jérusalem. Lorsqu'ils furent montés et arrivés~; ils s'arrêtèrent à l'aqueduc de l'étang supérieur, qui est sur le chemin du champ du foulon.
\VS{18}Ils appelèrent le roi tout haut~; alors Eliakim, fils de Hilkija, chef de la maison du roi, Schebna, le secrétaire et Joach, fils d'Asaph, l'archiviste, se rendirent auprès d'eux.
\VS{19}Rabschaké leur dit~: Dites maintenant à Ezéchias~: Ainsi parle le grand roi, le roi d'Assyrie~: Quelle est cette confiance sur laquelle tu t'appuies~?
\VS{20}Tu as dit~: Il faut pour la guerre le conseil et la force. Mais ce ne sont que des paroles. Mais en qui donc as-tu placé ta confiance, pour te rebeller contre moi~?
\VS{21}Voici maintenant, tu l'as placée dans l'Egypte, dans ce roseau cassé, qui pénètre et perce la main de quiconque s'appuie dessus~: Tel est Pharaon, roi d'Egypte, pour tous ceux qui se confient en lui.
\VS{22}Peut-être me direz-vous~: Nous nous confions en Yahweh, notre Dieu, mais n'est-ce pas celui dont Ezéchias a détruit les hauts lieux et les autels, en disant à Juda et à Jérusalem~: Vous vous prosternerez devant cet autel à Jérusalem~?
\VS{23}Maintenant, donne des otages au roi d'Assyrie, mon maître, et je te donnerai deux mille chevaux, si tu peux donner autant de cavaliers pour les monter.
\VS{24}Comment donc repousserais-tu un seul gouverneur d'entre les serviteurs de mon maître~? Mais tu mets ta confiance dans l'Egypte, à cause des chars et des cavaliers.
\VS{25}D'ailleurs, est-ce sans l'ordre de Yahweh que je suis monté contre ce lieu, pour le détruire~? Yahweh m'a dit~: Monte contre ce pays et détruis-le.
\TextTitle{Menaces de Rabschaké\FTNTT{2 Ch. 32:16,18-19~; Es. 36:11-21.}}
\VS{26}Alors Eliakim, fils de Hilkija, Schebna et Joach dirent à Rabschaké~: Nous te prions de parler en araméen à tes serviteurs, car nous le comprenons~; et ne nous parle pas en langue judaïque, aux oreilles du peuple qui est sur la muraille.
\VS{27}Rabschaké leur répondit~: Est-ce à ton maître et à toi que mon maître m'a envoyé dire ces paroles~? Ne m'a-t-il pas envoyé vers les hommes qui se tiennent sur la muraille pour leur dire qu'ils mangeront leurs propres excréments et qu'ils boiront leur urine avec vous~?
\VS{28}Rabschaké, s'étant avancé, cria à haute voix en langue judaïque, il parla et dit~: Ecoutez la parole du grand roi, le roi d'Assyrie.
\VS{29}Ainsi parle le roi~: Qu'Ezéchias ne vous trompe pas, car il ne pourra pas vous délivrer de ma main.
\VS{30}Qu'Ezéchias ne vous amène pas à vous confier en Yahweh, en disant~: Yahweh nous délivrera certainement et cette ville ne sera pas livrée entre les mains du roi d'Assyrie.
\VS{31}N'écoutez pas Ezéchias~; car ainsi parle le roi d'Assyrie~: Faites la paix avec moi et rendez-vous à moi~; et chacun de vous mangera de sa vigne, et de son figuier, et chacun boira de l'eau de sa citerne,
\VS{32}jusqu'à ce que je vienne et que je vous emmène dans un pays comme le vôtre, dans un pays de blé et de bon vin, un pays de pain et de vignes, un pays d'oliviers qui portent de l'huile, et de miel, et vous vivrez et vous ne mourrez pas. Mais n'écoutez pas Ezéchias~; car il pourrait vous séduire, en disant~: Yahweh nous délivrera.
\VS{33}Les dieux des nations ont-ils délivré chacun leur pays de la main du roi d'Assyrie~?
\VS{34}Où sont les dieux de Hamath et d'Arpad~? Où sont les dieux de Sépharvaïm, d'Héna et d'Ivva~? Et même ont-ils délivré Samarie de ma main~?
\VS{35}Parmi tous les dieux de ces pays, quels sont ceux qui ont délivré leur pays de ma main, pour dire que Yahweh délivrera Jérusalem de ma main~?
\VS{36}Le peuple se tut et on ne lui répondit pas un mot~; car le roi avait donné cet ordre~: Vous ne lui répondrez point.
\VS{37}Après cela, Eliakim, fils de Hilkija, chef de la maison du roi, et Schebna le secrétaire, et Joach, fils d'Asaph, l'archiviste, vinrent auprès d'Ezéchias, les vêtements déchirés, et ils lui rapportèrent les paroles de Rabschaké.
\Chap{19}
\TextTitle{Ezéchias demande à Esaïe de consulter Yahweh\FTNTT{2 Ch. 32:20-22~; Es. 36:22-37:5.}}
\VerseOne{}Et il arriva qu'aussitôt que le roi Ezéchias entendit ces choses, il déchira ses vêtements, se couvrit d'un sac et entra dans la maison de Yahweh.
\VS{2}Puis il envoya Eliakim, chef de la maison du roi, et Schebna le secrétaire, et les plus anciens des prêtres, couverts de sacs, vers Esaïe, le prophète, fils d'Amots.
\VS{3}Ils lui dirent~: Ainsi parle Ezéchias~: Ce jour est un jour d'angoisse, de châtiment et d'opprobre~; car les enfants sont près du sein maternel, mais il n'y a point de force pour enfanter.
\VS{4}Peut-être Yahweh, ton Dieu, a-t-il entendu toutes les paroles de Rabschaké, que le roi d'Assyrie, son maître, a envoyé pour blasphémer le Dieu vivant, et peut-être Yahweh, ton Dieu, exercera-t-il ses châtiments à cause des paroles qu'il a entendues. Fais donc une prière pour le reste qui subsiste encore.
\VS{5}Les serviteurs du roi Ezéchias vinrent donc vers Esaïe.
\TextTitle{Réponse de Yahweh\FTNTT{Es. 37:6-7.}}
\VS{6}Et Esaïe leur dit~: Voici ce que vous direz à votre maître~: Ainsi parle Yahweh~: Ne t'effraie point des paroles que tu as entendues, par lesquelles les serviteurs du roi d'Assyrie m'ont blasphémé.
\VS{7}Voici, je vais mettre en lui un esprit, tel que sur une nouvelle qu'il recevra, il retournera dans son pays~; et je le ferai tomber par l'épée dans son pays.
\TextTitle{Défi du roi d'Assyrie au Dieu d'Israël\FTNTT{2 Ch. 32:17~; Es. 37:8-13.}}
\VS{8}Rabschaké s'étant retiré, trouva le roi d'Assyrie qui attaquait Libna, car il avait appris qu'il était parti de Lakis.
\VS{9}Le roi d'Assyrie reçut une nouvelle au sujet de Tirhaka, roi d'Ethiopie~; on lui dit~: Voici, il est sorti pour te combattre. C'est pourquoi le roi d'Assyrie retourna dans son pays, mais il envoya des messagers à Ezéchias, en leur disant~:
\VS{10}Vous parlerez ainsi à Ezéchias, roi de Juda, et lui direz~: Que ton Dieu, en qui tu te confies, ne t'abuse pas en te disant~: Jérusalem ne sera point livrée entre les mains du roi d'Assyrie.
\VS{11}Voici, tu as entendu ce que les rois d'Assyrie ont fait à tous les pays, et comment ils les ont détruits entièrement~; et tu échapperais~?
\VS{12}Les dieux des nations que mes ancêtres ont détruites, savoir de Gozan, de Charan, de Retseph, et des fils d'Eden, qui sont en Telassar, les ont-ils délivrées~?
\VS{13}Où sont le roi de Hamath, le roi d'Arpad, et le roi de la ville de Sepharvaïm, d'Héna et d'Ivva~?
\TextTitle{Ezéchias dans le temple, sa prière à Yawheh\FTNTT{2 Ch. 32:20~; Es. 37:14-20.}}
\VS{14}Quand Ezéchias reçut la lettre de la main des messagers, il la lut. Puis il monta à la maison de Yahweh et la déploya devant Yahweh~;
\VS{15}puis Ezéchias lui adressa cette prière et dit~: Ô Yahweh, Dieu d'Israël~! Qui est assis entre les chérubins, c'est toi qui es le seul Dieu de tous les royaumes de la terre, c'est toi qui as fait les cieux et la terre.
\VS{16}Ô Yahweh~! Incline ton oreille et écoute. Ouvre tes yeux et regarde. Ecoute les paroles de Sanchérib, et de celui qu’il a envoyé pour blasphémer le Dieu vivant.
\VS{17}Il est vrai, ô Yahweh~! Que les rois d'Assyrie ont détruit ces nations et ravagé leurs pays,
\VS{18}et qu'ils ont jeté dans le feu leurs dieux~; mais ils n'étaient pas des dieux, mais des ouvrages de mains d'homme, du bois, et de la pierre, c'est pourquoi ils les ont détruits.
\VS{19}Maintenant donc, ô Yahweh, notre Dieu~! Je te prie, délivre-nous de la main de Sanchérib, afin que tous les royaumes de la terre sachent que c'est toi, ô Yahweh, qui es le seul Dieu.
\TextTitle{Yahweh répond au travers d'Esaïe\FTNTT{Es. 37:21-35.}}
\VS{20}Alors Esaïe, fils d'Amots, envoya dire à Ezéchias~: Ainsi parle Yahweh, le Dieu d'Israël~: Je t'ai exaucé dans ce que tu m'as demandé au sujet de Sanchérib, roi d'Assyrie.
\VS{21}Voici la parole que Yahweh a prononcée contre lui~: Elle te méprise, elle se moque de toi, la fille, vierge de Sion~; elle hoche la tête après toi, la fille de Jérusalem.
\VS{22}Qui as-tu outragé et blasphémé~? Contre qui as-tu élevé la voix~? Tu as porté tes yeux en haut, vers le Saint d'Israël~!
\VS{23}Tu as insulté le Seigneur par le moyen de tes messagers et tu as dit~: J'ai gravi le sommet des montagnes avec la multitude de mes chars, les extrémités du Liban~; je couperai les plus hauts de ses cèdres et les plus beaux de ses cyprès, et j'atteindrai sa dernière cime, la forêt de son verger.
\VS{24}J'ai creusé des sources, après avoir bu les eaux étrangères et je tarirai avec la plante de mes pieds tous les fleuves de l'Egypte.
\VS{25}N'as-tu pas appris que j'ai préparé cette ville déjà dès longtemps, et que dès les temps anciens je l'ai ainsi formée~? Et maintenant l'aurais-je conservée pour être réduite en désolation, et les villes fortes, en monceaux de ruines~?
\VS{26}Il est vrai que leurs habitants sont impuissants, épouvantés et confus~; ils sont devenus comme l'herbe des champs et la tendre verdure, comme le gazon des toits et le blé brûlé avant la formation de sa tige.
\VS{27}Mais je connais ta demeure, ta sortie et ton entrée, et comment tu es furieux contre moi.
\VS{28}Parce que tu es furieux contre moi et que ton insolence est montée à mes oreilles, je mettrai ma boucle à tes narines, et mon mors entre tes lèvres, et je te ferai retourner par le chemin par lequel tu es venu.
\VS{29}Que ceci soit un signe pour toi, ô Ezéchias~: On mangera cette année le produit du grain tombé, et la deuxième année, ce qui croît de soi-même~; mais la troisième année, vous sèmerez et vous moissonnerez, vous planterez des vignes et vous en mangerez le fruit.
\VS{30}Ce qui aura été épargné de la maison de Juda, ce qui sera resté poussera encore des racines par-dessous et produira du fruit par-dessus.
\VS{31}Car il sortira de Jérusalem un reste, et de la montagne de Sion des réchappés. Voilà ce que fera le zèle de Yahweh des armées.
\VS{32}C'est pourquoi ainsi parle Yahweh, sur le roi d'Assyrie~: Il n'entrera point dans cette ville, il n'y lancera aucune flèche, il ne se présentera point contre elle avec le bouclier et il n'élèvera point des retranchements contre elle.
\VS{33}Il s'en retournera par le chemin par lequel il est venu et il n'entrera point dans cette ville, dit Yahweh.
\VS{34}Car je protègerai cette ville, afin de la délivrer, par amour pour moi et par amour pour David, mon serviteur.
\TextTitle{L'ange de Yahweh dans le camp des Assyriens\FTNTT{Es. 37:36-38.}}
\VS{35}Il arriva nuit-là que l'Ange de Yahweh sortit et frappa cent quatre-vingt-cinq mille hommes dans le camp des Assyriens. Et quand on se leva de bon matin, voici, ils étaient tous morts.
\TextTitle{Mort de Sanchérib, roi d'Assyrie\FTNTT{Es. 37:37-38~; 2 Ch. 32:21.}}
\VS{36}Alors Sanchérib, roi d'Assyrie, leva son camp, partit et s'en retourna~; et il resta à Ninive.
\VS{37}Il arriva, comme il était prosterné dans la maison de Nisroc, son dieu, qu'Adrammélec et Scharetser, ses fils, le tuèrent avec l'épée, puis ils se sauvèrent au pays d'Ararat~; et Esar-Haddon, son fils, régna à sa place.
\Chap{20}
\TextTitle{Ezéchias malade puis guéri par Yahweh\FTNTT{2 Ch. 32:24~; Es. 38.}}
\VerseOne{}En ce temps-là, Ezéchias fut malade à la mort. Le prophète Esaïe, fils d'Amots, vint auprès de lui, et lui dit~: Ainsi parle Yahweh~: Donne tes ordres à ta maison, car tu vas mourir et tu ne vivras plus.
\VS{2}Alors Ezéchias tourna son visage contre le mur et fit sa prière à Yahweh, en disant~:
\VS{3}Je te prie, ô Yahweh~! Souviens-toi que j'ai marché devant toi avec fidélité et intégrité de cœur, et que j'ai fait ce qui est agréable à tes yeux~! Et Ezéchias pleura abondamment.
\VS{4}Esaïe n'était pas encore sorti de la cour du milieu, que la parole de Yahweh lui fut adressée, en disant~:
\VS{5}Retourne et dis à Ezéchias, chef de mon peuple~: Ainsi parle Yahweh, le Dieu de David, ton père~: J'ai exaucé ta prière, j'ai vu tes larmes. Voici je te guérirai~; dans trois jours tu monteras à la maison de Yahweh.
\VS{6}J'ajouterai quinze ans à tes jours, je te délivrerai, toi et cette ville, de la main du roi d'Assyrie~; et je protégerai cette ville, par amour pour moi et par amour pour David, mon serviteur.
\VS{7}Puis Esaïe dit~: Prenez une masse de figues sèches. Et ils la prirent et l'appliquèrent sur l'ulcère. Et Ezéchias fut guéri.
\VS{8}Ezéchias avait dit à Esaïe~: A quel signe connaîtrai-je que Yahweh me guérira et qu'au troisième jour, je monterai à la maison de Yahweh~?
\VS{9}Esaïe répondit~: Voici, de la part de Yahweh, le signe auquel tu connaîtras que Yahweh accomplira la parole qu'il a prononcée~: L'ombre s'avancera-t-elle de dix degrés, ou reculera-t-elle en arrière de dix degrés~?
\VS{10}Ezéchias dit~: C'est peu de chose que l'ombre s'avance de dix degrés~; mais plutôt que l'ombre recule en arrière de dix degrés.
\VS{11}Alors Esaïe, le prophète, invoqua Yahweh, qui fit reculer l'ombre de dix degrés sur les degrés d'Achaz, où elle était descendue.
\TextTitle{Visite des ambassadeurs babyloniens~; prophétie sur la captivité babylonienne\FTNTT{2 Ch. 32:25-31~; Es. 39.}}
\VS{12}En ce temps-là, Berodac-Baladan, fils de Baladan, roi de Babylone, envoya une lettre avec un présent à Ezéchias, parce qu'il avait appris la maladie d'Ezéchias.
\VS{13}Et Ezéchias, donna audience aux envoyés et il leur montra tous les lieux où étaient ses objets les plus précieux, l'argent, l'or, les aromates, l'huile précieuse, tout son arsenal et tout ce qui se trouvait dans ses trésors. Il n'y eut rien qu'Ezéchias ne leur montra dans sa maison et dans tous ses domaines.
\VS{14}Esaïe, le prophète, vint ensuite auprès du roi Ezéchias, et lui dit~: Qu'ont dit ces gens-là~? Et d'où sont-ils venus vers toi~? Ezéchias répondit~: Ils sont venus d'un pays très éloigné, ils sont venus de Babylone.
\VS{15}Esaïe dit~: Qu'ont-ils vu dans ta maison~? Et Ezéchias répondit~: Ils ont vu tout ce qui est dans ma maison~; il n'y a rien dans mes trésors que je ne leur aie montré.
\VS{16}Alors Esaïe dit à Ezéchias~: Ecoute la parole de Yahweh~:
\VS{17}Voici, les jours viendront où tout ce qui est dans ta maison et ce que tes pères ont amassé dans leurs trésors jusqu'à ce jour, sera emporté à Babylone~; il n'en restera rien dit Yahweh\FTNT{La déportation des juifs à Babylone~: Voir 2 R. 24-25.}.
\VS{18}On prendra même de tes fils\FTNT{2 R. 24:12~; 2 Ch. 33:11~; Da. 1.} qui seront sortis de toi, que tu auras engendrés, afin qu'ils soient eunuques dans le palais du roi de Babylone.
\VS{19}Ezéchias répondit à Esaïe~: La parole de Yahweh que tu as prononcée est bonne. Et il ajouta~: N'y aura-t-il pas paix et sécurité pendant mes jours~?
\TextTitle{Mort d'Ezéchias~; Manassé règne sur Juda\FTNTT{2 Ch. 32:32-33.}}
\VS{20}Le reste des actions d'Ezéchias, tous ses exploits, et comment il fit l'étang et l'aqueduc par lequel il fit entrer les eaux dans la ville, cela n'est-il pas écrit dans le livre des Chroniques des rois de Juda~?
\VS{21}Ezéchias se coucha avec ses pères. Et Manassé, son fils, régna à sa place.
\Chap{21}
\TextTitle{Abominations et idolâtrie de Manassé\FTNTT{2 Ch. 33:1-9.}}
\VerseOne{}Manassé était âgé de douze ans, lorsqu'il commença à régner. Il régna cinquante-cinq ans à Jérusalem. Sa mère s'appelait Hephtsiba.
\VS{2}Il fit ce qui est mal aux yeux de Yahweh, selon les abominations des nations que Yahweh avait chassées devant les enfants d'Israël.
\VS{3}Car il rebâtit les hauts lieux qu'Ezéchias, son père, avait détruits, et redressa des autels à Baal, il fit une idole d'Asherah\FTNT{Voir commentaire en Jg. 2:13.}, comme avait fait Achab, roi d'Israël, il se prosterna devant toute l'armée des cieux et les servit.
\VS{4}Il bâtit aussi des autels dans la maison de Yahweh, quoique Yahweh ait dit~: C'est dans Jérusalem que j'établirai mon nom.
\VS{5}Il bâtit des autels à toute l'armée des cieux dans les deux parvis de la maison de Yahweh.
\VS{6}Il fit aussi passer son fils par le feu, il pratiquait l'astrologie et la divination, il établit des gens qui évoquaient les esprits des morts et qui prédisaient l'avenir. Il fit de plus en plus ce qui est mal aux yeux de Yahweh pour l'irriter.
\VS{7}Il plaça aussi l'idole d'Asherah qu'il avait faite, dans la maison de laquelle Yahweh avait dit à David, et à Salomon, son fils~: C'est dans cette maison, et c'est dans Jérusalem, que j'ai choisie parmi toutes les tribus d'Israël, que je veux à toujours établir mon nom.
\VS{8}Je ne ferai plus errer le pied d'Israël hors de cette terre que j'ai donnée à leurs pères, pourvu seulement qu'ils aient soin de mettre en pratique tout ce que je leur ai ordonné et toute la loi que Moïse, mon serviteur, leur a prescrite.
\VS{9}Mais ils n'obéirent point~; car Manassé les fit s'égarer, jusqu'à faire le mal plus que les nations que Yahweh avait exterminées devant les enfants d'Israël.
\TextTitle{Jugement de Yahweh contre Juda~; mort d'Ezéchias\FTNTT{2 Ch. 33:10-20.}}
\VS{10}Alors Yahweh parla par ses serviteurs les prophètes, en disant~:
\VS{11}Parce que Manassé, roi de Juda, a commis ces abominations, parce qu'il a fait pire que tout ce qu'avaient fait avant lui les Amoréens, et parce qu'il a aussi fait pécher Juda par ses idoles,
\VS{12}à cause de cela, Yahweh, le Dieu d'Israël, dit~: Voici, je vais faire venir sur Jérusalem et sur Juda des malheurs qui étourdiront les oreilles de quiconque en entendra parler.
\VS{13}Car j'étendrai sur Jérusalem le cordeau de Samarie, et le niveau de la maison d'Achab, et je nettoierai Jérusalem comme un plat qu'on nettoie, et qu’on renverse sur son fond après l'avoir nettoyé.
\VS{14}J'abandonnerai le reste de mon héritage, et je les livrerai entre les mains de leurs ennemis~; et ils seront le butin et la proie de tous leurs ennemis~;
\VS{15}parce qu'ils ont fait ce qui est mal à mes yeux, et qu'ils m'ont irrité depuis le jour où leurs pères sont sortis d'Egypte, jusqu'à ce jour.
\TextTitle{Meurtres de Manassé~; sa mort\FTNTT{2 Ch. 33:11-20.}}
\VS{16}Manassé répandit aussi beaucoup de sang innocent, jusqu'à en remplir Jérusalem d'un bout à l'autre, outre son péché par lequel il fit pécher Juda en faisant ce qui est mal aux yeux de Yahweh.
\VS{17}Le reste des actions de Manassé, tout ce qu'il a fait~; et les péchés auxquels il se livra, cela n'est-il pas écrit dans le livre des Chroniques des rois de Juda~?
\VS{18}Manassé se coucha avec ses pères, et il fut enseveli dans le jardin de sa maison, dans le jardin d'Uzza. Amon, son fils, régna à sa place.
\TextTitle{Amon règne sur Juda~; sa mort\FTNTT{2 Ch. 33:20-25.}}
\VS{19}Amon était âgé de vingt-deux ans lorsqu'il commença à régner. Il régna deux ans à Jérusalem. Sa mère s'appelait Meschullémeth, fille de Haruts, de Jotba.
\VS{20}Il fit ce qui est mal aux yeux de Yahweh, comme avait fait Manassé, son père.
\VS{21}Car il marcha dans toute la voie où avait marché son père, il servit les idoles que son père avait servies et se prosterna devant elles.
\VS{22}Il abandonna Yahweh, le Dieu de ses pères et il ne marcha point dans la voie de Yahweh.
\TextTitle{Josias, roi de Juda\FTNTT{2 Ch. 33:24-25.}}
\VS{23}Les serviteurs d'Amon firent une conspiration contre lui et le tuèrent dans sa maison.
\VS{24}Mais le peuple du pays frappa tous ceux qui avaient conspiré contre le roi Amon~; et ils établirent Josias, son fils, roi à sa place.
\VS{25}Le reste des actions d'Amon, ce qu'il a fait, cela n'est-il pas écrit dans le livre des Chroniques des rois de Juda~?
\VS{26}On l'ensevelit dans son sépulcre, dans le jardin d'Uzza. Et Josias, son fils, régna à sa place.
\Chap{22}
\TextTitle{Droiture de Josias~; réparations dans le temple\FTNTT{2 Ch. 34:2-13.}}
\VerseOne{}Josias était âgé de huit ans lorsqu'il commença à régner. Il régna trente et un ans à Jérusalem. Sa mère s'appelait Jedida, fille d'Adaja, de Botskath.
\VS{2}Il fit ce qui est droit aux yeux de Yahweh et il marcha dans toute la voie de David, son père~; il ne s'en détourna ni à droite ni à gauche.
\VS{3}La dix-huitième année du roi Josias, le roi envoya dans la maison de Yahweh, Schaphan, le secrétaire, fils d'Atsalia, fils de Meschullam.
\VS{4}Il lui dit~: Monte vers Hilkija, le grand-prêtre, et dis-lui d'amasser l'argent qui a été apporté dans la maison de Yahweh et que ceux qui ont la garde du seuil ont recueilli du peuple.
\VS{5}On remettra cet argent entre les mains de ceux qui sont chargés de faire exécuter l'ouvrage dans la maison de Yahweh. Et ils l'emploieront pour ceux qui travaillent dans la maison de Yahweh, pour réparer les brèches de la maison,
\VS{6}pour les charpentiers, les architectes et les maçons, pour les achats du bois et des pierres de taille pour réparer la maison.
\VS{7}Mais on ne leur demandera pas de comptes pour l'argent remis entre leurs mains, parce qu'ils agissent fidèlement.
\TextTitle{Découverte et lecture du livre de la loi\FTNTT{2 Ch. 34:14-19.}}
\VS{8}Alors Hilkija, le grand-prêtre, dit à Schaphan, le secrétaire~: J'ai trouvé le livre de la loi dans la maison de Yahweh. Et Hilkija donna ce livre à Schaphan qui le lut.
\VS{9}Schaphan, le secrétaire, alla vers le roi et lui rapporta la chose, et dit~: Tes serviteurs ont amassé l'argent qui se trouvait dans la maison et l'ont remis entre les mains de ceux qui sont chargés de faire l'ouvrage dans la maison de Yahweh.
\VS{10}Schaphan, le secrétaire, dit aussi au roi~: Le prêtre Hilkija m'a donné un livre. Et Schaphan le lut devant le roi.
\VS{11}Lorsque le roi eut entendu les paroles du livre de la loi, il déchira ses vêtements.
\TextTitle{Annonce du jugement de Yahweh par Hulda\FTNTT{2 Ch. 34:20-28.}}
\VS{12}Il donna cet ordre au prêtre Hilkija, à Achikam, fils de Schaphan, à Acbor, fils de Michée, à Schaphan, le secrétaire, et à Asaja, serviteur du roi~:
\VS{13}Allez, consultez Yahweh pour moi, pour le peuple et pour tout Juda, au sujet des paroles de ce livre qui a été trouvé~; car grande est la colère de Yahweh, qui s'est enflammée contre nous, parce que nos pères n'ont point obéi aux paroles de ce livre et n'ont pas mis en pratique tout ce qui nous y est prescrit.
\VS{14}Le prêtre Hilkija, Achikam, Acbor, Schaphan et Asaja, allèrent auprès de la prophétesse Hulda, femme de Schallum, fils de Thikva, fils de Harhas, gardien des vêtements. Elle habitait dans un autre quartier de Jérusalem.
\TextTitle{Yahweh rassure Josias par la prophétesse Hulda\FTNTT{2 Ch. 34:22-28.}}
\VS{15}Après qu'ils eurent parlé avec elle, elle leur répondit~: Ainsi parle Yahweh, le Dieu d'Israël~: Dites à l'homme qui vous a envoyé vers moi~:
\VS{16}Ainsi parle Yahweh~: Voici, je vais faire venir le malheur sur cette ville et sur ses habitants, selon toutes les paroles du livre que le roi de Juda a lu.
\VS{17}Parce qu'ils m'ont abandonné et qu'ils ont offert des parfums à d'autres dieux, pour m'irriter par toutes les actions de leurs mains, ma colère s'est enflammée contre cette ville et elle ne s'éteindra point.
\VS{18}Mais quant au roi de Juda qui vous a envoyé pour consulter Yahweh, vous lui direz~: Ainsi parle Yahweh, le Dieu d'Israël, au sujet des paroles que tu as entendues~:
\VS{19}Parce que ton cœur a été touché, et que tu t'es humilié devant Yahweh en entendant ce que j'ai prononcé contre cette ville et contre ses habitants, qui seront un objet d'épouvante et de malédiction, et parce que tu as déchiré tes vêtements, et que tu as pleuré devant moi, je t'ai exaucé, dit Yahweh.
\VS{20}C'est pourquoi voici, je vais te recueillir auprès de tes pères, et tu seras recueilli dans ton sépulcre en paix, et tes yeux ne verront point tout ce mal que je vais faire venir sur cette ville. Ils rapportèrent toutes ces paroles au roi.
\Chap{23}
\TextTitle{Le livre de la loi lu au peuple\FTNTT{2 Ch. 34:29-30.}}
\VerseOne{}Alors, le roi Josias, fit assembler auprès de lui tous les anciens de Juda et de Jérusalem.
\VS{2}Le roi monta à la maison de Yahweh, avec tous les hommes de Juda, tous les habitants de Jérusalem, les prêtres, les prophètes, et tout le peuple, depuis le plus petit jusqu'au plus grand. Il lut devant eux toutes les paroles du livre de l'alliance, qui avait été trouvé dans la maison de Yahweh.
\TextTitle{Engagement de Josias et du peuple à suivre la loi de Yahweh\FTNTT{2 Ch. 34:31-32.}}
\VS{3}Le roi se tenait sur l'estrade et il traita alliance devant Yahweh, s'engageant à suivre Yahweh, à observer ses ordonnances, ses préceptes et ses lois, de tout son cœur, à persévérer dans les paroles de cette alliance, écrites dans ce livre. Et tout le peuple entra dans cette alliance.
\TextTitle{Josias débarrasse Juda de tous ses faux dieux\FTNTT{2 Ch. 34:33.}}
\VS{4}Alors le roi donna cet ordre à Hilkija, le grand-prêtre, aux prêtres du second ordre et à ceux qui gardaient le seuil, de sortir hors du temple de Yahweh tous les ustensiles qui avaient été faits pour Baal\FTNT{Voir commentaire en Jg. 2:12.}, pour Asherah\FTNT{Voir commentaire en Jg. 2:13.}, et pour toute l'armée des cieux~; et il les brûla hors de Jérusalem, dans les champs de Cédron, et en fit porter la poussière à Béthel.
\VS{5}Il chassa les prêtres des idoles, que les rois de Juda avaient établis pour brûler des parfums sur les hauts lieux, dans les villes de Juda et aux environs de Jérusalem, et ceux qui offraient des parfums à Baal, au soleil, à la lune, au zodiaque et à toute l'armée des cieux.
\VS{6}Il sortit de la maison de Yahweh l'idole d'Asherah, qu'il transporta hors de Jérusalem vers le torrent de Cédron~; il la brûla au torrent de Cédron et la réduisit en poudre, et il en jeta la poussière sur le sépulcre des fils du peuple.
\VS{7}Ensuite, il démolit les maisons des prostituées qui étaient dans la maison de Yahweh, où les femmes tissaient des tentes pour Asherah.
\VS{8}Il fit venir des villes de Juda tous les prêtres~; il profana les hauts lieux où les prêtres brûlaient des parfums, depuis Guéba jusqu'à Beer-Schéba~; il renversa les hauts lieux des portes, celui qui était à l'entrée de la porte de Josué, chef de la ville, et celui qui était à gauche de la porte de la ville.
\VS{9}Toutefois, les prêtres des hauts lieux ne montaient pas à l'autel de Yahweh à Jérusalem, mais ils mangeaient des pains sans levain parmi leurs frères.
\VS{10}Le roi profana aussi Topheth, dans la vallée des fils de Hinnom, afin que personne ne fasse plus passer son fils ou sa fille par le feu, en l'honneur de Moloc\FTNT{Lé. 20:2-3.}.
\VS{11}Il fit disparaître de l'entrée de la maison de Yahweh les chevaux que les rois de Juda avaient consacrés au soleil, près de la chambre de l'eunuque Nethan-Mélec, situé à Parvarim, et il brûla au feu les chars du soleil.
\VS{12}Le roi démolit les autels qui étaient sur le toit de la chambre haute d'Achaz, que les rois de Juda avaient faits et les autels que Manassé avait faits dans les deux parvis de la maison de Yahweh~; après les avoir brisés et enlevés de là, il en jeta la poussière dans le torrent de Cédron.
\VS{13}Le roi profana aussi les hauts lieux qui étaient en face de Jérusalem, sur la droite de la montagne de perdition, que Salomon, roi d'Israël, avait bâtis à Astarté, l'abomination des Sidoniens, à Kemosch, l'abomination des Moabites, et à Milcom, l'abomination des fils d'Ammon.
\VS{14}Il brisa aussi les statues, et abattit les Asherah, et il remplit d'ossements d'hommes les lieux où elles étaient.
\VS{15}Il renversa l'autel qui était à Béthel et le haut lieu qu'avait fait Jéroboam, fils de Nebath, qui avait fait pécher Israël~; il brûla le haut lieu et le réduisit en poudre, et il brûla l'Asherah.
\VS{16}Josias s'étant tourné et ayant vu les sépulcres qui étaient là dans la montagne, envoya prendre les ossements des sépulcres, et il les brûla sur l'autel et le profana, selon la parole de Yahweh prononcée à haute voix par l'homme de Dieu.
\VS{17}Le roi dit~: Quel est ce monument que je vois~? Et les hommes de la ville lui répondirent~: C'est le sépulcre de l'homme de Dieu qui est venu de Juda qui a crié contre l'autel de Béthel ces choses que tu as accomplies.
\VS{18}Et il dit~: Laissez-le~; que personne ne remue ses os~! Ils conservèrent ainsi ses os, avec les os du prophète qui était venu de Samarie.
\VS{19}Josias fit encore disparaître toutes les maisons des hauts lieux, qui étaient dans les villes de Samarie, et qu'avaient faites les rois d'Israël pour irriter Yahweh~; et il fit à leur égard entièrement comme il avait fait à Béthel.
\VS{20}Il immola sur les autels tous les prêtres des hauts lieux qui étaient là, et il y brûla des ossements d'hommes. Puis il retourna à Jérusalem.
\TextTitle{Josias rétablit la fête de la Pâque\FTNTT{2 Ch. 35:1-19.}}
\VS{21}Alors le roi donna cet ordre à tout le peuple, en disant~: Célébrez la Pâque en l'honneur de Yahweh, votre Dieu, comme il est écrit dans le livre de cette alliance\FTNT{Jésus-Christ est notre Pâque. Voir Ex. 12 et 1 Co. 5:7.}.
\VS{22}Aucune Pâque pareille à celle-ci n'avait été célébrée depuis le temps où les juges jugeaient Israël et pendant tous les jours des rois d'Israël et des rois de Juda.
\VS{23}Ce fut la dix-huitième année du roi Josias qu'on célébra cette Pâque en l'honneur de Yahweh à Jérusalem.
\VS{24}Josias, extermina aussi, ceux qui évoquaient les esprits des morts et les devins, les théraphim, les idoles, et toutes les abominations qui se voyaient dans le pays de Juda et à Jérusalem, afin de mettre en pratique les paroles de la loi, écrites dans le livre que Hilkija, le prêtre, avait trouvé dans la maison de Yahweh.
\TextTitle{Témoignage de Josias~; confirmation du jugement de Yahweh}
\VS{25}Avant Josias, il n'y eut point de roi qui, comme lui, revienne à Yahweh de tout son cœur, de toute son âme et de toute sa force, selon toute la loi de Moïse~; et après lui, il n'en a point paru de semblable.
\VS{26}Toutefois, Yahweh ne se détourna point de l'ardeur de sa grande colère dont il était enflammé contre Juda, à cause de tout ce que Manassé avait fait pour l'irriter.
\VS{27}Et Yahweh dit~: J'ôterai Juda de devant ma face, comme j'ai ôté Israël, et je rejetterai cette ville de Jérusalem que j'avais choisie, et la maison de laquelle j'avais dit~: Là sera mon Nom.
\VS{28}Le reste des actions de Josias, tout ce qu'il a fait, cela n'est-il pas écrit dans le livre des Chroniques des rois de Juda~?
\TextTitle{Mort de Josias~; Joachaz règne sur Juda\FTNTT{2 Ch. 35:20-27~; 2 Ch. 36:1-2.}}
\VS{29}De son temps, Pharaon Néco, roi d'Egypte, monta contre le roi d'Assyrie, vers le fleuve d'Euphrate. Le roi Josias s'en alla au-devant de lui~; mais dès que pharaon le vit, il le tua à Meguiddo.
\VS{30}Ses serviteurs l'emportèrent mort sur un char~; ils l'amenèrent de Meguiddo à Jérusalem et l'ensevelirent dans son sépulcre. Et le peuple du pays prit Joachaz, fils de Josias, ils l'oignirent et l'établirent roi à la place de son père.
\TextTitle{Joachaz mis en prison par Pharaon\FTNTT{2 Ch. 36:3.}}
\VS{31}Joachaz était âgé de vingt-trois ans, lorsqu'il commença à régner. Il régna trois mois à Jérusalem. Sa mère s'appelait Hamuthal, fille de Jérémie, de Libna.
\VS{32}Il fit ce qui est mal aux yeux de Yahweh, entièrement comme avaient fait ses pères.
\VS{33}Et pharaon Néco l'emprisonna à Ribla, dans le pays de Hamath, afin qu'il ne règne plus à Jérusalem~; et il imposa sur le pays un tribut de cent talents d'argent et d'un talent d'or.
\TextTitle{Pharaon établit Jojakim roi de Juda\FTNTT{2 Ch. 36:4-5.}}
\VS{34}Puis pharaon Néco établit roi Eliakim, fils de Josias, à la place de Josias, son père, et il changea son nom en celui de Jojakim. Il prit Joachaz, qui alla en Egypte, où il mourut.
\VS{35}Jojakim donna cet argent et cet or à pharaon~; mais il taxa le pays pour fournir cet argent, selon l'ordre de pharaon~; il détermina la part de chacun et exigea du peuple du pays l'argent et l'or qu'il devait livrer à pharaon Néco.
\VS{36}Jojakim était âgé de vingt-cinq ans lorsqu'il commença à régner. Il régna onze ans à Jérusalem. Sa mère s'appelait Zebudda, fille de Pedaja, de Ruma.
\VS{37}Il fit ce qui est mal aux yeux de Yahweh, entièrement comme avaient fait ses pères.
\Chap{24}
\TextTitle{Asservissement de Jojakim au roi de Babylone~; destruction de Juda\FTNTT{2 Ch. 36:6-7.}}
\VerseOne{}De son temps, Nebucadnetsar, roi de Babylone, monta contre Jojakim, et Jojakim lui fut asservi pendant trois ans~; mais il se révolta de nouveau contre lui.
\VS{2}Alors Yahweh envoya contre Jojakim des troupes de Chaldéens, des armées de Syriens, des troupes de Moabites et des troupes des fils d'Ammon~; il les envoya contre Juda, pour le détruire, selon la parole que Yahweh avait prononcée par ses serviteurs les prophètes.
\VS{3}Cela arriva uniquement sur l'ordre de Yahweh, qui voulait ôter Juda de devant sa face, à cause de tous les péchés commis par Manassé,
\VS{4}et à cause aussi du sang innocent qu'il avait répandu, et dont il avait rempli Jérusalem. C'est pourquoi Yahweh ne voulut point lui pardonner.
\TextTitle{Mort de Jojakim~; Jojakin règne sur Juda\FTNTT{2 Ch. 36:8-9.}}
\VS{5}Le reste des actions de Jojakim et tout ce qu'il a fait, cela n'est-il pas écrit dans le livre des Chroniques des rois de Juda~?
\VS{6}Ainsi Jojakim se coucha avec ses pères. Et Jojakin, son fils, régna à sa place.
\VS{7}Le roi d'Egypte ne sortit plus de son pays, parce que le roi de Babylone avait pris tout ce qui était au roi d'Egypte, depuis le torrent d'Egypte jusqu'au fleuve d'Euphrate.
\VS{8}Jojakin était âgé de dix-huit ans lorsqu'il commença à régner. Il régna trois mois à Jérusalem. Sa mère s'appelait Nehuschtha, fille d'Elnathan, de Jérusalem.
\VS{9}Il fit ce qui est mal aux yeux de Yahweh, entièrement comme avait fait son père.
\TextTitle{Jérusalem et son roi en captivité à Babylone~; les pauvres restent\FTNTT{2 Ch. 36:10.}}
\VS{10}En ce temps-là, les serviteurs de Nebucadnetsar, roi de Babylone, montèrent contre Jérusalem, et la ville fut assiégée.
\VS{11}Nebucadnetsar, roi de Babylone, arriva devant la ville, pendant que ses serviteurs l'assiégeaient.
\VS{12}Alors Jojakin, roi de Juda, se rendit vers le roi de Babylone, avec sa mère, ses serviteurs, ses chefs et ses eunuques. Et le roi de Babylone le fit prisonnier, la huitième année de son règne.
\VS{13}Il emporta de là, tous les trésors de la maison de Yahweh et les trésors de la maison royale~; et il mit en pièces tous les ustensiles d'or que Salomon, roi d'Israël, avait faits pour le temple de Yahweh, comme Yahweh l'avait ordonné.
\VS{14}Il emmena en captivité tout Jérusalem\FTNT{Première déportation~: 2 R. 24:1-4 et 2 Ch. 36:6-7. La première déportation eut lieu en 597 av. J.-C. pendant le règne de Jojakim, roi de Juda. Les premiers exilés furent installés dans la région du fleuve Kebar (Ez. 1:1-3), un canal de 90 km de long reliant l'Euphrate au nord de Babylone au même fleuve au sud d'Ur en Chaldée. Jérémie savait que leur séjour à l'étranger serait long. Il avait prophétisé qu'il durerait soixante-dix ans (Jé. 25:1~; Jé. 25:11-12) et leur conseilla de se construire des maisons, de cultiver des jardins et de se multiplier (Jé. 29). Daniel et ses compagnons furent déportés à Babylone lors de la première déportation (Da. 1). Daniel fut déporté environ huit ans avant Ezéchiel.}, à savoir, tous les chefs, et tous les vaillants hommes de guerre, au nombre de dix mille captifs, avec les charpentiers et les serruriers, de sorte qu'il ne resta plus que le peuple pauvre du pays.
\VS{15}Ainsi il transporta Jojakin à Babylone, avec la mère du roi, les femmes du roi et ses eunuques. Il emmena captifs à Babylone tous les grands du pays, de Jérusalem à Babylone,
\VS{16}avec tous les guerriers au nombre de sept mille, les charpentiers, les serruriers au nombre de mille, tous les hommes vaillants et propres à la guerre. Le roi de Babylone les emmena captifs à Babylone.
\TextTitle{Nebucadnetsar établit Sédécias roi de Juda\FTNTT{2 Ch. 36:10-12.}}
\VS{17}Et le roi de Babylone établit roi, à la place de Jojakin, Matthania, son oncle, et il changea son nom en celui de Sédécias.
\VS{18}Sédécias était âgé de vingt et un ans lorsqu'il commença à régner. Il régna onze ans à Jérusalem. Sa mère s'appelait Hamuthal, fille de Jérémie, de Libna.
\VS{19}Il fit ce qui est mal aux yeux de Yahweh, entièrement comme avait fait Jojakim.
\TextTitle{Sédécias se révolte\FTNTT{2 Ch. 36:13-16.}}
\VS{20}Cela arriva à cause de la colère de Yahweh contre Jérusalem et contre Juda, qu'il voulait rejeter de devant sa face. Et Sédécias se révolta contre le roi de Babylone.
\Chap{25}
\TextTitle{Siège de Jérusalem\FTNTT{Jé. 39:1.}}
\VerseOne{}Et il arriva dans la neuvième année du règne de Sédécias, le dixième jour du dixième mois, que Nebucadnetsar\FTNT{Jérusalem fut assiégée pendant deux ans. Lors de ce siège, des femmes juives faisaient cuire leurs enfants pour les consommer (La. 2:20~; La. 4:10).}, roi de Babylone, vint avec toute son armée contre Jérusalem~; il campa devant elle et éleva des retranchements tout autour.
\VS{2}La ville fut assiégée jusqu'à la onzième année du roi Sédécias.
\VS{3}Le neuvième jour du 4ème mois, la famine\FTNT{La. 4:10.} augmenta dans la ville, de sorte qu'il n'y avait pas de pain pour le peuple du pays.
\TextTitle{Sédécias lié et emmené à Babylone\FTNTT{Jé. 39:2-7.}}
\VS{4}Alors la brèche fut faite à la ville~; et tous les gens de guerre s'enfuirent de nuit par le chemin de la porte entre les deux murailles près du jardin du roi, pendant que les Chaldéens environnaient la ville. Les fuyards et le roi prirent le chemin de la plaine.
\VS{5}Mais l'armée des Chaldéens poursuivit le roi et l'atteignit dans les plaines de Jéricho, et toute son armée se dispersa loin de lui.
\VS{6}Ils saisirent donc le roi et le firent monter vers le roi de Babylone à Ribla~; et l'on prononça contre lui un jugement.
\VS{7}Et on égorgea les fils de Sédécias en sa présence~; puis on creva les yeux à Sédécias, et on le lia de doubles chaînes d'airain, et on le mena à Babylone.
\TextTitle{Destruction de Jérusalem, du temple et des murailles\FTNTT{2 Ch. 36:17-21~; Jé. 39~:8-10.}}
\VS{8}Le septième jour du cinquième mois, c'était la dix-neuvième année du roi Nebucadnetsar, roi de Babylone, Nebuzaradan, chef des gardes, serviteur du roi de Babylone,\FTNT{Troisième déportation~: Le temple fut brûlé, la ville de Jérusalem fut totalement rasée et ses habitants furent déportés (De. 28:49-68). Contrairement à ce que l'on pense, il y a eu d'autres déportations. Voir Jé. 52.} entra dans Jérusalem.
\VS{9}Il brûla la maison de Yahweh, la maison royale et toutes les maisons de Jérusalem~; il brûla par le feu toutes les grandes maisons.
\VS{10}Toute l'armée des Chaldéens, qui était avec le chef des gardes, démolit les murailles qui entouraient Jérusalem.
\VS{11}Et Nebuzaradan, chef des gardes, emmena captifs le reste du peuple, ceux qui étaient restés dans la ville, ceux qui s'étaient rendus au roi de Babylone et le reste de la multitude.
\VS{12}Cependant le chef des gardes laissa quelques-uns des plus pauvres du pays comme vignerons et comme laboureurs.
\VS{13}Les Chaldéens brisèrent les colonnes d'airain qui étaient dans la maison de Yahweh, les bases, la mer d'airain qui était dans la maison de Yahweh, et ils en emportèrent l'airain à Babylone.
\VS{14}Ils prirent aussi les cendriers, les pelles, les couteaux, les tasses et tous les ustensiles d'airain avec lesquels on faisait le service.
\VS{15}Le chef des gardes emporta aussi les encensoirs et les coupes, ce qui était d'or et ce qui était d'argent.
\VS{16}Les deux colonnes, la mer et les bases, que Salomon avait faits pour la maison de Yahweh, tous ces ustensiles d'airain avaient un poids inconnu.
\VS{17}La hauteur d'une colonne était de dix-huit coudées, et il y avait au-dessus un chapiteau d'airain dont la hauteur était de trois coudées~; autour du chapiteau il y avait un treillis et des grenades, le tout d'airain~; il en était de même pour la seconde colonne avec le treillis.
\VS{18}Le chef des gardes emmena aussi Seraja, le premier prêtre, et Sophonie, le second prêtre, et les trois gardiens du seuil.
\VS{19}Et dans la ville, il prit un eunuque qui avait sous son commandement des hommes de guerre, cinq hommes de ceux qui voyaient la face du roi et qui furent trouvés dans la ville, il prit aussi le secrétaire du chef de l'armée qui était chargé d'enrôler le peuple du pays, et soixante hommes du peuple du pays qui se trouvaient dans la ville.
\VS{20}Nebuzaradan, chef des gardes, les prit et les conduisit vers le roi de Babylone à Ribla.
\VS{21}Le roi de Babylone les frappa, et les fit mourir à Ribla, dans le pays de Hamath. Ainsi Juda fut transporté captif hors de sa terre.
\TextTitle{Guedalia nommé gouverneur de Juda\FTNTT{Jé. 40:7-11.}}
\VS{22}Nebucadnetsar, roi de Babylone, plaça le reste du peuple, qu'il laissa dans le pays de Juda, sous le commandement de Guedalia, fils d'Achikam, fils de Schaphan.
\VS{23}Lorsque tous les chefs des troupes et leurs hommes, eurent appris que le roi de Babylone avait établi Guedalia pour gouverneur, ils allèrent trouver Guedalia à Mitspa, à savoir Ismaël, fils de Nethania, Jochanan, fils de Karéach, Seraja, fils de Thanhumeth, de Nethopha, Jaazania, fils du Maacathien, eux et leurs hommes.
\VS{24}Guedalia leur jura, à eux et à leurs hommes, et leur dit~: Ne craignez pas d'être serviteurs des Chaldéens~; demeurez dans le pays et servez le roi de Babylone, et vous vous en trouverez bien.
\TextTitle{Fuite du peuple en Egypte\FTNTT{Jé. 41:1-3~; Jé. 43:4-7.}}
\VS{25}Mais il arriva au septième mois, qu'Ismaël, fils de Nethania, fils d'Elischama, qui était de race royale, vint, accompagné de dix hommes, et ils frappèrent mortellement Guedalia, ainsi que les Juifs et les Chaldéens qui étaient avec lui à Mitspa.
\VS{26}Alors tout le peuple, depuis le plus petit jusqu'au plus grand, avec les chefs des troupes, se levèrent et s'en allèrent en Egypte, parce qu'ils avaient peur des Chaldéens.
\TextTitle{Jojakin à la table du roi de Babylone\FTNTT{Jé. 52:31-34.}}
\VS{27}La trente-septième année de la captivité de Jojakin, roi de Juda, le vingt-septième jour du douzième mois, Evil-Merodac, roi de Babylone, dans la première année de son règne, releva la tête de Jojakin, roi de Juda et le tira de prison.
\VS{28}Il lui parla avec bonté et il mit son trône au-dessus du trône des rois qui étaient avec lui à Babylone.
\VS{29}Il lui fit changer ses vêtements de prison, et Jojakin mangea du pain tout le temps de sa vie en sa présence.
\VS{30}Et quant à son entretien, un entretien perpétuel, lui fut accordé par le roi pour chaque jour, tous les jours de sa vie.
\PPE{}
\end{multicols}

\clearpage
\ShortTitle{Esaïe}\BookTitle{Esaïe}\BFont
\noindent\hrulefill
{\footnotesize
\textit{
\bigskip
{\centering{}
\\Auteur : Esaïe
\\(Heb. : Yesha'yah)
\\Signification : YAHWEH a sauvé
\\Thème : Le Messie d'Israël
\\Date de rédaction : 8\up{ème} siècle av. J.-C.\\}
}
%\bigskip
\textit{
\\Prophète en Israël, Esaïe fut une figure marquante en raison du contenu et de l'impact de son message. Véritable porte-parole de Dieu, il parla de la ruine morale d'Israël, de la déportation à Babylone et des jugements de Dieu sur son peuple. Il prophétisa également sur le retour de l'exil, la restauration finale et la reconstruction de Jérusalem. Plus qu'aucun autre livre, les écrits d'Esaïe annoncent clairement la naissance du Messie, son service, sa mission rédemptrice, son sacrifice et son futur règne millénaire. 
%\bigskip
\\L'autorité et l'exactitude de ses prophéties ont été une source d'édification au fil des siècles.\bigskip
}
}
\par\nobreak\noindent\hrulefill
\begin{multicols}{2}
\Chap{1}
\TextTitle{Prophéties concernant Juda}
\VerseOne{}La vision d'Esaïe, fils d'Amots, qu'il a vue touchant Juda et Jérusalem, au jour d'Ozias, de Jotham, d'Achaz, et d'Ezéchias, rois de Juda.
\VS{2}Cieux, écoutez ! Et toi, terre, prête l'oreille ! Car Yahweh parle. J'ai nourri des enfants, je les ai élevés, mais ils se sont rebellés contre moi.
\VS{3}Le bœuf connaît son possesseur, et l'âne la crèche de son maître, mais Israël n'a point de connaissance, mon peuple n'a point d'intelligence.
\VS{4}Ah! Nation pécheresse, peuple chargé d'iniquités, race de gens méchants, enfants qui ne font que se corrompre ! Ils ont abandonné Yahweh, ils ont irrité par leur mépris le Saint d'Israël, ils se sont retirés en arrière.
\VS{5}Pourquoi serez-vous encore frappés ? Vous ajouterez la révolte ! La tête entière est malade, et tout le cœur est languissant.
\VS{6}Depuis la plante du pied jusqu'à la tête, il n'y a rien de sain en lui : Il n'y a que blessures, meurtrissures et plaies pourries, qui n'ont été ni nettoyées, ni bandées, et dont aucune n'a été adoucie par l'huile.
\VS{7}Votre pays n'est que désolation, et vos villes sont en feu ; des étrangers dévorent votre terre sous vos yeux, et cette désolation est comme un bouleversement fait par des étrangers.
\VS{8}Car la fille de Sion est restée comme une cabane dans une vigne, comme une cabane dans un champ de concombres, comme une ville assiégée.
\VS{9}Si Yahweh des armées ne nous avait pas laissé un petit reste, qui est même bien peu, nous serions comme Sodome, nous ressemblerions à Gomorrhe.
\TextTitle{Yahweh rejette la religiosité et recherche la justice}
\VS{10}Ecoutez la parole de Yahweh, chefs de Sodome, prêtez l'oreille à la loi de notre Dieu, peuple de Gomorrhe !
\VS{11}Qu'ai-je à faire, dit Yahweh, de la multitude de vos sacrifices ? Je suis rassasié des holocaustes de béliers et de la graisse des veaux ; je ne prends point plaisir au sang des taureaux, ni des agneaux, ni des boucs\FTNT{1 S. 15:22 ; Os. 8:13 ; Mt. 9:13.}.
\VS{12}Quand vous entrez pour vous présenter devant ma face, qui a requis cela de votre main, que vous fouliez de vos pieds mes parvis ?
\VS{13}Ne continuez plus à m'apporter de vaines offrandes : Le parfum m'est en abomination, quant aux nouvelles lunes, aux sabbats et à la publication de vos convocations ; je ne puis plus supporter votre méchanceté ni vos assemblées solennelles.
\VS{14}Mon âme hait vos nouvelles lunes et vos fêtes solennelles ; elles me sont fâcheuses, je suis las de les supporter.
\VS{15}C'est pourquoi, quand vous étendez vos mains, je cache mes yeux de vous ; quand vous multipliez vos prières, je ne les exauce pas ; vos mains sont pleines de sang\FTNT{Es. 59:1-3 ;Mi. 3:4.}.
\VS{16}Lavez-vous, purifiez-vous, ôtez de devant mes yeux la méchanceté de vos actions ; cessez de faire le mal.
\VS{17}Apprenez à bien faire, recherchez la droiture, redressez celui qui est foulé ; faites justice à l'orphelin, défendez la cause de la veuve.
\TextTitle{Mise en garde ; appel à la justice de Yahweh}
\VS{18}Venez maintenant, dit Yahweh, et débattons nos droits. Si vos péchés sont comme l'écarlate, ils seront blanchis comme la neige ; s'ils sont rouges comme le vermillon ils seront blanchis comme la laine.
\VS{19}Si vous obéissez volontairement, vous mangerez le meilleur du pays.
\VS{20}Mais si vous refusez d'obéir et si vous êtes rebelles, vous serez dévorés par l'épée, car la bouche de Yahweh a parlé.
\VS{21}Comment la cité fidèle est-elle devenue une prostituée ? Elle était pleine de droiture et la justice y habitait ; mais maintenant elle est pleine de meurtriers !
\VS{22}Ton argent s'est changé en scories; ton breuvage est mêlé d'eau. 
\VS{23}Les chefs de ton peuple sont rebelles et compagnons des voleurs ; chacun d'eux aime les présents, ils courent après les récompenses ; ils ne font point droit à l'orphelin, et la cause de la veuve ne vient point devant eux. 
\VS{24}C'est pourquoi le Seigneur, Yahweh des armées, le Puissant d'Israël dit : Ah ! Je me satisferai en punissant mes adversaires, et je me vengerai de mes ennemis. 
\VS{25}Et je remettrai ma main sur toi, je refondrai tes scories comme avec la potasse, et j'ôterai tout ton étain ;
\VS{26}mais je rétablirai tes juges, tels qu'ils étaient autrefois, et tes conseillers, tels qu'ils étaient au commencement\FTNT{Dans le royaume messianique, le gouvernement théocratique sera restauré et la fonction des juges sera rétablie (voir livre des Juges ; Mt. 19:28 ; 1 Co. 6:2-3).}. Après cela, on t'appellera cité de la justice, ville fidèle.
\VS{27}Sion sera rachetée par la droiture et ceux qui s'y convertiront seront rachetés par la justice.
\VS{28}Mais les rebelles et les pécheurs seront détruits ensemble, et ceux qui abandonnent Yahweh seront consumés.
\VS{29}Car on sera honteux à cause des térébinthes que vous avez désirés, et vous rougirez à cause des jardins que vous avez choisis\FTNT{Des cultes idolâtres avaient lieu autour des térébinthes et dans des jardins (De. 16:21 ; Es. 57:4-5 ; Es. 65:3 ; Jé. 2:20 ; Ez. 20:28 ; Os. 4:13).}.
\VS{30}Car vous serez comme le térébinthe dont le feuillage tombe, et comme un jardin qui n'a pas d'eau.
\VS{31}Et le fort sera de l'étoupe, et son œuvre une étincelle ; et tous deux brûleront ensemble, et il n'y aura personne pour éteindre le feu.
\Chap{2}
\TextTitle{Vision du règne messianique}
\VerseOne{}La parole qu'Esaïe, fils d'Amots a vue touchant Juda et Jérusalem.
\VS{2}Or il arrivera, dans les derniers jours\FTNT{Voir Ge. 49:1-2.}, que la montagne de la maison de Yahweh sera affermie au  sommet des montagnes, qu'elle sera élevée par-dessus les collines et que toutes les nations y afflueront.
\VS{3}Et plusieurs peuples iront et diront : Venez, et montons à la montagne de Yahweh, à la maison du Dieu de Jacob ; et il nous instruira ses voies, et nous marcherons dans ses sentiers ; car la loi sortira de Sion, et la parole de Yahweh sortira de Jérusalem. 
\VS{4}Il exercera le jugement parmi les nations, et reprendra plusieurs peuples. De leurs épées ils forgeront des hoyaux, et de leurs lances des serpes ; une nation ne lèvera plus l'épée contre une autre et ils ne s'adonneront plus à la guerre.
\VS{5}Venez, ô maison de Jacob, et marchons dans la lumière de Yahweh.
\TextTitle{L'orgueilleux abaissé au jour de Yahweh}
\VS{6}Certes tu as rejeté ton peuple, la maison de Jacob, parce qu'ils se sont remplis d'orient et adonnés à la divination comme les Philistins, et parce qu'ils s'allient aux enfants des étrangers\FTNT{De. 18:8-13 ; Os. 13:2 ; Mi. 5:11-13.}.
\VS{7}Son pays est rempli d'argent et d'or, et il n'y a pas de fin à ses trésors ; son pays est rempli de chevaux, et il n'y a pas de fin à ses chars.
\VS{8}Son pays est rempli d'idoles ; ils se prosternent devant l'ouvrage de leurs mains et devant ce que leurs doigts ont fabriqué.
\VS{9}Et ceux du commun sont abattus, et les personnes de qualité sont abaissées ; ne leur pardonne donc point.
\VS{10}Entre dans les rochers et cache-toi dans la poussière, à cause de la frayeur de Yahweh, et à cause de la gloire de sa majesté\FTNT{Ap. 6:15-16.}.
\VS{11}Les yeux hautains des hommes seront abaissés et les hommes qui s'élèvent seront humiliés, Yahweh sera seul haut élevé en ce jour-là.
\VS{12}Car il y a un jour assigné par Yahweh des armées contre tout homme orgueilleux et hautain, et contre tout homme qui s'élève, afin qu'il soit abaissé ;
\VS{13}contre tous les cèdres du Liban, hauts et élevés, et contre tous les chênes de Basan ;
\VS{14}contre toutes les hautes montagnes, et contre toutes les collines élevées ;
\VS{15}contre toutes les hautes tours, et contre toutes les murailles fortes ;
\VS{16}contre tous les navires de Tarsis, et contre toutes les peintures de plaisance.
\VS{17}Et l'arrogance des hommes sera humiliée, et les hommes qui s'élèvent seront abaissés :
\VS{18}Yahweh seul sera élevé en ce jour-là. Quant aux idoles, elles tomberont toutes.
\VS{19}Et les hommes entreront dans les cavernes des rochers et dans les trous de la terre, à cause de la frayeur de Yahweh et à cause de sa gloire magnifique, lorsqu'il se lèvera pour faire trembler la terre.
\VS{20}En ce jour-là, les hommes jetteront aux taupes et aux chauves-souris leurs idoles d'argent et leurs idoles d'or, qu'ils s'étaient faites pour se prosterner devant elles ;
\VS{21}et ils entreront dans les fentes des rochers et dans les creux des rochers, à cause de la frayeur de Yahweh, et à cause de sa gloire magnifique, quand il se lèvera pour punir la terre.
\VS{22}Retirez-vous de l'homme, dans les narines duquel il n'y a qu'un souffle : Car quel cas mérite-t-il qu'on en fasse ?
\Chap{3}
\TextTitle{Le péché, cause de dissolution nationale}
\VerseOne{}Car voici, le Seigneur, Yahweh des armées, va ôter de Jérusalem et de Juda tout appui et toute ressource, toute ressource de pain et toute ressource d'eau.
\VS{2}L'homme fort et l'homme de guerre, le juge et le prophète, le devin et l'ancien,
\VS{3}le chef de cinquante et l'homme d'autorité, le conseiller, l'expert d'entre les artisans et l'habile enchanteur.
\VS{4}Et je leur donnerai de jeunes gens pour chefs, et des enfants domineront sur eux.
\VS{5}Le peuple sera opprimé ; l'un opprimera l'autre, chacun son prochain. Le jeune homme se portera arrogamment contre le vieillard, et l'homme de rien contre l'honorable.
\VS{6}Même un homme ira jusqu'à saisir son frère dans la maison paternelle et lui dira : Tu as un manteau, sois notre chef ! Et prends en main ces ruines !
\VS{7}Ce jour même il répondra : Je ne suis pas médecin, et dans ma maison il n'y a ni pain ni manteau ; ne m'établissez donc pas chef du peuple.
\VS{8}Certes Jérusalem est renversée, et Juda est tombée, parce que leurs langues et leurs actions sont contre Yahweh, pour braver les regards de sa gloire.
\VS{9}L'aspect de leur visage témoigne contre eux, ils publient leur péché comme Sodome, ils ne le cachent pas. Malheur à leur âme, car ils ont attiré le mal sur eux !
\VS{10}Dites au juste que du bien lui arrivera, car il mangera le fruit de ses œuvres.
\VS{11}Malheur au méchant qui ne cherche qu'à faire le mal, car la rétribution de ses mains lui sera rendue.
\VS{12}Quant à mon peuple, il a pour oppresseur des enfants, et des femmes dominent sur lui. Mon peuple, ceux qui te conduisent t'égarent, ils corrompent le chemin dans lequel tu marches.
\VS{13}Yahweh se présente pour plaider, il se tient debout pour juger les peuples.
\VS{14}Yahweh entre en jugement avec les anciens de son peuple et avec ses chefs ; car vous avez brouté la vigne, et ce que vous avez ravi au pauvre est dans vos maisons.
\VS{15}Que vous revient-il de fouler mon peuple, et d'écraser le visage des affligés ? Dit le Seigneur, Yahweh des armées.
\TextTitle{Les filles hautaines de Sion}
\VS{16}Yahweh dit aussi : Parce que les filles de Sion sont hautaines, et qu'elles marchent le cou tendu et les yeux pleins de convoitise, parce qu'elles marchent avec une fière démarche faisant du bruit avec leurs pieds,
\VS{17}Yahweh rendra chauve le sommet de la tête des filles de Sion, Yahweh découvrira leur nudité.
\VS{18}En ce temps-là, le Seigneur ôtera l'ornement de leurs anneaux de cheville, et les filets et les croissants ;
\VS{19}les pendants d'oreilles, les bracelets et les voiles ;
\VS{20}les parures de la tête, les chaînettes des pieds et les ceintures, les boîtes à parfum et les amulettes ;
\VS{21}les anneaux et les bagues qui leur pendent sur le nez ;
\VS{22}les vêtements de fête et les larges tuniques, les manteaux et les gibecières ;
\VS{23}les miroirs et les chemises fines, les tiares et les voiles légers.
\VS{24}Et il arrivera qu'au lieu du parfum, il y aura de la puanteur ; au lieu de ceintures, des cordes ; au lieu de cheveux bouclés, des têtes chauves ; au lieu de robes flottantes, des sacs étroits ; et au lieu d'un beau teint, un teint tout hâlé.
\VS{25}Tes hommes tomberont par l'épée et ta force par la guerre.
\VS{26}Et ses portes gémiront et mèneront deuil ; désolée, elle s'assiéra par terre.
\Chap{4}
\TextTitle{Vision du règne messianique\FTNTT{Es. 11:1-16.}}
\VerseOne{}Et en ce jour sept femmes saisiront un seul homme, et diront : Nous mangerons notre pain, et nous nous vêtirons de nos habits ; seulement fais-nous porter ton nom ; ôte notre opprobre.
\VS{2}En ce temps-là, le germe de Yahweh\FTNT{Jésus est le « germe » de Yahweh (Es. 4:2) et le germe de David (Jé. 23:5 ; Za. 3:8 ; Za. 6:12). Ce germe a été placé par la vertu du Saint-Esprit dans le sein d'une vierge (Es. 7:14 ; Lu. 1:34-35) et l'enfant qui naquit d'elle fut appelé « Fils de Dieu » tout en étant le Dieu Tout-Puissant. Il existe de toute éternité en forme de Dieu (Jn. 1:1 ; Es. 9:5), mais il a été fait chair pour nous sauver (Jn. 1:14. 1 Ti. 3:16).
C'est le plus grand des miracles et la démonstration de sa divinité, de sa sagesse et de son amour envers les hommes.
} sera plein de noblesse et de gloire, et le fruit de la terre plein de grandeur et d'excellence pour les réchappés d'Israël.
\VS{3}Et il arrivera que les restes de Sion, et les restes de Jérusalem, seront appelés saints; et ceux de Jérusalem seront inscrits parmi les vivants\FTNT{Es. 10:20-22 ; Ro. 9:27 ; Ro. 11:5 ;}.
\VS{4}Quand le Seigneur aura lavé la souillure des filles de Sion, et purifié Jérusalem du sang qui est au milieu d'elle, par l'esprit de jugement et par l'esprit qui consume;
\VS{5}aussi Yahweh créera, sur toute l'étendue du mont Sion et sur ses assemblées, une nuée avec une fumée pendant le jour, et une splendeur de feu flamboyant pendant la nuit, car la gloire se répandra partout.
\VS{6}Et il y aura un tabernacle pour donner de l'ombre contre la chaleur du jour, pour servir de refuge et d'asile contre la tempête et la pluie\FTNT{Ap. 21:3.}.
\Chap{5}
\TextTitle{Israël, vigne de Yahweh}
\VerseOne{}Je chanterai maintenant pour mon bien-aimé le cantique de mon bien-aimé sur sa vigne. Mon bien-aimé avait une vigne sur un coteau fertile.
\VS{2}Il l'environna d'une haie, en ôta les pierres, et y planta des ceps exquis ; il bâtit une tour au milieu d'elle, et il y creusa aussi une cuve. Puis il espéra qu'elle produirait des raisins, mais elle a produit des grappes sauvages\FTNT{Lu. 13:6-9.}.
\VS{3}Maintenant donc, vous habitants de Jérusalem et vous hommes de Juda, jugez, je vous prie, entre moi et ma vigne.
\VS{4}Qu'y avait-il encore à faire à ma vigne que je ne lui aie fait ? Pourquoi, quand j'ai attendu qu'elle produirait des raisins, a-t-elle produit des grappes sauvages ?
\VS{5}Maintenant donc je vous dirai ce que je vais faire à ma vigne : J'ôterai sa haie, et elle sera broutée ; je romprai sa clôture et elle sera foulée.
\VS{6}Et je la réduirai en désert, elle ne sera plus taillée, ni cultivée ; les ronces et les épines y croîtront ; et je commanderai aux nuées qu'elles ne laissent plus tomber de pluie sur elle.
\VS{7}Or la maison d'Israël est la vigne de Yahweh des armées, et les hommes de Juda sont la plante en laquelle il prenait plaisir. Il en attendait de la droiture, et voici du saccagement ! De la justice, et voici des cris de détresse !
\TextTitle{Six malheurs en punition de l'infidélité d'Israël}
\VS{8}Malheur à ceux qui ajoutent maison à maison, et qui joignent champ à champ, jusqu'à ce qu'il n'y ait plus d'espace et qu'ils habitent seuls au milieu du pays.
\VS{9}Yahweh des armées m'a fait entendre : Certainement, ces maisons nombreuses seront réduites en désolation, ces maisons grandes et belles seront sans habitants.
\VS{10}Même dix arpents de vigne ne produiront qu'un bath, et un homer de semence ne produira qu'un épha.
\VS{11}Malheur à ceux qui se lèvent de bon matin, qui recherchent les boissons fortes, qui demeurent jusqu'au soir, et jusqu'à ce que le vin les échauffe !
\VS{12}La harpe et le luth, le tambourin, la flûte et le vin sont dans leurs festins ; mais ils ne regardent pas l'œuvre de Yahweh, et ils ne voient pas l'ouvrage de ses mains.
\VS{13}C'est pourquoi mon peuple sera emmené captif, parce qu'il n'a pas de connaissance\FTNT{2 R. 24:14-16 ; Os. 4:6.} ; et les plus honorables parmi eux seront des pauvres qui mourront de faim, et leur multitude sera asséchée par la soif.
\VS{14}C'est pourquoi le scheol s'élargit, il ouvre sa gueule outre mesure ; et sa magnificence y descend, sa multitude, sa pompe et tous ceux qui s'y réjouissent.
\VS{15}Ceux du commun seront abattus, les personnes de qualité seront humiliées, et les yeux des hautains seront humiliés.
\VS{16}Et Yahweh des armées sera haut élevé en jugement, et le Dieu saint sera sanctifié dans la justice.
\VS{17}Les agneaux paîtront selon qu'ils seront parqués, et les étrangers dévoreront les champs désolés des riches.
\VS{18}Malheur à ceux qui tirent l'iniquité avec des cordes de vanité, et le péché avec les traits d'un char,
\VS{19}et qui disent : Qu'il hâte et qu'il fasse venir son œuvre bientôt, afin que nous la voyions ! Que le conseil du Saint d'Israël s'avance et vienne, afin que nous le connaissions !
\VS{20}Malheur à ceux qui appellent le mal bien et le bien mal\FTNT{Mi. 7:2.} ; qui font les ténèbres lumière, et la lumière ténèbres ; qui font l'amertume douceur, et la douceur amertume.
\VS{21}Malheur à ceux qui sont sages à leurs yeux, en se considérant eux-mêmes intelligents !
\VS{22}Malheur à ceux qui sont forts pour boire le vin et vaillants pour mêler des boissons fortes ;
\VS{23}qui justifient le méchant pour des présents, et qui ôtent à chacun des justes sa justice.
\VS{24}C'est pourquoi, comme le flambeau de feu consume le chaume, et la flamme consume l'herbe sèche, ainsi leur racine sera comme la pourriture, et leur fleur sera détruite comme la poussière ; parce qu'ils ont rejeté la loi de Yahweh des armées, et ils ont méprisé la parole du Saint d'Israël.
\VS{25}C'est pourquoi la colère de Yahweh s'enflamme contre son peuple, il étend sa main sur lui, et il le frappe ; les montagnes tremblent, et leurs cadavres ont été mis en pièces au milieu des rues. Malgré tout cela, sa colère ne se détourne pas, mais sa main est encore étendue.
\VS{26}Il élève une bannière pour les nations éloignées, et il siffle à chacune d'elles depuis les extrémités de la terre ; et voici chacune viendra promptement et légèrement.
\VS{27}Nul n'est fatigué, nul ne chancelle de lassitude, personne ne sommeille ni ne dort ; et la ceinture de leurs reins ne sera point déliée, et la courroie de leurs souliers ne sera point rompue.
\VS{28}Leurs flèches sont aiguës et tous leurs arcs tendus ; les sabots de leurs chevaux ressemblent à des cailloux, et les roues de leurs chars à un tourbillon.
\VS{29}Leur rugissement est comme celui d'un vieux lion ; ils rugissent comme des lionceaux ; ils grondent et saisissent la proie, il l'emportent et personne ne vient à son secours.
\VS{30}En ce jour-là, on mènera un bruit sur lui, semblable au mugissement de la mer ; en regardant la terre, on ne verra que ténèbres et détresse ; la lumière sera obscurcie dans le ciel.
\Chap{6}
\TextTitle{Révélation de Yahweh à Esaïe}
\VerseOne{}L'année de la mort du roi Ozias, je vis le Seigneur assis sur un trône haut et élevé, et les pans de sa robe remplissaient le temple\FTNT{2 Ch. 26:23.}.
\VS{2}Les séraphins se tenaient au-dessus de lui ; et chacun d'eux avait six ailes ; deux dont ils se couvraient la face, deux dont ils se couvraient les pieds et deux dont ils se servaient pour voler.
\VS{3}Et ils criaient l'un à l'autre, et disaient : Saint, saint, saint est Yahweh des armées ! Toute la terre est pleine de sa gloire !
\VS{4}Et les poteaux des seuils furent ébranlés dans leurs fondements par la voix de celui qui criait ; et la maison fut remplie de fumée.
\VS{5}Alors je dis : Malheur à moi ! Je suis perdu, car je suis un homme dont les lèvres sont impures, j'habite au milieu d'un peuple dont les lèvres sont impures et mes yeux ont vu le Roi, Yahweh des armées\FTNT{Jg. 13:21-22.}.
\VS{6}Mais l'un des séraphins vola vers moi, tenant à la main un charbon ardent, qu'il avait pris sur l'autel avec des pincettes.
\VS{7}Il en toucha ma bouche, et dit : Voici, ceci a touché tes lèvres, c'est pourquoi ton iniquité est ôtée, et la propitiation est faite pour ton péché.
\VS{8}Puis j'entendis la voix du Seigneur, disant : Qui enverrai-je et qui marchera pour nous ? Je répondis : Me voici, envoie-moi.
\TextTitle{Mission d'Esaïe}
\VS{9}Et il dit : Va et dis à ce peuple : En entendant vous entendrez, mais vous ne comprendrez point ; et en voyant vous verrez, mais vous n'apercevrez point.
\VS{10}Engraisse le cœur de ce peuple, et rends ses oreilles pesantes, et bouche-lui les yeux ; de peur qu'il ne voie de ses yeux, et qu'il n'entende de ses oreilles, et que son cœur ne comprenne, et qu'il ne se convertisse, et qu'il ne recouvre la santé\FTNT{Mt. 13:15 ; Mc. 4:12 ; Jn. 12:40 ; Ac. 28:27.}.
\VS{11}Je dis : Jusqu'à quand, Seigneur ? Et il répondit : Jusqu'à ce que les villes soient dévastées, jusqu'à ce qu'il n'y ait plus d'habitants, ni d'hommes dans les maisons, et que la terre soit mise en entière désolation ;
\VS{12} et que Yahweh ait dispersé au loin les hommes, et que l'abandon ait été grand au milieu du pays.
\VS{13}Toutefois s'il y reste un dixième des habitants, ils reviendront pour être la proie des flammes. Mais comme le térébinthe et le chêne conservent leur tronc quand ils sont abattus, une sainte postérité renaîtra de ce peuple\FTNT{Ro. 11:17-25.}.
\Chap{7}
\TextTitle{Retsin et Pékach complote contre Juda}
\VerseOne{}Or il arriva du temps d'Achaz, fils de Jotham, fils d'Ozias, roi de Juda, que Retsin, roi de Syrie, et Pékach, fils de Remalia, roi d'Israël, montèrent contre Jérusalem pour lui faire la guerre ; mais ils ne purent l'assiéger.
\VS{2}Et on rapporta à la maison de David : La Syrie s'est reposée sur Ephraïm. Et le cœur d'Achaz, et le cœur de son peuple furent ébranlés comme les arbres des forêts qui sont ébranlés par le vent.
\VS{3}Alors Yahweh dit à Esaïe : Sors maintenant au devant d'Achaz, toi et Schear-Jaschub, ton fils, vers l'extrémité de l'aqueduc de l'étang supérieur, sur la route du champ du foulon.
\VS{4}Et dis-lui : Prends garde à toi, et demeure tranquille, ne crains point, et que ton cœur ne devienne point lâche à cause des deux queues de ces tisons fumants, à cause de l'ardeur, dis-je, de la colère de Retsin et de la Syrie, et du fils de Remalia,
\VS{5}de ce que la Syrie délibère avec Ephraïm et le fils de Remalia de te faire du mal, en disant :
\VS{6}Montons contre Juda, assiégeons la ville, battons-la en brèche, et établissons pour roi le fils de Tabeel au milieu d'elle.
\VS{7}Ainsi parle le Seigneur, Yahweh : Cela n'aura point d'effet, et cela ne se fera point.
\VS{8}Car la tête de la Syrie c'est Damas, et le chef de Damas c'est Retsin. Encore soixante-cinq ans, Ephraïm sera froissé pour n'être plus un peuple.
\VS{9}Et la tête d'Ephraïm c'est la Samarie, et le chef de la Samarie c'est le fils de Remalia. Si vous ne croyez pas, certainement vous ne serez point affermis.
\TextTitle{Annonce de la naissance d'Emmanuel}
\VS{10}Et Yahweh parla de nouveau à Achaz, en disant :
\VS{11}Demande pour toi un signe à Yahweh ton Dieu, demande-le, soit dans les bas lieux, soit dans les lieux élevés.
\VS{12}Et Achaz répondit : Je ne demanderai rien, et je ne tenterai point Yahweh.
\VS{13}Alors Esaïe dit : Ecoutez maintenant, ô maison de David ! Est-ce trop peu pour vous de lasser les hommes, que vous lassiez aussi mon Dieu ?
\VS{14}C'est pourquoi le Seigneur lui-même vous donnera un signe : Voici, une vierge sera enceinte, et elle enfantera un fils, et elle lui donnera le nom d'Emmanuel\FTNT{Le nom « Emmanuel » est dérivé de l'hébreu « Immanuw'el » qui signifie « Dieu est avec nous ». Jésus a dit aux disciples dans Mt. 28:20 : « Et moi, je suis avec vous tous les jours jusqu'à la fin des temps ». Jésus est Emmanuel, Dieu avec nous jusqu'à la fin des temps.}.
\VS{15}Il mangera du lait et du miel, jusqu'à ce qu'il sache rejeter le mal et choisir le bien.
\VS{16}Mais avant que l'enfant sache rejeter le mal et choisir le bien, la terre que tu as en détestation sera abandonnée par ses deux rois.
\TextTitle{Prophétie sur l'imminente invasion de Juda\FTNTT{2 Ch. 28:1-20.}}
\VS{17}Yahweh fera venir sur toi, sur ton peuple et sur la maison de ton père, par le roi d'Assyrie, des jours tels qu'il n'y en a point eu de semblable depuis le jour où Ephraïm s'est séparé de Juda.
\VS{18}Et il arrivera qu'en ce jour-là, Yahweh sifflera aux mouches qui sont à l'extrémité des ruisseaux d'Egypte, et aux abeilles qui sont au pays d'Assyrie.
\VS{19}Elles viendront, et se poseront dans toutes les vallées désertes, et dans les fentes des rochers, et par tous les buissons, et par tous les halliers.
\VS{20}En ce jour-là, le Seigneur rasera avec le rasoir pris à louage au-delà du fleuve, avec le roi d'Assyrie, la tête et les poils des pieds, et il enlèvera aussi la barbe\FTNT{2 R. 16:5-9.}.
\VS{21}Et il arrivera, en ce jour-là, qu'un homme nourrira une jeune vache et deux brebis.
\VS{22}Et il arrivera que de l'abondance du lait qu'elles rendront, il mangera du beurre ; car tous ceux qui seront restés dans le pays mangeront du beurre et du miel.
\VS{23}Et il arrivera, en ce jour-là, que tout lieu où il y aura mille vignes, valant mille sicles d'argent, sera réduit en ronces et en épines.
\VS{24}On y entrera avec des flèches et avec l'arc, car tout le pays ne sera que ronces et épines.
\VS{25}Et dans toutes les montagnes que l'on cultivait avec la bêche, on ne craindra plus de voir des ronces et des épines ; mais on y lâchera les bœufs, et la brebis en foulera le sol.
\Chap{8}
\TextTitle{Annonce de la défaite de Damas et de la Samarie}
\VerseOne{}Et Yahweh me dit : Prends un grand rouleau et écris dessus en grosses lettres : Qu'on se dépêche de butiner, qu'on se hâte de piller.
\VS{2}Et je pris avec moi des témoins fidèles : Urie, le sacrificateur, et Zacharie, fils de Bérékia.
\VS{3}Puis je m'étais approché de la prophétesse ; elle conçut et elle enfanta un fils. Et Yahweh me dit : Donne-lui pour nom Maher-Schalal-Chasch-Baz\FTNT{« Maher-Schalal-Chasch-Baz » signifie « rapide au butin, rapide sur la proie ».}.
\VS{4}Car avant que l'enfant sache dire : Mon père ! Ma mère ! On enlèvera la puissance de Damas et le butin de Samarie, devant le roi d'Assyrie.
\VS{5}Et Yahweh continua encore de me parler, en disant :
\VS{6}Parce que ce peuple a rejeté les eaux de Siloé qui coulent doucement, et qu'il s'est réjoui au sujet de Retsin, et du fils de Remalia,
\VS{7}à cause de cela, voici, le Seigneur va faire monter contre eux les puissantes et grandes eaux du fleuve : Le roi d'Assyrie et toute sa gloire. Il s'élèvera partout au-dessus de son lit, et il se répandra sur toutes ses rives.
\VS{8}Et il pénétrera dans Juda, il débordera et inondera, il atteindra jusqu'au cou. Et les étendues de ses ailes rempliront la largeur de ton pays, ô Emmanuel !
\TextTitle{Exhortation aux disciples de Yahweh à rester fidèles}
\VS{9}Alliez-vous, peuples ! Et vous serez brisés ; prêtez l'oreille, vous tous qui êtes d'un pays éloigné ! Equipez-vous, et vous serez brisés ; équipez-vous, et vous serez brisés.
\VS{10}Prenez conseil, et il sera dissipé ; dites la parole, et elle sera sans effet : Car Dieu est avec nous.
\VS{11}Car ainsi m'a parlé Yahweh, avec une main forte, et il m'instruisit de ne point aller par le chemin de ce peuple-ci, en me disant :
\VS{12}Ne dites point : Conjuration, toutes les fois que ce peuple dit conjuration ; ne craignez point ce qu'il craint, et ne vous en épouvantez point.
\VS{13}Sanctifiez Yahweh des armées, lui-même, c'est lui que vous devez craindre et redouter.
\VS{14}Et il sera un sanctuaire, mais aussi une pierre d'achoppement\FTNT{Yahweh s'est présenté comme une pierre d'achoppement et un rocher de scandale. En Es. 44:8 il affirme d'ailleurs ne pas connaître d'autre rocher que lui. Esaïe n'est pas le seul prophète à qui le Seigneur s'est révélé comme étant une pierre et un rocher. Dans le Ps. 118:22-23, il est dit : « La pierre qu'ont rejetée ceux qui bâtissaient est devenue la principale de l'angle ». Daniel et Zacharie ont également prophétisé au sujet de cette pierre : « Tu regardais, lorsqu'une pierre se détacha sans le secours d'aucune main, frappa les pieds de fer et d'argile de la statue, et les mit en pièces. Mais la pierre qui avait frappé la statue devint une grande montagne, et remplit toute la terre » (Da. 2:34-35). « Car voici, pour ce qui est de la pierre que j'ai placée devant Josué, il y a sept yeux sur cette seule pierre ; voici, je graverai moi-même ce qui doit y être gravé, dit Yahweh des armées; et j'enlèverai l'iniquité de ce pays, en un jour » (Za. 3:9). Ces prophéties se sont accomplies en Jésus-Christ, l'Agneau de Dieu qui ôte le péché du monde (Jn. 1:29). Le Seigneur s'est d'ailleurs clairement identifié à la pierre angulaire, affirmant ainsi sa divinité (Lu. 20:17-19). En Mt. 16:18, il s'est présenté comme le rocher inébranlable sur lequel il allait bâtir son Eglise. De plus, il est à noter que dans le livre de l'Apocalypse, l'Agneau possède sept yeux comme la pierre vue par Zacharie (Ap. 5:6). Ces sept yeux sont aussi les sept lampes du chandelier d'or que Zacharie et Jean avaient également vues (Za. 4:2 ; Ap. 4:5 ). Or le chiffre sept symbolise la plénitude et la perfection divines. Esaïe prophétisa encore en ces termes : « Voici, j'ai mis pour fondement en Sion une pierre, une pierre éprouvée, une pierre angulaire de prix, solidement posée; celui qui la prendra pour appui n'aura point hâte de fuir » (Es. 28:16). Les écrits de la Nouvelle Alliance attestent l'accomplissement de cette prophétie en Jésus-Christ, notamment par la bouche de Paul et de Pierre : « Vous avez été édifiés sur le fondement des apôtres et des prophètes, Jésus-Christ lui-même étant la pierre angulaire » (Ep. 2:20). « Car personne ne peut poser un autre fondement que celui qui a été posé, savoir Jésus-Christ » (1 Co. 3:11). « Approchez-vous de lui, pierre vivante, rejetée par les hommes, mais choisie et précieuse devant Dieu ; et vous-mêmes, comme des pierres vivantes, édifiez-vous pour former une maison spirituelle, un saint sacerdoce, afin d'offrir des victimes spirituelles, agréables à Dieu, par Jésus-Christ ». (1 Pi. 2:4-5).}, un rocher de scandale pour les deux maisons d'Israël, un filet et un piège pour les habitants de Jérusalem.
\VS{15}Plusieurs d'entre eux trébucheront, ils tomberont et se briseront, ils seront enlacés et pris.
\VS{16}Enveloppe ce témoignage, scelle cette loi\FTNTT{"towrah" ou "torah" en hébreu.} parmi mes disciples.
\VS{17}Je m'attends à Yahweh, qui cache sa face à la maison de Jacob, et je regarde à lui.
\VS{18}Me voici, avec les enfants que Yahweh m'a donnés, pour être un signe et un miracle en Israël, de la part de Yahweh des armées, qui habite sur la montagne de Sion.
\VS{19}Si l'on vous dit : Consultez ceux qui évoquent les morts et les diseurs de bonne aventure, qui poussent des sifflements et des soupirs, répondez : Un peuple ne consultera-t-il pas son Dieu ? S'adressera-t-il aux morts en faveur des vivants ?
\VS{20}A la loi et au témoignage ! Si l'on ne parle pas ainsi, il n'y aura certainement point d'aurore pour le peuple.
\VS{21}Et il sera errant dans le pays, accablé et affamé ; et il arrivera que dans sa faim, il s'irritera, maudira son roi et son Dieu, et tournera les yeux en haut ;
\VS{22}puis il regardera vers la terre, et voici, il n'y aura que détresse, ténèbres et de sombres angoisses : Il sera enfoncé dans l'obscurité.
\VS{23}Mais l'obscurité ne sera pas autant qu'elle avait été dans son humiliation ; quand au commencement, il affligea légèrement le pays de Zabulon et le pays de Nephthali, et ensuite, l'affligea plus sévèrement près de la mer, au-delà du Jourdain, dans la Galilée des Gentils.
\Chap{9}
\TextTitle{Annonce de la naissance et du règne du Messie}
\VerseOne{}Le peuple qui marchait dans les ténèbres voit une grande lumière, et la lumière resplendit sur ceux qui habitaient le pays de l'ombre de la mort\FTNT{Mt. 4:15-16.}.
\VS{2}Tu multiplies la nation, tu lui accordes de grandes joies, ils se réjouissent devant toi, comme on se réjouit à la moisson, comme on s'égaye quand on partage le butin.
\VS{3}Car tu as mis en pièces le joug dont il était chargé, et le bâton dont on lui battait ordinairement les épaules, et la verge de celui qui l'opprimait, comme au jour de Madian.
\VS{4}Parce que toute bataille de guerrier se fait dans un bruit confus, et que le vêtement est vautré dans le sang ; mais ceci sera comme un embrasement, quand le feu dévore quelque chose.
\VS{5}Car un enfant nous est né, un Fils nous a été donné\FTNT{Jésus-Christ est 100\% Dieu  et 100\% homme. Il  existe depuis toute éternité en tant que Dieu. Il est devenu homme au moment de son incarnation (Ph. 2 :5-7).}, et l'empire reposera sur son épaule : On l'appellera l'Admirable, le Conseiller, le Dieu Puissant, le Père d'éternité\FTNT{Philippe, disciple de Jésus-Christ voulait rencontrer le Père. Il posa au Seigneur cette question « Seigneur, montre-nous le Père, et cela nous suffit » (Jn. 14:8). Jésus lui répondit : « Il y a si longtemps que je suis avec vous, et tu ne m'as pas connu, Philippe ! » (Jn. 14:9).}, le Prince de paix,
\VS{6}pour accroître l'empire, et une paix sans fin au trône de David et à son royaume, pour l'affermir et le soutenir par le droit et par la justice, dès maintenant et à toujours\FTNT{Lu. 1:32-33.}. Voilà ce que fera le zèle de Yahweh des armées.
\TextTitle{Jugement sur le royaume du nord}
\VS{7}Le Seigneur envoie une parole à Jacob, et elle tombe sur Israël\FTNT{Ge. 32:28.}.
\VS{8}Et tout le peuple en aura connaissance, Ephraïm et les habitants de Samarie, qui disent avec orgueil et avec un cœur hautain :
\VS{9}Des briques sont tombées, mais nous bâtirons en pierres de taille ; des sycomores ont été coupés, mais nous les changerons en cèdres.
\VS{10}Yahweh élèvera contre eux les ennemis de Retsin, et il armera les ennemis d'Israël ;
\VS{11}la Syrie à l'orient, et les Philistins à l'occident ; et ils dévoreront Israël à gueule ouverte. Malgré tout cela, sa colère ne s'apaise point, et sa main est encore étendue.
\VS{12}Parce que le peuple ne revient pas à celui qui le frappe, et il ne cherche pas Yahweh des armées.
\VS{13}A cause de cela Yahweh retranchera d'Israël en un seul jour la tête et la queue, la branche de palmier et le roseau.
\VS{14}L'ancien et le magistrat, c'est la tête ; et le prophète qui enseigne le mensonge, c'est la queue.
\VS{15}Ceux donc qui font croire à ce peuple qu'il est heureux sont des séducteurs\FTNT{1 Ti. 4:1 ; Tit. 1:10.}; et ceux qui se laissent diriger par eux se perdent.
\VS{16}C'est pourquoi le Seigneur ne saurait prendre plaisir à leurs jeunes hommes ni avoir pitié de leurs orphelins et de leurs veuves, car tous sont des hypocrites et des méchants, et toute bouche ne profère que des infamies. Malgré tout cela, sa colère ne s'apaise point et sa main est encore étendue.
\VS{17}Car la méchanceté consume comme un feu, elle dévore les ronces et les épines ; elle embrase l'épaisseur de la forêt, d'où s'élèvent des colonnes de fumée.
\VS{18}A cause de la fureur de Yahweh des armées, la terre est obscurcie, et le peuple est comme la proie du feu ; nul n'a compassion de son frère.
\VS{19}On pille à droite, et l'on a faim ; on dévore à gauche, et l'on n'est pas rassasié ; chacun mange la chair de son bras.
\VS{20}Manassé dévore Ephraïm, Ephraïm dévore Manassé, et ensemble ils fondent sur Juda. Malgré tout cela, sa colère ne s'apaise point, et sa main est encore étendue.
\Chap{10}
\VerseOne{}Malheur à ceux qui décrètent des ordonnances iniques, et à ceux qui écrivent pour ordonner l'oppression,
\VS{2}pour refuser la justice aux pauvres et ravir leur droit aux malheureux de mon peuple, afin d'avoir les veuves pour leur butin, et de piller les orphelins !
\VS{3}Et que ferez-vous au jour de la visitation, et de la ruine éclatante qui viendra de loin ? Vers qui fuirez-vous pour avoir du secours et où laisserez-vous votre gloire\FTNT{Os. 9:7 ; Mt. 24:17-21 ; Lu. 19:41-44.} ?
\VS{4}Les uns seront courbés parmi les prisonniers, les autres tomberont parmi les morts. Malgré tout cela, sa colère ne s'apaise point, et sa main est encore étendue.
\TextTitle{Jugement sur l'Assyrie}
\VS{5}Malheur à l'Assyrie, verge de ma colère ! La verge dans leur main c'est l'instrument de ma colère.
\VS{6}Je l'ai envoyé contre une nation impie, et je l'ai fait marcher contre le peuple de ma fureur, afin qu'il se livre au pillage et fasse du butin, pour qu'il le foule aux pieds comme la boue des rues.
\VS{7}Mais il n'en juge pas ainsi, et ce n'est pas là la pensée de son cœur ; il ne songe qu'à détruire, qu'à exterminer beaucoup de nations.
\VS{8}Car il dit : Mes princes ne sont-ils pas autant de rois ?
\VS{9}Calno n'est-elle pas comme Carkemisch ? Hamath n'est-elle pas comme Arpad ? Et Samarie n'est-elle pas comme Damas ?
\VS{10}Puisque ma main a soumis les royaumes qui avaient des idoles, où il y avait plus d'images taillées qu'à Jérusalem et à Samarie,
\VS{11}ne ferai-je pas aussi à Jérusalem et à ses dieux, comme j'ai fait à Samarie et à ses idoles ?
\VS{12}Mais il arrivera que, quand le Seigneur aura achevé toute son œuvre sur la montagne de Sion et à Jérusalem, je punirai le roi d'Assyrie pour le fruit de son cœur orgueilleux, et pour la gloire de ses regards hautains.
\VS{13}Parce qu'il dit : C'est par la force de ma main que j'ai agi, c'est par ma sagesse, car je suis intelligent ; j'ai reculé les bornes des peuples, et j'ai pillé ce qu'ils avaient de plus précieux ; et comme un homme vaillant, j'ai fait descendre ceux qui étaient assis.
\VS{14}Ma main a trouvé les richesses des peuples, comme on trouve un nid ; comme on rassemble des œufs délaissés, ainsi ai-je rassemblé toute la terre ; nul n'a remué l'aile, ni ouvert le bec, ni poussé un cri.
\VS{15}La hache se glorifie-t-elle envers celui qui s'en sert ? Ou la scie s'élève-t-elle au-dessus de celui qui la manie ? Comme si la verge faisait mouvoir celui qui la lève, et que le bâton se levait comme s'il n'était pas du bois !
\VS{16}C'est pourquoi le Seigneur, Yahweh des armées, enverra la maigreur sur ses hommes gras ; et sous sa gloire éclatera l'embrasement d'un feu.
\VS{17}Car la lumière d'Israël deviendra un feu, et son Saint une flamme qui embrasera et consumera ses épines et ses ronces tout en un jour ;
\VS{18}et il consumera la gloire de sa forêt et de ses campagnes, depuis l'âme jusqu'à la chair. Il en sera comme quand celui qui porte l'étendard est défait.
\VS{19}Le reste des arbres de sa forêt pourra être compté, et un enfant en écrirait le nombre.
\TextTitle{Conversion et délivrance du reste d'Israël}
\VS{20}Et il arrivera en ce jour-là, que le reste d'Israël et les réchappés de la maison de Jacob ne s'appuieront plus sur celui qui les frappait, mais ils s'appuieront avec confiance sur Yahweh, le Saint d'Israël.
\VS{21}Le reste se convertira, le reste, dis-je, de Jacob se convertira au Dieu puissant.
\VS{22}Car quand ton peuple, ô Israël, serait comme le sable de la mer, un reste seulement se convertira ; la destruction est résolue, elle fera déborder la justice.
\VS{23}Car la destruction qu'il a résolue, le Seigneur, Yahweh des armées, va l'exécuter au milieu de toute la terre.
\VS{24}C'est pourquoi ainsi parle le Seigneur, Yahweh des armées : Mon peuple qui habites en Sion, ne crains pas le roi d'Assyrie ; il te frappe de la verge, et il lève son bâton sur toi comme faisait l'Egypte.
\VS{25}Mais encore un peu de temps, un peu de temps, et le châtiment cessera, puis ma colère se tournera contre lui pour l'exterminer.
\VS{26}Et Yahweh des armées lèvera le fouet contre lui, comme il frappa Madian au rocher d'Oreb ; et de même qu'il leva son bâton sur la mer, il le lèvera aussi comme contre les Egyptiens.
\VS{27}En ce jour-là, son fardeau sera ôté de dessus ton épaule et son joug de dessus ton cou ; et l'onction fera rompre le joug.
\TextTitle{Défaite des Assyriens\FTNTT{Es. 35-36 ; 37.7.}}
\VS{28}Il marche sur Ajjath, traverse Migron et il met ses bagages à Micmasch.
\VS{29}Ils passent le défilé, ils couchent à Guéba ; Rama est effrayée ; Guibea de Saül prend la fuite.
\VS{30}Pousse des cris, fille de Gallim ! Malheur à toi Anathoth ! Prends garde Laïs !
\VS{31}Madména se disperse, les habitants de Guébim se sauvent en foule.
\VS{32}Encore un jour d'arrêt à Nob, et il menace de sa main la montagne de la fille de Sion, la colline de Jérusalem.
\VS{33}Voici, le Seigneur, Yahweh des armées, brise les rameaux avec force ; et ceux qui sont les plus hauts élevés sont coupés, et les hauts montés sont abaissés.
\VS{34}Et il taille avec le fer les lieux les plus épais de la forêt, et le Liban tombe sous le Puissant.
\Chap{11}
\TextTitle{Rétablissement du règne de David par le Messie}
\VerseOne{}Mais il sortira un rameau du tronc d'Isaï, et un rejeton naîtra de ses racines\FTNT{Mt. 1:6-16 ; Lu. 1:31-32 ; Ro. 15:12 ; Ap. 5:5.}.
\VS{2}L'Esprit de Yahweh reposera sur lui, Esprit de sagesse et d'intelligence, Esprit de conseil et de force, Esprit de connaissance et de crainte de Yahweh\FTNT{Es. 61:1 ; Lu. 4:18}.
\VS{3}Il respirera la crainte de Yahweh, il ne jugera point sur l'apparence et il ne reprendra point sur un ouï-dire\FTNT{Jé. 11:20 ; Mt. 22:16 ; Ap. 2:23.}.
\VS{4}Mais il jugera les pauvres avec justice, et il prononcera avec droiture un jugement sur les malheureux de la terre, et il frappera la terre par la verge de sa bouche, et il fera mourir le méchant par le souffle de ses lèvres\FTNT{Job. 4:9 ; Job. 15:30 ; 2 Thess. 2:8.}.
\VS{5}La justice sera la ceinture de ses reins, et la fidélité, la ceinture de ses flancs\FTNT{Ep. 6:14.}.
\VS{6}Le loup habitera avec l'agneau, et le léopard se couchera avec le chevreau ; le veau, et le lionceau, et le bétail qu'on engraisse seront ensemble, et un petit enfant les conduira.
\VS{7}La jeune vache paîtra avec l'ourse, leurs petits auront un même gîte, et le lion, comme le bœuf, mangera de la paille\FTNT{Es. 65:25.}.
\VS{8}Le nourrisson s'ébattra sur l'antre de l'aspic, et l'enfant sevré mettra sa main dans la caverne du vipère.
\VS{9}Il ne se fera ni tort ni dommage sur toute ma montagne sainte, car la terre sera remplie de la connaissance de Yahweh, comme le fond de la mer des eaux qui le couvrent.
\VS{10}En ce jour-là, les nations rechercheront le rejeton d'Isaï qui sera comme une bannière\FTNT{Jésus, notre bannière doit être élevé afin que les pécheurs soient sauvés (Jn. 3:14-15). Le livre du Cantique des cantiques est une superbe image de l'amour de Dieu manifesté en Jésus-Christ, pour nous son Église, qui sommes sa bien-aimée. « Il m'a fait entrer dans la maison du vin ; et la bannière qu'il déploie sur moi, c'est l'amour » (Ca. 2:4). Le vin dont il est question c'est le Saint-Esprit que le Seigneur déverse sur nous jour après jour et qui nous désaltère spirituellement. « Moïse bâtit un autel, et lui donna pour nom : Yahweh ma bannière » (Ex. 17:15). Autrefois, lors des combats, les différentes armées portaient bien haut leur bannière en tête des troupes pour témoigner de leur appartenance et pour indiquer pour quel pays elles combattaient. En portant en nous Jésus, nous proclamons notre appartenance à Dieu et à son Royaume. Le Ps. 20:6 dit ceci : « Nous nous réjouirons de ton salut, nous lèverons l'étendard au nom de notre Dieu ; Yahweh exaucera tous tes vœux ». Et dans le Ps. 60:6 il est dit : « Tu as donné à ceux qui te craignent une bannière pour qu'elle s'élève à cause de la vérité ». La vérité se trouve en Jésus-Christ qui est le chemin, la vérité et la vie (Jn. 14:6). En élevant Jésus comme notre bannière, nous proclamons l'œuvre parfaite accomplie à la croix. Jésus, notre bannière, est le point de rassemblement des chrétiens de tous horizons. Ce rassemblement forme le corps de Christ, l'Eglise dont nous sommes les membres. Tout comme les douze tribus d'Israël se réunissaient pour combattre, nous nous réunissons tous sous la même bannière. Jésus, notre bannière, est le signe de la victoire contre les puissances des ténèbres. En élevant le nom de Jésus comme une bannière, nous faisons fuir toute l'armée de Satan. Dans Jn. 12:32 le Seigneur nous dit : « Et moi, quand j'aurai été élevé de la terre, j'attirerai tous les hommes à moi ». Jésus a été élevé comme étant la bannière qui a réconcilié Dieu avec les pécheurs. Lorsque cette bannière est élevée, les pécheurs sont attirés vers Dieu, ils passent des ténèbres à la lumière, de la mort à la vie.} pour les peuples, et son séjour ne   sera que gloire.
\TextTitle{Etablissement du règne du Messie}
\VS{11}Et il arrivera en ce jour-là, que le Seigneur mettra encore sa main une seconde fois pour acquérir le reste de son peuple dispersé en Assyrie, en Egypte, à Pathros, en Ethiopie, à Elam, à Schinear, à Hamath et dans les îles de la mer.
\VS{12}Il élèvera une bannière parmi les nations, il rassemblera les exilés d'Israël qui auront été chassés, et il recueillera les dispersés de Juda des quatre extrémités de la terre.
\VS{13}Et la jalousie d'Ephraïm sera ôtée, et les oppresseurs de Juda seront retranchés ; Ephraïm ne sera plus jaloux de Juda, et Juda n'opprimera plus Ephraïm.
\VS{14}Mais ils voleront sur l'épaule des Philistins vers la mer ; ils pilleront ensemble les fils de l'orient ; Edom et Moab seront la proie de leurs mains et les enfants d'Ammon leur obéiront.
\VS{15}Yahweh exterminera aussi à la façon de l'interdit la langue de la mer d'Egypte, et il lèvera sa main contre le fleuve par la force de son vent, et il le frappera sur les sept rivières, et fera qu'on y marche avec des souliers.
\VS{16}Et il y aura un chemin pour le reste de son peuple, qui sera échappé de l'Assyrie, comme il y en eut un pour Israël le jour où il remonta du pays d'Egypte.
\Chap{12}
\TextTitle{Louange au sein du royaume}
\VerseOne{}Tu diras en ce jour-là : Je te loue, ô Yahweh ! Car tu as été irrité contre moi, ta colère s'est apaisée, et tu m'as consolé.
\VS{2}Voici, Dieu est ma délivrance, j'aurai confiance et je ne craindrai rien ; car Yahweh, Yahweh est ma force et ma louange ; il est mon Sauveur.
\VS{3}Et vous puiserez de l'eau avec joie aux sources du salut\FTNT{Jn. 4:10-14.},
\VS{4}et vous direz en ce jour-là : Louez Yahweh, invoquez son Nom, publiez ses œuvres parmi les peuples, rappelez que son Nom est une haute retraite !
\VS{5}Psalmodiez à Yahweh car il a fait des choses magnifiques : Cela est connu dans toute la terre !
\VS{6}Habitante de Sion, égaye-toi, et réjouis-toi avec chant de triomphe ! Car le Saint d'Israël est grand au milieu de toi.
\Chap{13}
\TextTitle{Yahweh lève une armée}
\VerseOne{}Prophétie sur Babylone, révélé à Esaïe, fils d'Amots.
\VS{2}Elevez la bannière sur la haute montagne, élevez la voix vers eux, faites des signes avec la main, et qu'on entre dans les portes des magnifiques !
\VS{3}C'est moi qui ai donné des ordres à ceux qui me sont consacrés, j'ai appelé mes hommes forts pour exécuter ma colère, ceux qui se réjouissent de ma grandeur.
\VS{4}Il y a sur les montagnes un bruit d'une multitude, comme celui d'un grand peuple ; on entend un tumulte de royaumes, de nations rassemblées : Yahweh des armées passe en revue l'armée pour le combat.
\VS{5}D'un pays éloigné, de l'extrémité des cieux, Yahweh vient avec les instruments de sa colère pour détruire tout le pays.
\TextTitle{Jugement de Yahweh sur Babylone}
\VS{6}Hurlez, car le jour de Yahweh est proche, il vient comme un ravage du Tout-Puissant.
\VS{7}C'est pourquoi toutes les mains deviennent lâches, et tout cœur d'homme se fond.
\VS{8}Ils sont épouvantés ; les détresses et les douleurs les saisissent ; ils sont en travail comme celle qui enfante ; ils se regardent les uns les autres avec stupeur, leurs visages sont comme des visages enflammés.
\VS{9}Voici, le jour de Yahweh arrive, jour cruel, jour de colère et d'ardente fureur\FTNT{Mal. 4:1 ; Ap. 19:15}, qui réduira le pays en désolation, et en exterminera les pécheurs.
\VS{10}Même les étoiles des cieux et leurs astres ne feront plus briller leur lumière ; le soleil s'obscurcira dès son lever, et la lune ne fera plus resplendir sa lueur\FTNT{Joë. 2:31 ; Mt. 24:29 ; Mc. 13:24.}.
\VS{11}Je punirai le monde habitable à cause de sa malice, et les méchants à cause de leur iniquité ; je ferai cesser l'orgueil des hautains et j'abaisserai l'arrogance des tyrans.
\VS{12}Je ferai qu'un homme sera plus précieux que l'or fin, et une personne plus que l'or d'Ophir.
\VS{13}C'est pourquoi j'ébranlerai les cieux, et la terre sera secouée de sa base\FTNT{Ag. 2:6}, à cause de la fureur de Yahweh des armées, et à cause du jour de son ardente colère.
\VS{14}Et chacun sera comme un chevreuil qui est chassé, et comme une brebis que personne ne retire, chacun se tournera vers son peuple, chacun fuira vers son pays.
\VS{15}Quiconque sera trouvé, sera transpercé ; et quiconque s'y sera joint, tombera par l'épée.
\VS{16}Et leurs petits enfants seront écrasés sous leurs yeux\FTNT{Na. 3:10.}, leurs maisons seront pillées, et leurs femmes violées.
\TextTitle{Yahweh envoie les Mèdes contre Babylone}
\VS{17}Voici, je vais susciter contre eux les Mèdes, qui ne font point cas de l'argent, et qui ne convoitent point l'or.
\VS{18}Leurs arcs écraseront les jeunes gens, et ils seront sans pitié pour le fruit des entrailles, leur œil n'épargnera point les enfants.
\VS{19}Ainsi Babylone, l'ornement des royaumes, la parure et l'orgueil des Chaldéens, sera comme Sodome et Gomorrhe que Dieu détruisit.
\VS{20}Elle ne sera plus jamais habitée, elle ne sera point habitée de génération en génération ; même les Arabes n'y dresseront point leurs tentes, et les bergers n'y feront plus reposer leurs troupeaux.
\VS{21}Mais les bêtes sauvages des déserts y prendront leur gîte, et les hiboux rempliront ses maisons, les autruches en feront leur demeure, et les boucs y sauteront.
\VS{22}Les chacals hurleront dans ses palais, et les dragons dans ses maisons de plaisance. Son temps est près d'arriver et ses jours ne se prolongeront pas.
\Chap{14}
\TextTitle{Chant d'Israël après la chute de Babylone}
\VerseOne{}Car Yahweh aura pitié de Jacob, il choisira encore Israël, et il les rétablira dans leur terre ; les étrangers se joindront à eux et s'attacheront à la maison de Jacob.
\VS{2}Et les peuples les prendront, et les ramèneront à leur demeure, et la maison d'Israël les possédera en droit d'héritage sur la terre de Yahweh, comme serviteurs et comme servantes ; ils retiendront captifs ceux qui les avaient tenus captifs, et ils domineront sur leurs oppresseurs.
\VS{3}Et il arrivera qu'au jour que Yahweh fera cesser ton travail, ton tourment, et la dure servitude qui te fut imposée,
\VS{4}alors tu prononceras ce proverbe sur le roi de Babylone, et tu diras : Comment a-t-il fini le tyran ? Comment se repose celle qui était si avide de richesses ?
\VS{5}Yahweh a brisé le bâton des méchants, et la verge des dominateurs.
\VS{6}Celui qui frappait avec fureur les peuples de coups qu'on ne pouvait point détourner, qui dominait sur les nations avec colère, est poursuivi sans ménagement.
\VS{7}Toute la terre jouit du repos et de la paix ; on éclate en chants de triomphe à gorge déployée.
\VS{8}Même les cyprès et les cèdres du Liban se réjouissent de toi en disant : Depuis que tu es tombé, personne n'est monté pour nous abattre.
\TextTitle{Le roi de Babylone dépouillé de sa gloire}
\VS{9}Le scheol s'émeut jusque dans ses profondeurs, pour t'accueillir à ton arrivée ; il réveille à cause de toi les morts, et il fait lever de leurs sièges tous les principaux de la terre.
\VS{10}Tous prennent la parole pour te dire : Toi aussi, tu es sans force comme nous, tu es devenu semblable à nous !
\VS{11}Ta hauteur est descendue dans le scheol, avec le son de tes luths ; tu es couché sur une couche de vers, et la vermine est ta couverture.
\TextTitle{Orgueil, rébellion et chute de Satan}
\VS{12}Comment es-tu tombé du ciel, astre brillant, fils de l'aurore ? Toi qui foulais les nations, tu es abattu jusqu'à terre !
\VS{13}Tu disais en ton cœur : Je monterai aux cieux, je placerai mon trône au-dessus des étoiles de Dieu ; je m'assiérai sur la montagne de l'assemblée, du côté d'Aquilon\FTNT{Aquilon est un dieu des vents septentrionaux, froids et violents, dans la mythologie romaine.} ;
\VS{14}je monterai au dessus des hauts lieux des nuées, je serai semblable au Très-Haut.
\VS{15}Et cependant tu as été précipité dans le scheol, dans les profondeurs de la fosse\FTNT{Voir commentaire Ge. 1:1-2.}.
\VS{16}Ceux qui te voient fixent sur toi leurs regards, ils te considèrent attentivement, en disant : N'est-ce pas celui qui faisait trembler la terre, qui ébranlait les royaumes,
\VS{17}qui réduisait le monde habitable en désert, qui détruisait les villes, et ne relâchait pas ses prisonniers, ni ne les renvoyait chez eux ?
\TextTitle{Babylone anéantie}
\VS{18}Tous les rois des nations, oui, tous, reposent avec honneur, chacun dans sa maison.
\VS{19}Mais toi, tu as été jeté loin de ton sépulcre, comme un rejeton pourri, comme une dépouille de gens tués, transpercés avec l'épée, qu'on jette sous les pierres d'une fosse, comme un cadavre foulé aux pieds.
\VS{20}Tu ne seras point rangé comme eux dans le sépulcre, car tu as ravagé ta terre, tu as tué ton peuple. La race des méchants ne sera point renommée à toujours.
\VS{21}Préparez la tuerie pour ses enfants, à cause de l'iniquité de leurs pères ; afin qu'ils ne se relèvent point, et qu'ils n'héritent point la terre, et ne remplissent point de villes le dessus de la terre habitable.
\VS{22}Je m'élèverai contre eux, dit Yahweh des armées, et je retrancherai à Babylone le nom et le reste qu'elle a, ses descendants et sa postérité\FTNT{Ap. 14:8 ; Ap. 18:2.}, dit Yahweh.
\VS{23}J'en ferai l'habitation du butor et un marécage, et je la balayerai avec le balai de la destruction, dit Yahweh des armées.
\TextTitle{Jugement sur le roi d'Assyrie}
\VS{24}Yahweh des armées l'a juré, en disant : Certainement ce que j'ai décidé arrivera, ce que j'ai résolu s'accomplira.
\VS{25}Je briserai le roi d'Assyrie dans ma terre, je le foulerai aux pieds sur mes montagnes ; et son joug leur sera ôté, et son fardeau sera ôté de dessus leurs épaules.
\VS{26}C'est là le conseil arrêté contre toute la terre, c'est là la main étendue sur toutes les nations.
\TextTitle{Jugement sur le pays des Philistins}
\VS{27}Car Yahweh des armées l'a arrêté en son conseil : Qui l'empêchera ? Sa main est étendue : Qui la détournera\FTNT{Ec. 7:13.} ?
\VS{28}L'année de la mort du roi Achaz, cette prophétie fut prononcée : 
\VS{29}Ne te réjouis pas, toi pays des Philistins, de ce que la verge de celui qui te frappait est brisée ! Car de la racine du serpent sortira un vipère, et son fruit sera un serpent brûlant qui vole.
\VS{30}Alors les plus misérables seront repus, et les pauvres reposeront en assurance ; mais je ferai mourir de faim ta racine, et ce qui restera de toi sera tué.
\VS{31}Porte, hurle ! Ville, crie ! Tremble, pays tout entier des Philistins ! Car d'Aquilon, vient une fumée, et il ne restera pas un homme dans ses habitations.
\VS{32}Et que répondra-t-on aux envoyés de cette nation ? On répondra que Yahweh a fondé Sion, et que les affligés de son peuple y trouvent un refuge.
\Chap{15}
\TextTitle{Jugement sur Moab}
\VerseOne{}Prophétie sur Moab. La nuit même où elle est ravagée, Ar-Moab est détruite ! La nuit même où elle est saccagée, Kir-Moab est détruite !
\VS{2}Il monte à Bajith et à Dibon, dans les hauts lieux, pour pleurer ; Moab est en lamentations sur Nebo et sur Médeba : Toutes les têtes sont rasées et toutes les barbes sont coupées.
\VS{3}On sera couvert de sacs dans les rues ; chacun hurle, fondant en larmes sur ses toits et dans ses places\FTNT{Jé. 48:38.}.
\VS{4}Hesbon et Elealé poussent des cris, et l'on entend leur voix jusqu'à Jahats ; c'est pourquoi les guerriers de Moab se lamentent, ils ont l'effroi dans l'âme.
\VS{5}Mon cœur crie à cause de Moab, dont les fugitifs s'enfuient jusqu'à Tsoar, comme une génisse de trois ans ; car ils montent par la montée de Luchith avec des pleurs, et ils jettent des cris de détresse sur le chemin de Choronaïm.
\VS{6}Même les eaux de Nimrim ne sont que désolations, même le foin est déjà séché, l'herbe est consumée, et il n'y a point de verdure.
\VS{7}C'est pourquoi ils surveillent les richesses abondantes qu'ils ont acquises, afin que ce qu'ils ont réservé soit porté dans la vallée des saules.
\VS{8}Car les cris environnent les frontières de Moab, ses lamentations retentissent jusqu'à Eglaïm, ses lamentations retentissent jusqu'à Beer-Elim.
\VS{9}Même les eaux de Dimon sont pleines de sang ; car j'ajouterai un surcroît sur Dimon : Des lions contre les réchappés de Moab, et le reste du pays.
\Chap{16}
\TextTitle{Lamentation sur Moab}
\VerseOne{}Envoyez l'agneau au souverain du pays, envoyez-le du rocher du désert, à la montagne de la fille de Sion.
\VS{2}Car il arrivera que les filles de Moab seront au passage de l'Arnon, comme un oiseau volant ça et là, comme une nichée chassée de son nid.
\VS{3}Mets en avant le conseil, fais l'ordonnance, sers d'ombre comme une nuit au milieu de midi ; cache ceux qui ont été chassés, et ne trahis pas ceux qui sont errants.
\VS{4}Que ceux de mon peuple qui ont été chassés séjournent chez toi, ô Moab ! Sois pour eux un refuge contre le dévastateur ! Car celui qui use d'extorsion cessera, la dévastation finira, celui qui foule le pays sera consumé de dessus la terre.
\VS{5}Et le trône s'affermira par la clémence ; et sur ce trône sera assis en vérité, dans le tabernacle de David, un juge recherchant le droit, et se hâtant de faire justice\FTNT{Mi. 4:7 ; Da. 7:14 ; Lu. 1:33 ; Ap. 11:15}.
\VS{6}Nous avons entendu l'orgueil de Moab, le peuple extrêmement orgueilleux, sa fierté, son orgueil, son arrogance et ses vains discours.
\VS{7}C'est pourquoi Moab gémit sur Moab, chacun gémit ; vous soupirez pour les fondements de Kir-Haréseth, il n'y aura que des gens blessés à mort.
\VS{8}Car les campagnes de Hesbon et le vignoble de Sibma languissent ; les maîtres des nations ont foulé ses meilleurs ceps, qui s'étendaient jusqu'à Jaezer, qui couraient ça et là par le désert ; ses rameaux s'étendaient et passaient au-delà de la mer.
\VS{9}C'est pourquoi je pleure sur la vigne de Sibma, comme sur Jaezer ; je vous arrose de mes larmes, ô Hesbon et Elealé ! Car l'ennemi avec des cris s'est jeté sur tes fruits d'été et sur ta moisson.
\VS{10}Et la joie et l'allégresse se sont retirées du champ fertile ; on ne se réjouit plus et on ne s'égaye plus dans les vignes, le vendangeur ne foule plus dans les cuves, j'ai fait cesser la chanson de la vendange\FTNT{Jé. 48:31-34.}.
\VS{11}C'est pourquoi mes entrailles gémissent sur Moab, comme une harpe, et mon intérieur sur Kir-Harès.
\VS{12}Et on voit Moab qui se fatigue sur les hauts lieux ; il entre dans son sanctuaire pour prier mais il ne peut rien obtenir.
\VS{13}Telle est la parole que Yahweh a prononcée depuis longtemps sur Moab.
\VS{14}Et maintenant Yahweh a parlé, en disant : Dans trois ans, comme les années d'un mercenaire, la gloire de Moab sera avilie, avec toute cette grande multitude ; et le reste sera petit, ce sera peu de chose, ce ne sera rien de considérable.
\Chap{17}
\TextTitle{Prophétie sur la chute de Damas et de ses alliés}
\VerseOne{}Prophétie sur Damas. Voici, Damas est détruite pour ne plus être une ville, et elle ne sera qu'un monceau de ruines\FTNT{Jé. 49:23-27.}.
\VS{2}Les villes d'Aroër sont abandonnées, elles sont livrées aux troupeaux qui s'y reposent, et il n'y a personne qui les effraie.
\VS{3}Il n'y aura plus de forteresse en Ephraïm, ni de royaume à Damas et dans le reste de la Syrie ; ils seront comme la gloire des enfants d'Israël, dit Yahweh des armées.
\VS{4}Et il arrivera en ce jour-là que la gloire de Jacob sera affaiblie et la graisse de sa chair sera fondue.
\VS{5}Il en sera comme quand le moissonneur cueille les blés, et qu'il moissonne les épis avec son bras\FTNT{Joë. 3:13 ; Mt. 13:24-30.} ; comme quand on ramasse les épis dans la vallée de Rephaïm.
\VS{6}Mais il en restera quelques grappillages, comme quand on secoue l'olivier, et qu'il reste deux ou trois olives en haut de la cime, et qu'il y en a quatre ou cinq que l'olivier a produites dans ses branches fruitières, dit Yahweh, le Dieu d'Israël.
\VS{7}En ce jour-là, l'homme regardera vers celui qui l'a fait, et ses yeux se tourneront vers le Saint d'Israël.
\VS{8}Et il ne regardera plus vers les autels, qui sont l'ouvrage de ses mains, et il ne regardera plus ce que ses doigts ont fabriqué, ni les images d'Asherah, ni les statues du soleil.
\VS{9}En ce jour-là, ses villes fortes seront abandonnées à cause des enfants d'Israël, ils seront comme un bois taillis et des rameaux abandonnés, et ce sera un désert.
\VS{10}Parce que tu as oublié le Dieu de ton salut, et que tu ne t'es pas souvenue du rocher\FTNT{Voir commentaire Es. 8:13-14.} de ta force, à cause de cela tu as transplanté des plantes de plaisance, et tu as planté des ceps étrangers.
\VS{11}De jour tu as fais croître ce que tu as planté, et le matin tu as fait levé ta semence; mais la moisson a été enlevée au jour que l'on voulait en jouir, et il y a eu une douleur désespérée.
\VS{12}Malheur à la multitude de peuples nombreux, qui font un bruit comme le bruit des mers ; et à la tempête éclatante des nations, qui font du bruit comme une tempête éclatante d'eaux impétueuses !
\VS{13}Les nations font un bruit comme une tempête éclatante de grosses eaux, mais il les menace, et elles s'enfuient ; elles seront poursuivies comme la balle des montagnes chassée par le vent, et comme une boule poussée par un tourbillon.
\VS{14}Au temps du soir, voici une terreur soudaine ; mais avant le matin, ils ne sont plus ! C'est là le partage de ceux qui nous dépouillent, et le lot de ceux qui nous pillent.
\Chap{18}
\TextTitle{Jugement sur l'Ethiopie}
\VerseOne{}Malheur à la terre qui fait ombre avec des ailes, qui est au-delà des fleuves de l'Ethiopie ;
\VS{2}qui envoie par mer des messagers, dans des navires de jonc, voguant à la surface des eaux ! Allez, messagers rapides, vers la nation robuste et vigoureuse, vers le peuple redoutable, depuis là où il est et par delà ; nation puissante et qui écrase tout, et dont les fleuves ravagent son pays.
\VS{3}Vous tous, habitants du monde, et vous qui habitez dans le pays, quand la bannière sera élevée sur les montagnes, regardez ; et quand le shofar sonnera, écoutez !
\VS{4}Car ainsi m'a parlé Yahweh : Je me tiens tranquillement, et je regarde de ma demeure, par la chaleur de la lumière, et par la vapeur de la rosée, au temps de la chaude moisson.
\VS{5}Car avant la moisson, quand le bourgeon vient en sa perfection, et que la fleur devient un raisin qui mûrit, il coupe les sarments avec des serpes, il enlève les sarments, les ayant retranchés.
\VS{6}Ils seront tous ensemble abandonnés aux oiseaux de proie qui demeurent dans les montagnes, et aux bêtes de la terre ; les oiseaux de proie seront sur eux tout le long de l'été, et toutes les bêtes de la terre y passeront l'hiver.
\VS{7}En ce temps-là, un présent sera apporté à Yahweh des armées, par le peuple robuste et vigoureux, de la part, dis-je, du peuple terrible depuis là où il est et au-delà, nation puissante et qui écrase tout, et dont le pays est ravagé par ses fleuves ; il sera apporté dans la demeure du Nom de Yahweh des armées, sur la montagne de Sion.
\Chap{19}
\TextTitle{Chute de l'Egypte}
\VerseOne{}Prophétie sur l'Egypte. Voici, Yahweh est monté sur une nuée rapide, il entre en Egypte ; et les idoles d'Egypte s'enfuient de toutes parts devant sa face, et le cœur des Egyptiens se fond au milieu d'elle\FTNT{Jé. 43:12.}.
\VS{2}Et je ferai venir pêle-mêle l'Egyptien contre l'Egyptien, et chacun fera la guerre contre son frère, et chacun contre son ami, ville contre ville, et royaume contre royaume.
\VS{3}L'esprit de l'Egypte disparaîtra du milieu d'elle, et je dissiperai son conseil ; et ils consulteront les idoles et les enchanteurs, ceux qui évoquent les morts et ceux qui prédisent l'avenir.
\VS{4}Et je livrerai l'Egypte entre les mains d'un maître sévère ; et un roi cruel dominera sur eux, dit le Seigneur, Yahweh des armées.
\VS{5}Les eaux de la mer tariront, le fleuve séchera et tarira\FTNT{Jé. 51:36.}.
\VS{6}Et on fera détourner les fleuves ; les ruisseaux des digues s'abaisseront et sécheront ; les roseaux et les joncs seront coupés.
\VS{7}Les prairies qui sont près des ruisseaux, et sur l'embouchure du fleuve, tout ce qui aura été semé le long des ruisseaux, séchera, sera jeté au loin, et ne sera plus.
\VS{8}Et les pêcheurs gémiront, tous ceux qui jettent l'hameçon dans le fleuve mèneront deuil, et ceux qui étendent des filets sur les eaux languiront.
\VS{9}Ceux qui travaillent en fin lin et en fin crêpe, et ceux qui tissent les filets seront confus.
\VS{10}Les fondements du pays seront rompus, et tous ceux qui font des écluses de viviers auront l'âme attristée.
\VS{11}Certes les chefs de Tsoan ne sont que des insensés, les sages d'entre les conseillers de Pharaon forment un conseil stupide. Comment osez-vous dire à Pharaon : Je suis fils des sages, fils des anciens rois ?
\VS{12}Où sont-ils maintenant? Où sont, dis-je, tes sages ? Qu'ils t'annoncent, je te prie, s'ils le savent, ce que Yahweh des armées a décrété contre l'Egypte.
\VS{13}Les chefs de Tsoan sont devenus insensés, les chefs de Noph se sont trompés, les chefs des tribus font égarer l'Egypte.
\VS{14}Yahweh a versé au milieu d'elle un esprit de vertige\FTNT{1 R. 22:18-22.}, pour qu'ils fassent chanceler les Egyptiens dans toutes leurs actions, comme un homme ivre se vautre dans son vomissement.
\VS{15}Et l'Egypte sera hors d'état de faire ce que font la tête et la queue, la branche de palmier et le roseau.
\TextTitle{L'Egypte et l'Assyrie dans le royaume du Messie}
\VS{16}En ce jour-là, l'Egypte sera comme des femmes : Elle sera étonnée et épouvantée à cause de la main de Yahweh des armées, quand il élèvera la main contre elle.
\VS{17}Et la terre de Juda sera pour l'Egypte un objet d'effroi ; quiconque fera mention d'elle, en sera épouvanté en lui-même, à cause du conseil décrété contre elle par Yahweh des armées.
\VS{18}En ce jour-là, il y aura cinq villes au pays d'Egypte, qui parleront la langue de Canaan, et qui jureront par Yahweh des armées ; l'une sera appelée ville de la destruction.
\VS{19}En ce jour-là, il y aura un autel à Yahweh au milieu du pays d'Egypte, et un monument dressé à Yahweh sur la frontière.
\VS{20}Et ce sera un signe et un témoignage pour Yahweh des armées dans le pays d'Egypte ; car ils crieront à Yahweh à cause des oppresseurs, et il leur enverra un sauveur, quelqu'un de grand, et il les délivrera\FTNT{Es. 43:11.}.
\VS{21}Et Yahweh se fera connaître aux Egyptiens, et les Egyptiens connaîtront Yahweh en ce jour-là ; ils le serviront, ils offriront des sacrifices et des offrandes, et ils feront des vœux à Yahweh et les accompliront.
\VS{22}Ainsi Yahweh frappera les Egyptiens, il les frappera, mais il les guérira ; et ils retourneront à Yahweh, qui les exaucera et les guérira.
\VS{23}En ce jour-là, il y aura un chemin battu de l'Egypte en Assyrie ; et l'Assyrie viendra en Egypte, et l'Egypte en Assyrie, et l'Egypte servira avec l'Assyrie.
\VS{24}En ce même temps, Israël sera, lui troisième, uni à l'Egypte et à l'Assyrie, et la bénédiction sera au milieu de la terre.
\VS{25}Yahweh des armées les bénira, en disant : Bénis soit l'Egypte mon peuple, et l'Assyrie œuvre de mes mains, et Israël mon héritage !
\Chap{20}
\TextTitle{Conquête de l'Egypte et de l'Ethiopie}
\VerseOne{}L'année où Tharthan, envoyé par Sargon, roi d'Assyrie, vint et combattit contre Asdod, et la prit.
\VS{2}En ce temps-là, Yahweh parla par Esaïe, fils d'Amots, et lui dit : Va, délie le sac de dessus tes reins et ôte tes souliers de tes pieds. Il fit ainsi, marchant nu et déchaussé.
\VS{3}Puis Yahweh dit : De même que mon serviteur Esaïe marche nu et déchaussé, ce qui sera dans trois ans un signe et un prodige contre l'Egypte et contre l'Ethiopie,
\VS{4}de même le roi d'Assyrie emmènera de l'Egypte et de l'Ethiopie prisonniers et captifs les jeunes et les vieux, nus et déchaussés, ayant les hanches découvertes, ce qui sera l'opprobre de l'Egypte\FTNT{2 S. 10:4 ; Es. 3:17 ; Jé. 13:22-26.}.
\VS{5}Ils seront effrayés, et ils seront honteux à cause de l'Ethiopie à qui ils s'attendaient, et à cause de l'Egypte dont ils se glorifiaient.
\VS{6}Et les habitants de cette côte diront en ce jour-là : Voilà ce qu'est devenu le peuple à qui nous nous attendions, celui vers qui nous courions chercher du secours, afin d'être délivrés du roi d'Assyrie ! Comment pourrons-nous échapper ?
\Chap{21}
\TextTitle{Annonce de la conquête de Babylone}
\VerseOne{}Prophétie sur le désert de la mer. Il vient du désert, de la terre redoutable, comme des tourbillons qui s'élèvent au pays du midi pour traverser.
\VS{2}Une vision terrible m'a été révélée. Le traître demeure traître, celui qui saccage, saccage toujours. Monte, Elam ! Assiège, Médie ! Je fais cesser tous les soupirs.
\VS{3}C'est pourquoi mes reins sont remplis de douleur ; les angoisses me saisissent comme les douleurs de celle qui enfante ; je suis tourmenté à cause de ce que j'ai entendu, et j'ai été tout troublé à cause de ce que j'ai vu.
\VS{4}Mon cœur est agité de toutes parts, la terreur s'empare de moi ; la nuit de mes plaisirs devient une nuit de crainte.
\VS{5}Qu'on dresse la table, que la sentinelle veille, qu'on mange, qu'on boive! Levez-vous, chefs ! Oignez le bouclier !
\VS{6}Car ainsi m'a parlé le Seigneur : Va, place la sentinelle, et qu'elle rapporte ce qu'elle verra\FTNT{Ez. 33:1-19.}.
\VS{7}Et elle vit un char, un couple de cavaliers, un char tiré par des ânes, un char tiré par des chameaux ; et elle les considéra fort attentivement.
\VS{8}Et elle s'écria : C'est un lion ! Seigneur, je me tiens en sentinelle toute la journée et je suis à mon poste toutes les nuits ;
\VS{9}et voici venir le char d'un homme et un couple de cavaliers ! Alors elle parla et dit : Elle est tombée, elle est tombée, Babylone\FTNT{Prophétie sur la chute de Babylone. Voir Jé. 50 et 51 ; Ap.18.}, et toutes les images taillées de ses dieux sont brisées par terre.
\VS{10}C'est ce que j'ai foulé, et le grain que j'ai battu dans mon aire. Je vous ai annoncé ce que j'ai entendu de Yahweh des armées, du Dieu d'Israël.
\VS{11}Prophétie sur Duma. On me crie de Séir : Ô sentinelle ! Qu'en est-il de la nuit ? Ô sentinelle ! Qu'en est-il de la nuit ?
\VS{12}La sentinelle répond : Le matin vient et la nuit aussi. Si vous demandez, demandez. Retournez, venez.
\TextTitle{Jugement sur l'Arabie}
\VS{13}Prophétie contre l'Arabie. Vous passerez pêle-mêle la nuit dans la forêt, caravanes de Dedan !
\VS{14}Les habitants du pays de Théma portent de l'eau à ceux qui ont soif ; ils  viennent au-devant du fugitif avec du pain pour lui.
\VS{15}Car ils fuient devant les épées, devant l'épée dégainée, devant l'arc tendu, devant le fort de la bataille.
\VS{16}Car ainsi m'a parlé le Seigneur : Encore une année, comme les années d'un mercenaire, et toute la gloire de Kédar prendra fin.
\VS{17}Et le reste du nombre des forts archers des fils de Kédar sera diminué, car Yahweh, le Dieu d'Israël, a parlé.
\Chap{22}
\TextTitle{Malédiction sur la vallée des visions, Jérusalem}
\VerseOne{}Prophétie sur la vallée des visions. Qu'as-tu maintenant, que tu sois toute montée sur les toits ?
\VS{2}Toi ville bruyante, pleine de tumulte, ville joyeuse ! Tes blessés à morts ne seront pas blessés à mort par l'épée, et ils ne mourront pas par la guerre.
\VS{3}Tous tes chefs fuient ensemble, ils sont liés par les archers ; tous ceux des tiens qui sont trouvés sont liés ensemble tandis qu'ils s'enfuient au loin.
\VS{4}C'est pourquoi je dis : Détournez de moi vos regards, que je pleure amèrement. Ne vous empressez pas pour me consoler du désastre de la fille de mon peuple.
\VS{5}Car c'est le jour de trouble, d'oppression et de confusion\FTNT{Lam. 1:5 ; Lam. 2:2.}, envoyé par le Seigneur, Yahweh des armées, dans la vallée des visions. Il démolit la muraille et les cris retentissent jusqu'à la montagne.
\VS{6}Même Elam prend son carquois, il y a des hommes montés sur des chars et des cavaliers ; Kir découvre le bouclier.
\VS{7}Et tes plus belles vallées sont remplies de chars, et les cavaliers se rangent tous en bataille à tes portes.
\VS{8}Et on découvre ce qui couvrait Juda, et en ce jour là tu regardes vers les armes de la maison de la forêt.
\VS{9}Vous voyez que les brèches de la cité de David sont nombreuses ; et vous assemblez les eaux de l'étang inférieur.
\VS{10}Vous faites le dénombrement des maisons de Jérusalem, et vous démolissez les maisons pour fortifier la muraille.
\VS{11}Et vous faites aussi un réservoir d'eau entre les deux murailles, pour les eaux de l'ancien étang. Mais vous ne regardez pas à celui qui a fait ces choses, qui les a formées il y a longtemps.
\VS{12}Le Seigneur, Yahweh des armées, vous appelle ce jour-là aux pleurs et au deuil, à vous raser la tête, et à ceindre le sac\FTNT{Ez. 7:18 ; Joë 1:13}.
\VS{13}Et voici il y a de la joie et de l'allégresse ! On égorge des bœufs et l'on tue des moutons, on mange la viande et l'on boit du vin ; puis on dit : Mangeons et buvons, car demain nous mourrons\FTNT{Es. 56:12 ; 1 Co. 15:32.} !
\VS{14}Or il m'a été révélé à l'oreille, par Yahweh des armées : Sûrement cette iniquité ne vous sera pas pardonnée jusqu'à ce que vous mouriez, a dit le Seigneur, Yahweh des armées.
\TextTitle{Eliakim succède à Schebna}
\VS{15}Ainsi parle le Seigneur, Yahweh des armées : Va, entre chez ce trésorier, chez Schebna, gouverneur du palais et dis-lui :
\VS{16}Qu'as-tu à faire ici, et qu'as-tu ici qui t'appartienne, que tu te tailles ici un sépulcre ? Il taille un sépulcre en hauteur, il se taille une demeure dans le rocher.
\VS{17}Voici, ô homme ! Yahweh te chassera au loin d'un bras vigoureux ; il t'enveloppera entièrement.
\VS{18}Il te fera rouler fort vite, comme une balle sur une terre large et spacieuse ; là tu mourras, là seront les chars de ta gloire, ô toi qui es la honte de la maison de ton Seigneur !
\VS{19}Je te jetterai hors de ton rang, et on t'arrachera de ton service.
\VS{20}Et il arrivera en ce jour-là que j'appellerai mon serviteur Eliakim, fils de Hilkija.
\VS{21}Je le revêtirai de ta tunique, je le ceindrai de ta ceinture, et je remettrai ton autorité entre ses mains, il sera un père pour les habitants de Jérusalem et pour la maison de Juda.
\VS{22}Et je mettrai la clef de la maison de David sur son épaule ; et il ouvrira, et il n'y aura personne qui ferme ; et il fermera, et il n'y aura personne qui ouvre\FTNT{La clé de David est le symbole de l'autorité du Messie (Es. 9:5 ; Mt. 28:18 ; Ap. 3:7-8)}.
\VS{23}Je l'enfoncerai comme un clou dans un lieu sûr, et il sera un trône de gloire pour la maison de son père.
\VS{24}Et on y pendra toute la gloire de la maison de son père, de ses parents et de celles qui lui appartiennent ; tous les ustensiles des plus petites choses,  des bassins comme des vases. 
\VS{25}En ce jour-là, dit Yahweh des armées, le clou enfoncé dans un lieu sûr sera ôté ; et étant retranché il tombera, et le fardeau qui était sur lui sera retranché, car Yahweh a parlé.
\Chap{23}
\TextTitle{Effondrement de Tyr}
\VerseOne{}Prophétie sur Tyr. Hurlez, navires de Tarsis ! Car elle est détruite, il n'y a plus de maisons, on n'y entre plus ! Ceci leur a été révélé du pays de Kittim.
\VS{2}Vous qui habitez dans l'île, taisez-vous ! Toi qui étais remplie de marchands de Sidon, et de ceux qui traversaient la mer !
\VS{3}A travers les grandes eaux, les grains de Shichor, la moisson du Nil était pour elle son revenu ; elle était le marché des nations\FTNT{Ez. 27.}.
\VS{4}Sois honteuse, ô Sidon ! Car la mer, la forteresse de la mer, a parlé en disant : Je n'ai point eu de douleurs, je n'ai point enfanté, je n'ai point nourri de jeunes gens ni élevé aucune vierge.
\VS{5}Selon la nouvelle qui a été touchant l'Egypte, ainsi sera-t-on en travail quand on entendra la nouvelle touchant Tyr.
\VS{6}Passez à Tarsis, hurlez, vous qui habitez dans l'île !
\VS{7}N'est-ce pas ici votre ville joyeuse ? Elle avait une origine antique et ses propres pieds la mènent séjourner dans un pays étranger.
\VS{8}Qui a pris ce conseil contre Tyr, celle qui couronnait les siens, dont les marchands étaient des princes, et dont les trafiquants étaient les plus honorables de la terre\FTNT{Ap. 18:9-18.} ?
\VS{9}Yahweh des armées a pris ce conseil, pour flétrir l'orgueil de toute la noblesse, et pour avilir tous les honorables de la terre.
\VS{10}Traverse ton pays, comme une rivière, ô fille de Tarsis ! Il n'y a plus de ceinture.
\VS{11}Il a étendu sa main sur la mer, il a fait trembler les royaumes ; Yahweh a ordonné la destruction des forteresses de Canaan.
\VS{12}Il a dit : Tu ne te livreras plus à la joie, vierge opprimée, fille de Sidon ! Lève-toi, passe au pays de Kittim ! Même là, il n'y aura pas de repos pour toi.
\VS{13}Voilà le pays des Chaldéens ; ce peuple-là n'était pas autrefois ; Assur\FTNT{Assur : Le second fils de Sem (Ge. 10:22). L'ancêtre des Assyriens.} l'a fondé pour les gens du désert ; on a dressé ses forteresses, on a élevé ses palais, et il l'a mis en ruines.
\VS{14}Hurlez, navires de Tarsis ! Car votre force est détruite !
\VS{15}Et il arrivera en ce jour-là que Tyr tombera dans l'oubli durant soixante-dix ans, selon les jours d'un roi. Mais au bout de soixante-dix ans\FTNT{Jé. 25 : 11-12.}, on chantera une chanson à Tyr comme à une femme prostituée :
\VS{16}Prends la harpe, fais le tour de la ville, ô prostituée qu'on oublie ! Sonne avec force, chante et rechante, afin qu'on se ressouvienne de toi!
\VS{17}Et il arrivera au bout de soixante-dix ans que Yahweh visitera Tyr, mais elle retournera au salaire de sa prostitution, et elle se prostituera avec tous les royaumes de la terre, sur le dessus de la terre.
\VS{18}Mais son trafic et son salaire seront sanctifiés à Yahweh ; il n'en sera rien réservé, ni serré ; car son trafic sera pour ceux qui habitent dans la présence de Yahweh, pour en manger à satiété, et pour avoir des vêtements durables.
\Chap{24}
\TextTitle{Désastre après l'invasion babylonienne}
\VerseOne{}Voici, Yahweh s'en va rendre le pays vide et l'épuiser, il en renverse le dessus, et disperse ses habitants\FTNT{Ge. 11:1-8.}.
\VS{2}Et il en est du sacrificateur comme du peuple, du maître comme de son serviteur, de la dame comme de sa servante, du vendeur comme de l'acheteur, de celui qui prête comme de celui qui emprunte, du créancier comme du débiteur.
\VS{3}Le pays est entièrement vidé et entièrement pillé, car Yahweh a prononcé cet arrêt.
\VS{4}La terre mène le deuil, elle est déchue ; le pays habité est devenu languissant, il est déchu ; les plus distingués du peuple de la terre sont languissants.
\VS{5}Le pays était profané par ses habitants qui marchent sur lui ; car ils ont transgressé les lois, ils ont changé les ordonnances et ont enfreint l'alliance éternelle\FTNT{Da. 7:25.}.
\VS{6}C'est pourquoi la malédiction dévore le pays, et ses habitants portent la peine de leurs crimes ; c'est pourquoi les habitants du pays sont brûlés et il n'en reste qu'un petit nombre.
\VS{7}Le vin excellent pleure, la vigne languit, et tous ceux qui avaient le cœur joyeux soupirent.
\VS{8}La joie des tambours a cessé ; le bruit de ceux qui s'égayent a pris fin, la joie de la harpe a cessé.
\VS{9}On ne boit plus de vin en chantant ; les boissons fortes sont amères à ceux qui les boivent.
\VS{10}La ville confuse est en ruines ; toutes les maisons sont fermées, on n'y entre plus.
\VS{11}On crie dans les rues parce que le vin manque ; toute la joie est tournée en obscurité, l'allégresse du pays s'en est allée.
\VS{12}La désolation est restée dans la ville et la porte est frappée d'une ruine éclatante.
\VS{13}Car il arrivera au milieu de la terre et parmi les peuples, comme quand on secoue l'olivier, et comme quand on grappille après la vendange.
\TextTitle{Un reste de rescapés célèbre Yahweh}
\VS{14}Ils élèvent leur voix, ils se réjouissent avec chant de triomphe ; et s'égayent du côté de la mer, ils célèbrent la majesté de Yahweh.
\VS{15}C'est pourquoi glorifiez Yahweh dans les vallées, le Nom de Yahweh, le Dieu d'Israël, dans les îles de la mer !
\VS{16}De l'extrémité de la terre, nous entendons des cantiques à la gloire du Juste ; mais moi je dis : Maigreur sur moi ! Maigreur sur moi ! Malheur à moi ! Les perfides ont agi perfidement ; et ils ont imité la mauvaise foi des perfides.
\TextTitle{Manifestation des jugements de Yahweh}
\VS{17}La frayeur, la fosse, et le piège sont sur toi, habitant du pays !
\VS{18}Et il arrivera que celui qui fuit à cause du bruit de la frayeur tombe dans la fosse, et celui qui remonte hors de la fosse se prend au filet ; car les écluses d'en haut s'ouvrent et les fondements de la terre tremblent.
\VS{19}La terre est entièrement brisée, la terre s'écrase entièrement, la terre se remue de sa place.
\VS{20}La terre chancelle entièrement comme un homme ivre, elle est transportée comme une cabane ; son péché pèse sur elle, elle tombe et ne se relève plus.
\VS{21}Et il arrivera en ce jour là, que Yahweh punira dans le lieu élevé l'armée d'en haut, et sur la terre les rois de la terre.
\VS{22}Ils seront assemblés en troupes comme des prisonniers dans une fosse, et ils seront enfermés dans une prison, et après plusieurs jours ils seront visités.
\VS{23}La lune rougira et le soleil sera honteux quand Yahweh des armées régnera sur la montagne de Sion et à Jérusalem, resplendissant de gloire en présence de ses anciens\FTNT{Mt. 24:29-30 ; 2 Pi. 3:10-12 ; Ap. 6:12.}.
\Chap{25}
\TextTitle{Le royaume de Yahweh}
\VerseOne{}Ô Yahweh, tu es mon Dieu ; je t'exalterai, je célébrerai ton nom, car tu as fait des choses merveilleuses ; tes conseils conçus d'avance sont fidèlement accomplis.
\VS{2}Car tu as fait de la ville un monceau de pierres, et de la cité forte une ruine ; le palais des étrangers qui était dans la ville ne sera jamais rebâti.
\VS{3}C'est pourquoi le peuple fort te glorifie, la ville des nations redoutables te révère.
\VS{4}Parce que tu as été la force du faible, la force du misérable dans sa détresse, le refuge contre la tempête, l'ombrage contre la chaleur ; car le souffle des tyrans est comme la tempête qui abat une muraille.
\VS{5}Tu as rabaissé la tempête éclatante des étrangers ; comme la chaleur, dis-je, dans un pays sec, comme la chaleur par l'ombre d'une nuée, le branchage des tyrans sera abattu.
\VS{6}Et Yahweh des armées prépare à tous les peuples sur cette montagne un banquet de choses grasses, un banquet de vins vieux, un banquet, dis-je, de choses grasses et moelleuses, et de vins vieux bien purifiés\FTNT{Mt. 22:2 ; Ap. 3:20.}.
\VS{7}Et il détruit sur cette montagne l'enveloppe redoublée qu'on voit sur tous les peuples, et la couverture qui est étendue sur toutes les nations.
\VS{8}Il détruit la mort par sa victoire\FTNT{1 Co. 15:54.} ; et le Seigneur Yahweh essuie les larmes de tous les visages\FTNT{Ap. 7:17.}, et il ôte l'opprobre de son peuple de toute la terre\FTNT{Lu. 1:25.}, car Yahweh a parlé.
\VS{9}Et l'on dira en ce jour-là : Voici, c'est ici notre Dieu, auquel nous nous attendons, aussi c'est lui qui nous sauve ; c'est ici Yahweh, auquel nous nous attendons ; soyons dans l'allégresse, et réjouissons-nous de son salut !
\VS{10}Car la main de Yahweh repose sur cette montagne ; mais Moab est foulé aux pieds sous lui, comme on foule la paille pour en faire du fumier.
\VS{11}Et il étend ses mains au milieu d'eux, comme le nageur étend ses mains pour nager ; et Yahweh abat son orgueil, ainsi que l'artifice de ses mains.
\VS{12}Il abaisse la forteresse des plus hautes retraites de tes murailles, il les renverse, il les fait crouler à terre, et les réduit en poussière.
\Chap{26}
\TextTitle{Adoration à Yahweh}
\VerseOne{}En ce jour-là, ce cantique sera chanté dans le pays de Juda : Nous avons une ville forte ; le salut\FTNT{Le mot salut vient du mot « Yeshuw'ah ». Cette même racine a donné le prénom Jésus qui signifie Yahweh sauve. Jésus est notre muraille et notre rempart. Dans Ex. 15:2, Moïse identifie Yahweh à « Yeshuw'ah » c'est-à-dire à Jésus. Dans 1 Ch. 16:23, il est dit que « Yeshuw'ah » doit être annoncé tous les jours. Dans Ps. 62:2, il est présenté comme Dieu et le Rocher. Dans Es. 12:2, il est le Dieu qui sauve. Jacob et David avaient mis en lui leur espoir (Ge. 49:18 ; Ps. 119:166). Dans Es. 49:6, il est dit que le salut (« Yeshuw'ah » ou Jésus) doit être annoncé aux extrémités de la terre, et cela est répété et confirmé en Mt. 28:18-20. Es. 56:1 nous apprend que celui qui vient s'appelle « Yeshuw'ah ». Es. 59:17 le présente comme notre casque, ce qui fait écho au casque du salut en Ep. 6:17. Les murs de la Nouvelle Jérusalem portent son Nom (Es. 60:18). Ha. 3:8 nous dit que « Yeshuw'ah » montera sur ses chevaux, corroborant le récit de son retour en gloire dans Ap. 19:11-20. « Yeshuw'ah » est notre flambeau selon Es. 62:1 et Ap. 21:23.} y sera mis pour muraille et pour rempart.
\VS{2}Ouvrez les portes, et la nation juste, celle qui garde la fidélité, y entrera.
\VS{3}Tu gardes dans une paix parfaite celui dont l'esprit s'appuie sur toi, parce qu'il se confie en toi\FTNT{Es. 57:19 ; Ph. 4:6-7.}.
\VS{4}Confiez-vous en Yahweh à perpétuité, car le Rocher \FTNT{Voir commentaire en Es. 8:13-14. } des siècles est en Yahweh Dieu.
\VS{5}Car il a abaissé ceux qui habitaient aux lieux haut élevés, il a renversé la ville de haute retraite, il l'a renversée jusqu'à terre, il l'a réduite jusqu'à la poussière.
\VS{6}Le pied marchera dessus ; les pieds, dis-je, des pauvres, les plantes des misérables marcheront dessus.
\VS{7}Le sentier du juste est la droiture ; toi qui est juste, tu dresses au niveau le chemin du juste.
\VS{8}Aussi t'avons-nous attendu, ô Yahweh, dans le sentier de tes jugements ! Ton Nom et ton souvenir sont le désir de notre âme.
\VS{9}De nuit, je te désire de mon âme, et dès le point du jour, mon esprit qui est en moi te recherche ; car lorsque tes jugements s'exercent sur la terre, les habitants du monde apprennent la justice.
\VS{10}Est-il fait grâce au méchant ? Il n'en apprend point la justice, mais il agit méchamment sur la terre de la droiture, et il ne regarde pas à la majesté de Yahweh.
\VS{11}Yahweh, quand ta main est élevée, ils ne le voient pas. Mais ils verront et seront honteux à cause de leur jalousie pour ton peuple ; et le feu dont tu punis tes ennemis les dévorera.
\VS{12}Yahweh, tu ordonnes la paix pour nous, car aussi tout ce que nous faisons, c'est toi qui l'accomplis en nous.
\VS{13}Yahweh, notre Dieu, d'autres seigneurs que toi nous ont maîtrisés, mais c'est par toi seul que nous pouvons faire mention de ton Nom.
\VS{14}Ils sont morts, ils ne revivront plus, ils sont trépassés, ils ne se relèveront pas ; car tu les as châtiés et exterminés, et tu as fait périr toute mémoire d'eux\FTNT{Ec. 9:5.}.
\VS{15}Yahweh, tu avais accru la nation, tu avais accru la nation, tu as été glorifié, mais tu les as jetés loin dans toutes les extrémités de la terre.
\TextTitle{Un reste épargné de la colère de Yahweh}
\VS{16}Yahweh, étant en détresse ils se sont rendus auprès de toi ; ils se sont répandus en prières quand ton châtiment a été sur eux.
\VS{17}Comme celle qui est enceinte est en travail, et crie dans ses tranchées, lorsqu'elle est prête d'enfanter, ainsi avons-nous été devant ta face ô Yahweh !
\VS{18}Nous avons conçu et nous avons éprouvé des douleurs et nous avons comme enfanté du vent. Nous ne saurions en aucune manière délivrer le pays et les habitants de la terre habitable ne tomberaient point par notre force.
\VS{19}Tes morts vivront ! Même mon corps mort vivra ! Ils se relèveront. Réveillez-vous et réjouissez-vous avec des chants de triomphe, vous, habitants de la poussière ; car ta rosée est comme la rosée des herbes, et la terre jettera dehors les morts\FTNT{Os. 13:14 ; Da. 12:2 ; 1 Co. 15:52.}.
\VS{20}Va, mon peuple, entre dans tes cabinets et ferme ta porte derrière toi\FTNT{Mt. 6:6.} ; cache-toi pour un petit moment, jusqu'à ce que l'indignation soit passée.
\VS{21}Car voici, Yahweh s'en va sortir de son lieu pour visiter l'iniquité des habitants de la terre, commise contre lui ; alors la terre découvrira le sang qu'elle aura reçu et ne couvrira plus ceux qu'on a mis à mort.
\Chap{27}
\TextTitle{Israël rétabli}
\VerseOne{}En ce jour-là, Yahweh frappera de sa dure, grande et forte épée le Léviathan\FTNT{Ps. 104:26 ; Job. 40:20}, le serpent fuyard, le Léviathan, dis-je, le serpent tortueux, et il tuera le monstre qui est dans la mer.
\VS{2}En ce jour-là, chantez sur la vigne désirable\FTNT{Esaïe annonce ici le rétablissement d'Israël. Voir également Ro. 11:1-24.}.
\VS{3}C'est moi Yahweh qui la garde, je l'arrose à chaque instant, je la garde nuit et jour, afin que personne ne lui fasse du mal.
\VS{4}Il n'y a point de fureur en moi ; qu'on me donne des ronces, des épines pour les combattre ! Je marcherai contre elles, je les brulerai toutes ensemble.
\VS{5}Ou bien, qu'il saisisse ma force, qu'il fasse la paix avec moi, qu'il fasse la paix avec moi.
\VS{6}Il fera que Jacob prendra racine, Israël  fleurira, et s'épanouira ; et il remplira de fruits le dessus de la terre habitable.
\VS{7}L'a-t-il frappé comme il a frappé celui qui le frappaient ? L'a-t-il tué comme il a tué ceux qui le tuaient ?
\VS{8}Tu as plaidé avec elle modérément, quand tu l'as renvoyée ; en l'emportant par le vent rude au jour du vent d'orient.
\VS{9}C'est pourquoi l'expiation de l'iniquité de Jacob sera faite par ce moyen, et ceci en sera le fruit entier, que son péché sera ôté ; quand il aura transformé toutes les pierres des autels comme des pierres de chaux réduites en poussière ; et lorsque les idoles d'Asherah et les statues consacrées au soleil ne seront plus debout.
\VS{10}Car la ville fortifiée est désolée, la demeure agréable est abandonnée et délaissée comme le désert. Là pâture le veau, il y gîte et broute les branches.
\VS{11}Quand son branchage est sec, il est brisé ; et les femmes y venant en allument un feu. Car c'est un peuple sans intelligence\FTNT{De. 32:28 ; Es. 1:3.},  c'est pourquoi celui qui l'a fait n'a point eu pitié de lui, et celui qui l'a formé ne lui a point fait grâce.
\VS{12}Il arrivera en ce jour-là que Yahweh secouera, depuis le cours du fleuve jusqu'au torrent d'Egypte ; mais vous serez glanés un à un, ô enfants d'Israël.
\VS{13}Et il arrivera en ce jour-là qu'on sonnera du grand shofar, et ceux qui étaient exilés au pays d'Assyrie, et ceux qui avaient été chassés au pays d'Egypte, reviendront et se prosterneront devant Yahweh, sur la sainte montagne, à Jérusalem.
\Chap{28}
\TextTitle{Malheur et captivité d'Ephraïm en Assyrie}
\VerseOne{}Malheur à la couronne de fierté des ivrognes d'Ephraïm, la noblesse de la gloire qui n'est qu'une fleur qui tombe ; ceux qui sont sur le sommet de la grasse vallée sont étourdis de vin !
\VS{2}Voici, le Seigneur a dans sa main un homme fort et puissant, semblable à une tempête de grêle, à un tourbillon destructeur, à une tempête de grosses eaux débordées ; il la fera tomber à terre avec la main.
\VS{3}Elle seront foulées aux pieds, la couronne de fierté et les ivrognes d'Ephraïm.
\VS{4}Et la noblesse de sa gloire qui est sur le sommet de la fertile vallée, ne sera qu'une fleur qui tombe ; ils seront comme les fruits précoces avant l'été, aussitôt que celui qui regarde les voit, à peine ils sont dans sa main, il les dévore.  
\VS{5}En ce jour-là, Yahweh des armées sera une couronne de noblesse et un diadème de gloire pour le reste de son peuple ;
\VS{6}et un esprit de jugement pour celui qui sera assis au siège de jugement, et une force à ceux qui dans le combat repousseront l'ennemi jusqu'à la porte.
\VS{7}Mais eux aussi, s'oublient dans le vin, et se fourvoient dans les boissons fortes ; le sacrificateur et le prophète s'oublient dans les boissons fortes ; ils sont engloutis par le vin, ils se fourvoient à cause des boissons fortes ; ils s'oublient dans la vision, ils vacillent dans le jugement.
\VS{8}Car toutes leurs tables sont couvertes de vomissements et d'ordures ; aussi il n'y a plus de place !
\VS{9}A qui enseigne-t-on la connaissance ? A qui fait-on comprendre l'enseignement ? Est-ce à ceux qu'on vient de sevrer et de retirer de la mamelle ?
\VS{10}Car il faut leur donner précepte après précepte, précepte après précepte, règle après règle, règle après règle, un peu ici, un peu là\FTNT{Hé. 5:12.}.
\VS{11}C'est pourquoi, il parlera à ce peuple par des lèvres qui balbutient et une langue étrangère.
\VS{12}Il leur disait : Voici le repos, donnez du repos à celui qui est fatigué ;  voici le soulagement ! Mais ils n'ont point voulu écouter.
\VS{13}Ainsi la parole de Yahweh sera pour eux précepte après précepte, précepte après précepte, règle après règle, règle après règle, un peu ici, un peu là ; afin qu'ils aillent et tombent à la renverse, et qu'ils soient brisés, et afin qu'ils tombent dans le piège et qu'ils soient pris.
\TextTitle{Yahweh rompt le pacte du scheol par une pierre angulaire}
\VS{14}C'est pourquoi écoutez la parole de Yahweh, vous hommes moqueurs, qui dominez sur ce peuple qui est à Jérusalem !
\VS{15}Car vous dites : Nous avons fait un pacte avec la mort, et nous avons un accord avec le scheol ; quand le fléau débordé passera, il ne viendra pas sur nous, car nous avons le mensonge pour refuge et nous nous sommes cachés sous la fausseté.
\VS{16}C'est pourquoi ainsi parle le Seigneur Yahweh : Voici, je mettrai pour fondement en Sion une pierre\FTNT{Voir commentaire en Es. 8:13-16.}, une pierre éprouvée, la pierre angulaire la plus précieuse, pour être un fondement solide ; celui qui croira ne se hâtera point.
\VS{17}Et je mettrai le jugement à l'équerre, et la justice au niveau ; et la grêle détruira le refuge du mensonge, et les eaux inonderont le lieu où l'on se retirait.
\VS{18}Et votre pacte avec la mort sera détruit, votre accord avec le scheol ne tiendra pas ; quand le fléau débordé passera, vous en serez foulés.
\VS{19}Dès qu'il passera, il vous emportera. Or il passera tous les matins, le jour et la nuit ; et dès qu'on en entendra le bruit, il n'y aura que terreur.
\VS{20}Car le lit sera trop court, et on ne pourra pas s'y étendre, et la couverture trop étroite pour s'en envelopper.
\VS{21}Car Yahweh se lèvera comme à la montagne de Peratsim, et il sera ému comme dans la vallée de Gabaon, pour faire son œuvre, son œuvre extraordinaire, et pour faire son travail, son travail non accoutumé.
\VS{22}Maintenant donc, ne vous moquez plus, de peur que vos liens ne soient renforcés, car j'ai entendu de par le Seigneur, Yahweh des armées, que la destruction est déterminée sur tout le pays. 
\VS{23}Prêtez l'oreille, et écoutez ma voix ; soyez attentifs, et écoutez mon discours !
\VS{24}Celui qui laboure pour semer, laboure-t-il tous les jours ? Ne casse-t-il pas et ne rompt-il pas les mottes de sa terre ? 
\VS{25}Quand il en aura aplani la surface, ne sèmera-t-il pas la vesce\FTNT{La vesce est un genre de plante herbacées de la famille des légumineuse} ; ne répandra-t-il pas le cumin, ne mettra-t-il pas le froment au meilleur endroit, et l'orge en son lieu assigné, et l'épeautre\FTNT{L'épeautre est une espèce de blé} en son quartier ?
\VS{26}Parce que son Dieu l'a instruit, et lui a enseigné ce qu'il faut faire.
\VS{27}Car on ne foule pas la vesce avec la herse\FTNT{La herse est un instrument agricole permettant de travailler la terre en surface}, et on ne tourne point la roue du chariot sur le cumin ; mais on bat la vesce avec la verge, et le cumin avec le bâton.
\VS{28}Le blé avec lequel on fait le pain se menuise, car le laboureur ne le foule pas entièrement ; et quoiqu'il l'écrase avec la roue de son chariot, néanmoins il ne le menuisera pas avec ses chevaux.
\VS{29}Cela aussi vient de Yahweh des armées qui est admirable en conseil et magnifique en moyens.
\Chap{29}
\TextTitle{Avertissement d'un châtiment imminent}
\VerseOne{}Malheur à Ariel\FTNT{Ariel : Lion de Dieu, nom appliqué à Jérusalem.}, à Ariel, la ville dont David fit sa demeure ! Ajoutez année à année, qu'on égorge des victimes pour les fêtes.
\VS{2}Mais je mettrai Ariel à l'étroit, il n'y aura que tristesse et deuil ; et elle sera pour moi comme Ariel.
\VS{3}Car je camperai en rond contre toi, et je t'assiégerai avec des tours, et je dresserai contre toi des retranchements.
\VS{4}Et tu seras abaissée, et tu parleras depuis la terre, et ta parole sortira étouffée par la poussière ; et ta voix sortira de terre comme celle d'un esprit de Python , et ta parole marmottera comme si elle sortait de la poussière.
\VS{5}La multitude de tes étrangers sera comme une fine poussière ; et la multitude des guerriers sera comme la balle qui passe, et cela sera pour un petit moment.
\VS{6}Elle sera visitée par Yahweh des armées avec des tonnerres, des tremblements de terre, et un grand bruit\FTNT{Za. 14:13-14 ; Ap. 16:18-19.} ; avec la tempête, le tourbillon, et avec la flamme d'un feu dévorant.
\VS{7}Et la multitude de toutes les nations qui feront la guerre à Ariel, et tous ceux qui la combattront, et ceux qui la serreront de près seront comme un songe d'une vision de nuit.
\VS{8}Et il arrivera que comme celui qui a faim rêve qu'il mange, mais quand il se réveille son âme est vide ; et comme celui qui a soif rêve qu'il boit, mais quand il se réveille il est épuisé, et son âme est altérée ; ainsi sera-t-il de la multitude de toutes les nations qui combattront contre la montagne de Sion.
\TextTitle{Yahweh donne les raisons du châtiment}
\VS{9}Arrêtez-vous et soyez étonnés ! Ecriez-vous et criez ! Ils sont ivres, mais non de vin ; ils chancellent, mais non pas à cause des boissons fortes.
\VS{10}Car Yahweh a répandu sur vous un esprit d'un profond sommeil\FTNT{Ro. 11:8.} ; il a fermé vos yeux, il a bandé ceux de vos prophètes et de vos principaux voyants.
\VS{11}Et toute vision est pour vous comme les paroles d'un livre cacheté que l'on donne à un homme de lettres en lui disant : Nous te prions, lis donc cela ! Et qui répond : Je ne le puis, car il est cacheté ;
\VS{12}puis si on le donne à quelqu'un qui n'est pas un homme de lettres, en lui disant : Nous te prions, lis donc cela ! Et qui répond : Je ne sais pas lire.
\VS{13}C'est pourquoi le Seigneur dit : Parce que ce peuple s'approche de moi de sa bouche et qu'il m'honore de ses lèvres, mais que son cœur est éloigné de moi ; et parce que la crainte qu'il a de moi lui a été enseigné par un commandement d'hommes\FTNT{Mt. 15:8-9 ; Mc. 7:6-7.}.
\VS{14}A cause de cela, voici, je continuerai de faire à l'égard de ce peuple-ci des merveilles et des prodiges étranges ; et la sagesse de ses sages périra, et l'intelligence de ses hommes intelligents disparaîtra.
\VS{15}Malheur à ceux qui cachent profondément leurs desseins, pour les dissimuler à Yahweh, et dont les œuvres sont dans les ténèbres, et qui disent : Qui nous voit, et qui nous connaît\FTNT{Es. 47:10 ; Ez. 8:12 ; Ps. 10:11 ; Ps. 94:7.} ?
\VS{16}Ce que vous renversez ne sera-t-il pas réputé comme l'argile d'un potier ? Même l'ouvrage dira-t-il de celui qui l'a fait : Il ne m'a point fait ? Et la chose formée dira-t-elle de celui qui l'a formée : Il n'a point d'intelligence\FTNT{Ps. 100:3.} ?
\TextTitle{Yahweh rachète Jacob}
\VS{17}Le Liban ne sera-t-il pas encore dans très peu de temps changé en un Carmel ? Et Carmel ne sera-t-il pas considéré comme une forêt ?
\VS{18}En ce jour-là, les sourds entendront les paroles du livre, et les yeux des aveugles, étant délivrés de l'obscurité et des ténèbres, verront\FTNT{Mt. 11:5 ; Lu. 7:22.}.
\VS{19}Les humbles auront joie sur joie en Yahweh, et les pauvres d'entre les hommes se réjouiront dans le Saint d'Israël\FTNT{Mt. 5:3-11.}.
\VS{20}Car l'oppresseur prendra fin, le moqueur sera consumé, et tous ceux qui veillaient pour commettre l'iniquité seront retranchés\FTNT{Ap. 20:10.},
\VS{21}ceux qui rendaient coupable les hommes pour une parole, qui tendaient des pièges à celui qui les reprenait à la porte, et qui faisaient tomber le juste en confusion. 
\VS{22}C'est pourquoi ainsi parle Yahweh, lui qui a racheté Abraham, à la maison de Jacob : Jacob ne sera plus honteux, et sa face ne pâlira plus.
\VS{23}Car quand il verra ses fils, ouvrage de mes mains, au milieu de lui, ils sanctifieront mon Nom ; ils sanctifieront, dis-je, le Saint de Jacob, et ils craindront le Dieu d'Israël.
\VS{24}Et ceux dont l'esprit s'était fourvoyé deviendront intelligents, et ceux qui murmuraient apprendront la doctrine.
\Chap{30}
\TextTitle{Mise en garde contre les alliances étrangères}
\VerseOne{}Malheur aux enfants rebelles, dit Yahweh, qui prennent des conseils, et non pas de moi, et qui se forgent des idoles de métal où mon esprit n'est point, afin d'ajouter péché sur péché.
\VS{2}Qui sans avoir interrogé ma bouche, marchent pour descendre en Egypte, afin de se fortifier de la force de Pharaon et se retirer sous l'ombre de l'Egypte\FTNT{Jé. 42:19}.
\VS{3}Car la force de Pharaon sera pour vous une honte, et le refuge sous l'ombre de l'Egypte votre confusion.
\VS{4}Car ses princes sont à Tsoan, et ses messagers ont atteint Hanès.
\VS{5}Tous seront rendus honteux par un peuple qui ne leur profitera de rien, ils n'en recevront aucun secours ni aucun avantage, il sera leur honte et leur opprobre.
\VS{6}Les bêtes sont chargées pour aller au midi, ils portent leurs richesses sur les dos des ânons, et leurs trésors sur la bosse des chameaux, vers le peuple qui ne leur profitera point dans le pays de détresse et d'angoisse, d'où viennent le vieux lion et le lion, la vipère et le serpent volant ; .
\VS{7}Car le secours de l'Egypte n'est que vanité et néant ; c'est pourquoi je crie ceci : Leur force est de se tenir tranquille.
\VS{8}Va maintenant, et écris-le en leur présence sur une table, et rédige-le par écrit dans un livre, afin que cela demeure pour le temps à venir, à perpétuité, à jamais ;
\VS{9}que c'est ici un peuple rebelle, des enfants menteurs, des enfants qui ne veulent point écouter la loi de Yahweh\FTNT{No. 20: 3-5 ; De. 9:7 ; Ac. 7:51.} ;
\VS{10}qui disent aux voyants : Ne voyez pas ! Et aux prophètes : Ne nous prophétisez pas des choses droites, mais dites-nous des choses agréables, voyez des choses trompeuses\FTNT{2 Ti. 4:3-4 ; Mi. 2:6.} !
\VS{11}Retirez-vous du chemin, détournez-vous du sentier, éloignez de notre présence le Saint d'Israël\FTNT{Jn. 14:6.}.
\VS{12}C'est pourquoi ainsi dit le Saint d'Israël : Parce que vous rejetez cette parole et que vous vous confiez dans l'oppression et dans les détours, et que vous vous êtes appuyés sur ces choses,
\VS{13}à cause de cela, cette iniquité sera pour vous comme la fente d'une muraille qui va tomber, un renflement dans un mur élevé, dont la ruine vient soudainement, et en un instant.
\VS{14}Il la brise donc comme on brise un vase de terre, que l'on n'épargne point, et de ses pièces, il ne se trouve pas un tesson pour prendre du feu au foyer, ou pour puiser de l'eau à la citerne.
\TextTitle{La confiance en Yahweh, la vraie force}
\VS{15}Car ainsi a parlé le Seigneur Yahweh, le Saint d'Israël : En vous tenant tranquille et en repos vous serez sauvés ; votre force sera en vous tenant en repos et en espérance. Mais vous ne l'avez point voulu.
\VS{16}Et vous avez dit : Non, mais nous nous enfuirons sur des chevaux ; à cause de cela vous vous enfuirez. Et vous avez dit : Nous monterons sur des chevaux rapides ; à cause de cela ceux qui vous poursuivront seront rapides.
\VS{17}Mille d'entre vous s'enfuiront à la menace d'un seul ; vous vous enfuirez à la menace de cinq ; jusqu'à ce que vous soyez abandonnés comme un arbre tout ébranché au sommet d'une montagne, et comme un étendard sur la colline.
\VS{18}Cependant Yahweh attend pour vous faire grâce, et ainsi il sera exalté pour vous faire miséricorde ; car Yahweh est le Dieu de jugement : Ô bienheureux sont tous ceux qui se confient en lui !
\VS{19}Car le peuple demeurera dans Sion et dans Jérusalem. Tu ne pleureras point ! Certes, il te fera grâce dès qu'il entendra ton cri ; dès qu'il aura entendu, il t'exaucera.
\VS{20}Le Seigneur vous donnera du pain de détresse, et de l'eau d'angoisse, mais tes enseignants ne s'envoleront plus, et tes yeux verront tes enseignants.
\VS{21}Et tes oreilles entendront la parole de celui qui sera derrière toi, disant : Voici le chemin, marchez-y ,soit que vous tiriez à droite, soit que vous tiriez à gauche !
\VS{22}Et vous tiendrez pour souillés les chapiteaux des images taillées faites d'argent, et les ornements faits d'or fondu ; tu les jetteras au loin comme un sang impur, et tu leur diras : Hors d'ici ! 
\VS{23}Alors il donnera la pluie sur la semence que tu auras semées en terre, et le grain du revenu de la terre sera abondant et bien nourri ; en ce jour-là, ton bétail paîtra dans un pâturage spacieux\FTNT{Jn. 14:6.}.
\VS{24}Les bœufs et les ânes qui labourent la terre mangeront le pur fourrage de ce qui aura été vanné avec la pelle et le van.
\VS{25}Et il y aura des ruisseaux d'eau courante sur toute haute montagne, et sur toute colline haut élevée, au jour de la grande tuerie, quand les tours tomberont.
\VS{26}Et la lumière de la lune sera comme la lumière du soleil ; et la lumière du soleil sera sept fois plus grande, comme si c'était la lumière de sept jours, le jour où Yahweh bandera la blessure de son peuple, et qu'il guérira la blessure de sa plaie.
\TextTitle{Jugement de Yahweh sur les Assyriens}
\VS{27}Voici, le Nom de Yahweh vient de loin, sa colère est ardente, et une pesante charge ; ses lèvres sont pleines d'indignation, et sa langue est comme un feu dévorant.
\VS{28}Son Esprit est comme un torrent qui déborde et atteint jusqu'au milieu du cou, pour disperser les nations d'une telle dispersion qu'elles seront réduites à néant, et il est comme une bride aux mâchoires des peuples, qui les fera errer.
\VS{29}Vous aurez un cantique comme la nuit où l'on célèbre une fête solennelle ; vous aurez le cœur joyeux comme celui qui marche au son de la flûte, pour aller à la montagne de Yahweh, vers le Rocher d'Israël.
\VS{30}Et Yahweh fera entendre sa voix, pleine de majesté, et il montrera où aura assené son bras dans l'indignation de sa colère, avec une flamme de feu dévorant, avec éclat, tempête, et pierres de grêle.
\VS{31}Car l'Assyrien, qui frappait du bâton, sera effrayé par la voix de Yahweh.
\VS{32}Et partout où passe le bâton dont Yahweh l'a assené, et par lequel il combattra dans les batailles à bras élevé, on entendra les tambourins et les harpes.
\VS{33}Car Topheth\FTNT{Topheth : Lieu pour brûler. Un lieu à l'extrémité sud-est de la vallée de Hinnom au sud de Jérusalem.} est déjà préparée, et même elle est apprêtée pour le roi ; on a fait son bûcher profond et large ; son bûcher c'est du feu et du bois en abondance ; le souffle de Yahweh l'allume comme un torrent de soufre.
\Chap{31}
\TextTitle{Le secours de Yahweh préférable à celui de l'Egypte}
\VerseOne{}Malheur à ceux qui descendent en Egypte pour avoir de l'aide, et qui s'appuient sur les chevaux, et qui mettent leur confiance dans leurs chars parce qu'ils sont nombreux, et en leurs cavaliers quand ils sont bien forts, mais qui ne regardent pas vers le Saint d'Israël, et ne recherchent pas Yahweh.
\VS{2}Et cependant, c'est lui qui est sage, et il fait venir le malheur et ne révoque point sa parole ; il s'élève contre la maison des méchants et contre ceux qui aident les ouvriers d'iniquité.
\VS{3}Or les Egyptiens sont des hommes et non Dieu ; et leurs chevaux sont chair et non esprit. Quand Yahweh étendra sa main, et celui qui donne du secours sera renversé ; et celui à qui le secours est donné tombera ; et eux tous ensemble seront consumés.
\VS{4}Mais ainsi m'a dit Yahweh : Comme le lion, comme le lionceau rugit sur sa proie, et quoiqu'on appelle contre lui un grand nombre de bergers, il ne se laisse ni effrayer par leur cri, ni abaisser par leur bruit ; ainsi Yahweh des armées descendra pour combattre en faveur de la montagne de Sion et de sa colline.
\VS{5}Comme les oiseaux volent, ainsi Yahweh des armées défendra Jérusalem, la défendant et la délivrant, passant outre et la sauvant\FTNT{De. 32:11 ; Ps. 91:4 ; Mt. 23:37.}.
\VS{6}Retournez vers celui de qui les enfants d'Israël se sont étrangement éloignés.
\VS{7}Car en ce jour-là, chacun rejettera ses idoles d'argent et ses idoles d'or que vos propres mains ont fabriquées pour vous faire pécher.
\VS{8}Et l'Assyrien tombera par l'épée qui n'est pas celle d'un vaillant homme, et l'épée qui n'est pas celle d'un homme le dévorera ; et il s'enfuira devant l'épée, et ses jeunes hommes seront rendus tributaires.
\VS{9}Et saisi de frayeur, il s'enfuira à sa forteresse, et ses chefs seront effrayés à cause de la bannière, dit Yahweh, qui a son feu dans Sion et son fourneau dans Jérusalem.
\Chap{32}
\TextTitle{La venue de l'Esprit annonce la paix et la justice}
\VerseOne{}Voici, un roi régnera selon la justice, et les princes gouverneront avec équité.
\VS{2}Et un homme sera comme le lieu où l'on se cache du vent et comme un asile contre la tempête ; comme des ruisseaux d'eau dans un pays sec, et l'ombre d'un grand rocher dans une terre altérée.
\VS{3}Alors les yeux de ceux qui voient ne seront point retenus, et les oreilles de ceux qui entendent seront attentives.
\VS{4}Et le cœur des étourdis entendra la science, et la langue de ceux qui balbutient parlera aisément et nettement.
\VS{5}Le chiche ne sera plus appelé libéral, et l'avare trompeur ne sera plus nommé magnifique.
\VS{6}Car l'homme vil dira des choses viles, et son cœur ne machine qu'iniquité, pour exécuter son hypocrisie et pour proférer des faussetés contre Yahweh, pour rendre vide l'âme de celui qui a faim, et faire tarir la boisson de celui qui a soif\FTNT{Jn. 10:10.}.
\VS{7}Les instruments de l'avare sont pernicieux ; il prend des conseils pleins de machinations, pour attraper par des paroles de mensonge les affligés, même quand la cause du pauvre est juste\FTNT{2 Pi. 2:3.}.
\VS{8}Mais le libéral forme des conseils de libéralité et se lève pour user de libéralité.
\VS{9}Femmes qui êtes à votre aise, levez-vous, écoutez ma voix ! Filles qui vous tenez assurées, prêtez l'oreille à ma parole !
\VS{10}Dans un an et quelques jours, vous qui vous tenez assurées serez troublées ; car la vendange a manqué, la récolte n'arrivera plus.
\VS{11}Vous qui êtes à votre aise, tremblez ! Vous qui vous tenez assurées, soyez troublées ! Dépouillez-vous, quittez vos habits et ceignez de sacs vos reins !
\VS{12}On se frappe la poitrine à cause de la vigne abondante en fruits.
\VS{13}Les épines et les ronces montent sur la terre de mon peuple, même sur toutes les maisons où il y a de la joie et sur la ville joyeuse.
\VS{14}Car le palais est abandonné, la multitude de la cité est délaissée ; les lieux inaccessibles du pays et les forteresses serviront de cavernes à toujours ; les ânes sauvages y joueront, et les troupeaux y paîtront,
\VS{15}jusqu'à ce que l'Esprit soit répandu d'en haut sur nous\FTNT{Joë. 2:28 ; Za.12:10 ; Ac. 2:17-18.}, et que le désert devienne un Carmel et que Carmel  soit considéré comme une forêt.
\VS{16}Le jugement habitera dans le désert et la justice se tiendra en Carmel.
\VS{17}La justice produira de la paix, et le fruit de la justice sera le repos et la sécurité pour toujours.
\VS{18}Mon peuple habitera dans une demeure paisible, et dans des habitations assurées, et dans un repos fort tranquille.
\VS{19}Mais la grêle tombera sur la forêt, et la ville sera entièrement abaissée.
\VS{20}Heureux vous qui semez sur toutes les eaux, et qui laissez sans entraves le pied du bœuf et de l'âne !
\Chap{33}
\TextTitle{Yahweh se lève}
\VerseOne{}Malheur à toi qui dépouilles et qui n'as pas été dépouillé ! Qui pilles et qu'on n'a pas encore pillé ! Quand tu auras fini de dépouiller, tu seras dépouillé ; et quand tu auras achevé de piller, on te pillera.
\VS{2}Yahweh, aie pitié de nous ! Nous nous attendons à toi ! Sois leur bras dès le matin et notre délivrance au temps de la détresse !
\VS{3}Au son du tumulte, les peuples s'enfuient ; quand tu te lèves, les nations se dispersent.
\VS{4}Et votre butin est recueilli comme on rassemble les sauterelles ; on saute  dessus comme sautellent les sauterelles.
\VS{5}Yahweh est élevé, car il habite dans les lieux élevés ; il remplit Sion de jugement et de justice\FTNT{Ps. 97:9.}.
\VS{6}Et la sagesse et la science seront la certitude de ta durée, et la force de ton salut ; la crainte de Yahweh est son trésor.
\VS{7}Voici, leurs hérauts poussent des cris au-dehors, et les messagers de paix pleurent amèrement.
\VS{8}Les routes sont réduites en désolation, les passants n'y passent plus. Il a rompu l'alliance, il rejette les villes, il ne fait plus cas des hommes.
\VS{9}On mène le deuil, la terre languit. Le Liban est honteux et flétri. Le Saron est comme un désert. Le Basan et le Carmel secouent leur feuillage.
\VS{10}Maintenant je me lèverai, dit Yahweh, maintenant je serai exalté, maintenant je serai élevé.
\VS{11}Vous avez conçu du foin, et vous enfanterez de la paille ; votre souffle vous dévorera comme le feu.
\VS{12}Et les peuples seront des fourneaux de chaux ; ils seront brûlés au feu comme des épines coupées.
\VS{13}Vous qui êtes loin, écoutez ce que j'ai fait ! Et vous qui êtes près, connaissez ma force !
\TextTitle{Yahweh assure la paix aux justes}
\VS{14}Les pécheurs sont effrayés dans Sion, et le tremblement saisit les hypocrites, tellement qu'ils disent : Qui de nous pourra séjourner avec le feu dévorant\FTNT{Hé. 12:29.} ? Qui de nous pourra séjourner avec les flammes éternelles ?
\VS{15}Celui qui observe la justice et qui profère des choses droites ; celui qui rejette le gain déshonnête d'extorsion, et qui secoue ses mains pour ne pas accepter un présent ; celui qui bouche ses oreilles pour ne pas entendre des propos sanguinaires, et qui ferme ses yeux pour ne pas voir le mal,
\VS{16}Celui-là habitera dans des lieux élevés, des forteresses assises sur des rochers seront sa haute retraite ; son pain lui sera donné, et ses eaux ne lui manqueront point\FTNT{Jn. 4:14 ; Jn. 6:33-35 ; Ap 21:6.}.
\VS{17}Tes yeux contempleront le roi dans sa beauté ; et ils regarderont la terre éloignée.
\VS{18}Ton cœur méditera-il la frayeur, en disant : Où est le secrétaire, où est le trésorier ? Où est celui qui tient le compte des tours ?
\VS{19}Tu ne verras plus le peuple fier, le peuple au langage inconnu qu'on n'entend pas, et de langue bégayante qu'on ne comprend pas.
\VS{20}Regarde Sion, la ville de nos fêtes solennelles ! Que tes yeux voient Jérusalem, séjour tranquille, tabernacle qui ne sera pas transportée, et dont les pieux ne seront jamais ôtés, et dont les cordages ne seront point rompus\FTNT{Ap. 21:2.}.
\VS{21}C'est là que Yahweh nous est glorieux ; c'est le lieu de fleuves, de vastes rivières, où n'ira pas de navire à rame et où aucun gros navire passera.
\VS{22}Parce que Yahweh est notre Juge, Yahweh est notre Législateur, Yahweh est notre Roi\FTNT{Jésus-Christ exerce toutes les fonctions gouvernementales : législatives, exécutives et judiciaires.} ; c'est lui qui vous sauvera.
\VS{23}Tes cordages sont lâchés ; et ainsi ils ne tiennent point ferme leur mât et on n'étendra point la voile. Alors la dépouille d'un grand butin est partagé ; même les boiteux pillent le butin.
\VS{24}Et celui qui fait sa demeure dans la maison ne dit point : Je suis malade ! Le peuple qui habite en elle reçoit le pardon de ses iniquités.
\Chap{34}
\TextTitle{Le jugement des nations\FTNTT{Ap. 19:17-21.}}
\VerseOne{}Approchez-vous nations, pour écouter ! Et vous peuples, soyez attentifs ! Que la terre et tout ce qui la remplit écoute ! Que le monde habitable et tout ce qui y est produit écoute !
\VS{2}Car l'indignation de Yahweh est sur toutes les nations, et sa fureur sur toute leur armée ; il les voue à l'interdit, il les livre pour être tuées.
\VS{3}Leurs blessés à morts sont jetés là, et la puanteur de leurs corps morts se répand et les montagnes découlent de leur sang.
\VS{4}Et toute l'armée des cieux se fond ; les cieux sont roulés comme un livre\FTNT{Ap. 6:14.}, et toute leur armée tombe, comme tombe la feuille de la vigne, et comme tombe celle du figuier\FTNT{Mt. 24:28 ; Mc. 13:25.}.
\VS{5}Parce que mon épée s'est enivrée dans les cieux, voici, elle va descendre en jugement contre Edom, et contre le peuple que j'ai voué à l'interdit.
\VS{6}L'épée de Yahweh est pleine de sang ; engraissée de graisse, et du sang des agneaux et des boucs, et de la graisse des reins de béliers ; car il y a des sacrifices de Yahweh à Botsra, et une grande tuerie dans le pays d'Edom.
\VS{7}Les licornes descendent avec eux, et les bœufs avec les taureaux ; leur terre est enivrée de sang, et leur poussière engraissée de graisse.
\VS{8}Car c'est un jour de vengeance pour Yahweh, une année de rétribution pour maintenir la cause de Sion\FTNT{Jé. 46:10 ; Joë. 2:2 ; So. 1:15.}.
\VS{9}Et ces torrents d'Edom seront changés en poix, et sa poussière en soufre, et sa terre deviendra de la poix ardente.
\VS{10}Elle ne sera point éteinte ni jour ni nuit ; sa fumée montera éternellement, elle sera désolée de génération en génération ; il n'y aura personne qui passe par elle à jamais.
\VS{11}Le pélican et le hérisson la posséderont, la chouette et le corbeau  y habiteront ; et on étendra sur elle la ligne de la désolation et le niveau de désordre.
\VS{12}Ses magistrats crieront qu'il n'y a plus là de royaume, et tous ses princes seront réduits à néant.
\VS{13}Les épines croîtront dans ses palais, les chardons et les buissons dans ses forteresses; elle sera la demeure des dragons, et le parvis des hiboux.
\VS{14}Les bêtes sauvages des déserts rencontreront les bêtes sauvages des îles ; et les boucs s'y appelleront les uns les autres ; là aussi, la Lilith\FTNT{Lilith est le nom d'une déesse de la nuit connue pour être un démon nocturne qui hantait les lieux déserts d'Edom.} aura sa demeure et trouvera son lieu de repos ;
\VS{15}là le martinet fera son nid, déposera ses œufs, les couvera, et recueillera ses petits à son ombre ; et là aussi se rassembleront tous les vautours.
\VS{16}Consultez le livre de Yahweh et lisez : Il n'en manquera pas un seul point ; ni l'un ni l'autre ne manqueront ; car c'est ma bouche qui l'a ordonné, et son Esprit qui les rassemblera.
\VS{17}Car il leur a jeté le sort, et sa main leur a partagé cette terre au cordeau, ils la posséderont toujours, ils l'habiteront d'âge en âge.
\Chap{35}
\TextTitle{Yahweh se révèle et sauve son peuple}
\VerseOne{}Le désert et le lieu aride seront dans la joie ; le lieu solitaire se réjouira et fleurira comme une rose.
\VS{2}Il fleurira abondamment, et se réjouira, se réjouissant même et chantant en triomphe. La gloire du Liban lui est donnée, avec la magnificence de Carmel et de Saron ; ils verront la gloire de Yahweh et la magnificence de notre Dieu.
\VS{3}Renforcez les mains lâches, et fortifiez les genoux tremblants\FTNT{Hé. 12:12.}.
\VS{4}Dites à ceux qui ont le cœur troublé : Prenez courage et ne craignez plus\FTNT{Jn. 14:1 ; Jn. 16:33.} ; voici votre Dieu, la vengeance viendra, la rétribution de Dieu ; il viendra lui-même et vous délivrera.
\VS{5}Alors les yeux des aveugles seront ouverts, et les oreilles des sourds seront débouchées.
\VS{6}Alors le boiteux sautera comme un cerf, et la langue du muet chantera en triomphe\FTNT{Esaïe a annoncé la venue de Yahweh lui-même. Cette prophétie s'est parfaitement accomplie en Jésus-Christ qui a réalisé tout ce qui avait été prédit. « Allez rapporter à Jean ce que vous entendez et ce que vous voyez : Les aveugles voient, les boiteux marchent, les lépreux sont purifiés, les sourds entendent, les morts ressuscitent, et l'Evangile est annoncé aux pauvres » (Mt. 11:4-5).}. Car des eaux jailliront dans le désert, et des torrents dans le lieu solitaire.
\VS{7}Et les lieux secs deviendront des étangs, et la terre desséchée deviendra des sources d'eaux ; et dans les repaires où des dragons faisaient leur gîte, il y aura un parvis à roseaux et à joncs.
\VS{8}Il y aura là un sentier et un chemin, qu'on appellera le chemin de sainteté ; celui qui est souillé n'y passera point, mais il sera pour ceux-là ; celui qui va son chemin, et les insensés ne s'y égareront point\FTNT{Mt. 7:13-14 ; Jn. 14:6.}.
\VS{9}Là il n'y aura point de lion ; et aucune des bêtes qui ravissent les autres, n'y montera, et ne s'y trouvera ; mais les rachetés y marcheront.
\VS{10}Ceux dont Yahweh a payé la rançon\FTNT{Jésus-Christ est Yahweh qui a payé notre rançon (Mc. 10:45).}, retourneront, et viendront en Sion avec chant de triomphe, et une joie éternelle sera sur leur tête ; ils obtiendront la joie et l'allégresse ; la douleur et le gémissement s'enfuiront.
\Chap{36}
\TextTitle{Invasion de Sanchérib, menaces de Rabschaké\FTNTT{2 R. 18:9-37 ; 2 Ch. 32:1-19.}}
\VerseOne{}La quatorzième année du roi Ezéchias, Sanchérib, roi d'Assyrie, monta contre toutes les villes fortes de Juda et les prit\FTNT{2 R. 18:17.}.
\VS{2}Puis le roi d'Assyrie envoya de Lakis à Jérusalem, vers le roi Ezéchias, Rabschaké avec une puissante armée. Rabschaké s'arrêta à l'aqueduc de l'étang supérieur, sur le chemin du champ du foulon.
\VS{3}Alors Eliakim, fils de Hilkija, chef de la maison du roi, Schebna, le secrétaire, et Joach, fils d'Asaph, l'archiviste, sortirent vers lui.
\VS{4}Rabschaké leur dit : Dites maintenant à Ezéchias : Ainsi parle le grand roi, le roi d'Assyrie : Quelle est cette confiance que tu as ?
\VS{5}Je te le dis, ce ne sont là que des paroles ; mais il faut pour la guerre de la prudence et de la force. Or maintenant en qui t'es tu confié pour t'être rebellé contre moi ?
\VS{6}Voici, tu t'es confié sur ce bâton qui n'est qu'un roseau cassé, sur l'Egypte, qui perce et traverse la main de celui qui s'appuie dessus ; tel est Pharaon, roi d'Egypte, à tous ceux qui se confient en lui.
\VS{7}Que si tu me dis : Nous nous confions en Yahweh, notre Dieu. Mais n'est-ce pas lui dont Ezéchias a ôté les hauts lieux et les autels, en disant à Juda et à Jérusalem : Vous vous prosternerez devant cet autel-ci ?
\VS{8}Maintenant donc, donne des otages au roi d'Assyrie, mon maître ; et je te donnerai deux mille chevaux, si tu peux donner autant d'hommes pour monter dessus.
\VS{9}Et comment ferais-tu tourner le visage à un seul gouverneur d'entre les moindres serviteurs de mon maître ? Mais tu te confies en l'Egypte pour les chars et pour les cavaliers.
\VS{10}Mais suis-je monté sans Yahweh dans ce pays pour le détruire ? Yahweh m'a dit : Monte contre ce pays et détruis-le.
\VS{11}Alors Eliakim, Schebna et Joach dirent à Rabschaké : Nous te prions de parler en langue araméenne à tes serviteurs, car nous la comprenons ; mais ne parle pas en langue judaïque, pendant que le peuple qui est sur la muraille l'écoute.
\VS{12}Et Rabschaké répondit : Mon maître m'a-t-il envoyé vers ton maître ou vers toi, pour dire ces paroles là ? Ne m'a-t-il pas envoyé vers les hommes qui se tiennent sur la muraille, pour leur dire qu'ils mangeront leur propre fiente, et qu'ils boiront leur urine avec vous ?
\VS{13}Puis Rabschaké se dressa et s'écria à haute voix en langue judaïque, et dit : Ecoutez les paroles du grand roi, du roi d'Assyrie !
\VS{14}Ainsi parle le roi : Qu'Ezéchias ne vous séduise pas, car il ne pourra pas vous délivrer.
\VS{15}Qu'Ezéchias ne vous fasse pas confier en Yahweh, en disant : Yahweh nous délivrera certainement ; cette ville ne sera point livrée entre les mains du roi d'Assyrie.
\VS{16}N'écoutez point Ezéchias ; car ainsi parle le roi d'Assyrie : Faites un accord avec moi pour votre bien, et sortez vers moi, et vous mangerez chacun  de sa vigne, et chacun de son figuier, et vous boirez chacun de l'eau de sa citerne,
\VS{17}jusqu'à ce que je vienne, et que je vous emmène dans un pays qui est comme votre pays, un pays de blé et de bon vin, un pays de pain et de vignes.
\VS{18}Qu'Ezéchias donc ne vous séduise point, en disant : Yahweh nous délivrera. Les dieux des nations ont-ils délivré chacun leur pays de la main du roi d'Assyrie ?
\VS{19}Où sont les dieux de Hamath et d'Arpad ? Où sont les dieux de Sepharvaïm ? Ont-ils délivré Samarie de ma main ?
\VS{20}Qui sont ceux d'entre tous les dieux de ces pays qui aient délivré leur pays de ma main, pour que Yahweh délivre Jérusalem de ma main ?
\VS{21}Mais ils se turent et ne lui répondirent pas un mot ; car le roi avait donné cet ordre, disant : Vous ne lui répondrez pas.
\TextTitle{Ezéchias informé des menaces}
\VS{22}Après cela, Eliakim fils de Hilkija, chef de la maison du roi, Schebna, le secrétaire, et Joach, fils d'Asaph l'archiviste, s'en revinrent auprès d'Ezéchias, les vêtements déchirés, et lui rapportèrent les paroles de Rabschaké.
\Chap{37}
\TextTitle{Ezéchias recherche Yahweh auprès d'Esaïe\FTNTT{2 R. 19:1-7 ; 2 Ch. 32:20.}}
\VerseOne{}Et il arriva qu'aussitôt que le roi Ezéchias eut entendu ces choses, il déchira ses vêtements, se couvrit d'un sac, et entra dans la maison de Yahweh\FTNT{2 R. 19:1-7 ; 2 Ch. 32:20.}.
\VS{2}Puis il envoya Eliakim, chef de la maison du roi, et Schebna, le secrétaire, et les plus anciens des sacrificateurs couverts de sacs, vers Esaïe, le prophète, fils d'Amots.
\VS{3}Et ils lui dirent : Ainsi parle Ezéchias : Ce jour est un jour d'angoisse, de répréhension et de blasphème ; car les enfants sont près de sortir du sein maternel, mais il n'y a point de force pour enfanter.
\VS{4}Peut-être que Yahweh, ton Dieu, a-t-il entendu les paroles de Rabschaké, que le roi d'Assyrie, son maître, a envoyé pour blasphémer le Dieu vivant et lui faire outrage ; selon les paroles que Yahweh, ton Dieu, a entendues ; fais donc requête pour le reste qui subsiste encore.
\VS{5}Les serviteurs du roi Ezéchias vinrent vers Esaïe.
\VS{6}Et Esaïe leur dit : Voici ce que vous direz à votre maître : Ainsi parle Yahweh : Ne crains point pour les paroles que tu as entendues, par lesquelles les serviteurs du roi d'Assyrie m'ont blasphémé.
\VS{7}Voici, je vais mettre en lui un esprit tel qu'ayant entendu une certaine rumeur, il retournera dans son pays, et je le ferai tomber par l'épée dans son pays.
\TextTitle{Provocation et menace de Sanchérib\FTNTT{2 R. 19:8-13 ; 2 Ch. 32:17-19.}}
\VS{8}Or quand Rabschaké s'en fut retourné, il alla trouver le roi d'Assyrie qui attaquait Libna, car il avait appris qu'il était parti de Lakis.
\VS{9}Alors le roi d'Assyrie ayant entendu dire au sujet de Tirhaka, roi d'Ethiopie : Il est sorti pour te faire la guerre. Dès qu'il eut entendu cela, il envoya des messagers à Ezéchias, en leur disant :
\VS{10}Vous parlerez ainsi à Ezéchias, roi de Juda : Que ton Dieu, auquel tu te confies, ne te séduise point, en disant : Jérusalem ne sera point livrée entre les mains du roi d'Assyrie.
\VS{11}Voilà, tu as entendu ce que les rois d'Assyrie ont fait à tous les pays, en les détruisant entièrement ; et toi, tu échapperais ?
\VS{12}Les dieux des nations que mes ancêtres ont détruites, à savoir Gozan, Charan, Retseph, et les fils d'Eden, qui sont à Telassar, les ont-ils délivrées ?
\VS{13}Où sont le roi de Hamath, le roi d'Arpad, et le roi de la ville de Sepharvaïm, d'Héna et d'Ivva ?
\TextTitle{Prière d'Ezéchias à Yahweh\FTNTT{2 R. 19:14-19 ; 2 Ch. 32:20.}}
\VS{14}Et quand Ezéchias reçut les lettres de la main des messagers et les lut, il monta à la maison de Yahweh, et Ezéchias les déploya devant Yahweh.
\VS{15}Puis Ezéchias fit sa prière à Yahweh, en disant :
\VS{16}Ô Yahweh des armées ! Dieu d'Israël qui es assis entre les chérubins ! C'est toi qui es le seul Dieu de tous les royaumes de la terre, c'est toi qui as fait les cieux et la terre.
\VS{17}Ô Yahweh ! Incline ton oreille et écoute ! Ô Yahweh ! Ouvre tes yeux et regarde ! Ecoute les paroles de Sanchérib, qu'il m'a envoyé dire pour blasphémer le Dieu vivant.
\VS{18}Il est bien vrai, ô Yahweh, que les rois d'Assyrie ont détruit tous les pays et leurs contrées ;
\VS{19}et qu'ils ont jeté dans le feu leurs dieux ; mais ce n'étaient point des dieux, mais un ouvrage de mains d'homme, du bois et de la pierre ; c'est pourquoi ils les ont détruits.
\VS{20}Maintenant donc, ô Yahweh notre Dieu ! Délivre-nous de la main de Sanchérib, afin que tous les royaumes de la terre sachent que toi seul es Yahweh.
\TextTitle{Esaïe transmet la réponse de Yahweh\FTNTT{2 R. 19:20-34.}}
\VS{21}Alors Esaïe, fils d'Amots, envoya dire à Ezéchias : Ainsi parle Yahweh, le Dieu d'Israël : J'ai entendu la prière que tu m'as faite au sujet de Sanchérib, roi d'Assyrie.
\VS{22}C'est ici la parole que Yahweh a prononcée contre lui : La vierge, fille de Sion, te méprise et se moque de toi ; la fille de Jérusalem hoche la tête après toi.
\VS{23}Contre qui as-tu élevé ta voix, et levé tes yeux en haut ? C'est contre le Saint d'Israël.
\VS{24}Tu as outragé le Seigneur par le moyen de tes serviteurs, et tu as dit : Je suis monté avec la multitude de mes chars sur le haut des montagnes, aux côtés du Liban, je couperai les plus hauts cèdres, et les plus beaux cyprès qui y soient, et j'entrerai jusqu'en son plus haut bout, et en la forêt de son Carmel.
\VS{25}J'ai creusé des sources, et j'en ai bu les eaux, et je tarirai avec la plante de mes pieds tous les fleuves de l'Egypte.
\VS{26}N'as-tu pas appris, qu'il y a déjà longtemps, j'ai fait cette ville, et que dès les temps anciens je l'ai ainsi formée ? Et maintenant l'aurais-je conservée pour être réduite en désolation, et les villes fortes en monceaux de ruines ?
\VS{27}Or leurs habitants, étant dénués de force, ont été épouvantés et confus ; ils sont devenus comme l'herbe des champs ; et l'herbe verte, comme le foin des toits, et le blé brûlé avant la formation de sa tige.
\VS{28}Mais je sais quand tu t'assieds, quand tu sors et quand tu entres, et comment tu es furieux contre moi\FTNT{Ps. 139:2.}.
\VS{29}Parce que tu es furieux contre moi, et que ton insolence est montée à mes oreilles, je mettrai ma boucle à tes narines, et mon mors en ta bouche, et je te ferai retourner par le chemin par lequel tu es venu.
\VS{30}Et ceci te sera pour signe, ô Ezéchias, c'est qu'on mangera cette année ce qui viendra de soi-même aux champs ; et en la deuxième année ce qui croîtra encore sans semer ; mais la troisième année, vous sèmerez, vous moissonnerez, vous planterez des vignes, et vous en mangerez le fruit.
\VS{31}Et ce qui est réchappé, et demeuré de reste dans la maison de Juda, étendra sa racine par-dessous, et elle produira du fruit par-dessus.
\VS{32}Car il sortira de Jérusalem un reste, et de la montagne de Sion quelques réchappés, la jalousie de Yahweh des armées fera cela.
\VS{33}C'est pourquoi ainsi parle Yahweh sur le roi d'Assyrie : Il n'entrera point dans cette ville, il n'y jettera aucune flèche, il ne se présentera point contre elle avec le bouclier, et il ne dressera point de retranchements contre elle.
\VS{34}Il s'en retournera par le chemin par lequel il est venu, et il n'entrera point dans cette ville, dit Yahweh.
\VS{35}Car je protégerai cette ville pour la délivrer pour l'amour de moi, et pour l'amour de David, mon serviteur.
\TextTitle{Yahweh frappe Sanchérib\FTNTT{2 R. 19:35-37 ; 2 Ch. 32:21.}}
\VS{36}L'ange de Yahweh\FTNT{Ge. 16:7.} sortit et frappa cent quatre-vingt-cinq mille hommes dans le camp des Assyriens. Et quand on se leva le matin, voici, ils étaient tous morts.
\VS{37}Alors Sanchérib, roi d'Assyrie, partit de là ; il s'en alla et s'en retourna, et il se tint à Ninive.
\VS{38}Et il arriva qu'étant prosterné dans la maison de Nisroc\FTNT{Le nom Nisroc signifie « le grand aigle ». C'était une idole de Ninive adorée par Sanchérib, symbolisée par un aigle à figure humaine.}, son dieu, Adrammélec et Scharetser, ses fils, le tuèrent avec l'épée ; puis ils s'enfuirent au pays d'Ararat. Et Esar-Haddon, son fils, régna à sa place.
\Chap{38}
\TextTitle{Maladie et guérison d'Ezéchias\FTNTT{2 R. 20:1-11 ; 2 Ch. 32:24-30.}}
\VerseOne{}En ces jours-là, Ezéchias fut malade à la mort\FTNT{2 R. 20:1-11 ; 2 Ch. 32:24-30.}. Et Esaïe le prophète, fils d'Amots, vint auprès de lui, et lui dit : Ainsi parle Yahweh : Donne tes ordres à ta maison, car tu vas mourir et tu ne vivras plus.
\VS{2}Alors Ezéchias tourna sa face contre la muraille et fit sa prière à Yahweh,
\VS{3}et dit : Ô Yahweh, souviens-toi maintenant je te prie que j'ai marché devant toi en vérité et en intégrité de cœur, et que j'ai fait ce qui est agréable à tes yeux ! Et Ezéchias pleura abondamment.
\VS{4}Puis la parole de Yahweh fut adressée à Esaïe, en disant :
\VS{5}Va, et dis à Ezéchias ainsi parle Yahweh, le Dieu de David, ton père : J'ai exaucé ta prière, j'ai vu tes larmes. Voici, j'ajouterai à tes jours quinze années.
\VS{6}Et je te délivrerai de la main du roi d'Assyrie, toi et cette ville, et je défendrai cette ville.
\VS{7}Et ce signe t'est donné par Yahweh, pour voir que Yahweh accomplira la parole qu'il a prononcée.
\VS{8}Voici, je ferai retourner de dix degrés en arrière avec le soleil l'ombre des degrés qui est descendue sur les degrés d'Achaz. Et le soleil retourna de dix degrés par les degrés par lesquels il était descendu.
\VS{9}Or c'est ici l'écrit d'Ezéchias, roi de Juda, sur sa maladie et sur son rétablissement.
\VS{10}J'avais dit dans le retranchement de mes jours : Je m'en irai aux portes du scheol, je suis privé de ce qui restait de mes années.
\VS{11}Je disais : Je ne contemplerai plus Yahweh, Yahweh sur la terre des vivants ; je ne verrai plus aucun homme parmi les habitants du monde !
\VS{12}Ma durée s'en est allée, et a été transportée loin de moi, comme une cabane de berger ; ma vie est coupée je suis retranché comme la toile que le tisserand détache de sa trame. Du matin au soir tu m'auras enlevé\FTNT{Aux versets 12 et 13, le mot qui a été traduit par « enlevé » est « shalam » : « être dans une alliance de paix, être en paix ».} !
\VS{13}Je pensais en moi-même jusqu'au matin ; comme un lion, qui briserait ainsi tous mes os ; du matin au soir tu m'auras enlevé !
\VS{14}Je grommelais comme la grue et l'hirondelle ; je gémissais comme la colombe ; mes yeux défaillaient à force de regarder en haut : Ô Yahweh, je suis opprimé, sois mon garant !
\VS{15}Que dirai-je ? Il m'a parlé et lui-même l'a fait. Je m'en irai tout doucement tous les ans de ma vie, dans l'amertume de mon âme.
\VS{16}Seigneur, par ces choses-là on a la vie, et dans toutes ces choses est la vie de mon esprit. Ainsi tu me rétabliras et me feras revivre.
\VS{17}Voici, dans ma paix, une grande amertume m'est survenue, mais tu as embrassé mon âme afin qu'elle ne tombe pas dans la fosse de la pourriture, car tu as jeté tous mes péchés derrière ton dos.
\VS{18}Car le scheol ne te loue point,  la mort ne te célèbre point ; ceux qui sont descendus dans la fosse ne s'attendent plus à ta vérité\FTNT{Ps. 115:17.}.
\VS{19}Mais le vivant, le vivant est celui qui te célèbre, comme moi aujourd'hui ; le père conduira ses enfants à la connaissance de ta vérité\FTNT{Pr. 22:6 ; Ep. 6:4.}.
\VS{20}Yahweh est venu me délivrer, et à cause de cela, nous jouerons sur les instruments mes cantiques, tous les jours de notre vie dans la maison de Yahweh.
\VS{21}Or Esaïe avait dit : Qu'on prenne une masse de figues sèches et qu'on en fasse un emplâtre sur l'ulcère ; et Ezéchias guérira.
\VS{22}Et Ezéchias avait dit : Quel est le signe que je monterai à la maison de Yahweh ?
\Chap{39}
\TextTitle{Ezéchias montre toutes ses richesses aux Babyloniens\FTNTT{2 R. 20:12-19}}
\VerseOne{}En ce temps-là\FTNT{2 R. 20:12-19.}, Mérodac-Baladan, fils de Baladan, roi de Babylone, envoya des lettres avec un présent à Ezéchias, parce qu'il avait entendu qu'il avait été malade, et qu'il était guéri.
\VS{2}Et Ezéchias en eut de la joie, et il leur montra les cabinets où étaient ses choses précieuses, l'argent, l'or, et les aromates, et l'huile précieuse, tout son arsenal, et tout ce qui se trouvait dans ses trésors ; il n'y eut rien qu'Ezéchias ne leur montra dans sa maison et dans tous ses domaines.
\VS{3}Puis le prophète Esaïe vint vers le roi Ezéchias, et lui dit : Qu'ont dit ces hommes-là, et d'où sont-ils venus vers toi ? Et Ezéchias répondit : Ils sont venus vers moi d'un pays éloigné, de Babylone.
\VS{4}Puis Esaïe dit : Qu'ont-ils vu dans ta maison ? Ezéchias répondit : Ils ont vu tout ce qui est dans ma maison ; il n'y a rien dans mes trésors que je ne leur aie montré.
\VS{5}Et Esaïe dit à Ezéchias : Ecoute la parole de Yahweh des armées :
\VS{6}Voici, les jours viennent où l'on emportera à Babylone tout ce qui est dans ta maison, et ce que tes pères ont amassé dans leurs trésors jusqu'à aujourd'hui ; il n'en restera rien, dit Yahweh\FTNT{2 R. 24:13 ; 2 R. 25:13-15 ; Jé. 20:5.}.
\VS{7}Même on prendra de tes fils qui sortiront de toi, et que tu auras engendrés afin qu'ils soient eunuques dans le palais du roi de Babylone\FTNT{Da. 1:3-4.}.
\VS{8}Et Ezéchias répondit à Esaïe : La parole de Yahweh, que tu as prononcée, est bonne ; et, il ajouta, au moins qu'il y ait paix et sécurité pendant mes jours.
\Chap{40}
\TextTitle{Un nouveau message pour Esaïe}
\VerseOne{}Consolez, consolez mon peuple, dit votre Dieu.
\VS{2}Parlez à Jérusalem selon son cœur, et criez-lui que son temps marqué est accompli, que son iniquité est tenue pour acquittée, qu'elle a reçu de la main de Yahweh le double pour tous ses péchés.
\TextTitle{Mission de Jean-Baptiste\FTNTT{Mt. 3:3.}}
\VS{3}La voix de celui qui crie au désert\FTNT{L'accomplissement de cette prophétie se trouve en Mt 3:3, où il nous est dit que la voix qui devait crier ces choses était celle de Jean-Baptiste (voir aussi Mal. 3:1 ; Mal. 4:5-6 ; Mt. 17:10-13).} est : Préparez le chemin de Yahweh\FTNT{Les évangiles nous enseignent que Jean-Baptiste a été envoyé pour préparer le chemin du Seigneur Jésus (Jn. 1:19-27 ; Jn. 1:29-34 ; Jn. 3:28-31).}, aplanissez parmi les lieux arides un chemin pour notre Dieu.
\VS{4}Toute vallée sera comblée, toute montagne et toute colline seront abaissées, et les lieux tortueux seront redressés, et les lieux raboteux seront aplanis.
\VS{5}Alors la gloire de Yahweh sera manifestée, et toute chair en même temps la verra, car la bouche de Yahweh a parlé.
\TextTitle{La grandeur de Dieu échappe à l'homme}
\VS{6}La voix dit : Crie ! Et on a répondu : Que crierai-je ? Toute chair est comme l'herbe, et toute sa grâce est comme la fleur d'un champ\FTNT{Ja. 1:10 ; 1 Pi. 1:24-25.}.
\VS{7}L'herbe sèche, et la fleur tombe, parce que le vent de Yahweh souffle dessus. Certainement le peuple est comme l'herbe.
\VS{8}L'herbe sèche, et la fleur tombe, mais la parole de notre Dieu demeure éternellement.
\VS{9}Sion, qui annonce de bonnes nouvelles, monte sur une haute montagne ; Jérusalem, qui annonce de bonnes nouvelles, élève ta voix avec force ; élève-la, ne crains point ; dis aux villes de Juda : Voici votre Dieu !
\VS{10}Voici, le Seigneur Yahweh\FTNT{Jésus-Christ est Yahweh qui vient (Es. 35:4 ; Es. 40:10-11 ; Es. 60:1 ; Es. 62:11-12 ; Es. 66:15-16 ; Za. 14:1-7 ; Mt. 24 ; Jn. 14:1-3; Ac. 1:10-12 ; Ap. 3:11 ; Ap. 19:11-12 ; Ap. 22:7 ; Ap. 22:12 ; Ap. 22:20).} viendra contre le fort, et son bras dominera sur lui ; voici son salaire est avec lui, et ses rétributions sont devant lui.
\VS{11}Il paîtra son troupeau comme un berger, il rassemblera les agneaux dans ses bras, il les placera dans son sein ; il conduira celles qui allaitent\FTNT{Jn. 10.}.
\VS{12}Qui est celui qui a mesuré les eaux avec le creux de sa main, et qui a pris les dimensions des cieux avec la paume, qui a rassemblé toute la poussière de la terre dans un boisseau, et qui a pesé au crochet les montagnes et les collines à la balance ?
\VS{13}Qui a dirigé l'Esprit de Yahweh, ou qui a été son conseiller pour l'enseigner\FTNT{1 Co. 2:16 ; Ro. 11:34.} ?
\VS{14}Avec qui a-t-il pris conseil, et qui l'a instruit, et lui a enseigné le sentier de jugement ? Qui lui a enseigné la science, et lui a montré le chemin de l'intelligence ?
\VS{15}Voilà, les nations sont comme une goutte qui tombe d'un seau, et elles sont réputée comme la menue poussière d'une balance ; voila, il a jeté çà et là les îles comme de la poudre.
\VS{16}Et le Liban ne suffirait pas pour faire le feu, et les bêtes qui y sont ne seraient pas suffisantes pour l'holocauste.
\VS{17}Toutes les nations sont devant lui comme un rien, et il ne les considère que comme de la poussière, et comme un néant.
\VS{18}A qui donc ferez-vous ressembler Dieu ? Et à quelle ressemblance l'égalerez-vous ?
\VS{19}L'ouvrier fond l'image, et l'orfèvre la couvre d'or, et y soude des chaînettes d'argent.
\VS{20}Celui qui est si pauvre qu'il n'a pas de quoi faire une offrande, choisit un bois qui ne pourrisse point ; il se cherche un habile ouvrier pour faire une image taillée qui ne bouge pas\FTNT{Es. 44:9-20.}.
\VS{21}Ne le savez-vous pas ? Ne l'avez-vous pas entendu ? Cela ne vous a-t-il pas été déclaré dès le commencement ? Ne l'avez-vous pas entendu dès les fondements de la terre ?
\VS{22}C'est lui qui est assis au-dessus du globe de la terre, et à qui ses habitants sont comme des sauterelles ; c'est lui qui étend les cieux comme un voile, il les déploie même comme une tente pour y demeurer.
\VS{23}C'est lui qui réduit les princes à rien, et qui fait des chefs de la terre une chose de néant.
\VS{24}Ils ne sont pas même plantés, pas même semés, même leur tronc n'a point de racine en terre ; il souffle sur eux, et ils sèchent, et le tourbillon les emporte comme de la paille.
\VS{25}A qui donc me ferez-vous ressembler, et à qui serais-je égalé ? Dit le Saint.
\VS{26}Elevez vos yeux en haut et regardez ! Qui a créé ces choses ? C'est lui qui fait sortir leur armée par ordre, et qui les appelle toutes par leur nom ; il n'y en a pas une qui fait défaut, à cause de la grandeur de sa force, et parce qu'il excelle en puissance.
\VS{27}Pourquoi donc dis-tu, ô Jacob, pourquoi dis-tu, ô Israël : Ma voie est cachée à Yahweh, et mon jugement passe inaperçu devant mon Dieu ?
\VS{28}Ne sais-tu pas ? N'as-tu pas entendu que le Dieu d'éternité, Yahweh, a créé les extrémités de la terre ; il ne se fatigue point, il ne se lasse point, et il n'y a pas moyen de sonder son intelligence.
\VS{29}C'est lui qui donne de la force à celui qui est las, et il multiplie la force de celui qui n'a aucune vigueur.
\VS{30}Les jeunes gens se lassent et se fatiguent, même les jeunes hommes tombent sans force.
\VS{31}Mais ceux qui s'attendent à Yahweh renouvellent leur force. Ils s'élèvent avec des ailes, comme des aigles ; ils courent, et ne se fatiguent point ; ils marchent, et ne se lassent point.
\Chap{41}
\TextTitle{Dénonciation des idoles}
\VerseOne{}Iles, faites moi silence ! Que les peuples renouvellent leurs forces ; qu'ils s'approchent et qu'alors ils parlent ; allons ensemble en jugement.
\VS{2}Qui a fait levé l'homme droit de l'orient ? Qui l'a appelé à sa suite ? Qui a soumis à son commandement les nations ? Qui lui a donné la domination sur les rois ? Qui les a livrés à son épée comme de la poussière, et à son arc comme de la paille poussée par le vent ?
\VS{3}Il les a poursuivis, il est passé en paix par le chemin que son pied n'avait jamais foulé.
\VS{4}Qui est celui qui a opéré et fait ces choses ? C'est celui qui a appelé les âges dès le commencement. Moi, Yahweh, JE SUIS le premier, et JE SUIS avec les derniers\FTNT{Ap. 1:8 ; Ap. 21:6 ; Ap. 22:13.}.
\VS{5}Les îles voient, et sont dans la crainte, les extrémités de la terre sont effrayées, ils s'approchent, ils viennent.
\VS{6}Chacun aide son prochain, et chacun dit à son frère : Fortifie-toi.
\VS{7}L'ouvrier encourage le fondeur ; celui qui frappe doucement du marteau encourage celui qui frappe sur l'enclume, et il dit : Cela est bon pour souder, puis il fixe l'idole avec des clous, afin qu'elle ne bouge pas.
\VS{8}Mais toi, Israël, tu es mon serviteur, et toi, Jacob, tu es celui que j'ai élu, la race d'Abraham qui m'a aimé !
\VS{9}Car je t'ai pris aux extrémités de la terre, je t'ai appelé en te préférant aux plus excellents qui sont en elle, et je t'ai dit : C'est toi qui est mon serviteur, je t'ai élu, et je ne te rejette point\FTNT{De. 7:6 ; Ps. 77:8.}.
\VS{10}Ne crains rien, car je suis avec toi ; ne sois pas étonné, car je suis ton Dieu ; je te fortifie, et je t'aide, même je te soutiens par la droite de ma justice.
\VS{11}Voici, tous ceux qui sont indignés contre toi seront honteux et confus ; ils seront réduits à néant, et les hommes qui ont querelle avec toi périront.
\VS{12}Tu les chercheras, et tu ne les trouveras plus, ceux qui te suscitaient querelle ; ils seront réduits à néant, et ceux qui te font la guerre seront comme ce qui n'est plus.
\VS{13}Car je suis Yahweh, ton Dieu, qui soutient ta main droite, et te dis : Ne crains rien, c'est moi qui te secours.
\VS{14}Ne crains point, vermisseau de Jacob, hommes mortels d'Israël ; je viens à ton secours dit Yahweh, et ton défenseur, le Saint d'Israël.
\VS{15}Voici, je fais de toi un traîneau aigu, tout neuf, ayant des dents ; tu fouleras les montagnes et les menuiseras, et tu rendras les collines semblables à de la balle.
\VS{16}Tu les vanneras, et le vent les emportera, et le tourbillon les dispersera. Mais toi, tu te réjouiras en Yahweh, tu te glorifiera au Saint d'Israël.
\VS{17}Quant aux affligés et aux misérables qui cherchent des eaux, et n'en ont point ; dont la langue est tellement altérée qu'elle n'en peut plus ; moi, Yahweh, je les exaucerai ; moi, le Dieu d'Israël, je ne les abandonnerai pas\FTNT{Ge. 28:15 ; Jos. 1:5 ; Hé. 13:5.}.
\VS{18}Je ferai jaillir des fleuves sur les hauteurs et des fontaines au milieu des vallées ; et je ferai du désert des étang d'eaux et de la terre sèche des sources d'eaux.
\VS{19}Je ferai croître au désert le cèdre, l'acacia, le myrte et l'olivier ; je mettrai dans les lieux stériles le cyprès, l'orme et le buis ensemble,
\VS{20}afin qu'on voit, qu'on sache, qu'on pense, et qu'on comprenne que la main de Yahweh a fait cela, et que le Saint d'Israël a créé cela.
\VS{21}Plaidez votre cause, dit Yahweh ; et mettez en avant les fondements de votre cause, dit le Roi de Jacob.
\VS{22}Qu'ils les amènent et qu'ils nous déclarent ce qui doit arriver. Déclarez-nous que veulent dire les choses qui ont été auparavant et nous y prendrons garde, et nous saurons leur issue, ou faites-nous entendre ce qui est prêt à arriver. 
\VS{23}Déclarez les choses qui doivent arriver dorénavant, et nous saurons que vous êtes des dieux ; faites aussi du bien ou du mal, et nous en serons tout étonnés puis nous regarderons ensemble.
\VS{24}Voici, vous n'êtes rien, et votre œuvre est le néant ; celui qui vous choisit n'est qu'abomination.
\VS{25}Je l'ai suscité du nord, et il est venu ; il invoque mon Nom de devant le soleil levant ; et marche sur les princes comme sur le mortier, et les foule comme le potier foule la boue.
\VS{26}Qui est celui qui a manifesté ces choses dès le commencement, afin que nous le connaissions ? Et longtemps d'avance, que nous puissions dire : Il est juste. Mais il n'y a personne qui les annonce, même il n'y a personne qui les donne à entendre, même il n'y a personne qui entende vos paroles.
\VS{27}Le premier sera pour Sion, disant : Voici, les voici ! Et je donnerai quelqu'un à Jérusalem qui annoncera de bonnes nouvelles\FTNT{Es. 52:7 ; Ap. 14:6.}.
\VS{28}Je regarde, et il n'y a point d'homme même entre ceux-là, et il n'y a aucun homme de conseil ; je les interroge aussi afin qu'il réponde quelque chose. 
\VS{29}Voici, quant à eux tous, leurs œuvres ne sont que vanité, leurs idoles de fonte sont du vent et de la confusion.
\Chap{42}
\TextTitle{Le messie, serviteur de Yahweh}
\VerseOne{}Voici mon serviteur, que je soutiens, c'est mon élu, en qui mon âme prend son bon plaisir ; j'ai mis mon Esprit sur lui, il manifestera le jugement aux nations\FTNT{Mt. 3:17 ; Mt. 17:5 ; Mc. 9:7.}.
\VS{2}Il ne criera point, et il ne haussera, ni ne fera entendre sa voix dans les rues.
\VS{3}Il ne brisera point le roseau cassé, et il n'éteindra point le lumignon qui fume\FTNT{Mt. 12:18-20.} ; il mettra en avant le jugement en vérité.
\VS{4}Il ne se retirera point et ne s'affaiblira point, jusqu'à ce qu'il ait établi la justice sur la terre, et que les îles s'attendent à sa loi.
\VS{5}Ainsi parle Dieu, Yahweh, qui a créé les cieux, et qui les a étendus, qui a aplani la terre avec ce qu'elle produit, qui donne la respiration au peuple qui est sur elle, et l'esprit à ceux qui y marchent.
\VS{6}Moi Yahweh, je t'ai appelé en justice, et je prendrai ta main et te garderai, et je te ferai être l'alliance du peuple et la lumière des nations\FTNT{Voir commentaire en Ge. 1:3-5.},
\VS{7}afin d'ouvrir les yeux des aveugles, et de faire sortir les prisonniers hors du lieu où on les tient enfermés, et ceux qui habitent dans les ténèbres hors de la prison.
\TextTitle{Israël n'a pas été attentif à Yahweh}
\VS{8}Je suis Yahweh, c'est là mon Nom ; et je ne donnerai pas ma gloire à un autre, ni ma louange aux images taillées\FTNT{Es. 48:11.}.
\VS{9}Voici, les choses qui ont été prédites auparavant se sont accomplies. Et je vous en annonce de nouvelles ; et je vous les fait entendre avant qu'elles arrivent.
\VS{10}Chantez à Yahweh un cantique nouveau, et que sa louange éclate aux extrémités de la terre, vous qui descendez en la mer, et tout ce qui est en elle,  les îles et leurs habitants !
\VS{11}Que le désert et ses villes élèvent la voix ! Que les villages où habite Kédar et ceux qui habitent dans les rochers éclatent en chant de triomphe ! Qu'ils s'écrient du sommet des montagnes !
\VS{12}Qu'on donne gloire à Yahweh, et qu'on publie sa louange dans les îles !
\VS{13}Yahweh sort comme un homme vaillant, il réveille sa jalousie comme un homme de guerre, il jette, dis-je, des cris de joie, il jette de grands cris, et il prévaut sur ses ennemis.
\VS{14}Je me suis tu dès longtemps ; me tiendrais-je en repos ? Me retiendrais-je ? Je crierai comme celle qui enfante, je détruirai, et j'engloutirai tout à la fois.
\VS{15}Je réduirai les montagnes et les collines en désert, et j'en dessécherai toute la verdure, je réduirai les fleuves en îles, et je ferai tarir les étangs.
\VS{16}Je conduirai les aveugles sur un chemin qu'ils ne connaissent pas, je les ferai marcher par des sentiers qu'ils ne connaissent pas ; je réduirai devant eux les ténèbres en lumière, et les choses tortues en choses droites ; voilà ce que je ferai, et je ne les abandonnerai point.
\VS{17}Ils se retireront en arrière, et ils seront tout honteux, ceux qui se confient aux images taillées, et qui disent aux images de fonte : Vous êtes nos dieux !
\VS{18}Sourds, écoutez ! Et vous aveugles, regardez et voyez !
\VS{19}Qui, dis-je, est aveugle, sinon mon serviteur ? Et qui est sourd, comme mon messager que j'envoie ? Qui est aveugle, comme celui que j'ai comblé de grâces ? Qui est aveugle, comme le serviteur de Yahweh ?
\VS{20}Vous voyez beaucoup de choses, mais vous ne prenez garde à rien ; vous avez les oreilles ouvertes, mais vous n'entendez rien.
\VS{21}Yahweh a plaisir en lui à cause de sa justice ; il a magnifié la loi et l'a  rendu honorable. 
\VS{22}Mais c'est ici un peuple pillé et dépouillé ! Ils sont enlacés dans les cavernes, et sont cachés dans des prisons ; ils sont un butin, et il n'y a personne qui les délivre ; une proie, et il n'y a personne qui dise : Restituez !
\VS{23}Qui est celui d'entre vous qui prêtera l'oreille à ces choses ? Qui s'y rendra attentif et l'écoutera à l'avenir ?
\VS{24}Qui est-ce qui a livré Jacob au pillage, et Israël aux pillards\FTNT{Jg. 2:13-16.} ? N'est-ce pas Yahweh, contre lequel nous avons péché ? Car on n'a point voulu marcher dans ses voies et on n'a point obéi à sa loi.
\VS{25}C'est pourquoi il a répandu sur lui la fureur de sa colère, et une forte guerre ; et il l'a embrasé tout alentour, mais Israël ne l'a point connu ; et il l'a brûlé, mais il n'y a point pris garde.
\Chap{43}
\TextTitle{Yahweh veut racheter Israël}
\VerseOne{}Mais maintenant ainsi parle Yahweh, qui t'a créé, ô Jacob ! Celui qui t'a formé, ô Israël ! Ne crains point, car je te rachète, je t'appelle par ton nom, tu es à moi !
\VS{2}Si tu passes par les eaux, je serai avec toi ; et si tu passes par les fleuves, ils ne te noieront pas ; si tu marches dans le feu, tu ne seras pas brûlé, et la flamme ne t'embrasera pas.
\VS{3}Car je suis Yahweh, ton Dieu, le Saint d'Israël, ton Sauveur. Je donne l'Egypte pour ta rançon, l'Ethiopie et Saba à ta place.
\VS{4}Parce que tu es précieux à mes yeux, tu es rendu honorable et je t'aime, je donne des hommes à ta place, et des peuples pour ta vie.
\VS{5}Ne crains point, car je suis avec toi ; je ferai venir ta postérité de l'orient, et je t'assemblerai de l'occident.
\VS{6}Je dirai au nord : Donne ! Et au midi : Ne retiens point ! Fais venir mes fils de loin, et mes filles du bout de la terre,
\VS{7}savoir tous ceux qui s'appellent de mon Nom\FTNT{Dans les Ecritures, le Nom de Dieu le plus cité est YHWH. Jésus, dont le nom signifie « YHWH est salut » correspond au nom et à l'identité que Dieu a révélé à tous ceux qui l'ont rencontré quand il était sur cette terre. Dans sa dernière prière à Gethsémané, Jésus dit : « J'ai fait connaître ton Nom » (Jn. 17:6), et « Je leur ai fait connaître ton Nom » (Jn. 17:26). Ce nom n'est autre que le sien puisque Jésus (YHWH est salut) était et est le Nom de Dieu. Moïse n'avait pas reçu la révélation de ce Nom (Ex. 3:13-14) car cette révélation était réservée à l'Eglise. En tant qu'épouse de Christ, l'Eglise porte le Nom du Seigneur et bénéficie de l'autorité qu'il confère. Ainsi, Jésus est le seul Nom par lequel nous pouvons être sauvés (Ac. 4:12). C'est aussi en son Nom que nous devons être baptisés (Ac. 8:16 ; Ac. 19:5), que nous recevons l'exaucement de nos prières (Jn. 14:13-14 ; Jn. 16:24), que nous sommes délivrés de l'ennemi et que nous obtenons la victoire sur le camp de l'ennemi (Mc. 16:17 ; Ph. 2:9-11).} ; car je les ai créés pour ma gloire ; je les ai formés et les ai faits.
\TextTitle{Yahweh appelle ses témoins}
\VS{8}Amène dehors le peuple aveugle qui a des yeux, et les sourds qui ont des oreilles.
\VS{9}Que toutes les nations soient ramassées ensemble, et que les peuples soient assemblés. Lequel d'entre eux a annoncé ces choses-là ? Et qui sont ceux qui nous ont fait entendre les choses qui ont été ci-devant ? Qu'ils produisent leurs témoins et qu'ils se justifient ; qu'on les entende et qu'on dise : C'est vrai !
\VS{10}Vous êtes mes témoins\FTNT{Ac. 1:8.}, dit Yahweh, et mon serviteur que j'ai élu, afin que vous connaissiez, que vous me croyiez et que vous compreniez que JE SUIS. Avant moi il n'a pas été formé de Dieu, et il n'y en aura point après moi.
\VS{11}Moi, JE SUIS Yahweh, et à part moi il n'y a point de Sauveur\FTNT{Yahweh dit qu'à part lui, il n'y a pas d'autres sauveurs. Or les écrits de la nouvelle alliance affirment que Jésus-Christ est le seul Sauveur (Lu. 1:67-80 ; Ac. 4:11-12).}.
\VS{12}C'est moi qui ai prédit ce qui devait arriver, qui vous ai sauvés, et qui vous ai fait entendre l'avenir, quand il n'y avait point de dieu étranger parmi vous ; et vous êtes mes témoins, dit Yahweh, que je suis Dieu.
\VS{13}Et même avant que le jour fût, JE SUIS, et il n'y a personne qui puisse délivrer de ma main ; je ferai l'œuvre, qui m'en empêchera ?
\TextTitle{Yahweh fera une chose nouvelle car Jacob ne l'a pas honoré}
\VS{14}Ainsi parle Yahweh, votre Rédempteur\FTNT{Es. 60:16 ; 1 Co. 1:30 ; Ro. 3:24 ; Ep. 1:7.}, le Saint d'Israël : J'envoie pour l'amour de vous contre Babylone, et je les fais descendre tous fugitifs, et le cri des Chaldéens sera dans les navires.
\VS{15}Je suis Yahweh, votre Saint, le Créateur d'Israël, votre Roi.
\VS{16}Ainsi parle Yahweh, qui fraya un chemin dans la mer, et un sentier parmi les eaux impétueuses ;
\VS{17}qui amena des chars et des chevaux, et de grandes forces ; ils ont été étendus ensemble, et ils ne se relèveront point, ils ont été étouffés, ils ont été éteints comme un lumignon :
\VS{18}Ne pensez plus aux choses passées, et ne considérez point les choses anciennes.
\VS{19}Voici, je m'en vais faire une chose nouvelle\FTNT{2 Co. 5:17.}, qui paraîtra bientôt, ne la connaîtrez-vous pas ? Je mettrai un chemin dans le désert, et des fleuves dans le lieu de désolation.
\VS{20}Les bêtes des champs me glorifieront, les serpents et les autruches, parce que j'aurai mis des eaux dans le désert, et des fleuves dans la solitude, pour abreuver mon peuple que j'ai élu.
\VS{21}Ce peuple que je me suis formé racontera mes louanges.
\VS{22}Mais toi, Jacob, tu ne m'as pas invoqué, car tu t'es lassé de moi, ô Israël !
\VS{23}Tu ne m'as pas offert le menu bétail de tes holocaustes, et tu ne m'as pas glorifié dans tes sacrifices ; je ne t'ai point asservi pour me faire des offrandes, et je ne t'ai point fatigué pour de l'encens.
\VS{24}Tu ne m'as pas acheté à prix d'argent du roseau aromatique, et tu ne m'as pas rassasié de la graisse de tes sacrifices ; mais tu m'as asservi par tes péchés, et tu m'as peiné par tes iniquités.
\VS{25}Moi, JE SUIS celui qui efface tes transgressions pour l'amour de moi, et je ne me souviendrai plus de tes péchés.
\VS{26}Réveille ma mémoire, et plaidons ensemble ; toi, déclare pour que tu puisses être justifié.
\VS{27}Ton premier père a péché, et tes docteurs se sont rebellés contre moi.
\VS{28}C'est pourquoi j'ai profané les chefs du lieu saint, et j'ai livré Jacob à la destruction, et Israël à l'opprobre.
\Chap{44}
\TextTitle{Promesse de l'Esprit, folie de l'idolâtrie}
\VerseOne{}Ecoute maintenant, ô Jacob, mon serviteur, et toi Israël que j'ai choisi !
\VS{2}Ainsi parle Yahweh, qui t'a fait et formé dès le ventre, celui qui te soutient : Ne crains point, ô Jacob, mon serviteur ! Et toi Jeshurun que j'ai élu.
\VS{3}Car je répandrai des eaux sur celui qui est altéré, et des rivières sur la terre sèche ; je répandrai mon esprit sur ta postérité, et ma bénédiction sur ta descendance.
\VS{4}Et ils germeront comme au milieu de l'herbe, comme les saules auprès des courants d'eau.
\VS{5}L'un dira : Je suis à Yahweh ; et l'autre se réclamera du nom de Jacob ; et un autre écrira de sa main : Je suis à Yahweh, et se nommera du nom d'Israël.
\VS{6}Ainsi parle Yahweh, le Roi d'Israël et son Rédempteur, Yahweh des armées : Je suis le premier, et je suis le dernier ; et à part moi il n'y a point de Dieu.
\VS{7}Et qui, comme moi, a appelé, déclaré et ordonné cela, depuis que j'ai établi le peuple ancien ? Qu'ils déclarent les choses à venir, les choses qui arriveront ci-après !
\VS{8}Ne soyez point effrayés et ne soyez point troublés ; ne te l'ai-je pas fait entendre et déclarer dès ce temps-là ? Vous êtes mes témoins ; y a-t-il un autre Dieu que moi ? Certes il n'y a pas d'autre Rocher\FTNT{Yahweh dit qu'il ne connaît pas d'autre rocher. Jésus-Christ est ce rocher qui suivait les Hébreux dans le désert (Mt. 16:18 ; 1 Co. 10:1-4. Voir aussi commentaire en Es. 8:14). }, je n'en connais pas.
\VS{9}Les ouvriers d'images taillées ne sont tous que vanité, et leurs choses les plus désirables ne sont d'aucun profit ; elles le témoignent elles-mêmes, elles ne voient point, et ne connaissent point, afin qu'ils soient honteux.
\VS{10}Mais qui est-ce qui fabrique un dieu, ou fond une image taillée, pour n'en avoir aucun profit ?
\VS{11}Voici, tous ses compagnons seront honteux, car ces ouvriers-là sont d'entre les hommes. Qu'ils s'assemblent tous, qu'ils se tiennent là ! Ils seront effrayés et rendus honteux tous ensemble.
\VS{12}Le forgeron fait une hache, et il travaille avec le charbon, et il le forme à coups de marteau ; il le fait à force de bras, même il a faim et il est sans force, il ne boit point d'eau, et il est tout fatigué.
\VS{13}Le charpentier étend sa règle, il trace sa forme au crayon avec de la craie ; il le fait avec des équerres, et le forme au compas, et le fait à la ressemblance d'un homme, selon la beauté d'un homme, afin qu'il demeure dans la maison.
\VS{14}Il se coupe des cèdres, et prend un cyprès, ou un chêne, qu'il a laissé croître parmi les arbres de la forêt ; il plante des pins, et la pluie les fait croître.
\VS{15}Ces arbres servent à l'homme pour brûler, car il en prend et il s'en chauffe. Il en fait du feu, dis-je, et en cuit du pain ; et il en fait aussi un dieu et se prosterne devant lui ; il en fait une image taillée et l'adore.
\VS{16}Il en brûle au feu une partie, et d'une autre partie il mange sa chair, laquelle il rôtit, et s'en rassasie ; il s'en chauffe aussi, et il dit : Ah ! Ah ! Je me chauffe, je vois la flamme !
\VS{17}Puis avec le reste il fait un dieu pour être son image taillée ; il se prosterne devant elle, il l'adore, il lui fait sa requête et dit : Délivre-moi, car tu es mon dieu !
\VS{18}Ils ne savent et n'entendent rien, car on leur a plâtré les yeux afin qu'ils ne voient point, et les cœurs pour qu'ils ne comprennent point.
\VS{19}Nul ne rentre en lui-même\FTNT{So. 2:1 ; 2 Co. 13:5.}, et il n'a ni la connaissance ni l'intelligence pour dire : J'en ai brûlé une partie au feu, et même j'ai cuit du pain sur les charbons, j'ai rôti de la viande et je l'ai mangée ; et avec le reste ferais-je une abomination ? Adorerais-je une branche de bois ?
\VS{20}Il se repaît de cendres, et son cœur abusé l'égare, et il ne délivrera point son âme, et ne dira point : N'est-ce pas du mensonge que j'ai dans ma main droite ?
\TextTitle{Yahweh rachète son peuple}
\VS{21}Souviens-toi de ces choses, ô Jacob ! Ô Israël, car tu es mon serviteur ; je t'ai formé, tu es mon serviteur, ô Israël ! Je ne t'oublierai pas.
\VS{22}J'efface tes transgressions comme une nuée épaisse, et tes péchés comme une nuée ; reviens à moi, car je t'ai racheté.
\VS{23}Ô cieux ! Réjouissez-vous avec chants de triomphe, car Yahweh a opéré ; profondeurs de la terre, jetez des cris de réjouissance ! Montagnes, éclatez de joie avec chant de triomphe ! Et vous aussi forêts et tous les arbres qui êtes en elles ! Parce que Yahweh a racheté Jacob, et s'est manifesté glorieusement en Israël.
\VS{24}Ainsi parle Yahweh, ton Rédempteur, celui qui t'a formé dès le ventre : Je suis Yahweh qui ai fait toutes choses, qui seul ai étendu les cieux, et qui ai par moi-même étendu la terre ;
\VS{25}qui dissipe les signes des menteurs, qui rends insensés les devins ; qui renverse l'esprit des sages, et qui fait que leur science devient une folie.
\VS{26}C'est lui qui confirme la parole de son serviteur, et accomplit le conseil  de ses messagers ; qui dit à Jérusalem : Tu seras encore habitée ! Et aux villes de Juda : Vous serez rebâties ! Et je redresserai ses lieux déserts.
\VS{27}Qui dit à l'abîme : Sois asséchée, et je tarirai tes fleuves.
\TextTitle{Prophétie sur le rétablissement d'Israël par Cyrus}
\VS{28}Qui dit de Cyrus\FTNT{Esaïe prophétisa la destruction de Babylone deux siècles avant la réalisation de cet événement  le 5 octobre 539 av. J.-C. Fait remarquable : il précisa même le nom du commandant Cyrus qui dompta le lion babylonien. L’historien Hérodote donnera  par ailleurs raison au prophète sur le déroulement de la prise de Babylone.} : Il est mon berger, et il accomplira tout mon bon plaisir ; disant même à Jérusalem : Tu seras rebâtie ! Et au temple : Tu seras fondé.
\Chap{45}
\TextTitle{Cyrus suscité par Yahweh}
\VerseOne{}Ainsi parle Yahweh à son oint, à Cyrus\FTNT{Cyrus le Grand (580 av. J.-C. - 530 av. J.-C.). Voir Esd. 1.},
\VS{2}que je tiens par la main droite, pour terrasser les nations devant lui, et pour délier les ceintures des rois, pour ouvrir devant lui les portes, afin qu'elles ne soient point fermées.
\VS{3}J'irai devant toi, et j'aplanirai les lieux tortueux ; je romprai les portes d'airain, et je mettrai en pièces les barres de fer. Et je te donnerai des trésors cachés, et des richesses le plus secrètement gardées, afin que tu saches que je suis Yahweh, le Dieu d'Israël, qui t'appelle par ton nom.
\VS{4}Pour l'amour de Jacob, mon serviteur, et d'Israël mon élu ; je t'ai, dis-je, appelé par ton nom, et je t'ai surnommé avant que tu me connaisses.
\TextTitle{Yahweh, le seul Dieu}
\VS{5}Je suis Yahweh, et il n'y en a point d'autre ; à part moi, il n'y a point de Dieu. Je t'ai ceint avant que tu me connaisses,
\VS{6}afin que l'on sache, du soleil levant au soleil couchant, qu'à part moi, il n'y a point de Dieu. Je suis Yahweh, et il n'y en a point d'autre.
\VS{7}Je forme la lumière, et je crée les ténèbres ; je fais la paix et je crée l'adversité ; moi, Yahweh, je fais toutes ces choses.
\VS{8}Ô cieux ! Répandez la rosée d'en haut, et que les nuées laissent couler la justice ! Que la terre s'ouvre, qu'elle produise le salut, qu'elle fasse également germer la justice ! Moi, Yahweh, je crée ces choses.
\VS{9}Malheur à celui qui conteste avec celui qui l'a façonné ! Vase parmi des vases de terre ! L'argile dit-elle à celui qui la façonne : Que fais-tu ? Et l'œuvre dit-elle à l'ouvrier : Tu n'as point de mains\FTNT{Jé. 18:6 ; Ro. 9:21.} ?
\VS{10}Malheur à celui qui dit à son père : Qu'engendres-tu ? Et à sa mère : Qu'enfantes-tu ?
\VS{11}Ainsi parle Yahweh, le Saint d'Israël, qui est son Créateur : Interrogez-moi sur les choses à venir, mes fils ; me commanderez-vous sur l'œuvre de mes mains ?
\VS{12}C'est moi qui ai fait la terre et qui ai créé l'homme sur elle ; c'est moi qui ai étendu les cieux de mes mains, et qui ai donné la loi à toute leur armée.
\VS{13}C'est moi qui ai suscité Cyrus dans ma justice, et j'aplanirai toutes ses voies ; il rebâtira ma ville, et libérera mes captifs\FTNT{Cyrus le grand libéra les Juifs après 70 ans de captivité (Esd. 1).}, sans rançon ni présents, dit Yahweh des armées.
\TextTitle{Les autres peuples reconnaîtront la main de Yahweh sur Israël}
\VS{14}Ainsi parle Yahweh : Le travail de l'Egypte, et le trafic de l'Ethiopie, et ceux des Sabéens, gens de grande stature, passeront chez toi Jérusalem, et seront à toi ; ils marcheront à ta suite, ils passeront enchaînés, ils se prosterneront devant toi, ils te diront en suppliant : Certainement, Dieu est au milieu de toi, et il n'y a point d'autre Dieu que lui.
\VS{15}En vérité, tu es le Dieu qui te caches, le Dieu d'Israël, le Sauveur.
\VS{16}Ils sont tous honteux et confus, ils s'en vont tous avec ignominie, les fabricants d'idoles.
\VS{17}Mais Israël a été sauvé par Yahweh, d'un salut éternel ; vous ne serez ni honteux ni confus jusque dans l'éternité.
\VS{18}Car ainsi parle Yahweh qui a créé les cieux, Dieu lui-même qui a formé la terre, qui l'a faite et qui l'a affermie ; qui l'a créée pour qu'elle ne soit pas informe\FTNT{Informe : de l'hébreu « tohuw » qui signifie « informe, confusion, solitude, désert, néant ». On retrouve ce mot dès Ge. 1:2.}, qui l'a formée pour qu'elle soit habitée ; je suis Yahweh, et il n'y en a point d'autre.
\VS{19}Je n'ai point parlé en secret ni dans quelque lieu ténébreux de la terre ; je n'ai point dit à la postérité de Jacob : Cherchez-moi vainement ! Je suis Yahweh, qui prononce ce qui est juste, qui déclare ce qui est droit.
\VS{20}Assemblez-vous et venez, approchez-vous ensemble, vous les réchappés des nations ! Ceux qui portent le bois de leur image taillée ne savent rien, et invoquent un dieu qui ne sauve pas.
\VS{21}Déclarez-le, et faites-les approcher ! Qu'ils prennent conseil ensemble ! Qui a fait entendre ces choses dès l'origine, et les a déclarées dès longtemps ? N'est-ce pas moi, Yahweh ? Or il n'y a point d'autre Dieu à part moi ; un Dieu juste et un Sauveur, il n'y en a pas d'autre à part moi.
\VS{22}Vous tous qui êtes aux extrémités de la terre, regardez vers moi, et soyez sauvés ; car je suis Dieu, et il n'y en a point d'autre.
\VS{23}Je le jure par moi-même, la parole sort en justice de ma bouche, et elle ne sera point révoquée : Tout genou fléchira devant moi, et toute langue jurera par moi\FTNT{Ph. 2:9-11.}.
\VS{24}Certainement, on dira de moi : En Yahweh seul sont la justice et la force ; à lui viendront, pour être confondus, tous ceux qui étaient irrités contre lui.
\VS{25}Toute la postérité d'Israël sera justifiée, et elle se glorifiera en Yahweh.
\Chap{46}
\TextTitle{La puissance de Yahweh, l'incapacité des idoles}
\VerseOne{}Bel s'incline sur ses genoux, Nebo est renversé ; leurs faux dieux sont mis sur leurs bêtes et leur bétail ; les idoles que vous portiez, ont été chargées, elles sont un fardeau pour la bête fatiguée !
\VS{2}Elles se sont courbées, elles se sont inclinées ensemble sur leurs genoux, et ne peuvent échapper au fardeau, et elles-mêmes s'en vont en captivité.
\VS{3}Ecoutez-moi, maison de Jacob, et vous tous, tout le reste de la maison d'Israël, dont je me suis chargé dès le ventre, et que j'ai porté dès le sein maternel.
\VS{4}Jusqu'à votre vieillesse, JE SUIS ; et je vous chargerai sur moi jusqu'à votre blanche vieillesse, je l'ai fait, et je vous porterai encore, je vous chargerai sur moi et vous sauverai.
\VS{5}A qui me ferez-vous ressembler, et à qui m'égalerez-vous ? A qui me comparerez-vous pour que nous soyons semblable ? 
\VS{6}Ils tirent l'or de la bourse, et pèsent l'argent à la balance, et ils engagent un orfèvre pour en faire un dieu ; ils l'adorent, et se prosternent devant lui.
\VS{7}On le porte sur les épaules, on s'en charge ; on le pose en sa place où il se tient debout et ne bouge point de son lieu, puis on crie à lui, mais il ne répond pas, et il ne délivre pas de la détresse ceux qui crient vers lui.
\VS{8}Souvenez-vous de cela, et montrez-vous des hommes ; rappelez-le à votre pensée, ô vous transgresseurs !
\VS{9}Souvenez-vous des premières choses d'autrefois ; je suis Dieu, et il n'y en a point d'autre, je suis Dieu et il n'y en a point comme moi ;
\VS{10}qui déclare dès le commencement ce qui doit arriver à la fin, et longtemps auparavant, les choses qui n'ont pas encore été faites ; qui dis : Mon conseil tiendra, et j'exécuterai tout mon bon plaisir ;
\VS{11}qui appelle de l'orient l'oiseau de proie, et d'une terre éloignée un homme pour exécuter mon conseil. Oui, j'ai parlé, aussi je ferai venir la chose ; je l'ai formé, aussi je l'accomplirai. 
\VS{12}Ecoutez-moi, vous qui avez le cœur endurci et qui êtes éloignés de la justice.
\VS{13}Je fais approcher ma justice, elle ne s'éloignera point loin ; et mon salut, il ne tardera pas. Je mettrai le salut en Sion pour Israël, qui est ma gloire.
\Chap{47}
\TextTitle{Jugement sur Babylone}
\VerseOne{}Descends, et assieds-toi dans la poussière, vierge, fille de Babylone ! Assieds-toi à terre, il n'y a plus de trône pour la fille des Chaldéens ! Car tu ne te feras plus appeler la délicate et la voluptueuse.
\VS{2}Mets la main aux meules, et fais moudre la farine ; délie tes tresses, déchausse-toi, découvre tes jambes et traverse les fleuves !
\VS{3}Ta honte sera découverte et ton opprobre sera vue ; je prendrai vengeance, je n'irai point contre toi en homme.
\VS{4}Quant à notre Rédempteur, son Nom est Yahweh des armées, le Saint d'Israël.
\VS{5}Assieds-toi sans dire mot, et entre dans les ténèbres, fille des Chaldéens, car tu ne te feras plus appeler la dame des royaumes.
\VS{6}J'ai été embrasé de colère contre mon peuple, j'ai profané mon héritage, c'est pourquoi je les ai livrés entre tes mains, mais tu n'as point usé de miséricorde envers eux, tu as durement appesanti ton joug sur le vieillard.
\VS{7}Et tu as dit : Je serai dame à toujours ! De sorte que tu n'as point mis ces choses-là dans ton cœur, tu ne t'es point souvenue ce qu'en serait la fin.
\VS{8}Maintenant donc écoute ceci, toi voluptueuse qui habite avec assurance, et qui dis en ton cœur : C'est moi, et il n'y en a point d'autre que moi ; je ne deviendrai point veuve, et je ne saurai point ce que c'est que d'être privée d'enfants.
\VS{9}Mais ces deux choses t'arriveront en un moment, en un même jour, la privation d'enfants et le veuvage ; elles viendront sur toi dans leur perfection, pour le  grand nombre de tes sortilèges, et pour la grande abondance de tes enchantements\FTNT{Ap. 18:7-8.}.
\VS{10}Et tu t'es confiée dans ta méchanceté, et disais : Personne ne me voit ! Ta sagesse et ta science t'ont pervertie, et tu disais en ton cœur : C'est moi, et il n'y en a point d'autre que moi.
\VS{11} C'est pourquoi le mal viendra sur toi, et tu ne sauras pas quand il sera près d'arriver, et le malheur qui tombera sur toi sera tel, que tu ne pourras pas le détourner ; et la ruine éclatante que tu n'as pas soupçonnée viendra sur toi subitement.
\VS{12}Tiens-toi maintenant avec tes enchantements, et avec le grand nombre de tes sortilèges, après lesquels tu as travaillé dès ta jeunesse ; peut-être pourras-tu en tirer quelque profit ; peut-être en seras-tu renforcé.
\VS{13}Tu t'es lassée à force de demander des conseils. Que les spectateurs des cieux qui contemplent les étoiles, et qui font leurs prédictions selon les lunes, comparaissent maintenant, et qu'ils te délivrent des choses qui viendront sur toi.
\VS{14}Voici, ils sont devenus comme de la paille, le feu les consume, ils ne délivreront pas leur vie du pouvoir de la flamme ; il n'y a point de charbon pour se chauffer et il n'y a point de lueur de feu pour s'asseoir vis-à-vis. 
\VS{15}Tels te sont devenus ceux avec lesquels tu as travaillé et avec lesquels tu as trafiqué dès ta jeunesse, chacun s'en est fui en son quartier comme un vagabon ; il n'y a personne pour te sauver.
\Chap{48}
\TextTitle{Yahweh rappelle ses promesses}
\VerseOne{}Ecoutez ceci, maison de Jacob, qui êtes appelés du nom d'Israël, et qui êtes sortis des eaux de Juda ; qui jurez par le nom de Yahweh, et qui faites mention du Dieu d'Israël, mais non pas conformément à la vérité et à la justice\FTNT{Jé. 5:2.}.
\VS{2}Car ils prennent leur nom de la sainte cité, et ils s'appuient sur le Dieu d'Israël, dont le nom est Yahweh des armées\FTNT{Ex. 20:7.}.
\VS{3}J'ai déclaré les premières choses dès le commencement, elles sont sorties de ma bouche et je les ai publiées ; je les ai faites subitement et elles se sont accomplies.
\VS{4}Parce que j'ai connu que tu es obstiné, que ton cou est une barre de fer, et que ton front est d'airain,
\VS{5}je t'ai déclaré ces choses dès lors, et je les ai faites entendre avant qu'elles arrivent, de peur que tu ne dises : Mes dieux ont fait ces choses ; mon image taillée, et mon image de fonte les ont ordonnées.
\VS{6}Tu l'entends ! Vois tout ceci ! Et vous, ne l'annoncerez-vous pas ? Je te fais entendre dès maintenant des choses nouvelles, et qui étaient en réserve et que tu ne savais pas.
\VS{7}Elles sont créées maintenant, et non pas depuis le commencement ; et avant ce jour-ci tu n'en avais rien entendu, afin que tu ne dises pas : Voici, je les savais bien.
\VS{8}Oui, tu n'en avais pas entendu parler, oui, tu ne savais pas ; oui, depuis ce temps ton oreille n'a pas été ouverte ; car j'ai connu que tu agirais perfidement; aussi tu as été appelé transgresseur dès le ventre.
\VS{9}Pour l'amour de mon Nom, je diffère ma colère ; et pour l'amour de ma louange, je retiens mon courroux contre toi, afin de ne pas te retrancher.
\VS{10}Voici, je t'ai épuré, mais non pas comme on épure l'argent ; je t'ai éprouvé au creuset de l'affliction.
\VS{11}Pour l'amour de moi, pour l'amour de moi, je le ferai, car comment mon Nom serait-il profané ? Certes, je ne donnerai pas ma gloire à un autre
\VS{12}Ecoute-moi, Jacob ! Et toi Israël, mon appelé ; moi, JE SUIS le premier, JE SUIS aussi le dernier.
\VS{13}Ma main aussi a fondé la terre et ma droite a étendu les cieux ; quand je les appelle, ils comparaissent ensemble.
\VS{14}Vous tous, assemblez-vous et écoutez ! Lequel parmi eux a déclaré ces choses ? Yahweh l'aime et exécutera son bon plaisir contre Babylone, et son bras sera contre sur les Chaldéens.
\VS{15}Moi, JE SUIS celui qui ai parlé, je l'ai aussi appelé, je l'ai amené, et ses desseins réussiront.
\VS{16}Approchez-vous de moi et écoutez ceci ! Dès le commencement, je n'ai point parlé en secret, depuis l'origine de ces choses, JE SUIS. Or maintenant, le Seigneur, Yahweh, et son Esprit m'ont envoyé.
\VS{17}Ainsi parle Yahweh, ton Rédempteur, le Saint d'Israël : Je suis Yahweh, ton Dieu, qui t'enseigne pour ton profit, et qui te guide dans le chemin où tu dois marcher.
\VS{18}Ô ! Si tu étais attentif à mes commandements, ta paix serait comme un fleuve, et ta justice comme les flots de la mer\FTNT{Jos. 1:8 ; Ps. 1:2 ; Jn. 14:21 ; Ja. 1:22.},
\VS{19}ta postérité serait comme le sable, et ceux qui sortent de tes entrailles comme les grains de sable\FTNT{Ge. 15:5 ; Ge. 22:17 ; Ge. 32:12.} ; son nom ne serait point retranché ni effacé de devant ma face.
\VS{20}Sortez de Babylone, fuyez loin des Chaldéens ! Publiez ceci avec une voix de chant de triomphe, annoncez le, portez ceci jusqu'aux extrémités de la terre, dites : Yahweh a racheté son serviteur Jacob !
\VS{21}Et ils n'auront pas soif quand il les fera marcher dans les déserts ; il fera découler pour eux l'eau hors du rocher, même il leur fendra le rocher, et les eaux couleront.
\VS{22}Il n'y a point de paix pour les méchants, dit Yahweh.
\Chap{49}
\TextTitle{Le Messie, la lumière de tous les peuples}
\VerseOne{}Iles, écoutez-moi ! Soyez attentifs, vous peuples éloignés ! Yahweh m'a appelé dès le ventre, il a fait mention de mon nom dès les entrailles de ma mère\FTNT{Jé. 1:5 ; Ps. 139:16.}.
\VS{2}Et il a rendu ma bouche semblable à une épée aiguë ; il m'a caché dans l'ombre de sa main, et m'a rendu semblable à une flèche bien polie, il m'a serré dans son carquois.
\VS{3}Et il m'a dit : Tu es mon serviteur, ô Israël, en qui je serai glorifié.
\VS{4}Et moi j'ai dit : J'ai travaillé en vain, j'ai consumé ma force pour néant et sans fruit ; toutefois mon jugement est auprès de Yahweh, et ma récompense est auprès de mon Dieu.
\VS{5}Maintenant donc, Yahweh, qui m'a formé dès le ventre pour être à son service, m'a dit que je lui ramène Jacob, mais Israël ne se rassemble point ; toutefois je serai honoré aux yeux de Yahweh, et mon Dieu sera ma force.
\VS{6}Il me dit : C'est peu de chose que tu sois serviteur pour relever les tribus de Jacob et pour ramener les restes d'Israël ; c'est pourquoi je te donne pour lumière aux nations, afin que tu sois mon salut jusqu'aux extrémités de la terre.
\VS{7}Ainsi parle Yahweh, le Rédempteur, le Saint d'Israël, à celui qu'on méprise, à celui qui est abominable au peuple, au serviteur de ceux qui dominent ; les rois le verront, et se lèveront, et les princes aussi, et ils se prosterneront devant lui, pour l'amour de Yahweh, qui est fidèle, et du Saint d'Israël qui t'a élu.
\VS{8}Ainsi parle Yahweh : Je t'ai exaucé au temps de la bienveillance, et je t'ai aidé au jour du salut ; je te garderai, et je te donnerai pour être l'alliance du peuple, pour relever la terre, afin que tu possèdes les héritages désolés ;
\VS{9}disant à ceux qui sont emprisonnés : Sortez ! Et à ceux qui sont dans les ténèbres : Montrez-vous ! Ils paîtront sur les chemins, et leurs pâturages seront sur tous les lieux élevés.
\VS{10}Ils n'auront pas faim et ils n'auront pas soif ; la chaleur et le soleil ne les frapperont plus, car celui qui a pitié d'eux sera leur guide, et les conduira vers des sources d'eaux\FTNT{Ps. 121:6 ; Lu. 1:67-79.}.
\VS{11}Et je réduirai toutes mes montagnes en chemins, et mes sentiers seront relevés.
\VS{12}Voici, ceux-ci viennent de loin, et voici ceux-là viennent du nord et de l'occident, et les autres du pays de Sinim.
\VS{13}Ô cieux, réjouissez-vous avec des chants de triomphe! Et toi, ô terre, sois dans l'allégresse ! Et vous, ô montagnes, éclatez de joie avec des chants de triomphe ! Car Yahweh console son peuple, il a compassion de ceux qu'il a affligés.
\VS{14}Mais Sion disait : Yahweh me délaisse, le Seigneur m'oublie !
\VS{15}Une femme peut-elle oublier son enfant qu'elle allaite de sorte qu'elle n'ait pas pitié du fils de ses entrailles ? Mais quand les femmes les oublieraient, moi je ne t'oublierai point.
\VS{16}Voici, je t'ai gravé sur les paumes de mes mains ; tes murs sont continuellement devant moi.
\VS{17}Tes enfants viennent à grande hâtent, mais ceux qui te détruisaient et ceux qui te réduisaient en désert, sortiront du milieu de toi.
\VS{18}Elève tes yeux autour de toi, et regarde : Tous ceux-ci s'assemblent, ils viennent à toi. Je suis vivant, dit Yahweh, tu te revêtiras de tous comme d'une parure, et tu t'en orneras comme une épouse.
\VS{19}Car tes déserts, tes ruines, et ton pays détruit seront désormais trop étroits pour ses habitants, et ceux qui t'engloutissaient s'éloigneront.
\VS{20}Les enfants que tu auras après avoir perdu les autres diront encore, à tes oreilles : Le lieu est trop étroit pour moi, fais-moi de la place pour que je puisse y demeurer.
\VS{21}Et tu diras en ton cœur : Qui m'a engendré ceux-ci vu que j'avais perdu mes enfants et que j'étais stérile, emmenée en captivité et agitée ? Et qui m'a nourri ceux-ci ? Voici, j'étais restée toute seule, et ceux-ci où étaient-ils ?
\VS{22}Ainsi parle le Seigneur Yahweh : Voici, je lèverai ma main vers les nations et je dresserai ma bannière vers les peuples ; et ils ramèneront tes fils entre leurs bras, et ils porteront tes filles sur les épaules.
\VS{23}Et les rois seront tes nourriciers et leurs princesses, leurs femmes, tes nourrices ; ils se prosterneront devant toi le visage contre terre, et ils lécheront la poussière de tes pieds ; et tu sauras que je suis Yahweh, et que ceux qui se confient en moi ne seront point confus\FTNT{Ps. 22:5-6 ; Ps. 69:7 ; Ro. 9:33 ; 1 Pi. 2:6.}.
\VS{24}Le butin sera-t-il ôté à l'homme puissant ? Et les captifs du juste seront-ils délivrés ?
\VS{25}Car ainsi parle Yahweh : Même les captifs pris par l'homme puissant lui seront ôtés, et le butin de l'homme fort lui sera enlevé ; car je plaiderai moi-même avec ceux qui plaident contre toi, et je délivrerai tes enfants.
\VS{26}Et je ferai manger leur propre chair à ceux qui t'oppriment ; et ils s'enivreront de leur sang comme du moût, et toute chair connaîtra que je suis Yahweh, ton Sauveur, ton Rédempteur, le Puissant de Jacob.
\Chap{50}
\TextTitle{Avertissements de Yahweh par son serviteur}
\VerseOne{}Ainsi parle Yahweh : Où est la lettre de divorce par laquelle j'ai répudié votre mère\FTNT{De. 24:1 ; Jé. 3:8 ; Mt. 5:31.} ? Ou bien, auquel de mes créanciers vous ai-je vendus ? Voici, vous avez été vendus à cause de vos iniquités, et votre mère a été répudiée à cause de vos transgressions.
\VS{2}Je suis venu : Pourquoi ne s'est-il trouvé personne ? J'ai appelé : Pourquoi personne n'a-t-il répondu ? Ma main est-elle trop courte pour racheter\FTNT{No. 11:23 ; Es. 59:1.} ? Ou n'y a-t-il plus de force en moi pour délivrer ? Voici, par ma menace, je dessèche la mer, je réduis les fleuves en désert ; leurs poissons se corrompent faute d'eau, et ils meurent de soif.
\VS{3}Je revêts les cieux de noirceur, et je fais d'un sac leur couverture.
\VS{4}Le Seigneur, Yahweh, m'a donné la langue des savants, pour que je sache soutenir par la parole celui qui est accablé de maux\FTNT{Job. 6:14 ; 1 Th. 5:14. } ; chaque matin il me réveille soigneusement afin que je prête l'oreille aux discours des sages.
\VS{5}Le Seigneur Yahweh m'a ouvert l'oreille et je n'ai pas été rebelle, et je ne me suis pas retiré en arrière.
\VS{6}J'ai exposé mon dos à ceux qui me frappaient et mes joues à ceux qui me tiraient le poil ; je n'ai pas caché mon visage aux opprobres et aux crachats\FTNT{Mt. 5:39 ; Mt. 26:67 ; Lu. 6:29 ; Lu. 18:32.}.
\VS{7}Mais le Seigneur, Yahweh m'a aidé, c'est pourquoi je n'ai point été confus, et ainsi j'ai rendu mon visage semblable à un caillou\FTNT{Ez. 3:8-9.}, car je sais que je ne serais point rendu honteux.
\VS{8}Celui qui me justifie est proche ; qui plaidera contre moi ? Comparaissons ensemble ! Qui est mon adversaire ? Qu'il s'approche de moi.
\VS{9}Voici, le Seigneur, Yahweh m'aidera, qui est celui qui me condamnera ? Voici, tous seront usés comme un vêtement, la teigne les dévorera.
\VS{10}Qui est celui d'entre vous qui craint Yahweh, et qui obéit à la voix de son serviteur ! Que celui qui marche dans les ténèbres, et qui n'a pas de clarté, se confie dans le Nom de Yahweh, et qu'il s'appuie sur son Dieu.
\VS{11}Voici, vous tous qui allumez le feu, et qui vous ceignez d'étincelles, marchez à la lueur de votre feu et des étincelles que vous avez embrasées ; voici ce que vous aurez de ma main ; vous vous coucherez dans les tourments.
\Chap{51}
\TextTitle{Exhortation à ceux qui recherchent Yahweh}
\VerseOne{}Ecoutez-moi, vous qui poursuivez la justice et qui cherchez Yahweh ! Regardez au rocher d'où vous avez été taillés, et au creux de la citerne dont vous avez été tirés.
\VS{2}Regardez à Abraham, votre père, et à Sara qui vous a enfantés ; car lui seul je l'ai appelé, je l'ai béni et multiplié\FTNT{Ro. 4:1-16 ; Hé. 11:8-12.}.
\VS{3}Car Yahweh console Sion, il console de toutes ses désolations, il rendra son désert semblable à Eden, et sa terre aride à un jardin de Yahweh. En elle sera trouvée la joie et l'allégresse, la reconnaissance et la voie de mélodie.
\VS{4}Ecoutez-moi donc attentivement, mon peuple, et prêtez-moi l'oreille, vous ma nation ; car la loi sortira de moi, et j'établirai mon jugement pour être la lumière des peuples.
\VS{5}Ma justice est proche, mon salut va paraître, et mes bras jugeront les peuples ; les îles espéreront en moi, elles se confieront en mon bras.
\VS{6}Levez les yeux vers les cieux et regardez en bas sur la terre ! Car les cieux s'évanouiront comme la fumée, et la terre tombera en lambeaux comme un vêtement, et ses habitants périront pareillement ; mais mon salut demeurera éternellement, et ma justice ne sera point anéantie.
\VS{7}Ecoutez-moi, vous qui connaissez la justice, peuple dans le cœur duquel est ma loi ! Ne craignez point l'opprobre des hommes et ne soyez point effrayés devant leurs outrages.
\VS{8}Car la teigne les rongera comme un vêtement\FTNT{Mt. 6:19 ; Lu. 12:33 ; Ja. 5:2.}, et la gerce les dévorera comme de la laine ; mais ma justice demeurera toujours, et mon salut d'âge en âge.
\VS{9}Réveille-toi, réveille-toi, revêts-toi de force, bras de Yahweh ! Réveille-toi comme aux jours anciens, aux siècles passés. N'es-tu pas celui qui tailla en pièce l'Egypte, et qui blessa mortellement le dragon ?
\VS{10}N'est-ce pas toi qui fis tarir la mer, les eaux du grand abîme ? Qui réduisit les lieux les plus profonds de la mer en un chemin afin que les rachetés y passent ?
\VS{11}Ainsi ceux dont Yahweh aura payé la rançon, retourneront, ils iront à Sion avec chants de triomphe ; et une allégresse éternelle couronnera leurs têtes ; ils obtiendront la joie et l'allégresse ; la douleur et le gémissement s'enfuiront.
\VS{12}C'est moi qui suis celui qui vous console. Qui es-tu pour avoir peur de l'homme mortel qui mourra, et du fils de l'homme qui deviendra comme du foin ?
\VS{13}Et tu oublierais Yahweh qui t'a fait, qui a étendu les cieux et fondé la terre ; et chaque jour tu tremblerais continuellement à cause de la fureur de ton oppresseur parce qu'il s'apprête à détruire ! Et où est maintenant la fureur de ton oppresseur ?
\VS{14}Il se hâtera de faire que celui qui aura été transporté d'un lieu à l'autre, soit mis en liberté, afin qu'il ne meure point dans la fosse, et que son pain ne lui manque pas.
\VS{15}Car je suis Yahweh, ton Dieu, qui fend la mer, et les flots rugissants. Yahweh des armées est son Nom.
\VS{16}Or je mets mes paroles dans ta bouche, et je te couvre de l'ombre de ma main, afin que j'affermisse les cieux, que je fonde la terre, et que je dise à Sion : Tu es mon peuple !
\VS{17}Réveille-toi, réveille-toi ! Lève-toi, Jérusalem, qui as bu de la main de Yahweh la coupe de sa fureur ; tu as bu, tu as sucé la lie de la coupe d'étourdissement\FTNT{Ps. 60:5 ; Ap. 14:10.} !
\VS{18}Il n'y a pas un de tous les enfants qu'elle a enfantés qui te conduise, et de tous les enfants qu'elle a nourris, il n'y en a pas un qui la prenne par la main.
\VS{19}Ces deux choses te sont arrivées ; qui te plaindra ? Le ravage et la ruine, la famine et l'épée ; par qui te consolerai-je ?
\VS{20}Tes enfants en défaillance gisaient aux carrefours de toutes les rues, comme un bœuf sauvage pris dans les filets, pleins de la fureur de Yahweh, de la répréhension de ton Dieu.
\VS{21}C'est pourquoi, écoute maintenant ceci, ô affligée, ivre, mais non pas de vin.
\VS{22}Ainsi parle Yahweh, ton Seigneur et ton Dieu, qui plaide la cause de son peuple : Voici, je prends de la main la coupe d'étourdissement, la lie de la coupe de ma fureur, tu n'en boiras plus désormais !
\VS{23}Car je la mettrai dans la main de ceux qui t'ont affligée, et qui disaient à ton âme : Courbe-toi, et nous passerons ! C'est pourquoi tu as exposé ton corps  comme la terre, comme une rue pour les passants.
\Chap{52}
\TextTitle{Le réveil de Jérusalem, la ville sainte}
\VerseOne{}Réveille-toi, réveille-toi, Sion ! Revêts-toi de ta force ! Jérusalem, ville sainte ! Revêts-toi de tes vêtements magnifiques ! Car l'incirconcis et le souillé ne passeront plus désormais parmi toi. 
\VS{2}Jérusalem, secoue ta poussière, lève-toi, et assieds-toi ! Détache les liens de ton cou, captive, fille de Sion !
\VS{3}Car ainsi parle Yahweh : Vous avez été vendus pour rien, et vous serez aussi rachetés sans argent.
\VS{4}Car ainsi parle le Seigneur, Yahweh : Mon peuple descendit jadis en Egypte pour y séjourner ; mais les Assyriens l'opprimèrent sans cause.
\VS{5}Et maintenant, qu'ai-je à faire ici, dit Yahweh, quand mon peuple a été enlevé pour rien ? Ceux qui dominent sur lui le font hurler, dit Yahweh, et mon Nom est blasphémé continuellement chaque jour.
\VS{6}C'est pourquoi mon peuple connaîtra mon Nom ; c'est pourquoi il saura, en ce jour-là, que JE SUIS parle : Voici JE SUIS !
\VS{7}Combien sont beaux sur les montagnes les pieds de celui qui apporte de bonnes nouvelles, qui publie la paix\FTNT{Na. 2:1 ; Ro. 10:15.}, qui apporte de bonnes nouvelles concernant le bien, qui publie le salut, qui dit à Sion : Ton Dieu règne !
\VS{8}Tes sentinelles élèvent leurs voix, elles se réjouissent ensemble avec chants de triomphe ; car de leurs propres yeux elles voient comment Yahweh ramène Sion.
\VS{9}Déserts de Jérusalem, éclatez, réjouissez-vous ensemble avec chants de triomphe ! Car Yahweh console son peuple, il rachète Jérusalem.
\VS{10}Yahweh manifeste le bras de sa sainteté aux yeux de toutes les nations\FTNT{Es. 53:1.}, et toutes les extrémités de la terre verront le salut\FTNT{Toutes les extrémités de la terre verront le salut de Yahweh, c'est-à-dire Jésus (Mt. 28:18-20). } de notre Dieu.
\VS{11}Retirez-vous, retirez-vous, sortez de là ! Ne touchez rien d'impur ! Sortez du milieu d'elle\FTNT{Jé. 51:45 ; 2 Co. 6:17 ; Ap. 18:4.} ! Nettoyez-vous, vous qui portez les vases de Yahweh.
\VS{12}Car vous ne sortirez pas en hâte, et vous ne marcherez pas en fuyant, car Yahweh ira devant vous, et le Dieu d'Israël sera votre arrière-garde.
\TextTitle{Le serviteur de Yahweh}
\VS{13}Voici, mon serviteur prospérera, il sera fort exalté, élevé et glorifié.
\VS{14}Comme plusieurs ont été étonnés en te voyant, son visage était défiguré plus que celui d'aucun homme, et son apparence plus que celle d'aucun fils d'homme ;
\VS{15}ainsi, il aspergera plusieurs nations, et les rois fermeront la bouche sur lui ; car ceux auxquels on n'en avait point parlé le verront ; et ceux qui ne l'avait point entendu l'entendront.
\Chap{53}
\TextTitle{Le sacrifice du Messie, serviteur de Yahweh}
\VerseOne{}Qui a cru à notre prédication ? Et à qui le bras de Yahweh\FTNT{Jésus-Christ homme est le bras de Yahweh. Le bras de Yahweh est le symbole de la puissance divine. Cette puissance s'est manifestée dans l'œuvre du Messie accomplissant le salut du monde. Le prophète est transporté au moment où le peuple juif, après avoir rejeté son Messie, ouvrira enfin les yeux et acceptera celui qu'il a percé (Za. 12:10 ; Ap. 1:7). Voir aussi Jé. 27:4-5 ; Jé. 32:17.} a-t-il été révélé ?
\VS{2}Toutefois il s'est élevé devant lui comme une jeune plante, comme un rejeton qui sort d'une terre desséchée ; il n'y avait en lui ni beauté, ni splendeur, quand nous le regardions, ni apparence qui nous le fasse désirer.
\VS{3}Il était le méprisé et le rejeté des hommes\FTNT{Ps. 22:6-7 ; Mt. 27:27-31 ; Mc. 9:12 ; Jn. 16:32.}, homme de douleur, et sachant ce que c'est que la maladie ; et nous avons comme caché notre visage arrière de lui, tant il était méprisé ; et nous ne l'avons pas estimé.
\VS{4}En vérité, il a porté nos maladies, et il s'est chargé de nos douleurs\FTNT{Mt. 8:17 ; 1 Pi. 2:24.} ; et nous l'avons considéré comme frappé, battu par Dieu et humilié.
\VS{5}Mais il était transpercé pour nos péchés, brisé pour nos iniquités, le châtiment qui nous apporte la paix est tombé sur lui, et c'est par ses meurtrissures que nous avons la guérison.
\VS{6}Nous avons tous été errants\FTNT{Pierre, apôtre de l'Agneau, confirme que le Messie est bel et bien le Bon Berger (1 Pi. 2:25).} comme des brebis, nous nous sommes détournés, chacun suivait son propre chemin, et Yahweh a fait venir sur lui l'iniquité de nous tous.
\VS{7}Opprimé et humilié, il n'a point ouvert sa bouche\FTNT{Mt. 26:62-63 ; Mc. 15:3-5 ; Jn. 19:9 ; Ac. 8:32-33.}, semblable à un agneau qu'on mène à la boucherie, à une brebis muette devant celui qui la tond, et il n'a point ouvert sa bouche.
\VS{8}Il a été enlevé de la force de l'angoisse et de la condamnation ; mais qui racontera sa durée ? Car il a été retranché de la terre des vivants, et la plaie lui a été faite pour les péchés de mon peuple.
\VS{9}On a mis son sépulcre parmi les méchants, et dans sa mort, il a été avec le riche, quoiqu'il n'ait point commis de violence, et qu'il n'y ait point eu de fraude dans sa bouche\FTNT{Mc. 15:28 ; Lu. 23:32-33.}.
\VS{10}Toutefois il a plu à Yahweh de le briser ; il l'a mis dans la souffrance. Après avoir mis son âme en sacrifice pour le péché, il verra une postérité et prolongera ses jours ; et le bon plaisir de Yahweh prospérera en sa main\FTNT{Jé. 23:5.}.
\VS{11}Il jouira du travail de son âme et en sera rassasié ; mon serviteur juste justifiera beaucoup d'hommes par la connaissance qu'ils auront de lui ; et lui-même portera leurs iniquités.
\VS{12}C'est pourquoi je lui donnerai sa part parmi les grands ; il partagera le butin avec les puissants, parce qu'il a livré son âme à la mort, qu'il a été mis au rang des transgresseurs, et que lui-même a porté les péchés de plusieurs, et qu'il a intercédé pour les transgresseurs.
\Chap{54}
\TextTitle{Yahweh réhabilite Israël la délaissée}
\VerseOne{}Réjouis-toi avec chants de triomphe, stérile, toi qui n'enfantes point, toi qui n'a pas connu les douleurs de l'accouchement! Eclate de joie avec chant de triomphe et réjouis-toi  ! Car les enfants de la délaissée seront plus nombreux que les enfants de celle qui est mariée, dit Yahweh.
\VS{2}Elargis l'espace de ta tente, et qu'on étende les couvertures de ton tabernacle : Ne retiens rien ! Allonge tes cordages et affermis tes pieux !
\VS{3}Car tu te répandras à droite et à gauche, et ta postérité possédera les nations et peuplera les villes désertes.
\VS{4}Ne crains pas, car tu ne seras point honteuse, ni confuse, et tu ne rougiras pas ; mais tu oublieras la honte de ta jeunesse, et tu ne te souviendras plus de l'opprobre de ton veuvage.
\VS{5}Car ton Créateur est ton époux : Yahweh des armées est son Nom ; et ton Rédempteur est le Saint d'Israël : Il sera appelé le Dieu de toute la terre.
\VS{6}Car Yahweh t'appelle comme une femme délaissée et à l'esprit affligé, comme une femme qu'on a épousée dans la jeunesse, et qui a été répudiée, dit ton Dieu.
\VS{7}Je t'avais délaissée pour un petit moment, mais je te rassemblerai avec de grandes compassions.
\VS{8}Dans une courte colère, je t'avais un moment caché ma face, mais j'aurai compassion de toi avec une bonté éternelle, dit Yahweh, ton Rédempteur.
\VS{9}Car il en sera pour moi comme les eaux de Noé : De même que j'avais juré que les eaux de Noé ne se répandraient plus sur la terre\FTNT{Ge. 9:11 ; Ge. 8:21.} ; je jure de ne plus m'irriter contre toi, et de ne plus te menacer.
\VS{10}Car quand les montagnes s'en iraient, quand les collines chancelleraient, ma bonté ne s'en ira point de toi, et mon alliance de paix ne chancellera point, dit Yahweh, qui a compassion de toi.
\VS{11}Ô affligée, agitée de la tempête, dénuée de consolation, voici, je coucherai tes pierres d'antimoine, et je te fonderai sur des saphirs ;
\VS{12}et je ferai tes fenêtrages d'agates, et tes portes de rubis, et toute ton enceinte de pierres précieuses.
\VS{13}Aussi tous tes enfants seront enseignés de Yahweh, et grande sera la paix de tes fils.
\VS{14}Tu seras établie en justice, tu seras loin de l'oppression, et tu ne craindras rien ; tu seras, dis-je, loin de la frayeur, car elle n'approchera pas de toi.
\VS{15}Voici, on ne manquera pas de comploter contre toi, cela ne viendra pas de moi ; quiconque complotera contre toi tombera pour l'amour de toi\FTNT{Ps. 91:7 ; Ge. 37.}.
\VS{16}Voici, c'est moi qui ai créé le forgeron soufflant le charbon au feu, et formant un instrument pour son travail, et j'ai créé aussi le destructeur pour détruire.
\VS{17}Aucune arme forgée contre toi ne réussira, et toute langue qui se lèvera en jugement contre toi, tu la condamneras\FTNT{Ps. 23:4.}. Tel est l'héritage des serviteurs de Yahweh, et telle est la justice qui leur viendra de moi, dit Yahweh.
\Chap{55}
\TextTitle{Le salut gratuit par la grâce de Dieu}
\VerseOne{}Vous tous qui avez soif, venez aux eaux, et vous qui n'avez pas d'argent, venez, achetez et mangez ; venez, dis-je, achetez du vin et du lait sans argent, et sans rien payer !
\VS{2}Pourquoi dépensez-vous de l'argent pour ce qui ne nourrit pas ? Pourquoi travaillez-vous pour ce qui ne rassasie pas\FTNT{Ro. 14:17.} ? Ecoutez-moi attentivement, et vous mangerez de ce qui est bon, et votre âme se délectera de la graisse.
\VS{3}Inclinez l'oreille, et venez à moi\FTNT{Mt. 11:28.}, écoutez, et votre âme vivra ; et je traiterai avec vous une alliance éternelle, les miséricordes immuables promises à David.
\VS{4}Voici, je l'ai donné comme témoin auprès des peuples, comme chef et dominateur des peuples.
\VS{5}Voici, tu appelleras des nations que tu ne connais pas, et les nations qui ne te connaissent pas accourront vers toi, à cause de Yahweh, ton Dieu, et du Saint d'Israël, qui t'auras glorifié.
\VS{6}Cherchez Yahweh pendant qu'il se trouve, invoquez-le tandis qu'il est près.
\VS{7}Que le méchant abandonne sa voie, et l'homme injuste ses pensées ; et qu'il retourne à Yahweh, qui aura pitié de lui, et à notre Dieu qui pardonne abondamment\FTNT{Jé. 18:11 ; Ez. 33:11 ; Jon. 3:10 ; 1 Ti. 2:1-4 ; 2 Pi. 3:9.}.
\VS{8}Car mes pensées ne sont pas vos pensées, et mes voies ne sont pas vos voies, dit Yahweh.
\VS{9}Mais autant les cieux sont élevés au-dessus de la terre, autant mes voies sont élevées au-dessus de vos voies, et mes pensées au-dessus de vos pensées.
\VS{10}Car comme la pluie et la neige descendent des cieux et n'y retournent plus, mais arrosent la terre, et la font produire et germer, afin de donner de la semence au semeur, et du pain à celui qui mange,
\VS{11}ainsi en est-il de ma parole qui sort de ma bouche, elle ne retourne point vers moi sans effet, mais elle fait tout ce en quoi je prends plaisir, et prospérera dans l'œuvre pour laquelle je l'ai envoyée.
\VS{12}Car vous sortirez avec joie, et vous serez conduits en paix ; les montagnes et les collines éclateront de joie avec chants de triomphe devant vous, et tous les arbres des champs battront des mains.
\VS{13}Au lieu de l'épine s'élèvera le cyprès, au lieu de la ronce croîtra le myrte ; et ceci fera connaître le nom de Yahweh, et ce sera un signe perpétuel, qui ne sera jamais retranché.
\Chap{56}
\TextTitle{Exhortation à s'attacher à Yahweh}
\VerseOne{}Ainsi parle Yahweh : Observez le jugement, faites ce qui est juste, car mon salut ne tardera pas à venir, et ma justice à être révélée.
\VS{2}Bienheureux l'homme qui fait cela, et le fils de l'homme qui s'y tient, observant le sabbat pour ne pas le profaner, et gardant ses mains pour ne faire aucun mal.
\VS{3}Et que l'enfant de l'étranger qui se joint à Yahweh ne parle pas en disant : Yahweh me séparera entièrement de son peuple ! Et que l'eunuque ne dise pas : Voici, je suis un arbre sec.
\VS{4}Car ainsi parle Yahweh touchant les eunuques : Ceux qui garderont mes sabbats, et qui choisiront ce en quoi je prends plaisir, et qui tiendront dans mon alliance,
\VS{5}je leur donnerai dans ma maison et dans mes murailles une place et un nom meilleur que le nom de fils ou de filles ; je leur donnerai à chacun un nom éternel qui ne périra jamais\FTNT{Ap. 2:17.}.
\VS{6}Et les enfants des étrangers qui se joindront à Yahweh pour le servir, pour aimer le Nom de Yahweh, pour être ses serviteurs, savoir tous ceux qui garderont le sabbat pour ne pas le profaner et qui tiendront dans mon alliance\FTNT{Ex. 31:14.},
\VS{7}je les amènerai sur ma montagne sainte, et je les réjouirai dans ma maison de prière ; leurs holocaustes et leurs sacrifices seront agréés sur mon autel, car ma maison sera appelée une maison de prière\FTNT{Mt. 21:13 ; Mc. 11:17 ; Lu. 19:46.} pour tous les peuples.
\VS{8}Le Seigneur, Yahweh, parle, lui qui rassemble les exilés d'Israël : Je réunirai d'autres peuples à lui, outre ceux déjà rassemblés.
\VS{9}Bêtes des champs, bêtes des forêts, venez toutes pour manger !
\VS{10}Toutes ses sentinelles sont aveugles, elles ne connaissent rien ; ce sont tous des chiens muets, qui ne peuvent aboyer, dormant et demeurant couchés, et aimant à sommeiller.
\VS{11}Ce sont des chiens voraces et insatiables ; ce sont des pasteurs qui ne savent rien comprendre ; tous suivent leur propre voie, chacun à son gain injuste dans son quartier, en disant\FTNT{Mt. 23:24 ; Tit. 1:7-11 ; 1 Pi. 5:2.} :
\VS{12}Venez, je vais chercher du vin, et nous nous enivrerons de boissons fortes ! Nous en ferons autant demain, et même beaucoup plus encore !
\Chap{57}
\TextTitle{Yahweh expose la fausseté et défend le juste}
\VerseOne{}Le juste périt, et nul ne le prend à cœur ; et les gens de bien sont recueillis, sans qu'on y soit attentif, sans qu'on considère que le juste a été recueilli devant le mal\FTNT{Mi. 7:2 ; Ec. 7:15.}.
\VS{2}Il entrera en paix, il reposera sur sa couche, celui qui aura marché dans la droiture\FTNT{Mt. 25:23 ; Lu. 19:17.}.
\VS{3}Mais vous, approchez ici, enfants de l'enchanteresse, race de l'adultère et de la prostituée !
\VS{4}De qui vous êtes-vous moqués ? Contre qui avez-vous ouvert la bouche et tirez-vous la langue ? N'êtes-vous pas des enfants de rébellion, une race de mensonge ?
\VS{5}S'échauffant près des faux dieux, sous tout arbre vert ; égorgeant les enfants dans les vallées, sous les fentes des rochers\FTNT{Lé. 18:21 ; 1 R. 14:23 ; Jé. 2:20 ; Jé. 32:35.}.
\VS{6}Parmi les pierres polies des torrents est ta portion, ce sont elles, ce sont elles qui sont ton lot ; tu leur a aussi répandu ton aspersion, tu leur as aussi offert des offrandes ; puis-je être content de ces choses ?
\VS{7}Tu dresses ta couche sur les montagnes hautes et élevées ; c'est aussi là que tu montes pour offrir des sacrifices.
\VS{8}Et tu mets ton souvenir derrière la porte et les poteaux ; car tu te découvres loin de moi et tu montes, tu élargis ta couche, et tu te l'est taillé plus grande que n'ont fait ceux-là ; tu as aimé leur couche, tu as pris garde aux belles places.
\VS{9}Tu voyages vers le roi avec de l'huile précieuse, et tu ajoutes parfums sur parfums ; tu envoies au loin tes ambassades, tu t'abaisses jusqu'au scheol.
\VS{10}Tu te fatigues par la longueur du chemin, et tu ne dis pas : C'est sans espoir ! Tu trouves encore de la vigueur dans ta main ; c'est pourquoi tu n'as pas été languissante.
\VS{11}Et qui redoutais-tu, qui craignais-tu pour que tu me mentes, pour ne pas te souvenir et te soucier de moi ? N'ai-je pas gardé le silence, et même depuis longtemps, et tu ne me crains pas.
\VS{12}Je vais déclarer ta justice et tes œuvres, qui ne te profiteront pas.
\VS{13}Quand tu crieras, que ceux que tu assembles te délivrent ! Mais le vent les emmènera tous, la vanité les enlèvera ; mais celui qui met sa confiance en moi, héritera la terre et possédera ma montagne sainte\FTNT{Es. 2:3 ; Ps. 2:6 ;  Hé. 12:22.}.
\VS{14}On dira : Frayez, frayez, préparez le chemin, enlevez tout obstacle loin du chemin de mon peuple !
\TextTitle{Yahweh aime l'homme contrit}
\VS{15}Car ainsi parle celui qui est haut et élevé, qui habite dans l'éternité et dont le nom est le Saint : J'habiterai dans les lieux hauts et saints, avec celui qui a le cœur brisé et qui est humble d'esprit, afin de vivifier l'esprit des humbles, et afin de vivifier ceux qui ont le cœur brisé\FTNT{Ps. 34:19 ; Ps. 51:19.}.
\VS{16}Parce que je ne veux pas contester à toujours, et que je ne serai pas irrité à jamais ;  car devant moi tombent en défaillance les esprits, et les âmes que j'ai faites\FTNT{Mi. 7:18 ; Ps. 85:6 ; Ps. 103:9.}.
\VS{17}A cause de l'iniquité de ses gains déshonnêtes, je me suis irrité et je l'ai frappé, je me suis caché ma dans ma colère ; et le rebelle a suivi la voie de son cœur.
\VS{18}J'ai vu ses voies, et toutefois je le guérirai ; je le conduirai et je le restaurerai, lui et ceux qui mènent deuil avec lui.
\VS{19}Je crée les fruits des lèvres. Paix, paix à celui qui est loin et à celui qui est près ! dit Yahweh, car je le guérirai.
\VS{20}Mais les méchants sont comme la mer agitée, quand elle ne peut se calmer, et dont les eaux rejettent la boue et le bourbier.
\VS{21}Il n'y a point de paix pour les méchants, dit mon Dieu.
\Chap{58}
\TextTitle{Le vrai et le faux jeûne}
\VerseOne{}Crie à plein gosier, ne te retiens pas, élève ta voix comme un shofar, et annonce à mon peuple ses iniquités et à la maison de Jacob ses péchés !
\VS{2}Car ils me cherchent tous les jours, ils prennent plaisir à connaître mes voies ; comme une nation qui aurait pratiqué la justice, et qui n'aurait pas abandonné les ordonnances de son Dieu ; ils me demandent des jugements justes, ils prennent plaisir à s'approcher de Dieu, et puis ils disent :
\VS{3}Pourquoi jeûnons-nous, et tu ne le vois pas ? Pourquoi affligeons-nous nos âmes, si tu n'y as point connaissance ? Voici, le jour de votre jeûne, vous trouvez votre plaisir, et vous oppressez tous vos travailleurs.
\VS{4}Voici, vous jeûnez pour faire des querelles et vous disputer, et pour frapper du poing méchamment ; vous ne jeûnez pas comme le veut ce jour, pour que votre voix soit exaucée d'en haut.
\VS{5}Est-ce là le jeûne que j'ai choisi, que l'homme afflige son âme un jour ? Est-ce en courbant sa tête comme le jonc et en étendant le sac et la cendre ? Appelleras-tu cela un jeûne et un jour agréable à Yahweh ?
\VS{6}N'est-ce pas plutôt ici le jeûne que j'ai choisi : Que tu détaches les liens de la méchanceté, que tu délies les cordages du joug, que tu laisses aller libres les opprimés, et que l'on rompe toute espèce de joug ?
\VS{7}N'est ce-pas que tu partages ton pain avec celui qui a faim ? Et que tu fasses venir dans ta maison les affligés errants ? Quand tu vois un homme nu, que tu le couvres, et que tu ne te caches pas de ta propre chair ?
\TextTitle{Bénédiction pour ceux qui pratiquent le bien}
\VS{8}Alors ta lumière éclatera comme l'aurore, et ta guérison germera rapidement ; ta justice ira devant toi, et la gloire de Yahweh sera ton arrière-garde.
\VS{9}Alors tu prieras, et Yahweh t'exaucera ; tu crieras, et il dira : Me voici ! Si tu ôtes du milieu de toi le joug, si tu cesses de lever le doigt et de dire des outrages ;
\VS{10}si tu ouvres ton âme à celui qui a faim, si tu rassasies l'âme affligée ; ta lumière se lèvera sur les ténèbres, et l'obscurité sera comme le midi.
\VS{11}Et Yahweh te conduira continuellement, il rassasiera ton âme dans les grandes sécheresses, il fortifiera tes os, et tu seras comme un jardin arrosé, et comme une source dont les eaux ne tarissent pas\FTNT{Jn. 4:14 ; Ap. 21:6.}.
\VS{12}Et ceux qui sortiront de toi rebâtiront les lieux déserts depuis longtemps, tu rétabliras les fondements ruinés depuis plusieurs générations ; et on t'appellera le réparateur des brèches et le restaurateur des chemins, afin qu'on habite au pays.
\VS{13}Si tu détournes ton pied pendant le sabbat pour ne pas faire ta volonté en mon saint jour ; si tu appelles le sabbat tes délices, et honorable ce qui est saint à Yahweh, et si tu l'honores en ne suivant point tes voies, en ne te livrant pas à tes désirs et à des vains discours,
\VS{14}alors tu prendras plaisir en Yahweh, et je te ferai monter comme à cheval par-dessus les lieux haut élevés de la terre, et je te donnerai à manger de l'héritage de Jacob, ton père ; car la bouche de Yahweh a parlé.
\Chap{59}
\TextTitle{Le péché sépare de Yahweh}
\VerseOne{}Voici, la main de Yahweh n'est pas trop courte pour pouvoir sauver, ni son oreille trop pesante pour pouvoir entendre.
\VS{2}Mais ce sont vos iniquités qui mettent une séparation entre vous et votre Dieu ; ce sont vos péchés qui vous cachent sa face, afin qu'il ne vous entende point\FTNT{De. 31:17-18 ; Ez. 39:23-24.}.
\VS{3}Car vos mains sont souillées de sang, et vos doigts d'iniquité ; vos lèvres profèrent le mensonge, et votre langue déclare la perversité.
\VS{4}Nul ne crie pour la justice, nul ne plaide pour la vérité ; ils s'appuient sur des choses vaines et disent des faussetés, ils conçoivent le mal et enfantent l'iniquité.
\VS{5}Ils font éclore des œufs de vipère, et ils tissent des toiles d'araignée ; celui qui mange de leurs œufs meurt ; et si on les écrase, il en sort une vipère.
\VS{6}Leurs toiles ne servent point à faire des vêtements, et on ne se couvre pas de leurs ouvrages ; car leurs ouvrages sont des ouvrages d'iniquité, et il y a en leurs mains des actions de violence.
\VS{7}Leurs pieds courent au mal, et se hâtent pour répandre le sang innocent ; leurs pensées sont des pensées d'iniquité ; le ravage et la ruine sont sur leurs voies.
\VS{8}Ils ne connaissent point le chemin de la paix, et il n'y a point de jugement dans leurs voies, ils se sont pervertis dans leurs sentiers, tous ceux qui y marchent ignorent la paix\FTNT{Pr. 1:16 ; Pr. 6:16-19.}.
\VS{9}C'est pourquoi le jugement s'est éloigné de nous, et la justice ne parvient pas jusqu'à nous ; nous attendions la lumière, et voici les ténèbres, la clarté, et nous marchons dans l'obscurité.
\VS{10}Nous tâtonnons comme des aveugles le long du mur, nous tâtonnons comme ceux qui sont sans yeux ; nous chancelons en plein midi comme la nuit, et nous sommes dans les lieux abondants comme y sont des morts.
\VS{11}Nous rugissons tous comme des ours, et nous ne cessons de gémir comme des colombes ; nous attendons le jugement, et il n'y en a point, la délivrance, et elle est éloignée de nous.
\VS{12}Car nos transgressions se sont multipliées devant toi, et chacun de nos péchés témoignent contre nous ; parce que nos transgressions sont avec nous, et nous connaissons nos iniquités ;
\VS{13}qui sont de pécher et de mentir contre Yahweh, de s'éloigner de notre Dieu, de proférer l'oppression et la révolte, de concevoir et prononcer du cœur des paroles de mensonge.
\VS{14}C'est pourquoi le jugement s'est éloigné et la justice se tient éloignée ; car la vérité est tombée par les rues, et la droiture ne peut y entrer.
\VS{15}Même la vérité a disparu, et quiconque se retire du mal est exposé au pillage ; Yahweh voit, et cela lui a déplu, parce qu'il n'y a plus de droiture.
\TextTitle{Yahweh cherche un homme, il suscite le Messie}
\VS{16}Il voit aussi qu'il n'y a aucun homme, il s'étonne que personne ne se tienne à la brèche ; c'est pourquoi son bras lui vient en aide, et sa propre justice lui sert d'appui\FTNT{Es. 53:1 ; Es. 63:5 ; Ps. 77:15-16 ; Ac. 13:17.}.
\VS{17}Car il se revêt de la justice comme d'une cuirasse, et le casque du salut est sur sa tête\FTNT{Ep. 6:14-17.} ; il se revêt de la vengeance comme d'un vêtement, et se couvre de la jalousie comme d'un manteau.
\VS{18}Selon leurs actes, il rendra à chacun la pareille\FTNT{Jé. 17:10 ; Job. 34:11 ; Mt. 16:27 ; Ap. 2:23 ; Ap. 20:13.}, la fureur à ses adversaires, la rétribution à ses ennemis ; il rendra ainsi la rétribution aux îles.
\VS{19}Et on craindra le Nom de Yahweh depuis l'occident, et sa gloire depuis le soleil levant ; car l'ennemi viendra comme un fleuve, mais l'Esprit de Yahweh lèvera la bannière\FTNT{En hébreu « Yahweh Nissi », c'est-à-dire « Yahweh est ma bannière ». C'est le nom donné par Moïse à l'autel qu'il construisit pour célébrer la défaite d'Amalek (Ex. 17:15). En No. 21:8-9, Moïse éleva une bannière sur laquelle il avait fixé un serpent d'airain pour la guérison des malades. } contre lui.
\VS{20}Et le Rédempteur\FTNT{Le Rédempteur qui viendra pour Sion est le Seigneur Jésus-Christ (Ro. 11:26). Voir aussi Es. 60 : 16. } viendra en Sion, et vers ceux de Jacob qui se convertiront de leur péché, dit Yahweh.
\VS{21}Et quant à moi, c'est ici mon alliance que je ferai avec eux, dit Yahweh : Mon Esprit qui est sur toi, et mes paroles que j'ai mises dans ta bouche, ne se retireront point de ta bouche, ni de la bouche de ta postérité, ni de la bouche de la postérité de ta postérité, dit Yahweh, dès maintenant et à jamais.
\Chap{60}
\TextTitle{La gloire de Yahweh se lèvera sa gloire sur Sion}
\VerseOne{}Lève-toi, sois illuminée, car ta lumière arrive, et la gloire de Yahweh se lève sur toi.
\VS{2}Car voici, les ténèbres couvrent la terre, et l'obscurité couvre les peuples ; mais Yahweh se lève sur toi, et sa gloire apparaît sur toi.
\VS{3}Des nations marchent à ta lumière, et des rois à la splendeur qui se lève sur toi\FTNT{Ap. 21:24.}.
\VS{4}Elève tes yeux alentour, et regarde : Tous ceux-ci s'assemblent, ils viennent vers toi ; tes fils viennent de loin, et tes filles sont nourries par des nourriciers, étant portées sur les côtés.
\VS{5}Alors tu verras et tu seras éclairée, et ton cœur s'étonnera et s'épanouira de joie, quand l'abondance de la mer se sera tournée vers toi, et que la puissance des nations sera venue chez toi.
\VS{6}Tu seras couverte d'une foule de chameaux, des dromadaires de Madian et d'Epha ; et tous ceux de Séba viendront, ils apporteront de l'or et de l'encens, et publieront les louanges de Yahweh.
\VS{7}Toutes les brebis de Kédar seront assemblées vers toi, les béliers de Nebajoth seront à ton service ; ils seront agréables étant offerts sur mon autel, et je rendrai magnifique la maison de ma gloire.
\VS{8}Qui sont ceux-là qui volent comme des nuées, comme des colombes vers leur colombier ?
\VS{9}Car les îles s'attendent à moi, et les navires de Tarsis les premiers, afin d'amener de loin tes enfants, avec leur argent et leur or, à cause du Nom de Yahweh, ton Dieu, et du Saint d'Israël qui te glorifie.
\VS{10}Les fils des étrangers rebâtiront tes murailles, et leurs rois seront employés à ton service ; car je t'ai frappée dans ma colère, mais j'ai eu pitié de toi au temps de mon bon plaisir.
\VS{11}Tes portes seront continuellement ouvertes, elles ne seront fermées ni nuit ni jour, afin que les forces des nations te soient amenées et que leur roi y soient conduits\FTNT{Ap. 21:25-26.}.
\VS{12}Car la nation et le royaume qui ne te serviront pas périront, et ces nations-là seront réduites en une entière désolation.
\VS{13}La gloire du Liban viendra vers toi, le cyprès, l'orme, et le buis, tous ensemble pour rendre honorable le lieu de mon sanctuaire ; et je rendrais glorieux le lieu de mes pieds.
\VS{14}Mais les enfants de tes oppresseurs viendront vers toi en se courbant, et tous ceux qui te méprisaient se prosterneront à tes pieds et t'appelleront la ville de Yahweh, la Sion du Saint d'Israël.
\VS{15}Au lieu d'avoir été délaissée et haïe, si bien que personne ne passait par toi, je te mettrai dans une élévation éternelle et dans une joie qui sera de génération en génération.
\VS{16}Et tu suceras le lait des nations, et tu suceras la mamelle des rois, et tu sauras que je suis Yahweh, ton Sauveur, ton Rédempteur\FTNT{Le verbe « ga'al » et le nom correspondant « go'el », ont été traduits respectivement en français par « racheter » et « rédempteur ». Selon la loi de Moïse, si quelqu'un perdait son héritage à cause d'une dette ou s'il se vendait comme esclave, lui et ses biens pouvaient être rachetés par un proche parent qui devait payer le prix de la rédemption (Lé. 25:23-55). Yahweh se présente comme le Rédempteur par excellence (Es. 49:26 ; Es. 60:16 ; Ps. 78:35 ; Ps. 130:7; Job. 19:25). Or Jésus-Christ «[…] a été fait pour nous sagesse, justice, sanctification et rédemption » (1 Co. 1:30). Les épîtres nous révèlent la rédemption qu'il a acquise pour nous : « nous avons la rédemption par son sang » (Ep. 1:7). La rédemption est le paiement d'une rançon, or il est écrit : « Jésus-Christ s'est donné en rançon pour nous tous » (1 Ti. 2:6). « Vous avez été rachetés à grand prix » (1 Co. 6:20).}, le Puissant de Jacob.
\VS{17}Je ferai venir de l'or au lieu de l'airain, et de l'argent au lieu du fer, et de l'airain au lieu du bois, et du fer au lieu des pierres ; et je ferai régner la paix et dominer la justice.
\VS{18}On n'entendra plus parler de violence dans ton pays ni de ravage et de ruine dans ton territoire ; mais tu appelleras tes murailles : Salut ; et tes portes : Louange.
\VS{19}Tu n'auras plus le soleil pour la lumière du jour, et la lueur de la lune ne t'éclairera plus, mais Yahweh sera pour toi la lumière éternelle\FTNT{Voir le commentaire en Ge 1:3.}, et ton Dieu sera ta gloire.
\VS{20}Ton soleil ne se couchera plus, et ta lune ne se retirera plus, car Yahweh te sera pour lumière perpétuelle, et les jours de ton deuil seront finis.
\VS{21}Quant à ton peuple, ils seront tous justes, ils posséderont la terre à toujours ; savoir le germe de mes plantes, l'œuvre de mes mains pour y être glorifié\FTNT{Es. 11:1 ; Ro. 15:12 ; Ap. 5:5 ; Ap. 22:16.}.
\VS{22}La petite famille deviendra un millier de personnes, et la moindre deviendra une nation puissante. Je suis Yahweh, je hâterai ces choses en leur temps.
\Chap{61}
\TextTitle{La mission du Messie}
\VerseOne{}L'Esprit du Seigneur Yahweh est sur moi, car Yahweh m'a oint pour évangéliser les malheureux ; il m'a envoyé pour guérir ceux qui ont le cœur brisé, pour proclamer aux captifs la liberté, et aux prisonniers l'ouverture de la prison ;
\VS{2}pour publier une année de grâce de Yahweh, et le jour de vengeance de notre Dieu ; pour consoler tous ceux qui mènent deuil\FTNT{Lu. 4:14-19.} ;
\VS{3}pour annoncer à ceux de Sion qui mènent deuil, que la magnificence leur sera donnée au lieu de la cendre, une huile de joie au lieu du deuil, un manteau de louange au lieu d'un esprit abattu\FTNT{Job. 29:14 ; Ja. 1:12 ; 1 Co.9:25 ; 2 Ti. 4:8.}, afin qu'on les appelle des térébinthes de la justice, une plantation de Yahweh, pour servir à sa gloire.
\VS{4}Et ils rebâtiront les ruines antiques, ils relèveront les lieux qui étaient auparavant désolés, et ils renouvelleront des villes ravagées, et les choses désolées d'âge en âge.
\VS{5}Et des étrangers s'y tiendront là et feront paître vos troupeaux, et les enfants de l'étranger seront vos laboureurs et vos vignerons.
\VS{6}Mais vous, vous serez appelés sacrificateurs de Yahweh, et on vous nommera serviteurs de notre Dieu\FTNT{Ap. 1:6 ; Ap. 5:10.} ; vous mangerez les richesses des nations, et vous vous glorifierez de leur gloire.
\VS{7}Au lieu de la honte que vous avez eue, les nations en auront le double, et elles crieront tout haut que la confusion est leur portion ; c'est pourquoi ils posséderont le double dans leur pays, et leur joie sera éternelle.
\VS{8}Car je suis Yahweh qui aime le jugement et qui hait la rapine pour l'holocauste ; j'établirai leur œuvre dans la vérité et je traiterai avec eux une alliance éternelle.
\VS{9}Et leur race sera connue parmi les nations, et ceux qui seront sortis d'eux seront connus parmi les peuples ; tous ceux qui les verront connaîtront qu'ils sont la race que Yahweh aura bénie.
\VS{10}Je me réjouirai extrêmement en Yahweh, et mon âme se réjouira en mon Dieu ; car il m'a revêtu des vêtements du salut, il m'a couvert du manteau de la justice, comme un époux qui se pare de magnificence, et comme une épouse qui s'orne de ses joyaux\FTNT{Os. 2:21-22 ; Ap. 19:7-8.}.
\VS{11}Car comme la terre fait éclore son germe, et comme un jardin fait germer ses semences, ainsi le Seigneur Yahweh fera germer la justice, et la louange en présence de toutes les nations.
\Chap{62}
\TextTitle{Yahweh proclamme la restauration d'Israël}
\VerseOne{}Pour l'amour de Sion, je ne me tiendrai pas tranquille, et pour l'amour de Jérusalem je ne prendrai point de repos, jusqu'à ce que sa justice sorte dehors comme une splendeur, et que sa délivrance ne soit allumée comme une lampe.
\VS{2}Alors les nations verront ta justice, et tous les rois ta gloire ; et on t'appellera d'un nouveau nom\FTNT{Ap. 2:17.}, que la bouche de Yahweh aura expressément déclaré.
\VS{3}Tu seras une couronne de gloire dans la main de Yahweh, un turban royal dans la main de ton Dieu.
\VS{4}On ne te nommera plus la délaissée, et on ne nommera plus ta terre la désolation ; mais on t'appellera mon bon plaisir en elle ; et on appellera ta terre l'épouse ; car Yahweh prend son bon plaisir en toi, et ta terre aura un époux.
\VS{5}Car comme le jeune homme épouse la vierge, comme tes enfants se marient chez toi, ainsi ton Dieu se réjouira en toi, de la joie qu'un époux a de son épouse.
\VS{6}Jérusalem, j'ai placé des gardes sur tes murailles tout le jour et toute la nuit, et ils ne se tairont point. Vous qui faites mention de Yahweh, ne gardez point le silence !
\VS{7}Et ne vous arrêtez pas de l'invoquer jusqu'à ce qu'il rétablisse Jérusalem et lui rende sa renommée sur la terre.
\VS{8}Yahweh l'a juré par sa droite et par son bras puissant : Je ne donnerai plus ton froment pour nourriture à tes ennemis, et les enfants des étrangers ne boiront plus ton vin excellent pour lequel tu as travaillé.
\VS{9}Mais ceux qui auront amassé le froment le mangeront et loueront Yahweh, et ceux qui auront récolté le vin le boiront dans les parvis de ma sainteté.
\VS{10}Passez, passez les portes ! Disant : Préparez le chemin du peuple ! Frayez, frayez la route, et ôtez-en les pierres ! Elevez une bannière vers les peuples.
\VS{11}Voici ce que Yahweh proclame aux extrémités de la terre : Dites à la fille de Sion : Voici, ton Sauveur vient\FTNT{De nombreux passages, notamment dans le livre d'Esaïe, présentent Dieu comme le sauveur, le seul sauveur (Es. 43:3 ; Es. 43:11 ; Os. 13:4) qui viendra pour délivrer son peuple (Es. 35:4 ; Es. 60:1 ; Za. 14:1-7). Jésus-Christ a accompli en tous points les prophéties relatives à la venue de Yahweh. Dieu est bel et bien venu sur terre il y a plus de 2000 ans et ce même Dieu revient bientôt (Ac. 1:11 ; Ap. 1:7).} ; voici, son salaire est avec lui, et sa récompense marche devant lui.
\VS{12}Et on les appellera le peuple saint, les rachetés de Yahweh\FTNT{1 Pi. 2:9 ; Ap. 5:9.} ; et toi, on t'appellera la recherchée, la ville non abandonnée.
\Chap{63}
\TextTitle{Le jour de vengeance du Messie\FTNTT{Es. 2:10-22 ; Ap. 19:11-21.}}
\VerseOne{}Qui est celui-ci qui vient d'Edom, de Botsra, en habits rouges, magnifiquement paré en son vêtement, marchant selon la grandeur de sa force ? C'est moi qui parle en justice et qui ai tout pouvoir de sauver.
\VS{2}Pourquoi tes vêtements sont-ils rouges, et pourquoi tes habits sont comme les habits de ceux qui foulent dans la cuve ?
\VS{3}J'ai été seul à fouler au pressoir, et nul homme d'entre les peuples n'était avec moi. Cependant, j'ai marché sur eux dans ma colère, et je les ai foulés dans ma fureur ; et leur sang a rejailli sur mes vêtements, et j'ai souillé tous mes habits.
\VS{4}Car le jour de la vengeance était dans mon cœur, et l'année de mes rachetés est venue.
\VS{5}Je regardais donc, il n'y avait personne pour m'aider ; et j'étais étonné, et  il n'y avait personne pour me soutenir ; mais mon bras m'a sauvé et ma fureur m'a soutenu.
\VS{6}Ainsi j'ai foulé des peuples dans ma colère, et je les ai enivrés dans ma fureur ; et j'ai abattu leur force par terre.
\TextTitle{Esaïe confesse les péchés du peuple}
\VS{7}Je ferai mention des bontés de Yahweh, qui sont les louanges de Yahweh, pour tous les bienfaits que Yahweh nous a faits ; car grande est la bonté envers la maison d'Israël, qu'il a traitée selon ses compassions et la richesse de sa miséricorde.
\VS{8}Car il a dit : Certainement, ils sont mon peuple, des enfants qui ne tricheront pas ! Et il a été pour eux un Sauveur.
\VS{9}Et dans toutes leurs détresses, il a été en détresse, et l'ange qui est devant sa face les a délivrés\FTNT{Ge. 16:7-10 ; Jg. 6:11-14 ; Za.1:11.} ; lui-même les a rachetés dans son amour et sa miséricorde, et il les a soutenus et portés, tous les jours d'autrefois.
\VS{10}Mais ils ont été rebelles, et ils ont attristé son Esprit saint\FTNT{Ep. 4:30.}, c'est pourquoi il est devenu leur ennemi, et il a lui-même combattu contre eux.
\VS{11}Et on se souvint des anciens jours de Moïse et de son peuple. Où est celui, a-t-on dit, qui les fit monter de la mer, avec les pasteurs de son troupeau ? Où est celui qui mit au milieu d'eux son Esprit saint ;
\VS{12}qui les dirigea par la droite de Moïse et par son bras glorieux ; qui fendit les eaux devant eux pour se faire un nom éternel ?
\VS{13}Qui les dirigea à travers les flots, comme un cheval dans le désert, sans qu'ils ne bronchent ?
\VS{14}L'Esprit de Yahweh les a menés au repos comme on mène une bête qui descend dans la vallée. C'est ainsi que tu as conduit ton peuple, afin de t'acquérir un nom glorieux.
\VS{15}Regarde du ciel et vois de ta demeure sainte et glorieuse : Où sont ton zèle et ta puissance ? Le son de tes entrailles et de tes compassions se retiennent-ils envers moi ?
\VS{16}Certes tu es notre Père, encore qu'Abraham ne nous connaisse pas, et qu'Israël ne nous reconnaisse pas ; Yahweh, c'est toi qui es notre Père, et ton Nom est notre Rédempteur de tout temps.
\VS{17}Pourquoi nous as-tu fait égarer loin de tes voies, ô Yahweh, et endurcis-tu notre cœur contre ta crainte ? Reviens, pour l'amour de tes serviteurs, des tribus de ton héritage !
\VS{18}Ton peuple saint n'a possédé le pays que peu de temps ; nos ennemis ont foulé ton sanctuaire.
\TextTitle{Prière du reste d'Israël à Yahweh pour sa délivrance}
\VS{19}Nous sommes comme ceux sur lesquels tu ne domines pas depuis longtemps, et sur lesquels ton Nom n'est point réclamé. Ô ! Si tu fendais les cieux, et si tu descendais, les montagnes s'ébranleraient devant toi !
\Chap{64}
\VerseOne{}Comme un feu de fonte est ardent, le feu fait bouillir l'eau, afin de faire connaître ton Nom à tes ennemis, et que les nations tremblent en ta présence.
\VS{2}Lorsque tu fis les choses redoutables que nous n'attendions pas, tu descendis et les montagnes tremblèrent devant toi.
\VS{3}Jamais on n'a appris ni entendu dire, et jamais l'œil n'a vu qu'un autre dieu que toi fît de telles choses pour ceux qui s'attendent à lui\FTNT{1 Co. 2:9.}.
\VS{4}Tu viens à la rencontre de celui qui se réjouit et qui agit avec justice, et  se souviennent de toi dans tes voies. Voici tu as été irrité parce que nous avons péché ; tes compassions sont éternelles, c'est pourquoi nous serons sauvés.
\VS{5}Or nous sommes tous devenus comme une chose souillée, et toute notre justice est comme le linge le plus souillé\FTNT{Ap. 19:8.} ; nous sommes tous flétris comme la feuille, et nos iniquités nous emportent comme le vent.
\VS{6}Il n'y a personne qui invoque ton Nom, qui se réveille pour s'attacher fortement à toi ; c'est pourquoi tu nous as caché ta face, et tu nous fais fondre par l'effet de nos iniquités.
\VS{7}Cependant, ô Yahweh, tu es notre Père ; nous sommes l'argile, et c'est toi qui nous as formés, et nous sommes tous l'ouvrage de ta main\FTNT{Es. 29:16 ; Es. 45:9 ; Jé. 18:6 ; Ro. 9:20-21.}.
\VS{8}Ne t'irrite pas à l'extrême, ô Yahweh, et ne te souviens pas à toujours de notre iniquité. Voici, regarde, nous te prions, nous sommes tous ton peuple.
\VS{9}Tes villes saintes sont devenues un désert ; Sion est devenue un désert, et Jérusalem une désolation.
\VS{10}Notre maison sainte et glorieuse, où nos pères te louaient, a été brûlée par le feu ; tout ce que nous avions de précieux a été dévasté.
\VS{11}Après cela, ô Yahweh, ne te retiendras-tu pas ? Ne cesseras-tu pas, et nous affligeras-tu à l'excès ?
\Chap{65}
\TextTitle{Réponse de Yahweh}
\VerseOne{}Je me suis fait recherché de ceux qui ne me demandaient point, et je me suis laissé trouver par ceux qui ne me cherchaient pas\FTNT{Mt. 7:7 ; Lu. 11:9.} ; j'ai dit à la nation qui ne s'appelait pas de mon Nom : Me voici, me voici !
\VS{2}J'ai tendu mes mains tous les jours vers un peuple rebelle, à ceux qui marche dans une mauvaise voie, au gré de ses pensées ;
\VS{3}vers un peuple qui m'irrite continuellement en face, qui sacrifie dans les jardins, et qui fait des parfums sur les autels de briques,
\VS{4}qui habite les sépulcres et passe la nuit dans les lieux désolés, qui mangent la chair de porc, et ayant dans ses vases le jus des choses abominables.
\VS{5}Qui dit : Retire-toi, ne m'approche pas, car je suis plus saint que toi ! Ceux-là sont une fumée dans mes narines, un feu ardent tout le jour.
\VS{6}Voici, ceci est écrit devant moi, je ne me tairai point, mais je leur ferai porter la peine, oui je leur ferai porter la peine
\VS{7}de vos iniquités, dit Yahweh, et les iniquités de vos pères ensemble, qui ont brûlé de l'encens sur les montagnes, et qui m'ont blasphémé sur les collines ; c'est pourquoi je leur mesurerai aussi dans leur sein le salaire de ce qu'ils ont fait au commencement.
\VS{8}Ainsi parle Yahweh : Comme quand on trouve du vin dans une grappe, on dit : Ne la détruis pas, car il y a là une bénédiction ! J'agirai de même à cause de mes serviteurs, afin de ne pas tous les détruire.
\VS{9}Je ferai sortir de Jacob une postérité, et de Juda celui qui héritera de mes montagnes ; et mes élus hériteront le pays, et mes serviteurs y habiteront.
\VS{10}Et Saron servira de pâturage au menu bétail, et la vallée d'Acor sera le gîte du gros bétail, pour mon peuple qui m'aura recherché.
\VS{11}Mais vous, qui abandonnez Yahweh et qui oubliez ma montagne sainte, qui dressez la table pour Gad\FTNT{Gad : Dieu de la fortune.}, et qui remplissez une coupe pour Meni\FTNT{Meni : divinité païenne assimilée à la lune et dont le nom signifie « destin, sort ou fortune ».},
\VS{12}je vous destine aussi à l'épée, et vous serez tous courbés pour être égorgés ; parce que j'ai appelé, et vous n'avez point répondu ; j'ai parlé, et vous n'avez point écouté ; mais vous avez fait ce qui me déplaît, et vous avez choisi les choses auxquelles je ne prends pas plaisir.
\VS{13}C'est pourquoi, ainsi parle le Seigneur, Yahweh : Voici, mes serviteurs mangeront, et vous aurez faim ; voici, mes serviteurs boiront, et vous aurez soif ; voici mes serviteurs se réjouiront, et vous serez honteux.
\VS{14}Voici, mes serviteurs se réjouiront avec chants de triomphe pour la joie qu'ils auront au cœur ; mais vous, vous crierez pour la douleur que vous aurez au cœur, et vous crierez à cause de l'accablement de votre esprit.
\VS{15}Et vous laisserez votre nom à mes élus comme malédiction ; et le Seigneur Yahweh vous fera mourir ; et il donnera à ses serviteurs un autre nom.
\VS{16}Celui qui se bénira sur la terre, se bénira par le Dieu de vérité ; et celui qui jurera sur la terre jurera par le Dieu de vérité ; car les détresses du passé seront oubliées, et même elles seront cachées devant mes yeux.
\TextTitle{De nouveaux cieux et une nouvelle terre}
\VS{17}Car voici, je vais créer de nouveaux cieux et une nouvelle terre\FTNT{Es. 66:22 ; 2 Pi. 3:13 ; Ap. 21:1.} ; et on ne se souviendra plus des choses précédentes, elles ne reviendront plus au cœur.
\VS{18}Réjouissez-vous plutôt et soyez à toujours dans l'allégresse, à cause de ce que je vais créer ; car voici je vais créer Jérusalem pour n'être que joie, et son peuple pour n'être qu'allégresse.
\VS{19}Je ferai de Jérusalem mon allégresse, et de mon peuple ma joie ; on n'y entendra plus le bruit des pleurs et le bruit des clameurs.
\VS{20}Il n'y aura plus désormais ni nourrisson ni vieillard qui n'accomplissent leurs jours ; car celui qui mourra âgé de cent ans sera encore jeune ; mais le pécheur âgé de cent ans sera maudit.
\VS{21}Ils bâtiront des maisons et y habiteront ; ils planteront des vignes et ils en mangeront le fruit.
\VS{22}Ils ne bâtiront pas des maisons pour qu'un autre y habite ; ils ne planteront pas des vignes pour qu'un autre en mange le fruit ; car les jours de mon peuple seront comme les jours des arbres ; et mes élus jouiront de l'œuvre de leurs mains.
\VS{23}Ils ne travailleront plus en vain, et ils n'engendreront plus des enfants pour être exposés à la frayeur ; car ils seront la postérité des bénis de Yahweh, et ceux qui sortiront d'eux seront avec eux.
\VS{24}Et il arrivera qu'avant qu'ils crient, je les exaucerai ; et lorsqu'encore ils parleront, je les aurai déjà entendus.
\VS{25}Le loup et l'agneau paîtront ensemble, le lion comme le bœuf mangeront de la paille, et la poussière sera la nourriture du serpent\FTNT{Es. 2:4 ; Es. 11:6-7.}. On ne nuira point et on ne fera aucun dommage sur toute ma montagne sainte, dit Yahweh.
\Chap{66}
\TextTitle{Yahweh réprouve l'hypocrisie et agrée ceux qui le craignent}
\VerseOne{}Ainsi parle Yahweh : Le ciel est mon trône, et la terre est le marchepied de mes pieds\FTNT{Mt. 5:34-35 ; Ac. 7:49.}. Quelle maison me bâtiriez-vous, et quel serait le lieu de mon repos ?
\VS{2}Car ma main a fait toutes ces choses, et c'est par moi que toutes ces choses ont eu leur être, dit Yahweh. Mais à qui regarderai-je ? A celui qui est affligé, qui a l'esprit abattu, et qui tremble à ma parole.
\VS{3}Celui qui égorge un bœuf est comme celui qui tuerait un homme ; celui qui sacrifie une brebis est comme celui qui romprait la nuque à un chien ; celui qui présente une offrande est comme celui qui offrirait le sang d'un pourceau ; celui qui fait un parfum d'encens est comme celui qui bénirait une idole ; tous ceux-là ont choisi leurs voies, et leur âme trouve du plaisir dans leurs abominations.
\VS{4}Moi aussi je ferai attention à leurs tromperies, et je ferai venir sur eux les choses qu'ils craignent ; parce que j'ai appelé, et personne n'a répondu, parce que j'ai parlé, et qu'ils n'ont point écouté ; mais ils ont fait ce qui est mal à mes yeux, et ils ont choisi les choses auxquelles je ne prends pas de plaisir. 
\VS{5}Ecoutez la parole de Yahweh, vous qui tremblez à sa parole ; vos frères, qui vous haïssent et qui vous repoussent comme une chose abominable, à cause de mon Nom disent : Que Yahweh montre sa gloire ! Il sera donc vu à votre joie mais eux seront honteux. 
\VS{6}Un son éclatant sort de la ville, un son sort du temple, le son de Yahweh, qui rend à ses ennemis selon leurs œuvres.
\TextTitle{Israël renaît en un jour}
\VS{7}Elle a enfanté, avant d'éprouver les douleurs de l'enfantement ; elle a donné naissance à un enfant mâle, avant que les souffrances lui viennent.
\VS{8}Qui a jamais entendu une telle chose ? Qui en a jamais vu de semblable ? Ferait-on qu'un pays naisse en un jour ? Ou une nation naîtrait-elle d'un seul coup\FTNT{Cette prophétie fait allusion à la création de l'Etat d'Israël le 14 mai 1948.} ? Car dès que Sion a été en travail, elle a enfanté ses enfants !
\VS{9}Moi qui fais enfanter les autres, ne ferais-je point enfanter Sion ? Dit Yahweh. Moi qui donne de la postérité aux autres, l'empêcherais-je d'enfanter ? Dit ton Dieu.
\TextTitle{Réjouissance à Jérusalem et consolation}
\VS{10}Réjouissez-vous avec Jérusalem, faites d'elle le sujet de votre allégresse, vous tous qui l'aimez ; vous tous qui menez deuil sur elle, réjouissez-vous avec elle d'une grande joie ;
\VS{11}afin que vous soyez allaités et rassasiés de la mamelle de ses consolations, afin que vous suciez le lait et que vous jouissiez à plaisir de la plénitude de sa gloire.
\VS{12}Car ainsi parle Yahweh : Voici, je ferai couler vers elle la paix comme un fleuve, et la gloire des nations comme un torrent débordé, et vous serez allaités, vous serez portés sur les côtés et caressés sur les genoux.
\VS{13}Je vous consolerai pour vous apaiser, comme quelqu'un que sa mère caresse pour l'apaiser, vous serez consolés dans Jérusalem.
\VS{14}Vous le verrez et votre cœur se réjouira, et vos os germeront comme l'herbe ; et la main de Yahweh sera connue de ses serviteurs ; mais il sera indigné contre ses ennemis.
\TextTitle{Jugement de Yahweh}
\VS{15}Car voici, Yahweh viendra avec le feu, et ses chars seront comme la tempête ; afin qu'il tourne sa colère en fureur, et sa menace en flamme de feu.
\VS{16}Car Yahweh exercera jugement contre toute chair par le feu et avec son épée ; et le nombre de ceux qui seront mis à mort par Yahweh sera grand.
\VS{17}Ceux qui se sanctifient et se purifient au milieu des jardins, l'un après l'autre, qui mangent de la chair de porc et des choses abominables, comme des souris, seront ensemble consumés, dit Yahweh.
\VS{18}Mais pour moi, voyant leurs œuvres et leurs pensées, le temps est venu de rassembler toutes les nations et les langues ; ils viendront et verront ma gloire.
\TextTitle{Toutes les nations adoreront Yahweh}
\VS{19}Car je mettrai un signe en eux, et j'enverrai ceux d'entre eux qui seront réchappés, vers les nations, à Tarsis, à Pul, à Lud, gens tirant de l'arc, à Tubal et à Javan, et vers les îles lointaines, qui n'ont point entendu ma renommée, et qui n'ont pas vu ma gloire ; et ils annonceront ma gloire parmi les nations.
\VS{20}Et ils amèneront tous vos frères d'entre toutes les nations, sur des chevaux, sur des chars et dans des litières, sur des mulets et sur des dromadaires, en offrande à Yahweh, à la montagne sainte, à Jérusalem, dit Yahweh, comme lorsque les enfants d'Israël apportent l'offrande dans un vase pur, à la maison de Yahweh.
\VS{21}Et même je prendrai aussi parmi eux des sacrificateurs, des Lévites, dit Yahweh.
\VS{22}Car comme les nouveaux cieux et la nouvelle terre que je vais faire subsisteront devant moi, dit Yahweh, ainsi subsistera votre postérité et votre nom.
\VS{23}Et il arrivera que de nouvelle lune en nouvelle lune, et de sabbat en sabbat, toute chair viendra se prosterner devant ma face, dit Yahweh.
\VS{24}Et quand ils sortiront dehors, ils verront les cadavres des hommes qui se sont rebellés contre moi ; car leur ver ne mourra point, et leur feu ne s'éteindra point\FTNT{Mc. 9:48.} ; et ils seront méprisés de tout le monde.
\PPE{}
\end{multicols}

\clearpage\ShortTitle{Jérémie}\BookTitle{Jérémie}\BFont
\noindent\hrulefill
{\footnotesize
\textit{
\bigskip
{\centering{}
\\Auteur : Jérémie
\\(Heb. : Yirmeyah)
\\Signification : Celui que Yahweh a désigné
\\Thème : Avertissements et jugements
\\Date de rédaction : 7\up{ème} siècle av J.C\\}
}
%\bigskip
\textit{
\\Issu d'une famille de sacrificateurs, Jérémie fut appelé dès son plus jeune âge au service de Yahweh et exerça un ministère prophétique avant et pendant les premières années de déportation. Outre son message à Israël et aux nations, le livre de Jérémie révèle sa personnalité. On découvre alors que l'opposition de ses pairs fut l'une de ses expériences les plus douloureuses. En effet, ce récit raconte ses combats contre les faux prophètes et met en évidence les signes accompagnant les prophètes authentiques, à savoir la souffrance, la solitude, l'incompréhension et le rejet.
%\bigskip
\\Son message annonçait le jugement imminent de Dieu et invitait le peuple à la repentance pour éviter le châtiment de Yahweh. Après la chute de Jérusalem, alors que Nebucadnetsar lui avait laissé le choix, Jérémie décida de rester avec les plus pauvres plutôt que de partir pour Babylone. Cependant, des Israélites décidèrent de s'expatrier en Egypte et l'entraînèrent avec eux de force. En terre étrangère, Jérémie continua de porter le fardeau de son peuple, l'exhortant à réformer ses voies. 
%\bigskip
\\Parmi les prophéties de Jérémie, figure le retour du peuple d'Israël sur la terre promise avant la seconde venue de Christ.\bigskip
}
}
\par\nobreak\noindent\hrulefill
\begin{multicols}{2}
\Chap{1}
\TextTitle{Yahweh appelle Jérémie à son service}
\VerseOne{}Les Paroles de Jérémie, fils de Hilkija, d'entre les sacrificateurs qui étaient à Anathoth, dans le pays de Benjamin;
\VS{2}auquel fut adressée la parole de Yahweh aux jours de Josias, fils d'Amon, roi de Juda, la treizième année de son règne,
\VS{3}laquelle lui fut aussi adressée aux jours de Jojakim, fils de Josias, roi de Juda, jusqu'à la fin de la onzième année de Sédécias, fils de Josias, roi de Juda; savoir jusqu'au temps où  Jérusalem fut transportée, ce qui arriva au cinquième mois.
\VS{4}La parole de Yahweh me fut adressée, en disant :
\VS{5}Avant que je t'aie formé dans le ventre de ta mère, je te connaissais, et avant que tu sois sorti de son sein, je t'avais consacré, je t'avais établi prophète pour les nations\FTNT{Es. 49:5 ; Ga. 1:15.}.
\VS{6}Je répondis : Ah ! Seigneur Yahweh ! Voici, je ne sais pas parler, car je suis un enfant\FTNT{Ex. 4:10-11.}.
\VS{7}Et Yahweh me dit : Ne dis pas : Je suis un enfant. Car tu iras partout où je t'enverrai, et tu diras tout ce que je t'ordonnerai.
\VS{8}Ne crains pas de te montrer devant eux, car je suis avec toi pour te délivrer, dit Yahweh.
\VS{9}Puis Yahweh avança sa main et toucha ma bouche ; et Yahweh me dit : Voici, je mets mes paroles dans ta bouche.
\VS{10}Regarde, je t'établis aujourd'hui sur les nations et sur les royaumes, pour que tu arraches et que tu démolisses, pour que tu ruines et que tu détruises, pour que tu bâtisses et que tu plantes\FTNT{Jérémie devait d'abord arracher, démolir, ruiner et détruire avant de bâtir et de planter. Il y avait dans le temple de Jérusalem les autels de Baal et le pieu d'Asherah (2 R. 21). De même, avant de planter la Parole de Dieu qui est une semence plantée dans les cœurs (Mc. 4 : 3-17), il est nécessaire au préalable d'arracher et de renverser les fausses doctrines et le péché en les dénonçant.}.
\TextTitle{Yahweh confirme la mission de Jérémie et l'établit sur Juda}
\VS{11}Puis la parole de Yahweh me fut adressée, en disant : Que vois-tu, Jérémie ? Et je répondis : Je vois une branche d'amandier.
\VS{12}Et Yahweh me dit : Tu as bien vu ; car je me hâte d'exécuter ma parole.
\VS{13}La parole de Yahweh me fut adressée pour la seconde fois, en disant : Que vois-tu ? Et je répondis : Je vois un pot bouillant dont le devant est tourné vers le nord.
\VS{14}Et Yahweh me dit : le mal se découvrira du côté du nord sur tous les habitants de ce pays-ci.
\VS{15}Car voici, je vais appeler toutes les familles des royaumes du nord, dit Yahweh ; elles viendront et mettront chacune leur trône à l'entrée des portes de Jérusalem, contre toutes ses murailles à l'entour, et contre toutes les villes de Juda.
\VS{16}Et je prononcerai mes jugements contre eux, à cause de toute leur méchanceté, par laquelle ils m'ont délaissé, et ont fait des parfums à d'autres dieux, et se sont prosternés devant l'ouvrage de leurs mains. 
\VS{17}Toi donc, ceins tes reins, lève-toi, et dis-leur tout ce que je t'ordonnerai. Ne crains pas de te montrer devant eux, de peur que je ne te mette en pièces en leur présence.
\VS{18}Car voici, je t'établis aujourd'hui sur tout le pays comme une ville forte, une colonne de fer, et un mur d'airain, contre les rois de Juda, contre les chefs du pays, contre ses sacrificateurs, et contre le peuple du pays.
\VS{19}Et ils combattront contre toi, mais ils ne seront pas plus forts que toi ; car je suis avec toi, dit Yahweh, pour te délivrer.
\Chap{2}
\TextTitle{Yahweh dénonce l'attitude d'Israël et l'avertit}
\VerseOne{}La parole de Yahweh me fut adressée, en disant :
\VS{2}Va et crie aux oreilles de Jérusalem, et dis : Ainsi parle Yahweh : Je me souviens de la fidélité de ta jeunesse, de l'amour de tes fiançailles, quand tu me suivais au désert, dans une terre qu'on n'ensemence pas. 
\VS{3}Israël était une chose sainte à Yahweh, il était les prémices de son revenu\FTNT{Lé. 23:20 ; Pr. 3:9 ; Né. 10:35.}; tous ceux qui le dévoraient étaient coupables, il leur en arrivait du mal dit Yahweh.
\VS{4}Ecoutez la parole de Yahweh, maison de Jacob, et vous toutes, familles de la maison d'Israël !
\VS{5}Ainsi parle Yahweh : Quelle iniquité vos pères ont-ils trouvée en moi, pour qu’ils se soient éloignés de moi, et qu’ils aient marché après la vanité et soient devenus vains ?
\VS{6}Ils n'ont pas dit : Où est Yahweh qui nous a fait remonter du pays d'Egypte, qui nous a conduits par un désert, par un pays de landes et montagneux, par un pays aride et d'ombre de mort, par un pays où aucun homme n'avait passé, et où personne n'avait habité ? 
\VS{7}Je vous ai fait entrer dans un pays de verger, pour que vous en mangiez les fruits et les biens ; mais sitôt vous y êtes entrés, vous avez souillé mon pays, et vous avez rendu abominable mon héritage.
\VS{8}Les sacrificateurs n'ont pas dit : Où est Yahweh ? Les dépositaires de la loi ne m'ont pas connu, les pasteurs se sont révoltés contre moi, les prophètes ont prophétisé par Baal\FTNT{Baal. Voir Jg. 2:13.}, et sont allés après ce qui n'est d'aucun profit.
\VS{9}A cause de cela, je veux encore contester avec vous, dit Yahweh, je veux contester avec les fils de vos fils.
\VS{10}Passez par les îles de Kittim et voyez ! Envoyez quelqu'un à Kédar ; observez bien, et voyez s'il n'y a rien de semblable !
\VS{11}Y a-t-il une nation qui change ses dieux, quoiqu'ils ne soient pas des dieux ? Et mon peuple a changé sa gloire contre ce qui n'est d'aucun profit\FTNT{Ro. 1:23.} !
\VS{12}Cieux, soyez étonnés de cela ; frémissez d'horreur et soyez stupéfaits ! dit Yahweh.
\VS{13}Car mon peuple a commis doublement le mal : Ils m'ont abandonné, moi qui suis la source d'eaux vives\FTNT{Yahweh est la Source d'eaux vives. Jésus-Christ se présente aussi comme la Source d'eau vive (Jn. 4:13-14 ; Ap. 21:6).}, pour se creuser des citernes, des citernes crevassées qui ne peuvent pas retenir l'eau.
\VS{14}Israël est-il un esclave, ou un esclave né dans la maison ? Pourquoi donc est-il mis au pillage ?
\VS{15}Les lionceaux rugissent, poussent leurs cris contre lui, et ils mettent son pays en désolation ; ses villes sont brûlées, de sorte que personne n'y habite.
\VS{16}Même les fils de Noph et de Tachpanès te casseront le sommet de la tête.
\VS{17}Cela ne t'arrive-t-il pas parce que tu as abandonné Yahweh, ton Dieu, à l'époque où il te conduisait par le chemin ?
\VS{18}Et maintenant, qu'as-tu à faire d'aller en Egypte, pour boire l'eau du Schichor\FTNT{Schichor : Sombre, noir, boueux. Le Nil, une rivière ou un canal affluent du fleuve. Les Israélites préféraient ces eaux à Yahweh.} ? Qu'as-tu à faire d'aller en Assyrie, pour boire l'eau du fleuve ?
\VS{19}Ta méchanceté te châtiera, et tes débauches te jugeront, tu sauras et tu verras que c'est une chose mauvaise et amère d'abandonner Yahweh, ton Dieu, et de n'avoir de moi aucune crainte, dit le Seigneur, Yahweh des armées.
\VS{20}Tu as dès longtemps brisé ton joug, rompu tes liens, et tu as dit : Je ne veux plus être dans la servitude ! Mais sur toute haute colline et sous tout arbre vert tu t'es incliné, tu t'es prostitué.
\VS{21}Je t'avais moi-même plantée comme une vigne exquise, dont tout le plant était franc ; comment t'es-tu changée en sarments d'une vigne étrangère ?
\VS{22}Quand tu te laverais avec du nitre, et que tu prendrais beaucoup de savon, ton iniquité resterait encore marquée devant moi, dit le Seigneur, Yahweh.
\VS{23}Comment dirais-tu : Je ne me suis pas souillée, je ne suis pas allée après les Baals ? Regarde tes pas dans la vallée, reconnais ce que tu as fait, dromadaire à la course légère et vagabonde !
\VS{24}Anesse sauvage, accoutumée au désert, humant le vent à son plaisir. Qui l'arrêtera dans son ardeur ? Tous ceux qui la cherchent n'ont pas à se fatiguer ; ils la trouvent pendant son mois.
\VS{25}Garde ton pied de se déchausser, ton gosier d'avoir soif ! Mais tu dis : C'est en vain, non ! Car j'aime les dieux étrangers, et j'irai après eux.
\VS{26}Comme un voleur est confus quand il est surpris, ainsi seront confus ceux de la maison d'Israël, eux, leurs rois, leurs chefs, leurs sacrificateurs et leurs prophètes.
\VS{27}Ils disent au bois : Tu es mon père ! Et à la pierre : Tu m'as enfanté ! Car ils me tournent le dos, et non la face. Et ils disent dans le temps de leur malheur : Lève-toi, et sauve-nous !
\VS{28}Où donc sont tes dieux que tu t'es faits ? Qu'ils se lèvent, s'ils peuvent te sauver au temps de ton malheur ! Car tu as autant de dieux que de villes, ô Juda !
\VS{29}Pourquoi contesteriez-vous avec moi ? Vous vous êtes tous rebellés contre moi, dit Yahweh.
\VS{30}En vain ai-je frappé vos fils ; ils n'ont pas reçu d'instruction ; votre épée a dévoré vos prophètes comme un lion destructeur.
\VS{31}Hommes de cette génération, considérez la parole de Yahweh ! Ai-je été un désert pour Israël, ou un pays de ténèbres ? Pourquoi mon peuple dit-il : Nous sommes libres, nous ne viendrons plus à toi ?
\VS{32}La vierge oublie-t-elle ses ornements, la fiancée sa ceinture ? Mais mon peuple m'a oublié depuis des jours sans nombre.
\VS{33}Comme tu es habile dans tes voies pour chercher ce que tu aimes ! C'est pourquoi aussi tu accoutumes tes voies aux crimes.
\VS{34}Même sur les pans de ta robe se trouve le sang des pauvres, des innocents que tu n'as pas trouvés en effraction.
\VS{35}Malgré cela, tu dis : Oui, je suis innocent ! Certainement sa colère s'est détournée de moi ! Voici, je vais entrer en jugement avec toi, sur ce que tu as dit : Je n'ai pas péché.
\VS{36}Pourquoi tant te précipiter pour changer ton chemin ? Tu auras autant de confusion de l'Egypte que tu en as eu de l'Assyrie.
\VS{37}Tu sortiras même d'ici, ayant tes mains sur la tête ; car Yahweh rejette ceux en qui tu te confies, et tu n'auras aucune prospérité par eux.
\Chap{3}
\TextTitle{Israël comparé à une prostituée}
\VerseOne{}Il dit : Si un homme répudie sa femme, qu'elle le quitte et se joigne à un autre, cet homme retourne-t-il encore vers elle\FTNT{Lé. 21:7 ; De. 24:2.} ? Le pays même n'en serait-il pas entièrement souillé ? Or toi, tu t'es prostituée à plusieurs amants, et tu reviendrais à moi ! dit Yahweh.
\VS{2}Lève tes yeux vers les lieux élevés et regarde ! Où ne t'es-tu pas prostituée ! Tu te tenais sur les chemins, comme un Arabe dans le désert, et tu as souillé le pays par tes prostitutions et par ta méchanceté.
\VS{3}Aussi les pluies ont été retenues, et il n'y a pas eu de pluie de l'arrière-saison ; mais tu as eu le front d'une femme prostituée, tu n'as pas voulu avoir honte.
\VS{4}Maintenant, n'est-ce pas ? Tu cries vers moi : Mon père ! Tu as été l'ami de ma jeunesse !
\VS{5}Gardera-t-il à toujours sa colère ? La conservera-t-il à jamais\FTNT{Es. 57:16 ; Ps.103:9.} ? Voici, tu as ainsi parlé, tu as fait ces maux-là autant que tu as pu.
\TextTitle{Yahweh appelle Israël à la repentance}
\VS{6}Yahweh me dit au temps du roi Josias : As-tu vu ce qu'a fait Israël, l'infidèle ? Elle est allée sur toute haute colline et sous tout arbre vert, et elle s'y est prostituée.
\VS{7}Je disais : Après avoir fait toutes ces choses, elle reviendra à moi. Mais elle n'est pas revenue. Et sa sœur Juda, la perfide, l'a vu.
\VS{8}Quoique j'aie répudié Israël, l'infidèle, à cause de tous ses adultères, et que je lui aie donné sa lettre de divorce, j'ai vu que la perfide Juda, sa sœur, n'a pas eu de crainte, mais elle s'en est allée et s'est aussi prostituée.
\VS{9}Par le bruit de sa prostitution, elle a souillé le pays, elle a commis un adultère avec la pierre et le bois.
\VS{10}Malgré tout cela, sa sœur Juda, la perfide, n'est pas revenue à moi de tout son cœur ; c'est avec fausseté qu'elle l'a fait, dit Yahweh.
\VS{11}Et Yahweh me dit : Israël, l'infidèle, se montre plus juste que Juda, la perfide.
\VS{12}Va, crie ces paroles vers le nord, et dis : Reviens, Israël, l'infidèle, dit Yahweh. Je ne jetterai pas sur vous un regard sévère ; car je suis miséricordieux, dit Yahweh, je ne garde pas ma colère à toujours.
\VS{13}Reconnais seulement ton iniquité, que tu t'es rebellée contre Yahweh, ton Dieu, que tu as tourné çà et là tes pas vers les étrangers, sous tout arbre vert, et que tu n'as pas écouté ma voix, dit Yahweh.
\VS{14}Fils rebelles, convertissez-vous, dit Yahweh, car je suis votre maître. Je vous prendrai, un d'une ville, deux d'une famille, et je vous ferai entrer dans Sion.
\VS{15}Je vous donnerai des pasteurs selon mon cœur, qui vous paîtront avec intelligence et avec sagesse\FTNT{Jé. 23:5.}.
\VS{16}Lorsque vous aurez multiplié et fructifié dans le pays, en ces jours-là, dit Yahweh, on ne parlera plus de l'arche de l'alliance de Yahweh, elle ne viendra plus à la pensée ; on ne s'en souviendra plus, on ne s'apercevra plus de son absence, et l'on n'en fera pas une autre.
\VS{17}En ce temps-là, on appellera Jérusalem le trône de Yahweh ; toutes les nations s'assembleront à Jérusalem, au Nom de Yahweh, et elles ne marcheront plus suivant les penchants de leur mauvais cœur.
\VS{18}En ces jours-là, la maison de Juda marchera avec la maison d'Israël ; elles viendront ensemble du pays du nord au pays que j'ai donné en héritage à vos pères.
\VS{19}Je disais : Comment te mettrai-je parmi mes fils et te donnerai-je un pays désirable, le plus bel héritage des armées des nations ? Je disais : Tu m'appelleras : Mon père ! Et tu ne te détourneras pas de moi.
\VS{20}Mais, comme une femme est infidèle à son compagnon, ainsi vous m'avez été infidèles, maison d'Israël, dit Yahweh.
\VS{21}Une voix se fait entendre sur les lieux élevés ; ce sont les pleurs, les supplications des fils d'Israël ; car ils ont perverti leur voie, ils ont oublié Yahweh, leur Dieu.
\VS{22}Fils rebelles, convertissez-vous, je guérirai vos infidélités. Nous voici, nous venons à toi, car tu es Yahweh, notre Dieu.
\VS{23}Certainement, on s'attend en vain aux collines et à la multitude des montagnes ; mais c'est en Yahweh, notre Dieu, qu'est la délivrance d'Israël.
\VS{24}Car la honte a dévoré dès notre jeunesse le travail de nos pères, leurs brebis et leurs bœufs, leurs fils et leurs filles.
\VS{25}Nous serons gisants dans notre honte, et notre ignominie nous couvrira ; parce que nous avons péché contre Yahweh, notre Dieu, nous et nos pères, dès notre jeunesse jusqu'à ce jour, et nous n'avons pas obéi à la voix de Yahweh, notre Dieu.
\Chap{4}
\TextTitle{Prophétie sur l'invasion du pays}
\VerseOne{}Israël, si tu reviens, dit Yahweh, si tu reviens à moi, si tu ôtes tes abominations de devant moi, tu ne seras plus errant ça et là.
\VS{2}Alors tu jureras avec vérité, avec droiture et avec justice : Yahweh est vivant ! Et les nations seront bénies en lui, et se glorifieront en lui.
\VS{3}Car ainsi parle Yahweh aux hommes de Juda et de Jérusalem : Labourez pour vous une terre arable et ne semez pas parmi les épines\FTNT{Mt. 13:7 ; Mt. 13:22 ; Mc. 4:7 ; Mc. 4:18 ; Lu. 8:14.}.
\VS{4}Hommes de Juda, et vous habitants de Jérusalem, circoncisez-vous pour Yahweh, circoncisez vos cœurs\FTNT{Ro. 2:29.}, de peur que ma fureur ne sorte comme un feu et qu'elle ne brûle sans qu'on puisse l'éteindre, à cause de la méchanceté de vos actions.
\VS{5}Annoncez en Juda, publiez dans Jérusalem, et dites : Sonnez du shofar dans le pays ! Criez à pleine voix et dites : Assemblez-vous et nous entrerons dans les villes fortes !
\VS{6}Elevez une bannière vers Sion, fuyez, ne vous arrêtez pas ! Car je fais venir du nord le malheur et une grande calamité.
\VS{7}Le lion\FTNT{Lion est ici une allusion à Nebucadnetsar, roi de Babylone. Voir 2 R. 24 et 25 ; Da. 7:4.} est sorti de la caverne, le destructeur des nations est en marche, il est sorti de son lieu, pour réduire ton pays en désert ; tes villes seront ruinées, il n'y aura personne pour y habiter.
\VS{8}C'est pourquoi ceignez-vous de sacs, lamentez-vous et gémissez ; car l'ardeur de la colère de Yahweh ne se détourne pas de nous.
\VS{9}Et il arrivera ce jour-là, dit Yahweh, que le cœur du roi et le cœur des chefs seront épouvantés et que les sacrificateurs seront étonnés, et que les prophètes seront stupéfaits.
\VS{10}C'est pourquoi je dis : Ah ! Seigneur Yahweh ! Oui certainement tu as abusé ce peuple et Jérusalem, en disant : Vous aurez la paix ! Et cependant l'épée est venue jusqu'à l'âme.
\VS{11}En ce temps-là on dira à ce peuple et à Jérusalem : Un vent brûlant souffle des lieux élevés du désert sur le chemin de la fille de mon peuple, non pas pour vanner ni pour nettoyer.
\VS{12}C'est un vent impétueux qui vient de là jusqu'à moi et je leur ferai maintenant leur procès. 
\VS{13}Voici, il monte comme des nuées ; ses chars sont comme un tourbillon, ses chevaux sont plus légers que les aigles. Malheur à nous, car nous sommes détruits !
\VS{14}Jérusalem, lave ton cœur du mal afin que tu sois délivrées ! Jusqu'à quand séjourneront-tu au-dedans de toi les pensées de ton injustice ?
\VS{15}Car une voix apporte des nouvelles de Dan, elle publie depuis la montagne d'Ephraïm le tourment.
\VS{16}Rappelez-le aux nations, faites-le entendre à Jérusalem : Des observateurs viennent d'un pays éloigné ; ils poussent des cris contre les villes de Juda.
\VS{17}Ils se sont mis tout autour d'elle comme ceux qui gardent un champ, parce qu'elle s'est rebellée contre moi, dit Yahweh.
\VS{18}Ta conduite et tes actions t'ont produit ces choses, telle a été ta méchanceté, parce que cela a été une chose amère, certainement elle t'atteindra jusqu'à ton cœur.
\VS{19}Mes entrailles ! Mes entrailles : Je suis dans la douleur au-dedans de mon cœur, mon cœur bat, je ne puis me taire ; car, ô mon âme, tu entends le son du shofar, la clameur de la guerre.
\VS{20}On annonce brèche sur brèche, car tout le pays est dévasté ; mes tentes sont détruites tout à coup, mes pavillons en un moment.
\VS{21}Jusqu'à quand verrai-je la bannière et entendrai-je le son du shofar ?
\VS{22}Car mon peuple est insensé ; ils ne m'ont pas reconnu, ce sont des enfants insensés qui n'ont pas d'intelligence ; ils sont habiles pour faire le mal, et ils ne savent pas faire le bien.
\VS{23}Je regarde la terre, et voici, elle est informe et vide\FTNT{Dieu n'a pas créé la terre informe et vide, mais elle l'est devenue à cause du péché. Voir le commentaire en Gn. 1:2.} ; les cieux et leur lumière ne sont plus.
\VS{24}Je regarde les montagnes, et voici, elles sont ébranlées ; et toutes les collines sont renversées.
\VS{25}Je regarde, et voici, il n'y a pas un seul homme et tous les oiseaux des cieux se sont enfuis.
\VS{26}Je regarde, et voici, le Carmel est un désert ; et toutes ses villes sont détruites, devant Yahweh, devant l'ardeur de sa colère.
\VS{27}Car ainsi parle Yahweh : Tout le pays sera dévasté, mais je ne ferai pas une entière destruction.
\VS{28}C'est pourquoi le pays mènera deuil et les cieux en haut seront obscurcis, parce que je l'ai dit, je l'ai résolu, et je ne m'en repentirai pas et je le révoquerai pas.
\VS{29}Toute la ville s'enfuit à cause du bruit des cavaliers et des archers ; ils entrent dans les bois fourrés et montent sur les rochers ; toute la ville est abandonnée, et aucun homme n'y habite.
\VS{30}Et quand tu auras été détruite que fais-tu ? Quoique tu te revêtes de pourpre, que tu te pares d'ornements d'or, et que tu bordes tes yeux de fard, tu t'embellis en vain: tes amants t'ont méprisée; c'est ta vie qu'ils cherchent.
\VS{31}Car j'entends un cri comme celui d'une femme qui est en travail, et une angoisse comme celle d'une femme qui est en travail de son premier-né ; c'est le cri de la fille de Sion ; elle soupire, elle étend ses mains, en disant : Malheur maintenant à moi, car mon âme a défailli à cause des meurtriers. 
\Chap{5}
\TextTitle{Raisons du jugement de Yahweh}
\VerseOne{}Parcourez les rues de Jérusalem et regardez maintenant, sachez et cherchez dans les places, si vous y trouvez un homme de bien, s'il y a quelqu'un qui fasse ce qui est droit, qui cherche la vérité, et je pardonne à Jérusalem\FTNT{Es. 59:15 ; Mi. 7:2 ; Pr. 20:6.}.
\VS{2}Même s'ils disent : Yahweh est vivant ! En cela, ils jurent faussement.
\VS{3}Yahweh, tes yeux ne regardent-ils pas à la fidélité ? Tu les frappes, et ils ne sentent pas de douleur ; tu les consumes, et ils refusent de recevoir l'instruction ; ils endurcissent leurs faces plus qu'un rocher, ils refusent de se convertir.
\VS{4}Je disais : Certainement ce ne sont que les plus petits ; ils se montrent insensés parce qu'ils ne connaissent pas la voie de Yahweh, le droit de leur Dieu.
\VS{5}J'irai donc vers les plus grands, et je leur parlerai ; car cela connaissent la voie de Yahweh, le droit de leur Dieu ; mais ceux-là même ont brisés le joug et ont rompu les liens.
\VS{6}C'est pourquoi le lion de la forêt les tue, le loup du soir les détruit, et le léopard est aux aguets contre leurs villes ; quiconque en sortira sera déchiré ; car leurs transgressions sont nombreuses, et leurs infidélités se sont renforcées.
\VS{7}Comment te pardonnerais-je en cela ? Tes fils m'ont abandonné, et ils jurent par ce qui ne sont pas dieux. Je les ai rassasiés, mais ils commettent l'adultère et ils se pressent en foule dans la maison de la prostituée.
\VS{8}Ils sont comme des chevaux bien nourris, quand ils se lèvent le matin, chacun hennit après la femme de son prochain.
\VS{9}Ne punirais-je pas ces choses-là, dit Yahweh ? Et mon âme ne se vengerait-elle pas d'une telle nation ?
\VS{10}Montez sur ses murailles et détruisez-les, mais ne les achevez pas entièrement ! Otez ses sarments car ils ne sont pas à Yahweh\FTNT{Jn. 15:5.} !
\VS{11}Car la maison d'Israël et la maison de Juda m'ont été infidèles, dit Yahweh.
\VS{12}Ils démentent Yahweh, et disent : Cela n'arrivera pas, et le malheur ne viendra pas sur nous, nous ne verrons ni l'épée ni la famine.
\VS{13}Et les prophètes sont légers comme le vent, et la parole n'est pas en eux. Qu'il leur soit fait ainsi !
\VS{14}C'est pourquoi ainsi parle Yahweh, le Dieu des armées : Parce que vous avez prononcé cette parole-là, voici, je vais mettre mes paroles dans ta bouche pour y être comme une feu, et ce peuple sera comme le bois, et ce feu les consumera.
\VS{15}Maison d'Israël, voici, je fais venir contre vous une nation d'un pays éloigné\FTNT{Il s'agit de Babylone. Voir 2 R. 24 et 25.}, dit Yahweh, une nation puissante, une nation ancienne, une nation dont tu ne connais pas la langue, et dont tu ne comprendras pas ce qu'elle dira.
\VS{16}Son carquois est comme un sépulcre ouvert, et ils sont tous des hommes vaillants.
\VS{17}Et elle dévorera ta moisson et ton pain, que tes fils et tes filles devaient manger; elle dévorera tes brebis et tes bœufs; elle dévorera les fruits ta vigne et ton figuier et réduira à la pauvreté par l'épée tes villes fortes dans lesquelles tu te confies.
\VS{18}Toutefois en ces jours-là, dit Yahweh, je ne vous achèverai pas entièrement.
\VS{19}Et il arrivera que vous direz : Pourquoi Yahweh, notre Dieu, nous a-t-il fait toutes ces choses ? Tu leur diras ainsi : Comme vous m'avez abandonné et que vous avez servi les dieux étrangers dans votre pays, ainsi vous servirez des étrangers dans un pays qui n'est pas le vôtre.
\VS{20}Annoncez ceci dans la maison de Jacob, et publiez-le dans Juda, en disant :
\VS{21}Ecoutez maintenant ceci, peuple insensé, et qui n'avez pas d'intelligence; qui avez des yeux et ne voyez pas; et qui avez des oreilles et n'entendez pas\FTNT{Ez. 12:2 ; Jn. 12:40.}.
\VS{22}Ne me craindrez-vous pas, dit Yahweh, ne tremblerez-vous pas devant ma face ? C'est moi qui ai mis le sable pour limite à la mer, par une ordonnance perpétuelle et qui ne passera pas ; ses vagues s'agitent, mais elles sont impuissantes ; elles grondent, mais elles ne la passent pas\FTNT{Pr. 8:29 ; Job. 38:8.}.
\VS{23}Mais ce peuple-ci a un cœur indocile et rebelle ; ils reculent en arrière et s'en vont.
\VS{24}Et ils ne disent pas dans leur cœur : Craignons maintenant Yahweh, notre Dieu, qui nous donne la pluie en son temps, de la première et de l'arrière-saison, et qui nous réserve les semaines ordonnées pour la moisson.
\VS{25}Vos iniquités ont détourné ces choses, vos péchés retiennent loin de vous le bien.
\VS{26}Car il se trouve parmi mon peuple des méchants ; ils épient comme l'oiseleur qui dresse des pièges, ils tendent des filets et prennent des hommes\FTNT{Ps. 91:3 ; Ps. 124:7.}.
\VS{27}Comme la cage est remplie d'oiseaux, ainsi leurs maisons sont remplies de fraude ; c'est par ce moyen qu'ils deviennent grands et riches.
\VS{28}Ils s'engraissent, ils sont brillants ; ils surpassent les actions des méchants, ils ne jugent pas la cause, la cause de l'orphelin, et ils prospèrent ; ils ne font pas droit aux pauvres.
\VS{29}Ne punirais-je pas ces choses-là, dit Yahweh ? Et mon âme ne se vengerait-elle pas d'une telle nation ?
\VS{30}Il est arrivé dans le pays une chose étonnante et horrible :
\VS{31}C'est que les prophètes prophétisent le mensonge, et les sacrificateurs dominent par leur moyen, et mon peuple prend plaisir à cela. Que ferez-vous donc quand elle prendra fin ?
\Chap{6}
\TextTitle{Jérusalem dans la confusion}
\VerseOne{}Fils de Benjamin, fuyez par troupes du milieu de Jérusalem, et sonnez du shofar à Tekoa, et élevez un signal de feu à Beth-Hakkérem ! Car on voit venir du nord un malheur et une grande ruine.
\VS{2}La belle et la délicate, la fille de Sion, je la détruis !
\VS{3}Les pasteurs avec leurs troupeaux viennent contre elle ; ils plantent leurs tentes autour d'elle, chacun paîtra en son quartier.
\VS{4}Préparez le combat contre elle ! Levez-vous, et montons en plein midi !… Malheur à nous, car le jour décline, les ombres du soir s'étendent.
\VS{5}Levez-vous ! Montons de nuit, et ruinons ses palais !
\VS{6}Car ainsi parle Yahweh des armées : Coupez des arbres, élevez des terrasses contre Jérusalem ! C'est la ville qui doit être visitée ; tout est oppression au milieu d'elle.
\VS{7}Comme le puits fait jaillir ses eaux, ainsi elle fait jaillir sa méchanceté ; on n'entend continuellement en elle, devant moi, que violence et ruine, avec des maladies et des plaies.
\VS{8}Jérusalem, reçois l'instruction, de peur que mon âme ne se retire de toi, et que je ne fasse de toi un désert, et une terre inhabitée !
\VS{9}Ainsi parle Yahweh des armées : On grappillera entièrement comme une vigne les restes d'Israël. Remets ta main dans les paniers, comme un vendangeur.
\VS{10}A qui parlerai-je, et qui prendrai-je à témoin, pour qu'ils écoutent ? Voici, leur oreille est incirconcise, et ils ne peuvent entendre ; voici, la parole de Yahweh leur est en opprobre, ils n'y prennent point de plaisir.
\VS{11}C'est pourquoi je suis plein de la fureur de Yahweh, et je suis las de la contenir. Répands-la sur les enfants dans la rue, et sur les assemblées des jeunes gens. Car tant le mari que la femme seront pris, le vieillard et celui qui est chargé de jours.
\VS{12}Et leurs maisons passeront à d'autres, les champs et les femmes aussi, quand j'étendrai ma main sur les habitants du pays, dit Yahweh.
\VS{13}Car depuis le plus petit d'entre eux jusqu'au plus grand, chacun s'adonne au gain déshonnête, tant le prophète que le sacrificateur, tous se agissent faussement.
\VS{14}Et ils pansent à la légère la plaie de la fille de mon peuple, disant : Paix ! Paix ! et il n'y a pas de paix\FTNT{1 Th. 5:3.}.
\VS{15}Sont-ils confus d'avoir commis des abominations ? Ils n'en ont même aucune honte, et ils ne savent pas ce que c'est que de rougir ; c'est pourquoi ils tomberont parmi ceux qui tombent, ils seront renversés au temps où je les visiterai, dit Yahweh.
\VS{16}Ainsi parle Yahweh : Tenez-vous sur les chemins, regardez et enquérez-vous des sentiers des siècles passés, quel est le bon chemin ; et marchez-y, et vous trouverez le repos de vos âmes ! Et ils répondent : Nous n'y marcherons pas.
\VS{17}J'ai aussi établi sur vous des sentinelles\FTNT{Es. 21:6 ; Ez. 33:1-19.} qui disent : Soyez attentifs au son du shofar ! Mais ils répondent : Nous n'y serons pas attentifs.
\VS{18}Vous donc, nations, écoutez, et toi assemblée, connais ce qui est entre eux.
\VS{19}Ecoute, terre ! Voici, je fais venir un mal sur ce peuple, à savoir le fruit de leurs pensées ; car ils n'ont pas été attentifs à mes paroles, et qu'ils ont rejeté ma loi.
\VS{20}Pourquoi m'offrir de l'encens venu de Séba, et le bon roseau aromatique du pays éloigné ? Vos holocaustes ne me plaisent pas, et vos sacrifices ne me sont pas agréables.
\VS{21}C'est pourquoi ainsi parle Yahweh : Voici, je mettrai devant ce peuple des pierres d'achoppement, auxquels les pères et les fils, le voisin et son compagnon, se heurteront ensemble et ils périront.
\VS{22}Ainsi parle Yahweh : Voici, un peuple vient du pays du nord, et une grande nation se réveille des extrémités de la terre.
\VS{23}Ils prendront l'arc et le javelot ; ils sont cruels et n'ont pas de pitié ; leur voix gronde comme la mer ; ils sont montés sur des chevaux, ils sont rangés comme un seul homme en bataille contre toi, fille de Sion !
\VS{24}Nous en entendons le bruit, nos mains en deviennent lâches, l'angoisse nous saisit, et une douleur comme celle d'une femme qui enfante.
\VS{25}Ne sortez pas dans les champs, n'allez pas par les chemins ; car l'épée de l'ennemi, la terreur est partout.
\VS{26}Fille de mon peuple, ceins-toi d'un sac et roule-toi dans la cendre, prends le deuil comme pour un fils unique, fais une lamentation très amère ! Car le dévastateur vient subitement sur nous.
\VS{27}Je t'avais établi en observateur au milieu de mon peuple, comme une forteresse, pour que tu connaisses et que tu éprouves leur voie.
\VS{28}Ils sont tous rebelles et plus que rebelles, des calomniateurs, ils sont comme de l'airain et du fer ; ils sont tous corrompus.
\VS{29}Le soufflet est brûlant, le plomb est consumé par le feu ; c'est en vain que l'on fond et refond, car les mauvais ne sont pas séparés.
\VS{30}On les appelle de l'argent réprouvé, car Yahweh les a réprouvés.
\Chap{7}
\TextTitle{Hypocrisie de Juda}
\VerseOne{}La parole fut adressée à Jérémie de la part de Yahweh, en disant :
\VS{2}Tiens-toi debout à la porte de la maison de Yahweh, et là, crie cette parole, et dis : Ecoutez la parole de Yahweh, vous tous, hommes de Juda, qui entrez par ces portes, pour vous prosterner devant Yahweh !
\VS{3}Ainsi parle Yahweh des armées, le Dieu d'Israël : Amendez vos voies et vos actions, et je vous ferai habiter en ce lieu-ci.
\VS{4}Ne vous confiez pas en des paroles trompeuses, en disant : C'est ici le temple de Yahweh, le temple de Yahweh, le temple de Yahweh !
\VS{5}Mais amendez sérieusement vos voies et vos actions, et appliquez-vous à faire droit à ceux qui plaident l'un contre l'autre,
\VS{6}et ne faites pas de tort à l'étranger, ni à l'orphelin, ni à la veuve, et ne répandez pas en ce lieu-ci le sang innocent, et ne marchez pas après les dieux étrangers, pour votre malheur.
\VS{7}Et je vous ferai habiter depuis un siècle jusqu'à l'autre siècle en ce lieu-ci, dans le pays que j'ai donné à vos pères.
\VS{8}Voici, vous vous confiez en des paroles trompeuses, sans aucun profit.
\VS{9}Ne dérobez-vous pas ? Ne tuez-vous pas ? Ne commettez-vous pas adultère ? Ne jurez-vous pas faussement ? Ne faites-vous pas des encensements à Baal ? N'allez-vous pas après les dieux étrangers, que vous ne connaissez point ?
\VS{10}Toutefois vous venez et vous vous présentez devant moi, dans cette maison sur laquelle mon Nom est invoqué, et vous dites : Nous sommes délivrés !… Pour faire toutes ces abominations !
\VS{11}N'est-elle plus à vos yeux qu'une caverne de voleurs\FTNT{Mt. 21:13 ; Mc. 11:17 ; Lu. 19:46.}, cette maison sur laquelle mon Nom est invoqué ? Et voici, moi-même je le vois, dit Yahweh.
\VS{12}Mais allez maintenant à mon lieu qui était à Silo, où j'avais fait demeurer mon Nom au commencement. Et regardez ce que je lui ai fait, à cause de la méchanceté de mon peuple d'Israël.
\VS{13}Maintenant donc, puisque vous avez fait toutes ces actions, dit Yahweh, puisque je vous ai parlé, parlé dès le matin, et que vous n'avez pas écouté, puisque je vous ai appelés et que vous n'avez pas répondu ;
\VS{14}je ferai à cette maison sur laquelle mon Nom est invoqué, et sur laquelle vous vous confiez, et à ce lieu que je vous ai donné à vous et à vos pères, comme j'ai fait à Silo ;
\VS{15}et je vous chasserai de devant ma face, comme j'ai chassé tous vos frères, avec toute la postérité d'Ephraïm.
\VS{16}Toi donc ne prie pas pour ce peuple, et n'élève pour eux ni cri ni prière, et n'intercède pas auprès de moi\FTNT{Ez. 3:26-27.} ; car je ne t'écouterai pas.
\VS{17}Ne vois-tu pas ce qu'ils font dans les villes de Juda et dans les rues de Jérusalem ?
\VS{18}Les fils ramassent le bois, et les pères allument le feu, et les femmes pétrissent la pâte pour faire des gâteaux à la reine des cieux\FTNT{La reine des cieux est une déesse qui change de nom en fonction des pays. Asherah, Astarté, Isis, Junon, Cybèle, Diane ou encore la vierge Marie, proclamée mère de Dieu en 431 au concile d'Ephèse. Voir De. 16:2-3.}, et pour faire des libations aux dieux étrangers, afin de m'irriter.
\VS{19}Est-ce moi qu'ils irritent ? dit Yahweh ; n'est-ce pas contre eux-mêmes, à la confusion de leurs faces ?
\VS{20}C'est pourquoi ainsi parle le Seigneur Yahweh : Voici, ma colère et ma fureur se répandent sur ce lieu-ci, sur les hommes et sur les bêtes, sur les arbres des champs et sur le fruit de la terre ; ma colère brûlera et ne s'éteindra pas.
\VS{21}Ainsi parle Yahweh des armées, le Dieu d'Israël : Ajoutez vos holocaustes à vos sacrifices, et mangez-en la chair !
\VS{22}Car je n'ai pas parlé avec vos pères et je ne leur ai pas donné d'ordre au sujet des holocaustes et des sacrifices, le jour où je les ai fait sortir du pays d'Egypte.
\VS{23}Mais voici la parole que je leur ai commandée, disant : Ecoutez ma voix, et je serai votre Dieu, et vous serez mon peuple ; marchez dans toutes les voies que je vous ordonne, afin que vous soyez heureux\FTNT{Ex. 15:26.}.
\VS{24}Mais ils n'ont pas écouté, et n'ont pas prêté l'oreille ; mais ils ont suivi d'autres conseils, les penchants de leur mauvais cœur ; ils se sont éloignés et ne sont pas revenus à moi.
\VS{25}Depuis le jour où vos pères sont sortis du pays d'Egypte, jusqu'à ce jour, je vous ai envoyé tous mes serviteurs les prophètes, je les ai envoyés chaque jour, dès le matin.
\VS{26}Mais ils ne m'ont pas écouté, et ils n'ont pas prêté l'oreille ; mais ils ont raidi leur cou, ils ont fait le mal plus que leurs pères.
\VS{27}Tu leur diras toutes ces paroles, mais ils ne t'écouteront pas ; et tu crieras après eux, mais ils ne te répondront pas.
\VS{28}C'est pourquoi tu leur diras : C'est ici la nation qui n'écoute pas la voix de Yahweh, son Dieu, et qui ne reçoit pas d'instruction ; la vérité a disparu, elle s'est retirée de leur bouche.
\VS{29}Coupe ta chevelure, ô Jérusalem ! Et jette-la au loin, et prononce à haute voix ta complainte sur les lieux élevés ! Car Yahweh rejette et abandonne la génération qui a provoqué sa fureur.
\VS{30}Car les fils de Juda ont fait ce qui est mal à mes yeux, dit Yahweh ; ils ont mis leurs abominations dans cette maison sur laquelle mon Nom est invoqué, afin de la souiller.
\VS{31}Et ils ont bâti les hauts lieux de Topheth, qui est dans la vallée de Ben-Hinnom\FTNT{Voir commentaire en Ap. 16:16.}, pour brûler au feu leurs fils et leurs filles\FTNT{Lé. 18:21. Voir commentaire en Lé. 20:2.} : Ce que je n'avais pas ordonné, et à quoi je n'ai jamais pensé.
\VS{32}C'est pourquoi voici, les jours viennent, dit Yahweh, qu'elle ne sera plus appelée Topheth, ni la vallée de Ben-Hinnom, mais la vallée de la tuerie ; et on enterrera les morts à Topheth, à cause qu'il n'y aura plus d'autre lieu.
\VS{33}Et les cadavres de ce peuple seront la pâture des oiseaux des cieux et des bêtes de la terre ; sans qu'il n'y ait personne qui les effraye.
\VS{34}Je ferai aussi cesser dans les villes de Juda et dans les rues de Jérusalem les cris de joie et les cris d'allégresse, la voix de l'époux et la voix de l'épouse ; car le pays sera un désert.
\Chap{8}
\TextTitle{Juda dans l'égarement}
\VerseOne{}En ce temps-là, dit Yahweh, on sortira les os des rois de Juda, et les os de ses chefs, les os des sacrificateurs, et les os des prophètes, et les os des habitants de Jérusalem, hors de leurs sépulcres.
\VS{2}Et on les étendra devant le soleil, et devant la lune, et devant toute l'armée des cieux, qui sont des choses qu'ils ont aimées, qu'ils ont servies et après lesquelles ils ont marché ; des choses qu'ils ont recherchées, et devant lesquelles ils se sont prosternés ; ils ne seront pas recueillis ni ensevelis, ils seront comme du fumier sur la face du sol.
\VS{3}Et la mort sera plus désirable que la vie pour tous ceux qui resteront de cette race mauvaise, ceux, dis-je, qui seront restés dans tous les lieux où je les aurai chassés, dit Yahweh des armées.
\VS{4}Dis-leur donc : Ainsi parle Yahweh : Si on tombe, ne se relève-t-on pas ? Et si on se détourne, ne revient-on pas ?
\VS{5}Pourquoi donc ce peuple de Jérusalem s'abandonne-t-il à de perpétuels égarements ? Ils tiennent ferme à la tromperie, et ils refusent de convertir.
\VS{6}Je suis attentif et j'écoute, mais nul ne parlent selon la justice ; il n'y a personne qui se repente de sa méchanceté, disant : Qu'ai-je fait ? Ils retournent tous vers les objets qui les entraînent, comme le cheval qui se jette avec impétuosité parmi la bataille.
\VS{7}Même la cigogne connaît dans les cieux ses saisons ; la tourterelle et l'hirondelle, et la grue observent le temps où elles doivent venir ; mais mon peuple ne connaît pas les ordonnances de Yahweh.
\VS{8}Comment dites-vous : Nous sommes les sages, et la loi de Yahweh est avec nous ? Voilà, certes on a agi faussement, et la plume des scribes est une plume de fausseté.
\VS{9}Les sages sont confus, ils sont épouvantés et pris ; car ils ont rejeté la parole de Yahweh, et quelle sagesse ont-ils ?
\VS{10}C'est pourquoi je donnerai leurs femmes à d'autres, et leurs champs à des gens qui les posséderont en héritage. Car depuis le plus petit jusqu'au plus grand, chacun s'adonne au gain déshonnête, tant le prophète que le sacrificateur, tous agissent faussement.
\VS{11}Ils pansent à la légère la plaie de la fille de mon peuple, en disant : Paix ! Paix ! Et il n'y a pas de paix.
\VS{12}Sont-ils confus d'avoir commis des abominations ? Ils n'en ont même aucune honte, et ils ne savent pas ce que c'est que de rougir ; c'est pourquoi ils tomberont parmi ceux qui tombent, ils seront renversés au temps où je les visiterai, dit Yahweh.
\VS{13}Je les ramasserai, j'en finirai avec eux, dit Yahweh ; il n'y aura plus de raisins à la vigne, et il n'y aura plus de figues au figuier, les feuilles se flétriront ; et ce que je leur avais donné sera transporté avec eux.
\VS{14}Pourquoi restons-nous assis ? Assemblez-vous et entrons dans les villes fortes, et nous serons là en repos ! Car Yahweh, notre Dieu, nous réduit au silence, et il nous fait boire des eaux empoisonnées, parce que nous avons péché contre Yahweh.
\VS{15}On attendait la paix, et il n'y a rien de bon ; on attend le temps de guérison, et voici la terreur !
\VS{16}Le hennissement de ses chevaux se fait entendre de Dan, et tout le pays tremble au bruit des hennissements de ses puissants chevaux ; ils viennent et dévorent le pays et ce qu'il contient, la ville et ceux qui l'habitent.
\VS{17}Qui plus est, voici, j'envoie contre vous des serpents, des basilics, contre lesquels il n'y a pas d'enchantement, et ils vous mordront, dit Yahweh.
\VS{18}J'ai voulu prendre des forces pour soutenir la douleur, mais mon cœur est languissant au dedans de moi.
\VS{19}Voici la voix du cri de la fille de mon peuple, qui crie d'un pays éloigné : Yahweh n'est-il plus à Sion ? Son Roi n'est-il plus au milieu d'elle ? Pourquoi m'ont-ils irrité par leurs images taillées, par les vanités\FTNT{Idoles que Dieu appelle vanité, vapeur ou souffle} (2) étrangères ?
\VS{20}La moisson est passée, l'été est fini, et nous ne sommes pas sauvés !
\VS{21}Je suis brisé par la blessure de la fille de mon peuple, je suis sombre, l'épouvante me saisit.
\VS{22}N'y a-t-il pas de baume en Galaad ? N'y a-t-il pas là de médecin ? Pourquoi donc la guérison de la fille de mon peuple ne s'opère-t-elle pas ?
\Chap{9}
\TextTitle{Jérémie pleure sur son peuple}
\VerseOne{}Plaise à Dieu que ma tête soit comme un réservoir d'eau, et que mes yeux soient une vive fontaine de larmes, et je pleurerais jour et nuit les blessés à mort de la fille de mon peuple !
\VS{2}Plaise à Dieu que j'aie au désert une cabane de voyageurs, j'abandonnerais mon peuple, je m'en irais loin de lui ! Car ils sont tous des adultères, et une assemblée de perfides.
\VS{3}Ils ont tendu leur langue, qui a été comme leur arc pour décrocher le mensonge\FTNT{Ps. 64:3-4.} ; et ils se sont renforcés dans la terre contre la fidélité ; car ils sont allés de méchanceté en méchanceté, et ne m'ont pas reconnu, dit Yahweh.
\VS{4}Gardez-vous chacun de son intime ami, et ne vous confiez en aucun frère\FTNT{Mi. 7:5.} ; car tout frère fait métier de supplanter, et tout intime ami marche dans la calomnie.
\VS{5}Et chacun se moque de son intime ami, et on ne parle pas selon la vérité ; ils ont instruit leur langue à dire le mensonge, ils se tourmentent extrêmement pour faire le mal.
\VS{6}Ta demeure est au milieu de la tromperie ; ils refusent, à cause de la tromperie, de me connaître, dit Yahweh.
\VS{7}C'est pourquoi, ainsi parle Yahweh des armées : Voici, je vais les fondre, je les éprouverai\FTNT{Mal. 3:3.}. Car comment en agirais-je autrement à l'égard de la fille de mon peuple ?
\VS{8}Leur langue est une flèche meurtrière, elle profère des tromperies ; chacun de sa bouche parle de la paix avec son ami, mais au-dedans il lui dresse des embûches\FTNT{Ps. 12:3; Ps. 28:3.}.
\VS{9}Ne les punirais-je pas pour ces choses-là, dit Yahweh ? Mon âme ne se vengerait-elle pas d'une telle nation ?
\VS{10}J'élèverai ma voix avec larmes, et je prononcerai à haute voix une lamentation à cause des montagnes, et une complainte à cause des cabanes du désert, parce qu'elles sont brûlées, de sorte que personne n'y passe et qu'on n'y entend plus la voix des troupeaux ; les oiseaux des cieux et le bétail ont fui, ils s'en sont allés.
\VS{11}Et je ferai de Jérusalem des monceaux de ruines, elle sera un repaire de serpents, et je ferai des villes de Juda un désert sans habitants.
\VS{12}Qui est l'homme sage qui comprenne ceci ? Qui est celui à qui la bouche de Yahweh a parlé ? Qu'il le déclare et qu'il dise pourquoi le pays est-il détruit, brûlé comme un désert, sans que personne y passe ?
\VS{13}Yahweh donc dit : Parce qu'ils ont abandonné ma loi que j'avais mise devant eux ; parce qu'ils n'ont pas écouté ma voix, et qu'ils n'ont pas marché selon elle ;
\VS{14}mais parce qu'ils ont marché suivant les penchants de leur cœur, et après les Baals, comme leurs pères le leur ont enseigné.
\VS{15}C'est pourquoi, ainsi parle Yahweh des armées, le Dieu d'Israël : Voici, je vais faire manger de l'absinthe à ce peuple-ci, et je leur ferai boire des eaux empoisonnées.
\VS{16}Je les disperserai parmi les nations que n'ont connues ni eux ni leurs pères, et j'enverrai après eux l'épée, jusqu'à ce que je les aie exterminés.
\VS{17}Ainsi parle Yahweh des armées : Considérez, et appelez des pleureuses, afin qu'elles viennent, et mandez les femmes sages, et qu'elles viennent !
\VS{18}Qu'elles se hâtent, et qu'elles prononcent à haute voix une lamentation sur nous ! Et que nos larmes tombent de nos yeux et que l'eau coule de nos paupières !
\VS{19}Car une voix de lamentation se fait entendre de Sion, disant : Eh quoi ! Nous sommes dévastés ! Nous sommes couverts de honte ! Car nous avons abandonné le pays, car nos demeures nous ont jetés dehors !
\VS{20}C'est pourquoi, vous, femmes, écoutez la parole de Yahweh, et que votre oreille reçoive la parole de sa bouche ! Enseignez vos filles à se lamenter, et chacune sa compagne à faire des complaintes !
\VS{21}Car la mort est montée par nos fenêtres, elle est entrée dans nos palais, pour exterminer les enfants dans les rues, et les jeunes hommes dans les places.
\VS{22}Dis : Ainsi parle Yahweh : Même les cadavres des hommes tomberont comme du fumier sur le dessus des champs, et comme une gerbe après le moissonneur, sans que personne les ramasse !
\VS{23}Ainsi parle Yahweh : Que le sage ne se glorifie pas de sa sagesse, que le fort ne se glorifie pas de sa force, et que le riche ne se glorifie pas de sa richesse.
\VS{24}Mais que celui qui se glorifie, se glorifie d'avoir de l'intelligence et de me connaître, car je suis Yahweh, qui fais miséricorde, droit et justice sur la terre ; car je prends plaisir en ces choses-là, dit Yahweh\FTNT{Ps. 62:10 ; 1 Co. 1:31 ; 2 Co. 10:17 ; 1 Ti. 6:17.}.
\VS{25}Voici, les jours viennent, dit Yahweh, où je punirai tout circoncis incirconcis,
\VS{26}l'Egypte, Juda, Edom, les fils d'Ammon, Moab, et tous ceux qui se coupent les coins de leur barbe et qui habitent dans le désert ; car toutes les nations sont incirconcises, et toute la maison d'Israël a le cœur incirconcis.
\Chap{10}
\TextTitle{Dénonciation de l'idolâtrie en Israël}
\VerseOne{}Ecoutez la parole que Yahweh vous adresse, maison d'Israël !
\VS{2}Ainsi parle Yahweh : N'apprenez pas les façons de faire des nations\FTNT{Lé. 18:3 ; De. 12:30.}, et ne craignez pas les signes des cieux, parce que les nations les craignent.
\VS{3}Car les lois des peuples ne sont que vanité\FTNT{Les lois des peuples, ou encore statuts, coutumes, ordonnances ne sont que vanité. Nous devons nous soumettre aux lois des nations tant que celles-ci ne s'opposent pas à la Loi de Dieu (1 Pi. 2 :13). Quand celles-ci sont contraires aux règles morales établies par le Seigneur, nous devons obéir à Dieu, car il vaut mieux obéir à Dieu plutôt qu'aux hommes (Ac. 4:19 ; Ac. 5:29).}. On coupe le bois dans la forêt ; la main de l'ouvrier le travaille avec la hache\FTNT{Es. 40 :20 ; Es. 44 :12-18.} ;
\VS{4}on l'embellit avec de l'argent et de l'or, on le fait tenir avec des clous et à coups de marteau, afin qu'il ne vacille pas.
\VS{5}Ils sont façonnés tout droits comme des colonnes massives, et ils ne parlent pas ; on les porte par nécessité, parce qu'ils ne peuvent pas marcher. Ne les craignez pas, car ils ne sauraient faire aucun mal, et aussi ils sont incapables de faire du bien.
\VS{6}Nul n'est semblable à toi, ô Yahweh ! Tu es grand, et ton Nom est grand par ta puissance.
\VS{7}Qui ne te craindrait, Roi des nations ? Car cela t'est dû ; car, parmi tous les sages des nations et dans tous leurs royaumes, nul n'est semblable à toi\FTNT{Ap. 15:4.}.
\VS{8}Et ils sont tous ensemble stupides et insensés ; le bois ne leur enseigne que des vanités\FTNT{Ha. 2:18.}.
\VS{9}L'argent qui est étendu en plaques est apporté de Tarsis, et l'or d'Uphaz, pour être mis en œuvre par l'ouvrier et par les mains du fondeur ; et la pourpre et l'écarlate sont leur vêtement ; toutes ces choses sont l'ouvrage de gens habiles.
\VS{10}Mais Yahweh est le Dieu de vérité, c'est le Dieu vivant et le Roi éternel ; la terre tremble devant sa colère, et les nations ne supportent pas sa fureur.
\VS{11}Vous leur parlerez ainsi : Les dieux qui n'ont pas fait les cieux et la terre périront de la terre et de dessous les cieux.
\VS{12}Mais Yahweh est celui qui a fait la terre par sa puissance, qui a fondé le monde habitable par sa sagesse, et qui a étendu les cieux par son intelligence.
\VS{13}Sitôt qu'il fait retentir sa voix, il y a un tumulte d'eaux dans les cieux ; il fait monter les vapeurs des extrémités de la terre, il fait les éclairs et la pluie, et il fait sortir le vent de ses réservoirs.
\VS{14}Tout homme devient stupide par sa connaissance, tout fondeur est honteux par les images taillées ; car les idoles en métal fondu ne sont que mensonge, il n'y a pas de souffle en elles ;
\VS{15}elles ne sont que vanité, une œuvre de tromperie ; elles périront au temps de leur châtiment.
\VS{16}La portion de Jacob n'est pas comme ces choses-là ; car c'est lui qui a tout formé, et Israël est la tribu de son héritage. Son Nom est Yahweh des armées.
\VS{17}Toi qui es assise dans la détresse, rassemble du pays tes paquets !
\VS{18}Car ainsi parle Yahweh : Voici, cette fois je vais lancer au loin, comme avec une fronde, les habitants du pays ; je vais les mettre à l'étroit, afin qu'on les atteigne.
\VS{19}Malheur à moi, diront-ils, à cause de ma blessure ! Ma plaie est douloureuse ! Mais moi, je dis : Quoi qu'il en soit, c'est une maladie qu'il faut que je supporte.
\VS{20}Ma tente est dévastée, tous mes cordages sont rompus ; mes fils m'ont quittée, et ils ne sont plus ; il n'y a plus personne qui dresse ma tente, qui relève mes pavillons.
\VS{21}Car les pasteurs ont été stupides, ils n'ont pas cherché Yahweh ; c'est pour cela qu'ils n'ont pas réussi et que tous leurs troupeaux s'éparpillent.
\VS{22}Voici, une rumeur se fait entendre ; avec une grande secousse qui vient du pays du nord, pour faire des villes de Juda un désert, un repaire de serpents.
\VS{23}Yahweh ! Je sais que la voie de l'homme ne dépend pas de lui\FTNT{Pr. 16:1.}, et qu'il n'est pas au pouvoir de l'homme qui marche de diriger ses pas.
\VS{24}Ô Yahweh ! Châtie-moi, mais avec équité, et non dans ta colère, de peur que tu ne me réduises à rien\FTNT{Es. 27:8 ; Ps. 38:2.}.
\VS{25}Répands ta fureur sur les nations qui ne te connaissent pas, et sur les familles qui n'invoquent pas ton Nom ! Car ils ont dévoré Jacob, ils l'ont, dis-je, dévoré et consumé, et ils ont mis en désolation son agréable demeure.
\Chap{11}
\TextTitle{Yahweh dénonce la prostitution de Juda}
\VerseOne{}La parole fut adressée à Jérémie de la part de Yahweh, en disant :
\VS{2}Ecoutez les paroles de cette alliance, et parlez aux hommes de Juda et aux habitants de Jérusalem !
\VS{3}Dis-leur : Ainsi parle Yahweh, le Dieu d'Israël : Maudit soit l'homme qui n'écoute pas les paroles de cette alliance\FTNT{De. 27:26 ; Ga. 3:10.},
\VS{4}que j'ai ordonnée à vos pères, le jour où je les ai fait sortir du pays d'Egypte, de la fournaise de fer, en disant : Ecoutez ma voix et faites toutes les choses que je vous ordonnerai ; alors vous serez mon peuple, et je serai votre Dieu\FTNT{Lé. 26:12 ; De. 4:20.},
\VS{5}afin que j'accomplisse le serment que j'ai juré à vos pères, de leur donner un pays où coulent le lait et le miel, comme vous le voyez aujourd'hui. Et je répondis et dis : Amen ! Ô Yahweh !
\VS{6}Puis Yahweh me dit : Crie toutes ces paroles dans les villes de Juda et dans les rues de Jérusalem, en disant : Ecoutez les paroles de cette alliance et observez-les !
\VS{7}Car j'ai averti vos pères, depuis le jour où je les ai fait monter du pays d'Egypte jusqu'à ce jour, je les ai avertis dès le matin, en disant : Ecoutez ma voix !
\VS{8}Mais ils n'ont pas écouté, ils n'ont pas prêté l'oreille, ils ont marché chacun suivant les penchants de leur mauvais cœur ; c'est pourquoi j'ai fait venir sur eux toutes les paroles de cette alliance, que je leur avais donné l'ordre d'observer, et qu'ils n'ont pas observée.
\VS{9}Yahweh me dit : Il y a une conspiration entre les hommes de Juda et entre les habitants de Jérusalem.
\VS{10}Ils sont retournés aux iniquités de leurs premiers pères, qui ont refusé d'écouter mes paroles, et ils sont allés après d'autres dieux pour les servir. La maison d'Israël et la maison de Juda ont rompu mon alliance, que j'avais faite avec leurs pères.
\VS{11}C'est pourquoi ainsi parle Yahweh : Voici, je fais venir sur eux un mal dont ils ne pourront sortir. Ils crieront vers moi, et je ne les écouterai pas\FTNT{Es. 1:15 ; Ez. 8:18 ; Mi. 3:4 ; Pr. 1:28.}.
\VS{12}Et les villes de Juda et les habitants de Jérusalem s'en iront et crieront vers les dieux auxquels ils brûlent de l'encens, mais ces dieux-là ne les sauveront pas au temps de leur malheur.
\VS{13}Car, ô Juda ! Tu as eu autant de dieux que de villes ; et toi, Jérusalem, tu as dressé autant d'autels aux choses honteuses que tu as de rues, des autels, dis-je, pour brûler de l'encens à Baal\FTNT{Ez. 16:24-31 ; Ac. 17:23.}…
\VS{14}Toi donc, n'intercède pas pour ce peuple, et n'élève pour eux ni cri ni prière ; car je ne les écouterai pas au temps où ils crieront vers moi dans leur malheur.
\VS{15}Qu'est-ce que mon bien-aimé a à faire dans ma maison, que tant de gens se servent d'elle pour y faire leurs complots? la chair sainte est transportée loin de toi, et encore quand tu fais le mal, c'est alors que tu triomphes !
\VS{16}Yahweh avait appelé ton nom Olivier verdoyant et beau par la forme de ton fruit ; mais au bruit d'un grand fracas, il y a mis le feu, et ses rameaux sont brisés.
\VS{17}Yahweh des armées, qui t'a plantée, prononce le mal contre toi, à cause de la méchanceté de la maison d'Israël et de la maison de Juda, qui ont agi pour m'irriter, en brûlant de l'encens à Baal.
\TextTitle{Jugement des ennemis de Jérémie}
\VS{18}Et Yahweh me l'a fait savoir, et je l'ai su ; alors tu m'as fait voir leurs actions.
\VS{19}Mais moi, comme un agneau, ou comme un bœuf qu'on mène pour être égorgé, je ne savais pas qu'ils projetaient de mauvais desseins contre moi, en disant : Détruisons l'arbre avec son fruit ! Exterminons-le de la terre des vivants, et qu'on ne se souvienne plus de son nom !
\VS{20}Mais toi, Yahweh des armées, qui juges justement, et qui éprouve les reins et le cœur ! Fais que je voie ta vengeance s'exercer contre eux, car je t'ai découvert ma cause\FTNT{1 S. 16:7 ; Ps. 26:2 ; 1 Ch. 28:9 ; Ap. 2:23.}.
\VS{21}C'est pourquoi ainsi parle Yahweh contre les gens d'Anathoth, qui cherchent ta vie et qui disent : Ne prophétise plus au Nom de Yahweh, et tu ne mourras pas par nos mains\FTNT{Es. 30:10 ; Mi. 2:6.} !
\VS{22}C'est pourquoi donc ainsi parle Yahweh des armées : Voici, je vais les punir ; les jeunes hommes mourront par l'épée, leurs fils et leurs filles mourront par la famine.
\VS{23}Et il ne restera rien d'eux ; car je ferai venir le mal sur les gens d'Anathoth, l'année de leur châtiment.
\Chap{12}
\TextTitle{Prière de Jérémie et réponse de Yahweh}
\VerseOne{}Yahweh, quand je contesterai avec toi, tu seras trouvé juste ; mais toutefois j'entrerai en contestation avec toi : Pourquoi la voie des méchants est-elle prospère ? Pourquoi tous les perfides vivent-ils en paix\FTNT{Job. 21:7-9 ; Ro. 3:4.} ?
\VS{2}Tu les as plantés, et ils ont pris racine, ils s'avancent, et ils portent du fruit. Tu es près de leur bouche, mais tu es loin de leurs cœurs\FTNT{Es. 29:13 ; Job. 21:7-8.}.
\VS{3}Mais, ô Yahweh, tu me connais, tu me vois, tu éprouves mon cœur qui est avec toi. Traîne-les comme des brebis qu'on mène pour être égorgées, et mets-les à part pour le jour de la tuerie !
\VS{4}Jusqu'à quand le pays mènera-t-il deuil, et l'herbe de tous les champs séchera-t-elle à cause de la méchanceté des habitants qui sont en la terre ? Les bêtes et les oiseaux ont été consumés par la disette, parce que ces méchants ont dit : On ne verra pas notre dernière fin. 
\VS{5}Si tu cours avec des piétons et qu'ils te fatiguent, comment lutteras-tu avec les chevaux ? Et si tu te crois en sûreté dans une terre de paix, que feras-tu devant l'orgueil du Jourdain ?
\VS{6} Certainement, mêmes tes frères et la maison de ton père, ceux-là mêmes ont agi perfidement contre toi, eux-mêmes ont crié après toi à plein gosier ; ne les crois point, quoiqu'ils te parlent amicalement\FTNT{Pr. 26:25.}.
\VS{7}J'ai abandonné ma maison, j'ai quitté mon héritage, ce que mon âme aimait le plus je l'ai livré aux mains de ses ennemis.
\VS{8}Mon héritage a été pour moi comme un lion dans la forêt, il a poussé contre moi ses rugissements ; c'est pourquoi je l'ai pris en haine.
\VS{9}Mon héritage a-t-il donc été pour moi comme un oiseau de proie tacheté ? Les oiseaux de proie ne sont-ils pas autour de lui ? Venez, assemblez-vous, vous tous les animaux des champs, venez pour le dévorer\FTNT{Es. 56:9.} !
\VS{10}Plusieurs pasteurs ravagent ma vigne, ils foulent mon champ ; ils réduisent le champ de mes délices en un désert, en une désolation.
\VS{11}Ils le réduisent en un désert ; il est en deuil, il est désolé devant moi. Tout le pays est ravagé, car nul n'y prend garde.
\VS{12}Les destructeurs viennent sur tous les lieux élevés du désert, car l'épée de Yahweh dévore le pays d'un bout à l'autre ; il n'y a de paix pour aucune chair.
\VS{13}Ils ont semé du froment, et ils moissonnent des épines, ils se sont fatigués sans profit. Soyez honteux de vos récoltes, à cause de l'ardeur de la colère de Yahweh\FTNT{Lé. 26:16.}.
\VS{14}Ainsi parle Yahweh contre tous mes mauvais voisins, qui mettent la main sur l'héritage que j'ai donné à mon peuple d'Israël : Voici, je les arracherai de leur pays, et j'arracherai la maison de Juda du milieu d'eux.
\VS{15}Mais il arrivera qu'après que je les avoir arrachés, j'aurai encore compassion d'eux, et je les ramènerai chacun dans son héritage, chacun dans son pays\FTNT{De. 30:3.}.
\VS{16}Et il arrivera que s'ils apprennent bien les voies de mon peuple, pour jurer par mon Nom, en disant : Yahweh est vivant ! Comme ils ont enseigné à mon peuple à jurer par Baal, ils seront édifiés au milieu de mon peuple.
\VS{17}Mais s'ils n'écoutent pas, j'arracherai entièrement une telle nation, et je la ferai périr, dit Yahweh\FTNT{Es. 60:12.}.
\Chap{13}
\TextTitle{La ceinture pourrie, illustration du jugement}
\VerseOne{}Ainsi m'a parlé Yahweh : Va, et achète-toi une ceinture de lin et mets-la sur tes reins ; et ne la mets pas dans l'eau.
\VS{2}J'achetai donc une ceinture, selon la parole de Yahweh, et je la mis sur mes reins.
\VS{3}Et la parole de Yahweh me fut adressée pour la seconde fois, en disant :
\VS{4}Prends la ceinture que tu as achetée et qui est sur tes reins ; lève-toi, va-t'en vers l'Euphrate, et là, cache-la dans la fente d'un rocher.
\VS{5}J'allai donc et je la cachai près de l'Euphrate, comme Yahweh me l'avait ordonné.
\VS{6}Et il arriva que plusieurs jours après Yahweh me dit : Lève-toi, va vers l'Euphrate et reprends la ceinture que je t'avais ordonné d'y cacher.
\VS{7}Et j'allai vers l'Euphrate, je creusai, et je pris la ceinture dans le lieu où je l'avais cachée ; mais voici, la ceinture était pourrie, elle n'était plus bonne à rien.
\VS{8}Alors la parole de Yahweh me fut adressée, en disant :
\VS{9}Ainsi parle Yahweh : Je ferai ainsi pourrir l'orgueil de Juda et le grand orgueil de Jérusalem.
\VS{10}L'orgueil de ce peuple très méchant, qui refuse d'écouter mes paroles, qui marche selon les penchants de son cœur, et qui va après d'autres dieux, pour les servir et pour se prosterner devant eux, qu'il devienne comme cette ceinture qui n'est plus bonne à rien !
\VS{11}Car comme une ceinture est attachée aux reins d'un homme, ainsi je m'étais attaché toute la maison d'Israël et toute la maison de Juda, dit Yahweh, afin qu'elles soient mon peuple, mon Nom, ma louange, et ma gloire. Mais ils ne m'ont pas écouté.
\VS{12}Tu leur diras donc cette parole-ci : Ainsi parle Yahweh, le Dieu d'Israël : Toute outre sera remplie de vin. Et ils te diront : Ne savons-nous pas que toute outre sera remplie de vin ?
\VS{13}Mais tu leur diras : Ainsi parle Yahweh : Voici, je vais remplir d'ivresse tous les habitants de ce pays, les rois qui sont assis sur le trône de David, les sacrificateurs, les prophètes, et tous les habitants de Jérusalem.
\VS{14}Et je les briserai les uns contre les autres, les pères et les fils ensemble, dit Yahweh\FTNT{Es. 51:17-20 ; Ps. 60:5.} ; je n'aurai pas de compassion, je n'épargnerai pas, et je n'aurai pas de miséricorde ; rien ne m'empêchera de les détruire.
\VS{15}Écoutez et prêtez l'oreille ! Ne vous élevez pas ! Car Yahweh parle.
\VS{16}Donnez gloire à Yahweh, votre Dieu, avant qu'il fasse venir les ténèbres, avant que vos pieds se heurtent contre les montagnes du crépuscule ; vous attendrez la lumière, et il la changera en ombre de la mort, il la réduira en obscurité profonde\FTNT{Es. 59:9 ; Jn. 12:35.}.
\VS{17}Que si vous n'écoutez pas ceci, mon âme pleurera en secret, à cause de votre orgueil ; mes yeux verseront des larmes en abondance, ils se fondront en larmes, parce que le troupeau de Yahweh sera emmené captif\FTNT{La. 1:2-16.}.
\VS{18}Dis au roi et à la reine : Humiliez-vous et asseyez-vous sur la cendre ! Car elle est tombée de vos têtes, la couronne de votre gloire.
\VS{19}Les villes du midi sont fermées, il n'y a personne qui les ouvre ; tout Juda est transporté en captivité, il est transporté entièrement.
\VS{20}Levez vos yeux et voyez ceux qui viennent du nord. Où est le troupeau qui t'avait été donné, le troupeau qui faisait ta gloire ?
\VS{21}Que diras-tu quand il te punit ? Car tu les as enseignés à dominer en maîtres sur toi. Les douleurs ne te saisiront-elles pas, comme elles saisissent une femme qui enfante ?
\VS{22}Que si tu dis en ton cœur : Pourquoi cela m'arrive-t-il ? C'est à cause de la multitude de tes iniquités que les pans de ta robe sont relevés, et que tes talons sont violemment mis à nu\FTNT{Es. 47:2-3.}.
\VS{23}L'éthiopien peut-il changer sa peau et le léopard ses taches ? Pourriez-vous, aussi, faire quelque bien, vous qui êtes accoutumés à faire le mal ?
\VS{24}C'est pourquoi je les disperserai, comme du chaume, qui est emporté çà et là par le vent du désert.
\VS{25}Voilà ton sort, la portion que je te mesure, dit Yahweh, parce que tu m'as oublié, et que tu as mis ta confiance dans le mensonge.
\VS{26}A cause de cela, je relèverai les pans de ta robe sur ton visage, et ta honte se verra.
\VS{27}Tes adultères et tes hennissements, l'énormité de tes prostitutions sur les collines et dans les champs, tes abominations, je les ai vues. Malheur à toi, Jérusalem ! Ne seras-tu pas purifiée ? Jusqu'à quand cela durera-t-il ?
\Chap{14}
\TextTitle{Le pays frappé par la sécheresse}
\VerseOne{}La parole de Yahweh, qui fut adressée à Jérémie, à l'occasion de la sécheresse.
\VS{2}Juda est dans le deuil, et ses portes sont dans un état pitoyable. Ils sont tous en deuil, gisant par terre ; et les cris de Jérusalem montent au ciel.
\VS{3}Et les personnes distinguées envoient les petits chercher de l'eau, et les petits vont aux citernes, ne trouvent pas d'eau, et reviennent leurs vases vides ; ils sont honteux et confus, ils couvrent leur tête.
\VS{4}Parce que la terre est crevassée, parce qu'il n'y a pas eu de pluie dans le pays, les laboureurs sont honteux, ils se couvrent la tête.
\VS{5}Même la biche met bas son faon dans le champ et l'abandonne, parce qu'il n'y a pas d'herbe.
\VS{6}Et les ânes sauvages se tiennent sur les lieux élevés, humant l'air comme des serpents ; leurs yeux se consument, parce qu'il n'y a pas d'herbe.
\VS{7}Si nos iniquités témoignent contre nous, agis à cause de ton Nom, ô Yahweh\FTNT{Es. 59:12.} ! Car nos infidélités sont nombreuses, c'est contre toi que nous avons péché.
\VS{8}Toi qui es l'espérance d'Israël, son sauveur au temps de la détresse, pourquoi serais-tu dans le pays comme un étranger, comme un voyageur qui se détourne pour passer la nuit ?
\VS{9}Pourquoi serais-tu comme un homme stupéfait, et comme un héros qui ne peut sauver ? Or tu es au milieu de nous, ô Yahweh, et ton Nom est invoqué sur nous : Ne nous abandonne pas !
\VS{10}Voici ce que Yahweh dit de ce peuple : Parce qu'ils aiment à errer ainsi çà et là, et qu'ils ne savent retenir leurs pieds, Yahweh ne prend pas plaisir en eux, il se souvient maintenant de leurs iniquités, et il punit leurs péchés\FTNT{Os. 8:13.}.
\VS{11}Puis Yahweh me dit : N'intercède pas en faveur de ce peuple.
\VS{12}Quand ils jeûnent, je n'écouterai pas leurs cris ; et quand ils offrent des holocaustes et des offrandes, je n'y prendrai pas plaisir ; mais je les consumerai par l'épée, par la famine et par la peste.
\VS{13}Et je répondis : Ah ! ah ! Seigneur Yahweh ! Voici, les prophètes leur disent : Vous ne verrez pas l'épée, et vous n'aurez pas de famine ; mais je vous donnerai dans ce lieu-ci une paix assurée.
\VS{14}Et Yahweh me dit : C'est le mensonge ce que ces prophètes prophétisent en mon Nom ; je ne les ai pas envoyés, je ne leur ai pas donné d'ordre, je ne leur ai pas parlé ; ils vous prophétisent des visions de mensonge, des divinations, de l'idolâtrie et des tromperies de leur cœur\FTNT{De. 18:20-22 ; Ez. 13:2-3.}.
\VS{15}C'est pourquoi ainsi parle Yahweh sur les prophètes qui prophétisent en mon Nom, sans que je les ai envoyés, et qui disent : Il n'y aura ni épée ni la famine dans ce pays : Ces prophètes-là seront consumés par l'épée et par la famine.
\VS{16}Et le peuple à qui ils prophétisent sera jeté dans les rues de Jérusalem à cause de la famine et de l'épée ; et il n'y aura personne pour les enterrer, ni eux, ni leurs femmes, ni leurs fils, ni leurs filles ; je répandrai sur eux leur méchanceté.
\VS{17}Tu leur diras donc cette parole-ci : Que mes yeux se fondent en larmes nuit et jour, et qu'ils ne cessent pas\FTNT{La. 1:16.} ; car la vierge, fille de mon peuple, a été frappée d'un grand coup, d'une plaie très douloureuse.
\VS{18}Si je sors dans les champs, voici les gens tués par l'épée ; si j'entre dans la ville, voici les gens consumés par la faim ; même le prophète et le sacrificateur parcourent le pays, sans savoir où ils vont.
\VS{19}As-tu entièrement rejeté Juda, et ton âme a-t-elle Sion en horreur ? Pourquoi nous frappes-tu sans qu'il y ait pour nous de guérison ? On attend la paix, mais il n'y a rien de bon, un temps de guérison, et voici la terreur !
\VS{20}Yahweh, nous reconnaissons notre méchanceté, l'iniquité de nos pères ; car nous avons péché contre toi\FTNT{Ps. 106:6 ; Da. 9:8.}.
\VS{21}Ne nous rejette pas, à cause de ton Nom, et ne déshonore pas le trône de ta gloire ! Souviens-toi de ton alliance avec nous, et ne la romps pas !
\VS{22}Parmi les vanités\FTNT{Ce terme veut aussi dire « idole ».} des nations, y en a-t-il qui fassent pleuvoir, et les cieux donnent-ils des ondées\FTNT{Es. 30:23 ; Ac. 14:17.} ? N'est-ce pas toi, ô Yahweh, notre Dieu ? C'est pourquoi nous nous attendons à toi, car c'est toi qui as fait toutes ces choses.
\Chap{15}
\TextTitle{Yahweh fermement décidé à juger son peuple}
\VerseOne{}Et Yahweh me dit : Quand Moïse et Samuel se tiendraient devant moi, je n'aurais pourtant point d'affection pour ce peuple ; chasse-les de devant ma face, et qu'ils sortent.
\VS{2}Que s'ils te disent : Où irons-nous ? Tu leur répondras : Ainsi parle Yahweh : Ceux qui sont destinés à la mort iront à la mort ; et ceux qui sont destinés à l’épée iront à l’épée ; et ceux qui sont destinés à la famine, iront à la famine ; et ceux qui sont destinés à la captivité iront en captivité\FTNT{Za. 11:9.} !
\VS{3}J'établirai aussi sur eux quatre espèces de punitions, dit Yahweh, l'épée pour tuer, et les chiens pour traîner, et les oiseaux des cieux, et les bêtes de la terre pour dévorer et pour détruire. 
\VS{4}Et je les livrerai à être agités par tous les Royaumes de la terre, à cause de Manassé, fils d'Ezéchias, Roi de Juda, pour les choses qu'il a faites dans Jérusalem. 
\VS{5}Car qui aurait compassion de toi, Jérusalem, ou qui te plaindrait ? Ou qui se détournerait pour s'informer de ta paix ? 
\VS{6}Tu m'as abandonné, dit Yahweh, et tu t'en es allée en arrière ; c'est pourquoi j'étends ma main sur toi, et je te détruis, je suis las d'avoir compassion.
\VS{7}Je les vanne avec un van aux portes du pays\FTNT{Mt. 3:12.} ; je prive d'enfants, je fais périr mon peuple, et ils ne se sont pas détournés de leurs voies.
\VS{8}Je multiplie ses veuves plus que le sable de la mer ; je fais venir sur eux, sur la mère du jeune homme, le dévastateur en plein midi ; je fais tomber subitement sur elle l'angoisse et les frayeurs.
\VS{9}Celle qui en avait enfanté sept languit, elle rend l'âme ; son soleil se couche pendant qu'il est encore jour\FTNT{Am. 8:9.} ; elle est confuse, couverte de honte. Ceux qui restent, je les livre à l'épée devant leurs ennemis, dit Yahweh.
\VS{10}Malheur à moi, ô ma mère, de ce que tu m'as enfanté\FTNT{Job. 3:1-2.} pour être un homme de contestation et un homme de dispute pour tout le pays ! Je n'emprunte ni ne prête, et néanmoins tous me maudissent et me méprisent.
\VS{11}Alors Yahweh dit : En vérité tout ira bien pour ton reste ; en vérité je ferai que l’ennemi te traite bien au temps du malheur, et au temps de la détresse.
\VS{12}Le fer brisera-t-il le fer du nord et l'airain ?
\VS{13}Je livre au pillage, sans en faire le prix, tes richesses et tes trésors, et cela à cause de tous tes péchés, sur tout ton territoire.
\VS{14}Je te fais passer avec tes ennemis dans un pays que tu ne connais pas, car le feu de ma colère s'est allumé, il brûle sur vous\FTNT{De. 32:22.}.
\TextTitle{La mise à part de Jérémie}
\VS{15}Yahweh ! Tu sais tout, souviens-toi de moi, visite-moi, venge-moi de ceux qui me persécutent\FTNT{Ps. 106:4.} ! Ne m'enlève pas, tandis que tu te montres lent à la colère ! Sache que je supporte l'opprobre à cause de toi.
\VS{16}J'ai trouvé tes paroles, je les ai aussitôt dévorées\FTNT{Ez. 3:3 ; Ap. 10:9.} ; tes paroles ont fait la joie et l'allégresse de mon cœur ; car ton Nom est invoqué sur moi, ô Yahweh, Dieu des armées !
\VS{17}Je ne me suis pas assis dans l'assemblée des moqueurs, et je ne m'y suis pas réjoui ; mais je me suis assis tout seul à cause de ta main, car tu me remplissais d'indignation.
\VS{18}Pourquoi ma douleur est-elle continuelle ? Pourquoi ma plaie est-elle incurable et refuse-t-elle d'être guérie ? Serais-tu pour moi comme une source trompeuse, comme des eaux qui ne durent pas ?
\VS{19}C'est pourquoi ainsi parle Yahweh : Si tu reviens, je te ramènerai, et tu te tiendras devant moi ; et si tu sépares la chose précieuse de la méprisable, tu seras comme ma bouche. Qu'ils reviennent vers toi, mais toi, ne retourne pas vers eux.
\VS{20}Je ferai que tu sois pour ce peuple une muraille d'airain bien forte ; ils combattront contre toi, mais il n'auront pas le dessus contre toi ; car je suis avec toi pour te sauver et te délivrer, dit Yahweh.
\VS{21}Et je te délivrerai de la main des malins, et te rachèterai de la main des méchants.
\Chap{16}
\TextTitle{Célibat de Jérémie, illustration du jugement sur Juda}
\VerseOne{}Puis la parole de Yahweh me fut adressée, en disant :
\VS{2}Tu ne prendras pas de femme, et tu n'auras pas de fils ni de filles dans ce lieu-ci.
\VS{3}Car ainsi parle Yahweh sur les fils et les filles qui naîtront en ce lieu-ci, sur leurs mères qui les auront enfantés, et sur leurs pères qui les auront engendrés dans ce pays :
\VS{4}Ils mourront de maladie mortelle ; ils ne seront ni pleurés ni enterrés ; ils seront comme du fumier sur la face du sol ; ils seront consumés par l'épée et par la famine ; et leurs cadavres seront la pâture des oiseaux des cieux et des bêtes de la terre.
\VS{5}Car ainsi parle Yahweh : N'entre pas dans une maison de deuil, ne vas pas te lamenter ni te plaindre avec eux ; car j'ai retiré de ce peuple dit Yahweh, ma paix, ma miséricorde et mes compassions.
\VS{6}Et les grands et petits mourront dans ce pays ; ils ne seront pas enterrés ; on ne les pleurera pas, on ne se fera pas d'incision, et on ne se rasera pas pour eux\FTNT{Lé. 19:28 ; De. 14:1 ; Ez. 7:11}.
\VS{7}On ne rompra pas le pain dans le deuil pour consoler quelqu'un au sujet d'un mort, et on ne leur donnera pas à boire de la coupe de consolation pour leur père ou pour leur mère.
\VS{8}Aussi n'entre pas non plus dans une maison de festin pour t'asseoir avec eux, pour manger et pour boire.
\VS{9}Car ainsi parle Yahweh des armées, le Dieu d'Israël : Voici, je vais faire cesser dans ce lieu-ci, devant vos yeux et en vos jours, les cris de joie et les cris d'allégresse, la voix de l'époux et la voix de l'épouse.
\VS{10}Et il arrivera que quand tu annonceras à ce peuple toutes ces paroles-là, ils te diront : Pourquoi Yahweh parle-t-il de tout ce grand mal contre nous ? Quelle est notre iniquité ? Quel est le péché que nous avons commis contre Yahweh, notre Dieu ?
\VS{11}Et tu leur diras : Parce que vos pères m'ont abandonné, dit Yahweh, et sont allés après d'autres dieux, et les ont servis, et se sont prosternés devant eux, et m'ont abandonné et n'ont pas gardé ma loi ; 
\VS{12}et que vous avez fait le mal plus encore que vos pères. Car voici, chacun de vous marche selon les penchants de son mauvais cœur pour ne pas m'écouter.
\VS{13}A cause de cela, je vous jetterai de ce pays dans un pays que vous n'avez pas connu, ni vous ni vos pères ; et là, vous servirez jour et nuit les autres dieux, car je ne vous aurai pas fait grâce\FTNT{De. 28:64-65.}.
\VS{14}Néanmoins voici, les jours viennent, dit Yahweh, qu'on ne dira plus : Yahweh est vivant, lui qui a fait monter les fils d'Israël du pays d'Egypte !
\VS{15}Mais on dira : Yahweh est vivant, lui qui a fait monter les fils d'Israël du pays du nord et de tous les pays où il les avait chassés ; après que je les aurai ramenés dans leur pays, que j'avais donné à leurs pères.
\VS{16}Voici, j'envoie plusieurs pêcheurs, dit Yahweh, et ils les pêcheront ; et ensuite, j'enverrai plusieurs chasseurs, et ils les chasseront de toutes les montagnes et de toutes les collines, et des fentes des rochers.
\VS{17}Car mes yeux sont sur toutes leurs voies, elles ne sont pas cachées devant ma face, et leur iniquité n'est pas couverte devant mes yeux\FTNT{Pr. 5:21 ; Job. 34:21.}.
\VS{18}Mais premièrement je leur rendrai le double de leur iniquité et de leur péché, parce qu'ils ont souillé mon pays par les cadavres de leurs idoles, et parce qu'ils ont rempli mon héritage de leurs abominations.
\VS{19}Yahweh, qui est ma force et ma forteresse, et mon refuge au jour de la détresse ! Les nations viendront à toi des extrémités de la terre, et diront : Certes nos pères ont hérité le mensonge et la vanité, et les choses auxquelles il n'y a pas de profit.
\VS{20}L'homme se fera-t-il lui-même des dieux, qui ne sont pas dieux ?
\VS{21}C'est pourquoi voici, je leur fais connaître, cette fois, je leur fais connaître ma main et ma force ; et ils sauront que mon Nom est Yahweh.
\Chap{17}
\TextTitle{Le caractère sinueux du cœur}
\VerseOne{}Le péché de Juda est écrit avec un burin de fer, et avec une pointe de diamant ; il est gravé sur la table de leur cœur, et sur les cornes de leurs autels.
\VS{2}De sorte que leurs fils se souviennent de leurs autels, et de leurs poteaux d'Asherah, auprès des arbres verts sur les hautes collines.
\VS{3}Ma montagne, je livre par les champs tes richesses et tous tes trésors au pillage ; tes hauts lieux sont pleins de péché sur tout ton territoire.
\VS{4}Et toi, et ceux qui sont avec toi, vous laisserez vacant l'héritage que je t'avais donné ; et je t'asservirai à tes ennemis dans un pays que tu ne connais pas ; car vous avez allumé le feu de ma colère, et il brûlera à toujours.
\VS{5}Ainsi parle Yahweh : Maudit soit l'homme qui se confie dans l'homme, et qui fait de la chair sa force, et dont le cœur se retire de Yahweh !
\VS{6}Car il sera comme la bruyère dans le désert, et il ne voit pas venir le bien ; mais il demeure dans des lieux brûlés du désert, dans une terre salée et inhabitable.
\VS{7}Béni soit l'homme qui se confie en Yahweh, et dont Yahweh est l'espérance !
\VS{8}Il est comme un arbre planté près des eaux\FTNT{Ps. 23.}, et qui étend ses racines le long d'une eau courante ; quand la chaleur vient, il ne s'en aperçoit pas, et sa feuille reste verte ; il n'est pas en peine dans l'année de la sécheresse, et ne cesse de porter du fruit.
\VS{9}Le cœur est rusé et désespérément malin par-dessus tout : Qui peut le connaître\FTNT{Ps. 64:7.} ?
\VS{10}Je suis Yahweh, qui sonde le cœur, et qui éprouve les reins ; même pour rendre à chacun selon sa voie, et selon le fruit de ses actions.
\VS{11}Celui qui acquiert des richesses, sans observer la justice, est une perdrix qui couve ce qu'elle n'a pas pondu ; il les laissera au milieu de ses jours, et à la fin il sera trouvé insensé\FTNT{Ec. 4:8}.
\VS{12}Le lieu de notre sanctuaire est un trône de gloire, un lieu haut élevé dès le commencement.
\VS{13}Yahweh, qui es l'espérance d'Israël ! Tous ceux qui t'abandonnent seront honteux : Ceux qui se détournent de moi seront écrits sur la terre, car ils abandonnent la source des eaux vives, Yahweh\FTNT{Es. 1:28 ; Ps. 73:28.}.
\VS{14}Yahweh, guéris-moi, et je serai guéri ; sauve-moi, et je serai sauvé ; car tu es ma louange.
\VS{15}Voici, ceux-ci me disent : Où est la parole de Yahweh ? Qu'elle vienne présentement\FTNT{Es. 5:19 ; Ez. 12:23 ; 2 Pi. 3:3-4.} !
\VS{16}Mais je ne me suis pas avancé plus qu’un pasteur après toi, je n'ai pas non plus désiré le jour du malheur, tu le sais ; et ce qui est sorti de mes lèvres est présent devant toi.
\VS{17}Ne sois pas pour moi un sujet d'effroi, toi, mon refuge au jour du malheur !
\VS{18}Que ceux qui me persécutent soient honteux, mais que je ne sois pas honteux ; qu'ils soient brisés, mais que je ne sois pas brisé ! Fais venir sur eux le jour du malheur, frappe-les d'une double plaie !
\TextTitle{Message à propos du sabbat}
\VS{19}Ainsi m'a parlé Yahweh : Va, et tiens-toi debout à la porte des fils du peuple, par laquelle les rois de Juda entrent et par laquelle ils sortent, et à toutes les portes de Jérusalem.
\VS{20}Tu leur diras : Ecoutez la parole de Yahweh, rois de Juda, et vous tous homme de Juda, et vous tous habitants de Jérusalem qui entrez par ces portes !
\VS{21}Ainsi parle Yahweh : Prenez garde à vos âmes ; ne portez aucun fardeau le jour du sabbat, et ne les faites pas passer par les portes de Jérusalem\FTNT{Né. 13:19.}.
\VS{22}Ne faites sortir de vos maisons aucun fardeau le jour du sabbat, et ne faites aucune œuvre ; mais sanctifiez le jour du sabbat, comme je l'ai ordonné à vos pères\FTNT{Ex. 20:8 ; Ex. 23:12.}.
\VS{23}Mais ils n'ont pas écouté, ils n'ont pas prêté l'oreille ; ils ont raidi leur cou, pour ne pas écouter et ne pas recevoir d'instruction.
\VS{24}Il arrivera donc, si vous m'écoutez attentivement, dit Yahweh, pour ne faire passer aucun fardeau par les portes de cette ville le jour du sabbat, et si vous sanctifiez le jour du sabbat, en ne faisant aucune œuvre ce jour-là,
\VS{25}que les rois et les chefs, ceux qui sont assis sur le trône de David, montés sur des chars et sur des chevaux, eux et les chefs d'entre eux, les hommes de Juda et les habitants de Jérusalem, entreront par les portes de cette ville, et cette ville sera habitée à toujours.
\VS{26}On viendra aussi des villes de Juda et des environs de Jérusalem, et du pays de Benjamin, et du bas pays, des montagnes et du midi, pour apporter des holocaustes, des sacrifices, des offrandes et de l'encens ; pour apporter aussi des sacrifices de louanges dans la maison de Yahweh.
\VS{27}Mais si vous ne m'écoutez pas pour sanctifier le jour du sabbat, pour ne porter aucun fardeau, et n'en faire entrer aucun par les portes de Jérusalem le jour du sabbat, je mettrai le feu à ses portes, et il consumera les palais de Jérusalem et ne s'éteindra pas\FTNT{2 R. 25:9.}.
\Chap{18}
\TextTitle{La maison du potier ; appel à la repentance et avertissement}
\VerseOne{}Cette parole fut adressée à Jérémie de la part de Yahweh, disant :
\VS{2}Lève-toi et descends dans la maison d'un potier ; et là, je te ferai entendre mes paroles.
\VS{3}Je descendis donc dans la maison d'un potier, et voici, il faisait son ouvrage, assis sur sa selle.
\VS{4}Et le vase qu'il faisait avec l'argile qu'il tenait dans sa main, fut gâté ; et il en fit encore un autre vase, comme il lui sembla bon de le faire.
\VS{5}Alors la parole de Yahweh me fut adressée, en disant :
\VS{6}Maison d'Israël, ne puis-je pas faire de vous comme a fait ce potier ? Dit Yahweh. Voici, comme l'argile est dans la main d'un potier, ainsi vous êtes dans ma main, maison d'Israël !
\VS{7}En un instant je parle contre une nation et contre un royaume, pour arracher, pour démolir, et pour détruire ;
\VS{8}mais si cette nation, contre laquelle j'ai parlé, revient de sa méchanceté, je me repentirai aussi du mal que j'avais pensé de lui faire\FTNT{Jon. 3:6-10.}.
\VS{9}Et si en un instant je parle d'une nation et d'un royaume, pour l'édifier et pour le planter ;
\VS{10} et que cette nation fasse ce qui est mal à mes yeux, en sorte qu'elle n'écoute pas ma voix, je me repentirai aussi du bien que j'avais dit que je lui ferais.
\VS{11}Or donc, parle maintenant aux hommes de Juda et aux habitants de Jérusalem, en disant : Ainsi parle Yahweh : Voici, je projette du mal contre vous, et je forme un dessein contre vous. Détournez-vous donc chacun de votre mauvaise voie, et amendez votre voie et vos actions !
\VS{12}Et ils répondent : Il n'y a plus d'espérance ; c'est pourquoi nous suivrons nos pensées, chacun de nous fera selon les penchants de son mauvais cœur.
\VS{13}C'est pourquoi ainsi parle Yahweh : Demandez maintenant aux nations ! Qui a entendu de telles choses ? La vierge d'Israël a fait une chose très horrible\FTNT{1 Co. 5:1.}.
\VS{14}La neige du Liban abandonnerait-elle le rocher du champ ? Ou les eaux fraîches et ruisselantes qui viennent de loin tariraient-elles ?
\VS{15}Mais mon peuple m'a oublié, et il brûle de l'encens à ce qui n'est que vanité, et qui les a fait chanceler leurs voies, pour les faire retirer des anciens sentiers, afin de marcher dans les sentiers d'un chemin non frayé ;
\VS{16}pour faire venir sur leur pays une désolation et un opprobre perpétuel ; quiconque passera par là, en sera étonné et secouera la tête.
\VS{17}Je les disperserai devant l'ennemi, comme par le vent d'orient ; je leur tournerai le dos, et non pas la face, au jour de leur calamité\FTNT{Es. 27:8.}.
\VS{18}Et ils ont dit : Venez, et faisons des complots contre Jérémie ! Car la loi ne périra pas chez le sacrificateur, ni le conseil chez le sage, ni la parole chez le prophète. Venez, et tuons-le avec la langue, et ne soyons pas attentifs à ses discours !
\VS{19}Yahweh ! Fais attention à moi, et écoute la voix de ceux qui contestent avec moi !
\VS{20}Le mal sera-t-il rendu pour le bien\FTNT{Ps. 35:12 ; Ps. 109:5.} ? Car ils ont creusé une fosse pour mon âme. Souviens-toi que je me suis tenu devant toi, afin de parler pour leur bien, et afin de détourner d'eux ta grande colère.
\VS{21}C'est pourquoi livre leurs fils à la famine, et fais couler leur sang à coups d’épée ; que leurs femmes soient privées d'enfants, et deviennent veuves, et que leurs maris soient enlevés par la mort ; et leurs jeunes gens frappés par l'épée dans la bataille\FTNT{Ps. 109:9-13.} !
\VS{22}Qu'on entende le cri de leurs maisons, quand tu feras venir subitement des troupes contre eux ! Car ils ont creusé une fosse pour me prendre, et ils ont caché des pièges pour mes pieds.
\VS{23}Or tu sais, ô Yahweh ! Que tout leur conseil est contre moi pour me mettre à mort ; ne sois pas apaisé à l'égard de leur iniquité, et n'efface pas leur péché de devant ta face, mais qu'on les fasse tomber en ta présence ; agis contre eux au temps de ta colère.
\Chap{19}
\TextTitle{Le vase brisé : Image de Juda}
\VerseOne{}Ainsi a parlé Yahweh : Va, et achète un vase de terre d'un potier, et prends avec toi des anciens du peuple et des anciens des sacrificateurs.
\VS{2}Et sors à la vallée de Ben-Hinnom, qui est auprès de l'entrée de la porte de la poterie, et crie là les paroles que je te dirai.
\VS{3}Dis donc : Rois de Juda, et vous, habitants de Jérusalem, écoutez la parole de Yahweh ! Ainsi parle Yahweh des armées, le Dieu d'Israël : Voici, je vais faire venir sur ce lieu-ci un mal, tel que quiconque l'entendra, les oreilles lui tinteront\FTNT{1 S. 3:11 ; 2 R. 21:12.} ; 
\VS{4}parce qu'ils m'ont abandonné, et qu'ils ont profané ce lieu, et y ont brûlé de l'encens à d'autres dieux, que ni eux, ni leurs pères, ni les rois de Juda n'ont connus, et parce qu'ils ont rempli ce lieu du sang des innocents ;
\VS{5}et qu'ils ont bâti des hauts lieux à Baal, afin de brûler au feu leurs fils pour en faire des holocaustes à Baal : Ce que je n'avais pas ordonné, et dont je n'avais pas parlé, et qui ne m'était pas monté à cœur.
\VS{6}A cause de cela, voici les jours viennent, dit Yahweh, que ce lieu-ci ne sera plus appelé Topheth, ni la vallée de Ben-Hinnom, mais la vallée de la tuerie.
\VS{7}Et j'anéantirai dans ce lieu-ci le conseil de Juda et de Jérusalem ; et je les ferai tomber par l'épée devant leurs ennemis et par la main de ceux qui cherchent leur vie ; et je donnerai leurs cadavres en pâture aux oiseaux des cieux et aux bêtes de la terre.
\VS{8}Je ferai de cette ville un objet de désolation et de moquerie ; quiconque passera près d'elle sera étonné et sifflera à cause de toutes ses plaies.
\VS{9}Et je leur ferai manger la chair de leurs fils et la chair de leurs filles ; et chacun mangera la chair de son compagnon durant le siège, et dans la détresse où les réduiront leurs ennemis et ceux qui cherchent leur vie\FTNT{Lé. 26:29 ; De. 28:53 ; La. 2:20.}.
\VS{10}Puis tu briseras le vase, sous les yeux des hommes qui seront allés avec toi.
\VS{11}Et tu leur diras : Ainsi parle Yahweh des armées : Je briserai ce peuple et cette ville, de même qu'on brise un vase de potier, qui ne peut être réparé. Et ils seront enterrés à Topheth parce qu'il n'y aura plus d'autre lieu pour les enterrer.
\VS{12}Je ferai ainsi à ce lieu-ci, dit Yahweh, et à ses habitants, et je rendrai cette ville semblable à Topheth ;
\VS{13}et les maisons de Jérusalem, et les maisons des rois de Juda, seront impures comme le lieu de Topheth, à cause de toutes les maisons sur les toits desquelles ils brûlaient de l'encens à toute l'armée des cieux, et faisaient des libations à d'autres dieux.
\VS{14}Puis Jérémie revint de Topheth, là où Yahweh l'avait envoyé pour prophétiser. Et il se tint debout dans le parvis de la maison de Yahweh, et il dit à tout le peuple :
\VS{15}Ainsi parle Yahweh des armées, le Dieu d'Israël : Voici, je vais faire venir sur cette ville et sur toutes ses villes tout le mal que j'ai prononcés contre elle, parce qu'ils ont raidi leur cou pour ne pas écouter mes paroles.
\Chap{20}
\TextTitle{Paschhur outrage Jérémie}
\VerseOne{}Alors Paschhur, fils d'Immer, qui était sacrificateur et inspecteur en chef dans la maison de Yahweh, entendit Jérémie qui prophétisait ces choses.
\VS{2}Et Paschhur frappa le prophète Jérémie, et le mit dans la prison qui était à la porte supérieure de Benjamin, dans la maison de Yahweh.
\VS{3}Et il arriva que dès le lendemain, que Paschhur tira Jérémie hors de la prison. Et Jérémie lui dit : Yahweh ne t'appelle pas du nom de Paschhur, mais Magor-Missabib\FTNT{« Magor-Missabib » veut dire « terreur de chaque côté ».}.
\VS{4}Car ainsi parle Yahweh : Voici, je vais te livrer à la terreur, toi et tous tes amis qui tomberont par l'épée de leurs ennemis, et tes yeux le verront. Je livrerai tous ceux de Juda entre les mains du roi de Babylone, qui les transportera à Babylone et les frappera de l'épée.
\VS{5}Et je livrerai toutes les richesses de cette ville, et tout son travail, et tout ce qu'elle a de précieux, je livrerai, dis-je, tous les trésors des rois de Juda entre les mains de leurs ennemis, qui les pilleront, les enlèveront et les conduiront à Babylone.
\VS{6}Et toi, Paschhur, et tous ceux qui demeurent dans ta maison, vous irez en captivité ; tu iras à Babylone, tu y mourras, et y seras enterré, toi et tous tes amis auxquels tu as prophétisé le mensonge.
\TextTitle{Jérémie gémit auprès de Yahweh}
\VS{7}Ô Yahweh ! Tu m'as persuadé, et je me suis laissé persuader ; tu m'as saisi, et tu m'as vaincu. Je suis un objet de moquerie chaque jour, chacun se moque de moi.
\VS{8}Car depuis que je parle, je crie, je crie violence et dévastation ! Et la parole de Yahweh est pour moi un sujet d'opprobre et de moquerie chaque jour\FTNT{Es. 57:4.}.
\VS{9}C'est pourquoi j'ai dit : Je ne ferai plus mention de lui, je ne parlerai plus en son Nom, mais il y a eu dans mon cœur comme un feu ardent, renfermé dans mes os ; je me fatigue à le contenir, et je ne le puis.
\VS{10}Car j'entends les mauvais propos de plusieurs, la frayeur m'a saisi de tous côtés ; rapportez, disent-ils, et nous le rapporterons ! Tous ceux qui étaient en paix avec moi observent si je bronche, et disent : Peut-être se laissera-t-il séduire, et nous le vaincrons, nous tirerons vengeance de lui !
\VS{11}Mais Yahweh est avec moi comme un héros puissant ; c'est pourquoi ceux qui me persécutent seront renversés, ils ne me vaincront pas ; ils seront honteux, car ils n'ont pas réussi : Ce sera une honte éternelle qui ne s'oubliera jamais.
\VS{12}Yahweh des armées qui éprouve les justes, qui voit les reins et les cœurs, fais que je voie ta vengeance s'exercer contre eux, car je t'ai découvert ma cause.
\VS{13}Chantez à Yahweh, louez Yahweh ! Car il délivre l'âme des pauvres de la main des méchants.
\VS{14}Maudit soit le jour où je suis né ! Que le jour où ma mère m'a enfanté ne soit pas béni !
\VS{15}Maudit soit l'homme qui porta cette nouvelle à mon père, en lui disant : Un fils mâle t'est né, et qui le combla de joie !
\VS{16}Que cet homme-là soit comme les villes que Yahweh a renversées sans s'en repentir ! Qu'il entende la clameur le matin, et le cri de guerre au temps du midi\FTNT{Ge. 19:24-25 ; So. 2:4.} !
\VS{17}Que ne m'a-t-on fait mourir dans le sein de ma mère ! Pourquoi ma mère ne m'a-t-elle pas servi de sépulcre ? Et pourquoi n'est-elle pas restée éternellement enceinte ?
\VS{18}Pourquoi suis-je sorti de son sein pour ne voir que peine et douleur, et pour consumer mes jours dans la honte ?
\Chap{21}
\TextTitle{Prophétie sur les rois de Juda : Sédécias}
\VerseOne{}La parole qui fut adressée à Jérémie de la part de Yahweh, lorsque le roi Sédécias envoya vers lui Paschhur, fils de Malkija, et Sophonie, fils de Maaséja, le sacrificateur, pour lui dire :
\VS{2}Consulte maintenant Yahweh pour nous ; car Nebucadnetsar, roi de Babylone, combat contre nous ; peut-être que Yahweh fera-t-il en notre faveur un de ses miracles, afin qu'il se retire de nous.
\VS{3}Et Jérémie leur dit : Vous direz ainsi à Sédécias :
\VS{4}Ainsi parle Yahweh, le Dieu d'Israël : Voici, je vais détourner les armes de guerre qui sont dans vos mains, et avec lesquelles vous combattez en dehors des murailles contre le roi de Babylone et contre les Chaldéens qui vous assiègent, et je les rassemblerai au milieu de cette ville.
\VS{5}Et je combattrai contre vous, avec une main étendue, et avec un bras puissant, avec colère, avec fureur, et avec une grande indignation.
\VS{6}Et je frapperai les habitants de cette ville, les hommes, et les bêtes ; et ils mourront d'une grande peste.
\VS{7}Et après cela, dit Yahweh, je livrerai Sédécias, roi de Juda, et ses serviteurs, et le peuple, et ceux qui dans cette ville survivront à la peste, à l'épée et à la famine, entre les mains de Nebucadnetsar\FTNT{2 R. 24 et 25. }, roi de Babylone, et entre les mains de leurs ennemis, et entre les mains de ceux qui cherchent leur vie ; et il les frappera au tranchant de l'épée, il ne les épargnera pas, il n'en aura pas de compassion, il n'en aura pas de pitié.
\VS{8}Tu diras aussi à ce peuple : Ainsi parle Yahweh : Voici, je mets devant vous le chemin de la vie et le chemin de la mort\FTNT{De. 30:19.}.
\VS{9}Quiconque restera dans cette ville mourra par l'épée, ou par la famine, ou par la peste, mais celui qui en sortira, et se rendra aux Chaldéens qui vous assiègent vivra et aura sa vie pour butin.
\VS{10}Car je dresse ma face en mal et non en bien contre cette ville, dit Yahweh ; elle sera livrée entre les mains du roi de Babylone, et il la brûlera par le feu.
\VS{11}Et quant à la maison du roi de Juda : Ecoutez la parole de Yahweh !
\VS{12}Maison de David ! Ainsi parle Yahweh : Rendez la justice dès le matin, et délivrez celui qui aura été pillé d'entre les mains de l'oppresseur, de peur que ma fureur ne sorte comme un feu, et qu'elle ne brûle sans qu'on puisse l'éteindre, à cause de la méchanceté de vos actions.
\VS{13}Voici, j'en veux à toi qui habites dans la vallée, et qui es le rocher de la plaine, dit Yahweh ; à vous qui dites : Qui descendra contre nous, et qui entrera dans nos demeures ?
\VS{14}Et je vous punirai selon le fruit de vos actions, dit Yahweh ; et je mettrai le feu dans sa forêt qui consumera tout ce qui est autour d'elle\FTNT{Ez. 21:2-3.}.
\Chap{22}
\TextTitle{Sédécias averti de la destruction de Jérusalem}
\VerseOne{}Ainsi parle Yahweh : Descends dans la maison du roi de Juda, et là prononce cette parole.
\VS{2}Tu diras donc : Ecoute la parole de Yahweh, ô roi de Juda qui es assis sur le trône de David, toi et tes serviteurs, et ton peuple, qui entrez par ces portes !
\VS{3}Ainsi parle Yahweh : Faites droit et justice ; et délivrez celui qui aura été pillé d'entre les mains de l'oppresseur ; ne maltraitez pas l'orphelin, ni l'étranger, ni la veuve ; et n'usez d'aucune violence, et ne répandez pas le sang innocent dans ce lieu-ci.
\VS{4}Car si vous mettez exactement en effet cette parole, alors les rois qui sont assis à la place de David sur son trône, montés sur des chars et sur des chevaux, entreront par les portes de cette maison, eux et leurs serviteurs, et leur peuple.
\VS{5}Mais si vous n'écoutez pas ces paroles, je le jure par moi-même\FTNT{Es. 45:23 ; Hé. 6:13.}, dit Yahweh, que cette maison deviendra une ruine.
\VS{6}Car ainsi parle Yahweh sur la maison du roi de Juda : Tu es pour moi un Galaad, et le sommet du Liban ; mais certainement, je ferai de toi un désert, une ville sans habitants.
\VS{7}Je prépare contre toi des destructeurs, chacun avec ses armes, qui couperont tes cèdres de choix, et les jetteront au feu.
\VS{8}Et plusieurs nations passeront près de cette ville, et chacun dira à son compagnon : Pourquoi Yahweh a-t-il fait ainsi à cette grande ville\FTNT{De. 29:24-28 ; 1 R. 9:8.} ?
\VS{9}Et on dira : C'est parce qu'ils ont abandonné l'alliance de Yahweh, leur Dieu, et qu'ils se sont prosternés devant d'autres dieux et les ont servis.
\TextTitle{Prophétie sur les rois de Juda : Joachaz (Schallum)}
\VS{10}Ne pleurez pas celui qui est mort, et ne vous lamentez pas sur lui ; mais pleurez amèrement celui qui s'en va, car il ne reviendra plus, il ne reverra plus le pays de sa naissance.
\VS{11}Car ainsi parle Yahweh sur Schallum, fils de Josias, roi de Juda, qui régnait à la place de Josias, son père, et qui est sorti de ce lieu : Il n'y reviendra plus ;
\VS{12}mais il mourra dans le lieu où on l'a transporté, et ne verra plus ce pays.
\TextTitle{Prophétie sur les rois de Juda : Jojakim}
\VS{13}Malheur à celui qui bâtit sa maison par l'injustice, et ses chambres hautes sans droiture ; qui fait travailler son prochain pour rien, sans lui donner le salaire de son travail\FTNT{Lé. 19:13 ; De. 24:14-15 ; Ha. 2:9.}.
\VS{14}Qui dit : Je me bâtirai une grande maison et des chambres spacieuses, et qui s'y fait percer des fenêtres ; elle est lambrissée de cèdre, et peinte de vermillon.
\VS{15}Régneras-tu, parce que tu t’enfermes dans du cèdre ? Ton père n'a-t-il pas mangé et bu ? Quand il a fait jugement et justice, alors il a prospéré.
\VS{16}Il jugeait la cause du pauvre et de l'indigent, alors il a prospéré. N'est-ce pas là me connaître ? Dit Yahweh.
\VS{17}Mais tes yeux et ton cœur ne sont adonnés qu'à ton gain déshonnête, qu'à répandre le sang innocent, qu'à faire du tord et qu'à opprimer.
\VS{18}C'est pourquoi ainsi parle Yahweh sur Jojakim, fils de Josias, roi de Juda : On ne le pleurera pas en disant : Hélas, mon frère ! et hélas, ma sœur ! On ne le pleurera pas en disant : Hélas, seigneur ! Et hélas, sa majesté !
\VS{19}Il sera enterré de la sépulture d'un âne, étant traîné et jeté hors des portes de Jérusalem.
\TextTitle{Prophétie sur les rois de Juda : Jojakin}
\VS{20}Monte sur le Liban, et crie ! Donne de la voix sur le Basan ! Crie du haut d'Abarim ! A cause que tous ceux qui t'aiment sont brisés.
\VS{21}Je t'ai parlé durant ta grande prospérité, mais tu disais : Je n'écouterai pas ; telle est ta voie depuis ta jeunesse, que tu n'as pas écouté ma voix.
\VS{22}Tous tes pasteurs seront la pâture du vent, et ceux qui t'aiment iront en captivité ; certainement tu seras alors honteuse et confuse, à cause de toute ta méchanceté.
\VS{23}Toi qui habites sur le Liban, et qui fais ton nid dans les cèdres, que tu seras à plaindre quand les douleurs t'atteindront, les douleurs comme celles d'une femme qui enfante.
\VS{24}Je suis vivant, dit Yahweh, que quand Jéconia, fils de Jojakim, roi de Juda, serait une bague à ma main droite, je t'arracherais de là.
\VS{25}Je te livrerai entre les mains de ceux qui cherchent ta vie, et entre les mains devant qui tu es craintif, et entre les mains de Nebucadnetsar, roi de Babylone, et entre les mains des Chaldéens\FTNT{2 R. 24:14 ; Ez. 17:12 ; 2 Ch. 36:10.}.
\VS{26}Et je te jetterai, toi et ta mère qui t'a enfanté, dans un autre pays où vous n'êtes pas nés, et vous y mourrez.
\VS{27}Et quant au pays qu'ils désirent pour y retourner, ils n'y retourneront pas.
\VS{28}Cet homme, Jéconia, est-il un vase méprisé et brisé ? Est-il un objet qui ne fait plus plaisir ? Pourquoi sont-ils jetés là, lui et sa postérité, lancés, dis-je, dans un pays qu'ils ne connaissent pas\FTNT{Os. 8:8.} ?
\VS{29}Ô terre, terre, terre ! Ecoute la parole de Yahweh !
\VS{30}Ainsi parle Yahweh : Ecrivez que cet homme-là est privé d'enfant, que c'est un homme qui ne prospérera pas pendant ses jours, et que même il n'y aura pas d'homme de sa postérité qui prospère, et qui soit assis sur le trône de David, ni qui domine plus en dominer sur Juda\FTNT{2 R. 24:8-16.}.
\Chap{23}
\TextTitle{Israël sera rassemblé par le Messie}
\VerseOne{}Malheur aux pasteurs qui détruisent et dispersent le troupeau de mon pâturage ! Dit Yahweh.
\VS{2}C'est pourquoi ainsi parle Yahweh, le Dieu d'Israël, sur les pasteurs qui paissent mon peuple : Vous avez dispersé\FTNT{Les brebis du Seigneur sont dispersées ou éparpillées par les faux pasteurs (Ez. 34).} mes brebis, et vous les avez chassées, et ne vous en êtes pas occupés ; voici, je vous punirai à cause de la méchanceté de vos actions, dit Yahweh.
\VS{3}Mais je rassemblerai le reste de mes brebis de tous les pays où je les ai chassées ; et je les ramènerai à leur pâturage, et elles seront fécondes et multiplieront.
\VS{4}Je susciterai aussi sur elles des pasteurs qui les paîtront, et elles n'auront plus de peur, et ne s'épouvanteront plus, et il n'en manquera aucune, dit Yahweh.
\VS{5}Voici, les jours viennent, dit Yahweh, où je susciterai à David un Germe juste, qui régnera en Roi ; il prospérera, et exercera le droit et la justice dans le pays\FTNT{Es. 4:2 ; Za. 6:12-13 ; Ps. 96:13 ; Lu. 1:32-33.}.
\VS{6}En son temps, Juda sera sauvé, Israël demeurera en sécurité ; et c'est ici le nom dont on l'appellera : Yahweh notre justice.
\VS{7}C'est pourquoi, voici, les jours viennent, dit Yahweh, qu'on ne dira plus : Yahweh est vivant, lui qui a fait monter les fils d'Israël du pays d'Egypte !
\VS{8}Mais : Yahweh est vivant, lui qui a fait monter et qui a ramené la postérité de la maison d'Israël, du pays du nord et de tous les pays où je les avais chassés, et ils habiteront dans leur pays.
\TextTitle{Jugement sur les faux prophètes}
\VS{9}A cause des prophètes mon cœur est brisé au-dedans de moi, tous mes os se relâchent ; je suis comme un homme ivre, et comme un homme que le vin a surmonté, à cause de Yahweh, et à cause des paroles de sa sainteté.
\VS{10}Car le pays est rempli d'hommes qui commettent l'adultère ; et le pays est en deuil à cause de la malédiction : Les pâturages du désert sont desséchés, leur course ne va qu'au mal, et leur force à ce qui n'est pas droit.
\VS{11}Car le prophète et le sacrificateur sont corrompus ; j'ai même trouvé dans ma maison leur méchanceté, dit Yahweh.
\VS{12}C'est pourquoi leur chemin sera comme des lieux glissants dans l'obscurité, ils y seront poussés et ils tomberont\FTNT{Ps. 35:6 ; Pr. 4:19.} ; car je ferai venir du mal sur eux, dans l'année de leur châtiment, dit Yahweh.
\VS{13}Or j'ai vu de la folie dans les prophètes de Samarie, car ils prophétisaient par Baal, et faisaient égarer mon peuple Israël.
\VS{14}Mais j'ai vu des choses horribles dans les prophètes de Jérusalem car ils commettent des adultères, et ils marchent dans le mensonge ; ils fortifient les mains de ceux qui font le mal, afin qu'aucun ne se détourne de sa méchanceté ; ils me sont tous comme Sodome, et les habitants de la ville comme Gomorrhe\FTNT{Es. 1:9.}.
\VS{15}C'est pourquoi, ainsi parle Yahweh des armées sur les prophètes : Voici, je vais leur faire manger de l'absinthe, et leur ferai boire des eaux empoisonnées ; car c'est par les prophètes que la profanation est venue dans tout le pays.
\VS{16}Ainsi parle Yahweh des armées : N'écoutez pas les paroles des prophètes qui vous prophétisent ! Ils vous font devenir vains, ils disent les visions de leur cœur, et ils ne les tiennent pas de la bouche de Yahweh.
\VS{17}Ils ne cessent de dire à ceux qui me méprisent : Yahweh a dit : Vous aurez la paix ; et ils disent à tous ceux qui marchent suivant les penchants de leur cœur : Il ne vous arrivera aucun mal\FTNT{Ez. 13:10.}.
\VS{18}Car qui s'est trouvé au conseil secret de Yahweh ? Et qui a aperçu et entendu sa parole\FTNT{Es. 40:13 ; Job. 15:8 ; 1 Co. 2:16.} ? Qui a été attentif à sa parole, et l'a entendue ?
\VS{19}Voici la tempête de Yahweh, sa fureur va se montrer, et le tourbillon prêt à fondre tombera sur la tête des méchants.
\VS{20}La colère de Yahweh ne se détournera pas jusqu'à ce qu'il ait accompli, exécuté les desseins de son cœur. Vous aurez une claire intelligence de ceci dans les derniers jours\FTNT{Gn. 49:1-2.}.
\VS{21}Je n'ai pas envoyé ces prophètes-là, et ils ont couru ; je ne leur ai pas parlé, et ils ont prophétisé.
\VS{22}S'ils s'étaient trouvés dans mon conseil secret, ils auraient aussi fait entendre mes paroles à mon peuple, et ils les auraient ramenés de leur mauvaise voie, de la méchanceté de leurs actions.
\VS{23}Suis-je un Dieu de près, dit Yahweh, et ne suis-je pas aussi un Dieu de loin ?
\VS{24}Quelqu'un se cachera-t-il dans un lieu secret sans que je ne le voie ? Dit Yahweh. Ne remplis-je pas, moi, les cieux et la terre ? Dit Yahweh\FTNT{Ps. 139:7-8 ; Am. 9:2-3.}.
\VS{25}J'ai entendu ce que les prophètes disent, prophétisant le mensonge en mon nom, et disant : J'ai eu un songe ! J'ai eu un songe !
\VS{26}Jusqu'à quand ceci sera-t-il au cœur des prophètes qui prophétisent le mensonge, et qui prophétisent la tromperie de leur cœur ?
\VS{27}Qui pensent comment ils feront oublier mon nom à mon peuple, par les songes que chacun d'eux raconte à son compagnon, comme leurs pères ont oublié mon Nom pour Baal\FTNT{Jg. 2:13.}.
\VS{28}Que le prophète qui a eu un songe raconte ce songe ; et que celui qui a ma parole proclame ma parole en vérité. Quelle convenance y a-t-il entre la paille et le froment ? Dit Yahweh.
\VS{29}Ma parole n'est-elle pas comme un feu, dit Yahweh, et comme un marteau qui brise le roc ?
\VS{30}C'est pourquoi voici, dit Yahweh, j'en veux aux prophètes qui se dérobent mes paroles l'un à l'autre.
\VS{31}Voici, dit Yahweh, j'en veux aux prophètes qui accommodent leurs langues, et qui disent : Il dit.
\VS{32}Voici, dit Yahweh, j'en veux à ceux qui prophétisent des songes faux, et qui les racontent, et font égarer mon peuple par leurs mensonges, et par leur témérité, quoique je ne les ai pas envoyés, et que je ne leur aie pas donné d'ordre ; c'est pourquoi ils ne sont d'aucune utilité à ce peuple, dit Yahweh\FTNT{So. 3:4.}.
\VS{33}Si donc ce peuple t'interroge, ou qu'il interroge le prophète, ou le sacrificateur, en disant : Quel est l'oracle de Yahweh ? Tu leur diras : Quel est cet oracle ? Je vous abandonnerai, dit Yahweh.
\VS{34}Et quant au prophète, et au sacrificateur, et au peuple qui dira : Oracle de Yahweh ; je punirai cet homme-là et sa maison.
\VS{35}Vous direz ainsi chacun à son compagnon, et chacun à son frère : Qu'a répondu Yahweh ? Qu'a dit Yahweh ?
\VS{36}Et vous ne mentionnerez plus : Oracle de Yahweh ; car la parole de chacun sera pour lui un oracle ; parce que vous tordez les paroles du Dieu vivant\FTNT{2 Pi. 3:15-16.}, les paroles de Yahweh des armées, notre Dieu.
\VS{37}Tu diras au prophète : Que t'a répondu Yahweh et que t'a dit Yahweh ?
\VS{38}Et si vous dites : Oracle de Yahweh ; à cause de cela, parle Yahweh, parce que vous dites cette parole : Oracle de Yahweh ; et que j'ai envoyé vers vous pour dire : Ne dites plus : Oracle de Yahweh !
\VS{39}A cause de cela, me voici, et je vous oublierai entièrement, et je vous rejetterai loin de ma présence, vous et la ville que j'ai donnée à vous et à vos pères.
\VS{40}Et je mettrai sur vous un opprobre éternel et une honte éternelle, qui ne s'oublieront pas.
\Chap{24}
\TextTitle{Bonnes figues: Futur retour en Juda des captifs de Babylone ; mauvaises figues: Jugement sur Jérusalem}
\VerseOne{}Yahweh me fit voir une vision, et voici deux paniers de figues posés devant le temple de Yahweh, après que Nebucadnetsar, roi de Babylone, eut transporté de Jérusalem, Jéconia, fils de Jojakim, roi de Juda, les chefs de Juda, avec les charpentiers et les serruriers, et les eut conduits à Babylone.
\VS{2}L'un des paniers avait de très bonnes figues, comme les figues de la première récolte ; et l'autre panier avait de très mauvaises figues, qu'on ne pouvait manger à cause de leur mauvaise qualité.
\VS{3}Et Yahweh me dit : Que vois-tu, Jérémie ? Et je répondis : Des figues. Les bonnes figues sont très bonnes, et les mauvaises sont très mauvaises et ne peuvent être mangées à cause de leur mauvaise qualité.
\VS{4}Alors la parole de Yahweh me fut adressée, en disant :
\VS{5}Ainsi parle Yahweh, le Dieu d'Israël : Comme tu distingues ces bonnes figues, ainsi je me souviendrai, pour leur faire du bien, des captifs de Juda, que j'ai envoyés hors de ce lieu dans le pays des Chaldéens.
\VS{6}Et je les regarderai d'un œil favorable, et je les ramènerai dans ce pays, je les y rétablirai et je ne les détruirai plus, je les planterai et je ne les arracherai pas.
\VS{7}Et je leur donnerai un cœur pour me connaître, pour connaître, dis-je, que je suis Yahweh ; et ils seront mon peuple, et je serai leur Dieu : Car ils reviendront à moi de tout leur cœur\FTNT{De. 30:6 ; Ez. 11:19.}.
\VS{8}Et comme les mauvaises figues, qui ne peuvent être mangées à cause de leur mauvaise qualité ; ainsi certainement, dit Yahweh, je ferai devenir Sédécias, roi de Juda, ses chefs, et le reste de Jérusalem, ceux qui sont restés dans ce pays, et ceux qui habitent dans le pays d'Egypte.
\VS{9}Et je les livrerai pour être agités pour leur malheur par tous les royaumes de la terre, et pour être en opprobre, en de proverbe, en raillerie, et en malédiction, par tous les lieux où je les aurai chassé\FTNT{De. 28:37.}.
\VS{10}Et j'enverrai sur eux l'épée, la famine et la peste, jusqu'à ce qu'ils soient consumés du pays que j'avais donné à eux et à leurs pères.
\Chap{25}
\TextTitle{Prophétie sur les soixante-dix ans de captivité babylonienne\FTNTT{Da.9:2.}}
\VerseOne{}La parole qui fut adressée à Jérémie touchant tout le peuple de Juda, la quatrième année de Jojakim, fils de Josias, roi de Juda, qui est la première année de Nebucadnetsar, roi de Babylone,
\VS{2}parole que Jérémie, le prophète, prononça à tout le peuple de Juda, et à tous les habitants de Jérusalem, en disant :
\VS{3}Depuis la treizième année de Josias, fils d'Amon, roi de Juda, jusqu'à ce jour, qui est la vingt-troisième année, la parole de Yahweh m'a été adressée ; et je vous ai parlé, me levant dès le matin et parlant, et vous n'avez pas écouté.
\VS{4}Et Yahweh vous a envoyé tous ses serviteurs, les prophètes, se levant dès le matin et les envoyant ; et vous ne les avez pas écoutés, vous n'avez pas prêté l'oreille pour écouter.
\VS{5}Lorsqu'ils disaient : Détournez-vous maintenant chacun de sa mauvaise voie et de la méchanceté de vos actions, et vous habiterez d'un siècle à l'autre dans le pays que Yahweh a donné à vous et à vos pères\FTNT{Jon. 3:8 ; 2 R. 17:13.} ;
\VS{6}et n'allez pas après d'autres dieux pour les servir et pour vous prosterner devant eux, ne m'irritez pas par les œuvres de vos mains, et je ne vous ferai aucun mal.
\VS{7}Mais vous ne m'avez désobéi, dit Yahweh, pour m'irriter par les œuvres de vos mains, pour votre malheur.
\VS{8}C'est pourquoi ainsi parle Yahweh des armées : Parce que vous n'avez pas écouté mes paroles,
\VS{9}voici j'enverrai et j'assemblerai toutes les familles du nord, dit Yahweh, et j'enverrai, dis-je, vers Nebucadnetsar, roi de Babylone, mon serviteur ; et je les ferai venir contre ce pays et contre ses habitants, et contre toutes ces nations d'alentour, je les détruirai à la façon de l'interdit, je les mettrai en désolation, en opprobre et en ruines éternelles\FTNT{De. 28:37 ; Es. 10:6.}.
\VS{10}Et je ferai cesser parmi eux les cris de joie et les cris d'allégresse, la voix de l'époux et la voix de l'épouse, le bruit des meules et la lumière des lampes\FTNT{Es. 24:7 ; Ez. 26:13.}.
\VS{11}Et tout ce pays sera une ruine jusqu'à s'en étonner, et ces nations seront asservies au roi de Babylone pendant soixante-dix ans\FTNT{Voir Jé. 29 : 10. Les soixante-dix ans se rapportent également au temps de la domination mondiale babylonienne. Le peuple avait une dette envers Yahweh de 70 ans de sabbats (Lé. 26 34-43 ; 2 Ch. 36:21).}.
\TextTitle{Jugement sur Babylone et les nations impies}
\VS{12}Et il arrivera que quand ces soixante-dix ans seront accomplis, je punirai le roi de Babylone et cette nation, dit Yahweh, à cause de leurs iniquités ; je punirai le pays des Chaldéens, que je mettrai en désolations éternelles\FTNT{Da. 9:2.}.
\VS{13}Et je ferai venir sur ce pays-là toutes mes paroles que j'ai prononcées contre lui, toutes les choses qui sont écrites dans ce livre, ce que Jérémie a prophétisé contre toutes ces nations.
\VS{14}Car de grands rois aussi et de grandes nations se serviront d'eux, et je leur rendrai selon leurs actions et selon l'œuvre de leurs mains.
\VS{15}Car ainsi m'a parlé Yahweh, le Dieu d'Israël : Prends de ma main cette coupe du vin, savoir de cette fureur-ci, et fais-la boire à toutes les nations vers lesquelles je t'enverrai\FTNT{Ab. 16.}.
\VS{16}Ils en boiront, et ils chancelleront et seront comme fous, à cause de l'épée que j'enverrai parmi eux.
\VS{17}Je pris donc la coupe de la main de Yahweh, et je la fis boire à toutes les nations vers lesquelles Yahweh m'envoyait :
\VS{18}Savoir : A Jérusalem et aux villes de Juda, à ses rois et à ses chefs, pour les mettre en désolation, en étonnement, en opprobre et en malédiction, comme il paraît aujourd'hui ;
\VS{19}à Pharaon, roi d'Egypte, à ses serviteurs, à ses chefs, et à tout son peuple ;
\VS{20}et à tout le mélange des peuples d'Arabie, à tous les rois du pays d'Uts, à tous les rois du pays des Philistins, à Askalon, à Gaza, à Ekron, et au reste d'Asdod ;
\VS{21}à Edom, à Moab, et aux fils d'Ammon ;
\VS{22}à tous les rois de Tyr, à tous les rois de Sidon, et aux rois des îles qui sont au-delà de la mer ;
\VS{23}à Dedan, à Théma, à Buz, et à tous ceux qui se coupent les coins de la barbe ;
\VS{24}à tous les rois d'Arabie, et à tous les rois des Arabes qui habitent au désert ;
\VS{25}à tous les rois de Zimri, à tous les rois d'Elam, et à tous les rois de Médie ;
\VS{26}à tous les rois du nord, tant proches qu'éloignés l'un de l'autre, et à tous les royaumes du monde qui sont sur la face de la terre. Et le roi de Schéschac boira après eux.
\VS{27}Et tu leur diras : Ainsi parle Yahweh des armées, le Dieu d'Israël : Buvez et soyez enivrés, même vomissez, et tombez sans vous relever, à cause de l'épée que j'enverrai parmi vous !
\VS{28}Or il arrivera qu'ils refuseront de prendre la coupe de ta main pour boire; mais tu leur diras : Ainsi parle Yahweh des armées : Vous en boirez certainement !
\VS{29}Car voici, je commence à envoyer du mal sur la ville sur laquelle mon nom est invoqué ; et vous, en seriez exempts en quelque sorte ? Vous n'en serez pas exempts ; car je m'en vais appeler l'épée sur tous les habitants de la terre, dit Yahweh des armées\FTNT{1 Pi. 4:17-18.}.
\VS{30}Tu prophétiseras donc contre eux toutes ces paroles-là, et tu leur diras : Yahweh rugira d'en haut ; il fera entendre sa voix de la demeure de sa sainteté ; il rugira, il rugira contre son son agréable demeure ; il poussera un cri  contre tous les habitants de la terre\FTNT{Joë. 3:16 ; Am. 1:2.}, comme ceux qui foulent au pressoir,
\VS{31}le son éclatant est parvenu jusqu'à l'extrémité de la terre ; car Yahweh plaide avec les nations, et il conteste contre toute chair. On livrera les méchants à l'épée, dit Yahweh.
\VS{32}Ainsi parle Yahweh des armées : Voici, le mal va sortir d'une nation à l'autre, et une grande tempête se se lèvera des extrémités de la terre.
\VS{33}Et en ce jour-là, ceux qui auront été mis à mort par Yahweh seront étendus d'un bout de la terre à l'autre bout ; ils ne seront ni pleurés, ni recueillis, ni enterrés, mais ils seront comme du fumier sur la face du sol.
\VS{34}Vous, pasteurs, hurlez et criez ! Et vous, les nobles du troupeau, roulez-vous dans la cendre ; car les jours pour vous massacrer sont accomplis. Je vous disperserai et vous tomberez comme un vase précieux.
\VS{35}Et les pasteurs n'auront aucun moyen de s'enfuir, ni les nobles d'échapper. 
\VS{36}Il y aura la voix du cri des bergers, les hurlements des nobles du troupeau ; parce que Yahweh s'en va ravager leur pâturage.
\VS{37}Et les demeures paisibles seront abattues, à cause de l'ardeur de la colère de Yahweh.
\VS{38}Il a abandonné son tabernacle comme un lionceau ; car leur pays est réduit en désert, à cause de l'ardeur du destructeur et à cause, dis-je, de l'ardeur de sa colère.
\Chap{26}
\TextTitle{Avertissement dans le parvis du temple}
\VerseOne{}Au commencement du règne de Jojakim, fils de Josias, roi de Juda, cette parole fut adressée à Jérémie part Yahweh, en disant :
\VS{2}Ainsi parle Yahweh : Tiens-toi debout dans le parvis de la maison de Yahweh, et prononce à toutes les villes de Juda qui viennent pour se prosterner dans la maison de Yahweh toutes les paroles que je t'ordonne de leur dire ; n'en retranche pas un mot.
\VS{3}Peut-être qu'ils écouteront et qu'ils se détourneront chacun de sa mauvaise voie ; et je me repentirai du mal que j'avais pensé leur faire à cause de la méchanceté de leurs actions.
\VS{4}Tu leur diras donc : Ainsi parle Yahweh : Si vous ne m'écoutez pas pour marcher selon ma loi que je vous ai proposée,
\VS{5}pour obéir aux paroles des prophètes, mes serviteurs, que je vous envoie, me levant dès le matin, et les envoyant, lesquels que vous n'avez pas écoutés,
\VS{6}je mettrai cette maison dans le même état que Silo, et je livrerai cette ville à la malédiction, à toutes les nations de la terre.
\VS{7}Or les sacrificateurs, les prophètes, et tout le peuple, entendirent Jérémie prononcer ces paroles dans la maison de Yahweh.
\TextTitle{Jérémie menacé de mort par les sacrificateurs et les prophètes}
\VS{8}Et il arrivera qu'aussitôt que Jérémie eut achevé prononcer tout ce que Yahweh lui avait ordonné de dire à tout le peuple, les sacrificateurs, les prophètes, et tout le peuple, le saisirent en disant : Tu mourras, tu mourras\FTNT{Gn. 2:17.} !
\VS{9}Pourquoi as-tu prophétisé au nom de Yahweh, en disant : Cette maison sera comme Silo, et cette ville sera déserte tellement que personne n'y habitera ? Et tout le peuple s'assembla autour de Jérémie dans la maison de Yahweh.
\VS{10}Et les les chefs de Juda ayant entendu toutes ces choses, montèrent de la maison du roi à la maison de Yahweh, et s'assirent à l'entrée de la porte neuve de la maison de Yahweh.
\VS{11}Et les sacrificateurs et les prophètes parlèrent aux chefs et à tout le peuple, en disant : Cet homme mérite d'être condamné à la mort ; car il a prophétisé contre cette ville, comme vous l'avez entendu de vos oreilles.
\VS{12}Et Jérémie parla à tous les chefs et à tout le peuple, en disant : Yahweh m'a envoyé pour prophétiser contre cette maison et contre cette ville toutes les paroles que vous avez entendues.
\VS{13}Maintenant donc, amendez votre conduite et vos actions, écoutez la voix de Yahweh, votre Dieu, et Yahweh se repentira du mal qu'il a prononcé contre vous.
\VS{14}Pour moi, me voici entre vos mains ; faites-moi ce qui vous semblera bon et juste.
\VS{15}Mais sachez comme une chose certaine, que si vous me faites mourir, vous mettrez du sang innocent sur vous, sur cette ville et sur ses habitants ; car en vérité Yahweh m'a envoyé vers vous pour prononcer à vos oreilles toutes ces paroles.
\VS{16}Alors les chefs et tout le peuple dirent aux sacrificateurs et aux prophètes : Cet homme ne mérite pas d'être condamné à la mort, car il nous a parlé au nom de Yahweh, notre Dieu.
\VS{17}Et quelques-uns des anciens du pays se levèrent et parlèrent à toute l'assemblée du peuple, en disant :
\VS{18}Michée, de Moréscheth, prophétisait aux jours d'Ezéchias, roi de Juda, et il parlait à tout le peuple de Juda, en disant : Ainsi parle Yahweh des armées : Sion sera labourée comme un champ, Jérusalem sera réduite en un monceau de pierres, et la montagne du temple en des hauts lieux d'une forêt\FTNT{Mi. 1:1 ; Mi. 3:12.}.
\VS{19}Ezéchias, roi de Juda, et tous ceux de Juda l'ont-ils fait mourir ? Ezéchias ne craignit-il pas Yahweh ? N'implora-t-il pas Yahweh ? Et Yahweh se repentit du mal qu'il avait prononcé contre eux. Et nous, nous ferions donc un grand mal contre nos âmes\FTNT{2 Ch. 32:26.} !
\VS{20}Mais aussi dirent les autres, y eut aussi un homme qui prophétisait au nom de Yahweh, savoir, Urie, fils de Schemaeja, de Kirjath-Jearim. Il prophétisa contre cette même ville et contre ce même pays, de la même manière que Jérémie.
\VS{21}Et le roi Jojakim, tous ses vaillants hommes, et tous ses chefs entendirent ses paroles, et le roi chercha à le faire mourir. Urie, qui en fut informé, eut peur, prit la fuite, et se retira en Egypte.
\VS{22}Et le roi Jojakim envoya des hommes en Egypte, savoir, Elnathan, fils d'Acbor, et quelques hommes avec lui, qui allèrent en Egypte.
\VS{23}Et ils firent sortir d'Egypte Urie et l'amenèrent au roi Jojakim qui le frappa avec l'épée et jeta son cadavre sur les sépulcres des fils du peuple.
\VS{24}Toutefois la main d'Achikam, fils de Schaphan, fut avec Jérémie, et empêcha qu'il ne soit livré au peuple pour être mis à mort.
\Chap{27}
\TextTitle{Prophétie : Les nations seront asservies à Nebucadnetsar}
\VerseOne{}Au commencement du règne de Jojakim\FTNT{Il est probable que ce soit une erreur de copiste, car bien que l'hébreu dit « Jojakim », le contexte se rapporte à Sédécias. Voir Jé. 27:3 ; Jé. 27:12 ; Jé. 27:20 ; Jé 28:1.}, fils de Josias, roi de Juda, cette parole fut adressée à Jérémie de la part de Yahweh, en disant :
\VS{2}Ainsi m'a parlé Yahweh : Fais-toi des liens et des jougs, et mets-les sur ton cou\FTNT{Ez. 7:23.}.
\VS{3}Et envoie-les au roi d'Edom, et au roi de Moab, et au roi des fils d'Ammon, et au roi de Tyr et au roi de Sidon, par les mains des messagers qui sont venus à Jérusalem vers Sédécias, roi de Juda ;
\VS{4}et tu leur donneras mes ordres pour leurs maîtres, en disant : Ainsi parle Yahweh des armées, le Dieu d'Israël : Vous direz ainsi à vos maîtres :
\VS{5}J'ai fait la terre, les hommes et les bêtes qui sont sur la terre, par ma grande force et par mon bras étendu, et je la donne à qui cela me plaît\FTNT{De. 32:8.}.
\VS{6}Et maintenant j'ai livré tous ces pays entre les mains de Nebucadnetsar, roi de Babylone, mon serviteur ; et même je lui ai donné les bêtes des champs pour qu'elles lui soient asservies\FTNT{Da. 2:38.}.
\VS{7}Et toutes les nations lui seront asservies, à lui, à son fils, et au fils de son fils, jusqu'à ce que le temps de son pays vienne aussi, et que plusieurs nations et de grands rois l'asservissent.
\VS{8}Et il arrivera que la nation ou le royaume qui ne se soumettra pas à lui, à Nebucadnetsar, roi de Babylone, et qui ne soumettra pas son cou au joug du roi de Babylone, je punirai cette nation par l'épée, par la famine et par la peste, dit Yahweh, jusqu'à ce que je les aie consumés par sa main.
\VS{9}Vous donc, n'écoutez pas vos prophètes, ni vos devins, ni vos songeurs, ni vos augures, ni vos magiciens, qui vous parlent, en disant : Vous ne serez pas asservis au roi de Babylone.
\VS{10}Car ils vous prophétisent le mensonge pour vous faire aller loin de votre pays, afin que je vous chasse et que vous périssiez.
\VS{11}Mais la nation qui livrera son cou au joug du roi de Babylone, et qui le servira, je la laisserai dans son pays, dit Yahweh, pour qu'elle le cultive et qu'elle y demeure.
\VS{12}Puis j'ai parlé à Sédécias, roi de Juda, selon toutes ces paroles-là, en disant : Soumettez votre cou au joug du roi de Babylone, et rendez-vous sujets, à lui et son peuple, et vous vivrez.
\VS{13}Pourquoi mourriez-vous, toi et ton peuple, par l'épée, par la famine et par la peste, selon que Yahweh a parlé contre la nation qui ne sera pas soumise au roi de Babylone ?
\VS{14}N'écoutez donc pas les paroles des prophètes qui vous parlent en disant : Vous ne serez pas asservis au roi de Babylone ! Car ils vous prophétisent le mensonge.
\VS{15}Même je ne les ai pas envoyés, dit Yahweh, et ils vous prophétisent faussement en mon nom, afin que je vous rejette et que vous périssiez, vous et les prophètes qui vous prophétisent.
\VS{16}J'ai aussi parlé aussi aux sacrificateurs et à tout ce peuple, en disant : Ainsi parle Yahweh : N'écoutez pas les paroles de vos prophètes qui vous prophétisent, en disant : Voici, les ustensiles de la maison de Yahweh seront bientôt rapportés de Babylone ! Car ils vous prophétisent le mensonge.
\VS{17}Ne les écoutez donc pas, rendez-vous sujets au roi de Babylone, et vous vivrez. Pourquoi cette ville serait-elle réduite en un désert ?
\VS{18}Et s'ils sont prophètes et si la parole de Yahweh est en eux, qu'ils intercèdent maintenant auprès de Yahweh des armées, afin que les ustensiles qui restent dans la maison de Yahweh, dans la maison du roi de Juda, et dans Jérusalem, n'aillent pas à Babylone.
\VS{19}Car ainsi parle Yahweh des armées au sujet des colonnes, de la mer, des bases, et des autres ustensiles qui sont restés dans cette ville,
\VS{20}que Nebucadnetsar, roi de Babylone, n'a pas emportés, quand il a transporté de Jérusalem à Babylone, Jéconia, fils de Jojakim, roi de Juda, et tous les nobles de Juda et de Jérusalem,
\VS{21}Yahweh, dis-je, des armées le Dieu d'Israël, parle ainsi au sujet des ustensiles qui restent dans la maison de Yahweh, dans la maison du roi de Juda et dans Jérusalem :
\VS{22}Ils seront emportés à Babylone, et ils y demeureront jusqu'au jour où je les visiterai, dit Yahweh, et où je les ferai remonter et revenir dans ce lieu\FTNT{2 R. 24:14-15 ; Esd. 1:7-11 ; 2 Ch. 25:13-16 ; 2 Ch. 36:18.}.
\Chap{28}
\TextTitle{Hanania meurt suite à sa prophétie mensongère}
\VerseOne{}Il arriva aussi, en cette même année, au commencement du règne de Sédécias, roi de Juda, savoir, au cinquième mois de la quatrième année, que Hanania, fils d'Azzur, prophète de Gabaon, me parla dans la maison de Yahweh, aux yeux des sacrificateurs et de tout le peuple, en disant :
\VS{2}Ainsi parle Yahweh des armées, le Dieu d'Israël : Je romps le joug du roi de Babylone !
\VS{3}Dans deux années accomplis, et je ferai rapporter dans ce lieu tous les ustensiles de la maison de Yahweh, que Nebucadnetsar, roi de Babylone, a pris de ce lieu, et qu'il a transportés à Babylone.
\VS{4}Et je ferai revenir dans ce lieu, dit Yahweh, Jéconia, fils de Jojakim, roi de Juda, et tous les captifs de Juda qui sont allés à Babylone ; car je romprai le joug du roi de Babylone.
\VS{5}Alors Jérémie, le prophète, répondit à Hanania, le prophète, aux yeux des sacrificateurs, et aux yeux de tout le peuple qui se tenait dans la maison de Yahweh.
\VS{6}Et Jérémie, le prophète, dit : Ainsi soit-il ! Que Yahweh fasse ainsi ! Que Yahweh accomplisse les paroles que tu as prophétisées, et qu'il fasse revenir de Babylone dans ce lieu-ci les ustensiles de la maison de Yahweh, et tous les captifs de Babylone !
\VS{7}Toutefois, écoute maintenant cette parole que je prononce, à tes oreilles et aux oreilles de tout le peuple :
\VS{8}Les prophètes qui ont été avant moi et avant toi, dès les temps anciens, ont prophétisé contre plusieurs pays et de grands royaumes, la guerre, le malheur et la peste ;
\VS{9}Le prophète qui aura prophétisé la paix, quand la parole de ce prophète sera accomplie, ce prophète-là sera reconnu pour avoir été véritablement envoyé par Yahweh.
\VS{10}Alors Hanania, le prophète, prit le joug de dessus le cou de Jérémie, le prophète, et le rompit.
\VS{11}Puis Hanania parla aux yeux de tout le peuple, en disant : Ainsi parle Yahweh : C'est ainsi que dans deux années, je romprai le joug de Nebucadnetsar, roi de Babylone, de dessus le cou de toutes les nations. Et Jérémie, le prophète, alla au loin par la route.
\VS{12}Mais la parole de Yahweh fut adressée à Jérémie, après que Hanania, le prophète, eut rompu le joug de dessus le cou de Jérémie, le prophète, en disant :
\VS{13}Va, et parle à Hanania, en disant : Ainsi parle Yahweh : Tu as rompu les jougs de bois, et tu auras à la place un joug de fer.
\VS{14}Car ainsi parle Yahweh des armées, le Dieu d'Israël : Je mets un joug de fer sur le cou de toutes ces nations, afin qu'elles servent Nebucadnetsar, roi de Babylone, et elles le serviront ; et je lui donne aussi les bêtes des champs\FTNT{De. 28:48.}.
\VS{15}Puis Jérémie, le prophète, dit à Hanania, le prophète : Ecoute maintenant, ô Hanania ! Yahweh ne t'a pas envoyé, et tu as fait que ce peuple se confie au mensonge\FTNT{Ez. 13:3-9.}.
\VS{16}C'est pourquoi ainsi parle Yahweh : Voici, je te chasse de la face de la terre ; et tu mourras cette année ; car tu as parlé de révolte contre Yahweh.
\VS{17}Et Hanania, le prophète, mourut cette année-là, dans le septième mois.
\Chap{29}
\TextTitle{Message à l'attention des Juifs captifs à Babylone}
\VerseOne{}Or ce sont ici les paroles de la lettre que Jérémie, le prophète, envoya de Jérusalem au reste des anciens en captivité, aux sacrificateurs et aux prophètes, et à tout le peuple, que Nebucadnetsar avait transportés de Jérusalem à Babylone,
\VS{2}après que le roi Jéconia fut sorti de Jérusalem, avec la reine, et les eunuques, et les chefs de Juda et de Jérusalem, et les charpentiers et les serruriers\FTNT{2 R. 24:12.}.
\VS{3}C'est par la main d'Eleasa, fils de Schaphan, et Guemaria, fils de Hilkija, que Sédécias, roi de Juda, l'envoya à Babylone vers Nebucadnetsar, roi de Babylone. La lettre disait :
\VS{4}Ainsi parle Yahweh des armées, le Dieu d'Israël, à tous les captifs que j'ai fait transporter de Jérusalem à Babylone.
\VS{5}Bâtissez des maisons, et habitez-les ; plantez des jardins, et mangez-en les fruits.
\VS{6}Prenez des femmes, et engendrez des fils et des filles ; prenez aussi des femmes pour vos fils, et donnez vos filles à des hommes, afin qu'elles enfantent des fils et des filles ; multipliez-vous là, et ne diminuez pas.
\VS{7}Et cherchez la paix de la ville où je vous ai transporté, et priez Yahweh pour elle ; parce que dans sa paix vous aurez la paix.
\VS{8}Car ainsi parle Yahweh des armées, le Dieu d'Israël : Que vos prophètes qui sont au milieu de vous, et vos devins, ne vous séduisent pas, et n'écoutez pas vos songes que vous vous songez\FTNT{Tous les songes ne viennent pas toujours du Seigneur. Les visions et les songes doivent être en accord avec la Parole de Dieu.}.
\VS{9}Parce qu'ils vous prophétisent faussement en mon nom. Je ne les ai pas envoyés, dit Yahweh.
\VS{10}Car ainsi parle Yahweh : Lorsque les soixante-dix ans seront accomplis pour Babylone, je vous visiterai, et j'accomplirai ma bonne parole à votre égard, pour vous faire revenir dans ce lieu.
\VS{11}Car je sais que les pensées que j'ai pour vous, dit Yahweh, sont des pensées de paix et non pas d'adversité, pour vous donner une fin telle que vous espérez\FTNT{Jos. 1:8.}.
\VS{12}Alors vous m'invoquerez, et vous partirez ; vous me prierez, et je vous exaucerai\FTNT{Os. 5:15.}.
\VS{13}Vous me chercherez, et vous me trouverez, après que vous m'aurez recherché de tout votre cœur\FTNT{Mt. 7:7.}.
\VS{14}Car je me laisserai trouver par vous, dit Yahweh, je ramènerai vos captifs ; et je vous rassemblerai d'entre toutes les nations et de tous les lieux où je vous ai chassés, dit Yahweh, et je vous ramènerai dans le lieu d'où je vous ai transportés.
\VS{15}Cependant si vous dites : Yahweh nous a suscité des prophètes à Babylone !
\VS{16} A cause de cela, ainsi parle Yahweh sur le roi qui est assis sur le trône de David, sur tout le peuple qui habite dans cette ville, sur vos frères qui ne sont pas allés avec vous en captivité ;
\VS{17}ainsi parle Yahweh des armées : Voici, je vais envoyer sur eux l'épée, la famine, et la peste, et je les ferai devenir comme des figues affreuses qui ne peuvent être mangées à cause de leur mauvaise qualité.
\VS{18}Et je les poursuivrai par l'épée, par la famine et par la peste, je les abandonnerai pour être agités par tous les royaumes de la terre, et pour être une malédiction, un étonnement, une moquerie et un opprobre parmi toutes les nations où je les chasserai\FTNT{De. 28:25-37.},
\VS{19}parce qu'ils n'ont pas écouté mes paroles, dit Yahweh, eux à qui j'ai envoyé mes serviteurs, les prophètes, en me levant dès le matin ; et ils n'ont pas écouté, dit Yahweh.
\VS{20}Vous tous donc, écoutez la parole de Yahweh, vous les captifs que j'ai envoyés de Jérusalem à Babylone !
\VS{21}Ainsi parle Yahweh des armées, le Dieu d'Israël sur Achab, fils de Kolaja, et sur Sédécias, fils de Maaséja, qui vous prophétisent faussement en mon nom : Voici, je vais les livrer entre les mains de Nebucadnetsar, roi de Babylone ; et il les frappera sous vos yeux.
\VS{22}Et on se servira d'eux comme une formule de malédiction, parmi tous les captifs de Juda qui sont à Babylone, en disant : Que Yahweh te mette dans un tel état, comme Sédécias et comme Achab, que le roi de Babylone a fait rôtir au feu !
\VS{23}parce qu'ils ont commis des impuretés en Israël, parce qu'ils ont commis l'adultère avec les femmes de leurs prochains, et qu'ils ont dit en mon nom des paroles fausses, alors que je ne leur avais pas commandées. Je le sais, et j'en suis témoin, dit Yahweh.
\VS{24}Parle aussi à Schemaeja, Néchélamite, en disant :
\VS{25}Ainsi parle Yahweh des armées, le Dieu d'Israël : Tu as envoyé en ton nom une lettre à tout le peuple de Jérusalem, à Sophonie, fils de Maaséja, le sacrificateur, et à tous les sacrificateurs, en disant :
\VS{26}Yahweh t'a établi sacrificateur à la place de Jehojada, le sacrificateur, afin qu'il y ait dans la maison de Yahweh des inspecteurs pour surveiller tout homme qui est fou et se donne pour prophète, et afin que tu le mettes en prison et dans les fers.
\VS{27}Et maintenant, pourquoi n'as-tu pas réprimé pas Jérémie d'Anathoth, qui prophétise parmi vous,
\VS{28}car à cause de cela il nous a envoyé dire à Babylone : La captivité sera longue ; bâtissez des maisons, et habitez-les ; plantez des jardins, et mangez-en les fruits !
\VS{29}Or Sophonie, le sacrificateur, lut cette lettre aux oreilles de Jérémie, le prophète.
\VS{30}C'est pourquoi la parole de Yahweh fut adressée à Jérémie, en disant :
\VS{31}Envoie dire à tous les captifs : Ainsi parle Yahweh sur Schemaeja, Néchélamite : Parce que Schemaeja vous prophétise, quoique que je ne l'aie envoyé, et qu'il vous a fait vous confier dans le mensonge,
\VS{32}à cause de cela, dit Yahweh : Je vais punir Schemaeja, Néchélamite, et sa postérité ; il n'y aura personne de sa race qui  habite au milieu de ce peuple, et il ne verra pas le bien que je ferai à mon peuple, dit Yahweh ; car il a parlé de révolte contre Yahweh.
\Chap{30}
\TextTitle{Le jour de Yahweh}
\VerseOne{}La parole qui fut adressée à Jérémie de la part de Yahweh, en disant :
\VS{2}Ainsi parle Yahweh, le Dieu d'Israël : Ecris pour toi dans un livre toutes les paroles que je t'ai dites.
\VS{3}Car voici, les jours viennent, dit Yahweh, où je ramènerai les captifs de mon peuple d'Israël et de Juda, dit Yahweh ; je les ramènerai dans le pays que j'ai donné à leurs pères, et ils le posséderont.
\VS{4}Ce sont ici les paroles que Yahweh a prononcées sur Israël et Juda.
\VS{5}Ainsi parle Yahweh : Nous entendons des cris d'effroi et de terreur, il n'y a pas de paix.
\VS{6}Informez-vous, je vous prie et voyez si un mâle enfante ! Pourquoi vois-je les hommes les mains sur leurs reins, comme une femme qui enfante ? Pourquoi tous les visages sont-ils devenus pâles ?
\VS{7}Malheur ! Que ce jour est grand ; il n'y en a pas eu de semblable. Il sera un temps de détresse pour Jacob ; mais il en sera pourtant délivré\FTNT{Joë. 2:11 ; So. 1:15 ; Da. 12:1 ; Mt. 24:21.}.
\VS{8}Et il arrivera en ce jour-là, dit Yahweh des armées, que je briserai son joug de dessus ton cou, je romprai tes liens, et les étrangers ne t'asserviront plus.
\VS{9}Mais ils serviront Yahweh, leur Dieu, et David, leur roi, que je leur susciterai\FTNT{Ez. 34:23-24.}.
\VS{10}Toi donc, mon serviteur Jacob, ne crains pas, dit Yahweh, et ne t'épouvante pas, ô Israël ! Car, voici, je te délivrerai du pays éloigné, et ta postérité du pays de leur captivité ; et Jacob reviendra, il sera en repos et sera en paix, et il n'y aura personne qui lui fasse peur\FTNT{Es. 41:13.}.
\VS{11}Car je suis avec toi, dit Yahweh, pour te délivrer ; et même je consumerai entièrement toutes les nations parmi lesquelles je t'ai dispersé, mais quand à toi, je ne te consumerai pas entièrement; je te châtierai avec équité, je ne te tiens pas entièrement pour innocent\FTNT{Es. 27:7-8.}.
\VS{12}Ainsi parle Yahweh : Ta blessure est incurable, ta plaie est très douloureuse\FTNT{Mi. 1:9 ; 2 Ch. 36:16.}.
\VS{13}Il n'y a personne qui défende ta cause, pour panser ta plaie ; il n'y a pour toi aucun remède, aucun moyen de guérison.
\VS{14}Tous tes amoureux t'oublient, ils ne te recherchent pas ; car je t'ai frappée d'une plaie d'ennemi, d'un châtiment d'homme cruel, à cause de la multitude de tes iniquités, tes péchés se sont renforcés\FTNT{La. 1:2.}.
\VS{15}Pourquoi cries-tu à cause de ta plaie ? Ta douleur est hors d'espérance ; je t'ai fait ces choses à cause de la grandeur de tes iniquités, du grand nombre de tes péchés.
\VS{16}Néanmoins tous ceux qui te dévorent seront dévorés, et tous ceux qui te mettent dans la détresse, iront en captivité ; ceux qui te dépouillent seront dépouillés, et je livrerai au pillage tous ceux qui te pillent\FTNT{Es. 41:11 ; Ab. 15.}.
\VS{17}Même je guérirai tes plaies, et je te guérirai tes blessures, dit Yahweh. Car ils t'appellent la repoussée, cette Sion que personne ne recherche.
\TextTitle{Israël délivré par Yahweh}
\VS{18}Ainsi parle Yahweh : Voici, je ramène les captifs des tentes de Jacob, j'ai compassion de ses demeures ; la ville sera rebâtie sur le monceau de ses ruines et le palais sera rétabli comme il était.
\VS{19}Et il en sortira des remerciements et des cris de joie ; et je les multiplierai, et ils ne diminueront pas ; je les honorerai, et ils ne seront pas amoindris.
\VS{20}Et ses enfants seront comme autrefois, son assemblée sera affermie devant moi, et je punirai tous ceux qui l'oppriment.
\VS{21}Et son chef sera tiré de son sein, son dominateur sortira du milieu de lui ; je le ferai approcher, et il viendra vers moi ; car qui disposerait son cœur pour venir vers moi ? dit Yahweh.
\VS{22}Et vous serez mon peuple, et je serai votre Dieu.
\VS{23}Voici, la tempête de Yahweh, la fureur éclate, un tourbillon qui s'entasse ; il tombera sur la tête des méchants. 
\VS{24}L'ardeur de la colère de Yahweh ne se détournera pas, jusqu'à ce qu'il ait exécuté, accompli les desseins de son cœur ; vous le comprendrez dans les derniers jours.
\Chap{31}
\TextTitle{Communion retrouvée : La paix et et la joie}
\VerseOne{}En ce temps-là, dit Yahweh, je serai le Dieu de toutes les familles d'Israël, et ils seront mon peuple.
\VS{2}Ainsi parle Yahweh : Le peuple survivant à l'épée a trouvé grâce dans le désert ; Israël marche vers son lieu de repos.
\VS{3}De loin Yahweh m'est apparu, et m'a dit : Je t'aime d'un amour éternel, c'est pourquoi j'ai prolongé ma bonté envers toi.
\VS{4}Je te rétablirai encore, et tu seras rétablie, ô vierge d'Israël ! Tu te pareras encore de tes tambours, et tu sortiras au milieu des danses joyeuses.
\VS{5}Tu planteras encore des vignes sur les montagnes de Samarie ; les vignerons planteront et recueilleront les fruits pour leur usage\FTNT{Es. 65:21.}.
\VS{6}Car il y a un jour où les gardes crieront sur la montagne d'Ephraïm : Levez-vous, et montons à Sion, vers Yahweh, notre Dieu !
\VS{7}Car ainsi parle Yahweh : Réjouissez-vous avec chant de triomphe, et avec allégresse à cause de Jacob, et vous égayez à cause du chef des nations ! Faites-le entendre, chantez des louanges, et dites : Yahweh, délivre ton peuple, le reste d’Israël !
\VS{8}Voici, je vais les faire venir du pays du nord, et je les rassemblerai des extrémités de la terre ; l'aveugle et le boiteux, la femme enceinte et celle qui enfante seront ensemble parmi eux ; une grande assemblée qui reviendra ici.
\VS{9}Ils y seront allés en pleurant, mais je les ferai retourner avec des supplications, et je les conduirai aux torrents d'eaux, et par un droit chemin, où ils ne broncheront pas ; car je suis un père pour Israël, et Ephraïm est mon premier-né\FTNT{Ex. 4:22.}.
\VS{10}Nations, écoutez la parole de Yahweh, et annoncez-la aux îles éloignées ! Dites : Celui qui a dispersé Israël le rassemblera, et il le gardera comme un berger garde son troupeau.
\VS{11}Car Yahweh rachète Jacob, et le retire de la main d'un ennemi plus fort que lui.
\VS{12}Ils viendront donc, et se réjouiront avec des chants de triomphe sur les hauteurs de Sion ; ils afflueront vers les biens de Yahweh, le blé, le vin, l'huile, et le fruit du gros et du menu bétail ; et leur âme sera comme un jardin arrosé, et ils ne seront plus dans la souffrance\FTNT{Es. 61:11.}.
\VS{13}Alors la vierge se réjouira à la danse, les jeunes hommes et les anciens ensemble ; je changerai leur deuil en joie, et je les consolerai ; et je les réjouirai en les délivrant de leur douleur.
\VS{14}Je rassasierai aussi de graisse l'âme des sacrificateurs, et mon peuple sera rassasié de mes biens, dit Yahweh.
\VS{15}Ainsi parle Yahweh : On entend des cris à Rama, des lamentations, des larmes amères ; Rachel pleure ses fils ; elle refuse d'être consolée sur ses fils, car ils ne sont plus\FTNT{Mt. 2:17-18.}.
\VS{16}Ainsi parle Yahweh : Retiens ta voix de pleurer, et tes yeux de verser des larmes, car ton œuvre aura son salaire, dit Yahweh ; et ils reviendront des terres de l'ennemi.
\VS{17}Et il y a de l'espérance pour tes derniers jours, dit Yahweh ; et tes fils reviendront dans leur territoire.
\VS{18}J'ai très bien entendu Ephraïm se plaignant, et disant : Tu m'as châtié, et j'ai été châtié comme un veau qui n'est pas dompté. Fais-moi revenir, et je reviendrai, car tu es Yahweh, mon Dieu\FTNT{Ps. 119:67-71.}.
\VS{19}Certes, après m'être détourné, je me repens ; et après avoir reconnu mes fautes, je frappe sur ma cuisse ; je suis honteux et confus, car je porte l'opprobre de ma jeunesse\FTNT{Ez. 21:17.}.
\VS{20}Ephraïm est-il donc pour moi un cher fils, un fils qui fait mes délices ? Car plus je parle de lui, plus encore son souvenir est en moi ; aussi mes entrailles sont émues en sa faveur : J'aurai certainement pitié de lui, dit Yahweh\FTNT{Es. 5:7.}.
\VS{21}Dresse-toi des signes sur les chemins, place des poteaux, prends garde à la route, au chemin par lequel tu es venue… Reviens, vierge d'Israël, reviens dans tes villes !
\VS{22}Jusqu'à quand seras-tu errante, fille rebelle ? Car Yahweh crée une chose nouvelle sur la terre : La femme entourera l'homme.
\VS{23}Ainsi parle Yahweh des armées, le Dieu d'Israël : On dira encore cette parole-ci dans le pays de Juda et dans ses villes, quand j'aurai ramené leurs captifs : Que Yahweh te bénisse, ô agréable demeure de la justice, montagne de sainteté !
\VS{24}Juda et toutes ses villes ensemble, les laboureurs, et ceux qui conduisent les troupeaux, y habiteront.
\VS{25}Car j'abreuverai l'âme épuisée par le travail, et je remplirai toute âme languissante.
\VS{26}C'est pourquoi je me suis réveillé, et j'ai regardé ; mon sommeil m'avait été agréable.
\TextTitle{Promesse d'une nouvelle alliance}
\VS{27}Voici, les jours viennent, dit Yahweh, où j'ensemencerai la maison d'Israël et la maison de Juda d'une semence d'hommes et d'une semence de bêtes.
\VS{28}Et il arrivera que comme j'ai veillé sur eux pour arracher et démolir, pour détruire, pour perdre et pour faire du mal ; ainsi je veillerai sur eux pour bâtir et pour planter, dit Yahweh.
\VS{29}En ces jours-là, on ne dira plus : Les pères ont mangé des raisins verts, et les dents des fils en ont été agacées\FTNT{Ez. 18:2-3.}.
\VS{30}Mais chacun mourra pour son iniquité ; tout homme qui mangera des raisins verts, ses dents en seront agacées.
\VS{31}Voici, les jours viennent, dit Yahweh, où je traiterai une nouvelle alliance\FTNT{Il s'agit de l'alliance du sang que Jésus, notre Messie, est venu inaugurer en prenant sur lui tous nos péchés et en mourant sur la croix à notre place (Mt. 26:27-29 ; Hé. 8:7-13).} avec la maison d'Israël et avec la maison de Juda,
\VS{32}non comme l'alliance que je traitai avec leurs pères, le jour où je les pris par la main, pour les faire sortir du pays d'Egypte, mon alliance qu'ils ont violée ; et toutefois j’avais été pour eux un mari, dit Yahweh.
\VS{33}Car c'est ici l'alliance que je traiterai avec la maison d'Israël, après ces jours-là, dit Yahweh, je mettrai ma loi au-dedans d'eux, je l'écrirai dans leur cœur ; et je serai leur Dieu, et ils seront mon peuple.
\VS{34}Aucun homme parmi eux n'enseignera plus son prochain, ni personne son frère, en disant : Connaissez Yahweh ! Car tous me connaîtront, depuis le plus petit jusqu'au plus grand, dit Yahweh ; parce que je pardonnerai leur iniquité, et que je ne me souviendrai plus de leur péché\FTNT{Es. 54:13 ; Ha. 2:14 ; Jn. 6:45.}.
\VS{35}Ainsi parle Yahweh, qui a donné le soleil pour être la lumière du jour, et qui a réglé la lune et les étoiles pour être la lumière de la nuit, qui remue la mer, et fait gronder ses flots, lui dont le nom est Yahweh des armées\FTNT{Ge. 1:16 ; Es. 51:15.} :
\VS{36}Si ces lois\FTNT{Les lois de l'univers ont été établies par Yahweh. Ces lois sont : la loi de la gravité, la loi de l'attraction et la loi de la résonance (voir Ps. 148:5-6 ; Job. 38:33).} viennent à cesser devant moi, dit Yahweh, la race d'Israël aussi cessera d'être à jamais une nation devant moi.
\VS{37}Ainsi parle Yahweh : Si les cieux en haut peuvent être mesurés, si les fondements de la terre en bas peuvent être sondés, alors je rejetterai toute la race d'Israël, à cause de toutes les choses qu'ils ont faites, dit Yahweh.
\VS{38}Voici, les jours viennent, dit Yahweh, où cette ville sera rebâtie à Yahweh, depuis la tour de Hananeel, jusqu'à la porte de l'angle\FTNT{Za. 14:10 ; Né. 3:1 ; 2 Ch. 26:9.}.
\VS{39}Le cordeau à mesurer sera encore tiré vis-à-vis d'elle, sur la colline de Gareb, et tournera vers Goath.
\VS{40}Et toute la vallée des cadavres et des cendres, et tous les champs jusqu'au torrent de Cédron, jusqu'à l'angle de la porte des chevaux à l'orient, seront consacrés à Yahweh, et ne seront plus jamais arrachés ni détruits.
\Chap{32}
\TextTitle{Le champ de Hanameel : La pérennité d'Israël}
\VerseOne{}La parole qui fut adressée à Jérémie de la part de Yahweh, la dixième année de Sédécias, roi de Juda. C'était la dix-huitième année de Nebucadnetsar.
\VS{2}Or l'armée du roi de Babylone assiégeait alors Jérusalem ; et Jérémie le prophète était enfermé dans la cour de la prison, qui était dans la maison du roi de Juda ;
\VS{3}car Sédécias, roi de Juda, l'avait fait enfermer, et lui avait dit : Pourquoi prophétises-tu, en disant : Ainsi parle Yahweh : Voici, je vais livrer cette ville entre les mains du roi de Babylone, et il la prendra ;
\VS{4}et Sédécias, roi de Juda, n'échappera pas aux mains des Chaldéens ; mais il sera livré entre les mains du roi de Babylone, et lui parlera bouche à bouche, et ses yeux verront les yeux de ce roi ;
\VS{5}il emmènera Sédécias à Babylone, qui y demeurera jusqu'à ce que je le visite, dit Yahweh ; si vous combattez contre les Chaldéens, vous ne prospérerez pas.
\VS{6}Jérémie donc dit : La parole de Yahweh m'a été adressée, en disant :
\VS{7}Voici Hanameel, fils de Schallum, ton oncle, qui vient vers toi pour te dire : Achète mon champ qui est à Anathoth, car tu as le droit de rachat pour l'acquérir\FTNT{Lé. 25:48 ; Ru. 3:12.}.
\VS{8}Hanameel donc, fils de mon oncle, vint à moi, selon la parole de Yahweh, dans la cour de la prison, et me dit : Achète, je te prie, mon champ, qui est à Anathoth, dans le pays de Benjamin, car tu as le droit d'héritage et de rachat, achète-le ! Et je connus alors que c'était la parole de Yahweh.
\VS{9}Ainsi j'achetai de Hanameel, fils de mon oncle, le champ qui est à Anathoth, et je lui pesai l'argent, qui fut dix-sept sicles d'argent.
\VS{10}Puis j'écrivis le contrat, que je cachetai, je pris des témoins après avoir pesé l'argent sur la balance.
\VS{11}Et je pris le contrat d'acquisition, celui qui était cacheté, selon les ordonnances et les statuts, et celui qui était ouvert ;
\VS{12}Et je remis le contrat d'acquisition à Baruc, fils de Nérija, fils de Machséja, sous les yeux de Hanameel, fils de mon oncle, des témoins qui avaient signé le contrat d'acquisition, et sous les yeux de tous les juifs qui étaient assis dans la cour de la prison.
\VS{13}Puis je donnai sous leurs yeux cet ordre à Baruc, en disant :
\VS{14}Ainsi parle Yahweh des armées, le Dieu d'Israël : Prends ces contrats-ci, à savoir, ce contrat d'acquisition, celui qui est scellé, et celui qui est ouvert, et mets-les dans un vase de terre, afin qu'ils se conservent longtemps.
\VS{15}Car ainsi parle Yahweh des armées, le Dieu d'Israël : On achètera encore des maisons, des champs et des vignes, dans ce pays.
\TextTitle{Promesse du retour des Juifs en Israël}
\VS{16}Et après que j'eus donné à Baruc, fils de Nérija, le contrat d'acquisition, je fis cette prière à Yahweh, en disant :
\VS{17}Ah ! Ah ! Seigneur Yahweh, voici, tu as fait les cieux et la terre par ta grande puissance et par ton bras étendu : Aucune chose n'est étonnante de ta part.
\VS{18}Tu fais miséricorde jusqu'à la millième génération, et tu punis l'iniquité des pères dans le sein de leurs fils après eux\FTNT{Ex. 34:7 ; Es. 65:7 ; Ps. 79:12.}. Tu es le Dieu, le Grand, le Puissant, dont le nom est Yahweh des armées.
\VS{19}Tu es grand en conseil et puissant en actions ; tes yeux sont ouverts sur toutes les voies des fils des hommes, pour rendre à chacun selon ses voies, et selon le fruit de ses œuvres.
\VS{20}Tu as fait dans le pays d'Egypte des miracles et des prodiges qui sont connus jusqu'à ce jour, et en Israël et parmi les hommes, tu t'es fait un nom tel qu'il est aujourd'hui.
\VS{21}Car tu as fait sortir du pays d'Egypte ton peuple d'Israël, avec des miracles et des prodiges, et avec une main forte, et avec un bras étendu, et en répandant partout une grande terreur ;
\VS{22}Et tu leur as donné ce pays, que tu avais juré à leurs pères de leur donner, pays où coulent le lait et le miel.
\VS{23}Et ils y sont entrés, ils l'ont possédé ; mais ils n'ont pas obéi à ta voix, et n'ont pas marché dans ta loi, et n'ont pas fait tout ce que tu leur avais ordonné de faire. C'est pourquoi tu as fait arriver sur eux tout ce mal-ci !
\VS{24}Voilà, les terrasses sont élevées, on est venu contre la ville pour la prendre, et à cause de l’épée, de la famine, et de la peste, la ville est livrée entre la main des Chaldéens qui combattent contre elle ; et ce que tu as dit est arrivé, et voici, tu le vois.
\VS{25}Et cependant, Seigneur Yahweh ! Tu m'as dit : Achète-toi ce champ à prix d'argent, et prends-en des témoins, quoique la ville soit livrée entre les mains des Chaldéens.
\VS{26}Mais la parole de Yahweh fut adressée à Jérémie, en disant :
\VS{27}Voici, je suis Yahweh, le Dieu de toute chair. Y a-t-il quelque chose d'étonnant de ma part ?
\VS{28}C'est pourquoi ainsi parle Yahweh : Voici, je vais livrer cette ville entre les mains des Chaldéens, et entre les mains de Nebucadnetsar, roi de Babylone, qui la prendra.
\VS{29}Et les Chaldéens qui combattent contre cette ville, y entreront, et mettront le feu à cette ville, et la brûleront, avec les maisons sur les toits desquelles on a brûlé de l'encens à Baal, et où l'on a fait des libations à d'autres dieux pour m'irriter.
\VS{30}Car les fils d'Israël et les fils de Juda n'ont fait, dès leur jeunesse, que ce qui est mal à mes yeux ; les fils d'Israël n'ont fait que m'irriter par les œuvres de leurs mains, dit Yahweh.
\VS{31}Car cette ville a été portée à provoquer ma colère et ma fureur, depuis le jour qu'ils l'ont bâtie, jusqu'à ce jour, afin que je l'ôte de devant ma face ;
\VS{32}à cause de tout le mal que les fils d'Israël et les fils de Juda ont fait pour m'irriter, eux, leurs rois, leurs chefs, leurs sacrificateurs et leurs prophètes, les hommes de Juda et les habitants de Jérusalem.
\VS{33}Ils m'ont tourné le dos, et non la face ; je les ai enseignés, je les ai enseignés dès le matin, mais ils n'ont pas écouté pour recevoir l'instruction.
\VS{34}Mais ils ont mis leurs abominations dans la maison sur laquelle mon Nom est invoqué, pour la souiller.
\VS{35}Et ils ont bâti les hauts lieux de Baal, qui sont dans la vallée de Ben-Hinnom, pour faire passer par le feu leurs fils et leurs filles à Moloc\FTNT{Voir commentaire en Lé. 20:2.} ; ce que je ne leur avais pas ordonné, et il ne m'était pas monté à la pensée qu'ils feraient cette abomination pour faire pécher Juda.
\VS{36}Et maintenant, à cause de cela Yahweh, le Dieu d'Israël, ainsi parle sur cette ville dont vous dites qu’elle est livrée entre les mains du roi de Babylone, à cause que l’épée, la famine, et la peste sont en elle :
\VS{37}Voici, je vais les rassembler de tous les pays où je les ai chassés, dans ma colère, dans ma fureur et dans mon grand courroux ; et je les ramènerai dans ce lieu-ci, et je les y ferai habiter en sécurité.
\VS{38}Et Ils seront mon peuple, et je serai leur Dieu.
\VS{39}Et je leur donnerai un même cœur et une même voie, afin qu'ils me craignent à toujours, pour leur bien et celui de leurs fils après eux.
\VS{40}Et je traiterai avec eux une alliance éternelle, à savoir, que je ne me détournerai plus d'eux pour leur faire du bien ; et je mettrai ma crainte dans leur cœur, afin qu'ils ne se détournent pas de moi\FTNT{Es. 54:10.}.
\VS{41}Et je me réjouirai à leur faire du bien, et je les planterai dans ce pays-ci solidement, de tout mon cœur, et de toute mon âme.
\VS{42}Car ainsi parle Yahweh : Comme j'ai fait venir tous ce grand mal sur ce peuple, ainsi je ferai venir sur eux tout le bien que je prononce en leur faveur.
\VS{43}Et on achètera des champs dans ce pays, duquel vous dites que ce n'est que désolation, sans hommes ni bêtes, et qui est livré entre les mains des Chaldéens.
\VS{44}On achètera, dis-je, des champs à prix d'argent, et on en écrira les contrats, et on les cachettera, et on en prendra des témoins dans le pays de Benjamin, et aux environs de Jérusalem, dans les villes de Juda, tant dans les villes des montagnes, que dans les villes de la plaine, et dans les villes du midi. Car je ramènerai leurs captifs, dit Yahweh.
\Chap{33}
\TextTitle{Jésus, le Germe appelé à régner\FTNTT{Voir 2 S. 7:8-16.}}
\VerseOne{}Et la parole de Yahweh fut adressée une seconde fois à Jérémie, quand il était encore enfermé dans la cour de la prison, en disant :
\VS{2}Ainsi parle Yahweh, qui fait ces choses, Yahweh qui les forme et les établit, lui dont le nom est Yahweh :
\VS{3}Crie vers moi\FTNT{Yahweh, qui demandait qu'on l'invoque, n'est autre que Jésus-Christ, notre Seigneur (Joë. 2:32 ; 1 Co. 1:2 ; Ro. 10:13).}, je te répondrai, et je t'annoncerai des choses grandes, des choses cachées, que tu ne connais pas.
\VS{4}Car ainsi parle Yahweh, le Dieu d'Israël, touchant les maisons de cette ville-ci et les maisons des rois de Juda ; elles seront abattues par les terrasses et par l'épée.
\VS{5}Ils sont venus pour combattre contre les Chaldéens, mais ça été pour remplir leurs maisons des cadavres des hommes que j'ai frappé dans ma colère et dans ma fureur, et parce que j'ai caché ma face de cette ville à cause toute leur méchanceté.
\VS{6}Voici, je vais lui donner la santé et la guérison, je les guérirai, et je leur ferai découvrir une abondance de paix et de fidélité\FTNT{Ap. 22:1-2.}.
\VS{7}Et je ramènerai les captifs de Juda, et les captifs d'Israël, et je les rétablirai comme autrefois.
\VS{8}Et je les purifierai de toute leur iniquité, par laquelle ils ont péché contre moi ; et je pardonnerai toutes leurs iniquités par lesquelles ils ont péché contre moi, et par lesquelles ils se sont révoltés contre moi\FTNT{Ez. 37:23.}.
\VS{9}Et cette ville sera pour moi un sujet de joie, de louange et de gloire, parmi toutes les nations de la terre qui entendront parler de tout le bien que je leur ferai, et elles seront dans la crainte et trembleront à cause de tout le bien et de toute la prospérité que je vais lui donner.
\VS{10}Ainsi parle Yahweh : Dans ce lieu-ci duquel vous dites : Il est désert, il n'y a plus d'hommes, plus de bêtes, dans les villes de Juda, et dans les rues de Jérusalem, qui sont désolées, privées d'hommes, d'habitants, de bêtes,
\VS{11}on y entendra encore les cris de joie et les cris d'allégresse, la voix de l'époux et la voix de l'épouse, et la voix de ceux qui disent : Louez Yahweh des armées ; car Yahweh est bon, parce que sa miséricorde demeure à toujours, lorsqu'ils offriront des offrandes de reconnaissance dans la maison de Yahweh ; car je ramènerai les captifs de ce pays, et je les rétablirai comme autrefois, dit Yahweh.
\VS{12}Ainsi parle Yahweh des armées : Dans ce lieu désert, où il n'y a ni hommes ni bêtes, et dans toutes ses villes, il y aura encore des demeures de bergers qui y feront reposer leurs troupeaux ;
\VS{13}dans les villes des montagnes, et dans les villes de la plaine, dans les villes du midi, dans le pays de Benjamin et aux environs de Jérusalem, et dans les villes de Juda ; les brebis passeront encore sous les mains de celui qui les compte, dit Yahweh.
\VS{14}Voici, les jours viennent, dit Yahweh, où j'accomplirai la bonne parole que j'ai prononcée sur la maison d'Israël et la maison de Juda.
\VS{15}En ces jours et en ce temps-là, je ferai germer à David le Germe de justice, qui exercera le jugement et la justice dans le pays.
\VS{16}En ces jours-là, Juda sera sauvé, Jérusalem habitera en sécurité ; et voici comment on l'appellera : Yahweh notre justice.
\VS{17}Car ainsi parle Yahweh : David ne manquera jamais d'un successeur assis sur le trône de la maison d'Israël ;
\VS{18}et d'entre les sacrificateurs Lévites, il ne manquera jamais d'y avoir devant moi d'homme offrant des holocaustes, brûlant de l'encens avec les offrandes, et faisant des sacrifices tous les jours.
\VS{19}La parole de Yahweh fut encore adressée à Jérémie, en disant :
\VS{20}Ainsi parle Yahweh : Si vous pouvez rompre mon alliance avec le jour et mon alliance avec la nuit, de sorte que le jour et la nuit ne soient plus en leur temps,
\VS{21}alors aussi mon alliance avec David, mon serviteur, sera rompue ; de sorte qu'il n'aura plus de fils régnant sur son trône ; et avec les Lévites sacrificateurs, faisant mon service.
\VS{22}Car comme on ne peut compter l'armée des cieux, ni mesurer le sable de la mer, ainsi je multiplierai la postérité de David mon serviteur, et les Lévites qui font mon service\FTNT{Ge. 2:1 ; Ge. 15:5.}.
\VS{23}La parole de Yahweh fut encore adressée à Jérémie, en disant :
\VS{24}N'as-tu pas vu ce que ce peuple prononce, en disant : Yahweh a rejeté les deux familles qu'il avait élues ? Ainsi ils méprisent mon peuple, ils ne sont plus une nation devant eux.
\VS{25}Ainsi parle Yahweh : Si je n'ai pas fait mon alliance avec le jour et la nuit, et si je n'ai pas établi les ordonnances des cieux et de la terre ;
\VS{26}aussi rejetterai-je la postérité de Jacob, et celle de David mon serviteur, pour ne plus prendre de sa postérité des gens qui dominent sur les descendants d'Abraham, d'Isaac et de Jacob ; car je ramènerai leurs captifs, et j'aurai compassion d'eux.
\Chap{34}
\TextTitle{Désobéissance du peuple : Jérusalem dévastée}
\VerseOne{}La parole qui fut adressée à Jérémie de la part de Yahweh, lorsque Nebucadnetsar, roi de Babylone, et toute son armée, et tous les royaumes de la terre, et tous les peuples qui étaient sous la puissance de sa main, combattaient contre Jérusalem, et contre toutes ses villes, en disant\FTNT{2 R. 25:1-2.} :
\VS{2}Ainsi parle Yahweh, le Dieu d'Israël : Va, et parle à Sédécias, roi de Juda, et dis-lui : Ainsi parle Yahweh : Voici, je vais livrer cette ville entre les mains du roi de Babylone, et il la brûlera par le feu.
\VS{3}Et tu n'échapperas pas de sa main, car certainement tu seras pris et tu seras livré entre ses mains, et tes yeux verront les yeux du roi de Babylone, et il te parlera bouche à bouche, et tu iras à Babylone.
\VS{4}Toutefois écoute la parole de Yahweh, ô Sédécias, roi de Juda ! Ainsi parle Yahweh sur toi : Tu ne mourras pas par l'épée ;
\VS{5}mais tu mourras en paix, et on brûlera pour toi des parfums aromatiques, comme on en a brûlé pour tes pères, les rois précédents qui ont été avant toi ; et on te pleurera, en disant : Hélas, Seigneur ! Car j'ai prononcé cette parole, dit Yahweh\FTNT{2 Ch. 16:14.}.
\VS{6}Jérémie, le prophète, dit toutes ces paroles à Sédécias, roi de Juda, à Jérusalem.
\VS{7}Et l'armée du roi de Babylone combattait contre Jérusalem et contre toutes les villes de Juda qui restaient, à savoir, contre Lakis et contre Azéka, car c'étaient les seules villes fortifiées qui restaient parmi les villes de Juda\FTNT{2 R. 18:13.}.
\TextTitle{Jérusalem deviendra une désolation à cause de la désobéissance}
\VS{8}La parole fut adressée à Jérémie de la part de Yahweh, après que le roi Sédécias eut traité une alliance avec tout le peuple de Jérusalem, pour proclamer la liberté,
\VS{9}afin que chacun renvoie libre son esclave et chacun sa servante, l'hébreu ou la femme de l'hébreu, et qu'aucun juif ne soit l'esclave de son frère.
\VS{10}Tous les chefs et tout le peuple, qui étaient entrés dans cette alliance, entendirent que chacun devait renvoyer libre son serviteur et chacun sa servante, sans plus les asservir ; ils obéirent et les renvoyèrent.
\VS{11}Mais ensuite, ils changèrent d'avis ; ils firent revenir leurs esclaves et leurs servantes, qu'ils avaient renvoyés libres, et les assujettirent pour être leurs esclaves et leurs servantes.
\VS{12}Et la parole de Yahweh fut adressée à Jérémie en disant :
\VS{13}Ainsi parle Yahweh, le Dieu d'Israël : J'ai traité une alliance avec vos pères, le jour où je les ai fait sortir du pays d'Egypte, de la maison de servitude, en disant :
\VS{14}A la fin de la septième année, chacun renverra libre son frère hébreu qui aura été vendu ; il te servira six années, puis tu le renverras libre de chez toi. Mais vos pères ne m'ont pas écouté, ils n'ont pas prêté l'oreille\FTNT{Ex. 21:2 ; Lé. 25:10-15 ; De. 15:12.}.
\VS{15}Et vous, qui aujourd'hui étiez revenus à vous-mêmes, et vous aviez fait ce qui était droit à mes yeux, en publiant la liberté chacun pour son prochain, vous aviez traité une alliance devant moi, dans la maison sur laquelle mon Nom est invoqué.
\VS{16}Mais vous êtes revenus en arrière, et vous avez souillé mon Nom ; vous avez fait revenir chacun ses esclaves et ses servantes, que vous aviez renvoyés libres, rendus à eux-mêmes, et vous les avez assujettis, afin qu'ils soient pour vous des serviteurs et des servantes.
\VS{17}C'est pourquoi ainsi parle Yahweh : Vous ne m'avez pas obéi, en publiant la liberté chacun à son frère, et chacun à son prochain. Voici, je vais publier contre vous, dit Yahweh, la liberté contre vous à l'épée, à la peste, et à la famine ;  et je vous livrerai pour être transportés par tous les royaumes de la terre.
\VS{18}Et je livrerai les hommes qui ont transgressé mon alliance, et qui n'ont pas observé les paroles de l'alliance qu'ils avaient traitée devant moi, lorsqu'ils sont passés entre les morceaux du veau qu'ils ont coupé en deux ;
\VS{19}les chefs de Juda, et les chefs de Jérusalem, les eunuques, et les sacrificateurs, et tout le peuple du pays, qui sont passés au travers des morceaux du veau ;
\VS{20}je les livrerai, dis-je, entre les mains de leurs ennemis, entre les mains de ceux qui cherchent leur vie ; et leurs cadavres seront la pâture des oiseaux des cieux et des bêtes de la terre.
\VS{21}Je livrerai aussi Sédécias, roi de Juda, et les chefs de sa cour, entre les mains de leurs ennemis, entre les mains de ceux qui cherchent leur vie, entre les mains de l'armée du roi de Babylone, qui s'est retiré de devant vous.
\VS{22}Voici, je vais leur donner mes ordres, dit Yahweh, et je les ramènerai contre cette ville-ci ; et ils combattront contre elle, et la prendront, et la brûleront au feu ; et je ferai des villes de Juda un désert sans habitants.
\Chap{35}
\TextTitle{L'obéissance des Récabites}
\VerseOne{}C'est ici la parole qui fut adressée à Jérémie de la part de Yahweh, au temps de Jojakim, fils de Josias, roi de Juda, en disant :
\VS{2}Va à la maison des Récabites, et parle-leur, et fais les venir à la maison de Yahweh, dans l'une des chambres, et présente-leur du vin à boire\FTNT{2 S. 4:2 ; 1 Ch. 2:55.}.
\VS{3}Je pris donc Jaazania, fils de Jérémie, fils de Habazinia, et ses frères, et tous ses fils, et toute la maison des Récabites,
\VS{4}et je les fis venir dans la maison de Yahweh, dans la chambre des fils de Hanan, fils de Jigdalia, homme de Dieu, qui était près de la chambre des chefs, au-dessus de la chambre de Maaséja, fils de Schallum, garde du seuil.
\VS{5}Et je mis devant les fils de la maison des Récabites des coupes pleines de vin, et des calices, et je leur dis : Buvez du vin !
\VS{6}Et ils répondirent : Nous ne buvons pas de vin ; car Jonadab, fils de Récab, notre père, nous a donné cet ordre en disant : Vous ne boirez jamais de vin, ni vous ni vos fils\FTNT{Lé. 10:9 ; No. 6:2-4.} ;
\VS{7}vous ne bâtirez aucune maison, vous ne sèmerez aucune semence, vous ne planterez aucune vigne, et vous n'en aurez pas ; mais vous habiterez sous des tentes toute votre vie, afin que vous viviez longtemps sur la terre où vous êtes étrangers.
\VS{8}Nous avons donc obéi à la voix de Jonadab, fils de Récab, notre père dans toutes les choses qu'il nous a ordonnées, de sorte que nous n'avons pas bu de vin tous les jours de notre vie, ni nous, ni nos femmes, ni nos fils, ni nos filles.
\VS{9}Nous n'avons bâti aucune maison pour notre demeure, et nous n'avons eu ni vigne, ni champ, ni semence.
\VS{10}Mais nous avons habité sous des tentes, et nous avons obéi, et nous avons fait selon toutes les choses que Jonadab, notre père, nous a ordonnées.
\VS{11}Mais il est arrivé que quand Nebucadnetsar, roi de Babylone, est monté au pays, nous avons dit : Venez, et entrons dans Jérusalem, pour fuir de devant l'armée des Chaldéens, et de devant l'armée de Syrie. C'est ainsi que nous habitons à Jérusalem.
\VS{12}Alors la parole de Yahweh fut adressée à Jérémie, en disant :
\VS{13}Ainsi parle Yahweh des armées, le Dieu d'Israël : Va, et dis aux hommes de Juda, et aux habitants de Jérusalem : Ne recevrez-vous pas d'instruction pour obéir à mes paroles ? Dit Yahweh.
\VS{14}Toutes les paroles de Jonadab, fils de Récab, qui a ordonné à ses fils de ne pas boire de vin, ont été observées, et ils n'en ont pas bu jusqu'à ce jour ; mais ils ont obéi au commandement de leur père ; mais moi, je vous ai parlé, je vous ai parlé dès le matin, et vous ne m'avez pas obéi.
\VS{15}Car je vous ai envoyé tous les prophètes, mes serviteurs, je les ai envoyés dès le matin, pour vous dire : Revenez maintenant chacun de votre mauvaise voie, et amendez vos actions, et n'allez pas après d'autres dieux pour les servir, afin que vous demeureriez dans le pays que j'ai donné à vous et à vos pères. Mais vous n'avez pas prêté l'oreille, et vous ne m'avez pas écouté.
\VS{16}Parce que les fils de Jonadab, fils de Récab, ont observé le commandement que leur avait donné leur père, et que ce peuple ne m'écoute pas ;
\VS{17}à cause de cela, Yahweh le Dieu des armées, le Dieu d'Israël, parle ainsi : Voici, je vais faire venir sur Juda et sur tous les habitants de Jérusalem tout le mal que j'ai prononcés contre eux ; parce que je leur ai parlé, et ils n'ont pas écouté ; et que je les ai appelés, et ils n'ont pas répondu.
\VS{18}Et Jérémie dit à la maison des Récabites: Ainsi parle Yahweh des armées, le Dieu d'Israël : Parce que vous avez obéi au commandement de Jonadab, votre père, et que vous avez gardé tous ses commandements, et avez fait selon tout ce qu'il vous a ordonné\FTNT{Les Récabites furent bénis parce qu'ils obéirent aux commandements de leur père (Ep. 6:1-3)} ;
\VS{19}c'est pourquoi, ainsi parle Yahweh des armées, le Dieu d'Israël : Jonadab, fils de Récab, ne manquera jamais de descendants qui se tiennent debout devant moi.
\Chap{36}
\TextTitle{Le roi Jojakim brûle le manuscrit de Jérémie}
\VerseOne{}Or il arriva, dans la quatrième année de Jojakim, fils de Josias, roi de Juda, que cette parole fut adressée à Jérémie de la part de Yahweh, en disant :
\VS{2}Prends-toi un rouleau de livre, et tu y écriras toutes les paroles que je t'ai dites contre Israël et contre Juda, et contre toutes les nations, depuis le jour où je t'ai parlé, c'est à dire, depuis le jours de Josias, jusqu'à ce jour.
\VS{3}Peut-être que la maison de Juda entendra tout le mal que je pense de leur faire, afin que chaque homme se détourne de sa mauvaise voie, et que je leur pardonne leur iniquité, et leur péché.
\VS{4}Jérémie donc appela Baruc, fils de Nérija, et Baruc écrivit, sous la dictée de Jérémie, dans le rouleau de livre, toutes les paroles que Yahweh lui avait dites.
\VS{5}Puis Jérémie donna cet ordre à Baruc, en disant : Je suis retenu, et je ne peux pas entrer dans la maison de Yahweh.
\VS{6}Tu y entreras donc et tu liras dans le rouleau que tu as écrit sous ma dictée, toutes les paroles de Yahweh, aux oreilles du peuple dans la maison de Yahweh le jour du jeûne ; tu les liras, dis-je, aussi aux oreilles de tous ceux de Juda qui seront venus de leurs villes.
\VS{7}Peut-être que Yahweh écoutera leur supplication et qu'ils reviendront chacun de leur mauvaise voie ; car grande est la colère, la fureur que Yahweh a déclarée contre ce peuple.
\VS{8}Baruc donc, fils de Nérija, fit selon tout ce que lui avait ordonné Jérémie le prophète, lisant dans le rouleau les paroles de Yahweh, dans la maison de Yahweh.
\VS{9}Or il arriva dans la cinquième année de Jojakim, fils de Josias, roi de Juda, le neuvième mois, qu'on publia le jeûne devant Yahweh à tout le peuple de Jérusalem et à tout le peuple venu des villes de Juda à Jérusalem.
\VS{10}Et Baruc lut dans le livre les paroles de Jérémie, aux oreilles de tout le peuple, dans la maison de Yahweh, dans la chambre de Guemaria, fils de Schaphan, le secrétaire, dans le parvis supérieur, à l'entrée de la porte neuve de la maison de Yahweh.
\VS{11}Et quand Michée fils de Guemaria, fils de Schaphan, eut entendu toutes les paroles de Yahweh contenues dans le livre ;
\VS{12}il descendit dans la maison du roi, vers la chambre du secrétaire, et voici tous les chefs y étaient assis, à savoir, Elischama le secrétaire, Delaja, fils de Schemaeja, Elnathan, fils de Acbor, et Guemaria, fils de Schaphan, et Sédécias, fils de Hanania, et tous les chefs.
\VS{13}Et Michée leur rapporta toutes les paroles qu'il avait entendues, quand Baruc lisait dans le livre, aux oreilles du peuple.
\VS{14}C'est pourquoi tous les chefs envoyèrent vers Baruc, Jehudi, fils de Nethania, fils de Schélémia, fils de Cuschi, pour lui dire : Prends en ta main le rouleau que tu as lu aux oreilles du peuple, et viens ici ! Baruc donc, fils de Nérija, prit le rouleau en sa main, et vint vers eux.
\VS{15}Et ils lui dirent : Assieds-toi maintenant, et lis-le à nos oreilles ; et Baruc le lut à leurs oreilles.
\VS{16}Et il arriva que sitôt qu'ils eurent entendu toutes les paroles, ils furent effrayés entre eux, et dirent à Baruc : Nous ne manquerons pas de rapporter au roi toutes ces paroles.
\VS{17}Et ils interrogèrent Baruc, en disant : Dis-nous comment tu as écrit toutes ces paroles sous sa dictée.
\VS{18}Et Baruc leur dit : Il me dictait de sa bouche toutes ces paroles, et je les écrivais avec de l'encre dans le livre.
\VS{19}Alors les chefs dirent à Baruc : Va, et cache-toi, ainsi que Jérémie, et que personne ne sache où vous serez.
\VS{20}Puis ils s'en allèrent vers le roi dans la cour, mais ils prirent soin de laisser le rouleau dans la chambre d'Elischama le secrétaire ; et ils racontèrent toutes ces paroles aux oreilles du roi.
\VS{21}Et le roi envoya Jehudi pour prendre le rouleau ; et quand Jehudi l'eut prit de la chambre d'Elischama le secrétaire, et il le lut aux oreilles du roi et de tous les chefs qui étaient autour de lui.
\VS{22}Or le roi était assis dans la maison d'hiver, au neuvième mois, et un brasier était allumé devant lui.
\VS{23}Et il arriva qu'aussitôt que Jehudi en eut lu trois ou quatre feuilles, le roi déchira le rouleau avec le canif du secrétaire, et le jeta au feu du brasier, jusqu'à ce que tout le rouleau fut consumé au feu du brasier.
\VS{24}Et ni le roi ni tous ses serviteurs qui entendirent toutes ces paroles, n'en furent pas effrayés, et ne déchirèrent pas leurs vêtements.
\VS{25}Toutefois Elnathan, et Delaja et Guemaria intercédèrent envers le roi, afin qu'il ne brûle pas le rouleau ; mais il ne les écouta pas.
\VS{26}Même le roi ordonna à Jerachmeel, fils de Hammélec, et à Seraja, fils d'Azriel, et à Schélémia, fils de Abdeel, de saisir Baruc, le secrétaire, et Jérémie le prophète ; mais Yahweh les cacha.
\TextTitle{Remplacement du manuscrit brûlé ; jugement sur Jojakim}
\VS{27}Et la parole de Yahweh fut adressée à Jérémie, après que le roi eut brûlé le rouleau contenant les paroles que Baruc avait écrites sous la dictée de Jérémie, en disant :
\VS{28}Prends encore un autre rouleau, et tu y écriras toutes les premières paroles qui étaient dans le premier rouleau que Jojakim, roi de Juda, a brûlé.
\VS{29}Et tu diras à Jojakim, roi de Juda : Ainsi parle Yahweh : Tu as brûlé ce rouleau, et tu as dit : Pourquoi y as-tu écrit ces paroles : Le roi de Babylone viendra certainement, il détruira ce pays, et il exterminera les hommes et les bêtes ?
\VS{30}C'est pourquoi ainsi parle Yahweh sur Jojakim, roi de Juda : Aucun des siens ne sera assis sur le trône de David, et son cadavre sera jeté de jour à la chaleur et de nuit à la gelée.
\VS{31}Je le punirai, lui, sa postérité, et ses serviteurs, à cause de leur iniquité ; et je ferai venir sur eux, et sur les habitants de Jérusalem, et sur les hommes de Juda, tout le mal que je leur ai prononcé, et qu'ils n'ont pas écouter.
\VS{32}Jérémie donc prit un autre rouleau, et le donna à Baruc, fils de Nérija secrétaire, lequel y écrivit, sous la dictée de Jérémie, toutes les paroles du rouleau que Jojakim, roi de Juda, avait brûlé au feu. Beaucoup de paroles semblables y furent encore ajoutées.
\Chap{37}
\TextTitle{Sédécias sollicite l'intercession de Jérémie}
\VerseOne{}Or le roi Sédécias, fils de Josias, régna à la place de Jéconia, fils de Jojakim, et il fut établi roi dans le pays de Juda par Nebucadnetsar, roi de Babylone.
\VS{2}Mais, ni lui, ni ses serviteurs, ni le peuple du pays, n'obéirent pas aux paroles que Yahweh prononça par Jérémie le prophète.
\VS{3}Toutefois le roi Sédécias envoya Jucal, fils de Schélémia, et Sophonie, fils de Maaséja sacrificateur, vers Jérémie le prophète, pour lui dire : Intercède pour nous auprès de Yahweh, notre Dieu.
\VS{4}Car Jérémie allait et venait parmi le peuple, parce qu'on ne l'avait pas encore mis en prison.
\VS{5}Alors l'armée de Pharaon sortit d'Egypte, et quand les Chaldéens qui assiégeaient Jérusalem en entendirent cette nouvelle, ils se retirèrent de devant Jérusalem.
\VS{6}Et la parole de Yahweh fut adressée à Jérémie le prophète, en disant :
\VS{7}Ainsi parle Yahweh, le Dieu d'Israël : Vous direz ainsi au roi de Juda, qui vous a envoyés me consulter : Voici, l'armée de Pharaon, qui était sortie à votre secours, retourne dans son pays, en Egypte ;
\VS{8}et les Chaldéens reviendront, et combattront contre cette ville, et la prendront, et la brûleront au feu.
\VS{9}Ainsi parle Yahweh : Ne vous abusez pas vous-mêmes, en disant : Les Chaldéens s'en iront loin de nous ; car ils ne s'en iront pas.
\VS{10}Même quand vous auriez battu toute l'armée des Chaldéens qui combattent contre vous, et qu'il n'y aurait de reste entre eux que des hommes percés de blessures, ils se relèveront pourtant chacun dans sa tente, et brûleront cette ville au feu.
\TextTitle{Jérémie calomnié et emprisonné}
\VS{11}Or il arriva que quand l'armée des Chaldéens se fut retirée de Jérusalem, à cause de l'armée de Pharaon,
\VS{12}Jérémie sortit de Jérusalem, pour s'en aller dans le pays de Benjamin, se glissant hors de là au milieu du peuple.
\VS{13}Mais quand il fut à la porte de Benjamin, il y avait là un commandant de la garde, nommé Jireija, fils de Schélémia, fils de Hanania, qui saisit Jérémie le prophète, en lui disant : Tu vas te rendre aux Chaldéens !
\VS{14}Et Jérémie répondit : C'est un mensonge ! Je ne vais pas me rendre aux Chaldéens. Mais il ne l'écouta pas, et Jireija prit Jérémie, et l'amena vers les chefs.
\VS{15}Et les chefs se mirent en colère contre Jérémie, et le frappèrent et le mirent en prison dans la maison de Jonathan le secrétaire, car ils en avaient fait une prison.
\VS{16}Et ainsi Jérémie entra dans la fosse de la maison et dans les cachots ; et Jérémie y demeura plusieurs jours.
\VS{17}Mais le roi Sédécias y envoya, et l'en tira, et il l'interrogea en secret dans sa maison, et lui dit : Y a-t-il une parole de la part de Yahweh ? Et Jérémie répondit : Il y en a une ; Et lui dit : Tu seras livré entre les mains du roi de Babylone.
\VS{18}Puis Jérémie dit au roi Sédécias : Quel péché ai-je commis contre toi, contre tes serviteurs, et contre ce peuple, pour que vous m'ayez mis en prison ?
\VS{19}Mais où sont vos prophètes qui vous prophétisaient, en disant : Le roi de Babylone ne reviendra pas contre vous, ni contre ce pays ?
\VS{20}Or écoute maintenant, je te prie, ô roi, mon seigneur ! Et que maintenant ma supplication soit reçue devant ta face, et ne me renvoie pas dans la maison de Jonathan le secrétaire, de peur que je n'y meure !
\VS{21}C'est pourquoi le roi Sédécias ordonna qu'on garde Jérémie dans la cour de la prison, et qu'on lui donne chaque jour un pain de la rue des boulangers, jusqu'à ce que tout le pain de la ville soit épuisé. Ainsi Jérémie demeura dans la cour de la prison.
\Chap{38}
\TextTitle{Jérémie jeté dans la fosse puis délivré par Ebed-Mélec l'éthiopien}
\VerseOne{}Mais Schephathia, fils de Matthan, et Guedalia, fils de Paschhur, et Jucal, fils de Schélémia, et Paschhur, fils de Malkija, entendirent les paroles que Jérémie prononçait à tout le peuple, en disant :
\VS{2}Ainsi parle Yahweh : Celui qui restera dans cette ville mourra par l'épée, par la famine, ou par la peste ; mais celui qui sortira vers les Chaldéens vivra, et sa vie sera son butin, et il vivra.
\VS{3}Ainsi parle Yahweh : Cette ville sera livrée certainement aux mains de l'armée du roi de Babylone, qui la prendra.
\VS{4}Et les chefs dirent au roi : Qu'on fasse mourir cet homme ! Car il décourage les mains des hommes de guerre qui restent dans cette ville, et les mains de tout le peuple, en leur disant de telles paroles ; parce que cet homme ne cherche pas le bien de ce peuple, mais le mal.
\VS{5}Et le roi Sédécias dit : Voici, il est entre vos mains ; car le roi ne peut rien contre vous.
\VS{6}Ils prirent donc Jérémie, et le jetèrent dans la fosse de Malkija, fils de Hammélec, laquelle était dans la cour de la prison, et ils descendirent Jérémie avec des cordes dans cette fosse où Il n'y avait pas d'eau mais de la boue ; et ainsi Jérémie enfonça dans la boue.
\VS{7}Mais Ebed-Mélec l'éthiopien, eunuque, qui était dans la maison du roi, apprit qu'ils avaient mis Jérémie dans cette fosse ; et le roi était assis à la porte de Benjamin.
\VS{8}Et Ebed-Mélec sortit de la maison du roi, et parla au roi, en disant :
\VS{9}Ô roi, mon seigneur ! Ces hommes-là ont mal fait dans tout ce qu'ils ont fait contre Jérémie le prophète, en le jetant dans la fosse, car il serait déjà mort de faim dans le lieu où il était parce qu'il n'y a plus de pain dans la ville.
\VS{10}C'est pourquoi le roi donna cet ordre à Ebed-Mélec l'éthiopien, en disant : Prends ici trente hommes avec toi, et fais remonter hors de la fosse Jérémie le prophète, avant qu'il meure.
\VS{11}Ebed-Mélec donc prit des hommes avec lui, et entra dans la maison du roi, dans un lieu au-dessous du trésor, d'où il prit de vieux lambeaux et de vieux chiffons, et les descendit avec des cordes à Jérémie dans la fosse.
\VS{12}Et Ebed-Mélec l'éthiopien dit à Jérémie : Mets ces vieux lambeaux et ces chiffons sous les aisselles de tes bras, au-dessous des cordes. Et Jérémie fit ainsi.
\VS{13}Ainsi ils tirèrent Jérémie dehors avec les cordes, et le firent remonter hors de la fosse ; et Jérémie demeura dans la cour de la prison.
\TextTitle{Jérémie appelle Sédécias à la repentance}
\VS{14}Et le roi Sédécias envoya chercher Jérémie le prophète, et le fit amener vers lui à la troisième entrée qui était dans la maison de Yahweh. Et le roi dit à Jérémie : Je vais te demander une chose, ne me cache rien.
\VS{15}Et Jérémie répondit à Sédécias : Quand je te l'aurais déclarée, n'est-il pas vrai que tu me feras mourir ? Et quand je t'aurai donné conseil, tu ne m'écouteras pas.
\VS{16}Alors le roi Sédécias jura secrètement à Jérémie, en disant : Yahweh est vivant, qui nous a fait cette âme, je ne te ferai pas mourir, et que je ne te livrerai pas entre les mains de ces hommes qui cherchent ta vie.
\VS{17}Alors Jérémie dit à Sédécias : Ainsi parle Yahweh, le Dieu des armées, le Dieu d'Israël : Si tu sors volontairement pour aller vers les chefs du roi de Babylone, tu auras la vie, et cette ville ne sera pas brûlée par le feu ; et tu vivras toi et ta maison.
\VS{18}Mais si tu ne sors pas vers les chefs du roi de Babylone, cette ville sera livrée entre les mains des Chaldéens, qui la brûleront par le feu ; et tu n'échapperas pas à leurs mains.
\VS{19}Et le roi Sédécias dit à Jérémie : Je crains à cause des Juifs qui se sont rendus aux Chaldéens, je crains qu'on ne me livre entre leurs mains et qu’ils ne se moquent de moi.
\VS{20}Et Jérémie lui répondit : On ne te livrera pas à eux. Je te prie, écoute la voix de Yahweh dans ce que je te dis ; tu t'en trouveras bien, et tu auras la vie.
\VS{21}Que si tu refuses de sortir, voici ce que Yahweh m'a fait voir :
\VS{22}C'est que, voici toutes les femmes qui restent dans la maison du roi de Juda seront menées aux chefs du roi de Babylone, et elles diront : Tu as été séduit, vaincu, par les hommes qui te prédisaient la paix ; et quand tes pieds sont enfoncés dans la boue, ils se sont retirés en arrière.
\VS{23}Toutes tes femmes et tes fils seront menés dehors aux Chaldéens ; et tu n'échapperas pas à leurs mains, mais tu seras pris, pour être livré entre les mains du roi de Babylone, et à cause de toi, cette ville sera brûlée par le feu.
\VS{24}Alors Sédécias dit à Jérémie : Que personne ne sache rien de ces paroles, et tu ne mourras pas.
\VS{25}Et si les chefs entendent que je t'ai parlé, et qu'ils viennent vers toi, et te dise : Déclare-nous maintenant ce que tu as dit au roi, et ce que le roi t'a dit, ne nous en cache rien, et nous ne te ferons pas mourir ;
\VS{26}tu leur diras : J'ai présenté ma supplication devant le roi afin qu'il ne me renvoie pas dans la maison de Jonathan, pour y mourir.
\VS{27}Tous les chefs donc vinrent vers Jérémie, et l'interrogèrent ; mais il leur répondit exactement comme le roi lui avait ordonné ; et ils gardèrent le silence, car l'affaire n'avait pas été divulguée.
\VS{28}Ainsi Jérémie demeura dans la cour de la prison, jusqu'au jour où Jérusalem fut prise, et il y était lorsque Jérusalem fut prise.
\Chap{39}
\TextTitle{Prise de Jérusalem ; Sédécias déporté à Babylone\FTNTT{2 R. 25:1-7; Jé. 52:4-17 ; 2 Ch. 36:17-21.}}
\VerseOne{}La neuvième année de Sédécias, roi de Juda, au dixième mois, Nebucadnetsar, roi de Babylone, vint avec toute son armée contre Jérusalem, et ils l'assiégèrent.
\VS{2}Et la onzième année de Sédécias, le neuvième jour du quatrième mois, une brèche fut faite à la ville.
\VS{3}Et tous les chefs du roi de Babylone y entrèrent, et s'assirent à la porte du milieu, à savoir, Nergal-Scharetser, Samgar-Nebu, Sarsekim, chef des eunuques, Nergal-Scharetser, chef des devins et tous les autres chefs du roi de Babylone.
\VS{4}Or il arriva qu'aussitôt que Sédécias, roi de Juda, et tous les hommes de guerre les eurent vus, ils s'enfuirent et sortirent de nuit hors de la ville, par le chemin du jardin du roi, par la porte entre les deux murailles, et ils s'en allèrent par le chemin de la plaine.
\VS{5}Mais l'armée des Chaldéens les poursuivit et atteignit Sédécias dans les plaines de Jéricho. Ils le prirent, et le firent monter vers Nebucadnetsar, roi de Babylone, à Ribla, dans le pays de Hamath, où il prononça contre lui une sentence.
\VS{6}Et le roi de Babylone fit égorger à Ribla les fils de Sédécias sous ses yeux ; le roi de Babylone fit aussi égorger tous les nobles de Juda.
\VS{7}Puis il fit crever les yeux à Sédécias, et le fit lier de doubles chaînes d'airain, pour le conduire à Babylone.
\VS{8}Les Chaldéens brûlèrent par le feu la maison royale et les maisons du peuple, et démolirent\FTNT{Ici débute le « temps des nations » (587-586 av J.-C.), Jérusalem est foulée aux pieds par les nations. Voir aussi 2 R. 25:8-24 ; 2 Ch. 36:17-21.} les murailles de Jérusalem.
\VS{9}Et Nebuzaradan, chef des gardes, transporta à Babylone le reste du peuple qui était resté dans la ville, et ceux qui s'étaient rendus à lui, le reste, dis-je, du peuple qui avait été épargné.
\VS{10}Mais Nebuzaradan, chefs des gardes, laissa dans le pays de Juda les plus pauvres du peuple qui n'avaient rien ; et en ce jour-là, il leur donna des vignes et des champs.
\TextTitle{Jérémie libéré de prison}
\VS{11}Or Nebucadnetsar, roi de Babylone, avait donné cet ordre au sujet de Jérémie, à Nebuzaradan, chef des gardes, en disant :
\VS{12}Prends cet homme et veille sur lui ; ne lui fais aucun mal, mais fais pour lui tout ce qu'il te dira.
\VS{13}Nebuzaradan donc chefs des gardes, envoya, et aussi Nebuschazban, Rabsaris, chef des eunuques, Nergal-Scharetser, Rabmag, chef des devins, et tous les chefs du roi de Babylone ;
\VS{14}ils envoyèrent, dis-je, chercher Jérémie dans la cour de la prison, et le remirent à Guedalia, fils d'Achikam, fils de Schaphan, pour qu'il le conduise dans sa maison. Ainsi il demeura au milieu du peuple.
\TextTitle{Yahweh épargne Ebed-Mélec}
\VS{15}Or la parole de Yahweh fut adressée à Jérémie pendant qu'il était enfermé dans la cour de la prison, en disant :
\VS{16}Va, et parle à Ebed-Mélec l'éthiopien, et dis-lui : Ainsi parle Yahweh des armées, le Dieu d'Israël : Voici, je vais faire venir mes paroles sur cette ville pour son malheur et non pas pour son bien, et elles s'accompliront en ce jour-là devant toi.
\VS{17}Mais je te délivrerai en ce jour-là, dit Yahweh, et tu ne seras pas livré entre les mains des hommes que tu crains.
\VS{18}Car certainement je te ferai échapper, et tu ne tomberas pas sous l'épée ; mais ta vie sera ton butin, parce que tu as eu confiance en moi, dit Yahweh.
\Chap{40}
\TextTitle{Assassinat de Guedalia et meurtres en série d'Ismaël}
\VerseOne{}La parole qui fut adressée à Jérémie de la part de Yahweh, quand Nebuzaradan, chef des gardes, l'eut renvoyé de Rama, après l'avoir pris lorsqu'il était lié de chaînes parmi tous les captifs de Jérusalem et de Juda qu'on transportait à Babylone.
\VS{2}Quand donc le chef des gardes prit Jérémie, et il lui dit : Yahweh, ton Dieu, a prononcé ce mal contre ce lieu-ci ;
\VS{3}et Yahweh l'a fait venir et a fait comme il avait dit, parce que vous avez péché contre Yahweh, et que vous n'avez pas écouté sa voix, à cause de cela ceci vous est arrivé.
\VS{4}Maintenant donc voici, je t'affranchis aujourd'hui des chaînes que tu as aux mains ; s'il est bon à tes yeux de venir avec moi à Babylone, viens, et j'aurai les yeux sur toi ; mais s'il est mauvais de venir avec moi à Babylone, ne viens pas ; regarde, tout le pays est à ta disposition, va où il te semblera bon et convenable d'aller.
\VS{5}Or Guedalia ne retournera plus ici ; retourne, dit-il, vers Guedalia, fils d'Achikam, fils de Schaphan, que le roi de Babylone a établi sur les villes de Juda, et demeure avec lui parmi le peuple ; ou bien, va partout où il conviendra à tes yeux d'aller. Et le chef des gardes lui donna des vivres et quelques présents, et le renvoya.
\VS{6}Jérémie donc alla vers Guedalia, fils d'Achikam, à Mitspa, et demeura avec lui parmi le peuple qui était resté dans le pays.
\VS{7}Et tous les chefs des armées qui étaient dans les champs, eux et leurs hommes, entendirent que le roi de Babylone avait établi Guedalia, fils d'Achikam, sur le pays, et qu'il lui avait commis les hommes, et les femmes, et les enfants, et ceux-là d'entre les plus pauvres du pays, à savoir, de ceux qui n'avaient pas été transportés à Babylone.
\VS{8}Alors ils allèrent vers Guedalia à Mitspa ; à savoir, Ismaël fils de Nethania, et Jochanan et Jonathan fils de Karéach, et Seraja fils de Thanhumeth, et les fils d'Ephaï de Nethopha, et Jezania fils du Maacatite, eux et leurs hommes.
\VS{9}Et Guedalia, fils d'Achikam, fils de Schaphan, leur jura, à eux et à leurs hommes, en disant : Ne craignez pas de servir les Chaldéens ; demeurez dans le pays, et servez le roi de Babylone, et vous vous en trouverez bien.
\VS{10}Et pour moi, voici, je resterai à Mitspa, pour me tenir prêt à recevoir les ordres des Chaldéens qui viendront vers nous ; mais vous, recueillez le vin, les fruits d'été et l'huile, et mettez-les dans vos vases, et demeurez dans vos villes que vous avez prises pour votre demeure.
\VS{11}Pareillement aussi tous les Juifs qui étaient au pays de Moab, et parmi les Ammonites, et au pays d'Edom, et dans toutes ces contrées, quand il eurent entendu que le roi de Babylone avait laissé quelque reste à Juda, et qu'il avait établi sur eux Guedalia, fils d'Achikam, fils de Schaphan ;
\VS{12}tous ces juifs-là retournèrent de tous les lieux où ils avaient été chassés, et vinrent dans le pays de Juda vers Guedalia à Mitspa, et recueillirent du vin et des fruits d'été en grande abondance.
\VS{13}Mais Jochanan, fils de Karéach, et tous les chefs des armées qui étaient dans les champs, vinrent vers Guedalia à Mitspa,
\VS{14}et lui dirent : Ne sais-tu pas certainement que Baalis, roi des Ammonites, a envoyé Ismaël, le fils de Nethania, pour t'ôter la vie ? Mais Guedalia, fils d'Achikam, ne les crut pas.
\VS{15}Et Jochanan, fils de Karéach parla en secret à Guedalia à Mitspa, en disant : Laisse-moi aller et frapper Ismaël, fils de Nethania, et personne ne le saura. Pourquoi t'ôterait-il la vie, afin que tous les Juifs qui se sont rassemblés vers toi soient dissipés, et que les restes de Juda périssent ?
\VS{16}Mais Guedalia, fils d'Achikam, dit à Jochanan, fils de Karéach : Ne fais pas cela, car tu parles faussement d'Ismaël.
\Chap{41}
\TextTitle{Assassinat de Guedalia}
\VerseOne{}Or il arriva, au septième mois, qu'Ismaël, fils de Nethania, fils d'Elischama, de la race royale, et l'un des grands du roi et dix hommes avec lui, vinrent vers Guedalia, fils d'Achikam, à Mitspa ; et ils mangèrent là du pain ensemble à Mitspa\FTNT{2 R. 25:25.}.
\VS{2}Mais Ismaël, fils de Nethania, se leva, et les dix hommes qui étaient avec lui, et ils frappèrent avec l'épée Guedalia, fils d'Achikam, fils de Schaphan, et on le fit mourir, lui que le roi de Babylone avait établi sur le pays.
\VS{3}Ismaël frappa aussi tous les juifs qui étaient avec Guedalia à Mitspa, et les Chaldéens, gens de guerre, qui se trouvaient là.
\VS{4}Et il arriva que le second jour après après qu'on eut fait mourir Guedalia, avant que personne le sût,
\VS{5}quelques hommes de Sichem, de Silo et de Samarie, au nombre de quatre-vingts hommes, ayant la barbe rasée et les vêtements déchirés, et s'étant fait des incisions, vinrent avec des dons et de l'encens dans leurs mains pour les apporter dans la maison de Yahweh.
\VS{6}Alors Ismaël, fils de Nethania, sortit de Mitspa au-devant d'eux, et il marchait en pleurant, et quand il les rencontra, il leur dit : Venez vers Guedalia, fils d'Achikam.
\VS{7}Mais sitôt qu'ils arrivèrent au milieu de la ville, Ismaël, fils de Nethania, accompagné des hommes qui étaient avec lui, les égorgea et les jeta dans une fosse.
\VS{8}Mais il se trouva parmi eux dix hommes, qui dirent à Ismaël : Ne nous fais pas mourir, car nous avons dans les champs des provisions cachées de froment, d'orge, d'huile et de miel ; et il les laissa, et ne les fit pas mourir avec leurs frères.
\VS{9} Et la fosse dans laquelle Ismaël jeta les cadavres des hommes qu'il tua, à l'occasion de Guedalia, est celle que le roi Asa avait faite, lorsqu'il craignait Baescha, roi d'Israël ; et Ismaël, fils de Nethania, la remplit de gens tués\FTNT{1 R. 15:22.}.
\VS{10}Et Ismaël emmena captif tout le reste du peuple qui était à Mitspa, les filles du roi et tous ceux du peuple qui demeuraient à Mitspa, que Nebuzaradan, chef des gardes, avait commis à Guedalia, fils d'Achikam ; Ismaël, fils de Nethania, les emmena captifs, et s'en alla pour passer vers les Ammonites.
\TextTitle{Jochanan délivre le peuple ; fuite d'Ismaël}
\VS{11}Mais Jochanan, fils de Karéach, et tous les chefs des armées qui étaient avec lui, entendirent tout le mal qu'Ismaël, fils de Nethania, avait fait ;
\VS{12}et ils prirent tous les hommes, et s'en allèrent pour combattre contre Ismaël, fils de Nethania. Ils le trouvèrent près des grandes eaux qui sont à Gabaon.
\VS{13}Et il arriva qu'aussitôt que tout le peuple qui était avec Ismaël vit Jochanan, fils de Karéach, et tous les chefs des armées qui étaient avec lui, ils s'en réjouirent ;
\VS{14}et tout le peuple qu'Ismaël avait emmené captif de Mitspa tourna visage, et revenant sur leur pas, il s'en alla vers Jochanan, fils de Karéach.
\VS{15}Mais Ismaël, fils de Nethania, échappa avec huit hommes devant Jochanan, et s'en alla vers les Ammonites.
\VS{16}Et Jochanan, fils de Karéach, et tous les chefs des armées qui étaient avec lui, prirent tout le reste du peuple qu'ils avaient retiré des mains d'Ismaël, fils de Nethania, qu'il emmenait captif de Mitspa, après avoir tué Guedalia, fils d'Achikam, à savoir, les vaillants hommes de guerre, et les femmes, et les enfants et les eunuques ; et les ramenèrent de Gabaon.
\VS{17}Et ils s'en allèrent et demeurèrent à l'hôtellerie de Kimham, près de Bethléhem, pour se retirer ensuite en Egypte,
\VS{18}à cause des Chaldéens ; car ils avaient peur d'eux, parce qu'Ismaël, fils de Nethania, avait tué Guedalia, fils d'Achikam, qui avait été établi sur le pays par le roi de Babylone.
\Chap{42}
\TextTitle{Yahweh défend au reste du peuple de se réfugier en Egypte }
\VerseOne{}Alors tous les chefs des armées, et Jochanan, fils de Karéach, et Jezania, fils d'Hosée, et tout le peuple, depuis le plus petit jusqu'au plus grand, s'approchèrent,
\VS{2}et dirent à Jérémie le prophète : Que notre supplication soit favorable devant toi ! Intercède auprès de Yahweh, ton Dieu, pour nous, à savoir, pour tout ce reste-ci ; car de beaucoup de monde que nous étions, nous sommes restés peu, comme tes yeux nous voient ;
\VS{3}et que Yahweh, ton Dieu, nous déclare le chemin par lequel nous aurons à marcher, et ce que nous avons à faire !
\VS{4}Et Jérémie le prophète, leur répondit : J'ai entendu votre demande ; voici, je vais prier Yahweh, votre Dieu, selon vos paroles ; et il arrivera que je vous déclarerai tout ce que Yahweh vous répondra, et je ne vous en cacherai pas un mot.
\VS{5}Et ils dirent à Jérémie : Yahweh soit entre nous un témoin véritable et fidèle, si nous ne faisons pas selon toutes les paroles que Yahweh, ton Dieu, t'enverra vers nous !
\VS{6}Soit bien, soit mal, nous obéirons à la voix de Yahweh, notre Dieu, vers qui nous t'envoyons, afin qu'il nous arrive du bien, quand nous aurons obéi à la voix de Yahweh, notre Dieu.
\VS{7}Et il arriva, au bout de dix jours, que la parole de Yahweh fut adressée à Jérémie.
\VS{8}Et il appela Jochanan, fils de Karéach, tous les chefs des armées qui étaient avec lui, et tout le peuple, depuis le plus petit jusqu'au plus grand ;
\VS{9}et leur dit : Ainsi parle Yahweh, le Dieu d'Israël, vers qui vous m'avez envoyé, pour présenter votre supplication devant lui :
\VS{10}Si vous continuez à demeurer dans ce pays, je vous rétablirai et je ne vous détruirai pas ; je vous y planterai et je ne vous arracherai pas, car je me repens du mal que je vous ai fait.
\VS{11}Ne craignez pas le roi de Babylone, dont vous avez peur, ne craignez pas, dit Yahweh, car je suis avec vous pour vous sauver et pour vous délivrer de sa main.
\VS{12}Même je vous ferai obtenir miséricorde, tellement qu'il aura pitié de vous, et vous fera retourner dans votre pays.
\VS{13}Que si vous dites : Nous ne demeurerons pas dans ce pays, et nous n'écouterons pas la voix de Yahweh, notre Dieu,
\VS{14}en disant : Non ; mais nous irons au pays d'Egypte, afin que nous ne voyons pas de guerre, et que nous n'entendions pas le son du shofar, et que nous ne manquions pas de pain, et nous demeurerons là.
\VS{15}A cause de cela écoutez maintenant la parole de Yahweh, vous les restes de Juda ! Ainsi parle Yahweh des armées, le Dieu d'Israël : Si vous tournez le visage pour aller en Egypte, et que vous y entriez pour y demeurer ;
\VS{16}il arrivera que l'épée dont vous avez peur vous attrapera là au pays d'Egypte ; et la famine que vous craignez si fort vous suivra en Egypte, et vous y mourrez\FTNT{Ez. 30:9-11.}.
\VS{17}Et il arrivera que tous les hommes qui tourneront le visage pour aller en Egypte afin d'y demeurer, mourront par l'épée, par la famine et par la peste ; et il n'y aura ni survivant ni réchappé devant le mal que je vais faire venir sur eux.
\VS{18}Car ainsi parle Yahweh des armées, le Dieu d'Israël : Comme ma colère et ma fureur se sont répandues sur les habitants de Jérusalem, ainsi ma fureur sera versée sur vous, quand vous serez entrés en Egypte ; et vous serez un sujet d'exécration, d'épouvante, de malédiction et d'opprobre, et vous ne verrez plus ce lieu-ci.
\VS{19}Vous, les restes de Juda, Yahweh dit contre vous : N'allez pas en Egypte ! Sachez certainement que je vous ai avertis aujourd'hui.
\VS{20}Car vous vous êtes séduits vous-mêmes dans vos âmes, quand vous m'avez envoyé vers Yahweh, votre Dieu, en me disant : Intercède pour nous auprès de Yahweh, notre Dieu, et déclare-nous tout ce que Yahweh, notre Dieu, te dira, et nous le ferons.
\VS{21}Et je vous l'ai déclaré aujourd'hui ; mais vous n'écoutez pas la voix de Yahweh, votre Dieu, ni rien de tout ce pour quoi il m'a envoyé vers vous.
\VS{22}Maintenant donc sachez certainement que vous mourrez par l'épée, par la famine et par la peste, dans le lieu où vous avez désiré d’aller pour y demeurer.
\Chap{43}
\TextTitle{Désobéissance des Hébreux ; jugement sur l'Egypte}
\VerseOne{}Or il arriva qu'aussitôt que Jérémie eut achevé de prononcer à tout le peuple toutes les paroles de Yahweh, leur Dieu, pour lesquelles Yahweh, leur Dieu, l'avait envoyé vers eux, à savoir, toutes ces choses-là ;
\VS{2}Azaria, fils d'Hosée, et Jochanan, fils de Karéach, et tous ces hommes orgueilleux, dirent à Jérémie : Tu dis un mensonge ; Yahweh, notre Dieu, ne t'a pas envoyé nous dire : N'allez pas en Egypte pour y demeurer.
\VS{3}Mais Baruc, fils de Nérija, t'incite contre nous, afin de nous livrer entre les mains des Chaldéens, pour nous faire mourir, et pour nous faire transporter à Babylone.
\VS{4}Ainsi Jochanan, fils de Karéach, et tous les chefs des armées, et tout le peuple, n'obéirent pas à la voix de Yahweh, pour demeurer dans le pays de Juda.
\VS{5}Car Jochanan, fils de Karéach, et tous les chefs des armées, prirent tous les restes de Juda qui étaient revenus de toutes les nations, parmi lesquelles ils avaient été chassés, pour demeurer dans le pays de Juda ;
\VS{6}les hommes, et les femmes, et les enfants, et les filles du roi, et toutes les personnes que Nebuzaradan, chef des gardes, avait laissées avec Guedalia, fils d'Achikam, fils de Schaphan ; ils prirent aussi Jérémie le prophète et Baruc, fils de Nérija.
\VS{7}Et ils entrèrent dans le pays d'Egypte, car ils n'obéirent pas à la voix de Yahweh, et ils vinrent jusqu'à Tachpanès.
\VS{8}Alors la parole de Yahweh fut adressée à Jérémie, à Tachpanès, en disant :
\VS{9}Prends dans ta main de grandes pierres, et cache-les dans l'argile, dans le four à briques qui est à l'entrée de la maison de Pharaon à Tachpanès, sous les yeux des Juifs ;
\VS{10}et dis-leur : Ainsi parle Yahweh des armées, le Dieu d'Israël : Voici, j'enverrai chercher Nebucadnetsar, roi de Babylone, mon serviteur, et je mettrai son trône sur ces pierres que j'ai cachées, et il étendra son dais sur elles ;
\VS{11}et Il viendra et frappera le pays d'Egypte. Ceux qui sont destinés à la mort, iront à la mort ; et ceux qui sont destinés à la captivité, iront en captivité ; et ceux qui sont destinés à l’épée, seront livrés à l’épée\FTNT{Ez. 29:9.} !
\VS{12}Et j'allumerai le feu dans les maisons des dieux d'Egypte, Nebucadnetsar les brûlera, et il emmènera captifs ceux d'Egypte, et il se parera des richesses du pays d'Egypte, comme le pasteur s'enveloppe de son vêtement, et il en sortira en paix\FTNT{Es. 19:1 ; Ez. 30:13.}.
\VS{13}Il brisera aussi les statues de Beth-Schémesch, qui est au pays d'Egypte, et il brûlera par le feu les maisons des dieux d'Egypte.
\Chap{44}
\TextTitle{Yahweh avertit les Juifs d'Egypte\FTNTT{Jé. 43:8-13}}
\VerseOne{}La parole qui fut adressée à Jérémie sur tous les Juifs qui demeuraient au pays d'Egypte, qui habitaient à Migdol, à Tachpanès, à Noph, et au pays de Pathros, en disant :
\VS{2}Ainsi parle Yahweh des armées, le Dieu d'Israël : Vous avez vu tous les malheurs que j'ai fait venir sur Jérusalem et sur toutes les villes de Juda : Voici, elles ne sont plus aujourd'hui que des ruines, et personne n'y habite,
\VS{3}à cause des méchancetés qu'ils ont faites pour m'irriter, en allant brûler de l'encens pour servir d'autres dieux, qu'ils n'ont pas connu, ni eux, ni vous, ni vos pères.
\VS{4}Et je vous ai envoyé tous mes serviteurs, les prophètes, me levant dès le matin, et les envoyant, pour vous dire : Ne commettez pas maintenant cette chose abominable que je hais.
\VS{5}Mais ils n'ont pas écouté, ils n'ont pas prêté l'oreille pour se détourner de leur méchanceté, afin de ne pas faire brûler de l'encens à d'autres dieux.
\VS{6}C'est pourquoi ma fureur et ma colère se sont répandues sur eux, et ont embrasé les villes de Juda et les rues de Jérusalem, qui ne sont réduites en désert et en désolation, comme il paraît aujourd'hui.
\VS{7}Maintenant donc, ainsi parle Yahweh, le Dieu des armées, le Dieu d'Israël : Pourquoi faites-vous ce grand mal contre vos âmes, pour vous faire exterminer du milieu de Juda, hommes et femmes, petits enfants et ceux qui tètent, afin qu'on ne vous laisse aucun reste ?
\VS{8}En m'irritant par les œuvres de vos mains, en brûlant de l'encens à d'autres dieux au pays d'Egypte, où vous êtes venus pour y demeurer, afin de vous faire exterminer et d'être un objet de malédiction et d'opprobre parmi toutes les nations de la terre ?
\VS{9}Avez-vous oublié les crimes de vos pères, les crimes des rois de Juda, les crimes de leurs femmes, vos propres crimes et les crimes de vos femmes, commis dans le pays de Juda et dans les rues de Jérusalem ?
\VS{10}Jusqu'à ce jour, ils ne se sont pas humiliés, ils n'ont pas eu de crainte, ils n'ont pas marché dans ma loi ni dans mes ordonnances, que j'ai mises devant vous et devant vos pères.
\VS{11}C'est pourquoi ainsi parle Yahweh des armées, le Dieu d'Israël : Voici, je tourne ma face contre vous pour vous nuire et vous retrancher tout Juda\FTNT{Am. 9:4.}.
\VS{12}Et je prendrai les restes de ceux de Juda qui ont tourné le visage pour aller au pays d'Egypte afin d'y demeurer ; ils seront tous consumés, ils tomberont dans le pays d'Egypte ; ils seront consumés par l'épée, par la famine, depuis le plus petit jusqu'au plus grand ; ils mourront par l'épée et par la famine ; et ils seront en exécration, en étonnement, en malédiction et en opprobre.
\VS{13}Et je punirai ceux qui demeurent au pays d'Egypte, comme j'ai puni Jérusalem, par l'épée, par la famine, et par la peste.
\VS{14}Il n'y aura personne du reste de Juda qui est venu dans le pays d’Égypte pour y séjourner, échappera ou restera, pour retourner dans le pays de Juda, où ils aspirent à retourner pour y demeurer ; car pas un ne retournera, sinon les rescapés.
\VS{15}Mais les tous hommes qui savaient que leurs femmes brûlaient de l'encens à d'autres dieux, toutes les femmes qui se tenaient là en grande compagnie, et tout le peuple qui demeurait au pays d'Egypte, à Pathros, répondirent à Jérémie, en disant :
\VS{16}Quant à la parole que tu nous as dite au nom de Yahweh, nous ne t'écouterons pas.
\VS{17}Mais nous ferons assurément selon toute parole qui est sortie de notre bouche, brûler de l'encens à la reine des cieux\FTNT{Voir commentaire en Jé. 7:18.}, et lui faire des libations, comme nous l'avons fait, nous et nos pères, nos rois et nos chefs, dans les villes de Juda et dans les rues de Jérusalem. Alors nous étions rassasiés de pain, nous étions heureux, et nous ne voyions pas le malheur\FTNT{Ez. 16:24 ; Ez. 20:32.}.
\VS{18}Mais depuis le temps que nous avons cessé de brûler de l'encens à la reine des cieux et de lui faire des libations, nous avons manqué de tout, et nous avons été consumés par l'épée et par la famine…
\VS{19}Quand nous brûlions de l'encens à la reine des cieux et que nous lui faisions des libations, est-ce à l'insu de nos maris que nous lui faisons des gâteaux sur lesquels elle est représentée et que nous lui faisons des libations ?
\VS{20}Alors Jérémie parla à tout le peuple, aux hommes, aux femmes, et à tous ceux qui lui avaient donné cette réponse, et leur dit :
\VS{21}Yahweh ne s'est-il pas souvenu, ne lui est-il pas monté à cœur l'encens que vous avez brûlé dans les villes de Juda et dans les rues de Jérusalem, vous et vos pères, vos rois et vos chefs, et le peuple du pays ?
\VS{22}Yahweh n'a pas pu le supporter davantage, à cause de la méchanceté de vos actions, à cause des abominations que vous avez faites ; et votre pays est devenu une ruine, un désert et un objet de malédiction, sans que personne y habite, comme on le voit aujourd'hui.
\VS{23}C'est parce que vous avez brûlé de l'encens et que vous avez péché contre Yahweh, parce que vous n'avez pas écouté la voix de Yahweh, et que vous n'avez pas marché dans sa loi, ni dans ses ordonnances, ni dans ses témoignages, c'est pour cela que ces malheurs vous sont arrivés, comme on le voit aujourd'hui.
\VS{24}Jérémie dit à tout le peuple et à toutes les femmes : Vous tous de Juda, qui êtes au pays d'Egypte, écoutez la parole de Yahweh !
\VS{25}Ainsi parle Yahweh des armées, le Dieu d'Israël, en disant : Vous et vos femmes, vous avez parlé de vos bouches et accompli de vos mains, en disant : Certainement, nous accomplirons nos vœux que nous avons faits, brûler de l'encens à la reine des cieux, et lui faire des libations. Vous avez entièrement accompli vos vœux, vous les avez effectués très exactement.
\VS{26}C'est pourquoi, écoutez la parole de Yahweh, vous tous de Juda, qui demeurez au pays d'Egypte ! Voici, je le jure par mon grand Nom, dit Yahweh, mon Nom ne sera plus invoqué par la bouche d'aucun homme de Juda, et dans tout le pays d'Egypte aucun ne dira : Le Seigneur Yahweh est vivant !
\VS{27}Voici, je veille sur eux pour leur mal et non pour leur bien ; et tous les hommes de Juda qui sont dans le pays d'Egypte seront consumés par l'épée et par la famine, jusqu'à ce qu'ils soient exterminés\FTNT{Da. 9:14.}.
\VS{28}Et ceux qui seront échappés à l'épée, retourneront du pays d'Egypte au pays de Juda en fort petit nombre. Mais tout le reste de Juda, tous ceux qui sont venus dans le pays d'Egypte pour y demeurer, sauront quelle est la parole qui s'accomplira, la mienne ou la leur.
\VS{29}Et ceci sera pour vous le signe, dit Yahweh, que je vous punirai dans ce lieu, afin que vous sachiez que mes paroles s'accompliront infailliblement pour votre malheur.
\VS{30}Ainsi parle Yahweh : Voici, je livrerai Pharaon Hophra, roi d'Egypte, entre les mains de ses ennemis, entre les mains de ceux qui cherchent sa vie, comme j'ai livré Sédécias, roi de Juda, entre les mains de Nebucadnetsar, roi de Babylone, son ennemi, et qui cherchait sa vie.
\Chap{45}
\TextTitle{Yahweh explique son dessein à Baruc}
\VerseOne{}La parole que Jérémie, le prophète, adressa à Baruc, fils de Nérija, quand il écrivit dans un livre ces paroles, sous la dictée de Jérémie, la quatrième année de Jojakim, fils de Josias, roi de Juda. Il dit :
\VS{2}Ainsi parle Yahweh, le Dieu d'Israël, sur toi, Baruc :
\VS{3}Tu dis : Malheur à moi ! Car Yahweh ajoute la tristesse à ma douleur ; je me suis lassé dans mon gémissement, et je ne trouve pas de repos.
\VS{4}Tu lui diras : Ainsi parle Yahweh : Voici, je vais détruire ce que j'ai bâti, et arracher ce que j'ai planté, à savoir tout ce pays.
\VS{5}Et toi, chercherais-tu de grandes choses ? Ne les cherche pas ! Car voici, je vais faire venir du mal sur toute chair, dit Yahweh ; je te donnerai ta vie pour butin, dans tous les lieux où tu iras.
\Chap{46}
\TextTitle{Prophétie contre l'Egypte}
\VerseOne{}La parole de Yahweh qui fut adressée à Jérémie, le prophète, sur les nations.
\VS{2}A l'égard de l'Egypte, contre l'armée de Pharaon Neco, roi d'Egypte, qui était près du fleuve de l'Euphrate, à Carkemisch, et qui fut battue par Nebucadnetsar, roi de Babylone, la quatrième année de Jojakim, fils de Josias, roi de Juda\FTNT{2 R. 24:7.}.
\VS{3}Préparez le bouclier et l'écu et approchez-vous pour la bataille !
\VS{4}Attelez les chevaux, montez, cavaliers ! Présentez-vous avec vos casques, polissez vos lances, revêtez l'armure !
\VS{5}D'où vient que je vois ceci ? Ils sont effrayés, ils reviennent en arrière ; leurs hommes vaillants sont battus ; ils s'enfuient avec précipitation sans regarder derrière eux… La frayeur les environne, dit Yahweh.
\VS{6}Que l'homme léger à la course ne s'enfuie pas, et que le fort ne se sauve pas\FTNT{Am. 2:14-16.} ! Ils sont renversés et tombés vers le nord, auprès du rivage du fleuve de l'Euphrate.
\VS{7}Qui est celui-ci qui s'élève comme le Nil et dont les eaux sont agitées comme les fleuves ?
\VS{8}C'est l'Egypte. Elle s'élève comme le Nil et ses eaux agitées comme les fleuves ; et elle dit : Je m'élèverai et je couvrirai la terre ; je détruirai la ville et ceux qui y habitent.
\VS{9}Montez, chevaux ! Agissez en insensés, chars ! Que les hommes vaillants sortent, ceux d'Ethiopie et de Puth qui manient le bouclier, et ceux de Lud qui manient et tendent l'arc\FTNT{Ez. 30:5-9 ; Na. 3:9-10.} !
\VS{10}Car c'est le jour du Seigneur,  Yahweh des armées ; c'est un jour de vengeance, où il se venge de ses ennemis. L'épée dévore, elle se rassasie, elle s'enivre de leur sang. Car il y a des sacrifices pour le Seigneur, Yahweh des armées, dans le pays du nord, sur le fleuve de l'Euphrate\FTNT{Es. 34:5-6 ; Ez. 39:17 ; So. 1:7.}.
\VS{11}Monte en Galaad, prends du baume, vierge, fille de l'Egypte ! En vain tu multiplies les remèdes, il n'y a pas de guérison pour toi\FTNT{Ez. 30:21-25 ; Na. 3:19.}.
\VS{12}Les nations apprennent ta honte, et tes cris remplissent la terre, car les hommes forts chancellent l'un sur l'autre, ils tombent tous deux ensemble.
\VS{13}La parole que Yahweh prononça à Jérémie, le prophète, sur la venue de Nebucadnetsar, roi de Babylone, pour frapper le pays d'Egypte :
\VS{14}Déclarez-le en Egypte, et publiez-le à Migdol, à Noph, et à Tachpanès et Dites : Présente-toi, tiens-toi prêt car l'épée dévore ce qui est autour de toi !
\VS{15}Pourquoi tes vaillants hommes sont-ils emportés ? Ils ne tiennent pas ferme, parce que Yahweh les pousse.
\VS{16}Il en a terrassé un grand nombre, et même chacun tombe sur son compagnon, et ils disent : Levons-nous, retournons vers notre peuple, au pays de notre naissance, loin de l'épée de l'oppresseur !
\VS{17}Là, ils s'écrient : Pharaon, roi d'Egypte, n'est qu'un bruit ; il a laissé passer le temps fixé.
\VS{18}Je suis vivant ! dit le Roi, dont le Nom est Yahweh des armées ; comme le Thabor entre les montagnes, comme le Carmel qui s'avance dans la mer, ainsi viendra-t-il.
\VS{19}O fille, habitante de l'Egypte, fais tes bagages pour la captivité ! Car Noph sera un désert, elle sera brûlée, elle n'aura plus d'habitants.
\VS{20}L'Egypte est une très belle génisse… la destruction vient, elle vient du nord.
\VS{21}Memes les mercenaires aussi sont au milieu d'elle comme des veaux engraissés. Et eux aussi tournent le dos, ils fuient tous sans résister. Car le jour de leur malheur, le temps de leur châtiment est venu sur eux.
\VS{22}Elle sifflera comme un serpent ; car ils marcheront avec une puissante armée, ils viendront contre elle avec des haches, comme des bûcherons.
\VS{23}Ils couperont sa forêt, dit Yahweh, quoiqu'elle soit impénétrable ; parce que leur armée est en plus grand nombre que les sauterelles, on ne saurait la compter.
\VS{24}La fille de l'Egypte est confuse, elle est livrée entre les mains du peuple du nord.
\VS{25}Yahweh des armées, le Dieu d'Israël, dit : Voici, je vais punir Amon de No, Pharaon, l'Egypte, ses dieux et ses rois, Pharaon et ceux qui se confient en lui.
\VS{26}Et je les livrerai entre les mains de ceux qui cherchent leur vie, entre les mains, dis-je, de Nebucadnetsar, roi de Babylone, et entre les mains de ses serviteurs ; mais après cela, l'Egypte sera habitée comme aux temps passés, dit Yahweh.
\VS{27}Et toi, Jacob, mon serviteur, ne crains pas ; ne t'épouvante pas Israël ! Car voici, je te sauverai de la terre lointaine, je sauverai ta postérité du pays de leur captivité ; Jacob reviendra, il sera en repos et en paix, et il n'y aura personne qui lui fasse peur.
\VS{28}Toi donc,Jacob, mon serviteur, ne crains pas ! dit Yahweh ; car je suis avec toi. Et même je consumerai entièrement  toutes les nations parmi lesquelles je t'ai chassé, mais je ne te consumerai pas entièrement ; et je te châtierai avec justice, je ne te tiendrai pas tout à fait pour innocent.
\Chap{47}
\TextTitle{Prophétie contre la Philistie et la Phénicie}
\VerseOne{}La parole de Yahweh, qui fut adressée à Jérémie, le prophète, contre les Philistins, avant que Pharaon frappe Gaza.
\VS{2}Ainsi parle Yahweh : Voici des eaux montent du nord, elles sont comme un torrent qui déborde ; elles inondent le pays et ce qu'il contient, les villes et leurs habitants. Les hommes poussent des cris, et tous les habitants du pays se lamentent,
\VS{3}à cause du bruit des battements de sabots de ses puissants chevaux, du bruit de ses chars et au son de ses roues ; les pères ne se tournent pas vers leurs fils, tant les mains sont affaiblies,
\VS{4}parce que le jour vient où seront détruits tous les Philistins, exterminés tout le reste de ceux qui servaient de secours à Tyr et à Sidon ; car Yahweh va détruire les Philistins, les restes de l'île de Caphtor.
\VS{5}Gaza est devenue chauve, Askalon est perdue, le reste de leur plaine aussi. Jusqu'à quand te feras-tu des incisions ?
\VS{6}Ah ! Epée de Yahweh, quand te reposeras-tu ? Rentre dans ton fourreau, repose-toi, et sois tranquille !
\VS{7}Mais comment te reposerais-tu ? Car Yahweh lui donne ses ordres, il l'a assignée contre Askalon et contre le rivage de la mer.
\Chap{48}
\TextTitle{Prophétie sur Moab}
\VerseOne{}Sur Moab. Ainsi parle Yahweh des armées, le Dieu d'Israël : Malheur à Nebo, car elle est dévastée ! Kirjathaïm est honteuse, elle est prise ; Misgab est honteuse et brisée.
\VS{2} Moab ne se glorifiera plus à Hesbon, car on a machiné du mal contre elle en disant : Allons, exterminons-la, qu'elle ne soit plus une nation ! Toi aussi, Madmen, tu seras détruite ; l'épée te poursuivra.
\VS{3}Il y a un bruit de clameur qui vient de Choronaïm ; c'est un ravage, une grande ruine.
\VS{4}Moab est brisé ! On entend les cris des plus jeunes.
\VS{5}Pleurs sur pleurs s’élèveront à la montée de Luchith, car on entendra à la descente de Choronaïm\FTNT{Es. 15:5.}. ceux qui crieront à cause des plaies que les ennemis leur auront faites.
\VS{6}Fuyez, dira-t-on, sauvez vos vies, et soyez comme un misérable dans le désert !
\VS{7}Car, parce que tu as eu confiance dans tes ouvrages, et dans tes trésors, tu seras pris, et Kemosch sortira pour être transporté, avec ses sacrificateurs et ses chefs\FTNT{Es. 46:1-7.}.
\VS{8}Et le dévastateur entrera dans toutes les villes, et aucune ville n'échappera ; la vallée périra et la plaine sera détruite, comme Yahweh l'a dit.
\VS{9}Donnez des ailes à Moab, et qu'il parte en volant ! Ses villes seront réduites en désert, elles n'auront plus d'habitants.
\VS{10}Maudit soit celui qui fait l'œuvre de Yahweh avec paresse, maudit soit celui qui garde son épée pour répandre le sang !
\VS{11}Moab était tranquille depuis sa jeunesse, il reposait sur sa lie, il n'était pas vidé de vase en vase, et il n'allait pas en captivité. C'est pourquoi son goût lui est resté, et son odeur ne s'est pas changée.
\VS{12}Mais voici, les jours viennent, dit Yahweh, où je lui enverrai des gens qui le transvaseront, qui videront ses vases, et qui briseront ses outres.
\VS{13}Moab aura honte à cause de Kemosch, comme la maison d'Israël a eu honte à cause de Béthel, qui était sa confiance.
\VS{14}Comment dites-vous : Nous sommes de vaillants hommes, des soldats prêts à combattre ?
\VS{15}Moab est dévasté, et chacune de ses villes monte en fumée, l'élite de sa jeunesse est descendue pour être égorgée, dit le Roi, dont le nom est Yahweh des armées.
\VS{16}La calamité de Moab est proche, son malheur avance à grands pas.
\VS{17}Vous tous qui êtes autour de lui, soyez-en émus à compassion, et vous tous qui connaissez son nom, dites : Comment a été rompue cette forte verge, et ce sceptre d'honneur ? 
\VS{18}Toi qui te tiens chez la fille de Dibon, descends de ta gloire, et assieds-toi dans un lieu desséché ! Car le dévastateur de Moab monte contre toi, il détruit tes forteresses.
\VS{19}Habitante d'Aroër, tiens-toi sur le chemin, et regarde ! Interroge celui qui s'enfuit, celui qui s'échappe, et dis : Qu'est-il arrivé ?
\VS{20}Moab est rendu honteux, car il est brisé. Poussez des gémissements et des cris\FTNT{Es. 15:5 ; Es. 16:7.} ! Rapportez dans Arnon que Moab est dévasté !
\VS{21}Et que jugement est venu sur le pays de la plaine, sur Holon, sur Jahats, sur Méphaath,
\VS{22}Et sur Dibon, sur Nebo, sur Beth-Diblathaïm,
\VS{23}Et sur Kirjathaïm, sur Beth-Gamul, sur Beth-Meon,
\VS{24}Et sur Kerijoth, sur Botsra, sur toutes les villes du pays de Moab, éloignées et proches.
\VS{25}La force de Moab est abattue, et son bras est brisé, dit Yahweh.
\VS{26}Enivrez-le, car il s'est élevé contre Yahweh ! Moab se vautrera dans le vin qu'il aura rendu et deviendra aussi un sujet de moquerie !
\VS{27}Car ô Moab! Israël n'a-t-il pas été pour toi un objet de moquerie ? Avait-il été trouvé parmi les voleurs, pour que tu ne dises des paroles qu'en secouant la tête ?
\VS{28}Habitants de Moab, quittez les villes, et demeurez dans les rochers ! Soyez comme les colombes qui font leur nid aux côtés de l'entrée des cavernes !
\VS{29}Nous avons appris l'extrême orgueil de Moab, son arrogance, sa fierté, et son cœur hautain\FTNT{Es. 16:6 ; So. 2:9-10.}.
\VS{30}J'ai connu son orgueil, dit Yahweh ; mais il n'en sera pas ainsi ; j'ai connu ceux sur lesquels il s'appuie ; ils n'ont rien fait de droit. 
\VS{31}Je hurlerai donc à cause de Moab, même je crierai à cause de Moab tout entier ; on gémira sur les gens de Kir-Hérès.
\VS{32}O vignoble de Sibma, je pleurerai sur toi du pleur de Jaezer ; tes rameaux allaient au-delà de la mer, ils atteignaient la mer de Jaezer ; le dévastateur s'est jeté sur tes fruits d'été et sur ta vendange.
\VS{33}L'allégresse ausi et la et la joie se sont retirées loin des campagnes et du pays de Moab ; j'ai fait cesser le vin des cuves ; on ne foule plus gaîment au pressoir ; il y a des cris de guerre, et non des cris de joie\FTNT{Es. 16:10.}.
\VS{34}A cause de cris de Hesbon qui est parvenu jusqu'à Elealé, et ils font entendre leurs cris jusqu'à Jahats, même depuis Tsoar jusqu'à Choronaïm, jusqu'à Eglath-Schelischija ; car les eaux de Nimrim seront aussi réduites en désolation.
\VS{35}Je ferai cesser en Moab, dit Yahweh, celui qui offre sur les hauts lieux et celui qui brûle de l’encens à ses dieux. 
\VS{36}C’est pourquoi mon cœur mènera un bruit sur Moab, comme des flûtes ; mon cœur mènera un bruit comme des flûtes sur les hommes de Kir-Hérès, parce que tous les biens qu'ils ont acquis ont péri.
\VS{37}Car toutes les têtes sont chauves, toutes les barbes sont coupées ; et il y a des incisions sur toutes les mains, et sur les reins des sacs.
\VS{38}Il y aura des lamentations sur tous les toits de Moab et dans ses places, parce que j'aurai brisé Moab comme un vase auquel on ne prend nul plaisir, dit Yahweh.
\VS{39}Hurlez, en disant : Comment a-t-il été mis en pièces ? brisé ! Comment Moab a-t-il tourné honteusement le dos ! Car Moab sera un objet de moquerie et de frayeur pour tous ceux qui sont autour de lui.
\VS{40}Car ainsi parle Yahweh : Voici, il volera comme un aigle, et il étendra ses ailes sur Moab.
\VS{41}Kerijoth est prise, les forteresses sont saisies, et le cœur des hommes forts de Moab est en ce jour comme le cœur d'une femme qui est en travail.
\VS{42}Et Moab sera exterminé, il ne sera plus un peuple, parce qu'il s'est élevé contre Yahweh.
\VS{43}Habitant de Moab, la frayeur, la fosse, et le filet sont sur toi ! dit Yahweh.
\VS{44}Celui qui s'enfuira à cause de la frayeur tombera dans la fosse, et celui qui remontera de la fosse sera au filet ; car je fera venir sur lui, sur Moab, l'année de son châtiment, dit Yahweh\FTNT{Es. 24:18.}.
\VS{45}Ils se sont arrêtés à l'ombre de Hesbon, voulant éviter la force ; mais le feu sort de Hesbon, une flamme du milieu de Sihon ; elle dévore les flancs de Moab, et le sommet de la tête des fils du tumulte\FTNT{No. 21:28.}.
\VS{46}Malheur à toi, Moab ! Le peuple de Kemosch est perdu ! Car tes fils sont enlevés et emmenés captifs, et tes filles ont été emmenées captives.
\VS{47}Toutefoi je ramènerai et mettrai en repos les captifs de Moab, aux derniers jours\FTNT{Gn. 49:1-2.}, dit Yahweh. Là est le jugement de Moab.
\Chap{49}
\TextTitle{Prophétie sur Ammon}
\VerseOne{}Sur les fils d'Ammon. Ainsi parle Yahweh : Israël n'a-t-il pas de fils ? N'a-t-il pas d'héritier ? Pourquoi donc Malcom hérite-t-il de Gad, et pourquoi son peuple demeure-t-il dans ses villes ?
\VS{2}C'est pourquoi, les jours viennent, dit Yahweh, où je ferai entendre le cri de guerre contre Rabbath des fils d'Ammon ; elle sera réduite en un monceau de ruines, et les villes de son ressort seront brûlées par le feu ; Israël possédera ceux qui l'auront possédé, dit Yahweh.
\VS{3}Hurle, ô Hesbon, car Aï est dévastée ! Poussez des cris, filles de Rabba, ceignez-vous de sacs, lamentez-vous, courez ça et là le long des murailles ! Car Malcom s'en va en captivité avec ses sacrificateurs et ses chefs.
\VS{4}Pourquoi te glorifies-tu de tes vallées ? Ta vallée se fond, fille rebelle, qui te confiais dans tes trésors : Qui viendra contre moi ?
\VS{5}Voici, je fais venir sur toi la terreur, dit le Seigneur, Yahweh des armées, de tous les alentours ; vous serez chassés chacun çà et là, et il n'y aura personne qui rassemblera les fuyards.
\VS{6}Mais après cela, je ramènerai les captifs des fils d'Ammon, dit Yahweh.
\TextTitle{Prophétie sur Edom}
\VS{7}Sur Edom. Ainsi parle Yahweh des armées : N'y a-t-il plus de sagesse dans Théman ? Le conseil a-t-il manqué aux hommes intelligents ? Leur sagesse s'est-elle évanouie\FTNT{Ab. 1:8.} ?
\VS{8}Fuyez, tournez le dos, demeurez dans les cavernes, habitants de Dedan ! Car je fais venir la détresse sur Esaü, le temps de son châtiment.
\VS{9}Si des vendangeurs entrent chez toi, ne laissent-ils rien à grappiller ? Si des voleurs viennent de nuit, ils ne pillent que ce qu'ils peuvent.
\VS{10}Mais je dépouillerai Esaü, je découvrirai ses lieux secrets, il ne pourra se cacher ; sa postérité, ses frères, et ses voisins, périront, et il ne sera plus.
\VS{11}Laisse tes orphelins, je les ferai vivre, et que tes veuves se confient en moi !
\VS{12}Car ainsi parle Yahweh : Voici, ceux dont le jugement n'était pas de boire la coupe, la boiront certainement ; et toi, tu resterais impuni ! Tu ne resteras pas impuni, tu la boiras.
\VS{13}Car je le jure par moi-même, dit Yahweh, que Botsra sera un objet de désolation, d'opprobre, de dévastation, et de malédiction, et que toutes ses villes deviendront des ruines éternelles.
\VS{14}J'ai entendu de Yahweh une nouvelle, et un messager a été envoyé parmi les nations : Assemblez-vous, et venez contre elle ! Levez-vous pour la guerre !
\VS{15}Car voici, je te rendrai petit entre les nations, méprisé entre les hommes.
\VS{16}Mais ta présomption, l'orgueil de ton cœur t'a séduit, toi qui habites dans le creux des rochers, et qui occupes le sommet des collines. Quand tu aurais élevé ton nid comme l'aigle, je t'en ferai descendre, dit Yahweh.
\VS{17}Edom sera un objet de désolation ; quiconque passera près de lui sera étonné, et sifflera à cause de toutes ses plaies.
\VS{18}Comme Sodome et Gomorrhe et les villes voisines qui furent détruites, dit Yahweh, il ne sera plus habité par des hommes, il ne sera le séjour d'aucun fils d'homme\FTNT{Ge. 19:25 ; Am. 4:11.}…
\VS{19}Voici, il monte comme un lion des rives orgueilleuses du Jourdain, vers la demeure forte ; soudain, j'en ferai fuir Edom, et j'établirai sur elle celui que j'ai choisi. Car qui est semblable à moi ? Qui me donnera des ordres ? Et quel est le chef qui me résistera en face\FTNT{Job. 41:1.} ?
\VS{20}C'est pourquoi écoutez le conseil que Yahweh a donné contre Edom, et les desseins qu'il a projetés contre les habitants de Théman ! Certainement, on les traînera comme les plus petits du troupeau, certainement on dévastera leur demeure.
\VS{21}La terre tremble au bruit de leur chute ; le bruit de leur cri se fait entendre jusqu'à la Mer Rouge…
\VS{22}Voici, il monte comme un aigle, il vole, il étend ses ailes sur Botsra, et le cœur des hommes forts d'Edom est en ce jour comme le cœur d'une femme en travail.
\TextTitle{Prophétie sur Damas}
\VS{23}Sur Damas. Hamath et Arpad sont honteuses parce qu'elles ont entendu des mauvaises nouvelles, elles tremblent ; il y a une tourmente dans la mer qui ne peut se calmer.
\VS{24}Damas est défaillante, elle se tourne pour fuir, et la panique la saisit ; l'angoisse et les douleurs la saisissent comme une femme qui enfante.
\VS{25}Ah ! Elle n'est pas abandonnée, la ville glorieuse, ma ville de plaisance !
\VS{26}C'est pourquoi ses jeunes gens tomberont dans les places, et tous ses hommes de guerre périront en ce jour, dit Yahweh des armées.
\VS{27}Je mettrai le feu à la muraille de Damas, qui dévorera les palais de Ben-Hadad.
\TextTitle{Prophétie sur Kédar (les Arabes) et Hatsor}
\VS{28}Sur Kédar et les royaumes de Hatsor, que Nebucadnetsar, roi de Babylone, frappa. Ainsi parle Yahweh : Levez-vous, montez vers Kédar, et détruisez les fils d'orient !
\VS{29}On prendra leurs tentes et leurs troupeaux, on prendra leurs tentes, tous leurs bagages et leurs chameaux, et l'on jettera de toutes parts contre eux des cris de terreur.
\VS{30}Fuyez, fuyez de toutes vos forces, cherchez une demeure dans les cavernes, vous habitants de Hatsor ! dit Yahweh ; car Nebucadnetsar, roi de Babylone, a pris une résolution contre vous, il a imaginé un plan contre vous.
\VS{31}Levez-vous, montez vers la nation tranquille qui habite en sécurité, dit Yahweh ; elle n'a ni portes ni barres, elle habite seule\FTNT{Ez. 38:11.}.
\VS{32}Leurs chameaux seront au pillage, et la multitude de leur bétail sera une proie ; je les disperserai à tout vent, vers ceux qui se coupent le coin de la barbe, et je ferai venir de tous les côtés leur détresse, dit Yahweh.
\VS{33}Hatsor sera le repaire des serpents, un désert pour toujours ; personne n'y habitera, et aucun fils d'homme n'y séjournera.
\TextTitle{Prophétie sur Elam}
\VS{34}La parole de Yahweh fut adressée à Jérémie, le prophète, sur Elam, au commencement du règne de Sédécias, roi de Juda, en disant :
\VS{35}Ainsi parle Yahweh des armées : Voici, je vais briser l'arc d'Elam, qui est sa principale force\FTNT{Ez. 32:24-27.}.
\VS{36}Je ferai venir contre Elam les quatre vents des quatre extrémités des cieux, je les disperserai par tous ces vents ; et il n'y aura pas une nation où ne viennent ceux qui seront chassés d'Elam éternellement.
\VS{37}Je ferai trembler les habitants d'Elam devant leurs ennemis, et devant ceux qui cherchent leur vie, je ferai venir le malheur sur eux, l'ardeur de ma colère, dit Yahweh, et j'enverrai l'épée après eux, jusqu'à ce que je les aie consumés.
\VS{38}Je mettrai mon trône dans Elam, et j'en détruirai les rois et les chefs, dit Yahweh.
\VS{39}Mais dans les derniers jours\FTNT{Gn. 49:1-2.}, je ramènerai les captifs d'Elam, dit Yahweh.
\Chap{50}
\TextTitle{Prophétie sur Babylone}
\VerseOne{}La parole que Yahweh prononça sur Babylone, sur le pays des Chaldéens, par le moyen de Jérémie, le prophète :
\VS{2}Annoncez-le parmi les nations, entendez-le, levez une bannière ! Entendez-le, ne le cachez pas ! Dites : Babylone est prise ! Bel est confus, Merodac est brisé ! Ses idoles sont confuses et brisées\FTNT{Es. 46:1.} !
\VS{3}Car une nation monte contre elle du nord, qui mettra son pays en ruines, il n'y aura plus personne qui y habite ; les hommes et les bêtes fuient, ils s'en vont.
\VS{4}En ces jours, et en ce temps-là, dit Yahweh, les fils d'Israël et les fils de Juda viendront ensemble ; ils marcheront en pleurant, et en cherchant Yahweh, leur Dieu.
\VS{5}Ils demanderont la route de Sion, ils tourneront leur visage vers elle : Venez, attachez-vous à Yahweh, par une alliance éternelle qui ne soit jamais oubliée !
\VS{6}Mon peuple était comme un troupeau de brebis perdues ; leurs bergers les égaraient, les rendaient errantes par les montagnes ; elles allaient de montagne en colline, oubliant leur bercail\FTNT{Ez. 34:5-6 ; Za. 10:2 ; Mt. 9:36.}.
\VS{7}Tous ceux qui les trouvaient les dévoraient, et leurs ennemis disaient : Nous ne sommes coupables d'aucun mal, parce qu'ils ont péché contre Yahweh, contre la demeure de la justice, contre Yahweh, l'espérance de leurs pères.
\VS{8}Fuyez du milieu de Babylone, sortez du pays des Chaldéens, et soyez comme les boucs qui vont devant le troupeau\FTNT{Es. 48:20 ; 2 Co. 6:17 ; Ap. 18:4.} !
\VS{9}Car voici, je vais susciter et faire monter contre Babylone une multitude de grandes nations du pays du nord ; elles se rangeront en bataille contre elle, de sorte qu'elle sera prise ; leurs flèches font des ravages comme celles d'un habile guerrier qui ne retourne pas à vide\FTNT{Es. 13:18.}.
\VS{10}Et la Chaldée sera abandonnée au pillage ; tous ceux qui la pilleront seront rassasiés, dit Yahweh.
\VS{11}Oui, réjouissez-vous, soyez dans l'allégresse, vous qui avez pillé mon héritage ! Oui, bondissez comme une génisse qui est dans l'herbe, hennissez comme de puissants chevaux !
\VS{12}Votre mère est fort honteuse, celle qui vous a enfantés rougit de honte ; voici, elle est la dernière entre les nations, c'est un désert, un pays sec et aride.
\VS{13}Elle ne sera plus habitée à cause de la colère de Yahweh, elle ne sera plus qu'une désolation. Quiconque passera près de Babylone sera étonné et sifflera à cause de toutes ses plaies.
\VS{14}Rangez-vous en bataille contre Babylone, mettez-vous tout alentour vous tous qui tendez l'arc ! Tirez contre elle, n'épargnez pas les flèches ! Car elle a péché contre Yahweh.
\VS{15}Poussez des cris de guerre contre elle tout alentour ! Elle tend les mains ; ses fondements tombent ; ses murs sont renversés. Car c'est ici la vengeance de Yahweh. Vengez-vous sur elle ! Faites-lui comme elle a fait\FTNT{Ab. 1:15 ; Ps. 137:8 ; Lu. 6:38.} !
\VS{16}Retranchez de Babylone le semeur, et celui qui manie la faucille au temps de la moisson ! Devant l'épée de l'oppresseur, que chacun se tourne vers son peuple, que chacun s'enfuie vers son pays.
\VS{17}Israël est comme une brebis égarée que les lions ont chassée ; le roi d'Assyrie l'a dévorée le premier ; mais ce dernier, Nebucadnetsar, roi de Babylone, lui a brisé les os.
\VS{18}C'est pourquoi ainsi parle Yahweh des armées, le Dieu d'Israël : Voici, je punirai le roi de Babylone et son pays, comme j'ai puni le roi d'Assyrie\FTNT{Es. 37:36 ; 2 R. 19:35.}.
\VS{19}Je ramènerai Israël dans sa demeure ; il paîtra au Carmel et au Basan, et son âme se rassasiera sur la montagne d'Ephraïm et de Galaad.
\VS{20}En ces jours, et en ce temps-là, dit Yahweh, on cherchera l'iniquité d'Israël, mais il n'y en aura pas, les péchés de Juda ne seront pas trouvés ; car je pardonnerai au reste que j'aurai fait demeurer.
\VS{21}Monte contre ce pays doublement rebelle, contre les habitants, et châtie-les ! Massacre, extermine-les ! dit Yahweh, fais selon toutes les choses que je t'ai ordonnées.
\VS{22}Des cris de guerre retentissent dans le pays, et la ruine est grande.
\VS{23}Eh quoi ! Il est rompu, brisé, le marteau de toute la terre ! Babylone est réduite en une désolation parmi les nations !
\VS{24}Je t'ai tendu un piège, et tu as été prise, Babylone, et tu n'en savais rien ; tu as été trouvée, et même attrapée, parce que tu as lutté contre Yahweh.
\VS{25}Yahweh a ouvert son arsenal et en a sorti les armes de sa colère ; c'est là une œuvre du Seigneur, de Yahweh des armées, dans le pays des Chaldéens.
\VS{26}Venez de toutes parts dans Babylone, ouvrez ses greniers, faites-y des monceaux, comme des tas de gerbes, et détruisez-la ! Qu'il ne reste plus rien d'elle !
\VS{27}Tuez tous ses taureaux et qu'ils descendent à l'abattage ! Malheur à eux ! Car le jour est venu, le temps de leur châtiment.
\VS{28}Ecoutez la voix de ceux qui s'enfuient, de ceux qui sont échappés du pays de Babylone pour annoncer dans Sion la vengeance de Yahweh, notre Dieu, la vengeance de son temple !
\VS{29}Appelez les archers contre Babylone, vous tous qui tendez l'arc ! Campez-vous contre elle tout alentour, que personne n'échappe, rendez-lui selon ses œuvres, faites-lui selon tout ce qu'elle a fait ! Car elle s'est fièrement élevée contre Yahweh, contre le Saint d'Israël\FTNT{Es. 13:11 ; Joë. 3:4-9 ; La. 1:22.}.
\VS{30}C'est pourquoi ses jeunes gens tomberont dans les places, et tous ses hommes de guerre périront en ce jour, dit Yahweh.
\VS{31}Voici, j'en veux à toi, orgueilleuse ! dit le Seigneur, Yahweh des armées ; car ton jour est venu, le temps de ton châtiment.
\VS{32}L'orgueilleuse chancellera et tombera, et il n'y aura personne pour la relever ; je mettrai le feu à ses villes, et il dévorera tous ses environs.
\VS{33}Ainsi parle Yahweh des armées : Les fils d'Israël et les fils de Juda sont ensemble opprimés ; tous ceux qui les ont emmenés captifs les retiennent, et refusent de les laisser aller.
\VS{34}Leur Rédempteur est fort, son nom est Yahweh des armées ; il défendra certainement leur cause, pour donner du repos au pays, et pour faire trembler les habitants de Babylone.
\VS{35}L'épée est sur les Chaldéens ! dit Yahweh, sur les habitants de Babylone, ses chefs, et ses sages !
\VS{36}L'épée est tirée contre ses devins de mensonges ! Qu'ils soient comme des insensés ! L'épée contre ses hommes forts ! Qu'ils soient épouvantés !
\VS{37}L'épée est sur ses chevaux et sur ses chars ! Contre les foules de toute espèce qui sont au milieu d'elle ! Qu'ils deviennent comme des femmes ! L'épée est sur ses trésors ! Qu'ils soient pillés !
\VS{38}La sécheresse est sur ses eaux ! Qu'elles soient mises à sec ! Parce que c'est un pays d'idoles ; ils agissent en insensés à l'égard de leurs idoles\FTNT{Es. 2:8.}.
\VS{39}C'est pourquoi les animaux du désert y habiteront avec les chacals, et les autruches y habiteront aussi ; elle ne sera plus jamais habitée, et on n'y demeurera plus jamais.
\VS{40}Comme Sodome et Gomorrhe, et les villes voisines que Dieu détruisit, dit Yahweh, elle ne sera plus habitée par des hommes, elle ne sera le séjour d'aucun fils d'homme.
\VS{41}Voici, un peuple vient du nord, une grande nation et plusieurs rois se lèvent des extrémités de la terre.
\VS{42}Ils saisissent l'arc et le javelot ; ils sont cruels, et ils n'ont pas de compassion ; leur voix mugit comme la mer ; ils sont montés sur des chevaux, chacun d'eux est rangé en bataille comme un seul homme, contre toi, fille de Babylone !
\VS{43}Le roi de Babylone entend la nouvelle, et ses mains s'affaiblissent, l'angoisse le saisit comme la douleur de celle qui enfante…
\VS{44}Voici, il monte comme un lion des rives orgueilleuses du Jourdain vers la demeure forte ; soudain, je les ferai courir, et je désignerai sur elle celui que j'ai choisi. Car qui est semblable à moi ? Qui me donnera des ordres ? Et quel est le chef qui me résistera en face ?
\VS{45}C'est pourquoi écoutez le conseil que Yahweh a donné contre Babylone, et les desseins qu'il a projetés contre le pays des Chaldéens ! Certainement, on les traînera comme les plus petits du troupeau, certainement on dévastera leur demeure.
\VS{46}La terre tremble au bruit de la prise de Babylone, et le cri se fait entendre parmi les nations.
\Chap{51}
\TextTitle{Le jugement de Babylone par Yahweh}
\VerseOne{}Ainsi parle Yahweh : Voici, je fais lever un vent destructeur contre Babylone et contre ceux qui habitent au cœur du royaume.
\VS{2}J'envoie contre Babylone des vanneurs qui la vanneront, qui videront son pays ; car de tous côtés ils seront contre elle, au jour du malheur.
\VS{3}Qu'on bande l'arc contre celui qui bande son arc, contre celui qui s'élève dans son armure ! N'épargnez pas ses jeunes hommes ! Exterminez toute son armée !
\VS{4}Qu'ils tombent les blessés à mort dans le pays des Chaldéens, percés de coups dans les rues de Babylone !
\VS{5}Car Israël et Juda ne sont pas abandonnés de leur Dieu, de Yahweh des armées, quoique leur pays ai été trouvé par le Saint d'Israël plein de crimes.
\VS{6}Fuyez hors de Babylone, et que chacun sauve sa vie, ne soyez point exterminés dans son iniquité ; car c'est le temps de la vengeance de Yahweh ; il lui rend ce qu'elle a mérité.
\VS{7}Babylone était comme une coupe d'or dans la main de Yahweh, enivrant toute la terre ; les nations ont bu de son vin : C'est pourquoi les nations ont agi comme des insensées.
\VS{8}Babylone est tombée\FTNT{Ap. 18.} en un instant, elle est brisée ! Gémissez sur elle, prenez du baume pour sa douleur : Peut-être qu'elle guérira.
\VS{9}Nous avons pansé Babylone, mais elle n'a pas guéri. Laissons-la et allons-nous-en chacun dans son pays ; car son jugement atteint les cieux et s'élève jusqu'aux nues.
\VS{10}Yahweh a rendu la justice de notre cause ; venez et racontons dans Sion l'œuvre de Yahweh, notre Dieu.
\VS{11}Aiguisez les flèches, remplissez vos mains avec les boucliers ! Yahweh a réveillé l'esprit des rois de Médie, car sa pensée est de détruire Babylone ; c'est ici la vengeance de Yahweh, la vengeance de son temple.
\VS{12}Elevez une bannière contre les murs de Babylone ! Fortifiez les postes, levez des gardes, préparez des embuscades ! Car Yahweh a formé un projet, il fait ce qu'il a dit contre les habitants de Babylone.
\VS{13}Toi qui habites près des grandes eaux, abondantes en trésors, ta fin est venue, ta cupidité est à son terme !
\VS{14}Yahweh des armées a juré par lui-même, en disant : Je te remplirai d'hommes comme de sauterelles, et ils pousseront contre toi des cris de guerre.
\VS{15}C'est lui qui a fait la terre par sa puissance, qui a fondé le monde habitable par sa sagesse, et qui a étendu les cieux par son intelligence\FTNT{Ge. 1:1 ; Es. 40:22 ; Ps. 104:2; Job. 9:8.}.
\VS{16}Lorsqu'il donne de la voix, il y a un tumulte d'eaux dans les cieux, il fait monter les vapeurs des extrémités de la terre, il fait les éclairs et la pluie, il fait sortir le vent de ses réservoirs.
\VS{17}Tout homme devient stupide par sa connaissance, tout fondeur est honteux par les images taillées ; car ses idoles en métal fondu ne sont que mensonge, il n'y a pas de souffle en elles.
\VS{18}Elles ne sont que vanité, une œuvre de tromperie ; elles périront au temps de leur châtiment.
\VS{19}La portion de Jacob n'est pas comme ces choses-là ; car c'est lui qui a tout formé, et Israël est la tribu de son héritage. Son nom est Yahweh des armées.
\VS{20}Tu as été pour moi un marteau, un instrument de guerre. Par toi j'ai brisé des nations, par toi j'ai détruit des royaumes.
\VS{21}Par toi j'ai brisé le cheval et son cavalier ; par toi j'ai brisé le char et celui qui était monté dessus.
\VS{22}Par toi j'ai brisé l'homme et la femme ; par toi j'ai brisé le vieillard et le jeune garçon ; par toi j'ai brisé le jeune homme et la jeune fille.
\VS{23}Par toi j'ai brisé le berger et son troupeau ; par toi j'ai brisé le laboureur et ses bœufs ; par toi j'ai brisé les gouverneurs et les chefs.
\VS{24}Mais je rendrai à Babylone et à tous les habitants de la Chaldée tout le mal qu'ils ont fait à Sion sous vos yeux, dit Yahweh\FTNT{La. 1:21.}.
\VS{25}Voici, j'en veux à toi, montagne de destruction, dit Yahweh, à toi qui détruisais toute la terre ! J'étendrai ma main sur toi, je te roulerai du haut des rochers, je ferai de toi une montagne embrasée.
\VS{26}On ne pourra prendre de toi aucune pierre pour la placer à l'angle de l'édifice ni aucune pierre pour servir de fondement ; car tu seras une ruine éternelle, dit Yahweh\FTNT{Es. 13:19-20.}…
\VS{27}Elevez une bannière dans le pays ! Sonnez du shofar parmi les nations ! Préparez les nations contre elle, appelez contre elle les royaumes d'Ararat, de Minni et d'Aschkenaz ! Établissez contre elle des chefs ! Faites monter ses chevaux comme des sauterelles hérissées !
\VS{28}Préparez contre elle les nations, les rois de Médie, ses gouverneurs et tous ses chefs, et tout le pays sous leur domination\FTNT{Es. 13:17.} !
\VS{29}La terre tremble, elle se tord ; car la pensée de Yahweh se dresse contre Babylone ; il va faire du pays de Babylone un désert sans habitants\FTNT{Es. 13:14 ; Joë. 3:16.}.
\VS{30}Les hommes forts de Babylone cessent de combattre, ils demeurent dans les forteresses ; leur force est épuisée, ils sont comme des femmes. On met le feu aux demeures, on brise les barres.
\VS{31}Le courrier rencontre le courrier, et le messager rencontre le messager, pour annoncer au roi de Babylone que sa ville est prise par tous les côtés,
\VS{32}que les gués sont saisis, les marais brûlés par le feu, et les hommes de guerre épouvantés.
\VS{33}Car ainsi parle Yahweh des armées, le Dieu d'Israël : La fille de Babylone est comme une aire dans le temps où on la foule ; encore un peu de temps, et le moment de la moisson sera venu pour elle.
\VS{34}Nebucadnetsar, roi de Babylone, m'a dévorée, m'a détruite ; il a fait de moi un vase vide ; il m'a engloutie tel un dragon, il a rempli son ventre de mes délices ; il m'a chassée au loin.
\VS{35}Que la violence envers moi et ma chair déchirée retombe sur Babylone ! dit l'habitante de Sion. Que mon sang retombe sur les habitants de la Chaldée ! dit Jérusalem.
\VS{36}C'est pourquoi ainsi parle Yahweh : Voici, je défendrai ta cause, je te vengerai ! Je dessécherai la mer de Babylone, et je ferai tarir sa source.
\VS{37}Babylone sera un monceau de ruines, un repaire de serpents, un objet d'épouvante et de moquerie ; sans que personne n'y habite.
\VS{38}Ils rugiront ensemble comme des lions, ils pousseront des cris comme des lionceaux.
\VS{39}Quand ils seront échauffés, je les ferai boire, et les enivrerai, afin qu'ils se réjouissent et qu'ils dorment d'un sommeil éternel et qu'ils ne se réveillent plus, dit Yahweh.
\VS{40}Je les ferai descendre comme des agneaux à la boucherie, comme des béliers et des boucs.
\VS{41}Eh quoi ! Schéschac est prise ! Celle dont la louange remplissait toute la terre est conquise ! Eh quoi ! Babylone est réduite en désolation parmi les nations !
\VS{42}La mer est montée sur Babylone, elle a été couverte de la multitude de ses flots\FTNT{Es. 8:8 ; Ez. 26:3-19 ; Lu. 21:25.}.
\VS{43}Ses villes sont en ruines, une terre sèche et déserte ; c'est un pays où personne ne demeure, et où il ne passe aucun fils d'homme.
\VS{44}Je punirai aussi Bel à Babylone, je sortirai de sa bouche ce qu'il a englouti, et les nations n'aborderont plus vers lui. Le mur même de Babylone est tombé !
\VS{45}Mon peuple, sortez du milieu d'elle, et que chacun sauve sa vie de l'ardeur de la colère de Yahweh.
\VS{46}Que votre cœur ne se trouble pas, et ne craignez pas les nouvelles qu'on entendra dans tout le pays ; car cette année viendra une nouvelle, et l'année d'après une autre nouvelle, et il y aura violence dans le pays, et un dominateur s'élèvera contre un autre dominateur.
\VS{47}C'est pourquoi voici, les jours viennent où je punirai les idoles de Babylone, et tout son pays sera honteux ; tous ses morts tomberont au milieu d'elle.
\VS{48}Les cieux, la terre, et tout ce qui y est, pousseront des cris de joie contre Babylone ; car du nord les dévastateurs viendront contre elle, dit Yahweh.
\VS{49}Babylone tombera, ô morts d'Israël, comme Babylone a fait tomber les morts de tout le pays.
\VS{50}Vous qui avez échappé à l'épée, allez, ne vous arrêtez pas ! Souvenez-vous de Yahweh dans ces pays éloignés, et que Jérusalem revienne à vos cœurs !
\VS{51}Nous étions honteux des reproches que nous entendions ; la honte couvrait nos visages, quand les étrangers sont venus dans le sanctuaire de la maison de Yahweh.
\VS{52}C'est pourquoi, voici, les jours viennent, dit Yahweh, où je châtierai ses idoles ; et les blessés gémiront dans tout son pays.
\VS{53}Quand Babylone monterait jusqu'aux cieux et qu'elle rendrait inaccessible le plus haut de sa forteresse, alors les dévastateurs viendront contre elle, dit Yahweh\FTNT{Am. 9:2 ; Ab. 1:4.}…
\VS{54}Un grand cri s'entend de Babylone, et la ruine est grande dans le pays des Chaldéens.
\VS{55}Parce que Yahweh dévaste Babylone, il en fait échapper de grands cris ; les flots des dévastateurs mugissent comme de grandes eaux, le bruit du mugissement s'étend.
\VS{56}Car le destructeur est venu contre elle, contre Babylone ; ses hommes forts sont pris, leurs arcs sont brisés. Car Yahweh est un Dieu qui rend à chacun selon ses œuvres, qui paie à chacun son salaire.
\VS{57}J'enivrerai ses princes et ses sages, ses gouverneurs, ses chefs, et ses hommes forts ; ils dormiront d'un sommeil éternel, et ils ne se réveilleront plus, dit le Roi dont le nom est Yahweh des armées.
\VS{58}Ainsi parle Yahweh des armées : Les larges murs de Babylone seront renversés, ses portes qui sont si hautes, seront brûlées par le feu ; ainsi les peuples auront travaillé en vain, les nations se seront fatiguées pour le feu.
\VS{59}C'est ici l'ordre que Jérémie, le prophète, donna à Seraja, fils de Nérija, fils de Machséja, lorsqu'il alla avec Sédécias, roi de Juda, la quatrième année du règne de Sédécias. Or Seraja était premier chambellan.
\VS{60}Jérémie écrivit dans un livre tous les malheurs qui devaient venir sur Babylone, toutes ces paroles qui sont écrites contre Babylone.
\VS{61}Jérémie dit à Seraja : Lorsque tu seras venu à Babylone, et que tu auras vu, tu liras toutes ces paroles,
\VS{62}et tu diras : Yahweh, c'est toi qui as déclaré que ce lieu serait exterminé, en sorte qu'il n'y ait aucun habitant, depuis l'homme jusqu'à la bête, mais qu'il deviendrait un désert pour toujours.
\VS{63}Et quand tu auras achevé de lire ce livre, tu le lieras à une pierre et tu le jetteras dans l'Euphrate,
\VS{64}et tu diras : Ainsi Babylone sera submergée, elle ne se lèvera pas des malheurs que je ferai venir sur elle ; ils seront épuisés. Jusqu'ici sont les paroles de Jérémie.
\Chap{52}
\TextTitle{Chute de Jérusalem et destruction du temple ; Juda déporté à Babylone\FTNTT{2 R. 25:1-26; Jé. 39:1-10.}}
\VerseOne{}Sédécias avait vingt et un ans lorsqu'il devint roi, et il régna onze ans à Jérusalem. Sa mère se nommait Hamuthal, fille de Jérémie, de Libna\FTNT{2 R. 24 et 25.}.
\VS{2}Il fit ce qui est mal aux yeux de Yahweh, comme avait fait Jojakim.
\VS{3}Et cela arriva à cause de la colère de Yahweh contre Jérusalem et Juda, qu'il voulait chasser de devant sa face. Sédécias se rebella contre le roi de Babylone.
\VS{4}La neuvième année de son règne, le dixième jour du dixième mois, Nebucadnetsar, roi de Babylone, vint contre Jérusalem, lui et toute son armée ; ils campèrent devant elle et construisirent des retranchements tout alentour.
\VS{5}La ville fut assiégée jusqu'à la onzième année du roi Sédécias.
\VS{6}Le neuvième jour du quatrième mois, la famine était forte dans la ville, et il n'y avait pas de pain pour le peuple du pays\FTNT{La. 2:11-12.}.
\VS{7}Alors la brèche fut faite à la ville ; et tous les gens de guerre s'enfuirent et sortirent de nuit hors de la ville par le chemin de la porte entre les deux murailles près du jardin du roi, tandis que les Chaldéens entouraient la ville. Ils s'en allèrent par le chemin de la plaine.
\VS{8}Mais l'armée des Chaldéens poursuivit le roi, et ils atteignirent Sédécias dans les plaines de Jéricho ; et toute son armée se dispersa loin de lui.
\VS{9}Ils prirent le roi, et le firent monter vers le roi de Babylone à Ribla, dans le pays de Hamath ; et il prononça contre lui une sentence.
\VS{10}Le roi de Babylone fit égorger les fils de Sédécias sous ses yeux ; il fit aussi égorger tous les chefs de Juda à Ribla.
\VS{11}Puis il fit crever les yeux à Sédécias, et le fit lier avec des chaînes d'airain ; le roi de Babylone l'emmena à Babylone, et le mit en prison jusqu'au jour de sa mort.
\VS{12}Le dixième jour du cinquième mois, c'était la dix-neuvième année du règne de Nebucadnetsar, roi de Babylone, Nebuzaradan, chef des gardes, qui se tenait devant le roi de Babylone, entra dans Jérusalem.
\VS{13}Il brûla la maison de Yahweh, la maison du roi, et toutes les maisons de Jérusalem ; il brûla toutes les maisons des personnes considérées.
\VS{14}Toute l'armée des Chaldéens, qui était avec le chef des gardes, renversa toutes les murailles qui entouraient Jérusalem.
\VS{15}Nebuzaradan, chef des gardes, transporta à Babylone une partie des plus pauvres du peuple, le reste du peuple qui était demeuré dans la ville, ceux qui s'étaient rendus au roi de Babylone, et le reste de la multitude.
\VS{16}Toutefois, Nebuzaradan, chef des gardes, laissa quelques-uns des plus pauvres du pays pour être vignerons et laboureur.
\VS{17}Les Chaldéens brisèrent les colonnes d'airain qui étaient dans la maison de Yahweh, les bases, la mer d'airain qui était dans la maison de Yahweh, et ils emmenèrent tout l'airain à Babylone.
\VS{18}Ils prirent les cendriers, les pelles, les couteaux, les coupes, les tasses, et tous les ustensiles d'airain avec lesquels on faisait le service.
\VS{19}Le chef des gardes prit aussi les coupes, les encensoirs, les cendriers, les chandeliers, les tasses et les calices, ce qui était d'or et ce qui était d'argent.
\VS{20}Les deux colonnes, la mer, et les douze bœufs d'airain qui servaient de base, et que le roi Salomon avait faits pour la maison de Yahweh, tous ces ustensiles d'airain ne pouvaient être pesés.
\VS{21}La hauteur de l'une des colonnes était de dix-huit coudées, et un cordon de douze coudées l'entourait ; elle était épaisse de quatre doigts, et creuse;
\VS{22}il y avait par-dessus un chapiteau d'airain, et la hauteur d'un des chapiteaux était de cinq coudées ; il y avait aussi un treillis et des grenades tout autour du chapiteau, le tout d'airain ; il en était de même pour la seconde colonne avec des grenades\FTNT{1 R. 7:15-20.}.
\VS{23}Il y avait quatre-vingt-seize grenades de chaque côté, et les grenades qui étaient autour du treillis étaient au nombre de cent.
\VS{24}Le chef des gardes prit Seraja, qui était le premier sacrificateur, Sophonie, qui était le second sacrificateur, et les trois gardiens du seuil.
\VS{25}Il prit de la ville un eunuque qui avait sous son commandement des hommes de guerre, sept hommes de ceux qui voyaient la face du roi et qui furent trouvés dans la ville, le secrétaire du chef de l'armée qui enrôlait le peuple du pays, et soixante hommes du peuple du pays, qui se trouvèrent dans la ville.
\VS{26}Nebuzaradan, chef des gardes, les prit et les emmena vers le roi de Babylone à Ribla.
\VS{27}Le roi de Babylone les frappa et les fit mourir à Ribla, dans le pays de Hamath. Ainsi Juda fut transporté hors de son pays.
\VS{28}Et c'est ici le peuple que Nebucadnetsar emmena en captivité : La septième année, trois mille vingt-trois juifs ;
\VS{29}la dix-huitième année de Nebucadnetsar, il emmena de Jérusalem huit cent trente-deux personnes ;
\VS{30}La vingt-troisième année de Nebucadnetsar, Nebuzaradan, chef des gardes, transporta en exil sept cent quarante-cinq personnes des Juifs ; en tout quatre mille six cents personnes.
\VS{31}La trente-septième année de la captivité de Jojakin, roi de Juda, le vingt-cinquième jour du douzième mois, Evil-Merodac, roi de Babylone, dans la première année de son règne, leva la tête de Jojakin, roi de Juda, et le fit sortir de prison.
\VS{32}Il lui parla avec bonté, et mit son trône au-dessus du trône des autres rois qui étaient avec lui à Babylone.
\VS{33}Il fit changer ses vêtements de prison, il mangea du pain tous les jours de sa vie en présence du roi.
\VS{34}Le roi de Babylone lui donna continuellement des vivres pour chaque jour, jusqu'au jour de sa mort, tout le temps de sa vie.
\PPE{}
\end{multicols}

\clearpage\ShortTitle{Ezéchiel}\BookTitle{Ezéchiel}\BFont
\noindent\hrulefill
{\footnotesize
\textit{
\bigskip
{\centering{}
\\Auteur : Ezéchiel
\\(Heb. : Yechezqe'l)
\\Signification : Dieu fortifie
\\Thème : Jugements et gloire
\\Date de rédaction : 6\up{ème} siècle av. J.-C.\\}
}
%\bigskip
\textit{
\\Déporté à Babylone alors qu'il remplissait la fonction de sacrificateur, Ezéchiel eut la particularité d'exercer le ministère prophétique hors de la terre d'Israël. Sa mission était en partie d'affermir la foi des déportés par la promesse du jugement de leurs ennemis et du rétablissement de la nation. Il leur rappela aussi que les péchés de leurs pères étaient la raison de leur captivité et que l'occasion leur était donnée de réformer leurs voies.
%\bigskip
\\Ezéchiel reçut aussi des oracles concernant ceux qui étaient restés à Jérusalem et dont la condition n'était guère meilleure que celle des exilés et pour lesquels le pire était à venir. Ses prophéties s'exprimaient en songes et en visions, c'est donc ainsi qu'il vit la gloire de Yahweh quittant le temple de Jérusalem. Il reçut plusieurs prophéties sur les derniers temps, notamment la promesse d'un cœur nouveau en vue de la conversion, le retour de la gloire de Dieu lors du règne millénaire et aussi le rétablissement total d'Israël.\bigskip
}
}
\par\nobreak\noindent\hrulefill
\begin{multicols}{2}
\Chap{1}
\TextTitle{Vision de la gloire de Yahweh}
\VerseOne{}Or il arriva en la trentième année, au cinquième jour du quatrième mois, comme j'étais parmi ceux qui avaient été transportés sur le fleuve de Kebar, que les cieux furent ouverts, et je vis des visions de Dieu.
\VS{2}Au cinquième jour du mois de cette année, qui fut la cinquième après que le roi Jojakin\FTNT{2 R. 24:12-16.}ait été mené en captivité,
\VS{3}la parole de Yahweh fut adressée expressément à Ezéchiel, le sacrificateur, fils de Buzi, dans le pays des Chaldéens\FTNT{Ezéchiel était en exil à Babylone.}, sur le fleuve de Kebar, et la main de Yahweh fut là sur lui.
\VS{4}Je regardai donc, et voici, un vent impétueux vint du nord, une grosse nuée, et un feu qui prenait de tous côtés. Il y avait autour de la nuée une splendeur, et au milieu de la nuée paraissait comme de l'airain poli, lorsqu'il sort du milieu du feu.
\VS{5}Et du milieu aussi paraissait une ressemblance de quatre animaux\FTNT{Les quatre animaux représentent quatre aspects de Jésus. La face de l'homme correspond à l'humanité du Seigneur mise en exergue dans l'évangile de Luc. La face de lion symbolise la royauté de Christ, mise en évidence dans l'évangile de Matthieu. La face de bœuf fait écho à l'évangile de Marc où le Seigneur y est présenté comme serviteur. La face d'aigle symbolise la divinité du Messie mise en évidence dans l'évangile de Jean. Le Seigneur y est présenté comme le Fils de Dieu et le Dieu véritable.} et voici leur forme: Ils avaient une ressemblance humaine.
\VS{6}Et chacun d'eux avait quatre faces, et chacun avait quatre ailes.
\VS{7}Et leurs pieds étaient des pieds droits, et la plante de leurs pieds était comme la plante d'un pied de veau, ils étincelaient comme la couleur d'un airain poli.
\VS{8}Et il y avait des mains d'homme sous leurs ailes à leurs quatre côtés ; et tous quatre avaient leurs faces et leurs ailes.
\VS{9}Leurs ailes étaient jointes l'une à l'autre ; ils ne se tournaient point quand ils marchaient, mais chacun marchait droit devant soi.
\VS{10}Leurs faces ressemblaient à la face d'un homme, à la face de lion à la main droite, à la face de bœuf à la gauche des quatre, et à la face d'aigle à tous les quatre.
\VS{11}Leurs faces et leurs ailes étaient divisées par le haut ; chacun avait des ailes qui se joignaient l'une à l'autre, et deux couvraient leurs corps.
\VS{12}Chacun marchait droit devant soi ; ils allaient partout où l'Esprit les poussait à aller, et ils ne se tournaient point lorsqu'ils marchaient.
\VS{13}Et quant à l'aspect des animaux, leur regard était comme des charbons de feu ardent, comme des torches ; le feu courait parmi les animaux ; et le feu avait une splendeur, et de ce feu sortait un éclair.
\VS{14}Et les animaux couraient et revenaient selon que l'éclair paraissait.
\VS{15}Je regardais les animaux, et voici, une roue apparut sur la terre auprès des animaux, devant leurs quatre faces.
\VS{16}Et l'aspect et la forme des roues étaient comme la couleur d'un chrysolithe, et toutes les quatre avaient un même aspect ; leur aspect et leur structure étaient comme si chaque roue avait été au milieu d'une autre roue.
\VS{17}En marchant, elles allaient de leurs quatre côtés, et elles ne se tournaient point quand elles marchaient.
\VS{18}Elles avaient des jantes, elles étaient si hautes qu'elles faisaient peur, et leurs jantes étaient pleines d'yeux tout autour des quatre roues.
\VS{19}Quand ils marchaient, elles marchaient auprès d'eux ; et quand ils s'élevaient au-dessus de la terre, elles aussi s'élevaient.
\VS{20}Ils allaient partout où l'Esprit les poussait à aller ; l'Esprit tendait-il là, ils y allaient, et les roues s'élevaient avec eux, car l'esprit des animaux était dans les roues.
\VS{21}Quand ils marchaient, elles marchaient ; et quand ils s'arrêtaient, elles s'arrêtaient ; et quand ils s'élevaient au-dessus de la terre, les roues aussi s'élevaient avec eux, car l'esprit des animaux était dans les roues.
\VS{22}L'aspect de ce qui était au-dessus des têtes des animaux, était une étendue semblable à un cristal étincelant et terrible à voir, laquelle s'étendait au-dessus de leurs têtes.
\VS{23}Sous l'étendue, leurs ailes se tenaient droites l'une contre l'autre ; ils avaient chacun deux ailes dont ils se couvraient, chacun, dis-je, en avait deux qui couvraient leurs corps.
\VS{24}Puis j'entendis le bruit que faisaient leurs ailes quand ils marchaient, semblable au bruit des grandes eaux, et au bruit du Tout-Puissant, un bruit éclatant comme le bruit d'une armée ; quand ils s'arrêtaient, ils baissaient leurs ailes.
\VS{25}Et lorsqu'ils s'arrêtaient et laissaient tomber leurs ailes, il se faisait un bruit au-dessus de l'étendue qui était sur leurs têtes.
\VS{26}Et au-dessus de cette étendue, qui était sur leurs têtes, il y avait quelque chose de semblable à une pierre de saphir, en forme de trône ; et sur cette forme de trône apparaissait comme une figure d'homme\FTNT{Il est question ici de la manifestation du Messie} placée dessus en hauteur.
\VS{27}Je vis encore comme de l'airain poli semblable à un feu, au dedans duquel était cet homme, et qui l'environnait ; depuis la forme de ses reins jusqu'en haut et depuis la forme de ses reins jusqu'en bas, je vis comme du feu, et il y avait une lumière éclatante autour de lui.
\VS{28}Et la splendeur qui se voyait autour de lui, était comme l'arc qui se fait dans la nuée en un jour de pluie. C'est là la vision de la représentation de la gloire de Yahweh. A sa vue je tombai sur ma face, et j'entendis une voix qui parlait.
\Chap{2}
\TextTitle{Mandat d'Ezechiel}
\VerseOne{}Il me dit : Fils de l'homme, tiens-toi sur tes pieds, et je parlerai avec toi.
\VS{2}Alors l'Esprit entra en moi, après qu'il m'eut parlé, et il me releva sur mes pieds, et j'entendis celui qui me parlait.
\VS{3}Il me dit : Fils de l'homme, je t'envoie vers les fils d'Israël, vers des nations rebelles qui se sont rebellées contre moi. Eux et leurs pères ont péché contre moi jusqu'à ce jour même\FTNT{Jé. 3:25.}.
\VS{4}Ce sont des enfants à la face dure et au cœur obstiné, vers lesquels je t'envoie vers eux ; c'est pourquoi tu leur diras que le Seigneur Yahweh a ainsi parlé.
\VS{5}Et soit qu'ils écoutent, ou qu'ils n'en fassent rien, car ils sont une maison rebelle ; ils sauront pourtant qu'il y aura eu un prophète parmi eux\FTNT{Es. 6:9-10.}.
\VS{6}Mais toi, fils de l'homme, ne les crains point, et ne crains point leurs paroles ; quoique des gens rebelles et dont les langues sont perçantes comme des épines soient avec toi, et que tu habites parmi des scorpions ; ne crains point leurs paroles, et ne t'effraie point à cause d'eux, quoiqu'ils soient une maison rebelle\FTNT{Jé. 1:8 ; 1 Pi. 3:14.}.
\VS{7}Tu leur prononceras mes paroles, qu'ils écoutent ou qu'ils n'en fassent rien, car ils ne sont que rébellion.
\VS{8}Mais toi, fils de l'homme, écoute ce que je te dis, et ne sois point rebelle comme cette maison rebelle ; ouvre ta bouche et mange ce que je vais te donner\FTNT{Ap. 10:9 ; Jé. 15:16.}.
\VS{9}Alors je regardai, et voici, une main fut envoyée vers moi, et voici, elle avait un livre en rouleau.
\VS{10}Et elle l'ouvrit devant moi, et voici, il était écrit dedans et dehors ; des lamentations, des soupirs, et des gémissements y étaient écrits.
\Chap{3}
\TextTitle{Yahweh établit Ezéchiel comme sentinelle}
\VerseOne{}Puis il me dit : Fils de l'homme, mange ce que tu trouveras, mange ce rouleau, et va, parle à la maison d'Israël !
\VS{2}J'ouvris donc ma bouche, et il me fit manger ce rouleau.
\VS{3}Il me dit : Fils de l'homme, nourris ton ventre et remplis tes entrailles de ce rouleau que je te donne ! Je le mangeai, et il fut doux dans ma bouche comme du miel\FTNT{Ps. 119:103.}.
\VS{4}Puis il me dit : Fils de l'homme, lève-toi et va vers la maison d'Israël, et prononce-leur mes paroles !
\VS{5}Car tu n'es point envoyé vers un peuple au langage inconnu, ou à la langue barbare ; c'est vers la maison d'Israël ;
\VS{6}ni vers plusieurs peuples ayant un langage inconnu ou une langue barbare, dont tu ne puisses comprendre les paroles. Si je t'envoyais vers eux, ils t'écouteraient.
\VS{7}Mais la maison d'Israël ne voudra pas t'écouter, parce qu'ils ne veulent point m'écouter ; car toute la maison d'Israël a le front dur et le cœur obstiné.
\VS{8}Voici, j'endurcirai ta face contre leurs faces, et j'endurcirai ton front contre leurs fronts\FTNT{Jé. 1:18 ; Mi. 3:8.}.
\VS{9}Et j'ai rendu ton front semblable à un diamant, plus dur que le roc. Ne les crains donc point, et ne t'effraie point à cause d'eux, quoiqu'ils soient une maison rebelle.
\VS{10}Puis il me dit : Fils de l'homme, reçois dans ton cœur et écoute de tes oreilles toutes les paroles que je te dirai.
\VS{11}Lève-toi donc, va vers ceux qui ont été emmenés captifs, vers les enfants de ton peuple, parle-leur et dis-leur que le Seigneur Yahweh a ainsi parlé ; soit qu'ils écoutent ou qu'ils n'en fassent rien.
\VS{12}Puis l'Esprit m'enleva, et j'entendis derrière moi le bruit d'un grand tremblement, disant : Bénie soit la gloire de Yahweh du lieu de sa demeure !
\VS{13}Et j'entendis le bruit des ailes des animaux, qui s'entre-touchaient les unes les autres, et le bruit des roues auprès d'eux, et le bruit d'un grand tremblement.
\VS{14}L'Esprit donc m'enleva, et me prit, et j'allai, l'esprit rempli d'amertume et de colère, mais la main de Yahweh me fortifia.
\VS{15}Je vins donc vers ceux qui avaient été transportés à Thel-Abib, vers ceux qui demeuraient auprès du fleuve de Kebar ; et je me tins là où ils se tenaient, même je me tins là parmi eux sept jours, tout étonné.
\VS{16}Et au bout de sept jours, la parole de Yahweh me fut adressée, en disant :
\VS{17}Fils de l'homme, je t'établis pour être sentinelle sur la maison d'Israël ; tu écouteras donc la parole de ma bouche, et tu les avertiras de ma part\FTNT{Es. 52:8 ; Es. 62:6 ; Jé. 6:17.}.
\VS{18}Quand je dirai au méchant : Tu mourras, tu mourras ! Si tu ne l'avertis pas, et si tu ne parles pas pour l'avertir de se détourner de ses mauvaises voies, afin de lui sauver la vie ; ce méchant-là mourra dans son iniquité, mais je redemanderai son sang de ta main.
\VS{19}Et si tu avertis le méchant, et qu'il ne se détourne pas de sa méchanceté ni de ses mauvaises voies, il mourra dans son iniquité, mais toi, tu sauveras ton âme\FTNT{Ez. 18:23-24 ; Ez. 33:6.}.
\VS{20}Pareillement, si le juste se détourne de sa justice et commet l'iniquité, lorsque j'aurai mis quelque obstacle devant lui, il mourra ; parce que tu ne l'auras point averti, il mourra dans son péché, et il ne sera point fait mention de ses justices qu'il aura faites ; mais je te redemanderai son sang de ta main.
\VS{21}Et si tu avertis le juste de ne point pécher, et qu'il ne pèche point, il vivra, il vivra parce qu'il aura été averti, et toi pareillement tu sauveras ton âme.
\VS{22}Et la main de Yahweh fut sur moi, et il me dit : Lève-toi, et sors vers la vallée, et là je te parlerai.
\VS{23}Je me levai donc, et sortis dans la vallée ; voici, la gloire de Yahweh se tenait là, telle que je l'avais vue près du fleuve de Kebar, et je tombai sur ma face.
\VS{24}Alors l'Esprit entra en moi et me releva sur mes pieds ; il me parla et me dit : Entre, et enferme-toi dans ta maison.
\VS{25}Fils de l'homme, voici, on mettra des cordes sur toi, on te liera, et tu ne sortiras point pour aller parmi eux.
\VS{26} Et j'attacherai ta langue à ton palais, tu seras muet, et tu ne les reprendras point ; parce qu'ils sont une maison  rebelle\FTNT{Jn. 1:20-22.}.
\VS{27}Mais quand je te parlerai, j'ouvrirai ta bouche, et tu leur diras : Ainsi parle le Seigneur Yahweh : Que celui qui écoute, écoute ; et que celui qui n'écoute pas, n'écoute pas ; car ils sont une maison rebelle.
\Chap{4}
\TextTitle{Signes annonciateurs du jugement de Jérusalem : 
\\La brique, la plaque de fer et les cordes}
\VerseOne{}Toi, fils de l'homme, prends une brique et place-la devant toi, et traces-y la ville de Jérusalem.
\VS{2}Puis tu mettras le siège contre elle, tu bâtiras contre elle des retranchements, tu élèveras contre elle des terrasses, tu mettras des camps contre elle, et tu mettras autour d'elle des béliers pour la battre\FTNT{2 R. 25:1.}.
\VS{3}Tu prendras aussi une plaque de fer, et tu la mettras comme un mur de fer entre toi et la ville ; tu dresseras ta face contre elle, elle sera assiégée, et tu l'assiégeras ; ce sera un signe pour la maison d'Israël.
\VS{4}Après, tu dormiras sur ton côté gauche, mets-y l'iniquité de la maison d'Israël, et tu porteras leur iniquité autant de jours que tu seras couché sur ce côté.
\VS{5}Et je t'ai assigné un nombre de jours égal à celui des années de leur iniquité : Trois cent quatre-vingt-dix jours ; ainsi tu porteras l'iniquité de la maison d'Israël.
\VS{6}Et quand tu auras accompli ces jours-là, tu dormiras une seconde fois sur ton côté droit, et tu porteras l'iniquité de la maison de Juda pendant quarante jours ; un jour pour chaque année, car je t'ai assigné un jour pour chaque année.
\VS{7}Tu tourneras ta face et ton bras nu vers Jérusalem assiégée, et tu prophétiseras contre elle.
\VS{8}Et voici, j'ai mis sur toi des cordes, afin que tu ne puisses pas te tourner d'un côté sur l'autre, jusqu'à ce que tu aies accompli les jours de ton siège.
\TextTitle{Le pain impur}
\VS{9}Tu prendras aussi du froment, de l'orge, des fèves, des lentilles, du millet, et de l'épeautre ; tu les mettras dans un vase, et tu en feras du pain autant de jours que tu seras couché sur ton côté ; tu en mangeras pendant trois cent quatre-vingt-dix jours.
\VS{10}La viande que tu mangeras sera du poids de vingt sicles par jour ; et tu la mangeras de temps à autre.
\VS{11}Et tu boiras de l'eau par mesure; savoir la sixième de hin ; tu la boiras de temps à autre.
\VS{12}Et tu mangeras aussi des gâteaux d'orge, que tu feras cuire avec des excréments humains en leur présence.
\VS{13}Puis Yahweh dit : Les fils d'Israël mangeront ainsi leur pain souillé parmi les nations vers lesquelles je les chasserai\FTNT{Os. 9:3 ; Da. 1:8.}.
\VS{14}Et je dis : Ah ! Seigneur Yahweh, voici, mon âme n'a point été souillée, et je n'ai mangé d'aucune bête morte d'elle-même, ou déchirée par les bêtes sauvages, depuis ma jeunesse jusqu'à présent, et aucune chair impure n'est entrée dans ma bouche\FTNT{Lé 17:15 ; De. 14:3 ; Ac. 10:14.}.
\VS{15}Il me répondit : Voici, je te donne des excréments de bœuf au lieu d'excréments humains, et tu y feras cuire ton pain.
\VS{16}Puis il me dit : Fils de l'homme, voici, je m'en vais rompre le bâton du pain dans Jérusalem ; et ils mangeront leur pain au poids et avec chagrin ; et ils boiront de l'eau par mesure et avec horreur\FTNT{Lé. 26:26 ; Es. 3:1 ; Ps. 105:16 ; La. 5:4.}.
\VS{17}Parce que le pain et l'eau leur manqueront, ils seront épouvantés, se regardant les uns les autres, et ils se décomposeront à cause de leur iniquité.
\Chap{5}
\TextTitle{Les cheveux coupés et divisés en trois}
\VerseOne{}Et toi, fils de l'homme, prends un couteau tranchant, prends un rasoir de barbier, et fais-le passer sur ta tête et sur ta barbe. Puis, tu prendras une balance à peser, et tu partageras ce que tu auras rasé\FTNT{Lé. 21:5 ; Ez. 44:20.}.
\VS{2}Brûles-en un tiers dans le feu, au milieu de la ville, lorsque les jours du siège seront accomplis ; prends-en un tiers, et frappe-le avec l'épée tout autour de la ville ; disperses-en un tiers au vent, car je tirerai l'épée derrière eux\FTNT{Lé. 26:25 ; La. 1:20.}.
\VS{3}Tu en prendras une petite quantité que tu serreras aux pans de ton manteau.
\VS{4}De ceux-là, tu en prendras encore, les jetteras au milieu du feu, et les brûleras au feu. De là sortira un feu contre toute la maison d'Israël.
\VS{5}Ainsi parle le Seigneur Yahweh : C'est là cette Jérusalem que j'avais placée au milieu des nations et des pays qui sont autour d'elle.
\VS{6}Elle a changé mes ordonnances et s'est rendue plus coupable que les nations et les pays d'alentour ; car ils ont rejeté mes ordonnances, et n'ont point marché dans mes ordonnances.
\VS{7}C'est pourquoi ainsi parle le Seigneur Yahweh : Parce que vous avez multiplié vos méchancetés plus que les nations qui vous entourent, et que vous n'avez point suivi mes ordonnances et observé mes lois, et que vous n'avez pas agi selon les ordonnances des nations qui vous entourent ;
\VS{8}à cause de cela, ainsi parle le Seigneur Yahweh : Voici, j'en veux à toi et j'exécuterai au milieu de toi mes jugements, sous les yeux des nations.
\VS{9}Je te ferai, à cause de toutes tes abominations, des choses que je n'ai jamais faites, et ce que je ne ferai jamais\FTNT{Da. 9:12 ; Mt. 24:21.}.
\VS{10}Des pères mangeront leurs fils au milieu de toi, et des fils mangeront leurs pères ; j'exécuterai mes jugements sur toi, et je disperserai à tous les vents tout ce qui restera de toi\FTNT{Lé. 26:33 ; De. 28:64 ; Jé. 9:16 ; Za. 2:6.}.
\VS{11}Je suis vivant, dit le Seigneur Yahweh, parce que tu as souillé mon lieu saint par toutes tes infamies, et par toutes tes abominations, moi-même je te raserai, et mon oeil ne t'épargnera point, et je n'aurai point de compassion\FTNT{Jé. 7:9-11.}.
\VS{12}Un tiers d'entre vous mourra de la peste, et sera consumé par la famine au milieu de toi ; un tiers tombera par l'épée autour de toi ; et je disperserai à tous les vents l'autre tiers, je tirerai l'épée derrière eux.
\VS{13}Car ma colère sera portée à son comble, je ferai reposer ma fureur sur eux, et je me donnerai satisfaction ; ils sauront que moi, Yahweh, j'ai parlé dans ma jalousie, quand j'aurai consumé ma fureur sur eux.
\VS{14}Je ferai de toi un désert, un sujet d'opprobre parmi les nations qui sont autour de toi, aux yeux de tous les passants\FTNT{Lé. 26:31-32 ; Né. 2:17.}.
\VS{15}Tu seras en opprobre, en ignominie, un exemple et un sujet d'étonnement pour les nations qui t'entourent, quand j'aurai exécuté mes jugements sur toi, avec colère, avec fureur, et par des châtiments pleins de fureur ; moi, Yahweh, j'ai parlé\FTNT{De. 28:37 ; 1 R. 9:7 ; Ps. 79:4 ; Jé. 24:9 ; Es. 26:9.}.
\VS{16}Quand je lancerai sur eux les flèches douloureuses de la famine, qui seront mortelles, quand je les enverrai pour vous détruire, j'ajouterai la famine sur vous, et romprai pour vous le bâton du pain\FTNT{De. 32:24.}.
\VS{17}Je vous enverrai la famine, et des bêtes féroces, qui te priveront d'enfants ; la peste et le sang passeront au milieu de toi, et je ferai venir l'épée sur toi. Moi,Yahweh, j'ai parlé.
\Chap{6}
\TextTitle{Grâce de Yahweh pour quelques réchappés d'Israël}
\VerseOne{}La parole de Yahweh me fut encore adressée, en disant :
\VS{2}Fils de l'homme, tourne ta face contre les montagnes d'Israël, et prophétise contre elles !
\VS{3}Et dis : Montagnes d'Israël, écoutez la parole du Seigneur Yahweh. Ainsi parle le Seigneur Yahweh aux montagnes et aux collines, aux cours des rivières, et aux vallées : Me voici, je vais faire venir l'épée sur vous, et je détruirai vos hauts lieux\FTNT{Lé. 26:30.}.
\VS{4}Vos autels seront dévastés, vos autels d'encens seront brisés, et je ferai tomber vos morts devant vos idoles.
\VS{5}Car je mettrai les cadavres des fils d'Israël devant leurs idoles, et je disperserai vos os autour de vos autels\FTNT{2 R. 23:14-20.}.
\VS{6}Les villes seront désertes, là où sont vos demeures, et les hauts lieux seront dévastés, vos autels seront délaissés et abandonnés, et vos idoles seront brisées et ne seront plus ; vos autels d'encens abattus, et vos ouvrages seront nettoyés.
\VS{7}Les tués tomberont parmi vous ; et vous saurez que je suis Yahweh.
\VS{8}Mais je laisserai quelques restes d'entre vous, afin que vous ayez quelques réchappés de l'épée parmi les nations, quand vous serez dispersés parmi les pays.
\VS{9}Vos réchappés se souviendront de moi\FTNT{Jé. 51:50.} parmi les nations où ils seront captifs, parce que j'aurai brisé leur cœur adonné à la fornication, qui s'est détourné de moi, et à cause de leurs yeux qui se sont livrés à la prostitution après leurs idoles ; ils se prendront eux-mêmes en dégoût, à cause du mal qu'ils ont commis, à cause de leurs abominations.
\VS{10}Ils sauront que je suis Yahweh, que ce n'est point en vain que je les ai menacés.
\TextTitle{Sentence envers les idolâtres}
\VS{11}Ainsi parle le Seigneur Yahweh : Frappe de ta main et bats de ton pied, et dis : Hélas ! A cause de toutes les abominations, des maux de la maison d'Israël ; car ils tomberont par l'épée, par la famine, et par la peste.
\VS{12}Celui qui sera loin mourra de la peste, et celui qui sera près tombera par l'épée ; et celui qui restera et sera assiégé, mourra par la famine, ainsi je consumerai ma fureur sur eux\FTNT{Am. 4:10.}.
\VS{13}Vous saurez que je suis Yahweh quand les blessés à morts seront au milieu de leurs idoles, autour de leurs autels, sur toute colline élevée, sur tous les sommets des montagnes, sous tout arbre vert, et sous tout chêne touffu, là où ils offraient des parfums de bonne odeur à toutes leurs idoles\FTNT{Os. 4:13.}.
\VS{14}J'étendrai donc ma main sur eux, et je rendrai leur pays désolé et désert dans toutes leurs demeures, plus que le désert qui est vers Dibla. Et ils sauront que je suis Yahweh.
\Chap{7}
\TextTitle{Attaque babylonienne imminente}
\VerseOne{}Puis la parole de Yahweh me fut adressée, en disant :
\VS{2}Et toi, fils de l'homme, écoute : Ainsi parle le Seigneur Yahweh à la terre d'Israël : La fin, la fin vient sur les quatre coins de la terre !
\VS{3}Maintenant la fin vient sur toi, j'enverrai sur toi ma colère, et je te jugerai selon ta voie, et je mettrai sur toi toutes tes abominations\FTNT{Ro. 2:6.}.
\VS{4}Et mon œil ne t'épargnera point, et je n'aurai point de compassion ; mais je te chargerai de tes voies, et tes abominations seront au milieu de toi ; et vous saurez que je suis Yahweh.
\VS{5}Ainsi parle le Seigneur Yahweh : Voici un mal, un seul mal qui vient !
\VS{6}La fin vient, la fin vient, elle se réveille contre toi ; voici, le mal vient !
\VS{7}Ton tour arrive, habitant du pays ! Le temps vient, le jour est près de toi, il ne sera que frayeur, et non pas une invitation des montagnes\FTNT{So. 1:14-15.} à s'entre-réjouir.
\VS{8}Maintenant, je répandrai bientôt ma fureur sur toi, et je consumerai ma colère sur toi ; je te jugerai selon ta voie, je mettrai sur toi toutes tes abominations.
\VS{9}Mon œil ne t'épargnera point, et je n'aurai point de compassion, je te punirai selon ta voie, et tes abominations seront au milieu de toi ; et vous saurez que je suis Yahweh qui frappe.
\VS{10}Voici le jour, voici il vient, le matin paraît, la verge fleurit, l'orgueil bourgeonne.
\VS{11}La violence s'élève pour servir de verge à la méchanceté ; il ne restera rien d'eux, ni de leur multitude, ni de leur tumulte, et on ne se lamentera point sur eux.
\VS{12}Le temps vient, le jour est tout proche : Que celui donc qui achète ne se réjouisse point, et que celui qui vend ne se lamente point ; car il y a une ardente colère sur toute leur multitude.
\VS{13}Car le vendeur ne recouvrera pas ce qu'il a vendu, serait-il encore parmi les vivants ; car la vision touchant toute leur multitude ne sera point révoquée ; et à cause de son iniquité, nul ne conservera sa vie.
\VS{14}On sonne de la trompette, tout est prêt, mais il n'y a personne pour aller au combat, parce que l'ardeur de ma colère est sur toute leur multitude.
\VS{15}L'épée est au-dehors, la peste et la famine au-dedans ! Celui qui est aux champs mourra par l'épée ; et celui qui est dans la ville, la famine et la peste le dévoreront.
\VS{16}Les réchappés s'enfuiront et seront sur les montagnes comme les pigeons des vallées, tous gémissant, chacun sur son iniquité.
\VS{17}Toutes les mains deviendront lâches, et tous les genoux se fondront en eau\FTNT{Es. 13:7 ; Jé. 6:24.}.
\VS{18}Ils se ceindront de sacs, et le tremblement les couvrira, la confusion sera sur tous leurs visages, et leurs têtes deviendront chauves\FTNT{Es. 3:24 ; Jé. 48:37 ; Am. 8:10.}.
\VS{19}Ils jetteront leur argent par les rues, et leur or s'en ira au loin ; leur argent et leur or ne pourront pas les délivrer au jour de la grande colère de Yahweh\FTNT{Pr. 11:4 ; So. 1:18.} ; ils ne rassasieront point leurs âmes, et ne rempliront point leurs entrailles, parce que leur iniquité aura été leur ruine.
\TextTitle{Violation du temple}
\VS{20}Ils étaient fiers de leur magnifique parure ; mais ils y ont placé des images de leurs abominations et de leurs infamies, c'est pourquoi je la rendrai pour eux un objet d'horreur.
\VS{21}Je l'ai livrée au pillage dans la main des étrangers, et en proie aux méchants de la terre qui la profaneront\FTNT{Jé. 20:5.}.
\VS{22}Je détournerai aussi ma face d'eux, et on violera mon lieu secret, et des furieux entreront et le profaneront.
\VS{23}Fais une chaîne ! Car le pays est plein de crimes, de meurtre, et la ville est pleine de violence.
\VS{24}C'est pourquoi je ferai venir les plus méchants des nations, qui possèderont leurs maisons, et je ferai cesser l'orgueil des puissants, et leurs saints lieux seront profanés.
\VS{25}La destruction vient, et ils chercheront la paix, mais il n'y en aura point.
\VS{26}Il viendra malheur sur malheur, et il y aura rumeur sur rumeur ; ils demanderont la vision aux prophètes\FTNT{La. 2:9.} ; la loi périra chez le sacrificateur, et le conseil chez les anciens.
\VS{27}Le roi se lamentera, les princes se vêtiront de désolation, et les mains du peuple du pays tomberont de frayeur. Je les traiterai selon leur voie, je les jugerai comme ils le méritent et ils sauront que je suis Yahweh.
\Chap{8}
\TextTitle{Visions divines}
\VerseOne{}Puis il arriva dans la sixième année, au cinquième jour du sixième mois, comme j'étais assis dans ma maison, et que les anciens de Juda étaient assis devant moi, que la main du Seigneur Yahweh tomba là sur moi.
\VS{2}Je regardai, et voici c'était une figure ayant l'aspect d'un feu qui frappe les regards ; depuis ses reins jusqu'en bas c'était du feu, et depuis ses reins jusqu'en haut, c'était d'un aspect brillant comme de l'airain poli.
\VS{3}Il étendit une forme de main et me prit par les cheveux de ma tête. L'Esprit m'enleva entre la terre et le ciel et me transporta à Jérusalem, dans des visions de Dieu, à l'entrée de la porte intérieure, du côté nord, où était posée l'idole de jalousie\FTNT{L'idole de la jalousie : Dans le temple de Jérusalem à l'époque d'Ezéchiel, l'idolâtrie s'y développait sans retenue (2 R. 21, 22 et 23). Il y avait dans ce temple les idoles d'Astarté et les autels de Baal. Le temple était souillé.} qui provoque la jalousie.
\VS{4}Voici, la gloire du Dieu d'Israël était là, telle que je l'avais vue en vision dans la vallée.
\TextTitle{Abominations dans le temple}
\VS{5}Il me dit : Fils de l'homme, lève maintenant tes yeux vers le chemin qui tend vers le nord ! J'élevai mes yeux vers le chemin qui tend vers le nord, et voici du côté nord, à la porte de l'autel, était cette idole de jalousie, à l'entrée.
\VS{6}Il me dit : Fils de l'homme, ne vois-tu pas ce qu'ils font, les grandes abominations que la maison d'Israël commet ici, pour que je me retire de mon lieu saint ? Mais tourne-toi encore, tu verras de grandes abominations.
\VS{7}Il me conduisit donc à l'entrée du parvis. Je regardai, et voici, il y avait un trou dans le mur.
\VS{8}Il me dit : Fils de l'homme, perce maintenant le mur ; et quand je perçai le mur, il y avait une porte.
\VS{9}Puis il me dit : Entre et regarde les méchantes abominations qu'ils commettent ici.
\VS{10}J'entrai donc et je regardai ; et voici, toutes sortes de figures de reptiles et de bêtes abominables, et toutes les idoles de la maison d'Israël étaient peintes sur le mur tout autour\FTNT{Ex. 20:4 ; De. 4:16-18 ; Ro. 1:23.}.
\VS{11}Soixante-dix hommes des anciens de la maison d'Israël, au milieu desquels était Jaazania, fils de Schaphan, se tenaient debout devant ces idoles, chacun l'encensoir à la main, d'où s'élevait une épaisse nuée d'encens.
\VS{12}Alors il me dit : Fils de l'homme, n'as-tu pas vu ce que les anciens de la maison d'Israël font dans les ténèbres, chacun dans sa chambre pleine de figures ? Car ils disent : Yahweh ne nous voit point, Yahweh a abandonné le pays\FTNT{Es. 29:15.}.
\VS{13}Puis il me dit : Tourne-toi encore, et tu verras les grandes abominations qu'ils commettent.
\VS{14}Il me conduisit donc à l'entrée de la porte de la maison de Yahweh qui est vers le nord. Et voici, il y avait là des femmes assises qui pleuraient Thammuz\FTNT{Thammuz ou Adonis.}.
\VS{15}Il me dit : Fils de l'homme, n'as-tu pas vu ? Tourne-toi encore, et tu verras des abominations plus grandes que celles-ci.
\VS{16}Il me fit donc entrer dans le parvis intérieur de la maison de Yahweh. Et voici, à l'entrée du temple de Yahweh, entre le portique et l'autel, environ vingt-cinq hommes avaient le dos tourné contre le temple de Yahweh, leurs visages tournés vers l'orient ; et ils se prosternaient vers l'orient, devant le soleil\FTNT{De. 4:19.}.
\VS{17}Alors il me dit : Fils de l'homme, n'as-tu pas vu ? Est-ce une chose légère à la maison de Juda de commettre ces abominations qu'ils commettent ici ? Car ils ont rempli le pays de violence, et ils se sont ainsi tournés pour m'irriter ; mais voici ils approchent le rameau de leurs nez.
\VS{18}Et moi, j'agirai dans ma fureur ; mon œil ne les épargnera point, et je n'aurai point de compassion ; quand ils crieront à haute voix à mes oreilles, je ne les exaucerai point\FTNT{Pr. 1:28 ; Es. 1:15 ; Jé. 11:11 ; Mi. 3:4 ; Za. 7:13.}.
\Chap{9}
\TextTitle{Marque de Yahweh sur les justes ; extermination des impies}
\VerseOne{}Puis il cria d'une voix forte à mes oreilles : Faites approcher ceux qui châtient la ville, chacun avec son instrument de destruction à la main !
\VS{2}Et voici, six hommes venaient par le chemin de la haute porte qui regarde vers le nord, et chacun avait dans sa main son instrument de destruction. Il y avait au milieu d'eux un homme vêtu de lin, qui avait une écritoire sur ses reins ; ils entrèrent et se tinrent près de l'autel d'airain.
\VS{3}Alors la gloire du Dieu d'Israël s'éleva du chérubin sur lequel elle était, et vint sur le seuil de la maison. Il cria à l'homme qui était vêtu de lin et qui avait l'écritoire sur ses reins.
\VS{4}Yahweh lui dit : Passe par le milieu de la ville, par le milieu de Jérusalem, et marque la lettre Thau sur les fronts des hommes qui gémissent et qui soupirent à cause de toutes les abominations qui s'y commettent\FTNT{Ap. 7:3 ; Ap. 9:4 ; Ap. 13:16-17 ; Ap. 20:4 ; Ex. 12:7-23.}.
\VS{5}Et s'adressant aux autres en ma présence, il dit : Passez dans la ville après lui, et frappez ; que votre oeil soit sans pitié et n'ayez point de compassion !
\VS{6}Tuez-les tous, les vieillards, les jeunes gens, les vierges, les enfants et les femmes\FTNT{2 Ch. 36:17.} ; mais n'approchez pas de ceux qui ont la lettre Thau\FTNT{La lettre Thau ou Tav est la marque, le signe, le symbole ou le sceau Divin. La lettre Tav est formée par la réunion des lettres Daleth et Nun. Ces deux lettres forment le mot « dan » qui veut dire « juge ». Selon la Bible, la marque des chrétiens est représentée par : Le Saint-Esprit, le nom de Jésus-Christ (Ep. 1:13-14 ; Ep. 4:30 ; Ap. 14:1), le nom de la Nouvelle Jérusalem (Ap. 3 : 12) et le nom du Père. Les chrétiens fidèles à Dieu sont marqués par l'Esprit de Dieu qui est notre sceau. Le Saint-Esprit est saint, la sainteté est donc la marque des chrétiens (1 Pi. 1:2). Il est aussi l'Esprit de vérité, donc la vérité est aussi la marque des chrétiens (Jn. 16:13). Il est aussi amour, l'amour étant également la marque distinctive des véritables chrétiens (Ro. 5:5).}, et commencez par mon lieu saint\FTNT{Le jugement commence par la maison de Dieu (1 Pi. 4:17-18).}. Ils commencèrent donc par les vieillards qui étaient devant la maison.
\VS{7}Il leur dit : Profanez la maison, et remplissez de morts les parvis !… Sortez !… Et ils sortirent, et ils frappèrent dans la ville.
\VS{8}Or il arriva que comme ils frappaient, je restai là, et m'étant prosterné le visage contre terre, je criai et dis : Ah ! Seigneur Yahweh ! Vas-tu donc détruire tous les restes d'Israël en répandant ta fureur sur Jérusalem ?
\VS{9}Il me dit : L'iniquité de la maison d'Israël et de Juda est excessivement grande, le pays est rempli de meurtres et la ville remplie de crimes ; car ils ont dit : Yahweh a abandonné le pays, Yahweh ne nous voit point.
\VS{10}Quant à moi, mon oeil aussi ne les épargnera point, et je n'en aurai point compassion ; je mettrai leur voie sur leur tête.
\VS{11}Et voici, l'homme vêtu de lin, qui avait une écritoire sur ses reins, rapporta ce qui avait été fait, et il dit : J'ai fait comme tu m'as ordonné.
\Chap{10}
\TextTitle{La gloire de Yahweh quitte le temple}
\VerseOne{}Je regardai, et voici, sur l'étendue qui était au-dessus de la tête des chérubins, parut comme une pierre de saphir ; on voyait au-dessus d'eux quelque chose de semblable à un trône.
\VS{2}On parla à l'homme vêtu de lin, et on lui dit : Va entre les roues, sous les chérubins, et remplis tes mains de charbons ardents que tu prendras entre les chérubins, et répands-les sur la ville\FTNT{Es. 6:6 ; Ap. 8:5.} ; il y entra devant mes yeux.
\VS{3}Les chérubins étaient à la droite de la maison quand l'homme entra ; et une nuée remplit le parvis intérieur\FTNT{1 R. 8:10-11.}.
\VS{4}Puis la gloire de Yahweh s'éleva de dessus les chérubins pour venir sur le seuil de la maison, et la maison fut remplie d'une nuée, et le parvis fut rempli de la splendeur de la gloire de Yahweh.
\VS{5}On entendit le bruit des ailes des chérubins jusqu'au parvis extérieur, pareil à la voix du Dieu Tout-Puissant lorsqu'il parle.
\VS{6}Ainsi Yahweh donna cet ordre à l'homme qui était vêtu de lin : Prends du feu d'entre les roues des chérubins ; il entra et se tint auprès des roues.
\VS{7}L'un des chérubins étendit sa main entre les chérubins, vers le feu qui était entre les chérubins ; il en prit et le mit entre les mains de l'homme vêtu de lin. Et cet homme le prit et sortit.
\VS{8}On voyait aux chérubins une forme de main d'homme sous leurs ailes.
\VS{9}Puis je regardai, et voici, il y avait quatre roues près des chérubins, une roue près de chaque chérubin ; et ces roues avaient l'aspect d'une pierre de chrysolithe.
\VS{10}A leur aspect, toutes les quatre avaient la même forme ; chaque roue paraissait être au milieu d'une autre roue.
\VS{11}Quand elles marchaient, elles allaient de leurs quatre côtés, et elles ne se tournaient point dans leur marche ; mais elles allaient dans la direction de la tête, sans se tourner dans leur marche.
\VS{12}Tout le corps des chérubins, leur dos, leurs mains, leurs ailes, étaient remplis d'yeux, aussi bien que les roues tout autour, les quatre roues\FTNT{Ap. 4:6-8.}.
\VS{13}J'entendis qu'on appela les roues tourbillon.
\VS{14}Chaque animal avait quatre faces : La première face était la face d'un chérubin ; la seconde face était la face d'un homme ; la troisième était la face d'un lion ; et la quatrième la face d'un aigle\FTNT{Ez. 1 ; Ap. 4:7.}.
\VS{15}Puis les chérubins s'élevèrent. Ce sont là les animaux que j'avais vus près du fleuve de Kebar.
\VS{16}Lorsque les chérubins marchaient, les roues aussi marchaient à côté d'eux ; et quand les chérubins élevaient leurs ailes pour s'élever de terre, les roues ne se détournaient point d'eux.
\VS{17}Lorsqu'ils s'arrêtaient, elles s'arrêtaient ; et lorsqu'ils s'élevaient, elles s'élevaient ; car l'esprit des animaux était dans les roues.
\VS{18}Puis la gloire de Yahweh se retira de dessus le seuil de la maison, et se tint au-dessus des chérubins.
\VS{19}Les chérubins élevant leurs ailes, s'élevèrent de terre sous mes yeux quand ils partirent ; les roues s'élevèrent aussi. Et chacun d'eux s'arrêta à l'entrée de la porte orientale de la maison de Yahweh ; la gloire du Dieu d'Israël était sur eux en haut.
\VS{20}C'étaient les animaux que j'avais vus sous le Dieu d'Israël près du fleuve de Kebar ; et je reconnus que c'étaient des chérubins.
\VS{21}Chacun avait quatre faces, et chacun quatre ailes, une forme de main d'homme était sous leurs ailes.
\VS{22}Quant à l'aspect de leurs faces, c'étaient les faces que j'avais vues près du fleuve de Kebar, c'était le même aspect, c'étaient eux-mêmes. Et chacun marchait droit devant soi.
\Chap{11}
\TextTitle{Sentences sur les princes infidèles}
\VerseOne{}Puis l'Esprit m'enleva et me transporta à la porte orientale de la maison de Yahweh, à celle qui regarde vers l'orient. Et il y avait vingt-cinq hommes à l'entrée de la porte, et je vis au milieu d'eux Jaazania, fils d'Azzur, et Pelathia, fils de Benaja, les princes du peuple.
\VS{2}Il me dit : Fils de l'homme, ce sont les hommes qui ont des pensées d'iniquité, et qui donnent un mauvais conseil dans cette ville\FTNT{Mi. 2:1.}.
\VS{3}Ils disent : Ce n'est pas le moment ! Bâtissons des maisons ! La ville est la chaudière et nous sommes la viande.
\VS{4}C'est pourquoi prophétise contre eux, prophétise, fils de l'homme !
\VS{5}L'Esprit de Yahweh tomba sur moi. Et il me dit : Ainsi parle Yahweh : Vous parlez de la sorte, maison d'Israël, et je connais toutes les pensées de votre esprit.
\VS{6}Vous avez multiplié les meurtres dans cette ville, et vous avez rempli ses rues de gens que vous avez tués.
\VS{7}C'est pourquoi, ainsi parle le Seigneur Yahweh : Les gens que vous avez tués, et que vous avez mis au milieu d'elle, sont la viande, et elle est la chaudière, mais je vous tirerai hors du milieu d'elle\FTNT{Mi. 3:3.}.
\VS{8}Vous avez eu peur de l'épée, mais je ferai venir l'épée sur vous, dit le Seigneur Yahweh\FTNT{Jé. 42:16.}.
\VS{9}Je vous tirerai hors de la ville, je vous livrerai entre les mains des étrangers, et j'exécuterai mes jugements contre vous.
\VS{10}Vous tomberez par l'épée ; je vous jugerai dans le pays d'Israël, et vous saurez que je suis Yahweh.
\VS{11}Elle ne sera point une chaudière pour vous, et vous ne serez point au dedans d'elle comme la viande ; je vous jugerai dans le pays d'Israël.
\VS{12}Et vous saurez que je suis Yahweh ; car vous n'avez point suivi mes ordonnances, et vous n'avez pas observé mes lois, mais vous avez agi selon les ordonnances des nations qui sont autour de vous.
\VS{13}Or il arriva comme je prophétisais, que Pelathia, fils de Benaja, mourut. Alors je me prosternai sur mon visage, je criai à haute voix, et dis : Ah ! Seigneur Yahweh ! Vas-tu consumer entièrement le reste d'Israël ?
\TextTitle{Restauration d'Israël et de ses exilés}
\VS{14}La parole de Yahweh me fut adressée, en disant :
\VS{15}Fils de l'homme, tes frères, tes frères, les hommes de ta parenté, et la maison d'Israël tout entière, à qui les habitants de Jérusalem ont dit : Eloignez-vous de Yahweh, la terre nous a été donnée en héritage.
\VS{16}C'est pourquoi dis-leur : Ainsi parle le Seigneur Yahweh : Quoique je les aie éloignés des nations, et que je les aie dispersés dans divers pays, je serai pour eux quelque temps un lieu saint\FTNT{Alors que le lieu saint ou maison terrestre était souillée, Yahweh se présente comme le Lieu Sacré pour son peuple.} dans les pays où ils sont venus.
\VS{17}C'est pourquoi dis-leur : Ainsi parle le Seigneur Yahweh : Je vous rassemblerai du milieu des peuples, et je vous recueillerai des pays auxquels vous avez été dispersés, et je vous donnerai la terre d'Israël\FTNT{Es. 11:11-16 ; Jé. 24:6 ; Ez. 28:25 ; Ez. 34:13 ; Ez. 36:24.}.
\VS{18}C'est là qu'ils iront, et ils ôteront hors d'elle toutes ses infamies et toutes ses abominations.
\VS{19}Je leur donnerai un même cœur, et je mettrai en eux un esprit nouveau ; j'ôterai de leur corps le cœur de pierre, et je leur donnerai un cœur de chair\FTNT{Il s'agit d'une allusion à la nouvelle alliance (Jé. 31:31-34 ; Hé. 8).},
\VS{20}afin qu'ils suivent mes ordonnances, et qu'ils gardent et observent mes lois ; ils seront mon peuple, et je serai leur Dieu.
\VS{21}Quant à ceux dont le cœur se plaît à leurs idoles et à leurs abominations, quant à ceux-là, je ferai tomber sur leur tête les peines que mérite leur conduite, dit le Seigneur Yahweh.
\TextTitle{La gloire de Dieu en mouvement vers le Mont des Oliviers\FTNTT{Cp. Ez. 43:1-4.}}
\VS{22}Puis les chérubins élevèrent leurs ailes, accompagnés des roues ; et la gloire du Dieu d'Israël était sur eux, en haut.
\VS{23}La gloire de Yahweh s'éleva du milieu de la ville\FTNT{Le départ de la gloire de Dieu du temple de Jérusalem marque la fin de la théocratie (règne de Dieu) en Israël. Cet événement, comparable au retrait de l'Esprit de Dieu en Ge. 6:3, fut consécutif à la décadence morale d'Israël (voir Ez. 8) qui fut désormais livré aux nations. Certains estiment que la théocratie a cessé au moment où les israélites ont demandé un roi (voir 1 S. 8). Or bien que cette demande déplut à Yahweh, il continua néanmoins à diriger Israël au travers des souverains tels que David, qu'il établissait à la tête de son peuple. Les Hébreux avaient déjà reçu un sérieux avertissement avec la destruction du temple lors de la première déportation babylonienne (2 R. 24). Cet événement, bien que traumatisant pour beaucoup, n'avait cependant pas provoqué une réelle repentance, c'est pourquoi les israélites retombèrent rapidement dans leurs travers. Ainsi, comme en témoigne Mal. 2 :17 qui rapporte les propos de certains juifs : « Où est le Dieu de la justice ? », en dépit de la reconstruction du temple sous Néhémie et Esdras, la gloire de Dieu ne s'y manifestait plus depuis longtemps. Ezéchiel ne fait donc qu'assister à la conséquence de plusieurs siècles d'infidélité des juifs à l'égard de leur Dieu.}, et s'arrêta sur la montagne qui est à l'orient de la ville.
\VS{24}Puis l'Esprit m'enleva et me transporta en Chaldée, vers ceux qui avaient été emmenés captifs, le tout en vision par l'Esprit de Dieu ; et la vision que j'avais vue disparut au-dessus de moi.
\VS{25}Alors je dis à ceux qui avaient été emmenés captifs toutes les paroles que Yahweh m'avait révélées.
\Chap{12}
\TextTitle{Fuite d'Ezéchiel, un signe pour Israël}
\VerseOne{}La parole de Yahweh me fut encore adressée en ces mots :
\VS{2}Fils de l'homme : Tu habites au milieu d'une maison rebelle, au milieu de gens qui ont des yeux pour voir, et ne voient point ; et qui ont des oreilles pour entendre, et n'entendent point ; parce qu'ils sont une maison de rebelles\FTNT{Es. 6:9 ; Es. 49:19-20 ; Jé. 5:21 ; Ac. 28:26.}.
\VS{3}Toi donc fils d’homme, fais-toi des bagages d’un homme qui s'exile et pars en exil de jour, sous leurs yeux, pars en exil, dis-je de ton lieu pour aller dans un autre lieu, sous leurs yeux. Peut-être qu'ils y prendront garde, quoi qu’ils soient une maison rebelle.
\VS{4}Tu mettras donc dehors pendant le jour tes bagages comme les bagages d'un homme qui s'exile sous leurs yeux, et le soir, tu sortiras sous leurs yeux, comme quand on sort pour s'exiler.
\VS{5}Perce-toi le mur sous leurs yeux et sors par là tes bagages.
\VS{6}Tu les porteras sur tes épaules, sous leurs yeux, et tu sortiras tes bagages pendant l'obscurité. Tu couvriras aussi ton visage, afin que tu ne voies point la terre ; car je t'ai mis pour être un signe pour la maison d'Israël.
\VS{7}Je fis donc ce qui m'avait été ordonné : Je portai dehors pendant le jour mes bagages comme des bagages d'exil ; le soir je perçai le mur avec la main et je les sortis pendant l'obscurité, je les portai sur l'épaule, sous leurs yeux.
\VS{8}Au matin, la parole de Yahweh me fut adressée en ces mots :
\VS{9}Fils de l'homme, la maison d'Israël, maison rebelles, ne t'a-t-elle pas dit : Qu'est-ce que tu fais ?
\VS{10}Dis-leur : Ainsi parle le Seigneur Yahweh : Cet ordre dont je suis chargé s'adresse au prince qui est à Jérusalem et à toute la maison d'Israël qui s'y trouve.
\VS{11}Dis : Je suis pour vous un signe ; comme j'ai fait, ainsi il leur sera fait ; ils iront en exil, en captivité.
\VS{12}Et le prince qui est parmi eux, mettra son bagage sur l'épaule et sortira ; on percera le mur pour le tirer dehors ; il couvrira son visage, afin qu'il ne voie point de ses yeux la terre\FTNT{2 R. 25:4.}.
\VS{13}J'étendrai mon rets sur lui, et il sera pris dans mes filets ; je le ferai entrer dans Babylone, au pays des Chaldéens, mais il ne la verra point, et il y mourra.
\VS{14}Je disperserai à tout vent tout ce qui est autour de lui, son secours, et tous ses corps d'armées ; et je tirerai l'épée sur eux.
\VS{15}Ils sauront que je suis Yahweh, quand je les aurai répandus parmi les nations, et que je les aurai dispersés dans divers pays.
\VS{16}Je laisserai un reste d'entre eux, quelques hommes, préservés de l'épée, de la famine, et de la peste, afin qu'ils racontent toutes leurs abominations parmi les nations où ils iront ; et ils sauront que je suis Yahweh.
\TextTitle{La captivité du peuple imminente\FTNTT{Cp. 2 R. 25:1-10.}}
\VS{17}Puis la parole de Yahweh me fut adressée en ces mots :
\VS{18}Fils de l'homme, mange ton pain dans l'agitation, et bois ton eau en tremblant et avec inquiétude.
\VS{19}Puis tu diras au peuple du pays : Ainsi parle le Seigneur Yahweh, sur les habitants de Jérusalem, à la terre d'Israël : Ils mangeront leur pain avec chagrin, et ils boiront leur eau avec frayeur, parce que son pays sera désolé, étant privé de son abondance, à cause de la violence de tous ceux qui y habitent.
\VS{20}Les villes peuplées seront désertes, et le pays ne sera que désolation ; et vous saurez que je suis Yahweh.
\VS{21}La parole de Yahweh me fut encore adressée en ces mots :
\VS{22}Fils de l'homme, que signifient ces discours moqueurs que vous tenez sur la terre d'Israël, en disant : Les jours seront prolongés, et toute vision périra\FTNT{Es. 5:19 ; Am. 6:3 ; 2 Pi. 3:3.} ?
\VS{23}C'est pourquoi dis-leur : Ainsi parle le Seigneur Yahweh : Je ferai cesser ce proverbe, et on ne s'en servira plus comme proverbe en Israël ; et dis-leur : Les jours approchent, et toutes les visions s'accompliront.
\VS{24}Car il n'y aura plus désormais aucune vision de vanité ni aucune divination de flatteur, au milieu de la maison d'Israël.
\VS{25}Car moi, Yahweh, je parlerai, et la parole que j'aurai prononcée sera mis en exécution, elle ne sera plus différée ; mais ô maison rebelle! Je prononcerai en vos jours la parole, et je l'exécuterai dit le Seigneur Yahweh.
\VS{26}La parole de Yahweh me fut encore adressée en ces mots :
\VS{27}Fils de l'homme, voici, ceux de la maison d'Israël disent : La vision que celui-ci voit n'arrivera pas avant longtemps, et il prophétise pour des temps qui sont encore éloignés.
\VS{28}C'est pourquoi dis-leur : Ainsi parle le Seigneur Yahweh : Aucune de mes paroles ne sera plus différée, mais la parole que j'aurai prononcée sera exécutée incessament, dit le Seigneur Yahweh.
\Chap{13}
\TextTitle{Jugement sur ceux qui égarent le peuple de Dieu}
\VerseOne{}La parole de Yahweh me fut encore adressée en ces mots :
\VS{2}Fils de l'homme, prophétise contre les prophètes d'Israël qui prophétisent, et dis à ces prophètes qui prophétisent selon leur propre cœur : Ecoutez la parole de Yahweh !
\VS{3}Ainsi parle le Seigneur Yahweh : Malheur aux prophètes insensés qui suivent leur propre esprit, et qui n'ont point eu de vision.
\VS{4}Israël, tes prophètes ont été comme des renards dans les déserts.
\VS{5}Vous n'êtes point montés devant les brèches, et vous n'avez point réparé les murs pour la maison d'Israël, afin de vous tenir debout pour le combat au jour de Yahweh.
\VS{6}Ils ont eu des visions vaines et des divinations de mensonge, ils disent : Yahweh a dit ; et toutefois Yahweh ne les a point envoyés ; et ils font espérer que leur parole s'accomplira\FTNT{Les faux prophètes (Jé. 23) ; Jé. 14:14 ; Jé. 28:15.}.
\VS{7}N'avez-vous pas vu des visions de vanité, et prononcé des divinations de mensonge ? Cependant vous dites : Yahweh a parlé ; et je n'ai point parlé.
\VS{8}C'est pourquoi ainsi parle le Seigneur Yahweh : Parce que vous avez prononcé des choses vaines, et que vous avez eu des visions de mensonge, à cause de cela j'en veux à vous, dit le Seigneur Yahweh.
\VS{9}Et ma main sera sur les prophètes qui ont des visions de vanité et des divinations de mensonge ; ils ne seront plus admis dans le conseil de mon peuple, ils ne seront plus écrits dans les registres de la maison d'Israël, ils n'entreront plus dans la terre d'Israël ; et vous saurez que je suis le Seigneur Yahweh.
\VS{10}Parce, oui parce qu'ils ont abusé mon peuple, en disant : Paix ! Et il n'y avait point de paix\FTNT{Jé. 6:14 ; Jé. 8:11.}. L'un bâtissait le mur, et les autres l'induissaient de mortier mal lié.
\VS{11}Dis à ceux qui enduisent le mur de mortier mal lié, qu'il tombera ; il y aura une pluie débordante, et vous, pierres de grêle, vous tomberez sur lui, et un vent de tempête le fendra.
\VS{12}Et voici, le mur est tombé ; ne vous sera-t-il donc pas dit : Où est l'enduit dont vous l'avez couvert ?
\VS{13}C'est pourquoi, ainsi parle le Seigneur Yahweh : Je ferai dans ma fureur éclater un vent impétueux, et dans ma colère, il surviendra une pluie débordante et des pierres de grêle dans ma fureur, pour détruire entièrement.
\VS{14}Je démolirai le mur que vous avez enduit de mortier mal lié, je le jetterai par terre, tellement que son fondement sera découvert, et il tombera ; vous serez consumés au milieu de lui, et vous saurez que je suis Yahweh.
\VS{15}Ainsi j'accomplirai ma colère contre le mur, et contre ceux qui l'enduisent de mortier mal lié ; et je vous dirai : Le mur n'est plus ni ceux qui l'ont enduit ;
\VS{16}à savoir les prophètes d'Israël, qui prophétisent sur Jérusalem et qui voient pour elle des visions de paix ; et néanmoins il n'y a point de paix, dit le Seigneur Yahweh.
\VS{17}Aussi, toi, fils de l'homme, tourne ta face contre les filles de ton peuple qui prophétisent selon leur propre cœur, prophétise contre elles !
\VS{18} Et dis : ainsi parle le Seigneur Yahweh : Malheur à celles qui cousent des coussins\FTNT{Le mot « coussin » vient du terme hébreu « keceth », et signifie : bande, filet, faux phylactères,  tissu utilisé par les fausses prophétesses en Israël pour étayer leurs plans démoniaques de diseuses de bonne aventure. } pour s'accouder le long du bras jusqu'aux mains, et qui font des voiles pour mettre sur la tête des personnes de toute taille, pour séduire les âmes. Séduiriez-vous les âmes de mon peuple\FTNT{Ge. 10:9.} ; et conserveriez-vous vos âmes ?
\VS{19}Et me profaneriez-vous envers mon peuple pour des poignées d'orge et pour des morceaux de pain, en faisant mourir les âmes qui ne devaient point mourir et en faisant vivre les âmes qui ne devaient point vivre, en mentant à mon peuple qui écoute le mensonge ?
\VS{20}C'est pourquoi ainsi parle le Seigneur Yahweh : Voici, j'en veux à vos coussins, par lesquels vous séduisez les âmes pour les faire voler vers vous ; et je déchirerai ces coussins de vos bras, et je ferai échapper les âmes que vous avez attirées afin qu'elles volent vers vous\FTNT{Ap. 18:11-13 ; 1 Co. 6:10 ; 2 Pi. 2:14}.
\VS{21}Je déchirerai aussi vos voiles, et je délivrerai mon peuple d'entre vos mains, et ils ne seront plus entre vos mains pour en faire votre proie ; et vous saurez que je suis Yahweh.
\VS{22}Parce que vous avez affligé sans raison le cœur du juste, quand moi-même je ne l'ai point attristé, et que vous avez renforcé les mains du méchant, afin qu'il ne se détourne point de son mauvais chemin, et que je lui sauve la vie.
\VS{23}C'est pourquoi, vous n'aurez plus aucune vision de vanité ni aucune divination, mais je délivrerai mon peuple d'entre vos mains ; et vous saurez que je suis Yahweh.
\Chap{14}
\TextTitle{Jugement sur les anciens idolâtres}
\VerseOne{}Or quelques-uns des anciens d'Israël vinrent auprès de moi et s'assirent devant moi.
\VS{2}Et la parole de Yahweh me fut adressée en ces mots :
\VS{3}Fils de l'homme, ces gens élèvent leurs idoles dans leurs cœurs, et ils attachent les regards sur ce qui les fait tomber dans l'iniquité. Serais-je consulté par eux sérieusement ?
\VS{4}C'est pourquoi parle-leur et dis-leur : Ainsi parle le Seigneur Yahweh. Quiconque de la maison d'Israël aura élevé ses idoles dans son cœur, et aura mis devant sa face ce qui l'a fait tomber dans son iniquité, s'il vient vers le prophète, je suis Yahweh, je lui répondrai puisqu'il vient avec la multitude de ses idoles,
\VS{5}afin que je saisisse la maison d'Israël par leur propre cœur; car eux tous se sont éloignés de moi par leurs idoles.
\VS{6}C'est pourquoi dis à la maison d'Israël : Ainsi parle le Seigneur Yahweh : Revenez, et détournez-vous de vos idoles, détournez les regards de toutes vos abominations\FTNT{Es. 55:6-7.}.
\VS{7}Car quiconque de la maison d'Israël, ou des étrangers qui séjournent en Israël, qui s'est séparé de moi, qui éleve ses idoles dans son cœur, et attache ses regards sur ce qui l'a fait tomber dans l'iniquité, s'il vient vers le prophète pour me consulter par de lui, je suis Yahweh, on lui répondra tout ce qu'on a à lui répondre.
\VS{8}Je me tournerai contre cet homme\FTNT{Lé. 17:10 ; Lé. 20:3-6 ; Jé. 44:11.}, et je ferai de lui un signe, et un sujet de sarcasme\FTNT{No. 26:10 ; De. 28:37}. Je le retrancherai du milieu de mon peuple ; et vous saurez que je suis Yahweh.
\VS{9}S'il arrive que le prophète soit séduit, et qu'il profère quelque parole, moi, Yahweh, je séduirai ce prophète-là\FTNT{1 R. 22:23 ; Job. 12:16 ; 2 Th. 2:11.} ; et j'étendrai ma main sur lui, et je l'exterminerai du milieu de mon peuple d'Israël.
\VS{10}Et ils porteront la peine de leur iniquité ; la peine de l'iniquité du prophète sera comme la peine de celui qui l'aura interrogé ;
\VS{11}afin que la maison d'Israël ne s'éloigne plus de moi, et qu'ils ne se souillent plus par tous leurs crimes\FTNT{Jé. 31:18-19 ; Hé. 12:11 ; Ja. 1:1-3.} ; alors ils seront mon peuple, et je serai leur Dieu, dit le Seigneur Yahweh.
\TextTitle{Châtiments d'Israël ; Yahweh épargne un reste}
\VS{12}Puis la parole de Yahweh me fut adressée en ces mots :
\VS{13}Fils de l'homme, lorsqu'un pays aura péché contre moi, en commettant une infidélité, et que j'aurai étendu ma main contre lui, et que je lui aurai rompu le bâton du pain, envoyé la famine et retranché du milieu de lui tant les hommes que les bêtes,
\VS{14}et que ces trois hommes, Noé, Daniel et Job s'y trouvent, ils sauveraient leurs âmes par leur justice, dit le Seigneur Yahweh.
\VS{15}Si je fais passer les bêtes féroces par ce pays-là et qu'elles le privent d'enfants, tellement qu'il soit devenu un désert où personne ne passe à cause des bêtes,
\VS{16}et que ces trois hommes-là s'y trouvent, je suis vivant, dit le Seigneur Yahweh, ils ne sauveraient ni fils ni filles, eux seulement seraient sauvés, et le pays sera un désert.
\VS{17}Si je faisais venir l'épée sur ce pays-là et si je disais : Que l'épée passe par le pays, et qu'elle en retranche les hommes et les bêtes !
\VS{18}Si ces trois hommes-là se trouvent au milieu du pays, je suis vivant, dit le Seigneur, ils ne sauveraient ni fils ni filles ; mais eux seulement seraient sauvés.
\VS{19}Ou si j'envoyais la peste dans ce pays, et que je répandais ma colère contre lui jusqu'à faire ruisseler le sang, au point de retrancher du milieu de lui les hommes et les bêtes,
\VS{20}et que Noé\FTNT{Ge. 6:8.}, Daniel\FTNT{Da. 1:8-12.} et Job\FTNT{Job. 1:8.}, s'y trouvent, je suis vivant, dit le Seigneur Yahweh, ils ne sauveraient ni fils ni filles ; mais ils sauveraient leurs âmes par leur justice.
\VS{21}Car ainsi parle le Seigneur Yahweh : J'envoie mes quatre plaies mortelles, l'épée, la famine, les bêtes féroces, et la peste, contre Jérusalem, pour en retrancher les hommes et les bêtes\FTNT{Jé. 15:2-3.} ;
\VS{22}Et toutefois, il y aura un reste qui échappera, qui en sortira, des fils et des filles. Voici, ils viennent vers vous, et vous verrez leur conduite et leurs actions, et vous serez consolés du malheur que je fais venir contre Jérusalem, de tout ce que je fais venir sur elle.
\VS{23}Vous serez consolés, lorsque vous verrez leur conduite et leurs actions ; et vous reconnaîtrez que ce n'est pas sans raison que je fais tout ce que je lui fais, dit le Seigneur Yahweh\FTNT{Jé. 22:8-9.}.
\Chap{15}
\TextTitle{Infidélités d'Israël\FTNTT{Cp. Es. 5:1-24.}}
\VerseOne{}La parole de Yahweh me fut encore adressée, en disant :
\VS{2}Fils de l'homme, que vaut le bois de la vigne de plus que les autres bois ? Et les sarments de plus que les branches des arbres d'une forêt ?
\VS{3}Et prendra-t-on du bois pour en faire quelque ouvrage ? Ou prendra-t-on un clou pour y pendre quelque chose ?
\VS{4}Voici, on le met au feu pour être consumé ; le feu consume aussitôt ses deux bouts, et le milieu est en feu ; serait-il bon pour quelque ouvrage ?
\VS{5}Voici, quand il est entier, on n'en fait aucun ouvrage ; à plus forte raison quand le feu l'aura consumé et qu'il sera brûlé, sera-t-il bon pour quelque ouvrage ?
\VS{6}C'est pourquoi ainsi parle le Seigneur Yahweh : Comme le bois de la vigne est parmi les arbres d'une forêt, que j'ai assigné au feu pour être consumé, ainsi je livrerai les habitants de Jérusalem.
\VS{7}Je me tournerai contre eux, seront-ils sortis du feu ? Encore le feu les consumera ; et vous saurez que je suis Yahweh, quand je tournerai ma face contre eux.
\VS{8}Je ferai de ce pays une désolation, parce qu'ils ont commis une infidélité, dit le Seigneur Yahweh.
\Chap{16}
\TextTitle{Bonté de Yahweh, prostitutions d'Israël}
\VerseOne{}La parole de Yahweh me fut aussi adressée en ces mots :
\VS{2}Fils de l'homme, fais connaître à Jérusalem ses abominations.
\VS{3}Et dis : Ainsi parle le Seigneur Yahweh à Jérusalem : Tu as tiré ton origine et ta naissance du pays de Canaan ; ton père était Amoréen, et ta mère Héthienne.
\VS{4}Quant à ta naissance, le jour où tu naquis, ton cordon ombilical n'a pas été coupé, tu n'as pas été lavée dans l'eau pour être nettoyée ; tu n'as pas été salée de sel ni emmaillotée.
\VS{5}Il n'y a pas eu d'œil qui ait eu pitié de toi pour te faire une seule de ces choses, en ayant compassion pour toi ; mais tu as été jetée sur la face des champs le jour de ta naissance, parce qu'on avait horreur de toi.
\VS{6}Et passant près de toi, je te vis gisante par terre, dans ton sang, et je te dis : Vis dans ton sang ! Et je te redis encore : Vis dans ton sang !
\VS{7}Je t'ai fait croître par millions comme l'herbe des champs. Et tu pris de l'accroissement et tu devins grande, tu parvins à une parfaite bauté, tes seins se formèrent, ta chevelure poussa, tu devins nubile; mais tu étais abandonnée et sans habits.
\VS{8}Je passai près de toi, je te regardai, et voici, le temps était là, le temps des amours. J'étendis sur toi le pan de ma robe, et je couvris ta nudité. Je te jurai, j'entrai en alliance avec toi, dit le Seigneur Yahweh, et tu devins mienne.
\VS{9}Je te lavai dans l'eau en t'y plongeant, j'ôtai le sang de dessus toi, et je t'oignis d'huile.
\VS{10}Je te revêtis de vêtements brodés, je te chaussai de fourrure, je te ceignis de fin lin, et je te couvris de soie.
\VS{11}Je te parai d'ornements : Je mis des bracelets sur tes mains, et un collier à ton cou.
\VS{12}Je mis un anneau à ton nez, des pendants à tes oreilles, et une couronne de gloire sur ta tête.
\VS{13}Tu fus donc parée d'or et d'argent, et ton vêtement était de fin lin, de soie, et de broderie ; tu mangeas la fleur de farine, le miel, et l'huile ; tu devins extrêmement belle, et tu prospéras jusqu'à régner.
\VS{14}Ta renommée se répandit parmi les nations à cause de ta beauté, car elle était parfaite, à cause de ma gloire que j'avais mise sur toi, dit le Seigneur Yahweh.
\VS{15}Mais tu t'es confiée dans ta beauté, et tu t'es prostituée à cause de ta renommée, tu t'es abandonnée à tous les passants\FTNT{Es. 1:21 ; Jé. 2:20 ; Jé. 3:2-6 ; Os. 1:2.}.
\VS{16}Tu as pris tes vêtements pour t'en faire des hauts lieux de diverses couleurs, tels qu'il n'y en a point eu et n'en aura jamais, et tu t'y es prostituée.
\VS{17}Tu as pris ta magnifique parure d'or et d'argent, que je t'avais donnée, et tu t'en es fait des images d'hommes, tu as commis la fornication avec elles.
\VS{18}Tu as pris tes vêtements brodés, tu les en as couvertes, et tu as mis mon huile et mon encens devant elles.
\VS{19}Mon pain que je t'avais donné, la fleur de farine, l'huile, et le miel que je t'avais donné à manger, tu as mis cela devant elle en sacrifice de bonne odeur ; il a été fait ainsi, dit le Seigneur Yahweh.
\VS{20}Tu as aussi pris tes fils et tes filles que tu m'avais enfantés, et tu les as sacrifiés pour être mangés\FTNT{Lé. 18:21 ; Lé. 20:2 ; Es. 57:5 ; Jé. 19:5, Jé. 32:35.}. N'était-ce pas assez de tes prostitutions ?
\VS{21}Tu as égorgé mes fils, et tu les as livrés pour les faire passer par le feu, en l'honneur de ces idoles\FTNT{2 R. 17:17.}.
\VS{22}Et parmi toutes tes abominations et tes adultères, tu ne t'es point souvenue du temps de ta jeunesse, quand tu étais sans habits et toute nue, gisante par terre dans ton sang.
\VS{23}Après toutes tes méchancetés, malheur, malheur à toi ! dit le Seigneur Yahweh.
\VS{24}Tu t'es bâti un lieu éminent, et tu t'es fait des hauts lieux dans toutes les places.
\VS{25}A l'entrée de chaque chemin tu as bâti un haut lieu, et tu as rendu ta beauté abominable, tu t'es prostituée à tous les passants, tu as multiplié tes adultères.
\VS{26}Tu t'es abandonnée aux fils d'Egypte, tes voisins au corps avantageux ; tu as multiplié tes adultères pour m'irriter.
\VS{27}Et voici, j'ai étendu ma main sur toi, j'ai diminué la portion que je t'avais prescrite, et je t'ai abandonnée à la volonté de celles qui te haïssaient, des filles des Philistins, lesquelles ont honte de tes voies qui ne sont que méchanceté.
\VS{28}Tu t'es aussi abandonnée aux fils des Assyriens\FTNT{2 R. 16:7-10 ; Jé. 2:18-36.}, parce que tu n'étais pas encore rassasiée ; et après avoir commis l'adultère avec eux, tu n'as point encore été rassasiée.
\VS{29}Tu as multiplié tes adultères dans le pays de Canaan jusqu'en Chaldée, et avec cela tu n'as pas encore été rassasiée.
\VS{30}Quelle faiblesse de cœur tu as eue, dit le Seigneur, Yahweh, d'avoir fait toutes ces choses-là, qui sont les actions d'une femme qui se prostitue avec arrogance.
\VS{31}De t'être bâti un lieu éminent à chaque entrée de chemin, et d'avoir fait ton haut lieu dans toutes les places. Et tu n'as pas été comme la prostituée, car tu n'as point tenu compte du salaire.
\VS{32}Femme adultère, tu prends des étrangers au lieu de ton mari.
\VS{33}On donne un salaire à toutes les prostituées, mais toi tu as donné à tous tes amants des présents\FTNT{Es. 57:8-9 ; Os. 8:9-10.}, tu les as gagnés par des présents, afin que de toutes parts ils viennent vers toi, pour se plonger avec toi dans le crime.
\VS{34}Tu as été le contraire des autres prostituées, parce qu'on ne te recherchait pas ; et en donnant un salaire au lieu d'en recevoir un, tu as été le contraire des autres.
\TextTitle{Conséquences de l'infidélité de Jérusalem}
\VS{35}C'est pourquoi, ô adultère, écoute la parole de Yahweh !
\VS{36}Ainsi parle le Seigneur, Yahweh : Parce que ton venin s'est répandu, et que dans tes excès tu t'es abandonnée à ceux que tu aimais, à tes abominables idoles, et que tu as mis à mort tes fils que tu leur as donnés ;
\VS{37}à cause de cela, voici, je vais rassembler tous tes amants, avec lesquels tu te plaisais, et tous ceux que tu as aimés, avec tous ceux que tu as haïs ; je les assemblerai de toutes parts contre toi, et je découvrirai ta honte à leurs yeux et ils verront ton infamie.
\VS{38}Et je te jugerai comme on juge les femmes adultères, et celles qui répandent le sang\FTNT{Lé. 20:10 ; De. 22:22-30.} ; je te livrerai pour être mise à mort selon ma fureur et ma jalousie.
\VS{39}Je te livrerai, dis-je, entre leurs mains ; et ils détruiront tes maisons de prostitution, et ils détruiront tes hauts lieux ; ils te dépouilleront de tes vêtements, emporteront ta magnifique parure, et ils te laisseront sans habits et entièrement nue.
\VS{40}Et on fera monter contre toi une foule de gens qui te lapideront de pierres, et qui te perceront avec leurs épées.
\VS{41}Puis ils brûleront tes maisons, et feront justice de toi aux yeux d'un grand nombre de femmes, je ferai cesser tes prostitutions et tu ne donneras plus de salaires.
\VS{42}J'abandonnerai alors ma colère contre toi, et ma jalousie se retirera de toi ; je serai en repos, et je ne m'irriterai plus.
\VS{43}Parce que tu ne t'es point souvenue du temps de ta jeunesse, et que tu m'as provoqué par toutes ces choses-là ; à cause de cela, voici, j'ai fait tomber la peine de tes crimes sur ta tête, dit le Seigneur,Yahweh ; et tu ne feras plus de mauvais projets avec toutes tes abominations.
\VS{44}Voici, tous ceux qui usent de proverbes feront un proverbe de toi, en disant : Telle mère, telle fille !
\VS{45}Tu es la fille de ta mère, qui a dédaigné son mari et ses fils ; et tu es la sœur de chacune de tes sœurs, qui ont dédaigné leurs maris et leurs fils. Votre mère était Héthienne, et votre père était Amoréen.
\VS{46}Ta grande sœur qui demeure à ta gauche, c'est Samarie avec ses filles ; et ta petite sœur qui demeure à ta droite, c'est Sodome avec ses filles.
\VS{47}Et tu n'as pas seulement marché dans leurs voies et fait selon leurs abominations, c'était fort peu ; mais tu t'es corrompue plus qu'elles dans toutes tes voies.
\VS{48}Je suis vivant ! dit le Seigneur Yahweh, Sodome, ta sœur et tes filles, n'ont point fait comme tu as fait, toi et tes filles.
\VS{49}Voici quel a été le crime de Sodome, ta sœur : Elle avait de l'orgueil, elle vivait dans l'abondance de pain, et dans une insouciante tranquillité, elle et ses filles, elle ne fortifiait pas la main du pauvre et de l'indigent.
\VS{50}Elles se sont élevées, elles ont commis des abominations devant moi, et je me suis détourné quand j'ai vu cela.
\VS{51}Quant à Samarie, elle n'a pas commis la moitié de tes péchés ; car tu as multiplié tes abominations plus qu'elle, et tu as justifié tes sœurs par toutes les abominations que tu as commises.
\VS{52}Porte ta honte, toi qui as jugé chacune de tes sœurs, à cause de tes péchés, par lesquels tu as été rendue plus abominable qu'elles ; elles sont plus justes que toi ; c'est pourquoi sois honteuse, et porte ta confusion, vu que tu as justifié tes sœurs.
\VS{53}Quand je ramènerai leurs captifs, les captifs, dis-je, de Sodome et des villes de son ressort; et les captifs de Samarie et de villes de son ressort; je ramenèrai aussi les captifs de la captivité parmi elles!
\VS{54}afin que tu portes ta honte, et que tu sois confuse à cause de tout ce que tu as fait, et que tu les consoles.
\VS{55}Quand ta sœur Sodome, et les villes de son ressort retourneront à leur état précédent ; Samarie et les villes de son ressort retourneront à leur état précédent ; toi aussi, et les villes de ton ressort retournerez à votre état précédent.
\VS{56}Or ta bouche n'a point fait mention de ta sœur, Sodome, au jour de tes fiertés,
\VS{57}avant que ta méchanceté soit découverte ; lorsque tu as reçu les outrages des filles de Syrie, et de tous ses alentours, des filles des Philistins, qui te pillèrent de tous côtés !
\VS{58}Tu portes sur toi tes méchancetés et tes abominations, dit Yahweh.
\VS{59}Car ainsi parle le Seigneur Yahweh : Je te ferai comme tu as fait, quand tu as méprisé le serment en rompant l'alliance.
\TextTitle{Fidélité de Yahweh à son alliance}
\VS{60}Mais je me souviendrai de l'alliance que j'ai traitée avec toi dans les jours de ta jeunesse, et j'établirai avec toi une alliance éternelle\FTNT{Lé. 26:42-45 ; Ps. 106:45.}.
\VS{61}Et tu te souviendras de tes voies, et en seras confuse, lorsque tu recevras tes sœurs, tant tes plus grandes que tes plus petites, et je te les donnerai pour filles ; mais pas selon ton alliance.
\VS{62}Car j'établirai mon alliance avec toi, et tu sauras que je suis Yahweh,
\VS{63}afin que tu te souviennes de ta vie passée, que tu en sois honteuse, et que tu n'ouvres plus la bouche, à cause de ta confusion, après que j'aurai été apaisé envers toi, pour tout ce que tu auras fait, dit le Seigneur Yahweh.
\Chap{17}
\TextTitle{Enigme de Yahweh}
\VerseOne{}La parole de Yahweh me fut adressée en ces mots :
\VS{2}Fils de l'homme, propose une énigme, une parabole à la maison d'Israël.
\VS{3}Tu diras : Ainsi parle le Seigneur Yahweh : Un grand aigle à grandes ailes, aux ailes déployées, couvert de plumes de toutes les couleurs, vint au Liban, et enleva la cime d'un cèdre.
\VS{4}Il arracha la tête de ses rameaux, l'emmena dans un pays de commerce, et la mit dans une ville marchande.
\VS{5}Il prit de la semence du pays, et la mit dans un champ propre à semer, l'apporta près des grosses eaux, la planta comme un saule.
\VS{6}Cette semence poussa et devint un cep de vigne étendu, mais de peu d'élévation ; ses rameaux étaient tournés vers l'aigle, et ses racines étaient sous lui ; il devint une vigne, donna des jets, et produisit des branches.
\VS{7}Mais il y avait un autre grand aigle, aux longues ailes, et au plumage épais. Et voici, cette vigne serra vers lui ses racines, et étendit ses rameaux vers lui, afin qu'il l'arrose des eaux qui coulent des terrasses.
\VS{8}Elle était donc plantée dans une bonne terre, près des grosses eaux, en sorte qu'il y sortait des sarments et portait du fruit\FTNT{Mt. 13:8-23 ; Mc. 4:8-20 ; Lu. 8:8-15.}. Elle était devenue une vigne magnifique.
\VS{9}Dis : Ainsi parle le Seigneur Yahweh, prospèrera-t-elle ? N'arrachera-t-il pas ses racines, et ne coupera-t-il pas ses fruits pour qu'ils deviennent secs ? Tous les sarments qu'il a jetés sécheront, et il ne faudra pas un grand effort et beaucoup de monde pour l'enlever de dessus ses racines.
\VS{10}Mais voici, quoique plantée, prospèrera-t-elle ? Quand le vent d'orient l'aura touchée, ne séchera-t-elle pas entièrement ? Elle séchera sur le terrain où elle était plantée.
\TextTitle{Jugement de Dieu sur Sédécias\FTNTT{2 R. 24:17-20 ; 25:1-10.}}
\VS{11}Puis la parole de Yahweh me fut adressée en ces mots :
\VS{12}Parle maintenant à la maison rebelle : Ne savez-vous pas ce que veulent dire ces choses ? Dis : Voici, le roi de Babylone est venu à Jérusalem. Il a pris le roi, et les princes, et les a emmenés avec lui à Babylone.
\VS{13}Il a pris un de la race royale, il a traité alliance avec lui, il lui a fait prêter serment, et il a retenu les puissants du pays,
\VS{14}afin que le royaume soit tenu dans l'abaissement, et qu'il ne s'élève point, mais qu'en gardant son alliance, il subsiste.
\VS{15}Mais celui-ci s'est rebellé contre lui, envoyant ses messagers en Egypte, pour qu'on lui donne des chevaux et un grand peuple. Celui qui fait de telles choses prospérera-t-il, échappera-t-il ? Ayant enfreint l'alliance, échappera-t-il ?
\VS{16}Je suis vivant, dit le Seigneur Yahweh, c'est dans le pays du roi qui l'a établi pour roi, envers qui il a violé son serment et dont il a rompu l'alliance, c'est près de lui, au milieu de Babylone, qu'il mourra\FTNT{Référence à Nebucadnetsar. Sédécias eut les yeux crevés avant d'être emmené captif (2 R. 25:7 ; Jé. 34:3 ; Jé. 52:11).}.
\VS{17}Pharaon n'ira pas avec une grande armée et un peuple nombreux pour le secourir dans cette guerre, lorsque l'ennemi élèvera des terrasses et fera des retranchements pour exterminer beaucoup d'âmes.
\VS{18}Car il a méprisé le serment en violant l'alliance ; car voici, après avoir donné sa main, il a fait néanmoins toutes ces choses-là ; il n'échappera point !
\VS{19}C'est pourquoi ainsi parle le Seigneur Yahweh : Je suis vivant, si je ne fais tomber sur sa tête mon serment qu'il a méprisé, et mon alliance qu'il a enfreinte.
\VS{20}Et j'étendrai mon rets sur lui, et il sera pris dans mes filets, je le ferai entrer dans Babylone, et là j'entrerai en jugement contre lui pour le crime qu'il a commis contre moi.
\VS{21}Et tous ses fugitifs avec toutes ses troupes tomberont par l'épée, et ceux qui resteront seront dispersés à tout vent ; et vous saurez que moi, Yahweh, j'ai parlé.
\VS{22}Ainsi parle le Seigneur Yahweh : Je prendrai aussi un rameau de la cime de ce haut cèdre, et je le planterai ; je couperai, dis-je, du bout de ses jeunes branches, un tendre rameau, et je le planterai sur une montagne haute et éminente.
\VS{23}Je le planterai sur la haute montagne d'Israël, et là il produira des branches et produira du fruit, et il deviendra un excellent cèdre ; et des oiseaux de tout plumage demeureront sous lui, et habiteront sous l'ombre de ses branches.
\VS{24}Et tous les bois des champs connaîtront que moi, Yahweh, j'aurai abaissé le grand arbre, et élevé le petit arbre, fait sécher le bois vert, et fait reverdir le bois sec ; moi, Yahweh, j'ai parlé, et je le ferai.
\Chap{18}
\TextTitle{Chacun responsable de son péché}
\VerseOne{}La parole de Yahweh me fut encore adressée, en disant :
\VS{2}Que voulez-vous dire, vous qui usez ordinairement de ce proverbe touchant le pays d'Israël, en disant : Les pères ont mangé des raisins verts et les dents des enfants ont été agacées\FTNT{Jé. 31:29 ; La. 5:7.} ?
\VS{3}Je suis vivant, dit le Seigneur Yahweh et vous n'userez plus de ce proverbe en Israël.
\VS{4}Voici, toutes les âmes sont à moi ; l'âme du fils est à moi comme l'âme du père ; l'âme qui pèche sera celle qui mourra.
\VS{5}Mais l'homme qui est juste, et qui pratique la droiture et la justice,
\VS{6}qui ne mange pas sur les montagnes, et qui ne lève pas ses yeux vers les idoles de la maison d'Israël, et qui ne souille pas la femme de son prochain et ne s'approche pas de la femme dans son état d'impureté\FTNT{Lé 18:18 ; Lé. 20:18.},
\VS{7}qui n'opprime personne, qui rend le gage à son débiteur\FTNT{Ex. 22:26 ; De. 24:12-13.}, qui ne ravit pas le bien d'autrui, qui donne son pain à celui qui a faim et qui couvre d'un vêtement celui qui est nu\FTNT{De. 15:11 ; Es. 58:7.},
\VS{8}qui ne prête pas à intérêt, et ne tire pas d'usure, qui détourne sa main de l'iniquité et qui juge selon la vérité entre les parties qui plaident ensemble\FTNT{Ex. 22:25 ; Lé. 25:35-37 ; De. 23:19.},
\VS{9}qui suit mes lois et garde mes ordonnances pour agir avec fidélité, celui-là est juste, certainement il vivra, dit le Seigneur Yahweh.
\VS{10}Et s'il a engendré un fils qui soit un meurtrier, répandant le sang, et commettant des choses semblables ;
\VS{11}et qui ne fasse aucune de ces choses que j'ai ordonnées, s'il mange sur les montagnes, s'il déshonore la femme de son prochain,
\VS{12}s'il opprime le malheureux et le pauvre, s'il ravit le bien d'autrui, s'il ne rend pas le gage, s'il lève ses yeux vers les idoles et commet des abominations,
\VS{13}s'il prête à intérêt, et tire une usure, ce fils-là, vivrait ? Il ne vivra pas, quand il aura commis toutes ces abominations, on le fera mourir, et son sang retombera sur lui.
\VS{14}Mais s'il engendre un fils qui voie tous les péchés que commet son père, qui les voie et n'agisse pas de la même manière ;
\VS{15}S'il ne mange pas sur les montagnes et qu'il ne lève point ses yeux vers les idoles de la maison d'Israël, s'il ne déshonore pas la femme de son prochain,
\VS{16}s'il n'opprime personne, s'il ne prend point de gages, s'il ne ravit point le bien d'autrui, s'il donne de son pain à celui qui a faim et couvre celui qui est nu,
\VS{17}s'il retire sa main du pauvre, s'il n'exige ni usure ni intérêt, s'il garde mes ordonnances, et s'il suit mes lois ; il ne mourra point pour l'iniquité de son père, mais certainement il vivra.
\VS{18}Mais son père, parce qu'il a usé de fraude, et qu'il a ravi ce qui était à son frère, et fait parmi son peuple ce qui n'est pas bon, voici, il mourra pour son iniquité.
\VS{19}Mais, direz-vous : Pourquoi le fils ne porte-t-il pas l'iniquité de son père\FTNT{Ex. 20:5 ; De. 5:9.} ? Parce que le fils a fait ce qui était juste et droit, et qu'il a gardé toutes mes lois et les a observées, certainement il vivra.
\VS{20}L'âme qui pèche est celle qui mourra. Le fils ne portera point l'iniquité du père, et le père ne portera point l'iniquité du fils. La justice du juste sera sur le juste, et la méchanceté du méchant sera sur le méchant.
\VS{21}Si le méchant se détourne de tous ses péchés qu'il aura commis, et qu'il garde toutes mes lois, et fasse ce qui est juste et droit, certainement il vivra, il ne mourra point.
\VS{22}Il ne lui sera point fait mention de tous ses crimes qu'il aura commis, mais il vivra pour sa justice, à laquelle il se sera adonné.
\VS{23}Ce que je désire, est-ce que le méchant meure ? dit le Seigneur Yahweh. N'est-ce pas qu'il se détourne de ses mauvaises voies et qu'il vive ?
\VS{24}Mais si le juste se détourne de sa justice, et commet l'iniquité, selon toutes les abominations que le méchant a l'habitude de commettre, vivra-t-il ? Il ne sera point fait mention de toutes ses justices qu'il aura faites, à cause de son crime qu'il aura commis, et à cause de son péché qu'il aura fait ; il mourra à cause de ces choses-là.
\VS{25}Et vous, vous dites : La voie du Seigneur n'est pas bien réglée. Ecoutez maintenant maison d'Israël, ma voie n'est-elle pas bien réglée ? Ne sont-ce pas plutôt vos voies qui ne sont pas bien réglées ?
\VS{26}Si le juste se détourne de sa justice, et commet l'iniquité, il mourra à cause de ces choses-là ; il mourra à cause de son iniquité qu'il aura commise.
\VS{27}Si le méchant se détourne de sa méchanceté qu'il aura commise, et pratique ce qui est juste et droit, il fera vivre son âme.
\VS{28}Ayant donc considéré sa conduite, et s'étant détourné de tous ses crimes qu'il aura commis, certainement il vivra, il ne mourra point.
\VS{29}La maison d'Israël dit : La voie du Seigneur Yahweh n'est pas bien réglée. Ô maison d'Israël ! Mes voies ne sont-elles pas bien réglées ? Ne sont-ce pas plutôt vos voies qui ne sont pas bien réglées ?
\VS{30}C'est pourquoi je jugerai chacun de vous selon ses voies, ô maison d'Israël ! dit le Seigneur. Revenez, et détournez-vous de tous vos péchés, et l'iniquité ne vous ruinera pas.
\VS{31}Rejetez loin de vous tous les crimes par lesquels vous avez péché ; et faites-vous un nouveau cœur et un esprit nouveau ; pourquoi mourriez-vous, ô maison d'Israël ?
\VS{32}Car je ne désire pas la mort de celui qui meurt, dit le Seigneur Yahweh. Convertissez-vous donc, et vivez\FTNT{Ac. 3:19-20.}.
\Chap{19}
\TextTitle{Complaintes sur les dirigeants d'Israël}
\VerseOne{}Et toi, prononce à haute voix une complainte touchant les princes d'Israël.
\VS{2}Et dis : Ta mère, qu'était-ce ? C'était une lionne couchée parmi les lions, et qui a élevé ses petits parmi les jeunes lions.
\VS{3}Elle fit croître un de ses petits, qui devint un jeune lion, et qui apprit à déchirer la proie et a dévorer les hommes.
\VS{4}Les nations en entendirent parler, il fut attrapé dans leur fosse ; et elles l'emmenèrent avec des boucles au pays d'Egypte\FTNT{2 R. 23 : 33-34.}.
\VS{5}Puis ayant vu qu'elle attendait en vain, qu'elle était trompée dans son espérance, elle prit un autre de ses petits, et en fit un jeune lion.
\VS{6}Il marcha parmi les lions et devint un jeune lion, il apprit à déchirer la proie et a dévorer les hommes.
\VS{7}Il désola leurs palais, il ravagea leurs villes, de sorte que le pays, et tout ce qui y est, fut épouvanté par le cri de son rugissement.
\VS{8}Les nations s'armèrent contre lui de toutes les provinces, elles étendirent leurs rets contre lui, et il fut attrapé dans leur fosse\FTNT{2 R. 24:2.}.
\VS{9}Puis ils l'enfermèrent et l'enchaînèrent, pour l'amener au roi de Babylone, et le mettre dans une forteresse, afin que sa voix ne soit plus entendue sur les montagnes d'Israël.
\VS{10}Ta mère était comme une vigne dans ton sang plantée auprès des eaux, et elle est devenue chargée de fruits et de rameaux, à cause des grandes eaux.
\VS{11}Elle avait de puissantes branches pour en faire des sceptres de souverains ; son tronc s'était élevé jusqu'à ses branches touffues, et on la voyait dans sa hauteur avec la multitude de ses rameaux.
\VS{12}Mais elle a été arrachée avec fureur, et jetée par terre ; et le vent d'orient a séché son fruit ; ses puissantes branches se sont rompues et ont séché ; le feu les a consumées.
\VS{13}Maintenant elle est plantée dans le désert, dans une terre sèche et aride.
\VS{14}Le feu est sorti de ses branches, et a consumé son fruit ; et il n'y a plus en elle de puissantes branches pour un sceptre de souverain. C'est là une complainte, et cela servira de complainte.
\Chap{20}
\TextTitle{Compassions de Yahweh face aux infidélités d'Israël}
\VerseOne{}Or il arriva la septième année, au dixième jour du cinquième mois, que quelques-uns des anciens d'Israël vinrent pour consulter Yahweh, et s'assirent devant moi.
\VS{2}La parole de Yahweh me fut adressée en ces mots :
\VS{3}Fils de l'homme, parle aux anciens d'Israël, et dis-leur : Ainsi parle le Seigneur Yahweh : Est-ce pour me consulter que vous venez ? Je suis vivant, dit le Seigneur Yahweh, si vous me consultez.
\VS{4}Ne les jugeras-tu pas, ne les jugeras-tu pas, fils de l'homme ? Donne-leur à connaître les abominations de leurs pères.
\VS{5}Et dis-leur : Ainsi parle le Seigneur Yahweh : Le jour où j'ai choisi Israël, j'ai levé ma main vers la postérité de la maison de Jacob, et je me suis fait connaître à eux dans le pays d'Egypte, et j'ai levé ma main vers eux, en disant : Je suis Yahweh, votre Dieu.
\VS{6}En ce jour, j'ai levé ma main vers eux, pour les faire sortir du pays d'Egypte, pour les amener dans un pays que j'avais cherché pour eux, pays où coulent le lait et le miel, et qui est la noblesse de tous les pays\FTNT{Ex. 3:8 ; Ex. 6:7.}.
\VS{7}Alors je leur dis : Que chacun de vous rejette les abominations qui attirent ses regards, et ne vous souillez point par les idoles d'Egypte ! Je suis Yahweh, votre Dieu\FTNT{Jos. 24:14-23.}.
\VS{8}Mais ils se rebellèrent contre moi, et ils ne voulurent point m'écouter. Aucun ne rejeta les abominations qui attiraient ses regards, et ils n'abandonnèrent point les idoles de l'Egypte. Et je dis que je répandrais ma fureur sur eux, que je consumerais ma colère sur eux au milieu du pays d'Egypte.
\VS{9}Mais je les ai tirés hors du pays d'Egypte, je l'ai fait pour l'amour de mon Nom, afin qu'il ne soit point profané aux yeux des nations parmi lesquelles ils se trouvaient, et aux yeux desquelles je m'étais fait connaître à eux, pour les faire sortir du pays d'Egypte.
\VS{10}Je les fit donc sortir du pays d'Egypte, et je les conduisis dans le désert.
\VS{11}Je leur donnai mes lois et leur fis connaître mes ordonnances, que l'homme doit mettre en pratique, afin de vivre par elles\FTNT{Lé. 18:5 ; Ro. 10:5 ; Ga. 3:12.}.
\VS{12}Je leur donnai aussi mes sabbats, pour être un signe entre moi et eux, afin qu'ils sachent que je suis Yahweh qui les sanctifie\FTNT{Ex. 20:8 ; Ex. 31:13.}.
\VS{13}Mais ceux de la maison d'Israël se rebellèrent contre moi dans le désert. Ils ne suivirent point mes lois, et ils rejetèrent mes ordonnances que l'homme doit mettre en pratique, afin de vivre par elles, et ils profanèrent à l'excès mes sabbats. C'est pourquoi je dis que je répandrais sur eux ma fureur dans le désert pour les consumer\FTNT{Ex. 16:28.}.
\VS{14}Je l'ai fait pour l'amour de mon Nom, afin qu'il ne soit point profané devant les nations, en présence desquelles je les avais fait sortir d'Egypte\FTNT{Ex. 32:12 ; No. 14:13-14 ; De. 9:28 ; Jos. 7:9.}.
\VS{15}Je levai ma main vers eux dans le désert pour ne pas les amener dans le pays que je leur avais donné, pays où coulent le lait et le miel, et qui est la noblesse de tous les pays,
\VS{16}parce qu'ils ont rejeté mes ordonnances, qu'ils n'ont point suivi mes lois, et qu'ils ont profané mes sabbats, car leur cœur ne s'est pas éloigné de leurs idoles.
\VS{17}Toutefois, j'eus pour eux un regard de pitié pour ne point les détruire, et je ne les consumai point entièrement dans le désert.
\VS{18}Mais je dis à leurs fils dans le désert : Ne marchez point dans les statuts de vos pères, et ne gardez point leurs ordonnances, et ne vous souillez point par leurs idoles.
\VS{19}Je suis Yahweh, votre Dieu. Marchez dans mes statuts, et gardez mes ordonnances et accomplissez-les.
\VS{20}Sanctifiez mes sabbats, et ils seront un signe entre moi et vous, afin que vous reconnaissiez que je suis Yahweh, votre Dieu.
\VS{21}Mais les fils se rebellèrent aussi contre moi, et ils ne marchèrent point dans mes statuts, et ne gardèrent point mes ordonnances pour les faire ; ce que l'homme doit accomplir, pour vivre par elles. Ils profanèrent mes sabbats ; c'est pourquoi je dis que je répandrais ma fureur sur eux, et que je consumerais ma colère sur eux dans le désert.
\VS{22}Toutefois, je retirai ma main, et je le fis pour l'amour de mon Nom, afin qu'il ne soit point profané devant les nations, en présence desquelles je les avais sortis d'Egypte.
\VS{23}Néanmoins, je levai ma main vers eux dans le désert, pour les répandre parmi les nations, et les disperser dans les pays\FTNT{Lé. 26:13-33.},
\VS{24}parce qu'ils n'ont point accompli mes ordonnances, et qu'ils ont rejeté mes statuts, profané mes sabbats, et que leurs yeux se sont attachés aux idoles de leurs pères.
\VS{25}A cause de cela, je leur donnai des statuts qui n'étaient pas bons, et des ordonnances par lesquelles ils ne vivraient point.
\VS{26}Je les souillai par leurs dons, quand ils firent passer par le feu tous les premiers-nés, afin de les punir, et que l'on sache que je suis Yahweh.
\VS{27}C'est pourquoi, toi fils de l'homme, parle à la maison d'Israël, et dis-leur : Ainsi parle le Seigneur Yahweh : Vos pères m'ont encore outragé, car ils ont commis un crime contre moi.
\VS{28}Je les ai conduits dans le pays que j'avais juré de leur donner, et ils ont regardé toute haute colline, et tout arbre touffu, ils y ont fait leurs sacrifices, ils y ont posé leur oblation pour m'irriter, ils y ont mis leurs parfums, et ils y ont répandu leurs aspersions.
\VS{29}Je leur ai dit : Que veulent dire ces hauts lieux où vous allez ? Et le nom de hauts lieux leur a été donné jusqu'à ce jour.
\VS{30}C'est pourquoi dis à la maison d'Israël : Ainsi parle le Seigneur Yahweh : Ne vous souillez-vous pas selon les voies de vos pères, et ne vous prostituez-vous point à leurs idoles abominables,
\VS{31}en offrant vos dons, en faisant passer vos fils par le feu, en vous souillant par toutes vos idoles jusqu'à ce jour ? Est-ce ainsi que vous me consultez, ô maison d'Israël ? Je suis vivant, dit le Seigneur Yahweh, vous ne me consultez point.
\VS{32}Ce que vous pensez n'arrivera nullement, quand vous dites : Nous serons comme les nations, et comme les familles des pays, en servant le bois et la pierre.
\TextTitle{Restauration future d'Israël}
\VS{33}Je suis vivant ! dit le Seigneur Yahweh. Je règnerai sur vous avec une main forte, et un bras étendu, et avec effusion de colère.
\VS{34}Je vous sortirai du milieu des peuples, et vous rassemblerai hors des pays dans lesquels vous êtes dispersés, avec une main forte, et un bras étendu et avec effusion de colère.
\VS{35}Je vous ferai venir dans le désert des peuples, et je contesterai là contre vous, face à face,
\VS{36}comme j'ai contesté contre vos pères dans le désert du pays d'Egypte, ainsi je contesterai contre vous, dit le Seigneur Yahweh.
\VS{37}Je vous ferai passer sous la verge, et vous ramènerai au lieu de l'alliance\FTNT{Es. 65:12.}.
\VS{38}Je séparerai de vous les rebelles, et ceux qui se révoltent contre moi ; je les ferai sortir du pays dans lequel ils séjournent, mais ils n'entreront point dans la terre d'Israël ; et vous saurez que je suis Yahweh.
\VS{39}Vous donc, ô maison d'Israël, ainsi parle le Seigneur Yahweh : Allez, servez chacun vos idoles, puisque vous ne voulez pas m'écouter ! Ainsi vous ne profanerez plus mon saint Nom par vos dons et par vos idoles.
\VS{40}Mais ce sera sur ma sainte montagne, sur la haute montagne d'Israël, dit le Seigneur Yahweh, que toute la maison d'Israël me servira, dans le pays\FTNT{Jn. 4:21-24.}. Là, je prendrai plaisir en eux, et là je demanderai vos offrandes et les prémices de vos dons, et tout ce que vous me consacrerez.
\VS{41}Je prendrai plaisir en vous par vos parfums d'une agréable odeur, quand je vous aurai fait sortir du milieu des peuples, et que je vous aurai rassemblés des pays dans lesquels vous êtes dispersés ; je serai sanctifié par vous, aux yeux des nations.
\VS{42}Vous saurez que je suis Yahweh, quand je vous aurai fait revenir dans le pays d'Israël, dans le pays où j'ai levé ma main pour le donner à vos pères.
\VS{43}Et là, vous vous souviendrez de vos voies, et de toutes vos actions, par lesquelles vous vous êtes souillés ; et vous vous prendrez vous-mêmes en dégoût à cause de tous vos maux que vous aurez faits.
\VS{44}Vous saurez que je suis Yahweh, par tout ce que j'aurai fait pour vous, à cause de mon Nom, et non pas selon vos méchantes voies et vos actions corrompues, ô maison d'Israël ! dit le Seigneur Yahweh.
\Chap{21}
\TextTitle{L'épée de Yahweh}
\VerseOne{}La parole de Yahweh me fut encore adressée en ces mots :
\VS{2}Fils de l'homme, tourne ta face vers Jérusalem, parle en direction du sud, et prophétise contre la forêt du champ du sud.
\VS{3}Dis à la forêt du sud : Ecoute la parole de Yahweh. Ainsi parle le Seigneur Yahweh : Voici, je m'en vais allumer au dedans de toi un feu qui consumera tout bois vert et tout bois sec au dedans de toi ; la flamme de l'embrasement ne s'éteindra point, et tout le dessus en sera brûlé, depuis le sud jusqu'au nord\FTNT{Jé. 21:14 ; Jé. 22:7 ; Jé. 46:23 ; Lu. 23:31.}.
\VS{4}Toute chair verra que moi, Yahweh, j'ai allumé le feu ; et il ne s'éteindra point.
\VS{5}Je dis : Ah ! Seigneur Yahweh, ils disent de moi : N'est-il pas vrai que celui-ci ne fait que mettre en avant des similitudes ?
\TextTitle{Parabole de l'épée de Yahweh}
\VS{6}La parole de Yahweh me fut adressée en ces mots :
\VS{7}Fils de l'homme, tourne ta face vers Jérusalem, et parle en direction du lieu saint, et prophétise contre la terre d'Israël.
\VS{8}Dis à la terre d'Israël : Ainsi parle Yahweh : Voici, j'en veux à toi, je tirerai mon épée de son fourreau, et je retrancherai du milieu de toi le juste et le méchant.
\VS{9}Parce que je retrancherai du milieu de toi le juste et le méchant, à cause de cela mon épée sortira de son fourreau contre toute chair, depuis le sud jusqu'au nord.
\VS{10}Toute chair saura que moi, Yahweh, j'ai tiré mon épée de son fourreau, et elle n'y retournera plus.
\VS{11}Aussi, toi, fils de l'homme, gémis en te rompant les reins de douleur, et soupire avec amertume dans leur présence.
\VS{12}Quand ils te diront : Pourquoi gémis-tu ? Alors tu répondras : C'est à cause d'une nouvelle, car elle vient, et tout cœur se fondra, et toutes les mains seront baissées, tout esprit sera affaibli, et tous les genoux se fondront en eau ; voici, elle vient, elle arrive, dit le Seigneur Yahweh\FTNT{Jé. 6:24 ; Jé. 49:23.}.
\VS{13}Puis la parole de Yahweh me fut adressée en ces mots :
\VS{14}Fils de l'homme, prophétise, et dis : Ainsi parle Yahweh : Dis : L'épée ! L'épée a été aiguisée, elle est polie !
\VS{15}Elle a été aiguisée pour faire un grand carnage, elle a été polie afin qu'elle brille… Nous réjouirons-nous ? C'est la verge de mon fils, elle dédaigne tout bois.
\VS{16}Yahweh l'a donnée à polir, afin qu'on la tienne à la main ; l'épée a été aiguisée, et elle a été polie pour la mettre dans la main du destructeur.
\VS{17}Crie et hurle, fils de l'homme, car elle est contre mon peuple, elle est contre tous les princes d'Israël ; ils sont livrés à l'épée à cause de mon peuple. C'est pourquoi frappe sur ta cuisse !
\VS{18}Oui l'épreuve sera faite ; et que sera-ce si ce sceptre qui méprise tout est anéanti ? dit le Seigneur Yahweh.
\VS{19}Toi donc, fils de l'homme, prophétise, et frappe d'une main contre l'autre, que les coups de l'épée soient doublés, soient triplés, c'est l'épée du carnage, l'épée du grand carnage, l'épée qui doit les poursuivre.
\VS{20}J'ai mis à toutes leurs portes l'épée étincelante, afin que le cœur se fonde, et que les ruines soient multipliées. Ah ! Elle est faite pour briller et réservée pour tuer.
\VS{21}Joins-toi épée, frappe à la droite ! Avance-toi, frappe à la gauche, à tous côtés que tu rencontres !
\VS{22}Je frapperai aussi d'une main contre l'autre, je donnerai du repos à ma colère. Moi, Yahweh, j'ai parlé.
\VS{23}La parole de Yahweh me fut adressée en ces mots :
\VS{24}Toi, fils de l'homme, pose deux chemins où l'épée du roi de Babylone pourrait venir ; que les deux chemins sortent d'un même pays, et forme-les, forme-les de ta main à l'endroit où commence le chemin de la ville.
\VS{25}Tu poseras le chemin par lequel l'épée doit venir contre Rabbath des fils d'Ammon, et le chemin qui va en Judée, et à Jérusalem, ville forte.
\VS{26}Car le roi de Babylone se tient au carrefour, à l'entrée des deux chemins, pour consulter les devins ; il aiguise les flèches, il interroge les théraphim, il examine le foie.
\VS{27}Dans sa main droite est la divination contre Jérusalem, pour y dresser des béliers, pour publier le carnage, pour pousser des cris de guerre, pour ranger les béliers contre les portes, pour élever des terrasses et construire des remparts.
\VS{28}Mais ce sera pour eux, à leurs yeux, une divination vaine ; il y a de grands serments entre eux. Mais lui, il se souvient de leur iniquité, en sorte qu'ils seront pris.
\VS{29}C'est pourquoi ainsi parle le Seigneur Yahweh : Parce que vous avez fait revenir le souvenir de votre iniquité, lorsque vos crimes se sont découverts, au point de voir vos péchés dans toutes vos actions ; parce que vous avez fait qu'on se souvienne de vous, vous serez saisis par sa main.
\TextTitle{Quand l'iniquité arrive à son terme\FTNTT{Ap. 19:11-20:6.}}
\VS{30}Et toi, profane, méchant, prince d'Israël, dont le jour arrive au temps où l'iniquité est à son terme !
\VS{31}Ainsi parle le Seigneur Yahweh : Qu'on ôte cette tiare, et qu'on enlève cette couronne. Ce ne sera plus celle-ci ; j'élèverai ce qui est bas, et j'abaisserai ce qui est haut\FTNT{Job. 5:11 ; 1 Co. 1:27.}.
\VS{32}J'en ferai une ruine, une ruine, une ruine, et elle ne sera plus. Mais cela n'aura lieu qu'à la venue de celui à qui appartient le jugement et à qui je le lui donnerai.
\VS{33}Toi, fils de l'homme, prophétise, et dis : Ainsi parle le Seigneur Yahweh, au sujet des fils d'Ammon, et de leur opprobre : Dis donc, épée, épée dégainée, polie pour le massacre, pour dévorer avec son éclat !
\VS{34}Au milieu de tes visions vaines et de tes oracles menteurs, elle te fera tomber parmi les cadavres des méchants, dont le jour arrive au temps où l'iniquité est à son terme.
\VS{35}La remettrait-on dans son fourreau ? Je te jugerai sur le lieu où tu as été créé, au pays de ta naissance.
\VS{36}Je répandrai ma colère sur toi, j'allumerai sur toi le feu de ma fureur, et je te livrerai entre les mains d'hommes brutaux, qui ne travaillent qu'à détruire\FTNT{Jé. 25:11 ; Jé. 52:30.}.
\VS{37}Tu seras destiné au feu pour être dévoré ; ton sang sera au milieu de la terre : On ne se souviendra plus de toi, car c'est moi, Yahweh, qui parle.
\Chap{22}
\TextTitle{Les péchés d'Israël}
\VerseOne{}La parole de Yahweh me fut encore adressée en ces mots :
\VS{2}Et toi, fils de l'homme, ne jugeras-tu pas, ne jugeras-tu pas la ville sanguinaire, et ne lui donneras-tu pas à connaître toutes ses abominations\FTNT{Na. 3:1-4 ; Ha. 1:13 ; Ez. 24:6-9.} ?
\VS{3}Tu diras donc, ainsi parle le Seigneur Yahweh : Ville qui répands le sang au milieu de toi, afin que ton temps vienne, et qui as fait des idoles à ton préjudice, pour en être souillée.
\VS{4}Tu t'es rendue coupable par ton sang que tu as répandu, et tu t'es souillée par tes idoles que tu as faites ; tu as fait approcher tes jours, et tu es venue au terme de tes années ; c'est pourquoi je t'ai exposée en opprobre aux nations, et en dérision dans tous les pays\FTNT{2 R. 21:16 ; Jé. 26:21-23.}.
\VS{5}Celles qui sont près de toi, et celles qui en sont loin, se moqueront de toi, infâme de réputation, et remplie de troubles.
\VS{6}Voici, les princes d'Israël ont contribué au dedans de toi, chacun selon sa force, à répandre le sang.
\VS{7}Au dedans de toi, on méprise père et mère, on use de tromperie à l'égard de l'étranger, on opprime l'orphelin et la veuve.
\VS{8}Tu méprises ma sainteté, et profanes mes sabbats.
\VS{9}Des gens médisants sont au milieu de toi pour répandre le sang, ceux qui sont chez toi mangent sur les montagnes ; on commet des actions énormes au milieu de toi\FTNT{Es. 57:7 ; Jé. 2:20.}.
\VS{10}L'enfant découvre la nudité du père au milieu de toi, et on humilie au milieu de toi la femme dans le temps de son impureté\FTNT{Lé. 18:6-9 ; Ge. 9:22-23.}.
\VS{11}L'un commet l'abomination avec la femme de son prochain ; et l'autre se souille par l'inceste avec sa belle-fille ; chacun humilie sa sœur, fille de son père\FTNT{Ge. 19:32-36 ; Lé. 18:15-20 ; Jé. 5:8.}.
\VS{12}Chez toi, on reçoit des présents pour répandre le sang ; tu exiges un intérêt et une usure, tu dépouilles ton prochain par l'extorsion, et tu m'oublies, dit le Seigneur Yahweh\FTNT{Ex. 23:8 ; De. 27:25.}.
\VS{13}Voici, je frappe de mes mains l'une contre l'autre à cause de ton gain déshonnête que tu fais, et à cause de ton sang qui se répand au milieu de toi.
\VS{14}Ton cœur pourra-t-il tenir ferme, tes mains seront-elles fortes dans les jours où j'agirai contre toi ? Moi, Yahweh, j'ai parlé, et je le ferai.
\VS{15}Je te disperserai parmi les nations, je t'éparpillerai en divers pays, et je consumerai ta souillure, jusqu'à ce qu'il n'y en ait plus en toi.
\VS{16}Tu seras souillée par toi-même aux yeux des nations, et tu sauras que je suis Yahweh.
\TextTitle{La fureur de Yahweh}
\VS{17}Puis la parole de Yahweh me fut adressée en ces mots :
\VS{18}Fils de l'homme, la maison d'Israël m'est devenue comme de l'écume ; eux tous sont de l'airain, de l'étain, du fer et du plomb dans un creuset ; ils sont devenus comme une écume d'argent.
\VS{19}C'est pourquoi ainsi parle le Seigneur Yahweh : Parce que vous êtes tous devenus comme de l'écume, voici, je vais à cause de cela vous rassembler au milieu de Jérusalem,
\VS{20}comme on assemble de l'argent, de l'airain, du fer, du plomb, et de l'étain dans un creuset, afin d'y souffler le feu pour les fondre ; je vous rassemblerai ainsi dans ma colère et dans ma fureur, et je vous fondrai.
\VS{21}Je vous assemblerai, je soufflerai contre vous le feu de ma fureur, et vous serez fondus au milieu de Jérusalem.
\VS{22}Comme l'argent se fond dans le creuset, ainsi vous serez fondus au milieu d'elle, et vous saurez que moi,Yahweh, j'ai répandu ma fureur sur vous.
\TextTitle{Se tenir à la brèche devant Yahweh}
\VS{23}La parole de Yahweh me fut encore adressée en ces mots :
\VS{24}Fils de l'homme, dis-lui : Tu es une terre qui n'est pas purifiée ni arrosée de pluie au jour de la colère.
\VS{25}Il y a un complot de ses prophètes au milieu d'elle ; ils seront comme des lions rugissants, qui ravissent la proie : Ils dévorent les âmes, ils emportent les richesses et la gloire, ils multiplient les veuves au milieu d'elle\FTNT{Mt. 23:13 ; 1 Pi. 5:8.}.
\VS{26}Ses sacrificateurs ont fait violence à ma loi et ont profané mes choses saintes ; ils ne font pas de différence entre la chose sainte et profane; ils ne donnent pas à connaître la diffrence qu'il y a entre la chose impure et la pure, et ils cachent leurs yeux de mes sabbats, et je suis profané au milieu d'eux.
\VS{27}Ses princes sont au milieu d'elle comme des loups qui ravissent la proie, pour répandre le sang et pour détruire les âmes, pour s'adonner au gain déshonnête\FTNT{2 Pi. 2:16 ; Mt. 10:16 ; Mi. 3:11.}.
\VS{28}Ses prophètes ont pour eux des enduits de plâtre, des visions fausses, et des oracles menteurs, en disant : Ainsi parle le Seigneur Yahweh ; et cependant Yahweh n'a point parlé.
\VS{29}Le peuple du pays use de tromperies, ravit le bien d'autrui, opprime l'affligé et le pauvre, et foule l'étranger contre tout droit.
\VS{30}Je cherche parmi eux un homme\FTNT{Dieu n'a pas besoin d'une foule de gens avant d'agir. Une seule personne suffit. } qui élève un mur, et qui se tient à la brèche devant moi pour le pays, afin que je ne le détruise point ; mais je n'en trouve pas.
\VS{31}C'est pourquoi je répandrai sur eux ma colère, et je les consumerai par le feu de ma fureur ; je mettrai leur voie sur leur tête, dit le Seigneur Yahweh.
\Chap{23}
\TextTitle{Prostitutions d'Israël et de Juda}
\VerseOne{}La parole de Yahweh me fut encore adressée en ces mots :
\VS{2}Fils de l'homme, il y a eu deux femmes, filles d'une même mère,
\VS{3}qui se sont prostituées en Egypte, elles se sont prostituées dans leur jeunesse. Là, leur sein fut déshonoré et leur virginité touchée.
\VS{4}Et c'était ici leurs noms, celui de la plus grande était Ohola, et celui de sa sœur Oholiba\FTNT{Ohola signifie « sa propre tente », et Oholiba « la femme de la tente ». 2 R. 17:23-24.} ; elles étaient à moi, et elles ont enfanté des fils et des filles ; leurs noms donc étaient Ohola, qui était Samarie, et Oholiba, qui est Jérusalem.
\VS{5}Or Ohola a commis adultère étant ma femme, et s'est rendue amoureuse de ses amoureux, c'est-à-dire, des Assyriens ses voisins,
\VS{6}vêtus de pourpre, gouverneurs et magistrats, tous jeunes et aimables, tous cavaliers, montés sur des chevaux.
\VS{7}Elle a commis ses adultères avec toute l'élite des fils des Assyriens, et avec tous ceux pour qui elle s'était enflammée, et s'est souillée avec toutes leurs idoles.
\VS{8}Elle n'a pas abandonné ses fornications d'Egypte, car ils avaient couché avec elle dans sa jeunesse, ils avaient déshonoré sa virginité et s'étaient livrés à l'impureté avec elle\FTNT{Ac. 7:42.}.
\VS{9}C'est pourquoi je l'ai livrée entre les mains de ses amoureux, entre les mains des fils des Assyriens, dont elle s'était rendue amoureuse.
\VS{10}Ils l'ont couverte d'opprobre, ils ont enlevé ses fils et ses filles, et l'ont tuée elle-même avec l'épée ; elle a été en renom parmi les femmes, après avoir exercé des jugements sur elle.
\VS{11}Quand sa sœur Oholiba a vu cela, et fut plus déréglée qu'elle dans ses passions ; ses prostitutions dépassèrent celles de sa sœur.
\VS{12}Elle s'enflamma pour les fils des Assyriens, des gouverneurs et des magistrats, ses voisins, vêtus magnifiquement, et des cavaliers montés sur des chevaux, tous jeunes et bien faits.
\VS{13}J'ai vu qu'elle s'était souillée, et que l'une et l'autre avaient suivi la même voie.
\VS{14}Elle alla même plus loin dans ses prostitutions. Elle aperçut contre des murailles des peintures d'hommes, des images de Chaldéens peints en couleur rouge,
\VS{15}avec des ceintures autour des reins, avec des turbans de couleurs variées flottant sur la tête. Tous ayant l'apparence de princes, et figurant des fils de Babylone.
\VS{16}Elle s'enflamma pour eux au premier regard, et leur envoya des messagers en Chaldée.
\VS{17}Les fils de Babylone vinrent vers elle au lit de ses prostitutions, et la souillèrent par leurs adultères ; elle s'est aussi souillée avec eux, et après cela son cœur s'est détaché d'eux.
\VS{18}Elle a manifesté ses fornications et fait connaître son opprobre ; et mon cœur s'est détaché d'elle, comme mon cœur s'était détaché de sa sœur.
\VS{19}Car elle a multiplié ses adultères, jusqu'à rappeler le souvenir des jours de sa jeunesse, lorsqu'elle s'était abandonnée au pays d'Egypte.
\VS{20}Elle s'est enflammée pour des impudiques dont la chair était comme celle des ânes, et dont la force égale celle des chevaux.
\VS{21}Tu as donc repris les méchancetés de ta jeunesse, lorsque tu as été déshonorée, depuis que tu étais en Egypte, à cause du sein de ta jeunesse.
\VS{22}C'est pourquoi, Oholiba, ainsi parle le Seigneur Yahweh : Voici, je m'en vais réveiller contre toi tous tes amants, ceux dont ton cœur s'est détaché, et je les amènerai contre toi de toutes parts.
\VS{23}Les fils de Babylone, et tous les Chaldéens, Pekod, Shoa, Koa, et tous les Assyriens avec eux, tous jeunes gens d'élite, gouverneurs et magistrats, grands seigneurs et renommés, tous montant à cheval.
\VS{24}Ils viendront contre toi avec des armes, des chars, et des roues, avec une multitude de peuples, avec le grand bouclier et le petit bouclier, avec les casques, et je leur mettrai le jugement en main, ils te jugeront selon leur jugement.
\VS{25}Je mettrai ma jalousie contre toi, et ils agiront contre toi avec fureur ; ils te retrancheront le nez et les oreilles, et ce qui restera de toi tombera par l'épée. Ils enlèveront tes fils et tes filles, et ce qui restera de toi sera dévoré par le feu.
\VS{26}Ils te dépouilleront de tes vêtements, et t'enlèveront les ornements dont tu te pares.
\VS{27}Je mettrai un terme à tes méchancetés et tes prostitutions du pays d'Egypte ; tu ne lèveras plus tes yeux vers eux, et tu ne te souviendras plus de l'Egypte.
\VS{28}Car ainsi parle le Seigneur Yahweh : Voici, je te livre entre les mains de ceux que tu hais, entre les mains de ceux dont ton cœur s'est détaché.
\VS{29}Ils te traiteront avec haine ; ils enlèveront tout ton travail, et te laisseront sans habits et découverte ; et la turpitude de tes adultères, de ton énormité, et de tes fornications, sera découverte.
\VS{30}On te fera ces choses-là parce que tu t'es prostituée aux nations, avec lesquelles tu t'es souillée par leurs idoles.
\VS{31}Tu as marché dans la voie de ta sœur, c'est pourquoi je mets sa coupe dans ta main.
\VS{32}Ainsi parle le Seigneur Yahweh : Tu boiras la coupe profonde et large de ta sœur ; elle sera une coupe d'une grande mesure ; tu seras un sujet de risée et de moquerie\FTNT{Ps. 75:9 ; Es. 51:17 ; Jé. 25:15.}.
\VS{33}Tu seras remplie d'ivresse et de douleur, par la coupe de désolation et de dégât, qui est la coupe de ta sœur Samarie.
\VS{34}Tu la boiras et la videras, tu briseras ce pot de terre et tu déchireras ton sein. Car j'ai parlé, dit le Seigneur Yahweh.
\VS{35}C'est pourquoi ainsi parle le Seigneur Yahweh : Parce que tu m'as oublié, et que tu m'as jeté derrière ton dos, aussi porteras-tu la peine de ta méchanceté, et de tes prostitutions.
\TextTitle{Jugement sur Israël et Juda}
\VS{36}Puis Yahweh me dit : Fils de l'homme, ne jugeras-tu pas Ohola et Oholiba ? Déclare-leur donc leurs abominations.
\VS{37}Déclare-leur comment elles ont commis l'adultère et comment il y a du sang dans leurs mains ; comment, dis-je, elles ont commis l'adultère avec leurs idoles, et ont même fait passer par le feu leurs fils pour les consumer, ces enfants qu'elles m'avaient enfantés.
\VS{38}Voici encore ce qu'elles m'ont fait : Elles ont souillé mon lieu saint ce même jour, et ont profané mes sabbats.
\VS{39}Car après avoir égorgé leurs fils à leurs idoles, elles sont entrées ce même jour-là dans mon lieu saint pour le profaner ; et voilà, comment elles ont fait au milieu de ma maison\FTNT{2 R. 21:4.}.
\VS{40}Et qui plus est, elles ont fait chercher des hommes venant de loin, elles leur ont envoyé des messagers, et voici, ils sont venus. Pour eux tu t'es lavée, tu as fardé ton visage, et t'es parée d'ornements.
\VS{41}Tu t'es assise sur un lit magnifique, devant lequel a été apprêtée une table, sur laquelle tu as mis mon encens et mon huile.
\VS{42}On entendait le bruit d'une multitude tranquille ; et parmi cette foule d'hommes, on a fait venir du désert des Sabéens, qui ont mis des bracelets aux mains des deux sœurs, et de superbes couronnes sur leurs têtes.
\VS{43}J'ai dit au sujet de celle qui avait vieilli dans l'adultère : Maintenant ses impudicités prendront fin, et elle aussi.
\VS{44}Toutefois on est venu vers elle comme on vient vers une femme prostituée ; ils sont ainsi venus vers Ohola, et vers Oholiba, femmes pleines de méchanceté.
\VS{45}Les hommes justes donc les jugeront comme on juge les femmes adultères, et comme on juge celles qui répandent le sang ; car elles sont adultères, et le sang est dans leurs mains.
\VS{46}C'est pourquoi ainsi parle le Seigneur Yahweh : Qu'on fasse monter l'assemblée contre elles, et qu'elles soient abandonnées au tumulte et au pillage.
\VS{47}Que l'assemblée les lapide de pierres et les taille en pièces avec leurs épées ; qu'ils tuent leurs fils et leurs filles, et qu'ils brûlent au feu leurs maisons.
\VS{48}Et ainsi je ferai cesser la méchanceté dans le pays, et toutes les femmes seront enseignées à ne point faire selon votre méchanceté.
\VS{49}On mettra votre méchanceté sur vous, et vous porterez les péchés de vos idoles ; et vous saurez que je suis le Seigneur Yahweh.
\Chap{24}
\TextTitle{Malheur à la ville sanguinaire}
\VerseOne{}La neuvième année, au dixième jour du dixième mois, la parole de Yahweh me fut adressée en ces mots :
\VS{2}Fils de l'homme, mets par écrit la date de ce jour, de ce jour-ci ! Car en ce même jour le roi de Babylone s'approche contre Jérusalem\FTNT{2 R. 25:1.}.
\VS{3}Propose une parabole à la famille de rebelles, et dis-leur : Ainsi parle le Seigneur Yahweh : Mets, mets la chaudière, et verse de l'eau dedans.
\VS{4}Mets-y les morceaux, tous les bons morceaux, la cuisse, et l'épaule, et remplis-la des meilleurs os.
\VS{5}Prends la meilleure bête du troupeau, et fais brûler des os sous la chaudière, fais-la bouillir à gros bouillons, et que les os cuisent au-dedans.
\VS{6}Car ainsi parle le Seigneur Yahweh : Malheur à la ville sanguinaire, à la chaudière pleine de rouille, et de laquelle la rouille n'est point sortie ! Vide-la morceau par morceau, et que le sort ne soit point jeté sur elle.
\VS{7}Parce que son sang est au milieu d'elle, qu'elle l'a mis sur le rocher brillant, et qu'elle ne l'a point répandu sur la terre pour le couvrir de poussière,
\VS{8}j'ai mis son sang sur un rocher brillant, afin qu'il ne soit point couvert, pour faire monter la fureur, et pour me venger.
\VS{9}C'est pourquoi ainsi parle le Seigneur Yahweh : Malheur à la ville sanguinaire ! J'en ferai aussi un grand tas de bois à brûler !
\VS{10}Amasse beaucoup de bois, allume le feu, fais cuire la chair entièrement, et fais-la consumer, et que les os soient brûlés.
\VS{11}Puis mets sur les charbons ardents la chaudière toute vide, afin qu'elle s'échauffe et que son airain se brûle, et que sa souillure soit fondue à l'intérieur, et que sa rouille soit consumée.
\VS{12}Les efforts sont inutiles, sa rouille dont elle est pleine n'est point sortie d'elle ; sa rouille ne s'en ira que par le feu.
\VS{13}L'impureté est dans ta souillure ; car je t'avais purifiée, et tu n'as point été pure ; tu ne seras pas encore nettoyée de ta souillure, jusqu'à ce que j'aie assouvi sur toi ma fureur.
\VS{14}Moi, Yahweh, j'ai parlé, cela arrivera, et je le ferai ; et je ne me retirerai point en arrière, je n'épargnerai point, et je ne serai point apaisé. On t'a jugée selon tes voies et selon tes actions, dit le Seigneur Yahweh.
\TextTitle{La vie d'Ezéchiel, un signe pour Israël}
\VS{15}La parole de Yahweh me fut adressée en ces mots :
\VS{16}Fils de l'homme, voici, je vais t'ôter par une plaie ce que tes yeux voient avec le plus de plaisir. Ne mène point de deuil, ne pleure point, ne fais point couler tes larmes\FTNT{Jé. 16:6-7.}.
\VS{17}Garde-toi de gémir, et ne fais pas le deuil des morts ; attache ton turban sur ta tête, mets tes souliers à tes pieds, ne te couvre pas la barbe, et ne mange pas le pain des autres\FTNT{Lé. 10:6.}.
\VS{18}Je parlai au peuple le matin, et ma femme mourut le soir ; le lendemain matin je fis comme il m'avait été ordonné.
\VS{19}Le peuple me dit : Ne nous déclareras-tu point ce que nous signifient ces choses-là que tu fais ?
\VS{20}Je leur répondis : La parole de Yahweh m'a été adressée en ces mots :
\VS{21}Parle à la maison d'Israël : Ainsi parle le Seigneur Yahweh : Voici, je m'en vais profaner mon lieu saint, la magnificence de votre force, ce qui est le plus agréable à vos yeux, ce que vous voudriez épargner sur toutes choses ; et vos fils et vos filles, que vous aurez laissés, tomberont par l'épée.
\VS{22}Vous ferez alors comme j'ai fait ; vous ne couvrirez point vos barbes, et vous ne mangerez point le pain des autres.
\VS{23}Vos turbans seront sur vos têtes, et vos souliers à vos pieds ; vous ne mènerez point de deuil ni ne pleurerez ; mais vous pourrirez à cause de vos iniquités, et vous gémirez les uns avec les autres.
\VS{24}Ezéchiel sera pour vous un signe ; vous ferez selon toutes les choses qu'il a faites ; et quand cela sera arrivé, vous saurez que je suis le Seigneur Yahweh.
\VS{25}Quant à toi, fils de l'homme, au jour que je leur ôterai leur force, la joie de leur ornement, l'objet le plus agréable à leurs yeux, et l'objet de leurs cœurs, leurs fils et leurs filles,
\VS{26}ce jour-là un fuyard ne viendra-t-il pas vers toi pour te le raconter ?
\VS{27}En ce jour-là ta bouche sera ouverte envers celui qui sera échappé, et tu parleras, et ne seras plus muet ; ainsi tu seras pour eux un signe, et ils sauront que je suis Yahweh.
\Chap{25}
\TextTitle{Jugement de Dieu sur Ammon}
\VerseOne{}Puis la parole de Yahweh me fut adressée en ces mots :
\VS{2}Fils de l'homme, tourne ta face vers les fils d'Ammon, et prophétise contre eux\FTNT{Jé. 49:1.}.
\VS{3}Dis aux fils d'Ammon : Ecoutez la parole du Seigneur Yahweh : Parce que vous avez dit : Ah ! Ah ! contre mon lieu saint, parce qu'il était profané ; et contre la terre d'Israël, parce qu'elle était désolée ; et contre la maison de Juda, parce qu'ils allaient en captivité\FTNT{Am. 1:13 ; So. 2:8.};
\VS{4}à cause de cela, voici, je m'en vais te donner en héritage aux fils d'orient, et ils bâtiront des palais dans tes villes, et ils demeureront chez toi ; ils mangeront tes fruits et boiront ton lait.
\VS{5}Je livrerai Rabba pour être le repaire des chameaux, et le pays des fils d'Ammon pour être le gîte des brebis, et vous saurez que je suis Yahweh.
\VS{6}Car ainsi parle le Seigneur Yahweh : Parce que tu as frappé des mains, que tu as battu des pieds, et que tu t'es réjoui de bon cœur avec tout le mépris que tu as eu pour la terre d'Israël,
\VS{7}à cause de cela voici, j'ai étendu ma main sur toi, et je te livrerai pour être pillée par les nations, et je te retrancherai du milieu des peuples, je te ferai périr d'entre les pays ; je te détruirai ; et tu sauras que je suis Yahweh.
\TextTitle{Jugement sur Moab}
\VS{8}Ainsi parle le Seigneur Yahweh : Parce que Moab et Séir ont dit : Voici, la maison de Juda est comme toutes les autres nations ;
\VS{9}à cause de cela voici, j'ouvre le territoire de Moab du côté des villes, de ses villes frontières, la beauté du pays de Beth-Jeschimoth, de Baal-Meon et de Kirjathaïm\FTNT{Jos. 12:3 ; No. 32:38.},
\VS{10}je l'ouvre aux fils d'orient, qui sont au-delà du pays des fils d'Ammon, je leur donne en possession, afin qu'on ne se souvienne plus des fils d'Ammon parmi les nations.
\VS{11}J'exercerai aussi des jugements contre Moab, et ils sauront que je suis Yahweh.
\TextTitle{Jugement sur Edom}
\VS{12}Ainsi parle le Seigneur Yahweh : A cause de ce qu'Edom a fait quand il s'est vengé de la maison de Juda, et parce qu'il s'en est rendu coupable en se vengeant d'eux\FTNT{Ps. 137:7.},
\VS{13}à cause de cela, le Seigneur Yahweh dit : J'étendrai ma main sur Edom, j'en retrancherai les hommes et les bêtes, et j'en ferai un désert ; depuis Théman à Dedan ils tomberont par l'épée\FTNT{Jé. 49:7-9 ; Am. 1:12 ; Ab. 1:9.}.
\VS{14}J'exercerai ma vengeance sur Edom à cause de mon peuple d'Israël, et on traitera Edom selon ma colère, et selon ma fureur, et ils reconnaîtront ma vengeance, dit le Seigneur Yahweh.
\TextTitle{Jugement sur les Philistins}
\VS{15}Ainsi parle le Seigneur Yahweh : Puisque les Philistins ont agi par vengeance, et qu'ils se sont vengés avec mépris et du fond de leur âme, voulant tout détruire dans leur haine éternelle ;
\VS{16}à cause de cela le Seigneur Yahweh dit : Voici, je m'en vais étendre ma main sur les Philistins, j'exterminerai les Kéréthiens, et je ferai périr le reste sur le rivage de la mer.
\VS{17}J'exercerai sur eux de grandes vengeances par des châtiments de fureur ; et ils sauront que je suis Yahweh, quand j'aurai exécuté sur eux ma vengeance\FTNT{Es. 14:29 ; Jé. 25:20 ; So. 2:7.}.
\Chap{26}
\TextTitle{Jugement sur Tyr}
\VerseOne{}Il arriva dans la onzième année, le premier jour du mois, que la parole de Yahweh me fut adressée en ces mots :
\VS{2}Fils de l'homme, parce que Tyr a dit au sujet de Jérusalem : Ah ! Ah ! Celle qui était la porte des peuples a été rompue, elle s'est réfugiée chez moi, je serai remplie parce qu'elle a été rendue déserte\FTNT{Am. 1:9 ; Za. 9:2-3.} !
\VS{3}A cause de cela, ainsi parle le Seigneur Yahweh : Voici, j'en veux à toi, Tyr, et je ferai monter contre toi plusieurs nations, comme la mer fait monter ses flots\FTNT{Jé. 51:42.}.
\VS{4}Elles détruiront les murailles de Tyr, et démoliront ses tours ; j'en raclerai sa poussière, et la rendrai semblable à un rocher nu\FTNT{ Es. 23:15.}.
\VS{5}Elle servira à étendre les filets au milieu de la mer ; car j'ai parlé, dit le Seigneur Yahweh, et elle sera en pillage aux nations.
\VS{6}Ses filles sur sa terre seront tuées par l'épée, et elles sauront que je suis Yahweh.
\VS{7}Car ainsi parle le Seigneur Yahweh : Voici, je m'en vais faire venir du nord contre Tyr, Nebucadnetsar, roi de Babylone, le roi des rois, avec des chevaux, des chars, des cavaliers, et un grand peuple assemblé de toutes parts.
\VS{8}Il tuera par l'épée tes filles sur ta terre, il fera des remparts contre toi, il dressera des terrasses contre toi, et il lèvera les boucliers contre toi.
\VS{9}Il donnera des coups de béliers contre tes murs, et renversera tes tours avec ses épées.
\VS{10}La multitude de ses chevaux te couvrira de poussière, tes murs trembleront au bruit des cavaliers, des roues, et des chars, quand il entrera par tes portes, comme on entre dans une ville qu'on a divisée.
\VS{11}Il foulera toutes tes rues avec les sabots de ses chevaux, il tuera ton peuple avec l'épée, et les trophées de ta force tomberont par terre\FTNT{Jé. 47:3 ; Es. 5:8.}.
\VS{12}Puis ils retireront tes biens, et pilleront ta marchandise ; ils renverseront tes murs, et renverseront tes maisons de plaisance ; et ils mettront tes pierres, ton bois et ta poussière au milieu des eaux.
\VS{13}Je ferai cesser le bruit de tes chansons, et le son de tes harpes ne sera plus entendu.
\VS{14}Je te rendrai semblable à un rocher nu ; tu seras un lieu pour étendre les filets, et tu ne seras plus rebâtie, parce que moi,Yahweh, j'ai parlé, dit le Seigneur Yahweh.
\VS{15}Ainsi parle le Seigneur Yahweh, à Tyr : Les îles ne trembleront-elles pas du bruit de ta ruine, quand ceux qui seront blessés à mort gémiront, quand le carnage se fera au milieu de toi ?
\VS{16}Tous les princes de la mer descendront de leurs trônes, ôteront leurs manteaux, dépouilleront leurs vêtements brodés, et s'envelopperont de frayeur ; ils s'assiéront sur la terre, ils seront effrayés à chaque instant, et seront désolés à cause de toi.
\VS{17}Ils prononceront à haute voix une complainte sur toi, et te diront : Comment as-tu péri, toi qui étais fréquentée par ceux qui vont sur la mer, ville renommée, qui étais forte dans la mer, toi et tes habitants qui inspiraient la terreur à tous ceux qui habitent chez elle\FTNT{Es. 23:15-16 ; Ap. 18:9.}?
\VS{18}Maintenant les îles seront effrayées au jour de ta ruine, et les îles qui sont dans la mer seront terrifiées à cause de ta fuite.
\VS{19}Car ainsi parle le Seigneur Yahweh : Quand je ferai de toi une ville désolée, comme sont les villes qui ne sont point habitées, quand j'aurai fait tomber sur toi l'abîme, et que les grosses eaux t'auront couverte ;
\VS{20}alors je te ferai descendre avec ceux qui descendent dans la fosse, vers le peuple d'autrefois, et je te placerai aux lieux les plus bas de la terre, aux endroits désolés depuis longtemps, avec ceux qui descendent dans la fosse, afin que tu ne sois plus habitée, mais je donnerai la gloire pour la terre des vivants.
\VS{21}Je ferai qu'on sera épouvanté à cause de toi, de ce que tu n'es plus ; et quand on te cherchera, on ne te trouvera plus jamais, dit le Seigneur Yahweh.
\Chap{27}
\TextTitle{Lamentation sur Tyr\FTNTT{Cp. Ap. 18:1-24.}}
\VerseOne{}La parole de Yahweh me fut encore adressée en ces mots :
\VS{2}Toi donc, fils de l'homme, prononce à haute voix une complainte sur Tyr.
\VS{3}Tu diras à Tyr : Toi qui demeures au bord de la mer, qui trafiques avec les peuples dans plusieurs îles ; ainsi parle le Seigneur Yahweh : Tyr, tu disais : Je suis parfaite en beauté !
\VS{4}Ton territoire est au cœur de la mer, ceux qui t'ont bâtie t'ont rendue parfaite en beauté.
\VS{5}Ils t'ont bâti de tous les côtés des navires de sapins de Senir ; ils ont pris les cèdres du Liban pour te faire des mâts.
\VS{6}Ils ont fait tes rames de chênes de Basan, et la troupe des Assyriens a fait tes bancs d'ivoire, apporté des îles de Kittim.
\VS{7}Le fin lin d'Egypte, avec des broderies, te servait de voiles et de pavillon ; des étoffes teintes en bleu et en pourpre des îles d'Elischa formaient tes couvertures.
\VS{8}Les habitants de Sidon et d'Arvad étaient tes rameurs, ô Tyr ! Les plus sages du milieu de toi étaient tes pilotes.
\VS{9}Les anciens de Guebal et ses hommes experts furent parmi toi, réparant tes brèches ; tous les navires de la mer, et leurs mariniers étaient chez toi, pour faire l'échange de tes marchandises.
\VS{10}Ceux de Perse, de Lud, et de Puth servaient dans ton armée. C'étaient des hommes de guerre, ils suspendaient chez toi le bouclier et le casque ; ils t'ont rendue magnifique.
\VS{11}Les fils d'Arvad avec ton armée étaient autour de tes murs, et des hommes braves étaient dans tes tours ; ils ont suspendu leurs boucliers à tous tes murs, ils ont achevé de te rendre parfaite en beauté.
\VS{12}Ceux de Tarsis ont trafiqué avec toi de toutes sortes de richesses, d'argent, de fer, d'étain et de plomb.
\VS{13}Javan, Tubal, et Méschec trafiquaient avec toi ; ils donnaient des personnes et des ustensiles d'airain en échange de tes marchandises.
\VS{14}Ceux de la maison de Togarma pourvoyaient tes marchés de chevaux, de cavaliers, et de mulets.
\VS{15}Les fils de Dedan trafiquaient avec toi ; tu avais dans ta main le commerce de plusieurs îles ; et on t'a rendu en échange des dents d'ivoire et de l'ébène.
\VS{16}La Syrie trafiquait avec toi, en quantité d'ouvrages faits pour toi ; elle pourvoyait tes marchés d'escarboucles, d'écarlate, de broderie, de fin lin, de corail, et d'agate.
\VS{17}Juda et le pays d'Israël trafiquaient avec toi, faisant valoir ton commerce en blé de Minnith, en pâtisseries, en miel, en huile, et en baume.
\VS{18}Damas trafiquait avec toi en quantité d'ouvrages faits pour toi, en toutes sortes de richesses, en vin de Helbon, et en laine blanche.
\VS{19}Vedan, et Javan depuis Uzal, pourvoyaient tes marchés ; le fer luisant, la casse et le roseau aromatique furent dans ton commerce.
\VS{20}Dedan trafiquait avec toi en couvertures pour s'asseoir à cheval.
\VS{21}Les arabes, et tous les princes de Kédar, étaient des marchands dans ta main, trafiquant avec toi en agneaux, en moutons, et en boucs.
\VS{22}Les marchands de Séba et de Raema trafiquaient avec toi de tous les meilleurs aromates, de toute sorte de pierres précieuses et d'or.
\VS{23}Charan, Canné, et Eden, les marchands de Séba, d'Assyrie, de Kilmad, trafiquaient avec toi.
\VS{24}Ils trafiquaient avec toi toutes sortes de belles choses, des manteaux teints en bleu, en broderie, en riches étoffes contenues dans des coffres attachés avec des cordes, faits en bois de cèdre, et amenés sur tes marchés.
\VS{25}Les navires de Tarsis naviguaient pour ton commerce ; tu étais au comble de la force et de la richesse, au cœur des mers.
\VS{26}Tes rameurs t'ont amenée dans de grosses eaux, le vent d'orient t'a brisée au cœur de la mer.
\VS{27}Tes richesses, tes marchés et tes marchandises, tes mariniers et tes pilotes, ceux qui réparaient tes brèches, et ceux qui s'occupaient de ton commerce, tous tes hommes de guerre qui étaient chez toi, et toute ta multitude au milieu de toi, tomberont dans le cœur de la mer au jour de ta ruine\FTNT{Ap. 18:9.}.
\VS{28}Les faubourgs trembleront au bruit du cri de tes pilotes.
\VS{29}Tous ceux qui manient la rame descendront de leurs navires, les mariniers, et tous les pilotes de la mer ; ils se tiendront sur la terre ;
\VS{30}ils feront entendre leur voix, et crieront amèrement ; ils jetteront de la poussière sur leurs têtes, et se vautreront dans la cendre ;
\VS{31}ils arracheront leurs cheveux, et rendront leur tête chauve à cause de toi, ils se ceindront de sacs, et te pleureront avec l'amertume dans leur âme, en menant un deuil amer.
\VS{32}Ils prononceront à haute voix sur toi une complainte dans leur lamentation, et feront leur complainte sur toi, en disant : Qui fut jamais comme Tyr, comme cette ville détruite au cœur de la mer ?
\VS{33}Tu as rassasié plusieurs peuples par la traite des marchandises qu'on apportait de tes marchés au-delà des mers ; et tu as enrichi les rois de la terre par la multitude de tes richesses et de ton commerce.
\VS{34}Quand tu as été brisée par la mer au fond des eaux, ton commerce et toute ta multitude sont tombés avec toi.
\VS{35}Tous les habitants des îles sont désolés à cause de toi ; et leurs rois sont saisis d'épouvante, et leur visage pâlit.
\VS{36}Les marchands parmi les peuples t'insultent, tu es réduite au néant, tu ne seras plus à jamais !
\Chap{28}
\TextTitle{Yahweh réprime l'arrogance du roi de Tyr}
\VerseOne{}La parole de Yahweh me fut encore adressée en ces mots :
\VS{2}Fils de l'homme, dis au prince de Tyr : Ainsi parle le Seigneur Yahweh : Parce que ton cœur s'est élevé et que tu as dit : Je suis Dieu, je suis assis sur le siège de Dieu, au cœur de la mer, quoique tu sois un homme, et non Dieu, et parce que tu as élevé ton cœur comme si tu étais un Dieu.
\VS{3}Voici, tu es plus sage que Daniel, rien de caché ne t'a été rendu obscur.
\VS{4}Tu t'es acquis de la puissance par ta sagesse et par ton intelligence ; et tu as amassé de l'or et de l'argent dans tes trésors\FTNT{Za. 9:2-3.}.
\VS{5}Tu as multiplié ta puissance par la grandeur de ta sagesse dans ton commerce, puis ton cœur s'est élevé à cause de ta puissance.
\VS{6}C'est pourquoi ainsi parle le Seigneur Yahweh : Parce que tu as élevé ton cœur, comme si tu étais un Dieu,
\VS{7}à cause de cela voici, je m'en vais faire venir contre toi des étrangers, les plus terribles parmi les nations, qui tireront leurs épées sur la beauté de ta sagesse, et souilleront ta splendeur.
\VS{8}Ils te feront descendre dans la fosse, et tu mourras comme ceux qui tombent percés de coups, au milieu de la mer.
\VS{9}En face de ton meurtrier diras-tu : Je suis Dieu ? Tu seras homme et non Dieu sous la main de celui qui te tuera.
\VS{10}Tu mourras de la mort des incirconcis par la main des étrangers ; car j'ai parlé, dit le Seigneur Yahweh.
\TextTitle{Chute du roi de Tyr représentant satan\FTNTT{Cp. Es. 14:12-17.}}
\VS{11}La parole de Yahweh me fut encore adressée en ces mots :
\VS{12}Fils de l'homme, prononce à haute voix une complainte sur le roi de Tyr, et dis-lui : Ainsi parle le Seigneur Yahweh : Toi à qui rien ne manquait, plein de sagesse, et parfait en beauté ;
\VS{13}tu étais en Eden, le jardin de Dieu ; ta couverture était de pierres précieuses de toutes sortes, de sardoine, de topaze, de diamant, de chrysolithe, d'onyx, de jaspe, de saphir, d'escarboucle, d'émeraude, et d'or ; tes tambourins et tes flûtes étaient à ton service ; préparés pour le jour où tu fus créé.
\VS{14}Tu étais un chérubin, oint pour servir de protection ; je t'avais établi, et tu étais sur la sainte montagne de Dieu ; tu marchais entre les pierres éclatantes.
\VS{15}Tu étais parfait dans tes voies dès le jour où tu fus créé, jusqu'à celui où l'injustice fut trouvée en toi.
\VS{16}Selon la grandeur de ton trafic\FTNT{Satan est le premier commerçant. Il avait transformé ses sanctuaires célestes en un lieu de trafic, en un marché. Il avait reçu gratuitement du Seigneur, notre Dieu, plusieurs dons : la beauté, des pierres précieuses, des instruments de musique, la sagesse, un sanctuaire. Au lieu de les utiliser pour la gloire de Dieu, il en fit un trafic pour son propre profit égoïste. Il est le père de tous ceux qui vendent les dons de Dieu pour s'enrichir, de tous ceux qui font du commerce avec l'Evangile. Or Jésus nous a donné cet ordre formel : « Vous avez reçu gratuitement, donnez gratuitement » (Mt. 10 : 8). De même que le temple de Dieu était devenu une caverne de voleurs, plusieurs pasteurs ont transformé les bâtiments de leurs églises en véritables boutiques pour vendre toutes sortes de produits dérivés qui ne servent pas à l'avancement du Royaume de Dieu mais à enrichir des dirigeants cupides esclaves du dieu Mammon (Jn. 2:13-17 ; Mt. 6:24 ; Lu. 16:13 ; 1 Ti. 6:10 ; Hé. 13:5).}, tu as été rempli de violence, et tu as péché ; c'est pourquoi je te jette comme une chose souillée hors de la montagne de Dieu\FTNT{Ap. 12:1-12.}, et je te détruis d'entre les pierres éclatantes, ô chérubin protecteur !
\VS{17}Ton cœur s'est élevé à cause de ta beauté, tu as corrompu ta sagesse à cause de ton éclat ; je te jette par terre, je te donne en spectacle aux rois, afin qu'ils te regardent.
\VS{18}Tu as profané tes sanctuaires par la multitude de tes iniquités, par l'injustice de ton commerce ; et je fais sortir du milieu de toi un feu qui te consume, je te réduis en cendres sur la terre, dans la présence de tous ceux qui te regardent.
\VS{19}Tous ceux qui te connaissent parmi les peuples sont désolés à cause de toi ; tu es réduit à néant, tu ne seras plus à jamais.
\TextTitle{Jugement sur Sidon}
\VS{20}Puis la parole de Yahweh me fut adressée en ces mots :
\VS{21}Fils de l'homme, tourne ta face vers Sidon, et prophétise contre elle.
\VS{22}Tu diras : Ainsi parle le Seigneur Yahweh : Voici j'en veux à toi, Sidon ! Je serai glorifié au milieu de toi ; et on saura que je suis Yahweh, quand j'aurai exercé des jugements contre elle et que je serai sanctifié.
\VS{23}J'enverrai la peste dans son sein, je ferai couler le sang dans ses rues. Les morts tomberont au milieu d'elle par l'épée qui viendra de toutes parts sur elle ; et ils sauront que je suis Yahweh.
\VS{24}Elle ne sera plus pour la maison d'Israël une épine qui blesse, une ronce déchirante, parmi tous ceux qui l'entourent et qui la méprisent. Et ils sauront que je suis le Seigneur Yahweh.
\TextTitle{Rétablissement d'Israël}
\VS{25}Ainsi parle le Seigneur Yahweh : Quand j'aurai rassemblé la maison d'Israël d'entre les peuples parmi lesquels ils auront été dispersés, je manifesterai en elle ma sainteté, aux yeux des nations, et ils habiteront sur leur terre que j'ai donnée à mon serviteur Jacob.
\VS{26}Ils y habiteront en sûreté, ils bâtiront des maisons, ils planteront des vignes ; ils y habiteront, dis-je, en sûreté, lorsque j'aurai exercé des jugements contre ceux qui les auront pillés de toutes parts ; et ils sauront que je suis Yahweh leur Dieu.
\Chap{29}
\TextTitle{Jugement sur l'Egypte}
\VerseOne{}La dixième année, au douzième jour du dixième mois, la parole de Yahweh me fut adressée en ces mots :
\VS{2}Fils de l'homme, tourne ta face contre Pharaon, roi d'Egypte, prophétise contre lui, et contre toute l'Egypte\FTNT{Jé. 43:8-11.}.
\VS{3}Parle, et dis : Ainsi parle le Seigneur Yahweh : Voici, j'en veux à toi, Pharaon, roi d'Egypte, grand serpent couché au milieu de tes fleuves, qui dis : Mes fleuves sont à moi, et je me les suis faits\FTNT{Ps. 74:13-14 ; Es. 27:1.} !
\VS{4}C'est pourquoi je mettrai des crocs dans ta mâchoire, j'attacherai à tes écailles les poissons de tes fleuves ; je te tirerai hors de tes fleuves, avec tous les poissons de tes fleuves, qui seront attachés à tes écailles.
\VS{5}Et t'ayant tiré dans le désert, je te laisserai là, toi, et tous les poissons de tes fleuves ; tu tomberas sur la face des champs, tu ne seras point recueilli ni ramassé ; je te livrerai aux bêtes de la terre, et aux oiseaux des cieux, pour en être dévoré.
\VS{6}Et tous les habitants d'Egypte sauront que je suis Yahweh ; parce qu'ils ont été un soutien de roseau pour la maison d'Israël\FTNT{2 R. 18:21 ; Es. 36:6.}.
\VS{7}Quand ils t'ont pris par la main, tu t'es rompu, et tu leur as percé toute l'épaule ; et quand ils se sont appuyés sur toi, tu t'es cassé, et tu les as fait tomber à la renverse.
\VS{8}C'est pourquoi ainsi parle le Seigneur Yahweh : Voici, je m'en vais faire venir l'épée sur toi, et j'exterminerai du milieu de toi les hommes et les bêtes.
\VS{9}Le pays d'Egypte sera dans la désolation et dans le désert, et ils sauront que je suis Yahweh, parce que le roi d'Egypte a dit : Les fleuves sont à moi, et je les ai faits !
\VS{10}C'est pourquoi voici, j'en veux à toi, et à tes fleuves, et je réduirai le pays d'Egypte en désert de sécheresse et de désolation, depuis Migdol jusqu'à Syène, et aux frontières de l'Ethiopie.
\VS{11}Nul pied d'homme ne passera par là, et il n'y passera non plus aucun pied d'animal, elle sera quarante ans sans être habitée.
\VS{12}Car je réduirai le pays d'Egypte en désolation entre les pays désolés, et ses villes entre les villes réduites en désert ; elles seront en désolation durant quarante ans, je disperserai les Egyptiens parmi les nations, et je les répandrai parmi les pays.
\VS{13}Toutefois, ainsi parle le Seigneur Yahweh : Au bout de quarante ans, je ramasserai les Egyptiens d'entre les peuples parmi lesquels ils auront été dispersés ;
\VS{14}je ramènerai les captifs d'Egypte, et les ferai retourner au pays de Pathros, au pays de leur origine, mais ils seront là un royaume rabaissé.
\VS{15}Il sera le plus bas des royaumes, et il ne s'élèvera plus au-dessus des nations, je le diminuerai, afin qu'il ne domine point sur les nations.
\VS{16}Ce royaume ne sera plus pour la main d'Israël un sujet de confiance ; il lui rappellera son iniquité, quand elle se tournait vers eux ; et ils sauront que je suis le Seigneur Yahweh.
\VS{17}Il arriva la vingt-septième année, au premier jour du premier mois, que la parole de Yahweh me fut adressée en ces mots :
\VS{18}Fils de l'homme, Nebucadnetsar, roi de Babylone, a fait servir son armée dans un service pénible contre Tyr ; toute tête en est devenue chauve, et toute épaule en a été foulée, mais il n'a point eu de salaire, ni lui ni son armée, à cause de Tyr, pour le service qu'il a fait contre elle.
\VS{19}C'est pourquoi ainsi parle le Seigneur Yahweh : Voici, je m'en vais donner à Nebucadnetsar, roi de Babylone, le pays d'Egypte ; il enlèvera la multitude, il emportera le butin et fera le pillage ; ce sera là le salaire de son armée.
\VS{20}Pour prix du service qu'il a fait contre Tyr, je lui ai donné le pays d'Egypte, parce qu'ils ont travaillé pour moi, dit le Seigneur Yahweh.
\VS{21}En ce jour-là, je ferai germer la corne de la maison d'Israël, et j'ouvrirai ta bouche au milieu d'eux, et ils sauront que je suis Yahweh.
\Chap{30}
\TextTitle{Disgrâce de l'Egypte}
\VerseOne{}La parole de Yahweh me fut encore adressée en ces mots :
\VS{2}Fils de l'homme, prophétise, et dis : Ainsi parle le Seigneur Yahweh : Hurlez, et dites : Malheureux jour !
\VS{3}Car le jour est proche, oui le jour de Yahweh est proche, c'est un jour ténébreux ; ce sera le temps des nations.
\VS{4}L'épée viendra sur l'Egypte, et il y aura de l'effroi en Ethiopie, quand ceux qui seront blessés à mort tomberont dans l'Egypte, et quand on enlèvera la multitude de son peuple, et que ses fondements seront ruinés.
\VS{5}L'Ethiopie, Puth, Lud, toute l'Arabie, Cub, et les fils du pays allié tomberont par l'épée avec eux\FTNT{Jé. 46:9 ; Na. 3:9-10.}.
\VS{6}Ainsi parle Yahweh : Ceux qui soutiendront l'Egypte, tomberont ; et l'orgueil de sa force sera renversé ; ils tomberont par l'épée de Migdol à Syène, dit le Seigneur Yahweh.
\VS{7}Ils seront désolés au milieu des pays désolés, et ses villes seront au milieu des villes désertes.
\VS{8}Ils sauront que je suis Yahweh, quand j'aurai mis le feu en Egypte ; et tous ceux qui lui donneront du secours, seront brisés.
\VS{9}En ce jour-là, des messagers sortiront de ma part sur des navires pour effrayer l'Ethiopie dans sa sécurité, et il y aura entre eux un tourment au jour de l'Egypte ; car voici, il vient.
\VS{10}Ainsi parle le Seigneur Yahweh : Je ferai périr la multitude d'Egypte par la puissance de Nebucadnetsar, roi de Babylone.
\VS{11}Lui et son peuple avec lui, les plus terribles d'entre les nations, seront amenés pour ruiner le pays, et ils tireront leurs épées contre les Egyptiens, et rempliront la terre de morts.
\VS{12}Je mettrai à sec les fleuves et je livrerai le pays entre les mains des méchants ; je désolerai le pays, et tout ce qui y est, par la puissance des étrangers ; moi, Yahweh, j'ai parlé.
\VS{13}Ainsi parle le Seigneur Yahweh : Je détruirai aussi les idoles, j'anéantirai les faux dieux de Noph, et il n'y aura point de prince qui soit du pays d'Egypte ; je mettrai la frayeur dans le pays d'Egypte\FTNT{Es. 19:1-13 ; Jé. 43:12 ; Jé. 46:13.}.
\VS{14}Je désolerai Pathros, je mettrai le feu à Tsoan, et j'exercerai mes jugements sur No\FTNT{Jé. 44:1.}.
\VS{15}Je répandrai ma fureur sur Sin, qui est la place forte de l'Egypte, et j'exterminerai la multitude qui est à No.
\VS{16}Quand je mettrai le feu en Egypte, Sin sera grièvement tourmentée, et No sera rompue par diverses brèches, et il n'y aura à Noph que détresses en plein jour.
\VS{17}Les jeunes hommes d'On et de Pi-Béseth tomberont par l'épée, et ces villes iront en captivité.
\VS{18}Le jour s'obscurcira à Tachpanès, lorsque j'y romprai le joug de l'Egypte, et que l'orgueil de sa force aura cessé ; un nuage la couvrira, et les villes de son ressort iront en captivité.
\VS{19}J'exercerai des jugements en Egypte ; et ils sauront que je suis Yahweh.
\TextTitle{Chute et dispersion de l'Egypte}
\VS{20}Dans la onzième année, au septième jour du premier mois, la parole de Yahweh me fût adressée en ces mots :
\VS{21}Fils de l'homme, j'ai rompu le bras de Pharaon, roi d'Egypte ; et voici on ne l'a point bandé pour le guérir, on ne lui a point mis de linges pour le bander, et pour le fortifier, afin qu'il puisse manier l'épée.
\VS{22}C'est pourquoi ainsi parle le Seigneur Yahweh : Voici, j'en veux à Pharaon, roi d'Egypte, et je romprai ses bras, tant celui qui est fort que celui qui est rompu, et je ferai tomber l'épée de sa main.
\VS{23}Je disperserai les Egyptiens parmi les nations, et les répandrai parmi les pays.
\VS{24}Je fortifierai les bras du roi de Babylone, je lui mettrai mon épée dans la main ; mais je romprai les bras de Pharaon, et il gémira devant lui comme gémissent les mourants.
\VS{25}Je fortifierai donc les bras du roi de Babylone, mais les bras de Pharaon tomberont ; et on saura que je suis Yahweh, quand j'aurai mis mon épée dans la main du roi de Babylone, et qu'il l'a tournera contre le pays d'Egypte.
\VS{26}Je disperserai les Egyptiens parmi les nations, les répandrai parmi les pays ; et ils sauront que je suis Yahweh.
\Chap{31}
\TextTitle{Avertissement contre l'arrogance de Pharaon}
\VerseOne{}Il arriva aussi dans la onzième année, au premier jour du troisième mois, que la parole de Yahweh me fut adressée en ces mots :
\VS{2}Fils de l'homme, parle à Pharaon, roi d'Egypte, et à la multitude de son peuple : A qui ressembles-tu dans ta grandeur ?
\VS{3}Voici, le roi d'Assyrie a été comme un cèdre du Liban, ayant de belles branches, et des rameaux qui faisaient une grande ombre, et qui étaient d'une grande hauteur ; sa cime était fort touffue.
\VS{4}Les eaux l'ont fait croître, l'abîme l'a fait pousser en hauteur, ses fleuves ont coulé autour de ses plantes, et il a envoyé ses eaux abondantes vers tous les arbres des champs.
\VS{5}C'est pourquoi il s'est élevé au-dessus de tous les autres arbres des champs, ses branches se sont multipliées, et ses rameaux croissaient par les grandes eaux qui faisaient pousser ses branches.
\VS{6}Tous les oiseaux des cieux ont fait leurs nids dans ses branches, toutes les bêtes des champs ont fait leurs petits sous ses rameaux, et toutes les grandes nations ont habité sous son ombre.
\VS{7}Il était beau par sa grandeur, et par l'étendue de ses branches, parce que sa racine était sur de grandes eaux.
\VS{8}Les cèdres du jardin de Dieu ne le surpassaient point ; les cyprès n'égalaient point ses branches, et les platanes n'égalaient point comme ses rameaux ; aucun arbre du jardin de Dieu ne lui était comparable en beauté.
\VS{9}Je l'avais embelli par la multitude de ses rameaux, au point que tous les arbres d'Eden, qui étaient dans le jardin de Dieu, lui portaient envie.
\VS{10}C'est pourquoi le Seigneur Yahweh dit : Parce qu'il s'est élevé, parce qu'il lançait sa cime au milieu d'épais rameaux et que son cœur était fier de sa hauteur,
\VS{11}je l'ai livré entre les mains du plus fort des nations, qui l'a traité comme il fallait, et je l'ai chassé à cause de sa méchanceté.
\VS{12}Les étrangers les plus terrifiants parmi les nations l'ont coupé et l'ont laissé là, ses branches sont tombées sur les montagnes et sur toutes les vallées ; ses rameaux se sont rompus dans tous les ravins de la terre, et tous les peuples de la terre se sont retirés de dessous son ombre, et l'ont laissé là.
\VS{13}Tous les oiseaux des cieux se sont tenus sur ses ruines, et toutes les bêtes des champs se sont retirées vers ses rameaux,
\VS{14}afin que tous les arbres près des eaux n'élèvent plus leur hauteur, et qu'ils ne lancent plus leur cime au milieu d'épais rameaux, afin que tous les chênes arrosés d'eau ne gardent plus leur hauteur ; car tous sont livrés à la mort, aux profondeurs de la terre, parmi les fils des hommes, avec ceux qui descendent dans la fosse.
\VS{15}Ainsi parle le Seigneur Yahweh : Le jour qu'il descendit dans le scheol, j'ai répandu deuil sur lui, j'ai couvert l'abîme devant lui, j'ai empêché ses fleuves de couler, et les grosses eaux ont été retenues ; j'ai fait que le Liban soit en deuil à cause de lui, et tous les arbres des champs ont été desséchés.
\VS{16}J'ai ébranlé les nations par le bruit de sa ruine, quand je l'ai fait descendre dans le scheol, avec ceux qui descendent dans la fosse\FTNT{Es. 14:9.} ; et tous les arbres d'Eden, les plus beaux et les plus agréables du Liban, tous arrosés par les eaux, ont été consolés dans les profondeurs de la terre.
\VS{17}Eux aussi sont descendus avec lui dans le scheol, vers ceux qui ont péri par l'épée ; ils étaient son bras et ils habitaient sous son ombre parmi les nations.
\VS{18}A qui ressembles-tu ainsi en gloire et en grandeur parmi les arbres d'Eden ? Tu seras précipité avec les arbres d'Eden dans les profondeurs de la terre, tu seras gisant au milieu des incirconcis, avec ceux qui ont péri par l'épée. Voilà Pharaon et toute sa multitude ! dit le Seigneur Yahweh.
\Chap{32}
\TextTitle{Lamentation sur le pays d'Egypte}
\VerseOne{}Dans la douzième année, le premier jour du douzième mois, la parole de Yahweh me fut adressée en ces mots :
\VS{2}Fils de l'homme, prononce à haute voix une complainte sur Pharaon, roi d'Egypte, et dis-lui : Tu as été parmi les nations semblable à un lionceau, et comme un serpent dans les mers ; tu t'élançais dans tes fleuves, et tu troublais les eaux avec tes pieds, et remplissais de bourbe leurs fleuves.
\VS{3}Ainsi parle le Seigneur Yahweh : J'étendrai mon rets sur toi dans une assemblée nombreuse de peuples qui te tireront dans mes filets.
\VS{4}Je te laisserai à l'abandon sur la terre ; je te jetterai sur le dessus des champs, et je ferai demeurer sur toi tous les oiseaux des cieux, et rassasierai de toi les bêtes de toute la terre.
\VS{5}Car je mettrai ta chair sur les montagnes, et je remplirai les vallées de tes débris.
\VS{6}J'arroserai de ton sang jusqu'aux montagnes, la terre où tu nages, et les lits des eaux seront remplis de toi.
\VS{7}Quand je t'éteindrai, je couvrirai les cieux et j'obscurcirai leurs étoiles, je couvrirai le soleil de nuages, et la lune ne donnera plus sa lumière\FTNT{Es. 13:10 ; Joë. 2:31 ; Mt. 24:29.}.
\VS{8}J'obscurcirai à cause de toi tous les luminaires des cieux, et je répandrai les ténèbres sur ton pays, dit le Seigneur Yahweh.
\VS{9}J'affligerai le cœur de beaucoup de peuples, quand j'annoncerai ta ruine parmi les nations, à des pays que tu ne connaissais pas.
\VS{10}Je frapperai de stupeur beaucoup de peuples à cause de toi, et leurs rois seront saisis d'épouvante à cause de toi, quand je ferai luire mon épée à leurs yeux ; ils seront effrayés à chaque instant, chacun pour sa vie, au jour de ta ruine.
\VS{11}Car ainsi parle le Seigneur Yahweh : L'épée du roi de Babylone viendra sur toi.
\VS{12}J'abattrai ta multitude par les épées des hommes forts, qui tous sont les plus terribles d'entre les nations ; ils détruiront l'orgueil de l'Egypte, et toute la multitude de son peuple sera ruinée.
\VS{13}Je ferai périr tout son bétail près des grandes eaux, et aucun pied d'homme ne les troublera plus, ni aucun pied d'animaux ne les agitera plus.
\VS{14}Alors je rendrai profondes leurs eaux, et je ferai couler leurs fleuves comme de l'huile, dit le Seigneur Yahweh.
\VS{15}Quand j'aurai réduit le pays d'Egypte en désolation, et que le pays sera dénué des choses dont il était rempli ; quand je frapperai tous ceux qui y habitent, ils sauront alors que je suis Yahweh.
\VS{16}C'est ici la complainte qu'on fera sur elle, les filles des nations feront cette complainte sur elle ; elles feront cette complainte sur l'Egypte et sur toute la multitude de son peuple, dit le Seigneur.
\VS{17}Il arriva aussi dans la douzième année, le quinzième jour du mois, que la parole de Yahweh me fut adressée en ces mots :
\VS{18}Fils de l'homme, dresse une lamentation sur la multitude d'Egypte, et fais-la descendre, elle et les filles des nations magnifiques, aux plus bas lieux de la terre, avec ceux qui descendent dans la fosse\FTNT{Jé. 1:10 ; Jé. 18:7.}.
\VS{19}Qui surpasses-tu en beauté ? Descends, et couche-toi avec les incirconcis !
\VS{20}Ils tomberont au milieu de ceux qui seront tués par l'épée. L'épée a déjà été donnée : Entraînez l'Egypte et toute sa multitude !
\VS{21}Les plus forts d'entre les puissants lui parleront du milieu du scheol, avec ceux qui lui donnaient du secours, et diront : Ils sont descendus, ils sont couchés, les incirconcis, tués par l'épée.
\VS{22}Là est l'Assyrien, et toute son assemblée ; ses sépulcres sont autour de lui, eux tous, mis à mort, sont tombés par l'épée.
\VS{23}Car ses sépulcres sont posés au fond de la fosse et son assemblée autour de sa sépulture ; eux tous qui avaient répandu leur terreur sur la terre des vivants sont tombés morts par l'épée.
\VS{24}Là est Elam et toute sa multitude autour de son sépulcre ; eux tous sont tombés morts par l'épée, ils sont descendus incirconcis dans les plus bas lieux de la terre ; et après avoir répandu leur terreur sur la terre des vivants, ils ont porté leur ignominie avec ceux qui descendent dans la fosse.
\VS{25}On a mis sa couche parmi ceux qui ont été tués, avec toute sa multitude ; ses sépulcres sont autour de lui ; eux tous incirconcis, tués par l'épée, quoiqu'ils aient répandu leur terreur sur la terre des vivants, toutefois ils ont porté leur ignominie avec ceux qui descendent dans la fosse ; ils ont été placés parmi les morts.
\VS{26}Là est Méschec, Tubal, et toute leur multitude ; leurs sépulcres sont autour d'eux ; eux tous incirconcis, tués par l'épée, quoiqu'ils aient répandu leur terreur sur la terre des vivants.
\VS{27}Ils ne se sont point couchés avec les hommes vaillants qui sont tombés d'entre les incirconcis, lesquels sont descendus dans le scheol avec leurs armes de guerre, dont on a mis les épées sous leurs têtes, et dont les iniquités ont reposé sur leurs os ; parce que la terreur des hommes forts est dans la terre des vivants.
\VS{28}Toi aussi tu seras brisé au milieu des incirconcis, et tu seras couché avec ceux qui sont tués par l'épée.
\VS{29}Là est Edom, ses rois, et tous ses princes, qui ont été placés malgré leur force avec ceux qui sont tués par l'épée ; ils seront couchés avec les incirconcis, et avec ceux qui sont descendus dans la fosse.
\VS{30}Là sont tous les princes du nord, et tous les Sidoniens, qui sont descendus avec ceux qui sont tués, malgré la terreur qu'inspirait leur force ; ils sont couchés incirconcis avec ceux qui sont tués par l'épée ; ils ont porté leur ignominie avec ceux qui sont descendus dans la fosse.
\VS{31}Pharaon les verra, et il se consolera au sujet de toute la multitude de son peuple ; Pharaon, dit le Seigneur Yahweh, verra les blessés par l'épée et toute son armée.
\VS{32}Car je mettrai ma terreur dans la terre des vivants, c'est pourquoi Pharaon avec toute la multitude de son peuple se couchera au milieu des incirconcis, avec ceux qui sont tués par l'épée, dit le Seigneur Yahweh.
\Chap{33}
\TextTitle{Ezéchiel établi comme sentinelle pour avertir le pécheur}
\VerseOne{}La parole de Yahweh me fut encore adressée en ces mots :
\VS{2}Fils de l'homme, parle aux fils de ton peuple, et dis-leur : Quand je ferai venir l'épée sur un pays, et que le peuple du pays aura choisi quelqu'un d'entre eux, et l'aura établi pour leur servir de sentinelle,
\VS{3}et que voyant venir l'épée sur le pays, il sonnera du shofar et avertira le peuple,
\VS{4}si le peuple ayant bien entendu le son du shofar, ne se tient pas sur ses gardes, et qu'ensuite l'épée vienne le prendre, son sang sera sur sa tête.
\VS{5}Car il a entendu le son du shofar, et ne s'est point tenu sur ses gardes ; son sang sera sur lui ; mais s'il se tient sur ses gardes, il sauvera sa vie.
\VS{6}Si la sentinelle voit venir l'épée, et qu'elle ne sonne point du shofar, en sorte que le peuple ne se tienne point sur ses gardes, et qu'ensuite l'épée survienne et ôte la vie à l'un d'entre eux, celui-ci sera emmené en captivité à cause de son iniquité, mais je redemanderai son sang de la main de la sentinelle.
\VS{7}Toi donc, fils de l'homme, je t'ai établi pour sentinelle sur la maison d'Israël ; tu écouteras donc la parole qui sort de ma bouche, et tu les avertiras de ma part.
\VS{8}Quand j'aurai dit au méchant : Méchant, tu mourras ! et que tu n'auras point parlé au méchant pour l'avertir de se détourner de sa voie, ce méchant mourra dans son iniquité ; mais je redemanderai son sang de ta main.
\VS{9}Mais si tu as averti le méchant de se détourner de sa voie, et qu'il ne se détourne pas de sa voie, il mourra dans son iniquité ; mais toi tu auras délivré ton âme.
\VS{10}Toi donc, fils de l'homme, dis à la maison d'Israël : Vous avez parlé ainsi, en disant : Puisque nos crimes et nos péchés sont sur nous, et que nous périssons à cause d'eux, comment pourrions-nous vivre\FTNT{Lé. 26:39.} ?
\VS{11}Dis-leur : Je suis vivant, dit le Seigneur Yahweh, je ne prends point plaisir dans la mort du méchant, mais que le méchant se détourne de sa voie et qu'il vive. Détournez-vous, détournez-vous de votre méchante voie ! Pourquoi mourriez-vous, maison d'Israël ?
\VS{12}Toi donc, fils de l'homme, dis aux fils de ton peuple : La justice du juste ne le délivrera point au jour de son péché, le méchant ne tombera point par sa méchanceté au jour où il s'en détournera ; et le juste ne pourra pas vivre par sa justice au jour de son péché.
\VS{13}Quand j'aurai dit au juste qu'il vivra certainement, et que lui, se confiant sur sa justice, aura commis l'iniquité, on ne se souviendra plus d'aucune de ses justices, mais il mourra dans son iniquité qu'il aura commise.
\VS{14}Aussi quand j'aurai dit au méchant : Tu mourras ! s'il se détourne de son péché, et qu'il fasse ce qui est juste et droit ;
\VS{15}si le méchant rend le gage et qu'il restitue ce qu'il aura ravi, et qu'il marche dans les statuts de la vie, sans commettre d'iniquité, certainement il vivra, il ne mourra point.
\VS{16}On ne se souviendra plus des péchés qu'il aura commis ; il a fait ce qui est juste et droit ; certainement il vivra.
\VS{17}Or les fils de ton peuple ont dit : La voie du Seigneur n'est pas bien réglée ; mais c'est plutôt leur voie qui n'est pas bien réglée.
\VS{18}Quand le juste se détournera de sa justice, et qu'il commettra l'iniquité, il mourra à cause de cela.
\VS{19}Quand le méchant se détournera de sa méchanceté, et qu'il fera ce qui est juste et droit, il vivra à cause de cela.
\VS{20}Vous avez dit : La voie du Seigneur n'est pas bien réglée ! Je vous jugerai, maison d'Israël, chacun selon sa voie.
\TextTitle{Exécution du jugement de Yahweh}
\VS{21}Or il arriva dans la douzième année de notre captivité, au cinquième jour du dixième mois, qu'un homme qui s'était échappé de Jérusalem vint vers moi, en disant : La ville est prise !
\VS{22}La main de Yahweh fut sur moi le soir, avant l'arrivée du fugitif, et Yahweh ouvrit ma bouche lorsqu'il vint auprès de moi le matin. Ma bouche était ouverte et je n'étais plus muet.
\TextTitle{Ne pas se contenter d'écouter la Parole de Dieu}
\VS{23}La parole de Yahweh me fut adressée en ces mots :
\VS{24}Fils de l'homme, ceux qui habitent dans ces ruines, sur la terre d'Israël, discourent en disant : Abraham était seul, et il a possédé le pays\FTNT{Ge. 15:7.}; mais nous sommes un grand nombre de gens, et le pays nous a été donné en héritage.
\VS{25}C'est pourquoi tu leur diras : Ainsi parle le Seigneur Yahweh : Vous mangez la chair avec le sang, et vous levez vos yeux vers vos idoles, vous répandez le sang ; et vous posséderiez le pays\FTNT{Ge. 9:4 ; Lé. 3:17 ; Lé. 17:10.} ?
\VS{26}Vous vous appuyez sur votre épée, vous commettez des abominations, vous souillez chacun de vous la femme de son prochain ; et vous posséderiez le pays ?
\VS{27}Tu leur diras : Ainsi parle le Seigneur Yahweh : Je suis vivant, ceux qui sont dans ces ruines tomberont par l'épée, et je livrerai aux bêtes celui qui est dans les champs, afin qu'elles le mangent ; et ceux qui sont dans les forteresses et dans les cavernes mourront par la peste.
\VS{28}Ainsi je réduirai le pays en désolation et en désert, l'orgueil de sa force sera aboli, et les montagnes d'Israël seront désolées, en sorte qu'il n'y passera plus personne.
\VS{29}Ils sauront que je suis Yahweh, quand j'aurai réduit leur pays en désolation et en désert, à cause de toutes leurs abominations qu'ils ont commises.
\VS{30}Quant à toi, fils de l'homme, les fils de ton peuple parlent de toi près des murs et aux entrées des maisons, et parlent l'un à l'autre, chacun avec son prochain, en disant : Venez maintenant, et écoutez la parole qui vient de Yahweh.
\VS{31}Ils viennent vers toi en foule, et mon peuple s'assied devant toi, ils écoutent tes paroles, mais ils ne les mettent point en pratique ; ils les répètent comme si c'était une chanson profane, mais leur cœur marche toujours après leur gain déshonnête.
\VS{32}Voici tu es pour eux comme un homme qui leur chante une chanson profane avec une belle voix, qui résonne bien ; car ils écoutent bien tes paroles, mais ils ne les mettent point en pratique.
\VS{33}Mais quand ces choses arriveront, et voici, elles arrivent, ils sauront qu'il y avait un prophète au milieu d'eux.
\Chap{34}
\TextTitle{Jugement de Dieu sur les faux bergers}
\VerseOne{}La parole de Yahweh me fut encore adressée en ces mots :
\VS{2}Fils de l'homme, prophétise contre les pasteurs d'Israël\FTNT{Les faux pasteurs prennent en otage les brebis du Seigneur (Jé. 23). La véritable fonction pastorale consiste au service envers les frères et sœurs et non le contraire. Un vrai pasteur sert les autres, il n'aime pas être servi comme un roi. Il ne dit pas aux autres de faire les choses, mais il les fait et les autres l'imitent (Jn. 10).} ! Prophétise, et dis à ces pasteurs : Ainsi parle le Seigneur Yahweh : Malheur aux pasteurs d'Israël ! Qui ne paissent qu'eux-mêmes ! Les pasteurs ne paissent-ils pas le troupeau ?
\VS{3}Vous en mangez la graisse, et vous vous habillez de laine ; vous tuez ce qui est gras, vous ne paissez point le troupeau !
\VS{4}Vous n'avez point fortifié les brebis languissantes, vous n'avez point donné de remède à celle qui était malade, vous n'avez point bandé la plaie de celle qui avait la jambe rompue, vous n'avez point ramené celle qui était chassée, et vous n'avez point cherché celle qui était perdue\FTNT{Lu. 15:4-6 ; 1 Pi. 5:1-3.} ; mais vous les avez maîtrisées avec dureté et rigueur.
\VS{5}Elles se sont dispersées, parce qu'elles n'avaient pas de pasteurs, et elles se sont exposées à toutes les bêtes des champs, pour en être dévorées, étant dispersées.
\VS{6}Mes brebis sont errantes sur toutes les montagnes, et sur toutes les collines élevées ; mes brebis sont dispersées sur toute la surface de la terre ; et il n'y a personne qui les recherche, et il n'y a personne pour s'en soucier\FTNT{Mc. 14:27 ; Za. 13:7 ; Mt. 26:31.}.
\VS{7}C'est pourquoi pasteurs, écoutez la parole de Yahweh :
\VS{8}Je suis vivant, dit le Seigneur Yahweh, parce que mes brebis sont pillées, et que mes brebis sont la nourriture de toutes les bêtes des champs, parce qu'elles n'ont point de pasteur ; car mes pasteurs n'ont point recherché mes brebis, mais les pasteurs se sont nourris simplement eux-mêmes, et n'ont point fait paître mes brebis.
\VS{9}C'est pourquoi pasteurs, écoutez la parole de Yahweh !
\VS{10}Ainsi parle le Seigneur Yahweh : Voici, j'en veux à ces pasteurs-là, et je redemanderai mes brebis de leur main ; ils cesseront de paître les brebis, et les pasteurs ne se repaîtront plus eux-mêmes, mais je délivrerai mes brebis de leur bouche, et elles ne seront plus dévorées par eux.
\TextTitle{Yahweh, le bon berger qui restaure son troupeau\FTNT{Jn. 10:1-18}}
\VS{11}Car ainsi parle le Seigneur Yahweh : Me voici, je redemanderai mes brebis, et je les rechercherai.
\VS{12}Comme le pasteur prend soin de son troupeau quand il est au milieu de ses brebis dispersées, ainsi je rechercherai mes brebis, et les retirerai de tous les lieux où elles auront été dispersées au jour des nuages et de l'obscurité.
\VS{13}Je les retirerai d'entre les peuples et les rassemblerai des territoires, les ramènerai dans leur terre, et les nourrirai sur les montagnes d'Israël, auprès des cours d'eau et dans toutes les demeures du pays.
\VS{14}Je les paîtrai dans de bons pâturages, et leur demeure sera sur les hautes montagnes d'Israël ; et là elles coucheront dans une agréable demeure, et paîtront dans de gras pâturages, sur les montagnes d'Israël.
\VS{15}Moi-même je paîtrai mes brebis et les ferai reposer, dit le Seigneur Yahweh\FTNT{Ps. 23.}.
\VS{16}Je chercherai celle qui était perdue, et je ramènerai celle qui était chassée, je banderai la plaie de celle qui a la jambe rompue, et je fortifierai celle qui est malade ; mais je détruirai la grasse et la forte ; je les paîtrai avec justice.
\VS{17}Quant à vous, mes brebis, ainsi parle le Seigneur Yahweh : Voici, je m'en vais mettre à part les brebis, les béliers, et les boucs.
\VS{18}Et vous, est-ce peu de chose de vous faire paître dans de bons pâturages, pour que vous fouliez de vos pieds le reste de votre pâture ? Et de boire des eaux claires, pour que vous troubliez le reste avec vos pieds ?
\VS{19}Mais mes brebis sont nourries du pâturage que vous foulez de vos pieds, et boivent ce que vos pieds ont troublé.
\VS{20}C'est pourquoi le Seigneur Yahweh leur dit : Me voici, je mettrai moi-même à part la brebis grasse et la brebis maigre.
\VS{21}Parce que vous poussez du côté et de l'épaule, et que vous heurtez de vos cornes toutes celles qui sont languissantes, jusqu'à ce que vous les ayez chassées dehors,
\VS{22}je sauverai mes brebis, au point qu'elles ne seront plus au pillage. Voici, je jugerai entre brebis et brebis.
\VS{23}Je susciterai sur elles un pasteur qui les paîtra, mon serviteur David ; il les paîtra, et lui-même sera leur pasteur.
\VS{24}Moi, Yahweh, je serai leur Dieu, et mon serviteur David sera prince au milieu d'elles ; moi,Yahweh, j'ai parlé.
\VS{25}Je traiterai avec elles une alliance de paix ; et je détruirai dans le pays les mauvaises bêtes ; les brebis habiteront dans le désert en sécurité, et dormiront dans les forêts.
\VS{26}Je les comblerai de bénédictions, elles, et tous les environs de mes collines ; je ferai tomber la pluie en sa saison ; ce seront des pluies de bénédiction.
\VS{27}Les arbres des champs produiront leur fruit, et la terre rapportera son revenu ; elles seront dans leur terre en sécurité, et sauront que je suis Yahweh, quand j'aurai rompu les bois de leur joug, et que je les aurai délivrées de la main de ceux qui se les asservissent.
\VS{28}Elles ne seront plus au pillage parmi les nations, et les bêtes de la terre ne les dévoreront plus ; mais elles habiteront en sécurité, et il n'y aura personne pour les effrayer.
\VS{29}Je leur susciterai une plantation de renom ; elles ne mourront plus de faim sur la terre, et ne porteront plus l'opprobre des nations.
\VS{30}Ils sauront que moi, Yahweh, leur Dieu, suis avec eux, et qu'eux, la maison d'Israël, sont mon peuple, dit le Seigneur Yahweh.
\VS{31}Or vous êtes mes brebis, vous hommes, les brebis de mon pâturage, et je suis votre Dieu, dit le Seigneur Yahweh.
\Chap{35}
\TextTitle{Jugement sur Edom}
\VerseOne{}La parole de Yahweh me fut encore adressée en ces mots :
\VS{2}Fils de l'homme, tourne ta face contre la montagne de Séir, et prophétise contre elle\FTNT{Am. 1:11.}.
\VS{3}Dis-lui : Ainsi parle le Seigneur Yahweh : Voici, j'en veux à toi, montagne de Séir, et j'étendrai ma main contre toi, et te réduirai en désolation et en désert.
\VS{4}Je réduirai tes villes en désert, tu ne seras que désolation, et tu sauras que je suis Yahweh.
\VS{5}Parce que tu as eu une inimitié immortelle, et que tu as fait couler le sang des fils d'Israël à coups d'épée, au temps de leur détresse, au temps où l'iniquité était à son terme\FTNT{Ps. 137:7.}.
\VS{6}C'est pourquoi je suis vivant, dit le Seigneur Yahweh, je te mettrai à sang, et le sang te poursuivra ; parce que tu n'as point haï le sang, le sang aussi te poursuivra.
\VS{7}Je réduirai la montagne de Séir en désolation et en désert, et j'en éloignerai tous ceux qui la fréquentaient.
\VS{8}Je remplirai de morts ses montagnes ; tes hommes tués par l'épée tomberont sur tes collines, dans tes vallées, et dans tous tes courants d'eau.
\VS{9}Je te réduirai en désolations éternelles, et tes villes ne seront plus habitées ; vous saurez que je suis Yahweh.
\VS{10}Parce que tu as dit : Les deux nations, et les deux pays seront à moi, nous les posséderons, quand même Yahweh était là ;
\VS{11}à cause de cela, je suis vivant, dit le Seigneur Yahweh, j'agirai avec la colère et la jalousie que tu as montrées dans ta haine contre eux ; et je me ferai connaître au milieu d'eux, quand je te jugerai.
\VS{12}Tu sauras que moi, Yahweh, j'ai entendu toutes les paroles insultantes que tu as prononcées contre les montagnes d'Israël, en disant : Elles sont dévastées, elles nous sont livrées comme une proie.
\VS{13}Vous m'avez bravé par vos discours, et vous avez multiplié vos paroles contre moi ; je l'ai entendu.
\VS{14}Ainsi parle le Seigneur Yahweh : Quand toute la terre se réjouira, je te réduirai en désolation.
\VS{15}Comme tu t'es réjouie sur l'héritage de la maison d'Israël et de sa désolation, j'en ferai de même envers toi ; tu ne seras que désolation, ô montagne de Séir ! Ainsi qu'Edom tout entier ; et ils sauront que je suis Yahweh\FTNT{Ab. 1:11-16.}.
\Chap{36}
\TextTitle{Yahweh rétablit Israël}
\VerseOne{}Toi, fils de l'homme, prophétise sur les montagnes d'Israël, et dis : Montagnes d'Israël, écoutez la parole de Yahweh !
\VS{2}Ainsi parle le Seigneur Yahweh : Parce que l'ennemi a dit contre vous : Ah ! Ah ! Tous ces hauts lieux éternels sont devenus notre possession !
\VS{3}Prophétise, et dis : Ainsi parle le Seigneur Yahweh : Oui, parce qu'on vous a réduites en désolation, et que de toutes parts, on vous a englouties pour que vous soyez la propriété des autres nations, et qu'on vous a exposées à la langue et aux insultes des nations,
\VS{4}à cause de cela, montagnes d'Israël, écoutez la parole du Seigneur Yahweh : Ainsi parle le Seigneur Yahweh, aux montagnes, aux collines, aux courants d'eau, aux vallées, aux lieux détruits et désolés, et aux villes abandonnées qui sont pillées et sont un sujet de moquerie aux autres nations d'alentour ;
\VS{5}à cause de cela, ainsi parle le Seigneur Yahweh : Je parle dans le feu de ma jalousie contre les autres nations, et contre tous ceux d'Edom qui se sont attribués ma terre en possession, avec toute la joie de leur cœur et le mépris de leur âme, afin d'en piller le butin\FTNT{Lé. 25:23 ; Es. 14:2 ; Jé. 2:7.}.
\VS{6}C'est pourquoi prophétise sur la terre d'Israël, et dis aux montagnes et aux collines, aux courants d'eau et aux vallées : Ainsi parle le Seigneur Yahweh : Voici, j'ai parlé avec jalousie, et avec fureur, parce que vous avez porté l'ignominie des nations.
\VS{7}C'est pourquoi ainsi parle le Seigneur Yahweh : J'ai levé ma main, si les nations qui sont tout autour de vous ne portent leur ignominie.
\VS{8}Mais vous, montagnes d'Israël, vous pousserez vos branches, et vous porterez votre fruit pour mon peuple d'Israël ; car ils sont prêts à venir.
\VS{9}Car me voici, je viens à vous, et je retournerai vers vous, et vous serez labourées et semées.
\VS{10}Je mettrai sur vous des hommes en grand nombre, la maison d'Israël tout entière, et les villes seront habitées, les lieux déserts seront rebâtis.
\VS{11}Je multiplierai sur vous les hommes et les animaux, ils multiplieront et seront féconds ; je veux que vous soyez habitées comme auparavant, et je vous ferai plus de bien que vous n'en avez eu au commencement ; et vous saurez que je suis Yahweh.
\VS{12}Je ferai marcher sur vous des hommes, mon peuple d'Israël, qui vous posséderont, vous serez leur héritage, et vous ne les consumerez plus.
\VS{13}Ainsi parle le Seigneur Yahweh : Parce qu'on dit de vous : Tu es un pays qui dévore les hommes, et tu as consumé tes habitants ;
\VS{14}à cause de cela, tu ne dévoreras plus les hommes et ne consumeras plus tes habitants, dit le Seigneur Yahweh.
\VS{15}Je ne te ferai plus entendre l'ignominie des nations, tu ne porteras plus l'opprobre des peuples ; et tu ne feras plus périr tes habitants, dit le Seigneur Yahweh.
\VS{16}Puis la parole de Yahweh me fut adressée en ces mots :
\VS{17}Fils de l'homme, ceux de la maison d'Israël habitant sur leur terre l'ont souillée par leur voie et par leurs actions ; leur voie est devenue devant moi comme la souillure d'une femme pendant son impureté\FTNT{Lé. 12:2 ; Lé. 15:19.} ;
\VS{18}j'ai répandu ma fureur sur eux à cause du sang qu'ils ont répandu sur le pays, et parce qu'ils l'ont souillé par leurs idoles.
\VS{19}Je les ai dispersés parmi les nations, et ils ont été disséminés en divers pays ; je les ai jugés selon leur voie, et selon leurs actions.
\VS{20}Ils sont arrivés chez les nations où ils allaient, ils ont profané mon saint Nom en sorte qu'on disait d'eux : Ceux-ci sont le peuple de Yahweh, c'est de son pays qu'ils sont sortis\FTNT{Ro. 2:24.}.
\VS{21}Mais j'ai épargné mon saint Nom, que la maison d'Israël avait profané parmi les nations où elle est allée.
\VS{22}C'est pourquoi dis à la maison d'Israël : Ainsi parle le Seigneur Yahweh : Je ne le fais point à cause de vous, ô maison d'Israël ! Mais à cause de mon saint Nom, que vous avez profané parmi les nations où vous êtes allés\FTNT{De. 7:7 ; De. 9:5 ; Ps. 25:11 ; Es. 43:25.}.
\VS{23}Je sanctifierai mon grand nom, qui a été profané parmi les nations, et que vous avez profané au milieu d'elles ; et les nations sauront que je suis Yahweh, dit le Seigneur Yahweh, quand je serai sanctifié par vous, sous leurs yeux.
\VS{24}Je vous retirerai d'entre les nations, je vous rassemblerai de tous les pays, et je vous ramènerai dans votre terre.
\VS{25}Je répandrai sur vous une eau pure\FTNT{Il est question ici de la Nouvelle Alliance (Jé. 31:31-34 ; Hé. 8:7-13).}, et vous serez nettoyés ; je vous nettoierai de toutes vos souillures et de toutes vos idoles.
\TextTitle{Prophétie sur la naissance d'en haut}
\VS{26}Je vous donnerai un nouveau cœur, je mettrai au dedans de vous un Esprit nouveau ; j'ôterai de votre chair le cœur de pierre, et je vous donnerai un cœur de chair\FTNT{Jé. 32:39 ; 2 Co. 3:3 ; Ez. 11:19.}.
\VS{27}Je mettrai mon Esprit au dedans de vous, je ferai en sorte que vous suiviez mes ordonnances, et que vous observiez et pratiquiez mes lois.
\VS{28}Vous habiterez le pays que j'ai donné à vos pères, vous serez mon peuple, et je serai votre Dieu.
\VS{29}Je vous délivrerai de toutes vos souillures, j'appellerai le blé, je le multiplierai, et je ne vous enverrai plus la famine.
\VS{30}Je multiplierai le fruit des arbres et le revenu des champs, afin que vous ne portiez plus l'opprobre de la famine parmi les nations.
\VS{31}Vous vous souviendrez de votre mauvaise voie et de vos actions, qui n'étaient pas bonnes, et vous prendrez vous-mêmes en dégoût vos iniquités et vos abominations.
\VS{32}Je ne le fais point par amour pour vous, dit le Seigneur Yahweh ; sachez-le ! Soyez honteux et confus à cause de votre voie, ô maison d'Israël !
\VS{33}Ainsi parle le Seigneur Yahweh : Le jour où je vous aurai purifiés de toutes vos iniquités, je vous ferai habiter dans des villes, et les lieux déserts seront rebâtis.
\VS{34}La terre désolée sera cultivée, tandis qu'elle n'était que désolation aux yeux de tous les passants.
\VS{35}On dira : Cette terre-ci qui était désolée est devenue comme le jardin d'Eden ; et ces villes qui étaient désertes, désolées, et détruites, sont fortifiées et habitées\FTNT{Jé. 22:8-9 ; Es. 33:20.}.
\VS{36}Les nations qui resteront autour de vous sauront que moi, Yahweh, j'ai rebâti les lieux détruits et planté le pays désolé ; moi, Yahweh, j'ai parlé, et je le ferai.
\VS{37}Ainsi parle le Seigneur Yahweh : Je me laisserai rechercher par la maison d'Israël. Voici ce que je ferai pour eux : Je multiplierai les hommes comme un troupeau de brebis.
\VS{38}Les villes qui sont désertes seront remplies de troupeaux d'hommes, pareils aux troupeaux consacrés, aux troupeaux qu'on amène à Jérusalem pendant ses fêtes solennelles ; et ils sauront que je suis Yahweh.
\Chap{37}
\TextTitle{Vision des ossements desséchés, image de la restauration d'Israël}
\VerseOne{}La main de Yahweh fut sur moi, et Yahweh me transporta par son Esprit et me déposa au milieu d'une vallée remplie d'ossements\FTNT{Les ossements desséchés représentent les Israélites dispersés dans les nations.}.
\VS{2}Il me fit passer auprès d'eux, tout autour ; et voici, ils étaient fort nombreux à la surface de cette vallée et complètement secs.
\VS{3}Puis il me dit : Fils de l'homme, ces os pourront-ils revivre ? Et je répondis : Seigneur Yahweh, tu le sais.
\VS{4}Alors il me dit : Prophétise sur ces os, et dis-leur : Ossements desséchés, écoutez la parole de Yahweh !
\VS{5}Ainsi parle le Seigneur Yahweh à ces os : Voici, je ferai entrer un esprit en vous, et vous vivrez\FTNT{Ro. 8:11 ; Ps. 71:20.};
\VS{6}je mettrai des nerfs sur vous, je ferai croître de la chair sur vous, et j'étendrai la peau sur vous ; puis je mettrai un esprit en vous, et vous vivrez. Et vous saurez que je suis Yahweh.
\VS{7}Alors je prophétisai selon l'ordre que j'avais reçu. Et comme je prophétisais, il se fit un bruit, et voici, il se fit un mouvement, et ces os s'approchèrent les uns des autres.
\VS{8}Puis je regardai, et voici, il vint des nerfs sur eux, et il y crût de la chair, la peau fut étendue par dessus ; mais il n'y avait pas en eux d'esprit.
\VS{9}Alors il me dit : Prophétise à l'Esprit ! Prophétise, fils de l'homme ! Et dis à l'Esprit : Ainsi parle le Seigneur Yahweh : Esprit, viens des quatre vents, et souffle sur ces morts, et qu'ils revivent !
\VS{10}Je prophétisai donc selon l'ordre qu'il m'avait donné. Et l'Esprit entra en eux, ils reprirent vie, et se tinrent sur leurs pieds ; c'était une armée extrêmement grande.
\VS{11}Alors il me dit : Fils de l'homme, ces os sont toute la maison d'Israël ; voici, ils disent : Nos os sont desséchés, et notre attente est perdue, c'en est fait de nous !
\VS{12}C'est pourquoi prophétise, et dis-leur : Ainsi parle le Seigneur Yahweh : Mon peuple, voici, je m'en vais ouvrir vos sépulcres, je vous tirerai hors de vos sépulcres, et vous ferai entrer dans la terre d'Israël\FTNT{Les sépulcres représentent les nations dans lesquelles les Israélites se sont établis. Dieu annonce le retour de son peuple sur la terre d'Israël. Es. 26:19 ; Os. 13:14.}.
\VS{13}Et vous, mon peuple, vous saurez que je suis Yahweh quand j'aurai ouvert vos sépulcres, et que je vous aurai tirés hors de vos sépulcres.
\VS{14}Je mettrai mon Esprit en vous, et vous vivrez, je vous rétablirai sur votre terre ; et vous saurez que moi, Yahweh, j'ai parlé et que je l'ai fait, dit Yahweh.
\TextTitle{Prophétie sur l'unité d'Israël}
\VS{15}Puis la parole de Yahweh me fut adressée en ces mots :
\VS{16}Et toi, fils de l'homme, prends un bois et écris dessus : Pour Juda, et pour les fils d'Israël ses compagnons. Prends encore un autre bois, et écris dessus : Le bois d'Ephraïm et de toute la maison d'Israël, ses compagnons, pour Joseph.
\VS{17}Puis tu les joindras l'un à l'autre pour ne former qu'un même bois, ils seront unis dans ta main.
\VS{18}Quand les fils de ton peuple demanderont, en disant : Ne nous déclareras-tu pas ce que tu veux dire par ces choses ?
\VS{19}Dis-leur : Ainsi parle le Seigneur Yahweh : Voici, je m'en vais prendre le bois de Joseph qui est dans la main d'Ephraïm, et des tribus d'Israël, ses compagnons ; je les joindrai au bois de Juda, et j'en formerai un seul bois, ils ne seront qu'un seul bois dans ma main.
\VS{20}Ainsi les bois sur lesquels tu écriras seront dans ta main, sous leurs yeux.
\VS{21}Dis-leur : Ainsi parle le Seigneur Yahweh : Voici, je m'en vais prendre les fils d'Israël d'entre les nations parmi lesquelles ils sont allés, je les rassemblerai de toutes parts, et je les ferai entrer dans leur terre.
\VS{22}Je ferai d'eux une seule nation dans le pays, sur les montagnes d'Israël ; un seul roi sera leur roi à tous, ils ne seront plus deux nations, et ils ne seront plus divisés en deux royaumes\FTNT{Os. 2:2 ; Es. 11:12-13 ; Jn. 10:16.}.
\VS{23}Ils ne se souilleront plus par leurs idoles, ni par leurs infamies, ni par tous leurs crimes, et je les retirerai de toutes leurs demeures dans lesquelles ils ont péché, et je les purifierai ; ils seront mon peuple, et je serai leur Dieu\FTNT{Es. 1:18 ; Jé. 33:8 ; Jé. 24:7 ; Jé. 32:38 ; Za. 8:8 ; 2 Co. 6:16.}.
\VS{24}David, mon serviteur, sera leur roi, et ils auront tous un seul pasteur ; ils suivront mes ordonnances, ils garderont mes lois et les mettront en pratique.
\VS{25}Ils habiteront dans le pays que j'ai donné à Jacob, mon serviteur, dans lequel vos pères ont habité ; ils y habiteront, dis-je, eux, et leurs fils, et les fils de leurs fils, pour toujours ; et David mon serviteur sera leur prince pour toujours.
\VS{26}Je traiterai avec eux une alliance de paix, et il y aura une alliance éternelle avec eux ; je les établirai, et les multiplierai, je mettrai mon lieu saint au milieu d'eux pour toujours.
\VS{27}Ma demeure sera parmi eux ; je serai leur Dieu, et ils seront mon peuple.
\VS{28}Les nations sauront que je suis Yahweh qui sanctifie Israël, quand mon lieu saint sera au milieu d'eux pour toujours.
\Chap{38}
\TextTitle{Jugement sur Gog}
\VerseOne{}La parole de Yahweh me fut encore adressée en ces mots :
\VS{2}Fils de l'homme, tourne ta face vers Gog au pays de Magog\FTNT{Gog est un prince et Magog le pays. Ce chapitre doit être mis en parallèle avec Za. 12:1-4 ; Za. 14:1-9 ; Mt. 24:14-30 ; Ap. 14:14-20 ; Ap. 20:8.}, vers le prince de Rosch, de Méschec et de Tubal, et prophétise contre lui !
\VS{3}Tu diras : Ainsi parle le Seigneur Yahweh : Voici, j'en veux à toi, Gog, prince des chefs de Méschec et de Tubal !
\VS{4}Je te ferai retourner en arrière, et je mettrai des boucles dans tes mâchoires, et te ferai sortir avec toute ton armée, avec les chevaux, et les cavaliers, tous parfaitement bien équipés, une grande multitude portant le grand et le petit bouclier, et tous maniant l'épée ;
\VS{5}ceux de Perse, d'Ethiopie, et de Puth avec eux, qui tous ont des boucliers et des casques.
\VS{6}Gomer et toutes ses troupes, la maison de Togarma à l'extrême nord, avec toutes ses troupes, et plusieurs peuples avec toi.
\VS{7}Apprête-toi, tiens-toi prêt, toi, et toute la multitude assemblée autour de toi ! Sois leur chef !
\VS{8}Après plusieurs jours, tu seras à leur tête, et dans la suite des années, tu marcheras contre le pays dont les habitants, délivrés de l'épée, auront été rassemblés d'entre plusieurs peuples sur les montagnes d'Israël longtemps désertes ; retirés du milieu des peuples, ils seront en sécurité dans leurs demeures.
\VS{9}Tu monteras, tu viendras comme une dévastation, tu seras comme une nuée pour couvrir la terre, toi, toutes tes troupes, et plusieurs peuples avec toi\FTNT{Da. 11:40.}.
\VS{10}Ainsi parle le Seigneur Yahweh : Il arrivera dans ces jours-là que des pensées s'élèveront dans ton cœur, et que tu formeras un dessein pernicieux.
\VS{11}Car tu diras : Je monterai contre le pays dont les villes sont sans murailles ; j'envahirai ceux qui sont en repos, qui habitent en sécurité, qui demeurent tous dans des villes sans murs, lesquelles n'ont ni barres ni portes\FTNT{Jé. 49:31.} ;
\VS{12}pour enlever un grand butin et faire un grand pillage ; pour remettre ta main sur les déserts qui de nouveau étaient habités, et sur le peuple rassemblé d'entre les nations, ayant des troupeaux et des biens, et occupant les lieux élevés du pays.
\VS{13}Séba, et Dedan, les marchands de Tarsis, et tous ses lionceaux, te diront : Ne vas-tu pas pour faire du butin, et n'as-tu pas assemblé ta multitude pour faire un grand pillage, pour emporter de l'argent et de l'or, pour prendre le bétail et les biens, pour enlever un grand butin ?
\VS{14}Toi donc, fils de l'homme, prophétise, et dis à Gog : Ainsi parle le Seigneur Yahweh : En ce jour-là, quand mon peuple d'Israël habitera en sécurité, ne le sauras-tu pas ?
\VS{15}Ne viendras-tu pas de ton lieu, de l'extrême nord, toi, et plusieurs peuples avec toi, tous montés sur des chevaux, une grande multitude, et une grosse armée ?
\VS{16}Ne monteras-tu pas contre mon peuple d'Israël, comme une nuée pour couvrir la terre ? Dans la suite de ces jours, je te ferai venir sur ma terre, afin que les nations me connaissent, quand je serai sanctifié par toi sous leurs yeux, ô Gog !
\VS{17}Ainsi parle le Seigneur Yahweh : N'est-ce pas de toi que j'ai parlé autrefois par le ministère de mes serviteurs, les prophètes d'Israël, qui ont prophétisé dans ces jours-là pendant plusieurs années, qu'on te ferait venir contre eux ?
\VS{18}Mais il arrivera dans ce jour-là, au jour de la venue de Gog sur la terre d'Israël, dit le Seigneur Yahweh, que ma colère éclatera.
\VS{19}Je le déclare, dans ma jalousie, dans l'ardeur de ma fureur, en ce jour-là, il y aura une grande agitation sur la terre d'Israël.
\VS{20}Les poissons de la mer, les oiseaux des cieux, et les bêtes des champs, et tous les reptiles qui rampent sur la terre, et tous les hommes qui sont sur la surface de la terre seront épouvantés par ma présence ; les montagnes seront renversées, les parois des rochers tomberont, et tous les murs chuteront par terre.
\VS{21}J'appellerai contre lui l'épée sur toutes mes montagnes, dit le Seigneur Yahweh ; l'épée de chacun d'eux sera contre son frère.
\VS{22}J'entrerai en jugement avec lui par la peste, et par le sang ; je ferai pleuvoir sur lui, sur ses troupes, et sur les grands peuples qui seront avec lui, des torrents d'eau, des pierres de grêle, du feu et du soufre\FTNT{Ap. 8:7 ; Ps. 11:6 ; Ap. 16:21 ; Ap. 11:19.}.
\VS{23}Je me glorifierai, je me sanctifierai, je serai connu aux yeux de plusieurs nations ; et elles sauront que je suis Yahweh.
\Chap{39}
\TextTitle{Jugement sur Gog, suite}
\VerseOne{}Toi donc, fils de l'homme, prophétise contre Gog, et dis : Ainsi parle le Seigneur Yahweh : Voici, j'en veux à toi, Gog, prince des chefs de Méschec et de Tubal !
\VS{2}Je te ferai retourner en arrière, je te conduirai, je te ferai monter de l'extrême nord, et je t'amènerai sur les montagnes d'Israël.
\VS{3}Car je frapperai ton arc dans ta main gauche, et je ferai tomber tes flèches de ta main droite.
\VS{4}Tu tomberas sur les montagnes d'Israël, toi et toutes tes troupes, et les peuples qui seront avec toi ; je te livrerai aux oiseaux de proie, à tout ce qui a des ailes, et aux bêtes des champs, pour en être dévoré.
\VS{5}Tu tomberas sur la face des champs, parce que j'ai parlé, dit le Seigneur Yahweh.
\VS{6}Je mettrai le feu dans Magog, et parmi ceux qui demeurent en sécurité dans les îles ; et ils sauront que je suis Yahweh.
\VS{7}Je ferai connaître mon saint Nom au milieu de mon peuple d'Israël ; et je ne profanerai plus mon saint Nom ; les nations sauront que je suis Yahweh, le Saint d'Israël.
\VS{8}Voici, cela arrive et sera fait, dit le Seigneur Yahweh ; c'est ici le jour dont j'ai parlé.
\VS{9}Les habitants des villes d'Israël sortiront, allumeront le feu, brûleront les armes, les petits et les grands boucliers, les arcs, les flèches, les bâtons qu'on lance de la main, et les javelots ; ils en feront du feu pendant sept ans.
\VS{10}On n'apportera point du bois des champs, et on n'en coupera point dans les forêts, parce qu'ils feront du feu de ces armes, lorsqu'ils dépouilleront ceux qui les avaient dépouillés, et qu'ils pilleront ceux qui les avaient pillés, dit le Seigneur Yahweh.
\VS{11}Il arrivera ce jour-là que je donnerai à Gog dans ces quartiers-là un lieu pour sépulcre en Israël, à savoir la vallée des passants, qui est au-devant de la mer ; elle réduira les passants au silence ; on enterrera là Gog, et toute la multitude de son peuple, et on l'appellera la vallée d'Hamon-Gog\FTNT{La vallée d'Hamon-Gog : La vallée de la multitude de Gog.}.
\VS{12}Ceux de la maison d'Israël les enterreront, et cela durera sept mois, afin de purifier le pays.
\VS{13}Tout le peuple du pays les enterrera, et il en aura du renom, le jour où je serai glorifié, dit le Seigneur Yahweh.
\VS{14}Ils mettront à part des gens qui ne feront autre chose que parcourir le pays, et qui enterreront, avec l'aide des passants, les corps restés à la surface de la terre, pour la purifier, et ils seront à la recherche pendant sept mois.
\VS{15}Ils parcourront le pays, et celui qui verra l'os d'un homme, dressera auprès de lui un signal ; jusqu'à ce que les fossoyeurs l'aient enterré dans la vallée d'Hamon-Gog.
\VS{16}Il y aura aussi une ville nommée Hamona\FTNT{Hamona signifie « multitude ».}, et on nettoiera le pays.
\VS{17}Toi donc, fils de l'homme, ainsi parle le Seigneur Yahweh : Dis aux oiseaux de toutes espèces, et à toutes les bêtes des champs : Assemblez-vous et venez ; amassez-vous de toutes parts vers mon sacrifice que je fais pour vous, qui est un grand sacrifice sur les montagnes d'Israël ! Vous mangerez de la chair, et vous boirez du sang.
\VS{18}Vous mangerez la chair des hommes puissants, et vous boirez le sang des princes de la terre, le sang des moutons, des agneaux, des boucs, et des veaux engraissés sur le Basan\FTNT{Es. 34:6 ; Jé. 46:10 ; So. 1:7 ; Mt. 24:28 ; Job. 39:33.}.
\VS{19}Vous mangerez de la graisse jusqu'à en être rassasiés, et vous boirez du sang jusqu'à en être ivres, de la graisse et du sang de mon sacrifice, que j'aurai sacrifié pour vous.
\VS{20}Vous serez rassasiés à ma table, de chevaux et de bêtes d'attelage, d'hommes forts, et de tous hommes de guerre, dit le Seigneur Yahweh.
\VS{21}Je mettrai ma gloire parmi les nations, et toutes les nations verront mon jugement que j'aurai exercé, et comment j'aurai mis ma main sur eux.
\VS{22}La maison d'Israël connaîtra dès ce jour-là, et dans la suite, que je suis Yahweh, leur Dieu.
\VS{23}Les nations sauront que la maison d'Israël avait été emmenée en captivité à cause de son iniquité, parce qu'ils avaient péché contre moi, et que je leur avais caché ma face ; aussi je les avais livrés entre les mains de leurs ennemis pour qu'ils périssent par l'épée\FTNT{De. 31:17-18 ; Ps. 13:2.}.
\VS{24}Je leur avais fait selon leurs souillures, et selon leurs crimes, et je leur avais caché ma face.
\TextTitle{Rétablissement et conversion d'Israël}
\VS{25}C'est pourquoi, ainsi parle le Seigneur Yahweh : Maintenant, je ramènerai la captivité de Jacob, et j'aurai pitié de toute la maison d'Israël, et je serai jaloux de mon saint Nom,
\VS{26}après avoir porté leur ignominie, et tout leur crime, lorsqu'ils avaient péché contre moi, quand ils demeuraient en sûreté dans leur terre, sans qu'il y eût personne pour les effrayer.
\VS{27}Parce que je les ramènerai d'entre les peuples, que je les rassemblerai des pays de leurs ennemis, et que je serai sanctifié par eux, sous les yeux de plusieurs nations.
\VS{28}Ils sauront que je suis Yahweh, leur Dieu, lorsqu'après les avoir enlevés parmi les nations, je les rassemblerai sur leurs terres, et que je n'en laisserai chez elles aucun d'eux.
\VS{29}Je ne leur cacherai plus ma face, car je répandrai mon Esprit sur la maison d'Israël, dit le Seigneur Yahweh\FTNT{Joë. 2:28 ; Ac. 2:17.}.
\Chap{40}
\TextTitle{Mesures du futur temple}
\VerseOne{}Dans la vingt-cinquième année de notre captivité, au commencement de l'année, au dixième jour du mois, la quatorzième année après que la ville fut prise, en ce même jour, la main de Yahweh fut sur moi, et il m'amena là.
\VS{2}Il m'amena par des visions de Dieu, au pays d'Israël, et me posa sur une montagne fort élevée, sur laquelle du côté sud il y avait comme une ville construite.
\VS{3}Après qu'il m'y fît entrer, voici un homme, dont l'aspect était comme de l'airain, qui avait dans sa main un cordeau de lin, et une canne à mesurer, et qui se tenait debout à la porte.
\VS{4}Cet homme me parla ainsi : Fils de l'homme, regarde de tes yeux, écoute de tes oreilles, et applique ton cœur à toutes les choses que je m'en vais te faire voir, car tu as été amené ici afin que je te les fasse voir, et que tu fasses savoir à la maison d'Israël toutes les choses que tu vas voir.
\VS{5}Voici, un mur extérieur entourait la maison. Cet homme avait dans la main une canne à mesurer longue de six coudées, chaque coudée étant d'une coudée normale et une largeur de main en plus. Il mesura la largeur de ce mur bâti, laquelle était d'une canne, et sa hauteur d'une autre canne.
\VS{6}Puis il vint vers la porte orientale, et monta par ses étages. Il mesura l'un des poteaux de la porte d'une canne en largeur, et l'autre poteau d'une autre canne en largeur.
\VS{7}Puis il mesura chaque chambre d'une canne en longueur, et d'une canne en largeur. L'espace entre les deux chambres était de cinq coudées. Il mesura d'une canne chacun des poteaux de la porte près du vestibule qui menait à la porte la plus intérieure.
\VS{8}Puis il mesura d'une canne le vestibule qui menait à la porte la plus intérieure.
\VS{9}Il mesura de huit coudées le vestibule de la porte et ses poteaux, le vestibule de la porte était en dedans.
\VS{10}Les chambres de la porte orientale étaient au nombre de trois d'un côté et de trois de l'autre, toutes les trois avaient la même mesure.
\VS{11}Puis il mesura de dix coudées la largeur de l'ouverture de la première porte, et de treize coudées la longueur de la même porte.
\VS{12}Ensuite, il mesura d'un côté un espace limité au-devant des chambres d'une coudée, et une autre coudée d'espace limité de l'autre côté ; chaque chambre avait six coudées d'un côté, et six coudées de l'autre.
\VS{13}Après cela, il mesura le portail depuis le toit d'une chambre jusqu'au toit de l'autre, de la largeur de vingt-cinq coudées entre les deux ouvertures opposées.
\VS{14}Il compta soixante coudées pour les poteaux, près desquels était une cour, autour de la porte.
\VS{15}L'espace entre la porte d'entrée et le vestibule de la porte intérieure était de cinquante coudées.
\VS{16}Il y avait des fenêtres closes aux chambres et à leurs poteaux, à l'intérieur de la porte tout autour. Il y avait aussi des fenêtres dans les vestibules tout autour intérieurement, des palmes étaient sculptées sur les poteaux.
\VS{17}Il me mena dans le parvis extérieur, où se trouvaient des chambres et un pavé tout autour. Il y avait trente chambres sur ce pavé.
\VS{18}Le pavé était au côté des portes et répondait à la longueur des portes ; c'était le pavé inférieur.
\VS{19}Il mesura la largeur du parvis depuis la porte qui menait vers le bas jusqu'au parvis intérieur en dehors. Il y avait cent coudées à l'orient et au nord.
\VS{20}Après cela, il mesura la longueur et la largeur de la porte nord du parvis extérieur.
\VS{21}Quant aux chambres, au nombre de trois d'un côté et trois de l'autre, ses poteaux et ses vestibules avaient la même mesure que la première porte, cinquante coudées en longueur, et vingt-cinq coudées en largeur.
\VS{22}Ses fenêtres, son vestibule, et ses palmes avaient la même mesure que la porte orientale ; on y montait par sept étages, devant lesquels étaient son vestibule.
\VS{23}La porte du parvis intérieur était vis-à-vis de la première porte du nord, et vis-à-vis de la porte orientale. Il mesura depuis une porte jusqu'à l'autre cent coudées.
\VS{24}Après cela, il me conduisit du côté sud, où se trouvait la porte méridionale, il en mesura les poteaux et les vestibules qui avaient la même mesure.
\VS{25}Cette porte et ses vestibules avaient des fenêtres tout autour, comme les autres fenêtres, cinquante coudées de long, et vingt-cinq coudées de large.
\VS{26}On y montait par sept étages, devant lesquels était son vestibule ; il avait de chaque côté des palmes sur ses poteaux.
\VS{27}Pareillement, le parvis intérieur avait sa porte du côté sud ; il mesura d'une porte à l'autre au sud cent coudées.
\VS{28}Après cela il me fit entrer dans le parvis intérieur par la porte sud, et il mesura la porte sud, selon les mesures précédentes.
\VS{29}Ses chambres, ses poteaux et ses vestibules avaient la même mesure. Cette porte et ses vestibules avaient des fenêtres tout autour, cinquante coudées de long, et vingt-cinq coudées de large.
\VS{30}Il y avait tout autour des vestibules de vingt-cinq coudées de long, et cinq coudées de large.
\VS{31}Les vestibules de la porte aboutissaient au parvis extérieur ; il y avait des palmes sur ses poteaux, et huit étages pour y monter.
\VS{32}Il me conduisit dans le parvis intérieur, par l'entrée orientale. Il mesura la porte, qui avait la même mesure.
\VS{33}Ses chambres, ses poteaux et ses vestibules avaient la même mesure. Cette porte et ses vestibules avaient des fenêtres tout autour, cinquante coudées de long, et vingt-cinq de large.
\VS{34}Ses vestibules aboutissaient au parvis extérieur ; il y avait de chaque côté des palmes sur ses poteaux, et huit étages pour y monter.
\VS{35}Il me conduisit vers la porte nord. Il la mesura et trouva la même mesure.
\VS{36}Ainsi qu'à ses chambres, à ses poteaux et à ses vestibules ; elle avait des fenêtres tout autour, cinquante coudées de long, et vingt-cinq coudées de large.
\VS{37}Ses vestibules aboutissaient au parvis extérieur ; il y avait de chaque côté des palmes sur ses poteaux, et huit étages pour y monter.
\VS{38}Il y avait une chambre qui s'ouvrait vers les poteaux des portes, et où l'on devait laver les holocaustes.
\VS{39}Il y avait aussi dans le vestibule de la porte de chaque côté deux tables, pour y égorger les bêtes qu'on sacrifierait pour l'holocauste, et le sacrifice pour l'expiation et le sacrifice pour la culpabilité.
\VS{40}Vers l'un des côtés de la porte, au dehors, vers le lieu où l'on montait, à l'entrée de la porte nord, il y avait deux tables, et de l'autre côté, vers le vestibule de la porte, deux autres tables.
\VS{41}Il se trouvait ainsi, aux côtés de la porte, quatre tables d'une part, et quatre tables de l'autre, en tout huit tables, sur lesquelles on devait abattre les victimes.
\VS{42}Les quatre tables qui étaient pour l'offrande entièrement consumée, étaient en pierres de taille, de la longueur d'une coudée et demie, et de la largeur d'une coudée et demie, et de la hauteur d'une coudée ; et même on devait poser sur elles les instruments avec lesquels on tuait les victimes pour les offrandes entièrement consumées, et les autres sacrifices.
\VS{43}Il y avait aussi à l'intérieur de la maison tout autour, des chevilles pour accrocher, larges d'une paume, bien adaptées, d'où l'on apportait la chair des sacrifices sur les tables.
\TextTitle{Répartition des pièces du futur temple}
\VS{44}En dehors de la porte intérieure, il y avait des chambres pour les chantres dans le parvis intérieur, l'une était à côté de la porte nord et avait la face au sud, l'autre était à côté de la porte orientale et avait la face au nord.
\VS{45}Il me dit : Ces chambres, dont la face est au sud, sont pour les sacrificateurs qui ont la charge de la maison.
\VS{46}Mais ces chambres, dont la face est au nord, sont pour les sacrificateurs qui ont la charge de l'autel, qui sont les fils de Tsadok, qui, parmi les fils de Lévi, s'approchent de Yahweh pour faire son service.
\VS{47}Puis il mesura un parvis de la longueur et de la largeur de cent coudées, en carré ; et l'autel était devant la maison.
\VS{48}Ensuite, il me fit entrer dans le vestibule de la maison ; et il mesura les poteaux du vestibule de cinq coudées d'un côté, et de cinq coudées de l'autre, puis la largeur de la porte de trois coudées d'un côté, et de trois coudées de l'autre.
\VS{49}Le vestibule avait une longueur de vingt coudées, et une largeur de onze coudées ; on y montait par des étages. Il y avait des colonnes près des poteaux, l'une d'un côté, et l'autre de l'autre.
\Chap{41}
\TextTitle{Description du temple}
\VerseOne{}Puis il me fit entrer dans le temple, et il mesura des poteaux de six coudées de largeur d'un côté, et de six coudées de largeur de l'autre côté, largeur de la tente.
\VS{2}Ensuite il mesura la largeur de l'ouverture de la porte qui était de dix coudées, et les côtés de l'ouverture de cinq coudées, d'une part, et de cinq coudées de l'autre part. Puis il mesura la longueur du temple, quarante coudées, et la largeur, vingt coudées.
\VS{3}Il entra à l'intérieur, et il mesura un poteau d'une ouverture de porte, deux coudées, la hauteur de cette ouverture, six coudées, et la largeur de cette ouverture, sept coudées.
\VS{4}Puis il mesura une longueur de vingt coudées, et une largeur de vingt coudées en face du temple ; et il me dit : C'est ici le Saint des saints.
\VS{5}Il mesura l'épaisseur du mur de la maison, qui fut de six coudées, et la largeur des chambres qui étaient tout autour de la maison, de quatre coudées.
\VS{6}Les chambres latérales étaient les unes à côté des autres, au nombre de trente, et il y avait trois poutres ; elles entraient dans un mur construit pour ces chambres tout autour de la maison, elles y étaient appuyées sans entrer dans le mur même de la maison.
\VS{7}Les chambres occupaient plus d'espace, à mesure qu'elles s'élevaient, et l'on allait en tournant, car on montait autour de la maison par un escalier tournant. Il y avait plus d'espace dans le haut de la maison, et l'on montait de l'étage inférieur à l'étage supérieur par celui du milieu.
\VS{8}Je considérai la hauteur autour de la maison. Les chambres latérales, à partir de leur fondement avaient une canne pleine, six grandes coudées.
\VS{9}La largeur du mur extérieur des chambres latérales était de cinq coudées ; l'espace libre entre les chambres latérales de la maison,
\VS{10}et les chambres autour de la maison avait une largeur de vingt coudées.
\VS{11}L'ouverture des chambres latérales donnait sur l'espace libre, une ouverture au nord, et une autre ouverture au sud ; la largeur de l'espace libre était de cinq coudées tout autour.
\VS{12}Le bâtiment qui était devant la place vide, du côté de l'occident, avait une largeur de soixante-dix coudées, un mur de cinq coudées de largeur tout autour, et une longueur de quatre-vingt-dix coudées.
\VS{13}Il mesura la maison, qui avait cent coudées de longueur ; la place vide, le bâtiment et les murs avaient une longueur de cent coudées.
\VS{14}La largeur de la face de la maison et de la place vide, du côté oriental, était de cent coudées.
\VS{15}Il mesura la longueur du bâtiment devant la place vide, sur le derrière, et ses galeries de chaque côté : Il y avait cent coudées.
\VS{16}Les seuils, les fenêtres closes, les galeries du pourtour aux trois étages, en face des seuils, étaient recouverts de bois tout autour. Depuis le sol jusqu'aux fenêtres fermées,
\VS{17}jusqu'au-dessus des ouvertures, et jusqu'à la maison au-dedans comme au dehors, tout le mur du pourtour, à l'intérieur et à l'extérieur, tout était d'après la mesure,
\VS{18}et fait de chérubins et de palmes. Il y avait une palme entre deux chérubins, et chaque chérubin avait deux faces.
\VS{19}Une face d'homme était tournée vers la palme d'un côté, et une face de jeune lion était tournée vers la palme de l'autre côté ; il en était ainsi tout autour de la maison.
\VS{20}Depuis le sol jusqu'au-dessus des ouvertures il y avait des chérubins et des palmes et aussi sur le mur du temple.
\VS{21}Les poteaux du temple étaient carrés ; et la face du lieu saint avait la même apparence.
\VS{22}L'autel était de bois, de la hauteur de trois coudées, et de deux coudées de longueur ; ses angles, ses pieds et ses côtés étaient de bois. Puis il me dit : C'est ici la table qui est devant Yahweh.
\VS{23}Le temple et le lieu saint avaient deux portes.
\VS{24}Il y avait deux portes, deux battants, qui tous deux tournaient sur les portes, deux battants pour une porte et deux pour l'autre.
\VS{25}Il y avait aussi des chérubins et des palmes façonnés sur les portes du temple, comme sur les murs. Un entablement en bois était sur le front du vestibule en dehors.
\VS{26}Il y avait des fenêtres fermées, et des palmes de part et d'autre, ainsi qu'aux côtés du vestibule, aux chambres latérales de la maison, et aux entablements.
\Chap{42}
\TextTitle{Mesures supplémentaires du temple}
\VerseOne{}Après cela, il me fit sortir vers le parvis extérieur, du côté nord ; et il me conduisit vers les chambres qui étaient vis-à-vis de la place vide et vis-à-vis du bâtiment, au nord.
\VS{2}Sur la face où se trouvait une ouverture au nord, il y avait une longueur de cent coudées, et la largeur était de cinquante coudées.
\VS{3}C'était vis-à-vis des vingt coudées du parvis intérieur, et vis-à-vis du pavé extérieur, là où se trouvaient les galeries des trois étages.
\VS{4}Devant les chambres, il y avait une promenade large de dix coudées, et une voie d'une coudée ; leurs ouvertures donnaient au nord.
\VS{5}Les chambres supérieures étaient plus étroites que les inférieures et que celles du milieu du bâtiment parce que les galeries leur ôtaient de la place.
\VS{6}Car elles étaient à trois étages, et n'avaient point de colonnes, comme les colonnes des parvis ; c'est pourquoi à partir du sol, les chambres du haut étaient plus étroites que celles du bas et du milieu.
\VS{7}Le mur extérieur parallèle aux chambres, du côté du parvis extérieur devant les chambres, avait cinquante coudées de long.
\VS{8}Car la longueur des chambres du côté du parvis extérieur était de cinquante coudées. Mais sur la face du temple, il y avait cent coudées.
\VS{9}Au bas de ces chambres était l'entrée orientale quand on y venait du parvis extérieur.
\VS{10}Il y avait encore des chambres sur la largeur du mur du parvis du côté oriental, vis-à-vis de la place vide et vis-à-vis du bâtiment.
\VS{11}Devant elles, il y avait un chemin, comme devant les chambres qui étaient du côté nord. La longueur et la largeur étaient les mêmes ; leurs issues, leur disposition et leurs ouvertures étaient semblables.
\VS{12}Il en était de même pour les ouvertures des chambres du côté sud. Il y avait une ouverture à la tête du chemin, du chemin qui se trouvait droit devant le mur du côté oriental par où l'on y entrait.
\VS{13}Après cela, il me dit : Les chambres du parvis nord et les chambres du parvis sud, qui sont devant la place vide, ce sont les chambres du lieu saint où les sacrificateurs qui s'approchent de Yahweh, mangeront les choses très saintes. Ils déposeront là les choses très saintes, savoir les gâteaux, les offrandes pour l'expiation et les offrandes pour la culpabilité ; car ce lieu est saint.
\VS{14}Quand les sacrificateurs seront entrés, ils ne sortiront point du lieu saint pour venir au parvis extérieur, mais ils déposeront là leurs vêtements avec lesquels ils font le service ; car ces vêtements sont saints ; ils en mettront d'autres pour s'approcher du peuple.
\VS{15}Lorsqu'il eut achevé de mesurer la maison intérieure, il me fit sortir par la porte qui était du côté oriental, puis il mesura l'enceinte tout autour.
\VS{16}Il mesura le côté oriental avec la canne qui servait de mesure, et il y avait tout autour cinq cents cannes.
\VS{17}Ensuite il mesura le côté nord, avec la canne qui servait de mesure, et il y avait tout autour cinq cents cannes.
\VS{18}Puis il mesura le côté sud avec la canne qui servait de mesure, et il y avait cinq cents cannes.
\VS{19}Il se tourna du côté occidental, et mesura cinq cents cannes avec la canne qui servait de mesure.
\VS{20}Il mesura des quatre côtés le mur formant l'enceinte de la maison ; la longueur était de cinq cents cannes, et la largeur de cinq cents cannes, ce mur marquait la séparation entre le saint et le profane.
\Chap{43}
\TextTitle{La gloire de Yahweh remplit la maison\FTNTT{Cp. Ez. 11:22-24.}}
\VerseOne{}Puis il me ramena à la porte, à la porte qui était du côté oriental.
\VS{2}Et voici, la gloire du Dieu d'Israël s'avançait de l'orient, sa voix était pareille au bruit des grandes eaux, et la terre resplendissait de sa gloire\FTNT{Ap. 1:15.}.
\VS{3}La vision que j'eus alors était semblable à celle que j'avais vue lorsque j'étais venu pour détruire la ville, ces visions étaient comme la vision que j'avais vue sur le fleuve de Kebar ; et je me prosternai le visage contre terre.
\VS{4}Puis la gloire de Yahweh entra dans la maison par la porte qui était du côté oriental.
\VS{5}L'Esprit m'enleva et me fit entrer dans le parvis intérieur, et voici la gloire de Yahweh remplissait la maison.
\TextTitle{Le trône de Yahweh}
\VS{6}Je l'entendis s'adressant à moi depuis la maison, et l'homme qui me conduisait était debout près de moi.
\VS{7}Yahweh me dit : Fils de l'homme, c'est ici le lieu de mon trône, et le lieu des plantes de mes pieds, dans lequel je ferai ma demeure éternellement parmi les fils d'Israël ; et la maison d'Israël ne souillera plus mon saint Nom, ni eux, ni leurs rois, par leurs fornications ; mais ils souilleront leurs hauts lieux par les cadavres de leurs rois.
\VS{8}Car ils ont mis leur seuil près de mon seuil, et leur poteau près de mon poteau, il y avait un mur entre moi et eux ; ils ont souillé mon saint Nom par leurs abominations qu'ils ont faites, c'est pourquoi je les ai consumés dans ma colère.
\VS{9}Maintenant ils rejetteront loin de moi leurs adultères et les cadavres de leurs rois, et je ferai ma demeure éternellement parmi eux.
\VS{10}Toi donc, fils de l'homme, montre ce temple à la maison d'Israël ; et qu'ils soient confus à cause de leurs iniquités ; et qu'ils aient honte de leur iniquité.
\VS{11}S'ils rougissent de tout ce qu'ils ont fait, fais-leur connaître la forme de ce temple, sa disposition, avec ses sorties et ses entrées, toutes ses figures et toutes ses ordonnances, toutes ses formes, toutes ses lois, et écris-les sous leurs yeux, afin qu'ils gardent toutes ses formes, et toutes les ordonnances, et qu'ils les pratiquent.
\VS{12}Tel est la loi de la maison. Sur le sommet de la montagne, tout le territoire sera un lieu très saint tout autour. Voilà donc la loi de la maison.
\TextTitle{L'autel pour les holocaustes et les sacrifices}
\VS{13}Voici les mesures de l'autel, d'après les coudées dont chacune était d'une largeur de main plus longue que la coudée ordinaire. Le fond avait une coudée de hauteur et une coudée de largeur, et le rebord qui terminait son contour avait un empan de largeur ; c'était le dos de l'autel.
\VS{14}Depuis le fond sur le sol jusqu'à l'encadrement inférieur, il y avait deux coudées, et une coudée de largeur, et depuis le petit jusqu'au grand encadrement, il y avait quatre coudées et une coudée de largeur.
\VS{15}L'autel avait quatre coudées ; et quatre cornes s'élevaient de l'autel.
\VS{16}L'autel avait douze coudées de longueur, douze coudées de largeur, et formait un carré par ses quatre côtés.
\VS{17}L'encadrement avait quatorze coudées de longueur sur quatorze coudées de largeur à ses quatre côtés, le rebord qui terminait son contour avait une demi-coudée, le fond avait une coudée tout autour, et les étages étaient tournés vers l'orient.
\VS{18}Il me dit : Fils de l'homme, ainsi parle le Seigneur Yahweh : Ce sont ici les lois au sujet de l'autel pour le jour où on le fera, afin qu'on y offre l'holocauste, et qu'on y répande le sang.
\VS{19}C'est que tu donneras aux sacrificateurs, aux Lévites, qui sont de la race de Tsadok, et qui s'approchent de moi, dit le Seigneur Yahweh, afin qu'ils y fassent mon service, un jeune veau en sacrifice pour le péché.
\VS{20}Et tu prendras de son sang, et en mettras sur les quatre cornes de l'autel, et sur les quatre angles de l'encadrement et sur le rebord qui l'entoure, ainsi tu purifieras l'autel, et tu feras propitiation pour lui\FTNT{Ex. 29:36-39.}.
\VS{21}Tu prendras le jeune taureau expiatoire, et on le brûlera dans un lieu réservé de la maison, en dehors du lieu saint.
\VS{22}Le second jour, tu offriras en expiation un bouc, sans défaut, et on purifiera l'autel comme on l'aura purifié avec le jeune taureau.
\VS{23}Quand tu auras achevé de purifier l'autel, tu offriras un jeune taureau sans défaut, et un bélier du troupeau sans défaut.
\VS{24}Tu les offriras devant Yahweh, et les sacrificateurs jetteront du sel par dessus, et les offriront en holocauste à Yahweh\FTNT{Lé. 2:13.}.
\VS{25}Durant sept jours, tu sacrifieras chaque jour un bouc comme victime expiatoire, et les sacrificateurs sacrifieront un jeune taureau et un bélier du troupeau sans défaut.
\VS{26}Pendant sept jours, les sacrificateurs feront la propitiation pour l'autel, on le purifiera et chacun d'eux sera consacré\FTNT{Le terme « consacré » veut dire littéralement « remplir sa main ». Voir aussi Jg.17:5 et 17:12}
\VS{27}Lorsque ces jours seront accomplis, dès le huitième jour, et à l'avenir, les sacrificateurs offriront sur cet autel vos holocaustes et vos sacrifices d'offrande de paix. Et je serai apaisé envers vous, dit le Seigneur Yahweh.
\Chap{44}
\TextTitle{La porte fermée du sanctuaire}
\VerseOne{}Puis il me ramena vers la porte extérieure du lieu saint, du côté oriental, mais elle était fermée.
\VS{2}Yahweh me dit : Cette porte-ci sera fermée, et ne sera point ouverte, personne n'y passera, parce que Yahweh, le Dieu d'Israël, est entré par cette porte ; elle sera donc fermée\FTNT{Ap. 3:8.}.
\VS{3}Elle sera pour le prince ; le prince sera le seul qui s'y assiéra pour manger le pain devant Yahweh ; il entrera par le chemin du vestibule de la porte, et sortira par le même chemin.
\TextTitle{[La gloire dans la maison de Yahweh]}
\VS{4}Il me fit revenir par le chemin de la porte nord jusque sur le devant de la maison, je regardai, et voici, la gloire de Yahweh avait rempli la maison de Yahweh, et je me prosternai sur ma face.
\VS{5}Alors Yahweh me dit : Fils de l'homme, applique ton cœur, et regarde de tes yeux, écoute de tes oreilles tout ce dont je vais te parler, concernant toutes les ordonnances et toutes les lois qui concernent la maison de Yahweh. Applique ton cœur en ce qui concerne l'entrée de la maison et toutes les sorties du lieu saint.
\VS{6}Tu diras aux rebelles, à la maison d'Israël : Ainsi parle le Seigneur Yahweh : Maison d'Israël ! Assez de toutes vos abominations !
\VS{7}Vous avez fait entrer les fils de l'étranger, incirconcis de cœur et incirconcis de chair, pour être dans mon lieu saint, pour profaner ma maison. Vous avez offert mon pain, la graisse et le sang, à toutes vos abominations, vous avez enfreint mon alliance\FTNT{Lé. 3:11-16 ; Lé. 22:25 ; No. 28:2.}.
\VS{8}Vous n'avez pas observé l'office de mon lieu saint, mais vous les avez mis à votre place pour faire l'office dans mon lieu saint.
\TextTitle{Recommandations aux sacrificateurs du futur temple}
\VS{9}Ainsi parle le Seigneur Yahweh : Pas un de tous ceux qui seront fils d'étranger, incirconcis de cœur et incirconcis de chair, n'entrera dans mon lieu saint, pas même un d'entre tous les fils d'étrangers qui seront parmi les fils d'Israël.
\VS{10}Mais les Lévites qui se sont éloignés de moi, lorsque Israël s'est égaré, et qui se sont égarés de moi pour suivre leurs idoles, porteront la peine de leur iniquité.
\VS{11}Toutefois, ils seront employés dans mon lieu saint aux charges qui sont vers les portes de la maison, et ils feront le service de la maison ; ils égorgeront pour le peuple les bêtes pour l'holocauste, et pour les autres sacrifices, et se tiendront prêts devant lui pour le servir.
\VS{12}Parce qu'ils l'ont servi se présentant devant leurs idoles, et qu'ils ont fait tomber dans l'iniquité la maison d'Israël, à cause de cela j'ai levé ma main en jurant contre eux, dit le Seigneur Yahweh, qu'ils porteront la peine de leur iniquité.
\VS{13}Ils n'approcheront plus de moi pour exercer la sacrificature, ni pour approcher mes sanctuaires, mes lieux très saints ; mais ils porteront leur confusion et leurs abominations qu'ils ont commises.
\VS{14}C'est pourquoi je les établirai pour avoir la garde de la maison pour tout son service, et pour tout ce qui s'y fait.
\VS{15}Mais quant aux sacrificateurs et aux Lévites, fils de Tsadok, qui ont soigneusement administré ce qu'il fallait faire dans mon lieu saint, lorsque les fils d'Israël se sont éloignés de moi, ceux-là s'approcheront de moi pour faire mon service, et se tiendront devant moi pour m'offrir la graisse et le sang, dit le Seigneur Yahweh.
\VS{16}Ceux-là entreront dans mon lieu saint, et s'approcheront de ma table, pour faire mon service, et ils administreront soigneusement ce que j'ai ordonné de faire.
\VS{17}Lorsqu'ils franchiront les portes des parvis intérieurs, ils se vêtiront de robes de lin ; et il n'y aura point de laine sur eux pendant qu'ils feront le service aux portes des parvis intérieurs et dans la maison.
\VS{18}Ils auront des ornements de lin sur leur tête, et des caleçons de lin sur leurs reins, et ne se ceindront point de manière à provoquer la sueur.
\VS{19}Quand ils sortiront pour aller dans le parvis extérieur, dans le parvis extérieur, vers le peuple, ils se dévêtiront de leurs habits, avec lesquels ils font le service, et les poseront dans les chambres saintes, et se revêtiront d'autres habits, afin qu'ils ne sanctifient point le peuple avec leurs habits.
\VS{20}Ils ne se raseront point la tête, ni ne laisseront point croître leurs cheveux, mais simplement ils tondront leur tête\FTNT{Lé. 19:27.}.
\VS{21}Pas un des sacrificateurs ne boira du vin quand ils entreront au parvis intérieur.
\VS{22}Ils ne prendront point pour femme une veuve, ni une répudiée ; mais ils prendront des vierges, de la race de la maison d'Israël, ou une veuve qui soit veuve d'un sacrificateur\FTNT{Lé. 21:13-14.}.
\VS{23}Ils enseigneront à mon peuple la différence qu'il y a entre le saint et le profane, et leur feront entendre la différence qu'il y a entre ce qui est souillé et ce qui est pur.
\VS{24}Quand il surviendra quelque procès, ils assisteront au jugement, et jugeront suivant les lois que j'ai données ; et ils garderont mes lois et mes statuts dans toutes mes fêtes, et ils sanctifieront mes sabbats.
\VS{25}Un sacrificateur n'ira pas vers un mort, de peur d'en être souillé, il pourra se rendre impur que pour un père, pour une mère, pour un fils, pour une fille, pour un frère, et pour une sœur qui n'aura point eu de mari\FTNT{Lé. 21:1-2.}.
\VS{26}Et après que chacun d'eux se sera purifié, on lui comptera sept jours.
\VS{27}Le jour où il entrera dans le lieu saint, dans le parvis intérieur pour faire le service dans le lieu saint, il offrira son sacrifice pour son péché, dit le Seigneur Yahweh.
\VS{28}Et cela leur sera pour héritage. Ce sera moi leur héritage, car vous ne leur donnerez aucune possession en Israël, ce sera moi leur possession\FTNT{No. 18:20 ; De. 18:1-2.}.
\VS{29}Ils mangeront donc les gâteaux et ce qui s'offrira pour l'expiation, et ce qui s'offrira pour la culpabilité ; et tout interdit en Israël leur appartiendra.
\VS{30}Les prémices de tous les fruits et toutes les offrandes que vous présenterez par élévation, appartiendront aux sacrificateurs ; vous donnerez aussi les prémices de votre pâte aux sacrificateurs, afin que la bénédiction repose sur votre maison.
\VS{31}Les sacrificateurs ne mangeront aucune créature volante, aucun animal mort ou déchiré\FTNT{Ex. 22:31 ; Lé. 22:8.}.
\Chap{45}
\TextTitle{Zone réservée à Yahweh et aux sacrificateurs}
\VerseOne{}Quand vous partagerez au sort le pays en héritage, vous prélèverez comme une offrande en élévation pour Yahweh, une portion du pays longue de vingt-cinq mille cannes, et large de dix mille ; ce sera une chose sainte dans tous ses territoires et aux environs.
\VS{2}De cette portion, vous prendrez pour le lieu saint cinq cents cannes sur cinq cents en carré, et cinquante coudées tout autour pour ses faubourgs.
\VS{3}Sur cette étendue de vingt-cinq mille en longueur, et de dix mille en largeur, tu mesureras un emplacement pour le lieu saint, pour le Saint des saints.
\VS{4}C'est la portion sainte du pays, elle appartiendra aux sacrificateurs qui font le service du lieu saint, qui s'approchent de Yahweh pour le servir ; c'est là que seront leur maison, et ce sera un lieu saint pour le lieu saint.
\VS{5}Vingt-cinq mille cannes en longueur, et dix mille en largeur, formeront la propriété des Lévites, serviteurs de la maison, avec vingt chambres.
\VS{6}Vous donnerez pour la possession de la ville la largeur de cinq mille et la longueur de vingt-cinq mille, suivant la proportion de la portion sanctifiée, qui aura été levée pour toute la maison d'Israël.
\TextTitle{Zone réservée au prince}
\VS{7}Pour le prince vous réserverez un espace aux deux côtés de la portion sainte et de la propriété de la ville, le long de la portion sainte et le long de la propriété de la ville, du côté de l'occident vers l'occident, et du côté de l'orient vers l'orient, sur une longueur parallèle à l'une des parts, depuis la limite de l'occident jusqu'à la limite de l'orient.
\VS{8}Ce sera sa terre, sa propriété en Israël ; et mes princes que j'établirai ne fouleront plus mon peuple, mais ils distribueront le pays à la maison d'Israël, selon leurs tribus.
\TextTitle{Le prince, exemple au milieu du peuple ; prescriptions sur les offrandes}
\VS{9}Ainsi parle le Seigneur Yahweh : Assez, princes d'Israël ! Otez la violence et le pillage, et jugez avec justice ; ôtez vos extorsions de dessus mon peuple ! dit le Seigneur Yahweh.
\VS{10}Ayez la balance juste, l'épha juste, et le bath juste\FTNT{Lé. 19:35-36.}.
\VS{11}L'épha et le bath seront de même mesure ; on prendra un bath pour la dixième partie d'un homer, et l'épha sera la dixième partie d'un homer, la mesure de l'un et de l'autre se rapportera à l'homer.
\VS{12}Le sicle sera de vingt guéras ; vingt sicles, vingt-cinq sicles et quinze sicles feront la mine\FTNT{Ex. 30:13 ; Lé. 27:25.}.
\VS{13}Voici l'offrande que vous élèverez en offrande : La sixième partie d'un épha d'un homer de blé ; et vous donnerez la sixième partie d'un épha d'un homer d'orge.
\VS{14}Le bath est la mesure pour l'huile, l'offrande ordonnée pour l'huile sera la dixième partie d'un bath sur un cor, qui est égal à un homer de dix baths ; car dix baths feront un homer.
\VS{15}Pareillement l'offrande ordonnée des bêtes du menu bétail sera de deux cents l'une, même des meilleurs pâturages d'Israël ; toute cette offrande sera employée en gâteaux et en holocaustes, et en offrandes de paix, afin de faire propitiation pour vous, dit le Seigneur Yahweh.
\VS{16}Tout le peuple du pays sera tenu à cette offrande élevée, pour celui qui sera prince en Israël.
\VS{17}Mais le prince sera tenu de fournir les holocaustes, les offrandes et les libations qu'il faudra offrir aux fêtes solennelles, aux nouvelles lunes et aux sabbats, et dans toutes les solennités de la maison d'Israël. Il tiendra prêtes les bêtes qu'on sacrifiera pour l'expiation, et les gâteaux, et les bêtes qu'on sacrifiera pour l'holocauste, et les bêtes qu'on sacrifiera pour les offrandes de paix, afin de faire propitiation pour la maison d'Israël.
\VS{18}Ainsi parle le Seigneur Yahweh : Au premier mois, au premier jour du mois, tu prendras un jeune taureau sans défaut, et tu feras l'expiation du lieu saint.
\VS{19}Le sacrificateur prendra du sang de ce sacrifice offert pour le péché, et en mettra sur les poteaux de la maison, et sur les quatre angles de l'encadrement de l'autel, et sur les poteaux de la porte du parvis intérieur.
\VS{20}Tu en feras ainsi au septième jour du même mois, à cause des hommes qui pèchent involontairement et à cause des hommes simples ; et vous ferez ainsi propitiation pour la maison.
\VS{21}Au premier mois, au quatorzième jour du mois, vous aurez la Pâque, fête solennelle qui durera sept jours, pendant lesquels on mangera des pains sans levain\FTNT{Lé. 25:5 ; No. 9:3 ; Ex. 12.}.
\VS{22}En ce jour-là, le prince offrira un taureau pour le sacrifice d'expiation, tant pour lui que pour tout le peuple du pays.
\VS{23}Pendant les sept jours de cette fête solennelle, il offrira chaque jour sept taureaux et sept béliers sans défaut, pour l'holocauste qu'on offrira à Yahweh, et un bouc en sacrifice d'expiation, chaque jour.
\VS{24}Il offrira un épha pour chaque taureau, et un épha pour chaque bélier, avec un hin d'huile par épha.
\VS{25}Au septième mois, le quinzième jour du mois, à la fête solennelle, il offrira durant sept jours les mêmes choses, le même sacrifice expiatoire, le même holocauste, et la même offrande avec l'huile.
\Chap{46}
\TextTitle{Le service le jour du sabbat et les jours de fêtes}
\VerseOne{}Ainsi parle le Seigneur Yahweh : La porte du parvis intérieur, du côté oriental, sera fermée les six jours ouvrables, mais elle sera ouverte le jour du sabbat, elle sera aussi ouverte le jour de la nouvelle lune.
\VS{2}Et le prince y entrera par le chemin du vestibule de la porte du parvis extérieure, et se tiendra près de l'un des poteaux de l'autre porte, les sacrificateurs prépareront son holocauste et ses sacrifices d'offrande de paix ; il se prosternera sur le seuil de cette porte, et ensuite il sortira ; et la porte ne sera point fermée jusqu'au soir.
\VS{3}Tellement que le peuple du pays se prosternera devant Yahweh à l'entrée de cette porte, les jours de sabbat et des nouvelles lunes.
\VS{4}L'holocauste que le prince offrira à Yahweh le jour du sabbat sera de six agneaux sans défaut, et d'un bélier sans défaut.
\VS{5}L'offrande pour le bélier sera d'un épha, et l'offrande pour chacun des agneaux sera selon ce qu'il pourra donner ; mais il y aura un hin d'huile pour chaque épha.
\VS{6}Au jour de la nouvelle lune, son holocauste sera d'un jeune taureau, sans défaut, de six agneaux et d'un bélier, aussi sans défaut.
\VS{7}Son offrande pour le taureau sera d'un épha, pour l'offrande du bélier, un autre épha, et pour chacun des agneaux selon ce qu'il pourra donner ; mais il y aura un hin d'huile pour chaque épha.
\VS{8}Lorsque le prince entrera, il entrera par le chemin du vestibule de la porte, et il sortira par le même chemin.
\VS{9}Quand le peuple du pays entrera pour se présenter devant Yahweh, aux fêtes solennelles, celui qui y entrera par le chemin de la porte nord pour y adorer Yahweh, sortira par le chemin de la porte sud ; et celui qui y entrera par le chemin de la porte sud, sortira par le chemin de la porte nord ; personne ne retournera par le chemin de la porte par laquelle il sera entré, mais il sortira par celle qui lui est opposée.
\VS{10}Alors le prince entrera parmi eux, quand ils entreront ; et quand ils sortiront, ils sortiront ensemble.
\VS{11}Or,dans ces fêtes solennelles et dans ces solennités, l'offrande d'un taureau sera d'un épha, et l'offrande d'un bélier d'un autre épha, l'offrande de chacun des agneaux sera selon ce que le prince pourra donner, et il y aura un hin d'huile pour chaque épha.
\VS{12}Et si le prince offre un sacrifice volontaire, quelque un holocauste, soit quelques un sacrifices d'offrande de paix en offrande à Yahweh, on lui ouvrira la porte qui est du côté oriental, et il offrira son holocauste et ses sacrifices d'offrande de paix comme il les offre le jour du sabbat, puis il sortira, et après qu'il sera sorti, on fermera cette porte.
\VS{13}Tu sacrifieras chaque jour en holocauste à Yahweh un agneau d'un an sans défaut, tu le sacrifieras tous les matins.
\VS{14}Tu lui offriras tous les matins l'offrande, faite de la sixième partie d'un épha, et de la troisième d'un hin d'huile pour pétrir la farine ; c'est l'offrande à Yahweh qu'il faut offrir par ordonnances perpétuelles.
\VS{15}Ainsi on offrira tous les matins en holocauste perpétuel cet agneau et l'offrande avec cette huile.
\VS{16}Ainsi a dit le Seigneur Yahweh : quand le Prince aura fait un don de quelque pièce de son héritage à quelqu'un de ses fils, ce don appartiendra à ses fils ; parce qu'ils ont droit de possession en l'héritage.
\VS{17}Mais s'il fait un don pris de son héritage à l'un de ses serviteurs, le don lui appartiendra bien, mais seulement jusqu'à l'année de la liberté, puis il retournera au prince ; ses fils seuls posséderont ce qu'il leur donnera de son héritage\FTNT{Lé. 25:10.}.
\VS{18}Et le prince ne prendra pas de l'héritage du peuple en les opprimant, les chassant de leur possession : c'est de sa propre possession qu'il fera hériter ses fils, afin qu'aucun de mon peuple ne soit pas dispersé loin de sa possession.
\VS{19}Puis il me mena par l'entrée qui était vers le côté de la porte, aux chambres saintes qui appartenaient aux sacrificateurs, vers le nord. Et voici, il y avait un certain lieu dans le fond du côté occidental.
\VS{20}Il me dit : C'est là le lieu où les sacrificateurs feront bouillir le reste de la bête qu'on aura sacrifiée pour la culpabilité, et le reste de la bête qu'on aura sacrifiée pour l'expiation, et où ils feront cuire les offrandes, afin qu'ils ne les emportent point au parvis extérieur de manière à sanctifier le peuple\FTNT{No. 18:9.}.
\VS{21}Puis il me fit sortir vers le parvis extérieur, et me fit traverser vers les quatre angles du parvis, et voici, il y avait une cour à chacun des angles du parvis.
\VS{22}Aux quatre angles de ce parvis, il y avait des cours voûtées, longues de quarante coudées, et larges de trente ; et toutes les quatre avaient la même mesure dans les angles.
\VS{23}Un mur les entourait toutes les quatre, et des foyers étaient faits au bas du campement tout autour.
\VS{24}Et il me dit : Ce sont ici les cuisines, où ceux qui font le service de la maison cuiront les sacrifices du peuple.
\Chap{47}
\TextTitle{Les eaux pures du sanctuaire\FTNTT{Cp. Za. 14:8-9 ; Ap. 22:1-2.}}
\VerseOne{}Puis il me ramena vers l'entrée de la maison, et voici, des eaux sortaient sous le seuil de la maison, vers l'orient, car la face de la maison était vers l'orient ; et ces eaux-là descendaient du côté droit de la maison, du côté sud de l'autel\FTNT{Ps. 46:5 ; Joë. 3:18 ; Za. 13:1 ; Za. 14:8 ; Ap. 22:1.}.
\VS{2}Puis il me fit sortir par le chemin de la porte nord, et me fit faire le tour par dehors, jusqu'à la porte extérieure, du côté de l'orient, et voici, les eaux coulaient du côté droit.
\VS{3}Quand cet homme s'avança vers l'orient, il avait dans sa main un cordeau ; et il mesura mille coudées, puis il me fit traverser ces eaux-là, et j'avais de l'eau jusqu'aux chevilles.
\VS{4}Puis il mesura mille autres coudées, et me fit traverser les eaux, j'avais de l'eau jusqu'aux genoux ; puis il mesura mille autres coudées, et me fit traverser, et j'avais de l'eau jusqu' aux reins.
\VS{5}Il mesura mille autres coudées ; mais ces eaux-là étaient déjà un torrent que je ne pouvais traverser ; car ces eaux-là étaient si profondes qu'il fallait y traverser à la nage, c'était un torrent que l'on ne pouvait traverser.
\VS{6}Alors il me dit : Fils de l'homme, as-tu vu ? Puis il me fit aller et revenir vers le bord du torrent.
\VS{7}Quand je revins, il y avait un grand nombre d'arbres sur les deux bords du torrent.
\VS{8}Il me dit : Ces eaux couleront vers la Galilée orientale, et elles descendront à la campagne, puis elles entreront dans la mer, et quand elles se seront jetées dans la mer, les eaux deviendront saines.
\VS{9}Il arrivera que tout être vivant qui se meut vivra partout où les deux torrents couleront, et il y aura une grande quantité de poissons ; car là où ces eaux entreront, les eaux deviendront saines, et tout vivra là où ce torrent parviendra.
\VS{10}Il arrivera que des pêcheurs se tiendront le long de cette mer, depuis En-Guédi jusqu'à En-Eglaïm ; on étendra les filets, il y aura des poissons de diverses espèces, comme les poissons de la grande mer, et ils seront très nombreux.
\VS{11}Ses marais et ses fosses ne seront pas assainis, ils seront abandonnés au sel.
\VS{12}Auprès de ce torrent et sur ses deux bords, il croîtra des arbres fruitiers de toutes sortes. Leur feuillage ne se flétrira point, et l'on trouvera toujours du fruit. Tous les mois, ils produiront des fruits mûrs, parce que les eaux de ce torrent sortent du lieu saint, et à cause de cela leur fruit sera bon à manger, et leur feuillage servira de remède\FTNT{Ap. 22:2.}.
\TextTitle{Délimitations du pays\FTNTT{Cp. Ge. 15:18-21.}}
\VS{13}Ainsi parle le Seigneur Yahweh : Voici les frontières du pays que vous aurez en héritage, selon les douze tribus d'Israël. Joseph aura deux portions.
\VS{14}Vous en aurez la possession l'un comme l'autre de ce pays. J'ai levé ma main de le donner à vos pères ; et ce pays-là vous sera donc échu en héritage\FTNT{Ge. 12:7 ; Ge. 17:8.}.
\VS{15}Voici la frontière du pays, du côté nord, depuis la grande mer, le chemin de Hethlon jusqu'à Tsedad,
\VS{16}Hamath, Bérotha, et Sibraïm, entre la frontière de Damas et la frontière de Hamath, Hatzer-Hatthicon, vers la frontière de Havran.
\VS{17}La frontière sera depuis la mer, Hatsar-Enon ; la frontière de Damas, Tsaphon au nord et la frontière de Hamath : Ce sera le côté nord.
\VS{18}Le côté oriental sera le Jourdain entre Havran, Damas et Galaad, et le pays d'Israël ; vous mesurerez depuis la frontière nord jusqu'à la mer orientale : Ce sera le côté oriental.
\VS{19}Le côté méridional, au midi, ira depuis Thamar jusqu'aux eaux de Meriba à Kadès, jusqu'au torrent vers la grande mer : Ce sera le côté méridional.
\VS{20}Le côté occidental sera la grande mer, depuis la frontière jusque vis-à-vis de Hamath : Ce sera le côté occidental.
\VS{21}Vous partagerez ce pays entre vous, selon les tribus d'Israël.
\VS{22}Vous le diviserez en héritage par le sort pour vous et pour les étrangers qui séjourneront au milieu de vous, qui engendreront des fils au milieu de vous ; vous les regarderez comme natifs des fils d'Israël ; ils partageront au sort l'héritage avec vous parmi les tribus d'Israël.
\VS{23}Vous donnerez à l'étranger son héritage dans la tribu où il séjournera, dit le Seigneur Yahweh.
\Chap{48}
\TextTitle{Héritage de sept tribus\FTNTT{Cp. Jos. 13:1-19:51.}}
\VerseOne{}Voici les noms des tribus. Depuis l'extrémité qui regarde vers le nord, le long de la contrée du chemin de Hethlon du quartier par lequel on entre à Hamath, jusqu'à Hatsar-Enon, qui est la frontière de Damas, du côté qui regarde vers le nord, le long de la contrée de Hamath, tellement que cette extrémité est le canton de l'orient et celui de l'occident: il y aura une portion pour Dan.
\VS{2}Et tout joignant les frontières de Dan, depuis le canton de l'orient jusqu'au canton qui regarde vers l'occident : il y aura une partion pour Aser.
\VS{3}Et tout joignant les frontières d'Aser, depuis le canton qui regarde vers l'orient jusqu'au canton qui regarde vers l'occident : il y aura une partion pour Nephthali.
\VS{4}Et tout les frontières de Nephthali, depuis le canton qui regarde vers l'orient jusqu'au canton qui regarde vers  l'occident : il y aura une partion pour Manassé. 
\VS{5}Et tout joignant les frontières de Manassé, depuis le canton qui regarde vers de l'occident jusqu'au canton qui regarde vers  l'orient : il y aura une part pour Ephraïm. 
\VS{6}Et tout joignant les frontières d'Ephraïm, encore depuis le canton de l'orient,jusqu'au canton qui regarde vers l'occident: il y aura une partion pour Ruben. 
\VS{7}Et joignant les frontièrs de Ruben, depuis le canton de l'orient jusqu'au canton qui regarde vers l'occident : il y aura une partion pour Juda.
\VS{8}Et tout le long des frontières de Juda, depuis le canton de l'orient, jusqu'au canton qui regarde vers l'occident; il y aura une portion que vous prélèverez sur toute la masse du pays, comme une offrande élevée et elle aura vingt-cinq mille cannes de largueur et de longueur, autant que l'une des autres partions depuis le canton qui regarde vers l'orient jusqu'au canton qui regarde vers l'occident; de sorte que le sanctuaire sera au milieu.
\VS{9}La portion que vous lèverez pour Yahweh et offerte en offrande élevée, aura vingt-cinq mille cannes de longueur, et dix mille de largeur.
\TextTitle{Territoire réservé aux sacrificateurs et aux Lévites}
\VS{10}Et cette portion sainte sera pour les sacrificateurs, vingt-cinq mille cannes de longueur au nord, et dix mille de largeur, à l'occident, dix mille en largeur à l'orient, et vingt-cinq mille en longueur au sud, et le sanctuaire de Yahweh sera au milieu.
\VS{11}Elle sera pour les sacrificateurs, et quiconque aura été sanctifié d'entre les fils de Tsadok, qui ont fait ce que j'ai ordonné, et qui ne se sont point égarés quand les fils d'Israël se sont égarés, comme se sont égarés les autres Lévites,
\VS{12}Ceux-là auront une portion ainsi levée sur l'autre très sainte, prélevée sur la portion du pays qui aura été prélevée, à côté de la frontière des Lévites.
\VS{13}Les Lévites auront, parallèlement à la frontière des sacrificateurs, vingt-cinq mille cannes en longueur et dix mille de largeur, vingt-cinq mille pour toute la longueur et dix mille pour toute la largeur.
\VS{14}Ils n'en pourront ni vendre, ni échanger, et les prémices du pays ne seront point transgressées car elles sont mises à part pour Yahweh.
\VS{15}Les cinq mille cannes qui resteront en largeur sur les vingt-cinq mille cannes seront destinées à la ville, pour les habitations et le faubourg, et la ville sera au milieu.
\VS{16}En voici les mesures : Du côté nord, quatre mille cinq cents cannes, du côté sud, quatre mille cinq cents, du côté oriental, quatre mille cinq cents, et du côté occidental, quatre mille cinq cents.
\VS{17}Puis il y aura des faubourgs pour la ville, vers le nord. La ville aura un faubourg au nord de deux cent cinquante cannes, de deux cent cinquante au sud, de deux cent cinquante à l'orient, et de deux cent cinquante à l'occident.
\VS{18}Quant à ce qui sera de reste sur la longueur et qui sera tout joignant à la portion sanctifiée, et qui aura dix mille cannes à l'orient, et dix mille autres cannes à l'occident, parallèlement à la portion sanctifiée, le revenu qu'on en tirera sera pour nourriture de ceux qui feront le service qu'il faut dans la ville.
\VS{19}Le sol sera travaillé par ceux de toutes les tribus d'Israël qui travailleront pour la ville.
\VS{20}Toute la portion prélevée sera de vingt-cinq mille cannes en longueur sur vingt-cinq mille en largeur ; vous en séparerez un carré pour la propriété de la ville.
\VS{21}PUis le reste sera pour le prince aux deux côtés de la portion sainte et de la possession de la ville, des vingt-cinq mille coudées de la portion prélevée jusqu'à la frontière de l'orient, et à l'occident, des vingt-cinq mille coudées jusqu'à la frontière de l'occident, le long des parts, pour le prince, et la portion sainte et le sanctuaire de la Maison seront au milieu de tout le pays. 
\VS{22}Ce qui sera donc pour le prince sera l'espace compris depuis la possession des Lévites, et depuis la possession de la ville ; ce qui sera entre ces possessions là et la frontière de Juda, et la frontière de Benjamin, sera pour le prince.
\TextTitle{Héritage de cinq tribus}
\VS{23}Or ce qui sera de reste sera pour les autres tribus; depuis le canton de ce qui regarde vers l'orient, jusqu'au canton de ce qui regarde vers l'occident, il y aura une portion pour Benjamin.
\VS{24}Puis tout joignant les frontières de Benjamin, depuis le canton de ce qui regarde vers l'orient, jusqu'au canton de ce qui regarde vers l'occident, il y aura une autre portion pour Siméon.
\VS{25}Puis tout joignant les frontières de Siméon, depuis le canton de ce qui regarde vers l'orient, jusqu'au canton de ce qui regarde vers l'occident, il y aura une autre portion pour Issacar.
\VS{26}Puis tout joignant sur les frontières d'Issacar, depuis le canton de ce qui regarde vers l'orient, jusqu'au canton de ce qui regarde vers l'occident, il y aura une autre portion pour Zabulon.
\VS{27}Puis joignant les frontières de Zabulon, depuis le canton de ce qui regarde vers l'orient, jusqu'au canton de ce qui regarde vers l'occident, il y aura une autre portion pour Gad.
\VS{28}Or ce qui appartient au côté du midi qui regarde proprement le vent d'autant, et sur la frontière de Gad, et cette frontière sera depuis Thamar jusqu'aux eaux de contestation, à Kadès, le long du torrent jusqu'à la grande mer.
\VS{29}C'est là le pays que vous partagerez par le sort en héritage aux tribus d'Israël, et ce sont là leurs portions, dit le Seigneur Yahweh.
\VS{30}Et ce sont ici les sorties de la ville : Du côté du nord, il y aura quatre mille cinq cents mesures.
\VS{31}Et les portes de la ville seront selon les noms des tribus d'Israël : Trois portes vers le nord, une porte de Ruben, une porte de Juda, une porte de Lévi.
\VS{32}Et du côté de l'orient, quatre mille cinq cents mesures, et trois portes : Une porte de Joseph, une porte de Benjamin, une porte de Dan.
\VS{33}Et du côté sud, quatre mille cinq cents mesures, et trois portes : Une porte de Simeon, une porte d'Issacar, une porte de Zabulon.
\VS{34}Et du côté ouest, quatre mille cinq cents mesures, avec leurs trois portes : Une porte de Gad, une porte d'Aser, une porte de Nephthali.
\VS{35}Ainsi le circuit de la ville sera de dix-huit mille mesures ; et le nom de la ville depuis ce jour-là sera : Yahweh est ici.
\PPE{}
\end{multicols}

\clearpage\ShortTitle{Osée}\BookTitle{Osée}\BFont
\noindent\hrulefill
{\footnotesize
\textit{
\bigskip
{\centering{}
\\Auteur : Osée
\\(Heb. : Hoswhéa')
\\Signification : Salut, Sauve
\\Thème : Israël sera rejeté à cause de son apostasie. D'autres nations seront appelées à sa place.
\\Date de rédaction : 8\up{ème} siècle av. J.-C.\\}
}
%\bigskip
\textit{
\\Osée, fils de Béeri, exerça son service dans le royaume du nord au temps de Joas, roi d'Israël. Il était contemporain des prophètes Amos, Michée et Esaïe.
%\bigskip
\\Yahweh demanda à Osée d'épouser une prostituée pour que le prophète puisse partager plus profondément son fardeau
et la tristesse qu'il subissait en raison de l'infidélité du peuple qu'il aimait tant. Malgré la rébellion et les mauvais
agissements d'Israël, Yahweh manifesta une fois de plus sa patience et utilisa Osée pour avertir et inviter les fils de Jacob à la repentance.\bigskip
}
}
\par\nobreak\noindent\hrulefill
\begin{multicols}{2}
\Chap{1}
\VerseOne{}La parole de Yahweh qui fut adressée à Osée fils de Béeri, au temps d'Ozias, de Jotham, d'Achaz et d'Ezéchias, rois de Juda, et au temps de Jéroboam, fils de Joas, roi d'Israël.
\TextTitle{Mariage d'Osée et naissance de Jizreel}
\VS{2}La première fois que Yahweh parla à Osée, Yahweh dit à Osée : Va, prends une femme prostituée, et aie d'elle des enfants de prostitution ; car le pays s'est entièrement prostitué en abandonnant Yahweh !
\VS{3}Il alla, et il prit Gomer, fille de Diblaïm. Elle conçut, et lui enfanta un fils.
\VS{4}Et Yahweh lui dit : Donne-lui le nom de Jizreel ; car encore un peu de temps, et je punirai la maison de Jéhu pour le sang versé à Jizreel, et je ferai cesser le royaume de la maison d'Israël\FTNT{Cette prophétie s'est accomplie en 722 av. J-C. Voir 2 R.17.}.
\VS{5}Et il arrivera qu'en ce jour-là, je briserai l'arc d'Israël dans la vallée de Jizreel.
\TextTitle{Naissance de Lo-Ruchama}
\VS{6}Elle conçut encore, et enfanta une fille. Et Yahweh lui dit : Donne-lui le nom de Lo-Ruchama\FTNT{Tout au long de ce livre, certains mots sont symboliquement utilisées pour nommer Israël. A savoir : 
\\- « ruchama » : miséricorde ;
\\- « Lo-Ruchama » : celle dont on ne fait pas miséricorde ; 
\\- « ammi » : mon peuple ;
\\- « Lo-Ammi » : pas mon peuple.
\\Ces noms illustrent ainsi l'infidélité d'Israël envers Yahweh.} ; car je ne continuerai plus à faire miséricorde à la maison d'Israël, mais je les enlèverai entièrement.
\VS{7}Mais je ferai miséricorde à la maison de Juda; et je les délivrerai par Yahweh, leur Dieu, et je ne les délivrerai ni par l'arc, ni par l'épée, ni par les combats, ni par les chevaux, ni par les cavaliers.
\TextTitle{Naissance de Lo-Ammi}
\VS{8}Puis quand elle sevra Lo-Ruchama ; elle conçut, et enfanta un fils.
\VS{9}Et Yahweh dit : Appelle-le du nom de Lo-Ammi ; car vous n'êtes point mon peuple, et je ne suis pas votre Dieu.
\Chap{2}
\TextTitle{Futur rétablissement d'Israël}
\VerseOne{}Cependant le nombre des fils d'Israël sera comme le sable de la mer, qui ne peut ni se mesurer ni se compter ; et dans la ville où il leur est dit : Vous n'êtes pas mon peuple ! On leur dira : Vous êtes les fils du Dieu vivant !
\VS{2}Aussi les fils de Juda et les fils d'Israël se rassembleront, et ils s'établiront un chef, et monteront hors du pays ; car la journée de Jizreel sera grande.
\VS{3}Dites à vos frères : Ammi ! Et à vos sœurs : Ruchama !
\TextTitle{Châtiment d'Israël, la prostituée\FTNTT{2 R. 17:1-18.}}
\VS{4}Plaidez, plaidez contre votre mère, car elle n'est point ma femme, et je ne suis point son mari ! Qu'elle ôte ses prostitutions de son visage, et ses adultères de son sein !
\VS{5}De peur que je ne la dépouille à nu, et que je ne l'expose comme au jour de sa naissance, et que je ne la rende semblable à un désert, à une terre aride, et ne la fasse mourir de soif ;
\VS{6}et je n'aurai point de miséricorde pour ses enfants, car ce sont des enfants de prostitution.
\VS{7}Car leur mère s'est prostituée, celle qui les a conçus s'est déshonorée, car elle a dit : Je m'en irai après mes amants, qui me donnent mon pain et mes eaux, ma laine et mon lin, mon huile, et mes boissons.
\VS{8}C'est pourquoi voici, je vais fermer ton chemin avec des épines, j'y élèverai un mur, afin qu'elle ne trouve plus ses sentiers.
\VS{9}Elle poursuivra ses amants, mais ne les atteindra pas ; elle les cherchera, mais elle ne les trouvera point. Puis elle dira : Je m'en irai, et je retournerai vers mon premier mari, car alors j'étais plus heureuse que maintenant.
\VS{10}Mais elle n'a pas reconnu que c'était moi qui lui donnais le blé, le vin et l'huile ; et l'on a fait des offrandes à Baal\FTNT{Baal : Voir commentaire en Jg. 2:13.} avec l'argent et l'or que je lui prodiguais.
\VS{11}C'est pourquoi je reprendrai mon froment en son temps, mon vin en sa saison, et je retirerai ma laine et mon lin qui couvraient sa nudité.
\VS{12}Et maintenant je découvrirai sa honte aux yeux de ses amants, et personne ne la délivrera de ma main.
\VS{13}Je ferai cesser toute sa joie, ses fêtes, ses nouvelles lunes, ses sabbats, et toutes ses solennités.
\VS{14}Je ravagerai ses vignes et ses figuiers, dont elle disait : Voici le salaire que mes amants m'ont donné ! Je les réduirai en une forêt, et les bêtes des champs les dévoreront.
\VS{15}Je la punirai pour les jours où elle encensait les Baals, où elle se parait de ses anneaux et de ses colliers, et s'en allait après ses amants, et m'oubliait, dit Yahweh.
\TextTitle{La femme adultère revient dans son foyer : Israël revient à Yahweh}
\VS{16}Néanmoins, voici, je veux l'attirer et la mener au désert, là je parlerai à son cœur.
\VS{17}Là, je lui accorderai ses vignes et la vallée d'Acor, telle une porte d'espérance, et là, elle chantera comme au temps de sa jeunesse, et comme au jour où elle remonta du pays d'Egypte.
\VS{18} Et il arrivera en ce jour-là, dit Yahweh, tu m'appelleras: Mon Mari ! Et tu ne m'appelleras plus: Mon Maître\FTNT{Littéralement : Baal.} !
\VS{19}Car j'ôterai de sa bouche les noms des Baals, et on ne fera plus mention de leurs noms.
\VS{20}Aussi en ce temps-là, je traiterai pour eux une alliance avec les bêtes des champs, avec les oiseaux du ciel, et avec les reptiles de la terre ; je briserai et j'ôterai du pays l'arc, l'épée et la guerre, et je les ferai se reposer en sécurité.
\VS{21}Et je te fiancerai pour moi à toujours ; je te fiancerai, dis-je, pour moi, par la justice, la droiture, la grâce et la miséricorde;
\VS{22}je serai ton fiancé par la fidélité, et tu reconnaîtras Yahweh.
\VS{23}Et il arrivera en ce jour-là, j'exaucerai, dit Yahweh, je témoignerai aux cieux, et les cieux exauceront la terre;
\VS{24}la terre exaucera le blé, le bon vin et l'huile, et ils exauceront Jizreel. 
\VS{25}Puis, je la sèmerai pour moi dans ce pays, et je ferai miséricorde à Lo-Ruchama ; je dirai à Lo-Ammi : Tu es mon peuple ! Et il me répondra : Mon Dieu !
\Chap{3}
\TextTitle{Soumission d'Israël à Yahweh}
\VerseOne{}Après cela Yahweh me dit : Va encore, et aime une femme aimée d'un ami, et adultère; aime-la comme Yahweh aime les enfants d'Israël, qui se tournent toutefois vers d'autres dieux et aiment les gâteaux de raisins.
\VS{2}J'achetai donc cette femme pour quinze pièces d'argent, un homer et demi d'orge\FTNT{Voir dans les annexes les tableaux des poids et mesures.}.
\VS{3}Et je lui dis : Assieds-toi avec moi pendant plusieurs jours, ne t'abandonne plus à la prostitution, ne sois à aucun homme, et je serai fidèle envers toi.
\VS{4}Car les enfants d'Israël demeureront plusieurs jours sans roi, sans chef, sans sacrifice, sans statue, sans éphod, et sans théraphim\FTNT{Cette prophétie s'est accomplie d'une manière extraordinaire à travers l'histoire du peuple d'Israël depuis la première venue de Jésus-Christ. Les Israélites étaient dispersés, sans unité politique faute de roi, empêchés d'offrir des sacrifices depuis la destruction du temple par Titus (39-81), fils de l'empereur romain Vespasien (9-79), en l'an 70.}.
\VS{5}Mais après cela, les enfants d'Israël se repentiront\FTNT{La repentance et la conversion nationale d'Israël auront lieu lors du retour du Messie (Es. 59:20-21 ; Ro. 11:26-27). Dieu n'a pas abandonné son peuple, il viendra lui-même le délivrer.}; et rechercheront Yahweh, leur Dieu, et David, leur roi; ils seront dans la crainte à la vue de Yahweh et de sa bonté, dans les derniers jours\FTNT{Les derniers jours : Voir commentaire en Ge. 49:1.}.
\Chap{4}
\TextTitle{Israël, la nation pécheresse}
\VerseOne{}Ecoutez la parole de Yahweh, fils d'Israël ! Car Yahweh a un procès avec les habitants du pays, parce qu'il n'y a ni de vérité, ni de miséricorde, ni de connaissance de Dieu dans le pays.
\VS{2}Il n'y a que parjures et mensonges, meurtres, vols et adultères ; on use de violence, et un meurtre touche l'autre.
\VS{3}C'est pourquoi le pays sera dans le deuil, et tous ceux qui l'habitent seront languissants, et avec eux toutes les bêtes des champs et tous les oiseaux du ciel ; même les poissons de la mer périront.
\VS{4}Mais que nul ne conteste, et que nul ne reprenne ; car ton peuple est comme ceux qui disputent avec le sacrificateur.
\VS{5}Tu tomberas donc en plein jour, et le prophète aussi tombera avec toi de nuit, et j'exterminerai ta mère.
\TextTitle{Israël dans l'ignorance}
\VS{6}Mon peuple est détruit, parce qu'il lui manque la connaissance\FTNT{Le verbe « détruire » vient de l'hébreu « damah » qui signifie aussi « égorger ». Satan est celui qui vient égorger, dérober et détruire, notamment avec ses faux prophètes (Jn. 10:10). Chaque disciple de Jésus-Christ doit avoir une vie de prière et de méditation quotidienne afin de résister aux attaques de l'ennemi.}. Parce que tu as rejeté la connaissance, je te rejetterai, afin que tu n'exerces plus la sacrificature; puisque tu as oublié la loi de ton Dieu, moi aussi j'oublierai tes enfants.
\VS{7}Plus ils se sont multipliés, plus ils ont péché contre moi : Je changerai leur gloire en ignominie.
\VS{8}Ils se nourrissent des péchés de mon peuple, leur âme soutient leur iniquité.
\VS{9}C'est pourquoi le sacrificateur sera traité comme le peuple; je le châtierai selon ses voies, et je lui rendrai selon ses œuvres.
\VS{10}Et ils mangeront mais ils ne seront point rassasiés, ils se prostitueront mais ils ne multiplieront point, parce qu'ils ont cessé de prendre garde à Yahweh.
\VS{11}La prostitution, le vin et le moût, font perdre l'entendement.
\TextTitle{Israël dans l'idolâtrie}
\VS{12}Mon peuple consulte son bois, et c'est son bâton qui lui répond ; car l'esprit de prostitution égare, et ils se prostituent loin de leur Dieu.
\VS{13}Ils sacrifient sur le sommet des montagnes, ils brûlent de l'encens sur les collines, sous les chênes, sous les peupliers, et les térébinthes, parce que leur ombrage est agréable. C'est pourquoi vos filles se prostituent, et vos belles-filles commettent l'adultère.
\VS{14}Je ne punirai pas vos filles parce qu'elles se prostituent, ni vos belles-filles parce qu'elles commettent l'adultère, car eux-mêmes se retirent avec des prostituées, et sacrifient avec des femmes débauchées. Ainsi le peuple qui est sans intelligence sera ruiné.
\VS{15}Si tu te prostitues, ô Israël, au moins que Juda ne se rende point coupable ! N'entrez donc point dans Guilgal, et ne montez pas à Beth-Aven, et ne jurez point : Yahweh est vivant !
\VS{16}Parce qu'Israël se révolte comme une vache indomptable, maintenant Yahweh les fera paître comme des agneaux dans de vastes plaines.
\VS{17}Ephraïm s'est associé aux idoles ; abandonne-le !
\VS{18}Leur breuvage est devenu aigre ; ils n'ont fait que se prostituer ; ils n'aiment qu'à dire : apportez ; ce n'est qu'ignominie que ses protecteurs.
\VS{19}Le vent l'a enfermé dans ses ailes, et ils auront honte de leurs sacrifices.
\Chap{5}
\TextTitle{Yahweh abandonne son peuple}
\VerseOne{}Ecoutez ceci, sacrificateurs ! Maison d'Israël, sois attentive ! Maison du roi, tendez l'oreille ! Car c'est à vous que s'adresse le jugement, parce que vous avez été un piège à Mitspa, et un filet tendu sur le Thabor.
\VS{2}Les infidèles s'enfoncent dans le crime. Et moi, je les châtierai tous.
\VS{3}Je connais Ephraïm, et Israël ne m'est point caché; car maintenant, Ephraïm, tu t'es prostitué, et Israël est souillé.
\VS{4}Leurs œuvres ne leur permettent pas de revenir à leur Dieu, parce que l'esprit de prostitution est au milieu d'eux, et parce qu'ils ne connaissent point Yahweh.
\VS{5}L'orgueil d'Israël témoigne contre lui; Israël et Ephraïm tomberont par leur iniquité ; Juda aussi tombera avec eux.
\VS{6}Ils iront avec leurs brebis et leurs bœufs chercher Yahweh, mais ils ne le trouveront point, il s'est retiré du milieu d'eux.
\VS{7}Ils se sont montrés infidèles envers Yahweh, car ils ont engendré des fils étrangers ; maintenant un mois suffira pour les dévorer avec leurs biens.
\VS{8}Sonnez du shofar à Guibea. Sonnez de la trompette à Rama ! Poussez des cris de guerre à Beth-Aven ! Derrière toi, Benjamin !
\VS{9}Ephraïm sera un sujet d'épouvante au jour du châtiment ; je le fais savoir parmi les tribus d'Israël comme une chose certaine.
\VS{10}Les chefs de Juda sont comme ceux qui déplacent les bornes; je répandrai sur eux ma fureur comme un torrent.
\VS{11}Ephraïm est opprimé, brisé par le jugement, car il a vécu selon les préceptes qui lui plaisaient.
\VS{12}Je serai comme une teigne pour Ephraïm, comme de la pourriture pour la maison de Juda.
\VS{13}Ephraïm voit sa maladie, et Juda ses plaies ; Ephraïm s'en est allé vers le roi d'Assyrie, et s'est adressé au roi Jareb. Mais ce roi ne pourra ni vous guérir, ni panser vos plaies.
\VS{14}Je serai comme un lion pour Ephraïm, comme un lionceau pour la maison de Juda. Moi, moi je déchirerai, puis je m'en irai, j'emporterai la proie, et nul ne me l'enlèvera.
\TextTitle{Israël revient à Yahweh}
\VS{15}Je m'en irai, je reviendrai dans ma demeure, jusqu'à ce qu'ils se reconnaissent coupables, et qu'ils cherchent ma face. Quand ils seront dans la détresse, dans leur angoisse, ils me chercheront.
\Chap{6}
\VerseOne{}Venez, retournons à Yahweh ! Car il a déchiré, mais il nous guérira ; il a frappé, mais il bandera nos plaies.
\VS{2}Il nous rendra la vie dans deux jours; et le troisième jour il nous relèvera, et nous vivrons en sa présence.
\VS{3}Car nous connaîtrons Yahweh, et nous continuerons à le connaître ; sa venue\FTNT{Il est question ici du retour du Seigneur Jésus-Christ : Voir  commentaire en Za. 14:1.} est aussi certaine que celle de l'aurore. Il viendra pour nous comme la pluie, comme la pluie de l'arrière-saison\FTNT{Voir commentaire en Joë. 2:23.} qui arrose la terre.
\TextTitle{Yahweh dénonce le péché d'Ephraïm}
\VS{4}Que te ferai-je, Ephraïm ? Que te ferai-je, Juda ? Votre piété est comme la nuée du matin, comme la rosée qui se dissipe dès le matin.
\VS{5}C'est pourquoi je les taillerai en pièces par mes prophètes, je les tuerai par les paroles de ma bouche\FTNT{Hé. 4:12 ; Ap. 1:16 ; Ap. 19:15.}, et mes jugements apporteront la lumière.
\VS{6}Car je prends plaisir à la miséricorde et non aux sacrifices, et à la connaissance de Dieu plus qu'aux holocaustes.
\VS{7}Mais ils ont transgressé l'alliance, comme si elle avait été d'un homme, en quoi ils se sont portés perfidement contre moi.
\VS{8}Galaad est une ville d'ouvriers d'iniquité, couverte de traces de sang.
\VS{9}Et comme les bandes des voleurs attendent quelqu'un, ainsi les sacrificateurs, après avoir comploté, tuent les gens sur le chemin du côté de Sichem ; car ils exécutent leurs méchants desseins.
\VS{10}J'ai vu des choses infâmes dans la maison d'Israël: Là Ephraïm se prostitue, Israël en est souillé.
\VS{11}A toi aussi Juda, une moisson est préparée, quand je ramènerai les captifs de mon peuple.
\Chap{7}
\TextTitle{Transgression d'Ephraïm}
\VerseOne{}Lorsque je guérissais Israël, l'iniquité d'Ephraïm et la méchanceté de Samarie se sont révélées, car ils ont agi frauduleusement ; le voleur vient tandis que la bande dépouille au-dehors.
\VS{2}Ils n'ont point pensé dans leur cœur que je me souviens de toute leur méchanceté ; maintenant leurs œuvres les entourent, elles sont devant ma face.
\VS{3}Ils réjouissent le roi par leur méchanceté, et les chefs par leurs mensonges.
\VS{4}Ils sont tous adultères, comme un four allumé par le boulanger : Il cesse d'attiser le feu depuis qu'il a pétri la pâte jusqu'à ce qu'elle soit levée.
\VS{5}Au jour de notre roi, les chefs se rendent malades par les excès de vin ; il tend la main aux moqueurs.
\VS{6}Lorsqu'ils dressent des embuscades, leur cœur s'embrase comme un four ; leur boulanger dort toute la nuit, le matin le four est embrasé comme un feu accompagné de flammes.
\VS{7}Ils sont tous ardents comme un four, et ils dévorent leurs chefs ; tous leurs rois tombent, et il n'y a aucun d'entre eux qui crie à moi.
\VS{8}Ephraïm se mêle avec les peuples, Ephraïm est un gâteau qui n'a pas été retourné.
\VS{9}Les étrangers ont dévoré sa force, et il ne s'en doute pas ; les cheveux gris sont aussi parsemés sur lui, et il ne s'en doute pas.
\VS{10}L'orgueil d'Israël rendra témoignage contre lui ; car ils ne reviennent pas à Yahweh, leur Dieu, et ils ne le recherchent pas malgré tout cela. 
\VS{11}Ephraïm est comme une colombe troublée, sans intelligence ; car ils appellent l'Egypte, et s'en vont vers le roi d'Assyrie.
\VS{12}Quand ils s'en iront, j'étendrai mon filet sur eux, et je les précipiterai comme les oiseaux du ciel ; je les châtierai, comme ils en ont été avertis au sein de leurs assemblées.
\VS{13}Malheur à eux, parce qu'ils me fuient ! Ruine sur eux, car ils se révoltent contre moi ! Je voudrais les sauver, mais ils profèrent contre moi des paroles mensongères.
\VS{14}Ils ne crient pas vers moi dans leur cœur, mais ils gémissent sur leurs couches ; ils se rassemblent pour le froment et le bon vin, et ils s'éloignent de moi.
\VS{15}Je les ai châtiés, et j'ai fortifié leurs bras, mais ils méditent le mal contre moi.
\VS{16}Ce n'est pas au Très-Haut qu'ils retournent ; ils sont comme un arc trompeur. Leurs chefs tomberont par l'épée, à cause de l'insolence de leur langue. C'est ce qui en fera un objet de moquerie dans le pays d'Egypte.
\Chap{8}
\TextTitle{Conséquences de la désobéissance}
\VerseOne{}Crie comme si tu avais un shofar dans ta bouche ! Il vient comme un aigle contre la maison de Yahweh, parce qu'ils ont transgressé mon alliance, et qu'ils ont agi méchamment contre ma loi.
\VS{2}Ils crieront à moi : Mon Dieu, nous te connaissons, dira Israël !
\VS{3}Israël a rejeté le bien ; l'ennemi le poursuivra.
\VS{4}Ils ont fait régner, mais non pas de ma part, ils ont établi des chefs, et je n'en ai rien su ; ils se sont fait des idoles avec leur argent et leur or ; c'est pourquoi ils seront retranchés.
\VS{5}Samarie, ton veau t'a chassée loin ! Ma colère s'est embrasée contre eux. Jusqu'à quand ne pourront-ils pas s'adonner à l'innocence ?
\VS{6}Car il vient d'Israël, c'est un orfèvre qui l'a fait, et il n'est pas Dieu ; c'est pourquoi le veau de Samarie sera mis en pièces.
\VS{7}Parce qu'ils ont semé du vent, ils moissonneront la tempête ; ils n'auront pas un épi de blé ; le grain qui poussera ne donnera point de farine, et s'il en faisait, les étrangers la dévoreraient.
\VS{8}Israël est dévoré ! Il est maintenant parmi les nations comme un vase dont on ne se soucie pas.
\VS{9}Car ils sont montés vers le roi d'Assyrie, qui est un âne sauvage se tenant seul à part ; Ephraïm a fait des présents à ceux qui l'aimait.
\VS{10}Et parce qu'ils ont fait des présents aux nations, je les rassemblerai maintenant ; et ils commenceront à être amoindris à cause de l'impôt pour le roi des princes.
\VS{11}Parce qu'Ephraïm a fait plusieurs autels pour pécher, ils auront des autels pour pécher.
\VS{12}Je lui ai écrit les grandes choses de ma loi, mais elles sont estimées comme des lois étrangères.
\VS{13}Quant aux sacrifices qui me sont offerts, ils sacrifient de la chair, et la mangent ; mais Yahweh ne les accepte point. Et maintenant il se souviendra de leur iniquité, et punira leurs péchés ; ils retourneront en Egypte.
\VS{14}Israël a oublié celui qui l'a fait, et il a bâti des palais ; et Juda a multiplié les villes fortes ; c'est pourquoi j'enverrai le feu dans les villes de celui-ci, quand il aura dévoré les palais de celui-là.
\Chap{9}
\TextTitle{Ephraïm châtié et rejeté}
\VerseOne{}Israël, ne te réjouis point, ne sois pas dans l'allégresse, comme les autres peuples, de ce que tu t'es prostitué en abandonnant ton Dieu, de ce que tu as obtenu un salaire de tes amants dans toutes les aires à blé !
\VS{2}L'aire et la cuve ne les nourriront pas, et le vin doux les trompera.
\VS{3}Ils ne resteront pas dans le pays de Yahweh; Ephraïm retournera en Egypte, et ils mangeront en Assyrie ce qui est impur.
\VS{4}Ils ne feront pas d'aspersions de vin à Yahweh : Elles ne lui seraient point agréables. Leurs sacrifices seront pour eux comme le pain de deuil ; tous ceux qui en mangeront se rendront impurs ; car leur pain ne sera que pour eux, il n'entrera point dans la maison de Yahweh.
\VS{5}Que ferez-vous aux jours des fêtes solennelles, aux jours des fêtes de Yahweh ?
\VS{6}Car voici, ils partent à cause de la dévastation ; l'Egypte les recueillera, Moph les enterrera ; ce qu'ils ont de précieux, leur argent, sera la proie des ronces, et l'épine sera dans leurs tentes.
\VS{7}Les jours du châtiment sont venus, les jours de la rétribution sont venus, et Israël le saura ! Les prophètes sont fous, les hommes de révélation sont insensés, à cause de la grandeur de ton iniquité, et de ta grande aversion.
\VS{8}Ephraïm est une sentinelle avec mon Dieu ; mais le prophète est un filet d'oiseleur sur toutes ses voies, en rébellion contre la maison de son Dieu.
\VS{9}Ils se sont profondément corrompus, comme aux jours de Guibea ; Yahweh se souviendra de leur iniquité, il punira leurs péchés.
\VS{10}J'ai, dira-t-il, trouvé Israël comme des raisins dans le désert ; j'ai vu vos pères comme les premiers fruits d'un figuier ; mais ils sont allés vers Baal-Peor\FTNT{Baal-Peor, « seigneur de la brèche », était une divinité adorée à Peor avec des rites licencieux (No. 23:28 ; No. 25:1-3 ; Ps. 106:28-29).}, ils se sont consacrés à l'infâme idole, et ils sont devenus abominables comme ce qu'ils ont aimé.
\VS{11}La gloire d'Ephraïm s'envolera comme un oiseau : Point d'enfantement, point de grossesse, point de conception.
\VS{12}Que s'ils élèvent leurs enfants, je les en priverai tellement, que pas un d'entre eux ne deviendra homme ; malheur à eux, quand je me retirerai d'eux !
\VS{13}Ephraïm était comme j'ai vu Tyr, plantée dans un lieu agréable ; mais Ephraïm mènera ses fils à celui qui les tuera.
\VS{14}Ô Yahweh, donne-leur ! Mais que leur donnerais-tu ? Donne-leur un sein qui avorte et des mamelles desséchées.
\VS{15}Toute leur méchanceté s'est manifestée à Guilgal ; c'est là que je les ai pris en aversion. Je les chasserai de ma maison à cause de la malice de leurs actions. Je ne les aimerai plus ; tous leurs chefs sont des rebelles.
\VS{16}Ephraïm est frappé, sa racine est devenue sèche ; ils ne porteront plus de fruit ; et s'ils engendrent des enfants, je mettrai à mort les fruits désirables de leur ventre.
\VS{17}Mon Dieu les rejettera, parce qu'ils ne l'ont point écouté, et ils seront vagabonds parmi les nations.
\Chap{10}
\TextTitle{Yahweh annonce la destruction du royaume d'Israël}
\VerseOne{}Israël était une vigne dévastée, elle ne fait de fruit que pour elle-même. Selon l'abondance de son fruit, il a multiplié les autels ; selon la beauté de son pays, il a rendu belles ses statues.
\VS{2}Leur cœur est partagé. Ils vont être déclarés coupables. Yahweh renversera leurs autels, il détruira leurs statues.
\VS{3}Car bientôt ils diront : Nous n'avons point de roi, parce que nous n'avons point craint Yahweh ; et le roi, que pourrait-il faire pour nous ?
\VS{4}Ils prononcent des paroles vaines, des faux serments, lorsqu'ils concluent une alliance. C'est pourquoi le châtiment germera dans les sillons des champs, comme une plante vénéneuse.
\VS{5}Les habitants de Samarie seront épouvantés à cause des jeunes vaches de Beth-Aven; car le peuple mènera deuil sur son idole ; et les prêtres de ses idoles, qui s'en étaient réjouis, mèneront deuil parce que sa gloire est transportée loin d'elle.
\VS{6}Elle sera transportée en Assyrie, pour en faire un présent au roi Jareb. Ephraïm sera dans la confusion, et Israël aura honte de ses desseins.
\VS{7}C'en est fait de Samarie, et de son roi, qui sera retranché comme l'écume qui est à la surface des eaux.
\VS{8}Les hauts lieux de Beth-Aven, qui sont le péché d'Israël, seront détruits ; l'épine et la ronce croîtront sur leurs autels. Et on dira aux montagnes : Couvrez-nous ! Et aux collines : Tombez sur nous !
\VS{9}Israël, tu as péché dès les jours de Guibea ! Là ils restèrent debout, la guerre contre les fils d'iniquité ne les atteignit pas à Guibea.
\VS{10}Je les châtierai selon ma volonté, et les peuples s'assembleront contre eux, lorsqu'on les enchaînera pour leur double iniquité.
\VS{11}Ephraïm est une génisse bien dressée, qui aime à fouler le blé, mais je m'approcherai de son superbe cou ; j'attellerai Ephraïm, Juda labourera, Jacob brisera ses mottes.
\VS{12}Semez selon la justice, moissonnez selon la miséricorde, défrichez-vous un champ nouveau ! Car il est temps de chercher Yahweh, jusqu'à ce qu'il vienne, et répande sur vous sa justice.
\VS{13}Vous avez cultivé la méchanceté, et vous avez moissonné l'iniquité, vous avez mangé le fruit du mensonge; parce que vous avez eu confiance dans vos voies, dans la multitude de vos vaillants hommes.
\VS{14}Il s'élèvera un tumulte parmi ton peuple, et on détruira toutes tes forteresses, comme Schalman a détruit Beth-Arbel, au jour de la bataille, où la mère fut écrasée avec les enfants.
\VS{15}Béthel vous fera de même, à cause de votre extrême méchanceté ; le roi d'Israël sera entièrement exterminé dès l'aurore.
\Chap{11}
\TextTitle{L'amour de Yahweh pour Israël}
\VerseOne{}Quand Israël était jeune enfant, je l'aimais, et j'appelai mon fils hors d'Egypte\FTNT{La sortie des Hébreux de l'Egypte sous Moïse était une préfiguration de celle de Jésus-Christ lorsqu'il fuyait le massacre décrété par Hérode (Mt. 2:15).}.
\VS{2}Lorsqu'on les appelait ils se sont éloignés ; ils ont sacrifié aux Baals, et offert de l'encens aux idoles.
\VS{3}J'appris à Ephraïm à marcher en le prenant par les bras; et ils n'ont pas vu que je les guérissais.
\VS{4}Je les tirai avec des liens d'humanité, et avec des cordages d'amour, et je fus pour eux comme ceux qui enlèveraient le joug de dessus leur mâchoire, et je leur présentai de la nourriture.
\VS{5}Ils ne retourneront pas au pays d'Egypte ; mais le roi d'Assyrie sera leur roi, parce qu'ils n'ont point voulu revenir à moi.
\VS{6}L'épée fondra sur leurs villes, les réduira à néant, consumera leurs forces, et les dévorera, à cause des desseins qu'ils ont eus.
\VS{7}Mon peuple tient à se détourner de moi ; on les appelle vers le Très-Haut, mais aucun d'eux ne l'exalte.
\VS{8}Que ferai-je de toi, Ephraïm ? Te livrerais-je, Israël ? Te traiterai-je comme Adma ? Te rendrai-je semblable à Tseboïm ? Mon cœur s'agite au-dedans de moi, mes compassions sont émues.
\VS{9}Je n'exécuterai pas l'ardeur de ma colère, je ne reviendrai pas pour détruire Ephraïm ; car je suis Dieu, et non pas un homme, je suis le Saint au milieu de toi; et je n'entrerai point dans la ville.
\VS{10}Ils marcheront après Yahweh, qui rugira comme un lion\FTNT{Yahweh rugit comme un lion : Jésus-Christ est le lion de la tribu de Juda, car selon la chair, il est issu de la postérité de Juda (Lu. 3:23-38 ; Ap. 5:5). Le lion est le roi des animaux, or Jacob fut le premier a avoir annoncé la venue du Schilo, c'est-à-dire celui à qui appartient le sceptre (Ge. 49:8-12).}, et quand il rugira, les enfants accourront en hâte de la mer.
\VS{11}Ils accourront en hâte hors d'Egypte, comme des oiseaux, et hors du pays d'Assyrie, comme des colombes. Et je les ferai habiter dans leurs maisons, dit Yahweh.
\Chap{12}
\TextTitle{Dénonciation du péché d'Ephraïm}
\VerseOne{}Ephraïm m'entoure avec des mensonges, et la maison d'Israël avec des tromperies ; lorsque Juda erre sans frein vis-à-vis du Dieu Puissant, vis-à-vis du Saint fidèle.
\VS{2}Ephraïm se repaît de vent, et poursuit le vent d'orient ; il multiplie chaque jour le mensonge et la violence, et il traite alliance avec l'Assyrie, et l'on porte des huiles de senteur en Egypte.
\VS{3}Yahweh a aussi un procès avec Juda, et il punira Jacob pour sa conduite, il lui rendra selon ses œuvres.
\VS{4}Dans le ventre Jacob saisit son frère par le talon\FTNT{Ge. 25:26}, puis dans sa vigueur, il lutta avec Dieu\FTNT{Ge. 32:24-28.}.
\VS{5}Il lutta avec l'Ange, et il fut vainqueur, il pleura, et lui demanda grâce. Jacob l'avait rencontré à Béthel, et c'est là que Dieu nous a parlé.
\VS{6}Yahweh est le Dieu des armées ; Yahweh est son mémorial.
\VS{7}Et toi donc, reviens à ton Dieu, garde la miséricorde et la justice, et espère toujours en ton Dieu.
\VS{8}Ephraïm est un marchand\FTNT{Ephraïm est appelé « marchand », littéralement « Canaan ». Notez que l'ange de Laodicée s'exprime comme Ephraïm : « Je suis riche, je me suis enrichi… » (Ap. 3:14-19).}, qui a dans sa main des balances fausses, il aime à frauder.
\VS{9}Et Ephraïm dit : Quoi qu'il en soit, je suis devenu riche ; je me suis acquis des richesses ; c'est entièrement le produit de mon travail ; on ne trouvera en moi aucune iniquité, rien qui soit un péché.
\VS{10}Et moi, je suis Yahweh, ton Dieu, dès le pays d'Egypte ; je te ferai encore habiter dans des tentes, comme aux jours des fêtes solennelles.
\VS{11}Je parlerai par les prophètes, et je multiplierai les visions, et par les prophètes, je proposerai des paraboles.
\VS{12}Certainement Galaad n'est qu'iniquité, certainement ils ne seront que vanité. Ils sacrifient des bœufs dans Guilgal ; même leurs autels seront comme des monceaux de pierres sur les sillons des champs.
\VS{13}Jacob s'enfuit au pays de Syrie, et Israël servit pour une femme, et pour une femme il garda les troupeaux.
\VS{14}Par un prophète, Yahweh fit monter Israël hors d'Egypte, et par un prophète, Israël fut gardé.
\VS{15}Mais Ephraïm a provoqué Yahweh à une amère colère ; son Seigneur laissera sur lui le sang qu'il a répandu, et lui rendra ses mépris.
\Chap{13}
\TextTitle{Ephraïm persiste dans sa méchanceté}
\VerseOne{}Quand Ephraïm parlait, c'était une terreur ; il s'éleva en Israël. Mais il se rendit coupable par Baal, et mourut.
\VS{2}Et maintenant ils continuent de pécher, et se sont fait avec leur argent des images de fonte, des idoles selon leurs pensées; toutes sont un travail d'artisans, desquelles ils disent : Que les hommes qui sacrifient embrassent\FTNT{C'est une expression d'hommage.} les veaux !
\VS{3}C'est pourquoi ils seront comme la nuée du matin, et comme la rosée qui bientôt disparaît ; comme la balle qui est emportée par le vent hors de l'aire, comme la fumée sortant de la cheminée.
\VS{4}Et moi, je suis Yahweh, ton Dieu, dès le pays d'Egypte. Et tu ne devrais reconnaître d'autre dieu que moi, et il n'y a pas d'autre Sauveur que moi\FTNT{Yahweh dit qu'il n'y a pas d'autre sauveur que lui (Es. 43:11). Et les Ecrits de la Nouvelle Alliance nous présentent clairement notre Sauveur : Jésus-Christ (Mt.1:21; Ac.13:23 ; 2 Ti. 1:10 ; Tit : 1:4).}.
\VS{5}Je t'ai connu dans le désert, dans une terre aride.
\VS{6}Ils se sont rassasiés dans leurs pâturages ; ils se sont rassasiés, et leur cœur s'est enflé ; alors ils m'ont oublié.
\VS{7}Je serai pour eux comme un lion ; je les épierai sur la route comme un léopard.
\VS{8}Je les attaquerai, comme une ourse à qui on a enlevé ses petits, et je déchirerai l'enveloppe de leur cœur ; et là, je les dévorerai comme un lion ; les bêtes des champs les mettront en pièces.
\TextTitle{Châtiment d'Ephraïm}
\VS{9}Ta ruine, ô Israël, c'est que tu as été contre moi, alors que moi seul pouvais te secourir !
\VS{10}Où donc est ton roi ? Qu'il te délivre dans toutes tes villes ! Où sont tes juges, au sujet desquels tu as dit : Donne-moi un roi et des princes ?
\VS{11}Je t'ai donné un roi\FTNT{Ce passage concerne Saül, premier roi d'Israël (1 S. 8, 9 et 10)} dans ma colère, et je l'ôterai dans ma fureur.
\VS{12}L'iniquité d'Ephraïm est enveloppée, et son péché est mis en réserve.
\VS{13}Les douleurs comme de celle qui enfante le surprendront ; c'est un enfant qui n'est pas sage, qui, au temps marqué, ne sort pas du sein maternel.
\VS{14}Je les rachèterai de la puissance du scheol, je les délivrerai de la mort\FTNT{L'auteur de l'épître aux Hébreux applique ce passage à la victoire que le Seigneur Jésus-Christ a remportée face à la mort lors de sa résurrection (Hé. 2 :14-18). Depuis la chute d'Adam, les hommes ont toujours eu peur de la mort. Cette peur est d'autant plus forte de nos jours, car la plupart des gens sont angoissés par son aspect imprévisible, inévitable et par son non-sens. Et bien que beaucoup ne croient pas à l'existence de la vie après la mort (au paradis ou à l'enfer), la mort associée à l'annihilation, au non-être, apparaît d'autant plus monstrueuse et insupportable. Or notre Seigneur Jésus-Christ a vaincu la mort et il promet la vie éternelle à ceux qui croient en lui (Jn. 3:16 ; Jn. 5:24-29 ; Ap. 1:18). En plaçant notre foi en lui, nous avons non seulement la victoire sur la mort, mais aussi sur l'angoisse qu'elle produit dans le cœur de tout homme.} ; ô mort, où est ta peste ? Scheol, où est ta destruction\FTNT{1 Co. 15:55-57} ? Mais le repentir se cache à mes yeux !
\VS{15}Ephraïm a beau être fertile au milieu de ses frères, le vent d'orient, le vent de Yahweh s'élèvera du désert, viendra, desséchera ses sources et tarira ses fontaines. On pillera le trésor de tous ses objets précieux.
\VS{16}Samarie sera châtiée, car elle s'est rebellée contre son Dieu. Ils tomberont par l'épée; leurs petits enfants seront écrasés, et l'on fendra le ventre de leurs femmes enceintes.
\Chap{14}
\TextTitle{Bénédiction future d'Israël}
\VerseOne{}Israël, reviens à Yahweh ton Dieu ; car tu es tombé par ton iniquité.
\VS{2}Apportez avec vous des paroles, et revenez à Yahweh. Dites-lui : Pardonne toutes nos iniquités, et reçois le bien, pour le mettre à sa place! Et nous t'offrirons pour sacrifices la louange de nos lèvres.
\VS{3}L'Assyrie ne nous sauvera pas, nous ne monterons pas sur des chevaux, et nous ne dirons plus à l'ouvrage de nos mains : Notre dieu ! Car c'est auprès de toi que l'orphelin trouve de la compassion.
\VS{4}Je guérirai leur rébellion, et les aimerai volontairement ; parce que ma colère s'est détournée d'eux.
\VS{5}Je serai comme la rosée pour Israël ; il fleurira comme le lis, et il poussera ses racines comme le Liban.
\VS{6}Ses branches s'étendront, et sa magnificence sera comme celle de l'olivier, avec un parfum comme celui du Liban.
\VS{7}Ils reviendront s'asseoir à son ombre, et ils redonneront la vie au froment, et ils fleuriront comme la vigne ; et l'odeur de chacun d'eux sera comme celle du vin du Liban.
\VS{8}Ephraïm dira : Qu'ai-je à faire encore avec les idoles ? Je l'exaucerai, je le regarderai, je serai pour lui comme un cyprès verdoyant. C'est de moi que tu recevras ton fruit.
\VS{9}Qui est celui qui est sage ? Qu'il entende ces choses ! Et qui est celui qui est prudent ? Qu'il les connaisse ! Car les voies de Yahweh sont droites ; aussi les justes y marcheront, mais les rebelles y tomberont.
\PPE{}
\end{multicols}

\clearpage\ShortTitle{Joël}\BookTitle{Joël}\BFont
\noindent\hrulefill
{\footnotesize
\textit{
\bigskip
{\centering{}
\\Auteur : Joël
\\(Heb. : Yow'el)
\\Signification : Yahweh est Dieu
\\Thème : Le jour de Yahweh
\\Date de rédaction : 9ème ou 8ème siècle av. J.-C.\\}
}
%\bigskip
\textit{
\\Joël, fils de Pethuel, exerça son ministère dans le royaume de Juda. Son message faisait suite à deux fléaux qui s’étaient abattus sur Juda, à savoir une invasion de sauterelles et la sécheresse. Il s’agissait d’un avertissement de Yahweh qui appelait le peuple à revenir à lui avec la promesse de le restaurer dans tout ce qu’il avait perdu. Joël annonça en outre l’effusion de l’Esprit sur toute chair dans un avenir lointain, prophétie ayant trouvé son accomplissement à la naissance de l’Eglise lors de la Pentecôte.\bigskip
}
}
\par\nobreak\noindent\hrulefill
\begin{multicols}{2}
\Chap{1}
\VerseOne{}La parole de Yahweh qui fut adressée à Joël, fils de Pethuel.
\VS{2}Anciens écoutez ceci ! Et vous, tous les habitants du pays, prêtez l'oreille ! Rien de pareil est-il arrivé de votre temps, ou même du temps de vos pères ?
\VS{3}Racontez-le à vos enfants, et que vos enfants le racontent à leurs enfants, et leurs enfants à la génération suivante !
\TextTitle{Désolation  après  l'invasion des sauterelles}
\VS{4}La sauterelle a dévoré les restes du gazam, le jélek a dévoré les restes de la sauterelle, et le hasil a dévoré les restes du jélek.
\VS{5}Ivrognes, réveillez-vous, et pleurez ; et vous tous buveurs de vin hurlez à cause du vin nouveau, parce qu’il est retranché de votre bouche.
\VS{6}Car une nation puissante et innombrable est montée contre mon pays. Elle a les dents d’un lion et les mâchoires d’un vieux lion.
\VS{7}Elle a réduit ma vigne en désert ; et a ôté l’écorce de mes figuiers ; elle les a entièrement dépouillés, et les a abattus, leurs branches en sont devenues blanches.
\VS{8}Lamente-toi, comme une jeune fille qui se revêt d'un sac pour pleurer le mari de sa jeunesse !
\VS{9}L'offrande et la libation sont retranchées de la maison de Yahweh, et les sacrificateurs qui font le service de Yahweh mènent deuil.
\VS{10}Les champs sont ravagés, la terre est dans le deuil ; parce que le blé est détruit, le moût est tari, l'huile est desséchée.
\VS{11}Les laboureurs sont confus, les vignerons gémissent, à cause du froment et de l'orge, car la moisson des champs est perdue.
\VS{12}La vigne est desséchée, le figuier languissant ; le grenadier, le palmier, le pommier, tous les arbres des champs ont séché, c'est pourquoi la joie a cessé parmi les fils de l’homme !
\VS{13}Sacrificateurs, ceignez-vous et pleurez ! Poussez des gémissements, vous qui faites le service de l’autel, hurlez, vous qui faites le service de mon Dieu ; entrez, passez la nuit vêtus de sacs car il est défendu à l'offrande et à la libation d’entrer dans la maison de votre Dieu.
\TextTitle{Désolation après la sécheresse et la famine}
\VS{14}Sanctifiez le jeûne, publiez l’assemblée solennelle, assemblez les anciens, et tous les habitants du pays dans la maison de Yahweh votre Dieu, et criez à Yahweh en disant :
\VS{15}Hélas ! Quel jour ! Car le jour de Yahweh\FTNT{Jour de Yahweh : Voir commentaire en Za. 14:1.} est proche : Il vient comme un ravage fait par le Tout-Puissant.
\VS{16}La nourriture n’est-elle pas retranchée sous nos yeux ? Et la joie et l'allégresse de la maison de notre Dieu ?
\VS{17}Les semances sont pourries sous leurs mottes, les magasins sont dévastés, les greniers sont renversés parce que le blé a manqué.
\VS{18}Ô combien ont gémi les bêtes, et dans quelle peine ont été les troupeaux de bœufs, parce qu’ils n’ont point de pâturage ! Aussi les troupeaux de brebis sont dévastés.
\VS{19}Ô Yahweh, je crierai à toi, car le feu a consumé les pâturages du désert, et la flamme a brûlé tous les arbres des champs.
\VS{20}Même toutes les bêtes des champs crient aussi vers toi ; car les torrents d’eau sont à sec, et le feu a consumé les pâturages du désert.
\Chap{2}
\TextTitle{Le jour de Yahweh, invasion future}
\VerseOne{}Sonnez du shofar en Sion, et sonnez avec un retentissement bruyant dans la montagne de ma sainteté ; que tous les habitants du pays tremblent ; car le jour de Yahweh vient ; car il est proche,
\VS{2}jour de ténèbres et d'obscurité, jour de nuées et de brouillards, il vient comme l'aurore s'étend sur les montagnes. Voici un peuple nombreux et puissant, tel qu’il n’y en a jamais eu, et qu’il n’y en aura jamais dans la suite des siècles.
\VS{3}Devant lui est un feu dévorant, et derrière lui la flamme brûle ; le pays était, avant sa venue, comme le jardin d’Eden, et après qu’il sera parti il sera comme un désert affreux ; et même il n’y aura rien qui lui échappe.
\VS{4}Leur aspect est comme l’aspect des chevaux, et ils courent comme des cavaliers.
\VS{5}C’est comme le bruit de chariots, quand ils sautent au sommet des montagnes, comme le bruit d’une flamme de feu, qui dévore le chaume, comme un peuple puissant rangé en bataille.
\VS{6}Les peuples tremblent en le voyant ; tous les visages en deviennent pâles et livides.
\VS{7}Ils courent comme des hommes vaillants, et montent sur les murailles comme des gens de guerre ; chacun va son chemin, sans se détourner de son chemin.
\VS{8}Ils ne se pressent point les uns les autres, chacun va son chemin ; ils se jettent au travers des épées sans être blessés.
\VS{9}Ils courent çà et là dans la ville, se précipitent sur les murailles, montent sur les maisons, entrent par les fenêtres comme le voleur.
\VS{10}La terre tremble devant eux, les cieux sont ébranlés, le soleil et la lune s’obscurcissent, et les étoiles retirent leur éclat.
\VS{11}Aussi Yahweh fait entendre sa voix devant son armée ; parce que son camp est très grand, car l'exécuteur de sa parole est puissant. Certainement le jour de Yahweh est grand et terrible. Qui peut le supporter ?
\TextTitle{Repentance et miséricorde}
\VS{12}Maintenant encore, dit Yahweh, revenez à moi de tout votre cœur, avec des jeûnes, avec des pleurs et des lamentations !
\VS{13}Déchirez vos cœurs et non vos vêtements, et revenez à Yahweh, votre Dieu ; car il est compatissant et miséricordieux, lent à la colère et riche en bonté, et il se repent d’avoir affligé.
\VS{14}Qui sait si Yahweh, votre Dieu, ne reviendra pas et ne se repentira pas, et s'il ne laissera point après lui la bénédiction, des offrandes et des libations ?
\VS{15}Sonnez du shofar en Sion ! Sanctifiez le jeûne, publiez l'assemblée solennelle !
\VS{16}Assemblez le peuple, sanctifiez la congrégation ! Réunissez les anciens, assemblez les enfants, même les nourrissons à la mamelle ! Que l’époux sorte de sa demeure, et l’épouse de sa chambre nuptiale !
\VS{17}Que les sacrificateurs qui font le service de Yahweh pleurent entre le portique et l'autel, et qu'ils disent : Yahweh ! Epargne ton peuple ! N’expose pas ton héritage à l'opprobre, que les nations n’en fassent pas un sujet de railleries ! Pourquoi dirait-on parmi les peuples : Où est leur Dieu ?
\TextTitle{Promesse de restauration}
\VS{18}Or Yahweh est jaloux pour son pays, et il est ému de compassion envers son peuple.
\VS{19}Yahweh répond et il dit à son peuple : Voici, je vous enverrai du blé, du moût, et de l'huile, et vous en serez rassasiés ; et je ne vous exposerai plus à l'opprobre parmi les nations.
\VS{20}J'éloignerai de vous l’armée venue du nord, je la chasserai vers une terre aride et déserte, son avant-garde dans la mer orientale, son arrière-garde dans la mer occidentale ; et sa puanteur montera, et son infection s’élèvera, après avoir fait de grandes choses.
\VS{21}Terre, ne crains pas, sois dans l’allégresse et réjouis-toi, car Yahweh fait de grandes choses !
\VS{22}Ne craignez point, bêtes des champs, car les pâturages du désert ont poussé leur jet, et même les arbres portent leur fruit ; le figuier et la vigne ont poussé avec vigueur.
\VS{23}Et vous, enfants de Sion, soyez dans l’allégresse et réjouissez-vous en Yahweh, votre Dieu, car il vous donnera la pluie selon sa justice, il vous enverra la pluie de la première\FTNT{La pluie de la première saison : En Orient, la première pluie tombe au moment des semailles d’automne. Elle est nécessaire afin que la semence puisse germer. Sous l'influence des pluies fertilisantes, les tendres pousses sortent du sol.} et de l’arrière-saison\FTNT{La pluie de l’arrière-saison : Elle tombe vers la fin de la saison, mûrit le grain et le prépare pour la moisson. C’est la pluie du printemps. Voir Jé. 5:24 ; Os. 6:1-3 ; Za. 10:1.}, au premier mois.
\VS{24}Et les aires se rempliront de blé, et les cuves regorgeront de moût et d'huile.
\VS{25}Ainsi je vous rendrai les fruits des années qu'ont dévoré la sauterelle, le jélek, le hasil et le gazam, ma grande armée que j’avais envoyée contre vous.
\VS{26}Vous aurez donc abondamment de quoi manger et être rassasiés, et vous louerez le Nom de Yahweh votre Dieu, qui aura fait pour vous des choses merveilleuses ; et mon peuple ne sera plus jamais dans la confusion.
\VS{27}Et vous saurez que je suis au milieu d'Israël, que je suis Yahweh, votre Dieu, et qu'il n'y en a point d'autre, et mon peuple ne sera plus jamais dans la confusion.
\TextTitle{La promesse de l'Esprit}
\VS{28}Et il arrivera après cela, que je répandrai mon Esprit sur toute chair\FTNT{Cette promesse s’est réalisée dans Actes 2. Elle se réalise encore aujourd’hui dans la vie de chaque enfant de Dieu. Enfin, elle sera pleinement réalisée lors du retour du Messie en Israël (Za. 12:10-14) puisque cette prophétie annonce la repentance nationale d’Israël (Ro. 11:26-27).} ; et vos fils et vos filles prophétiseront ; vos vieillards songeront des songes, et vos jeunes gens verront des visions.
\VS{29}Et même en ces jours-là, je répandrai mon Esprit sur les serviteurs et sur les servantes.
\TextTitle{Prodiges précédant le jour de Yahweh\FTNTT{Es. 13:9-10 ; 24:21-23 ; Ez. 32:7-10 ; Mt. 24:29-30.}}
\VS{30}Je ferai des prodiges dans les cieux et sur la terre, du sang, et du feu, et des colonnes de fumée ;
\VS{31}Le soleil se changera en ténèbres, et la lune en sang, avant que le grand et terrible jour de Yahweh vienne.
\VS{32}Et il arrivera que quiconque invoquera le Nom de Yahweh\FTNT{Quiconque invoquera le Nom de Yahweh sera sauvé. Ce passage nous confirme que Jésus-Christ est vraiment Yahweh. En effet, Paul, apôtre des païens, attribue le Nom de Yahweh et cette prophétie à Jésus-Christ (Ro. 10:9-13). C’est bien le Nom de Jésus-Christ qu'il faut invoquer pour être sauvé (Ac. 4:12 ; Ac. 9:21 ; 1 Co. 1:2). Les éditeurs de la Traduction du Monde Nouveau (bible des témoins de Jéhovah) se sont permis de « restituer » le Nom divin YHWH qui apparaît près de 6000 fois dans le Tanakh, en 237 endroits dans les écrits de la nouvelle alliance, alors qu’aucun ancien manuscrit de la nouvelle alliance (testament de Jésus) ne le contient. Ils affirment, sur la base d’éléments de preuves indirectes, que les scribes du IIème siècle remplaçaient le Nom divin dans la Nouvelle Alliance par « Seigneur » ou « Dieu ». Pour restituer ce Nom (YHWH), ils se basent sur les citations du Tanakh où celui-ci figure et sur des versions hébraïques de la nouvelle alliance dont la plus ancienne date du XIVème siècle pour la plupart des copies de textes plus anciens. On constate cependant qu’ils n’ont pas restitué le Nom divin en 1 Pierre 2:3 qui est pourtant une citation du Psaumes 34:8. Pourquoi ? 
Parce que l’application de ce texte à Jésus-Christ, la pierre rejetée, est évidente. Si ce texte du Tanakh mentionnant Yahweh est appliqué à Jésus que penser des autres ? Jésus-Christ est vraiment Yahweh qui s’est incarné pour nous sauver. D'ailleurs, le Nom de Jésus veut dire « YHWH est Sauveur » (Es. 7:14 ; Es. 9:5 ; Mt. 1 ; Lu. 1 ; 1 Ti. 3:16).} sera sauvé ; car le salut sera sur la montagne de Sion et dans Jérusalem, comme l’a dit Yahweh, et parmi les réchappés que Yahweh appellera.
\Chap{3}
\TextTitle{Rétablissement d'Israël\FTNTT{Es. 11:10-12 ; Jé. 23:5-8 ; Ez. 37:21-28 ; Ac. 15:15-17.}}
\VerseOne{}Car voici, en ces jours-là, et en ce temps-là, quand je ramènerai les captifs de Juda et de Jérusalem,
\TextTitle{Jugements des nations étrangères\FTNTT{Za. 12:2-3.}}
\VS{2}J'assemblerai toutes les nations\FTNT{Dieu rassemblera les nations dans la vallée de Josaphat (de l'hébreu « Yehowshaphat », « Yahweh a jugé ») pour leur jugement. Cette vallée est peut-être celle où le roi Josaphat remporta une grande victoire, avec beaucoup de facilité, sur les Moabites, les Ammonites et les Maonites (2 Ch. 20). Cette vallée s'étend à l'orient de Jérusalem, entre la ville et le Mont des Oliviers, et traverse le torrent de Cédron.}, et je les ferai descendre dans la vallée de Josaphat ; là, j'entrerai en jugement avec elles, à cause de mon peuple, et d'Israël, mon héritage, lequel ils ont dispersé parmi les nations, et parce qu’ils ont partagé entre eux mon pays ;
\VS{3}et qu'ils ont tiré mon peuple au sort ; ils ont donné l’enfant pour une prostituée, ils ont vendu la jeune fille pour du vin, et ils ont bu.
\VS{4}Et qu’ai-je aussi affaire de vous, Tyr et Sidon, et de vous, toutes les limites de la Palestine, me rendrez-vous ma récompense, ou voulez-vous m'irriter ? Je vous rendrai promptement et sans délai votre récompense sur votre tête.
\VS{5}Car vous avez pris mon argent et mon or ; et vous avez emporté dans vos temples ce que j’avais de plus précieux et de plus beau.
\VS{6}Vous avez vendu les enfants de Juda et de Jérusalem aux enfants des Grecs, afin de les éloigner de leur territoire.
\VS{7}Voici, je les ferai lever\FTNT{Le verbe lever vient de l'hébreu «'uwr » qui signifie « se réveiller », « éveiller », « être éveillé » , « inciter », « veiller »,  « se lever », « sortir de l’assoupissement », «prendre courage». Yahweh annonce le réveil des hébreux depuis les nations, d'où ils sont établis. Ce réveil est une prise de conscience qui aboutira au retour à la terre sainte.} du lieu où ils ont été transportés après que vous les avez vendus ; et je ferai retourner votre récompense sur votre tête.
\VS{8}Je vendrai donc vos fils et vos filles entre les mains des enfants de Juda, et ils les vendront à ceux de Séba, qui les transporteront vers une nation éloignée ; car Yahweh a parlé.
\VS{9}Publiez ceci parmi les nations ! Préparez la guerre ! Réveillez les hommes vaillants ! Qu’ils s’approchent, et qu’ils montent, tous les hommes de guerre !
\VS{10}Forgez des épées de vos hoyaux, et des lances de vos serpes ! Et que le faible dise : Je suis fort !
\VS{11}Hâtez-vous et venez, vous toutes les nations d'alentour, et rassemblez-vous ! Là, ô Yahweh, fais descendre tes hommes vaillants !
\VS{12}Que les nations se réveillent, et qu'elles montent à la vallée de Josaphat ! Car là je siégerai pour juger toutes les nations d'alentour.
\VS{13}Saisissez la faucille, car la moisson est mûre ! Venez, et descendez, car le pressoir est plein, les cuves regorgent ! Car leur méchanceté est grande,
\VS{14}Des multitudes, des multitudes, dans la vallée du jugement ; car le jour de Yahweh est proche, dans la vallée du jugement.
\VS{15}Le soleil et la lune s’obscurcissent, et les étoiles retirent leur éclat.
\VS{16}De Sion Yahweh rugit, de Jérusalem il fait entendre sa voix ; les cieux et la terre sont ébranlés. Mais Yahweh est le refuge pour son peuple, et la forteresse\FTNT{Jésus-Christ est notre rocher (commentaire Es. 8:14 ; Ps. 78:35 ; 1 Co. 10:4).} pour les enfants d’Israël.
\VS{17}Et vous saurez que je suis Yahweh, votre Dieu, qui habite à Sion, ma sainte montagne. Jérusalem sera sainte, et les étrangers n'y passeront plus.
\TextTitle{Restauration finale et pleine bénédiction du royaume}
\VS{18}Et il arrivera en ce jour-là, le moût ruissellera des montagnes, le lait coulera des collines, il y aura de l’eau dans tous les torrents de Juda ; et une source\FTNT{Jésus est celui qui fait jaillir en nous une source d’eau qui étanche notre soif à jamais et nous donne la vie éternelle (Jé. 2:13 ; Jé. 17:13 ; Ez. 47:1-12 ; Za. 14:8 ; Jn. 4:14 ; Ap. 22:1).} sortira de la maison de Yahweh, et arrosera la vallée de Sittim.
\VS{19}L'Egypte sera dévastée, Edom sera réduit en désert de désolation, à cause de la violence faite aux enfants de Juda, dont ils ont répandu le sang innocent dans leur pays.
\VS{20}Mais là, Judas sera éternellement habitée, et Jérusalem, d’âge en âge.
\VS{21}Et je nettoierai leur sang que je n’avais point nettoyé ; car Yahweh habite en Sion.

\PPE{}
\end{multicols}

\clearpage%& -output-directory="./pdf"
% type document & taille police
\documentclass[11pt]{book}
% package format document
\usepackage[paperwidth=6.5in, paperheight=9.05in, top=0in, bottom=0in, left=0in, right=0in]{geometry}
% formatage marges, etc.
\setlength{\voffset}{-0.7in} % offset haut
%\setlength{\hoffset}{-0.3in} % offset gauche
\setlength{\topmargin}{0in} % marge en tête
\setlength{\headsep}{0.2in} % marge header/body
\setlength{\oddsidemargin}{-0.5in} % marge texte gauche
\setlength{\evensidemargin}{-0.5in} % marge texte droite
\setlength{\textheight}{8in} % hauteur du texte
\setlength{\textwidth}{5.5in} % largeur du texte
\setlength{\columnseprule}{0.4pt} % épaisseur séparateur colonne
\setlength{\parskip}{0pt} % espace entre paragraphes
% package pour afficher les cadres
%\usepackage{showframe}
% package langue
\usepackage[francais]{babel}
% package polices système
\usepackage{fontspec}
% définition police
\setmainfont[Ligatures=TeX,Scale=0.95]{Liberation Serif}
\setsansfont{Liberation Sans}
\setmonofont{Liberation Mono}
% package titlesec
\usepackage{titlesec}
% package multicolonne
\usepackage{multicol}
% package liens cliquables
\usepackage[xetex]{hyperref}
% package inclusion copyright (dépandant de hyperref)
\usepackage{hyperxmp}
% copyright
\hypersetup{
pdfauthor = {ANJC Productions},
pdftitle = {Bible de Jésus-Christ},
pdfkeywords = {BJC, Bible, Jesus},
pdfcopyright = {ANJC Productions. Distribution et Diffusion Libres - Pas d'Utilisation Commerciale - Pas de Dénaturation de l'Œuvre - International},
pdflicenseurl = {http://www.bibledejesuschrist.org/}
}
% ???
\setcounter{collectmore}{-1}
% style
\pagestyle{myheadings}
% ???
\sloppy\hyphenpenalty=2000
% titres de livres
\newcommand{\ShortTitle}[1]{\def\webbook{#1}\par\goodbreak\bigskip\setcounter{footnote}{0}}
\newcommand{\BookTitle}[1]{\par\goodbreak\bigskip{\parindent=0mm\begin{center}{\small\bfseries{\LARGE #1\nopagebreak}}\end{center}}\addcontentsline{toc}{subsection}{#1}\nopagebreak\par\nobreak}
% chapitres
\newcommand{\Chap}[1]{\def\webchap{#1:}\def\webvs{1}\def\vchap{#1}\ssubsection{\centerline{\textbf{{CHAPITRE\ #1}}}}}
% versets
\newcommand{\VerseOne}{\def\webvs{1}{\up{\footnotesize 1}}\markboth{\webbook\ \webchap 1}{\webbook\ \webchap 1}}
\newcommand{\VS}[1]{\def\webvs{#1}{\up{\footnotesize #1}}\markboth{\webbook\ \webchap #1}{\webbook\ \webchap #1}}
\newcommand{\vref}[1]{\NoAutoSpaceBeforeFDP{#1}}
% commentaires
%\interfootnotelinepenalty=10000 % longueur max commentaires
\renewcommand{\thefootnote}{\alph{footnote}} % repères alphabetiques
\renewcommand{\footnoterule}{\hrule width \textwidth} % longueur ligne
\newcommand{\FTNT}[1]{\ifnum\value{footnote}>25\setcounter{footnote}{0}\fi\footnote{[\NoAutoSpaceBeforeFDP{\webchap\webvs}]\ #1}}
% commentaire sur les titres
\newcounter{webvst}
\newcommand{\FTNTT}[1]{
% intialisation de l'indice de note
\ifnum \value{footnote}>25 \setcounter{footnote}{0} \fi
% initialisation de la référence du numéro de verset
\setcounter{webvst}{\webvs}
% si le titre est sur le premier verset, incrémenter de 1
\ifnum \value{webvst}>1 \addtocounter{webvst}{1} \fi
% écriture note
\footnote{[\NoAutoSpaceBeforeFDP{\webchap\thewebvst}]\ #1}
}
% titres de paragraphes
\titlespacing*{\subsection}{0pt}{5pt plus 0pt minus 0pt}{5pt plus 0pt minus 0pt}
\titlespacing*{\subsubsection}{0pt}{5pt plus 0pt minus 0pt}{5pt plus 0pt minus 0pt}
\newcommand{\ssubsection}[1]{\subsection*{\centering\footnotesize\normalfont #1}\PP}
\newcommand{\ssubsubsection}[1]{\subsubsection*{\centering\footnotesize\normalfont #1}\PP}
\newcommand{\TextTitle}[1]{\ssubsubsection{[\textit{#1}]}}
\newcommand{\TextDial}[1]{{\scriptsize[\textit{#1}]}}
% dictionnaire
\newcommand{\DicoEntry}[1]{\smallskip\parindent=0mm{\textbf{#1}}\markboth{#1}{#1}}
% commandes diverses
\newcommand{\BFont}{\normalfont\small}
\newcommand{\PP}{\par\parindent=0mm}
\newcommand{\PPE}{\par\parindent=4mm}
% debut document
\begin{document}
% en-tête pages
\makeatletter
\def\@evenhead{{\NoAutoSpaceBeforeFDP{\small{\rightmark\hfil\thepage\hfil\leftmark}}}}
\def\@oddhead{{\NoAutoSpaceBeforeFDP{\small{\rightmark\hfil\thepage\hfil\leftmark}}}}
\makeatother
% inclusion des livres
\pagenumbering{arabic}
\clearpage\ShortTitle{Amos}\BookTitle{Amos}\BFont
\noindent\hrulefill
{\footnotesize
\textit{
\bigskip
{\centering{}
\\Auteur : Amos
\\(Heb. : Amowc)
\\Signification : Fardeau, porteur de fardeau
\\Thème : Jugement sur le péché
\\Date de rédaction : 8ème siècle av. J.-C.\\}
}
%\bigskip
\textit{
\\Originaire de Tekoa, Amos exerça son ministère dans le royaume du nord, au temps d’Ozias,  roi de Juda, et Jéroboam II, roi d’Israël. Il fut aussi le contemporain des prophètes Osée, Michée, Jonas et Esaïe.
%\bigskip
\\Alors que le peuple juif jouissait d’une certaine prospérité, l’immoralité et les sacrilèges prirent place dans le royaume. Amos avertit le peuple de son péché et du jugement qu'il encourait. Il lui rappela la bonté de Dieu et l’invita à revenir à Yahweh et à lui rester fidèle.\bigskip
}
}
\par\nobreak\noindent\hrulefill
\begin{multicols}{2}
\Chap{1}
\VerseOne{}Paroles d'Amos, berger de Tekoa, qui prophétisa sur Israël, du temps d’Ozias, roi de Juda, et de Jéroboam, fils de Joas, roi d'Israël, deux ans avant le tremblement de terre\FTNT{Za. 14:5}.
\VS{2}Il dit : Yahweh rugit de Sion, et fait entendre sa voix de Jérusalem. Les habitations des bergers sont en deuil, et le sommet du Carmel est desséché\FTNT{Jé. 25:30 ; Joë 3:16}.
\TextTitle{Yahweh annonce ses jugements sur les villes et les pays d'alentour}
\VS{3}Ainsi parle Yahweh : A cause de trois crimes de Damas, et même de quatre, je ne rappellerai point cela, mais je le ferai\FTNT{Il est question du jugement de Dieu.}, parce qu'ils ont foulé Galaad avec des herses de fer\FTNT{Es. 17:1}.
\VS{4}J'enverrai le feu dans la maison de Hazaël, et il dévorera le palais de Ben-Hadad.
\VS{5}Je briserai aussi les verrous de Damas, j'exterminerai de Bikath-Aven ses habitants, et de Beth-Eden celui qui tient le sceptre. Et le peuple de Syrie sera mené captif à Kir, dit Yahweh.
\VS{6}Ainsi parle Yahweh : A cause de trois crimes de Gaza, et même de quatre, je ne rappellerai point cela, mais je le ferai\FTNT{Il est question du jugement de Dieu.} parce qu'ils ont emmené des captifs en grand nombre pour les livrer à Edom\FTNT{Ez. 25:13-17}.
\VS{7}J'enverrai le feu dans les murs de Gaza, et il dévorera ses palais.
\VS{8}J'exterminerai d'Asdod les habitants, et d'Askalon celui qui tient le sceptre ; je tournerai ma main contre Ekron, et le reste des Philistins périra, dit le Seigneur, Yahweh.
\VS{9}Ainsi parle Yahweh : A cause de trois crimes de Tyr, et même de quatre, je ne rappellerai point cela, mais je le ferai\FTNT{Il est question du jugement de Dieu.}, parce qu'ils ont livré à Edom des captifs en grand nombre sans se souvenir de l'alliance fraternelle\FTNT{Ez. 26:2}.
\VS{10}J'enverrai le feu dans les murs de Tyr, et il dévorera ses palais.
\VS{11}Ainsi parle Yahweh : A cause de trois crimes d'Edom, et même de quatre, je ne rappellerai point cela,, mais je le ferai\FTNT{Il est question du jugement de Dieu.}, parce qu'il a poursuivi son frère avec l'épée, refoulant toute compassion, parce que sa colère déchire continuellement et qu'il garde sa fureur éternellement.
\VS{12}J'enverrai le feu dans Théman, et il dévorera les palais de Botsra\FTNT{Jé. 49:7 ; Abd. 1:9}.
\VS{13}Ainsi parle Yahweh : A cause de trois crimes des enfants d'Ammon, et même de quatre, je ne rappellerai point cela, mais je le ferai\FTNT{Il est question du jugement de Dieu.}, parce qu’ils ont fendu le ventre des femmes enceintes de Galaad pour étendre leurs frontières\FTNT{Ez. 21:33 ; So. 2:8}.
\VS{14}J'allumerai le feu dans les murs de Rabba, et il dévorera les palais, au bruit des cris de guerre au jour du combat, et au milieu de l’ouragan au jour de la tempête.
\VS{15}Et leur roi ira en captivité, lui et ses chefs, dit Yahweh.
\Chap{2}
\TextTitle{Suite des jugements prononcés sur les villes et les pays d'alentour}
\VerseOne{}Ainsi parle Yahweh : A cause de trois crimes de Moab, et même de quatre, je ne rappellerai point cela, mais je le ferai, parce qu'il a brûlé les os du roi d'Edom jusqu’à les calciner.
\VS{2}J'enverrai le feu dans Moab, et il dévorera les palais de Kerijoth ; et Moab périra dans le tumulte, au milieu des cris de guerre et du bruit du shofar\FTNT{Ez. 25:8-9}.
\VS{3}J'exterminerai les juges de son pays, et je tuerai tous ses chefs, dit Yahweh.
\TextTitle{Juda et Israël jugés à cause de leurs iniquités}
\VS{4}Ainsi parle Yahweh : A cause de trois crimes de Juda, et même de quatre, je ne rappellerai point cela, mais je le ferai, parce qu'ils ont rejeté la loi de Yahweh et n'ont point gardé ses ordonnances ; parce qu’ils ont été égarés par les mensonges après lesquels leurs pères ont marché.
\VS{5}J'enverrai le feu dans Juda, et il dévorera les palais de Jérusalem.
\VS{6}Ainsi parle Yahweh : A cause de trois crimes d'Israël, et même de quatre, je ne rappellerai point cela, mais je le ferai, parce qu'ils ont vendu le juste pour de l'argent, et le pauvre pour une paire de souliers.
\VS{7}Ils aspirent à voir la poussière de la terre sur la tête des misérables, et ils pervertissent la voie des pauvres. Le fils et le père vont vers la même jeune fille, pour profaner mon Saint Nom.
\VS{8}Ils se couchent près de chaque autel, sur les vêtements qu'ils ont pris en gage, et boivent dans la maison de leurs dieux le vin de ceux qu’ils châtient.
\VS{9}Pourtant j'ai détruit devant eux les Amoréens qui étaient hauts comme les cèdres et forts comme les chênes ; j'ai détruit son fruit en haut, et ses racines en bas\FTNT{No. 21:24 ; Jos. 24:8}.
\VS{10}Je vous ai fait monter du pays d'Egypte et je vous ai conduits dans le désert quarante ans pour que vous possédiez le pays des Amoréens.
\VS{11}J'ai suscité quelques-uns d'entre vos fils pour être prophètes, et quelques-uns d'entre vos jeunes gens pour être nazaréens\FTNT{Le mot nazaréen vient de l'hébreu nâzîr, de la racine nâzar qui signifie séparer. Il y avait deux types de nazaréens. Premièrement ceux qui étaient appelés par Dieu. Par exemple : Samson Jg. 13:1-7 ; Samuel 1 S. 1:11 ; Jean-Baptiste Lu. 1:15. Deuxièmement les personnes qui voulaient se consacrer à Dieu. No. 6:13.}. N'en est-il pas ainsi, enfants d'Israël ? dit Yahweh.
\VS{12}Mais vous avez fait boire du vin aux nazaréens, et vous avez donné cet ordre aux prophètes disant : Ne prophétisez pas\FTNT{Es. 30:10 ; Jé. 11:21 ; Mi. 2:6} !
\VS{13}Voici, je m’en vais fouler le lieu où vous habitez, comme un chariot plein de gerbes foule tout par où il passe.
\VS{14}Tellement que l’homme agile ne pourra pas fuir, et le fort ne pourra pas se servir de sa vigueur, et l’homme vaillant ne sauvera pas sa vie\FTNT{Jé. 46:6}.
\VS{15}Celui qui manie l'arc, ne pourra pas tenir ferme, et celui qui a les pieds légers n'échappera pas, et le cavalier ne sauvera pas sa vie.
\VS{16}Le plus courageux d’entre les hommes vaillants s'enfuira tout nu en ce jour-là, dit Yahweh.
\Chap{3}
\TextTitle{La maison de Jacob coupable devant Yahweh}
\VerseOne{}Enfants d’Israël écoutez la parole que Yahweh prononce contre vous, contre toutes les familles que j'ai fait monter du pays d'Egypte.
\VS{2}Je vous ai connu vous seuls d'entre toutes les familles de la terre ; c'est pourquoi je vous châtierai pour toutes vos iniquités\FTNT{Ex. 19:5-6 ; Ps. 147:19-20}.
\VS{3}Deux hommes marchent-ils ensemble s’ils ne sont pas accordés ?
\VS{4}Le lion rugit-il dans la forêt sans qu’il n'ait de proie ? Le lionceau jette-t-il son cri de sa tanière sans qu’il n'ait rien attrapé ?
\VS{5}L'oiseau tombe-t-il dans le filet posé à terre sans que ce ne soit un piège ? Le filet est-il ramassé par terre sans qu’il n’y ait rien de capturé ?
\VS{6}Le shofar sonne-t-il dans une ville sans que le peuple en étant tout effrayé s’assemble ? Arrive-t-il un malheur dans une ville sans que Yahweh ne l’ait causé\FTNT{Es. 45:7 ; La. 3:37-38} ?
\VS{7}Car le Seigneur, Yahweh, ne fait aucune chose qu'il n'ait révélé son secret aux prophètes ses serviteurs.
\VS{8}Le lion rugit, qui ne serait pas effrayé ? Le Seigneur, Yahweh, parle, qui ne prophétiserait\FTNT{Lorsque Yahweh parle, des prophètes sont suscités : Jé. 20:9 ; Mi. 3:8 ; Ac. 4:20.} ?
\VS{9}Faites entendre votre voix dans les palais d'Asdod, et dans les palais du pays d'Egypte, et dites : Assemblez-vous sur les montagnes de Samarie, et voyez l’important tumulte interne et quelles oppressions dans son sein !
\VS{10}Ils ne savent pas faire ce qui est droit, dit Yahweh, ils amassent la violence et la rapine dans leurs palais.
\VS{11}C'est pourquoi ainsi parle le Seigneur, Yahweh : L'ennemi viendra, il cernera le pays, il t'ôtera ta force et tes palais seront pillés.
\VS{12}Ainsi parle Yahweh : Comme un berger arrache de la gueule d'un lion deux jambes ou un bout d'oreille, ainsi les enfants d'Israël qui habitent dans Samarie seront arrachés de l’angle d’un lit et de l’asile de Damas.
\VS{13}Ecoutez et soyez mes témoins contre la maison de Jacob, dit le Seigneur, Yahweh, le Dieu des armées :
\VS{14}Le jour où je punirai Israël pour ses péchés, j’exercerai mon châtiment sur les autels de Béthel ; les cornes de l'autel seront brisées, et tomberont à terre.
\VS{15}J’abattrai la maison d'hiver et la maison d'été ; les maisons d'ivoire seront détruites, et un grand nombre de maisons disparaîtront, dit Yahweh.
\Chap{4}
\TextTitle{Yahweh condamne les sacrifices du peuple}
\VerseOne{}Ecoutez cette parole, vaches de Basan, qui êtes sur la montagne de Samarie, vous qui opprimez les faibles, qui maltraitez les pauvres, qui dites à leurs maîtres : Apportez, et que nous buvions !
\VS{2}Le Seigneur, Yahweh, l’a juré par sa sainteté : Voici, les jours viennent sur vous, où l’on vous enlèvera avec des hameçons, et votre postérité avec des crochets de pêche\FTNT{Jé. 16:16 ; Ha. 1.14-16}.
\VS{3}Vous sortirez dehors par les brèches, chacune devant soi, et vous serez jetées dans la forteresse, dit Yahweh.
\VS{4}Entrez dans Béthel, et péchez ! Multipliez vos péchés dans Guilgal ! Amenez vos sacrifices dès le matin, et vos dîmes tous les trois ans\FTNT{Voir commentaires en Mal. 3:10 et No. 18:21.} !
\VS{5}Brûlez de l’encens avec du pain levé pour l’offrande de remerciement ; proclamez et publiez les offrandes volontaires ; car c’est là ce que vous aimez, enfants d'Israël, dit le Seigneur, Yahweh\FTNT{Lé. 2:1}.
\TextTitle{Endurcissement du peuple malgré les châtiments de Yahweh}
\VS{6}C'est pourquoi je vous ai envoyé la famine dans toutes vos villes, et la disette de pain dans toutes vos demeures ; mais malgré cela, vous n’êtes pas revenus vers moi, dit Yahweh.
\VS{7}Je vous ai aussi privés de pluie, alors qu’il restait encore trois mois jusqu'à la moisson ; j'ai fait pleuvoir sur une ville et je n'ai pas fait pleuvoir sur une autre ville ; une parcelle a été arrosée par la pluie, et l'autre parcelle, sur laquelle il n'a pas plu, est desséchée\FTNT{1 R. 8:35 ; 1 R. 17:1 ; Es. 5:6 ; Ag. 1:11}.
\VS{8}Et deux, même trois villes sont allées vers une autre ville pour boire de l'eau et n'ont pas été désaltérées, mais vous n’êtes pas revenus vers moi, dit Yahweh.
\VS{9}Je vous ai frappés par la rouille et par la nielle, et la sauterelle a brouté autant de jardins et de vignes, de figuiers et d'oliviers que vous aviez, mais vous n’êtes pas revenus vers moi, dit Yahweh\FTNT{De. 28:22-39 ; 1 R. 8:37 ; Ag. 2:17 ; 2 Ch. 6:28}.
\VS{10}J’ai envoyé parmi vous la peste comme celle en Egypte ; j'ai tué par l'épée vos jeunes hommes et vos chevaux en captivité ; j'ai fait remonter, jusque dans votre nez, la puanteur de vos camps ; mais vous n’êtes pas revenus vers moi, dit Yahweh\FTNT{Ez. 14:19}.
\VS{11}Je vous ai détruits comme Dieu détruisit Sodome et Gomorrhe, et vous avez été comme un tison arraché du feu, mais vous n’êtes pas revenus vers moi, dit Yahweh\FTNT{Ge. 19:24 ; Jé. 49:18 ; Za. 3:2},
\VS{12}C'est pourquoi je te traiterai de la même manière ô Israël ; et parce que je te traiterai ainsi, prépare-toi à la rencontre de ton Dieu, ô Israël !
\VS{13}Car voici celui qui a formé les montagnes et créé le vent, et qui déclare à l'homme quelle est sa pensée, qui fait l'aube et l'obscurité, et qui marche sur les hauteurs de la terre ; son nom est Yahweh, le Dieu des armées.
\Chap{5}
\TextTitle{Israël invité à revenir entièrement à Yahweh}
\VerseOne{}Ecoutez cette parole, cette complainte que je prononce sur vous, maison d'Israël !
\VS{2}Elle est tombée, elle ne se relèvera plus, la vierge d'Israël ; elle est couchée par terre, et personne ne la relève.
\VS{3}Car ainsi a parlé le Seigneur, Yahweh : La ville qui mettait en campagne mille hommes n'en aura de reste que cent ; et celle qui mettait en campagne cent hommes n'en aura de reste que dix dans la maison d’Israël.
\VS{4}Car ainsi a parlé Yahweh à la maison d'Israël : Cherchez-moi, et vous vivrez !
\VS{5}Ne cherchez pas Béthel, et n'allez pas à Guilgal, et ne passez point à Beer-Schéba. Car Guilgal sera transportée en captivité, et Béthel sera détruite\FTNT{Os. 4:15}.
\VS{6}Cherchez Yahweh, et vous vivrez, de peur qu'il ne saisisse comme un feu la maison de Joseph, et que ce feu ne la consume, sans qu'il y ait personne à Béthel pour l’éteindre.
\VS{7}Ils changent le jugement en absinthe, et ils foulent à terre la justice\FTNT{Es. 5:26-28 ; Ha. 1:1-3}.
\VS{8}Celui qui a créé les Pléiades et l'Orion, qui change les profonds ténèbres en aurore, et qui obscurcit le jour en nuit, qui appelle les eaux de la mer, et les répand sur la surface de la Terre, Yahweh est son nom\FTNT{Es. 58:8-10 ; Job 9:9 ; Job 38:31}.
\VS{9}Il fait éclater la ruine sur les puissants, et la ruine vient sur les forteresses.
\VS{10}Ils haïssent à la porte ceux qui les reprennent, et ils ont en abomination celui qui parle en intégrité.
\VS{11}C'est pourquoi, puisque vous opprimez le pauvre, et que vous prenez de lui du blé en présent, vous avez bâti des maisons en pierres de taille, mais vous n'y habiterez pas ; vous avez planté des vignes délicieuses, mais vous n'en boirez pas le vin.
\VS{12}Car j'ai connu vos crimes, ils sont en grand nombre, et vos péchés se sont multipliés : Vous opprimez le juste, vous recevez des présents, et vous violez à la porte le droit des pauvres.
\VS{13}C'est pourquoi, en ce temps-ci, le sage se tait, car les temps sont mauvais.
\VS{14}Recherchez le bien et non le mal, afin que vous viviez ; et qu’ainsi Yahweh, le Dieu des armées, soit avec vous, comme vous l'avez dit.
\VS{15}Haïssez le mal, et aimez le bien, faites régner la justice à la porte ; peut-être Yahweh, le Dieu des armées, aura pitié des restes de Joseph.
\TextTitle{Le jour de Yahweh}
\VS{16}C'est pourquoi ainsi parle Yahweh, le Dieu des armées, le Seigneur, parle ainsi : Dans toutes les places on se lamentera, dans toutes les rues on dira : Hélas ! Hélas ! On appellera au deuil le laboureur, et à la lamentation ceux qui en savent le métier.
\VS{17}Dans toutes les vignes on se lamentera, quand je passerai au milieu de toi, dit Yahweh.
\VS{18}Malheur à ceux qui désirent le jour de Yahweh\FTNT{L’expression «~le jour du Seigneur~» ou «~le jour de Yahweh~» est une période durant laquelle Jésus-Christ interviendra ouvertement dans les affaires des hommes. Elle est utilisé dix-neuf fois dans le Tanakh (Es. 2:12 ; Es. 13:6-9 ; Ez. 13:5 ; Ez. 30:3 ; Joë. 1:15 ; Joë. 2:1, 11, 31 ; Joë. 3:14 ; Am. 5:18, 20 ; Ab. 1:15 ; So. 1:7, 14 ; Za. 14:1 ; Mal. 4:5) et quatre fois dans le Testament de Jésus (Ac. 2:20 ; 2 Th. 2:2 ; 2 Pi. 3:10). On y fait également allusion dans d’autres passages (Ap. 6:17 ; Ap. 16:14).}. Qu’attendez-vous du jour de Yahweh ? Ce sont des ténèbres, et non pas une lumière.
\VS{19}Vous serez comme un homme qui fuit devant un lion et qui rencontre un ours, ou qui entre dans sa maison, appuie sa main sur le mur et un serpent le mord.
\TextTitle{Mépris du droit et de la justice}
\VS{20}Le jour de Yahweh n’est-il pas ténèbres et non lumière ? Obscurité et non clarté ?
\VS{21}Je hais, je méprise vos fêtes, je ne prends pas plaisir à vos assemblées solennelles.
\VS{22}Si vous me présentez des holocaustes, je n’agréerai pas vos offrandes, je ne regarderai pas les bêtes grasses de vos offrandes de paix.
\VS{23}Eloigne de moi le bruit de tes cantiques ; je n’écouterai pas la mélodie de tes luths.
\VS{24}Mais le jugement roule comme de l'eau, et la justice comme un torrent intarissable.
\VS{25}Est-ce à moi, maison d'Israël, que vous avez offert des sacrifices et des gâteaux dans le désert pendant quarante ans ? 
\VS{26}Au contraire, vous avez porté la tente de votre roi, et de vos idoles Kijun\FTNT{«~Kijun~» : Probablement une statue d'un dieu Assyro-Babylonien de la planète Saturne et utilisé pour symboliser l'apostasie d'Israél}, l'étoile de votre dieu que vous vous êtes fabriqué.
\VS{27}C'est pourquoi je vous transporterai au-delà de Damas, dit Yahweh, dont le nom est le Dieu des armées.
\Chap{6}
\TextTitle{Ceux qui prospèrent seront emmenés captifs}
\VerseOne{}Hélas vous qui êtes à votre aise en Sion, et qui vous confiez en la montagne de Samarie, lieux les plus renommés d’entre les principaux des nations, auprès desquels va la maison d'Israël.
\VS{2}Passez à Calné, et regardez ; allez de là à Hamath la grande, puis descendez à Gath chez les Philistins. Ces villes sont-elles plus prospères que vos deux royaumes, ou leur pays n'est-il pas plus étendu que votre pays ?
\VS{3}Vous qui éloignez le jour du malheur, et qui approchez le règne de la violence.
\VS{4}Vous qui vous couchez sur des lits d'ivoire, et qui sont étendus sur vos coussins ; qui mangez les agneaux du troupeau, et les veaux pris du lieu où on les engraisse ;
\VS{5}qui fredonnez au son du luth ; qui inventez des instruments de musique comme David.
\VS{6}Qui buvez le vin dans de grandes coupes, et qui parfumez des parfums les plus exquis, et qui n’êtes pas affligés pour la plaie de Joseph.
\VS{7}A cause de cela ils vont être emmenés à la tête des captifs, et les cris de joie de ces personnes voluptueuses prendront fin.
\VS{8}Le Seigneur, Yahweh, l’a juré par lui-même. Yahweh, Dieu des armées, dit : J'ai en détestation l'orgueil de Jacob, et j'ai en haine ses palais, c'est pourquoi je livrerai la ville, et tout ce qui est en elle.
\VS{9}Et si il reste dix hommes dans une maison, ils mourront.
\VS{10}Un proche parent prendra un mort et le brûlera pour emporter les os hors de la maison ; il dira à celui qui est au fond de la maison : Y a-t-il encore quelqu'un avec toi ? Et il répondra : Il n’y a plus personne. Puis il dira : Silence ! Ce n’est pas le moment de prononcer le nom de Yahweh.
\VS{11}Car voici, Yahweh ordonne : Et il frappera les grandes maisons par des débordements d'eau, et la petite maison en débris.
\VS{12}Les chevaux courent-ils sur les rochers, y laboure-t-on avec des bœufs, pour que vous ayez changé la droiture en poison, et le fruit de la justice en absinthe ?
\VS{13}Vous vous réjouissez de choses qui ne sont que néant, vous dites : N’est-ce pas par notre force que nous avons acquis de la puissance ?
\VS{14}Voici, je ferai lever contre vous, maison d’Israël, dit Yahweh, le Dieu des armées, une nation qui vous opprimera depuis l'entrée de Hamath jusqu'au torrent du désert.
\Chap{7}
\TextTitle{Avertissement\FTNTT{Am. 8:1 ; 9:10}}
\VerseOne{}Le Seigneur, Yahweh, me fit voir cette vision : Voici, il formait des sauterelles au temps où le regain commençait à croître ; et voici le regain poussait après les récoltes du roi.
\VS{2}Et quand elles eurent achevé de dévorer l'herbe de la terre, je dis : Seigneur Yahweh, pardonne, je te prie ! Comment Jacob subsistera-t-il ? Car il est faible.
\VS{3}Yahweh se repentit de cela. Cela n'arrivera pas, dit Yahweh.
\VS{4}Le Seigneur, Yahweh, me fit voir cette vision : Voici, le Seigneur, Yahweh, proclamait le jugement par le feu. Et le feu dévorait le grand abîme et dévorait les champs.
\VS{5}Et je dis : Seigneur Yahweh ! Arrête, je te prie ! Comment Jacob subsistera-t-il ? Car il est faible.
\VS{6}Yahweh se repentit de cela. Cela non plus n'arrivera pas, dit le Seigneur, Yahweh.
\VS{7}Il me fit voir cette vision : Voici, le Seigneur se tenait debout sur un mur fait au niveau, et il avait un niveau dans la main.
\VS{8}Et Yahweh me dit : Que vois-tu, Amos ? Et je répondis : Un niveau. Et le Seigneur me dit : Je mettrai le niveau au milieu de mon peuple d'Israël, je ne lui pardonnerai plus.
\VS{9}Et les hauts lieux d'Isaac seront ravagés, et les sanctuaires d'Israël seront détruits ; et je me lèverai contre la maison de Jéroboam avec l'épée.
\TextTitle{Amatsia accuse Amos devant Jéroboam}
\VS{10}Alors Amatsia, sacrificateur de Béthel, fit dire à Jéroboam roi d'Israël : Amos conspire contre toi au milieu de la maison d'Israël ; le pays ne saurait supporter toutes ses paroles.
\VS{11}Car voici ce que dit Amos : Jéroboam mourra par l'épée, et Israël sera emmené captif hors de son pays.
\VS{12}Et Amatsia dit à Amos : Voyant\FTNT{Voyant ou prophète.}, va-t’en, fuis dans le pays de Juda, et manges-y ton pain, et là tu prophétiseras.
\VS{13}Mais ne continue pas à prophétiser à Béthel\FTNT{Bethel, qui signifie «~maison de Dieu~», était devenue le sanctuaire du roi Jéroboam. De même aujourd’hui des églises du Seigneur sont devenues la propriété des hommes et les brebis de Dieu sont devenues la propriété des pasteurs.}, car c'est le sanctuaire du roi, et c'est une maison royale.
\TextTitle{Amos répond}
\VS{14}Amos répondit à Amatsia : Je n'étais ni prophète ni fils de prophète ; j'étais un berger, et je cueillais des figues sauvages.
\VS{15}Or Yahweh m’a pris derrière le troupeau, et Yahweh m’a dit : Va, prophétise à mon peuple d'Israël.
\VS{16}Ecoute maintenant la parole de Yahweh, tu dis : Ne prophétise pas contre Israël, et ne parle pas contre la maison d'Isaac.
\VS{17}C'est pourquoi ainsi parle Yahweh : Ta femme se prostituera dans la ville, tes fils et tes filles tomberont par l'épée, ton champ sera partagé au cordeau, et toi, tu mourras sur une terre souillée, et Israël sera emmené captif hors de son pays.
\Chap{8}
\TextTitle{Vision du panier de fruit, la fin pour le peuple d'Israël}
\VerseOne{}Le Seigneur, Yahweh, me fit voir cette vision : Voici, je vis un panier de fruits d'été.
\VS{2}Il dit : Que vois-tu, Amos ? Et je répondis : Un panier de fruits. Et Yahweh me dit : La fin est venue pour mon peuple d'Israël, je ne continuerai plus à lui pardonner.
\VS{3}En ce jour-là, les chants du palais seront des gémissements, dit le Seigneur, Yahweh ; en tout lieu, il y aura beaucoup de cadavres que l'on jettera en silence.
\VS{4}Ecoutez ceci vous qui dévorez les pauvres et qui faites périr les pauvres misérables du pays,
\VS{5}et qui dites : Quand la nouvelle lune sera-t-elle passée pour que nous vendions du blé ? Quand finira le sabbat pour que nous ouvrions les greniers ? Nous diminuerons l’épha, nous augmenterons le sicle, nous falsifierons les balances pour tromper,
\VS{6}nous achèterons les faibles pour de l’argent, et le pauvre pour une paire de souliers, et nous vendrons la criblure du froment.
\VS{7}Yahweh l’a juré par la gloire de Jacob : Jamais je n’oublierai toutes leurs actions !
\VS{8}La terre ne sera-t-elle point émue d'une telle chose, et tous ses habitants ne se lamenteront-ils point ? Le pays tout entier montera comme le fleuve. Il se soulèvera et s’affaissera comme le fleuve d'Egypte.
\VS{9}Il arrivera en ce jour-là, dit le Seigneur, Yahweh, que je ferai coucher le soleil à midi, et que j’obscurcirai la terre en plein jour.
\VS{10}Je changerai vos fêtes en deuil, et tous vos chants en lamentations ; je couvrirai de sacs tous les reins, et je rendrai chauves toutes les têtes ; je mettrai le pays dans le deuil comme pour un fils unique, et sa fin sera un jour d’amertume.
\VS{11}Voici, les jours viennent, dit le Seigneur, Yahweh, où j'enverrai la famine\FTNT{Nous sommes dans une époque où la Parole de Dieu, l’Evangile véritable a presque disparu au profit de l’évangile de prospérité. L’esprit de commerce a pris place au sein de beaucoup d’églises. C’est le temps de l’église de Laodicée, une église qui fait l’apologie de la richesse matérielle.} dans le pays ; non une famine de pain, ni une soif d'eau, mais d’entendre les paroles de Yahweh.
\VS{12}Ils erreront d’une mer jusqu'à l'autre, du nord à l'orient, ils iront çà et là pour chercher la parole de Yahweh, et ils ne la trouveront pas.
\VS{13}En ce jour-là, les belles vierges et les jeunes hommes mourront de soif.
\VS{14}Ceux qui jurent par le péché de Samarie disent : Vive ton Dieu, ô Dan ! Vive la voie de Beer-Schéba ! Mais ils tomberont et ne se relèveront plus.
\Chap{9}
\TextTitle{Prophétie annonçant la destruction\FTNTT{De. 28:63-68.}}
\VerseOne{}Je vis le Seigneur qui se tenait debout sur l'autel. Et il dit : Frappe le chapiteau et que les seuils s’ébranlent ; et brise-les sur leurs têtes à tous ! Je tuerai par l'épée ce qui restera d'eux. Il ne s’enfuira pas un fugitif, il ne s’échappera pas un fuyard.
\VS{2}S’ils pénètrent dans le séjour des morts, ma main les enlèvera de là ; s’ils montent aux cieux, je les en ferai descendre.
\VS{3}S’ils se cachent au sommet du Carmel, je les y rechercherai et je les enlèverai de là ; s’ils se dérobent à mes yeux dans le fond de la mer, là j’ordonnerai au serpent de les mordre.
\VS{4}Lorsqu'ils s'en iront en captivité devant leurs ennemis, là j’ordonnerai à l’épée de les tuer ; je fixerai mon regard sur eux pour leur faire du mal et non du bien.
\VS{5}Le Seigneur, Yahweh des armées, touche la terre, et elle tremble, et tous ses habitants sont dans le deuil ; elle monte tout entière comme le fleuve, et elle s’affaisse comme le fleuve d'Egypte.
\VS{6}Il a bâti sa demeure dans les cieux, et fondé sa voûte sur la terre ; il appelle les eaux de la mer, et les répand sur la surface de la Terre. Son nom est Yahweh.
\VS{7}N'êtes-vous pas pour moi comme les enfants des Ethiopiens, enfants d'Israël ? dit Yahweh. N'ai-je pas fait monter Israël du pays d'Egypte, les Philistins de Caphtor et les Syriens de Kir ?
\VS{8}Voici, les yeux du Seigneur, Yahweh, sont sur ce royaume pécheur. Je le détruirai de dessus la surface de la terre. Cependant, je ne détruirai pas entièrement la maison de Jacob, dit Yahweh.
\VS{9}Car voici, je donnerai mes ordres, et je secouerai la maison d'Israël parmi toutes les nations, comme on secoue le blé dans le crible, sans qu'il en tombe un grain à terre.
\VS{10}Tous les pécheurs de mon peuple mourront par l'épée, ceux qui disent : Le mal n'approchera pas, il ne nous atteindra pas.
\TextTitle{Yahweh relève la maison de David}
\VS{11}En ce temps-là, je relèverai le tabernacle de David qui est tombé, j’en réparerai les brèches, j’en redresserai les ruines, et je le rebâtirai comme il était autrefois,
\VS{12}afin qu'ils possèdent le reste d’Edom et toutes les nations sur lesquelles mon nom a été invoqué, dit Yahweh, qui accomplira cela.
\TextTitle{Restauration d'Isarël}
\VS{13}Voici, les jours viennent, dit Yahweh, où le laboureur suivra de près le moissonneur, et celui qui foule les raisins atteindra celui qui répand la semence ; et le moût ruissellera des montagnes et découlera de toutes les collines.
\VS{14}Je ramènerai les captifs de mon peuple d'Israël ; ils rebâtiront les villes dévastées, et y habiteront, ils planteront des vignes, et en boiront le vin ; ils feront des jardins et en mangeront les fruits.
\VS{15}Je les planterai sur leur terre, et ils ne seront plus arrachés du pays que je leur ai donné\FTNT{Cette prophétie annonce la restauration de la maison de David. Personne ne chassera Israël de sa terre, aucune nation n’a le pouvoir de le déloger, car c’est le Seigneur qui l’a établi.}, dit Yahweh, ton Dieu.
\PPE{}
\end{multicols}

\end{document}

\clearpage\ShortTitle{Abdias}\BookTitle{Abdias}\BFont
\begin{multicols}{2}
\Chap{1}
\VerseOne{}La vision d'Abdias. Ainsi a dit le Seigneur l'Eternel touchant Edom : Nous avons ouï une publication de par l'Eternel, et un Envoyé a été dépêché parmi les nations, [et elles ont dit] : Courage levons-nous contre lui pour le combattre.
\VS{2}Voici, je te rendrai petit entre les nations, tu seras fort méprisé.
\VS{3}L'orgueil de ton cœur t'a séduit, toi qui habites dans les fentes des rochers, qui sont ta haute demeure, et qui dis en ton cœur : Qui est-ce qui me renversera par terre ?
\VS{4}Quand tu aurais élevé ton nid comme l'aigle, et quand même tu l'aurais mis entre les étoiles, je te jetterai de là par terre, dit l'Eternel.
\VS{5}Sont-ce des larrons qui sont entrés chez toi, ou des voleurs de nuit ? Comment [donc] as-tu été rasé ? N'eussent-ils pas dérobé jusqu'à ce qu'ils en eussent eu assez ? Si des vendangeurs fussent entrés chez toi, n'eussent-ils pas laissé quelque grappillage ?
\VS{6}Comment a été fouillé Esaü ? Comment ont été examinés ses lieux secrets ?
\VS{7}Tous tes alliés t'ont conduit jusqu'à la frontière ; ceux qui étaient en paix avec toi t'ont trompé, et ont eu le dessus sur toi ; ceux [qui mangeaient] ton pain t'ont donné le coup par-dessous, sans qu'on l'aperçût.
\VS{8}Ne sera-ce pas en ce temps-là, dit l'Eternel, que je ferai périr les sages au milieu d'Edom, et la prudence dans la montagne d'Esaü ?
\VS{9}Tes hommes forts seront aussi étonnés, ô Téman ! afin que les hommes soient retranchés de la montagne d'Esaü, à force de les [y] tuer.
\VS{10}La honte te couvrira, et tu seras retranché pour jamais, à cause de la violence [faite] à ton frère Jacob.
\VS{11}Lorsque tu te tenais vis-à-vis quand les étrangers menaient son armée en captivité, et que les forains entraient dans ses portes, et qu'ils jetaient le sort sur Jérusalem, tu étais aussi comme l'un d'eux.
\VS{12}Mais tu ne devais pas prendre plaisir à voir la journée de ton frère quand il a été livré aux étrangers, et tu ne devais pas te réjouir sur les enfants de Juda au jour qu'ils ont été détruits, et tu ne [les] devais pas braver au jour de la détresse.
\VS{13}Et tu ne devais pas entrer dans la porte de mon peuple au jour de sa calamité, et tu ne devais pas prendre plaisir, toi, dis-je, à voir son mal au jour de sa calamité, et [tes mains] ne se devaient pas avancer sur son bien, au jour de sa calamité.
\VS{14}Et tu ne te devais pas tenir sur les passages, pour exterminer ses réchappés ; ni livrer ceux qui étaient restés, au jour de la détresse.
\VS{15}Car la journée de l'Eternel est proche sur toutes les nations ; comme tu as fait, il te sera ainsi fait ; ta récompense retournera sur ta tête.
\VS{16}Car comme vous avez bu sur la montagne de ma sainteté [la coupe de ma fureur], ainsi toutes les nations [la] boiront sans interruption ; oui, elles [la] boiront, et [l']avaleront, et elles seront comme si elles n'avaient point été.
\VS{17}Mais il y aura des réchappés sur la montagne de Sion, et elle sera sainte, et la maison de Jacob possédera ses possessions.
\VS{18}Et la maison de Jacob sera un feu, et la maison de Joseph une flamme, et la maison d'Esaü du chaume ; ils s'allumeront parmi eux, et les consumeront ; et il n'y aura rien de reste dans la maison d'Esaü ; car l'Eternel a parlé.
\VS{19}Ils posséderont le Midi, [savoir] la montagne d'Esaü ; et la campagne, [savoir] les Philistins, et ils posséderont le territoire d'Ephraïm, et le territoire de Samarie ; et Benjamin [possédera] Galaad.
\VS{20}Et ces bandes des enfants d'Israël qui auront été transportés, [posséderont] ce qui était des Cananéens, jusqu'à Sarepta ; et ceux de Jérusalem qui auront été transportés, [posséderont] ce qui [est] jusqu'à Sépharad, ils le posséderont avec les villes du Midi.
\VS{21}Car les libérateurs monteront en la montagne de Sion, pour juger la montagne d'Esaü ; et le Royaume sera à l'Eternel.
\PPE{}
\end{multicols}

\clearpage\ShortTitle{Jonas}\BookTitle{Jonas}\BFont
\begin{multicols}{2}
\TextTitle{[I. Désobéissance et fuite de Jonas
\\Introduction]}
\Chap{1}
\VerseOne{}La parole de Yahweh fut adressée à Jonas, fils d'Amitthaï, en ces mots :
\TextTitle{[Jonas fuit la face de Yahweh]}
\VS{2}Lève-toi, va à Ninive (1), la grande ville, et crie contre elle ! Car leur malice est montée jusqu'à moi.
\VS{3}Mais Jonas se leva pour s'enfuir à Tarsis, loin de la face de Yahweh. Il descendit à Japho, où il trouva un navire qui allait à Tarsis ; il paya le prix du transport et y entra pour aller à Tarsis, loin de la face de Yahweh.
\VS{4}Mais Yahweh fit lever un grand vent sur la mer, et il y eut une grande tempête sur la mer, de sorte que le navire semblait se briser.
\VS{5}Et les mariniers eurent peur, et ils crièrent chacun à son dieu, et jetèrent dans la mer les objets qui étaient dans le navire, pour l’alléger. Jonas descendit au fond du navire, se coucha et s’endormit profondément.
\VS{6}Le chef des marins s'approcha de lui, et lui dit : Qu’as-tu dormeur ? Lève-toi, invoque ton Dieu ! Peut-être ton Dieu pensera à nous et nous ne périrons pas.
\VS{7}Puis ils se dirent l'un à l'autre : Venez, tirons au sort pour savoir qui est la cause de ce malheur. Ils tirèrent au sort, et le sort tomba sur Jonas.
\VS{8}Alors ils lui dirent : Dis-nous quelle est la cause de ce malheur. Quel est ton métier, et d'où viens-tu ? Quel est ton pays, et de quel peuple es-tu ?
\VS{9}Il leur répondit : Je suis Hébreu, et je crains Yahweh, le Dieu des cieux, qui a fait la mer et la terre sèche.
\VS{10}Alors ces hommes furent saisis d'une grande crainte, et lui dirent : Pourquoi as-tu fait cela ? Car ces hommes savaient qu’il fuyait loin de la face de Yahweh, parce qu'il le leur avait déclaré.
\VS{11}Ils lui dirent : Que te ferons-nous pour que la mer se calme ? Car la mer était de plus en plus agitée.
\TextTitle{[II. Jonas et le poisson
\\Jonas englouti par le poisson]}
\VS{12}Il leur répondit : Prenez-moi, et jetez-moi dans la mer, et la mer se calmera ; car je sais que c’est moi la cause de cette grande tempête.
\VS{13}Ces hommes ramaient pour revenir sur la terre sèche, mais ils ne le purent, car la mer s'agitait toujours plus contre eux.
\VS{14}Alors ils invoquèrent Yahweh, et dirent : O Yahweh, ne nous fais pas périr à cause de la vie de cet homme, et ne mets pas sur nous le sang innocent ! Car toi, Yahweh, tu fais comme il te plait (2).
\VS{15}Alors ils prirent Jonas, et le jetèrent dans la mer. Et la fureur de la mer s'arrêta.
\VS{16}Ces hommes furent saisis d’une grande crainte envers Yahweh, et ils offrirent des sacrifices à Yahweh, et firent des vœux.
\VS{17}Yahweh ordonna à un grand poisson d’engloutir Jonas, et Jonas fut dans le ventre du poisson trois jours et trois nuits.
\Chap{2}
\VerseOne{}Jonas pria Yahweh, son Dieu, dans le ventre du poisson.
\TextTitle{[Prière de Jonas et l'exaucement de Yahweh]}
\VS{2}Il dit : Dans ma détresse j’ai invoqué Yahweh, et il m'a exaucé ; du sein du scheol j’ai crié, et tu as entendu ma voix (1).
\VS{3}Tu m'as jeté dans les profondeurs, au cœur de la mer, et le courant m'a environné ; tous tes flots et toutes tes vagues ont passé sur moi.
\VS{4}Je disais : Je suis chassé loin de tes yeux ! Cependant je verrai encore le temple de ta sainteté.
\VS{5}Les eaux m'ont environné jusqu'à l'âme. L'abîme m'a enveloppé, les roseaux ont lié ma tête.
\VS{6}Je suis descendu jusqu'aux bases des montagnes, la terre fermait sur moi ses barres pour toujours ; mais tu m’as fait remonter vivant de la fosse, Yahweh, mon Dieu !
\VS{7}Quand mon âme s’était affaiblie en moi, je me suis souvenu de Yahweh, et ma prière est parvenue jusqu’à toi, dans le temple de ta sainteté.
\VS{8}Ceux qui s’adonnent à des vanités mensongères abandonnent ta miséricorde.
\VS{9}Mais moi, je t’offrirai des sacrifices avec un cri de louange, j’accomplirai les vœux que j’ai faits : Car le salut vient de Yahweh (2).
\VS{10}Alors Yahweh parla au poisson, et le poisson vomit Jonas sur la terre sèche.
\TextTitle{[III. Le plus grand réveil de l'histoire
\\Ninive se repent et elle est épargnée]}
\Chap{3}
\VerseOne{}La parole de Yahweh fut adressée à Jonas une seconde fois, en disant :
\VS{2}Lève-toi, va à Ninive, la grande ville, et proclames-y à haute voix ce que je t'ordonne !
\VS{3}Jonas se leva, et alla à Ninive, suivant la parole de Yahweh. Or Ninive était une grande ville devant Dieu, de trois jours de marche.
\VS{4}Jonas commença dans la ville le chemin d'une journée de marche ; il criait et disait : Encore quarante jours, et Ninive sera renversée !
\VS{5}Les hommes de Ninive crurent à Dieu, ils publièrent un jeûne, et se vêtirent de sacs, depuis le plus grand d'entre eux jusqu'au plus petit.
\VS{6}Cette parole parvint au roi de Ninive ; il se leva de son trône, ôta de dessus lui son manteau, se couvrit d'un sac, et s'assit sur la cendre.
\VS{7}Puis il fit faire une proclamation, et publier dans Ninive par décret du roi et de ses grands : Que les hommes, les bêtes, les bœufs et les brebis, ne goûtent de rien, ne paissent point, et ne boivent point d'eau !
\VS{8}Que les hommes et les bêtes soient couverts de sacs, qu'ils crient à Dieu avec force, et que chacun revienne de sa mauvaise voie, et des actions violentes que ses mains ont commises !
\VS{9}Qui sait si Dieu ne reviendra pas et ne se repentira pas, et s'il ne se détournera pas de son ardente colère, en sorte que nous ne périssions point ?
\VS{10}Dieu vit ce qu’ils faisaient et comment ils revenaient de leur mauvaise voie. Alors Dieu se repentit du mal qu'il avait déclaré de leur faire, et il ne le fit point.
\TextTitle{[IV. Miséricorde infinie de Dieu
\\Mécontentement de Jonas]}
\Chap{4}
\VerseOne{}Mais cela déplut fortement à Jonas, et il fut furieux.
\VS{2}Il pria Yahweh, et dit : Oh ! Yahweh, n'est-ce pas là ce que je disais quand j'étais encore dans mon pays ? C'est pourquoi j'ai voulu m'enfuir à Tarsis. Car je savais que tu es un Dieu compatissant, miséricordieux, lent à la colère et riche en bonté, et qui te repens du mal (1).
\VS{3}Maintenant, Yahweh, prends-moi donc la vie, car la mort m'est meilleure que la vie.
\TextTitle{[Reproches de Yahweh à Jonas]}
\VS{4}Et Yahweh répondit : Fais-tu bien de te mettre en colère ?
\VS{5}Alors Jonas sortit de la ville, et s'assit à l'orient de la ville, là il se fit une cabane, et y resta à l'ombre, jusqu'à ce qu'il vît ce qui arriverait à la ville.
\VS{6}Yahweh Dieu ordonna à un ricin de croître au-dessus de Jonas, pour donner de l’ombre sur sa tête et pour le délivrer de son mal. Jonas éprouva une grande joie à cause de ce ricin.
\VS{7}Mais le lendemain, à l’aurore, Dieu ordonna à un ver d’attaquer le ricin, et le ricin sécha.
\VS{8}Au lever du soleil, Dieu ordonna à un vent chaud d’orient de souffler, et le soleil frappa la tête de Jonas, au point qu’il s’évanouit. Il demanda la mort, et dit : La mort m'est meilleure que la vie.
\VS{9}Dieu dit à Jonas : Fais-tu bien de te mettre en colère à cause du ricin ? Et il répondit : Je fais bien de m’irriter jusqu’à la mort.
\VS{10}Et Yahweh dit : Tu as pitié du ricin pour lequel tu n'as point travaillé et que tu n'as point fait croître, qui est né dans une nuit et qui a péri dans une nuit.
\VS{11}Et moi, je n’aurais pas pitié de Ninive, la grande ville, dans laquelle il y a plus de cent vingt mille personnes qui ne savent point distinguer leur main droite de leur main gauche, et des animaux en grand nombre !
\PPE{}
\end{multicols}

\clearpage\ShortTitle{Michée}\BookTitle{Michée}\BFont
\noindent\hrulefill
{\footnotesize
\textit{
\bigskip
{\centering{}
\\(Mikha)
\\Signifie : Qui est semblable à Dieu ?
\\Thème : Le jugement et le royaume
\\Auteur : Michée 
\\Date de rédaction : 8ème siècle av. J.-C.\\}
}
%\bigskip
\textit{
\\Originaire de Moréscheth, Michée exerça son ministère dans le royaume du sud au temps d’Ezéchias, roi de Juda et fut contemporain d’Osée, Amos et Esaïe. Alors que la corruption et l’idolâtrie régnaient en Samarie et à Jérusalem, Michée appela le peuple à se détourner de ses iniquités et les prévint du danger qui les menaçait. Il prophétisa également le rétablissement final de la nation juive et mit en exergue la miséricorde divine.\bigskip
}
}
\par\nobreak\noindent\hrulefill
\begin{multicols}{2}
\TextTitle{[Jugement de Yahweh sur Israël infidèle]}
\Chap{1}
\VerseOne{}La Parole de Yahweh qui fut adressée à Michée, de Moréscheth, au temps de Jotham, Achaz, et Ezéchias, rois de Juda, laquelle lui fut adressée dans une vision contre Samarie et Jérusalem.
\VS{2}Vous tous, peuples, écoutez ! Et toi, terre, et tout ce qui est en elle, soyez attentifs ! Et que le Seigneur, Yahweh, soit témoin contre vous, le Seigneur, sortant du palais de sa sainteté.
\VS{3}Car voici, Yahweh sortira de son lieu, il descendra, et marchera sur les hauts lieux de la terre\FTNT{Cette prophétie annonce à la fois la destruction du royaume du nord par Salmanasar V (régna de 727-722 av. J.-C.) en 722 av. J.-C. (2 R. 17:1-23), l’invasion de Sanchérib (régna de 705 à 681 av. J.-C.), (2 R. 18:13 à 2 R. 19:37) ainsi que celle de Nebucadnetsar (régna de 604 à 562 av. J.-C.), (2 R. 24 et 25).} ;
\VS{4}et les montagnes se fondront sous lui, et les vallées se fendront, elles seront comme de la cire devant le feu et comme des eaux qui coulent sur une pente.
\VS{5}Tout ceci arrivera à cause du crime de Jacob, et à cause des péchés de la maison d'Israël. Or quel est le crime de Jacob ? N'est-ce pas Samarie ? Et quels sont les hauts lieux de Juda ? N'est-ce pas Jérusalem ?
\TextTitle{[Chutes futures de Samarie et Jérusalem]}
\VS{6}C'est pourquoi je réduirai Samarie en un monceau de pierres dans les champs, un lieu où l'on plante des vignes ; et je ferai rouler ses pierres dans la vallée, et je découvrirai ses fondements.
\VS{7}Toutes ses images taillées seront brisées, tous ses salaires de prostitution seront brûlés au feu, et je mettrai tous ses faux dieux en désolation ; parce qu'elle les a entassés par le moyen du salaire de sa prostitution, ils serviront de salaire à une prostituée.
\VS{8}C'est pourquoi je me plaindrai, et je hurlerai ; je m'en irai dépouillé et nu ; je ferai une lamentation comme celle des dragons, et je mènerai le deuil comme celui des autruches.
\VS{9}Car sa plaie est incurable, elle est venue jusqu'en Juda, et est parvenue jusqu'à la porte de mon peuple, jusqu'à Jérusalem.
\VS{10}Ne l'annoncez point dans Gath, et ne pleurez nullement ! Vautre-toi dans la poussière à Beth-Leaphra.
\VS{11}Passez habitants de Schaphir, dans la nudité et la honte ! Les habitants de Tsaanan ne sont point sortis ; le deuil de Beth-Haëtsel vous prive de son abri.
\VS{12}L’habitante de Maroth est dans l'angoisse à cause de son bien ; parce que le mal est descendu de par Yahweh sur la porte de Jérusalem.
\VS{13}Attelle le cheval au char, habitante de Lakisch ! Toi qui es le commencement du péché de la fille de Sion ; car en toi ont été trouvés les crimes d'Israël.
\VS{14}C'est pourquoi donne des présents à cause de Moréschet-Gath ; les maisons d'Aczib mentiront aux rois d'Israël.
\VS{15}Je t’amènerai un autre héritier, habitante de Maréscha ; et la gloire d'Israël s'en ira jusqu'à Adullam.
\VS{16}Arrache tes cheveux et fais-toi tondre, à cause de tes fils qui font tes délices ; arrache tout le poil de ton corps, comme un aigle qui mue, car ils sont emmenés prisonniers loin de toi.
\TextTitle{[Causes du jugement de Dieu sur Israël]}
\Chap{2}
\VerseOne{}Malheur à ceux qui pensent à faire outrage, qui forgent le mal sur leurs lits, et qui l'exécutent dès le point du jour, parce qu'ils ont le pouvoir en main.
\VS{2}S'ils convoitent des possessions, ils les ont aussitôt ravies, des maisons, ils les ont aussitôt prises ; ainsi ils oppriment l'homme et sa maison, l'homme et son héritage.
\VS{3}C'est pourquoi ainsi parle Yahweh : Voici, je médite contre cette famille-ci un mal duquel vous ne pourrez point préserver votre cou, et vous ne marcherez point la tête levée, car ce temps est mauvais.
\VS{4}En ce temps-là on fera de vous un proverbe lugubre, et l'on gémira d'un gémissement lamentable, en disant : Nous sommes entièrement détruits ; la part de mon peuple, il la change de mains ! Comment nous enlève-t-il et partage-t-il notre terre à l'infidèle ?
\VS{5}C'est pourquoi il n'y aura personne qui jettera le cordeau pour ton lot, dans l'assemblée de Yahweh.
\VS{6}Ne prophétisez point, disent-ils, ne prophétisez point de telles choses ; l'opprobre ne s'éloignera point.
\VS{7}Or toi qui es appelée maison de Jacob, l'Esprit de Yahweh est-il amoindri ? Sont-ce là ses actes ? Mes paroles ne sont-elles pas bonnes pour celui qui marche droitement ?
\VS{8}Mais celui qui était hier mon peuple, s'élève à la manière d'un ennemi ; vous dépouillez le manteau avec le vêtement à ceux qui passent en assurance, de retour de la guerre.
\VS{9}Vous chassez les femmes de mon peuple hors des maisons de leurs délices ; vous ôtez pour toujours ma gloire de dessus leurs enfants.
\VS{10}Levez-vous et marchez, car ce pays n'est plus pour vous un lieu de repos ; à cause de la souillure, il vous détruira d’une violente destruction.
\VS{11}S'il y a quelque homme qui court après le vent et le mensonge, et qui mente en disant : Je te prophétiserai sur du vin et sur les boissons fortes, ce sera le prophète de ce peuple.
\TextTitle{[Yahweh, le Dieu qui rassemble son peuple]}
\VS{12}Mais je t'assemblerai tout entier, ô Jacob ! Et je ramasserai entièrement le reste d'Israël, et le mettrai tout ensemble comme les brebis d'une bergerie, comme un troupeau au milieu de son pâturage ; il y aura un grand bruit pour la foule des hommes.
\VS{13}Celui qui fera la brèche\FTNT{«~Celui qui fera la brèche~» est Jean-Baptiste, qui fut suscité à une époque où la gloire de Dieu ne se manifestait plus depuis quatre cents ans. En effet, Dieu n’avait plus suscité de ministère prophétique ou de messagers pour parler à son peuple. Le ciel était comme fermé. Il est donc venu ouvrir une brèche, c’est-à-dire préparer le chemin du Seigneur selon Es. 40:3-5, Mal. 3:1, Mal. 4:5-6.} montera  devant eux, on brisera, et on passera outre, et ils sortiront par la porte ; et leur Roi marchera devant eux\FTNT{Le Roi qui marchera devant eux est bien Jésus-Christ. Le message de Jean-Baptiste était clair : «~Repentez-vous car le Royaume des cieux arrive~»,  or ce royaume est celui du Messie (Mt. 3:1-3).}, Yahweh sera à leur tête.
\TextTitle{[Corruption et méchanceté des chefs]}
\Chap{3}
\VerseOne{}C'est pourquoi je dis : Ecoutez, chefs de Jacob, et vous conducteurs de la maison d'Israël ! N'est-ce point à vous de connaître ce qui est juste ?
\VS{2}Ils haïssent le bien, et aiment le mal ; ils leur arrachent la peau et la chair de dessus les os.
\VS{3}Ils dévorent la chair de mon peuple, lui arrachent la peau, et lui brisent les os ; ils le mettent en pièces comme dans un pot, comme de la viande dans une chaudière.
\VS{4}Alors ils crieront à Yahweh, mais il ne les exaucera point, et il leur cachera sa face en ce temps-là, parce qu’ils se sont mal conduits dans leurs actions
\VS{5}Ainsi parle Yahweh contre les prophètes qui égarent mon peuple, qui annoncent la paix si leurs dents ont quelque chose à mordre, et qui publient la guerre si on ne leur met rien dans la bouche :
\VS{6}C'est pourquoi la nuit sera sur vous, afin que vous n'ayez plus de vision ; et elle s'obscurcira, afin que vous ne deviniez plus ; le soleil se couchera sur ces prophètes-là, et le jour leur sera ténébreux.
\VS{7}Les voyants seront honteux, et les devins seront confondus ; eux tous se couvriront la barbe, parce qu'il n'y aura aucune réponse de Dieu.
\VS{8}Mais moi, je suis rempli de force, de justice, et de courage, par l'Esprit de Yahweh, pour déclarer à Jacob son crime, et à Israël son péché.
\TextTitle{[Future destruction de Jérusalem]}
\VS{9}Ecoutez maintenant ceci, chefs de la maison de Jacob, et vous conducteurs de la maison d'Israël, qui avez la justice en abomination, et qui pervertissez tout ce qui est droit,
\VS{10}vous qui bâtissez Sion avec le sang, et Jérusalem avec l'injustice.
\VS{11}Ses chefs jugent pour des présents, ses sacrificateurs enseignent pour un salaire, et ses prophètes devinent pour de l'argent\FTNT{Ce passage est encore d’actualité de nos jours. En effet, nombreux sont les dirigeants chrétiens qui exigent un salaire, notamment par le moyen de la dîme pour la plupart, en échange de leurs prières, enseignements,  conseils, formations bibliques… Le même constat se fait avec les chantres qui font des concerts payants alors qu’ils ont reçu leurs grâces du Seigneur gratuitement. Voir commentaire en Mt. 10:8.}, puis ils s'appuient sur Yahweh, en disant : Yahweh n'est-il pas parmi nous ? Le mal ne nous atteindra pas.
\VS{12}C'est pourquoi, à cause de vous, Sion sera labourée comme un champ, et Jérusalem sera réduite en ruines, et la montagne du temple en hauts lieux de forêt.
\TextTitle{[Marcher au nom de Yahweh]}
\Chap{4}
\VerseOne{}Mais il arrivera dans les derniers jours\FTNT{Voir commentaire en Ge. 49:1-2.}, que  la montagne de la maison de Yahweh\FTNT{Dans les Ecritures, les montagnes symbolisent parfois une grande puissance terrestre, et les collines celles de moindre importance. Cette prophétie confirme l’établissement du Royaume messianique dont la capitale sera Jérusalem (2 S. 7:14-16). Esaïe avait reçu la même prophétie que l’on peut découvrir au chapitre 2 de son livre.} sera affermie au sommet des montagnes, et sera élevée par-dessus les collines ; les peuples y afflueront.
\VS{2}Et des nations nombreuses iront et diront : Venez, et montons à la montagne de Yahweh, à la maison du Dieu de Jacob ; il nous enseignera ses voies, et nous marcherons dans ses sentiers ; car la loi sortira de Sion, et la parole de Yahweh de Jérusalem.
\VS{3}Il exercera le jugement parmi des peuples nombreux, et il sera l'arbitre de nations puissantes et lointaines ; et de leurs épées elles forgeront des hoyaux ; et de leurs hallebardes, des serpes ; une nation ne lèvera plus l'épée contre une autre, et on n'apprendra plus la guerre.
\VS{4}Mais chacun demeurera sous sa vigne et sous son figuier, et il n'y aura personne qui les épouvante ; car la bouche de Yahweh des armées aura parlé.
\VS{5}Les peuples marchent chacun au nom de leur dieu ; mais nous, nous marchons au Nom de Yahweh, notre Dieu, à toujours et à perpétuité.
\TextTitle{[Future restauration d'Israël]}
\VS{6}En ce jour-là, dit Yahweh, j'assemblerai les boiteux, je recueillerai ceux que j'avais chassés et ceux que j'avais maltraités.
\VS{7}Je ferai de ceux qui boitent un reste, et de ceux qui étaient éloignés une nation robuste ; Yahweh régnera sur eux, à la montagne de Sion, dès lors et à toujours.
\VS{8}Et toi, tour du troupeau, citadelle de la fille de Sion, jusqu'à toi viendra, à toi arrivera la souveraineté première ; le royaume sera à la fille de Jérusalem.
\TextTitle{[Yahweh, le Dieu qui rachète son peuple]}
\VS{9}Pourquoi maintenant pousses-tu des cris ? N'y a-t-il point de roi au milieu de toi ? Ou ton conseiller est-il mort, que la douleur t'ait saisie comme celle qui enfante ?
\VS{10}Souffre et gémis, fille de Sion, comme celle qui enfante ; car tu sortiras bientôt de la ville, tu demeureras aux champs, et tu iras jusqu'à Babylone ; là tu seras délivrée ; là Yahweh te rachètera de la main de tes ennemis.
\TextTitle{[Nations rassemblés pour l'Harmaguedon]}
\VS{11}Maintenant plusieurs nations se sont rassemblées contre toi\FTNT{Il est ici question de la guerre d’Harmaguédon. Voir commentaire en Ap. 16:12-16.} et disent : Qu'elle soit profanée ! Et que notre œil voie en Sion ce qu’il y voudrait voir\FTNT{Jérusalem est une horloge de Dieu. Le Seigneur a fait en sorte que les nations aient les yeux tournés vers ce bout de terre, car c'est là que débutera la troisième guerre mondiale, l’Harmaguédon (Mt. 24:15-28 ; Ap. 16:12-16 ; Ap. 19:11-21). Là aura lieu le jugement des nations, dans la vallée de Josaphat (Joë. 3:2-12). Le Messie reviendra (Es. 59:20-21 ; Za. 14:1-8 ; Ac. 1:10-11) et gouvernera le monde depuis Jérusalem (Za. 14:9-21). L'actuel conflit israélo-palestinien nous confirme bien ces prophéties. Il ne se passe pas un jour sans que les informations nous rapportent des événements venant de cet endroit du monde. D'ailleurs le Seigneur lui-même nous invite à suivre de près ce qu'il s'y passe (Mt. 24:32-34).}.
\VS{12}Mais ils ne connaissent point les pensées de Yahweh, et ne comprennent pas ses desseins ; car il les a assemblées comme des gerbes dans l'aire.
\VS{13}Lève-toi, et foule, fille de Sion ! Car je te ferai une corne de fer, et te mettrai des ongles d'airain ; et tu écraseras des peuples nombreux, et tu consacreras par interdit leurs profits à Yahweh, et leurs richesses au Seigneur de toute la terre.
\VS{14}Maintenant, fille de troupes, rassemble tes troupes ; on a mis le siège contre nous, on frappera le juge d'Israël avec la verge sur la joue.
\TextTitle{[Naissance du roi: le Messie]
\\(cp. Mt. 2:1-6 ; 27:24-37)}
\Chap{5}
\VerseOne{}Mais toi, Bethléhem Ephrata, petite pour être entre les milliers de Juda, de toi sortira quelqu’un pour être dominateur en Israël, dont l'origine remonte aux temps anciens, aux jours de l'éternité\FTNT{Il est question ici de Jésus-Christ. Ce passage nous parle de sa préexistence éternelle. Les pharisiens, scribes et principaux sacrificateurs avaient la connaissance de cette prophétie concernant le Messie (Mt. 2:1-6).}.
\VS{2}C'est pourquoi il les livrera jusqu'au temps où enfantera celle qui doit enfanter ; et le reste de ses frères retournera avec les enfants d'Israël.
\VS{3}Et il se maintiendra et gouvernera par la force de Yahweh, avec la magnificence du Nom de Yahweh, son Dieu ; et ils auront une demeure assurée, car dès lors il sera élevé jusqu'aux extrémités de la terre.
\VS{4}C'est lui qui sera la paix. Lorsque l'Assyrien sera entré dans notre pays, et qu'il aura mis le pied dans nos palais, nous élèverons contre lui sept pasteurs et huit princes du peuple.
\VS{5}Ils ravageront le pays d'Assyrie avec l'épée, et le pays de Nimrod à ses portes. Il nous délivrera ainsi des Assyriens, quand ils seront entrés dans notre pays, et qu'ils auront mis le pied dans nos quartiers.
\VS{6}Et le reste de Jacob sera au milieu de peuples nombreux, comme une rosée qui vient de Yahweh, et comme une pluie qui tombe sur l'herbe, qui ne s’attend à aucun homme, et qui n'espère pas des enfants des hommes.
\VS{7}Aussi le reste de Jacob sera parmi les nations, et au milieu de peuples nombreux, comme un lion parmi les bêtes de la forêt, et comme un lionceau parmi les troupeaux de brebis ; qui, en passant, foule et déchire, sans que personne ne puisse les sauver.
\TextTitle{[Jugement de Dieu sur les ennemis d'Isrël]}
\VS{8}Ta main se lèvera sur tes adversaires, et tous tes ennemis seront exterminés.
\VS{9}Et il arrivera en ce temps-là, dit Yahweh, que j'exterminerai du milieu de toi tes chevaux, et ferai périr tes chars.
\VS{10}J'exterminerai les villes de ton pays et renverserai toutes tes forteresses.
\VS{11}J'exterminerai aussi de ta main les sorcelleries, et tu n'auras plus aucun devin.
\VS{12}Et j'exterminerai du milieu de toi tes idoles et tes statues, et tu ne te prosterneras plus devant l'ouvrage de tes mains.
\VS{13}J'arracherai aussi du milieu de toi les poteaux d'Asherah\FTNT{Ex. 34:13.}, et détruirai tes ennemis.
\VS{14}Et j'exercerai ma vengeance avec colère et avec fureur contre toutes les nations qui ne m'auront pas écouté.
\TextTitle{[Yahweh appelle son peuple à l'humilité]}
\Chap{6}
\VerseOne{}Ecoutez maintenant ce que dit Yahweh : Lève-toi, plaide devant les montagnes, et que les collines entendent ta voix !
\VS{2}Ecoutez, montagnes, le procès de Yahweh, vous solides fondements de la terre ! Car Yahweh a un procès avec son peuple, et il plaidera avec Israël.
\VS{3}Mon peuple, que t'ai-je fait, ou en quoi t'ai-je causé de la peine ? Réponds-moi !
\VS{4}Car je t'ai fait sortir du pays d'Égypte et t'ai délivré de la maison de servitude, et j'ai envoyé devant toi Moïse, Aaron et Marie.
\VS{5}Mon peuple, rappelle-toi quel conseil Balak, roi de Moab, avait pris contre toi, et de ce que Balaam, fils de Beor, lui répondit ; et de ce que j'ai fait depuis Sittim jusqu'à Guilgal, afin que tu connaisses les justices de Yahweh.
\TextTitle{[Pratiquer la justice]}
\VS{6}Avec quoi me présenterai-je devant Yahweh, et me prosternerai-je devant le Dieu Très-Haut ? Me présenterai-je avec des holocaustes, et avec des veaux d'un an ?
\VS{7}Yahweh prendra-t-il plaisir à des milliers de béliers ou à des myriades de torrents d'huile ? Donnerai-je pour mon crime mon premier-né\FTNT{Selon la loi (Ex. 13 : 2 ; Ex. 3 : 12), les premiers-nés de l’homme et des animaux appartenaient au Seigneur. Ceux des animaux étaient offerts en sacrifice alors que le sacrifice des enfants était formellement interdit sous peine de mort (Lé. 18:21 ; Lé. 20:2-5 ; De. 12:31 ; De. 18:10).}, le fruit de mes entrailles pour le péché de mon âme ?
\VS{8}Ô homme ! Il t'a fait connaître ce qui est bon, et ce que Yahweh exige de toi : Que tu fasses ce qui est juste, que tu aimes la miséricorde, et que tu marches en toute humilité avec ton Dieu.
\VS{9}La voix de Yahweh crie à la ville, et le sage reconnaît son Nom. Écoutez la verge, et celui qui la dirige !
\VS{10}Y a-t-il encore dans la maison du méchant des trésors iniques, et un épha court et détestable ?
\VS{11}Tiendrai-je pour pur celui qui a de fausses balances et de faux poids dans son sac ?
\VS{12}Ses riches sont pleins de violence, ses habitants usent de mensonge, et ils ont une langue trompeuse dans leur bouche.
\VS{13}C'est pourquoi je te rendrai languissante en te frappant, et te ravagerai à cause de tes péchés.
\VS{14}Tu mangeras, mais tu ne seras pas rassasiée, et la faim sera au-dedans de toi-même ; tu mettras de côté, mais tu ne sauveras point, et ce que tu auras sauvé je le livrerai à l'épée.
\VS{15}Tu sèmeras, mais tu ne moissonneras point ; tu presseras l'olive, mais tu ne feras pas d'onctions d'huile ; et tu presseras le moût, mais tu ne boiras pas le vin.
\VS{16}Car tu as gardé les ordonnances d'Omri, et toutes les œuvres de la maison d'Achab, et tu as marché dans leurs conseils. C'est pourquoi je te livrerai à la désolation, je ferai de tes habitants un objet de raillerie, et vous porterez l'opprobre de mon peuple.
\TextTitle{[Le mal appelé bien, et le bien appelé mal]}
\Chap{7}
\VerseOne{}Malheur à moi ! Car je suis comme quand on a cueilli les fruits d'été et les grappillages de la vendange : Il n'y a ni grappe pour manger ni les premiers fruits que mon âme désirait.
\VS{2}Le fidèle est exterminé du pays, et il n'y a plus de juste entre les hommes ; ils sont tous en embûche pour verser le sang, chacun chasse son frère avec des filets.
\VS{3}Leurs mains sont habiles à faire le mal : Le gouverneur exige, le juge demande un salaire, le grand déclare ce qu'il convoite, et ils s'unissent.
\VS{4}Le meilleur d'entre eux est comme une ronce, et le plus juste est pire qu'une haie d'épines. Le jour annoncé par tes sentinelles, ton châtiment arrive. C'est alors qu'ils seront dans la confusion.
\VS{5}Ne crois pas à ton ami intime, et ne te confie pas en tes conducteurs ; garde-toi d'ouvrir ta bouche devant la femme qui dort dans ton sein.
\VS{6}Car le fils déshonore le père, la fille s'élève contre sa mère, la belle-fille contre sa belle-mère, et chacun a pour ennemis les gens de sa maison.
\TextTitle{[Espérance en Yahweh, le Dieu de notre salut]}
\VS{7}Mais moi, je regarderai vers Yahweh, je m'attendrai au Dieu de mon salut ; mon Dieu m'exaucera.
\VS{8}Toi, mon ennemie, ne te réjouis pas sur moi ; si je suis tombée, je me relèverai ; si j'ai été gisante dans les ténèbres, Yahweh m'éclairera.
\VS{9}Je supporterai la colère de Yahweh, car j'ai péché contre lui, jusqu'à ce qu'il défende ma cause, et qu'il me fasse justice ; il me conduira à la lumière, je verrai sa justice.
\VS{10}Et mon ennemie le verra, et la honte la couvrira ; elle qui me disait : Où est Yahweh, ton Dieu ? Mes yeux la verront, et alors elle sera foulée aux pieds comme la boue des rues.
\VS{11}Le jour où il rebâtira tes murs, en ce jour-là tes limites seront reculées.
\VS{12}En ce jour-là on viendra jusqu'à toi d'Assyrie et des villes d'Égypte, et depuis les villes d'Égypte jusqu'au fleuve, et depuis une mer jusqu'à l'autre mer, et depuis une montagne jusqu'à l'autre montagne ;
\VS{13}après que le pays aura été en désolation à cause de ses habitants, et du fruit de leurs actions.
\VS{14}Pais ton peuple avec ta houlette, le troupeau de ton héritage, qui demeure seul dans les forêts au milieu de Carmel ! Et fais qu'ils paissent en Basan et en Galaad, comme aux temps anciens.
\VS{15}Je lui ferai voir des choses merveilleuses, comme au jour où tu sortis du pays d'Égypte.
\VS{16}Les nations le verront, et elles seront honteuses avec toute leur force ; elles mettront la main sur la bouche, et leurs oreilles seront sourdes.
\VS{17}Elles lécheront la poussière comme le serpent, comme les reptiles de la terre ; elles trembleront dans leurs forteresses et accourront toutes effrayées vers Yahweh, notre Dieu, et te craindront.
\VS{18}Quel dieu est semblable à toi, qui est un Dieu qui pardonne l'iniquité, et qui passe par-dessus les péchés du reste de son héritage ? Il ne garde pas à toujours sa colère, parce qu'il prend plaisir à la miséricorde.
\VS{19}Il aura encore compassion de nous ; il effacera nos iniquités, et jettera tous nos péchés au fond de la mer.
\VS{20}Tu feras voir ta fidélité à Jacob, et ta miséricorde à Abraham, comme tu l’as juré à nos pères dès les temps anciens\FTNT{Les versets 18 à 20 de Mi. 7 sont lus chaque année dans les synagogues le jour des expiations.}.
\PPE{}
\end{multicols}

\clearpage%& -output-directory="./pdf"
% type document & taille police
\documentclass[11pt]{book}
% package format document
\usepackage[paperwidth=6.5in, paperheight=9.05in, top=0in, bottom=0in, left=0in, right=0in]{geometry}
% formatage marges, etc.
\setlength{\voffset}{-0.7in} % offset haut
%\setlength{\hoffset}{-0.3in} % offset gauche
\setlength{\topmargin}{0in} % marge en tête
\setlength{\headsep}{0.2in} % marge header/body
\setlength{\oddsidemargin}{-0.5in} % marge texte gauche
\setlength{\evensidemargin}{-0.5in} % marge texte droite
\setlength{\textheight}{8in} % hauteur du texte
\setlength{\textwidth}{5.5in} % largeur du texte
\setlength{\columnseprule}{0.4pt} % épaisseur séparateur colonne
\setlength{\parskip}{0pt} % espace entre paragraphes
% package pour afficher les cadres
%\usepackage{showframe}
% package langue
\usepackage[francais]{babel}
% package polices système
\usepackage{fontspec}
% définition police
\setmainfont[Ligatures=TeX,Scale=0.95]{Liberation Serif}
\setsansfont{Liberation Sans}
\setmonofont{Liberation Mono}
% package titlesec
\usepackage{titlesec}
% package multicolonne
\usepackage{multicol}
% package liens cliquables
\usepackage[xetex]{hyperref}
% package inclusion copyright (dépandant de hyperref)
\usepackage{hyperxmp}
% copyright
\hypersetup{
pdfauthor = {ANJC Productions},
pdftitle = {Bible de Jésus-Christ},
pdfkeywords = {BJC, Bible, Jesus},
pdfcopyright = {ANJC Productions. Distribution et Diffusion Libres - Pas d'Utilisation Commerciale - Pas de Dénaturation de l'Œuvre - International},
pdflicenseurl = {http://www.bibledejesuschrist.org/}
}
% ???
\setcounter{collectmore}{-1}
% style
\pagestyle{myheadings}
% ???
\sloppy\hyphenpenalty=2000
% titres de livres
\newcommand{\ShortTitle}[1]{\def\webbook{#1}\par\goodbreak\bigskip\setcounter{footnote}{0}}
\newcommand{\BookTitle}[1]{\par\goodbreak\bigskip{\parindent=0mm\begin{center}{\small\bfseries{\LARGE #1\nopagebreak}}\end{center}}\addcontentsline{toc}{subsection}{#1}\nopagebreak\par\nobreak}
% chapitres
\newcommand{\Chap}[1]{\def\webchap{#1:}\def\webvs{1}\def\vchap{#1}\ssubsection{\centerline{\textbf{{CHAPITRE\ #1}}}}}
% versets
\newcommand{\VerseOne}{\def\webvs{1}{\up{\footnotesize 1}}\markboth{\webbook\ \webchap 1}{\webbook\ \webchap 1}}
\newcommand{\VS}[1]{\def\webvs{#1}{\up{\footnotesize #1}}\markboth{\webbook\ \webchap #1}{\webbook\ \webchap #1}}
\newcommand{\vref}[1]{\NoAutoSpaceBeforeFDP{#1}}
% commentaires
%\interfootnotelinepenalty=10000 % longueur max commentaires
\renewcommand{\thefootnote}{\alph{footnote}} % repères alphabetiques
\renewcommand{\footnoterule}{\hrule width \textwidth} % longueur ligne
\newcommand{\FTNT}[1]{\ifnum\value{footnote}>25\setcounter{footnote}{0}\fi\footnote{[\NoAutoSpaceBeforeFDP{\webchap\webvs}]\ #1}}
% commentaire sur les titres
\newcounter{webvst}
\newcommand{\FTNTT}[1]{
% intialisation de l'indice de note
\ifnum \value{footnote}>25 \setcounter{footnote}{0} \fi
% initialisation de la référence du numéro de verset
\setcounter{webvst}{\webvs}
% si le titre est sur le premier verset, incrémenter de 1
\ifnum \value{webvst}>1 \addtocounter{webvst}{1} \fi
% écriture note
\footnote{[\NoAutoSpaceBeforeFDP{\webchap\thewebvst}]\ #1}
}
% titres de paragraphes
\titlespacing*{\subsection}{0pt}{5pt plus 0pt minus 0pt}{5pt plus 0pt minus 0pt}
\titlespacing*{\subsubsection}{0pt}{5pt plus 0pt minus 0pt}{5pt plus 0pt minus 0pt}
\newcommand{\ssubsection}[1]{\subsection*{\centering\footnotesize\normalfont #1}\PP}
\newcommand{\ssubsubsection}[1]{\subsubsection*{\centering\footnotesize\normalfont #1}\PP}
\newcommand{\TextTitle}[1]{\ssubsubsection{[\textit{#1}]}}
\newcommand{\TextDial}[1]{{\scriptsize[\textit{#1}]}}
% dictionnaire
\newcommand{\DicoEntry}[1]{\smallskip\parindent=0mm{\textbf{#1}}\markboth{#1}{#1}}
% commandes diverses
\newcommand{\BFont}{\normalfont\small}
\newcommand{\PP}{\par\parindent=0mm}
\newcommand{\PPE}{\par\parindent=4mm}
% debut document
\begin{document}
% en-tête pages
\makeatletter
\def\@evenhead{{\NoAutoSpaceBeforeFDP{\small{\rightmark\hfil\thepage\hfil\leftmark}}}}
\def\@oddhead{{\NoAutoSpaceBeforeFDP{\small{\rightmark\hfil\thepage\hfil\leftmark}}}}
\makeatother
% inclusion des livres
\pagenumbering{arabic}
\clearpage%& -output-directory="./pdf"
% type document & taille police
\documentclass[11pt]{book}
% package format document
\usepackage[paperwidth=6.5in, paperheight=9.05in, top=0in, bottom=0in, left=0in, right=0in]{geometry}
% formatage marges, etc.
\setlength{\voffset}{-0.7in} % offset haut
%\setlength{\hoffset}{-0.3in} % offset gauche
\setlength{\topmargin}{0in} % marge en tête
\setlength{\headsep}{0.2in} % marge header/body
\setlength{\oddsidemargin}{-0.5in} % marge texte gauche
\setlength{\evensidemargin}{-0.5in} % marge texte droite
\setlength{\textheight}{8in} % hauteur du texte
\setlength{\textwidth}{5.5in} % largeur du texte
\setlength{\columnseprule}{0.4pt} % épaisseur séparateur colonne
\setlength{\parskip}{0pt} % espace entre paragraphes
% package pour afficher les cadres
%\usepackage{showframe}
% package langue
\usepackage[francais]{babel}
% package polices système
\usepackage{fontspec}
% définition police
\setmainfont[Ligatures=TeX,Scale=0.95]{Liberation Serif}
\setsansfont{Liberation Sans}
\setmonofont{Liberation Mono}
% package titlesec
\usepackage{titlesec}
% package multicolonne
\usepackage{multicol}
% package liens cliquables
\usepackage[xetex]{hyperref}
% package inclusion copyright (dépandant de hyperref)
\usepackage{hyperxmp}
% copyright
\hypersetup{
pdfauthor = {ANJC Productions},
pdftitle = {Bible de Jésus-Christ},
pdfkeywords = {BJC, Bible, Jesus},
pdfcopyright = {ANJC Productions. Distribution et Diffusion Libres - Pas d'Utilisation Commerciale - Pas de Dénaturation de l'Œuvre - International},
pdflicenseurl = {http://www.bibledejesuschrist.org/}
}
% ???
\setcounter{collectmore}{-1}
% style
\pagestyle{myheadings}
% ???
\sloppy\hyphenpenalty=2000
% titres de livres
\newcommand{\ShortTitle}[1]{\def\webbook{#1}\par\goodbreak\bigskip\setcounter{footnote}{0}}
\newcommand{\BookTitle}[1]{\par\goodbreak\bigskip{\parindent=0mm\begin{center}{\small\bfseries{\LARGE #1\nopagebreak}}\end{center}}\addcontentsline{toc}{subsection}{#1}\nopagebreak\par\nobreak}
% chapitres
\newcommand{\Chap}[1]{\def\webchap{#1:}\def\webvs{1}\def\vchap{#1}\ssubsection{\centerline{\textbf{{CHAPITRE\ #1}}}}}
% versets
\newcommand{\VerseOne}{\def\webvs{1}{\up{\footnotesize 1}}\markboth{\webbook\ \webchap 1}{\webbook\ \webchap 1}}
\newcommand{\VS}[1]{\def\webvs{#1}{\up{\footnotesize #1}}\markboth{\webbook\ \webchap #1}{\webbook\ \webchap #1}}
\newcommand{\vref}[1]{\NoAutoSpaceBeforeFDP{#1}}
% commentaires
%\interfootnotelinepenalty=10000 % longueur max commentaires
\renewcommand{\thefootnote}{\alph{footnote}} % repères alphabetiques
\renewcommand{\footnoterule}{\hrule width \textwidth} % longueur ligne
\newcommand{\FTNT}[1]{\ifnum\value{footnote}>25\setcounter{footnote}{0}\fi\footnote{[\NoAutoSpaceBeforeFDP{\webchap\webvs}]\ #1}}
% commentaire sur les titres
\newcounter{webvst}
\newcommand{\FTNTT}[1]{
% intialisation de l'indice de note
\ifnum \value{footnote}>25 \setcounter{footnote}{0} \fi
% initialisation de la référence du numéro de verset
\setcounter{webvst}{\webvs}
% si le titre est sur le premier verset, incrémenter de 1
\ifnum \value{webvst}>1 \addtocounter{webvst}{1} \fi
% écriture note
\footnote{[\NoAutoSpaceBeforeFDP{\webchap\thewebvst}]\ #1}
}
% titres de paragraphes
\titlespacing*{\subsection}{0pt}{5pt plus 0pt minus 0pt}{5pt plus 0pt minus 0pt}
\titlespacing*{\subsubsection}{0pt}{5pt plus 0pt minus 0pt}{5pt plus 0pt minus 0pt}
\newcommand{\ssubsection}[1]{\subsection*{\centering\footnotesize\normalfont #1}\PP}
\newcommand{\ssubsubsection}[1]{\subsubsection*{\centering\footnotesize\normalfont #1}\PP}
\newcommand{\TextTitle}[1]{\ssubsubsection{[\textit{#1}]}}
\newcommand{\TextDial}[1]{{\scriptsize[\textit{#1}]}}
% dictionnaire
\newcommand{\DicoEntry}[1]{\smallskip\parindent=0mm{\textbf{#1}}\markboth{#1}{#1}}
% commandes diverses
\newcommand{\BFont}{\normalfont\small}
\newcommand{\PP}{\par\parindent=0mm}
\newcommand{\PPE}{\par\parindent=4mm}
% debut document
\begin{document}
% en-tête pages
\makeatletter
\def\@evenhead{{\NoAutoSpaceBeforeFDP{\small{\rightmark\hfil\thepage\hfil\leftmark}}}}
\def\@oddhead{{\NoAutoSpaceBeforeFDP{\small{\rightmark\hfil\thepage\hfil\leftmark}}}}
\makeatother
% inclusion des livres
\pagenumbering{arabic}
\clearpage\input{../bjc_2014/34-Nahum}
\end{document}

\end{document}

\clearpage\ShortTitle{Habakuk}\BookTitle{Habakuk}\BFont
\noindent\hrulefill
{\footnotesize
\textit{
\bigskip
{\centering{}
\\Signifie : Embrasser, amour
\\Thème : Du doute à la foi
\\Auteur : Habakuk
\\Date de rédaction : 7ème siècle av. J.-C.\\}
}
%\bigskip
\textit{
\\Habakuk, contemporain de Nahum, Sophonie et Jérémie, exerça son ministère dans le royaume de Juda. Véritable sentinelle, il fut chargé d’annoncer le châtiment de Juda par les chaldéens. Ce récit, qui est en partie un dialogue entre Dieu et Habakuk, témoigne de la relation qui les liait. Il est aussi une invitation à la patience et la foi en Yahweh.\bigskip
}
}
\par\nobreak\noindent\hrulefill
\begin{multicols}{2}
\Chap{1}
\TextTitle{[Quand la méchanceté semble triompher de la justice]}
\VerseOne{}Oracle qu’Habakuk, le prophète a vu.
\VS{2}Ô Yahweh ! Jusqu’à quand crierai-je sans que tu m'écoutes ? Jusqu'à quand crierai-je vers toi ? On me traite avec violence sans que tu me délivres !
\VS{3}Pourquoi me fais-tu voir la méchanceté\FTNT{La perplexité d’Habakuk était la même que celle de Job (Job. 21:7), d’Asaph (Ps. 73) et de Jérémie (Jé. 12:1-2). Les méchants semblent prospérer tandis que les justes pleurent et sont persécutés (Mal. 3 : 12-15).}, et vois-tu la perversité ? Pourquoi y a-t-il de l’oppression et de la violence devant moi, et des gens qui excitent des procès et des querelles ?
\VS{4}Parce que la loi est sans force, et que la justice ne se fait jamais, à cause de cela le méchant environne le juste, et à cause de cela on rend des jugements corrompus\FTNT{Jé. 5:26 ; Am. 5:7.}.
\TextTitle{[La réponse de Yahweh]}
\VS{5}Regardez parmi les nations, et voyez, et soyez étonnés et stupéfaits ! Car je vais faire en vos jours une œuvre que vous ne croiriez pas si on vous la racontait\FTNT{Ac. 13:41.}.
\VS{6}Car voici, je vais susciter les Chaldéens, ce peuple cruel et impétueux, marchant sur l'étendue de la terre, pour posséder des demeures qui ne lui appartiennent pas.
\VS{7}Il est redoutable et terrible, son gouvernement et son autorité viennent de lui-même .
\VS{8}Ses chevaux sont plus légers que les léopards, et ils ont la vue plus aiguë que les loups du soir ; et ses cavaliers se répandront çà et là, même ses cavaliers viendront de loin ; ils voleront comme un aigle qui fond sur sa proie\FTNT{Jé. 5:6 ; So. 3:3.}.
\VS{9}Ils viendront tous pour la violence ; ce qu'ils engloutiront de leurs regards sera porté vers l'orient, et ils amasseront les prisonniers comme du sable.
\VS{10}Ce peuple se moque des rois, et les princes sont l’objet de ses railleries ; il se rit de toutes les forteresses ; il amoncelle de la terre, et il s’en empare.
\VS{11}Alors il traverse comme le vent, il passe outre et se rend coupable, car sa force est son dieu.
\TextTitle{[La souveraineté de Dieu]}
\VS{12}N'es-tu pas de toute éternité, ô Yahweh ! Mon Dieu, mon Saint ? Nous ne mourrons point ! Ô Yahweh, tu l'as établi pour exécuter tes jugements ; et toi, mon rocher\FTNT{Voir commentaire en  Es. 8:13-17.}, tu l'as fondé pour punir.
\VS{13}Tu as les yeux trop purs pour voir le mal, et tu ne saurais prendre plaisir à regarder le mal qu'on fait à autrui. Pourquoi regarderais-tu les perfides, et te tairais-tu quand le méchant dévore son prochain qui est plus juste que lui ?
\VS{14}Or tu as fait les hommes comme les poissons de la mer, et comme le reptile qui n'a point de maître.
\VS{15}Il a tout enlevé avec l'hameçon ; il l'a amassé avec son filet, et l'a assemblé dans son rets ; c'est pourquoi il se réjouira et s'égayera\FTNT{Am. 4:2.}.
\VS{16}A cause de cela, il sacrifie à son filet, et il offre de l’encens à ses rets, parce qu'il aura eu par leur moyen une grasse portion, et que sa viande est une chose moelleuse.
\VS{17}Videra-t-il à cause de cela son filet ? Et ne cessera-t-il jamais de faire le carnage des nations ?
\TextTitle{[S'attendre à Yahweh]}
\Chap{2}
\VerseOne{}Je me tenais en sentinelle, j'étais debout dans la forteresse et je faisais le guet, pour voir ce qu’il me dira, et ce que je répondrais après ma plainte\FTNT{Jé. 6:17 ; Es. 21:1-6 ; Ez. 33:1-19.}.
\TextTitle{[Le juste vivra par la foi]}
\VS{2}Et Yahweh m'a répondu et m'a dit : Ecris la vision, et grave-la sur des tablettes, afin qu'on la lise couramment.
\VS{3}Car la vision est encore différée jusqu'à un certain temps, et Yahweh parlera de ce qui arrivera à la fin, et il ne mentira point. S'il tarde, attends-le, car il ne manquera point de venir, et il ne tardera point\FTNT{Hé. 10:37.}.
\VS{4}Voici, l'âme de celui qui s'élève n'est point droite en lui ; mais le juste vivra de sa foi\FTNT{Ro. 1:17 ; Hé. 10:38.}.
\VS{5}Et combien plus l'homme adonné au vin est-il perfide, et l'homme puissant est-il orgueilleux, ne se tenant point tranquille chez lui ; il élargit son âme comme le scheol, et il est insatiable comme la mort, il rassemble vers lui toutes les nations, et réunit à lui tous les peuples.
\VS{6}Tous ceux-là ne feront-ils pas de lui un sujet de raillerie et d’énigmes ? Et ne dira-t-on pas : Malheur à celui qui accumule ce qui ne lui appartient point ; jusqu'à quand le fera-t-il, et entassera-t-il sur lui de la boue épaisse ?
\VS{7}Ne se lèveront-ils pas soudain, ceux qui le mordront ? Ne se réveilleront-ils pas pour te tourmenter ? Et tu deviendras leur proie.
\VS{8}Parce que tu as pillé beaucoup de nations, tout le reste des peuples te pillera, et à cause aussi des meurtres des hommes, et de la violence faite dans le pays, contre la ville, et contre tous ses habitants\FTNT{Es. 33:1 ; Na. 3:1.}.
\VS{9}Malheur à celui qui amasse pour sa maison des gains injustes, afin de placer son nid dans un lieu élevé, pour échapper à l’atteinte de la calamité !
\VS{10}C’est pour la confusion de ta maison que tu as pris conseil, en détruisant beaucoup de peuples, et c’est contre ton âme que tu as péché.
\VS{11}Car la pierre crie du milieu de la muraille, et de la charpente la poutre lui répond.
\VS{12}Malheur à celui qui bâtit des villes avec le sang et qui fonde des cités sur l'iniquité.
\VS{13}Voici, n'est-ce pas la volonté de Yahweh des armées que les peuples travaillent pour le feu, et que les peuples se lassent pour le néant ?
\VS{14}Car la terre sera remplie de la connaissance de la gloire de Yahweh\FTNT{Es. 11:9.}, comme le fond de la mer par les eaux qui le couvrent.
\VS{15}Malheur à celui qui fait boire son compagnon en lui approchant sa bouteille, et qui l’enivre afin qu'on voie sa nudité\FTNT{Es. 5:22 ; Ge. 9:21-24.}.
\VS{16}Tu seras rassasié de honte plutôt que de gloire ; toi aussi bois, et découvre-toi. La coupe de la droite de Yahweh fera le tour jusqu’à toi, et l'ignominie sera répandue sur ta gloire.
\VS{17}Car la violence faite au Liban retombera sur toi ; et les ravages des bêtes t’effrayeront, parce que tu as répandu le sang des hommes, et commis  des violences dans le pays, contre la ville et tous ses habitants.
\VS{18}A quoi sert l'image taillée  pour qu’un ouvrier la taille ? A quoi sert l’image de fonte, docteur de mensonge, a quoi sert-elle pour que l'ouvrier qui l’a faite place en elle sa confiance en fabriquant des idoles muettes ?
\VS{19}Malheur à ceux qui disent au bois : Réveille-toi ! Et à la pierre muette : Réveille-toi ! Enseignera-t-elle ? Voici, elle est couverte d'or et d'argent, et il n'y a aucun esprit au-dedans d’elle.
\VS{20}Mais Yahweh est dans le temple de sa sainteté. Que toute la terre fasse silence devant lui !
\TextTitle{[Habakuk reconnait et accepte la volonté de Dieu]}
\Chap{3}
\VerseOne{}Prière d'Habakuk, le prophète, sur le mode des chants lyriques.
\VS{2}Yahweh, j'ai entendu ce que tu m'as fait entendre, et j'ai été saisi de crainte, ô Yahweh ! Dans le cours des années, ravive ton œuvre ; dans le cours des années, fais-la connaître; dans ta colère souviens-toi de tes compassions.
\VS{3}Dieu vient de Théman, et le Saint vient du mont de Paran. Sélah. Sa majesté couvre les cieux, et la terre est remplie de sa louange.
\VS{4}Sa splendeur est comme la lumière même, et des rayons sortent de sa main ; c'est là où réside sa force.
\VS{5}La peste marche devant lui, et une  flamme ardente sort sous ses pieds.
\VS{6}Il s'arrête et mesure la terre ; il regarde et met en déroute les nations ; les montagnes antiques sont  brisées en éclats,  et les collines éternelles s’affaissent. Ses voies sont les voies anciennes.
\VS{7}Je vois les tentes de Cuschan accablées sous la punition ; les pavillons du pays de Madian sont ébranlés.
\VS{8}Est-ce contre les fleuves que s’irrite Yahweh ? Ta colère est-elle contre les fleuves, et ta fureur contre la mer, que tu sois monté sur tes chevaux et sur tes chars de délivrance ?
\VS{9}Ton arc est mis à nu et tire toutes les flèches, selon le serment fait aux tribus, à savoir ta parole. Sélah. Tu fends la terre et tu en fais sortir des fleuves\FTNT{Ps. 78:15-16 ; Ps. 105:41.}.
\VS{10}Les montagnes te voient et elles tremblent\FTNT{Ps. 114:4-7.}; des torrents d’eau se précipitent, l'abîme fait retentir sa voix de la profondeur, il élève ses mains en haut.
\VS{11}Le soleil et la lune s'arrêtent dans leur habitation\FTNT{Jos. 10:12 ; Ap. 22:5.}, ils marchent à la lueur de tes flèches, et à la splendeur de l'éclat de ta lance étincelante.
\VS{12}Tu marches sur la terre avec indignation, et foules les nations avec colère.
\VS{13}Tu sors pour la délivrance de ton peuple, tu sors avec ton Oint pour la délivrance ; tu transperces le chef, afin qu'il n'y en ait plus dans la maison du méchant, tu en découvres le fondement  jusqu’au fond. Sélah.
\VS{14}Tu perces avec ses flèches  la tête de ses chefs, quand ils viennent comme une tempête pour me dissiper ; ils s'égaient comme pour dévorer l'affligé dans sa retraite.
\VS{15}Tu marches avec tes chevaux par la mer, les grandes eaux ayant été amoncelées.
\VS{16}J'ai entendu ce que tu m'as déclaré, et mes entrailles en sont émues ; à ta voix le tremblement saisit mes lèvres ; la pourriture entre dans mes os, et je tremble en moi-même, car je serai en repos au jour de la détresse, lorsque montant vers le peuple, il le mettra en pièces.
\VS{17}Car le figuier ne fleurira pas, et il n'y aura point de fruit dans les vignes ; ce que l'olivier produit mentira, et aucun champ ne produira rien à manger ; les brebis seront retranchées du parc, et il n'y aura point de bœufs dans les étables.
\VS{18}Mais moi, je me réjouis en Yahweh, et je me réjouis dans le Dieu de ma délivrance.
\VS{19}Yahweh, le Seigneur, est ma force, et il rend mes pieds semblables à ceux des biches, et me fait marcher sur mes lieux élevés\FTNT{Ps. 18:33-34 ; De. 32:13.}. Au chef des chantres avec instruments à cordes.
\PPE{}
\end{multicols}

\clearpage\ShortTitle{So.}\BookTitle{Sophonie}\BFont
\noindent\hrulefill
{\footnotesize
\textit{
\bigskip
{\centering{}
\\Auteur~: Sophonie
\\(Heb.~: Tsephanyah)
\\Signification~: Yahweh a caché, protégé
\\Thème~: Le jour de Yahweh
\\Date de rédaction~: 7\up{ème} siècle av. J.-C.\\}
}
\textit{
\\De lignée royale, Sophonie exerça son service dans le royaume de Juda au temps du roi Josias et fut contemporain de Jérémie, Habakuk, Ezéchiel et Abdias. A une époque où l'iniquité s'était accrue au point où les quelques personnes fidèles à Dieu étaient persécutées, Sophonie fut suscité par Yahweh pour annoncer le jugement de Juda, d'Israël et de quelques nations païennes.\bigskip
}
}
\par\nobreak\noindent\hrulefill
\begin{multicols}{2}
\Chap{1}
\TextTitle{Yahweh annonce son jugement sur Juda, conséquence de son idolâtrie}
\VerseOne{}C'est ici la parole de Yahweh qui fut adressée à Sophonie, fils de Cuschi, fils de Guedalia, fils d'Amaria, fils d'Ezéchias, du temps de Josias, fils d'Amon, roi de Juda.
\VS{2}Je ferai entièrement périr toutes choses de dessus cette terre, dit Yahweh.
\VS{3}Je ferai périr l'homme et le bétail~; je consumerai les oiseaux des cieux et les poissons de la mer~; et la ruine arrivera aux méchants, et je retrancherai les hommes de dessus cette terre, dit Yahweh.
\VS{4}J'étendrai ma main sur Juda, et sur tous les habitants de Jérusalem~; je retrancherai de ce lieu-ci le reste de Baal\FTNT{Voir commentaire en Jg. 2:13.}, les noms des prêtres des faux dieux, les prêtres,
\VS{5}ceux qui se prosternent sur les toits devant l'armée des cieux, ceux qui se prosternent devant Yahweh, qui jurent par lui, et qui jurent aussi par Malcom\FTNT{2 R. 17:33~; 2 R. 23:11-12~; Jé. 19:13.},
\VS{6}ceux qui se détournent de Yahweh, ceux qui n'ont point cherché Yahweh, qui ne l'ont point consulté.
\VS{7}Silence, à cause de la présence du Seigneur Yahweh, car le jour de Yahweh est proche\FTNT{Voir commentaire en Za. 14:1.}~; Yahweh a préparé le sacrifice, il a invité ses conviés.
\VS{8}Et il arrivera au jour du sacrifice de Yahweh que je punirai les chefs, et les enfants du roi, et tous ceux qui portent des vêtements étrangers.
\VS{9}Et je punirai, en ce jour-là, tous ceux qui sautent par-dessus le seuil, et ceux qui remplissent de violence et de fraude la maison de leurs maîtres.
\VS{10}Et en ce jour-là dit Yahweh, il y aura de grands cris vers la porte des poissons, et des hurlements vers la seconde partie de la ville, et une grande désolation sur les collines.
\VS{11}Vous qui habitez dans Macthesch\FTNT{Macthesch était un bas-quartier de Jérusalem où se trouvaient les marchés.}, hurlez~! Car tous ceux qui trafiquaient ont été détruits, et tous ceux qui apportaient de l'argent ont été retranchés.
\VS{12}Et il arrivera en ce temps-là que je fouillerai Jérusalem avec des lampes, que je punirai les hommes qui sont figés sur leurs lies, et qui disent dans leurs cœurs~: Yahweh ne nous fera ni bien ni mal.
\VS{13}Leurs biens seront au pillage et leurs maisons en désolation~; et ils auront bâti des maisons, mais ils ne les habiteront pas~; ils auront planté des vignes, mais ils n'en boiront pas le vin.
\VS{14}Le grand jour de Yahweh est proche, il est proche, et il se hâte beaucoup~; le jour de Yahweh n'est que bruit~; celui qui est dans l'amertume, crie de toute sa force. Là sont les hommes vaillants\FTNT{Jé. 30:7~; Joë. 2:11~; Am. 5:18.}.
\VS{15}Ce jour est un jour de fureur, un jour de détresse et d'angoisse, un jour de bruit éclatant et effrayant, un jour de ténèbres et d'obscurité, un jour de nuées et de brouillards~;
\VS{16}un jour de shofar et de cris de guerre contre les villes fortifiées, et contre les hautes tours.
\VS{17}Je mettrai les hommes dans la détresse, et ils marcheront comme des aveugles, parce qu'ils ont péché contre Yahweh~; et leur sang sera répandu comme de la poussière, et leur chair comme des ordures.
\VS{18}Ni leur argent ni leur or ne pourront les délivrer au jour de la fureur de Yahweh~; et tout ce pays sera dévoré par le feu de sa jalousie, car il se hâtera de consumer tous les habitants de ce pays\FTNT{Ez. 7:19~; Pr. 11:4.}.
\Chap{2}
\TextTitle{Yahweh invite Israël à la repentance}
\VerseOne{}Examinez-vous, examinez-vous avec soin ô nation non désirée\FTNT{1 Th. 5:21~; 2 Co. 13:5~; Ep. 5:10.}~!
\VS{2}Avant que le décret enfante, et que le jour passe comme la balle~; avant que l'ardeur de la colère de Yahweh vienne sur vous, avant que le jour de la colère de Yahweh vienne sur vous~!
\VS{3}Vous, tous les pauvres du pays, qui faites ce qu'il ordonne, cherchez Yahweh, cherchez la justice, cherchez l'humilité~; peut-être serez-vous protégés au jour de la colère de Yahweh\FTNT{Am. 5:15.}.
\VS{4}Mais Gaza sera abandonnée, et Askalon sera en désolation~; on chassera les habitants d'Asdod en plein midi, et Ekron sera arrachée\FTNT{Am. 8:9~; Za. 9:5.}.
\VS{5}Malheur aux habitants de la contrée maritime, à la nation des Kéréthiens~! La parole de Yahweh est contre vous~; Canaan, qui est le pays des Philistins, je te détruirai, si bien que, personne n'y habitera.
\VS{6}Et la contrée maritime sera des pâturages, des demeures pour les bergers, et des parcs pour les troupeaux.
\VS{7}Et cette contrée sera pour le reste de la maison de Juda~; ils paîtront dans ces lieux-là, et le soir ils feront leur gîte dans les maisons d'Askalon~; car Yahweh, leur Dieu, les visitera, et il ramènera leurs captifs.
\VS{8}J'ai entendu les insultes de Moab, et les outrages des fils d'Ammon, quand ils ont diffamé mon peuple, et l'ont bravé sur leur frontière\FTNT{Ez. 25:3-6.}.
\VS{9}C'est pourquoi, je suis vivant, dit Yahweh des armées, le Dieu d'Israël, Moab sera comme Sodome, et les fils d'Ammon comme Gomorrhe, un lieu couvert d'orties, et une carrière de sel et de désolation à jamais~; les restes de mon peuple les pilleront, et les restes de ma nation les posséderont.
\VS{10}Ceci leur arrivera en échange de leur orgueil, parce qu'ils ont usé d'insultes et d'arrogance, contre le peuple de Yahweh des armées\FTNT{Es. 16:6~; Jé. 48:29.}.
\VS{11}Yahweh sera terrible contre eux, car il anéantira tous les dieux du pays~; et on se prosternera devant lui, chacun de son lieu, même dans toutes les îles des nations\FTNT{Mal. 1:11~; Jn. 4:21.}.
\VS{12}Vous aussi, habitants de Cusch, vous serez blessés à mort par mon épée.
\VS{13}Il étendra aussi sa main sur le nord, et il détruira l'Assyrie, et il fera de Ninive une désolation, dans un lieu aride comme un désert.
\VS{14}Et les troupeaux feront leur gîte au milieu d'elle, et toutes les bêtes des nations, même le pélican et le hérisson, habiteront parmi les chapiteaux de ses colonnes~; la voix des oiseaux retentira à la fenêtre, la désolation sera au seuil, parce qu'il en aura abattu les cèdres\FTNT{Es. 14:23~; Es. 34:11.}.
\VS{15}C'est là cette ville remplie de joie, qui se tenait assurée, et qui disait en son cœur~: C'est moi, et il n'y en a point d'autre que moi~! Comment a-t-elle été réduite en désert, pour être le repère des bêtes~? Quiconque passera près d'elle sifflera et secouera sa main.
\Chap{3}
\TextTitle{Israël persiste dans l'immoralité}
\VerseOne{}Malheur à la ville immonde et souillée et qui ne fait qu'opprimer~!
\VS{2}Elle n'a point écouté la voix, elle n'a point reçu d'instruction, elle ne s'est point confiée en Yahweh, elle ne s'est point approchée de son Dieu.
\VS{3}Ses chefs au milieu d'elle sont des lions rugissants, et ses juges sont des loups du soir, qui ne gardent pas les os pour les ronger le matin\FTNT{Ez. 22:27~; Pr. 28:15.}.
\VS{4}Ses prophètes sont des téméraires, et des hommes infidèles~; ses prêtres ont souillé les choses saintes, ils ont fait violence à la loi\FTNT{Jé. 23:11-32.}.
\VS{5}Yahweh est juste au milieu d'elle, il ne commet point d'iniquité\FTNT{De. 32:4.}. Chaque matin il met en lumière son jugement, il n'y manque pas~; mais celui qui est inique ne sait ce que c'est que d'avoir honte.
\VS{6}J'ai exterminé les nations, et leurs forteresses ont été désolées~; j'ai rendu désertes leurs places, si bien que personne n'y passe~; leurs villes ont été détruites, sans qu'il y soit resté un seul homme, et sans qu'il y ait aucun habitant.
\VS{7}Et je disais~: Au moins tu me craindras, tu recevras instruction, et sa demeure ne sera pas retranchée, quelque soit la punition que je lui envoie. Mais ils se sont levés de bon matin, ils ont corrompu toutes leurs actions.
\VS{8}C'est pourquoi attendez-moi, dit Yahweh, au jour où je me lèverai pour le butin~; car j'ai résolu de rassembler les nations et de réunir les royaumes, pour répandre sur eux mon indignation, et toute l'ardeur de ma colère~; car tout le pays sera dévoré par le feu de ma jalousie.
\TextTitle{Un reste trouve refuge en Yahweh}
\VS{9}Alors je transformerai les langues\FTNT{Il est question ici de la conversion des peuples issus des nations (Ap. 7:9-17).} des nations en des langues pures, afin qu'elles invoquent toutes le Nom de Yahweh, pour qu'elles le servent d'un commun accord.
\VS{10}Mes adorateurs qui sont au-delà des fleuves de Cusch, à savoir la fille de mes dispersés, m'apporteront mes offrandes\FTNT{Es. 19:21~; Es. 27:13~; Ps. 68:31-32~; Ps. 72:10-11.}.
\VS{11}En ce jour-là, tu ne seras plus confuse à cause de toutes tes actions, par lesquelles tu as péché contre moi~; parce qu'alors j'aurai ôté du milieu de toi ceux qui se réjouissent de ton orgueil, et désormais tu ne t'enorgueilliras plus de la montagne de ma sainteté.
\VS{12}Et je laisserai au milieu de toi un peuple humble et faible, et il mettra sa confiance dans le Nom de Yahweh.
\VS{13}Les restes d'Israël ne commettront point d'iniquité, et ne proféreront point de mensonge, et il n'y aura point dans leur bouche de langue trompeuse~; aussi ils paîtront et se reposeront, et il n'y aura personne qui les épouvante.
\TextTitle{Israël délivré et restauré}
\VS{14}Réjouis-toi avec chant de triomphe, fille de Sion~! Pousse des cris de réjouissance, ô Israël~! Réjouis-toi et triomphe de tout ton cœur, fille de Jérusalem~!
\VS{15}Yahweh a aboli ta condamnation, il a éloigné ton ennemi. Le Roi d'Israël, Yahweh, est au milieu de toi~; tu ne verras plus de mal\FTNT{Ps. 46:5-6~; Col. 2:14.}.
\VS{16}En ce temps-là, on dira à Jérusalem~: Ne crains point Sion, que tes mains ne défaillent point~!
\VS{17}Yahweh, ton Dieu, est au milieu de toi comme le Puissant qui sauve~; il se réjouira à cause de toi d'une grande joie~; il se taira à cause de son amour, et se réjouira à cause de toi avec chant de triomphe.
\VS{18}Je rassemblerai ceux qui sont tristes à cause de l'assemblée solennelle, ils sont sortis de toi~; sur eux pèse l'opprobre.
\VS{19}Voici, je détruirai en ce temps-là tous ceux qui t'auront affligé~; je sauverai la boiteuse, je recueillerai celle qui avait été chassée, et je les ferai louer et devenir célèbres, dans tous les pays où ils auront été couverts de honte.
\VS{20}En ce temps-là, je vous ramènerai, et en ce temps-là je vous rassemblerai~; car je vous rendrai célèbres et un sujet de louange parmi tous les peuples de la terre, quand je ramènerai vos captifs sous vos yeux, dit Yahweh.
\PPE{}
\end{multicols}

\clearpage\ShortTitle{Ag.}\BookTitle{Aggée}\BFont
\noindent\hrulefill
{\footnotesize
\textit{
\bigskip
{\centering{}
\\Auteur~: Aggée
\\(Heb.~: Chaggay)
\\Signification~: En fête, né un jour de fête
\\Thème~: Reconstruction du temple
\\Date de rédaction~: 6\up{ème} siècle av. J.-C.\\}
}
\textit{
\\Aggée, contemporain de Zacharie, exerça son service dans le royaume de Juda après le retour de l'exil. Alors que la reconstruction du temple était négligée, Aggée reçut un message rappelant au peuple quelles devaient être ses priorités et redéfinissant les exigences de Yahweh en matière de sainteté. Ce récit montre la bénédiction accompagnant celui qui oublie ses propres intérêts et qui prend véritablement à cœur l'œuvre de Dieu.\bigskip
}
}
\par\nobreak\noindent\hrulefill
\begin{multicols}{2}
\Chap{1}
\TextTitle{Israël coupable de négligence}
\VerseOne{}La seconde année du roi Darius, le premier jour du sixième mois, la parole de Yahweh vint par le moyen d'Aggée, le prophète, à Zorobabel, fils de Schealthiel, gouverneur de Juda, et à Josué, fils de Jotsadak, le grand-prêtre, en ces mots\FTNT{Esd. 4:24.}~:
\VS{2}Ainsi parle Yahweh des armées, en disant~:~Ce peuple dit : Le temps n'est pas encore venu, le temps de rebâtir la maison de Yahweh.
\VS{3}C'est pourquoi la parole de Yahweh a été adressée par le moyen d'Aggée, le prophète, en disant~:
\VS{4}Est-il temps pour vous d'habiter dans vos maisons lambrissées pendant que cette maison est en ruine~?
\VS{5}Maintenant donc, ainsi parle Yahweh des armées~: Considérez attentivement votre conduite~!
\VS{6}Vous avez semé beaucoup, mais vous avez récolté peu. Vous avez mangé, mais non pas jusqu'à être rassasiés. Vous avez bu, mais vous n'avez pas eu de quoi boire abondamment. Vous avez été vêtus, mais non pas jusqu'à en être échauffés. Et celui qui se loue, se loue pour mettre son salaire dans un sac percé\FTNT{Mi. 6:14-15.}.
\VS{7}Ainsi parle Yahweh des armées~: Considérez attentivement vos chemins~!
\VS{8}Montez à la montagne, apportez du bois, et bâtissez cette maison~; et j'y prendrai mon plaisir et je serai glorifié, a dit Yahweh.
\VS{9}Vous comptiez sur beaucoup, et voici, il y a eu peu~; vous l'avez apporté à la maison et j'ai soufflé dessus. Pourquoi~? A cause de ma maison, dit Yahweh des armées, parce qu'elle est en ruine pendant que vous vous empressez chacun pour sa maison.
\VS{10}A cause de cela, les cieux au-dessus de vous retiennent la rosée, et la terre a retenu ses fruits\FTNT{Lé. 26:19~; De. 28:23.}.
\VS{11}Et j'ai appelé la sécheresse sur la terre, et sur les montagnes, et sur le blé, et sur le moût, et sur l'huile, et sur tout ce que la terre produit, et sur les hommes et sur les bêtes, et sur tout le travail des mains\FTNT{Am. 4:7~; Ps. 105:16.}.
\TextTitle{Yahweh réveille son peuple}
\VS{12}Zorobabel donc, fils de Schealthiel, et Josué, fils de Jotsadak, le grand-prêtre, et tout le reste du peuple, entendirent la voix de Yahweh, leur Dieu, et les paroles d'Aggée, le prophète, ainsi que Yahweh, leur Dieu, l'avait envoyé~; et le peuple eut de la crainte devant Yahweh.
\VS{13}Et Aggée, messager de Yahweh, parla au peuple, suivant le message de Yahweh, en disant~: Je suis avec vous, dit Yahweh.
\VS{14}Et Yahweh réveilla l'esprit de Zorobabel, fils de Schealthiel, gouverneur de Juda, et l'esprit de Josué, fils de Jotsadak, le grand-prêtre, et l'esprit de tout le reste du peuple. Et ils vinrent et travaillèrent à la maison de Yahweh, leur Dieu,
\VS{15}le vingt-quatrième jour du sixième mois, de la seconde année du roi Darius.
\Chap{2}
\TextTitle{Encouragements à poursuivre la construction}
\VerseOne{}Le vingt et unième jour du septième mois, la parole de Yahweh vint par le moyen d'Aggée, le prophète, en disant~:
\VS{2}Parle maintenant à Zorobabel, fils de Schealthiel, gouverneur de Juda, et à Josué, fils de Jotsadak, le grand-prêtre, et au reste du peuple, en disant~:
\VS{3}Quel est parmi vous le survivant qui ait vu cette maison dans sa première gloire~? Et comment la voyez-vous maintenant~? N'est-elle pas comme un rien devant vos yeux, au prix de celle-là\FTNT{Esd. 3:12.}~?
\VS{4}Maintenant donc Zorobabel, fortifie-toi~! dit Yahweh. Toi aussi, Josué, fils de Jotsadak, grand-prêtre, fortifie-toi~! Vous aussi, tout le peuple du pays, fortifiez-vous~! dit Yahweh. Et travaillez, car je suis avec vous, dit Yahweh des armées.
\VS{5}La parole de l'Alliance que je traitai avec vous, quand vous sortîtes d'Egypte, et mon Esprit, demeurent au milieu de vous~; ne craignez point\FTNT{Za. 4:6.}~!
\VS{6}Car ainsi parle Yahweh des armées~: Encore un peu de temps, et j'ébranlerai les cieux et la terre, la mer et le sec\FTNT{Hé. 12:26.}~;
\VS{7}j'ébranlerai toutes les nations~; et les trésors de toutes les nations viendront, et je remplirai de gloire cette maison, dit Yahweh des armées.
\VS{8}L'argent est à moi, et l'or est à moi, dit Yahweh des armées.
\VS{9}La gloire de cette dernière maison sera plus grande que celle de la première, dit Yahweh des armées~; et je mettrai la paix en ce lieu, dit Yahweh des armées.
\TextTitle{Purification et sanctification du peuple}
\VS{10}Le vingt-quatrième jour du neuvième mois de la seconde année de Darius, la parole de Yahweh vint par le moyen d'Aggée le prophète, en disant~:
\VS{11}Ainsi parle Yahweh des armées~: Interroge maintenant les prêtres sur la loi en ces mots~:
\VS{12}Si quelqu'un porte de la chair consacrée dans le pan de son vêtement, et que ce vêtement touche du pain, ou un mets cuit, ou du vin, ou de l'huile, ou un aliment quelconque, cela devient-il sanctifié~? Et les prêtres répondirent et dirent~: Non~!
\VS{13}Alors Aggée dit~: Si celui qui est souillé pour un mort touche toutes ces choses-là, ne seront-elles pas souillées~? Et les prêtres répondirent et dirent~: Elles seront souillées\FTNT{Lé. 17:15~; No. 19:22~; Tit. 1:15.}.
\VS{14}Alors Aggée répondit et dit~: Tel est ce peuple et telle est cette nation devant ma face, dit Yahweh~; et telles sont toutes les œuvres de leurs mains~; même ce qu'ils offrent ici est souillé.
\VS{15}Maintenant donc mettez ceci, je vous prie, dans votre cœur, depuis ce jour et par la suite, avant qu'on ait mis pierre sur pierre au temple de Yahweh~!
\VS{16}Avant cela, dis-je, quand on venait à un monceau de blé, au lieu de vingt mesures, il n'y en avait que dix~; et quand on  venait au pressoir, au lieu de puiser de la cuve cinquante mesures, il n'y en avait que vingt.
\VS{17}Je vous ai frappés de brûlure, de rouille, de grêle, dans tout le travail de vos mains. Et vous n'êtes point revenus à moi, dit Yahweh\FTNT{De. 28:22~; 1 R. 8:37~; Am. 4:9~; 2 Ch. 6:28.}.
\VS{18}Mettez maintenant ceci dans votre cœur~; depuis ce jour-ci et dans la suite~; depuis, dis-je, le vingt-quatrième jour du neuvième mois, depuis le jour où les fondements du temple de Yahweh ont été posés, mettez ceci dans votre cœur~!
\VS{19}Ya-t-il encore de la semence dans les greniers~? Même jusqu'à la vigne, au figuier, au grenadier, et à l'olivier, rien n'a rapporté~; mais depuis ce jour-ci, je donnerai la bénédiction.
\TextTitle{Destruction des royaumes des nations}
\VS{20}Et la parole de Yahweh vint pour la seconde fois à Aggée, le vingt-quatrième jour du mois, en disant~:
\VS{21}Parle à Zorobabel, gouverneur de Juda, et dis-lui~: J'ébranlerai les cieux et la terre~;
\VS{22}je renverserai le trône des royaumes, je détruirai la force des royaumes des nations, je renverserai les chars et ceux qui les montent~; et les chevaux et ceux qui les montent seront abattus, chacun par l'épée de son frère.
\VS{23}En ce jour-là, dit Yahweh des armées, je te prendrai, ô Zorobabel, fils de Schealthiel, mon serviteur, dit Yahweh~; et je te mettrai comme un sceau\FTNT{Le sceau est un symbole d'autorité.}, car je t'ai choisi, dit Yahweh des armées.
\PPE{}
\end{multicols}

\clearpage\ShortTitle{Zacharie}\BookTitle{Zacharie}\BFont
\noindent\hrulefill
{\footnotesize
\textit{
\bigskip
{\centering{}
\\Signifie : Yahweh se souvient
\\Thème : Les deux avènements du Messie
\\Auteur : Zacharie
\\Date de rédaction : 6ème siècle av. J.-C.\\}
}
%\bigskip
\textit{
\\Zacharie, contemporain d’Aggée, exerça son ministère en Juda au retour des exilés de Babylone, où il était né. Il annonça la venue du Messie et raconta de manière très précise différents épisodes de sa vie, également retrouvés dans le récit des évangiles. Il dévoila également quelques-uns des attributs du Sauveur, premièrement rejeté mais finalement accepté par le peuple juif pendant le millenium. On y découvre ainsi le Christ en tant que souverain sacrificateur, germe, serviteur, ange de l’Yahweh, roi de paix, fils de David…\bigskip
}
}
\par\nobreak\noindent\hrulefill
\begin{multicols}{2}
\Chap{1}
\TextTitle{Yahweh avertit son peuple}
\VerseOne{}Le huitième mois de la deuxième année de Darius, la parole de Yahweh fut adressée à Zacharie, le prophète, fils de Bérékia, fils d’Iddo, en ces mots :
\VS{2}Yahweh a été extrêmement irrité contre vos pères.
\VS{3}C'est pourquoi tu leur diras : Ainsi parle Yahweh des armées : Revenez à moi, dit Yahweh des armées, et je reviendrai à vous, dit Yahweh des armées\FTNT{Joë. 2:12 ; Es. 31:6 ; Jé. 3:12.}.
\VS{4}Ne soyez point comme vos pères, auxquels s’adressaient les premiers prophètes, en disant : Ainsi a dit Yahweh des armées : Détournez-vous maintenant de vos mauvaises voies et de vos mauvaises actions ! Mais ils n’écoutèrent pas, ils ne furent pas attentifs à ce que je leur disais, dit Yahweh\FTNT{2 Ch. 29:6 ; Esd. 9:7 ; Né. 9:16 ; La. 5:7.}.
\VS{5}Vos pères où sont-ils ? Et ces prophètes-là pouvaient-ils vivre éternellement ?
\VS{6}Cependant mes paroles et mes ordonnances que j'avais données aux prophètes, mes serviteurs, n'ont-elles pas atteint vos pères ? De sorte qu’étant revenus, ils ont dit : Yahweh des armées nous a traités comme il avait résolu de le faire, selon nos voies et nos actions.
\TextTitle{Le cavalier sur le cheval roux}
\VS{7}Le vingt-quatrième jour du onzième mois, qui est le mois de Schebat, la deuxième année de Darius, la parole de Yahweh fut adressée à Zacharie, le prophète, fils de Bérékia, fils d’Iddo, en ces mots :
\VS{8}Je voyais de nuit une vision, et voici, un homme était monté sur un cheval roux, et il se tenait parmi des myrtes qui étaient dans un lieu creux ; il y avait derrière lui des chevaux roux, fauves et blancs\FTNT{Ap. 6:2-4.}.
\VS{9}Je dis : Mon Seigneur ! Que signifient ces choses ? Et l’Ange qui me parlait me dit : Je te montrerai ce que signifient ces choses.
\VS{10}L’homme qui se tenait parmi les myrtes répondit et dit : Ce sont ceux que Yahweh a envoyés pour parcourir la terre.
\VS{11}Et ils répondirent à l'Ange de Yahweh\FTNT{Voir commentaire en Ge. 16:7.} qui se tenait parmi les myrtes, et dirent : Nous avons parcouru la terre ; et voici, toute la terre est en repos et tranquille.
\TextTitle{La compassion de Yahweh pour Jérusalem}
\VS{12}Alors l'Ange de Yahweh répondit et dit : Yahweh des armées, jusqu'à quand n'auras-tu pas compassion de Jérusalem et des villes de Juda, contre lesquelles tu es irrité depuis soixante-dix ans\FTNT{Jérémie prophétisa que la captivité babylonienne durerait soixante-dix ans (Jé. 25:11-12 ; Jé. 29:10). Les soixante-dix ans commencèrent à la déportation de la famille royale à Babylone en 605 av. J.-C. (2 R. 24 ; Da. 1) et se terminèrent avec la première vague de retours conduite par Zorobabel (Esd. 1). Les Israélites furent emmenés en captivité en plusieurs vagues. Le livre d’Esdras raconte les deux premières. En 538 av. J.-C., Zorobabel mena la première vague et fut nommé gouverneur (Ag. 1:1). Le sacrificateur Josué  (Esd. 3:2) et les prophètes Aggée et Zacharie (Es. 5:1-2) le secondaient. Leur plus grand défi fut de rebâtir le temple. Puisque la seule tribu à retourner en masse fut celle de Juda, dès lors, le reste du peuple fut appelé «~les Juifs~» (Esd. 4:23).} ?
\VS{13}Yahweh répondit à l'Ange qui me parlait, par de bonnes paroles, par des paroles de consolation.
\VS{14}Puis l'Ange qui me parlait me dit : Crie, en disant : Ainsi parle Yahweh des armées : Je suis ému d'une grande jalousie pour Jérusalem et pour Sion,
\VS{15}et je suis extrêmement irrité contre les nations qui sont à leur aise ; car je n’étais que peu irrité, mais elles ont contribué au mal.
\VS{16}C'est pourquoi ainsi parle Yahweh : Je reviens à Jérusalem avec compassion, et ma maison y sera rebâtie, dit Yahweh des armées ; et le cordeau sera étendu sur Jérusalem.
\VS{17}Crie encore, et dis : Ainsi parle Yahweh des armées : Mes villes regorgeront encore de biens, et Yahweh consolera encore Sion, il choisira encore Jérusalem.
\TextTitle{Les quatre cornes et les quatre forgerons}
\VS{18}Puis je levai les yeux et je regardai ; et voici, quatre cornes\FTNT{Da. 7:7-11 ; Da. 8:22 ; Ap. 13:1-11.}.
\VS{19}Alors je dis à l'Ange qui me parlait : Que veulent dire ces choses ? Et il me répondit : Ce sont les cornes qui ont dispersé Juda, Israël et Jérusalem.
\VS{20}Puis Yahweh me fit voir quatre forgerons.
\VS{21}Je dis : Que viennent-ils faire ? Et il répondit et dit : Ce sont les cornes qui ont dispersé Juda, au point que personne ne lève la tête ; et ces forgerons sont venus pour les effrayer, et pour abattre les cornes des nations qui ont levé la corne contre le pays de Juda, pour le disperser.
\Chap{2}
\TextTitle{L'homme tenant dans sa main le cordeau pour mésurer}
\VerseOne{}Je levai encore mes yeux et je regardai, et voici, il y avait un homme tenant dans la main un cordeau pour mesurer,
\VS{2}auquel je dis : Où vas-tu ? Et il me répondit : Je vais mesurer Jérusalem, pour voir quelle est sa largeur et quelle est sa longueur.
\VS{3}Et voici, l'Ange qui me parlait s’avança, et un autre ange sortit à sa rencontre.
\TextTitle{Yahweh, la gloire de Jérusalem}
\VS{4}Il lui dit : Cours, et parle à ce jeune homme, et dis : Jérusalem sera habitée comme les villes sans murailles, à cause de la multitude d'hommes et de bêtes qui seront au milieu d'elle\FTNT{Né. 1:3 ; Né. 2:13.}.
\VS{5}Mais je serai pour elle, dit Yahweh, une muraille de feu tout autour, et je serai sa gloire au milieu d'elle\FTNT{Es. 60:19.}.
\VS{6}Ha ! Fuyez, fuyez hors du pays du nord ! dit Yahweh. Car je vous ai dispersés aux quatre vents des cieux, dit Yahweh.
\VS{7}Ha ! Sauve-toi, Sion, toi qui habites chez la fille de Babylone\FTNT{Jé. 50:8 ; Es. 48:20 ; Es. 52:11 ; Jé. 51:6.} !
\VS{8}Car ainsi parle Yahweh des armées, lequel après la gloire, il m'a envoyé vers les nations qui ont fait de vous leur proie ; car celui qui vous touche, touche à la prunelle de son œil\FTNT{De. 32:10 ; Ps. 17:8.}.
\VS{9}Car voici, je vais lever ma main contre elles, et elles seront la proie de ceux qui leur étaient asservis. Et vous saurez que Yahweh des armées m'a envoyé.
\VS{10}Pousse des cris d’allégresse et réjouis-toi, fille de Sion ! Car voici, je viens\FTNT{Jésus-Christ est Yahweh qui vient ! C’est la seconde venue de Jésus-Christ qui est évoquée ici (Es. 40:10-11 ; Za. 12:10-14 ; Za. 14:1-10 ; Ac. 1:1-11 ; Ap. 1:7-8 ; Ap. 22:12-17).}, et j'habiterai au milieu de toi, dit Yahweh.
\VS{11}Beaucoup de nations se joindront à Yahweh en ce jour-là, et deviendront mon peuple ; et j'habiterai au milieu de toi ; et tu sauras que Yahweh des armées m'a envoyé vers toi.
\VS{12}Yahweh possédera Juda comme sa part dans la terre sainte, et il choisira encore Jérusalem.
\VS{13}Que toute chair fasse silence devant la face de Yahweh ! Car il s'est réveillé de sa demeure sainte.
\Chap{3}
\TextTitle{Yahweh enlève l'iniquité du pays}
\VerseOne{}Puis Yahweh me fit voir Josué, le souverain sacrificateur, se tenant debout devant l'Ange de Yahweh\FTNT{Voir commentaire en Ge. 16:7.}, et Satan qui se tenait debout à sa droite, pour l’accuser.
\VS{2}Yahweh dit à Satan : Que Yahweh te réprime, ô Satan ! Que Yahweh, dis-je, qui a choisi Jérusalem, te réprime ! N’est-ce pas là un tison qui a été retiré du feu\FTNT{Jud. 1:9 ; Am. 4:11.} ?
\VS{3}Or Josué était vêtu de vêtements sales, et il se tenait debout devant l'Ange.
\VS{4}L’Ange prit la parole et dit à ceux qui étaient debout devant lui : Otez-lui ces vêtements sales ! Et il dit à Josué : Regarde, je t’enlève ton iniquité, et je te revêts d’habits de fête.
\VS{5}Je dis : Qu'on mette sur sa tête un turban pur ! Et ils mirent un turban pur sur sa tête, puis ils lui mirent des vêtements\FTNT{Ap. 19:8.}. L’Ange de Yahweh était présent.
\VS{6}Alors l'Ange de Yahweh fit à Josué cette déclaration, en disant :
\VS{7}Ainsi parle Yahweh des armées : Si tu marches dans mes voies, et si tu observes mes commandements, tu jugeras ma maison, tu garderas mes parvis, et je te donnerai libre accès parmi ceux qui se tiennent devant moi.
\VS{8}Ecoute maintenant, Josué, souverain sacrificateur, toi, et tes compagnons qui sont assis devant toi ! Car ce sont des hommes qui serviront de signes. Certainement voici je ferai venir mon serviteur, le Germe\FTNT{Le Germe est un autre nom de Jésus-Christ, notre Seigneur (Es. 4:2).}.
\VS{9}Car voici, quant à la pierre\FTNT{Jésus-Christ est le Rocher des âges (Es. 8:13-17; Ap. 5:1-7).} que j'ai mise devant Josué, sur cette pierre, qui n'est qu'une\FTNT{Cette pierre est UNE (~E’had~) c’est-à-dire indivisible (De. 6:4).}, il y a sept yeux. Voici, je graverai moi-même ce qui doit y être gravé, dit Yahweh des armées ; et j'ôterai en un jour l'iniquité de ce pays.
\VS{10}En ce jour-là, dit Yahweh des armées, chacun de vous appellera son prochain sous la vigne et sous le figuier.
\Chap{4}
\TextTitle{Le peuple de Yahweh peut tout par son Esprit}
\VerseOne{}Puis l'Ange qui me parlait revint, et il me réveilla comme un homme que l’on réveille de son sommeil.
\VS{2}Il me dit : Que vois-tu ? Et je répondis : Je regarde, et voici, il y a un chandelier tout en or, surmonté d’un vase et portant ses sept lampes, avec sept conduits pour les sept lampes qui sont au sommet du chandelier\FTNT{Ap. 1:12-13.} ;
\VS{3}et il y a deux oliviers près de lui, l'un à la droite du vase, et l'autre à sa gauche.
\VS{4}Alors je pris la parole et je dis à l'Ange qui me parlait : Mon Seigneur, que signifient ces choses ?
\VS{5}L'Ange qui me parlait répondit et me dit : Ne sais-tu pas ce que signifient ces choses ? Je dis : Non, mon Seigneur !
\VS{6}Alors il reprit et me dit : C'est ici la parole que Yahweh adresse à Zorobabel : Ce n'est point par la puissance ni par la force, mais par mon Esprit, dit Yahweh des armées.
\VS{7}Qui es-tu, grande montagne, devant Zorobabel ? Tu seras aplanie. Il fera sortir la pierre principale ; il y aura des sons éclatants : Grâce, grâce pour elle !
\TextTitle{Yahweh encourage son peuple à achever l'oeuvre commencée}
\VS{8}Aussi la parole de Yahweh me fut adressée en ces mots :
\VS{9}Les mains de Zorobabel ont fondé cette maison, et ses mains l'achèveront ; et tu sauras que Yahweh des armées m'a envoyé vers vous.
\VS{10}Car qui est-ce qui a méprisé le jour des faibles commencements ? Ils se réjouiront en voyant le niveau dans la main de Zorobabel.  Ces sept\FTNT{Les sept yeux de Yahweh sont aussi les sept yeux de l’Agneau (Ap. 5 : 6). Ces yeux représentent l’omniscience et l’omniprésence de Jésus-Christ (Za. 3:9 ; Za 4:10. ; Za. 14:7 ; Jn. 16:30 ; Ac. 1:24 ; Ap. 21:17).} sont les yeux de Yahweh qui parcourent toute la terre.
\VS{11}Je pris la parole et je lui dis : Que signifient ces deux oliviers\FTNT{L’identité de ces deux individus est inconnue.  Selon Ap. 11:3, ces deux hommes recevront des pouvoirs incroyables pour les trois années et demi de la grande tribulation qui précéderont le retour du Christ. Si quiconque tente de leur faire du mal ou d’interférer dans leur ministère et leur témoignage, «~… du feu sortirait de leur bouche et dévorerait leurs ennemis~», (Ap. 11:5). Ils auront aussi le pouvoir de provoquer la sécheresse et la famine sur la terre, tout comme l’avait fait Elie (1 R. 17:1-7 ; 2 R. 1:9-15 ; Lu. 4:25). Ils auront également le pouvoir de frapper la Terre par des plaies diverses, semblables à celles provoquées par Moïse (Chapitres 7, 8, 9, 10, 11 d’Exode ; Ap. 11:6).}, à la droite et à la gauche du chandelier ?
\VS{12}Je pris la parole pour la seconde fois et je lui dis : Que signifient ces deux branches d'olivier qui sont près des deux conduits d'or, d’où l'or découle ?
\VS{13}Il me répondit et dit : Ne sais-tu pas ce que signifient ces choses ? Et je dis : Non, mon Seigneur.
\VS{14}Et il dit : Ce sont les  deux fils oints, qui se tiennent devant le Seigneur de toute la terre.
\Chap{5}
\TextTitle{La malédiction se répands sur Israël}
\VerseOne{}Puis je me retournai, et levai mes yeux pour regarder ; et voici, un rouleau qui volait.
\VS{2}Alors il me dit : Que vois-tu ? Je répondis : Je vois un rouleau qui vole, dont la longueur est de vingt coudées, et la largeur de dix coudées.
\VS{3}Et il me dit : C'est l’exécration du serment qui sort sur la face de tout le pays ; car selon elle, quiconque d'entre ce peuple-ci vole, sera puni comme elle ; et selon elle, quiconque d'entre ce peuple parjure, sera puni comme elle.
\VS{4}Je déploierai cette exécration dit Yahweh des armées, et elle entrera dans la maison du voleur, et dans la maison de celui qui jure faussement en mon Nom, et elle logera au milieu de leur maison, et la consumera avec son bois et ses pierres.
\TextTitle{L'épha au pays de Schinear}
\VS{5}L’Ange qui me parlait sortit, et me dit : Lève maintenant tes yeux, et regarde ce qui sort là.
\VS{6}Et je dis : Qu'est-ce ? Et il répondit : C'est l’épha\FTNT{L'épha était une unité de mesure utilisée dans le commerce des céréales, souvent à des fins frauduleuses (De. 25:14 ; Mi. 6:10 ; Am. 8:5).} qui sort dehors. Puis il dit : C'est ici leur aspect dans tout le pays.
\VS{7}Et voici, on portait une masse de plomb, et une femme était assise au milieu de l'épha\FTNT{Zacharie voit une femme assise au milieu de l'épha. L'ange déclare : «~C'est la méchanceté ou l’iniquité~». Elle représente la grande prostituée décrite en Ap. 17, avec sa coupe d'or pleine de ses abominations et des impuretés de sa fornication (v. 4). Cette femme est la figure du «~mystère de l'iniquité qui opère déjà~» (2 Th. 2:7).}.
\VS{8}Il dit : C'est là l’iniquité\FTNT{L’iniquité ou la méchanceté.} ; puis il la repoussa dans l'épha, et il jeta la masse de plomb sur l’ouverture.
\VS{9}Je levai les yeux et je regardai, et voici deux femmes sortirent\FTNT{Les deux femmes ayant «~des ailes de cigogne~» apparaissent portées par le vent. Sous Moïse, cet oiseau devait être considéré comme impur (Lé. 11 : 19). Dans les Ecritures, le vent est constamment en relation avec le jugement (Job. 27:20-22 ; Job. 30:22 ; Es. 7:2 ; Es. 26:6 ; Es. 41:16). Elles soulèvent l'épha et l'emportent dans son lieu d'origine, le pays de Schinear, c’est-à-dire Babylone, pour lui bâtir une maison, au siège même de l'idolâtrie et de la révolte contre Dieu. (Ge. 11:2-9 ; 2 R. 17:24).}. Le vent soufflait dans leurs ailes : Elles avaient des ailes comme les ailes de la cigogne. Et elles enlevèrent l'épha entre la terre et le ciel.
\VS{10}Je dis à l'Ange qui me  parlait : Où emportent-elles l'épha ?
\VS{11}Il me répondit : C'est pour lui bâtir une maison dans le pays de Schinear\FTNT{Schinear ou Babylone (Ge. 10:6-12).} ; et quand elle sera prête, il sera déposé là, sur sa base.
\Chap{6}
\TextTitle{Les quatre vents des cieux}
\VerseOne{}Je levai encore les yeux et je regardai, et voici quatre chars\FTNT{Dans les Ecritures, les chars et les chevaux représentent souvent la puissance de Dieu exerçant un jugement sur la terre (Jé. 46:9-10 ; Joë. 2:3-11). Ce jugement concerne le monde entier (Ap. 6:1-8).} sortaient d'entre deux montagnes ; et ces montagnes étaient des montagnes d'airain.
\VS{2}Au premier char, il y avait des chevaux roux ; au deuxième char, des chevaux noirs,
\VS{3}au troisième char, des chevaux blancs, et au quatrième char, des chevaux tachetés, rouges.
\VS{4}Je pris la parole et je dis à l'Ange qui me parlait : Mon Seigneur, que veulent dire ces choses ?
\VS{5}L’Ange répondit et me dit : Ce sont les quatre vents des cieux, qui sortent du lieu où ils se tenaient devant le Seigneur de toute la terre.
\VS{6}Quant au char où sont les chevaux noirs, ils se dirigent vers le pays du nord, et les blancs sortent après eux ; les tachetés se dirigent vers le pays du midi.
\VS{7}Ensuite les rouges sortirent et demandèrent à aller parcourir la terre. L’Ange leur dit : Allez, et parcourez la terre ! Et ils parcoururent la terre.
\VS{8}Puis il m'appela, et me parla, en disant : Voici, ceux qui se dirigent vers le pays du nord ont apaisé mon Esprit dans le pays du nord.
\TextTitle{Prophétie sur le règne du germe de Yahweh}
\VS{9}La parole de Yahweh me fut adressée en ces mots :
\VS{10}Tu recevras les dons de ceux qui sont de retour de la captivité : Heldaï, Tobija et Jedaeja. Et tu iras toi-même ce même jour-là, et tu iras dans la maison de Josias, fils de Sophonie, où ils se sont rendus en arrivant de Babylone.
\VS{11}Tu prendras de l'argent et de l'or, et tu en feras des couronnes que tu mettras sur la tête de Josué, fils de Jotsadak, le souverain sacrificateur.
\VS{12}Tu lui diras : Ainsi parle Yahweh des armées : Voici un homme, dont le nom est Germe\FTNT{Es. 4:2.}, germera dans son lieu, et bâtira le temple de Yahweh\FTNT{C’est Yahweh, c’est-à-dire Jésus-Christ lui-même, qui bâtit son temple (Ps. 127:1-2 ; Mt. 16:18).}.
\VS{13}Oui, lui-même bâtira le temple de Yahweh ; et lui-même sera rempli de majesté. Il s’assiéra et dominera sur son trône, il sera Sacrificateur\FTNT{Jésus-Christ est Souverain Sacrificateur (Hé. 6:20 ; Hé. 7:1-28).}, étant sur son trône ; et il y aura un conseil de paix entre les deux.
\VS{14}Les couronnes seront pour Hélem, Tobija et Jedaeja, et pour Hen, fils de Sophonie, un souvenir dans le temple de Yahweh.
\VS{15}Ceux qui sont éloignés viendront, et travailleront au temple de Yahweh ; et vous saurez que Yahweh des armées m'a envoyé vers vous. Cela arrivera, si vous écoutez attentivement la voix de Yahweh, votre Dieu.
\Chap{7}
\TextTitle{Yahweh dénonce le jeûne formaliste}
\VerseOne{}La quatrième année du roi Darius, la parole de Yahweh fut adressée à Zacharie, le quatrième jour du neuvième mois, qui est le mois de Kisleu.
\VS{2}On avait envoyé à Béthel Scharetser et Réguem-Mélec avec ses gens, pour supplier Yahweh,
\VS{3}et pour parler aux sacrificateurs de la maison de Yahweh des armées, et aux prophètes, en disant : Dois-je pleurer au cinquième mois, et faire abstinence, comme j'ai déjà fait pendant plusieurs années ?
\VS{4}La parole de Yahweh des armées me fut adressée en ces mots :
\VS{5}Parle à tout le peuple du pays et aux sacrificateurs, et dis-leur : Quand vous avez jeûné et pleuré au cinquième mois et au septième, et cela depuis soixante-dix ans, avez-vous célébré ce jeûne par amour pour moi ?
\VS{6}Et quand vous buvez et mangez, n'est-ce pas vous qui mangez et vous qui buvez\FTNT{Es. 58:3-4.} ?
\VS{7}Ne connaissez-vous pas les paroles qu’a proclamées Yahweh par les premiers prophètes, lorsque Jérusalem était habitée et paisible avec ses villes à l’entour, et que le midi et la plaine étaient habités ?
\TextTitle{Yahweh n'exauce pas les pécheurs}
\VS{8}Puis la parole de Yahweh fut adressée à Zacharie en ces mots :
\VS{9}Ainsi parlait Yahweh des armées, en disant : Rendez véritablement la justice, et exercez la miséricorde et la compassion chacun envers son frère.
\VS{10}N’opprimez pas la veuve et l'orphelin, l'étranger et le pauvre, et ne méditez aucun mal dans vos cœurs chacun contre son frère\FTNT{Ex. 22:21 ; Es. 1:23 ; Jé. 5:28 ; Pr. 22:22-23.}.
\VS{11}Mais ils refusèrent d’être attentifs, ils eurent l'épaule rebelle, et ils endurcirent leurs oreilles pour ne pas entendre.
\VS{12}Ils rendirent leur cœur dur comme le diamant, pour ne pas écouter la loi et les paroles que Yahweh des armées adressait par son Esprit, par les premiers prophètes. C’est pourquoi  Yahweh des armées s’enflamma d’une grande colère.
\VS{13}Quand il appelait, ils n'ont pas écouté. Aussi n’ai-je pas écouté, quand ils ont appelé, dit Yahweh des armées\FTNT{Pr. 1:28 ; Es. 1:15 ; Jé. 11:11.}.
\VS{14}Je les ai dispersés comme par un tourbillon parmi toutes les nations qu'ils ne connaissaient pas ; le pays a été dévasté derrière eux, il n’y a plus eu ni allants ni venants ; et d’un pays de délices ils ont fait un désert.
\Chap{8}
\TextTitle{Futur royaume d'Israël rétabli dans la justice}
\VerseOne{}La parole de Yahweh des armées me fut encore adressée en ces mots :
\VS{2}Ainsi parle Yahweh des armées : Je suis jaloux pour Sion d'une grande jalousie, et je suis jaloux pour elle d’une grande fureur.
\VS{3}Ainsi parle Yahweh : Je retourne à Sion, et j'habiterai au milieu de Jérusalem ; et Jérusalem sera appelée ville fidèle ; et la montagne de Yahweh des armées sera appelée montagne sainte\FTNT{Es. 1:26.}.
\VS{4}Ainsi parle Yahweh des armées : Il y aura encore des vieillards et des femmes âgées, assis dans les rues de Jérusalem, et chacun aura son bâton à la main, à cause du grand nombre de leurs jours.
\VS{5}Les rues de la ville seront remplies de fils et de filles, jouant dans les rues.
\VS{6}Ainsi parle Yahweh des armées : S’il semble difficile aux yeux du reste de ce peuple que cela arrive, en ces jours-là, sera-t-il de même difficile à mes yeux ? dit Yahweh des armées.
\VS{7}Ainsi parle Yahweh des armées : Voici, je délivre mon peuple du pays de l'orient et du pays du soleil couchant.
\VS{8}Je les ramènerai, et ils habiteront au milieu de Jérusalem ; ils seront mon peuple, et je serai leur Dieu avec vérité et droiture.
\TextTitle{Juger selon la vérité}
\VS{9}Ainsi parle Yahweh des armées : Que vos mains soient fortifiées, vous qui entendez aujourd’hui ces paroles de la bouche des prophètes qui parurent au jour où la maison de Yahweh fut fondée, et où le temple allait être bâti\FTNT{Ag. 2:4.}.
\VS{10}Car avant ces jours-là, il n'y avait pas de salaire pour l'homme ni de salaire pour la bête ; et il n'y avait pas de paix pour ceux qui entraient et sortaient, à cause de la détresse ; et je lâchais tous les hommes les uns contre les autres.
\VS{11}Mais maintenant je ne serai pas pour le reste de ce peuple comme les premiers jours,  dit Yahweh des armées.
\VS{12}Car les semailles prospéreront, la semence de paix sera là ; la vigne rendra son fruit, et la terre donnera ses produits ; les cieux donneront leur rosée, et je ferai hériter toutes ces choses au reste de ce peuple.
\VS{13}De même que vous avez été en malédiction parmi les nations, ô maison de Juda et maison d'Israël, de même je vous délivrerai, et vous serez en bénédiction. Ne craignez pas, mais que vos mains soient fortifiées\FTNT{Ge. 1:11 ; Ap. 22:2.}.
\VS{14}Car ainsi parle Yahweh des armées : Comme j'ai eu la pensée de vous affliger, lorsque vos pères ont provoqué ma colère, dit Yahweh des armées, et que je ne m'en suis point repenti,
\VS{15}ainsi je reviens en arrière et j’ai résolu en ces jours de faire du bien à Jérusalem, et à la maison de Juda. Ne craignez pas !
\VS{16}Voici les choses que vous devez faire : Que chacun dise la vérité à son prochain ; jugez selon la vérité et prononcez un jugement en vue de la paix dans vos portes\FTNT{Ep. 4:25 ; Ex. 20:16 ; Mt. 19:18 ; Lu. 18:20.} ;
\VS{17}que personne ne projette du mal dans son cœur contre son prochain ; et n'aimez point le faux serment, car ce sont là des choses que je hais, dit Yahweh\FTNT{Ps. 5:5 ; Ps. 11:5 ; Pr. 6:16-19.}.
\VS{18}Puis la parole de Yahweh des armées me fut adressée en ces mots :
\VS{19}Ainsi parle Yahweh des armées : Le jeûne du quatrième mois, le jeûne du cinquième, le jeûne du septième et le jeûne du dixième seront changés pour la maison de Juda en joie et en allégresse, et en fêtes solennelles de réjouissance. Aimez donc la vérité et la paix\FTNT{Ep. 4:15.}.
\TextTitle{Les nations reconnaissent que Yahweh est le seul Dieu}
\VS{20}Ainsi parle Yahweh des armées : Il viendra encore des peuples et des habitants de plusieurs villes.
\VS{21}Les habitants d’une ville  iront à l'autre, en disant : Allons, allons implorer Yahweh et chercher Yahweh des armées ! Nous irons aussi !
\VS{22}Et beaucoup de peuples et de puissantes nations viendront rechercher Yahweh des armées à Jérusalem\FTNT{Jérusalem est appelée à devenir le centre d’adoration de la terre et la capitale du monde à cause de la présence de Dieu (Es. 66:23 ; Za. 14:16-21).}, et implorer Yahweh.
\VS{23}Ainsi parle Yahweh des armées : En ce jour-là, dix hommes de toutes les langues des nations saisiront le pan de la robe d'un homme Juif, et diront : Nous irons avec vous, car nous avons entendu que Dieu est avec vous.
\Chap{9}
\TextTitle{Le jugement de Yahweh sur les nations}
\VerseOne{}Oracle, parole de Yahweh sur le pays de Hadrac. Elle s’arrête sur Damas, car Yahweh a l’œil sur les hommes et sur toutes les tribus d'Israël.
\VS{2}Il s’arrête aussi sur Hamath, à la frontière de Damas, sur Tyr, et Sidon, quoique chacune d'elles soit fort sage.
\VS{3}Car Tyr s'est bâti une forteresse ; elle a amassé l'argent comme la poussière, et l’or fin comme la boue des rues\FTNT{Ez. 28:3-17.}.
\VS{4}Voici, le Seigneur l'appauvrira, et en la frappant, il jettera sa puissance dans la mer, et elle sera consumée par le feu\FTNT{Ez. 26:3-4.}.
\VS{5}Askalon le verra, et elle sera dans la crainte ; Gaza aussi le verra, et un violent tremblement la saisira ; Ekron aussi, car son espoir sera confondu. Et il n'y aura plus de roi à Gaza, et Askelon ne sera plus habitée\FTNT{So. 2:4.}.
\VS{6}Et le bâtard habitera à Asdod ; et j’abattrai l'orgueil des Philistins.
\VS{7}J'ôterai le sang de la bouche de chacun d'eux, et leurs abominations d'entre leurs dents ; et lui aussi restera pour notre Dieu, il sera comme un chef en Juda, et Ekron sera comme le Jébusien.
\VS{8}Je camperai autour de ma maison, pour la défendre contre une armée, contre les allants et les venants, et l’oppresseur ne passera plus près d’eux ; car maintenant mes yeux sont fixés sur elle.
\TextTitle{Prophétie sur la première venue du Messie}
\VS{9}Sois transportée d’allégresse, fille de Sion ! Pousse des cris de joie, fille de Jérusalem ! Voici, ton Roi vient à toi ; il est juste et vainqueur, il est monté sur un âne, sur un âne, le petit d'une ânesse\FTNT{Cette prophétie s’est accomplie 500 ans après. Jésus est effectivement entré à Jérusalem monté sur un âne (Mt. 21:1-11 ; Lu. 19:28-40 ; Jn. 12:12-19).}.
\TextTitle{La vision du Messie pour Israël}
\VS{10}Je détruirai les chars d'Ephraïm, et les chevaux de Jérusalem ; et les arcs de guerre seront aussi retranchés.  Et le Roi parlera de paix aux nations ; et sa domination s'étendra d’une mer à l’autre, depuis le fleuve jusqu'aux extrémités de la terre\FTNT{Es. 57:19 ; Ps. 2:8 ; Ps. 72:8.}.
\VS{11}Quant à toi, à cause de ton alliance scellée par le sang, je retirerai tes captifs de la fosse où il n'y a pas d'eau.
\VS{12}Retournez à la forteresse, captifs pleins d’espérance ! Aujourd'hui même je le déclare, je te rendrai le double.
\VS{13}Car je bande Juda comme un arc, je m’arme d’Ephraïm comme d’un arc, et j’exciterai tes enfants, ô Sion, contre tes enfants, ô Javan ! Je te rendrai pareille à l’épée d’un vaillant homme.
\VS{14}Alors Yahweh au-dessus d’eux apparaîtra, et ses dards partiront comme l'éclair, et le Seigneur, Yahweh, sonnera du shofar, il s’avancera dans le tourbillon du midi.
\VS{15}Yahweh des armées sera leur protecteur ; ils dévoreront, après avoir subjugué ceux qui tirent les pierres de fronde ; ils boiront, ils seront bruyants comme des hommes ivres, ils se rempliront de vin comme un bassin, et comme les coins de l'autel.
\VS{16}Yahweh, leur Dieu, les sauvera en ce jour-là, comme le troupeau de son peuple ; car ils sont les pierres d’une couronne  qui brilleront dans son pays.
\VS{17}Car combien est grande sa bonté ! Quelle beauté ! Le froment fera croître les jeunes hommes, et le vin doux rendra ses vierges éloquentes.
\Chap{10}
\TextTitle{Yahweh rassemblera son peuple}
\VerseOne{}Demandez à Yahweh la pluie\FTNT{Les pluies en Israël : En Israël, la saison des pluies commence généralement vers la fin du mois d’octobre avec de légères pluies qui ramollissent la terre (Ps. 65:10), et se poursuit ensuite par de fortes précipitations intermittentes durant deux ou trois jours, tout au long des mois de novembre et de décembre. Ces fortes précipitations étaient appelées dans les écritures «~la pluie de la première saison~» (en hébreu «~yoreb~» ou «~moreh~). Les fermiers dépendaient de la pluie de la première saison pour que la terre dure comme le roc soit rendue apte au labour et à l’ensemencement. Quand ces fortes précipitations s’achèvent, des pluies plus fines continuent encore de façon intermittente. Toutefois, à l’approche de la moisson,  la forte pluie revenait gonfler le grain et le fruit en préparation. Celle-ci était connue comme étant «~la pluie de l’arrière-saison~» (Jé. 5:24; Joë. 2:23-24 ; Os. 6:3).}, la pluie au temps de la dernière saison ! Yahweh produira des éclairs, et il vous donnera une abondante pluie, il donnera à chacun de l'herbe dans son champ.
\VS{2}Car les théraphim ont des paroles vaines, et les devins prophétisent le mensonge, ils profèrent des songes vains et consolent par la vanité. C'est pourquoi ils sont errants comme des brebis, ils sont malheureux, parce qu'il n'y a point de pasteur\FTNT{Mt. 9:36 ; Ez. 34:2 ; Jé. 23:21-30.}.
\VS{3}Ma colère s'est enflammée contre ces pasteurs, et je châtierai ces boucs ; car Yahweh des armées visite son troupeau, la maison de Juda ; et il les a rangés en bataille comme son cheval d'honneur.
\VS{4}De lui sortira l’Angle\FTNT{De lui (Juda) sortira l’Angle ou la pierre angulaire (Jésus-Christ), (1 Pi. 2:7 ; Es. 8:13-17).}, de lui sortira le clou, de lui sortira l'arc de bataille, et de lui sortiront tous les chefs ensemble.
\VS{5}Ils seront comme des vaillants hommes foulant la boue des rues dans la bataille, et ils combattront, parce que Yahweh sera avec eux ; et les cavaliers seront confus.
\VS{6}Car je fortifierai la maison de Juda, et je sauverai la maison de Joseph ; je les ramènerai, et je les ferai habiter en repos, parce que j'aurai compassion d'eux, et ils seront comme si je ne les avais point rejetés ; car je suis Yahweh, leur Dieu, et je les exaucerai.
\VS{7}Et ceux d'Ephraïm seront comme un héros, et leur cœur se réjouira comme par le vin ; leurs fils le verront, et se réjouiront ; leur cœur se réjouira en Yahweh.
\VS{8}Je les sifflerai et les rassemblerai, car je les rachète ; et ils seront multipliés comme ils l'ont été auparavant.
\VS{9}Et après que je les aurai dispersés parmi les peuples, ils se souviendront de moi dans les pays éloignés, et ils vivront avec leurs enfants, et ils reviendront.
\VS{10}Ainsi je les ramènerai du pays d'Egypte, je les rassemblerai de l'Assyrie, je les ferai venir au pays de Galaad, et au Liban, et il n'y aura point assez d'espace pour eux.
\VS{11}Il passera la mer de détresse, et il frappera les flots de la mer ; et toutes les profondeurs du fleuve seront desséchées ; l'orgueil de l'Assyrie sera abattu, et le sceptre d'Egypte sera ôté.
\VS{12}Je les fortifierai en Yahweh, et ils marcheront en son Nom, dit Yahweh.
\Chap{11}
\TextTitle{Les houlettes du vrai berger}
\VerseOne{}Liban, ouvre tes portes, et que le feu dévore tes cèdres !
\VS{2}Cyprès, gémis, car le cèdre est tombé, parce que les choses magnifiques ont été ravagées ! Chênes de Basan, gémissez, car la forêt inaccessible est coupée !
\VS{3}Les pasteurs poussent des cris de lamentations, parce que leur magnificence est ravagée ; on entend le rugissement des lionceaux, parce que l'orgueil du Jourdain est abattu.
\VS{4}Ainsi parle Yahweh, mon Dieu : Pais les brebis exposées au carnage !
\VS{5}Leurs possesseurs les égorgent, sans qu'on les tienne pour coupables, et celui qui les vend dit : Béni soit Yahweh, car je m’enrichis !  Et leurs pasteurs ne les épargnent pas.
\VS{6}Car je n’ai plus de pitié pour les habitants du pays, dit Yahweh ; et voici, je livre les hommes aux mains les uns des autres et aux mains de leur roi ; ils ravageront le pays, et je ne le délivrerai pas de leur main.
\VS{7}Alors je me mis donc à paître les brebis exposées au carnage, qui sont véritablement les plus misérables du troupeau. Puis je pris deux verges : J'appelai l'une Grâce, et l'autre Cordon ; et je me mis à paître les brebis.
\VS{8}Et je supprimai les trois pasteurs en un mois ; car mon âme était impatiente à leur sujet, et leur âme aussi avait pour moi du dégoût.
\VS{9}Et je dis : Je ne vous paîtrai plus ; que celle qui va mourir meure, et que celle qui va périr périsse, et que celles qui restent se dévorent la chair les unes les autres.
\VS{10}Puis je pris ma verge, appelée Grâce, et je la brisai, pour rompre mon alliance que j'avais traitée avec tous ces peuples.
\VS{11}Elle fut rompue en ce jour-là ; et les plus malheureuses brebis\FTNT{Les plus malheureuses des brebis sont le reste d’Israël.}, qui prirent garde à moi, reconnurent ainsi que c'était la parole de Yahweh.
\VS{12}Je leur dis : S'il vous semble bon, donnez-moi mon salaire ; sinon, ne me le donnez pas.  Alors ils pesèrent\FTNT{Mt. 26:15 ; Mt. 27:9-10.}  pour mon salaire trente pièces d'argent\FTNT{Selon la loi de Moïse, pour racheter un mâle de 20 à 60 ans, ayant fait un vœu, il fallait payer cinquante sicles d'argent (Lé. 27:3). Pour dédommager un préjudice causé par un bœuf ayant frappé un esclave, on devait donner trente sicles d'argent au maître de l'esclave et lapider le bœuf (Ex. 21:32). Or le prix du Seigneur a été estimé à trente sicles d’argent, comme pour les esclaves.}.
\VS{13}Yahweh me dit : Jette-le au potier, ce prix honorable auquel ils m’ont estimé ! Alors je pris les trente pièces d'argent, et les jetai dans la maison de Yahweh, pour le potier.
\VS{14}Puis je brisai ma seconde verge, appelée Cordon, pour rompre la fraternité entre Juda et Israël.
\TextTitle{Caractéristiques du faux berger}
\VS{15}Yahweh me dit : Prends-toi encore l'équipage d'un berger insensé.
\VS{16}Car voici, je susciterai dans le pays un pasteur, qui ne visitera pas les brebis qui périssent ; il ne cherchera pas celles qui s’égarent, il ne guérira pas celles qui sont blessées, et il ne soutiendra pas celles qui sont saines, mais il dévorera la chair des plus grasses, et il déchirera jusqu’aux cornes de leurs pieds.
\VS{17}Malheur au pasteur inutile qui abandonne les brebis ! Que l’épée fonde sur son bras et sur son œil droit ! Que son bras se dessèche, et que son œil droit s’éteigne entièrement !
\Chap{12}
\TextTitle{Jérusalem, une coupe d'étourdissement pour les nations}
\VerseOne{}Oracle, parole de Yahweh, sur Israël. Ainsi parle Yahweh, qui a étendu les cieux et fondé la terre, et qui a formé l'esprit de l'homme au-dedans de lui :
\VS{2}Voici, je ferai de Jérusalem une coupe d'étourdissement pour tous les peuples d'alentour ; et aussi pour Juda dans le siège de Jérusalem\FTNT{Ap. 16:12-16.}.
\VS{3}En ce jour-là, je ferai de Jérusalem une pierre pesante pour tous les peuples ; tous ceux qui en porteront le poids seront entièrement écrasés, car toutes les nations de la terre s'assembleront contre elle.
\VS{4}En ce temps-là, dit Yahweh, je frapperai d'étourdissement tous les chevaux, et de folie ceux qui les monteront ; mais j’aurai les yeux ouverts sur la maison de Juda, et je frapperai d'aveuglement tous les chevaux des peuples.
\VS{5}Les chefs de Juda diront en leur cœur : Les habitants de Jérusalem sont notre force, par Yahweh des armées, leur Dieu.
\VS{6}En ce jour-là, je ferai des chefs de Juda comme un foyer de feu parmi du bois, et comme une torche enflammée parmi des gerbes ; ils dévoreront à droite et à gauche tous les peuples d'alentour ; et Jérusalem sera encore habitée à sa place, à Jérusalem.
\VS{7}Yahweh sauvera premièrement les tentes de Juda, afin que la gloire de la maison de David, la gloire des habitants de Jérusalem, ne s'élève point au-dessus de Juda.
\VS{8}En ce jour-là, Yahweh sera le protecteur des habitants de Jérusalem ; et le plus faible parmi eux sera en ce jour-là comme David ; la maison de David sera comme Dieu, comme l'Ange de Yahweh devant leur face.
\VS{9}En ce jour-là, je chercherai à détruire toutes les nations qui viendront contre Jérusalem.
\TextTitle{Repentance et délivrance d'Israël}
\VS{10}Et je répandrai sur la maison de David, et sur les habitants de Jérusalem, l'Esprit de grâce et de supplications\FTNT{Joë. 2:28-30.}, et ils regarderont vers moi\FTNT{Au retour du Messie, il y aura une repentance et une conversion nationale d’Israël (Ro. 11:26).}, celui qu’ils ont percé, et ils pleureront sur lui\FTNT{Celui qu’ils ont percé : Il est question ici du Seigneur Jésus, le Messie (Ap. 1:7).}, comme on pleure sur un fils unique, et ils pleureront amèrement sur lui, comme quand on pleure sur un premier-né.
\VS{11}En ce jour-là, il y aura un grand deuil à Jérusalem, comme le deuil d'Hadadrimmon dans la vallée de Meguiddon.
\VS{12}Le pays sera dans le deuil, chaque famille à part : La famille de la maison de David à part, et les femmes de cette maison-là à part ; la famille de la maison de Nathan à part, et les femmes de cette maison-là à part.
\VS{13}La famille de la maison de Lévi à part, et les femmes de cette maison-là à part ; la famille de Schimeï à part, et ses femmes à part.
\VS{14}Toutes les autres  familles, chaque famille à part, et leurs femmes à part.
\Chap{13}
\TextTitle{Dieu frappe les faux prophètes}
\VerseOne{}En ce jour-là, il y aura une source ouverte en faveur de la maison de David et des habitants de Jérusalem, pour le péché et pour la souillure.
\VS{2}En ce jour-là, dit Yahweh des armées, je retrancherai du pays les noms des faux dieux, et on n'en fera plus mention. J'ôterai aussi du pays les faux prophètes et l'esprit d'impureté.
\VS{3}Et il arrivera que si quelqu'un prophétise encore, son père et sa mère qui l’ont engendré, lui diront : Tu ne vivras plus ; car tu as prononcé des mensonges au Nom de Yahweh ; et son père et sa mère qui l’ont engendré, le transperceront quand il prophétisera.
\VS{4}En ce jour-là, les prophètes seront confus de leurs visions, quand ils prophétiseront ; et ils ne revêtiront plus un manteau de poil pour mentir.
\VS{5}Chacun d’eux dira : Je ne suis pas prophète, mais je suis laboureur, car on m'a appris à gouverner du bétail dès ma jeunesse.
\VS{6}Et si on lui demande : Que veulent donc dire ces blessures que tu as aux mains ? Et il répondra : C’est dans la maison de mes amis qu’on me les a faites.
\TextTitle{Prophétie sur le vrai berger, le Messie}
\VS{7}Epée, réveille-toi contre mon Berger\FTNT{Mon Berger : Il est question de Jésus-Christ, le Bon Berger (Ps. 23 ; Jn. 10:1-17).}, et sur l'homme qui est mon compagnon ! dit Yahweh des armées frappe le Berger, et les brebis seront dispersées\FTNT{Frappe le Berger : Cette prophétie fait référence à la crucifixion du Seigneur Jésus-Christ (Ge. 3:15 ; Mt. 26:31 ; Mc. 14:27 ; Mc. 14:50 ; Mc. 15:19).} ; et je tournerai ma main vers les faibles.
\TextTitle{Le reste de Yahweh épuré à travers l'épreuve}
\VS{8}Dans tout le pays, dit Yahweh, les deux tiers seront retranchées et périront, et l’autre tiers restera.
\VS{9}Je mettrai ce tiers dans le feu, et je le purifierai comme on purifie l'argent, je les éprouverai comme on éprouve l'or. Il invoquera mon Nom, et je l'exaucerai ; je dirai : C'est ici mon peuple ! Et il dira : Yahweh est mon Dieu\FTNT{1 Pi. 1:6-7 ; Ps. 50:15 ; Ps. 91:15 ; Ps. 144:15.} !
\Chap{14}
\TextTitle{Imminence du jour de Yahweh}
\VerseOne{}Voici, le jour de Yahweh\FTNT{L’expression «~le jour du Seigneur~» ou «~le jour de Yahweh~» est utilisée dix-neuf fois dans le Tanakh (Es. 2:12 ;  Es. 13:6 ; Es. 13:9 ; Ez. 13:5 ; Ez. 30:3 ; Joë. 1:15 ; Joë. 2:1 ; Joë. 2:11 ; Joë. 2:31 ;  Joë. 3:14 ; Am. 5:18-20 ; Ab. 1:15 ; So. 1:7 ; So. 1:14 ; Za. 14:1 ; Mal. 4:5) et quatre fois dans les textes de la nouvelle alliance (Ac. 2:20 ; 2 Th. 2:2 ; 2 Pi. 3:10 ; Ap. 6:17 ; Ap. 16:14). Cette expression désigne habituellement des événements qui se déroulent à la fin des temps (Es. 7:18-25). Elle désigne un espace de temps au cours duquel Dieu va intervenir personnellement dans l’histoire des hommes. Appelé  «~jour de colère~», «~jour de  visitation~», et «~grand jour du Dieu Tout-Puissant~» ; il se réfère ainsi à un accomplissement encore futur, quand la colère de Dieu viendra s’abattre sur l’Israël qui n’aura pas cru (Es. 22 ; Jé. 30:1-17 ; Joë. 1 et 2 ; Am. 5 ; So. 1) et sur tous les incrédules du monde (Ez. 38 et 39 ; Za. 14). Ce jour sera aussi un temps de salut puisque Dieu va délivrer «~le reste~» d’Israël, accomplissant ainsi sa promesse selon laquelle «~tout Israël sera sauvé~» (Ro. 11:26) : il pardonnera leurs péchés et restaurera le peuple qu’Il s’est choisi sur la terre promise à Abraham (Es. 10:27 ; Jé. 30:19-31 ; Mi. 4 ; Za. 13).} arrive, et tes dépouilles seront partagées au milieu de toi, Jérusalem.
\VS{2}Je rassemblerai toutes les nations à Jérusalem pour qu’elles lui fassent la guerre\FTNT{Joë. 3 ; Ap. 16:12-16.} ; la ville sera prise, les maisons pillées, et les femmes violées ; la moitié de la ville ira en captivité, mais le reste du peuple ne sera pas retranché de la ville.
\VS{3}Yahweh sortira, et il combattra contre ces nations, comme il a combattu au jour de la bataille.
\TextTitle{Retour visible et en gloire du Seigneur}
\VS{4}Ses pieds se poseront en ce jour sur la Montagne des Oliviers\FTNT{Ce sont les pieds de Jésus-Christ (Ac. 1:10-11).}, qui est vis-à-vis de Jérusalem, du côté de l’orient ; et la Montagne des Oliviers se fendra par le milieu, à l'orient et à l'occident, de sorte qu'il y aura une très grande vallée ; une moitié de la montagne reculera vers le nord, et l'autre moitié vers le midi.
\VS{5}Vous fuirez alors dans la vallée de mes montagnes ; car la vallée des montagnes s’étendra jusqu'à Atzel ; et vous fuirez comme vous avez fui devant le tremblement de terre, aux jours d’Ozias, roi de Juda. Alors Yahweh, mon Dieu, viendra, et tous les saints seront avec lui\FTNT{Ce passage confirme clairement que Jésus-Christ est Yahweh (1 Th. 3:13 ; Jud. 14-15 ; Es. 34:5 ; Es. 40:10-11 ; Es. 62:11-15).}.
\VS{6}Et il arrivera qu'en ce jour-là, la lumière précieuse ne sera pas mêlée de ténèbres.
\VS{7}Ce sera un jour unique, connu de Yahweh, et qui ne sera ni jour ni nuit ; mais au temps du soir il y aura de la lumière.
\VS{8}Et il arrivera qu'en ce jour-là, des eaux vives\FTNT{Ez. 47:1-12 ;  Ap. 22:1-2.} sortiront de Jérusalem, la moitié d'elles coulera vers la mer orientale, et l'autre moitié, vers la mer occidentale ; il en sera ainsi été et hiver.
\TextTitle{Le royaume messianique}
\VS{9}Yahweh sera Roi sur toute la terre ; en ce jour-là, Yahweh sera Un, et son nom sera Un\FTNT{Littéralement «~E’had~». Le jour du Seigneur est un comme le jour un de Ge. 1:5. Yahweh est Un et non trois (De. 6:4). Son Nom est Un (Ac. 4:12).}.
\VS{10}Toute la terre deviendra comme la plaine, depuis Guéba jusqu'à Rimmon, au midi de Jérusalem ; et Jérusalem sera exaltée et restera à sa place, depuis la porte de Benjamin, jusqu'à l'endroit de la première porte, jusqu'à la porte des angles, et depuis la tour de Hananeel, jusqu'aux pressoirs du roi.
\VS{11}On habitera dans son sein, et il n'y aura plus d'interdit, mais Jérusalem sera habitée en sûreté.
\VS{12}Voici la plaie dont Yahweh frappera tous les peuples qui auront fait la guerre contre Jérusalem ; il fera que la chair de chacun tombera en pourriture tandis qu’ils seront sur leurs pieds, leurs yeux tomberont en pourriture dans leurs orbites, et leur langue tombera en pourriture dans leur bouche.
\VS{13}Et il arrivera en ce jour-là que Yahweh produira un grand trouble parmi eux ; car chacun saisira la main de son prochain, et la main de l'un s'élèvera contre la main de l'autre.
\VS{14}Juda combattra aussi dans Jérusalem, et les richesses de toutes les nations d'alentour y seront amassées : L'or, l'argent, et des vêtements en très grand nombre.
\VS{15}Et la même plaie sera sur les chevaux, les mulets, les chameaux, les ânes et sur toutes les bêtes qui seront dans ces camps, cette plaie sera semblable à l’autre.
\TextTitle{Adoration de Yahweh des armées dans le royaume}
\VS{16}Et il arrivera que tous ceux qui resteront de toutes les nations venues contre Jérusalem, monteront en foule chaque année pour adorer le Roi, Yahweh des armées, et pour célébrer la fête des tabernacles.
\VS{17}S’il y a des familles de la terre qui ne montent pas à Jérusalem, pour adorer le Roi, Yahweh des armées, la pluie ne tombera pas sur elles.
\VS{18}Si la famille d'Egypte ne monte pas, si elle ne vient pas, la pluie ne tombera pas sur elle ; elle sera frappée de la plaie dont Yahweh frappera les nations qui ne monteront pas pour célébrer la fête des tabernacles.
\VS{19}Ce sera la peine du péché de l’Egypte, et du péché de toutes les nations qui ne monteront pas pour célébrer la fête des tabernacles.
\VS{20}En ce jour-là, il sera écrit sur les clochettes des chevaux : Sainteté à Yahweh ! Et les chaudières dans la maison de Yahweh seront comme les coupes devant l'autel.
\VS{21}Toute chaudière qui sera à Jérusalem et dans Juda, sera consacrée à Yahweh des armées ; et tous ceux qui offriront des sacrifices viendront, et s’en serviront pour cuire  les viandes ; et il n'y aura plus de marchands dans la maison de Yahweh des armées, en ce jour-là.
\PPE{}
\end{multicols}

\clearpage\ShortTitle{Malachie}\BookTitle{Malachie}\BFont
\noindent\hrulefill
\textit{
\bigskip
{\centering{}
\\(Malakhi)
\\Signifie : Mon messager, mon ange
\\Thème : Message final de l'ancienne alliance à une nation désobéissante
\\Auteur : Malachie
\\Date de rédaction : Vème siècle av. J.-C.\\}
}
%\bigskip
\textit{
\\Dernier prophète de l’ancienne alliance, Malachie exerça son ministère en Juda après la reconstruction du temple et la reprise des cultes. Il annonça la venue du Messie et du messager qui devait le précéder, le nouvel Elie que Jésus-Christ reconnut en Jean-Baptiste. Ses écrits mettent en évidence l’importance de l’obéissance à la loi de Yahweh et la justice divine.
\bigskip
\\message. 
\bigskip
\\message.\bigskip
}
\par\nobreak\noindent\hrulefill
\begin{multicols}{2}
\TextTitle{[L'amour de Yahweh pour son peuple]}
\Chap{1}
\VerseOne{}Oracle, parole de Yahweh contre Israël, par le moyen de Malachie.
\VS{2}Je vous ai aimés, dit Yahweh ; et vous dites : En quoi nous as-tu aimés ? Esaü n'était-il pas frère de Jacob ? dit Yahweh. Or j'ai aimé Jacob,
\VS{3}Mais j'ai eu de la haine pour Esaü, et j'ai fait de ses montagnes une solitude, j’ai livré son héritage aux chacals du désert.
\VS{4}Si Edom dit : Nous sommes détruits, nous rebâtirons les lieux ruinés ! Ainsi parle Yahweh des armées : Ils rebâtiront, mais je détruirai, et on les appellera pays de méchanceté, peuple contre lequel Yahweh est irrité pour toujours.
\VS{5}Vos yeux le verront, et vous direz : Yahweh est grand par-delà les frontières d'Israël !
\TextTitle{[Le péché des sacrificateurs après le retour d'exil]}
\VS{6}Un fils honore son père, et un serviteur son maître. Si donc je suis Père, où est l'honneur qui m'appartient ? Si je suis maître, où est la crainte qu'on a de moi ? dit Yahweh des armées, à vous sacrificateurs, qui méprisez mon Nom, et qui dites : En quoi avons-nous méprisé ton Nom ?
\VS{7}Vous offrez sur mon autel des aliments souillés, et vous dites : En quoi t'avons-nous profané ? C'est en disant : La table de Yahweh est méprisable !
\VS{8}Et quand vous amenez une bête aveugle pour la sacrifier, n'y a-t-il point de mal en cela ? Quand vous en offrez une boiteuse ou malade, n’est-ce pas mal ? Offre-la à ton gouverneur ! T’agréera-t-il, te recevra-t-il favorablement ? dit Yahweh des armées.
\VS{9}Maintenant, donc suppliez Dieu, pour qu'il ait pitié de nous ! Cela vient de vos mains : Vous recevra-t-il favorablement ? dit Yahweh des armées.
\VS{10}Lequel de vous fermera les portes pour que vous n’allumiez pas en vain le feu sur mon autel ? Je ne prends aucun plaisir en vous, dit Yahweh des armées, et je n’agrée pas l'offrande de vos mains.
\VS{11}Car depuis le soleil levant jusqu'au soleil couchant, mon Nom est grand parmi les nations, et en tous lieux on brûle de l’encens en l'honneur de mon Nom, et des offrandes pures ; car mon Nom est grand parmi les nations, dit Yahweh des armées.
\VS{12}Mais vous, vous le profanez, en disant : La table de Yahweh est souillée, et ce qu’elle rapporte est un aliment méprisable.
\VS{13}Vous dites aussi : Quelle fatigue ! Et vous le dédaignez, dit Yahweh des armées ; vous amenez ce qui a été dérobé, ce qui est boiteux, et malade, ce sont là les offrandes que vous faites ! Accepterai-je cela de vos mains ? dit Yahweh.
\VS{14}C'est pourquoi, maudit soit l'homme trompeur, qui a dans son troupeau un mâle, et qui voue et sacrifie à Yahweh ce qui est corrompu ! Car je suis un grand roi, dit Yahweh des armées, et mon Nom est redoutable parmi les nations.
\TextTitle{[Mise en garde de Yahweh aux sacrificateurs]}
\Chap{2}
\VerseOne{}Maintenant, c’est à vous, sacrificateurs que s'adresse ce commandement :
\VS{2}Si vous n'écoutez pas, et que vous ne preniez point pas à cœur de donner gloire à mon Nom, dit Yahweh des armées, j'enverrai sur vous la malédiction, et je maudirai vos bénédictions ; et déjà même je les ai maudites, parce que vous ne prenez pas cela à cœur.
\VS{3}Voici, je vais détruire vos semences, et je répandrai les excréments de vos victimes sur vos visages, les excréments, dis-je, de vos solennités, et on vous emportera avec eux.
\VS{4}Alors vous saurez que je vous ai adressé ce commandement, afin que mon alliance avec Lévi subsiste, dit Yahweh des armées.
\VS{5}Mon alliance avec lui était la vie et la paix, c’est ce que je lui accordai pour qu’il me craigne ; il a eu pour moi de la crainte, et il a tremblé devant mon Nom.
\VS{6}La loi de la vérité était dans sa bouche, et il ne s'est point trouvé de perversité sur ses lèvres ; il a marché avec moi dans la paix et dans la droiture, et il en a détourné beaucoup de l'iniquité.
\VS{7}Car les lèvres du sacrificateur doivent garder la science, et c’est de sa bouche qu’on demande la loi, parce qu'il est un messager de Yahweh des armées.
\VS{8}Mais vous, vous vous êtes écartés de la voie, vous avez fait de la loi une occasion de chute pour beaucoup, et vous avez corrompu l'alliance de Lévi, dit Yahweh des armées.
\TextTitle{[Infidélités envers des frères et envers Yahweh]}
\VS{9}C'est pourquoi je vous rendrai méprisables et abjects aux yeux de tout le peuple. Parce que vous n’avez pas gardé mes voies, et vous avez égard à l'apparence des personnes quand vous enseignez la loi.
\VS{10}N'avons-nous pas tous un seul Père ? N’est-ce pas un seul Dieu qui nous a créés ? Pourquoi donc agissons-nous avec perfidie l’un avec l’autre, en violant l'alliance de nos pères ?
\VS{11}Juda s’est montré infidèle, et une abomination a été commise en Israël et à Jérusalem ; car Juda a profané ce qui est consacré à Yahweh, ce qu’il aime, il s'est marié à la fille d'un dieu étranger.
\VS{12}Yahweh retranchera l’homme qui fait cela, celui qui veille et qui répond, il le retranchera des tentes de Jacob, et il retranchera celui qui présente une offrande à Yahweh des armées.
\VS{13}Voici une autre chose que vous faites : Vous couvrez l'autel de Yahweh de larmes, de plaintes et de gémissements, en sorte qu’il n’a plus égard aux offrandes et qu’il ne peut rien agréer de vos mains.
\VS{14}Et vous dites : Pourquoi ?... C'est parce que Yahweh est intervenu comme témoin entre toi et la femme de ta jeunesse, envers laquelle tu as été infidèle, bien qu’elle soit ta compagne et la femme de ton alliance.
\VS{15}Nul n’a fait cela, avec un reste de bon esprit. Un seul l’a fait, et pourquoi ? Parce qu'il cherchait une postérité de Dieu. Prenez donc garde en votre esprit, et qu’aucun ne soit infidèle à la femme de sa jeunesse !
\VS{16}Car je hais la répudiation, dit Yahweh, le Dieu d'Israël, et celui qui couvre de violence son vêtement, dit Yahweh des armées. Prenez donc garde en votre esprit, et ne soyez pas infidèles !
\TextTitle{[Fausse profession religieuse]}
\VS{17}Vous fatiguez Yahweh par vos paroles, et vous dites : En quoi l'avons-nous fatigué ? C'est quand vous dites : Quiconque fait le mal plaît à Yahweh, et il prend plaisir à de tels gens ! Autrement : Où est le Dieu du jugement ?
\TextTitle{[Venue du précurseur du messie]}
\Chap{3}
\VerseOne{}Voici, j'enverrai mon messager\FTNT{Ce messager, ou Elie le prophète, est Jean-Baptiste (Es. 40:1-3 ; Mal. 3:1 ; Mt. 3:1-15 ; Mt. 11:14 ; Mt. 17:10-13 ; Mc. 1:1-11 ; Mc. 9:11-13 ; Lu. 1:17 ; Lu. 3:1-5).} ; il préparera le chemin devant moi. Et soudain entrera dans son temple le Seigneur que vous cherchez ; l’ange de l'alliance\FTNT{L’ange de l’alliance est le Seigneur Jésus-Christ. Voir aussi commentaire en Mt. 1:20}, que vous désirez, voici, il vient, dit Yahweh des armées.
\VS{2}Mais qui pourra soutenir le jour de sa venue ? Qui pourra subsister quand il paraîtra ? Car il sera comme le feu du fondeur et comme la potasse des foulons.
\VS{3}Et il sera assis comme celui qui raffine et purifie l'argent ; il nettoiera les fils de Lévi, il les épurera comme l’or et l'argent, et ils présenteront à Yahweh des offrandes avec justice.
\VS{4}Alors l’offrande de Juda et de Jérusalem sera agréable à Yahweh, comme aux anciens jours, comme aux années d'autrefois.
\VS{5}Je m'approcherai de vous pour le jugement, et je me hâterai de témoigner contre les enchanteurs et les adultères, contre ceux qui jurent faussement, et contre ceux qui retiennent le salaire du mercenaire, qui oppriment la veuve et l'orphelin, qui font tort à l'étranger, et qui ne me craignent point, dit Yahweh des armées.
\VS{6}Parce que je suis Yahweh et que je n'ai point changé ; à cause de cela, enfants de Jacob, vous n'avez point été consumés.
\TextTitle{[Le peuple infidèle qui vole Yahweh]}
\VS{7}Depuis le temps de vos pères, vous vous êtes écartés de mes ordonnances, vous ne les avez point observées. Revenez à moi, et je reviendrai à vous, dit Yahweh des armées. Et vous dites : En quoi nous convertirons-nous ?
\VS{8}L'homme pillera-t-il Dieu, que vous me pilliez ? Et vous dites : En quoi t'avons-nous pillé ? Vous l'avez fait dans les dîmes et dans les offrandes.
\VS{9}Vous êtes certainement maudits, parce que vous me pillez, vous, toute la nation !
\VS{10}Apportez toutes les dîmes\FTNT{Il est question ici de la dîme de la dîme que les levites donnaient aux sacrificateurs. Cette dîme était rapportée aux magasins, ou greniers (Né. 10:35-39), là aussi était stocké toute sorte de trésor. Pour les autres dîmes, voir le commentaire dans Dt. 14:22-29.} aux magasins, afin qu'il y ait provision dans ma maison ; et dès maintenant éprouvez moi en cela, a dit Yahweh des armées, si je ne vous ouvre pas les écluses des cieux, et si je ne répands pas en votre faveur la bénédiction, jusqu'à ce qu'il n'y ait plus assez de place.
\VS{11}Et je réprimerai pour l'amour de vous le dévorateur, et il ne vous ravagera pas les fruits de la terre, et vos vignes ne seront pas stériles dans vos campagnes, a dit Yahweh des armées.
\VS{12}Toutes les nations vous diront heureux, car vous serez un pays de délices, dit Yahweh des armées.
\VS{13}Vos paroles sont rudes contre moi, a dit Yahweh. Et vous dites : Qu'avons-nous donc dit contre toi ?
\VS{14}Vous avez dit : C'est en vain que l’on sert Dieu ; et qu'avons-nous gagné à observer ses ordonnances, et à marcher en pauvre état pour l'amour de Yahweh des armées ?
\VS{15}Et maintenant nous tenons pour heureux les orgueilleux ; et même ceux qui commettent la mechanceté, sont avancés ; et s'ils ont tenté Dieu, ils ont été délivrés !
\TextTitle{[Le "reste d'Israël" demeure fidèle à Yahweh"]}
\VS{16}Alors ceux qui craignent Yahweh se parlèrent l'un à l’autre ; et Yahweh fut attentif, et il écouta ; et un livre de souvenir fut écrit devant lui pour ceux qui craignent Yahweh et qui pensent à son Nom.
\VS{17}Ils seront à moi, a dit Yahweh des armées, le jour où je mettrai à part mes plus précieux joyaux, et je leur pardonnerai comme un homme pardonne à son fils qui le sert.
\VS{18}Convertissez-vous donc, et vous verrez la différence qu'il y a entre le juste et le méchant, entre celui qui sert Dieu et celui qui ne le sert pas.
\TextTitle{[Avènement du jour de Yahweh]}
\Chap{4}
\VerseOne{}Car voici, le jour vient, ardent comme une fournaise. Tous les orgueilleux et tous les méchants seront comme du chaume ; et ce jour qui vient, dit Yahweh des armées, les embrasera, il ne leur laissera ni racine ni rameau.
\VS{2}Mais pour vous qui craignez mon Nom, se lèvera le Soleil de justice\FTNT{Le Soleil de justice : Jésus-Christ est notre Soleil (Lu. 1:78-79). Cet aspect de Jésus-Christ nous parle de la grâce de Dieu : « il fait lever son soleil sur les méchants, et sur les gens de bien » (Mt. 5:45). Le soleil évoque aussi le jugement de Dieu. Ainsi, en plein midi, il est le feu de la justice et de la colère de Dieu. Son pardon et son amour pour nous sont alors comparés à une ombre fraîche qui nous sauve de sa chaleur ardente (Ps. 121 ; Es. 25:4). Dans Ps. 19:6, le soleil est comparé à un époux. Or Jésus-Christ est notre époux et le soleil qui nous apporte la guérison.}, et la guérison sera sous ses ailes ; vous sortirez, et bondirez comme les veaux d’une étable.
\VS{3}Et vous foulerez les méchants, car ils seront comme de la cendre sous les plantes de vos pieds, au jour où je ferai mon œuvre, dit Yahweh des armées.
\VS{4}Souvenez-vous de la loi de Moïse, mon serviteur, auquel j’ai prescrit en Horeb, pour tout Israël, des statuts et des ordonnances.
\TextTitle{[Retour d'Elie avant le jour de Yahweh]}
\VS{5}Voici, je vous enverrai Elie, le prophète\FTNT{Voir commentaire en Mal. 3:1.}, avant que le jour grand et redoutable de Yahweh vienne.
\VS{6}Il ramènera le cœur des pères à leurs enfants, et le cœur des enfants à leurs pères, de peur que je ne vienne et que je ne frappe la terre d’interdit.
\PPE{}
\end{multicols}
\clearpage
\addcontentsline{toc}{section}{Ketouvim (Ecrits)}\clearpage
\clearpage\makeatletter\def\@evenhead{}\def\@oddhead{}\makeatother

\vspace*{\fill}
\begin{center}
{\Huge Tanakh : Ketouvim}
\end{center}
\vspace*{\fill}

\clearpage

\makeatletter\def\@evenhead{{\NoAutoSpaceBeforeFDP{\small{\rightmark\hfil\thepage\hfil\leftmark}}}}\def\@oddhead{{\NoAutoSpaceBeforeFDP{\small{\rightmark\hfil\thepage\hfil\leftmark}}}}\makeatother

\clearpage\ShortTitle{Psaumes}\BookTitle{Psaumes}\BFont
\noindent\hrulefill
{\footnotesize
\textit{
\bigskip
{\centering{}
\\Auteurs : David essentiellement et d'autres écrivains
\\(Heb. : Tehilim)
\\Signification : Louanges 
\\Thème : La louange et l'adoration
\\Date de rédaction : A compter du 10\up{ème} siècle et au-delà\\}
}
%\bigskip
\textit{
\\Le terme « psaume » désigne un poème chanté avec l'accompagnement d'un instrument. C'est ainsi que furent initialement contés les récits de la création divine, la captivité ou encore la gloire de Jérusalem. Expressions de joie, de reconnaissance, de repentance, d'angoisse ou de vulnérabilité de l'homme, ces hymnes étaient des prières adressées à Dieu.
%\bigskip
\\Prophétiques, certains psaumes annoncent les événements de la fin des temps, notamment les souffrances de Christ. Utilisé comme recueil de chants, le livre des Psaumes exalte la grandeur de Dieu, sa souveraineté, sa miséricorde et son omniscience. Il est le fruit d'une grande variété d'expériences spirituelles du fait de la diversité de ses auteurs. De plus, il contient une richesse de styles considérable, ce qui en fait le chef-d'œuvre de la poésie hébraïque.\bigskip
}
}
\par\nobreak\noindent\hrulefill
\begin{multicols}{2}
\Chap{1}
\TextTitle{La voie du juste et du pécheur}
\VerseOne{}Heureux l'homme qui ne marche pas selon le conseil des méchants, et qui ne s'arrête pas sur la voie des pécheurs, qui ne s'assied pas dans l'assemblée des moqueurs\FTNT{Jé. 15:17 ; 1 Co. 15 : 33 ; Ep. 5:11.},
\VS{2}mais qui prend plaisir dans la loi de Yahweh, et qui médite sa loi jour et nuit\FTNT{De. 6:6 ; De. 17:19 ; Jos. 1:8.}.
\VS{3}Il est comme un arbre planté près des ruisseaux d'eaux, qui rend son fruit en sa saison, et dont le feuillage ne se flétrit point\FTNT{Jé. 17:7-8 ; Ez. 47:12 ; Jn. 15:8 ; Ap. 22:2.}. Et ainsi tout ce qu'il fera réussira.
\VS{4}Il n'en est pas ainsi des méchants : Ils sont comme la balle que le vent chasse au loin\FTNT{Job. 21:17-18 ; Os. 13:3.}.
\VS{5}C'est pourquoi les méchants ne résistent pas dans le jugement, ni les pécheurs dans l'assemblée des justes.
\VS{6}Car Yahweh connaît la voie des justes, mais la voie des méchants périra.
\Chap{2}
\TextTitle{Complot des nations contre le Messie}
\VerseOne{}Pourquoi cette agitation parmi les nations, et pourquoi les peuples projettent-ils des choses vaines ?
\VS{2}Pourquoi les rois de la terre se lèvent-ils en personne, et les princes se liguent-ils avec eux contre Yahweh, et contre son Messie\FTNT{Cette prophétie concerne le complot des Juifs, de Pilate et d'Hérode contre Jésus-Christ, notre Seigneur. Il est également question du gouvernement mondial dirigé par Satan. Mt. 12:14 ; Mt. 26:3-4 ; Mt. 26:59-66. ; Mt. 27:1-2 ; Mc. 3:6 ; Mc. 11:18 ; Ac. 4:23-29.}?
\VS{3}Rompons leurs liens et jetons loin de nous leurs cordes!
\VS{4}Celui qui habite dans les cieux se rit d'eux, le Seigneur se moque d'eux.
\VS{5}Il leur parle dans sa colère, et il les remplira de terreur par la grandeur de son courroux\FTNT{Pr. 1:26.}:
\VS{6}C'est moi qui ai consacré mon Roi sur Sion, la montagne de ma sainteté\FTNT{Mi. 4:7.}!
\VS{7}Je vous réciterai cette ordonnance ; Yahweh m'a dit : Tu es mon Fils ! Je t'ai engendré aujourd'hui\FTNT{Ac. 13:33 ; Hé. 1:5 ; Hé. 5:5.}.
\VS{8}Demande-moi, et je te donnerai les nations pour héritage, et les extrémités de la terre pour possession.
\VS{9}Tu les briseras avec un sceptre de fer et tu les mettras en pièces comme un vase de potier\FTNT{Da. 2:44 ; Ap. 2:27.}.
\VS{10}Maintenant donc, rois, ayez de l'intelligence ! Juges de la terre, recevez instruction !
\VS{11}Servez Yahweh avec crainte, et réjouissez-vous avec tremblement\FTNT{Ps. 19:10.}.
\VS{12}Embrassez le Fils, de peur qu'il ne s'irrite et que vous ne périssiez dans cette conduite, quand sa colère s'embrasera promptement. Heureux sont tous ceux qui se confient en lui !
\Chap{3}
\TextTitle{Yahweh, le véritable secours}
\VerseOne{}Psaume de David au sujet de sa fuite devant Absalom, son fils.
\VS{2}Ô Yahweh, que mes adversaires sont nombreux ! Beaucoup de gens se lèvent contre moi!
\VS{3}Plusieurs disent à mon âme : Plus de salut pour lui auprès de Dieu! Sélah\FTNT{Le mot hébreu « Sélah » signifie « élever, exalter ». Il peut aussi traduire une pause dans le cantique ou le texte. C'est sûrement un terme technique musical montrant probablement une accentuation, une pause, une interruption.}.
\VS{4}Mais toi, ô Yahweh ! Tu es un bouclier autour de moi, tu es ma gloire, et tu relèves ma tête.
\VS{5}De ma voix je crie à Yahweh, et il me répond de sa sainte montagne. Sélah.
\VS{6}Je me couche, je m'endors, je me réveille, car Yahweh me soutient\FTNT{Lé. 26:6.}.
\VS{7}Je ne crains pas les myriades de peuples quand ils se rangent contre moi de toutes parts.
\VS{8}Lève-toi, Yahweh, mon Dieu ! Délivre-moi ! Car tu frappes à la joue tous mes ennemis, tu brises les dents des méchants.
\VS{9}La délivrance vient de Yahweh\FTNT{Es. 43:11 ; Jé. 3:23 ; Pr. 21:31 ; Ap. 7:10.} ! Que ta bénédiction soit sur ton peuple! Sélah.
\Chap{4}
\TextTitle{Yahweh, la joie et la paix du juste}
\VerseOne{}Psaume de David, donné au chef des chantres pour le chanter sur Neguinoth.
\VS{2}Ô Dieu de ma justice, puisque je crie, réponds-moi ! Quand j'étais à l'étroit, tu m'as mis au large ! Aie pitié de moi, et exauce ma prière\FTNT{Ps. 28:1-2.} !
\VS{3}Fils des hommes, jusqu'à quand ma gloire sera-t-elle diffamée ? Jusqu'à quand aimerez-vous la vanité et chercherez-vous le mensonge ? Sélah.
\VS{4}Or sachez que Yahweh s'est choisi un bien-aimé. Yahweh m'exauce quand je crie à lui\FTNT{1 Jn. 5:14.}.
\VS{5}Tremblez et ne péchez point ; parlez en vos cœurs sur votre couche et taisez-vous. Sélah.
\VS{6}Offrez des sacrifices de justice\FTNT{Ps. 51:19.} et confiez-vous en Yahweh.
\VS{7}Plusieurs disent : Qui nous fera voir le bonheur ? Lève sur nous la lumière de ta face, ô Yahweh !
\VS{8}Tu mets plus de joie dans mon cœur qu'ils n'en ont, quand abondent leur froment et leur vin.
\VS{9}Je me couche et je m'endors en paix, car toi seul, ô Yahweh ! Tu me fais reposer en sécurité\FTNT{Pr. 3:24.}.
\Chap{5}
\TextTitle{Recours à la protection de Yahweh}
\VerseOne{}Psaume de David, donné au chef des chantres, pour le chanter sur Nehiloth.
\VS{2}Yahweh, prête l'oreille à mes paroles ! Ecoute ma méditation !
\VS{3}Mon Roi et mon Dieu ! Sois attentif à la voix de mon cri ; car c'est à toi que j'adresse ma requête.
\VS{4}Yahweh, le matin tu entends ma voix, dès le matin je me tourne vers toi, et je veille.
\VS{5}Car tu n'es point un Dieu qui prenne plaisir au mal ; le méchant n'a point sa demeure auprès de toi.
\VS{6}Les orgueilleux ne subsistent pas devant tes yeux ; tu hais tous ceux qui commettent l'iniquité\FTNT{Ps. 1:5 ; Ha. 1:13.}.
\VS{7}Tu fais périr les menteurs ; Yahweh a en abomination l'homme sanguinaire et le trompeur.
\VS{8}Mais moi, comblé de tes bienfaits, j'entrerai dans ta maison, je me prosternerai dans le palais de ta sainteté avec les sentiments d'une crainte respectueuse.
\VS{9}Yahweh, conduis-moi dans ta justice, à cause de mes ennemis, aplanis ta voie sous mes pas\FTNT{Ps. 25:4-5 ; Ps. 27:11.}.
\VS{10}Car il n'y a rien de droit dans leur bouche; leur cœur est rempli de malice, leur gosier est un sépulcre ouvert, ils flattent de leur langue\FTNT{Ps. 10:7 ; Ps. 12:3 ; Ro. 3:13.}.
\VS{11}Ô Dieu ! Fais-leur leur procès, qu'ils échouent dans leurs entreprises ! Chasse-les au loin, à cause du grand nombre de leurs transgressions ! Car ils se sont rebellés contre toi.
\VS{12}Mais que tous ceux qui se confient en toi se réjouiront, qu'ils soient dans la joie perpétuellement, et que tu sois leur protecteur ; et que ceux qui aiment ton Nom s'égayent en toi !
\VS{13}Car tu bénis le juste, ô Yahweh ! Et tu l'entoures de ta bienveillance comme d'un bouclier.
\Chap{6}
\TextTitle{La miséricorde de Yahweh}
\VerseOne{}Psaume de David, donné au chef des chantres, pour le chanter pour le chanter en Neguinoth, sur Sheminith.
\VS{2}Yahweh, ne me punis pas dans ta colère et ne me châtie pas dans ta fureur\FTNT{Jé. 10:24.}.
\VS{3}Yahweh, aie pitié de moi ! Car je suis sans aucune force. Guéris-moi, ô Yahweh ! Car mes os sont épouvantés.
\VS{4}Même mon âme est fort troublée ; et toi, ô Yahweh ! Jusqu'à quand ?
\VS{5}Reviens, Yahweh ! Délivre mon âme. Sauve-moi, à cause de ta miséricorde.
\VS{6}Car celui qui meurt n'a plus ton souvenir ; qui te célébrera dans le scheol\FTNT{Es. 38 : 18 ; Ps. 88 : 11 ; Ps. 115 : 17.} ?
\VS{7}Je m'épuise à force de gémir; chaque nuit ma couche est baignée de mes larmes\FTNT{Job 7 : 3-4.}, mon lit est arrosé de mes pleurs.
\VS{8}J'ai le visage usé par le chagrin\FTNT{Ps. 31 : 10.}; il vieillit à cause de tous ceux qui m'oppriment.
\VS{9}Retirez-vous loin de moi, vous tous ouvriers d'iniquité\FTNT{Mt. 7:23 ; Mt. 25 : 41 ; Lu. 13 : 27.}! Car Yahweh a entendu la voix de mes pleurs.
\VS{10}Yahweh a entendu ma supplication, Yahweh a reçu ma prière.
\VS{11}Tous mes ennemis sont confondus, saisis d'épouvante; ils reculent soudain, honteux.
\Chap{7}
\TextTitle{La délivrance se trouve auprès de Yahweh}
\VerseOne{}Shiggaïon de David, chantée à Yahweh, au sujet de Cusch, le Benjamite.
\VS{2}Yahweh, mon Dieu ! Je cherche en toi mon refuge. Sauve-moi de tous mes persécuteurs et délivre-moi,
\VS{3}afin qu'ils ne me déchirent pas, comme un lion qui dévore sans qu'il n'y ait personne qui me secoure.
\VS{4}Yahweh, mon Dieu ! Si j'ai commis une telle action, s'il y a de l'iniquité dans mes mains,
\VS{5}si j'ai rendu le mal à celui qui était paisible envers moi, si j'ai dépouillé celui qui m'opprimait sans cause,
\VS{6}que l'ennemi me poursuive et m'atteigne, qu'il foule à terre ma vie, et qu'il couche ma gloire dans la poussière ! Sélah.
\VS{7}Lève-toi, ô Yahweh ! Dans ta colère, lève-toi contre la fureur de mes adversaires. Réveille-toi pour me secourir, ordonne un jugement !
\VS{8}Que l'assemblée des peuples t'environne ! Monte au-dessus d'elle vers les lieux élevés !
\VS{9}Yahweh juge les peuples : Rends-moi justice, ô Yahweh\FTNT{Ps. 9 : 5.} ! Selon ma droiture et selon mon intégrité !
\VS{10}Que la malice des méchants prenne fin, et affermis le juste, toi qui sondes les cœurs et les reins\FTNT{Jé. 11:20 ; Jé. 17:10.}, ô Dieu juste !
\VS{11}Mon bouclier est en Dieu, qui délivre ceux qui sont droits de cœur.
\VS{12}Dieu est un juste juge, Dieu s'irrite en tout temps.
\VS{13}Si le méchant ne se convertit pas, Dieu aiguise son épée\FTNT{De. 32 : 41.}, il bande son arc, et vise.
\VS{14}Il dirige sur lui des traits meurtriers, il rend ses flèches brûlantes.
\VS{15}Voici, le méchant prépare le mal, il conçoit l'iniquité, et il enfante le mensonge\FTNT{Ja. 1:15.}.
\VS{16}Il fait une fosse, il la creuse, et il tombe dans la fosse qu'il a faite\FTNT{Ps. 9 : 16.}.
\VS{17}Son travail retourne sur sa tête, et sa violence redescend sur son front.
\VS{18}Je célébrerai Yahweh à cause de sa justice, je psalmodierai le Nom de Yahweh, du Très-Haut.
\Chap{8}
\TextTitle{Magnificence de Dieu et vanité de l'homme}
\VerseOne{}Psaume de David, donné au chef des chantres, pour le chanter sur guitthith.
\VS{2}Yahweh, notre Seigneur ! Que ton Nom est magnifique sur toute la terre ! Ta majesté s'élève au-dessus des cieux\FTNT{Es. 6 : 3.}.
\VS{3}Par la bouche des petits enfants et de ceux qui tètent\FTNT{Mt. 21 : 16.}, tu as fondé ta puissance, à cause de tes adversaires, afin de faire cesser l'ennemi et le vindicatif.
\VS{4}Quand je regarde tes cieux, l'ouvrage de tes doigts, la lune et les étoiles que tu as fixées:
\VS{5}Qu'est-ce que l'homme, pour que tu te souviennes de lui ? Et le fils de l'homme, pour que tu le visites\FTNT{Dans ce passage, il est question de l'incarnation de Yahweh afin de nous sauver (1 Co. 15:45-49 ; 1 Ti. 3:16 ; Hé. 2:14). Jésus-Christ s'est lui-même nommé « fils de l'homme » (Lu. 9:22-26), littéralement « fils d'Adam », d'ailleurs cette expression apparaît dans les évangiles plus de quatre-vingt fois.} ?
\VS{6}Tu l'as fait de peu inférieur aux anges, et tu l'as couronné de gloire et d'honneur.
\VS{7}Tu lui as donné la domination sur les œuvres de tes mains, tu as tout mis sous ses pieds\FTNT{1 Co. 15:27.},
\VS{8}les brebis comme les bœufs, les animaux des champs,
\VS{9}les oiseaux du ciel et les poissons de la mer, tout ce qui parcourt les sentiers des mers.
\VS{10}Yahweh, notre Seigneur ! Que ton Nom est magnifique sur toute la terre !
\Chap{9}
\TextTitle{Louange à Yahweh, l'auteur de nos victoires}
\VerseOne{}Psaume de David, donné au chef des chantres, pour le chanter sur Muth-Labben.
\VS{2}Je célébrerai de tout mon cœur Yahweh, je raconterai toutes tes merveilles.
\VS{3}Je me réjouirai et je m'égaierai en toi, je chanterai ton Nom, ô Très-Haut !
\VS{4}Mes ennemis reculent, ils trébuchent, ils périssent devant ta face.
\VS{5}Car tu soutiens mon droit et ma cause, tu sièges sur ton trône en juste juge.
\VS{6}Tu châties les nations, tu détruis le méchant, tu effaces leur nom pour toujours, et à perpétuité.
\VS{7}Plus d'ennemis ! Les désolations ont-elles pris fin ? As-tu aussi rasé les villes pour toujours ? Leur mémoire est perdue avec elles.
\VS{8}Mais Yahweh sera assis éternellement, il a établi son trône pour juger.
\VS{9}Il juge le monde avec justice, il juge les peuples avec droiture\FTNT{Ps. 96:13 ; Ps. 98:9.}.
\VS{10}Yahweh est un refuge pour l'opprimé, un refuge au temps de la détresse\FTNT{Ps. 37:39 ; Ps. 46:2 ; Ps. 91:2.}.
\VS{11}Ceux qui connaissent ton Nom se confient en toi\FTNT{Pr. 3:5.}. Car tu n'abandonnes point ceux qui te cherchent, ô Yahweh !
\VS{12}Chantez à Yahweh qui habite en Sion, annoncez ses exploits parmi les peuples !
\VS{13}Lorsqu'il recherche le sang versé, il se souvient des malheureux? il n'oublie pas le cris des affligés.
\VS{14}Aie pitié de moi, Yahweh ! Vois la misère où me réduisent mes ennemis, enlève-moi des portes de la mort,
\VS{15}afin que je raconte toutes tes louanges, dans les portes de la fille de Sion. Je me réjouirai de la délivrance\FTNT{Voir commentaire en Es. 26:1.} que tu m'auras donnée.
\VS{16}Les nations tombent dans la fosse qu'elles ont faite\FTNT{Ps. 10:2 ; Ps. 35:7.}, leur pied se prend au filet qu'elles ont caché.
\VS{17}Yahweh se fait connaître, il fait justice, le méchant est enlacé dans l'ouvrage de ses mains. Jeu d'instruments. Sélah.
\VS{18}Les méchants retournent dans le scheol, toutes les nations qui oublient Dieu.
\VS{19}Car le pauvre n'est point oublié à jamais, l'espérance des affligés ne périt pas à toujours.
\VS{20}Lève-toi, ô Yahweh ! Que l'homme mortel ne triomphe point ! Que les nations soient jugées devant ta face !
\VS{21}Frappe-les de terreur, ô Yahweh ! Que les peuples sachent qu'ils ne sont que des hommes mortels\FTNT{Es. 51:12.}! Sélah.
\Chap{10}
\TextTitle{Appel au jugement de Dieu sur les méchants}
\VerseOne{}Pourquoi, ô Yahweh ! Te tiens-tu éloigné ? Pourquoi te caches-tu au temps où nous sommes dans la détresse\FTNT{Ps. 13:2 ; Ps. 44:24.} ?
\VS{2}Le méchant par son orgueil poursuit ardemment les affligés, mais ils seront pris par les machinations qu'ils ont préméditées\FTNT{Ps. 7:15-16 ; Ps. 9:16 ; Ps. 35:8.}.
\VS{3}Car le méchant se glorifie du désir de son âme, il bénit l'avare et il méprise Yahweh.
\VS{4}Le méchant dit avec arrogance : Il ne fera pas d'enquête ! Il n'y a point de Dieu\FTNT{Ps. 14:1 ; Ps. 53:2.} ! Voilà toutes ses pensées.
\VS{5}Ses voies réussissent en tout temps; tes jugements sont éloignés de lui, il souffle contre tous ses adversaires.
\VS{6}Il dit en son cœur : Je ne chancelle pas, je suis pour toujours à l'abri du malheur !
\VS{7}Sa bouche est pleine de malédictions, de tromperies et de fraudes ; il n'y a sous sa langue qu'oppression et outrage\FTNT{Ps. 59:7-8 ; Ps. 64:3-4 ; Job. 20:12.}.
\VS{8}Il se tient aux embûches dans des villages, il tue l'innocent dans des lieux cachés, ses yeux épient le malheureux.
\VS{9}Il se tient aux aguets dans un lieu caché, comme un lion dans sa tanière, il se tient aux aguets pour attraper l'affligé ; il attrape l'affligé, l'attirant dans son filet.
\VS{10}Il se courbe, il se baisse, et les malheureux tombent dans ses griffes.
\VS{11}Il dit en son cœur : Dieu oublie ! Il cache sa face, il ne le verra jamais\FTNT{Ps 94:7.}!
\VS{12}Lève-toi, ô Yahweh ! Lève ta main ! N'oublie pas les malheureux !
\VS{13}Pourquoi le méchant méprise-t-il Dieu ? Il dit en son cœur que tu ne punis pas.
\VS{14}Tu regardes cependant, car tu vois la peine et la souffrance, pour prendre en main leur cause ; c'est toi qui viens en aide à l'orphelin.
\VS{15}Brise le bras du méchant, punis ses iniquités et qu'il disparaisse à tes yeux !
\VS{16}Yahweh est Roi à toujours et à perpétuité\FTNT{Ps. 29:10 ; Ps. 145:13 ; Ps. 146:10 ; La. 5:19.} ; les nations sont exterminées de sa terre.
\VS{17}Tu entends les vœux de ceux qui souffrent, ô Yahweh ! Tu affermis leur cœur; tu prêtes l'oreille
\VS{18}pour rendre justice à l'orphelin et à l'opprimé, afin que l'homme mortel tiré de la terre cesse d'inspirer l'effroi.
\Chap{11}
\TextTitle{Yahweh, le refuge des hommes droits}
\VerseOne{}Psaume de David, donné au chef des chantres. C'est en Yahweh que je cherche un refuge. Comment un homme peut-il dire à mon âme : Fuis dans vos montagnes, comme un oiseau ?
\VS{2}En effet, les méchants bandent l'arc\FTNT{Ps. 37:14.}, ils ajustent leur flèche sur la corde, pour tirer dans l'ombre sur ceux dont le cœur est droit.
\VS{3}Quand les fondements sont renversés, que fera le juste ?
\VS{4}Yahweh est dans son saint temple, Yahweh a son trône dans les cieux ; ses yeux contemplent, ses paupières sondent les fils des hommes.
\VS{5}Yahweh sonde le juste et le méchant ; et son âme hait celui qui aime la violence.
\VS{6}Il fait pleuvoir sur les méchants des charbons, du feu et du soufre\FTNT{Ez. 38:22.} ; un vent brûlant, c'est le calice qu'ils ont en partage.
\VS{7}Car Yahweh est juste, il aime la justice ; les hommes droits contemplent sa face.
\Chap{12}
\TextTitle{Le langage des lèvres arrogantes}
\VerseOne{}Psaume de David, donné au chef des chantres pour le chanter sur Sheminith.
\VS{2}Sauve, ô Yahweh ! Car les hommes pieux s'en vont, les fidèles disparaissent parmi les fils de l'homme.
\VS{3}Chacun dit des faussetés à son compagnon avec des lèvres flatteuses et ils parlent avec un cœur double.
\VS{4}Que Yahweh retranche toutes les lèvres flatteuses, la langue qui parle fièrement\FTNT{Ps. 17:10.},
\VS{5}parce qu'ils disent : Nous sommes puissants par nos langues, nous avons nos lèvres avec nous ; qui serait notre maître ?
\VS{6}A cause du mauvais traitement que l'on fait aux malheureux, à cause du gémissement des pauvres, je me lèverai maintenant, dit Yahweh, je mettrai en sûreté celui à qui l'on tend des pièges.
\VS{7}Les paroles de Yahweh sont des paroles pures, c'est un argent éprouvé sur terre au creuset\FTNT{Ps. 19:10 ; Ps. 119:140 ; Pr. 30:5.}, et sept fois épuré.
\VS{8}Toi, Yahweh ! Garde-les, préserve cette race à jamais.
\VS{9}Les méchants se promènent de toutes parts, tandis que des gens abjects sont élevés parmi les fils des hommes.
\Chap{13}
\TextTitle{Savoir attendre le secours de Dieu}
\VerseOne{}Psaume de David, donné au chef des chantres.
\VS{2}Yahweh, jusqu'à quand m'oublieras-tu ? Sera-ce pour toujours ? Jusqu'à quand me cacheras-tu ta face\FTNT{Ps. 10:1 ; Ps. 27:9.} ?
\VS{3}Jusqu'à quand consulterai-je mon âme, affligerai-je mon cœur tous les jours ? Jusqu'à quand mon ennemi s'élèvera-t-il contre moi ?
\VS{4}Yahweh, mon Dieu ! Regarde, exauce-moi, illumine mes yeux, de peur que je ne dorme du sommeil de la mort,
\VS{5}de peur que mon ennemi ne dise : J'ai eu le dessus! Que mes adversaires ne se réjouissent, si je venais à tomber\FTNT{Ps. 25:2.}.
\VS{6}Mais moi, je me confie en ta bonté, mon cœur se réjouira de la délivrance que tu m'auras donnée ; je chanterai à Yahweh, parce qu'il m'a fait du bien.
\Chap{14}
\TextTitle{L'insensé ne cherche pas Dieu}
\VerseOne{}Psaume de David, donné au chef des chantres. L'insensé dit en son cœur : Il n'y a point de Dieu\FTNT{Ceux qui ne croient pas en l'existence de Dieu sont appelés insensés. En effet, la création révèle l'existence du Créateur (Ro. 1:19-20).} ! Ils se sont corrompus, ils ont commis des actions abominables ; il n'y a personne qui fasse le bien.
\VS{2}Yahweh regarde des cieux les fils de l'homme, pour voir s'il y a quelqu'un qui soit intelligent, qui cherche Dieu\FTNT{Ps. 33:13 ; Job. 28:24.}.
\VS{3}Ils se sont tous égarés, ils se sont tous ensemble rendus odieux, il n'y a personne qui fasse le bien, pas même un seul\FTNT{Tous les hommes naissent pécheurs (Ro. 3:10-23).}.
\VS{4}Tous ces ouvriers d'iniquité n'ont-ils point de connaissance ? Ils dévorent mon peuple, ils le prennent pour nourriture ; ils n'invoquent point Yahweh.
\VS{5}Là, ils seront saisis d'une grande frayeur, car Dieu est avec la race des justes.
\VS{6}Jetez l'opprobre sur l'espérance du malheureux…Yahweh est son refuge.
\VS{7}Oh ! Qui fera partir de Sion la délivrance d'Israël\FTNT{C'est le Messie qui délivrera Israël (Ro. 11:25-27).} ? Quand Yahweh ramenera son peuple captif, Jacob se réjouira, Israël se réjouira.
\Chap{15}
\TextTitle{L'homme que Yahweh agrée}
\VerseOne{}Psaume de David. Yahweh, qui séjournera dans ta tente ? Qui demeurera sur ta montagne sainte\FTNT{Ps. 24:3-4.} ?
\VS{2}Celui qui marche dans l'intégrité, qui fait ce qui est juste, et qui profère la vérité telle qu'elle est dans son cœur,
\VS{3}qui ne calomnie point avec sa langue, qui ne fait point de mal à son ami, qui ne diffame point son prochain.
\VS{4}Il regarde avec dédain celui qui est méprisable, mais il honore ceux qui craignent Yahweh; il ne se rétracte point s'il fait un serment à son préjudice.
\VS{5}Il n'exige point d'intérêt de son argent, et il n'accepte point de présent contre l'innocent\FTNT{Lé. 25:36 ; De. 16:19 ; De. 27:25.}. Celui qui fait ces choses ne sera jamais ébranlé.
\Chap{16}
\TextTitle{Yahweh, la source de la vie}
\VerseOne{}Mictam de David. Garde-moi, ô Dieu ! Car je cherche en toi mon refuge.
\VS{2}Je dis à Yahweh : Tu es mon Seigneur, tu es mon bonheur!
\VS{3}Les saints qui sont dans le pays, les hommes pieux sont l'objet de toute mon affection.
\VS{4}On multiplie les peines, on court après les dieux étrangers : Je ne répands pas leurs libations de sang et je ne mets pas leurs noms sur mes lèvres.
\VS{5}Yahweh est la part de mon héritage et ma coupe ; tu maintiens mon lot;
\VS{6}un héritage délicieux m'est échu, une belle possession m'est accordée.
\VS{7}Je bénirai Yahweh qui me donne conseille; je le bénirai même durant les nuits dans lesquelles mes reins m'enseignent.
\VS{8}J'ai constamment Yahweh sous mes yeux ; quand il est à ma droite, je ne chancelle pas\FTNT{Ps. 109:31 ; Ps. 110:5 ; Ac. 2:25.}.
\VS{9}C'est pourquoi mon cœur se réjouit, mon esprit se réjouit et mon corps repose en sécurité.
\VS{10}Car tu n'abandonneras point mon âme au scheol, tu ne permettras point que ton bien-aimé voie la corruption\FTNT{Le roi David prophétise ici la résurrection du Messie.}.
\VS{11}Tu me feras connaître le chemin de la vie ; il y a d'abondantes joies devant ta face, des délices éternels à ta droite.
\Chap{17}
\TextTitle{L'assurance en Dieu}
\VerseOne{}Prière de David. Yahweh, écoute la droiture, sois attentif à mon cri, prête l'oreille à ma prière faite avec des lèvres sans tromperie !
\VS{2}Que ma justice paraisse devant ta face, que tes yeux contemplent mon intégrité !
\VS{3}Tu as sondé mon cœur\FTNT{Ps. 139:1 ; Jé. 12:3.}, tu l'as visité de nuit, tu m'as examiné, tu n'as rien trouvé : Ma pensée ne va point au-delà de ma parole.
\VS{4}Quant aux actions des hommes, selon la parole de tes lèvres, je me tiens en garde contre la voie du violent.
\VS{5}Mes pas sont fermes dans tes sentiers, mes pieds ne chancellent point.
\VS{6}Je t'invoque, car tu m'exauces, ô Dieu ! Incline ton oreille vers moi, écoute mes paroles !
\VS{7}Signale ta bonté, toi qui sauves ceux qui cherchent un refuge, et qui par ta droite les délivres de leurs adversaires !
\VS{8}Garde-moi comme la prunelle de l'œil, cache-moi à l'ombre de tes ailes\FTNT{Mt. 23:37.},
\VS{9}contre les méchants qui me traitent violemment, mes ardents ennemis qui m'entourent.
\VS{10}Ils sont enfermés dans leur propre graisse, leur bouche parle avec orgueil.
\VS{11}Maintenant, ils nous environnent à chaque pas que nous faisons ; ils jettent leur regard pour nous étendre par terre.
\VS{12}Ils ressemblent au lion qui ne demande qu'à déchirer, et au lionceau qui se tient dans les lieux cachés.
\VS{13}Lève-toi, ô Yahweh, devance-les, renverse-les ! Délivre mon âme du méchant par ton épée.
\VS{14}Yahweh, délivre-moi par ta main de ces gens, des gens de ce monde ! Leur part est dans cette vie et tu remplis leur ventre de tes biens ; leurs enfants sont rassasiés et ils laissent leurs restes à leurs petits-enfants.
\VS{15}Mais moi, dans mon innocence, je verrai ta face\FTNT{Job. 19:26-27 ; Ps. 16:10-11.}, et je me rassasierai de ton image, dès mon réveil.
\Chap{18}
\TextTitle{Louange à Dieu, le bouclier des saints}
\VerseOne{}Psaume de David, serviteur de Yahweh, qui adressa à Yahweh les paroles de ce cantique le jour où Yahweh l'eut délivré de la main de Saül. Au chef des chantres.
\VS{2}Il dit donc : Je t'aime, ô Yahweh, ma force !
\VS{3}Yahweh est mon rocher\FTNT{Yahweh est le rocher sur lequel s'appuyait David. Paul enseigne que ce rocher était Jésus-Christ (1 Co. 10:1-4). Voir commentaire en Es. 8:13-17.}, ma forteresse et mon libérateur ! Mon Dieu, mon rocher où je trouve un refuge ! Mon bouclier, la force qui me sauve, ma haute retraite !
\VS{4}Je crie : Loué soit Yahweh ! Et je suis délivré de mes ennemis.
\VS{5}Les liens de la mort m'avaient environné et des torrents de destruction m'avaient épouvanté.
\VS{6}Les liens du scheol m'avaient entouré, les filets de la mort m'avaient surpris\FTNT{Ps. 116:3.}.
\VS{7}Dans ma détresse, j'ai invoqué Yahweh, j'ai crié à mon Dieu ; il a entendu ma voix de son palais, mon cri est parvenu devant lui à ses oreilles.
\VS{8}La terre fut ébranlée et trembla, les fondements des montagnes croulèrent\FTNT{Es. 5:25 ; Es. 64:1-3 ; Jé. 4:24 ; Ps. 104:32.}, et ils furent ébranlés, parce qu'il était irrité.
\VS{9}Une fumée montait de ses narines, et de sa bouche sortait un feu dévorant, des charbons embrasés.
\VS{10}Il abaissa les cieux et descendit : Il y avait une épaisse nuée sous ses pieds.
\VS{11}Il était monté sur un chérubin, et il volait, il était porté sur les ailes du vent\FTNT{Ps. 104:3.}.
\VS{12}Il faisait des ténèbres sa demeure secrète, autour de lui était sa tente, il était enveloppé des eaux obscures et de sombres nuages.
\VS{13}De la splendeur qui le précédait s'échappaient les nuées, lançant de la grêle et des charbons de feu.
\VS{14}Yahweh tonna dans les cieux, le Très-Haut fit retentir sa voix avec de la grêle et des charbons de feu.
\VS{15}Il tira ses flèches, et écarta mes ennemis, il lança des éclairs et les mit en déroute\FTNT{Ps. 77:18.}.
\VS{16}Le fond des eaux parut, les fondements du monde furent découverts, par ta menace, ô Yahweh ! Par le souffle du vent de tes narines.
\VS{17}Il étendit la main d'en haut, il m'enleva et me retira des grandes eaux\FTNT{2 S. 22:17.} ;
\VS{18}il me délivra de mon puissant ennemi, et de ceux qui me haïssaient, car ils étaient plus forts que moi.
\VS{19}Ils m'avaient surpris au jour de ma détresse, mais Yahweh, fut mon appui.
\VS{20}Il m'a mis au large, il m'a délivré, parce qu'il m'aime.
\VS{21}Yahweh m'a rendu selon ma justice, il m'a traité selon la pureté de mes mains\FTNT{Ps. 18:25 ; Ps. 7:9.},
\VS{22}car j'ai observé les voies de Yahweh et je n'ai point été coupable envers mon Dieu.
\VS{23}Car j'ai eu devant moi toutes ses ordonnances et je ne me suis point écarté de ses lois.
\VS{24}J'ai été intègre envers lui, et je me suis tenu en garde contre mon iniquité.
\VS{25}Aussi Yahweh m'a rendu selon ma justice, selon la pureté de mes mains devant ses yeux,
\VS{26}avec celui qui est bon, tu te montres bon, avec l'homme droit tu agis selon la droiture.
\VS{27}Avec celui qui est pur, tu te montres pur, et avec le pervers tu agis selon sa perversité.
\VS{28}Car tu sauves le peuple affligé et tu abaisses les yeux hautains\FTNT{Es. 2:11 ; Es. 5:15.}.
\VS{29}Tu fais briller ma lumière ; Yahweh, mon Dieu, éclaire mes ténèbres.
\VS{30}Avec toi, je me précipite sur un corps d'armée, avec mon Dieu je franchis la muraille.
\VS{31}Les voies de Dieu sont sans défaut ; la parole de Yahweh est éprouvée\FTNT{De. 32:4 ; Ps. 19:8-9 ; Da. 4:37.} ; il est un bouclier pour tous ceux qui se confient en lui.
\VS{32}Car qui est Dieu, si ce n'est Yahweh ? Et qui est un rocher, si ce n'est notre Dieu ?\FTNT{1 S. 2:2 ; 2 S. 22:32.}
\VS{33}C'est le Dieu qui me ceint de force, et qui me conduit dans la voie droite.
\VS{34}Il rend mes pieds semblables à ceux des biches\FTNT{2 S. 2:18.}, et il me place sur mes lieux élevés.
\VS{35}Il exerce mes mains au combat, tellement qu'un arc d'airain a été rompu avec mes bras.\FTNT{Job. 20:24.}.
\VS{36}Tu me donnes le bouclier de ton salut, ta droite me soutient, et je deviens puissant par ta bonté.
\VS{37}Tu élargis le chemin sous mes pas, et mes pieds ne chancellent point.
\VS{38}Je poursuis mes ennemis, je les atteints, et je ne reviens pas avant de les avoir anéantis.
\VS{39}Je les brise et ils ne peuvent se relever ; ils tombent sous mes pieds.
\VS{40}Car tu m'as ceint de force pour le combat, tu fais plier sous moi ceux qui s'élevaient contre moi.
\VS{41}Tu fais tourner le dos à mes ennemis devant moi, et j'extermine ceux qui me haïssaient.
\VS{42}Ils crient, mais il n'y a point de libérateur! Ils crient à Yahweh, mais il ne leur répond point!
\VS{43}Je les brise comme la poussière qui est dispersée par le vent et je les foule comme la boue des rues.
\VS{44}Tu me délivres des séditions du peuple, tu m'établis chef des nations. Un peuple que je ne connais point m'est asservi.
\VS{45}Ils m'obéissent au premier ordre, les fils de l'étranger me flattent.
\VS{46}Les étrangers s'enfuient et ils tremblent de peur dans leurs forteresses.
\VS{47}Yahweh est vivant, et béni soit mon rocher ! Que le Dieu de mon salut soit exalté !
\VS{48}Le Dieu qui est mon vengeur et qui m'assujettit les peuples,
\VS{49}c'est lui qui me délivre de mes ennemis ! Tu m'élèves au-dessus de mes adversaires, tu me sauves de l'homme violent.
\VS{50}C'est pourquoi, ô Yahweh, je te célébrerai parmi les nations ! Et je chanterai des psaumes à ton Nom.
\VS{51}Il accorde de grandes délivrances à son roi, et il fait miséricorde à son oint, à David, et à sa postérité, pour toujours.
\Chap{19}
\TextTitle{La création exalte la grandeur de Dieu}
\VerseOne{}Psaume de David, donné au chef des chantres.
\VS{2}Les cieux racontent la gloire de Dieu, et l'étendue met en évidence l'oeuvre de ses mains.
\VS{3}Un jour en instruit un autre jour, et une nuit fait connaître sa science à l'autre nuit.
\VS{4}Ce n'est pas un langage, ce ne sont pas des paroles dont le cri ne soit point entendu :
\VS{5}Leur retentissement couvre toute la terre, et leur voix est allée jusqu'aux extrémités du monde\FTNT{Ro. 10:18.}. Il a dressé une tente pour le soleil.
\VS{6}Et le soleil est semblable à un époux sortant de sa chambre ; il s'élance sur le sentier avec la joie d'un homme vaillant;
\VS{7}il se lève à l'extrémité des cieux et achève sa course à l'autre extrémité\FTNT{Ec. 1:5.}: Rien ne se dérobe à sa chaleur.
\VS{8}La loi de Yahweh est parfaite, elle restaure l'âme ; le témoignage de Yahweh est fidèle, il donne la sagesse au simple\FTNT{2 S. 22:31 ; Ps. 18:31 ; Ps. 119:130.}.
\VS{9}Les ordonnances de Yahweh sont droites, elles réjouissent le cœur ; les commandements de Yahweh sont purs, ils éclairent les yeux.
\VS{10}La crainte de Yahweh est pure, elle subsiste à toujours ; les jugements de Yahweh sont vrais, et ils sont tous justes.
\VS{11}Ils sont plus précieux que l'or, que beaucoup d'or fin ; et plus doux que le miel, que celui qui coule des rayons de miel\FTNT{Ps. 119:103.}.
\VS{12}Ton serviteur aussi en reçoit l'éclairage ; pour qui les observe la récompense est grande.
\VS{13}Qui connaît ses fautes commises par erreur ? Purifie-moi de mes fautes cachées.
\VS{14}Eloigne aussi ton serviteur des actions commises par fierté, en sorte qu'elles ne dominent point sur moi, qu'elles cessent et que je sois nettoyé de mes grands péchés!
\VS{15}Que les propos de ma bouche et la méditation de mon cœur te soient agréables, ô Yahweh ! Mon rocher et mon rédempteur\FTNT{Voir commentaire en Es. 60:16.}!
\Chap{20}
\TextTitle{Recours à l'intervention de Dieu}
\VerseOne{}Psaume de David, donné au chef des chantres.
\VS{2}Que Yahweh te réponde au jour de la détresse, que le Nom du Dieu de Jacob te protège !
\VS{3}Qu'il envoie ton secours du saint lieu, et qu'il te soutienne de Sion !
\VS{4}Qu'il se souvienne de toutes tes offrandes, qu'il réduise en cendres ton holocauste ! Sélah.
\VS{5}Qu'il te donne ce que ton cœur désire, et qu'il fasse réussir tes desseins !
\VS{6}Nous triompherons dans ton salut, nous lèverons la bannière au Nom de notre Dieu ; Yahweh exaucera tous tes vœux.
\VS{7}Je sais déjà que Yahweh sauve son oint ; il l'exaucera des cieux, de sa sainte demeure, par le secours puissant de sa droite.
\VS{8}Les uns se vantent de leurs chars, et les autres de leurs chevaux ; mais nous, nous glorifierons le Nom de Yahweh, notre Dieu.
\VS{9}Eux ils plient, et ils tombent ; nous, nous tenons ferme, et restons debout.
\VS{10}Yahweh, sauve le roi ! Qu'il nous réponde quand nous crions à lui !
\Chap{21}
\TextTitle{La protection de Dieu sur le roi}
\VerseOne{}Psaume de David, donné au chef des chantres.
\VS{2}Yahweh, le roi se réjouit de ta puissance, ton secours le remplit d'allégresse !
\VS{3}Tu lui as donné ce que désirait son cœur et tu n'as point refusé ce que demandaient ses lèvres. Sélah.
\VS{4}Car tu l'as prévenu par les bénédictions de ta bonté, et tu as mis sur sa tête une couronne d'or pur.
\VS{5}Il t'avait demandé la vie, et tu la lui as donnée, une vie longue pour toujours et à perpétuité.
\VS{6}Sa gloire est grande à cause de ton salut, tu l'as couvert de majesté et d'honneur.
\VS{7}Tu le rends à jamais un objet de bénédictions, tu le combles de joie devant ta face\FTNT{Ps. 16:11.}.
\VS{8}Le roi se confie en Yahweh, et par la bonté du Très-Haut, il ne chancelle pas\FTNT{Ps. 16:8.}.
\VS{9}Ta main trouvera tous tes ennemis, ta droite trouvera tous ceux qui te haïssent.
\VS{10}Tu les rendras tels qu'une fournaise ardente le jour où l'on verra ta face ; Yahweh les engloutira dans sa colère, et le feu les consumera.
\VS{11}Tu feras périr leur fruit de la terre et leur race du milieu des fils des hommes.
\VS{12}Car ils ont projeté du mal contre toi et ils ont conçu de mauvais desseins dont ils ne pourront venir à bout.
\VS{13}Parce que tu leur feras tourner le dos, et avec ton arc tu tireras sur eux.
\VS{14}Elève-toi, Yahweh, par ta force ! Nous chanterons et célébrerons ta puissance.
\Chap{22}
\TextTitle{Les souffrances du Messie}
\VerseOne{}Psaume de David, donné au chef des chantres, pour le chanter sur Ajéleth-Hashakhar.
\VS{2}Mon Dieu ! Mon Dieu ! Pourquoi m'as-tu abandonné\FTNT{Le Ps. 22 est une description détaillée de la mort par crucifixion du Seigneur Jésus-Christ (Mt. 27:45-46).}, et t'éloignes-tu sans me secourir, sans écouter mes plaintes ?
\VS{3}Mon Dieu ! Je crie le jour, mais tu ne réponds point ; la nuit, et je n'ai point de repos.
\VS{4}Pourtant tu es le Saint, tu habites au milieu des louanges d'Israël.
\VS{5}Nos pères se sont confiés en toi ; ils se sont confiés, et tu les as délivrés.
\VS{6}Ils ont crié vers toi, et ils ont été délivrés ; ils se sont appuyés sur toi, et ils n'ont point été confus\FTNT{Es. 49:23 ; Ps. 25:3 ; Ps. 31:2.}.
\VS{7}Et moi, je suis un ver et non un homme, l'opprobre des hommes et le méprisé du peuple\FTNT{Es. 53:2-3.}.
\VS{8}Tous ceux qui me voient se moquent de moi, ils ouvrent les lèvres, secouent la tête\FTNT{Ps. 109:25 ; Mt. 27:39.} :
\VS{9}Recommande-toi à Yahweh ! Qu'il te délivre, et qu'il te sauve, puisqu'il prend plaisir en toi\FTNT{Mt. 27:43.} !
\VS{10}Cependant, c'est toi qui m'as tiré hors du ventre de ma mère, qui m'as mis en sûreté lorsque j'étais sur les mamelles de ma mère.
\VS{11}J'ai été sous ta garde, dès le sein maternel, tu as été mon Dieu dès le ventre de ma mère\FTNT{Es. 49:1.}.
\VS{12}Ne t'éloigne point de moi, car la détresse est près de moi, et il n'y a personne qui me secoure\FTNT{Ps. 69:21.} !
\VS{13}Plusieurs taureaux sont autour de moi, de puissants taureaux de Basan m'entourent.
\VS{14}Ils ouvrent leur gueule contre moi, comme un lion qui déchire et rugit.
\VS{15}Je suis comme de l'eau qui s'écoule, et tous mes os se séparent ; mon cœur est comme de la cire, il se fond dans mes entrailles.
\VS{16}Ma force se dessèche comme l'argile, et ma langue s'attache à mon palais ; tu me réduis à la poussière de la mort.
\VS{17}Car des chiens m'environnent, une assemblée de méchants m'entoure, ils ont percé mes mains et mes pieds.
\VS{18}Je pourrais compter tous mes os un par un. Eux, ils m'examinent, ils me regardent.
\VS{19}Ils se partagent mes vêtements et tirent au sort ma tunique\FTNT{Mt. 27:35 ; Mc. 15:24 ; Lu. 23:33.}.
\VS{20}Et toi, Yahweh, ne t'éloigne point ! Ma force, hâte-toi de me secourir !
\VS{21}Délivre ma vie de l'épée, ma vie contre le pouvoir des chiens !
\VS{22}Sauve-moi de la gueule du lion, délivre-moi des cornes du buffle !
\VS{23}Je déclarerai ton Nom à mes frères, je te louerai au milieu de l'assemblée\FTNT{Hé. 2:12.}.
\VS{24}Vous qui craignez Yahweh, louez-le ! Toute la race de Jacob, glorifiez-le ! Toute la race d'Israël, redoutez-le !
\VS{25}Car il n'a ni mépris ni dédain pour les peines du misérable, et il ne lui cache point sa face, mais il l'écoute quand il crie à lui.
\VS{26}Tu seras l'objet de mes louanges dans la grande assemblée ; j'accomplirai mes vœux en présence de ceux qui te craignent\FTNT{Ps. 56:13.}.
\VS{27}Les malheureux mangeront et seront rassasiés, ceux qui cherchent Yahweh le loueront. Votre cœur vivra à perpétuité !
\VS{28}Toutes les extrémités de la terre s'en souviendront, ils se convertiront à Yahweh, et toutes les familles des nations se prosterneront devant toi\FTNT{Ps. 72:8-11 ; Ps. 86:9.}.
\VS{29}Car le règne appartient à Yahweh : Il domine sur les nations.
\VS{30}Tous les gens de la terre mangeront et se prosterneront devant lui ; tous ceux qui descendent dans la poussière s'inclineront, même celui qui ne peut conserver sa vie.
\VS{31}La postérité le servira, on parlera du Seigneur de génération en génération\FTNT{Es. 59:21 ; Es. 65:23 ; Ps. 110:3.}.
\VS{32}Ils viendront et ils publieront sa justice au peuple qui naîtra, parce qu'il aura fait ces choses.
\Chap{23}
\TextTitle{Le bon Berger}
\VerseOne{}Psaume de David. Yahweh est mon berger\FTNT{Yahweh, le bon berger, est notre Seigneur Jésus-Christ. Es. 40:11 ; Jé. 23:4 ; Jn. 10:11.}: Je ne manquerai de rien.
\VS{2}Il me fait reposer dans de verts pâturages, il me dirige près des eaux paisibles.
\VS{3}Il restaure mon âme, et me conduit dans les sentiers de la justice, à cause de son Nom.
\VS{4}Quand je marche dans la vallée de l'ombre de la mort, je ne crains aucun mal\FTNT{Ps. 118:6.}, car tu es avec moi : Ton bâton et ta houlette me consolent.
\VS{5}Tu dresses devant moi une table, en face de mes adversaires; tu oins d'huile ma tête et ma coupe déborde.
\VS{6}Le bonheur et la grâce m'accompagneront tous les jours de ma vie, et j'habiterai dans la maison de Yahweh jusqu'à la fin de mes jours.
\Chap{24}
\TextTitle{Accueil de Yahweh, le Roi de gloire}
\VerseOne{}Psaume de David. La terre appartient à Yahweh, avec tout ce qui est en elle\FTNT{Ex. 19:5 ; De. 10:14 ; Ps. 50:12 ; Job. 41:2 ; 1 Co. 10:26.}, le monde et ceux qui y habitent!
\VS{2}Car il l'a fondée sur les mers, et affermie sur les fleuves.
\VS{3}Qui pourra monter à la montagne de Yahweh ? Qui s'élèvera jusqu'à son lieu saint\FTNT{Ps. 15:1-2 ; Ps. 118:19.} ?
\VS{4}Celui qui a les mains pures et le cœur pur, qui ne livre point son âme au mensonge, et qui ne jure pas pour tromper.
\VS{5}Il obtiendra la bénédiction de Yahweh et la justice du Dieu de son salut.
\VS{6}Voilà le partage de la génération qui l'invoque, de ceux qui cherchent ta face de Jacob ! Sélah.
\VS{7}Portes, élevez vos linteaux; élevez-vous portes éternelles ! Que le Roi de gloire fasse son entrée !
\VS{8}Qui est ce Roi de gloire ? C'est Yahweh fort et puissant, Yahweh puissant dans les combats.
\VS{9}Portes, élevez vos linteaux; élevez-les aussi, vous portes éternelles! Que le Roi de gloire fasse son entrée !
\VS{10}Qui est ce Roi de gloire ? Yahweh des armées : Voilà le Roi de gloire! Sélah.
\Chap{25}
\TextTitle{La crainte de Dieu mène à la voie de Yahweh}
\VerseOne{}Psaume de David. [Aleph.] Yahweh, j'élève mon âme à toi.
\VS{2}[Beth.] Mon Dieu ! Je me confie en toi : Que je ne sois point honteux\FTNT{Ps. 22:5 ; Ps. 31:2.} ! Que mes ennemis ne triomphent point de moi!
\VS{3}[Guimel.] Tous ceux qui espèrent en toi ne seront point confus\FTNT{Ro. 10:11.} ; ceux qui agissent avec tromperie sans cause seront honteux.
\VS{4}[Daleth.] Yahweh ! Fais-moi connaître tes voies, enseigne-moi tes sentiers\FTNT{Ps. 27:11 ; Ps. 86:11 ; Ps. 143:10.}.
\VS{5}[He. Vav.] Fais-moi marcher selon la vérité, et instruis-moi, car tu es le Dieu de ma délivrance, je m'attends à toi tous les jours.
\VS{6}[Zayin.] Yahweh ! Souviens-toi de ta miséricorde et de ta bonté, car elles sont éternelles\FTNT{Jé. 33:11 ; Ps. 103:17 ; Ps. 106:1 ; Ps. 107:1 ; Ps. 117:2 ; Ps. 136:1-2.}.
\VS{7}[Heth.] Ne te souviens point des péchés de ma jeunesse ni de mes transgressions ; souviens-toi de moi selon ta miséricorde, à cause de ta bonté, ô Yahweh !
\VS{8}[Teth.] Yahweh est bon et droit : C'est pourquoi il enseigne aux pécheurs la voie.
\VS{9}[Yod.] Il conduit les humbles dans la justice, et il leur enseigne sa voie.
\VS{10}[Kaf.] Tous les sentiers de Yahweh sont miséricorde et fidélité, pour ceux qui gardent son alliance et son témoignage.
\VS{11}[Lamed.] Pour l'amour de ton Nom, ô Yahweh ! Tu me pardonneras mon iniquité, car elle est grande\FTNT{2 S. 24:10.}.
\VS{12}[Mem.] Qui est l'homme qui craint Yahweh ? Yahweh lui enseignera la voie qu'il doit choisir.
\VS{13}[Nun.] Son âme demeurera dans le bonheur, et sa postérité possédera la terre en héritage.
\VS{14}[Samech.] Le secret de Yahweh est pour ceux qui le craignent, et son alliance leur donne le savoir.
\VS{15}[Ayin.] Mes yeux sont continuellement sur Yahweh, car c'est lui qui sortira mes pieds du filet.
\VS{16}[Pe.] Tourne ta face vers moi, et aie pitié de moi, car je suis seul et affligé.
\VS{17}[Tsade.] Les angoisses de mon cœur augmentent ; sors-moi de ma détresse.
\VS{18}[Resh.] Vois ma misère et ma peine, et pardonne tous mes péchés.
\VS{19}[Resh.] Vois combien mes ennemis sont nombreux, et me haïssent d'une haine pleine de violence\FTNT{Jn. 15:25.}.
\VS{20}[Shin.] Garde mon âme et délivre-moi ! Que je ne sois point confus, car je me suis réfugié en toi!
\VS{21}[Tav.] Que l'innocence et la droiture me protègent, car je m'attends à toi!
\VS{22}[Pe.] Ô Dieu ! Rachète Israël de toutes ses détresses !
\Chap{26}
\TextTitle{Demeurer dans l'intégrité}
\VerseOne{}Psaume de David. Yahweh, rends-moi justice\FTNT{Ps. 43:1 ; Ps. 54:3.} ! Car je marche dans l'intégrité, je me confie en Yahweh, je ne chancelle pas.
\VS{2}Sonde-moi et éprouve-moi\FTNT{Ps. 11:4-5 ; Ps. 17:3 ; Ps. 139:23.}, Yahweh ! Fais passer au creuset mes reins et mon cœur ;
\VS{3}car ta grâce est devant mes yeux, et je marche dans ta vérité.
\VS{4}Je ne m'assieds pas avec les hommes faux\FTNT{Ps. 1:1 ; 1 Co. 5:9-11 ; 1 Co. 15:33.}, et je ne vais point avec les gens dissimulés.
\VS{5}Je hais la compagnie de ceux qui font le mal\FTNT{Ps. 101:2-5 ; Ps. 119:113.}, et je ne m'assieds pas avec les méchants.
\VS{6}Je lave mes mains dans l'innocence et je fais le tour de ton autel\FTNT{Ps. 73:13.}, ô Yahweh !
\VS{7}Pour faire entendre le cri de reconnaissance, et pour raconter toutes tes merveilles.
\VS{8}Yahweh, j'aime la demeure de ta maison, le lieu dans lequel est le tabernacle de ta gloire.
\VS{9}N'enlève pas mon âme avec les pécheurs, ma vie avec les hommes de sang,
\VS{10}dont les mains sont criminelles, et la droite pleine de présents.
\VS{11}Moi, je marche dans l'intégrité ; délivre-moi et aie pitié de moi !
\VS{12}Mon pied se tient dans la droiture ; je bénirai Yahweh dans les assemblées.
\Chap{27}
\TextTitle{La foi qui triomphe des épreuves}
\VerseOne{}Psaume de David. Yahweh est ma lumière\FTNT{Es. 60:19-20 ; Mi. 7:8 ; Ps. 118:6 ; Jn. 8:12 ; Ap. 21:23.} et mon salut : De qui aurai-je peur ? Yahweh est le soutien de ma vie : De qui aurai-je peur ?
\VS{2}Lorsque les méchants s'avancent contre moi pour dévorer ma chair, ce sont mes adversaires et mes ennemis qui chancellent et tombent.
\VS{3}Si toute une armée campait contre moi, mon cœur ne craindrait point ; si une guerre s'élevait contre moi, je serai plein de confiance.
\VS{4}Je demande une chose à Yahweh, que je désire ardemment : C'est d'habiter dans la maison de Yahweh tous les jours de ma vie, pour contempler la beauté de Yahweh et pour admirer son temple.
\VS{5}Car il me cachera dans son tabernacle au jour du malheur, il me tiendra caché sous l'abri de sa tente; il m'élèvera sur un rocher.
\VS{6}Même maintenant ma tête s'élève par-dessus mes ennemis qui m'entourent ; et j'offrirai des sacrifices dans sa tente, au son de la trompette; je chanterai et célèbrerai Yahweh.
\VS{7}Yahweh ! Ecoute ma voix, je t'invoque : Aie pitié de moi et exauce-moi !
\VS{8}Mon cœur dit de ta part : Cherche ma face ! Je chercherai ta face, ô Yahweh !
\VS{9}Ne me cache point ta face, ne rejette point avec colère ton serviteur ! Tu es mon secours, ne me laisse pas, ne m'abandonne pas, Dieu de mon salut !
\VS{10}Car mon père et ma mère m'abandonnent, mais Yahweh me recueillera\FTNT{Es. 49:15.}.
\VS{11}Yahweh, enseigne-moi ta voie, et conduis-moi dans le sentier de la droiture, à cause de mes ennemis\FTNT{Ps. 5:9 ; Ps. 25:4-5.}.
\VS{12}Ne me livre pas au désir de mes adversaires, car s'élèvent contre moi de faux témoins et des gens qui ne respirent que la violence.
\VS{13}Oh ! Si je n'étais pas sûr de voir la bonté de Yahweh sur la terre des vivants…
\VS{14}Espère en Yahweh ! Fortifie-toi et que ton cœur s'affermisse\FTNT{Es. 33:2 ; Ps. 31:25.} ! Espère en Yahweh !
\Chap{28}
\TextTitle{Louange à Yahweh, le Rocher de son peuple}
\VerseOne{}Psaume de David. Je crie à toi, ô Yahweh ! Mon rocher! Ne te rends point sourd envers moi, de peur que si tu ne me réponds pas, je ne sois semblable à ceux qui descendent dans la fosse\FTNT{Ps. 4:2 ; Ps. 143:7. 
Voir commentaire en Es. 8:13-17.}.
\VS{2}Ecoute la voix de mes supplications, lorsque je crie à toi, quand j'élève mes mains vers ton saint sanctuaire.
\VS{3}Ne m'emporte pas avec les méchants ni avec les ouvriers d'iniquité, qui parlent de paix avec leur prochain pendant que la malice est dans leur cœur\FTNT{Jé. 9:8 ; Ps. 26:9.}.
\VS{4}Traite-les selon leurs œuvres et selon la malice de leurs actions, traite-les selon l'ouvrage de leurs mains, rends-leur ce qu'ils ont mérité\FTNT{2 Ti. 4:14.}.
\VS{5}Parce qu'ils ne prennent point garde aux œuvres de Yahweh, à l'œuvre de ses mains. Qu'il les renverse et ne les édifie point !
\VS{6}Béni soit Yahweh ! Car il exauce la voix de mes supplications.
\VS{7}Yahweh est ma force et mon bouclier ; mon cœur se confie en lui, et je suis secouru ; mon cœur se réjouit, c'est pourquoi je le loue par mes chants.
\VS{8}Yahweh est la force de son peuple, il est le refuge des délivrances de son oint.
\VS{9}Sauve ton peuple et bénis ton héritage ! Nourris-les et élève-les éternellement.
\Chap{29}
\TextTitle{La suprématie de Dieu}
\VerseOne{}Psaume de David. Fils de Dieu, rendez à Yahweh, rendez à Yahweh la gloire et la force\FTNT{Ps. 96:7-8.} !
\VS{2}Rendez à Yahweh la gloire due à son Nom ! Prosternez-vous devant Yahweh avec des ornements sacrés !
\VS{3}La voix de Yahweh est sur les eaux, le Dieu de gloire fait tonner ; Yahweh est sur les grandes eaux.
\VS{4}La voix de Yahweh est forte, la voix de Yahweh est majestueuse.
\VS{5}La voix de Yahweh brise les cèdres, Yahweh brise les cèdres du Liban,
\VS{6}il les fait sauter comme un veau, le Liban et le Sirion comme de jeunes buffles.
\VS{7}La voix de Yahweh fait jaillir des flammes de feu.
\VS{8}La voix de Yahweh fait trembler le désert; Yahweh fait trembler le désert de Kadès.
\VS{9}La voix de Yahweh fait naître les biches, et dépouille les forêts. Dans son palais tout s'écrie : Gloire !
\VS{10}Yahweh était assis lors du déluge ; Yahweh est assis comme roi éternellement\FTNT{Ps. 146:10.}.
\VS{11}Yahweh donne de la force à son peuple ; Yahweh bénit son peuple en paix.
\Chap{30}
\TextTitle{De la délivrance découle la louange}
\VerseOne{}Psaume. Cantique pour la dédicace de la maison de David.
\VS{2}Yahweh, je t'exalte parce que tu m'as relevé, tu n'as pas voulu que mes ennemis se réjouissent à mon sujet.
\VS{3}Yahweh, mon Dieu ! J'ai crié à toi, et tu m'as guéri.
\VS{4}Yahweh ! Tu as fait remonter mon âme du scheol, tu m'as rendu la vie, afin que je ne descende point dans la fosse.
\VS{5}Chantez à Yahweh, vous ses bien-aimés, et célébrez la mémoire de sa sainteté\FTNT{Ps. 97:12.} !
\VS{6}Car sa colère dure un instant, mais sa grâce toute la vie. Le soir arrivent les pleurs, et le matin les cris de louange.
\VS{7}Dans ma sécurité, je disais : Je ne serai jamais ébranlé\FTNT{Ps. 10:6.} !
\VS{8}Yahweh ! Par ta faveur tu avais affermi ma montagne… Tu cachas ta face, et je fus terrifié\FTNT{Ps. 13:2 ; Ps. 88:15 ; Ps. 102:3 ; Ps. 143:7.}.
\VS{9}Yahweh, j'ai crié à toi, j'ai présenté ma supplication à Yahweh :
\VS{10}Que gagnes-tu à verser mon sang si je descends dans la fosse ? La poussière te célébrera-t-elle ? Racontera-t-elle ta fidélité\FTNT{Es. 38:18.} ?
\VS{11}Yahweh, écoute, et aie pitié de moi ! Yahweh, secours-moi !
\VS{12}Tu as changé mon deuil en allégresse, tu as détaché mon sac, et tu m'as ceint de joie,
\VS{13}afin que ma langue te loue\FTNT{Ps. 57:10.} et ne se taise point. Yahweh, mon Dieu ! Je te célébrerai toujours.
\Chap{31}
\TextTitle{Recherche de la protection divine}
\VerseOne{}Psaume de David. Au chef des chantres.
\VS{2}Yahweh ! Tu es mon refuge : Que je ne sois jamais confus ! Délivre-moi par ta justice\FTNT{Ps. 25:2-20 ; Ps. 71:1-2.} !
\VS{3}Incline ton oreille vers moi, hâte-toi de me délivrer ! Sois pour moi un rocher protecteur, une forteresse, afin que je puisse m'y sauver !
\VS{4}Car tu es mon rocher, ma forteresse ; tu me dirigeras et tu me donneras du repos, à cause de ton Nom.
\VS{5}Tire-moi du filet qu'ils m'ont tendu en secret, car tu es ma vigueur.
\VS{6}Je remets mon esprit entre tes mains\FTNT{Lu. 23:46.} ; tu me rachèteras, Yahweh, Dieu de vérité !
\VS{7}Je hais ceux qui s'adonnent aux vanités trompeuses, et je me confie en Yahweh.
\VS{8}Je serai par ta bonté dans l'allégresse et dans la joie ; car tu vois mon affliction, tu sais les angoisses de mon âme,
\VS{9}tu ne m'as pas livré entre les mains de l'ennemi, mais tu feras tenir mes pieds au large.
\VS{10}Yahweh, aie pitié de moi! Car je suis dans la détresse. Mes yeux, mon âme et mon corps dépérissent de chagrin\FTNT{Ps. 6:8 ; Ps. 88:10.}.
\VS{11}Ma vie se consume dans la douleur, et mes années dans les soupirs ; ma force chancelle à cause de mon iniquité, et mes os sont consumés.
\VS{12}J'ai été un objet d'opprobre à cause de tous mes adversaires, de grand opprobre pour mes voisins, et de terreur pour ceux qui me connaissent ; ceux qui me voient dehors s'enfuient loin de moi\FTNT{Ps. 38:12 ; Job. 19:13-14.}.
\VS{13}Je suis oublié des cœurs comme un mort, je suis comme un vase détruit.
\VS{14}J'entends les calomnies de plusieurs, la crainte m'environne, quand ils se concertent unis contre moi : Ils projettent de m'ôter la vie\FTNT{Jé. 20:10.}.
\VS{15}Toutefois, je me confie en toi, ô Yahweh ! Je dis : Tu es mon Dieu !
\VS{16}Ma destinée est entre tes mains ; délivre-moi de la main de mes ennemis et de ceux qui me poursuivent !
\VS{17}Fais luire ta face sur ton serviteur\FTNT{Ps. 4:7 ; Ps. 67:2.}, délivre-moi par ta bonté !
\VS{18}Yahweh, que je ne sois point confus puisque je t'ai invoqué. Que les méchants soient confus, qu'ils soient couchés dans le scheol !
\VS{19}Que les lèvres menteuses soient muettes, elles profèrent des paroles dures contre le juste, avec orgueil et avec mépris!
\VS{20}Que ta bonté est grande\FTNT{Ps. 36:6.} ! Toi qui la réserves pour ceux qui te craignent, tu leur fais un refuge à la vue des fils de l'homme !
\VS{21}Tu les caches sous l'abri de ta face, loin du complot des hommes, tu les caches sous ton abri contre les langues querelleuses.
\VS{22}Béni soit Yahweh ! Car il a rendu merveilleuse sa bonté envers moi, comme si j'avais été dans une ville retranchée.
\VS{23}Je disais dans ma précipitation : Je suis retranché loin de ton regard ! Mais tu as entendu la voix de mes supplications quand j'ai crié vers toi.
\VS{24}Aimez Yahweh, vous tous, ses bien-aimés! Yahweh garde les fidèles, et il punit sévèrement les orgueilleux.
\VS{25}Fortifiez-vous et que votre esprit s'affermisse, espérez en Yahweh\FTNT{Ps. 27:14.} !
\Chap{32}
\TextTitle{La puissance du pardon}
\VerseOne{}Cantique de David. Heureux celui à qui la transgression est pardonnée, et dont le péché est couvert !
\VS{2}Heureux l'homme à qui Yahweh n'impute point son iniquité\FTNT{Ro. 4:6-8.}, et dans l'esprit duquel il n'y a point de fraude !
\VS{3}Quand je me suis tu, mes os se sont consumés, je n'ai fait que gémir tout le jour;
\VS{4}parce que jour et nuit ta main s'appesantissait sur moi\FTNT{Ps. 38:3.}, ma vigueur s'est changée en une sécheresse d'été. Sélah.
\VS{5}Je t'ai fait connaître mon péché et je n'ai point caché mon iniquité; j'ai dit : J'avouerai mes transgressions à Yahweh\FTNT{Pr. 28:13 ; 1 Jn 1:9.} ! Et tu as porté la peine de mon péché. Sélah.
\VS{6}Que tout fidèle te prie au temps convenable\FTNT{Es. 55:6 ; So. 2:3 ; Ps. 69:14.}! Si de grandes eaux débordent, elles ne l'atteindront point.
\VS{7}Tu es mon asile, tu me gardes de la détresse, tu m'environnes de chants de triomphe à cause de ta délivrance. Sélah.
\VS{8}Je te rendrai intelligent, je t'enseignerai la voie dans laquelle tu dois marcher ; je te guiderai, mon oeil sera sur toi.
\VS{9}Ne soyez pas comme le cheval ni comme le mulet qui sont sans intelligence ; il faut brider leur bouche avec un mors et un frein, de peur qu'ils ne s'approchent de toi\FTNT{Ja. 3:3.}.
\VS{10}Beaucoup de douleurs atteindront le méchant\FTNT{Pr. 19:29.}, mais la bonté environne l'homme qui se confie en Yahweh.
\VS{11}Vous justes, réjouissez-vous en Yahweh, soyez dans l'allégresse ! Criez de joie, vous tous qui êtes droits de cœur\FTNT{Ps. 33:1 ; Ps. 64:11.} !
\Chap{33}
\TextTitle{Louanges à Yahweh, le Dieu fidèle}
\VerseOne{}Vous justes, poussez un cri de joie à cause de Yahweh\FTNT{Ps. 32:11 ; Ps. 97:12 ; Ps. 147:1.} ! Sa louange sied aux hommes droits.
\VS{2}Célébrez Yahweh avec la harpe, chantez-le sur le luth à dix cordes!
\VS{3}Chantez-lui un cantique nouveau\FTNT{Ps. 40:4 ; Ps. 96:1 ; Ps. 98:1 ; Ps. 144:9 ; Ap. 5:9 ; Ap. 14:3.} ! Jouez de vos instruments avec un cri de réjouissance !
\VS{4}Car la parole de Yahweh est droite, et toutes ses œuvres s'accomplissent avec fidélité ;
\VS{5}il aime la justice et la droiture\FTNT{Ps. 45:8 ; Hé. 1:9.} ; la terre est remplie de la bonté de Yahweh.
\VS{6}Les cieux ont été faits par la parole de Yahweh, et toute leur armée par le souffle de sa bouche\FTNT{Ge. 2:1-2.}.
\VS{7}Il amoncelle en un tas les eaux de la mer, il met les abîmes dans des réservoirs.
\VS{8}Que toute la terre craigne Yahweh ! Que tous les habitants du monde le redoutent !
\VS{9}Car il dit, et la chose arrive ; il ordonne, et la chose se présente.
\VS{10}Yahweh rompt le conseil des nations, il anéantit les desseins des peuples ;
\VS{11}mais le conseil de Yahweh subsiste à toujours, les desseins de son cœur subsistent d'âge en âge\FTNT{Pr. 19:21.}.
\VS{12}Heureuse la nation dont Yahweh est le Dieu\FTNT{Ps. 144:15.} et le peuple qu'il s'est choisi pour héritage !
\VS{13}Yahweh regarde des cieux, il voit tous les fils des hommes\FTNT{Job. 28:24.};
\VS{14}du lieu de sa demeure, il observe tous les habitants de la terre.
\VS{15}C'est lui qui forme également leur cœur et qui prend garde à toutes leurs actions.
\VS{16}Le roi n'est point sauvé par une grande armée, l'homme puissant n'échappe point par sa grande force;
\VS{17}le cheval est impuissant pour sauver, et ne délivre point par la grandeur de sa force\FTNT{Ps. 147:10.}.
\VS{18}Voici, l'œil de Yahweh est sur ceux qui le craignent\FTNT{Ps. 34:16 ; 1 Pi. 3:12.}, sur ceux qui s'attendent à sa bonté,
\VS{19}afin qu'il les délivre de la mort, et les fasse vivre durant la famine.
\VS{20}Notre âme espère en Yahweh; il est notre aide et notre bouclier.
\VS{21}Notre cœur se réjouit en lui, car nous avons confiance en son saint Nom.
\VS{22}Que ta bonté soit sur nous, ô Yahweh ! Nous nous attendons à toi!
\Chap{34}
\TextTitle{Yahweh libère les siens}
\VerseOne{}Psaume de David, lorsqu'il contrefît l'insensé en présence d'Abimélec, qui s'en alla, chassé par lui.
\VS{2}[Aleph.] Je bénirai Yahweh en tout temps, sa louange sera continuellement dans ma bouche.
\VS{3}[Beth.] Mon âme se glorifie en Yahweh ! Que les pauvres écoutent et se réjouissent.
\VS{4}[Guimel.] Glorifiez Yahweh avec moi ! Elevons son Nom tous ensemble !
\VS{5}[Daleth.] J'ai cherché Yahweh et il m'a répondu ; il m'a délivré de toutes mes frayeurs.
\VS{6}[He. Vav.] Quand on le regarde, on est illuminé, et la face n'est point confuse.
\VS{7}[Zayin.] Cet affligé a crié et Yahweh l'a exaucé, et l'a délivré de toutes ses détresses.
\VS{8}[Heth.] L'ange de Yahweh campe tout autour de ceux qui le craignent, et les équipe.
\VS{9}[Teth.] Goûtez et voyez combien Yahweh est bon ! Heureux l'homme qui se confie en lui !
\VS{10}[Yod.] Craignez Yahweh vous ses saints ! Car rien ne manque à ceux qui le craignent.
\VS{11}[Kaf.] Les lionceaux éprouvent la disette et la faim, mais ceux qui cherchent Yahweh ne manquent d'aucun bien.
\VS{12}[Lamed.] Venez, mes fils, écoutez-moi ! Je vous enseignerai la crainte de Yahweh.
\VS{13}[Mem.] Qui est l'homme qui prend plaisir à la vie, qui aime la prolonger pour jouir du bonheur ?
\VS{14}[Nun.] Garde ta langue du mal et tes lèvres des paroles trompeuses\FTNT{1 Pi. 3:10.} ;
\VS{15}[Samech.] détourne-toi du mal et fais-le bien ; cherche la paix et poursuis-la\FTNT{Hé. 12:14.}.
\VS{16}[Ayin.] Les yeux de Yahweh sont sur les justes et ses oreilles sont attentives à leur cri.
\VS{17}[Pe.] La face de Yahweh est contre ceux qui font le mal, pour retrancher de la terre leur mémoire\FTNT{Jé. 44:11 ; Lé. 17:10.}.
\VS{18}[Tsade.] Quand les justes crient, Yahweh les exauce et il les délivre de toutes leurs détresses.
\VS{19}[Qof.] Yahweh est près de ceux qui ont le cœur déchiré par la douleur, et il délivre ceux qui ont l'esprit abattu.
\VS{20}[Resh.] Le juste a des maux en grand nombre, mais Yahweh le délivre de tous\FTNT{2 Ti. 3:11.}.
\VS{21}[Shin.] Il garde tous ses os, aucun d'eux n'est brisé.
\VS{22}[Tav.] Le mauvais tue le méchant, et ceux qui haïssent le juste sont détruits.
\VS{23}[Pe.] Yahweh rachète l'âme de ses serviteurs, et aucun de ceux qui se confient en lui ne sera détruit.
\Chap{35}
\TextTitle{Prière au juste Juge}
\VerseOne{}Psaume de David. Yahweh, défends-moi contre mes adversaires, combats ceux qui me combattent !
\VS{2}Prends le petit et le grand bouclier, et lève-toi pour me secourir !
\VS{3}Brandis la lance et le javelot contre mes persécuteurs ! Dis à mon âme : Je suis ta délivrance !
\VS{4}Que ceux qui en veulent à ma vie soient honteux et confus\FTNT{Jé. 17:18 ; Ps. 40:15 ; Ps. 70:3.} ! Que ceux qui méditent ma perte reculent et rougissent !
\VS{5}Qu'ils soient comme la balle emportée par le vent\FTNT{Es. 29:5 ; Os. 13:3.}, et que l'Ange de Yahweh les chasse !
\VS{6}Que leur chemin soit ténébreux et glissant, et que l'Ange de Yahweh les poursuive.
\VS{7}Car sans cause ils m'ont tendu leur filet sur une fosse, sans cause ils l'ont creusée pour m'ôter la vie\FTNT{Jé. 18:20 ; Ps. 57:7 ; Ps. 140:5 ; Ps. 141:9.}.
\VS{8}Que la ruine les atteigne sans qu'ils le sachent, qu'ils soient capturés dans le filet qu'ils ont caché. Qu'ils y tombent et soient ravagés !
\VS{9}Mon âme aura de la joie en Yahweh, de l'allégresse en sa délivrance.
\VS{10}Tous mes os diront : Yahweh ! Qui est semblable à toi ? Qui délivre l'affligé de la main de celui qui est plus fort que lui ? L'affligé et le pauvre de celui qui le pille ?
\VS{11}De faux témoins s'élèvent contre moi : On m'interroge sur ce que j'ignore.
\VS{12}Ils me rendent le mal pour le bien, tâchant de m'ôter la vie\FTNT{Ps. 38:21 ; Ps. 109:5.}.
\VS{13}Mais moi, quand ils étaient malades, je me couvrais d'un sac, j'affligeais mon âme par le jeûne, je priais dans mon sein,
\VS{14}comme pour un ami, pour un frère, j'étais abattu, en pleurs, comme pour le deuil d'une mère.
\VS{15}Mais quand je chancelle, ils se réjouissent et s'assemblent, ils s'assemblent contre moi sans que je le sache pour me frapper, ils me déchirent pour que je sois silencieux ;
\VS{16}avec les hypocrites d’entre les railleurs qui suivent les bonnes tables, et ils ont grincé les dents contre moi.
\VS{17}Seigneur ! Jusqu'à quand le verras-tu ? Détourne mon âme de leurs tempêtes, mon unique des lionceaux.
\VS{18}Je te célébrerai dans la grande assemblée, je te louerai parmi un peuple nombreux\FTNT{Ps. 111:1.}.
\VS{19}Que ceux qui sont mes ennemis par leur mensonge ne se réjouissent point de moi, que ceux qui me haïssent sans cause ne m'insultent point par leurs regards\FTNT{Jn. 15:25.}.
\VS{20}Car ils ne parlent point de paix, mais ils préméditent des choses pleines de fraudes contre les gens tranquilles de la terre.
\VS{21}Ils ont ouvert leur bouche autant qu'ils ont pu contre moi, et ont dit : Ah ! Ah ! Nos yeux l'ont vu !
\VS{22}Yahweh ! Tu le vois : Ne te tais point\FTNT{Ps. 83:2.} ! Seigneur, ne t'éloigne point de moi !
\VS{23}Réveille-toi, réveille-toi pour me rendre justice\FTNT{Ps. 44:24.} ! Mon Dieu et mon Seigneur, défends ma cause !
\VS{24}Juge-moi selon ta justice, Yahweh mon Dieu ! et qu'ils ne se réjouissent point de moi !
\VS{25}Qu'ils ne disent point dans leur cœur : Ah ! Notre âme ! Et qu'ils ne disent point : Nous l'avons englouti !
\VS{26}Que ceux qui se réjouissent de mon mal soient honteux et rougissent tous ensemble ! Que ceux qui s'élèvent contre moi soient couverts de honte et de confusion !
\VS{27}Mais que ceux qui prennent plaisir à ma justice se réjouissent avec des chants de triomphe, qu'ils disent sans cesse : Grand est Yahweh qui désire la paix de son serviteur !
\VS{28}Alors ma langue publiera ta justice et ta louange tous les jours.
\Chap{36}
\TextTitle{Opposition : Les justes et les méchants}
\VerseOne{}Psaume de David, serviteur de Yahweh, donné au chef des chantres.
\VS{2}La transgression du méchant me dit, au dedans de mon cœur, qu'il n'y a point de crainte de Dieu devant ses yeux.
\VS{3}Car il se flatte à ses propres yeux pour consumer, pour assouvir sa haine.
\VS{4}Les paroles de sa bouche ne sont que méchanceté et tromperie, il cesse d'être sage et de faire le bien.
\VS{5}Il projette le malheur sur sa couche, il se tient sur un chemin qui n'est pas bon, il ne rejette pas le mal.
\VS{6}Yahweh ! Ta bonté atteint jusqu'aux cieux, ta fidélité jusqu'aux nues\FTNT{Ps. 57:11 ; Ps. 108:5.}.
\VS{7}Ta justice est comme les montagnes de Dieu, tes jugements sont un grand abîme. Yahweh ! Tu sauves les hommes et les bêtes.
\VS{8}Ô Dieu ! Combien est précieuse ta bonté ! Aussi les fils des hommes se retirent à l'ombre de tes ailes\FTNT{Ps. 17:8 ; Ps. 57:2.}.
\VS{9}Ils seront abondamment rassasiés de la graisse de ta maison et tu les abreuveras au fleuve de tes délices.
\VS{10}Car la source de la vie est auprès de toi, et par ta lumière nous voyons la lumière.
\VS{11}Etends ta bonté sur ceux qui te connaissent, et ta justice sur ceux qui ont le cœur droit !
\VS{12}Que le pied de l'orgueilleux ne s'avance point sur moi, et que la main des méchants ne m'ébranle point !
\VS{13}Là sont tombés les ouvriers d'iniquité ; ils sont renversés et ne peuvent se relever.
\Chap{37}
\TextTitle{Se confier en la justice de Yahweh}
\VerseOne{}Psaume de David. [Aleph.] Ne t'irrite pas contre les méchants, ne jalouse pas ceux qui s'adonnent à la perversité\FTNT{Pr. 23:17 ; Pr. 24:19.}.
\VS{2}Car ils seront soudainement retranchés comme le foin, et ils se faneront comme l'herbe verte.
\VS{3}[Beth.] Confie-toi en Yahweh, et fais ce qui est bon ; aie le pays pour demeure et la fidélité pour pâture.
\VS{4}Fais de Yahweh tes délices et il t'accordera ce que ton cœur désire.
\VS{5}[Guimel.] Recommande tes voies à Yahweh, confie-toi en lui et il agira\FTNT{Ps. 22:9 ; Ps. 55:23 ; Pr. 16:3.}.
\VS{6}Il manifestera ta justice comme la lumière et ton droit comme le soleil à son midi\FTNT{Pr. 4:18.}.
\VS{7}[Daleth.] Garde le silence devant Yahweh et tremble devant lui ; ne t'irrite point contre celui qui réussit dans ses voies, contre celui qui vient à bout de ses mauvais desseins.
\VS{8}[He.] Laisse la colère et abandonne la rage\FTNT{Ep. 4:26.} ; ne t'irrite pas pour faire le mal.
\VS{9}Car les méchants seront retranchés, mais ceux qui se confient en Yahweh hériteront la terre.
\VS{10}[Vav.] Encore un peu de temps et le méchant ne sera plus ; tu regardes le lieu où il était et il n'y est plus\FTNT{Job. 7:10 ; Job. 20:9.}.
\VS{11}Les pauvres prennent possession du pays et jouissent abondamment de la paix.
\VS{12}[Zayin.] Le méchant complote contre le juste et grince ses dents contre lui.
\VS{13}Le Seigneur se rit de lui, car il voit que son jour approche.
\VS{14}[Heth.] Les méchants tirent leur épée et bandent leur arc pour faire tomber le malheureux et le pauvre, pour massacrer ceux qui marchent dans la droiture\FTNT{Ps. 11:2.}.
\VS{15}Mais leur épée entre dans leur propre cœur, et leurs arcs se brisent.
\VS{16}[Teth.] Mieux vaut au juste le peu qu'il a, que l'abondance de beaucoup de méchants\FTNT{Pr. 15:16-17 ; Ec. 4:6.} ;
\VS{17}car les bras des méchants seront brisés, mais Yahweh soutient les justes.
\VS{18}[Yod.] Yahweh connaît les jours de ceux qui sont intègres, et leur héritage demeure à jamais.
\VS{19}Ils ne sont pas honteux au jour du malheur, mais ils sont rassasiés au jour de la famine.
\VS{20}[Kaf.] Mais les méchants périssent, et les ennemis de Yahweh, comme les beaux pâturages, s'évanouissent, ils s'évanouissent en fumée.
\VS{21}[Lamed.] Le méchant emprunte et ne rend point ; mais le juste a compassion et donne.
\VS{22}Car les bénis de Yahweh hériteront la terre, mais ceux qu'il a maudits seront retranchés.
\VS{23}[Mem.] Yahweh affermit les pas de l'homme, et il prend plaisir à ses voies.
\VS{24}S'il tombe, il ne sera pas entièrement abattu, car Yahweh le soutient de sa main.
\VS{25}[Nun.] J'ai été jeune et j'ai vieilli ; et je n'ai point vu le juste abandonné, ni sa postérité mendiant son pain.
\VS{26}Il est compatissant tout le temps, et il prête ; et sa postérité est bénie.
\VS{27}[Samech.] Retire-toi du mal et fais le bien ; et tu auras une demeure éternelle.
\VS{28}Car Yahweh aime ce qui est juste, et il n'abandonne point ses fidèles ; c'est pourquoi ils sont sous sa garde pour toujours, mais la postérité des méchants est retranchée.
\VS{29}[Ayin.] Les justes hériteront la terre et y habiteront à perpétuité.
\VS{30}[Pe.] La bouche du juste prononce la sagesse et sa langue déclare la justice.
\VS{31}La loi de son Dieu est dans son cœur\FTNT{Ps. 40:8-9.}, aucun de ses pas ne chancellera.
\VS{32}[Tsade.] Le méchant épie le juste et cherche à le faire mourir.
\VS{33}Yahweh ne l'abandonne point entre ses mains et ne le laisse point condamner quand on le juge.
\VS{34}[Qof.] Espère en Yahweh et garde sa voie, et il t'élèvera pour que tu hérites la terre ; tu verras les méchants retranchés.
\VS{35}[Resh.] J'ai vu le méchant dans toute sa puissance, il s'étendait comme un arbre verdoyant.
\VS{36}Il a passé, et voici, il n'est plus ; je le cherche et il ne se trouve plus.
\VS{37}[Shin.] Observe l'homme intègre et considère l'homme droit, car il y a une issue pour l'homme de paix.
\VS{38}Mais les rebelles seront tous détruits et ce qui sera resté des méchants sera retranché.
\VS{39}[Tav.] Mais la délivrance des justes viendra de Yahweh, il sera leur force au temps de la détresse.
\VS{40}Yahweh les secourt et les délivre ; il les délivre des méchants et les sauve, parce qu'ils se confient en lui.
\Chap{38}
\TextTitle{La tristesse du péché mène à la repentance}
\VerseOne{}Psaume de David. Pour souvenir.
\VS{2}Yahweh ! Ne me juge pas dans ta colère et ne me châtie pas dans ta fureur.
\VS{3}Car tes flèches m'ont atteint et ta main s'est appesantie sur moi.
\VS{4}Il n'y a rien de sain dans ma chair, à cause de ta colère, ni de paix dans mes os, à cause de mon péché.
\VS{5}Car mes iniquités s'élèvent au-dessus de ma tête, elles se sont appesanties comme un pesant fardeau, au-delà de mes forces\FTNT{Ps. 40:13.}.
\VS{6}Mes plaies ont une mauvaise odeur et sont purulentes à cause de ma folie.
\VS{7}Je suis courbé et abattu outre mesure ; je marche en pleurs tout le jour.
\VS{8}Car un mal brûlant remplit mes reins, et dans ma chair il n'y a rien de sain.
\VS{9}Je suis affaibli et brisé, je rougis le cœur troublé.
\VS{10}Seigneur, tout mon désir est devant toi, et mon soupir ne t'est point caché.
\VS{11}Mon cœur est agité çà et là, ma force m'abandonne, et la lumière de mes yeux n'est plus avec moi.
\VS{12}Ceux qui m'aiment, et même mes amis intimes, se tiennent loin de ma plaie, et mes proches se tiennent loin de moi\FTNT{Job. 19:13-14.}.
\VS{13}Ceux qui en veulent à ma vie me tendent des pièges ; ceux qui cherchent ma perte parlent de calamités et méditent des tromperies tous les jours.
\VS{14}Mais moi je suis comme un sourd, comme un muet qui n'ouvre point sa bouche.
\VS{15}Je suis, dis-je, comme un homme qui n'entend pas et qui n'a point de réplique dans sa bouche.
\VS{16}Car je m'attends à toi, ô Yahweh ! Tu me répondras, Seigneur mon Dieu !
\VS{17}Je dis : Il faut prendre garde qu'ils ne triomphent de moi ; quand mon pied chancelle, ils s'élèvent contre moi\FTNT{Ps. 94:18.} !
\VS{18}Car je suis près de tomber et ma douleur est continuellement devant moi.
\VS{19}Car je reconnais mon iniquité et je suis dans la crainte à cause de mon péché.
\VS{20}Cependant mes ennemis qui sont vivants se renforcent, et ceux qui me haïssent à tort se multiplient.
\VS{21}Ceux qui me rendent le mal pour le bien sont mes adversaires, parce que je recherche le bien\FTNT{Ps. 109:5 ; Jé. 18:20.}.
\VS{22}Ne m'abandonne pas Yahweh ! Mon Dieu, ne t'éloigne pas de moi !
\VS{23}Hâte-toi de venir à mon secours, Seigneur, tu es ma délivrance !
\Chap{39}
\TextTitle{La faiblesse de l'homme}
\VerseOne{}Psaume de David, donné au chef des chantres, à Jeduthun.
\VS{2}J'ai dit : Je prends garde à mes voies, de peur de pécher par ma langue ; je mettrai un frein à ma bouche tant que le méchant sera devant moi.
\VS{3}Je suis resté muet, dans le silence ; je me suis tu, quoique malheureux ; et ma douleur n'était pas moins vive.
\VS{4}Mon cœur brûlait au-dedans de moi, un feu intérieur me consumait, et la parole est venue sur ma langue.
\VS{5}Yahweh ! Dis-moi quel est le terme de ma vie et quelle est la mesure de mes jours\FTNT{Ps. 119:84.} ; que je sache combien je suis fragile.
\VS{6}Voici, tu as réduit mes jours à la largeur de ma main, et ma vie est comme un rien devant toi. Oui, tout homme debout n'est qu'un souffle\FTNT{Ja. 4:14.}. Sélah.
\VS{7}Oui, l'homme se promène comme une ombre, il s'agite inutilement ; il amasse des biens et il ne sait pas qui les recueillera.
\VS{8}Maintenant que puis-je espérer, Seigneur ? Mon espérance est en toi.
\VS{9}Délivre-moi de toutes mes transgressions ! Ne permets pas l'opprobre des insensés.
\VS{10}Je me suis tu et je n'ai point ouvert ma bouche, parce que c'est toi qui agis.
\VS{11}Détourne de moi tes coups ! Je suis consumé par les attaques de ta main.
\VS{12}Aussitôt que tu châties quelqu'un, en le punissant à cause de son iniquité, tu détruis comme la teigne ce qu'il a de plus cher. Oui, tout homme est une vapeur. Sélah.
\VS{13}Yahweh, écoute ma prière et prête l'oreille à mon cri ! Ne sois point sourd à mes larmes ! Car je suis un voyageur et un étranger chez toi, comme tous mes pères\FTNT{Lé. 25:23 ; Ps. 119:19 ; 1 Pi. 2:11 ; Hé. 11:13.}.
\VS{14}Détourne ton regard de moi, afin que je reprenne mes forces, avant que je m'en aille et que je ne sois plus.
\Chap{40}
\TextTitle{Un cantique nouveau à Yahweh}
\VerseOne{}Psaume de David, donné au chef des chantres.
\VS{2}J'ai attendu patiemment Yahweh, et il s'est tourné vers moi et a entendu mon cri.
\VS{3}Il m'a retiré de la fosse de destruction, du fond de la boue ; il a mis mes pieds sur un roc et a assuré mes pas.
\VS{4}Il a mis dans ma bouche un cantique nouveau, qui est la louange de notre Dieu ; plusieurs verront cela et ils craindront et se confieront en Yahweh.
\VS{5}Heureux l'homme qui place sa confiance en Yahweh et qui ne se tourne pas vers les orgueilleux et les menteurs !
\VS{6}Yahweh, mon Dieu ! Tu as multiplié tes merveilles et tes desseins envers nous ; nul n'est comparable à toi ; je voudrais les annoncer et les déclarer, mais leur nombre est trop grand pour que je les raconte.
\VS{7}Tu ne désires ni sacrifice ni offrande. Tu m'as percé les oreilles ; tu ne demandes ni holocauste ni victime expiatoire pour le péché\FTNT{Hé. 10:5.}.
\VS{8}Alors je dis : Voici, je viens avec le rouleau du livre écrit pour moi.
\VS{9}Mon Dieu, je prends plaisir à faire ta volonté, et ta loi est au fond de mes entrailles\FTNT{Ps. 37:31 ; Es. 51:7.}.
\VS{10}J'annonce ta justice dans la grande assemblée ; voilà, je ne ferme pas mes lèvres, Yahweh, tu le sais !
\VS{11}Je ne cache pas ta justice, qui est dans mon cœur ; je déclare ta fidélité et ta délivrance ; je ne cache pas ta bonté ni ta vérité dans la grande assemblée.
\VS{12}Toi, Yahweh ! Ne m'épargne point tes compassions, que ta bonté et ta vérité me gardent continuellement.
\VS{13}Car des maux sans nombre m'environnent ; mes iniquités m'atteignent et je ne supporte pas leur vue ; elles surpassent en nombre les cheveux de ma tête, et mon cœur m'abandonne.
\VS{14}Yahweh, veuille me délivrer ! Yahweh, hâte-toi de venir à mon secours !
\VS{15}Que tous ensemble ils soient honteux et confus, ceux qui cherchent mon âme pour la perdre ; et que ceux qui prennent plaisir à mon malheur retournent en arrière et rougissent.
\VS{16}Que ceux qui disent de moi : Ah ! Ah ! Soient consumés, en récompense de la honte qu'ils m'ont faite.
\VS{17}Que tous ceux qui te cherchent soient dans l'allégresse et se réjouissent en toi\FTNT{Ps. 70:5.} ! Que ceux qui aiment ta délivrance disent continuellement : Grand est Yahweh !
\VS{18}Moi, je suis affligé et misérable, mais le Seigneur prend soin de moi. Tu es mon secours et mon libérateur : Mon Dieu ne tarde point\FTNT{Ps. 70:6.} !
\Chap{41}
\TextTitle{Secours de Yahweh dans le malheur}
\VerseOne{}Psaume de David, donné au chef des chantres.
\VS{2}Heureux celui qui s'intéresse au pauvre ! Yahweh le délivrera au jour du malheur ;
\VS{3}Yahweh le garde et lui conserve la vie. Il est heureux sur la terre, et tu ne le livres pas au bon plaisir de ses ennemis.
\VS{4}Yahweh le soutient sur son lit de douleur ; tu le soulages dans toutes ses maladies.
\VS{5}Je dis : Yahweh ! Aie pitié de moi, guéris mon âme, car j'ai péché contre toi.
\VS{6}Mes ennemis disent du mal de moi : Quand mourra-t-il ? Et quand périra son nom ?
\VS{7}Si quelqu'un vient me voir, il dit des mensonges, il recueille de mauvais desseins\FTNT{Ps. 5:10 ; Ps. 10:7 ; Ps. 12:3.}, il s'en va et il parle au-dehors.
\VS{8}Tous ceux qui m'ont en haine murmurent sourdement ensemble contre moi, et machinent du mal contre moi.
\VS{9}Quelque action criminelle\FTNT{Le mot « criminelle » donne en hébreu « Belial »} pèse sur lui ; le voilà couché, disent–ils, il ne se relèvera plus !
\VS{10}Même celui qui était en paix avec moi, qui avait ma confiance et qui mangeait mon pain, a levé le talon contre moi\FTNT{Il est question ici de la trahison du Messie par Judas (Jn. 13:18-19).}.
\VS{11}Mais toi, ô Yahweh ! Aie pitié de moi et relève-moi ! Et je leur rendrai ce qui leur est dû.
\VS{12}Je connaîtrai que tu prends plaisir en moi, si mon ennemi ne triomphe pas de moi.
\VS{13}Pour moi, tu m'as soutenu à cause de mon intégrité, et tu m'as établi pour toujours en ta présence.
\VS{14}Béni soit Yahweh, le Dieu d'Israël, d'éternité en éternité. Amen ! Amen !
\Chap{42}
\TextTitle{Avoir soif de Dieu}
\VerseOne{}Cantique des fils de Koré, donné au chef des chantres.
\VS{2}Comme une biche soupire après des courants d'eau, ainsi mon âme soupire ardemment après toi, ô Dieu !
\VS{3}Mon âme a soif de Dieu, du Dieu vivant\FTNT{Ps. 63:2 ; Ps. 84:3.} : Ô quand entrerai-je et me présenterai-je devant la face de Dieu ?
\VS{4}Mes larmes sont ma nourriture jour et nuit, quand on me dit chaque jour : Où est ton Dieu\FTNT{Ps. 80:6 ; Ps. 115:2.} ?
\VS{5}Je rappelais ces choses dans mon souvenir, en répandant mon âme au-dedans de moi, savoir que je marchais dans la foule, et que je m'en allais tout doucement en leur compagnie, avec une voix de triomphe et de louange, jusqu'à la maison de Dieu, et qu'une grande multitude de gens sautait alors de joie.
\VS{6}Mon âme, pourquoi t'abats-tu et murmures-tu au-dedans de moi ? Attends-toi à Dieu, car je le célébrerai encore ; sa face est la délivrance même.
\VS{7}Mon Dieu ! Mon âme est abattue au-dedans de moi, aussi je me souviens de toi depuis la terre du Jourdain, depuis l'Hermon, et depuis la montagne de Mitsear.
\VS{8}Un flot appelle un autre flot au bruit de tes ondées ; toutes tes vagues et tes flots passent sur moi.
\VS{9}Toutefois, Yahweh enverra sa bonté compatissante de jour, et de nuit son cantique sera avec moi, et ma prière sera au Dieu qui est ma vie.
\VS{10}Je dis à Dieu, mon rocher : Pourquoi m'oublies-tu ? Pourquoi marcherai-je dans la tristesse à cause de l'oppression de l'ennemi ?
\VS{11}Comme avec une épée dans mes os, mes ennemis m'outragent, tandis qu'ils me disent chaque jour : Où est ton Dieu ?
\VS{12}Mon âme, pourquoi t'abats-tu et pourquoi murmures-tu au-dedans de moi ? Attends-toi à Dieu, car je le célébrerai encore, il est ma délivrance et mon Dieu.
\Chap{43}
\TextTitle{Espérer dans la délivrance de Dieu}
\VerseOne{}Fais-moi justice, ô Dieu ! Et défends ma cause contre une nation infidèle\FTNT{Ps. 26:1 ; Ps. 35:1.} ! Délivre-moi de l'homme trompeur et pervers.
\VS{2}Toi, mon Dieu protecteur, pourquoi me repousses-tu ? Pourquoi marcherai-je dans la tristesse à cause de l'oppression de l'ennemi ?
\VS{3}Envoie ta lumière et ta vérité, afin qu'elles me conduisent et m'introduisent dans ta sainte montagne, et dans tes demeures.
\VS{4}Alors je viendrai à l'autel de Dieu, au Dieu de ma joie et mon allégresse, et je te célébrerai sur la harpe, ô Dieu ! Mon Dieu !
\VS{5}Mon âme, pourquoi t'abats-tu et pourquoi murmures-tu au-dedans de moi ? Attends-toi à Dieu, car je le célébrerai encore ; il est ma délivrance et mon Dieu\FTNT{Ps. 42:6.}.
\Chap{44}
\TextTitle{Prière des affligés}
\VerseOne{}Cantique des fils de Koré, donné au chef des chantres.
\VS{2}Ô Dieu ! Nous avons entendu de nos oreilles et nos pères nous ont raconté les exploits que tu as faits de leur temps, aux jours d'autrefois\FTNT{Jg. 6:13 ; Ps. 77:12.}.
\VS{3}Tu as de ta main chassé les nations, et tu as affermi nos pères, tu as affligé les peuples, et tu as fait prospérer nos pères.
\VS{4}Car ce n'est point par leur épée qu'ils ont conquis le pays, et ce n'est pas leur bras qui les a délivrés, mais c'est ta droite, c'est ton bras, c'est la lumière de ta face, parce que tu les aimais.
\VS{5}Ô Dieu ! Tu es mon Roi : Ordonne la délivrance de Jacob !
\VS{6}Avec toi nous battrons nos adversaires, par ton Nom nous foulerons ceux qui s'élèvent contre nous.
\VS{7}Car je ne me confie point en mon arc, et ce n'est pas mon épée qui me délivrera.
\VS{8}Mais tu nous délivreras de nos adversaires, et tu rendras confus ceux qui nous haïssent.
\VS{9}Nous nous glorifierons en Dieu chaque jour et nous célébrerons à jamais ton Nom. Sélah.
\VS{10}Mais tu nous rejettes, tu nous confonds, et tu ne sors plus avec nos armées.
\VS{11}Tu nous fais reculer devant l'adversaire, et ceux qui nous haïssent enlèvent nos dépouilles.
\VS{12}Tu nous livres comme des brebis destinées à être dévorées, et tu nous as dispersés parmi les nations.
\VS{13}Tu as vendu ton peuple pour rien, et tu ne l'estimes pas d'une grande valeur\FTNT{Es. 52:3 ; Jé. 15:13.}.
\VS{14}Tu nous as mis en opprobre chez nos voisins, en dérision, et en sujet de moquerie auprès de ceux qui habitent autour de nous\FTNT{Jé. 24:9 ; Ps. 79:4.}.
\VS{15}Tu fais de nous un objet de sarcasmes parmi les nations, et de hochement de tête parmi les peuples.
\VS{16}Ma confusion est tout le jour devant moi, et la honte couvre ma face,
\VS{17}à cause des discours de celui qui nous fait des reproches et qui nous injurie, et à cause de l'ennemi et du vindicatif.
\VS{18}Tout cela nous est arrivé, et cependant nous ne t'avons point oublié, et nous n'avons point violé ton alliance.
\VS{19}Notre cœur ne s'est point détourné, nos pas ne se sont point éloignés de tes sentiers,
\VS{20}pour que tu nous écrases dans le lieu du serpent, et que tu nous couvres de l'ombre de la mort\FTNT{Ps. 23:4.}.
\VS{21}Si nous avions oublié le Nom de notre Dieu et étendu nos mains vers un dieu étranger,
\VS{22}Dieu ne le saurait-il pas, lui qui connaît les secrets du cœur ?
\VS{23}Mais nous sommes tous les jours mis à mort pour l'amour de toi, nous sommes regardés comme des brebis destinées à la boucherie\FTNT{Es. 53:7.}.
\VS{24}Lève-toi, pourquoi dors-tu Seigneur ? Réveille-toi ! Ne nous rejette point à jamais !
\VS{25}Pourquoi caches-tu ta face, pourquoi oublies-tu notre affliction et notre oppression ?
\VS{26}Car notre âme est abattue dans la poussière et notre ventre est attaché à la terre.
\VS{27}Lève-toi pour nous secourir ! Délivre-nous à cause de ta bonté.
\Chap{45}
\TextTitle{La beauté du Roi}
\VerseOne{}Cantique des fils de Koré, qui est un chant nuptial donné au chef des chantres pour le chanter sur Shoshannim.
\VS{2}Des paroles agréables bouillonnent dans mon cœur, et j'ai dit : Mon œuvre est pour le roi ! Ma langue sera comme la plume d'un habile écrivain !\FTNT{Ca. 5:13 ; Ca. 5:16.}
\VS{3}Tu es le plus beau des fils de l'homme, la grâce est répandue sur tes lèvres : C'est pourquoi Dieu t'a béni éternellement.
\VS{4}Ô héros, ceins ton épée sur ta cuisse, ta majesté et ta magnificence,
\VS{5}et prospère dans ta magnificence. Sois porté sur la parole de vérité, de douceur, et de justice, et ta droite versera des choses terribles !
\VS{6}Tes flèches sont aiguës, les peuples tomberont sous toi, elles perceront le cœur des ennemis du roi.
\VS{7}Ô Dieu, ton trône est à toujours et à perpétuité ! Le sceptre de ton règne est un sceptre d'équité.
\VS{8}Tu aimes la justice et tu hais la méchanceté : C'est pourquoi, ô Dieu, ton Dieu t'a oint d'une huile de joie par privilège sur tes compagnons\FTNT{Hé. 1:8-9.}.
\VS{9}Tous tes vêtements sont parfumés de myrrhe, d'aloès et de casse. Dans les palais d'ivoire les instruments à cordes te réjouissent.
\VS{10}Des filles de rois sont parmi tes bien-aimées ; la reine est à ta droite, parée d'or d'Ophir.
\VS{11}Ecoute, jeune fille, vois et prête l'oreille ; oublie ton peuple et la maison de ton père.
\VS{12}Le roi porte ses désirs sur ta beauté ; puisqu'il est ton Seigneur, prosterne-toi devant lui.
\VS{13}La fille de Tyr et les plus riches des peuples te supplieront avec des présents.
\VS{14}La fille du roi est intérieurement pleine de gloire. Elle porte un vêtement tissé d'or.
\VS{15}Elle sera présentée au roi en vêtements de broderie, et les filles qui viennent après elle, et qui sont ses compagnes, seront amenées vers toi.
\VS{16}Elles te seront présentées avec réjouissance et allégresse, et elles entreront au palais du roi.
\VS{17}Tes fils seront au lieu de tes pères, tu les établiras pour princes sur toute la terre.
\VS{18}Je rendrai ton Nom mémorable dans tous les âges, et à cause de cela les peuples te célébreront pour toujours et à perpétuité\FTNT{Ps. 67:3-5.}.
\Chap{46}
\TextTitle{L'assurance du peuple de Dieu}
\VerseOne{}Cantique des fils de Koré, donné au chef des chantres pour le chanter sur Alamoth. Cantique.
\VS{2}Dieu est notre retraite, notre force, et notre secours qui ne manque jamais dans les détresses\FTNT{Ps. 9:10.}.
\VS{3}C'est pourquoi nous ne craindrons point quand la terre est bouleversée et que les montagnes chancellent au cœur des mers\FTNT{Es. 54:10.},
\VS{4}quand ses eaux mugissent et écument, se soulèvent jusqu'à faire trembler les montagnes. Sélah.
\VS{5}Il est un fleuve dont les courants réjouissent la cité de Dieu, le lieu saint des demeures du Très-Haut\FTNT{Ez. 47:1-2 ; Za. 14:8-9 ; Jn. 7:38 ; Ap. 22:1-2.}.
\VS{6}Dieu est au milieu d'elle : Elle n'est point ébranlée. Dieu la secourt dès le point du jour\FTNT{So. 3:16-17.}.
\VS{7}Les nations murmurent, les royaumes s'ébranlent ; il a fait entendre sa voix et la terre se fond.
\VS{8}Yahweh des armées est avec nous, le Dieu de Jacob est pour nous une haute retraite. Sélah.
\VS{9}Venez, contemplez les œuvres de Yahweh et voyez quels ravages il a faits sur la terre.
\VS{10}Il a fait cesser les guerres jusqu'au bout de la terre, il a brisé l'arc et rompu la lance, il a consumé par le feu les chars de guerre\FTNT{Es. 2:4.}.
\VS{11}Arrêtez, et sachez que je suis Dieu : Je suis élevé parmi les nations, je suis élevé sur toute la terre.
\VS{12}Yahweh des armées est avec nous, le Dieu de Jacob est pour nous une haute retraite. Sélah.
\Chap{47}
\TextTitle{Yahweh, le Dieu élevé}
\VerseOne{}Psaume des fils de Koré, donné au chef des chantres.
\VS{2}Peuples battez tous des mains ! Poussez vers Dieu des cris de joie avec une voix de triomphe !
\VS{3}Car Yahweh, le Très-Haut, est terrible. Il est un grand Roi sur toute la terre.
\VS{4}Il nous assujettit des peuples et des nations sous nos pieds.
\VS{5}Il nous choisit notre héritage, la gloire de Jacob qu'il aime. Sélah.
\VS{6}Dieu est monté avec un cri de réjouissance, Yahweh monte au son du shofar.
\VS{7}Chantez à Dieu, chantez ! Chantez à notre Roi, chantez !
\VS{8}Car Dieu est le Roi de toute la terre : Chantez un cantique !
\VS{9}Dieu règne sur les nations, Dieu est assis sur son saint trône.
\VS{10}Les princes des peuples se rassemblent vers le peuple du Dieu d'Abraham, car les boucliers de la terre sont à Dieu : Il est puissamment élevé.
\Chap{48}
\TextTitle{Sion, splendeur du grand Roi}
\VerseOne{}Cantique. Psaume des fils de Koré.
\VS{2}Yahweh est grand, il est l'objet de toutes les louanges dans la ville de notre Dieu, sur sa montagne sainte.
\VS{3}Belle est la colline, joie de toute la terre, la montagne de Sion, le côté nord, c'est la ville du grand Roi.
\VS{4}Dieu est connu dans ses palais pour une haute retraite.
\VS{5}Ils l'ont vue, et aussitôt ils ont été émerveillés ; ils ont été troublés et se sont enfuis à la hâte.
\VS{6}Ils ont regardé tout stupéfaits, ils ont eu peur et ont pris la fuite.
\VS{7}Là un tremblement les a saisis, une douleur comme celle de l'enfantement\FTNT{Es. 13:8.}.
\VS{8}Ils ont été chassés comme par le vent d'orient qui brise les navires de Tarsis.
\VS{9}Comme nous l'avions entendu, ainsi l'avons-nous vu dans la ville de Yahweh des armées, dans la ville de notre Dieu : Dieu l'établira à toujours. Sélah.
\VS{10}Ô Dieu ! Nous pensons à ta bonté au milieu de ton temple.
\VS{11}Ô Dieu ! Comme ton Nom, ta louange retentit jusqu'aux extrémités de la terre ; ta droite est pleine de justice.
\VS{12}La montagne de Sion se réjouit, et les filles de Juda sont dans la joie, à cause de tes jugements.
\VS{13}Entourez Sion, faites-en le tour, comptez ses tours.
\VS{14}Observez son rempart, examinez ses palais pour le raconter à la génération future.
\VS{15}Car ce Dieu-là est notre Dieu éternellement et à jamais ; il nous accompagnera jusqu'à la mort.
\Chap{49}
\TextTitle{Vanité des richesses terrestres}
\VerseOne{}Psaume des fils de Koré, au chef des chantres.
\VS{2}Vous tous peuples, entendez ceci, vous habitants du monde, prêtez l'oreille,
\VS{3}petits et grands, riches et pauvres !
\VS{4}Ma bouche prononcera des discours pleins de sagesse, et les pensées de mon cœur sont pleines de sens.
\VS{5}Je prête l'oreille aux sentences qui me sont inspirées, je poserai mes questions au son de la harpe.
\VS{6}Pourquoi craindrai-je au jour du malheur, quand l'iniquité de mes adversaires m'entoure ?
\VS{7}Ils mettent leur confiance dans leurs biens et se glorifient de l'abondance de leurs richesses.
\VS{8}Ils ne peuvent se racheter l'un l'autre ni donner à Dieu le prix du rachat\FTNT{Mt. 16:26 ; Mc. 8:36-37 ; Lu. 12:15-21.}.
\VS{9}Car le rachat de leur âme est trop considérable, et il ne se fera jamais ;
\VS{10}ils ne vivront pas toujours et n'éviteront pas la vue de la fosse.
\VS{11}Car on voit que les sages meurent, l'insensé et le stupide périssent également, et ils laissent à d'autres leurs biens\FTNT{Ec. 2:21 ; Ec. 6:2.}.
\VS{12}Leur intention est que leurs maisons durent éternellement, et que leurs habitations demeurent d'âge en âge, ils ont donné leurs noms à leurs terres.
\VS{13}Mais l'homme qui est en honneur n'a point de durée, il est semblable aux bêtes que l'on égorge.
\VS{14}Tel est leur chemin, leur folie, et ceux qui les suivent se plaisent à leurs discours. Sélah.
\VS{15}Ils seront mis dans le scheol comme des brebis, la mort en fait sa pâture, et au matin les hommes droits les foulent aux pieds, leur beau rocher s'use, le scheol est leur résidence\FTNT{Job. 24:19.}.
\VS{16}Mais Dieu rachètera mon âme du pouvoir du scheol, quand il m'enlèvera de sa captivité\FTNT{Ps. 68:19 ; Ep. 4:8-9.}. Sélah.
\VS{17}Ne crains point quand tu verras quelqu'un s'enrichir et quand les trésors de sa maison se multiplient.
\VS{18}Car lorsqu'il mourra, il n'emportera rien, ses trésors ne descendront point après lui\FTNT{Job. 27:16-19 ; 1 Ti. 6:7.}.
\VS{19}Il aura beau s'estimer heureux pendant sa vie, on aura beau te louer des jouissances que tu te donnes,
\VS{20}tu iras néanmoins au séjour de tes pères, qui jamais ne reverront la lumière.
\VS{21}L'homme qui est en honneur, qui n'a pas d'intelligence, est semblable aux bêtes que l'on égorge.
\Chap{50}
\TextTitle{Yahweh, le juste Juge}
\VerseOne{}Psaume d'Asaph. Le Dieu puissant, Dieu, Yahweh a parlé et il a appelé toute la terre, depuis le soleil levant jusqu'au soleil couchant.
\VS{2}De Sion, Dieu a fait luire sa splendeur qui est d'une beauté parfaite,
\VS{3}notre Dieu viendra, il ne se taira point : Il y aura devant lui un feu dévorant, et tout autour de lui une grosse tempête.
\VS{4}Il appellera les cieux d'en haut, et la terre pour juger son peuple :
\VS{5}Rassemblez-moi mes bien-aimés qui ont traité alliance avec moi par le sacrifice\FTNT{Mt. 24:29-31.}.
\VS{6}Les cieux aussi annonceront sa justice parce que Dieu est le juge. Sélah.
\VS{7}Ecoute, ô mon peuple ! Et je parlerai. Entends, Israël ! Et je t'avertirai. Moi je suis Dieu, ton Dieu.
\VS{8}Je ne te réprimande pas pour tes sacrifices, tes holocaustes sont continuellement devant moi.
\VS{9}Je ne prendrai point de taureaux de ta maison ni de boucs de tes bergeries\FTNT{Ps. 40:7.}.
\VS{10}Car tous les animaux des forêts sont à moi, toutes les bêtes qui paissent sur mille montagnes.
\VS{11}Je connais tous les oiseaux des montagnes, et tout ce qui se meut dans les champs m'appartient.
\VS{12}Si j'avais faim, je ne t'en dirais rien, car le monde est à moi et tout ce qu'il renferme.
\VS{13}Mangerais-je la chair des gros taureaux ? Et boirais-je le sang des boucs ?
\VS{14}Offre à Dieu la reconnaissance et accomplis tes vœux envers le Très-Haut.
\VS{15}Invoque-moi au jour de ta détresse, je te délivrerai, et tu me glorifieras\FTNT{Ps. 37:5.}.
\VS{16}Dieu dit au méchant : Quoi donc ? Tu énumères mes lois ! Et tu as mon alliance dans ta bouche !
\VS{17}Toi qui hais la correction, et qui jettes mes paroles derrière toi !
\VS{18}Si tu vois un voleur, tu te plais avec lui, et ta part est avec les adultères.
\VS{19}Tu livres ta bouche au mal, et ta langue est un tissu de tromperies.
\VS{20}Tu t'assieds et parles contre ton frère, tu couvres d'opprobre le fils de ta mère.
\VS{21}Tu as fait ces choses-là, et je me suis tu. Tu as estimé que je te ressemble, mais je vais te reprendre et tout mettre sous tes yeux.
\VS{22}Comprenez cela maintenant, vous qui oubliez Dieu, de peur que je ne déchire sans que personne ne vous délivre.
\VS{23}Celui qui offre la louange me glorifie, et à celui qui veille sur sa voie, je lui montrerai le salut de Dieu.
\Chap{51}
\TextTitle{Le cœur repentant, sacrifice agréable à Dieu}
\VerseOne{}Psaume de David, au chef des chantres.
\VS{2}Lorsque Nathan le prophète vint à lui, après que David fut allé vers Bath-Schéba\FTNT{2 S. 11 ; 2 S. 12.}.
\VS{3}Ô Dieu ! Aie pitié de moi dans ta bonté, selon ta grande miséricorde, efface mes transgressions ;
\VS{4}lave-moi parfaitement de mon iniquité et purifie-moi de mon péché.
\VS{5}Car je reconnais mes transgressions, et mon péché est continuellement devant moi\FTNT{Es. 59:12.}.
\VS{6}J'ai péché contre toi, contre toi seul, et j'ai fait ce qui déplaît à tes yeux : En sorte que tu seras juste dans ta sentence, sans reproche dans ton jugement.
\VS{7}Voici, je suis né dans l'iniquité, et ma mère m'a conçu dans le péché.
\VS{8}Mais tu prends plaisir à la vérité au fond du cœur, et tu me fais connaître la sagesse au-dedans de moi.
\VS{9}Purifie-moi de mon péché avec de l'hysope, et je serai pur ; lave-moi, et je serai plus blanc que la neige.
\VS{10}Fais-moi entendre la joie et l'allégresse, et les os que tu as brisés se réjouiront.
\VS{11}Détourne ta face de mes péchés, et efface toutes mes iniquités.
\VS{12}Ô Dieu ! Crée en moi un cœur pur et renouvelle en moi un esprit ferme\FTNT{Mt. 5:8.}.
\VS{13}Ne me rejette pas loin de ta face et ne m'ôte pas ton Esprit Saint.
\VS{14}Rends-moi la joie de ton salut et qu'un esprit bien disposé me soutienne.
\VS{15}J'enseignerai tes voies aux transgresseurs et les pécheurs reviendront à toi.
\VS{16}Ô Dieu, Dieu de mon salut ! Délivre-moi de tant de sang, et ma langue chantera hautement ta justice.
\VS{17}Seigneur, ouvre mes lèvres, et ma bouche annoncera ta louange.
\VS{18}Car tu ne prends point plaisir aux sacrifices, autrement je t'en donnerais ; l'holocauste ne t'est point agréable.
\VS{19}Les sacrifices à Dieu, c'est un esprit brisé. Ô Dieu ! Tu ne méprises point un cœur brisé et contrit.
\VS{20}Répands par ta grâce, tes bienfaits sur Sion, édifie les murs de Jérusalem.
\VS{21}Alors tu prendras plaisir aux sacrifices de justice, à l'holocauste, et aux sacrifices qui se consument entièrement par le feu ; alors on offrira des taureaux sur ton autel.
\Chap{52}
\TextTitle{Sort de l'homme qui se confie en ses richesses}
\VerseOne{}Cantique de David, donné au chef des chantres.
\VS{2}A l'occasion du rapport que Doëg, l'Edomite, vint faire à Saül, en lui disant : David s'est rendu dans la maison d'Achimélec.
\VS{3}Pourquoi te vantes-tu du mal, vaillant homme ? La bonté de Dieu dure à toujours.
\VS{4}Ta langue trame des méchancetés, elle est comme un rasoir affilé qui trompe.
\VS{5}Tu aimes plus le mal que le bien, le mensonge plutôt que de dire la vérité. Sélah.
\VS{6}Tu aimes tous les discours pernicieux, le langage trompeur.
\VS{7}Aussi Dieu te détruira pour toujours, il t'enlèvera et t'arrachera de ta tente ; il te déracinera de la terre des vivants. Sélah.
\VS{8}Les justes le verront et auront de la crainte, et ils se riront d'un tel homme, disant :
\VS{9}Voilà cet homme qui ne tenait point Dieu pour sa protection, mais qui se confiait en ses grandes richesses et qui mettait sa force dans ses mauvais désirs\FTNT{Es. 47:10 ; Lu. 12:15-21.}.
\VS{10}Mais moi, je serai dans la maison de Dieu comme un olivier verdoyant. Je me confie dans la bonté de Dieu pour toujours et à jamais.
\VS{11}Je te célébrerai à jamais, car tu agis ; et je mettrai mon espérance en ton Nom, parce qu'il est bon envers tes fidèles.
\Chap{53}
\TextTitle{Egarement des impies}
\VerseOne{}Cantique de David, donné au chef des chantres, pour le chanter sur la flûte.
\VS{2}L'insensé dit en son cœur : Il n'y a point de Dieu ! Ils se sont corrompus, ils ont commis des injustices abominables ; il n'y a personne qui fasse le bien\FTNT{Ps. 10:4 ; Ro. 1:20-21 ; Ro. 3:12.}.
\VS{3}Dieu a regardé des cieux les fils des hommes, pour voir s'il y a quelqu'un qui soit intelligent, qui cherche Dieu.
\VS{4}Ils se sont tous détournés et se sont tous rendus odieux. Il n'y a personne qui fasse le bien, pas même un seul.
\VS{5}Les ouvriers d'iniquité n'ont-ils point de connaissance ? Ils mangent mon peuple comme s'ils mangeaient du pain. Ils n'invoquent point Dieu.
\VS{6}Ils seront épouvantés sans qu'il y ait sujet d'épouvante, car Dieu a dispersé les os de celui qui campe contre toi. Tu les confondras, car Dieu les a rejetés.
\VS{7}Oh ! Qui fera partir de Sion les délivrances d'Israël ? Quand Dieu aura ramené son peuple captif, Jacob s'égayera, Israël se réjouira.
\Chap{54}
\TextTitle{La délivrance vient de Yahweh}
\VerseOne{}Cantique de David, donné au chef des chantres, pour le chanter avec instruments à cordes.
\VS{2}Lorsque les Ziphiens vinrent dire à Saül : David n'est-il pas caché parmi nous\FTNT{1 S. 23:19 ; 1 S. 26:1.} ?
\VS{3}Ô Dieu ! Délivre-moi par ton Nom et fais-moi justice par ta puissance.
\VS{4}Ô Dieu ! Ecoute ma prière, prête l'oreille aux paroles de ma bouche !
\VS{5}Car des étrangers se sont élevés contre moi, et des gens terribles qui ne mettent pas Dieu devant eux en veulent à ma vie. Sélah.
\VS{6}Voilà, Dieu m'accorde son secours, le Seigneur est de ceux qui soutiennent mon âme.
\VS{7}Il fera retourner le mal sur ceux qui m'épient ; détruis-les selon ta vérité.
\VS{8}Je t'offrirai de bon cœur des sacrifices ; Yahweh ! Je célébrerai ton Nom parce qu'il est bon.
\VS{9}Car il m'a délivré de toute détresse ; et mes yeux se réjouissent à la vue de mes ennemis.
\Chap{55}
\TextTitle{Se garder des méchants}
\VerseOne{}Cantique de David, donné au chef des chantres, pour le chanter avec instruments à cordes.
\VS{2}Ô Dieu ! Prête l'oreille à ma prière et ne te cache pas de mes supplications !
\VS{3}Ecoute-moi et réponds-moi ! J'erre çà et là dans ma méditation et je suis agité
\VS{4}à cause du bruit que fait l'ennemi, à cause de l'oppression du méchant ; car ils font tomber sur moi les outrages, et ils me haïssent jusqu'à la fureur.
\VS{5}Mon cœur tremble au-dedans de moi et les terreurs de la mort tombent sur moi.
\VS{6}La crainte et l'épouvante m'atteignent et le frisson m'habille.
\VS{7}Je dis : Qui me donnera des ailes de colombe ? Je m'envolerais et je trouverais ma demeure.
\VS{8}Voilà, je m'enfuirais bien loin et je me tiendrais au désert. Sélah.
\VS{9}Je m'échapperais en toute hâte, plus rapide que le vent impétueux, que la tempête.
\VS{10}Seigneur, réduis à néant, divise leur langue, car j'ai vu la violence et les querelles dans la ville.
\VS{11}Elles font jour et nuit le tour sur les murailles ; l'iniquité et la malice sont dans son sein.
\VS{12}Les calamités sont au milieu d'elle, et la tromperie et la fraude ne partent point de ses places.
\VS{13}Car ce n'est pas mon ennemi qui m'a diffamé, je le supporterais ; ce n'est point celui qui m'a en haine qui s'élève contre moi, je me cacherais de lui.
\VS{14}Mais c'est toi, ô homme ! Que j'estimais mon égal, mon confident et mon ami\FTNT{Ps. 41:10.} !
\VS{15}Nous prenions plaisir à communiquer nos secrets ensemble, nous allions avec la multitude dans la maison de Dieu.
\VS{16}Que la mort les séduise ! Qu'ils descendent vivants dans le scheol ! Car le mal est dans leur demeure, parmi eux dans leur assemblée.
\VS{17}Mais moi je crie à Dieu, et Yahweh me délivrera.
\VS{18}Le soir, le matin, et à midi je me plains et je gémis, et il entendra ma voix.
\VS{19}Il délivrera mon âme de la guerre et me rendra la paix ; car ils sont nombreux contre moi.
\VS{20}Dieu entendra et témoignera en ma faveur. Lui qui de toute éternité est assis sur son trône. Sélah. Car il n'y a point de changement en eux, et ils ne craignent point Dieu.
\VS{21}Chacun d'eux porte la main sur ceux qui vivaient en paix avec lui, et viole son alliance.
\VS{22}Les paroles de sa bouche sont plus douces que la crème, mais la guerre est dans son cœur ; ses paroles sont plus douces que l'huile, néanmoins elles sont tout autant d'épées nues.
\VS{23}Remets ton sort à Yahweh et il te soulagera, il ne permettra jamais que le juste tombe.
\VS{24}Mais toi, ô Dieu ! Tu les précipiteras au puits de la perdition ; les hommes sanguinaires et trompeurs ne parviendront point à la moitié de leurs jours. C'est en toi que je me confie.
\Chap{56}
\TextTitle{Se glorifier en la Parole de Yahweh}
\VerseOne{}Hymne de David, donné au chef des chantres, pour le chanter sur « Colombe des térébinthes lointains ». Lorsque les Philistins le saisirent à Gath\FTNT{1 S. 21:10-14.}.
\VS{2}Dieu ! Aie pitié de moi, car des hommes m'écrasent et m'oppriment, me faisant tout le jour la guerre, ils m'oppressent.
\VS{3}Mes adversaires me piétinent tout le jour ; car, ô Très-Haut, plusieurs me font la guerre comme des hautains.
\VS{4}Le jour où j'aurai peur, je me confierai en toi.
\VS{5}Je me glorifierai en Dieu, en sa parole ; je me confie en Dieu, je ne craindrai rien. Que peuvent me faire les hommes\FTNT{Ps. 118:6 ; Hé. 13:6.} ?
\VS{6}Tout le jour ils tordent mes propos, et toutes leurs pensées tendent à me nuire.
\VS{7}Ils s'assemblent, ils se tiennent cachés, ils observent mes pas, s'attendant à m'ôter la vie.
\VS{8}C'est par l'iniquité qu'ils espèrent échapper. Dans ta colère, ô Dieu, précipite les peuples !
\VS{9}Tu comptes mes allées et venues ; recueille mes larmes dans tes outres : Ne sont-elles pas écrites dans ton livre ?
\VS{10}Le jour où je crierai à toi, mes ennemis reculeront ; je sais que Dieu est pour moi.
\VS{11}Je me glorifierai en Dieu, en sa parole, je me glorifierai en Yahweh, en sa parole.
\VS{12}Je me confie en Dieu, je ne craindrai rien : Que me fera l'homme ?
\VS{13}Ô Dieu ! Les vœux que je t'ai fait s'accompliront, je te louerai.
\VS{14}Car tu as délivré mon âme de la mort, tu as garanti mes pieds de la chute, afin que je marche devant Dieu, à la lumière des vivants.
\Chap{57}
\TextTitle{Avoir confiance en Dieu dans les difficultés}
\VerseOne{}Hymne de David, donné au chef des chantres, pour le chanter sur Al-Thasheth\FTNT{Al-Thasheth signifie « Ne détruis pas »}. Lorsqu'il se réfugia dans la caverne, poursuivi par Saül\FTNT{1 S. 22:1.}.
\VS{2}Aie pitié de moi, ô Dieu, aie pitié de moi ! Car mon âme cherche un refuge ; je cherche un refuge à l'ombre de tes ailes, jusqu'à ce que les calamités soient passées\FTNT{Ps. 17:8.}.
\VS{3}Je crie au Dieu Très-Haut, au Dieu qui accomplit son œuvre pour moi.
\VS{4}Il m'enverra des cieux la délivrance, il rendra honteux celui qui veut me dévorer. Sélah. Dieu enverra sa bonté et sa vérité.
\VS{5}Mon âme est parmi des lions ; je suis couché au milieu de gens qui vomissent la flamme, parmi des hommes dont les dents sont des lances et des flèches, et dont la langue est une épée aiguë\FTNT{Ps. 59:8 ; Ps. 64:4 ; Ja. 3:5-12.}.
\VS{6}Ô Dieu, élève-toi sur les cieux ! Que ta gloire soit sur toute la terre !
\VS{7}Ils avaient tendu un filet sous mes pas : Mon âme se courbait. Ils avaient creusé une fosse devant moi, mais ils y sont tombés. Sélah.
\VS{8}Mon cœur est affermi, ô Dieu ! Mon cœur est affermi, je chanterai et je ferai retentir mes instruments.
\VS{9}Réveille-toi ma gloire ! Réveillez-vous mon luth et ma harpe ! Je me réveillerai à l'aube du jour.
\VS{10}Seigneur, je te célébrerai parmi les peuples, je te chanterai parmi les nations.
\VS{11}Car ta bonté est grande jusqu'aux cieux, et ta vérité jusqu'aux nues\FTNT{Ps. 118:4-5.}.
\VS{12}Ô Dieu ! Elève-toi sur les cieux ! Que ta gloire soit sur toute la terre !
\Chap{58}
\TextTitle{Yahweh rend justice sur la terre}
\VerseOne{}Hymne de David, donné au chef des chantres, pour le chanter sur Al-Thasheth « Ne détruis pas ».
\VS{2}En vérité, vous, gens de l'assemblée, prononcez-vous ce qui est juste ? Vous, fils des hommes, jugez-vous avec droiture ?
\VS{3}Au contraire, vous tramez des injustices dans votre cœur. Sur la terre, c'est la violence de vos mains que vous placez sur la balance.
\VS{4}Les méchants se sont égarés dès le sein maternel, ils ont erré dès le ventre de leur mère, en parlant faussement.
\VS{5}Ils ont un venin semblable au venin du serpent, ils sont comme l'aspic sourd, qui ferme son oreille,
\VS{6}qui n'entend pas la voix des enchanteurs, du magicien le plus sage.
\VS{7}Ô Dieu, brise-leur les dents dans leur bouche ! Yahweh, brise les mâchoires des lionceaux !
\VS{8}Qu'ils s'écoulent comme de l'eau, et qu'ils se fondent ! Que chacun d'eux bande son arc, mais que ses flèches soient comme si elles étaient rompues !
\VS{9}Qu'ils s'en aillent comme un limaçon qui se fond ! Qu'ils ne voient point le soleil comme l'avorton d'une femme !
\VS{10}Avant que vos chaudières aient senti le feu des épines, l'ardeur de la colère, semblable à un tourbillon, enlèvera chacun d'eux comme de la chair crue..
\VS{11}Le juste se réjouira quand il aura vu la vengeance, il lavera ses pieds avec le sang du méchant.
\VS{12}Et chacun dira : Quoi qu'il en soit, il y a une récompense pour le juste ; quoi qu'il en soit, il y a un Dieu qui juge sur la terre.
\Chap{59}
\TextTitle{Intervention divine}
\VerseOne{}Hymne de David, donné au chef des chantres, pour le chanter sur Al-Thasheth « Ne détruis pas ». Lorsque Saül envoya des gens qui épièrent sa maison afin de le tuer\FTNT{1 S. 19:11.}.
\VS{2}Mon Dieu ! Délivre-moi de mes ennemis, protège-moi de ceux qui s'élèvent contre moi !
\VS{3}Délivre-moi des ouvriers d'iniquité et garde-moi des hommes sanguinaires !
\VS{4}Car voici, ils m'ont dressé des embûches, et des gens robustes se sont assemblés contre moi, bien qu'il n'y ait point en moi de transgression ni de péché, ô Yahweh !
\VS{5}Ils courent çà et là, et se mettent en ordre, bien qu'il n'y ait point d'iniquité en moi. Réveille-toi pour venir au-devant de moi ! Et regarde !
\VS{6}Toi donc, ô Yahweh ! Dieu des armées, Dieu d'Israël, réveille-toi pour visiter toutes les nations ! Ne fais point de grâce à aucun de ceux qui me trahissent ! Sélah.
\VS{7}Ils reviennent chaque soir, ils hurlent comme des chiens, ils font le tour de la ville.
\VS{8}Voici, de leur bouche ils font jaillir le mal, il y a des épées sur leurs lèvres\FTNT{Ja. 3:5-12.} ; car, disent-ils, qui nous entend ?
\VS{9}Mais toi, Yahweh ! Tu te riras d'eux, tu te moqueras de toutes les nations\FTNT{Ps. 2:4.}.
\VS{10}Quelle que soit leur force, je m'attends à toi, car Dieu est ma haute retraite.
\VS{11}Dieu qui me favorise me préviendra, Dieu me fera voir mes adversaires\FTNT{Ps. 118:7.}.
\VS{12}Ne les tue pas, de peur que mon peuple ne l'oublie ; fais-les errer par ta puissance et abats-les ; Seigneur, notre bouclier !
\VS{13}Leur bouche pèche à chaque parole de leurs lèvres ; qu'ils soient pris par leur orgueil ! Ils ne tiennent que des discours de malédiction et de mensonge.
\VS{14}Consume-les avec fureur, consume-les de sorte qu'ils ne soient plus ! Qu'on sache que Dieu domine sur Jacob et jusqu'aux extrémités de la terre ! Sélah.
\VS{15}Qu'ils reviennent le soir, et qu'ils hurlent comme des chiens, et qu'ils fassent le tour de la ville.
\VS{16}Qu'ils errent çà et là cherchant leur nourriture, et qu'ils passent la nuit sans être rassasiés.
\VS{17}Mais moi je chanterai ta force, je louerai dès le matin à haute voix ta bonté\FTNT{Ps. 88:14.}. Car tu es pour moi une haute retraite, et mon asile au jour de ma détresse.
\VS{18}Ma force ! Je te chanterai ; car Dieu est ma haute retraite, le Dieu qui me favorise.
\Chap{60}
\TextTitle{Yahweh, le meilleur secours}
\VerseOne{}Hymne de David, pour enseigner, donné au chef des chantres, pour le chanter sur le lis lyrique,
\VS{2}lorsqu'il fit la guerre contre les Syriens de Mésopotamie, et contre les Syriens de Tsoba, et que Joab revint et défit douze mille Edomites dans la vallée du sel\FTNT{2 S. 8:3-13 ; 1 Ch. 18:3-12.}.
\VS{3}Ô Dieu ! Tu nous as rejetés, tu nous as dispersés, tu t'es irrité : Reviens vers nous !
\VS{4}Tu as ébranlé la terre et l'as mise en pièces ; répare ses brèches, car elle chancelle !
\VS{5}Tu as fait voir à ton peuple des choses dures, tu nous as abreuvés d'un vin d'étourdissement\FTNT{Es. 51:17-21 ; Ap. 14:10.}.
\VS{6}Mais tu as donné une bannière à ceux qui te craignent, afin de l'élever bien haut pour l'amour de ta vérité. Sélah.
\VS{7}Afin que ceux que tu aimes soient délivrés ; sauve-moi par ta droite et exauce-moi\FTNT{Ps. 108:6.}.
\VS{8}Dieu a parlé dans son lieu saint : Je me réjouirai, je partagerai Sichem, je mesurerai la vallée de Succoth ;
\VS{9}Galaad est à moi, Manassé aussi est à moi, et Ephraïm est la protection de ma tête et Juda mon sceptre.
\VS{10}Moab est le bassin où je me lave ; je jette mon soulier sur Edom ; pays des Philistins, pousse des cris de guerre à mon sujet\FTNT{2 S. 8:2 ; 1 Ch. 18:2.}.
\VS{11}Qui me conduira dans la ville forte ? Qui me conduira jusqu'en Edom ?
\VS{12}Ne sera-ce pas toi, ô Dieu, qui nous avais rejetés, et qui ne sortais plus, ô Dieu, avec nos armées ?
\VS{13}Donne-nous du secours pour sortir de la détresse ! Car la délivrance qu'on attend de l'homme est vanité\FTNT{Jé. 17:5 ; Ps. 118:8.}.
\VS{14}Avec le secours de Dieu, nous ferons des exploits, et il foulera nos ennemis.
\Chap{61}
\TextTitle{Dieu, le parfait Refuge}
\VerseOne{}Psaume de David, donné au chef des chantres, pour le chanter sur instruments à cordes.
\VS{2}Ô Dieu, je crie à toi, sois attentif à ma prière !
\VS{3}Je crie à toi du bout de la terre, le cœur abattu ; conduis-moi sur le rocher qui est trop haut pour moi !
\VS{4}Car tu es mon refuge, une tour forte au-devant de l'ennemi.
\VS{5}Je séjournerai éternellement dans ta tente, je me retirerai à l'ombre de tes ailes. Sélah.
\VS{6}Car, ô Dieu ! Tu exauces mes vœux, et tu me donnes l'héritage de ceux qui craignent ton Nom.
\VS{7}Tu ajoutes des jours aux jours du roi ; que ses années se prolongent à jamais !
\VS{8}Qu'il demeure toujours dans la présence de Dieu ! Que la bonté et la vérité le gardent !
\VS{9}Ainsi je chanterai ton Nom à perpétuité, en rendant mes vœux chaque jour.
\Chap{62}
\TextTitle{La confiance en Dieu}
\VerseOne{}Psaume de David, donné au chef des chantres, d'après Jeduthun.
\VS{2}Quoiqu'il en soit, mon âme se repose en Dieu ; c'est de lui que vient ma délivrance.
\VS{3}Quoiqu'il en soit, il est mon rocher, ma délivrance, et ma haute retraite ; je ne serai pas entièrement ébranlé.
\VS{4}Jusqu'à quand accablerez-vous de maux un homme ? Vous serez tous mis à mort, et vous serez comme le mur qui penche, comme une cloison qui a été ébranlée.
\VS{5}Ils ne font que consulter pour le faire déchoir de son élévation ; ils prennent plaisir au mensonge ; ils bénissent de leur bouche, mais au-dedans ils maudissent. Sélah.
\VS{6}Mais toi mon âme, demeure tranquille, regarde à Dieu, car mon espérance est en lui.
\VS{7}Quoiqu'il en soit, il est mon rocher, ma délivrance, et ma haute retraite ; je ne serai point ébranlé.
\VS{8}En Dieu est ma délivrance et ma gloire ; en Dieu est le rocher de ma force et ma retraite.
\VS{9}Peuples, confiez-vous en lui en tout temps, déchargez votre cœur sur lui ! Dieu est notre retraite. Sélah.
\VS{10}Oui, vanité, les fils de l'homme ! Mensonge, les fils de l'homme ! Dans une balance, ils monteraient tous ensemble, plus légers qu'un souffle.
\VS{11}Ne vous confiez pas dans la violence ni dans la rapine ; ne devenez point vains ; quand les richesses abonderont, n'y mettez point votre cœur.
\VS{12}Dieu a parlé une fois, j'ai entendu cela deux fois : C'est que la force est à Dieu.
\VS{13}Et c'est à toi, Seigneur, qu'appartient la bonté ; certainement tu rendras à chacun selon son œuvre\FTNT{Jé. 32:19 ; Pr. 24:12 ; Job. 34:11 ; Mt. 16:27 ; Ro. 2:6.}.
\Chap{63}
\TextTitle{Soif de la présence de Dieu}
\VerseOne{}Psaume de David, lorsqu'il était dans le désert de Juda\FTNT{1 S. 22:5 ; 1 S. 23:14-15.}.
\VS{2}Ô Dieu ! Tu es mon Dieu, je te cherche au point du jour ; mon âme a soif de toi, mon corps soupire après toi sur cette terre aride, desséchée, et sans eau\FTNT{Ps. 42:2 ; Ps. 84:3 ; Ps. 143:6.}.
\VS{3}Ainsi je te contemple dans ton lieu saint pour voir ta force et ta gloire.
\VS{4}Car ta bonté vaut mieux que la vie, mes lèvres te louent.
\VS{5}Et ainsi je te bénirai donc toute ma vie\FTNT{Ps. 104:33.}, j'élèverai mes mains en ton Nom.
\VS{6}Mon âme est rassasiée comme de mets gras et succulents, et ma bouche te loue avec un chant de réjouissance.
\VS{7}Quand je me souviens de toi dans mon lit, je médite sur toi durant les veilles de la nuit\FTNT{Ps. 16:7 ; Ps. 119:55.}.
\VS{8}Car tu m'as secouru, je me réjouirai à l'ombre de tes ailes.
\VS{9}Mon âme s'est attachée à toi pour te suivre, ta droite me soutient.
\VS{10}Mais ceux-ci qui demandent que mon âme tombe en ruine, entreront au plus bas de la terre.
\VS{11}On les détruira à coups d'épée, ils seront la proie des chacals.
\VS{12}Mais le roi se réjouira en Dieu ; quiconque jure par lui s'en glorifiera, car la bouche de ceux qui mentent sera fermée\FTNT{Ps. 107:42 ; Job. 5:16.}.
\Chap{64}
\TextTitle{Yahweh, le seul abri}
\VerseOne{}Psaume de David, donné au chef des chantres.
\VS{2}Ô Dieu ! Ecoute ma voix quand je m'écrie. Protège ma vie contre l'ennemi que je crains !
\VS{3}Cache-moi des complots des méchants, de l'assemblée tumultueuse des ouvriers d'iniquité !
\VS{4}Ils aiguisent leur langue comme une épée\FTNT{Jé. 9:3 ; Ps. 11:2 ; Ps. 59:8.}, ils tirent comme des flèches leurs paroles amères,
\VS{5}afin de tirer sur l'innocent dans sa cachette ; ils tirent soudainement sur lui et n'ont aucune crainte.
\VS{6}Ils se fortifient dans leur méchanceté, tiennent des discours pour tendre des pièges, ils disent : Qui les verra\FTNT{Job. 24:15.} ?
\VS{7}Ils cherchent curieusement des méchancetés ; ils ont sondé tout ce qui se peut sonder, même ce qui peut être au–dedans de l'homme, et au cœur le plus profond.
\VS{8}Mais Dieu lance contre eux ses traits, soudain les voilà frappés.
\VS{9}Leur langue a causé leur chute ; tous ceux qui les voient secouent leur tête.
\VS{10}Et tous les hommes craindront et raconteront l'œuvre de Dieu, et considéreront ce qu'il aura fait.
\VS{11}Le juste se réjouira en Yahweh, et se retirera vers lui, et tous ceux qui sont droits de cœur s'en glorifieront\FTNT{Ps. 63:12 ; Ps. 97:12.}.
\Chap{65}
\TextTitle{Le règne de Yahweh sur la nature}
\VerseOne{}Psaume de David. Cantique. Donné au chef des chantres.
\VS{2}Ô Dieu ! Dans le calme, on te louera dans Sion, et l'on accomplira nos vœux\FTNT{Ps. 50:14 ; Ps. 66:13.}.
\VS{3}Tu entends nos prières, toute chair viendra jusqu'à toi.
\VS{4}Les iniquités prévalent sur moi, mais tu feras la propitiation de nos transgressions.
\VS{5}Heureux celui que tu choisis et que tu admets dans ta présence pour qu'il habite dans tes parvis ! Nous serons rassasiés des biens de ta maison, des biens du saint lieu de ton temple.
\VS{6}Dans ta justice, tu nous réponds par des choses terribles, ô Dieu de notre salut, espoir de toutes les extrémités lointaines de la terre et de la mer.
\VS{7}Il affermit les montagnes par sa force, il est ceint de puissance.
\VS{8}Il apaise le mugissement de la mer, le mugissement de leurs flots, et le tumulte des peuples.
\VS{9}Ceux qui habitent aux extrémités de la terre ont peur de tes prodiges ; tu réjouis l'orient et l'occident.
\VS{10}Tu visites la terre, tu lui donnes l'abondance, tu la combles de richesses ; le ruisseau de Dieu est plein d'eau ; tu prépares le blé, quand tu l'établis ainsi.
\VS{11}Tu arroses ses sillons, et tu aplanis ses mottes ; tu l'amollis par la pluie, et tu bénis son germe\FTNT{Es. 55:10 ; Ps. 104:13-14.}.
\VS{12}Tu couronnes l'année de tes biens, et tes voies versent l'abondance.
\VS{13}Les plaines du désert sont abreuvées et les collines sont ceintes de joie.
\VS{14}Les pâturages se couvrent de brebis, et les vallées se revêtent de froments ; les cris de joie et les chants retentissent.
\Chap{66}
\TextTitle{Louange au Dieu de grâces}
\VerseOne{}Cantique. Psaume, donné au chef des chantres. Vous tous habitants de toute la terre, poussez des cris de triomphe à Dieu.
\VS{2}Chantez la gloire de son Nom, faites éclater sa gloire par vos louanges.
\VS{3}Dites à Dieu : Que tes œuvres sont redoutables ! Tes ennemis te mentiront à cause de la grandeur de ta force.
\VS{4}Toute la terre se prosterne devant toi et te chante ; elle chante ton Nom. Sélah.
\VS{5}Venez et voyez les œuvres de Dieu : Il est redoutable quand il agit sur les fils des hommes.
\VS{6}Il a fait de la mer une terre sèche ; on a passé le fleuve à pied sec ; là, nous nous sommes réjouis en lui\FTNT{Ex. 14:21 ; Jos. 3:14-17.}.
\VS{7}Il domine par sa puissance éternellement ; ses yeux prennent garde sur les nations\FTNT{Ps. 14:2 ; Ps. 33:13 ; Job. 28:24.} ; les rebelles ne pourront point s'élever. Sélah.
\VS{8}Peuples, bénissez notre Dieu, et faites retentir le son de sa louange.
\VS{9}C'est lui qui a remis notre âme en vie, et qui n'a point permis que nos pieds chancellent.
\VS{10}Car, ô Dieu, tu nous as éprouvés ! Tu nous a fait passer au creuset comme l'argent.
\VS{11}Tu nous as amenés dans le filet, tu as mis sur nos reins un pesant fardeau.
\VS{12}Tu as fait monter des hommes sur notre tête, et nous avons passé par le feu et par l'eau. Mais tu nous as fait entrer dans un lieu d'abondance.
\VS{13}J'entrerai dans ta maison avec des holocaustes, j'accomplirai mes vœux envers toi\FTNT{Ps. 22:26 ; Ps. 76:12 ; Ps. 116:14.}.
\VS{14}Pour eux, mes lèvres se sont ouvertes et ma bouche les a prononcés dans ma détresse.
\VS{15}Je t'offrirai en holocauste des brebis grasses, avec la graisse des béliers, je te sacrifierai des taureaux et des boucs. Sélah.
\VS{16}Vous tous qui craignez Dieu, venez, écoutez, et je raconterai ce qu'il a fait à mon âme.
\VS{17}Je l'ai invoqué de ma bouche, et la louange a été sur ma langue.
\VS{18}Si j'avais conçu l'iniquité dans mon cœur, le Seigneur ne m'aurait pas écouté\FTNT{Jn. 9:31.}.
\VS{19}Mais certainement Dieu m'a écouté, il a été attentif à la voix de ma prière.
\VS{20}Béni soit Dieu qui n'a point rejeté ma prière, et qui n'a point éloigné de moi sa bonté.
\Chap{67}
\TextTitle{Louange des peuples}
\VerseOne{}Psaume. Cantique donné au chef des chantres, pour le chanter avec instruments à cordes.
\VS{2}Que Dieu ait pitié de nous et qu'il nous bénisse, qu'il fasse luire sa face sur nous\FTNT{No. 6:25 ; Ps. 4:7 ; Ps. 31:17 ; Ps. 119:135.}. Sélah.
\VS{3}Afin que ta voie soit connue sur la terre et ta délivrance parmi toutes les nations.
\VS{4}Les peuples te célébreront, ô Dieu ! Tous les peuples te célébreront\FTNT{Ps. 22:27 ; Ps. 68:33.} !
\VS{5}Les peuples se réjouissent et chantent de joie, car tu juges les peuples avec droiture et tu conduis les nations sur la terre\FTNT{Ps. 96:10.}. Sélah.
\VS{6}Les peuples te célébreront, ô Dieu ! Tous les peuples te célébreront !
\VS{7}La terre produira son fruit ; Dieu, notre Dieu, nous bénira.
\VS{8}Dieu nous bénira, et toutes les extrémités de la terre le craindront.
\Chap{68}
\TextTitle{Yahweh, le Dieu glorieux}
\VerseOne{}Psaume. Cantique de David, donné au chef des chantres.
\VS{2}Que Dieu se lève, et ses ennemis seront dispersés, et ceux qui le haïssent s'enfuiront devant lui\FTNT{No. 10:35.}.
\VS{3}Tu les chasseras comme la fumée est chassée par le vent ; comme la cire se fond devant le feu, ainsi les méchants périront devant Dieu\FTNT{Ps. 37:20 ; Ps. 97:5.}.
\VS{4}Mais les justes se réjouiront et s'égayeront devant Dieu, et tressailliront de joie\FTNT{Ps. 67:4-5.}.
\VS{5}Chantez à Dieu, célébrez son Nom ! Exaltez celui qui est monté sur les cieux ! Son Nom est Yahweh ! Réjouissez-vous dans sa présence.
\VS{6}Il est le père des orphelins et le juge des veuves ; Dieu est dans sa demeure sainte\FTNT{Ps. 146:9}.
\VS{7}Dieu donne une famille à ceux qui étaient abandonnés, il délivre ceux qui étaient enchaînés, mais les rebelles habitent sur une terre déserte.
\VS{8}Ô Dieu ! Quand tu sortis devant ton peuple, quand tu marchais dans le désert ! Sélah.
\VS{9}La terre trembla et les cieux répandirent leurs eaux à cause de la présence de Dieu, le mont Sinaï trembla à cause de la présence de Dieu, du Dieu d'Israël\FTNT{Ex. 19:18 ; Jg. 5:5.}.
\VS{10}Ô Dieu ! Tu as fait tomber une pluie abondante sur ton héritage, et quand il était épuisé, tu l'as rétabli.
\VS{11}Ton troupeau établit sa demeure dans le pays, que par ta bonté tu avais préparé pour les malheureux, ô Dieu !
\VS{12}Le Seigneur donne une parole, et les messagères de bonnes nouvelles sont une grande armée.
\VS{13}Les rois des armées se sont enfuis, ils se sont enfuis, et celle qui se tenait à la maison a partagé le butin\FTNT{1 S. 30:16.}.
\VS{14}Tandis que vous vous couchez dans les étables, les ailes de la colombe sont couvertes d'argent, et son plumage est d'un jaune d'or.
\VS{15}Quand le Tout-Puissant dispersa les rois dans le pays, il devint blanc comme la neige du Tsalmon.
\VS{16}La montagne de Dieu est un mont de Basan ; une montagne élevée, un mont de Basan.
\VS{17}Pourquoi l'insultez-vous, montagnes dont le sommet est élevé ? Dieu a désiré cette montagne pour y habiter, et Yahweh y demeurera à jamais.
\VS{18}Les chars de Dieu se comptent par vingt-mille, par milliers et par milliers ; le Seigneur est au milieu d'eux ; le Sinaï est dans le sanctuaire.
\VS{19}Tu es monté dans les hauteurs, tu as emmené des captifs, tu as pris des dons pour les distribuer parmi les hommes, et même parmi les rebelles, afin qu'ils habitent dans le lieu de Yahweh Dieu\FTNT{Ep. 4:8-10. Cette prophétie concerne la résurrection du Seigneur Jésus-Christ.}.
\VS{20}Béni soit le Seigneur, qui tous les jours nous comble de ses biens ; Dieu est notre délivrance. Sélah.
\VS{21}Dieu est pour nous le Dieu de délivrance, et les issues de la mort sont à Yahweh le Seigneur.
\VS{22}Certainement, Dieu écrasera la tête de ses ennemis\FTNT{Ge. 3:15.}, le sommet de la tête chevelue de celui qui marche dans ses péchés.
\VS{23}Le Seigneur dit : Je les ramènerai de Basan\FTNT{No. 21:33-35.}, je les ramènerai du fond de la mer.
\VS{24}Afin que tu plonges ton pied dans le sang\FTNT{Ps. 58:11.}, et que la langue de tes chiens ait sa part de tes ennemis.
\VS{25}Ils voient ta marche, ô Dieu ! Ils ont vu ta marche dans le lieu saint, la marche de mon Dieu, mon Roi.
\VS{26}Les chantres allaient devant, ensuite les joueurs d'instruments, et au milieu les jeunes filles jouant du tambour\FTNT{Ex. 15:20 ; 1 S. 18:6.}.
\VS{27}Bénissez Dieu dans les assemblées, bénissez le Seigneur, vous qui êtes descendants d'Israël.
\VS{28}Là sont Benjamin, le plus jeune qui domine sur eux, les chefs de Juda et leur corps d'armée, les chefs de Zabulon, et les chefs de Nephthali.
\VS{29}Ton Dieu ordonne que tu sois puissant. Affermis, ô Dieu, ce que tu as fait pour nous.
\VS{30}Dans ton temple, à Jérusalem, les rois t'amèneront des présents\FTNT{1 R. 10:10 ; Ps. 72:10 ; 2 Ch. 32:23.}.
\VS{31}Epouvante les bêtes sauvages des roseaux, la troupe des taureaux, et les veaux des peuples, et ceux qui se prosternent avec des pièces d'argent. Disperse les peuples qui prennent plaisir à la guerre.
\VS{32}De grands seigneurs viendront d'Egypte ; l'Ethiopie se hâtera d'étendre ses mains vers Dieu.
\VS{33}Royaumes de la terre, chantez à Dieu, célébrez le Seigneur ! Sélah.
\VS{34}Chantez celui qui est monté dans les cieux des cieux, les cieux éternels ; voilà, il fait retentir de sa voix un son puissant.
\VS{35}Attribuez la force à Dieu ; sa majesté est sur Israël, et sa force est dans les nuées.
\VS{36}Dieu ! Tu es redouté à cause de ton lieu saint. Le Dieu d'Israël est celui qui donne la force et la puissance à son peuple. Béni soit Dieu !
\Chap{69}
\TextTitle{Dieu attentif à la prière de ceux qui s'humilient}
\VerseOne{}Psaume de David, donné au chef des chantres, pour le chanter sur les lis.
\VS{2}Délivre-moi, ô Dieu, car les eaux menacent ma vie\FTNT{Ps. 124:4 ; Ps. 144:7.}.
\VS{3}Je suis enfoncé dans un bourbier profond, sans appui ; je suis entré au plus profond des eaux, et les courants d'eau me submergent.
\VS{4}Je suis las de crier, mon gosier se dessèche, mes yeux se consument pendant que je m'attends à Dieu.
\VS{5}Ceux qui me haïssent sans cause\FTNT{Jn. 15:25.} dépassent en nombre les cheveux de ma tête ; ceux qui tâchent de me ruiner et qui sont mes ennemis à tort se sont renforcés ; je dois rendre ce que je n'avais point ravi.
\VS{6}Ô Dieu ! Tu connais ma folie et mes fautes ne te sont point cachées.
\VS{7}Ô Seigneur Yahweh des armées ! Que ceux qui se confient en toi ne soient point honteux à cause de moi ; et que ceux qui te cherchent ne soient point humiliés à cause de moi, ô Dieu d'Israël !
\VS{8}Car pour l'amour de toi j'ai souffert l'opprobre, la honte a couvert mon visage.
\VS{9}Je suis devenu un étranger pour mes frères, et un homme de dehors pour les fils de ma mère\FTNT{Ge. 31:14-15 ; Jn. 7:3-5}.
\VS{10}Car le zèle de ta maison me dévore\FTNT{Jn. 2:17 ; Ro. 15:3.}, et les outrages de ceux qui t'insultaient sont tombés sur moi.
\VS{11}Je pleure et je jeûne : C'est ce qui m'attire l'opprobre.
\VS{12}Je prends un sac pour vêtement, et je suis l'objet de leurs discours moqueurs.
\VS{13}Ceux qui sont assis à la porte parlent de moi, et les buveurs de boissons fortes me mettent en chanson\FTNT{Job. 30:9 ; La. 3:14.}.
\VS{14}Mais je t'adresse ma prière, ô Yahweh\FTNT{Ps. 102:2.} ! Que ce soit le temps favorable, ô Dieu ! Par ta grande bonté. Réponds-moi en m'assurant ta délivrance.
\VS{15}Délivre-moi de la boue, que je ne m'y enfonce point\FTNT{Ps. 40:3.}, et que je sois délivré de ceux qui me haïssent, et des eaux profondes.
\VS{16}Que les courants d'eau ne me submergent plus, que l'abîme ne m'engloutisse point, et que le puits ne ferme point sa bouche sur moi.
\VS{17}Yahweh ! Exauce-moi, car ta bonté est agréable ; dans tes grandes compassions, tourne ta face vers moi ;
\VS{18}et ne cache point ta face à ton serviteur, car je suis en détresse. Hâte-toi, exauce-moi !
\VS{19}Approche-toi de mon âme, rachète-la ; délivre-moi à cause de mes ennemis.
\VS{20}Tu connais toi-même mon opprobre, et ma honte, et mon ignominie ; tous mes ennemis sont devant toi.
\VS{21}L'opprobre m'a brisé le cœur, et je suis languissant ; j'ai attendu que quelqu'un ait compassion de moi, mais il n'y en a point eu. J'ai attendu des consolateurs, mais je n'en ai point trouvé.
\VS{22}Ils m'ont au contraire donné du fiel\FTNT{Mt. 27:34 ; Mt. 27:48.} pour mon repas ; et dans ma soif, ils m'ont abreuvé de vinaigre.
\VS{23}Que leur table soit pour eux un piège et un appât au sein de leur perfection.
\VS{24}Que leurs yeux soient tellement obscurcis, qu'ils ne puissent point voir ; et fais continuellement chanceler leurs reins.
\VS{25}Répands ton indignation sur eux, et que l'ardeur de ta colère les saisisse.
\VS{26}Que leur campement soit désolé, et qu'il n'y ait personne qui habite dans leurs tentes.
\VS{27}Car ils persécutent celui que tu avais frappé, et racontent les souffrances de ceux que tu blesses.
\VS{28}Mets des iniquités à leurs iniquités ; et qu'ils n'entrent point dans ta justice.
\VS{29}Qu'ils soient effacés du livre de vie, et qu'ils ne soient point inscrits avec les justes.
\VS{30}Mais pour moi, qui suis affligé et dans la douleur, ta délivrance, ô Dieu, m'élèvera en une haute retraite.
\VS{31}Je louerai le Nom de Dieu par des cantiques et je le glorifierai par des louanges.
\VS{32}Cela est agréable à Yahweh plus qu'un taureau avec des cornes et des sabots fendus.
\VS{33}Les malheureux le voient et ils se réjouissent ; que votre cœur vive, vous qui cherchez Dieu.
\VS{34}Car Yahweh exauce les misérables et ne méprise point ses prisonniers.
\VS{35}Que les cieux et la terre le louent ; que la mer et tout ce qui s'y meut le louent aussi\FTNT{Ps. 96:11.}.
\VS{36}Car Dieu délivrera Sion et bâtira les villes de Juda ; on y habitera et on la possèdera.
\VS{37}Et la postérité de ses serviteurs en fera son héritage, et ceux qui aiment son Nom y auront leur demeure.
\Chap{70}
\TextTitle{Le pauvre et l'indigent}
\VerseOne{}Psaume de David, pour souvenir, donné au chef des chantres.
\VS{2}Dieu ! Hâte-toi de me délivrer, ô Dieu ! Hâte-toi de venir à mon secours\FTNT{Ps. 40:14 ; Ps. 71:12.}.
\VS{3}Que ceux qui cherchent mon âme soient honteux et rougissent\FTNT{Ps. 35:4 ; Ps. 71:13.} ; et que ceux qui prennent plaisir à mon mal soient repoussés en arrière et soient confus.
\VS{4}Que ceux qui disent : Aha ! Aha ! Retournent en arrière par l'effet de leur honte.
\VS{5}Que tous ceux qui te cherchent exultent et se réjouissent en toi ; et que ceux qui aiment ta délivrance disent toujours : Glorifié soit Dieu !
\VS{6}Moi, je suis affligé et misérable, ô Dieu ! Hâte-toi de venir vers moi ; tu es mon secours et mon libérateur, ô Yahweh ! Ne tarde point.
\Chap{71}
\TextTitle{Demeurer en Dieu jusqu'au bout}
\VerseOne{}Yahweh ! Je cherche en toi mon refuge : Que je ne sois jamais confus !
\VS{2}Délivre-moi par ta justice et sauve-moi. Incline ton oreille vers moi, mets-moi en sûreté.
\VS{3}Sois pour moi le rocher de mon refuge, afin que je puisse toujours m'y retirer ; tu as donné l'ordre de me mettre en sûreté, car tu es mon rocher et ma forteresse.
\VS{4}Mon Dieu ! Délivre-moi de la main du méchant, de la main du pervers et de l'oppresseur.
\VS{5}Car tu es mon espérance, Seigneur Yahweh ! Ma confiance dès ma jeunesse.
\VS{6}Je m'appuie sur toi dès le ventre de ma mère ; c'est toi qui m'as tiré hors des entrailles de ma mère\FTNT{Ps. 22:10-11.} ; tu es le sujet continuel de mes louanges.
\VS{7}Je suis pour plusieurs comme un miracle, mais tu es mon puissant refuge.
\VS{8}Que ma bouche soit remplie de ta louange et de ta gloire chaque jour.
\VS{9}Ne me rejette point au temps de ma vieillesse ; ne m'abandonne point maintenant que ma force est consumée.
\VS{10}Car mes ennemis ont parlé de moi, et ceux qui épient mon âme ont pris conseil ensemble,
\VS{11}disant : Dieu l'a abandonné. Poursuivez-le et saisissez-le, car il n'y a personne qui le délivre.
\VS{12}Dieu, ne t'éloigne point de moi ! Mon Dieu hâte-toi de venir à mon secours !
\VS{13}Que ceux qui sont les ennemis de mon âme soient honteux et défaits ; et que ceux qui cherchent mon malheur soient enveloppés d'opprobre et de honte.
\VS{14}Mais moi, j'espèrerai toujours et je te louerai tous les jours davantage.
\VS{15}Ma bouche racontera chaque jour ta justice et ta délivrance, bien que je n'en sache point le nombre.
\VS{16}Je marcherai par la force du Seigneur Yahweh ; je raconterai ta seule justice.
\VS{17}Ô Dieu ! Tu m'as enseigné dès ma jeunesse et j'ai annoncé jusqu'à présent tes merveilles.
\VS{18}Ô Dieu ! Ne m'abandonne pas, même dans la blanche vieillesse. Afin que j'annonce ta force à cette génération présente, ta puissance à la génération à venir.
\VS{19}Car ta justice, ô Dieu, est haut élevée, car tu as fait de grandes choses. Ô Dieu, qui est semblable à toi ?
\VS{20}Tu m'as fait éprouver bien des détresses et des malheurs, mais tu me redonneras la vie et tu me feras remonter hors des abîmes de la terre.
\VS{21}Relève ma grandeur et console-moi encore.
\VS{22}Je te louerai au son du luth, je chanterai ta fidélité, mon Dieu, je te célèbrerai avec la harpe, Saint d'Israël !
\VS{23}Mes lèvres et mon âme, que tu as rachetée, pousseront des cris de joie quand je te chanterai.
\VS{24}Ma langue aussi publiera chaque jour ta justice, car ceux qui cherchent mon malheur seront honteux et rougiront.
\Chap{72}
\TextTitle{Le royaume messianique}
\VerseOne{}De Salomon. Ô Dieu, donne tes jugements au roi et ta justice au fils du roi.
\VS{2}Qu'il juge avec justice ton peuple, et tes malheureux avec équité.
\VS{3}Que les montagnes portent la paix pour le peuple, et que les collines la portent en justice.
\VS{4}Qu'il fasse droit aux malheureux du peuple, qu'il délivre les fils du misérable, et qu'il écrase l'oppresseur !
\VS{5}Ils te craindront tant que le soleil et la lune dureront d'âge en âge.
\VS{6}Il descendra comme la pluie sur l'herbe fauchée, comme les ondées qui arrosent la terre.
\VS{7}En son temps, le juste fleurira, et il y aura abondance de paix jusqu'à ce qu'il n'y ait plus de lune.
\VS{8}Il dominera depuis une mer jusqu'à l'autre, et depuis le fleuve jusqu'aux extrémités de la terre.
\VS{9}Les habitants des déserts se courberont devant lui, et ses ennemis lécheront la poussière.
\VS{10}Les rois de Tarsis et des îles lui rapporteront des dons ; les rois de Saba et de Séba lui apporteront des présents.
\VS{11}Tous les rois aussi se prosterneront devant lui, toutes les nations le serviront.
\VS{12}Car il délivrera le pauvre qui crie vers lui, l'affligé et celui qui n'a personne qui l'aide\FTNT{Ps. 34:18 ; Job. 29:12.}.
\VS{13}Il aura compassion du pauvre et du misérable, et il sauvera les âmes des misérables.
\VS{14}Il garantira leur âme de la fraude et de la violence, et leur sang sera précieux devant ses yeux.
\VS{15}Il vivra donc, et on lui donnera de l'or de Séba, et on fera des prières pour lui continuellement ; on le bénira chaque jour.
\VS{16}Les blés abonderont dans le pays, au sommet des montagnes, et leurs épis s'agiteront comme les arbres du Liban ; les hommes fleuriront dans les villes comme l'herbe de la terre.
\VS{17}Sa renommée durera à toujours ; sa renommée ira de père en fils tant que le soleil durera ; et on se bénira en lui ; toutes les nations le diront heureux.
\VS{18}Béni soit Yahweh Dieu, le Dieu d'Israël, qui seul fait des choses merveilleuses !
\VS{19}Béni soit éternellement son Nom glorieux, et que toute la terre soit remplie de sa gloire. Amen ! Oui, amen !
\VS{20}Fin des prières de David, fils d'Isaï.
\Chap{73}
\TextTitle{L'orgueil des méchants}
\VerseOne{}Psaume d'Asaph. Quoi qu'il en soit, Dieu est bon pour Israël, pour ceux qui ont le cœur pur\FTNT{Mt. 5:8.}.
\VS{2}Toutefois, mes pieds allaient fléchir, mes pas étaient sur le point de glisser.
\VS{3}Car j'ai porté envie aux insensés en voyant la prospérité des méchants.
\VS{4}Rien ne les tourmente jusqu'à leur mort, et leur corps est gras.
\VS{5}Ils n'ont point de part aux peines des humains, et ils ne sont point frappés avec les autres hommes.
\VS{6}C'est pourquoi l'orgueil les environne comme un collier, et un vêtement de violence les couvre.
\VS{7}Les yeux leur sortent dehors à force de graisse ; ils surpassent les desseins de leur cœur.
\VS{8}Ils sont pernicieux, et parlent méchamment d'opprimer ; ils parlent d'une manière hautaine.
\VS{9}Ils élèvent leur bouche jusqu'aux cieux et leur langue parcourt la terre.
\VS{10}C'est pourquoi son peuple se tourne de leur côté, il avale l'eau abondamment.
\VS{11}Ils disent : Comment Dieu saurait-il ? Comment le Très-Haut connaîtrait-il\FTNT{Es. 29:15 ; Ez. 8:12 ; Ps. 94:7 ; Job. 22:12-13.} ?
\VS{12}Voilà, ceux-ci sont méchants, ils prospèrent toujours dans ce monde et acquièrent de plus en plus de richesses.
\VS{13}Quoi qu'il en soit, c'est donc en vain que j'ai purifié mon cœur et que j'ai lavé mes mains dans l'innocence\FTNT{Mal. 3:14 ; Job. 35:3}.
\VS{14}Je suis frappé tous les jours, et tous les matins mon châtiment est là.
\VS{15}Si je disais : Je veux parler comme eux, voici je trahirais la génération de tes fils.
\VS{16}Toutefois, j'ai tâché de connaître cela, mais cela m'a paru fort difficile,
\VS{17}jusqu'à ce que je sois entré dans le sanctuaire de Dieu et que j'aie considéré la fin de telles gens.
\VS{18}Quoi qu'il en soit, tu les as mis sur des voies glissantes, tu les fais tomber dans des précipices.
\VS{19}Comment ont-ils été ainsi détruits en un instant ? Ont-ils défailli ? Ont-ils été consumés d'épouvante ?
\VS{20}Ils sont comme un songe lorsqu'on s'est réveillé. Seigneur, tu méprises leur image à ton réveil.
\VS{21}Quand mon cœur s'aigrissait et que je me sentais percé dans les entrailles,
\VS{22}j'étais alors stupide, et je n'avais aucune connaissance ; j'étais comme une bête dans ta présence.
\VS{23}Je serai donc toujours avec toi ; tu m'as pris par la main droite.
\VS{24}Tu me conduiras par ton conseil, et tu me recevras dans la gloire.
\VS{25}Quel autre ai-je au ciel ? Or sur la terre je ne prends plaisir qu'en toi seul.
\VS{26}Ma chair et mon cœur étaient consumés, mais Dieu est le rocher de mon cœur, et mon partage pour toujours.
\VS{27}Car voilà, ceux qui s'éloignent de toi périront ; tu retrancheras tous ceux qui se détournent de toi.
\VS{28}Mais pour moi, m'approcher de Dieu c'est mon bien ; j'ai mis toute mon espérance dans le Seigneur Yahweh, afin de raconter toutes tes œuvres.
\Chap{74}
\TextTitle{Appel au secours du peuple de Dieu}
\VerseOne{}Cantique d'Asaph. Ô Dieu, pourquoi nous as-tu rejetés pour toujours ? Et pourquoi ta colère fume-t-elle contre le troupeau de ton pâturage\FTNT{Ps. 79:5.} ?
\VS{2}Souviens-toi de ton assemblée que tu as acquise autrefois. Tu t'es approprié cette montagne de Sion, sur laquelle tu habitais, afin qu'elle soit la portion de ton héritage.
\VS{3}Elève tes pas vers les lieux constamment dévastés ; l'ennemi a tout renversé dans le lieu saint.
\VS{4}Tes adversaires ont rugi au milieu de ton assemblée ; ils ont mis leurs signes pour signes.
\VS{5}On les a vus pareils à celui qui lève la cognée dans une épaisse forêt.
\VS{6}Et maintenant, avec des haches et des marteaux, ils brisent les sculptures.
\VS{7}Ils ont mis le feu à ton lieu saint. Ils ont abattu à terre et profané la demeure dédiée à ton Nom\FTNT{2 R. 25:9.}.
\VS{8}Ils ont dit en leur cœur : Saccageons-les tous ensemble ! Ils ont brûlé dans le pays tous les lieux saints de Dieu.
\VS{9}Nous ne voyons plus nos signes ; il n'y a plus de prophètes ; et personne parmi nous qui sache jusqu'à quand\FTNT{La. 2:9-10.}.
\VS{10}Ô Dieu ! Jusqu'à quand l'adversaire te couvrira-t-il d'opprobres et l'ennemi méprisera-t-il ton Nom à jamais ?
\VS{11}Pourquoi retires-tu ta main, même ta droite ? Consume-les en la tirant du milieu de ton sein !
\VS{12}Or Dieu est mon Roi dès les temps anciens, faisant des délivrances au milieu de la terre.
\VS{13}Tu as fendu la mer par ta force ; tu as brisé les têtes des serpents sur les eaux.
\VS{14}Tu as brisé les têtes du léviathan, tu l'as donné pour nourriture au peuple du désert.
\VS{15}Tu as ouvert la fontaine et le torrent, tu as desséché les grosses rivières.
\VS{16}A toi est le jour, à toi aussi est la nuit ; tu as établi la lumière et le soleil.
\VS{17}Tu as posé toutes les limites de la terre ; tu as formé l'été et l'hiver.
\VS{18}Souviens-toi de ceci : Que l'ennemi a blasphémé Yahweh et qu'un peuple insensé a outragé ton Nom.
\VS{19}Ne livre pas aux vivants l'âme de la tourterelle, n'oublie pas à toujours la vie de tes affligés.
\VS{20}Regarde à ton alliance, car les lieux ténébreux de la terre sont remplis d'habitations de violence.
\VS{21}Ne permets pas que celui qui est foulé s'en retourne tout confus. Que l'affligé et le pauvre louent ton Nom !
\VS{22}Ô Dieu ! Lève-toi, défends ta cause, souviens-toi de l'opprobre qui t'est fait tous les jours par l'insensé !
\VS{23}N'oublie pas le cri de tes adversaires, le bruit de ceux qui s'élèvent contre toi monte continuellement !
\Chap{75}
\TextTitle{L'élevation vient de Yahweh}
\VerseOne{}Psaume d'Asaph. Cantique donné au chef des chantres, pour le chanter sur Al-Thasheth\FTNT{Voir Ps. 57:1.}.
\VS{2}Nous te célébrons, ô Dieu ! Nous te célébrons et ton Nom est près de nous ; nous racontons tes merveilles.
\VS{3}Au temps que j'aurai fixé, je jugerai avec droiture.
\VS{4}La terre se dissout avec tous ceux qui y habitent, mais j'affermis ses piliers. Sélah.
\VS{5}Je dis aux insensés : N'agissez point follement ; et aux méchants : N'élevez pas la tête.
\VS{6}N'élevez pas si haut votre tête, et ne parlez point avec fierté.
\VS{7}Car l'élévation ne vient point d'orient, ni d'occident ni du désert.
\VS{8}Car c'est Dieu qui gouverne ; il abaisse l'un, et élève l'autre\FTNT{1 S. 2:7.}.
\VS{9}Il y a une coupe dans la main de Yahweh\FTNT{Es. 51:17-22 ; Jé. 25:27-28 ; Ap. 14:10 ; Ap. 16:19.}, et le vin rougit dedans ; il est plein de mélange, et Dieu en verse ; certainement, tous les méchants de la terre en suceront et en boiront jusqu'à la lie.
\VS{10}Mais moi, je raconterai ces choses à jamais, je chanterai au Dieu de Jacob.
\VS{11}J'humilierai tous les méchants, mais les justes seront élevés.
\Chap{76}
\TextTitle{La Puissance du Dieu redoutable}
\VerseOne{}Psaume d'Asaph. Cantique donné au chef des chantres, pour le chanter avec instruments à cordes.
\VS{2}Dieu est connu en Judée, sa renommée est grande en Israël ;
\VS{3}sa tente est à Salem et sa demeure à Sion.
\VS{4}Là il a brisé les arcs étincelants, le bouclier, l'épée et les armes de guerre. Sélah.
\VS{5}Tu es resplendissant, plus magnifique que les montagnes des ravisseurs.
\VS{6}Les plus courageux sont étourdis, ils sont dans un profond assoupissement, et aucun de ces hommes vaillants n'a trouvé ses mains.
\VS{7}Ô Dieu de Jacob, les cavaliers et les chevaux se sont endormis quand tu les as menacés.
\VS{8}Tu es redoutable, toi. Qui peut se tenir devant toi quand ta colère éclate ?
\VS{9}Tu fais entendre des cieux le jugement ; la terre en a eu peur et s'est tenue dans le silence.
\VS{10}Quand tu te lèves, ô Dieu, pour faire jugement, pour délivrer tous les malheureux de la terre ! Sélah.
\VS{11}L'homme te célèbre, même dans sa fureur, quand tu te ceins de toute ta colère.
\VS{12}Faites vos vœux à Yahweh votre Dieu et accomplissez-les ! Que tous ceux qui l'environnent apportent des dons au Dieu terrible !
\VS{13}Il coupe le souffle des princes ; il est redoutable aux rois de la terre.
\Chap{77}
\TextTitle{Se souvenir des prodiges de Yahweh}
\VerseOne{}Psaume d'Asaph, donné au chef des chantres, d'après Jeduthun.
\VS{2}Ma voix s'élève à Dieu, et je crie ; ma voix s'adresse à Dieu, et il m'écoutera.
\VS{3}Je cherche le Seigneur au jour de ma détresse ; sans cesse mes mains s'étendent durant la nuit ; mon âme refuse d'être consolée.
\VS{4}Je me souviens de Dieu, et je gémis ; je médite, et mon esprit est affaibli. Sélah.
\VS{5}Tu empêches mes yeux de dormir ; je suis troublé, et ne peux parler.
\VS{6}Je pense aux jours d'autrefois et aux années des siècles passées\FTNT{Ps. 143:5.}.
\VS{7}Je me souviens de mes chants pendant la nuit, je médite en mon cœur, et mon esprit cherche diligemment.
\VS{8}Le Seigneur m'a-t-il rejeté pour toujours ? Ne me sera-t-il plus favorable ?
\VS{9}Sa bonté est-elle disparue pour toujours ? Sa parole a-t-elle pris fin pour l'éternité ?
\VS{10}Dieu a-t-il oublié d'avoir compassion ? A-t-il dans sa colère retiré sa miséricorde ? Sélah.
\VS{11}Je dis : Ce qui me fait devenir malade, je me souviendrai des années de la droite du Très–Haut.
\VS{12}Je me souviens des exploits de Yahweh ; je me suis, dis-je, souvenu de tes merveilles d'autrefois.
\VS{13}Je méditerai toutes tes œuvres, et je parlerai de tes œuvres.
\VS{14}Ô Dieu ! Tes voies sont saintes. Quel dieu est grand comme Dieu ?
\VS{15}Tu es le Dieu qui fait des merveilles ! Tu as fait connaître ta force parmi les peuples.
\VS{16}Tu as délivré par ton bras ton peuple, les fils de Jacob et de Joseph. Sélah.
\VS{17}Les eaux t'ont vu, ô Dieu ! Les eaux t'ont vu et ont tremblé, même les abîmes en ont été émus.
\VS{18}Les nuées ont versé un déluge d'eau, les nuées ont fait retentir leur son ; tes flèches ont volé de toutes parts.
\VS{19}La voix de ton tonnerre était dans le tourbillon, les éclairs ont éclairé le monde, la terre en a été émue et en a tremblé.
\VS{20}Tu te frayas un chemin par la mer, un sentier par les grosses eaux ; et tes traces ne furent plus reconnues.
\VS{21}Tu as mené ton peuple comme un corps d'armée sous la conduite de Moïse et d'Aaron\FTNT{Mi. 6:4.}.
\Chap{78}
\TextTitle{Les œuvres de Dieu dans l'histoire d'Israël}
\VerseOne{}Cantique d'Asaph. Mon peuple, écoute ma loi, prêtez vos oreilles aux paroles de ma bouche.
\VS{2}J'ouvrirai ma bouche en une parabole ; je proférerai les énigmes cachées des temps anciens\FTNT{Mt. 13:35.}.
\VS{3}Ce que nous avons entendu et connu, et que nos pères nous ont raconté\FTNT{Ps. 44:2.},
\VS{4}nous ne le cacherons point à leurs fils. Ils raconteront à la génération à venir les louanges de Yahweh, sa puissance et ses merveilles qu'il a faites.
\VS{5}Car il a établi le témoignage en Jacob, et il a mis la loi en Israël ; il a donné cet ordre à nos pères de la faire connaître à leurs fils\FTNT{De. 4:9.},
\VS{6}pour qu'elle soit connue de la génération future, des fils qui naîtraient, et pour que lorsqu'ils seront grands, ils la relatent à leurs fils,
\VS{7}afin qu'ils mettent leur confiance en Dieu, et qu'ils n'oublient point les œuvres de Dieu, et qu'ils gardent ses commandements.
\VS{8}Afin qu'ils ne soient point comme leurs pères, une génération revêche et rebelle, une génération insoumise de cœur, dont l'esprit est infidèle à Dieu\FTNT{Ex. 32:9 ; Ac. 7:51.}.
\VS{9}Les fils d'Ephraïm, armés et tirant de l'arc, tournèrent le dos le jour de la bataille.
\VS{10}Ils ne gardèrent point l'alliance de Dieu et refusèrent de marcher selon sa loi.
\VS{11}Ils oublièrent ses œuvres et ses merveilles qu'il leur avait fait voir.
\VS{12}Il avait fait des miracles en présence de leurs pères, dans le pays d'Egypte, dans le champ de Tsoan.
\VS{13}Il fendit la mer et les fit passer au travers ; et il fit arrêter les eaux comme un monceau de pierres.
\VS{14}Il les conduisit de jour par la nuée, et toute la nuit par une lumière de feu\FTNT{Ex. 13:21.}.
\VS{15}Il fendit les rochers au désert, et leur donna à boire d'abondantes eaux, comme s'il eût puisé des abîmes.
\VS{16}Il fit sortir des ruisseaux de la roche\FTNT{Ex. 17:6 ; No. 20:11 ; 1 Co. 10:4.} et fit couler des eaux comme des rivières.
\VS{17}Toutefois, ils continuèrent à pécher contre lui, irritant le Très-Haut dans le désert.
\VS{18}Ils tentèrent Dieu dans leurs cœurs, en demandant de la viande selon leur désir.
\VS{19}Ils parlèrent contre Dieu, disant : Dieu pourrait-il dresser une table dans ce désert\FTNT{No. 11:4.} ?
\VS{20}Voilà, dirent-ils, il a frappé le rocher, et les eaux ont coulé et des torrents ont débordé ; mais pourrait-il aussi nous donner du pain ? Fournirait-il de la viande à son peuple ?
\VS{21}C'est pourquoi, Yahweh les ayant entendus, se mit dans une grande colère, et le feu s'embrasa contre Jacob, et sa colère s'excita contre Israël.
\VS{22}Parce qu'ils n'avaient point cru en Dieu et ne s'étaient point confiés en sa délivrance.
\VS{23}Il ordonna aux nuées d'en haut et il ouvrit les portes des cieux ;
\VS{24}il fit pleuvoir la manne sur eux pour leur nourriture et il leur donna le blé du ciel\FTNT{Ex. 16:14 ; Jn. 6:31.}.
\VS{25}Ils mangèrent tous le pain des grands. Il leur envoya de la viande pour s'en rassasier.
\VS{26}Il excita dans les cieux le vent d'orient et il amena par sa puissance le vent du sud.
\VS{27}Il fit pleuvoir sur eux de la viande comme de la poussière, et comme le sable des mers des oiseaux ailées.
\VS{28}Il les fit tomber au milieu du camp, autour de leurs demeures.
\VS{29}Ils en mangèrent et en furent pleinement rassasiés, car il leur donna selon leur désir.
\VS{30}Mais ils ne furent pas encore dégoûtés de leur désir, et leur viande était encore dans leur bouche
\VS{31}quand la colère de Dieu s'excita contre eux, et qu'il mit à mort les plus gras d'entre eux, et abattit les gens d'élite d'Israël\FTNT{1 Co. 10:5.}.
\VS{32}Malgré cela, ils péchèrent encore et ne crurent point à ses prodiges\FTNT{No. 14:2.}.
\VS{33}C'est pourquoi il consuma leurs jours par la vanité et leurs années par une fin soudaine.
\VS{34}Quand il les mettait à mort, alors ils le recherchaient ; ils se repentaient et ils cherchaient Dieu dès le matin.
\VS{35}Ils se souvenaient que Dieu était leur rocher, et Dieu, le Très-Haut, était leur libérateur.
\VS{36}Mais ils le trompaient de leur bouche et ils lui mentaient de leur langue\FTNT{Es. 29:13 ; Jé. 12:2 ; Mt. 15:8.} ;
\VS{37}car leur cœur n'était point droit envers lui, et ils ne furent point fidèles à son alliance.
\VS{38}Toutefois, comme il est compatissant, il pardonna leur iniquité, au point qu'il ne les détruisit pas ; mais il détourna souvent sa colère et ne réveilla pas toute sa fureur.
\VS{39}Il se souvint qu'ils n'étaient que chair, qu'un vent qui passe et qui ne revient point.
\VS{40}Combien de fois l'ont–ils irrité au désert, et combien de fois l'ont–ils attristé dans ce lieu inhabitable ?
\VS{41}Ils ne cessèrent de tenter Dieu et de provoquer le Saint d'Israël.
\VS{42}Ils ne se souvinrent point de sa puissance, du jour où il les délivra de la main de l'ennemi,
\VS{43}des miracles qu'il accomplit en Egypte, et de ses merveilles dans les champs de Tsoan.
\VS{44}Il changea en sang leurs fleuves et leurs ruisseaux et ils ne purent en boire les eaux\FTNT{Ex. 7:20.}.
\VS{45}Il envoya contre eux des mouches qui les dévorèrent et des grenouilles qui les détruisirent\FTNT{Ex. 8:6-24.}.
\VS{46}Il livra leurs récoltes aux sauterelles, le produit de leur travail aux sauterelles\FTNT{Ex. 10:13.}.
\VS{47}Il détruisit leurs vignes par la grêle, et leurs sycomores par les orages\FTNT{Ex. 9:23.}.
\VS{48}Il livra leur bétail à la grêle, et leurs troupeaux aux foudres étincelantes.
\VS{49}Il envoya sur eux l'ardeur de sa colère, la fureur, la rage et la détresse, un corps d'armée de messagers de malheur.
\VS{50}Il donna libre cours à sa colère, et ne retira point leur âme de la mort ; il livra leur vie à la peste\FTNT{Ex. 9:6.}.
\VS{51}Il frappa tout premier-né en Egypte, les prémices de la vigueur dans les tentes de Cham\FTNT{Ex. 12:29.}.
\VS{52}Il fit partir son peuple comme des brebis, il les mena comme un corps d'armée dans le désert.
\VS{53}Il les conduisit sûrement, et sans qu'ils eussent aucune frayeur, là où la mer couvrit leurs ennemis.
\VS{54}Il les amena vers sa frontière sainte, vers cette montagne que sa droite a acquise\FTNT{Ex. 15:17.}.
\VS{55}Il chassa devant eux les nations, leur distribua le pays en héritage, et fit habiter les tribus d'Israël dans les tentes de ces nations.
\VS{56}Mais ils tentèrent et irritèrent le Dieu Très-Haut, et ne gardèrent point ses préceptes.
\VS{57}Et ils se retirèrent en arrière et furent infidèles comme leurs pères ; ils tournèrent comme un arc trompeur.
\VS{58}Ils le provoquèrent à la colère par leurs hauts lieux, et l'émurent à la jalousie par leurs images taillées\FTNT{De. 32:16-21.}.
\VS{59}Dieu l'entendit et se mit dans une grande colère, et il méprisa fortement Israël.
\VS{60}Il abandonna la demeure de Silo, la tente où il habitait parmi les hommes.
\VS{61}Il livra en captivité sa force et son ornement entre les mains de l'ennemi.
\VS{62}Il livra son peuple à l'épée et se mit dans une grande colère contre son héritage.
\VS{63}Le feu consuma leurs gens d'élite, et leurs vierges ne furent point louées.
\VS{64}Leurs sacrificateurs tombèrent par l'épée, et leurs veuves ne les pleurèrent point.
\VS{65}Puis le Seigneur se réveilla comme un homme qui se serait endormi, et comme un puissant homme qui s'écrie ayant encore le vin dans la tête.
\VS{66}Il frappa ses adversaires par derrière et les mit en opprobre perpétuel.
\VS{67}Mais il dédaigna la tente de Joseph, et ne choisit point la tribu d'Ephraïm.
\VS{68}Mais il choisit la tribu de Juda, la montagne de Sion, celle qu'il aime.
\VS{69}Il bâtit son lieu saint dans les lieux élevés, et l'établit comme la terre qu'il a fondée pour toujours.
\VS{70}Il choisit David, son serviteur, et le prit de la bergerie\FTNT{1 S. 16:11 ; 2 S. 7:8.} ;
\VS{71}il le prit derrière les brebis qui allaitent et l'amena pour paître Jacob, son peuple, et Israël, son héritage.
\VS{72}Aussi il les dirigea selon l'intégrité de son cœur, et les conduisit avec des mains intelligentes.
\Chap{79}
\TextTitle{Appel au jugement de Dieu}
\VerseOne{}Psaume d'Asaph. Ô Dieu ! Les nations sont entrées dans ton héritage ; on a profané ton saint temple, on a mis Jérusalem en monceaux de pierres.
\VS{2}On a livré les cadavres de tes serviteurs pour viande aux oiseaux du ciel, et la chair de tes fidèles aux bêtes de la terre.
\VS{3}On a répandu leur sang comme de l'eau autour de Jérusalem, et il n'y a eu personne pour les enterrer.
\VS{4}Nous sommes un sujet d'opprobre à nos voisins, de moquerie et de risée à ceux qui habitent autour de nous\FTNT{Ps. 44:14 ; Ps. 80:7.}.
\VS{5}Jusqu'à quand, ô Yahweh, t'irriteras-tu sans cesse et ta jalousie s'embrasera-t-elle comme un feu\FTNT{Ps. 89:47.} ?
\VS{6}Répands ta fureur sur les nations qui ne te connaissent point et sur les royaumes qui n'invoquent point ton Nom\FTNT{Jé. 10:25.}.
\VS{7}Car on a dévoré Jacob et on a ravagé ses demeures.
\VS{8}Ne rappelle point devant nous les iniquités passées. Que tes compassions viennent en hâte au-devant de nous, car nous sommes dans une extrême détresse.
\VS{9}Ô Dieu de notre délivrance ! Aide-nous pour l'amour de la gloire de ton Nom, et délivre-nous ! Pardonne-nous nos péchés pour l'amour de ton Nom !
\VS{10}Pourquoi les nations diraient-elles : Où est leur Dieu ? Que la vengeance du sang de tes serviteurs, qui a été répandu, soit manifestée parmi les nations en notre présence.
\VS{11}Que le gémissement des captifs parviennent jusqu'à toi. Par ton bras puissant sauve tes fils, ceux qui vont périr !
\VS{12}Et rends à nos voisins, dans leur sein, sept fois au double l'opprobre qu'ils t'ont fait, ô Yahweh !
\VS{13}Mais nous, ton peuple, et le troupeau de ton pâturage, nous te louerons pour toujours, et de génération en génération nous publierons tes louanges.
\Chap{80}
\TextTitle{Implorer Yahweh}
\VerseOne{}Psaume d'Asaph, donné au chef des chantres, pour le chanter Sosannim-héduth.
\VS{2}Toi qui pais Israël, prête l'oreille ! Toi qui mènes Joseph comme un troupeau, toi qui es assis entre les chérubins\FTNT{2 S. 6:2 ; Es. 37:16 ; Ps. 99:1.}, fais briller ta splendeur !
\VS{3}Réveille ta puissance au-devant d'Ephraïm, de Benjamin et de Manassé ; et viens pour notre délivrance !
\VS{4}Dieu, ramène-nous et fais briller ta face ! Et nous serons délivrés !
\VS{5}Ô Yahweh, Dieu des armées, jusqu'à quand seras-tu irrité contre la prière de ton peuple ?
\VS{6}Tu les nourris de pain de larmes et tu les abreuves de larmes à pleine mesure.
\VS{7}Tu fais de nous un sujet de dispute entre nos voisins, et nos ennemis se moquent de nous.
\VS{8}Ô Dieu des armées, ramène-nous et fais briller ta face ! Et nous serons délivrés.
\VS{9}Tu avais retiré une vigne hors d'Egypte, tu as chassé les nations, et tu l'as plantée\FTNT{Es. 5:1-7 ; Os. 10:1 ; Mt. 20:1 ; Mt. 21:28-33.}.
\VS{10}Tu as préparé une place devant elle, tu lui as fait prendre racine, et elle a rempli la terre.
\VS{11}Les montagnes étaient couvertes de son ombre, et ses rameaux étaient comme de hauts cèdres de Dieu.
\VS{12}Elle étendait ses branches jusqu'à la mer, et ses rejetons jusqu'au fleuve.
\VS{13}Pourquoi as-tu rompu ses clôtures, de sorte que tous les passants sur la route cueillent ses raisins ?
\VS{14}Les sangliers de la forêt l'ont détruite, et toutes les bêtes des champs en font leur pâture.
\VS{15}Ô Dieu des armées, reviens ! Regarde des cieux, vois, et visite cette vigne ;
\VS{16}et le plant que ta droite avait planté, et le fils que tu t'es choisi.
\VS{17}Elle est brûlée par le feu, elle est coupée ; ils périssent devant ta face menaçante.
\VS{18}Que ta main soit sur l'homme de ta droite, sur le fils de l'homme que tu t'es choisi.
\VS{19}Et nous ne nous éloignerons plus de toi. Rends-nous la vie, et nous invoquerons ton Nom.
\VS{20}Ô Yahweh ! Dieu des armées, ramène-nous, fais briller ta face, et nous serons délivrés !
\Chap{81}
\TextTitle{Se débarasser des dieux étrangers}
\VerseOne{}Psaume d'Asaph, donné au chef des chantres, pour le chanter sur la Guitthith.
\VS{2}Chantez avec allégresse à notre Dieu, notre force ! Poussez des cris de joie en l'honneur du Dieu de Jacob.
\VS{3}Sonnez du shofar, prenez le tambour, la harpe mélodieuse et le luth.
\VS{4}Sonnez du shofar à la nouvelle lune, à la pleine lune, au jour de notre fête\FTNT{No. 10:10.}.
\VS{5}Car c'est une loi pour Israël, une ordonnance du Dieu de Jacob.
\VS{6}Il établit un statut à Joseph, lorsqu'il marcha contre le pays d'Egypte, où j'entendis un langage que je ne connaissais pas.
\VS{7}J'ai retiré son épaule du fardeau, et ses mains ont lâché les corbeilles.
\VS{8}Tu as crié dans la détresse, et je t'ai sauvé ; je t'ai répondu dans le lieu caché du tonnerre ; je t'ai éprouvé auprès des eaux de Mériba. Sélah.
\VS{9}Ecoute mon peuple, je te relèverai. Israël, si tu m'écoutais !
\VS{10}Qu'il n'y ait point de dieu étranger au milieu de toi, et ne te prosterne point devant les dieux des étrangers.
\VS{11}Je suis Yahweh, ton Dieu, qui t'ai fait monter hors du pays d'Egypte. Ouvre ta bouche et je la remplirai.
\VS{12}Mais mon peuple n'a point écouté ma voix, et Israël ne m'a point obéi.
\VS{13}C'est pourquoi je les ai abandonnés aux penchants de leur cœur, et ils ont suivi leurs propres conseils\FTNT{Es. 63:17 ; Es. 65:2 ; 2 Pi. 3:3.}.
\VS{14}Ô si mon peuple m'écoutait ! Si Israël marchait dans mes voies !
\VS{15}J'abattrais en un instant leurs ennemis et je tournerais ma main contre leurs adversaires.
\VS{16}Ceux qui haïssent Yahweh le flatteraient, et le bonheur de mon peuple durerait toujours.
\VS{17}Dieu le nourrirait du meilleur froment ; et je le rassasierais du miel du rocher.
\Chap{82}
\TextTitle{Dieu dénonce l'injustice des hommes}
\VerseOne{}Psaume d'Asaph. Dieu se tient dans l'assemblée de Dieu, il juge au milieu des juges.
\VS{2}Jusqu'à quand jugerez-vous injustement et aurez-vous égard à l'apparence de la personne des méchants\FTNT{Ps. 58:2.} ? Sélah.
\VS{3}Faites droit à celui qu'on opprime et à l'orphelin ; faites justice à l'affligé et au pauvre ;
\VS{4}délivrez celui qu'on maltraite et le misérable, retirez-le de la main des méchants.
\VS{5}Ils ne connaissent ni n'entendent rien ; ils marchent dans les ténèbres, tous les fondements de la terre sont ébranlés.
\VS{6}J'ai dit : Vous êtes des dieux\FTNT{Jn. 10:34.}, et vous êtes tous fils du Très-Haut.
\VS{7}Toutefois, vous mourrez comme des hommes, et vous les princes vous tomberez comme les autres.
\VS{8}Ô Dieu ! Lève-toi, juge la terre ; car tu auras en héritage toutes les nations\FTNT{Ps. 2:8 ; Hé. 1:2.}.
\Chap{83}
\TextTitle{Dessein et confusion des ennemis d'Israël}
\VerseOne{}Cantique et psaume d'Asaph.
\VS{2}Ô Dieu ! Ne garde point le silence, ne te tais point, et ne te tiens point en repos, ô Dieu\FTNT{Ps. 35:22.} !
\VS{3}Car voici, tes ennemis s'agitent, et ceux qui te haïssent ont levé la tête.
\VS{4}Ils ont consulté finement en secret contre ton peuple, et ils ont tenu conseil contre ceux qui se sont retirés vers toi pour se cacher\FTNT{Ps. 2:2.}.
\VS{5}Ils disent : Venez et détruisons-les, en sorte qu'ils ne soient plus une nation, et qu'on ne fasse plus mention du nom d'Israël\FTNT{Ce passage fait allusion aux désirs qu'ont certaines nations de voir Israël détruite Mi. 4:11 ; Ap. 11:1-2.}.
\VS{6}Car ils consultent ensemble d'un même esprit ; ils font alliance contre toi.
\VS{7}Les tentes d'Edom et des Ismaélites, des Moabites et des Hagaréniens ;
\VS{8}de Guebal, d'Ammon, d'Amalek, les Philistins avec les habitants de Tyr.
\VS{9}L'Assyrie aussi se joint à eux ; ils ont servi de bras aux fils de Lot. Sélah.
\VS{10}Fais-leur comme tu fis à Madian\FTNT{Jg. 7:15.}, comme à Sisera\FTNT{Jg. 4:15.}, et comme à Jabin, auprès du torrent de Kison !
\VS{11}Ils furent détruits à En-Dor et servirent de fumier à la terre.
\VS{12}Que leurs chefs soient traités comme Oreb et comme Zeeb ; et que tous leurs princes soient comme Zébach et Tsalmunna\FTNT{Jg. 7:25.} ;
\VS{13}parce qu'ils ont dit : Prenons possession des habitations agréables de Dieu.
\VS{14}Mon Dieu ! Rends-les semblables au tourbillon et au chaume chassé par le vent,
\VS{15}comme le feu brûle une forêt, et comme la flamme embrase les montagnes.
\VS{16}Poursuis-les ainsi par ta tempête et épouvante-les par ton tourbillon !
\VS{17}Couvre leurs visages d'ignominie afin qu'on cherche ton Nom, ô Yahweh !
\VS{18}Qu'ils soient honteux et épouvantés à jamais, qu'ils rougissent, et qu'ils périssent ;
\VS{19}afin qu'on sache que toi seul, dont le nom est Yahweh, tu es le Très-Haut sur toute la terre.
\Chap{84}
\TextTitle{Délices pour ceux qui ont Yahweh comme appui}
\VerseOne{}Psaume des fils de Koré, donné au chef des chantres, pour le chanter sur la Guitthith.
\VS{2}Yahweh des armées, que tes demeures sont aimables !
\VS{3}Mon âme soupire et languit après les parvis de Yahweh ; mon cœur et ma chair poussent des cris de joie vers le Dieu vivant.
\VS{4}Le passereau même trouve sa maison, et l'hirondelle son nid où elle a mis ses petits… Tes autels, ô Yahweh des armées ! Mon Roi et mon Dieu !
\VS{5}Heureux ceux qui habitent ta maison et qui te louent sans cesse ! Sélah.
\VS{6}Heureux l'homme dont la force est en toi, ils trouvent dans leur cœur des chemins tout tracés !
\VS{7}Passant par la vallée de Baca, ils la réduisent en fontaine ; la pluie la couvre de bénédictions.
\VS{8}Ils marchent avec force pour se présenter devant Dieu à Sion.
\VS{9}Yahweh Dieu des armées, écoute ma prière, Dieu de Jacob, prête l'oreille. Sélah.
\VS{10}Ô Dieu, notre bouclier, vois et regarde la face de ton oint !
\VS{11}Car mieux vaut un jour dans tes parvis, que mille ailleurs. J'aimerais mieux me tenir à la porte dans la maison de mon Dieu, que de demeurer dans les tentes des méchants.
\VS{12}Car Yahweh Dieu est un soleil et un bouclier\FTNT{Ge. 15:1 ; Ps. 89:19 ; Ps. 144:2.} ; Yahweh donne la grâce et la gloire, et il ne refuse aucun bien à ceux qui marchent dans l'intégrité.
\VS{13}Yahweh des armées, heureux l'homme qui se confie en toi\FTNT{Ps. 2:12.} !
\Chap{85}
\TextTitle{Supplication des rescapés de l'exil}
\VerseOne{}Psaume des fils de Koré, donné au chef des chantres.
\VS{2}Yahweh, tu as été favorable à ta terre, tu as ramené et mis en repos les prisonniers de Jacob.
\VS{3}Tu as pardonné l'iniquité de ton peuple, tu as couvert tous leurs péchés. Sélah.
\VS{4}Tu as retiré toute ta colère, tu es revenu de l'ardeur de ton indignation.
\VS{5}Ô Dieu de notre délivrance, rétablis-nous et fais cesser la colère que tu as contre nous.
\VS{6}Seras-tu irrité à jamais contre nous ? Feras-tu durer ta colère de génération en génération ?
\VS{7}Ne reviendras–tu pas nous rendre la vie\FTNT{Ps. 71:20.}, afin que ton peuple se réjouisse en toi ?
\VS{8}Yahweh, fais-nous voir ta miséricorde et accorde-nous ta délivrance !
\VS{9}J'écouterai ce que dira Dieu, Yahweh ; car il parlera de paix à son peuple et à ses bien-aimés, pourvu que jamais ils ne retournent à leur folie.
\VS{10}Certainement sa délivrance est proche de ceux qui le craignent, la gloire habite dans notre pays.
\VS{11}La bonté et la vérité se rencontrent ; la justice et la paix s'embrassent\FTNT{Hé. 7:2.}.
\VS{12}La vérité germe de la terre et la justice regarde des cieux.
\VS{13}Yahweh aussi donne le bien et notre terre rendra son fruit.
\VS{14}La justice marchera devant lui, et il la mettra partout où il passera.
\Chap{86}
\TextTitle{Coeur disposé à la crainte de Dieu}
\VerseOne{}Prière de David. Yahweh, écoute, réponds-moi, car je suis affligé et misérable.
\VS{2}Garde mon âme, car je suis un de tes bien-aimés ; ô toi mon Dieu, délivre ton serviteur qui se confie en toi !
\VS{3}Seigneur, aie pitié de moi, car je crie à toi tout le jour.
\VS{4}Réjouis l'âme de ton serviteur, car j'élève mon âme à toi, Seigneur.
\VS{5}Yahweh ! Tu es bon et clément, et d'une grande bonté envers tous ceux qui t'invoquent\FTNT{Joë. 2:13.}.
\VS{6}Yahweh, prête l'oreille à ma prière, et sois attentif à la voix de mes supplications.
\VS{7}Je t'invoque au jour de ma détresse, car tu m'exauces\FTNT{Ps. 50:15.}.
\VS{8}Seigneur, nul n'est comme toi parmi les dieux, et rien ne ressemble à tes œuvres\FTNT{De. 3:24 ; Ps. 95:3.}.
\VS{9}Seigneur, toutes les nations que tu as faites viendront et se prosterneront devant toi, et glorifieront ton Nom,
\VS{10}car tu es grand, et tu fais des choses merveilleuses ; tu es Dieu, toi seul.
\VS{11}Yahweh ! Enseigne-moi tes voies et je marcherai dans ta vérité\FTNT{Ps. 25:4 ; Ps. 27:11.} ; lie mon cœur à la crainte de ton Nom.
\VS{12}Seigneur, mon Dieu, je te célébrerai de tout mon cœur, et je glorifierai ton Nom à toujours.
\VS{13}Car ta bonté est grande envers moi, et tu as retiré mon âme du profond scheol.
\VS{14}Ô Dieu ! Des gens orgueilleux se sont élevés contre moi, et un corps d'armée de méchants en veut à ma vie ; ils ne portent pas leurs pensées sur toi.
\VS{15}Mais toi, Seigneur, tu es le Dieu compatissant, miséricordieux, lent à la colère, riche en bonté et en vérité.
\VS{16}Tourne-toi vers moi, et aie pitié de moi ! Donne ta force à ton serviteur, délivre le fils de ta servante !
\VS{17}Accorde–moi un signe de ta faveur, et que ceux qui me haïssent le voient et soient honteux, parce que tu m'aideras, ô Yahweh ! Tu me consoleras !
\Chap{87}
\TextTitle{Sion, la cité de Dieu}
\VerseOne{}Psaume. Cantique des fils de Koré. Elle est fondée sur les montagnes saintes.
\VS{2}Yahweh aime les portes de Sion, plus que toutes les demeures de Jacob.
\VS{3}Ce qui se dit de toi, cité de Dieu, sont des choses glorieuses. Sélah.
\VS{4}Je ferai mention de Rahab et de Babylone parmi ceux qui me connaissent ; voici le pays des philistins, et Tyr avec l'Ethiopie. C'est dans Sion qu'ils sont nés.
\VS{5}Et de Sion il est dit : Un homme y est né ; le Très-Haut lui-même l'établira.
\VS{6}Yahweh compte en inscrivant les peuples : C'est là qu'ils sont nés. Sélah.
\VS{7}Et les chantres, de même que les joueurs de flûte, toutes mes sources sont en toi.
\Chap{88}
\TextTitle{Lamentation dans l'affliction}
\VerseOne{}Cantique. Psaume des fils de Koré, donné au chef des chantres. Pour chanter sur la flûte. Cantique d'Héman, l'Ezrahite.
\VS{2}Yahweh ! Dieu de ma délivrance, je crie jour et nuit devant toi\FTNT{Lu. 18:7.}.
\VS{3}Que ma prière parvienne en ta présence ; étends ton oreille à mon cri.
\VS{4}Car mon âme est rassasiée de maux, et ma vie atteint le scheol\FTNT{Lu. 16:23.}.
\VS{5}On m'a mis au rang de ceux qui descendent dans la fosse\FTNT{Ps. 28:1 ; Ps. 31:13.} ; je suis devenu comme un homme qui n'a plus de vigueur.
\VS{6}Je suis étendu parmi les morts, semblable à ceux qui sont tués et couchés dans la tombe, à ceux dont tu n'as plus le souvenir, et qui sont séparés par ta main.
\VS{7}Tu m'as jeté dans une fosse profonde, dans les ténèbres, dans les abîmes.
\VS{8}Ta fureur se pose sur moi, et tu m'as accablé de tous tes flots. Sélah.
\VS{9}Tu as éloigné de moi ceux de qui j'étais connu, tu m'as mis en abomination devant eux ; je suis enfermé et je ne peux sortir.
\VS{10}Mes yeux se consument dans la souffrance ; Yahweh ! Je crie à toi tout le jour ! J'étends mes mains vers toi\FTNT{Ex. 9:29 ; 1 R. 8:22. ; Job. 17:7} !
\VS{11}Est-ce pour les morts que tu fais des miracles ? Les morts se relèveront-ils pour te célébrer\FTNT{1 Th. 4:16 ; 1 Co. 15:12-13.} ? Sélah.
\VS{12}Parle-t-on de ta bonté dans le sépulcre, de ta fidélité dans le tombeau\FTNT{Ep. 4:9-10 ; 1 Pi. 3:18-20.} ?
\VS{13}Connaîtra-t-on tes merveilles dans les ténèbres et ta justice dans la terre de l'oubli ?
\VS{14}Mais moi, ô Yahweh ! J'implore ton secours, ma prière s'élève dès le matin.
\VS{15}Yahweh ! Pourquoi rejettes-tu mon âme, pourquoi me caches-tu ta face\FTNT{Mt. 27:46 ; Mc. 15:34.} ?
\VS{16}Je suis malheureux et moribond dès ma jeunesse ; j'ai été exposé à tes terreurs, et je ne sais pas où j'en suis.
\VS{17}Les ardeurs de ta colère sont passées sur moi et tes terreurs m'anéantissent\FTNT{Es. 53:5.}.
\VS{18}Elles m'environnent tout le jour comme des eaux, elles m'enveloppent toutes à la fois.
\VS{19}Tu as éloigné de moi mon ami et mon compagnon, mes connaissances ont disparu\FTNT{Mt. 26:56.}.
\Chap{89}
\TextTitle{«Heureux le peuple qui connaît le son de la trompette »}
\VerseOne{}Cantique d'Ethan, l'Ezrachite.
\VS{2}Je chanterai toujours les bontés de Yahweh ; je ferai connaître de ma bouche ta fidélité de génération en génération.
\VS{3}Car je dis : Ta bonté a des fondements éternels, tu établis ta fidélité dans les cieux quand tu dis :
\VS{4}J'ai traité alliance avec mon élu, j'ai fait serment à David mon serviteur :
\VS{5}J'affermirai ta postérité pour toujours, et j'établirai ton trône de génération en génération\FTNT{2 S. 7:8-16.}. Sélah.
\VS{6}Les cieux célèbrent tes merveilles, ô Yahweh ! Ta fidélité aussi est célébrée dans l'assemblée des saints.
\VS{7}Car qui dans le ciel peut se comparer à Yahweh ? Qui est semblable à Yahweh parmi les fils de Dieu ?
\VS{8}Dieu se rend extrêmement terrible dans le conseil secret des saints, il est plus redouté que tous ceux qui sont à l'entour de lui.
\VS{9}Ô Yahweh Dieu des armées ! Qui est semblable à toi, puissant Yahweh ? Aussi ta fidélité t'environne.
\VS{10}Tu domines l'élévation des flots de la mer ; quand ses vagues s'élèvent, tu les calmes\FTNT{Job. 26:12 ; Job. 38:8-12.}.
\VS{11}Tu écrasas Rahab\FTNT{Ce terme hébreu fait référence au nom emblèmatique de l'Egypte, il signifie « largeur », « arrogance ».} comme un homme blessé à mort ; tu dispersas tes ennemis par le bras de ta force.
\VS{12}A toi sont les cieux, à toi aussi est la terre ; tu as fondé le monde, et tout ce qui est en lui.
\VS{13}Tu as créé le nord et le sud ; le Thabor et l'Hermon se réjouissent en ton Nom.
\VS{14}Ton bras est puissant, ta main est forte, ta droite est haut élevée.
\VS{15}La justice et l'équité sont la base de ton trône ; la bonté et la vérité marchent devant ta face.
\VS{16}Heureux le peuple qui connaît le son de la trompette\FTNT{1 Co. 15:52 ; Ap. 10:7.} ! Il marche, ô Yahweh ! A la clarté de ta face.
\VS{17}Il se réjouit chaque jour en ton Nom, et il se glorifie de ta justice.
\VS{18}Parce que tu es la gloire de leur force ; et notre pouvoir est distingué par ta faveur.
\VS{19}Car notre bouclier est Yahweh, et notre Roi est le Saint d'Israël.
\VS{20}Tu as autrefois parlé en vision touchant ton bien-aimé, et tu as dit : J'ai ordonné mon secours en faveur d'un homme vaillant ; j'ai élevé l'élu du milieu du peuple.
\VS{21}J'ai trouvé David mon serviteur, je l'ai oint de ma sainte huile\FTNT{1 S. 16:13 ; Ac. 13:22.} ;
\VS{22}ma main sera ferme avec lui, et mon bras le renforcera.
\VS{23}L'ennemi ne le surprendra point, et l'inique ne l'affligera point ;
\VS{24}mais j'écraserai devant lui ses adversaires, et je détruirai ceux qui le haïssent.
\VS{25}Ma fidélité et ma bonté seront avec lui, et sa gloire sera élevée en mon Nom.
\VS{26}Je mettrai sa main sur la mer, et sa droite sur les fleuves.
\VS{27}Il m'invoquera : Tu es mon Père, mon Dieu, et le rocher\FTNT{Voir commentaire en Es. 8:14.} de ma délivrance.
\VS{28}Aussi je ferai de lui le premier-né\FTNT{Col. 1:15.}, le plus élevé des rois de la terre.
\VS{29}Je lui garderai ma bonté à toujours, et mon alliance lui sera assurée.
\VS{30}Je rendrai éternelle sa postérité, et son trône comme les jours des cieux.
\VS{31}Mais si ses fils abandonnent ma loi, et ne marchent point selon mes ordonnances ;
\VS{32}s'ils violent mes statuts, et qu'ils ne gardent point mes commandements ;
\VS{33}je punirai de la verge leur transgression, et de plaie leur iniquité.
\VS{34}Mais je ne retirerai point de lui ma bonté, et je ne lui trahirai point ma fidélité.
\VS{35}Je ne violerai point mon alliance, et je ne changerai point ce qui est sorti de mes lèvres.
\VS{36}J'ai une fois juré par ma sainteté : Mentirais-je à David\FTNT{Hé. 6:13.} ?
\VS{37}Sa postérité sera à toujours, et son trône sera devant moi comme le soleil.
\VS{38}Il aura une durée éternelle comme la lune ; le témoin qui est dans le ciel est fidèle. Sélah.
\VS{39}Néanmoins, tu l'as rejeté et dédaigné\FTNT{Es. 53:3.} ; tu t'es mis en grande colère contre ton oint.
\VS{40}Tu as rejeté l'alliance faite avec ton serviteur ; tu as souillé sa couronne en la jetant par terre.
\VS{41}Tu as rompu toutes ses murailles ; tu as mis en ruines ses forteresses.
\VS{42}Tous ceux qui passaient par le chemin l'ont pillé ; il a été mis en opprobre à ses voisins.
\VS{43}Tu as élevé la droite de ses adversaires, tu as réjoui tous ses ennemis.
\VS{44}Tu as fait reculer le tranchant de son épée, et tu ne l'as point élevé dans le combat.
\VS{45}Tu as fait cesser sa splendeur, et tu as jeté par terre son trône.
\VS{46}Tu as abrégé les jours de sa jeunesse et l'as couvert de honte. Sélah.
\VS{47}Jusqu'à quand, ô Yahweh ? Te cacheras-tu à jamais ? Ta fureur s'embrasera-t-elle comme un feu ?
\VS{48}Souviens-toi quelle est la durée de ma vie ; pourquoi aurais-tu créé en vain tous les fils des hommes ?
\VS{49}Qui est l'homme qui vivra et ne verra point la mort, et qui garantira son âme de la main du scheol\FTNT{1 Co. 15:54-57.} ? Sélah.
\VS{50}Seigneur, où sont tes bontés premières que tu juras à David dans ta fidélité ?
\VS{51}Seigneur ! Souviens-toi de l'opprobre de tes serviteurs, et comment je porte dans mon sein l'opprobre qui nous a été fait par tous les peuples nombreux.
\VS{52}Souviens-toi des outrages de tes ennemis, ô Yahweh ! Des outrages contre les pas de ton oint.
\VS{53}Béni soit à toujours Yahweh ; amen ! Oui, amen !
\Chap{90}
\TextTitle{Mortalité de l'homme}
\VerseOne{}Prière de Moïse, homme de Dieu\FTNT{De. 33:1.}. Seigneur ! Tu as été pour nous un refuge de génération en génération.
\VS{2}Avant que les montagnes soient nées et que tu aies formé la terre et le monde, d'éternité en éternité, tu es Dieu\FTNT{Ge. 17:1 ; Es. 40:28.}.
\VS{3}Tu fais revenir l'homme à la poussière, et tu dis : Fils des hommes, retournez\FTNT{Ge. 3:19 ; Ec. 12:7.} !
\VS{4}Car mille ans sont à tes yeux comme le jour d'hier qui est passé, et comme une veille de la nuit\FTNT{Ps. 39:5 ; 2 Pi. 3:8.}.
\VS{5}Tu les emportes semblables à un songe qui, le matin, passe comme l'herbe :
\VS{6}Elle fleurit au matin et reverdit ; le soir on la coupe et elle se fane\FTNT{1 Pi. 1:24.}.
\VS{7}Car nous sommes consumés par ta colère et nous sommes troublés par ta fureur.
\VS{8}Tu as mis devant toi nos iniquités, et à la lumière de ta face nos fautes cachées.
\VS{9}Car tous nos jours s'en vont par ta grande colère, et nos années se consument dans un soupir.
\VS{10}Les jours de nos années reviennent à soixante-dix ans, et pour les plus forts, à quatre-vingts ans ; l'orgueil qu'ils en tirent n'est que peine et misère ; car il passe vite, et nous nous envolons.
\VS{11}Qui connaît, selon ta crainte, la force de ton indignation et de ta grande colère ?
\VS{12}Enseigne-nous à compter nos jours, afin que nous puissions avoir un cœur rempli de sagesse.
\VS{13}Yahweh ! Reviens ! Jusqu'à quand ? Sois apaisé envers tes serviteurs.
\VS{14}Rassasie-nous chaque matin de ta bonté, afin que nous nous réjouissions et que nous soyons joyeux tout le long de nos jours.
\VS{15}Réjouis-nous autant de jours que tu nous as affligés, autant d'années que nous avons vu le malheur.
\VS{16}Que ton œuvre se voie sur tes serviteurs, et ta gloire sur leurs fils.
\VS{17}Que la grâce de Yahweh, notre Dieu, soit sur nous, et affermis l'œuvre de nos mains ; oui, affermis l'œuvre de nos mains !
\Chap{91}
\TextTitle{La sécurité et la fidélité de Yahweh}
\VerseOne{}Celui qui demeure sous la couverture\FTNT{Jésus est notre couverture spirituelle.} du Très-Haut, repose à l'ombre du Tout-Puissant.
\VS{2}Je dis à Yahweh : Tu es ma retraite et ma forteresse, tu es mon Dieu en qui je me confie.
\VS{3}Certes, il te délivre du filet de l'oiseleur, de la peste et de ses ravages.
\VS{4}Il te couvrira de ses plumes et tu trouveras un refuge sous ses ailes ; sa fidélité est un bouclier et une cuirasse.
\VS{5}Tu ne craindras ni les terreurs de la nuit, ni la flèche qui vole le jour\FTNT{Pr. 3:23-24.},
\VS{6}ni la peste qui marche dans les ténèbres, ni la destruction qui frappe en plein midi.
\VS{7}Que mille tombent à ton côté et dix mille à ta droite, tu ne seras pas atteint.
\VS{8}De tes yeux tu regarderas et tu verras la rétribution des méchants.
\VS{9}Car tu es mon refuge, ô Yahweh ! Tu fais du Très-Haut ta demeure.
\VS{10}Aucun malheur ne s'approchera de toi, aucun fléau n'approchera de ta tente\FTNT{Ex. 8:18-19 ; Ps. 121:6-8.}.
\VS{11}Car il ordonnera à ses anges de te garder dans toutes tes voies.
\VS{12}Ils te porteront sur les mains, de peur que ton pied ne heurte contre une pierre\FTNT{Mt. 4:5-6 ; Lu. 4:9-11.}.
\VS{13}Tu marcheras sur le lion et sur l'aspic, tu piétineras le lionceau et le dragon.
\VS{14}Puisqu'il m'aime, je le délivrerai ; je le mettrai sur les hauteurs, parce qu'il connaît mon Nom.
\VS{15}Il m'invoquera et je l'exaucerai ; je serai avec lui dans la détresse, je le délivrerai et le glorifierai.
\VS{16}Je le rassasierai de jours et je lui ferai voir ma délivrance.
\Chap{92}
\TextTitle{Proclamer la louange de Dieu}
\VerseOne{}Psaume. Cantique pour le jour du sabbat.
\VS{2}C'est une belle chose que de célébrer Yahweh, et de chanter ton Nom, ô Très-Haut\FTNT{Ps. 147:1.} !
\VS{3}Afin d'annoncer chaque matin ta bonté et ta fidélité toutes les nuits\FTNT{Ps. 59:17 ; Ps. 88:14 ; Ps. 89:2.}.
\VS{4}Sur l'instrument à dix cordes, sur le luth, et par un cantique prémédité sur la harpe.
\VS{5}Car, ô Yahweh ! Tu me réjouis par tes œuvres, je me réjouis des œuvres de tes mains.
\VS{6}Ô Yahweh ! Que tes œuvres sont magnifiques ! Tes pensées sont merveilleusement profondes\FTNT{Es. 55:8-9 ; Job. 5:9.}.
\VS{7}L'homme stupide n'y connaît rien et le fou n'y prend point garde\FTNT{Es. 5:12 ; Ro. 1:21.}.
\VS{8}Les méchants croissent comme l'herbe, et tous les ouvriers d'iniquité fleurissent pour être exterminés éternellement\FTNT{Jé. 12:1-2 ; Mal. 3:15 ; Ps. 37:2 ; Ps. 73:1-20.}.
\VS{9}Mais toi, ô Yahweh ! Tu es élevé à toujours.
\VS{10}Car voici tes ennemis, ô Yahweh ! Car voici, tes ennemis périssent, tous les ouvriers d'iniquité sont dispersés.
\VS{11}Mais tu élèveras ma corne comme celle d'un buffle, je serai oint d'une huile fraîche\FTNT{Ps. 23:5 ; Hé. 1:9.}.
\VS{12}Mes yeux se plaisent à regarder ceux qui m'épient, et mes oreilles à entendre les méchants qui s'élèvent contre moi.
\VS{13}Le juste fleurit comme le palmier, il croît comme le cèdre au Liban.
\VS{14}Etant plantés dans la maison de Yahweh, ils fleurissent dans les parvis de notre Dieu.
\VS{15}Ils portent encore des fruits dans la blanche vieillesse ; ils sont gras et verdoyants\FTNT{Os. 14:6 ; Ps. 1:3.},
\VS{16}afin d'annoncer que Yahweh est droit ; c'est mon rocher, et il n'y a point d'injustice en lui.
\Chap{93}
\TextTitle{Majesté et puissance de Yahweh}
\VerseOne{}Yahweh règne, il est revêtu de majesté ; Yahweh est revêtu de force, il s'en est ceint ; aussi le monde est ferme, tellement qu'il ne sera point ébranlé.
\VS{2}Ton trône est établi dès lors, tu es de toute éternité\FTNT{Ps. 9:8 ; Hé. 1:8.}.
\VS{3}Les fleuves élevés, ô Yahweh ! Les fleuves augmentent leur bruit, les fleuves élèvent leurs flots\FTNT{Ps. 46:4 ; Ps. 65:7-8.} ;
\VS{4}Yahweh, qui est dans les lieux élevés, est plus puissant que le bruit des grandes eaux, et que les fortes vagues de la mer\FTNT{Es. 57:15 ; Ac. 7:49.}.
\VS{5}Tes préceptes sont entièrement fidèles. Yahweh ! La sainteté orne ta maison pour une longue durée.
\Chap{94}
\TextTitle{A Dieu seul la vengeance}
\VerseOne{}Ô Yahweh ! Dieu des vengeances, Dieu des vengeances, fais briller ta splendeur !
\VS{2}Toi, juge de la terre, élève-toi ! Rends aux orgueilleux selon leurs œuvres.
\VS{3}Jusqu'à quand les méchants, ô Yahweh ! Jusqu'à quand les méchants se réjouiront-ils ?
\VS{4}Jusqu'à quand tous les ouvriers d'iniquité discourront-ils et diront-ils des paroles rudes et se vanteront-ils ?
\VS{5}Yahweh, ils écrasent ton peuple et affligent ton héritage.
\VS{6}Ils tuent la veuve et l'étranger, et ils mettent à mort les orphelins.
\VS{7}Ils disent : Yahweh ne le voit point, le Dieu de Jacob n'entend rien.
\VS{8}Vous les plus abrutis d'entre les peuples, prenez garde à ceci ; et vous insensés, quand serez-vous intelligents ?
\VS{9}Celui qui a planté l'oreille, n'entendrait-il point ? Celui qui a formé l'œil, ne verrait-t-il point\FTNT{Ex. 4:11 ; Pr. 20:12.} ?
\VS{10}Celui qui châtie les nations, celui qui enseigne la science aux hommes, ne réprimanderait-il point\FTNT{Ap. 19:15.} ?
\VS{11}Yahweh connaît les pensées des hommes qui ne sont que vanité.
\VS{12}Heureux l'homme que tu châties, ô Yahweh\FTNT{Hé. 12:6.} ! Que tu instruis par ta loi,
\VS{13}afin qu'il soit dans la paix aux jours du malheur, jusqu'à ce que la fosse soit creusée pour le méchant !
\VS{14}Car Yahweh ne délaisse point son peuple et n'abandonne point son héritage\FTNT{Es. 49:15 ; Ro. 11:2.}.
\VS{15}C'est pourquoi le jugement s'unira à la justice, et tous ceux qui sont droits de cœur le suivront.
\VS{16}Qui se lèvera pour moi contre les méchants\FTNT{Job. 19:25 ; Ro. 8:31.} ? Qui m'assistera contre les ouvriers d'iniquité ?
\VS{17}Si Yahweh n'était pas mon secours, mon âme serait bien vite dans la demeure du silence.
\VS{18}Quand je dis : Mon pied chancelle, ta bonté me soutient, ô Yahweh !
\VS{19}Quand j'ai beaucoup de pensées au-dedans de moi, tes consolations font les délices de mon âme.
\VS{20}Serais–tu l'allié du trône de méchanceté, qui forge des injustices contre les règles de la justice ?
\VS{21}Ils se rassemblent contre l'âme du juste et condamnent le sang innocent\FTNT{Mt. 27:1-4 ; Mt. 27:24.}.
\VS{22}Or Yahweh est pour moi une haute retraite ; mon Dieu est le rocher de mon refuge.
\VS{23}Il fera retourner sur eux leur iniquité et les détruira par leur propre méchanceté. Yahweh notre Dieu les détruira\FTNT{Mt. 13:30 ; Ap. 20:14-15.}.
\Chap{95}
\TextTitle{Adoration à Yahweh}
\VerseOne{}Venez, chantons à Yahweh, poussons des cris de réjouissance au rocher de notre salut.
\VS{2}Allons au-devant de lui en lui présentant nos louanges ; et poussons devant lui des cris de réjouissance en chantant des psaumes.
\VS{3}Car Yahweh est un grand Dieu, et il est un grand Roi au-dessus de tous les dieux.
\VS{4}Les lieux les plus profonds de la terre sont dans sa main, et les sommets des montagnes sont à lui.
\VS{5}C'est à lui qu'appartient la mer, car lui-même l'a faite, et ses mains ont formé la terre.
\VS{6}Venez, prosternons-nous, inclinons-nous, et mettons-nous à genoux devant Yahweh qui nous a faits\FTNT{Ps. 96:9 ; Ph. 2:10-11.}.
\VS{7}Car il est notre Dieu, et nous sommes le peuple de son pâturage, et les brebis que sa main conduit\FTNT{Ps. 23:1 ; Ps. 100:3 ; Jn. 10:11.}. Si vous entendez aujourd'hui sa voix,
\VS{8}n'endurcissez point votre cœur\FTNT{Hé. 3:8 ; Hé. 4:7.}, comme à Meriba, comme à la journée de Massa, au désert ;
\VS{9}là où vos pères m'ont tenté et éprouvé bien qu'ils virent mes œuvres\FTNT{Ex. 17:7.}.
\VS{10}J'ai eu cette génération en dégoût durant quarante ans, et j'ai dit : C'est un peuple dont le cœur s'égare ; et ils n'ont point connu mes voies ;
\VS{11}c'est pourquoi j'ai juré dans ma colère, ils n'entreront pas dans mon repos\FTNT{No. 14:22-23 ; Hé. 3:15-19 ; Hé. 4:3.}.
\Chap{96}
\TextTitle{La grandeur et la gloire de Dieu}
\VerseOne{}Chantez à Yahweh un cantique nouveau\FTNT{Es. 42:10 ; Ps. 98:1 ; Ap. 5:9 ; Ap. 14:3.} ! Vous tous habitants de la terre chantez à Yahweh !
\VS{2}Chantez à Yahweh, bénissez son Nom ! Prêchez de jour en jour sa délivrance !
\VS{3}Racontez sa gloire parmi les nations, ses merveilles parmi tous les peuples\FTNT{Ps. 67:5.}.
\VS{4}Car Yahweh est grand et digne d'être loué ; il est redoutable au-dessus de tous les dieux\FTNT{Ph. 2:9 ; Ap. 5:9.} ;
\VS{5}car tous les dieux des peuples ne sont que des idoles, mais Yahweh a fait les cieux.
\VS{6}La splendeur et la magnificence marchent devant lui, la force et la beauté sont dans son lieu saint.
\VS{7}Familles des peuples, rendez à Yahweh, rendez à Yahweh la gloire et la puissance !
\VS{8}Rendez à Yahweh la gloire due à son Nom ! Apportez des offrandes et entrez dans ses parvis !
\VS{9}Prosternez–vous devant Yahweh avec des ornements sacrés ; tremblez devant lui, vous toute la terre !
\VS{10}Dites parmi les nations : Yahweh règne ; même le monde est affermi, il ne sera point ébranlé ; il jugera les peuples avec équité.
\VS{11}Que les cieux se réjouissent et que la terre soit dans l'allégresse ! Que la mer tonne avec tout ce qui la remplit !
\VS{12}Que les champs s'égayent avec tout ce qui est en eux. Alors tous les arbres de la forêt chanteront de joie
\VS{13}devant Yahweh, car il vient ! Car il vient pour juger la terre ; il jugera avec justice le monde habitable et les peuples selon sa fidélité.
\Chap{97}
\TextTitle{Aimer Dieu, c'est haïr le mal}
\VerseOne{}Yahweh règne, que la terre soit dans l'allégresse et que les îles nombreuses s'en réjouissent\FTNT{Es. 42:10 ; Ps. 86:9 ; Ps. 93:1 ; Ps. 99:1} !
\VS{2}La nuée et l'obscurité sont autour de lui ; la justice et le jugement sont la base de son trône.
\VS{3}Le feu marche devant lui et embrase tout autour ses adversaires.
\VS{4}Ses éclairs éclairent le monde, et la terre le voit et tremble tout étonnée\FTNT{Job. 38:35 ; Ap. 4:5.}.
\VS{5}Les montagnes se fondent comme de la cire\FTNT{Mi. 1:4.}, à cause de la présence de Yahweh, à cause de la présence du Seigneur de toute la terre.
\VS{6}Les cieux annoncent sa justice et tous les peuples voient sa gloire.
\VS{7}Que tous ceux qui servent les images et qui se glorifient des idoles soient confus\FTNT{De. 4:25-26 ; 1 S. 5:1-5.} ; vous dieux, prosternez-vous tous devant lui.
\VS{8}Sion l'a entendu et s'en est réjouie ; et les filles de Juda se sont égayées pour l'amour de tes jugements, ô Yahweh !
\VS{9}Yahweh, tu es le Très-Haut sur toute la terre ; tu es élevé au-dessus de tous les dieux.
\VS{10}Vous qui aimez Yahweh, haïssez le mal\FTNT{Am. 5:14-15 ; Ro. 12:9.} ! Il garde les âmes de ses bien-aimés et les délivre de la main des méchants\FTNT{Ps. 34:8 ; Jn. 10:28-29.}.
\VS{11}La lumière est faite pour le juste\FTNT{Mt. 5:15-16.} et la joie pour ceux qui sont droits de cœur.
\VS{12}Justes, réjouissez-vous en Yahweh et célébrez la mémoire de sa sainteté.
\Chap{98}
\TextTitle{Invitation à la louange}
\VerseOne{}Psaume. Chantez à Yahweh un cantique nouveau, car il a fait des choses merveilleuses ; sa droite et le bras de sa sainteté l'ont délivré\FTNT{Es. 52:10 ; Es 53:1 ; Es. 63:3-5.}.
\VS{2}Yahweh a fait connaître son salut\FTNT{Il est question de la révélation de Jésus. Voir commentaire en Es. 26:1.}, il a révélé sa justice devant les yeux des nations.
\VS{3}Il s'est souvenu de sa bonté et de sa fidélité envers la maison d'Israël ; toutes les extrémités de la terre ont vu la délivrance de notre Dieu\FTNT{Es. 49:6 ; Lu. 1:72 ; Ac. 13:47.}.
\VS{4}Vous tous habitants de la terre, poussez des cris de réjouissance à Yahweh ! Faites retentir vos cris et chantez de joie !
\VS{5}Chantez à Yahweh avec la harpe, avec la harpe et avec une voix mélodieuse !
\VS{6}Poussez des cris de réjouissance avec le shofar au son du cor devant le Roi, Yahweh !
\VS{7}Que la mer tonne avec tout ce qu'elle contient, que la terre et ceux qui y habitent fassent éclater leurs cris !
\VS{8}Que les fleuves frappent des mains et que les montagnes chantent de joie
\VS{9}devant Yahweh ! Car il vient pour juger la terre\FTNT{Yahweh, qui vient pour juger la terre, est Jésus-Christ ( 
Za. 14:1-7 ; 2 Ti. 4:1 ; Ap. 19:15).} ; il jugera le monde habitable avec justice et les peuples avec équité.
\Chap{99}
\TextTitle{Grandeur, justice, et sainteté de Dieu}
\VerseOne{}Yahweh règne, que les peuples tremblent ; il est assis entre les chérubins, que la terre soit ébranlée\FTNT{Ex. 25:22 ; Es. 37:16.}.
\VS{2}Yahweh est grand en Sion, et il est élevé au-dessus de tous les peuples.
\VS{3}Ils célébreront ton Nom, grand et terrible, car il est saint.
\VS{4}Qu'on célèbre la force du roi qui aime la justice ! Tu as ordonné l'équité, tu as prononcé des jugements justes en Jacob.
\VS{5}Exaltez Yahweh notre Dieu et prosternez-vous devant son marchepied ! Il est saint !
\VS{6}Moïse et Aaron étaient parmi ses sacrificateurs\FTNT{Ex. 31:10 ; Lé. 2:2.} ; et Samuel parmi ceux qui invoquaient son Nom ; ils invoquaient Yahweh et il leur répondait\FTNT{1 S. 12:18-19.}.
\VS{7}Il leur parlait de la colonne de nuée ; ils ont gardé ses préceptes et l'ordonnance qu'il leur avait donnée.
\VS{8}Ô Yahweh, mon Dieu ! Tu les as exaucés, tu as été pour eux un Dieu qui pardonne\FTNT{Hé. 10:16-17.}, mais tu les as punis de leurs fautes.
\VS{9}Exaltez Yahweh notre Dieu ! Prosternez-vous sur la montagne de sa sainteté ! Car Yahweh, notre Dieu est saint !
\Chap{100}
\TextTitle{Célébrer et bénir le nom de Yahweh}
\VerseOne{}Psaume de louange. Vous tous habitants de la terre, poussez des cris de réjouissance à Yahweh !
\VS{2}Servez Yahweh avec allégresse, venez devant lui avec un chant de joie !
\VS{3}Sachez que Yahweh est Dieu. C'est lui qui nous a faits, ce n'est pas nous qui nous sommes faits ; nous sommes son peuple et le troupeau de son pâturage\FTNT{Ps. 79:13 ; Ps. 80:2 ; Ps. 95:6 ; Ps. 119:73.}.
\VS{4}Entrez dans ses portes avec des louanges ; et dans ses parvis, avec des cantiques. Célébrez-le, bénissez son Nom !
\VS{5}Car Yahweh est bon ; sa bonté demeure à toujours et sa fidélité de génération en génération.
\Chap{101}
\TextTitle{Appel à l'intégrité}
\VerseOne{}Psaume de David. Je chanterai la miséricorde et la justice ; Yahweh ! Je te chanterai.
\VS{2}Je me rendrai attentif à une conduite pure jusqu'à ce que tu viennes à moi ; je marcherai dans l'intégrité de mon cœur au milieu de ma maison.
\VS{3}Je ne mettrai point devant mes yeux des choses de Bélial\FTNT{Ps. 26:5 ; Ps. 119:115.}; j'ai en haine les actions de ceux qui se détournent ; elles ne s'attacheront pas à moi.
\VS{4}Le cœur mauvais s'éloignera de moi ; je ne connaîtrai pas le méchant.
\VS{5}Je retrancherai celui qui calomnie en secret son prochain ; je ne supporterai pas celui qui a les yeux élevés et le cœur enflé\FTNT{Pr. 6:16-17.}.
\VS{6}Je prendrai garde aux gens de bien du pays afin qu'ils demeurent avec moi ; celui qui marche dans la voie de l'intégrité me servira.
\VS{7}Celui qui usera de tromperie ne demeurera point dans ma maison ; celui qui profèrera des mensonges ne sera point affermi devant mes yeux.
\VS{8}Je retrancherai chaque matin tous les méchants du pays, afin d'exterminer de la cité de Yahweh tous les ouvriers d'iniquité.
\Chap{102}
\TextTitle{Yahweh, le Dieu immuable}
\VerseOne{}Prière de l'affligé étant dans l'angoisse et répandant sa plainte devant Yahweh.
\VS{2}Yahweh ! Ecoute ma prière, et que mon cri parvienne jusqu'à toi\FTNT{Ps. 69:14.}.
\VS{3}Ne cache pas ta face arrière de moi ; au jour où je suis en détresse, prête l'oreille à ma prière ; au jour où je t'invoque, hâte-toi de me répondre.
\VS{4}Car mes jours se sont évanouis comme la fumée et mes os brûlent comme dans un foyer.
\VS{5}Mon cœur est frappé et se dessèche comme l'herbe, car j'ai oublié de manger mon pain\FTNT{Mt. 4:4 ; Lu. 4:4.}.
\VS{6}Le gémissement de ma voix est tel que mes os s'attachent à ma chair\FTNT{Job. 19:20.}.
\VS{7}Je suis devenu semblable au pélican du désert ; et je suis comme la chouette des lieux sauvages.
\VS{8}Je veille et je suis semblable au passereau solitaire sur le toit.
\VS{9}Mes ennemis m'outragent tous les jours, et ceux qui sont furieux contre moi jurent contre moi.
\VS{10}Car j'ai mangé la cendre comme le pain et j'ai mêlé des larmes à ma boisson,
\VS{11}à cause de ta colère et de ta fureur ; car après m'avoir soulevé, tu m'as jeté par terre.
\VS{12}Mes jours sont comme l'ombre qui décline et je deviens sec comme l'herbe.
\VS{13}Mais toi, ô Yahweh ! Tu demeures éternellement, et ta mémoire est de génération en génération.
\VS{14}Tu te lèveras, tu auras compassion de Sion ; car il est temps d'en avoir pitié, parce que le temps assigné est échu.
\VS{15}Car tes serviteurs aiment ses pierres et chérissent sa poussière.
\VS{16}Alors les nations redouteront le Nom de Yahweh, et tous les rois de la terre ta gloire.
\VS{17}Quand Yahweh aura édifié Sion, quand il aura été vu dans sa gloire,
\VS{18}quand il aura eu égard à la prière du désolé et qu'il n'aura point méprisé leur supplication.
\VS{19}Cela sera enregistré pour la génération à venir, le peuple qui sera créé louera Yahweh !
\VS{20}Car il regarde du lieu élevé de sa sainteté. Du haut des cieux, Yahweh regarde la terre,
\VS{21}pour entendre le gémissement des prisonniers, pour délier ceux qui étaient voués à la mort\FTNT{Es. 42:6-7 ; Es. 61:1 ; Lu. 4:18-19.},
\VS{22}afin qu'on annonce le Nom de Yahweh dans Sion et sa louange dans Jérusalem,
\VS{23}quand les peuples se seront joints ensemble, et les royaumes aussi, pour servir Yahweh.
\VS{24}Il a brisé ma force en chemin, il a abrégé mes jours.
\VS{25}Je dis : Mon Dieu, ne m'enlève point au milieu de mes jours dont les années durent éternellement.
\VS{26}Tu as jadis fondé la terre, et les cieux sont l'ouvrage de tes mains.
\VS{27}Ils périront, mais tu subsisteras, ils s'useront tous comme un vêtement ; tu les changeras comme un habit, et ils seront changés.
\VS{28}Mais toi, tu es toujours le même et tes années ne seront jamais achevées.
\VS{29}Les fils de tes serviteurs habiteront près de toi et leur postérité sera établie devant toi.
\Chap{103}
\TextTitle{Yahweh, le Dieu miséricordieux et compatissant}
\VerseOne{}Psaume de David. Mon âme, bénis Yahweh, et que tout ce qui est en moi bénisse son saint Nom.
\VS{2}Mon âme, bénis Yahweh, et n'oublie pas un de ses bienfaits\FTNT{De. 6:12.}.
\VS{3}C'est lui qui pardonne toutes tes iniquités, qui guérit toutes tes infirmités\FTNT{Es. 33:24 ; Es. 53:5 ; Jé. 17:14 ; Ps. 130:3-4 ; Mt. 9:6 ; Lu. 7:47.} ;
\VS{4}qui garantit ta vie de la fosse\FTNT{Es. 59:20 ; Ps. 106:10.}, qui te couronne de bonté et de compassions ;
\VS{5}qui rassasie ta bouche de biens ; ta jeunesse est renouvelée comme celle de l'aigle\FTNT{Es. 40:31.}.
\VS{6}Yahweh fait justice et droit à tous les opprimés\FTNT{Ps. 146:7.}.
\VS{7}Il a fait connaître ses voies à Moïse, et ses exploits aux fils d'Israël\FTNT{Ex. 33:12-17.}.
\VS{8}Yahweh est compatissant, miséricordieux, lent à la colère, et riche en bonté.
\VS{9}Il ne conteste pas éternellement, et il ne garde point à toujours sa colère\FTNT{Es. 57:16 ; Jé. 3:5 ; Mi. 7:18.}.
\VS{10}Il ne nous traite pas selon nos péchés, et ne nous rend point selon nos iniquités\FTNT{Esd. 9:13.}.
\VS{11}Car autant les cieux sont élevés au-dessus de la terre, autant sa bonté est grande sur ceux qui le craignent.
\VS{12}Il éloigne de nous nos transgressions, autant que l'orient est éloigné de l'occident\FTNT{Es. 38:17.}.
\VS{13}Comme un père a compassion de ses fils, Yahweh a compassion de ceux qui le craignent\FTNT{Mal. 3:17 ; Lu. 11:11-13.}.
\VS{14}Car il sait bien de quoi nous sommes faits, se souvenant que nous ne sommes que poussière.
\VS{15}L'homme ! Ses jours sont comme l'herbe, il fleurit comme la fleur d'un champ.
\VS{16}Car le vent étant passé par-dessus, elle n'est plus, et son lieu ne la reconnaît plus.
\VS{17}Mais la miséricorde de Yahweh est de tout temps, et elle sera pour toujours en faveur de ceux qui le craignent ; et sa justice en faveur des fils de leurs fils ;
\VS{18}pour ceux qui gardent son alliance, et qui se souviennent de ses commandements pour les faire\FTNT{De. 7:9.}.
\VS{19}Yahweh a établi son trône dans les cieux, et son règne domine sur tout.
\VS{20}Bénissez Yahweh, vous ses anges puissants en force, qui faites ses affaires, en obéissant à la voix de sa parole.
\VS{21}Bénissez Yahweh, vous toutes ses armées, qui êtes ses serviteurs faisant sa volonté.
\VS{22}Bénissez Yahweh, vous toutes ses œuvres, par tous les lieux de sa domination. Mon âme, bénis Yahweh !
\Chap{104}
\TextTitle{Yahweh, le Dieu de toute la création}
\VerseOne{}Mon âme, bénis Yahweh. Ô Yahweh mon Dieu, tu es merveilleusement grand, tu es revêtu de majesté et de splendeur.
\VS{2}Il s'enveloppe de lumière comme d'un vêtement, il étend les cieux comme un voile\FTNT{Es. 40:22 ; Job. 9:8 ; 1 Ti. 6:16.}.
\VS{3}Avec les eaux, il va à la rencontre de sa demeure ; il fait des grosses nuées son char, il se promène sur les ailes du vent\FTNT{Es. 19:1 ; Ps. 18:10 ; Ap. 14:14.}.
\VS{4}Il fait des vents ses messagers, et des flammes de feu ses serviteurs\FTNT{Ps. 148:8 ; Hé. 1:7 ; Jn. 3:8.}.
\VS{5}Il a fondé la terre sur ses bases, elle ne sera jamais ébranlée\FTNT{Ps. 24:1-2 ; Ps. 78:69 ; Ps. 93:1 ; Job. 26:7 ; Job. 38:4-6.}.
\VS{6}Tu l'avais couverte de l'abîme comme d'un vêtement, les eaux se tenaient sur les montagnes\FTNT{Ge. 1:2.}.
\VS{7}Elles s'enfuirent à ta menace et se mirent promptement en fuite au son de ton tonnerre.
\VS{8}Les montagnes s'élevèrent et les vallées s'abaissèrent au même lieu que tu leur avais fixé.
\VS{9}Tu as posé une limite que les eaux ne doivent point franchir, afin qu'elles ne reviennent plus couvrir la terre\FTNT{Ge. 1:9 ; Jé. 5:2 ; Pr. 8:29 ; Job. 26:10.}.
\VS{10}C'est lui qui conduit les sources par les vallées, elles se promènent entre les monts.
\VS{11}Elles abreuvent toutes les bêtes des champs, les ânes sauvages y étanchent leur soif.
\VS{12}Les oiseaux des cieux se tiennent auprès d'elles, et font résonner leur voix parmi les rameaux.
\VS{13}Il abreuve les montagnes de ses chambres hautes ; la terre est rassasiée du fruit de tes œuvres.
\VS{14}Il fait germer l'herbe pour le bétail, et les plantes pour le besoin de l'homme, faisant sortir le pain de la terre,
\VS{15}et le vin qui réjouit le cœur de l'homme\FTNT{Jg. 9:11 ; Pr. 31:6-7.}, qui fait resplendir son visage avec l'huile, et qui soutient le cœur de l'homme avec le pain.
\VS{16}Les hauts arbres de Yahweh en sont rassasiés, ainsi que les cèdres du Liban qu'il a plantés,
\VS{17}afin que les oiseaux y fassent leurs nids. Quant à la cigogne, les sapins sont sa demeure.
\VS{18}Les hautes montagnes sont pour les chamois, et les rochers sont la retraite des lapins.
\VS{19}Il a fait la lune pour les saisons, et le soleil sait quand il doit se coucher\FTNT{Ge. 1:16.}.
\VS{20}Tu amènes les ténèbres, et il fait nuit ; alors toutes les bêtes de la forêt sont en mouvement.
\VS{21}Les lionceaux rugissent après la proie pour demander à Dieu leur nourriture.
\VS{22}Le soleil se lève-t-il ? Ils se retirent et se couchent dans leurs tanières.
\VS{23}Alors l'homme sort pour se rendre à son ouvrage, et à son travail jusqu'au soir.
\VS{24}Ô Yahweh, que tes œuvres sont en grand nombre ! Tu les as toutes faites avec sagesse ; la terre est pleine de tes richesses.
\VS{25}Cette mer grande et spacieuse, là où des animaux sans nombre se meuvent, des petites bêtes avec des grandes !
\VS{26}Là se promènent les navires, et ce léviathan que tu as formé pour jouer dans les flots.
\VS{27}Ils s'attendent tous à toi afin que tu leur donnes la nourriture en leur temps.
\VS{28}Quand tu la leur donnes, ils la recueillent, et quand tu ouvres ta main, ils sont rassasiés de biens.
\VS{29}Caches-tu ta face ? Ils sont troublés ; retires-tu leur souffle ? Ils défaillent et retournent dans leur poussière.
\VS{30}Tu envoies ton souffle, ils sont créés ; et tu renouvelles la face de la terre.
\VS{31}Que la gloire de Yahweh subsiste à toujours, que Yahweh se réjouisse dans ses œuvres !
\VS{32}Il jette son regard sur la terre et elle tremble ; il touche les montagnes et elles fument.
\VS{33}Je chanterai à Yahweh durant ma vie ; je chanterai à mon Dieu tant que j'existerai.
\VS{34}Ma méditation lui sera agréable, et je me réjouirai en Yahweh.
\VS{35}Que les pécheurs soient consumés de dessus la terre et qu'il n'y ait plus de méchants ! Mon âme, bénis Yahweh ! Louez Yahweh !
\Chap{105}
\TextTitle{Yahweh, le Dieu fidèle}
\VerseOne{}Célébrez Yahweh, invoquez son Nom, faites connaître parmi les peuples ses exploits.
\VS{2}Chantez-le, chantez-le, parlez de toutes ses merveilles !
\VS{3}Glorifiez-vous de son saint Nom, et que le cœur de ceux qui cherchent Yahweh se réjouisse.
\VS{4}Recherchez Yahweh et sa puissance ; cherchez continuellement sa face.
\VS{5}Souvenez-vous de ses merveilles qu'il a faites, de ses miracles, et des jugements de sa bouche.
\VS{6}La postérité d'Abraham sont ses serviteurs ; les enfants de Jacob sont ses élus !
\VS{7}Il est Yahweh notre Dieu, ses jugements sont sur toute la terre.
\VS{8}Il s'est souvenu pour toujours de son alliance, de la parole qu'il a ordonnée pour mille générations,
\VS{9}du traité qu'il a fait avec Abraham et du serment qu'il a fait à Isaac\FTNT{Ge. 17:2 ; Ge. 22:16 ; Ge. 26:3 ; Ge. 28:13 ; Ge. 33:11 ; Lu. 1:73.}.
\VS{10}Il l'a érigé pour être une ordonnance à Jacob, et à Israël pour être une alliance éternelle,
\VS{11}en disant : Je te donnerai le pays de Canaan, comme héritage qui vous est échu\FTNT{Ge. 13:15 ; Ge. 15:18.}.
\VS{12}Ils étaient alors un petit nombre de gens, très peu nombreux, et étrangers dans le pays.
\VS{13}Car ils allaient de nation en nation, et d'un royaume vers un autre peuple.
\VS{14}Il ne permit à personne de les opprimer, il châtia des rois à cause d'eux\FTNT{Ge. 35:5.},
\VS{15}disant : Ne touchez point à mes oints et ne faites point de mal à mes prophètes\FTNT{1 Ch. 16:22.} !
\VS{16}Il appela aussi la famine sur la terre, rompit le bâton du pain\FTNT{Lé. 26:26 ; Es. 3:1 ; Ez. 4:16.}.
\VS{17}Il envoya un homme devant eux ; Joseph fut vendu pour esclave\FTNT{Ge. 37:28-36.}.
\VS{18}On serra ses pieds dans des ceps, sa personne fut mise aux fers.
\VS{19}Jusqu'au temps où arriva ce qu'il avait annoncé, et où la parole de Yahweh l'éprouva.
\VS{20}Le roi le relâcha et le laissa aller ; le dominateur des peuples le délivra.
\VS{21}Il l'établit pour maître sur sa maison, et pour gouverneur sur tout son domaine\FTNT{Ge. 41:40.} ;
\VS{22}pour soumettre les princes à ses désirs, et pour instruire ses anciens.
\VS{23}Puis Israël entra en Egypte, et Jacob séjourna dans le pays de Cham\FTNT{Ge. 46:6 ; Ps. 78:51.}.
\VS{24}Yahweh rendit son peuple très fécond et le rendit plus puissant que ceux qui l'opprimaient.
\VS{25}Il changea leur cœur, au point qu'ils haïrent son peuple jusqu'à conspirer contre ses serviteurs\FTNT{Ex. 1:7-12.}.
\VS{26}Il envoya Moïse son serviteur, et Aaron, qu'il avait élu\FTNT{Ex. 4:14.}.
\VS{27}Ils accomplirent au milieu d'eux des prodiges et des miracles qu'ils avaient eu la charge de faire dans le pays de Cham.
\VS{28}Il envoya les ténèbres et fit venir l'obscurité ; et ils ne furent point rebelles à sa parole.
\VS{29}Il changea leurs eaux en sang et fit mourir leurs poissons.
\VS{30}Leur terre produisit en abondance des grenouilles jusque dans les chambres de leurs rois.
\VS{31}Il dit, et des mouches vinrent, des poux sur tout leur pays.
\VS{32}Il leur donna pour pluie de la grêle, et un feu flamboyant sur la terre.
\VS{33}Il frappa leurs vignes et leurs figuiers, et il brisa les arbres du pays.
\VS{34}Il ordonna et les sauterelles vinrent, des jeunes sauterelles sans nombre
\VS{35}qui dévorèrent toute l'herbe du pays, et qui dévorèrent le fruit de leur terroir.
\VS{36}Il frappa tous les premiers-nés du pays, les prémices de toute leur vigueur\FTNT{Lire Ex. 7 à 12.}.
\VS{37}Puis il les fit sortir avec de l'or et de l'argent, et nul ne chancela parmi ses tribus.
\VS{38}Les Egyptiens se réjouirent à leur départ, car la peur qu'ils avaient d'eux les avait saisis.
\VS{39}Il étendit la nuée pour couverture, et le feu pour éclairer la nuit.
\VS{40}Le peuple demanda et il fit venir des cailles ; et il les rassasia du pain des cieux\FTNT{Ex. 16:12-13.}.
\VS{41}Il ouvrit le rocher et les eaux en coulèrent ; elles se répandirent comme un fleuve dans les lieux arides\FTNT{Ex. 17:6.}.
\VS{42}Car il se souvint de sa parole sainte qu'il avait donnée à Abraham son serviteur\FTNT{Ge. 15:13-16.}.
\VS{43}Il fit sortir son peuple dans l'allégresse, ses élus au milieu des cris retentissants\FTNT{Ex. 15:1.}.
\VS{44}Il leur donna les terres des nations et ils possédèrent le fruit du travail des peuples,
\VS{45}afin qu'ils gardent ses statuts et qu'ils observent ses lois. Louez Yahweh !
\Chap{106}
\TextTitle{L'infidélité d'Israël}
\VerseOne{}Louez Yahweh ! Célébrez Yahweh car il est bon, car sa bonté demeure à toujours !
\VS{2}Qui pourrait réciter les exploits de Yahweh ? Qui pourrait faire retentir toute sa louange ?
\VS{3}Heureux ceux qui observent la justice, qui font en tout temps ce qui est juste !
\VS{4}Yahweh, souviens-toi de moi selon la bienveillance que tu portes à ton peuple, aie soin de moi selon ta délivrance !
\VS{5}Afin que je voie le bien de tes élus, que je me réjouisse dans la joie de ta nation, que je me glorifie avec ton héritage.
\VS{6}Nous avons péché avec nos pères, nous avons agi dans l'iniquité, nous avons fait le mal\FTNT{Da. 9:16 ; Esd. 9:7 ; Né. 1:6.}.
\VS{7}Nos pères n'ont point été attentifs à tes merveilles en Egypte ; ils ne se sont point souvenus de la multitude de tes faveurs ; mais ils furent rebelles près de la mer, vers la Mer Rouge\FTNT{Ex. 14:11.}.
\VS{8}Toutefois, il les délivra pour l'amour de son Nom, afin de faire connaître sa puissance.
\VS{9}Il menaça la Mer Rouge et elle se sécha ; et il les conduisit à travers les profondeurs de la mer comme un désert ;
\VS{10}il les délivra de la main de ceux qui les haïssaient et les racheta de la main de l'ennemi.
\VS{11}Les eaux couvrirent leurs oppresseurs, il n'en resta pas un seul\FTNT{Ex. 14:27.}.
\VS{12}Alors ils crurent à ses paroles et ils chantèrent sa louange.
\VS{13}Mais ils oublièrent vite ses œuvres et ne s'attendirent point à son conseil.
\VS{14}Ils furent épris de convoitise au désert et ils tentèrent Dieu dans le désert.
\VS{15}Alors il leur donna ce qu'ils avaient demandé, toutefois il leur envoya le dépérissement dans leur corps.
\VS{16}Ils jalousèrent dans le camp Moïse et Aaron, le saint de Yahweh.
\VS{17}La terre s'ouvrit et engloutit Dathan ; et recouvrit de terre Abiram\FTNT{No. 16.}.
\VS{18}Le feu s'alluma au milieu de leur assemblée, la flamme brûla les méchants.
\VS{19}Ils firent un veau en Horeb, et se prosternèrent devant une image de métal fondu\FTNT{Ex. 32.}.
\VS{20}Ils changèrent leur gloire contre la figure d'un bœuf qui mange l'herbe.
\VS{21}Ils oublièrent Dieu, leur libérateur, qui avait fait de grandes choses en Egypte,
\VS{22}des choses merveilleuses dans le pays de Cham, et des prodiges sur la Mer Rouge.
\VS{23}C'est pourquoi il dit qu'il les détruirait ; mais Moïse, son élu, se tint à la brèche devant lui pour détourner sa fureur, afin qu'il ne les détruisît point\FTNT{Ex. 32:11.}.
\VS{24}Ils méprisèrent le pays désirable et ne crurent point à sa parole.
\VS{25}Ils murmurèrent dans leurs tentes et n'obéirent point à la voix de Yahweh.
\VS{26}C'est pourquoi il leur jura la main levée de les faire tomber dans le désert,
\VS{27}d'accabler leur postérité parmi les nations, et de les disperser au milieu des pays\FTNT{No. 14:22.}.
\VS{28}Ils se joignirent aux adorateurs de Baal-Peor et mangèrent des victimes sacrifiées aux morts.
\VS{29}Ils irritèrent Dieu par leurs actions, au point qu'une plaie fit une brèche parmi eux.
\VS{30}Mais Phinées se présenta et fit justice ; et la plaie fut arrêtée.
\VS{31}Cela lui fut imputé à justice de génération en génération, pour toujours\FTNT{No. 25:3-8.}.
\VS{32}Ils excitèrent aussi sa colère près des eaux de Meriba, et Moïse fut puni à cause d'eux.
\VS{33}Car ils aigrirent son esprit et il parla avec légèreté de ses lèvres\FTNT{No. 20:12.}.
\VS{34}Ils ne détruisirent point les peuples que Yahweh leur avait dit de détruire,
\VS{35}mais ils se mêlèrent parmi ces nations et apprirent leurs manières de faire.
\VS{36}Ils servirent leurs faux dieux qui furent un piège pour eux.
\VS{37}Car ils sacrifièrent leurs fils et leurs filles aux démons\FTNT{Lé. 18:21 ; De. 12:31 ; 2 R. 16:3 ; Ez. 20:26.}.
\VS{38}Ils répandirent le sang innocent, le sang de leurs fils et de leurs filles, ils sacrifièrent aux faux dieux de Canaan ; et le pays fut souillé de sang\FTNT{No. 35:33.}.
\VS{39}Ils se souillèrent par leurs œuvres et se prostituèrent par leurs actions.
\VS{40}C'est pourquoi la colère de Yahweh s'embrasa contre son peuple et il eut en abomination son héritage.
\VS{41}Il les livra entre les mains des nations, et ceux qui les haïssaient dominèrent sur eux.
\VS{42}Leurs ennemis les opprimèrent et ils furent humiliés sous leur main.
\VS{43}Il les délivra souvent, mais ils se montrèrent rebelles dans leurs desseins et furent humiliés par leur iniquité.
\VS{44}Toutefois, il vit leur détresse lorsqu'il entendit leurs supplications.
\VS{45}Il se souvint en leur faveur de son alliance et se repentit selon la grandeur de ses compassions.
\VS{46}Il fit que ceux qui les avaient emmenés captifs eurent pitié d'eux.
\VS{47}Yahweh notre Dieu, délivre-nous et rassemble-nous du milieu des nations ! Afin que nous célébrions ton saint Nom et que nous mettions notre gloire à te louer !
\VS{48}Béni soit Yahweh, le Dieu d'Israël, d'éternité en éternité ! Et que tout le peuple dise amen ! Louez Yahweh.
\Chap{107}
\TextTitle{La grâce de Yahweh pour ses rachetés}
\VerseOne{}Célébrez Yahweh car il est bon, parce que sa bonté demeure à toujours.
\VS{2}Qu'ainsi disent les rachetés de Yahweh, ceux qu'il a rachetés de la main de l'oppresseur,
\VS{3}et qu'il a rassemblés de tous les pays, de l'orient et de l'occident, du nord et de la mer.
\VS{4}Ils erraient dans le désert, ils marchaient dans la solitude, sans trouver une ville où ils puissent habiter.
\VS{5}Ils étaient affamés et assoiffés, leur âme était languissante.
\VS{6}Alors ils crièrent vers Yahweh dans leur détresse et il les délivra de leurs angoisses ;
\VS{7}il les conduisit sur le droit chemin pour aller dans une ville habitée.
\VS{8}Qu'ils célèbrent Yahweh pour sa bonté et ses merveilles envers les fils des hommes !
\VS{9}Parce qu'il a désaltéré l'âme altérée et rassasié de ses biens l'âme affamée\FTNT{Ps. 146:7 ; Lu. 1:53.}.
\VS{10}Ceux qui avaient pour demeure les ténèbres et l'ombre de la mort, vivaient captifs dans l'affliction et dans les chaînes,
\VS{11}parce qu'ils furent rebelles aux paroles de Dieu, et parce qu'ils avaient rejeté le conseil du Très-Haut\FTNT{De. 31:20 ; La. 3:42.}.
\VS{12}Il humilia leur cœur par la souffrance, ils furent abattus ; et personne ne les secourut.
\VS{13}Alors ils crièrent vers Yahweh dans leur détresse, et il les délivra de leurs angoisses.
\VS{14}Il les fit sortir hors des ténèbres et de l'ombre de la mort ; et il rompit leurs liens\FTNT{Ps. 68:19 ; Ep. 4:8 ; Col. 1:12-13.}.
\VS{15}Qu'ils célèbrent Yahweh pour sa bonté et ses merveilles envers les fils des hommes !
\VS{16}Parce qu'il a brisé les portes d'airain et cassé les barreaux de fer.
\VS{17}Les insensés sont affligés à cause de leurs transgressions et à cause de leurs iniquités.
\VS{18}Leur âme avait en horreur toute nourriture, et ils touchaient aux portes de la mort.
\VS{19}Alors ils crièrent vers Yahweh dans leur détresse, et il les délivra de leurs angoisses\FTNT{Ps. 50:15 ; Os. 5:15.}.
\VS{20}Il envoya sa parole et les guérit ; et il les délivra de leurs tombeaux.
\VS{21}Qu'ils célèbrent Yahweh pour sa bonté et ses merveilles envers les fils des hommes !
\VS{22}Qu'ils offrent des sacrifices de remerciements, et qu'ils racontent ses œuvres avec des cris de joie.
\VS{23}Ceux qui descendaient sur la mer dans des navires, faisant commerce sur les grandes eaux,
\VS{24}ceux-là virent les œuvres de Yahweh et ses merveilles dans les lieux profonds,
\VS{25}car il dit, et il fit paraître la tempête qui souleva les vagues de la mer.
\VS{26}Ils montaient vers les cieux, ils descendaient dans l'abîme ; leur âme se fondait d'angoisse.
\VS{27}Saisis de vertiges, ils chancelaient comme un homme ivre ; et toute leur sagesse était anéantie\FTNT{Es. 51:17-21 ; Jé. 13:13.}.
\VS{28}Alors ils crièrent vers Yahweh dans leur détresse, et il les tira hors de leurs angoisses.
\VS{29}Il arrêta la tempête, la changeant en calme, et les ondes se turent.
\VS{30}Puis ils se réjouirent de ce qu'elles s'étaient apaisées, et il les conduisit au port qu'ils désiraient.
\VS{31}Qu'ils célèbrent Yahweh pour sa bonté et ses merveilles envers les fils des hommes !
\VS{32}Et qu'ils l'exaltent dans l'assemblée du peuple et le louent dans l'assemblée des anciens.
\VS{33}Il réduit les fleuves en désert, et les sources d'eaux en sécheresse ;
\VS{34}la terre fertile en terre salée, à cause de la méchanceté de ses habitants\FTNT{Jé. 12:4 ; Jé. 17:6.}.
\VS{35}Il transforme le désert en étangs d'eaux, et la terre sèche en des sources d'eaux\FTNT{Es. 41:18.} ;
\VS{36}il y établit ceux qui sont affamés, ils bâtissent des villes pour l'habiter.
\VS{37}Ils ensemencent des champs et plantent des vignes qui rendent du fruit tous les ans.
\VS{38}Il les bénit et ils se multiplient extrêmement ; et il ne laisse point diminuer leur bétail.
\VS{39}Puis ils sont amoindris et humiliés par l'oppression, le malheur et la souffrance.
\VS{40}Il répand le mépris sur les princes et les fait errer dans des lieux déserts sans chemin.
\VS{41}Mais il relève le pauvre et le délivre de la misère, il établit les familles comme des troupeaux\FTNT{1 S. 2:8 ; Ps. 113:7.}.
\VS{42}Les hommes droits le voient et se réjouissent, mais toute iniquité a la bouche fermée.
\VS{43}Quiconque est sage prendra garde à ces choses, afin qu'on considère les bontés de Yahweh.
\Chap{108}
\TextTitle{Yahweh, le secours}
\VerseOne{}Cantique. Psaume de David. Mon cœur est affermi, ô Dieu ! Je chante et je joue de mes instruments, c'est ma gloire !
\VS{2}Réveillez-vous, mon luth et ma harpe ! Je me réveillerai à l'aube du jour.
\VS{3}Yahweh, je te célébrerai parmi les peuples et je te chanterai parmi les nations.
\VS{4}Car ta bonté est grande par-dessus les cieux, et ta vérité atteint jusqu'aux nues.
\VS{5}Ô Dieu ! Elève-toi sur les cieux, et que ta gloire soit sur toute la terre !
\VS{6}Afin que ceux que tu aimes soient délivrés ; sauve-moi par ta droite et exauce-moi !
\VS{7}Dieu a dit dans sa sainteté : Je me réjouirai, je partagerai Sichem et mesurerai la vallée de Succoth.
\VS{8}Galaad sera à moi, Manassé sera à moi, et Ephraïm sera le sommet de ma forteresse, Juda mon législateur.
\VS{9}Moab sera le bassin où je me laverai, je jetterai mon soulier sur Edom, je triompherai des Philistins. 
\VS{10}Qui me conduira dans la ville forte ? Qui me conduira jusqu'en Edom ?
\VS{11}N'est-ce pas toi, ô Dieu, qui nous avais rejetés, et qui ne sortais plus, ô Dieu, avec nos armées ?
\VS{12}Donne–nous du secours pour sortir de la détresse ! Car la délivrance qu'on attend de l'homme est vaine.
\VS{13}Avec Dieu, nous ferons des exploits ; il foulera nos ennemis\FTNT{Ps. 60:5-14.}.
\Chap{109}
\TextTitle{La méchanceté de l'homme}
\VerseOne{}Psaume de David ; Donné au chef des chantres\FTNT{Les Psaumes d'imprécations (Ps. 35 ; 52 ; 55 ; 58, 59 ; 79 ; 109 ; 137) sont des demandes faites à Dieu pour qu'il punisse les méchants. Le Seigneur Jésus-Christ nous demande aujourd'hui de bénir nos ennemis (Lu. 6:27-37).}. Dieu de ma louange, ne te tais point !
\VS{2}Car la bouche du méchant et la bouche remplie de fraude se sont ouvertes contre moi ; ils parlent contre moi avec une langue mensongère !
\VS{3}Ils m'entourent de paroles pleines de haine et ils me font la guerre sans cause !
\VS{4}Tandis que je les aime, ils sont mes ennemis ; mais moi, je n'ai fait que prier en leur faveur !
\VS{5}Ils me rendent le mal pour le bien, et la haine pour l'amour que je leur porte.
\VS{6}Etablis le méchant sur lui, et que Satan se tienne à sa droite !
\VS{7}Quand il sera jugé, fais qu'il soit déclaré méchant, et que sa prière soit regardée comme un crime !
\VS{8}Que sa vie soit courte\FTNT{Il est question ici du suicide de Judas (Mt. 27:3-5).} et qu'un autre prenne sa charge\FTNT{Ce passage parle de Judas (Ac. 1:20).} !
\VS{9}Que ses enfants soient orphelins et sa femme veuve !
\VS{10}Que ses enfants soient entièrement vagabonds, et qu'ils mendient et quêtent en sortant de leurs maisons détruites\FTNT{Job. 20:10.} !
\VS{11}Que le créancier usant d'exaction attrape tout ce qui est à lui et que les étrangers butinent tout son travail !
\VS{12}Qu'il n'y ait personne qui étende sa compassion sur lui, et qu'il n'y ait personne qui ait pitié de ses orphelins !
\VS{13}Que sa postérité soit exposée à être retranchée ; que leur nom soit effacé dans la génération qui le suivra !
\VS{14}Que l'iniquité de ses pères revienne en mémoire à Yahweh, et que le péché de sa mère ne soit point effacé !
\VS{15}Qu'ils soient continuellement devant Yahweh, et qu'il retranche leur mémoire de la terre\FTNT{Ps. 34:17.},
\VS{16}parce qu'il ne s'est point souvenu d'user de miséricorde, mais il a persécuté l'homme affligé et misérable, dont le cœur est brisé, et cela pour le faire mourir !
\VS{17}Puisqu'il aime la malédiction, que la malédiction tombe sur lui ! Puisqu'il ne prend pas plaisir à la bénédiction, que la bénédiction aussi s'éloigne de lui !
\VS{18}Et qu'il soit revêtu de la malédiction comme de sa robe ; qu'elle entre dans son corps comme de l'eau, et dans ses os comme de l'huile !
\VS{19}Qu'elle lui soit comme un vêtement dont il se couvre, et comme une ceinture dont il se ceigne continuellement !
\VS{20}Telle soit, de la part de Yahweh, la récompense de mes adversaires, et de ceux qui parlent mal de moi !
\VS{21}Mais toi, Yahweh, Seigneur, agis avec moi pour l'amour de ton Nom ! Et parce que ta miséricorde est grande, délivre-moi !
\VS{22}Car je suis affligé et misérable, et mon cœur est blessé au-dedans de moi.
\VS{23}Je m'en vais comme l'ombre quand elle décline, et je suis chassé comme une sauterelle.
\VS{24}Mes genoux sont affaiblis par le jeûne, et mon corps est épuisé de maigreur au lieu d'être gras.
\VS{25}Je suis pour eux un objet d'opprobre ; quand ils me voient, ils secouent la tête.
\VS{26}Yahweh, mon Dieu ! Aide-moi, délivre-moi selon ta miséricorde.
\VS{27}Afin qu'on sache que c'est ta main, que c'est toi, ô Yahweh, qui l'as fait.
\VS{28}Ils maudiront, mais tu béniras ; ils s'élèveront, mais ils seront confus ; et ton serviteur se réjouira.
\VS{29}Que mes adversaires soient revêtus de confusion et couverts de leur honte comme d'un manteau.
\VS{30}Je célébrerai hautement de ma bouche Yahweh, et je le louerai au milieu de plusieurs nations.
\VS{31}De ce qu'il se tient à la droite du misérable pour le délivrer de ceux qui condamnent son âme.
\Chap{110}
\TextTitle{Yahweh, le Roi et le Sacrificateur}
\VerseOne{}Psaume de David. Yahweh a dit à mon Seigneur : Assieds-toi à ma droite, jusqu'à ce que je fasse de tes ennemis le marchepied de tes pieds\FTNT{Ce psaume affirme la divinité de Jésus-Christ (Mt. 22:41-46 ; Mc. 12:35-37 ; Lu. 20:41-44 ; Ac. 2:34-35 ; Hé. 1:13 ; Hé. 10:12-13).}.
\VS{2}Yahweh étendra de Sion le sceptre de ta puissance, en disant : Domine au milieu de tes ennemis\FTNT{Es. 2:2-3 ; Da. 7:14.} !
\VS{3}Ton peuple est plein d'ardeur quand tu rassembles ton armée ; avec des ornements sacrés, du sein de l'aurore, ta jeunesse vient à toi comme une rosée.
\VS{4}Yahweh l'a juré, et il ne s'en repentira point que tu es sacrificateur éternellement, à la manière de Melchisédek\FTNT{Hé. 5:6 ; Hé. 6:20 ; Hé. 7:17 ; Ge. 14:18.}.
\VS{5}Le Seigneur est à ta droite, il brisera les rois au jour de sa colère.
\VS{6}Il exercera le jugement sur les nations, il remplira tout de cadavres ; il brisera le chef qui domine sur un grand pays\FTNT{Ap. 14 ; Ap. 16.}.
\VS{7}Il boit au torrent pendant la marche : C'est pourquoi il lève haut la tête.
\Chap{111}
\TextTitle{Les oeuvres magnifiques de Dieu}
\VerseOne{}Louez Yahweh. [Aleph.] Je célébrerai Yahweh de tout mon cœur, [Beth.] dans la compagnie des hommes droits et dans l'assemblée.
\VS{2}[Guimel.] Les œuvres de Yahweh sont grandes, [Daleth.] elles sont recherchées par tous ceux qui y prennent plaisir.
\VS{3}[He.] Son œuvre n'est que majesté et magnificence, [Vav.] et sa justice demeure à perpétuité.
\VS{4}[Zayin.] Il a rendu ses merveilles mémorables. [Heth.] Yahweh est miséricordieux et compatissant.
\VS{5}[Teth.] Il a donné de la nourriture à ceux qui le craignent ; [Yod.] il s'est souvenu pour toujours de son alliance.
\VS{6}[Kaf.] Il a manifesté à son peuple la puissance de ses œuvres, [Lamed.] en leur donnant l'héritage des nations.
\VS{7}[Mem.] Les œuvres de ses mains ne sont que vérité et équité. [Nun.] Tous ses commandements sont véritables,
\VS{8}[Samech.] appuyés à perpétuité, éternellement, [Ayin.] faits avec fidélité et droiture.
\VS{9}[Pe.] Il a envoyé la rédemption à son peuple\FTNT{Ex. 6:6 ; Jn. 3:16.} ; [Tsade.] il lui a donné une alliance éternelle ; [Qof.] son nom est saint et redoutable.
\VS{10}[Resh.] Le commencement de la sagesse c'est la crainte de Yahweh : [Shin.] Tous ceux qui s'adonnent à faire ce qu'elle prescrit sont sages\FTNT{Pr. 1:7 ; Pr. 9:10 ; Pr. 8:13 ; De. 4:6.}. [Tav.] Sa louange demeure à perpétuité.
\Chap{112}
\TextTitle{La crainte de Yahweh enrichit et donne de l'assurance}
\VerseOne{}Louez Yahweh. [Aleph.] Heureux l'homme qui craint Yahweh [Beth.] et qui prend un grand plaisir à ses commandements !
\VS{2}[Guimel.] Sa postérité sera puissante sur la terre, [Daleth.] la génération des hommes droits sera bénie\FTNT{Pr. 20:7.}.
\VS{3}[He.] Il y aura des biens et des richesses dans sa maison ; [Vav.] et sa justice demeure à perpétuité.
\VS{4}[Zayin.] La lumière s'est levée dans les ténèbres sur ceux qui sont justes\FTNT{Pr. 4:18 ; Ps. 37:6.} ; [Heth.] il est compatissant, miséricordieux et juste.
\VS{5}[Teth.] Heureux l'homme de bien qui exerce la miséricorde et prête, [Yod.] qui règle ses actions avec justice.
\VS{6}[Kaf.] Il ne chancelle jamais. [Lamed.] La mémoire du juste dure toujours\FTNT{Pr. 10:7.}.
\VS{7}[Mem.] Il ne craint pas les mauvaises nouvelles ; [Nun.] son cœur est ferme, confiant en Yahweh.
\VS{8}[Samech.] Son cœur est bien affermi, il ne craint pas, [Ayin.] jusqu'à ce qu'il mette son plaisir à regarder ses adversaires.
\VS{9}[Pe.] Il fait des largesses, il donne aux pauvres ; [Tsade.] sa justice demeure à perpétuité ; [Qof.] sa corne s'élève en gloire.
\VS{10}[Resh.] Le méchant le voit et s'irrite. [Shin.] Il grince des dents et se consume ; [Tav.] les désirs des méchants périssent.
\Chap{113}
\TextTitle{Yahweh, le Dieu élevé au-dessus de tout}
\VerseOne{}Louez Yahweh\FTNT{Les psaumes disant « Alléluia » sont les Ps. 104 à 106, 111 à 113, 115 à 117, 135 à 136, 146 à 150. Parmi eux, les psaumes 135 et 146 à 150, étaient chantés durant le service quotidien d'adoration dans la synagogue. Les psaumes 115 à 118, appelés « le grand Hallel », étaient chantés lors des fêtes de Pâque. Alléluia veut dire « Louez Yahweh » (Ap. 19:1).} ! Louez, vous serviteurs de Yahweh, louez le Nom de Yahweh !
\VS{2}Que le Nom de Yahweh soit béni dès maintenant et à toujours !
\VS{3}Le Nom de Yahweh est digne de louanges depuis le soleil levant jusqu'au soleil couchant.
\VS{4}Yahweh est élevé par-dessus toutes les nations, sa gloire est au-dessus des cieux.
\VS{5}Qui est semblable à Yahweh notre Dieu, qui habite dans les lieux très hauts ?
\VS{6}Il s'abaisse pour regarder sur le ciel et sur la terre,
\VS{7}Il relève l'affligé de la poussière, et retire le pauvre\FTNT{1 S. 2:8 ; Ps. 107:41.} de dessus le fumier,
\VS{8}pour le faire asseoir avec les nobles, avec les nobles de son peuple\FTNT{Job. 36:7.}.
\VS{9}Il donne une maison à la femme stérile, il en fait une mère joyeuse au milieu de ses fils\FTNT{Ge. 17:17-21 ; 1 S. 2:5 ; Ps. 68:6.}. Louez Yahweh !
\Chap{114}
\TextTitle{La création tremble devant le Tout-Puissant}
\VerseOne{}Quand Israël sortit d'Egypte, quand la maison de Jacob s'éloigna d'un peuple barbare,
\VS{2}Juda devint son lieu saint, Israël son domaine\FTNT{Jé. 2:2-3.}.
\VS{3}La mer le vit et s'enfuit, le Jourdain retourna en arrière\FTNT{Jos. 3:13-16 ; Ps. 77:17.}.
\VS{4}Les montagnes sautèrent comme des béliers, les collines comme des agneaux\FTNT{Jg. 5:5 ; Ha. 3:10 ; Ps. 68:9.}.
\VS{5}Ô Mer ! Qu'avais-tu pour t'enfuir ? Jourdain, pour retourner en arrière ?
\VS{6}Et vous montagnes, pour sauter comme des béliers ? Et vous collines, comme des agneaux ?
\VS{7}Ô Terre ! Tremble devant la présence du Seigneur, devant la présence du Dieu de Jacob,
\VS{8}qui a changé le rocher en un étang d'eaux, la pierre très dure en une source d'eaux.
\Chap{115}
\TextTitle{Louange au Dieu de gloire}
\VerseOne{}Non point à nous, ô Yahweh ! Non point à nous, mais à ton Nom donne gloire, à cause de ta bonté, à cause de ta fidélité !
\VS{2}Pourquoi les nations diraient-elles : Où est maintenant leur Dieu ?
\VS{3}Certes notre Dieu est au ciel, il fait tout ce qu'il veut\FTNT{Ps. 135:6 ; Job. 23:13.}.
\VS{4}Leurs idoles sont des dieux d'or et d'argent, elles sont l'ouvrage de mains d'homme.
\VS{5}Elles ont une bouche, et ne parlent point ; elles ont des yeux, et ne voient point ;
\VS{6}elles ont des oreilles, et n'entendent point ; elles ont un nez, et ne sentent point ;
\VS{7}elles ont des mains, et elles ne touchent point ; elles ont des pieds, et elles ne marchent point ; et elles ne rendent aucun son de leur gosier\FTNT{Ex. 32:2-8 ; 1 R. 18:25-26 ; Es. 44:9 ; Ez. 8:8-12.}.
\VS{8}Ils leur ressemblent, ceux qui les fabriquent, tous ceux qui se confient en eux.
\VS{9}Israël confie-toi en Yahweh ; il est leur secours et leur bouclier de ceux qui se confient en lui.
\VS{10}Maison d'Aaron, confie-toi en Yahweh ; il est leur secours et leur bouclier.
\VS{11}Vous qui craignez Yahweh, confiez-vous en Yahweh ; il est leur secours et leur bouclier.
\VS{12}Yahweh s'est souvenu de nous, il bénira, il bénira la maison d'Israël, il bénira la maison d'Aaron.
\VS{13}Il bénira ceux qui craignent Yahweh, tant les petits que les grands.
\VS{14}Yahweh vous multipliera ses bénédictions, à vous et à vos fils.
\VS{15}Vous êtes bénis de Yahweh, qui a fait les cieux et la terre.
\VS{16}Les cieux, sont les cieux de Yahweh, mais il a donné la terre aux fils des hommes.
\VS{17}Ce ne sont pas les morts qui célèbrent Yahweh, ce n'est aucun de ceux qui descendent dans le lieu du silence\FTNT{Ps. 88:11 ; Es. 38:18-19 ; Ps. 6:6.}.
\VS{18}Mais nous, nous bénirons Yahweh dès maintenant et pour toujours. Louez Yahweh !
\Chap{116}
\TextTitle{Psaume des rachetés}
\VerseOne{}J'aime Yahweh, car il a entendu ma voix et mes supplications.
\VS{2}Car il a incliné son oreille vers moi, c'est pourquoi je l'invoquerai durant mes jours.
\VS{3}Les liens de la mort m'avaient environné, et les angoisses du scheol m'avaient trouvé\FTNT{2 S. 22:5 ; Ps. 18:5.} ; j'avais trouvé la détresse et la douleur.
\VS{4}Mais j'invoquai le Nom de Yahweh en disant : Je te prie, délivre mon âme, ô Yahweh !
\VS{5}Yahweh est compatissant et juste, et notre Dieu fait miséricorde.
\VS{6}Yahweh garde les simples ; j'étais devenu misérable et il m'a sauvé.
\VS{7}Mon âme, retourne dans ton repos, car Yahweh t'a fait du bien.
\VS{8}Parce que tu as retiré mon âme de la mort, mes yeux des larmes et mes pieds de la chute,
\VS{9}je marcherai dans la présence de Yahweh, sur la terre des vivants.
\VS{10}J'ai cru, c'est pourquoi j'ai parlé\FTNT{2 Co. 4:13.} ; j'ai été fort affligé.
\VS{11}Je disais dans ma précipitation : Tout homme est menteur\FTNT{Ro. 3:4.}.
\VS{12}Que rendrai-je à Yahweh ? Tous ses bienfaits envers moi ?
\VS{13}J'élèverai la coupe des délivrances et j'invoquerai le Nom de Yahweh.
\VS{14}J'accomplirai maintenant mes vœux à Yahweh, devant tout son peuple.
\VS{15}Elle a du prix aux yeux de Yahweh, la mort de ceux qu'il aime.
\VS{16}Ecoute-moi, ô Yahweh ! Car je suis ton serviteur, je suis ton serviteur, fils de ta servante. Tu as délié mes liens.
\VS{17}Je t'offrirai le sacrifice de remerciement et j'invoquerai le Nom de Yahweh.
\VS{18}J'accomplirai maintenant mes vœux à Yahweh, devant tout son peuple,
\VS{19}dans les parvis de la maison de Yahweh, au milieu de toi, Jérusalem ! Louez Yahweh !
\Chap{117}
\TextTitle{Toutes les nations louent Yahweh}
\VerseOne{}Toutes les nations, louez Yahweh ! Tous les peuples, célébrez-le !
\VS{2}Car sa miséricorde est grande envers nous, et sa fidélité dure à toujours. Louez Yahweh !
\Chap{118}
\TextTitle{Yahweh, le Dieu de mon secours}
\VerseOne{}Célébrez Yahweh, car il est bon, parce que sa bonté dure à toujours !
\VS{2}Qu'Israël dise maintenant : Car sa bonté dure à toujours !
\VS{3}Que la maison d'Aaron dise maintenant : Car sa bonté dure à toujours !
\VS{4}Que ceux qui craignent Yahweh disent maintenant : Car sa bonté dure à toujours !
\VS{5}Me trouvant dans la détresse, j'ai invoqué Yahweh\FTNT{Ps. 120:1.} ; et Yahweh m'a répondu et m'a mis au large.
\VS{6}Yahweh est pour moi, je ne craindrai point. Que me ferait l'homme ?
\VS{7}Yahweh est pour moi parmi ceux qui me secourent, c'est pourquoi je verrai en ceux qui me haïssent ce que je désire.
\VS{8}Mieux vaut se confier en Yahweh que se confier en l'homme\FTNT{Es. 2:22 ; Jé. 17:5 ; Ps. 62:9.}.
\VS{9}Mieux vaut se confier en Yahweh que se reposer sur les grands d'entre les peuples.
\VS{10}Toutes les nations m'avaient environné, mais au Nom de Yahweh je les taille en pièces.
\VS{11}Elles m'avaient environné, elles m'avaient, dis-je, environné ; mais au Nom de Yahweh je les taille en pièces.
\VS{12}Elles m'avaient environné comme des abeilles, elles s'éteignent comme un feu d'épines\FTNT{De. 1:44.}, car au Nom de Yahweh je les taille en pièces.
\VS{13}Tu me poussais violemment pour me faire tomber, mais Yahweh m'a secouru.
\VS{14}Yahweh est ma force et le sujet de mes louanges, et il a été ma délivrance\FTNT{Ex. 15:2 ; Es. 12:2.}.
\VS{15}Une voix de chant de triomphe et de délivrance retentit dans les tentes des justes : La droite de Yahweh exerce la puissance !
\VS{16}La droite de Yahweh est élevée, la droite de Yahweh exerce sa puissance.
\VS{17}Je ne mourrai pas, je vivrai et je raconterai les œuvres de Yahweh.
\VS{18}Yahweh m'a châtié sévèrement, mais il ne m'a point livré à la mort.
\VS{19}Ouvrez-moi les portes de la justice ; j'y entrerai et je célébrerai Yahweh.
\VS{20}C'est ici la porte de Yahweh, les justes y entreront.
\VS{21}Je te célébrerai parce que tu m'as exaucé et tu as été mon libérateur.
\VS{22}La Pierre que les architectes avaient rejetée, est devenue la principale de l'angle\FTNT{Le Messie est présenté comme la pierre ou le rocher. (cp. Es. 8:13-17 ; 1 Pi. 2:7).}.
\VS{23}Ceci a été fait par Yahweh, c'est un prodige à nos yeux.
\VS{24}C'est ici la journée que Yahweh a faite, qu'elle soit pour nous un sujet d'allégresse et de joie.
\VS{25}Yahweh, je te prie, délivre maintenant. Yahweh, je te prie, donne maintenant la prospérité !
\VS{26}Béni soit celui qui vient au Nom de Yahweh ! Nous vous bénissons de la maison de Yahweh.
\VS{27}Yahweh est Dieu, et il nous a éclairés. Liez avec des cordes la bête du sacrifice, et amenez-la jusqu'aux cornes de l'autel.
\VS{28}Tu es mon Dieu, c'est pourquoi je te célébrerai. Tu es mon Dieu, je t'exalterai.
\VS{29}Célébrez Yahweh car il est bon, parce que sa miséricorde demeure à toujours !
\Chap{119}
\TextTitle{La Parole de Yahweh éclaire}
\VerseOne{}[Aleph.] Heureux ceux qui sont intègres dans leur voie, qui marchent selon la loi de Yahweh.
\VS{2}Heureux sont ceux qui gardent ses préceptes et qui le cherchent de tout leur cœur\FTNT{Jos. 1:8.} ;
\VS{3}qui ne font point d'iniquité, qui marchent dans ses voies\FTNT{1 Jn. 3:9 ; 1 Jn. 5:18.}.
\VS{4}Tu as donné tes commandements afin qu'on les garde soigneusement.
\VS{5}Oh ! Que mes voies soient bien établies pour garder tes statuts !
\VS{6}Et je ne rougirai point de honte quand je regarderai à tous tes commandements.
\VS{7}Je te célébrerai avec droiture de cœur quand j'aurai appris les ordonnances de ta justice.
\VS{8}Je veux garder tes statuts, ne me délaisse point entièrement.
\VS{9}[Beth.] Par quel moyen le jeune homme rendra-t-il pure sa voie ? Ce sera en y prenant garde selon ta parole.
\VS{10}Je te recherche de tout mon cœur, ne me laisse pas m'égarer loin de tes commandements.
\VS{11}Je serre ta parole dans mon cœur afin de ne pas pécher contre toi.
\VS{12}Yahweh ! Tu es béni ; enseigne-moi tes statuts.
\VS{13}De mes lèvres je raconte toutes les ordonnances de ta bouche.
\VS{14}Je me réjouis dans le chemin de tes préceptes comme si je possédais toutes les richesses du monde.
\VS{15}Je médite tes commandements et j'observe tes voies.
\VS{16}Je prends plaisir à tes statuts et je n'oublie pas tes paroles.
\VS{17}[Guimel.] Fais du bien à ton serviteur afin que je vive, et je garderai ta parole\FTNT{Ps. 116:7.}.
\VS{18}Ouvre mes yeux afin que je regarde aux merveilles de ta loi\FTNT{Ep. 1:18.} !
\VS{19}Je suis voyageur sur la terre, ne me cache pas tes commandements.
\VS{20}Mon âme est brisée par le désir qui toujours me porte vers tes ordonnances.
\VS{21}Tu réprimandes les orgueilleux, ces maudits, qui se détournent de tes commandements.
\VS{22}Décharge-moi de l'opprobre et du mépris, car j'ai gardé tes préceptes\FTNT{Ps. 3:9.}.
\VS{23}Même les princes s'assoient et parlent contre moi pendant que ton serviteur médite tes statuts.
\VS{24}Tes préceptes font mes délices, ce sont mes conseillers.
\VS{25}[Daleth.] Mon âme est attachée à la poussière, fais-moi revivre selon ta parole\FTNT{Ps. 44:26 ; Ps. 143:11.}.
\VS{26}Je te raconte mes voies et tu me réponds ; enseigne-moi tes statuts.
\VS{27}Fais-moi entendre la voie de tes commandements, et je parlerai de tes merveilles\FTNT{Ps. 145:6.}.
\VS{28}Mon âme pleure de chagrin, relève-moi selon tes paroles.
\VS{29}Eloigne de moi la voie du mensonge et accorde–moi la grâce d'observer ta loi\FTNT{Le mot « loi » vient de l'hebreu « towrah » qui donne « Thora » en français}.
\VS{30}Je choisis la voie de la vérité et je place tes ordonnances sous mes yeux.
\VS{31}Je m'attache à tes préceptes, ô Yahweh ! Ne me fais point rougir de honte.
\VS{32}Je cours dans la voie de tes commandements car tu élargis mon cœur.
\VS{33}[He.] Yahweh, enseigne-moi la voie de tes statuts, et je la garderai jusqu'au bout.
\VS{34}Donne-moi de l'intelligence ; je garderai ta loi et je l'observerai de tout mon cœur\FTNT{Pr. 2:6 ; Ja. 1:5.}.
\VS{35}Fais-moi marcher sur le sentier de tes commandements car j'y prends plaisir.
\VS{36}Incline mon cœur à tes préceptes et non point au profit\FTNT{Ez. 33:31 ; Mc 7:21-22 ; Hé. 13:5.}.
\VS{37}Détourne mes yeux de la vue des choses vaines ; fais-moi vivre dans ta voie.
\VS{38}Accomplis ta parole envers ton serviteur, parole qui est pour ceux qui te craignent.
\VS{39}Eloigne de moi l'opprobre que je redoute, car tes ordonnances sont bonnes.
\VS{40}Voici, je désire pratiquer tes commandements, fais-moi vivre dans ta justice.
\VS{41}[Vav.] Que ta miséricorde vienne sur moi, ô Yahweh ! Et ta délivrance aussi, selon ta promesse !
\VS{42}Et je pourrai répondre à celui qui m'outrage, car je me confie en ta parole.
\VS{43}N'arrache pas de ma bouche la parole de vérité, car j'espère en tes jugements.
\VS{44}Je garderai continuellement ta loi, à toujours et à perpétuité.
\VS{45}Je marcherai au large parce que je recherche tes commandements.
\VS{46}Je parlerai de tes préceptes devant les rois et je ne rougirai pas de honte\FTNT{Ps. 138:1-4 ; Mt. 10:18-19 ; Ac. 26.}.
\VS{47}Je fais mes délices de tes commandements que j'aime ;
\VS{48}j'étends mes mains vers tes commandements que j'aime ; et je médite tes statuts.
\VS{49}[Zayin.] Souviens-toi de la parole donnée à ton serviteur, sur laquelle tu m'as fait espérer.
\VS{50}C'est ici ma consolation dans mon affliction, car ta parole me rend la vie.
\VS{51}Les orgueilleux se sont fort moqués de moi, mais je ne me suis pas détourné de ta loi.
\VS{52}Yahweh, je me souviens de tes jugements anciens et je me suis consolé en eux.
\VS{53}L'horreur me saisit à cause des méchants qui abandonnent ta loi.
\VS{54}Tes statuts sont le sujet de mes cantiques dans la maison où je suis étranger.
\VS{55}Yahweh, je me souviens de ton Nom pendant la nuit et je garde ta loi.
\VS{56}Cela m'arrive parce que je garde tes commandements.
\VS{57}[Heth.] Ô Yahweh ! J'en conclus que ma part est de garder tes paroles.
\VS{58}Je te supplie de tout mon cœur : Aie pitié de moi selon ta parole.
\VS{59}Je fais le compte de mes voies et je rebrousse chemin vers tes préceptes\FTNT{Os. 6:3 ; La. 3:40.}.
\VS{60}Je me hâte, je ne diffère point de garder tes commandements.
\VS{61}Une compagnie de méchants me pille, mais je n'oublie pas ta loi.
\VS{62}Je me lève au milieu de la nuit pour te célébrer à cause des ordonnances de ta justice.
\VS{63}Je suis l'ami de tous ceux qui te craignent et qui gardent tes commandements.
\VS{64}Yahweh, la terre est pleine de ta bonté ; enseigne-moi tes statuts.
\VS{65}[Teth.] Yahweh, tu fais du bien à ton serviteur selon ta parole.
\VS{66}Enseigne-moi le bon sens et la connaissance car je crois à tes commandements.
\VS{67}Avant d'avoir été humilié, je m'égarais, mais maintenant j'observe ta parole.
\VS{68}Tu es bon et bienfaisant, enseigne-moi tes statuts.
\VS{69}Les orgueilleux imaginent des faussetés contre moi, mais je garde de tout mon cœur tes commandements.
\VS{70}Leur cœur est insensible comme la graisse, mais moi, je prends plaisir dans ta loi\FTNT{De. 32:15 ; Jé. 5:28.}.
\VS{71}Il est bon que je sois humilié afin que j'apprenne tes statuts.
\VS{72}La loi que tu as prononcée de ta bouche m'est plus précieuse que mille pièces d'or ou d'argent\FTNT{Ps. 19:10-11 ; Job. 22:2.}.
\VS{73}[Yod.] Tes mains m'ont façonné, elles m'ont formé\FTNT{Jé. 1:5 ; Job. 10:9.} ; donne-moi l'intelligence afin que j'apprenne tes commandements.
\VS{74}Ceux qui te craignent me verront et se réjouiront, parce que j'espère en tes promesses.
\VS{75}Je reconnais, ô Yahweh, que tes jugements sont justes, et que tu m'as humilié par ta fidélité\FTNT{Hé. 12:10.}.
\VS{76}Que ta bonté soit ma consolation, comme tu l'as promis à ton serviteur.
\VS{77}Que tes compassions viennent sur moi et je vivrai ; car ta loi fait mes délices.
\VS{78}Que les orgueilleux rougissent de honte, de ce qu'ils m'oppriment sans cause ; mais moi, je médite sur tes ordonnances.
\VS{79}Que ceux qui te craignent et ceux qui connaissent tes préceptes reviennent vers moi.
\VS{80}Que mon cœur soit intègre dans tes statuts afin que je ne sois pas couvert de honte.
\VS{81}[Kaf.] Mon âme se consume en attendant ta délivrance ; j'espère en ta promesse.
\VS{82}Mes yeux s'épuisent en attendant ta promesse, lorsque je dis : Quand me consoleras-tu ?
\VS{83}Car je suis comme une outre dans la fumée, je n'oublie pas tes statuts.
\VS{84}Quel est le nombre de jours de ton serviteur ? Quand jugeras-tu ceux qui me poursuivent\FTNT{Ap. 6:10.} ?
\VS{85}Les orgueilleux me creusent des fosses, ils n'agissent pas selon ta loi.
\VS{86}Tous tes commandements ne sont que fidélité ; on me persécute sans cause, aide-moi\FTNT{Mt. 5:10.} !
\VS{87}On m'a presque réduit à rien et mis par terre ; mais je n'ai point abandonné tes commandements.
\VS{88}Fais-moi revivre selon ta miséricorde et je garderai les préceptes de ta bouche.
\VS{89}[Lamed.] Ô Yahweh ! Ta parole subsiste à toujours dans les cieux.
\VS{90}Ta fidélité dure d'âge en âge ; tu as établi la terre, et elle demeure ferme\FTNT{Pr. 1:4.}.
\VS{91}Ces choses subsistent aujourd'hui selon tes ordonnances, car toutes choses te servent.
\VS{92}Si ta loi n'avait pas fait mes délices, j'aurais déjà péri dans mon affliction.
\VS{93}Je n'oublierai jamais tes commandements car c'est par eux que tu m'as fait revivre.
\VS{94}Je suis à toi, sauve-moi ; car je recherche tes commandements.
\VS{95}Les méchants m'attendent pour me faire périr, mais je suis attentif à tes préceptes.
\VS{96}Je vois des bornes à tout ce qui est parfait, mais tes commandements n'ont point de limites.
\VS{97}[Mem.] Combien j'aime ta loi\FTNT{Ps. 1:2.} ! Elle est tout le jour l'objet de ma méditation.
\VS{98}Par tes commandements, tu m'as rendu plus sage que mes ennemis, parce que tes commandements sont toujours avec moi.
\VS{99}J'ai surpassé en prudence tous ceux qui m'avaient enseigné parce que tes préceptes sont l'objet de ma méditation.
\VS{100}Je suis devenu plus intelligent que les vieillards parce que j'observe tes commandements.
\VS{101}Je garde mes pieds de toute mauvaise voie afin d'observer ta parole.
\VS{102}Je ne me suis point détourné de tes ordonnances parce que tu me les enseignes.
\VS{103}Que ta parole est douce à mon palais ! Plus douce que le miel à ma bouche.
\VS{104}Je suis devenu intelligent par tes commandements, c'est pourquoi je hais toute voie de mensonge.
\VS{105}[Nun.] Ta parole est une lampe à mes pieds et une lumière sur mon sentier\FTNT{Pr. 6:23 ; 2 Pi. 1:19.}.
\VS{106}J'ai juré et je le tiendrai, d'observer les lois de ta justice\FTNT{Né. 10:29.}.
\VS{107}Yahweh, je suis extrêmement affligé, fais-moi revivre selon ta parole.
\VS{108}Yahweh, je te prie, agrée les sentiments que ma bouche exprime, et enseigne-moi tes ordonnances\FTNT{Os. 14:2 ; Hé. 13:15.}.
\VS{109}Ma vie est continuellement en danger, toutefois je n'oublie pas ta loi.
\VS{110}Les méchants m'ont tendu des pièges, toutefois je ne me suis point égaré de tes commandements.
\VS{111}J'ai pris pour héritage perpétuel tes préceptes car ils sont la joie de mon cœur.
\VS{112}J'ai incliné mon cœur à accomplir toujours tes statuts jusqu'au bout.
\VS{113}[Samech.] Je hais les hommes indécis\FTNT{1 R. 18:21 ; Ja. 1:6 ; Ja. 4:8.}, mais j'aime ta loi.
\VS{114}Tu es mon refuge et mon bouclier, je m'attends à ta parole.
\VS{115}Méchants, retirez-vous de moi\FTNT{Mt. 7:23 ; Ps. 6:9.} ! Et je garderai les commandements de mon Dieu.
\VS{116}Soutiens-moi suivant ta parole, et je vivrai ; et ne me fais point rougir de honte en me refusant ce que j'espérais.
\VS{117}Soutiens-moi, et je serai en sûreté ; et j'aurai continuellement les yeux sur tes statuts.
\VS{118}Tu as foulé aux pieds tous ceux qui se détournent de tes statuts, car le mensonge est le moyen dont ils se servent pour tromper.
\VS{119}Tu réduis à néant tous les méchants de la terre, comme de l'écume ; c'est pourquoi j'aime tes préceptes.
\VS{120}Ma chair frissonne de l'effroi que tu m'inspires et je crains tes jugements\FTNT{Ha. 3:16.}.
\VS{121}[Ayin.] J'ai exercé le jugement et la justice, ne m'abandonne pas à ceux qui me font tort.
\VS{122}Prends sous ta garantie le bien de ton serviteur et ne permets pas que je sois opprimé par les orgueilleux.
\VS{123}Mes yeux s'épuisent en attendant ta délivrance et la parole de ta justice.
\VS{124}Agis envers ton serviteur suivant ta miséricorde et enseigne-moi tes statuts.
\VS{125}Je suis ton serviteur, donne-moi l'intelligence, et je connaîtrai tes préceptes\FTNT{Pr. 1:4 ; Pr. 6:23.}.
\VS{126}Il est temps que Yahweh opère ; ils ont aboli ta loi.
\VS{127}C'est pourquoi j'aime tes commandements, plus que l'or et l'or fin.
\VS{128}C'est pourquoi je trouve justes tous tes commandements, je hais toute voie de mensonge.
\VS{129}[Pe.] Tes préceptes sont merveilleux, c'est pourquoi mon âme les garde.
\VS{130}La révélation de tes paroles éclaire, elle donne de l'intelligence aux simples.
\VS{131}J'ouvre ma bouche et je soupire, car je désire tes commandements.
\VS{132}Regarde-moi, et aie pitié de moi, selon tes jugements à l'égard de ceux qui aiment ton Nom.
\VS{133}Affermis mes pas sur ta parole, et que l'iniquité n'ait point d'emprise sur moi.
\VS{134}Délivre-moi de l'oppression des hommes afin que je garde tes commandements.
\VS{135}Fais luire ta face sur ton serviteur et enseigne-moi tes statuts.
\VS{136}Mes yeux répandent des torrents d'eau parce qu'on n'observe point ta loi.
\VS{137}[Tsade.] Tu es juste, ô Yahweh, et droit dans tes jugements.
\VS{138}Tu ordonnes tes préceptes avec justice et grande fidélité.
\VS{139}Mon zèle me consume parce que mes adversaires oublient tes paroles.
\VS{140}Ta parole est entièrement éprouvée, c'est pourquoi ton serviteur l'aime.
\VS{141}Je suis petit et méprisé, toutefois je n'oublie point tes commandements.
\VS{142}Ta justice est une justice éternelle, et ta loi est la vérité.
\VS{143}La détresse et l'angoisse m'atteignent, mais tes commandements font mes délices.
\VS{144}Tes préceptes ne sont que justice éternelle ; donne-moi l'intelligence afin que je vive.
\VS{145}[Qof.] Je crie de tout mon cœur, réponds-moi, ô Yahweh ! Je garde tes statuts.
\VS{146}Je crie vers toi, sauve-moi afin que j'observe tes préceptes.
\VS{147}Je devance l'aurore et je crie ; je m'attends à ta parole.
\VS{148}Mes yeux ont devancé les veilles de la nuit pour méditer ta parole.
\VS{149}Ecoute ma voix selon ta miséricorde, ô Yahweh ! Fais-moi revivre selon ton ordonnance.
\VS{150}Ceux qui poursuivent le crime s'approchent de moi, et ils s'éloignent de ta loi.
\VS{151}Yahweh, tu es près de moi ; et tous tes commandements ne sont que vérité.
\VS{152}Depuis longtemps, je sais par tes préceptes, que tu les as établis pour toujours.
\VS{153}[Resh.] Regarde mon affliction et sauve-moi, car je n'oublie pas ta loi.
\VS{154}Soutiens ma cause et rachète-moi ; fais-moi revivre selon ta parole.
\VS{155}La délivrance est loin des méchants parce qu'ils ne recherchent pas tes statuts.
\VS{156}Tes compassions sont en grand nombre, ô Yahweh ! Fais-moi revivre selon tes ordonnances.
\VS{157}Ceux qui me persécutent et qui me pressent sont en grand nombre, toutefois je ne me détourne pas de tes préceptes.
\VS{158}Je vois avec dégoût les traîtres et je suis rempli de tristesse car ils n'observent pas ta parole.
\VS{159}Regarde combien j'aime tes commandements, Yahweh ! Fais-moi revivre selon ta miséricorde !
\VS{160}Le fondement de ta parole est la vérité, et toutes les lois de ta justice sont éternelles.
\VS{161}[Shin.] Les princes du peuple me persécutent sans cause, mais mon cœur tremble à cause de ta parole.
\VS{162}Je me réjouis de ta parole comme ferait celui qui aurait trouvé un grand butin.
\VS{163}J'ai en haine et en abomination le mensonge ; j'aime ta loi.
\VS{164}Sept fois le jour je te loue à cause des ordonnances de ta justice.
\VS{165}Il y a une grande paix pour ceux qui aiment ta loi, et rien ne peut les renverser\FTNT{Es. 32:17 ; Ph. 4:7.}.
\VS{166}Yahweh, j'espère en ta délivrance et je pratique tes commandements.
\VS{167}Mon âme observe tes préceptes, je les aime beaucoup.
\VS{168}J'observe tes commandements et tes préceptes, car toutes mes voies sont devant toi.
\VS{169}[Tav.] Yahweh, que mon cri parvienne jusqu'à toi, donne-moi l'intelligence selon ta parole.
\VS{170}Que ma supplication vienne devant toi, délivre-moi selon ta parole.
\VS{171}Mes lèvres publieront ta louange quand tu m'auras enseigné tes statuts.
\VS{172}Ma langue ne s'entretiendra que de ta parole, parce que tous tes commandements ne sont que justice.
\VS{173}Que ta main me soit en aide, parce que j'ai choisi tes commandements.
\VS{174}Yahweh, je souhaite ta délivrance, et ta loi fait mes délices.
\VS{175}Que mon âme vive afin qu'elle te loue, et que tes ordonnances me soient en aide !
\VS{176}Je suis errant comme une brebis perdue\FTNT{Es. 53:6 ; Lu. 15:4 ; 1 Pi. 2:25.} ; cherche ton serviteur, car je n'oublie pas tes commandements.
\Chap{120}
\TextTitle{Cri de détresse}
\VerseOne{}Cantique des degrés\FTNT{Les psaumes 120 à 134 sont appelés « psaumes des degrés » ou de « l'ascension ». Ces psaumes furent chantés par les Israélites montant à Jérusalem au retour de la captivité de Babylone.}. J'ai invoqué Yahweh dans ma grande détresse, et il m'a exaucé.
\VS{2}Yahweh, délivre mon âme des lèvres mensongères et de la langue trompeuse.
\VS{3}Que te donne, que te rapporte la langue trompeuse ?
\VS{4}Ce sont des flèches aiguës tirées par un homme puissant et des charbons ardents du genêt\FTNT{Jé. 9:3 ; Ja. 3:5-6.}.
\VS{5}Malheureux que je suis de séjourner à Méschec et de demeurer aux tentes de Kédar !
\VS{6}Assez longtemps mon âme a demeuré auprès de ceux qui haïssent la paix !
\VS{7}Je ne cherche que la paix, mais lorsque j'en parle, ils sont pour la guerre.
\Chap{121}
\TextTitle{Yahweh ne dort ni ne sommeille}
\VerseOne{}Cantique des degrés. J'élève mes yeux vers les montagnes, d'où me viendra le secours.
\VS{2}Mon secours vient de Yahweh qui a fait les cieux et la terre\FTNT{Ps. 124:8.}.
\VS{3}Il ne permettra point que ton pied chancelle, celui qui te garde ne sommeillera point\FTNT{Es. 27:3 ; Pr. 3:23.}.
\VS{4}Voici, il ne sommeille ni ne dort celui qui garde Israël.
\VS{5}Yahweh est celui qui te garde, Yahweh est ton ombre à ta main droite\FTNT{Es. 25:4.}.
\VS{6}Pendant le jour, le soleil ne te frappera point, ni la lune pendant la nuit\FTNT{Es. 49:10. Ap. 7:16.}.
\VS{7}Yahweh te gardera de tout mal, il gardera ton âme.
\VS{8}Yahweh gardera ton départ et ton arrivée, dès maintenant et à jamais\FTNT{De. 28:6.}.
\Chap{122}
\TextTitle{Jérusalem, la ville de Yahweh}
\VerseOne{}Cantique des degrés de David. Je me réjouis à cause de ceux qui me disent : Allons à la maison de Yahweh\FTNT{Ps. 84:1-5.} !
\VS{2}Nos pieds s'arrêtent dans tes portes, ô Jérusalem !
\VS{3}Jérusalem, qui est bâtie comme une ville dont les édifices sont joints ensemble,
\VS{4}à laquelle montent les tribus, les tribus de Yahweh, selon le témoignage d'Israël, pour célébrer le Nom de Yahweh.
\VS{5}Car c'est là qu'ont été posés les trônes pour juger\FTNT{Mt. 19:28.}. Les trônes de la maison de David.
\VS{6}Demandez la paix de Jérusalem ; que ceux qui t'aiment jouissent du repos.
\VS{7}Que la paix soit dans tes murs et la tranquillité dans tes palais.
\VS{8}Pour l'amour de mes frères et de mes amis, je prie maintenant pour ta paix.
\VS{9}A cause de la maison de Yahweh notre Dieu, je fais une requête pour ton bonheur.
\Chap{123}
\TextTitle{Les regards fixés sur Yahweh}
\VerseOne{}Cantique des degrés. J'élève mes yeux vers toi qui habites dans les cieux.
\VS{2}Voici, comme les yeux des serviteurs regardent la main de leurs maîtres, comme les yeux de la servante regardent la main de sa maîtresse, ainsi nos yeux regardent à Yahweh notre Dieu, jusqu'à ce qu'il ait pitié de nous\FTNT{Ps. 25:15.}.
\VS{3}Aie pitié de nous, ô Yahweh ! Aie pitié de nous ! Car nous sommes assez rassasiés de mépris !
\VS{4}Notre âme est assez rassasiée des moqueries des orgueilleux, du mépris des hautains.
\Chap{124}
\TextTitle{Yahweh, le Dieu qui secours et protège son peuple}
\VerseOne{}Cantique des degrés, de David. Sans Yahweh, qui nous protégea, qu'Israël le dise !
\VS{2}Sans Yahweh, qui nous protégea, quand les hommes s'élevèrent contre nous ?
\VS{3}Ils nous auraient engloutis tous vivants quand leur colère s'enflamma contre nous.
\VS{4}Alors les eaux nous auraient submergés, les torrents auraient passé sur notre âme.
\VS{5}Alors les flots impétueux auraient passé sur notre âme.
\VS{6}Béni soit Yahweh qui ne nous a point livrés en proie à leurs dents !
\VS{7}Notre âme s'est échappée comme l'oiseau du filet des oiseleurs ; le filet a été rompu, et nous nous sommes échappés\FTNT{Pr. 6:5.}.
\VS{8}Notre secours est dans le Nom de Yahweh\FTNT{Ac. 4:11-12.} qui a fait les cieux et la terre.
\Chap{125}
\TextTitle{Yahweh entoure tous ceux qui se confient en lui}
\VerseOne{}Cantique des degrés. Ceux qui se confient en Yahweh sont comme la montagne de Sion : Elle ne chancelle point et est affermie pour toujours.
\VS{2}Quant à Jérusalem, il y a des montagnes autour d'elle, ainsi Yahweh entoure son peuple, dès maintenant et à jamais.
\VS{3}Car la verge de la méchanceté ne restera pas sur le lot des justes, de peur que les justes n'étendent leurs mains vers l'iniquité\FTNT{Es. 14:5.}.
\VS{4}Yahweh, répands tes bienfaits sur les bons et sur ceux dont le cœur est droit.
\VS{5}Mais ceux qui s'engagent dans des voies détournées, que Yahweh les fasse marcher avec les ouvriers d'iniquité\FTNT{Mt. 7:23.}. La paix sera sur Israël.
\Chap{126}
\TextTitle{Yahweh, le libérateur}
\VerseOne{}Cantique des degrés. Quand Yahweh ramena les captifs de Sion, nous étions comme ceux qui font un rêve.
\VS{2}Alors notre bouche était remplie de joie, et notre langue de chants de triomphe, alors on disait parmi les nations : Yahweh a fait de grandes choses pour eux !
\VS{3}Yahweh a fait de grandes choses pour nous ; nous sommes dans la joie.
\VS{4}Ô Yahweh ! Ramène nos captifs, comme des ruisseaux dans le midi\FTNT{Os. 6:11 ; Joë. 3:11.} !
\VS{5}Ceux qui sèment avec larmes moissonneront avec chants d'allégresse\FTNT{Ga. 6:9.}.
\VS{6}Celui qui marche en pleurant quand il porte la semence pour la mettre en terre, revient avec des chants d'allégresse quand il porte ses gerbes.
\Chap{127}
\TextTitle{Yahweh, le plus grand architecte}
\VerseOne{}Cantique des degrés, de Salomon. Si Yahweh ne bâtit la maison, ceux qui la bâtissent travaillent en vain ; si Yahweh ne garde la ville, celui qui la garde fait le guet en vain.
\VS{2}C'est en vain que vous vous levez de grand matin, que vous vous couchez tard, et que vous mangez le pain de douleurs ; certes c'est Dieu qui donne du repos à celui qu'il aime\FTNT{Ez. 20:20 ; Mc. 2:27.}.
\VS{3}Voici, les fils sont un héritage donné par Yahweh et le fruit du ventre est une récompense de Dieu\FTNT{Ps. 113:9 ; Ps. 128:3-6.}.
\VS{4}Telles sont les flèches dans la main d'un homme puissant, tels sont les fils de la jeunesse.
\VS{5}Heureux l'homme qui en a rempli son carquois ! Ils ne seront pas honteux quand ils parleront avec leurs ennemis à la porte.
\Chap{128}
\TextTitle{Yahweh assure la paix à celui qui le craint}
\VerseOne{}Cantique des degrés. Heureux tout homme qui craint Yahweh et marche dans ses voies !
\VS{2}Tu jouis du travail de tes mains ; tu es heureux et tu prospères\FTNT{Es. 3:10.}.
\VS{3}Ta femme est dans ta maison comme une vigne qui porte du fruit ; tes fils sont autour de ta table comme des plants d'oliviers.
\VS{4}Voici, certainement ainsi sera béni l'homme qui craint Yahweh.
\VS{5}Yahweh te bénira de Sion et tu verras le bien de Jérusalem tous les jours de ta vie.
\VS{6}Tu verras les fils de tes fils. La paix sera sur Israël.
\Chap{129}
\TextTitle{L'opprimé plus que vainqueur en Yahweh}
\VerseOne{}Cantique des degrés. Qu'Israël dise maintenant : Ils m'ont souvent tourmenté dès ma jeunesse.
\VS{2}Ils m'ont assez opprimé dès ma jeunesse, mais ils ne m'ont pas vaincu.
\VS{3}Des laboureurs ont labouré mon dos, ils y ont tracé de longs sillons.
\VS{4}Yahweh est juste, il a coupé les cordes des méchants.
\VS{5}Qu'ils soient honteux et qu'ils reculent, tous ceux qui haïssent Sion !
\VS{6}Qu'ils soient comme l'herbe des toits qui sèche avant qu'on l'arrache !
\VS{7}Le moissonneur n'en remplit point sa main, ni celui qui lie les gerbes n'en remplit point ses bras ;
\VS{8}et les passants ne disent pas : Que la bénédiction de Yahweh soit sur vous ! Nous vous bénissons au nom de Yahweh !
\Chap{130}
\TextTitle{La rédemption en abondance auprès de Yahweh}
\VerseOne{}Cantique des degrés. Ô Yahweh ! Je t'invoque du fond de l'abîme.
\VS{2}Seigneur, écoute ma voix ! Que tes oreilles soient attentives à la voix de mes supplications !
\VS{3}Yahweh ! si tu prends garde aux iniquités, Seigneur, qui subsistera ?
\VS{4}Mais le pardon se trouve auprès de toi, afin qu'on te craigne\FTNT{Mt. 26:28 ; Ro. 3:24 ; Col. 1:12-14.}.
\VS{5}J'espère en Yahweh, mon âme espère, et j'attends sa parole.
\VS{6}Mon âme attend le Seigneur plus que les sentinelles n'attendent le matin, plus que les sentinelles n'attendent le matin.
\VS{7}Israël, attends-toi à Yahweh, car Yahweh est miséricordieux et la rédemption est auprès de lui en abondance.
\VS{8}Lui-même rachètera Israël de toutes ses iniquités.
\Chap{131}
\TextTitle{Mettre son espoir en Yahweh seul}
\VerseOne{}Cantique des degrés, de David. Ô Yahweh ! Je n'ai ni un cœur qui s'élève ni un regard hautain\FTNT{Pr. 16:5 ; Pr. 6:17.} ; je ne m'occupe pas de choses trop grandes et trop extraordinaires pour moi.
\VS{2}J'ai l'âme calme et tranquille comme un enfant sevré de sa mère ; j'ai l'âme comme un enfant sevré.
\VS{3}Israël attends-toi à Yahweh dès maintenant et à jamais !
\Chap{132}
\TextTitle{Sion, le trône de Yahweh}
\VerseOne{}Cantique des degrés. Ô Yahweh ! Souviens-toi de David et de toute son affliction !
\VS{2}Il a juré à Yahweh et fait ce vœu au puissant de Jacob :
\VS{3}Je n'entrerai pas dans la tente où j'habite, je ne monterai pas sur le lit où je couche,
\VS{4}je ne donnerai pas du sommeil à mes yeux, je ne laisserai pas sommeiller mes paupières,
\VS{5}jusqu'à ce que j'aie trouvé un lieu pour Yahweh, une demeure pour le puissant de Jacob\FTNT{1 Ch. 15:1.}.
\VS{6}Voici, nous avons entendu parler d'elle à Ephrata, nous l'avons trouvée dans les champs de Jaar.
\VS{7}Entrons dans sa demeure, prosternons-nous devant son marchepied.
\VS{8}Lève-toi, ô Yahweh, pour venir à ton lieu de repos, toi et l'arche de ta force\FTNT{No. 10:35-36 ; 2 Ch. 6:41.}.
\VS{9}Que tes sacrificateurs soient revêtus de justice et que tes bien-aimés chantent de joie\FTNT{Es. 11:5 ; Ap. 19:8.} !
\VS{10}Pour l'amour de David, ton serviteur, ne permets pas que ton oint retourne en arrière !
\VS{11}Yahweh a juré la vérité à David, et il ne se rétractera pas, disant : Je mettrai le fruit de tes entrailles\FTNT{2 S. 7:12 ; 1 R. 8:25 ; 2 Ch. 6:16 ; Lu. 1:69 ; Ac. 2:30.} sur ton trône.
\VS{12}Si tes fils gardent mon alliance et mon témoignage que je leur enseignerai, leurs fils aussi seront assis à perpétuité sur ton trône.
\VS{13}Car Yahweh a choisi Sion, il l'a préférée pour être son trône :
\VS{14}Elle est mon lieu de repos à perpétuité, j'y habiterai parce que je l'ai désirée.
\VS{15}Je bénirai abondamment sa nourriture, je rassasierai de pain ses pauvres.
\VS{16}Je revêtirai de salut ses sacrificateurs, et ses bien-aimés chanteront avec des cris de joie.
\VS{17}Je ferai qu'en elle germera une corne à David ; je préparerai une lampe à mon oint,
\VS{18}je revêtirai de honte ses ennemis, et sur lui fleurira son diadème.
\Chap{133}
\TextTitle{La bénédiction dans la communion fraternelle}
\VerseOne{}Cantique des degrés. De David. Voici, oh ! Que c'est une chose bonne et que c'est une chose agréable que des frères demeurent unis ensemble\FTNT{Hé. 13:1 ; Ac. 2:46.} !
\VS{2}C'est comme cette huile précieuse, répandue sur la tête, qui coule sur la barbe d'Aaron\FTNT{Ex. 30:22-30.}, sur le bord de ses vêtements ;
\VS{3}comme la rosée de l'Hermon, celle qui descend sur les montagnes de Sion. Car c'est là que Yahweh a ordonné la bénédiction et la vie, pour l'éternité.
\Chap{134}
\TextTitle{Bénissez Yahweh, vous tous ses serviteurs}
\VerseOne{}Cantique des degrés. Voici, bénissez Yahweh ! Vous tous les serviteurs de Yahweh ! Qui vous tenez toutes les nuits dans la maison de Yahweh !
\VS{2}Elevez vos mains vers le lieu saint ! Et bénissez Yahweh !
\VS{3}Que Yahweh, qui a fait les cieux et la terre, te bénisse de Sion !
\Chap{135}
\TextTitle{La souveraineté de Dieu}
\VerseOne{}Louez le Nom de Yahweh ! Vous serviteurs de Yahweh ! Louez-le !
\VS{2}Vous qui vous tenez dans la maison de Yahweh, dans les parvis de la maison de notre Dieu,
\VS{3}louez Yahweh, car Yahweh est bon ! Chantez son Nom, car il est agréable !
\VS{4}Car Yahweh s'est choisi Jacob et Israël pour sa possession\FTNT{Ex. 19:5 ; De. 7:6 ; Tit. 2:14 ; 1 Pi. 2:9.}.
\VS{5}Certainement, je sais que Yahweh est grand et que notre Seigneur est au-dessus de tous les dieux.
\VS{6}Yahweh fait tout ce qu'il lui plaît, dans les cieux et sur la terre, dans la mer et dans tous les abîmes.
\VS{7}C'est lui qui fait monter les vapeurs des extrémités de la terre ; il fait les éclairs et la pluie ; il tire le vent hors de ses trésors.
\VS{8}C'est lui qui a frappé les premiers-nés d'Egypte, tant des hommes que des bêtes ;
\VS{9}qui a envoyé des prodiges et des miracles au milieu de toi, ô Egypte ! Contre Pharaon et contre tous ses serviteurs ;
\VS{10}qui a frappé plusieurs nations et tué les puissants rois ;
\VS{11}Sihon, roi des Amoréens, et Og, roi de Basan, et ceux de tous les royaumes de Canaan\FTNT{No. 21:33-35 ; De. 3:11.} ;
\VS{12}qui a donné leur pays en héritage, en héritage à Israël son peuple.
\VS{13}Yahweh, ton Nom est pour toujours ! Yahweh, ta mémoire de génération en génération !
\VS{14}Car Yahweh jugera son peuple et se repentira à l'égard de ses serviteurs.
\VS{15}Les dieux des nations ne sont que de l'or et de l'argent, un ouvrage de mains d'homme.
\VS{16}Ils ont une bouche, et ne parlent point ; ils ont des yeux, et ne voient point ;
\VS{17}ils ont des oreilles, et n'entendent point ; il n'y a point de souffle dans leur bouche.
\VS{18}Ils leur ressemblent ceux qui les font, et tous ceux qui s'y confient.
\VS{19}Maison d'Israël, bénissez Yahweh ! Maison d'Aaron, bénissez Yahweh !
\VS{20}Maison des Lévites, bénissez Yahweh ! Vous qui craignez Yahweh, bénissez Yahweh !
\VS{21}Béni soit de Sion Yahweh qui habite dans Jérusalem ! Louez Yahweh !
\Chap{136}
\TextTitle{La bonté de Yahweh demeure à toujours}
\VerseOne{}Célébrez Yahweh, car il est bon, car sa bonté demeure à toujours !
\VS{2}Célébrez le Dieu des dieux, car sa bonté demeure à toujours !
\VS{3}Célébrez le Seigneur des seigneurs, car sa bonté demeure à toujours !
\VS{4}Célébrez celui qui seul fait de grandes merveilles, car sa bonté demeure à toujours !
\VS{5}Celui qui a fait avec intelligence les cieux, car sa bonté demeure à toujours !
\VS{6}Celui qui a étendu la terre sur les eaux, car sa bonté demeure à toujours !
\VS{7}Celui qui a fait les grands luminaires, car sa bonté demeure à toujours !
\VS{8}Le soleil pour dominer sur le jour, car sa bonté demeure à toujours !
\VS{9}La lune et les étoiles pour dominer la nuit, car sa bonté demeure à toujours !
\VS{10}Celui qui a frappé l'Egypte dans leurs premiers-nés, car sa bonté demeure à toujours !
\VS{11}Qui a fait sortir Israël du milieu d'eux, car sa bonté demeure à toujours.
\VS{12}Et cela avec main forte et bras étendu, car sa bonté demeure à toujours !
\VS{13}Il a fendu la Mer Rouge en deux, car sa bonté demeure à toujours !
\VS{14}Il a fait passer Israël par le milieu d'elle, car sa bonté demeure à toujours !
\VS{15}Il a renversé Pharaon et son armée dans la Mer Rouge, car sa bonté demeure à toujours !
\VS{16}Il a conduit son peuple dans le désert, car sa bonté demeure à toujours !
\VS{17}Il a frappé les grands rois, car sa bonté demeure à toujours !
\VS{18}Qui a tué des grands rois, car sa bonté demeure à toujours !
\VS{19}Sihon, roi des Amoréens, car sa bonté demeure à toujours !
\VS{20}Og, roi de Basan, car sa bonté demeure à toujours !
\VS{21}Il a donné leur pays en héritage, car sa bonté demeure à toujours\FTNT{Jos. 12:7.} !
\VS{22}En héritage à Israël son serviteur, car sa bonté demeure à toujours !
\VS{23}Et qui, lorsque nous étions humiliés, s'est souvenu de nous, car sa bonté demeure à toujours !
\VS{24}Il nous a délivrés de la main de nos adversaires, car sa bonté demeure à toujours !
\VS{25}Il donne la nourriture à toute chair, car sa bonté demeure à toujours\FTNT{Ps. 104:21 ; Mt. 6:26 ; Ps. 147:9.} !
\VS{26}Célébrez le Dieu des cieux, car sa bonté demeure à toujours !
\Chap{137}
\TextTitle{Le coeur des captifs}
\VerseOne{}Sur les bords des fleuves de Babylone, nous étions assis et nous pleurions en nous souvenant de Sion.
\VS{2}Nous avions suspendu nos harpes au milieu des saules.
\VS{3}Là, ceux qui nous avaient emmenés en captivité, nous ont demandé des paroles de chants, et nos oppresseurs de la joie, en nous disant : Chantez-nous quelques cantiques de Sion ! Nous avons répondu :
\VS{4}Comment chanterions-nous les cantiques de Yahweh sur une terre étrangère ?
\VS{5}Si je t'oublie, Jérusalem, que ma droite s'oublie elle-même.
\VS{6}Que ma langue soit attachée à mon palais\FTNT{Ez. 3:26.}, si je ne me souviens pas de toi, si je ne fais pas de Jérusalem le sujet de ma réjouissance.
\VS{7}Ô Yahweh, souviens-toi des fils d'Edom, qui dans la journée de Jérusalem disaient : Rasez, rasez jusqu'à ses fondements\FTNT{Jé. 25:15-21 ; Jé. 49:7-8 ; Ez. 25:12 ; La. 4:21 ; Am. 1:11.} !
\VS{8}Fille de Babylone, qui va être détruite, heureux celui qui te rend la pareille de ce que tu nous as fait\FTNT{Jé. 50:15-29 ; Ap. 18:6.} !
\VS{9}Heureux celui qui saisit tes petits enfants et qui les écrase contre le rocher\FTNT{Es. 13:16.} !
\Chap{138}
\TextTitle{La renommée de Yahweh dans les nations}
\VerseOne{}Psaume de David. Je te célèbre de tout mon cœur, je te chante des louanges dans la présence de Dieu.
\VS{2}Je me prosterne dans ton saint temple, et je célèbre ton Nom à cause de ta bonté et de ta fidélité ; car ta renommée s'est accrue par l'accomplissement de ta promesse.
\VS{3}Le jour où je t'ai invoqué, tu m'as exaucé, tu m'as rassuré, tu m'as fortifié d'une nouvelle force en mon âme.
\VS{4}Yahweh ! Tous les rois de la terre te célèbrent, quand ils entendent les paroles de ta bouche.
\VS{5}Ils chantent les voies de Yahweh, car la gloire de Yahweh est grande.
\VS{6}Car Yahweh est haut élevé, il voit les humbles et il reconnaît de loin les orgueilleux.
\VS{7}Quand je marche au milieu de l'adversité, tu me rends la vie, tu avances ta main contre la colère de mes ennemis, et ta droite me délivre.
\VS{8}Yahweh achèvera ce qui me concerne. Yahweh, ta bonté demeure toujours ; tu n'abandonnes pas l'œuvre de tes mains\FTNT{Ph. 1:6.}.
\Chap{139}
\TextTitle{L'omniscience de Yahweh}
\VerseOne{}Psaume de David, donné au chef des chantres. Yahweh, tu me sondes et tu me connais\FTNT{Jé. 12:3 ; Ps. 17:3.}.
\VS{2}Tu sais quand je m'assieds et quand je me lève ; tu discernes de loin ma pensée.
\VS{3}Tu sais quand je marche et quand je me couche ; tu connais parfaitement toutes mes voies.
\VS{4}Avant que la parole soit sur ma langue, voici, ô Yahweh, tu la connais déjà !
\VS{5}Tu m'entoures par derrière et par devant, et tu mets ta main sur moi.
\VS{6}Ta science est trop merveilleuse pour moi, elle est si haut élevée que je ne saurais l'atteindre\FTNT{Job. 42:3 ; Ps. 92:6 ; Ro. 11:33.}.
\VS{7}Où irai-je loin de ton Esprit, et où fuirai-je loin de ta face\FTNT{Jé. 23:24 ; Am. 9:2-4 ; Jon. 1:3.} ?
\VS{8}Si je monte aux cieux, tu y es ; si je me couche dans le scheol, t'y voilà.
\VS{9}Si je prends les ailes de l'aurore et que je demeure à l'extrémité de la mer,
\VS{10}là aussi ta main me conduira et ta droite me saisira.
\VS{11}Si je dis : Au moins les ténèbres me couvriront, la nuit même sera une lumière tout autour de moi.
\VS{12}Même les ténèbres ne me cacheront point de toi, et la nuit resplendira comme le jour, et les ténèbres comme la lumière.
\VS{13}Tu as créé mes reins, tu me couvres du sein de ma mère.
\VS{14}Je te célèbre de ce que je suis une créature redoutée et merveilleuse ; tes œuvres sont merveilleuses, et mon âme le reconnaît très bien.
\VS{15}Mon corps n'était pas caché devant toi lorsque j'ai été fait dans un lieu secret et brodé dans les profondeurs de la terre\FTNT{Ps. 119:73 ; Ec. 11:5.}.
\VS{16}Tes yeux me voyaient quand je n'étais qu'un embryon, et sur ton livre étaient inscrits tous les jours qui m'étaient destinés\FTNT{Ph. 4:3 ; Ap. 3:5 ; Ap. 20:15.}.
\VS{17}Dieu ! Que tes pensées sont précieuses ! Que le nombre en est grand !
\VS{18}Si je les compte, elles sont plus nombreuses que les grains de sable. Je m'éveille et je suis encore avec toi.
\VS{19}Ô Dieu ! Ne tueras-tu pas le méchant ? C'est pourquoi, hommes sanguinaires, retirez-vous loin de moi !
\VS{20}Car ils ont parlé de toi en pensant à quelque méchanceté ; ils ont élevé tes ennemis en mentant.
\VS{21}Yahweh, n'aurais-je point en haine ceux qui te haïssent ; et ne serais-je point irrité contre ceux qui s'élèvent contre toi ?
\VS{22}Je les hais d'une parfaite haine ; ils sont pour moi des ennemis.
\VS{23}Ô Dieu ! Sonde-moi et considère mon cœur ! Eprouve-moi et considère mes discours !
\VS{24}Et regarde si je suis sur une mauvaise voie ; conduis-moi sur la voie de l'éternité.
\Chap{140}
\TextTitle{Yahweh, le protecteur}
\VerseOne{}Psaume de David, donné au chef des chantres. Yahweh, délivre-moi de l'homme méchant, garde-moi de l'homme violent.
\VS{2}Ils méditent des méchancetés dans leur cœur, tous les jours ils complotent des guerres.
\VS{3}Ils aiguisent leur langue comme un serpent, il y a du venin de vipère sous leurs lèvres. Sélah.
\VS{4}Yahweh, garde-moi de la main du méchant, préserve-moi de l'homme violent, de ceux qui méditent de me faire tomber.
\VS{5}Les orgueilleux me tendent un piège et des filets, et ils étendent des rets le long du chemin, ils me dressent des embûches. Sélah.
\VS{6}Je dis à Yahweh : Tu es mon Dieu, Yahweh ! Prête l'oreille à la voix de mes supplications !
\VS{7}Ô Yahweh ! Seigneur ! La force de mon salut ! Tu couvres ma tête au jour de la bataille.
\VS{8}Yahweh n'accorde point au méchant ses désirs ; qu'il n'apporte pas ses méchants desseins, ils s'élèveraient. Sélah.
\VS{9}Quant à la tête de ceux qui m'environnent, que la méchanceté de leurs lèvres les recouvre.
\VS{10}Que des charbons ardents soient jetés sur eux ! Qu'ils tombent sur eux ! Qu'il les fasse tomber dans le feu, et dans des fosses profondes, sans qu'ils se relèvent\FTNT{Pr. 25:21-22 ; Ro. 12:20.} !
\VS{11}Que l'homme à la langue méchante ne soit point affermi sur la terre ; quant à l'homme violent et mauvais, qu'on le chasse jusqu'à ce qu'il soit exterminé.
\VS{12}Je sais que Yahweh fera justice au malheureux et droit aux indigents.
\VS{13}Quoi qu'il en soit, les justes célébreront ton Nom, les hommes droits habiteront devant ta face.
\Chap{141}
\TextTitle{Yahweh, garde-moi du mal !}
\VerseOne{}Psaume de David. Yahweh, je t'invoque, hâte-toi de venir vers moi ; prête l'oreille à ma voix lorsque je crie à toi.
\VS{2}Que ma prière te soit agréable comme l'encens, et l'élévation de mes mains comme l'offrande du soir\FTNT{Ex. 30:1 ; Ap. 5:8 ; Ap. 8:3.}.
\VS{3}Yahweh, mets une garde à ma bouche, garde l'entrée de mes lèvres.
\VS{4}N'incline point mon cœur à des choses mauvaises, au point que je commette quelques méchantes actions par malice, avec les hommes qui font le mal ; et que je ne mange point de leurs délices.
\VS{5}Que le juste me frappe, ce me sera une faveur ; et qu'il me réprimande, ce sera pour moi un baume excellent\FTNT{Pr. 27:6 ; Ec. 7:5.} ; il ne blessera point ma tête ; car ma prière sera pour eux leur calamité.
\VS{6}Que leurs juges soient précipités le long des rochers, et l'on écoutera mes paroles, car elles sont agréables.
\VS{7}Nos os sont dispersés dans la bouche du scheol comme quand on laboure la terre et on fend le bois.
\VS{8}C'est pourquoi, ô Yahweh, Seigneur, mes yeux sont sur toi, je me suis retiré vers toi, n'abandonne point mon âme !
\VS{9}Garde-moi du piège qu'ils m'ont tendu et des filets de ceux qui font le mal.
\VS{10}Que tous les méchants tombent dans leurs filets, jusqu'à ce que je sois passé.
\Chap{142}
\TextTitle{Yahweh, mon refuge}
\VerseOne{}Cantique de David. Prière qu'il fit lorsqu'il était dans la caverne\FTNT{1 S. 24:4.}.
\VS{2}Je crie de ma voix à Yahweh, je supplie de ma voix Yahweh.
\VS{3}Je répands devant lui ma complainte, je déclare mon angoisse devant lui\FTNT{1 S. 1:15 ; La. 2:19.}.
\VS{4}Quand mon esprit est abattu en moi, toi, tu connais mon sentier. Ils me tendent un piège sur le chemin par lequel je marche.
\VS{5}Je contemple à ma droite et je regarde, et il n'y a personne qui me reconnaît ; tout refuge s'évanouit devant moi, il n'y a personne qui prend soin de mon âme.
\VS{6}Yahweh, je crie vers toi ; je dis : Tu es mon refuge, ma part sur la terre des vivants.
\VS{7}Sois attentif à mon cri car je suis devenu très affaibli. Délivre-moi de ceux qui me poursuivent car ils sont plus puissants que moi.
\VS{8}Retire mon âme de sa prison afin que je célèbre ton Nom ! Les justes viendront m'entourer quand tu m'auras fait du bien.
\Chap{143}
\TextTitle{Yahweh, enseigne-moi à faire ta volonté}
\VerseOne{}Psaume de David. Yahweh, écoute ma requête, prête l'oreille à mes supplications ! Exauce-moi dans ta fidélité, réponds-moi à cause de ta justice !
\VS{2}N'entre point en jugement avec ton serviteur, car aucun homme vivant n'est juste devant toi.
\VS{3}Car l'ennemi poursuit mon âme, il foule ma vie par terre ; il me fait habiter dans les ténèbres comme ceux qui sont morts depuis longtemps.
\VS{4}Et mon esprit est abattu au-dedans de moi, mon cœur est épouvanté en mon sein.
\VS{5}Je me souviens des jours anciens, je médite sur toutes tes œuvres, je médite sur l'ouvrage de tes mains\FTNT{Ps. 77:11-13.}.
\VS{6}J'étends mes mains vers toi ; mon âme s'adresse à toi comme une terre desséchée\FTNT{Ps. 28:1 ; Ps. 42:1-3.}. Sélah.
\VS{7}Ô Yahweh, hâte-toi, réponds-moi ! Mon esprit se consume ! Ne me cache point ta face au point que je devienne semblable à ceux qui descendent dans la fosse !
\VS{8}Fais-moi entendre dès le matin ta miséricorde, car je me confie en toi ; fais-moi connaître le chemin par lequel je dois marcher, car j'ai élevé mon cœur vers toi\FTNT{Ps. 25:1.}.
\VS{9}Yahweh, délivre-moi de mes ennemis, car je me suis réfugié auprès de toi !
\VS{10}Enseigne-moi à faire ta volonté, car tu es mon Dieu ! Que ton bon Esprit me conduise sur la voie de la droiture\FTNT{Jn. 16:13.} !
\VS{11}Yahweh, rends-moi la vie pour l'amour de ton Nom ! Retire mon âme de la détresse à cause de ta justice !
\VS{12}Et selon la bonté que tu as pour moi, retranche mes ennemis ! Détruis tous ceux qui tiennent mon âme oppressée, parce que je suis ton serviteur !
\Chap{144}
\TextTitle{Se confier en Yahweh, le Rocher}
\VerseOne{}Psaume de David. Béni soit Yahweh, mon rocher\FTNT{Voir commentaire en Es. 8:13-17.} qui exerce mes mains au combat et mes doigts à la bataille,
\VS{2}qui déploie sa bonté envers moi, qui est ma forteresse, ma haute retraite, mon libérateur\FTNT{Es. 59:20-21 ; Ro. 11:26.}, mon bouclier\FTNT{Ep. 6:16.}, mon refuge\FTNT{Ps. 91 ; Mt. 11:28-30.}, qui m'assujettit mon peuple.
\VS{3}Ô Yahweh ! Qu'est-ce que l'homme pour que tu aies soin de lui\FTNT{Ps. 8:5 ; Job. 7:17 ; Hé. 2:6-7.} ? Le fils de l'homme mortel pour que tu prennes garde à lui ?
\VS{4}L'homme est semblable à la vanité, ses jours sont comme une ombre qui passe\FTNT{Ps. 102:12 ; Job. 14:1-2 ; Ec. 6:12.}.
\VS{5}Yahweh abaisse tes cieux et descends ! Touche les montagnes et qu'elles soient fumantes\FTNT{Es. 63:19 ; Ps. 18:7-8.}.
\VS{6}Lance les éclairs et disperse mes ennemis ! Lance tes flèches et mets-les en déroute !
\VS{7}Etends tes mains d'en haut ; sauve-moi et délivre-moi des grandes eaux, de la main des fils de l'étranger,
\VS{8}dont la bouche profère le mensonge, et dont la droite est une droite trompeuse !
\VS{9}Ô Dieu ! Je chanterai un cantique nouveau ! Je te célèbrerai sur le luth à dix cordes !
\VS{10}Toi qui donnes la délivrance aux rois et qui délivres de l'épée meurtrière David, ton serviteur.
\VS{11}Retire-moi et délivre-moi de la main des fils de l'étranger, dont la bouche profère le mensonge et dont la droite est une droite trompeuse ;
\VS{12}afin que nos fils soient comme des plantes qui croissent dans leur jeunesse et nos filles comme des pierres angulaires taillées pour l'ornement d'un palais.
\VS{13}Que nos greniers soient pleins, fournissant toute espèce de provision ; que nos troupeaux multiplient par milliers, même par dix milliers dans nos rues.
\VS{14}Que nos bœufs soient chargés de graisse. Qu'il n'y ait ni brèche, ni sortie dans nos murailles, ni cri dans nos places.
\VS{15}Heureux le peuple pour qui il en est ainsi ! Heureux le peuple dont Yahweh est le Dieu !
\Chap{145}
\TextTitle{Louange à Yahweh pour tout ce qu'il est}
\VerseOne{}Psaume de louange, composé par David. [Aleph.] Mon Dieu, mon roi, je t'exalterai et je bénirai ton Nom à toujours, et à perpétuité !
\VS{2}[Beth.] Je te bénirai chaque jour, et je louerai ton Nom à toujours, et à perpétuité !
\VS{3}[Guimel.] Yahweh est grand et très digne de louanges, il n'est pas possible de sonder sa grandeur.
\VS{4}[Daleth.] Que chaque génération célèbre tes œuvres et publie tes hauts faits !
\VS{5}[He.] Je dirai la splendeur glorieuse de ta majesté et de tes faits merveilleux.
\VS{6}[Vav.] On parlera de ta puissance redoutable, et je raconterai ta grandeur.
\VS{7}[Zayin.] Ils proclameront le souvenir de ton immense bonté, et ils raconteront avec chants de triomphe ta justice.
\VS{8}[Heth.] Yahweh est miséricordieux et compatissant, lent à la colère et grand en bonté.
\VS{9}[Teth.] Yahweh est bon envers tous et ses compassions sont au-dessus de toutes ses œuvres.
\VS{10}[Yod.] Yahweh, toutes tes œuvres te célébreront, et tes fidèles te béniront.
\VS{11}[Kaf.] Ils diront la gloire de ton règne, et ils proclameront ta puissance
\VS{12}[Lamed.] pour faire connaître aux fils de l'homme ta puissance et la splendeur glorieuse de ton règne.
\VS{13}[Mem.] Ton règne est un règne de tous les siècles et ta domination subsiste dans tous les âges.
\VS{14}[Samech.] Yahweh soutient tous ceux qui tombent et redresse tous ceux qui sont courbés\FTNT{Ps. 146:8.}.
\VS{15}[Ayin.] Les yeux de tous les animaux s'attendent à toi et tu leur donnes leur nourriture en leur temps.
\VS{16}[Pe.] Tu ouvres ta main et tu rassasies à souhait toute créature vivante.
\VS{17}[Tsade.] Yahweh est juste dans toutes ses voies et plein de bonté dans toutes ses œuvres\FTNT{Da. 4:37.}.
\VS{18}[Qof.] Yahweh est près de tous ceux qui l'invoquent, de tous ceux qui l'invoquent avec vérité\FTNT{Ps. 34:18.}.
\VS{19}[Resh.] Il accomplit le désir de ceux qui le craignent, il entend leur cri et les délivre.
\VS{20}[Shin.] Yahweh garde tous ceux qui l'aiment, mais il exterminera tous les méchants.
\VS{21}[Tav.] Ma bouche racontera la louange de Yahweh, et toute chair bénira le Nom de sa sainteté à toujours, et à perpétuité\FTNT{Ps. 103:1.}.
\Chap{146}
\TextTitle{La fidélité de Yahweh dure à toujours}
\VerseOne{}Louez Yahweh ! Mon âme, loue Yahweh !
\VS{2}Je louerai Yahweh durant ma vie, je chanterai mon Dieu tant que je vivrai !
\VS{3}Ne vous confiez pas aux grands, ni en aucun fils de l'homme qui ne peuvent délivrer.
\VS{4}Son esprit s'en va et l'homme retourne dans sa terre, et ce même jour ses desseins périssent.
\VS{5}Heureux celui qui a pour secours le Dieu de Jacob, qui met son espoir en Yahweh, son Dieu !
\VS{6}Il a fait les cieux et la terre, la mer et tout ce qui s'y trouve. Il garde la vérité à toujours !
\VS{7}Il fait droit aux opprimés, il donne du pain aux affamés ; Yahweh délie ceux qui sont liés\FTNT{Jn. 11:43-44.}.
\VS{8}Yahweh ouvre les yeux des aveugles\FTNT{Les miracles de Jésus-Christ confirment sa divinité (Es. 35:4-6 ; Lu. 7:19-23).} ; Yahweh redresse ceux qui sont courbés\FTNT{Lu. 13:11-13.} ; Yahweh aime les justes.
\VS{9}Yahweh protège les étrangers, il soutient l'orphelin et la veuve, mais il renverse la voie des méchants.
\VS{10}Yahweh règne éternellement. Ô Sion ! Ton Dieu subsiste d'âge en âge. Louez Yahweh !
\Chap{147}
\TextTitle{Yahweh aime ceux qui le craignent et qui s'attendent à sa bonté}
\VerseOne{}Louez Yahweh ! Car il est beau de chanter à notre Dieu ! Car il est doux et bienséant de le louer !
\VS{2}Yahweh est celui qui bâtit Jérusalem ; il rassemblera ceux d'Israël qui sont dispersés çà et là.
\VS{3}Il guérit ceux qui ont le cœur brisé et il bande leurs plaies\FTNT{Ex. 15:26 ; De. 32:39 ; Job. 5:18.}.
\VS{4}Il compte le nombre des étoiles, il les appelle toutes par leur nom.
\VS{5}Notre Seigneur est grand, puissant par sa force, son intelligence n'a point de limites.
\VS{6}Yahweh soutient les malheureux, mais il abaisse les méchants jusqu'à terre.
\VS{7}Chantez à Yahweh avec reconnaissance ! Célébrez notre Dieu avec la harpe !
\VS{8}Il couvre les cieux de nuées, il prépare la pluie pour la terre ; il fait germer l'herbe sur les montagnes.
\VS{9}Il donne la nourriture au bétail et aux petits du corbeau qui crient.
\VS{10}Il ne prend point plaisir dans la force du cheval ; il ne fait point cas des jambes de l'homme.
\VS{11}Yahweh aime ceux qui le craignent, ceux qui s'attendent à sa bonté.
\VS{12}Jérusalem, loue Yahweh ! Sion, loue ton Dieu !
\VS{13}Car il a affermi les barres de tes portes, il a béni tes fils au milieu de toi.
\VS{14}Il rend la paix à son territoire et te rassasie du meilleur froment.
\VS{15}C'est lui qui envoie ses ordres sur la terre, sa parole court avec rapidité\FTNT{Es. 55:10-11.}.
\VS{16}C'est lui qui donne la neige comme des flocons de laine et qui répand la gelée blanche comme de la cendre.
\VS{17}C'est lui qui lance sa glace comme par morceaux, qui peut résister devant son froid ?
\VS{18}Il envoie sa parole, et il les fond ; il fait souffler son vent, et les eaux coulent\FTNT{Ps. 135:7.}.
\VS{19}Il déclare ses paroles à Jacob, ses statuts et ses ordonnances à Israël\FTNT{Ps. 78:5.}.
\VS{20}Il n'a pas agi de même pour toutes les nations, c'est pourquoi elles ne connaissent point ses ordonnances. Louez Yahweh !
\Chap{148}
\TextTitle{La création loue son Dieu}
\VerseOne{}Louez Yahweh ! Louez des cieux Yahweh ! Louez-le dans les lieux élevés !
\VS{2}Louez-le, vous tous anges ! Louez-le, vous toutes ses armées !
\VS{3}Louez-le, vous, soleil et lune ! Louez-le, vous toutes, étoiles lumineuses !
\VS{4}Louez-le, vous, cieux des cieux ! Et vous, eaux qui êtes au-dessus des cieux !
\VS{5}Qu'ils louent le Nom de Yahweh ! Car il a commandé et ils ont été créés\FTNT{Ge. 1:3-6 ; Jé. 31:35.}.
\VS{6}Il les a établis à perpétuité et à toujours ; il a donné des lois, et il ne les violera pas\FTNT{Ps. 104:5 ; Ps. 119:91 ; Job. 14:5.}.
\VS{7}De la terre, louez Yahweh ! Louez-le, monstres marins et tous les abîmes !
\VS{8}Feu et grêle, neige et brouillard, vent impétueux qui exécutez ses ordres,
\VS{9}montagnes et toutes les collines, arbres fruitiers et tous les cèdres,
\VS{10}bêtes sauvages et tout le bétail, reptiles et oiseaux ailés,
\VS{11}rois de la terre et tous les peuples, princes et tous les juges de la terre,
\VS{12}ceux qui sont à la fleur de leur âge, et les vierges aussi, les vieillards, et les jeunes gens !
\VS{13}Qu'ils louent le Nom de Yahweh ! Car son Nom seul est haut élevé ! Sa majesté est au-dessus de la terre et des cieux.
\VS{14}Il a relevé la force de son peuple, sujet de louange pour tous ses fidèles, pour les fils d'Israël, du peuple qui est près de lui. Louez Yahweh !
\Chap{149}
\TextTitle{Adorons Yahweh}
\VerseOne{}Louez Yahweh ! Chantez à Yahweh un cantique nouveau et louez-le dans l'assemblée de ses fidèles !
\VS{2}Qu'Israël se réjouisse en celui qui l'a fait ! Et que les fils de Sion soient dans l'allégresse à cause de leur Roi\FTNT{Ps. 100:3 ; Za. 9:9 ; Mt. 21:5.} !
\VS{3}Qu'ils louent son Nom avec des danses ! Qu'ils le chantent avec le tambourin et la harpe !
\VS{4}Car Yahweh prend plaisir à son peuple, il glorifie les pauvres en les délivrant.
\VS{5}Que les fidèles se réjouissent dans la gloire, qu'ils poussent des cris de joie sur leur couche.
\VS{6}Les louanges de Dieu sont dans leur bouche et les épées affilées à deux tranchants dans leur main,
\VS{7}pour se venger des nations, pour châtier les peuples,
\VS{8}pour lier leurs rois avec des chaînes, et les plus honorables parmi eux avec des ceps de fer,
\VS{9}pour exercer sur eux le jugement qui est écrit ! Cet honneur est pour tous ses fidèles. Louez Yahweh !
\Chap{150}
\TextTitle{Que tout ce qui respire loue Yahwe !}
\VerseOne{}Louez Yahweh ! Louez Dieu à cause de sa sainteté ! Louez-le dans l'étendue de toute sa puissance !
\VS{2}Louez-le pour ses hauts faits ! Louez-le selon la grandeur de sa magnificence !
\VS{3}Louez-le au son du shofar ! Louez-le avec le luth et la harpe !
\VS{4}Louez-le avec le tambour et avec des danses ! Louez-le avec des instruments à cordes et le chalumeau !
\VS{5}Louez-le avec les cymbales sonores ! Louez-le avec les cymbales de cri de joie !
\VS{6}Que tout ce qui respire loue Yahweh ! Louez Yahweh !
\PPE{}
\end{multicols}

\clearpage\ShortTitle{Proverbes}\BookTitle{Proverbes}\BFont
\noindent\hrulefill
{\footnotesize
\textit{
\bigskip
{\centering{}
\\Auteurs : Salomon, Agur et Lemuel
\\(Heb. : Mishlei)
\\Signification : Paraboles
\\Thème : La sagesse
\\Date de rédaction : 10\up{ème} siècle av J.-C.\\}
}
%\bigskip
\textit{
\\Le mot « proverbe » désigne un genre littéraire appliqué à une sentence, une énigme, une comparaison, un oracle, une
parabole ou une parole de sagesse. Le livre des proverbes est donc un recueil de sentences dont la majeure partie
est attribuée à Salomon. Véritable collection de maximes morales et spirituelles, la sagesse, la crainte de Dieu et la
tempérance en sont les thèmes principaux.
%\bigskip
\\Ce livre met en évidence l'opposition entre la voie du méchant et celle du juste, entre la femme étrangère et la femme
vertueuse, entre l'orgueil et l'humilité, entre la sagesse et la folie, entre le chemin de la vie et celui de la mort. Comme il était coutume au Moyen-Orient, ces écrits s'adressaient particulièrement aux jeunes gens en vue de leur instruction.\bigskip
}
}
\par\nobreak\noindent\hrulefill
\begin{multicols}{2}
\Chap{1}
\TextTitle{[But du livre : connaître la sagesse}
\VerseOne{}Les Proverbes de Salomon, fils de David et roi d'Israël.
\VS{2}Pour connaître la sagesse et l'instruction, pour discerner les paroles d'intelligence ;
\VS{3}pour recevoir une leçon de bon sens, de justice, de jugement et d'équité.
\VS{4}Pour donner du discernement aux simples, aux jeunes gens de la connaissance et de la réflexion.
\VS{5}Le sage écoutera, et il augmentera son savoir, et l'homme intelligent acquerra de la prudence ;
\VS{6}afin d'entendre les paraboles et les énigmes ; les discours des sages et leurs énigmes.
\TextTitle{Le fondement de la sagesse : la crainte de Dieu}
\VS{7}La crainte de Yahweh\FTNT{Pr. 8:13.} est la principale de la science ; mais les fous méprisent la sagesse et l'instruction.
\VS{8}Mon fils, écoute l'instruction de ton père, et n'abandonne pas l'enseignement\FTNT{Loi vient de Torah (instruction, enseignement, direction etc.).} de ta mère.
\VS{9}Car se sont des grâces enfilées ensemble autour de ta tête, et des colliers autour de ton cou.
\VS{10}Mon fils, si les pécheurs veulent t'attirer, ne t'y accorde pas.
\VS{11}S'ils disent : Viens avec nous, dressons des embûches pour tuer; épions secrètement l'innocent, quoiqu'il ne nous en ai point donné de sujet aucune raison.
\VS{12}Engloutissons-les tout vifs, comme le scheol ; et tout entiers, comme ceux qui descendent dans la fosse ;
\VS{13}nous trouverons toutes sortes de biens précieux, nous remplirons nos maisons de butin ;
\VS{14}tu auras ta part avec nous, il n'y aura qu'une bourse pour nous tous.
\VS{15}Mon fils, ne te mets point en chemin avec eux ; retires ton pied de leur sentier ;
\VS{16}parce que leurs pieds courent au mal, et se hâtent pour répandre le sang\FTNT{Es. 59:7.}.
\VS{17}Car c'est en vain qu'on jette le filet devant les yeux de tout Baal ailé\FTNT{Ec. 10:20.} ;
\VS{18}ainsi ceux-ci dressent des embûches contre le sang de ceux-là et épient secrètement leurs vies.
\VS{19}Tel est le train de tout homme convoiteux de gain déshonnête, qui ôte la vie à ceux qui y sont adonnés.
\TextTitle{La sagesse crie}
\VS{20}La souveraine sagesse crie hautement au-dehors, elle fait retentir sa voix dans les rues.
\VS{21}Elle crie dans les carrefours, là où on fait le plus de bruit, aux entrées des portes, elle prononce ses paroles dans la ville :
\VS{22}Stupides, dit-elle, jusqu'à quand aimerez-vous la stupidité ? Et jusqu'à quand les moqueurs prendront-ils plaisir à la moquerie, et les haïront-ils la connaissance ?
\VS{23}Etant repris par moi, convertissez-vous ; voici, je vous donnerai de mon Esprit en abondance, et je vous ferai connaître mes paroles.
\VS{24}Parce que je crie, et que vous refusez d'entendre ; parce que j'étends ma main, et que personne n'y prend garde ;
\VS{25}et parce que vous rejetez tout mon conseil, et que vous n'avez accepté je vous reprenne ;
\VS{26}moi aussi je rirai quand vous serez dans le malheur, je me moquerai quand la terreur viendra sur vous.
\VS{27}Quand votre effroi surviendra comme une ruine, et que votre calamité viendra comme un tourbillon; vous enveloppera comme un tourbillon ; quand la détresse et l'angoisse viendront  sur vous ;
\VS{28}alors on criera vers moi, mais je ne répondrai point ; on me cherchera de grand matin, mais on ne me trouvera pas\FTNT{De. 31:18 ; Job. 35:12.}.
\VS{29}Parce qu'ils auront haï la connaissance, et qu'ils n'auront point choisi la crainte de Yahweh.
\VS{30}Ils n'ont point aimé mon conseil ; ils ont rejeté toutes mes réprimandes.
\VS{31}Qu'ils mangent donc le fruit de leur voie, et qu'ils se rassasient de leurs conseils.
\VS{32}Car l'égarement des sots les tue, et la prospérité des insensés les perd.
\VS{33}Mais celui qui m'écoute habitera en sécurité et sera tranquille, sans être effrayé d'aucun mal.
\Chap{2}
\TextTitle{La sagesse nous libère du mal}
\VerseOne{}Mon fils, si tu reçois mes paroles, et que tu gardes précieusement en toi mes commandements,
\VS{2}si tu rends ton oreille attentive à la sagesse, et que tu inclines ton cœur à l'intelligence ;
\VS{3}si tu appelles à toi la sagesse, et que tu adresses ta voix à l'intelligence,
\VS{4}si tu la cherches comme de l'argent, et si tu la recherches soigneusement comme des trésors,
\VS{5}alors tu connaîtras la crainte de Yahweh, et tu trouveras la connaissance de Dieu.
\VS{6}Car Yahweh donne la sagesse, et de sa bouche procède la connaissance et l'intelligence.
\VS{7}Il réserve le salut pour ceux qui sont droits, et il est le bouclier de ceux qui marchent dans l'intégrité,
\VS{8}pour garder les sentiers de la justice ; il gardera la voie de ses bien-aimés.
\VS{9}Alors tu comprendras la justice, le jugement, l'équité, et tout bon chemin.
\VS{10}Si la sagesse vient dans ton coeur, et si la connaissance est agréable à ton âme ;
\VS{11}la réflexion veillera sur toi, et l'intelligence te gardera,
\VS{12}pour te délivrer du mauvais chemin, et de l'homme qui tient de mauvais discours ,
\VS{13}de ceux qui abandonnent les voies de la droiture pour marcher dans les chemins ténébreux,
\VS{14}qui sont joyeux de mal faire, et qui se réjouissent dans la perversité des méchants.
\VS{15}Eux dont les sentiers sont tortueux, et qui dans leur conduite vont de travers.
\VS{16}Afin qu'il te délivre de la femme étrangère\FTNT{La femme étrangère est la prostituée ou l'esprit de Jézabel qui séduit les hommes. Voir Pr. 6:24 ; Pr. 7:5.}, et de la femme d'autrui, dont les paroles sont flatteuses ;
\VS{17}qui abandonne l'ami de sa jeunesse et qui oublie l'alliance de son Dieu.
\VS{18}Car sa maison penche vers la mort, et son chemin mène vers les morts.
\VS{19}Pas un de ceux qui vont vers elle n'en retourne, ni ne reprend les sentiers de la vie.
\VS{20}Ainsi tu marcheras dans la voie des gens de bien, et tu garderas les sentiers des justes.
\VS{21}Car ceux qui sont droits habiteront la terre, les hommes intègres y demeureront.
\VS{22}Mais les méchants seront retranchés de la terre, et ceux qui agissent perfidement seront arrachés.
\Chap{3}
\TextTitle{La sagesse bénit et protège}
\VerseOne{}Mon fils, ne mets pas en oubli mon enseignement, et que ton cœur garde mes commandements.
\VS{2}Car ils t'apportent de longs jours et des années de vie et de paix.
\VS{3}Que la bonté et la vérité ne t'abandonnent pas : Lie-les à ton cou, et écris-les sur la table de ton coeur ;
\VS{4}et tu trouveras la grâce et la prudence au yeux de Dieu et des hommes.
\VS{5}Confie-toi de tout ton coeur en Yahweh et ne t'appuie point sur ton intelligence.
\VS{6}Considère-le dans toutes tes voies et il dirigera tes sentiers.
\VS{7}Ne sois point sage à tes yeux ; crains Yahweh, et détourne-toi du mal.
\VS{8}Ce sera la guérison de ton nombril et un rafraîchissement pour tes os.
\VS{9}Honore Yahweh avec tes biens et les prémices de tout ton revenu\FTNT{De. 12:6.} :
\VS{10}Alors tes greniers seront remplis d'abondance, et tes cuves regorgeront de vin nouveau.
\VS{11}Mon fils, ne rebute pas l'instruction de Yahweh, et ne te fâche pas de ce qu'il te reprend.
\VS{12}Car Yahweh châtie\FTNT{Hé. 12:4-11.} celui qu'il aime, comme un père le fils auquel il prend plaisir.
\VS{13}Heureux l'homme qui a trouvé la sagesse, et l'homme qui possède l'intelligence !
\VS{14}Car le trafic qu'on peut faire d'elle est meilleur que le trafic l'argent; et le profit qu'on en tire est meilleur que l'or fin.
\VS{15}Elle est plus précieuse que les perles, et toutes tes choses désirables ne la valent point.
\VS{16}Il y a de longs jours dans sa main droite, des richesses et de la gloire en sa gauche.
\VS{17}Ses voies sont des voies agréables, et tous ses sentiers ne sont que paix.
\VS{18}Elle est l'arbre de vie pour ceux qui l'embrassent ; et tous ceux qui la tiennent sont heureux\FTNT{Ge. 2:9 ; Ap. 22:2.}.
\VS{19}Yahweh a fondé la terre par la sagesse, il a disposé les cieux par l'intelligence.
\VS{20}C'est par sa science que les abîmes se sont ouverts, et que les nuages distillent la rosée.
\VS{21}Mon fils, que ces enseignements ne s'écartent point de devant tes yeux ; garde la sagesse et la réflexion :
\VS{22}Elles seront la vie de ton âme et l'ornement de ton cou.
\VS{23}Alors tu marcheras avec assurance dans ton chemin, et ton pied ne bronchera pas.
\VS{24}Si tu te couches, tu seras sans crainte, et quand tu seras couché ton sommeil sera doux.
\VS{25}Ne crains ni une terreur soudaine, ni la ruine des méchants, quand elle arrivera.
\VS{26}Car Yahweh sera ton assurance, et il gardera ton pied de toute embûche.
\VS{27}Ne retiens pas le bien à ceux à qui il est dû, quand il est au pouvoir de ta main de le faire\FTNT{Ga. 6:10.}.
\VS{28}Ne dis pas à ton prochain : Va, et reviens, demain je te donnerai ! Quand tu as de quoi donner.
\VS{29}Ne médite pas le mal contre ton prochain, lorsqu'il demeure tranquillement près de toi.
\VS{30}Ne conteste pas sans motif avec quelqu'un, à moins qu'il ne t'ait causé quelque tort\FTNT{Ro. 12:18.}.
\VS{31}Ne porte pas envie à l'homme violent, et ne choisis aucune de ses voies.
\VS{32}Car celui qui va de travers est en abomination à Yahweh ; mais son intimité est pour ceux qui sont justes.
\VS{33}La malédiction de Yahweh est dans la maison du méchant ; mais il bénit la demeure des justes.
\VS{34}Certes il se moque des moqueurs, mais il fait grâce à ceux qui s'humilient.
\VS{35}Les sages hériteront la gloire ; mais la honte élève des insensés.
\Chap{4}
\TextTitle{Instructions et conseils d'un père}
\VerseOne{}Ecoutez, mes fils, l'instruction du père, et soyez attentifs pour connaître l'intelligence.
\VS{2}Car je vous donne une bonne doctrine, ne rejetez donc pas mon enseignement.
\VS{3}J'ai été un fils pour mon père. Un fils tendre et unique auprès de ma mère.
\VS{4}Il m'a enseigné, et m'a dit : Que ton coeur retienne mes paroles ; garde mes commandements et tu vivras.
\VS{5}Acquiers la sagesse, acquiers l'intelligence ; n'oublie pas les paroles de ma bouche, et ne t'en détourne pas.
\VS{6}Ne l'abandonne point, et elle te gardera ; aime-la, et elle te protégera.
\VS{7}La principale chose c'est la sagesse ; donc acquiers la sagesse ; et sur toutes tes acuisitions, acquiers la prudence.
\VS{8}Exalte-la, et elle t'élèvera ; elle te glorifiera quand tu l'auras embrassée.
\VS{9}Elle posera sur ta tête une couronne de grâce, et elle t'ornera d'un magnifique diadème.
\VS{10}Ecoute, mon fils, et reçois mes paroles, ainsi les années de ta vie te seront multipliées.
\VS{11}Je t'ai enseigné le chemin de la sagesse, et je t'ai conduit dans les sentiers de la droiture\FTNT{Ps. 23:3.}.
\VS{12}Quand tu y marcheras, ton pas ne sera pas gêné ; et si tu cours, tu ne chancelleras pas\FTNT{Ps. 121:3.}.
\VS{13}Embrasse l'instruction, ne la lâche pas ; garde-la ; car elle est ta vie.
\VS{14}N'entre pas dans le sentier des méchants, et ne marche pas dans la voie des hommes mauvais.
\VS{15}Détourne-t'en, ne passe pas par là, détourne-t'en, et passe outre.
\VS{16}Car ils ne dormiraient pas, s'ils n'avaient fait quelque mal, et le sommeil leur serait ôté, s'ils n'avaient fait tomber quelqu'un.
\VS{17}Parce qu'ils mangent le pain de méchanceté, et qu'ils boivent le vin de la violence.
\VS{18}Mais le sentier des justes est comme la lumière resplendissante, dont l'éclat augmente jusqu'à ce que le jour soit dans sa perfection.
\VS{19}La voie des méchants est comme l'obscurité ; ils n'aperçoivent pas ce qui les fera tomber.
\VS{20}Mon fils, sois attentif à mes paroles, incline ton oreille à mes discours.
\VS{21}Qu'ils ne s'écarte pas de tes yeux ; garde-les dans le fond de ton coeur.
\VS{22}Car ils sont la vie pour ceux qui les trouvent, et la santé de tout le corps de chacun d'eux.
\VS{23}Garde ton coeur de tout ce dont il faut se garder ; car de lui procèdent les sources de la vie\FTNT{Mt 12:35 ; Mt. 15:18-19.}.
\VS{24}Eloigne de toi la perversité de la bouche et la dépravation des lèvres.
\VS{25}Que tes yeux regardent droit et que tes paupières dirigent ton chemin devant toi.
\VS{26}Pèse le chemin de tes pieds, et que toutes tes voies soient bien stables.
\VS{27}Ne te tourne ni à droite ni à gauche ; détourne ton pied du mal.
\Chap{5}
\TextTitle{[Se garder de l'immoralité]}
\VerseOne{}Mon fils, sois attentif à ma sagesse, incline ton oreille à mon intelligence ;
\VS{2}afin que tu gardes mes avis, et que tes lèvres conservent la connaissance.
\VS{3}Car les lèvres de l'étrangère distillent des rayons de miel, et son palais est plus doux que l'huile.
\VS{4}Mais ce qui en provient est amère comme de l'absinthe, et aigu comme une épée à deux tranchants.
\VS{5}Ses pieds descendent à la mort, ses pas atteignent le scheol.
\VS{6}Afin que tu ne balance pas sur le chemin de la vie car, ses chemins en sont écartés; tu ne le connaîtras pas.
\VS{7}Maintenant donc, fils, écoutez-moi, et ne vous détournez pas des paroles de ma bouche.
\VS{8}Eloigne ton chemin de la femme étrangère et n'approche pas de l'entrée de sa maison.
\VS{9}De peur que tu ne donnes ton honneur à d'autres, et tes années à un homme cruel.
\VS{10}De peur que les étrangers ne se rassasient de tes biens, et que le fruit de ton travail ne soit dans la maison d'un étranger.
\VS{11}De peur que tu ne gémisses quand tu seras près de ta fin, quand ta chair et ton corps seront consumés ;
\VS{12}et que tu ne dises : Comment donc ai-je pu haïr la correction, et comment mon coeur a-t-il dédaigné les réprimandes ?
\VS{13}Et comment n'ai-je point obéi à la voix de ceux qui m'instruisaient, et n'ai-je point incliné mon oreille à ceux qui m'enseignaient ?
\VS{14}Peu s'en est fallu que je n'aie été dans toute sorte de mal, au milieu du peuple et de l'assemblée.
\VS{15}Bois des eaux de ta citerne et de ce qui coule du milieu de ton puits ;
\VS{16}que tes sources se répandent dehors, et les ruisseaux d'eau sur les rues ;
\VS{17}qu'elles soient à toi seul, et non aux étrangers avec toi.
\VS{18}Que ta source soit bénie, et réjouis-toi de la femme de ta jeunesse,
\VS{19}comme d'une biche des amours, et d'une chevrette gracieuse ; que ses mamelles te rassasient en tout temps, et sois continuellement épris de son amour.
\VS{20}Et pourquoi, mon fils, irais-tu errant après l'étrangère et embrasserais-tu le sein de l'inconnue ?
\VS{21}Vu que les voies de l'homme sont devant les yeux de Yahweh et qu'il pèse toutes ses voies\FTNT{Jé 16:17 ; Hé. 4:13.}.
\VS{22}Les iniquités du méchant l'attraperont, et il sera retenu par les cordes de son péché.
\VS{23}Il mourra faute d'instruction et il s'égarera par l'excès de sa folie.
\Chap{6}
\TextTitle{Recommandations diverses}
\VerseOne{}Mon fils, si tu t'es porté caution pour ton prochain, si tu as engagé ta main pour un étranger,
\VS{2}tu es enlacé par les paroles de ta bouche, tu es pris par les paroles de ta bouche.
\VS{3}Mon fils, fais maintenant ceci, et dégage-toi, puisque tu es tombé entre les mains de ton intime ami, va, prosterne-toi, et importune tes amis.
\VS{4}Ne donne point de sommeil à tes yeux et ne laisse point sommeiller tes paupières.
\VS{5}Dégage-toi comme la gazelle de la main du chasseur, et comme l'oiseau de la main de l'oiseleur.
\VS{6}Va, paresseux, vers la fourmi, regarde ses voies, et sois sage.
\VS{7}Elle n'a ni chef, ni directeur, ni gouverneur,
\VS{8}et cependant elle prépare en été son pain, et amasse durant la moisson de quoi manger.
\VS{9}Paresseux, jusqu'à quand resteras-tu couché ? Quand te lèveras-tu de ton sommeil ?
\VS{10}Un peu de sommeil, dis-tu, un peu d'assoupissement, un peu croiser les mains afin de rester couché ;
\VS{11}et ta pauvreté viendra comme un voyageur, et ta disette comme un soldat.
\VS{12}Celui qui marche, la fausseté dans sa bouche, est un homme de Bélial\FTNT{1 S. 2:12.}, un homme inique.
\VS{13}Il cligne des yeux, parle du pied, enseigne de ses doigts.
\VS{14}Il y a la perversité dans son cœur, il machine du mal en tout temps, il fait naître des querelles.
\VS{15}C'est pourquoi sa calamité viendra subitement, il sera subitement brisé, il n'y aura point de guérison.
\VS{16}Il y a six choses que Yahweh hait, et il y en a sept qui sont en abomination à son âme ;
\VS{17}savoir, les yeux hautains\FTNT{Ps. 101:5.}, la langue mensongère\FTNT{Ps. 120:2-3.}, les mains qui répandent le sang innocent\FTNT{Es. 1:15.},
\VS{18}le coeur qui médite des projets iniques\FTNT{Ps. 36:5.}, les pieds qui se hâtent de courir au mal\FTNT{Es. 59:7.},
\VS{19}le faux témoin qui profère des mensonges\FTNT{Ps. 27:12.}, et celui qui sème des querelles entre les frères\FTNT{Jud. 1:16-19.}.
\VS{20}Mon fils, garde le commandement de ton père, et n'abandonne pas l'enseignement de ta mère ;
\VS{21}attache-les continuellement à ton coeur, lie-les à ton cou.
\VS{22}Quand tu marcheras, il te conduira ; et quand tu te coucheras, il te gardera ; et quand tu te réveilleras, il s'entretiendra avec toi.
\VS{23}Car le commandement est une lampe ; et l'enseignement une lumière\FTNT{Ps. 119:105.} ; et les réprimandes propres à instruire sont le chemin de la vie.
\VS{24}Ils te préserveront de la mauvaise femme, de la langue doucereuse de l'étrangère.
\VS{25}Ne la convoite pas dans ton coeur pour sa beauté, et ne te laisse pas prendre par ses yeux\FTNT{Mt. 5:28.}.
\VS{26}Car pour l'amour de la femme prostituée on est réduit à un morceau de pain, et la femme adultère chasse après l'âme précieuse de l'homme.
\VS{27}Un homme peut-il prendre du feu dans son sein, sans que ses habits brûlent ?
\VS{28}Un homme marchera-t-il sur des charbons ardents, sans que ses pieds en soient brûlés ?
\VS{29}Il en est de même pour celui qui va vers la femme de son prochain ; quiconque la touchera ne restera pas impuni.
\VS{30}On ne méprise pas un voleur, s'il vole pour satisfaire son âme quand il a faim ;
\VS{31}si on le trouve, il rendra sept fois autant, il donnera tout ce qu'il a dans sa maison.
\VS{32}Mais celui qui commet un adultère avec une femme est dépourvu de sens ; et celui qui le fera, détruira son âme.
\VS{33}Il trouvera des plaies et de l'ignominie, et son opprobre ne sera pas effacé.
\VS{34}Car la jalousie d'un mari est une fureur, il n'épargnera pas l'adultère au jour de la vengeance.
\VS{35}Il n'aura égard à aucune rançon, et il n'acceptera rien, quand même tu multiplierais les présents.
\Chap{7}
\TextTitle{Mise en garde contre la femme prostituée}
\VerseOne{}Mon fils, observe mes paroles, et garde avec toi mes commandements.
\VS{2}Garde mes commandements, et tu vivras, garde mes enseignements comme la prunelle de tes yeux\FTNT{Lé. 18:5.}.
\VS{3}Lie-les sur tes doigts, écris-les sur la table de ton coeur.
\VS{4}Dis à la sagesse : Tu es ma soeur ; et appelle l'intelligence, ton amie.
\VS{5}Afin qu'elles te préservent de la femme étrangère, de l'étrangère qui emploie des paroles doucereuses.
\VS{6}Comme je regardais de la fenêtre de ma maison à travers mon treillis,
\VS{7}je vis parmi les stupides, et je remarquai parmi les jeunes gens un jeune homme dépourvu de sens.
\VS{8}Il passait dans la rue, près de l'angle où se tenait une de ces femmes, et qui suivait le chemin de sa maison,
\VS{9}au crépuscule, au soir du jour, au milieu de la nuit et de l'obscurité.
\VS{10}Et voici, il fut abordé par une femme, vêtue en tenue de prostituée, et pleine de ruse dans le cœur.
\VS{11}Elle était bruyante et rebelle, ses pieds ne restaient point dans sa maison ;
\VS{12}tantôt dehors, tantôt sur les places, elle était aux aguets à chaque coin de rue.
\VS{13}Elle le saisit, et l'embrassa ; et avec un visage effronté, lui dit :
\VS{14}J'ai chez moi des sacrifices d'offrande de paix ; j'ai aujourd'hui accompli mes voeux.
\VS{15}C'est pourquoi je suis sortie à ta rencontre pour chercher ton visage, et je t'ai trouvé.
\VS{16}J'ai orné mon lit de couvertures, d'étoffes de fil d'Egypte.
\VS{17}J'ai parfumé ma couche de myrrhe, d'aloès et de cinnamome.
\VS{18}Viens, enivrons-nous de plaisir jusqu'au matin, réjouissons-nous en amours.
\VS{19}Car mon mari n'est point à la maison, il est parti pour un voyage lointain.
\VS{20}Il a pris un sac d'argent dans sa main, il ne reviendra à la maison qu'à la nouvelle lune.
\VS{21}Elle l'a fait détourner par beaucoup de douces paroles, et l'a attiré par la flatterie de ses lèvres.
\VS{22}Il s'en alla aussitôt après elle, comme un boeuf qui va à la boucherie, comme le fou qu'on lie pour être châtié ;
\VS{23}jusqu'à ce que la flèche lui ait transpercé le foie ; comme l'oiseau qui se hâte vers le filet, sans savoir que c'est au prix de sa vie.
\VS{24}Maintenant donc, fils, écoutez-moi, et soyez attentifs aux paroles de ma bouche.
\VS{25}Que ton coeur ne se détourne pas vers les voies d'une telle femme, ne t' égare pas dans ses sentiers.
\VS{26}Car elle a fait tomber plusieurs blessés à mort, et tous ceux qu'elle a tués sont nombreux.
\VS{27}Sa maison est le chemin du scheol, qui descend vers les demeures de la mort.
\Chap{8}
\TextTitle{[La sagesse préférable aux richesses]}
\VerseOne{}La sagesse ne crie-t-elle pas ? Et l'intelligence ne fait-elle pas entendre sa voix ?
\VS{2}Elle s'est présentée sur le sommet des lieux élevés, sur le chemin, aux carrefours.
\VS{3}Elle crie près des portes, devant la ville, à l'entrée des portes,
\VS{4}ô vous ! Hommes de qualité, je vous appelle ; et ma voix s'adresse aussi aux fils des hommes.
\VS{5}Vous stupides, apprenez le discernement, et vous tous, devenez intelligents de coeur.
\VS{6}Écoutez, car je dirai des choses importantes : Et j'ouvrirai mes lèvres pour enseigner des choses droites.
\VS{7}Parce que ma bouche proclame la vérité, et mes lèvres ont en horreur le mensonge.
\VS{8}Tous les discours de ma bouche sont selon la justice, il n'y a rien en eux de faux, ni de déformé.
\VS{9}Ils sont tous clairs à l'homme intelligent, et droits pour ceux qui ont trouvé la connaissance.
\VS{10}Recevez mon instruction plutôt que de l'argent, la connaissance à l'or le plus précieux.
\VS{11}Car la sagesse vaut mieux que les perles, et tout ce qu'on pourrait souhaiter ne la vaut pas\FTNT{Ps. 19:11 ; Ps. 119:127 ; Job. 28:18.}.
\VS{12}Moi, la Sagesse, j'habite avec le discernement, et je possède la connaissance de la réflexion.
\VS{13}La crainte de Yahweh c'est la haine du mal. Je hais l'orgueil et l'arrogance, la voie du mal, et la bouche perverse.
\VS{14}A moi appartiennent le conseil et le succès ; je suis l'intelligence, à moi appartient la force.
\VS{15}Par moi règnent les rois, et par moi les princes décrètent ce qui est juste.
\VS{16}Par moi gouvernent les seigneurs, les princes, et tous les juges de la terre.
\VS{17}J'aime ceux qui m'aiment ; et ceux qui me cherchent soigneusement me trouveront\FTNT{Mt. 7:7 ; Lu. 11:9 ; Jn 14:23-24.}.
\VS{18}Avec moi sont la richesse et la gloire, les biens durables et la justice.
\VS{19}Mon fruit est meilleur que le fin or, même que l'or raffiné ; et mon revenu est meilleur que l'argent choisi.
\VS{20}Je marche dans le chemin de la justice, au milieu des sentiers de la droiture ;
\VS{21}pour donner des biens en héritage à ceux qui m'aiment, et pour remplir leurs trésors.
\VS{22}Yahweh m'a acquise dès le commencement de ses voies, avant ses œuvres les plus anciennes.
\VS{23}J'ai été déclarée princesse depuis l'éternité, dès le commencement, avant l'origine de la terre.
\VS{24}J'ai été engendrée lorsqu'il n'y avait point encore d'abîmes, ni de sources chargées d'eaux.
\VS{25}Avant que les montagnes soient affermies, avant que les collines existent, j'ai été engendrée.
\VS{26}Lorsqu'il n'avait pas encore fait la terre et les campagnes, et le commencement de la poussière du monde habitable.
\VS{27}Lorsqu'il disposa les cieux, j'étais là ; lorsqu'il traça un cercle à la surface de l'abîme ;
\VS{28}lorsqu'il fixa les nuages en haut ; et que les sources de l'abîme jaillirent avec force ;
\VS{29}lorsqu'il donna une limite à la mer, pour que les eaux ne franchissent pas les bords ; lorsqu'il posa les fondements de la terre,
\VS{30}j'étais à l'œuvre auprès de lui, je faisais ses délices tous les jours, et toujours j'étais en joie en sa présence.
\VS{31}Je me réjouissais dans la partie habitable de sa terre, trouvant mes délices avec les fils de l'homme.
\VS{32}Maintenant donc, mes fils, écoutez-moi : Heureux sont ceux qui observent mes voies.
\VS{33}Ecoutez l'instruction, et soyez sages, et ne la rejetez point.
\VS{34}Ô ! Heureux est l'homme qui m'écoute, qui veille chaque jour à mes portes. Et qui monte la garde aux montants de mes portes !
\VS{35}Car celui qui me trouve a trouvé la vie, et obtient la faveur de Yahweh.
\VS{36}Mais celui qui pèche contre moi nuit à son âme ; tous ceux qui me haïssent aiment la mort.
\Chap{9}
\TextTitle{[La sagesse, source de vie]}
\VerseOne{}La Souveraine Sagesse a bâti sa maison, elle a taillé ses sept colonnes.
\VS{2}Elle a apprêté sa viande, elle a mêlé son vin ; elle a aussi dressé sa table.
\VS{3}Elle a envoyé ses servantes, elle crie du haut des lieux les plus élevés de la ville, disant : 
\VS{4}Que celui qui est stupide, entre ici ; et elle dit à ceux qui sont dépourvus de sens :
\VS{5}Venez, mangez de mon pain, et buvez du vin que j'ai mêlé.
\VS{6}Abandonnez la stupidité, et vous vivrez ; et marchez droit dans la voie de l'intelligence.
\VS{7}Celui qui instruit le moqueur, en reçoit de l'ignominie ; et celui qui reprend le méchant en reçoit une tache.
\VS{8}Ne reprends point le moqueur, de crainte qu'il ne te haïsse ; reprends le sage, et il t'aimera\FTNT{Ps. 141:5.}.
\VS{9}Donne l'instruction au sage, et il deviendra encore plus sage ; enseigne le juste, et il croîtra en science.
\VS{10}Le commencement de la sagesse est la crainte de Yahweh\FTNT{Ps. 19:10.} ; et la connaissance des saints, c'est l'intelligence.
\VS{11}Car tes jours se multiplieront par moi, et les années de vie augmenteront.
\VS{12}Si tu es sage, tu es sage pour toi-même ; si tu es moqueur, tu en porteras seul la peine.
\VS{13}La femme folle est bruyante, stupide et elle ne connaît rien.
\VS{14}Et elle s'assied à la porte de sa maison sur un siège, dans les lieux élevés de la ville ;
\VS{15}pour appeler les passants qui vont droit leur chemin, disant :
\VS{16}Que celui qui est stupide entre ici ! Elle dit à celui qui est dépourvu de sens :
\VS{17}Les eaux dérobées sont douces, et le pain pris en secret est agréable.
\VS{18}Et il ne sait pas que là sont les défunts, et que ceux qu'elle a conviés sont dans le scheol.
\Chap{10}
\TextTitle{[La justice s'oppose à la méchanceté]}
\VerseOne{}Proverbes de Salomon. Le fils sage réjouit son père, mais le fils insensé est l'ennui de sa mère.
\VS{2}Les trésors de méchanceté ne profitent pas, mais la justice délivre de la mort.
\VS{3}Yahweh ne laisse pas l'âme du juste avoir faim, mais il repousse au loin l'avidité des méchants.
\VS{4}Celui qui agit d'une main nonchalante s'appauvrit, mais la main des diligents enrichit.
\VS{5}L'enfant prudent amasse en été, mais celui qui dort durant la moisson est un enfant qui fait honte.
\VS{6}Les bénédictions seront sur la tête du juste, mais la violence couvrira la bouche des méchants.
\VS{7}La mémoire du juste est en bénédiction\FTNT{Ps 112:6.}, mais la réputation des méchants tombe en pourriture.
\VS{8}Celui qui est sage de coeur reçoit les commandements, mais celui qui est insensé des lèvres, tombera.
\VS{9}Celui qui marche dans l'intégrité marche avec assurance, mais celui qui pervertit ses voies, sera connu.
\VS{10}Celui qui cligne de l'oeil cause du chagrin, et celui qui a les lèvres insensées sera renversé.
\VS{11}La bouche du juste est une source de vie, mais la cruauté couvre la bouche des méchants.
\VS{12}La haine excite les querelles, mais la charité couvre toutes les fautes\FTNT{1 Pi 4:8.}.
\VS{13}La sagesse se trouve sur les lèvres de l'homme intelligent, mais la verge est pour le dos de celui qui est dépourvu de sens.
\VS{14}Les sages tiennent la connaissance en réserve, mais la bouche de l'insensé est une ruine prochaine.
\VS{15}Les biens du riche sont la ville de sa force, mais la pauvreté des misérables est leur ruine.
\VS{16}L'oeuvre du juste est pour la vie, mais le revenu du méchant est pour le péché.
\VS{17}Celui qui garde l'instruction est dans le chemin de la vie, mais celui qui néglige la correction s'y égare.
\VS{18}Celui qui dissimule la haine a des lèvres menteuses, et celui qui répand la calomnie est un insensé.
\VS{19}Dans la multitude de paroles le péché ne manque pas, mais celui qui retient ses lèvres est prudent.
\VS{20}La langue du juste est un argent de choix, mais le coeur des méchants est bien peu de chose.
\VS{21}Les lèvres du juste en instruisent plusieurs, mais les insensés mourront faute de sens.
\VS{22}La bénédiction de Yahweh est celle qui enrichit, et il n'y ajoute aucune peine.
\VS{23}C'est comme un jeu à un insensé de pratiquer l'infamie, mais la sagesse appartient à l'homme intelligent.
\VS{24}Ce que redoute le méchant, c'est ce qui lui arrive ; mais Dieu accorde aux justes ce qu'ils désirent.
\VS{25}Comme le tourbillon passe, ainsi le méchant n'est plus ; mais le juste est un fondement perpétuel.
\VS{26}Ce qu'est le vinaigre aux dents et la fumée aux yeux, tel est le paresseux à ceux qui l'envoient.
\VS{27}La crainte de Yahweh augmente les jours, mais les années des méchants sont raccourcies.
\VS{28}L'espérance des justes n'est que joie, mais l'espérance des méchants périra.
\VS{29}La voie de Yahweh est le refuge de l'homme intègre, mais elle est la ruine pour ceux qui pratiquent l'iniquité.
\VS{30}Le juste ne sera jamais ébranlé, mais les méchants ne demeureront pas sur la terre.
\VS{31}La bouche du juste produit la sagesse, mais la langue perverse sera retranchée.
\VS{32}Les lèvres du juste connaissent ce qui est agréable ; mais la bouche des méchants n'est que perversité.
\Chap{11}
\TextTitle{[La justice s'oppose à la méchanceté (suite)]}
\VerseOne{}La fausse balance est une abomination à Yahweh, mais le poids juste lui est agréable\FTNT{Lé 19:35-36 ; De. 25:13-16.}.
\VS{2}Quand l'orgueil vient, la honte vient aussi ; mais la sagesse est avec ceux qui sont modestes.
\VS{3}L'intégrité des hommes droits les conduit, mais la perversité des perfides les détruit.
\VS{4}Les richesses ne servent à rien au jour de la colère, mais la justice délivre de la mort.
\VS{5}La justice de l'homme intègre rend droite sa voie, mais le méchant tombe par sa méchanceté.
\VS{6}La justice des hommes droits les délivre, mais les perfides sont pris par leur méchanceté.
\VS{7}Quand l'homme méchant meurt, son espoir périt ; et l'espérance des hommes iniques périt.
\VS{8}Le juste est délivré de la détresse, et le méchant y entre à sa place.
\VS{9}Par sa bouche l'impie corrompt son prochain, mais les justes en sont délivrés par la connaissance.
\VS{10}La ville se réjouit quand les justes sont heureux, et quand les méchants périssent, c'est un triomphe.
\VS{11}La ville est élevée par la bénédiction des hommes droits, mais elle est renversée par la bouche des méchants.
\VS{12}Celui qui méprise son prochain est dépourvu de sens, mais l'homme prudent se tait.
\VS{13}Celui qui va rapportant, révèle les secrets, mais celui qui a l'esprit qui supporte les paroles, les couvre.
\VS{14}Le peuple tombe par faute de prudence, mais la délivrance est dans la multitude de conseillers.
\VS{15}Celui qui se porte garant pour un étranger en souffrira, et celui qui hait le cautionnement est assuré.
\VS{16}La femme gracieuse obtient de l'honneur, et les hommes robustes obtiennent les richesses.
\VS{17}L'homme doux fait du bien à son âme, mais le cruel trouble sa chair.
\VS{18}Le méchant fait une oeuvre qui le trompe, mais la récompense est assurée à celui qui sème la justice\FTNT{Os. 10:12.}.
\VS{19}Ainsi la justice conduit à la vie, mais celui qui poursuit le mal aboutit à sa mort.
\VS{20}Ceux qui ont le cœur pervers sont en abomination à Yahweh, mais ceux qui sont intègres dans leurs voies lui sont agréables.
\VS{21}De main en main le méchant ne demeurera point impuni, mais la race des justes sera délivrée.
\VS{22}Une belle femme qui se détourne de la raison est comme un anneau d'or au nez d'un pourceau.
\VS{23}Le souhait des justes n'est que le bien, mais l'attente des méchants c'est l'indignation.
\VS{24}Tel, qui donne libéralement, devient plus riche ; et tel qui épargne à l'excès ne fait que s'appauvrir.
\VS{25}Celui qui bénit sera engraisssé ; et celui qui arrose abondamment sera lui-même arrosé.
\VS{26}Sera maudit du peuple, celui qui cache le froment, mais la bénédiction est sur la tête de celui qui le vend.
\VS{27}Qui recherche le bien cherche la faveur, mais le mal arrive à qui le recherche.
\VS{28}Celui qui se confie dans ses richesses tombera, mais les justes verdiront comme le feuillage\FTNT{Ps. 1:3 ; Jé 17: 8.}.
\VS{29}Celui qui ne gouverne pas sa maison avec ordre, aura le vent pour héritage, et le fou sera le serviteur de celui qui a le coeur sage.
\VS{30}Le fruit du juste est un arbre de vie, et celui qui gagne les âmes est sage.
\VS{31}Voici, le juste reçoit sur la terre sa rétribution, combien plus le méchant et le pécheur la recevront-ils ?
\Chap{12}
\TextTitle{[La justice s'oppose à la méchanceté (suite)]}
\VerseOne{}Celui qui aime la correction aime la connaissance, mais celui qui hait la réprimande est un stupide.
\VS{2}L'homme de bien obtient la faveur de Yahweh, mais Yahweh condamne l'homme qui a des mauvaises pensées.
\VS{3}L'homme ne sera point affermi par la méchanceté, mais la racine des justes ne sera point ébranlée.
\VS{4}La femme vertueuse est la couronne de son mari\FTNT{Pr. 31:10.}, mais celle qui fait honte est comme la pourriture dans ses os.
\VS{5}Les pensées des justes ne sont que jugement, mais les conseils des méchants ne sont que fraude.
\VS{6}Les paroles des méchants ne tendent qu'à dresser des embûches pour répandre le sang, mais la bouche des hommes droits les délivrera.
\VS{7}Les méchants sont renversés, et ils ne sont plus, mais la maison des justes se maintiendra.
\VS{8}L'homme est estimé en raison de sa prudence, mais celui qui a le coeur pervers est l'objet du mépris.
\VS{9}Mieux vaut l'homme qui ne fait pas cas de lui-même, bien qu'il ait des serviteurs, que celui qui se glorifie, et qui manque de pain.
\VS{10}Le juste a égard à la vie de sa bête, mais les entrailles des méchants sont cruelles.
\VS{11}Celui qui cultive son champ sera rassasié de pain, mais celui qui court après des futilités est dépourvu de sens.
\VS{12}Ce que le méchant désire, est un filet des hommes mauvais, mais la racine des justes donnera son fruit.
\VS{13}Il y a dans le péché des lèvres un piège pernicieux, mais le juste sortira de la détresse.
\VS{14}L'homme sera rassasié de biens par le fruit de sa bouche, et on rendra à l'homme la rétribution de ses mains.
\VS{15}La voie de l'insensé est droite à son opinion, mais celui qui écoute le conseil est sage.
\VS{16}Quand à l'insensé, sa colère est révélée le jour même, mais l'homme bien avisé couvre son ignominie.
\VS{17}Celui qui prononce des choses véritables rend un témoignage juste, mais le faux témoin fait des rapports trompeurs.
\VS{18}Il y a tel homme dont les paroles blessent comme des pointes d'épée, mais la langue des sages apporte la guérison.
\VS{19}La lèvre véridique est affermie pour toujours, mais la fausse langue n'est que pour un moment\FTNT{Ps. 52: 6-7.}.
\VS{20}Il y a de la tromperie dans le coeur de ceux qui méditent le mal, mais il y a de la joie pour ceux qui conseillent la paix.
\VS{21}Il n'arrivera aucun outrage aux justes, mais les méchants seront remplis de mal.
\VS{22}Les fausses lèvres sont une abomination à Yahweh\FTNT{Ap. 22:15.}, mais ceux qui agissent fidèlement lui sont agréables.
\VS{23}L'homme bien avisé cache sa connaissance, mais le coeur des insensés publie la folie.
\VS{24}La main des diligents dominera, mais la main paresseuse sera tributaire.
\VS{25}Le chagrin qui est au cœur de l'homme, l'accable ; mais la bonne parole le réjouit.
\VS{26}Le juste a plus de reste que son voisin, mais la voie des méchants les égare.
\VS{27}L'homme paresseux ne rôtit point son gibier ; mais les biens précieux de l'homme sont au diligent.
\VS{28}La vie est dans le chemin de la justice, et la voie de son sentier ne tend point à la mort.
\Chap{13}
\TextTitle{[La justice s'oppose à la méchanceté (suite)]}
\VerseOne{}Un fils sage écoute l'instruction de son père, mais le moqueur n'écoute pas la réprimande\FTNT{Ps. 1:1.}.
\VS{2}L'homme mange du bien par le fruit de sa bouche, mais l'âme de ceux qui agissent perfidement mangent l'injustice.
\VS{3}Celui qui garde sa bouche, garde son âme ; mais celui qui ouvre à tout propos ses lèvres, tombera en ruine\FTNT{Ps. 39:2.}.
\VS{4}L'âme du paresseux a des désirs qu'il ne peut satisfaire, mais l'âme des diligents sera engraissée.
\VS{5}Le juste hait la parole mensongère, mais elle rend le méchant odieux et le fait tomber dans la confusion.
\VS{6}La justice garde celui qui est intègre dans sa voie, mais la méchanceté renversera celui qui s'égare.
\VS{7}Tel fait le riche et n'a rien du tout, tel fait le pauvre et a de grandes fortunes.
\VS{8}Les richesses d'un homme servent de rançon pour sa vie, mais le pauvre n'entend pas des réprimandes.
\VS{9}La lumière des justes remplit de joie, mais la lampe des méchants s'éteint.
\VS{10}L'orgueil ne produit que querelle, mais la sagesse est avec ceux qui écoutent les conseils.
\VS{11}Les richesses provenues de la fraude seront diminuées, mais celui qui amasse peu à peu les augmentera.
\VS{12}Un espoir différé fait languir le cœur, mais un désir accompli est comme un arbre de vie.
\VS{13}Celui qui méprise la parole périra à cause d'elle, mais celui qui craint le commandement en sera récompensé.
\VS{14}L'enseignement du sage est une source de vie, pour se détourner des pièges de la mort.
\VS{15}Le bon sens donne de la grâce ; mais la voie de ceux qui agissent perfidement est raboteuse.
\VS{16}Tout homme bien avisé agira avec connaissance, mais l'insensé fera l'étalage de sa folie\FTNT{Da.11:32}.
\VS{17}Le méchant messager tombe dans le mal, mais l'ambassadeur fidèle apporte la guérison.
\VS{18}La pauvreté et l'ignominie arrivent à celui qui rejette l'instruction, mais celui qui garde la réprimande est honoré.
\VS{19}Le souhait accompli est une chose douce à l'âme, mais se détourner du mal est une abomination aux insensés.
\VS{20}Celui qui marche avec les sages deviendra sage, mais le compagnon des insensés sera accablé.
\VS{21}Le mal poursuit les pécheurs, mais le bien sera rendu aux justes.
\VS{22}L'homme de bien laissera de quoi hériter aux fils de ses fils, mais les richesses du pécheur sont réservées aux justes.
\VS{23}Il y a beaucoup à manger dans les terres défrichées des pauvres, mais il y a tel qui est consumé faute de règles.
\VS{24}Celui qui épargne sa verge hait son fils, mais celui qui l'aime se hâte de le châtier.
\VS{25}Le juste mangera jusqu'à être rassasié à son souhait, mais le ventre des méchants aura la disette.
\Chap{14}
\TextTitle{[La justice s'oppose à la méchanceté (suite)]}
\VerseOne{}Toute femme sage bâtit sa maison, mais la folle la ruine de ses mains.
\VS{2}Celui qui marche dans la droiture craint Yahweh, mais celui dont les voies sont perverses le méprise.
\VS{3}La verge d'orgueil est dans la bouche de l'insensé, mais les lèvres des sages les garderont.
\VS{4}Où il n'y a point de boeuf, la grange est vide ; et l'abondance du revenu provient de la force du boeuf.
\VS{5}Le témoin véritable ne ment jamais, mais le faux témoin avance volontiers des mensonges.
\VS{6}Le moqueur cherche la sagesse et ne la trouve pas, mais la connaissance est aisée à trouver pour l'homme intelligent.
\VS{7}Eloigne-toi de l'homme insensé, puisque tu n'as pas trouvé sur ses lèvres la connaissance.
\VS{8}La sagesse d'un homme avisé est de connaître les règles de sa voie, mais la folie des insensés est la tromperie.
\VS{9}Les insensés se moquent du péché, mais parmi les hommes droits se trouve la bienveillance.
\VS{10}Le cœur d'un chacun connaît l'amertume de son âme, et un autre ne saurait partager sa joie.
\VS{11}La maison des méchants sera abolie, mais la tente des hommes droits fleurira.
\VS{12}Il y a telle voie qui semble droite à l'homme, mais dont l'issue sont les voies de la mort.
\VS{13}Même en riant le coeur sera triste, et la joie finit par l'ennui.
\VS{14}Celui qui a un cœur hypocrite, sera rassasié de ses voies ; mais l'homme de bien de ce qui est en lui.
\VS{15}Le simple croit à toute parole ; mais l'homme bien avisé considère ses pas.
\VS{16}Le sage craint et se retire du mal, mais l'insensé se met en colère et est confiant.
\VS{17}Celui qui est prompt à la colère agit follement\FTNT{Ps. 37:8.}, et l'homme plein de ruse est haï.
\VS{18}Les naïfs hériteront la folie ; mais les prudents seront couronnés de connaissance.
\VS{19}Les malins seront humiliés devant les bons, et les méchants, devant les portes du juste.
\VS{20}Le pauvre est haï même de son ami, mais les amis du riche sont en grand nombre.
\VS{21}Celui qui méprise son prochain commet un péché, mais celui qui a pitié des pauvres affligés est heureux.
\VS{22}Ceux qui méditent le mal ne s'égarent-ils pas ? Mais la bonté et la vérité sont pour ceux qui méditent le bien.
\VS{23}En tout travail il y a quelque profit, mais les vains discours ne tournent qu'à la disette.
\VS{24}Les richesses des sages leur sont comme une couronne, mais la stupidité des insensés est toujours stupidité.
\VS{25}Le témoin fidèle délivre les âmes, mais celui qui prononce des mensonges est trompeur.
\VS{26}En la crainte de Yahweh il y a une ferme assurance, et une retraite pour ses fils.
\VS{27}La crainte de Yahweh est une source de vie pour se détourner des pièges de la mort.
\VS{28}La gloire d'un roi, c'est la multitude du peuple, mais quand le peuple manque, c'est la ruine du prince.
\VS{29}Celui qui est lent à la colère a une grande intelligence, mais celui qui est prompt à s'emporter excite la folie.
\VS{30}Un coeur sain est la vie de la chair, mais l'envie est la pourriture des os.
\VS{31}Celui qui fait tort au pauvre déshonore celui qui l'a fait, mais celui qui a pitié de l'indigent honore Yahweh\FTNT{De. 24:11 ; Ps. 107:41.}.
\VS{32}Le méchant est chassé par sa malice, mais le juste trouve un refuge même dans sa mort.
\VS{33}La sagesse repose au coeur de l'homme intelligent, et elle est même reconnue au milieu des insensés.
\VS{34}La justice élève une nation, mais le péché est l'ignominie des peuples.
\VS{35}Le roi prend plaisir au serviteur prudent, mais son indignation sera contre celui qui lui fait honte.
\Chap{15}
\TextTitle{[La justice s'oppose à la méchanceté (suite)]}
\VerseOne{}La réponse douce apaise la fureur ; mais la parole douloureuse excite la colère
\VS{2}La langue des sages se réjouit de la connaissance, mais la bouche des insensés profère la sottise.
\VS{3}Les yeux de Yahweh sont en tous lieux, observant les méchants et les bons.
\VS{4}La langue qui corrige le prochain est comme l'arbre de vie, mais celle où il y a de la perversité est comme une brèche dans l'esprit.
\VS{5}L'insensé méprise l'instruction de son père, mais celui qui prend garde à la réprimande agit avec prudence.
\VS{6}Il y a un grand trésor dans la maison du juste, mais il y a du trouble dans les revenus du méchant.
\VS{7}Les lèvres des sages répandent partout la connaissance, mais le coeur des insensés ne fait pas ainsi.
\VS{8}Le sacrifice des méchants est en abomination à Yahweh, mais la requête des hommes droits lui est agréable.
\VS{9}La voie du méchant est en abomination à Yahweh, mais il aime celui qui poursuit soigneusement la justice.
\VS{10}Le châtiment est fâcheux à celui qui quitte le droit chemin, mais celui qui hait d'être repris, mourra.
\VS{11}Le schéol et le gouffre sont devant Yahweh ; combien plus les coeurs des fils des hommes !
\VS{12}Le moqueur n'aime pas qu'on le reprenne, et il ne va pas vers les sages.
\VS{13}Le cœur joyeux rend le visage beau, mais l'esprit est abattu par l'ennui du cœur.
\VS{14}Le cœur de l'homme prudent cherche la science ; mais la bouche des insensés se repaît de folie.
\VS{15}Tous les jours de l'affligé sont mauvais, mais quand on a le coeur gai, c'est un festin perpétuel.
\VS{16}Un peu de bien vaut mieux avec la crainte de Yahweh, qu'un grand trésor avec lequel il y a du trouble\FTNT{Ps. 37:16.}.
\VS{17}Mieux vaut un repas d'herbes où il y a de l'amitié, qu'un repas de boeuf bien gras où il y a de la haine.
\VS{18}L'homme furieux excite la querelle, mais l'homme lent à la colère apaise la dispute.
\VS{19}La voie du paresseux est comme une haie d'épines, mais le chemin des hommes droits est aplani.
\VS{20}Un fils sage réjouit le père, et un homme insensé méprise sa mère.
\VS{21}La stupidité est la joie de celui qui est dépourvu de sens, mais un homme prudent dresse ses pas au chemin de la droiture.
\VS{22} Les résolutions deviennent inutiles où il n'y a point de conseil ; mais il y a de la fermeté dans la multitude des conseillers.
\VS{23}L'homme a de la joie dans les réponses de sa bouche ; et combien est bonne une parole dite en son temps !
\VS{24}Le chemin de la vie élève l'homme prudent, afin qu'il se détourne du scheol qui est en bas.
\VS{25}Yahweh renverse la maison des orgueilleux, mais il affermit la borne de la veuve.
\VS{26}Les pensées du malin sont en abomination à Yahweh, mais celles de ceux qui sont purs sont des paroles agréables à ses yeux.
\VS{27}Celui qui est entièrement adonné au gain déshonnête trouble sa maison, mais celui qui hait les présents vivra.
\VS{28}Le coeur du juste médite ce qu'il doit répondre, mais la bouche des méchants profère des choses mauvaises.
\VS{29}Yahweh est loin des méchants, mais il exauce la requête des justes.
\VS{30}La clarté des yeux réjouit le coeur ; et la bonne renommée fortifie les os.
\VS{31}L'oreille qui écoute la correction qui donne la vie habite parmi les sages.
\VS{32}Celui qui rejette l'instruction a en dédain son âme, mais celui qui écoute la réprimande s'acquiert du sens.
\VS{33}La crainte de Yahweh enseigne la sagesse, et l'humilité précède la gloire\FTNT{Ps. 19:10.}.
\Chap{16}
\TextTitle{[La justice s'oppose à la méchanceté (suite)]}
\VerseOne{}Les préparations du cœur sont à l'homme, mais le discours réponse de la langue est de par Yahweh.
\VS{2}Chacune des voies de l'homme lui semble pure à ses yeux; mais Yahweh pèse les esprits.
\VS{3}Recommande tes affaires à Yahweh, et tes pensées seront bien ordonnées.
\VS{4}Yahweh a fait toutes choses pour lui-même ; et même le méchant pour le jour de l'affliction.
\VS{5}Yahweh a en abomination tout homme hautain de coeur ; assurément, il ne demeurera pas impuni.
\VS{6}Il y aura propitiation de l'iniquité par la miséricorde et la vérité ; on se détourne du mal par la crainte de Yahweh. 
\VS{7}Quand Yahweh prend plaisir aux voies d'un homme, il apaise\FTNT{Apaiser vient de shalom qui signifie : être dans une alliance de paix, être en paix, apaiser, vivre dans la paix etc.} envers lui même ses ennemis.
\VS{8}Il vaut mieux un peu de bien avec justice, qu'un gros revenu là où on n'a pas de droit.
\VS{9}Le cœur de l'homme médite sur sa voie, mais Yahweh conduit ses pas.
\VS{10}La divination est sur les lèvres du roi : Sa bouche ne doit pas s'égarer du droit.
\VS{11}La balance et le poids justes sont à Yahweh, tous les poids du sachet sont aussi son oeuvre.
\VS{12}Commettre une injustice doit être en abomination aux rois, parce que le trône est affermi par la justice.
\VS{13}Les rois doivent prendre plaisir aux lèvres de justice, et aimer celui qui profère des paroles justes.
\VS{14}Ce sont autant de messagers de mort que la colère du roi, mais l'homme sage l'apaisera.
\VS{15}Le visage serein du roi c'est la vie, et sa faveur est comme la nuée portant la pluie de la dernière saison.
\VS{16}Combien est-il plus précieux que l'or fin, d'acquérir de la sagesse! Et combien est-il plus excellent que l'argent, d'acquérir de la prudence ! 
\VS{17}Le chemin aplani des hommes droits, c'est de se détourner du mal ; celui qui prend garde de sa voie garde son âme.
\VS{18}L'orgueil va devant l'écrasement, et la fierté d'esprit devant la ruine.
\VS{19}Mieux vaut être humilié d'esprit avec les débonnaires, que de partager le butin avec les orgueilleux.
\VS{20}Celui qui prend garde à la parole trouvera le bien, et celui qui se confie en Yahweh est heureux\FTNT{Ps. 2:12.}.
\VS{21}On appellera prudent le sage de cœur, et la douceur des lèvres augmente l'instruction.
\VS{22}La prudence est à ceux qui la possèdent une source de vie ; mais le l'instruction des fous c'est leur folie.
\VS{23}Celui qui est sage de coeur conduit prudemment sa bouche, et ajoute l'instruction sur ses lèvres.
\VS{24}Les paroles agréables sont des rayons de miel, douces à l'âme et santé pour les os.
\VS{25} II y a telle voie qui semble droite à l'homme, mais dont la fin sont les voies de la mort.
\VS{26}Celui qui travaille, travaille pour lui-même, parce que sa bouche se courbe devant lui\FTNT{Ec. 6:7.}.
\VS{27}L'homme méchant creuse le mal, et il y a comme un feu brûlant sur ses lèvres.
\VS{28}L'homme qui use de perversité sème des querelles, et le rapporteur divise les grands amis.
\VS{29}L'homme violent attire son compagnon et le fait marcher dans une voie qui n'est pas bonne.
\VS{30}Il fait signe des yeux pour méditer des choses perverses, et remuant ses lèvres il exécute le mal.
\VS{31}Les cheveux blancs sont une couronne d'honneur ; elle se trouvera dans la voie de la justice.
\VS{32}Celui qui est lent à la colère vaut mieux que l'homme fort, et celui qui est maître de son cœur, vaut mieux que celui qui prend des villes.
\VS{33}On jette le sort dans le pan de la robe, mais tout ce qui doit arriver est de part Yahweh.
\Chap{17}
\TextTitle{[La justice s'oppose à la méchanceté (suite)]}
\VerseOne{}Mieux vaut un morceau de pain sec là où il y a la paix, qu'une maison pleine de viandes, là où il y a des querelles.
\VS{2}Le serviteur prudent sera maître sur l'enfant qui fait honte, et il partagera l'héritage entre les frères.
\VS{3}Le creuset est pour éprouver l'argent, et le fourneau l'or ; mais Yahweh éprouve les coeurs\FTNT{Jé. 17:10 ; Mal. 3:3 ; Ps. 26:2.}.
\VS{4}L'homme mauvais est attentif à la lèvre trompeuse, et le menteur écoute la mauvaise langue.
\VS{5}Celui qui se moque du pauvre déshonore celui qui l'a fait ; et celui qui se réjouit de l'affliction ne demeurera pas impuni.
\VS{6}Les petits-fils sont la couronne des vieillards\FTNT{Ps. 127:3 ; Ps. 128:3.}, et les pères sont la gloire de leurs fils.
\VS{7}La parole distinguée ne convient pas à un fou ; combien moins aux principaux du peuple des paroles de mensonge!
\VS{8}Le présent est comme une pierre précieuse aux yeux de ceux qui y sont adonnés ; de quelque côté qu'ils se tournent, ils réussissent.
\VS{9}Celui qui couvre les fautes cherche l'amitié, mais celui qui rapporte la chose divise les plus grands amis.
\VS{1}La répréhension se fait mieux sentir sur l'homme prudent que cent coups au fou.
\VS{11}Le méchant ne cherche que rébellion, mais le messager cruel sera envoyé contre lui.
\VS{12}Que l'homme rencontre plutôt une ourse qui a perdu ses petits qu'un fou dans sa folie.
\VS{13}Le mal ne partira point de la maison de celui qui rend le mal pour le bien.
\VS{14}Le commencement d'une querelle est comme quand on lâche une l'eau; mais avant qu'on en vienne à la dispute, retire-toi.
\VS{15}Celui qui déclare juste le méchant et celui qui déclare méchant le juste, sont tous deux en abomination à Yahweh\FTNT{Ex. 23:7 ; Es. 5:23.}.
\VS{16}A quoi sert le prix dans la main du fou pour acheter la sagesse, vu qu'il n'a pas de sens?
\VS{17}L'ami intime aime en tout temps, et il naît comme un frère dans la détresse.
\VS{18}Celui là est dépourvu de sens qui touche à la main et se rend caution pour son ami.
\VS{19}Celui qui aime les querelles aime le péché ; celui qui élève sa porte cherche sa ruine.
\VS{20}Celui qui est pervers de coeur ne trouve pas le bien; et l'hypocrite tombe dans le malheur.
\VS{21}Celui qui engendre un sot en aura de l'ennui, et le père du sot ne se réjouira pas.
\VS{22}Le coeur joyeux est un remède, mais l'esprit abattu dessèche les os.
\VS{23}Le méchant rend les présents en secret, pour pervertir les voies du jugement.
\VS{24}La sagesse est en présence de l'homme prudent; mais les yeux du fou sont à l'extrémité de la terre.
\VS{25}Le fils fou est l'ennui de son père, et l'amertume de celle qui l'a enfanté.
\VS{26}Il n'est pas bon de condamner l'innocent à l'amende, ni que les principaux frappent quelqu'un pur avoir agi avec droiture.
\VS{27}L'homme retenu dans ses paroles sait ce qu'est la connaissance, et l'homme qui est d'un esprit calme est un homme intelligent.
\VS{28}Même le fou, quand il se tait, est réputé sage ; et celui qui ferme ses lèvres est réputé intelligent.
\Chap{18}
\TextTitle{[La justice s'oppose à la méchanceté (suite)]}
\VerseOne{}Celui qui se sépare cherche ce qui lui fait plaisir, et se mêle de savoir comment tout doit aller.
\VS{2}Le fou ne prend pas plaisir à l'intelligence, mais à ce que son cœur soit manifesté.
\VS{3}Quand le méchant vient, le mépris vient aussi, et le reproche avec l'ignominie.
\VS{4}Les paroles de la bouche d'un homme sont des eaux profondes ; et la source de la sagesse est un torrent qui bouillonne\FTNT{Jn. 4:14.}.
\VS{5}Il n'est pas bon d'avoir égard à l'apparence de la personne du méchant, pour renverser le juste en jugement.
\VS{6}La bouche du fou entrent en querelles, et sa bouche appelle les combats.
\VS{7}La bouche du fou lui est une ruine, et ses lèvres sont un piège à son âme.
\VS{8}Les paroles du flatteur sont de ceux qui font semblant d'y toucher ; mais elles pénètrent jusqu'au-dedans des entrailles.
\VS{9}Celui qui se relâche dans son ouvrage est frère de celui qui dissipe ce qu'il a.
\VS{10}Le Nom de Yahweh est une tour forte, le juste y court et y trouve une haute retraite.
\VS{11}Les biens du riche sont sa ville forte et comme une haute muraille de retraite, selon son imagination.
\VS{12}Le coeur de l'homme s'élève avant que la ruine arrive, mais l'humilité précède la gloire.
\VS{13}Celui qui répond à quelque propos avant de l'avoir entendu, agit en fou et s'attire le reproche.
\VS{14}L'esprit d'un homme fort soutiendra dans son infirmité ; mais l'esprit abattu, qui le relèvera ?
\VS{15}Le coeur de l'homme intelligent acquiert la connaissance, et l'oreille des sages cherche la connaissance.
\VS{16}Le présent d'un homme lui fait faire place, et le conduit devant les grands.
\VS{17}Celui qui plaide le premier paraît juste; mais sa partie adverse vient, et examine le tout.
\VS{18}Le sort fait cesser les procès et fait les partages entre les puissants.
\VS{19}Un frère offensé se rend plus difficile qu'une ville forte, et les discordes entre frères sont comme les verrous d'un palais.
\VS{20}Le ventre de chacun est rassasié du fruit de sa bouche, il se rassasie du revenu de ses lèvres.
\VS{21}La mort et la vie sont au pouvoir de la langue\FTNT{Mt. 12:37.}, et celui qui aime à parler mangera de ses fruits.
\VS{22}Celui qui trouve une femme vertueuse trouve le bonheur et il obtient une faveur de Yahweh.
\VS{23}Le pauvre ne prononce que des supplications, mais le riche ne répond que des paroles dures.
\VS{24}L'homme qui a des intimes amis se tiennent à leur amitié parce qu'il y a tel ami qui est plus attaché que le frère.
\Chap{19}
\TextTitle{[La justice s'oppose à la méchanceté (suite)]}
\VerseOne{}Le pauvre qui marche dans son intégrité, vaut mieux que celui qui pervertit ses lèvres et qui est fou.
\VS{2}La vie même sans connaissance n'est pas une bonne personne ; et celui qui hâte ses pas dans le péché, s'égare.
\VS{3}La folie de l'homme renverse son chemin ; et cependant, c'est contre Yahweh que son coeur s'irrite.
\VS{4}Les richesses attirent un grand nombre d'amis, mais celui qui est pauvre est abandonné même par son ami.
\VS{5}Le faux témoin ne restera pas impuni, et celui qui profère des mensonges n'échappera pas.
\VS{6}Plusieurs supplient celui qui est en état de faire du bien, et chacun est ami de celui qui donne.
\VS{7}Tous les frères du pauvre le haïssent ; combien plus ses amis se retirent-ils de lui ! Il les supplie, mais il n'y a que des paroles pour lui.
\VS{8}Celui qui acquiert du sens aime son âme, et celui qui prend garde à l'intelligence c'est pour trouve le bonheur.
\VS{9}Le faux témoin ne restera pas impuni, et celui qui profère des mensonges périra.
\VS{10}Il ne sied pas à un fou de vivre dans les délices ; combien moins sied-il à un esclave de dominer sur les personnes de distinction !
\VS{11}La prudence de l'homme retient à la colère ; c'est un honneur pour lui de passer par dessus le tort qu'on lui fait.
\VS{12}La colère du roi est comme le rugissement d'un jeune lion, mais sa faveur est comme la rosée sur l'herbe.
\VS{13}Un fils insensé est un grand malheur pour son père, et les querelles d'une femme sont une gouttière continuelle.
\VS{14}On peut hériter de ses pères une maison et des richesses, mais la femme prudente est un don de Yahweh.
\VS{15}La paresse fait venir le sommeil, et l'âme paresseuse a faim.
\VS{16}Celui qui garde le commandement garde son âme, mais celui qui méprise ses voies mourra.
\VS{17}Celui qui a pitié du pauvre prête à Yahweh, qui lui rendra son bienfait.
\VS{18}Châtie ton fils tandis qu'il y a de l'espérance, mais ne va pas jusqu'à le faire mourir.
\VS{19}Celui qui est de grande colère en porte la peine ; et si tu l'en retires, tu y ajoute davantage.
\VS{20}Ecoute le conseil et reçois l'instruction, afin que tu deviennes sage en ton dernier temps.
\VS{21}Il y a dans le cœur de l'homme plusieurs pensées, mais le conseil de Yahweh est\FTNT{Es. 46:10 ; Ps. 33:11.}.
\VS{22}Ce que l'homme doit désirer, c'est d'exercer la miséricorde ; et le pauvre vaut mieux qu'un menteur.
\VS{23}La crainte de Yahweh conduit à la vie, et celui qui l'a, passe la nuit étant rassasié, sans qu'il soit visité par aucun mal.
\VS{24}Le paresseux cache sa main dans le sein, et il ne daigne même pas la ramener à sa bouche.
\VS{25}Si tu bats le moqueur, le sot en rend garde ; et si tu reprends l'homme intelligent, il discernera ce qu'il faut savoir.
\VS{26}L'enfant qui fait honte et sème la confusion, détruit le père et met en fuite sa mère.
\VS{27}Mon fils, cesse d'écouter ce qui pourrait t'apprendre à te détourner des paroles de la connaissance.
\VS{28}Le témoin indigne\FTNT{Le mot « pervers » vient de l'hébreu « beliya'al » : « sans valeur », « vaurien » (Jg. 19:22 ; 1. S. 2:12). Bélial est aussi un autre nom de Satan (2 Co. 6:15).} se moque de la justice, et la bouche des méchants avale l'iniquité.
\VS{29}Les jugements sont préparés pour les moqueurs, et les grands coups pour le dos des fous.
\Chap{20}
\TextTitle{[La justice s'oppose à la méchanceté (suite)]}
\VerseOne{}Le vin est moqueur et les boissons fortes sont tumultueuses, quiconque en fait excès, n'est pas sage.
\VS{2}La terreur du roi est comme le rugissement d'un jeune lion, celui qui l'irrite pèche contre sa propre âme.
\VS{3}C'est une gloire à l'homme de s'abstenir des disputes, mais tout insensé s'y engage.
\VS{4}Le paresseux ne labourera pas à cause de l'hiver, lors de la moisson il mendiera et n'aura rien.
\VS{5}Les conseils dans le coeur d'un homme sage sont comme des eaux profondes, et l'homme intelligent sait y puiser.
\VS{6}Beaucoup de gens vantent leur bonté ; mais l'homme fidèle, qui le trouvera ?
\VS{7}Ô, que les fils du juste qui marchent dans son intégrité seront heureux après lui !
\VS{8}Le roi assis sur le trône de justice dissipe tout mal par son regard.
\VS{9}Qui est-ce qui peut dire : J'ai purifié mon cœur, je suis net de mon péché ?
\VS{10}Le double poids et la double mesure sont tous deux en abomination à Yahweh.
\VS{11}Un jeune enfant fait connaître par ses actions si son oeuvre sera pure et droite.
\VS{12}L'oreille qui entend et l'oeil qui voit, Yahweh les a faits tous les deux.
\VS{13}N'aime point le sommeil, de peur que tu ne deviennes pauvre ; ouvre tes yeux, et tu auras suffisamment de pain.
\VS{14}Il est mauvais, il est mauvais, dit l'acheteur ; puis il s'en va, et se vante.
\VS{15}Il y a de l'or et beaucoup de perles ; mais les lèvres qui gardent la connaissance sont un vase précieux.
\VS{16}Quand quelqu'un se porte garant pour l'étranger, prends son vêtement ; exige de lui des gages pour cet étranger.
\VS{17}Le pain acquis par la tromperie est doux à l'homme, mais ensuite sa bouche sera remplie de gravier.
\VS{18}Les projets s'affermissent par le conseil ; fais donc la guerre avec prudence.
\VS{19}Celui qui médit révèle les secrets ; ne te mêle donc pas avec celui qui séduit par ses lèvres.
\VS{20}Celui qui traite avec mépris son père ou sa mère, sa lampe s'éteindra au milieu des ténèbres les plus noires\FTNT{Ex. 21:17 ; Lé. 20:9 ; Mt. 15:4.}.
\VS{21}L'héritage pour lequel on s'est trop hâté dès l'origine, ne sera pas béni à la fin.
\VS{22}Ne dis point : Je rendrai le mal ; espère en Yahweh, et il te délivrera.
\VS{23}Le double poids est en horreur à Yahweh, et la balance fausse n'est pas une chose bonne.
\VS{24}Les pas de l'homme sont dirigés par Yahweh, comment donc l'homme comprendrait-il sa voie ?
\VS{25}C'est un piège à l'homme que prendre à la légère un engagement sacré, et de ne réfléchir qu'après avoir fait un vœu.
\VS{26}Un roi sage disperse les méchants et ramène la roue sur eux.
\VS{27}L'esprit de l'homme est une lampe de Yahweh, il pénètre jusqu'au fond des entrailles.
\VS{28}La bienveillance et la vérité protègent le roi, et il soutient son trône par la bienveillance.
\VS{29}La force est la gloire des jeunes gens, et les cheveux blancs sont l'honneur des vieillards.
\VS{30}Les meurtrissures et les plaies nettoient le mal, de même les coups qui pénètrent jusqu'au fond des entrailles.
\Chap{21}
\TextTitle{[La justice s'oppose à la méchanceté (suite)]}
\VerseOne{}Le coeur du roi est un courant d'eau dans la main de Yahweh ; il l'incline partout où il veut.
\VS{2}Toutes les voies de l'homme sont droites à ses yeux, mais c'est Yahweh qui pèse les coeurs.
\VS{3}Faire ce qui est juste et droit est une chose que Yahweh préfère aux sacrifices.
\VS{4}Des regards hautains et le coeur qui s'enfle sont la lampe des méchants, ce n'est que péché.
\VS{5}Les projets de l'homme diligent ne mènent qu'à l'abondance, mais celui qui agit avec précipitation ne court qu'à l'indigence.
\VS{6}Des trésors acquis par une langue mensongère, c'est une vanité qu'on ne peut retenir, un signe avant-coureur de la mort.
\VS{7}La violence des méchants les emporte, parce qu'ils refusent de faire ce qui est droit.
\VS{8}La voie d'un homme coupable est détournée, mais l'oeuvre de celui qui est innocent est droite.
\VS{9}Il vaut mieux habiter à l'angle d'un toit qu'avec une femme querelleuse dans une grande maison.
\VS{10}L'âme du méchant désire le mal, son prochain ne trouve pas de grâce à ses yeux.
\VS{11}Quand on punit le moqueur, le sot devient sage ; et quand on instruit le sage, il reçoit la connaissance.
\VS{12}Il y a un juste qui considère attentivement la maison du méchant, Yahweh renverse les méchants dans le malheur.
\VS{13}Celui qui bouche son oreille pour ne pas entendre le cri du pauvre, criera aussi lui-même, et on ne lui répondra point.
\VS{14}Un don fait en secret apaise la colère, et un présent fait en cachette calme une fureur violente.
\VS{15}C'est une joie pour le juste de pratiquer la justice, mais c'est la ruine pour les ouvriers d'iniquité.
\VS{16}L'homme qui s'écarte du chemin de la sagesse aura sa demeure dans l'assemblée des morts.
\VS{17}Celui qui aime les réjouissances reste dans l'indigence ; et celui qui aime le vin et l'huile ne s'enrichira pas.
\VS{18}Le méchant sert de rançon pour le juste, et le déloyal pour les hommes intègres.
\VS{19}Il vaut mieux habiter dans une terre déserte qu'avec une femme querelleuse et qui se dépite.
\VS{20}Des précieux trésors et l'huile sont dans la demeure du sage, mais l'homme insensé les engloutit.
\VS{21}Celui qui poursuit la justice et la bonté, trouve la vie, la justice et la gloire.
\VS{22}Le sage entre dans la ville des forts et il abat la force qui lui donnait de l'assurance.
\VS{23}Celui qui veille sur sa bouche et sur sa langue préserve son âme des angoisses.
\VS{24}On appelle moqueur un superbe arrogant, qui agit avec colère et orgueil.
\VS{25}Les désirs du paresseux le tuent, parce que ses mains refusent de travailler.
\VS{26}Tout le jour il désire avidement, mais le juste donne sans parcimonie.
\VS{27}Le sacrifice des méchants est une abomination ; combien plus quand ils l'apportent avec des mauvaises intentions\FTNT{1 S. 15:22.} ?
\VS{28}Le témoin menteur périra, mais l'homme qui écoute parlera avec gain de cause.
\VS{29}L'homme méchant prend un air effronté, mais l'homme droit règle sa conduite.
\VS{30}Il n'y a ni sagesse, ni intelligence, ni conseil, contre Yahweh.
\VS{31}Le cheval est équipé pour le jour de la bataille, mais la délivrance vient de Yahweh.
\Chap{22}
\TextTitle{[La justice s'oppose à la méchanceté (suite)]}
\VerseOne{}La renommée est préférable aux grandes richesses\FTNT{Ec.7:1.}, et la bonne grâce plus que l'argent et l'or.
\VS{2}Le riche et le pauvre se rencontrent ; celui qui les a faits l'un et l'autre, c'est Yahweh\FTNT{Lu. 16.}.
\VS{3}L'homme prudent voit le mal et se cache, mais les stupides passent et en portent la peine.
\VS{4}Les fruits de l'humilité et de la crainte de Yahweh sont les richesses, la gloire et la vie.
\VS{5}Il y a des épines et des pièges dans la voie de l'homme pervers ; celui qui aime son âme s'en retirera loin.
\VS{6}Instruis le jeune enfant selon la voie qu'il doit suivre, et quand il sera vieux, il ne s'en détournera pas.
\VS{7}Le riche domine sur les pauvres\FTNT{Ja. 2:6.}, et celui qui emprunte est l'esclave de celui qui prête.
\VS{8}Celui qui sème l'injustice moissonne le malheur\FTNT{Job. 4:8 ; Ga. 6:7.}, et la verge de sa fureur prendra fin.
\VS{9}Celui qui a l'oeil bienveillant sera béni, parce qu'il aura donné de son pain au pauvre.
\VS{10}Chasse le moqueur, et la querelle prendra fin ; les disputes et l'ignominie cesseront.
\VS{11}Le roi est ami de celui qui aime la pureté de coeur, et qui a de la grâce dans ses paroles.
\VS{12}Les yeux de Yahweh veillent sur la connaissance, mais il confond les paroles du perfide.
\VS{13}Le paresseux dit : Il y a un lion dehors ! Je serais tué dans les rues !
\VS{14}La bouche des courtisanes est une fosse profonde, celui contre qui Yahweh est irrité y tombera.
\VS{15}La folie est liée au coeur du jeune enfant, mais la verge de la correction l'éloignera de lui.
\VS{16}Celui qui fait tort au pauvre pour s'enrichir et pour donner au riche, ne peut manquer de tomber dans l'indigence.
\VS{17}Prête ton oreille et écoute les paroles des sages, et applique ton coeur à ma connaissance.
\VS{18}Car ce sera une chose agréable pour toi si tu les gardes au-dedans de toi, et qu'elles soient toutes présentes sur tes lèvres.
\VS{19}Je te les ai fait connaître à toi aujourd'hui, dis-je, afin que ta confiance soit en Yahweh.
\VS{20}N'ai-je pas déjà pour toi mis par écrit des choses qui conviennent à ceux qui gouvernent, des conseils et des réflexions,
\VS{21}pour te faire connaître la certitude des paroles vraies, afin que tu répondes par des paroles vraies à celui qui t'envoie ?
\VS{22}Ne dépouille pas le pauvre, parce qu'il est pauvre ; et n'opprime pas le malheureux à la porte.
\VS{23}Car Yahweh défendra leur cause et privera de la vie ceux qui les auront volés.
\VS{24}Ne fréquente pas quelqu'un de coléreux, ne va pas avec l'homme violent ;
\VS{25}de peur que tu n'apprennes ses manières, et qu'ils ne deviennent un piège pour ton âme.
\VS{26}Ne sois pas parmi ceux qui prennent des engagements ni de ceux qui cautionnent les dettes.
\VS{27}Si tu n'as pas de quoi payer, pourquoi prendrait-on ton lit de dessous toi ?
\VS{28}Ne déplace pas la borne ancienne, que tes pères ont posée.
\VS{29}As-tu vu un homme habile en son travail ? Il sera au service des rois, il ne se tiendra pas devant des gens obscurs.
\Chap{23}
\TextTitle{[La justice s'oppose à la méchanceté (suite)]}
\VerseOne{}Quand tu t'assieds pour manger avec un gouverneur, considère avec attention celui qui est devant toi.
\VS{2}Autrement tu te mettras le couteau à la gorge, si ton appétit te domine.
\VS{3}Ne convoite pas ses friandises, car c'est un pain trompeur.
\VS{4}Ne travaille pas en vue d'acquérir des richesses ; désiste-toi de la résolution que tu as prise.
\VS{5}Jetteras-tu tes yeux sur ce qui bientôt n'est plus ? Car certainement, il se fera des ailes, il s'envolera comme un aigle dans les cieux.
\VS{6}Ne mange pas le pain de celui dont le regard est envieux, et ne désire pas ses friandises.
\VS{7}Car il est tel qu'il pense dans son âme. Il te dira bien : Mange et bois, mais son coeur n'est pas avec toi.
\VS{8}Tu voudrais vomir le morceau que tu auras mangé, et tu auras perdu tes paroles agréables.
\VS{9}Ne parle pas aux oreilles de l'insensé, car il méprise le bon sens de ton discours.
\VS{10}Ne déplace pas la borne ancienne et n'entre pas dans les champs des orphelins :
\VS{11}Car leur vengeur est puissant, il défendra leur cause contre toi.
\VS{12}Applique ton coeur à l'instruction, et tes oreilles aux paroles de la connaissance.
\VS{13}Ne te retiens pas de corriger le jeune enfant ; quand tu l'auras frappé de la verge, il n'en mourra pas.
\VS{14}En le frappant de la verge, tu préserves son âme du scheol.
\VS{15}Mon fils, si ton coeur est sage, mon coeur s'en réjouira, oui, moi-même.
\VS{16}Certes, mes reins tressailliront de joie quand tes lèvres proféreront ce qui est droit.
\VS{17}Que ton coeur ne porte pas d'envie aux pécheurs, mais adonne-toi tout le jour à la crainte de Yahweh.
\VS{18}Car il y a véritablement un avenir, et ton espérance ne sera pas retranchée.
\VS{19}Toi, mon fils, écoute et sois sage, et dirige ton coeur dans la bonne voie.
\VS{20}Ne fréquente point les ivrognes ni les gourmands\FTNT{Ro. 13:13 ; Ep. 5:18 ; Ga. 5:18-21.}.
\VS{21}Car l'ivrogne et le gourmand s'appauvrissent ; et l'assoupissement fait porter des vêtements déchirés.
\VS{22}Ecoute ton père, c'est celui qui t'a engendré ; et ne méprise pas ta mère, quand elle est devenue vieille.
\VS{23}Acquiers la vérité, et ne la vends point, acquiers la sagesse, l'instruction et l'intelligence.
\VS{24}Le père du juste aura beaucoup de joie, et celui qui donne naissance à un sage se réjouira en lui.
\VS{25}Que ton père et ta mère se réjouissent, que celle qui t'a enfanté soit dans l'allégresse !
\VS{26}Mon fils, donne-moi ton coeur, et que tes yeux prennent plaisir à mes voies.
\VS{27}Car la femme prostituée est une fosse profonde, et la courtisane un puits de détresse.
\VS{28}Aussi se tient-elle en embûche comme un voleur, et elle augmente parmi les hommes le nombre des infidèles.
\VS{29}Pour qui les « ah » ? Pour qui les « malheur à moi ! » Pour qui les disputes ? Pour qui les plaintes ? Pour qui les blessures sans raison ? Pour qui les yeux rouges ?
\VS{30}Pour ceux qui s'attardent auprès du vin, pour ceux qui vont chercher des vins mélangés.
\VS{31}Ne regarde pas le vin parce qu'il est d'un beau rouge, qu'il donne son éclat dans la coupe, et qu'il coule aisément.
\VS{32}Il finit par mordre par derrière comme un serpent, et par piquer comme un basilic.
\VS{33}Ensuite tes yeux regarderont les femmes étrangères, et ton coeur parlera d'une manière perverse.
\VS{34}Tu seras comme un homme qui dort au milieu de la mer, et comme un homme couché sur le sommet d'un mât.
\VS{35}On m'a battu, diras-tu, et je n'en ai pas été malade, on m'a frappé, et je ne l'ai pas senti, quand me réveillerai-je ? Je me remettrai encore à chercher le vin.
\Chap{24}
\TextTitle{[La justice s'oppose à la méchanceté (suite)]}
\VerseOne{}N'envie pas les hommes qui font le mal, et ne désire pas être avec eux.
\VS{2}Car leur coeur médite la destruction, et leurs lèvres parlent d'iniquité.
\VS{3}C'est par la sagesse qu'une maison est bâtie, et par l'intelligence qu'elle s'affermit.
\VS{4}C'est par la connaissance que les chambres seront remplies de tous les biens précieux et agréables.
\VS{5}Un homme sage est accompagné de force, et celui qui a de la connaissance affermit sa vigueur.
\VS{6}Car avec de bonnes directives tu feras la guerre avantageusement, et le salut est dans le grand nombre des bons conseillers.
\VS{7}La sagesse est trop élevée pour l'insensé, il n'ouvrira pas sa bouche à la porte.
\VS{8}Celui qui médite de faire le mal s'appelle un homme plein de malice.
\VS{9}Le projet de la folie est un péché, et le moqueur est en abomination parmi les hommes.
\VS{10}Si tu perds courage au jour de la détresse, ta force n'est que détresse.
\VS{11}Ne te retiens pas de délivrer ceux qu'on traîne à la mort, ceux qu'on va égorger, agis pour qu'on les épargne !
\VS{12}Si tu dis : Ah ! nous n'en savions rien… Celui qui pèse les cœurs, lui, ne le considérera-t-il pas ? Celui qui garde ton âme, lui, le sait, et il rend à chacun selon son œuvre.
\VS{13}Mon fils, mange le miel, car il est bon ; un rayon de miel sera doux à ton palais.
\VS{14}Ainsi sera à ton âme la connaissance de la sagesse, quand tu l'auras trouvée ; il y a un avenir, et ton espérance ne sera pas anéantie.
\VS{15}Méchant, ne mets pas des embûches dans le domaine du juste, et ne dévaste pas le lieu où il se repose.
\VS{16}Car le juste tombera sept fois, et sera relevé\FTNT{Ps. 34:20. ; Job. 5:19.} ; mais les méchants trébuchent pour tomber dans le malheur.
\VS{17}Si ton ennemi tombe, ne t'en réjouis pas, et que ton coeur ne soit pas dans l'allégresse quand il chancelle,
\VS{18}de peur que Yahweh ne le voie et que cela ne lui déplaise, tellement qu'il détourne sa colère de dessus de lui sur toi.
\VS{19}Ne t'irrite pas à cause de ceux qui font le mal, n'envie pas les méchants,
\VS{20}car il n'y a pas d'avenir pour celui qui fait le mal, la lampe des méchants sera éteinte.
\VS{21}Mon fils, crains Yahweh et le roi ; et ne te mêle pas avec les gens agités.
\VS{22}Car leur ruine s'élèvera tout d'un coup ; et qui sait le malheur qui arrivera à l'un et à l'autre ?
\VS{23}Voici encore ce qui vient des sages : Il n'est pas bon d'avoir égard à l'apparence des personnes dans le jugement.
\VS{24}Celui qui dit au méchant : Tu es juste ! Les peuples le maudiront, et les nations seront indignées contre lui.
\VS{25}Mais pour ceux qui le reprennent, ils en retireront de la satisfaction. Et la bénédiction vient sur eux pour leur bonheur.
\VS{26}Celui qui répond avec justesse fait plaisir à celui qui l'écoute.
\VS{27}Prépare ton ouvrage au-dehors, et apprête ton champ, et après, tu bâtiras ta maison.
\VS{28}Ne témoigne pas sans cause contre ton prochain ; car voudrais-tu tromper de tes lèvres\FTNT{Ep. 4:25.} ?
\VS{29}Ne dis pas : Comme il m'a fait, ainsi je lui ferai ; je rendrai à cet homme selon ce qu'il m'a fait.
\VS{30}J'ai passé près du champ de l'homme de paresseux, et près de la vigne d'un homme dépourvu de sens ;
\VS{31}et voici, tout y était monté en chardons, et les ronces avaient couvert la surface, et le mur de pierres était écroulé.
\VS{32}Et j'ai regardé, j'y ai appliqué mon coeur, j'ai vu et j'en ai tiré instruction.
\VS{33}Un peu de sommeil, un peu d'assoupissement, un peu croiser les mains pour dormir ! …
\VS{34}Et la pauvreté te surprendra comme un rôdeur ; et la disette, comme un homme armé.
\Chap{25}
\TextTitle{[Avertissements et conseils]}
\VerseOne{}Ce sont ici aussi des proverbes de Salomon\FTNT{1 R. 4: 32.}, que les gens d'Ezéchias, roi de Juda, ont transcrits.
\VS{2}La gloire de Dieu est de cacher les choses, et la gloire des rois est de sonder les choses.
\VS{3}Les cieux dans leur hauteur, la terre dans sa profondeur, le coeur des rois sont impénétrables.
\VS{4}Ôte de l'argent les scories, et il en sortira un vase pour l'orfèvre.
\VS{5} De même, ôte le méchant de devant le roi, et son trône sera affermi par la justice.
\VS{6}Ne t'élève pas devant le roi et ne te tiens pas à la place des grands.
\VS{7}Car il vaut mieux qu'on te dise: Monte-ici ! Que si l'on t'abaisse devant un seigneur que tes yeux ont vu\FTNT{Lu. 14:8-11.}.
\VS{8}Ne te hâte pas d'entrer en contestation, de peur que tu ne saches que faire à la fin, lorsque ton prochain t'aura confondu.
\VS{9}Plaide ta cause contre ton prochain, mais ne révèle pas le secret d'autrui.
\VS{10}De peur que celui qui l'écoute ne te couvre de honte, et que ton opprobre ne s'efface pas.
\VS{11}Telles que sont des pommes d'or sur des ciselures d'argent, telle est une parole dite quand il faut.
\VS{12}Comme un anneau d'or ou comme un joyau d'or fin, ainsi est l'oreille obéissante pour le sage qui réprimande.
\VS{13}Le messager fidèle est à ceux qui l'envoient, comme la fraîcheur de la neige au temps de la moisson, il restaure l'âme de son maître.
\VS{14}Celui qui se glorifie faussement de ses libéralités, est comme les nuages et le vent sans pluie.
\VS{15}Un prince est fléchi par la patience, et la langue douce brise les os.
\VS{16}As-tu trouvé du miel, manges-en autant qu'il t'en faut, de peur que tu n'en sois rassasié, que tu ne le vomisses.
\VS{17}De même, mets rarement le pied dans la maison de ton prochain, de peur qu'étant rassasié de toi, il ne te haïsse.
\VS{18}Comme une massue, une épée et une flèche aiguë, ainsi est un homme qui porte un faux témoignage contre son prochain.
\VS{19}Comme une dent qui se rompt, un pied qui glisse, telle est la confiance qu'on met en un traître au jour de la détresse.
\VS{20}Celui qui chante des chansons à un coeur affligé est comme celui qui ôte sa robe dans un jour froid, et comme du vinaigre répandu sur du nitre.
\VS{21}Si celui qui te hait a faim, donne-lui à manger du pain ; et s'il a soif, donne-lui à boire de l'eau\FTNT{Mt 5:39-44.}.
\VS{22}Car ce sont des charbons ardents que tu lui mets sur sa tête, et Yahweh te le rendra.
\VS{23}Le vent du nord engendre les averses, et la langue qui médit en cachette un visage irrité.
\VS{24}Il vaut mieux habiter à l'angle d'un toit que de partager la demeure d'une femme querelleuse.
\VS{25}Comme de l'eau fraîche pour une personne fatiguée et lasse, ainsi est une bonne nouvelle venant d'une terre lointaine.
\VS{26}Le juste qui bronche devant le méchant est une fontaine troublée et une source gâtée.
\VS{27}Comme il n'est pas bon de manger beaucoup de miel, aussi n'y a-t-il pas de gloire pour les hommes de rechercher leur gloire avec ardeur.
\VS{28}L'homme qui n'est pas maître de son esprit est comme une ville où il y a une brèche, et qui est sans murailles.
\Chap{26}
\TextTitle{[Avertissements et conseils (suite)]}
\VerseOne{}Comme la neige en été et la pluie pendant la moisson, ainsi la gloire ne convient pas à un insensé.
\VS{2}Comme l'oiseau est prompt à s'échapper et l'hirondelle à s'envoler, ainsi la malédiction sans cause n'atteint pas.
\VS{3}Le fouet est pour le cheval, le mors pour l'âne, et la verge pour le dos des insensés.
\VS{4}Ne réponds pas à l'insensé selon sa folie, de peur que tu ne lui ressembles toi-même.
\VS{5}Réponds à l'insensé selon sa folie, de peur qu'il ne devienne sage à ses propres yeux.
\VS{6}Celui qui envoie des messages par l'intermédiaire d'un insensé, se coupe les pieds et boit la peine du tort qu'il s'est fait.
\VS{7}Faites marcher un homme boiteux, ainsi il en sera d'un proverbe dans la bouche des insensés.
\VS{8}Celui qui donne de la gloire à un insensé, c'est comme s'il jetait une pierre précieuse dans un monceau de pierres.
\VS{9}Comme une épine dans la main d'un homme ivre, ainsi est un proverbe dans la bouche des insensés.
\VS{10}Les puissants donnent de l'ennui à tout le monde, et prennent à leur service les insensés et les passants.
\VS{11}Comme le chien retourne à ce qu'il a vomi, ainsi l'insensé réitère sa folie\FTNT{2 Pi. 2:22.}.
\VS{12}As-tu vu un homme qui croit être sage ? Il y a plus à espérer d'un insensé que de lui.
\VS{13}Le paresseux dit : Il y a un lion rugissant sur le chemin, il y a un lion dans les rues.
\VS{14}Comme une porte tourne sur ses gonds, ainsi fait le paresseux sur son lit.
\VS{15}Le paresseux plonge sa main dans le plat, et il trouve fatigant de la ramener à sa bouche.
\VS{16}Le paresseux se croit plus sage que sept hommes qui répondent avec bon sens.
\VS{17}Celui qui en passant se met en colère pour une dispute qui ne le touche en rien, est comme celui qui prend un chien par les oreilles.
\VS{18}Tel est celui qui fait l'insensé, et qui cependant jette des feux, des flèches, et des choses propres à tuer,
\VS{19}tel est l'homme qui a trompé son ami, et qui après cela dit : N'était-ce pas pour plaisanter ?
\VS{20}Le feu s'éteint faute de bois, ainsi quand il n'y a pas de rapporteurs la querelle s'apaise.
\VS{21}Le charbon est pour faire de la braise, et le bois pour faire du feu, et l'homme querelleur pour exciter des querelles.
\VS{22}Les paroles d'un rapporteur sont comme des friandises, elles descendent jusqu'au fond des entrailles.
\VS{23}Les lèvres brûlantes de zèle et le coeur mauvais sont comme des scories d'argent appliquées sur un vase de terre.
\VS{24}Celui qui a de la haine se déguise par ses discours, mais au-dedans de lui il nourrit la trahison.
\VS{25}Lorsqu'il prend une voix douce, ne le crois pas, car il y a sept abominations dans son coeur.
\VS{26}S'il cache sa haine sous la dissimulation, sa méchanceté sera révélée dans l'assemblée.
\VS{27}Celui qui creuse la fosse y tombe ; et la pierre retourne sur celui qui la roule\FTNT{Ps. 7:16-17 ; Ps. 57:7 ; Ec. 10:8.}.
\VS{28}La fausse langue hait ceux qu'elle a écrasés ; et la bouche flatteuse prépare la ruine.
\Chap{27}
\TextTitle{[Avertissements et conseils (suite)]}
\VerseOne{}Ne te vante point du lendemain, car tu ne sais pas quelle chose le jour enfantera\FTNT{Ja. 4:13-15.}.
\VS{2}Qu'un autre te loue, et non pas ta propre bouche ; que ce soit l'étranger, et non pas tes lèvres.
\VS{3}La pierre est pesante, et le sable est lourd ; mais l'irritation de l'insensé est plus pesante que tous les deux.
\VS{4}Il y a de la cruauté dans la fureur, et du débordement dans la colère ; mais qui pourra subsister devant la jalousie ?
\VS{5}Mieux vaut une réprimande ouverte qu'une amitié cachée.
\VS{6}Les blessures d'un ami sont dignes de confiance, mais les baisers d'un ennemi sont à craindre\FTNT{Il est question ici de Judas}.
\VS{7}L'âme rassasiée foule aux pieds les rayons de miel ; mais l'âme qui a faim trouve doux même ce qui est amer.
\VS{8}Tel qu'est un oiseau errant loin de son nid, tel est l'homme qui s'écarte de son lieu.
\VS{9}L'huile et les parfums réjouissent le coeur, et il en est ainsi de la douceur d'un ami dont le fruit est un conseil qui vient du coeur.
\VS{10}Ne quitte point ton ami ni l'ami de ton père, et n'entre pas dans la maison de ton frère au jour de ta détresse ; car le voisin qui est proche vaut mieux que le frère qui est éloigné.
\VS{11}Mon fils sois sage, et réjouis mon coeur, afin que j'aie de quoi répondre à celui qui m'outrage.
\VS{12}L'homme prudent voit le malheur arriver et se cache ; mais les stupides passent outre et en payent l'amende.
\VS{13}Quand quelqu'un se portera garant pour l'étranger, prends son vêtement, exige de lui des gages, à cause des étrangers.
\VS{14}Celui qui bénit son ami à haute voix, se levant de grand matin, on le lui comptera comme une malédiction.
\VS{15}Une gouttière continuelle en un jour de grosse pluie, et une femme querelleuse, cela se ressemble.
\VS{16}Celui qui veut la retenir est comme s'il voulait arrêter le vent, et retenir dans sa main une huile qui s'écoule.
\VS{17}Comme le fer aiguise le fer, ainsi l'homme aiguise la personnalité de son prochain.
\VS{18}Comme celui qui garde le figuier mangera de son fruit, ainsi celui qui garde son maître sera honoré.
\VS{19}Comme dans l'eau le visage répond au visage, ainsi le coeur d'un homme répond à celui d'un autre homme.
\VS{20}Le scheol et le gouffre ne sont jamais rassasiés ; de même, les yeux des hommes sont insatiables\FTNT{Ec. 1:8 ; 2 Pi. 2:14.}.
\VS{21}Le fourneau est pour éprouver l'argent, et le creuset pour l'or ; mais un homme est jugé d'après sa renommée.
\VS{22}Quand tu pilerais un insensé dans un mortier, parmi du grain qu'on pile avec un pilon, sa stupidité ne se détacherait pas de lui.
\VS{23}Sois diligent à reconnaître l'état de chacune de tes brebis, et applique ton coeur aux troupeaux.
\VS{24}Car l'abondance ne dure pas à toujours, et une couronne dure-t-elle d'âge en âge ?
\VS{25}Le foin s'enlève et la verdure paraît, et les herbes des montagnes sont amassées.
\VS{26}Les agneaux sont pour te vêtir, et les boucs pour payer le champ ;
\VS{27}et l'abondance du lait des chèvres sera pour ta nourriture et celle de ta maison, et pour la subsistance de tes servantes.
\Chap{28}
\TextTitle{[Avertissements et conseils (suite)]}
\VerseOne{}Le méchant prend la fuite sans qu'on le poursuive, mais les justes seront assurés comme un jeune lion.
\VS{2}Il y a plusieurs chefs, à cause de la rébellion d'un pays, mais pour l'amour de l'homme avisé et intelligent, il y aura prolongation du même gouvernement.
\VS{3}L'homme qui est pauvre et qui opprime les pauvres, est comme une pluie violente qui cause la disette du pain.
\VS{4}Ceux qui abandonnent la loi louent le méchant, mais ceux qui gardent la loi leur font la guerre.
\VS{5}Les gens adonnés au mal n'entendent point ce qui est droit ; mais ceux qui cherchent Yahweh comprennent tout.
\VS{6}Le pauvre qui marche dans son intégrité vaut mieux que le pervers qui marche par deux chemins et qui est riche.
\VS{7}Celui qui garde la loi est un fils prudent, mais celui qui entretient les gourmands fait honte à son père.
\VS{8}Celui qui augmente ses biens par l'intérêt et l'usure, les amasse pour celui qui en fera des libéralités aux pauvres.
\VS{9}Celui qui détourne son oreille pour ne pas écouter la loi, sa prière même est une abomination\FTNT{La prière doit être faite selon la Parole de Dieu, en conformité avec sa volonté. Le Seigneur n'exauce que ceux qui obéissent à sa Parole (Mt. 6:9-10 ; Jn. 9:31 ; Jn. 15:7 ; 1 Jn. 5:14-15).}.
\VS{10}Celui qui égare les hommes droits dans le mauvais chemin tombera dans la fosse qu'il aura creusée, mais ceux qui sont intègres hériteront le bonheur.
\VS{11}L'homme riche pense être sage, mais le pauvre qui est intelligent le sondera.
\VS{12}Quand les justes se réjouissent, la gloire est grande, mais quand les méchants sont élevés, chacun se déguise.
\VS{13}Celui qui cache ses transgressions ne prospère point, mais celui qui les confesse et les délaisse, obtient miséricorde\FTNT{Ec. 1:8 ; 2 Pi. 2:14.}.
\VS{14}Heureux est l'homme qui est continuellement dans la crainte, mais celui qui endurcit son coeur tombera dans la calamité.
\VS{15}Le méchant qui domine sur un peuple pauvre est un lion rugissant et comme un ours quêtant sa proie.
\VS{16}Le conducteur qui manque d'intelligence fait beaucoup d'extorsions, mais celui qui hait le gain déshonnête prolonge ses jours.
\VS{17}L'homme chargé du sang d'une personne fuira jusqu'à la fosse sans qu'aucun ne le retienne.
\VS{18}Celui qui marche dans l'intégrité sera sauvé, mais le pervers qui suit deux chemins tombera tout à coup.
\VS{19}Celui qui laboure sa terre sera rassasié de pain, mais celui qui suit les fainéants sera accablé de misère.
\VS{20}L'homme fidèle abondera en bénédictions, mais celui qui se hâte de s'enrichir ne restera pas impuni.
\VS{21}Il n'est pas bon d'avoir égard à l'apparence des personnes, car pour un morceau de pain l'homme commet un crime.
\VS{22}L'homme qui a l'oeil malin se hâte pour avoir des richesses, et il ne sait pas que la disette lui arrivera.
\VS{23}Celui qui reprend les hommes obtient ensuite plus de faveur que celui qui flatte de sa langue.
\VS{24}Celui qui pille son père ou sa mère, et qui dit que ce n'est point un péché, est compagnon de l'homme dissipateur.
\VS{25}Celui qui a l'âme enflée excite les querelles, mais celui qui se confie en Yahweh sera rassasié.
\VS{26}Celui qui se confie dans son propre coeur est un fou, mais celui qui marche sagement sera délivré.
\VS{27}Celui qui donne au pauvre n'aura point de disette, mais celui qui en détourne ses yeux abondera en malédictions.
\VS{28}Quand les méchants s'élèvent, l'homme se cache ; mais quand ils périssent, les justes se multiplient.
\Chap{29}
\TextTitle{[Avertissements et conseils (suite)]}
\VerseOne{}L'homme qui étant repris, raidit son cou, sera subitement brisé et sans qu'il y ait de guérison.
\VS{2}Quand les justes sont nombreux, le peuple se réjouit ; mais quand le méchant domine, le peuple gémit.
\VS{3}L'homme qui aime la sagesse, réjouit son père, mais celui qui se plaît avec les femmes prostituées dissipe ses richesses.
\VS{4}Le roi affermit le pays par la justice, mais l'homme qui est adonné aux présents le ruinera.
\VS{5}L'homme qui flatte son prochain lui tend un piège sous ses pas.
\VS{6}Le péché de l'homme méchant lui tend un piège dangereux, mais le juste triomphe et se réjouit.
\VS{7}Le juste prend connaissance de la cause des pauvres, mais le méchant n'en prend pas connaissance.
\VS{8}Les hommes moqueurs troublent la ville, mais les sages apaisent la colère.
\VS{9}Un homme sage qui conteste avec un insensé, qu'il se fâche ou qu'il rie, la paix n'aura pas lieu.
\VS{10}Les hommes de sang ont en haine l'homme intègre, mais les hommes droits tiennent chère sa vie.
\VS{11}L'insensé pousse au-dehors tout ce qu'il a dans l'esprit, mais le sage le calme et le retient en arrière.
\VS{12}Tous les serviteurs d'un prince qui prêtent l'oreille à la parole de mensonge sont méchants.
\VS{13}Le pauvre et l'oppresseur se rencontrent, c'est Yahweh qui illumine les yeux de l'un et de l'autre.
\VS{14}Le trône du roi qui fait justice selon la vérité aux pauvres, sera établi à perpétuité.
\VS{15}La verge et la réprimande donnent la sagesse, mais l'enfant livré à lui-même fait honte à sa mère.
\VS{16}Quand les méchants se multiplient, les péchés s'accroissent, mais les justes verront leur ruine.
\VS{17}Corrige ton fils, et il te donnera du repos, et il procurera du plaisir à ton âme.
\VS{18}Lorsqu'il n'y a pas de vision\FTNT{Le manque de vision n'est bon pour personne. Dieu donne une vision aux personnes qu'il a appelées. La vision peut être un songe, une directive, une prophétie, etc. Il s'agit des objectifs à atteindre.}, le peuple est sans frein, mais heureux est celui qui garde la loi !
\VS{19}L'esclave ne se corrige pas par des paroles, même s'il comprend, il n'obéit pas.
\VS{20}As-tu vu un homme irréfléchi dans ses paroles ? Il y a plus à espérer d'un insensé que de lui.
\VS{21}Le serviteur qu'on a traité délicatement dès sa jeunesse finit par se croire un fils.
\VS{22}L'homme coléreux excite des querelles, et l'homme furieux commet beaucoup de péchés.
\VS{23}L'orgueil de l'homme l'abaisse, mais celui qui est humble d'esprit obtient la gloire\FTNT{Mt. 23:12 ; Lu. 14:11 ; 1 Pi. 5:5.}.
\VS{24}Celui qui partage avec un voleur hait son âme ; il entend la malédiction, et il ne révèle rien.
\VS{25}La crainte qu'on a des hommes tend un piège, mais celui qui se confie en Yahweh est élevé dans une haute retraite.
\VS{26}Plusieurs recherchent la face de celui qui domine, mais c'est de Yahweh que vient le jugement des hommes.
\VS{27}L'homme inique est en abomination aux justes, et celui dont la voie est droite est en abomination au méchant.
\Chap{30}
\TextTitle{[Proverbe d'Agur]}
\VerseOne{}Les paroles d'Agur, fils de Jaké, à savoir la sentence prononcée par cet homme pour Ithiel, pour Ithiel et Ucal.
\VS{2}Certainement je suis le plus stupide de tous les hommes, et il n'y a pas en moi l'intelligence humaine.
\VS{3}Et je n'ai pas appris la sagesse ; et je n'ai pas la connaissance des saints.
\VS{4}Qui est celui qui est monté aux cieux, ou qui en est descendu\FTNT{Jn. 3:13 ; Ro. 10:6-7.} ? Qui est celui qui a recueilli le vent dans le creux de sa main, qui a serré les eaux dans son manteau, qui a dressé toutes les bornes de la terre ? Quel est son nom, et quel est le nom de son fils, le sais-tu ?
\VS{5}Toute la parole de Dieu est éprouvée ; il est un bouclier pour ceux qui se réfugient en lui\FTNT{Ps. 18:31 ; Ps. 115:9-11.}.
\VS{6}N'ajoute rien à ses paroles, de peur qu'il ne te reprenne et que tu ne sois trouvé menteur.
\VS{7}Je te demande deux choses : Ne me les refuse pas durant ma vie.
\VS{8}Eloigne de moi la vanité et la parole mensongère ; ne me donne ni pauvreté ni richesse, nourris-moi du pain qui m'est nécessaire.
\VS{9}De peur que dans l'abondance je ne te renie, et que je ne dise : Qui est Yahweh ? Ou que dans la pauvreté, je ne dérobe et que je ne porte atteinte au Nom de mon Dieu.
\VS{10}N'accuse pas un serviteur devant son maître, de peur que ce serviteur ne te maudisse, et qu'il ne t'en arrive du mal.
\VS{11}Il est une race de gens qui maudit son père et qui ne bénit pas sa mère.
\VS{12}Il est une race de gens qui croit être pure, et qui toutefois n'est point lavée de son ordure.
\VS{13}Il est une race de gens dont les yeux sont fort hautains, et les paupières élevées.
\VS{14}Il est une race de gens dont les dents sont des épées, et les mâchoires sont des couteaux pour dévorer les malheureux sur la terre et les pauvres d'entre les hommes.
\VS{15}La sangsue a deux filles qui disent : Apporte ! Apporte ! Il y a trois choses qui sont insatiables, il y en a même quatre qui ne disent point : C'est assez !
\VS{16}Le scheol, la matrice stérile, la terre qui n'est pas rassasiée d'eau, et le feu qui ne dit jamais : C'est assez !
\VS{17}L'oeil de celui qui se moque de son père et qui méprise l'enseignement de sa mère, les corbeaux des torrents le crèveront, et les petits de l'aigle le mangeront.
\VS{18}Il y a trois choses qui sont trop merveilleuses pour moi, même quatre, que je ne connais point :
\VS{19}La trace de l'aigle dans le ciel, la trace du serpent sur un rocher, le chemin d'un navire au milieu de la mer, et la trace de l'homme chez la jeune femme.
\VS{20}Telle est la trace de la femme adultère : Elle mange, et s'essuie la bouche, puis elle dit : Je n'ai pas commis d'iniquité.
\VS{21}La terre tremble pour trois choses, même pour quatre, qu'elle ne peut supporter :
\VS{22}Pour l'esclave quand il vient à régner, pour l'insensé quand il est rassasié de pain,
\VS{23}pour la femme odieuse quand elle se marie, et pour la servante quand elle hérite de sa maîtresse.
\VS{24}Il y a quatre choses des plus petites de la terre qui toutefois sont bien sages entre les sages :
\VS{25}Les fourmis, qui sont un peuple sans force, et qui néanmoins préparent durant l'été leur nourriture ;
\VS{26}les damans, qui sont un peuple qui n'est pas puissant, et qui néanmoins font leurs maisons dans les rochers ;
\VS{27}les sauterelles, qui n'ont point de roi, et qui toutefois sortent toutes par divisions ;
\VS{28}les lézards que tu peux saisir avec les mains, et qui sont pourtant dans les palais des rois.
\VS{29}Il y a trois choses qui ont une belle allure, même quatre, qui ont une belle démarche :
\VS{30}Le lion, qui est le plus fort d'entre les animaux, et qui ne recule pas à la rencontre de qui que ce soit ;
\VS{31}le cheval, qui a les flancs bien troussés ; le bouc, et le roi devant qui personne ne résiste.
\VS{32}Si tu t'es conduit follement en t'emportant, et si tu as des mauvaises intentions, mets la main sur ta bouche.
\VS{33}Comme celui qui bat le lait en fait sortir le beurre, et comme celui qui presse le nez en fait sortir le sang, ainsi celui qui provoque la colère excite la querelle.
\Chap{31}
\TextTitle{Proverbe de Lemuel}
\VerseOne{}Les paroles du roi Lemuel et l'instruction que sa mère lui donna.
\VS{2}Quoi, mon fils ? Quoi, fils de mes entrailles ? Eh quoi, mon fils, pour lequel j'ai tant fait de voeux ?
\VS{3}Ne livre pas ta vigueur aux femmes et tes voies à celles qui perdent les rois.
\VS{4}Lemuel, ce n'est point aux rois, ce n'est point aux rois de boire le vin, ni aux princes de boire la cervoise\FTNT{La cervoise est une bière faite avec de l'orge ou d'autres céréales.}
\VS{5}de peur qu'ayant bu, ils n'oublient ce qui a été prescrit, et qu'ils n'altèrent le jugement de tous les pauvres affligés.
\VS{6}Donnez de la cervoise à celui qui va périr, et du vin à celui qui a l'amertume dans le coeur ;
\VS{7}afin qu'il en boive, et qu'il oublie sa pauvreté, et ne se souvienne plus de ses peines.
\VS{8}Ouvre ta bouche en faveur du muet, pour la cause de tous les fils délaissés qui vont périr.
\VS{9}Ouvre ta bouche, fais justice, et plaide pour le malheureux et l'indigent.
\TextTitle{[La femme vertueuse]}
\VS{10}[Aleph.] Qui est-ce qui trouvera une femme vertueuse ? Car son prix surpasse de beaucoup les perles.
\VS{11}[Beth.] Le coeur de son mari a confiance en elle, et il ne manquera point de dépouilles.
\VS{12}[Guimel.] Elle lui fera du bien tous les jours de sa vie, et jamais du mal.
\VS{13}[Daleth.] Elle cherche de la laine et du lin, et elle en fait ce qu'elle veut avec ses mains.
\VS{14}[He.] Elle est semblable aux navires d'un marchand, elle amène son pain de loin.
\VS{15}[Vav.] Elle se lève lorsqu'il est encore nuit, elle donne la nourriture nécessaire à sa maison et elle donne à ses servantes leur tâche.
\VS{16}[Zayin.] Elle pense à un champ, et l'acquiert ; et elle plante la vigne du fruit de ses mains.
\VS{17}[Heth.] Elle ceint ses reins de force, et affermit ses bras.
\VS{18}[Teth.] Elle sent que ce qu'elle gagne est bon ; sa lampe ne s'éteint pas pendant la nuit.
\VS{19}[Yod.] Elle met sa main à la quenouille, et ses doigts tiennent le fuseau.
\VS{20}[Kaf.] Elle tend sa main au malheureux, elle tend ses mains à l'indigent.
\VS{21}[Lamed.] Elle ne craint point la neige pour sa famille, car toute sa famille est vêtue de vêtements doubles.
\VS{22}[Mem.] Elle se fait des couvertures, le fin lin et l'écarlate sont ce dont elle s'habille.
\VS{23}[Nun.] Son mari est considéré aux portes, lorsqu'il siège avec les anciens du pays.
\VS{24}[Samech.] Elle fait des chemises et les vend, et elle livre des ceintures au marchand.
\VS{25}[Ayin.] Elle est revêtue de force et de gloire, elle se rit du jour à venir.
\VS{26}[Pe.] Elle ouvre sa bouche avec sagesse, et la loi de la charité est sur sa langue.
\VS{27}[Tsade.] Elle veille sur ce qui se passe dans sa maison, et elle ne mange pas le pain de la paresse.
\VS{28}[Qof.] Ses fils se lèvent et la disent bienheureuse ; son mari aussi, et il la loue, en disant :
\VS{29}[Resh.] Plusieurs filles se sont conduites vertueusement, mais toi, tu les surpasses toutes.
\VS{30}[Shin.] La grâce est trompeuse et la beauté vaine, mais la femme qui craint Yahweh est celle qui sera louée.
\VS{31}[Tav.] Récompensez-la du fruit de ses mains, et que ses oeuvres la louent aux portes.
\PPE{}
\end{multicols}

\clearpage\ShortTitle{Job}\BookTitle{Job}\BFont
\noindent\hrulefill
{\footnotesize
\textit{
\bigskip
{\centering{}
\\Auteur : Inconnu
\\(Heb. : Iyov)
\\Signification : haï, ennemi et « je m'exclamerai »
\\Thème : La souffrance
\\Date de rédaction : Incertaine\\}
}
%\bigskip
\textit{
\\Job était un homme prospère et intègre auquel Dieu rendit témoignage. Il subit une succession de malheurs en très peu de temps en perdant tout ce qui lui était cher. Après avoir cherché à se justifier et subi les railleries de sa femme et les accusations de ses amis, Job s'humilia devant Dieu et comprit l'impuissance de sa propre justice. Cette histoire, dont on n'a aucune indication spatio-temporelle et qui pourtant parle à tous, est un encouragement pour le juste éprouvé.
%\bigskip
\\Rappelant que la souffrance peut être le moyen choisi par Dieu pour enseigner et se révéler, ce récit illustre la fidélité et la bonté de Yahweh envers ceux qui le craignent.\bigskip
}
}
\par\nobreak\noindent\hrulefill
\begin{multicols}{2}
\Chap{1}
\TextTitle{Job et sa famille}
\VerseOne{}Il y avait dans le pays d'Uts\FTNT{Ge. 36:28} un homme dont le nom était Job\FTNT{Ez. 14:14; Ja. 5:11.}. Cet homme était intègre\FTNT{1 R. 8:61.} et droit, craignant\FTNT{Ps. 19:10; Pr. 1:7.} Dieu et se détournant du mal.
\VS{2}Il eut sept fils et trois filles.
\VS{3}Et son bétail était de sept mille brebis, trois mille chameaux, cinq cents paires de bœufs, cinq cents ânesses, avec un très grand nombre de serviteurs\FTNT{Job 42:12-13.}; tellement que cet homme était le plus puissant de tous les Orientaux.
\VS{4}Or ses fils allaient et faisaient des festins les uns chez les autres chacun à son jour, et ils envoyaient appeler leurs trois sœurs pour manger et boire avec eux.
\VS{5}Quand les jours de festin étaient passés, Job envoyait chercher ses fils pour les sanctifier, et se levant de bon matin, il offrait un holocauste selon le nombre de ses enfants ; car Job disait : Peut-être mes fils ont-ils péché, et ont-ils blasphémé contre Dieu dans leurs cœurs. Job faisait toujours ainsi.\FTNT{Job 42:8.}
\VS{6}Or, il arriva un jour que les fils de Dieu\FTNT{Ps. 89:7 ; Job 38:7.} vinrent se présenter devant Yahweh, et Satan\FTNT{Es. 14:12; Ap. 12:9-10.} aussi vint au milieu d'eux.
\VS{7}Yahweh dit à Satan : D'où viens-tu ? Et Satan répondit à Yahweh : De courir çà et là sur la terre et de m'y promener\FTNT{1 Pi. 5:8.}.
\VS{8}Yahweh dit à Satan : N'as-tu point considéré mon serviteur Job, qui n'a point d'égal sur la terre ; homme intègre et droit, craignant Dieu, et se détournant du mal ?
\VS{9}Et Satan répondit à Yahweh : Est-ce en vain que Job craint Dieu ?
\VS{10}N'as-tu pas mis une haie \FTNT{La haie est une protection, une barrière végétale entretenue afin de protéger et de clôturer un terrain.} tout autour de lui, autour de sa maison, autour de tout ce qui lui appartient ? Tu as béni l'œuvre de ses mains, et ses troupeaux se répandent sur la terre.
\VS{11}Mais étends maintenant ta main, touche à tout ce qui lui appartient, et tu verras s'il ne te maudit pas en face.
\VS{12}Et Yahweh dit à Satan : Voilà, tout ce qui lui appartient est en ton pouvoir ; seulement ne porte pas la main sur lui. Et Satan sortit de devant la face de Yahweh\FTNT{1 R. 22:22.}.
\TextTitle{Première attaque de Satan}
\VS{13}Il arriva donc qu'un jour comme les fils et les filles de Job mangeaient et buvaient du vin dans la maison de leur frère aîné, un messager vint vers Job,
\VS{14}et lui dit : Les bœufs labouraient, et les ânesses paissaient à côté d'eux;
\VS{15}et ceux de Séba se sont jetés dessus, les ont pris, et ont frappé les serviteurs au fil de l'épée. Et je me suis échappé, moi seul, pour te l'annoncer.
\VS{16}Cet homme parlait encore, lorsqu'un autre vint et dit : Le feu de Dieu est tombé du ciel, il a brûlé les brebis et les serviteurs, et les a consumés\FTNT{2 R. 1:10-12.}. Et je me suis échappé moi seul, pour te l'annoncer.
\VS{17}Cet homme parlait encore, lorsqu'un autre vint et dit : Des Chaldéens\FTNT{Ge. 11:28.} ont fait trois bandes, se sont jetés sur les chameaux et les ont pris, ils ont frappé les serviteurs au fil de l'épée, et je me suis échappé moi seul, pour te l'annoncer.
\VS{18}Cet homme parlait encore, lorsqu'un autre vint et dit : Tes fils et tes filles mangeaient et buvaient du vin dans la maison de leur frère aîné ;
\VS{19}voici, un grand vent est venu de l'autre côté du désert et a frappé contre les quatre coins de la maison ; elle est tombée sur les jeunes gens, et ils sont morts. Et je me suis échappé, moi seul, pour te l'annoncer.
\VS{20}Alors Job se leva, déchira\FTNT{Job 2:12 ; Est. 4:1.} son manteau et  rasa la tête ; et se jetant par terre, se prosterna,
\VS{21}et dit : Je suis sorti nu du ventre de ma mère, et nu je retournerai dans le sein de la terre\FTNT{Ec. 5:14. ; 1 Ti. 6:7.} ; Yahweh a donné, Yahweh a enlevé\FTNT{1 S. 2:6.} ; que le nom de Yahweh soit béni !
\VS{22}En tout cela, Job ne pécha pas et n'attribua rien d'injuste à Dieu.
\Chap{2}
\TextTitle{Deuxième attaque de Satan}
\VerseOne{}Or il arriva un jour que les fils de Dieu vinrent un jour se présenter devant Yahweh, Satan\FTNT{Za. 3:1-2.} vint aussi au milieu d'eux se présenter devant Yahweh.
\VS{2}Yahweh dit à Satan : D'où viens-tu ? Satan répondit à Yahweh : De courir çà et là sur la terre et de m'y promener.
\VS{3}Yahweh dit à Satan: N'as-tu point considéré mon serviteur Job,qui n'a point d'égal sur la terre ; homme sincère et droit, craignant Dieu, et se détournant du mal ? Il demeure ferme dans son intégrité, quoique tu m'aies incité contre lui à le détruire sans cause\FTNT{Job 9:17.}.
\VS{4}Et Satan répondit à l’Eternel, en disant : Chacun donnera peau pour peau, et tout ce qu’il a, pour sa vie.
\VS{5}Mais étends maintenant ta main, et frappe à ses os et à sa chair\FTNT{Job 19:20.}, et tu verras s'il ne te maudit pas en face. 
\VS{6}Yahweh dit à Satan : Voici, il est en ta main : Seulement garde sa vie.
\VS{7}Ainsi Satan sortit de devant l’Eternel, et frappa Job d’un ulcère malin, depuis la plante de ses pieds jusqu’au sommet de la tête.
\VS{8}Job prit un tesson pour se gratter et s'assit au milieu de la cendre\FTNT{Jé. 6:26 ; Jon. 3:6.}.
\TextTitle{Réaction de Job et de sa femme}
\VS{9}Et sa femme lui dit : Conserveras-tu encore ton intégrité ?  Bénis\FTNT{Job 1:11.} Dieu, et meurs !
\VS{10}Et il lui dit : Tu parles comme une femme insensée ! Nous recevons le bien de la part de Dieu, et nous n'en recevrions pas le mal !\FTNT{Es. 45:7 ; Am. 3:6 ; La. 3:37.} En tout cela, Job ne pécha pas par ses lèvres.
\TextTitle{Job et ses trois amis}
\VS{11}Et trois des amis de Job, Eliphaz de Théman, Bildad de Schuach, et Tsophar de Naama, ayant appris tous les maux qui lui étaient arrivés, vinrent chacun du lieu de leur demeure, après s'être convenus ensemble d'un jour pour venir le plaindre et le consoler.
\VS{12}Ayant de loin levé les yeux sur lui, ils ne le reconnurent pas, alors ils élevèrent la voix et ils pleurèrent. Ils déchirèrent leurs manteaux, et jetèrent de la poussière vers le ciel au-dessus de leur tête.
\VS{13}Et ils s'assirent à terre avec lui, sept jours et sept nuits, et aucun d'eux ne lui dit une parole, car ils voyaient que sa douleur était fort grande.
\Chap{3}
\TextTitle{Lamentations de Job}
\VerseOne{}Après cela, Job ouvrit la bouche et maudit le jour de sa naissance.\FTNT{Jé. 20:14 ; Job 10:18.}
\VS{2}Car prenant la parole, il dit :
\VS{3}Périsse le jour où je suis né, et la nuit qui a dit : Un homme est conçu !
\VS{4}Que ce jour-là ne soit que ténèbres ; que Dieu ne le recherche point d'en haut, et qu'il ne soit point éclairé de la lumière ! 
\VS{5}Que les ténèbres et l'ombre de la mort\FTNT{Job 10:21-22.} s'en emparent, que les nuées demeurent sur lui, qu'il soit rendu terrible comme le jour de ceux à qui la vie est amère ! 
\VS{6}Que l'obscurité prenne cette nuit, qu'elle ne se réjouisse pas au milieu des jours de l'année, qu'elle n'entre pas dans le compte des mois !
\VS{7}Voici, que cette nuit soit stérile, et qu'aucun cri de joie n'y survienne !
\VS{8}Qu'ils la maudissent ceux qui maudissent les jours, ceux qui sont prêts à réveiller le Léviathan !
\VS{9}Que les étoiles de son crépuscule soient obscurcies ; qu'elle attende la lumière, mais qu'il n'y en ait point, et qu'elle ne voie point les rayons de l'aube du jour ! \FTNT{Job 41:9.} !
\VS{10}Parce qu'elle n'a pas fermé le sein qui me conçut ni caché la souffrance à mes yeux.
\VS{11}Pourquoi ne suis-je pas mort dans le sein de ma mère ? Pourquoi n'ai-je pas expiré aussitôt que je suis sorti de ses entrailles ?\FTNT{Job 10:18.}
\VS{12}Pourquoi des genoux m'ont-ils reçu? Pourquoi des mamelles m'ont-elles allaité ?
\VS{13}Je serais couché maintenant, je h serais tranquille, je dormirais, je me reposerais\FTNT{Job 17:16.},
\VS{14}avec les rois et les grands de la terre, qui se bâtirent des mausolées,
\VS{15}avec les princes qui possedèrent de l'or, et qui remplirent d'argent leurs maisons.
\VS{16}Ou comme l'avorton caché, je n'existerais pas\FTNT{Ps. 58:9.}, comme les petits enfants qui n'ont pas vu la lumière.
\VS{17}Là les méchants n'agitent plus personne, et là se reposent ceux qui sont fatigués. 
\VS{18}Pareillement ceux qui avaient été dans les liens, jouissent là du repos, et n'entendent plus la voix de l'oppresseur. 
\VS{19}le petit et le grand sont là, et l'esclave est délivré de son maître.
\VS{20}Pourquoi la lumière est-elle donnée au misérable, et la vie à ceux qui ont le cœur dans l'amertume ;
\VS{21}qui désirent en vain la mort, et qui la recherchent plus que le trésor,\FTNT{Ap. 9:6.}
\VS{22}qui seraient ravis de joie et seraient dans l'allégresse s'ils avaient trouvé le tombeau ?
\VS{23}Pourquoi, dis-je, la lumière est-elle donnée à l'homme à qui le chemin est caché, et que Dieu a enfermé de toutes parts\FTNT{Job 19:8 ; La. 3:7.} ?
\VS{24}Car avant que je mange, mon soupir vient, et mes cris se répandent comme de l'eau. 
\VS{25}Ce que je crains le plus, m'arrive, et ce que je redoute le plus, m'atteint. 
\VS{26}Je n'ai point eu de paix, je n'ai point eu de repos, ni de calme, depuis que ce trouble m'est arrivé. 
\Chap{4}
\TextTitle{Premier discours d'Eliphaz}
\VerseOne{}Alors Eliphaz de Théman prit la parole et dit :
\VS{2}Si l'on tente de te parler, en seras-tu peiné ? Mais qui pourrait retenir ses paroles ?
\VS{3}Voici, tu as souvent instruit les autres, et tu as fortifié les mains affaiblies\FTNT{Es. 35:3 ; Hé. 12:12.},
\VS{4}Tes paroles ont affermi ceux qui chancelaient, et tu as fortifié les genoux qui pliaient\FTNT{Job 16:5.}.
\VS{5}Et maintenant que le malheur t'arrive, tu faiblis ! Maintenant que tu es atteint, tu en es tout troublé !
\VS{6} Ta crainte de Yahweh n'a-t-elle pas été ton espérance ? Et l'intégrité de tes voies n'a-t-elle pas été ton attente ? 
\VS{7}Rappelle, je te prie, dans ton souvenir : Quel est l'innocent qui a péri ? Quels sont les justes qui ont été exterminés ?\FTNT{Job 8:20.}
\VS{8}Selon ce que j'ai vu, ceux qui labourent l'iniquité et qui sèment la peine en moissonnent les fruits ;\FTNT{Job 15:35 ; Ga. 6:7.}
\VS{9}ils périssent par le souffle de Dieu, et ils sont consumés par le vent de ses narines.\FTNT{Ex. 15:8 ; Es. 11:4 ; 30:33 ; Job 15:30 ; 2 Th. 2:8.}
\VS{10}Il étouffe le rugissement du lion, et le cri d'un grand lion, et il arrache les dents des lionceaux ;
\VS{11}le lion périt faute de proie, et les petits de la lionne sont dispersés.
\VS{12}Une parole m'est furtivement arrivée, et mon oreille en a saisi les sons légers.
\VS{13}Au moment où les visions de la nuit agitent la pensée, quand un profond sommeil tombe sur les hommes\FTNT{Job 33:15.},
\VS{14}une frayeur et un tremblement me saisirent, et tous mes os tremblèrent.
\VS{15}Un esprit passa devant moi, et mes cheveux en furent tout hérissés. 
\VS{16}Il se tint là et je ne reconnus pas son visage ; une figure était devant mes yeux. Et j'entendis un léger murmure et une voix :
\VS{17}L'homme serait-il juste devant Dieu ? L'homme serait-il pur devant celui qui l'a fait ?\FTNT{Job 25:4.}
\VS{18}Voici, il ne se fie pas à ses serviteurs, il trouve des erreurs à ses anges\FTNT{Job 15:15 ; Job 25:5 ; 2 Pi 2:4. }
\VS{19}combien plus chez ceux qui habitent des maisons d'argile, qui ont leurs fondements dans la poussière, qu'on écrase comme des vermisseaux !\FTNT{Job 25:6.}
\VS{20}Du matin au soir ils sont brisés, et, sans qu'on s'en aperçoive, ils périssent pour toujours. 
\VS{21}L'excellence qui était en eux, n'a-t-elle pas été emportée ? Ils meurent sans être sages. 
\Chap{5}
\VerseOne{}Crie maintenant ! Y aura-t-il quelqu'un qui te réponde ? Et vers quel saint te tourneras-tu ?\FTNT{Job 15:15.}
\VS{2}La colère tue l'insensé, et le fou meurt dans ses emportements.
\VS{3}J'ai vu l'insensé qui s'enracinait \FTNT{Jé. 12:1-2.}, mais j'ai aussitôt maudit sa demeure.
\VS{4}Ses fils sont loin de tout secours ; ils sont écrasés à la porte, et personne ne les délivre !\FTNT{Ps. 119:155.}
\VS{5} Sa moisson est dévorée par l'affamé, qui même la ravit d'entre les épines ; et le voleur convoite ses biens.
\VS{6}Le malheur ne sort pas de la poussière, et le travail ne germe pas de la terre ;
\VS{7}l'homme naît pour la peine\FTNT{Ge. 3:17-19 ; Job 14:1-5.}, comme l'étincelle pour voler et s'élever.
\VS{8}Mais moi, j'aurais recours à Dieu, et j'adresserais ma parole à Dieu.
\VS{9}Il fait de grandes choses qu'on ne peut sonder, de merveilleuses choses qu'on ne peut compter\FTNT{Ps. 72:18. Ps. 92:5 ; Job 9:10.}.
\VS{10}Il répand la pluie sur la face de la terre, et envoie les eaux sur les campagnes\FTNT{De. 28:12 ; Ps. 135:7 ; Job 28:26; Job 38:25-26 ; Ac. 14:17.};
\VS{11}il met en haut ceux qui sont abaissés, et délivre les affligés\FTNT{1 S. 2:7; Ez. 21:31 ; Ps. 113:7-8.} ;
\VS{12}il anéantit les projets des hommes rusés, de sorte qu'ils ne viennent pas à bout de leurs entreprises\FTNT{Es. 8:10 ; Ps. 33:10 ; Né. 4:15.} ;
\VS{13}il prend les sages dans leur propre ruse\FTNT{1 Co. 3:19.}, et les desseins des hommes pervers sont renversés :
\VS{14}De jour ils rencontrent les ténèbres, et ils marchent à tâtons en plein midi, comme dans la nuit.
\FTNT{De. 28:29.}.
\VS{15}Ainsi Dieu délivre le pauvre de l'épée de leur bouche, et le sauve de la main des puissants\FTNT{Ps. 12:3-4; Ps 52:2; Ps. 57:4.} ;
\VS{16}et l'espérance soutient le malheureux\FTNT{1S. 2:8.}, et la méchanceté a la bouche fermée\FTNT{Es. 52:15 ; Ps. 63: 11; Ps. 107: 42; Pr. 10:6.}.
\VS{17}Voici, heureux est celui que Yahweh châtie ! Ne rejette donc point le châtiment de Yahweh.
\FTNT{Ps 94:12 ; Pr.3:11-12 ; Hé. 12:5-6; Ap. 3:19.}.
\VS{18}Car c'est lui qui fait la plaie, et la bande ; il blesse et ses mains guérissent\FTNT{De. 32:39; 1S. 2: 6-7 ; Cp. Es. 30:26 ; Os. 6:1.}.
\VS{19}Six fois il te délivrera de l'angoisse, et sept fois le mal ne te touchera pas\FTNT{Ps 34:20; Ps. 91:3; Pr.24:16.}.
\VS{20}Il te sauvera de la mort pendant la famine, et du tranchant de l'épée pendant la guerre\FTNT{Ps. 33: 19; Ps. 37:19.}.
\VS{21}Tu seras à l'abri du fléau de la langue, et tu n’auras point peur de la dévastation, quand elle arrivera.
\FTNT{Ps. 31:21.}.
\VS{22}Tu riras de la dévastation et de la famine, et tu n'auras pas peur des bêtes de la terre\FTNT{Es. 65:25; Ez. 34:25; Os. 2:20.};
\VS{23}car tu feras une alliance avec les pierres des champs, et les bêtes des champs seront en paix avec toi\FTNT{Os. 2:20.}.
\VS{24}Tu jouiras en paix de la prospérité sous ta tente, tu pourvoiras à ta demeure et tu n'y seras point trompé ;
\VS{25}tu verras ta postérité s'accroître, et tes descendants se multiplier comme l'herbe de la terre\FTNT{Ps. 72:16; Ps. 127: 3-5; Ps. 128:6.}.
\VS{26}Tu entreras au tombeau dans ta vieillesse, comme une gerbe qu'on emporte en son temps\FTNT{Pr. 9:11; Pr. 10:27.}.
\VS{27}Voilà ce que nous avons examiné, voilà ce qui est ; à toi d'entendre et de choisir.
\Chap{6}
\TextTitle{Réponse de Job}
\VerseOne{}Job prit la parole et dit :
\VS{2}Oh ! si l’on pesait ma douleur, et si l’on mettait en même temps mes calamités dans la balance !
\VS{3}Car elle serait plus pesante que le sable de la mer ; c’est pourquoi mes paroles sont englouties.
\FTNT{Pr. 27:3.} !
\VS{4}Car les flèches du Tout-Puissant sont sur moi, mon âme en boit le venin ; les terreurs\FTNT{Job 30:15 ; Ps. 88:16-17.} de Dieu se rangent en bataille contre moi\FTNT{Job 19:12 ; Ps. 38:2-3.}.
\VS{5}L'âne sauvage\FTNT{Job 39:8.} brait-il auprès de l’herbe ? Le bœuf mugit-il auprès de son fourrage ?
\VS{6}Mange-t-on sans sel ce qui est fade ? Trouve-t-on du goût dans un blanc d’œuf ?
\VS{7}Ce que mon âme voudrait ne pas toucher, c'est là ma nourriture, si dégoûtante soit-elle !
\VS{8}Oh ! Puisse ma prière s'accomplir et Dieu me donner ce que j'attends !
\VS{9}Qu’il plaise à Dieu de me réduire en poussière, qu’il laisse aller sa main pour m’achever !
!\FTNT{Job 7:16; 9:21; 10:1; cp. No. 11:15; 1R. 19:4; Jon. 4:3, 8.}
\VS{10}Mais j’ai encore cette consolation, quoique la douleur me consume, et qu’elle ne m’épargne point, je n'ai pas transgressé les paroles du Saint.
\VS{11}Quelle est ma force pour que j’espère, et quelle est ma fin pour que je prenne patience ?
\VS{12}Ma force est-elle une force de pierre ? Ma chair est-elle d'airain ?
\VS{13}Ne suis-je pas sans secours, et le salut n'est-il pas loin de moi ?
\VS{14}A celui qui souffre, est due la compassion de son ami ; mais il a abandonné la crainte. \FTNT{Ps. 19:10.} du Tout-Puissant\FTNT{Pr. 17:17.}.
\VS{15}Mes frères m'ont trompé comme un torrent, comme le lit des torrents qui passent\FTNT{Ps. 38:12; Ps 41:10; Ps 69:9; Jé. 15:19.}.
\VS{16}Les glaçons en troublent le cours, la neige s'y cache ;
\VS{17}mais au temps de la sécheresse, ils tarissent, et dans les chaleurs, ils disparaissent de leur place.
\VS{18}Les caravanes se détournent de leur route, elles montent dans le désert et périssent.
\VS{19}Les caravanes de Théma\FTNT{Ge. 25:15.} fixent le regard, les voyageurs de Séba\FTNT{1R. 10:1; Ps. 72:10; Ez. 27:22-23.} s'attendent à eux;
\VS{20}ils sont honteux d'avoir eu cette confiance, ils restent confondus quand ils arrivent.
\VS{21}Certes, vous m'êtes devenus inutiles ; vous voyez mon angoisse, et vous en avez horreur !\FTNT{Job 19:13 ; Ps. 31:12.}
\VS{22}Mais vous ai-je dit : Donnez-moi quelque chose, et de vos biens, faites des présents en ma faveur ? 
\VS{23}délivrez-moi de la main de l'ennemi, et rachetez-moi de la main des violents ?
\VS{24}Instruisez-moi, et je me tairai ; faites-moi comprendre en quoi je me suis égaré.
\VS{25}Ô combien sont fortes les paroles de vérité ! Mais que veut censurer votre argumentation ?
\VS{26}Voulez-vous donc blâmer ce que j'ai dit, et ne voir que du vent dans les paroles d'un homme désespéré ?\FTNT{Ec. 9:16.}
\VS{27}Vous vous jetez même sur un orphelin, vous persécutez votre ami.
\VS{28}Regardez-moi, je vous prie ! Et voyez si je vous mens en face ?
\VS{29}Revenez\FTNT{Job 17:10.} donc, soyez sans injustice; revenez, et reconnaissez mon innocence\FTNT{Job 27:5-6 ; 34:5 ; cp. Job 23:10 ; 42:1-6.}.
\VS{30}Y a-t-il de l'injustice dans ma langue ? et mon palais ne sait-il pas discerner le mal ? 
\Chap{7}
\VerseOne{}N'y a-t-il pas un temps de guerre limité à l'homme sur la terre ? Et ses jours ne sont-ils pas comme les jours d'un mercenaire ?
\VS{2}Comme un esclave, il soupire après l'ombre, comme un mercenaire\FTNT{Es. 16:14.}, il attend son salaire\FTNT{Ps. 39:5.}.
\VS{3}Ainsi j'ai reçu en partage des mois en vain, et l'on m'a assigné des nuits de peine\FTNT{Ps. 6:6.}.
\VS{4}Si je suis couché, je dis : Quand me lèverai-je ? Quand finira la nuit ? Et je suis rassasié d'agitations jusqu'au point du jour\FTNT{De. 28:67.}.
\VS{5}Ma chair se couvre de vers et d'une croûte terreuse, ma peau se crevasse et coule.
\VS{6}Mes jours sont plus rapides que la navette du tisserand, ils se consument sans espoir !\FTNT{Es. 38:12 ; Job 9:25 ; 17:11; Ja. 4:14.}
\VS{7}Souviens-toi que ma vie est un souffle ! Et que mes yeux ne reverront plus le bonheur\FTNT{Es. 40:6 ; Ps. 78:39 ; Ps. 89:48 ; Ps. 102: 12 ; Ps. 103:15 ; Job 8:9 ; Job 14:1-2 ; 1P. 1:24.}.
\VS{8}L'œil de ceux qui me regarndent ne me verra plus ; tes yeux seront sur moi, et je ne serai plus.
\VS{9}La nuée se dissipe et s'en va, ainsi celui qui descend au scheol\FTNT{cp. Ha. 2:5 ; Lu. 16:23.} ne remontera pas\FTNT{Job 10:21-22 ; Job 14:7-14.};
\VS{10}il ne reviendra plus dans sa maison, et le lieu qu'il habitait ne le reconnaîtra plus\FTNT{Ps. 37:35-36 ; Ps. 103:16 ; Job 10:21.}.
\VS{11}C'est pourquoi, je ne retiendrai pas ma bouche, je parlerai dans l'angoisse de mon esprit, je me plaindrai dans l'amertume de mon âme\FTNT{Job 10:1.}.
\VS{12}Suis-je une mer ? Suis-je un monstre marin, pour que tu poses autour de moi des gardes ?
\VS{13}Quand je dis : Mon lit me consolera, ma couche calmera ma plainte,
\VS{14}alors tu me terrifies par des songes, et tu m'épouvantes par des visions.
\VS{15}C'est pourquoi je choisirais d'être étranglé, et de mourir, plutôt que de conserver mes os.
\VS{16}Je les méprise !… Je ne vivrai pas toujours… Laisse-moi car mes jours sont un souffle\FTNT{Job 10:20.}.
\VS{17}Qu'est-ce que l'homme pour que tu en fasses tant de cas, pour que tu poses ta main sur son cœur,\FTNT{Ps. 8:5 ; Ps. 144:3 ; Hé. 2:6.}
\VS{18}pour que tu le visites tous les matins, pour que tu l'éprouves\FTNT{Job 23:10.} à chaque instant ?
\VS{19}Quand finiras-tu de me regarder? Ne me lâcheras-tu pas, pour que j'avale ma salive ?\FTNT{Job 9:18.}
\VS{20}J'ai péché ; que te ferai-je, gardien des hommes ?  \FTNT{1 Ti. 4:10.}  Pourquoi m'as-tu mis en butte à tes coups, et pourquoi suis-je à charge à moi-même ?
\VS{21}Et pourquoi ne pardonnes-tu pas mon péché, et ne fais-tu pas passer mon iniquité ? Car je vais maintenant me coucher dans la poussière ; tu me chercheras, et je ne serai plus.
\Chap{8}
\TextTitle{Premier discours de Bildad}
\VerseOne{}Bildad de Schuach prit la parole et dit :
\VS{2}Jusqu'à quand parleras-tu ainsi, et les paroles de ta bouche seront-elles un vent impétueux ?\FTNT{Job 15:2.}
\VS{3}Dieu renverserait-il le droit, et le Tout-puissant renverserait-il la justice ? \FTNT{Cp. Ge. 18:25.} ?\FTNT{De. 32:4 ; Job 34:12 ; Da. 9:14 ; 2 Ch. 19:7.}
\VS{4}Si tes fils ont péché contre lui, il les a livrés à leur crime.
\VS{5}Mais toi si tu cherches Dieu, si tu demandes grâce au Tout-Puissant ;\FTNT{Cp. Job 5:17-27.}
\VS{6}si tu es pur et droit, il veillera certainement sur toi, il rendra le bonheur à la demeure de ta justice ;
\VS{7}tes commencements\FTNT{Za. 4:10.} auront été peu de chose, et ta fin sera bien plus grande.\FTNT{Job 42:12.}
\VS{8}Interroge ceux des générations précédentes, applique-toi à l'expérience de leurs pères.\FTNT{De. 4:32 ; De. 32:7.}
\VS{9}Car nous sommes d'hier, et nous ne savons rien, parce que nos jours sur la terre ne sont qu'une ombre.\FTNT{Ps. 102:12 ; Ps. 144:41 ; Ch. 29:15.}
\VS{10}Ils t'instruiront, ils te parleront, ils tireront de leur cœur ces discours :
\VS{11}Le roseau croît-il sans marais ? Le jonc pousse-t-il sans eau ?
\VS{12}Il est encore en sa verdure, sans qu'on le coupe, il sèche plus vite que toutes les herbes.\FTNT{Cp. Jé. 17:5-8 ; Ps. 129:6.}
\VS{13}Ainsi est la voie de tous ceux qui oublient Dieu\FTNT{Ps. 9:18.}, et l'espérance de l'impie périra\FTNT{Ps. 1:4 ; Ps. 112:10 ; Pr. 10:28 ; Job 11:20 ; Job 27:8.}.
\VS{14}Sa confiance est brisée, son soutien est une toile d'araignée.
\VS{15}Il s'appuie sur sa maison, et elle ne tient pas ; il s'y cramponne, et elle ne reste pas debout.
\VS{16}Dans toute sa vigueur, en plein soleil, il étend ses rameaux sur son jardin,
\VS{17}mais ses racines s'entrelacent parmi des monceaux de pierres, il pénètre dans les rochers.
\VS{18}S'Il l'ôte de sa place, celle-ci le renie, disant je ne t'ai pas connu ! 
\VS{19}Telle est la joie que ses voies lui procurent. Puis sur le même sol, d'autres s'élèvent après lui.
\VS{20}Dieu ne rejette pas l'homme intègre, il ne soutient pas la main des méchants.\FTNT{Job 4:7.}
\VS{21}Il remplira encore ta bouche de cris de joie, et tes lèvres de chants d'allégresse.\FTNT{Ps. 126:2.}
\VS{22}Ceux qui te haïssent seront revêtus de honte, et la tente des méchants ne sera plus. \FTNT{Ps. 35:26 ; Ps. 109:29.}
\Chap{9}
\TextTitle{Réponse de Job}
\VerseOne{}Job prit la parole et dit :
\VS{2}Certainement, je sais qu'il en est ainsi ; et comment l'homme mortel se justifierait-il devant Dieu ? \FTNT{Ha. 2:4 ; Ga. 3:11 ; Ro. 1:17 ; Hé. 10:38.} devant Dieu ?\FTNT{Ps. 25:4 ; Ps. 143:2 ; Job 15:14-16 ; Da. 9:11 ; Ro 3:19.}
\VS{3}S'il veut plaider avec lui, il ne lui répondra pas une fois sur mille. \FTNT{Es. 45:9-10.}
\VS{4} Dieu est sage de coeur, et puissant en force. Qui est-ce qui s'est opposé à lui, et s'en est bien trouvé ? \FTNT{Job 12:13 ; Job 36:5 ; Job 37:23.}
\VS{5}Il transporte les montagnes, et quand il les renverse dans sa fureur, elles n'en connaissent rien.\FTNT{Ps. 144:5.}
\VS{6}Il remue la terre de sa place, et ses piliers sont ébranlés.\FTNT{Ag. 2:6, 21 ; Hé. 12:26.}
\VS{7}Il commande au soleil, et le soleil ne se lève pas ; et il met un sceau sur les étoiles.\FTNT{Jos. 10:12.}
\VS{8} C'est lui seul qui étend les cieux\FTNT{Ge 1:6-8 ; Es. 44:24; Es. 51:13 ; Ps. 104:2.},qui marche sur les hauteurs de la mer\FTNT{Cp. Mt. 14:25.}.
\VS{9}Il a fait la grande ourse, l'orion, les pléiades, et les étoiles des régions australes.\FTNT{Ge. 1:16 ; Am. 5:8 ; Ps. 89:12 ; Job 38:31-32.}
\VS{10}Il fait de grandes choses qu'on ne peut sonder, des merveilles sans nombre.\FTNT{Ps. 86:10 ; Ps. 139:6, 17-18 ; Job 5:9 ; Job 37:5.}
\VS{11}Voici, il passe près de moi, et je ne le vois pas ; il passe encore, et je ne l'aperçois pas.\FTNT{Job 23:8-9 ; 35:14.}
\VS{12}S'il enlève, qui l'en détournera? Qui lui dira : Que fais-tu ?\FTNT{Es. 45: 9-10 ; Da. 4:35 ; Ro. 11:33-35.}
\VS{13}Dieu ne revient pas sur sa colère ; sous lui s'inclinent les appuis de l'orgueil.\FTNT{Job 26:12; Cp. Es. 30:7.}
\VS{14}Combien moins lui répondrais-je, moi et comment choisirais-je mes paroles contre lui ? 
\VS{15}Quand je serais juste, je ne répondrais pas ; je demanderais grâce à mon juge.\FTNT{Job 23:1-7.}
\VS{16}Si je l'invoque et qu'il me réponde, ne croirais-je pas qu'il ait écouté ma voix,
\VS{17}lui qui m'assaille comme par une tempête, qui multiplie mes plaies sans motif,\FTNT{Job 6:29.}
\VS{18}qui ne me permet pas de reprendre haleine ; qui me rassasie d'amertume.\FTNT{Job 7:19.}
\VS{19}S'il est question de savoir qui est le plus fort ; voilà, il est fort ; et s'il est question d'aller en justice, qui est-ce qui m'y fera comparaître ? 
\VS{20}Si je me justifie, ma propre bouche me condamnera ; si je me fais parfait, il me convaincra d'être coupable.
\VS{21}Je suis innocent ! Je ne me soucie pas de vivre, je méprise ma vie.\FTNT{Job 10:1.}
\VS{22}Tout se vaut! C'est pourquoi j'ai dit: Il détruit l'innocent comme l'impie. \FTNT{ Cp. Ez. 21:3 ; Ec. 9:2-3 ; Mt 5:45.}
\VS{23}Au moins si le fléau dont il frappe faisait mourir tout aussitôt ; mais il se rit de l'épreuve des innocents. 
\VS{24}[C'est par lui que] la terre est livrée entre les mains du méchant ; c'est lui qui couvre la face des juges de la [terre] ; et si ce n'est pas lui, qui est-ce donc ? 
\VS{25}Or mes jours vont plus vite qu'un courrier ; ils s'en fuient sans avoir vu le bonheur ;\FTNT{Job 7:6-7.}
\VS{26}ils passent comme les navires de roseaux, comme l'aigle qui fond sur sa proie.
\VS{27}i je dis : J'oublierai ma plainte, je renoncerai à ma colère, je me fortifierai; 
\VS{28}Je suis épouvanté de tous mes tourments. Je sais que tu ne me jugeras pas innocent. .\FTNT{Cp. Ps. 130:3.}
\VS{29}Je serai jugé coupable ; pourquoi travaillerais-je en vain ?
\VS{30}Quand je me laverais dans de l'eau de neige, et que je nettoierais mes mains dans la pureté, \FTNT{Jé. 2:22.}
\VS{31}tu me plongerais dans le fossé, et mes vêtements m'auraient en horreur.
\VS{32}Car il n'est pas comme moi un homme, pour que je lui réponde, [et] que nous allions ensemble en jugement. .\FTNT{Es. 45:9 ; Jé 49:19 ; Ec. 6:10; Ro. 9:20.}
\VS{33}Mais il n'y a personne qui prend connaissance de la cause qui serait entre nous, et qui pose la main sur nous deux. \FTNT{Cp. 1 S. 2:25.}
\VS{34}Qu'il ôte donc sa verge de dessus moi, et que la frayeur que j'ai de lui ne me trouble plus. ;
\VS{35}Je parlerai, et je ne le craindrai pas ; mais dans l'état où je suis je ne suis plus à moi-même. 
\Chap{10}
\VerseOne{}Mon âme a pris en dégoût la vie ! Je laisserai aller ma plainte, je parlerai dans l'amertume de mon âme.
\VS{2}Je dirai à Dieu : Ne me condamne pas ; montre-moi pourquoi tu plaides contre moi ?
\VS{3}Te plais-tu à m'opprimer, et à dédaigner l'ouvrage de tes mains, et à bénir les desseins des méchants\FTNT{Es. 64:7-8.} ?
\VS{4}As-tu des yeux de chair ? Vois-tu comme voit un homme mortel?
\VS{5}Tes jours sont-ils comme les jours de l'homme mortel ? Tes années sont-elles comme les jours de l'homme, 
\VS{6}Que tu recherches mon iniquité, et que tu t'informes de mon péché,
\VS{7}tu sais que je n'ai point commis de crime, et qu'il n'y a personne qui me délivre de ta main?
\VS{8}Tes mains m'ont formé, et elles ont rangé toutes les parties de mon corps; et tu me détruirais !\FTNT{Ge. 2:7 ; Ps. 119:73 ; Ps. 139:14-15.} !
\VS{9}Souviens-toi, je te prie, que tu m'as formé comme de la boue, et que tu me feras retourner en poudre?
\VS{10}Ne m'as-tu pas coulé comme du lait ? et ne m'as-tu pas fait cailler comme un fromage ?
\VS{11}Tu m'as revêtu de peau et de chair, et tu m'as composé d'os et de nerfs;
\VS{12}Tu m'as donné la vie, et tu as usé de miséricorde envers moi, et [par] tes soins continuels tu as gardé mon esprit.
\VS{13}Et cependant tu gardais ces choses en ton cœur ; mais je connais que cela était devant toi. 
\VS{14}Si j'ai pèche, tu m'observes, et tu ne me tiens pas pour innocent de mon iniquité.
\VS{15}Si j'agis méchamment, malheur à moi ! si je suis juste, je n'en lève pas la tête plus haut. Je suis rempli d'ignominie ; mais regarde mon affliction. 
\VS{16}Si je redresse la tête, tu me poursuis comme à un lion, et tu multiplie tes exploits contre moi; \FTNT{Zq. 38:13 ; La. 3:10.}.
\VS{17}Tu renouvelles tes témoins contre moi, et ton indignation augmente contre moi. De nouvelles troupes toutes fraîches viennent contre moi.
\VS{18}Mais pourquoi m'as-tu fait sortir du sein de ma mère? J'aurais expiré, et aucun œil ne m'aurait vu ;
\VS{19}et j'aurais été comme n'ayant jamais été, et j'aurais été porté du ventre de ma mère au tombeau.
\VS{20}Mes jours ne sont-ils pas en petit nombre ? Cesse donc et retire-toi de moi, et que je me renforce un peu. .
\VS{21}Avant que j'aille au lieu d'où je ne reviendrai plus, en la terre de ténèbres et de l'ombre de la mort,
\VS{22}terre d'une grande obscurité, comme les ténèbres de l'ombre de la mort, où il n'y a aucun ordre, et où rien ne luit que des ténèbres. 
\Chap{11}
\TextTitle{Première accusation de Tsophar}
\VerseOne{}Tsophar de Naama prit la parole et dit :
\VS{2}Ne répondra-t-on point à tant de discours, et suffira-t-il d'être un grand parleur pour être justifié ?
\VS{3}Tes vains feront-ils taire les gens ? Et quand tu te seras moqué, n'y aura-t-il personne qui te fasse honte ?
\VS{4}Car tu as dit : Ma doctrine est pure, et je suis sans tache devant tes yeux. 
\VS{5}Mais je voudrais que Dieu parle, et qu'il ouvre sa bouche pour te répondre; ,
\VS{6}Qu'il te montre les secrets de sa sagesse, de son immense sagesse; et que tu reconnaisse que Dieu oublie une partie de ton iniquité. .
\VS{7}Trouveras-tu Dieu en le sondant ? Connaîtras-tu parfaitement le Tout-puissant ? 
\VS{8}Ce sont les hauteurs des cieux : Qu'y feras-tu ? C'est plus profond que le scheol : Qu'y connaîtras-tu ?
\VS{9}Son étendue est plus longue que la terre, et plus large que la mer.
\VS{10}S'il remue, et qu'il resserre, ou qu'il rassemble, qui l'en détournera ?
\VS{11}Car il connaît les hommes vicieux, il discerne par le regard les coupables\FTNT{Ps. 10:11-14 ; Ps. 35:22.}.
\VS{12}Mais l'homme vide de sens devient intelligent, quoique l'homme naisse comme un ânon sauvage\FTNT{Ec. 3:18.}.
\VS{13}Si tu disposes ton cœur, et que tu étendes tes mains vers lui,
\VS{14}Si tu éloignes de toi l'iniquité qui est en ta main, et si tu ne permets pas que la méchanceté habite dans tes tentes ; 
\VS{15}Alors certainement tu pourras élever ton visage sans tache ; tu seras ferme et tu ne craindras rien;
\VS{16}tu oublieras tes peines, tu t'en souviendras comme des eaux écoulées.
\VS{17}La vie se lèvera pour toi plus brillante que le midi, et l'obscurité même sera comme le matin\FTNT{Ps. 37:6 ; Ps. 112:4.}.
\VS{18} Tu seras plein de confiance, parce qu'il y aura de l'espérance pour toi; tu creuseras, et tu reposeras sûrement. \FTNT{Lé. 26:6 ; Ps. 3:6 ; Pr. 3:24.}.
\VS{19}Tu te coucheras, et il n'y aura personne qui t'épouvante, et plusieurs te feront la cour. 
\VS{20}Mais les yeux des méchants seront consumés; tout refuge leur sera ôté et toute leur espérance sera de rendre l'âme !
\Chap{12}
\TextTitle{Réplique de Job}
\VerseOne{}Job reprit la parole, et dit :
\VS{2}On dirait vraiment que vous êtes tout un peuple, et qu'avec vous doit mourir la sagesse.
\VS{3}J'ai du bon sens aussi bien que vous, et je ne vous suis point inférieur ; et qui ne sait de telles choses ?
\VS{4}Je suis pour mes amis un objet de raillerie, quand je m'écrie à Dieu pour qu'il me réponde; on se moque d'un homme qui est juste et droit.
\VS{5} Mépris au malheur! telle est la pensée des heureux; le mépris est réservé à ceux dont le pied chancelle !
\VS{6}Elles sont en paix, les tentes des pillards, et toutes les sécurités sont pour ceux qui irritent Dieu, qui se font un dieu de leur bras. \FTNT{ Jé. 12:1 ; Ps. 73:12.}.
\VS{7}Mais interroge donc les bêtes, et elles t'instruiront, ou les oiseaux des cieux, et ils te l'annonceront ;
\VS{8}Ou parle à la terre, et elle t'enseignera ; même les poissons de la mer te le raconteront ; 
\VS{9}Qui est-ce qui ne sait toutes ces choses; que c'est la main de Yahweh qui a fait cela ?
\VS{10} Qu'il tient en sa main, l'âme de tout ce qui vit, et l'esprit de toute chair humaine,
\VS{11}L'oreille ne discerne-t-elle pas les discours, ainsi que le palais savoure les aliments ?
\VS{12}La sagesse est dans les vieillards, et l'intelligence est le fruit d'une longue vie.
\VS{13}Mais en Dieu est la sagesse et la force ; à lui appartient le conseil et l'intelligence. \FTNT{Da. 2:20.}.
\VS{14}Voici, il démolit, et on ne rebâtit pas ; il enferme un homme, et on ne lui ouvre pas\FTNT{Es. 22:22 ; Ap. 3:7.}.
\VS{15}Voilà, il retient les eaux, et tout devient sec ; il les lâche, et elles bouleversent la terre.
\VS{16} En lui résident la puissance et la sagesse; de lui dépendent celui qui s'égare et celui qui égare.
\VS{17} Il emmène dépouillés les conseillers, et il met hors de sens les juges. \FTNT{2 S. 15:31 ; 2 S. 17:14-23 ; Es. 19:12 ; Es. 29:14 ; 1 Co. 1:19.}.
\VS{18}Il rend impuissant le gouvernement des rois, et lie de chaînes leurs reins. 
\VS{19}Il fait marcher pieds nus les sacrificateurs ; et il renverse les puissants.
\VS{20}Il ôte la parole à ceux qui sont les plus assurés en leurs discours, et il prive de sens les anciens.
\VS{21}Il verse le mépris sur les nobles ; il relâche la ceinture des forts\FTNT{Es. 40:23.}.
\VS{22}Il met en évidence les choses qui étaient cachées dans les ténèbres, et il produit en lumière l'ombre de la mort. \FTNT{Ps. 139:11-12 ; Ec. 12:16 ; Mt. 10:26 ; 1 Co. 4:6.}.
\VS{23} Il multiplie les nations, et les fait périr ; il répand çà et là les nations, et puis il les ramène. 
\VS{24}Il ôte la raison aux Chefs des peuples de la terre, et les fait errer dans les déserts où il n'y a point de chemin;
\VS{25}ils tâtonnent dans les ténèbres, sans aucune clarté, et il les fait chanceler comme des gens ivres. 
\Chap{13}
\VerseOne{}Voici, mon œil a vu toutes ces choses, mon oreille l'a entendu et compris.
\VS{2}Comme vous les savez, je les sais aussi ; je ne vous suis pas inférieur. 
\VS{3}Mais je veux parler au Tout-Puissant, je veux plaider auprès de Dieu. .
\VS{4}Et certes vous inventez des mensonges ; vous êtes tous des médecins inutiles.
\VS{5}Plaît à Dieu que vous demeuriez entièrement dans le silence ; et cela vous sera réputé à sagesse. \FTNT{Pr. 17:28.}.
\VS{6}Ecoutez donc maintenant ma cause, et soyez attentifs à la défense de mes lèvres.
\VS{7}Tiendrez-vous des discours injustes en faveur de Dieu, et, pour le défendre, direz-vous des mensonges?
\VS{8}Ferez-vous acception de personnes en sa faveur? Prétendrez-vous plaider pour Dieu? 
\VS{9}S'il vous sonde, vous trouvera-t-il bon ? Comme on trompe un homme, le tromperez-vous? 
\VS{10}Certainement il vous reprendra, si même en secret vous faites acception de personnes.
\VS{11}Sa majesté ne vous épouvantera-t-elle pas ? Et sa frayeur ne tombera-t-elle pas sur vous ? 
\VS{12}Vos discours mémorables sont des sentences de cendre, et vos éminences sont des éminences de boue. 
\VS{13}Taisez-vous devant moi, et que je parle ; et il m'arrivera ce qui pourra. 
\VS{14}Pourquoi porterais-je ma chair entre mes dents, et tiendrais-je mon âme entre mes mains ? \FTNT{Jg. 12:3 ; 1 S. 19:5.}.
\VS{15}Voilà, qu'il me tue, je ne cesserai pas d'espérer en lui ; et je défendrai ma conduite en sa présence.
\VS{16}Et qui plus est, il sera lui-même mon salut ; mais l'hypocrite ne viendra point devant sa face. \FTNT{Ps. 1:5.}.
\VS{17}Ecoutez attentivement mes paroles, et prêtez l'oreille à ce que je vais vous déclarer. 
\VS{18}Voici, j'ai préparé ma cause. Je sais que je serai justifié.
\VS{19}Qui est-ce qui veut disputer contre moi ? car maintenant si je me tais, je mourrai. 
\VS{20}Seulement ne me fais pas ces deux choses, et alors je ne me cacherai point devant ta face :
\VS{21}Retire ta main de dessus moi, et que tes terreurs ne me troublent pas.
\VS{22}Puis appelle-moi, et je répondrai ; ou bien je parlerai, et tu me répondras. 
\VS{23}Combien ai-je d'iniquités et de péchés ? Montre-moi mon crime et mon péché. 
\VS{24}Pourquoi caches-tu ta face, et me tiens-tu pour ton ennemi ?
\VS{25}Déploieras-tu tes forces contre une feuille que le vent emporte ? Poursuivras-tu du chaume tout sec \FTNT{1 S. 24:15.} ?
\VS{26}Que tu écrives contre moi des choses amères, et que tu me fasses porter la peine des péchés de ma jeunesse? \FTNT{Ps. 25:7.} ?
\VS{27}Que tu mettes mes pieds aux ceps, et observes tous mes chemins, et que tu suives les traces de mes pieds,
\VS{28}Quand mon corps s'en va par pièces comme du bois vermoulu, et comme une robe que la teigne a rongée? 
\Chap{14}
\VerseOne{}L'homme né de la femme est de courte vie, et rassasié d'agitations. \FTNT{Ps. 102:12 ; Ps. 103:15 ; Ps. 144:4 ; Ja. 4:14.}.
\VS{2}Il sort comme une fleur, puis il est coupé, et il s'enfuit comme une ombre qui ne s'arrête pas. \FTNT{Es. 40:6 ; Ps. 90:61 ; 1 Pi. 1:24.}.
\VS{3}Cependant tu as ouvert tes yeux sur lui, et tu me conduis en justice avec toi.
\VS{4}Qui est-ce qui tirera le pur de l'impur ? Personne. \FTNT{Es. 48:8 ; Pr. 22:15.}.
\VS{5}Les jours de l'homme sont déterminés, le nombre de ses mois est entre tes mains, tu lui as prescrit ses limites, et il ne passera point au delà.
\VS{6}Retire-toi de lui, afin qu'il ait du relâche, jusqu'à ce que comme un mercenaire il ait achevé sa journée.
\VS{7}Car si un arbre est coupé, il y a de l'espérance, et il poussera encore, et ne manquera pas de rejetons ; 
\VS{8}quoique sa racine ait vieilli dans la terre, et que son tronc soit mort dans la poussière;
\VS{9}Dès qu'il sent l'eau il regerme, et produit des branches, comme un arbre nouvellement planté. 
\VS{10}Mais l'homme meurt et perd toute sa force; il expire et puis où est-il ?
\VS{11}Les eaux s'écoulent de la mer, et une rivière s'assèche, et tarit ;
\VS{12} Ainsi l'homme est couché par terre, et ne se relève plus ; jusqu'à ce qu'il n'y ait plus de cieux, ils ne se réveillera plus, et ne sera pas réveillé de son sommeil. 
\VS{13}Oh que tu me caches dans le scheol, que tu me gardes à l'abri jusqu'à ce que ta colère soit passée, que tu me donnes un temps arrêté, après lequel tu te souviendrais de moi !
\VS{14}Si un homme meurt, revivra-t-il ? Tous les jours de ma détresse, j'attendrais jusqu'à ce que mon état vînt à changer.
\VS{15} Tu appellerais, et moi je te répondrais, tu ne dédaignerais pas l'ouvrage de tes mains.
\VS{16}Mais maintenant tu comptes mes pas, et tu n'exceptes rien de mon péché. \FTNT{Ps. 56:9 ; Ps. 139:2-4 ; Pr. 5:21.} ;
\VS{17}Mes péchés sont scellés dans un sac, et tu as cousu ensemble mes iniquités. \FTNT{Os. 13:12.}.
\VS{18}Car comme une montagne s'éboule en tombant, et comme un rocher est transporté de sa place ; 
\VS{19} et comme les eaux minent les pierres, et entraînent par leur débordement la poussière de la terre, avec tout ce qu'elle a produit, tu fais ainsi périr l'attente de l'homme. 
\VS{20}Tu te montres toujours plus fort que lui, et il s'en va, et lui ayant défiguré le visage, tu le renvoies.
\VS{21}Quand ses fils sont honorés, il n'en sait rien ; et quand ils sont abaissés, il ne s'en aperçoit pas.
\VS{22}Seulement sa chair sur lui, a de la douleur, et son âme en lui s'afflige. 
\Chap{15}
\TextTitle{Deuxième discours d'Eliphaz}
\VerseOne{}Eliphaz de Théman prit la parole et dit :
\VS{2}Un homme sage profère-t-il dans ses réponses une science aussi légère que le vent, des opinions vaines ? Remplit-il son ventre du vent d'orient ?
\VS{3}Contestant avec des discours qui ne servent de rien, et avec des paroles dont on ne peut tirer aucun profit ?
\VS{4}Certainement tu abolis la crainte de Dieu, et tu anéantis peu à peu la prière qu'on doit présenter à Dieu. 
\VS{5} Car ta bouche fait connaître ton iniquité, et tu as choisi un langage trompeur. 
\VS{6}C'est ta bouche qui te condamne, et non pas moi ; et tes lèvres témoignent contre toi. 
\VS{7}Es-tu le premier homme né ? Ou as-tu été formé avant les montagnes ? \FTNT{Ps. 90:2 ; Pr. 8:25.} ?
\VS{8}As-tu été instruit dans le conseil secret de Dieu, et renfermes-tu seul la sagesse ? \FTNT{Es. 40:13 ; Jé. 23:18 ; Ro. 11:34.} ?
\VS{9}Que sais-tu que nous ne sachions pas ? Quelle connaissance as-tu que nous n'ayons pas ?
\VS{10}Parmi nous, il y a des hommes à cheveux blancs, et des gens d'une fort grande vieillesse, il y en a même de plus âgés que ton père. 
\VS{11}Les consolations du Dieu te semblent-elles trop petites ? As-tu quelque chose de caché par-devant toi ? …
\VS{12}Pourquoi ton coeur s'emporte-il et pourquoi tes yeux clignent-ils ?
\VS{13}C'est contre Dieu que tu tournes ta colère, et que tu fais sortir de ta bouche de tels discours! 
\VS{14}Qu'est-ce que de l'homme, pour qu'il soit pur, et celui qui est né de femme, pour qu'il soit juste ? \FTNT{Ps. 14:3 ; Pr. 20:9 ; Ec. 7:20.} ?
\VS{15}Si Voici, Dieu ne se fie pas à ses saints, et les cieux ne sont pas purs à ses yeux,
\VS{16}Combien plus est abominable et corrompu, l'homme qui boit l'iniquité comme l'eau !  
\VS{17}Je t'enseignerai, écoute-moi, et je te raconterai ce que j'ai vu ,
\VS{18} savoir ce que les sages ont déclaré, et qu'ils n'ont point caché ; ce qu'ils avaient [reçu] de leurs pères.
\VS{19}Eux à qui seuls la terre a été donnée, et parmi lesquels l'étranger n'est point passé.
\VS{20}Toute sa vie, le méchant est tourmenté, et un petit nombre d'années sont réservées au malfaiteur.\FTNT{Es. 48:22 ; Es. 57:21.}.
\VS{21}Un cri de frayeur est dans ses oreilles ; au milieu de la paix [il croit] que le destructeur se jette sur lui. \FTNT{1 Th. 5:3.} ;
\VS{22}Il ne croit pas pouvoir sortir des ténèbres, car il voit la menace de   l’épée;
\VS{23}il court çà et là pour chercher son pain, il sait que le jour des ténèbres lui est préparé \FTNT{Ps. 109:10.}.
\VS{24}La détresse et l'angoisse l'épouvantent, elles l'assaillent comme un roi prêt à combattre ;
\VS{25}Parce qu'il a élevé sa main contre Dieu, et qu'il s'est levé contre le Tout-puissant ;
\VS{26}Il lui a sauté au collet, et sur l'épaisseur de ses gros boucliers. 
\VS{27}Parce que la graisse a couvert son visage, et qu'elle a fait des replis sur son ventre;
\VS{28} il habite des villes détruites, des maisons désertes, tout près de n'être plus que des monceaux de pierres. 
\VS{29}Et il ne s'enrichira plus, car ses biens ne subsisteront pas, et ses richesses ne se répandront pas sur la terre. 
\VS{30}Il ne pourra pas se détourner des ténèbres, la flamme desséchera ses rejetons, et Dieu le fera disparaître par le souffle de sa bouche.
\VS{31}S'il a confiance dans la vanité, il se trompe, car la vanité sera sa récompense.
\VS{32}Ce sera fait de lui avant son temps, ses branches ne reverdiront plus. 
\VS{33}On arrachera ses fruits non mûrs, comme à une vigne; on jettera sa fleur, comme celle d'un olivier. 
\VS{34}Car la famille des hypocrites est stérile, et le feu dévore les tentes de l'homme corrompu.
\VS{35}Ils conçoivent le travail, et ils enfantent la misère, et machinent dans le cœur des fraudes. \FTNT{Es. 59:4 ; Os. 10:13.}.
\Chap{16}
\TextTitle{Réponse de Job}
\VerseOne{}Job répondit, et dit :
\VS{2}J'ai souvent entendu de pareils discours ; vous êtes tous des consolateurs fâcheux.
\VS{3}Y aura-t-il une fin à [ces] paroles de vent ? Qu'est-ce qui t'irrite, que tu répondes ?
\VS{4}Parlerais-je comme vous faites, si vous étiez en ma place ; accumulerais-je des paroles contre vous, ou secouerais-je ma tête contre vous ? 
\VS{5}Je vous fortifierais de ma bouche, et le mouvement de mes lèvres vous soulagerait.
\VS{6}Si je parle, ma douleur ne sera point soulagée. Si je me tais, en sera-t-elle diminuée?
\VS{7}Maintenant il m'a épuisé... Tu as dévasté toute ma famille, ;
\VS{8}Tu m'as tout couvert de rides, qui sont un témoignage des maux que je souffre ; et il s'est élevé en moi une maigreur qui en rend aussi témoignage sur mon visage. 
\VS{9}Sa fureur me déchire, il se déclare mon ennemi, il grince des dents contre moi, et étant devenu mon ennemi, il étincelle des yeux contre moi.
\VS{10}Ils ouvrent contre moi leur bouche; ils me frappent à la joue pour m'outrager; ils se réunissent tous ensemble contre moi. 
\VS{11}Dieu m'a livré à l'impie, et m'a jeté entre les mains des méchants. 
\VS{12}J'étais tranquille, et il m'a secoué, il m'a saisi par la nuque et m'a brisé, il m'a posé en butte à ses traits.
\VS{13}Ses archers m'ont environné, il me perce les reins, et ne m'épargne pas ; il répand mon fiel par terre. 
\VS{14}Il m'a brisé en me faisant plaie sur plaie, il a couru sur moi comme un homme fort.
\VS{15}J'ai cousu un sac sur ma peau, j'ai souillé ma tête dans la poussière\FTNT{Ps. 44 : 25 ; Ps. 119 : 25.},
\VS{16}J'ai le visage tout enflammé, à force de pleurer, et l'ombre de la mort est sur mes paupières, 
\VS{17}Quoiqu'il n'y ait point de violence dans mes mains, et que ma prière fut toujours pure.
\VS{18}Ô terre, ne cache pas mon sang, et qu'il n'y ait aucun lieu où s'arrête mon cri !
\VS{19}Mais maintenant voilà, mon témoin est aux cieux, mon témoin est dans les lieux élevés. \FTNT{Ap. 1:5 ; Ap. 3:14.}.
\VS{20}Mes amis se moquent de moi: c'est vers Dieu que mon œil se tourne en pleurant,
\VS{21}pour qu'il fasse justice entre l'homme et Dieu, entre le fils d'Adam et son semblable.
\VS{22}Car les années de mon compte arrivent à leur terme, et j'entre dans un sentier d'où je ne reviendrai plus. 
\Chap{17}
\VerseOne{}Mon souffle se perd, mes jours s'éteignent, le sépulcre m'attend.
\VS{2}Je suis environné de moqueurs, et mon œil veille toute la nuit au milieu de leurs insultes.
\VS{3}Dépose un gage, sois ma caution auprès de toi-même; car qui voudrait répondre pour moi? 
\VS{4}C'est pourquoi tu ne les élèveras pas\FTNT{De. 29:4 ; Mt. 11:25.}.
\VS{5}Celui qui trahit ses amis pour qu'ils soient pillés, les yeux de ses fils se consument.
\VS{6}On a fait de moi la fable des peuples, un être à qui l'on crache au visage.
\VS{7}Mon œil est obscurci par le chagrin, tous mes membres sont comme une ombre\FTNT{Ps. 6:7 ; Ps. 31:10.}.
\VS{8}Les hommes droits en sont consternés, et l'innocent est irrité contre l'impie. 
\VS{9}Toutefois le juste se tient ferme dans sa voie, et celui qui a les mains pures, se renforce.
\VS{10}Retournez donc vous tous, et revenez, je vous prie ; car je ne trouve pas de sages parmi vous. 
\VS{11}Mes jours sont passés; mes desseins, chers à mon cœur, sont renversés.…
\VS{12}On me change la nuit en jour, et on fait que la lumière se trouve proche des ténèbres !
\VS{13}Certes je n'ai plus à attendre que le sépulcre, qui va être ma maison ; j'ai dressé mon lit dans les ténèbres ;
\VS{14}J'ai crié à la fosse : tu es mon père ; et aux vers : vous êtes ma mère et ma Soeur. 
\VS{15}Où est donc mon espérance? Et mon espérance, qui pourrait la voir? \VS{16}Elle descendra au fond du sépulcre ; certes elle reposera avec moi dans la poussière. 
\Chap{18}
\TextTitle{Deuxième discours de Bildad}
\VerseOne{}Bildad de Schuach prit la parole et dit :
\VS{2}Quand finirez-vous ces discours ? Ecoutez, et puis nous parlerons.
\VS{3}Pourquoi sommes-nous regardés comme des bêtes, et sommes-nous stupides à vos yeux?
\VS{4}Ô toi qui déchires ton âme dans ta colère, la terre sera-t-elle abandonnée à cause de toi, et le rocher sera-t-il transporté de sa place ? 
\VS{5}Certainement, la lumière du méchant s'éteindra, et la flamme de son feu ne brillera pas\FTNT{Ps. 37:9-10.}.
\VS{6}La lumière sera ténèbres dans sa tente, et sa lampe sera éteinte au-dessus de lui. 
\VS{7}Les pas de sa force seront resserrés, et son propre conseil le renversera.
\VS{8}Car il est poussé dans le filet par ses propres pieds ; et il marche sur les mailles du filet.
\VS{9}Le piège le prend par le talon, et le filet s'empare de lui;
\VS{10}la corde est cachée dans la terre, et la trappe est sur son sentier.
\VS{11}Les terreurs l'assiègent de tous côtés, et le font courir ses pieds çà et là.\FTNT{Jé. 6:25 ; Jé. 46:5 ; Jé. 49:29.}.
\VS{12}Sa vigueur sera affamée, la détresse est à ses côtés.
\VS{13}Il dévorera les membres de son corps, il dévorera ses membres, le premier-né de la mort ! 
\VS{14} Les choses en quoi il mettait sa confiance seront arrachées de sa tente, et il sera conduit vers le Roi des épouvantements. 
\VS{15}On habitera dans sa tente, qui ne sera plus à lui; le soufre sera répandu sur sa demeure. 
\VS{16}Ses racines sèchent au dessous, et ses branches sont coupées en haut. 
\VS{17}Sa mémoire périt sur la terre, et on ne parle plus de son nom dans les places \FTNT{Ps. 109:13 ; Pr. 10:7.}.
\VS{18}Il est chassé de la lumière dans les ténèbres, et il est exterminé du monde. 
\VS{19}Il n'a ni lignée, ni descendance au milieu de son peuple, ni survivant dans ses habitations. \FTNT{Es. 14:20-22 ; Jé. 22:30 ; Ps. 37:28 ; Ps. 109:13.}. 
\VS{20}Ceux qui seront venus après lui, seront étonnés de sa ruine ; et ceux qui auront été avant lui en seront saisis d'horreur. 
\VS{21}Tel est le sort de l'injuste. Telle est la destinée de celui qui ne connaît pas Dieu. 
\Chap{19}
\TextTitle{Réponse de Job}
\VerseOne{}Job prit la parole et dit :
\VS{2}Jusqu'à quand affligerez-vous mon âme, et m'accablerez-vous de paroles ?
\VS{3} Voilà déjà dix fois que vous m'outragez: vous n'avez pas honte de me maltraiter? 
\VS{4}Vraiment si j'ai failli, ma faute demeure avec moi. 
\VS{5}Si réellement vous voulez vous élever contre moi et faire valoir mon opprobre contre moi, 
\VS{6}Sachez donc que c'est Dieu qui me renverse, et qui tend son filet autour de moi.
\VS{7}Voici je crie pour la violence qui m'est faite, et je ne suis pas exaucé ; je m'écrie, et il n'y a point de justice !
\VS{8}Il a fermé mon chemin, et je ne puis passer; il a mis des ténèbres sur mes sentiers. 
\VS{9}Il m'a dépouillé de ma gloire, il a ôté la couronne de ma tête.
\VS{10}Il m'a détruit de tous côtés, et je m'en vais ; il a arraché mon espérance comme un arbre.
\VS{11}Il s'est enflammé de colère contre moi, et m'a traité comme un de ses ennemis\FTNT{La. 2:5.}.
\VS{12}Ses troupes sont venues ensemble, et elles ont dressé leur chemin contre moi, et se sont campées autour de ma tente\FTNT{La. 2:22.}.
\VS{13}Il a éloigné de moi mes frères, et ceux qui me connaissaient se sont écartés comme des étrangers\FTNT{Ps. 88:9.} ;
\VS{14}mes proches m'ont abandonné, et ceux que je connaissais m'ont oublié.
\VS{15}Ceux qui séjournent dans ma maison et mes servantes m'ont traité comme un étranger; je suis devenu un inconnu pour eux. 
\VS{16}J'appelle mon serviteur, il ne me répond; de ma propre bouche, je le supplie en vain. 
\VS{17}Mon haleine est devenue dégoûtante à ma femme, et ma plainte aux fils de mes entrailles.
\VS{18}Je suis méprisé même par des enfants ; si je me lève, ils parlent contre moi.
\VS{19}Ceux que j'avais pour confidents m'ont en horreur, ceux que j'aimais se sont tournés contre moi\FTNT{Ps. 55:13-14.}.
\VS{20}Mes os sont attachés à ma peau et à ma chair ; et je me suis échappé avec la peau de mes dents\FTNT{La. 4:8.}.
\VS{21}Ayez pitié, ayez pitié de moi, vous, mes amis ! Car la main de Dieu m'a frappé.
\VS{22}Pourquoi, comme Dieu, me poursuivez-vous et n'êtes-vous pas rassasiés de ma chair \FTNT{Ps. 27:2.} ?
\VS{23}Oh! je voudrais que mes paroles fussent écrites quelque part, je voudrais qu'elles fussent inscrites dans un livre; 
\VS{24}Qu'avec un burin de fer et avec du plomb, elles fussent gravées sur le roc, pour toujours... 
\VS{25}Mais je sais que mon rédempteur est vivant, il demeurera le dernier sur la terre.
\VS{26}Et après que cette peau aura été détruite, hors de ma chair, je verrai Dieu \FTNT{Ps. 17:15.}.
\VS{27}Je le verrai moi-même, et mes yeux le verront, et non un autre. Mes reins se consument dans mon sein. 
\VS{28}Vous direz : Comment le poursuivrons-nous, et trouverons-nous en lui la cause de son malheur? 
\VS{29}Ayez peur de l'épée ; car la fureur [avec laquelle vous me persécutez], est [du nombre] des iniquités qui attirent l'épée ; c'est pourquoi sachez qu'il y a un jugement. 
\Chap{20}
\TextTitle{Dernier discours de Tsophar}
\VerseOne{}Tsophar de Naama prit la parole et dit :
\VS{2}C'est à cause de cela que mes pensées diverses me poussent à répondre, et que cette promptitude est en moi. 
\VS{3}J'ai entendu la correction dont tu veux me faire honte, mais mon esprit tirera de mon intelligence la réponse pour moi. 
\VS{4}Ne sais-tu pas que, de tout temps, depuis que Dieu a mis l'homme sur la terre, 
\VS{5}Le triomphe des méchants est de peu de durée, et la joie de l'hypocrite n'est que pour un moment \FTNT{Ps. 37:35-36.} ?
\VS{6}Quand son élévation monterait jusqu'aux cieux, et que sa tête atteindrait les nues,
\VS{7}il périra pour toujours comme ses ordures, et ceux qui le voyaient diront : Où est-il ?
\VS{8}Il s'envolera comme un songe, et on ne le trouvera plus ; il se  retirera comme une vision nocturne\FTNT{Ps. 73:19-20.} ;
\VS{9}l'œil qui le regardait ne le regardera plus, le lieu qu'il habitait ne le contemplera plus.
\VS{10}Ses fils rechercheront la faveur des pauvres, et ses mains restitueront ce que sa violence a ravi\FTNT{Ps. 109:10.}.
\VS{11}Ses os seront pleins de la punition, à  cause des péchés de sa jeunesse, et elle reposera avec lui dans la poussière.
\VS{12}Le mal était doux à sa bouche, il le cachait sous sa langue,
\VS{13}s'il l'épargne, et ne le rejette point, mais le retient dans son palais ; 
\VS{14}Ce qu'il mangera se changera dans ses entrailles en un fiel d'aspic.
\VS{15}Il a englouti des richesses, il les vomira ; Dieu les arrachera de son ventre.
\VS{16}Il a sucé du venin d'aspic, la langue de la vipère le tuera.
\VS{17}Il ne verra plus les ruisseaux, les fleuves, les torrents de miel et de lait.
\VS{18}Il rendra le fruit de son travail, et ne l'avalera pas; il restituera à proportion de ce qu'il aura amassé, et ne s'en réjouira pas\FTNT{So. 2:10.}.
\VS{19}Car il a opprimé, délaissé les pauvres; il a pillé des maisons et ne les a pas rebâties.
\VS{20}Certainement il ne sentira pas dans son ventre la satisfaction de son avidité, et il ne sauvera rien de ce qu'il aura tant convoité \FTNT{Ec. 5:12.}.
\VS{21}Rien n'échappait à sa voracité, mais son bonheur ne durera pas.
\VS{22}Après que la mesure de ses biens aura été remplie, il sera dans la misère ; toutes les mains de ceux qu'il aura opprimés se jetteront sur lui.
\VS{23}Il arrivera que pour lui remplir le ventre, Dieu enverra contre lui l'ardeur de sa colère; il la fera pleuvoir sur lui et entrer dans sa chair.
\VS{24}S’il s’enfuit de devant les armes de fer, l’arc d’airain le transpercera.
\VS{25}Il arrachera la flèche, et elle sortira de son corps, et le fer étincelant, de son foie; les frayeurs de la mort viendront sur lui.
\VS{26}Toutes les ténèbres sont renfermées dans ses demeures les plus secrètes ; un feu qu'on n'aura point soufflé, le consumera ; l'homme qui restera dans sa tente sera malheureux\FTNT{Ps. 12:6.}.
\VS{27}Les cieux découvriront son iniquité, et la terre s'élèvera contre lui. 
\VS{28}Le revenu de sa maison sera emporté. Tout s'écoulera au jour de la colère de Dieu.
\VS{29}C'est là la portion que Dieu réserve à l'homme méchant, et l'héritage qu'il aura de Dieu pour ses discours.
\Chap{21}
\TextTitle{Réponse de Job}
\VerseOne{}Job répondit, et dit :
\VS{2}Ecoutez, écoutez mes discours, donnez-moi seulement cette consolation.
\VS{3}Supportez-moi, et je parlerai ; et quand j'aurai parlé, tu pourras te moquer.
\VS{4}Mais est-ce contre un homme que s'adresse ma plainte ? Et pourquoi mon âme ne serait-elle pas impatiente ?
\VS{5}Regardez-moi, soyez étonnés, et mettez la main sur la bouche.
\VS{6}Quand j'y pense, cela m'épouvante, et un frisson saisit mon corps.
\VS{7}Pourquoi les méchants vivent-ils, vieillissent-ils, et croissent-ils en puissance\FTNT{Jé. 12:1 ; Ha. 1:3 ; Mal. 3:14-15.}?
\VS{8}Leur postérité s'établit avec eux et en leur présence, leurs rejetons prospèrent sous leurs yeux.
\VS{9}Dans leurs maisons règne la paix, loin de la crainte ; la verge de Dieu ne vient pas les frapper.
\VS{10}Leurs taureaux sont féconds, leurs génisses conçoivent et n'avortent pas\FTNT{Ps. 144:13-14.}.
\VS{11}Ils laissent courir leurs enfants comme un troupeau, et les enfants prennent leurs ébats.
\VS{12}Ils chantent au son du tambourin et de la harpe, ils se réjouissent au son du chalumeau.
\VS{13}Ils passent leurs jours dans le bonheur, et ils descendent en un instant au scheol.
\VS{14}Ils disaient pourtant à Dieu : Éloigne-toi de nous, nous ne voulons pas connaître tes voies.
\VS{15}Qu'est-ce que le Tout-Puissant pour que nous le servions ? Que gagnerions-nous à lui adresser nos prières\FTNT{Ex. 5:2.} ?
\VS{16}Quoi donc ! Ne sont-ils pas en possession du bonheur entre leurs mains ? Loin de moi le conseil des méchants\FTNT{Ps. 1:1-2.} !
\VS{17}Mais arrive-t-il que la lampe des méchants s'éteigne, que la ruine vienne sur eux, que Dieu leur distribue leur part dans sa colère\FTNT{Ps. 11:5-6 ; Pr. 13:9.},
\VS{18}qu'ils soient comme la paille face au vent, comme la balle enlevée par le tourbillon\FTNT{Ps. 1:4.}?
\VS{19}Dieu réservera-t-il aux enfants du méchant la punition de ses violences? Il la leur rendra, et il le connaîtra!
\VS{20}Il verra de ses propres yeux sa ruine, c'est lui qui devrait boire la colère du Tout-Puissant\FTNT{Es. 51:17-22 ; Jé. 25:15 ; Ez. 23:31-32 ; Ap. 14:10.}.
\VS{21}Car que lui importe sa maison après lui, quand le nombre de ses mois est achevé ?
\VS{22}Enseignerait-on la science à Dieu, lui qui juge les esprits élevés\FTNT{Ro. 11:34 ; 1 Co. 2:16.} ?
\VS{23}L'un meurt au sein du bien-être, tout à son aise et en joie,
\VS{24}les flancs chargés de graisse, et ses os comme abreuvés de mœlle ;
\VS{25}l'autre meurt l'amertume dans l'âme, n'ayant jamais mangé ce qui est bon.
\VS{26}Et tous deux se couchent dans la poussière, tous deux deviennent couverts de vers.
\VS{27}Je sais bien quelles sont vos pensées, quels jugements iniques vous portez sur moi.
\VS{28}Vous dites : Où est la maison de l'homme puissant ? Où est la tente, demeure des méchants ?
\VS{29}Mais quoi ! N'avez-vous pas interrogé les voyageurs, et n'avez-vous pas appris par les rapports qu'ils vous on faits ?
\VS{30}Au jour du malheur, le méchant est épargné ; au jour de la colère, il échappe\FTNT{Pr. 16:4 ; Ec. 9:12.}.
\VS{31}Qui lui dit en face sa conduite ? Qui lui rend ce qu'il a fait ?
\VS{32}Il est porté au tombeau, et il veille encore sur sa tombe.
\VS{33}Les mottes de la vallée lui sont légères ; et tous après lui suivront la même voie, comme une multitude l'a déjà suivie.
\VS{34}Pourquoi donc m'offrir de vaines consolations ? Ce qui reste de vos réponses n'est que transgression.
\Chap{22}
\TextTitle{Dernier discours d'Eliphaz}
\VerseOne{}Eliphaz de Théman prit la parole et dit :
\VS{2}Un homme peut-il être utile à Dieu ? Mais le sage n'est utile qu'à lui-même.
\VS{3}Si tu es juste, est-ce à l'avantage du Tout-Puissant ? Si tu es intègre dans tes voies, qu'y gagne-t-il ?
\VS{4}Est-ce par crainte de toi qu'il te reprend, qu'il entre en jugement avec toi?
\VS{5}Ta méchanceté n'est-elle pas grande ? Tes iniquités ne sont-elles pas sans fin ?
\VS{6}Tu a pris sans raison le gage de tes frères, tu privais de leurs vêtements ceux qui étaient nus\FTNT{Ex. 22:21.} ;
\VS{7}tu ne donnais pas d'eau à boire à l'homme altéré, tu refusais du pain à l'homme affamé.
\VS{8}Le pays était à l'homme le plus fort, et le puissant s'y établissait.
\VS{9}Tu renvoyais les veuves à vide, les bras des orphelins étaient brisés.
\VS{10}C'est pour cela que tu es entouré de pièges, et que la terreur t'a saisi tout à coup.
\VS{11}Ne vois-tu donc pas ces ténèbres, ces eaux débordées qui te couvrent ?
\VS{12}Dieu n'est-il pas là-haut dans les cieux ? Regarde le sommet des étoiles, comme il est élevé !
\VS{13}Et tu dis : Qu'est-ce que Dieu connaît ? Peut-il juger à travers l'obscurité\FTNT{So. 1:12 ; Ps. 10:11-13 ; Ps. 94:7.} ?
\VS{14}Les nuées l'enveloppent, et il ne voit rien ; il ne parcourt que la voûte des cieux.
\VS{15}Eh quoi ! N'as-tu pas pris garde à l'ancienne route qu'ont suivie les hommes d'iniquité ?
\VS{16}Ils ont été emportés avant le temps, ils ont eu la durée d'un torrent qui s'écoule.
\VS{17}Ils disaient à Dieu : Éloigne-toi de nous ; que peut faire pour nous le Tout-Puissant ?
\VS{18}Dieu cependant avait rempli leurs maisons de biens ! Loin de moi le conseil des méchants !
\VS{19}Les justes le verront, se réjouiront, et l'innocent se moquera d'eux\FTNT{Ps. 107:42.} :
\VS{20}Certainement, notre adversaire a été détruit, le feu a dévoré ce qui en restait\FTNT{Ps. 37:20 ; Ec. 8:12-13.} !
\VS{21}Attache-toi donc à Dieu, et tu auras la paix, tu atteindras ainsi le bonheur.
\VS{22}Reçois de sa bouche l'instruction, et mets ses paroles dans ton cœur\FTNT{Ps. 119:72.}.
\VS{23}Si tu reviens au Tout-Puissant, tu seras rétabli ; si tu éloignes l'iniquité de ta tente.
\VS{24}Jette l'or dans la poussière, l'or d'Ophir parmi les rochers des torrents ;
\VS{25}et le Tout-Puissant sera ton or, ton argent, ta richesse.
\VS{26}Alors tu feras du Tout-Puissant tes délices, tu élèveras vers Dieu ta face ;
\VS{27}tu le prieras, et il t'exaucera, et tu lui rendras tes vœux\FTNT{Ps. 50:14-15.}.
\VS{28}Quand tu prendras des résolutions elles s'accompliront, sur tes sentiers brillera la lumière\FTNT{Ps. 97:11.}.
\VS{29}Quand on aura abaissé quelqu'un et que tu auras dit qu'il soit élevé; alors Dieu délivrera celui qui tenait les yeux abaissés\FTNT{Pr. 29:23.}.
\VS{30}Il délivrera le coupable ; il sera délivré par la pureté de tes mains.
\Chap{23}
\TextTitle{Réponse de Job}
\VerseOne{}Job répondit, et dit :
\VS{2}Maintenant encore ma plainte est une révolte, et pourtant ma main appesantit mes soupirs.
\VS{3}Oh ! Si je savais où le trouver, j'irais jusqu'à son trône,
\VS{4}je disposerais en ordre ma cause devant lui, je remplirais ma bouche d'arguments,
\VS{5}je saurais ce qu'il peut avoir à répondre, je comprendrais ce qu'il peut avoir à me dire.
\VS{6}Contesterait-il avec moi dans la grandeur de sa force? Ne prendrait-il pas le temps de m'écouter ?
\VS{7}Ce serait un homme juste qui argumenterait avec lui, et je serais pour toujours absous par mon juge.
\VS{8}Mais, si je vais à l'orient, il n'y est pas ; si je vais à l'occident, je ne l'aperçois pas ;
\VS{9}est-il occupé au nord, je ne le vois pas ; se cache-t-il au midi, je ne l'aperçois pas.
\VS{10}Il connaît la voie que j'ai suivie ; et s'il m'éprouvait, j'en sortirai pur comme l'or\FTNT{1 Pi. 1:7.}.
\VS{11}Mon pied s'est attaché à ses pas ; j'ai gardé sa voie, et je ne m'en suis pas détourné.
\VS{12}Je n'ai pas abandonné les commandements de ses lèvres ; j'ai fait plier ma volonté aux paroles de sa bouche.
\VS{13}Mais il n'a qu'une pensée ; qui l'en fera revenir ? Ce que son âme désire, il le fait\FTNT{Ps. 115:3 ; Ps. 135:6.}.
\VS{14}Il achèvera donc ses desseins à mon égard, et il en concevra beaucoup d'autres encore.
\VS{15}C'est pourquoi je suis terrifié à cause de sa présence, et quand je le considère, je suis effrayé devant lui.
\VS{16}Dieu a brisé mon cœur, le Tout-Puissant m'a épouvanté.
\VS{17}Car ce n'est pas la présence des ténèbres qui m'anéantit, ce n'est pas l'obscurité dont ma face est couverte.
\Chap{24}
\VerseOne{}Pourquoi le Tout-Puissant ne met-il pas des temps en réserve, et pourquoi ceux qui le connaissent ne voient-ils pas ses jours ?
\VS{2}On déplace les bornes, on ravit des troupeaux, et on les fait paître\FTNT{De. 19:14 ; De. 27:17 ; Pr. 13:10 ; Pr. 22:28.} ;
\VS{3}on emmène l'âne de l'orphelin, on prend pour gage le bœuf de la veuve ;
\VS{4}on fait écarter les pauvres du chemin, on force tous les affligés du pays à se cacher.
\VS{5}Et voici, comme les ânes sauvages du désert, ils sortent le matin pour chercher de la nourriture, ils n'ont que le désert pour trouver le pain de leurs enfants ;
\VS{6}ils moissonnent le fourrage qui reste dans les champs, ils grappillent dans la vigne de l'impie ;
\VS{7}ils passent la nuit nus, sans vêtements, sans couverture contre le froid\FTNT{Lé. 19:13 ; De. 24:12-13.} ;
\VS{8}ils sont percés par la pluie des montagnes, et ils embrassent les rochers comme unique refuge.
\VS{9}On arrache l'orphelin à la mamelle, on prend des gages sur le pauvre.
\VS{10}Ils font aller sans habits l'homme qu'ils ont dépouillé; et ils enlèvent à ceux qui n'avaient pas de quoi manger, ce qu'ils avaient glâné.
\VS{11}Dans les enclos de l'impie, ils font de l'huile, ils foulent le pressoir à raisin et ils ont soif.
\VS{12}Ils font gémir les gens dans la ville, l'âme de ceux qu'ils ont fait mourir, crient; Dieu ne fait rien d'indigne de lui.
\VS{13}En voici d'autres qui se révoltent contre la lumière, ils n'en connaissent pas les voies, ils ne restent pas sur leurs sentiers.
\VS{14}Le meurtrier se lève au point du jour ; il tue le pauvre et l'indigent, et il dérobe pendant la nuit\FTNT{Ps. 10:8-9.}.
\VS{15}L'œil de l'adultère épie le crépuscule ; aucun œil ne me verra, dit-il, et il met un voile sur le visage\FTNT{Ps. 64:6 ; Pr. 7:7-10.}.
\VS{16}Ils percent durant les ténèbres les maisons, qu'ils avaient marquées le jour, ils haïssent la lumière.\FTNT{Jn. 3:20.}.
\VS{17}Pour eux, le matin c'est l'ombre de la mort ; si quelqu'un les reconnaît, ils ont des terreurs.
\VS{18}Eh quoi ! L'impie est d'un poids léger sur la surface de l'eau, il n'a sur la terre qu'un héritage maudit, il ne prend jamais le chemin des vignes !
\VS{19}Comme la sécheresse et la chaleur absorbent les eaux de la neige, ainsi le scheol engloutit ceux qui pèchent\FTNT{Ps. 49:15.} !
\VS{20}Quoi ! Le sein maternel l'oublie, les vers en font leurs délices, on ne se souvient plus de lui ! L'injuste est brisé comme du bois,
\VS{21}lui qui dépouille la femme stérile et sans enfants, lui qui ne répand aucun bien sur la veuve !…
\VS{22}Non ! Dieu par sa force prolonge les jours des violents, et les voilà s'élever quand ils ne croyaient plus en la vie.
\VS{23}Il leur donne de la sécurité et de la confiance, ses yeux sont sur leurs voies.
\VS{24}Ils se sont élevés ; et en un peu de temps ils ne sont plus, ils s'affaissent, ils meurent en chemin comme tous les hommes, ils sont coupés comme une tête d'épi.
\VS{25}S'il n'en est pas ainsi, qui me fera mentir, qui fera de mes paroles un rien ?
\Chap{25}
\TextTitle{Dernier discours de Bildad}
\VerseOne{}Bildad de Schuach prit la parole et dit :
\VS{2}La domination et la terreur appartiennent à Dieu ; il fait régner la paix dans ses hautes régions.
\VS{3}Ses armées peuvent-elles se compter ? Sur qui sa lumière ne se lève-t-elle pas\FTNT{Mt. 5:45.} ?
\VS{4}Comment l'homme serait-il juste devant Dieu ? Comment celui qui est né de la femme serait-il pur ?
\VS{5}Voici, la lune même n'est pas brillante, et les étoiles ne sont pas pures à ses yeux ;
\VS{6}combien moins l'homme qui n'est qu'un ver, le fils de l'homme qui n'est qu'un vermisseau\FTNT{Ps. 22:7.} !
\Chap{26}
\TextTitle{Réponse de Job}
\VerseOne{}Job répondit, et dit :
\VS{2}Comme tu as aidé celui qui était sans force ! Comme tu as secouru le bras sans force !
\VS{3}Quels bons conseils tu donnes à celui qui manque de sagesse ! Tu fais connaître l'abondance de ton intelligence !
\VS{4}A qui s'adressent tes paroles ? Et de qui est l'esprit qui est sorti de toi ?
\VS{5}Devant Dieu les ombres des morts tremblent au-dessous des eaux, et de leurs habitants ;
\VS{6}devant lui le scheol est nu, l'abîme est sans voile\FTNT{Ps. 139:8-12 ; Pr. 15:11 ; Hé. 4:13.}.
\VS{7}Il étend la direction nord sur le vide, il suspend la terre sur le néant.
\VS{8}Il renferme les eaux dans ses nuages, et la nuée n'éclate pas sous leur poids\FTNT{Ps. 104:2-3.}.
\VS{9}Il couvre la face de son trône, il répand sur lui sa nuée.
\VS{10}Il a tracé un cercle à la surface des eaux, comme limite entre la lumière et les ténèbres\FTNT{Ge. 1:9 ; Jé. 5:22 ; Ps. 33:7 ; Ps. 104:9 ; Pr. 8:29.}.
\VS{11}Les colonnes du ciel s'ébranlent et s'étonnent à sa menace.
\VS{12}Par sa force il soulève la mer, par son intelligence il en brise l'orgueil\FTNT{Ps. 89:10.}.
\VS{13}Il a orné les cieux par son Esprit, et  de sa main, il transperce le serpent fuyard.
\VS{14}Ce sont là les bords de ses voies, c'est le discours fait en chuchotant que nous entendons ; mais qui comprendra le tonnerre de sa puissance\FTNT{Ec. 3:10.} ?
\Chap{27}
\VerseOne{}Et Job continuant, reprit son discours sentencieux, et dit :
\VS{2}Dieu, qui met mon droit à l'écart, et le Tout-puissant qui remplit mon âme d'amertume, est vivant.
\VS{3}Aussi longtemps que j'aurai ma respiration et que l'esprit de Dieu sera dans mes narines,
\VS{4}mes lèvres ne prononceront rien d'injuste, et ma langue ne dira pas de chose fausse\FTNT{Es. 33:15 ; Ps. 15:2 ; Ps. 24:4.}.
\VS{5}Loin de moi la pensée de vous reconnaître pour justes ! Tant que je vivrai je n'abandonnerai pas mon intégrité.
\VS{6}Je conserve ma justice, et je ne l'abandonne pas ; et mon cœur ne me reproche rien en mes jours.
\VS{7}Qu'il en soit de mon ennemi comme du méchant ; et de celui qui se lève contre moi, comme de l'injuste !
\VS{8}Quelle espérance reste-t-il à l'hypocrite quand Dieu coupe le fil de sa vie, quand il lui retire son âme\FTNT{Mt. 16:26 ; Lu. 12:20.} ?
\VS{9}Est-ce que Dieu entend ses cris, quand l'angoisse vient sur lui\FTNT{ Es. 1:15 ; Jé. 14:12 ; Ez. 8:18 ; Mi. 3:4 ; Ps. 18:41 ; Pr. 1:28 ; Jn. 9:31 ; Ja. 4:3.} ?
\VS{10}Trouvera-t-il son plaisir dans le Tout-Puissant ? Invoque-t-il Dieu en tout temps ?
\VS{11}Je vous enseignerai comment la main de Dieu agit, je ne vous cacherai pas les desseins du Tout-Puissant.
\VS{12}Voilà, vous avez tous vu ces choses, et pourquoi vous laissez-vous aller à des pensées vaines ?
\VS{13}Voici la part que Dieu réserve à l'homme méchant, l'héritage que les violents reçoivent du Tout-Puissant.
\VS{14}S'il a des fils en grand nombre, c'est pour l'épée, et ses rejetons ne seront pas rassasiés de pain ;
\VS{15}ses survivants sont ensevelis par la peste, et leurs veuves ne les pleurent pas\FTNT{Ps. 78:64.}.
\VS{16}Parce qu’il entasse l'argent comme la poussière, et qu'il entasse des habits comme on amasse de la boue,
\VS{17}le riche tombe, et il n’est pas relevé ; il ouvre ses yeux, et il ne trouve rien.
\VS{18}Il se bâtit une maison comme celle de la teigne, comme la cabane que fait un gardien\FTNT{Ps. 49:18.}.
\VS{19}Il se couche riche, et il périt dépouillé ; il ouvre les yeux, et tout a disparu.
\VS{20}Les frayeurs l'atteignent comme des eaux ; le tourbillon l'enlève de nuit.
\VS{21}Le vent d'orient l'emporte, et il s'en va ; il l'arrache de sa demeure comme un tourbillon.
\VS{22}Dieu le précipite à terre et ne l'épargne pas, et le méchant voudrait fuir devant sa main.
\VS{23}On applaudit à sa chute, et on le siffle au lieu où il se tient.
\Chap{28}
\VerseOne{}Il y a pour l'argent une mine d'où on le fait sortir, et pour l'or un lieu d'où on le purifie pour l'affiner ;
\VS{2}le fer se tire de la poussière, et la pierre se fond pour produire l'airain.
\VS{3}L'homme fait cesser les ténèbres, il explore jusqu'aux extrêmes limites les pierres cachées dans l'obscurité et dans l'ombre de la mort.
\VS{4}Il creuse un puits, loin des lieux habités ; ne se souvenant plus de ses pieds, il est suspendu, balancé, loin des humains.
\VS{5}La terre, d'où sort le pain, est bouleversée dans ses entrailles comme par le feu.
\VS{6}Ses pierres sont la demeure du saphir, et l'on y trouve de la poudre d'or.
\VS{7}L'oiseau de proie n'en connaît pas le chemin, l'œil du vautour ne l'aperçoit pas ;
\VS{8}les plus jeunes et fiers animaux n'y ont pas marché, le lion n'y a jamais passé.
\VS{9}L'homme avance sa main sur le roc, il renverse les montagnes depuis la racine ;
\VS{10}il fend des tranchées dans les rochers, et son œil voit tout ce qu'il y a de précieux ;
\VS{11}il arrête l'écoulement des eaux, et il fait sortir ce qui est caché.
\VS{12}Mais la sagesse, où se trouve-t-elle ? Où est le lieu où se tient l'intelligence ?
\VS{13}L'homme n'en connaît pas le prix, elle ne se trouve pas dans la terre des vivants.
\VS{14}L'abîme dit : Elle n'est pas en moi ; et la mer dit : Elle n'est pas avec moi.
\VS{15}Elle ne se donne pas contre de l'or pur, elle ne s'achète pas au poids de l'argent\FTNT{Pr. 3:14 ; Pr. 8:11 ; Pr. 16:16.} ;
\VS{16}elle ne se pèse pas contre de l'or d'Ophir, ni contre le précieux onyx, ni contre le saphir.
\VS{17}Elle ne peut se comparer à l'or ni au verre, elle ne peut s'échanger pour un vase d'or fin.
\VS{18}On ne se souvient ni du corail ni du cristal auprès d'elle : La sagesse vaut plus que les perles.
\VS{19}On ne la compare pas avec la topaze d'Ethiopie ; on ne la met pas en balance avec l'or pur.
\VS{20}D'où vient donc la sagesse ? Où est la demeure de l'intelligence ?
\VS{21}Elle est cachée aux yeux de tous les vivants, elle est cachée aux oiseaux des cieux.
\VS{22}L'abîme et la mort disent : Nous en avons entendu parler de nos oreilles.
\VS{23}C'est Dieu qui en sait le chemin, c'est lui qui en connaît la demeure ;
\VS{24}car il regarde jusqu'aux extrémités de la terre, il voit tout sous les cieux\FTNT{Ps. 14:2 ; Ps. 33:13-14 ; Ps. 102:20.}.
\VS{25}Quand il façonna le poids du vent, et qu'il estima la mesure des eaux\FTNT{Pr. 8:29.},
\VS{26}quand il ordonna des lois à la pluie, et qu'il fit un chemin à l'éclair et au tonnerre,
\VS{27}alors il vit la sagesse et la manifesta ; il l'établit et la sonda.
\VS{28}Puis il dit à l'homme : Voici, la crainte du Seigneur, c'est la sagesse ; se détourner du mal, c'est l'intelligence\FTNT{De. 4:6 ; Jé. 9:24 ; Ps. 111:10 ; Pr. 1:7 ; Pr. 9:10 ; Ec. 12:15.}.
\Chap{29}
\TextTitle{La postérité passée de Job}
\VerseOne{}Job prit de nouveau la parole sous forme sentencieuse et dit :
\VS{2}Oh ! Que ne puis-je être comme aux mois du passé, comme aux jours où Dieu me gardait,
\VS{3}quand sa lampe brillait sur ma tête, quand je marchais à sa lumière dans les ténèbres !
\VS{4}Que ne suis-je comme aux jours de mon automne, où Dieu veillait en ami sur ma tente,
\VS{5}quand le Tout-Puissant était encore avec moi, et que mes serviteurs m'entouraient ;
\VS{6}quand je lavais mes pieds dans le lait, et que le rocher répandait près de moi des torrents d'huile\FTNT{De. 32:13.} !
\VS{7}Si je sortais pour aller à la porte de la ville, et si je me faisais préparer un siège dans la place,
\VS{8}les jeunes gens se retiraient en me voyant, les vieillards se levaient et se tenaient debout.
\VS{9}Les princes s'abstenaient de parler, et mettaient la main sur leur bouche ;
\VS{10}la voix des chefs se taisait, et leur langue s'attachait à leur palais.
\VS{11}L'oreille qui m'entendait me disait heureux, l'œil qui me voyait me rendait témoignage ;
\VS{12}car je délivrais l'affligé qui criait au secours, et l'orphelin qui n'avait personne pour le secourir\FTNT{Ps. 72:12 ; Pr. 21:13.}.
\VS{13}La bénédiction de celui qui allait périr venait sur moi ; je remplissais de joie le cœur de la veuve.
\VS{14}Je me revêtais de la justice et elle se revêtait de moi, j'avais ma droiture pour manteau et pour turban\FTNT{Es. 59:17 ; 1 Th. 5:8 ; Ep. 6:14-17.}.
\VS{15}J'étais les yeux de l'aveugle et les pieds du boiteux.
\VS{16}J'étais le père des pauvres, j'examinais la cause de l'inconnu\FTNT{Pr. 29:7.} ;
\VS{17}je brisais les mâchoires de l'injuste, et j'arrachais la proie d'entre ses dents\FTNT{Ps. 58:7.}.
\VS{18}Alors je disais : Je mourrai dans mon nid, mes jours seront aussi nombreux que le sable ;
\VS{19}l'eau pénétrera dans mes racines, la rosée passera la nuit sur mes branches\FTNT{Jé. 17:5-8 ; Ps. 1:3.} ;
\VS{20}ma gloire se renouvellera sans cesse en moi, et mon arc se renouvellera dans ma main.
\VS{21}On m'écoutait et l'on restait dans l'attente, on gardait le silence devant mes conseils.
\VS{22}Après mes discours, nul ne répliquait, et ma parole était pour tous une bienfaisante rosée ;
\VS{23}ils s'attendaient à moi comme à la pluie, ils ouvraient la bouche comme pour une pluie de printemps.
\VS{24}Je souriais quand ils perdaient confiance, et l'on ne pouvait faire tomber la sérénité de mon visage.
\VS{25}J'aimais à aller avec eux, et je m'asseyais à leur tête ; j'étais comme un roi au milieu de ses gardes, comme un consolateur auprès des affligés.
\Chap{30}
\TextTitle{Son humiliation}
\VerseOne{}Mais maintenant !… Chaque jour je suis la risée de plus jeunes que moi, de ceux dont je dédaignais de mettre les pères parmi les chiens de mon troupeau.
\VS{2}Mais à quoi me servirait la force de leurs mains ? En eux avait péri toute vigueur.
\VS{3}Desséchés par la disette et la faim, ils fuient dans les lieux arides, depuis longtemps abandonnés et déserts ;
\VS{4}ils arrachent près des buissons l'herbe sauvage, et la racine des genêts est leur nourriture.
\VS{5}On les chasse du milieu des hommes, on crie après eux comme après un voleur.
\VS{6}Ils habitent dans le creux des torrents, dans les trous de la terre et des rochers ;
\VS{7}ils hurlent parmi les buissons, ils se rassemblent sous les ronces.
\VS{8}Peuple insensé et sans nom, on les repousse du pays !
\VS{9}Et maintenant, je suis le sujet de leurs chansons, je suis en butte à leurs propos\FTNT{Ps. 69:12 ; La. 3:14.}.
\VS{10}Ils m'ont en horreur, ils s'éloignent de moi, ils ne se retiennent pas de me cracher leur salive au visage.
\VS{11}Ils n'ont aucune retenue et ils m'humilient, ils rejettent tout frein devant moi.
\VS{12}Ces misérables se lèvent à ma droite et me poussent les pieds, ils se fraient contre moi des routes pour ma ruine\FTNT{Ps. 35:15.} ;
\VS{13}ils détruisent mon propre sentier et travaillent à ma perte, eux à qui personne ne viendrait en aide ;
\VS{14}ils viennent contre moi comme par une brèche large, et ils se sont jetés sur moi à cause de ma désolation.
\VS{15}Toutes les terreurs se tournent contre moi ; ma gloire est emportée comme par le vent, mon bonheur a passé comme un nuage\FTNT{Os. 13:3.}.
\VS{16}Et maintenant, mon âme se répand en mon sein, les jours d'affliction m'ont saisi.
\VS{17}La nuit me perce et m'arrache les os, la douleur qui me ronge ne se donne aucun repos.
\VS{18}Par la violence du mal, mon vêtement se déforme, il se colle à mon corps comme ma tunique.
\VS{19}Dieu m'a jeté dans la boue, et je ressemble à la poussière et à la cendre.
\VS{20}Je crie vers toi, et tu ne me réponds pas ; je me tiens debout, et tu m'aperçois.
\VS{21}Tu deviens cruel contre moi, tu t'opposes à moi avec la force de ta main.
\VS{22}Tu me soulèves, tu me fais chevaucher sur le vent, et tu me fais fondre au bruit de la tempête.
\VS{23}Car, je le sais, tu me mènes à la mort, à la demeure fixée pour tous les vivants\FTNT{Hé. 9:27.}.
\VS{24}Mais celui qui va périr n'étend-il pas les mains ? Celui qui est dans le malheur n'implore-t-il pas du secours ?
\VS{25}Ne pleurais-je pas sur l'homme qui passait des jours difficiles ? Mon âme n'avait-elle pas pitié du pauvre\FTNT{Ro. 12:15.} ?
\VS{26}J'attendais le bonheur, et le malheur est arrivé ; j'espérais la lumière, et les ténèbres sont venues.
\VS{27}Mes entrailles bouillonnent sans repos, les jours d'affliction m'ont confronté.
\VS{28}Je marche noirci, mais non par le soleil ; je me lève en pleine assemblée, et je crie.
\VS{29}Je suis devenu le frère des serpents, le compagnon des autruches\FTNT{Ps. 102:7-8.}.
\VS{30}Ma peau noircit et tombe, mes os brûlent et se dessèchent\FTNT{La. 4:8 ; La. 5:10.}.
\VS{31}Ma harpe n'est plus qu'un instrument de deuil, et mon chalumeau ne peut rendre que des voix en pleurs.
\Chap{31}
\TextTitle{Job se justifie}
\VerseOne{}J'avais fait une alliance avec mes yeux, et je n'aurais pas regardé une vierge.
\VS{2}Quelle part Dieu m'eût-il réservée d'en haut ? Quel héritage le Tout-Puissant m'aurait-il envoyé des cieux ?
\VS{3}La ruine n'est-elle pas pour l'injuste, et le malheur pour ceux qui commettent l'iniquité ?
\VS{4}Dieu ne voit-il pas mes voies ? Ne compte-t-il pas tous mes pas\FTNT{Pr. 5:21 ; Pr. 15:3 ; 2 Ch. 16:9.} ?
\VS{5}Si j'ai marché dans le mensonge, si mon pied s'est hâté pour tromper,
\VS{6}que Dieu me pèse dans des balances justes, et il reconnaîtra mon intégrité !
\VS{7}Si mes pas se sont détournés du droit chemin, si mon cœur a suivi mes yeux, si quelque souillure s'est attachée à mes mains,
\VS{8}Que je sème et qu'un autre mange, et tout ce que j'aurais fait produire soit déraciné!
\VS{9}Si mon cœur a été séduit par une femme, si j'ai fait le guet à la porte de mon prochain\FTNT{Pr. 7.},
\VS{10}que ma femme broie le grain pour un autre, et que d'autres se penchent sur elle !
\VS{11}Car c'est un crime, une iniquité punie par les juges ;
\VS{12}c'est un feu qui dévore jusqu'à la destruction, et qui aurait détruit toutes mes récoltes dans leur racine.
\VS{13}Si j'ai méprisé le droit de mon serviteur ou de ma servante, lorsqu'ils étaient en contestation avec moi,
\VS{14}qu'ai-je à faire, quand Dieu se lève ? Qu'ai-je à répondre, quand il châtie ?
\VS{15}Celui qui m'a fait dans le ventre de ma mère ne l'a-t-il pas fait aussi ? Un même Dieu ne nous a-t-il pas formés dans le sein maternel\FTNT{Pr. 14:31 ; Pr. 17:5.} ?
\VS{16}Si j'ai refusé aux pauvres leur désir, si j'ai laissé se consumer les yeux de la veuve\FTNT{Es. 10:2 ; Lu. 18:2-3.},
\VS{17}si j'ai mangé seul mon morceau de pain, sans que l'orphelin en ait sa part,
\VS{18}moi qui l'ai dès ma jeunesse fait grandir près de moi comme un père, et qui dès le sein de ma mère, ai été le guide de la veuve ;
\VS{19}si j'ai vu le malheureux périr faute de vêtements, le pauvre manquer de couverture\FTNT{Mt. 25:41-45.},
\VS{20}sans que ses reins m'aient béni, sans qu'il ait été réchauffé par la toison de mes agneaux ;
\VS{21}si j'ai levé la main contre l'orphelin, parce que je me voyais comme un appui dans les portes\FTNT{Pr. 22:22.} ;
\VS{22}que mon épaule tombe de sa jointure, que mon bras tombe et qu'il se brise l'os !
\VS{23}Car les châtiments de Dieu m'épouvantent, et je ne pourrais pas prévaloir devant sa majesté.
\VS{24}Si j'ai mis dans l'or ma confiance, si j'ai dit à l'or fin : Tu es mon espoir\FTNT{Mc. 10:24 ; 1 Ti. 6:17.} ;
\VS{25}si je me suis réjoui de ma grande puissance, de la quantité des richesses que ma main a acquise\FTNT{Ps. 62:11.} ;
\VS{26}si j'ai regardé le soleil quand il brillait, la lune quand elle s'avançait de façon majestueuse,
\VS{27}et si mon cœur s'est laissé secrètement séduire, si ma main a envoyé des baisers de ma bouche ;
\VS{28}c'est encore une iniquité que doit punir le juge, et j'aurais renié le Dieu d'en haut !
\VS{29}Si je me suis réjoui du malheur de mon ennemi, si j'ai sauté d'allégresse quand le mal l'a atteint\FTNT{Mt. 5:43-44.},
\VS{30}moi qui n'ai pas permis à ma langue de pécher en demandant sa mort par des malédictions ;
\VS{31}si les gens de ma tente ne disaient pas : Où est celui qui n'a pas été rassasié de sa viande\FTNT{Ps. 27:2.} ?
\VS{32}Si l'étranger passait la nuit dehors, si je n'ouvrais pas ma porte au voyageur\FTNT{Ge. 19:1-2 ; De. 10:19 ; 1 Pi. 4:9 ; Hé. 13:2.} ;
\VS{33}si, comme les hommes, j'ai caché mes transgressions et mon crime dans mon sein\FTNT{Ge. 3:10-12 ; Pr. 28:13.},
\VS{34}parce que je craignais la multitude, et je craignais le mépris des familles, en sorte que je restais tranquille et n'osais franchir ma porte…
\VS{35}Oh ! Qui me fera trouver quelqu'un qui m'écoute ? Voilà ma défense toute signée : Que le Tout-Puissant me réponde ! Qui me donnera la plainte écrite par mon adversaire ?
\VS{36}Je porterai son écrit sur mon épaule, je l'attacherai sur mon front comme une couronne ;
\VS{37}je lui déclarerai le nombre de mes pas, je m'approcherai de lui comme un prince.
\VS{38}Si ma terre crie contre moi, et que ses sillons pleurent ;
\VS{39}si j'en ai mangé le produit sans l'avoir payée, et que j'aie attristé l'âme de ses anciens maîtres ;
\VS{40}Qu'elle en produise des épines au lieu du froment, et de l'ivraie au lieu de l'orge ! C'est ici la fin des paroles de Job.
\Chap{32}
\TextTitle{Discours d'Elihu : reproches à Job et à ses amis}
\VerseOne{}Ces trois hommes-là cessèrent de répondre à Job, parce qu'il se regardait comme juste.
\VS{2}Élihu, fils de Barakeel de Buz, de la famille de Ram, s'enflamma de colère contre Job, parce qu'il disait son âme juste devant Dieu.
\VS{3}Et sa colère s'enflamma contre ses trois amis, parce qu'ils ne trouvaient rien à répondre et que néanmoins ils condamnaient Job.
\VS{4}Comme ils étaient plus âgés que lui, Elihu avait attendu jusqu'à ce moment pour parler à Job.
\VS{5}Mais, voyant que ces trois hommes n'avaient plus aucune réponse à la bouche, Elihu se mit en colère.
\VS{6}Et Elihu, fils de Barakeel de Buz, prit la parole et dit : Je suis jeune et vous êtes des vieillards ; c'est pourquoi j'ai craint, j'ai eu peur de vous faire connaître mon sentiment.
\VS{7}Je disais: les jours parleront, et le grand nombre des années fera connaître la sagesse.
\VS{8}L'esprit dans l'homme, c'est l'esprit, le souffle du Tout-Puissant qui rend intelligent\FTNT{Da. 1:17 ; Da. 2:21 ; Pr. 2:6 ; Ec. 2:26.} ;
\VS{9}ce ne sont pas les aînés qui sont sages, ce ne sont pas les vieillards qui comprennent ce qui est juste.
\VS{10}C'est pourquoi je dis : Ecoute-moi ! Et je dirai aussi ma pensée.
\VS{11}J'ai attendu la fin de vos discours, j'ai écouté vos raisonnements, jusqu'à ce que vous ayez bien examiné les discours de Job.
\VS{12}J'ai pris le soin de vous écouter ; et voici, aucun de vous n'a convaincu Job, aucun n'a répondu à ses paroles.
\VS{13}Qu'il ne vous n'arrive pas de dire: nous avons trouvé la sagesse ; c'est Dieu qui le poursuit, et non pas l'homme !
\VS{14}Il n'a pas dirigé ses discours contre moi : Aussi je ne lui répondrai pas à votre manière.
\VS{15}Ils sont étonnés! Ils ne répondent plus rien! On leur a ôté la parole!
\VS{16}J'ai attendu jusqu'à ce qu'ils n'ont plus rien dit, car ils sont demeurés muets, et ils n'ont plus su que répondre.
\VS{17}A mon tour, je veux répondre pour moi, et je veux donner mon avis.
\VS{18}Car je suis rempli de discours, l'esprit qui est en mon sein me presse.
\VS{19}Mon sein est comme vin sans air, comme des outres neuves qui vont éclater\FTNT{Mt. 9:17 ; Mc. 2:22 ; Lu. 5:38.}.
\VS{20}Je parlerai pour respirer à l'aise, j'ouvrirai mes lèvres et je répondrai.
\VS{21}Je ne ferai pas acception de personnes, et je flatterai aucun homme.
\VS{22}Car je ne sais pas flatter : Mon Créateur m'enlèverait bien vite.
\Chap{33}
\TextTitle{Discours d'Elihu : la justice de Dieu}
\VerseOne{}C'est pourquoi Job, écoute mon discours, je te prie et prête l'oreille à toutes mes paroles !
\VS{2}Voici, j'ouvre la bouche, ma langue parle dans mon palais.
\VS{3}Mes paroles exprimeront la droiture de mon cœur, mes lèvres diront la vérité pure.
\VS{4}L'Esprit de Dieu m'a fait, et le souffle du Tout-Puissant me donne la vie\FTNT{Ge. 2:7.}.
\VS{5}Si tu peux, réponds-moi, dresse-toi contre moi, demeure ferme!
\VS{6}Voici, je suis pour le Dieu fort, selon que tu en as parlé; j'ai été formé de la terre tout comme toi.\FTNT{Ac. 14:15.} ;
\VS{7}Voici ma terreur ne te trouble pas, et ma main ne s'appesantit pas sur toi
\VS{8}Quoi qu'il en soit, tu as dit, moi l'entendant, j'ai entendu la voix de tes discours:
\VS{9}Je suis pur, sans péché, je suis net, il n'y a pas d'iniquité en moi.
\VS{10}Voici il cherche à rompre avec moi, il me considère comme son ennemi ;
\VS{11}il met mes pieds dans les ceps, il surveille tous mes chemins.
\VS{12}Je te répondrai qu'en cela tu n'as pas été juste, car Dieu sera toujours plus grand que l'homme.
\VS{13}Pourquoi as-tu donc plaidé contre lui ? Car il ne rend pas compte de toutes ses actions.
\VS{14}Car Dieu parle une première fois, et une seconde fois à celui qui n'aura pas pris garde à la première.
\VS{15}Par des songes, par des visions nocturnes, quand les hommes tombent dans un profond sommeil, quand ils dorment sur leur couche.
\VS{16}Alors il ouvre l'oreille de l'homme d'une mauvaise action et de rabaisser la fierté de l'homme.
\VS{17}afin de détourner l'homme de son œuvre et de le préserver de l'orgueil,
\VS{18}il garantit son âme de la fosse, et sa vie de l'épée.
\VS{19}L'homme est aussi châtié par des douleurs sur son lit, à cause d'une lutte perpétuelle en ses os\FTNT{Ps. 38:4.}.
\VS{20}Alors sa vie prend en horreur le pain et son âme les mets les plus désirés\FTNT{Ps. 107:18.} ;
\VS{21}Sa chair est tellement consumée qu'elle paraît plus, ses os sont tellement brisés, qu'on y connaît plus rien;
\VS{22}son âme s'approche de la fosse, et sa vie des messagers de la mort.
\VS{23}Mais s'il y a pour cet homme un messager qui interprète, un d'entre les mille, pour lui annoncer la voie de la droiture,
\VS{24}alors Dieu prend pitié de lui et dit : Garantis-le, afin qu'il ne descende pas dans la fosse ; j'ai trouvé la propitiation!
\VS{25}Sa chair devient plus délicate qu'elle n'était dan son enfance; il revient aux jours de sa jeunesse.
\VS{26}Il supplie Dieu par ses prières, et Dieu lui est favorable, il lui laisse voir sa face avec joie, et lui rend sa justice\FTNT{Es. 58:9.}.
\VS{27}Il regarde vers les hommes et dit : J'ai péché, j'ai violé la justice, et je n'ai pas été puni comme je le méritais ;
\VS{28}Dieu a racheté mon âme afin qu'elle ne passe pas dans la fosse, et ma vie voit encore la lumière !
\VS{29}Voilà ce que Dieu fait, deux fois, trois fois, envers l'homme\FTNT{Ps. 62:11.},
\VS{30}pour ramener son âme de la fosse, pour l'éclairer de la lumière des vivants\FTNT{Ps. 56:14.}.
\VS{31}Sois attentif, Job, écoute-moi ! Tais-toi, et je parlerai !
\VS{32}Si tu as quelque chose à dire, réponds-moi ! Parle, car je désire te justifier.
\VS{33}Sinon, écoute, tais-toi et je t'enseignerai la sagesse.
\Chap{34}
\TextTitle{Discours d'Elihu : il accuse Job de se révolter}
\VerseOne{}Elihu reprit la parole, et dit :
\VS{2}Sages, écoutez mes discours ! Vous qui avez la connaissance, prêtez-moi l'oreille !
\VS{3}Car l'oreille discerne les discours, comme le palais savoure ce qu'il mange.
\VS{4}Choisissons ce qui est juste, voyons entre nous ce qui est bon.
\VS{5}Job dit : Je suis juste, et Dieu a écarté ma justice;
\VS{6}mentirai-je à mon droit? Ma flèche est mortelle sans que j'aie commis de crime.
\VS{7}où y a-t-il un homme comme Job, qui boit le péché la moquerie comme l'eau,
\VS{8}qui marche en compagnie des ouvriers d'iniquité, et qui fréquente  avec les hommes marchant de pair avec les hommes méchants ?
\VS{9}Car il a dit : Il est inutile à l'homme de plaire à Dieu\FTNT{Mal. 3:14.}.
\VS{10} C'est pourquoi écoutez, vous qui avez de l'intelligence, écoutez-moi ! Loin de Dieu la méchanceté, loin du Tout-Puissant l'injustice\FTNT{De. 32:4 ; Ps. 92:16 ; Ro. 9:14.} !
\VS{11}Car il rend à l'homme selon son œuvre, il fait trouver à chacun selon sa voie\FTNT{Jé. 17:10 ; Jé. 32:19 ; Ez. 7:27 ; Pr. 24:12 ; Mt. 16:27 ; Ro. 2:6 ; 2 Co. 5:10 ; Ep. 6:8 ; Ap. 22:12.}.
\VS{12}Certes, Dieu ne commet pas l'injustice ; le Tout-Puissant ne renverse pas le droit.
\VS{13}Qui lui a donné la terre en charge ? Ou Qui a placé la terre habitable?
\VS{14}S'il ne pensait qu'à lui-même, s'il retirait à lui son esprit et son souffle\FTNT{Ps. 104:29.},
\VS{15}toute chair périrait ensemble, et l'homme retournerait dans la poussière\FTNT{Ge. 3:19 ; Ec. 3:20 ; Ec. 12:9.}.
\VS{16}Si donc tu as de l'intelligence, écoute ceci, prête l'oreille à ce que tu entendras de moi.
\VS{17}Comment celui qui n'aimerait pas à faire la justice jugerait-il le monde? Et condamneras-tu celui qui est souverainement juste ?
\VS{18}Dira-t-on à un roi, qu'il est un scélérat ? Et aux princes, qu'ils sont des méchants ?
\VS{19}Combien moins le dira-t-on à celui qui n'a point d'égard à la personne des grands, et qui ne connaît point les riches pour les préférer aux pauvres, parce qu'ils sont tous l'ouvrage de ses mains\FTNT{De. 10:17 ; 2 Ch. 19:7 ; Ac. 10:34 ; Ga. 2:6 ; Ro. 2:11 ; Ep. 6:9 ; Col. 3:25.} ?
\VS{20}En un moment, ils mourront ; au milieu de la nuit, un peuple est ébranlé et passe ; le puissant s'en va, sans la main d'aucun homme.
\VS{21}Car les yeux de Dieu sont sur les voies de l'homme, il regarde tous ses pas.
\VS{22}Il n'y a ni ténèbres ni ombre de la mort où puissent se cacher les ouvriers d'iniquité.
\VS{23}Dieu ne regarde pas à deux fois un homme, pour le faire aller en jugement avec lui.
\VS{24}Il brise les puissants par des voies incompréhensibles, et il n établit d'autres à leur place ;
\VS{25}car il connaît leurs œuvres. Il les renverse de nuit, et ils sont écrasés ;
\VS{26}il les frappe comme des impies, au lieu où se tiennent tous les regards.
\VS{27}Du fait qu'ils se sont détournés de lui, et qu'ils n'ont considéré aucune de ses voies.
\VS{28}ils ont fait monter à Dieu le cri du pauvre, et il a entendu le cri des affligés\FTNT{Ja. 5:4.}.
\VS{29}S'il donne le repos, qui est-ce qui causera du trouble? S'il cache sa face à quelqu'un, qui le regardera, qu'il s'agisse de toute une nation ou d'un seul un homme?
\VS{30}afin que l'hypocrite ne règne pas, de peur qu'il plus un piège pour le peuple.
\VS{31}Car a-t-il jamais dit à Dieu : J'ai été pardonné, je ne pécherai plus ;
\VS{32}montre-moi ce que je ne vois pas ; si j'ai fait le mal, je ne le ferai plus ?
\VS{33}Mais Dieu ne te le rendra-t-il pas, puisque tu as rejeté son châtiment, quand tu as fait le choix que tu as fait? Pour moi, je ne sais que dire à cela; mais toi, si tu as quelque chose à répondre, parle.
\VS{34}Les gens de bon sens diront avec moi, et tout homme sage en conviendra,
\VS{35}que Job ne parle pas avec connaissance, et ses paroles manquent d'intelligence.
\VS{36}Ah! Mon père, que Job soit éprouvé jusqu'à ce qu'il soit vaincu, puisqu'il vaincu, puisqu'il répond comme les impies.
\VS{37}Car il ajoute péché sur péché; il applaudit au milieu de nous ; il parle de plus en plus contre Dieu.
\Chap{35}
\TextTitle{Discours d'Elihu : il reproche à Job ses propos irréfléchis}
\VerseOne{}Elihu reprit la parole et dit :
\VS{2}Penses-tu avoir raison de dire : Je suis juste devant Dieu ?
\VS{3}Quand tu dis : Que me sert-il, et que gagnerais-je de plus sans pécher ?
\VS{4}Je te répondrai en ces termes, et à tes amis qui sont avec toi.
\VS{5}Regarde les cieux, et considère-les ! Vois les nuées, comme elles sont plus hautes que toi !
\VS{6}Si tu pèches, quel mal fais-tu à Dieu ? Et si tes péchés se multiplient, quel mal reçoit-il ?
\VS{7}Si tu es juste, que lui donnes-tu ? Que reçoit-il de ta main ?
\VS{8}C'est à un homme, comme tu es, que ta méchanceté peut seule nuire, et c'est au fils d'un homme que ta justice peut seule être utile.
\VS{9}On fait crier les opprimés par la grandeur des maux qu'on leur afflige; ils crient à cause de la violence des grands;
\VS{10}Et nul ne dit : Où est le Dieu qui m'a fait, qui donne de quoi chanter pendant la nuit ,
\VS{11}qui nous instruit plus que les animaux de la terre, et plus intelligent que les oiseaux des cieux ?
\VS{12}On crie donc à cause de la fierté des méchants; mais Dieu ne l'exauce pas.
\FTNT{Es. 1:15 ; Ez. 8:18 ; Mi. 3:4 ; Jn. 9:31.}.
\VS{13} Cependant que ce soit en vain; que Dieu n'écoute pas, et que le Tout-puissant n'y a pas égard.
\VS{14}Encore moins dois-tu lui dire, tu ne le vois pas; car le jugement est devant lui, attends-le donc!
\VS{15}Mais maintenant, ce n'est rien ce que sa colère exécute, et il n'a pas encore pris connaissance en profondeur toutes les choses que tu as faites.
\VS{16}Job ouvre donc sa bouche pour se plaindre, il multiplie les paroles sans intelligence.
\Chap{36}
\TextTitle{Discours d'Elihu : Dieu traite les hommes selon leurs oeuvres}
\VerseOne{}Elihu continua de parler, et dit :
\VS{2}Attends un peu, et je te montrerai qu'il y a encore d'autres raisons pour la cause de Dieu.
\VS{3}Je tirerai de loin mes raisons, et je défendrai la justice du Créateur.
\VS{4}Car certainement il n'y aura rien de faux en tout ce que je dirai, et celui qui est avec toi, est parfait dans sa connaissance.
\VS{5}Dieu est puissant, mais il ne méprise personne ; il est puissant par la force de son coeur.
\VS{6}Il ne laisse pas vivre le méchant, et il fait droit aux pauvres.
\VS{7}Il ne détourne pas ses yeux de dessus les justes, il les place sur le trône avec les rois, il les y fait asseoir pour toujours, afin qu'ils soient élevés\FTNT{Ps. 33:18 ; Ps. 34:16.}.
\VS{8}S'ils sont liés de chaînes, s'ils sont pris dans les liens de l'affliction,
\VS{9}il leur fait connaître leurs œuvres, leurs transgressions, leur orgueil.
\VS{10}Alors il ouvre leur oreille pour leur discipline, il leur dit de se détourner de l'iniquité.
\VS{11}S'ils écoutent, et s'ils le servent, ils achèvent leurs jours dans le bonheur, leurs années dans la joie.
\VS{12}S'ils n'écoutent pas, ils passent par l'épée, ils expirent dans leur aveuglement.
\VS{13}Ceux qui sont hypocrites dans leur cœur, ils ne crient pas à lui quand il les a liés ;
\VS{14}leur personne meurt dans sa jeunesse, leur vie s'éteint parmi les débauchés.
\VS{15}Mais Dieu sauve celui qui est affligé de son oppression, et c'est par la détresse qu'il lui ouvre les oreilles.
\VS{16}Il t'écartera aussi de la détresse, pour te mettre au large, loin de toute angoisse, et ta table sera chargée de viandes grasses\FTNT{Ps. 50:15 ; Ps. 63:6.}.
\VS{17}Or tu remplis le jugement du méchant, mais le jugement et le droit subsisteront.
\VS{18}Certainement Dieu et irrité; prends garde qu'il ne te  plonge dans l'affliction, car il n'y aura pas alors de rançon si grande pour te délivrer\FTNT{Ps. 49:8.} !
\VS{19}Tes cris valent-ils ton or, et même toutes les forces qui se trouvent dans tes richesses ?
\VS{20}Ne soupire pas après la nuit, qui enlève les peuples de leur place.
\VS{21}Garde-toi de te retourner vers l'iniquité, car la souffrance t'y dispose.
\VS{22}Dieu est élevé par sa puissance ; qui saurait enseigner comme lui ?
\VS{23}Qui lui prescrit le chemin qu'il devait tenir? Qui lui dit : Tu a fait une injustice ?
\VS{24}Souviens-toi de célébrer ses ouvrages, que tous les hommes voient.
\VS{25}Tout homme les voit, chacun les contemple de loin.
\VS{26}Dieu est grand, mais nous ne le connaissons pas, quant au nombre de ses années il est insondable\FTNT{Es. 63:16 ; Ps. 92:8 ; Ps. 93:2 ; Ps. 102:13 ; La. 5:19.}.
\VS{27}Parce qu'il met les eaux  en petites gouttes, elle répandent la pluie selon la vapeur d'eau qui la contient ;
\VS{28}les nuées la font dégoutter, elles coulent sur les hommes en abondance.
\VS{29}Et qui pourra comprendre l'étendue des nuages et le son éclatant de sa tente ?
\VS{30}Voici, il étend sa lumière sur elle, et il se cache jusque dans les profondeurs de la mer.
\VS{31} Or c'est par ces choses qu'il juge les peuples, qu' il donne la nourriture en abondance.
\VS{32} Il tient caché dans les paumes de ses mains le feu étincelant, et lui ordonne de frapper de ce qui se présente à sa rencontre.
\VS{33} Son bruit porte les nouvelles ; les troupeaux font connaître qu'il approche.
\Chap{37}
\TextTitle{Discours d'Elihu : conclusion}
\VerseOne{}Mon cœur même à cause de cela est tout tremblant, il sort de sa place.
\VS{2}Ecoutez attentivement et en tremblant le bruit de sa voix, le grondement qui sort de sa bouche\FTNT{Ps. 29:3-9.} !
\VS{3}Il le conduit dans toute l'étendue des cieux, et son éclair brille jusqu'aux extrémités de la terre\FTNT{Ps. 97:4.}.
\VS{4}Après lui de l'élève un grand bruit, il tonne de sa voix majestueuse; il ne tarde pas après que sa voix a été entendue\FTNT{Jé. 10:13.}.
\VS{5}Dieu tonne avec sa voix d'une manière étonnante ; il fait de grandes choses que nous ne comprenons pas.
\VS{6}Car il dit à la neige : Tombe sur la terre ! Il le dit à la pluie, même aux plus fortes pluies.
\VS{7}Il met un sceau à la main de tous les hommes pour reconnaître tous les hommes qui sont son ouvrage.
\VS{8}Les bêtes entrent dans leurs tanières, et elles demeurent dans leurs repaires.
\VS{9}L'ouragan vient du fond du sud, et le froid vient des vents du nord.
\VS{10}Par son souffle, Dieu donne la glace, et il réduit l'espace où se répondaient au large les eaux\FTNT{Ps. 147:17-18.}.
\VS{11}Il lasse les nuages à force d'arroser, il écarte les nuages par sa lumière.
\VS{12}Et ceux-ci font plusieurs tours pour faire ce qu'il a commandé, sur la face de la terre sur la face de la terre habitée ;
\VS{13}Il les fait venir pour s'en servir soit comme une verge pour la terre, soit pour répandre ses bienfaits\FTNT{Ex. 9:18-23 ; 1 S. 12:18-19.}.
\VS{14}Job, arrête-toi, prête l'oreille à ces choses ! Considère encore les merveilles de Dieu !
\VS{15}Sais-tu comment Dieu les dispose, et fait briller la lumière de ses nuages?
\VS{16}Connais-tu le balancement des nuages, les merveilles de celui dont la science est parfaite ?
\VS{17}Sais-tu pourquoi tes vêtements sont chauds quand la terre se repose par le vent du midi ?
\VS{18}Peux-tu étendre avec lui les cieux, aussi fermes qu'un miroir de fonte ?
\VS{19}Montre-nous ce que nous pouvons lui dire ; car nous ne saurions rien dire par ordre à cause de nos ténèbres. 
\VS{20}Lui racontera-t-on quand je parlerai ? S'il y a un homme qui en parle, certainement il en sera englouti ?
\VS{21}Et maintenant on ne voit pas la lumière du soleil qui resplendit dans les cieux, lorsque le vent passe et le nettoie ;
\VS{22}Le temps qui la reluit comme l'or vient du nord. Il y a en Dieu une majesté redoutable.
\VS{23}Nous ne saurions comprendre le Tout-Puissant, grand en puissance, en jugement et en abondante justice, il n'opprime personne !
\VS{24}C'est pourquoi les hommes le craignent ; mais il ne les voit pas tous sages de cœur\FTNT{Ps. 92:7 ; Ro. 1:21.}.
\Chap{38}
\TextTitle{Yahweh interroge Job}
\VerseOne{}Yahweh répondit à Job du milieu de le tourbillon et dit :
\VS{2}Qui est celui qui obscurcit mes décisions par des paroles sans connaissance ?
\VS{3}Ceins maintenant tes reins comme un vaillant homme ; je t'interrogerai, et tu me fera voir ta science.
\VS{4}Où étais-tu quand je fondais la terre ? Dis-le, si tu as de l'intelligence\FTNT{Pr. 8:29.}.
\VS{5}Qui en a réglé les mesures, le sais-tu ? Ou qui a appliqué sur elle le niveau ?
\VS{6}Sur quoi ses bases sont-elles plantées ? Ou qui en a posé la pierre angulaire pour la soutenir\FTNT{Ps. 104:5.}?
\VS{7}Quand les étoiles du matin se réjouissent ensemble, et que tous les fils de Dieu poussent des cris de joie \FTNT{Ps. 148:3.} ?
\VS{8}Qui a renfermé la mer dans ses bords, quand elle fut tirée de la matrice et qu'elle sortit? 
\VS{9}quand je lui donnai la nuée pour vêtement, et l'obscurité pour langes ;
\VS{10}que je lui imposai ma loi, et que je lui mis des barrières et des portes;
\VS{11} et quand je dis : Tu viendras jusqu'ici, tu n'iras pas plus loin ; ici s'arrêtera l'orgueil de tes flots ?
\VS{12}Depuis que tu es au monde, as-tu commandé au matin et as-tu montré à l'aube du jour le lieu où elle doit se lever,
\VS{13}pour qu'elle saisisse les extrémités de la terre, et que les méchants en soient chassés ;
\VS{14}pour que la terre prenne une forme comme l'argile qui reçoit un sceau, et qu'elle soit parée comme d'un vêtement nouveau ;
\VS{15}pour que la lumière soit ôtée aux méchants, et que le bras qui se lève soit brisé\FTNT{Ps. 10:15.} ?
\VS{16}As-tu pénétré jusqu'aux sources de la mer ? T'es-tu promené dans les profondeurs de l'abîme ?
\VS{17}Les portes de la mort se sont-elles découvertes à toi ? As-tu vu les portes de l'ombre de la mort ?
\VS{18}As-tu compris l'étendue de la terre ? Si tu sais tout cela, dis-le !
\VS{19}Où est la demeure de la lumière, et où est le lieu des ténèbres ?
\VS{20}Pour que tu les prennes à leur limite, et que tu connaisses le chemin de leur maison ?
\VS{21}Tu le sais, car alors tu étais né, et le nombre de tes jours est grand !
\VS{22}Es tu entré dans les trésors de la neige ? As-tu vu les trésors de grêle,
\VS{23}que je réserve pour les temps de détresse, pour les jours de guerre et de bataille\FTNT{Ex. 9:23 ; Jos. 10:11 ; Ap. 8 :7.} ?
\VS{24}Par quel chemin la lumière se partage la lumière, et le vent d'orient se répand-il sur la terre\FTNT{Jn. 3:8.} ?
\VS{25}Qui a ouvert un conduit aux inondations, et tracé la route de l'éclair et du tonnerre,
\VS{26}pour qu'elle pleuve sur une terre sans habitants, sur un désert sans hommes\FTNT{Ps. 104:13-14 ; PS. 147:8 ; Ac. 14:17.} ;
\VS{27}pour qu'elle abreuve les lieux solitaires et arides, et qu'elle fasse germer et sortir l'herbe ?
\VS{28}La pluie a-t-elle un père ? Qui enfante les gouttes de la rosée ?
\VS{29}De quel sein est sortie la glace ? Et qui enfante le givre du ciel,
\VS{30}pour que les eaux se cachent comme une pierre, et que le dessus de l'abîme soit enchaîné ?
\VS{31}Peux-tu resserer les liens des pléiades ou détacher les chaînes d'orient\FTNT{Am. 5:8.}?
\VS{32}Fais-tu sortir en leur temps les signes du zodiaque, et conduis-tu la Grande Ourse avec ses petits ?
\VS{33}Connais-tu les lois du ciel ? Disposes-tu de son pouvoir sur la terre\FTNT{Jé. 31:35-36 ; Ps. 104:4.} ?
\VS{34}Élèves-tu la voix jusqu'aux nuées, pour que des eaux abondantes te couvrent ?
\VS{35}Envoies-tu les éclairs ? Partent-ils ? Te disent-ils : Nous voici ?
\VS{36}Qui a mis la sagesse dans le cœur, ou qui a donné l'intelligence à l'esprit\FTNT{Ec. 2:26.} ?
\VS{37}Qui Est-ce qui peut avec intelligence compter les nuages, et pour placer les outres des cieux,
\VS{38}Quand la poussière est détrompée par les eaux qui l'arrosent, et que les mottes viennent à se joindre ?
\Chap{39}
\TextTitle{Yahweh démontre son omnipotence}
\VerseOne{}Chasses-tu de la proie pour la lionne, et apaises-tu la faim des lionceaux\FTNT{Ps. 104:21.},
\VS{2}Quand ils se tapissent dans leurs tanières et se tiennent aux aguets dans leur repaire ?
\VS{3}Qui est-ce qui apprête la nourriture au corbeau, quand ses petits crient à Dieu, et qu'ils vont errants, parce qu'ils n'ont point de quoi manger\FTNT{Ps. 104:27 ; Ps. 147:9 ; Mt. 6:26.} ?
\VS{4}Sais-tu quand les boucs de rochers mettent bas ? Observes-tu les biches de rochers quand elles font leurs petits\FTNT{Ps. 29:9.} ?
\VS{5}Comptes-tu les mois de leur gestation, et sais-tu le temps auquel elles font leurs petits ?
\VS{6}Et qu'elles se courbent pour mettre bas leurs petits et se délivrent de leurs douleurs ?
\VS{7}Leurs petits se fortifient, ils croissent en plein air, ils s’en vont et ne reviennent plus vers elles.
\VS{8}Qui a laissé aller libre l’âne sauvage ? et qui a délié les liens de l’âne farouche ?
\VS{9}Auquel j’ai donné le désert pour maison, et la terre inhabitée pour ses retraites\FTNT{Jé. 2:24.} ?
\VS{10}Il se rit du bruit des villes, il n'entend pas les cris d'un exacteur.
\VS{11}Les montagnes qu'il va épiant çà et là, sont ses pâturages, et il cherche toute sorte de verdure. 
\VS{12}Le buffle voudra-t-il te servir, ou demeurera-t-il à ta crèche ? 
\VS{13}Lies-tu le buffle avec son licou pour labourer ? ou rompra-t-il les mottes des vallées après toi ? 
\VS{14}Te fies-tu à lui parce que sa force est grande, et lui abandonnes-tu ton travail? 
\VS{15}Comptes-tu sur lui pour rentrer ta semence, et pour l'amasser sur ton aire ? 
\VS{16}As-tu donné aux paons ce plumage qui est si brillant, ou à l'autruche les ailes et les plumes ? 
\VS{17}Néanmoins elle abandonne ses oeufs à terre, et les fait échauffer sur la poussière ;
\VS{18}et elle oublie que le pied peut les écraser, ou que les bêtes des champs peuvent les fouler. 
\VS{19}Elle est dure envers ses petits, comme s’ils n’étaient pas siens. Son travail est vain, elle ne s’en inquiète pas.
\VS{20}Car Dieu l'a privée de sagesse et ne lui a pas donné l'intelligence.
\VS{21}A la première occasion elle se dresse en haut, et se moque du cheval et de celui qui le monte. 
\VS{22}As-tu donné la force au cheval ? et as-tu revêtu son cou d'un hennissement éclatant comme le tonnerre ? 
\VS{23}Fais-tu bondir le cheval comme la sauterelle ? le son magnifique de ses narines est effrayant.
\VS{24}Il creuse la terre de son pied, il s'égaie dans sa force, il va à la rencontre d'un homme armé ;
\VS{25}Il se rit de la frayeur, il ne s'épouvante de rien, et il ne se détourne point de devant l'épée.
\VS{26}Il n'a point peur des flèches qui sifflent tout autour de lui, ni du fer luisant de la lance et du javelot. 
\VS{27}Il creuse la terre, plein d'émotion et d'ardeur au son de la trompette, et il ne peut se retenir. 
\VS{28}Au son bruyant de la trompette, il dit : En avant ! En avant ! Il flaire de loin la bataille, le tonnerre des capitaines, et le cri de triomphe.
\VS{29}Est-ce par ta sagesse que l'épervier prend son vol, et qu'il étend ses ailes vers le midi ?
\VS{30}Est-ce par ton commandement que l'aigle s'élève, et qu'il place son nid sur les hauteurs\FTNT{Jé. 49:16 ; Abd. 1:4.} ?
\VS{31}Elle habite sur les rochers, et elle s'y tient ; même sur les sommets des rochers et dans des lieux forts. 
\VS{32}De là il découvre le gibier, ses yeux voient de loin.
\VS{33}Ses petits auss sucent le sang ; et là où sont des cadavres, il s'y trouve aussitôt\FTNT{Mt. 24:28 ; Lu. 17:37.}.
\TextTitle{Yahweh lui pose une question}
\VS{34}Yahweh prit encore la parole et dit à Job :
\VS{35}Celui qui conteste avec le Tout-puissant, lui apprendra-t-il quelque chose ? Que celui qui dispute avec Dieu réponde à ceci.
\TextTitle{Réponse de Job}
\VS{36}Alors Job répondit à Yahweh et dit :
\VS{37}Voici, je suis un homme vil ; que te répondrais-je ? Je mets ma main sur ma bouche\FTNT{Ps. 39:10.}.
\VS{38}J'ai parlé une fois, mais je ne répondrai plus ; j'ai même parlé deux fois mais je n'ajouterai plus.
\Chap{40}
\TextTitle{Yahweh questionne encore Job}
\VerseOne{}Et Yahweh répondit à Job du milieu d'un tourbillon, et lui dit :
\VS{2}Ceins maintenant tes reins comme un vaillant homme ; je t'interrogerai et tu m'enseigneras.
\VS{3}Anéantiras-tu mon jugement ? me condamneras-tu pour te justifier\FTNT{Ps. 51:6 ; Ro. 3:4.} ?
\VS{4}As-tu un bras comme celui de Dieu ; tonnes-tu de la voix, comme lui ?
\VS{5}Pare-toi maintenant de magnificence et de grandeur, et revêts-toi de majesté et de gloire.
\VS{6}Répands les fureurs de ta colère, d'un regard, humilie tous les orgueilleux
\VS{7}D'un regard humilie les orgueilleux, écrase sur place les méchants,
\VS{8}cache-les tous ensemble dans la poussière, enferme leur face dans les ténèbres !
\VS{9}Alors je rends hommage à mon sauveur qui me sauve par sa droite.
\VS{10}Voici le béhémoth, que j'ai façonné comme toi ! Il mange de l'herbe comme le bœuf.
\VS{11}Regarde donc, sa force est dans ses reins, et sa puissance dans les muscles de son ventre ;
\VS{12}il plie sa queue aussi ferme qu'un cèdre ; les tendons de ses cuisses sont entrelacés ;
\VS{13}ses os sont des tubes d'airain, ses membres sont comme des barres de fer.
\VS{14}C’est le chef-d’œuvre de Dieu ; celui qui l’a fait lui a donné son épée.
\VS{15}Car les montagnes lui apportent sa pâture, là où se jouent toutes les bêtes des champs.
\VS{16}Il se couche sous les lotus, caché dans les roseaux et les marécages ;
\VS{17}les lotus le couvrent de leur ombre, les saules du torrent l'enveloppent.
\VS{18}Voilà, il engloutit une rivière en buvant, et il ne s'en retire pas vite ; et il ne s'étonnerait pas quand le Jourdain se dégorgerait dans sa gueule. 
\VS{19}Il l'engloutit en le voyant, et son nez passe au travers des empêchements qu'il rencontre.  
\VS{20}Attireras-tu le léviathan à l'hameçon ? Saisiras-tu sa langue avec une corde ?
\VS{21}Mettras-tu un jonc dans ses narines ? Lui perceras-tu la mâchoire avec un crochet ?
\VS{22}Accumulera-t-il les supplications ? Te parlera-t-il d'une voix douce ?
\VS{23}Fera-t-il une alliance avec toi, pour te prendre pour toujours comme esclave ?
\VS{24}Joueras-tu avec lui comme avec un oiseau ? L'attacheras-tu pour amuser les jeunes filles ?
\VS{25}Les pêcheurs en trafiquent-ils ? Le partagent-ils entre les marchands ?
\VS{26}Couvriras-tu sa peau de dards, et sa tête de harpons ?
\VS{27}Mets ta main contre lui, et tu ne te souviendras plus de l'attaquer.
\VS{28}Voici, on est trompé dans son attente ; à sa vue n'est-on pas terrassé ?
\Chap{41}
\VerseOne{}Nul n'est assez féroce pour l'exciter ; qui donc me résisterait en face ?
\VS{2}De qui suis-je le débiteur ? Je le paierai. Sous le ciel tout m'appartient\FTNT{Ex. 19:5 ; De. 10:14 ; Ps. 24:1 ; Ps. 50:12 ; 1 Co. 10:26 ; Ro. 11:35.}.
\VS{3}Je veux encore parler de ses discours, et de sa force, et de la beauté de sa structure.
\VS{4}Qui découvrira son vêtement devant ma face ? Qui viendra freiner ses mâchoires par un mors ?
\VS{5}Qui ouvrira les portes devant sa face ? Autour du lion habite la terreur.
\VS{6}Ses magnifiques et puissants boucliers sont fermés comme un sceau ;
\VS{7}ils se serrent l'un contre l'autre, et l'air n'entrerait pas entre eux ;
\VS{8}ce sont des frères qui s'embrassent, se saisissent, demeurent inséparables.
\VS{9}Ses éternuements font briller la lumière, ses yeux sont comme les paupières de l'aurore.
\VS{10}Des flammes viennent de sa bouche, des étincelles de feu s'en échappent.
\VS{11}Une fumée sort de ses narines, comme d'un chaudron qui bout, d'une chaudière ardente.
\VS{12}Son souffle allume les charbons, de sa bouche sort la flamme.
\VS{13}La force a son cou pour demeure, et l'effroi bondit devant lui.
\VS{14}Ses parties charnues sont jointes ensemble, fondues sur lui, inébranlables.
\VS{15}Son cœur est dur comme la pierre, dur comme la meule inférieure.
\VS{16}Quand il se lève, les plus vaillants ont peur, et l'épouvante les fait quitter le droit chemin.
\VS{17}C'est en vain qu'on l'attaque avec l'épée ; la lance, le javelot, la cuirasse ne servent à rien.
\VS{18}Il regarde le fer comme de la paille, l'airain comme du bois pourri.
\VS{19}La flèche de l'arc ne le met pas en fuite, les pierres de la fronde sont pour lui changés en chaume.
\VS{20}Il ne voit dans la massue qu'un brin de paille, il rit au sifflement des dards.
\VS{21}Sous son ventre sont des pointes aiguës : On dirait une herse qu'il étend sur la boue.
\VS{22}Il fait bouillir les profondeurs de la mer comme une chaudière, il la traite comme un vase rempli de parfums.
\VS{23}Il laisse après lui un sentier lumineux ; l'abîme prend la chevelure d'un vieillard.
\VS{24}Sur la terre nul n'est son maître ; il a été façonné pour ne rien craindre.
\VS{25}Il regarde avec dédain tout ce qui est élevé, il est le roi des plus fiers animaux.
\Chap{42}
\TextTitle{Job reconnaît la souveraineté de Dieu et s'humilie}
\VerseOne{}Job répondit à Yahweh et dit :
\VS{2}Je sais que tu peux tout, et qu'on ne saurait t'empêcher de faire ce que tu penses.
\VS{3}Quel est celui qui a la folie d'obscurcir mes conseils ? Oui, j'ai parlé sans les comprendre, de merveilles qui me dépassent et que je ne connais pas\FTNT{Ps. 40:6 ; Ps. 131:1 ; Ps. 139:6.}.
\VS{4}Écoute-moi maintenant, et je parlerai ; je t'interrogerai et tu m'instruiras.
\VS{5}J'avais entendu parler de toi ; mais maintenant mon œil t'a vu.
\VS{6}C'est pourquoi je me condamne et je me repens d'avoir ainsi parlé et je m'en sur la poussière et sur la cendre.
\VS{7}Après que Yahweh eut ainsi parlé à Job, il dit à Eliphaz de Théman : Ma colère est embrasée contre toi et contre tes deux amis, parce que vous n'avez pas parlé de moi avec droiture comme Job, mon serviteur.
\VS{8} C'est pourquoi prenez maintenant sept taureaux et sept béliers, allez auprès de mon serviteur Job, et offrez un holocauste pour vous. Job, mon serviteur, priera pour vous, et certainement j'exaucerai sa prière, afin que je ne vous traite pas selon votre folie ; car vous n'avez pas parlé de moi avec droiture, comme mon serviteur Job.
\VS{9} Ainsi Eliphaz de Théman, Bildad de Schuach, et Tsophar de Naama allèrent et firent comme Yahweh leur avait commandé ; et Yahweh exauça la prière de Job.
\VS{10}Yahweh rétablit Job de sa captivité, quand il eut prié pour ses amis ; et Yahweh lui ajouta le double de tout ce qu'il avait possédé.
\VS{11}Ses frères, ses sœurs, et tous ceux qui l'avaient connu auparavant vinrent tous le visiter, et ils mangèrent avec lui dans sa maison. Ils compatirent et le consolèrent au sujet de tout le mal que Yahweh avait fait venir sur lui, et chacun lui donna une kesita et un anneau d'or.
\VS{12}Pendant ses dernières années, Job reçut de Yahweh plus de bénédictions qu'il n'en avait reçu dans les premières. Il posséda quatorze mille brebis, six mille chameaux, mille paires de bœufs, et mille ânesses.
\VS{13}Il eut aussi sept fils et trois filles :
\VS{14}Il donna à la première le nom de Jemima, à la seconde celui de Ketsia, à la troisième celui de Kéren-Happuc.
\VS{15}Et il ne se trouvait pas de femmes aussi belles que les filles de Job dans tout le pays. Leur père leur donna une part de l'héritage parmi leurs frères.
\VS{16}Job vécut, après ces choses, cent quarante ans, et il vit ses fils et les fils de ses fils jusqu'à la quatrième génération.
\VS{1} Job mourut âgé et rassasié de jours.
\PPE{}
\end{multicols}

\clearpage%& -output-directory="./pdf"
% type document & taille police
\documentclass[11pt]{book}
% package format document
\usepackage[paperwidth=6.5in, paperheight=9.05in, top=0in, bottom=0in, left=0in, right=0in]{geometry}
% formatage marges, etc.
\setlength{\voffset}{-0.7in} % offset haut
%\setlength{\hoffset}{-0.3in} % offset gauche
\setlength{\topmargin}{0in} % marge en tête
\setlength{\headsep}{0.2in} % marge header/body
\setlength{\oddsidemargin}{-0.5in} % marge texte gauche
\setlength{\evensidemargin}{-0.5in} % marge texte droite
\setlength{\textheight}{8in} % hauteur du texte
\setlength{\textwidth}{5.5in} % largeur du texte
\setlength{\columnseprule}{0.4pt} % épaisseur séparateur colonne
\setlength{\parskip}{0pt} % espace entre paragraphes
% package pour afficher les cadres
%\usepackage{showframe}
% package langue
\usepackage[francais]{babel}
% package polices système
\usepackage{fontspec}
% définition police
\setmainfont[Ligatures=TeX,Scale=0.95]{Liberation Serif}
\setsansfont{Liberation Sans}
\setmonofont{Liberation Mono}
% package titlesec
\usepackage{titlesec}
% package multicolonne
\usepackage{multicol}
% package liens cliquables
\usepackage[xetex]{hyperref}
% package inclusion copyright (dépandant de hyperref)
\usepackage{hyperxmp}
% copyright
\hypersetup{
pdfauthor = {ANJC Productions},
pdftitle = {Bible de Jésus-Christ},
pdfkeywords = {BJC, Bible, Jesus},
pdfcopyright = {ANJC Productions. Distribution et Diffusion Libres - Pas d'Utilisation Commerciale - Pas de Dénaturation de l'Œuvre - International},
pdflicenseurl = {http://www.bibledejesuschrist.org/}
}
% ???
\setcounter{collectmore}{-1}
% style
\pagestyle{myheadings}
% ???
\sloppy\hyphenpenalty=2000
% titres de livres
\newcommand{\ShortTitle}[1]{\def\webbook{#1}\par\goodbreak\bigskip\setcounter{footnote}{0}}
\newcommand{\BookTitle}[1]{\par\goodbreak\bigskip{\parindent=0mm\begin{center}{\small\bfseries{\LARGE #1\nopagebreak}}\end{center}}\addcontentsline{toc}{subsection}{#1}\nopagebreak\par\nobreak}
% chapitres
\newcommand{\Chap}[1]{\def\webchap{#1:}\def\webvs{1}\def\vchap{#1}\ssubsection{\centerline{\textbf{{CHAPITRE\ #1}}}}}
% versets
\newcommand{\VerseOne}{\def\webvs{1}{\up{\footnotesize 1}}\markboth{\webbook\ \webchap 1}{\webbook\ \webchap 1}}
\newcommand{\VS}[1]{\def\webvs{#1}{\up{\footnotesize #1}}\markboth{\webbook\ \webchap #1}{\webbook\ \webchap #1}}
\newcommand{\vref}[1]{\NoAutoSpaceBeforeFDP{#1}}
% commentaires
%\interfootnotelinepenalty=10000 % longueur max commentaires
\renewcommand{\thefootnote}{\alph{footnote}} % repères alphabetiques
\renewcommand{\footnoterule}{\hrule width \textwidth} % longueur ligne
\newcommand{\FTNT}[1]{\ifnum\value{footnote}>25\setcounter{footnote}{0}\fi\footnote{[\NoAutoSpaceBeforeFDP{\webchap\webvs}]\ #1}}
% commentaire sur les titres
\newcounter{webvst}
\newcommand{\FTNTT}[1]{
% intialisation de l'indice de note
\ifnum \value{footnote}>25 \setcounter{footnote}{0} \fi
% initialisation de la référence du numéro de verset
\setcounter{webvst}{\webvs}
% si le titre est sur le premier verset, incrémenter de 1
\ifnum \value{webvst}>1 \addtocounter{webvst}{1} \fi
% écriture note
\footnote{[\NoAutoSpaceBeforeFDP{\webchap\thewebvst}]\ #1}
}
% titres de paragraphes
\titlespacing*{\subsection}{0pt}{5pt plus 0pt minus 0pt}{5pt plus 0pt minus 0pt}
\titlespacing*{\subsubsection}{0pt}{5pt plus 0pt minus 0pt}{5pt plus 0pt minus 0pt}
\newcommand{\ssubsection}[1]{\subsection*{\centering\footnotesize\normalfont #1}\PP}
\newcommand{\ssubsubsection}[1]{\subsubsection*{\centering\footnotesize\normalfont #1}\PP}
\newcommand{\TextTitle}[1]{\ssubsubsection{[\textit{#1}]}}
\newcommand{\TextDial}[1]{{\scriptsize[\textit{#1}]}}
% dictionnaire
\newcommand{\DicoEntry}[1]{\smallskip\parindent=0mm{\textbf{#1}}\markboth{#1}{#1}}
% commandes diverses
\newcommand{\BFont}{\normalfont\small}
\newcommand{\PP}{\par\parindent=0mm}
\newcommand{\PPE}{\par\parindent=4mm}
% debut document
\begin{document}
% en-tête pages
\makeatletter
\def\@evenhead{{\NoAutoSpaceBeforeFDP{\small{\rightmark\hfil\thepage\hfil\leftmark}}}}
\def\@oddhead{{\NoAutoSpaceBeforeFDP{\small{\rightmark\hfil\thepage\hfil\leftmark}}}}
\makeatother
% inclusion des livres
\pagenumbering{arabic}
\clearpage%& -output-directory="./pdf"
% type document & taille police
\documentclass[11pt]{book}
% package format document
\usepackage[paperwidth=6.5in, paperheight=9.05in, top=0in, bottom=0in, left=0in, right=0in]{geometry}
% formatage marges, etc.
\setlength{\voffset}{-0.7in} % offset haut
%\setlength{\hoffset}{-0.3in} % offset gauche
\setlength{\topmargin}{0in} % marge en tête
\setlength{\headsep}{0.2in} % marge header/body
\setlength{\oddsidemargin}{-0.5in} % marge texte gauche
\setlength{\evensidemargin}{-0.5in} % marge texte droite
\setlength{\textheight}{8in} % hauteur du texte
\setlength{\textwidth}{5.5in} % largeur du texte
\setlength{\columnseprule}{0.4pt} % épaisseur séparateur colonne
\setlength{\parskip}{0pt} % espace entre paragraphes
% package pour afficher les cadres
%\usepackage{showframe}
% package langue
\usepackage[francais]{babel}
% package polices système
\usepackage{fontspec}
% définition police
\setmainfont[Ligatures=TeX,Scale=0.95]{Liberation Serif}
\setsansfont{Liberation Sans}
\setmonofont{Liberation Mono}
% package titlesec
\usepackage{titlesec}
% package multicolonne
\usepackage{multicol}
% package liens cliquables
\usepackage[xetex]{hyperref}
% package inclusion copyright (dépandant de hyperref)
\usepackage{hyperxmp}
% copyright
\hypersetup{
pdfauthor = {ANJC Productions},
pdftitle = {Bible de Jésus-Christ},
pdfkeywords = {BJC, Bible, Jesus},
pdfcopyright = {ANJC Productions. Distribution et Diffusion Libres - Pas d'Utilisation Commerciale - Pas de Dénaturation de l'Œuvre - International},
pdflicenseurl = {http://www.bibledejesuschrist.org/}
}
% ???
\setcounter{collectmore}{-1}
% style
\pagestyle{myheadings}
% ???
\sloppy\hyphenpenalty=2000
% titres de livres
\newcommand{\ShortTitle}[1]{\def\webbook{#1}\par\goodbreak\bigskip\setcounter{footnote}{0}}
\newcommand{\BookTitle}[1]{\par\goodbreak\bigskip{\parindent=0mm\begin{center}{\small\bfseries{\LARGE #1\nopagebreak}}\end{center}}\addcontentsline{toc}{subsection}{#1}\nopagebreak\par\nobreak}
% chapitres
\newcommand{\Chap}[1]{\def\webchap{#1:}\def\webvs{1}\def\vchap{#1}\ssubsection{\centerline{\textbf{{CHAPITRE\ #1}}}}}
% versets
\newcommand{\VerseOne}{\def\webvs{1}{\up{\footnotesize 1}}\markboth{\webbook\ \webchap 1}{\webbook\ \webchap 1}}
\newcommand{\VS}[1]{\def\webvs{#1}{\up{\footnotesize #1}}\markboth{\webbook\ \webchap #1}{\webbook\ \webchap #1}}
\newcommand{\vref}[1]{\NoAutoSpaceBeforeFDP{#1}}
% commentaires
%\interfootnotelinepenalty=10000 % longueur max commentaires
\renewcommand{\thefootnote}{\alph{footnote}} % repères alphabetiques
\renewcommand{\footnoterule}{\hrule width \textwidth} % longueur ligne
\newcommand{\FTNT}[1]{\ifnum\value{footnote}>25\setcounter{footnote}{0}\fi\footnote{[\NoAutoSpaceBeforeFDP{\webchap\webvs}]\ #1}}
% commentaire sur les titres
\newcounter{webvst}
\newcommand{\FTNTT}[1]{
% intialisation de l'indice de note
\ifnum \value{footnote}>25 \setcounter{footnote}{0} \fi
% initialisation de la référence du numéro de verset
\setcounter{webvst}{\webvs}
% si le titre est sur le premier verset, incrémenter de 1
\ifnum \value{webvst}>1 \addtocounter{webvst}{1} \fi
% écriture note
\footnote{[\NoAutoSpaceBeforeFDP{\webchap\thewebvst}]\ #1}
}
% titres de paragraphes
\titlespacing*{\subsection}{0pt}{5pt plus 0pt minus 0pt}{5pt plus 0pt minus 0pt}
\titlespacing*{\subsubsection}{0pt}{5pt plus 0pt minus 0pt}{5pt plus 0pt minus 0pt}
\newcommand{\ssubsection}[1]{\subsection*{\centering\footnotesize\normalfont #1}\PP}
\newcommand{\ssubsubsection}[1]{\subsubsection*{\centering\footnotesize\normalfont #1}\PP}
\newcommand{\TextTitle}[1]{\ssubsubsection{[\textit{#1}]}}
\newcommand{\TextDial}[1]{{\scriptsize[\textit{#1}]}}
% dictionnaire
\newcommand{\DicoEntry}[1]{\smallskip\parindent=0mm{\textbf{#1}}\markboth{#1}{#1}}
% commandes diverses
\newcommand{\BFont}{\normalfont\small}
\newcommand{\PP}{\par\parindent=0mm}
\newcommand{\PPE}{\par\parindent=4mm}
% debut document
\begin{document}
% en-tête pages
\makeatletter
\def\@evenhead{{\NoAutoSpaceBeforeFDP{\small{\rightmark\hfil\thepage\hfil\leftmark}}}}
\def\@oddhead{{\NoAutoSpaceBeforeFDP{\small{\rightmark\hfil\thepage\hfil\leftmark}}}}
\makeatother
% inclusion des livres
\pagenumbering{arabic}
\clearpage\input{../bjc_2014/22-Cantiques}
\end{document}

\end{document}

\clearpage%& -output-directory="../../pdf/books/"
% type document & taille police
\documentclass[11pt]{book}
% package format document
\usepackage[paperwidth=6.5in, paperheight=9.05in, top=0in, bottom=0in, left=0in, right=0in]{geometry}
% formatage marges, etc.
\setlength{\voffset}{-0.7in} % offset haut
%\setlength{\hoffset}{-0.3in} % offset gauche
\setlength{\topmargin}{0in} % marge en tête
\setlength{\headsep}{0.2in} % marge header/body
\setlength{\oddsidemargin}{-0.5in} % marge texte gauche
\setlength{\evensidemargin}{-0.5in} % marge texte droite
\setlength{\textheight}{8in} % hauteur du texte
\setlength{\textwidth}{5.5in} % largeur du texte
\setlength{\columnseprule}{0.4pt} % épaisseur séparateur colonne
\setlength{\parskip}{0pt} % espace entre paragraphes
% package pour afficher les cadres
%\usepackage{showframe}
% package langue
\usepackage[francais]{babel}
% package polices système
\usepackage{fontspec}
% définition police
\setmainfont[Ligatures=TeX,Scale=0.95]{Liberation Serif}
\setsansfont{Liberation Sans}
\setmonofont{Liberation Mono}
% package parskip pour espaces entre paragraphes
\usepackage{parskip}
% package multicolonne
\usepackage{multicol}
% package liens cliquables
\usepackage[xetex]{hyperref}
% package inclusion copyright (dépandant de hyperref)
\usepackage{hyperxmp}
% copyright
\hypersetup{
    pdfauthor = {ANJC Productions},
    pdftitle = {Bible de Jésus-Christ},
    pdfkeywords = {BJC, Bible, Jesus},
    pdfcopyright = {ANJC Productions. Distribution et Diffusion Libre - Pas d'Utilisation Commerciale - Pas de Dénaturation de l'Œuvre - International},
    pdflicenseurl = {http://www.bible-de-jesus.org/}
}
% ???
\setcounter{collectmore}{-1}
% style
\pagestyle{myheadings}
% ???
\sloppy\hyphenpenalty=2000
% titres de livres
\newcommand{\ShortTitle}[1]{\def\webbook{#1}\par\goodbreak\bigskip\setcounter{footnote}{0}}
\newcommand{\BookTitle}[1]{\par\goodbreak\bigskip{\parindent=0mm\begin{center}{\small\bfseries{\LARGE #1\nopagebreak}}\end{center}}\addcontentsline{toc}{subsection}{#1}\nopagebreak\par\nobreak}
% chapitres
\newcommand{\Chap}[1]{\def\webchap{#1:}\def\webvs{0}\def\vchap{#1}\ssubsection{\centerline{\textbf{{CHAPITRE\ #1}}}}}
% versets
\newcommand{\VerseOne}{\def\webvs{1}{\up{\footnotesize 1}}\markboth{\webbook\ \webchap 1}{\webbook\ \webchap 1}}
\newcommand{\VS}[1]{\def\webvs{#1}{\up{\footnotesize #1}}\markboth{\webbook\ \webchap #1}{\webbook\ \webchap #1}}
\newcommand{\vref}[1]{\NoAutoSpaceBeforeFDP{#1}}
% commentaires
%\interfootnotelinepenalty=10000 % longueur max commentaires
\renewcommand{\thefootnote}{\alph{footnote}} % repères alphabetiques
\renewcommand{\footnoterule}{\hrule width \textwidth} % longueur ligne
\newcommand{\FTNT}[1]{\ifnum\value{footnote}>25\setcounter{footnote}{0}\fi\footnote{[\webchap\webvs]\ #1}}
\newcounter{webvst}
\newcommand{\FTNTT}[1]{
    \ifnum\value{footnote}>25\setcounter{footnote}{0}\fi
    \setcounter{webvst}{\webvs}\addtocounter{webvst}{1}
    \footnote{[\webchap\thewebvst]\ #1}
}
% titres de paragraphes
\newcommand{\ssubsection}[1]{\subsection*{\centering\footnotesize\normalfont #1}}
\newcommand{\ssubsubsection}[1]{\subsubsection*{\centering\footnotesize\normalfont #1}}
\newcommand{\TextTitle}[1]{\ssubsubsection{[\textit{#1}]}}
% dictionnaire
\newcommand{\DicoEntry}[1]{\smallskip\parindent=0mm{\textbf{#1}}\markboth{#1}{#1}}
% commandes diverses
\newcommand{\BFont}{\normalfont\small}
\newcommand{\PP}{\par\parindent=0mm}
\newcommand{\PPE}{\par\parindent=4mm}
% debut document
\begin{document}
% en-tête vide
\makeatletter
    \def\@evenhead{}
    \def\@oddhead{}
\makeatother
% inclusion intro
%\begin{center}{\LARGE Introduction}\end{center}
\begin{small}
\subsection*{Pourquoi cette Bible révisée~?}

En novembre 2013, alors que j'étais en prière, je demandais au Seigneur ce qu'il attendait de moi. Ce dernier m'a répondu à travers plusieurs songes dans lesquels il me disait de réviser la Bible. Je dois dire que j'ai eu du mal à croire que Dieu puisse me demander une telle chose. De plus, je me sentais incapable d'assumer un si grand projet, aussi je lui ai demandé à plusieurs reprises de me confirmer que c'était bien sa volonté, chose qu'il a faite. J'ai ensuite parlé de ce que j'avais reçu à des frères et sœurs qui travaillent avec moi et ces derniers m'ont confirmé que cette vision venait bien du Seigneur. Une dynamique s'est créée aussitôt et bien qu'aucun d'entre nous ne se sentît à la hauteur de la tâche qui nous était confiée, nous nous sommes rapidement organisés pour concrétiser cette vision, comptant sur le Seigneur pour qu'il nous donne les capacités et la sagesse dont nous avions besoin.\bigskip

Deux constats majeurs nous ont amenés à la conclusion qu'une révision de la Bible était plus que nécessaire. Tout d'abord, la plupart des bibles modernes les plus diffusées sont basées sur le texte minoritaire comportant une quantité importante de fautes de traduction, d'omissions et de rajouts qui altèrent la compréhension du message et induisent par conséquent le lecteur en erreur. Or il est du devoir de tout chrétien de mettre en pratique la Parole, notamment en veillant sur son authenticité.\bigskip

«~\emph{Car, je vous le dis en vérité, tant que le ciel et la terre ne passeront point, il ne disparaîtra pas de la loi un seul iota ou un seul trait de lettre jusqu'à ce que tout soit arrivé. Celui donc qui aura violé l'un de ces petits commandements, et qui aura enseigné les hommes à faire de même, sera appelé le plus petit au Royaume des cieux~; mais celui qui les observera, et qui enseignera à les observer, celui-là sera appelé grand au Royaume des cieux.}~» Matthieu 5:18-19.\bigskip

«~\emph{Je le déclare à quiconque entend les paroles de la prophétie de ce livre~: Si quelqu'un y ajoute quelque chose, Dieu le frappera des fléaux décrits dans ce livre. Et si quelqu'un retranche quelque chose des paroles du livre de cette prophétie, Dieu retranchera sa part de l'arbre de vie, et de la ville sainte, décrits dans ce livre.}~» Apocalypse 22:18-19.\bigskip

Nous ne devons pas oublier que la Bible a été initialement écrite en trois langues, à savoir l'hébreu, le grec et quelques versets en araméen. En réalisant cette révision, notre but est de restituer le sens des mots d'origine et d'expurger toute l'influence de l'ennemi. Ce travail a permis de mettre en lumière une évidence~: la personne de Jésus-Christ occupe une place centrale de Genèse à Apocalypse, ce qui ne fait que confirmer et attester sa divinité.\bigskip

«~\emph{Puis il leur dit~: C'est là ce que je disais lorsque j'étais encore avec vous, qu'il fallait que s'accomplisse tout ce qui est écrit de moi dans la loi de Moïse, dans les prophètes, et dans les psaumes.}~» Luc 24:44.\bigskip

Ensuite, nous déplorons le fait que la majorité des bibles en circulation soient vendues alors que Jésus-Christ a dit «~\emph{Vous l'avez reçu gratuitement, donnez-le gratuitement}~» (Mt. 10:8). Il est donc impensable que celui qui a chassé du temple vendeurs et changeurs puisse approuver un seul instant le commerce qui est fait avec sa Parole (Jn. 2:14-16).\bigskip

«~\emph{Vous tous qui avez soif, venez aux eaux, et vous qui n'avez pas d'argent, venez, achetez et mangez~; venez, dis-je, achetez du vin et du lait sans argent et sans rien payer~!}~» Esaïe 55:1.\bigskip

«~\emph{Il me dit aussi~: Tout est accompli. Je suis l'Alpha et l'Oméga, le commencement et la fin. A celui qui a soif, je lui donnerai de la source d'eau vive, gratuitement.}~» Apocalypse 21:6.\bigskip

«~\emph{Et l'Esprit et l'épouse disent~: Viens. Et que celui qui entend dise~: Viens. Et que celui qui a soif vienne~; que celui qui veut prenne gratuitement de l'eau de la vie.}~» Apocalypse 22:17.\bigskip

Les apôtres ont scrupuleusement respecté l'ordre du Seigneur en Matthieu 10:8. Pierre a dénoncé avec la plus grande sévérité Simon, le magicien qui avait eu la folie de croire que le don de Dieu pouvait être monnayé. Et durant tout son service, Paul a enseigné l'Evangile gratuitement.\bigskip

«~\emph{Puis ils leur imposèrent les mains, et ils reçurent le Saint-Esprit. Lorsque Simon vit que le Saint-Esprit était donné par l’imposition des mains des apôtres, il leur présenta de l’argent, en leur disant : Donnez-moi aussi ce pouvoir, afin que tous ceux à qui j’imposerai les mains reçoivent le Saint-Esprit. Mais Pierre lui dit~: Que ton argent périsse avec toi, puisque tu as estimé que le don de Dieu s’acquérait avec de l’argent. Tu n’as point de part ni d’héritage en cette affaire ; car ton coeur n’est point droit devant Dieu. Repens-toi donc de cette méchanceté, et prie Dieu, afin que, s’il est possible, la pensée de ton coeur te soit pardonnée. Car je vois que tu es dans un fiel très amer et dans un lien d’iniquité.}~» Actes 8:17-23.\bigskip

«~\emph{Je n'ai désiré ni l'argent, ni l'or, ni les vêtements de personne.}~» Actes 20:33.\bigskip

«~\emph{Quelle récompense en ai-je donc~? C’est qu’en prêchant l’Evangile, je prêche l’Evangile de Christ sans qu’il en coûte rien, afin que je n’abuse pas de mon pouvoir dans l’Evangile.}~» 1 Corinthiens 9:18.\bigskip

Nous pensons qu'il est juste et honnête que la Bible porte le nom de son véritable auteur et qu'elle soit gratuitement diffusée selon sa volonté et l'ordre clair qu'il a donné. Cette Bible s'appelle donc La Bible de Jésus-Christ et est gratuitement mise à la disposition de ceux qui souhaitent se la procurer.\bigskip

\subsection*{Comment a été réalisée cette révision~?}

Pour réaliser cette révision, nous nous sommes appuyés sur le texte majoritaire (originaux et traductions). Ainsi, tout en essayant de conserver un vocabulaire qui soit à la portée de tous, certains mots et expressions ont été changés pour restituer pleinement leur signification initiale. A titre d'exemple, vous constaterez régulièrement que certains mots sont répétés deux fois de suite. Cela n'est pas une erreur mais la restitution littérale de certaines expressions qui insistent sur une vérité (voir commentaire en Gn. 2:15-17). En effet, Dieu parle une fois et une seconde fois pour avertir les hommes (Job. 33:14). «~Et quant à ce que le songe a été réitéré à Pharaon pour la seconde fois, c'est que la chose est arrêtée de la part de Dieu, et que Dieu se hâtera de l'exécuter~» (Genèse 41:32).\bigskip

Les Ecrits ont été classés dans l'ordre de la tradition juive pour le Tanakh et dans l'ordre chronologique de leur rédaction pour les épîtres afin de permettre au lecteur de mieux comprendre le contexte et le déroulement de la prophétie biblique. L'appellation «~Ancien Testament~» a été remplacée par l'acronyme hébreu Tanakh (voir sommaire). Quant à ce qu'on appelle communément le «~Nouveau Testament~», il sera désormais question du Testament de Jésus. En effet, l'Ancienne Alliance n'étant pas un testament, on ne peut donc pas parler de «~Nouveau Testament~» mais plutôt d'une Nouvelle Alliance (voir commentaires en Ex. 19:5~; Mt. 27:51~; Jn. 19:30).\bigskip

Je remercie tout d'abord le Seigneur pour son aide précieuse qu'il m'a apportée pour la révision de cette Bible, ainsi qu'à celles et ceux qui m’ont assisté dans ce travail.\newline

\begin{flushright}
Shora KUETU
\end{flushright}
\end{small}

%% formatage sommaire
%\makeatletter
%\renewcommand\tableofcontents{
%    \begin{center}{\LARGE Sommaire}\end{center}
%    \setlength{\columnseprule}{0pt} % désactivation séparateur colonne temp
%    \begin{multicols}{2}\medskip\footnotesize{\@starttoc{toc}}\end{multicols}
%    \setlength{\columnseprule}{0.4pt} % réactivation séparateur colonne
%}
%\makeatother
%% inclusion table des matières
%\clearpage\tableofcontents\clearpage
%% en-tête pages
%\makeatletter
\def\@evenhead{{\NoAutoSpaceBeforeFDP{\small{\rightmark\hfil\thepage\hfil\leftmark}}}}
\def\@oddhead{{\NoAutoSpaceBeforeFDP{\small{\rightmark\hfil\thepage\hfil\leftmark}}}}
\makeatother
% inclusion des livres
\addcontentsline{toc}{chapter}{Tanakh}\pagenumbering{arabic}\clearpage
%\addcontentsline{toc}{section}{Torah (Loi)}\clearpage
%\clearpage\ShortTitle{Genèse}\BookTitle{Genèse}\BFont
\noindent\hrulefill
{\footnotesize
\textit{
\bigskip
{\centering{}
\\Auteur : Probablement Moïse
\\(Heb. : Bereshit)
\\Signification : Au commencement
\\Thème : Le Messie d'Israël
\\Date de rédaction : Env. 1450-1410 av. J.-C.\\}
}
%\bigskip
\textit{
\\Premier livre du Tanakh, la Genèse est le livre des commencements.
Elle relate l’histoire des origines de l’humanité, la création des cieux, de la terre et de tout ce qui s’y trouve par Yahweh, le Dieu créateur.
%\bigskip
\\Il y est décrit le péché de l’homme et sa séparation d’avec Dieu, ainsi que la décadence de l’univers qui en résulta. En réponse à la méchanceté du cœur de l’homme, Yahweh  exerça sa justice en détruisant la terre par le déluge.
Dans sa prescience, Yahweh avait cependant résolu de se réconcilier avec l’homme. Il se révéla donc comme sauveur en accordant sa grâce à Noé et à sa famille. Après cet événement, les hommes se tournèrent une fois de plus vers le mal en tentant Dieu par la construction de la tour de Babel, œuvre à l’origine de la dispersion des nations.
%\bigskip
\\Ce livre présente aussi l’élection d’Abraham, originaire d’Ur en Chaldée - actuelle Mésopotamie - qui reçut la promesse divine de devenir une grande nation, en qui toutes les familles de la terre seraient bénies. Le récit se poursuit par l’histoire de ses descendants Isaac, Jacob et ses douze fils,  qui formèrent par la suite la nation d’Israël.\bigskip
}
}
\par\nobreak\noindent\hrulefill
\begin{multicols}{2}
\Chap{1}
\VerseOne{}Au commencement, Dieu créa les cieux et la terre.
\TextTitle{La terre devient informe et vide}
\VS{2}Et la terre devint informe et vide\FTNT{Les termes «~informe~» et «~vide~» viennent des mots hébreux «~tohuw~» et «~bohuw~»  qui désignent la confusion, le chaos, la vanité.}, les ténèbres étaient à la surface de l'abîme ; et l'Esprit de Dieu se mouvait au-dessus des eaux.
\TextTitle{Jour «~un~» : Apparition de la lumière}
\VS{3}Dieu dit : Que la lumière apparaisse\FTNT{«~Que la lumière apparaisse !~» (Es. 9:1 ; Mt. 4:16 ; Jn. 1:1-5). Cette lumière n’est autre que Yahweh lui-même qui va s’incarner en la personne de Jésus-Christ pour chasser les ténèbres (2 S. 22:9-12 ; Es. 60:1 ; 60:19-20 ; Jn. 1:1-2 ; 8:12-14 ; 2 Co. 4:6).} ; et la lumière apparut.
\VS{4}Et Dieu vit que la lumière était bonne ; et Dieu sépara la lumière des ténèbres.
\VS{5}Dieu appela la lumière jour, et il appela les ténèbres nuit\FTNT{La lumière et les ténèbres, ainsi que leurs champs lexicaux respectifs, personnifient  souvent Jésus et Satan.  Ainsi, Jésus est la Lumière du monde (Jn. 9:5), l’Etoile brillante du matin (Ap. 22:16), le Soleil levant ou le Soleil de la justice (Mal. 4:2 ; Ps. 19:6 ; Lu. 1:78). Il est associé au jour (Jn. 9:4) d’où les expressions  «~Jour du Seigneur~» (1 Th. 5:2) ou «~Jour de Yahweh~» (Joë. 1:15). A l’inverse, la Bible associe Satan aux ténèbres (Es. 8:23 ; Ps. 143:3 ; Ep. 6:12 ; Col. 1:13) et à la nuit (Jn. 9:4 ; Ro. 13:12).}. Ainsi fut le soir, ainsi fut le matin\FTNT{Contrairement au calendrier grégorien où le jour commence à minuit, selon Dieu et le calendrier hébraïque, le jour commence le soir à 18 heures pour se terminer le lendemain à la même heure. Voir commentaire en Mc. 16:9.} ; ce fut le  jour un\FTNT{L’hébreu utilise le terme «~ehad~» qui signifie «~un~», au sens de l’indivisible, pour qualifier le premier jour. Ce jour nous parle de Yahweh tel qu’il s’est présenté à son peuple sur le mont Sinaï en De. 6:4:«~Shema Yisrael Yahweh elohénou Yahweh ehad~» («~écoute Israël, Yahweh [est] notre Dieu, Yahweh [est] UN~»). Un n’est pas divisible sinon on obtient un zéro ce qui équivaut au néant. Dieu est tout sauf le néant, il remplit tout (Ep. 1:23), il est partout (Ps. 139:7-13), les cieux des cieux ne peuvent le contenir (1 R. 8:27).}.
\TextTitle{Second jour : Une étendue entre les eaux}
\VS{6}Puis Dieu dit : Qu'il y ait une étendue entre les eaux, et qu'elle sépare les eaux d'avec les eaux.
\VS{7}Dieu donc fit l'étendue, et il sépara les eaux qui sont au-dessous de l'étendue d'avec celles qui sont au-dessus de l'étendue, et il fut ainsi.
\VS{8}Et Dieu appela l'étendue cieux. Ainsi fut le soir, ainsi fut le matin ; ce fut le second jour.
\TextTitle{Troisième jour : Les mers, la terre et la végétation}
\VS{9}Puis Dieu dit : Que les eaux qui sont au-dessous des cieux soient rassemblées en un lieu, et que le sec paraisse ; et il fut ainsi.
\VS{10}Et Dieu appela le sec terre ; et il appela l'amas des eaux mers ; et Dieu vit que cela était bon.
\VS{11}Puis Dieu dit : Que la terre produise de la verdure, de l'herbe portant de la semence, et des arbres fruitiers portant du fruit selon leur espèce, qui aient leur semence en eux-mêmes sur la terre ; et il fut ainsi.
\VS{12}La terre donc produisit de la verdure, de l'herbe portant de la semence selon son espèce ; et des arbres portant du fruit qui avaient leur semence en eux-mêmes selon leur espèce ; et Dieu vit que cela était bon.
\VS{13}Ainsi fut le soir, ainsi fut le matin ; ce fut le troisième jour.
\TextTitle{Quatrième jour : Les luminaires du ciel}
\VS{14}Puis Dieu dit : Qu'il y ait des luminaires dans l'étendue du ciel pour séparer la nuit d'avec le jour, et qui servent de signes pour les saisons, pour les jours, et pour les années ;
\VS{15}et qu’ils servent de luminaires dans l'étendue du ciel afin d'éclairer la terre ; et il fut ainsi.
\VS{16}Dieu donc fit deux grands luminaires, le plus grand luminaire pour présider au jour, et le plus petit luminaire pour présider à la nuit ; il fit aussi les étoiles.
\VS{17}Dieu les plaça dans l'étendue du ciel pour éclairer la terre,
\VS{18}pour présider au jour et à la nuit, et pour séparer la lumière d’avec les ténèbres ; et Dieu vit que cela était bon.
\VS{19}Ainsi fut le soir, ainsi fut le matin ; ce fut le quatrième jour.
\TextTitle{Cinquième jour : Les animaux vivant dans les eaux et les airs\FTNTT{ Ge. 2:19}}
\VS{20}Puis Dieu dit : Que les eaux produisent en toute abondance des reptiles vivants ; et qu'il y ait des oiseaux qui volent sur la terre vers l'étendue du ciel.
\VS{21}Dieu créa les grands poissons et tous les animaux vivants qui se meuvent et que les eaux produisirent en toute abondance selon leur espèce ; il créa aussi tout oiseau ayant des ailes selon son espèce ; et Dieu vit que cela était bon.
\VS{22}Dieu les bénit en disant : Soyez féconds, multipliez, et remplissez les eaux des mers ; et que les oiseaux multiplient sur la terre.
\VS{23}Ainsi fut le soir, ainsi fut le matin ; ce fut le cinquième jour.
\TextTitle{Sixième jour : Les animaux terrestres}
\VS{24}Puis Dieu dit : Que la terre produise des animaux selon leur espèce, le bétail, les reptiles, et les bêtes de la terre selon leur espèce ; et il fut ainsi.
\VS{25}Dieu donc fit les animaux de la terre selon leur espèce, et le bétail selon son espèce, et les reptiles de la terre selon leur espèce ; et Dieu vit que cela était bon.
\TextTitle{Mission confiée à l’homme ; son autorité sur la création}
\VS{26}Puis Dieu dit : Faisons l'homme à notre image, selon notre ressemblance\FTNT{L’image de Dieu n’est autre que Jésus-Christ lui-même (Col. 1:15). Adam, qui signifie terrien, a été créé à  l’image du dernier Adam (1 Co. 15:40-49) qui est venu comme Fils, afin de nous montrer le modèle de fils et de filles que Dieu souhaite (Ro. 8:29). Nous avons ici une autre image de l’incarnation de Dieu en la personne de Jésus-Christ. Ainsi, avant que l’homme ne pèche, le projet de la rédemption était déjà là (1 Pi. 1:19-21).}, et qu'il domine sur les poissons de la mer, sur les oiseaux du ciel, sur le bétail, sur toute la terre, et sur tout reptile qui rampe sur la terre.
\VS{27}Dieu créa l'homme à son image, il le créa à l'image de Dieu, il les créa mâle et femelle.
\TextTitle{Autorité de l'homme sur la création}
\VS{28}Dieu les bénit et leur dit : Soyez féconds, multipliez, remplissez la terre, et assujettissez-la ; et dominez sur les poissons de la mer, sur les oiseaux du ciel, et sur toute bête qui se meut sur la terre.
\VS{29}Et Dieu dit : Voici, je vous donne toute herbe portant de la semence qui est sur toute la terre, et tout arbre ayant en lui du fruit d'arbre et portant de la semence, ce sera votre nourriture.
\VS{30}Et à tout animal de la terre, à tout oiseau du ciel, et à tout ce qui se meut sur la terre, ayant en soi un souffle de vie, je donne toute herbe verte pour nourriture. Et cela fut ainsi.
\VS{31}Dieu vit tout ce qu'il avait fait, et voici cela était très bon. Ainsi fut le soir, ainsi fut le matin ; ce fut le sixième jour.
\Chap{2}
\TextTitle{Septième jour : Le sabbat}
\VerseOne{}Les cieux donc et la terre furent achevés, avec toute leur armée.
\VS{2}Dieu acheva au septième jour son œuvre qu'il avait faite, et il se reposa au septième jour de toute son œuvre qu'il avait faite.
\VS{3}Dieu bénit le septième jour, et le sanctifia, parce qu'en ce jour-là il s'était reposé de toute son œuvre qu'il avait créée en la faisant.
\VS{4}Telles sont les origines des cieux et de la terre, lorsqu'ils furent créés.
\VS{5}Lorsque Yahweh Dieu fit la terre et les cieux, aucun arbuste des champs n’était encore sur la terre, et aucune herbe des champs ne germait encore ; car Yahweh Dieu n'avait pas fait pleuvoir sur la terre, et il n'y avait point d'homme pour cultiver la terre.
\TextTitle{Yahweh forme l’homme et le place en Eden\FTNTT{Job 10:8-9 ; Ps. 119:73}}
\VS{6}Et il monta une vapeur de la terre qui arrosa toute la surface de la terre.
\VS{7}Yahweh Dieu forma l'homme de la poussière de la terre, et il souffla dans ses narines un souffle de vie ; et l'homme devint une âme vivante.
\VS{8}Aussi Yahweh Dieu planta un jardin en Eden, du côté de l’orient, et il y mit l'homme qu'il avait formé.
\TextTitle{Description du jardin en Eden\FTNTT{Ge. 1:28-3:6}}
\VS{9}Yahweh Dieu fit germer de la terre des arbres de toute espèce, agréables à voir et bons à manger, et l'arbre de la vie au milieu du jardin, et l'arbre de la connaissance du bien et du mal.
\VS{10}Un fleuve sortait d'Eden pour arroser le jardin ; et de là il se divisait en quatre bras.
\VS{11}Le nom du premier est Pischon ; c'est le fleuve qui coule en entourant tout le pays de Havila où se trouve l'or.
\VS{12}L'or de ce pays est bon ; c'est là aussi que se trouvent le bdellium et la pierre d'onyx.
\VS{13}Le nom du second fleuve est Guihon ; c'est celui qui coule en entourant tout le pays de Cusch.
\VS{14}Le nom du troisième fleuve est Hiddékel, qui coule vers l'Assyrie ; et le quatrième fleuve est l'Euphrate.
\TextTitle{Commandement donné par Yahweh à l’homme\FTNTT{Ge. 1:28}}
\VS{15}Yahweh Dieu prit donc l'homme et le mit dans le jardin d'Eden pour le cultiver et pour le garder.
\VS{16}Puis Yahweh Dieu donna cet ordre à l'homme, en disant : Tu mangeras, tu mangeras\FTNT{En hebreu, le mot «akal» signifie «~manger~», «~se nourrir~», «~goûter~», «~jouir~», «~dévorer~», «~consumer~», et il a été utilisé deux fois de suite dans ce passage.} de tout arbre du jardin.
\VS{17}Mais quant à l'arbre de la connaissance du bien et du mal, tu n'en mangeras point, car le jour où tu en mangeras, tu mourras, tu mourras\FTNT{Dans la plupart des versions ce passage est mal traduit par «~tu mourras certainement~», alors que le terme mort en hébreu «~muwth~» est utilisé deux fois dans ce passage, et les écritures nous parlent de la mort physique et de la seconde mort, qui est le lac de feu (Ap. 2:11 ; Ap. 20:6,14). La mort physique précède la seconde physique.}.
\TextTitle{Yahweh forme une femme pour l'homme\FTNTT{Ge. 1:27}}
\VS{18}Yahweh Dieu dit : Il n'est pas bon que l'homme soit seul ; je lui ferai une aide semblable à lui.
\VS{19}Car Yahweh Dieu forma de la terre tous les animaux des champs et tous les oiseaux du ciel, puis il les fit venir vers Adam pour voir comment il les nommerait, et afin que le nom qu'Adam donnerait à tout animal fût, son nom.
\VS{20}Et Adam donna des noms à tout le bétail, et aux oiseaux du ciel, et à tous les animaux des champs ; mais pour Adam, il ne trouva point d'aide semblable à lui.
\VS{21}Et Yahweh Dieu fit tomber un profond sommeil sur Adam, qui s'endormit ; et Dieu prit une de ses côtes, et referma la chair à la place de cette côte.
\VS{22}Yahweh Dieu forma une femme de la côte qu'il avait prise d'Adam, et il l’amena vers Adam\FTNT{1 Co. 11:8.}.
\TextTitle{Union d’Adam et Eve}
\VS{23}Alors Adam dit : Voici cette fois celle qui est os de mes os et chair de ma chair ; on l’appellera femme, parce qu'elle a été prise de l'homme.
\VS{24}C'est pourquoi l'homme quittera son père et sa mère et s’attachera à sa femme, et ils deviendront une seule chair\FTNT{Ep. 5:30-31 ; Mt. 19:5 ; Mc. 10:7 ; 1 Co. 6:16.}.
\VS{25}Adam et sa femme étaient tous deux nus, et ils n’en avaient pas honte.
\Chap{3}
\TextTitle{Séduction du serpent et chute de l’homme}
\VerseOne{}Or le serpent\FTNT{Satan ou le serpent ancien (Ap. 12:9 ; Ap. 20:2).} était le plus prudent\FTNT{La prudence, la ruse, la subtilité du serpent, sont marquées dans l'Ecriture comme des qualités qui le distinguent des autres animaux (Mt. 10:16).} de tous les animaux des champs que Yahweh Dieu avait faits ; et il dit à la femme : Quoi ! Dieu a dit : Vous ne mangerez pas de tous les arbres du jardin ?
\VS{2}La femme répondit au serpent : Nous mangeons du fruit des arbres du jardin ;
\VS{3}mais quant au fruit de l'arbre qui est au milieu du jardin, Dieu a dit : Vous n'en mangerez point, et vous ne le toucherez point, de peur que vous ne mouriez.
\VS{4}Alors le serpent dit à la femme : Vous ne mourrez nullement ;
\VS{5}mais Dieu sait que le jour où vous en mangerez, vos yeux seront ouverts, et vous serez comme des dieux, connaissant le bien et le mal.
\VS{6}La femme donc voyant que le fruit de l'arbre était bon à manger et agréable à la vue, et que cet arbre était désirable pour donner de la science ; elle prit de son fruit, et en mangea, et elle en donna aussi à son mari qui était auprès d’elle, et il en mangea.
\TextTitle{La connaissance du bien et du mal}
\VS{7}Les yeux de tous les deux s’ouvrirent, ils connurent qu'ils étaient nus, et ils cousirent ensemble des feuilles de figuier, et s'en firent des ceintures.
\VS{8}Alors ils entendirent au vent du jour la voix de Yahweh Dieu qui se promenait par le jardin ; et Adam et sa femme se cachèrent loin de la face de Yahweh Dieu, au milieu des arbres du jardin.
\VS{9}Mais Yahweh Dieu appela Adam et lui dit : Où es-tu ?
\VS{10}Il répondit : J'ai entendu ta voix dans le jardin, et j'ai eu peur parce que je suis nu, et je me suis caché.
\VS{11}Et Dieu dit : Qui t'a appris que tu es nu ? Est-ce que tu as mangé du fruit de l'arbre dont je t'avais défendu de manger ?
\VS{12}Adam répondit : La femme que tu m'as donnée pour être avec moi m'a donné du fruit de l'arbre, et j'en ai mangé.
\VS{13}Et Yahweh Dieu dit à la femme : Pourquoi as-tu fait cela ? Et la femme répondit : Le serpent m'a séduite, et j'en ai mangé.
\TextTitle{La création soumise à la vanité\FTNTT{Ro. 8:20-22}}
\VS{14}Alors Yahweh Dieu dit au serpent : Parce que tu as fait cela, tu seras maudit entre tout le bétail et entre tous les animaux des champs ; tu marcheras sur ton ventre, et tu mangeras la poussière\FTNT{La poussière dont il est question n’est autre que l’homme pécheur (Ge. 3:19). Satan ne peut rien contre les véritables enfants de Dieu (Mt. 16:18 ; Lu. 10:19).} tous les jours de ta vie.
\VS{15}Je mettrai inimitié entre toi et la femme\FTNT{La femme représente en premier lieu Eve, la mère de tous les hommes. Ici, elle représente aussi Israël, l’épouse de Yahweh selon Ge. 37:5-11 et Ap. 12:1.}, et entre ta postérité\FTNT{La postérité du serpent regroupe l’homme impie (2 Th. 2:3-4 ; 1 Jn. 2:18-22), et tous ceux qui n’ont pas reçu Jésus-Christ comme Seigneur et Sauveur. En effet, seuls ceux qui ont reçu Jésus dans leur vie sont appelés enfants de Dieu (Jn. 1:12 ; 1 Jn. 3:8-10 ; 1 Jn. 5:19).} et sa postérité\FTNT{La postérité de la femme regroupe Jésus-Christ homme (Es. 7:14 ; Lu. 2:4-7), et l’Église, le Corps de Christ (Col. 1:24).} ; celle-ci te brisera la tête, et tu lui blesseras le talon.
\VS{16}Et il dit à la femme : J'augmenterai beaucoup la souffrance de tes grossesses ; tu enfanteras dans la douleur tes enfants ; tes désirs se porteront vers ton mari, et il dominera sur toi.
\VS{17}Puis il dit à Adam : Parce que tu as obéi à la parole de ta femme, et que tu as mangé le fruit de l'arbre au sujet duquel je t'avais donné cet ordre, en disant : Tu n'en mangeras point, la terre sera maudite à cause de toi ; tu en mangeras les fruits dans la peine, tous les jours de ta vie.
\VS{18}Et elle te produira des épines et des chardons ; et tu mangeras l'herbe des champs.
\VS{19}C’est à la sueur de ton visage que tu mangeras du pain, jusqu'à ce que tu retournes dans la terre, d’où tu as été pris ; car tu es poussière, et tu retourneras dans la poussière.
\TextTitle{L’homme et la femme revêtu de tuniques de peaux}
\VS{20}Et Adam appela sa femme Eve, parce qu'elle a été la mère de tous les vivants.
\VS{21}Yahweh Dieu fit à Adam et à sa femme des tuniques de peaux, et il les en revêtit.
\VS{22}Yahweh Dieu dit : Voici, l'homme est devenu comme l'un de nous, connaissant le bien et le mal. Mais maintenant il faut prendre garde, qu’il n’avance sa main, et aussi qu’il ne prenne de l'arbre de vie, et qu’il n’en mange, et ne vive éternellement.
\TextTitle{L’homme chassé du jardin}
\VS{23}Et Yahweh Dieu le chassa du jardin d’Eden pour qu’il cultive la terre d’où il avait été pris.
\VS{24}C’est ainsi qu’il chassa l'homme, et il mit à l’orient du jardin d’Eden des chérubins qui tournent ça et là une épée flamboyante pour garder le chemin de l'arbre de vie.
\Chap{4}
\TextTitle{La jalousie de Caïn contre son frère Abel}
\VerseOne{}Adam connut Eve sa femme ; elle conçut, et enfanta Caïn ; et elle dit : J'ai acquis un homme de par Yahweh.
\VS{2}Elle enfanta encore Abel, son frère ; et Abel fut berger, et Caïn laboureur.
\VS{3}Or, au bout de quelque temps, Caïn offrit à Yahweh une offrande des fruits de la terre\FTNT{Caïn était du diable, il est l’archétype du religieux qui pense pouvoir être sauvé par les œuvres (Lu. 11:51 ; 1 Jn. 3:12). Son offrande fut rejetée car il avait apporté devant Dieu le fruit de la terre qui avait été maudite (Ge. 3:17).  Cela revenait à offrir à Dieu le péché, la malédiction.} ;
\VS{4}et Abel, de son côté, offrit des premiers-nés de son troupeau, et de leur graisse\FTNT{Abel était juste et pieux, aussi il sut instinctivement apporter une offrande agréable à Dieu (Mt. 23:35 ; Lu. 11:51 ; Hé. 11:4). En l’occurrence, son offrande préfigurait le sacrifice du Seigneur.}. Yahweh eut égard à Abel, et à son offrande.
\VS{5}Mais il n'eut point d'égard à Caïn ni à son offrande ; et Caïn fut fort irrité, et son visage fut abattu.
\TextTitle{Yahweh avertit Caïn}
\VS{6}Et Yahweh dit à Caïn : Pourquoi es-tu irrité, et pourquoi ton visage est-il abattu ?
\VS{7}Si tu agis bien, tu relèveras ton visage, et si tu agis mal, le péché est couché à la porte, et ses désirs se portent vers toi ;  mais toi, domine sur lui.
\TextTitle{Caïn tue son frère Abel\FTNTT{Ge. 4:23}}
\VS{8}Et Caïn parla avec Abel son frère, et comme ils étaient dans les champs, Caïn se jeta sur Abel, son frère, et le tua.
\VS{9}Yahweh dit à Caïn : Où est Abel ton frère ? Et il lui répondit : Je ne sais, suis-je le gardien de mon frère, moi ?
\VS{10}Et Dieu dit : Qu'as-tu fait ? La voix du sang de ton frère crie de la terre à moi.
\VS{11}Maintenant donc tu seras maudit de la terre, qui a ouvert sa bouche pour recevoir de ta main le sang de ton frère.
\VS{12}Quand tu cultiveras la terre, elle ne te donnera plus son fruit, et tu seras vagabond et fugitif sur la terre.
\VS{13}Caïn dit à Yahweh : Mon châtiment est trop grand pour être supporté.
\VS{14}Voici, tu me chasses aujourd'hui de cette terre ; je serai caché loin de ta face, je serai vagabond et fugitif sur la terre, et quiconque me trouvera me tuera.
\VS{15}Yahweh lui dit : Si quelqu’un tuait Caïn, Caïn serait vengé sept fois. Ainsi Yahweh mit une marque sur Caïn afin que quiconque le trouverait ne le tue point.
\TextTitle{Caïn bâtit une cité loin de Yahweh}
\VS{16}Alors, Caïn s’éloigna de la face de Yahweh, et habita dans la terre de Nod, à l'orient d’Eden.
\VS{17}Puis Caïn connut sa femme ; elle conçut et enfanta Hénoc. Il bâtit une ville, et il donna à cette ville le nom de son fils Hénoc.
\VS{18}Hénoc engendra Irad, Irad engendra Mehujaël, Mehujaël engendra Metuschaël, et Métuschaël engendra Lémec.
\VS{19}Lémec prit deux femmes ; le nom de l'une était Ada, et le nom de l'autre Tsilla.
\VS{20}Ada enfanta Jabal : Il fut le père de ceux qui habitent dans les tentes et près des troupeaux.
\VS{21}Le nom de son frère était Jubal : Il fut le père de tous ceux qui jouent de la harpe et du chalumeau.
\VS{22}Tsilla aussi enfanta Tubal-Caïn, qui forgeait toutes sortes d'instruments d'airain et de fer. La soeur de Tubal-Caïn était Naama.
\VS{23}Lémec dit à Ada et à Tsilla ses femmes : Ecoutez ma voix femmes de Lémec, écoutez ma parole ! J’ai tué un homme pour ma blessure et un jeune homme pour ma meurtrissure.
\VS{24}Car si Caïn est vengé sept fois, Lémec le sera soixante-dix-sept fois.
\TextTitle{Naissance de Seth}
\VS{25}Adam connut encore sa femme ; elle enfanta un fils, et il l’appela du nom de Seth, car, dit-il, Dieu m'a donné un autre fils à la place d'Abel, que Caïn a tué.
\VS{26}Il naquit aussi un fils à Seth, et il l'appela du nom d’Enosch. C’est alors que l’on commença à proclamer le nom de Yahweh.
\Chap{5}
\TextTitle{La postérité d'Adam soumise à la mort\FTNTT{Ro. 5:12}}
\VerseOne{}Voici le livre de la postérité d'Adam, depuis le jour où Dieu créa l'homme, il le fit à la ressemblance de Dieu.
\VS{2}Il les créa mâle et femelle, et les bénit, et il leur donna le nom d'homme, le jour où ils furent créés.
\VS{3}Adam vécut cent trente ans, et engendra un fils à sa ressemblance, selon son image\FTNT{Désormais les hommes naissent à la ressemblance d’Adam, c’est-à-dire pécheurs (Ro. 3:23 ; Ro. 5:14-17).}, et il lui donna le nom de Seth.
\VS{4}Les jours d'Adam, après qu'il eut engendré Seth, furent de huit cents ans, et il engendra des fils et des filles.
\VS{5}Tous les jours qu'Adam vécut furent de neuf cent trente ans ; puis il mourut.
\TextTitle{De Seth aux fils de Noé\FTNTT{Ro. 5:12}}
\VS{6}Seth aussi vécut cent cinq ans, et engendra Enosch.
\VS{7}Seth, après qu'il eut engendré Enosch, vécut huit cent sept ans ; et il engendra des fils et des filles.
\VS{8}Tous les jours que Seth vécut furent de neuf cent douze ans ; puis il mourut.
\VS{9}Enosch, ayant vécu quatre-vingt-dix ans, engendra Kénan.
\VS{10}Enosch, après qu'il eut engendré Kénan, vécut huit cent quinze ans, et il engendra des fils et des filles.
\VS{11}Tous les jours qu'Enosch vécut furent de neuf cent cinq ans ; puis il mourut.
\VS{12}Kénan, ayant vécu soixante-dix ans, engendra Mahalaleel.
\VS{13}Kénan, après qu'il eut engendré Mahalaleel, vécut huit cent quarante ans ; et il engendra des fils et des filles.
\VS{14}Tous les jours que Kénan vécut furent de neuf cent dix ans ; puis il mourut.
\VS{15}Mahalaleel vécut soixante-cinq ans ; et il engendra Jéred.
\VS{16}Et Mahalaleel, après qu'il eut engendré Jéred, vécut huit cent trente ans, et il engendra des fils et des filles.
\VS{17}Tous les jours donc que Mahalaleel vécut furent de huit cent quatre-vingt-quinze ans ; puis il mourut.
\VS{18}Jéred, ayant vécu cent soixante-deux ans, engendra Hénoc.
\VS{19}Jéred, après avoir engendré Hénoc, vécut huit cents ans, et il engendra des fils et des filles.
\VS{20}Tous les jours que Jéred vécut furent de neuf cent soixante-deux ans ; puis il mourut.
\VS{21}Hénoc vécut soixante-cinq ans, et engendra Metuschélah.
\VS{22}Hénoc, après qu'il eut engendré Metuschélah, marcha avec Dieu trois cents ans ; et il engendra des fils et des filles.
\VS{23}Tous les jours qu'Hénoc vécut furent de trois cent soixante-cinq ans.
\VS{24}Hénoc marcha avec Dieu ; mais il ne parut plus parce que Dieu le prit.
\VS{25}Metuschélah, ayant vécu cent quatre-vingt-sept ans, engendra Lémec.
\VS{26}Metuschélah, après qu'il eut engendré Lémec, vécut sept cent quatre-vingt-deux ans ; et il engendra des fils et des filles.
\VS{27}Tous les jours que Metuschélah vécut furent de neuf cent soixante-neuf ans ; puis il mourut.
\VS{28}Lémec aussi vécut cent quatre-vingt-deux ans, et il engendra un fils.
\VS{29}Il l'appela Noé, en disant : Celui-ci nous consolera de notre oeuvre, et du travail pénible de nos mains, sur la terre que Yahweh a maudite.
\VS{30}Lémec, après qu'il eut engendré Noé, vécut cinq cent quatre-vingt-quinze ans ; et il engendra des fils et des filles.
\VS{31}Tous les jours que Lémec vécut furent de sept cent soixante-dix-sept ans ; puis il mourut.
\VS{32}Noé, âgé de cinq cents ans, engendra Sem, Cham, et Japhet.
\Chap{6}
\TextTitle{Le mal dans le cœur de l'homme\FTNTT{Ro. 5:12}}
\VerseOne{}Lorsque les hommes eurent commencé à se multiplier sur la face de la terre, et qu'ils eurent engendré des filles,
\VS{2}les fils de Dieu\FTNT{Ici, les fils de Dieu sont des anges qui ont quitté leur demeure (Jud. 1:5-7).} virent que les filles des hommes étaient belles, et ils en prirent pour femmes parmi toutes celles qu'ils choisirent.
\TextTitle{Yahweh ne conteste plus avec les hommes}
\VS{3}Yahweh dit : Mon Esprit ne contestera point à toujours avec les hommes\FTNT{C’est le Saint-Esprit qui nous convainc de péché, de jugement et de justice (Jn. 16:8). Lorsqu’il constate que le cœur d’une personne est définitivement endurci au point de refuser la repentance, il renonce à la convaincre de péché et il se retire. La génération antédiluvienne avait définitivement rejeté Dieu en choisissant de faire du mal son idole (Ge. 6:5). Elle était allée si loin dans l’abomination au point de s’accoupler avec des anges déchus (Ge. 6:4), ce qui laisse supposer un culte volontaire aux démons. Lorsque le Saint-Esprit est retiré d’une personne, il est remplacé par l’esprit d’égarement qui enferme le pécheur dans l’erreur et l’entraîne ainsi à sa condamnation éternelle (Mt. 12:31 ; 2 Th. 2:11 )}, car les hommes ne sont que chair, et leurs jours seront de cent vingt ans.
\TextTitle{Le monde avant le déluge\FTNTT{Lu. 17:27}}
\VS{4}Les géants étaient sur la terre en ce temps-là. Il en fut de même après que les fils de Dieu furent venus vers les filles des hommes, et qu’elles leur eurent donné des enfants. Ce sont ces hommes vaillants qui furent des gens de renom dans l’antiquité.
\TextTitle{Yahweh prépare un jugement}
\VS{5}Yahweh vit que la méchanceté des hommes était très grande sur la terre, et que toute l'imagination des pensées de leur cœur n'était que mal en tout temps.
\VS{6}Yahweh se repentit d'avoir fait l'homme sur la terre, et il fut affligé en son cœur.
\VS{7}Et Yahweh dit : J'exterminerai de la face de la terre les hommes que j'ai créés, depuis les hommes jusqu'au bétail, jusqu'aux reptiles, et même jusqu'aux oiseaux du ciel ; car je me repens de les avoir faits.
\TextTitle{La grâce de Yahweh sur Noé : Construction de l'arche}
\VS{8}Mais Noé trouva grâce aux yeux de Yahweh.
\VS{9}Voici la postérité de Noé. Noé était un homme juste et intègre en son temps ; Noé marchait avec Dieu.
\VS{10}Noé engendra trois fils : Sem, Cham, et Japhet.
\VS{11}Et la terre était corrompue devant Dieu, et remplie de violence.
\VS{12}Dieu donc regarda la terre, et voici elle était corrompue ; car toute chair avait corrompu sa voie sur la terre.
\VS{13}Et Dieu dit à Noé : La fin de toute chair est venue devant moi ; car ils ont rempli la terre de violence, et voici, je les détruirai avec la terre.
\VS{14}Fais-toi une arche\FTNT{L’arche  est un type de Christ et du salut en lui et par lui.  On peut voir plusieurs aspects de Jésus-Christ et de la rédemption dans la structure de l’arche :
- L’arche a été une révélation de Jésus-Christ donnée à Noé.  C’est en Jésus-Christ que nous avons le salut et la protection (Col. 1:12-13 ; Col. 3:3). 
-L’arche était faite de bois de gopher, probablement du cèdre.  Ce bois est un bois qui ne pourrit pas en condition normale. Ce bois préfigurait l’incorruptibilité de Jésus-Christ homme (Es. 53:9 ;  Hé. 4:15 ; 1 Pi. 2:22). 
-Au verset 14, on lit que Dieu demande à Noé d’enduire l’arche en dedans et en dehors avec de la  poix, c’est-à-dire du bitume.  Le mot «~poix~» vient de l’hébreu «~kaphar~», qui signifie «~expiation~».  Ce mot est traduit près de 70 fois dans le Tanakh par expiation.  Il est également traduit par «~réconciliation~», «~pardon~», «~miséricordieux~» et «~apaiser~».  L’allusion à l’expiation des péchés faite par Jésus-Christ est claire. Par son sacrifice, nous sommes rendus parfaits à jamais (Hé. 10:14-15).} de bois de gopher ; tu feras cette arche en cellules, et tu l’enduiras de poix en dedans et en dehors.
\VS{15}Et voici comment tu la feras : La longueur de l'arche sera de trois cents coudées ; sa largeur de cinquante coudées, et sa hauteur de trente coudées.
\VS{16}Tu feras une fenêtre à l'arche, et feras son comble d'une coudée de hauteur, et tu mettras la porte de l'arche à son côté, et tu la feras avec un bas, un second, et un troisième étage.
\VS{17}Et voici, je ferai venir un déluge d'eau sur la terre, pour détruire toute chair dans laquelle il y a souffle de vie sous les cieux ; et tout ce qui est sur la terre expirera.
\VS{18}Mais j'établirai mon alliance avec toi ; et tu entreras dans l'arche toi et tes fils, et ta femme, et les femmes de tes fils avec toi.
\VS{19}Et de tout ce qui a vie d'entre toute chair, tu en feras entrer deux de chaque espèce dans l'arche, pour les conserver en vie avec toi, à savoir le mâle et la femelle.
\VS{20}Des oiseaux, selon leur espèce, des bêtes à quatre pattes, selon leur espèce, et de tous les reptiles, selon leur espèce. Ils y entreront tous par paires avec toi, afin que tu les conserves en vie.
\VS{21}Prends aussi avec toi de tous les aliments que l’on mange, et rassemble-les auprès de toi, afin qu'ils servent pour ta nourriture et pour celle des animaux.
\VS{22}Et Noé fit selon tout ce que Dieu lui avait ordonné ; il le fit ainsi.
\Chap{7}
\TextTitle{Le jugement par le déluge}
\VerseOne{}Yahweh dit à Noé : Entre dans l’arche, toi et toute ta maison ; car je t'ai vu juste devant moi parmi cette génération. 
\VS{2} Tu prendras de toutes les bêtes pures sept de chaque espèce, le mâle et sa femelle ; mais des bêtes qui ne sont point pures, un couple, le mâle et la femelle.
\VS{3}Tu prendras aussi des oiseaux du ciel sept de chaque espèce, le mâle et sa femelle ; afin d'en conserver la race sur toute la terre.
\VS{4}Car dans sept jours, je ferai pleuvoir sur la terre pendant quarante jours et quarante nuits ; et j'exterminerai de la surface de la terre tous les êtres qui subsistent que j'ai faits.
\VS{5}Noé fit selon tout ce que Yahweh lui avait ordonné.
\VS{6}Noé était âgé de six cents ans quand le déluge des eaux vint sur la terre.
\VS{7}Noé donc entra dans l’arche avec ses fils, sa femme, et les femmes de ses fils, pour échapper aux eaux du déluge.
\VS{8}Des bêtes pures, des bêtes qui ne sont point pures, des oiseaux, et tout ce qui se meut sur la terre.
\VS{9}Elles entrèrent deux à deux vers Noé dans l'arche, le mâle et la femelle, comme Dieu l’avait ordonné à Noé.
\VS{10}Sept jours après, les eaux du déluge furent sur la terre.
\VS{11}En l'an six cent de la vie de Noé, au second mois, le dix-septième jour du mois, en ce jour-là toutes les sources du grand abîme furent rompues, et les écluses des cieux furent ouvertes.
\VS{12}La pluie tomba sur la terre pendant quarante jours et quarante nuits.
\VS{13}Ce même jour entrèrent dans l’arche Noé, Sem, Cham, et Japhet, fils de Noé, avec la femme de Noé, et les trois femmes de ses fils avec eux.
\VS{14}Eux, et tous les animaux selon leur espèce, et tout le bétail selon son espèce, et tous les reptiles qui se meuvent sur la terre selon leur espèce, et tous les oiseaux selon leur espèce ; et tout petit oiseau ayant des ailes, de quelque sorte que ce soit.
\VS{15}Ils entrèrent dans l'arche auprès de Noé, deux à deux, de toute chair ayant souffle de vie.
\VS{16}Il en entra mâle et femelle de toute chair comme Dieu l’avait ordonné à Noé, puis Yahweh ferma l'arche sur lui.
\VS{17}Le déluge fut pendant quarante jours sur la terre ; et les eaux crurent et élevèrent l'arche, et elle fut élevée au-dessus de la terre.
\VS{18}Les eaux  grossirent et s'accrurent beaucoup sur la terre, et l'arche flottait au-dessus des eaux.
\VS{19}Les eaux grossirent de plus en plus sur la terre, et toutes les hautes montagnes qui sont sous le ciel entier en furent couvertes.
\VS{20}Les eaux s’élevèrent de quinze coudées au-dessus des montagnes  qui furent couvertes.
\VS{21}Toute chair qui se mouvait sur la terre périt, tant les oiseaux que le bétail et les animaux, tous les reptiles qui rampaient sur la terre, et tous les hommes.
\VS{22}Tout ce qui avait respiration, souffle de vie dans ses narines, et qui était sur la terre sèche mourut.
\VS{23}Tous les êtres qui étaient sur la face de la terre furent donc exterminés, depuis les hommes jusqu’au bétail, aux reptiles et aux oiseaux du ciel ; ils furent exterminés de la face de la terre ; il ne resta seulement que Noé, et ce qui était avec lui dans l'arche.
\VS{24}Les eaux furent grosses sur la terre pendant cent cinquante jours.
\Chap{8}
\TextTitle{Fin du déluge}
\VerseOne{}Dieu se souvint de Noé, de tous les animaux et de tout le bétail qui étaient avec lui dans l'arche ; et Dieu fit passer un vent sur la terre, et les eaux s’apaisèrent.
\VS{2}Les sources de l'abîme et les écluses des cieux furent fermées et la pluie ne tomba plus du ciel.
\VS{3}Au bout de cent cinquante jours, les eaux se retirèrent sans interruption de dessus la terre, et diminuèrent.
\VS{4}Le dix-septième jour du septième mois, l'arche s'arrêta sur les montagnes d'Ararat.
\VS{5}Les eaux allèrent en diminuant de plus en plus jusqu'au dixième mois ; et au premier jour du dixième mois, les sommets des montagnes apparurent.
\VS{6}Au bout de quarante jours, Noé ouvrit la fenêtre qu’il avait faite à l'arche.
\VS{7}Il lâcha le corbeau, qui sortit, allant et revenant, jusqu'à ce que les eaux aient séché sur la terre.
\VS{8}Il lâcha aussi une colombe pour voir si les eaux avaient diminué à la surface de la terre.
\VS{9}Mais la colombe ne trouvant aucun lieu pour poser la plante de son pied, retourna à lui dans l'arche, car les eaux étaient sur toute la terre ; et Noé avançant sa main la reprit et la fit entrer dans l'arche.
\VS{10}Il attendit encore sept autres jours, il lâcha de nouveau la colombe hors de l'arche.
\VS{11}Sur le soir, la colombe revint à lui ; et voici, elle avait dans son bec une feuille d'olivier qu'elle avait arrachée ; et Noé connut que les eaux avaient diminué sur la terre.
\VS{12}Il attendit encore sept autres jours, puis il lâcha la colombe qui ne retourna plus à lui.
\VS{13}L’an six cent un de l'âge de Noé, le premier jour du premier mois, les eaux avaient diminué sur la terre. Noé ôta la couverture de l'arche, regarda, et voici, la surface de la terre avait séché.
\VS{14}Le vingt-septième jour du second mois la terre fut sèche.
\TextTitle{Noé sort de l'arche: Le règne des hommes\FTNTT{Ge. 8:11-15:32}}
\VS{15}Puis Dieu parla à Noé, en disant :
\VS{16}Sors de l'arche, toi et ta femme, tes fils, et les femmes de tes fils avec toi.
\VS{17}Fais sortir avec toi tous les animaux qui sont avec toi, de toute chair, tant les oiseaux que le bétail, et tous les reptiles qui rampent sur la terre ; qu'ils se répandent sur la terre, et qu'ils soient féconds et multiplient sur la terre.
\VS{18}Noé donc sortit, et avec lui ses fils, sa femme, et les femmes de ses fils.
\VS{19}Tous les animaux, tous les reptiles, tous les oiseaux, tout ce qui se meut sur la terre, selon leurs espèces, sortirent de l'arche.
\VS{20}Noé bâtit un autel à Yahweh, il prit de toutes les bêtes pures, et de tout oiseau pur, et il offrit des holocaustes sur l'autel.
\VS{21}Yahweh respira une odeur agréable, et dit en son cœur : Je ne maudirai plus la terre à cause des hommes, quoique les dispositions du coeur des hommes soient mauvaises dès leur jeunesse ; et je ne frapperai plus tout ce qui est vivant, comme je l’ai fait.
\VS{22}Tant que la terre subsistera, les semailles et les moissons, le froid et la chaleur, l'été et l'hiver, le jour et la nuit ne cesseront point.
\Chap{9}
\TextTitle{Yahweh établit une alliance avec Noé\FTNTT{Ge. 9:16}}
\VerseOne{}Dieu bénit Noé et ses fils, et leur dit : Soyez féconds, multipliez, et remplissez la terre.
\VS{2}Vous serez un sujet de crainte et d’effroi pour tout animal de la terre, pour tout oiseau du ciel, pour tout ce qui se meut sur la terre, et pour tous les poissons de la mer : Ils sont livrés entre vos mains.
\VS{3}Tout ce qui se meut et qui a vie sera votre nourriture ; je vous donne tout cela comme l'herbe verte.
\VS{4}Seulement, vous ne mangerez point de chair avec son âme, c'est-à-dire, son sang.
\VS{5}Sachez-le aussi, je redemanderai votre sang, le sang de vos âmes, je le redemanderai à tout animal ; et je redemanderai l’âme de l’homme de la main de l’homme, de la main de son frère.
\VS{6}Celui qui aura versé le sang de l'homme, par l'homme son sang sera versé ; car Dieu a fait l'homme à son image.
\VS{7}Vous donc, soyez féconds et multipliez, répandez-vous sur la terre et multipliez sur elle.
\VS{8}Dieu parla aussi à Noé et à ses fils qui étaient avec lui, en disant :
\VS{9}Et quant à moi, voici, j'établis mon alliance avec vous, et avec votre postérité après vous ;
\VS{10}avec tous les êtres vivants qui sont avec vous, tant les oiseaux que le bétail, et tous les animaux de la terre qui sont avec vous, tous ceux qui sont sortis de l'arche jusqu'à tous les animaux de la terre.
\VS{11}J'établis donc mon alliance avec vous ; aucune chair ne sera plus exterminée par les eaux du déluge, et il n'y aura plus de déluge pour détruire la terre.
\VS{12}Puis Dieu dit : C'est ici le signe de l'alliance que j’établis entre moi et vous, et tous les êtres vivants qui sont avec vous, pour les générations à toujours :
\VS{13}J’ai placé mon arc dans la nuée, et il servira de signe de l'alliance entre moi et la terre.
\VS{14}Quand j’aurai rassemblé des nuages au-dessus de la terre, l’arc paraîtra dans la nuée ;
\VS{15}et je me souviendrai de mon alliance entre moi et vous, et tous les êtres vivants de toute chair, et les eaux ne deviendront plus un déluge pour détruire toute chair.
\VS{16}L'arc donc sera dans la nuée, et je le regarderai, et je me souviendrai de l'alliance perpétuelle entre Dieu et tous les êtres vivants  de toute chair qui est sur la terre.
\VS{17}Dieu donc dit à Noé : C'est là le signe de l'alliance que j'ai établie entre moi et toute chair qui est sur la terre.
\VS{18}Les fils de Noé qui sortirent de l'arche étaient Sem, Cham, et Japhet. Cham fut père de Canaan.
\VS{19}Ce sont là les trois fils de Noé, et c’est leur postérité qui peupla toute la terre.
\TextTitle{Le péché de Noé}
\VS{20}Or, Noé commença à cultiver la terre, et planta de la vigne.
\VS{21}Il but du vin, s'enivra, et se découvrit au milieu de sa tente.
\VS{22}Cham, père de Canaan, vit la nudité de son père\FTNT{Lé. 18:6-19 ; Lé. 20:11-21.}, et il le rapporta dehors à ses deux frères.
\VS{23}Alors Sem et Japhet prirent un manteau qu'ils mirent sur leurs deux épaules, et marchant à reculons, ils couvrirent la nudité de leur père ; et leurs visages étaient tournés en arrière, de sorte qu'ils ne virent point la nudité de leur père.
\VS{24}Et quand Noé se réveilla de son vin, il apprit ce que lui avait fait son fils cadet.
\TextTitle{Noé prononce une malédiction contre Canaan}
\VS{25}C'est pourquoi il dit : Maudit soit Canaan\FTNT{Une idée erronée selon laquelle les noirs auraient été maudits par Dieu au travers de la malédiction de Canaan s’est répandue pendant des siècles. On a ainsi légitimé la domination des peuples africains par les puissances occidentales blanches, et par la même occasion l’esclavage. Il faut préciser que les descendants de Cham furent Cush (Ethiopie), Mitsraïm (Egypte), Puth (les Celtes) et Canaan (Palestine, pays que Dieu a donné aux descendants de Sem, selon Ge. 15). Cham est le fils cadet, c’est-à-dire le coupable aux yeux de Noé. Mais c’est à Canaan (Palestine), le fils de Cham, donc petit-fils de Noé, que s’adresse la malédiction. Selon la Bible, les peuples africains sont des descendants de Cham, mais par son fils Cush et non par Canaan. La prétendue malédiction des noirs n’a donc aucun fondement.} ; il sera serviteur des serviteurs de ses frères.
\VS{26}Il dit aussi : Béni soit Yahweh, Dieu de Sem ; et que Canaan soit leur serviteur.
\VS{27}Que Dieu étende en douceur Japhet, et qu’il habite dans les tentes de Sem ; et que Canaan soit leur serviteur.
\VS{28}Noé vécut après le déluge trois cent cinquante ans.
\VS{29}Tout le temps donc que Noé vécut fut de neuf cent cinquante ans ; puis il mourut.
\Chap{10}
\TextTitle{La postérité de Noé}
\VerseOne{}Voici la postérité des enfants de Noé, Sem, Cham et Japhet ; il leur naquit des fils après le déluge.
\VS{2}Les fils de Japhet furent : Gomer, Magog, Madaï, Javan, Tubal, Mésech, et Tiras.
\VS{3}Les fils de Gomer : Aschkenaz, Riphat, et Togarma.
\VS{4}Les fils de Javan : Elischa, Tarsis, Kittim, et Dodanim.
\VS{5}C’est par eux qu’ont été peuplées les îles des nations selon leurs terres, chacun selon sa langue, selon leurs familles, entre leurs nations.
\VS{6}Les fils de Cham furent : Cusch, Mitsraïm, Puth, et Canaan.
\VS{7}Les fils de Cusch : Saba, Havila, Sabta, Raema, et Sabteca. Les fils de Raema : Séba et Dedan.
\VS{8}Cusch engendra aussi Nimrod\FTNT{Nimrod ou Nemrod, dont le nom signifie «~rebelle~», fut le premier roi de l’histoire biblique. Fils de Cusch (Ethiopie), lui-même premier-né de Cham, fils de Noé (Ge. 10:8-10), il fut à la tête du premier empire après le déluge. Il se distingua en qualité de puissant chasseur  «~devant Yahweh~» ou «~contre Yahweh~». Le contexte du chapitre 10 laisse entendre que Nimrod était un puissant chasseur qui provoquait Dieu. Fondateur de Ninive, il est surtout connu pour avoir été à l’origine du projet de la tour de Babel.}, c’est lui qui commença à être puissant sur la terre.
\VS{9}Il fut un puissant chasseur devant Yahweh, c'est pourquoi l'on a dit : Comme Nimrod, le puissant chasseur devant Yahweh.
\VS{10}Il régna d’abord sur Babel\FTNT{Le nom Babel signifie confusion par le mélange.}, Erec, Accad, et Calné au pays de Schinear.
\VS{11}De ce pays-là sortit Assur, et il bâtit Ninive et les rues de la ville, Rehoboth-Hir et Calach,
\VS{12}et Résen, entre Ninive et Calach, qui est une grande ville.
\VS{13}Mitsraïm engendra les Ludim, les Anamim, les Lehabim, les Naphtuhim,
\VS{14}les Patrusim, les Casluhim, d’où sont sortis les Philistins, et les Caphtorim.
\VS{15}Canaan engendra Sidon, son premier-né, et Heth ;
\VS{16}et les Jébusiens, les Amoréens, les Guirgasiens,
\VS{17}les Héviens, les Arkiens, les Siniens,
\VS{18}les Arvadiens, les Tsemariens, les Hamathiens. Ensuite, les familles des Cananéens se sont dispersées.
\VS{19}Les limites des Cananéens furent depuis Sidon, quand on vient vers Guérar, jusqu'à Gaza, en allant vers Sodome et Gomorrhe, Adma, et Tseboïm, jusqu'à Léscha.
\VS{20}Ce sont là les fils de Cham selon leurs familles et leurs langues, selon leurs pays, et selon leurs nations.
\VS{21}Il naquit aussi des fils à Sem, père de tous les fils d'Héber, et frère aîné de Japhet.
\VS{22}Les fils de Sem furent : Elam, Assur, Arpacschad, Lud et Aram.
\VS{23}Les fils d'Aram : Uts, Hul, Guéter et Masch.
\VS{24}Arpacschad engendra Schélach ; et Schélach engendra Héber.
\VS{25}Il naquit à Héber deux fils : Le nom de l'un était Péleg, parce que de son temps la terre fut partagée ; et le nom de son frère était Jokthan.
\VS{26}Jokthan engendra Almodad, Schéleph, Hatsarmaveth, Jérach,
\VS{27}Hadoram, Uzal, Dikla,
\VS{28}Obal, Abimaël, Séba,
\VS{29}Ophir, Havila, et Jobab. Tous ceux-là sont les enfants de Jokthan.
\VS{30}Ils habitèrent depuis Méscha, du côté de Sephar jusqu’à la montagne de l’orient.
\VS{31}Ce sont là les fils de Sem, selon leurs familles, selon leurs langues, selon leurs pays, et selon leurs nations.
\VS{32}Telles sont les familles des fils de Noé, selon leurs lignées, selon leurs nations. Et c’est d’eux que sont sorties les nations qui se sont répandues sur la terre après le déluge.
\Chap{11}
\TextTitle{Un projet humain : La tour de Babel}
\VerseOne{}Alors toute la terre avait un même langage et les mêmes paroles.
\VS{2}Mais il arriva qu'étant partis d'orient, ils trouvèrent une vallée au pays de Schinear où ils habitèrent.
\VS{3}Et ils se dirent l'un à l'autre : Allons ! Faisons des briques, et cuisons-les très bien au feu. Et la brique leur servit de pierre, et le bitume leur servit d’argile.
\VS{4}Puis ils dirent : Allons ! Bâtissons-nous une ville, et une tour dont le sommet soit jusqu’aux cieux ; et faisons-nous un nom, de peur que nous ne soyons dispersés sur toute la terre.
\VS{5}Alors Yahweh descendit pour voir la ville et la tour que les fils des hommes bâtissaient.
\VS{6}Et Yahweh dit : Voici, ce n'est qu'un seul et même peuple, ils ont un même langage, et ils commencent à travailler ; et maintenant rien ne les empêchera d'exécuter ce qu'ils ont projeté.
\TextTitle{Yahweh confond le langage humain}
\VS{7}Allons ! Descendons, et là confondons leur langage afin qu'ils n'entendent point le langage les uns des autres.
\VS{8}Ainsi Yahweh les dispersa de là par toute la terre, et ils cessèrent de bâtir la ville.
\VS{9}C'est pourquoi on l’appela du nom de Babel, car c’est là que Yahweh confondit le langage de toute la terre, et c’est de là que Yahweh les dispersa sur toute la terre.
\TextTitle{La postérité de Sem, ancêtre d’Abram}
\VS{10}Voici la postérité de Sem : Sem, âgé de cent ans, engendra Arpacschad, deux ans après le déluge.
\VS{11}Sem, après qu'il eut engendré Arpacschad, vécut cinq cents ans, et engendra des fils et des filles.
\VS{12}Arpacschad vécut trente-cinq ans, et engendra Schélach.
\VS{13}Arpacschad, après qu'il eut engendré Schélach, vécut quatre cent trois ans, et engendra des fils et des filles.
\VS{14}Schélach, ayant vécu trente ans, engendra Héber.
\VS{15}Schélach, après qu'il eut engendré Héber, vécut quatre cent trois ans, et engendra des fils et des filles.
\VS{16}Héber, ayant vécu trente-quatre ans, engendra Péleg.
\VS{17}Héber, après qu'il eut engendré Péleg, vécut quatre cent trente ans, et engendra des fils et des filles.
\VS{18}Péleg, ayant vécu trente ans, engendra Rehu.
\VS{19}Péleg, après qu'il eut engendré Rehu, vécut deux cent neuf ans, et engendra des fils et des filles.
\VS{20}Rehu, ayant vécu trente-deux ans, engendra Serug.
\VS{21}Rehu, après qu'il eut engendré Serug, vécut deux cent sept ans, et engendra des fils et des filles.
\VS{22}Serug, ayant vécu trente ans, engendra Nachor.
\VS{23}Serug, après qu'il eut engendré Nachor, vécut deux cents ans, et engendra des fils et des filles.
\VS{24}Nachor, ayant vécu vingt-neuf ans, engendra Térach.
\VS{25}Nachor, après qu'il eut engendré Térach, vécut cent dix-neuf ans, et engendra des fils et des filles.
\VS{26}Térach, ayant vécu soixante-dix ans, engendra Abram, Nachor, et Haran.
\VS{27}Voici la postérité de Térach : Térach engendra Abram, Nachor, et Haran ; et Haran engendra Lot.
\VS{28}Et Haran mourut en présence de son père, au pays de sa naissance, à Ur en Chaldée.
\VS{29}Abram et Nachor prirent chacun une femme. Le nom de la femme d'Abram était Saraï ; et le nom de la femme de Nachor était Milca, fille de Haran, père de Milca et de Jisca.
\VS{30}Saraï était stérile, et n'avait point d'enfants.
\TextTitle{Séjour à Charan}
\VS{31}Térach prit son fils Abram, et Lot fils de son fils, qui était fils de Haran, et Saraï, sa belle-fille, femme d'Abram, son fils, et ils sortirent ensemble d'Ur en Chaldée pour aller au pays de Canaan, et ils vinrent jusqu'à Charan, et ils y habitèrent.
\VS{32}Les jours de Térach furent de deux cent cinq ans ; puis il mourut à Charan.
\Chap{12}
\TextTitle{Appel d'Abram : La promesse de Yahweh\FTNTT{Ge. 12:2 ; 13:14-18 ; 15:1-21 ; 17:4-8 ; 22:15-24 ; 26:1-5 ; 28:10-15}}
\VerseOne{}Yahweh dit à Abram : Va pour toi, hors de ta terre, de ta patrie, et de la maison de ton père, vers la terre que je te montrerai\FTNT{Ac. 7:3 ; Hé. 11:8.}.
\VS{2}Je te ferai devenir une grande nation, et je te bénirai, je rendrai ton nom grand, et tu seras béni.
\VS{3}Je bénirai ceux qui te béniront, et je maudirai ceux qui te maudiront ; et toutes les familles de la terre seront bénies en toi\FTNT{Ac. 3:25 ; Ga. 3:8.}.
\TextTitle{Abram sur la terre de Canaan}
\VS{4}Abram donc partit, comme Yahweh le lui avait dit, et Lot alla avec lui. Abram était âgé de soixante-quinze ans quand il sortit de Charan.
\VS{5}Abram prit aussi Saraï, sa femme, et Lot, fils de son frère, avec tous les biens qu'ils avaient acquis, et les personnes qu'ils avaient eues à Charan ; et ils partirent pour aller dans le pays de Canaan, et ils arrivèrent au pays de Canaan\FTNT{Ac. 7:4.}.
\VS{6}Abram parcourut le pays jusqu'au lieu nommé Sichem, et jusqu'aux chênes de Moré ; et les Cananéens étaient alors dans le pays.
\VS{7}Yahweh apparut à Abram, et lui dit : Je donnerai ce pays à ta postérité. Et Abram bâtit là un autel à Yahweh qui lui était apparu.
\VS{8}Il se transporta de là vers la montagne, à l'orient de Béthel, et il dressa ses tentes, ayant Béthel à l'occident, et Aï à l'orient ; et il bâtit là un autel à Yahweh, et invoqua le nom de Yahweh.
\VS{9}Puis Abram partit de là, marchant et s'avançant vers le midi.
\TextTitle{Abram en Egypte}
\VS{10}Mais la famine étant survenue dans le pays, Abram descendit en Egypte pour s'y retirer, car la famine était grande dans le pays.
\VS{11}Comme il était près d'entrer en Egypte, il dit à Saraï, sa femme : Voici, je sais que tu es une fort belle femme ;
\VS{12}c'est pourquoi, quand les Egyptiens te verront, ils diront : C'est la femme de cet homme, et ils me tueront, mais ils te laisseront vivre.
\VS{13}Dis donc, je te prie, que tu es ma sœur, afin que je sois bien traité à cause de toi, et que par ton moyen, ma vie soit préservée.
\VS{14}Il arriva donc qu'aussitôt qu'Abram fut arrivé en Egypte, les Egyptiens virent que cette femme était fort belle.
\VS{15}Les principaux de la cour de Pharaon la virent aussi et la vantèrent à Pharaon, et elle fut enlevée pour être menée dans la maison de Pharaon.
\VS{16}Il traita bien Abram à cause d'elle, de sorte qu'il en eut des brebis, des bœufs, des ânes, des serviteurs, des servantes, des ânesses, et des chameaux.
\VS{17}Mais Yahweh frappa de grandes plaies Pharaon et sa maison, à cause de Saraï, femme d'Abram.
\VS{18}Alors Pharaon appela Abram, et lui dit : Qu'est-ce que tu m'as fait ? Pourquoi ne m'as-tu pas déclaré que c'était ta femme ?
\VS{19}Pourquoi as-tu dit : C'est ma sœur ? Car je l'avais prise pour ma femme ; mais maintenant, voici ta femme, prends-la, et va-t'en.
\VS{20}Et Pharaon ayant donné ordre à ses gens, ils le renvoyèrent, lui, sa femme, et tout ce qui était à lui.
\Chap{13}
\TextTitle{Retour d'Abram à Canaan}
\VerseOne{}Abram donc monta d'Egypte vers le midi, lui, sa femme, et tout ce qui lui appartenait, et Lot avec lui.
\VS{2}Et Abram était très riche en bétail, en argent, et en or.
\VS{3}Et il s'en retourna en suivant la route qu'il avait suivie du midi à Béthel, jusqu'au lieu où il avait dressé ses tentes au commencement, entre Béthel et Aï,
\VS{4}au même lieu où était l'autel qu'il y avait bâti au commencement, et Abram invoqua là le nom de Yahweh.
\TextTitle{Abram se sépare de Lot\FTNTT{Ge. 13:12}}
\VS{5}Lot aussi, qui marchait avec Abram, avait des brebis, des boeufs, et des tentes.
\VS{6}Et le pays ne pouvait les porter pour demeurer ensemble ; car leurs biens étaient si grand qu'ils ne pouvaient demeurer ensemble.
\VS{7}Il y eut querelle entre les bergers du bétail d'Abram et les bergers du bétail de Lot ; or en ce temps-là, les Cananéens et les Phérésiens habitaient dans le pays.
\VS{8}Et Abram dit à Lot : Je te prie qu'il n'y ait point de dispute entre moi et toi, ni entre mes bergers et les tiens, car nous sommes frères.
\VS{9}Tout le pays n'est-il pas devant toi ? Sépare-toi je te prie d'avec moi. Si tu vas à gauche, j’irai à droite ; et si tu vas à droite, j’irai à gauche.
\TextTitle{Lot s'établit à Sodome\FTNTT{Ge. 13:10}}
\VS{10}Lot, levant les yeux, vit que toute la plaine du Jourdain était entièrement arrosée. Avant que Yahweh ait détruit Sodome et Gomorrhe, c’était, jusqu'à Tsoar, comme le jardin de Yahweh, et comme le pays d'Egypte.
\VS{11}Lot choisit pour lui toute la plaine du Jourdain, et alla du côté de l’orient ; ainsi ils se séparèrent l'un de l'autre.
\VS{12}Abram habita dans le pays de Canaan, et Lot habita dans les villes de la plaine, et dressa ses tentes jusqu'à Sodome.
\VS{13}Les habitants de Sodome étaient méchants et de grands pécheurs contre Yahweh.
\TextTitle{Yahweh confirme son alliance avec Abram}
\VS{14}Yahweh dit à Abram, après que Lot se fut séparé de lui : Lève maintenant tes yeux, et regarde du lieu où tu es vers le nord, le midi, l'orient, et l'occident.
\VS{15}Car je te donnerai, à toi et à ta postérité pour toujours, tout le pays que tu vois.
\VS{16}Je rendrai ta postérité comme la poussière de la terre ; en sorte que si quelqu'un peut compter la poussière de la terre, il comptera aussi ta postérité\FTNT{Ro. 4:18 ; Hé. 11:12.}.
\VS{17}Lève-toi donc et promène-toi dans le pays, dans sa longueur et dans sa largeur, car je te le donnerai.
\VS{18}Abram ayant transporté ses tentes, alla habiter dans les plaines de Mamré, qui sont près d’Hébron et là, il bâtit un autel à Yahweh.
\Chap{14}
\TextTitle{Abram va au secours de Lot}
\VerseOne{}Dans le temps d'Amraphel, roi de Schinear, d'Arjoc, roi d'Ellasar, de Kedorlaomer, roi d'Elam, et de Tideal, roi de Gojim,
\VS{2}il arriva qu’ils firent la guerre contre Béra, roi de Sodome, et contre Birscha, roi de Gomorrhe, et contre Schineab, roi d'Adma, et contre Schémeéber, roi de Tseboïm, et contre le roi de Béla, qui est Tsoar.
\VS{3}Tous ceux-ci se joignirent dans la vallée de Siddim, qui est la mer salée.
\VS{4}Ils avaient été asservis douze années à Kedorlaomer, et la treizième année, ils s'étaient révoltés.
\VS{5}A la quatorzième année, Kedorlaomer et les rois qui étaient avec lui vinrent et ils battirent les Rephaïm à Aschteroth-Karnaïm, les Zuzim à Ham, et les Emin à la plaine de Schavé-Kirjathaïm,
\VS{6}et les Horiens dans leur montagne de Séir, jusqu'au chêne de Paran, qui est près du désert.
\VS{7}Puis ils s’en retournèrent et vinrent à En-Mischpath, qui est Kadès ; et ils frappèrent tout le pays des Amalécites et des Amoréens qui habitaient dans Hatsatson-Thamar.
\VS{8}Alors le roi de Sodome, le roi de Gomorrhe, le roi d'Adma, le roi de Tseboïm, et le roi de Béla qui est Tsoar, sortirent et rangèrent leurs troupes contre eux dans la vallée de Siddim.
\VS{9}C'est-à-dire contre Kedorlaomer, roi d'Elam, et contre Tideal, roi de Gojim, et contre Amraphel, roi de Schinear, et contre Arjoc, roi d'Ellasar : Quatre rois contre cinq.
\VS{10}La vallée de Siddim était pleine de puits de bitume ; les rois de Sodome et de Gomorrhe s'enfuirent et y tombèrent, et le reste s'enfuit dans la montagne.
\VS{11}Ils prirent donc toutes les richesses de Sodome et de Gomorrhe, et tous leurs vivres ; puis ils se retirèrent.
\VS{12}Ils prirent aussi Lot, fils du frère d'Abram, qui habitait dans Sodome, et tous ses biens ; puis ils s'en allèrent.
\VS{13}Un fuyard vint avertir Abram, l’Hébreu, qui demeurait dans les plaines de Mamré, l’Amoréen, frère d'Eschcol, et frère d’Aner, qui avaient fait alliance avec Abram.
\VS{14}Dès qu’Abram eut appris que son frère avait été emmené prisonnier, il arma trois cent dix-huit de ses plus braves serviteurs, nés dans sa maison, et il poursuivit ces rois jusqu'à Dan.
\VS{15}Il divisa sa troupe, il se jeta sur eux de nuit, lui et ses serviteurs ; il les battit et les poursuivit jusqu'à Choba, qui est à la gauche de Damas.
\VS{16}Il ramena tous les biens qu'ils avaient pris ; il ramena aussi Lot, son frère, ses biens, les femmes et le peuple.
\TextTitle{Melchisédek, sacrificateur d'El Elyon (Dieu Très-Haut)}
\VS{17}Le roi de Sodome sortit à la rencontre d’Abram qui revenait vainqueur de Kedorlaomer, et des rois qui étaient avec lui, dans la vallée de la plaine, qui est la vallée royale.
\VS{18}Melchisédek\FTNT{Melchisédek est un type de Christ (Ps. 110:4 ; Hé. 5:5-6 ; Hé. 6:20 ; Hé. 7:1-2). Ce personnage nous montre l’aspect de Christ en tant que roi de Salem, ce qui signifie «~paix~», et Souverain Sacrificateur possédant un sacerdoce non transmissible (Hé. 7:24).}, roi de Salem, fit apporter du pain et du vin, or il était Sacrificateur du Dieu Très-Haut.
\VS{19}Il bénit Abram en disant : Béni soit Abram par le Dieu Très-Haut, Maître du ciel et de la terre.
\VS{20}Béni soit le Dieu Très-Haut qui a livré tes ennemis entre tes mains. Et Abram lui donna la dîme\FTNT{Voir commentaire sur la dîme en No. 18:21 et Mal. 3:10.} de tout.
\VS{21}Le roi de Sodome dit à Abram : Donne-moi les personnes, et prends pour toi les richesses.
\VS{22}Abram répondit au roi de Sodome : Je lève ma main vers Yahweh, le Dieu Très-Haut, Maître du ciel et de la terre :
\VS{23}Je ne prendrai rien de tout ce qui est à toi, pas même un fil, ni un cordon de soulier, afin que tu ne dises point : J'ai enrichi Abram.
\VS{24}Seulement, ce que les jeunes gens ont mangé, et la part des hommes qui sont venus avec moi, Aner, Eschcol, et Mamré, qui prendront leur part.
\Chap{15}
\TextTitle{Yahweh promet un enfant à Abram}
\VerseOne{}Après ces choses, la parole de Yahweh fut adressée à Abram dans une vision, en disant : Abram, ne crains point, je suis ton bouclier, et ta récompense sera très grande.
\VS{2}Abram répondit : Seigneur Yahweh, que me donneras-tu ? Je m'en vais sans laisser d'enfants après moi, et l’héritier de ma maison c'est Eliézer de Damas.
\VS{3}Abram dit aussi : Voici, tu ne m'as point donné d'enfants ; et voilà, le serviteur né dans ma maison sera mon héritier.
\VS{4}Alors la parole de Yahweh lui fut adressée ainsi : Ce n’est pas lui qui sera ton héritier, mais c’est celui qui sortira de tes entrailles qui sera ton héritier.
\VS{5}Puis l'ayant fait sortir dehors, il lui dit : Lève maintenant les yeux au ciel et compte les étoiles si tu peux les compter. Et il lui dit : Ainsi sera ta postérité.
\VS{6}Abram crut à Yahweh qui lui imputa cela à justice\FTNT{Ga. 3:6 ; Ja. 2:23 ; Ro. 4:3.}.
\TextTitle{Yahweh annonce l'esclavage de la postérité d'Abram}
\VS{7}Et il lui dit : Je suis Yahweh qui t'ai fait sortir d'Ur en Chaldée, afin de te donner ce pays-ci pour le posséder.
\VS{8}Abram répondit : Seigneur Yahweh, à quoi connaîtrai-je que je le posséderai ?
\VS{9}Et Yahweh lui répondit : Prends une génisse de trois ans,  une chèvre de trois ans, un bélier de trois ans, une tourterelle, et un pigeon.
\VS{10}Abram prit tous ces animaux, les coupa par le milieu, et mit chaque morceau l’un vis-à-vis de l’autre, mais il ne partagea point les oiseaux.
\VS{11}Les oiseaux de proie descendirent sur les cadavres, mais Abram les chassa.
\VS{12}Au coucher du soleil, un profond sommeil tomba sur Abram, et voici, une frayeur d'une grande obscurité tomba sur lui.
\VS{13}Et Yahweh dit à Abram : Sache comme une chose certaine que tes descendants habiteront quatre cents ans comme étrangers dans un pays qui ne leur appartiendra point, et qu’ils seront asservis aux habitants du pays qui les opprimera\FTNT{Ac. 7:6 ; Ga. 3:17.}.
\VS{14}Mais je jugerai la nation à laquelle ils seront asservis, et après cela ils sortiront avec de grands biens\FTNT{Ex. 3:22.}.
\VS{15}Et toi tu iras vers tes pères en paix, et tu seras enterré après une heureuse vieillesse.
\VS{16}A la quatrième génération, ils reviendront ici ; car l'iniquité des Amoréens n'est pas encore à son comble.
\VS{17}Quand le soleil fut couché, il y eut une obscurité profonde, et voici, ce fut une fournaise fumante, et des flammes passèrent entre les animaux qui avaient été partagés.
\VS{18}En ce jour-là, Yahweh traita alliance avec Abram, en disant : Je donne ce pays à ta postérité, depuis le fleuve d'Egypte jusqu'au grand fleuve, le fleuve d'Euphrate ;
\VS{19}le pays des Kéniens, des Keniziens, des Kadmoniens,
\VS{20}des Héthiens, des Phéréziens, des Rephaïm,
\VS{21}des Amoréens, des Cananéens, des Guirgasiens, et des Jébusiens.
\Chap{16}
\TextTitle{Saraï pousse Abram dans les bras de sa servante}
\VerseOne{}Saraï, femme d'Abram, ne lui avait enfanté aucun enfant, mais elle avait une servante égyptienne nommée Agar.
\VS{2}Et Saraï dit à Abram : Voici, Yahweh m'a rendue stérile ; viens je te prie vers ma servante, peut-être aurai-je des enfants par elle. Et Abram écouta la voix de Saraï.
\VS{3}Alors Saraï, femme d'Abram, prit Agar, sa servante égyptienne, et la donna pour femme à Abram, son mari, après qu’Abram eut habité dix ans dans le pays de Canaan.
\VS{4}Il alla donc vers Agar, et elle conçut. Quand Agar se vit enceinte, elle regarda sa maîtresse avec mépris.
\VS{5}Et Saraï dit à Abram : L'outrage qui m'est fait retombe sur toi. J’ai mis ma servante dans ton sein, mais quand elle a vu qu'elle avait conçu, elle m'a regardée avec mépris. Que Yahweh soit juge entre moi et toi !
\VS{6}Alors Abram répondit à Saraï : Voici, ta servante est entre tes mains, traite-la comme il te plaira. Saraï donc la maltraita, et Agar s'enfuit de devant elle.
\VS{7}Mais l'Ange de Yahweh la trouva auprès d'une fontaine d'eau dans le désert, près de la fontaine qui est sur le chemin de Schur.
\VS{8}Il lui dit : Agar, servante de Saraï, d'où viens-tu ? Et où vas-tu ? Et elle répondit : Je m'enfuis de devant Saraï, ma maîtresse.
\VS{9}L'Ange de Yahweh lui dit : Retourne vers ta maîtresse et humilie-toi sous sa main.
\VS{10}L'Ange de Yahweh lui dit : Je multiplierai beaucoup ta postérité, elle sera si nombreuse qu'on ne pourra la compter.
\VS{11}L'Ange de Yahweh lui dit aussi : Voici, tu as conçu, et tu enfanteras un fils que tu appelleras Ismaël, car Yahweh a entendu ton affliction.
\VS{12}Et ce sera un homme farouche comme un âne sauvage ; sa main sera contre tous, et la main de tous contre lui ; et il habitera en face de tous ses frères.
\VS{13}Alors elle appela Atta-El-roï (tu es le Dieu qui me voit) le nom de Yahweh qui lui avait parlé ; car elle dit : N'ai-je pas même, ici, vu celui qui me voyait ?
\VS{14}C'est pourquoi on a appelé ce puits le puits du vivant qui me voit ; lequel est entre Kadès et Bared.
\TextTitle{Naissance d'Ismaël}
\VS{15}Agar donc enfanta un fils à Abram ; et Abram donna le nom d’Ismaël  au fils qu'Agar lui avait enfanté\FTNT{Ga. 4:22.}.
\VS{16}Abram était âgé de quatre-vingt-six ans quand Agar enfanta Ismaël à Abram.
\Chap{17}
\TextTitle{El Schaddaï (Dieu Tout-Puissant) confirme sa promesse}
\VerseOne{}Lorsqu’Abram fut âgé de quatre-vingt-dix-neuf ans, Yahweh lui apparut et lui dit : Je suis le Dieu Tout-Puissant\FTNT{Dieu se révèle ici à Abraham comme le Dieu Tout-Puissant. Or Christ s’est présenté à l’apôtre Jean comme le Dieu Tout Puissant (Ap. 1:8).  Plus loin en Ap. 5:6, le Seigneur apparaît au milieu du trône céleste sous la forme d’un Agneau ayant sept cornes qui représentent sa toute-puissance. Jésus est bien le Dieu Tout-Puissant qui s’était révélé à Abraham (Da. 8:20-22).}. Marche devant ma face, et sois intègre.
\VS{2}J’établirai mon alliance entre moi et toi, et je te multiplierai très abondamment.
\VS{3}Alors Abram tomba sur sa face, et Dieu lui parla et lui dit :
\TextTitle{Abram devient Abraham}
\VS{4}Quant à moi, voici, mon alliance est avec toi, et tu deviendras père d'une multitude de nations\FTNT{Ro. 4:17.}.
\VS{5}On ne t’appellera plus Abram\FTNT{Né. 9:7.}, mais ton nom sera Abraham ; car je t'ai établi père d'une multitude de nations.
\TextTitle{Promesse d’une alliance éternelle}
\VS{6}Je te rendrai fécond  à l’extrême, et je te ferai devenir des nations ; même des rois sortiront de toi\FTNT{Mt. 1:6.}.
\VS{7}J'établirai donc mon alliance entre moi et toi, et entre ta postérité après toi, selon leurs générations, ce sera une alliance éternelle en vertu de laquelle je serai ton Dieu et celui de ta postérité après toi.
\VS{8}Je te donnerai, et à ta postérité après toi, le pays où tu demeures comme étranger, à savoir tout le pays de Canaan, en possession perpétuelle, et je serai leur Dieu.
\TextTitle{La circoncision, signe de l'alliance}
\VS{9}Dieu dit encore à Abraham : Tu garderas donc mon alliance, toi et ta postérité après toi, selon leurs générations.
\VS{10}C’est ici mon alliance entre moi et vous, et entre ta postérité après toi, que vous garderez : Tout mâle parmi vous sera circoncis.
\VS{11}Vous circoncirez la chair de votre prépuce ; et cela sera le signe de l'alliance entre moi et vous\FTNT{Ac. 7:8 ; Ro. 4:11.}.
\VS{12}Tout enfant mâle de huit jours sera circoncis parmi vous dans vos générations, tant celui qui est né dans la maison que l'esclave acquis à prix d’argent de tout étranger qui n'est point de ta race\FTNT{Lu. 2:21 ; Lé. 12:3.}.
\VS{13}On ne manquera donc point de circoncire celui qui est né dans ta maison, et celui qui est acquis à prix d’argent, et mon alliance sera dans votre chair pour être une alliance perpétuelle.
\VS{14}Et le mâle incirconcis qui n’aura pas été circoncis dans sa  chair sera retranché du milieu de son peuple parce qu'il aura violé mon alliance.
\TextTitle{Saraï devient Sara ; promesse de la naissance d’Isaac}
\VS{15}Dieu dit aussi à Abraham : Quant à Saraï, ta femme, tu n'appelleras plus son nom Saraï, mais son nom sera Sara.
\VS{16}Je la bénirai, et même je te donnerai un fils d'elle. Je la bénirai et elle deviendra des nations ; des rois, chefs de peuples sortiront d'elle.
\VS{17}Alors Abraham se prosterna la face contre terre, et sourit en disant en son cœur : Naîtrait-il un fils à un homme âgé de cent ans ? Et Sara, âgée de quatre-vingt-dix ans, aurait-elle un enfant ?
\VS{18}Et Abraham dit à Dieu : Je te prie, qu'Ismaël vive devant toi.
\VS{19}Et Dieu dit : Certainement Sara, ta femme, t'enfantera un fils, et tu appelleras son nom Isaac ; et j'établirai mon alliance avec lui pour être une alliance perpétuelle pour sa postérité après lui.
\TextTitle{Une nation sortira d'Ismaël}
\VS{20}Je t'ai aussi exaucé touchant Ismaël : Voici, je le bénirai, et je le ferai croître et multiplier très abondamment. Il engendrera douze princes, et je le ferai devenir une grande nation.
\VS{21}Mais j'établirai mon alliance avec Isaac, que Sara t'enfantera l'année qui vient, en cette même saison.
\VS{22}Et Dieu ayant achevé de parler, s’éleva au-dessus d'Abraham.
\VS{23}Et Abraham prit son fils Ismaël, avec tous ceux qui étaient nés dans sa maison, et tous ceux qu'il avait acquis à prix d’argent, tous les mâles qui étaient des gens de sa maison, et il circoncit la chair de leur prépuce en ce même jour-là, comme Dieu le lui avait dit.
\VS{24}Abraham était âgé de quatre-vingt-dix-neuf ans quand il circoncit la chair de son prépuce ;
\VS{25}et Ismaël, son fils, était âgé de treize ans lorsqu'il fut circoncis.
\VS{26}En ce même jour, Abraham fut circoncis, et son fils Ismaël aussi.
\VS{27}Et tous les gens de sa maison, tant ceux qui étaient nés dans sa maison que ceux qui avaient été acquis à prix d’argent des étrangers, furent circoncis avec lui.
\Chap{18}
\TextTitle{Abraham, ami de Yahweh\FTNTT{Jn 3:29 ; 15:13-15}}
\VerseOne{}Puis Yahweh lui apparut dans les plaines de Mamré, comme il était assis à la porte de sa tente, pendant la chaleur du jour.
\VS{2}Levant ses yeux, il regarda : Et voici, trois hommes parurent devant lui. Quand il les vit, il courut au-devant d'eux depuis la porte de sa tente, et se prosterna à terre\FTNT{Hé. 13:2.} ;
\VS{3}Et il dit : Mon Seigneur, je te prie, si j'ai trouvé grâce devant tes yeux, ne passe point outre, je te prie, et arrête-toi chez ton serviteur.
\VS{4}Qu'on prenne, je vous prie, un peu d'eau, et lavez vos pieds, et reposez-vous sous un arbre.
\VS{5}J’apporterai un morceau de pain pour fortifier votre cœur, après quoi vous passerez outre ; car c'est pour cela que vous êtes venus vers votre serviteur. Et ils dirent : Fais ce que tu as dit.
\VS{6}Abraham donc s'en alla en hâte dans la tente vers Sara, et lui dit : Hâte-toi, prends trois mesures de fleur de farine, pétris-les, et fais des gâteaux.
\VS{7}Puis Abraham courut au troupeau et prit un veau tendre et bon, et le donna à un serviteur qui se hâta de l'apprêter.
\VS{8}Ensuite, il prit du beurre et du lait, et le veau qu'on avait apprêté, et le mit devant eux ; et il se tint auprès d'eux sous l'arbre, et ils mangèrent.
\VS{9}Et ils lui dirent : Où est Sara ta femme ? Et il répondit : La voilà dans la tente.
\VS{10}Et l'un d'entre eux dit : Je ne manquerai pas de revenir vers toi en ce même temps où nous sommes, et voici, Sara, ta femme, aura un fils. Et Sara écoutait à la porte de la tente qui était derrière lui\FTNT{Ro. 9:9.}.
\VS{11}Or Abraham et Sara étaient vieux, fort avancés en âge ; et Sara n'avait plus ce que les femmes sont accoutumées d'avoir\FTNT{Ro. 4:19 ; Hé. 11:11.}.
\VS{12}Et Sara rit en elle-même et dit : Etant vieille, et mon Seigneur étant fort âgé, aurai-je encore des désirs ?
\VS{13}Et Yahweh dit à Abraham : Pourquoi Sara a-t-elle ri en disant : Serait-il vrai que j'aurais un enfant, étant vieille comme je suis ?
\VS{14}Y a-t-il quelque chose qui soit difficile à Yahweh ? Je reviendrai vers toi à cette époque, en ce même temps où nous sommes et Sara aura un fils\FTNT{Mt. 19:26 ; Lu. 1:37.}.
\VS{15}Et Sara le nia en disant : Je n'ai point ri ; car elle avait peur. Mais il dit : Cela n'est pas, car tu as ri.
\VS{16}Et ces hommes se levèrent de là, et regardèrent vers Sodome ; et Abraham alla  avec eux pour les accompagner.
\VS{17}Et Yahweh dit : Cacherai-je à Abraham ce que je vais faire ?
\VS{18}Abraham deviendra certainement une nation grande et puissante, et toutes les nations de la terre seront bénies en lui\FTNT{Ac. 3:25 ; Ga. 3:8.}.
\VS{19}Car je le connais, et je sais qu'il ordonnera à ses enfants, et à sa maison après lui, de garder la voie de Yahweh, pour faire ce qui est juste et droit ; afin que Yahweh fasse venir sur Abraham tout ce qu'il lui a dit.
\VS{20}Et Yahweh dit : Le cri contre Sodome et Gomorrhe s’est accru, et leur péché s’est fort aggravé.
\VS{21}Je descendrai maintenant, et je verrai s'ils ont fait entièrement selon le cri qui est venu jusqu'à moi ; et si cela n'est pas, je le saurai.
\VS{22}Ces hommes donc partant de là allèrent vers Sodome ; mais Abraham se tint encore devant Yahweh.
\TextTitle{Intercession d'Abraham}
\VS{23}Et Abraham s'approcha et dit : Feras-tu périr le juste avec le méchant ?
\VS{24}Peut-être y a-t-il cinquante justes dans la ville, les feras-tu périr aussi ? Ne pardonneras-tu point à la ville à cause des cinquante justes qui sont au milieu d’elle ?
\VS{25}Non, il n'arrivera pas que tu fasses une telle chose, que tu fasses mourir le juste avec le méchant, et que le juste soit traité comme le méchant ! Non, tu ne le feras point. Celui qui juge toute la terre ne fera-t-il point justice\FTNT{Ro. 3:5-6.} ?
\VS{26}Et Yahweh dit : Si je trouve dans Sodome cinquante justes au milieu de la ville, je pardonnerai à toute la ville à cause d'eux.
\VS{27}Et Abraham répondit en disant : Voici, j'ai pris maintenant la hardiesse de parler au Seigneur, moi qui ne suis que poussière et cendres.
\VS{28}Peut-être en manquera-t-il cinq des cinquante justes ; détruiras-tu toute la ville pour ces cinq-là ? Et Yahweh lui répondit : Je ne la détruirai point si j'y trouve quarante-cinq justes.
\VS{29}Abraham continua de lui parler en disant : Peut-être s'y trouvera-t-il quarante ? Et il dit : Je ne la détruirai point pour l'amour des quarante.
\VS{30}Abraham dit : Je prie le Seigneur de ne pas s'irriter si je parle encore. Peut-être s'en trouvera-t-il trente ? Et il dit : Je ne la détruirai point si j'y trouve trente.
\VS{31}Abraham dit : Voici, maintenant j'ai pris la hardiesse de parler au Seigneur : Peut-être s'en trouvera-t-il vingt ? Et il dit : Je ne la détruirai point pour l'amour des vingt.
\VS{32}Abraham dit : Je prie le Seigneur de ne pas s'irriter, je parlerai encore une seule fois : Peut-être s'y trouvera-t-il dix. Et Yahweh dit : Je ne la détruirai point pour l'amour des dix.
\VS{33}Yahweh s'en alla quand il eut achevé de parler avec Abraham. Et Abraham retourna dans sa demeure.
\Chap{19}
\TextTitle{Des anges chez Lot\FTNTT{Ge. 13:10, 12 ; 19:33}}
\VerseOne{}Sur le soir, les deux anges arrivèrent à Sodome, et Lot était assis à la porte de Sodome. Quand Lot les vit, il se leva pour aller au-devant d'eux, et se prosterna la face contre terre.
\VS{2}Et il leur dit : Voici, je vous prie, mes seigneurs, entrez maintenant dans la maison de votre serviteur, et passez-y la nuit ; lavez-vous les pieds ; puis vous vous lèverez dès le matin et continuerez votre chemin ; et ils dirent : Non, mais nous passerons la nuit dans la rue.
\VS{3}Mais il les pressa tellement qu'ils se retirèrent chez lui ; et quand ils furent entrés dans sa maison, il leur fit un festin, et fit cuire des pains sans levain, et ils mangèrent.
\VS{4}Ils n’étaient pas encore couchés que les hommes de la ville, les hommes de Sodome, environnèrent la maison, depuis les plus jeunes jusqu'aux vieillards, tout le peuple était ensemble.
\VS{5}Ils appelèrent Lot et ils lui dirent : Où sont les hommes qui sont venus cette nuit chez toi ? Fais-les sortir afin que nous les connaissions.
\VS{6}Mais Lot sortit de sa maison pour leur parler à la porte, et ayant fermé la porte après lui,
\VS{7}il leur dit : Je vous prie, mes frères, ne leur faites point de mal.
\VS{8}Voici, j'ai deux filles qui n'ont point encore connu d'homme ; je vous les amènerai et vous les traiterez comme il vous plaira. Seulement, ne faites pas de mal à ces hommes, car ils sont venus à l'ombre de mon toit.
\VS{9}Ils lui dirent : Retire-toi de là. Ils dirent aussi : Cet homme seul est venu pour habiter ici comme étranger, et il veut nous gouverner ? Maintenant nous te ferons pis qu'à eux. Et faisant violence à Lot,  ils s'approchèrent pour briser la porte\FTNT{2 Pi. 2:7-8.}.
\VS{10}Mais les hommes étendirent leurs mains, firent rentrer Lot vers eux dans la maison, et fermèrent la porte.
\VS{11}Et ils frappèrent d’aveuglement les hommes qui étaient à la porte de la maison, depuis le plus petit jusqu'au plus grand, de sorte qu'ils se lassèrent à chercher la porte.
\VS{12}Alors ces hommes dirent à Lot : Qui as-tu encore ici qui t'appartienne ? Gendres, fils et filles, et tout ce qui t'appartient dans la ville, fais-les sortir de ce lieu.
\VS{13}Car nous allons détruire ce lieu parce que le cri contre ses habitants est grand devant Yahweh. Yahweh nous a envoyés pour le détruire.
\VS{14}Lot sortit donc et parla à ses gendres, qui devaient prendre ses filles, et leur dit : Levez-vous, sortez de ce lieu, car Yahweh va détruire la ville. Mais aux yeux de ses gendres, il parut plaisanter.
\TextTitle{Jugement sur Sodome}
\VS{15}Dès l’aube du jour, les anges pressèrent Lot en disant : Lève-toi, prends ta femme et tes deux filles qui se trouvent ici, de peur que tu ne périsses dans le châtiment de la ville.
\VS{16}Et comme il tardait, ces hommes le prirent par la main, et ils prirent aussi par la main sa femme et ses deux filles, parce que Yahweh voulait l'épargner ; et ils l'emmenèrent et le mirent hors de la ville.
\VS{17}Après les avoir fait sortir, l'un d’eux dit : Sauve ta vie, ne regarde point derrière toi, et ne t'arrête en aucun endroit de la plaine ; sauve-toi sur la montagne, de peur que tu ne périsses.
\VS{18}Lot leur répondit : Non, Seigneur, je te prie.
\VS{19}Voici, ton serviteur a maintenant trouvé grâce devant toi, et tu as montré la grandeur de ta bonté à mon égard en préservant ma vie, mais je ne pourrai pas me sauver vers la montagne avant que le mal ne m'atteigne, et je mourrai.
\VS{20}Voici, je te prie, cette ville-là est proche ; je puis m'y enfuir, et elle est petite. Je te prie, que je m'y sauve ; n'est-elle pas petite ? Et mon âme vivra.
\VS{21}Et il lui dit : Voici, je t'ai exaucé encore en cela, de ne point détruire la ville dont tu as parlé.
\VS{22}Hâte-toi, sauve-toi là, car je ne pourrai rien faire jusqu'à ce que tu y sois entré ; c'est pourquoi cette ville fut appelée Tsoar.
\VS{23}Comme le soleil se levait sur la terre, Lot entra dans Tsoar.
\VS{24}Alors Yahweh fit pleuvoir du ciel, sur Sodome et sur Gomorrhe, du soufre et du feu, de la part de Yahweh\FTNT{De. 29:23 ; Lu. 17:29 ; Jud. 1:7.} ;
\VS{25}et il détruisit ces villes-là, et toute la plaine, et tous les habitants des villes, et les herbes de la terre.
\VS{26}Mais la femme de Lot regarda en arrière, et elle devint une statue de sel\FTNT{Lu. 17:31-33.}.
\VS{27}Abraham se leva de bon matin et vint au lieu où il s'était tenu devant Yahweh ;
\VS{28}et regardant vers Sodome et Gomorrhe, et vers toute la terre de cette plaine-là, il vit monter de la terre une fumée comme la fumée d'une fournaise.
\VS{29}Lorsque Dieu détruisit les villes de la plaine, il se souvint d'Abraham, et laissa Lot s’en aller  du milieu du désastre par lequel il détruisit les villes où Lot avait établi sa demeure.
\TextTitle{Une abomination commise dans la famille de Lot\FTNTT{Ge. 13:10,12 ; 19:1 ; Lu. 22:31-62}}
\VS{30}Lot quitta Tsoar et habita sur la montagne avec ses deux filles, car il craignait de demeurer dans Tsoar, et il se retira dans une caverne avec ses deux filles.
\VS{31}L'aînée dit à la plus jeune : Notre père est vieux, et il n'y a personne sur la terre pour venir vers nous, selon la coutume de tous les pays.
\VS{32}Viens, donnons du vin à notre père, et couchons avec lui  afin que nous conservions la race de notre père.
\VS{33}Elles donnèrent donc du vin à boire à leur père cette nuit-là ; et l'aînée vint, et coucha avec son père, mais il ne s'aperçut point ni quand elle se coucha ni quand elle se leva.
\VS{34}Le lendemain, l'aînée dit à la plus jeune : Voici, j'ai couché la nuit dernière avec mon père, donnons-lui encore du vin à boire cette nuit, puis va et couche avec lui, et nous conserverons la race de notre père.
\VS{35}Elles firent boire du vin à leur père encore cette nuit-là ; et la plus jeune se leva et coucha avec lui ; mais il ne s'aperçut point ni quand elle se coucha ni quand elle se leva.
\VS{36}Ainsi, les deux filles de Lot conçurent de leur père.
\VS{37}L’aînée enfanta un fils qu’elle appela du nom de Moab ; c'est le père des Moabites jusqu'à ce jour.
\VS{38}La plus jeune aussi enfanta un fils qu’elle appela du nom de Ben-Ammi ; c'est le père des Ammonites jusqu'à ce jour.
\Chap{20}
\TextTitle{Faute d'Abraham à Guérar\FTNTT{Ge. 26:6-32}}
\VerseOne{}Abraham s'en alla de là pour le pays du midi ; il demeura entre Kadès et Schur, et il habita comme étranger à Guérar.
\VS{2}Abraham disait de Sara sa femme : C'est ma sœur. Et Abimélec, roi de Guérar, envoya des gens prendre Sara.
\VS{3}Mais Dieu apparut la nuit dans un songe à Abimélec, et lui dit : Voici, tu vas mourir, à cause de la femme que tu as prise, car elle a un mari.
\VS{4}Abimélec, qui ne s'était point approché d'elle, répondit : Seigneur, feras-tu donc mourir une nation juste ?
\VS{5}Ne m'a-t-il pas dit : C'est ma sœur? Et elle-même aussi n'a-t-elle pas dit : C'est mon frère ? J'ai fait ceci dans l'intégrité de mon cœur et dans la pureté de mes mains.
\VS{6}Dieu lui dit en songe : Je sais que tu l'as fait dans l'intégrité de ton cœur, aussi ai-je empêché que tu ne pèches contre moi ; c'est pourquoi je n'ai pas permis que tu la touches.
\VS{7}Maintenant donc rends la femme de cet homme, car il est prophète ; et il priera pour toi et tu vivras. Mais si tu ne la rends pas, sache que tu mourras toi et tout ce qui est à toi.
\VS{8}Abimélec se leva de bon matin, appela tous ses serviteurs, et rapporta à leurs oreilles  toutes ces choses, et ils furent saisis de crainte.
\VS{9}Puis Abimélec appela Abraham et lui dit : Que nous as-tu fait ? Et en quoi t'ai-je offensé que tu aies fait venir sur moi et sur mon royaume un grand péché ? Tu m'as fait des choses qui ne doivent point se faire.
\VS{10}Abimélec dit aussi à Abraham : Qu'as-tu vu qui t'aie obligé de faire cela ?
\VS{11}Abraham répondit : C'est parce que je disais : Assurément, il n'y a point de crainte de Dieu dans ce pays, et ils me tueront à cause de ma femme.
\VS{12}De plus, il est vrai qu’elle est ma soeur, fille de mon père ; mais elle n'est pas fille de ma mère ; et elle m'a été donnée pour femme.
\VS{13}Lorsque Dieu me fit errer loin de la maison de mon père, je dis à Sara : Voici la grâce que tu me feras, dis de moi dans tous les lieux où nous irons : C'est mon frère.
\VS{14}Alors Abimélec prit des brebis, des bœufs, des serviteurs et des servantes, et les donna à Abraham, et lui rendit Sara, sa femme.
\VS{15}Abimélec lui dit : Voici, mon pays est à ta disposition, demeure où il te plaira.
\VS{16}Et il dit à Sara : Voici, je donne à ton frère mille pièces d'argent ; cela te sera un voile sur les yeux  pour tous ceux qui sont avec toi, et envers tous les autres ; et ainsi elle fut reprise.
\VS{17}Abraham pria Dieu, et Dieu guérit Abimélec, sa femme, et ses servantes ; et elles eurent des enfants.
\VS{18}Car Yahweh avait frappé de stérilité en  fermant toute matrice de la maison d'Abimélec, à cause de Sara, femme d'Abraham.
\Chap{21}
\TextTitle{Naissance d'Isaac}
\VerseOne{}Et Yahweh visita Sara, comme il avait dit ; et il agit selon ses paroles.
\VS{2}Sara donc conçut, et enfanta un fils à Abraham dans sa vieillesse, au temps précis que Dieu lui avait dit.
\VS{3}Abraham donna le nom d’Isaac au fils qui lui était né, que Sara lui avait enfanté.
\VS{4}Abraham circoncit son fils Isaac âgé de huit jours, comme Dieu le lui avait ordonné.
\VS{5}Abraham était âgé de cent ans quand Isaac, son fils, lui naquit.
\VS{6}Et Sara dit : Dieu m'a donné de quoi rire ; tous ceux qui l'apprendront riront avec moi.
\VS{7}Elle dit aussi : Qui aurait dit à Abraham que Sara allaiterait des enfants ? Car je lui ai enfanté un fils dans sa vieillesse.
\VS{8}L'enfant grandit et fut sevré ; et Abraham fit un grand festin le jour où Isaac fut sevré.
\TextTitle{Abraham chasse Agar avec Ismaël\FTNTT{Ga. 4:21-31}}
\VS{9}Sara vit rire le fils qu’Agar, l’Egyptienne, avait enfanté à Abraham ;
\VS{10}et elle dit à Abraham : Chasse cette servante et son fils, car le fils de cette servante n'héritera point avec mon fils, avec Isaac\FTNT{Ga. 4:30.}.
\VS{11}Cette parole déplut fort à Abraham à cause de son fils.
\VS{12}Mais Dieu dit à Abraham : N'aie point de chagrin au sujet de l'enfant ni de ta servante ;  écoute la parole de Sara dans toutes les choses qu’elle te dira, car en Isaac te sera donnée une postérité.
\VS{13}Je ferai aussi devenir le fils de la servante une nation, parce qu'il est ta semence.
\VS{14}Puis Abraham se leva de bon matin et prit du pain et une outre d'eau, et il les donna à Agar en les mettant sur son épaule. Il lui donna aussi l'enfant et la renvoya. Elle se mit en chemin et fut errante au désert de Beer-Schéba.
\VS{15}Quand l'eau de l’outre fut épuisée, elle jeta l'enfant sous un arbrisseau,
\VS{16}et elle alla s’asseoir vis-à-vis, à une portée d’arc, car elle dit : Que je ne voie pas mourir mon enfant. Elle s’assit donc vis-à-vis de lui, éleva la voix et pleura.
\VS{17}Dieu entendit la voix de l'enfant, et l'Ange de Dieu appela des cieux Agar et lui dit : Qu'as-tu Agar ? Ne crains point, car Dieu a entendu la voix de l'enfant du lieu où il est.
\VS{18}Lève-toi, lève l'enfant, et prends-le par la main, car je le ferai devenir une grande nation.
\VS{19}Et Dieu lui ouvrit les yeux et elle vit un puits d'eau ; elle alla remplir d'eau l’outre, et donna à boire à l'enfant.
\VS{20}Dieu fut avec l'enfant, qui devint grand, et demeura dans le désert ; et il fut tireur d'arc.
\VS{21}Il habita dans le désert de Paran ; et sa mère lui prit une femme du pays d'Egypte.
\TextTitle{Abraham à Beer-Schéba}
\VS{22}Et il arriva en ce temps-là qu'Abimélec, et Picol, chef de son armée, parla à Abraham en disant : Dieu est avec toi dans toutes les choses que tu fais.
\VS{23}Maintenant donc jure-moi ici par le nom de Dieu que tu ne me mentiras point, ni à mes enfants ni aux enfants de mes enfants, et que selon la faveur que je t'ai faite, tu agiras envers moi et envers le pays où tu séjournes comme étranger.
\VS{24}Abraham répondit : Je te le jurerai.
\VS{25}Mais Abraham fit des reproches à Abimélec au sujet d'un puits d'eau, dont les serviteurs d'Abimélec s'étaient emparés de force.
\VS{26}Abimélec répondit : J’ignore qui a fait cela, et aussi tu ne m'en as point informé, et moi, je ne l’apprends qu’aujourd’hui.
\VS{27}Alors Abraham prit des brebis et des bœufs, et les donna à Abimélec, et ils firent alliance ensemble.
\VS{28}Abraham mit à part sept jeunes brebis de son troupeau.
\VS{29}Et Abimélec dit à Abraham : Que veulent dire ces sept jeunes brebis que tu as mises à part ?
\VS{30}Il répondit : C'est que tu prendras ces sept jeunes brebis de ma main pour me servir de témoignage que j'ai creusé ce puits.
\VS{31}C'est pourquoi on appela ce lieu-là Beer-Schéba, car tous deux y jurèrent.
\VS{32}Ils traitèrent donc alliance à Beer-Schéba, puis Abimélec se leva avec Picol, chef de son armée, et ils retournèrent au pays des Philistins.
\VS{33}Abraham planta des tamaris à Beer-Schéba ; et là il invoqua le nom de Yahweh, le Dieu de l’éternité.
\VS{34}Abraham séjourna beaucoup de jours comme étranger dans le pays des Philistins.
\Chap{22}
\TextTitle{Abraham présente Isaac en sacrifice\FTNTT{Hé. 11:17-19}}
\VerseOne{}Or, il arriva après ces choses, que Dieu éprouva Abraham et lui dit : Abraham ! Et il répondit : Me voici.
\VS{2}Et Dieu lui dit : Prends maintenant ton fils, ton unique, celui que tu aimes, Isaac, et va-t'en au pays de Morija, et là offre-le en holocauste sur l'une des montagnes que je te dirai.
\VS{3}Abraham donc s'étant levé de bon matin, sella son âne, et prit deux de ses serviteurs avec lui, et Isaac son fils ; et ayant fendu le bois pour l'holocauste, il se mit en chemin et s'en alla au lieu que Dieu lui avait dit.
\VS{4}Le troisième jour, Abraham levant ses yeux, vit le lieu de loin.
\VS{5}Et Abraham dit à ses serviteurs : Restez ici avec l'âne ; moi et l'enfant nous irons jusque-là pour adorer, après quoi nous reviendrons auprès de vous.
\VS{6}Abraham prit le bois de l'holocauste et le mit sur Isaac, son fils, et prit le feu dans sa main, et un couteau ; et ils s'en allèrent tous deux ensemble.
\VS{7}Alors Isaac parla à Abraham, son père, et dit : Mon père ! Abraham répondit : Me voici mon fils. Et il dit : Voici le feu et le bois, mais où est l’agneau pour l'holocauste\FTNT{Isaac est un autre type de Christ qui s’offre en sacrifice pour l’expiation de nos péchés. La réponse à sa question au v. 7:«~Voici le feu et le bois, mais où est l’agneau pour l’holocauste ?~», a été apportée bien des siècles plus tard par Jean-Baptiste:«~Voici l’agneau de Dieu, qui ôte le péché du monde~». (Jn. 1:29).} ?
\VS{8}Abraham répondit : Mon fils, Dieu se pourvoira lui-même de l’agneau pour l'holocauste. Et ils marchèrent tous deux ensemble.
\VS{9}Et étant arrivés au lieu que Dieu lui avait dit, Abraham bâtit là un autel, et rangea le bois, et ensuite il lia Isaac, son fils, et le mit sur l'autel, par-dessus le bois\FTNT{Ja. 2:21.}.
\VS{10}Puis Abraham étendit sa main et prit le couteau pour égorger son fils.
\VS{11}Mais l'Ange de Yahweh l’appela des cieux et dit : Abraham, Abraham ! Il répondit : Me voici.
\VS{12}L’Ange lui dit : Ne porte pas ta main sur l'enfant, et ne lui fais rien ; car maintenant je sais que tu crains Dieu, puisque tu ne m’as point refusé ton fils, ton unique.
\VS{13}Abraham leva les yeux et regarda ; et voici,  il vit derrière lui un bélier qui était retenu à un buisson par ses cornes ; et Abraham alla prendre le bélier et l'offrit en holocauste à la place de son fils.
\VS{14}Abraham donna à ce lieu le nom de Yahweh-Jiré (Yahweh pourvoira) ; c'est pourquoi on dit aujourd'hui : Dans la montagne de Yahweh il y sera pourvu.
\VS{15}L'Ange de Yahweh appela des cieux Abraham pour la seconde fois,
\VS{16}et dit : Je le jure par moi-même\FTNT{Hé. 6:13-15.}, parole de Yahweh ! Parce que tu as fait cela, et que tu n'as point refusé ton fils, ton unique,
\VS{17}certainement je te bénirai, et je multiplierai très abondamment ta postérité, comme les étoiles du ciel et comme le sable qui est sur le bord de la mer ; et ta postérité possédera la porte de ses ennemis.
\VS{18}Toutes les nations de la terre seront bénies en ta postérité, parce que tu as obéi à ma voix.
\VS{19}Ainsi Abraham retourna vers ses serviteurs, et ils se levèrent et s'en allèrent ensemble à Beer-Schéba ; car Abraham demeurait à Beer-Schéba.
\VS{20}Après ces choses, quelqu'un apporta des nouvelles à Abraham, en disant : Voici, Milca a aussi enfanté des fils à Nachor, ton frère.
\VS{21}Uts, son premier-né, et Buz, son frère, Kemuel, père d'Aram,
\VS{22}Késed, Hazo, Pildasch, Jidlaph et Bethuel.
\VS{23}Bethuel a engendré Rebecca. Milca enfanta ces huit fils à Nachor, frère d'Abraham.
\VS{24}Sa concubine, nommée Réuma, enfanta aussi Thébach, Gaham, Tahasch, et Maaca.
\Chap{23}
\TextTitle{Mort de Sara}
\VerseOne{}Or, Sara vécut cent vingt-sept ans ; ce sont là les années de la vie de Sara.
\VS{2}Sara mourut à Kirjath-Arba, qui est Hébron, dans le pays de Canaan ; et Abraham vint pour mener deuil sur Sara et pour la pleurer.
\VS{3}Et Abraham se leva de devant son mort, il parla aux fils de Heth, en disant :
\VS{4}Je suis étranger et habitant parmi vous ; donnez-moi une possession de sépulcre parmi vous, afin que j'enterre mon mort et que je l'ôte de devant moi\FTNT{Ac. 7:5.}.
\VS{5}Les fils de Heth répondirent à Abraham et lui dirent :
\VS{6}Mon seigneur, écoute-nous ! Tu es un prince de Dieu parmi nous, enterre ton mort dans le plus distingué de nos sépulcres ; nul de nous ne te refusera son sépulcre afin que tu y enterres ton mort.
\VS{7}Alors Abraham se leva et se prosterna devant le peuple du pays, devant les Héthiens.
\VS{8}Et il leur parla et dit : S'il vous plaît que j'enterre mon mort et que je l'ôte de devant moi ; écoutez-moi, et intercédez pour moi envers Ephron, fils de Tsochar,
\VS{9}afin qu'il me cède sa caverne de Macpéla, qui est à l’extrémité de son champ ; qu'il me la cède contre sa valeur en argent, afin qu’elle me serve de possession sépulcrale au milieu de vous.
\VS{10}Ephron était assis parmi les fils de Heth. Et Ephron, l’Héthien, répondit à Abraham, en présence des fils de Heth qui l'écoutaient, devant tous ceux qui entraient par la porte de sa ville, et dit :
\VS{11}Non, mon seigneur, écoute-moi ! Je te donne le champ, je te donne aussi la caverne qui y est, je te la donne en présence des enfants de mon peuple ; enterres-y ton mort.
\VS{12}Abraham se prosterna devant le peuple du pays.
\VS{13}Et il parla ainsi à Ephron, en présence de tout le peuple du pays qui écoutait et dit : S'il te plaît, je te prie, écoute-moi ! Je donnerai l'argent du champ ; reçois-le de moi, et j'y enterrerai mon mort.
\VS{14}Et Ephron répondit à Abraham, en disant :
\VS{15}Mon seigneur, écoute-moi ! La terre vaut quatre cents sicles d'argent, qu’est-ce que cela entre moi et toi ? Enterre donc ton mort.
\VS{16}Abraham ayant entendu Ephron, lui paya l'argent dont il avait parlé, en présence des fils de Heth, à savoir quatre cents sicles d'argent ayant cours chez les marchands\FTNT{Ac. 7:16.}.
\VS{17}Le champ d'Ephron, qui était à Macpéla, vis-à-vis de Mamré, le champ et la caverne qui y est, et tous les arbres qui sont dans le champ et dans toutes ses limites alentour,
\VS{18}tout fut acquis comme propriété d’Abraham, en présence des fils de Heth, et de tous ceux qui entraient par la porte de la ville.
\VS{19}Après cela, Abraham enterra Sara, sa femme, dans la caverne du champ de Macpéla, vis-à-vis de Mamré, qui est Hébron, dans le pays de Canaan.
\VS{20}Le champ et la caverne qui y est demeurèrent à Abraham comme possession sépulcrale, acquise des fils de Heth.
\Chap{24}
\TextTitle{Abraham recherche une épouse pour Isaac}
\VerseOne{}Or, Abraham devint vieux et fort avancé en âge ; et Yahweh avait béni Abraham en toute chose.
\VS{2}Abraham dit à son serviteur, le plus ancien des serviteurs de sa maison, l’intendant de tout ce qui lui appartenait : Mets, je te prie, ta main sous ma cuisse ;
\VS{3}et je te ferai jurer par Yahweh, le Dieu du ciel et le Dieu de la terre, que tu ne prendras point de femme pour mon fils parmi les filles des Cananéens, au milieu desquels j'habite.
\VS{4}Mais tu iras dans mon pays et vers mes parents, et tu y prendras une femme pour mon fils Isaac.
\VS{5}Le serviteur lui répondit : Peut-être que la femme ne voudra-t-elle pas me suivre dans ce pays ; me faudra-t-il nécessairement ramener ton fils dans le pays d'où tu es sorti ?
\VS{6}Abraham lui dit : Garde-toi bien d'y ramener mon fils !
\VS{7}Yahweh, le Dieu du ciel, qui m'a fait sortir de la maison de mon père et de ma patrie, qui m'a parlé et qui m’a juré en disant : Je donnerai ce pays à ta postérité, enverra lui-même son ange devant toi ; et c’est là que tu prendras une femme pour mon fils.
\VS{8}Si la femme ne veut pas te suivre, tu seras quitte de ce serment que je te fais faire. Quoi qu'il en soit, tu n’y ramèneras point mon fils.
\VS{9}Le serviteur mit la main sous la cuisse d'Abraham, son Seigneur, et lui jura d’observer ces choses.
\VS{10}Alors le serviteur prit dix chameaux parmi les chameaux de son maître, et s'en alla, ayant à sa disposition tous les biens. Il partit donc et s'en alla en Mésopotamie, à la ville de Nachor.
\VS{11}Il fit reposer les chameaux sur leurs genoux hors de la ville, près d'un puits d'eau, sur le soir, au temps où sortent celles qui vont puiser de l'eau.
\VS{12}Et il dit : Ô Yahweh, Dieu de mon seigneur Abraham, fais que j'aie une heureuse rencontre aujourd'hui, et sois favorable à mon seigneur Abraham.
\VS{13}Voici, je me tiens près de la source d'eau, et les filles des gens de la ville vont sortir pour puiser de l'eau.
\VS{14}Fais donc que la jeune fille à laquelle je dirai : Penche ta cruche, je te prie, afin que je boive, et qui me répondra : Bois, et je donnerai aussi à boire à tes chameaux, soit celle que tu as destinée à ton serviteur Isaac, et par là je connaîtrai que tu es favorable à mon seigneur.
\VS{15}Il n’avait pas encore fini de parler que sortit sa cruche sur l’épaule, Rebecca, fille de Bethuel, fils de Milca, femme de Nachor, frère d'Abraham.
\VS{16}Et la jeune fille était très belle de figure ; elle était vierge, et aucun homme ne l'avait connue. Elle descendit donc à la source, et comme elle remontait après avoir rempli sa cruche,
\VS{17}le serviteur courut au-devant d'elle et lui dit : Laisse-moi boire, je te prie, un peu d’eau de ta cruche.
\VS{18}Elle répondit : Mon seigneur, bois. Elle s’empressa d’abaisser sa cruche sur sa main,  et elle lui donna à boire.
\VS{19}Quand elle eut achevé de lui donner à boire, elle dit : Je puiserai aussi pour tes chameaux jusqu'à ce qu'ils aient achevé de boire.
\VS{20}Et elle s’empressa de vider sa cruche dans l’abreuvoir ; elle courut encore au puits pour puiser de l'eau, et elle puisa pour tous ses chameaux.
\VS{21}L’homme la regardait avec étonnement et sans rien dire, pour voir si Yahweh faisait réussir son voyage ou non.
\VS{22}Quand les chameaux eurent fini de boire, l’homme prit un anneau d'or, du poids d'un demi-sicle, et deux bracelets, pour les mettre sur les mains de cette fille, pesant dix sicles d'or.
\VS{23}Et il lui dit : De qui es-tu fille ? Je te prie, fais-le-moi savoir. Y a-t-il dans la maison de ton père de la place pour nous loger ?
\VS{24}Elle lui répondit : Je suis fille de Bethuel, fils de Milca et de Nachor.
\VS{25}Elle lui dit encore : Il y a chez nous de la paille et du fourrage en abondance, et de la place pour loger.
\VS{26}Alors l’homme s'inclina et adora Yahweh,
\VS{27}et dit : Béni soit Yahweh, le Dieu de mon seigneur Abraham, qui n'a point cessé d'exercer sa bonté et sa fidélité envers mon Seigneur ! Lorsque j'étais en chemin, Yahweh m'a conduit dans la maison des frères de mon seigneur.
\VS{28}La jeune fille courut et rapporta toutes ces choses à la maison de sa mère.
\VS{29}Rebecca avait un frère nommé Laban, qui courut dehors vers l’homme près de la source.
\VS{30}Il avait vu l’anneau et les bracelets aux mains de sa sœur, et il avait entendu les paroles de Rebecca sa sœur, disant : Ainsi m’a parlé l’homme. Il vint donc à cet homme qui se tenait auprès des chameaux, près de la source,
\VS{31}et il lui dit : Entre, béni de Yahweh ! Pourquoi te tiens-tu dehors ? J'ai préparé la maison et une place pour tes chameaux.
\VS{32}L'homme donc entra dans la maison. Laban fit décharger les chameaux, et il donna de la paille et du fourrage  aux chameaux ; et il apporta de l'eau pour laver les pieds de l’homme et les pieds de ceux qui étaient avec lui.
\VS{33}Et il lui présenta à manger. Mais il dit : Je ne mangerai point avant d’avoir dit  ce que j'ai à dire. Parle ! dit Laban.
\VS{34}Alors il dit : Je suis serviteur d'Abraham.
\VS{35}Yahweh a comblé de bénédictions mon seigneur qui est devenu puissant. Il lui a donné des brebis, des bœufs, de l'argent, de l'or, des serviteurs, des servantes, des chameaux, et des ânes.
\VS{36}Sara, la femme de mon seigneur, a enfanté dans sa vieillesse un fils à mon seigneur ; et il lui a donné tout ce qu'il possède.
\VS{37}Mon seigneur m'a fait jurer en disant : Tu ne prendras point de femme pour mon fils parmi les filles des Cananéens dans le pays desquels j’habite ;
\VS{38}mais tu iras dans la maison de mon père et de ma famille prendre une femme pour mon fils.
\VS{39}J’ai dit à mon seigneur : Peut-être que la femme ne voudra-t-elle pas me suivre.
\VS{40}Et il m’a répondu : Yahweh, devant la face de qui j'ai marché, enverra son ange avec toi, et fera réussir ton voyage ; et tu prendras pour mon fils une femme de ma famille et de la maison de mon père.
\VS{41}Quand tu auras été vers ma famille, tu seras alors dégagé de la punition du serment que je te fais faire ; et si on ne te la donne pas, tu seras dégagé de la punition du serment que je te fais faire.
\VS{42}Je suis arrivé aujourd'hui à la source et j'ai dit : Ô Yahweh ! Dieu de mon seigneur Abraham, si tu daignes faire réussir le voyage que j'ai entrepris,
\VS{43}voici, je me tiendrai près de la source d'eau, et la jeune fille qui sortira pour puiser à qui je dirai : Laisse-moi boire, je te prie, un peu d’eau de ta cruche ; et qui me répondra :
\VS{44}Bois toi-même, et je puiserai aussi pour tes chameaux, que cette jeune fille soit la femme que Yahweh a destinée au fils de mon seigneur.
\VS{45}Avant que j’ai fini de parler en mon cœur, voici, Rebecca est sortie, ayant sa cruche sur son épaule ; elle est descendue à la source et a puisé de l'eau ; et je lui ai dit : Donne-moi, à boire, je te prie.
\VS{46}Elle s’est empressée d’abaisser sa cruche de dessus son épaule et m'a dit : Bois, et même je donnerai à boire à tes chameaux. J'ai donc bu, et elle a aussi donné à boire aux chameaux.
\VS{47}Puis je l'ai interrogée en disant : De qui es-tu fille ? Elle a répondu : Je suis fille de Bethuel, fils de Nachor et de Milca. Alors je lui ai mis un anneau à son nez et les bracelets à ses mains.
\VS{48}Puis je me suis incliné, j’ai adoré Yahweh, et j'ai béni Yahweh, le Dieu de mon seigneur Abraham, qui m'a conduit fidèlement, afin que je prenne la fille du frère de mon seigneur pour son fils.
\VS{49}Maintenant donc, si vous voulez user de bonté et de fidélité envers mon seigneur, déclarez-le-moi ; sinon, déclarez-le-moi aussi ; et je me tournerai à droite ou à gauche.
\VS{50}Laban et Bethuel répondirent et dirent : Cette affaire vient de Yahweh, nous ne pouvons te parler ni en bien ni en mal.
\VS{51}Voici Rebecca est devant toi, prends-la et va, et qu'elle soit la femme du fils de ton seigneur, comme Yahweh l’a dit.
\VS{52}Lorsque le serviteur d'Abraham eut entendu leurs paroles, il se prosterna à terre devant Yahweh.
\VS{53}Et le serviteur sortit des objets d'argent et d'or, et des vêtements, et les donna à Rebecca. Il donna aussi de riches présents à son frère et à sa mère.
\VS{54}Puis ils mangèrent et burent, lui et les gens qui étaient avec lui, et ils passèrent la nuit.  Le matin,  quand ils furent levés, le serviteur dit : Laissez-moi retourner vers mon seigneur.
\VS{55}Le frère et la mère lui dirent : Que la jeune fille reste avec nous quelques jours encore, une dizaine de jours ;  après quoi, elle s'en ira.
\VS{56}Il leur répondit : Ne me retardez pas puisque Yahweh a fait réussir mon voyage ; laissez-moi partir  afin que je m'en aille vers mon seigneur.
\VS{57}Alors ils dirent : Appelons la jeune fille et demandons-lui son avis.
\VS{58}Ils appelèrent donc Rebecca et lui dirent : Veux-tu aller avec cet homme ? Et elle répondit : J'irai.
\VS{59}Ainsi ils laissèrent partir Rebecca, leur sœur, et sa nourrice, avec le serviteur d'Abraham et ses gens.
\VS{60}Ils bénirent Rebecca et lui dirent : Tu es notre sœur, puisses-tu devenir des milliers de myriades, et que ta postérité possède la porte de ses ennemis !
\VS{61}Alors Rebecca se leva avec ses servantes, et elles montèrent sur les chameaux et suivirent l’homme. Et le serviteur prit Rebecca et s'en alla.
\VS{62}Or Isaac revenait du puits de Lachaï-roï, et il habitait dans le pays du midi.
\VS{63}Un soir qu’Isaac était sorti dans les champs pour prier, il leva les yeux et regarda, et voici, des chameaux arrivaient.
\VS{64}Rebecca leva aussi les yeux, vit Isaac, et descendit de son chameau ;
\VS{65}car elle avait dit au serviteur : Qui est cet homme qui marche dans les champs à notre rencontre ? Et le serviteur avait répondu : C'est mon seigneur ; et elle prit son voile et se couvrit.
\VS{66}Le serviteur raconta à Isaac toutes les choses qu'il avait faites.
\VS{67}Alors Isaac conduisit Rebecca dans la tente de Sara, sa mère ; il prit Rebecca pour sa femme\FTNT{Pr. 18:22 ; Pr. 31:10-31.} et l'aima. Ainsi Isaac fut consolé après la mort de sa mère.
\Chap{25}
\TextTitle{Ketura, femme d'Abraham}
\VerseOne{}Or, Abraham prit une autre femme nommée Ketura.
\VS{2}Elle lui enfanta Zimram, Jokschan, Medan, Madian, Jischbak, et Schuach.
\VS{3}Jokschan engendra Séba et Dedan. Les fils de Dedan furent Aschurim, Letuschim et Leummim.
\VS{4}Les fils de Madian furent Epha, Epher, Hénoc, Abida, Eldaa. Ce sont là tous les fils de Ketura.
\TextTitle{Isaac hérite d'Abraham\FTNTT{Hé. 1:2}}
\VS{5}Abraham donna tout ce qui lui appartenait à Isaac.
\VS{6}Mais il fit des dons aux fils de ses concubines, et tandis qu’il vivait encore, il les envoya loin de son fils Isaac, du côté de l'orient, dans le pays d’orient.
\TextTitle{Mort d'Abraham}
\VS{7}Voici les jours des années de la vie d’Abraham : Il vécut cent soixante-quinze ans.
\VS{8}Abraham expira et mourut après une heureuse vieillesse, fort âgé et rassasié de jours, et il fut recueilli auprès de son peuple.
\VS{9}Isaac et Ismaël, ses fils, l'enterrèrent dans la caverne de Macpéla, dans le champ d'Ephron, fils de Tschoar, le Héthien, qui est vis-à-vis de Mamré.
\VS{10}C’est le champ qu'Abraham avait acheté des fils de Heth. Là furent enterrés Abraham et Sara, sa femme.
\VS{11}Après la mort d'Abraham, Dieu bénit Isaac son fils.  Isaac habitait près du puits de Lachaï-roï.
\TextTitle{Postérité d'Ismaël}
\VS{12}Voici la postérité d'Ismaël, fils d'Abraham, qu'Agar l’Egyptienne, servante de Sara, avait enfanté à Abraham.
\VS{13}Voici les noms des fils d'Ismaël, par leurs noms, selon leurs générations. Le premier-né d'Ismaël fut Nebajoth, puis Kédar, Adbeel, Mibsam,
\VS{14}Mischma, Duma, Massa,
\VS{15}Hadad, Théma, Jethur, Naphisch, et Kedma.
\VS{16}Ce sont là les fils d'Ismaël, et ce sont là leurs noms, selon leurs parcs, et selon leurs enclos ; douze princes de leurs peuples.
\VS{17}Et voici les années de la vie d'Ismaël : Cent trente-sept ans. Il expira et mourut, et il fut recueilli auprès de son peuple.
\VS{18}Ses descendants habitèrent depuis Havila jusqu'à Schur, qui est vis-à-vis de l'Egypte, en allant vers l'Assyrie. Et le pays qui était échu à Ismaël était à la vue de tous ses frères.
\TextTitle{Postérité d'Isaac}
\VS{19}Voici la postérité d'Isaac, fils d'Abraham.
\VS{20}Abraham engendra Isaac. Isaac était âgé de quarante ans quand il épousa Rebecca, fille de Bethuel, le Syrien, de Paddan-Aram, sœur de Laban, le Syrien.
\VS{21}Isaac pria instamment Yahweh au sujet de sa femme parce qu'elle était stérile ; et Yahweh exauça ses prières ; et Rebecca, sa femme, conçut.
\VS{22}Mais les enfants se heurtaient dans son ventre, et elle dit : S'il en est ainsi, pourquoi suis-je enceinte ? Et elle alla consulter Yahweh.
\VS{23}Et Yahweh lui dit : Deux nations sont dans ton ventre, et deux peuples se sépareront au sortir de tes entrailles ; un de ces peuples sera plus fort que l'autre, et le plus grand sera asservi au plus petit\FTNT{Ro. 9:12.}.
\TextTitle{Naissance des jumeaux : Esaü et Jacob}
\VS{24}Les jours où elle devait accoucher s’accomplirent ; et voici, il y avait deux jumeaux dans son ventre.
\VS{25}Celui qui sortit le premier était roux et tout velu, comme un manteau de poil ; et on lui donna le nom d’Esaü.
\VS{26}Ensuite sortit son frère, tenant de sa main le talon d'Esaü ; c'est pourquoi il fut appelé Jacob\FTNT{Jacob:«~celui qui prend par le talon~» ou «~qui supplante~».}. Isaac était âgé de soixante ans quand ils naquirent.
\TextTitle{Esaü méprise son droit d'aînesse}
\VS{27}Depuis, les enfants devinrent grands. Esaü devint un habile chasseur, et un homme des champs ; mais Jacob fut un homme intègre, se tenant dans les tentes.
\VS{28}Isaac aimait Esaü ; car le gibier était sa nourriture. Mais Rebecca aimait Jacob.
\VS{29}Comme Jacob faisait cuire du potage, Esaü arriva des champs, et il était fatigué.
\VS{30}Et Esaü dit à Jacob : Donne-moi, je te prie, à manger de ce roux, de ce roux-là\FTNT{Probablement un plat de lentilles.} ; car je suis fatigué. C'est pourquoi on appela son nom, Edom\FTNT{Edom:«~rouge, de couleur rousse~».}.
\VS{31}Mais Jacob lui dit : Vends-moi aujourd'hui ton droit d'aînesse.
\VS{32}Et Esaü répondit : Voici, je m'en vais mourir ; et de quoi me servira le droit d'aînesse ?
\VS{33}Et Jacob dit : Jure-moi aujourd'hui ; et il lui jura ; ainsi il vendit son droit d'aînesse à Jacob\FTNT{Hé. 12:16.}.
\VS{34}Et Jacob donna à Esaü du pain et du potage de lentilles ; et il mangea et but ; puis il se leva et s'en alla ; ainsi Esaü méprisa son droit d'aînesse.
\Chap{26}
\TextTitle{Yahweh confirme son alliance à Isaac}
\VerseOne{}Or, il y eut une famine dans le pays, outre la première famine qui eut lieu du temps d'Abraham ; et Isaac s'en alla vers Abimélec, roi des Philistins, à Guérar.
\VS{2}Yahweh lui apparut et lui dit : Ne descends pas en Egypte ; demeure dans le pays que je te dirai.
\VS{3}Demeure dans ce pays-ci, et je serai avec toi, et je te bénirai ; car je donnerai toutes ces contrées à toi et à ta postérité, et j’accomplirai le serment que j'ai fait à ton père Abraham.
\VS{4}Je multiplierai ta postérité comme les étoiles du ciel ; et je donnerai ces contrées à ta postérité ; et toutes les nations de la terre seront bénies en ta postérité,
\VS{5}parce qu'Abraham a obéi à ma voix, et qu'il a gardé mon ordonnance, mes commandements, mes statuts et mes lois.
\TextTitle{Faute d'Isaac à Guérar\FTNTT{Ge. 20}}
\VS{6}Isaac donc demeura à Guérar.
\VS{7}Et quand les gens du lieu posaient des questions sur sa femme, il disait : C'est ma sœur ; car il craignait de dire : C'est ma femme ; de peur, disait-il, que les habitants du lieu ne me tuent à cause de Rebecca, car elle est belle de figure.
\VS{8}Comme son séjour se prolongeait, il arriva qu'Abimélec, roi des Philistins, regardant par la fenêtre, vit Isaac qui plaisantait avec Rebecca, sa femme\FTNT{Ge. 20.}.
\VS{9}Alors Abimélec appela Isaac et lui dit : Voici, c'est véritablement ta femme. Comment as-tu pu dire : C'est ma soeur ? Et Isaac lui répondit : C'est parce que j'ai dit : Il ne faut pas que je meure à cause d'elle.
\VS{10}Et Abimélec dit : Que nous as-tu fait ? Il s'en est peu fallu que quelqu'un du peuple n'ait couché avec ta femme, et tu nous aurais rendus coupables.
\VS{11}Abimélec donc fit une ordonnance à tout le peuple en disant : Celui qui touchera cet homme, ou à sa femme, sera certainement puni de mort.
\VS{12}Isaac sema dans cette terre-là et il recueillit cette année-là le centuple ; car Yahweh le bénit.
\VS{13}Cet homme devint riche, et il alla s’enrichissant de plus en plus, jusqu'à ce qu'il devint fort riche.
\VS{14}Il avait des troupeaux de menu bétail et des troupeaux de gros bétail, et un grand nombre de serviteurs ; et les Philistins lui portèrent envie ; 
\VS{15}Et tous les puits que les serviteurs de son père avaient creusés, du temps de son père Abraham, les Philistins les bouchèrent et les remplirent de terre.
\VS{16}Abimélec aussi dit à Isaac : Va-t’en de chez nous, car tu es devenu beaucoup plus puissant que nous.
\TextTitle{Les puits d'Isaac}
\VS{17}Isaac donc partit de là, et campa dans la vallée de Guérar, où il s’établit.
\VS{18}Isaac creusa de nouveau les puits d'eau qu'on avait creusés du temps d'Abraham, son père, et que les Philistins avaient bouchés après la mort d'Abraham, et il leur donna les mêmes noms que son père leur avait donnés.
\VS{19}Les serviteurs d'Isaac creusèrent dans cette vallée et y trouvèrent un puits d'eau vive.
\VS{20}Mais les bergers de Guérar eurent une querelle avec les bergers d'Isaac, disant : L'eau est à nous. Et il appela le nom du puits Esek parce qu'ils avaient contesté avec lui.
\VS{21}Ensuite, ils creusèrent un autre puits, pour lequel ils contestèrent aussi ; et il appela son nom Sitna.
\VS{22}Alors il se transporta de là et creusa un autre puits pour lequel ils ne contestèrent point, et il le nomma Rehoboth, en disant : C'est parce que Yahweh nous a maintenant mis au large, et nous fructifierons dans le pays.
\VS{23}Et de là il remonta à Beer-Schéba.
\VS{24}Yahweh lui apparut cette nuit-là et lui dit : Je suis le Dieu d'Abraham, ton père ; ne crains point, car je suis avec toi, je te bénirai et je multiplierai ta postérité à cause d'Abraham, mon serviteur.
\VS{25}Alors il bâtit là un autel, et invoqua le nom de Yahweh, et il y dressa ses tentes. Et les serviteurs d'Isaac y creusèrent un puits.
\VS{26}Abimélec vint à lui de Guérar avec Ahuzath, son ami, et Picol, chef de son armée.
\VS{27}Mais Isaac leur dit : Pourquoi venez-vous vers moi, puisque vous me haïssez et que vous m'avez renvoyé de chez vous ?
\VS{28}Ils répondirent : Nous avons vu clairement que Yahweh est avec toi ; et nous avons dit : Qu'il y ait maintenant un serment solennel entre nous, c'est-à-dire entre nous et toi ; et traitons alliance avec toi.
\VS{29}Jure que tu ne nous feras aucun mal, de même que nous ne t'avons point maltraité, que nous t'avons fait seulement du bien, et que nous t’avons laissé partir en paix. Toi qui es maintenant béni de Yahweh.
\VS{30}Alors il leur fit un festin, et ils mangèrent et burent.
\VS{31}Ils se levèrent de bon matin, et jurèrent l'un à l'autre. Puis Isaac les renvoya, et ils s'en allèrent en paix.
\VS{32}Ce même jour, les serviteurs d'Isaac vinrent lui parler du puits qu'ils avaient creusé, et lui dirent : Nous avons trouvé de l'eau.
\VS{33}Et il l'appela Schiba. C'est pourquoi le nom de la ville a été Beer-Schéba jusqu'à aujourd'hui.
\VS{34}Esaü, âgé de quarante ans, prit pour femmes Judith, fille de Beéri, le Héthien, et Basmath, fille d'Elon, le Héthien.
\VS{35}Elles furent un sujet d’amertume pour l’esprit d’Isaac et de Rebecca.
\Chap{27}
\TextTitle{Jacob prend la bénédiction d'Isaac à la place d'Esaü}
\VerseOne{}Et il arriva que quand Isaac fut devenu vieux, et que ses yeux furent si affaiblis qu'il ne pouvait plus voir, il appela Esaü, son fils aîné, et lui dit : Mon fils ! Et il lui répondit : Me voici.
\VS{2}Isaac lui dit : Voici, maintenant je suis devenu vieux, et je ne connais pas le jour de ma mort.
\VS{3}Maintenant donc, je te prie, prends tes armes, ton carquois et ton arc, va dans les champs, et chasse-moi du gibier.
\VS{4}Apprête-moi un mets comme j’aime, et apporte-le-moi, afin que je mange, et que mon âme te bénisse avant que je meure.
\VS{5}Or Rebecca écoutait pendant qu'Isaac parlait à Esaü, son fils. Esaü donc s'en alla dans les champs pour chasser du gibier et pour le rapporter.
\VS{6}Et Rebecca parla à Jacob, son fils, et lui dit : Voici, j'ai entendu parler ton père à Esaü, ton frère, disant :
\VS{7}Apporte-moi du gibier, et fais-moi un mets, afin que je le mange et je te bénirai devant Yahweh avant de mourir.
\VS{8}Maintenant donc, mon fils, obéis à ma parole, et fais ce que je vais te commander.
\VS{9}Va maintenant à la bergerie, et prends-moi là deux bons chevreaux parmi les chèvres, et j'en ferai un mets pour ton père comme il aime.
\VS{10}Et tu le porteras à ton père, afin qu'il le mange et qu'il te bénisse avant sa mort.
\VS{11}Jacob répondit à Rebecca sa mère : Voici, Esaü, mon frère, est un homme velu, et je suis un homme sans poil.
\VS{12}Peut-être que mon père me touchera-t-il, et il me regardera comme un homme qui a voulu le tromper, et j'attirerai sur moi sa malédiction et non pas sa bénédiction.
\VS{13}Sa mère lui dit : Mon fils, que la malédiction que tu crains retombe sur moi ! Obéis seulement à ma parole, et va me prendre ce que je t'ai dit.
\TextTitle{Déception d'Esaü\FTNTT{Hé. 12:16-17}}
\VS{14}Jacob alla les prendre et les apporta à sa mère ; et sa mère fit un mets comme son père aimait.
\VS{15}Puis Rebecca prit les plus précieux habits d'Esaü, son fils aîné, qu'elle avait dans la maison, et elle les fit mettre à Jacob, son fils cadet.
\VS{16}Elle couvrit ses mains et son cou, qui étaient sans poil, des peaux des chevreaux.
\VS{17}Puis elle mit entre les mains de son fils Jacob le mets et le pain qu'elle avait apprêtés.
\VS{18}Il vint vers son père, et lui dit : Mon père ! Il répondit : Me voici ; qui es-tu, mon fils ?
\VS{19}Jacob répondit à son père : Je suis Esaü, ton fils aîné ; j'ai fait ce que tu m’as dit. Lève-toi, je te prie, assieds-toi et mange de mon gibier, afin que ton âme me bénisse.
\VS{20}Isaac dit à son fils : Eh quoi ! Tu en as déjà trouvé, mon fils ! Et il dit : Yahweh ton Dieu l'a fait venir devant moi.
\VS{21}Isaac dit à Jacob : Approche-toi, je te prie, mon fils, et que je te touche, afin que je sache si tu es mon fils Esaü ou non.
\VS{22}Jacob donc s'approcha de son père Isaac, qui le toucha et dit : Cette voix est la voix de Jacob, mais ces mains sont les mains d'Esaü.
\VS{23}Et il ne le reconnut pas, car ses mains étaient velues comme les mains de son frère Esaü ; et il le bénit.
\VS{24}Il dit : C’est toi, mon fils Esaü ? Il répondit : Je le suis.
\VS{25}Isaac lui dit : Apporte-moi donc la viande, et que je mange du gibier de mon fils, afin que mon âme te bénisse. Jacob l'apporta, et Isaac mangea ; il lui apporta aussi du vin, et il but.
\VS{26}Puis Isaac, son père, lui dit : Approche-toi, je te prie, et embrasse-moi mon fils.
\VS{27}Jacob s'approcha et l’embrassa. Isaac sentit l'odeur de ses habits, et le bénit en disant : Voici l'odeur de mon fils, comme l'odeur d'un champ que Yahweh a béni.
\VS{28}Que Dieu te donne de la rosée du ciel, et de la graisse de la terre, du blé et du vin en abondance\FTNT{Hé. 11:20.} !
\VS{29}Que des peuples te servent, et que des nations se prosternent devant toi ! Sois le maître de tes frères, et que les fils de ta mère se prosternent devant toi ! Maudit soit quiconque te maudira, et béni soit quiconque te bénira.
\VS{30}Isaac avait fini de bénir Jacob, et Jacob avait à peine quitté son père Isaac, qu’Esaü, son frère, revint de la chasse.
\VS{31}Il apprêta aussi un mets, l’apporta à son père, et lui dit : Que mon père se lève et mange du gibier de son fils, afin que ton âme me bénisse.
\VS{32}Isaac, son père, lui dit : Qui es-tu ? Et il dit : Je suis ton fils, ton fils aîné, Esaü.
\VS{33}Isaac fut saisi d'une grande, d’une violente émotion, et dit : Qui est donc celui qui a chassé du gibier et me l’a apporté ? J'ai mangé de tout avant que tu ne viennes, et je l'ai béni. Aussi sera-t-il béni !
\VS{34}Dès qu'Esaü entendit les paroles de son père, il poussa de forts cris, pleins d’amertume, et il dit à son père : Bénis-moi aussi, bénis-moi, mon père !
\VS{35}Mais il dit : Ton frère est venu avec tromperie, et il a enlevé ta bénédiction.
\VS{36}Esaü dit : N'est-ce pas avec raison qu'on a appelé son nom Jacob ? Car il m'a déjà supplanté deux fois ; il m'a enlevé mon droit d'aînesse, et voici, maintenant il a enlevé ma bénédiction. Puis il dit : Ne m'as-tu point réservé de bénédiction ?
\VS{37}Isaac répondit à Esaü en disant : Voici, je l'ai établi ton maître, et lui ai donné tous ses frères pour serviteurs, et je l'ai pourvu de blé et de vin ; et que ferai-je maintenant pour toi, mon fils ?
\VS{38}Esaü dit à son père : N'as-tu qu'une bénédiction, mon père ? Bénis-moi aussi, bénis-moi, mon père ! Et Esaü éleva la voix et pleura\FTNT{Hé. 12:17.}.
\VS{39}Isaac, son père, répondit, et dit : Voici, ta demeure sera privée de la graisse de la terre, et de la rosée du ciel, d'en haut.
\VS{40}Tu vivras par ton épée, et tu seras asservi à ton frère ; mais il arrivera qu'étant devenu maître, tu briseras son joug de dessus ton cou.
\TextTitle{Fuite de Jacob chez Laban}
\VS{41}Esaü conçut de la haine contre Jacob, à cause de la bénédiction dont son père l'avait béni ; et Esaü dit en son cœur : Les jours du deuil de mon père approchent, et je tuerai Jacob, mon frère.
\VS{42}On rapporta à Rebecca les paroles d'Esaü, son fils aîné ; et elle fit alors appeler Jacob, son fils cadet, et lui dit : Voici, Esaü, ton frère, se console dans l'espérance qu'il a de te tuer.
\VS{43}Maintenant donc, mon fils, obéis à ma parole ! Lève-toi, et enfuis-toi à Charan, vers Laban, mon frère.
\VS{44}Et reste avec lui quelque temps, jusqu'à ce que la fureur de ton frère soit passée ;
\VS{45}jusqu’à ce que la colère de ton frère se détourne de toi, et qu'il oublie ce que tu lui as fait. Pourquoi serais-je privée de vous deux en un même jour ?
\VS{46}Rebecca dit à Isaac : Je suis dégoûtée de la vie, à cause de filles de Heth. Si Jacob prend une femme, comme celles-ci, parmi les filles de Heth, parmi les filles du pays, à quoi me sert la vie ?
\Chap{28}
\TextTitle{A Béthel, Yahweh confirme son alliance à Jacob}
\VerseOne{}Isaac donc appela Jacob, et le bénit, et lui donna cet ordre : Tu ne prendras point de femme parmi les filles de Canaan.
\VS{2}Lève-toi, va à Paddan-Aram, à la maison de Bethuel, père de ta mère, et prends-toi une femme de là, parmi les filles de Laban, frère de ta mère.
\VS{3}Que le Dieu Tout-Puissant te bénisse, te rende fécond et te multiplie, afin que tu deviennes une assemblée de peuples.
\VS{4}Qu’il te donne la bénédiction d'Abraham, à toi et à ta postérité avec toi, afin que tu obtiennes en héritage le pays où tu as été étranger, que Dieu a donné à Abraham.
\VS{5}Isaac donc fit partir Jacob, qui s'en alla à Paddan-Aram, vers Laban, fils de Bethuel, le Syrien, frère de Rebecca, mère de Jacob et d'Esaü.
\VS{6}Esaü vit qu'Isaac avait béni Jacob, et qu'il l'avait envoyé à Paddan-Aram afin qu'il prenne une femme de ce pays-là pour lui, et qu'il lui avait donné cet ordre, quand il le bénissait, disant : Ne prends point de femme parmi les filles de Canaan ;
\VS{7}il vit que Jacob avait obéi à son père et à sa mère, et qu’il était parti à Paddan-Aram.
\VS{8}Esaü comprit ainsi que les filles de Canaan déplaisaient à Isaac, son père.
\VS{9}Et Esaü s’en alla vers Ismaël. Il prit pour femme, outre ses autres femmes, Mahalath, fille d'Ismaël, fils d'Abraham, sœur de Nebajoth.
\VS{10}Jacob partit de Beer-Schéba et s'en alla à Charan.
\VS{11}Il arriva dans un lieu où il passa la nuit, parce que le soleil était couché. Il y prit donc une pierre\FTNT{1 Pi. 2:4. Voir  commentaire en Es. 8:13-15.}, et en fit son chevet, et il se coucha dans ce lieu-là.
\VS{12}Il eut un songe ; et voici, une échelle dressée sur la terre, dont le sommet touchait le ciel. Et voici, les anges de Dieu montaient et descendaient par cette échelle\FTNT{Jn. 1:51.}.
\VS{13}Et voici, Yahweh se tenait sur l'échelle, et il lui dit : Je suis Yahweh, le Dieu d'Abraham, ton père, et le Dieu d'Isaac ; je te donnerai à toi et à ta postérité, la terre sur laquelle tu es couché.
\VS{14}Ta postérité sera comme la poussière de la terre, et tu t'étendras à l'occident et à l'orient, au nord et au midi, et toutes les familles de la terre seront bénies en toi et en ta postérité.
\VS{15}Voici, je suis avec toi ; et je te garderai partout où tu iras ; et je te ramènerai dans ce pays ; car je ne t'abandonnerai point que je n'aie exécuté ce que je t'ai dit.
\VS{16}Et quand Jacob fut réveillé de son sommeil, il dit : Certainement, Yahweh est en ce lieu-ci, et moi, je ne le savais pas !
\VS{17}Il eut peur et dit : Que ce lieu-ci est effrayant ! C'est ici la maison de Dieu, et c'est ici la porte des cieux !
\VS{18}Et Jacob se leva de bon matin, prit la pierre dont il avait fait son chevet, il la dressa pour monument, et versa de l'huile sur son sommet.
\VS{19}Il donna à ce lieu le nom de Béthel ; mais auparavant la ville s'appelait Luz.
\VS{20}Jacob fit un vœu en disant : Si Dieu est avec moi, et s'il me garde pendant le voyage que je fais, s'il me donne du pain à manger, et des habits pour me vêtir,
\VS{21}et si je retourne en paix à la maison de mon père, certainement Yahweh sera mon Dieu.
\VS{22}Cette pierre que j'ai dressée pour monument sera la maison de Dieu ; et de tout ce que tu m'auras donné, je t'en donnerai entièrement la dîme\FTNT{Voir commentaire sur la dîme en No. 18:21 et Mal. 3:10.}.
\Chap{29}
\TextTitle{Jacob épouse Léa et Rachel chez Laban}
\VerseOne{}Jacob donc se mit en chemin, et s'en alla au pays des fils de l’orient.
\VS{2}Il regarda. Et voici, il y avait un puits dans un champ ; et voici il y avait à côté trois troupeaux de brebis couchées près du puits, car c’était à ce puits qu’on abreuvait les troupeaux.  Et il y avait une grosse pierre sur l'ouverture du puits.
\VS{3}Tous les troupeaux se rassemblaient là ; on roulait la pierre de dessus l'ouverture du puits, et on abreuvait les troupeaux ; et ensuite on remettait la pierre à sa place, sur l'ouverture du puits.
\VS{4}Jacob leur dit : Mes frères, d'où êtes-vous ? Ils répondirent : Nous sommes de Charan.
\VS{5}Il leur dit : Connaissez-vous Laban, fils de Nachor ? Ils répondirent : Nous le connaissons.
\VS{6}Il leur dit : Se porte-t-il bien ? Ils lui répondirent : Il se porte bien ; et voici Rachel, sa fille, qui vient avec le troupeau.
\VS{7}Il dit : Voici, il est encore grand jour, et il n'est pas temps de rassembler les troupeaux ; abreuvez les brebis, puis allez et faites-les paître.
\VS{8}Ils répondirent : Nous ne le pouvons pas, jusqu'à ce que tous les troupeaux soient rassemblés et qu'on ait ôté la pierre de dessus l'ouverture du puits, afin d'abreuver les troupeaux.
\VS{9}Comme il parlait encore avec eux, Rachel arriva avec le troupeau de son père ; car elle était bergère.
\VS{10}Lorsque Jacob vit Rachel, fille de Laban, frère de sa mère, et le troupeau de Laban, frère de sa mère, il s'approcha et roula la pierre de dessus l'ouverture du puits, et abreuva le troupeau de Laban, frère de sa mère.
\VS{11}Et Jacob embrassa Rachel, et il éleva sa voix et pleura.
\VS{12}Jacob apprit à Rachel qu'il était frère de son père, et qu'il était fils de Rebecca ; et elle courut le rapporter à son père.
\VS{13}Dès que Laban eut entendu parler de Jacob, fils de sa soeur, il courut au-devant de lui, il le prit dans ses bras et l’embrassa, et il le fit venir dans sa maison ; et Jacob raconta à Laban tout ce qui lui était arrivé.
\VS{14}Et Laban lui dit : Certainement, tu es mon os et ma chair. Jacob demeura un mois entier chez Laban.
\VS{15}Puis Laban dit à Jacob : Me serviras-tu pour rien parce que tu es mon frère ? Dis-moi quel sera ton salaire ?
\VS{16}Or Laban avait deux filles : L'aînée s'appelait Léa, et la cadette Rachel.
\VS{17}Léa avait les yeux délicats, mais Rachel était belle de taille et belle de figure.
\VS{18}Jacob aimait Rachel, et il dit : Je te servirai sept ans pour Rachel, ta cadette.
\VS{19}Et Laban répondit : Il vaut mieux que je te la donne que de la donner à un autre homme ; demeure avec moi.
\VS{20}Ainsi Jacob servit sept années pour Rachel ; et elles furent à ses yeux comme quelques jours, parce qu'il l'aimait.
\VS{21}Et Jacob dit à Laban : Donne-moi ma femme, car mon temps est accompli, et j’irai vers elle.
\VS{22}Laban réunit tous les gens du lieu et fit un festin.
\VS{23}Mais quand le soir fut venu, il prit Léa, sa fille, et l'amena vers Jacob qui s’approcha d’elle.
\VS{24}Et Laban donna Zilpa, sa servante, à Léa, sa fille, pour servante.
\VS{25}Le lendemain matin, voilà que c'était Léa. Alors Jacob dit à Laban : Qu'est-ce que tu m'as fait ? N'ai-je pas servi chez toi pour Rachel ? Et pourquoi m'as-tu trompé ?
\VS{26}Laban répondit : On ne fait pas ainsi dans ce lieu de donner la plus jeune avant l'aînée.
\VS{27}Achève la semaine avec celle-ci, et nous te donnerons aussi l'autre, pour le service que tu feras encore chez moi sept autres années.
\VS{28}Jacob donc fit ainsi, et il acheva la semaine avec Léa ; et Laban lui donna aussi pour femme Rachel, sa fille.
\VS{29}Et Laban donna Bilha, sa servante, à Rachel, sa fille, pour servante.
\VS{30}Jacob alla aussi vers Rachel, et il aima Rachel plus que Léa ; et il servit encore chez Laban sept autres années.
\VS{31}Yahweh vit que Léa était haïe, et il ouvrit sa matrice, tandis que Rachel était stérile.
\TextTitle{Les enfants de Jacob}
\VS{32}Léa conçut et enfanta un fils à qui elle donna le nom de Ruben, car elle dit : C'est parce que Yahweh a vu mon affliction, et maintenant mon mari m'aimera.
\VS{33}Elle conçut encore et enfanta un fils, et elle dit : Parce que Yahweh a entendu que j'étais haïe, il m'a aussi donné celui-ci. Et elle lui donna le nom de Siméon.
\VS{34}Elle conçut encore et enfanta un fils, et elle dit : Maintenant mon mari s'attachera à moi, car je lui ai enfanté trois fils. C'est pourquoi on lui donna le nom de Lévi.
\VS{35}Elle conçut encore et enfanta un fils, et elle dit : Cette fois je louerai Yahweh. C'est pourquoi elle lui donna le nom de Juda. Et elle cessa d'avoir des enfants.
\Chap{30}
\TextTitle{Les enfants de Jacob (suite)}
\VerseOne{}Alors Rachel, voyant qu’elle ne donnait point d'enfants à Jacob, fut jalouse de Léa, sa sœur, et elle dit à Jacob : Donne-moi des enfants, autrement je meurs !
\VS{2}La colère de Jacob s’enflamma contre Rachel, et il dit : Suis-je à la place de Dieu pour t’empêcher d'avoir des enfants ?
\VS{3}Elle dit : Voici ma servante Bilha ; va vers elle ; qu’elle enfante sur mes genoux, et que j’aie des fils par elle.
\VS{4}Et elle lui donna pour femme Bilha, sa servante, et Jacob alla vers elle.
\VS{5}Bilha conçut et enfanta un fils à Jacob.
\VS{6}Rachel dit : Dieu a jugé en ma faveur, et il a aussi exaucé ma voix, et m'a donné un fils ; c'est pourquoi elle l’appela du nom de Dan.
\VS{7}Bilha, servante de Rachel, conçut encore et enfanta un second fils à Jacob.
\VS{8}Rachel dit : J'ai fortement lutté contre ma sœur, aussi j'ai eu la victoire ; c'est pourquoi elle l’appela du nom de Nephthali.
\VS{9}Alors Léa, voyant qu'elle avait cessé de faire des enfants, prit Zilpa, sa servante, et la donna pour femme à Jacob.
\VS{10}Zilpa, servante de Léa, enfanta un fils à Jacob.
\VS{11}Léa dit : Le bonheur est arrivé, c'est pourquoi elle l’appela du nom de Gad.
\VS{12}Zilpa, servante de Léa, enfanta un second fils à Jacob.
\VS{13}Léa dit : C'est pour me rendre heureuse, car les filles me diront bienheureuse ; c'est pourquoi elle l’appela du nom d’Aser.
\VS{14}Ruben sortit au temps de la moisson des blés, trouva des mandragores\FTNT{La mandragore, appelée pomme d'amour, était utilisée comme excitant du désir sexuel ainsi que pour favoriser la procréation. On attribuait à cette plante aux propriétés hallucinogènes des vertus magiques.} aux champs, et les apporta à Léa, sa mère ; et Rachel dit à Léa : Donne-moi, je te prie, des mandragores de ton fils.
\VS{15}Elle lui répondit : Est-ce peu que tu aies pris mon mari, pour que tu prennes aussi les mandragores de mon fils ? Et Rachel dit : Qu'il couche donc cette nuit avec toi pour les mandragores de ton fils.
\VS{16}Le soir, comme Jacob revenait des champs, Léa sortit au-devant de lui et lui dit : Tu viendras vers moi, car je t'ai acheté pour les mandragores de mon fils ; et il coucha avec elle cette nuit-là.
\VS{17}Dieu exauça Léa, et elle conçut et enfanta à Jacob un cinquième fils.
\VS{18}Léa dit : Dieu m'a récompensée, parce que j'ai donné ma servante à mon mari ; c'est pourquoi elle l’appela du nom d’Issacar.
\VS{19}Léa conçut encore et enfanta un sixième fils à Jacob.
\VS{20}Léa dit : Dieu m'a donné un beau don ; maintenant mon mari habitera avec moi, car je lui ai enfanté six fils ; c'est pourquoi elle l’appela du nom de Zabulon.
\VS{21}Puis elle enfanta une fille et la nomma Dina.
\VS{22}Dieu se souvint de Rachel, il l’exauça et il ouvrit sa matrice.
\VS{23}Alors elle conçut et enfanta un fils, et elle dit : Dieu a ôté mon opprobre.
\VS{24}Et elle lui donna le nom de Joseph, en disant : Que Yahweh m'ajoute un autre fils !
\TextTitle{Jacob devient de plus en plus riche}
\VS{25}Lorsque Rachel eut enfanté Joseph, Jacob dit à Laban : Laisse-moi partir, pour que je m’en aille chez moi, dans mon pays.
\VS{26}Donne-moi mes femmes et mes enfants, pour lesquels je t'ai servi, et je m'en irai ; car tu sais de quelle manière je t'ai servi.
\VS{27}Laban lui répondit : Ecoute, je te prie, si j'ai trouvé grâce à tes yeux ; j’ai deviné que Yahweh m'a béni à cause de toi.
\VS{28}Il lui dit aussi : Fixe-moi le salaire que tu veux, et je te le donnerai.
\VS{29}Jacob lui répondit : Tu sais comment je t'ai servi et ce qu'est devenu ton bétail avec moi.
\VS{30}Car le peu que tu avais avant que je vienne s’est beaucoup accru, et Yahweh t'a béni depuis que j’ai mis mes pieds chez toi. Et maintenant, quand ferai-je aussi quelque chose pour ma maison ?
\VS{31}Laban lui dit : Que te donnerai-je ? Et Jacob répondit : Tu ne me donneras rien ; mais je ferai paître encore tes troupeaux, et je les garderai, si tu consens à ce que je vais te dire.
\VS{32}Je parcourrai aujourd'hui tes troupeaux, mets à part parmi toutes les brebis tachetées et marquetées, et tous les agneaux noirs, et les chèvres marquetées et tachetées. Ce sera mon salaire.
\VS{33}Ma justice me rendra témoignage à l’avenir devant toi ; quand tu viendras reconnaître mon salaire, en ta présence ; et tout ce qui ne sera pas marqueté ou tacheté parmi les chèvres, et noirs parmi les agneaux, sera considéré comme un vol s'il est trouvé chez moi.
\VS{34}Laban dit : Voici, qu'il te soit fait comme tu l'as dit.
\VS{35}Ce même jour, il sépara les boucs rayés et marquetés, et toutes les chèvres tachetées et marquetées, toutes celles où il y avait du blanc, et tous les agneaux noirs. Il les remit entre les mains de ses fils.
\VS{36}Puis il mit l'espace de trois journées de chemin entre lui et Jacob ; et Jacob fit paître le reste des troupeaux de Laban.
\VS{37}Mais Jacob prit des branches vertes de peuplier, d’amandier et de platane ; il y pela des bandes blanches,  mettant à nu le blanc qui était sur les branches.
\VS{38}Puis il plaça les branches qu’il avait pelées dans les auges, dans les abreuvoirs, sous les yeux des brebis qui venaient boire, et elles entraient en chaleur quand elles venaient boire.
\VS{39}Les brebis entraient en chaleur près des branches, et elles faisaient des brebis rayées, tachetées et marquetées.
\VS{40}Jacob séparait les agneaux, et il mettait ensemble ce qui était rayé et tout ce qui était noir dans les troupeaux de Laban. Il se fit ainsi des troupeaux à part, qu’il ne réunit point aux troupeaux de Laban.
\VS{41}Toutes les fois que les brebis vigoureuses entraient en chaleur, Jacob mettait les branches dans les auges sous les yeux des brebis, afin qu'elles entrent en chaleur près des  branches.
\VS{42}Mais pour les brebis chétives, il ne les mettait point ; de sorte que les chétives appartenaient à Laban et les vigoureuses à Jacob.
\VS{43}Ainsi cet homme devint de plus en plus riche ; il eut du menu bétail en abondance, des servantes et des serviteurs, des chameaux et des ânes.
\Chap{31}
\TextTitle{Yahweh demande à Jacob de rentrer dans la terre de se pères}
\VerseOne{}Or Jacob entendit les discours des fils de Laban qui disaient : Jacob a pris tout ce qui appartenait à notre père, et c’est de ce qui était à notre père qu’il s’est acquis toute cette richesse.
\VS{2}Jacob regarda le visage de Laban, et voici, il n'était plus à son égard comme auparavant.
\VS{3}Alors Yahweh dit à Jacob : Retourne au pays de tes pères et vers ta parenté, et je serai avec toi.
\VS{4}Jacob fit appeler Rachel et Léa qui étaient aux champs vers son troupeau,
\VS{5}et leur dit : Je vois au visage de votre père qu'il n'est plus envers moi comme il était auparavant ; toutefois le Dieu de mon père a été avec moi.
\VS{6}Vous savez que j'ai servi votre père de tout mon pouvoir.
\VS{7}Mais votre père s'est moqué de moi et a changé dix fois mon salaire ; mais Dieu ne lui a pas permis de me faire du mal.
\VS{8}Quand il disait : Les tachetées seront ton salaire, alors toutes les brebis faisaient des agneaux tachetés ; et quand il disait : Les marquetées seront ton salaire, alors toutes les brebis faisaient des agneaux marquetés.
\VS{9}Ainsi Dieu a ôté à votre père son bétail et me l'a donné.
\VS{10}Au temps où les brebis entraient en chaleur, je levai mes yeux et vis en songe que les boucs qui couvraient les brebis étaient rayés, tachetés, et marquetés.
\VS{11}Et l'Ange de Dieu\FTNT{Gn. 7:7} me dit en songe : Jacob ! Et je répondis : Me voici.
\VS{12}Il dit : Lève maintenant tes yeux et regarde : Tous les boucs qui couvrent les brebis sont rayés, tachetés et marquetés, car j'ai vu tout ce que te fait Laban.
\VS{13}Je suis le Dieu de Béthel, où tu oignis la pierre que tu dressas pour monument, où tu me fis un vœu.  Maintenant lève-toi, sors de ce pays, et retourne au pays de ta naissance.
\TextTitle{Jacob fuit de chez Laban avec sa famille}
\VS{14}Alors Rachel et Léa lui répondirent et dirent : Avons-nous encore quelque portion et quelque héritage dans la maison de notre père ?
\VS{15}Ne nous a-t-il pas traitées comme des étrangères ? Car il nous a vendues, et même il a entièrement mangé notre argent.
\VS{16}Car toutes les richesses que Dieu a ôtées à notre père nous appartenaient ainsi qu’à nos enfants. Maintenant donc fais tout ce que Dieu t'a dit.
\VS{17}Ainsi Jacob se leva, et fit monter ses enfants et ses femmes sur des chameaux.
\VS{18}Il emmena tout son bétail et tous les biens qu'il avait acquis, et tout ce qu'il possédait et qu'il avait acquis à Paddan-Aram, pour aller vers Isaac, son père, au pays de Canaan.
\VS{19}Or comme Laban était allé tondre ses brebis, Rachel déroba les théraphim de son père\FTNT{Les théraphim étaient des idoles utilisées dans un sanctuaire de maison ou dans un lieu de culte. Voir Jg. 18:14 ; 2 R. 23:24.}.
\VS{20}Et Jacob trompa Laban, le Syrien, en ne l’avertissant pas de son dessein, parce qu'il s'enfuyait.
\VS{21}Il s'enfuit avec tout ce qui lui appartenait ; il se leva, passa le fleuve, et se dirigea vers la montagne de Galaad.
\VS{22}Le troisième jour, on rapporta à Laban que Jacob s’était enfui.
\VS{23}Alors il prit avec lui ses frères, et il le poursuivit sept journées de marche, et l'atteignit à la montagne de Galaad.
\VS{24}Mais Dieu apparut à Laban, le Syrien, en songe la nuit, et lui dit : Garde-toi de parler à Jacob ni en bien ni en mal.
\VS{25}Laban donc atteignit Jacob. Jacob avait dressé ses tentes sur la montagne ; et Laban dressa aussi les siennes avec ses frères sur la montagne de Galaad.
\VS{26}Et Laban dit à Jacob : Qu'as-tu fait ? Tu m’as trompé, tu as emmené mes filles comme des prisonnières de guerre.
\VS{27}Pourquoi as-tu pris la fuite secrètement, m’as-tu trompé et ne m’as-tu pas averti ? Car je t'aurais laissé partir avec joie et avec des chansons, au son des tambours et des violons.
\VS{28}Tu ne m'as pas laissé embrasser mes fils et mes filles ! C’est en insensé que tu as agi.
\VS{29}J'ai en main le pouvoir de vous faire du mal, mais le Dieu de votre père m'a parlé la nuit passée et m'a dit : Garde-toi de ne parler à Jacob ni en bien ni en mal.
\VS{30}Maintenant que tu es parti, parce que tu  languissais après la maison de ton père, pourquoi as-tu dérobé mes dieux ?
\VS{31}Jacob répondit et dit à Laban : Je me suis enfui parce que je craignais ; car je me disais qu'il fallait prendre garde que tu ne me ravisses tes filles.
\VS{32}Mais celui chez qui tu trouveras tes dieux ne vivra point. En présence de nos frères, examine  s'il y a chez moi quelque chose qui t'appartienne, et prends-le ; car Jacob ignorait que Rachel les avait dérobés.
\VS{33}Alors Laban entra dans la tente de Jacob, et dans celle de Léa, et dans la tente des deux servantes, et il ne les trouva point ; et étant sorti de la tente de Léa, il entra dans la tente de Rachel.
\VS{34}Mais Rachel avait prit les théraphim et les avait mis dans le bât d'un chameau, et s’était assise dessus ; et Laban fouilla toute la tente et ne les trouva point.
\VS{35}Elle dit à son père : Que mon seigneur ne se fâche point de ce que je ne puis me lever devant lui, car j'ai ce que les femmes ont coutume d'avoir ; et il fouilla, mais il ne trouva point les théraphim.
\VS{36}Jacob se mit en colère et querella Laban. Il reprit la parole et lui dit : Quel est mon crime ? Quel est mon péché, pour que tu me poursuives avec tant d’ardeur ?
\VS{37}Car tu as fouillé tous mes effets, qu'as-tu trouvé des effets de ta maison ? Mets-les ici devant mes frères et les tiens, et qu'ils soient juges entre nous deux.
\VS{38}Voilà vingt ans que j’ai passés chez toi ; tes brebis et tes chèvres n'ont point avorté, je n'ai point mangé les moutons de tes troupeaux.
\VS{39}Je ne t'ai point rapporté de bêtes déchirées par les bêtes sauvages, j'en ai moi-même subi la perte ; et tu redemandais de ma main ce qui avait été dérobé de jour et ce qui avait été dérobé de nuit.
\VS{40}Le jour la chaleur me consumait, et la nuit le froid ; et le sommeil fuyait de mes yeux.
\VS{41}Voilà vingt ans que j’ai passés dans ta maison, quatorze ans pour tes deux filles, et six ans pour tes troupeaux, et tu m'as changé dix fois mon salaire.
\VS{42}Si je n’avais pas eu pour moi le Dieu de mon père, le Dieu d'Abraham, et celui que craint Isaac, certes tu m’aurais maintenant renvoyé à vide. Mais Dieu a regardé mon affliction et le travail de mes mains, et il t'a repris la nuit passée.
\VS{43}Laban répondit à Jacob et dit : Ces filles sont mes filles, et ces enfants sont mes enfants, et ces troupeaux sont mes troupeaux, et tout ce que tu vois est à moi ; et que ferais-je aujourd'hui à mes filles et aux enfants qu'elles ont enfantés ?
\VS{44}Maintenant donc, viens, faisons ensemble une alliance, et qu’elle serve de témoignage entre moi et toi.
\VS{45}Jacob prit une pierre et il la dressa pour monument.
\VS{46}Jacob dit à ses frères : Ramassez des pierres. Et ils prirent des pierres et ils en firent un monceau, et ils mangèrent là sur ce monceau.
\VS{47}Laban l'appela Jegar-Sahadutha, et Jacob l'appela Galed.
\VS{48}Et Laban dit : Ce monceau sera aujourd'hui témoin entre moi et toi ; c'est pourquoi il fut nommé Galed (poste d’observation).
\VS{49}Il fut aussi appelé Mitspa ; parce que Laban dit : Que Yahweh veille sur moi et sur toi, quand nous nous serons l'un et l'autre perdus de vue.
\VS{50}Si tu maltraites mes filles et si tu prends une autre femme que mes filles, ce n’est pas un homme qui sera témoin entre nous, prends-y garde ; c'est Dieu qui est témoin entre moi et toi.
\VS{51}Laban dit encore à Jacob : Regarde ce monceau, et considère le monument que j'ai dressé entre moi et toi.
\VS{52}Que ce monceau soit témoin et que ce monument soit témoin que je n’irai pas vers toi au-delà de ce monceau, et que tu ne viendras pas vers moi au-delà de ce monceau et de ce monument pour me faire du mal.
\VS{53}Que le Dieu d'Abraham et le Dieu de Nachor, le Dieu de leur père, juge entre nous ; mais Jacob jura par celui que craignait Isaac, son père.
\VS{54}Jacob offrit un sacrifice sur la montagne et invita ses frères pour manger du pain ; ils mangèrent donc du pain et passèrent la nuit sur la montagne.
\VS{55}Laban se leva de bon matin, embrassa ses fils et ses filles, et les bénit. Ensuite il s'en alla. Ainsi Laban retourna chez lui.
\Chap{32}
\TextTitle{Jacob devient Israël}
\VerseOne{}Et Jacob continua son chemin, et des anges de Dieu le rencontrèrent.
\VS{2}En les voyant, Jacob dit : C'est ici le camp de Dieu ! Et il donna à ce lieu le nom de Mahanaïm.
\VS{3}Jacob envoya devant lui des messagers vers Esaü, son frère, au pays de Séir, dans le territoire d'Edom.
\VS{4}Il leur donna cet ordre : Vous parlerez de cette manière à mon seigneur Esaü : Ainsi a dit ton serviteur Jacob : J'ai séjourné comme étranger chez Laban, et j’y ai habité jusqu'à présent ;
\VS{5}j’ai des bœufs, des ânes, des brebis, des serviteurs, et des servantes ; et j'envoie l’annoncer à mon seigneur, afin de trouver grâce à  tes yeux.
\VS{6}Et les messagers revinrent auprès de Jacob et lui dirent : Nous sommes allés vers ton frère Esaü, et il marche aussi à ta rencontre avec quatre cents hommes.
\VS{7}Alors Jacob fut très effrayé et rempli d’angoisse ; et il partagea le peuple qui était avec lui, et les brebis, et les boeufs, et les chameaux, en deux camps ; et  il dit :
\VS{8}Si Esaü attaque l'un des camps et le frappe, le camp qui restera pourra s’échapper.
\VS{9}Jacob dit aussi : Ô Dieu de mon père Abraham, Dieu de mon père Isaac, ô Yahweh qui m'as dit : Retourne dans ton pays, et vers ta parenté, et je te ferai du bien.
\VS{10}Je suis trop petit pour toutes les faveurs et pour toute la fidélité dont tu as usé envers ton serviteur ; car j'ai passé ce Jourdain avec mon bâton, et maintenant je forme deux camps.
\VS{11}Je te prie, délivre-moi de la main de mon frère Esaü ; car je crains qu'il ne vienne, et qu'il ne me frappe, et qu'il ne tue la mère avec les enfants.
\VS{12}Et toi, tu as dit : Certes, je te ferai du bien, et je rendrai ta postérité comme le sable de la mer, si abondant qu’on ne saurait le compter.
\VS{13}C’est dans ce lieu-là que Jacob passa la nuit.  Il prit de ce qu’il avait sous la main pour faire un présent à Esaü, son frère :
\VS{14}à savoir deux cents chèvres, vingt boucs, deux cents brebis et vingt béliers.
\VS{15}Trente femelles de chameaux qui allaitaient, et leurs petits ; quarante jeunes vaches, dix jeunes taureaux, vingt ânesses et dix ânes.
\VS{16}Il les mit entre les mains de ses serviteurs, chaque troupeau à part, et leur dit : Passez devant moi, et faites qu'il y ait un intervalle entre chaque troupeau.
\VS{17}Il donna cet ordre au premier, disant : Quand Esaü, mon frère, te rencontrera et te demandera, disant : A qui es-tu ? Et où vas-tu ? Et à qui sont ces choses qui sont devant toi ?
\VS{18}Alors tu diras : Je suis à ton serviteur Jacob ; c'est un présent qu'il envoie à mon seigneur Esaü ; et voici, il vient lui-même derrière nous.
\VS{19}Il donna le même ordre au deuxième, au troisième, et à tous ceux qui suivaient les troupeaux, disant : C’est ainsi que vous parlerez à mon seigneur Esaü, quand vous le rencontrerez.
\VS{20}Vous lui direz : Voici, ton serviteur Jacob vient aussi derrière nous. Car il se disait : J'apaiserai sa colère par ce présent qui va devant moi, et après cela, je verrai sa face ; peut-être qu'il me regardera favorablement.
\VS{21}Le présent passa devant lui ; mais il resta cette nuit-là dans le camp.
\VS{22}Il se leva cette nuit, et prit ses deux femmes, ses deux servantes, et ses onze enfants, et passa le gué de Jabbok.
\VS{23}Il les prit donc, et leur fit passer le torrent ; il fit aussi passer tout ce qu'il avait.
\VS{24}Jacob demeura seul. Alors un homme lutta avec lui jusqu'au lever de l’aurore.
\VS{25}Et quand cet homme vit qu'il ne pouvait pas le vaincre, il frappa à l'emboîture de la hanche de Jacob ; ainsi l'emboîture de l'os de la hanche de Jacob se démit pendant qu’il luttait avec lui.
\VS{26}Et cet homme lui dit : Laisse-moi, car l'aube du jour est levée. Mais il dit : Je ne te laisserai point que tu ne m'aies béni.
\VS{27}Cet homme lui dit : Quel est ton nom ? Il répondit : Jacob.
\VS{28}Alors il dit : Ton nom ne sera plus Jacob, mais tu seras appelé Israël ; car tu as été le vainqueur en luttant avec Dieu et avec les hommes, et tu as été le plus fort.
\VS{29}Jacob l’interrogea en disant : Je te prie, déclare-moi ton nom. Et il répondit : Pourquoi demandes-tu mon nom ? Et il le bénit là\FTNT{Jg. 13:18.}.
\VS{30}Jacob appela ce lieu du nom de Peniel ; car, dit-il, j’ai vu Dieu face à face, et mon âme a été délivrée.
\VS{31}Le soleil se levait lorsqu’il passa Peniel. Jacob boitait de la hanche.
\VS{32}C'est pourquoi, jusqu'à ce jour, les enfants d'Israël ne mangent point le tendon qui est à l’emboîture de la hanche ; parce que Dieu frappa Jacob à l'emboîture de la hanche, au tendon.
\Chap{33}
\TextTitle{Jacob demande pardon à son frère Esaü}
\VerseOne{}Et Jacob leva ses yeux et regarda ; et voici, Esaü arrivait avec quatre cents hommes. Et Jacob répartit les enfants entre Léa, Rachel, et les deux servantes.
\VS{2}Il plaça en tête les servantes avec leurs enfants ; Léa et ses enfants ensuite ; et Rachel et Joseph au dernier rang.
\VS{3}Quant à lui,  il passa devant eux et se prosterna à terre sept fois, jusqu'à ce qu'il soit près de son frère.
\VS{4}Esaü courut à sa rencontre ; il le prit dans ses bras, se jeta sur son cou, et l’embrassa. Et ils pleurèrent.
\VS{5}Esaü leva ses yeux, vit les femmes et les enfants, et dit : Qui sont ceux-là ? Sont-ils à toi ? Jacob lui répondit : Ce sont les enfants que Dieu, par sa grâce, a donnés à ton serviteur.
\VS{6}Les servantes s'approchèrent, elles et leurs enfants, et se prosternèrent.
\VS{7}Puis Léa aussi s'approcha avec ses enfants, et ils se prosternèrent, et ensuite Joseph et Rachel s'approchèrent et se prosternèrent aussi.
\VS{8}Esaü dit : Que veux-tu faire avec tout ce camp que j'ai rencontré ? Et Jacob répondit : C'est pour trouver grâce aux yeux de mon seigneur.
\VS{9}Esaü dit : Je suis dans l’abondance, mon frère ; garde ce qui est à toi.
\VS{10}Et Jacob répondit : Non, je te prie, si j'ai maintenant trouvé grâce à tes yeux, reçois ce présent de ma main ; parce que j'ai vu ta face comme si j'avais vu la face de Dieu, et parce que tu m’as accueilli favorablement.
\VS{11}Accepte, je te prie, mon présent qui t'a été offert ; car Dieu m’a comblé de grâce, et je ne manque de rien. Il le pressa tant qu'il le prit.
\VS{12}Esaü dit : Partons et marchons, et je marcherai devant toi.
\VS{13}Mais Jacob lui dit : Mon seigneur sait que ces enfants sont jeunes et que j’ai des brebis et des vaches qui allaitent ; si l’on forçait leur marche un seul jour, tout le troupeau mourra.
\VS{14}Je te prie que mon seigneur passe devant son serviteur, et je m’avancerai tout doucement, au pas de ce bétail qui est devant moi, et au pas de ces enfants, jusqu'à ce que j'arrive chez mon seigneur à Séir.
\VS{15}Esaü dit : Je te prie, je vais au moins laisser avec toi une partie de ce peuple qui est avec moi ; et il répondit : Pourquoi cela ? Je te prie que je trouve grâce aux yeux de mon seigneur.
\VS{16}Ainsi Esaü  retourna ce jour-là par son chemin à Séir.
\TextTitle{Jacob dresse un autel à El-Elohé-Israël (Dieu Fort, Dieu d'Israël)}
\VS{17}Jacob partit pour Succoth. Il bâtit une maison pour lui, et il fit des cabanes pour son bétail. C’est pourquoi il appela ce lieu du nom de Succoth.
\VS{18}A son retour de  Paddan-Aram, Jacob arriva sain et sauf à la ville de Sichem, dans le pays de Canaan, et il campa devant la ville.
\VS{19}Il acheta une portion du champ où il avait dressé sa tente de la main des fils de Hamor, père de Sichem, pour cent pièces d'argent.
\VS{20}Et là, il dressa un autel qu'il appela El-Elohé-Israël (le Dieu Fort, le Dieu d'Israël).
\Chap{34}
\TextTitle{Déshonneur de Dina et vengeance de ses frères}
\VerseOne{}Or Dina, la fille que Léa avait enfantée à Jacob, sortit pour voir les filles du pays.
\VS{2}Elle fut aperçue de Sichem, fils de Hamor, le Hévien, prince du pays. Il l'enleva et coucha avec elle, et la déshonora.
\VS{3}Son cœur fut attaché à Dina, fille de Jacob ; il aima la jeune fille et sut parler au cœur de la jeune fille.
\VS{4}Et Sichem parla à Hamor, son père, en disant : Prends-moi cette fille pour femme.
\VS{5}Jacob apprit qu'il avait déshonoré Dina, sa fille. Or ses fils étaient avec son bétail aux champs ; Jacob garda le silence jusqu’à leur retour.
\VS{6}Hamor, père de Sichem, sortit vers Jacob pour lui  parler.
\VS{7}Et les fils de Jacob revinrent des champs dès qu’ils apprirent ce qui était arrivé ; ces hommes furent dans  une grande douleur, et furent fort irrités de l'infamie que Sichem avait commise contre Israël, en couchant avec la fille de Jacob, ce qui ne devait point se faire.
\VS{8}Hamor leur parla en disant : L’âme de Sichem, mon fils, s’est attachée à votre fille ; donnez-la-lui je vous prie pour femme.
\VS{9}Alliez-vous avec nous, vous nous donnerez vos filles, et vous prendrez pour vous les nôtres.
\VS{10}Vous habiterez avec nous, et le pays sera à votre disposition ; restez pour y trafiquer et y acquérir des possessions.
\VS{11}Sichem dit aussi au père et aux frères de la fille : Que je trouve grâce à vos yeux, et je donnerai tout ce que vous me direz.
\VS{12}Exigez de moi une forte dot, et beaucoup de présents que vous voudrez, et je les donnerai comme vous me direz ; et donnez-moi la jeune fille pour femme.
\VS{13}Alors les fils de Jacob répondirent avec ruse à Sichem et à Hamor, son père ; ils parlèrent ainsi parce que Sichem avait déshonoré Dina, leur sœur.
\VS{14}Ils leur dirent : C’est une chose que nous ne pouvons pas faire, que de donner notre sœur à un homme incirconcis, car ce serait un opprobre pour nous.
\VS{15}Mais nous ne consentirons à ce que vous demandez que si vous deveniez semblables à nous en circoncisant tous les mâles qui sont parmi vous.
\VS{16}Alors nous vous donnerons nos filles, et nous prendrons vos filles pour nous, et nous habiterons avec vous, et nous ne serons qu'un seul peuple.
\VS{17}Mais si vous ne voulez pas nous écouter et vous circoncire, nous prendrons notre fille et nous nous en irons.
\VS{18}Leurs discours plurent à Hamor et à Sichem, fils d'Hamor.
\VS{19}Le jeune homme ne tarda point à faire ce qu'on lui avait proposé, car la fille de Jacob lui plaisait beaucoup ; et il était le plus considéré de tous ceux de la maison de son père.
\VS{20}Hamor et Sichem, son fils, se rendirent à la porte de leur ville et parlèrent aux gens de leur ville en leur disant :
\VS{21}Ces hommes sont paisibles à notre égard ; qu'ils habitent dans le pays et qu'ils y trafiquent ; car voici, le pays est assez vaste pour eux. Nous prendrons pour femmes leurs filles, et nous leur donnerons nos filles.
\VS{22}Mais ces hommes ne consentiront à habiter avec nous, pour former un seul peuple, que si tout mâle qui est parmi nous est circoncis, comme ils sont eux-mêmes circoncis.
\VS{23}Leur bétail, et leurs biens, et toutes leurs bêtes, ne seront-ils pas à nous ? Accordons-leur seulement cela, et qu'ils demeurent avec nous.
\VS{24}Tous ceux qui sortaient par la porte de leur ville obéirent à Hamor et à Sichem, son fils ; et tout mâle d'entre tous ceux qui sortaient par la porte de leur ville fut circoncis.
\VS{25}Le troisième jour, pendant qu’ils étaient souffrants, deux des fils de Jacob, Siméon et Lévi, frères de Dina, prirent leurs épées, entrèrent hardiment dans la ville et tuèrent tous les mâles.
\VS{26}Ils passèrent aussi au tranchant de l'épée Hamor et Sichem, son fils ; ils enlevèrent Dina de la maison de Sichem, et sortirent.
\VS{27}Les fils de Jacob se jetèrent sur les morts et pillèrent la ville, parce qu'on avait déshonoré leur sœur.
\VS{28}Ils prirent leurs troupeaux, leurs bœufs, leurs ânes, et ce qui était dans la ville et dans les champs ;
\VS{29}et toutes leurs richesses, leurs petits enfants, et ils emmenèrent prisonnières leurs femmes ; et ils les pillèrent avec tout ce qui était dans les maisons.
\VS{30}Alors Jacob dit à Siméon et Lévi : Vous m'avez troublé en me rendant odieux aux habitants du pays, aux Cananéens et aux Phérésiens, et je n'ai qu’un petit nombre d’hommes ; ils s'assembleront contre moi, et me frapperont, et me détruiront, moi et ma maison.
\VS{31}Ils répondirent : Doit-on traiter notre sœur comme une prostituée ?
\Chap{35}
\TextTitle{Jacob revient à Béthel pour adorer Yahweh}
\VerseOne{}Or Dieu dit à Jacob : Lève-toi, monte à Béthel, et demeures-y ; là, tu y dresseras un autel au Dieu qui t'apparut lorsque tu fuyais Esaü, ton frère.
\VS{2}Jacob dit à sa famille, et à tous ceux qui étaient avec lui : Otez les dieux des étrangers qui sont au milieu de vous, purifiez-vous, et changez de vêtements\FTNT{Jos. 24:23.}.
\VS{3}Levons-nous et montons à Béthel ; là je dresserai un autel au Dieu qui m'a exaucé dans le jour de ma détresse, et qui a été avec moi dans le chemin où j'ai marché.
\VS{4}Alors ils donnèrent à Jacob tous les dieux des étrangers qui étaient entre leurs mains, et les anneaux qui étaient à leurs oreilles, et il les cacha sous un térébinthe qui est près de Sichem.
\VS{5}Puis ils partirent. Et Dieu frappa de terreur les villes qui les entouraient, et l’on ne poursuivit point les fils de Jacob.
\VS{6}Ainsi Jacob, et tout le peuple qui était avec lui, arrivèrent à Luz, qui est Béthel, dans le pays de Canaan.
\VS{7}Il bâtit là un autel, et il appela ce lieu El-Béthel (le Dieu Puissant de Béthel) ; car c’est là que Dieu s’était révélé à lui lorsqu’il fuyait son frère.
\VS{8}Débora,  nourrice de Rebecca, mourut ; et elle fut ensevelie au-dessous de Béthel sous un chêne, auquel on donna le nom d’Allon-Bacuth (chêne des pleurs).
\VS{9}Dieu apparut encore à Jacob, après son retour de Paddan-Aram, et il le bénit\FTNT{Os. 12:5.}.
\VS{10}Dieu lui dit : Ton nom est Jacob, mais tu ne seras plus appelé Jacob, car ton nom sera Israël. Et il lui donna le nom d’Israël.
\VS{11}Dieu lui dit aussi : Je suis le Dieu Fort, Tout-Puissant. Sois fécond et multiplie : Une nation et une multitude de nations naîtront de toi, et des rois sortiront de tes reins.
\VS{12}Je te donnerai le pays que j'ai donné à Abraham et à Isaac, et je le donnerai à ta postérité après toi.
\VS{13}Dieu s’éleva au-dessus de lui dans le lieu où il lui avait parlé.
\VS{14}Et Jacob dressa un monument dans le lieu où Dieu lui avait parlé, à savoir un monument de pierre, et il fit dessus une aspersion et y versa de l'huile.
\VS{15}Jacob donna le nom de Béthel au lieu où Dieu lui avait parlé.
\VS{16}Puis ils partirent de Béthel, et il y avait encore une certaine distance jusqu’à Ephrata\FTNT{Ephrata : «~lieu de la fécondité~».} lorsque Rachel accoucha. Elle eut un accouchement difficile ;
\VS{17}et comme elle avait beaucoup de peine à accoucher, la sage-femme lui dit : Ne crains point, car tu as encore un fils.
\VS{18}Et comme elle rendait l'âme, car elle était mourante, elle lui donna le nom de Ben-Oni\FTNT{Ben-Oni : «~fils de ma douleur~».}, mais son père l’appela Benjamin\FTNT{Benjamin : «~fils de ma main droite~», «~fils de félicité~».}.
\VS{19}C'est ainsi que mourut Rachel, et elle fut ensevelie sur le chemin d'Ephrata, qui est Bethléhem.
\VS{20}Jacob dressa un monument sur son sépulcre. C'est le monument du sépulcre de Rachel qui subsiste encore aujourd'hui.
\VS{21}Puis Israël partit et dressa ses tentes au-delà de Migdal-Eder.
\VS{22}Pendant qu’Israël habitait dans ce pays, Ruben alla coucher avec Bilha, concubine de son père.  Et Israël l'apprit. Or Jacob avait douze fils.
\VS{23}Les fils de Léa étaient Ruben, premier-né de Jacob, Siméon, Lévi, Juda, Issacar, et Zabulon.
\VS{24}Les fils de Rachel : Joseph et Benjamin.
\VS{25}Les fils de Bilha, servante de Rachel : Dan et Nephthali.
\VS{26}Les fils de Zilpa, servante de Léa : Gad et Aser. Ce sont là les enfants de Jacob qui lui naquirent à Paddan-Aram.
\TextTitle{Jacob voit vers son père Isaac avant sa mort}
\VS{27}Jacob arriva auprès d’Isaac, son père, à la plaine de Mamré, à Kirjath-Arba, qui est Hébron, où Abraham et Isaac avaient séjourné comme étrangers.
\VS{28}Les jours d’Isaac furent de cent quatre-vingts ans.
\VS{29}Isaac expira et mourut, et fut recueilli auprès de son peuple, âgé et rassasié de jours ; et Esaü et Jacob ses fils l'ensevelirent.
\Chap{36}
\TextTitle{Postérité d'Esaü (Edom)}
\VerseOne{}Et voici la postérité d'Esaü, qui est Edom.
\VS{2}Esaü prit ses femmes parmi les filles de Canaan, à savoir Ada, fille d'Elon, le Héthien, Oholibama, fille d’Ana, petite-fille de Tsibeon, le Hévien.
\VS{3}Il prit aussi Basmath, fille d'Ismaël, sœur de Nebajoth.
\VS{4}Ada enfanta à Esaü Eliphaz ; et Basmath enfanta Réuel.
\VS{5}Et Oholibama enfanta Jéusch, Jaelam et Koré. Ce sont là les enfants d'Esaü qui lui naquirent dans le  pays de Canaan.
\VS{6}Esaü prit ses femmes, ses fils et ses filles, et toutes les personnes de sa maison, tous ses troupeaux, ses bêtes, et tout le bien qu'il avait acquis dans le pays de Canaan, et il s'en alla dans un autre pays, loin de Jacob, son frère.
\VS{7}Car leurs richesses étaient si grandes qu'ils n'auraient pas pu demeurer ensemble ; et le pays où ils séjournaient comme étrangers ne pouvait plus les contenir à cause de leurs troupeaux.
\VS{8}Ainsi Esaü habita dans la montagne de Séir ; Esaü est Edom.
\VS{9}Voici la postérité d'Esaü, père d'Edom, dans la montagne de Séir.
\VS{10}Voici les noms des fils d'Esaü : Eliphaz fils d’Ada, femme d'Esaü ; Réuel, fils de Basmath, femme d'Esaü.
\VS{11}Les fils d'Eliphaz furent : Théman, Omar, Tsepho, Gaetham et Kenaz.
\VS{12}Et Timna était la concubine d'Eliphaz, fils d'Esaü, et elle enfanta à Eliphaz Amalek. Ce sont là les fils d’Ada, femme d'Esaü.
\VS{13}Voici les fils de Réuel : Nahath, Zérach, Schamma et Mizza. Ce sont là les fils de Basmath, femme d'Esaü.
\VS{14}Voici les fils d'Oholibama, fille d’Ada, petite fille de Tsibeon, femme d'Esaü ; elle enfanta à Esaü Jéusch, Jaelam et Koré.
\VS{15}Voici les chefs des fils d'Esaü. Voici les fils d'Eliphaz, premier-né d'Esaü, le chef Théman, le chef Omar, le chef Tsepho, le chef Kenaz,
\VS{16}le chef Koré, le chef Gaetham, le chef Amalek. Ce sont là les chefs d'Eliphaz dans le  pays d'Edom. Ce sont les fils d’Ada.
\VS{17}Voici les fils de Réuel, fils d'Esaü : le chef Nahath, le chef Zérach, le chef Schamma, et le chef Mizza. Ce sont là les chefs sortis de Réuel, dans le pays d'Edom.  Ce sont là les fils de Basmath, femme d'Esaü.
\VS{18}Voici les fils d'Oholibama, femme d'Esaü : Le chef Jéusch, le chef Jaelam, le chef Koré. Ce sont là les chefs sortis d'Oholibama, fille d’Ana, femme d'Esaü.
\VS{19}Ce sont là les fils d'Esaü, qui est Edom, et ce sont là leurs chefs.
\VS{20}Voici les fils de Séir, le Horien, qui avaient habité dans le pays : Lothan, Schobal, Tsibeon, Ana,
\VS{21}Dischon, Etser, et Dischan. Ce sont là les chefs des Horiens, fils de Séir, dans le pays d'Edom.
\VS{22}Les fils de Lothan furent Hori et Héman.  Et Thimna était sœur de Lothan.
\VS{23}Voici les fils de Schobal : Alvan, Manahath, Ebal, Schepho et Onam.
\VS{24}Voici les fils de Tsibeon : Ajja et Ana. C’est cet Ana qui trouva les sources chaudes dans le désert, quand il faisait paître les ânes de Tsibeon, son père.
\VS{25}Voici les fils d’Ana : Dischon, et Oholibama, fille d’Ana.
\VS{26}Voici les fils de Dischon : Hemdan, Eschban, Jithran et Keran.
\VS{27}Voici les fils d'Etser : Bilhan, Zaavan et Akan.
\VS{28}Voci les fils de Dischan : Huts et Aran.
\VS{29}Voici les chefs des Horiens : Le chef Lothan, le chef Schobal, le chef Tsibeon, le chef Ana.
\VS{30}Le chef Dischon, le chef Etser, le chef Dischan. Ce sont là les chefs des Horiens, les chefs qu’ils établirent dans le pays de Séir.
\VS{31}Voici les rois qui ont régné dans le pays d'Edom, avant qu’un roi règne sur les enfants d'Israël.
\VS{32}Béla, fils de Béor, régna sur Edom, et le nom de sa ville était Dinhaba.
\VS{33}Béla mourut, et Jobab, fils de Zérach de Botsra, régna à sa place.
\VS{34}Jobab mourut, et Huscham, du pays des Thémanites, régna à sa place.
\VS{35}Huscham mourut, et Hadad, fils de Bédad, régna à sa place. C’est lui qui frappa Madian dans le territoire de Moab ; et le nom de sa ville était Avith.
\VS{36}Hadad mourut, et Samla, de Masréka, régna à sa place.
\VS{37}Samla mourut, et Saül de Réhoboth sur le fleuve, régna à sa place.
\VS{38}Saül mourut, et Baal-Hanan, fils d’Acbor, régna à sa place.
\VS{39}Baal-Hanan, fils de Hacbor mourut, et Hadar régna à sa place. Le nom de sa ville était Pau ; et le nom de sa femme Mehéthabeel, fille de Mathred, petite-fille de Mézahab.
\VS{40}Voici les noms des chefs d'Esaü selon leurs familles, selon leurs territoires, et d’après leurs noms : Le chef Thimna, le chef Alva, le chef Jétheth,
\VS{41}le chef Oholibama, le chef Ela, le chef Pinon,
\VS{42}Le chef Kenaz, le chef Théman, le chef Mibtsar,
\VS{43}le chef Magdiel, et le chef Iram. Ce sont là les chefs d'Edom, selon leurs habitations dans le pays qu’ils possédaient. C'est Esaü le père d'Edom.
\Chap{37}
\TextTitle{Jacob aime Joseph plus que ses autres fils}
\VerseOne{}Or Jacob demeura dans le pays de Canaan, pays où avait séjourné son père comme étranger.
\VS{2}Voici la postérité de Jacob. Joseph, âgé de dix-sept ans, faisait paître le troupeau avec ses frères ; et il était jeune garçon auprès des fils de Bilha et des fils de Zilpa, femmes de son père. Et Joseph rapportait à leur père leurs mauvais propos.
\VS{3}Or Israël aimait Joseph plus que tous ses autres fils, parce qu'il l'avait eu dans sa vieillesse, et il lui fit une tunique de plusieurs couleurs.
\VS{4}Ses frères voyant que leur père l'aimait plus qu'eux tous, le haïssaient et ne pouvaient lui parler paisiblement.
\VS{5}Joseph eut un songe et il raconta à ses frères ; et ils le haïrent encore davantage.
\VS{6}Il leur dit donc : Ecoutez, je vous prie, le songe que j'ai eu.
\VS{7}Voici, nous étions à lier des gerbes au milieu d'un champ ; et voici, ma gerbe se leva et se tint droite ; et voici, vos gerbes l’entourèrent et se prosternèrent devant elle.
\TextTitle{Joseph haï par ses frères}
\VS{8}Alors ses frères lui dirent : Régnerais-tu sur nous ? Et dominerais-tu sur nous ? Et ils le haïrent encore plus pour ses songes et pour ses paroles.
\VS{9}Il eut encore un autre songe, et il le raconta à ses frères, en disant : Voici, j'ai eu encore un songe ; et voici, le soleil, la lune et onze étoiles se prosternaient devant moi\FTNT{Ap. 12:1.}.
\VS{10}Il le raconta à son père et à ses frères. Son père le réprimanda et lui dit : Que veut dire ce songe que tu as eu ? Faut-il que nous venions moi, ta mère, et tes frères, nous prosterner à terre devant toi ?
\VS{11}Ses frères eurent de l'envie contre lui, mais son père garda ses discours\FTNT{Ac. 7:9.}.
\VS{12}Les frères de Joseph s'en allèrent paître les troupeaux de leur père à Sichem.
\VS{13}Israël dit à Joseph : Tes frères ne font-ils pas paître le troupeau à Sichem ? Viens, que je t'envoie vers eux ; et il lui répondit : Me voici.
\VS{14}Israël lui dit : Va maintenant, vois si tes frères se portent bien, et si le troupeau est en bon état, et rapporte-le-moi. Ainsi il l'envoya de la vallée d’Hébron, et il alla jusqu'à Sichem.
\VS{15}Un homme le rencontra, comme il errait dans les champs ; et cet homme le questionna et lui dit : Que cherches-tu ?
\VS{16}Joseph répondit : Je cherche mes frères ; je te prie, dis-moi où ils font paître leur troupeau.
\VS{17}Et l'homme dit : Ils sont partis d'ici, et je les ai entendus dire : Allons à Dothan. Joseph alla après ses frères et les trouva à Dothan.
\VS{18}Ils le virent de loin ; et avant qu'il soit près d’eux, ils complotèrent contre lui pour le tuer.
\VS{19}Ils se dirent l'un à l'autre : Voici ce maître songeur qui arrive.
\TextTitle{Joseph dans la citerne}
\VS{20}Venez maintenant, tuons-le, et jetons-le dans l’une de ces citernes ; et nous dirons qu'une bête féroce l'a dévoré, et nous verrons ce que deviendront ses songes.
\VS{21}Mais Ruben entendit cela et le délivra de leurs mains en disant : Ne lui ôtons point la vie.
\VS{22}Ruben leur dit encore : Ne répandez point le sang ; jetez-le dans cette citerne qui est au désert, mais ne mettez point la main sur lui. C'était pour le délivrer de leurs mains et le renvoyer à son père.
\VS{23}Lorsque Joseph fut arrivé auprès de ses frères, ils le dépouillèrent de sa tunique, de cette tunique de plusieurs couleurs qui était sur lui.
\VS{24}Ils le prirent et le jetèrent dans la citerne.  Cette citerne était vide, il n'y avait point d'eau.
\VS{25}Ensuite, ils s'assirent pour manger du pain ; et levant les yeux, ils virent une caravane d'Ismaélites qui passait et qui venait de Galaad ; et leurs chameaux étaient chargés d’aromates, du baume et de la myrrhe, qu’ils transportaient en Egypte.
\VS{26}Et Juda dit à ses frères : Que gagnerons-nous à tuer notre frère et à cacher son sang ?
\VS{27}Venez, vendons-le à ces Ismaélites, et ne mettons point notre main sur lui, car il est notre frère, notre chair ; et ses frères lui obéirent.
\TextTitle{Joseph vendu à des marchands et emmené en Egypte}
\VS{28}Et comme les marchands Madianites passaient, ils tirèrent et firent remonter Joseph de la citerne, et le vendirent pour vingt pièces d'argent aux Ismaélites, qui emmenèrent Joseph en Egypte\FTNT{Ps. 105:17.}.
\VS{29}Puis Ruben revint à la citerne, et voici, Joseph n'était plus dans la citerne. Alors il déchira ses vêtements.
\VS{30}Il retourna vers ses frères et leur dit : L'enfant n’y est plus ! Et moi ! Moi ! Où irai-je ?
\VS{31}Ils prirent la tunique de Joseph et tuèrent un bouc d'entre les chèvres, ils plongèrent la tunique dans le sang.
\VS{32}Puis ils envoyèrent et firent porter à leur père la tunique de plusieurs couleurs, en lui disant : Voici ce que nous avons trouvé ! Reconnais maintenant si c'est la tunique de ton fils ou non.
\VS{33}Jacob la reconnut, et dit : C'est la tunique de mon fils ! Une bête féroce l'a dévoré ! Certainement Joseph a été déchiré !
\VS{34}Et Jacob déchira ses vêtements, il mit un sac sur ses reins, et il porta le deuil de son fils durant plusieurs jours.
\VS{35}Tous ses fils et toutes ses filles vinrent pour le consoler, mais il rejeta toute consolation. Il disait : C’est en pleurant que je descendrai vers mon fils dans le scheol ! C'est ainsi que son père le pleurait.
\VS{36}Les Madianites le vendirent en Egypte à Potiphar, eunuque de Pharaon, chef des gardes.
\Chap{38}
\TextTitle{Péché de Juda}
\VerseOne{}Il arriva qu’en ce temps-là, Juda s’éloigna de ses frères et se retira vers un homme d’Adullam, nommé Hira.
\VS{2}Là, Juda vit la fille d'un Cananéen, nommé Schua, il la prit pour femme et alla vers elle.
\VS{3}Elle conçut et enfanta un fils qu’elle appela Er.
\VS{4}Elle conçut encore et enfanta un fils qu’elle appela Onan.
\VS{5}Elle enfanta de nouveau un fils qu’elle appela Schéla. Juda était à Czib quand elle l’enfanta.
\VS{6}Juda prit une femme pour Er, son premier-né, une femme nommée Tamar.
\VS{7}Mais Er le premier-né de Juda était méchant devant Yahweh, et Yahweh le fit mourir\FTNT{No. 26:19.}.
\VS{8}Alors Juda dit à Onan : Va vers la femme de ton frère, et prends-la pour femme, comme tu es son beau-frère, et suscite des enfants à ton frère\FTNT{Lé. 25:25  ; Lé. 25:48. Voir commentaire en Ru. 2:20.}.
\VS{9}Mais Onan, sachant que les enfants ne seraient pas à lui, se souillait à terre lorsqu’il allait vers la femme de son frère, afin de ne pas donner de postérité à son frère.
\VS{10}Ce qu'il faisait déplut à Yahweh, c'est pourquoi il le fit aussi mourir.
\VS{11}Et Juda dit à Tamar, sa belle-fille : Demeure veuve dans la maison de ton père, jusqu'à ce que Schéla, mon fils, soit grand ; car il dit : Il faut prendre garde qu'il ne meure comme ses frères. Ainsi Tamar s'en alla et demeura dans la maison de son père.
\VS{12}Et après plusieurs jours, la fille de Schua, femme de Juda, mourut ; lorsque Juda fut consolé, il monta vers ceux qui tondaient ses brebis à Thimna, avec Hira, l’Adullamite, son ami intime.
\VS{13}On en informa Tamar et on lui dit : Voici, ton beau-père monte à Thimna pour tondre ses brebis.
\VS{14}Alors elle ôta ses habits de veuve, se couvrit d'un voile, et s'enveloppa, et elle s’assit à l’entrée d’Enaïm, sur le chemin de Thimna ; car elle voyait que Schéla était devenu grand et qu’elle ne lui était point donnée pour femme.
\VS{15}Et quand Juda la vit, il s'imagina que c'était une prostituée, car elle avait couvert son visage.
\VS{16}Il l’aborda sur le chemin et lui dit : Permets, je te prie, que je vienne vers toi ; car il ne savait pas que c’était sa belle-fille. Et elle répondit : Que me donneras-tu pour venir vers moi ?
\VS{17}Il répondit : Je t'enverrai un chevreau d'entre les chèvres du troupeau. Elle répondit : Me donneras-tu un gage jusqu'à ce que tu l'envoies ?
\VS{18}Il répondit : Quel gage te donnerai-je ? Et elle répondit : Ton cachet, ton cordon, et ton bâton que tu as à la main. Et il les lui donna. Il alla vers elle, et elle devint enceinte de lui.
\VS{19}Puis elle se leva et s'en alla ; elle ôta son voile et remit ses habits de veuve.
\VS{20}Juda envoya un chevreau d'entre ses chèvres par son ami intime l’Adullamite, pour qu'il retire le gage de la main de la femme, mais il ne la trouva point.
\VS{21}Il interrogea les hommes du lieu où elle avait été, en disant : Où est cette prostituée qui était à Enaïm, sur le chemin ? Ils répondirent : Il n'y a point eu ici de prostituée.
\VS{22}Il retourna auprès de Juda et lui dit : Je ne l'ai point trouvée ; et même les gens du lieu m'ont dit : Il n'y a point eu ici de prostituée.
\VS{23}Juda dit : Qu'elle garde le gage, il ne faut pas nous faire mépriser. Voici, j'ai envoyé ce chevreau, mais tu ne l'as point trouvée.
\VS{24}Environ trois mois après, on fit un rapport à Juda, en disant : Tamar, ta belle-fille, a commis un adultère, et voici elle est même enceinte. Et Juda dit : Faites-la sortir, et qu'elle soit brûlée.
\VS{25}Comme on la faisait sortir, elle envoya dire à son beau-père : Je suis enceinte de l'homme à qui ces choses appartiennent. Elle dit aussi : Reconnais, je te prie, à qui est ce cachet, ce cordon, et ce bâton.
\VS{26}Alors Juda les reconnut et il dit : Elle est plus juste que moi, parce que je ne l'ai point donnée à Schéla, mon fils ; et il ne la connut plus.
\VS{27}Quand elle fut au moment d'accoucher, voici, des jumeaux étaient dans son ventre.
\VS{28}Et pendant qu’elle accouchait, il y en eut un qui présenta la main ; la sage-femme la prit et y attacha un fil cramoisi, en disant : Celui-ci sort le premier.
\VS{29}Mais il retira la main, et son frère sortit. Alors la sage-femme dit : Quelle brèche tu as faite ! Et elle lui donna le nom de Pérets.
\VS{30}Ensuite sortit son frère, qui avait à la main le fil cramoisi ; et on lui donna le nom de Zérach.
\Chap{39}
\TextTitle{Joseph fidèle à Yahweh devant la tentation}
\VerseOne{}Or, quand on fit descendre Joseph en Egypte, Potiphar, eunuque de Pharaon, chef des gardes, Egyptien, l'acheta de la main des Ismaélites qui l'y avaient amené.
\VS{2}Yahweh était avec Joseph ; et il prospéra, et demeura dans la maison de son maître,  l’Egyptien.
\VS{3}Son maître vit que Yahweh était avec lui, et que Yahweh faisait prospérer entre ses mains tout ce qu'il faisait.
\VS{4}C'est pourquoi Joseph trouva grâce aux yeux de son maître, qui l’employa à son service. Et son maître l'établit sur sa maison, et lui remit entre les mains tout ce qui lui appartenait.
\VS{5}Dès que Potiphar l’eut établi sur sa maison et sur tout ce qu’il possédait, Yahweh bénit la maison de l’Egyptien, à cause de Joseph ; et la bénédiction de Yahweh fut sur tout ce qui lui appartenait, soit à la maison, soit aux champs.
\VS{6}Il abandonna aux mains de Joseph tout ce qui lui appartenait, et il n’avait avec lui d’autre soin que celui de prendre sa nourriture. Or Joseph était beau de  taille et beau de figure.
\VS{7}Après ces choses, il arriva que la femme de son maître porta les yeux sur Joseph, et elle lui dit : Couche avec moi\FTNT{Pr. 7:9-13.} !
\VS{8}Mais il le refusa, et dit à la femme de son maître : Voici, mon maître ne prend avec moi connaissance de rien dans la maison, et il a remis entre mes mains tout ce qui lui appartient.
\VS{9}Il n'y a personne dans cette maison qui soit plus grand que moi, et il ne m'a rien interdit excepté toi, parce que tu es sa femme ; et comment ferais-je un si grand mal et pécherais-je contre Dieu ?
\VS{10}Quoiqu’elle parlât tous les jours à Joseph, il refusa de coucher auprès d’elle, d’être avec elle.
\VS{11}Un jour qu'il était entré dans la maison pour faire son ouvrage, et qu'il n'y avait là aucun des gens dans la maison,
\VS{12}elle le saisit par son vêtement et lui dit : Couche avec moi ! Mais il laissa son vêtement entre ses mains, s'enfuit, et sortit dehors\FTNT{1 Co. 6:18.}.
\TextTitle{Fausse accusation contre Joseph}
\VS{13}Et lorsqu'elle vit qu'il lui avait laissé son vêtement entre les mains, et qu'il s'était enfui dehors,
\VS{14}elle appela les gens de sa maison, et leur parla en disant : Voyez, on nous a amené un Hébreu pour se moquer de nous.  Cet homme est venu vers moi pour coucher avec moi ; mais j'ai crié à haute voix.
\VS{15}Et dès qu’il a entendu que j’élevais la voix et que je  criais, il a laissé son vêtement à côté de moi, et s’est enfui dehors.
\VS{16}Et elle garda le vêtement de Joseph jusqu'à ce que son maître rentre à la maison.
\VS{17}Alors elle lui parla en ces mêmes termes et dit : Le serviteur Hébreu que tu nous as amené est venu vers moi pour se moquer de moi.
\VS{18}Mais comme j'ai élevé ma voix et que j'ai crié, il a laissé son vêtement à côté de moi et s'est enfui.
\VS{19}Et dès que le maître de Joseph eut entendu les paroles de sa femme qui lui disait : Ton serviteur m'a fait ce que je t'ai dit, sa colère s'enflamma.
\VS{20}Et le maître de Joseph le prit et le mit dans une étroite prison ; dans l'endroit où les prisonniers du roi étaient enfermés, et il fut là en prison.
\VS{21}Mais Yahweh fut avec Joseph ; il étendit sa bonté sur lui et lui fit trouver grâce auprès du chef de la prison.
\VS{22}Et le chef de la prison mit entre les mains de Joseph tous les prisonniers qui étaient dans la prison, et tout ce qu'il y avait à faire, il le faisait.
\VS{23}Le chef de la prison ne prenait aucune connaissance de ce que Joseph avait en main, parce que Yahweh était avec lui. Et Yahweh faisait prospérer tout ce qu'il faisait.
\Chap{40}
\TextTitle{Joseph demeure en prison}
\VerseOne{}Après ces choses, il arriva que l'échanson et le panetier du roi d'Egypte offensèrent leur maître, le roi d'Egypte.
\VS{2}Pharaon fut fort irrité contre ces deux eunuques, contre le chef des échansons, et contre le chef des panetiers.
\VS{3}Et il les fit mettre dans la maison du chef des gardes, dans la prison étroite, dans le même lieu où Joseph était enfermé.
\VS{4}Le chef des gardes les mit entre les mains de Joseph qui les servait ; et ils furent quelques jours en prison.
\VS{5}Pendant une même nuit, l’échanson et le panetier du roi d’Egypte, qui étaient enfermés dans la prison, eurent tous les deux un songe, chacun le sien, pouvant recevoir une explication distincte.
\VS{6}Joseph, étant venu le matin vers eux, les regarda ; et voici, ils étaient fort tristes.
\VS{7}Et il interrogea ces deux eunuques de Pharaon, qui étaient avec lui dans la prison de son maître, et leur dit : Pourquoi avez-vous mauvais visage aujourd'hui ?
\VS{8}Ils lui répondirent : Nous avons eu des songes, et il n'y a personne qui les interprète. Et Joseph leur dit : Les interprétations n’appartiennent-elles pas à Dieu ? Je vous prie, racontez-moi vos songes\FTNT{1 Co. 12:8-10 ; Job. 33:15.}.
\VS{9}Le chef des échansons raconta son songe à Joseph et lui dit : Dans mon songe, voici, il y avait un cep devant moi.
\VS{10}Ce cep avait trois sarments. Quand il eut poussé, sa fleur se développa et ses grappes donnèrent des raisins mûrs.
\VS{11}La coupe de Pharaon était dans ma main. Je pris les raisins, je les pressai dans la coupe de Pharaon, et je mis la coupe dans la main de Pharaon.
\VS{12}Joseph lui dit : Voici son interprétation : Les trois sarments sont trois jours.
\VS{13}Dans trois jours Pharaon élèvera ta tête et te rétablira dans ta charge, et tu mettras la coupe dans sa main, comme tu le faisais auparavant, lorsque tu étais son échanson.
\VS{14}Mais souviens-toi de moi quand tu  seras heureux, et use de bonté envers moi je te prie ; fais mention de moi à Pharaon, afin  qu’il me fasse sortir de cette maison.
\VS{15}Car certainement j'ai été enlevé du pays des Hébreux ; ici non plus je n’ai rien fait  pour  être mis en prison.
\VS{16}Le chef des panetiers, voyant que Joseph avait interprété favorablement ce songe, lui dit : Voici, il y avait aussi dans mon songe trois corbeilles de pain blanc sur ma tête.
\VS{17}Dans la corbeille la plus élevée, il y avait pour Pharaon des mets de toute espèce, cuits au four ; et les oiseaux les mangeaient dans la corbeille au-dessus de ma tête.
\VS{18}Joseph répondit et dit : Voici son interprétation : Les trois corbeilles sont trois jours.
\VS{19}Dans trois jours Pharaon enlèvera ta tête de dessus toi et te fera pendre à un bois, et les oiseaux mangeront ta chair sur toi.
\VS{20}Le troisième jour, jour de la naissance de Pharaon, il fit un festin à tous ses serviteurs ; et il éleva la tête du chef des échansons et la tête du chef des panetiers, au milieu de ses serviteurs.
\VS{21}Il rétablit le chef des échansons dans sa charge d’échanson, pour qu’il mette la coupe dans la main de Pharaon.
\VS{22}Mais il fit pendre le chef des panetiers, selon l’explication que Joseph leur avait donnée.
\VS{23}Cependant, le chef des échansons ne pensa plus à Joseph. Il l’oublia.
\Chap{41}
\TextTitle{Les songes de Pharaon}
\VerseOne{}Mais il arriva qu’au bout de deux ans entiers, Pharaon eut un songe. Et il lui semblait qu'il était près du fleuve.
\VS{2}Et voici, sept jeunes vaches belles à voir, grasses de chair, montèrent hors du fleuve et se mirent à paître dans les  prairies.
\VS{3}Et voici sept autres jeunes vaches, laides à voir, et maigres de chair, montèrent hors du fleuve derrière les autres et se tinrent auprès des autres jeunes vaches sur le bord du fleuve.
\VS{4}Les jeunes vaches laides à voir, et maigres, mangèrent les sept jeunes vaches belles à voir, et grasses. Alors Pharaon s'éveilla.
\VS{5}Il se rendormit et il eut un second songe. Voici, sept épis gras et beaux montèrent sur une même tige.
\VS{6}Et sept épis maigres et brûlés par le vent d’orient poussèrent après eux.
\VS{7}Les épis maigres engloutirent les sept épis gras et pleins. Et Pharaon s'éveilla ; et voilà le songe.
\VS{8}Le matin, Pharaon eut l’esprit troublé, et il envoya appeler tous les magiciens et tous les sages d'Egypte, et leur raconta ses songes.  Mais personne ne put les interpréter à Pharaon.
\VS{9}Alors le chef des échansons parla à Pharaon en disant : Je rappellerai aujourd'hui le souvenir de mes fautes.
\VS{10}Lorsque Pharaon fut irrité contre ses serviteurs, et nous fit mettre, le chef des panetiers et moi, en prison, dans la maison du chef des gardes,
\VS{11}nous eûmes l’un et l’autre un songe dans une même nuit ; et chacun de nous reçut une interprétation en rapport avec le songe qu’il avait eu.
\VS{12}Il y avait là avec nous un garçon Hébreu, esclave du chef des gardes. Nous lui racontâmes nos songes, et il nous les expliqua.
\VS{13}Les choses sont arrivées comme il nous les avait interprétées ; car le roi me rétablit dans ma charge et fit pendre le chef des panetiers.
\TextTitle{Joseph sort de prison et est établi sur l'Egypte par Pharaon}
\VS{14}Alors Pharaon envoya appeler Joseph.  On le fit sortir en hâte de la prison ; on le rasa, et on lui fit changer de vêtements ; puis il se rendit vers Pharaon.
\VS{15}Pharaon dit à Joseph : J'ai eu un songe, et personne ne peut  l'expliquer ; or j'ai appris que tu sais expliquer les songes.
\VS{16}Joseph répondit à Pharaon en disant : Ce n’est pas moi !  C’est Dieu qui donnera une réponse concernant la paix de Pharaon.
\VS{17}Pharaon dit alors à Joseph : Dans mon songe, voici, je me tenais sur le bord du fleuve.
\VS{18}Et voici, sept vaches grasses de chair et belles d’apparence montèrent hors du fleuve et se mirent à paître dans la prairie.
\VS{19}Sept autres vaches montèrent derrière elles, maigres, fort laides d’apparence, et décharnées ; je n’en ai point vu d’aussi laides dans tout le pays d’Egypte.
\VS{20}Les vaches décharnées et laides mangèrent les sept premières vaches qui étaient grasses ;
\VS{21}elles les engloutirent dans leur ventre, sans qu’on s’aperçoive qu’elles y étaient entrées ; et leur apparence était laide comme auparavant. Et je m’éveillai.
\VS{22}Je vis encore en songe sept épis pleins et beaux, qui montèrent sur une même tige.
\VS{23}Et sept épis vides, maigres, brûlés par le vent d’orient, poussèrent après eux.
\VS{24}Les épis maigres engloutirent les sept beaux épis. Je l’ai dit aux magiciens, mais personne ne m’a donné l’explication. 25 Joseph dit à Pharaon : Ce qu’a rêvé Pharaon est une seule chose ; Dieu a fait connaître à Pharaon ce qu’il va faire.
\VS{26}Les sept vaches belles sont sept années ; et les sept épis beaux sont sept années ; c’est un seul songe.
\VS{27}Les sept vaches décharnées et laides, qui montaient derrière les premières, sont sept années ; et les sept épis vides, brûlés par le vent d’orient, seront sept années de famine.
\VS{28}Ainsi, comme je viens de le dire à Pharaon, Dieu a fait connaître à Pharaon ce qu’il va faire.
\VS{29}Voici, il y aura sept années de grande abondance dans tout le pays d’Egypte.
\VS{30}Sept années de famine viendront après elles ; et l’on oubliera toute cette abondance au pays d’Egypte, et la famine consumera le pays.
\VS{31}Cette famine qui suivra sera si forte qu’on ne s’apercevra plus de l’abondance dans le pays.
\VS{32}Si Pharaon a vu le songe se répéter une seconde fois, c’est que la chose est arrêtée de la part de Dieu, et que Dieu se hâtera de l’exécuter.
\VS{33}Maintenant que Pharaon choisisse un homme intelligent et sage, et qu'il l'établisse sur le pays d'Egypte.
\VS{34}Que Pharaon établisse et institue des commissaires sur le pays, et qu'ils prennent la cinquième partie du revenu du pays d'Egypte durant les sept années d'abondance.
\VS{35}Qu’ils rassemblent tous les produits de ces bonnes années  qui viennent ; qu’ils fassent sous l’autorité de Pharaon des amas de blé, des approvisionnements dans les villes, et qu’ils en aient la garde.
\VS{36}Ces provisions seront en réserve pour le pays durant les sept années de famine qui seront dans le  pays d'Egypte, afin que le pays ne soit pas consumé par la famine.
\VS{37}Ces paroles plurent à Pharaon et à tous ses serviteurs\FTNT{Ac. 7:10.}.
\VS{38}Et Pharaon dit à ses serviteurs : Trouverions-nous un homme semblable à celui-ci, qui a l'Esprit de Dieu ?
\VS{39}Et Pharaon dit à Joseph : Puisque Dieu t'a fait connaître toutes ces choses, il n'y a personne qui soit aussi intelligent et aussi sage que toi.
\VS{40}C’est toi qui seras sur ma maison, et tout mon peuple obéira à tes ordres ; je serai seulement plus grand que toi par le trône.
\VS{41}Pharaon dit encore à Joseph : Regarde, je t’établis sur tout le pays d'Egypte.
\VS{42}Alors Pharaon ôta son anneau de sa main et le mit à la main de Joseph ; il le fit revêtir d'habits de fin lin et lui mit un collier d'or au cou.
\VS{43}Il le fit monter sur le char qui suivait le sien, et on criait devant lui : A genoux ! Et il l'établit sur tout le pays d'Egypte.
\VS{44}Et Pharaon dit à Joseph : Je suis Pharaon ! Et sans toi nul ne lèvera la main ni le pied dans tout le pays d'Egypte.
\TextTitle{Joseph épouse une égyptienne}
\VS{45}Pharaon appela Joseph du nom de Tsaphnath-Paenéach ; et il lui donna pour femme Asnath, fille de Poti-Phéra, prêtre d'On. Et Joseph alla visiter le pays d'Egypte.
\VS{46}Joseph était âgé de trente ans lorsqu’il se présenta devant Pharaon, roi d'Egypte ; et il quitta Pharaon et parcourut tout le pays d'Egypte.
\VS{47}Et la terre rapporta très abondamment pendant les sept années de fertilité.
\VS{48}Joseph rassembla tous les produits de ces sept années dans le pays d’Egypte ; il fit des approvisionnements dans les villes, mettant dans l’intérieur de chaque ville les productions des champs d’alentour.
\VS{49}Ainsi Joseph amassa une grande quantité de blé, comme le sable de la mer ; tellement qu'on cessa de le compter, parce qu’il n’y avait plus de nombre.
\VS{50}Avant les années de famine, il naquit à Joseph deux fils, que lui enfanta Asnath, fille de Poti-Phéra, prêtre  d'On.
\VS{51}Joseph donna au premier-né le nom de Manassé, parce que, dit-il, Dieu m'a fait oublier toute ma peine et toute la maison de mon père.
\VS{52}Et il donna au second le nom d’Ephraïm, parce que, dit-il, Dieu m'a fait fructifier dans le  pays de mon affliction.
\VS{53}Alors finirent les sept années de l'abondance qui avaient été dans le pays d'Egypte.
\VS{54}Et les sept années de la famine commencèrent à venir comme Joseph l'avait prédit. Et la famine fut dans tous les pays ; mais il y avait du pain dans tout le pays d'Egypte.
\VS{55}Ensuite tout le pays d'Egypte fut affamé, et le peuple cria à Pharaon pour avoir du pain. Et Pharaon répondit à tous les Egyptiens : Allez vers Joseph, et faites ce qu'il vous dira.
\VS{56}La famine régnait dans tout le pays. Joseph ouvrit tous les lieux d’approvisionnements et vendit du blé aux Egyptiens. La famine augmentait dans le pays d’Egypte.
\VS{57}On venait de tous les pays jusqu’en Egypte, pour acheter du blé  auprès de Joseph ; car la famine était fort grande sur toute la terre.
\Chap{42}
\TextTitle{Les frères de Joseph viennent acheter des vivres en Egypte}
\VerseOne{}Et Jacob, voyant qu'il y avait du blé à vendre en Egypte, dit à ses fils : Pourquoi vous regardez-vous les uns les autres ?
\VS{2}Il leur dit aussi : Voici, j'ai appris qu'il y a du blé à vendre en Egypte, descendez-y pour nous en acheter là, afin que nous vivions, et que nous ne mourions point.
\VS{3}Alors les frères de Joseph descendirent pour acheter du blé en Egypte.
\VS{4}Mais Jacob n'envoya point Benjamin, frère de Joseph, avec ses frères ; car il disait : Il faut prendre garde qu’un malheur ne lui arrive.
\VS{5}Ainsi les fils d'Israël allèrent en Egypte pour acheter du blé avec ceux qui y allaient, car la famine était dans le pays de Canaan.
\TextTitle{Joseph met ses frères à l'épreuve}
\VS{6}Joseph commandait dans le pays, et c’était lui qui vendait le blé à tous les peuples de la terre. Les frères de Joseph vinrent et se prosternèrent devant lui la face contre terre.
\VS{7}Joseph vit ses frères et les reconnut ; mais il feignit d’être un étranger pour eux, et il leur parla rudement, en leur disant : D'où venez-vous ? Et ils répondirent : Du pays de Canaan, pour acheter des vivres.
\VS{8}Joseph reconnut ses frères, mais eux ne le connurent point.
\VS{9}Alors Joseph se souvint des songes qu'il avait eus à leur sujet et leur dit : Vous êtes des espions, vous êtes venus pour observer les lieux faibles du pays.
\VS{10}Et ils lui répondirent : Non, mon seigneur, mais tes serviteurs sont venus pour acheter des vivres.
\VS{11}Nous sommes tous enfants d'un même homme, nous sommes des gens de bien ; tes serviteurs ne sont pas des espions.
\VS{12}Et il leur dit : Nullement ; vous êtes venus pour observer les lieux faibles du pays.
\VS{13}Et ils répondirent : Nous, tes serviteurs, étions douze frères, fils d'un même homme, dans le pays de Canaan. Et voici, le plus jeune est aujourd'hui avec notre père, et l'un n'est plus.
\VS{14}Joseph leur dit : C'est ce que je vous disais, vous êtes des espions.
\VS{15}Voici comment vous serez éprouvés : Par la vie de Pharaon ! Vous ne sortirez pas d'ici que votre jeune frère ne soit venu ici.
\VS{16}Envoyez l’un de vous et qu’il amène votre frère ; et  vous, restez prisonniers. Vos paroles seront éprouvées et je saurai si vous avez dit la vérité. Autrement, par la vie de Pharaon ! Vous êtes des espions.
\VS{17}Et il les mit tous ensemble en prison pendant trois jours.
\VS{18}Le troisième jour, Joseph leur dit : Faites ceci, et vous vivrez. Je crains Dieu !
\VS{19}Si vous êtes sincères, que l'un de vos frères reste enfermé dans votre prison ; et vous, partez et emportez du blé pour nourrir vos familles.
\VS{20}Puis amenez-moi votre jeune frère afin que vos paroles soient éprouvées, et vous ne mourrez point ; et ils firent ainsi.
\TextTitle{Siméon gardé en Egypte en attendant que Benjamin soit présenté à Joseph}
\VS{21}Et ils se dirent alors l'un à l'autre : Nous sommes certainement coupables à l'égard de notre frère ; car nous avons vu l'angoisse de son âme quand il nous demandait grâce, et nous ne l'avons point écouté ; c'est pour cela que cette détresse nous est arrivée.
\VS{22}Ruben leur répondit en disant : Ne vous disais-je pas : Ne commettez point ce péché contre l'enfant ? Et vous ne m’avez point écouté ; et voici que son sang vous est redemandé.
\VS{23}Ils ne savaient pas que Joseph les comprenait, parce qu'il se servait d’un interprète pour leur parler.
\VS{24}Il s’éloigna d’eux pour pleurer. Et il revint, leur parla ; puis il prit parmi eux Siméon, et le fit enchaîner sous leurs yeux.
\VS{25}Et Joseph ordonna qu'on remplisse leurs sacs de blé, et qu'on remette l'argent de chacun d’eux dans son sac, et qu'on leur donne de la provision pour la route ; et cela fut fait ainsi.
\VS{26}Ils chargèrent donc leur blé sur leurs ânes, et s'en allèrent.
\VS{27}L’un d'eux ouvrit son sac pour donner du fourrage à son âne dans l'hôtellerie ; et il vit son argent qui était à l’entrée de son sac.
\VS{28}Il dit à ses frères : Mon argent m'a été rendu ; et le voici dans mon sac. Alors leur cœur fut en défaillance ; et ils furent saisis de peur, et se dirent l'un à l'autre : Qu'est-ce que Dieu nous a fait ?
\VS{29}Et étant arrivés dans le pays de Canaan, vers Jacob leur père, ils lui racontèrent toutes les choses qui leur étaient arrivées, en disant :
\VS{30}L'homme qui est le seigneur du pays, nous a parlé rudement et nous a pris pour des espions du pays.
\VS{31}Mais nous lui avons répondu : Nous sommes sincères, nous ne sommes point des espions.
\VS{32}Nous étions douze frères, fils de notre père ; l'un n'est plus, et le plus jeune est aujourd'hui avec notre père dans le pays de Canaan.
\VS{33}Et cet homme, qui est le seigneur  du pays, nous a dit : A ceci je connaîtrai que vous êtes sincères : Laissez-moi l'un de vos frères, et prenez de quoi nourrir vos familles et partez.
\VS{34}Puis amenez-moi votre jeune frère, et je saurai que vous n'êtes point des espions, que vous êtes sincères ; je vous rendrai votre frère, et vous pourrez librement trafiquer dans le  pays.
\VS{35}Lorsqu’ils vidèrent leurs sacs, voici, le paquet d’argent de chacun était dans son sac. Ils virent, eux et leur père, leurs paquets d’argent, et ils eurent peur.
\VS{36}Jacob leur père leur dit : Vous me privez de mes enfants ! Joseph n'est plus, et Siméon n'est plus, et vous prendriez Benjamin ! C’est sur moi que tout cela retombe.
\VS{37}Ruben parla à son père et lui dit : Fais mourir deux de mes fils si je ne te ramène pas Benjamin ! Remets-le entre mes mains et je te le ramènerai.
\VS{38}Jacob répondit : Mon fils ne descendra point avec vous, car son frère est mort, et il reste seul ; s’il lui arrivait un malheur dans le voyage que vous allez faire, vous feriez descendre mes cheveux blancs avec douleur dans le scheol.
\Chap{43}
\TextTitle{Jacob renvoie ses fils en Egypte\FTNTT{Ge. 37:26-28}}
\VerseOne{}Or la famine devint fort grande dans le pays.
\VS{2}Et quand  ils eurent achevé de manger le blé qu'ils avaient apporté d'Egypte, leur père leur dit : Retournez, achetez-nous un peu de vivres.
\VS{3}Juda lui répondit et lui dit : Cet homme nous a expressément déclaré, disant : Vous ne verrez point ma face, à moins que votre frère ne soit avec vous.
\VS{4}Si donc tu envoies notre frère avec nous, nous descendrons en Egypte et nous t'achèterons des vivres.
\VS{5}Mais si tu ne l'envoies pas, nous n'y descendrons point ; car cet homme nous a dit : Vous ne verrez point ma face, à moins que votre frère ne soit avec vous.
\VS{6}Et Israël dit : Pourquoi avez-vous mal agi à mon égard, en disant à cet homme que vous aviez encore un frère ?
\VS{7}Ils répondirent : Cet homme nous a interrogés sur nous et sur notre famille, en disant : Votre père vit-il encore ? N'avez-vous point de frère ? Et nous lui avons déclaré selon ce qu'il nous avait demandé ; pouvions-nous savoir qu'il dirait : Faites descendre votre frère ?
\VS{8}Juda dit à Israël, son père : Laisse venir l'enfant avec moi, afin que nous nous levions et que nous partions ; et nous vivrons et nous ne mourons point, nous, toi et nos enfants.
\VS{9}Je réponds de lui, tu le redemanderas de ma main. Si je ne te le ramène pas auprès de toi et si je ne le remets pas devant ta face, je serai coupable toute ma vie envers toi.
\VS{10}Car si nous n’avions pas tardé, certainement nous serions déjà de retour deux fois.
\VS{11}Alors Israël leur père leur dit : Si cela est ainsi, faites ceci, prenez dans vos bagages les meilleures productions du pays, pour en  porter un présent à cet homme, un peu de baume, et un peu de miel, des épices, de la myrrhe, des dattes, et des amandes.
\VS{12}Prenez avec vous de l'argent au double dans vos mains, et rapportez l’argent qu’on avait mis à l’entrée de vos sacs ; peut-être était-ce une erreur.
\VS{13}Prenez votre frère, et levez-vous, retournez vers cet homme.
\VS{14}Que le Dieu Tout-Puissant vous fasse trouver grâce devant cet homme, afin qu'il relâche votre autre frère et Benjamin ; et s'il faut que je sois privé de ces deux fils, que j'en sois privé.
\VS{15}Alors ils prirent le présent, et ayant pris de l'argent au double dans leurs mains, et Benjamin, ils se levèrent et descendirent en Egypte ; puis ils se présentèrent devant Joseph.
\VS{16}Dès que Joseph vit Benjamin avec eux, il dit à l’intendant de sa maison : Fais entrer ces gens dans la maison, tue et apprête quelques bêtes, car ils mangeront à midi avec moi.
\VS{17}Cet homme fit ce que Joseph lui avait dit ; et il conduit ces gens dans la maison de Joseph.
\VS{18}Ils eurent peur lorsqu’ils furent conduits dans la maison de Joseph, et ils dirent : Nous sommes emmenés à cause de l'argent remis l’autre fois dans nos sacs ; c’est pour se jeter sur nous, se précipiter sur nous ; c’est pour nous prendre comme esclaves et s’emparer de nos ânes.
\VS{19}Ils s’approchèrent de l’intendant de la maison de Joseph, et lui adressèrent la parole, à l’entrée de la maison.
\VS{20}Ils dirent : Pardon ! Mon seigneur, nous sommes déjà descendus une fois pour acheter des vivres.
\VS{21}Puis, quand nous arrivâmes, au lieu où nous devions passer la nuit, nous avons ouvert nos sacs ; et voici, l’argent de chacun était à l’entrée de son sac, notre argent selon son poids ;  nous le rapportons avec nous.
\VS{22}Nous avons aussi apporté d'autre argent dans nos mains pour acheter des vivres ; et nous ne savons point qui a remis notre argent dans nos sacs.
\VS{23}L’intendant leur dit : Tout va bien pour vous, ne craignez point. C’est votre Dieu, le Dieu de votre père vous a donné un trésor dans vos sacs ; votre argent est parvenu jusqu'à moi ; et il leur amena Siméon.
\VS{24}Cet homme les fit entrer dans la maison de Joseph, et leur donna de l'eau, et ils lavèrent leurs pieds ; il donna aussi à manger à leurs ânes.
\VS{25}Ils préparèrent leur présent en attendant que Joseph revienne à midi ; car ils avaient appris qu'ils mangeraient du pain chez lui.
\VS{26}Quand  Joseph fut arrivé à la maison, ils lui offrirent le présent qu'ils avaient dans leurs mains, et se prosternèrent à terre devant lui dans la maison.
\VS{27}Il leur demanda comment ils se portaient et leur dit : Votre vieux père, dont vous m'avez parlé, se porte-t-il bien ? Vit-il encore ?
\VS{28}Ils répondirent : Ton serviteur, notre père, se porte bien, il vit encore. Et ils s’inclinèrent et se prosternèrent.
\VS{29}Joseph leva les yeux, il vit Benjamin, son frère, fils de sa mère, et il dit : Est-ce là votre jeune frère dont vous m'avez parlé ? Et il ajouta : Mon fils, Dieu te fasse grâce !
\VS{30}Et Joseph se retira promptement, car ses entrailles étaient émues à la vue de son frère, et il cherchait un lieu pour pleurer ; il entra dans sa chambre et il y pleura.
\VS{31}Après s’être lavé le visage, il sortit de là, et faisant des efforts pour se contenir, il dit : Servez le pain.
\VS{32}On servit Joseph à part, et ses frères à part, et les Egyptiens qui mangeaient avec lui furent aussi servis à part, car les Egyptiens ne pouvaient manger du pain avec les Hébreux,  parce que c’est à leurs yeux une abomination.
\VS{33}Les frères de Joseph s’assirent en sa présence, le premier-né selon son droit d’aînesse, et le plus jeune selon son âge ; et ils se regardaient les uns les autres avec étonnement.
\VS{34}Joseph leur fit porter des mets qui étaient devant lui, et Benjamin en eut cinq fois plus que les autres. Ils burent et s’enivrèrent  avec lui.
\Chap{44}
\TextTitle{Juda se rend esclave de Joseph à la place de Benjamin\FTNTT{Ge. 43:9}}
\VerseOne{}Et Joseph donna un ordre à son intendant, en disant : Remplis de vivres les sacs de ces gens, autant qu'ils en pourront porter, et remets l'argent de chacun à l’entrée de son sac.
\VS{2}Tu mettras aussi ma coupe, la coupe d'argent, à l’entrée du sac du plus petit avec l'argent de son blé ; et il fit comme Joseph lui avait dit.
\VS{3}Le matin, dès qu'il fit jour, on renvoya ces hommes avec leurs ânes.
\VS{4}Ils étaient sortis de la ville, ils n’en étaient guère éloignés, lorsque Joseph dit à son intendant : Va, poursuis ces hommes, et quand tu les auras atteints, tu leur diras : Pourquoi avez-vous rendu le mal pour le bien ?
\VS{5}N'est-ce pas la coupe dont se sert mon seigneur pour boire et pour deviner ? Vous avez mal fait d’agir ainsi.
\VS{6}L’intendant les atteignit, et leur dit ces paroles.
\VS{7}Ils lui répondirent : Pourquoi mon seigneur parle-t-il ainsi ? Loin de tes serviteurs la pensée de faire pareille chose !
\VS{8}Voici, nous t'avons rapporté du pays de Canaan l'argent que nous avions trouvé à l’entrée de nos sacs, et comment aurions-nous dérobé de l'argent ou de l'or de la maison de ton maître ?
\VS{9}Que celui de tes serviteurs sur qui se trouvera la coupe meure ; et nous serons aussi esclaves de mon seigneur !
\VS{10}Il leur dit : Qu'il soit fait maintenant selon vos paroles ! Qu’il en soit ainsi ! Que celui sur qui se trouvera la coupe soit mon esclave, et vous, vous serez innocents.
\VS{11}Et ils se hâtèrent de déposer chacun son sac à terre ; et chacun ouvrit son sac.
\VS{12}L’intendant les fouilla, en commençant par le plus âgé, et finissant par le plus jeune ; et la coupe fut trouvée dans le sac de Benjamin.
\VS{13}Alors ils déchirèrent leurs vêtements, et chacun rechargea son âne, et ils retournèrent à la ville.
\VS{14}Juda et ses frères arrivèrent à la maison de Joseph, qui était encore là, et ils se jetèrent à terre devant lui.
\VS{15}Joseph leur dit : Quelle action avez-vous faite ? Ne savez-vous pas qu'un homme tel que moi ne manque pas de deviner ?
\VS{16}Juda lui répondit : Que dirons-nous à mon seigneur ? Comment parlerons-nous ? Et comment nous justifierons-nous ? Dieu a trouvé l'iniquité de tes serviteurs ; voici, nous sommes esclaves de mon seigneur, nous, et celui entre les mains de qui la coupe a été trouvée.
\VS{17}Mais il dit : Loin de moi la pensée d’agir ainsi ! L’homme dans la main duquel la coupe a été trouvée sera mon esclave ; mais vous, remontez en paix vers votre père.
\VS{18}Alors Juda s'approcha de lui en disant : Pardon mon seigneur ! Je te prie, que ton serviteur dise un mot, je te prie aux oreilles de mon seigneur, et que ta colère ne s'enflamme point contre ton serviteur, car tu es comme Pharaon.
\VS{19}Mon seigneur interrogea ses serviteurs en disant : Avez-vous un père ou un frère ?
\VS{20}Nous avons répondu à mon seigneur : Nous avons notre père qui est âgé, et un enfant de sa vieillesse, et qui est le plus jeune d'entre nous ; son frère est mort, et celui-ci est resté le seul enfant  de sa mère ; et son père l'aime.
\VS{21}Tu as dis à tes serviteurs : Faites-le descendre vers moi, et que je le voie de mes yeux.
\VS{22}Nous avons répondu  à mon seigneur : Cet enfant ne peut quitter son père, car s'il le quitte, son père mourra.
\VS{23}Alors tu dis à tes serviteurs : Si votre petit frère ne descend avec vous, vous ne verrez plus ma face.
\VS{24}Lorsque nous sommes remontés auprès de ton serviteur, mon père, nous lui avons rapporté les paroles de mon seigneur.
\VS{25}Notre père nous a dit : Retournez, et achetez-nous un peu de vivres.
\VS{26}Nous lui avons répondu : Nous ne pouvons pas descendre ; mais si notre petit frère est avec nous, nous descendrons, car nous ne pouvons pas voir la face de cet homme, à moins que notre jeune frère ne soit avec nous.
\VS{27}Ton serviteur, mon père, nous répondit : Vous savez que ma femme m'a enfanté deux fils.
\VS{28}L’un étant sorti de chez moi, je pense qu’il a été sans doute déchiré, car je ne l’ai pas revu jusqu’à présent.
\VS{29}Si vous me prenez encore celui-ci, et qu’il lui arrive un malheur, vous ferez descendre mes cheveux blancs avec douleur dans le scheol.
\VS{30}Maintenant, si je retourne auprès de ton serviteur, mon père, sans avoir avec nous l’enfant à l’âme duquel son âme est attachée,
\VS{31}il mourra, en voyant que l’enfant n’y est pas ; et tes serviteurs feront descendre avec douleur dans le scheol les cheveux blancs de ton serviteur, notre père.
\VS{32}De plus, ton serviteur a répondu pour l'enfant, en le prenant à mon père, en disant : Si je ne te le ramène pas, je serai pour toujours coupable envers mon père.
\VS{33}Permets donc, je te prie, à ton serviteur de rester à la place de l’enfant, comme esclave de mon seigneur ; et que l’enfant remonte avec ses frères.
\VS{34}Car comment pourrai-je remonter vers mon père, si l'enfant n'est pas avec moi ? Que je ne voie point l'affliction qu'en aurait mon père !
\Chap{45}
\TextTitle{Joseph révèle son identité à ses frères}
\VerseOne{}Alors Joseph, ne pouvant plus se contenir devant tous ceux qui étaient là présents, cria : Faites sortir tout le monde ! Et il ne resta personne quand il se fit connaître à ses frères.
\VS{2}Et en pleurant, il éleva sa voix, et les Egyptiens l'entendirent, et la maison de Pharaon l'entendit aussi.
\VS{3}Et Joseph dit à ses frères : Je suis Joseph ! Mon père vit-il encore ? Mais ses frères ne pouvaient lui répondre, car ils étaient tout troublés en sa présence.
\VS{4}Joseph dit encore à ses frères : Je vous prie, approchez-vous de moi ; et ils s'approchèrent, et il leur dit : Je suis Joseph, votre frère, que vous avez vendu pour être mené en Egypte\FTNT{Ac. 7:13.}.
\VS{5}Mais maintenant ne soyez pas en peine, et n'ayez point de regret de ce que vous m'avez vendu pour être mené ici, car Dieu m'a envoyé devant vous pour la conservation de votre vie.
\VS{6}Car voici, il y a déjà deux ans que la famine est sur la terre, et il y aura encore cinq ans pendant lesquels il n'y aura ni labour ni moisson.
\VS{7}Mais Dieu m'a envoyé devant vous, pour vous faire subsister sur la terre, et vous faire vivre par une grande délivrance.
\VS{8}Maintenant donc ce n'est pas vous qui m'avez envoyé ici, mais c'est Dieu ; il m'a établi père de Pharaon, et seigneur sur toute sa maison, et gouverneur de tout le pays d'Egypte.
\VS{9}Hâtez-vous d'aller vers mon père, et dites-lui : Ainsi a dit ton fils, Joseph : Dieu m'a établi seigneur sur toute l'Egypte, descends vers moi, ne t'arrête point.
\VS{10}Et tu habiteras dans la contrée de Gosen, et tu seras près de moi, toi, tes fils, et les fils de tes fils, tes brebis, et tes bœufs, et tout ce qui est à toi.
\VS{11}Là, je te nourrirai, car il y aura encore cinq années de famine ; et ainsi tu ne périras point, toi et ta maison, et tout ce qui est à toi.
\VS{12}Et voici, vous voyez de vos yeux, et Benjamin mon frère voit aussi de ses yeux, que c'est moi qui vous parle de ma propre bouche.
\VS{13}Rapportez donc à mon père quelle est ma gloire en Egypte, et tout ce que vous avez vu ; hâtez-vous, et faites descendre ici mon père.
\VS{14}Alors il se jeta sur le cou de Benjamin, son frère, et pleura. Benjamin pleura aussi sur son cou.
\VS{15}Puis il embrassa tous ses frères et pleura sur eux ; après cela ses frères parlèrent avec lui.
\TextTitle{Jacob pardonne ses frères et fait venir son père Jacob\FTNTT{Ge. 43:9}}
\VS{16}Et le bruit se répandit dans la maison de Pharaon que les frères de Joseph étaient venus, ce qui plut fort à Pharaon et à ses serviteurs.
\VS{17}Alors Pharaon dit à Joseph : Dis à tes frères : Faites ceci : Chargez vos bêtes, et allez, retournez dans le pays de Canaan ;
\VS{18}et prenez votre père et vos familles, et revenez vers moi, et je vous donnerai le meilleur du pays d'Egypte ; et vous mangerez la graisse de la terre.
\VS{19}Tu as ordre de leur dire: Faites ceci : Prenez dans le pays d’Egypte des chars pour vos enfants et pour vos femmes; amenez votre père, et venez.
\VS{20}Ne regrettez point ce que vous laisserez, car ce qu’il y a de meilleur dans tout le pays d’Egypte sera pour vous.
\VS{21}Et les fils d'Israël firent ainsi. Et Joseph leur donna des chars selon l'ordre de Pharaon ; il leur donna aussi de la provision pour la route.
\VS{22}Il leur donna à chacun des vêtements de rechange ; et il donna à Benjamin trois cents pièces d'argent et cinq vêtements de rechange.
\VS{23}Il envoya aussi à son père dix ânes chargés des plus excellentes choses qu'il y avait en Egypte, et dix ânesses portant du blé, du pain, et des vivres à son père pour la route.
\VS{24}Il renvoya donc ses frères, et ils partirent ; et il leur dit : Ne vous querellez point en chemin.
\VS{25}Ainsi ils remontèrent d'Egypte, et vinrent dans le  pays de Canaan auprès de Jacob, leur père.
\VS{26}Et ils lui rapportèrent et lui dirent : Joseph vit encore, et même c’est lui qui gouverne tout le pays d'Egypte ; mais le cœur de Jacob resta froid, parce qu’il ne les croyait pas.
\VS{27}Et ils lui dirent toutes les paroles que Joseph leur avait dites ; puis il vit les chars que Joseph avait envoyés pour le porter ; et l'esprit de Jacob, leur père, se ranima.
\VS{28}Alors Israël dit : C'est assez ! Joseph, mon fils, vit encore ! J'irai, et je le verrai avant que je meure.
\Chap{46}
\TextTitle{Jacob en Egypte}
\VerseOne{}Israël donc partit avec tout ce qui lui appartenait, et vint à Beer-Schéba, et il offrit des sacrifices au Dieu de son père Isaac.
\VS{2}Et Dieu parla à Israël dans une vision pendant la nuit et lui dit : Jacob, Jacob ! Et il répondit : Me voici.
\VS{3}Et Dieu lui dit : Je suis le Dieu, le Dieu de ton père. Ne crains point de descendre en Egypte, car là je te ferai devenir une grande nation.
\VS{4}Je descendrai avec toi en Egypte, et je t'en ferai aussi très certainement remonter ; et Joseph te fermera les yeux avec sa main.
\VS{5}Ainsi Jacob partit de Beer-Schéba, et les fils d'Israël mirent Jacob, leur père, et leurs petits enfants, et leurs femmes, sur les chars que Pharaon avait envoyés pour le porter.
\VS{6}Ils emmenèrent aussi leur bétail et leur bien qu'ils avaient acquis dans le pays de Canaan ; et Jacob et toute sa famille avec lui vinrent en Egypte.
\VS{7}Il amena avec lui en Egypte ses fils, et les fils de ses fils, ses filles, et les filles de ses fils, et toute sa famille.
\TextTitle{Les fils de Jacob en Egypte}
\VS{8}Voici les noms des fils d'Israël qui vinrent en Egypte : Jacob et ses fils. Le premier-né de Jacob fut Ruben.
\VS{9}Et les fils de Ruben : Hénoc, Pallu, Hetsron, et Carmi.
\VS{10}Et les fils de Siméon : Jemuel, Jamin, Ohad, Jakin, Tsochar, et Saül, fils d'une Cananéenne.
\VS{11}Et les fils de Lévi : Guerschon, Kehath, et Merari.
\VS{12}Et les fils de Juda : Er, Onan, Schéla, Pérets et Zérach ; mais Er et Onan moururent au pays de Canaan. Les fils de Pérets furent Hetsron et Hamul.
\VS{13}Et les fils d'Issacar : Thola, Puva, Job et Schimron.
\VS{14}Et les fils de Zabulon : Séred, Elon et Jahleel.
\VS{15}Ce sont là les fils de Léa, qu'elle enfanta à Jacob à Paddan-Aram, avec Dina, sa fille. Ses fils et ses filles formaient en tout trente-trois personnes.
\VS{16}Et les fils de Gad : Tsiphjon, Haggi, Schuni, Etsbon, Eri, Arodi et Areéli.
\VS{17}Et les fils d'Aser : Jimna, Jischva, Jischvi, Beria et Sérach, leur sœur. Les fils de Beria : Héber et Malkiel.
\VS{18}Ce sont là les fils de Zilpa que Laban donna à Léa, sa fille ; et elle les enfanta à Jacob. En tout seize personnes.
\VS{19}Les fils de Rachel, femme de Jacob, furent Joseph et Benjamin.
\VS{20}Et il naquit à Joseph dans le  pays d'Egypte, Manassé et Ephraïm, qu'Asnath, fille de Poti-Phéra, prêtre d'On, lui enfanta.
\VS{21}Et les fils de Benjamin étaient Béla, Béker, Aschbel, Guéra, Naaman, Ehi, Rosch, Muppim, Huppim et Ard.
\VS{22}Ce sont là les fils de Rachel, qu'elle enfanta à Jacob. En tout quatorze personnes.
\VS{23}Et les fils de Dan : Huschim.
\VS{24}Et les enfants de Nephthali : Jahtseel, Guni, Jetser, et Schillem.
\VS{25}Ce sont là les fils de Bilha, que Laban donna à Rachel, sa fille, et elle les enfanta à Jacob. En tout sept personnes.
\VS{26}Toutes les personnes appartenant à Jacob qui vinrent en Egypte, et qui étaient issues de lui, sans les femmes des fils de Jacob, furent en tout soixante-dix.
\VS{27}Et les fils de Joseph qui lui étaient nés en Egypte furent deux personnes. Toutes les personnes de la maison de Jacob qui vinrent en Egypte furent soixante-dix.
\VS{28}Jacob envoya Juda devant lui vers Joseph, pour l’informer qu’il se rendait en Gosen. Ils vinrent donc dans la contrée de Gosen.
\VS{29}Et Joseph fit atteler son char, et y monta pour aller à la rencontre d'Israël, son père, en Gosen. Dès qu’il le vit, il se jeta à son cou, et pleura longtemps sur son cou.
\VS{30}Et Israël dit à Joseph : Que je meure à présent, puisque j'ai vu ton visage, et que tu vis encore.
\VS{31}Puis Joseph dit à ses frères et à la famille de son père : Je monterai pour informer Pharaon, et je lui dirai : Mes frères et la famille de mon père, qui étaient au pays de Canaan, sont arrivés auprès de moi.
\VS{32}Et ces hommes sont bergers, ils se sont toujours occupés du bétail, et ils ont amené leurs brebis et leurs bœufs, et tout ce qui était à eux.
\VS{33}Et quand Pharaon vous fera appeler et vous dira : Quel est votre métier ?
\VS{34}Vous direz : Tes serviteurs se sont toujours occupés de bétail dès leur jeunesse jusqu'à maintenant, nous, et nos pères. De cette manière, vous habiterez dans le pays de Gosen, car les Egyptiens ont en abomination les bergers.
\Chap{47}
\TextTitle{La famille de Jacob honorée en Egypte}
\VerseOne{}Joseph alla avertir Pharaon, et lui dit : Mon père  et mes frères sont arrivés du pays de Canaan avec leurs troupeaux et leurs bœufs, et tout ce qui est à eux ; et voici, ils sont dans le pays de Gosen.
\VS{2}Et il prit une partie de ses frères, à savoir cinq, et il les présenta à Pharaon.
\VS{3}Et Pharaon dit aux frères de Joseph : Quel est votre métier ? Ils répondirent à Pharaon : Tes serviteurs sont bergers, comme l'ont été nos pères.
\VS{4}Ils dirent aussi à Pharaon : Nous sommes venus séjourner comme étrangers dans ce pays, parce qu'il n'y a plus de pâturages pour les troupeaux de tes serviteurs, et il y a une grande famine au pays de Canaan ; maintenant nous te prions que tes serviteurs demeurent dans le pays de Gosen.
\VS{5}Et Pharaon parla à Joseph et lui dit : Ton père et tes frères sont arrivés auprès de toi.
\VS{6}Le pays d'Egypte est à ta disposition ; fais habiter ton père et tes frères dans le meilleur endroit du pays ; qu'ils demeurent dans la terre de Gosen ; et si tu connais parmi eux des hommes habiles tu les établiras chefs de tous mes troupeaux.
\VS{7}Alors Joseph amena Jacob, son père, et le présenta à Pharaon ; et Jacob bénit Pharaon.
\VS{8}Et Pharaon dit à Jacob : Quel est le nombre de jours de tes années ?
\VS{9}Jacob répondit à Pharaon : Les jours des années de mes pèlerinages sont de cent trente ans ; les jours des années de ma vie ont été courts et mauvais et n'ont point atteint les jours des années de la vie de mes pères, du temps de leurs pèlerinages.
\VS{10}Jacob donc bénit Pharaon, et sortit de devant lui.
\VS{11}Et Joseph assigna une demeure à son père et à ses frères, et leur donna une possession au pays d'Egypte, au meilleur endroit du pays, dans le pays d'Egypte, comme Pharaon l'avait ordonné.
\VS{12}Et Joseph fournit du pain à son père et à ses frères, et à toute la maison de son père, selon le nombre de leurs familles.
\VS{13}Or il n'y avait point de pain sur toute la terre, car la famine était très grande ; et le pays d'Egypte et le pays de Canaan étaient épuisés par la famine.
\VS{14}Et Joseph amassa tout l'argent qui se trouva dans le pays d'Egypte, et dans le pays de Canaan, contre le blé qu'on achetait ; et il apporta l'argent à la maison de Pharaon.
\VS{15}Quand l'argent du pays d'Egypte et du pays de Canaan fut épuisé, tous les Egyptiens vinrent à Joseph en disant : Donne-nous du pain ; et pourquoi mourrions-nous en ta présence, parce que l'argent manque ?
\VS{16}Joseph répondit : Donnez votre bétail, et je vous en donnerai pour votre bétail, puisque l'argent manque.
\VS{17}Alors ils amenèrent à Joseph leur bétail, et Joseph leur donna du pain pour des chevaux, pour des troupeaux de brebis, pour des troupeaux de boeufs, et pour des ânes ; ainsi il leur fournit du pain en échange de leurs troupeaux cette année-là.
\VS{18}Lorsque cette année fut écoulée, ils revinrent à Joseph l'année suivante et lui dirent : Nous ne cacherons point à mon seigneur que l'argent est épuisé et les troupeaux de bétail ont été amenés à mon seigneur, il ne nous reste plus rien devant mon seigneur que nos corps et nos terres.
\VS{19}Pourquoi mourrions-nous sous tes yeux ? Achète-nous avec nos terres, pour du pain ; et nous serons esclaves de Pharaon, et nos terres seront à lui ; donne-nous aussi de quoi semer, afin que nous vivions et ne mourions point, et que nos terres ne soient point désolées.
\VS{20}Ainsi, Joseph acheta toutes les terres de l’Egypte pour Pharaon ; car les Egyptiens vendirent chacun son champ, parce que la famine les pressait. Et le pays devint la propriété de Pharaon.
\VS{21}Et il fit passer le peuple dans les villes, d’un bout à l’autre des frontières de l’Egypte.
\VS{22}Seulement, il n’acheta point les terres des prêtres, parce qu’il y avait une loi de Pharaon en faveur des prêtres, qui vivaient du revenu que leur assurait Pharaon, c’est pourquoi ils ne vendirent point leurs terres.
\VS{23}Et Joseph dit au peuple : Voici, je vous ai achetés aujourd'hui, vous et vos terres pour Pharaon, voilà de la semence pour ensemencer la terre.
\VS{24}Et quand le temps de la récolte viendra, vous donnerez la cinquième partie à Pharaon, et les quatre autres seront à vous, pour ensemencer les champs, et pour votre nourriture, et pour celle de ceux qui sont dans vos maisons, et pour la nourriture de vos petits enfants.
\VS{25}Et ils dirent : Tu nous sauves la vie ! Que nous trouvions grâce aux yeux de mon seigneur, et nous serons esclaves de Pharaon.
\VS{26}Et Joseph fit de cela une loi qui a subsisté jusqu’à ce jour, et d’après laquelle un cinquième du revenu des terres de l’Egypte appartient à Pharaon ; il n’y a que les terres des prêtres qui ne soient point à Pharaon.
\TextTitle{Jacob demande à être enterré à Canaan}
\VS{27}Israël habita dans le pays d’Egypte, dans le pays de Gosen. Ils eurent des possessions, ils furent féconds et multiplièrent beaucoup.
\VS{28}Jacob vécut dix-sept ans dans le pays d’Egypte ; et les jours des années de la vie de Jacob furent de cent quarante-sept ans.
\VS{29}Et quand le jour de la mort d'Israël approcha, il appela Joseph, son fils, et lui dit : Je te prie, si j'ai trouvé grâce à tes yeux, mets présentement ta main sous ma cuisse, et jure-moi que tu useras envers moi de bonté et de fidélité : Je te prie, ne m'enterre point en Egypte !
\VS{30}Quand  je serai couché avec mes pères, tu me transporteras hors de l'Egypte, et m'enterreras dans leur sépulcre. Et il répondit : Je le ferai selon ta parole.
\VS{31}Et Jacob lui dit : Jure-le-moi ; et il le lui jura. Et Israël se prosterna sur le chevet du lit.
\Chap{48}
\TextTitle{Bénédiction de Jacob sur les fils de Joseph}
\VerseOne{}Or il arriva après ces choses que l'on vint dire à Joseph : Voici, ton père est malade. Et il prit avec lui ses deux fils, Manassé et Ephraïm.
\VS{2}On avertit Jacob et on lui dit : Voici Joseph, ton fils, qui vient vers toi. Alors Israël rassembla ses forces et s’assit sur son lit.
\VS{3}Puis Jacob dit à Joseph : Le Dieu Tout-Puissant m’est apparu  à Luz, au pays de Canaan, et m’a béni.
\VS{4}Et il m’a dit : Voici, je te ferai croître et multiplier, et je te ferai devenir une assemblée de peuples, et je donnerai ce pays en possession perpétuelle à ta postérité après toi.
\VS{5}Et maintenant tes deux fils, qui te sont nés au pays d'Egypte, avant mon arrivée vers toi, seront à moi : Ephraïm et Manassé seront à moi comme Ruben et Siméon.
\VS{6}Mais les enfants que tu auras engendrés après eux, seront à toi, et ils seront appelés selon le nom de leurs frères dans leur héritage.
\VS{7}A mon retour de Paddan, Rachel mourut en route auprès de moi, dans le pays de Canaan, à quelque distance d’Ephrata ; et c’est là que je l’ai enterrée, sur le chemin d’Ephrata, qui est Bethléhem.
\VS{8}Puis Israël vit les fils de Joseph, et il dit : Qui sont ceux-ci ?
\VS{9}Et Joseph répondit à son père : Ce sont mes fils que Dieu m'a donnés ici ; et il dit : Amène-les-moi, je te prie, afin que je les bénisse.
\VS{10}Or les yeux d'Israël étaient appesantis par la vieillesse, et il ne pouvait plus voir ; et il les fit approcher de lui, les embrassa et les prit dans ses bras.
\VS{11}Et Israël dit à Joseph : Je ne pensais pas revoir ton visage ; et voici, Dieu m'a fait voir et toi et ta postérité.
\VS{12}Et Joseph les retira des genoux de son père, et se prosterna le visage contre terre.
\VS{13}Puis Joseph les prit tous deux, Ephraïm de sa main droite à la gauche d’Israël, et Manassé de sa main gauche à la droite d’Israël, et il les fit approcher de lui.
\VS{14}Israël étendit sa main droite et la posa sur la tête d’Ephraïm qui était le plus jeune, et il posa sa main gauche sur la tête de Manassé ; ce fut avec intention qu’il posa ses mains ainsi, car Manassé était le premier-né.
\VS{15}Il bénit Joseph et dit : Que le Dieu en présence duquel ont marché mes pères, Abraham et Isaac, que le Dieu qui m’a conduit depuis que j’existe jusqu’à ce jour\FTNT{Hé. 11:21.},
\VS{16}que l’Ange qui m’a délivré de tout mal, bénisse ces enfants ! Qu’ils soient appelés de mon nom et du nom de mes pères, Abraham et Isaac, et qu’ils multiplient en abondance comme les poissons au milieu du pays.
\VS{17}Joseph vit avec déplaisir que son père posait sa main droite sur la tête d’Ephraïm ; il saisit la main de son père, pour la détourner de dessus la tête d’Ephraïm, et la diriger sur celle de Manassé.
\VS{18}Et Joseph dit à son père : Ce n'est pas ainsi mon père ! Car celui-ci est l'aîné ; mets ta main droite sur sa tête.
\VS{19}Mais son père le refusa en disant : Je le sais, mon fils, je le sais. Celui-ci deviendra aussi un peuple, et même il sera grand ; mais toutefois son frère, qui est plus jeune, sera plus grand que lui, et sa postérité sera une multitude de nations.
\VS{20}Il les bénit ce jour-là et dit : C’est par toi qu’Israël bénira en disant : Que Dieu te traite comme Ephraïm et comme Manassé ! Et il mit Ephraïm avant Manassé.
\VS{21}Puis Israël dit à Joseph : Voici, je  vais mourir, mais Dieu sera avec vous, et vous fera retourner au pays de vos pères.
\VS{22}Et je te donne une portion  de plus qu'à tes frères, celle que j'ai prise avec mon épée et mon arc sur les Amoréens.
\Chap{49}
\TextTitle{Prophétie de Jacob qui bénit ses fils}
\VerseOne{}Puis Jacob appela ses fils et leur dit : Assemblez-vous, et je vous annoncerai ce qui vous arrivera dans les derniers jours\FTNT{L’expression «~dans les derniers jours~» vient de l’hébreu «~achariyth~» qui veut dire «~dernier~». Son équivalent grec est «~eschatos~» : «~dernier~», «~extrémité~» etc. Jacob est le premier homme à avoir utilisé  cette expression. Cette promesse de Jacob devait arriver à Israël dans les derniers jours, selon leurs tribus. Ainsi, les promesses du droit d'aînesse de Ge. 49 étaient pour l'âge messianique, lequel est associé aux derniers jours, et a commencé à la Fête de la Pentecôte (Ac. 2:14-21). 
Ces jours impliquent :
- L’effusion de l’Esprit, le réveil de l’Eglise de Christ (Mt. 25:1-13 ; Ac. 2)
- Le réveil des faux prophètes ou l’apostasie (2 Pi. 3:3 ; 1 Jn. 2)
- La dégradation de la moralité (2 Ti. 3)
- L’enrichissement des hommes de ce monde (Ja. 5:3 ; Ap. 3:14-22)
- Le fait que Dieu nous parle par le Fils (Hé. 1:2)
- La future résurrection des saints lors du retour du Messie (Jn. 6:39-54 ; 1 Th. 4:12-17).Le temps des nations (fin des temps) s’achèvera lors du retour visible de Jésus-Christ pour établir son règne sur toute la terre. Le temps des nations a commencé lorsque, à la suite de l’infidélité d’Israël, la gloire de Dieu a quitté le temple et la ville de Jérusalem (Ez. 11), la puissance fut confiée aux nations en la personne de Nebucadnetsar qui s’empara de Jérusalem (2 R. 24 et 25 ; 2 Ch. 36:6-21 ; Da. 1 ; Jé 39). Ces temps dureront jusqu’à la destruction finale du dernier empire des nations représenté par la Bête romaine ressuscitée (Ap. 13:3). Cette destruction n’aura lieu que lorsque Jésus-Christ, la pierre détachée sans le secours d’aucune main, deviendra une grande montagne qui remplira toute la terre (Da. 2:34 ; Mi. 4). Jérusalem ne sera délivrée du joug des nations qu’à ce moment-là. Les temps des nations ne seront accomplis que lorsque le trône de Dieu sera de nouveau établi à Jérusalem.}.
\VS{2}Rassemblez-vous, et écoutez, fils de Jacob ; écoutez Israël\FTNT{«~Ecoutez Israël~» : Le «~shema~» Israël est le texte principal de la liturgie juive. Composé de trois extraits de la Torah, on le récite matin et soir accompagné de bénédictions. Voir De. 6:4-9.}, votre père.
\VS{3}Ruben, tu es mon premier-né, ma force et le commencement de ma vigueur, qui excelle en dignité et qui excelle aussi en force ;
\VS{4}impétueux comme les eaux ; tu n'auras pas la prééminence, car tu es monté sur la couche de ton père, et tu as souillé mon lit en y montant.
\VS{5}Siméon et Lévi, sont frères, leurs glaives sont des instruments de violence dans leurs demeures.
\VS{6}Que mon âme n'entre point dans leur conseil secret, que ma gloire ne soit point jointe à leur compagnie, car ils ont tué les gens dans leur colère, et ont enlevé les bœufs pour leur plaisir.
\VS{7}Maudite soit leur colère, car elle a été violente ; et leur fureur, car elle a été cruelle ; je les diviserai dans Jacob, et les disperserai dans Israël.
\VS{8}Juda, quant à toi, tes frères te loueront ; ta main sera sur la nuque de tes ennemis ; les fils de ton père se prosterneront devant toi.
\VS{9}Juda est un jeune lion. Mon fils, tu reviens du carnage, mon fils ! Il ploie les genoux, il se couche comme un lion, comme une lionne : Qui le fera lever ?
\VS{10}Le sceptre ne s’éloignera point de Juda, ni le bâton de législateur d'entre ses pieds, jusqu'à ce que le Schilo vienne, et que  les peuples lui obéissent.
\VS{11}Il attache à la vigne son ânon, et au cep excellent le petit de son ânesse ; il lavera son vêtement dans le vin, et son vêtement dans le sang des raisins.
\VS{12}Il a les yeux rouges de vin, et les dents blanches de lait.
\VS{13}Zabulon habitera sur la côte des mers, il sera un port des navires ; et ses côtés s'étendront vers Sidon.
\VS{14}Issacar est un âne robuste, couché entre les barres des étables.
\VS{15}Il voit que le lieu où il repose est agréable et que la contrée est magnifique ; et il courbe son épaule sous le fardeau, il s’assujettit à un tribut.
\VS{16}Dan jugera son peuple, comme l’une des tribus d'Israël.
\VS{17}Dan sera un serpent sur le chemin, une vipère sur le sentier, mordant les talons du cheval, pour que le cavalier tombe à la renverse.
\VS{18}Ô Yahweh ! J’espère en ton salut\FTNT{Le mot secours se dit «~yeshuw`ah~» en hébreu et veut littéralement dire  «~salut~», «~délivrance~». Avant de mourir, Jacob a  donc  placé son espérance en Jésus-Christ qui est la résurrection et la vie (Jn. 11:25). Voir commentaire en Es. 26:1.} !
\VS{19}Quant à Gad, des troupes viendront l’attaquer, mais il ravagera leur arrière-garde.
\VS{20}Le pain excellent viendra d'Aser, et il fournira les mets délicats des rois.
\VS{21}Nephtali est une biche en liberté ; il profère des belles paroles.
\VS{22}Joseph est un fils fertile, un rameau fertile près d'une fontaine ; ses branches se sont étendues sur la muraille.
\VS{23}Des archers l’ont provoqué, ils ont lancé des traits ; les archers l’ont poursuivi de leur haine.
\VS{24}Mais son arc est demeuré ferme, et ses mains ont été fortifiées par les mains du Puissant de Jacob : Il est ainsi devenu le pasteur, le rocher d’Israël.
\VS{25}C’est l’œuvre du Dieu de ton père qui t’aidera ; c’est l’œuvre du Tout-Puissant qui te bénira des bénédictions des cieux en haut, des bénédictions des eaux en bas, des bénédictions des mamelles et du sein maternel.
\VS{26}Les bénédictions de ton père ont surpassé les bénédictions de ceux qui m'ont engendré, jusqu'à la cime des antiques collines ; elles seront sur la tête de Joseph, et sur le sommet de la tête du Nazaréen d'entre ses frères.
\VS{27}Benjamin est un loup qui déchirera ; le matin il dévorera la proie, et sur le soir il partagera le butin.
\VS{28}Ce sont là tous ceux qui forment les douze tribus d'Israël.  Et c’est là ce que leur père leur dit en les bénissant. Il bénit chacun d'eux selon la bénédiction qui lui était propre.
\VS{29}Il leur donna aussi cet ordre : Je vais être recueilli auprès de mon peuple, enterrez-moi avec mes pères dans la caverne qui est au champ d'Ephron, le Héthien,
\VS{30}dans la caverne du champ de Macpéla, vis-à-vis de Mamré, dans le pays de Canaan. C’est le champ qu’Abraham a acheté d’Ephron, le Héthien, comme propriété sépulcrale.
\VS{31}C'est là qu'on a enterré Abraham avec Sara, sa femme ; c'est là qu'on a enterré Isaac et Rebecca, sa femme ; et c'est là que j'ai enterré Léa.
\VS{32}Le champ a été acquis des fils de Heth avec la caverne qui s’y trouve.
\VS{33}Lorsque Jacob eut achevé de donner ses ordres à ses fils, il retira ses pieds dans le lit, il expira, et fut recueilli auprès de son peuple.
\Chap{50}
\TextTitle{Mort de Jacob}
\VerseOne{}Alors Joseph se jeta sur le visage de son père, pleura sur lui et l’embrassa.
\VS{2}Et Joseph ordonna à ceux de ses serviteurs qui étaient médecins d'embaumer son père ; et les médecins embaumèrent Israël.
\VS{3}Et on employa quarante jours à l'embaumer, car c'était la coutume d'embaumer les corps pendant quarante jours ; et les Egyptiens le pleurèrent soixante-dix jours.
\VS{4}Quand les jours du deuil furent passés, Joseph s’adressa aux gens de la maison de Pharaon, et leur dit : Si j’ai trouvé grâce à vos yeux, rapportez, je vous prie, à Pharaon ce que je vous dis.
\VS{5}Mon père m’a fait jurer en disant : Voici, je vais mourir ! Tu m’enterreras dans le sépulcre que je me suis acheté au pays de Canaan. Je voudrais donc y monter, pour enterrer mon père ; et je reviendrai.
\VS{6}Et Pharaon répondit : Monte, et enterre ton père comme il t'a fait jurer.
\VS{7}Alors Joseph monta pour enterrer son père, et les serviteurs de Pharaon, les anciens de la maison de Pharaon, et tous les anciens du pays d'Egypte montèrent avec lui.
\VS{8}Et toute la maison de Joseph, et ses frères, et la maison de son père y montèrent aussi, laissant seulement leurs familles, et leurs troupeaux, et leurs bœufs dans le pays de Gosen.
\VS{9}Il y avait encore avec Joseph des chars et des cavaliers, en sorte que le cortège était très nombreux.
\VS{10}Arrivés à l’aire d’Athad, qui est au-delà du Jourdain, ils firent entendre de grandes et profondes lamentations ; et Joseph fit en l’honneur de son père un deuil de sept jours.
\VS{11}Et les Cananéens, habitants du pays, voyant ce deuil dans l'aire d'Athad, dirent : Ce deuil est grand pour les Egyptiens ; c'est pourquoi cette aire, qui est au-delà du Jourdain, fut nommée Abel-Mitsraïm\FTNT{Abel-Mitsraïm : «~Pré du deuil de l’Egypte~».}.
\VS{12}Les fils de Jacob firent à l'égard de son corps ce qu'il leur avait ordonné.
\VS{13}Ils le transportèrent au pays de Canaan, et l’enterrèrent dans la caverne du champ de Macpéla, qu’Abraham avait achetée d’Ephron, le Héthien, comme propriété sépulcrale, et qui est vis-à-vis de Mamré.
\VS{14}Et après que Joseph eut enseveli son père, il retourna en Egypte avec ses frères et tous ceux qui étaient montés avec lui pour ensevelir son père.
\VS{15}Et les frères de Joseph, voyant que leur père était mort, se dirent entre eux : Peut-être que Joseph nous aura en haine, et ne manquera pas de nous rendre tout le mal que nous lui avons fait.
\VS{16}Et ils firent dire à Joseph : Ton père a donné cet ordre avant de mourir en disant :
\VS{17}Vous parlerez ainsi à Joseph : Je te prie, pardonne maintenant l'iniquité de tes frères, et leur péché, car ils t'ont fait du mal. Maintenant, je te supplie, pardonne le crime des serviteurs du Dieu de ton père. Et Joseph pleura quand on lui parla.
\VS{18}Ses frères vinrent eux-mêmes se prosterner devant lui, et ils dirent : Nous sommes tes serviteurs.
\VS{19}Et Joseph leur dit : Ne craignez point, car suis-je à la place de Dieu ?
\VS{20}Vous aviez médité de me faire du mal : Dieu l’a changé en bien, pour accomplir ce qui arrive aujourd’hui, pour sauver la vie à un peuple nombreux.
\VS{21}Soyez donc sans crainte ; je vous entretiendrai, vous et vos familles ; et il les consola en parlant à leur cœur.
\VS{22}Joseph demeura donc en Egypte, lui et la maison de son père, et vécut cent dix ans.
\VS{23}Et Joseph vit les fils d'Ephraïm jusqu'à la troisième génération. Makir aussi, fils de Manassé, eut des fils qui furent élevés sur les genoux de Joseph.
\VS{24}Et Joseph dit à ses frères : Je vais mourir ! Mais Dieu ne manquera pas de vous visiter, et il vous fera remonter de ce pays au pays qu’il a juré  de donner à Abraham, Isaac et à Jacob.
\VS{25}Et Joseph fit jurer les enfants d'Israël et leur dit : Dieu ne manquera pas de vous visiter, et alors vous transporterez mes os d'ici\FTNT{Hé. 11:22 ; Ex. 13:19.}.
\VS{26}Puis Joseph mourut, âgé de cent dix ans. On l'embauma, et on le mit dans un cercueil en Egypte.
\PPE{}
\end{multicols}

%\clearpage\ShortTitle{Ex.}\BookTitle{Exode}\BFont
\noindent\hrulefill
{\footnotesize
\textit{
\bigskip
{\centering{}
\\Auteur~: Probablement Moïse
\\(Heb.~: Shemot)
\\Signification~: Noms
\\Thème~: La délivrance
\\Date de rédaction~: Env. 1450-1410 av. J.-C.\\}
}
%\bigskip
\textit{
\\Les fils de Jacob s'étaient retrouvés en Egypte pour survivre à une famine qui avait frappé la terre entière pendant plusieurs années. Grâce à leur frère Joseph, alors gouverneur d'Egypte, ils bénéficièrent d'un bon traitement. Mais la mort de ce dernier et la montée au pouvoir d'un nouveau Pharaon (probablement Ramsès II) inaugurèrent une période de quatre siècles de souffrances pour le peuple élu.
%\bigskip
\\En effet, les Hébreux avaient été réduits en esclavage. En réponse aux cris de douleur de son peuple, Dieu suscita Moïse, dont le nom signifie «~tiré de~». Ce descendant de Lévi fut élevé dans le palais de Pharaon, mais dut s'enfuir parce qu'il avait tué un Egyptien. Après quarante ans passés dans le pays de Madian, le Dieu qui s'appelle «~Je suis~» se révéla à Moïse sur la montagne d'Horeb et lui confia la mission d'aller délivrer son peuple du joug égyptien.
%\bigskip
\\Ce livre retrace la sortie d'Egypte et le début de la traversée du désert, jalonnée de prodiges exceptionnels.\bigskip
}
}
\par\nobreak\noindent\hrulefill
\begin{multicols}{2}
\Chap{1}
\TextTitle{Après la mort de Joseph}
\VerseOne{}Et ce sont ici les noms des fils d'Israël qui entrèrent en Egypte avec Jacob. Ils y entrèrent chacun avec sa famille~: 
\VS{2}Ruben, Siméon, Lévi, et Juda,
\VS{3}Issacar, Zabulon, et Benjamin,
\VS{4}Dan, et Nephthali, Gad, et Aser.
\VS{5}Toutes les personnes issues des reins de Jacob étaient soixante-dix âmes. Joseph était alors en Egypte.
\VS{6}Joseph mourut ainsi que tous ses frères et toute cette génération-là.
\VS{7}Les enfants d'Israël fructifièrent et s'accrurent abondamment, et se multiplièrent et devinrent extrêmement puissants, de sorte que le pays en fut rempli\FTNT{De. 26:5~; Ac. 7:17.}.
\TextTitle{Israël esclave en Egypte}
\VS{8}Depuis, il s'éleva un nouveau roi sur l'Egypte, qui n'avait point connu Joseph.
\VS{9}Et il dit à son peuple~: Voici, le peuple des enfants d'Israël est plus grand et plus puissant que nous.
\VS{10}Agissons donc prudemment avec lui, de peur qu'il ne se multiplie, et que s'il survenait une guerre, il ne se joigne à nos ennemis, ne fasse la guerre contre nous, et qu'il ne s'en aille du pays.
\VS{11}Ils établirent donc sur le peuple des commissaires d'impôts, pour l'affliger en le surchargeant~; car le peuple bâtit des villes à greniers pour Pharaon~; à savoir Pithom et Ramsès.
\VS{12}Mais plus ils l'affligeaient et plus il multipliait et croissait en toute abondance~; c'est pourquoi ils haïssaient les enfants d'Israël\FTNT{Ps. 105:24.}.
\VS{13}Et les Egyptiens assujettirent les enfants d'Israël à une rude servitude\FTNT{Ge. 15:13.}.
\VS{14}Tellement qu'ils leur rendirent la vie amère par un rude travail, en les employant à faire du mortier, des briques, et toute sorte d'ouvrage qui se fait aux champs~; c'était avec cruauté qu'ils leur imposaient toutes ces charges.
\VS{15}Le roi d'Egypte parla aussi aux sages-femmes des Hébreux, nommées l'une Schiphra et l'autre Pua.
\VS{16}Il leur dit~: Quand vous accoucherez les femmes des Hébreux, et que vous les verrez sur les sièges, si c'est un fils, mettez-le à mort~; mais si c'est une fille, qu'elle vive.
\VS{17}Mais les sages-femmes craignirent Dieu et ne firent pas ce que le roi d'Egypte leur avait dit~; car elles laissèrent vivre les fils.
\VS{18}Alors le roi d'Egypte appela les sages-femmes et leur dit~: Pourquoi avez-vous fait cela, et avez-vous laissé vivre les fils~?
\VS{19}Les sages-femmes répondirent à Pharaon~: Parce que les femmes des Hébreux ne sont point comme les femmes Egyptiennes~; car elles sont vigoureuses, elles ont accouché avant que la sage-femme ne soit arrivée chez elles.
\VS{20}Dieu fit du bien aux sages-femmes~; et le peuple multiplia et devint très puissant.
\VS{21}Parce que les sages-femmes craignirent Dieu, il leur édifia des maisons.
\VS{22}Alors Pharaon donna cet ordre à tout son peuple~: Jetez dans le fleuve tous les fils qui naîtront, mais laissez vivre toutes les filles.
\Chap{2}
\TextTitle{Naissance de Moïse\FTNTT{Hé. 11:23-27.}}
\VerseOne{}Un homme de la maison de Lévi s'en alla et prit une fille de Lévi\FTNT{No. 26:59.}.
\VS{2}Cette femme conçut et enfanta un fils. Voyant qu'il était beau, elle le cacha pendant trois mois\FTNT{Hé. 11:23}.
\VS{3}Mais ne pouvant le tenir caché plus longtemps, elle prit une arche de jonc, et l'enduisit de bitume et de poix, mit l'enfant dedans, et le posa parmi des roseaux sur le bord du fleuve.
\VS{4}Et la sœur de cet enfant se tenait loin pour savoir ce qu'il en arriverait.
\VS{5}La fille de Pharaon descendit à la rivière pour se baigner, et ses servantes se promenaient sur le bord de la rivière, et ayant vu le coffret au milieu des roseaux, elle envoya une de ses servantes pour le prendre.
\VS{6}Et l'ayant ouvert, elle vit l'enfant et voici l'enfant pleurait. Elle en fut touchée de compassion et dit~: C'est un des enfants de ces Hébreux~!
\VS{7}Alors la sœur de l'enfant dit à la fille de Pharaon~: Irai-je appeler une femme d'entre les Hébreux, qui allaite~? Et elle t'allaitera cet enfant.
\VS{8}La fille de Pharaon lui répondit~: Va~! Et la jeune fille s'en alla et appela la mère de l'enfant.
\VS{9}Et La fille de Pharaon lui dit~: Emporte cet enfant, et allaite-le moi, je te donnerai ton salaire~; et la femme prit l'enfant et l'allaita.
\VS{10}Quand l'enfant fut devenu grand, elle l'amena à la fille de Pharaon~; il fut pour elle comme un fils. Elle lui donna le nom de Moïse parce que, dit-elle, je l'ai tiré des eaux.
\TextTitle{Moïse prend à cœur le sort d'Israël~; fuite à Madian}
\VS{11}Or il arriva, en ce temps-là, que Moïse, étant devenu grand, sortit vers ses frères et vit leurs travaux~; il vit aussi un Egyptien qui frappait un Hébreu d'entre ses frères\FTNT{Hé. 11:24-25.}.
\VS{12}Et ayant regardé çà et là, et voyant qu'il n'y avait personne, il tua l'Egyptien et le cacha dans le sable.
\VS{13}Il sortit encore le second jour~; et voici, deux hommes Hébreux se querellaient. Il dit à celui qui avait tort~: Pourquoi frappes-tu ton prochain~?
\VS{14}Lequel répondit~: Qui t'a établi prince et juge sur nous~? Veux-tu me tuer comme tu as tué l'Egyptien~? Et Moïse craignit, et dit~: Certainement le fait est connu.
\VS{15}Or Pharaon ayant appris ce fait-là, chercha à faire mourir Moïse~; mais Moïse s'enfuit de devant Pharaon, s'arrêta au pays de Madian et s'assit près d'un puits.
\VS{16}Or le prêtre de Madian avait sept filles qui vinrent puiser de l'eau, et elles remplirent les auges pour abreuver le troupeau de leur père.
\VS{17}Mais des bergers survinrent et les chassèrent~; et Moïse se leva et les secourut, et abreuva leur troupeau.
\VS{18}Et quand elles furent revenues chez Réuel, leur père, il leur dit~: Comment êtes-vous revenues si tôt aujourd'hui~?
\VS{19}Elles répondirent~: Un homme Egyptien nous a délivrées de la main des bergers~; et même il nous a puisé abondamment de l'eau et a abreuvé le troupeau.
\VS{20}Il dit à ses filles~: Où est-il~? Pourquoi avez-vous ainsi laissé cet homme~? Appelez-le, et qu'il mange du pain.
\VS{21}Et Moïse s'accorda de demeurer avec cet homme-là, qui donna Séphora, sa fille, à Moïse.
\VS{22}Et elle enfanta un fils, et il le nomma Guerschom~: Car, dit-il, je séjourne dans un pays étranger.
\TextTitle{Yahweh entend les cris de son peuple}
\VS{23}Or il arriva longtemps après que le roi d'Egypte mourut, et les enfants d'Israël soupirèrent à cause de la servitude, et ils crièrent~; et leur cri monta jusqu'à Dieu, à cause de la servitude\FTNT{No. 20:15-16.}.
\VS{24}Dieu entendit leurs gémissements, et Dieu se souvint de l'alliance qu'il avait traitée avec Abraham, Isaac et Jacob.
\VS{25}Ainsi Dieu regarda les enfants d'Israël et il fit attention à leur état.
\Chap{3}
\TextTitle{Yahweh se révèle à Moïse dans le buisson ardent}
\VerseOne{}Or Moïse fut berger du troupeau de Jéthro, son beau-père, prêtre de Madian~; il mena le troupeau derrière le désert, et vint à la montagne de Dieu à Horeb.
\VS{2}Et l'Ange de Yahweh lui apparut dans une flamme de feu, du milieu d'un buisson. Il regarda, et voici, le buisson était tout en feu, et le buisson ne se consumait point.
\VS{3}Alors Moïse dit~: Je me détournerai maintenant, et je regarderai cette grande vision, pourquoi le buisson ne se consume point.
\VS{4}Et Yahweh vit que Moïse s'était détourné pour regarder~; et Dieu l'appela du milieu du buisson, en disant~: Moïse~! Moïse~! Et il répondit~: Me voici~!
\VS{5}Et Dieu dit~: N'approche point d'ici~; déchausse tes souliers de tes pieds, car le lieu où tu es arrêté est une terre sainte.
\VS{6}Il dit aussi~: Je suis le Dieu de ton père, le Dieu d'Abraham, le Dieu d'Isaac et le Dieu de Jacob\FTNT{Mt. 22:32~; Mc. 12:26~; Lu. 20:37~; Ac. 7:32.}~; Moïse cacha son visage, parce qu'il craignait de regarder vers Dieu.
\VS{7} Et Yahweh dit~: J'ai très bien vu l'affliction de mon peuple qui est en Egypte et j'ai entendu le cri qu'ils ont jeté à cause de leurs oppresseurs, car je connais leurs douleurs.
\VS{8}C'est pourquoi je suis descendu pour le délivrer de la main des Egyptiens et pour le faire remonter de ce pays-là, dans un pays bon et vaste, dans un pays découlant de lait et de miel~; au lieu où sont les Cananéens, les Héthiens, les Amoréens, les Phéréziens, les Héviens et les Jébusiens.
\VS{9}Et maintenant, voici le cri des enfants d'Israël est parvenu à moi, et j'ai vu aussi l'oppression dont les Egyptiens les oppriment.
\VS{10}Maintenant donc viens, et je t'enverrai vers Pharaon~; et tu retireras mon peuple, les enfants d'Israël, hors d'Egypte\FTNT{Os. 12:14~; Mi. 6:4.}.
\VS{11}Et Moïse répondit à Dieu~: Qui suis-je, moi, pour aller vers Pharaon, et pour retirer de l'Egypte les enfants d'Israël~?
\VS{12}Et Dieu lui dit~: Va, car je serai avec toi. Et tu auras ce signe que c'est moi qui t'envoie~: C'est que quand tu auras retiré mon peuple d'Egypte, vous servirez Dieu près de cette montagne.
\TextTitle{Yahweh révèle son Nom à Moïse}
\VS{13}Et Moïse dit à Dieu~: Voici, quand je serai venu vers les enfants d'Israël, et que je leur aurai dit~: Le Dieu de vos pères m'a envoyé vers vous, s'ils me disent alors~: Quel est son Nom~? Que leur dirai-je~?
\VS{14} Et Dieu dit à Moïse~: JE SUIS CELUI QUI SUIS. Il dit aussi~: Tu diras ainsi aux enfants d'Israël~: Celui qui s'appelle JE SUIS\FTNT{Je suis («~Ehyeh~» en hébreu), c'est de là que vient le Nom de Yahweh. Or le Nom de Jésus signifie «~Yahweh est Salut~». Dieu révèle son Nom à Moïse~: «~Je suis celui qui suis~». Or Jésus-Christ s'est ouvertement attribué ce Nom en Jn. 8:58. N'ayant compris ni le plan de Dieu ni qui était celui qui les visitait, les religieux Juifs ont voulu le lapider car ils estimaient qu'il blasphémait. Car en déclarant être «~Je suis~», Jésus-Christ proclamait ouvertement sa divinité (Ro. 9:5), chose que les Juifs ne pouvaient concevoir. Dans l'évangile de Jean, Jésus déclare clairement qu'il est le «~JE SUIS~» d'Ex. 3:14. «~Je suis le pain de vie~» (Jn. 6:35), «~Je suis la lumière du monde~» (Jn. 8:12), «~Je suis le bon berger~» (Jn. 10:11), «~Je suis la porte~» (Jn. 10:7), «~Je suis la résurrection~» (Jn. 11:25), «~Je suis le chemin, la vérité et la vie~» (Jn. 14:6), «~Je suis la vraie vigne~» (Jn. 15:1).}, m'a envoyé vers vous.
\VS{15}Dieu dit encore à Moïse~: Tu diras ainsi aux enfants d'Israël~: Yahweh, le Dieu de vos pères, le Dieu d'Abraham, le Dieu d'Isaac et le Dieu de Jacob m'a envoyé vers vous. C'est ici mon Nom éternellement, et c'est ici le souvenir que vous aurez de moi de génération en génération.
\VS{16}Va, et rassemble les anciens d'Israël, et dis leur~: Yahweh, le Dieu de vos pères, le Dieu d'Abraham, d'Isaac et de Jacob, m'est apparu, en disant~: Certainement je vous ai visités, et j'ai vu ce qu'on vous fait en Egypte.
\VS{17}Et j'ai dit~: Je vous ferai remonter de l'Egypte où vous êtes affligés, dans le pays des Cananéens, des Héthiens, des Amoréens, des Phéréziens, des Héviens et des Jébusiens, qui est un pays découlant de lait et de miel.
\VS{18}Et ils obéiront à ta parole~; et tu iras, toi et les anciens d'Israël, vers le roi d'Egypte, et vous lui direz~: Yahweh, le Dieu des Hébreux, est venu nous rencontrer. Et maintenant donc, laisse-nous aller, nous te prions, à trois jours de marche dans le désert, afin que nous puissions sacrifier à Yahweh, notre Dieu.
\VS{19}Or je sais que le roi d'Egypte ne vous permettra point de vous en aller, si ce n'est par une main forte.
\VS{20}Mais j'étendrai ma main et je frapperai l'Egypte par toutes les merveilles que je ferai au milieu d'elle~; et après cela, il vous laissera aller.
\VS{21}Je ferai que ce peuple trouve grâce envers les Egyptiens, et il arrivera que, quand vous partirez, vous ne vous en irez point à vide.
\VS{22}Mais chacune demandera à sa voisine, et à l'hôtesse de sa maison, des vases d'argent, des vases d'or, et des vêtements, que vous mettrez sur vos fils et sur vos filles~: Ainsi vous dépouillerez les Egyptiens.
\Chap{4}
\TextTitle{Moïse résiste en évoquant l'incrédulité du peuple}
\VerseOne{}Et Moïse répondit, et dit~: Mais voici, ils ne me croiront pas et n'obéiront pas à ma parole~; car ils diront~: Yahweh ne t'est point apparu.
\VS{2}Et Yahweh lui dit~: Qu'est-ce que tu as dans ta main~? Il répondit~: Une verge.
\VS{3}Et Dieu lui dit~: Jette-la par terre~; il la jeta par terre et elle devint un serpent. Et Moïse s'enfuyait de devant lui.
\VS{4}Et Yahweh dit à Moïse~: Etends ta main et saisis sa queue~; et il étendit sa main et l'empoigna~; et il redevint une verge dans sa main.
\VS{5}C'est là ce que tu feras, afin qu'ils croient que Yahweh, le Dieu de leurs pères, le Dieu d'Abraham, le Dieu d'Isaac et le Dieu de Jacob, t'est apparu.
\VS{6}Yahweh lui dit encore~: Mets maintenant ta main dans ton sein, et il mit sa main dans son sein~; puis il la tira~; et voici, sa main était blanche de lèpre comme la neige.
\VS{7}Et Dieu lui dit~: Remets ta main dans ton sein~; et il remit sa main dans son sein~; puis il la retira hors de son sein~; et voici, elle était redevenue comme son autre chair.
\VS{8}Mais s'il arrive qu'ils ne te croient point, et qu'ils n'obéissent point à la voix du premier signe, ils croiront à la voix du second signe.
\VS{9}S'il arrive qu'ils ne croient point à ces deux signes et qu'ils n'obéissent point à ta parole, tu prendras de l'eau du fleuve et tu la répandras sur la terre, et les eaux que tu auras prises du fleuve deviendront du sang sur la terre.
\TextTitle{Moïse résiste en évoquant son incapacité à parler}
\VS{10}Et Moïse répondit à Yahweh~: Hélas~! Seigneur~! Je ne suis point un homme qui ait, ni d'hier ni d'avant-hier, la parole aisée, ni même depuis que tu parles à ton serviteur~; car j'ai la bouche et la langue empêchées.
\VS{11}Et Yahweh lui dit~: Qui a fait la bouche de l'homme~? Ou qui a fait le muet, ou le sourd, ou le voyant, ou l'aveugle~? N'est-ce pas moi Yahweh\FTNT{Ps. 94:9.}~?
\VS{12}Va donc maintenant, je serai avec ta bouche et je t'enseignerai ce que tu auras à dire\FTNT{Lu. 12:12~; Mt. 10:19~; Mc. 13:11.}.
\VS{13}Et Moïse répondit~: Hélas~! Seigneur~! Envoie, je te prie, celui que tu dois envoyer.
\VS{14}Et la colère de Yahweh s'enflamma contre Moïse, et il lui dit~: Aaron, le Lévite, n'est-il pas ton frère~? Je sais qu'il parlera très bien, et même le voilà qui sort à ta rencontre, et quand il te verra, il se réjouira dans son cœur.
\VS{15}Tu lui parleras donc et tu mettras ces paroles dans sa bouche~; je serai avec ta bouche et avec la sienne, et je vous enseignerai ce que vous aurez à faire.
\VS{16}Et il parlera pour toi au peuple, et ainsi il te sera pour bouche, et tu lui seras pour Dieu.
\VS{17}Tu prendras aussi dans ta main cette verge, avec laquelle tu feras ces signes-là.
\TextTitle{Moïse accepte sa mission et part en Egypte}
\VS{18}Ainsi Moïse s'en alla, et retourna vers Jéthro, son beau-père, et lui dit~: Je te prie, que je m'en aille, et que je retourne vers mes frères qui sont en Egypte, pour voir s'ils vivent encore. Et Jéthro lui dit~: Va en paix~!
\VS{19}Or Yahweh dit à Moïse au pays de Madian~: Va, et retourne en Egypte~; car tous ceux qui cherchaient ta vie sont morts.
\VS{20}Moïse prit sa femme et ses fils, les mit sur un âne, et retourna au pays d'Egypte. Moïse prit aussi la verge de Dieu dans sa main.
\VS{21}Et Yahweh avait dit à Moïse~: Quand tu t'en iras pour retourner en Egypte, tu prendras garde à tous les miracles que j'ai mis dans ta main~; et tu les feras devant Pharaon~; mais j'endurcirai son cœur et il ne laissera point aller le peuple.
\VS{22}Tu diras donc à Pharaon, ainsi parle Yahweh~: Israël est mon fils, mon premier-né\FTNT{Os. 11:1.}.
\VS{23}Et je t'ai dit~: Laisse aller mon fils, afin qu'il me serve. Mais tu as refusé de le laisser aller~: Voici, je m'en vais tuer ton fils, ton premier-né.
\VS{24}Or il arriva que, comme Moïse était en chemin dans l'hotellerie, Yahweh le rencontra et chercha à le faire mourir.
\VS{25}Et Séphora prit un couteau tranchant, coupa le prépuce de son fils et le jeta à ses pieds, et dit~: Certes, tu es pour moi un époux de sang~!
\VS{26}Alors Yahweh se retira de lui~; et Séphora dit~: Epoux de sang~; à cause de la circoncision.
\TextTitle{Yahweh envoie Aaron vers Moïse}
\VS{27}Et Yahweh dit à Aaron~: Va dans le désert, au-devant de Moïse. Il y alla donc, et le rencontra sur la montagne de Dieu et l'embrassa.
\VS{28}Et Moïse raconta à Aaron toutes les paroles de Yahweh qui l'avait envoyé, et tous les signes qu'il lui avait ordonné de faire.
\VS{29}Moïse donc poursuivit son chemin avec Aaron~; et ils assemblèrent tous les anciens des enfants d'Israël.
\VS{30}Et Aaron rapporta toutes les paroles que Yahweh avait dites à Moïse, et il exécuta les signes aux yeux du peuple.
\VS{31}Et le peuple crut. Ils apprirent que Yahweh avait visité les enfants d'Israël, qu'il avait vu leur affliction~; et ils s'inclinèrent et se prosternèrent.
\Chap{5}
\TextTitle{Pharaon s'oppose à Moïse\FTNTT{Ex. 5-14.}}
\VerseOne{}Après cela, Moïse et Aaron se rendirent ensuite auprès de Pharaon et lui dirent~: Ainsi parle Yahweh, le Dieu d'Israël~: Laisse aller mon peuple, afin qu'il me célèbre une fête solennelle dans le désert.
\VS{2}Mais Pharaon dit~: Qui est Yahweh pour que j'obéisse à sa voix et que je laisse aller Israël~? Je ne connais point Yahweh et je ne laisserai point aller Israël.
\VS{3}Et ils dirent~: Le Dieu des Hébreux est venu au-devant de nous. Permets-nous de faire trois journées de marche dans le désert, et que nous sacrifions à Yahweh, notre Dieu~; de peur qu'il ne se jette sur nous par la peste ou par l'épée.
\VS{4}Et le roi d'Egypte leur dit~: Moïse et Aaron, pourquoi détournez-vous le peuple de son ouvrage~? Allez maintenant à vos charges.
\VS{5}Pharaon dit aussi~: Voici, le peuple de ce pays est maintenant en grand nombre, et vous lui feriez cesser leur travail~!
\VS{6}Et ce jour-là, Pharaon donna cet ordre aux oppresseurs établis sur le peuple et à ses commissaires, en disant~:
\VS{7}Vous ne donnerez plus de paille à ce peuple pour faire des briques comme auparavant, mais qu'ils aillent s'amasser de la paille.
\VS{8}Néanmoins, vous leur imposerez la quantité de briques qu'ils faisaient auparavant, sans en rien diminuer~; car ils sont paresseux, et c'est pour cela qu'ils crient, en disant~: Allons et sacrifions à notre Dieu~!
\VS{9}Que la servitude soit aggravée sur ces gens-là, et qu'ils s'occupent, et ne s'amusent plus à des paroles de mensonge.
\VS{10}Alors les oppresseurs du peuple et ses commissaires sortirent et dirent au peuple~: Ainsi parle Pharaon~: Je ne vous donnerai plus de paille.
\VS{11}Allez vous-mêmes et prenez de la paille où vous en trouverez~; mais il ne sera rien diminué de votre travail.
\VS{12}Alors le peuple se répandit par tout le pays d'Egypte, pour ramasser du chaume au lieu de paille.
\VS{13}Et les oppresseurs les pressaient en disant~: Achevez vos ouvrages, chaque jour sa tâche, comme lorsque la paille vous était fournie.
\VS{14}Même les commissaires des enfants d'Israël, que les oppresseurs de Pharaon avaient établis sur eux, furent battus, et on leur dit~: Pourquoi n'avez-vous point achevé votre tâche en faisant des briques hier et aujourd'hui, comme auparavant~?
\VS{15}Alors les commissaires des enfants d'Israël vinrent crier à Pharaon, en disant~: Pourquoi fais-tu ainsi à tes serviteurs~?
\VS{16}On ne donne point de paille à tes serviteurs, et toutefois on nous dit~: Faites des briques. Et voici, tes serviteurs sont battus, et ton peuple est traité comme coupable.
\VS{17}Et il répondit~: Vous êtes des paresseux, des paresseux~! C'est pourquoi vous dites~: Allons, sacrifions à Yahweh~!
\VS{18}Maintenant donc allez, travaillez~; car on ne vous donnera point de paille, et vous rendrez la même quantité de briques.
\VS{19}Les commissaires des enfants d'Israël virent qu'ils souffraient, puisqu'on disait~: Vous ne diminuerez rien de vos briques sur la tâche de chaque jour.
\VS{20}Et en sortant de chez Pharaon, ils rencontrèrent Moïse et Aaron, qui se trouvèrent au-devant d'eux~;
\VS{21}et ils leur dirent~: Que Yahweh vous regarde, et en juge, vu que vous nous avez mis en mauvaise odeur devant Pharaon et devant ses serviteurs, leur mettant l'épée à la main pour nous tuer.
\VS{22}Alors Moïse retourna vers Yahweh, et dit~: Seigneur~! Pourquoi as-tu fait maltraiter ce peuple~? Pourquoi m'as-tu envoyé~?
\VS{23}Car depuis que je suis allé vers Pharaon pour parler en ton Nom, il a maltraité ce peuple, et tu n'as point délivré ton peuple.
\Chap{6}
\TextTitle{Yahweh fortifie Moïse et rappelle son alliance avec Israël}
\VerseOne{}Et Yahweh dit à Moïse~: Tu verras maintenant ce que je ferai à Pharaon~; car il les laissera aller y étant contraint par une main puissante, étant, dis-je contraint par ma main puissante, il les chassera de son pays.
\VS{2}Dieu parla encore à Moïse et lui dit~: Je suis Yahweh.
\VS{3}Je suis apparu à Abraham, à Isaac et à Jacob, comme le Dieu Tout-Puissant, mais je n'ai point été connu d'eux par mon Nom YAHWEH.
\VS{4}J'ai aussi fait cette alliance avec eux, que je leur donnerai le pays de Canaan, le pays de leurs pèlerinages, dans lequel ils ont demeuré comme étrangers.
\VS{5}Et j'ai entendu les sanglots des enfants d'Israël, que les Egyptiens tiennent esclaves, et je me suis souvenu de mon alliance~;
\VS{6}c'est pourquoi dis aux enfants d'Israël~: Je suis Yahweh, et je vous retirerai de dessous les charges des Egyptiens, et je vous délivrerai de leur servitude, je vous rachèterai à bras étendu, et par de grands jugements.
\VS{7}Et je vous prendrai pour être mon peuple, je vous serai Dieu~; et vous connaîtrez que je suis Yahweh, votre Dieu, qui vous retire de dessous les charges des Egyptiens.
\VS{8}Et je vous ferai entrer dans le pays au sujet duquel j'ai levé ma main que je le donnerai à Abraham, à Isaac et à Jacob, et je vous le donnerai en héritage~; je suis Yahweh.
\VS{9}Moïse donc parla de cette manière aux enfants d'Israël. Mais ils n'écoutèrent point Moïse, à cause de l'angoisse de leur esprit, et à cause de leur dure servitude.
\VS{10}Et Yahweh parla à Moïse, en disant~:
\VS{11}Va, et dis à Pharaon, roi d'Egypte, qu'il laisse sortir les enfants d'Israël de son pays.
\VS{12}Alors Moïse parla devant Yahweh, en disant~: Voici, les enfants d'Israël ne m'ont point écouté, et comment Pharaon m'écoutera-t-il, moi, qui suis incirconcis des lèvres~?
\VS{13} Mais Yahweh parla à Moïse et à Aaron, et leur ordonna d'aller trouver les enfants d'Israël, et Pharaon, roi d'Egypte, pour retirer les fils d'Israël du pays d'Egypte.
\TextTitle{Les chefs d'Israël}
\VS{14}Voici les chefs des pères~: Les fils de Ruben, premier-né d'Israël~: Hénoc et Pallu, Hetsron et Carmi~; ce sont là les familles de Ruben\FTNT{Ge. 46:9~; No. 26:5~; 1 Ch. 5:3.}.
\VS{15}Les fils de Siméon~: Jemuel, Jamin, Ohad, Jakin et Tsochar, et Saül, fils d'une Cananéenne~; ce sont là les familles de Siméon.
\VS{16}Voici les noms des fils de Lévi selon leur naissance~: Guerschon, Kehath et Merari. Les années de la vie de Lévi furent de cent trente-sept ans.
\VS{17}Les fils de Guerschon~: Libni et Schimeï, selon leurs familles.
\VS{18}Les fils de Kehath~: Amram, Jitsehar, Hébron et Uziel. Et les années de la vie de Kehath furent de cent trente-trois ans.
\VS{19}Les fils de Merari~: Machli et Muschi~; ce sont là les familles de Lévi selon leurs générations.
\VS{20}Or Amram prit Jokébed, sa tante, pour femme, qui lui enfanta Aaron et Moïse~; les années de la vie d'Amram furent de cent trente-sept ans.
\VS{21}Et les fils de Jitsehar~: Koré, Népheg et Zicri.
\VS{22}Et les fils d'Uziel~: Mischaël, Eltsaphan, et Sithri.
\VS{23}Aaron prit pour femme Elischéba, fille d'Amminadab, sœur de Nachschon, qui lui enfanta Nadab, Abihu, Eléazar et Ithamar.
\VS{24}Et les fils de Koré~: Assir, Elkana, et Abiasaph. Ce sont là les familles des Korites.
\VS{25}Eléazar, fils d'Aaron, prit pour femme une des filles de Puthiel, qui lui enfanta Phinées. Ce sont là les chefs des pères des Lévites selon leurs familles.
\VS{26}Or c'est là cet Aaron et ce Moïse à qui Yahweh dit~: Retirez les enfants d'Israël du pays d'Egypte selon leurs armées.
\VS{27}Ce sont eux qui parlèrent à Pharaon, roi d'Egypte, pour retirer d'Egypte les enfants d'Israël. C'est ce Moïse et c'est cet Aaron.
\VS{28}Le jour où Yahweh parla à Moïse dans le pays d'Egypte,
\VS{29}Yahweh parla à Moïse et dit~: Je suis Yahweh~; dis à Pharaon, roi d'Egypte, toutes les paroles que je t'ai dites.
\VS{30}Et Moïse dit en présence de Yahweh~: Voici, je suis incirconcis des lèvres, comment Pharaon m'écoutera-t-il~?
\Chap{7}
\TextTitle{L'appel de Moïse confirmé}
\VerseOne{}Et Yahweh dit à Moïse~: Voici, je t'ai établi pour être Dieu à Pharaon, et Aaron, ton frère, sera ton prophète.
\VS{2}Tu diras tout ce que je t'ordonnerai, et Aaron, ton frère, parlera à Pharaon pour qu'il laisse aller les enfants d'Israël hors de son pays.
\VS{3}J'endurcirai le cœur de Pharaon, et je multiplierai mes signes et mes miracles dans le pays d'Egypte.
\VS{4}Pharaon ne vous écoutera point~; je mettrai ma main sur l'Egypte, et je sortirai mes armées, mon peuple, les enfants d'Israël, du pays d'Egypte, par de grands jugements.
\VS{5}Les Egyptiens connaîtront\FTNT{Les nations reconnaîtront que Jésus-Christ est le Dieu d'Israël lorsqu'il reviendra en Sion pour délivrer et restaurer son peuple (Za. 14).} que je suis Yahweh quand j'aurai étendu ma main sur l'Egypte, et que j'aurai retiré du milieu d'eux les enfants d'Israël.
\VS{6}Et Moïse et Aaron firent comme Yahweh leur avait ordonné~; ils firent ainsi.
\VS{7}Or Moïse était âgé de quatre-vingts ans, et Aaron de quatre-vingt-trois ans quand ils parlèrent à Pharaon.
\TextTitle{La verge d'Aaron devient un serpent}
\VS{8}Yahweh parla à Moïse et à Aaron, en disant~:
\VS{9}Quand Pharaon vous parlera, en disant~: Faites un miracle~; tu diras alors à Aaron~: Prends ta verge, jette-la devant Pharaon et elle deviendra un serpent.
\VS{10}Moïse donc et Aaron allèrent auprès de Pharaon, et firent comme Yahweh avait ordonné~; Aaron jeta sa verge devant Pharaon et devant ses serviteurs, et elle devint un serpent.
\VS{11}Mais Pharaon fit venir aussi les sages et les enchanteurs~; et les magiciens d'Egypte, et eux aussi firent autant par leurs enchantements.
\VS{12}Ils jetèrent donc chacun leurs verges et elles devinrent des serpents~; mais la verge d'Aaron engloutit leurs verges.
\VS{13}Le cœur de Pharaon s'endurcit et il ne les écouta point~; selon ce que Yahweh avait dit.
\TextTitle{Les eaux du fleuve changées en sang}
\VS{14}Yahweh dit à Moïse~: Le cœur de Pharaon est endurci, il a refusé de laisser aller le peuple.
\VS{15}Va-t'en dès le matin vers Pharaon~; voici, il sortira pour aller près de l'eau~; tu te présenteras donc devant lui sur le bord du fleuve, et tu prendras dans ta main la verge qui a été changée en serpent.
\VS{16}Et tu lui diras~: Yahweh, le Dieu des Hébreux, m'avait envoyé vers toi pour te dire~: Laisse aller mon peuple, afin qu'il me serve au désert~; mais voici, tu ne m'as point écouté jusqu'ici.
\VS{17}Ainsi parle Yahweh~: A ceci tu sauras que je suis Yahweh~; je m'en vais frapper de la verge qui est dans ma main les eaux du fleuve, et elles seront changées en sang.
\VS{18}Et le poisson qui est dans le fleuve mourra, le fleuve deviendra puant, et les Egyptiens éprouveront du dégoût à boire des eaux du fleuve.
\VS{19}Yahweh parla aussi à Moïse~: Dis à Aaron~: Prends ta verge et étends ta main sur les eaux des Egyptiens, sur leurs rivières, sur leurs ruisseaux, et sur leurs marais, et sur tous les amas de leurs eaux, et elles deviendront du sang~; il y aura du sang par tout le pays d'Egypte, dans les vases de bois et de pierre.
\VS{20}Moïse donc et Aaron firent ce que Yahweh avait ordonné. Aaron, ayant levé la verge, en frappa les eaux du fleuve, sous les yeux de Pharaon et de ses serviteurs~; et toutes les eaux du fleuve furent changées en sang.
\VS{21}Et le poisson qui était dans le fleuve mourut, et le fleuve en devint puant, tellement que les Egyptiens ne pouvaient point boire les eaux du fleuve~; il y eut du sang dans tout le pays d'Egypte.
\VS{22}Et les magiciens d'Egypte en firent de même par leurs enchantements. Et le cœur de Pharaon s'endurcit tellement, qu'il ne les écouta point, selon ce que Yahweh avait dit.
\VS{23}Et Pharaon leur ayant tourné le dos, alla dans sa maison, et ne prit même pas à cœur ces choses qu'il avait vues.
\VS{24}Or tous les Egyptiens creusèrent autour du fleuve pour trouver de l'eau à boire, parce qu'ils ne pouvaient pas boire de l'eau du fleuve.
\VS{25}Il se passa sept jours depuis que Yahweh eut frappé le fleuve.
\TextTitle{Invasion de grenouilles}
\VS{26}Après cela, Yahweh dit à Moïse~: Va vers Pharaon et dis-lui~: Ainsi parle Yahweh~: Laisse aller mon peuple, afin qu'il me serve.
\VS{27}Si tu refuses de le laisser aller, voici, je m'en vais frapper de grenouilles toutes tes contrées~;
\VS{28}et le fleuve fourmillera de grenouilles, qui monteront et entreront dans ta maison, et dans la chambre où tu couches, et sur ton lit, et dans les maisons de tes serviteurs, et parmi tout ton peuple, dans tes fours et dans tes maies.
\VS{29}Ainsi, les grenouilles monteront sur toi, sur ton peuple et sur tous tes serviteurs.
\Chap{8}
\VerseOne{}Yahweh donc dit à Moïse~: Dis à Aaron~: Etends ta main avec ta verge sur les fleuves, sur les rivières, et sur les marais, et fais monter les grenouilles sur le pays d'Egypte.
\VS{2}Et Aaron étendit sa main sur les eaux de l'Egypte, et les grenouilles montèrent et couvrirent le pays d'Egypte.
\VS{3}Mais les magiciens firent de même par leurs enchantements et firent monter des grenouilles sur le pays d'Egypte.
\VS{4}Alors Pharaon appela Moïse et Aaron, et leur dit~: Fléchissez Yahweh par vos prières, afin qu'il retire les grenouilles de dessus moi et de dessus mon peuple~; et je laisserai aller le peuple, afin qu'ils sacrifient à Yahweh.
\VS{5}Et Moïse dit à Pharaon~: Glorifie-toi sur moi~! Pour quel temps fléchirai-je par mes prières Yahweh pour toi, et pour tes serviteurs et pour ton peuple, afin que Yahweh retire les grenouilles loin de toi et de tes maisons~? Il en demeurera seulement dans le fleuve.
\VS{6}Alors il répondit~: Pour demain. Et Moïse dit~: Il sera fait selon ta parole, afin que tu saches qu'il n'y a nul Dieu tel que Yahweh, notre Dieu.
\VS{7}Les grenouilles donc se retireront de toi, de tes maisons, de tes serviteurs et de ton peuple~; il en demeurera seulement dans le fleuve.
\VS{8}Alors Moïse et Aaron sortirent de chez Pharaon~; Moïse cria à Yahweh au sujet des grenouilles qu'il avait fait venir sur Pharaon.
\VS{9}Et Yahweh fit selon la parole de Moïse. Ainsi les grenouilles moururent dans les maisons, dans les villages et dans les champs.
\VS{10}On les amassa par monceaux, et la terre en fut infectée.
\VS{11}Mais Pharaon, voyant qu'il y avait du relâche, endurcit son cœur et ne les écouta point, selon ce que Yahweh avait dit.
\TextTitle{Invasion de poux}
\VS{12}Et Yahweh dit à Moïse~: Dis à Aaron~: Etends ta verge et frappe la poussière de la terre, et elle deviendra des poux dans tout le pays d'Egypte.
\VS{13}Et ils firent ainsi~; et Aaron étendit sa main avec sa verge, et frappa la poussière de la terre~; et elle fut changée en poux, sur les hommes et sur les bêtes~; toute la poussière du pays fut changée en poux dans tout le pays d'Egypte.
\VS{14}Et les magiciens voulurent faire de même par leurs enchantements, pour produire des poux, mais ils ne purent pas. Les poux furent donc tant sur les hommes que sur les bêtes.
\VS{15}Alors les magiciens dirent à Pharaon~: C'est ici le doigt de Dieu\FTNT{Lu. 11:20.}~! Toutefois, le cœur de Pharaon s'endurcit et il ne les écouta point, selon ce que Yahweh avait dit.
\TextTitle{Invasion de mouches}
\VS{16}Puis Yahweh dit à Moïse~: Lève-toi de bon matin, et présente-toi devant Pharaon~; voici, il sortira près de l'eau, et tu lui diras~: Ainsi parle Yahweh~: Laisse aller mon peuple, afin qu'il me serve.
\VS{17}Car si tu ne laisses pas aller mon peuple, voici, je m'en vais envoyer contre toi, contre tes serviteurs, contre ton peuple et contre tes maisons, un mélange d'insectes~; et les maisons des Egyptiens seront remplies de ce mélange, et la terre aussi sur laquelle ils seront \FTNT{Ps. 105:31~; Ps. 78:43.}.
\VS{18}Mais je distinguerai ce jour-là le pays de Gosen, où se tient mon peuple, tellement qu'il n'y aura nul mélange d'insectes~; afin que tu saches que je suis Yahweh au milieu de la terre.
\VS{19}Et je ferai la différence entre ton peuple et mon peuple~; demain, ce signe-là se fera.
\VS{20}Et Yahweh le fit ainsi~; et un grand mélange d'insectes entra dans la maison de Pharaon et dans chaque maison de ses serviteurs, et dans tout le pays d'Egypte, de sorte que la terre fut gâtée par ce mélange.
\TextTitle{Pharaon tente de compromettre Moïse}
\VS{21} Et Pharaon appela Moïse et Aaron, et leur dit~: Allez, sacrifiez à votre Dieu dans ce pays.
\VS{22}Mais Moïse dit~: Il n'est pas convenable de faire ainsi~; car nous sacrifierions à Yahweh, notre Dieu, l'abomination des Egyptiens. Voici, si nous sacrifions l'abomination des Egyptiens devant leurs yeux, ne nous lapideraient-ils pas~?
\VS{23}Nous irons le chemin de trois jours au désert, et nous sacrifierons à Yahweh, notre Dieu, comme il nous dira.
\VS{24}Alors Pharaon dit~: Je vous laisserai aller pour sacrifier dans le désert à Yahweh, votre Dieu~; toutefois, vous ne vous éloignerez pas en y allant. Fléchissez Yahweh pour moi, par vos prières.
\VS{25}Moïse dit~: Voici, je sors de chez toi et je supplierai Yahweh, afin que le mélange d'insectes se retire demain de Pharaon, de ses serviteurs, et de son peuple. Mais que Pharaon ne continue point à se moquer en ne laissant point aller le peuple pour sacrifier à Yahweh.
\VS{26}Alors Moïse sortit de chez Pharaon et fléchit Yahweh par la prière.
\VS{27}Et Yahweh fit selon la parole de Moïse~; et le mélange d'insectes se retira de Pharaon, de ses serviteurs et de son peuple~; il n'en resta pas un seul insecte.
\VS{28}Mais Pharaon endurcit son cœur cette fois encore et ne laissa point aller le peuple.
\Chap{9}
\TextTitle{La mort des troupeaux}
\VerseOne{}Alors Yahweh dit à Moïse~: Va vers Pharaon et dis-lui~: Ainsi parle Yahweh, le Dieu des Hébreux~: Laisse aller mon peuple, afin qu'il me serve.
\VS{2}Car si tu refuses de les laisser aller et si tu le retiens encore,
\VS{3}voici, la main de Yahweh sera sur ton bétail qui est dans les champs, tant sur les chevaux que sur les ânes, sur les chameaux, sur les bœufs, et sur les brebis, et il y aura une très grande mortalité.
\VS{4}Et Yahweh distinguera le bétail des Israélites du bétail des Egyptiens, afin que rien de ce qui est aux enfants d'Israël ne meure.
\VS{5}Et Yahweh fixa un temps, en disant~: Demain, Yahweh fera ceci dans le pays.
\VS{6}Yahweh donc fit cela dès le lendemain~; et tout le bétail des Egyptiens mourut~; mais du bétail des enfants d'Israël, il ne mourut pas une seule bête.
\VS{7}Et Pharaon envoya examiner, et voici, il n'y avait pas une seule bête morte du bétail des enfants d'Israël. Toutefois, le cœur de Pharaon s'endurcit, et il ne laissa point aller le peuple.
\TextTitle{Des ulcères sur les Egyptiens et les bêtes}
\VS{8}Alors Yahweh dit à Moïse et à Aaron~: Remplissez vos mains de cendre de fournaise~; et que Moïse les répande vers les cieux en la présence de Pharaon.
\VS{9}Et elles deviendront de la poussière sur tout le pays d'Egypte, et il s'en fera des ulcères bourgeonnant en pustules tant sur les hommes que sur les bêtes, dans tout le pays d'Egypte.
\VS{10}Ils prirent donc de la cendre de fournaise et se tinrent devant Pharaon~; Moïse la répandit vers les cieux et il se forma des ulcères bourgeonnant en pustules tant sur les hommes que sur les bêtes.
\VS{11} Et les magiciens ne purent se tenir devant Moïse, à cause des ulcères~; car les magiciens avaient des ulcères, comme tous les Egyptiens.
\VS{12} Et Yahweh endurcit le cœur de Pharaon, et il ne les écouta point selon ce que Yahweh avait dit à Moïse.
\TextTitle{L'Egypte frappée par la grêle et le feu}
\VS{13}Puis Yahweh parla à Moïse~: Lève-toi de bon matin, et présente-toi devant Pharaon, et dis-lui~: Ainsi parle Yahweh, le Dieu des Hébreux~: Laisse aller mon peuple, afin qu'il me serve.
\VS{14}Car cette fois, je vais faire venir toutes mes plaies contre ton cœur, sur tes serviteurs et sur ton peuple, afin que tu saches qu'il n'y a nul Dieu semblable à moi sur toute la terre.
\VS{15}Car maintenant si j'avais étendu ma main, je t'aurais frappé de la peste, toi et ton peuple, et tu serais effacé de la terre.
\VS{16}Mais certainement, je t'ai fait subsister pour te faire voir ma puissance, afin que mon Nom soit célébré sur toute la terre\FTNT{Ro. 9:17.}.
\VS{17}T'élèves-tu encore contre mon peuple, pour ne point le laisser aller~?
\VS{18}Voici, je m'en vais faire pleuvoir demain à cette même heure, une grêle tellement forte qu'il n'y en a point eu de semblable en Egypte, depuis le jour où elle fut fondée jusqu'à maintenant.
\VS{19}Maintenant envoie rassembler ton bétail et tout ce que tu as à la campagne~; car la grêle tombera sur tous les hommes, sur le bétail qui se trouvera à la campagne, et qu'on n'aura pas renfermé, et ils mourront.
\VS{20}Celui d'entre les serviteurs de Pharaon, qui craignit la parole de Yahweh, fit promptement retirer dans les maisons ses serviteurs et ses bêtes.
\VS{21}Mais celui qui n'appliqua point son cœur à la parole de Yahweh, laissa ses serviteurs et ses bêtes à la campagne.
\VS{22}Et Yahweh dit à Moïse~: Etends ta main vers les cieux, et il y aura de la grêle sur tout le pays d'Egypte, sur les hommes et sur les bêtes, et sur toutes les herbes des champs au pays d'Egypte.
\VS{23}Moïse donc étendit sa verge vers les cieux, et Yahweh envoya des tonnerres et de la grêle, et le feu se promenait sur la terre. Yahweh fit pleuvoir de la grêle sur le pays d'Egypte.
\VS{24}Il y eut donc de la grêle et du feu entremêlé avec la grêle, laquelle était si grosse qu'il n'y en avait point eu de semblable sur toute la terre d'Egypte, depuis qu'elle a été habitée.
\VS{25}La grêle frappa dans tout le pays d'Egypte tout ce qui était aux champs, depuis les hommes jusqu'aux bêtes. La grêle frappa aussi toutes les herbes des champs et brisa tous les arbres des champs.
\VS{26}Il n'y eut que la contrée de Gosen, dans laquelle étaient les enfants d'Israël, où il n'y eut point de grêle.
\TextTitle{Pharaon continue d'endurcir son cœur}
\VS{27}Alors Pharaon envoya appeler Moïse et Aaron, et leur dit~: J'ai péché cette fois~; Yahweh est juste, mais moi et mon peuple sommes méchants.
\VS{28}Fléchissez par des prières Yahweh~: Que ce soit assez, et que Dieu ne fasse plus tonner ni grêler, car je vous laisserai aller, et on ne vous arrêtera plus.
\VS{29}Alors Moïse dit~: Aussitôt que je sortirai de la ville, j'étendrai mes mains vers Yahweh et les tonnerres cesseront. Il n'y aura plus de grêle, afin que tu saches que la terre est à Yahweh\FTNT{Ps. 24:1.}.
\VS{30}Mais quant à toi et tes serviteurs, je sais que vous ne craindrez pas encore Yahweh Dieu.
\VS{31}Or le lin et l'orge avaient été frappés, car l'orge était en épis et c'était la floraison du lin.
\VS{32} Mais le blé et l'épeautre ne furent point frappés, parce qu'ils sont tardifs.
\VS{33}Moïse donc sortit de chez Pharaon pour aller hors de la ville. Il étendit ses mains vers Yahweh, et les tonnerres cessèrent, et la grêle et la pluie ne tombèrent plus sur la terre.
\VS{34}Pharaon, voyant que la pluie, la grêle, et les tonnerres avaient cessé, continua encore à pécher, et il endurcit son cœur, lui et ses serviteurs.
\VS{35}Le cœur donc de Pharaon s'endurcit et il ne laissa point aller les enfants d'Israël, selon ce que Yahweh avait dit par l'intermédiaire de Moïse.
\Chap{10}
\TextTitle{Invasion de sauterelles}
\VerseOne{} Et Yahweh dit à Moïse~: Va vers Pharaon, car j'ai endurci son cœur et le cœur de ses serviteurs, afin que je mette au-dedans de lui les signes que je m'en vais faire~;
\VS{2}et afin que tu racontes à ton fils et au fils de ton fils, les signes que j'accomplirai sur les Egyptiens et les prodiges que je ferai au milieu d'eux, et que vous sachiez que je suis Yahweh.
\VS{3}Moïse donc et Aaron vinrent vers Pharaon, et lui dirent~: Ainsi parle Yahweh, le Dieu des Hébreux~: Jusqu'à quand refuseras-tu de t'humilier devant moi~? Laisse aller mon peuple, afin qu'il me serve.
\VS{4}Car si tu refuses de laisser aller mon peuple, voici, je ferai venir demain des sauterelles dans ton territoire.
\VS{5}Elles couvriront la face de la terre, et l'on ne pourra plus voir la terre~; elles dévoreront le reste de ce qui a échappé, ce que la grêle vous a laissé~; et elles dévoreront tous les arbres qui poussent dans vos champs.
\VS{6}Et elles rempliront tes maisons, et les maisons de tous tes serviteurs, et les maisons de tous les Egyptiens~; ce que tes pères n'ont point vu ni les pères de tes pères, depuis qu'ils existent sur la terre jusqu'à ce jour. Puis, ayant tourné le dos à Pharaon, il sortit d'auprès de lui.
\VS{7}Et les serviteurs de Pharaon lui dirent~: Jusqu'à quand celui-ci nous sera-t-il un piège~? Laisse aller ces gens et qu'ils servent Yahweh, leur Dieu. Attendras-tu de savoir avant cela que l'Egypte est perdue~?
\VS{8}Alors on fit revenir Moïse et Aaron vers Pharaon, il leur dit~: Allez, servez Yahweh, votre Dieu. Qui sont tous ceux qui iront~?
\VS{9} Et Moïse répondit~: Nous irons avec nos jeunes gens et nos vieillards, avec nos fils et nos filles~; nous irons avec nos brebis et nos bœufs~; car nous avons à célébrer une fête solennelle à Yahweh.
\VS{10}Alors il leur dit~: Que Yahweh soit avec vous, comme je laisserai aller vos petits enfants~! Prenez garde, car le mal est devant vous.
\VS{11}Il n'en sera pas ainsi que vous l'avez demandé~; mais vous, hommes, allez maitenant et servez Yahweh~; car c'est ce que vous demandiez. Et on les chassa de la présence de Pharaon.
\VS{12}Alors Yahweh dit à Moïse~: Etends ta main sur le pays d'Egypte, pour faire venir les sauterelles, afin qu'elles montent sur le pays d'Egypte, qu'elles dévorent toute l'herbe de la terre, tout ce que la grêle a laissé.
\VS{13}Moïse étendit donc sa verge sur le pays d'Egypte~; et Yahweh amena sur le pays, tout ce jour-là et toute la nuit, un vent d'orient~; le matin vint, et le vent d'orient enleva les sauterelles.
\VS{14}Et il fit monter les sauterelles sur tout le pays d'Egypte, et les mit dans toutes les contrées d'Egypte~; elles étaient fort grosses et il y n'en avait point eu avant elles de semblables, et il n'y en aura point de semblables après elles.
\VS{15}Et elles couvrirent la face de tout le pays, tellement que le pays en fut obscurci~; elles dévorèrent toute l'herbe de la terre, tout le fruit des arbres que la grêle avait laissé~; il ne resta aucune verdure aux arbres ni aux herbes des champs, dans tout le pays d'Egypte.
\VS{16}Aussitôt Pharaon se hâta d'appeler Moïse et Aaron, et dit~: J'ai péché contre Yahweh, votre Dieu, et contre vous.
\VS{17}Mais pardonne, je te prie, mon péché, pour cette fois seulement~; et suppliez Yahweh, votre Dieu, par vos prières, afin qu'il retire de moi cette mort-ci seulement.
\VS{18}Il sortit donc de chez Pharaon, et fléchit Yahweh par ses prières.
\VS{19}Et Yahweh fit lever un vent d'occident très fort qui enleva les sauterelles et les précipita dans la Mer Rouge. Il ne resta pas une seule sauterelle dans tout le territoire de l'Egypte.
\VS{20}Mais Yahweh endurcit le cœur de Pharaon et il ne laissa point aller les enfants d'Israël.
\TextTitle{Les ténèbres sur les Egyptiens}
\VS{21}Puis Yahweh dit à Moïse~: Etends ta main vers les cieux, qu'il y ait sur le pays d'Egypte des ténèbres si épaisses, qu'on puisse les toucher à la main.
\VS{22}Moïse étendit donc sa main vers les cieux, et il y eut d'épaisses ténèbres dans tout le pays d'Egypte, pendant trois jours\FTNT{Ps. 105:28.}.
\VS{23}On ne se voyait pas l'un l'autre, et nul ne se leva de sa place pendant trois jours. Mais pour tous les enfants d'Israël, il y eut de la lumière dans le lieu de leurs demeures.
\TextTitle{Pharaon tente encore de compromettre Moïse}
\VS{24}Alors Pharaon appela Moïse et dit~: Allez, servez Yahweh~; que vos brebis et vos bœufs seuls demeurent~; vos petits enfants iront aussi avec vous.
\VS{25}Moïse répondit~: Tu mettras toi-même entre nos mains de quoi faire des sacrifices et des holocaustes, que nous ferons à Yahweh, notre Dieu.
\VS{26}Et même, nos troupeaux viendront aussi avec nous, il n'en restera pas un sabot. Car nous en prendrons pour servir Yahweh, notre Dieu~; car nous ne savons pas ce que nous choisirons pour offrir à Yahweh, jusqu'à ce que nous soyons arrivés en ce lieu là.
\VS{27}Mais Yahweh endurcit le cœur de Pharaon et il ne voulut point les laisser aller.
\VS{28}Et Pharaon lui dit~:Va-t-en~!Arrière de moi~! Garde-toi de revoir ma face, car le jour où tu verras ma face, tu mourras.
\VS{29}Alors Moïse répondit~: Tu as bien dit, je ne reverrai plus ta face\FTNT{Hé. 11:27.}.
\Chap{11}
\TextTitle{Pharaon méprise l'avertissement sur la mort des premiers-nés}
\VerseOne{}Or Yahweh dit à Moïse~: Je ferai venir encore une plaie sur Pharaon, et sur l'Egypte, et après cela il vous laissera aller d'ici~; il vous laissera entièrement aller, et vous chassera tout à fait d'ici.
\VS{2}Parle maintenant aux oreilles du peuple, et dis leur~: Que chacun demande à son voisin, et chacune à sa voisine, des vases d'argent et des vases d'or.
\VS{3}Or Yahweh fit trouver grâce au peuple devant les Egyptiens~; et même Moïse passait pour un grand homme dans le pays d'Egypte, tant parmi les serviteurs de Pharaon que parmi le peuple.
\VS{4}Et Moïse dit~: Ainsi parle Yahweh~: Vers le milieu de la nuit, je passerai au travers de l'Egypte~;
\VS{5}et tout premier-né mourra dans le pays d'Egypte, depuis le premier-né de Pharaon, qui devait être assis sur son trône, jusqu'au premier-né de la servante qui est derrière la meule, et jusqu'à tous les premiers-nés des bêtes.
\VS{6}Et il y aura un grand cri dans tout le pays d'Egypte, tel qu'il n'y en a jamais eu et qu'il n'y en aura jamais de semblable.
\VS{7}Mais contre tous les enfants d'Israël, un chien même ne remuera point sa langue, depuis l'homme jusqu'aux bêtes~; afin que vous sachiez que Dieu fera la différence entre les Egyptiens et les Israélites.
\VS{8}Et tous tes serviteurs viendront vers moi, et se prosterneront devant moi, en disant~: Sors, toi, et tout le peuple qui est avec toi. Après cela, je sortirai. Ainsi, Moïse sortit de chez Pharaon dans une ardente colère.
\VS{9}Yahweh donc dit à Moïse~: Pharaon ne vous écoutera point, afin que mes miracles soient multipliés dans le pays d'Egypte.
\VS{10}Et Moïse et Aaron firent tous ces miracles-là devant Pharaon. Et Yahweh endurcit le cœur de Pharaon, tellement qu'il ne laissa point aller les enfants d'Israël hors de son pays.
\Chap{12}
\TextTitle{La première Pâque}
\VerseOne{}Or Yahweh dit à Moïse et à Aaron dans le pays d'Egypte~:
\VS{2}Ce mois-ci sera pour vous le premier des mois, il sera pour vous le premier des mois de l'année.
\VS{3}Parlez à toute l'assemblée d'Israël, en disant~: Jusqu'au dixième jour de ce mois, que chacun prenne un petit d'entre les brebis ou d'entre les chèvres, selon les familles des pères~; un petit, dis-je, d'entre les brebis ou d'entre les chèvres, par famille.
\VS{4}Mais si la famille est moindre qu'il ne faut pour manger un petit d'entre les brebis ou d'entre les chèvres, qu'elle le prenne avec son voisin qui est près de sa maison, selon le nombre de personnes~; vous compterez combien il en faudra pour manger d'entre les brebis ou d'entre les chèvres, ayant égard à ce que chacun de vous peut manger.
\VS{5}Or le petit d'entre les brebis ou d'entre les chèvres sera sans défaut, et sera un mâle ayant un an\FTNT{La Pâque juive était célébrée le 14ème jour du premier mois de l'année juive soit, le 14 du mois de Nissan (Ex. 12:2~; No. 9:1-5). L'agneau pascal était une préfiguration de Jésus-Christ~: L'agneau de Dieu qui ôte le péché du monde (Jn. 1:29). Ses caractéristiques sont les suivantes~: \\- L'agneau devait nécessairement être un mâle sans défaut (Ex. 12:5). Jésus est l'enfant mâle mis au monde par une vierge, il n'a pas été affecté par le sang corrompu d'Adam, il est donc sans défaut (Es. 7:14~; Mt. 1:20-21). Pour être certains de la perfection de l'animal, les hébreux devaient l'examiner pendant quatre jours avant de l'immoler (Ex. 12:3-6). Il est à noter que la loi juive exigeait que deux ou trois témoins soient présents pour constater un crime ou un péché (De. 17:6~; De. 19:15), ces quatre jours font donc office de quatre témoins pour attester de la pureté de l'animal. De même, les quatre auteurs de l'évangile attestent la sainteté du Seigneur. De plus, avant sa mise à mort, le Seigneur a été examiné par deux législations~: juive (le sanhédrin) et romaine (Ponce Pilate). Ces deux législations attestèrent, malgré elles, son innocence (Mt. 25:60~; Mt. 27:24~; Mc. 14:55-56~; Mc. 15:14~; Lu. 23:4~; Jn. 18:31~; Jn. 19:6) et confirmèrent qu'il était sans défaut et donc digne d'être offert en sacrifice.\\- Yahweh avait prescrit aux hébreux d'immoler l'agneau entre les deux soirs (Ex. 12:6), c'est-à-dire avant le crépuscule, entre la neuvième et la onzième heure. Jésus fut arrêté la nuit de Pâque (Mc. 14:12-41). Sa crucifixion eut lieu le lendemain, à la troisième heure (Mc. 15:25), et sa mort survint à la neuvième heure (Mt. 27:45). L'agneau devait être rôti au feu puis consommé avec du pain sans levain et des herbes amères (Ex. 12:8). Le feu symbolise le jugement que le Seigneur a pris sur lui à cause de nos péchés (Es. 53:5~; Ro. 4:25~; 1 Pi. 1:18-20). Le pain sans levain est une autre image de Jésus, le pain de vie (Jn. 6:35) sans aucun péché (1 Co. 5:8). Les herbes amères préfigurent, quant à elles, l'affliction et la souffrance du Seigneur (Hé. 2:10).}~; vous le prendrez d'entre les brebis ou d'entre les chèvres~;
\VS{6}et vous le garderez jusqu'au quatorzième jour de ce mois~; et toute la congrégation de l'assemblée d'Israël l'égorgera entre les deux soirs.
\VS{7} Et ils prendront de son sang, et le mettront sur les deux poteaux et sur le linteau de la porte des maisons où ils le mangeront.
\VS{8} Et ils en mangeront la chair rôtie au feu cette nuit-là~; et ils la mangeront avec des pains sans levain, et avec des herbes amères.
\VS{9}N'en mangez rien à demi cuit, ni qui ait été bouilli dans l'eau~; mais qu'il soit rôti au feu, sa tête, ses jambes et ses entrailles.
\VS{10} Et ne laissez aucun reste jusqu'au matin, mais s'il en reste quelque chose le matin, vous le brûlerez au feu.
\VS{11}Et vous le mangerez ainsi~: Vos reins seront ceints, vous aurez vos souliers à vos pieds, et votre bâton à la main, et vous le mangerez à la hâte. C'est la Pâque de Yahweh.
\TextTitle{Le sang qui sauve~; l'instauration de la fête de la Pâque}
\VS{12}Car je passerai cette nuit-là par le pays d'Egypte, et je frapperai tout premier-né au pays d'Egypte, depuis les hommes jusqu'aux bêtes~; et j'exercerai des jugements sur tous les dieux de l'Egypte. Je suis Yahweh.
\VS{13}Et le sang sera pour vous un signe sur les maisons où vous serez~; car je verrai le sang et je passerai par-dessus vous, et il n'y aura point de plaie à destruction quand je frapperai le pays d'Egypte.
\VS{14}Et ce jour là, vous conserverez le souvenir de ce jour, et vous le célébrerez comme une fête solennelle à Yahweh~; vous le célébrerez comme une fête solennelle par une ordonnance perpétuelle de génération en génération.
\VS{15}Vous mangerez pendant sept jours des pains sans levain, et dès le premier jour, vous ôterez le levain de vos maisons~; car quiconque mangera du pain levé, depuis le premier jour jusqu'au septième, cette personne-là sera retranchée d'Israël.
\VS{16}Au premier jour il y aura une sainte convocation, et il y aura de même au septième jour une sainte convocation~; il ne se fera aucune œuvre dans ces jours-là~; seulement, on vous apprêtera à manger ce qu'il faudra pour chaque personne.
\VS{17}Vous prendrez donc garde aux pains sans levain, parce qu'en ce même jour, j'aurai retiré vos armées du pays d'Egypte~; vous observerez donc ce jour-là de génération en génération par une ordonance perpétuelle.
\VS{18}Au premier mois, le quatorzième jour du mois, au soir, vous mangerez des pains sans levain jusqu'au vingt et unième jour du mois, au soir.
\VS{19}Il ne se trouvera point de levain dans vos maisons pendant sept jours, car quiconque mangera du pain levé, cette personne-là sera retranchée de l'assemblée d'Israël, tant celui qui habite comme étranger que celui qui est né au pays.
\VS{20}Vous ne mangerez point de pain levé~; mais vous mangerez dans tous les lieux où vous demeurerez des pains sans levain.
\VS{21}Moïse donc appela tous les anciens d'Israël et leur dit~: Choisissez et prenez un petit d'entre les brebis ou d'entre les chèvres selon vos familles, et égorgez la Pâque.
\VS{22}Puis vous prendrez un bouquet d'hysope et le tremperez dans le sang qui sera dans un bassin, et vous arroserez du sang qui sera dans le bassin, le linteau et les deux poteaux~; et nul de vous ne sortira de la porte de sa maison jusqu'au matin.
\VS{23}Car Yahweh passera pour frapper l'Egypte et il verra le sang sur le linteau et sur les deux poteaux, et Yahweh passera par-dessus la porte, et ne permettra point que le destructeur entre dans vos maisons pour frapper.
\VS{24}Vous garderez ceci comme une ordonnance perpétuelle pour toi et pour tes fils.
\VS{25}Quand donc vous serez entrés dans le pays que Yahweh vous donnera, selon qu'il en a parlé, vous observerez ce service.
\VS{26}Et quand vos fils vous diront~: Que signifie pour vous ce service~?
\VS{27}Alors vous répondrez~: C'est le sacrifice de la Pâque à Yahweh, qui passa en Egypte par-dessus les maisons des enfants d'Israël, quand il frappa l'Egypte, et qu'il préserva nos maisons. Alors le peuple s'inclina et se prosterna.
\VS{28}Ainsi les enfants d'Israël s'en allèrent et firent comme Yahweh l'ordonna à Moïse et à Aaron, ils le firent ainsi.
\TextTitle{Les premiers-nés d'Egypte frappés}
\VS{29}Et il arriva qu'à minuit Yahweh frappa tous les premiers-nés du pays d'Egypte, depuis le premier-né de Pharaon, qui devait être assis sur son trône, jusqu'aux premiers-nés des captifs qui étaient dans la prison, et tous les premiers-nés des bêtes.
\VS{30}Et Pharaon se leva de nuit, lui et ses serviteurs, et tous les Egyptiens~; et il y eut un grand cri en Egypte, parce qu'il n'y avait point de maison où il n'y ait eu un mort\FTNT{Hé. 11:28~; No. 8:17~; Ps. 78:51~; Ps. 105:36.}.
\TextTitle{Israël sort d'Egypte}
\VS{31}Il appela donc Moïse et Aaron de nuit, et leur dit~: Levez-vous, sortez du milieu de mon peuple, tant vous que les enfants d'Israël, allez et servez Yahweh, comme vous en avez parlé.
\VS{32}Prenez aussi votre menu et gros bétail, comme vous en avez parlé, et allez-vous-en et bénissez-moi.
\VS{33}Et les Egyptiens pressaient le peuple et se hâtaient de les faire sortir du pays, car ils disaient~: Nous sommes tous morts.
\VS{34}Le peuple donc prit sa pâte avant qu'elle fût levée, ayant leurs mains liées avec leurs vêtements, sur leurs épaules.
\VS{35}Or les enfants d'Israël firent selon la parole de Moïse, et demandèrent aux Egyptiens des vases d'argent et d'or, et des vêtements.
\VS{36}Et Yahweh fit trouver grâce au peuple auprès des Egyptiens, qui les leur prêtèrent~; de sorte qu'ils dépouillèrent les Egyptiens.
\VS{37}Ainsi, les fils d'Israël étant partis de Ramsès, vinrent à Succoth, environ six cent mille hommes de pied, sans les enfants.
\VS{38}Il s'en alla aussi avec eux un grand nombre de toutes sortes de gens~; et du menu et du gros bétail, en fort grands troupeaux.
\VS{39} Or parce qu'ils avaient été chassés d'Egypte, et qu'ils n'avaient pas pu tarder plus longtemps, et que même ils n'avaient fait aucune provision, ils cuisirent par gâteaux sans levain, la pâte qu'ils avaient emportée d'Egypte~; car ils ne l'avaient point fait lever. 
\VS{40}Or le séjour des enfants d'Israël en Egypte fut de quatre cent trente ans\FTNT{Ge. 15:13~; Ac. 7:6~; Ga. 3:17.}.
\VS{41}Il arriva donc au bout de quatre cent trente ans, il arriva dis-je, en ce propre jour-là, que toutes les armées de Yahweh sortirent du pays d'Egypte.
\VS{42}C'est la nuit qui doit être soigneusement observée en l'honneur de Yahweh, parce qu'alors il les retira du pays d'Egypte~; cette nuit-là est à observer en l'honneur de Yahweh, par tous les enfants d'Israël de génération en génération\FTNT{De. 16:1-6.}.
\VS{43}Yahweh dit aussi à Moïse et à Aaron~: C'est ici l'ordonnance de la Pâque~: Aucun étranger n'en mangera~;
\VS{44}mais tout esclave qu'on aura acheté par argent sera circoncis, et alors il en mangera.
\VS{45}L'étranger et le mercenaire n'en mangeront point.
\VS{46}On la mangera dans une même maison, et vous n'emporterez point de sa chair hors de la maison, et vous n'en casserez point les os.
\VS{47}Toute l'assemblée d'Israël la fera.
\VS{48}Et si quelque étranger qui habite chez toi veut faire la Pâque à Yahweh, que tout mâle qui lui appartient soit circoncis~; et alors il s'approchera pour la faire, et il sera comme celui qui est né dans le pays~; mais aucun incirconcis n'en mangera.
\VS{49}Il y aura une même loi pour celui qui est né dans le pays et pour l'étranger qui habite parmi vous.
\VS{50}Tous les enfants d'Israël firent ce que Yahweh avait ordonné à Moïse et à Aaron~; ils le firent ainsi.
\VS{51}Il arriva donc en ce même jour que Yahweh retira les enfants d'Israël du pays d'Egypte, selon leurs armées.
\Chap{13}
\TextTitle{Consécration des premiers-nés à Yahweh}
\VerseOne{}Et Yahweh parla à Moïse, et dit~:
\VS{2}Sanctifie-moi tout premier-né, tout premier-né issu du sein maternel parmi les fils d'Israël, tant des hommes que des bêtes, car il est à moi\FTNT{Lé. 27:26-27~; No. 3:13~; No. 8:17~; Lu. 2:22-23.}.
\VS{3}Moïse donc dit au peuple~: Souvenez-vous de ce jour où vous êtes sortis d'Egypte, de la maison de servitude~; car Yahweh vous en a retirés par sa main puissante~; on ne mangera donc point de pain levé.
\VS{4}Vous sortez aujourd'hui dans le mois où les épis mûrissent.
\VS{5}Quand donc Yahweh t'aura introduit dans le pays des Cananéens, des Héthiens, des Amoréens, des Héviens et des Jébusiens, qu'il a juré à tes pères de te donner, et qui est un pays découlant de lait et de miel, alors tu feras ce service durant ce mois-ci.
\VS{6}Pendant sept jours tu mangeras des pains sans levain, et au septième jour il y aura une fête solennelle à Yahweh.
\VS{7}On mangera durant sept jours des pains sans levain~; il ne sera point vu chez toi de pain levé et même il ne sera point vu de levain dans toutes tes contrées.
\VS{8}Et ce jour-là, tu feras entendre ces choses à tes enfants, en disant~: C'est à cause de ce que Yahweh m'a fait en me retirant d'Egypte.
\VS{9}Et ceci te sera pour signe sur ta main, et comme un rappel entre tes yeux, afin que la loi de Yahweh soit dans ta bouche, car Yahweh t'aura retiré d'Egypte par sa main puissante\FTNT{De. 6:8~; De. 11:18.}.
\VS{10}Tu observeras cette ordonnance au jour fixé d'année en année.
\VS{11}Aussi, quand Yahweh t'aura introduit dans le pays des Cananéens, selon qu'il a juré à toi et à tes pères, et qu'il te l'aura donné,
\VS{12}tu consacreras à Yahweh tout premier-né issu du sein de sa mère, même tout premier-né des animaux que tu auras~; les mâles appartiendront à Yahweh.
\VS{13}Et tu rachèteras avec un petit d'entre les brebis ou d'entre les chèvres, tout premier-né de l'ânesse, et si tu ne le rachètes point, tu lui briseras la nuque. Tu rachèteras aussi tout premier-né des hommes parmi tes fils.
\VS{14}Et quand ton fils t'interrogera à l'avenir, en disant~: Que veut dire ceci~? Alors tu lui diras~: Yahweh nous a retirés par main forte hors d'Egypte, de la maison de servitude.
\VS{15}Car il arriva que, quand Pharaon s'obstinait à ne point nous laisser aller, Yahweh tua tous les premiers-nés au pays d'Egypte, depuis les premiers-nés des hommes jusqu'aux premiers-nés des bêtes. Voilà pourquoi je sacrifie à Yahweh tout premier-né mâle issu du sein de sa mère, et je rachète tout premier-né de mes fils.
\VS{16}Ceci te sera donc pour signe sur ta main, et pour fronteaux entre tes yeux, que Yahweh nous a retirés d'Egypte par sa main puissante.
\TextTitle{Début du voyage, Yahweh dirige son peuple}
\VS{17}Or lorsque Pharaon laissa aller le peuple, Dieu ne les conduisit point par le chemin du pays des Philistins, bien qu'il fût le plus court~; car Dieu dit~: C'est afin qu'il n'arrive que le peuple se repente quand il verra la guerre, et qu'il ne retourne en Egypte.
\VS{18}Mais Dieu fit tourner le peuple par le chemin du désert, vers la Mer Rouge. Ainsi, les enfants d'Israël montèrent en armes hors du pays d'Egypte.
\VS{19}Et Moïse avait pris avec lui les ossements de Joseph, parce que Joseph avait expressément fait jurer les enfants d'Israël, en leur disant~: Dieu vous visitera très certainement, et vous transporterez donc avec vous mes ossements d'ici\FTNT{Ge. 50:25~; Jos. 24:32.}.
\VS{20}Et ils partirent de Succoth, et campèrent à Etham, qui est à l'extrémité du désert.
\VS{21}Et Yahweh allait devant eux, de jour dans une colonne de nuée pour les conduire par le chemin~; et de nuit dans une colonne de feu pour les éclairer, afin qu'ils marchent jour et nuit\FTNT{No. 9:13-23~; No. 10:43~; De. 1:33~; Né. 9:12-19~; 1 Co. 10:1.}.
\VS{22}Et il ne retira point la colonne de nuée le jour, ni la colonne de feu la nuit de devant le peuple.
\Chap{14}
\TextTitle{Pharaon et son armée à la poursuite d'Israël}
\VerseOne{}Et Yahweh parla à Moïse et dit~:
\VS{2}Parle aux enfants d'Israël et dis-leur: Qu'ils se détournent, et qu'ils campent devant Pi-Hahiroth, entre Migdol et la mer, vis-à-vis de Baal-Tsephon. Vous camperez vis-à-vis de ce lieu-là près de la mer\FTNT{No. 33:7.}.
\VS{3}Pharaon dira des enfants d'Israël~: Ils sont confus dans le pays, le désert les a enfermés.
\VS{4}Et j'endurcirai le cœur de Pharaon, et il vous poursuivra. Ainsi je serai glorifié en Pharaon et en toute son armée et les Egyptiens sauront que je suis Yahweh~; et ils firent ainsi.
\VS{5}Or on avait rapporté au roi d'Egypte que le peuple s'enfuyait, et le cœur de Pharaon et de ses serviteurs fut changé à l'égard du peuple, et ils dirent~: Qu'est-ce que nous avons fait en laissant aller Israël, de sorte qu'il ne nous servira plus~?
\VS{6}Alors il fit atteler son char et il prit son peuple avec lui.
\VS{7}Il prit donc six cents chars d'élite et tous les chars de l'Egypte~; et il y avait des capitaines sur tout cela.
\VS{8}Et Yahweh endurcit le cœur de Pharaon, roi d'Egypte, qui poursuivit les enfants d'Israël. Or les fils d'Israël étaient sortis à main levée\FTNT{Lé. 26:13~; No. 33:3.}.
\VS{9}Les Egyptiens donc les poursuivirent~; et tous les chevaux des chars de Pharaon, ses cavaliers et son armée les atteignirent comme ils étaient campés près de la mer, vers Pi-Hahiroth vis-à-vis de Baal-Tsephon.
\VS{10}Et Pharaon approchait. Les enfants d'Israël levèrent leurs yeux, et voici, les Egyptiens marchaient après eux. Et les fils d'Israël eurent une grande frayeur et crièrent à Yahweh.
\VS{11}Ils dirent aussi à Moïse~: Est-ce qu'il n'y avait pas des sépulcres en Egypte pour que tu nous aies emmenés pour mourir au désert~? Que nous as-tu fait en nous faisant sortir d'Egypte~?
\VS{12}N'est-ce pas ce que nous te disions en Egypte, en disant~: Retire-toi de nous et que nous servions les Egyptiens~? Car nous aimons mieux les servir que de mourir au désert.
\TextTitle{Délivrance miraculeuse par Yahweh}
\VS{13}Et Moïse dit au peuple~: Ne craignez point, arrêtez-vous et voyez la délivrance que Yahweh vous donnera aujourd'hui~; car les Egyptiens que vous voyez aujourd'hui, vous ne les verrez plus.
\VS{14}Yahweh combattra pour vous et vous demeurerez tranquilles.
\VS{15}Or Yahweh avait dit à Moïse~: Que cries-tu à moi~? Parle aux enfants d'Israël, qu'ils marchent.
\VS{16}Et toi, élève ta verge, étends ta main sur la mer, et fends-la~; et que les enfants d'Israël entrent au milieu de la mer à sec.
\VS{17} Et quant à moi, voici, je m'en vais endurcir le cœur des Egyptiens, afin qu'ils entrent après eux~; et je serai glorifié en Pharaon, et en toute son armée, en ses chars et en ses cavaliers.
\VS{18}Et les Egyptiens sauront que je suis Yahweh, quand j'aurai été glorifié en Pharaon, avec ses chars et ses cavaliers.
\VS{19}Et l'Ange de Dieu qui allait devant le camp d'Israël partit, et s'en alla derrière eux~; et la colonne de nuée partit de devant eux et se tint derrière eux.
\VS{20}Et elle vint entre le camp des Egyptiens et le camp d'Israël. Elle était aux uns une nuée et une obscurité~; et pour les autres, elle les éclairait la nuit. L'un des camps n'approcha point de l'autre durant toute la nuit.
\VS{21}Or Moïse avait étendu sa main sur la mer, et Yahweh fit reculer la mer toute la nuit par un vent d'orient qui souffla avec puissance~; il mit la mer à sec, et les eaux se fendirent\FTNT{Jos. 4:23~; Ps. 66:6~; Ps. 106:9~; Hé. 11:29.}.
\VS{22}Et les enfants d'Israël entrèrent au milieu de la mer à sec, et les eaux leur servaient de mur à droite et à gauche.
\VS{23}Et les Egyptiens les poursuivirent~; et ils entrèrent après eux au milieu de la mer, à savoir tous les chevaux de Pharaon, ses chars et ses cavaliers.
\VS{24}Mais il arriva que sur la veille du matin, Yahweh étant dans la colonne de feu et dans la nuée, regarda le camp des Egyptiens et le mit en déroute.
\VS{25}Il ôta les roues de leurs chars et alourdit leur marche. Alors les Egyptiens dirent~: Fuyons de devant les Israëlites, car Yahweh combat pour eux contre les Egyptiens.
\VS{26}Et Yahweh dit à Moïse~: Etends ta main sur la mer, et les eaux retourneront sur les Egyptiens, sur leurs chars et sur leurs cavaliers.
\VS{27}Moïse donc étendit sa main sur la mer, et la mer reprit son impétuosité vers le matin. Et les Egyptiens s'enfuyant rencontrèrent la mer qui s'était rejointe~; et ainsi Yahweh jeta les Egyptiens au milieu de la mer.
\VS{28}Car les eaux retournèrent et couvrirent les chars et les cavaliers de toute l'armée de Pharaon, qui étaient entrés après les Israélites dans la mer, et il n'en resta pas un seul.
\VS{29}Mais les enfants d'Israël marchèrent au milieu de la mer à sec, et les eaux leur servaient de mur à droite et à gauche.
\VS{30}Ainsi, Yahweh délivra, en ce jour-là, Israël de la main des Egyptiens~; et Israël vit sur le bord de la mer les Egyptiens morts.
\VS{31}Israël vit donc la grande puissance que Yahweh avait déployée contre les Egyptiens~; et le peuple craignit Yahweh, ils crurent en Yahweh, et en Moïse, son serviteur.
\Chap{15}
\TextTitle{Cantique de délivrance}
\VerseOne{}Alors Moïse et les enfants d'Israël chantèrent ce cantique à Yahweh, et dirent~: Je chanterai à Yahweh, car il est hautement élevé~; il a jeté dans la mer le cheval et celui qui le montait.
\VS{2}Yahweh est ma force et ma louange, et il a été mon Sauveur, mon Dieu. Je lui dresserai un tabernacle, c'est le Dieu de mon père, je l'exalterai.
\VS{3}Yahweh est un vaillant guerrier, son Nom est Yahweh.
\VS{4}Il a jeté dans la mer les chars de Pharaon et son armée~; l'élite de ses capitaines a été submergée dans la Mer Rouge.
\VS{5}Les gouffres les ont couverts, ils sont descendus au fond des eaux comme une pierre\FTNT{Né. 9:11.}.
\VS{6}Ta droite, ô Yahweh, s'est montrée magnifique en force~! Ta droite, ô Yahweh, a brisé l'ennemi\FTNT{Ps. 118:15-16~; Ps. 77:16.}~!
\VS{7}Tu as ruiné par la grandeur de ta majesté ceux qui s'élevaient contre toi~; tu as lâché ta colère et elle les a consumés comme du chaume.
\VS{8}Par le souffle de tes narines, les eaux ont été amoncelées~; les eaux courantes se sont arrêtés comme un monceau~; les gouffres ont été gelés au milieu de la mer.
\VS{9}L'ennemi disait~: Je poursuivrai, j'atteindrai, je partagerai le butin~; mon âme sera assouvie d'eux, je tirerai mon épée, ma main les détruira.
\VS{10}Tu as soufflé de ton vent, la mer les a couverts~; ils ont été enfoncés comme du plomb au plus profond des eaux.
\VS{11}Qui est comme toi parmi les dieux, ô Yahweh~! Qui est comme toi, magnifique en sainteté, digne d'être révéré et célébré, faisant des choses merveilleuses~?
\VS{12}Tu as étendu ta droite, la terre les a engloutis.
\VS{13}Tu as conduit par ta miséricorde ce peuple que tu as racheté~; tu l'as conduit par ta force à la demeure de ta sainteté.
\VS{14}Les peuples l'ont entendu, et ils en ont tremblé~; la douleur a saisi les habitants du pays des Philistins.
\VS{15}Alors les princes d'Edom seront troublés, et le tremblement saisira les puissants de Moab, tous les habitants de Canaan se fondront.
\VS{16}La frayeur et l'épouvante tomberont sur eux~; ils seront rendus muets comme une pierre par la grandeur de ton bras, jusqu'à ce que ton peuple soit passé, ô Yahweh~! Jusqu'à ce que ce peuple que tu as acquis soit passé\FTNT{De. 2:25~; De. 11:25~; Jos. 2:9.}.
\VS{17}Tu les introduiras et les planteras sur la montagne de ton héritage, au lieu que tu as préparé pour ta demeure, ô Yahweh~! Au lieu saint, ô Seigneur, que tes mains ont établi~!
\VS{18}Yahweh régnera à jamais et à perpétuité.
\VS{19}Car les chevaux de Pharaon, ses chars et ses cavaliers sont entrés dans la mer, et Yahweh a fait retourner sur eux les eaux de la mer~; mais les enfants d'Israël ont marché à sec au milieu de la mer.
\VS{20}Et Marie, la prophétesse, sœur d'Aaron, prit un tambour dans sa main, et toutes les femmes sortirent après elle, avec des tambours et des flûtes.
\VS{21}Et Marie leur répondait~: Chantez à Yahweh, car il est hautement élevé~; il a jeté dans la mer le cheval et celui qui le montait.
\TextTitle{Yahweh pourvoit pour son peuple}
\VS{22}Après cela, Moïse fit partir les Israélites de la Mer Rouge, et ils partirent vers le désert de Schur~; et ayant marché trois jours dans le désert, ils ne trouvèrent point d'eau.
\VS{23}De là, ils vinrent à Mara, mais ils ne purent boire les eaux de Mara, parce qu'elles étaient amères~; c'est pourquoi ce lieu fut appelé Mara.
\VS{24}Et le peuple murmura contre Moïse en disant~: Que boirons-nous~?
\VS{25}Et Moïse cria à Yahweh, et Yahweh lui montra\FTNT{«~Montra~» de l'hébreu «~yarah~» qui veut également dire «~enseigner~», «~signaler~», «~lancer~», «~instruire~», «~informer~», «~montrer~», «~jeter~» etc.} un certain bois qu'il jeta dans les eaux~; et les eaux devinrent douces. Il lui proposa là une ordonnance et une loi, et il l'éprouva là,
\VS{26}et lui dit~: Si tu écoutes attentivement la voix de Yahweh, ton Dieu, si tu fais ce qui est droit devant lui, si tu prêtes l'oreille à ses commandements, si tu gardes toutes ses ordonnances, je ne ferai venir sur toi aucune des infirmités que j'ai fait venir sur l'Egypte, car je suis Yahweh qui te guérit\FTNT{De. 7:12-15.}.
\VS{27}Puis ils vinrent à Elim, où il y avait douze fontaines d'eau, et soixante-dix palmiers. Et ils campèrent là, près des eaux.
\Chap{16}
\TextTitle{Yahweh envoie la manne}
\VerseOne{}Et toute l'assemblée des enfants d'Israël étant partie d'Elim, vint au désert de Sin, qui est entre Elim et Sinaï, le quinzième jour du second mois après qu'ils furent sortis du pays d'Egypte.
\VS{2}Et toute l'assemblée des enfants d'Israël murmura dans ce désert contre Moïse et Aaron.
\VS{3}Et les enfants d'Israël leur dirent~: Ah~! Pourquoi ne sommes-nous point morts par la main de Yahweh dans le pays d'Egypte, quand nous étions assis près des pots de viande, et que nous mangions du pain à satiété~? Car vous nous avez amenés dans ce désert pour faire mourir de faim toute cette assemblée\FTNT{1 Co. 10:10~; No. 11:4.}.
\VS{4}Et Yahweh dit à Moïse~: Voici, je vais vous faire pleuvoir des cieux du pain, et le peuple sortira et en recueillera chaque jour la provision d'un jour, afin que je l'éprouve, pour voir s'il observera ma loi ou non.
\VS{5}Mais qu'ils apprêtent au sixième jour ce qu'ils auront apporté, et qu'il y ait le double de ce qu'ils recueilleront chaque jour.
\VS{6}Moïse donc et Aaron dirent à tous les enfants d'Israël~: Ce soir vous saurez que Yahweh vous a tirés du pays d'Egypte.
\VS{7}Et au matin vous verrez la gloire de Yahweh, parce qu'il a entendu vos murmures, qui sont contre Yahweh~; car que sommes-nous pour que vous murmuriez contre nous~?
\VS{8}Moïse dit donc~: Ce sera quand Yahweh vous aura donné ce soir de la chair à manger, et qu'au matin, il vous aura rassasiés de pain, parce qu'il a entendu vos murmures, par lesquels vous avez murmuré contre lui. Car que sommes-nous~? Vos murmures ne sont pas contre nous, mais contre Yahweh.
\VS{9}Et Moïse dit à Aaron~: Dis à toute l'assemblée des enfants d'Israël~: Approchez-vous de la présence de Yahweh, car il a entendu vos murmures.
\VS{10}Or il arriva qu'aussitôt qu'Aaron eut parlé à toute l'assemblée des enfants d'Israël, ils regardèrent vers le désert, et voici, la gloire de Yahweh se montra dans la nuée.
\VS{11} Et Yahweh parla à Moïse, en disant~:
\VS{12}J'ai entendu les murmures des enfants d'Israël. Parle-leur et dis-leur~: Entre les deux soirs, vous mangerez de la chair, et au matin vous serez rassasiés de pain~; et vous saurez que je suis Yahweh, votre Dieu.
\VS{13}Sur le soir donc, il monta des cailles qui couvrirent le camp, et au matin il y eut une couche de rosée autour du camp.
\VS{14}Et cette couche de rosée étant évanouie, voici, sur la surface du désert, quelque chose de menu et de rond, comme du grain sur la terre.
\VS{15}Ce que les enfants d'Israël ayant vu, ils se dirent l'un à l'autre~: Qu'est-ce~? Car ils ne savaient ce que c'était. Et Moïse leur dit~: C'est le pain que Yahweh vous donne à manger\FTNT{Ps. 105:40.}.
\TextTitle{Récolte de la manne}
\VS{16}Or ce que Yahweh a ordonné, c'est que chacun en recueille autant qu'il lui en faut pour sa nourriture, un homer par tête, selon le nombre de vos personnes~; chacun en prendra pour ceux qui sont dans sa tente.
\VS{17}Les enfants d'Israël firent donc ainsi~; et les uns en recueillirent plus, les autres moins.
\VS{18}Et ils le mesuraient par homer~; et celui qui en avait recueilli beaucoup n'en avait pas plus qu'il ne lui en fallait~; ni celui qui en avait recueilli peu, n'en avait pas moins~; mais chacun en recueillait selon ce qu'il en pouvait manger.
\VS{19}Et Moïse leur avait dit~: Que personne n'en laisse rien de reste jusqu'au matin.
\VS{20}Mais il y en eut qui n'obéirent point à Moïse, car quelques-uns en réservèrent jusqu'au matin~; et il s'y engendra des vers, et cela puait. Et Moïse se mit en grande colère contre eux.
\VS{21}Ainsi, chacun en recueillait tous les matins autant qu'il lui en fallait pour se nourrir, et lorsque la chaleur du soleil était venue, elle se fondait.
\VS{22}Mais le sixième jour, ils recueillirent du pain en double, deux homers pour chacun~; et les principaux de l'assemblée vinrent pour le rapporter à Moïse.
\TextTitle{Le sabbat\FTNTT{Né. 9:13-14~; Mt. 12:1.}}
\VS{23} Et il leur dit~: C'est ce que Yahweh a dit~: Demain est le repos, le sabbat consacré à Yahweh~; faites cuire ce que vous avez à cuire, et faites bouillir ce que vous avez à bouillir, et serrez tout ce qui sera de surplus, pour le garder jusqu'au matin.
\VS{24}Ils le serrèrent donc jusqu'au matin, comme Moïse l'avait ordonné, et il ne pua point, et il n'y eut point de vers dedans.
\VS{25}Alors Moïse dit~: Mangez-le aujourd'hui, car c'est aujourd'hui le repos de Yahweh~; aujourd'hui vous n'en trouverez point dans les champs.
\VS{26}Durant six jours vous le recueillerez, mais le septième est le sabbat, il n'y en aura point ce jour-là.
\VS{27}Et au septième jour, quelques-uns du peuple sortirent pour en recueillir, mais ils n'en trouvèrent point.
\VS{28}Et Yahweh dit à Moïse~: Jusqu'à quand refuserez-vous de garder mes commandements et mes lois~?
\VS{29}Considérez que Yahweh vous a ordonné le sabbat, c'est pourquoi il vous donne au sixième jour du pain pour deux jours~; que chacun demeure au lieu où il sera, et qu'aucun ne sorte du lieu où il est le septième jour.
\VS{30}Le peuple donc se reposa le septième jour.
\VS{31}Et la maison d'Israël nomma ce pain manne\FTNT{Le mot «~manne~» vient de l'hébreu «~man~» et veut dire «~Qu'est-ce que cela~?~». La manne est une image de Jésus, le Pain de vie descendu du ciel (Jn. 6:32-52). La consommation quotidienne du Pain de vie, qui est aussi la Parole de Dieu, apporte la vie éternelle.}. Elle était comme de la semence de coriandre blanche, et ayant le goût d'un gâteau au miel.
\VS{32}Et Moïse dit~: Voici ce que Yahweh a ordonné~: Qu'on en remplisse un homer pour le garder pour vos générations, afin qu'on voie le pain que je vous ai fait manger au désert, après vous avoir retirés du pays d'Egypte.
\VS{33}Moïse dit à Aaron~: Prends un vase, et mets-y un plein d'homer de manne, et pose-le devant Yahweh, afin qu'il soit conservé pour vos générations.
\VS{34}Et Aaron le posa devant le témoignage pour y être gardé, selon que le Seigneur l'avait ordonné à Moïse.
\VS{35}Et les enfants d'Israël mangèrent la manne durant quarante ans, jusqu'à leur arrivée dans un pays habité~; ils mangèrent, dis-je, la manne, jusqu'à leur arrivée aux frontières du pays de Canaan.
\VS{36}Or un homer est la dixième partie d'un épha.
\Chap{17}
\TextTitle{Miracle de l'eau qui sort du rocher}
\VerseOne{}Et toute l'assemblée des enfants d'Israël partit du désert de Sin, selon l'ordre de marche que Yahweh leur avait ordonné, et ils campèrent à Rephidim, où il n'y avait point d'eau à boire pour le peuple.
\VS{2}Et le peuple se souleva contre Moïse et ils lui dirent~: Donnez-nous de l'eau à boire. Et Moïse leur dit~: Pourquoi vous soulevez-vous contre moi~? Pourquoi tentez-vous Yahweh\FTNT{No. 20:2-5.}~?
\VS{3}Le peuple donc eut soif en ce lieu-là, par faute d'eau~; et ainsi le peuple murmura contre Moïse, en disant~: Pourquoi nous as-tu fait monter hors d'Egypte, pour nous faire mourir de soif, nous, nos enfants, et nos troupeaux~?
\VS{4}Et Moïse cria à Yahweh, en disant~: Que ferai-je à ce peuple~? Encore un peu, et ils me lapideront.
\VS{5}Et Yahweh répondit à Moïse: Passe devant le peuple, et prends avec toi des anciens d'Israël, prends aussi dans ta main la verge avec laquelle tu as frappé le fleuve, et viens~!
\VS{6}Voici, je vais me tenir là devant toi sur le rocher d'Horeb~; et tu frapperas le rocher, et il en sortira des eaux, et le peuple en boira. Moïse donc fit ainsi aux yeux des anciens d'Israël\FTNT{De. 9:8~; Ps. 78:15~; 1 Co. 10:4.}.
\VS{7}Et il nomma le lieu Massa et Meriba, à cause de la querelle des enfants d'Israël, et parce qu'ils avaient tenté Yahweh, en disant~: Yahweh est-il au milieu de nous ou non~?
\TextTitle{Bataille et victoire contre Amalek}
\VS{8}Alors Amalek vint et livra bataille contre Israël à Rephidim\FTNT{De. 25:17-18.}.
\VS{9}Et Moïse dit à Josué~: Choisis-nous des hommes, et sors pour combattre contre Amalek~; et je me tiendrai demain sur le sommet de la colline, et la verge de Dieu sera dans ma main.
\VS{10}Et Josué fit comme Moïse lui avait ordonné en combattant contre Amalek. Mais Moïse, Aaron et Hur montèrent au sommet de la colline.
\VS{11}Et il arrivait que lorsque Moïse élevait sa main, Israël était alors le plus fort, mais quand il reposait sa main, alors Amalek était le plus fort.
\VS{12}Et les mains de Moïse étant devenues pesantes, ils prirent une pierre et la mirent sous lui, et il s'assit dessus~; Aaron et Hur soutenaient ses mains, l'un d'un côté, et l'autre de l'autre côté~; et ainsi ses mains furent fermes jusqu'au soleil couchant.
\VS{13}Josué donc défit Amalek et son peuple au tranchant de l'épée.
\VS{14}Et Yahweh dit à Moïse~: Ecris ceci pour mémoire dans un livre, et fais entendre à Josué que j'effacerai entièrement la mémoire d'Amalek de dessous les cieux.
\VS{15}Et Moïse bâtit un autel et le nomma Yahweh, ma bannière.
\VS{16}Il dit aussi~: Parce que la main a été levée contre le trône de Yahweh, Yahweh aura toujours la guerre contre Amalek.
\Chap{18}
\TextTitle{Jéthro conseille Moïse}
\VerseOne{}Or Jéthro, prêtre de Madian, beau-père de Moïse, apprit toutes les choses que Yahweh avait faites à Moïse, et à Israël, son peuple, à savoir comment Yahweh avait retiré Israël de l'Egypte.
\VS{2}Jéthro, beau-père de Moïse, prit Séphora la femme de Moïse, après que Moïse l'eut renvoyée,
\VS{3}et les deux fils de cette femme, dont l'un s'appelait Guerschom, car il avait dit~: J'habite un pays étranger~;
\VS{4}et l'autre Eliézer, car il avait dit~: Le Dieu de mon père m'a secouru et m'a délivré de l'épée de Pharaon.
\VS{5}Jéthro donc, beau-père de Moïse, vint vers Moïse avec ses fils et sa femme au désert, où il était campé, à la montagne de Dieu.
\VS{6}Il fit dire à Moïse~: Jéthro, ton beau-père, vient vers toi, et ta femme et ses deux fils avec elle.
\VS{7}Et Moïse sortit au-devant de son beau-père, et s'étant prosterné, il l'embrassa~; et ils s'enquirent l'un de l'autre, de leur santé, puis ils entrèrent dans la tente.
\VS{8}Et Moïse raconta à son beau-père toutes les choses que Yahweh avait faites à Pharaon et aux Egyptiens en faveur d'Israël, et toute la fatigue qu'ils avaient soufferte en chemin, et comment Yahweh les avait délivrés.
\VS{9}Et Jéthro se réjouit de tout le bien que Yahweh avait fait à Israël, parce qu'il les avait délivrés de la main des Egyptiens.
\VS{10}Puis Jéthro dit~: Béni soit Yahweh qui vous a délivrés de la main des Egyptiens et de la main de Pharaon, qui a, dis-je, délivré le peuple de la main des Egyptiens~!
\VS{11}Je sais maintenant que le Seigneur est plus grand que tous les dieux, car la chose même en laquelle ils se sont enorgueillis, il a eu le dessus sur eux.
\VS{12}Jéthro, beau-père de Moïse, apporta aussi un holocauste et des sacrifices pour les offrir à Dieu. Et Aaron et tous les anciens d'Israël vinrent pour manger du pain avec le beau-père de Moïse dans la présence de Dieu.
\VS{13}Et il arriva, le lendemain, comme Moïse siégeait pour juger le peuple, et que le peuple se tenait devant Moïse depuis le matin jusqu'au soir,
\VS{14}que le beau-père de Moïse vit tout ce qu'il faisait au peuple, et il lui dit~: Qu'est-ce que tu fais à l'égard de ce peuple~? Pourquoi es-tu assis seul, et tout le peuple se tient devant toi depuis le matin jusqu'au soir~?
\VS{15} Et Moïse répondit à son beau-père~: C'est que le peuple vient à moi pour s'enquérir de Dieu.
\VS{16}Quand ils ont quelque affaire, ils viennent à moi, et je juge entre l'un et l'autre, et je leur fais entendre les ordonnances de Dieu et ses lois.
\VS{17}Mais le beau-père de Moïse lui dit~: Ce que tu fais n'est pas bien.
\VS{18}Certainement, tu succomberas, toi et ce peuple qui est avec toi~; car cela est trop pesant pour toi, tu ne saurais faire cela toi seul.
\VS{19}Ecoute donc mon conseil~; je te conseillerai et Dieu sera avec toi: Sois pour ce peuple auprès de Dieu, et rapporte les causes à Dieu.
\VS{20}Et instruis-les des ordonnances et des lois~; et fais-leur connaître la voie par laquelle ils auront à marcher et ce qu'ils auront à faire.
\VS{21}Et choisis-toi d'entre tout le peuple des hommes vertueux, craignant Dieu~; des hommes véritables, haïssant le gain déshonnête, et établis-les chefs de milliers, chefs de centaines, chefs de cinquantaines et chef des dizaines.
\VS{22}Et qu'ils jugent le peuple en tout temps, mais qu'ils te rapportent toutes les grandes affaires, et qu'ils jugent toutes les petites causes~; ainsi ils te soulageront et porteront une partie de la charge avec toi.
\VS{23}Si tu fais cela, et que Dieu te l'ordonne, tu pourras subsister, et tout le peuple parviendra en paix à destination.
\VS{24}Moïse donc obéit à la parole de son beau-père, et fit tout ce qu'il lui avait dit.
\VS{25}Ainsi, Moïse choisit de tout Israël des hommes vertueux, et les établit chefs sur le peuple, chefs de milliers, chefs de centaines, chefs de cinquantaines, et chefs de dizaines,
\VS{26}lesquels devaient juger le peuple en tout temps, mais ils devaient rapporter à Moïse les choses difficiles, et juger de toutes les petites affaires.
\VS{27}Puis Moïse laissa partir son beau-père, qui s'en alla dans son pays.
\Chap{19}
\TextTitle{DEBUT DE LA PÉRIODE DE LA LOI MOSAÏQUE OU DE LA PREMIÈRE ALLIANCE}
\VerseOne{}Au premier jour du troisième mois, après que les enfants d'Israël furent sortis du pays d'Egypte, en ce même jour-là, ils vinrent au désert de Sinaï.
\VS{2}Etant donc partis de Rephidim, ils vinrent au désert de Sinaï et campèrent au désert. Et Israël campa vis-à-vis de la montagne.
\VS{3}Et Moïse monta vers Dieu, car Yahweh l'avait appelé de la montagne pour lui dire~: Tu parleras ainsi à la maison de Jacob, et tu annonceras ceci aux enfants d'Israël~:
\VS{4}Vous avez vu ce que j'ai fait aux Egyptiens~; comment je vous ai portés comme sur des ailes d'aigle et vous ai amenés à moi.
\VS{5}Maintenant donc, si vous obéissez exactement à ma voix, et si vous gardez mon alliance, vous serez aussi d'entre tous les peuples mon plus précieux joyau, car toute la terre m'appartient\FTNT{C'est ici que débute la période de la Loi ou Première Alliance. Le fait d'avoir réuni les textes de Genèse à Malachie sous l'appellation «~Ancien Testament~» a induit beaucoup de personnes en erreur quant à leur compréhension du plan de Dieu pour nos vies. Tout d'abord, l'emploi du mot «~testament~» est inapproprié puisqu'on ne peut parler de testament sans qu'il y ait au préalable la mort du testateur (Hé. 9:16-17). Certes, des animaux étaient tués sous la Loi pour couvrir les péchés. Toutefois, ces sacrifices étaient imparfaits et par conséquent prévus pour ne durer qu'un temps, en attendant le sacrifice parfait de Jésus-Christ (Hé. 10:1-14). De plus, il est évident que les animaux sacrifiés ne nous ont rien légué.\\ Ensuite, il est à noter que tous les textes classés dans ce que l'on appelle à tort «~Ancien Testament~» ne se rapportent pas exclusivement et nécessairement à la Loi. Ainsi, des prophètes, en commençant par Moïse en personne, ayant vécu sous la Loi, ont prophétisé et écrit sur d'autres sujets que la Loi, notamment sur la grâce et la fin des temps. N'oublions pas non plus que Jésus-Christ est né et a vécu sous la Loi (Ga. 4:4). En tant que juif, il l'a scrupuleusement respectée de telle sorte qu'elle fut totalement accomplie en Lui (Mt. 5:17-18~; Jn. 19:30). En conséquence, la fin de la Loi mosaïque eut lieu après la mort du Seigneur, précisément au moment où le Seigneur a dit «~Tout est accompli~», et lorsque le voile du temple s'est déchiré de haut en bas (Mt. 27:50-51~; Jn. 19:30). La Nouvelle Alliance, ou le Testament de Jésus débuta avec l'effusion de l'Esprit (Ac. 2). De la mort du Seigneur à la Pentecôte, une période de transition de cinquante jours s'est écoulée. Jésus-Christ s'est présenté pendant ce temps dans le sanctuaire céleste pour présenter son sang dans le Saint des saints. Une fois son sacrifice examiné et accepté, le Saint-Esprit qui avait été retiré de l'homme (Ge. 6:3) put de nouveau revenir habiter le cœurs des croyants.\\Mais qu'est-ce que la loi exactement~? Beaucoup de chrétiens sont dans la confusion à ce sujet. En réalité, il n'y avait pas qu'une loi mais trois sortes de lois~: Les lois morales et les lois cérémonielles qui préexistaient depuis l'éternité~; et les lois civiles qui ont débuté avec Moïse car elles ne concernaient que son peuple.\\- Les lois civiles régissaient le fonctionnement de la vie en communauté des Hébreux. Elles étaient exclusivement réservées au peuple d'Israël dans le camp puis dans le pays de Canaan (Ex. 21:1-2~; De. 23).\\- Les lois morales font référence à la nature de Dieu~: Son amour, sa justice, sa sainteté, etc. Les dix commandements, à l'exception du sabbat tel que prescrit par Moïse (Ex. 16:28-29~; Lé. 15:32), font partie des lois morales (Ex. 20:1-17). Les dix paroles ne constituent qu'une base, un résumé. Ainsi, d'autres règles morales sont énoncées tout au long des Ecritures notamment sur la sexualité (Lé. 18:1-22), l'interdiction des sacrifices humains et de l'occultisme (De. 18:10-13), le respect d'autrui et l'entraide (Lé. 19:10-18~; Lé. 19:29-36). Comme il est impossible de consigner dans un livre tous les péchés moraux, le Seigneur a inscrit les lois morales dans le cœur de l'homme afin qu'il sache instinctivement faire la différence entre le bien et mal (Ro. 2:14-15). Jésus les a résumées en ces quelques mots~: «~Tu aimeras le Seigneur ton Dieu de tout ton cœur, et de toute ton âme, et de toute ta pensée. Celui-ci est le premier et le plus grand commandement. Et le second semblable à celui-là, est~: Tu aimeras ton prochain comme toi-même.~» (Mt. 22:37-39). Ces lois sont encore en vigueur aujourd'hui et le resteront pour toujours.\\- Les lois cérémonielles étaient relatives au culte et au sanctuaire terrestre, c'est-à-dire le tabernacle puis le temple de Jérusalem (Hé. 9:1-10). Elles regroupent toutes les ordonnances concernant les sacrifices, les ablutions, les sabbats, les fêtes de Yahweh, la dîme des Lévites et des prêtres (voir commentaire en No. 18:21 et Mal. 3:10). Les livres du Lévitique et des Nombres exposent en détail toutes les ordonnances reçues par Moïse d'après le modèle céleste que Yahweh lui avait montré sur le Mont Sinaï (Ex. 26:30). Les lois cérémonielles préexistaient donc depuis l'éternité.\\Les lois cérémonielles représentent la Première Alliance qui avait pour fondement la loi morale. Or cette alliance a vieilli puis disparu car elle n'était que l'ombre des choses à venir (Hé. 8:13). En effet, elle était basée sur quatre points principaux~: Le temple, le culte centralisé, le sacrifice et les prêtres. En Christ, nous n'avons plus besoin d'un temple physique puisque nous sommes devenus les temples vivants de Yahweh (1 Co. 6:19~; Ep. 2:22). Nous pouvons désormais adorer le Seigneur en Esprit et en vérité, à tout moment et en tout lieu (Jn. 4:23). Le sacerdoce lévitique ayant été aboli, chaque enfant de Dieu est devenu un prêtre (Ap. 5:10) qui offre en sacrifice sa propre vie consacrée au Seigneur (Ro. 12:1).\\Les lois cérémonielles ont donc trouvé leur parfait accomplissement en Jésus-Christ~: Tous les sacrifices sanglants le préfiguraient, toutes les solennités ont été réalisées en Lui (voir note en Lé. 23). Christ est donc la fin de la Loi, non pas morale, mais cérémonielle (Ro. 10:4).\\Un lien étroit existe entre les lois morales et les lois cérémonielles. La loi morale est comme un diagnostic qui révèle une pathologie incurable comme le sida~: Le péché (Ro. 5:13-20~; Ro. 7:7-14). En la découvrant, l'homme se sent condamné car il réalise qu'il ne peut pas répondre aux exigences de la justice divine. La loi cérémonielle (le sang des animaux - Hé. 9:1-13~; Hé. 10:11) a donné aux hommes une sorte de trithérapie pour les soulager provisoirement de leurs péchés mais sans pour autant les ôter (guérir, délivrer, nettoyer, laver) définitivement. Seul le sang de la Nouvelle Alliance, c'est-à-dire le sang de Jésus-Christ, a pu nous délivrer une fois pour toutes (Jn. 1:29~; Hé. 9:11-26~; Hé. 10:1-23~; Ap. 1:6).}.
\VS{6}Et vous me serez un royaume de prêtres, et une nation sainte~; ce sont là les discours que tu tiendras aux enfants d'Israël.
\VS{7}Puis Moïse vint et appela les anciens du peuple, et proposa devant eux toutes ces choses-là que Yahweh lui avait ordonné.
\VS{8}Et tout le peuple répondit d'un commun accord, en disant~: Nous ferons tout ce que Yahweh a dit. Et Moïse rapporta à Yahweh toutes les paroles du peuple.
\TextTitle{Moïse doit sanctifier le peuple pour qu'il rencontre Yahweh}
\VS{9}Et Yahweh dit à Moïse~: Voici, je viendrai à toi dans une nuée épaisse, afin que le peuple entende quand je parlerai avec toi, et qu'il te croie aussi toujours~; car Moïse avait rapporté à Yahweh les paroles du peuple.
\VS{10}Yahweh dit aussi à Moïse~: Va-t'en vers le peuple, et sanctifie-les aujourd'hui et demain, et qu'ils lavent leurs vêtements.
\VS{11}Et qu'ils soient tous prêts pour le troisième jour, car au troisième jour, Yahweh descendra sur la montagne de Sinaï, à la vue de tout le peuple.
\VS{12}Or tu mettras des bornes pour le peuple tout autour, et tu diras~: Gardez-vous de monter sur la montagne et de toucher aucune de ses extrémités. Quiconque touchera la montagne sera puni de mort.
\VS{13}Aucune main ne la touchera, et certainement il sera lapidé, ou percé de flèches~; soit bête, soit homme, il ne vivra point. Quand la trompette sonnera longuement, ils monteront vers la montagne.
\VS{14}Et Moïse descendit de la montagne vers le peuple, et sanctifia le peuple, et ils lavèrent leurs vêtements.
\VS{15}Et il dit au peuple~: Soyez tous prêts pour le troisième jour, et ne vous approchez point de vos femmes.
\VS{16}Et le troisième jour au matin, il y eut des tonnerres, et des éclairs, et une grosse nuée sur la montagne, avec un très fort son de shofar, et tout le peuple dans le camp fut effrayé.
\VS{17}Alors Moïse fit sortir le peuple du camp pour aller au-devant de Dieu~; et ils s'arrêtèrent au pied de la montagne.
\VS{18}Or le mont Sinaï était tout couvert de fumée, parce que Yahweh y était descendu en feu~; et sa fumée montait comme la fumée d'une fournaise, et toute la montagne tremblait fort.
\VS{19}Et comme le son du shofar se renforçait de plus en plus, Moïse parla, et Dieu lui répondit par une voix.
\VS{20}Yahweh donc étant descendu sur la montagne de Sinaï, au sommet de la montagne, Yahweh appela Moïse au sommet de la montagne~; et Moïse y monta.
\VS{21}Et Yahweh dit à Moïse~: Descends. Somme le peuple qu'il ne rompe point les barrières pour monter vers Yahweh afin de regarder~; de peur qu'un grand nombre d'entre eux ne périsse.
\VS{22}Et même, que les prêtres qui s'approchent de Yahweh se sanctifient aussi, de peur qu'il n'arrive que Yahweh se jette sur eux.
\VS{23} Et Moïse dit à Yahweh~: Le peuple ne pourra pas monter sur la montagne de Sinaï, parce que tu nous as sommés en me disant~: Mets des bornes sur la montagne, et sanctifie-la.
\VS{24}Et Yahweh lui dit~: Va, descends~; puis tu monteras, toi, et Aaron avec toi~; mais que les prêtres et le peuple ne rompent point les bornes pour monter vers Yahweh, de peur qu'il n'arrive qu'il se jette sur eux.
\VS{25}Moïse descendit donc vers le peuple, et lui dit ces choses.
\Chap{20}
\TextTitle{Les dix paroles}
\VerseOne{}Alors Dieu prononça toutes ces paroles, disant~:
\VS{2}Je suis Yahweh, ton Dieu, qui t'ai retiré du pays d'Egypte, de la maison de servitude.
\VS{3}Tu n'auras point d'autres dieux devant ma face.
\VS{4}Tu ne te feras point d'image taillée, ni aucune ressemblance des choses qui sont là-haut aux cieux, ni ici-bas sur la terre, ni dans les eaux sous la terre\FTNT{Lé. 26:1.}.
\VS{5}Tu ne te prosterneras point devant elles, et ne les serviras point~; car je suis Yahweh, ton Dieu~; le Dieu qui est jaloux, punissant l'iniquité des pères sur les fils, jusqu'à la troisième et à la quatrième génération de ceux qui me haïssent~;
\VS{6}et faisant miséricorde en mille générations à ceux qui m'aiment et qui gardent mes commandements.
\VS{7}Tu ne prendras point le Nom de Yahweh, ton Dieu, en vain~; car Yahweh ne tiendra point pour innocent celui qui aura pris son Nom en vain\FTNT{Lé. 19:12~; Mt. 5:33.}.
\VS{8}Souviens-toi du jour du repos pour le sanctifier.
\VS{9}Tu travailleras six jours, et tu feras toute ton œuvre.
\VS{10}Mais le septième jour est le repos de Yahweh ton Dieu. Tu ne feras aucune œuvre en ce jour-là, ni toi, ni ton fils, ni ta fille, ni ton serviteur, ni ta servante, ni ton bétail, ni ton étranger qui est dans tes portes.
\VS{11}Car Yahweh a fait en six jours les cieux, la terre, la mer, et tout ce qui est en eux, et s'est reposé le septième jour~; c'est pourquoi Yahweh a béni le jour du repos et l'a sanctifié\FTNT{Ge. 2:3~; Ex. 31:14~; Ez. 20:12.}.
\VS{12}Honore ton père et ta mère, afin que tes jours soient prolongés sur la terre que Yahweh, ton Dieu, te donne\FTNT{Lé. 19:3~; De. 5:16~; Mt. 15:4~; Ep. 6:2.}.
\VS{13}Tu ne commettras pas de meurtre\FTNT{Mt. 5:21.}.
\VS{14}Tu ne commettras pas d'adultère\FTNT{Lé. 20:10~; De. 5:18~; Pr. 6:32~; Mt. 5:32~; Ro. 7:3.}.
\VS{15}Tu ne déroberas pas.
\VS{16}Tu ne diras pas de faux témoignage contre ton prochain.
\VS{17}Tu ne convoiteras pas la maison de ton prochain~; tu ne convoiteras pas la femme de ton prochain, ni son serviteur, ni sa servante, ni son bœuf, ni son âne, ni aucune chose qui soit à ton prochain.
\TextTitle{Le peuple tout tremblant devant Yahweh}
\VS{18}Or tout le peuple apercevait les tonnerres, les éclairs, le son du shofar, et la montagne fumante. Et le peuple voyant cela tremblait et se tenait loin.
\VS{19}Et ils dirent à Moïse~: Parle, toi, avec nous, et nous écouterons~; mais que Dieu ne parle point avec nous, de peur que nous ne mourrions\FTNT{De. 5:23-24~; Hé. 12:18-19.}.
\VS{20}Et Moïse dit au peuple~: Ne craignez point car Dieu est venu pour vous éprouver, et afin que sa crainte soit devant vous, et que vous ne péchiez point.
\VS{21}Le peuple donc se tint loin, mais Moïse s'approcha de l'obscurité dans laquelle Dieu était.
\VS{22}Et Yahweh dit à Moïse~: Tu diras ainsi aux enfants d'Israël~: Vous avez vu que je vous ai parlé des cieux.
\VS{23}Vous ne vous ferez point avec moi de dieux d'argent ni de dieux d'or.
\VS{24}Tu me feras un autel de terre, sur lequel tu sacrifieras tes holocaustes, et tes offrandes de paix\FTNT{Voir commentaire en Lé. 3:1.}, ton menu et ton gros bétail. En quelque lieu que ce soit où je mettrai la mémoire de mon Nom, je viendrai là à toi, et je te bénirai.
\VS{25}Si tu me fais un autel de pierres, ne les taille point~; car si tu fais passer le fer dessus, tu le souillerais.
\VS{26}Et tu ne monteras point à mon autel par des marches, de peur que ta nudité ne soit découverte en y montant.
\Chap{21}
\TextTitle{Lois sur les maîtres et leurs esclaves}
\VerseOne{}Ce sont ici les lois que tu leur proposeras.
\VS{2}Si tu achètes un esclave Hébreu, il te servira six ans, et au septième il sortira pour être libre, sans rien payer\FTNT{Lé. 25:39-43~; De. 15:12~; Jé. 34:14.}.
\VS{3}S'il est venu avec son corps seulement, il sortira avec son corps~; s'il avait une femme, sa femme sortira aussi avec lui.
\VS{4}Si son maître lui a donné une femme qui lui ait enfanté des fils ou des filles, sa femme et les enfants qu'il en aura seront à son maître, mais il sortira avec son corps.
\VS{5}Si l'esclave dit positivement: J'aime mon maître, ma femme, et mes fils, je ne sortirai point pour être libre.
\VS{6}Alors son maître le fera venir devant les juges, et le fera approcher de la porte ou du poteau, et son maître lui percera l'oreille avec un poinçon~; et il le servira pour toujours.
\VS{7}Si quelqu'un vend sa fille pour être esclave, elle ne sortira point comme les esclaves sortent.
\VS{8}Si elle déplaît à son maître, qui ne l'aura point fiancée, il la fera acheter~; mais il n'aura pas le pouvoir de la vendre à un peuple étranger, après lui avoir été infidèle.
\VS{9}Mais s'il l'a fiancée à son fils, il fera pour elle selon le droit des filles.
\VS{10}S'il en prend une pour lui, il ne retranchera rien de sa nourriture, de ses vêtements et du droit conjugal.
\VS{11}S'il ne fait pas pour elle ces trois choses-là, elle sortira sans payer aucun argent.
\TextTitle{Lois sur les dommages corporels}
\VS{12}Si quelqu'un frappe un homme et qu'il en meure,on le fera mourir de mort \FTNT{Lé. 24:17~; No. 35:11-16~; De. 19:2-11~; Jos. 20:2.}.
\VS{13}S'il ne lui a point dressé d'embûches, mais que Dieu l'ait fait tomber entre ses mains, je t'établirai un lieu où il s'enfuira.
\VS{14}Mais si quelqu'un s'élève de propos délibéré contre son prochain, pour le tuer par ruse, tu le tireras de mon autel, afin qu'il meure.
\VS{15}Celui qui aura frappé son père ou sa mère sera puni de mort\FTNT{Lé. 20:9~; De. 27:16~; Mt. 15:4.}.
\VS{16}Si quelqu'un dérobe un homme et le vend, ou s'il est trouvé entre ses mains, on le fera mourir de mort.
\VS{17}Celui qui aura maudit son père ou sa mère sera puni de mort.
\VS{18}Si quelques uns ont une querelle, et que l'un ait frappé l'autre d'une pierre ou du poing, sans causer sa mort, mais qu'il soit obligé de se mettre au lit,
\VS{19}s'il se lève et marche dehors en s'appuyant sur son bâton, celui qui l'aura frappé sera absous~; toutefois, il le dédommagera de ce qu'il a chômé et le fera guérir entièrement.
\VS{20}Si quelqu'un a frappé du bâton son serviteur ou sa servante, et qu'il soit mort sous sa main, on ne manquera point de le venger.
\VS{21}Mais s'il survit un jour ou deux, il ne sera point vengé, car c'est son argent.
\VS{22}Si des hommes se querellent, et que l'un d'eux frappe une femme enceinte, et qu'elle en accouche, s'il n'y a pas cas de mort, il sera condamné à l'amende que le mari de la femme lui imposera, et il la donnera selon que les juges en ordonneront.
\VS{23}Mais s'il y a cas de mort, tu donneras vie pour vie,
\VS{24}œil pour œil, dent pour dent, main pour main, pied pour pied\FTNT{Lé. 24:20~; De. 19:21~; Mt. 5:38.},
\VS{25}brûlure pour brûlure, plaie pour plaie, meurtrissure pour meurtrissure.
\VS{26}Si quelqu'un frappe l'œil de son serviteur, ou l'œil de sa servante, et lui gâte l'œil, il le laissera aller libre pour son œil~;
\VS{27}et s'il fait tomber une dent à son serviteur, ou à sa servante, il le laissera aller libre pour sa dent.
\VS{28}Si un bœuf heurte de sa corne un homme ou une femme, et que la personne en meure, le bœuf sera lapidé sans nulle exception, et on ne mangera point de sa chair, mais le maître du bœuf sera absous.
\VS{29}Si le bœuf était auparavant sujet à frapper de sa corne, et que son maître en ait été averti avec protestation, et qu'il ne l'ait point surveillé, s'il tue un homme ou une femme, le bœuf sera lapidé, et on fera aussi mourir son maître.
\VS{30}Si on lui impose un prix pour se racheter, il donnera la rançon de sa vie, selon tout ce qui lui sera imposé.
\VS{31}Si le bœuf heurte de sa corne un fils ou une fille, il lui sera fait selon cette même loi.
\VS{32}Si le bœuf heurte de sa corne un esclave, soit homme, soit femme, celui à qui est le bœuf donnera trente sicles d'argent au maître de l'esclave, et le bœuf sera lapidé.
\VS{33}Si quelqu'un découvre une fosse, ou si quelqu'un creuse une fosse, et ne la couvre point, et qu'il y tombe un bœuf ou un âne,
\VS{34}le maître de la fosse donnera satisfaction, et rendra l'argent au maître du bœuf, mais la bête morte lui appartiendra.
\VS{35}Et si le bœuf de quelqu'un blesse le bœuf de son prochain, et qu'il en meure, ils vendront le bœuf vivant, et en partageront l'argent par moitié, ils partageront aussi par moitié le bœuf mort.
\VS{36}Mais s'il est connu que le bœuf avait auparavant l'habitude de heurter avec sa corne, et que le maître ne l'ait point gardé, il restituera bœuf pour bœuf~; mais le bœuf mort sera pour lui.
\Chap{22}
\TextTitle{Lois sur les torts causés à autrui}
\VerseOne{}Si quelqu'un dérobe un bœuf, ou un chevreau, ou un agneau, et qu'il le tue, ou le vende, il restituera cinq bœufs pour le bœuf, et quatre agneaux ou chevreaux pour l'agneau ou pour le chevreau.
\VS{2}Si le voleur est trouvé dérobant avec effraction, et est frappé de sorte qu'il en meure, celui qui l'aura frappé ne sera point coupable de meurtre.
\VS{3}Mais si le soleil est levé sur lui, il sera coupable de meurtre. Il fera donc une entière restitution~; et s'il n'a pas de quoi, il sera vendu pour son vol.
\VS{4}Si ce qui a été dérobé est trouvé vivant entre ses mains, soit bœuf, soit âne, soit brebis ou chèvre, il rendra le double.
\VS{5}Si quelqu'un fait brouter dans un champ ou dans une vigne, en lâchant son bétail qui aille paître dans le champ d'autrui, il rendra le meilleur de son champ et le meilleur de sa vigne.
\VS{6}Si un feu éclate et rencontre des épines, et que le blé qui est en tas, ou sur pied, ou le champ, soit consumé, celui qui aura allumé le feu rendra entièrement ce qui en aura été brûlé.
\VS{7}Si quelqu'un donne à son prochain de l'argent ou des vases à garder, et qu'on le dérobe de sa maison, et si l'on trouve le voleur, il rendra le double\FTNT{Lé. 5:20-26.}.
\VS{8}Mais si on ne trouve point le voleur, on fera venir le maître de la maison devant les juges pour jurer s'il n'a point mis sa main sur le bien de son prochain.
\VS{9}Dans toute affaire d'infidélité concernant un bœuf, un âne, une brebis, une chèvre, un vêtement ou tout objet perdu, dont quelqu'un dira qu'il lui appartient, la cause des deux parties viendra devant les juges~; et celui que les juges auront condamné, rendra le double à son prochain.
\VS{10}Si quelqu'un donne à garder à son prochain un âne, un bœuf, quelque menue ou grosse bête, et qu'elle meure, ou qu'elle se soit cassée quelque membre, ou qu'on l'ait emmenée sans que personne l'ait vue,
\VS{11}le serment de Yahweh interviendra entre les deux parties\FTNT{Hé. 6:16.}, pour savoir s'il n'a point mis sa main sur le bien de son prochain, et le maître de la bête se contentera du serment, et l'autre ne la rendra point.
\VS{12}Mais s'il est vrai qu'elle lui a été dérobée, il la rendra à son maître.
\VS{13}S'il est vrai qu'elle ait été déchirée par les bêtes sauvages, il la produira en témoignage, et il ne rendra point ce qui a été déchiré.
\VS{14}Si quelqu'un a emprunté de son prochain quelque bête, et qu'elle se casse quelque membre, ou qu'elle meure, son maître n'étant point présent, il ne manquera pas de la rendre.
\VS{15}Mais si son maître est avec lui, il ne la rendra point~; si elle a été louée, on payera seulement son louage.
\TextTitle{Lois diverses}
\VS{16}Si un homme séduit une vierge non fiancée, et couche avec elle, il faut qu'il la dote, et qu'il la prenne pour femme\FTNT{De. 22:28.}.
\VS{17}Mais si le père de la fille refuse absolument de la lui donner, il lui comptera autant d'argent qu'on en donne pour la dot des vierges.
\VS{18}Tu ne laisseras point vivre la sorcière\FTNT{De. 18:10-11~; Lé. 20:27.}.
\VS{19}Celui qui couche avec une bête sera puni de mort\FTNT{Lé. 18:23~; Lé. 20:15~; De. 27:21.}.
\VS{20}Celui qui sacrifie à d'autres dieux qu'à Yahweh seul sera dévoué à la façon de l'interdit\FTNT{Lé. 17:7~; De. 13:6-16~; De. 17:2-5.}.
\VS{21}Tu ne fouleras ni n'opprimeras point l'étranger~; car vous avez été étrangers au pays d'Egypte\FTNT{Lé. 19:34.}.
\VS{22}Vous n'affligerez point la veuve ni l'orphelin\FTNT{De. 24:17-18~; Za. 7:10.}.
\VS{23}Si vous les affligez en quoi que ce soit, et qu'ils crient à moi, certainement j'entendrai leur cri.
\VS{24}Et ma colère s'embrasera, et je vous ferai mourir par l'épée~; et vos femmes seront veuves, et vos fils orphelins.
\VS{25}Si tu prêtes de l'argent à mon peuple, au pauvre qui est avec toi, tu ne te comporteras point avec lui en créancier, vous ne lui exigerez point d'intérêt.
\VS{26}Si tu prends en gage le vêtement de ton prochain, tu le lui rendras avant que le soleil soit couché\FTNT{De. 24:10-13.}.
\VS{27}Car c'est sa seule couverture, c'est son vêtement pour couvrir sa peau~; où coucherait-il~? S'il arrive donc qu'il crie à moi, je l'entendrai~; car je suis miséricordieux.
\VS{28}Tu ne maudiras point les juges, et tu ne maudiras point le prince de ton peuple\FTNT{Lé. 24:15-16.}.
\VS{29}Tu ne différeras point de m'offrir de ton abondance et de tes liqueurs~; tu me donneras le premier-né de tes fils\FTNT{Ex. 13:12-15~; De. 26:2-11.}.
\VS{30}Tu feras la même chose de ta vache, de ta brebis, et de ta chèvre. Il sera sept jours avec sa mère, et le huitième jour tu me le donneras.
\VS{31}Vous me serez saints, et vous ne mangerez point de la chair déchirée dans les champs, mais vous la jetterez aux chiens.
\Chap{23}
\TextTitle{Lois diverses (suite)}
\VerseOne{}Tu ne léveras point de faux bruit, et tu ne te joindras point au méchant pour être un faux témoin, afin que violence soit faite\FTNT{Ex. 20:16~; De. 19:16-21.}.
\VS{2}Tu ne suivras point la multitude pour faire le mal~; et tu ne témoigneras point dans un procès en sorte que tu te détournes après un grand nombre pour pervertir le droit.
\VS{3}Tu n'honoreras point le pauvre dans son procès\FTNT{De. 1:17.}.
\VS{4}Si tu rencontres le bœuf de ton ennemi, ou son âne égaré, tu ne manqueras point de le lui ramener.
\VS{5}Si tu vois l'âne de celui qui te hait, abattu sous sa charge, tu t'arrêteras pour le secourir, et tu ne manqueras pas de l'aider.
\VS{6}Tu ne pervertiras point le droit de l'indigent, qui est au milieu de toi, dans son procès.
\VS{7}Tu t'éloigneras de toute parole fausse, et tu ne feras point mourir l'innocent et le juste~; car je ne justifierai point le méchant.
\VS{8}Tu ne prendras point de présent~; car le présent aveugle les plus éclairés, et pervertit les paroles des justes.
\VS{9}Tu n'opprimeras point l'étranger~; car vous savez ce que c'est que d'être étrangers, parce que vous avez été étrangers au pays d'Egypte.
\TextTitle{Le sabbat, le repos de la terre}
\VS{10}Pendant six ans tu ensemenceras ta terre, et en recueilleras le revenu.
\VS{11}Mais la septième année, tu lui donneras du relâche, et la laisseras reposer, afin que les pauvres de ton peuple en mangent, et que les bêtes des champs mangent ce qui restera. Tu en feras de même de ta vigne et de tes oliviers.
\VS{12}Tu travailleras six jours, mais tu te reposeras au septième jour, afin que ton bœuf et ton âne se reposent, et que le fils de ta servante et l'étranger reprennent courage.
\VS{13}Vous prendrez garde à toutes les choses que je vous ai ordonnées. Vous ne ferez point mention du nom des dieux étrangers, on ne l'entendra point de ta bouche\FTNT{Jos. 23:7~; Ps. 16:4.}.
\TextTitle{Les fêtes solennelles}
\VS{14}Trois fois l'an, tu me célébreras une fête solennelle\FTNT{Lé. 23:4-44.}.
\VS{15}Tu garderas la fête solennelle des pains sans levain\FTNT{Ex. 29:2.}~; tu mangeras des pains sans levain pendant sept jours, comme je t'ai ordonné, en la saison et au mois où les épis mûrissent~; car c'est en ce mois-là que tu es sorti d'Egypte~; et nul ne se présentera devant ma face à vide.
\VS{16}Et la fête solennelle de la moisson des premiers fruits de ton travail, de ce que tu auras semé au champ~; et la fête de la récolte, après la fin de l'année, quand tu auras recueilli du champ les fruits de ton travail\FTNT{Ex. 34:22.}.
\VS{17}Trois fois l'an, tous les mâles d'entre vous se présenteront devant le Seigneur Yahweh.
\VS{18}Tu ne sacrifieras point le sang de mon sacrifice avec du pain levé~; et la graisse de ma fête solennelle ne passera point la nuit jusqu'au matin\FTNT{Ex. 34:25-26.}.
\VS{19}Tu apporteras dans la maison de Yahweh, ton Dieu, les prémices des premiers fruits de ta terre. Tu ne feras point cuire le chevreau dans le lait de sa mère.
\TextTitle{Mises en garde et promesses de Yahweh}
\VS{20}Voici, j'envoie un Ange devant toi, afin qu'il te garde dans le chemin, et qu'il t'introduise dans le lieu que je t'ai préparé.
\VS{21}Garde-toi de provoquer sa colère, et écoute sa voix, et ne l'irrite point, car il ne pardonnera point votre péché~; car mon Nom est en lui.
\VS{22}Mais si tu écoutes attentivement sa voix, et si tu fais tout ce que je te dirai, je serai l'ennemi de tes ennemis, et j'affligerai ceux qui t'affligeront.
\VS{23}Car mon Ange marchera devant toi, et t'introduira au pays des Amoréens, des Héthiens, des Phéréziens, des Cananéens, des Héviens, et des Jébusiens, et je les exterminerai.
\VS{24}Tu ne te prosterneras point devant leurs dieux, et tu ne les serviras point, et tu ne feras point selon leurs œuvres, mais tu les détruiras entièrement, et tu briseras entièrement leurs statues\FTNT{Ex. 20:5~; Ex. 34:13~; No. 33:52.}.
\VS{25}Vous servirez Yahweh, votre Dieu. Et il bénira ton pain et tes eaux~; et j'ôterai les maladies du milieu de toi\FTNT{Ex. 15:26~; De. 6:13~; De. 7:15-16~; Mt. 4:10.}.
\VS{26}Il n'y aura point dans ton pays de femme qui avorte, ou qui soit stérile~; j'accomplirai le nombre de tes jours.
\VS{27}J'enverrai la terreur de mon Nom devant toi, et j'effrayerai tout peuple vers lequel tu arriveras, et je ferai que tous tes ennemis tourneront le dos devant toi\FTNT{De. 7:23.}.
\VS{28}Et j'enverrai des frelons devant toi, qui chasseront les Héviens, les Cananéens, et les Héthiens, de devant ta face\FTNT{De. 7:20~; Jos. 24:12.}.
\VS{29}Je ne les chasserai point loin de devant ta face en une année, de peur que le pays ne devienne un désert, et que les bêtes des champs ne se multiplient contre toi.
\VS{30}Mais je les chasserai peu à peu loin de devant toi, jusqu'à ce que tu te sois accru, et que tu possèdes le pays.
\VS{31}Et je mettrai des bornes depuis la Mer Rouge jusqu'à la mer des Philistins, et depuis le désert jusqu'au fleuve~; car je livrerai entre tes mains les habitants du pays et je les chasserai de devant toi.
\VS{32}Tu ne traiteras point d'alliance avec eux ni avec leurs dieux.
\VS{33}Ils n'habiteront point dans ton pays, de peur qu'ils ne te fassent pécher contre moi~; car tu servirais leurs dieux, et ce serait un piège pour toi.
\Chap{24}
\TextTitle{La loi lue au peuple~; le sang de l'Alliance}
\VerseOne{}Puis il dit à Moïse~: Monte vers Yahweh, toi et Aaron, Nadab et Abihu, et soixante-dix des anciens d'Israël, et vous vous prosternerez de loin.
\VS{2}Et Moïse s'approchera seul de Yahweh, mais eux ne s'en approcheront point, et le peuple ne montera point avec lui.
\VS{3}Alors Moïse vint, et récita au peuple toutes les paroles de Yahweh, et toutes ses lois, et tout le peuple répondit d'une voix, et dit~: Nous ferons toutes les choses que Yahweh a dites.
\VS{4}Or Moïse écrivit toutes les paroles de Yahweh, et s'étant levé de bon matin, il bâtit un autel au bas de la montagne, et dressa pour monument douze pierres pour les douze tribus d'Israël.
\VS{5}Et il envoya des jeunes hommes, des enfants d'Israël, qui offrirent des holocaustes et qui sacrifièrent des veaux à Yahweh en sacrifice d'offrande de paix.
\VS{6}Et Moïse prit la moitié du sang, et le mit dans des bassins, et répandit l'autre moitié sur l'autel.
\VS{7}Ensuite, il prit le livre de l'Alliance et le lut, et le peuple qui l'écoutait dit~: Nous ferons tout ce que Yahweh a dit, et nous obéirons.
\VS{8}Moïse donc prit le sang, et le répandit sur le peuple, en disant~: Voici le sang de l'Alliance que Yahweh a traitée avec vous, selon toutes ces paroles\FTNT{Mt. 26:28~; Mc. 14:24~; Lu. 22:20~; 1 Co. 11:25~; Hé. 9:20.}.
\TextTitle{Yahweh fait monter Moïse sur la montagne}
\VS{9}Puis Moïse, Aaron, Nadab, Abihu, et les soixante-dix anciens d'Israël montèrent.
\VS{10}Et ils virent le Dieu d'Israël, et sous ses pieds comme un ouvrage de saphir transparent, comme le ciel dans toute sa pureté.
\VS{11}Et il ne mit point sa main sur ceux qui avaient été choisis d'entre les enfants d'Israël~; ainsi, ils virent Dieu, et ils mangèrent et burent.
\VS{12}Et Yahweh dit à Moïse~: Monte vers moi sur la montagne, et demeure là~; et je te donnerai des tables de pierre, la loi et les commandements que j'ai écrits pour les enseigner.
\VS{13}Alors Moïse se leva avec Josué qui le servait~; et Moïse monta sur la montagne de Dieu.
\VS{14}Et il dit aux anciens d'Israël~: Demeurez ici en nous attendant jusqu'à ce que nous retournions vers vous. Et voici, Aaron et Hur seront avec vous~; quiconque aura quelque affaire, qu'il s'adresse à eux.
\VS{15}Moïse donc monta sur la montagne, et une nuée couvrit la montagne\FTNT{Ex. 19:9-16.}.
\VS{16}Et la gloire de Yahweh demeura sur la montagne de Sinaï, et la nuée la couvrit pendant six jours. Et au septième jour, il appela Moïse du milieu de la nuée.
\VS{17}Et ce qu'on voyait de la gloire de Yahweh au sommet de la montagne, était comme un feu dévorant aux yeux des enfants d'Israël\FTNT{De. 4:24~; De. 9:3~; Hé. 12:29.}.
\VS{18}Et Moïse entra dans la nuée et monta sur la montagne. Moïse fut sur la montagne quarante jours et quarante nuits.
\Chap{25}
\TextTitle{Des offrandes volontaires pour les matériaux du tabernacle}
\VerseOne{}Et Yahweh parla à Moïse, en disant~:
\VS{2}Parle aux enfants d'Israël, et qu'on prenne une offrande pour moi. Vous prendrez mon offrande de tout homme dont le cœur me l'offrira volontairement.
\VS{3}Et c'est ici l'offrande que vous prendrez d'eux~: De l'or, de l'argent, de l'airain,
\VS{4}de la pourpre, de l'écarlate, du cramoisi\FTNT{La couleur cramoisi s'obtient grâce à la femelle cochenille aptère qui contient dans son corps et dans ses œufs un pigment rouge à base d'acide carminique qui permet à l'insecte et à ses larves de se protéger des prédateurs. Au moment de la ponte, cette dernière fixe fermement son corps au tronc d'un arbre puis libère ses œufs qui demeurent ainsi protégés en dessous d'elle jusqu'à leur éclosion. Ensuite, l'insecte meurt en libérant cette substance rouge qui se propage sur tout son corps et sur le bois hôte. C'est ce fluide que l'homme récupère pour en faire un colorant à la couleur caractéristique. Une subtile analogie peut être faire entre la cochenille et le Seigneur qui a versé son sang à la croix pour nous donner la vie. «~Et moi, je suis un ver, et non un homme, l'opprobre des hommes et le méprisé du peuple~» (Ps. 22~:7).}, du fin lin, du poil de chèvre,
\VS{5}des peaux de béliers teintes en rouge, des peaux de taissons\FTNT{Le mot hébreu employé ici est «~tachash~», il désigne le matériau servant à fabriquer la couverture extérieure de la tente d'assignation. Si tout le monde s'accorde pour dire qu'il s'agissait d'une fourrure ou d'une peau d'animal, un doute subsiste sur la race exacte de l'animal. On hésite entre le marsouin, le dauphin, le blaireau (taisson) ou peut-être le mouton. Dans de nombreuses bibles, on a pris le parti de traduire par «~peaux de dauphins~». Cette hypothèse est cependant très peu probable. D'une part parce que le dauphin n'a pas de fourrure~; d'autre part parce que sa peau n'est absolument pas adaptée à la vie terrestre. Elle est donc impossible à conserver et à transformer, en particulier dans le contexte d'un climat propre au désert. Certains pensent qu'il s'agit tout simplement de peaux de béliers. Dans ce cas, comment expliquer qu'on n'ait pas employé le terme «~'ayil~» comme cela est mentionné pour les peaux teintes en rouge~? Il reste donc les peaux de taissons, c'est-à-dire de blaireaux, dont la fourrure est utilisée depuis des siècles. On peut objecter qu'il est impossible que le Seigneur puisse accepter la peau d'un animal impur pour construire son sanctuaire. Si tel était le cas, la peau du dauphin, qui est également un animal impur, n'aurait pas non-plus été autorisée. Toutefois, d'un point de vue prophétique, le symbole est important~: La présence de cet animal impur préfigurait Christ qui a pris une chair semblable à celle du péché (Ro. 8:3) mais aussi le levain que l'on peut trouver dans la pate nouvelle (1 Co. 5:6-7), l'ivraie qui se glisse parmi le blé (Mt. 13:25-43).}, du bois d'acacia,
\VS{6}de l'huile pour le luminaire, des aromates pour l'huile d'onction et pour le parfum odoriférant,
\VS{7}des pierres d'onyx, et d'autres pierres pour la garniture de l'éphod et pour le pectoral.
\VS{8}Et ils me feront un sanctuaire, et j'habiterai au milieu d'eux\FTNT{Ex. 29:45-46.}.
\VS{9}Ils le feront conformément à tout ce que je vais te montrer, selon le modèle du tabernacle et selon le modèle de tous ses ustensiles~; vous le ferez donc ainsi.
\TextTitle{L'arche de l'alliance}
\VS{10}Et ils feront une arche de bois d'acacia~; et sa longueur sera de deux coudées et demie, et sa largeur d'une coudée et demie, et sa hauteur d'une coudée et demie.
\VS{11}Et tu la couvriras d'or pur, tu l'en couvriras en dehors et en dedans~; et tu feras sur elle un couronnement d'or tout autour\FTNT{Ex. 37:1-9.}.
\VS{12}Et tu fondras pour elle quatre anneaux d'or, que tu mettras à ses quatre coins, deux anneaux à l'un de ses côtés, et deux autres de l'autre côté.
\VS{13}Tu feras aussi des barres de bois d'acacia, et tu les couvriras d'or.
\VS{14}Puis tu feras entrer les barres dans les anneaux aux côtés de l'arche, pour porter l'arche avec elles.
\VS{15}Les barres seront dans les anneaux de l'arche, et on ne les en tirera point.
\VS{16}Et tu mettras dans l'arche le témoignage que je te donnerai\FTNT{Hé. 9:4.}.
\VS{17}Tu feras aussi un propitiatoire d'or pur, dont la longueur sera de deux coudées et demie, et la largeur d'une coudée et demie.
\VS{18}Et tu feras deux chérubins d'or~; tu les feras d'ouvrage étendu au marteau, tirés des deux extrémités du propitiatoire.
\VS{19}Fais donc un chérubin tiré des extrémités et un chérubin tiré de l'autre extrémité~; vous ferez les chérubins tirés du propitiatoire à ses deux extrémités.
\VS{20}Et les chérubins étendront les ailes en haut, couvrant de leurs ailes le propitiatoire, et leurs faces seront vis-à-vis l'une de l'autre~; et le regard des chérubins sera vers le propitiatoire\FTNT{1 R. 8:6-7~; Hé. 9:5.}.
\VS{21}Et tu poseras le propitiatoire au-dessus de l'arche, et tu mettras dans l'arche le témoignage que je te donnerai.
\VS{22}Et je me rencontrerai là avec toi, et je te dirai de dessus le propitiatoire, d'entre les deux chérubins qui seront sur l'arche du témoignage, toutes les choses que je t'ordonnerai pour les enfants d'Israël\FTNT{Ex. 29:42-43~; No. 7:89.}.
\TextTitle{La table des pains de proposition}
\VS{23}Tu feras aussi une table de bois d'acacia. Sa longueur sera de deux coudées, et sa largeur d'une coudée, et sa hauteur d'une coudée et demie.
\VS{24}Tu la couvriras d'or pur, et tu lui feras un couronnement d'or tout autour.
\VS{25}Tu lui feras aussi à l'entour une clôture d'une largeur de main, et tout autour de sa clôture tu feras un couronnement d'or.
\VS{26}Tu lui feras aussi quatre anneaux d'or que tu mettras aux quatre coins qui seront à ses quatre pieds.
\VS{27}Les anneaux seront à l'endroit de la clôture, afin d'y mettre les barres pour porter la table.
\VS{28}Tu feras les barres de bois d'acacia, et tu les couvriras d'or, et on portera la table avec elles.
\VS{29}Tu feras aussi ses plats, ses tasses, ses gobelets, et ses bassins, avec lesquels on fera les aspersions~; tu les feras d'or pur\FTNT{Ex. 37:10-16.}.
\VS{30}Et tu mettras sur cette table le pain de proposition continuellement devant moi\FTNT{Lé. 24:5-9.}.
\TextTitle{Le chandelier d'or pur}
\VS{31}Tu feras aussi un chandelier d'or pur\FTNT{Le chandelier avait une double symbolique. D'une part, il préfigurait Jésus-Christ, notre Lumière (Jn. 1:4-5~; Jn. 8:12). Les sept lampes évoquaient l'omniscience de l'Esprit de Jésus-Christ (Za. 3:9~; Jn. 16:29-30~; Ap. 1:4~; Ap. 3:1~; Ap. 4:5~; Ap. 5:6). Il est à noter que ce chandelier comportait des calices en forme de fleurs, de pommes et d'amandes (Ex. 25:33) qui symbolisaient les fruits de l'Esprit que nous devons nécessairement porter (Ga. 5:22). D'autre part, il est une image de l'Eglise (Ap. 1:20). Voir commentaire Ex. 37:17-24~; Es. 8:13-17.}. Le chandelier sera étendu au marteau~; son pied, sa tige et ses branches, ses plats, ses pommeaux et ses fleurs seront tirés de lui.
\VS{32}Six branches sortiront de ses côtés~: Trois branches d'un côté du chandelier, et trois autres de l'autre côté du chandelier.
\VS{33}Il y aura sur l'une des branches trois petits plats en forme d'amande, un pommeau et une fleur~; sur l'autre branche trois petits plats en forme d'amande, un pommeau et une fleur~; il en sera de même des six branches sortant du chandelier.
\VS{34}ll y aura aussi au chandelier quatre petits plats en forme d'amande, ses pommeaux et ses fleurs.
\VS{35}Un pommeau sous deux branches tirées du chandelier, un pommeau sous deux autres branches tirées de lui, et un pommeau sous deux autres branches tirées de lui~; il en sera de même des six branches sortant du chandelier.
\VS{36}Leurs pommeaux et leurs branches seront tirés de lui, et tout le chandelier sera un seul ouvrage étendu au marteau, et d'or pur.
\VS{37}Tu feras aussi ses sept lampes, et on les allumera afin qu'elles éclairent vis-à-vis du chandelier.
\VS{38}Et ses mouchettes et ses encensoirs destinés à recevoir ce qui tombe des lampes seront d'or pur.
\VS{39}On le fera avec tous ses ustensiles d'un talent d'or pur.
\VS{40}Regarde donc, et fais selon le modèle qui t'est montré sur la montagne.
\Chap{26}
\TextTitle{Les tapis de fin lin}
\VerseOne{}Tu feras aussi le tabernacle de dix tapis de fin lin retors, de pourpre, d'écarlate, et de cramoisi~; et tu les feras semés de chérubins d'un ouvrage exquis\FTNT{Ex. 36:8-38.}.
\VS{2}La longueur d'un tapis sera de vingt-huit coudées, et la largeur du même tapis de quatre coudées~; tous les tapis auront une même mesure.
\VS{3}Cinq de ces tapis seront joints l'un à l'autre, et les cinq autres seront aussi joints l'un à l'autre.
\VS{4}Fais aussi des lacets de pourpre sur le bord d'un tapis, au bord du premier assemblage~; et tu feras la même chose au bord du dernier tapis dans l'autre assemblage.
\VS{5}Tu feras donc cinquante lacets au premier tapis, et tu feras cinquante lacets au bord du tapis qui est dans le second assemblage. Les lacets seront vis-à-vis l'un de l'autre.
\VS{6}Tu feras aussi cinquante crochets d'or, et tu attacheras les tapis l'un à l'autre avec les crochets~; ainsi le tabernacle ne fera qu'un.
\TextTitle{Les tapis de poils de chèvre}
\VS{7}Tu feras aussi des tapis de poils de chèvres pour servir de tente sur le tabernacle~; tu feras onze de ces tapis.
\VS{8}La longueur d'un tapis sera de trente coudées, et la largeur du même tapis sera de quatre coudées~; les onze tapis auront une même mesure.
\VS{9}Puis tu joindras séparément cinq de ces tapis, et les six tapis à part~; mais tu redoubleras le sixième tapis sur le devant du tabernacle.
\VS{10}Tu feras aussi cinquante lacets sur le bord de l'un des tapis, à savoir au dernier qui est assemblé, et cinquante lacets au bord du tapis du second assemblage.
\VS{11}Tu feras aussi cinquante crochets d'airain, et tu feras entrer les crochets dans les lacets~; et tu assembleras ainsi la tente qui fera un tout.
\VS{12}Mais ce qu'il y aura en surplus dans les tapis de la tente, à savoir la moitié du tapis de reste, retombera sur le derrière du tabernacle.
\VS{13}La coudée d'une part, et la coudée d'autre part, qui seront de reste sur la longueur des tapis de la tente, retomberont sur les deux côtés du tabernacle, pour le couvrir.
\TextTitle{Les couvertures de peaux de béliers}
\VS{14}Tu feras aussi pour ce tabernacle une couverture de peaux de béliers teintes en rouge, et une couverture de peaux de taissons par-dessus\FTNT{Ex. 35:7~; Ex. 35:23~; Ex. 36:19~; Ex. 39:34.}.
\TextTitle{Les planches et leurs bases}
\VS{15}Et tu feras pour le tabernacle, des planches de bois d'acacia, qu'on fera tenir debout\FTNT{Ex. 36:20-34.}.
\VS{16}La longueur d'une planche sera de dix coudées, et la largeur d'une même planche d'une coudée et demie.
\VS{17}Il y aura à chaque planche deux tenons joints l'un à l'autre~; et tu feras de même pour toutes les planches du tabernacle.
\VS{18}Tu feras donc les planches du tabernacle, à savoir vingt planches qui regardent vers le midi.
\VS{19}Et au-dessous des vingt planches, tu feras quarante bases d'argent~; deux bases sous une planche pour ses deux tenons, et deux bases sous l'autre planche pour ses deux tenons.
\VS{20}Et vingt planches de l'autre côté du tabernacle, du coté nord.
\VS{21}Et leurs quarante bases seront d'argent, deux bases sous une planche, et deux bases sous l'autre planche.
\VS{22}Et pour le fond du tabernacle, vers l'occident, tu feras six planches.
\VS{23}Tu feras aussi deux planches pour les angles du tabernacle, aux deux cotés du fond.
\VS{24}Et elles seront égales par le bas, et elles seront jointes et unies par le haut avec un anneau~; il en sera de même des deux planches qui seront aux deux angles.
\VS{25}Il y aura donc huit planches, et seize bases d'argent~; deux bases sous une planche et deux bases sous une autre planche.
\VS{26}Après cela, tu feras cinq barres de bois d'acacia, pour les planches d'un des côtés du tabernacle.
\VS{27}Pareillement, tu feras cinq barres pour les planches de l'autre côté du tabernacle~; et cinq barres pour les planches du côté du tabernacle, pour le fond, vers le côté de l'occident.
\VS{28}Et la barre du milieu sera au milieu des planches d'une extrémité à l'autre.
\VS{29}Tu couvriras aussi d'or les planches, et tu feras d'or leurs anneaux pour mettre les barres, et tu couvriras d'or les barres.
\VS{30}Tu dresseras le tabernacle selon le modèle qui t'est montré sur la montagne.
\TextTitle{Les voiles intérieurs et extérieurs}
\VS{31}Et tu feras un voile\FTNT{Le voile intérieur symbolisait la chair de Jésus-Christ qui a été brisée à cause de nos péchés (Es. 53:5~; Hé. 10:20). Ex. 36:35-38~; Mt. 27:51~; Hé. 9:3.} de pourpre, d'écarlate, de cramoisi, et de fin lin retors~; on le fera d'ouvrage exquis, avec des chérubins.
\VS{32}Et tu le mettras sur quatre piliers de bois d'acacia couverts d'or, ayant leurs crochets d'or. Et ils seront sur quatre bases d'argent.
\VS{33}Puis tu mettras le voile sous les crochets, et tu feras entrer là-dedans, c'est-à-dire au-dedans du voile, l'arche du témoignage~; et ce voile vous fera la séparation entre le lieu saint et le Saint des saints.
\VS{34}Et tu poseras le propitiatoire sur l'arche du témoignage, dans le Saint des saints.
\VS{35}Et tu mettras la table au dehors de ce voile, et le chandelier vis-à-vis de la table, au côté du tabernacle, vers le sud~; et tu placeras la table côté nord.
\VS{36}Et à l'entrée du tabernacle, tu feras un rideau de pourpre, d'écarlate, de cramoisi et de fin lin retors, d'ouvrage de broderie.
\VS{37}Tu feras aussi pour ce rideau cinq piliers de bois d'acacia, que tu couvriras d'or, et leurs crochets seront d'or~; et tu fondras pour eux cinq bases d'airain.
\Chap{27}
\TextTitle{L'autel d'airain}
\VerseOne{}Tu feras aussi un autel de bois d'acacia, ayant cinq coudées de long, et cinq coudées de large~; l'autel sera carré, et sa hauteur sera de trois coudées.
\VS{2}Tu feras ses cornes à ses quatre coins~; ses cornes seront tirées de lui, et tu le couvriras d'airain\FTNT{C'est sur l'autel d'airain que les animaux étaient sacrifiés. Il préfigurait la croix et le jugement que Jésus-Christ a pris sur lui à notre place (Es. 53:5~; 2 Co. 13:4~; Ph. 2:8).}.
\VS{3}Tu feras ses chaudrons pour recevoir ses cendres, et ses racloirs, ses bassins, ses fourchettes, et ses encensoirs~; tu feras tous ses ustensiles d'airain.
\VS{4}Tu lui feras une grille d'airain en forme de treillis, et tu feras au treillis quatre anneaux d'airain à ses quatre coins.
\VS{5}Et tu le mettras au-dessous de l'enceinte de l'autel en bas, et le treillis s'étendra jusqu'au milieu de l'autel.
\VS{6}Tu feras aussi des barres pour l'autel, des barres de bois d'acacia, et tu les couvriras d'airain.
\VS{7}Et on fera passer ses barres dans les anneaux~; les barres seront aux deux côtés de l'autel pour le porter.
\VS{8}Tu le feras creux avec des planches~; ils le feront ainsi qu'il t'a été montré sur la montagne\FTNT{Ex. 38:1-7.}.
\TextTitle{Le parvis}
\VS{9}Tu feras aussi le parvis du tabernacle, du côté qui regarde vers le sud~; il y aura pour former le parvis, des courtines de fin lin retors~; la longueur de l'un des côtés sera de cent coudées.
\VS{10}Il y aura vingt piliers avec leurs vingt bases d'airain, mais les crochets des piliers et leurs filets seront d'argent.
\VS{11}Ainsi du côté nord, il y aura également des courtines sur une longueur de cent coudées, avec vingt piliers avec leurs vingt bases d'airain~; mais les crochets des piliers avec leurs filets seront d'argent.
\VS{12}La largeur du parvis du côté de l'occident sera de cinquante coudées de courtines, qui auront dix piliers, avec leurs dix bases.
\VS{13}Et la largeur du parvis du côté de l'orient, directement vers le levant, sera de cinquante coudées.
\VS{14}A l'un des côtés, il y aura quinze coudées de courtines, avec leurs trois piliers et leurs trois bases.
\VS{15}Et de l'autre côté, quinze coudées de courtines, avec leurs trois piliers et leurs trois bases.
\TextTitle{La porte du parvis}
\VS{16}Il y aura aussi pour la porte du parvis un rideau de vingt coudées, fait de pourpre, d'écarlate, de cramoisi, et de fin lin retors, ouvrage de broderie, avec quatre piliers et quatre bases.
\VS{17}Tous les piliers du parvis seront ceints d'un filet d'argent, et leurs crochets seront d'argent, mais leurs bases seront d'airain.
\VS{18}La longueur du parvis sera de cent coudées, et la largeur de cinquante, de chaque côté~; et la hauteur de cinq coudées. Il sera de fin lin retors, et les bases des piliers seront d'airain.
\VS{19}Que tous les ustensiles du tabernacle, pour tout son service, et tous ses pieux, avec les pieux du parvis, soient d'airain\FTNT{Ex. 38:9-20.}.
\TextTitle{L'huile d'olive vierge pour les lampes}
\VS{20}Tu ordonneras aux fils d'Israël qu'ils t'apportent de l'huile d'olive vierge pour le luminaire, afin de faire luire les lampes continuellement\FTNT{Ex. 35:8-28~; Lé. 24:1-4.}.
\VS{21}Aaron avec ses fils les prépareront dans la présence de Yahweh, depuis le soir jusqu'au matin, dans la tente d'assignation, hors du voile qui est devant le témoignage~; ce sera une ordonnance perpétuelle pour les enfants d'Israël.
\Chap{28}
\TextTitle{La prêtrise}
\VerseOne{}Et toi, fais approcher de toi Aaron, ton frère, et ses fils avec lui, d'entre les enfants d'Israël, pour m'exercer la prêtrise, à savoir Aaron, Nadab et Abihu, Eléazar et Ithamar, fils d'Aaron.
\VS{2}Et tu feras à Aaron, ton frère, de saints vêtements pour gloire et pour ornement.
\TextTitle{Les vêtements sacrés des prêtres}
\VS{3}Et tu parleras à tous les hommes d'esprit, à chacun de ceux que j'ai remplis de l'esprit de science, afin qu'ils fassent des vêtements à Aaron pour le sanctifier, afin qu'il m'exerce la prêtrise.
\VS{4}Et ce sont ici les vêtements qu'ils feront~: Le pectoral, l'éphod, la robe, la tunique brodée, la tiare, et la ceinture. Ils feront donc les saints vêtements à Aaron, ton frère, et à ses fils, pour m'exercer la prêtrise.
\VS{5}Et ils prendront de l'or, de la pourpre, de l'écarlate, du cramoisi, et du fin lin.
\TextTitle{L'éphod}
\VS{6}Et ils feront l'éphod d'or, de pourpre, d'écarlate et de cramoisi, et de fin lin retors~; d'un ouvrage exquis.
\VS{7}Il aura deux épaulettes qui se joindront par les deux bouts~; et c'est ainsi qu'il sera joint.
\VS{8}La ceinture exquise dont il sera ceint, et qui sera par-dessus, sera de même ouvrage, et tirée de lui, étant d'or, de pourpre, d'écarlate, de cramoisi, et de fin lin retors.
\VS{9}Et tu prendras deux pierres d'onyx, et tu graveras sur elles les noms des enfants d'Israël~:
\VS{10}Six de leurs noms sur une pierre et les six noms des autres sur l'autre pierre, selon leur naissance.
\VS{11}Tu graveras sur les deux pierres les noms des enfants d'Israël, comme on grave les pierres et les cachets, tu les entoureras de montures d'or.
\VS{12}Et tu mettras les deux pierres sur les épaulettes de l'éphod, afin qu'elles soient des pierres de souvenir pour les enfants d'Israël~; car Aaron portera leurs noms sur ses deux épaules devant Yahweh, pour souvenir.
\VS{13}Tu feras aussi des montures d'or,
\VS{14}et deux chaînettes d'or pur que tu tresseras en forme de cordons, et tu fixeras aux montures les chaînettes ainsi tressées.
\TextTitle{Le pectoral}
\VS{15}Tu feras aussi le pectoral du jugement d'un ouvrage exquis, comme l'ouvrage de l'éphod, d'or, de pourpre, d'écarlate, de cramoisi, et de fin lin retors.
\VS{16}Il sera carré et double~; et sa longueur sera d'un empan, et sa largeur d'un empan.
\VS{17}Et tu le rempliras de garniture de pierres, à quatre rangées de pierres précieuses. A la première rangée, on mettra une sardoine, une topaze, et une émeraude.
\VS{18}Et à la seconde rangée, une escarboucle, un saphir, et un jaspe.
\VS{19}Et à la troisième rangée, une opale, une agate, et une améthyste.
\VS{20}Et à la quatrième rangée, un chrysolithe, un onyx et un béryl, qui seront enchâssés dans de l'or, selon leur garniture.
\VS{21}Et ces pierres-là seront selon les noms des enfants d'Israël, douze selon leurs noms, chacune d'elles gravées comme des cachets, selon le nom qu'elle doit porter, et elles seront pour les douze tribus.
\VS{22}Tu feras donc pour le pectoral des chaînettes d'or pur, tressées en forme de cordon.
\VS{23}Et tu feras sur le pectoral deux anneaux d'or, et tu mettras les deux anneaux aux deux bouts du pectoral.
\VS{24}Et tu mettras les deux chaînettes d'or, faites en cordon, dans les deux anneaux à l'extrémité du pectoral.
\VS{25}Et tu mettras les deux autres bouts des deux chaînettes en cordon sur les deux montures, et tu les mettras sur les épaulettes de l'éphod, sur le devant de l'éphod.
\VS{26}Tu feras aussi deux autres anneaux d'or, que tu mettras aux deux autres bouts du pectoral, sur le bord qui sera du côté de l'éphod à l'intérieur.
\VS{27}Et tu feras deux autres anneaux d'or, que tu mettras aux deux épaulettes de l'éphod par le bas, sur le devant, à l'endroit où il se joint, au-dessus de la ceinture exquise de l'éphod.
\VS{28}Et ils joindront le pectoral élevé par ses anneaux, aux anneaux de l'éphod, avec un cordon de pourpre, afin qu'il tienne au-dessus de la ceinture exquise de l'éphod, et que le pectoral ne puisse pas se séparer de l'éphod.
\VS{29}Ainsi, Aaron portera sur son cœur les noms des enfants d'Israël gravés sur le pectoral du jugement, quand il entrera dans le lieu saint, pour souvenir devant Yahweh continuellement.
\TextTitle{L'urim et le thummim}
\VS{30}Et tu mettras sur le pectoral de jugement l'urim et le thummim\FTNT{L'urim («~lumières~») et le thummim («~perfections~») étaient deux pierres du pectoral que l'on utilisait ensemble pour déterminer la décision de Dieu sur certaines questions.}, qui seront sur le cœur d'Aaron, quand il viendra devant Yahweh~; et Aaron portera le jugement des enfants d'Israël sur son cœur devant Yahweh continuellement.
\TextTitle{La robe de l'éphod}
\VS{31}Tu feras aussi la robe de l'éphod entièrement de pourpre.
\VS{32}Il y aura, au milieu, une ouverture pour la tête, et cette ouverture aura tout autour un bord tissé, comme l'ouverture d'une cotte de mailles, afin que la robe ne se déchire pas.
\VS{33}Tu feras à ses bords des grenades de pourpre, d'écarlate, et de cramoisi tout autour, et des clochettes d'or entre elles tout autour.
\VS{34}Une clochette d'or, puis une grenade, une clochette d'or, puis une grenade, aux bords de la robe tout autour.
\VS{35}Et Aaron en sera revêtu quand il fera le service, et on en entendra le son lorsqu'il entrera dans le lieu saint devant Yahweh, et quand il en sortira, afin qu'il ne meure pas.
\TextTitle{La lame d'or gravée~: La sainteté à Yahweh}
\VS{36}Et tu feras une lame d'or pur, sur laquelle tu graveras ces mots, comme on grave un cachet~: La sainteté à Yahweh.
\VS{37}Tu l'attacheras avec un cordon de pourpre sur la tiare, sur le devant de la tiare.
\VS{38}Et elle sera sur le front d'Aaron~; et Aaron portera l'iniquité commise par les enfants d'Israël, en faisant leurs saintes offrandes, elle sera continuellement sur son front devant Yahweh, pour qu'il leur soit favorable.
\TextTitle{Les vêtements de service d'Aaron et ses fils}
\VS{39}Tu feras aussi une tunique de fin lin qui s'appliquera sur le corps, et tu feras aussi la tiare de fin lin~; mais tu feras la ceinture d'ouvrage de broderie\FTNT{Ex. 39:1-32.}.
\VS{40}Tu feras aussi aux fils d'Aaron des tuniques, des ceintures, et des bonnets, pour leur gloire et leur ornement.
\VS{41}Et tu en revêtiras Aaron, ton frère, et ses fils avec lui~; tu les oindras, tu les consacreras et tu les sanctifieras~; puis ils exerceront la prêtrise pour moi\FTNT{Lé. 8:12~; Lé. 16:32~; No. 3:3.}.
\VS{42}Et tu leur feras des caleçons de lin, pour couvrir leur nudité, qui tiendront depuis les reins jusqu'au bas des cuisses.
\VS{43}Et Aaron et ses fils seront ainsi habillés quand ils entreront dans la tente d'assignation, ou quand ils approcheront de l'autel pour faire le service dans le lieu saint~; et ils ne porteront point la peine d'aucune iniquité, et ne mourront point. Ce sera une ordonnance perpétuelle pour lui et pour sa postérité après lui.
\Chap{29}
\TextTitle{Les prêtres consacrés au service de Yahweh}
\VerseOne{}Or c'est ici ce que tu leur feras, quand tu les sanctifieras pour exercer la prêtrise pour moi~: Prends un veau du troupeau, et deux béliers sans tare\FTNT{Lé. 8:2~; Lé. 9:2~; Hé. 7:26-28.}~;
\VS{2}et des pains sans levain, et des gâteaux sans levain pétris à l'huile, et des beignets sans levain, oints d'huile~; et tu les feras de fine farine de froment\FTNT{Lé. 6:13.}.
\VS{3}Tu les mettras dans une corbeille, et tu les présenteras dans la corbeille~; tu présenteras aussi le veau et les deux moutons.
\VS{4}Puis, tu feras approcher Aaron et ses fils à l'entrée de la tente d'assignation, et tu les laveras avec de l'eau\FTNT{Ex. 40:12.}.
\VS{5}Ensuite, tu prendras les vêtements, et tu feras vêtir à Aaron la tunique et la robe de l'éphod, l'éphod et pectoral, et tu le ceindras par-dessus avec la ceinture exquise de l'éphod.
\VS{6}Puis, tu mettras sur sa tête la tiare et la couronne de sainteté sur la tiare.
\VS{7}Et tu prendras l'huile d'onction et la répandras sur sa tête~; et tu l'oindras ainsi.
\VS{8}Puis, tu feras approcher ses fils, et tu leur feras vêtir les tuniques.
\VS{9}Et tu les ceindras des ceintures, Aaron, dis-je, et ses fils\FTNT{Es. 11:5~; Ep. 6:14.}, et tu leur attacheras des bonnets, et ils posséderont la prêtrise par ordonnance perpétuelle. Et tu consacreras ainsi Aaron et ses fils.
\VS{10}Et tu feras approcher le veau devant la tente d'assignation, et Aaron et ses fils poseront leurs mains sur la tête du veau.
\VS{11}Et tu égorgeras le veau devant Yahweh, à l'entrée de la tente d'assignation.
\VS{12}Puis, tu prendras du sang du veau, et le mettras avec ton doigt sur les cornes de l'autel, et tu répandras tout le reste du sang au pied de l'autel.
\VS{13}Tu prendras aussi toute la graisse qui couvre les entrailles, et le lobe du foie, les deux rognons, la graisse qui les entoure, et tu les feras fumer sur l'autel.
\VS{14}Mais tu brûleras au feu la chair du veau, sa peau, et ses excréments, hors du camp. C'est un sacrifice pour le péché\FTNT{Lé. 1:3-13~; Hé. 9:11~; Hé. 13:11.}.
\VS{15}Puis, tu prendras l'un des béliers, et Aaron et ses fils poseront leurs mains sur la tête du bélier.
\VS{16}Puis, tu égorgeras le bélier, et prenant son sang, tu le répandras sur l'autel tout autour.
\VS{17}Après, tu couperas le bélier par pièces, et ayant lavé ses entrailles et ses jambes, tu les mettras sur ses pièces et sur sa tête.
\VS{18} Et tu feras fumer tout le bélier sur l'autel~; c'est un holocauste à Yahweh, c'est un sacrifice consumé par le feu d'une agréable odeur à Yahweh.
\VS{19}Puis, tu prendras l'autre bélier, et Aaron et ses fils mettront leurs mains sur sa tête.
\VS{20}Et tu égorgeras le bélier, et prenant de son sang, tu le mettras sur le lobe de l'oreille droite d'Aaron, et sur le lobe de l'oreille droite de ses fils, sur le pouce de leur main droite, sur le gros orteil de leur pied droit, et tu répandras le reste du sang sur l'autel tout autour.
\VS{21}Et tu prendras du sang qui sera sur l'autel, de l'huile d'onction, et tu en feras l'aspersion sur Aaron, sur ses vêtements, sur ses fils, et sur les vêtements de ses fils avec lui. Ainsi, lui, ses vêtements, ses fils, et les vêtements de ses fils, seront sanctifiés avec lui.
\VS{22}Tu prendras aussi la graisse du bélier, la queue, et la graisse qui couvre les entrailles, le grand lobe du foie, les deux rognons, la graisse qui est dessus, et l'épaule droite~; car c'est le bélier de consécration.
\VS{23}Tu prendras aussi un pain, un gâteau à l'huile, et un beignet dans la corbeille où seront ces choses sans levain, laquelle sera devant Yahweh.
\VS{24}Et tu mettras toutes ces choses sur les mains d'Aaron et sur les mains de ses fils, et tu les agiteras de côté et d'autre devant Yahweh\FTNT{No. 6:19.}.
\VS{25}Puis, les recevant de leurs mains, tu les feras fumer sur l'autel, sur l'holocauste, pour être une odeur agréable devant Yahweh~; c'est un sacrifice consumé par le feu à Yahweh.
\TextTitle{La part des prêtres}
\VS{26}Tu prendras aussi la poitrine du bélier des consécrations, qui est pour Aaron, et tu l'agiteras de côté et d'autre en offrande agitée devant Yahweh. Ce sera ta part.
\VS{27}Tu sanctifieras donc la poitrine de l'offrande agitée, et l'épaule de l'offrande élevée, tant ce qui aura été agité que ce qui aura été élevé du bélier de consécration, de ce qui est pour Aaron et de ce qui est pour ses fils. \FTNT{Lé. 10:14~; No. 18:18.}.
\VS{28}Et ceci sera une ordonnance perpétuelle pour Aaron et pour ses fils, de ce qui sera offert par les enfants d'Israël~; car c'est une offrande élevée. Quand il y aura une offrande élevée de celles qui sont faites par les enfants d'Israël, de leurs offrandes de paix, leur offrande élevée sera à Yahweh.
\VS{29}Et les saints vêtements qui seront pour Aaron, seront pour ses fils après lui, afin qu'ils soient oints et consacrés dans ces vêtements.
\VS{30}Le prêtre qui succédera à sa place d'entre ses fils, et qui viendra à la tente d'assignation, pour faire le service dans le lieu saint, en sera revêtu durant sept jours.
\VS{31}Or tu prendras le bélier des consécrations, et tu feras bouillir sa chair dans un lieu saint~;
\VS{32}et Aaron et ses fils mangeront à l'entrée de la tente d'assignation la chair du bélier et le pain qui sera dans la corbeille.
\VS{33}Ils mangeront donc ces choses, par lesquelles la propitiation aura été faite, pour les consacrer et les sanctifier~; mais l'étranger n'en mangera point, parce qu'elles sont saintes.
\VS{34}S'il y a des restes de la chair des consécrations et du pain jusqu'au matin, tu brûleras ces restes-là au feu~; on n'en mangera point, parce que c'est une chose sainte.
\VS{35}Tu feras donc ainsi à Aaron et à ses fils, selon toutes les choses que je t'ai ordonnées~; tu les consacreras durant sept jours\FTNT{Lé. 8:31-35.}.
\VS{36}Et tu offriras comme sacrifice pour l'expiation tous les jours un veau pour faire l'expiation, et tu purifieras l'autel par cette propitiation, et tu l'oindras pour le sanctifier\FTNT{Ez. 43:19-20.}.
\VS{37}Pendant sept jours, tu feras propitiation pour l'autel, et tu le sanctifieras~; l'autel sera une chose très sainte~; tout ce qui touchera l'autel sera saint\FTNT{No. 28:3.}.
\TextTitle{L'holocauste perpétuel}
\VS{38}Or c'est ici ce que tu feras sur l'autel~: Tu offriras chaque jour continuellement deux agneaux d'un an.
\VS{39}Tu sacrifieras l'un des agneaux au matin, et l'autre agneau entre les deux soirs~;
\VS{40}avec un dixième de fine farine pétrie dans la quatrième partie d'un hin d'huile vierge, et avec une libation de vin de la quatrième partie d'un hin pour chaque agneau,
\VS{41}et tu sacrifieras l'autre agneau entre les deux soirs, avec un gâteau comme au matin, et tu lui feras la même libation, en bonne odeur~; c'est un sacrifice consumé par le feu à Yahweh.
\VS{42}Ce sera l'holocauste perpétuel qui sera offert en vos générations, à l'entrée de la tente d'assignation, devant Yahweh, où je me trouverai avec vous pour te parler.
\VS{43}Je me trouverai là pour les enfants d'Israël, et la tente sera sanctifiée par ma gloire.
\VS{44}Je sanctifierai donc la tente d'assignation et l'autel. Je sanctifierai aussi Aaron et ses fils, afin qu'ils exercent la prêtrise pour moi.
\VS{45} Et j'habiterai au milieu des enfants d'Israël, et je serai leur Dieu.
\VS{46}Et ils sauront que je suis Yahweh, leur Dieu, qui les ai tirés du pays d'Egypte, pour habiter au milieu d'eux. Je suis Yahweh leur Dieu.
\Chap{30}
\TextTitle{L'autel des parfums}
\VerseOne{}Tu feras aussi un autel pour les parfums, et tu le feras de bois d'acacia.
\VS{2}Sa longueur sera d'une coudée, et sa largeur d'une coudée~; il sera carré~; mais sa hauteur sera de deux coudées, et ses cornes seront tirées de lui.
\VS{3}Tu le couvriras d'or pur, tant le dessus, que ses côtés tout autour, et ses cornes. Et tu lui feras un couronnement d'or tout autour.
\VS{4}Tu lui feras aussi deux anneaux d'or au-dessous de son couronnement, à ses deux côtés, lesquels tu mettras aux deux coins, pour y faire passer les barres qui serviront à le porter.
\VS{5}Tu feras les barres de bois d'acacia, et tu les couvriras d'or.
\VS{6}Et tu les mettras devant le voile, qui est au-devant de l'arche du témoignage, à l'endroit du propitiatoire qui est sur le témoignage, où je me trouverai avec toi.
\VS{7}Et Aaron fera sur cet autel un parfum de choses aromatiques~; il y fera un parfum chaque matin, quand il préparera les lampes.
\VS{8}Et quand Aaron allumera les lampes entre les deux soirs, il y fera aussi le parfum, à savoir le parfum perpétuel devant Yahweh dans vos générations\FTNT{Ex. 37:25-29~; 2 Ch. 13:11.}.
\VS{9}Vous n'offrirez point sur cet autel aucun parfum étranger, ni d'holocauste, ni d'offrande, et vous n'y répandrez aucune libation.
\VS{10}Mais Aaron fera une fois l'an la propitiation sur les cornes de cet autel~; il fera, dis-je, la propitiation une fois l'an sur cet autel dans vos générations, avec le sang de l'offrande pour l'expiation faite pour les propitiations. C'est une chose très sainte à Yahweh.
\TextTitle{L'offrande du rachat\FTNTT{Ex. 15:1-21~; Ps. 107:1-2.}}
\VS{11}Yahweh parla aussi à Moïse, et lui dit~:
\VS{12}Quand tu feras le dénombrement des fils d'Israël, selon leur nombre, ils donneront chacun à Yahweh le rachat de sa personne, quand tu en feras le dénombrement, et il n'y aura point de plaie sur eux quand tu en feras le dénombrement\FTNT{No. 1:2.}.
\VS{13}Tous ceux qui passeront par le dénombrement donneront un demi-sicle, selon le sicle du sanctuaire, qui est de vingt guéras~; le demi-sicle donc sera l'offrande que l'on donnera à Yahweh\FTNT{Lé. 27:25~; No. 3:47~; Ez. 45:12.}.
\VS{14}Tous ceux qui passeront par le dénombrement, depuis l'âge de vingt ans et au-dessus, donneront cette offrande à Yahweh.
\VS{15}Le riche n'augmentera rien, et le pauvre ne diminuera rien du demi-sicle, quand ils donneront à Yahweh l'offrande pour faire le rachat de vos personnes.
\VS{16}Tu prendras donc des enfants d'Israël l'argent des expiations, et tu l'appliqueras à l'œuvre de la tente d'assignation. Ce sera pour les fils d'Israël, un souvenir devant Yahweh pour faire le rachat de vos personnes.
\TextTitle{Purification par l'eau de la cuve d'airain\FTNTT{Jn. 13:3-10~; Hé. 10:22~; 1 Jn. 1:9.}}
\VS{17}Yahweh parla encore à Moïse, en disant~:
\VS{18}Fais aussi une cuve d'airain, avec sa base d'airain, pour laver. Et tu la mettras entre la tente d'assignation et l'autel, et tu mettras de l'eau dedans~;
\VS{19}Aaron et ses fils y laveront leurs mains et leurs pieds.
\VS{20}Quand ils entreront dans la tente d'assignation ils se laveront avec de l'eau, afin qu'ils ne meurent point, et quand ils approcheront de l'autel pour faire le service, afin de faire fumer l'offrande consumée par le feu à Yahweh.
\VS{21}Ils laveront donc leurs pieds et leurs mains, afin qu'ils ne meurent point~; ce leur sera une ordonnance perpétuelle, tant pour Aaron que pour ses fils, en leur génération.
\TextTitle{L'huile pour l'onction sainte\FTNTT{Jn. 4:23~; Ep. 2:18, 5:18-19.}}
\VS{22}Yahweh parla aussi à Moïse, en disant~:
\VS{23}Prends des choses aromatiques les plus exquises~; de la myrrhe franche le poids de cinq cent sicles, la moitié de cinnamome odoriférant, c'est-à-dire, le poids de deux cent cinquante sicles, et du roseau aromatique deux cent cinquante sicles.
\VS{24}De la casse le poids de cinq cent sicles, selon le sicle du sanctuaire, et un hin d'huile d'olive.
\VS{25}Et tu en feras de l'huile pour l'onction sainte, un onguent composé selon l'art du parfumeur, ce sera l'huile de l'onction sainte.
\VS{26}Puis tu en oindras la tente d'assignation, et l'arche du témoignage.
\VS{27}La table et tous ses ustensiles, le chandelier et ses ustensiles, et l'autel du parfum,
\VS{28} et l'autel des holocaustes et tous ses ustensiles, la cuve et sa base.
\VS{29}Ainsi, tu les sanctifieras, et ils seront une chose très-sainte~; tout ce qui les touchera sera saint.
\VS{30}Tu oindras aussi Aaron et ses fils, et les sanctifieras pour m'exercer la prêtrise.
\VS{31}Tu parleras aussi aux enfants d'Israël, en disant~: Ce me sera une huile d'onction sainte dans toutes vos générations.
\VS{32}On n'en oindra point la chair d'aucun homme, et vous n'en ferez point d'autre de même composition~; elle est sainte, elle vous sera sainte.
\VS{33}Quiconque composera un onguent semblable, et qui en mettra sur un autre, sera retranché de ses peuples.
\TextTitle{L'encens pur parfumé}
\VS{34}Yahweh dit aussi à Moïse~: Prends des aromates, à savoir de la gomme, de l'ongle odorant, du galbanum, le tout préparé, et de l'encens pur, le tout à poids égal.
\VS{35}Et tu en feras un parfum aromatique selon l'art du parfumeur, et tu y mettras du sel~; vous le ferez pur, et ce sera pour vous une chose sainte.
\VS{36}Et quand tu l'auras pilé bien menu, tu en mettras dans la tente d'assignation devant le témoignage, où je me trouverai avec toi. Ce sera pour vous une chose très sainte.
\VS{37}Quant au parfum que tu feras, vous ne ferez point pour vous de semblable composition~; ce sera une chose sainte pour Yahweh.
\VS{38}Quiconque en fera un semblable pour le sentir sera retranché de ses peuples.
\Chap{31}
\TextTitle{Yahweh suscite des artisans}
\VerseOne{}Yahweh parla aussi à Moïse, en disant~:
\VS{2}Regarde, j'ai appelé par son nom Betsaleel, fils d'Uri, fils de Hur, de la tribu de Juda.
\VS{3}Et je l'ai rempli de l'Esprit de Dieu, de sagesse, d'intelligence et de science pour toutes sortes d'ouvrages,
\VS{4}afin d'inventer des dessins pour travailler  l'or, l'argent et l'airain~;
\VS{5}dans la sculpture des pierres précieuses, pour les mettre en œuvre, et dans la menuiserie pour travailler dans toutes sortes d'ouvrages.
\VS{6}Et voici, je lui ai donné pour compagnon Oholiab, fils d'Ahisamac, de la tribu de Dan~; et au cœur de tout homme sage, j'ai mis de l'intelligence, afin qu'ils fassent toutes les choses que je t'ai ordonnées,
\VS{7}à savoir la tente d'assignation, l'arche du témoignage, et le propitiatoire qui doit être au-dessus, et tous les ustensiles du tabernacle~;
\VS{8}et la table avec tous ses ustensiles~; et le chandelier pur avec tous ses ustensiles~; et l'autel du parfum~;
\VS{9}et l'autel de l'holocauste avec tous ses ustensiles, la cuve et sa base~;
\VS{10}et les vêtements du service~; les saints vêtements pour le prêtre Aaron, et les vêtements de ses fils pour exercer la prêtrise~;
\VS{11}et l'huile d'onction, et le parfum des choses aromatiques pour le sanctuaire, et ils feront toutes les choses que je t'ai ordonnées.
\TextTitle{Le sabbat comme signe entre Yahweh et Israël}
\VS{12}Yahweh parla encore à Moïse, en disant~:
\VS{13}Toi aussi parle aux enfants d'Israël, en disant~: Certes, vous garderez mes sabbats, car c'est un signe entre moi et vous, et parmi vos générations, afin que vous sachiez que je suis Yahweh qui vous sanctifie.
\VS{14}Gardez donc le sabbat, car il doit vous être saint. Quiconque le violera sera puni de mort~; quiconque,dis-je, fera une œuvre en ce jour-là sera retranché du milieu de son peuple.
\VS{15}On travaillera six jours, mais le septième jour est le sabbat du repos, consacré à Yahweh~; quiconque fera une œuvre le jour du repos sera puni de mort.
\VS{16}Ainsi, les enfants d'Israël garderont le sabbat pour célébrer le jour du repos, en leur génération, par une alliance perpétuelle.
\VS{17}C'est un signe entre moi et les enfants d'Israël à perpétuité~; car Yahweh a fait en six jours les cieux et la terre, et il a cessé au septième, et s'est reposé\FTNT{Ge. 2:2~; Ez. 20:12.}.
\VS{18}Et Dieu donna à Moïse, après qu'il eut achevé de parler avec lui sur la montagne de Sinaï, les deux tables du témoignage~; tables de pierre, écrites du doigt de Dieu\FTNT{De. 9:10.}.
\Chap{32}
\TextTitle{Le culte du veau d'or}
\VerseOne{}Mais le peuple, voyant que Moïse tardait tant à descendre de la montagne, s'assembla autour d'Aaron, et lui dit~: Lève-toi, fais-nous des dieux qui marchent devant nous, car quant à ce Moïse, cet homme qui nous a fait monter du pays d'Egypte, nous ne savons ce qui lui est arrivé\FTNT{Ac. 7:40.}.
\VS{2}Et Aaron leur répondit~: Mettez en pièces les anneaux d'or qui sont aux oreilles de vos femmes, de vos fils, et de vos filles, et apportez-les-moi\FTNT{Ex. 35:22.}.
\VS{3} Et aussitôt, tout le peuple mit en pièces les anneaux d'or qui étaient à leurs oreilles, et ils les apportèrent à Aaron, 
\VS{4}qui les ayant reçu de leurs mains, forma l'or avec un burin, et en fit un veau\FTNT{Selon toute vraisemblance, les Israélites s'étaient inspirés d'une idole égyptienne, le taureau sacré Apis, pour faire le veau d'or. Dieu de la puissance sexuelle, de la fertilité et de la force, il était souvent représenté sous la forme d'un homme avec une tête de taureau, puis avec un disque solaire entre les cornes à partir du Nouvel Empire.} en métal fondu. Et ils dirent~: Ce sont ici tes dieux, ô Israël, qui t'ont fait monter du pays d'Egypte.
\VS{5}Ce qu'Aaron ayant vu, il bâtit un autel devant le veau, et cria en disant~: Demain, il y aura une fête solennelle à Yahweh.
\VS{6}Ainsi, ils se levèrent le lendemain, dès le matin, et ils offrirent des holocaustes, et présentèrent des offrandes de paix. Et le peuple s'assit pour manger et pour boire, puis ils se levèrent pour jouer\FTNT{1 Co. 10:7.}.
\TextTitle{Yahweh condamne l'idolâtrie d'Israël}
\VS{7}Alors Yahweh dit à Moïse~: Va, descends, car ton peuple que tu as fait monter du pays d'Egypte s'est corrompu\FTNT{De. 32:5.}.
\VS{8}Ils se sont promptement détournés de la voie que je leur avais ordonnée, et ils se sont fait un veau en métal fondu, et se sont prosternés devant lui, ils lui ont offert des sacrifices, puis ont dit~: Ce sont ici tes dieux, ô Israël, qui t'ont fait monter du pays d'Egypte\FTNT{1 R. 12:28.}.
\VS{9}Yahweh dit encore à Moïse~: J'ai regardé ce peuple, et voici, c'est un peuple au cou raide.
\VS{10}Maintenant laisse-moi, et ma colère s'embrasera contre eux, et je les consumerai~; mais je te ferai devenir une grande nation.
\TextTitle{Moïse implore Yahweh pour le peuple}
\VS{11}Alors Moïse supplia Yahweh, son Dieu, et dit~: Ô Yahweh, pourquoi ta colère s'embraserait-elle contre ton peuple, que tu as retiré du pays d'Egypte par une grande puissance et par une main forte\FTNT{Ps. 106:23.}~?
\VS{12}Pourquoi les Egyptiens diraient~: Il les a retirés dans de mauvaises vues, pour les tuer sur les montagnes, et pour les consumer de dessus la terre~? Reviens de l'ardeur de ta colère, et repens-toi de ce mal que tu veux faire à ton peuple\FTNT{No. 14:11-15~; De. 9:28.}.
\VS{13}Souviens-toi d'Abraham, d'Isaac et d'Israël, tes serviteurs, auxquels tu as juré par toi-même en leur disant~: Je multiplierai votre postérité comme les étoiles des cieux, et je donnerai à votre postérité tout ce pays, dont j'ai parlé, et ils l'hériteront à jamais\FTNT{De. 34:4.}.
\VS{14}Et Yahweh se repentit du mal qu'il avait dit qu'il ferait à son peuple.
\TextTitle{Jugement sur le peuple}
\VS{15}Moïse regarda, et descendit de la montagne, ayant dans sa main les deux tables du témoignage, et les tables étaient écrites des deux côtés, écrites de l'un et de l'autre côté.
\VS{16}Et les tables étaient l'ouvrage de Dieu, et l'écriture était l'écriture de Dieu, gravée sur les tables.
\VS{17}Et Josué, entendant la voix du peuple qui faisait un grand bruit, dit à Moïse~: Il y a un bruit de bataille au camp.
\VS{18}Et Moïse lui répondit~: Ce n'est pas une voix ni un cri de gens qui soient les plus forts, ni une voix ni un cri de gens qui soient les plus faibles~; mais j'entends une voix de gens qui chantent.
\VS{19}Et il arriva que lorsque Moïse fut approché du camp, il vit le veau et les danses, et la colère de Moïse s'embrasa~; et il jeta de ses mains les tables, et les rompit au pied de la montagne.
\VS{20}Il prit ensuite le veau qu'ils avaient fait, et le brûla au feu, et le moulut jusqu'à ce qu'il fut en poudre, puis il répandit cette poudre dans de l'eau, et il en fit boire aux enfants d'Israël\FTNT{De. 9:17-21.}.
\VS{21}Et Moïse dit à Aaron~: Que t'a fait ce peuple pour que tu aies fait venir sur lui un si grand péché~?
\VS{22}Et Aaron lui répondit~: Que la colère de mon seigneur ne s'embrase point, tu sais que ce peuple est porté au mal.
\VS{23}Ils m'ont dit~: Fais-nous un dieu qui marche devant nous, car ce Moïse, cet homme qui nous a fait monter du pays d'Egypte, nous ne savons ce qui lui est arrivé.
\VS{24}Alors je leur ai dit~: Que celui qui a de l'or, le mette en pièces~! Et ils me l'ont donné~; et je l'ai jeté au feu, et ce veau en est sorti.
\VS{25}Or Moïse vit que le peuple était dénudé, car Aaron l'avait dénudé pour être en opprobre parmi leurs ennemis.
\VS{26}Et Moïse se tenant à la porte du camp, dit~: Qui est pour Yahweh~? Qu'il vienne vers moi~! Et tous les fils de Lévi s'assemblèrent vers lui.
\VS{27}Et il leur dit~: Ainsi parle Yahweh, le Dieu d'Israël~: Que chacun mette son épée à son côté, passez et repassez de porte en porte par le camp, et que chacun de vous tue son frère, son ami, et son voisin.
\VS{28}Et les fils de Lévi firent selon la parole de Moïse~; et ce jour-là il tomba parmi le peuple environ trois mille hommes.
\VS{29}Car Moïse avait dit~: Consacrez aujourd'hui vos mains à Yahweh, chacun même contre son fils, et contre son frère, afin que vous attiriez aujourd'hui sur vous la bénédiction.
\TextTitle{Moïse intercède pour Israël}
\VS{30}Et le lendemain, Moïse dit au peuple~: Vous avez commis un grand péché~; mais je monterai vers Yahweh, et peut-être je ferais propitiation pour votre péché.
\VS{31}Moïse donc retourna vers Yahweh et dit~: Hélas~! Je te prie, ce peuple a commis un grand péché, en se faisant des dieux d'or.
\VS{32}Maintenant pardonne leur péché~! Sinon, efface-moi maintenant de ton livre que tu as écrit.
\VS{33}Et Yahweh répondit à Moïse~: C'est celui qui aura péché contre moi que j'effacerai de mon livre\FTNT{Ap. 3:5~; Ap. 20:15~; Ap. 21:27.}.
\VS{34}Va maintenant, conduis le peuple au lieu duquel je t'ai parlé. Voici, mon Ange ira devant toi~; et le jour où je ferai punition, je punirai sur eux leur péché.
\VS{35} Ainsi, Yahweh frappa le peuple, parce qu'ils avaient été les auteurs du veau qu'Aaron avait fait.
\Chap{33}
\TextTitle{Yahweh ne veut plus marcher avec Israël}
\VerseOne{}Yahweh donc dit à Moïse~: Va, monte d'ici, toi et le peuple que tu as fait monter du pays d'Egypte, au pays que j'ai juré de donner à Abraham, à Isaac, et à Jacob, en disant~: Je le donnerai à ta postérité.
\VS{2}Et j'enverrai un Ange devant toi, et je chasserai les Cananéens, les Amoréens, les Héthiens, les Phéréziens, les Héviens et les Jébusiens,
\VS{3}pour vous conduire au pays découlant de lait et de miel, mais je ne monterai point au milieu de toi, parce que tu es un peuple au cou raide, de peur que je ne te consume en chemin.
\VS{4}Et le peuple entendit ces tristes nouvelles, et en mena le deuil, et aucun d'eux ne mit ses ornements sur soi.
\VS{5}Car Yahweh avait dit à Moïse~: Dis aux enfants d'Israël~: Vous êtes un peuple au cou raide~; je monterai en un moment au milieu de toi, et je te consumerai. Maintenant donc ôte tes ornements de dessus toi, et je saurai ce que je te ferai.
\VS{6}Ainsi, les enfants d'Israël se dépouillèrent de leurs ornements vers la montagne d'Horeb.
\TextTitle{Moïse dresse la tente d'assignation hors du camp}
\VS{7}Et Moïse prit une tente, et la tendit pour soi hors du camp, l'éloignant du camp~; et il l'appela la tente d'assignation~; et tous ceux qui cherchaient Yahweh sortaient vers la tente d'assignation qui était hors du camp.
\VS{8}Et il arrivait qu'aussitôt que Moïse sortait vers la tente, tout le peuple se levait, et chacun se tenait à l'entrée de sa tente, et regardait Moïse par-derrière, jusqu'à ce qu'il soit entré dans la tente.
\VS{9}Et sitôt que Moïse était entré dans la tente, la colonne de nuée descendait et s'arrêtait à la porte de la tente, et Yahweh parlait avec Moïse.
\VS{10}Et tout le peuple voyant la colonne de nuée s'arrêtant à la porte de la tente se levait, et chacun se prosternait à la porte de sa tente.
\VS{11}Et Yahweh parlait à Moïse face à face, comme un homme parle avec son intime ami. Puis Moïse retournait au camp, mais son serviteur Josué, fils de Nun, jeune homme, ne bougeait point de la tente\FTNT{No. 12:8~; De. 34:10~; Jn. 15:14-15.}.
\TextTitle{Moïse demande que Yahweh marche avec Israël}
\VS{12}Moïse donc dit à Yahweh~: Regarde, tu m'as dit~: Fais monter ce peuple, et tu ne m'as point fait connaître celui que tu dois envoyer avec moi~; tu as même dit~: Je te connais par ton nom, et aussi, tu as trouvé grâce devant mes yeux.
\VS{13}Or maintenant, je te prie, si j'ai trouvé grâce devant tes yeux, fais-moi connaître ton chemin, et je te connaîtrai, afin que je trouve grâce devant tes yeux~; considère aussi que cette nation est ton peuple\FTNT{Ps. 25:4.}.
\VS{14}Et Yahweh dit~: Ma face ira, et je te donnerai du repos.
\VS{15}Et Moïse lui dit~: Si ta face ne vient, ne nous fais point monter d'ici.
\VS{16}Car en quoi connaîtra-t-on que nous avons trouvé grace devant tes yeux, moi et ton peuple~? Ne sera-ce pas quand tu marcheras avec nous~? Et alors, moi et ton peuple serons en admiration plus que tous les peuples qui sont sur la terre~?
\VS{17}Et Yahweh dit à Moïse~: Je ferai aussi ce que tu dis~; car tu as trouvé grâce devant mes yeux, et je te connais par ton nom.
\TextTitle{Moïse veut voir la gloire de Yahweh}
\VS{18}Moïse dit aussi~: Je te prie, fais-moi voir ta gloire~!
\VS{19}Et Dieu dit~: Je ferai passer toute ma bonté devant ta face, et je crierai le Nom de Yahweh devant toi~; et je ferai grâce à qui je ferai grâce, et j'aurai compassion de celui de qui j'aurai compassion\FTNT{Ro. 9:15.}.
\VS{20}Puis il dit~: Tu ne pourras pas voir ma face, car nul homme ne peut me voir et vivre\FTNT{Jn. 1:18~; Jn. 14:8-11.}.
\VS{21}Yahweh dit aussi~: Voici, il y a un lieu près de moi, et tu t'arrêteras sur le rocher\FTNT{Le rocher préfigurait Jésus-Christ, le Roc sur lequel nous devons bâtir nos vies et le fondement de l'Eglise (Ps. 18:32~; Mt. 7:24-25~; Mt. 16:18~; 1 Co. 3:11). Voir commentaire Es. 8:13-17.}.
\VS{22}Et quand ma gloire passera, je te mettrai dans un creux du rocher, et te couvrirai de ma main, jusqu'à ce que je sois passé.
\VS{23}Puis je retirerai ma main, et tu me verras par-derrière, mais ma face ne se verra point.
\Chap{34}
\TextTitle{De nouvelles tables~; la gloire de Yahweh\FTNTT{Ex. 33:18-23.}}
\VerseOne{}Et Yahweh dit à Moïse~: Aplanis-toi deux tables de pierre comme les premières, et j'écrirai sur elles les paroles qui étaient sur les premières tables que tu as rompues\FTNT{De. 10:1.}.
\VS{2}Et sois prêt au matin, et monte au matin sur la montagne de Sinaï, et présente-toi là devant moi sur le haut de la montagne.
\VS{3}Mais que personne ne monte avec toi, et même que personne ne paraisse sur toute la montagne~; et que ni menu ni gros bétail ne paisse sur cette montagne\FTNT{Ex. 19:12-13.}.
\VS{4}Moïse donc aplanit deux tables de pierre comme les premières, et se leva de bon matin, et monta sur la montagne de Sinaï, comme Yahweh le lui avait ordonné, et il prit dans sa main les deux tables de pierre.
\VS{5}Et Yahweh descendit dans la nuée, et s'arrêta là avec lui, et cria le Nom de Yahweh.
\VS{6}Comme donc Yahweh passait par devant lui, il cria~: Yahweh, Yahweh~! Le Dieu compatissant, miséricordieux, lent à la colère, abondant en bonté et en fidélité\FTNT{No. 14:18~; 2 Ch. 30:9~; Né. 9:17~; Ps. 103:8.}.
\VS{7}Qui conserve sa bonté jusqu'à mille générations, ôtant l'iniquité, le crime, et le péché, qui ne tient point le coupable pour innocent, et qui punit l'iniquité des pères sur les fils, et sur les fils des fils, jusqu'à la troisième et à la quatrième génération\FTNT{Ex. 20:6~; De. 5:10~; Jé. 32:18.}.
\VS{8}Et Moïse se hatant, baissa la tête contre terre et se prosterna. 
\VS{9}Et il dit~: Ô Seigneur~! Je te prie, si j'ai trouvé grâce à tes yeux, que le Seigneur marche maintenant au milieu de nous, car c'est un peuple au cou raide. Pardonne donc nos iniquités et notre péché, et prends-nous pour ta possession.
\TextTitle{Yahweh renouvelle ses promesses\FTNTT{Ex. 33:18-23.}}
\VS{10}Et il répondit~: Voici, je traite alliance devant tout ton peuple, je ferai des merveilles qui n'ont point été faites sur toute la terre ni dans aucune nation. Et tout le peuple au milieu duquel tu es, verra l'œuvre de Yahweh, car ce que je m'en vais faire avec toi sera une chose redoutable.
\VS{11}Garde soigneusement ce que je t'ordonne aujourd'hui. Voici, je m'en vais chasser devant toi les Amoréens, les Cananéens, les Héthiens, les Phéréziens, les Héviens, et les Jébusiens.
\VS{12}Garde-toi de traiter alliance avec les habitants du pays où tu dois entrer, de peur que peut-être ils ne soient un piège pour toi\FTNT{De. 7:2~; Jos. 23:12-13~; 2 Co. 6:14.}.
\VS{13}Mais vous démolirez leurs autels, vous briserez leurs statues, et vous couperez leurs emblèmes d'Asherah\FTNT{Cité au moins quarante fois dans le Tanakh, le terme hébreu «~Asherah~» fait référence à un «~arbre sacré, un pieu près d'un autel~» ou encore «~une idole~», terme par lequel il est majoritairement traduit. Il s'agit également de l'objet en bois utilisé dans le culte de la parèdre de Baal. Manassé, roi de Juda, introduisit l'emblème d'Asherah dans le temple (2 R. 21:1-7) en dépit de l'interdiction formelle de Yahweh (De. 16:21). Il n'en fut enlevé que lors des réformes de Josias et d'Ezéchias (2 R. 18:3-4~; 2 R. 23:6-14). Pourtant, Yahweh a toujours exigé la destruction de celle qu'il a nommé «~l'abomination des Sidoniens~», de peur que son peuple y trouve une occasion de chute (Jg. 2:13~; Jg. 10:6~; 1 S. 31:10~; 1 R. 11:5-33~; 2 R. 23:13). Bien qu'étant majoritairement citée dans les Ecritures en tant qu'objet de culte, Asherah est également associée à la divinité Astarté, connue pour être la Diane des Ephésiens (Ac. 19:23-40), la reine du ciel (Jé. 7:18~; Jé. 44:15-30), l'Isis des Egyptiens et l'épouse de Baal (voir commentaire en Jg. 2:13).}.
\VS{14}Car tu ne te prosterneras point devant un autre dieu, parce que Yahweh se nomme le Dieu jaloux~; c'est le Dieu qui est jaloux.
\VS{15}Afin qu'il n'arrive que tu traites alliance avec les habitants du pays, et que quand ils viendront à se prostituer après leurs dieux et à sacrifier à leurs dieux, quelqu'un ne t'invite et que tu ne manges de leurs sacrifices~;
\VS{16}et que tu ne prennes de leurs filles pour tes fils, lesquelles se prostituant après leurs dieux, n'entraînent tes fils à se prostituer après leurs dieux.
\VS{17}Tu ne te feras aucun dieu de métal fondu.
\TextTitle{Les fêtes et le sabbat\FTNTT{Lé. 23:4-44.}}
\VS{18}Tu garderas la fête solennelle des pains sans levain~; tu mangeras les pains sans levain pendant sept jours, comme je te l'ai ordonné, dans la saison où les épis mûrissent~; car c'est dans le mois des épis que tu es sorti du pays d'Egypte.
\VS{19}Tout premier-né sera à moi~; même le premier mâle qui naîtra de toutes les bêtes, tant du gros que du menu bétail.
\VS{20}Mais tu rachèteras avec un agneau ou un chevreau le premier-né d'un âne. Si tu ne le rachètes pas, tu lui couperas le cou. Tu rachèteras tout premier-né de tes fils~; et nul ne se présentera devant ma face à vide.
\VS{21}Tu travailleras six jours, mais au septième tu te reposeras~; tu te reposeras au temps du labourage et de la moisson.
\VS{22}Tu feras la fête solennelle des semaines au temps des premiers fruits de la moisson du froment~; et la fête solennelle de la récolte à la fin de l'année.
\VS{23}Trois fois l'an, tout mâle d'entre vous comparaîtra devant le Seigneur Yahweh, le Dieu d'Israël.
\VS{24}Car je déposséderai les nations de devant toi, et j'étendrai tes limites, et nul ne convoitera ton pays lorsque tu monteras pour comparaître trois fois l'an devant Yahweh, ton Dieu.
\VS{25}Tu n'offriras point le sang de mon sacrifice avec du pain levé~; on ne gardera rien du sacrifice de la fête solennelle de la Pâque jusqu'au matin.
\VS{26}Tu apporteras les prémices des premiers fruits de la terre dans la maison de Yahweh, ton Dieu. Tu ne feras point cuire le chevreau dans le lait de sa mère.
\VS{27}Yahweh dit aussi à Moïse~: Ecris ces paroles, car suivant la teneur de ces paroles, j'ai traité alliance avec toi et avec Israël.
\VS{28}Et Moïse demeura là avec Yahweh quarante jours et quarante nuits, sans manger de pain et sans boire d'eau~; et Yahweh écrivit sur les tables les paroles de l'alliance, c'est-à-dire les dix paroles.
\TextTitle{La gloire de Yahweh sur le visage de Moïse}
\VS{29}Or il arriva que lorsque Moïse descendait de la montagne de Sinaï, tenant dans sa main les deux tables du témoignage, lorsque, dis-je, il descendait de la montagne, il ne s'aperçut point que la peau de son visage était devenue rayonnante pendant qu'il parlait avec Dieu.
\VS{30}Mais Aaron et tous les enfants d'Israël ayant vu Moïse, et s'étant aperçus que la peau de son visage était rayonnante, ils craignirent de s'approcher de lui.
\VS{31}Mais Moïse les appela, et Aaron et tous les principaux de l'assemblée retournèrent vers lui~; et Moïse parla avec eux.
\VS{32}Après quoi, tous les enfants d'Israël s'approchèrent, et il leur ordonna toutes les choses que Yahweh lui avait dites sur la montagne de Sinaï.
\VS{33}Ainsi, Moïse acheva de leur parler~; or il avait mis un voile sur son visage.
\VS{34}Et quand Moïse entrait vers Yahweh pour parler avec lui, il ôtait le voile jusqu'à ce qu'il sorte~; et étant sorti, il disait aux enfants d'Israël ce qui lui avait été ordonné.
\VS{35}Or les enfants d'Israël avaient vu que le visage de Moïse, la peau, dis-je, de son visage rayonnait. C'est pourquoi Moïse remettait le voile sur son visage, jusqu'à ce qu'il entre pour parler avec Yahweh.
\Chap{35}
\TextTitle{Rappels sur le sabbat}
\VerseOne{}Moïse donc assembla toute la congrégation des enfants d'Israël, et leur dit~: Ce sont ici les choses que Yahweh a ordonnées de faire.
\VS{2}On travaillera six jours, mais le septième jour il y aura sainteté pour vous, car c'est le sabbat du repos consacré à Yahweh~; quiconque travaillera en ce jour-là sera puni de mort.
\VS{3}Vous n'allumerez point de feu dans aucune de vos demeures le jour du repos.
\TextTitle{Les offrandes pour le tabernacle\FTNTT{Ex. 25:1-8.}}
\VS{4}Puis Moïse parla à toute l'assemblée des enfants d'Israël, et leur dit~: C'est ici ce que Yahweh vous a ordonné, en disant~:
\VS{5}Prenez des choses qui sont chez vous une offrande pour Yahweh. Quiconque sera de bonne volonté, apportera cette offrande pour Yahweh, à savoir de l'or, de l'argent, de l'airain\FTNT{Ex. 25:2~; 2 Co. 8:12.},
\VS{6}de la pourpre, de l'écarlate, du cramoisi, du fin lin, du poil de chèvre,
\VS{7}des peaux de béliers teintes en rouge, des peaux de taissons, du bois d'acacia,
\VS{8}de l'huile pour le chandelier, des aromates pour l'huile d'onction et pour le parfum odoriférant,
\VS{9}des pierres d'onyx, et des pierres pour la garniture de l'éphod et pour le pectoral.
\VS{10}Et tous les hommes d'esprit d'entre vous viendront et feront tout ce que Yahweh a ordonné, 
\VS{11}à savoir le tabernacle, sa tente, et sa couverture, ses agrafes, ses planches, ses barres, ses colonnes, et ses bases~;
\VS{12}l'arche et ses barres, le propitiatoire et le voile qui sert de rideau~;
\VS{13}la table et ses barres, et tous ses ustensiles, et le pain de proposition~;
\VS{14}et le chandelier du luminaire, ses ustensiles, ses lampes et l'huile du luminaire~;
\VS{15}l'autel du parfum et ses barres~; l'huile d'onction, le parfum odoriférant, le rideau de la porte pour l'entrée du tabernacle~;
\VS{16}l'autel de l'holocauste, sa grille d'airain, ses barres et tous ses ustensiles~; la cuve avec sa base~;
\VS{17}les courtines du parvis, ses colonnes, ses bases, et le rideau de la porte du parvis~;
\VS{18}les pieux du tabernacle, les pieux du parvis et leur cordage~;
\VS{19}les vêtements du service pour faire le service dans le sanctuaire, les saints vêtements d'Aaron, le prêtre, et les vêtements de ses fils pour exercer la prêtrise.
\VS{20}Alors toute l'assemblée des enfants d'Israël sortit de la présence de Moïse.
\VS{21}Et quiconque fut ému en son cœur, quiconque, dis-je, se sentit porté à la libéralité, apporta l'offrande de Yahweh pour l'ouvrage de la tente d'assignation et pour tout son service et pour les saints vêtements.
\VS{22}Et les hommes vinrent avec les femmes~; quiconque fut de cœur volontaire, apporta des boucles, des bagues, des anneaux, des bracelets, et des joyaux d'or~; et quiconque offrit quelque offrande d'or à Yahweh.
\VS{23}Tout homme aussi chez qui se trouvait de la pourpre, de l'écarlate, du cramoisi, du fin lin, du poil de chèvre, des peaux de béliers teintes en rouge et des peaux de taissons, les apportèrent.
\VS{24}Tout homme qui avait de quoi faire une offrande d'argent et d'airain, l'apporta pour l'offrande de Yahweh~; tout homme aussi chez qui fut trouvé du bois d'acacia pour tout l'ouvrage du service, l'apporta.
\VS{25}Toute femme adroite fila de sa main et apporta ce qu'elle avait filé~: De la pourpre, de l'écarlate, du cramoisi, et du fin lin\FTNT{Pr. 31:19.}.
\VS{26}Toutes les femmes aussi dont le cœur les y porta en sagesse, filèrent du poil de chèvre.
\VS{27}Les principaux aussi de l'assemblée apportèrent des pierres d'onyx, et d'autres pierres pour la garniture de l'éphod et du pectoral~;
\VS{28}et des aromates, et de l'huile tant pour le chandelier que pour l'huile d'onction, et pour le parfum odoriférant.
\VS{29}Tout homme donc et toute femme que le cœur incita à la libéralité pour apporter de quoi faire l'ouvrage que Yahweh avait ordonné par le moyen de Moïse, tous les enfants, dis-je, d'Israël apportèrent volontairement des présents à Yahweh.
\TextTitle{Betsaleel et Oholiab oints pour l'œuvre du tabernacle}
\VS{30}Alors Moïse dit aux fils d'Israël~: Voyez, Yahweh a appelé par son nom Betsaleel, fils d'Uri, fils de Hur, de la tribu de Juda.
\VS{31}Et il l'a rempli de l'Esprit de Dieu, de sagesse, d'intelligence, de science, pour toutes sortes d'ouvrages.
\VS{32}Même afin d'inventer des dessins, pour travailler l'or, l'argent et l'airain~;
\VS{33}dans la sculpture des pierres précieuses pour les mettre en œuvre, et dans la menuiserie pour travailler en tout ouvrage exquis.
\VS{34}Et il lui a mis aussi au cœur, tant à lui qu'à Oholiab, fils d'Ahisamac, de la tribu de Dan, de l'enseigner.
\VS{35}Et il les a remplis de sagesse pour faire toutes sortes d'ouvrages d'ouvrier, même d'ouvrier en ouvrage exquis, et en broderie, en pourpre, en écarlate, en cramoisi, et en fin lin, et d'ouvrage de tisserand, faisant toutes sortes d'ouvrages et inventant toutes sortes de dessins\FTNT{Es. 28:26.}.
\Chap{36}
\TextTitle{Construction du tabernacle d'après le modèle donné par Yahweh\FTNTT{Ex. 36-39.}}
\VerseOne{}Et Betsaleel et Oholiab, et tous les hommes au coeur sage auxquels Yahweh avait donné de la sagesse et de l'intelligence pour savoir faire tout l'ouvrage du service du sanctuaire, firent selon toutes les choses que Yahweh avait ordonnées.
\VS{2}Moïse donc appela Betsaleel et Oholiab, et tous les hommes d'esprit, dans le cœur desquels Yahweh avait mis de la sagesse, et tous ceux qui furent émus en leur cœur de se présenter pour faire cet ouvrage.
\VS{3}Lesquels emportèrent de devant Moïse toute l'offrande que les enfants d'Israël avaient apportée pour faire l'ouvrage du service du sanctuaire. Or on apportait encore chaque matin quelques offrandes volontaires.
\VS{4}C'est pourquoi, tous les hommes sages qui faisaient tout l'ouvrage du sanctuaire, vinrent chacun de l'ouvrage qu'ils faisaient,
\VS{5}et parlèrent à Moïse, en disant~: Le peuple ne cesse d'apporter plus qu'il ne faut pour le service et pour l'ouvrage que Yahweh a ordonné de faire.
\VS{6}Alors, par l'ordre de Moïse, on fit crier dans le camp que ni homme ni femme ne fasse plus d'ouvrage pour l'offrande du sanctuaire~; et ainsi, on empêcha le peuple d'offrir.
\VS{7}Car ils avaient du travail suffisant pour tout l'ouvrage à faire, et il y en avait même de reste.
\TextTitle{Les tapis de fin lin}
\VS{8}Tous les hommes donc au coeur sage d'entre ceux qui faisaient l'ouvrage, firent le tabernacle, à savoir dix tapis de fin lin retors, de pourpre, d'écarlate, et de cramoisi~; et ils les firent semés de chérubins d'un ouvrage exquis.
\VS{9}La longueur d'un tapis était de vingt-huit coudées, et la largeur du même tapis de quatre coudées~; tous les tapis avaient une même mesure\FTNT{Ex. 26:1-6.}.
\VS{10}Et ils joignirent cinq tapis l'un à l'autre, et cinq autres tapis l'un à l'autre.
\VS{11}Et ils firent des lacets de pourpre sur le bord d'un tapis, à savoir au bord de celui qui était attaché~; ils en firent ainsi au bord du dernier tapis dans l'assemblage de l'autre.
\VS{12}Ils firent cinquante lacets à un tapis, et cinquante lacets au bord du tapis qui était dans l'assemblage de l'autre~; les lacets étant vis-à-vis l'un de l'autre.
\VS{13}Puis on fit cinquante agrafes d'or, et on attacha les tapis l'un à l'autre avec les agrafes~; ainsi fut fait le tabernacle.
\TextTitle{Les tapis de poils de chèvres}
\VS{14}Puis on fit des tapis de poils de chèvres, pour servir de tente au-dessus du tabernacle~; on fit onze de ces tapis.
\VS{15}La longueur d'un tapis était de trente coudées, et la largeur du même tapis de quatre coudées~; et les onze tapis étaient d'une même mesure.
\VS{16}Et on assembla cinq de ces tapis à part, et six tapis à part.
\VS{17}On fit aussi cinquante lacets sur le bord de l'un des tapis, à savoir au dernier qui était attaché, et cinquante lacets sur le bord de l'autre tapis, qui était attaché.
\VS{18}On fit aussi cinquante agrafes d'airain pour assembler la tente, afin qu'il n'y en eut qu'une.
\TextTitle{Les couvertures de peaux de béliers et de taissons}
\VS{19}Puis, on fit pour la tente une couverture de peaux de béliers teintes en rouge, et une couverture de peaux de taissons par-dessus.
\TextTitle{Les planches et leurs bases}
\VS{20}Et on fit pour le tabernacle des planches de bois d'acacia, qu'on fit tenir debout.
\VS{21}La longueur d'une planche était de dix coudées, et la largeur de la même planche d'une coudée et demie.
\VS{22}Il y avait deux tenons à chaque planche en façon d'échelons l'un après l'autre~; on fit la même chose pour toutes les planches du tabernacle.
\VS{23}On fit donc les planches pour le tabernacle~; à savoir vingt planches au côté qui regardaient directement vers le sud.
\VS{24}Et au-dessous des vingt planches, on fit quarante bases d'argent, deux bases sous une planche, pour ses deux tenons, et deux bases sous l'autre planche, pour ses deux tenons.
\VS{25}On fit aussi vingt planches pour l'autre côté du tabernacle, du côté nord,
\VS{26}et leurs quarante bases d'argent~: Deux bases sous une planche, et deux bases sous l'autre planche.
\VS{27}Et pour le fond du tabernacle, vers l'occident, on fit six planches.
\VS{28}Et on fit deux planches pour les angles du tabernacle aux deux cotés du fond~;
\VS{29}qui étaient égales par le bas, et qui étaient jointes et unies par le haut avec un anneau~; on fit la même chose aux deux planches qui étaient aux deux angles.
\VS{30}Il y avait donc huit planches et seize bases d'argent, à savoir deux bases sous chaque planche.
\VS{31}Puis on fit cinq barres de bois d'acacia, pour les planches de l'un des côtés du tabernacle~;
\VS{32}et cinq barres pour les planches de l'autre côté du tabernacle~; et cinq barres pour les planches du tabernacle pour le fond, vers le côté de l'occident.
\VS{33}Et on fit que la barre du milieu passait par le milieu des planches d'une extrémité à l'autre.
\VS{34}Et on couvrit d'or les planches, et on fit leurs anneaux d'or pour y faire passer les barres, et on couvrit d'or les barres.
\TextTitle{Le voile et le rideau extérieur}
\VS{35}On fit aussi le voile de pourpre, d'écarlate, de cramoisi, et de fin lin retors~; on le fit d'ouvrage exquis, avec des chérubins.
\VS{36}Et on lui fit quatre piliers de bois d'acacia, qu'on couvrit d'or, ayant leurs crochets d'or~; et on fondit pour eux quatre bases d'argent.
\VS{37}On fit aussi à l'entrée de la tente un rideau de pourpre, d'écarlate, de cramoisi, et de fin lin retors~; d'ouvrage de broderie~;
\VS{38}et ses cinq piliers avec leurs crochets~; et on couvrit d'or leurs chapiteaux et leurs filets~; mais leurs cinq bases étaient d'airain.
\Chap{37}
\TextTitle{L'arche de l'alliance}
\VerseOne{}Puis Betsaleel fit l'arche de bois d'acacia. Sa longueur était de deux coudées et demie, et sa largeur d'une coudée et demie, et sa hauteur d'une coudée et demie\FTNT{Ex. 23:10-31.}.
\VS{2}Et il la couvrit par dedans et par dehors de pur or, et lui fit un couronnement d'or tout autour.
\VS{3}Et il lui fondit pour elle quatre anneaux d'or pour les mettre sur ses quatre coins, à savoir deux anneaux à l'un de ses côtés, et deux autres à l'autre côté.
\VS{4}Et il fit aussi des barres de bois d'acacia, et les couvrit d'or.
\VS{5}Et il fit entrer les barres dans les anneaux aux côtés de l'arche, pour porter l'arche.
\TextTitle{Le propitiatoire}
\VS{6}Il fit aussi le propitiatoire d'or pur~; sa longueur était de deux coudées et demie, et sa largeur d'une coudée et demie.
\VS{7}Et il fit deux chérubins d'or~; il les fit d'ouvrage étendu au marteau, tirés des deux extrémités du propitiatoire~;
\VS{8}à savoir un chérubin tiré de l'une des extrémités et un chérubin tiré de l'autre extrémité~; il fit, dis-je, les chérubins tirés du propitiatoire~; à savoir de ses deux extrémités. 
\VS{9}Et les chérubins étendaient leurs ailes en haut, couvrant de leurs ailes le propitiatoire~; et leurs faces étaient vis-à-vis l'une de l'autre, et les chérubins regardaient vers le propitiatoire.
\TextTitle{La table des pains de proposition}
\VS{10}Il fit aussi la table de bois d'acacia~; sa longueur était de deux coudées, et sa largeur d'une coudée, et sa hauteur d'une coudée et demie.
\VS{11}Et il la couvrit d'or pur, et lui fit un couronnement d'or tout autour.
\VS{12}Il lui fit aussi à l'entour un rebord d'une largeur d'une paume, et à l'entour de sa bordure un couronnement d'or.
\VS{13}Et il lui fondit quatre anneaux d'or, et il mit les anneaux aux quatre coins, qui étaient à ses quatre pieds.
\VS{14}Les anneaux étaient à coté du rebord, pour y mettre les barres afin de porter la table avec elles.
\VS{15}Et il fit les barres de bois d'acacia, et les couvrit d'or pour porter la table.
\VS{16}Il fit aussi d'or pur des ustensiles pour poser sur la table, ses plats, ses tasses, ses bassins et ses gobelets avec lesquels on devait faire les aspersions.
\TextTitle{Le chandelier}
\VS{17}Il fit aussi le chandelier d'or pur~; il le fit d'ouvrage façonné au marteau~; sa tige, ses branches, ses plats, ses pommeaux et ses fleurs étaient tirés de lui.
\VS{18}Et six branches sortaient de ses côtés, trois branches d'un côté du chandelier, et trois de l'autre côté du chandelier.
\VS{19}Il y avait sur l'une des branches trois plats en forme d'amande, un pommeau et une fleur~; et sur l'autre branche trois plats en forme d'amande, un pommeau et une fleur~; il fit la même chose aux six branches qui sortaient du chandelier.
\VS{20}Et il y avait sur le chandelier quatre plats en forme d'amande, ses pommeaux et ses fleurs.
\VS{21}Et un pommeau sous deux branches tirées du chandelier, et un pommeau sous deux autres branches, tirées de lui, et un pommeau sous deux autres branches, tirées de lui, à savoir des six branches qui procédaient du chandelier.
\VS{22}Leurs pommeaux et leurs branches étaient tirés de lui, et tout le chandelier était un ouvrage d'une seule pièce étendue au marteau et d'or pur.
\VS{23}Il fit aussi ses sept lampes, ses mouchettes, et ses encensoirs d'or pur.
\VS{24}Et il le fit avec toute sa garniture d'un talent d'or pur.
\TextTitle{L'autel des parfums}
\VS{25}Il fit aussi de bois d'acacia l'autel des parfums. Sa longueur était d'une coudée, et sa largeur d'une coudée. Il était carré, et sa hauteur était de deux coudées, et ses cornes procédaient de lui\FTNT{Ex. 30:1-10.}.
\VS{26} Et il couvrit d'or pur le dessus de l'autel, ses côtés tout à l'entour, et ses cornes~; et il lui fit tout à l'entour un couronnement d'or.
\VS{27}Il fit aussi au-dessous de son couronnement deux anneaux d'or à ses deux côtés, lesquels il mit aux deux coins, pour y faire passer les barres afin de le porter avec elles.
\VS{28}Et il fit les barres de bois d'acacia, et les couvrit d'or.
\TextTitle{L'huile d'onction et le parfum}
\VS{29}Il composa aussi l'huile pour l'onction, qui était une chose sainte, et le parfum pur odoriférant, d'ouvrage de parfumeur.
\Chap{38}
\TextTitle{L'autel des holocaustes}
\VerseOne{}Il fit aussi de bois d'acacia l'autel des holocaustes. Et sa longueur était de cinq coudées, et sa largeur de cinq coudées. Il était carré, et sa hauteur était de trois coudées\FTNT{Ex. 27:1-8.}.
\VS{2}Et il fit ses cornes à ses quatre coins. Ses cornes sortaient de lui, et il le couvrit d'airain.
\VS{3}Il fit aussi tous les ustensiles de l'autel~: Les chaudrons, les racloirs, les bassins, les fourchettes et les encensoirs~; il fit tous ses ustensiles d'airain.
\VS{4}Et il fit pour l'autel une grille d'airain en forme de treillis, au-dessous de l'enceinte de l'autel, depuis le bas jusqu'au milieu.
\VS{5}Et il fondit quatre anneaux aux quatre coins de la grille d'airain pour mettre les barres.
\VS{6} Et il fit les barres de bois d'acacia, et les couvrit d'airain.
\VS{7}Et il fit passer les barres dans les anneaux, au cotés de l'autel, pour le porter avec elles. Il le fit creux, avec des planches.
\TextTitle{La cuve d'airain}
\VS{8}Il fit aussi la cuve d'airain et sa base d'airain avec les miroirs des femmes qui s'assemblaient à l'entrée de la tente d'assignation\FTNT{Ex. 30:14-18.}.
\TextTitle{Le parvis}
\VS{9}Il fit aussi un parvis, pour le côté qui regarde vers le sud, et des courtines de fin lin retors, de cent coudées, pour le parvis.
\VS{10}Il fit d'airain leurs vingt colonnes avec leurs vingt bases, mais les crochets des colonnes et leurs filets étaient d'argent.
\VS{11}Et pour le côté nord, il fit des courtines de cent coudées, leurs vingt colonnes et leurs vingt bases étaient d'airain, mais les crochets des colonnes et leurs filets étaient d'argent.
\VS{12}Pour le côté de l'occident, des courtines de cinquante coudées, leurs dix colonnes, et leurs dix bases. Les crochets des colonnes et leurs filets étaient d'argent.
\VS{13}Pour le côté de l'orient droit vers le levant, des courtines de cinquante coudées.
\VS{14}Il fit pour l'un des côtés quinze coudées de courtines, et leurs trois colonnes avec leurs trois bases.
\VS{15}Et pour l'autre côté, quinze coudées de courtines, afin qu'il y en ait autant de part et d'autre de la porte du parvis, et leurs trois colonnes avec leurs trois bases.
\VS{16}Il fit donc toutes les courtines du parvis qui étaient tout autour de fin lin retors.
\VS{17}Il fit aussi d'airain les bases des colonnes, mais il fit d'argent les crochets des colonnes et les filets, et leurs chapiteaux furent couverts d'argent~; et toutes les colonnes du parvis furent ceintes tout autour d'un filet d'argent.
\TextTitle{La porte du parvis}
\VS{18} Et il fit le rideau de la porte du parvis de pourpre, d'écarlate, et de cramoisi et de fin lin retors, d'ouvrage de broderie, de la longueur de vingt coudées, et de la hauteur qui était comme la largeur de cinq coudées, à la correspondance des courtines du parvis~;
\VS{19}et ses quatre colonnes avec leurs bases d'airain, et leurs crochets d'argent, la couverture aussi de leur chapiteaux et leurs filets d'argent~; 
\VS{20} et tous les pieux du tabernacle et du parvis tout autour d'airain.
\TextTitle{Les comptes du tabernacle}
\VS{21}C'est ici le compte des choses qui furent employées au tabernacle, à savoir à la tente d'assignation, selon que le compte en fut fait par l'ordre de Moïse, à quoi furent employés les Lévites, sous la conduite d'Ithamar, fils du prêtre Aaron.
\VS{22}Et Betsaleel, fils d'Uri, fils de Hur, de la tribu de Juda, fit toutes les choses que Yahweh avait ordonnées à Moïse~;
\VS{23}et avec lui Oholiab, fils d'Ahisamac, de la tribu de Dan, les ouvriers, et ceux qui travaillaient en ouvrage exquis, et les brodeurs en pourpre, en écarlate, en cramoisi, et en fin lin.
\VS{24}Tout l'or qui fut employé pour l'ouvrage, à savoir pour tout l'ouvrage du sanctuaire, qui était l'or des offrandes, fut de vingt-neuf talents et sept cent trente sicles, selon le sicle du sanctuaire.
\VS{25} Et l'argent de ceux de l'assemblée qui furent dénombrés fut de cent talents et mille sept cent soixante-quinze sicles, selon le sicle du sanctuaire.
\VS{26}Un demi-sicle par tête, la moitié d'un sicle selon le sicle du sanctuaire. Tous ceux qui passèrent par le dénombrement depuis l'âge de vingt ans et au-dessus, furent six cent trois mille cinq cent cinquante.
\VS{27}Il y eut donc cent talents d'argent pour fondre les bases du sanctuaire, et les bases du voile, à savoir cent bases de cent talents, un talent pour chaque base.
\VS{28}Mais des mille sept cent soixante-quinze sicles, il fit les crochets pour les colonnes, et il couvrit leurs chapiteaux et en fit des filets tout autour.
\VS{29}L'airain des offrandes fut de soixante-dix talents et deux mille quatre cents sicles~;
\VS{30}dont on fit les bases de la porte de la tente d'assignation, et l'autel d'airain avec sa grille d'airain, et tous les ustensiles de l'autel~;
\VS{31}et les bases tout autour du parvis, les bases de la porte du parvis, et tous les pieux du tabernacle, et tous les pieux du parvis tout autour.
\Chap{39}
\TextTitle{Les vêtements sacrés d'Aaron}
\VerseOne{}Ils firent aussi de pourpre, d'écarlate, et de cramoisi les vêtements du service, pour faire le service du sanctuaire. Et ils firent les saints vêtements sacrés pour Aaron, comme Yahweh l'avait ordonné à Moïse\FTNT{Ex. 28.}.
\VS{2}On fit donc l'éphod d'or, de pourpre, d'écarlate, de cramoisi, et de fin lin retors.
\VS{3}Or on étendit des lames d'or, et on les coupa par filets pour les brocher parmi la pourpre, l'écarlate, le cramoisi, et le fin lin d'ouvrage exquis.
\VS{4}On fit à l'éphod des épaulettes\FTNT{Voir annexe «~Les habits du grand-prêtre~».} qui s'attachaient, en sorte qu'il était joint par ses deux extrémités.
\VS{5}Et la ceinture exquise de laquelle il était ceint, était tirée de lui, et de même ouvrage, d'or, de pourpre, d'écarlate, de cramoisi, et de fin lin retors, comme Yahweh l'avait ordonné à Moïse.
\VS{6}On enchassa aussi les pierres d'onyx dans leurs montures d'or, ayant les noms des enfants d'Israël gravés comme on grave les cachets.
\VS{7}Et on les mit sur les épaulettes de l'éphod, afin qu'elles soient des pierres de souvenir pour les enfants d'Israël, comme Yahweh l'avait ordonné à Moïse.
\VS{8}On fit aussi le pectoral\FTNT{Voir annexe «~Les habits du grand-prêtre~».} d'ouvrage exquis, comme l'ouvrage de l'éphod, d'or, de pourpre, d'écarlate, de cramoisi, et de fin lin retors.
\VS{9}On fit le pectoral carré et double~; sa longueur était d'une paume, et sa largeur d'une paume de part et d'autre.
\VS{10}Et on le garnit de quatre rangs de pierres~: A la première rangée on mit une sardoine, une topaze et une émeraude.
\VS{11}A la seconde rangée une escarboucle, un saphir, et un jaspe.
\VS{12}A la troisième rangée, une opale, une agate, et une améthyste.
\VS{13}A la quatrième rangée, un chrysolithe, un onyx, et un béryl\FTNT{Ap. 21:18-19.}, enchâssés dans leur monture d'or.
\VS{14}Ainsi, il y avait autant de pierres qu'il y avait de noms des enfants d'Israël, douze selon leurs noms, chacune d'elles gravées comme des cachets, selon le nom qu'elle devait porter, et elles étaient pour les douze tribus.
\VS{15}Et on fit sur le pectoral des chaînettes à bouts en façon de cordon, d'or pur.
\VS{16}On fit aussi deux montures d'or et deux anneaux d'or, on mit les deux anneaux aux deux extrémités du pectoral.
\VS{17}Et on mit les deux chaînettes d'or faites à cordon dans les deux anneaux à l'extrémité du pectoral.
\VS{18}Et on mit les deux autres bouts des deux chaînettes faites à cordon aux deux montures, sur les épaulettes de l'éphod, sur le devant de l'éphod.
\VS{19}On fit aussi deux autres anneaux d'or, et on les mit aux deux autres extrémités du pectoral sur son bord, qui était du côté de l'éphod à l'intérieur.
\VS{20}On fit aussi deux autres anneaux d'or, et on les mit aux deux épaulettes de l'éphod par le bas, répondant sur le devant de l'éphod, à l'endroit où il se joignait au-dessus de la ceinture exquise de l'éphod.
\VS{21}Et on joignit le pectoral élevé par ses anneaux aux anneaux de l'éphod, avec un cordon de pourpre, afin qu'il tienne au-dessus de la ceinture exquise de l'éphod, et que le pectoral ne bouge de dessus l'éphod, comme Yahweh l'avait ordonné à Moïse.
\VS{22}On fit aussi la robe de l'éphod d'ouvrage tissé et entièrement de pourpre.
\VS{23} Et l'ouverture pour passer la tête était au milieu de la robe, comme l'ouverture d'une cotte de mailles~; et il y avait un ourlet à l'ouverture de la robe tout autour, afin qu'elle ne se déchire pas.
\VS{24}Aux bordures de la robe, on fit des grenades de pourpre, d'écarlate et de cramoisi, à fil retors.
\VS{25}On fit aussi des clochettes d'or pur~; et on mit les clochettes entre les grenades aux bordures de la robe tout autour, parmi les grenades~;
\VS{26}à savoir une clochette puis une grenade, une clochette puis une grenade, sur la bordure de la robe tout autour, pour faire le service, comme Yahweh l'avait ordonné à Moïse.
\VS{27}On fit aussi à Aaron et à ses fils des tuniques de fin lin d'ouvrage tissé.
\VS{28}Et la tiare de fin lin, et les ornements des calottes de fin lin, et les caleçons de lin, de fin lin retors.
\VS{29}Et la ceinture de fin lin retors, de pourpre, d'écarlate et de cramoisi, d'ouvrage de broderie~; comme Yahweh l'avait ordonné à Moïse~;
\VS{30}et la lame du saint diadème d'or pur, sur laquelle on écrivit comme on grave un cachet~: La sainteté à Yahweh.
\VS{31}Et on mit sur elle un cordon de pourpre, pour l'appliquer à la tiare par-dessus, comme Yahweh l'avait ordonné à Moïse.
\TextTitle{Le matériel pour excercer la prêtrise est prêt}
\VS{32}Ainsi fut achevé tout l'ouvrage du tabernacle, de la tente d'assignation. Les enfants d'Israël firent selon toutes les choses que Yahweh avait ordonnées à Moïse~; ils les firent ainsi.
\VS{33}Et ils apportèrent à Moïse le tabernacle, la tente, et tous ses ustensiles, ses crochets, ses planches, ses barres, ses colonnes, et ses bases~;
\VS{34}la couverture de peaux de béliers teintes en rouge, la couverture de peaux de taissons, et le voile qui sert de rideau devant le Saint des saints~;
\VS{35}l'arche du témoignage et ses barres, et le propitiatoire~;
\VS{36}la table avec tous ses ustensiles, et les pains de proposition\FTNT{Ex. 31:8-10.}~;
\VS{37}et le chandelier d'or pur avec toutes ses lampes arrangées, et tous ses ustensiles, et l'huile du chandelier~;
\VS{38}et l'autel d'or, l'huile d'onction, le parfum odoriférant, et le rideau de l'entrée de la tente~;
\VS{39}l'autel d'airain, avec sa grille d'airain, ses barres et tous ses ustensiles~; la cuve et sa base~;
\VS{40}et les courtines du parvis, ses colonnes, ses bases, le rideau pour la porte du parvis, son cordage, ses pieux, et tous les ustensiles pour le service du tabernacle, pour la tente d'assignation~;
\VS{41}les vêtements du service pour faire le service du sanctuaire, les saints vêtements pour le prêtre Aaron, et les vêtements de ses fils pour exercer la prêtrise.
\VS{42}Les enfants d'Israël donc firent tout l'ouvrage, comme Yahweh l'avait ordonné à Moïse.
\VS{43}Et Moïse vit tout l'ouvrage, et voici, on l'avait fait ainsi que Yahweh l'avait ordonné, on l'avait, dis-je, fait ainsi. Et Moïse les bénit.
\Chap{40}
\TextTitle{Moïse dresse le tabernacle}
\VerseOne{}Et Yahweh parla à Moïse, en disant~:
\VS{2}Au premier jour du premier mois, tu dresseras le tabernacle de la tente d'assignation.
\VS{3}Et tu y mettras l'arche du témoignage, au-devant de laquelle tu tendras le voile.
\VS{4}Puis tu apporteras la table et y arrangeras ce qui doit y être arrangé. Tu apporteras aussi le chandelier et allumeras ses lampes.
\VS{5}Tu mettras aussi l'autel d'or pour le parfum au-devant de l'arche du témoignage, et tu mettras le rideau de l'entrée au tabernacle.
\VS{6}Tu mettras aussi l'autel de l'holocauste vis-à-vis de l'entrée du tabernacle de la tente d'assignation.
\VS{7}Tu mettras aussi la cuve entre la tente d'assignation et l'autel, et y mettras de l'eau.
\VS{8}Tu mettras aussi le parvis tout autour, et tu mettras le rideau à la porte du parvis.
\VS{9}Tu prendras aussi l'huile de l'onction, et tu en oindras le tabernacle, et tout ce qui y est, et tu le sanctifieras avec tous ses ustensiles~; et il sera saint.
\VS{10}Tu oindras aussi l'autel de l'holocauste, et tous ses ustensiles, et tu sanctifieras l'autel, et l'autel sera très saint.
\VS{11}Tu oindras aussi la cuve et sa base, et la sanctifieras.
\VS{12}Tu feras aussi approcher Aaron et ses fils à l'entrée de la tente d'assignation, et les laveras avec de l'eau.
\VS{13}Et tu feras vêtir à Aaron les saints vêtements, et tu l'oindras et le sanctifieras~; et il exercera la prêtrise pour moi.
\VS{14}Tu feras aussi approcher ses fils que tu revêtiras de tuniques.
\VS{15} Et tu les oindras comme tu auras oint leur père~; et ils m'exerceront la prêtrise, et leur onction leur sera pour exercer la prêtrise à toujours parmi leur génération.
\VS{16} Ce que Moïse fit selon toutes les choses que Yahweh lui avait ordonnées~; il le fit ainsi.
\VS{17}Car au premier jour du premier mois de la seconde année, le tabernacle fut dressé.
\VS{18}Moïse donc dressa le tabernacle, mit ses bases, posa ses planches, mit ses barres et dressa ses colonnes.
\VS{19}Et il étendit la tente sur le tabernacle, et mit la couverture de la tente au-dessus du tabernacle par le haut, comme Yahweh l'avait ordonné à Moïse.
\VS{20}Puis il prit et posa le témoignage dans l'arche et mit les barres à l'arche~; il mit aussi le propitiatoire au-dessus de l'arche.
\VS{21}Et il apporta l'arche dans le tabernacle, et posa le voile qui sert de rideau, et le mit au-devant de l'arche du témoignage, comme Yahweh l'avait ordonné à Moïse.
\VS{22}Il mit aussi la table dans la tente d'assignation, au côté du tabernacle vers le nord, en dehors du voile.
\VS{23}Et il arrangea sur elle les rangées de pains devant Yahweh, comme Yahweh l'avait ordonné à Moïse.
\VS{24}Il mit aussi le chandelier dans la tente d'assignation, vis-à-vis de la table, du côté du tabernacle, vers le sud.
\VS{25}Et il alluma les lampes devant Yahweh, comme Yahweh l'avait ordonné à Moïse.
\VS{26}Il posa aussi l'autel d'or dans la tente d'assignation, devant le voile.
\VS{27}Et il fit fumer sur lui le parfum odoriférant, comme Yahweh l'avait ordonné à Moïse.
\VS{28}Il mit aussi le rideau de l'entrée du tabernacle.
\VS{29}Et il mit l'autel de l'holocauste à l'entrée du tabernacle de la tente d'assignation~; et offrit sur lui l'holocauste et l'offrande, comme Yahweh l'avait ordonné à Moïse.
\VS{30}Et il plaça la cuve entre la tente d'assignation et l'autel, et y mit de l'eau pour se laver.
\VS{31}Et Moïse et Aaron avec ses fils en lavèrent leurs mains et leurs pieds.
\VS{32}Et quand ils entraient dans la tente d'assignation, et qu'ils approchaient de l'autel, ils se lavaient, selon que Yahweh l'avait ordonné à Moïse.
\VS{33}Il dressa aussi le parvis tout autour du tabernacle et de l'autel, et tendit le rideau de la porte du parvis. Ainsi Moïse acheva l'ouvrage.
\TextTitle{La gloire de Yahweh sur le tabernacle}
\VS{34}Et la nuée couvrit la tente d'assignation, et la gloire de Yahweh remplit le tabernacle\FTNT{No. 9:15~; 1 R. 8:10.},
\VS{35}tellement que Moïse ne put entrer dans la tente d'assignation, car la nuée se tenait dessus et la gloire de Yahweh remplissait le tabernacle.
\VS{36} Or quand la nuée se levait de dessus le tabernacle, les enfants d'Israël partaient dans toutes leurs marches.
\VS{37}Mais si la nuée ne se levait point, ils ne partaient point, jusqu'au jour où elle se levait.
\VS{38}Car la nuée de Yahweh était le jour sur le tabernacle, et le feu y était la nuit, devant les yeux de toute la maison d'Israël, dans toutes leurs marches.
\PPE
\end{multicols}

%\clearpage\ShortTitle{Lé.}\BookTitle{Lévitique}\BFont
\noindent\hrulefill
{\footnotesize
\textit{
\bigskip
{\centering{}
\\Auteur : Probablement Moïse
\\(Heb. : Vayiqra)
\\Signification : Et Il (Yahweh) appela
\\Thème : La sainteté
\\Date de rédaction : Env. 1450-1410 av. J.-C.\\}
}
%\bigskip
\textit{
\\Après avoir construit et dressé le tabernacle selon le modèle que Yahweh avait donné à Moïse, les fils d'Israël reçurent le détail des prescriptions relatives aux offrandes, aux sacrifices et aux fêtes en l'honneur de Yahweh. 
%\bigskip
\\Ce livre, dont le nom tire son origine de Lévi, explique la manière dont Aaron et ses fils devaient exercer la sacrificature et amener le peuple à s'approcher de Dieu dans le respect de ses ordonnances.
%\bigskip
\\Les lois que Moïse avait recueillies présentent la voie du pardon, laquelle est impossible sans effusion de sang. Bien que les mêmes sacrifices furent réitérés tous les ans, ces préceptes mettaient en évidence l'impuissance de l'homme à atteindre la justice de Dieu par ses propres moyens.\bigskip
}
}
\par\nobreak\noindent\hrulefill
\begin{multicols}{2}
\Chap{1}
\TextTitle{L'holocauste\FTNTT{voir Lé. 6:1-6.}}
\VerseOne{}Et Yahweh appela Moïse, et lui parla de la tente d'assignation, en disant :
\VS{2}Parle aux enfants d'Israël, et dis-leur : Quand quelqu'un d'entre vous offrira à Yahweh une offrande d'une bête à quatre pattes, il fera son offrande de gros ou de menu bétail.
\VS{3}Si son offrande pour un holocauste est de gros bétail, il offrira un mâle sans défaut\FTNT{L'holocauste était le sacrifice pour l'expiation par excellence. Contrairement aux autres sacrifices, l'holocauste était entièrement consumé sur l'autel. Il symbolisait d'une part le sacrifice parfait de Christ et d'autre part notre vie, volontairement offerte à Dieu (Ro. 12:1). Les animaux aptes à être offerts en holocauste devaient être des mâles sans défaut :
\\- Le veau (Lé. 1:5), image de Christ, l'humble serviteur, soumis et obéissant (Mt. 20:28 ; Ph. 2:5-8).
\\- L'agneau ou le chevreau, image de Christ qui livre sa vie à la croix sans résistance ni contestation, et qui prend sur lui nos péchés (Es. 53:7 ; Mt. 26:63 ; Ac. 8:32). 
\\- Les tourterelles ou les jeunes pigeons, image de la simplicité de Christ (Mt. 10:16).
\\Toutes les étapes de la réalisation de ce sacrifice enseignent le disciple sur la mort à soi-même et le dépouillement des œuvres de la chair (Ga. 5:19-21).
\\Le sang de l'animal égorgé devait être répandu sur l'autel (Lé. 1:5), image de la croix. L'âme (contenue dans le sang selon Lé. 17:14), liée à la chair et ses désirs, doit être crucifiée (Ga. 2:20 ; Ga. 5:24). L'objet de la mise à mort était certainement un couteau tranchant comme une épée, image de la Parole de Dieu (Hé. 4:12). La mise en pratique de la Parole nous amène nécessairement à nous séparer du monde et à renoncer à soi-même.} ; il l'offrira de son bon gré à l'entrée de la tente d'assignation ; devant Yahweh\FTNT{Ex. 29:10-11.}.
\VS{4}Et il posera sa main sur la tête de l'holocauste, et il sera agréé pour lui, afin de faire la propitiation pour lui.
\VS{5}Puis, on égorgera le jeune taureau devant Yahweh ; et les fils d'Aaron, les prêtres, en offriront le sang et ils répandront le sang sur l'autel tout autour, qui est à l'entrée de la tente d'assignation.
\VS{6}Et on égorgera l'holocauste et le coupera en morceaux.
\VS{7}Les fils du prêtre Aaron mettront le feu sur l'autel, et disposeront le bois sur le feu.
\VS{8}Et les fils d'Aaron, les prêtres, poseront les morceaux, la tête et la graisse sur le bois qui sera au feu sur l'autel.
\VS{9}Mais il lavera avec de l'eau les entrailles et les jambes ; et le prêtre brûlera toutes ces choses sur l'autel. C'est un holocauste, un sacrifice consumé par le feu, d'une bonne odeur à Yahweh.
\VS{10}Si son offrande est un holocauste de menu bétail, d'entre les agneaux ou d'entre les chèvres, il offrira un mâle sans défaut.
\VS{11}Et on l'égorgera à côté de l'autel, vers le nord, devant Yahweh ; et les prêtres, fils d'Aaron, en répandront le sang sur l'autel tout autour.
\VS{12}Puis on le coupera en morceaux, avec sa tête et sa graisse ; et le prêtre les posera sur le bois qui sera au feu sur l'autel.
\VS{13}Mais il lavera avec de l'eau les entrailles et les jambes. Puis le prêtre offrira toutes ces choses, et les brûlera sur l'autel. C'est un holocauste, un sacrifice consumé par le feu, d'une agréable odeur à Yahweh\FTNT{Ez. 40:38.}.
\VS{14}Si son offrande à Yahweh est un holocauste d'oiseaux, il offrira son offrande de tourterelles, ou de jeunes pigeons.
\VS{15}Le prêtre l'apportera sur l'autel, lui ouvrira la tête avec l'ongle, la brûlera sur l'autel, et il en exprimera le sang contre un côté de l'autel.
\VS{16}Il ôtera son jabot avec ses plumes, et le jettera près de l'autel, vers l'orient, dans le lieu où seront les cendres.
\VS{17}Il le déchirera avec ses ailes, sans le séparer ; et le prêtre le brûlera sur l'autel, sur le bois qui sera au feu. C'est un holocauste, un sacrifice consumé par le feu, d'une agréable odeur à Yahweh.
\Chap{2}
\TextTitle{L'offrande de gâteau\FTNTT{Lé. 6:7-16.}}
\VerseOne{}Lorsque quelqu'un offrira l'offrande de gâteau\FTNT{L'offrande de farine ou de gâteau correspond aux perfections de la vie du Seigneur Jésus-Christ en tant qu'homme. Ce sacrifice ne comporte ni victime ni sang, mais seulement de la farine, de l'huile, de l'encens et du sel. Jésus, le grain de blé (Jn. 12:24), a été complètement broyé, pétri et oint d'huile, éprouvé par toutes sortes de douleurs. Sa vie sainte était pour le Père un parfum de bonne odeur. Son amour pour les âmes, sa dépendance totale au Père, sa persévérance, sa douceur, sa sagesse et sa bonté, n'ont pas varié malgré toutes les souffrances par lesquelles il est passé. Voilà quelques-uns des fruits admirables qui correspondent à l'offrande de gâteau saupoudrée d'encens. Le levain, image du péché (1 Co. 5:6-8), n'y entrait pas, ni le miel, symbole des affections humaines (Pr. 5:3). Quant au sel, il préserve de la corruption des aliments, il est comparé à la saveur des disciples de Christ (Mt. 5:13).} à Yahweh, son offrande sera de fine farine ; il versera de l'huile dessus, et mettra de l'encens.
\VS{2}Il l'apportera aux fils d'Aaron, les prêtres, et le prêtre prendra une pleine poignée de cette fine farine, et d'huile, avec tout l'encens, et il brûlera son souvenir\FTNT{En hébreu « azkarah », offrande de souvenir, la portion de nourriture offerte et qui est consumée.} sur l'autel. C'est une offrande d'une bonne odeur à Yahweh.
\VS{3}Ce qui restera du gâteau sera pour Aaron et ses fils ; c'est une chose très sainte parmi les offrandes consumées par le feu à Yahweh.
\VS{4}Et quand tu offriras une offrande de gâteaux cuits au four, ce sera de fine farine, des gâteaux sans levain, pétris avec de l'huile, et des galettes sans levain, ointes d'huile.
\VS{5}Si ton offrande est un gâteau cuit sur la plaque, elle sera de fine farine pétrie à l'huile, sans levain.
\VS{6}Tu la rompras en morceaux, et tu verseras de l'huile sur elle ; c'est une offrande de gâteau.
\VS{7}Si ton offrande est un gâteau cuit sur le gril, elle sera faite de fine farine avec de l'huile.
\VS{8}Puis tu apporteras à Yahweh l'offrande de gâteaux qui sera faite de ces choses, et on la présentera au prêtre, qui l'apportera sur l'autel.
\VS{9}Le prêtre lèvera de l'offrande de gâteaux, son souvenir, et le brûlera sur l'autel. C'est une offrande consumée par le feu de bonne odeur à Yahweh.
\VS{10}Ce qui restera de l'offrande de gâteau sera pour Aaron et ses fils ; c'est une chose très sainte parmi les offrandes consumées par le feu devant Yahweh.
\VS{11}Aucune offrande de gâteau que vous offrirez à Yahweh ne sera faite avec du levain ; car vous ne brûlerez point de levain ni de miel, parmi l'offrande consumée par le feu devant Yahweh.
\VS{12}Vous pourrez bien les offrir à Yahweh dans l'offrande des prémices, mais ils ne seront point mis sur l'autel comme offrande d'une bonne odeur.
\VS{13}Tu mettras du sel\FTNT{Voir No. 18:19 ; 2 Ch. 13:5. Le sel est un agent purificateur (2 R. 2:19-22). Le sel préserve de la corruption et conserve les aliments. Les chrétiens sont le sel de la terre (Mt. 5:13). Nos paroles doivent être assaisonnées de sel (Col. 4:6).} sur toutes tes offrandes de gâteaux, et tu ne laisseras point ton offrande de gâteau manquer de sel, signe de l'alliance de ton Dieu ; mais sur toutes tes offrandes, tu offriras du sel.
\VS{14}Si tu offres à Yahweh une offrande de gâteau des premiers fruits, tu offriras, pour l'offrande de gâteau des premiers fruits, des épis qui commencent à mûrir, rôtis au feu, les grains de quelques épis bien grenés, broyés entre les mains.
\VS{15}Puis tu mettras de l'huile sur le gâteau, et tu mettras aussi de l'encens dessus : C'est une offrande de gâteaux.
\VS{16}Et le prêtre brûlera son souvenir, pris de ses grains broyés, et de son huile avec tout l'encens. C'est une offrande consumée par le feu à Yahweh.
\Chap{3}
\TextTitle{Le sacrifice d'offrande de paix\FTNTT{Lé. 7:11-21.}}
\VerseOne{}Si son offrande est un sacrifice d'offrande de paix\FTNT{La plupart des traducteurs ont traduit par « sacrifice d'actions de grâces », or l'étymologie hébraïque du mot grâce est « shelem », ce qui signifie d'abord « paix ». Ce terme peut aussi vouloir dire « remerciement » ou « reconnaissance ». La racine de « shelem » est « shalam » : « être dans une alliance de paix », « être en paix ».Il est donc question ici d'une offrande de paix qui préfigure l'ensemble de l'œuvre de la croix accomplie par le Messie, et grâce à laquelle nous sommes réconciliés avec le Père (Col. 1:20 ; Ep. 2:14-17). Cette offrande préfigure aussi la Pâque incarnée par le Messie (1 Co. 5:7) ainsi que le repas du Seigneur. En effet, sur cette offrande, Dieu prenait pour lui la graisse et la queue entière (Lé. 3:3 ; Lé. 3:9-17), le prêtre prenait la poitrine et l'épaule droite (Lé. 7:31-34), et celui qui offrait l'animal pouvait consommer le reste avec d'autres personnes pures (Lé. 7:20). Ainsi, comme pour le repas du Seigneur, tous ceux qui étaient saints pouvaient participer au repas (1 Co. 11:27-34).}, et qu'il offre du gros bétail, soit mâle, soit femelle, il l'offrira sans défaut devant Yahweh.
\VS{2}Il posera sa main sur la tête de son offrande, et l'égorgera à l'entrée de la tente d'assignation, et les fils d'Aaron, les prêtres, répandront le sang sur l'autel tout autour.
\VS{3}Puis on offrira de cette offrande de paix, un sacrifice consumé par le feu à Yahweh, à savoir la graisse qui couvre les entrailles et toute la graisse qui est sur les entrailles ;
\VS{4}les deux rognons avec la graisse qui est dessus et qui est sur les flancs ; et on ôtera le grand lobe qui est sur le foie pour le mettre avec les rognons.
\VS{5}Les fils d'Aaron brûleront tout cela sur l'autel, sur l'holocauste, qui sera sur le bois mis au feu. C'est une offrande consumée par le feu d'agréable odeur à Yahweh\FTNT{Ex. 29:13-25.}.
\VS{6}Si son offrande pour le sacrifice d'offrande de paix à Yahweh est de menu bétail, soit mâle, soit femelle, il l'offrira sans défaut.
\VS{7}S'il offre un agneau pour son offrande, il l'offrira devant Yahweh.
\VS{8}Il posera sa main sur la tête de son offrande, et l'égorgera devant la tente d'assignation, et les fils d'Aaron répandront son sang sur l'autel tout autour.
\VS{9}De ce sacrifice d'offrande de paix, il offrira en offrande consumée par le feu à Yahweh, sa graisse et sa queue entière, séparée jusqu'à l'échine, avec la graisse qui couvre les entrailles et toute la graisse qui est sur les entrailles,
\VS{10}les deux rognons avec la graisse qui est dessus, sur les flancs, et il ôtera le grand lobe qui est sur le foie, jusqu'aux rognons.
\VS{11}Le prêtre brûlera tout cela sur l'autel. C'est un aliment d'offrande consumée par le feu à Yahweh\FTNT{No. 28:2.}.
\VS{12}Si son offrande est une chèvre, il l'offrira devant Yahweh.
\VS{13}Il posera sa main sur sa tête, et l'égorgera devant la tente d'assignation ; et les fils d'Aaron répandront son sang sur l'autel tout autour.
\VS{14}Puis il offrira son offrande en sacrifice consumé par le feu à Yahweh, la graisse qui couvre les entrailles et toute la graisse qui est sur les entrailles,
\VS{15}les deux rognons, et la graisse qui est dessus, sur les flancs, et il ôtera le grand lobe qui est sur le foie, jusqu'aux rognons.
\VS{16}Puis le prêtre brûlera toutes ces choses sur l'autel. C'est un aliment d'offrande consumée par le feu de bonne odeur. Toute graisse appartient à Yahweh.
\VS{17}C'est une loi perpétuelle pour vos descendants, dans toutes vos demeures : Vous ne mangerez ni graisse ni sang\FTNT{Ge. 9:4 ; 1 S. 14:33.}.
\Chap{4}
\TextTitle{Le sacrifice pour l'expiation\FTNTT{Lé. 6:17-23.}}
\VerseOne{}Yahweh parla encore à Moïse en disant :
\VS{2}Parle aux enfants d'Israël, et dis-leur : Quand une personne aura péché involontairement\FTNT{Avant la promulgation de la loi, certains hommes péchaient par ignorance (Ro. 5:13). Néanmoins, ces péchés étaient tout de même punis et nécessitaient un sacrifice (Lé. 4:13-14. No. 15:22-36 ; Job. 1). Sous la grâce, l'excuse du péché par ignorance ne peut être invoquée puisque nous sommes scellés du Saint-Esprit qui nous enseigne toutes choses (1 Jn. 2:20 et 27).} contre l'un des commandements de Yahweh, en commettant des choses qui ne doivent point se faire, et qu'il aura fait une de ces choses ;
\VS{3} si c'est le prêtre oint qui ait commis un péché, semblable à quelque faute du peuple, il offrira à Yahweh pour son péché qu'il aura fait, un jeune taureau sans défaut, pris du troupeau en sacrifice pour l'expiation.
\VS{4}Il amènera le taureau à l'entrée de la tente d'assignation, devant Yahweh, il posera sa main sur la tête du taureau, et l'égorgera devant Yahweh.
\VS{5}Et le prêtre oint prendra du sang du taureau, et l'apportera dans la tente d'assignation.
\VS{6}Le prêtre trempera son doigt dans le sang, et fera sept fois l'aspersion du sang devant Yahweh, en face du voile du lieu saint\FTNT{No. 19:4.}.
\VS{7}Le prêtre mettra aussi devant Yahweh du sang sur les cornes de l'autel des parfums odoriférants, qui est dans la tente d'assignation ; et il répandra tout le reste du sang du taureau au pied de l'autel de l'holocauste, qui est à l'entrée de la tente d'assignation.
\VS{8}Il enlèvera toute la graisse du taureau du sacrifice pour l'expiation, à savoir, la graisse qui couvre les entrailles, et toute la graisse qui est sur les entrailles,
\VS{9}et les deux rognons avec la graisse qui les entoure, qui couvre les flancs, et il ôtera le grand lobe qui est sur le foie, pour le mettre sur les rognons.
\VS{10}Comme on les enlève du taureau du sacrifice d'offrande de paix\FTNT{Voir commentaire en Lé. 3:1.}, et le prêtre brûlera toutes ces choses-là sur l'autel de l'holocauste.
\VS{11}Mais quant à la peau du taureau et toute sa chair, avec sa tête, ses jambes, ses entrailles, et ses excréments,
\VS{12}et même tout le taureau, il l'emportera hors du camp, dans un lieu pur, où l'on répand les cendres, et il le brûlera au feu sur du bois : Il sera brûlé au lieu où l'on répand les cendres.
\VS{13}Et si toute l'assemblée d'Israël a péché involontairement, et que la chose soit restée cachée aux yeux de l'assemblée, et qu'ils aient violé l'un des commandements de Yahweh, en commettant des choses qui ne doivent pas se faire, et s'en soit rendu coupable,
\VS{14}et que le péché qu'ils ont fait vienne en évidence, l'assemblée offrira en sacrifice pour l'expiation un jeune taureau pris du troupeau, et on l'amènera devant la tente d'assignation.
\VS{15}Les anciens de l'assemblée poseront leurs mains sur la tête du taureau devant Yahweh, et on égorgera le taureau devant Yahweh.
\VS{16}Et le prêtre oint, apportera du sang du taureau dans la tente d'assignation ;
\VS{17}ensuite le prêtre trempera son doigt dans le sang, et en fera aspersion devant Yahweh en face du voile, par sept fois.
\VS{18}Et il mettra du sang sur les cornes de l'autel, qui est devant Yahweh dans la tente d'assignation ; et il répandra tout le reste du sang au pied de l'autel de l'holocauste, qui est à l'entrée de la tente d'assignation.
\VS{19}Il enlèvera toute sa graisse et la brûlera sur l'autel.
\VS{20}Et il fera de ce taureau comme il l'a fait du taureau pour le sacrifice d'expiation. Le prêtre fera ainsi ; il fera propitiation pour eux, et il leur sera pardonné.
\VS{21}Puis il emportera le taureau hors du camp, et le brûlera comme il a brûlé le premier taureau. Car c'est le sacrifice pour l'expiation de l'assemblée.
\VS{22}Que si un chef a péché involontairement, en violant l'un des commandements de Yahweh son Dieu, ce qui ne doit point se faire, et s'en soit rendu coupable,
\VS{23}et qu'on vienne à connaître le péché qu'il a commis, il amènera pour sacrifice un jeune bouc, mâle, sans défaut ;
\VS{24}et il posera sa main sur la tête du bouc, et l'égorgera au lieu où l'on égorge l'holocauste devant Yahweh. C'est un sacrifice pour expiation.
\VS{25}Puis le prêtre prendra avec son doigt du sang de l'offrande pour l'expiation, et le mettra sur les cornes de l'autel de l'holocauste, et il répandra le reste de son sang au pied de l'autel de l'holocauste.
\VS{26}Et il brûlera toute sa graisse sur l'autel, comme la graisse du sacrifice d'offrande de paix. Ainsi le prêtre fera propitiation pour lui de son péché, et il lui sera pardonné.
\VS{27}Que si quelqu'un du peuple du pays a péché involontairement, en violant l'un des commandements de Yahweh, et en commettant des choses qui ne doivent point se faire, et s'en soit rendu coupable,
\VS{28}et qu'on vienne à connaître le péché qu'il a commis, il amènera pour offrande une jeune chèvre, femelle, sans défaut, pour le péché qu'il a commis.
\VS{29}Et il posera sa main sur la tête de l'offrande pour le péché, et égorgera l'offrande pour l'expiation au lieu où l'on égorge l'holocauste.
\VS{30}Puis le prêtre prendra du sang de la chèvre avec son doigt, et le mettra sur les cornes de l'autel de l'holocauste, et il répandra tout le reste de son sang au pied de l'autel.
\VS{31}Et il ôtera toute sa graisse, comme on ôte la graisse de dessus le sacrifice d'offrande de paix, et le prêtre la brûlera sur l'autel, en bonne odeur à Yahweh. Il fera propitiation pour lui, et il lui sera pardonné.
\VS{32}Que s'il amène un agneau comme offrande, pour le sacrifice d'expiation, il amènera une femelle sans défaut.
\VS{33}Et il posera sa main sur la tête de l'offrande d'expiation, et on l'égorgera en sacrifice pour l'expiation au lieu où l'on égorge l'holocauste.
\VS{34}Puis le prêtre prendra avec son doigt du sang de l'offrande pour l'expiation, et le mettra sur les cornes de l'autel de l'holocauste, et il répandra tout le reste de son sang au pied de l'autel.
\VS{35}Et il ôtera toute sa graisse, comme on ôte la graisse de l'agneau du sacrifice d'offrande de paix, et le prêtre la brûlera sur l'autel, par-dessus les sacrifices de Yahweh consumés par le feu, et il fera propitiation pour lui, pour son péché qu'il aura commis, et il lui sera pardonné.
\Chap{5}
\TextTitle{Le sacrifice de culpabilité\FTNTT{Lé. 7:1-7.}}
\VerseOne{}Et quand quelqu'un, étant témoin, après avoir entendu la parole du serment, aura péché en ne déclarant pas ce qu'il a vu ou ce qu'il sait, il portera son iniquité\FTNT{Pr. 29:24.}.
\VS{2}Et quand quelqu'un, à son insu, aura touché une chose souillée, soit le cadavre d'un animal impur, soit le cadavre d'une bête sauvage impure, soit le cadavre d'un reptile impur, il sera souillé et coupable\FTNT{Ag. 2:14 ; 2 Co. 6:17.}.
\VS{3}Ou quand il aura touché à l'impureté d'un homme, quelle que soit son impureté par laquelle il se rend impur, et que cela lui soit resté caché, quand il le sait, alors il est coupable.
\VS{4}Ou quand quelqu'un, parlant légèrement de ses lèvres, a juré de faire du mal ou du bien, selon tout ce que l'homme profère légèrement en jurant, et que cela lui soit resté caché, quand il le sait, alors il est coupable dans l'un de ces points-là.
\VS{5}Quand donc quelqu'un sera coupable sur l'un de ces points là, il confessera ce en quoi il aura péché.
\VS{6}Et il amènera son sacrifice de culpabilité à Yahweh pour le péché qu'il a commis, à savoir, une femelle du menu bétail, soit une brebis, soit une chèvre, pour l'offrande d'expiation. Et le prêtre fera pour lui propitiation de son péché.
\VS{7}Et s'il n'a pas le moyen de trouver une brebis ou une chèvre, il apportera en offrande pour le péché à Yahweh, pour sa culpabilité, deux tourterelles ou deux jeunes pigeons, l'un comme sacrifice pour l'expiation, l'autre pour l'holocauste\FTNT{Lu. 2:24.}.
\VS{8}Il les apportera au prêtre, qui offrira premièrement celui qui est pour l'offrande d'expiation. Il leur ouvrira la tête avec l'ongle, près du cou, sans la séparer ;
\VS{9}puis il fera l'aspersion du sang du sacrifice d'expiation sur un côté de l'autel, et ce qui restera du sang sera exprimé au pied de l'autel : C'est un sacrifice pour l'expiation.
\VS{10}Et il fera de l'autre un holocauste, selon l'ordonnance. Et le prêtre fera pour lui la propitiation pour son péché qu'il aura commis, et il lui sera pardonné.
\VS{11}Si celui qui aura péché n'a pas le moyen de trouver deux tourterelles ou deux jeunes pigeons, il apportera pour son offrande un dixième d'épha de fine farine en offrande pour le sacrifice d'expiation ; il ne mettra ni huile ni encens, car c'est un sacrifice d'expiation.
\VS{12}Il l'apportera au prêtre, et le prêtre qui en prendra une pleine poignée pour souvenir\FTNT{Lé. 2:2.}, la brûlera sur l'autel, comme offrande consumée par le feu à Yahweh : C'est un sacrifice d'expiation.
\VS{13}Ainsi le prêtre fera propitiation pour lui, pour le péché qu'il a commis dans l'une de ces choses, et il lui sera pardonné. Le reste sera pour le prêtre, comme étant une offrande de gâteau.
\VS{14}Yahweh parla aussi à Moïse, en disant :
\VS{15}Quand quelqu'un aura commis une transgression et péchera involontairement, en retenant des choses consacrées à Yahweh, il amènera en sacrifice de culpabilité à Yahweh, à savoir un bélier sans défaut, pris du troupeau, avec l'estimation que tu feras de la chose sainte, la faisant en sicles d'argent, selon le sicle du sanctuaire, à cause de son péché.
\VS{16}Il restituera donc ce en quoi il aura péché en retenant de la chose sainte et il y ajoutera un cinquième par dessus, et le donnera au prêtre ; et le prêtre fera propitiation pour lui, par le bélier du sacrifice de culpabilité, et il lui sera pardonné.
\VS{17}Lorsque quelqu'un aura péché, en violant, sans le savoir, l'un des commandements de Yahweh, des choses qu'on ne doit point faire, il sera coupable et portera son iniquité.
\VS{18} Il amènera donc en sacrifice de culpabilité au prêtre un bélier sans tâche, pris du troupeau, avec l'estimation que tu feras du péché involontaire ; et le prêtre fera propitiation pour lui du péché involontaire qu'il a commis et dont il ne se sera point aperçu ; et ainsi il lui sera pardonné.
\VS{19}C'est un sacrifice de culpabilité. Il s'est rendu coupable contre Yahweh.
\TextTitle{La restitution au jour du sacrifice de culpabilité\FTNTT{Lé. 7:1-7.}}
\VS{20}Yahweh parla aussi à Moïse, en disant :
\VS{21}Quand quelqu'un aura péché et aura commis une transgression contre Yahweh, en mentant à son prochain pour un dépôt, pour une chose qu'on aura mise entre ses mains, un vol, ou qu'il ait extorqué son prochain,
\VS{22}ou s'il a trouvé quelque chose perdue, et qu'il mente à ce sujet, ou s'il jure faussement concernant l'une des choses qu'un homme fait en péchant ;
\VS{23}quand il péchera et se rendra coupable, il rendra la chose qu'il a volée ou extorquée, ou le dépôt qui lui a été donné en garde, ou la chose perdue qu'il a trouvée,
\VS{24}ou tout ce dont il aura juré faussement. Il le restituera totalement, et il y ajoutera un cinquième ; il le donnera à celui à qui il appartenait, le jour de son sacrifice de culpabilité.
\VS{25}Et il amènera pour Yahweh, au prêtre le sacrifice de culpabilité, à savoir un bélier sans défaut, pris du troupeau, avec l'estimation que tu feras de la culpabilité.
\VS{26}Et le prêtre fera propitiation pour lui devant Yahweh, et il lui sera pardonné, quelle que soit la faute dont il se sera rendu coupable. 
\Chap{6}
\TextTitle{Loi de l'holocauste\FTNTT{Lé. 1:1-17.}}
\VerseOne{}Yahweh parla aussi à Moïse, en disant :
\VS{2}Ordonne à Aaron et à ses fils, et dis-leur : C'est ici la loi de l'holocauste. L'holocauste demeurera sur le foyer de l'autel toute la nuit jusqu'au matin, et le feu brûlera sur l'autel.
\VS{3}Et le prêtre revêtira sa tunique de lin, mettra ses caleçons de lin sur son corps, et il enlèvera la cendre de l'holocauste que le feu aura consumé sur l'autel, puis il la mettra près de l'autel.
\VS{4}Alors il ôtera ses vêtements et portera d'autres vêtements pour transporter les cendres hors du camp, dans un lieu pur.
\VS{5}Et quant au feu qui brûle sur l'autel, il continuera de brûler, on ne l'éteindra point ; le prêtre y brûlera du bois tous les matins, il préparera l'holocauste sur le bois, et y brûlera les graisses des offrandes de paix.
\VS{6}Le feu brûlera continuellement sur l'autel, on ne le laissera point s'éteindre.
\TextTitle{Loi de l'offrande de gâteau\FTNTT{Lé. 2:1-16.}}
\VS{7}Et c'est ici la loi de l'offrande de gâteau. Les fils d'Aaron l'offriront devant Yahweh sur l'autel\FTNT{No. 15:4.}.
\VS{8}Et on lèvera une poignée de la fine farine du gâteau et de son huile, avec tout l'encens qui est sur le gâteau, et on le brûlera sur l'autel, en bonne odeur, en mémorial à Yahweh.
\VS{9}Aaron et ses fils mangeront ce qui en restera ; ils le mangeront sans levain dans un lieu saint, ils le mangeront dans le parvis de la tente d'assignation\FTNT{Ex. 29:26-37.}.
\VS{10}On ne le cuira point avec du levain. Je leur ai donné cela pour leur portion d'entre mes offrandes consumées par le feu. C'est une chose très sainte, comme le sacrifice d'expiation et le sacrifice de culpabilité.
\VS{11}Tout mâle d'entre les fils d'Aaron en mangera. C'est une ordonnance perpétuelle pour vos descendants concernant les offrandes consumées par le feu à Yahweh : Quiconque les touchera sera sanctifié.
\VS{12}Yahweh parla aussi à Moïse, en disant :
\VS{13}C'est ici l'offrande d'Aaron et de ses fils, qu'ils offriront à Yahweh le jour où il sera oint : Un dixième d'épha de fine farine, comme offrande de gâteau perpétuelle, une moitié le matin et une moitié le soir.
\VS{14}Elle sera apprêtée sur une plaque avec de l'huile, tu l'apporteras mélangée, et tu offriras les morceaux cuits du gâteau en bonne odeur à Yahweh.
\VS{15}Et le prêtre, d'entre ses fils, qui sera oint à sa place, fera cela. C'est une ordonnance perpétuelle devant Yahweh : On le brûlera tout entier.
\VS{16}Tout le gâteau du prêtre sera entièrement consumé ; on n'en mangera pas.
\TextTitle{Loi de l'offrande pour le péché\FTNTT{Lé. 4:1-35.}}
\VS{17}Yahweh parla aussi à Moïse, en disant :
\VS{18}Parle à Aaron et à ses fils, et dis-leur : C'est ici la loi du sacrifice d'expiation. L'offrande pour l'expiation sera égorgée devant Yahweh, dans le même lieu où l'on égorge l'holocauste : C'est une chose très sainte.
\VS{19}Le prêtre qui offrira l'offrande pour l'expiation la mangera ; elle se mangera dans un lieu saint, dans le parvis de la tente d'assignation\FTNT{No. 18:10.}.
\VS{20}Quiconque touchera sa chair sera saint. Et s'il en jaillit du sang sur le vêtement, ce sur quoi il aura jailli sera lavé dans un lieu saint.
\VS{21}Et le vase de terre dans lequel on l'aura fait cuire sera brisé ; mais si on l'a fait cuire dans un vase d'airain, il sera nettoyé et lavé dans l'eau.
\VS{22}Tout mâle d'entre les prêtres en mangera ; car c'est une chose très sainte.
\VS{23}Aucune offrande pour le sacrifice d'expiation, dont on portera le sang dans la tente d'assignation pour faire la propitiation dans le sanctuaire, ne sera mangée, mais elle sera brûlée au feu\FTNT{Hé. 13:11.}.
\Chap{7}
\TextTitle{Loi du sacrifice de culpabilité\FTNTT{Lé. 5:1-26.}}
\VerseOne{}Or c'est ici la loi du sacrifice de culpabilité : C'est une chose très sainte.
\VS{2}Au même lieu où l'on égorgera l'holocauste, on égorgera le sacrifice de culpabilité. On en répandra le sang sur l'autel tout autour.
\VS{3}Puis on en offrira toute la graisse, avec la queue, et toute la graisse qui couvre les entrailles,
\VS{4}les deux rognons, la graisse qui est dessus sur les flancs, et le grand lobe qui est sur le foie, qu'on ôtera jusqu'aux rognons.
\VS{5}Le prêtre brûlera toutes ces choses sur l'autel comme offrande consumée par le feu à Yahweh : C'est un sacrifice pour la culpabilité.
\VS{6}Tout mâle d'entre les prêtres en mangera ; il sera mangé dans un lieu saint ; car c'est une chose très sainte.
\VS{7}Le sacrifice pour l'expiation sera semblable au sacrifice de culpabilité, il y aura une même loi pour les deux ; et la victime appartiendra au prêtre qui aura fait propitiation par elle.
\VS{8}Et le prêtre qui offrira l'holocauste de quelqu'un aura la peau de l'holocauste qu'il aura offert.
\VS{9}Et toute offrande de gâteau cuit au four, apprêtée sur le gril ou sur la plaque, appartiendra au prêtre qui l'offre.
\VS{10}Et toute offrande pétrie à l'huile, ou sèche, sera pour tous les fils d'Aaron, pour l'un comme pour l'autre.
\TextTitle{Loi du sacrifice d'offrande de paix\FTNTT{Lé. 3:1-17.}} 
\VS{11}Et c'est ici, la loi du sacrifice d'offrande de paix\FTNT{Voir commentaire en Lé. 3:1.} qu'on offrira à Yahweh.
\VS{12}Si quelqu'un l'offre pour un sacrifice de reconnaissance, il offrira avec le sacrifice de reconnaissance, des gâteaux sans levain pétris à l'huile, des galettes sans levain ointes d'huile, et des gâteaux de fine farine mêlés et pétris à l'huile.
\VS{13}En plus des gâteaux, il offrira pour son offrande du pain levé avec le sacrifice de reconnaissance de ses offrandes de paix.
\VS{14}Il présentera une part de chaque offrande, qu'il offrira comme offrande élevée à Yahweh ; elle sera pour le prêtre qui a répandu le sang du sacrifice d'offrande de paix.
\VS{15}Mais la chair du sacrifice de reconnaissance de ses offrandes de paix sera mangée le jour où elle sera offerte ; on n'en laissera rien jusqu'au matin.
\VS{16}Que si le sacrifice de son offrande est un vœu ou une offrande volontaire, son sacrifice sera mangé le jour où il l'aura offert ; ce qui en restera sera mangé le lendemain.
\VS{17}Mais ce qui restera de la chair du sacrifice sera brûlé au feu le troisième jour.
\VS{18}Que si on mange de la chair du sacrifice d'offrande de paix le troisième jour, celui qui l'aura offert ne sera point agréé, il ne lui sera point imputé, ce sera une chose infâme, et la personne qui en mangera portera son iniquité\FTNT{Ez. 4:14.}.
\VS{19}Et la chair de ce sacrifice qui a touché quelque chose d'impure ne sera point mangée, elle sera brûlée au feu. Mais quiconque sera pur, mangera de cette chair.
\VS{20}Car une personne qui mangera de la chair du sacrifice d'offrande de paix, laquelle appartient à Yahweh, et qui aura sur elle son impureté, cette personne-là sera retranchée de son peuple.
\VS{21}Si une personne touche quelque chose d'impure, soit une impureté d'homme, soit une bête impure, ou quelque autre chose impure, et qu'il mange de la chair du sacrifice d'offrande de paix qui appartient à Yahweh, cette personne-là sera retranchée d'entre son peuple.
\VS{22}Yahweh parla à Moïse, en disant :
\VS{23}Parle aux enfants d'Israël, et dis-leur : Vous ne mangerez aucune graisse de bœuf, ni d'agneau, ni de chèvre.
\VS{24}On pourra se servir pour un usage quelconque de la graisse d'une bête morte ou de la graisse d'une bête déchirée ; mais vous n'en mangerez point.
\VS{25}Car quiconque mangera de la graisse d'une bête que l'on offre comme offrande consumée par le feu à Yahweh, la personne qui en mangera, sera retranché de son peuple.
\VS{26}Vous ne mangerez point de sang, ni d'oiseaux, ni d'autres bêtes, dans aucune de vos demeures.
\VS{27}Toute personne, qui aura mangé de quelque sang que ce soit, sera retranchée de son peuple.
\VS{28}Yahweh parla à Moïse, en disant :
\VS{29}Parle aux enfants d'Israël, et dis-leur : Celui qui offrira son sacrifice d'offrande de paix à Yahweh, apportera son offrande à Yahweh, prise sur son sacrifice d'offrande de paix.
\VS{30}Il apportera de ses mains les offrandes consumées par le feu devant Yahweh. Il apportera la graisse avec la poitrine, la poitrine pour l'agiter d'un côté et de l'autre devant Yahweh.
\VS{31}Puis le prêtre brûlera la graisse sur l'autel, mais la poitrine sera pour Aaron et ses fils.
\VS{32}Vous donnerez aussi au prêtre pour offrande élevée, l'épaule droite de vos sacrifices d'offrande de paix\FTNT{No. 18:18.}.
\VS{33}Celui des fils d'Aaron qui offrira le sang et la graisse de l'offrande de paix, aura pour sa part l'épaule droite.
\VS{34}Car je prends sur les enfants d'Israël, la poitrine qu'on agite d'un côté et de l'autre, et l'épaule qu'on présente par élévation, de tous les sacrifices d'offrande de paix, et je les donne à Aaron le prêtre et à ses fils, par une ordonnance perpétuelle, de la part des fils d'Israël.
\VS{35}C'est là, le droit de l'onction d'Aaron et de l'onction de ses fils sur ces offrandes consumées par le feu devant Yahweh, depuis le jour où on les aura présentés pour exercer la sacrificature à Yahweh.
\VS{36}Et c'est ce que Yahweh ordonne aux enfants d'Israël de leur donner, depuis le jour où on les aura oints ; par une loi perpétuelle parmi leurs descendants\FTNT{Ex. 40:15.}.
\VS{37}Telle est donc la loi de l'holocauste, du gâteau, du sacrifice pour l'expiation, du sacrifice pour la culpabilité, de la consécration et du sacrifice d'offrande de paix.
\VS{38}Yahweh l'ordonna à Moïse sur la montagne de Sinaï, le jour où il ordonna aux enfants d'Israël d'offrir leurs offrandes à Yahweh dans le désert de Sinaï.
\Chap{8}
\TextTitle{Consécration d'Aaron et ses fils}
\VerseOne{}Yahweh parla aussi à Moïse en disant :
\VS{2}Prends Aaron et ses fils avec lui, les vêtements, l'huile d'onction, un jeune taureau pour le sacrifice d'expiation, deux béliers et une corbeille de pains sans levain\FTNT{Ex. 29:1-2 ; Ex. 30:25.} ;
\VS{3}et convoque toute l'assemblée à l'entrée de la tente d'assignation.
\VS{4}Et Moïse fit comme Yahweh lui avait ordonné ; et l'assemblée se rassembla à l'entrée de la tente d'assignation.
\VS{5}Moïse dit à l'assemblée : Voici ce que Yahweh a ordonné de faire.
\TextTitle{La purification avec l'eau} 
\VS{6}Et Moïse fit approcher Aaron et ses fils, et les lava avec de l'eau.
\TextTitle{Les vêtements d'Aaron} 
\VS{7}Et il mit sur Aaron la tunique, il le ceignit de la ceinture, le revêtit de la robe, mit sur lui l'éphod, et le ceignit avec la ceinture de l'éphod dont il le lia.
\VS{8}Puis il mit sur lui le pectoral, après avoir mis au pectoral l'urim et le thummim.
\VS{9}Il lui mit aussi la tiare sur la tête, et il mit sur le devant de la tiare la lame d'or, la couronne de sainteté, comme Yahweh l'avait ordonné à Moïse\FTNT{Ex. 28.}.
\TextTitle{L'onction d'huile} 
\VS{10}Puis Moïse prit l'huile d'onction, et oignit le tabernacle et toutes les choses qui y étaient, et les sanctifia.
\VS{11}Et il en fit l'aspersion sur l'autel par sept fois, et il oignit l'autel, tous ses ustensiles, et la cuve avec sa base, pour les sanctifier.
\VS{12}Il versa aussi de l'huile d'onction sur la tête d'Aaron, et l'oignit pour le sanctifier\FTNT{Ps. 133:2.}.
\TextTitle{Les vêtements des fils d'Aaron}
\VS{13}Puis Moïse fit approcher les fils d'Aaron, les revêtit des tuniques, les ceignit des ceintures et leur attacha des turbans, comme Yahweh l'avait ordonné à Moïse.
\TextTitle{Les offrandes et les sacrifices}
\VS{14}Alors il fit approcher le jeune taureau pour le sacrifice d'expiation, et Aaron et ses fils posèrent leurs mains sur la tête du taureau pour le sacrifice d'expiation.
\VS{15}Et Moïse l'égorgea, prit de son sang, et en mit avec son doigt sur les cornes de l'autel tout autour, et purifia l'autel ; et il répandit le reste du sang au pied de l'autel, ainsi il le sanctifia pour faire la propitiation sur lui.
\VS{16}Puis il prit toute la graisse qui était sur les entrailles, le grand lobe du foie, les deux rognons avec leur graisse, et Moïse les brûla sur l'autel.
\VS{17}Mais il brûla au feu, hors du camp, le jeune taureau avec sa peau, sa chair, et ses excréments, comme Yahweh l'avait ordonné à Moïse.
\VS{18}Il fit aussi approcher le bélier de l'holocauste, et Aaron et ses fils posèrent leurs mains sur la tête du bélier.
\VS{19}Et Moïse l'égorgea et répandit le sang sur l'autel tout autour.
\VS{20}Puis il coupa le bélier en morceaux, et Moïse brûla la tête, les morceaux, et la graisse.
\VS{21}Et il lava dans l'eau les entrailles et les jambes, et brûla tout le bélier sur l'autel : Ce fut un holocauste d'une agréable odeur, c'était une offrande consumée par le feu à Yahweh, comme Yahweh l'avait ordonné à Moïse.
\VS{22}Il fit aussi approcher l'autre bélier, le bélier de consécration, et Aaron et ses fils posèrent les mains sur la tête du bélier.
\VS{23}Et Moïse l'égorgea, prit de son sang, et le mit sur le lobe de l'oreille droite d'Aaron, et sur le pouce de sa main droite et sur le gros orteil de son pied droit.
\VS{24}Il fit aussi approcher les fils d'Aaron, et mit du même sang sur le lobe de leur oreille droite, et sur le pouce de leur main droite, et sur le gros orteil de leur pied droit, et Moïse répandit le reste du sang sur l'autel tout autour.
\VS{25}Après, il prit la graisse, la queue, toute la graisse qui est sur les entrailles, et le grand lobe du foie, et les deux rognons avec leur graisse, et l'épaule droite.
\VS{26}Il prit aussi de la corbeille des pains sans levain, qui étaient devant Yahweh, un gâteau sans levain, et un gâteau de pain fait à l'huile et une galette, et il les mit sur les graisses, et sur l'épaule droite.
\VS{27}Puis il mit toutes ces choses sur les paumes des mains d'Aaron et sur les paumes des mains de ses fils, et les agita d'un côté et de l'autre devant Yahweh.
\VS{28}Puis Moïse les prit de leurs mains et les brûla sur l'autel, sur l'holocauste : Ce fut l'offrande de consécration de bonne odeur, c'est une offrande consumée par le feu devant Yahweh.
\VS{29}Moïse prit aussi la poitrine du bélier de consécration, et l'agita d'un côté et de l'autre devant Yahweh : Ce fut la part de Moïse, comme Yahweh l'avait ordonné à Moïse.
\TextTitle{L'aspersion d'huile et de sang}
\VS{30}Moïse prit de l'huile d'onction et du sang qui était sur l'autel, et il en fit l'aspersion sur Aaron et sur ses vêtements, sur ses fils et sur les vêtements de ses fils ; ainsi il sanctifia Aaron et ses vêtements, les fils d'Aaron et les vêtements de ses fils.
\TextTitle{La nourriture de consécration\FTNTT{Ex. 29:26 ; Lé. 7:31-34 ; 8:29.}}
\VS{31}Après cela, Moïse dit à Aaron et à ses fils : Faites cuire la chair à l'entrée de la tente d'assignation, et vous la mangerez là, avec le pain qui est dans la corbeille de consécration, comme je l'ai ordonné, en disant : Aaron et ses fils la mangeront.
\VS{32}Mais vous brûlerez au feu ce qui restera de la chair et du pain.
\TextTitle{Les prêtres mis à part}
\VS{33}Et vous ne sortirez point pendant sept jours, de l'entrée de la tente d'assignation, jusqu'à ce que vos jours de consécration soient accomplis ; car on emploiera sept jours à vous consacrer.
\VS{34}Yahweh a ordonné de faire en ces autres jours comme on a fait en celui-ci, pour faire la propitiation en votre faveur.
\VS{35}Vous resterez donc pendant sept jours à l'entrée de la tente d'assignation, jour et nuit, et vous observerez ce que Yahweh vous a ordonné d'observer, afin que vous ne mouriez pas ; car il m'a été ainsi ordonné.
\VS{36}Ainsi Aaron et ses fils firent toutes les choses que Yahweh avait ordonnées par Moïse.
\Chap{9}
\TextTitle{Aaron et ses fils commencent leur service dans le tabernacle}
\VerseOne{}Et il arriva au huitième jour, que Moïse appela Aaron et ses fils, et les anciens d'Israël.
\VS{2}Et il dit à Aaron : Prends un jeune taureau du troupeau pour l'offrande d'expiation, et un bélier pour l'holocauste, tous deux sans défaut, et offre-les devant Yahweh.
\VS{3}Et tu parleras aux enfants d'Israël, en disant : Prenez un bouc pour l'offrande d'expiation, un jeune taureau et un agneau, tous deux d'un an et sans défaut, pour l'holocauste ;
\VS{4}un bœuf et un bélier pour l'offrande de paix\FTNT{Voir commentaire en Lé. 3:1.}, pour les sacrifier devant Yahweh ; et un gâteau pétri à l'huile. Car aujourd'hui Yahweh vous apparaîtra.
\VS{5}Ils prirent donc les choses que Moïse avait ordonné et les amenèrent devant la tente d'assignation, et toute l'assemblée s'approcha, et se tint devant Yahweh.
\VS{6}Et Moïse dit : Faites ce que Yahweh vous a ordonné, et la gloire de Yahweh vous apparaîtra.
\VS{7}Moïse dit à Aaron : Approche-toi de l'autel, fais ton sacrifice pour l'expiation et ton holocauste, et fais propitiation pour toi et pour le peuple ; présente l'offrande pour le peuple, et fais propitiation pour eux, comme Yahweh l'a ordonné\FTNT{Hé. 7:26-27.}.
\VS{8}Alors Aaron s'approcha de l'autel et égorgea le veau de son sacrifice d'expiation.
\VS{9}Et les fils d'Aaron lui présentèrent le sang, et il trempa son doigt dans le sang et le mit sur les cornes de l'autel ; puis il répandit le reste du sang au pied de l'autel.
\VS{10}Mais il brûla sur l'autel la graisse, et les rognons, et le grand lobe du foie de l'offrande pour le péché, comme Yahweh l'avait ordonné à Moïse.
\VS{11}Et il brûla au feu la chair et la peau hors du camp.
\VS{12}Il égorgea aussi l'holocauste. Les fils d'Aaron lui présentèrent le sang, lequel il répandit sur l'autel tout autour.
\VS{13}Puis ils lui présentèrent l'holocauste coupé en morceaux, avec la tête, et il les brûla sur l'autel.
\VS{14}Et il lava les entrailles et les jambes, qu'il brûla sur l'holocauste, sur l'autel.
\VS{15}Il offrit l'offrande du peuple. Il prit le bouc pour le sacrifice d'expiation du peuple, il l'égorgea et l'offrit pour le péché, comme la première offrande.
\VS{16}Il l'offrit en holocauste, faisant selon l'ordonnance.
\VS{17}Ensuite, il offrit l'offrande du gâteau, et il en remplit la paume de sa main, et la brûla sur l'autel, outre l'holocauste du matin.
\VS{18}Il égorgea aussi le bœuf et le bélier pour le sacrifice d'offrande de paix, qui était pour le peuple. Les fils d'Aaron lui présentèrent le sang, lequel il répandit sur l'autel tout autour.
\VS{19}Ils présentèrent la graisse du bœuf et du bélier, la queue, ce qui couvre les entrailles, les rognons, et le grand lobe du foie ;
\VS{20}ils mirent les graisses sur les poitrines, et il brûla les graisses sur l'autel.
\VS{21}Et Aaron agita d'un côté et de l'autre devant Yahweh les poitrines et l'épaule droite, comme Yahweh l'avait ordonné à Moïse.
\VS{22}Aaron éleva aussi ses mains vers le peuple, et le bénit. Puis il descendit, après avoir offert le sacrifice pour l'expiation, l'holocauste et l'offrande de paix.
\VS{23}Moïse donc et Aaron entrèrent dans la tente d'assignation, puis ils sortirent et ils bénirent le peuple. Et la gloire de Yahweh apparut à tout le peuple.
\VS{24}Car le feu sortit de devant Yahweh, et consuma sur l'autel l'holocauste et les graisses. Tout le peuple le vit et ils poussèrent des cris de joie et tombèrent sur leur face\FTNT{1 R. 18:38 ; 2 Ch. 7:1.}.
\Chap{10}
\TextTitle{Un feu étranger présenté à Yahweh}
\VerseOne{}Or les fils d'Aaron, Nadab et Abihu, prirent chacun leur encensoir, mirent du feu, et ils posèrent dessus du parfum ; ils offrirent devant Yahweh un feu étranger\FTNT{Ce passage nous avertit du danger auquel s'exposent ceux qui apportent un feu étranger dans le temple. Les feux étrangers sont les fausses doctrines, le péché, les conceptions cartésiennes, pernicieuses, mercantiles, destinés à remplacer la Parole de Dieu et à conduire le chrétien dans les ténèbres.}, ce qu'il ne leur avait point été ordonné.
\VS{2}Et le feu sortit de devant Yahweh, et les dévora ; ils moururent devant Yahweh\FTNT{No. 3:4.}.
\VS{3}Moïse dit à Aaron : C'est ce dont Yahweh avait parlé, en disant : Je serai sanctifié par ceux qui s'approchent de moi, et je serai glorifié en présence de tout le peuple. Et Aaron se tut.
\VS{4}Et Moïse appela Mischaël et Eltsaphan, les fils d'Uziel, oncle d'Aaron, et leur dit : Approchez-vous, emportez vos frères de devant le sanctuaire, hors du camp.
\VS{5}Alors ils s'approchèrent et les emportèrent avec leurs tuniques hors du camp, comme Moïse l'avait dit.
\TextTitle{Instructions données par Moïse} 
\VS{6}Puis Moïse dit à Aaron, à Eléazar et à Ithamar, ses fils : Ne découvrez point vos têtes, et ne déchirez point vos vêtements, de peur que vous ne mouriez, et que Yahweh ne se mette en colère contre toute l'assemblée. Mais que vos frères, toute la maison d'Israël, pleurent à cause de l'embrasement que Yahweh a allumé\FTNT{Ez. 24:17.}.
\VS{7}Et ne sortez point de l'entrée de la tente d'assignation, de peur que vous ne mouriez, car l'huile de l'onction de Yahweh est sur vous. Et ils firent selon la parole de Moïse.
\VS{8}Et Yahweh parla à Aaron, en disant :
\VS{9}Vous ne boirez point de vin, ni de boisson forte, ni toi ni tes fils avec toi, quand vous entrerez dans la tente d'assignation, de peur que vous ne mouriez ; c'est une ordonnance perpétuelle pour vos descendants\FTNT{No. 6:3 ; Jg. 13:7.},
\VS{10}afin que vous puissiez discerner entre ce qui est saint et ce qui est profane, entre ce qui est impur et ce qui est pur,
\VS{11}afin que vous enseigniez aux enfants d'Israël toutes les ordonnances que Yahweh leur a prononcées par Moïse.
\VS{12}Puis Moïse parla à Aaron, à Eléazar et à Ithamar, ses fils qui lui restaient : Prenez l'offrande de gâteau, leur dit-il, ce qui reste des offrandes de Yahweh consumées par le feu, et mangez-la avec des pains sans levain auprès de l'autel, car c'est une chose très sainte.
\VS{13}Vous la mangerez dans un lieu saint, parce que c'est la portion qui est assignée à toi et à tes fils sur les offrandes consumées par le feu à Yahweh ; car il m'a été ainsi ordonné.
\VS{14}Vous mangerez aussi la poitrine offerte par agitation et l'épaule présentée par élévation dans un lieu pur, toi, tes fils et tes filles avec toi ; car ces choses-là t'ont été données, dans les sacrifices d'offrande de paix\FTNT{Voir commentaire en Lé. 3:1.} des enfants d'Israël, comme ton droit et le droit de tes fils.
\VS{15}Ils apporteront l'épaule présentée par élévation et la poitrine offerte par agitation, avec les offrandes consumées par le feu, qui sont les graisses, pour les agiter en offrande çà et là devant Yahweh : Cela t'appartiendra, et à tes fils avec toi, par une ordonnance perpétuelle, comme Yahweh l'a ordonné.
\VS{16}Or Moïse cherchait soigneusement le bouc de l'offrande pour l'expiation mais voici, il avait été brûlé. Et Moïse se mit en grande colère contre Eléazar et Ithamar, les fils d'Aaron qui lui restaient, et leur dit :
\VS{17}Pourquoi n'avez-vous point mangé l'offrande pour l'expiation dans un lieu saint ? Car c'est une chose très sainte ; vu qu'elle vous a été donnée pour porter l'iniquité de l'assemblée, afin de faire propitiation pour eux devant Yahweh.
\VS{18}Voici, son sang n'a point été porté dans l'intérieur du sanctuaire ; ne manquez donc plus à la manger dans le lieu saint, comme je l'avais ordonné.
\VS{19}Alors Aaron répondit à Moise : Voici, ils ont offert aujourd'hui leur offrande pour l'expiation et leur holocauste devant Yahweh, et ces choses-ci me sont arrivées. Si j'avais mangé aujourd'hui l'offrande pour le péché, cela aurait-il plu à Yahweh ?
\VS{20}Et Moïse l'entendit, et cela fut bon à ses yeux.
\Chap{11}
\TextTitle{Lois de purification : les bêtes pures et impures}
\VerseOne{}Et Yahweh parla à Moïse et à Aaron, et leur dit :
\VS{2}Parlez aux enfants d'Israël, et dites-leur : Ce sont ici les bêtes dont vous mangerez d'entre toutes les bêtes qui sont sur la terre\FTNT{De. 14:4 ; Ac. 10:11-14.}.
\VS{3}Vous mangerez d'entre les bêtes de tous ceux qui ont le sabot fendu, qui ont le pied fourchu, et qui ruminent.
\VS{4}Mais vous ne mangerez point de celles qui ruminent uniquement, ou qui ont uniquement le sabot fendu : Comme le chameau, car il rumine mais il n'a point le sabot fendu : Il vous sera impur.
\VS{5}Et le lapin, car il rumine mais il n'a point le sabot fendu : Il vous sera impur.
\VS{6}Le lièvre, car il rumine mais il n'a point le sabot fendu : Il vous sera impur.
\VS{7}Le porc, car il a bien le sabot fendu et le pied fourchu, mais il ne rumine pas : Il vous sera impur.
\VS{8}Vous ne mangerez point de leur chair, même vous ne toucherez point leur cadavre : Ils vous seront impurs.
\VS{9}Vous mangerez de ceci d'entre tout ce qui est dans les eaux. Vous mangerez de tout ce qui a des nageoires et des écailles dans les eaux, soit dans la mer, soit dans les fleuves.
\VS{10}Mais vous ne mangerez rien de ce qui n'a point de nageoires et d'écailles, soit dans la mer, soit dans les fleuves, tant des reptiles des eaux, que de toute chose vivante qui est dans les eaux, cela vous sera en abomination.
\VS{11}Elles vous seront donc en abomination, vous ne mangerez point de leur chair, et vous tiendrez pour une chose abominable leur cadavre.
\VS{12}Tout ce donc qui vit dans les eaux et qui n'a point de nageoires et d'écailles, vous sera en abomination.
\VS{13}Et d'entre les oiseaux vous tiendrez ceux-ci pour abominables, on n'en mangera point, ils vous seront en abomination : L'aigle, l'orfraie, l'aigle de mer ;
\VS{14}le vautour, et le milan, selon leur espèce ;
\VS{15}tout corbeau, selon son espèce ;
\VS{16}l'autruche, le hibou, la mouette, et l'épervier selon leur espèce ;
\VS{17}le chat-huant, le plongeon, la chouette ;
\VS{18}le cygne, le cormoran, le pélican ;
\VS{19}la cigogne, le héron selon leur espèce, la huppe et la chauve-souris,
\VS{20}et tout reptile volant qui marche sur quatre pattes vous sera en abomination.
\VS{21}Mais, vous pourrez manger de toute chose rampante qui vole et qui va sur quatre pattes qui ont des jambes au-dessus de leurs pieds, pour sauter avec celles-ci sur la terre.
\VS{22}Ce sont donc ici ceux dont vous mangerez : La sauterelle selon son espèce, le solam\FTNT{« Solam », « hargol » et « hagab » sont diverses espèces de sauterelles.} selon son espèce, le hargol, selon son espèce et le hagab, selon son espèce.
\VS{23}Mais tout autre reptile volant qui a quatre pattes vous sera en abomination.
\VS{24}Vous serez donc impurs par ces bêtes ; quiconque touchera leur cadavre sera impur jusqu'au soir,
\VS{25}et quiconque aussi portera leur cadavre lavera ses vêtements et sera impur jusqu'au soir.
\VS{26}Toute bête qui a le sabot fendu, et qui n'a point le pied fourchu et ne rumine point, vous sera impur : Quiconque les touchera sera impur.
\VS{27}Tout ce qui marche sur ses pattes, entre tous les animaux qui marchent à quatre pieds, vous sera impur : Quiconque touchera leur cadavre sera impur jusqu'au soir,
\VS{28}et celui qui portera leur cadavre lavera ses vêtements et sera impur jusqu'au soir. Ils vous seront impurs.
\VS{29}Ceci aussi vous sera impur entre les reptiles, qui rampent sur la terre : La taupe, la souris et la tortue, selon leur espèce ;
\VS{30}le hérisson, la grenouille, le lézard, la limace et le caméléon.
\VS{31}Ces choses vous seront impures entre les reptiles : Quiconque les touchera mortes sera impur jusqu'au soir.
\VS{32}Aussi, tout ce sur quoi il en tombera quelque chose quand elles seront mortes sera impur, soit ustensile de bois, soit vêtement, soit peau, ou sac, quelque objet que ce soit dont on se sert pour faire quelque chose ; il sera mis dans l'eau, et sera impur jusqu'au soir ; puis il sera pur.
\VS{33}Mais s'il en tombe quelque chose dans quelque vase de terre que ce soit, tout ce qui est dedans sera impur, et vous casserez le vase.
\VS{34}Et tout aliment qu'on mange, sur lequel il y aura eu de cette eau, sera impur ; tout breuvage qu'on boit dans quelque vase que ce soit, en sera impur.
\VS{35}Et s'il tombe quelque chose de leur cadavre sur quoi que ce soit, cela sera impur ; le four et le foyer seront détruits : Ils seront impurs, et ils vous seront impurs.
\VS{36}Toutefois, la source, le puits ou tel autre amas d'eaux resteront purs ; mais celui donc qui touchera leur cadavre sera impur.
\VS{37}Et s'il est tombé de leur cadavre sur quelque semence qui se sème, elle restera pure.
\VS{38}Mais si on avait mis de l'eau sur la semence, et que quelque chose de leur cadavre tombe sur elle, elle vous sera impure.
\VS{39}Et si une des bêtes qui vous servent pour nourriture meurt, celui qui en touchera le cadavre sera impur jusqu'au soir ;
\VS{40}celui qui mangera de son cadavre lavera ses vêtements et sera impur jusqu'au soir, et celui aussi qui portera le cadavre de cette bête, lavera ses vêtements et sera impur jusqu'au soir.
\VS{41}Tout reptile donc qui rampe sur la terre vous sera en abomination ; et on n'en mangera point\FTNT{cp. Ge. 3:14.}.
\VS{42}Vous ne mangerez point de tout ce qui rampe sur la poitrine, ni de tout ce qui marche sur les quatre pieds, ni de tout ce qui a plusieurs pieds entre tous les reptiles qui se traînent sur la terre ; car ils seront en abomination.
\VS{43}Ne rendez point vos personnes abominables par aucun reptile qui se traîne ; ne vous rendez point impurs par eux, ne vous souillez point par eux.
\VS{44}Car je suis Yahweh, votre Dieu ; vous vous sanctifierez donc et vous serez saints, car je suis saint\FTNT{1 Pi. 1:16.} ! Ainsi, vous ne rendrez point vos personnes impures par aucun reptile qui se traîne sur la terre.
\VS{45}Car je suis Yahweh, qui vous ai fait monter du pays d'Egypte, afin que je sois votre Dieu, et que vous soyez saints ; car je suis saint !
\VS{46}Telle est la loi touchant les animaux, les oiseaux, tout être vivant, qui se meut dans les eaux, et toute être vivant, qui se rampe sur la terre,
\VS{47}afin de discerner entre la chose impure et la chose pure, entre les animaux qu'on peut manger et les animaux dont on ne doit point manger.
\Chap{12}
\TextTitle{Lois de purification : Le flux de sang\FTNTT{Ps. 51:7.}}
\VerseOne{}Yahweh parla aussi à Moïse, en disant :
\VS{2}Parle aux enfants d'Israël, et dis-leur : Si la femme après avoir conçu, enfante un mâle, elle sera impure pendant sept jours ; elle sera impure comme au temps de son indisposition menstruelle.
\VS{3}Et au huitième jour, on circoncira la chair du prépuce de l'enfant\FTNT{Les parents de Jésus ont observé cette loi (Lu. 2:21-24). Jn. 7:22.}.
\VS{4}Et elle demeurera trente-trois jours à se purifier de son sang ; elle ne touchera aucune chose sainte, et ne viendra point au sanctuaire, jusqu'à ce que les jours de sa purification soient accomplis.
\VS{5}Si elle enfante une fille, elle sera impure deux semaines, comme au temps de son indisposition menstruelle, et elle restera soixante-six jours à se purifier de son sang.
\VS{6}Après que le temps de sa purification sera accompli, soit pour un fils ou pour une fille, elle présentera au prêtre un agneau d'un an en holocauste, et un jeune pigeon ou une tourterelle en sacrifice d'expiation, à l'entrée de la tente d'assignation\FTNT{No. 6:10.}.
\VS{7}Et le prêtre offrira ces choses devant Yahweh, et fera propitiation pour elle ; et elle sera purifiée du flux de son sang. Telle est la loi pour celle qui enfante un fils ou une fille.
\VS{8}Et que, si elle n'a pas le moyen de trouver un agneau, alors elle prendra deux tourterelles ou deux jeunes pigeons, l'un pour l'holocauste, et l'autre pour le sacrifice d'expiation. Le prêtre fera propitiation pour elle, et elle sera pure.
\Chap{13}
\TextTitle{Lois de purification : La lèpre}
\VerseOne{}Yahweh parla aussi à Moïse et à Aaron, en disant :
\VS{2}L'homme qui aura sur la peau de son corps une tumeur, une dartre, ou une tache blanche, et que cela paraîtra sur la peau de son corps comme une plaie de lèpre, on l'amènera à Aaron, le prêtre, ou à l'un de ses fils prêtres.
\VS{3}Et le prêtre regardera la plaie qui est sur la peau du corps. Si le poil de la plaie est devenu blanc, et si la plaie, à la voir, est plus profonde que la peau du corps, c'est une plaie de lèpre : Le prêtre donc le regardera et le jugera impur.
\VS{4}Mais si la tache est blanche sur la peau du corps, et qu'à la voir, elle n'est point plus profonde que la peau, et si son poil n'est pas devenu blanc, le prêtre fera enfermer pendant sept jours celui qui a la plaie.
\VS{5}Et le prêtre la regardera le septième jour. Si à ses yeux la plaie s'est arrêtée, et qu'elle ne s'est point étendue sur la peau, le prêtre le fera renfermer pendant sept autres jours.
\VS{6}Et le prêtre la regardera une seconde fois le septième jour suivant. Si la plaie est devenue pâle, et qu'elle ne s'est point étendue sur la peau, le prêtre le jugera pur : C'est de la dartre ; il lavera ses vêtements, et sera pur.
\VS{7}Mais si la dartre s'est étendue sur la peau, après avoir été vu par le prêtre pour être jugé pur, il se fera examiner pour la seconde fois par le prêtre.
\VS{8}Le prêtre le regardera encore. S'il aperçoit que la dartre s'est étendue sur la peau, le prêtre le jugera impur : C'est de la lèpre.
\VS{9}Quand il y aura une plaie de lèpre sur un homme, on l'amènera au prêtre.
\VS{10}Le prêtre le regardera. Et s'il aperçoit qu'il y a une tumeur blanche sur la peau, que le poil est devenu blanc, et qu'il y a une trace de chair vive dans la tumeur,
\VS{11}c'est une lèpre invétérée dans la peau du corps : Le prêtre le jugera impur ; il ne le fera point enfermer, car il est jugé impur.
\VS{12}Si la lèpre fait une éruption sur la peau, et qu'elle couvre toute la peau de celui qui a la plaie, depuis la tête de cet homme jusqu'à ses pieds, partout où pourra voir le prêtre, le prêtre le regardera,
\VS{13}et si le prêtre voit que la lèpre couvre tout le corps de cet homme, alors il jugera pur celui qui a la plaie : La plaie est devenue toute blanche, il est pur.
\VS{14}Mais le jour où l'on apercevra de la chair vive, il sera impur ;
\VS{15}alors le prêtre regardera la chair vive, et le jugera impur : La chair vive est impure, c'est de la lèpre.
\VS{16}Si la chair vive se change et devient blanche, alors il viendra vers le prêtre ;
\VS{17}et le prêtre le regardera, et s'il aperçoit que la plaie est devenue blanche, le prêtre jugera pur celui qui a la plaie : Il est pur.
\VS{18}Si le corps a eu sur la peau un ulcère qui soit guéri,
\VS{19}et qu'à l'endroit où était l'ulcère il y ait une tumeur blanche, ou une tache blanche rougeâtre, il sera regardé par le prêtre.
\VS{20}Le prêtre donc la regardera. Et s'il aperçoit, qu'à la voir, elle paraît plus enfoncée que la peau, et que son poil est devenu blanc, alors le prêtre le jugera impur : C'est une plaie de lèpre qui a fait éruption dans l'ulcère.
\VS{21}Si le prêtre la regardant, voit que le poil n'est point blanc, et qu'elle n'est point plus enfoncée que la peau, mais qu'elle est devenue pâle, le prêtre le fera enfermer pendant sept jours.
\VS{22}Si elle s'est étendue sur la peau en quelque sorte que ce soit, le prêtre le jugera impur : C'est une plaie.
\VS{23}Mais si la tache est restée à la même place et ne s'est pas étendue, c'est une cicatrice d'ulcère : Ainsi le prêtre le jugera pur.
\VS{24}Si le corps a sur la peau une brûlure par le feu, et que la chair vive de la partie brûlée soit une tache blanche rougeâtre ou blanc seulement, le prêtre la regardera,
\VS{25}et si le poil est devenu blanc dans la tache, et qu'à la voir, elle est plus profonde que la peau, c'est de la lèpre, elle a fait éruption dans la brûlure ; le prêtre donc le jugera impur : C'est une plaie de lèpre.
\VS{26}Mais si le prêtre la regardant aperçoit qu'il n'y a point de poil blanc dans la tache, et qu'elle n'est point plus basse que la peau, qu'elle est devenue pâle, le prêtre le fera enfermer pendant sept jours.
\VS{27}Puis le prêtre la regardera le septième jour. Si la tache s'est étendue sur la peau, le prêtre le jugera impur : C'est une plaie de lèpre.
\VS{28}Si la tache est restée à la même place, ne s'est pas étendue, et est devenue pâle, c'est la tumeur de la brûlure ; et le prêtre le jugera pur ; c'est la cicatrice de la brûlure.
\VS{29}Si l'homme ou la femme a une plaie à la tête, ou l'homme à la barbe,
\VS{30}le prêtre regardera la plaie, et si à la voir, elle est plus profonde que la peau, et qu'il y ait en elle du poil jaunâtre et fin, le prêtre le jugera impur : C'est de la teigne, c'est une lèpre de la tête ou de la barbe.
\VS{31}Si le prêtre regardant la plaie de la teigne, voit qu'elle n'est point plus profonde que la peau, et n'a en elle aucun poil noir, le prêtre fera enfermer pendant sept jours celui qui a la plaie de la teigne.
\VS{32}Et le septième jour le prêtre regardera la plaie. Si la teigne ne s'est point étendue, qu'elle n'a aucun poil jaunâtre, et, qu'à voir la teigne, elle n'est pas plus profonde que la peau,
\VS{33}celui qui a la plaie de la teigne se rasera, mais il ne se rasera point à l'endroit de la teigne, et le prêtre fera enfermer pendant sept autres jours celui qui a la teigne.
\VS{34}Puis le prêtre regardera la teigne au septième jour. Si la teigne ne s'est point étendue sur la peau et, qu'à la voir, elle n'est point plus profonde que la peau, le prêtre le jugera pur, et cet homme lavera ses vêtements, et il sera pur.
\VS{35}Mais si la teigne s'est étendue sur la peau, après sa purification, le prêtre la regardera,
\VS{36}et si la teigne s'est étendue sur la peau, le prêtre ne cherchera point de poil jaunâtre : Il est impur.
\VS{37}Mais si la teigne s'est arrêtée, et qu'il y ait poussé du poil noir, la teigne est guérie : Il est pur, et le prêtre le jugera pur.
\VS{38}Si l'homme ou la femme ont sur la peau de leur corps des taches, des taches qui sont blanches,
\VS{39}le prêtre les regardera. Si sur la peau de leur corps il y a des taches d'un blanc pâle, c'est une tache blanche qui a fait éruption sur la peau : Il est donc pur.
\VS{40}Si l'homme a la tête dépouillée de cheveux, c'est un chauve : Il est pur.
\VS{41}Et si sa tête est dépouillée de cheveux du côté de son visage, c'est un front chauve : Il est pur.
\VS{42}Et si dans la partie chauve de devant ou de derrière, il y a une plaie d'un blanc rougeâtre, c'est une lèpre qui a fait éruption dans sa partie chauve de derrière ou de devant.
\VS{43}Et le prêtre le regardera. S'il aperçoit que la tumeur de la plaie est d'un blanc rougeâtre dans sa partie chauve de derrière ou de devant, semblable à la lèpre de la peau du corps,
\VS{44}l'homme est lépreux, il est impur : Le prêtre ne manquera pas de le juger impur ; sa plaie est à la tête.
\VS{45}Or le lépreux en qui sera la plaie aura ses vêtements déchirés, et sa tête nue ; et il se couvrira sur la lèvre de dessus et il criera : Impur ! Impur !
\VS{46}Pendant tout le temps qu'il aura cette plaie, il sera jugé impur : Il est impur. Il demeurera seul ; sa demeure sera hors du camp\FTNT{2 R. 7:3 ; La. 4:15 ; Lu. 17:12-13.}.
\VS{47}Et si le vêtement est infecté de la plaie de la lèpre, soit sur un vêtement de laine, soit sur un vêtement de lin,
\VS{48}à la chaîne ou à la trame du lin, ou de laine, sur la peau ou sur quelque ouvrage de peau,
\VS{49}et si cette plaie est verdâtre ou rougeâtre sur le vêtement ou sur la peau, à la chaîne ou à la trame, ou sur un objet quelconque de peau, ce sera une plaie de lèpre, et elle sera montrée au prêtre.
\VS{50}Et le prêtre regardera la plaie, et fera enfermer pendant sept jours celui qui a la plaie.
\VS{51}Et au septième jour, il regardera la plaie. Si la plaie s'est étendue sur le vêtement, à la chaîne ou à la trame, sur la peau ou sur quelque ouvrage de peau, la plaie est une lèpre invétérée : La chose est impure.
\VS{52}Il brûlera le vêtement, la chaîne ou la trame de laine ou de lin, et toutes les choses de peau, qui auront cette plaie, car c'est une lèpre rongeuse : Cela sera brûlé au feu.
\VS{53}Mais si le prêtre regarde, et que la plaie ne s'est point étendue sur le vêtement, sur la chaîne ou sur la trame, ou sur quelque objet de peau,
\VS{54}le prêtre ordonnera qu'on lave la chose où est la plaie, et il le fera enfermer pendant sept autres jours.
\VS{55}Si le prêtre, après qu'on aura fait laver la plaie, la regarde, et s'il aperçoit que la plaie n'a point changé sa couleur, et qu'elle ne s'est point étendue, c'est une chose impure : Tu la brûleras au feu ; c'est une partie de l'endroit ou de l'envers qui a été rongée.
\VS{56}Si le prêtre regarde, et aperçoit que la plaie est devenue pâle, après qu'on l'ait fait laver, il la déchirera du vêtement ou de la peau, de la chaîne ou de la trame.
\VS{57}Si elle paraît encore sur le vêtement, à la chaîne ou à la trame, ou sur quelque chose de peau, c'est une lèpre qui a fait éruption : Vous brûlerez au feu la chose où est la plaie.
\VS{58}Mais si tu as lavé le vêtement, la chaîne ou la trame, ou quelque chose de peau, et que la plaie s'en est allée, il sera lavé une seconde fois, puis il sera pur.
\VS{59}Telle est la loi sur la plaie de la lèpre sur un vêtement de laine ou de lin, la chaîne ou la trame, ou quelque chose de peau, pour la juger pure ou impure.
\Chap{14}
\TextTitle{Loi du lépreux pour le jour de sa purification}
\VerseOne{}Yahweh parla aussi à Moïse, en disant :
\VS{2}C'est ici la loi du lépreux pour le jour de sa purification. Il sera amené au prêtre\FTNT{Mt. 8:2-4 ; Mc. 1:42-44 ; Lu. 5:12-14.}.
\VS{3}Le prêtre sortira hors du camp et le regardera. Si la plaie de la lèpre du lépreux est guérie,
\VS{4}le prêtre ordonnera qu'on prenne pour celui qui doit être purifié, deux oiseaux vivants et purs, avec du bois de cèdre, du cramoisi et de l'hysope\FTNT{Ex. 12:22.}.
\VS{5}Et le prêtre ordonnera qu'on égorge l'un des oiseaux sur un vase de terre, sur de l'eau vive.
\VS{6}Puis il prendra l'oiseau vivant, le bois de cèdre, le cramoisi et l'hysope ; et il trempera toutes ces choses avec l'oiseau vivant, dans le sang de l'autre oiseau qui aura été égorgé sur de l'eau vive.
\VS{7}Il en fera sept fois l'aspersion sur celui qui doit être purifié de la lèpre. Il le déclarera pur, et il laissera aller par les champs, l'oiseau vivant.
\VS{8}Et celui qui doit être purifié lavera ses vêtements, rasera tout son poil, et se lavera dans l'eau ; et il sera pur. Ensuite il entrera dans le camp, mais il demeurera sept jours hors de sa tente.
\VS{9}Au septième jour, il rasera tout son poil, sa tête, sa barbe, les sourcils de ses yeux, tout son poil ; il rasera tout son poil ; puis il lavera ses vêtements et son corps, et il sera pur.
\VS{10}Et au huitième jour, il prendra deux agneaux sans défaut, une brebis d'un an sans défaut, et trois dixièmes de fine farine en offrande de gâteau, pétrie à l'huile, et un log d'huile.
\VS{11}Le prêtre qui fait la purification présentera celui qui doit être purifié et ces choses-là devant Yahweh, à l'entrée de la tente d'assignation.
\VS{12}Puis le prêtre prendra l'un des agneaux et l'offrira en sacrifice pour la culpabilité avec un log d'huile ; il agitera ces choses devant Yahweh, en offrande agitée.
\VS{13}Et il égorgera l'agneau au lieu où l'on égorge l'offrande pour l'expiation et l'holocauste, dans le lieu saint ; car le sacrifice pour la culpabilité appartient au prêtre, comme le sacrifice pour l'expiation ; c'est une chose très sainte.
\VS{14}Le prêtre prendra du sang de l'offrande pour la culpabilité ; il le mettra sur le lobe de l'oreille droite de celui qui doit être purifié, sur le pouce de sa main droite et sur le gros orteil de son pied droit.
\VS{15}Puis le prêtre prendra du log d'huile et en versera dans la paume de sa main gauche.
\VS{16}Et le prêtre trempera le doigt de sa main droite dans l'huile qui est dans sa paume gauche, et fera l'aspersion de l'huile avec son doigt sept fois devant Yahweh.
\VS{17}Et du reste de l'huile qui sera dans sa paume, le prêtre en mettra sur le lobe de l'oreille droite de celui qui doit être purifié, sur le pouce de sa main droite et sur le gros orteil de son pied droit, sur le sang pris de l'offrande pour la culpabilité.
\VS{18}Mais ce qui restera de l'huile sur la paume du prêtre, il le mettra sur la tête de celui qui doit être purifié ; et ainsi le prêtre fera propitiation pour lui devant Yahweh.
\VS{19}Ensuite le prêtre offrira le sacrifice pour l'expiation et fera propitiation pour celui qui doit être purifié de sa souillure. Puis il égorgera l'holocauste.
\VS{20}Le prêtre offrira l'holocauste et le gâteau sur l'autel, et fera propitiation pour celui qui doit être purifié, et il sera pur.
\VS{21}Mais s'il est pauvre et s'il n'a pas le moyen de fournir ces choses, il prendra un agneau en offrande agitée pour la culpabilité, afin de faire propitiation pour lui. Et un dixième de fine farine pétrie à l'huile pour le gâteau, avec un log d'huile.
\VS{22}Et deux tourterelles ou deux jeunes pigeons, selon ce qu'il pourra fournir, dont l'un sera pour le péché et l'autre pour l'holocauste.
\VS{23}Et le huitième jour de sa purification, il les apportera au prêtre, à l'entrée de la tente d'assignation, devant Yahweh.
\VS{24}Et le prêtre recevra l'agneau du sacrifice pour la culpabilité et le log d'huile, et les agitera devant Yahweh en offrande agitée.
\VS{25}Et il égorgera l'agneau du sacrifice pour la culpabilité. Puis le prêtre prendra du sang de l'offrande pour la culpabilité, il le mettra sur le lobe de l'oreille droite de celui qui doit être purifié, sur le pouce de sa main droite et sur le gros orteil de son pied droit.
\VS{26}Puis le prêtre versera de l'huile dans la paume de sa main gauche.
\VS{27}Et avec le doigt de sa main droite, il fera l'aspersion de l'huile qui est dans sa main gauche sept fois devant Yahweh.
\VS{28}Il mettra de cette huile qui est dans sa paume, sur le lobe de l'oreille droite de celui qui doit être purifié et sur le pouce de sa main droite et sur le gros orteil de son pied droit, sur le lieu du sang pris de l'offrande pour la culpabilité.
\VS{29}Après il mettra le reste de l'huile qui est dans sa paume sur la tête de celui qui doit être purifié, afin de faire propitiation pour lui devant Yahweh.
\VS{30}Puis il sacrifiera l'une des tourterelles ou l'un des jeunes pigeons, selon ce qu'il aura pu fournir.
\VS{31}De ce donc qu'il aura pu fournir, l'un sera pour le sacrifice d'expiation et l'autre pour l'holocauste, avec le gâteau ; ainsi le prêtre fera propitiation devant Yahweh pour celui qui doit être purifié.
\VS{32}Telle est la loi de celui qui a une plaie de lèpre, et dont les ressources sont insuffisantes à sa purification.
\TextTitle{Lois de purification d'une maison lépreuse}
\VS{33}Puis Yahweh parla à Moïse et à Aaron, en disant :
\VS{34}Quand vous serez entrés dans le pays de Canaan, que je vous donne en possession, si j'envoie une plaie de lèpre sur une maison du pays que vous posséderez,
\VS{35}celui à qui la maison appartiendra viendra et le fera savoir au prêtre, en disant : Il me semble que j'aperçois comme une plaie dans ma maison.
\VS{36}Alors le prêtre ordonnera qu'on vide la maison avant qu'il y entre pour regarder la plaie, afin que rien de ce qui est dans la maison ne soit impur, puis le prêtre entrera pour voir la maison.
\VS{37}Et il regardera la plaie. Si la plaie qui est sur les murs de la maison a des creux verdâtres ou rougeâtres, qui soient, à les voir, plus enfoncés que le mur ;
\VS{38}le prêtre sortira de la maison, à l'entrée, et fera fermer la maison pendant sept jours.
\VS{39}Au septième jour, le prêtre retournera et la regardera. Si la plaie s'est étendue sur les murs de la maison,
\VS{40}alors il ordonnera de retirer les pierres sur lesquelles est la plaie, et de les jeter hors de la ville, dans un lieu impur.
\VS{41}Il fera aussi racler l'enduit de la maison à l'intérieur, tout autour ; et l'enduit qu'on aura raclé, on le jettera hors de la ville, dans un lieu impur.
\VS{42}Puis on prendra d'autres pierres, et on les mettra à la place des premières pierres ; et on prendra d'autres mortiers pour recrépir la maison.
\VS{43}Mais si la plaie revient et fait éruption dans la maison, après avoir retiré les pierres, après avoir raclé et recrépi la maison,
\VS{44}le prêtre y entrera et la regardera. Si la plaie s'est étendue dans la maison, c'est une lèpre invétérée dans la maison : Elle est impure.
\VS{45}On démolira la maison, ses pierres, son bois, et tout le mortier de la maison ; et on les transportera hors de la ville, dans un lieu impur.
\VS{46}Si quelqu'un est entré dans la maison pendant tout le temps que le prêtre l'avait faite fermer, il sera impur jusqu'au soir.
\VS{47}Celui qui dormira dans cette maison lavera ses vêtements. Celui aussi qui mangera dans cette maison lavera ses vêtements.
\VS{48}Mais quand le prêtre y sera entré, et qu'il aura aperçu que la plaie ne s'est point étendue dans cette maison, après l'avoir recrépie, il jugera la maison pure, car sa plaie est guérie.
\VS{49}Alors il prendra pour purifier la maison deux oiseaux, du bois de cèdre, du cramoisi et de l'hysope.
\VS{50}Il égorgera l'un des oiseaux sur un vase de terre, sur de l'eau vive.
\VS{51}Il prendra le bois de cèdre, l'hysope, le cramoisi et l'oiseau vivant ; il trempera le tout dans le sang de l'oiseau qu'on aura égorgé et dans l'eau vive, puis il fera sept fois l'aspersion sur la maison.
\VS{52}Il purifiera la maison avec le sang de l'oiseau, avec l'eau vive, avec l'oiseau vivant, le bois de cèdre, l'hysope et le cramoisi.
\VS{53}Puis il laissera aller hors de la ville par les champs l'oiseau vivant. C'est ainsi qu'il fera propitiation pour la maison, et elle sera pure.
\VS{54}Telle est la loi pour toute plaie de lèpre et de teigne,
\VS{55}de lèpre de vêtement et de maison,
\VS{56}de tumeur, de dartre, et de tache ;
\VS{57}pour enseigner quand une chose est impure et quand elle est pure. Telle est la loi sur la lèpre.
\Chap{15}
\TextTitle{Lois de purification : Gonorrhée et flux menstruel\FTNTT{Jn. 13:3-10 ; Ep. 5:25-27 ; 1 Jn. 1:9.}}
\VerseOne{}Yahweh parla aussi à Moïse et à Aaron, en disant :
\VS{2}Parlez aux enfants d'Israël et dites-leur : Tout homme qui a une gonorrhée\FTNT{Gonorrhée : infection des organes génito-urinaires.} sera impur à cause de son flux.
\VS{3}Et telle sera l'impureté de son flux : Quand sa chair laissera aller son flux, ou que sa chair retiendra son flux, c'est son impureté.
\VS{4}Tout lit sur lequel se couchera celui qui est atteint d'un flux sera impur ; et toute chose sur laquelle il se sera assis sera impure.
\VS{5}L'homme aussi qui touchera son lit lavera ses vêtements et se lavera avec de l'eau ; et il sera impur jusqu'au soir.
\VS{6}Et celui qui s'assiéra sur quelque chose sur laquelle celui qui a ce flux s'est assis, lavera ses vêtements et se lavera dans l'eau, et il sera impur jusqu'au soir.
\VS{7}Et celui qui touchera la chair de celui qui a ce flux lavera ses vêtements et se lavera dans l'eau, et il sera impur jusqu'au soir.
\VS{8}Si celui qui a ce flux crache sur celui qui est pur, celui qui était pur lavera ses vêtements et se lavera dans l'eau, et il sera impur jusqu'au soir.
\VS{9}Toute monture que celui qui a ce flux aura montée sera impure.
\VS{10}Quiconque touchera quelque chose qui aura été sous lui sera impur jusqu'au soir ; et quiconque portera une telle chose lavera ses vêtements, et se lavera dans l'eau ; il sera impur jusqu'au soir.
\VS{11}Quiconque aura été touché par celui qui a ce flux, sans qu'il ait lavé ses mains dans l'eau, lavera ses vêtements et il se lavera dans l'eau, et il sera impur jusqu'au soir.
\VS{12}Et le vase de terre que celui qui a ce flux aura touché sera cassé, mais tout vase de bois sera lavé dans l'eau.
\VS{13}Or quand celui qui a ce flux sera purifié de son flux, il comptera sept jours pour sa purification ; il lavera ses vêtements et sa chair avec de l'eau vive, et ainsi il sera pur.
\VS{14}Au huitième jour, il prendra pour lui deux tourterelles ou deux jeunes pigeons, et il viendra devant Yahweh à l'entrée de la tente d'assignation, et les donnera au prêtre.
\VS{15}Et le prêtre les sacrifiera, l'un en sacrifice pour l'expiation et l'autre en holocauste ; ainsi le prêtre fera propitiation pour lui devant Yahweh à cause de son flux.
\VS{16}L'homme aussi duquel sera sortie de la semence lavera dans l'eau tout son corps, et il sera impur jusqu'au soir.
\VS{17}Et tout vêtement et toute peau sur lequel il y aura de la semence seront lavés dans l'eau, et seront impurs jusqu'au soir.
\VS{18}Même la femme qui couchera avec un tel homme se lavera dans l'eau avec son mari, et ils seront impurs jusqu'au soir.
\VS{19}Et quand la femme aura un flux, un flux de sang en sa chair, elle sera séparée sept jours. Quiconque la touchera sera impur jusqu'au soir\FTNT{Mt. 9:18-22 ; Mc. 5:21-34 ; Lu. 8:41-48.}.
\VS{20}Toute chose sur laquelle elle aura couché durant sa séparation sera impure, toute chose aussi sur laquelle elle aura été assise sera impure.
\VS{21}Quiconque aussi touchera le lit de cette femme lavera ses vêtements et se lavera dans l'eau, et il sera impur jusqu'au soir.
\VS{22}Et quiconque touchera quelque chose sur laquelle elle se sera assise lavera ses vêtements et se lavera dans l'eau, et il sera impur jusqu'au soir.
\VS{23}Même si la chose que quelqu'un aura touchée était sur le lit ou sur quelque chose sur laquelle elle était assise, quand quelqu'un aura touché cette chose-là, il sera impur jusqu'au soir.
\VS{24}Et si un homme a couché avec elle et que son impureté soit sur lui, il sera impur sept jours, et toute couche sur laquelle il dormira sera impure.
\VS{25}La femme qui aura un flux de sang pendant plusieurs jours, hors de l'époque de ses menstruations, ou dont le flux durera plus longtemps que l'époque de ses menstruations, sera impure tout le temps du flux de son impureté, comme au temps de sa séparation.
\VS{26}Toute couche sur laquelle elle couchera tous les jours de son flux lui sera comme la couche de sa séparation, et toute chose sur laquelle elle s'assiéra sera impure comme pour l'impureté de sa séparation.
\VS{27}Et quiconque aura touché ces choses-là sera impur ; il lavera ses vêtements et se lavera dans l'eau, et il sera impur jusqu'au soir.
\VS{28}Mais si elle est purifiée de son flux, elle comptera sept jours, et après elle sera pure.
\VS{29}Au huitième jour, elle prendra deux tourterelles ou deux jeunes pigeons, et les apportera au prêtre à l'entrée de la tente d'assignation.
\VS{30}Et le prêtre en sacrifiera l'un en sacrifice pour l'expiation et l'autre en holocauste ; ainsi le prêtre fera propitiation pour elle devant Yahweh, à cause du flux de son impureté.
\VS{31}Ainsi, vous séparerez les enfants d'Israël de leurs impuretés, et ils ne mourront point à cause de leurs impuretés, en rendant impur mon tabernacle, qui est au milieu d'eux.
\VS{32}Telle est la loi pour celui qui a une gonorrhée ou de celui duquel sort la semence qui le rend impur.
\VS{33}Telle est aussi la loi pour celle qui a son indisposition menstruelle ou de toute personne qui découle et qui a son flux, soit mâle, soit femelle, et de celui qui couche avec celle qui est impure.
\Chap{16}
\TextTitle{Expiation pour le prêtre, sa maison et le peuple.\FTNTT{Hé. 9:1-14.}}
\VerseOne{}Or Yahweh parla à Moïse après la mort des deux fils d'Aaron, qui moururent lorsqu'ils s'étaient approchés de la présence de Yahweh.
\VS{2}Yahweh donc dit à Moïse : Parle à Aaron, ton frère, et dis-lui qu'il n'entre point en tout temps dans le sanctuaire, au-dedans du voile, devant le propitiatoire qui est sur l'arche, afin qu'il ne meure point ; car j'apparaîtrai dans une nuée sur le propitiatoire.
\VS{3}Aaron entrera dans le sanctuaire de cette manière, après avoir offert un jeune taureau du troupeau pour le péché, et un bélier pour l'holocauste.
\VS{4}Il se revêtira de la sainte tunique de lin, et portera les caleçons de lin sur son corps ; il se ceindra de la ceinture de lin\FTNT{La ceinture de vérité (Ep. 6:14).}, et se couvrira la tête de la tiare\FTNT{La tiare, le casque du salut (Ep. 6:17).} de lin, qui sont les saints vêtements, et il s'en vêtira après avoir lavé son corps avec de l'eau\FTNT{Le lavement préfigure ici la régénération (Tit. 3:5).}.
\VS{5}Et il prendra de l'assemblée des enfants d'Israël deux jeunes boucs en offrande pour le péché et un bélier pour l'holocauste.
\VS{6}Puis Aaron offrira son veau en sacrifice pour l'expiation, et fera propitiation tant pour lui que pour sa maison.
\TextTitle{Les deux boucs expiatoires\FTNTT{2 Co. 5:21.}}
\VS{7}Et il prendra les deux boucs, et les présentera devant Yahweh, à l'entrée de la tente d'assignation.
\VS{8}Puis Aaron jettera le sort sur les deux boucs, un sort pour Yahweh et un sort pour le bouc qui doit être Azazel.
\VS{9}Et Aaron offrira le bouc sur lequel le sort sera échu pour Yahweh, et l'offrira en sacrifice pour l'expiation.
\VS{10}Mais le bouc sur lequel le sort sera tombé pour être Azazel, sera présenté vivant devant Yahweh pour faire propitiation par lui, et on l'enverra dans le désert pour être Azazel.
\VS{11}Aaron donc, présentera le veau en sacrifice pour l'expiation, et fera propitiation pour lui et pour sa maison. Et il égorgera, dis-je, son veau qui est le sacrifice pour l'expiation.
\VS{12}Puis il prendra un encensoir plein de charbons ardents, de dessus l'autel devant Yahweh, et deux poignées de parfum odoriférant en poudre ; et il les apportera au-dedans du voile ;
\VS{13}et il mettra le parfum sur le feu devant Yahweh, afin que la nuée du parfum couvre le propitiatoire qui est sur le témoignage, ainsi il ne mourra point.
\VS{14}Il prendra aussi du sang du veau, et il en fera l'aspersion avec son doigt au-devant du propitiatoire vers l'orient ; il fera l'aspersion de ce sang-là sept fois avec son doigt devant le propitiatoire.
\VS{15}Il égorgera aussi le bouc du peuple, qui est l'offrande pour l'expiation, et il apportera son sang au-dedans du voile. Il fera de son sang comme il a fait du sang du veau, en faisant l'aspersion sur le propitiatoire et sur le devant du propitiatoire.
\VS{16}Et il fera propitiation pour le sanctuaire, le purifiant des impuretés des enfants d'Israël, et de leurs transgressions, selon tous leurs péchés. Il fera la même chose pour la tente d'assignation, qui demeure avec eux au milieu de leurs impuretés.
\VS{17}Et personne ne sera dans la tente d'assignation quand le prêtre y entrera pour faire propitiation dans le sanctuaire, jusqu'à ce qu'il en sorte, lorsqu'il fera propitiation pour lui et pour sa maison, et pour toute l'assemblée d'Israël.
\VS{18}Puis il sortira vers l'autel qui est devant Yahweh, et fera propitiation pour lui ; il prendra du sang du veau et du sang du bouc, il le mettra sur les cornes de l'autel tout autour.
\VS{19}Et il fera par sept fois l'aspersion du sang avec son doigt sur l'autel, et le purifiera et le sanctifiera des impuretés des enfants d'Israël.
\VS{20}Et quand il achèvera de faire propitiation pour le sanctuaire, pour la tente d'assignation et pour l'autel, alors il offrira le bouc vivant.
\VS{21}Et Aaron posera ses deux mains sur la tête du bouc vivant, et il confessera sur lui toutes les iniquités des enfants d'Israël et toutes leurs transgressions, selon tous leurs péchés ; et il les mettra sur la tête du bouc, et l'enverra au désert par un homme prêt pour cela.
\VS{22}Et le bouc portera sur lui toutes leurs iniquités dans une terre inhabitable, puis cet homme laissera aller le bouc par le désert.
\VS{23}Et Aaron reviendra dans la tente d'assignation ; il quittera les vêtements de lin dont il s'était vêtu quand il était entré dans le sanctuaire, et les posera là.
\VS{24}Il lavera aussi son corps avec de l'eau dans le lieu saint, et se revêtira de ses vêtements. Puis il sortira, il offrira son holocauste et l'holocauste du peuple, et fera propitiation pour lui et pour le peuple.
\VS{25}Il brûlera aussi sur l'autel la graisse de l'offrande pour le péché.
\VS{26}Et celui qui aura conduit le bouc pour être Azazel lavera ses vêtements et son corps avec de l'eau ; après cela, il rentrera dans le camp.
\VS{27}Mais on tirera hors du camp le veau et le bouc qui auront été offerts en sacrifice pour l'expiation, et desquels le sang aura été porté dans le sanctuaire pour y faire propitiation, et on brûlera au feu leurs peaux, leur chair et leurs excréments\FTNT{Hé. 13:11.}.
\VS{28}Et celui qui les aura brûlés lavera ses vêtements et son corps avec de l'eau ; après cela, il rentrera dans le camp.
\VS{29}Et ceci sera pour vous une ordonnance perpétuelle : Le dixième jour du septième mois, vous affligerez vos âmes, et vous ne ferez aucune œuvre, tant celui qui est du pays que l'étranger qui fait son séjour parmi vous\FTNT{La fête des expiations (ou yom kippour) avait lieu une fois par an, le dixième jour du septième mois (Ex. 30:10 ; Lé. 16:29). A cette occasion, le grand prêtre jetait le sort sur deux boucs : un sort pour Yahweh et un sort pour Azazel (Lé. 16:8-10). Le bouc pour Yahweh était sacrifié, il préfigurait la mort expiatoire de Christ. Le bouc émissaire, pour Azazel, n'avait lui-même rien fait de mal, mais il était choisi par Dieu pour porter le péché du peuple afin qu'il soit dégagé de toute accusation. Ce que l'on faisait de ce bouc, préfigurait l'œuvre de Jésus-Christ. Il symbolisait le Seigneur qui s'est chargé de nos péchés pour les emporter loin de nous (Es. 53 ; Ps. 103:12 ; Hé. 10:17 ; Hé. 13:12-14). Christ est mort et ressuscité hors du camp et c'est là qu'il nous appelle à le rejoindre : hors du monde et des systèmes religieux (Hé. 13:10-14).}.
\VS{30}Car en ce jour-là le prêtre fera propitiation pour vous, afin de vous purifier : Ainsi vous serez purifiés de tous vos péchés devant Yahweh.
\VS{31}Ce sera pour vous donc un sabbat, un jour de repos, et vous affligerez vos âmes. C'est une ordonnance perpétuelle.
\VS{32}Et le prêtre qu'on aura oint, et qu'on aura consacré pour exercer la sacrificature à la place de son père, fera propitiation, s'étant revêtu des vêtements de lin, qui sont les saints vêtements.
\VS{33}Et il fera propitiation pour le saint sanctuaire et il fera propitiation pour la tente d'assignation et pour l'autel, et pour les prêtres et pour tout le peuple de l'assemblée.
\VS{34}Ceci donc sera pour vous une ordonnance perpétuelle, afin de faire propitiation pour les enfants d'Israël de tous leurs péchés une fois par an. On fit comme Yahweh l'avait ordonné à Moïse.
\Chap{17}
\TextTitle{Les sacrifices apportés à l'entrée de la tente d'assignation}
\VerseOne{}Yahweh parla aussi à Moïse, en disant :
\VS{2}Parle à Aaron et à ses fils, et à tous les enfants d'Israël, et dis-leur : C'est ici ce que Yahweh a ordonné, en disant :
\VS{3}Quiconque de la maison d'Israël aura égorgé un bœuf, un agneau ou une chèvre dans le camp, ou qui l'aura égorgé hors du camp\FTNT{De. 12:6.},
\VS{4}et ne l'aura point amené à l'entrée de la tente d'assignation, pour en faire une offrande à Yahweh, devant le tabernacle de Yahweh, le sang sera imputé à cet homme-là ; il a répandu du sang, c'est pourquoi cet homme-là sera retranché du milieu de son peuple.
\VS{5}C'est afin que les enfants d'Israël amènent leurs sacrifices, qu'ils sacrifient dans les champs, qu'ils les amènent à Yahweh, à l'entrée de la tente d'assignation, vers le prêtre, et qu'ils les sacrifient en sacrifices d'offrande de paix\FTNT{Voir commentaire en Lé. 3:1.} à Yahweh ;
\VS{6}et que le prêtre en répande le sang sur l'autel de Yahweh, à l'entrée de la tente d'assignation, et en brûle la graisse en bonne odeur à Yahweh.
\VS{7}Et qu'ils n'offrent plus leurs sacrifices aux démons, avec lesquels ils se sont prostitués. Ceci leur sera une ordonnance perpétuelle pour eux et leurs descendants\FTNT{De. 32:17 ; Ps. 106:37.}.
\VS{8}Tu leur diras donc : Si un homme de la maison d'Israël, ou des étrangers qui font leur séjour parmi eux, aura offert un holocauste ou un sacrifice,
\VS{9}et qui ne l'aura point amené à l'entrée de la tente d'assignation, pour le sacrifier à Yahweh, cet homme-là sera retranché d'entre ses peuples.
\TextTitle{Importance du sang}
\VS{10}Quiconque de la maison d'Israël ou des étrangers qui font leur séjour parmi eux, aura mangé de quelque sang que ce soit, je mettrai ma face contre cette personne qui aura mangé du sang, et je la retrancherai du milieu de son peuple\FTNT{Ge. 9:4 ; De. 12:16-23 ; 1 S. 14:33.}.
\VS{11}Car l'âme de la chair est dans le sang. C'est pourquoi je vous ai ordonné qu'il soit mis sur l'autel, afin de faire propitiation pour vos âmes, car c'est le sang qui fera propitiation pour l'âme.
\VS{12}C'est pourquoi j'ai dit aux enfants d'Israël : Que personne d'entre vous ne mange du sang, que même l'étranger qui fait son séjour parmi vous ne mange point de sang.
\VS{13}Et quiconque des enfants d'Israël, et des étrangers qui font leur séjour parmi eux, aura pris à la chasse une bête sauvage ou un oiseau que l'on mange, il répandra leur sang et le couvrira de poussière.
\VS{14}Car l'âme de toute chair est dans son sang, c'est son âme. C'est pourquoi j'ai dit aux enfants d'Israël : Vous ne mangerez point le sang d'aucune chair ; car l'âme de toute chair est son sang : Quiconque en mangera sera retranché.
\VS{15}Et toute personne qui aura mangé de la chair de quelque bête morte d'elle-même ou déchirée par les bêtes sauvages, tant celui qui est né dans le pays que l'étranger, lavera ses vêtements et se lavera avec de l'eau, et il sera impur jusqu'au soir ; puis il sera pur.
\VS{16}S'il ne lave pas ses vêtements et son corps, il portera son iniquité.
\Chap{18}
\TextTitle{Condamnation des incestes}
\VerseOne{}Yahweh parla encore à Moïse, en disant :
\VS{2}Parle aux enfants d'Israël et dis-leur : Je suis Yahweh, votre Dieu.
\VS{3}Vous ne ferez point ce qui se fait dans le pays d'Egypte où vous avez habité, ni ce qui se fait dans le pays de Canaan, auquel je vous amène : Vous ne vivrez point selon leurs statuts\FTNT{Jé. 10:2.}.
\VS{4}Mais vous ferez selon mes statuts, et vous garderez mes ordonnances pour marcher en elles. Je suis Yahweh, votre Dieu.
\VS{5}Vous garderez donc mes statuts et mes ordonnances, l'homme qui les pratiquera vivra par elles. Je suis Yahweh\FTNT{Ez. 20:11-13 ; Ga. 3:12 ; Ro. 10:5.}.
\VS{6}Que nul ne s'approche de celle qui est sa proche parente pour découvrir sa nudité. Je suis Yahweh.
\VS{7}Tu ne découvriras point la nudité de ton père, ni la nudité de ta mère. C'est ta mère ; tu ne découvriras point sa nudité.
\VS{8}Tu ne découvriras point la nudité de la femme de ton père. C'est la nudité de ton père\FTNT{De. 22:30 ; 1 Co. 5:1.}.
\VS{9}Tu ne découvriras point la nudité de ta sœur, fille de ton père ou fille de ta mère, née dans la maison ou hors de la maison. Tu ne découvriras point leur nudité.
\VS{10}Quant à la nudité de la fille de ton fils ou de la fille de ta fille, tu ne découvriras point leur nudité. Car elles sont ta nudité.
\VS{11}Tu ne découvriras point la nudité de la fille de la femme de ton père, née de ton père. C'est ta sœur.
\VS{12}Tu ne découvriras point la nudité de la sœur de ton père. Elle est la proche parente de ton père.
\VS{13}Tu ne découvriras point la nudité de la sœur de ta mère ; car elle est la proche parente de ta mère.
\VS{14}Tu ne découvriras point la nudité du frère de ton père. Et tu ne t'approcheras point de sa femme. Elle est ta tante.
\VS{15}Tu ne découvriras point la nudité de ta belle-fille. Elle est la femme de ton fils ; tu ne découvriras point sa nudité.
\VS{16}Tu ne découvriras point la nudité de la femme de ton frère. C'est la nudité de ton frère.
\VS{17}Tu ne découvriras point la nudité d'une femme et de sa fille. Et tu ne prendras point la fille de son fils, ni la fille de sa fille pour découvrir leur nudité. Elles sont tes proches parentes : C'est un crime.
\VS{18}Tu ne prendras point aussi une femme avec sa sœur pour exciter une rivalité en découvrant sa nudité à côté d'elle pendant sa vie.
\TextTitle{Condamnation des abominations}
\VS{19}Tu ne t'approcheras point d'une femme durant son impureté menstruelle, pour découvrir sa nudité.
\VS{20}Tu ne coucheras point avec la femme de ton prochain pour te souiller avec elle\FTNT{Ex. 20:17 ; De. 5:21 ; Mt. 5:28.}.
\VS{21}Tu ne donneras point tes enfants pour les faire passer par le feu devant Moloc\FTNT{Moloc est le nom du dieu auquel les Ammonites, peuple issu de la relation incestueuse de Loth et sa fille, sacrifiaient leurs premiers-nés en les jetant dans un brasier. De. 18:9-10 ; 1 R. 11:5-7 ; 2 R. 23:10 ; Jé. 32:35.}, et tu ne profaneras point le nom de ton Dieu. Je suis Yahweh.
\VS{22}Tu ne coucheras pas aussi avec un homme, comme on couche avec une femme. C'est une abomination\FTNT{1 Co. 6:9-10 ; Ge. 13:13 ; Ro. 1:26-27.}.
\VS{23}Tu ne coucheras point aussi avec une bête pour te souiller avec elle ; et la femme ne se prostituera point à une bête ; c'est une confusion\FTNT{1 Co. 6:9-10 ; Ro. 1:26-27.}.
\VS{24}Ne vous rendez point impurs par aucune de ces choses, car les nations que je vais chasser de devant vous se sont rendues impures par toutes ces choses.
\VS{25}Le pays a été rendu impur ; et je punirai sur lui son iniquité, et le pays vomira ses habitants.
\VS{26}Mais quant à vous, vous garderez mes ordonnances et mes jugements, et vous ne ferez aucune de ces abominations, tant celui qui est né dans le pays que l'étranger qui fait son séjour parmi vous.
\VS{27}Car les gens de ce pays-là qui ont été avant vous, ont fait toutes ces abominations, et le pays en a été rendu impur.
\VS{28}Prenez garde que le pays ne vous vomisse, si vous le rendez impur, comme il aura vomi les nations qui y étaient avant vous.
\VS{29}Car tous ceux qui feront l'une de toutes ces abominations, seront retranchés du milieu de leur peuple.
\VS{30}Vous garderez donc ce que j'ai ordonné de garder, et vous ne pratiquerez aucune de ces coutumes abominables qui ont été pratiquées avant vous, et vous ne vous rendrez point impurs par elles. Je suis Yahweh, votre Dieu.
\Chap{19}
\TextTitle{Mise en garde contre l'idolâtrie}
\VerseOne{}Yahweh parla aussi à Moïse, en disant :
\VS{2}Parle à toute l'assemblée des enfants d'Israël, et dis-leur : Soyez saints, car je suis saint, moi, Yahweh, votre Dieu.
\VS{3}Chacun de vous craindra sa mère et son père, et vous garderez mes sabbats. Je suis Yahweh, votre Dieu\FTNT{Ex. 20:12 ; De. 5:16 ; Mt. 15:4.}.
\VS{4}Vous ne vous tournerez point vers les idoles, et vous ne vous ferez aucun dieu de fonte. Je suis Yahweh, votre Dieu\FTNT{Ex. 20:3-5.}.
\TextTitle{Recommandation pour les sacrifices}
\VS{5}Si vous offrez un sacrifice d'offrande de paix\FTNT{Voir commentaire en Lé. 3:1.} à Yahweh, vous le sacrifierez de votre bon gré.
\VS{6}II se mangera le jour où vous l'aurez sacrifié, et le lendemain, mais ce qui restera jusqu'au troisième jour sera brûlé au feu.
\VS{7}Si on en mange au troisième jour, ce sera une abomination : Il ne sera point agréé.
\VS{8}Quiconque aussi en mangera portera son iniquité ; car il aura profané la chose sainte de Yahweh : Cette personne-là sera retranchée d'entre ses peuples.
\TextTitle{La justice de Yahweh, l'amour pour son prochain}
\VS{9}Quand vous ferez la moisson de votre pays, tu n'achèveras point de moissonner le bout de ton champ, et tu ne glaneras point ce qui restera à cueillir de ta moisson.
\VS{10}Tu ne grappilleras point ta vigne, ni ne recueilleras point les grains tombés de ta vigne, mais tu les laisseras au pauvre et à l'étranger\FTNT{De. 24:19.}. Je suis Yahweh, votre Dieu.
\VS{11}Vous ne déroberez point, et vous ne vous tromperez point les uns les autres ; et aucun de vous ne mentira à son prochain\FTNT{Ex. 20:15 ; Ep. 4:25 ; Col. 3:9.}.
\VS{12}Vous ne jurerez point par mon Nom en mentant, car tu profanerais le Nom de ton Dieu\FTNT{Ex. 20:7 ; De. 5:11.}. Je suis Yahweh.
\VS{13}Tu n'opprimeras point ton prochain, et tu ne le pilleras point\FTNT{De. 24:14-15 ; Ja. 5:4.}. Le salaire de ton mercenaire ne demeurera point chez toi jusqu'au lendemain.
\VS{14}Tu ne maudiras point le sourd, et tu ne mettras point d'achoppement devant l'aveugle, mais tu craindras ton Dieu. Je suis Yahweh.
\VS{15}Vous ne ferez point d'iniquité dans vos jugements : Tu n'auras point d'égard à la personne du pauvre, et tu n'honoreras point la personne du grand, mais tu jugeras ton prochain selon la justice.
\VS{16}Tu ne répandras point de calomnies parmi ton peuple. Tu ne t'élèveras point contre le sang de ton prochain. Je suis Yahweh.
\VS{17}Tu ne haïras point ton frère dans ton cœur ; tu reprendras soigneusement ton prochain\FTNT{Ge. 4:8 ; Mt. 18:15 ; 1 Jn. 2:9-11.}, et tu ne te chargeras point d'un péché à cause de lui.
\VS{18}Tu n'useras point de vengeance, et tu ne la garderas point aux enfants de ton peuple ; mais tu aimeras ton prochain comme toi-même\FTNT{Mt. 7:12 ; Mc. 12:28-34.}. Je suis Yahweh.
\VS{19}Vous garderez mes ordonnances. Tu n'accoupleras point tes bêtes de deux espèces différentes ; tu ne sèmeras point ton champ de diverses sortes de grains ; et tu ne mettras point sur toi de vêtements de diverses espèces, comme de la laine et du lin.
\VS{20}Si un homme couche et a commerce avec une femme, si c'est une esclave, fiancée à un homme, qui n'a pas été rachetée, et que la liberté ne lui a pas été donnée, ils auront le fouet, mais on ne les fera point mourir, parce qu'elle n'a pas été affranchie.
\VS{21}L'homme amènera son sacrifice pour la culpabilité à Yahweh à l'entrée de la tente d'assignation, à savoir un bélier pour la culpabilité.
\VS{22}Et le prêtre fera propitiation pour lui devant Yahweh par le bélier du sacrifice pour la culpabilité, à cause de son péché qu'il aura commis, et son péché qu'il aura commis lui sera pardonné.
\TextTitle{Ordonnances diverses}
\VS{23}Et quand vous serez entrés dans le pays, et que vous y aurez planté quelque arbre fruitier, vous considérerez son fruit comme incirconcis ; il vous sera incirconcis pendant trois ans, on n'en mangera point.
\VS{24}Mais à la quatrième année, tout son fruit sera une chose sainte à la louange de Yahweh.
\VS{25}Et à la cinquième année, vous mangerez son fruit, afin qu'il vous multiplie son produit. Je suis Yahweh, votre Dieu.
\VS{26}Vous ne mangerez rien avec le sang. Vous n'userez point de divinations, et vous ne pronostiquerez point le temps\FTNT{De. 12:23.}.
\VS{27}Vous ne couperez point en rond les coins de votre chevelure, et vous ne raserez point les coins de votre barbe.
\VS{28}Vous ne ferez point d'incisions dans votre chair pour un mort, et vous n'imprimerez point de caractères sur vous. Je suis Yahweh.
\VS{29}Tu ne profaneras point ta fille en la prostituant ; afin que le pays ne se prostitue point et ne se remplisse point de crimes.
\VS{30}Vous garderez mes sabbats et vous aurez en révérence mon sanctuaire. Je suis Yahweh.
\VS{31}Ne vous tournez point vers ceux qui évoquent les morts, ni vers les devins\FTNT{Ac. 16:16.} ; ne cherchez point à vous rendre impurs avec eux. Je suis Yahweh, votre Dieu.
\VS{32}Lève-toi devant les cheveux blancs, et tu honoreras la personne du vieillard. Tu craindras ton Dieu. Je suis Yahweh.
\VS{33}Si quelque étranger séjourne dans votre pays, vous ne lui ferez point de tort.
\VS{34}L'étranger qui séjourne parmi vous, vous sera comme celui qui est né parmi vous, et vous l'aimerez comme vous-mêmes, car vous avez été étrangers dans le pays d'Egypte. Je suis Yahweh, votre Dieu.
\VS{35}Vous ne ferez point d'iniquité dans les jugements, ni dans les mesures de dimension, ni dans les poids, ni dans les mesures de capacité.
\VS{36}Vous aurez les balances justes, les pierres à peser justes, l'épha juste et le hin juste. Je suis Yahweh, votre Dieu, qui vous ai fait sortir du pays d'Egypte.
\VS{37}Gardez donc toutes mes ordonnances et mes jugements, et pratiquez-les. Je suis Yahweh.
\Chap{20}
\TextTitle{Abominations diverses et leurs châtiments}
\VerseOne{}Yahweh parla aussi à Moïse, en disant :
\VS{2}Tu diras aux enfants d'Israël : Quiconque des enfants d'Israël ou des étrangers qui demeurent en Israël, qui donnera de sa postérité à Moloc, sera puni de mort : Le peuple du pays le lapidera.
\VS{3}Et je mettrai ma face contre un tel homme, et je le retrancherai du milieu de son peuple, parce qu'il a donné de sa postérité à Moloc, pour rendre impur mon sanctuaire et profaner le Nom de ma sainteté.
\VS{4}Si le peuple du pays ferme les yeux en quelque manière que ce soit sur cet homme-là, qui donne de sa postérité à Moloc, et s'il ne le fait pas mourir,
\VS{5}je mettrai ma face contre cet homme-là, contre sa famille, et je le retrancherai du milieu de mon peuple, avec tous ceux qui se prostituent comme lui, en se prostituant après Moloc.
\VS{6}Quant à la personne qui se tournera vers ceux qui évoquent les morts, vers les devins, en se prostituant après eux, je mettrai ma face contre cette personne-là, et je la retrancherai du milieu de son peuple.
\VS{7}Sanctifiez-vous donc, et soyez saints, car je suis Yahweh, votre Dieu.
\VS{8}Gardez aussi mes lois et pratiquez-les. Je suis Yahweh, qui vous sanctifie.
\VS{9}Un homme qui maudit son père ou sa mère sera puni de mort ; il a maudit son père ou sa mère : Son sang retombera sur lui.
\VS{10}Quant à l'homme qui commet un adultère avec la femme d'un autre, parce qu'il a commis un adultère avec la femme de son prochain, l'homme et la femme adultères seront mis à mort.
\VS{11}L'homme qui couche avec la femme de son père, découvre la nudité de son père, les deux seront mis à mort, leur sang est sur eux.
\VS{12}Quant un homme couche avec sa belle-fille, ils seront mis à mort, tous deux ; ils ont fait une confusion : Leur sang est sur eux.
\VS{13}Quant un homme couche avec un homme comme on couche avec une femme, ils ont tous deux fait une chose abominable ; ils seront mis à mort : Leur sang est sur eux.
\VS{14}Et si un homme prend pour femmes la fille et la mère, c'est un crime : Il sera brûlé au feu avec elles, afin que ce crime n'existe pas au milieu de vous.
\VS{15}Si un homme couche avec une bête, il sera puni de mort ; et vous tuerez aussi la bête.
\VS{16}Et si une femme s'approche d'une bête, tu tueras cette femme et la bête ; ils seront mis à mort : Leur sang sera sur eux.
\VS{17}Si un homme prend sa sœur, fille de son père ou fille de sa mère, et voit sa nudité, et qu'elle voit la nudité de cet homme, c'est une chose infâme ; ils seront donc retranchés sous les yeux des fils de leur peuple : Il a découvert la nudité de sa sœur, il portera son iniquité.
\VS{18}Si un homme couche avec une femme qui a son indisposition menstruelle, et qu'il découvre la nudité de cette femme, en découvrant son flux, et qu'elle découvre le flux de son sang, ils seront tous deux retranchés du milieu de leur peuple.
\VS{19}Tu ne découvriras point la nudité de la sœur de ta mère, ni de la sœur de ton père, car c'est découvrir sa proche parente, ils porteront tous deux leur iniquité.
\VS{20}Si un homme couche avec sa tante, il a découvert la nudité de son oncle ; ils porteront leur péché, et ils mourront privés d'enfants.
\VS{21}Si un homme prend la femme de son frère, c'est une impureté ; il a découvert la nudité de son frère, ils seront privés d'enfants.
\VS{22}Vous garderez toutes mes ordonnances et mes jugements et vous les pratiquerez, afin que le pays où je vous fais entrer pour y habiter ne vous vomisse point.
\VS{23}Vous ne suivrez point les statuts des nations que je vais chasser devant vous ; car elles ont fait toutes ces choses-là, et je les ai eues en abomination.
\VS{24}Et je vous ai dit : Vous posséderez leur pays, je vous le donnerai en possession : C'est un pays où coulent le lait et le miel. Je suis Yahweh, votre Dieu, qui vous ai séparés des autres peuples.
\VS{25}C'est pourquoi séparez les bêtes pures de celles qui sont impures, les oiseaux purs de ceux qui sont impurs, et ne rendez point abominables vos personnes en mangeant des bêtes et des oiseaux impurs, ni rien qui rampe sur la terre, rien de ce que je vous ai défendu comme une chose impure.
\VS{26}Vous me serez donc saints, car je suis saint, moi, Yahweh ; je vous ai séparés des autres peuples afin que vous soyez à moi.
\VS{27}Si un homme ou une femme évoquent les morts ou se livrent à la divination, on les mettra à mort ; on les lapidera : Leur sang sera sur eux.
\Chap{21}
\TextTitle{Recommandations aux prêtres}
\VerseOne{}Yahweh dit aussi à Moïse : Parle aux prêtres, fils d'Aaron, et dis-leur : Aucun d'eux ne se rendra impur parmi son peuple pour un mort,
\VS{2}excepté pour son proche parent, pour sa mère, pour son père, pour son fils, pour sa fille, et pour son frère,
\VS{3}et aussi pour sa sœur vierge, qui lui est proche, et qui n'aura point eu de mari, il se rendra impur pour elle.
\VS{4}Chef parmi son peuple, il ne se rendra point impur en se profanant.
\VS{5}Ils ne se feront point de place chauve sur la tête, ils ne raseront point les coins de leur barbe, ni ne feront point d'incisions dans leur chair.
\VS{6}Ils seront consacrés à leur Dieu, et ils ne profaneront point le Nom de leur Dieu ; car ils offrent à Yahweh les sacrifices consumés par le feu, qui sont la nourriture de leur Dieu : C'est pourquoi ils seront très saints.
\VS{7}Ils ne prendront point une femme prostituée ou déshonorée ; ils ne prendront point une femme répudiée par son mari, car ils sont saints pour leur Dieu.
\VS{8}Tu regarderas chacun d'eux comme saint, parce qu'ils offrent la nourriture de ton Dieu ; ils seront saints, car je suis saint, moi, Yahweh, qui vous sanctifie.
\VS{9}Si la fille du prêtre se profane en se prostituant, elle déshonore son père : Qu'elle soit brûlée au feu.
\VS{10}Le grand prêtre d'entre ses frères, sur la tête duquel l'huile d'onction a été répandue, et qui se sera consacré pour vêtir les saints vêtements, ne découvrira point sa tête et ne déchirera point ses vêtements.
\VS{11}Il n'ira vers aucune personne morte, il ne se rendra point impur pour son père ni pour sa mère.
\VS{12}Il ne sortira point du sanctuaire, et ne profanera point le sanctuaire de son Dieu ; car l'huile d'onction de son Dieu est une couronne sur lui. Je suis Yahweh.
\VS{13}Il prendra pour femme une vierge.
\VS{14}Il ne prendra point une veuve, ni une répudiée, ni une femme déshonorée ou prostituée ; mais il prendra pour femme une vierge parmi son peuple.
\VS{15}Il ne profanera point sa postérité parmi son peuple ; car je suis Yahweh qui le sanctifie.
\VS{16}Yahweh parla aussi à Moïse, en disant :
\VS{17}Parle à Aaron, et dis-lui : Si quelqu'un de ta postérité, parmi tes descendants, qui a quelque défaut corporel, il ne s'approchera point pour offrir la nourriture de son Dieu.
\VS{18}Car tout homme en qui il y aura un défaut n'en approchera point ; l'homme aveugle, boiteux, ayant le nez camus ou qui aura un membre allongé ;
\VS{19}ou l'homme qui aura une fracture aux pieds ou aux mains ;
\VS{20}ou qui sera bossu ou grêle, qui aura une tache à l'œil, qui aura une gale sèche, une dartre, ou qui aura les testicules écrasés.
\VS{21}Nul homme de la postérité d'Aaron, le prêtre, en qui il y aura un défaut corporel, ne s'approchera pour offrir les offrandes consumées par le feu à Yahweh ; il y a un défaut en lui, il ne s'approchera donc point pour offrir la nourriture de son Dieu.
\VS{22}Il pourra manger la nourriture de son Dieu, des choses très saintes et des choses saintes.
\VS{23}Mais il n'entrera point vers le voile, ni ne s'approchera point de l'autel, car il a un défaut corporel, et il ne profanera point mes sanctuaires, car je suis Yahweh, qui les sanctifie.
\VS{24}Moïse parla ainsi à Aaron et à ses fils, et à tous les enfants d'Israël.
\Chap{22}
\TextTitle{Consécration d'Aaron et de ses fils}
\VerseOne{}Puis Yahweh parla à Moïse, en disant :
\VS{2}Parle à Aaron et à ses fils, afin qu'ils s'abstiennent des choses saintes des enfants d'Israël, et qu'ils ne profanent point le Nom de ma sainteté dans les choses qu'ils me consacrent. Je suis Yahweh.
\VS{3}Dis-leur donc : Tout homme parmi votre génération et de vos descendants qui, étant impur, s'approchera des choses saintes que les enfants d'Israël auront sanctifiées à Yahweh, cette personne-là sera retranchée de devant moi. Je suis Yahweh.
\VS{4}Tout homme de la postérité d'Aaron, qui aura la lèpre ou une gonorrhée, ne mangera point des choses saintes jusqu'à ce qu'il soit pur. Il en sera de même pour celui qui touchera quelqu'un s'étant rendu impur en touchant un mort, ou celui qui aura une perte séminale,
\VS{5}et celui qui touchera un reptile et qui en aura été impur, ou un homme atteint d'une impureté quelconque, il en sera rendu impur.
\VS{6}La personne qui touchera ces choses sera rendu impur jusqu'au soir ; il ne mangera point des choses saintes s'il n'a point lavé son corps dans l'eau ;
\VS{7}Ensuite il sera pur après le coucher du soleil, et il mangera des choses saintes, car c'est sa nourriture.
\VS{8}Il ne mangera de la chair d'aucune bête morte d'elle-même ou déchirée par les bêtes sauvages, pour se rendre impur par elle. Je suis Yahweh.
\VS{9}Ils garderont ce que j'ai ordonné de garder, et ils ne commettront point de péché au sujet de la nourriture sainte, afin qu'ils ne meurent point, pour l'avoir profanée. Je suis Yahweh, qui les sanctifie.
\VS{10}Aucun étranger ne mangera des choses saintes ; l'étranger logé chez le prêtre et le mercenaire ne mangeront point des choses saintes.
\VS{11}Mais si le prêtre achète une personne avec son argent, elle en mangera, de même pour celui qui sera né dans sa maison ; ils mangeront de sa nourriture.
\VS{12}Si la fille du prêtre est mariée à un homme étranger, elle ne mangera point des choses saintes présentées en offrande par élévation.
\VS{13}Mais si la fille du prêtre est veuve ou répudiée, et si elle n'a point d'enfants, et est retournée dans la maison de son père, comme dans sa jeunesse, elle mangera de la nourriture de son père. Mais aucun étranger n'en mangera.
\VS{14}Si quelqu'un, pèche involontairement en mangeant d'une chose sainte, il y ajoutera un cinquième et le donnera au prêtre avec la chose sainte.
\VS{15}Et ils ne profaneront point les choses sanctifiées des enfants d'Israël, qu'ils auront offertes à Yahweh.
\VS{16}Mais on leur fera porter la peine du péché, parce qu'ils auront mangé de leurs choses saintes : Car je suis Yahweh, qui les sanctifie.
\TextTitle{Des animaux sans défaut pour les sacrifices\FTNTT{Hé. 9:14.}}
\VS{17}Yahweh parla encore à Moïse, en disant :
\VS{18}Parle à Aaron, à ses fils, et à tous les enfants d'Israël, et dis-leur : Quiconque de la maison d'Israël ou des étrangers qui sont en Israël, offrira son offrande, selon tous ses vœux, ou toutes ses offrandes volontaires, qu'on offre en holocauste à Yahweh,
\VS{19}il offrira de son bon gré, un mâle sans défaut, parmi les bœufs, les agneaux ou les chèvres.
\VS{20}Vous n'offrirez aucune chose qui ait un défaut, car elle ne serait point agréée pour vous.
\VS{21}Si un homme offre à Yahweh un sacrifice d'offrande de paix\FTNT{Voir commentaire en Lé. 3:1.} en s'acquittant d'un vœu, ou en faisant une offrande volontaire, soit de gros ou de menu bétail, elle sera sans défaut pour être agréée ; il ne doit y avoir aucun défaut.
\VS{22}Vous n'offrirez point à Yahweh ce qui sera aveugle, estropié, ou mutilé, qui ait un ulcère, une gale sèche ou une dartre ; et vous n'en ferez point sur l'autel un sacrifice consumé par le feu pour Yahweh.
\VS{23}Tu pourras bien faire une offrande volontaire d'un bœuf, ou d'une brebis, ou d'une chèvre ayant quelques membres allongés, ou quelque défaut dans ses membres, mais ils ne seront point agréés pour le vœu.
\VS{24}Vous n'offrirez point à Yahweh, et ne sacrifierez point dans votre pays un animal qui ait les testicules froissés, cassés, arrachés ou taillés.
\VS{25}Vous ne prendrez point de la main de l'étranger aucune de toutes ces choses pour les offrir comme nourriture à votre Dieu ; car la corruption qui est en eux est un défaut en elles : Elles ne seront point agréées pour vous.
\VS{26}Yahweh parla encore à Moïse, en disant :
\VS{27}Quand un veau, un agneau ou une chèvre seront nés, et qu'ils auront été sept jours sous leur mère, depuis le huitième jour et les suivants, ils seront agréables pour l'offrande du sacrifice consumé par le feu à Yahweh.
\VS{28}Vous n'égorgerez point aussi en un même jour la vache, ou la brebis, ou la chèvre, avec son petit.
\VS{29}Quand vous offrirez un sacrifice de remerciement à Yahweh, vous le sacrifierez de votre bon gré.
\VS{30}Il sera mangé le jour même ; vous n'en laisserez rien jusqu'au matin. Je suis Yahweh.
\VS{31}Gardez mes commandements et pratiquez-les. Je suis Yahweh.
\VS{32}Ne profanez point le nom de ma sainteté, car je serai sanctifié entre les enfants d'Israël. Je suis Yahweh, qui vous sanctifie,
\VS{33}et qui vous ai fait sortir du pays d'Egypte, pour être votre Dieu. Je suis Yahweh.
\Chap{23}
\TextTitle{Les fêtes de Yahweh}
\VerseOne{}Yahweh parla aussi à Moïse en disant :
\VS{2}Parle aux enfants d'Israël et dis-leur : Les fêtes\FTNT{Les fêtes de Yahweh étaient des jours solennels, c'est-à-dire des temps fixés pour s'approcher de Dieu et présenter des sacrifices (Voir le tableau en annexe « Les 7 fêtes de Yahweh » et également le dictionnaire).} solennelles de Yahweh, que vous publierez, seront de saintes convocations. Ce sont ici mes fêtes solennelles.
\VS{3}On travaillera six jours ; mais au septième jour, qui est le sabbat, le jour du repos, il y aura une sainte convocation. Vous ne ferez aucune œuvre, car c'est le sabbat à Yahweh, dans toutes vos demeures.
\TextTitle{La Pâque}
\VS{4}Ce sont ici les fêtes solennelles de Yahweh, qui seront de saintes convocations, que vous publierez en leur saison.
\VS{5}Au premier mois, le quatorzième jour du mois, entre les deux soirs, sera la Pâque\FTNT{La pâque était une fête qui commémorait la sortie d'Egypte (Ex. 12:1-14). Elle préfigurait la rédemption en Jésus-Christ, notre Pâque (1 Co. 5:7). Elle était fixée au 14ème jour du mois de Nisan, le premier mois.} à Yahweh.
\TextTitle{La fête des pains sans levain\FTNTT{Ex. 12:18 ; 13:6-8 ; 1 Co. 11:23-26.}}
\VS{6}Et le quinzième jour de ce même mois, sera la fête solennelle des pains sans levain\FTNT{La fête des pains sans levain commençait le 15ème jour du même mois (Nisan) et durait sept jours. Elle annonçait Christ, notre Pain descendu du ciel (Jn. 6:32-35). Seul le Seigneur Jésus a été sans levain, c'est-à-dire sans aucun péché. Le croyant est sauvé à la Pâque de Christ et doit vivre une vie sans péché (la fête des pains sans levain).} à Yahweh ; vous mangerez des pains sans levain pendant sept jours.
\VS{7}Le premier jour, vous aurez une sainte convocation : Vous ne ferez aucune œuvre servile.
\VS{8}Mais vous offrirez à Yahweh pendant sept jours des offrandes consumées par le feu. Et au septième jour, il y aura une sainte convocation : Vous ne ferez aucune œuvre servile.
\TextTitle{La fête des prémices\FTNTT{1 Co. 15:23.}}
\VS{9}Yahweh parla aussi à Moïse, en disant :
\VS{10}Parle aux enfants d'Israël et dis-leur : Quand vous serez entrés dans le pays que je vous donne, et que vous en aurez fait la moisson, vous apporterez alors au prêtre une gerbe des premiers fruits\FTNT{La fête des prémices annonce d'abord la résurrection du Seigneur Jésus-Christ, ensuite celle de tous ceux qui lui appartiennent (1 Th. 4:13-18 ; 1 Co. 15:23). Elle commençait le premier jour de la semaine suivant le sabbat de la Pâque, au mois de Nisan.} de votre moisson.
\VS{11}Et il agitera cette gerbe-là devant Yahweh, afin qu'elle soit agréée pour vous : Le prêtre l'agitera le lendemain du sabbat.
\VS{12}Et le jour où vous agiterez cette gerbe, vous sacrifierez un agneau sans défaut et d'un an, en holocauste à Yahweh ;
\VS{13}et le gâteau de cet holocauste sera de deux dixièmes de fine farine, pétrie à l'huile, pour offrande consumée par le feu, en bonne odeur à Yahweh ; et sa libation de vin sera le quart d'un hin.
\VS{14}Vous ne mangerez ni pain, ni grain rôti, ni grain en épi, jusqu'à ce jour-là, même jusqu'à ce que vous ayez apporté l'offrande à votre Dieu. C'est une loi perpétuelle pour vos descendants, dans toutes vos demeures.
\TextTitle{La Pentecôte ou la fête des semaines}
\VS{15}Vous compterez aussi dès le lendemain du sabbat, à savoir dès le jour où vous aurez apporté la gerbe qu'on doit agiter, sept semaines entières.
\VS{16}Vous compterez donc cinquante jours\FTNT{La fête des semaines ou fête de la moisson est désignée également comme la Pentecôte. Elle avait lieu au mois de Sivan et préfigurait l'effusion du Saint-Esprit et l'inauguration de la Nouvelle Alliance (Ac. 2:1-4). Le levain autorisé lors de cette fête évoquait par avance la présence de l'ivraie, symbole du péché et des fils du malin, parmi le blé, c'est-à-dire les enfants de Dieu (Mt. 13:24-41). Cinquante jours séparent la Pâque de la Pentecôte. Cet intervalle correspond exactement à la période séparant la résurrection du Seigneur Jésus-Christ de la naissance de l'Eglise (Ac. 2:1-4).} jusqu'au lendemain du septième sabbat ; et vous offrirez à Yahweh un gâteau nouveau.
\VS{17}Vous apporterez de vos demeures deux pains pour en faire une offrande agitée, ils seront de deux dixièmes, et de fine farine, pétris avec du levain : Ce sont les premiers fruits à Yahweh.
\VS{18}Vous offrirez aussi avec ce pain-là sept agneaux sans défaut et d'un an, un jeune taureau pris du troupeau et deux béliers, qui seront un holocauste à Yahweh, avec leurs gâteaux et leurs libations, des sacrifices consumés par le feu, en bonne odeur à Yahweh.
\VS{19}Vous sacrifierez aussi un jeune bouc en sacrifice pour l'expiation, et deux agneaux d'un an pour le sacrifice d'offrande de paix\FTNT{Voir commentaire en Lé. 3:1.}.
\VS{20}Et le prêtre les agitera avec le pain des premiers fruits, et avec les deux agneaux, en offrande agitée devant Yahweh : Ils seront saints à Yahweh, pour le prêtre.
\VS{21}Vous publierez donc, en ce même jour-là, une sainte convocation : Vous ne ferez aucune œuvre servile. C'est une ordonnance perpétuelle dans toutes vos demeures, pour vos descendants.
\VS{22}Et quand vous ferez la moisson de votre pays, tu n'achèveras point de moissonner le bout de ton champ, et tu ne glaneras point les épis qui resteront de ta moisson. Mais tu les laisseras pour le pauvre et pour l'étranger. Je suis Yahweh, votre Dieu.
\TextTitle{La fête des trompettes}
\VS{23}Yahweh parla aussi à Moïse, en disant :
\VS{24}Parle aux enfants d'Israël et dis-leur : Au septième mois, le premier jour du mois, il y aura un jour de repos pour vous, un mémorial de jubilation\FTNT{La fête des trompettes préfigure le rassemblement futur du peuple d'Israël après sa longue dispersion et l'enlèvement de l'Eglise. Cette fête était fixée au premier jour du septième mois (Tishri).}, et une sainte convocation.
\VS{25}Vous ne ferez aucune œuvre servile, et vous offrirez à Yahweh des offrandes consumées par le feu.
\TextTitle{Le jour des expiations\FTNTT{Hé. 9:1-16.}}
\VS{26}Yahweh parla aussi à Moïse, en disant :
\VS{27}Pareillement en ce même mois, qui est le septième, le dixième jour sera le jour des expiations\FTNT{Le jour des expiations ou du grand pardon (Voir Lé. 16) était célébré le dixième jour du septième mois (Tishri). Le Seigneur Jésus-Christ a fait l'expiation de nos péchés afin de nous amener à Dieu. Le propitiatoire au lieu d'être le trône du jugement, devenait ainsi le lieu de rencontre de Dieu avec le croyant (Ex. 25:22). Christ est la propitiation pour nos péchés (1 Jn. 2:2), mais il est aussi lui-même le propitiatoire (Ro. 3:25). Le péché ôté, les fautes confessées, le pardon acquis, l'holocauste offert, le chemin est ouvert pour la joie de la fête des tabernacles.} : Vous aurez une sainte convocation, vous humilierez vos âmes, et vous offrirez à Yahweh des sacrifices consumés par le feu.
\VS{28}En ce jour-là, vous ne ferez aucune œuvre, car c'est le jour des expiations, afin de faire propitiation pour vous devant Yahweh, votre Dieu.
\VS{29}Toute personne qui ne s'humiliera point en ce jour-là sera retranchée d'entre son peuple.
\VS{30}Et toute personne qui aura fait quelque œuvre en ce jour-là, je ferai périr cette personne-là du milieu de son peuple.
\VS{31}Vous ne ferez donc aucune œuvre. C'est une ordonnance perpétuelle pour vos descendants dans toutes vos demeures.
\VS{32}Ce sera pour vous un sabbat, un jour de repos, et vous humilierez vos âmes. Le neuvième jour du mois, au soir, depuis le soir jusqu'à l'autre soir, vous célébrerez votre sabbat.
\TextTitle{La fête des tabernacles\FTNTT{Esd. 3:4.}}
\VS{33}Yahweh parla aussi à Moïse, en disant :
\VS{34}Parle aux enfants d'Israël, et dis-leur : Au quinzième jour de ce septième mois sera la fête solennelle des tabernacles\FTNT{La fête des tabernacles ou des récoltes, était la fête du souvenir et de la joie. Célébrée au mois de Tishri, elle était aussi celle du repos, dans l'accomplissement des promesses. Elle préfigure le Royaume millénaire (Za. 14).} pendant sept jours, à Yahweh.
\VS{35}Au premier jour, il y aura une sainte convocation : Vous ne ferez aucune œuvre servile.
\VS{36}Pendant sept jours, vous offrirez à Yahweh des offrandes consumées par le feu. Et au huitième jour, vous aurez une sainte convocation, et vous offrirez à Yahweh des offrandes consumées par le feu ; ce sera une assemblée solennelle : Vous ne ferez aucune œuvre servile.
\VS{37}Ce sont là les fêtes solennelles de Yahweh, que vous publierez pour être des convocations saintes, afin d'offrir à Yahweh des offrandes consumées par le feu ; à savoir un holocauste, un gâteau, un sacrifice et une libation, chacune de ces choses en son jour ;
\VS{38}outre les sabbats de Yahweh, et outre vos dons, outre tous vos vœux, outre toutes les offrandes volontaires que vous présenterez à Yahweh.
\VS{39}Et aussi au quinzième jour du septième mois, quand vous aurez recueilli le produit du pays, vous célébrerez la fête solennelle de Yahweh pendant sept jours : Le premier jour sera un jour de repos, le huitième aussi sera un jour de repos.
\VS{40}Et au premier jour, vous prendrez du fruit d'un bel arbre, des branches de palmier, des rameaux d'arbres touffus et des saules de rivière ; et vous vous réjouirez pendant sept jours, devant Yahweh, votre Dieu.
\VS{41}Et vous célébrerez à Yahweh cette fête solennelle pendant sept jours dans l'année. C'est une loi perpétuelle pour vos descendants. Vous la célébrerez le septième mois.
\VS{42}Vous demeurerez sept jours sous des tentes ; tous ceux qui seront nés entre les Israélites demeureront sous des tentes,
\VS{43}afin que votre postérité sache que j'ai fait habiter les enfants d'Israël sous des tentes, quand je les ai fait sortir du pays d'Egypte. Je suis Yahweh, votre Dieu.
\VS{44}Moïse déclara ainsi aux enfants d'Israël les fêtes solennelles de Yahweh.
\Chap{24}
\TextTitle{L'huile du chandelier\FTNTT{Ex. 25:6.}}
\VerseOne{}Yahweh parla aussi à Moïse, en disant :
\VS{2}Ordonne aux enfants d'Israël de t'apporter de l'huile pure d'olives pressées pour le chandelier, afin de faire brûler les lampes continuellement.
\VS{3}Aaron les arrangera devant Yahweh continuellement, depuis le soir jusqu'au matin, en dehors du voile du témoignage dans la tente d'assignation. C'est une ordonnance perpétuelle pour vos descendants.
\VS{4}Il arrangera, dis-je, continuellement les lampes sur le chandelier pur, devant Yahweh.
\TextTitle{Les pains de proposition\FTNTT{Ex. 25:23-30.}}
\VS{5}Tu prendras aussi de la fine farine\FTNT{La fine farine est une farine de blé très pure, la première qui passe à travers les tamis de bluterie.}, et tu en feras cuire douze gâteaux\FTNT{Les pains de proposition étaient au nombre de douze et ne pouvaient être consommés que par les prêtres (Lé. 24:9). Ils préfiguraient Christ, le véritable pain de vie descendu du ciel (Jn. 6:48-51). Sous la Nouvelle Alliance, chaque enfant de Dieu est également un prêtre (Ap. 1:6), et est invité par conséquent à manger ce pain. Le nombre douze nous parle du fondement sur lequel nous devons êtres bâtis, à savoir Jésus-Christ lui-même et l'enseignement des apôtres et des prophètes (1 Co. 3:11 ; Ep. 2:20).}, chaque gâteau sera de deux dixièmes.
\VS{6}Et tu les exposeras devant Yahweh en deux rangées sur la table d'or pur, six à chaque rangée.
\VS{7}Et tu mettras de l'encens pur sur chaque rangée, qui sera comme un souvenir\FTNT{Voir commentaire en Lé. 2:2.} pour le pain, c'est une offrande consumée par le feu à Yahweh.
\VS{8}On les arrangera chaque jour de sabbat continuellement devant Yahweh, de la part des enfants d'Israël : C'est une alliance perpétuelle.
\VS{9}Et ils appartiendront à Aaron et à ses fils, qui les mangeront dans un lieu saint ; car ce sera pour eux une chose très sainte d'entre les offrandes de Yahweh consumées par le feu. C'est une ordonnance perpétuelle.
\TextTitle{Le blasphème contre le Nom de Yahweh\FTNTT{Jn. 8:59 ; 10:31.}}
\VS{10}Or le fils d'une femme israélite, qui était aussi fils d'un homme égyptien, sortit parmi les fils d'Israël, et ce fils de la femme israélite se querella dans le camp avec un homme israélite.
\VS{11}Et le fils de la femme israélite blasphéma et maudit le Nom de Yahweh. On l'amena à Moïse. Or sa mère s'appelait Schelomith, fille de Dibri, de la tribu de Dan.
\VS{12}Et on le mit en prison, jusqu'à ce que Moïse ait déclaré ce qu'il devrait faire selon la parole de Yahweh.
\VS{13}Et Yahweh parla à Moïse, en disant :
\VS{14}Tire hors du camp celui qui a maudit ; et que tous ceux qui l'ont entendu mettent les mains sur sa tête, et que toute l'assemblée le lapide.
\VS{15}Tu parleras aux enfants d'Israël, et tu leur diras : Quiconque aura maudit son Dieu, portera la peine de son péché.
\VS{16}Et celui qui aura blasphémé le Nom de Yahweh sera puni de mort : Toute l'assemblée ne manquera pas de le lapider, on fera mourir tant l'étranger que celui qui est né au pays, lequel aura blasphémé le Nom de Yahweh.
\TextTitle{La violence punie}
\VS{17}On punira aussi de mort celui qui aura frappé à mort quelque personne que ce soit.
\VS{18}Celui qui aura frappé une bête à mort, la remplacera : Vie pour vie.
\VS{19}Et quand quelque homme aura fait une blessure à son prochain, on lui fera comme il a fait :
\VS{20}Fracture pour fracture, œil pour œil, dent pour dent, selon le mal qu'il aura fait à un homme, il lui sera fait de même.
\VS{21}Celui qui frappera une bête à mort, la remplacera ; mais on fera mourir celui qui aura frappé un homme à mort.
\VS{22}Vous rendrez un même jugement. Vous traiterez l'étranger comme celui qui est né au pays ; car je suis Yahweh, votre Dieu.
\VS{23}Moïse parla aux enfants d'Israël, qui firent sortir hors du camp celui qui avait maudit, et le lapidèrent. Ainsi les fils d'Israël firent comme Yahweh l'avait ordonné à Moïse.
\Chap{25}
\TextTitle{L'année sabbatique}
\VerseOne{}Yahweh parla aussi à Moïse sur la montagne de Sinaï, en disant :
\VS{2}Parle aux enfants d'Israël, et dis-leur : Quand vous serez entrés dans le pays que je vous donne, la terre se reposera : Ce sera un sabbat à Yahweh.
\VS{3}Pendant six ans tu sèmeras ton champ, et pendant six ans tu tailleras ta vigne ; et tu en recueilleras le produit.
\VS{4}Mais la septième année il y aura un sabbat, un temps de repos pour la terre, ce sera un sabbat à Yahweh : Tu ne sèmeras point ton champ, et tu ne tailleras point ta vigne.
\VS{5}Tu ne moissonneras point ce qui proviendra des grains tombés dans ta moisson, et tu ne vendangeras point les raisins de ta vigne non taillée : Ce sera une année de repos total pour la terre.
\VS{6}Mais ce qui proviendra de la terre l'année du sabbat vous servira de nourriture, à toi, à ton serviteur et à ta servante, à ton mercenaire et à l'étranger qui demeurent avec toi,
\VS{7}à ton bétail et aux animaux qui sont dans ton pays ; tout son produit servira de nourriture.
\TextTitle{L'année du jubilé}
\VS{8}Tu compteras aussi sept sabbats d'années, à savoir sept fois sept ans, et les jours de sept sabbats feront quarante-neuf ans.
\VS{9}Puis tu feras sonner le shofar de jubilation le dixième jour du septième mois ; le jour, dis-je, des expiations, vous ferez sonner le shofar dans tout votre pays.
\VS{10}Et vous sanctifierez la cinquantième année, et publierez la liberté dans le pays à tous ses habitants : Ce sera pour vous l'année du jubilé ; et vous retournerez chacun dans sa possession, et chacun dans sa famille.
\VS{11}Cette cinquantième année vous sera l'année du jubilé : Vous ne sèmerez point et vous ne moissonnerez point ce que la terre rapportera d'elle-même, et vous ne vendangerez point les fruits de la vigne non taillée.
\VS{12}Car c'est l'année du jubilé, elle vous sera sainte. Vous mangerez ce que les champs rapporteront cette année-là.
\VS{13}En cette année du jubilé chacun de vous retournera dans sa possession.
\VS{14}Et si tu fais une vente à ton prochain, ou si tu achètes quelque chose de ton prochain, que nul de vous ne trompe son frère.
\VS{15}Mais tu achèteras de ton prochain selon le nombre des années après le jubilé. Pareillement on te fera les ventes selon le nombre des années de rapport.
\VS{16}Selon qu'il y aura plus d'années, tu augmenteras le prix de ce que tu achètes ; et selon qu'il y aura moins d'années, tu le diminueras ; car on te vend le nombre des récoltes.
\VS{17}Que nul de vous ne trompe son prochain, mais craignez votre Dieu ; car je suis Yahweh, votre Dieu.
\VS{18}Pratiquez mes ordonnances, gardez mes jugements et observez-les, et vous habiterez en sécurité dans le pays.
\VS{19}Et le pays vous donnera ses fruits, vous en mangerez, vous en serez rassasiés, et vous y habiterez en sécurité.
\VS{20}Et si vous dites : Que mangerons-nous la septième année si nous ne semons point, et si nous ne recueillons point notre récolte ?
\VS{21}J'ordonnerai à ma bénédiction de se répandre sur vous dans la sixième année, et la terre rapportera pour trois ans.
\VS{22}Puis vous sèmerez la huitième année, et vous mangerez de l'ancienne récolte jusqu'à la neuvième année ; jusqu'à ce que sa récolte soit venue, vous mangerez de l'ancienne.
\VS{23}La terre ne sera point vendue à perpétuité ; car le pays est à moi, et vous êtes étrangers et forains\FTNT{Forain : Quelqu'un d'extérieur, d'étranger à un lieu.} chez moi.
\VS{24}C'est pourquoi dans tout le pays dont vous aurez la possession, vous donnerez le droit de rachat\FTNT{Pour voir un exemple de ce droit de rachat, voir Ru. 4:1-13.} pour la terre.
\TextTitle{Le droit de rachat}
\VS{25}Si ton frère est devenu pauvre et vend quelque chose de ce qu'il possède, celui qui a le droit de rachat, à savoir son plus proche parent, viendra et rachètera la chose vendue par son frère.
\VS{26}Si cet homme n'a personne qui ait le droit de rachat, et qu'il ait trouvé de lui-même suffisamment de quoi faire le rachat de ce qu'il a vendu,
\VS{27}il comptera les années du temps qu'il a fait la vente, et il restituera le surplus à l'homme auquel il l'avait faite, et ainsi il retournera dans sa possession.
\VS{28}Mais s'il n'a pas trouvé suffisamment de quoi lui rendre, la chose qu'il aura vendue sera dans les mains de celui qui l'aura acheté, jusqu'à l'année du jubilé ; puis l'acheteur en sortira au jubilé, et le vendeur retournera dans sa possession.
\VS{29}Et si quelqu'un a vendu une maison d'habitation dans quelque ville entourée de murs, il aura le droit de rachat jusqu'à la fin de l'année de sa vente ; son droit de rachat sera d'une année.
\VS{30}Mais si elle n'est point rachetée dans l'année accomplie, la maison qui est dans la ville entourée de murs, demeurera à l'acheteur absolument et à ses descendants ; il n'en sortira point au jubilé.
\VS{31}Mais les maisons des villages, qui ne sont point entourés de murs, seront comptées comme des fonds de terre ; le vendeur aura droit de rachat, et l'acheteur sortira au jubilé.
\VS{32}Et quant aux villes des Lévites, les Lévites auront un droit de rachat perpétuel des maisons des villes de leur possession.
\VS{33}Et celui qui achètera une maison des Lévites, sortira au jubilé de la maison vendue, qui est dans la ville de sa possession ; car les maisons des villes des Lévites sont leur possession parmi les enfants d'Israël.
\VS{34}Mais les champs situés autour des villes des Lévites ne seront point vendus ; car c'est leur possession perpétuelle.
\TextTitle{Les traitements du frère pauvre}
\VS{35}Quand ton frère sera devenu pauvre, et qu'il tendra vers toi ses mains tremblantes, tu le soutiendras, tu soutiendras aussi l'étranger, et le forain, afin qu'il vive avec toi.
\VS{36}Tu ne prendras point de lui d'usure ni d'intérêt, mais tu craindras ton Dieu, et ton frère vivra avec toi.
\VS{37}Tu ne lui prêteras point ton argent à intérêt ni ne lui prêteras de tes vivres pour en tirer du profit.
\VS{38}Je suis Yahweh, votre Dieu qui vous ai fait sortir du pays d'Egypte, pour vous donner le pays de Canaan, afin d'être votre Dieu.
\VS{39}Pareillement, quand ton frère sera devenu pauvre auprès de toi, et qu'il se sera vendu à toi, tu ne te serviras point de lui comme on se sert des esclaves.
\VS{40}Mais il sera chez toi comme serait le mercenaire et l'étranger, et il te servira jusqu'à l'année du jubilé.
\VS{41}Alors il sortira de chez toi avec ses fils, il s'en retournera dans sa famille, et rentrera dans la possession de ses pères.
\VS{42}Car ils sont mes serviteurs, parce que je les ai retirés du pays d'Egypte ; c'est pourquoi ils ne seront point vendus comme on vend les esclaves.
\VS{43}Tu ne domineras point sur lui avec dureté, et tu craindras ton Dieu.
\VS{44}C'est des nations qui vous entourent que tu prendras ton esclave et ta servante qui t'appartiendront ; c'est d'elles que vous achèterez l'esclave et la servante.
\VS{45}Vous pourrez aussi en acheter des fils des étrangers qui demeureront chez toi, et même de leurs familles qui seront parmi vous, qui auront engendré dans votre pays, et vous les posséderez.
\VS{46}Vous les aurez comme un héritage pour les laisser à vos enfants après vous, afin qu'ils en héritent la possession, et vous vous servirez d'eux à perpétuité. Mais quant à vos frères, les fils d'Israël, nul ne dominera avec dureté sur son frère.
\VS{47}Et lorsque l'étranger ou le forain qui est avec toi se sera enrichi, et que ton frère qui est avec lui sera devenu si pauvre qu'il se soit vendu à l'étranger, ou au forain qui est avec toi, ou à quelqu'un de la postérité de la famille de l'étranger,
\VS{48}après s'être vendu, il y aura droit de rachat pour lui : Un de ses frères le rachètera.
\VS{49}Son oncle, ou le fils de son oncle, ou quelque autre proche parent de son sang d'entre ceux de sa famille, le rachètera ; ou lui-même, s'il en trouve le moyen, se rachètera.
\VS{50}Et il comptera avec son acheteur depuis l'année qu'il s'est vendu à lui, jusqu'à l'année du jubilé ; de sorte que l'argent du prix pour lequel il s'est vendu, se comptera à raison du nombre des années, le temps qu'il aura servi lui sera compté comme les journées d'un mercenaire.
\VS{51}S'il y a encore plusieurs années, il restituera le prix de son achat à raison de ces années, selon le prix pour lequel il a été acheté ;
\VS{52}et s'il reste peu d'années jusqu'à l'année du jubilé, il comptera avec lui, et restituera le prix de son achat à raison des années qu'il a servi.
\VS{53}Il aura été avec lui comme un mercenaire qui se loue d'année en année, et cet étranger ne dominera point sur lui avec dureté en ta présence.
\VS{54}S'il n'est pas racheté par quelqu'un de ces moyens, il sortira l'année du jubilé, lui et ses fils avec lui.
\VS{55}Car c'est de moi que les enfants d'Israël sont esclaves ; ce sont mes esclaves que j'ai fait sortir du pays d'Egypte. Je suis Yahweh, votre Dieu.
\Chap{26}
\TextTitle{Mise en garde contre le péché}
\VerseOne{}Vous ne vous ferez point d'idoles, vous ne vous dresserez point d'image taillée, ni de statue, et vous ne mettrez point de pierre sculptée dans votre pays, pour vous prosterner devant elles ; car je suis Yahweh, votre Dieu.
\VS{2}Vous garderez mes sabbats et vous révérerez mon sanctuaire. Je suis Yahweh.
\TextTitle{La bénédiction conditionnelle à l'obéissance à Yahweh}
\VS{3}Si vous marchez dans mes ordonnances et si vous gardez mes commandements et les pratiquez,
\VS{4}je vous donnerai les pluies en leur temps, la terre donnera ses produits, et les arbres des champs donneront leurs fruits.
\VS{5}Le foulage des grains atteindra la vendange chez vous, et la vendange atteindra les semailles ; vous mangerez votre pain à satiété et vous habiterez en sécurité dans votre pays.
\VS{6}Je donnerai la paix au pays, vous dormirez sans que personne ne vous trouble ; je ferai disparaître les bêtes méchantes du pays, et l'épée ne passera point par votre pays.
\VS{7}Vous poursuivrez vos ennemis, et ils tomberont par l'épée devant vous.
\VS{8}Cinq d'entre vous en poursuivront cent, et cent en poursuivront dix mille, et vos ennemis tomberont par l'épée devant vous.
\VS{9}Je me tournerai vers vous, je vous ferai fructifier et multiplier, et j'établirai mon alliance avec vous.
\VS{10}Vous mangerez de vieilles provisions, et vous sortirez le vieux pour y loger le nouveau.
\VS{11}Même, je mettrai mon tabernacle au milieu de vous, et mon âme ne vous aura point en horreur.
\VS{12}Mais je marcherai au milieu de vous, je serai votre Dieu, et vous serez mon peuple.
\VS{13}Je suis Yahweh, votre Dieu, qui vous ai fait sortir du pays d'Egypte, afin que vous ne soyez point leurs esclaves ; j'ai brisé les liens de votre joug, et je vous ai fait marcher la tête levée.
\TextTitle{Les châtiments en cas de désobéissance à Yahweh}
\VS{14}Mais si vous ne m'écoutez point et que vous ne pratiquez pas tous ces commandements,
\VS{15}et si vous rejetez mes ordonnances, et que votre âme a en horreur mes jugements, afin de ne point pratiquer tous mes commandements, et que vous rompiez mon alliance,
\TextTitle{La domination par les ennemis}
\VS{16}aussi je vous ferai ceci : Je répandrai sur vous la frayeur, la langueur et l'ardeur, qui vous consumerons les yeux et feront languir votre âme ; et vous sèmerez en vain votre semence car vos ennemis la mangeront.
\VS{17}Je tournerai ma face contre vous, vous serez battus devant vos ennemis ; ceux qui vous haïssent domineront sur vous ; et vous fuirez sans que personne ne vous poursuive.
\TextTitle{Le manque de fertilité de la terre}
\VS{18}Si après ces choses vous ne m'écoutez point, je vous châtierai sept fois plus à cause de vos péchés.
\VS{19}Je briserai l'orgueil de votre force et je ferai que votre ciel soit pour vous comme du fer, et votre terre comme de l'airain.
\VS{20}Votre force se consumera en vain, votre terre ne donnera point ses produits, et les arbres de la terre ne donneront point leurs fruits.
\TextTitle{Les attaques des bêtes des champs}
\VS{21}Si vous marchez en opposition avec moi et que vous ne voulez point m'écouter, je vous frapperai sept fois plus, selon vos péchés.
\VS{22}J'enverrai contre vous les bêtes des champs, qui vous priveront de vos enfants, qui détruiront votre bétail, et vous réduiront à un petit nombre, et vos chemins seront déserts.
\TextTitle{La peste}
\VS{23}Si après ces choses, vous ne recevez pas ma correction, et que vous marchiez en opposition avec moi,
\VS{24}je marcherai aussi en opposition avec vous, et je vous frapperai sept fois plus, selon vos péchés.
\VS{25}Et je ferai venir sur vous l'épée qui fera la vengeance de mon alliance ; et quand vous vous rassemblerez dans vos villes, j'enverrai la peste au milieu de vous, et vous serez livrés entre les mains de l'ennemi.
\TextTitle{Le manque de nourriture}
\VS{26}Lorsque je vous briserai le bâton du pain, dix femmes cuiront votre pain dans un seul four, et vous rendront votre pain au poids ; vous en mangerez, et vous n'en serez point rassasiés.
\VS{27}Si avec cela vous ne m'écoutez point, et que vous marchiez en opposition avec moi,
\VS{28}je marcherai aussi en opposition avec vous, avec fureur, et je vous châtierai aussi sept fois plus, selon vos péchés ;
\VS{29}vous mangerez la chair de vos fils, et vous mangerez aussi la chair de vos filles\FTNT{La. 4:10.}.
\VS{30}Je détruirai vos hauts lieux, j'abattrai vos statues consacrées au soleil, je mettrai vos cadavres sur les cadavres de vos idoles, et mon âme vous aura en horreur.
\VS{31}Je réduirai vos villes en désert, je dévasterai vos sanctuaires, et je ne respirerai plus l'agréable odeur de vos parfums.
\TextTitle{La dispersion dans les nations\FTNTT{De. 28:58-67.}}
\VS{32}Je dévasterai le pays, et vos ennemis qui l'habiteront, en seront étonnés.
\VS{33}Je vous disperserai parmi les nations, et je tirerai l'épée après vous, et votre pays sera dévasté, et vos villes désertes.
\VS{34}Alors la terre prendra plaisir à ses sabbats\FTNT{2 Ch. 36:21.}, tout le temps qu'elle sera dévastée, et lorsque vous serez dans le pays de vos ennemis, la terre se reposera et prendra plaisir à ses sabbats.
\VS{35}Tout le temps qu'elle sera dévastée, elle se reposera parce qu'elle ne s'était point reposée dans vos sabbats, lorsque vous y habitiez.
\VS{36}Et quant à ceux d'entre vous qui survivront dans le pays de leurs ennemis, je rendrai leur cœur lâche, de sorte que le bruit d'une feuille agitée les poursuivra, ils fuiront comme on fuit devant l'épée, et ils tomberont sans que personne ne les poursuive.
\VS{37}Et ils trébucheront les uns sur les autres comme devant l'épée, sans que personne ne les poursuive ; et vous ne tiendrez point devant vos ennemis ;
\VS{38}vous périrez parmi les nations, et le pays de vos ennemis vous consumera.
\VS{39}Et ceux d'entre vous qui survivront, se fondront à cause de leurs iniquités, dans les pays de vos ennemis ; ils se fondront aussi à cause des iniquités de leurs pères.
\TextTitle{Repentance et restauration de l'alliance d'Abraham, d'Isaac et de Jacob}
\VS{40}Alors, ils confesseront leurs iniquités et les iniquités de leurs pères, selon les transgressions qu'ils auront commises contre moi ; et aussi parce qu'ils auront marché en opposition avec moi.
\VS{41}Moi aussi, je marcherai en opposition avec eux, je les amènerai dans le pays de leurs ennemis. Et alors, leur cœur incirconcis s'humiliera, et ils recevront la peine de leur iniquité.
\VS{42}Et alors je me souviendrai de mon alliance avec Jacob, et de mon alliance avec Isaac, et je me souviendrai aussi de mon alliance avec Abraham, et je me souviendrai de la terre.
\VS{43}Quand donc la terre sera abandonnée par eux, et prendra plaisir à ses sabbats, ayant été désolée à cause d'eux ; et qu'ils recevront la peine de leur iniquité, parce qu'ils ont rejeté mes ordonnances, et que leur âme a eu mes ordonnances en horreur.
\VS{44}Je m'en souviendrai, dis-je, lorsqu'ils seront dans le pays de leurs ennemis, je ne les rejetterai point, et je ne les aurai point en horreur pour les consumer entièrement jusqu'à rompre mon alliance avec eux ; car je suis Yahweh, leur Dieu.
\VS{45}Je me souviendrai en leur faveur de la Première Alliance, par laquelle je les ai fait sortir du pays d'Egypte, aux yeux des nations, pour être leur Dieu. Je suis Yahweh.
\VS{46}Ce sont là les statuts, les ordonnances, et les lois que Yahweh établit entre lui et les enfants d'Israël sur la montagne de Sinaï, par Moïse.
\Chap{27}
\TextTitle{Lois des personnes et des biens voués à Yahweh}
\VerseOne{}Yahweh parla aussi à Moïse, en disant :
\VS{2}Parle aux enfants d'Israël, et dis-leur : quand quelqu'un aura fait un vœu important, les personnes vouées à Yahweh seront mises à ton estimation.
\VS{3}Et l'estimation que tu feras d'un homme, depuis l'âge de vingt ans jusqu'à l'âge de soixante ans, sera du prix de cinquante sicles d'argent, selon le sicle du sanctuaire.
\VS{4}Mais si c'est une femme, alors ton estimation sera de trente sicles.
\VS{5}Si c'est un homme de cinq ans jusqu'à vingt ans, alors ton estimation sera de vingt sicles ; et quant à la femme, de dix sicles.
\VS{6}Et si c'est un homme d'un mois jusqu'à cinq ans, ton estimation sera de cinq sicles d'argent ; et l'estimation d'une femme sera de trois sicles d'argent.
\VS{7}Et lorsque c'est un homme de soixante ans et au-dessus, ton estimation sera de quinze sicles ; et si c'est une femme, de dix sicles.
\VS{8}Et si celui qui a fait le vœu est plus pauvre que ton estimation, on le présentera devant le prêtre, qui en fera l'estimation, et le prêtre fera l'estimation selon les ressources de celui qui a fait le vœu.
\VS{9}Si c'est d'une des bêtes que l'on présente en offrande à Yahweh, tout ce qu'on donnera à Yahweh de la sorte sera saint.
\VS{10}On ne la changera point, et on n'en mettra point une autre à la place, d'une bonne pour une mauvaise, ou une mauvaise pour une bonne ; si l'on remplace une bête par une autre bête, elles seront l'une et l'autre chose sainte.
\VS{11}Si c'est d'une bête impure, qu'on ne peut présenter en offrande à Yahweh, on présentera la bête devant le prêtre,
\VS{12}qui en fera l'évaluation selon qu'elle sera bonne ou mauvaise, et il en sera fait ainsi, selon l'estimation du prêtre.
\VS{13}Mais si on veut la racheter, on ajoutera un cinquième à ton estimation.
\VS{14}Et quand quelqu'un sanctifiera sa maison pour être sainte à Yahweh, le prêtre l'estimera selon qu'elle sera bonne ou mauvaise, et on se tiendra à l'estimation que le prêtre en aura faite.
\VS{15}Mais si celui qui l'a sanctifiée veut racheter sa maison, il ajoutera par-dessus un cinquième de l'argent de ton estimation, et elle lui appartiendra.
\VS{16}Et si l'homme sanctifie à Yahweh une partie du champ de sa possession, ton estimation sera selon ce qu'on y sème, le homer de semence d'orge à cinquante sicles d'argent.
\VS{17}S'il a sanctifié son champ dès l'année du jubilé, on s'en tiendra à ton estimation ;
\VS{18}mais s'il sanctifie son champ après le jubilé, le prêtre estimera l'argent selon le nombre des années qui restent jusqu'à l'année du jubilé, et il sera fait une réduction sur ton estimation.
\VS{19}Et si celui qui a sanctifié le champ veut le racheter en quelque sorte que ce soit, il ajoutera par-dessus un cinquième de l'argent de ton estimation, et il lui restera.
\VS{20}Mais s'il ne rachète point le champ, et que le champ se vende à un autre homme, il ne se rachètera plus.
\VS{21}Et ce champ-là ayant passé le jubilé sera consacré à Yahweh, comme un champ d'interdit, la possession en sera au prêtre.
\VS{22}Et s'il sanctifie à Yahweh un champ qu'il ait acheté, qui ne soit point des champs de sa possession,
\VS{23}le prêtre lui comptera la somme de ton estimation jusqu'à l'année du jubilé, et il donnera en ce jour-là ton estimation, afin que ce soit une chose consacrée à Yahweh.
\VS{24}Mais l'année du jubilé, le champ retournera à celui de qui il avait été acheté, et auquel était la possession de la terre.
\VS{25}Et toute estimation que tu auras faite, sera selon le sicle du sanctuaire : Le sicle est de vingt guéras.
\TextTitle{Consécration des premiers-nés du bétail}
\VS{26}Toutefois, nul ne pourra consacrer le premier-né d'entre les bêtes, car il appartient à Yahweh par droit de primogéniture, soit de bœuf, soit d'agneau, il est à Yahweh.
\VS{27}Mais s'il s'agit d'une bête impure, il le rachètera selon ton estimation, et il ajoutera à ton estimation un cinquième ; et s'il n'est point racheté, il sera vendu selon ton estimation.
\TextTitle{Consécration des choses et personnes dévouées par interdit à Yahweh}
\VS{28}Or toute chose dévouée que quelqu'un dévouera à la façon de l'interdit à Yahweh, de tout ce qui est sien, soit homme, ou bête, ou champ de sa possession, ne se revendra ni ne se rachètera ; toute chose dévouée sera entièrement consacrée à Yahweh.
\VS{29}Nul interdit dévoué par interdit d'entre les hommes ne pourra être racheté, mais on le fera mettre à mort.
\TextTitle{Consécration de la dîme de la terre et du bétail}
\VS{30}Toute dîme de la terre, tant du grain de la terre que du fruit des arbres, est à Yahweh ; c'est une chose consacrée à Yahweh.
\VS{31}Mais si quelqu'un veut racheter en quelque sorte que ce soit quelque chose de sa dîme, il y ajoutera un cinquième par-dessus.
\VS{32}Mais toute dîme de bœufs, de brebis et de chèvres, à savoir tout ce qui passe sous la verge, le dixième en sera consacrée à Yahweh.
\VS{33}On ne choisira point le bon ou le mauvais, et l'on ne fera point d'échange ; si on l'échange, la bête changée et l'autre seront consacrées, et ne seront point rachetées.
\VS{34}Ce sont là les commandements que Yahweh donna à Moïse sur la montagne de Sinaï, pour les enfants d'Israël.
\PPE{}
\end{multicols}

%\clearpage\ShortTitle{No.}\BookTitle{Nombres}\BFont
\noindent\hrulefill
{\footnotesize
\textit{
\bigskip
{\centering{}
\\Auteur~: Probablement Moïse
\\(Heb.~: Bamidbar)
\\Signification~: Dans le désert
\\Thème~: Pérégrination dans le désert
\\Date de rédaction~: Env. 1450-1410 av. J.-C.\\}
}
\textit{
\\Ce livre commence par le recensement des fils d'Israël et relate trente-huit des quarante années qu'ils passèrent dans le désert du Sinaï. Il couvre une période qui s'étend de la deuxième année après la sortie d'Egypte à la veille de l'entrée en Canaan, terre que Dieu avait promis de donner à la descendance d'Abraham. Ce pays où coulaient le lait et le miel s'étendait de Sidon jusqu'à Lesha, en passant par Gaza et Sodome. En plus des Cananéens, il accueillait en son sein des enfants d'Anak, les Amalécites, les Hétiens, les Jébusiens et les Amoréens.
\\Ces écrits retracent les premières victoires d'Israël et regroupent diverses lois et instructions sur le partage de la terre promise. Ils témoignent également de la révolte et de l'incrédulité de la génération sortie d'Egypte dont la quasi-totalité périt dans le désert.\bigskip
}
}
\par\nobreak\noindent\hrulefill
\begin{multicols}{2}
\Chap{1}
\TextTitle{Dénombrement des hommes de guerre}
\VerseOne{}Or Yahweh parla à Moïse dans le désert de Sinaï, dans la tente d'assignation, le premier jour du second mois, la seconde année, après qu'ils furent sortis du pays d'Egypte, en disant~:
\VS{2}Faites le dénombrement de toute l'assemblée des fils d'Israël, selon leurs familles, selon les maisons de leurs pères, en comptant nom par nom, savoir tous les mâles\FTNT{Ex. 30:12~; Ex. 38:26.}, chacun par tête~;
\VS{3}depuis l'âge de vingt ans et au-dessus, tous ceux d'Israël qui peuvent aller à la guerre, vous les compterez selon leurs armées, toi et Aaron.
\VS{4}Il y aura avec vous un homme par tribu, celui qui est le chef de la maison de ses pères.
\VS{5}Voici les noms des hommes qui vous assisteront. Pour la tribu de Ruben~: Elitsur, fils de Schedéur~;
\VS{6}pour celle de Siméon~: Schelumiel, fils de Tsurischaddaï~;
\VS{7}pour celle de Juda~: Nachschon, fils d'Amminadab~;
\VS{8}pour celle d'Issacar~: Nethaneel, fils de Tsuar~;
\VS{9}pour celle de Zabulon~: Eliab, fils de Hélon~;
\VS{10}pour les fils de Joseph, pour la tribu d'Ephraïm~: Elischama, fils d'Ammihud~; pour celle de Manassé~: Gamliel, fils de Pedahtsur~;
\VS{11}pour la tribu de Benjamin~: Abidan, fils de Guideoni~;
\VS{12}pour celle de Dan~: Ahiézer, fils d'Ammischaddaï~;
\VS{13}pour celle d'Aser~: Paguiel, fils d'Ocran~;
\VS{14}pour celle de Gad~: Eliasaph, fils de Déuel~;
\VS{15}pour celle de Nephthali~: Ahira, fils d'Enan.
\VS{16}C'étaient là ceux qu'on appelait pour tenir l'assemblée~; ils étaient les princes des tribus de leurs pères, chefs des milliers d'Israël.
\VS{17}Alors Moïse et Aaron prirent ces hommes qui avaient été désignés par leurs noms,
\VS{18}et ils convoquèrent toute l'assemblée, le premier jour du second mois. On les enregistra selon leurs familles et selon la maison de leurs pères, en comptant les noms depuis l'âge de vingt ans et au-dessus, chacun par tête.
\VS{19}Comme Yahweh l'avait commandé à Moïse, il les dénombra au désert de Sinaï.
\VS{20}Les fils donc de Ruben, premier-né d'Israël, selon leurs générations, leurs familles, et les maisons de leurs pères, dont on fit le dénombrement par leur nom, et par tête, savoir tous les mâles de l'âge de vingt ans, et au dessus, tous ceux qui pouvaient aller à la guerre.
\VS{21}Ceux, dis-je, de la tribu de Ruben, qui furent dénombrés, furent quarante-six mille cinq cents.
\VS{22}Des enfants de Siméon, selon leurs générations, leurs familles, et les maisons de leurs pères, ceux qui furent dénombrés par leur nom et par tête, savoir tous les mâles de l'âge de vingt ans, et au dessus, tous ceux qui pouvaient aller à la guerre~;
\VS{23}ceux, dis-je, de la tribu de Siméon, qui furent dénombrés, furent cinquante-neuf mille trois cents.
\VS{24}Des fils de Gad, selon leurs générations, leurs familles, et les maisons de leurs pères, dénombrés chacun par leur nom, depuis l'âge de vingt ans, et au dessus, tous ceux qui pouvaient aller à la guerre~;
\VS{25}ceux, dis-je, de la tribu de Gad, qui furent dénombrés, furent quarante-cinq mille six cent cinquante.
\VS{26}Des enfants de Juda, selon leurs générations, leurs familles, et les maisons de leurs pères, dénombrés chacun par leur nom, depuis l'âge de vingt ans, et au dessus, tous ceux qui pouvaient aller à la guerre~;
\VS{27}ceux, dis-je, de la tribu de Juda, qui furent dénombrés, furent soixante-quatorze mille six cents.
\VS{28}Des fils d'Issacar, selon leurs générations, leurs familles, et les maisons de leurs pères, dénombrés chacun par leur nom, depuis l'âge de vingt ans, et au dessus, tous ceux qui pouvaient aller à la guerre~;
\VS{29}ceux, dis-je, de la tribu d'Issacar, qui furent dénombrés, furent cinquante-quatre mille quatre cents.
\VS{30}Des enfants de Zabulon, selon leurs générations, leurs familles, et les maisons de leurs pères, dénombrés chacun par leur nom, depuis l'âge de vingt ans, et au dessus, tous ceux qui pouvaient aller à la guerre~;
\VS{31}ceux, dis-je, de la tribu de Zabulon, qui furent dénombrés, furent cinquante-sept mille quatre cents.
\VS{32}Quant aux fils de Joseph~; les fils d'Ephraïm, selon leurs générations, leurs familles, et les maisons de leurs pères, dénombrés chacun par leur nom, depuis l'âge de vingt ans, et au dessus, tous ceux qui pouvaient aller à la guerre~;
\VS{33}ceux, dis-je, de la tribu d'Ephraïm, qui furent dénombrés, furent quarante mille cinq cents.
\VS{34}Des fils de Manassé, selon leurs générations, leurs familles, et les maisons de leurs pères, dénombrés chacun par leur nom, depuis l'âge de vingt ans, et au dessus, tous ceux qui pouvaient aller à la guerre~;
\VS{35}ceux, dis-je, de la tribu de Manassé, qui furent dénombrés, furent trente-deux mille deux cents.
\VS{36}Des fils de Benjamin, selon leurs générations, leurs familles, et les maisons de leurs pères, dénombrés chacun par leur nom, depuis l'âge de vingt ans, et au dessus, tous ceux qui pouvaient aller à la guerre~;
\VS{37}ceux, dis-je, de la tribu de Benjamin, qui furent dénombrés, furent trente-cinq mille quatre cents.
\VS{38}Des fils de Dan, selon leurs générations, leurs familles, et les maisons de leurs pères, dénombrés chacun par leur nom, depuis l'âge de vingt ans, et au dessus, tous ceux qui pouvaient aller à la guerre~;
\VS{39}ceux, dis-je, de la tribu de Dan qui furent dénombrés, furent soixante-deux mille sept cents.
\VS{40}Des fils d'Aser, selon leurs générations, leurs familles, et les maisons de leurs pères, dénombrés chacun par leur nom, depuis l'âge de vingt ans, et au dessus, tous ceux qui pouvaient aller à la guerre~;
\VS{41}ceux, dis-je, de la tribu d'Aser, qui furent dénombrés, furent quarante et un mille cinq cents.
\VS{42}Des fils de Nephthali, selon leurs générations, leurs familles, et les maisons de leurs pères, dénombrés chacun par leur nom, depuis l'âge de vingt ans, et au dessus, tous ceux qui pouvaient aller à la guerre~;
\VS{43}ceux, dis-je, de la tribu de Nephthali, qui furent dénombrés, furent cinquante-trois mille quatre cents.
\VS{44}Ce sont là ceux dont Moïse et Aaron firent le dénombrement, les douze princes d'entre les enfants d'Israël y étant, un pour chaque maison de leurs pères.
\VS{45}Ainsi tous ceux des enfants d'Israël, dont on fit le dénombrement, selon les maisons de leurs pères, depuis l'âge de vingt ans, et au dessus, tous ceux d'entre les Israélites, qui pouvaient aller à la guerre~;
\VS{46}tous ceux, dis-je, dont on fit le dénombrement, furent six cent trois mille cinq cent cinquante.
\VS{47}Mais les Lévites ne furent point dénombrés avec eux, selon la tribu de leurs pères.
\VS{48}Car Yahweh avait parlé à Moïse, en disant~:
\VS{49}Tu ne feras aucun dénombrement de la tribu de Lévi, et tu n'en lèveras point la somme avec les autres enfants d'Israël.
\VS{50}Mais tu donneras aux Lévites la charge du tabernacle du témoignage, et de tous ses ustensiles, et de tout ce qui lui appartient~; ils porteront le tabernacle, et tous ses ustensiles~; ils y serviront, et camperont autour du tabernacle.
\VS{51}Et quand le tabernacle partira, les Lévites le démonteront, et quand le tabernacle campera, les Lévites le dresseront. Que si quelque étranger en approche, on le fera mourir\FTNT{Ez. 44:8-9.}.
\VS{52}Or les enfants d'Israël camperont chacun dans son camp, et chacun sous sa bannière, selon leurs armées.
\VS{53}Mais les Lévites camperont autour du tabernacle du témoignage, afin qu'il n'y ait point d'indignation sur l'assemblée des enfants d'Israël, et ils prendront en leur charge le tabernacle du Témoignage.
\VS{54}Et les enfants d'Israël firent selon toutes les choses que Yahweh avait commandées à Moïse~; ils le firent ainsi.
\Chap{2}
\TextTitle{Disposition du camp d'Israël par tribu}
\VerseOne{}Et Yahweh parla à Moïse et à Aaron, en disant~:
\VS{2}Les enfants d'Israël camperont chacun sous sa bannière, avec les enseignes des maisons de leurs pères, tout autour de la tente d'assignation, vis-à-vis de lui.
\VS{3}Ceux de la bannière du camp de Juda camperont droit vers l'est, selon ses armées~; et Nachschon, fils d'Amminadab, sera le chef des fils de Juda~;
\VS{4}et son armée, et ses dénombrés, soixante-quatorze mille six cents.
\VS{5}Près de lui campera la tribu d'Issacar, et Nethanaël, fils de Tsuar, sera le chef des enfants d'Issacar~;
\VS{6}et son armée, et ses dénombrés, cinquante-quatre mille quatre cents.
\VS{7}Puis la tribu de Zabulon, et Eliab, fils de Hélon, sera le chef des enfants de Zabulon~;
\VS{8}et son armée, et ses dénombrés, cinquante-sept mille quatre cents.
\VS{9}Tous les dénombrés du camp de Juda, cent quatre-vingt-six mille quatre cents, selon leurs armées, partiront les premiers.
\VS{10}La bannière du camp de Ruben, selon ses armées, sera vers le sud, et Elitsur, fils de Schedéur, sera le chef des enfants de Ruben~;
\VS{11}et son armée, et ses dénombrés, quarante-six mille cinq cents.
\VS{12}Près de lui campera la tribu de Siméon, et Schelumiel, fils de Tsurischaddaï, sera le chef des enfants de Siméon~;
\VS{13}et son armée, et ses dénombrés, cinquante-neuf mille trois cents.
\VS{14}Puis la tribu de Gad, et Eliasaph, fils de Déuel, sera le chef des enfants de Gad~;
\VS{15}et son armée, et ses dénombrés, quarante-cinq mille six cent cinquante.
\VS{16}Tous les dénombrés du camp de Ruben, cent cinquante et un mille quatre cent cinquante, selon leurs armées, partiront les seconds.
\VS{17}Ensuite la tente d'assignation partira avec le camp des Lévites, au milieu des camps qui partiront comme ils auront campés, chacune en sa place, selon leurs bannières.
\VS{18}La bannière du camp d'Ephraïm, selon ses armées, sera vers l'occident~; et Elischama, fils de Ammihud, sera le chef des enfants d'Ephraïm~;
\VS{19}et son armée, et ses dénombrés, quarante mille cinq cents.
\VS{20}Près de lui campera la tribu de Manassé, et Gamliel, fils de Pedahtsur, sera le chef des fils de Manassé~;
\VS{21}et son armée, et ses dénombrés, trente-deux mille deux cents.
\VS{22}Puis la tribu de Benjamin, et Abidan, fils de Guideoni, sera le chef des fils de Benjamin~;
\VS{23}et son armée, et ses dénombrés, trente-cinq mille et quatre cents.
\VS{24}Tous les dénombrés pour le camp d'Ephraïm, cent huit mille et cent, selon leurs armées, partiront les troisièmes.
\VS{25}La bannière du camp de Dan, selon ses armées, sera vers le nord, et Ahiézer, fils de Ammischaddaaï, sera le chef des fils de Dan~;
\VS{26}et son armée, et ses dénombrés, soixante-deux mille sept cents.
\VS{27}Près de lui campera la tribu d'Aser, et Paguiel, fils de Ocran, sera le chef des fils d'Aser~;
\VS{28}et son armée, et ses dénombrés, quarante et un mille cinq cents.
\VS{29}Puis la tribu de Nephthali, et Ahira, fils d'Enan, sera le chef des fils de Nephthali~;
\VS{30}et son armée, et ses dénombrés, cinquante-trois mille quatre cents.
\VS{31}Tous les dénombrés du camp de Dan, cent cinquante-sept mille six cents, partiront les derniers des bannières.
\VS{32}Ce sont là ceux des enfants d'Israël dont on fit le dénombrement selon les maisons de leurs pères. Tous les dénombrés des camps selon leurs armées furent six cent trois mille cinq cent cinquante.
\VS{33}Mais les Lévites ne furent point dénombrés avec les autres enfants d'Israël, comme Yahweh l'avait commandé à Moïse.
\VS{34}Et les enfants d'Israël firent selon toutes les choses que Yahweh avait commandées à Moïse, et campèrent ainsi selon leurs bannières, et partirent ainsi, chacun selon leurs familles, et selon la maison de leurs pères.
\Chap{3}
\TextTitle{Organisation des prêtres et des Lévites}
\VerseOne{}Or ce sont ici les générations d'Aaron et de Moïse, au temps que Yahweh parla à Moïse sur la montagne de Sinaï.
\VS{2}Et ce sont ici les noms des fils d'Aaron~; Nadab, qui était l'aîné, Abihu, Eléazar, et Ithamar.
\VS{3}Ce sont là les noms des fils d'Aaron, les prêtres, qui furent oints et consacrés pour exercer la prêtrise\FTNT{Ex. 40:15~; Lé. 8:30.}.
\VS{4}Mais Nadab et Abihu moururent en la présence de Yahweh, quand ils offrirent un feu étranger devant Yahweh au désert de Sinaï, et ils n'eurent point d'enfants~; mais Eléazar et Ithamar exercèrent la prêtrise en la présence d'Aaron leur père\FTNT{Lé. 10:1-2~; 1 Ch. 24:2.}.
\VS{5}Yahweh parla à Moïse, en disant~:
\VS{6}Fais approcher la tribu de Lévi, et fais qu'elle se tienne devant Aaron, le prêtre, afin qu'ils le servent.
\VS{7}Et qu'ils aient la charge de ce qu'il leur ordonnera de garder, et de ce que toute l'assemblée leur ordonnera de garder, devant la tente d'assignation, en faisant le service du tabernacle.
\VS{8}Et qu'ils gardent tous les ustensiles de la tente d'assignation, et ce qui leur sera donné en charge par les enfants d'Israël, pour faire le service du tabernacle.
\VS{9}Ainsi tu donneras les Lévites à Aaron et à ses fils~; ils lui sont complètement donnés d'entre les enfants d'Israël.
\VS{10}Tu établiras donc Aaron et ses fils, et ils exerceront leur prêtrise. Que si quelque étranger en approche, on le fera mourir.
\VS{11}Et Yahweh parla à Moïse, en disant~:
\VS{12}Voici, j'ai pris les Lévites d'entre les enfants d'Israël, à la place de tout premier-né qui ouvre la matrice parmi les enfants d'Israël~; c'est pourquoi les Lévites seront à moi.
\VS{13}Car tout premier-né m'appartient, depuis le jour où je frappai tout premier-né au pays d'Egypte~; je me suis sanctifié tout premier-né en Israël, depuis les hommes jusqu'aux bêtes~; ils seront à moi, je suis Yahweh\FTNT{Ex. 13:2~; Ex. 22:29~; Ex. 34:19~; Lé. 27:26.}.
\TextTitle{Les familles des Lévites}
\VS{14}Yahweh parla aussi à Moïse au désert de Sinaï, en disant~:
\VS{15}Dénombre les enfants de Lévi, par les maisons de leurs pères, et par leurs familles, en comptant tout mâle depuis l'âge d'un mois, et au dessus.
\VS{16}Et Moïse les dénombra, selon le commandement de Yahweh, ainsi qu'il lui avait été ordonné.
\VS{17}Or ce sont ici les fils de Lévi selon leurs noms~: Guerschon, Kehath, et Merari.
\VS{18}Et ce sont ici les noms des fils de Guerschon, selon leurs familles, Libni, et Schimeï.
\VS{19}Et les fils de Kehath selon leurs familles, Amram, Jitsehar, Hébron et Uziel~;
\VS{20}et les fils de Merari, selon leurs familles, Machli et Muschi~; ce sont là les familles de Lévi, selon les maisons de leurs pères.
\VS{21}De Guerschon est sortie la famille de Libni, et la famille de Schimeï~; ce sont les familles des Guerschonites.
\VS{22}Ceux dont on fit le dénombrement, en comptant de tous les mâles depuis l'âge d'un mois et au dessus, furent au nombre de sept mille cinq cents.
\VS{23}Les familles des Guerschonites camperont derrière le tabernacle à l'occident.
\VS{24}Et Eliasaph, fils de Laël, sera le chef de la maison des pères des Guerschonites.
\TextTitle{Les fonctions des Lévites}
\VS{25}Et les fils de Guerschon auront en charge à la tente d'assignation, la tente, le tabernacle, sa couverture, le rideau de l'entrée de la tente d'assignation.
\VS{26}Et les courtines du parvis avec le rideau de l'entrée du parvis, qui servent pour tabernacle et pour l'autel, tout autour, et son cordage, pour tout son service.
\VS{27}Et de Kehath est sortie la famille des Amramites, la famille des Jitseharites, la famille des Hébronites, et la famille des Uziélites~; ce furent là les familles des Kehathites,
\VS{28}dont tous les mâles depuis l'âge d'un mois, et au dessus, furent au nombre de huit mille six cents, ayant la charge du sanctuaire.
\VS{29}Les familles des fils de Kehath camperont du côté du tabernacle vers le sud.
\VS{30}Et Elitsaphan, fils d'Uziel, sera le chef de la maison des pères des familles des Kehathites.
\VS{31}Et ils auront en charge l'arche, la table, le chandelier, les autels, et les ustensiles du sanctuaire avec lesquels on fait le service, et le rideau, avec tout ce qui y sert.
\VS{32}Et le chef des chefs des Lévites sera Eléazar, fils d'Aaron, le prêtre~; qui aura la surveillance sur ceux qui auront la charge du sanctuaire.
\VS{33}Et de Merari est sortie la famille des Machlites, et la famille des Muschi~; ce furent là les familles de Merari~;
\VS{34}ceux dont on fit le dénombrement, après le compte qui fut fait de tous les mâles, depuis l'âge d'un mois et au dessus, furent six mille deux cents.
\VS{35}Et Esuriel, fils d'Abihaïl, sera le chef de la maison des pères des familles des Merarites~; ils camperont du côté du tabernacle vers le nord.
\VS{36}Et on donnera aux enfants de Merari la surveillance des planches du tabernacle, de ses barres, de ses piliers, de ses bases, et de tous ses ustensiles, avec tout ce qui y sera~;
\VS{37}et des piliers du parvis tout autour, avec leurs bases, leurs pieux, et leurs cordes.
\VS{38}Et Moïse, et Aaron et ses fils campaient devant le tabernacle, à l'orient, devant la tente d'assignation, vers l'orient~; ils avaient la garde et le soin du sanctuaire, remis à la garde des enfants d'Israël~; et si quelque étranger en approche, on le fera mourir.
\VS{39}Tous ceux des Lévites dont on fit le dénombrement, lesquels Moïse et Aaron comptèrent par leurs familles, suivant le commandement de Yahweh, tous les mâles de l'âge d'un mois et au dessus, furent de vingt-deux mille.
\TextTitle{Le rachat des premiers-nés}
\VS{40}Yahweh dit à Moïse~: Fais le dénombrement de tous les premiers-nés mâles des enfants d'Israël, depuis l'âge d'un mois, et au dessus, et relève le nombre de leurs noms.
\VS{41}Et tu prendras pour moi, je suis Yahweh, les Lévites, à la place de tous les premiers-nés qui sont entre les enfants d'Israël~; tu prendras aussi les bêtes des Lévites, à la place de tous les premiers-nés des bêtes des enfants d'Israël.
\VS{42}Moïse fit le dénombrement, comme Yahweh lui avait commandé, de tous les premiers-nés qui étaient parmi les enfants d'Israël.
\VS{43}Et tous les premiers-nés des mâles, selon le nombre des noms, depuis l'âge d'un mois et au dessus, selon leur dénombrement, furent vingt-deux mille deux cent soixante-treize.
\VS{44}Et Yahweh parla à Moïse, en disant~:
\VS{45}Prends les Lévites à la place de tous les premiers-nés qui sont parmi les enfants d'Israël, et les bêtes des Lévites, à la place de leurs bêtes~; et les Lévites seront à moi~; je suis Yahweh.
\VS{46}Et quant à ceux qu'il faut racheter, les deux cent soixante-treize parmi les premiers-nés des fils d'Israël, qui sont de plus que les Lévites,
\VS{47}tu prendras cinq sicles par tête, tu les prendras selon le sicle du sanctuaire~; le sicle est de vingt guéras\FTNT{Ex. 30:13~; Lé. 27:6~; Lé. 27:25~; Ez. 45:12.}.
\VS{48}Et tu donneras à Aaron et à ses fils l'argent de ceux qui auront été rachetés, dépassant le nombre des Lévites.
\VS{49}Moïse donc prit l'argent du rachat de ceux qui étaient de plus, outre ceux qui avaient été rachetés par l'échange des Lévites.
\VS{50}Et il reçut l'argent des premiers-nés des enfants d'Israël, qui fut mille trois cent soixante-cinq sicles, selon le sicle du sanctuaire.
\VS{51}Et Moïse donna l'argent des rachetés à Aaron, et à ses fils, selon le commandement de Yahweh, ainsi que Yahweh le lui avait commandé.
\Chap{4}
\TextTitle{Les fonctions des fils de Kehath}
\VerseOne{}Et Yahweh parla à Moïse et à Aaron, en disant~:
\VS{2}Faites le dénombrement des fils de Kehath d'entre les enfants de Lévi par leurs familles, et par les maisons de leurs pères,
\VS{3}depuis l'âge de trente ans et au dessus, jusqu'à l'âge de cinquante ans, tous ceux qui entrent en rang, pour s'employer à la tente d'assignation.
\VS{4}C'est ici le service des fils de Kehath à la tente d'assignation, c'est-à-dire, le Saint des saints.
\VS{5}Quand le camp partira, Aaron et ses fils viendront démonter le voile\FTNT{Le voile intérieur est l'image du corps humain de Christ (Mt. 26:26). Ce voile fut déchiré de haut en bas lorsque le Seigneur est mort sur la croix (Mt. 27:50-51). Désormais, le croyant peut pénétrer dans la présence du Père (Hé. 10:19-20).} qui sert de rideau, et en couvriront l'arche du témoignage~;
\VS{6}puis ils mettront au dessus une couverture de peaux de taissons, ils étendront par dessus un drap de pourpre, et ils y mettront ses barres.
\VS{7}Et ils étendront un drap de pourpre sur la table des pains de proposition, et mettront sur elle les plats, les tasses, les bassins, et les calices de libations. Le pain continuel sera sur elle.
\VS{8}Ils étendront au dessus un drap teint de cramoisi, ils le couvriront d'une couverture de peaux de taissons, et ils y mettront ses barres.
\VS{9}Et ils prendront un drap de pourpre, en couvriront le chandelier du luminaire avec ses lampes, ses mouchettes, ses vases à cendre, et tous ses vases à huile, dont on fait usage pour son service\FTNT{Ex. 25:30-38.}~;
\VS{10}ils le mettront avec tous ses ustensiles, dans une couverture de peaux de taissons, et le mettront sur une perche.
\VS{11}Ils étendront sur l'autel d'or un drap de pourpre, ils le couvriront d'une couverture de peaux de taissons, et ils y mettront ses barres.
\VS{12}Ils prendront aussi tous les ustensiles du service dont on se sert dans le lieu saint, ils les mettront dans un drap de pourpre, et ils les couvriront d'une couverture de peaux de taissons, et les mettront sur des perches.
\VS{13}Ils ôteront les cendres de l'autel, et étendront dessus un drap de pourpre.
\VS{14}Et ils mettront dessus les ustensiles dont on se sert pour l'autel, les brasiers, les fourchettes, les pelles, les bassins, et tous les ustensiles de l'autel~; ils étendront dessus une couverture de peaux de taissons, et ils y mettront ses barres.
\VS{15}Le camp partira après qu'Aaron et ses fils auront achevé de couvrir le lieu saint et tous ses ustensiles, et après cela les fils de Kehath viendront pour le porter, et ils ne toucheront point les choses saintes, de peur qu'ils ne meurent~; c'est là ce que les fils de Kehath porteront de la tente d'assignation.
\TextTitle{Les fonctions d'Eléazar}
\VS{16}Et Eléazar fils d'Aaron, le prêtre, aura la surveillance de l'huile du luminaire, du parfum odoriférant, de l'offrande continuelle, et de l'huile de l'onction~; la charge de tout le tabernacle, et de toutes les choses qui sont dans le lieu saint, et de ses ustensiles\FTNT{Ex. 30:23-35.}.
\VS{17}Yahweh parla à Moïse et à Aaron, en disant~:
\VS{18}Ne retranchez pas la tribu des familles des Kehathites d'entre les Lévites.
\VS{19}Mais faites ceci pour eux, afin qu'ils vivent et ne meurent point~; c'est que quand ils approcheront du Saint des saints, Aaron et ses fils viendront, qui les placeront chacun à son service, et à sa charge.
\VS{20}Et ils n'entreront point pour regarder quand on enveloppera les choses saintes, afin qu'ils ne meurent point.
\TextTitle{Les fonctions des fils de Guerschon}
\VS{21}Yahweh parla à Moïse, en disant~:
\VS{22}Fais aussi le dénombrement des fils de Guerschon selon les maisons de leurs pères, et selon leurs familles~;
\VS{23}depuis l'âge de trente ans, et au dessus, jusqu'à l'âge de cinquante ans, dénombrant tous ceux qui entrent pour tenir leur rang, afin de s'employer à servir à la tente d'assignation.
\VS{24}C'est ici le service des familles des Guerschonites, ce à quoi, ils doivent servir et en ce qu'ils doivent porter.
\VS{25}Ils porteront donc les tapis du tabernacle, et la tente d'assignation, sa couverture, la couverture de peaux de taissons qui est sur lui par dessus, et le rideau de l'entrée de la tente d'assignation~;
\VS{26}les courtines du parvis, et le rideau de l'entrée de la porte du parvis, qui servent pour le tabernacle et pour l'autel tout autour, leurs cordages, et tous les ustensiles de leur service, et tout ce qui est fait pour eux~; c'est ce en quoi ils serviront.
\VS{27}Tout le service des fils de Guerschonites en tout ce qu'ils doivent porter, et en tout ce à quoi ils doivent servir, sera réglé par les ordres d'Aaron et de ses fils, et vous les chargerez d'observer tout ce qu'ils doivent porter.
\VS{28}C'est là le service des familles des fils des Guerschonites dans la tente d'assignation~; et leur charge sera sous la conduite d'Ithamar, fils d'Aaron, le prêtre.
\TextTitle{Les fonctions des fils de Merari}
\VS{29}Tu dénombreras aussi les fils de Mérari selon leurs familles et selon les maisons de leurs pères.
\VS{30}Tu les dénombreras depuis l'âge de trente ans et au dessus, jusqu'à l'âge de cinquante ans, tous ceux qui entrent en rang pour s'employer au service dans la tente d'assignation.
\VS{31}Or c'est ici la charge de ce qu'ils auront à porter, selon tout le service qu'ils auront à faire à la tente d'assignation, savoir les planches du tabernacle, ses barres, et ses piliers, avec ses bases\FTNT{Ex. 26:15.},
\VS{32}et les piliers du parvis tout autour, et leurs bases, leurs pieux, leurs cordages, tous leurs ustensiles, et tout ce dont on se sert en ces choses-là, et vous leur compterez, en les désignant par nom, tous les ustensiles, qu'ils auront en charge de porter, pièce par pièce.
\VS{33}C'est là le service des familles des fils de Merari, pour tout leur service à la tente d'assignation, sous la conduite d'Ithamar, fils d'Aaron, le prêtre.
\VS{34}Moïse, Aaron et les princes de l'assemblée dénombrèrent les fils des Kéhathites, selon leurs familles, et selon les maisons de leurs pères.
\VS{35}Depuis l'âge de trente ans, et au dessus, jusqu'à l'âge de cinquante ans, tous ceux qui entraient en rang pour servir à la tente d'assignation.
\VS{36}Et ceux dont on fit le dénombrement selon leurs familles, étaient deux mille sept cent cinquante.
\VS{37}Ce sont là les dénombrés des familles des Kéhathites, tous servant à la tente d'assignation, que Moïse et Aaron dénombrèrent selon le commandement que Yahweh avait fait par le moyen de Moïse.
\VS{38}Or quant aux dénombrés des fils de Guerschon selon leurs familles, et selon les maisons de leurs pères,
\VS{39}depuis l'âge de trente ans, et au dessus, jusqu'à l'âge de cinquante ans, tous ceux qui entraient en rang pour servir à la tente d'assignation,
\VS{40}ceux, dis-je, qui en furent dénombrés selon leurs familles, et selon les maisons de leurs pères, étaient deux mille six cent trente.
\VS{41}Ce sont là, les dénombrés des familles des fils de Guerschon, tous servant dans la tente d'assignation, que Moïse et Aaron dénombrèrent selon le commandement de Yahweh.
\VS{42}Et quant aux dénombrés des familles des fils de Merari, selon leurs familles, et selon les maisons de leurs pères,
\VS{43}depuis l'âge de trente ans, et au dessus, jusqu'à l'âge de cinquante ans, tous ceux qui entraient en rang, pour servir à la tente d'assignation~;
\VS{44}ceux, dis-je, qui en furent dénombrés selon leurs familles, étaient trois mille deux cents.
\VS{45}Ce sont là, les dénombrés des familles des fils de Merari, que Moïse et Aaron dénombrèrent selon le commandement que Yahweh avait fait par le moyen de Moïse.
\VS{46}Ainsi tous ces dénombrés, que Moïse, Aaron et les princes d'Israël dénombrèrent d'entre les Lévites, selon leurs familles, et selon les maisons de leurs pères~;
\VS{47}depuis l'âge de trente ans, et au dessus, jusqu'à l'âge de cinquante ans, tous ceux qui entraient en service pour s'employer en ce à quoi il fallait servir, et à ce qu'il fallait porter de la tente d'assignation.
\VS{48}Tous ceux, dis-je, qui en furent dénombrés, étaient huit mille cinq cent quatre-vingts.
\VS{49}On les dénombra selon le commandement que Yahweh en avait fait par le moyen de Moïse, chacun selon ce en quoi il avait à servir, et ce qu'il avait à porter, et la charge de chacun fut telle que Yahweh l'avait commandé à Moïse.
\Chap{5}
\TextTitle{Mise en garde contre toute souillure~; lois diverses}
\VerseOne{}Et Yahweh parla à Moïse, en disant~:
\VS{2}Ordonne aux enfants d'Israël qu'ils mettent hors du camp tout lépreux, tout homme ayant une gonorrhée, et tout homme souillé pour un mort\FTNT{Lé. 13~; Lé. 15.}.
\VS{3}Vous les mettrez dehors, tant l'homme que la femme, vous les mettrez, dis-je, hors du camp, afin qu'ils ne souillent point le camp au milieu duquel j'habite.
\VS{4}Et les enfants d'Israël firent ainsi, et les envoyèrent hors du camp, comme Yahweh l'avait dit à Moïse~; les enfants d'Israël firent ainsi.
\VS{5}Et Yahweh parla à Moïse, en disant~:
\VS{6}Parle aux enfants d'Israël~; quand un homme ou une femme aura commis un des péchés que l'homme commet en faisant un crime contre Yahweh, et qu'une telle personne en sera trouvée coupable~;
\VS{7}alors ils confesseront leur péché, qu'ils auront commis~; et le coupable restituera la somme totale de ce en quoi il aura été trouvé coupable, et il y ajoutera un cinquième par-dessus, et le donnera à celui contre qui il aura commis le délit.
\VS{8}Que si cet homme n'a personne à qui appartienne le droit de restituer pour retirer ce en quoi aura été commis le délit, cette chose-là sera restituée à Yahweh, et elle appartiendra au prêtre, outre le bélier expiatoire avec lequel on fera propitiation pour lui.
\VS{9}De même, toute offrande élevée d'entre toutes les choses sanctifiées des enfants d'Israël, qu'ils présenteront\FTNT{Ez. 44:30.} au prêtre, lui appartiendra.
\VS{10} Les choses donc que quelqu'un aura sanctifiées appartiendront au prêtre~; ce que chacun lui aura donné, lui appartiendra\FTNT{Lé. 10:12-13.}.
\VS{11}Yahweh parla à Moïse, en disant~:
\VS{12}Parle aux enfants d'Israël, et dis leur~: Si la femme de quelqu'un se détourne et lui devienne infidèle~;
\VS{13}et que quelqu'un aura couché avec elle, et l'aura connue, sans que son mari en ait rien su, mais qu'elle se soit cachée, et qu'elle se soit souillée, et qu'il n'y ait point de témoin contre elle, et qu'elle n'ait point été surprise~;
\VS{14}et que l'esprit de jalousie saisisse son mari, tellement qu'il soit jaloux de sa femme, parce qu'elle s'est souillée~; ou que l'esprit de jalousie le saisisse tellement, qu'il soit jaloux de sa femme, encore qu'elle ne se soit point souillée~;
\VS{15}cet homme-là fera venir sa femme devant le prêtre, et il apportera l'offrande de cette femme pour elle, savoir la dixième partie d'un epha de farine d'orge~; mais il ne répandra point d'huile dessus~; et il n'y mettra point d'encens~; car c'est un gâteau de jalousie, un gâteau de souvenir, pour remettre en mémoire l'iniquité\FTNT{Lé. 5:11.}.
\VS{16}Le prêtre la fera approcher et la fera tenir debout devant Yahweh.
\VS{17}Puis le prêtre prendra de l'eau sainte dans un vase de terre, et il prendra de la poussière qui sera sur le sol du tabernacle, et la mettra dans l'eau.
\VS{18}Ensuite le prêtre fera tenir debout la femme devant Yahweh, il découvrira la tête de cette femme, et lui posera sur les paumes des mains le gâteau de souvenir, le gâteau de jalousie~; le prêtre tiendra dans sa main les eaux amères, qui apportent la malédiction.
\VS{19}Et le prêtre fera jurer la femme et lui dira~: Si aucun homme n'a couché avec toi, et si étant sous la puissance de ton mari tu ne t'es point détournée et souillée, sois exempte du mal de ces eaux amères qui apportent la malédiction.
\VS{20}Mais si, étant sous la puissance de ton mari, tu t'es détournée et souillée, et si un autre homme que ton mari a couché avec toi,
\VS{21}alors le prêtre fera jurer la femme avec un serment d'imprécation et lui dira~: Que Yahweh te livre à la malédiction et à l'exécration au milieu de ton peuple, en faisant flétrir ta cuisse et enfler ton ventre,
\VS{22}et que ces eaux qui apportent la malédiction, entrent dans tes entrailles pour te faire enfler le ventre et flétrir ta cuisse~! Alors la femme répondra~: Amen~! Amen~!
\VS{23}Ensuite le prêtre écrira dans un livre ces imprécations, et les effacera avec les eaux amères.
\VS{24}Et il fera boire à la femme les eaux amères qui apportent la malédiction, et les eaux qui apportent la malédiction entreront en elle pour être amères.
\VS{25}Le prêtre donc prendra des mains de la femme le gâteau de jalousie, et l'agitera de côté et d'autre devant Yahweh, et l'offrira sur l'autel~;
\VS{26}le prêtre prendra une poignée de cette offrande comme souvenir\FTNT{Voir commentaire en Lé. 2:2.}, et il la brûlera sur l'autel. C'est après cela qu'il fera boire les eaux à la femme.
\VS{27}Et après qu'il lui aura fait boire les eaux, s'il est vrai qu'elle se soit souillée et qu'elle a été infidèle à son mari, les eaux qui apportent la malédiction entreront en elle et lui seront amères, et son ventre enflera, sa cuisse se flétrira, et cette femme sera assujettie à l'exécration du serment au milieu de son peuple.
\VS{28}Mais si la femme ne s'est point souillée, mais qu'elle soit pure, elle sera reconnue innocente et aura des enfants.
\VS{29}Telle est la loi sur la jalousie, quand la femme qui est sous la puissance de son mari se détourne et se souille,
\VS{30}ou quand un mari saisi d'un esprit de jalousie a des soupçons sur sa femme~: Le prêtre la fera tenir debout devant Yahweh et fera à l'égard de cette femme tout ce qui est ordonné par cette loi.
\VS{31}Le mari sera exempt de faute, mais cette femme portera son iniquité.
\Chap{6}
\TextTitle{Le vœu de naziréat}
\VerseOne{}Yahweh parla à Moïse, en disant~:
\VS{2}Parle aux enfants d'Israël, et dis-leur~: Lorsqu'un homme ou une femme se consacrera en faisant un vœu de naziréat pour se consacrer à Yahweh,
\VS{3}il s'abstiendra de vin et de boisson forte, il ne boira ni vinaigre fait de vin, ni vinaigre fait avec une boisson forte~; il ne boira d'aucune liqueur de raisins, et il ne mangera point de raisins, frais ou secs.
\VS{4}Durant tous les jours de son naziréat il ne mangera d'aucun fruit de la vigne, depuis les pépins jusqu'à la peau du raisin\FTNT{Jg. 13:7~; Lu. 1:15.}.
\VS{5}Le rasoir ne passera point sur sa tête durant tous les jours de son naziréat. Il sera saint jusqu'à ce que les jours pour lesquels il s'est consacré à Yahweh soient accomplis, et il laissera croître les cheveux de sa tête\FTNT{Jg. 13:5~; 1 S. 1:11.}.
\VS{6}Durant tous les jours pour lesquels il s'est consacré à Yahweh il ne s'approchera d'aucune personne morte\FTNT{Lé. 21:1-4.}~;
\VS{7}il ne se souillera point à la mort de son père, ni de sa mère, ni de son frère, ni de sa sœur, car il porte sur sa tête la consécration de son Dieu.
\VS{8}Durant tous les jours de son naziréat, il sera consacré à Yahweh.
\VS{9}Que si quelqu'un vient à mourir subitement près de lui, la tête de son naziréat sera souillée, et il rasera sa tête au jour de sa purification, il la rasera le septième jour.
\VS{10}Le huitième jour, il apportera au prêtre deux tourterelles ou deux pigeonneaux, à l'entrée de la tente d'assignation\FTNT{Lé. 1~; Lé. 12:6.}.
\VS{11}Et le prêtre en sacrifiera l'un pour le sacrifice d'expiation et l'autre en holocauste, et il fera propitiation pour lui de ce qu'il a péché à l'occasion du mort. Il sanctifiera donc ainsi sa tête en ce jour-là.
\VS{12}Et il séparera à Yahweh les jours de son naziréat, offrant un agneau d'un an pour le délit, et les premiers jours seront comptés pour rien, car son naziréat a été souillé.
\VS{13}Or c'est ici la loi du naziréen. Lorsque les jours de son naziréat seront accomplis, on le fera venir à la porte de la tente d'assignation.
\VS{14}Il présentera son offrande à Yahweh~: Un agneau d'un an et sans défaut pour l'holocauste, une brebis d'un an et sans défaut pour le sacrifice d'expiation, et un bélier sans défaut pour le sacrifice d'offrande de paix\FTNT{Voir commentaire en Lé. 3:1.}.
\VS{15}Une corbeille de pains sans levain, de gâteaux de fine farine, pétrie à l'huile, et de galettes sans levain, oints d'huile, avec leur gâteau, et leurs libations~;
\VS{16}lesquels, le prêtre offrira devant Yahweh~; il sacrifiera aussi son offrande pour le péché, et son holocauste.
\VS{17}Et il offrira le bélier en sacrifice d'offrande de paix à Yahweh, avec la corbeille des pains sans levain~; le prêtre offrira aussi son gâteau, et sa libation.
\VS{18}Et le naziréen rasera la tête de son naziréat à l'entrée de la tente d'assignation, et prendra les cheveux de la tête de son naziréat, et les mettra sur le feu qui est sous le sacrifice d'offrande de paix.
\VS{19}Et le prêtre prendra l'épaule cuite du bélier, et un gâteau sans levain de la corbeille, et une galette sans levain, et les mettra sur les paumes des mains du naziréen, après qu'il se sera fait raser son naziréat.
\VS{20}Et le prêtre les agitera de côté et d'autre devant Yahweh~: C'est une chose sainte qui appartient au prêtre, avec la poitrine agitée et l'épaule offerte par élévation. Et après cela le naziréen boira du vin\FTNT{Lé. 7:32-34~; Ex. 29:24-27.}.
\VS{21}Telle est la loi du naziréen qui aura voué à Yahweh son offrande pour son naziréat, outre ce qu'il aura encore moyen d'offrir~; il fera selon son vœu qu'il aura voué, suivant la loi de son naziréat.
\TextTitle{Aaron et ses fils bénissent Israël}
\VS{22}Yahweh parla à Moïse, en disant~:
\VS{23}Parle à Aaron et à ses fils, et dis-leur~: Vous bénirez ainsi les enfants d'Israël, en leur disant~:
\VS{24}Yahweh te bénisse, et te garde~!
\VS{25}Yahweh fasse luire sa face sur toi, et te fasse grâce\FTNT{Ps. 67:2~; Ps. 119:135.}~!
\VS{26}Yahweh tourne sa face vers toi, et te donne la paix~!
\VS{27}Ils mettront donc mon Nom sur les enfants d'Israël, et je les bénirai.
\Chap{7}
\TextTitle{Les offrandes des princes}
\VerseOne{}Or il arriva le jour que Moïse eut achevé de dresser le tabernacle, et qu'il l'eut oint et sanctifié avec tous ses ustensiles, de même que l'autel avec tous ses ustensiles, il arriva, dis-je, après qu'il les eut oints et sanctifiés~;
\VS{2}que les princes d'Israël, et les chefs des maisons de leurs pères, qui sont les princes des tribus, et qui avaient assisté à faire les dénombrements, firent leur offrande.
\VS{3}Et ils amenèrent leur offrande devant Yahweh~: Six chars couverts et douze bœufs~; chaque char pour deux des princes, et chaque bœuf pour chacun d'eux~; ils les offrirent devant le tabernacle.
\VS{4}Alors Yahweh parla à Moïse, en disant~:
\VS{5}Prends d'eux ces choses, et elles seront employées pour le service de la tente d'assignation~; et tu les donneras aux Lévites, à chacun selon ses fonctions.
\VS{6}Moïse prit donc les chars et les bœufs, et il les remit aux Lévites.
\VS{7}Il donna aux fils de Guerschon deux chars et quatre bœufs, selon leurs fonctions.
\VS{8}Mais il donna aux fils de Merari quatre chars et huit bœufs, selon leurs fonctions, sous la conduite d'Ithamar, fils d'Aaron, le prêtre.
\VS{9}Or il n'en donna point aux fils de Kehath, parce que le service du sanctuaire était de leur charge~; ils portaient ces choses saintes sur les épaules.
\VS{10}Et les princes présentèrent leur offrande pour la dédicace de l'autel, le jour où on l'oignit~; les princes, dis-je, présentèrent leur offrande devant l'autel.
\VS{11}Et Yahweh dit à Moïse~: Un des princes offrira un jour, et un autre l'autre jour, son offrande pour la dédicace de l'autel.
\VS{12}Le premier jour donc, Nachschon, fils d'Amminadab, présenta son offrande pour la tribu de Juda.
\VS{13}Il offrit un plat d'argent du poids de cent trente sicles, un bassin d'argent de soixante-dix sicles, selon le sicle du sanctuaire, tous deux pleins de fine farine pétrie à l'huile, pour l'offrande~;
\VS{14}une coupe d'or de dix sicles pleine de parfum~;
\VS{15}un jeune taureau, un bélier, un agneau d'un an, pour l'holocauste~;
\VS{16}un jeune bouc pour le sacrifice d'expiation~;
\VS{17}et pour le sacrifice d'offrande de paix, deux bœufs, cinq béliers, cinq boucs, et cinq agneaux d'un an. Telle fut l'offrande de Nachschon, fils d'Amminadab.
\VS{18}Le second jour, Nethaneel, fils de Tsuar, chef de la tribu d'Issacar, présenta son offrande.
\VS{19}Et il offrit pour son offrande un plat d'argent du poids de cent trente sicles, un bassin d'argent de soixante-dix sicles, selon le sicle du sanctuaire, tous deux pleins de fine farine pétrie à l'huile, pour l'offrande~;
\VS{20}une coupe d'or de dix sicles pleine de parfum~;
\VS{21}un jeune taureau, un bélier, un agneau d'un an, pour l'holocauste~;
\VS{22}un jeune bouc pour le sacrifice d'expiation~;
\VS{23}et pour le sacrifice d'offrande de paix, deux bœufs, cinq béliers, cinq boucs, et cinq agneaux d'un an. Telle fut l'offrande de Nethaneel, fils de Tsuar.
\VS{24}Le troisième jour, Eliab, fils de Hélon, chef des fils de Zabulon, présenta son offrande.
\VS{25}Il offrit un plat d'argent du poids de cent trente sicles, un bassin d'argent de soixante-dix sicles, selon le sicle du sanctuaire, tous deux pleins de fine farine pétrie à l'huile, pour l'offrande~;
\VS{26}une coupe d'or de dix sicles pleine de parfum~;
\VS{27}un jeune taureau, un bélier, un agneau d'un an, pour l'holocauste~;
\VS{28}un jeune bouc pour le sacrifice d'expiation~;
\VS{29}et pour le sacrifice d'offrande de paix, deux bœufs, cinq béliers, cinq boucs, et cinq agneaux d'un an. Telle fut l'offrande d'Eliab, fils de Hélon.
\VS{30}Le quatrième jour, Elitsur, fils de Schedéur, prince des fils de Ruben, présenta son offrande.
\VS{31}Il offrit un plat d'argent du poids de cent trente sicles, un bassin d'argent de soixante-dix sicles, selon le sicle du sanctuaire, tous deux pleins de fine farine pétrie à l'huile, pour l'offrande~;
\VS{32}une coupe d'or de dix sicles pleine de parfum~;
\VS{33}un jeune taureau, un bélier, un agneau d'un an, pour l'holocauste~;
\VS{34}un jeune bouc pour le sacrifice d'expiation~;
\VS{35}et pour le sacrifice d'offrande de paix, deux bœufs, cinq béliers, cinq boucs, et cinq agneaux d'un an. Telle fut l'offrande d'Elitsur, fils de Schedéur.
\VS{36}Le cinquième jour, Schelumiel, fils de Tsurischaddaï, prince des fils de Siméon, présenta son offrande.
\VS{37}Il offrit un plat d'argent du poids de cent trente sicles, un bassin d'argent de soixante-dix sicles, selon le sicle du sanctuaire, tous deux pleins de fine farine pétrie à l'huile pour l'offrande~;
\VS{38}une coupe d'or de dix sicles pleine de parfum~;
\VS{39}un jeune taureau, un bélier, un agneau d'un an, pour l'holocauste~;
\VS{40}un jeune bouc pour le sacrifice d'expiation~;
\VS{41}et pour le sacrifice d'offrande de paix, deux bœufs, cinq béliers, cinq boucs, et cinq agneaux d'un an. Telle fut l'offrande de Schelumiel, fils de Tsurischaddaï.
\VS{42}Le sixième jour, Eliasaph, fils de Déuel, prince des fils de Gad, présenta son offrande.
\VS{43}Il offrit un plat d'argent du poids de cent trente sicles, un bassin d'argent de soixante-dix sicles, selon le sicle du sanctuaire, tous deux pleins de fine farine pétrie à l'huile pour l'offrande~;
\VS{44}une coupe d'or de dix sicles pleine de parfum~;
\VS{45}un jeune taureau, un bélier, un agneau d'un an, pour l'holocauste~;
\VS{46}un jeune bouc pour le sacrifice d'expiation~;
\VS{47}et pour le sacrifice d'offrande de paix, deux bœufs, cinq béliers, cinq boucs, et cinq agneaux d'un an. Telle fut l'offrande d'Eliasaph, fils de Déuel.
\VS{48}Le septième jour, Elischama, fils d'Ammihud, prince des fils d'Ephraïm, présenta son offrande.
\VS{49}Il offrit un plat d'argent, du poids de cent trente sicles, un bassin d'argent de soixante-dix sicles, selon le sicle du sanctuaire, tous deux pleins de fine farine pétrie à l'huile pour l'offrande~;
\VS{50}une coupe d'or de dix sicles pleine de parfum~;
\VS{51}un jeune taureau, un bélier, un agneau d'un an, pour l'holocauste~;
\VS{52}un jeune bouc pour le sacrifice d'expiation~;
\VS{53}et pour le sacrifice d'offrande de paix, deux bœufs, cinq béliers, cinq boucs, et cinq agneaux d'un an. Telle fut l'offrande d'Elischama, fils d'Ammihud.
\VS{54}Le huitième jour, Gamliel, fils de Pedahtsur, prince des fils de Manassé, présenta son offrande.
\VS{55}Il offrit un plat d'argent, du poids de cent trente sicles, un bassin d'argent de soixante-dix sicles, selon le sicle du sanctuaire, tous deux pleins de fine farine pétrie à l'huile pour l'offrande~;
\VS{56}une coupe d'or de dix sicles pleine de parfum~;
\VS{57}un jeune taureau, un bélier, un agneau d'un an, pour l'holocauste~;
\VS{58}un jeune bouc pour le sacrifice d'expiation~;
\VS{59}et pour le sacrifice d'offrande de paix, deux bœufs, cinq béliers, cinq boucs, et cinq agneaux d'un an. Telle fut l'offrande de Gamliel, fils de Pedahtsur.
\VS{60}Le neuvième jour, Abidan, fils de Guideoni, prince des fils de Benjamin, présenta son offrande.
\VS{61}Il offrit un plat d'argent, du poids de cent trente sicles, un bassin d'argent de soixante-dix sicles, selon le sicle du sanctuaire, tous deux pleins de fine farine pétrie à l'huile pour l'offrande~;
\VS{62}une coupe d'or de dix sicles pleine de parfum~;
\VS{63}un jeune taureau, un bélier, un agneau d'un an, pour l'holocauste~;
\VS{64}un jeune bouc pour le sacrifice d'expiation~;
\VS{65}et pour le sacrifice d'offrande de paix, deux bœufs, cinq béliers, cinq boucs, et cinq agneaux d'un an. Telle fut l'offrande d'Abidan, fils de Guideoni.
\VS{66}Le dixième jour, Ahiézer, fils d'Ammischaddaï, prince des fils de Dan, présenta son offrande.
\VS{67}Il offrit un plat d'argent du poids de cent trente sicles, un bassin d'argent de soixante-dix sicles, selon le sicle du sanctuaire, tous deux pleins de fine farine pétrie à l'huile pour l'offrande~;
\VS{68}une coupe d'or de dix sicles pleine de parfum~;
\VS{69}Un jeune taureau, un bélier, un agneau d'un an, pour l'holocauste~;
\VS{70}un jeune bouc pour le sacrifice d'expiation~;
\VS{71}et pour le sacrifice d'offrande de paix, deux bœufs, cinq béliers, cinq boucs, et cinq agneaux d'un an. Telle fut l'offrande d'Ahiézer, fils d'Ammischaddaï.
\VS{72}Le onzième jour, Paguiel, fils d'Ocran, prince des fils d'Aser, présenta son offrande.
\VS{73}Il offrit un plat d'argent, du poids de cent trente sicles, un bassin d'argent de soixante-dix sicles, selon le sicle du sanctuaire, tous deux pleins de fine farine pétrie à l'huile pour l'offrande~;
\VS{74}une coupe d'or de dix sicles pleine de parfum~;
\VS{75}un jeune taureau, un bélier, un agneau d'un an, pour l'holocauste~;
\VS{76}un jeune bouc pour le sacrifice d'expiation~;
\VS{77}et pour le sacrifice d'offrande de paix, deux bœufs, cinq béliers, cinq boucs, et cinq agneaux d'un an. Telle fut l'offrande de Paguiel, fils d'Ocran.
\VS{78}Le douzième jour, Ahira, fils d'Enan, prince des fils de Nephthali, présenta son offrande.
\VS{79}Il offrit un plat d'argent du poids de cent trente sicles, un bassin d'argent de soixante-dix sicles, selon le sicle du sanctuaire, tous deux pleins de fine farine pétrie à l'huile pour l'offrande~;
\VS{80}une coupe d'or de dix sicles pleine de parfum~;
\VS{81}un jeune taureau, un bélier, un agneau d'un an, pour l'holocauste~;
\VS{82}un jeune bouc pour le sacrifice d'expiation~;
\VS{83}et pour le sacrifice d'offrande de paix, deux bœufs, cinq béliers, cinq boucs, et cinq agneaux d'un an. Telle fut l'offrande d'Ahira, fils d'Enan.
\TextTitle{Les dons des princes}
\VS{84}Telle fut la dédicace de l'autel, qui fut faite par les princes d'Israël, lorsqu'il fut oint. Douze plats d'argent, douze bassins d'argent, douze tasses d'or~;
\VS{85}chaque plat d'argent était de cent trente sicles, et chaque bassin de soixante-dix~; tout l'argent de ces ustensiles montait à deux mille quatre cents sicles, selon le sicle du sanctuaire~;
\VS{86}douze coupes d'or pleines de parfum, chacune de dix sicles, selon le sicle du sanctuaire~; tout l'or des tasses montait à cent-vingt sicles.
\VS{87}Tous les animaux pour l'holocauste étaient douze veaux, douze béliers, et douze agneaux d'un an, avec leurs offrandes, et douze jeunes boucs pour le sacrifice d'expiation.
\VS{88}Tous les animaux du sacrifice d'offrande de paix étaient vingt-quatre veaux, avec soixante béliers, soixante boucs, et soixante agneaux d'un an. Telle fut donc la dédicace de l'autel, après qu'on l'eut oint.
\VS{89}Et quand Moïse entrait dans la tente d'assignation pour parler avec Yahweh, il entendait une voix qui lui parlait du haut du propitiatoire placé sur l'arche du témoignage, entre les deux chérubins. Et il lui parlait\FTNT{Ex. 25:22.}.
\Chap{8}
\TextTitle{Les lampes sur le chandelier}
\VerseOne{}Yahweh parla à Moïse, en disant~:
\VS{2}Parle à Aaron, et tu lui diras~: Quand tu allumeras les lampes, les sept lampes éclaireront sur le devant du chandelier\FTNT{Ex. 25:37.}.
\VS{3}Et Aaron fit ainsi~; il plaça les lampes pour éclairer sur le devant du chandelier, comme Yahweh l'avait commandé à Moïse.
\VS{4}Or le chandelier était fait de telle manière, qu'il était d'or battu au marteau, d'ouvrage fait au marteau, sa tige aussi, et ses fleurs. On fit ainsi le chandelier selon le modèle que Yahweh en avait fait voir à Moïse\FTNT{Ex. 25:31-40.}.
\TextTitle{Purification des Lévites}
\VS{5}Puis Yahweh parla à Moïse, en disant~:
\VS{6}Prends les Lévites du milieu des enfants d'Israël, et purifie-les.
\VS{7}Tu leur feras ainsi pour les purifier. Tu feras aspersion sur eux de l'eau de purification~; ils feront passer le rasoir sur toute leur chair, ils laveront leurs vêtements, et ils se purifieront.
\VS{8}Puis ils prendront un jeune taureau avec son offrande de gâteau de fine farine pétrie à l'huile~; et tu prendras un autre jeune taureau pour le sacrifice d'expiation.
\VS{9}Alors tu feras approcher les Lévites devant la tente d'assignation, et tu convoqueras toute l'assemblée des enfants d'Israël.
\VS{10}Tu feras, dis-je, approcher les Lévites devant Yahweh, et les enfants d'Israël poseront leurs mains sur les Lévites.
\VS{11}Et Aaron fera tourner de côté et d'autre les Lévites devant Yahweh, comme offrande de la part des enfants d'Israël, et ils seront employés au service de Yahweh.
\VS{12}Et les Lévites poseront leurs mains sur la tête des veaux~; puis tu offriras l'un en sacrifice pour l'expiation, et l'autre en holocauste à Yahweh, afin de faire propitiation pour les Lévites.
\VS{13}Après tu feras tenir les Lévites devant Aaron et devant ses fils, et tu les présenteras en offrande à Yahweh.
\VS{14}Ainsi tu sépareras les Lévites du milieu des enfants d'Israël, et les Lévites m'appartiendront.
\VS{15}Après cela, les Lévites viendront pour servir dans la tente d'assignation quand tu les auras purifiés et présentés en offrande.
\VS{16}Car ils me sont entièrement donnés du milieu des enfants d'Israël~; je les ai pris pour moi à la place des premiers-nés~; de tous les premiers-nés des fils d'Israël.
\VS{17}Car tout premier-né des enfants d'Israël est à moi, tant des hommes que des animaux~; je me les suis consacrés le jour où j'ai frappé tous les premiers-nés dans le pays d'Egypte.
\VS{18}Or j'ai pris les Lévites au lieu de tous les premiers-nés d'entre les enfants d'Israël.
\VS{19}Et j'ai entièrement donné, d'entre les enfants d'Israël, les Lévites à Aaron et à ses fils, pour faire le service des enfants d'Israël dans la tente d'assignation, et pour faire propitiation pour les enfants d'Israël~; afin qu'il n'y ait point de plaie sur les enfants d'Israël, comme il y aurait si les enfants d'Israël s'approchaient du sanctuaire.
\VS{20}Moïse, Aaron et toute l'assemblée des enfants d'Israël firent à l'égard des Lévites tout ce que Yahweh avait ordonné à Moïse touchant les Lévites~; ainsi firent les enfants d'Israël.
\VS{21}Les Lévites donc se purifièrent, et lavèrent leurs vêtements, et Aaron les fit tourner de côté et d'autre comme une offrande devant Yahweh, et il fit propitiation pour eux afin de les purifier.
\VS{22}Cela étant fait, les Lévites vinrent faire leur service dans la tente d'assignation devant Aaron, et devant ses fils, selon ce que Yahweh avait commandé à Moïse touchant les Lévites~; ainsi fut-il fait à leur égard.
\VS{23}Puis Yahweh parla à Moïse, en disant~:
\VS{24}Voici ce qui concerne les Lévites. Depuis l'âge de vingt-cinq ans et au-dessus, tout Lévite entrera en fonction dans la tente d'assignation~;
\VS{25}dès l'âge de cinquante ans, il sortira du service et ne servira plus.
\VS{26}Cependant il servira ses frères dans la tente d'assignation, pour garder ce qui leur a été commis, mais il ne fera plus de service. Tu agiras ainsi à l'égard des Lévites pour ce qui concerne leurs fonctions.
\Chap{9}
\TextTitle{La Pâque}
\VerseOne{}Yahweh avait aussi parlé à Moïse dans le désert de Sinaï, le premier mois de la seconde année, après qu'ils furent sortis du pays d'Egypte, en disant~:
\VS{2}Que les enfants d'Israël célèbrent la Pâque\FTNT{Ex. 12~; 1 Co. 5:7.} au temps fixé.
\VS{3}Vous la ferez en sa saison, le quatorzième jour de ce mois entre les deux soirs, selon toutes ses ordonnances et selon tout ce qu'il faut y faire.
\VS{4}Moïse donc, parla aux enfants d'Israël afin qu'ils célèbrent la Pâque.
\VS{5}Et ils firent la Pâque le quatorzième jour du premier mois, entre les deux soirs, dans le désert de Sinaï~; selon tout ce que Yahweh avait commandé à Moïse, les enfants d'Israël le firent ainsi.
\VS{6}Or il y eut quelques-uns qui étaient impurs à cause d'un mort et qui ne purent célébrer la Pâque ce jour-là. Ils se présentèrent ce même jour devant Moïse et devant Aaron,
\VS{7}et ces hommes leur dirent~: Nous sommes impurs à cause d'un mort, pourquoi serions-nous privés de présenter l'offrande à Yahweh dans sa saison au milieu des enfants d'Israël~?
\VS{8}Et Moïse leur dit~: Arrêtez-vous, et j'entendrai ce que Yahweh commandera sur votre sujet.
\VS{9}Alors Yahweh parla à Moïse, en disant~:
\VS{10}Parle aux enfants d'Israël, et dis-leur~: Si quelqu'un d'entre vous, ou de votre postérité, est impur à cause d'un mort, ou est en voyage dans un lieu éloigné, il célébrera cependant la Pâque en l'honneur de Yahweh.
\VS{11}Ils la feront le quatorzième jour du second mois, entre les deux soirs~; et ils la mangeront avec du pain sans levain et des herbes amères\FTNT{Ex. 12:10~; Ex. 23:18~; Ex. 34:25~; De. 16:4~; Jn. 19:33-36.}.
\VS{12}Ils n'en laisseront rien jusqu'au matin, et n'en briseront point les os. Ils la feront selon toutes les ordonnances de la Pâque.
\VS{13}Mais si celui qui est pur et qui n'est pas en voyage s'abstient de célébrer la Pâque, il sera retranché d'entre ses peuples parce qu'il n'a pas présenté l'offrande de Yahweh en sa saison.
\VS{14}Et si un étranger en séjour chez vous célèbre la Pâque de Yahweh, il la fera selon l'ordonnance de la Pâque. II y aura une même ordonnance entre vous, pour l'étranger comme pour celui qui est né au pays\FTNT{Ex. 12:49.}.
\TextTitle{La nuée conduit Israël}
\VS{15}Or le jour où le tabernacle fut dressé, la nuée couvrit le tabernacle de la tente d'assignation~; et le soir jusqu'au matin, elle parut sur le tabernacle avec l'apparence d'un feu\FTNT{Ex. 13:21-22~; Ex. 40:34-38~; De. 1:33.}.
\VS{16}Il en fut ainsi continuellement~; la nuée le couvrait, mais elle paraissait la nuit comme du feu.
\VS{17}Et selon que la nuée se levait de dessus le tabernacle, les enfants d'Israël partaient~; et au lieu où la nuée s'arrêtait, les enfants d'Israël y campaient.
\VS{18}Les enfants d'Israël marchaient sur le commandement de Yahweh, et ils campaient sur le commandement de Yahweh~; ils campaient aussi longtemps que la nuée se tenait sur le tabernacle.
\VS{19}Et quand la nuée restait plusieurs jours sur le tabernacle, les enfants d'Israël observaient l'ordre de Yahweh, et ne partaient point.
\VS{20}Et pour peu de jours que la nuée fût sur le tabernacle, ils campaient sur le commandement de Yahweh, et ils partaient sur le commandement de Yahweh.
\VS{21}Et quand la nuée y était depuis le soir jusqu'au matin, et que la nuée se levait au matin, ils partaient~; fût-ce de jour ou de nuit, quand la nuée se levait, ils partaient.
\VS{22}Si la nuée s'arrêtait sur le tabernacle deux jours, ou un mois, ou plus longtemps, les enfants d'Israël restaient campés, et ne partaient point~; mais quand elle se levait, ils partaient.
\VS{23}Ils campaient donc au commandement de Yahweh, et ils partaient au commandement de Yahweh~; et ils prenaient garde à Yahweh, suivant le commandement de Yahweh, qu'il leur faisait savoir par Moïse.
\Chap{10}
\TextTitle{Les trompettes d'argent}
\VerseOne{}Puis Yahweh parla à Moïse, en disant~:
\VS{2}Fais-toi deux trompettes d'argent, battues au marteau. Elles te serviront pour convoquer l'assemblée, et pour le départ des camps.
\VS{3}Quand on en sonnera, toute l'assemblée s'assemblera auprès de toi à l'entrée de la tente d'assignation.
\VS{4}Et quand on sonnera d'une seule, les princes, qui sont les chefs des milliers d'Israël, s'assembleront vers toi.
\VS{5}Mais quand vous sonnerez avec un retentissement bruyant, ceux qui campent à l'orient partiront.
\VS{6}Et quand vous sonnerez la seconde fois avec un retentissement bruyant, ceux qui campent au midi partiront, on sonnera avec un retentissement bruyant, pour leur départ.
\VS{7}Lorsque vous convoquerez l'assemblée, vous ne sonnerez pas avec un retentissement bruyant.
\VS{8}Or les fils d'Aaron, les prêtres, sonneront des trompettes. Ce sera une loi perpétuelle pour vous et pour vos descendants.
\VS{9}Et lorsque, dans votre pays, vous irez à la guerre contre l'ennemi qui vous combattra, vous sonnerez des trompettes avec un retentissement bruyant, et Yahweh votre Dieu, se souviendra de vous, et vous serez délivrés de vos ennemis.
\VS{10}Aussi dans vos jours de joie, dans vos fêtes solennelles, et au commencement de vos mois, vous sonnerez des trompettes en offrant vos holocaustes et vos sacrifices d'offrande de paix, et elles vous serviront de souvenir devant votre Dieu. Je suis Yahweh, votre Dieu.
\TextTitle{La nuée se lève, reprise de la marche dans le désert}
\VS{11}Or il arriva le vingtième jour du second mois de la seconde année, que la nuée se leva de dessus le tabernacle du témoignage.
\VS{12}Et les enfants d'Israël partirent du désert de Sinaï, selon l'ordre fixé pour leur marche. La nuée se posa dans le désert de Paran.
\VS{13}Ils partirent donc pour la première fois, suivant le commandement de Yahweh, déclaré par Moïse.
\VS{14}Et la bannière du camp des fils de Juda partit la première, selon leurs armées. Nachschon, fils d'Amminadab, commandait l'armée de Juda~;
\VS{15}et Nethaneel, fils de Tsuar, commandait l'armée de la tribu des fils d'Issacar~;
\VS{16}et Eliab, fils de Hélon, commandait l'armée de la tribu des fils de Zabulon.
\VS{17}Et le tabernacle fut démonté~; et les fils de Guerschon, et les fils de Merari, qui portaient le tabernacle, partirent.
\VS{18}Puis la bannière du camp de Ruben partit, selon leurs armées. Et Elitsur, fils de Schedéur, commandait l'armée de Ruben~;
\VS{19}et Schelumiel, fils de Tsurischaddaï, commandait l'armée de la tribu des fils de Siméon~;
\VS{20}et Eliasaph, fils de Déuel, commandait l'armée des fils de Gad.
\VS{21}Alors les Kehathites, qui portaient le sanctuaire, partirent~; cependant on dressait le tabernacle, en attendant leur arrivée.
\VS{22}Puis la bannière du camp des fils d'Ephraïm partit, selon leurs armées. Elischama, fils d'Ammihud, commandait l'armée d'Ephraïm~;
\VS{23}et Gamliel, fils de Pedahtsur, commandait l'armée de la tribu des fils de Manassé~;
\VS{24}et Abidan, fils de Guideoni, commandait l'armée de la tribu des fils de Benjamin.
\VS{25}Enfin la bannière des camps des fils de Dan, qui faisait l'arrière-garde, partit, selon leurs armées~; et Ahiézer, fils d'Ammischaddaï, commandait l'armée de Dan.
\VS{26}Et Paguiel, fils d'Ocran, commandait l'armée de la tribu des fils d'Aser~;
\VS{27}et Ahira, fils d'Enan, commandait l'armée de la tribu des fils de Nephthali.
\VS{28}Tel fut l'ordre d'après lequel les enfants d'Israël se mirent en marche selon leurs armées, c'est ainsi qu'ils partirent.
\VS{29}Or Moïse dit à Hobab, fils de Réuel, le Madianite, beau-père de Moïse~: Nous allons au lieu dont Yahweh a dit~: Je vous le donnerai. Viens avec nous, et nous te ferons du bien, car Yahweh a promis de faire du bien à Israël.
\VS{30}Et Hobab lui répondit~: Je n'irai point, mais je m'en irai dans mon pays, et vers ma parenté.
\VS{31}Et Moïse lui dit~: Je te prie, ne nous quitte pas~; car tu nous serviras de guide, parce que tu connais les lieux où nous aurons à camper dans le désert.
\VS{32}Et il arrivera que, quand tu seras venu avec nous, et que le bien que Yahweh doit nous faire sera arrivé, nous te ferons aussi du bien.
\VS{33}Ainsi ils partirent de la montagne de Yahweh et ils marchèrent trois jours~; et l'arche de l'alliance de Yahweh alla devant eux, et fit une marche de trois jours pour leur chercher un lieu de repos.
\VS{34}Et la nuée de Yahweh était sur eux le jour, quand ils partaient du camp.
\VS{35}Or il arrivait qu'au départ de l'arche, Moïse disait~: Lève-toi, ô Yahweh, et tes ennemis seront dispersés, et ceux qui te haïssent s'enfuiront de devant toi\FTNT{Ps. 68:2.}~!
\VS{36}Et quand on la posait, il disait~: Reviens Yahweh, aux dix mille milliers d'Israël~!
\Chap{11}
\TextTitle{Jugement contre les murmures du peuple}
\VerseOne{}Après, il arriva que le peuple murmura et cela déplut aux oreilles de Yahweh. Lorsque Yahweh l'entendit, sa colère s'enflamma, et le feu de Yahweh s'alluma parmi eux et en consuma l'extrémité du camp.
\VS{2}Alors le peuple cria à Moïse. Moïse pria Yahweh, et le feu s'éteignit.
\VS{3}Et on nomma ce lieu-là Tabeéra, parce que le feu de Yahweh s'était allumé parmi eux.
\TextTitle{Le peuple regrette l'Egypte}
\VS{4}Et le peuple nombreux qui se trouvaient au milieu d'Israël fut épris de convoitise~; et même, les enfants d'Israël se mirent à pleurer disant~: Qui nous donnera de la viande à manger\FTNT{Ex. 16:3~; Ps. 106:14~; 1 Co. 10:6.}~?
\VS{5}Nous nous souvenons des poissons que nous mangions en Egypte, et qui ne nous coûtaient rien, des concombres, des melons, des poireaux, des oignons, et de l'ail.
\VS{6}Et maintenant nos âmes sont asséchées~; nos yeux ne voient que de la manne\FTNT{Ps. 78:24.}.
\VS{7}Or la manne était comme la graine de coriandre, et avait l'apparence du bdellium\FTNT{Ex. 16:14-31~; Jn. 6:31-58.}.
\VS{8}Le peuple se dispersait et la ramassait, il la moulait aux meules, ou la pilait dans un mortier, il la cuisait au pot et en faisait des gâteaux. Elle avait le goût d'une liqueur d'huile fraîche.
\VS{9}Et quand la rosée descendait la nuit sur le camp, la manne y descendait aussi.
\TextTitle{Moïse dans l'affliction}
\VS{10}Moïse donc entendit le peuple qui pleurait, chacun dans sa famille et à l'entrée de sa tente. La colère de Yahweh s'enflamma fortement et Moïse en fut attristé.
\VS{11}Et Moïse dit à Yahweh~: Pourquoi affliges-tu ton serviteur et pourquoi n'ai-je pas trouvé grâce à tes yeux, que tu aies mis sur moi la charge de tout ce peuple~?
\VS{12}Est-ce moi qui ai conçu tout ce peuple ou l'ai-je engendré pour que tu me dises~: Porte-le dans ton sein comme le nourricier porte un enfant qui tète, porte-le jusqu'au pays que tu as juré à ses pères~?
\VS{13}D'où aurais-je de la viande pour en donner à tout ce peuple~? Car il pleure auprès de moi, en disant~: Donne-nous de la viande à manger~!
\VS{14}Je ne puis, à moi seul, porter tout ce peuple, car il est trop pesant pour moi\FTNT{De. 1:9-12.}.
\VS{15}Si tu agis ainsi à mon égard, tue-moi, je te prie donc, si j'ai trouvé grâce à tes yeux, et que je ne voie pas mon malheur.
\TextTitle{Yahweh établit soixante-dix anciens autour de Moïse\FTNTT{Ex. 18:19.}}
\VS{16}Alors Yahweh dit à Moïse~: Assemble-moi soixante-dix hommes des anciens d'Israël, que tu connais être les anciens du peuple et ses officiers, et amène-les à la tente d'assignation, et qu'ils s'y présentent avec toi.
\VS{17}Puis je descendrai, et je parlerai là avec toi, je mettrai de l'Esprit qui est sur toi sur eux~; afin qu'ils portent avec toi la charge du peuple, et que tu ne la portes pas toi seul.
\VS{18}Et tu diras au peuple~: Sanctifiez-vous pour demain, et vous mangerez de la viande~; puisque vous avez pleuré aux oreilles de Yahweh, en disant~: Qui nous fera manger de la viande~? Car nous étions bien en Egypte. Ainsi Yahweh vous donnera de la viande, et vous en mangerez.
\VS{19}Vous n'en mangerez pas un jour, ni deux jours, ni cinq jours, ni dix jours, ni vingt jours,
\VS{20}mais jusqu'à un mois entier, jusqu'à ce qu'elle vous sorte par les narines, et que vous en ayez du dégoût, parce que vous avez rejeté Yahweh qui est au milieu de vous~; vous avez pleuré devant lui, en disant~: Pourquoi sommes-nous sortis d'Egypte~?
\VS{21}Moïse dit~: Six cent mille hommes de pied forment ce peuple au milieu duquel je suis, et tu as dit~: Je leur donnerai de la viande afin qu'ils en mangent un mois entier~!
\VS{22}Leur tuera-t-on des brebis ou des bœufs, en sorte qu'il y en ait assez pour eux~? Ou leur assemblera-t-on tous les poissons de la mer, en sorte qu'ils en aient assez~?
\VS{23}Yahweh répondit à Moïse: La main de Yahweh serait-elle trop courte~? Tu verras maintenant si ce que je t'ai dit arrivera ou non\FTNT{Es. 50:2~; Es. 59:1-2.}.
\VS{24}Moïse donc sortit et rapporta au peuple les paroles de Yahweh. Il assembla soixante-dix hommes des anciens du peuple, et les plaça autour de la tente.
\VS{25}Yahweh descendit dans la nuée et parla à Moïse~; il prit de l'Esprit qui était sur lui et le mit sur les soixante-dix hommes anciens. Et dès que l'Esprit reposa sur eux, ils prophétisèrent~; mais ils ne continuèrent pas.
\TextTitle{Prophétie d'Eldad et de Médad}
\VS{26}Or il y eut deux hommes restés au camp, l'un s'appelait Eldad, et l'autre Médad, sur lesquels l'Esprit reposa. Ils étaient de ceux qui avaient été inscrits, mais ils n'étaient pas allés à la tente, et ils prophétisaient dans le camp.
\VS{27}Alors un garçon courut le rapporter à Moïse, en disant~: Eldad et Médad prophétisent dans le camp.
\VS{28}Et Josué, fils de Nun, qui servait Moïse, l'un de ses jeunes gens, répondit, en disant~: Mon seigneur Moïse, empêche-les.
\VS{29}Et Moïse lui répondit~: Es-tu jaloux pour moi~? Plût à Dieu que tout le peuple de Yahweh fût prophète, et que Yahweh mît son Esprit sur eux~!
\VS{30}Puis Moïse se retira au camp, lui et les anciens d'Israël.
\TextTitle{Les cailles et le jugement de Yahweh}
\VS{31}Alors Yahweh fit lever un vent de la mer qui amena des cailles et les répandit sur le camp environ le chemin d'une journée, de çà et de là, tout autour du camp~; et il y en avait presque la hauteur de deux coudées sur la terre\FTNT{Ex. 16:13-15~; Ps. 78:26-29~; Ps. 105:40.}.
\VS{32}Et le peuple se leva tout ce jour-là, et toute la nuit, et tout le jour suivant, et amassa des cailles~; celui qui en avait amassé le moins en avait dix homers~; et ils les étendirent soigneusement pour eux tout autour du camp.
\VS{33}Mais la chair était encore entre leurs dents, avant qu'elle fût mâchée, la colère de Yahweh s'embrasa contre le peuple, et il frappa le peuple d'une très grande plaie\FTNT{Ps. 78:30-31.}.
\VS{34}Et on nomma ce lieu-là Kibroth-Hattaava~; car on ensevelit là le peuple qui avait convoité.
\VS{35}Et de Kibroth-Hattaava le peuple s'en alla pour Hatséroth, et il s'arrêta à Hatséroth.
\Chap{12}
\TextTitle{Marie et Aaron murmurent contre Moïse}
\VerseOne{}Alors Marie et Aaron parlèrent contre Moïse au sujet de la femme éthiopienne\FTNT{Voir commentaire en Ge. 2:13.} qu'il avait prise, car il avait pris une femme éthiopienne.
\VS{2}Et ils dirent~: Est-ce seulement par Moïse que Yahweh parle~? N'est-ce pas aussi par nous qu'il parle~? Et Yahweh entendit cela. 
\VS{3}Or cet homme Moïse était un homme fort doux, plus que tous les hommes qui étaient sur la terre.
\VS{4}Et soudain Yahweh dit à Moïse, à Aaron, et à Marie~: Venez vous trois à la tente d'assignation~; et ils y allèrent eux trois.
\VS{5}Alors Yahweh descendit dans la colonne de nuée et se tint à l'entrée de la tente. Puis il appela Aaron et Marie, qui s'avancèrent tous les deux.
\VS{6}Et il dit~: Ecoutez maintenant mes paroles~! Lorsqu'il y aura parmi vous un prophète, moi qui suis Yahweh je me ferai bien connaître à lui en vision, et je lui parlerai en songe.
\VS{7}Il n'en est pas ainsi de mon serviteur Moïse, qui est fidèle dans toute ma maison\FTNT{Hé. 3:2.}.
\VS{8}Je parle avec lui bouche à bouche, et il me voit en effet, et non point en obscurité, ni dans aucune représentation de Yahweh. Pourquoi donc n'avez-vous pas craint de parler contre mon serviteur, contre Moïse~?
\FTNT{Ex. 33:11~; De. 34:10.} 
\VS{9}Ainsi la colère de Yahweh s'embrasa contre eux. Et il s'en alla.
\VS{10}Car la nuée se retira de dessus la tente. Et voici, Marie était frappée d'une lèpre blanche comme la neige~; et Aaron se tourna vers Marie et la vit lépreuse.
\VS{11}Alors Aaron dit à Moïse~: Hélas, de grâce, mon seigneur~! Je te prie ne mets point sur nous ce péché, car nous avons fait follement, et nous avons péché.
\VS{12}Je te prie qu'elle ne soit pas comme un enfant mort-né, dont la moitié de la chair est déjà consumée quand il sort du ventre de sa mère~!
\VS{13}Alors Moïse cria à Yahweh, en disant~: Ô Dieu, je te prie, guéris-la, je t'en prie.
\VS{14}Et Yahweh répondit à Moïse~: Si son père lui avait craché au visage, ne serait-elle pas dans l'ignominie pendant sept jours~? Qu'elle soit enfermée sept jours en dehors du camp, après quoi, elle y sera reçue\FTNT{Lé. 13:46.}.
\VS{15}Ainsi Marie fut enfermée hors du camp sept jours~; et le peuple ne partit pas de là jusqu'à ce que Marie fût rentrée.
\VS{16}Après cela le peuple partit de Hatséroth, et il campa dans le désert de Paran.
\Chap{13}
\TextTitle{Douze espions envoyés pour explorer Canaan}
\VerseOne{}Et Yahweh parla à Moïse, en disant~:
\VS{2}Envoie des hommes pour explorer le pays de Canaan, que je donne aux enfants d'Israël. Tu enverras un homme de chaque tribu de leurs pères, tous seront des principaux d'entre eux.
\VS{3}Moïse donc les envoya du désert de Paran, d'après l'ordre de Yahweh~; et tous ces hommes étaient chefs des enfants d'Israël.
\VS{4}Et ce sont ici leurs noms~: De la tribu de Ruben~: Schammua, fils de Zaccur~;
\VS{5}de la tribu de Siméon~: Schaphath, fils de Hori~;
\VS{6}de la tribu de Juda~: Caleb, fils de Jephunné~;
\VS{7}de la tribu d'Issacar~: Jigual, fils de Joseph~;
\VS{8}de la tribu d'Ephraïm~: Hosée, fils de Nun~;
\VS{9}de la tribu de Benjamin~: Palthi, fils de Raphu~;
\VS{10}de la tribu de Zabulon~: Gaddiel, fils de Sodi~;
\VS{11}de l'autre tribu de Joseph~: la tribu de Manassé, Gaddi, fils de Susi~;
\VS{12}de la tribu de Dan~: Ammiel, fils de Guemalli~;
\VS{13}de la tribu d'Aser~: Sethur, fils de Micaël~;
\VS{14}de la tribu de Nephthali~: Nachbi, fils de Vophsi~;
\VS{15}de la tribu de Gad~: Guéuel, fils de Maki.
\VS{16}Ce sont là les noms des hommes que Moïse envoya pour explorer le pays. Moïse donna à Hosée, fils de Nun, le nom de Josué\FTNT{Moïse changea le nom d'Hosée en y ajoutant le Nom de Yahweh. Hosée signifie «~sauveur~» et Josué (ou Jésus) «~Yahweh est salut~». Josué préfigurait Jésus-Christ qui nous a délivrés et transportés dans le Royaume des cieux (Col. 1:12-14). Moïse avait compris prophétiquement que seul Jésus peut nous faire rentrer dans notre héritage.}.
\VS{17}Moïse les envoya pour explorer le pays de Canaan, et il leur dit~: Montez de ce côté par le sud~; et vous monterez sur la montagne.
\VS{18}Et vous verrez quel est ce pays-là, et quel est le peuple qui l'habite, s'il est fort ou faible~; s'il est en petit ou en grand nombre.
\VS{19}Et quel est le pays où il habite, s'il est bon ou mauvais~; et quelles sont les villes dans lesquelles il habite, si c'est dans des camps, ou dans des villes fortifiées.
\VS{20}Et quelle est la terre, si elle est grasse ou maigre, s'il y a des arbres, ou non. Ayez bon courage, et prenez du fruit du pays. Or c'était alors le temps des premiers raisins.
\VS{21}Etant donc partis, ils examinèrent le pays, depuis le désert de Tsin jusqu'à Rehob, à l'entrée de Hamath.
\VS{22}Ils montèrent par le sud, et ils allèrent jusqu'à Hébron, où étaient Ahiman, Schéschaï, et Talmaï, enfants d'Anak. Hébron avait été bâtie sept ans avant Tsoan en Egypte.
\VS{23}Et ils vinrent jusqu'au torrent d'Eschcol, et coupèrent de là un sarment de vigne, avec une grappe de raisins~; ils étaient deux à le porter avec une perche. Ils apportèrent aussi des grenades et des figues.
\VS{24}Et on donna à ce lieu le nom de vallée d'Eschcol~; à cause de la grappe que les fils d'Israël y coupèrent.
\VS{25}Et au bout de quarante jours, ils furent de retour du pays qu'ils étaient allés explorer.
\TextTitle{Comptes rendus des envoyés}
\VS{26}Et à leur arrivée, ils se rendirent auprès de Moïse et d'Aaron, et de toute l'assemblée des enfants d'Israël, dans le désert de Padan à Kadès. Ils leur firent le rapport, ainsi qu'à toute l'assemblée, ils leur montrèrent les fruits du pays.
\VS{27}Ils firent donc leur rapport à Moïse, et lui dirent~: Nous avons été dans le pays où tu nous as envoyés. Véritablement, c'est un pays où coulent le lait et le miel, et en voici les fruits.
\VS{28}Seulement, le peuple qui habite ce pays est puissant, les villes sont fortifiées, très grandes~; nous y avons vu des enfants d'Anak\FTNT{De. 1:24-28.}.
\VS{29}Les Amalécites habitent la contrée du midi~; les Héthiens, les Jébusiens et les Amoréens habitent la montagne~; les Cananéens habitent le long de la mer, et vers le rivage du Jourdain.
\VS{30}Caleb fit taire le peuple devant Moïse, et il dit~: Montons, possédons ce pays, car nous y serons vainqueurs~!
\VS{31}Mais les hommes qui y étaient montés avec lui dirent~: Nous ne pouvons pas monter contre ce peuple-là, car il est plus fort que nous.
\VS{32}Et ils décrièrent devant les enfants d'Israël le pays qu'ils avaient exploré, en disant~: Le pays que nous avons parcouru pour l'explorer est un pays qui dévore ses habitants et tous ceux que nous y avons vus sont des gens de grande taille.
\VS{33}Et nous y avons vu aussi des géants, des enfants d'Anak, de la race des géants et nous étions à nos yeux et à leurs yeux comme des sauterelles.
\Chap{14}
\TextTitle{Rébellion et incrédulité d'Israël\FTNTT{1 Co. 10:1-5~; Hé. 3:7-19}}
\VerseOne{}Alors toute l'assemblée éleva la voix et se mit à pousser des cris, et le peuple pleura cette nuit-là.
\VS{2}Et tous les enfants d'Israël murmurèrent contre Moïse et Aaron, et toute l'assemblée leur dit~: Oh~! Si nous étions morts dans le pays d'Egypte~! Ou si nous étions morts dans ce désert\FTNT{De. 1:26-27.}~!
\VS{3}Et pourquoi Yahweh nous fait-il aller dans ce pays, où nous tomberons par l'épée, où nos femmes et nos petits enfants deviendront une proie~? Ne vaut-il pas mieux retourner en Egypte~?
\VS{4}Et ils se dirent l'un à l'autre~: Etablissons-nous un chef, et retournons en Egypte.
\VS{5}Alors Moïse et Aaron tombèrent sur leurs visages devant toute l'assemblée des enfants d'Israël.
\VS{6}Et Josué, fils de Nun, et Caleb, fils de Jephunné, qui étaient parmi ceux qui avaient exploré le pays, déchirèrent leurs vêtements,
\VS{7}et parlèrent à toute l'assemblée des enfants d'Israël, en disant~: Le pays que nous avons exploré est un très bon pays.
\VS{8}Si nous sommes agréables à Yahweh, il nous fera entrer dans ce pays, et il nous le donnera. C'est un pays où coulent le lait et le miel.
\VS{9}Seulement, ne soyez point rebelles contre Yahweh, et ne craignez point le peuple de ce pays-là, car ils seront notre pain, leur protection s'est retirée de dessus eux. Yahweh est avec nous, ne les craignez point\FTNT{De. 20:3-4.}~!
\VS{10}Alors toute l'assemblée parlait de les lapider~; mais la gloire de Yahweh apparut à tous les enfants d'Israël, devant la tente d'assignation.
\TextTitle{Moïse intercède pour le pardon d'Israël}
\VS{11}Et Yahweh dit à Moïse~: Jusqu'à quand ce peuple-ci m'irritera-t-il par mépris et jusqu'à quand ne croira-t-il point en moi, malgré tous les signes que j'ai faits au milieu de lui~?
\VS{12}Je le frapperai par la peste et je le détruirai, mais je ferai de toi une nation plus grande et plus puissante que lui.
\VS{13}Et Moïse dit à Yahweh~: Mais les Egyptiens l'entendront, car tu as fait monter par ta puissance ce peuple-ci du milieu d'eux\FTNT{Ex. 32:10-12.},
\VS{14}et ils diront avec les habitants de ce pays qui auront entendu que tu étais, ô Yahweh, au milieu de ce peuple, et que tu apparaissais, ô Yahweh à vue d'œil, que ta nuée s'arrêtait sur eux, et que tu marchais devant eux le jour dans la colonne de nuée, et la nuit dans la colonne de feu~;
\VS{15}si tu fais mourir ce peuple comme un seul homme, les nations qui ont entendu parler de toi diront~:
\VS{16}Yahweh n'avait pas le pouvoir de faire entrer ce peuple dans le pays qu'il avait juré de leur donner, il l'a égorgé dans le désert.
\VS{17}Maintenant, je te prie, que la puissance du Seigneur se montre dans sa grandeur, comme tu l'as déclaré en disant~:
\VS{18}Yahweh est lent à la colère et riche en bonté, il ôte l'iniquité et pardonne la rébellion, mais il ne tient point le coupable pour innocent, et il punit l'iniquité des pères sur les fils, jusqu'à la troisième et à la quatrième génération\FTNT{Ex. 20:5~; Ex. 34:6~; Ex. 34:7~; Ps. 86:15~; Ps. 103:8~; Ps. 145:8~; Jon. 4:2~; De. 5:9.}.
\VS{19}Pardonne, je te prie, l'iniquité de ce peuple, selon la grandeur de ta miséricorde, comme tu as pardonné à ce peuple depuis l'Egypte jusqu'ici.
\TextTitle{Réponse de Yahweh à Moïse}
\VS{20}Et Yahweh dit~: Je pardonne selon ta parole.
\VS{21}Mais certainement je suis vivant, et la gloire de Yahweh remplira toute la terre.
\VS{22}Car tous ceux qui ont vu ma gloire, et les prodiges que j'ai faits en Egypte et dans le désert, qui m'ont déjà tenté par dix fois, et qui n'ont point écouté ma voix,
\VS{23}tous ceux-là ne verront point le pays que j'ai juré à leurs pères de leur donner, tous ceux, dis-je, qui m'ont irrité par mépris, ne le verront pas\FTNT{De. 1:35-38.}.
\VS{24}Mais parce que mon serviteur Caleb a été animé d'un autre esprit, et qu'il a persévéré à me suivre, je le ferai entrer dans le pays où il a été, et ses descendants le posséderont en héritage.
\VS{25}Or les Amalécites et les Cananéens habitent la vallée. Demain, tournez-vous et partez pour le désert, dans la direction de la Mer Rouge.
\VS{26}Yahweh parla à Moïse et à Aaron, en disant~:
\VS{27}Jusqu'à quand laisserai-je cette méchante assemblée murmurer contre moi~? J'ai entendu les murmures des enfants d'Israël, qui murmuraient contre moi\FTNT{Ps. 106:25.}.
\VS{28}Dis-leur~: Je suis vivant, dit Yahweh, je vous ferai ainsi que vous avez parlé à mes oreilles.
\VS{29}Vos cadavres tomberont dans ce désert, et tous ceux d'entre vous qui ont été dénombrés, selon tout le compte que vous en avez fait, depuis l'âge de vingt ans, et au dessus, vous tous qui avez murmuré contre moi~;
\VS{30}vous n'entrerez pas dans le pays que j'avais juré de vous faire habiter, excepté Caleb, fils de Jephunné, et Josué, fils de Nun.
\VS{31}Et quant à vos petits enfants, dont vous avez dit~: Ils deviendront une proie~! Je les y ferai entrer, et ils connaîtront le pays que vous avez méprisé.
\VS{32}Mais quant à vous, vos cadavres tomberont dans ce désert~;
\VS{33}mais vos enfants paîtront dans ce désert quarante ans et ils porteront la peine de vos prostitutions, jusqu'à ce que vos cadavres soient tous consumés dans le désert.
\VS{34}Selon le nombre des jours que vous avez mis à reconnaître le pays, qui ont été quarante jours, un jour pour une année, vous porterez la peine de vos iniquités quarante ans, et vous connaîtrez ma rupture de promesse.
\VS{35}Je suis Yahweh, j'ai parlé~! C'est ainsi que je traiterai cette méchante assemblée, qui s'est assemblée contre moi~; ils seront consumés dans ce désert, et ils y mourront.
\VS{36}Les hommes donc que Moïse avait envoyés pour épier le pays, et qui étant de retour avaient fait murmurer contre lui toute l'assemblée, en diffamant le pays~;
\VS{37}ces hommes-là, qui avaient décrié le pays, moururent frappés d'une plaie devant Yahweh.
\VS{38}Mais Josué, fils de Nun, et Caleb, fils de Jephunné, restèrent seuls vivants parmi ceux qui étaient allés pour explorer le pays.
\TextTitle{Israël battu par les Amalécites et les Cananéens}
\VS{39}Or Moïse dit ces choses à tous les enfants d'Israël, et le peuple fut dans un grand deuil.
\VS{40}Puis ils se levèrent de bon matin et montèrent au sommet de la montagne, en disant~: Nous voici, et nous monterons au lieu dont Yahweh a parlé car nous avons péché.
\VS{41}Mais Moïse leur dit~: Pourquoi transgressez-vous le commandement de Yahweh~? Cela ne réussira point.
\VS{42}Ne montez pas~; car Yahweh n'est pas au milieu de vous~; afin que vous ne soyez pas battus devant vos ennemis\FTNT{De. 1:41-42.}.
\VS{43}Car les Amalécites et les Cananéens sont là devant vous, et vous tomberez par l'épée~; parce que vous vous êtes détournés de Yahweh, Yahweh ne sera point avec vous.
\VS{44}Toutefois ils s'obstinèrent à monter au sommet de la montagne~; mais l'arche de l'alliance de Yahweh et Moïse ne sortirent point du milieu du camp.
\VS{45}Alors les Amalécites et les Cananéens qui habitaient sur cette montagne descendirent, les battirent, et les taillèrent en pièces jusqu'à Horma.
\Chap{15}
\TextTitle{Consignes pour le pays de Canaan}
\VerseOne{}Puis Yahweh parla à Moïse, en disant~:
\VS{2}Parle aux enfants d'Israël, et dis-leur~: Quand vous serez entrés au pays que je vous donne, où vous devez demeurer,
\VS{3}et que vous voudrez faire un sacrifice consumé par le feu à Yahweh, un holocauste, ou un sacrifice en accompagnement d'un vœu, ou en offrande volontaire, ou bien dans vos fêtes, pour produire avec votre gros ou votre menu bétail une agréable odeur à Yahweh\FTNT{Ex. 29:18~; Lé. 22:21.},
\VS{4}celui qui offrira son offrande à Yahweh présentera en offrande un dixième de fleur de farine, pétrie dans un quart de hin d'huile\FTNT{Lé. 2:1-2.},
\VS{5}et un quart de hin de vin pour la libation que tu feras sur l'holocauste, ou sur un autre sacrifice pour chaque agneau.
\VS{6}Si c'est pour un bélier, tu feras en offrande deux dixièmes de fleur de farine, pétrie dans un tiers de hin d'huile,
\VS{7}et un tiers de hin de vin pour la libation, comme offrande d'une bonne odeur à Yahweh.
\VS{8}Et si tu sacrifies un veau, soit comme holocauste, soit comme sacrifice en accompagnement d'un vœu, ou comme sacrifice d'offrande de paix à Yahweh,
\VS{9}on présentera en offrande avec le veau trois dixièmes de fleur de farine, pétrie dans un demi-hin d'huile.
\VS{10}Et tu offriras la moitié d'un hin de vin pour la libation, en offrande consumée par le feu d'une bonne odeur à Yahweh.
\VS{11}On fera de même pour chaque bœuf, chaque bélier, et chaque petit des brebis ou des chèvres.
\VS{12}Selon le nombre que vous en sacrifierez, vous ferez ainsi à chacun, d'après leur nombre.
\VS{13}Tous ceux qui sont nés au pays feront ces choses de cette manière, en offrant un sacrifice consumé par le feu, d'une bonne odeur à Yahweh.
\TextTitle{Loi sur l'étranger vivant au milieu d'Israël}
\VS{14}Si un étranger séjournant chez vous, ou se trouvant au milieu de vous en vos générations, offre un sacrifice consumé par le feu d'une bonne odeur à Yahweh, il l'offrira de la même manière que vous.
\VS{15}Ô assemblée~! Il y aura une même ordonnance pour vous et pour l'étranger qui fait son séjour parmi vous, il y aura une même ordonnance perpétuelle en vos âges~; il en sera de l'étranger comme de vous en la présence de Yahweh.
\VS{16}Il y aura une même loi et une seule ordonnance pour vous et pour l'étranger qui séjourne au milieu de vous.
\TextTitle{Lois diverses}
\VS{17}Yahweh parla à Moïse, en disant~:
\VS{18}Parle aux enfants d'Israël, et dis-leur~: Quand vous serez arrivés dans le pays où je vous ferai entrer,
\VS{19}et que vous mangerez du pain de ce pays, vous en offrirez à Yahweh une offrande élevée.
\VS{20}Vous offrirez en offrande élevée un gâteau, les prémices de votre pâte~; vous l'offrirez comme ce qu'on prélève de l'aire.
\VS{21}Vous donnerez pour Yahweh une offrande des prémices de votre pâte, dans les temps à venir.
\VS{22}Et lorsque vous aurez péché involontairement\FTNT{Voir commentaire en Lé. 4:2.}, et que vous n'aurez pas fait tous ces commandements que Yahweh a fait connaître à Moïse,
\VS{23}tout ce que Yahweh vous a commandé par Moïse, depuis le jour où Yahweh a commencé de donner ses commandements, et dans la suite dans vos générations,
\VS{24}s'il arrive que la chose ait été faite involontairement, sans que l'assemblée s'en soit aperçue, toute l'assemblée sacrifiera un jeune taureau en holocauste d'une bonne odeur à Yahweh, avec l'offrande et la libation, d'après les règles établies~; elle offrira encore un jeune bouc en sacrifice pour l'expiation.
\VS{25}Ainsi le prêtre fera propitiation pour toute l'assemblée des enfants d'Israël, et il leur sera pardonné parce que c'est une chose arrivée involontairement, et ils ont apporté leur offrande, un sacrifice consumé par le feu à Yahweh et l'offrande pour l'expiation devant Yahweh, à cause de leur péché involontaire.
\VS{26}Alors il sera pardonné à toute l'assemblée des enfants d'Israël, et à l'étranger qui séjourne au milieu d'eux, car c'est involontairement que tout le peuple a péché.
\VS{27}Si c'est une seule personne qui a péché involontairement, elle offrira une chèvre d'un an en offrande pour le péché\FTNT{Lé. 4:27-28.}.
\VS{28}Et le prêtre fera propitiation pour la personne qui aura péché involontairement, de ce qu'elle aura péché involontairement devant Yahweh, et faisant propitiation pour elle, il lui sera pardonné.
\VS{29}Il y aura une même loi pour celui qui aura fait quelque chose involontairement, tant pour celui qui est né au pays des enfants d'Israël, que pour l'étranger qui fait son séjour parmi eux.
\VS{30}Mais quant à celui qui aura péché par fierté, tant celui qui est né au pays, que l'étranger, il a outragé Yahweh, cette personne-là sera retranchée du milieu de son peuple.
\VS{31}Parce qu'il a méprisé la parole de Yahweh, et qu'il a enfreint son commandement. Cette personne donc sera certainement retranchée~; son iniquité est sur elle.
\TextTitle{Un homme lapidé selon la loi\FTNTT{Ro. 3:19~; 7:7-11~; 2 Co. 3:7-9~; Ga. 3:10.}}
\VS{32}Or comme les enfants d'Israël étaient dans le désert, on trouva un homme qui ramassait du bois le jour du sabbat.
\VS{33}Et ceux qui l'avaient trouvé ramassant du bois, l'amenèrent à Moïse, à Aaron, et à toute l'assemblée.
\VS{34}Et on le mit sous garde, car ce qu'on devait lui faire n'avait pas été déclaré.
\VS{35}Alors Yahweh dit à Moïse~: On punira de mort cet homme, et toute l'assemblée le lapidera hors du camp.
\VS{36}Toute l'assemblée donc le mena hors du camp et le lapida, et il mourut, comme Yahweh l'avait ordonné à Moïse.
\VS{37}Et Yahweh parla à Moïse, en disant~:
\VS{38}Parle aux enfants d'Israël, et dis-leur~: Qu'ils se fassent de génération en génération des franges aux bords de leurs vêtements, et qu'ils mettent sur les franges au bords de leurs vêtements un cordon de couleur pourpre\FTNT{De. 22:12~; Mt. 23:5.}.
\VS{39}Quand vous aurez cette frange, vous la regarderez et vous vous souviendrez de tous les commandements de Yahweh, pour les mettre en pratique, et vous ne suivrez pas les désirs de vos cœurs et de vos yeux, pour vous laisser entraîner à la prostitution.
\VS{40}Afin que vous vous souveniez de tous mes commandements, et que vous les fassiez, et que vous soyez saints à votre Dieu.
\VS{41}Je suis Yahweh, votre Dieu, qui vous ai retiré du pays d'Egypte, pour être votre Dieu. Je suis Yahweh, votre Dieu.
\Chap{16}
\TextTitle{La révolte de Koré\FTNTT{Jud. 11.}}
\VerseOne{}Or Koré\FTNT{Koré, Dathan et Abiram, s'étaient révoltés contre Aaron et Moïse, car ils voulaient s'attribuer l'honneur d'offrir à Dieu des sacrifices. Ils voulaient exercer la prêtrise (sacerdoce) alors que Yahweh ne les avait pas établis pour le service du culte. Vouloir servir Dieu sans avoir reçu un appel divin est dangereux.}, fils de Jitsehar, fils de Kehath, fils de Lévi, se révolta avec Dathan et Abiram, fils d'Eliab, et On, fils de Péleth, tous trois fils de Ruben.
\VS{2}Et ils s'élevèrent contre Moïse, avec deux cent cinquante hommes des fils d'Israël, qui étaient des principaux de l'assemblée, de ceux que l'on convoquait pour tenir le conseil, et qui étaient des gens de renom.
\VS{3}Et ils s'assemblèrent contre Moïse et contre Aaron, et leur dirent~: C'en est assez~! Puisque tous ceux de l'assemblée sont saints, et que Yahweh est au milieu d'eux, pourquoi vous élevez-vous au-dessus de l'assemblée de Yahweh~?
\VS{4}Quand Moïse eut entendu cela, il se jeta sur son visage.
\VS{5}Et il parla à Koré et à tous ceux qui étaient assemblés avec lui, et leur dit~: Demain au matin, Yahweh fera connaître celui qui lui appartient, et celui qui est saint, et il le fera approcher de lui~; il fera, dis-je, approcher de lui celui qu'il aura choisi.
\VS{6}Faites ceci, prenez des encensoirs, Koré et toute son assemblée.
\VS{7}Et demain, mettez-y du feu, et mettez-y du parfum devant Yahweh~; et celui que Yahweh choisira, c'est celui-là qui sera saint. C'en est assez, fils de Lévi~!
\VS{8}Moïse dit aussi à Koré~: Ecoutez maintenant, fils de Lévi~:
\VS{9}Est-ce trop peu de chose pour vous, que le Dieu d'Israël vous ait séparés de l'assemblée d'Israël, pour vous faire approcher de lui, afin de faire le service du tabernacle de Yahweh, et pour vous tenir devant l'assemblée, afin de la servir~?
\VS{10}Et qu'il t'ait fait approcher de lui, toi et tous tes frères, les fils de Lévi, et vous recherchez encore la prêtrise~!
\VS{11}C'est pourquoi toi et toute ton assemblée, vous vous êtes rassemblés contre Yahweh~! Car qui est Aaron pour que vous murmuriez contre lui~?
\VS{12}Et Moïse envoya appeler Dathan et Abiram, fils d'Eliab, qui répondirent~: Nous n'y monterons point.
\VS{13}Est-ce peu de chose que tu nous aies fait monter hors d'un pays où coulent le lait et le miel, pour nous faire mourir dans le désert, que tu veuilles aussi dominer sur nous~?
\VS{14}Certes, tu ne nous as pas fait venir dans un pays où coulent le lait et le miel~! Et tu ne nous as pas donné un héritage de champs ni de vignes~! Veux-tu crever les yeux de ces gens~? Nous ne monterons pas.
\VS{15}Alors Moïse fut très irrité, et il dit à Yahweh~: N'aie point égard à leur offrande. Je n'ai point pris d'eux un seul âne, et je n'ai fait de mal à aucun d'eux.
\VS{16}Puis Moïse dit à Koré~: Toi et tous ceux qui sont assemblés avec toi, trouvez-vous demain devant Yahweh, toi et eux avec Aaron.
\VS{17}Et prenez chacun vos encensoirs, et mettez-y du parfum~; et que chacun présente devant Yahweh son encensoir~: Il y aura deux cent cinquante encensoirs~; toi et Aaron aussi, chacun avec son encensoir.
\VS{18}Ils prirent donc chacun son encensoir, et y mirent du feu, et ensuite y posèrent du parfum, et ils se tinrent à l'entrée de la tente d'assignation, avec Moïse et Aaron.
\VS{19}Et Koré fit assembler contre eux toute l'assemblée à l'entrée de la tente d'assignation~; et la gloire de Yahweh apparut à toute l'assemblée.
\VS{20}Puis Yahweh parla à Moïse et à Aaron, en disant~:
\VS{21}Séparez-vous du milieu de cette assemblée, et je les consumerai en un seul instant\FTNT{Ex. 32:10.}.
\VS{22}Mais ils tombèrent sur leur visage et dirent~: Ô Dieu~! Dieu des esprits de toute chair~! Un seul homme a péché, et tu te mettrais en colère contre toute l'assemblée\FTNT{Hé. 12:9.}~?
\VS{23}Et Yahweh parla à Moïse, en disant~:
\VS{24}Parle à l'assemblée, et dis lui~: Retirez-vous d'auprès de la demeure de Koré, de Dathan, et d'Abiram.
\VS{25}Moïse donc se leva, et alla vers Dathan et Abiram~; et les anciens d'Israël le suivirent.
\VS{26}Et il parla à l'assemblée, en disant~: Eloignez-vous, je vous prie, d'auprès des tentes de ces méchants hommes, et ne touchez à rien qui leur appartienne, de peur que vous ne périssiez punis pour tous leurs péchés.
\VS{27}Ils se retirèrent donc d'auprès des demeures de Koré, de Dathan et d'Abiram. Et Dathan et Abiram sortirent et se tinrent debout à l'entrée de leurs tentes, avec leurs femmes, leurs fils, et leurs petits-enfants.
\VS{28}Et Moïse dit~: Vous connaîtrez à ceci que Yahweh m'a envoyé pour faire toutes ces choses, et que je n'agis pas de moi-même.
\VS{29}Si ces gens meurent comme tous les hommes meurent, et s'ils subissent le sort commun à tous les hommes, Yahweh ne m'a point envoyé~;
\VS{30}mais si Yahweh fait une chose nouvelle, et si la terre ouvre sa bouche pour les engloutir avec tout ce qui leur appartient, et qu'ils descendent vivants dans le scheol, vous saurez alors que ces hommes-là ont irrité par mépris Yahweh.
\VS{31}Et il arriva qu'aussitôt qu'il eut achevé de dire toutes ces paroles, la terre qui était sous eux se fendit.
\VS{32}Et la terre ouvrit sa bouche et les engloutit, avec leurs tentes et tous les hommes qui étaient à Koré, et tous leurs biens\FTNT{De. 11:6~; Ps. 106:17.}.
\VS{33}Ils descendirent donc vivants dans le scheol, eux et tout ceux qui leur appartenait~; la terre les recouvrit, et ils disparurent au milieu de l'assemblée.
\VS{34}Et tout Israël qui était autour d'eux s'enfuit à leurs cris~; car ils disaient~: Prenons garde que la terre ne nous engloutisse~!
\VS{35}Un feu sortit de part Yahweh et consuma les deux cent cinquante hommes qui offraient le parfum.
\VS{36}Puis Yahweh parla à Moïse, en disant~:
\VS{37}Dis à Eléazar, fils d'Aaron, le prêtre, qu'il ramasse les encensoirs du milieu de l'embrasement, et d'en répandre au loin le feu, car ils sont sanctifiés.
\VS{38}Avec les encensoirs de ceux qui ont péché contre leurs âmes, que l'on fasse des lames étendues dont on couvrira l'autel. Puisqu'ils ont été offerts devant Yahweh et qu'ils sont sanctifiés, ils serviront de signe aux enfants d'Israël.
\VS{39}Ainsi Eléazar, le prêtre, prit les encensoirs d'airain, que ces hommes qui furent brûlés avaient présentés, et on en fit des lames pour couvrir l'autel.
\VS{40}C'est un souvenir pour les enfants d'Israël, afin qu'aucun étranger qui n'est pas de la race d'Aaron, ne s'approche pour offrir du parfum devant Yahweh, et ne soit comme Koré, et comme ceux qui ont été assemblés avec lui~; selon ce que Yahweh avait déclaré par Moïse.
\TextTitle{Le peuple frappé à cause des murmures}
\VS{41}Or dès le lendemain, toute l'assemblée des enfants d'Israël murmura contre Moïse et contre Aaron, en disant~: Vous avez fait mourir le peuple de Yahweh.
\VS{42}Et il arriva comme l'assemblée s'amassait contre Moïse et contre Aaron, et comme ils tournaient les regards vers la tente d'assignation, voici la nuée la couvrit, et la gloire de Yahweh apparut.
\VS{43}Moïse donc et Aaron vinrent donc devant la tente d'assignation.
\VS{44}Et Yahweh parla à Moïse, en disant~:
\VS{45}Retirez-vous du milieu de cette assemblée, et je les consumerai en un instant. Alors ils se prosternèrent le visage contre terre~;
\VS{46}puis Moïse dit à Aaron~: Prends l'encensoir, et mets-y du feu de dessus l'autel, mets-y aussi du parfum, et va promptement à l'assemblée, et fais propitiation pour eux~; car une grande colère est sortie de devant Yahweh, la plaie a commencé.
\VS{47}Et Aaron prit l'encensoir, comme Moïse lui avait dit, et il courut au milieu de l'assemblée, et voici la plaie avait déjà commencé sur le peuple. Alors il mit du parfum et fit propitiation pour le peuple.
\VS{48}Et comme il se tenait entre les morts et les vivants, la plaie fut arrêtée.
\VS{49}Et il y en eut quatorze mille sept cents qui moururent de cette plaie, outre ceux qui étaient morts à cause de Koré.
\VS{50}Et Aaron retourna auprès de Moïse, à l'entrée de la tente d'assignation, et la plaie s'arrêta.
\Chap{17}
\TextTitle{Yahweh confirme l'appel d'Aaron, sa verge fleurit}
\VerseOne{}Après cela Yahweh parla à Moïse, en disant~:
\VS{2}Parle aux enfants d'Israël, et prends une verge de chacun d'eux selon la maison de leur père, de tous ceux qui sont les princes, selon la maison de leurs pères, douze verges, puis tu écriras le nom de chacun sur sa verge,
\VS{3}mais tu écriras le nom d'Aaron sur la verge de Lévi\FTNT{La verge d'Aaron est une image du Messie ressuscité. Elle avait produit la vie tandis que celles des autres princes n'avaient produit aucun fruit. Cette histoire nous parle également de la confirmation de l'appel d'Aaron face aux critiques dont il était l'objet. On reconnaît l'arbre par ses fruits (Mt. 7:16-20~; Lu. 7:17-22).}~; car il y aura une verge pour chaque chef des maisons de leurs pères.
\VS{4}Et tu les déposeras dans la tente d'assignation, devant le témoignage, où je me rencontre avec vous.
\VS{5}Et la verge de l'homme que j'aurai choisi fleurira~; et je ferai cesser de devant moi les murmures des enfants d'Israël, par lesquels ils murmurent contre vous.
\VS{6}Quand Moïse parla aux enfants d'Israël, tous leurs princes lui donnèrent une verge, chaque prince une verge, selon les maisons de leurs pères, soit douze verges~; or la verge d'Aaron était au milieu des leurs.
\VS{7}Et Moïse mit les verges devant Yahweh, dans la tente du témoignage.
\VS{8}Et le lendemain, lorsque Moïse entra dans la tente du témoignage, voici, la verge d'Aaron, avait fleuri, pour la maison de Lévi, et elle avait poussé des boutons, produit des fleurs et mûri des amandes.
\VS{9}Alors Moïse ôta de devant Yahweh toutes les verges et les porta à tous les fils d'Israël, afin qu'ils les voient et qu'ils prennent chacun leurs verges.
\VS{10}Et Yahweh dit à Moïse~: Reporte la verge d'Aaron devant le témoignage, pour être conservée comme un signe pour les fils de rébellion, afin que tu fasses cesser de devant moi leurs murmures et qu'ils ne meurent point\FTNT{Hé. 9:3-5.}.
\VS{11}Et Moïse fit ainsi~; il se conforma à l'ordre que Yahweh lui avait donné.
\VS{12}Les enfants d'Israël parlèrent à Moïse, en disant~: Voici, nous expirons, nous périssons, nous périssons tous~!
\VS{13}Quiconque s'approche du tabernacle de Yahweh, meurt. Serons-nous tous entièrement expirés~?
\Chap{18}
\TextTitle{Droits et devoirs des prêtres et des Lévites}
\VerseOne{}Alors Yahweh dit à Aaron~: Toi et tes fils, et la maison de ton père avec toi, vous porterez l'iniquité du sanctuaire~; et toi, et tes fils avec toi, vous porterez l'iniquité de votre prêtrise.
\VS{2}Fais aussi approcher de toi tes frères, la tribu de Lévi, qui est la tribu de ton père, afin qu'ils te soient attachés et qu'ils te servent, mais toi et tes fils avec toi, vous servirez devant la tente du témoignage.
\VS{3}Ils garderont ce que tu leur ordonneras de garder, et ce qu'il faut garder de toute la tente, mais ils n'approcheront point des ustensiles du sanctuaire, ni de l'autel de peur qu'ils ne meurent, et que vous ne mouriez avec eux.
\VS{4}Ils te seront donc attachés, et ils garderont tout ce qu'il faut garder dans la tente d'assignation, selon tout le service du tabernacle et aucun étranger n'approchera de vous.
\VS{5}Mais vous prendrez garde à ce qu'il faut faire dans le sanctuaire, et à ce qu'il faut faire à l'autel, afin qu'il n'y ait plus d'indignation sur les enfants d'Israël.
\VS{6}Car quant à moi voici, j'ai pris vos frères, les Lévites, du milieu des enfants d'Israël, qui sont donnés en pur don pour Yahweh, afin qu'ils soient employés au service de la tente d'assignation.
\VS{7}Mais toi et tes fils avec toi, vous observerez la fonction de votre prêtrise en tout ce qui concerne l'autel et ce qui est au dedans du voile, et vous y ferez le service. J'établis votre prêtrise en office de pur don~; c'est pourquoi si un étranger en approche, on le fera mourir.
\VS{8}Yahweh dit encore à Aaron~: Voici, je t'ai donné la garde de mes offrandes élevées sur toutes les choses consacrées par les enfants d'Israël~; je te les ai données, et à tes enfants, par ordonnance perpétuelle, à cause de l'onction.
\VS{9}Ceci t'appartiendra d'entre les choses très saintes qui ne sont pas brûlées, savoir toutes leurs offrandes, soit de tous leurs gâteaux, soit de tous leurs sacrifices pour l'expiation, et tous leurs sacrifices pour la culpabilité qu'ils m'apporteront~; ce sont des choses très saintes pour toi et pour tes enfants.
\VS{10}Vous les mangerez dans un lieu très saint~; tout mâle en mangera~; vous les regarderez comme saintes\FTNT{Lé. 6:17-22~; Lé. 7:6~; Lé. 10:13.}.
\VS{11}Voici encore ce qui t'appartiendra~: Tous les dons que les enfants d'Israël présenteront par élévation et en les agitant de côté et d'autre, je te les donne à toi, à tes fils, et à tes filles avec toi, par une loi perpétuelle~; quiconque sera pur dans ta maison en mangera\FTNT{Lé. 7:34~; Lé. 10:14.}.
\VS{12}Je te donne aussi leurs prémices qu'ils offriront à Yahweh~: Tout ce qu'il y aura de meilleur en huile, et tout le meilleur du moût et du blé.
\VS{13}Les premiers fruits de toutes les choses que leur terre produira, et qu'ils apporteront à Yahweh t'appartiendront~; quiconque sera pur dans ta maison, en mangera.
\VS{14}Tout ce qui sera dévoué en Israël t'appartiendra\FTNT{Lé. 27:28~; Ez. 44:29.}.
\VS{15}Tout premier-né de toute chair, qu'ils offriront à Yahweh, tant des hommes que des animaux t'appartiendront. Mais, tu feras racheter le premier-né de l'homme, et tu feras racheter le premier-né d'un animal impur.
\VS{16}Et ceux qui doivent être rachetés, depuis l'âge d'un mois, tu les rachèteras selon ton estimation que tu en feras, au prix de cinq sicles d'argent, selon le sicle du sanctuaire, qui est de vingt guéras.
\VS{17}Mais tu ne feras point racheter le premier-né du bœuf, ni le premier-né de la brebis, ni le premier-né de la chèvre~: Ce sont des choses saintes. Tu répandras leur sang sur l'autel, et tu brûleras leur graisse~: Ce sera un sacrifice consumé par le feu d'une bonne odeur à Yahweh.
\VS{18}Mais leur chair t'appartiendra, comme la poitrine qu'on agite de côté et d'autre, et comme l'épaule droite.
\VS{19}Je t'ai donné, à toi et à tes fils, et à tes filles avec toi, par une loi perpétuelle, toutes les offrandes présentées par élévation des choses sanctifiées, que les enfants d'Israël offriront à Yahweh. C'est une alliance de sel\FTNT{Le sel est un aliment pratiquement impérissable et incorruptible. Dans l'Antiquité, il symbolisait l'incorruptibilité (Lé. 2:13).} et à perpétuité devant Yahweh, pour toi et pour ta postérité avec toi.
\VS{20}Puis Yahweh dit à Aaron~: Tu ne posséderas rien dans leur pays, et il n'y aura point de part pour toi au milieu d'eux~; c'est moi qui suis ta part et ta possession, au milieu des enfants d'Israël\FTNT{De. 10:9~; De. 18:2~; Ez. 44:28.}.
\TextTitle{Lois sur les dîmes (De. 14:22-29)}
\VS{21}Et je donne comme possession aux fils de Lévi, toutes les dîmes\FTNT{Il y avait plusieurs sortes de dîmes dans la loi Mosaïque~:
\\- La 1ère dîme~: Le peuple devait payer une dîme générale au bénéfice des Lévites (No. 18:21).
\\Toutes les tribus d'Israël, à l'exception des Lévites, eurent une possession géographique qu'ils reçurent comme héritage après leur entrée en Canaan. Mais les Lévites devaient accomplir une tâche particulière au sein de la nation. Ils devaient s'occuper du service dans la tente d'assignation. En compensation de ce service, ils devaient percevoir un impôt de 10\% des revenus de tous les Israélites.
\\- La 2ème dîme~: Les Lévites devaient payer la «~dîme de la dîme~», au bénéfice des prêtres (No. 18:25-31).
\\Tous les prêtres étaient des Lévites, mais tous les Lévites n'étaient pas des prêtres. Les prêtres descendaient d'Aaron et ils exerçaient des responsabilités particulières dans la tente d'assignation, puis dans le temple. Cette seconde dîme permettait aux prêtres d'être nourris et assurait donc le bon fonctionnement du service du temple.
\\- La 3ème dîme~: Tous les Israélites devaient conserver une dîme de toute leur production en prévision de leurs pèlerinages annuels à Jérusalem (De. 14:22-26).
\\Trois fois par an, tout le peuple devait s'assembler à Jérusalem, l'endroit choisi par le Seigneur, à l'occasion des principales fêtes. Dieu avait prévu que chacun puisse disposer de ressources suffisantes pour leur permettre de se réjouir pleinement à ces occasions. C'est pour cela qu'ils devaient mettre de côté 10\% de leurs productions agricoles annuelles. Il est intéressant de noter que la dîme n'était jamais payée en argent, mais toujours en nature.
\\- La 4ème dîme~: Il fallait payer une dîme spéciale à l'intention des pauvres, des orphelins et des veuves (De. 14:28-29). 
\\Certains affirment que la dîme existait bien avant la loi. Mais ils ignorent que la Bible parle de plusieurs sortes de lois.
\\- Les lois cérémonielles (Hé. 9:1)
\\Ces lois étaient relatives au culte et concernaient le tabernacle puis le temple, les sacrifices, les ablutions (Lé. 16~; Hé. 9:1-10). Les dîmes (la dîme des prêtres) devaient être amenées dans le temple (Mal. 3:10), elles faisaient donc partie des lois cérémonielles. Or les Lévites et les prêtres de la Première Alliance n'existent plus sous la Nouvelle Alliance car les enfants de Dieu sont un royaume de rois et de prêtres (Ap. 1:6~; Ap. 5:10).
\\- Les lois morales (Ex. 20:1-17). Dieu est saint et il veut un peuple saint qui marche dans sa crainte, dans la sainteté et dans l'obéissance. Lé. 18 nous parle des lois morales~; elles n'ont pas été abolies, elles existent toujours. Elles sont inscrites dans la conscience de l'homme, elles sont gravées dans notre cœur (Hé. 8:10).
\\- Les lois sociales (Ex. 21:1-24). Ce sont des lois civiles régissant la vie sociale d'Israël, comme nous pouvons le lire dans Ex. 21 par exemple. Ces lois n'ont rien à voir avec les croyants de la Nouvelle Alliance. Les lois morales témoignent de la nature de Dieu, ce sont des lois éternelles qui existaient bien avant Abraham. Les lois cérémonielles ont commencé dès la fondation du monde (Ap. 13:8) car l'Agneau de Dieu était immolé avant la fondation du monde (1 Pi. 1:19-20). Seules les lois sociales ont débuté avec Moïse car elles concernaient exclusivement les Israélites. Ces trois sortes de lois ont été institutionnalisées par Moïse, mais les deux premières (morales et cérémonielles) existaient avant ce dernier. Les quatre sortes de dîmes faisaient bel et bien partie des lois sociales et cérémonielles. Or ces lois ne sont plus d'actualité sous la Nouvelle Alliance. En conclusion, nous pouvons dire que Jésus nous a rachetés en accomplissant les lois cérémonielles afin que nous pratiquions les lois morales (Ep. 2:10). Voir également commentaire en Mal 3~: 10.} d'Israël, pour le service auquel ils sont employés, le service de la tente d'assignation.
\VS{22}Et les enfants d'Israël n'approcheront plus de la tente d'assignation, afin qu'ils ne se chargent d'un péché et qu'ils ne meurent point.
\VS{23}Mais les Lévites s'emploieront au service de la tente d'assignation, et ils resteront chargés de leurs iniquités. Cette loi sera perpétuelle parmi vos descendants, et ils ne posséderont point d'héritage parmi les enfants d'Israël.
\VS{24}Car je donne comme possession aux Lévites les dîmes que les enfants d'Israël présenteront à Yahweh en offrande élevée~; c'est pourquoi je dis d'eux qu'ils n'auront point d'héritage parmi les fils d'Israël.
\VS{25}Puis Yahweh parla à Moïse, en disant~:
\VS{26}Tu parleras aussi aux Lévites, et tu leur diras~: Quand vous recevrez des enfants d'Israël les dîmes que je vous donne de leur part comme possession, vous en offrirez l'offrande élevée à Yahweh, la dîme de la dîme~;
\VS{27}et votre offrande élevée vous sera comptée comme le blé qu'on prélève de l'aire, et comme l'abondance qu'on prélève de la cuve.
\VS{28}C'est ainsi que vous prélèverez une offrande pour Yahweh de toutes les dîmes que vous recevrez des enfants d'Israël, et vous donnerez au prêtre Aaron l'offrande que vous en aurez prélevée pour Yahweh.
\VS{29}Sur tous les dons qui vous seront faits, vous prélèverez toute l'offrande élevée pour Yahweh~; sur tout ce qu'il y aura de meilleur, vous prélèverez la portion consacrée.
\VS{30}Et tu leur diras~: Quand vous aurez offert en offrande élevée le meilleur de la dîme, pris de la dîme même, il sera imputé aux Lévites comme le revenu de l'aire, et comme le revenu de la cuve.
\VS{31}Et vous la mangerez en tout lieu, vous et votre maison~; car c'est votre salaire pour le service auquel vous êtes employés dans la tente d'assignation.
\VS{32}Vous ne serez point coupables de péché au sujet de la dîme, quand vous en aurez offert en offrande élevée sur ce qu'il y aura de meilleur et vous ne souillerez point les choses saintes des enfants d'Israël et vous ne mourrez point.
\Chap{19}
\TextTitle{La jeune vache rousse~; l'eau de purification}
\VerseOne{}Yahweh parla à Moïse et à Aaron, en disant~:
\VS{2}Voici ce qui est ordonné par la loi que Yahweh a commandé, en disant~: Parle aux enfants d'Israël, et dis-leur qu'ils t'amènent une jeune vache rousse, entière, sans défaut, et qui n'ait point porté le joug.
\VS{3}Puis vous la donnerez à Eléazar, le prêtre, qui la mènera hors du camp, et on l'égorgera en sa présence\FTNT{Lé. 4:12~; Hé. 13:11-12.}.
\VS{4}Ensuite, Eléazar, le prêtre, prendra de son sang avec son doigt, et fera sept fois l'aspersion du sang vers le devant de la tente d'assignation.
\VS{5}Et on brûlera la jeune vache en sa présence~; on brûlera sa peau, sa chair, son sang et ses excréments\FTNT{Ex. 29:14.}.
\VS{6}Le prêtre prendra du bois de cèdre, de l'hysope, et du cramoisi, et les jettera dans le feu où sera brûlée la jeune vache.
\VS{7}Puis le prêtre lavera ses vêtements et son corps avec de l'eau~; après cela, il rentrera au camp, et le prêtre sera impur jusqu'au soir.
\VS{8}Celui qui l'aura brûlé, lavera ses vêtements dans l'eau, il lavera aussi dans l'eau son corps~; et il sera impur jusqu'au soir.
\VS{9}Et un homme pur ramassera les cendres de la jeune vache, et les mettra hors du camp, dans un lieu pur~; elles seront gardées pour l'assemblée des enfants d'Israël~; afin d'en faire l'eau de purification. C'est une purification pour le péché.
\VS{10}Celui qui aura ramassé les cendres de la jeune vache, lavera ses vêtements, et sera impur jusqu'au soir~; ce sera une loi perpétuelle pour les enfants d'Israël, et pour l'étranger en séjour au milieu d'eux.
\VS{11}Celui qui touchera un mort, un corps humain quel qu'il soit, sera impur pendant sept jours\FTNT{Ag. 2:13.}.
\VS{12}Il se purifiera avec cette eau le troisième jour et le septième jour, et il sera pur~; mais s'il ne se purifie pas le troisième jour, il ne sera pas pur le septième jour.
\VS{13}Alors celui qui touchera un mort, le corps d'un homme qui sera mort et qui ne se purifiera pas, souille le tabernacle de Yahweh~; celui-là sera retranché d'Israël. Il est impur, car l'eau de purification n'a pas été répandue sur lui, son impureté demeure encore sur lui.
\VS{14}Voici la loi. Lorsqu'un homme mourra dans une tente, quiconque entrera dans la tente, et quiconque se trouvera dans la tente sera impur pendant sept jours.
\VS{15}Aussi tout vase découvert, sur lequel il n'y aura point de couvercle attaché, sera impur.
\VS{16}Et quiconque touchera, dans les champs, un homme qui aura été tué par l'épée, ou un mort, ou des ossements humains, ou un sépulcre, sera impur durant sept jours.
\VS{17}Et on prendra, pour celui qui est impur, de la poudre de la jeune vache brûlée pour faire la purification, et on la mettra dans un vase, avec de l'eau vive par-dessus.
\VS{18}Puis un homme pur prendra de l'hysope, et la trempera dans l'eau~; il en fera aspersion sur la tente, et sur tous les ustensiles, et sur toutes les personnes qui auront été là, et sur celui qui a touché des ossements ou un homme tué, ou un mort, ou un sépulcre.
\VS{19}Celui qui est pur fera l'aspersion sur celui qui est impur, le troisième jour et le septième jour, et il le purifiera le septième jour~; puis il lavera ses vêtements, et se lavera dans l'eau, et il sera pur le soir.
\VS{20}Mais l'homme qui sera impur, et qui ne se purifiera point, sera retranché du milieu de l'assemblée, parce qu'il a souillé le sanctuaire de Yahweh~; comme l'eau de purification n'a pas été répandue sur lui, il est impur.
\VS{21}Et ce sera pour eux une loi perpétuelle, et celui qui fera l'aspersion de l'eau de purification lavera ses vêtements~; et quiconque touchera l'eau de purification sera impur jusqu'au soir.
\VS{22}Et tout ce que l'homme impur touchera sera souillé, et la personne qui le touchera sera impure jusqu'au soir.
\Chap{20}
\TextTitle{Mort de Marie}
\VerseOne{}Or toute l'assemblée des enfants d'Israël arriva dans le désert de Tsin au premier mois, et le peuple s'arrêta à Kadès. Marie mourut là, et y fut ensevelie.
\TextTitle{Murmures du peuple à cause du manque d'eau\FTNTT{De. 32:51~; cp. Ex. 17:1-7.}}
\VS{2}Et il n'y avait point d'eau pour l'assemblée~; et ils se soulevèrent contre Moïse et contre Aaron.
\VS{3}Et le peuple contesta contre Moïse et ils lui dirent~: Pourquoi ne sommes-nous pas morts quand nos frères moururent devant Yahweh~?
\VS{4}Et pourquoi avez-vous fait venir l'assemblée de Yahweh dans ce désert, pour que nous y mourions, nous et notre bétail\FTNT{Ex. 17:3.}~?
\VS{5}Et pourquoi nous avez-vous fait monter hors d'Egypte, pour nous amener dans ce méchant lieu qui n'est pas un lieu où l'on puisse semer, ni un lieu pour des figuiers, ni pour des vignes, ni pour des grenadiers, et sans eau pour boire~?
\VS{6}Alors Moïse et Aaron se retirèrent de devant l'assemblée à l'entrée de la tente d'assignation et ils tombèrent sur leurs faces~; et la gloire de Yahweh apparut.
\TextTitle{Incrédulité de Moïse et d'Aaron à Meriba}
\VS{7}Yahweh parla à Moïse, en disant~:
\VS{8}Prends la verge, et convoque l'assemblée, toi et Aaron, ton frère. Vous parlerez en leur présence au rocher\FTNT{Christ, le rocher des âges ( Es. 8:13-17~; 1 Co. 10:1-4).}, et il donnera son eau~; ainsi tu leur feras sortir de l'eau du rocher, et tu donneras à boire à l'assemblée et à leur bétail.
\VS{9}Moïse prit la verge qui était devant Yahweh, comme il lui avait ordonné.
\VS{10}Moïse et Aaron convoquèrent l'assemblée devant le rocher. Et il leur dit~: Ecoutez donc, rebelles~! Est-ce de ce rocher que nous vous ferons sortir de l'eau~?
\VS{11}Puis Moïse leva sa main, et frappa deux fois le rocher avec sa verge et il en sortit des eaux en abondance. L'assemblée but, et leur bétail aussi.
\VS{12}Alors Yahweh dit à Moïse et à Aaron~: Parce que vous n'avez pas cru en moi, pour me sanctifier aux yeux des enfants d'Israël, ainsi vous ne ferez point entrer cette assemblée dans le pays que je lui donne.
\VS{13}Ce sont là, les eaux de Meriba, où les enfants d'Israël contestèrent avec Yahweh, qui fut sanctifié en eux.
\TextTitle{La méchanceté d'Edom\FTNTT{Ge. 25:30~; Ab. 10.}}
\VS{14}Puis Moïse envoya des ambassadeurs de Kadès au roi d'Edom, pour lui dire~: Ainsi parle ton frère Israël~: Tu sais toutes les souffrances que nous avons eu.
\VS{15}Comment nos pères descendirent en Egypte, où nous avons demeuré longtemps~; et comment les Egyptiens nous ont maltraités, nous et nos pères.
\VS{16}Et nous avons crié à Yahweh, et il a entendu nos cris. Il a envoyé l'Ange et nous a retirés d'Egypte. Et voici, nous sommes à Kadès, ville qui est à l'extrémité de ton territoire\FTNT{Ex. 2:23~; Ex. 23:20~; Ac. 7:30-38.}.
\VS{17}Je te prie, laisse-nous passer par ton pays~; nous ne traverserons ni les champs ni les vignes, et nous ne boirons l'eau d'aucun puits~; nous marcherons par le chemin royal~; nous ne nous détournerons ni à droite ni à gauche, jusqu'à ce que nous ayons passé ton territoire.
\VS{18}Et Edom lui dit~: Tu ne passeras point par mon pays, de peur que je ne sorte en armes à ta rencontre.
\VS{19}Les enfants d'Israël lui répondirent~: Nous monterons par le grand chemin, et si nous buvons de tes eaux, moi et mes bêtes, je t'en payerai le prix~; je veux seulement passer à pied.
\VS{20}Mais il lui répondit~: Tu ne passeras pas~! Et sur cela, Edom sortit à sa rencontre avec une grande multitude, et à main armée.
\VS{21}Ainsi Edom ne voulut point permettre à Israël de passer par ses frontières~; c'est pourquoi Israël se détourna de lui.
\VS{22}Et toute l'assemblée des enfants d'Israël partit de Kadès et arriva à la montagne de Hor.
\TextTitle{Mort d'Aaron}
\VS{23}Et Yahweh parla à Moïse et à Aaron à la montagne de Hor, près des frontières du pays d'Edom, en disant~:
\VS{24}Aaron sera recueilli auprès de son peuple, car il n'entrera pas dans le pays que je donne aux enfants d'Israël, parce que vous avez été rebelles à mon commandement aux eaux de la dispute\FTNT{«~Meriba~»}.
\VS{25}Prends donc Aaron et Eléazar, son fils, et fais-les monter sur la montagne de Hor.
\VS{26}Puis fais dépouiller Aaron de ses vêtements, et fais-les revêtir à Eléazar, son fils. C'est là qu'Aaron sera recueilli et qu'il mourra.
\VS{27}Moïse fit ce que Yahweh avait ordonné~; et ils montèrent sur la montagne de Hor, aux yeux de toute l'assemblée.
\VS{28}Et Moïse dépouilla Aaron de ses vêtements et en fit revêtir Eléazar, son fils. Aaron mourut là, au sommet de la montagne. Moïse et Eléazar descendirent de la montagne\FTNT{De. 10:6.}.
\VS{29}Toute l'assemblée, toute la maison d'Israël, voyant qu'Aaron était mort, le pleurèrent trente jours.
\Chap{21}
\TextTitle{Les Cananéens livrés à Israël}
\VerseOne{}Quand le roi d'Arad, Cananéen, qui habitait le midi, eut appris qu'Israël venait par le chemin d'Atharim, il combattit Israël et emmena des prisonniers.
\VS{2}Alors Israël fit un vœu à Yahweh, en disant~: Si tu livres ce peuple entre mes mains, je dévouerai ses villes par le moyen de l'interdit.
\VS{3}Et Yahweh exauça la voix d'Israël et livra entre ses mains les Cananéens. On les dévoua par interdit, avec leurs villes~; et on donna à ce lieu le nom de Horma.
\TextTitle{Le serpent d'airain\FTNTT{Jn. 3:14-15~; 2 Co. 5:20.}}
\VS{4}Puis ils partirent de la montagne de Hor, par le chemin de la Mer Rouge, pour faire le tour du pays d'Edom. Le cœur du peuple s'impatienta en route,
\VS{5}et parla contre Dieu, et contre Moïse, en disant~: Pourquoi nous as-tu fait monter hors d'Egypte, pour mourir dans ce désert~? Car il n'y a point de pain ni d'eau, et notre âme est dégoûtée de cette nourriture misérable.
\VS{6}Et Yahweh envoya contre le peuple des serpents brûlants qui mordaient le peuple~; tellement qu'il en mourut un grand nombre en Israël\FTNT{1 Co. 10:9.}.
\VS{7}Alors le peuple vint vers Moïse, et dit~: Nous avons péché, car nous avons parlé contre Yahweh et contre toi. Invoque Yahweh afin qu'il éloigne de nous les serpents et Moïse pria pour le peuple.
\VS{8}Et Yahweh dit à Moïse~: Fais-toi un serpent brûlant, et mets-le sur une perche~; quiconque aura été mordu et le regardera conservera la vie.
\VS{9}Moïse fit un serpent d'airain\FTNT{Voir Jn. 3:14-16. Ceux qui regardent à Jésus-Christ, et non aux hommes, obtiennent la délivrance. L'airain nous parle du jugement (Job 20:24), le serpent de la malédiction (Ge. 3:14), et la perche parle de la croix (1 Co. 1:18). Jésus a pris nos malédictions sur la croix de Golgotha (Ga. 3:13).}, et le mit sur une perche~; quiconque avait été mordu par un serpent et regardait le serpent d'airain conservait la vie.
\VS{10}Les enfants d'Israël partirent et campèrent à Oboth.
\VS{11}Et ils partirent d'Oboth et ils campèrent en Ijjé-Abarim, dans le désert qui est vis-à-vis de Moab, vers le soleil levant.
\VS{12}Puis ils partirent de là et campèrent vers le torrent de Zéred.
\VS{13}Et ils partirent de là et campèrent de l'autre côté de l'Arnon, qui est dans le désert, en sortant du territoire des Amoréens~; car l'Arnon est la frontière de Moab, entre les Moabites et les Amoréens\FTNT{Jg. 11:18.}.
\VS{14}C'est pourquoi il est dit dans le livre des batailles de Yahweh~: Vaheb en Supha, et les torrents de l'Arnon,
\VS{15}et le cours des torrents qui s'étend du côté d'Ar et touche à la frontière de Moab.
\VS{16}De là ils allèrent à Beer. C'est là le puit où Yahweh dit à Moïse~: Rassemble le peuple, et je leur donnerai de l'eau.
\VS{17}Alors Israël chanta ce cantique~: Monte, puits~! Chantez-lui en vous répondant les uns aux autres.
\VS{18}Puits que des princes ont creusés. Que les grands du peuple ont creusé, avec le législateur, avec leurs bâtons~! Du désert ils vinrent à Matthana~;
\VS{19}de Matthana à Nahaliel~; et de Nahaliel à Bamoth~;
\VS{20}de Bamoth à la vallée qui est dans le territoire de Moab, au sommet de Pisga, et qui regarde vers Jeshimon.
\TextTitle{Israël bat le roi des Amoréens et le roi de Basan}
\VS{21}Puis Israël envoya des messagers à Sihon, roi des Amoréens, pour lui dire~:
\VS{22}Laisse-moi passer par ton pays~; nous ne nous détournerons ni dans les champs, ni dans les vignes, et nous ne boirons l'eau d'aucun des puits~; mais nous marcherons par la route royale, jusqu'à ce que nous ayons passé ton territoire.
\VS{23}Mais Sihon ne permit pas à Israël de passer sur son territoire~; il rassembla tout son peuple et sortit à la rencontre d'Israël, dans le désert~; il vint à Jahats, et combattit Israël\FTNT{De. 2:26-30~; Jg. 11:29-30.}.
\VS{24}Israël le fit passer au fil de l'épée et conquit son pays, depuis l'Arnon jusqu'à Jabbok, et jusqu'à la frontière des fils d'Ammon~; car la frontière des fils d'Ammon était forte\FTNT{De. 2:30~; De. 29:7~; Ps. 135:11-12.}.
\VS{25}Et Israël prit toutes les villes qui étaient là, et habitat dans toutes les villes des Amoréens, à Hesbon, et dans toutes les villes de son ressort.
\VS{26}Or Hesbon était la ville de Sihon, roi des Amoréens, qui avait le premier fait la guerre au roi de Moab, et pris sur lui tout son pays jusqu'à l'Arnon.
\VS{27}C'est pourquoi les poètes disent~: Venez à Hesbon~! Que la ville de Sihon soit rebâtie et fortifiée~!
\VS{28}Car le feu est sorti de Hesbon, et la flamme de la cité de Sihon~; elle a consumé Ar-Moab, les habitants des hauteurs de l'Arnon.
\VS{29}Malheur à toi, Moab~! Peuple de Kemosch, tu es perdu~! Il a livré ses fils qui se sauvaient et ses filles en captivité à Sihon, roi des Amoréens\FTNT{Jé. 48:46.}.
\VS{30}Nous les avons défaits à coups de flèches~: De Hesbon à Dibon tout est détruit~; nous les avons mis en déroute jusqu'à Nophach, jusqu'à Médeba.
\VS{31}Israël s'établit dans le pays des Amoréens.
\VS{32}Puis Moïse envoya des gens pour reconnaître Jaezer, ils prirent les villes de son ressort, et chassèrent les Amoréens qui y étaient.
\VS{33}Ensuite, ils se tournèrent et montèrent par le chemin de Basan. Og, roi de Basan, sortit à leur rencontre, avec tout son peuple pour les combattre à Edréï.
\VS{34}Et Yahweh dit à Moïse~: Ne le crains point, car je le livre entre tes mains, lui et tout son peuple, et son pays~; tu le traiteras comme tu as traité Sihon, roi des Amoréens, qui habitait à Hesbon\FTNT{De. 3:1-2.}.
\VS{35}Ils le battirent donc, lui et ses fils, et tout son peuple, sans en laisser échapper un seul, et ils s'emparèrent de son pays.
\Chap{22}
\TextTitle{Balak cherche à maudire Israël~; Balaam\FTNTT{2 Pi. 2:15~; Jud. 11~; Ap. 2:14} séduit par les honneurs}
\VerseOne{}Puis les enfants d'Israël partirent, et ils campèrent dans les plaines de Moab, au-delà du Jourdain, vis-à-vis de Jéricho.
\VS{2}Balak, fils de Tsippor, vit tout ce qu'Israël avait fait aux Amoréens.
\VS{3}Et Moab eut une grande frayeur du peuple, parce qu'il était en grand nombre, il fut saisi de terreur en face des enfants d'Israël.
\VS{4}Et Moab dit aux anciens de Madian~: Maintenant cette multitude va brouter tout ce qui nous entoure, comme le bœuf broute l'herbe des champs. Balak, fils de Tsippor, était alors roi de Moab.
\VS{5}Il envoya des messagers auprès de Balaam, fils de Beor, à Pethor, située sur le fleuve, dans le pays des fils de son peuple, afin de l'appeler et de lui dire~: Voici, un peuple est sorti d'Egypte, il couvre la surface de la terre, et il habite vis-à-vis de moi.
\VS{6}Viens donc maintenant, je te prie, maudis-moi ce peuple, car il est plus puissant que moi~; peut-être que je serai le plus fort, et que nous le battrons, et que je le chasserai du pays~; car je sais que celui que tu bénis est béni, et que celui que tu maudis est maudit.
\VS{7}Les anciens de Moab s'en allèrent avec les anciens de Madian, ayant dans leurs mains de quoi payer le devin. Ils arrivèrent auprès de Balaam, et lui rapportèrent les paroles de Balak.
\VS{8}Il leur répondit~: Demeurez ici cette nuit, et je vous répondrai d'après ce que Yahweh me dira. Et les chefs des Moabites restèrent chez Balaam.
\VS{9}Et Dieu vint à Balaam et dit~: Qui sont ces hommes que tu as chez toi~?
\VS{10}Et Balaam répondit à Dieu~: Balak, fils de Tsippor, roi de Moab, les a envoyés pour me dire~:
\VS{11}Voici, un peuple qui est sorti d'Egypte, et qui couvre la face de la terre~; viens donc, maudis-le-moi~; peut-être qu'ainsi je pourrai le combattre, et je le chasserai.
\VS{12}Et Dieu dit à Balaam~: Tu n'iras point avec eux, et tu ne maudiras point ce peuple, car il est béni.
\VS{13}Et Balaam se leva le matin, et il dit aux chefs qui avaient été envoyés par Balak~: Retournez dans votre pays, car Yahweh refuse de me laisser venir avec vous.
\VS{14}Ainsi les chefs des Moabites se levèrent et retournèrent auprès de Balak, et dirent~: Balaam a refusé de venir avec nous.
\VS{15}Et Balak envoya encore des chefs en plus grand nombre, et plus considérés que les premiers.
\VS{16}Ils arrivèrent auprès de Balaam, et lui dirent~: Ainsi parle Balak, fils de Tsippor~: Que l'on ne t'empêche donc pas de venir vers moi~;
\VS{17}car je te rendrai beaucoup d'honneur, et je ferai tout ce que tu me diras~; je te prie donc viens, maudis-moi ce peuple.
\VS{18}Et Balaam répondit et dit aux serviteurs de Balak~: Quand Balak me donnerait sa maison pleine d'or et d'argent, je ne pourrais point transgresser l'ordre de Yahweh, mon Dieu~; je ne pourrais faire aucune chose, ni petite ni grande.
\VS{19}Toutefois, je vous prie, demeurez maintenant ici encore cette nuit, et je saurai ce que Yahweh aura de plus à me dire.
\VS{20}Dieu vint, la nuit à Balaam, et lui dit~: Puisque ces hommes sont venus t'appeler, lève-toi, et va avec eux~; mais quoi qu'il en soit, tu feras ce que je te dirai.
\VS{21}Ainsi Balaam se leva le matin, et sella son ânesse, et partit avec les chefs de Moab.
\VS{22}Mais la colère de Dieu s'enflamma parce qu'il était parti~; et l'Ange de Yahweh se plaça sur le chemin pour lui résister. Balaam était monté sur son ânesse, et ses deux serviteurs étaient avec lui.
\VS{23}L'ânesse vit l'Ange de Yahweh qui se tenait sur le chemin, son épée nue dans la main~; elle se détourna du chemin et alla dans les champs. Balaam frappa l'ânesse pour la ramener dans le chemin\FTNT{2 Pi. 2:16~; Jud. 1:11.}.
\VS{24}L'Ange de Yahweh se plaça dans un sentier entre les vignes~; il y avait un mur de chaque côté.
\VS{25}L'ânesse vit l'Ange de Yahweh~; elle se serra contre le mur, et elle serra le pied de Balaam contre le mur. Balaam la frappa de nouveau.
\VS{26}Et l'Ange de Yahweh passa plus loin et s'arrêta dans un lieu étroit où il n'y avait point d'espace pour se détourner à droite ou à gauche.
\VS{27}Et l'ânesse vit l'Ange de Yahweh, et elle s'abattit sous Balaam. Balaam se mit en grande colère, et il frappa l'ânesse avec son bâton.
\VS{28}Alors Yahweh fit parler l'ânesse, et elle dit à Balaam~: Que t'ai-je fait, pour que tu m'aies déjà frappée trois fois~?
\VS{29}Et Balaam répondit à l'ânesse~: C'est parce que tu t'es moquée de moi~; si j'avais une épée dans la main, je te tuerai sur le champ !
\VS{30}Et l'ânesse dit à Balaam: Ne suis-je pas ton ânesse, sur laquelle tu montes depuis que je suis à toi, jusqu'à aujourd'hui~? Ai-je l'habitude de te faire ainsi~? Et il répondit~: Non.
\VS{31}Alors Yahweh ouvrit les yeux de Balaam, et il vit l'Ange de Yahweh qui se tenait sur le chemin, et qui avait dans sa main son épée nue~; et il s'inclina et se prosterna sur son visage.
\VS{32}Et l'Ange de Yahweh lui dit~: Pourquoi as-tu frappé ton ânesse déjà trois fois~? Voici je suis sorti pour m'opposer à toi~; car ta voie est devant moi une voie de perdition.
\VS{33}Mais l'ânesse m'a vu et elle s'est détournée de devant moi déjà trois fois~; autrement, si elle ne s'était détournée de moi, je t'aurais même déjà tué, et je lui aurais laissé la vie.
\VS{34}Alors Balaam dit à l'Ange de Yahweh~: J'ai péché, car je ne savais point que tu t'étais placé au-devant de moi sur le chemin~; et maintenant, si cela te déplaît, je m'en retournerai.
\VS{35}L'Ange de Yahweh dit à Balaam~: Va avec ces hommes~; mais tu ne feras que répéter les paroles que je te dirai. Et Balaam alla avec les chefs envoyés par Balak.
\VS{36}Et quand Balak apprit que Balaam arrivait, il sortit à sa rencontre jusqu'à la ville de Moab, qui est sur la limite de l'Arnon, à l'extrême frontière.
\VS{37}Et Balak dit à Balaam~: N'ai-je pas auparavant envoyé vers toi pour t'appeler~? Pourquoi n'es-tu pas venu vers moi~? Ne puis-je donc pas te traiter avec honneur~?
\VS{38}Et Balaam répondit à Balak~: Je suis venu vers toi~; mais pourrais-je maintenant dire quelque chose~? Je ne dirai que les paroles que Dieu m'aura mis dans la bouche.
\VS{39}Et Balaam alla avec Balak, et ils arrivèrent dans la cité de Kirjath-Hutsoth.
\VS{40}Et Balak sacrifia des bœufs et des brebis, et il en envoya à Balaam et aux chefs qui étaient venus avec lui.
\VS{41}Quand le matin fut venu, il prit Balaam et le fit monter à Bamoth-Baal, et de là il vit une partie du peuple.
\Chap{23}
\TextTitle{Balaam ne maudit pas mais bénit Israël des hauts lieux de Baal}
\VerseOne{}Et Balaam dit à Balak~: Bâtis-moi ici sept autels, et prépare-moi ici sept veaux et sept béliers.
\VS{2}Et Balak fit ce que Balaam avait dit~; et Balak offrit avec Balaam un veau et un bélier sur chaque autel.
\VS{3}Balaam dit à Balak~: Tiens-toi près de ton holocauste, et je m'éloignerai~; peut-être que Yahweh viendra à ma rencontre, et je te rapporterai tout ce qu'il me révélera. Ainsi il se retira à l'écart.
\VS{4}Et Dieu vint au-devant de Balaam, et Balaam lui dit~: J'ai dressé sept autels, et j'ai sacrifié un veau et un bélier sur chaque autel.
\VS{5}Et Yahweh mit des paroles dans la bouche de Balaam et lui dit~: Retourne vers Balak, et tu parleras ainsi.
\VS{6}Il s'en retourna donc vers lui~; et voici, Balak se tenait près de son holocauste, tant lui que tous les chefs de Moab.
\VS{7}Alors Balaam prononça son discours sentencieux et dit~: Balak, roi de Moab, m'a fait descendre d'Aram\FTNT{De l'hébreu «~Aram~» traduit par «~Aram~» ou «~Syrie~» (1 R. 11:25).}, des montagnes d'orient, en me disant~: Viens, maudis-moi Jacob~! Viens, dis-je, déteste Israël~!
\VS{8}Mais comment le maudirai-je~? Dieu ne l'a point maudit. Et comment le détesterai-je~? Yahweh ne l'a point détesté.
\VS{9}Car je le regarderai du sommet des rochers, et je le contemplerai du haut des collines~: Voici, ce peuple habitera à part, et il ne sera pas compté parmi les nations\FTNT{De. 33:28.}.
\VS{10}Qui comptera la poussière de Jacob, et dira le nombre du quart d'Israël~? Que je meure de la mort des justes, et que ma fin soit semblable à la leur~!
\VS{11}Alors Balak dit à Balaam~: Que m'as-tu fait~? Je t'ai pris pour maudire mes ennemis, et voici, tu les a bénis, tu les a bénis\FTNT{Le verbe bénir vient de l'hébreu «~Barak~», il est utilisé deux fois de suite dans ce passage. Voir commentaire en Ge. 2:16-17.}.~!
\VS{12}Et il répondit, et dit~: Ne prendrais-je pas garde de dire les paroles que Yahweh aura mis dans ma bouche~?
\TextTitle{Balaam bénit Israël au sommet de Pisga}
\VS{13}Alors Balak lui dit~: Viens, je te prie, avec moi dans un autre lieu, d'où tu pourras le voir, car tu en voyais seulement une extrémité, et tu ne le voyais pas tout entier~; maudis-le moi de là.
\VS{14}Puis l'ayant conduit au territoire de Tsophim, sur le sommet de Pisga~; il bâtit sept autels, et offrit un taureau et un bélier sur chaque autel.
\VS{15}Alors Balaam dit à Balak~: Tiens-toi ici près de ton holocauste, et je m'en irai à la rencontre de Dieu, comme j'ai déjà fait.
\VS{16}Yahweh donc vint au-devant de Balaam, il mit des paroles dans sa bouche et lui dit~: Retourne vers Balak, et tu parleras ainsi.
\VS{17}Il retourna vers Barak~; et voici, il se tenait près de son holocauste, et les chefs de Moab avec lui. Et Balak lui dit~: Qu'est-ce que Yahweh a dit~?
\VS{18}Alors il prononça son discours sentencieux et dit~: Lève-toi, Balak, écoute~! Fils de Tsippor, prête-moi l'oreille~!
\VS{19}Dieu n'est point un homme pour mentir ni fils d'un homme pour se repentir. Ce qu'il a dit, ne le fera-t-il pas~? Ce qu'il a déclaré, ne l'exécutera-t-il pas\FTNT{Ja. 1:17.}~?
\VS{20}Voici, j'ai reçu la parole pour bénir~: Puisqu'il a béni, je ne le révoquerai point.
\VS{21}Il n'a point aperçu d'iniquité en Jacob, il ne voit point de perversité en Israël~; Yahweh, son Dieu, est avec lui, et il y a en lui un chant de triomphe royal\FTNT{Jé. 50:20~; Ro. 4:7.}.
\VS{22}Dieu les a tirés d'Egypte, il est pour eux comme la vigueur du buffle.
\VS{23} Car il n'y a pas d'enchantement contre Jacob, ni la divination contre Israël. Au temps marqué, il sera dit à Jacob et à Israël~: Qu'est-ce que Dieu a fait~?
\VS{24}Voici, ce peuple se lèvera comme un vieux lion, et se dressera comme un lion qui est dans sa force~; il ne se couchera pas jusqu'à ce qu'il ait dévoré la proie, et bu le sang des blessés à mort.
\VS{25}Balak dit à Balaam~: Et bien~! Ne le maudis pas, mais du moins ne le bénis pas.
\VS{26}Et Balaam répondit à Balak~: Ne t'ai-je pas dit que tout ce que Yahweh dira, je le ferai~? 
\TextTitle{Balaam bénit Israël de Peor}
\VS{27}Balak dit encore à Balaam~: Viens maintenant, je te conduirai dans un autre lieu~; peut-être que Dieu trouvera bon que tu me le maudisses de là.
\VS{28}Balak conduisit donc Balaam sur le sommet de Peor, qui regarde du côté de Jeshimon.
\VS{29}Et Balaam lui dit~: Bâtis-moi ici sept autels, et apprête-moi ici sept veaux et sept béliers.
\VS{30}Et Balak fit donc comme Balaam lui avait dit~; puis il offrit un taureau et un bélier sur chaque autel.
\Chap{24}
\VerseOne{}Or Balaam, voyant que Yahweh voulait bénir Israël, n'alla plus comme les autres fois chercher des enchantements~; mais il tourna son visage du côté du désert.
\VS{2}Et Balaam leva les yeux, il vit Israël qui se tenait rangé selon ses tribus. Alors l'Esprit de Dieu fut sur lui.
\VS{3}Et il prononça à haute voix son discours sentencieux et dit~: Balaam, fils de Beor, dit, et l'homme qui a l'œil ouvert dit:
\VS{4}Celui qui entend les paroles de Dieu, qui voit la vision du Tout-Puissant, qui tombe à terre, et qui a les yeux ouverts dit~:
\VS{5}Que tes tentes sont belles, ô Jacob~! Et tes tabernacles, ô Israël~!
\VS{6}Ils sont étendus comme des torrents, comme des jardins près d'un fleuve, comme des arbres d'aloès que Yahweh a plantés, comme des cèdres auprès des eaux.
\VS{7}L'eau coule de ses seaux, et sa semence est parmi d'abondantes eaux. Et son roi s'élève au-dessus d'Agag, et son royaume sera haut élevé.
\VS{8}Dieu, qui l'a tiré d'Egypte, est pour lui comme la vigueur du buffle~; il consumera les nations qui sont ses ennemies~; il brisera leurs os, et les percera de ses flèches.
\VS{9}Il s'est courbé, il s'est couché comme un lion qui est dans sa force, et comme un vieux lion~; qui le réveillera~? Quiconque te bénit, sera béni, et quiconque te maudit, sera maudit.
\VS{10}Alors Balak se mit très en colère contre Balaam, il frappa des mains et Balak parla ainsi à Balaam~: Je t'ai appelé pour maudire mes ennemis, et voici, tu les as bénis, tu les a bénis\FTNT{voir le commentaire en Ge. 2:16-17.} trois fois déjà.
\VS{11}Et maintenant, fuis dans ton pays~! J'avais dit que je t'honorerais, je t'honorerais\FTNT{Le mot hébreu est utilisé deux fois dans ce passage. Voir le commentaire en Ge. 2:17.}, mais Yahweh t'empêche d'être honoré.
\VS{12}Et Balaam répondit à Balak~: N'ai-je pas dit à tes messagers que tu m'as envoyés~:
\VS{13}Quand Balak me donnerait sa maison pleine d'argent et d'or, je ne pourrais transgresser l'ordre de Yahweh pour faire de moi-même du bien ou du mal~; mais ce que Yahweh dira, je le dirai.
\VS{14}Maintenant donc je m'en vais vers mon peuple. Viens, je te donnerai un conseil, et je te dirai ce que ce peuple fera à ton peuple, dans les derniers jours.
\TextTitle{Prophétie sur le Roi qui sort de Jacob, le Messie}
\VS{15}Alors il prononça son discours sentencieux et dit~: Balaam, fils de Beor dit, et l'homme qui a l'œil ouvert dit~:
\VS{16}Celui qui entend les paroles de Dieu, qui connaît la science du Très-Haut, qui voit la vision du Tout-Puissant, qui tombe à terre, et qui a les yeux ouverts.
\VS{17}Je le vois, mais non pas maintenant~; je le regarde, mais non pas de près~; une Etoile est sortie de Jacob\FTNT{L'Etoile en question est Jésus-Christ, qui se révéla à Jean comme l'Etoile brillante du matin (Ap. 22:16).}, et un Sceptre s'est élevé d'Israël. Il transpercera les cotés de Moab, il détruira tous les enfants de Seth.
\VS{18}Edom sera sa possession, Séir sera possédé par ses ennemis, et Israël se portera vaillamment.
\VS{19}Et il y en aura un de Jacob qui dominera, il fera périr le reste de la ville.
\VS{20}Il vit aussi Amalek, il prononça son discours sentencieux et dit~: Amalek est le premier des nations, mais à la fin il sera détruit.
\VS{21}Il vit aussi les Kéniens\FTNT{Il y a plusieurs sens à ce mot~:
\\Caïn = «~possession~», «~artisan, forgeron~», fils d'Adam.
\\Kéniens = «~forgerons~» tribu du beau-père de Moïse qui vivait dans la région du sud de la Palestine.}. Il prononça à haute voix son discours sentencieux et dit~: Ta demeure est dans un lieu solide, et tu as mis ton nid dans le rocher~;
\VS{22}toutefois, le Kénien sera consumé, jusqu'à ce que l'Assyrien t'emmène en captivité.
\VS{23}Il continua à prononcer à haute voix son discours sentencieux, et il dit~: Malheur à celui qui vivra quand Dieu fera ces choses.
\VS{24}Et des navires viendront de Kittim, et ils humilieront l'Assyrien et l'Hébreu~; et lui aussi sera détruit.
\VS{25}Puis Balaam se leva, et s'en alla pour retourner chez lui. Balak aussi s'en alla son chemin.
\Chap{25}
\TextTitle{Prostitution d'Israël à Baal-Peor\FTNTT{No. 31:16~; Ja. 4:4~; Ap. 2:14.}}
\VerseOne{}Alors Israël demeurait à Sittim~; et le peuple commença à commettre la fornication avec les filles de Moab.
\VS{2}Car elles convièrent le peuple aux sacrifices de leurs dieux~; et le peuple mangea et se prosterna devant leurs dieux.
\VS{3}Et Israël s'accoupla à Baal-Peor, c'est pourquoi la colère de Yahweh s'enflamma contre Israël\FTNT{Ps. 106:28~; Os. 9:10.}.
\VS{4}Et Yahweh dit à Moïse~: Prends tous les chefs du peuple, et fais-les pendre devant Yahweh en face du soleil, afin que la colère de Yahweh se détourne d'Israël\FTNT{De. 4:3~; Jos. 22:17.}.
\VS{5}Moïse donc dit aux juges d'Israël~: Que chacun de vous fasse mourir les hommes qui sont à sa charge, et qui se sont joints à Baal-Peor.
\VS{6}Et voici, un homme des enfants d'Israël vint, et amena à ses frères une Madianite, devant Moïse et devant toute l'assemblée des fils d'Israël, tandis qu'ils pleuraient à l'entrée de la tente d'assignation.
\VS{7}Ce que Phinées, fils d'Eléazar, fils d'Aaron le prêtre, ayant vu, se leva du milieu de l'assemblée et prit une lance dans sa main.
\VS{8}Et il entra dans la tente de l'homme Israélite et les transperça tous deux, l'homme Israélite puis la femme, par le ventre. Et la plaie s'arrêta parmi les enfants d'Israël\FTNT{Ps. 106:30.}.
\VS{9}Or il y en eut vingt-quatre mille qui moururent de cette plaie.
\VS{10}Et Yahweh parla à Moïse, en disant~:
\VS{11}Phinées, fils d'Eléazar, fils d'Aaron, le prêtre, a détourné ma colère de dessus les enfants d'Israël, parce qu'il a été animé de mon zèle au milieu d'eux~; et je n'ai point, dans mon ardeur, consumé les fils d'Israël.
\VS{12}C'est pourquoi, dis-lui~: Voici, je lui donne mon alliance de paix.
\VS{13}Et l'alliance de prêtrise perpétuelle sera tant pour lui que pour sa postérité après lui, parce qu'il a été animé de zèle pour son Dieu, et qu'il a fait propitiation pour les enfants d'Israël.
\VS{14}Et le nom de l'homme Israélite tué, lequel fut tué avec la Madianite, était Zimri, fils de Salu, chef d'une maison de père des Siméonites.
\VS{15}Et le nom de la femme Madianite qui fut tuée était Cozbi, fille de Tsur, chef du peuple, et d'une maison de père en Madian.
\VS{16}Yahweh parla à Moïse, en disant~:
\VS{17}Mettez en détresse les Madianites, tuez-les~;
\VS{18}car ils vous ont serrés les premiers par leurs ruses, par lequelles ils vous surpris dans l'affaire de Peor, et dans l'affaire de Cozbi, fille d'un chef d'entre les Madianites, leur sœur, qui a été tuée le jour de la plaie, causée par l'affaire de Peor.
\Chap{26}
\TextTitle{Nouveau dénombrement des hommes de guerre}
\VerseOne{} Or il arriva qu'après cette plaie-là, que Yahweh parla à Moïse, et à Eléazar, fils d'Aaron, le prêtre, en disant~:
\VS{2}Faites le dénombrement de toute l'assemblée des enfants d'Israël, depuis l'âge de vingt ans et au-dessus, selon les maisons de leurs pères, à savoir de tous ceux d'Israël qui peuvent aller à la guerre.
\VS{3}Moïse donc et Eléazar, le prêtre, leur parlèrent donc dans les plaines de Moab, près du Jourdain de Jéricho, en disant~:
\VS{4}Qu'on fasse le dénombrement depuis l'âge de vingt ans et au-dessus, comme Yahweh l'avait ordonné à Moïse et aux enfants d'Israël, quand ils furent sortis du pays d'Egypte.
\VS{5}Ruben, premier-né d'Israël. Fils de Ruben~: Hénoc, de qui descend la famille des Hénokites~; Pallu, de qui descend la famille des Palluites~;
\VS{6}Hetsron, de qui descend la famille des Hetsronites~; Carmi, de qui descend la famille des Carmites.
\VS{7}Ce sont là les familles des Rubénites~: Ceux qui furent dénombrés étaient quarante-trois mille sept cent trente.
\VS{8}Et les fils de Pallu~: Eliab.
\VS{9}Fils d'Eliab~: Nemuel, Dathan et Abiram. Ce Dathan et cet Abiram, qui étaient de ceux qu'on appelait pour tenir l'assemblée, et qui se révoltèrent contre Moïse et contre Aaron dans l'assemblée de Koré, lors de leur révolte contre Yahweh.
\VS{10}Et lorsque la terre ouvrit sa bouche et les engloutit, ainsi que Koré, ceux qui s'étaient assemblés avec lui moururent. Et le feu dévora les deux cent cinquante hommes qui servirent d'avertissement.
\VS{11}Mais les fils de Koré ne moururent pas.
\VS{12}Les fils de Siméon selon leurs familles~: De Nemuel descend la famille des Némuélites~; de Jamin, la famille des Jaminites~; de Jakin, la famille des Jakinites~;
\VS{13}de Zérach, la famille des Zérachites~; de Saül, la famille des Saülites.
\VS{14}Ce sont là les familles des Siméonites, qui furent vingt-deux mille deux cents.
\VS{15}Fils de Gad selon leurs familles. De Tsephon, descend la famille des Tsephonites~; de Haggi, la famille des Haggites~; de Schuni, la famille des Schunites~;
\VS{16}d'Ozni, la famille des Oznites~; d'Eri, la famille des Erites~;
\VS{17}d'Arod, la famille des Arodites~; d'Areéli, la famille des Areélites.
\VS{18}Ce sont là les familles des fils de Gad, d'après leur dénombrement~: Quarante mille cinq cents.
\VS{19}Fils de Juda, Er, et Onan~; mais Er et Onan moururent au pays de Canaan\FTNT{Ge. 38:7-10~; Ge. 46:12.}.
\VS{20}Voici les fils de Juda selon leurs familles~: De Schéla descend la famille des Schélanites~; de Pérets, la famille des Péretsites~; de Zérach, la famille des Zérachites.
\VS{21}Les fils de Pérets furent~: Hetsron, de qui descend la famille des Hetsronites~; Hamul, de qui descend la famille des Hamulites.
\VS{22}Ce sont là les familles de Juda, selon leur dénombrement: Soixante-seize mille cinq cents.
\VS{23}Fils d'Issacar, selon leurs familles~: De Thola descend la famille des Tholaïtes~; de Puva, la famille des Puvites~;
\VS{24}de Jaschub, la famille des Jaschubites~; de Schimron, la famille des Schimronites.
\VS{25}Ce sont là les familles d'Issacar, d'après leur dénombrement~: Soixante-quatre mille trois cents.
\VS{26}Fils de Zabulon, selon leurs familles~: De Séred, descend la famille des Sardites~; d'Elon, la famille des Elonites~; de Jahleel, la famille des Jahleélites.
\VS{27}Ce sont là les familles des Zabulonites, d'après leur dénombrement~: Soixante mille cinq cents.
\VS{28}Fils de Joseph, selon leurs familles~: Manassé et Ephraïm.
\VS{29}Fils de Manassé. De Makir descend la famille des Makirites. Makir engendra Galaad. De Galaad descend la famille des Galaadites.
\VS{30}Voici les fils de Galaad~: Jézer, de qui descend la famille des Jézerites~; Hélek, la famille des Hélekites.
\VS{31}Asriel, la famille des Asriélites~; Sichem, la famille des Sichémites~;
\VS{32}Schemida, la famille des Schemidaïtes~; Hépher, la famille des Héphrites.
\VS{33}Tselophchad, fils de Hépher, n'eut point de fils, mais des filles. Voici les noms des filles de Tselophchad~: Machla, Noa, Hogla, Milca, et Thirtsa.
\VS{34}Ce sont là les familles de Manassé, d'après leur dénombrement~: Cinquante-deux mille sept cents.
\VS{35}Voici les fils d'Ephraïm, selon leurs familles~: De Schutélach descend la famille des Schutalchites~; de Béker, la famille des Bakrites~; de Thachan, la famille des Thachanites.
\VS{36}Voici les fils de Schutélach~: D'Eran est descendue la famille des Eranites.
\VS{37}Ce sont là les familles des fils d'Ephraïm, d'après leur dénombrement~: Trente-deux mille cinq cents. Ce sont là les fils de Joseph, selon leurs familles.
\VS{38}Fils de Benjamin, selon leurs familles~: De Béla descend la famille des Balites~; d'Aschbel, la famille des Aschbélites~; d'Achiram, la famille des Achiramites~;
\VS{39}De Schupham, la famille des Schuphamites~; de Hupham, la famille des Huphamites.
\VS{40}Les fils de Béla furent Ard et Naaman. D'Ard descend la famille des Ardites~; et de Naaman la famille des Naamanites.
\VS{41}Ce sont là les fils de Benjamin, d'après leurs familles~; et leur dénombrement~: Quarante-cinq mille six cents.
\VS{42}Voici les fils de Dan, selon leurs familles~: De Schucham descend la famille des Schuchamites. Ce sont là les familles de Dan, selon leurs familles.
\VS{43}Toutes les familles des Schuchamites, selon leur dénombrement~: Soixante-quatre mille quatre cents.
\VS{44}Fils d'Aser, selon leurs familles~: De Jimna descend la famille des Jimnites~; de Jischvi, la famille des Jischvites~; de Beria la famille des Beriites.
\VS{45}Des fils de Beria descendent~: De Héber, la famille des Hébrites~; de Malkiel, la famille des Malkiélites.
\VS{46}Et le nom de la fille d'Aser était Sérach.
\VS{47}Ce sont là les familles des fils d'Aser, d'après leur dénombrement~: Cinquante-trois mille quatre cents.
\VS{48}Fils de Nephthali, selon leurs familles~: De Jahtseel descend la famille des Jahtseélites~; de Guni, la famille des Gunites~;
\VS{49}de Jetser la famille des Jitsrites~; de Schillem, la famille des Schillémites.
\VS{50}Ce sont là les familles de Nephthali, selon leurs familles, et leur dénombrement~: Quarante-cinq mille quatre cents.
\VS{51}Voici les dénombrés des fils d'Israël, qui furent six cent un mille sept cent trente.
\VS{52}Yahweh parla à Moïse, en disant~:
\VS{53}Le pays sera partagé entre ceux-ci en héritage, selon le nombre des noms.
\VS{54}A ceux qui sont en plus grand nombre, tu donneras plus d'héritage, et à ceux qui sont en plus petit nombre tu donneras moins d'héritage~; on donnera à chacun son héritage selon le nombre de ses dénombrés.
\VS{55}Toutefois, que le pays soit partagé par le sort~; et qu'ils prennent leur héritage selon les noms des tribus de leurs pères\FTNT{Jos. 11:23~; Jos. 14:2~; Jos. 18:6-8.}.
\VS{56}L'héritage de chacun sera selon que le sort le montrera, et on aura égard au plus grand et au plus petit nombre.
\VS{57}Et ce sont ici les dénombrés de Lévi selon leurs familles~; de Guerschon, la famille des Guerschonites~; de Kehath, la famille des Kehathites~; de Merari, la famille des Merarites.
\VS{58}Ce sont ici les familles de Lévi~; la famille des Libnites, la famille des Hébronites, la famille des Machlites, la famille des Muschites, la famille des Korites. Kehath engendra Amram.
\VS{59}Et le nom de la femme d'Amram était Jokébed, fille de Lévi, qui naquit à Lévi en Egypte~; et elle enfanta à Amram: Aaron, Moïse, et Marie, leur sœur.
\VS{60}Et il naquit à Aaron~: Nadab et Abihu, Eléazar et Ithamar.
\VS{61}Nadab et Abihu moururent lorsqu'ils apportèrent du feu étranger devant Yahweh\FTNT{Lé. 10:1-2~; 1 Ch. 24:2.}.
\VS{62}Et tous les dénombrés des Lévites furent vingt-trois mille, tous mâles, depuis l'âge d'un mois, et au dessus, qui ne furent point dénombrés avec les autres enfants d'Israël, car on ne leur donna point d'héritage entre les enfants d'Israël.
\VS{63}Ce sont là ceux qui furent dénombrés par Moïse et Eléazar, le prêtre, qui firent le dénombrement des fils d'Israël dans les plaines de Moab, près du Jourdain de Jéricho.
\VS{64}Entre lesquels il ne s'en trouva aucun de ceux qui avaient été dénombrés par Moïse et Aaron le prêtre, quand ils firent le dénombrement des enfants d'Israël au désert de Sinaï.
\VS{65}Car Yahweh avait dit d'eux~: ils mourront certainement dans le désert, et qu'ainsi il n'en restera pas un, excepté Caleb, fils de Jephunné, et Josué, fils de Nun\FTNT{1 Co. 10:5.}.
\Chap{27}
\TextTitle{Loi sur les héritages\FTNTT{No. 36.}}
\VerseOne{}Or les filles de Tselophchad, fils de Hépher, fils de Galaad, fils de Makir, fils de Manassé, d'entre les familles de Manassé, fils de Joseph, s'approchèrent~; et ce sont ici les noms de ses filles~: Machla, Noa, Hogla, Milca, et Thirtsa.
\VS{2}Elles se présentèrent devant Moïse, devant Eléazar, le prêtre, et devant les princes et toute l'assemblée, à l'entrée de la tente d'assignation. Elles dirent~:
\VS{3}Notre père est mort dans le désert~; il n'était toutefois pas dans la troupe de ceux qui s'assemblèrent contre Yahweh, dans l'assemblée de Koré, mais il est mort dans son péché, et il n'avait point de fils.
\VS{4}Pourquoi le nom de notre père serait-il retranché de sa famille, parce qu'il n'a point eu de fils~? Donne-nous une possession parmi les frères de notre père.
\VS{5}Moïse rapporta leur cause devant Yahweh.
\VS{6}Et Yahweh parla à Moïse, en disant~:
\VS{7}Les filles de Tselophchad ont parlé droitement. Tu ne manqueras pas de leur donner un héritage à posséder parmi les frères de leur père, et tu leur feras passer l'héritage de leur père.
\VS{8}Tu parleras aussi aux enfants d'Israël, et tu leur diras~: Lorsqu'un homme mourra sans avoir de fils, vous ferez passer son héritage à sa fille.
\VS{9}S'il n'a pas de fille, vous donnerez son héritage à ses frères.
\VS{10}S'il n'a pas de frères, vous donnerez son héritage aux frères de son père.
\VS{11}Et si son père n'a pas de frère, vous donnerez son héritage à son parent le plus proche de sa famille, et il le possédera. Et ce sera pour les enfants d'Israël une ordonnance de droit, comme Yahweh l'a ordonné à Moïse.
\TextTitle{Moïse voit de loin le pays promis aux fils d'Israël}
\VS{12}Yahweh dit aussi à Moïse~: Monte sur cette montagne d'Abarim, et regarde le pays que je donne aux enfants d'Israël\FTNT{De. 32:48-49.}.
\VS{13}Tu le regarderas donc~; et puis tu seras toi aussi recueilli auprès de ton peuple, comme Aaron ton frère y a été recueilli~;
\VS{14}parce que vous avez été rebelles à mon ordre dans le désert de Tsin, lors de la contestation de l'assemblée, vous ne m'avez point sanctifié au sujet des eaux devant eux~; ce sont les eaux de Meriba, à Kadès, dans le désert de Tsin.
\TextTitle{Yahweh désigne Josué comme successeur de Moïse}
\VS{15}Moïse parla à Yahweh, en disant~:
\VS{16}Que Yahweh, le Dieu des esprits de toute chair, établisse sur l'assemblée un homme\FTNT{Hé. 12:9.},
\VS{17}qui sorte devant eux et qui entre devant eux, et qui les fasse sortir et qui les fasse entrer, afin que l'assemblée de Yahweh ne soit pas comme des brebis qui n'ont point de berger\FTNT{1 R. 22:17~; Mt. 9:36~; Mc. 6:34.}.
\VS{18}Alors Yahweh dit à Moïse~: Prends Josué, fils de Nun, un homme en qui est l'Esprit, et tu poseras ta main sur lui\FTNT{De. 34:9.}.
\VS{19}Tu le présenteras devant Eléazar, le prêtre, et devant toute l'assemblée~; et tu lui donneras des instructions sous leurs yeux.
\VS{20}Et tu lui feras part de ton autorité, afin que toute l'assemblée des enfants d'Israël l'écoute.
\VS{21}Et il se présentera devant Eléazar, le prêtre, qui consultera pour lui les jugements de l'urim\FTNT{Lé. 8:8.} devant Yahweh~; et à sa parole ils sortiront, et à sa parole ils entreront, lui, les enfants d'Israël, avec lui, et toute l'assemblée.
\VS{22}Moïse donc fit comme Yahweh lui avait ordonné. Il prit Josué et le présenta devant Eléazar, le prêtre, et devant toute l'assemblée.
\VS{23}Puis il posa ses mains sur lui, et lui donna des instructions, comme Yahweh l'avait dit par Moïse.
\Chap{28}
\TextTitle{Consignes relatives au temps des sacrifices}
\VerseOne{}Yahweh parla à Moïse, en disant~:
\VS{2}Donne cet ordre aux enfants d'Israël, et dis-leur~: Vous aurez soin de m'offrir en leur temps, mon offrande, ma nourriture, pour mes sacrifices consumés par le feu, qui me sont d'une bonne odeur\FTNT{Lé. 3:11~; Lé. 21:6.}.
\VS{3}Tu leur diras~: Voici le sacrifice consumé par le feu que vous offrirez à Yahweh~: Deux agneaux d'un an sans défaut, chaque jour, en holocauste perpétuel\FTNT{Ex. 29:38.}.
\VS{4}Tu sacrifieras l'un des agneaux le matin, et l'autre agneau entre les deux soirs,
\VS{5}et la dixième partie d'épha de fine farine pour le gâteau pétrie avec le quart d'un hin d'huile vierge\FTNT{Lé. 2:1~; Ex. 29:40~; Ex. 16:36.}.
\VS{6}C'est l'holocauste perpétuel, qui a été offert à la montagne de Sinaï, c'est un sacrifice consumé par le feu, d'une bonne odeur à Yahweh.
\VS{7}Et sa libation sera d'un quart de hin pour chaque agneau~: Et tu verseras dans le lieu saint la libation de boisson forte à Yahweh.
\VS{8}Et tu sacrifieras l'autre agneau entre les deux soirs, tu feras le même gâteau qu'au matin, et la même libation, en sacrifice consumé par le feu d'une bonne odeur à Yahweh.
\VS{9}Mais le jour du sabbat vous offrirez deux agneaux d'un an sans défaut, et deux dixièmes de fine farine pétrie à l'huile pour le gâteau, avec sa libation.
\VS{10}C'est l'holocauste du sabbat, pour chaque sabbat, outre l'holocauste perpétuel avec sa libation.
\VS{11}Et au commencement de vos mois, vous offrirez en holocauste à Yahweh deux jeunes taureaux, un bélier, et sept agneaux d'un an sans défaut~;
\VS{12}et trois dixièmes de fine farine pétrie à l'huile, pour le gâteau de chaque taureau, et deux dixièmes de fine farine pétrie à l'huile pour le gâteau du bélier~;
\VS{13}et un dixième de fine farine pétrie à l'huile, comme gâteau pour chaque agneau, en holocauste, d'une bonne odeur, et en sacrifice consumé par le feu à Yahweh.
\VS{14}Et leurs libations seront d'un demi-hin de vin pour chaque veau, d'un tiers de hin pour un bélier, et d'un quart de hin pour chaque agneau, c'est l'holocauste du commencement de chaque mois, selon tous les mois de l'année.
\VS{15}On sacrifiera aussi à Yahweh un jeune bouc en sacrifice d'expiation, outre l'holocauste perpétuel, et sa libation.
\VS{16}Au quatorzième jour du premier mois, ce sera la Pâque à Yahweh.
\VS{17}Et au quinzième jour du même mois sera un jour de fête. On mangera pendant sept jours des pains sans levain\FTNT{Ex. 12~; Lé. 23:5-6.}.
\VS{18}Au premier jour, il y aura une sainte convocation~: Vous ne ferez aucune œuvre servile.
\VS{19}Et vous offrirez un sacrifice consumé par le feu en holocauste à Yahweh~: Deux jeunes taureaux, un bélier, et sept agneaux d'un an, sans défaut.
\VS{20}Leur gâteau sera de fine farine pétrie à l'huile, vous en offrirez trois dixièmes pour chaque jeune taureau, et deux dixièmes pour un bélier~;
\VS{21}tu en offriras aussi un dixième pour chacun des sept agneaux,
\VS{22}et un bouc en sacrifice pour l'expiation, afin de faire propitiation pour vous.
\VS{23}Vous offrirez ces choses là, outre l'holocauste du matin, qui est l'holocauste perpétuel.
\VS{24}Vous offrirez ces choses-là chaque jour, pendant sept jours, comme l'aliment d'un sacrifice consumé par le feu, d'une bonne odeur à Yahweh. On offrira cela outre l'holocauste perpétuel, et sa libation.
\VS{25}Et au septième jour, vous aurez une sainte convocation~: Vous ne ferez aucune œuvre servile.
\VS{26}Et au jour des prémices, quand vous offrirez à Yahweh une offrande nouvelle de gâteau à votre fête des semaines, vous aurez une sainte convocation~: Vous ne ferez aucune œuvre servile.
\VS{27}Et vous offrirez en holocauste d'une bonne odeur à Yahweh, deux jeunes taureaux, un bélier, et sept agneaux d'un an.
\VS{28}Et leur gâteau sera de fine farine pétrie à l'huile, de trois dixièmes pour chaque jeune taureau, et de deux dixièmes pour le bélier,
\VS{29}et d'un dixième pour chacun des sept agneaux~;
\VS{30}et un jeune bouc, afin de faire propitiation pour vous.
\VS{31}Vous les offrirez, outre l'holocauste perpétuel et son offrande, lesquels seront sans défaut, avec leurs libations.
\Chap{29}
\TextTitle{Consignes relatives au temps des sacrifices - suite}
\VerseOne{}Et le premier jour du septième mois, vous aurez une sainte convocation~: Vous ne ferez aucune œuvre servile. Ce jour sera publié parmi vous au son des trompettes\FTNT{Lé. 23:24-25.}.
\VS{2}Et vous offrirez en holocauste de bonne odeur à Yahweh, un jeune taureau, un bélier, et sept agneaux d'un an, sans défaut.
\VS{3}Et leur gâteau sera de fine farine pétrie à l'huile, de trois dixièmes pour le jeune taureau, de deux dixièmes pour le bélier,
\VS{4}et un dixième pour chacun des sept agneaux.
\VS{5}Et un jeune bouc en sacrifice pour l'expiation, afin de faire propitiation pour vous,
\VS{6}outre l'holocauste du commencement du mois et son gâteau, et l'holocauste perpétuel et son gâteau, et leurs libations selon leur ordonnance. Ce sont des sacrifices consumés par le feu en bonne odeur à Yahweh.
\VS{7}Et au dixième jour de ce septième mois, vous aurez une sainte convocation, et vous affligerez vos âmes~: Vous ne ferez aucune œuvre\FTNT{Lé. 16:29-31~; Lé. 23:27.}.
\VS{8}Et vous offrirez en holocauste, de bonne odeur à Yahweh, un jeune taureau, un bélier, et sept agneaux d'un an, qui seront sans défaut.
\VS{9}Et leur gâteau sera de fine farine pétrie à l'huile, de trois dixièmes pour le taureau, et de deux dixièmes pour le bélier,
\VS{10}et d'un dixième pour chacun des sept agneaux.
\VS{11}Un jeune bouc aussi en sacrifice d'expiation, outre le sacrifice des expiations, l'holocauste perpétuel et son gâteau, avec leurs libations.
\VS{12}Et au quinzième jour du septième mois, vous aurez une sainte convocation~: Vous ne ferez aucune œuvre servile. Vous célébrerez une fête à Yahweh, pendant sept jours\FTNT{Lé. 23:34-43.}.
\VS{13}Et vous offrirez en holocauste un sacrifice consumé par le feu, d'une agréable odeur à Yahweh, treize jeunes taureaux, deux béliers, et quatorze agneaux d'un an, sans défaut.
\VS{14}Et leur gâteau sera de fine farine pétrie à l'huile, de trois dixièmes pour chacun des treize jeunes taureaux, de deux dixièmes pour chacun des deux béliers,
\VS{15}et d'un dixième pour chacun des quatorze agneaux.
\VS{16}Et un jeune bouc en sacrifice d'expiation, outre l'holocauste perpétuel, son gâteau, et sa libation.
\VS{17}Et au second jour, vous offrirez douze jeunes taureaux, deux béliers, et quatorze agneaux d'un an, sans défaut,
\VS{18}avec les gâteaux et les libations pour les jeunes taureaux, pour les béliers, et pour les agneaux, selon leur nombre, d'après les ordonnances.
\VS{19}Vous offrirez un jeune bouc en sacrifice d'expiation, outre l'holocauste perpétuel, et son offrande, avec leurs libations.
\VS{20}Et au troisième jour, vous offrirez onze taureaux, deux béliers, et quatorze agneaux d'un an, sans défaut~;
\VS{21}et les gâteaux et les libations pour les jeunes taureaux, les béliers et les agneaux, selon leur nombre, selon leur ordonnance.
\VS{22}Et un bouc en sacrifice d'expiation, outre l'holocauste continuel, son gâteau et sa libation.
\VS{23}Et au quatrième jour, vous offrirez dix jeunes taureaux, deux béliers, et quatorze agneaux d'un an, sans défaut,
\VS{24}les gâteaux et les libations pour les taureaux, les béliers, et les agneaux, selon leur nombre et leur ordonnance.
\VS{25}Et un jeune bouc en sacrifice d'expiation, outre l'holocauste perpétuel, son offrande, et sa libation.
\VS{26}Et au cinquième jour, vous offrirez neuf jeunes taureaux, deux béliers, et quatorze agneaux d'un an, sans défaut,
\VS{27}avec les gâteaux et les libations pour les taureaux, les béliers, et les agneaux, selon leur nombre et leur ordonnance.
\VS{28}Et un bouc en sacrifice d'expiation, outre l'holocauste continuel, son gâteau, et sa libation.
\VS{29}Et le sixième jour, vous offrirez huit jeunes taureaux, deux béliers et quatorze agneaux d'un an, sans défaut,
\VS{30}et les gâteaux, les libations pour les taureaux, les béliers, et les agneaux selon leur nombre leur ordonnance.
\VS{31}Et un bouc en sacrifice d'expiation, outre l'holocauste continuel, son offrande, et sa libation.
\VS{32}Et au septième jour, vous offrirez sept jeunes taureaux, deux béliers, et quatorze agneaux d'un an, sans défaut,
\VS{33}avec les gâteaux et les libations pour les jeunes taureaux, les béliers, et les agneaux, selon leur nombre et leur ordonnance.
\VS{34}Et un bouc en sacrifice d'expiation, outre l'holocauste continuel, son gâteau, et sa libation.
\VS{35}Et au huitième jour, vous aurez une assemblée solennelle~: Vous ne ferez aucune œuvre servile.
\VS{36}Et vous offrirez en holocauste un sacrifice consumé par le feu, d'une agréable odeur à Yahweh~: Un jeune taureau, un bélier, et sept agneaux d'un an, sans défaut,
\VS{37}avec les gâteaux et les libations pour le jeune taureau, le bélier, et les agneaux, selon leur nombre et leur ordonnance.
\VS{38}Et un bouc en sacrifice d'expiation, outre l'holocauste perpétuel, son offrande, et sa libation.
\VS{39}Vous offrirez ces choses à Yahweh dans vos fêtes solennelles, outre vos vœux, et vos offrandes volontaires, selon vos holocaustes, vos gâteaux, vos libations, et vos sacrifices d'offrande de paix.
\Chap{30}
\TextTitle{Les vœux}
\VerseOne{}Et Moïse parla aux enfants d'Israël selon toutes les choses que Yahweh lui avait ordonné.
\VS{2}Moïse parla aussi aux chefs des tribus des enfants d'Israël, en disant~: Voici ce que Yahweh ordonne.
\VS{3}Quand un homme fera un vœu à Yahweh, ou aura juré par serment, pour lier son âme par un vœu, il ne violera pas sa parole~; il fera selon toutes les choses qui sont sorties de sa bouche\FTNT{De. 23:21.}.
\VS{4}Mais quand une femme fera un vœu à Yahweh, et qu'elle se liera par un serment, dans sa jeunesse, étant encore dans la maison de son père,
\VS{5}et que son père aura entendu son vœu et le serment par lequel elle a lié son âme, si son père ne lui dit rien, tous ses vœux seront valables, et tout serment par lequel elle aura lié son âme sera valable~;
\VS{6}mais si son père la désapprouve le jour où il l'a entendue, aucun de ses vœux ou de ses serments par lesquels elle a lié son âme ne sera valable, et Yahweh lui pardonnera~; parce que son père l'a désapprouvée.
\VS{7}Et si elle a un mari, et qu'elle s'est engagée par quelque vœu ou par une parole échappée de ses lèvres par laquelle elle aura lié son âme,
\VS{8}et que son mari l'aura entendue, et que le jour même où il l'a entendue, il ne lui a rien dit, ses vœux alors seront valables, et ses serments par lesquels elle aura lié son âme seront valables~;
\VS{9}mais si son mari la désapprouve le jour où il l'a entendue, alors il annulera le vœu par lequel elle s'est engagée et la parole échappée de ses lèvres, par laquelle elle avait lié son âme~; et Yahweh lui pardonnera.
\VS{10}Mais le vœu de la veuve ou de la répudiée, tout ce par quoi elle aura lié son âme, sera valable pour elle.
\VS{11}Que si étant encore dans la maison de son mari elle a fait un vœu, ou si elle a lié son âme par serment,
\VS{12}et que son mari l'ait entendue, et ne lui en ait rien dit, et ne l'ait pas désapprouvée, alors tous ses vœux seront valables, et tout serment par lequel elle a lié son âme sera valable.
\VS{13}Mais si son mari les a entièrement annulés le jour où il les a entendus, alors rien de ce qui est sorti de ses lèvres, soit ses vœux, soit le serment par lequel elle a lié son âme ne seront valables~; parce que son mari les a annulés, et Yahweh lui pardonnera.
\VS{14}Son mari ratifiera ou son mari annulera tout vœu et toute obligation faite par serment, pour affliger l'âme.
\VS{15}Mais si son mari ne lui en a absolument rien dit, d'un jour à l'autre, il aura ratifié tous ses vœux ou toutes ses obligations dont elle était tenue~; il les aura, dis-je, ratifiés, parce qu'il ne lui en a rien dit le jour où il les a entendus.
\VS{16}Mais s'il les a expressément annulés après les avoir entendus, alors il portera l'iniquité de sa femme.
\VS{17}Telles sont les ordonnances que Yahweh ordonna à Moïse, entre un mari et sa femme~; entre un père et sa fille, étant encore dans la maison de son père, dans sa jeunesse.
\Chap{31}
\TextTitle{Jugements sur Madian\FTNTT{No. 25:6-18.}}
\VerseOne{}Yahweh parla à Moïse, en disant~:
\VS{2}Fais la vengeance des enfants d'Israël sur les Madianites, puis tu seras recueilli auprès de ton peuple.
\VS{3}Moïse donc parla au peuple, en disant~: Que quelques-uns d'entre vous s'équipent pour aller à la guerre, et qu'ils aillent contre Madian, pour exécuter la vengeance de Yahweh sur Madian.
\VS{4}Vous enverrez à la guerre mille hommes de chaque tribu, de toutes les tribus d'Israël.
\VS{5}On donna d'entre les milliers d'Israël mille hommes de chaque tribu, qui furent douze mille hommes équipés pour la guerre.
\VS{6}Moïse les envoya à la guerre, savoir mille de chaque tribu, et avec eux Phinées, fils d'Eléazar, le prêtre, qui portait les instruments sacrés et les trompettes retentissantes.
\VS{7}Ils s'avancèrent donc contre Madian, comme Yahweh l'avait ordonné à Moïse, et ils en tuèrent tous les mâles.
\VS{8}Ils tuèrent aussi les rois de Madian, outre les autres qui y furent tués, Evi, Rékem, Tsur, Hur, et Réba, cinq rois de Madian~; ils firent aussi passer au fil de l'épée Balaam, fils de Beor\FTNT{Jos. 13:21-22.}.
\VS{9}Et les fils d'Israël emmenèrent prisonniers les femmes de Madian, avec leurs petits enfants, et pillèrent tout leur gros et menu bétail, et tous leurs biens.
\VS{10}Ils brûlèrent par le feu toutes leurs villes, leurs demeures, et tous leurs châteaux.
\VS{11}Ils prirent tout le butin et tout le pillage, tant des hommes que du bétail\FTNT{De. 20:14.}~;
\VS{12}puis ils amenèrent les captifs, le pillage, et le butin, à Moïse, à Eléazar le prêtre, et à l'assemblée des enfants d'Israël, au camp, dans les plaines de Moab, qui sont près du Jourdain, vis-à-vis de Jéricho.
\VS{13}Moïse, Eléazar, le prêtre, et tous les princes de l'assemblée sortirent au-devant d'eux, hors du camp.
\VS{14}Et Moïse se mit en grande colère contre les officiers de l'armée, les chefs des milliers, et les chefs des centaines, qui revenaient de cet exploit de guerre.
\VS{15}Et Moïse leur dit~: N'avez-vous pas gardé en vie toutes les femmes~?
\VS{16}Voici ce sont elles qui, à la parole de Balaam, ont donné l'occasion aux fils d'Israël de pécher contre Yahweh dans l'affaire de Peor~; ce qui attira la plaie sur l'assemblée de Yahweh\FTNT{2 Pi. 2:15~; Ap. 2:14.}.
\VS{17}Or maintenant, tuez tous les mâles d'entre les petits enfants, et tuez toute femme qui a connu un homme en couchant avec lui\FTNT{Jg. 21:11.}~;
\VS{18}mais vous garderez en vie toutes les jeunes filles qui n'ont point connu la couche d'un homme.
\VS{19}Au reste, demeurez sept jours hors du camp~; quiconque aura tué quelqu'un, et quiconque aura touché quelqu'un qui aura été tué, se purifiera le troisième et le septième jour, tant vous que vos prisonniers.
\VS{20}Vous purifierez aussi tous vos vêtements, et tout ce qui sera fait de peau, et tout ouvrage de poil de chèvre, et toute vaisselle de bois.
\VS{21}Eléazar, le prêtre, dit aux hommes de guerre qui étaient allés au combat~: Voici l'ordonnance et la loi que Yahweh a ordonné à Moïse.
\VS{22}En général l'or, l'argent, l'airain, le fer, l'étain, le plomb~;
\VS{23}tout ce qui peut passer par le feu, vous le ferez passer par le feu pour le rendre pur. Seulement on purifiera avec l'eau de purification toutes les choses qui ne peuvent aller au feu, vous les ferez passer dans l'eau.
\VS{24}Vous laverez aussi vos vêtements le septième jour, ensuite vous serez purs~; puis vous entrerez au camp.
\TextTitle{Partage du butin}
\VS{25}Et Yahweh parla à Moïse, en disant~:
\VS{26}Fais le compte du butin et de tout ce qu'on a emmené, tant des personnes que des bêtes, toi et Eléazar, le prêtre, et les chefs des pères de l'assemblée.
\VS{27}Et partage par moitié le butin entre les combattants qui sont allés à la guerre et toute l'assemblée\FTNT{1 S. 30:24.}.
\VS{28}Tu prélèveras aussi pour Yahweh un tribut sur les hommes de guerre qui sont allés à la bataille, savoir un sur cinq cents, tant des personnes, que des bœufs, des ânes et des brebis.
\VS{29}On le prendra sur leur moitié, et tu le donneras à Eléazar, le prêtre, en offrande présentée par élévation à Yahweh.
\VS{30}Et sur la moitié qui appartient aux enfants d'Israël, tu prendras un sur cinquante, tant des personnes que des bœufs, des ânes, des brebis et de tous les autres animaux, et tu le donneras aux Lévites qui ont la charge de garder le tabernacle de Yahweh.
\VS{31}Moïse et Eléazar, le prêtre, firent comme Yahweh l'avait ordonné à Moïse.
\VS{32}Or le butin qui était resté du pillage du peuple qui était allé à la guerre, était de six cent soixante-quinze mille brebis~;
\VS{33}de soixante-douze mille bœufs~;
\VS{34}de soixante et un mille ânes,
\VS{35}quant aux femmes qui n'avaient point connu la couche d' un homme, elles étaient en tout trente-deux mille âmes.
\VS{36}Et la moitié du butin, à savoir la part de ceux qui étaient allés à la guerre, montait à trois cent trente-sept mille cinq cents brebis~;
\VS{37}dont le tribut pour Yahweh, quant aux brebis, était de six cent soixante-quinze.
\VS{38}Trente-six mille bœufs~; dont le tribut pour Yahweh, quant aux bœufs, était de soixante-douze bœufs,
\VS{39}trente mille cinq cents ânes~; dont le tribut pour Yahweh, quant aux ânes, était de soixante et un ânes~;
\VS{40}et de seize mille personnes, dont le tribut pour Yahweh était de trente-deux personnes.
\VS{41}Et Moïse donna à Eléazar, le prêtre, le tribut de l'offrande présentée par élévation à Yahweh, comme Yahweh le lui avait ordonné.
\VS{42}Et de l'autre moitié qui appartenait aux enfants d'Israël, que Moïse avait tiré des hommes qui étaient allés à la guerre~;
\VS{43}or de cette moitié qui fut pour l'assemblée, et qui montait à trois cent trente-sept mille cinq cents brebis,
\VS{44}trente-six mille bœufs,
\VS{45}trente mille cinq cents ânes,
\VS{46}et à seize mille personnes~;
\VS{47}de cette moitié, dis-je, qui appartenait aux enfants d'Israël, Moïse prit un sur cinquante, tant des personnes que des bêtes, et les donna aux Lévites qui avaient la charge de garder le tabernacle de Yahweh, comme Yahweh le lui avait ordonné.
\VS{48}Les commandants des milliers de l'armée, tant les chefs des milliers que les chefs des centaines, s'approchèrent de Moïse,
\VS{49}et lui dirent~: Tes serviteurs ont fait le compte des hommes de guerre qui étaient sous nos ordres, il ne manque pas un homme d'entre nous.
\VS{50}C'est pourquoi, nous offrons l'offrande de Yahweh, chacun les objets que nous avons trouvés~: Des joyaux d'or, des chaînes de cheville, des bracelets, des anneaux, des pendants d'oreilles et des colliers, afin de faire propitiation pour nos personnes devant Yahweh.
\VS{51}Moïse et Eléazar, le prêtre, reçurent d'eux cet or, tous ces objets travaillés.
\VS{52}Et tout l'or de l'offrande présentée par élévation à Yahweh, de la part des chefs de milliers et des chefs de centaines, montait à seize mille sept cent cinquante sicles.
\VS{53}Or les hommes de guerre gardèrent chacun pour soi ce qu'ils avaient pillé.
\VS{54}Moïse donc et Eléazar, le prêtre, prirent l'or des chefs des milliers et des chefs de centaines, et l'apportèrent à la tente d'assignation, comme souvenir pour les enfants d'Israël, devant Yahweh.
\Chap{32}
\TextTitle{Ruben et Gad en Galaad}
\VerseOne{}Les fils de Ruben et les fils de Gad avaient beaucoup de bétail, en très grande quantité, et ils virent que le pays de Jaezer et le pays de Galaad étaient un lieu propre pour du bétail.
\VS{2}Ainsi les fils de Gad et les fils de Ruben vinrent, et parlèrent à Moïse et à Eléazar, le prêtre, et aux princes de l'assemblée, en disant~:
\VS{3}Atharoth, Dibon, Jaezer, Nimra, Hesbon, et Elealé, Sebam, Nebo, et Beon,
\VS{4}ce pays-là que Yahweh a frappé devant l'assemblée d'Israël, est un pays propre pour le bétail, et tes serviteurs ont des troupeaux.
\VS{5}Ils dirent donc: Si nous avons trouvé grâce à tes yeux, que ce pays soit donné en possession à tes serviteurs~; et ne nous fais point passer le Jourdain.
\VS{6}Mais Moïse répondit aux fils de Gad, et aux fils de Ruben~: Vos frères iront-ils à la guerre, et vous, demeurerez-vous ici~?
\VS{7}Pourquoi voulez-vous décourager les enfants d'Israël de passer dans le pays que Yahweh leur a donné~?
\VS{8}C'est ainsi que firent vos pères quand je les envoyai de Kadès-Barnéa pour examiner le pays.
\VS{9}Car ils montèrent jusqu'à la vallée d'Eschcol, virent le pays, puis découragèrent les enfants d'Israël, afin qu'ils n'entrent point dans le pays que Yahweh leur avait donné.
\VS{10}C'est pourquoi la colère de Yahweh s'enflamma ce jour-là, et il jura en disant~:
\VS{11}Les hommes qui sont montés hors d'Egypte, depuis l'âge de vingt ans et au-dessus, ne verront point le pays que j'ai juré de donner à Abraham, Isaac, et à Jacob~; car ils n'ont point persévéré à me suivre\FTNT{De. 1:35.},
\VS{12}excepté Caleb, fils de Jephunné, le Kénizien, et Josué, fils de Nun, car ils ont persévéré à suivre Yahweh.
\VS{13}Ainsi la colère de Yahweh s'enflamma contre Israël et il les fit errer dans le désert pendant quarante ans, jusqu'à ce que toute la génération qui avait fait le mal aux yeux de Yahweh, ait été consumée.
\VS{14}Et voici, vous vous êtes levés à la place de vos pères, comme une race d'hommes pécheurs, pour augmenter encore l'ardeur de la colère de Yahweh contre Israël.
\VS{15}Si vous vous détournez de lui, il continuera encore à vous laisser au désert, et vous ferez détruire tout ce peuple.
\VS{16}Mais ils s'approchèrent de lui et lui dirent~: Nous bâtirons ici des cloisons pour nos troupeaux, et des villes pour nos petits enfants~;
\VS{17}et nous nous équiperons pour marcher promptement devant les enfants d'Israël, jusqu'à ce que nous les ayons introduits en leur lieu~; mais nos petits enfants demeureront dans les villes fortes, à cause des habitants du pays.
\VS{18}Nous ne retournerons point dans nos maisons avant que chacun des enfants d'Israël n'ait pris possession de son héritage~;
\VS{19}et nous ne posséderons rien en héritage avec eux au-delà du Jourdain, ni plus avant~; parce que nous aurons notre héritage de ce côté-ci du Jourdain, à l'orient.
\VS{20}Et Moïse leur dit~: Si vous faites cela, si vous vous équipez devant Yahweh pour aller à la guerre,
\VS{21}si chacun de vous étant équipé passe le Jourdain devant Yahweh, jusqu'à ce qu'il ait chassé ses ennemis loin de devant lui,
\VS{22}et que le pays soit assujetti devant Yahweh, et qu'ensuite vous vous en retournez, alors vous serez innocents envers Yahweh, et envers Israël~; et ce pays-ci vous appartiendra pour le posséder devant Yahweh.
\VS{23}Mais si vous ne faites point cela, vous péchez contre Yahweh~; et sachez que votre péché vous atteindra.
\VS{24}Bâtissez donc des villes pour vos petits enfants, et des cloisons pour vos troupeaux, et faites ce que vous avez dit.
\VS{25}Alors les fils de Gad et les fils de Ruben parlèrent à Moïse, en disant~: Tes serviteurs feront ce que mon seigneur a ordonné.
\VS{26}Nos petits enfants, nos femmes, nos troupeaux, et tout notre bétail demeureront ici dans les villes de Galaad~;
\VS{27}et tes serviteurs passeront chacun armés pour aller à la guerre devant Yahweh, prêts à combattre, comme mon seigneur a parlé.
\VS{28}Alors Moïse donna des ordres à leur sujet à Eléazar, le prêtre, à Josué, fils de Nun, et aux chefs des pères des tribus des fils d'Israël.
\VS{29}Il leur dit~: Si les fils de Gad et les fils de Ruben passent avec vous le Jourdain tous armés, prêts à combattre devant Yahweh, et que le pays vous soit assujetti, vous leur donnerez le pays de Galaad en possession.
\VS{30}Mais s'ils ne marchent point en armes avec vous, qu'ils s'établissent au milieu de vous dans le pays de Canaan.
\VS{31}Les fils de Gad et les fils de Ruben répondirent, en disant~: Nous ferons ce que Yahweh a dit à tes serviteurs.
\VS{32}Nous passerons en armes devant Yahweh au pays de Canaan, afin que nous possédions pour notre héritage ce qui est de ce côté-ci du Jourdain.
\VS{33}Ainsi Moïse donna aux fils de Gad et aux fils de Ruben, et à la demi-tribu de Manassé, fils de Joseph, le royaume de Sihon, roi des Amoréens~; et le royaume de Og, roi de Basan, le pays avec ses villes, selon les bornes des villes du pays tout autour.
\VS{34}Alors les fils de Gad rebâtirent Dibon, Atharoth, Aroër,
\VS{35}Athroth-Schophan, Jaezer, Jogbeha,
\VS{36}Beth-Nimra et Beth-Haran, villes fortifiées. Ils firent aussi des cloisons pour les troupeaux.
\VS{37}Et les fils de Ruben rebâtirent Hesbon, Elealé, Kirjathaim,
\VS{38}Nébo, Baal-Meon, et Sibma, dont ils changèrent les noms, et ils donnèrent des noms aux villes qu'ils rebâtirent.
\VS{39}Or les fils de Makir, fils de Manassé, allèrent en Galaad, le prirent et dépossédèrent les Amoréens qui y étaient.
\VS{40}Moïse donc donna Galaad à Makir, fils de Manassé, qui y habita\FTNT{De. 3:15.}.
\VS{41}Jaïr, fils de Manassé, se mit en marche, prit leurs villages, et les appela villages de Jaïr\FTNT{De. 3:14~; 1 Ch. 2:22.}.
\VS{42}Et Nobach se mit en marche, prit Kenath avec les villes de son ressort, et l'appela Nobach d'après son nom.
\Chap{33}
\TextTitle{Les stations de l'Egypte jusqu'au Jourdain}
\VerseOne{}Ce sont ici les étapes des enfants d'Israël, qui sortirent du pays d'Egypte, selon leurs armées, sous la main de Moïse et d'Aaron.
\VS{2}Moïse écrivit leurs départs, et leurs étapes, d'après l'ordre de Yahweh~! Et voici leurs étapes selon leurs départs.
\VS{3}Les enfants d'Israël donc partirent de Ramsès le quinzième jour du premier mois, dès le lendemain de la Pâque, et ils sortirent à main levée, à la vue de tous les Egyptiens\FTNT{Ex. 14:8.}.
\VS{4}Et les Egyptiens ensevelissaient ceux que Yahweh avait frappés parmi eux, à savoir tous les premiers-nés~; même Yahweh exerçait aussi ses jugements contre leurs dieux\FTNT{Ex. 12:12~; Ex. 18:11.}.
\VS{5}Et les enfants d'Israël partirent de Ramsès, et campèrent à Succoth\FTNT{Ex. 12:37.}.
\VS{6}Et ils partirent de Succoth et campèrent à Etham, qui est au bout du désert\FTNT{Ex. 13:20.}.
\VS{7}Et ils partirent d'Etham et se détournèrent vers Pi-Hahiroth, qui est vis-à-vis de Baal-Tsephon, et campèrent devant Migdol\FTNT{Ex. 14:2.}.
\VS{8}Et ils partirent de devant Pi-Hahiroth et passèrent au travers de la mer vers le désert, et firent trois journées de marche par le désert d'Etham et campèrent à Mara.
\VS{9}Puis ils partirent de Mara et vinrent à Elim où il y avait douze fontaines d'eaux et soixante-dix palmiers, et ils y campèrent\FTNT{Ex. 15:27.}.
\VS{10}Et ils partirent d'Elim et campèrent près de la Mer Rouge.
\VS{11}Puis ils partirent de la Mer Rouge et campèrent au désert de Sin\FTNT{Ex. 16:1.}.
\VS{12}Ils partirent du désert de Sin et campèrent à Dophka.
\VS{13}Puis ils partirent de Dophka et campèrent à Alusch.
\VS{14}Et ils partirent d'Alusch et campèrent à Rephidim où il n'y avait point d'eau à boire pour le peuple\FTNT{Ex. 17:1.}.
\VS{15}Puis ils partirent de Rephidim et campèrent dans le désert de Sinaï\FTNT{Ex. 17:1.}.
\VS{16}Ils partirent du désert de Sinaï et campèrent à Kibroth-Hattaava.
\VS{17}Et ils partirent de Kibroth-Hattaava et campèrent à Hatséroth.
\VS{18}Puis ils partirent de Hatséroth et campèrent à Rithma.
\VS{19}Et ils partirent de Rithma et campèrent à Rimmon-Pérets.
\VS{20}Ils partirent de Rimmon-Pérets et campèrent à Libna.
\VS{21}Et ils partirent de Libna et campèrent à Rissa.
\VS{22}Puis ils partirent de Rissa et campèrent vers Kehélatha.
\VS{23}Et ils partirent de Kehélatha et campèrent à la montagne de Schapher.
\VS{24}Ils partirent de la montagne de Schapher et campèrent à Harada.
\VS{25}Et ils partirent de Harada et campèrent à Makhéloth.
\VS{26}Puis ils partirent de Makhéloth et campèrent à Tahath.
\VS{27}Ils partirent de Tahath et campèrent à Tarach.
\VS{28}Et ils partirent de Tarach et campèrent à Mithka.
\VS{29}Puis ils partirent de Mithka et campèrent à Haschmona.
\VS{30}Ils partirent de Haschmona et campèrent à Moséroth.
\VS{31}Et ils partirent de Moséroth et campèrent à Bené-Jaakan.
\VS{32}Ils partirent de Bené-Jaakan et campèrent à Hor-Guidgad.
\VS{33}Puis ils partirent de Hor-Guidgad et campèrent vers Jothbatha.
\VS{34}Ils partirent de Jothbatha et campèrent à Abrona.
\VS{35}Et ils partirent d'Abrona et campèrent à Etsjon-Guéber.
\VS{36}Ils partirent d'Etsjon-Guéber et campèrent dans le désert de Tsin, qui est Kadès.
\VS{37}Puis ils partirent de Kadès et campèrent à la montagne de Hor, qui est au bout du pays d'Edom.
\VS{38}Et Aaron le prêtre, monta sur la montagne de Hor, suivant l'ordre de Yahweh, et mourut là, la quarantième année après que les enfants d'Israël furent sortis du pays d'Egypte, le premier jour du cinquième mois.
\VS{39}Et Aaron était âgé de cent vingt-trois ans quand il mourut sur la montagne de Hor.
\VS{40}Alors le Cananéen, roi d'Arad, qui habitait vers le midi au pays de Canaan, apprit que les enfants d'Israël venaient.
\VS{41}Et ils partirent de la montagne de Hor et campèrent à Tsalmona.
\VS{42}Puis ils partirent de Tsalmona et campèrent à Punon.
\VS{43}Et Ils partirent de Punon et campèrent à Oboth.
\VS{44}Ils partirent d'Oboth et campèrent à Ijjé-Abarim, sur les frontières de Moab.
\VS{45}Puis ils partirent d'Ijjé-Abarim et campèrent à Dibon-Gad.
\VS{46}Et ils partirent de Dibon-Gad, et campèrent à Almon-Diblathaïm.
\VS{47}Ils partirent d'Almon-Diblathaïm et campèrent aux montagnes de Abarim devant Nébo.
\VS{48}Et ils partirent des montagnes d'Abarim et campèrent aux plaines de Moab, près du Jourdain de Jéricho.
\VS{49}Pui ils campèrent près du Jourdain, depuis Beth-Jeschimoth jusqu'à Abel-Sittim, dans les plaines de Moab.
\TextTitle{Consignes pour les possessions attribuées à Israël}
\VS{50}Et Yahweh parla à Moïse dans les plaines de Moab, près du Jourdain de Jéricho, en disant~:
\VS{51}Parle aux enfants d'Israël, et dis-leur~: Puisque vous allez passer le Jourdain pour entrer au pays de Canaan,
\VS{52}vous chasserez de devant vous tous les habitants du pays, vous détruirez toutes leurs peintures, et vous ruinerez toutes leurs images de fonte, et vous démolirez tous leurs hauts lieux\FTNT{De. 7:5~; De. 12:2.}.
\VS{53}Et vous prendrez possession du pays, et vous y habiterez. Car je vous ai donné le pays pour le posséder.
\VS{54}Or vous recevrez le pays en héritage par le sort, selon vos familles. A ceux qui sont en plus grand nombre, vous donnerez plus d'héritage, et à ceux qui sont en plus petit nombre, vous donnerez moins d'héritage. Chacun aura selon ce qui lui sera échu par le sort, et vous hériterez selon les tribus de vos pères.
\VS{55}Mais si vous ne chassez pas de devant vous les habitants du pays, il arrivera que ceux d'entre eux que vous aurez laissés comme reste, seront comme des épines à vos yeux, et comme des pointes à vos côtés, et ils vous serreront de près dans le pays auquel vous habiterez\FTNT{Jos. 23:13.}.
\VS{56}Et il arrivera que je vous ferai tout comme j'ai eu dessein de leur faire.
\Chap{34}
\TextTitle{Consignes sur les limites de chaque tribu}
\VerseOne{}Yahweh parla aussi à Moïse, en disant~:
\VS{2}Donne l'ordre aux enfants d'Israël, et dis-leur~: Parce que vous allez entrer au pays de Canaan, ce pays deviendra votre héritage, le pays de Canaan selon ses limites.
\VS{3}Votre frontière du côté du sud sera depuis le désert de Tsin, le long d'Edom, et votre frontière du côté du sud commencera au bout de la mer salée, vers l'orient~;
\VS{4}et cette frontière tournera du sud vers la montée d'Akrabbim, et passera jusqu'à Tsin~; et elle aboutira du côté du sud de Kadès-Barnéa~; et sortira aussi par Hatsar-Addar, et passera jusqu'à Atsmon.
\VS{5}Et cette frontière tournera depuis Atsmon jusqu'au torrent d'Egypte~; et elle aboutira à la mer.
\VS{6}Quant à la frontière d'occident, vous aurez la grande mer et ses limites~; ce sera votre frontière occidentale.
\VS{7}Et ce sera ici votre frontière au nord~; depuis la grande mer, vous marquerez pour vos limites la montagne de Hor~;
\VS{8}et depuis la montagne de Hor, vous marquerez pour vos limites l'entrée de Hamath, et cette frontière aboutira vers Tsedad~;
\VS{9}cette frontière passera jusqu'à Ziphron, et elle aboutira à Hatsar-Enan~; telle sera votre frontière au nord.
\VS{10}Puis vous marquerez pour vos limites vers l'orient de Hatsar-Enan à Schepham.
\VS{11}Et cette frontière descendra de Schepham à Ribla, du côté de l'orient d'Aïn~; et cette frontière descendra et s'étendra le long de la Mer de Kinnéreth vers l'orient.
\VS{12}Cette frontière descendra au Jourdain pour aboutir à la Mer Salée~; tel sera le pays que vous aurez avec ses limites tout autour.
\VS{13}Et Moïse donna l'ordre aux enfants d'Israël, en disant~: C'est là le pays que vous hériterez par le sort, et que Yahweh a ordonné de donner à neuf tribus, et à la demi-tribu.
\VS{14}Car la tribu des fils de Ruben selon les familles de leurs pères, et la tribu des fils de Gad, selon les familles de leurs pères, ont pris leur héritage~; et la demi-tribu de Manassé a pris aussi son héritage.
\VS{15}Deux tribus, dis-je, et la demi-tribu ont pris leur héritage de l'autre côté du Jourdain, vis-à-vis de Jéricho, du côté du levant.
\VS{16}Et Yahweh parla à Moïse, en disant~:
\VS{17}Ce sont ici les noms des hommes qui vous partageront le pays~: Eléazar le prêtre, et Josué fils de Nun.
\VS{18}Vous prendrez aussi un prince de chaque tribu pour faire le partage du pays.
\VS{19}Et voici les noms de ces hommes. Pour la tribu de Juda~: Caleb, fils de Jephunné~;
\VS{20}pour la tribu des fils de Siméon~: Samuel, fils d'Ammihud~;
\VS{21}pour la tribu de Benjamin~: Elidad, fils de Kislon~;
\VS{22}pour la tribu des fils de Dan~: Celui qui en est le chef, Buki, fils de Jogli~;
\VS{23}pour les fils de Joseph, pour la tribu des fils de Manassé~: Celui qui en est le chef, Hanniel, fils d'Ephod~;
\VS{24}et pour la tribu des fils d'Ephraïm~: Celui qui en est le chef, Kemuel, fils de Schiphtan~;
\VS{25}pour la tribu des fils de Zabulon~: Celui qui en est le chef, Elitsaphan, fils de Parnac~;
\VS{26}pour la tribu des fils d'Issacar~: Celui qui en est le chef, Paltiel, fils d'Azzan~;
\VS{27}pour la tribu des fils d'Aser~: Celui qui en est le chef, Ahihud, fils de Schelomi~;
\VS{28}pour la tribu des fils de Nephthali~: Celui qui en est le chef, Pedahel, fils d'Ammihud.
\VS{29}Ce sont là, ceux à qui Yahweh donna l'ordre de partager l'héritage aux enfants d'Israël dans le pays de Canaan.
\Chap{35}
\TextTitle{Quarante-huit villes pour les Lévites dont six villes de refuge}
\VerseOne{}Yahweh parla à Moïse dans les plaines de Moab, près du Jourdain, vis-à-vis de Jéricho, en disant~:
\VS{2}Donne l'ordre aux enfants d'Israël qu'ils donnent aux Lévites, sur l'héritage qu'ils posséderont, des villes pour y habiter. Vous leur donnerez aussi les faubourgs qui sont autour de ces villes\FTNT{Jos. 21:2.}.
\VS{3}Ils auront donc les villes pour y habiter~; et les faubourgs de ces villes seront pour leurs bétails, pour leurs biens, et pour tous leurs animaux.
\VS{4}Les faubourgs des villes que vous donnerez aux Lévites, seront de mille coudées tout autour depuis la muraille de la ville en dehors.
\VS{5}Et vous mesurerez depuis le dehors de la ville du côté de l'orient, deux mille coudées~; et du côté du sud, deux mille coudées~; et du côté de l'occident, deux mille coudées~; et du côté du nord, deux mille coudées~; et la ville sera au milieu~; tels seront les faubourgs de leurs villes.
\VS{6}Et des villes que vous donnerez aux Lévites, il y aura six villes de refuge que vous donnerez pour que le meurtrier s'y enfuie, et outre celles-là, vous leur donnerez quarante-deux villes.
\VS{7}Toutes les villes que vous donnerez aux Lévites seront quarante-huit villes, elles et leurs faubourgs.
\VS{8}Et quant aux villes que vous leur donnerez sur la possession des enfants d'Israël, de ceux qui en auront plus vous en prendrez plus, et de ceux qui en auront moins vous en prendrez moins~; chacun donnera de ses villes aux Lévites, en proportion de l'héritage qu'il possédera.
\VS{9}Puis Yahweh parla à Moïse, en disant~:
\VS{10}Parle aux enfants d'Israël, et dis-leur~: Quand vous aurez passé le Jourdain, pour entrer au pays de Canaan~;
\VS{11}établissez-vous des villes qui vous soient des villes de refuge, afin que le meurtrier qui aura frappé à mort quelqu'un involontairement, s'y enfuie\FTNT{Jos. 20:2-3~; Ex. 21:13.}.
\VS{12}Et ces villes seront pour vous des villes de refuge contre le vengeur, afin que le meurtrier ne meure pas, jusqu'à ce qu'il ait comparu en jugement devant l'assemblée.
\VS{13}De ces villes que vous donnerez, il y en aura six de refuge pour vous.
\VS{14}Vous donnerez trois de ces villes au-delà du Jourdain, et les trois autres dans le pays de Canaan, qui seront des villes de refuge\FTNT{De. 19:2~; De. 4:41-42.}.
\VS{15}Ces six villes serviront de refuge aux enfants d'Israël, à l'étranger et à celui qui séjourne au milieu de vous, afin que quiconque aura frappé à mort quelqu'un involontairement, s'y enfuie.
\VS{16}Mais si un homme en frappe un autre avec un instrument de fer, et qu'il en meure, il est meurtrier~; on punira de mort le meurtrier.
\VS{17}Et s'il le frappe avec une pierre qu'il tenait à la main, dont on puisse mourir, et qu'il en meure, c'est un meurtrier~; le meurtrier sera puni de mort.
\VS{18}De même s'il le frappe d'un instrument de bois qu'il tenait à la main, dont on puisse mourir, et qu'il en meure, il est un meurtrier~; on punira de mort le meurtrier.
\VS{19}Et le vengeur du sang fera mourir le meurtrier quand il le rencontrera, il pourra le faire mourir.
\VS{20}Et s'il le pousse par haine, ou s'il jette quelque chose sur lui avec préméditation, et qu'il en meure~;
\VS{21}ou si par inimitié il le frappe de sa main, et qu'il en meure, on punira de mort celui qui l'a frappé, car il est meurtrier~; le vengeur du sang pourra le faire mourir quand il le rencontrera\FTNT{De. 19:11-12.}.
\VS{22}Mais s'il le pousse subitement, sans inimitié, ou s'il jette quelque chose sur lui, sans préméditation,
\VS{23}ou s'il fait tomber sur lui quelque pierre sans l'avoir vu, et qu'il en meure, n'étant pas son ennemi et ne lui cherchant pas du mal,
\VS{24}alors l'assemblée jugera entre celui qui a frappé et le vengeur du sang, selon ces ordonnances~;
\VS{25}l'assemblée délivrera le meurtrier de la main du vengeur de sang, et le fera retourner dans la ville de refuge où il s'était enfui. Il y demeurera jusqu'à la mort du grand-prêtre, qui aura été oint de la sainte huile.
\VS{26}Mais si le meurtrier sort de quelque manière que ce soit hors des bornes de la ville de son refuge, où il s'est enfui,
\VS{27}et si le vengeur du sang le rencontre hors des bornes de la ville de son refuge, et qu'il tue le meurtrier, il ne sera point coupable de meurtre.
\VS{28}Car il doit demeurer dans la ville de son refuge jusqu'à la mort du grand-prêtre~; et après la mort du grand-prêtre, le meurtrier pourra retourner dans sa possession.
\VS{29}Et ces choses-ci seront des ordonnances de jugement pour vous et pour vos générations, dans toutes vos demeures.
\VS{30}Celui qui fera mourir le meurtrier, le fera mourir sur la parole de deux témoins~; mais un seul témoin ne sera point reçu en témoignage contre quelqu'un, pour le faire mourir\FTNT{De. 17:6~; De. 19:15.}.
\VS{31}Et vous ne prendrez point de rançon pour la vie du meurtrier, qui est coupable et digne de mort~; mais il doit être puni de mort.
\VS{32}Vous ne prendrez point de rançon pour le laisser s'enfuir de sa ville de refuge, pour qu'il retourne habiter dans le pays, jusqu'à la mort du prêtre.
\VS{33}Et vous ne souillerez point le pays où vous serez, car le sang souille le pays~; et il ne se fera point de propitiation pour le pays, du sang qui y sera répandu que par le sang de celui qui l'aura répandu.
\VS{34}Vous ne souillerez donc point le pays où vous allez demeurer, et au milieu duquel j'habiterai~; car je suis Yahweh qui habite au milieu des enfants d'Israël.
\Chap{36}
\TextTitle{Loi sur les héritages\FTNTT{No. 27:1-11.}}
\VerseOne{}Or les chefs des pères de la famille des fils de Galaad, fils de Makir, fils de Manassé, d'entre les familles des fils de Joseph, s'approchèrent et parlèrent devant Moïse, et devant les princes, les chefs des pères des enfants d'Israël,
\VS{2}et ils dirent~: Yahweh a donné l'ordre à mon seigneur de donner aux enfants d'Israël le pays en héritage par le sort~; et mon seigneur a reçu l'ordre de Yahweh de donner l'héritage de Tselophchad, notre frère, à ses filles.
\VS{3}Si elles se marient à l'un des fils des autres tribus d'Israël, leur héritage sera retranché de l'héritage de nos pères et sera ajouté à l'héritage de la tribu de laquelle elles seront~; ainsi sera diminué l'héritage qui nous est échu par le sort.
\VS{4}Même quand viendra le jubilé pour les enfants d'Israël, on ajoutera leur héritage à l'héritage de la tribu à laquelle elles appartiendront, ainsi leur héritage sera retranché de l'héritage de la tribu de nos pères\FTNT{Lé. 25:10-13.}.
\VS{5}Et Moïse ordonna aux enfants d'Israël, suivant l'ordre de la bouche de Yahweh, en disant~: Ce que la tribu des fils de Joseph dit est juste.
\VS{6}C'est ici ce que Yahweh ordonne au sujet des filles de Tselophchad~: Elles se marieront à qui bon leur semblera, toutefois elles se marieront dans l'une des familles de la tribu de leurs pères.
\VS{7}Ainsi l'héritage ne sera point transporté entre les enfants d'Israël de tribu en tribu~; car chacun des enfants d'Israël se tiendra à l'héritage de la tribu de ses pères.
\VS{8}Et toute fille, qui possédera un héritage d'entre les tribus des enfants d'Israël, se mariera à quelqu'un de la famille de la tribu de son père, afin que chacun des enfants d'Israël possède l'héritage de ses pères.
\VS{9}L'héritage donc ne sera point transporté d'une tribu à une autre, mais chacune des tribus des enfants d'Israël se tiendra à son héritage.
\VS{10}Les filles de Tselophchad firent comme Yahweh avait donné à Moïse.
\VS{11}Machla, Thirtsa, Hogla, Milca, et Noa, filles de Tselophchad, se marièrent aux fils de leurs oncles.
\VS{12}Ainsi elles se marièrent à ceux qui étaient des familles des fils de Manassé, fils de Joseph~; et leur héritage demeura dans la tribu de la famille de leur père.
\VS{13}Ce sont là les ordonnances et les jugements que Yahweh ordonna par Moïse aux enfants d'Israël, dans les plaines de Moab, près du Jourdain, vis-à-vis de Jéricho.
\PPE{}
\end{multicols}

%\clearpage\ShortTitle{De.}\BookTitle{Deutéronome}\BFont
\noindent\hrulefill
{\footnotesize
\textit{
\bigskip
{\centering{}
\\Auteur~: Probablement Moïse
\\(Heb.~: Devarim)
\\Signification~: Paroles
\\Thème~: Rappel de la loi
\\Date de rédaction~: 1450-1410 av. J.-C.\\}
}
\textit{
\\Ce livre est un rappel de la loi de Yahweh. Après quarante années d'errance dans le désert, Moïse s'adresse à la nouvelle génération par des discours et des exhortations, depuis les plaines de Moab. Au travers de son serviteur, Dieu rappelle ainsi la loi donnée sur le mont Sinaï, les expériences vécues par la génération passée et par conséquent, l'importance de la soumission à Dieu. De leur obéissance dépendraient les bénédictions ou les malédictions contenues dans ce livre.\bigskip
}
}
\par\nobreak\noindent\hrulefill
\begin{multicols}{2}
\Chap{1}
\TextTitle{Rappel de l'infidélité d'Israël\FTNTT{No. 14.}}
\VerseOne{}Ce sont ici les paroles que Moïse déclara à tout Israël de l'autre côté du Jourdain, dans le désert, dans la plaine, qui est vis-à-vis de Suph, entre Paran, Tophel, Laban, Hatséroth, et Di-Zahab.
\VS{2}Il y a onze journées depuis Horeb, par le chemin de la montagne de Séir, jusqu'à Kadès-Barnéa.
\VS{3}Or il arriva dans la quarantième année, au onzième mois, le premier jour du mois, que Moïse parla aux enfants d'Israël selon tout ce que Yahweh lui avait ordonné de leur dire,
\VS{4}après qu'il eut battu Sihon, roi des Amoréens, qui habitait à Hesbon, et Og, roi de Basan, qui demeurait à Aschtaroth et à Edréi\FTNT{No. 21:23-24.}.
\VS{5}Moïse donc commença à expliquer cette loi, de l'autre côté du Jourdain, dans le pays de Moab, en disant~:
\VS{6}Yahweh, notre Dieu, nous a parlé à Horeb, en disant~: Vous avez assez demeuré dans cette montagne.
\VS{7}Tournez-vous, partez, et allez à la montagne des Amoréens et dans tous les lieux voisins, dans la plaine, dans la montagne, dans la vallée, vers le sud, sur le rivage de la mer, au pays des Cananéens et au Liban, jusqu'au grand fleuve, le fleuve d'Euphrate.
\VS{8}Regardez, j'ai mis devant vous le pays~; entrez et prenez possession du pays que Yahweh a juré de donner à vos pères, Abraham, Isaac et Jacob, et à leur postérité après eux.
\VS{9}Et je vous parlai en ce temps-là, et je vous dis~: Je ne puis pas, à moi seul, vous porter.
\VS{10}Yahweh, votre Dieu, vous a multipliés, et vous voici aujourd'hui comme les étoiles du ciel par le nombre.
\VS{11}Que Yahweh, le Dieu de vos pères, vous fasse croître mille fois au delà de ce que vous êtes et vous bénisse, comme il vous l'a dit~!
\VS{12}Comment porterais-je moi seul vos chagrins, vos charges, et vos procès~?
\VS{13}Prenez dans vos tribus des hommes sages, intelligents et connus, et je les établirai chefs sur vous.
\VS{14}Et vous me répondîtes, et dîtes~: Il est bon de faire ce que tu as dit.
\VS{15}Alors je pris les chefs de vos tribus, des hommes sages et connus, et je les établis chefs sur vous, chefs de milliers, chefs de centaines, chefs de cinquantaines, chefs de dizaines et officiers selon vos tribus. 
\VS{16}Puis j'ordonnai en ce temps-là à vos juges, en disant~: Ecoutez les différends qui seront entre vos frères, et jugez droitement entre l'homme et son frère, et entre l'étranger qui est avec lui\FTNT{Lé. 19:15~; De. 16:19~; Pr. 24:23.}.
\VS{17}Vous n'aurez point d'égard à l'apparence de la personne en jugement~; vous entendrez autant le petit que le grand~; vous ne craindrez personne, car le jugement est à Dieu~; et vous ferez venir devant moi la cause qui sera trop difficile pour vous, et je l'entendrai. 
\VS{18}Et en ce temps-là, je vous ordonnai toutes les choses que vous deviez faire.
\VS{19}Puis nous partîmes d'Horeb, et nous marchâmes dans tout ce grand et affreux désert que vous avez vu~; par le chemin de la montagne des Amoréens, ainsi que Yahweh, notre Dieu, nous l'avait ordonné, et nous vînmes jusqu'à Kadès-Barnéa.
\VS{20}Alors je vous dis~: Vous êtes arrivés jusqu'à la montagne des Amoréens, que Yahweh, notre Dieu, nous donne.
\VS{21}Regarde, Yahweh, ton Dieu, met le pays devant toi~; monte et prends-en possession, comme Yahweh, le Dieu de tes pères, te l'a dit~; ne crains point et ne t'effraie point.
\VS{22}Et vous vous approchâtes tous de moi, et dîtes~: Envoyons devant nous des hommes pour explorer le pays, et qui nous rapportent des nouvelles du chemin par lequel nous devrons monter, et des villes où nous devrons aller\FTNT{No. 13:2.}.
\VS{23}Et ce discours sembla bon à mes yeux~; et je pris douze hommes parmi vous, un homme par tribu.
\VS{24}Et ils se mirent en chemin et montèrent dans la montagne, et vinrent jusqu'au torrent d'Eschcol et explorèrent le pays.
\VS{25}Et ils prirent dans leurs mains des fruits du pays, et ils nous les apportèrent~; ils nous donnèrent des nouvelles, et nous dirent~: Le pays que Yahweh, notre Dieu, nous donne est bon. 
\VS{26}Mais vous refusâtes d'y monter, et vous fûtes rebelles à l'ordre de Yahweh, votre Dieu.
\VS{27}Et vous murmurâtes dans vos tentes, en disant~: C'est parce que Yahweh nous hait qu'il nous a fait sortir du pays d'Egypte, afin de nous livrer entre les mains des Amoréens pour nous exterminer.
\VS{28}Où monterions-nous~? Nos frères nous ont fait fondre le cœur, en disant~: Le peuple est plus grand que nous, et de plus haute taille~; les villes sont grandes et closes jusqu'au ciel~; et même nous avons vu là les fils des Anakim.
\VS{29}Mais je vous dis~: Ne tremblez point et ne les craignez point.
\VS{30}Yahweh, votre Dieu, qui marche devant vous, lui-même combattra pour vous, selon tout ce que vous avez vu qu'il a fait pour vous en Egypte~;
\VS{31}et au désert, où tu as vu de quelle manière Yahweh, ton Dieu, t'a porté comme un homme porterait son fils, sur tout le chemin où vous avez marché, jusqu'à ce que vous soyez arrivés dans ce lieu-ci.
\VS{32}Mais malgré cela, vous ne crûtes point encore en Yahweh, votre Dieu,
\VS{33}qui marchait devant vous sur le chemin afin de vous chercher un lieu pour camper, marchant de nuit dans la colonne de feu pour vous éclairer dans le chemin par lequel vous deviez marcher et de jour dans la nuée.
\VS{34}Et Yahweh entendit la voix de vos paroles et se mit en grande colère et jura, disant~:
\VS{35}Aucun des hommes de cette méchante génération ne verra ce bon pays que j'ai juré de donner à vos pères,
\VS{36}à l'exception de Caleb, fils de Jephunné~; lui le verra, et je donnerai à lui et à ses fils le pays sur lequel il a marché, parce qu'il a persévéré à suivre Yahweh\FTNT{No. 14:22-24.}.
\VS{37}Même Yahweh s'est mis en colère contre moi à cause de vous, disant~: Et toi aussi tu n'y entreras pas.
\VS{38}Josué, fils de Nun, qui te sert, y entrera~; fortifie-le, car c'est lui qui mettra les enfants d'Israël en possession de ce pays\FTNT{De. 34:4.}.
\VS{39}Et vos petits-enfants, dont vous avez dit qu'ils seront en proie, vos enfants, dis-je, qui aujourd'hui ne savent pas ce que c'est le bien ou le mal, eux y entreront, et je leur donnerai ce pays et ils le posséderont.
\VS{40}Mais vous, retournez vous-en en arrière, et allez dans le désert par le chemin de la Mer Rouge.
\VS{41}Et vous répondîtes et me dîtes~: Nous avons péché contre Yahweh, nous monterons et nous combattrons, comme Yahweh, notre Dieu, nous l'a ordonné. Et vous ceignîtes chacun vos armes de guerre, et vous entreprîtes hardiment de monter à la montagne.
\VS{42}Et Yahweh me dit~: Dis-leur~: Ne montez point et ne combattez point, car je ne suis point au milieu de vous~; afin que vous ne soyez point battus par vos ennemis.
\VS{43}Je vous parlai, mais vous ne m'écoutâtes point et vous vous rebellâtes contre l'ordre de Yahweh, et vous fûtes orgueilleux et vous montâtes sur la montagne.
\VS{44}Et les Amoréens, qui demeuraient sur cette montagne, sortirent contre vous et vous poursuivirent comme font les abeilles~; et ils vous battirent depuis Séir jusqu'à Horma.
\VS{45}Et étant retournés vous pleurâtes devant Yahweh~; mais Yahweh n'écouta point votre voix, et ne vous prêta point l'oreille.
\VS{46}Ainsi, vous demeurâtes à Kadès plusieurs jours, autant de temps que vous y aviez demeuré.
\Chap{2}
\TextTitle{Périple du peuple dans le désert}
\VerseOne{}Alors nous retournâmes en arrière, et nous partîmes pour le désert, par le chemin de la Mer Rouge, comme Yahweh me l'avait dit, et nous tournâmes autour de la montagne de Séir plusieurs jours.
\VS{2}Et Yahweh me parla, en disant~:
\VS{3}Vous avez assez tourné autour de cette montagne. Tournez-vous vers le nord.
\VS{4}Ordonne au peuple, en disant~: Vous allez passer la frontière de vos frères, les fils d'Esaü, qui demeurent en Séir. Ils auront peur de vous~; mais soyez bien sur vos gardes.
\VS{5}N'ayez pas de démêlé avec eux~; car je ne vous donnerai rien dans leur pays, pas même de quoi poser la plante du pied~: J'ai donné à Esaü la montagne de Séir en héritage.
\VS{6}Vous achèterez d'eux la nourriture à prix d'argent et vous en mangerez, et vous achèterez d'eux l'eau à prix d'argent et vous en boirez.
\VS{7}Car Yahweh, ton Dieu, t'a béni dans tout le travail de tes mains, il a connu ta marche dans ce grand désert. Yahweh, ton Dieu, a été avec toi pendant ces quarante années, et tu n'as manqué de rien.
\VS{8}Nous passâmes à distance de nos frères, les fils d'Esaü, qui demeuraient en Séir, à distance du chemin de la plaine, d'Elath et d'Etsjon-Guéber, et nous nous tournâmes, et nous passâmes par le chemin du désert de Moab.
\VS{9}Yahweh me dit~: N'assiège point Moab, et ne t'engage pas dans un combat avec lui~; car je ne te donnerai rien en héritage dans son pays~: J'ai donné Ar en héritage aux fils de Lot\FTNT{Ge. 19:36-38.}.
\VS{10}Les Emim y habitaient auparavant~; c'était un peuple grand, nombreux et de haute taille comme les Anakim.
\VS{11}Ils étaient considérés comme des Rephaïm, de même que les Anakim~; mais les Moabites les appelaient Emim.
\VS{12}Séir était habité autrefois par les Horiens~; mais les fils d'Esaü les en dépossédèrent, les détruisirent devant eux, et y habitèrent à leur place, comme l'a fait Israël dans le pays de son héritage que Yahweh lui a donné.
\VS{13}Mais maintenant, levez-vous, et passez le torrent de Zéred. Et nous passâmes le torrent de Zéred.
\VS{14}Or le temps que nous avons marché de Kadès-Barnéa, jusqu'à ce que nous ayons passé le torrent de Zéred, fut de trente-huit ans, jusqu'à ce que toute la génération des hommes de guerre eût été consumée du milieu du camp, comme Yahweh le leur avait juré.
\VS{15}La main de Yahweh fut aussi sur eux pour les détruire du milieu du camp, jusqu'à ce qu'ils eussent été consumés.
\VS{16}Or il est arrivé qu'après que tous les hommes de guerre eurent été consumés par la mort du milieu du peuple,
\VS{17}Yahweh me parla, et dit~:
\VS{18}Tu vas passer aujourd'hui la frontière de Moab, à savoir Har.
\VS{19}Tu t'approcheras en face des fils d'Ammon, mais ne les assiège point, et ne t'engage point dans un combat avec eux~; car je ne te donnerai rien en possession dans le pays des fils d'Ammon~: Je l'ai donné en héritage aux fils de Lot.
\VS{20} Ce pays était aussi considéré comme un pays de Rephaïm~; car les Rephaïm y habitaient auparavant, et les Ammonites les appelaient Zamzummim~;
\VS{21}c'était un peuple grand, nombreux, et de haute taille, comme les Anakim, Yahweh les détruisit devant eux, et ils les dépossédèrent, et habitèrent à leur place.
\VS{22}Comme il fit pour les fils d'Esaü qui demeurent en Séir, quand il détruisit les Horiens devant eux~; ils les dépossédèrent et habitèrent à leur place jusqu'à ce jour.
\VS{23}Or quant aux Avviens, qui demeuraient en Hatserim jusqu'à Gaza, ils furent détruits par les Caphtorim, sortis de Caphtor, qui demeurèrent à leur place.
\TextTitle{Yawheh livre Sihon, roi de Hesbon, entre les mains d'Israël}
\VS{24}Levez-vous, partez et passez le torrent de l'Arnon. Regarde, j'ai livré entre tes mains Sihon, roi de Hesbon, l'Amoréen, et son pays. Commence à en prendre possession, et fais-lui la guerre~!
\VS{25}Aujourd'hui, je vais commencer à mettre la frayeur et la crainte de toi sur les peuples qui sont sous les cieux~; et ayant entendu parler de toi, ils trembleront et seront dans l'angoisse à cause de ta présence.
\VS{26}J'envoyai, du désert de Kedémoth, des messagers à Sihon, roi de Hesbon, avec des paroles de paix, disant\FTNT{No. 21:21.}~:
\VS{27}Permets que je passe par ton pays~; et j'irai par le grand chemin, sans me détourner ni à droite ni à gauche.
\VS{28}Tu me vendras de la nourriture à prix d'argent, afin que je mange, et tu me donneras de l'eau à prix d'argent, afin que je boive~; seulement que j'y passe de mes pieds.
\VS{29}C'est ce qu'ont fait les fils d'Esaü qui demeurent en Séir, et les Moabites qui demeurent à Ar, jusqu'à ce que je passe le Jourdain pour entrer au pays que Yahweh, notre Dieu, nous donne.
\VS{30}Mais Sihon, roi de Hesbon, ne voulut point nous laisser passer par son pays~; car Yahweh, ton Dieu, avait endurci son esprit, et raidit son cœur afin de le livrer entre tes mains, comme tu le vois aujourd'hui.
\VS{31}Yahweh me dit~: Regarde, j'ai commencé à te livrer Sihon et son pays~; commence à posséder son pays, pour le tenir en héritage.
\VS{32}Sihon donc, sortit nous rencontrer avec tout son peuple pour nous combattre à Jahats.
\VS{33}Mais Yahweh, notre Dieu, nous le livra en face, et nous le battîmes, lui, ses fils, et tout son peuple.
\VS{34}Et en ce temps-là, nous prîmes toutes ses villes, et nous détruisîmes par le moyen de l'interdit les villes, les hommes, les femmes, et les petits enfants, sans laisser de survivants.
\VS{35}Seulement nous pillâmes les bêtes pour nous, et le butin des villes que nous avions prises.
\VS{36}Depuis Aroër, qui est sur le bord du torrent de l'Arnon, et la ville qui est dans la vallée, jusqu'à Galaad, il n'y eut pas une ville qui fût trop haute pour nous~: Yahweh, notre Dieu, nous livra tout.
\VS{37}Seulement tu n'approchas point du pays des fils d'Ammon, de tous les bords du torrent de Jabbok, des villes de la montagne, ni d'aucun lieu que Yahweh, notre Dieu, t'avait ordonné de ne point attaquer.
\Chap{3}
\TextTitle{Yawheh livre Og, roi de Basan, entre les mains d'Israël}
\VerseOne{}Alors nous nous tournâmes, et nous montâmes par le chemin de Basan. Et Og, roi de Basan, sortit nous rencontrer, avec tout son peuple, pour nous combattre à Edréi.
\VS{2}Et Yahweh me dit~: Ne le crains point~; car je le livre entre tes mains, lui, tout son peuple, et son pays~; et tu lui feras comme tu as fait à Sihon, roi des Amoréens, qui demeurait à Hesbon.
\VS{3}Ainsi Yahweh, notre Dieu, livra aussi entre nos mains Og, roi de Basan, avec tout son peuple~; nous le battîmes sans laisser de survivants.
\VS{4}En ce même temps, nous prîmes aussi toutes ses villes, et il n'y eut point de ville que nous ne lui prîmes pas~: Soixante villes, toute la contrée d'Argob, le royaume d'Og en Basan.
\VS{5}Toutes ces villes-là étaient fortifiées, avec de hautes murailles, des portes et des barres. Il y avait aussi des villes sans murailles en fort grand nombre.
\VS{6}Et nous les détruisîmes par le moyen de l'interdit, comme nous l'avions fait à Sihon, roi de Hesbon~; nous dévouâmes par le moyen de l'interdit toutes les villes, les hommes, les femmes, et les petits enfants.
\VS{7}Mais nous pillâmes pour nous toutes les bêtes et le butin des villes.
\VS{8}Nous prîmes donc, en ce temps-là, le pays de la main des deux rois des Amoréens, qui étaient de l'autre côté du Jourdain, depuis le torrent de l'Arnon jusqu'à la montagne de l'Hermon~;
\VS{9}or les Sidoniens donnent à l'Hermon le nom de Sirion, mais les Amoréens le nomment Senir~;
\VS{10}toutes les villes de la plaine, tout Galaad, et tout Basan jusqu'à Salca et Edréi, les villes du royaume d'Og en Basan.
\VS{11}Og, roi de Basan, avait survécu seul du reste des Rephaïm. Voici, son lit, un lit de fer, n'est-il pas dans Rabbath, ville des fils d'Ammon~? Sa longueur est de neuf coudées, et sa largeur de quatre coudées, en coudées d'homme.
\TextTitle{Premières terres attribuées à Ruben, Gad et à la demi-tribu de Manassé}
\VS{12}En ce temps-là donc, nous prîmes possession de ce pays. Je donnai aux Rubénites et aux Gadites le territoire à partir d'Aroër, sur le torrent de l'Arnon, et la moitié de la montagne de Galaad, avec ses villes\FTNT{Jos. 13:23-32.}.
\VS{13}Je donnai à la demi-tribu de Manassé le reste de Galaad et tout le royaume d'Og, en Basan~: Toute la contrée d'Argob avec tout le Basan, c'est ce qu'on appelait le pays des Réphaïm.
\VS{14}Jaïr, fils de Manassé, prit toute la contrée d'Argob jusqu'à la frontière des Gueschuriens et des Maacathiens, et il donna son nom à Basan, appelé villages de Jaïr jusqu'à aujourd'hui.
\VS{15}Je donnai aussi Galaad à Makir.
\VS{16}Mais aux Rubénites et aux Gadites, je donnai de Galaad jusqu'au torrent de l'Arnon, dont le milieu du torrent sert de frontière, et jusqu'au torrent de Jabbok, frontière des fils d'Ammon~;
\VS{17}la plaine, et le Jourdain, de la frontière de Kinnéreth jusqu'à la mer de la plaine, la Mer Salée, aux pieds de Pisga vers l'orient.
\VS{18}Or en ce temps-là, je vous ordonnai, en disant~: Yahweh, votre Dieu, vous donne ce pays pour le posséder. Vous tous, qui êtes vaillants, vous passerez armés devant vos frères, les fils d'Israël.
\VS{19}Seulement vos femmes, vos petits-enfants, et vos troupeaux, car je sais que vous avez beaucoup de troupeaux, resteront dans les villes que je vous ai données,
\VS{20}jusqu'à ce que Yahweh ait accordé du repos à vos frères comme à vous, et qu'eux aussi possèdent le pays que Yahweh, votre Dieu, leur donne de l'autre côté du Jourdain. Puis vous retournerez chacun dans l'héritage que je vous ai donné.
\VS{21}En ce temps-là, j'ordonnai à Josué, en disant~: Tes yeux ont vu tout ce que Yahweh, votre Dieu, a fait à ces deux rois~: Yahweh en fera de même à tous les royaumes vers lesquels tu vas passer.
\VS{22}Ne les craignez point~; car Yahweh, votre Dieu, combattra lui-même pour vous.
\TextTitle{Moïse n'entrera pas dans la terre promise}
\VS{23}En ce même temps, j'implorai la grâce de Yahweh, en disant~:
\VS{24}Seigneur Yahweh, tu as commencé à montrer à ton serviteur ta grandeur et ta main puissante~; car quel est le dieu dans le ciel et sur la terre qui puisse faire selon tes œuvres et selon ta puissance~?
\VS{25}Que je passe, je te prie, et que je voie ce bon pays de l'autre côté du Jourdain, ces bonnes montagnes et le Liban.
\VS{26}Mais Yahweh s'irrita contre moi, à cause de vous, et ne m'écouta point. Yahweh me dit~: C'est assez, ne me parle plus de cette affaire.
\VS{27}Monte au sommet du Pisga, et lève tes yeux à l'occident, au nord, au sud, et à l'orient, et regarde de tes yeux~; car tu ne passeras point ce Jourdain.
\VS{28}Donnes-en la charge à Josué, fortifie-le et affermis-le~; car c'est lui qui passera devant ce peuple et qui le mettra en possession du pays que tu verras.
\VS{29}Ainsi nous demeurâmes dans la vallée, vis-à-vis de Beth-Peor.
\Chap{4}
\TextTitle{Encouragement à garder la loi de Yahweh}
\VerseOne{}Et maintenant Israël, écoute les lois et les ordonnances que je vous enseigne, pour les pratiquer afin que vous viviez, que vous entriez et possédiez le pays que Yahweh, le Dieu de vos pères, vous donne.
\VS{2}Vous n'ajouterez\FTNT{De. 12:32~; Pr. 30:6~; Ap. 22:18-19.} rien à la parole que je vous ordonne, et vous n'en retrancherez rien~; afin de garder les commandements de Yahweh, votre Dieu, que je vous ordonne.
\VS{3}Vos yeux ont vu ce que Yahweh a fait à cause de Baal-Peor~: Yahweh, ton Dieu, a détruit du milieu de toi tout homme qui était allé après Baal-Peor\FTNT{No. 25:4-9.}.
\VS{4}Mais vous, qui vous êtes attachés à Yahweh, votre Dieu, vous êtes tous vivants aujourd'hui.
\VS{5}Regardez, je vous ai enseigné des lois et des ordonnances, comme Yahweh, mon Dieu, me l'a ordonné, afin que vous les pratiquiez au milieu du pays où vous allez pour le posséder.
\VS{6}Vous les garderez et vous les pratiquerez, car c'est là votre sagesse et votre intelligence aux yeux de tous les peuples, qui entendront ces lois, et qui diront~: Cette grande nation est un peuple sage et intelligent~!
\TextTitle{Israël, privilégié parmi tous les peuples}
\VS{7}Car quelle est la grande nation qui ait ses dieux proches d'elle, comme nous avons Yahweh, notre Dieu, toutes les fois que nous l'invoquons~?
\VS{8}Et quelle est la grande nation qui ait des lois et des ordonnances justes, comme toute cette loi que je mets aujourd'hui devant vous~?
\VS{9}Seulement, prends garde à toi et garde soigneusement ton âme, afin que tous les jours de ta vie tu n'oublies point les choses que tes yeux ont vues, et qu'elles ne sortent de ton cœur\FTNT{Pr. 4:23.}~; enseigne-les à tes fils, et aux fils de tes fils.
\VS{10}Rappelle-toi du jour où tu te tins face à Yahweh, ton Dieu, à Horeb, après que Yahweh me dit~: Convoque le peuple~! Je veux leur faire entendre mes paroles, pour qu'ils apprennent à me craindre tout le temps qu'ils seront vivants sur la terre~; et pour qu'ils les enseignent à leurs fils.
\VS{11}Et que vous vous approchâtes, et vous vous tîntes au pied de la montagne. Or la montagne était embrasée de feu jusqu'au milieu du ciel. Il y avait des ténèbres, une nuée, et une obscurité.
\VS{12}Et Yahweh vous parla du milieu du feu~; vous entendîtes le son de ses paroles, mais vous ne vîtes aucune image, vous entendîtes seulement la voix\FTNT{Ex. 19:17-19.}.
\VS{13}Et il déclara son alliance, qu'il vous ordonna d'observer, les dix paroles, qu'il écrivit sur deux tables de pierre.
\VS{14}Yahweh m'ordonna aussi, en ce temps-là, de vous enseigner les lois et les ordonnances, afin que vous les pratiquiez dans le pays que vous allez posséder.
\VS{15}Prenez bien garde à vos âmes, puisque vous n'avez vu aucune image le jour où Yahweh, votre Dieu, vous parla du milieu du feu à Horeb,
\VS{16}de peur que vous ne vous corrompiez et que vous ne vous fassiez une image taillée, une représentation d'idole ayant la forme d'un mâle ou d'une femelle,
\VS{17}ou la forme d'un animal qui soit sur la terre, ou la forme d'un oiseau ailé qui vole dans les cieux,
\VS{18}ou la forme d'un animal qui rampe sur la terre, ou la forme d'un poisson qui soit dans les eaux au-dessous de la terre.
\VS{19}De peur aussi qu'élevant tes yeux vers les cieux, et voyant le soleil, la lune, et les étoiles, toute l'armée des cieux, tu ne sois poussé à te prosterner devant elles, et que tu ne les serves~: C'est ce que Yahweh, ton Dieu, a donné en partage à tous les peuples, sous tous les cieux.
\VS{20}Mais vous, Yahweh vous a pris, et vous a fait sortir d'Egypte, du fourneau de fer, afin que vous fussiez un peuple de son héritage, comme vous l'êtes aujourd'hui.
\TextTitle{Conséquences de la désobéissance et de l'idolâtrie}
\VS{21}Or Yahweh s'irrita contre moi, à cause de vos paroles, et il jura que je ne passerais point le Jourdain, et que je n'entrerais point dans ce bon pays que Yahweh, ton Dieu, te donne en héritage.
\VS{22}Je mourrai dans ce pays-ci, je ne passerai point le Jourdain~; mais vous le passerez, et vous posséderez ce bon pays.
\VS{23}Gardez-vous d'oublier l'alliance de Yahweh, votre Dieu, qu'il a traitée avec vous, et que vous ne vous fassiez d'image taillée, de représentation quelconque, que Yahweh, votre Dieu, vous a défendues.
\VS{24}Car Yahweh, ton Dieu, est un feu dévorant\FTNT{Hé. 12:29.}, un Dieu jaloux.
\VS{25}Quand tu auras engendré des fils, et des fils de tes fils, et que vous serez depuis longtemps dans le pays, si vous vous corrompez, et que vous faites des images taillées, ou des représentations de quelque chose que ce soit, si vous faites ce qui est mal aux yeux de Yahweh, votre Dieu, afin de l'irriter
\VS{26}j'appelle aujourd'hui à témoin les cieux et la terre contre vous, certainement vous périrez promptement dans ce pays que vous allez posséder au-delà du Jourdain, vous n'y prolongerez point vos jours, car vous serez entièrement détruits.
\VS{27}Yahweh vous dispersera parmi les peuples, et vous ne resterez qu'un petit nombre parmi les nations, chez lesquelles Yahweh vous emmènera.
\VS{28}Et là, vous servirez des dieux, œuvres de main d'homme, du bois et de la pierre, qui ne peuvent voir, ni entendre, ni manger, ni sentir\FTNT{Es. 44:9~; Es. 46:7~; Ps. 115:4-7}.
\TextTitle{Yahweh, puissant, miséricordieux et fidèle à son alliance}
\VS{29}Mais de là, tu chercheras Yahweh, ton Dieu, et tu le trouveras, si tu le cherches de tout ton cœur et de toute ton âme.
\VS{30}Quand tu seras dans la détresse, et que toutes ces choses te seront arrivées, alors, dans les derniers jours, tu retourneras à Yahweh, ton Dieu, et tu obéiras à sa voix~;
\VS{31}parce que Yahweh, ton Dieu, est le Dieu puissant et miséricordieux, il ne t'abandonnera point et ne te détruira point, il n'oubliera point l'alliance de tes pères qu'il leur a jurée.
\VS{32}Interroge les premiers temps, qui ont été avant toi, depuis le jour que Dieu créa l'homme sur la terre, et d'une extrémité des cieux à l'autre, s'il a jamais été rien fait de semblable à cette grande chose, et s'il a été jamais rien entendu de semblable.
\VS{33}Est-ce qu'un peuple a entendu la voix de Dieu parlant du milieu du feu, comme tu l'as entendue, et qui soit demeuré en vie~?
\VS{34}Où Dieu a-t-il essayé de venir prendre pour lui une nation du milieu d'une nation, par des épreuves, des signes, des miracles, et des batailles, à main forte, et à bras étendu, et par des choses grandes et terribles, comme tout ce que Yahweh, notre Dieu, a fait pour vous en Egypte, sous vos yeux~?
\VS{35}Cela t'a été montré afin que tu reconnaisses que Yahweh est Dieu et qu'il n'y en a point d'autre.
\VS{36}Il t'a fait entendre sa voix des cieux pour t'instruire~; et il t'a montré son grand feu sur la terre, et tu as entendu ses paroles du milieu du feu.
\VS{37}Et parce qu'il a aimé tes pères, il a choisi leur postérité après eux et il t'a retiré d'Egypte en sa présence, par sa grande puissance~;
\VS{38}pour chasser de devant toi des nations plus grandes et plus puissantes que toi, pour te faire entrer dans leur pays, et pour te le donner en héritage, comme tu le vois aujourd'hui.
\VS{39}Sache donc aujourd'hui, et rappelle dans ton cœur que Yahweh est Dieu, en haut dans les cieux et sur la terre, et qu'il n'y en a point d'autre.
\VS{40}Garde donc ses lois et ses commandements que je t'ordonne aujourd'hui, afin que tu sois heureux, toi et tes fils après toi, et que tu prolonges tes jours sur la terre que Yahweh, ton Dieu, te donne\FTNT{Ex 20.}.
\TextTitle{Trois villes de refuge à l'est du Jourdain}
\VS{41}Alors Moïse sépara trois villes de l'autre côté du Jourdain vers le soleil levant,
\VS{42}afin que le meurtrier qui aurait tué son prochain involontairement, sans l'avoir haï auparavant, s'y enfuie~; et qu'en s'enfuyant dans l'une de ces villes-là, il eût sa vie sauve.
\VS{43}C'étaient~: Betser dans le désert, dans la plaine du pays, chez les Rubénites~; Ramoth en Galaad, chez les Gadites~; Golan en Basan, chez les Manassites.
\VS{44}C'est ici la loi que Moïse plaça face aux enfants d'Israël.
\VS{45}Voici les témoignages, les lois, et les ordonnances que Moïse déclara aux enfants d'Israël, après qu'ils furent sortis d'Egypte.
\VS{46}C'était de l'autre côté du Jourdain, dans la vallée, vis-à-vis de Beth-Peor, au pays de Sihon, roi des Amoréens, qui demeurait à Hesbon, et qui fut battu par Moïse et les enfants d'Israël après être sortis d'Egypte.
\VS{47}Et ils s'emparèrent de son pays avec le pays d'Og, roi de Basan, deux rois des Amoréens qui étaient de l'autre côté du Jourdain, vers le soleil levant.
\VS{48}Depuis Aroër, sur le bord du torrent de l'Arnon, jusqu'à la montagne de Sion, qui est l'Hermon,
\VS{49}et toute la plaine de l'autre côté du Jourdain vers l'orient, jusqu'à la mer de la plaine, au pied du Pisga.
\Chap{5}
\TextTitle{L'alliance établie à Horeb rappelée à la nouvelle génération}
\VerseOne{}Moïse appela tout Israël, et leur dit~: Ecoute, Israël, les lois et les ordonnances que je prononce aujourd'hui à vos oreilles, apprenez-les, et veillez à les mettre en pratique.
\VS{2}Yahweh, notre Dieu, a traité avec nous une alliance en Horeb\FTNT{Ex. 19:5.}.
\VS{3}Dieu n'a point traité cette alliance avec nos pères, mais avec nous, qui sommes ici aujourd'hui tous vivants.
\VS{4}Yahweh vous parla face à face sur la montagne du milieu du feu.
\VS{5}Je me tenais en ce temps-là entre Yahweh et vous, pour vous rapporter la parole de Yahweh~; parce que vous aviez peur face à ce feu, et vous ne montâtes point sur la montagne. Il dit\FTNT{Les dix paroles (Ex. 20).}~:
\VS{6}Je suis Yahweh, ton Dieu, qui t'ai fait sortir du pays d'Egypte, de la maison de servitude.
\VS{7}Tu n'auras point d'autres dieux devant ma face.
\VS{8}Tu ne te feras point d'image taillée, ni de représentation des choses qui sont en haut dans les cieux, ni sur la terre, ni dans les eaux sous la terre.
\VS{9}Tu ne te prosterneras point devant elles, et tu ne les serviras point~; car je suis Yahweh, ton Dieu, un Dieu jaloux, qui punis l'iniquité des pères sur les enfants jusqu'à la troisième et à la quatrième génération de ceux qui me haïssent,
\VS{10}et qui fais miséricorde jusqu'à mille générations à ceux qui m'aiment et qui gardent mes commandements.
\VS{11}Tu ne prendras point le Nom de Yahweh, ton Dieu, en vain~; car Yahweh ne tiendra pas pour innocent celui qui prendra son Nom en vain.
\VS{12}Garde le jour du sabbat pour le sanctifier, comme Yahweh, ton Dieu, te l'a ordonné.
\VS{13}Tu travailleras six jours, et tu feras toute ton œuvre,
\VS{14}mais le septième jour est le sabbat de Yahweh, ton Dieu~: Tu ne feras aucune œuvre, ni ton fils, ni ta fille, ni ton serviteur, ni ta servante, ni ton bœuf, ni ton âne, ni aucune de tes bêtes, ni l'étranger qui est dans tes portes, afin que ton serviteur et ta servante se reposent comme toi.
\VS{15}Et tu te souviendras que tu as été esclave au pays d'Egypte, et que Yahweh, ton Dieu, t'en a fait sortir à main forte et à bras étendu~: C'est pourquoi Yahweh, ton Dieu, t'a ordonné d'observer le jour du sabbat.
\VS{16}Honore ton père et ta mère, comme Yahweh, ton Dieu, te l'a ordonné, afin que tes jours se prolongent et que tu sois heureux sur la terre que Yahweh, ton Dieu, te donne.
\VS{17}Tu ne tueras point.
\VS{18}Tu ne commettras point d'adultère.
\VS{19}Tu ne déroberas point.
\VS{20}Tu ne diras point de faux témoignage contre ton prochain.
\VS{21}Tu ne convoiteras point la femme de ton prochain~; tu ne désireras point la maison de ton prochain, ni son champ, ni son serviteur, ni sa servante, ni son bœuf, ni son âne, ni aucune chose qui soit à ton prochain.
\VS{22}Yahweh déclara ces paroles à toute votre assemblée sur la montagne, du milieu du feu, des nuées et de l'obscurité, à voix forte sans rien ajouter. Il les écrivit sur deux tables de pierre qu'il me donna.
\TextTitle{Moïse, intermédiaire entre Yahweh et le peuple}
\VS{23}Or il arriva qu'aussitôt que vous eûtes entendu la voix du milieu de l'obscurité, parce que la montagne était embrasée par le feu, vos chefs de tribus et vos anciens s'approchèrent de moi,
\VS{24}et vous dîtes~: Voici, Yahweh, notre Dieu, nous a fait voir sa gloire et sa grandeur, et nous avons entendu sa voix du milieu du feu~; aujourd'hui, nous avons vu que Dieu a parlé avec l'homme, et qu'il est resté en vie.
\VS{25}Et maintenant pourquoi mourrions-nous~? Car ce grand feu là nous dévorera~; si nous entendons encore la voix de Yahweh, notre Dieu, nous mourrons.
\VS{26}Car qui, de toute chair, a entendu comme nous la voix du Dieu vivant parlant du milieu du feu, et qui soit resté en vie~?
\VS{27}Approche-toi et écoute tout ce que Yahweh, notre Dieu, dira~; puis tu nous diras tout ce que Yahweh, notre Dieu, t'aura dit~; nous l'entendrons, et nous le ferons.
\VS{28}Yahweh entendit la voix de vos paroles pendant que vous me parliez. Et Yahweh me dit~: J'ai entendu les paroles que ce peuple t'ont adressées~: Tout ce qu'ils ont dit est bien.
\VS{29}Ô~! S'ils avaient toujours ce même cœur pour me craindre et pour garder tous mes commandements, afin qu'ils fussent heureux, eux et leurs enfants, pour toujours~!
\VS{30}Va, dis-leur~: Retournez dans vos tentes.
\VS{31}Mais toi, reste ici avec moi, et je te dirai tous les commandements, les lois, et les ordonnances que tu leur enseigneras, afin qu'ils les pratiquent dans le pays que je leur donne en possession.
\VS{32}Vous prendrez donc garde de faire ce que Yahweh, votre Dieu, vous a ordonné~; vous ne vous en détournerez ni à droite ni à gauche.
\VS{33}Vous marcherez dans toute la voie que Yahweh, votre Dieu, vous a ordonnée, afin que vous viviez et que vous soyez heureux, et que vous prolongiez vos jours sur la terre que vous posséderez.
\Chap{6}
\TextTitle{Obéissance à la loi, source de bénédictions}
\VerseOne{}Voici les commandements, les lois et les ordonnances que Yahweh, votre Dieu, m'a ordonné de vous enseigner, afin que vous les pratiquiez dans le pays dans lequel vous allez passer pour le posséder~;
\VS{2}afin que tu craignes Yahweh, ton Dieu, en gardant durant tous les jours de ta vie, toi, ton fils, et le fils de ton fils, toutes ses lois et ses commandements que je t'ordonne, pour que tes jours soient prolongés.
\VS{3}Tu les écouteras donc, ô Israël, et tu auras soin de les mettre en pratique, afin que tu sois heureux, et que vous vous multipliiez sur la terre où coulent le lait et le miel, comme Yahweh, le Dieu de tes pères, l'a dit\FTNT{Ex. 3:8.}.
\VS{4}Ecoute Israël~! Yahweh, notre Dieu, Yahweh est Un\FTNT{Jacob fut le premier à faire cette prière qui affirme l'unicité de Dieu. Dieu est UN (en hébreu «~Echad~» ou «~Ehad~»). Loin de l'infirmer ou de la contredire, Jésus a confirmé cette prière et l'enseignement capital qu'elle contient (Mc. 12:29). Dieu n'est pas trois personnes en une, mais UN. Cette parole annonce un monothéisme absolu. Elle s'oppose catégoriquement au polythéisme des Cananéens qui adoraient de multiples dieux, les étoiles, la lune, le soleil, les arbres, les rois etc. Aussi les Hébreux avaient reçu l'ordre de la part de Yahweh de détruire toutes les idoles qu'ils trouveraient en la terre promise (De. 16:21). Voir également commentaire en Ge. 1:5.}.
\VS{5}Tu aimeras donc Yahweh, ton Dieu, de tout ton cœur, de toute ton âme, et de toute ta force\FTNT{Mt. 22:37~; Mc. 12:30.}.
\TextTitle{La loi de Yahweh doit être enseignée aux enfants}
\VS{6}Et ces paroles, que je t'ordonne aujourd'hui, seront dans ton cœur.
\VS{7}Tu les enseigneras soigneusement à tes enfants, et tu en parleras quand tu te tiendras dans ta maison, quand tu iras en voyage, quand tu te coucheras et quand tu te lèveras.
\VS{8}Et tu les lieras comme un signe sur tes mains, et elles seront comme des fronteaux entre tes yeux.
\VS{9}Tu les écriras aussi sur les poteaux de ta maison et sur tes portes.
\VS{10}Yahweh, ton Dieu, te fera entrer dans le pays qu'il a juré à tes pères, Abraham, Isaac, et Jacob, de te donner. Tu posséderas de grandes et bonnes villes que tu n'as point bâties,
\VS{11}des maisons pleines de toutes sortes de biens que tu n'as point remplies, des puits creusés que tu n'as point creusés, des vignes et des oliviers que tu n'as point plantés, tu mangeras, et tu te rassasieras.
\VS{12}Prends garde à toi, de peur que tu n'oublies Yahweh, qui t'a fait sortir du pays d'Egypte, de la maison de servitude.
\VS{13}Tu craindras Yahweh, ton Dieu, tu le serviras et tu jureras par son Nom.
\VS{14}Vous n'irez point après d'autres dieux, d'entre les dieux des peuples qui sont autour de vous~;
\VS{15}car Yahweh, ton Dieu, est un Dieu jaloux au milieu de toi~; de peur que la colère de Yahweh, ton Dieu, ne s'enflamme contre toi, et qu'il ne t'extermine de dessus la terre.
\VS{16}Vous ne tenterez point Yahweh, votre Dieu, comme vous l'avez tenté à Massa.
\VS{17}Vous garderez soigneusement les commandements de Yahweh, votre Dieu, ses ordonnances et ses lois qu'il vous a ordonnées.
\VS{18}Tu feras ce qui est droit et bon aux yeux de Yahweh, afin que tu sois heureux, que tu entres et que tu possèdes le bon pays que Yahweh a juré à tes pères,
\VS{19}après qu'il aura chassé tous tes ennemis de devant toi, comme Yahweh l'a dit.
\VS{20}Quand ton enfant t'interrogera à l'avenir, en disant~: Que veulent dire ces préceptes, ces lois, et ces ordonnances que Yahweh, notre Dieu, vous a ordonnés~?
\VS{21}Tu diras à ton enfant~: Nous étions esclaves de Pharaon en Egypte, et Yahweh nous a fait sortir de l'Egypte par sa main puissante.
\VS{22}Yahweh a fait sous nos yeux des signes et des miracles, grands et désastreux contre l'Egypte, contre Pharaon et contre toute sa maison~;
\VS{23}et il nous a fait sortir de là pour nous conduire dans le pays qu'il avait juré à nos pères de nous donner.
\VS{24}Yahweh nous a ordonné de pratiquer toutes ces lois, et de craindre Yahweh, notre Dieu, afin que nous soyons toujours heureux, et qu'il préserve notre vie, comme aujourd'hui.
\VS{25}Et ceci sera notre justice, que nous prenions garde de pratiquer tous ces commandements devant Yahweh, notre Dieu, comme il nous l'a ordonné.
\Chap{7}
\TextTitle{Yahweh interdit les alliances avec les peuples païens}
\VerseOne{}Quand Yahweh, ton Dieu, t'aura fait entrer dans le pays où tu vas entrer pour le posséder, et qu'il aura chassé de devant toi beaucoup de nations~: Les Héthiens, les Guirgasiens, les Amoréens, les Cananéens, les Phéréziens, les Héviens, et les Jébusiens, sept nations plus grandes et plus puissantes que toi~;
\VS{2}et que Yahweh, ton Dieu, te les aura livrées en face et que tu les auras battues, tu les dévoueras complètement à la façon de l'interdit, tu ne traiteras point d'alliance avec elles, et tu ne leur feras point de grâce.
\VS{3}Tu ne t'allieras point par mariage avec elles, tu ne donneras point tes filles à leurs fils, et tu ne prendras point leurs filles pour tes fils\FTNT{Jos. 23:12-13.}~;
\VS{4}car elles détourneraient de moi tes fils, et ils serviraient d'autres dieux, et la colère de Yahweh s'enflammerait contre vous~: Il te détruirait promptement.
\VS{5}Mais vous les traiterez de cette manière~: Vous renverserez leurs autels, vous briserez leurs statues, vous abattrez leurs Asherah\FTNT{Le mot idole vient de l'hébreu «~Asherah~». Il est cité au moins quarante fois dans le Tanakh. Il fait référence à un objet en bois utilisé dans le culte d'Astarté, l'épouse de Baal. Voir De. 19:21.}, et vous brûlerez au feu leurs images taillées.
\VS{6}Car tu es un peuple saint pour Yahweh, ton Dieu. Yahweh, ton Dieu, t'a choisi pour que tu sois pour lui un peuple précieux entre tous les peuples qui sont sur la face de la terre.
\VS{7}Ce n'est pas parce que vous êtes plus nombreux que tous les peuples que Yahweh vous a aimé et qu'il vous a choisis~; car vous êtes le plus petit de tous les peuples.
\VS{8}Mais c'est parce que Yahweh vous aime, et qu'il garde le serment qu'il a juré à vos pères, Yahweh vous a fait sortir par sa main puissante, et vous a rachetés de la maison de servitude, de la main de Pharaon, roi d'Egypte.
\VS{9}Sache que c'est Yahweh, ton Dieu, qui est Dieu. Ce Dieu fidèle garde son alliance et sa miséricorde jusqu'à mille générations envers ceux qui l'aiment et qui gardent ses commandements,
\VS{10}et qui rend la pareille en face à ceux qui le haïssent, et les fait périr~; il ne diffère point envers celui qui le hait, il lui rend la pareille en face.
\VS{11}Garde les commandements, les lois, et les ordonnances que je t'ordonne aujourd'hui, et mets-les en pratique.
\TextTitle{L'obéissance à Yahweh, source de bénédictions et de victoires}
\VS{12}Et il arrivera que si vous écoutez ces ordonnances, si vous les gardez et les mettez en pratique, Yahweh, ton Dieu, gardera l'alliance et la bonté qu'il a jurées à tes pères.
\VS{13}Et il t'aimera, te bénira, et te multipliera~; il bénira le fruit de tes entrailles, et le fruit de ta terre, ton blé, ton vin, et ton huile, les portées de ton gros et de ton menu bétail, sur la terre qu'il a juré de donner à tes pères.
\VS{14}Tu seras béni plus que tous les peuples~; il n'y aura chez toi et parmi tes bêtes, ni mâle ni femelle stérile\FTNT{Ex. 23:26.}.
\VS{15}Yahweh détournera de toi toute maladie~; il ne t'enverra aucun de ces mauvais maux d'Egypte qui te sont connus, mais il les fera venir sur tous ceux qui te haïssent.
\VS{16}Tu détruiras donc tous les peuples que Yahweh, ton Dieu, va te livrer, ton œil n'aura point de pitié, et tu ne serviras point leurs dieux, car cela te serait un piège.
\VS{17}Si tu dis dans ton cœur~: Ces nations sont plus nombreuses que moi, comment pourrai-je les déposséder~?
\VS{18}Ne les crains point. Rappelle-toi bien ce que Yahweh, ton Dieu, a fait à Pharaon, et à tous les Egyptiens,
\VS{19}de ces grandes épreuves que tes yeux ont vues, les signes et les miracles, la main forte et le bras étendu par lesquels Yahweh, ton Dieu, t'a fait sortir~; ainsi fera Yahweh, ton Dieu, à tous ces peuples que tu crains.
\VS{20}Yahweh, ton Dieu, enverra contre eux les frelons, jusqu'à ce que périssent ceux qui resteront, et ceux qui se seront cachés de devant toi.
\VS{21}Ne t'effraie point devant eux, car Yahweh, ton Dieu, le Dieu grand et terrible est au milieu de toi.
\VS{22}Or Yahweh, ton Dieu, chassera peu à peu ces nations de devant toi~; tu ne pourras pas les exterminer promptement, de peur que les bêtes des champs ne se multiplient contre toi.
\VS{23}Mais Yahweh, ton Dieu, les livrera devant toi~; et il les troublera par de grandes confusions, jusqu'à ce qu'elles soient détruites.
\VS{24}Et il livrera leurs rois entre tes mains, et tu feras disparaître leurs noms de dessous les cieux~; aucun homme ne tiendra face à toi, jusqu'à ce que tu les aies détruits.
\VS{25}Tu brûleras au feu les images taillées de leurs dieux. Tu ne convoiteras point et tu ne prendras point pour toi l'argent et l'or qui seront sur elles, de peur que tu en sois pris au piège~; car c'est une abomination pour Yahweh, ton Dieu.
\VS{26}Ainsi tu n'introduiras point de choses abominables dans ta maison, afin que tu ne sois pas, comme cette chose, dévoué par interdit~; tu la détesteras fortement, et tu l'auras en abomination, car c'est une chose dévouée par interdit.
\Chap{8}
\TextTitle{Le désert, lieu de formation, d'humiliation et d’épreuve}
\VerseOne{}Vous observerez et vous mettrez en pratique tous les commandements que je vous ordonne aujourd'hui, afin que vous viviez, que vous multipliiez, et que vous entriez en possession du pays que Yahweh a juré de donner à vos pères.
\VS{2}Et souviens-toi de tout le chemin par lequel Yahweh, ton Dieu, t'a fait marcher pendant ces quarante ans dans ce désert, afin de t'humilier et de t'éprouver, pour connaître ce qui était dans ton cœur, et si tu garderais ses commandements ou non.
\VS{3}Il t'a donc humilié, il t'a laissé avoir faim, mais il t'a nourri de la manne, que tu ne connaissais pas et que tes pères n'avaient pas connue, afin de te faire connaître que l'homme ne vivra pas de pain seulement, mais que l'homme vivra de tout ce qui sort de la bouche de Yahweh\FTNT{Mt. 4:4~; Lu. 4:4.}.
\VS{4}Ton vêtement ne s'est point usé sur toi, et ton pied ne s'est point enflé durant ces quarante années\FTNT{Né. 9:21.}.
\VS{5}Reconnais dans ton cœur que Yahweh, ton Dieu, te châtie comme un homme châtie son enfant\FTNT{Hé. 12:5-12.}.
\TextTitle{Se garder d'oublier Yahweh}
\VS{6}Et garde les commandements de Yahweh, ton Dieu, pour marcher dans ses voies, et pour le craindre.
\VS{7}Car Yahweh, ton Dieu, va te faire entrer dans un bon pays, un pays de torrents d'eaux, de fontaines et d'abîmes, qui jaillissent des vallées et des montagnes~;
\VS{8}un pays de blé, d'orge, de vignes, de figuiers, et de grenadiers~; un pays d'oliviers donnant de l'huile et du miel~;
\VS{9}un pays où tu ne mangeras point le pain avec disette, où tu ne manqueras de rien~; un pays dont les pierres sont du fer, et des montagnes desquelles tu tailleras l'airain.
\VS{10}Tu mangeras et tu te rassasieras, tu béniras Yahweh, ton Dieu, pour le bon pays qu'il t'a donné.
\VS{11}Prends garde à toi de peur que tu n'oublies Yahweh, ton Dieu, en ne gardant point ses commandements, ses ordonnances, et ses lois que je t'ordonne aujourd'hui~;
\VS{12}de peur que quand tu mangeras et que tu seras rassasié~; que tu bâtiras et habiteras de belles maisons~;
\VS{13}que ton gros et menu bétail se multipliera~; que ton argent et ton or augmentera, et que tout ce qui est à toi se multipliera,
\VS{14}que ton cœur ne s'élève point et que tu n'oublies point Yahweh, ton Dieu, qui t'a fait sortir du pays d'Egypte, de la maison de servitude,
\VS{15}qui t'a fait marcher dans ce grand et affreux désert de serpents brûlants et de scorpions, dans des lieux arides et sans eau, et qui a fait jaillir pour toi de l'eau du rocher le plus dur,
\VS{16}qui t'as fait manger dans ce désert la manne que tes pères n'avaient point connue, afin de t'humilier et de t'éprouver, pour te faire ensuite du bien,
\VS{17}et que tu ne dises dans ton cœur~: Ma force et la puissance de ma main m'ont acquis ces richesses.
\VS{18}Mais tu te souviendra de Yahweh, ton Dieu, car c'est lui qui te donne de la force pour acquérir ces richesses, afin de confirmer son alliance, qu'il a jurée à tes pères, comme tu le vois aujourd'hui.
\VS{19}Mais si tu oublies Yahweh, ton Dieu, et que tu vas après d'autres dieux, si tu les sers, et que tu te prosternes devant eux, je vous avertis aujourd'hui que vous périrez certainement.
\VS{20}Vous périrez comme les nations que Yahweh fait périr devant vous, parce que vous n'aurez pas obéi à la voix de Yahweh, votre Dieu.
\Chap{9}
\TextTitle{Yahweh, fidèle à son alliance malgré la rébellion du peuple}
\VerseOne{}Ecoute, Israël~! Tu vas passer aujourd'hui le Jourdain, pour aller posséder des nations plus grandes et plus puissantes que toi, des villes grandes et fortifiées jusqu'au ciel,
\VS{2}un peuple grand et de haute taille, les fils d'Anak, que tu connais, et dont tu as entendu dire~: Qui tiendra face aux fils d'Anak~?
\VS{3}Sache donc aujourd'hui que Yahweh, ton Dieu, passera devant toi, comme un feu dévorant, c'est lui qui les détruira, qui les humiliera devant toi~; tu les chasseras, et tu les feras périr promptement, comme Yahweh te l'a dit.
\VS{4}Ne parle pas en ton cœur, quand Yahweh, ton Dieu, les chassera de devant toi, en disant~: C'est à cause de ma justice que Yahweh me fait entrer en possession de ce pays. Car c'est à cause de la méchanceté de ces nations-là que Yahweh les chasse devant toi.
\VS{5}Ce n'est point pour ta justice ni pour la droiture de ton cœur que tu entres en possession de leur pays, mais c'est pour la méchanceté de ces nations-là que Yahweh, ton Dieu, les chasse de devant toi, et pour confirmer la parole que Yahweh a jurée à tes pères, Abraham, Isaac, et Jacob.
\VS{6}Sache donc que ce n'est point pour ta justice que Yahweh, ton Dieu, te donne ce bon pays pour que tu le possèdes~; car tu es un peuple au cou raide.
\VS{7}Souviens-toi, n'oublie pas que tu as excité la colère de Yahweh, ton Dieu, dans le désert. Depuis le jour où tu es sorti du pays d'Egypte jusqu'à ce que vous arriviez dans ce lieu, vous avez été rebelles contre Yahweh.
\VS{8}Même à Horeb, vous avez excité la colère de Yahweh~; et Yahweh s'irrita contre vous, pour vous détruire.
\VS{9}Quand je montai sur la montagne, pour prendre les tables de pierre, les tables de l'alliance que Yahweh a traitée avec vous, je demeurai sur la montagne quarante jours et quarante nuits, sans manger de pain et sans boire d'eau~;
\VS{10}et Yahweh me donna les deux tables de pierre écrites du doigt de Dieu, et contenant toutes les paroles que Yahweh avait déclarées sur la montagne, du milieu du feu, le jour de l'assemblée.
\VS{11}Et il arriva qu'au bout de quarante jours et quarante nuits, Yahweh me donna les deux tables de pierre, qui sont les tables de l'alliance.
\VS{12}Puis Yahweh me dit~: Lève-toi, descends promptement d'ici~; car ton peuple, que tu as fait sortir d'Egypte, s'est corrompu. Ils se sont détournés promptement de la voie que je leur avais ordonnée, ils se sont fait une image en métal fondu.
\VS{13}Yahweh me parla, en disant~: Je vois que ce peuple est un peuple au cou raide.
\VS{14}Laisse-moi les détruire et effacer leur nom de dessous les cieux~; et je te ferai devenir une nation plus puissante et plus grande que celle-ci.
\VS{15}Je retournai et je descendis de la montagne~; or la montagne était toute en feu, et j'avais les deux tables de l'alliance dans mes deux mains.
\VS{16}Puis je regardai, et voici, vous aviez péché contre Yahweh, votre Dieu, vous vous étiez fait un veau en métal fondu, vous vous étiez détournés promptement de la voie que vous avait ordonnée Yahweh.
\VS{17}Alors je saisis les deux tables, je les jetai de mes deux mains, et je les brisai devant vos yeux.
\TextTitle{Moïse, intercède pour Israël devant Yahweh}
\VS{18}Puis je me prosternai devant Yahweh, comme auparavant, quarante jours et quarante nuits, sans manger de pain et sans boire d'eau, à cause de tout votre péché, que vous aviez commis en faisant ce qui est mal aux yeux de Yahweh, afin de l'irriter.
\VS{19}Car je craignais face à la colère et à la fureur dont Yahweh était enflammé contre vous, pour vous détruire. Et Yahweh m'exauça encore cette fois.
\VS{20}Yahweh était très irrité contre Aaron, voulant le faire périr, mais j'intercédai pour Aaron en ce temps-là.
\VS{21}Puis je pris le veau\FTNT{Le veau d'or (Ex. 32).} que vous aviez fait, votre péché, et je le brûlai au feu, je le brisai en le broyant, jusqu'à ce qu'il soit réduit en poudre, et je jetai cette poudre dans le torrent qui descend de la montagne.
\VS{22}Vous avez fort irrité la colère de Yahweh à Tabeéra, à Massa, et à Kibroth-Hattaava.
\VS{23}Et quand Yahweh vous envoya à Kadès-Barnéa, en disant~: Montez, et prenez possession du pays que je vous donne~! Vous fûtes rebelles à la parole de Yahweh, votre Dieu, vous n'eûtes point confiance, et vous n'obéîtes point à sa voix.
\VS{24}Vous avez été rebelles à Yahweh depuis le jour où je vous ai connu.
\VS{25}Je me prosternai donc devant Yahweh, je me prosternai quarante jours et quarante nuits, parce que Yahweh avait dit qu'il vous détruirait.
\VS{26}Et je priai Yahweh, et je dis~: Ô Seigneur, Yahweh, ne détruis point ton peuple, ton héritage que tu as racheté par ta grandeur, et que tu as fait sortir d'Egypte par ta main puissante.
\VS{27}Souviens-toi de tes serviteurs Abraham, Isaac, et Jacob. Ne regarde point à l'obstination de ce peuple, ni à sa méchanceté, ni à son péché,
\VS{28}de peur que le pays d'où tu nous as fait sortir ne dise~: Parce que Yahweh n'était pas capable de les conduire dans le pays qu'il leur avait promis, et parce qu'il les haïssait, il les a fait sortir pour les faire mourir dans le désert.
\VS{29}Cependant ils sont ton peuple et ton héritage, que tu as fait sortir par ta grande puissance et par ton bras étendu.
\Chap{10}
\TextTitle{Rappel du remplacement des tables de la loi}
\VerseOne{}En ce temps-là Yahweh me dit~: Taille deux tables de pierre comme les premières, et monte vers moi sur la montagne~; tu feras une arche de bois\FTNT{Ex. 25:10~; 34:1-4.}.
\VS{2}Et j'écrirai sur ces tables les paroles qui étaient sur les premières tables que tu as brisées, et tu les mettras dans l'arche.
\VS{3}Ainsi je fis une arche de bois d'acacia, je taillai deux tables de pierre comme les premières, et je montai sur la montagne, les deux tables dans ma main\FTNT{Ex. 34:4.}.
\VS{4}Et Yahweh écrivit sur ces tables ce qui avait été écrit sur les premières, les dix paroles qu'il avait dites sur la montagne, du milieu du feu, le jour de l'assemblée~; puis Yahweh me les donna.
\VS{5}Je me retournai et je descendis de la montagne~; je mis les tables dans l'arche que j'avais faite, et elles y sont demeurées, comme Yahweh me l'avait ordonné.
\VS{6}Or les enfants d'Israël partirent de Beéroth-Bené-Jaakan pour Moséra. Là mourut Aaron, et il fut enseveli~; Eléazar, son fils, exerça la prêtrise à sa place.
\VS{7}De là ils partirent pour Gudgoda, et de Gudgoda pour Jothbatha, qui est un pays de torrents d'eau.
\VS{8}Or en ce temps-là, Yahweh sépara la tribu de Lévi afin de porter l'arche de l'alliance de Yahweh, de se tenir devant Yahweh, de le servir, et de bénir en son Nom, jusqu'à ce jour.
\VS{9}C'est pourquoi Lévi n'a ni portion ni d'héritage avec ses frères~: Yahweh est son héritage, comme Yahweh, ton Dieu, lui a dit.
\VS{10}Je restai sur la montagne, comme la première fois, quarante jours et quarante nuits. Yahweh m'exauça encore cette fois~; Yahweh ne voulut point te détruire.
\VS{11}Mais Yahweh me dit~: Lève-toi, va, marche devant ce peuple. Qu'ils aillent prendre possession du pays que j'ai juré à leurs pères de leur donner.
\TextTitle{Une alliance basée sur l'amour de Yahweh}
\VS{12}Maintenant donc, ô Israël, que demande de toi Yahweh, ton Dieu, sinon que tu craignes Yahweh, ton Dieu, afin de marcher dans toutes ses voies, d'aimer et de servir Yahweh, ton Dieu, de tout ton cœur, et de toute ton âme~;
\VS{13}de garder les commandements de Yahweh et ses lois que je t'ordonne aujourd'hui, afin que tu sois heureux~?
\VS{14}Voici, les cieux, et les cieux des cieux appartiennent à Yahweh, ton Dieu, la terre et tout ce qu'elle renferme.
\VS{15}Et Yahweh s'est attaché à tes pères, pour les aimer~; et après eux, il vous a choisis, vous, leur postérité, entre tous les peuples, comme vous le voyez aujourd'hui.
\VS{16}Circoncisez donc le prépuce de votre cœur, et vous ne raidirez plus votre cou.
\VS{17}Car Yahweh, votre Dieu, est le Dieu des dieux, le Seigneur des seigneurs\FTNT{Yahweh, le Dieu des dieux et le Seigneur des seigneurs n'est autre que Jésus-Christ, notre Seigneur qui s'est révélé à Jean comme le Seigneur des seigneurs et le Roi des rois (Ap. 19:16).}, le Fort, le Grand, le Puissant et le Redoutable, qui n'a point d'égard à l'apparence des personnes, et qui ne prend point de présents~;
\VS{18}qui fait justice à l'orphelin et à la veuve, qui aime l'étranger et lui donne le pain et le vêtement.
\VS{19}Vous aimerez donc l'étranger~; car vous avez été étrangers dans le pays d'Egypte.
\VS{20}Tu craindras Yahweh, ton Dieu, tu le serviras, tu t'attacheras à lui, et tu jureras par son Nom.
\VS{21}Il est ta louange, il est ton Dieu, qui a fait pour toi des choses grandes et redoutables que tes yeux ont vues.
\VS{22}Tes pères descendirent en Egypte au nombre de soixante-dix âmes~; et maintenant Yahweh, ton Dieu, t'a fait devenir comme les étoiles des cieux, tant tu es en grand nombre.
\Chap{11}
\TextTitle{Exhortation à la reconnaissance et à l'obéissance}
\VerseOne{}Tu aimeras donc Yahweh, ton Dieu, et tu garderas toujours ses lois, ses ordonnances, et ses commandements.
\VS{2}Et reconnaissez aujourd'hui, ce que n'ont point connu ni vu vos fils, le châtiment de Yahweh, votre Dieu, sa grandeur, sa main puissante, et son bras étendu,
\VS{3}ses signes, et les œuvres qu'il a accomplies au milieu de l'Egypte contre Pharaon, roi d'Egypte, et contre tout son pays~;
\VS{4}et ce que Yahweh a fait à l'armée d'Egypte, à ses chevaux et à ses chars, quand il a fait déborder sur eux les eaux de la Mer Rouge, car Yahweh les a détruits jusqu'à ce jour\FTNT{Ex. 14:28.}~;
\VS{5}ce qu'il a fait dans le désert, jusqu'à votre arrivée en ce lieu-ci~;
\VS{6}ce qu'il a fait à Dathan et à Abiram, fils d'Eliab, fils de Ruben, comment la terre ouvrit sa bouche et les engloutit, avec leurs maisons et leurs tentes, et tous les êtres qui les suivaient, au milieu de tout Israël\FTNT{No. 16:1-33.}.
\VS{7}Car ce sont vos yeux qui ont vu toutes les grandes œuvres que Yahweh a faites.
\TextTitle{Les bienfaits de la terre promise sont pour un peuple fidèle}
\VS{8}Vous garderez donc tous les commandements que je vous ordonne aujourd'hui, afin que vous ayez la force d'entrer et de vous emparer du pays où vous allez passer pour en prendre possession,
\VS{9}et afin que vous prolongiez vos jours sur la terre que Yahweh a juré à vos pères de leur donner, ainsi qu'à leur postérité, pays où coulent le lait et le miel.
\VS{10}Car le pays où tu vas entrer afin de le posséder n'est pas comme le pays d'Egypte, d'où vous êtes sortis, où tu semais ta semence, et l'arrosais avec ton pied, comme un jardin potager.
\VS{11}Mais le pays où vous allez passer pour le posséder est un pays de montagnes et de vallées, qui boit les eaux de la pluie du ciel~;
\VS{12}c'est un pays dont Yahweh, ton Dieu, prend soin, et sur lequel Yahweh, ton Dieu, a continuellement ses yeux, du commencement de l'année jusqu'à la fin de l'année.
\VS{13}Il arrivera donc que, si vous obéissez attentivement à mes commandements que je vous ordonne aujourd'hui, si vous aimez Yahweh, votre Dieu, et que vous le servez de tout votre cœur et de toute votre âme,
\VS{14}alors je donnerai à votre pays la pluie en son temps, la pluie de la première et de l'arrière-saison, et tu recueilleras ton blé, ton vin, et ton huile.
\VS{15}Je mettrai aussi dans ton champ de l'herbe pour ton bétail, tu mangeras et tu seras rassasié.
\VS{16}Prenez garde à vous, de peur que votre cœur ne soit trompé, et que vous ne vous détourniez, et ne serviez d'autres dieux, et ne vous prosterniez devant eux.
\VS{17}Et que la colère de Yahweh s'enflamme contre vous, et qu'il ne ferme les cieux, tellement qu'il n'y aurait point de pluie, la terre ne donnerait plus son produit, et vous péririez promptement dans ce bon pays que Yahweh vous donne.
\VS{18}Mettez donc dans votre cœur et dans votre âme ces paroles. Liez-les comme un signe sur vos mains, et qu'elles soient comme des fronteaux entre vos yeux.
\VS{19}Et enseignez-les à vos enfants, en leur en parlant, quand tu seras dans ta maison, quand tu partiras en voyage, quand tu te coucheras et quand tu te lèveras.
\VS{20}Tu les écriras aussi sur les poteaux de ta maison, et sur tes portes.
\VS{21}Afin que vos jours et les jours de vos fils, sur la terre que Yahweh a juré à vos pères de leur donner, soient aussi nombreux que les jours des cieux sur la terre.
\VS{22}Car si vous gardez et si vous pratiquez tous ces commandements que je vous ordonne de faire, aimant Yahweh, votre Dieu, marchant dans toutes ses voies, et vous attachant à lui,
\VS{23}alors Yahweh chassera devant vous toutes ces nations et vous prendrez possession de nations plus grandes et plus puissantes que vous.
\VS{24}Tout lieu que foulera la plante de votre pied sera à vous\FTNT{Jos. 1:3~; 14:9.}~: Votre territoire s'étendra du désert au Liban, et du fleuve, le fleuve de l'Euphrate, jusqu'à la Mer Occidentale.
\VS{25}Aucun homme ne tiendra face à vous. Yahweh, votre Dieu, mettra, comme il vous l'a dit, la frayeur et la crainte de vous sur tout le pays où vous marcherez.
\TextTitle{La malédiction et la bénédiction}
\VS{26}Regardez, je mets aujourd'hui devant vous la bénédiction et la malédiction~:
\VS{27}La bénédiction, si vous obéissez aux commandements de Yahweh, votre Dieu, que je vous ordonne aujourd'hui~;
\VS{28}la malédiction, si vous n'obéissez point aux commandements de Yahweh, votre Dieu, et si vous vous détournez du chemin que je vous ordonne aujourd'hui, pour aller après d'autres dieux que vous ne connaissez point.
\VS{29}Et quand Yahweh, ton Dieu, t'aura fait entrer dans le pays dont tu vas prendre possession, tu prononceras alors les bénédictions, étant sur la montagne de Garizim, et les malédictions, étant sur la montagne d'Ebal.
\VS{30}Ces montagnes ne sont-elles pas de l'autre côté du Jourdain, derrière le chemin du soleil couchant, au pays des Cananéens qui demeurent dans la plaine, vis-à-vis de Guilgal, près des chênes de Moré~?
\VS{31}Car vous allez passer le Jourdain, pour entrer et prendre possession du pays que Yahweh, votre Dieu, vous donne~; vous le posséderez, et vous y habiterez.
\VS{32}Vous garderez et pratiquerez toutes les lois et les ordonnances que je mets aujourd'hui devant vous.
\Chap{12}
\TextTitle{Lois sur les sacrifices offerts au lieu où résidera le Nom de Yahweh}
\VerseOne{}Ce sont ici les lois et les ordonnances que vous garderez et pratiquerez dans le pays que Yahweh, le Dieu de vos pères, vous a donné à posséder, tout le temps que vous vivrez sur cette terre.
\VS{2}Vous détruirez, vous détruirez tous les lieux où les nations que vous allez déposséder servent leurs dieux, sur les hautes montagnes et sur les collines, et sous tout arbre verdoyant.
\VS{3}Vous démolirez aussi leurs autels, vous briserez leurs statues, vous brûlerez au feu leurs asheras, vous mettrez en pièces les images taillées de leurs dieux, et vous ferez périr leur nom de ce lieu-là.
\VS{4}Vous ne ferez pas ainsi à Yahweh, votre Dieu.
\VS{5}Mais vous le chercherez dans sa demeure, et vous irez au lieu que Yahweh, votre Dieu, aura choisi d'entre toutes vos tribus, pour y mettre son Nom.
\VS{6}Et vous y apporterez vos holocaustes, vos sacrifices, vos dîmes, vos offrandes élevées, vos vœux, vos offrandes volontaires de vos mains, et les premiers-nés de votre gros et de votre menu bétail\FTNT{Lé. 17:3-4.}.
\VS{7}Et là, vous mangerez devant Yahweh, votre Dieu, et vous vous réjouirez, vous et vos familles, de toutes les choses auxquelles vous aurez mis la main, et dans lesquelles Yahweh, votre Dieu, vous aura bénis.
\VS{8}Vous ne ferez pas comme nous faisons ici aujourd'hui, où chacun fait ce qui lui semble juste à ses yeux,
\VS{9}car vous n'êtes point encore entrés dans le lieu de repos, et dans l'héritage que Yahweh, votre Dieu, vous donne.
\VS{10}Vous passerez le Jourdain, et vous habiterez dans le pays que Yahweh, votre Dieu, vous donne en héritage~; il vous donnera du repos de tous vos ennemis qui vous entourent, et vous y habiterez en sécurité.
\VS{11}Et il y aura un lieu que Yahweh, votre Dieu, choisira pour y faire habiter son Nom. Vous y apporterez tout ce que je vous ordonne, vos holocaustes, vos sacrifices, vos dîmes, vos offrandes élevées de vos mains, et toutes offrandes de choix pour les vœux que vous aurez voués à Yahweh.
\VS{12}Et là, vous vous réjouirez devant Yahweh, votre Dieu, vous, vos fils et vos filles, vos serviteurs et vos servantes, et le Lévite qui sera dans vos portes~; car il n'a ni part ni héritage avec vous.
\VS{13}Garde-toi d'offrir tes holocaustes dans tous les lieux que tu verras~;
\VS{14}mais tu offriras tes holocaustes dans le lieu que Yahweh choisira dans l'une de tes tribus, et tu y feras tout ce que je t'ordonne.
\VS{15}Toutefois, selon le désir de ton âme, tu pourras tuer et manger de la viande dans toutes tes portes, selon la bénédiction que t'accordera Yahweh, ton Dieu~; celui qui sera impur et celui qui sera pur en mangeront, comme on mange de la gazelle et du cerf.
\VS{16}Seulement, vous ne mangerez point de sang. Tu le répandras sur la terre, comme de l'eau.
\VS{17}Tu ne pourras pas manger dans tes portes la dîme de ton blé, de ton vin, de ton huile, ni les premiers-nés de ton gros et menu bétail, ni aucune de tes offrandes en accomplissement d'un vœu, ni tes offrandes volontaires, ni les offrandes élevées de tes mains.
\VS{18}Mais tu les mangeras devant Yahweh, ton Dieu, au lieu que Yahweh, ton Dieu, choisira~; toi, ton fils, ta fille, ton serviteur et ta servante, et le Lévite qui sera dans tes portes~; et tu te réjouiras devant Yahweh, ton Dieu, de tout ce à quoi tu auras mis la main.
\VS{19}Garde-toi, tout le temps que tu vivras sur la terre, d'abandonner le Lévite.
\VS{20}Quand Yahweh, ton Dieu, aura élargi tes frontières, comme il te l'a promis, et que tu diras~: Je mangerai de la chair, parce que ton âme désirera manger de la chair, tu en mangeras selon tous les désirs de ton âme.
\VS{21}Si le lieu que Yahweh, ton Dieu, aura choisi pour y mettre son Nom, est loin de toi, alors tu tueras de ton gros et menu bétail, comme je te l'ai ordonné, et tu en mangeras dans tes portes selon tous les désirs de ton âme.
\VS{22}Tu en mangeras comme on mange de la gazelle et du cerf~; celui qui sera impur et celui qui sera pur en mangeront également.
\VS{23}Seulement, garde-toi de manger le sang, car le sang c'est l'âme~; et tu ne mangeras point l'âme avec la chair\FTNT{Lé. 7:26.}.
\VS{24}Tu n'en mangeras point~: Tu le répandras sur la terre comme de l'eau.
\VS{25}Tu n'en mangeras point, afin que tu sois heureux, toi et tes enfants après toi, parce que tu auras fait ce qui est droit aux yeux de Yahweh.
\VS{26}Mais tu prendras les choses que tu auras consacrées, qui seront à toi, et ce que tu auras voué, tu les prendras et tu viendras au lieu que Yahweh aura choisi.
\VS{27}Et tu offriras tes holocaustes, la chair et le sang, sur l'autel de Yahweh, ton Dieu~; mais le sang de tes autres sacrifices sera versé sur l'autel de Yahweh, ton Dieu, et tu en mangeras la chair.
\VS{28}Garde et écoute toutes ces paroles que je t'ordonne, afin que tu sois heureux, toi et tes enfants après toi, à jamais, en faisant ce qui est bon et droit aux yeux de Yahweh, ton Dieu.
\TextTitle{Mise en garde contre la séduction et les dieux étrangers}
\VS{29}Quand Yahweh, ton Dieu, aura exterminé de devant toi les nations que tu vas prendre en possession, que tu les auras possédées, et que tu habiteras dans leur pays,
\VS{30} prends garde à toi, de peur que tu ne sois pris au piège après elles, quand elles auront été détruites de devant toi~; et que tu ne recherches leurs dieux, en disant~: Comme ces nations-là servaient leurs dieux, je le ferai aussi tout de même.
\VS{31}Tu ne feras point ainsi à Yahweh, ton Dieu~; car elles ont fait à leurs dieux tout ce qui est en abomination et qui est odieux à Yahweh, et même ils brûlaient au feu leurs fils et leurs filles à leurs dieux.
\VS{32}Vous prendrez garde de faire tout ce que je vous commande. Vous n'y ajouterez rien, et vous n'en retrancherez rien.
\Chap{13}
\TextTitle{Eprouver les faux prophètes, ôter le méchant du milieu de l'assemblée}
\VerseOne{}S'il s'élève au milieu de toi un prophète ou un songeur de songes, qui te donne un signe ou miracle,
\VS{2}et que ce signe ou ce miracle dont il t'a parlé, arrive, et qu'il te dise~: Allons après d'autres dieux que tu ne connais point, et servons-les~!
\VS{3}Tu n'écouteras point les paroles de ce prophète ni de ce songeur de songes, car Yahweh, votre Dieu, vous met à l'épreuve pour savoir si vous aimez Yahweh, votre Dieu, de tout votre cœur et de toute votre âme.
\VS{4}Vous marcherez après Yahweh, votre Dieu, vous le craindrez~; vous garderez ses commandements, vous obéirez à sa voix, vous le servirez, et vous vous attacherez à lui.
\VS{5}Mais on fera mourir ce prophète-là ou ce songeur de songes, parce qu'il a parlé de révolte\FTNT{Le mot «~révolte~» utilisé ici, traduit le terme hébreu «~carah~» et signifie «~apostasie~». L'apostasie est une déviation progressive. C'est tout d'abord l'abandon d'une vérité reçue. Paul, l'apôtre enseigne que deux événements doivent avoir lieu avant le retour du Seigneur sur la terre~: L'apostasie et la révélation de l'homme du péché, le fils de perdition, c'est-à-dire l'Antichrist (2 Th. 2:1-3~; 2 Ti. 4:1).} contre Yahweh, votre Dieu, qui vous a fait sortir du pays d'Egypte et vous a délivrés de la maison de servitude, pour vous conduire loin de la voie que Yahweh, votre Dieu, vous a ordonné de marcher. Tu ôteras le méchant du milieu de toi.
\VS{6}Quand ton frère, fils de ta mère, ou ton fils, ou ta fille, ou ta femme bien-aimée, ou ton intime ami, qui est comme ton âme, t'incitera, en te disant en secret~: Allons, et servons d'autres dieux, que tu n'as point connus, ni tes pères,
\VS{7}d'entre les dieux des peuples qui sont autour de vous, près ou loin de toi, d'une extrémité de la terre jusqu'à l'autre,
\VS{8}tu ne t'accorderas pas avec lui, et tu ne l'écouteras point. Ton œil ne le regardera pas avec pitié, tu ne l'épargneras point, et tu ne le cacheras point.
\VS{9}Mais tu le feras mourir, tu le feras mourir\FTNT{Répétition du mot «~mourir~», voir commentaire en Ge. 2:17}~; ta main sera la première sur lui pour le mettre à mort, et ensuite la main de tout le peuple.
\VS{10}Tu le lapideras avec des pierres, et il mourra, parce qu'il a cherché à t'éloigner loin de Yahweh, ton Dieu, qui t'a fait sortir du pays d'Egypte, de la maison de servitude.
\VS{11}Afin que tout Israël entende et craigne, et que l'on ne fasse plus une action aussi méchante au milieu de toi.
\TextTitle{Jugement des villes idolâtres}
\VS{12}Si tu entends dire dans l'une des villes que Yahweh, ton Dieu, t'a données pour y habiter~:
\VS{13}Des hommes, fils de Bélial, sont sortis du milieu de toi, et ont chassé les habitants de leur ville, en disant~: Allons et servons d'autres dieux, des dieux que tu ne connais point~!
\VS{14}Tu chercheras, tu examineras, tu t'enquerras bien. Et si c'est la vérité, si la chose est établie, si cette abomination a été faite au milieu de toi,
\VS{15}tu frapperas, tu frapperas\FTNT{Répétition du mot «~frapperas~». Dans les écrits hébraïque, la répétition de mots est utilisée afin d'accentuer une action, pour appuyer un fait précis et le renforcer.} du tranchant de l'épée les habitants de cette ville, tu la dévoueras par interdit, et tu passeras le bétail au fil de l'épée.
\VS{16}Tu assembleras tout son butin au milieu de la place, et tu brûleras entièrement au feu cette ville et tout son butin, devant Yahweh, ton Dieu~: Elle sera pour toujours un monceau de ruines, sans être jamais rebâtie.
\VS{17}Rien de ce qui sera dévoué ne s'attachera à ta main, afin que Yahweh revienne de l'ardeur de sa colère, qu'il te fasse miséricorde et grâce, et qu'il te multiplie, comme il a juré à tes pères,
\VS{18}si tu obéis à la voix de Yahweh, ton Dieu, en gardant tous ses commandements que je t'ordonne aujourd'hui, et en faisant ce qui est droit aux yeux de Yahweh, ton Dieu.
\Chap{14}
\TextTitle{Israël, peuple mis en part}
\VerseOne{}Vous êtes les enfants de Yahweh, votre Dieu. Vous ne vous ferez aucune incision, et vous ne vous ferez point de place chauve entre les yeux pour aucun mort.
\VS{2}Car tu es un peuple saint pour Yahweh, ton Dieu~; et Yahweh t'a choisi pour que tu lui sois un peuple qui lui appartienne entre tous les peuples qui sont sur la face de la terre.
\TextTitle{Lois sur l'alimentation}
\VS{3}Tu ne mangeras d'aucune chose abominable.
\VS{4}Ce sont ici les bêtes que vous mangerez~: Le bœuf, la brebis et la chèvre~;
\VS{5}le cerf, la gazelle et le daim~; le bouquetin, le chevreuil, la chèvre sauvage, et le mouflon.
\VS{6}Vous mangerez donc toute bête qui a le sabot divisé, le pied fendu, et qui rumine.
\VS{7}Mais vous ne mangerez point de ceux qui ruminent seulement, ou qui ont le sabot divisé et le pied fendu seulement, comme le chameau, le lièvre et le lapin, car ils ruminent bien, mais ils n'ont pas le sabot qui est fendu~: Ils vous seront impurs.
\VS{8}Le porc aussi, car il a le sabot fendu, mais il ne rumine point~: Il vous sera impur. Vous ne mangerez point de leur chair, et vous ne toucherez point à leur cadavre.
\VS{9}Voici ce que vous mangerez de tout ce qui est dans les eaux~: Vous mangerez de tout ce qui a des nageoires et des écailles.
\VS{10}Mais vous ne mangerez point de ce qui n'a ni nageoires ni écailles~: Cela vous sera impur.
\VS{11}Vous mangerez tout oiseau pur.
\VS{12}Mais voici ceux dont vous ne mangerez point~: L'aigle, l'orfraie, l'aigle de mer~;
\VS{13}le vautour, le milan, et l'autour, selon leur espèce~;
\VS{14}le corbeau, selon son espèce~;
\VS{15}l'autruche, le hibou, la mouette, l'épervier, selon son espèce~;
\VS{16}le chat-huant, la chouette et le cygne~;
\VS{17}le cormoran, le pélican, le plongeon~;
\VS{18}la cigogne, le héron, selon leur espèce, la huppe et la chauve-souris.
\VS{19}Et tout reptile qui vole sera impur pour vous~; on n'en mangera point.
\VS{20}Mais vous mangerez de tout ce qui vole et qui est pur.
\VS{21}Vous ne mangerez aucun cadavre~; tu le donneras à l'étranger qui sera dans tes portes, et il le mangera, ou tu le vendras à un étranger~; car tu es un peuple saint pour Yahweh, ton Dieu. Tu ne feras point cuire le chevreau dans le lait de sa mère.
\TextTitle{Lois sur les dîmes\FTNT{No. 18:21-32.}}
\VS{22}Tu prendras la dîme, tu prendras la dîme\FTNT{Il est question ici de la dîme que les Hébreux consommaient chaque année.} de tout le produit de ta semence, de ce qui sortira de ton champ, chaque année.
\VS{23}Et tu mangeras devant Yahweh, ton Dieu, au lieu qu'il aura choisi pour y faire habiter son Nom, la dîme de ton blé, de ton vin et de ton huile, et les premiers-nés de ton gros et menu bétail, afin que tu apprennes à toujours craindre Yahweh, ton Dieu.
\VS{24}Mais quand le chemin sera trop long pour que tu puisses les transporter, parce que le lieu que Yahweh, ton Dieu, aura choisi pour y mettre son Nom, sera trop loin de toi, lorsque Yahweh, ton Dieu, t'aura béni,
\VS{25}alors tu l'échangeras contre de l'argent, tu serreras l'argent dans ta main, et tu iras au lieu que Yahweh, ton Dieu, aura choisi.
\VS{26}Et tu donneras l'argent contre tout ce que ton âme désirera, des bœufs, des brebis, du vin et des liqueurs fortes, tout ce que ton âme demandera, tu le mangeras devant Yahweh, ton Dieu, et tu te réjouiras, toi et ta famille.
\VS{27}Tu n'abandonneras point le Lévite qui sera dans tes portes, parce qu'il n'a ni portion ni héritage avec toi\FTNT{Ce verset fait référence à la première dîme qui devait être donnée aux Lévites (Voir commentaire en No. 18:21 et Mal. 3:1).}.
\VS{28}Au bout de trois ans, tu feras sortir toutes les dîmes de tes produits de cette année-là, et tu les déposeras dans tes portes.
\VS{29}Alors le Lévite, qui n'a ni portion ni héritage avec toi, l'étranger, l'orphelin, et la veuve qui seront dans tes portes, viendront, mangeront et se rassasieront, afin que Yahweh, ton Dieu, te bénisse dans toute l'œuvre que tu feras de tes mains.
\Chap{15}
\TextTitle{Lois sur l'année de relâche~: la justice et la bonté de Yahweh}
\VerseOne{}Tous les sept ans, tu célébreras l'année de relâche\FTNT{Ex. 21:2, Jé. 34:14.}.
\VS{2}Et c'est ici la manière de célébrer l'année de relâche. Que tout homme ayant droit d'exiger quelque chose que ce soit, qu'il puisse exiger de son prochain, donnera relâche, et ne l'exigera point de son prochain ni de son frère, quand on aura proclamé le relâche, en l'honneur de Yahweh.
\VS{3}Tu l'exigeras de l'étranger~; mais ta main relâchera tout ce qui t'appartiendra chez ton frère,
\VS{4}afin qu'il n'y ait point d'indigent chez toi, car Yahweh te bénira, te bénira\FTNT{Voir commentaire en Ge. 2:16} abondamment dans le pays que Yahweh, ton Dieu, te donnera à posséder pour héritage~;
\VS{5}pourvu que tu obéisses, que tu obéisses\FTNT{Voir commentaire en Ge. 2:16} bien à la voix de Yahweh, ton Dieu, en prenant garde de pratiquer tous ces commandements que je t'ordonne aujourd'hui.
\VS{6}Parce que Yahweh, ton Dieu, te bénira comme il te l'a promis, tu prêteras sur gage à beaucoup de nations, et tu n'emprunteras point sur gage~; tu domineras sur beaucoup de nations, et elles ne domineront point sur toi.
\VS{7}Quand un de tes frères sera indigent au milieu de toi, dans l'une de tes portes, dans le pays que Yahweh, ton Dieu, te donne, tu n'endurciras point ton cœur, et tu ne fermeras point ta main à ton frère indigent.
\VS{8}Mais tu lui ouvriras, tu lui ouvriras\FTNT{Voir commentaire en Ge. 2:16} ta main, et tu lui prêteras, lui prêteras\FTNT{Voir commentaire en Ge. 2:16} sur gage autant qu'il en aura besoin pour son indigence, dans laquelle il se trouvera.
\VS{9}Prends garde à toi, de peur que tu n'aies dans ton cœur quelque chose de Bélial, et que tu ne dises~: La septième année, l'année du relâche approche~! Et que ton œil soit méchant envers ton frère indigent, afin de ne rien lui donner et qu'il ne crie à Yahweh contre toi, et qu'il n'y ait du péché en toi.
\VS{10}Tu lui donneras, lui donneras\FTNT{Voir commentaire en Ge. 2:16} et que ton cœur ne lui donne point à regret~; car à cause de cela, Yahweh, ton Dieu, te bénira dans toutes tes œuvres, et dans tout ce à quoi tu mettras tes mains.
\VS{11}Car il y aura toujours des indigents dans le pays~; c'est pourquoi je t'ordonne, et je te dis~: Tu ouvriras, tu ouvriras\FTNT{Voir commentaire en Ge. 2:16} ta main à ton frère, à l'affligé, et à l'indigent dans ton pays.
\TextTitle{Loi sur les esclaves}
\VS{12}Quand l'un de tes frères Hébreux, homme ou femme, te sera vendu, il te servira six ans~; mais la septième année, tu le renverras libre de chez toi.
\VS{13}Et quand tu le renverras libre de chez toi, tu ne le renverras point à vide.
\VS{14}Tu chargeras, chargeras\FTNT{Voir commentaire en Ge. 2:16} de quelque chose de ton menu bétail, de ton aire, de ton pressoir, et tu lui donneras de ce que Yahweh, ton Dieu, t'aura béni.
\VS{15}Et tu te souviendras que tu as été esclave au pays d'Egypte, et que Yahweh, ton Dieu, t'en a racheté~; et c'est pour cela que je t'ordonne ceci aujourd'hui.
\VS{16}Mais s'il arrive qu'il te dise~: Je ne sortirai point de chez toi~; parce qu'il t'aime, toi et ta maison, et qu'il se trouve bien chez toi,
\VS{17}alors tu prendras un poinçon\FTNT{Ex. 21:6} et tu lui perceras l'oreille contre la porte, et il sera ton serviteur pour toujours. Tu en feras de même à ta servante.
\VS{18}Ce ne sera point, à tes yeux, dur de le renvoyer libre de chez toi, car il t'a servi six ans, ce qui est le double salaire d'un mercenaire~; et Yahweh, ton Dieu, te bénira en tout ce que tu feras.
\TextTitle{Loi sur les premiers-nés des animaux}
\VS{19}Tu consacreras à Yahweh, ton Dieu, tout premier-né mâle qui naîtra parmi ton gros et ton menu bétail. Tu ne travailleras point avec le premier-né de ton bœuf, et tu ne tondras point le premier-né de tes brebis\FTNT{Ex. 13:2.}.
\VS{20}Tu le mangeras, toi et ta famille, chaque année devant Yahweh, ton Dieu, dans le lieu que Yahweh aura choisi.
\VS{21}Mais s'il a quelque défaut, boiteux ou aveugle, ou qu'il ait quelque autre mauvais défaut, tu ne le sacrifieras point à Yahweh, ton Dieu.
\VS{22}Mais tu le mangeras dans tes portes~; celui qui sera impur et celui qui sera pur en mangeront également, comme on mange de la gazelle et du cerf.
\VS{23}Seulement, tu n'en mangeras point le sang~; mais tu le répandras sur la terre comme de l'eau.
\Chap{16}
\TextTitle{La Pâque et la fête des pains sans levain}
\VerseOne{}Observe le mois des épis, et fais la Pâque à Yahweh, ton Dieu~; car c'est au mois des épis que Yahweh, ton Dieu, t'a fait sortir, de nuit, d'Egypte\FTNT{Ex. 12:2-29.}.
\VS{2}Et tu sacrifieras la Pâque à Yahweh, ton Dieu, du gros et du menu bétail, au lieu que Yahweh choisira pour y faire habiter son Nom.
\VS{3}Tu ne mangeras point de pain levé, mais tu mangeras sept jours des pains sans levain, du pain d'affliction, parce que tu es sorti précipitamment du pays d'Egypte, afin que tous les jours de ta vie tu te souviennes du jour où tu es sorti du pays d'Egypte.
\VS{4}Et il ne se verra point de levain chez toi, sur tout le territoire de ton pays pendant sept jours\FTNT{1 Co. 5:7.}~; et aucune chair que tu sacrifieras le soir du premier jour ne restera jusqu'au matin.
\VS{5}Tu ne pourras point sacrifier la Pâque dans l'une de tes portes que Yahweh, ton Dieu, te donne~;
\VS{6}mais c'est au lieu que Yahweh, ton Dieu, choisira pour y faire habiter son Nom, que tu sacrifieras la Pâque, le soir, au coucher du soleil, moment où tu es sorti d'Egypte.
\VS{7}Tu la cuiras et tu la mangeras dans le lieu que Yahweh, ton Dieu, aura choisi. Et le matin, tu t'en retourneras et tu t'en iras dans tes tentes.
\VS{8}Pendant six jours, tu mangeras des pains sans levain~; et le septième jour, il y aura une assemblée solennelle à Yahweh, ton Dieu~: Tu ne feras aucune œuvre.
\TextTitle{La fête des semaines}
\VS{9}Tu te compteras sept semaines~; tu commenceras à compter ces sept semaines dès que la faucille sera mise dans les blés.
\VS{10}Puis tu feras la fête des semaines à Yahweh, ton Dieu, en présentant l'offrande volontaire de ta main, que tu donneras, selon que Yahweh, ton Dieu, t'aura béni.
\VS{11}Et tu te réjouiras devant Yahweh, ton Dieu, toi, ton fils et ta fille, ton serviteur et ta servante, le Lévite qui sera dans tes portes, l'étranger, l'orphelin et la veuve qui seront au milieu de toi, dans le lieu que Yahweh, ton Dieu, aura choisi pour y faire habiter son Nom.
\VS{12}Et tu te souviendras que tu as été esclave en Egypte, et tu garderas et pratiqueras ces lois.
\TextTitle{La fête des tabernacles}
\VS{13}Tu feras la fête des tabernacles pendant sept jours, après que tu auras recueilli le produit de ton aire et de ton pressoir.
\VS{14}Et tu te réjouiras à cette fête, toi, ton fils et ta fille, ton serviteur et ta servante, le Lévite, l'étranger, l'orphelin, et la veuve qui seront dans tes portes.
\VS{15}Tu célébreras la fête pendant sept jours à Yahweh, ton Dieu, dans le lieu que Yahweh aura choisi~; car Yahweh, ton Dieu, te bénira dans toute ta récolte, et dans tout le travail de tes mains, et tu vivras dans la joie.
\TextTitle{Offrandes à Yahweh selon ses moyens}
\VS{16}Trois fois l'an, tout mâle d'entre vous se présentera devant Yahweh, ton Dieu, dans le lieu qu'il aura choisi, à la fête des pains sans levain, à la fête des semaines, et à la fête des tabernacles. On ne se présentera point devant Yahweh à vide.
\VS{17}Mais chacun donnera à proportion de ce qu'il aura, selon la bénédiction de Yahweh,ton Dieu, qu'il t'aura donnée. 
\TextTitle{Des juges établis pour faire respecter la justice de Yahweh}
\VS{18}Tu t'établiras des juges et des officiers dans toutes les villes que Yahweh, ton Dieu, te donne, selon tes tribus~; et ils jugeront le peuple d'un juste jugement.
\VS{19}Tu ne te détourneras point de la justice, tu ne prêteras point attention à l'apparence des personnes, et tu ne recevras point de présents, car les présents aveuglent les yeux des sages et corrompent les paroles des justes.
\VS{20}Tu suivras fermement la justice, afin que tu vives et que tu possèdes le pays que Yahweh, ton Dieu, te donne.
\TextTitle{Prescriptions sur les cultes}
\VS{21}Tu ne planteras point d'arbre d'Asherah\FTNT{Bois ou arbre d'Asherah~: Il est question d'un objet en bois, pieu sacré ou arbre utilisé dans le culte d'Astarté, l'épouse de Baal (Ex. 34:13~; De. 7:5~; De. 12:3~; Jg. 3:7~; Jg. 6:25-30~; 1 R. 14:15-23).}, près de l'autel que tu feras à Yahweh, ton Dieu.
\VS{22}Tu ne dresseras point non plus de statue~; Yahweh, ton Dieu, hait ces choses.
\Chap{17}
\VerseOne{}Tu ne sacrifieras à Yahweh, ton Dieu, ni bœuf, ni agneau qui ait quelque défaut ou quelque chose de mauvais~; car c'est une abomination à Yahweh, ton Dieu.
\TextTitle{Punition de l'idolâtrie}
\VS{2}S'il se trouve au milieu de toi dans l'une des villes que Yahweh, ton Dieu, te donne, un homme ou une femme faisant ce qui est mal aux yeux de Yahweh, ton Dieu, en transgressant son alliance,
\VS{3}et allant servir d'autres dieux et se prosterner devant eux, devant le soleil, devant la lune, ou devant toute l'armée des cieux, ce que je n'ai pas ordonné~;
\VS{4}et que cela t'aura été rapporté, et que tu l'auras entendu, alors tu feras des recherches avec soin. Si la chose est vraie, que le fait est établi, et que cette abomination a été commise en Israël,
\VS{5}alors tu feras sortir vers tes portes cet homme ou cette femme, qui aura fait cette mauvaise action, cet homme, dis-je, ou cette femme, et tu les lapideras avec des pierres, et ils mourront.
\VS{6}On fera mourir sur la parole de deux témoins ou de trois témoins\FTNT{Mt. 18:15-17.}, celui qui doit être mis à mort~; il ne sera pas mis à mort sur la parole d'un seul témoin.
\VS{7}La main des témoins sera la première sur lui pour le faire mourir, et ensuite la main de tout le peuple. Et ainsi tu ôteras le mal du milieu de toi.
\TextTitle{Soumission aux autorités}
\VS{8}Quand une affaire te paraîtra trop difficile à juger entre meurtre et meurtre, entre cause et cause, entre plaie et plaie, qui sont des affaires de procès dans tes portes, alors tu te lèveras et tu monteras au lieu que Yahweh, ton Dieu, aura choisi.
\VS{9}Et tu iras vers les prêtres, les Lévites, et vers le juge qu'il y aura en ce temps-là, tu les consulteras, et ils te feront connaître et te déclareront la sentence du jugement.
\VS{10}Tu feras conformément à la sentence qu'ils t'auront déclarée de leur bouche dans le lieu que Yahweh aura choisi, et tu prendras garde de faire tout ce qu'ils t'enseigneront.
\VS{11}Tu feras conformément à la loi qu'ils t'auront enseignée de leur bouche et selon la sentence qu'ils t'auront prononcée~; tu ne te détourneras ni à droite ni à gauche de ce qu'ils t'auront déclaré.
\VS{12}Mais l'homme qui agira par orgueil et n'obéira pas au prêtre qui se tient là pour servir Yahweh, ton Dieu, ou au juge, cet homme mourra. Tu ôteras le mal d'Israël,
\VS{13}et tout le peuple l'entendra et craindra, et n'agira plus par orgueil.
\TextTitle{Instructions sur la royauté}
\VS{14}Quand tu seras entré dans le pays que Yahweh, ton Dieu, te donne, que tu le posséderas, que tu y demeureras, et que tu diras~: J'établirai un roi sur moi, comme toutes les nations qui sont autour de moi,
\VS{15}tu ne manqueras pas de t'établir pour roi celui que Yahweh, ton Dieu, aura choisi, tu établiras un roi du milieu de tes frères, tu ne pourras point désigner un homme étranger qui ne soit pas ton frère\FTNT{Dans sa prescience, Yahweh savait que le peuple se détournerait de ses voies et réclamerait un roi, à l'identique des nations alentour. (1 S. 8). Or depuis leur sortie d'Egypte, seul Yahweh était leur Dieu et leur Roi.}.
\VS{16}Seulement, il n'aura pas de nombreux chevaux, et il ne ramènera point le peuple en Egypte pour augmenter le nombre de chevaux~; car Yahweh vous a dit~: Vous ne retournerez plus par ce chemin.
\VS{17}Il n'aura point un grand nombre de femmes, afin que son cœur ne se détourne point~; et qu'il n'accumule point beaucoup d'argent et d'or.
\VS{18}Et dès qu'il sera assis sur le trône de son royaume, il écrira pour lui, dans un livre, une copie de cette loi, qu'il prendra des prêtres, les Lévites.
\VS{19}Il l'aura auprès de lui et la lira tous les jours de sa vie, afin qu'il apprenne à craindre Yahweh, son Dieu, à prendre garde à toutes les paroles de cette loi, et à ces ordonnances, afin de les pratiquer~;
\VS{20}afin que son cœur ne s'élève point au-dessus de ses frères, et qu'il ne se détourne point de ce commandement ni à droite ni à gauche~; afin qu'il prolonge ses jours dans son royaume, lui et ses fils, au milieu d'Israël.
\Chap{18}
\TextTitle{Héritage des Lévites et des prêtres}
\VerseOne{}Les prêtres, les Lévites, et même toute la tribu de Lévi, n'auront ni part ni héritage avec Israël~; ils mangeront les sacrifices consumés par le feu de Yahweh, et de son héritage.
\VS{2}Ils n'auront point d'héritage parmi leurs frères~: Yahweh sera leur héritage, comme il leur a dit.
\VS{3}Or c'est ici le droit que les prêtres prendront du peuple, sur ceux qui offriront un sacrifice, un bœuf ou un agneau~: On donnera au prêtre l'épaule, les mâchoires et l'estomac.
\VS{4}Tu lui donneras les prémices de ton blé, de ton vin et de ton huile, et les prémices de la toison de tes brebis.
\VS{5}Car Yahweh, ton Dieu, l'a choisi d'entre toutes les tribus, afin qu'il se tienne devant lui, et qu'il fasse le service au Nom de Yahweh, lui et ses fils, à toujours.
\VS{6}Or quand le Lévite viendra de l'une de tes portes, de tout lieu où il habite en Israël, et qu'il viendra selon tout le désir de son âme, au lieu que Yahweh aura choisi,
\VS{7}et qu'il fera le service au Nom de Yahweh, son Dieu, comme tous ses frères Lévites qui se tiennent là devant Yahweh,
\VS{8}il mangera une portion égale à la leur, outre ce qu'il aura vendu de son patrimoine.
\TextTitle{Les abominations des nations interdites en Israël}
\VS{9}Quand tu seras entré dans le pays que Yahweh, ton Dieu, te donne, tu n'apprendras point à faire les abominations de ces nations-là.
\VS{10}Qu'on ne trouve au milieu de toi personne qui fasse passer par le feu son fils ou sa fille, personne qui pratique la divination, l'astrologie, l'augure, la sorcellerie,
\VS{11}ni d'enchanteur qui use d'enchantements, personne qui consulte les médiums ou disent la bonne aventure, personne qui interroge les morts\FTNT{Yahweh interdit tout contact avec le monde des esprits et des démons. Le croyant qui accepte l'Evangile comprendra sans peine et simplement en obéissant à la Parole que ce domaine est interdit. Voir Ex. 22:18~; Lé. 19:26~; Lé. 19:31~; Lé. 20:6~; Lé. 20:27~; Es. 8:19~; 2 Ch. 33:6~; Ac. 19:13-20.}.
\VS{12}Car quiconque fait ces choses est en abomination à Yahweh~; et à cause de ces abominations, Yahweh, ton Dieu, va chasser ces nations-là devant toi.
\VS{13}Tu seras intègre avec Yahweh, ton Dieu.
\VS{14}Car ces nations, que tu vas déposséder, écoutent les pronostiqueurs et les devins~; mais à toi, Yahweh, ton Dieu, ne le permet point.
\TextTitle{Annonce sur la venue du Messie}
\VS{15}Yahweh, ton Dieu, te suscitera du milieu de toi, d'entre tes frères, un prophète comme moi\FTNT{Moïse a annoncé la venue d'un prophète comme lui, c'est-à-dire un prophète de la délivrance et de l'exode. Ce prophète n'est autre que Jésus-Christ qui nous délivre de l'emprise de Satan et nous sort du monde pour nous amener dans la Nouvelle Jérusalem (Jn. 14:2~; Col. 1:13). Notons qu'au moment de la transfiguration, Elie et Moïse parlaient avec Jésus de son départ («~exodus~» en grec~; Lu. 9:31).}: Vous l'écouterez.
\VS{16}Selon tout ce que tu as demandé à Yahweh, ton Dieu, à Horeb, le jour de l'assemblée, quand tu disais~: Que je n'entende plus la voix de Yahweh, mon Dieu, et que je ne voie plus ce grand feu, de peur de mourir.
\VS{17}Alors Yahweh me dit~: Ce qu'ils ont dit est bien.
\VS{18}Je leur susciterai un prophète comme toi du milieu de leurs frères, je mettrai mes paroles dans sa bouche, et il leur dira tout ce que je lui ordonnerai.
\VS{19}Et il arrivera que si un homme n'écoute pas mes paroles qu'il dira en mon Nom, je lui en demanderai compte.
\TextTitle{Comment éprouver les prophètes~?}
\VS{20}Mais le prophète qui agira de manière orgueilleuse pour dire en mon Nom une parole que je ne lui aurai point ordonnée de dire, ou qui parlera au nom des autres dieux, ce prophète-là mourra.
\VS{21}Et si tu dis dans ton cœur~: Comment connaîtrons-nous la parole que Yahweh n'aura point dite~?
\VS{22}Quand le prophète parlera au Nom de Yahweh, et que ce qu'il aura dit n'arrivera pas, ce sera une parole que Yahweh ne lui aura point dite. C'est par orgueil que le prophète l'a dite~: N'aie point peur de lui.
\Chap{19}
\TextTitle{Les villes de refuge\FTNTT{No. 35:1-34.}}
\VerseOne{}Quand Yahweh, ton Dieu, aura exterminé les nations dont Yahweh, ton Dieu, te donne le pays, et que tu les auras dépossédées et que tu demeureras dans leurs villes, et dans leurs maisons,
\VS{2}alors tu sépareras trois villes au milieu du pays que Yahweh, ton Dieu, te donne à posséder.
\VS{3}Tu établiras des chemins, et tu diviseras en trois le territoire de ton pays, que Yahweh, ton Dieu, te donnera en héritage. Ce sera afin que tout meurtrier s'y enfuie.
\VS{4}Or voici comment on procédera envers le meurtrier qui s'enfuira pour sauver sa vie. Celui qui aura frappé son prochain involontairement, et sans l'avoir haï dans le passé~;
\VS{5}ainsi, si quelqu'un va couper du bois dans la forêt avec une autre personne, la hache à la main pour couper du bois, si le fer glisse du manche, trouve son compagnon, et s'il en meurt~; il s'enfuira alors dans une de ces villes, afin qu'il vive.
\VS{6}De peur que celui qui venge le sang ne poursuive le meurtrier, parce que son cœur est échauffé, et qu'il ne le rattrape, si le chemin est trop long, et ne le frappe à mort, alors qu'il ne mérite pas la mort, parce qu'il ne le haïssait pas auparavant\FTNT{No. 35:1-34.}.
\VS{7}C'est pourquoi je t'ordonne, en disant~: Sépare-toi trois villes.
\VS{8}Lorsque Yahweh, ton Dieu, aura élargi tes frontières, comme il l'a juré à tes pères, et qu'il t'aura donné tout le pays qu'il a promis à tes pères de te donner,
\VS{9}parce que tu auras gardé et mis en pratique tous ces commandements que je t'ordonne aujourd'hui, en aimant Yahweh, ton Dieu, et en marchant toujours dans ses voies, alors tu ajouteras encore trois villes à ces trois-là,
\VS{10}afin que le sang innocent ne soit versé au milieu du pays que Yahweh, ton Dieu, te donne en héritage, et que tu ne sois pas coupable de meurtre.
\VS{11}Mais si un homme hait son prochain, lui dresse un piège, se lève contre lui et frappe cette personne, de sorte qu'il meure, et qu'il s'enfuit dans l'une de ces villes,
\VS{12}alors les anciens de sa ville l'enverront saisir, et le livreront entre les mains du vengeur de sang, afin qu'il meure.
\VS{13}Ton œil ne l'épargnera point, mais tu feras disparaître d'Israël le sang innocent, et tu seras heureux.
\VS{14}Tu ne déplaceras point les bornes de ton prochain, fixées par tes ancêtres, dans l'héritage que tu posséderas, dans le pays que Yahweh, ton Dieu, te donne à posséder.
\TextTitle{Résoudre des différends}
\VS{15}Un seul témoin ne sera point valable contre un homme pour constater un crime ou un péché, quel que soit le péché~; mais sur la parole de deux témoins ou de trois témoins la chose sera valable.
\VS{16}Quand un faux témoin s'élèvera contre un homme pour témoigner contre lui d'un crime,
\VS{17}ces deux hommes en contestation comparaîtront devant Yahweh, en présence des prêtres et des juges qui seront là en ce temps-là.
\VS{18}Et les juges feront des recherches avec soin. Si le témoin est un faux témoin, s'il a donné un faux témoignage contre son frère,
\VS{19}tu lui feras comme il avait pensé faire à son frère. Tu ôteras ainsi le mal du milieu de toi.
\VS{20}Et les autres entendront et craindront, et ne feront plus une chose aussi méchante au milieu de toi.
\VS{21}Ton œil ne l'épargnera point~: Vie pour vie, œil pour œil, dent pour dent, main pour main, pied pour pied.
\Chap{20}
\TextTitle{Instructions diverses pour la guerre}
\VerseOne{}Quand tu iras à la guerre contre tes ennemis, et que tu verras des chevaux et des chars, et un peuple plus grand que toi, tu ne les craindras point, car Yahweh, ton Dieu, qui t'a fait monter du pays d'Egypte, est avec toi.
\VS{2}Et quand vous vous approcherez du combat, le prêtre s'avancera et parlera au peuple.
\VS{3}Et leur dira~: Ecoute Israël~: Vous vous approchez aujourd'hui pour combattre vos ennemis. Que votre cœur ne faiblisse pas~; ne craignez point, ne soyez point effrayés et ne soyez point terrifiés face à eux.
\VS{4}Car Yahweh, votre Dieu, marche avec vous, pour combattre vos ennemis, pour vous sauver.
\VS{5}Les officiers parleront au peuple, en disant~: Qui est l'homme qui a bâti une maison neuve et ne l'a pas inaugurée~? Qu'il s'en aille et retourne dans sa maison, de peur qu'il ne meure dans la bataille et qu'un autre homme ne l'inaugure.
\VS{6}Qui est celui qui a planté une vigne et n'en a point encore cueilli le fruit~? Qu'il s'en aille et retourne dans sa maison, de peur qu'il ne meure dans la bataille et qu'un autre homme n'en cueille le fruit.
\VS{7}Qui est celui qui a fiancé une femme et ne l'a point prise en mariage~? Qu'il s'en aille et retourne dans sa maison, de peur qu'il ne meure dans la bataille et qu'un autre homme ne la prenne en mariage.
\VS{8}Et les officiers continueront à parler au peuple, et diront~: Si un homme a peur et est timide, qu'il s'en aille et retourne dans sa maison, de peur que le cœur de ses frères ne devienne craintif comme le sien.
\VS{9}Quand les officiers auront fini de parler au peuple, ils désigneront les chefs des armées à la tête du peuple.
\VS{10}Quand tu t'approcheras d'une ville pour lui faire la guerre, tu l'inviteras à la paix.
\VS{11}Et si elle te donne une réponse de paix et s'ouvre à toi, tout le peuple qui s'y trouvera te sera tributaire et te servira.
\VS{12}Si elle ne fait pas la paix avec toi et qu'elle te fait la guerre, alors tu l'assiègeras.
\VS{13}Et quand Yahweh, ton Dieu, l'aura livrée entre tes mains, tu frapperas tous les mâles au fil de l'épée.
\VS{14}Mais les femmes, les enfants, le bétail, tout ce qui sera dans la ville, et tout son butin, tu le prendras pour toi et tu mangeras le butin de tes ennemis, que Yahweh, ton Dieu, t'aura donné.
\VS{15}Tu feras ainsi à toutes les villes qui sont très éloignées de toi, et qui ne sont point des villes de ces nations.
\VS{16}Mais dans les villes de ces peuples que Yahweh, ton Dieu, te donne en héritage, tu ne laisseras vivre personne qui respire.
\VS{17}Car tu ne manqueras point de les dévouer par interdit~: Héthiens, Amoréens, Cananéens, Phéréziens, Héviens, et Jébusiens, comme Yahweh, ton Dieu, te l'a ordonné.
\VS{18}Afin qu'ils ne vous enseignent point à faire toutes les abominations qu'ils font pour leurs dieux, et que vous ne péchiez point contre Yahweh, votre Dieu.
\VS{19}Quand tu assiégeras une ville durant plusieurs jours, en lui faisant la guerre pour la saisir, tu ne détruiras point les arbres à coups de hache, tu t'en nourriras et tu ne les couperas point, car l'arbre des champs est-il un homme pour être assiégé par toi~?
\VS{20}Mais seulement tu détruiras et tu couperas les arbres que tu sauras ne point être des arbres fruitiers, et tu construiras des retranchements contre la ville qui te fait la guerre, jusqu'à ce qu'elle tombe.
\Chap{21}
\TextTitle{Lois sur le meurtre anonyme}
\VerseOne{}S'il se trouve sur la terre que Yahweh, ton Dieu, te donne à posséder, un homme tué, étendu dans un champ, sans que l'on sache qui l'a frappé,
\VS{2}tes anciens et tes juges sortiront, et ils mesureront de l'homme tué jusqu'aux villes qui sont autour.
\VS{3}Puis les anciens de la ville la plus proche de l'homme tué prendront une génisse du troupeau qui n'a pas travaillé et qui n'a point tiré au joug.
\VS{4}Et les anciens de cette ville feront descendre cette génisse vers un torrent intarissable, où on ne travaille ni ne sème~; et là, ils briseront la nuque à la génisse dans le torrent.
\VS{5}Et les prêtres, fils de Lévi, s'approcheront~; car Yahweh, ton Dieu, les a choisis pour qu'ils le servent, et qu'ils bénissent au Nom de Yahweh~; et leur bouche doit décider de toute contestation et toute blessure.
\VS{6}Et tous les anciens de cette ville, qui seront les plus proches de l'homme qui aura été tué, laveront leurs mains sur la génisse à laquelle on aura brisé la nuque dans le torrent.
\VS{7}Et prenant la parole, ils diront~: Nos mains n'ont point répandu ce sang et nos yeux ne l'ont point vu.
\VS{8}Ô Yahweh~! Sois propice à ton peuple d'Israël que tu as racheté~; ne lui impute point le sang innocent qui a été répandu au milieu de ton peuple d'Israël~; et le meurtre sera expié pour eux.
\VS{9}Et tu ôteras le sang innocent du milieu de toi, en faisant ce qui est droit aux yeux de Yahweh.
\TextTitle{Lois sur le mariage et l'héritage}
\VS{10}Quand tu iras en guerre contre tes ennemis, que Yahweh, ton Dieu, les aura livrés entre tes mains, et que tu en auras emmené des captifs,
\VS{11}si tu vois parmi les captifs une femme belle de figure, et que tu désires la prendre pour femme,
\VS{12}alors tu la conduiras à l'intérieur de ta maison, et elle rasera sa tête et fera ses ongles,
\VS{13}elle ôtera les vêtements de sa captivité, elle demeurera dans ta maison, et pleurera son père et sa mère durant un mois. Puis tu iras vers elle, tu l'épouseras, et elle sera ta femme.
\VS{14}Si il arrive qu'elle ne te plaise plus, tu la renverras où elle voudra, mais tu ne la vendras certainement pas pour de l'argent ni la traiteras en esclave, parce que tu l'auras humiliée.
\VS{15}Quand un homme, qui a deux femmes, aime l'une et hait l'autre, si celle qu'il aime et celle qu'il hait enfantent des fils, et que le fils aîné est de celle qui est haïe,
\VS{16}alors, le jour où il laissera en héritage ce qu'il aura, il ne pourra pas reconnaître comme premier-né le fils de celle qu'il aime, à la place du fils de celle qui est haïe, et qui est le premier-né.
\VS{17}Mais il reconnaîtra pour premier-né le fils de celle qui est haïe, et il lui donnera la double portion de tout ce qui s'y trouvera être à lui~; car il est le commencement de sa vigueur, le droit d'aînesse lui appartient.
\TextTitle{Le fils indocile sous la loi\FTNTT{cp. Lu. 15:11-23.}}
\VS{18}Si un homme a un fils indocile et rebelle, n'obéissant point à la voix de son père, ni à la voix de sa mère, et qui, bien qu'ils l'aient châtié, ne les écoute point,
\VS{19}alors le père et la mère le prendront et le mèneront aux anciens de sa ville, et à la porte du lieu de sa demeure.
\VS{20}Et ils diront aux anciens de sa ville~: Voici notre fils qui est indocile et rebelle, qui n'obéit point à notre voix, et qui se livre à l'excès et à l'ivrognerie.
\VS{21}Et tous les gens de la ville le lapideront avec des pierres, et il mourra. Tu ôteras le mal du milieu de toi, afin que tout Israël entende et craigne.
\VS{22}Si un homme a commis un péché digne de mort, et qu'on le fait mourir, et que tu l'aies pendu à un bois,
\VS{23}son cadavre ne passera point la nuit sur le bois~; mais tu ne manqueras point de l'ensevelir le même jour, car celui qui est pendu est malédiction de Dieu\FTNT{Ga. 3:13.}, et tu ne souilleras point la terre que Yahweh, ton Dieu, te donne en héritage.
\Chap{22}
\TextTitle{Lois sur la vie en société}
\VerseOne{}Si tu vois le bœuf ou la brebis de ton frère s'égarer, tu ne t'en cacheras point, tu ne manqueras point de les ramener à ton frère.
\VS{2}Si ton frère ne demeure point près de toi, et que tu ne le connais point, tu les recueilleras dans ta maison et il sera chez toi jusqu'à ce que ton frère les cherche~; et alors tu les lui rendras.
\VS{3}Tu feras de même pour son âne, tu feras de même pour son vêtement, et tu feras de même pour tout ce que ton frère aura perdu et que tu trouveras~; tu ne devras point t'en détourner.
\VS{4}Si tu vois l'âne de ton frère ou son bœuf tombé dans le chemin, tu ne t'en détourneras point, et tu ne manqueras point de le relever.
\VS{5}La femme ne portera point l'habit d'un homme ni l'homme ne se vêtira point d'un habit de femme~; car celui qui fait ces choses est en abomination à Yahweh, ton Dieu\FTNT{Dans ce passage, Yahweh condamne le travestisme. Cette pratique était répandue chez les Cananéens. Le travestisme consiste à adopter le comportement, les habitudes sociales et la tenue vestimentaire du sexe opposé dans le but de lui ressembler.}.
\VS{6}Si tu rencontres sur le chemin, sur un arbre ou sur la terre, un nid d'oiseaux, ayant des petits ou des œufs, et la mère couchée sur les petits ou les œufs, tu ne prendras point la mère et les petits,
\VS{7}mais tu ne manqueras point de laisser aller la mère et tu ne prendras que les petits, afin que tu sois heureux et que tu prolonges tes jours.
\VS{8}Si tu bâtis une maison neuve, tu feras un parapet tout autour de ton toit, afin que tu ne mettes point de sang sur ta maison, si quelqu'un tombait de là.
\TextTitle{Lois sur les mélanges}
\VS{9}Tu ne sèmeras point dans ta vigne diverses sortes de grains~; de peur que le tout, à savoir les grains, que tu auras semés, et le rapport de ta vigne, ne soit souillé. 
\VS{10}Tu ne laboureras point avec un âne et un bœuf ensemble.
\VS{11}Tu ne te vêtiras point d'un tissu mélangé de laine et de lin ensemble.
\VS{12}Tu te feras des franges aux quatre pans du vêtement dont tu te couvriras.
\TextTitle{Lois sur la virginité, l'adultère et la fidélité}
\VS{13}Si un homme a pris une femme et est allé vers elle, et qu'il la haïsse,
\VS{14}et qu'il lui impute des choses qui donnent l'occasion de parler d'elle et de la diffamer, en disant~: J'ai pris cette femme, et quand je me suis approché d'elle, je ne l'ai point trouvé vierge,
\VS{15}alors le père et la mère de la jeune femme prendront et produiront les signes de la virginité de la jeune femme devant les anciens de la ville, à la porte.
\VS{16}Et le père de la jeune femme dira aux anciens~: J'ai donné ma fille à cet homme pour femme, et il l'a haïe~;
\VS{17}et voici, il lui impute des choses qui lui donnent l'occasion de parler d'elle, disant~: Je n'ai point trouvé ta fille vierge. Cependant, voici les signes de la virginité de ma fille. Et ils étendront le drap devant les anciens de la ville.
\VS{18}Alors les anciens de la ville prendront le mari, et le châtieront~;
\VS{19}et parce qu'il aura répandu une mauvaise réputation sur une vierge d'Israël, ils le condamneront à une amende de cent sicles d'argent, qu'ils donneront au père de la jeune femme. Elle sera sa femme, et il ne pourra pas la répudier, tant qu'il vivra.
\VS{20}Mais si la chose est vraie, si la jeune femme ne s'est point trouvée vierge,
\VS{21}alors ils feront sortir la jeune femme à l'entrée de la maison de son père~; les gens de sa ville la lapideront de pierres et elle mourra, car elle a commis une infamie en Israël, en se prostituant dans la maison de son père. Tu ôteras le mal du milieu de toi.
\VS{22}Si l'on trouve un homme couché avec une femme mariée, ils mourront tous les deux, l'homme qui a couché avec la femme, et la femme aussi. Tu ôteras ainsi le mal d'Israël.
\VS{23}Si une jeune fille vierge est fiancée à un homme, et qu'un homme la rencontre dans la ville, et couche avec elle,
\VS{24}vous les conduirez tous deux à la porte de la ville, vous les lapiderez de pierres, et ils mourront~; la jeune fille, parce qu'elle n'a point crié étant dans la ville, et l'homme parce qu'il a humilié la femme de son prochain. Tu ôteras le mal du milieu de toi.
\VS{25}Si l'homme rencontre dans les champs la jeune fille fiancée, et que l'homme lui fait violence et couche avec elle, alors l'homme qui aura couché avec elle mourra lui seul.
\VS{26}Mais tu ne feras rien à la jeune fille~; la jeune fille n'a point commis de péché digne de mort, car c'est comme si un homme s'élevait contre son prochain et lui ôtait la vie.
\VS{27}Parce que l'ayant trouvée dans les champs, la jeune fille fiancée a pu crier, sans que personne ne l'ait délivrée.
\VS{28}Si un homme rencontre une jeune fille vierge non fiancée, lui fait violence et couche avec elle, et qu'ils soient découverts,
\VS{29}l'homme qui aura couché avec elle donnera au père de la jeune fille cinquante sicles d'argent~; et il la prendra pour femme, parce qu'il l'a humiliée, et il ne pourra point la répudier, tant qu'il vivra.
\VS{30}Un homme ne prendra point la femme de son père ni ne découvrira le pan de la robe de son père.
\Chap{23}
\TextTitle{Lois sur l'accès à l'assemblée de Yahweh}
\VerseOne{}Celui dont les testicules ont été écrasés ou l'urètre coupé n'entrera point dans l'assemblée de Yahweh.
\VS{2}Le bâtard\FTNT{Le mot bâtard, «~mamzer~» en hébreu, désigne l'enfant illégitime, celui issu de l'inceste, celui né d'une population mélangée ou d'un père Juif et d'une mère païenne, et inversement.} n'entrera point dans l'assemblée de Yahweh~; même sa dixième génération n'entrera point dans l'assemblée de Yahweh.
\VS{3}L'Ammonite et le Moabite n'entreront point dans l'assemblée de Yahweh, même leur dixième génération, à jamais,
\VS{4}parce qu'ils ne sont point venus à votre rencontre avec du pain et de l'eau, sur le chemin, lorsque vous sortiez d'Egypte, et parce qu'ils ont engagé à prix d'argent contre vous Balaam, fils de Beor, de Pethor en Mésopotamie, pour qu'il vous maudisse.
\VS{5}Mais Yahweh, ton Dieu, n'a point voulu écouter Balaam~; et Yahweh, ton Dieu, a changé la malédiction en bénédiction, parce que Yahweh, ton Dieu, t'aime.
\VS{6}Tu ne chercheras jamais, tant que tu vivras, leur paix ni leur bien.
\VS{7}Tu n'auras point en abomination l'Edomite, car il est ton frère~; tu n'auras point en abomination l'Egyptien, car tu as été étranger dans son pays~:
\VS{8}Les enfants qui leur naîtront à la troisième génération entreront dans l'assemblée de Yahweh.
\TextTitle{La sainteté et la justice dans le camp de Yahweh}
\VS{9}Quand le camp sortira contre tes ennemis, garde-toi de toute chose mauvaise.
\VS{10}S'il y a parmi vous un homme qui ne soit point pur, par suite d'un accident nocturne, il sortira hors du camp, et n'entrera point dans le camp.
\VS{11}Et sur le soir, il se lavera dans l'eau, et dès que le soleil sera couché, il rentrera dans le camp.
\VS{12}Tu auras un endroit hors du camp, et tu sortiras là dehors.
\VS{13}Tu auras un pieu parmi tes bagages, et quand tu voudras aller dehors, tu creuseras, puis tu recouvriras tes excréments.
\VS{14}Car Yahweh, ton Dieu, marche au milieu de ton camp pour te délivrer et pour livrer tes ennemis devant toi~; que tout ton camp soit saint, afin qu'il ne voie chez toi aucune chose honteuse, et qu'il ne se détourne point de toi.
\VS{15}Tu ne livreras point à son maître l'esclave qui se sera sauvé chez toi d'auprès de son maître.
\VS{16}Il demeurera avec toi, au milieu de toi, dans le lieu qu'il choisira, dans l'une de tes villes, là où bon lui semblera~: Tu ne l'opprimeras point.
\VS{17}Il n'y aura, parmi les filles d'Israël, aucune prostituée, et il n'y aura, parmi les fils d'Israël, aucun qui se prostitue.
\VS{18}Tu n'apporteras point dans la maison de Yahweh, ton Dieu, le salaire d'une prostituée, ni le prix d'un chien, pour quelque vœu que ce soit~; car tous les deux sont en abomination devant Yahweh, ton Dieu.
\VS{19}Tu n'exigeras aucun intérêt à ton frère, ni intérêt pour de l'argent, ni intérêt pour des vivres, ni intérêt pour quelque chose que ce soit que l'on prête avec intérêt.
\VS{20}Tu prêteras avec intérêt à l'étranger, mais tu ne prêteras point avec intérêt à ton frère, afin que Yahweh, ton Dieu, te bénisse dans tout ce que ta main entreprendra dans le pays où tu vas entrer en possession.
\TextTitle{Vœux faits à Yahweh}
\VS{21}Si tu fais un vœu à Yahweh, ton Dieu, tu ne tarderas point à l'accomplir, car Yahweh, ton Dieu, ne manquerait point de te le redemander, ainsi il y aurait du péché en toi.
\VS{22}Mais si tu t'abstiens de faire un vœu, il n'y aura pas de péché en toi.
\VS{23}Mais tu prendras garde de faire ce qui sortira de tes lèvres, l'offrande volontaire que tu auras vouée à Yahweh, ton Dieu, et que ta bouche aura prononcée.
\TextTitle{Lois diverses}
\VS{24}Si tu entres dans la vigne de ton prochain, tu pourras manger des raisins selon ton appétit, jusqu'à en être rassasié~; mais tu n'en mettras point dans ton vase.
\VS{25}Si tu entres dans les blés de ton prochain, tu pourras arracher des épis avec ta main~; mais tu n'agiteras point la faucille sur les blés de ton prochain.
\Chap{24}
\TextTitle{Loi sur le divorce}
\VerseOne{}Quand un homme aura pris et épousé une femme, s'il arrive qu'elle ne trouve pas grâce à ses yeux, parce qu'il aura trouvé en elle quelque chose de honteux, il lui écrira une lettre de divorce, et après la lui avoir remise en main, il la renverra de sa maison.
\VS{2}Elle sortira de sa maison, s'en ira, et elle pourra devenir la femme d'un autre homme.
\VS{3}Si ce dernier homme la hait, écrit une lettre de divorce, la lui donne dans sa main, et la renvoie de sa maison, ou que ce dernier homme qui l'a prise pour femme, meure,
\VS{4}alors son premier mari qui l'avait renvoyée ne pourra pas la reprendre pour femme après avoir été souillée, car c'est une abomination devant Yahweh, ainsi tu ne feras point pécher le pays que Yahweh, ton Dieu, te donne en héritage.
\TextTitle{Lois diverses sur l'organisation de la société}
\VS{5}Quand un homme aura nouvellement épousé une femme, il n'ira point à la guerre, et on ne lui imposera aucune charge~; il en sera libre pour sa maison pendant un an, et il réjouira la femme qu'il a prise.
\VS{6}On ne prendra point pour gage les deux meules, pas même la meule de dessus~; parce qu'on prendrait pour gage la vie.
\VS{7}Si l'on trouve un homme qui ait dérobé l'un de ses frères, l'un des enfants d'Israël, qui en ait fait son esclave ou qui l'ait vendu, ce voleur mourra. Tu ôteras le mal du milieu de toi.
\VS{8}Prends garde à la plaie de la lèpre, afin de bien observer et de faire tout ce que les prêtres, les Lévites, vous enseigneront~; vous prendrez garde de faire selon ce que je leur ai ordonné.
\VS{9}Souviens-toi de ce que Yahweh, ton Dieu, fit à Marie, en chemin, après votre sortie d'Egypte.
\TextTitle{Lois en faveur des nécessiteux}
\VS{10}Lorsque tu feras à ton prochain un prêt quelconque, tu n'entreras point dans sa maison pour prendre son gage~;
\VS{11}mais tu te tiendras dehors, et l'homme à qui tu feras le prêt t'apportera le gage dehors.
\VS{12}Si cet homme est pauvre, tu ne te coucheras point ayant encore son gage~;
\VS{13}tu ne manqueras point de lui rendre le gage dès que le soleil sera couché, afin qu'il se couche dans son vêtement et qu'il te bénisse~; et cela te sera imputé à justice devant Yahweh, ton Dieu.
\VS{14}Tu n'opprimeras point le mercenaire, le pauvre et l'indigent, d'entre tes frères, ou d'entre les étrangers qui demeurent dans ton pays, dans tes portes.
\VS{15}Tu lui donneras son salaire le jour même avant que le soleil se couche~; car il est pauvre, et son désir s'y porte. Afin qu'il ne crie point contre toi à Yahweh, et que tu ne pèches point.
\VS{16}On ne fera point mourir les pères pour les fils, et on ne fera point mourir les fils pour les pères~; mais on fera mourir chacun pour son péché.
\VS{17}Tu ne feras pas d'injustice à l'étranger ni à l'orphelin, et tu ne prendras point en gage le vêtement de la veuve.
\VS{18}Et tu te souviendras que tu as été esclave en Egypte, et que Yahweh, ton Dieu, t'a racheté de là~; c'est pourquoi je t'ordonne de faire ces choses.
\VS{19}Quand tu moissonneras dans ton champ, et que tu auras oublié une gerbe dans ton champ, tu ne retourneras point la prendre~: Elle sera pour l'étranger, pour l'orphelin et pour la veuve, afin que Yahweh, ton Dieu, te bénisse dans toute l'œuvre de tes mains.
\VS{20}Quand tu secoueras tes oliviers, tu n'y retourneras point pour cueillir ce qui reste aux branches~: Ce sera pour l'étranger, pour l'orphelin et pour la veuve.
\VS{21}Quand tu vendangeras ta vigne, tu ne grappilleras point après~: Ce sera pour l'étranger, pour l'orphelin et pour la veuve.
\VS{22}Et tu te souviendras que tu as été esclave dans le pays d'Egypte~; c'est pourquoi je t'ordonne de faire ces choses.
\Chap{25}
\TextTitle{Le juste justifié et le méchant condamné}
\VerseOne{}Quand il y aura un différend entre des hommes et qu'ils viendront en jugement afin qu'on les juge, on justifiera le juste, et on condamnera le méchant.
\VS{2}Si le méchant mérite d'être battu, le juge le fera jeter par terre et frapper en sa présence par un certain nombre de coups, selon l'exigence de son crime.
\VS{3}Il le fera battre de quarante coups, pas plus, de peur que si l'on continuait à le frapper avec plus de coups, ton frère ne soit méprisé à tes yeux.
\VS{4}Tu n'emmuselleras point ton bœuf lorsqu'il foulera le grain.
\TextTitle{Loi sur la continuité de la postérité}
\VS{5}Quand des frères demeureront ensemble, et que l'un d'entre eux mourra sans fils, alors la femme du défunt ne se mariera point dehors avec un homme qui est étranger, mais son beau-frère viendra vers elle, la prendra pour femme, et l'épousera comme son beau-frère.
\VS{6}Et le premier-né qu'elle enfantera succédera au frère mort et portera son nom, afin que son nom ne soit point effacé d'Israël.
\VS{7}Et s'il ne plaît pas à cet homme-là de prendre sa belle-sœur, alors sa belle-sœur montera à la porte vers les anciens\FTNT{Ru. 4:1-10}, et dira~: Mon beau-frère refuse de relever le nom de son frère en Israël, et ne veut point m'épouser par droit de beau-frère.
\VS{8}Alors les anciens de la ville l'appelleront et lui parleront. S'il demeure ferme, et qu'il dit~: Il ne me plaît point de la prendre,
\VS{9}alors sa belle-sœur s'approchera de lui à la vue des anciens, lui ôtera son soulier du pied, et lui crachera au visage. Et prenant la parole, elle dira~: C'est ainsi qu'on fera à l'homme qui ne bâtit point la maison de son frère.
\VS{10}Et son nom sera appelé en Israël la maison du déchaussé.
\TextTitle{L'abomination sévèrement et justement punie}
\VS{11}Quand des hommes se querelleront ensemble, l'un contre l'autre, si la femme de l'un s'approche pour délivrer son mari de la main de celui qui le frappe, et qu'étendant sa main elle saisisse ses parties intimes,
\VS{12}tu lui couperas la main, et ton œil ne l'épargnera point.
\VS{13}Tu n'auras point dans ton sac deux poids différents, un grand et un petit.
\VS{14}Il n'y aura point dans ta maison deux épha différents, un grand et un petit\FTNT{Lé. 19:35-37.}.
\VS{15}Mais tu auras un poids exact et juste, tu auras un épha exact et juste, afin que tes jours se prolongent sur la terre que Yahweh, ton Dieu, te donne.
\VS{16}Car celui qui fait ces choses, celui qui commet une injustice, est en abomination à Yahweh, ton Dieu.
\TextTitle{Yahweh confirme le sort d'Amalek}
\VS{17}Souviens-toi ce que te fit Amalek en chemin, quand vous sortiez d'Egypte\FTNT{Ex. 17:8.},
\VS{18}comment il est venu te rencontrer sur le chemin, et, sans aucune crainte de Dieu, attaqua par derrière ceux qui étaient fatigués, quand toi-même tu étais épuisé.
\VS{19}Quand Yahweh, ton Dieu, t'aura accordé du repos de tous tes ennemis qui t'entourent, dans le pays que Yahweh, ton Dieu, te donne en héritage afin que tu le possèdes, alors tu effaceras la mémoire d'Amalek de dessous les cieux~: Ne l'oublie point.
\Chap{26}
\TextTitle{La loi des prémices\FTNTT{cp. Ex. 23:16-19.}}
\VerseOne{}Quand tu seras entré dans le pays que Yahweh, ton Dieu, te donne en héritage, et quand tu le posséderas et y habiteras,
\VS{2}alors tu prendras des prémices de tous les fruits que tu retireras de la terre dans le pays que Yahweh, ton Dieu, te donne~; tu les mettras dans une corbeille, et tu iras au lieu que Yahweh, ton Dieu, choisira pour y faire habiter son Nom\FTNT{Ex. 23:16-19.}.
\VS{3}Et tu viendras vers le prêtre qui sera en ce temps-là, et tu lui diras~: Je déclare aujourd'hui à Yahweh, ton Dieu, que je suis entré dans le pays que Yahweh a juré à nos pères de nous donner.
\VS{4}Et le prêtre prendra la corbeille de ta main, et la posera devant l'autel de Yahweh, ton Dieu.
\VS{5}Puis tu prendras la parole, et tu diras devant Yahweh, ton Dieu~: Mon père était un Araméen qui périssait, il descendit en Egypte avec un petit nombre de gens, il y séjourna et il y devint une nation grande, puissante, et nombreuse.
\VS{6}Puis les Egyptiens nous maltraitèrent, nous humilièrent, et nous imposèrent une dure servitude.
\VS{7}Nous criâmes à Yahweh, le Dieu de nos pères. Yahweh entendit notre voix, et il vit notre souffrance, notre travail, et notre oppression.
\VS{8}Et Yahweh nous fit sortir d'Egypte, à main forte et à bras étendu, avec une grande frayeur, avec des signes et des miracles.
\VS{9}Et il nous a conduits dans ce lieu, et nous a donné ce pays où coulent le lait et le miel.
\VS{10}Maintenant donc voici, j'apporte les prémices des fruits de la terre que tu m'as donnée, ô Yahweh~! Tu les poseras devant Yahweh, ton Dieu, et tu te prosterneras devant Yahweh, ton Dieu.
\VS{11}Et tu te réjouiras de tout le bien que Yahweh, ton Dieu, t'aura donné, et à ta maison, toi et le Lévite, et l'étranger qui sera au milieu de toi.
\VS{12}Quand tu auras achevé de lever toute la dîme de ta récolte, la troisième année, l'année de la dîme, tu la donneras au Lévite, à l'étranger, à l'orphelin, et à la veuve~; ils en mangeront dans tes portes, et ils en seront rassasiés.
\VS{13}Tu diras en la présence de Yahweh, ton Dieu~: J'ai fait disparaître de ma maison ce qui est consacré, et je l'ai donné au Lévite, à l'étranger, à l'orphelin, et à la veuve, selon tous tes commandements que tu m'as ordonnés~; je n'ai transgressé ni oublié aucun de tes commandements.
\VS{14}Je n'en ai point mangé dans mon affliction, et je n'en ai rien fait disparaître pour un usage impur, et je n'en ai point donné pour un mort~; j'ai obéi à la voix de Yahweh, mon Dieu~; j'ai fait selon tout ce que tu m'avais ordonné.
\VS{15}Regarde de ta sainte demeure, des cieux, et bénis ton peuple d'Israël et la terre que tu nous as donnée, comme tu l'avais juré à nos pères, pays où coulent le lait et le miel.
\VS{16}Aujourd'hui, Yahweh, ton Dieu, t'ordonne de mettre en pratique ces lois et ces ordonnances~; prends garde de les faire de tout ton cœur et de toute ton âme.
\VS{17}Tu as fait promettre aujourd'hui à Yahweh qu'il sera ton Dieu, pour que tu marches dans ses voies, que tu observes ses lois, ses commandements et ses ordonnances, et que tu obéisses à sa voix.
\VS{18}Et aujourd'hui, Yahweh t'a fait promettre que tu seras un peuple précieux, comme il te l'a dit, et que tu observeras tous ses commandements,
\VS{19}pour qu'il te donne sur toutes les nations qu'il a créées la supériorité en louange, en renom, et en beauté, et pour que tu sois un peuple saint à Yahweh, ton Dieu, comme il te l'a dit.
\Chap{27}
\TextTitle{La loi gravée sur des pierres au mont Ebal}
\VerseOne{}Or Moïse et les anciens d'Israël ordonnèrent au peuple, en disant~: Gardez tous les commandements que je vous ordonne aujourd'hui.
\VS{2}Le jour où vous aurez traversé le Jourdain, pour entrer dans le pays que Yahweh, ton Dieu, te donne, tu dresseras de grandes pierres, et tu les enduiras de chaux.
\VS{3}Puis tu écriras sur elles toutes les paroles de cette loi, quand tu auras traversé le Jourdain, pour entrer dans le pays que Yahweh, ton Dieu, te donne, pays où coulent le lait et le miel, comme te l'a dit Yahweh, le Dieu de tes pères.
\VS{4}Quand donc vous aurez traversé le Jourdain, vous dresserez ces pierres-là sur le mont Ebal, selon ce que je vous ordonne aujourd'hui, et tu les enduiras de chaux.
\VS{5}Tu bâtiras aussi là un autel à Yahweh, ton Dieu~; un autel, dis-je, de pierres, sur lesquelles tu ne lèveras point le fer.
\VS{6}Tu bâtiras l'autel de Yahweh, ton Dieu, de pierres entières. Tu y offriras des holocaustes à Yahweh, ton Dieu~;
\VS{7}tu y offriras aussi des offrandes de paix\FTNT{Voir commentaire en Lé. 3:1.}, et tu mangeras là et te réjouiras devant Yahweh, ton Dieu.
\VS{8}Et tu écriras sur ces pierres toutes les paroles de cette loi, en les gravant bien distinctement.
\TextTitle{Les malédictions prononcées sur le mont Ebal}
\VS{9}Et Moïse et les prêtres, les Lévites, parlèrent à tout Israël, en disant~: Ecoute et garde le silence, Israël~! Aujourd'hui, tu es devenu le peuple de Yahweh, ton Dieu.
\VS{10}Tu obéiras à la voix de Yahweh, ton Dieu, et tu feras ses commandements et ses lois que je t'ordonne aujourd'hui.
\VS{11}Moïse ordonna au peuple ce jour-là, disant~:
\VS{12}Quand vous aurez traversé le Jourdain, Siméon, Lévi, Juda, Issacar, Joseph, et Benjamin, se tiendront sur le mont Garizim, pour bénir le peuple~;
\VS{13}et Ruben, Gad, Aser, Zabulon, Dan et Nephthali, se tiendront sur le mont Ebal, pour maudire.
\VS{14}Et les Lévites prendront la parole, et diront à haute voix à tous les hommes d'Israël~:
\VS{15}Maudit soit l'homme qui fait une image taillée ou une image en métal fondu, car c'est une abomination à Yahweh, œuvre des mains d'un artisan, et qui la met dans un lieu secret~! Et tout le peuple répondra, et dira~: Amen~!
\VS{16}Maudit soit celui qui méprise son père et sa mère~! Et tout le peuple dira~: Amen~!
\VS{17}Maudit soit celui qui déplace les bornes de son prochain~! Et tout le peuple dira~: Amen~!
\VS{18}Maudit soit celui qui égare un aveugle dans le chemin~! Et tout le peuple dira~: Amen~!
\VS{19}Maudit soit celui qui fait injustice à l'étranger, à l'orphelin, et à la veuve~! Et tout le peuple dira~: Amen~!
\VS{20}Maudit soit celui qui couche avec la femme de son père, car il découvre le pan de la robe de son père~! Et tout le peuple dira~: Amen~!
\VS{21}Maudit soit celui qui couche avec une bête~! Et tout le peuple dira~: Amen~!
\VS{22}Maudit soit celui qui couche avec sa sœur, fille de son père, ou fille de sa mère~! Et tout le peuple dira~: Amen~!
\VS{23}Maudit soit celui qui couche avec sa belle-mère~! Et tout le peuple dira~: Amen~!
\VS{24}Maudit soit celui qui frappe son prochain en secret~! Et tout le peuple dira~: Amen~!
\VS{25}Maudit soit celui qui reçoit un présent pour mettre à mort un homme, en versant le sang innocent~! Et tout le peuple dira~: Amen~!
\VS{26}Maudit soit celui qui n'accomplit point les paroles de cette loi et ne les met pas en pratique~! Et tout le peuple dira~: Amen~!
\Chap{28}
\TextTitle{Les bénédictions accompagnent l'obéissance}
\VerseOne{}Or il arrivera que si tu écoutes attentivement la voix de Yahweh, ton Dieu, et que tu prennes garde de pratiquer tous ses commandements que je t'ordonne aujourd'hui, Yahweh, ton Dieu, te donnera la supériorité sur toutes les nations de la terre.
\VS{2}Voici toutes les bénédictions qui viendront sur toi, et qui t'atteindront, quand tu obéiras à la voix de Yahweh, ton Dieu~:
\VS{3}Tu seras béni dans la ville, et tu seras aussi béni aux champs.
\VS{4}Le fruit de tes entrailles, le fruit de ta terre, le fruit de tes troupeaux, les portées de ton gros et de ton menu bétail seront bénis.
\VS{5}Ta corbeille et ta huche seront bénies.
\VS{6}Tu seras béni en entrant, et tu seras béni en sortant.
\VS{7}Yahweh fera que tes ennemis qui s'élèveront contre toi seront battus devant toi, ils sortiront contre toi par un chemin, et ils s'enfuiront devant toi par sept chemins.
\VS{8}Yahweh ordonnera à la bénédiction d'être avec toi dans tes greniers et dans tout ce à quoi tu mettras ta main~; il te bénira dans le pays que Yahweh, ton Dieu, te donne.
\VS{9}Yahweh t'établira pour lui être un peuple saint, comme il te l'a juré, quand tu garderas les commandements de Yahweh, ton Dieu, et que tu marcheras dans ses voies.
\VS{10}Et tous les peuples de la terre verront que tu es appelé du Nom de Yahweh, et ils te craindront.
\VS{11}Yahweh te fera abonder de biens dans le fruit de tes entrailles, le fruit de tes troupeaux, et le fruit de ton sol, sur la terre que Yahweh a juré à tes pères de te donner.
\VS{12}Yahweh t'ouvrira son bon trésor, les cieux, pour donner à ton pays la pluie en sa saison et pour bénir tout le travail de tes mains~; tu prêteras à beaucoup de nations, et tu n'emprunteras point.
\VS{13}Yahweh te mettra à la tête et non à la queue, tu seras toujours en haut et jamais en bas, lorsque tu obéiras aux commandements de Yahweh, ton Dieu, que je t'ordonne aujourd'hui, afin que tu prennes garde de les faire,
\VS{14}et que tu ne te détournes ni à droite ni à gauche de toutes les paroles que je t'ordonne aujourd'hui, pour aller après d'autres dieux et pour les servir.
\TextTitle{Les malédictions accompagnent la désobéissance}
\VS{15}Mais si tu n'obéis point à la voix de Yahweh, ton Dieu, pour prendre garde de pratiquer tous ses commandements et ses lois que je t'ordonne aujourd'hui, voici toutes les malédictions qui viendront sur toi, et qui t'atteindront~:
\VS{16}Tu seras maudit dans la ville, et tu seras maudit dans les champs.
\VS{17}Ta corbeille et ta huche seront maudites.
\VS{18}Le fruit de tes entrailles, le fruit de ta terre, les portées de ton gros et de ton menu bétail seront maudits.
\VS{19}Tu seras maudit à ton entrée, et tu seras maudit à ta sortie.
\VS{20}Yahweh enverra sur toi la malédiction, la confusion, et la ruine dans tout ce à quoi tu mettras ta main et que tu feras, jusqu'à ce que tu sois détruit, et que tu périsses promptement, à cause de la méchanceté de tes pratiques, par lesquelles tu m'auras abandonné.
\VS{21}Yahweh fera que la peste s'attachera à toi, jusqu'à ce qu'elle te consume sur la terre où tu vas entrer pour en prendre possession.
\VS{22}Yahweh te frappera de tuberculose, de fièvre, d'inflammation, de chaleur brûlante, de l'épée, de sécheresse et de rouille, qui te poursuivront jusqu'à ce que tu périsses.
\VS{23}Les cieux sur ta tête seront d'airain, et la terre sous toi sera de fer.
\VS{24}Yahweh te donnera pour pluie à ton pays de la poussière et de la poudre, qui descendra des cieux sur toi jusqu'à ce que tu sois détruit.
\VS{25}Yahweh fera que tu seras battu devant tes ennemis~; tu sortiras par un chemin contre eux, et tu t'enfuiras devant eux par sept chemins~; et tu seras tremblant face à tous les royaumes de la terre.
\VS{26}Ton cadavre sera la viande de tous les oiseaux des cieux et des bêtes de la terre~; et il n'y aura personne qui les effraye.
\VS{27}Yahweh te frappera de l'ulcère d'Egypte, d'hémorroïdes, de gale, et de teigne, dont tu ne pourras guérir.
\VS{28}Yahweh te frappera de folie, d'aveuglement, et d'égarement d'esprit~;
\VS{29}et tu tâtonneras en plein midi comme tâtonne un aveugle dans l'obscurité, tu ne prospéreras pas dans tes voies, et tu seras opprimé et dépouillé tous les jours, et il n'y aura personne pour venir te sauver.
\VS{30}Tu fianceras une femme, mais un autre homme couchera avec elle et la violera~; tu bâtiras une maison, mais tu ne l'habiteras point~; tu planteras une vigne, mais tu n'en jouiras point.
\VS{31}Ton bœuf sera tué sous tes yeux, et tu n'en mangeras point~; ton âne sera enlevé devant toi, et on ne te le rendra point~; tes brebis seront livrées à tes ennemis, et il n'y aura personne pour te sauver.
\VS{32}Tes fils et tes filles seront livrés à un autre peuple, tes yeux le verront, et languiront tout le jour après eux, et tu n'auras aucun pouvoir en ta main.
\VS{33}Un peuple que tu n'auras point connu mangera le fruit de ta terre et tout ton travail, et tu seras opprimé et écrasé tous les jours.
\VS{34}Tu deviendras fou à cause de ce que tu verras de tes yeux.
\VS{35}Yahweh te frappera d'un ulcère malin sur les genoux et sur les cuisses dont tu ne pourras guérir, il t'en frappera depuis la plante du pied jusqu'au sommet de ta tête.
\VS{36}Yahweh te fera marcher, toi et ton roi que tu auras établi sur toi, vers une nation que tu n'auras point connue, ni toi ni tes pères. Et là, tu serviras d'autres dieux, du bois et de la pierre.
\VS{37}Et tu seras un sujet d'étonnement, de proverbes, de railleries, parmi tous les peuples vers lesquels Yahweh t'aura emmené.
\VS{38}Tu jetteras beaucoup de semence dans ton champ, et tu recueilleras peu, car les sauterelles la consumeront.
\VS{39}Tu planteras des vignes et tu les cultiveras~; mais tu n'en boiras point le vin et tu n'en recueilleras rien, car les vers la mangeront.
\VS{40}Tu auras des oliviers sur tout le territoire~; mais tu ne t'oindras point d'huile, car tes olives perdront leurs fruits.
\VS{41}Tu engendreras des fils et des filles~; mais ils ne seront pas à toi, car ils iront en captivité.
\VS{42}Les insectes posséderont tous tes arbres et le fruit de ta terre.
\VS{43}L'étranger qui sera au milieu de toi montera toujours plus au-dessus de toi, et toi, tu descendras toujours plus bas.
\VS{44}Il te prêtera, et tu ne lui prêteras point~; il sera à la tête, et tu seras à la queue.
\VS{45}Toutes ces malédictions viendront sur toi, elles te poursuivront et t'atteindront jusqu'à ce que tu sois détruit, parce que tu n'auras pas obéi à la voix de Yahweh, ton Dieu, pour garder ses commandements et ses lois qu'il t'a ordonnés.
\VS{46}Et ces choses seront à jamais pour toi et ta postérité comme des signes et des prodiges.
\VS{47}Et parce que tu n'auras pas servi Yahweh, ton Dieu, avec joie, et de bon cœur, malgré l'abondance de toutes choses,
\VS{48}tu serviras, dans la faim, dans la soif, dans la nudité, et dans la disette de toutes choses, ton ennemi que Yahweh enverra contre toi. Il mettra un joug de fer sur ton cou, jusqu'à ce qu'il t'ait détruit.
\TextTitle{Prophétie sur l'invasion babylonienne et la dispersion d'Israël}
\VS{49}Yahweh fera lever de loin, des extrémités de la terre, une nation qui volera comme l'aigle, une nation dont tu ne comprendras pas la langue,
\VS{50}une nation au visage féroce, et qui ne soutiendra point le vieillard et n'aura point pitié pour l'enfant\FTNT{Cette prophétie s'est accomplie en 587 av. J.-C. Voir 2 R. 24-25.}.
\VS{51}Elle mangera le fruit de tes troupeaux et les fruits de ta terre, jusqu'à ce que tu sois détruit~; elle n'épargnera ni blé, ni vin, ni huile, ni portée de ton gros et de ton menu bétail, jusqu'à ce qu'elle t'ait fait périr.
\VS{52}Et elle t'assiégera dans toutes tes portes, jusqu'à ce que tombent ces hautes et fortes murailles dans lesquelles tu auras mis ta confiance dans tout ton pays~; elle t'assiégera, dis-je, dans toutes tes portes, dans tout le pays que Yahweh, ton Dieu, te donne.
\VS{53}Tu mangeras le fruit de tes entrailles, la chair de tes fils et de tes filles que Yahweh, ton Dieu, t'aura donnés, durant le siège et la détresse dont ton ennemi te serrera.
\VS{54}L'homme le plus tendre et le plus délicat d'entre vous regardera d'un œil malin son frère, sa femme bien-aimée, et le reste de ses fils qu'il a épargnés~;
\VS{55}pour ne donner à aucun d'eux de la chair de ses fils, qu'il mangera, parce qu'il ne lui restera rien du tout, à cause du siège et de la détresse dont ton ennemi te serrera dans toutes tes portes.
\VS{56}La femme la plus tendre et la plus délicate d'entre vous, qui n'a point osé mettre la plante de son pied sur la terre, par délicatesse et par mollesse, regardera d'un œil malin son mari bien-aimé, son fils, et sa fille~;
\VS{57}et le placenta qui sortira d'entre ses jambes, et les fils qu'elle enfantera~; car manquant de tout, elle les mangera secrètement, à cause du siège et de la détresse, dont ton ennemi te serrera dans toutes les villes.
\VS{58}Si tu ne prends pas garde d'observer toutes les paroles de cette loi, qui sont écrites dans ce livre, en craignant le Nom glorieux et redoutable de Yahweh, ton Dieu,
\VS{59}alors Yahweh rendra difficile tes plaies et les plaies de ta postérité, par des plaies grandes et persistantes, des maladies malignes et persistantes. 
\VS{60}Et il fera retourner sur toi toutes les maladies d'Egypte, devant lesquelles tu avais peur~; et elles s'attacheront à toi.
\VS{61}Même Yahweh fera venir sur toi toutes maladies et toutes plaies, qui ne sont point écrites dans le livre de cette loi, jusqu'à ce que tu sois détruit.
\VS{62}Et vous resterez en petit nombre, après avoir été aussi nombreux que les étoiles des cieux, parce que tu n'auras point obéi à la voix de Yahweh, ton Dieu.
\VS{63}Et il arrivera que comme Yahweh s'est réjoui sur vous, en vous faisant du bien et en vous multipliant, de même Yahweh se réjouira sur vous en vous faisant périr et en vous détruisant~; et vous serez arrachés de la terre dans laquelle vous allez entrer en possession.
\VS{64}Et Yahweh te dispersera parmi tous les peuples, d'un bout de la terre jusqu'à l'autre~; et là, tu serviras d'autres dieux que ni toi ni tes pères n'avez connus, le bois et la pierre.
\VS{65}Tu n'auras aucun repos parmi ces nations, même la plante de ton pied n'aura aucun repos. Car Yahweh te donnera un cœur tremblant, des yeux languissants, et une âme souffrante.
\VS{66}Et ta vie sera en suspens devant toi, tu trembleras la nuit et le jour, et tu ne seras point sûr de ta vie.
\VS{67}Tu diras le matin~: Qui me fera voir le soir~? Et le soir tu diras~: Qui me fera voir le matin~? A cause de l'effroi dont ton cœur sera effrayé, et à cause des choses que tu verras de tes yeux.
\VS{68}Et Yahweh te fera retourner en Egypte sur des navires, pour faire le chemin dont je t'ai dit~: Tu ne le verras plus~; et là, vous vous vendrez à vos ennemis, comme esclaves et servantes~; et il n'y aura personne pour vous acheter.
\Chap{29}
\TextTitle{Yahweh rappelle sa fidélité à Israël}
\VerseOne{}Voici les paroles de l'alliance que Yahweh ordonna à Moïse de traiter avec les enfants d'Israël au pays de Moab, outre l'alliance qu'il avait traitée avec eux à Horeb.
\VS{2}Moïse donc appela tout Israël, et leur dit~: Vous avez vu tout ce que Yahweh a fait sous vos yeux, dans le pays d'Egypte, à Pharaon, à tous ses serviteurs, et à tout son pays,
\VS{3}les grandes épreuves que tes yeux ont vues, ces signes et ces grands miracles.
\VS{4}Mais, jusqu'à ce jour, Yahweh ne vous a point donné un cœur pour connaître, ni des yeux pour voir, ni des oreilles pour entendre.
\VS{5}Je t'ai conduit pendant quarante ans par le désert~; tes vêtements ne se sont point usés, et ton soulier ne s'est point usé à ton pied.
\VS{6}Vous n'avez point mangé de pain, ni bu de vin ni de liqueur forte, afin que vous connaissiez que je suis Yahweh, votre Dieu.
\VS{7}Et vous êtes parvenus dans ce lieu~; Sihon, roi de Hesbon, et Og, roi de Basan, sont sortis à notre rencontre, pour nous combattre, et nous les avons battus.
\VS{8}Et nous avons pris leur pays, et nous l'avons donné en héritage aux Rubénites, aux Gadites, et à la demi-tribu des Manassites.
\TextTitle{Béni celui qui reste fidèle à l'alliance}
\VS{9}Vous garderez donc les paroles de cette alliance, et vous les pratiquerez, afin de réussir dans tout ce que vous ferez.
\VS{10}Vous vous tiendrez aujourd'hui devant Yahweh, votre Dieu, vos chefs de tribus, vos anciens, vos officiers, tous les hommes d'Israël,
\VS{11}vos enfants, vos femmes, et l'étranger qui est au milieu de ton camp, depuis celui qui coupe ton bois jusqu'à celui qui puise ton eau~;
\VS{12}afin que tu entres dans l'alliance de Yahweh, ton Dieu, dans ce serment, que Yahweh, ton Dieu, traite aujourd'hui avec toi,
\VS{13}afin qu'il t'établisse aujourd'hui pour son peuple et qu'il soit ton Dieu, comme il te l'a dit, et comme il l'a juré à tes pères, Abraham, Isaac et Jacob.
\VS{14}Et ce n'est pas seulement avec vous que je traite cette alliance, ce serment.
\VS{15}Mais c'est avec ceux qui sont ici, avec nous aujourd'hui devant Yahweh, notre Dieu, et avec ceux qui ne sont point ici, avec nous aujourd'hui.
\TextTitle{Mise en garde contre celui qui abandonne l'alliance}
\VS{16}Car vous savez comment nous avons habité dans le pays d'Egypte, et comment nous sommes passés au milieu des nations, que vous avez traversées.
\VS{17}Et vous avez vu leurs abominations et leurs idoles, le bois et la pierre, l'argent et l'or qui sont parmi eux.
\VS{18}Qu'il n'y ait parmi vous ni homme, ni femme, ni famille, ni tribu qui détourne son cœur aujourd'hui de Yahweh, notre Dieu, pour aller servir les dieux de ces nations. Qu'il n'y ait parmi vous de racine qui produise du poison et de l'absinthe.
\VS{19}Et qu'il n'arrive que quelqu'un en entendant les paroles de cette malédiction, ne se bénisse dans son cœur, en disant~: J'aurai la paix, même si je marche dans les penchants de mon cœur, et que j'ajoute l'ivresse à la soif.
\VS{20}Yahweh ne voudra point lui pardonner. Mais la colère de Yahweh et la jalousie s'enflammeront contre cet homme, et toutes les malédictions écrites dans ce livre reposeront sur lui, et Yahweh effacera son nom de dessous les cieux.
\VS{21}Et Yahweh le séparera de toutes les tribus d'Israël, pour son malheur, selon toutes les malédictions de l'alliance écrite dans ce livre de la loi.
\VS{22}Et la génération à venir, vos fils qui se lèveront après vous, et l'étranger qui viendra d'un pays lointain, quand ils verront les plaies et les maladies, dont Yahweh aura frappé ce pays~;
\VS{23}et que toute la terre de ce pays-là ne sera que soufre, que sel, et qu'embrasement, qu'elle ne sera point semée, et qu'elle ne fera rien germer, et que nulle herbe n'en sortira, ainsi qu'en la subversion de Sodome, et de Gomorrhe, et d'Adma, et de Tseboïm, que Yahweh détruisit dans sa colère et dans sa fureur.
\VS{24}Mais toutes les nations diront~: Pourquoi Yahweh a-t-il traité ainsi ce pays~? D'où vient l'ardeur de cette grande colère~?
\VS{25}Et on répondra~: C'est parce qu'ils ont abandonné l'alliance de Yahweh, le Dieu de leurs pères, qu'il a traitée avec eux quand il les fit sortir du pays d'Egypte~;
\VS{26}car ils sont allés servir d'autres dieux et se sont prosternés devant eux~; des dieux qu'ils ne connaissaient point et qu'il ne leur avait point donnés en partage.
\VS{27}A cause de cela, la colère de Yahweh s'est enflammée contre ce pays, et il a fait venir sur lui toutes les malédictions écrites dans ce livre.
\VS{28}Et Yahweh les a arrachés de leur terre avec colère, avec fureur, avec une grande indignation, et il les a chassés sur un autre pays, comme on le voit aujourd'hui.
\VS{29}Les choses cachées sont à Yahweh, notre Dieu~; les choses révélées sont à nous et à nos fils, à jamais, afin que nous pratiquions toutes les paroles de cette loi.
\Chap{30}
\TextTitle{Yahweh bénira et restaurera le peuple repentant}
\VerseOne{}Or il arrivera que lorsque toutes ces choses seront venues sur toi, la bénédiction et la malédiction, que je mets devant toi, si tu les rappelles dans ton cœur, parmi toutes les nations vers lesquelles Yahweh, ton Dieu, t'aura chassé~;
\VS{2}si tu reviens à Yahweh, ton Dieu, et si tu obéis à sa voix de tout ton cœur, de toute ton âme, toi et tes fils, selon tout ce que je t'ordonne aujourd'hui,
\VS{3}Yahweh, ton Dieu, ramènera tes captifs et aura compassion de toi~; il te rassemblera encore du milieu de tous les peuples parmi lesquels Yahweh, ton Dieu, t'aura dispersé.
\VS{4}Quand tu seras dispersé à l'extrémité des cieux, Yahweh, ton Dieu, te rassemblera de là, et de là, il te prendra.
\VS{5}Yahweh, ton Dieu, dis-je, te ramènera dans le pays que tes pères possédaient, et tu le posséderas~; il te fera du bien, et te rendra plus nombreux que tes pères.
\VS{6}Yahweh, ton Dieu, circoncira ton cœur, et le cœur de ta postérité, pour que tu aimes Yahweh, ton Dieu, de tout ton cœur, et de toute ton âme, afin que tu vives\FTNT{Ro. 2:29.}.
\VS{7}Et Yahweh, ton Dieu, mettra toutes ces malédictions sur tes ennemis, et sur ceux qui te haïront et te persécuteront.
\VS{8}Ainsi tu retourneras à Yahweh, tu obéiras à sa voix, et tu feras tous ses commandements que je t'ordonne aujourd'hui.
\TextTitle{Faire connaître la loi aux futures générations}
\VS{9}Et Yahweh, ton Dieu, te fera abonder en bien dans toute l'œuvre de ta main, dans le fruit de tes entrailles, dans le fruit de tes troupeaux et dans le fruit de ta terre~; car Yahweh se réjouira de nouveau de ton bonheur, comme il s'est réjoui de celui de tes pères,
\VS{10}quand tu obéiras à la voix de Yahweh, ton Dieu, en gardant ses commandements et ses ordonnances écrites dans ce livre de la loi, quand tu reviendras à Yahweh, ton Dieu, de tout ton cœur et de toute ton âme.
\TextTitle{Le peuple devant un choix}
\VS{11}Car ce commandement que je t'ordonne aujourd'hui n'est pas trop difficile pour toi et hors de ta portée.
\VS{12}Il n'est pas aux cieux, pour dire~: Qui montera pour nous aux cieux, nous l'apportera et nous le fera entendre, pour que nous le fassions~?
\VS{13}Il n'est point aussi de l'autre côté de la mer pour dire~: Qui passera de l'autre côté de la mer pour nous, et nous l'apportera, et nous le fera entendre pour que nous le fassions~?
\VS{14}Car cette parole est fort près de toi, dans ta bouche et dans ton cœur, afin que tu la pratiques\FTNT{Ro. 10:6.}.
\VS{15}Regarde, je mets aujourd'hui devant toi la vie et le bien, la mort et le mal.
\VS{16}Car je t'ordonne aujourd'hui d'aimer Yahweh, ton Dieu, de marcher dans ses voies, de garder ses commandements, ses lois, et ses ordonnances, afin que tu vives, que tu multiplies, et que Yahweh, ton Dieu, te bénisse dans le pays où tu vas entrer en possession.
\VS{17}Mais si ton cœur se détourne, si tu n'obéis point, et si tu te laisses entraîner à te prosterner devant d'autres dieux et à les servir,
\VS{18}je vous déclare aujourd'hui que vous périrez certainement, et que vous ne prolongerez point vos jours sur la terre dont vous allez entrer en possession, après avoir passé le Jourdain.
\VS{19}J'en prends aujourd'hui à témoin les cieux et la terre contre vous~: J'ai mis devant toi la vie et la mort, la bénédiction et la malédiction. Choisis donc la vie\FTNT{cp. Mt. 7:13-14.}, afin que tu vives, toi et ta postérité.
\VS{20}en aimant Yahweh, ton Dieu, en obéissant à sa voix, et en t'attachant à lui~: Car c'est lui qui est ta vie et la longueur de tes jours, afin que tu demeures sur la terre que Yahweh a juré à tes pères, Abraham, Isaac, et Jacob, de leur donner.
\Chap{31}
\TextTitle{Moïse encourage et affermit le peuple}
\VerseOne{}Puis Moïse s'en alla, et dit ces paroles à tout Israël~:
\VS{2}Aujourd'hui, leur dit-il, je suis âgé de cent vingt ans, je ne pourrai plus sortir ni entrer, et Yahweh m'a dit~: Tu ne passeras point ce Jourdain.
\VS{3}Yahweh, ton Dieu, passera lui-même devant toi, il détruira ces nations devant toi, et tu les posséderas. Josué passera aussi devant toi, comme Yahweh l'a dit.
\VS{4}Et Yahweh leur fera comme il a fait à Sihon et à Og, rois des Amoréens, qu'il a détruits avec leurs pays.
\VS{5}Et Yahweh les livrera devant vous, et vous leur ferez selon tout le commandement que je vous ai ordonné.
\VS{6}Fortifiez-vous donc et prenez courage~! Ne craignez point et ne soyez point effrayés devant eux~; car Yahweh, ton Dieu, marchera avec toi, il ne te délaissera point et ne t'abandonnera point.
\VS{7}Et Moïse appela Josué, et lui dit en présence de tout Israël~: Fortifie-toi et prends courage, car tu entreras avec ce peuple dans le pays que Yahweh a juré à leurs pères de leur donner, et c'est toi qui les en mettras en possession.
\VS{8}Yahweh est celui qui marchera devant toi, il sera lui-même avec toi, il ne te délaissera point, il ne t'abandonnera point~; ne crains point, et ne t'effraie point.
\VS{9}Or Moïse écrivit cette loi, et il la donna aux prêtres, fils de Lévi, qui portaient l'arche de l'alliance de Yahweh, et à tous les anciens d'Israël.
\VS{10}Moïse leur ordonna, en disant~: Tous les sept ans, au temps fixé de l'année du relâche, à la fête des tabernacles,
\VS{11}quand tout Israël viendra se présenter devant Yahweh, ton Dieu, dans le lieu qu'il aura choisi, tu liras alors cette loi devant tout Israël, à leurs oreilles.
\VS{12}Tu rassembleras le peuple, les hommes, les femmes, les enfants et l'étranger qui sera dans tes portes, pour qu'ils t'entendent, et qu'ils apprennent à craindre Yahweh, votre Dieu, et qu'ils prennent garde de faire toutes les paroles de cette loi.
\VS{13}Et leurs fils qui ne la connaîtront point l'entendront, et ils apprendront à craindre Yahweh, votre Dieu, tous les jours que vous vivrez sur cette terre que vous allez posséder après avoir passé le Jourdain.
\TextTitle{Yahweh annonce les événements à venir}
\VS{14}Alors Yahweh dit à Moïse~: Voici, le jour où tu vas mourir est proche. Appelle Josué, et tenez-vous dans la tente d'assignation. Je lui donnerai mes ordres. Moïse et Josué allèrent et se présentèrent dans la tente d'assignation.
\VS{15}Et Yahweh apparut dans la tente, dans une colonne de nuée~; et la colonne de nuée s'arrêta à l'entrée de la tente.
\VS{16}Yahweh dit à Moïse~: Voici, tu vas te coucher avec tes pères. Et ce peuple se lèvera et se prostituera après les dieux étrangers du pays au milieu duquel il va entrer. Il m'abandonnera et violera mon alliance que j'ai traitée avec lui.
\VS{17}En ce jour-là ma colère s'enflammera contre lui. Je les abandonnerai, et je leur cacherai ma face. Il sera dévoré, une multitude de maux et d'angoisses l'atteindront, et il dira en ce jour-là~: N'est-ce pas parce que mon Dieu n'est point au milieu de moi, que ces maux m'ont atteint~?
\VS{18}En ce jour-là, dis-je, je cacherai entièrement ma face, à cause de tout le mal qu'il aura fait, parce qu'il se sera tourné vers d'autres dieux.
\VS{19}Maintenant donc, écrivez ce cantique. Enseigne-le aux enfants d'Israël, mets-le dans leur bouche, afin que ce cantique me serve de témoignage contre les fils d'Israël.
\VS{20}Car je le conduirai sur la terre que j'ai juré à ses pères, où coulent le lait et le miel~; il mangera, se rassasiera, et s'engraissera~; puis il se tournera vers d'autres dieux, et il les servira, il m'irritera par mépris et violera mon alliance~;
\VS{21}et il arrivera qu'il sera atteint par une multitude de maux et d'angoisses, ce cantique, qui ne sera point oublié et qui sera dans la bouche de la postérité, répondra comme témoin contre eux. Je connais ses desseins, qu'il a déjà préparés aujourd'hui, avant même que je l'aie fait entrer dans le pays que j'ai juré.
\VS{22}Ainsi Moïse écrivit ce cantique en ce jour-là, et l'enseigna aux enfants d'Israël.
\VS{23}Et Yahweh commanda à Josué, fils de Nun, en disant~: Fortifie-toi et prends courage, car c'est toi qui feras entrer les enfants d'Israël dans le pays que je leur ai juré~; et je serai avec toi.
\VS{24}Et il arrivera que quand Moïse eut achevé d'écrire dans un livre les paroles de cette loi jusqu'à ce qu'elle soit complète,
\VS{25}Moïse ordonna aux Lévites qui portaient l'arche de l'alliance de Yahweh, en disant~:
\VS{26}Prenez ce livre de la loi, et mettez-le à côté de l'arche de l'alliance de Yahweh, votre Dieu, et il sera là comme témoin contre toi.
\VS{27}Car je connais ta rébellion et ton cou raide. Voici, déjà aujourd'hui étant en vie avec vous, vous avez été rebelles contre Yahweh, combien plus le serez-vous après ma mort~?
\VS{28}Faites assembler devant moi tous les anciens de vos tribus, et vos officiers, et je dirai ces paroles en leur présence, et j'appellerai à témoin contre eux les cieux et la terre.
\VS{29}Car je sais qu'après ma mort vous vous corromprez, et que vous vous détournerez de la voie que je vous ai ordonnée~; mais à la fin, le malheur vous atteindra, parce que vous aurez fait ce qui déplaît aux yeux de Yahweh, en l'irritant par les œuvres de vos mains.
\VS{30}Ainsi Moïse prononça entièrement les paroles de ce cantique-ci, en présence de toute l'assemblée d'Israël.
\Chap{32}
\TextTitle{Cantique de Moïse}
\VerseOne{}Cieux~! Prêtez l'oreille, et je parlerai. Terre~! écoute les paroles de ma bouche.
\VS{2}Que mon enseignement tombe comme la pluie, que ma parole se répande comme la rosée, comme une pluie fine sur l'herbe naissante, et comme une averse sur la verdure~!
\VS{3}Car j'invoquerai le Nom de Yahweh~; attribuez la grandeur à notre Dieu.
\VS{4}L'œuvre du rocher\FTNT{Voir commentaire en Es. 8:13-17.} est parfaite, car toutes ses voies sont justes. C'est un Dieu fidèle et sans iniquité, il est juste et droit.
\VS{5}Ils se sont corrompus, à lui n'est point la faute~; la faute est à ses fils, c'est une génération fausse et tortueuse.
\TextTitle{Israël, le choix de Yahweh}
\VS{6}Est-ce ainsi que tu récompenses Yahweh, peuple insensé et dépourvu de sagesse~? N'est-il pas ton père, celui qui t'a acquis~? Il t'a fait et t'a façonné.
\VS{7}Souviens-toi des anciens jours, considère les années, de génération en génération, interroge ton père, et il te l'apprendra, et tes anciens, et ils te le diront.
\VS{8}Quand le Très-Haut laissa un héritage aux nations, quand il sépara les enfants des hommes, il fixa les limites des peuples selon le nombre des fils d'Israël~;
\VS{9}car la portion de Yahweh, c'est son peuple, Jacob est le lot de son héritage.
\VS{10}Il l'a trouvé dans un pays désert, dans la désolation des hurlements d'une solitude, il l'a entouré, il l'a dirigé, il l'a gardé comme la prunelle de son œil,
\VS{11}comme l'aigle éveille sa nichée, couve ses petits, étend ses ailes, les prend, les porte sur ses ailes.
\VS{12}Yahweh seul l'a conduit, et il n'y a point eu avec lui de dieu étranger.
\VS{13}Il l'a fait monter à cheval sur les hauteurs du pays, et il a mangé les fruits des champs~; il lui a donné à sucer le miel du rocher, l'huile du rocher le plus dur,
\VS{14}la crème des vaches, le lait des brebis, et la graisse des agneaux, des béliers de Basan, et des boucs, et la fleur du froment~; et tu as bu le vin qui était le sang de la grappe.
\TextTitle{Condamnation de l'apostasie d'Israël}
\VS{15}Jeshurun\FTNT{Littéralement «~Jeshurun~» en hébreu~: «~celui qui est droit~». Nom symbolique donné à Israël pour décrire son caractère idéal.} s'est engraissé, et a regimbé~; tu es devenu gras, gros et épais~! Et il a abandonné Dieu qui l'a fait, et il a méprisé le rocher de son salut.
\VS{16}Ils ont provoqué sa jalousie par des dieux étrangers, ils l'ont irrité par des abominations.
\VS{17}Ils ont sacrifié à des démons, qui ne sont point Dieu~; aux dieux qu'ils ne connaissaient point, dieux nouveaux, venus depuis peu, et que vos pères n'ont point redoutés.
\VS{18}Tu as oublié le rocher qui t'a engendré, et tu as oublié le Dieu qui t'a fait naître.
\VS{19}Yahweh l'a vu, et a été irrité, parce que ses fils et ses filles l'ont provoqué à la colère.
\VS{20}Et il a dit~: Je cacherai ma face, je verrai quelle sera leur fin~; car ils sont une génération perverse, des fils infidèles.
\VS{21}Ils ont excité ma jalousie par ce qui n'est point Dieu, ils m'ont irrité par leurs vanités~; ainsi je provoquerai leur jalousie par ce qui n'est point un peuple, et je les offenserai par une nation insensée.
\VS{22}Car le feu de ma colère s'est allumé, et brûlera jusqu'au fond du scheol, dévorera la terre et son fruit, et embrasera les fondements des montagnes.
\VS{23}Je rassemblerai sur eux des maux, et je détruirai toutes mes flèches sur eux.
\VS{24}Ils seront consumés par la famine, rongés par des charbons ardents, et par une destruction amère~; j'enverrai contre eux la dent des bêtes et le venin des serpents qui rampent sur la poussière.
\VS{25}L'épée venant de dehors les privera les uns des autres~; et au-dedans, la terreur les privera d'enfants. Il en sera du jeune homme comme de la vierge, de l'enfant à la mamelle comme de l'homme aux cheveux blancs.
\TextTitle{A Yahweh la vengeance et la rétribution}
\VS{26}Je dirais~: Je les détruirai, et je ferai disparaître leur mémoire d'entre les hommes~!
\VS{27}Si je ne craignais la colère de l'ennemi, de peur que leurs adversaires ne se méprennent, et ne disent~: Notre main est élevée, et ce n'est pas Yahweh qui a fait tout ceci.
\VS{28}Car c'est une nation qui se perd par ses conseils, et il n'y a en eux aucune intelligence.
\VS{29}Ô s'ils étaient sages, ils comprendraient ceci, et ils considéreraient leur fin.
\VS{30}Comment un seul en poursuivrait-il mille, et deux en mettraient-ils dix mille en fuite, si ce n'était que leur Rocher les avait vendus, et que Yahweh ne les avait enserrés~?
\VS{31}Car leur rocher n'est pas comme notre Rocher, nos ennemis en sont juges.
\VS{32}Car leur vigne est du plant de Sodome, et du terroir de Gomorrhe~; leurs raisins sont des raisins empoisonnés, leurs grappes sont amères.
\VS{33}Leur vin est un venin de dragon, et du poison cruel d'aspic.
\VS{34}Cela n'est-il pas caché près de moi, scellé dans mes trésors~?
\VS{35}A moi la vengeance et la rétribution, le temps où leur pied glissera~! Car le jour de leur calamité est près, et les choses qui doivent leur arriver se hâtent.
\VS{36}Mais Yahweh jugera son peuple~; et il se repentira en faveur de ses serviteurs, quand il verra que leur force a disparu, et qu'il n'y a personne de retenu ni d'abandonné.
\VS{37}Et il dira~: Où sont leurs dieux, le rocher en qui ils se confiaient,
\VS{38}qui mangeaient la graisse de leurs sacrifices, qui buvaient le vin de leurs libations~? Qu'ils se lèvent, qu'ils vous aident, et qu'ils vous servent de refuge~!
\VS{39}Voyez maintenant que moi,JE SUIS\FTNT{«~JE SUIS. Il s'agit ici du Nom que Dieu a révélé à Moïse et à Esaïe (Ex. 3:14) et sous lequel Jésus s'est présenté (Jn. 18:5-8).}, et il n'y a point de dieu avec moi\FTNT{Ce verset confirme que Dieu est un puisqu'il n'y a pas d'autres dieux à ses côtés.}~; je fais mourir et je fais vivre, je blesse et je guéris~; et il n'y a personne qui puisse délivrer de ma main.
\VS{40}Car je lève ma main au ciel, et je dis~: Je vis éternellement.
\VS{41}Si j'aiguise l'éclair de mon épée, et si ma main saisit la justice, je rendrai la vengeance à mes adversaires et je rétribuerai ceux qui me haïssent.
\VS{42} J'enivrerai mes flèches de sang et mon épée dévorera la chair, j'enivrerai, dis-je, mes flèches du sang des tués et des captifs, de la tête des chefs de l'ennemi.
\VS{43}Nations, réjouissez-vous avec son peuple~! Car il venge le sang de ses serviteurs, il tire vengeance de ses ennemis, et fait propitiation pour sa terre et pour son peuple.
\TextTitle{Fin du cantique, invitation à demeurer fidèle}
\VS{44}Moïse donc vint et prononça toutes les paroles de ce cantique, à l'oreille du peuple, lui et Josué, fils de Nun.
\VS{45}Et quand Moïse eut achevé de prononcer toutes ces paroles à tout Israël,
\VS{46}il leur dit~: Appliquez votre cœur à toutes ces paroles que je vous conjure aujourd'hui d'ordonner à vos fils, afin qu'ils prennent garde de faire toutes les paroles de cette loi.
\VS{47}Car ce n'est pas une parole vaine pour vous, mais c'est votre vie~; et par cette parole vous prolongerez vos jours sur la terre que vous posséderez, après avoir passé le Jourdain.
\TextTitle{Moïse, invité à monter sur le mont Nebo}
\VS{48}En ce même jour-là, Yahweh parla à Moïse, en disant~:
\VS{49}Monte sur cette montagne d'Abarim, sur le mont Nebo, au pays de Moab, vis-à-vis de Jéricho~; et regarde le pays de Canaan, que je donne en possession aux enfants d'Israël.
\VS{50}Tu mourras sur la montagne où tu vas monter, et tu seras recueilli vers ton peuple, comme Aaron, ton frère, est mort sur la montagne d'Hor, et a été recueilli vers son peuple,
\VS{51}parce que vous avez péché contre moi au milieu des fils d'Israël, aux eaux de Meriba, à Kadès, dans le désert de Tsin~; car vous ne m'avez point sanctifié au milieu des enfants d'Israël.
\VS{52}Tu verras le pays devant toi, mais tu n'entreras point dans le pays, que je donne aux enfants d'Israël.
\Chap{33}
\TextTitle{Moïse bénit les tribus d'Israël}
\VerseOne{}Or c'est ici la bénédiction dont Moïse, homme de Dieu, bénit les enfants d'Israël avant sa mort.
\VS{2}Il dit donc~: Yahweh est venu de Sinaï, il s'est levé sur eux de Séir, il a resplendi de la montagne de Paran, et il est sorti d'entre les dix milliers des saints, et de sa droite le feu de la loi est sorti vers eux.
\VS{3}En effet, il aime les peuples~; tous ses saints sont dans ta main. Ils se sont mis à tes pieds pour recevoir tes paroles.
\VS{4}Moïse nous a donné la loi, héritage de l'assemblée de Jacob.
\VS{5}Il était roi de Jeshurun\FTNT{Voir commentaire en De. 32:15.}, quand les chefs du peuple s'assemblaient ensemble, avec les tribus d'Israël.
\VS{6}Que Ruben vive et qu'il ne meure point, encore que ses hommes soient en petit nombre.
\VS{7}Et voici ce qu'il dit pour Juda~: Ô Yahweh~! Ecoute la voix de Juda, et ramène-le vers son peuple. Que ses mains soient puissantes, et sois-lui en aide contre ses ennemis.
\VS{8}Il dit aussi touchant Lévi~: Tes thummim et tes urim sont à l'homme fidèle que tu as éprouvé à Massa, et avec qui tu as contesté aux eaux de Meriba.
\VS{9}Il dit de son père et de sa mère~: Je ne les ai point vus~! Il ne reconnait point ses frères, et ne connait point ses fils. Car ils gardent tes paroles, et ils gardent ton alliance.
\VS{10}Ils enseignent tes ordonnances à Jacob, et ta loi à Israël~; ils mettent l'encens sous tes narines, et l'holocauste sur ton autel.
\VS{11}Ô Yahweh, bénis sa force~! Agrée l'œuvre de ses mains~! Brise les reins de ceux qui s'élèvent contre lui, et que ceux qui le haïssent ne se relèvent plus~!
\VS{12}Il dit de Benjamin~: Le bien-aimé de Yahweh habitera en sécurité avec lui~; il le protégera toujours, et demeurera entre ses épaules.
\VS{13}Il dit de Joseph~: Son pays est béni par Yahweh, de ce qu'il y a de plus précieux au ciel, de la rosée, et de l'abîme qui est en bas,
\VS{14}et du plus précieux des produits du soleil, et du plus précieux des produits de la lune, 
\VS{15}et de ce qui croît sur le sommet des montagnes d'ancienneté, du plus précieux des collines éternelles,
\VS{16}et du plus précieux de la terre et de sa plénitude. Que la grâce de celui qui demeura dans le buisson vienne sur la tête de Joseph, sur le sommet, sur le sommet de la tête de celui qui est consacré d'entre ses frères~!
\VS{17}Sa majesté est comme le premier-né de son taureau~; et ses cornes comme les cornes du buffle~; il poussera tous les peuples ensemble jusqu'aux extrémités de la terre~: Ce sont les dix milliers d'Ephraïm, et ce sont les milliers de Manassé.
\VS{18}Il dit de Zabulon~: Réjouis-toi, Zabulon, dans ta sortie, et toi, Issacar, dans tes tentes.
\VS{19}Ils appelleront les peuples sur la montagne, ils y offriront des sacrifices de justice, car ils suceront l'abondance des mers, et les trésors cachés dans le sable.
\VS{20}Il dit aussi de Gad~: Béni soit celui qui élargit Gad~! Il habite comme un lion, et il déchire le bras et la tête.
\VS{21}Il a choisi les prémices, parce que c'était là qu'était cachée la portion du législateur, et il est venu en tête du peuple~; il a exécuté la justice de Yahweh et ses jugements envers Israël.
\VS{22}Et il dit de Dan~: Dan est un jeune lion, il s'élance de Basan.
\VS{23}Il dit de Nephthali~: Nephthali, rassasié de faveur, et rempli de la bénédiction de Yahweh, possède l'occident et le sud.
\VS{24}Il dit aussi d'Aser~: Aser sera béni entre les fils~; il sera agréable à ses frères, et il trempera son pied dans l'huile.
\VS{25}Tes verrous seront de fer et d'airain, et ta force durera autant que tes jours.
\VS{26}Nul n'est comme le Dieu de Jeshurun\FTNT{Voir commentaire en De. 32:15.}, porté sur les cieux pour te venir en aide, et sur les nuées dans sa majesté.
\VS{27}Le Dieu d'éternité est un refuge, et au-dessous de toi sont ses bras éternels~; car il a chassé de devant toi tes ennemis, et il a dit~: Extermine.
\VS{28}Israël donc habitera en sécurité, la source de Jacob est à part dans un pays de blé et de vin, et ses cieux distilleront la rosée.
\VS{29}Ô que tu es heureux, Israël~! Qui est le peuple semblable à toi, qui ait été sauvé par Yahweh, le bouclier de ton secours et l'épée de ta majesté~? Tes ennemis dissimuleront devant toi, et tu fouleras de tes pieds leurs lieux élevés.
\Chap{34}
\TextTitle{Moïse voit le pays mais n'y entre pas}
\VerseOne{}Alors Moïse monta des plaines de Moab sur le mont Nebo, au sommet du Pisga, vis-à-vis de Jéricho. Et Yahweh lui fit voir tout le pays~: De Galaad jusqu'à Dan,
\VS{2}tout Nephthali, le pays d'Ephraïm et de Manassé, tout le pays de Juda, jusqu'à la Mer Occidentale,
\VS{3}le sud, les environs du Jourdain, la plaine de Jéricho, la ville des palmiers, jusqu'à Tsoar.
\VS{4}Yahweh lui dit~: C'est ici le pays que j'ai juré à Abraham, à Isaac, et à Jacob, en disant~: Je le donnerai à ta postérité. Je te l'ai fait voir de tes yeux~; mais tu n'y entreras point.
\TextTitle{Mort de Moïse}
\VS{5}Ainsi Moïse, serviteur de Yahweh, mourut là, dans le pays de Moab, selon la parole de Yahweh.
\VS{6}Et il l'ensevelit dans la vallée, au pays de Moab, vis-à-vis de Beth-Peor. Personne n'a connu son sépulcre jusqu'à aujourd'hui\FTNT{Jud. 1:9.}.
\VS{7}Or Moïse était âgé de cent vingt ans quand il mourut~; sa vue n'était point affaiblie, et sa vigueur n'était point passée.
\VS{8}Les enfants d'Israël pleurèrent Moïse trente jours dans les plaines de Moab~; et ces jours de pleurs et de deuil sur Moïse furent accomplis.
\TextTitle{Josué, successeur de Moïse}
\VS{9}Et Josué, fils de Nun, fut rempli de l'Esprit de sagesse, parce que Moïse lui avait imposé les mains\FTNT{Jos. 1:9.}. Les enfants d'Israël lui obéirent, et firent ce que Yahweh avait ordonné à Moïse.
\VS{10}Et il ne s'est plus levé en Israël de prophète comme Moïse, que Yahweh connaissait face à face.
\VS{11}Selon tous les signes et les miracles que Yahweh l'envoya faire au pays d'Egypte, devant Pharaon, et tous ses serviteurs, et tout son pays,
\VS{12}et selon toute cette main forte, et tous ces terribles prodiges, que Moïse fit sous les yeux de tout Israël.
\PPE{}
\end{multicols}

%\addcontentsline{toc}{section}{Nevi'im (Prophètes)}\clearpage
%\clearpage\ShortTitle{Josué}\BookTitle{Josué}\BFont
\noindent\hrulefill
{\footnotesize
\textit{
\bigskip
{\centering{}
\\Auteur : Probablement Josué
\\(Heb. : Yehowshuwa)
\\Signification : Yahweh est salut
\\Thème : La conquête de Canaan
\\Date de rédaction : 14\up{ème} siècle av. J.-C.\\}
}
%\bigskip
\textit{
\\Né en Egypte, Josué, fils de Nun, originaire de la tribu d'Ephraïm, servit Moïse de la sortie d'Egypte jusqu'à sa mort. Choisi par Dieu pour succéder au prophète, il fut le seul de l'ancienne génération, avec Caleb, à avoir survécu à la longue épreuve du désert. Ce livre relate les étapes du voyage du peuple et sa conquête de la terre promise. Il présente par ailleurs les victoires acquises par la puissance de Yahweh sous la conduite de Josué. C'est l'histoire de la prise de Canaan et de son partage aux douze tribus d'Israël.\bigskip
}
}
\par\nobreak\noindent\hrulefill
\begin{multicols}{2}
\Chap{1}
\TextTitle{Josué succède à Moïse à sa mort\FTNTT{De. 34:9.}}
\VerseOne{}Or, il arriva après la mort de Moïse, serviteur de Yahweh, que Yahweh parla à Josué, fils de Nun, qui avait servi Moïse, en disant :
\VS{2}Moïse, mon serviteur est mort ; maintenant donc, lève-toi, passe ce Jourdain, toi et tout ce peuple, pour entrer dans le pays que je donne aux enfants d'Israël.
\VS{3}Tout lieu que foulera la plante de votre pied, je vous l'ai donné, comme je l'ai déclaré à Moïse\FTNT{De. 11:24.}.
\VS{4}Vos frontières seront depuis ce désert et le Liban, jusqu'au grand fleuve, le fleuve de l'Euphrate, tout le pays des Héthiens jusqu'à la grande mer, vers le soleil couchant.
\VS{5}Nul ne tiendra devant toi, tous les jours de ta vie. Je serai avec toi comme j'ai été avec Moïse ; je ne te délaisserai point, et je ne t'abandonnerai point\FTNT{De. 31:6 ; Hé. 13:5-6.}.
\VS{6}Fortifie-toi et prends courage, car c'est toi qui mettras ce peuple en possession du pays dont j'ai juré à leurs pères de leur donner.
\VS{7}Seulement fortifie-toi et renforce-toi de plus en plus, afin que tu prennes garde de faire selon toute la loi que Moïse mon serviteur t'a ordonnée ; ne t'en détourne point ni à droite ni à gauche, afin que tu prospères partout où tu iras.
\VS{8}Que ce livre de la loi ne s'éloigne point de ta bouche, mais médite-le jour et nuit, pour agir fidèlement selon tout ce qui y est écrit\FTNT{La clé d'une vie chrétienne épanouie est la Parole de Dieu. Méditer signifie : 
\\- Murmurer la Parole de Dieu : Partout où nous sommes, nous pouvons dans nos cœurs murmurer les promesses du Seigneur (Ps. 63:5-8 ; Ps. 119:11).
\\- Proclamer à haute voix : Il est intéressant de noter que le mot hébreu traduit dans Jos. 1:8 par méditer est traduit par « proclamer » ou « dire » dans Pr. 8:7 ; Ps. 35:8 ; Ps. 77:13. 
\\- Réfléchir profondément : Il faut être dans le lieu secret (Mt. 6:5-6). En Israël il est de coutume d'aller étudier la Torah à l'ombre d'un figuier. Voir Jn. 1:43-51.} ; car c'est alors que tu auras du succès dans tes entreprises, c'est alors que tu réussiras.
\VS{9}Ne t'ai-je pas donné cet ordre, fortifie-toi et prends courage ? Ne t'épouvante point et ne t'effraie point ; car Yahweh ton Dieu est avec toi partout où tu iras.
\TextTitle{Josué prend la direction du peuple}
\VS{10}Après cela, Josué donna cet ordre aux officiers du peuple, en disant :
\VS{11}Passez par le camp, ordonnez au peuple et dites-lui : Préparez-vous des provisions, car dans trois jours vous passerez ce Jourdain pour aller prendre possession du pays que Yahweh, votre Dieu, vous donne afin que vous le possédiez.
\VS{12}Josué parla aussi aux Rubénites, aux Gadites et à la demi-tribu de Manassé, en disant :
\VS{13}Souvenez-vous de la parole que Moïse, serviteur de Yahweh, vous a prescrite, en disant : Yahweh votre Dieu vous a accordé du repos, et vous a donné ce pays.
\VS{14}Vos femmes, vos petits-enfants, et vos bêtes resteront dans le pays que Moïse vous a donné de l'autre côté du Jourdain ; mais vous tous, hommes vaillants, vous passerez en armes devant vos frères, et vous les aiderez\FTNT{Ex. 13:18.} ;
\VS{15}jusqu'à ce que Yahweh ait accordé du repos à vos frères comme à vous, et qu'ils soient aussi en possession du pays que Yahweh, votre Dieu, leur donne. Puis vous reviendrez prendre possession du pays qui est votre propriété, et que vous a donné Moïse, serviteur de Yahweh, de l'autre côté du Jourdain, vers l'orient.
\VS{16}Ils répondirent à Josué, en disant : Nous ferons tout ce que tu nous as ordonné, et nous irons partout où tu nous enverras.
\VS{17}Nous t'obéirons comme nous avons obéi à Moïse ; seulement que Yahweh ton Dieu soit avec toi, comme il a été avec Moïse.
\VS{18}Tout homme qui sera rebelle à ton ordre, et qui n'obéira point à tes paroles dans tout ce que tu lui commanderas, sera mis à mort ; seulement, fortifie-toi, et sois courageux !
\Chap{2}
\TextTitle{Josué envoie deux espions à Jéricho ; ils sont reçus par Rahab\FTNTT{Ja. 2:25.}}
\VerseOne{}Or, Josué fils de Nun, envoya secrètement de Sittim deux hommes, pour épier secrètement le pays, et il leur dit : Allez, examinez le pays, et Jéricho. Ils partirent donc et entrèrent dans la maison d'une femme prostituée, nommée Rahab\FTNT{Rahab avait entendu parler du Dieu des Hébreux et avait placé son espérance de salut en lui (Ro. 10 :11). Par cet acte de foi, sa destinée a changé. Cette femme qui était vouée à une double condamnation du fait de sa condition de prostituée (De.23:17) et de son appartenance à une nation païenne qui devait être dévouée à la façon de l'interdit (Jos. 6), a été sauvée avec sa famille (Ac. 2:21 ; Ac. 16:31 ). Ainsi, bien des siècles plus tard, on ne la mentionnera plus comme une prostituée, mais comme une ancêtre du Sauveur et une héroïne de la foi (Mt. 1:5 ; Hé. 11 :21). Rahab est donc l'archétype des païens qui sont rentrés dans l'alliance de Dieu par la foi.}, et ils y couchèrent.
\VS{2}Alors on dit au roi de Jéricho : Voici, des hommes sont venus ici cette nuit de la part des enfants d'Israël pour explorer le pays.
\VS{3}Et le roi de Jéricho envoya dire à Rahab : Fais sortir les hommes qui sont venus chez toi et qui sont entrés dans ta maison ; car ils sont venus pour explorer tout le pays.
\VS{4}Or la femme prit les deux hommes et les cacha ; et elle dit : Il est vrai que des hommes sont venus chez moi, mais je ne savais pas d'où ils étaient ;
\VS{5}et comme on fermait la porte sur le soir, ces hommes sont sortis ; je ne sais pas où ces hommes sont allés ; poursuivez-les bien vite car vous les atteindrez.
\VS{6}Or elle les avait fait monter sur le toit et les avait cachés sous des tiges de lin qu'elle avait arrangées sur le toit.
\VS{7}Et quelques gens les poursuivirent par le chemin du Jourdain jusqu'aux passages ; et on ferma la porte après que ceux qui les poursuivaient furent sortis.
\VS{8}Or, avant qu'ils se couchent, elle monta vers eux sur le toit ;
\VS{9}et leur dit : Je sais que Yahweh vous a donné ce pays, et que la terreur de votre nom nous a saisis, et que tous les habitants du pays perdent courage à cause de vous\FTNT{Ex. 23:27.}.
\VS{10}Car nous avons entendu que Yahweh a mis à sec devant vous les eaux de la Mer Rouge à votre sortie du pays d'Egypte ; et ce que vous avez fait aux deux rois des Amoréens qui étaient de l'autre côté du Jourdain, à Sihon et à Og, que vous avez détruits complètement en les dévouant par le moyen de l'interdit.
\VS{11}Nous l'avons entendu, et notre cœur a fondu, et depuis aucun homme n'a eu le courage à cause de vous. Car Yahweh, votre Dieu, est le Dieu des cieux en haut et de la terre\FTNT{De. 4:39.} en bas.
\VS{12}Maintenant donc, je vous prie, jurez-moi par Yahweh, que puisque j'ai usé de bonté envers vous, vous userez aussi de bonté envers la maison de mon père, 
\VS{13}et que vous me donnerez un signe de votre fidélité\FTNT{La couleur cramoisi s'obtient grâce à la femelle cochenille aptère qui contient dans son corps et dans ses œufs un pigment rouge à base d'acide carminique qui permet à l'insecte et à ses larves de se protéger des prédateurs. Au moment de la ponte, cette dernière fixe fermement son corps au tronc d'un arbre puis libère ses œufs qui demeurent ainsi protégés en dessous d'elle jusqu'à leur éclosion. Ensuite, l'insecte meurt en libérant cette substance rouge qui se propage sur tout son corps et sur le bois hôte. C'est ce fluide que l'homme récupère pour en faire un colorant à la couleur caractéristique. Une subtile analogie peut être faire entre la cochenille et le Seigneur qui a versé son sang à la croix pour nous donner la vie. « Et moi, je suis un ver, et non un homme, l'opprobre des hommes et le méprisé du peuple » (Ps. 22 :7).} d'une ferme assurance que vous laisserez vivre mon père, ma mère, mes frères, mes sœurs, et tous ceux qui leur appartiennent, et que vous sauverez nos âmes de la mort.
\VS{14}Et ces hommes lui répondirent : Nos personnes répondront pour vous jusqu'à la mort, pourvu que vous ne divulguez pas cette affaire ; et quand Yahweh nous aura donné le pays nous userons envers toi de bonté et de vérité. 
\TextTitle{Les espions s'enfuient aidés par Rahab}
\VS{15}Elle les fit donc descendre avec une corde par la fenêtre ; car sa maison était sur la muraille de la ville, et elle habitait sur la muraille de la ville. 
\VS{16}Et elle leur dit : Allez à la montagne, de peur que ceux qui vous poursuivent ne vous rencontrent, et cachez-vous là pendant trois jours jusqu'à ce qu'ils soient de retour. Après cela vous suivrez votre chemin.
\VS{17}Et ces hommes lui dirent : Voici comment nous serons quittes de ce serment que tu nous as fait faire.
\VS{18}Voici, quand nous entrerons dans le pays, tu lieras ce cordon de fil d'écarlate à la fenêtre par laquelle tu nous auras fait descendre, et tu recueilleras chez toi, dans cette maison, ton père et ta mère, tes frères, et toute la famille de ton père.
\VS{19}Et quiconque sortira hors de la porte de ta maison, son sang sera sur sa tête, et nous en serons quittes ; mais quiconque sera avec toi, dans la maison, son sang sera sur notre tête si quelqu'un met la main sur lui.
\VS{20}Et si tu divulgues cette affaire, nous serons quittes du serment que tu nous as fait faire.
\VS{21}Et elle répondit : Que cela soit ainsi que vous l'avez dit. Alors elle les laissa aller. Ils s'en allèrent et elle lia le cordon de fil d'écarlate à la fenêtre.
\VS{22}Et ils marchèrent et arrivèrent à la montagne, où ils restèrent trois jours, jusqu'à ce que ceux qui les poursuivaient soient de retour. Ceux qui les poursuivaient les cherchèrent par tout le chemin, mais ils ne les trouvèrent pas.
\VS{23}Ainsi ces deux hommes s'en retournèrent, descendirent de la montagne, passèrent le Jourdain. Ils vinrent auprès de Josué, fils de Nun. Ils lui racontèrent toutes les choses qui leur étaient arrivées.
\VS{24}Et ils dirent à Josué : Certainement, Yahweh a livré tout le pays entre nos mains, et même tous les habitants ont perdu le courage à notre vue.
\Chap{3}
\TextTitle{Israël traverse le Jourdain à sec}
\VerseOne{}Or Josué se leva de bon matin, lui et tous les enfants d'Israël partirent de Sittim, ils vinrent jusqu'au Jourdain, et ils logèrent là cette nuit, avant de le traverser.
\VS{2}Et au bout de trois jours les officiers traversèrent le milieu du camp,
\VS{3}et donnèrent cet ordre au peuple en disant : Dès que vous verrez l'arche de l'alliance de Yahweh, votre Dieu, portée par les prêtres, les Lévites, vous partirez de votre quartier, et vous marcherez après elle.
\VS{4}Et afin que vous n'approchez pas d'elle, il y aura entre vous et elle une distance de la mesure d'environ deux mille coudées. Elle vous fera connaître le chemin par lequel vous devez marcher ; car vous n'avez pas encore passé par ce chemin.
\VS{5}Josué dit au peuple : Sanctifiez-vous, car Yahweh fera demain des choses merveilleuses au milieu de vous\FTNT{Ex. 19:10-11.}.
\VS{6}Josué parla aussi aux prêtres, en disant : Portez l'arche de l'alliance, et passez devant le peuple. Ainsi ils portèrent l'arche de l'alliance, et marchèrent devant le peuple.
\VS{7}Or Yahweh dit à Josué : Aujourd'hui je commencerai à t'élever aux yeux de tout Israël, afin qu'ils sachent que je serai aussi avec toi, comme j'ai été avec Moïse.
\VS{8}Tu donneras cet ordre aux prêtres qui portent l'arche de l'alliance, en leur disant : Dès que vous arriverez au bord des eaux du Jourdain, vous vous arrêterez dans le Jourdain.
\VS{9}Et Josué dit aux enfants d'Israël : Approchez-vous d'ici, et écoutez les paroles de Yahweh, votre Dieu.
\VS{10}Puis Josué dit : Vous reconnaîtrez à ceci que le Dieu vivant est au milieu de vous et qu'il chassera et déshéritera devant vous les Cananéens, les Héthiens, les Héviens, les Phéréziens, les Guirgasiens, les Amoréens et les Jébusiens.
\VS{11}Voici, l'arche de l'alliance du Seigneur de toute la terre va passer devant vous dans le Jourdain.
\VS{12}Maintenant, prenez douze hommes des tribus d'Israël, un homme de chaque tribu.
\VS{13}Et il arrivera qu'aussitôt que les plantes des pieds des prêtres qui portent l'arche de Yahweh, le Seigneur de toute la terre, seront posés dans les eaux du Jourdain, les eaux du Jourdain seront coupées, les eaux, dis-je, qui descendent d'en haut, et elles s'arrêteront en un monceau\FTNT{Ps. 114:3.}.
\VS{14}Et il arriva que le peuple étant parti de ses tentes pour passer le Jourdain, et les prêtres qui portaient l'arche de l'alliance, étaient devant le peuple.
\VS{15}Aussitôt que ceux qui portaient l'arche furent arrivés au Jourdain, et que les pieds des prêtres qui portaient l'arche furent mouillés au bord de l'eau. Le Jourdain regorge par-dessus toutes ses rives durant tout le temps de la moisson\FTNT{1 Ch. 12:15}.
\VS{16}Les eaux qui descendent d'en haut, s'arrêtèrent, et s'élevèrent en un monceau, à une très grande distance, depuis la ville d'Adam, qui est à côté de Tsarthan ; et celles d'en bas, qui descendaient vers la mer de la plaine, qui est la mer salée, furent totalement coupées. Le peuple passa vis-à-vis de Jéricho.
\VS{17}Mais les prêtres qui portaient l'arche de l'alliance de Yahweh, s'arrêtèrent de pied ferme sur le sec, au milieu du Jourdain, pendant que tout Israël passait à sec, jusqu'à ce que tout le peuple ait achevé de passer le Jourdain.
\Chap{4}
\TextTitle{Josué dresse un monument de pierres en souvenir de la traversée}
\VerseOne{}Or il arriva que quand tout le peuple eut achevé de passer le Jourdain, que Yahweh parla à Josué et dit :
\VS{2}Prenez douze hommes parmi le peuple, un homme de chaque tribu.
\VS{3}et donnez-leur cet ordre, en disant : Prenez ici, du milieu du Jourdain, de la place où les prêtres se sont arrêtés de pied ferme, douze pierres, que vous emporterez avec vous, et vous les poserez au lieu où vous passerez cette nuit.
\VS{4}Josué appela les douze hommes qu'il choisit parmi les enfants d'Israël, un homme de chaque tribu.
\VS{5}Et il leur dit : Passez devant l'arche de Yahweh, votre Dieu, au milieu du Jourdain, et que chacun de vous charge une pierre sur son épaule, selon le nombre des tribus des enfants d'Israël ;
\VS{6}afin que cela soit un signe au milieu de vous. Et quand vos fils interrogeront à l'avenir leurs pères, en disant : Que signifient ces pierres-ci ?
\VS{7}Alors vous leur répondrez : Les eaux du Jourdain ont été coupées devant l'arche de l'alliance de Yahweh ; lorsqu'elle passa le Jourdain, les eaux du Jourdain ont été arrêtées ; c'est pourquoi ces pierres-là seront à jamais un souvenir pour les enfants d'Israël.
\VS{8}Les enfants d'Israël firent donc comme Josué leur avait ordonné. Ils prirent douze pierres du milieu du Jourdain, comme Yahweh l'avait ordonné à Josué, selon le nombre des tribus des enfants d'Israël. Ils les emportèrent avec eux et les posèrent au lieu où ils devaient passer la nuit.
\VS{9}Josué dressa aussi douze pierres au milieu du Jourdain, à l'endroit où les pieds des prêtres qui portaient l'arche de l'alliance s'étaient arrêtés ; et elles y sont restées jusqu'à ce jour.
\VS{10}Les prêtres donc qui portaient l'arche se tinrent debout au milieu du Jourdain, jusqu'à ce que tout ce que Yahweh avait ordonné à Josué de dire au peuple soit accompli, selon tout ce que Moïse avait prescrit à Josué. Et le peuple se hâta de passer.
\VS{11}Et quand tout le peuple eut achevé de passer, alors l'arche de Yahweh et les prêtres passèrent devant le peuple.
\VS{12}Et les fils de Ruben, les fils de Gad, et la demi-tribu de Manassé passèrent en armes devant les enfants d'Israël, comme Moïse le leur avait dit\FTNT{No. 32:20-29.}.
\VS{13}Ils passèrent, dis-je, dans les plaines de Jérico environ quarante mille hommes en équipage de guerre, devant Yahweh, pour combattre. 
\VS{14}Ce jour-là, Yahweh éleva Josué à la vue de tout Israël, et ils le craignirent, comme ils avaient craint Moïse, tous les jours de sa vie.
\VS{15}Yahweh parla à Josué, et dit :
\VS{16}Ordonne aux prêtres qui portent l'arche du témoignage qu'ils montent hors du Jourdain.
\VS{17}Et Josué donna cet ordre aux prêtres, en disant : Montez hors du Jourdain.
\VS{18}Or sitôt que les prêtres, qui portaient l'arche de l'alliance de Yahweh furent montés hors du milieu du Jourdain, et qu'ils eurent mis la plante de leurs pieds sur le sec, les eaux du Jourdain retournèrent à leur place, et coulèrent comme auparavant sur tous les rivages.
\VS{19}Le peuple donc monta hors du Jourdain le dixième jour du premier mois, et il campa à Guilgal, à l'orient de Jéricho.
\VS{20}Josué aussi dressa à Guilgal les douze pierres qu'ils avaient prises du Jourdain.
\VS{21}Et il parla aux enfants d'Israël et leur dit : Quand vos enfants interrogeront à l'avenir leurs pères, et leur diront : Que signifient ces pierres-ci ?
\VS{22}Vous l'apprendrez à vos enfants, en leur disant : Israël a passé ce Jourdain à sec.
\VS{23}Car Yahweh, votre Dieu, a fait tarir les eaux du Jourdain devant vous jusqu'à ce que vous eussiez passé, comme Yahweh, votre Dieu, l'avait fait à la Mer Rouge, qu'il mit à sec devant nous, jusqu'à ce que nous eussions passé,
\VS{24}afin que tous les peuples de la terre sachent que la main de Yahweh est puissante, et afin que vous ayez toujours la crainte de Yahweh, votre Dieu.
\Chap{5}
\TextTitle{La crainte s'empare des Amoréens}
\VerseOne{}Or il arriva qu'aussitôt que tous les rois des Amoréens qui étaient au-delà du Jourdain, vers l'occident, et tous les rois des Cananéens qui étaient près de la mer, apprirent que Yahweh avait mis à sec les eaux du Jourdain devant les enfants d'Israël, jusqu'à ce que nous eussions passé, leur cœur fut fondu, et il n'y avait plus de courage en eux à cause des enfants d'Israël.
\TextTitle{Israël circoncis à nouveau ; la fin de la manne}
\VS{2}En ce temps-là, Yahweh dit à Josué : Fais-toi des couteaux de pierre tranchants, et circoncis de nouveau les enfants d'Israël, une seconde fois.
\VS{3}Et Josué se fit des couteaux de pierre tranchants, et circoncit les enfants d'Israël sur la colline d'Araloth.
\VS{4}Or la raison pour laquelle Josué les circoncit, c'est que tout le peuple sorti d'Egypte, tous les mâles, dis-je, hommes de guerre étaient morts en chemin dans le désert, après leur sortie d'Egypte.
\VS{5}Et tout le peuple sorti d'Egypte était circoncis, mais aucun du peuple né dans le désert en chemin n'avait été circoncis, après leur sortie d'Egypte.
\VS{6}Car les enfants d'Israël avaient marché dans le désert quarante ans jusqu'à ce que soit consummée toute la nation des hommes de guerre qui étaient sortis d'Egypte, et qui n'avaient point écouté la voix de Yahweh ; auxquels Yahweh avait juré qu'il ne leur laisserait point voir le pays qu'il avait juré à leurs pères de nous donner, pays où coulent le lait et le miel\FTNT{No. 14:32-33.}.
\VS{7}Et il a suscité à leur place leurs enfants que Josué circoncit, parce qu'ils étaient incirconcis ; car on ne les avait pas circoncis pendant le voyage.
\VS{8}Et quand on eut achevé de circoncire tout le peuple, ils restèrent dans leur camp, jusqu'à ce qu'ils soient guéris.
\VS{9}Et Yahweh dit à Josué : Aujourd'hui j'ai roulé de dessus vous l'opprobre de l'Egypte. Et ce lieu-là fut appelé Guilgal jusqu'à ce jour.
\VS{10}Ainsi les enfants d'Israël campèrent à Guilgal, et célébrèrent la Pâque le quatorzième jour du mois, sur le soir, dans les plaines de Jéricho\FTNT{Ex. 12:6.}.
\VS{11}Et dès le lendemain de la Pâque, ils mangèrent du blé du pays, savoir, des pains sans levain et du grain rôti, en ce même jour\FTNT{Ex. 12:39 ; Lé. 2:14}.
\VS{12}Et la manne cessa dès le lendemain de la Pâque, après qu'ils eurent manger du blé du pays ; les enfants d'Israël n'eurent plus de manne, mais ils mangèrent les récoltes de la terre de Canaan cette année-là\FTNT{Ex. 16:35.}.
\TextTitle{Rencontre avec le chef de l'armée de Yahweh}
\VS{13}Or il arriva, comme Josué était près de Jéricho, qu'il leva les yeux et regarda. Voici, un homme qui avait son épée nue à la main, se tenait debout devant lui. Josué alla vers lui et lui dit : Es-tu des nôtres ou de nos ennemis ?
\VS{14}Et il répondit : Non, mais je suis le Chef de l'armée de Yahweh, je viens maintenant. Josué tomba à terre sur son visage, l'adora, et lui dit : Qu'est-ce que mon Seigneur dit à son serviteur ?
\VS{15}Et le Chef de l'armée de Yahweh dit à Josué : Délie tes souliers de tes pieds ; car le lieu sur lequel tu te tiens est saint\FTNT{Ex. 3:5.}. Et Josué fit ainsi.
\Chap{6}
\TextTitle{Jéricho miraculeusement livré à Israël ; Rahab sauvée}
\VerseOne{}Or Jéricho était barricadée et fermée soigneusement, à cause des enfants d'Israël. Personne ne sortait, et personne n'entrait.
\VS{2}Et Yahweh dit à Josué : Regarde, j'ai livré entre tes mains Jéricho et son roi, ses hommes vaillants.
\VS{3}Vous tous donc, hommes de guerre, vous ferez le tour de la ville, en tournant une fois autour d'elle. Tu feras ainsi durant six jours.
\VS{4}Et Sept prêtres porteront sept shofars retentissants devant l'arche. Mais au septième jour, vous ferez sept fois le tour de la ville et les prêtres sonneront des shofars.
\VS{5}Et quand ils sonneront avec la corne de bélier, aussitôt que vous entendrez le son du shofar retentissant, tout le peuple poussera un grand cri de joie et la muraille de la ville tombera sur elle. Et le peuple montera, les hommes devant lui.
\VS{6}Josué donc, fils de Nun, appela les prêtres et leur dit : Portez l'arche de l'alliance et que sept prêtres portent sept shofars devant l'arche de Yahweh.
\VS{7}Il dit aussi au peuple : Passez et faites le tour de la ville, que tous ceux qui seront armés passent devant l'arche de Yahweh.
\VS{8}Et quand Josué eut parlé au peuple, les sept prêtres qui portaient les sept cornes de béliers devant Yahweh passèrent et sonnèrent des cornes. Et l'arche de l'alliance de Yahweh les suivait.
\VS{9}Et les hommes qui étaient armés marchaient devant les prêtres qui sonnaient des shofars ; mais l'arrière-garde suivait derrière l'arche ; on sonnait des shofars en marchant.
\VS{10}Or Josué avait donné cet ordre au peuple, en disant : Vous ne pousserez point de cris de joie et vous ne ferez point entendre votre voix. Et il ne sortira point un seul mot de votre bouche, jusqu'au jour où je vous dirai : Poussez des cris de joie ! Alors vous crierez.
\VS{11}L'arche de Yahweh fit ainsi le tour de la ville, en tournant tout autour une fois, puis on revint au camp, et on y passa la nuit.
\VS{12}Ensuite Josué se leva de bon matin, et les prêtres portèrent l'arche de Yahweh.
\VS{13}Et les sept prêtres qui portaient les sept cornes de bélier devant l'arche de Yahweh se mirent en marche et sonnèrent du shofar. Et les hommes armés allaient devant eux ; puis l'arrière-garde suivait l'arche de Yahweh ; on sonnait des shofars en marchant.
\VS{14}Ainsi ils firent une fois le tour de la ville le deuxième jour, et ils retournèrent au camp. Ils firent de même durant six jours.
\VS{15}Mais quand le septième jour fut venu, ils se levèrent dès le matin à l'aube du jour, et ils firent sept fois le tour de la ville de la même manière ; ce fut le seul jour où ils firent sept fois le tour de la ville.
\VS{16}Et à la septième fois, comme les prêtres sonnaient des shofars, Josué dit au peuple : Poussez des cris de joie, car Yahweh vous a donné la ville !
\VS{17}La ville sera dévouée par le moyen de l'interdit à Yahweh, elle et toutes les choses qui y sont ; seulement Rahab, la prostituée\FTNT{Rahab sauva sa famille par sa foi en Dieu (Ac. 16:31). Voir Josué 2.}, vivra, elle et tous ceux qui seront avec elle dans la maison, parce qu'elle a caché soigneusement les messagers que nous avions envoyés.
\VS{18}Mais quoi qu'il en soit gardez-vous de l'interdit, de peur que vous ne vous mettiez en interdit, et que vous ne mettriez le camp d'Israël en interdit et que vous le troubliez\FTNT{De. 7:26.}.
\VS{19}Mais tout l'argent et tout l'or, tous les objets d'airain et de fer seront consacrés à Yahweh, ils entreront dans le trésor de Yahweh\FTNT{No. 31:54.}.
\VS{20}Le peuple donc poussa de cris de joie et on sonna des shofars. Et quand le peuple entendit le son des shofars, il poussa de grands cris de joie et la muraille tomba sur elle-même\FTNT{Hé. 11:30.}. Alors le peuple monta dans la ville, les hommes devant le peuple. Et ils prirent la ville. 
\VS{21}Et ils la dévouèrent entièrement par le moyen de l'interdit, et passèrent au fil de l'épée tout ce qui était dans la ville, depuis l'homme jusqu'à la femme, depuis l'enfant jusqu'au vieillard, même jusqu'aux bœufs, aux brebis et aux ânes.
\VS{22}Mais Josué dit aux deux hommes qui avaient espionné le pays : Entrez dans la maison de cette femme prostituée, et faites-la sortir de là, avec tous ceux qui lui appartiennent, selon que vous lui avez juré.
\VS{23}Les jeunes hommes donc qui avaient espionné le pays, entrèrent et firent sortir Rahab, et son père, et sa mère et ses frères, avec tous ceux qui lui appartenaient ; ils firent aussi sortir toutes les familles qui lui appartenaient, et les mirent hors du camp d'Israël.
\VS{24}Puis ils allumèrent le feu et brûlèrent la ville et tout ce qui s'y trouvait ; seulement ils mirent l'argent et l'or, les objets d'airain et de fer dans le trésor de la maison de Yahweh.
\VS{25}Ainsi Josué sauva la vie à Rahab la prostituée, la maison de son père, et tous ceux qui lui appartenaient ; et elle a habité au milieu d'Israël jusqu'à ce jour, parce qu'elle avait caché les messagers que Josué avait envoyés pour explorer Jéricho.
\VS{26}Et en ce temps-là Josué jura, en disant : Maudit soit devant Yahweh l'homme qui se mettra à rebâtir cette ville de Jéricho ! Il la fondera sur son premier-né, et il posera ses portes sur son plus jeune fils\FTNT{Cette parole s'est accomplie en 1 R. 16:34.}.
\VS{27}Yahweh fut avec Josué, et sa renommée se répandit dans tout le pays.
\Chap{7}
\TextTitle{Israël battu à Aï suite au péché d'Acan}
\VerseOne{}Mais les enfants d'Israël se rendirent coupables au sujet de l'interdit. Car Acan, fils de Carmi, fils de Zabdi, fils de Zérach, de la tribu de Juda, prit de l'interdit, et la colère de Yahweh s'enflamma contre les enfants d'Israël.
\VS{2}Car Josué envoya de Jéricho des hommes vers Aï, qui est près de Beth-Aven, à l'orient de Béthel. Il leur parla, et dit : Montez, et reconnaissez le pays. Ces hommes donc montèrent et reconnurent Aï.
\VS{3}Et étant retournés vers Josué, ils lui dirent : Que tout le peuple n'y monte point mais qu'environ deux mille ou trois mille hommes y montent, et ils battront Aï. Ne fatigue pas tout le peuple en l'envoyant là, car ils sont en petit nombre.
\VS{4}Ainsi, environ trois mille hommes du peuple y montèrent, mais ils s'enfuirent devant les gens d'Aï.
\VS{5}Et les gens d'Aï leur tuèrent environ trente-six hommes ; car ils les poursuivirent depuis la porte jusqu'à Schebarim, et les battirent à la descente. Le cœur du peuple se fondit et devint comme de l'eau.
\VS{6}Alors Josué déchira ses vêtements, et se jeta sur le visage contre terre, devant l'arche de Yahweh jusqu'au soir, lui et les anciens d'Israël, et ils jettèrent de la poussière sur leur tête.
\VS{7}Et Josué dit : Helas ! Seigneur Yahweh, pourquoi as-tu fait si magnifiquement passer le Jourdain à ce peuple, pour nous livrer entre les mains des Amoréens, et nous faire périr ? Oh ! Que n'avons-nous eu dans l'esprit de demeurer de l'autre côté du Jourdain !
\VS{8}Hélas ! Seigneur, que dirai-je, puisqu'Israël a tourné le dos devant ses ennemis ?
\VS{9}Les Cananéens et tous les habitants du pays l'entendront ; ils nous envelepperont, et ils retrancheront notre nom de dessus la terre. Et que feras-tu à ton grand Nom ?
\VS{10}Alors Yahweh dit à Josué : Lève-toi ! Pourquoi te jettes-tu ainsi le visage contre terre ?
\VS{11}Israël a péché ; ils ont transgressé mon alliance que je leur avais prescrite, même ils ont pris de l'interdit, même ils en ont dérobé, même ils ont menti, et même ils l'ont caché parmi leurs objets\FTNT{Il est impossible de remporter une victoire contre Satan en ayant avec soi des choses qui lui appartiennent (Jn. 14:30). Celui qui pèche est du diable nous dit la Parole de Dieu (1 Jn. 3:4-10). Les grandes victoires sont remportées par ceux qui se sanctifient et invoquent le Nom de Jésus-Christ.}.
\VS{12}C'est pourquoi les enfants d'Israël ne pourront subsister devant leurs ennemis ; ils tourneront le dos devant leurs ennemis ; car ils sont devenus un interdit. Je ne serai plus avec vous si vous ne détruisez pas l'interdit du milieu de vous.
\VS{13}Lève-toi, sanctifie le peuple, et dis : Sanctifiez-vous pour demain ; car ainsi parle Yahweh, le Dieu d'Israël : Il y a de l'interdit au milieu de toi, Israël ! Tu ne pourras subsister et faire face à tes ennemis jusqu'à ce que vous ayez ôté l'interdit du milieu de vous.
\VS{14}Vous vous approcherez donc le matin selon vos tribus ; et la tribu que Yahweh aura saisi s'approchera selon les familles, et la famille que Yahweh aura saisie s'approchera selon les maisons, et la maison que Yahweh aura saisie s'approchera selon les hommes.
\VS{15}Alors celui qui aura été saisi avec l'interdit sera brûlé au feu, lui et tout ce qui lui appartient parce qu'il a transgressé l'alliance de Yahweh, et qu'il a commis une infamie en Israël.
\VS{16}Josué donc se leva de bon matin, et fit approcher Israël selon ses tribus, et la tribu de Juda fut saisie.
\VS{17}Puis il fit approcher les familles de Juda, et la famille de Zérach fut saisie. Puis il fit approcher les hommes de la famille de ceux qui étaient descendants de Zérach, et Zabdi fut saisie.
\VS{18}Et quand il fit approcher la maison de Zabdi par hommes, Acan fils de Carmi, fils de Zabdi, fils de Zérach, de la tribu de Juda, fut saisi.
\VS{19}Josué dit à Acan : Mon fils, je te prie donne gloire à Yahweh, le Dieu d'Israël, et fais-lui confession. Déclare-moi je te prie ce que tu as fait, ne me le cache point.
\VS{20}Et Acan répondit à Josué, et dit : J'ai péché il est vrai, contre Yahweh, le Dieu d'Israël, et voici ce que j'ai fait.
\VS{21}J'ai vu parmi le butin un beau manteau de Schinear\FTNT{Ge. 10:6-12.}, deux cents sicles d'argent et un lingot d'or du poids de cinquante sicles ; je les ai convoités, je les ai pris et voilà, ces choses sont cachées dans la terre au milieu de ma tente, et l'argent est sous le manteau.
\VS{22}Alors Josué envoya des messagers qui coururent à cette tente ; et voici, le manteau était caché dans la tente d'Acan, et l'argent sous le manteau.
\VS{23}Ils les tirèrent donc du milieu de la tente et les apportèrent à Josué et à tous les enfants d'Israël, et ils les déposèrent devant Yahweh.
\VS{24}Alors Josué et tout Israël avec lui, prirent Acan, fils de Zérach, l'argent, le manteau, le lingot d'or, ses fils et ses filles, ses bœufs, ses ânes et ses brebis, sa tente et tout ce qui lui appartenait, et ils les firent monter dans la vallée d'Acor.
\VS{25}Et Josué dit : Pourquoi nous as-tu troublés ? Yahweh te troublera aujourd'hui. Et tout Israël le lapida avec des pierres, et les brûlèrent au feu, après les avoir lapidés avec des pierres.
\VS{26}Et ils dressèrent sur lui un grand monceau de pierres, qui dure jusqu'à ce jour. Et Yahweh apaisa l'ardeur de sa colère. C'est pourquoi ce lieu-là a été appelé jusqu'à aujourd'hui, la vallée d'Acor\FTNT{2 S. 18:17.}.
\Chap{8}
\TextTitle{Victoire d'Israël à Aï}
\VerseOne{}Puis Yahweh dit à Josué : Ne crains point, et ne t'effraie de rien\FTNT{De. 1:21 ; De. 7:18.} ! Prends avec toi tout le peuple propre à la guerre et lève-toi, et monte contre Aï. Regarde, j'ai livré entre tes mains le roi d'Aï et son peuple, sa ville et son pays.
\VS{2}Et tu traiteras Aï et son roi, comme tu as fais Jéricho et son roi : Seulement vous pillerez pour vous le butin et les bêtes. Place des gens en embuscade derrière la ville.
\VS{3}Josué donc se leva avec tout le peuple propre à la guerre, pour monter contre Aï. Josué choisit trente mille vaillants hommes armés, et les envoya de nuit.
\VS{4}Et il leur donna cet ordre en disant : Voyez, vous qui serez en embuscade derrière la ville ; ne vous éloignez pas beaucoup de la ville, mais tenez-vous prêts.
\VS{5}Et moi et tout le peuple qui est avec moi, nous nous approcherons de la ville. Et quand ils sortiront à notre rencontre, comme ils ont fait la première fois, nous nous enfuirons devant eux.
\VS{6}Ainsi ils sortiront après nous, jusqu'à ce que nous les ayons attirés hors de la ville ; car ils diront : Ils fuient devant nous comme la première fois ; parce que nous fuirons devant eux.
\VS{7}Alors vous vous lèverez de l'embuscade, et vous vous saisirez de la ville ; car Yahweh, votre Dieu, la livrera entre vos mains.
\VS{8}Et quand vous aurez pris la ville, vous y mettrez le feu ; vous agirez selon la parole de Yahweh. Regardez, je vous l'ai ordonné.
\VS{9}Josué donc les envoya, et ils allèrent se mettre en embuscade, et se tinrent entre Béthel et Aï, à l'occident d'Aï. Mais Josué resta cette nuit-là au milieu du peuple.
\VS{10}Puis Josué se leva de bon matin, et dénombra le peuple ; et il monta lui et les anciens d'Israël, devant le peuple vers Aï.
\VS{11}Et tout le peuple propre à la guerre qui étaient avec lui, monta et s'approcha ; et ils vinrent en face de la ville et campèrent du côté du nord d'Aï ; et la vallée était entre lui et Aï.
\VS{12}Il prit aussi environ cinq mille hommes, et les mit en embuscade entre Béthel et Aï, à l'occident de la ville.
\VS{13}Après que tout le camp eut pris position au nord de la ville, et l'embuscade à l'occident de la ville, cette nuit-là, Josué s'avança au milieu de la vallée.
\VS{14}Or il arriva qu'aussitôt que le roi de Aï l'eut vu, les hommes de la ville se hâtèrent, et se levèrent de bon matin, et au temps marqué, le Roi et tout son peuple sortirent à la campagne contre Israël pour le combattre. Or il ne savait pas qu'il y eût des gens en embuscade contre lui derrière la ville.
\VS{15}Alors Josué et tout Israël feignirent d'être battus devant eux, et ils s'enfuirent par le chemin du désert.
\VS{16}Alors tout le peuple qui était dans la ville d'Aï, fut assemblé à grand cri pour les poursuivre. Ils poursuivirent Josué, et ils furent ainsi attirés loin de la ville.
\VS{17}Il ne resta pas un seul homme dans Aï ni dans Béthel qui ne sortit contre Israël. Ils laissèrent la ville ouverte, et ils poursuivirent Israël.
\VS{18}Alors Yahweh dit à Josué : Etends vers Aï l'étandard qui est dans ta main, car je la livrerai entre tes mains. Et Josué étendit vers la ville l'étandard qui était dans sa main.
\VS{19}Aussitôt qu'il eut étendu sa main, les hommes qui étaient en embuscade se levèrent précipitamment du lieu où ils étaient ; ils pénétrèrent dans la ville, la prirent, et se hâtèrent de mettre le feu dans la ville.
\VS{20}Et les gens d'Aï, se tournant derrière eux, regardèrent ; et voici, la fumée de la ville montait vers le ciel, et ils n'y eut en eux aucune force pour fuir ça ou là. Et le peuple qui fuyait vers le désert se tourna contre ceux qui le poursuivaient ;
\VS{21}Et Josué et tout Israël, voyant que ceux qui étaient en embuscade avaient pris la ville, et que la fumée de la ville montait, se retournèrent, et frappèrent les gens d'Aï.
\VS{22}Les autres aussi sortirent de la ville contre eux, et ils furent enveloppés par les Israélites ayant les uns d'un côté et les autres de l'autre. Ils furent tellement battus qu'il n'en laissa aucun qui resta en vie ou qui échappât\FTNT{De 7:2.} ;
\VS{23}ils prirent aussi vivant le roi d'Aï, et le présentèrent à Josué.
\VS{24}Et quand les Israélites eurent achevé de tuer tous les habitants d'Aï dans la campagne, dans le désert, où ils les avaient poursuivis, et que tous furent tombés sous le tranchant de l'épée, jusqu'à être entièrement défaits, tous les Israélites revinrent vers Aï, et la frappèrent au tranchant de l'épée.
\VS{25}Et tous ceux qui tombèrent ce jour-là, tant des hommes que des femmes, furent au nombre de douze mille, tous gens d'Aï.
\VS{26}Et Josué ne retira point sa main qu'il tenait étendue avec l'étandard, jusqu'à ce que tous les habitants d'Aï aient été entièrement dévoués par le moyen de l'interdit.
\VS{27}Seulement les Israélites pillèrent pour eux les bêtes et le butin de cette ville-là, suivant ce que Yahweh avait prescrit à Josué\FTNT{No. 31:22-26.}.
\VS{28}Josué donc brûla Aï, et en fit un monceau perpétuel de ruines, jusqu'à aujourd'hui.
\VS{29}Puis il fit pendre le roi d'Aï à un arbre jusqu'au temps du soir. Et comme le soleil se couchait, Josué ordonna qu'on descende de l'arbre son cadavre ; on le jeta à l'entrée de la porte de la ville, puis on dressa sur lui un grand amas de pierres, qui subsiste encore aujourd'hui.
\TextTitle{Sacrifices offerts à Yahweh et lecture de la loi de Moïse}
\VS{30}Alors Josué bâtit un autel à Yahweh, le Dieu d'Israël, sur la montagne d'Ebal,
\VS{31}comme Moïse, serviteur de Yahweh, l'avait ordonné aux enfants d'Israël, ainsi qu'il est écrit dans le livre de la loi de Moïse : Il fit cet autel de pierres brutes sur lesquelles personne ne porta le fer\FTNT{L'autel devait être construit avec des pierres taillées par Dieu lui-même dans la nature (Ex. 20:25). L'Eglise du Seigneur est construite avec des pierres vivantes, taillées par Dieu et non par les hommes (Mt. 16:18). Babylone est construite avec des briques, œuvre des hommes (Ge. 11:1-3).} ; et ils offrirent dessus des holocaustes à Yahweh, et sacrifièrent des sacrifices d'offrande de paix\FTNT{Voir commentaire en Lé. 3:1.}.
\VS{32}Il écrivit aussi là, sur les pierres une copie de la loi que Moïse avait mise par écrit devant les enfants d'Israël.
\VS{33}Et tout Israël, ses anciens, ses officiers et ses juges étaient des deux côtés de l'arche, en face des prêtres qui sont de la race de Lévi, qui portaient l'arche de l'alliance de Yahweh, les étrangers comme les Hébreux naturels, une moitié du côté du mont Garizim\FTNT{Voir Jn. 4:19-24.}, et l'autre moitié du côté du mont Ebal, selon l'ordre qu'avait précédemment donné Moïse, serviteur de Yahweh, de bénir le peuple d'Israël.
\VS{34}Et après cela, il lut tout haut toutes les paroles de la loi, tant les bénédictions que les malédictions, selon tout ce qui est écrit dans le livre de la loi.
\VS{35}Il n'y eut rien de tout ce que Moïse avait prescrit, que Josué ne lise tout haut devant toute l'assemblée d'Israël, des femmes et des petits-enfants, et des étrangers qui marchaient au milieu d'eux.
\Chap{9}
\TextTitle{Josué tombe dans la ruse des Gabaonites}
\VerseOne{}Or, dès que tous les rois qui étaient au-delà du Jourdain, dans la montagne et dans la plaine, et sur toute la côte de la grande mer, jusque près du Liban, les Héthiens, les Amoréens, les Cananéens, les Phéréziens, les Héviens et les Jébusiens, eurent appris ces choses,
\VS{2}ils s'assemblèrent tous d'un commun accord pour faire la guerre à Josué et à Israël.
\VS{3}Mais les habitants de Gabaon\FTNT{Les Gabaonites étaient rusés. Ils poussèrent les Hébreux à faire alliance avec eux, comme le font les faux chrétiens aujourd'hui (Esd. 4 ; Es. 30:1). Il n'y a pas de rapport entre la lumière et les ténèbres (2 Co. 6:14-18). Combien de chrétiens ne se font-ils pas avoir par des loups ravisseurs dans le domaine du mariage ?}, ayant entendu ce que Josué avait fait à Jéricho et à Aï,
\VS{4}usèrent de ruse, car ils se mirent en chemin et contrefirent les ambassadeurs et prirent de vieux sacs pour leurs ânes, et de vieilles outres de vin déchirées et recousues,
\VS{5}Et ils avaient à leurs pieds de vieux souliers raccommodés et de vieux habits sur eux ; et tout le pain qu'ils avaient pour nourriture était sec et moisi.
\VS{6}Et ils arrivèrent auprès de Josué au camp de Guilgal, et lui dirent, ainsi qu'à tous les hommes d'Israël : Nous sommes venus d'un pays éloigné, maintenant donc traitez alliance avec nous.
\VS{7}Et les hommes d'Israël répondirent à ces Héviens : Peut-être que vous habitez au milieu de nous, et comment traiterions-nous alliance avec vous ?
\VS{8}Mais ils dirent à Josué : Nous sommes tes serviteurs. Alors Josué leur dit : Qui êtes-vous ? Et d'où venez-vous ?
\VS{9}Ils lui répondirent : Tes serviteurs sont venus d'un pays très éloigné, sur la renommée de Yahweh, ton Dieu ; car nous avons entendu sa renommée, et toutes les choses qu'il a faites en Egypte,
\VS{10}et tout ce qu'il a fait aux deux rois des Amoréens, qui étaient au-delà du Jourdain, Sihon, roi de Hesbon, et Og, roi de Basan, qui demeurait à Aschtaroth.
\VS{11}Et nos anciens et tous les habitants de notre pays nous ont dit : Prenez avec vous des provisions pour le chemin, et allez au-devant d'eux, et dites-leur : Nous sommes vos serviteurs, et maintenant traitez alliance avec nous.
\VS{12}Voci notre pain : Nous l'avons pris dans nos maisons tout chaud pour notre provision, le jour où nous sommes partis pour venir vers vous, mais maintenant voici, il est devenu sec et moisi.
\VS{13}Et voici aussi les outres de vin neuves que nous avons remplies, elles se sont déchirées ; nos habits et nos souliers sont usés à cause de la longueur de la marche.
\VS{14}Les hommes d'Israël prirent de leur provision, et aucun d'eux ne consulta la bouche de Yahweh\FTNT{Josué et les chefs ne consultèrent pas Yahweh avant de traiter alliance avec les Gabaonites. Prenez le temps dans la prière afin de connaître le cœur de la personne avec laquelle vous voulez marcher.}.
\VS{15}Car Josué fit la paix avec eux, et traita avec eux une alliance par laquelle il devait leur laisser la vie, et les chefs de l'assemblée le leur jurèrent.
\TextTitle{Les Gabaonites démasqués}
\VS{16}Mais il arriva, trois jours après l'alliance traitée avec eux, qu'ils apprirent que c'étaient leurs voisins et qu'ils habitaient parmi eux.
\VS{17}Car les enfants d'Israël partirent, et arrivèrent à leurs villes le troisième jour. Leurs villes étaient Gabaon, Kephira, Beéroth, et Kirjath-Jearim.
\VS{18}Et les enfants d'Israël ne les frappèrent point, parce que les chefs de l'assemblée leur avaient juré par Yahweh, le Dieu d'Israël. Mais toute l'assemblée murmura contre les chefs.
\VS{19}Alors tous les chefs dirent à toute l'assemblée : Nous leur avons juré par Yahweh, le Dieu d'Israël, c'est pourquoi maintenant nous ne pouvons pas les frapper.
\VS{20}Faisons-leur ceci, et qu'on les laisse vivre afin qu'il n'y ait pas de colère contre nous, à cause du serment que nous leur avons fait.
\VS{21}Ils vivront, leur dirent les chefs. Mais ils furent employés à couper le bois et à puiser l'eau pour toute l'assemblée, comme les chefs le leur avaient dit\FTNT{2 S. 21:1-14. La présence des Gabaonites en plein centre de Canaan tendait à isoler les tribus du nord de celles du sud, favorisant ainsi le schisme des deux royaumes (1 R. 12).}.
\VS{22}Car Josué les fit appeler, et leur parla, en disant : Pourquoi nous avez-vous trompés, en nous disant : Nous sommes très éloignés de vous, alors que vous habitez au milieu de nous ?
\VS{23}Maintenant vous êtes maudits ; il y aura toujours des esclaves parmi vous, des coupeurs de bois et des puiseurs d'eau pour la maison de mon Dieu.
\VS{24}Et ils répondirent à Josué, et dirent : Après qu'il ait été exactement rapporté à tes serviteurs les ordres que Yahweh, ton Dieu, avait ordonnés à Moïse, son serviteur, pour vous donner tout le pays et pour en exterminer tous les habitants devant vous ; nous avons extrêmement crains pour nos personnes à cause de vous et nous avons fait ceci. 
\VS{25}Et maintenant nous voici entre tes mains ; fais-nous comme il te semblera bon et juste de nous faire.
\VS{26}Il leur fit donc ainsi et il les délivra de la main des enfants d'Israël, de sorte qu'il ne les tuèrent point.
\VS{27}Et en ce jour-là, Josué les établit coupeurs de bois et puiseurs d'eau pour l'assemblée, et pour l'autel de Yahweh, jusqu'à aujourd'hui, dans le lieu qu'il choisirait.
\Chap{10}
\TextTitle{Josué secoure Gabaon des cinq rois des Amoréens}
\VerseOne{}Or quand Adoni-Tsédek, roi de Jérusalem, entendit que Josué avait pris Aï, et qu'il l'avait entièrement détruite par le moyen de l'interdit, ayant fait à Aï et à son roi, comme il avait fait à Jéricho et à son roi, et que les habitants de Gabaon avaient fait la paix avec Israël, et étaient au milieu d'eux.
\VS{2}Il eut une grande frayeur, parce que Gabaon était une grande ville, comme une ville royale, et elle était plus grande qu'Aï, et parce que tous ses hommes étaient vaillants.
\VS{3}C'est pourquoi Adoni-Tsédek, roi de Jérusalem, envoya dire à Hoham, roi d'Hébron, et à Piream, roi de Jarmuth, et à Japhia, roi de Lakis, et à Debir, roi d'Eglon :
\VS{4}Montez vers moi, et aidez-moi afin que nous frappions Gabaon, car elle a fait la paix avec Josué et avec les enfants d'Israël.
\VS{5}Ainsi cinq rois des Amoréens, savoir, le roi de Jérusalem, le roi d'Hébron, le roi de Jarmuth, le roi de Lakis, et le roi d'Eglon, s'assemblèrent et montèrent avec toutes leurs armées ; et ils campèrent près de Gabaon, et lui firent la guerre.
\VS{6}Alors les gens de Gabaon dirent à Josué au camp de Guilgal : Ne retire point tes mains de tes serviteurs, monte rapidement vers nous, délivre-nous, et donne-nous du secours ; car tous les rois des Amoréens qui habitent aux montagnes se sont rassemblés contre nous.
\VS{7}Josué donc monta de Guilgal, et avec lui tout le peuple qui était propre à la guerre, et tous les hommes forts et vaillants.
\TextTitle{Yahweh accorde à Israël une grande victoire à Makkéda}
\VS{8}Et Yahweh dit à Josué : Ne les crains point, car je les ai livré entre tes mains, et aucun d'eux ne tiendra devant toi.
\VS{9}Josué arriva subitement sur eux, après avoir marché toute la nuit depuis Guilgal.
\VS{10}Yahweh les mit en déroute devant Israël, qui en fit un grand carnage près de Gabaon, et les poursuivit par le chemin de la montagne de Beth-Horon, les battit jusqu'à Azéka, et jusqu'à Makkéda.
\VS{11}Et comme ils s'enfuyaient devant Israël, et qu'ils étaient à la descente de Beth-Horon, Yahweh fit tomber du ciel sur eux de grosses pierres jusqu'à Azéka, et ils périrent ; ceux qui moururent des pierres de grêle furent plus nombreux que ceux qui furent tués avec l'épée par les enfants d'Israël.
\VS{12}Alors Josué parla à Yahweh, le jour où Yahweh livra les Amoréens aux enfants d'Israël, et dit en présence d'Israël : Soleil, arrête-toi sur Gabaon, et toi lune, sur la vallée d'Ajalon !
\VS{13}Et le soleil s'arrêta, et la lune aussi s'arrêta, jusqu'à ce que le peuple ait tiré vengeance de ses ennemis. Cela n'est-il pas écrit dans le livre du Juste ? Le soleil s'arrêta au milieu du ciel et ne se hâta point de se coucher environ un jour entier\FTNT{Ha. 3:11.}.
\VS{14}Et il n'y a point eu de jour semblable à celui-là, ni avant ni après, où Yahweh exauça la voix d'un homme ; car Yahweh combattait pour Israël.
\VS{15}Et Josué, et tout Israël avec lui, retourna au camp à Guilgal.
\VS{16}Au reste, ces cinq rois restants s'enfuirent, et se cachèrent dans une caverne à Makkéda.
\VS{17}Et on le rapporta à Josué, en disant : On a trouvé les cinq rois cachés dans une caverne à Makkéda.
\VS{18}Et Josué dit : Roulez de grosses pierres à l'entrée de la caverne et mettez près d'elle quelques hommes pour les garder.
\VS{19}Mais vous, ne vous arrêtez pas, poursuivez vos ennemis, attaquez-les par-derrière jusqu'au dernier, ne les laissez pas entrer dans leurs villes, car Yahweh, votre Dieu, les a livrés entre vos mains.
\VS{20}Et quand Josué et les enfants d'Israël eurent achevé d'en faire une très grande boucherie, jusqu'à les détruire entièrement, ceux d'entre eux qui s'étaient échappés se retirèrent dans les villes fortifiées,
\VS{21}tout le peuple revint en paix au camp vers Josué à Makkéda, et personne ne remua sa langue contre les enfants d'Israël.
\VS{22}Alors Josué dit : Ouvrez l'entrée de la caverne, et amenez-moi ces cinq rois hors de la caverne.
\VS{23}Et ils firent ainsi, et ils lui amenèrent hors de la caverne ces cinq rois : Le roi de Jérusalem, le roi d'Hébron, le roi de Jarmuth, le roi de Lakis et le roi d'Eglon.
\VS{24}Et après qu'ils eurent amené à Josué ces cinq rois hors de la caverne, Josué appela tous les hommes d'Israël, et dit aux chefs des gens de guerre qui étaient allés avec lui : Approchez-vous, mettez vos pieds sur les cous de ces rois. Ils s'approchèrent, et mirent leurs pieds sur leurs cous\FTNT{Ps. 110:1.}.
\VS{25}Alors Josué leur dit : Ne craignez point, et ne soyez point effrayés, fortifiez-vous, et ayez du courage, car Yahweh traitera ainsi tous vos ennemis contre lesquels vous combattez.
\VS{26}Et après cela, Josué les frappa et les fit mourir, il les fit pendre à cinq arbres, et ils restèrent pendus à ces arbres jusqu'au soir.
\VS{27}Et comme le soleil se couchait, Josué ordonna qu'on les descende de ces arbres, et on les jeta dans la caverne où ils s'étaient cachés, et on mit à l'entrée de la caverne de grosses pierres qui y sont demeurées jusqu'à ce jour\FTNT{De. 21:23.}.
\VS{28}Josué prit aussi Makkéda le même jour, la frappa du tranchant de l'épée, et dévoua à la façon de l'interdit son roi et ses habitants, et ne laissa échapper personne qui était dans cette ville. Et il fit au roi de Makkéda comme il fait au roi de Jéricho.
\TextTitle{Conquête des territoires du sud}
\VS{29}Après cela, Josué, et tout Israël avec lui, passa de Makkéda à Libna, et fit la guerre à Libna.
\VS{30}Et Yahweh la livra aussi entre les mains d'Israël, avec son roi, et il la frappa du tranchant de l'épée, elle et tous ceux qui s'y trouvaient ; il n'en laissa échapper aucune personne qui était dans cette ville ; et il fit à son roi comme il avait fait au roi de Jéricho.
\VS{31}Ensuite Josué, et tout Israël avec lui, passa de Libna à Lakis, campa devant elle, et lui fit la guerre.
\VS{32}Et Yahweh livra Lakis entre les mains d'Israël, qui la prit le deuxième jour, et la frappa du tranchant de l'épée, et toutes les personnes qui s'y trouvaient, comme il avait fait à Libna.
\VS{33}Alors Horam, roi de Guézer, monta pour secourir Lakis. Josué le frappa, lui et son peuple, de sorte qu'il n'en laissa pas échapper un seul homme.
\VS{34}Après cela Josué, et tout Israël avec lui, passa de Lakis à Eglon ; ils campèrent devant elle, et lui firent la guerre.
\VS{35}Ils la prirent le jour même, la frappèrent du tranchant de l'épée ; et Josué dévoua à la façon de l'interdit ce jour-là toutes les personnes qui y étaient, comme il avait fait à Lakis.
\VS{36}Puis Josué, et tout Israël avec lui, monta d'Eglon à Hébron, et ils lui firent la guerre.
\VS{37}Et ils la prirent, et la frappèrent du tranchant de l'épée, avec son roi, toutes ses villes, et toutes les personnes qui y étaient ; il n'en laissa échapper aucune, comme il avait fait à Eglon ; et il dévoua à la façon de l'interdit, toutes les personnes qui y étaient.
\VS{38}Ensuite Josué, et tout Israël avec lui, retourna vers Debir, et ils lui firent la guerre.
\VS{39}Et il la prit, avec son roi et toutes ses villes ; et ils les frappèrent du tranchant de l'épée, et dévouèrent à la façon de l'interdit toutes les personnes qui y étaient ; il n'en laissa échapper aucune. Il fait à Debir et son roi comme il avait fait à Hébron, et comme il avait fait à Libna et à son roi.
\VS{40}Josué donc frappa tout ce pays, la montagne et le midi, la plaine et les coteaux, et tous leurs rois ; il n'en laissa échapper aucun, et il dévoua par le moyen de l'interdit toutes les personnes qui y respiraient, comme Yahweh, le Dieu d'Israël, l'avait ordonné\FTNT{De. 20:16-17.}.
\VS{41}Ainsi Josué les battit depuis Kadès-Barnéa jusqu'à Gaza, et tout le pays de Gosen jusqu'à Gabaon.
\VS{42}Josué prit tous ces rois en même temps et leur pays, parce que Yahweh, le Dieu d'Israël, combattait pour Israël.
\VS{43}Après quoi Josué, et tout Israël avec lui, retourna au camp à Guilgal.
\Chap{11}
\TextTitle{Conquête des territoires du nord}
\VerseOne{}Et aussitôt que Jabin, roi de Hatsor, eut appris ces choses, il envoya des messagers à Jobab, roi de Madon, au roi de Schimron, et au roi d'Acschaph,
\VS{2}et aux rois qui habitaient vers le nord, aux montagnes et dans la plaine, vers le midi de Kinnéreth, dans la vallée, et sur les hauteurs de Dor vers l'occident,
\VS{3}aux Cananéens qui étaient à l'orient et à l'occident, aux Amoréens, aux Héthiens, aux Phéréziens, aux Jébusiens dans les montagnes, et aux Héviens au pied de la montagne de l'Hermon, dans le pays de Mitspa.
\VS{4}Ils sortirent donc avec toutes leurs armées, un grand peuple par leur grand nombre, comme le sable qui est sur le bord de la mer, il y avait aussi des chevaux et des chars en très grand nombre.
\VS{5}Tous ces rois se réunirent, et campèrent ensemble près des eaux de Mérom, pour combattre contre Israël.
\VS{6}Et Yahweh dit à Josué : Ne les crains point, car demain, à cette même heure, je les livrerai tous, blessés à mort, devant Israël. Tu couperas les jarrets à leurs chevaux, et brûleras au feu leurs chars\FTNT{2 S. 8:4.}.
\VS{7}Josué donc, et tous les gens de guerre avec lui vinrent subitement sur eux près des eaux de Mérom, et ils se précipitèrent au milieu d'eux.
\VS{8}Et Yahweh les livra entre les mains d'Israël ; ils les battirent, et les poursuivirent jusqu'à Sidon la grande, jusqu'aux eaux de Misrephoth-Maïm, et jusqu'à la vallée de Mitspa vers l'orient, et ils les battirent tellement qu'ils ne laissèrent aucun survivant.
\VS{9}Et Josué leur fit comme Yahweh lui avait dit ; il coupa les jarrets de leurs chevaux, et brûla au feu leurs chars.
\VS{10}A son retour, et dans le même temps, Josué prit Hatsor, et frappa son roi avec l'épée ; car Hatsor avait été auparavant la capitale de tous ces royaumes.
\VS{11}On frappa aussi du tranchant de l'épée et l'on dévoua à la façon de l'interdit tous ceux qui s'y trouvaient, il ne resta rien de ce qui respirait, et l'on brûla au feu Hatsor.
\VS{12}Josué prit aussi toutes les villes de ces rois, et tous leurs rois, et les frappa du tranchant de l'épée, et il les dévoua à la façon de l'interdit, comme Moïse, serviteur de Yahweh, l'avait ordonné.
\VS{13}Mais Israël ne brûla aucune des villes situées sur des collines, excepté de Hatsor seule, que Josué brûla.
\VS{14}Et les enfants d'Israël pillèrent pour eux tout le butin de ces villes et le bétail ; mais ils frappèrent du tranchant de l'épée tous les hommes, jusqu'à ce qu'ils les aient exterminés, ils n'y laissèrent aucun qui respirait.
\VS{15}Comme Yahweh l'avait ordonné à Moïse son serviteur, ainsi Moïse l'avait ordonné à Josué ; et Josué le fit ainsi ; de sorte qu'il n'omit rien de tout ce que Yahweh avait ordonné à Moïse. 
\TextTitle{Josué s'empare de tout le pays}
\VS{16}Josué donc prit tout ce pays-là, la montagne et tout le pays du midi, avec tout le pays de Gosen, la vallée et la plaine, la montagne d'Israël et ses vallées.
\VS{17}Depuis la montagne de Halak, qui s'élève vers Séir, jusqu'à Baal-Gad dans la vallée du Liban, au pied de la montagne d'Hermon. Il prit aussi tous leurs rois, les battit et les fit mourir.
\VS{18} Josué fit la guerre plusieurs jours contre tous ces rois.
\VS{19}Il n'y eut aucune ville qui fit la paix avec les enfants d'Israël, excepté les Héviens qui habitaient à Gabaon ; ils les prirent toutes par la guerre.
\VS{20}Car cela venait de Yahweh, qu'ils endurcissent leur cœur pour qu'ils sortent en bataille contre Israël, afin qu'il les dévoue à la façon de l'interdit, sans qu'il y ait pour eux de miséricorde, et qu'il les extermine, comme Yahweh l'avait ordonné à Moïse\FTNT{Ex. 4:21 ; De. 2:30 ; 1 R. 12:15.}.
\VS{21}En ce même temps-là aussi, Josué se mit en marche, et il extermina les Anakim des montagnes d'Hébron, de Debir, d'Anab, et de toute la montagne de Juda, et de toute la montagne d'Israël ; Josué, dis-je, les dévoua à la façon de l'interdit avec leurs villes.
\VS{22}Il ne resta aucun Anakim dans le pays des enfants d'Israël ; il n'en resta seulement qu'à Gaza, à Gath et à Asdod\FTNT{2 S. 21:20.}.
\VS{23}Josué donc prit tout le pays, suivant tout ce que Yahweh avait dit à Moïse. Et Josué le donna en héritage à Israël, selon leurs portions, et leurs tribus. Et le pays fut en repos et sans avoir guerre.
\Chap{12}
\TextTitle{Liste des rois vaincus par Moïse et Josué}
\VerseOne{}Voici les rois du pays que les enfants d'Israël frappèrent, et dont ils possédèrent le pays de l'autre côté du Jourdain, vers l'orient, depuis le torrent de l'Arnon jusqu'à la montagne de l'Hermon, et toute la plaine vers l'orient.
\VS{2}Savoir, Sihon, roi des Amoréens, qui habitait à Hesbon, et qui dominait depuis Aroër, qui est sur le bord du torrent de l'Arnon, et depuis le milieu du torrent, sur la moitié de Galaad, jusqu'au torrent de Jabbok, qui est la frontière des enfants d'Ammon\FTNT{De. 3:8-16.} ;
\VS{3}et depuis la plaine jusqu'à la mer de Kinnéreth vers l'orient, et jusqu'à la mer de la plaine, qui est la mer salée, vers l'orient, au chemin de Beth-Jeschimoth ; et depuis le midi sur le pied du Pisga.
\VS{4}Et les contrées d'Og, roi de Basan, qui était seul reste des Rephaïm, et qui habitait à Aschtaroth et à Edréï.
\VS{5}Et sa domination s'étendait sur la montagne de l'Hermon, sur Salca, et sur tout Basan, jusqu'à la frontière des Gueschuriens et des Maacathiens, et sur la moitié de Galaad, frontière de Sihon, roi de Hesbon.
\VS{6}Moïse, serviteur de Yahweh, et les enfants d'Israël, les battirent ; et Moïse, serviteur de Yahweh, en donna la possession aux Rubénites, aux Gadites, et à la demi-tribu de Manassé\FTNT{No. 32:33.}.
\VS{7}Voici les rois du pays que Josué et les enfants d'Israël frappèrent de ce côté-ci du Jourdain vers l'occident, depuis Baal-Gad, dans la vallée du Liban, jusqu'à la montagne de Halak qui monte vers Séir, et que Josué donna aux tribus d'Israël en possession, selon leurs portions,
\VS{8}pays consistant en montagnes et en vallées, en plaines et en collines, en pays de désert et de midi: Les Héthiens, les Amoréens, les Cananéens, les Phéréziens, les Héviens et les Jébusiens.
\VS{9}Le roi de Jéricho, un ; le roi d'Aï, près de Béthel, un ;
\VS{10}le roi de Jérusalem, un ; le roi d'Hébron, un ;
\VS{11}le roi de Jarmuth, un ; le roi de Lakis, un ;
\VS{12}le roi d'Eglon, un ; le roi de Guézer, un ;
\VS{13}le roi de Debir, un ; le roi de Guéder, un ;
\VS{14}le roi de Horma, un ; le roi d'Arad, un ;
\VS{15}le roi de Libna, un ; le roi d'Adullam, un ;
\VS{16}le roi de Makkéda, un ; le roi de Béthel, un ;
\VS{17}le roi de Tappuach, un ; le roi de Hépher, un ;
\VS{18}le roi d'Aphek, un ; le roi de Lascharon, un ;
\VS{19}le roi de Madon, un ; le roi de Hatsor, un ;
\VS{20}le roi de Schimron-Meron, un ; le roi d'Acschaph, un ;
\VS{21}le roi de Taanac, un ; le roi de Meguiddo, un ;
\VS{22}le roi de Kédesch, un ; le roi de Jokneam, au Carmel, un ;
\VS{23}le roi de Dor, sur les hauteurs de Dor, un ; le roi de Gojim, près de Guilgal, un ;
\VS{24}le roi de Thirtsa, un ; en tout trente et un rois.
\Chap{13}
\TextTitle{Les territoires de Ruben, de Gad et de la demi-tribu de Manassé}
\VerseOne{}Or, quand Josué fut devenu vieux, fort avancé en âge, Yahweh lui dit : Tu es devenu vieux, fort avancé en âge, et il te reste encore un très grand pays à posséder.
\VS{2}Voici le pays qui reste, toutes les contrées des Philistins, et des Gueschuriens,
\VS{3}depuis le Schichor, qui coule devant l'Egypte, jusqu'à la frontière d'Ekron au nord, contrée qui doit être tenue pour Cananéenne, et qui est occupée par les cinq princes des Philistins, celui de Gaza, celui d'Asdod, celui d'Askalon, celui de Gath, celui d'Ekron, et par les Avviens ;
\VS{4}du côté du midi, tout le pays des Cananéens, et Meara qui est aux Sidoniens, jusqu'à Aphek, jusqu'à la frontière des Amoréens ;
\VS{5}le pays qui appartient aux Guibliens, et tout le Liban, vers l'orient, depuis Baal-Gad, au pied de la montagne d'Hermon, jusqu'à l'entrée de Hamath ;
\VS{6}tous les habitants de la montagne, depuis le Liban jusqu'aux eaux de Misrephoth-Maïm, tous les Sidoniens. Je les chasserai moi-même devant les fils d'Israël. Donne seulement ce pays en héritage par le sort à Israël, comme je te l'ai prescrit.
\VS{7}Maintenant donc divise ce pays en héritage aux neuf tribus, et à la demi-tribu de Manassé.
\VS{8}Avec l'autre moitié de laquelle les Rubénites et les Gadites ont pris leur héritage, lequel Moïse leur a donné au delà du Jourdain, vers l'orient, selon que Moïse, serviteur de Yahweh, le leur a donné ;
\VS{9}depuis Aroër, qui est sur le bord du torrent de l'Arnon, et la ville qui est au milieu de la vallée, et toute la plaine de Médeba, jusqu'à Dibon ;
\VS{10}et toutes les villes de Sihon, roi des Amoréens, qui régnait à Hesbon, jusqu'à la frontière des enfants d'Ammon ;
\VS{11}et Galaad, et les territoires des Gueschuriens et des Maacathiens, toute la montagne de l'Hermon, et tout Basan jusqu'à Salca ;
\VS{12}tout le royaume d'Og en Basan, qui régnait à Aschtaroth, et à Edréï, et qui était resté le seul reste des Rephaïm ; Moïse battit ces rois, et les chassa.
\VS{13}Or les fils d'Israël ne chassèrent point les Gueschuriens et les Maacathiens, mais les Gueschuriens et les Maacathiens ont habité au milieu d'Israël jusqu'à ce jour.
\VS{14}Seulement il ne donna point d'héritage à la tribu de Lévi ; les sacrifices consumés par le feu devant Yahweh, le Dieu d'Israël, tel fut son héritage, comme il le lui avait dit\FTNT{No. 18:20-24 ; De. 10:9 ; De. 18:2 ; Ez. 44:28.}.
\VS{15}Moïse donc donna un héritage à la tribu des fils de Ruben selon leurs familles.
\VS{16}Et leurs frontières furent depuis Aroër qui est sur le bord du torrent d'Arnon, et de la ville qui est au milieu du torrent, et toute la plaine qui est près de Médeba.
\VS{17}Hesbon et toutes ses villes, qui étaient dans la plaine, Dibon, Bamoth-Baal, Beth-Baal-Meon,
\VS{18}Jahats, Kedémoth et Méphaath,
\VS{19}Kirjathaïm, Sibma, Tséreth-Haschachar sur la montagne de la vallée,
\VS{20}Beth-Peor, les coteaux du Pisga et Beth-Jeschimoth,
\VS{21}et toutes les villes de la plaine, et tout le royaume de Sihon, roi des Amoréens qui régnait à Hesbon ; Moïse l'avait battu, lui et les princes de Madian, Evi, Rékem, Tsur, Hur, et Réba, princes qui relevaient de Sihon, et qui habitaient dans le pays.
\VS{22}Les enfants d'Israël firent passer aussi par l'épée Balaam\FTNT{Voir No. 22. Balaam était l'exemple type du prophète corrompu, soucieux de tirer profit de son service.}, fils de Beor, le devin, avec les autres qui y furent tués.
\VS{23}Et les frontières des enfants d'Israël fut le Jourdain et sa frontière. Tel fut l'héritage des fils de Ruben, selon leurs familles ; savoir, ces villes-là et leurs villages\FTNT{No. 34:14-15.}.
\VS{24}Moïse donna aussi un héritage à la tribu de Gad, pour les fils de Gad, selon leurs familles.
\VS{25}Et leur pays fut Jaezer, et toutes les villes de Galaad et la moitié du pays des enfants d'Ammon, jusqu'à Aroër, qui est vis-à-vis de Rabba,
\VS{26}et depuis Hesbon jusqu'à Ramath-Mitspé, et Bethonim, et depuis Mahanaïm jusqu'à la frontière de Debir,
\VS{27}et, dans la vallée, Beth-Haram, Beth-Nimra, Succoth et Tsaphon, reste du royaume de Sihon, roi de Hesbon, ayant le Jourdain pour frontière jusqu'à l'extrémité de la mer de Kinnéreth, de l'autre côté du Jourdain, vers l'orient.
\VS{28}Tel fut l'héritage des fils de Gad, selon leurs familles ; savoir, les villes et leurs villages.
\VS{29}Moïse donna aussi à la demi-tribu de Manassé un héritage, qui est resté à la demi-tribu des fils de Manassé, selon leurs familles.
\VS{30}Leur pays fut depuis Mahanaïm, tout Basan, et tout le royaume d'Og, roi de Basan, et tous les villages de Jaïr qui sont en Basan, soixante villes.
\VS{31}Et la moitié de Galaad, Aschtaroth et Edréï, villes du royaume d'Og en Basan, furent aux fils de Makir, fils de Manassé, à la moitié des enfants de Makir, selon leurs familles.
\VS{32}Ce sont là les pays que Moïse avait donnés en héritage, lorsqu'il était dans les plaines de Moab, de l'autre côté du Jourdain, vis-à-vis de Jéricho, à l'orient.
\VS{33}Mais Moïse ne donna point d'héritage à la tribu de Lévi ; car Yahweh, le Dieu d'Israël, fut leur héritage, comme il le lui avait dit.
\Chap{14}
\TextTitle{Caleb reçoit Hébron}
\VerseOne{}Voici les terres que les enfants d'Israël eurent pour héritage dans le pays de Canaan, ce que partagèrent entre eux le prêtre Eléazar, Josué, fils de Nun, et les chefs des familles des tribus des enfants d'Israël.
\VS{2}Selon le sort de leur héritage ; comme Yahweh l'avait ordonné par le moyen de Moïse ; savoir, à neuf tribus et à la demi-tribu\FTNT{No. 26:55.}.
\VS{3}Car Moïse avait donné un héritage aux deux tribus et à la demi-tribu de l'autre côté du Jourdain, mais il n'avait point donné de part aux Lévites parmi eux.
\VS{4}Parce que les fils de Joseph, savoir, Manassé et Ephraïm, formaient deux tribus ; et l'on ne donna point de part aux Lévites dans le pays, excepté des villes pour habitation, et les faubourgs pour leurs troupeaux, et pour le reste de leurs biens.
\VS{5}Les enfants d'Israël firent comme Yahweh l'avait ordonné à Moïse, et ils partagèrent le pays.
\VS{6}Or les fils de Juda s'approchèrent de Josué à Guilgal ; et Caleb, fils de Jephunné, le Kenizien, lui dit : Tu sais la parole que Yahweh a déclarée à Moïse, homme de Dieu, à mon sujet et au tien à Kadès-Barnéa\FTNT{No. 14:24 ; No. 32:12 ; De. 1:36.}.
\VS{7}J'étais âgé de quarante ans quand Moïse, serviteur de Yahweh, m'envoya à Kadès-Barnéa pour espionner le pays, et je lui fis un rapport avec droiture de cœur.
\VS{8}Et mes frères qui étaient montés avec moi découragèrent le cœur du peuple, mais moi je persévérai à suivre Yahweh, mon Dieu.
\VS{9}Et ce jour-là Moïse jura, en disant : La terre que ton pied a foulée sera ton héritage à perpétuité, pour toi et pour tes fils, parce que tu as persévéré à suivre Yahweh, mon Dieu.
\VS{10}Or maintenant voici, Yahweh m'a fait vivre comme il l'a dit. Il y a déjà quarante-cinq ans que Yahweh déclarait cette parole à Moïse, lorsqu'Israël marchait dans le désert. Et maintenant voici, je suis aujourd'hui âgé de quatre-vingt-cinq ans.
\VS{11}Et je suis encore aujourd'hui aussi vigoureux que j'étais le jour où Moïse m'envoya ; et j'ai maintenant la même force que j'avais alors pour le combat, soit pour sortir et pour entrer.
\VS{12}Maintenant, donne-moi donc cette montagne, dont Yahweh a parlé ce jour-là ; car tu as appris en ce jour qu'il s'y trouve des Anakim, et qu'il y a de grandes villes fortifiées. Yahweh sera peut-être avec moi, et je les chasserai, comme Yahweh a dit.
\VS{13}Josué donc bénit Caleb, fils de Jephunné, et lui donna Hébron pour héritage.
\VS{14}C'est ainsi que Caleb, fils de Jephunné, le Kenizien, a eu jusqu'à ce jour Hébron pour héritage, parce qu'il avait persévéré à suivre Yahweh, le Dieu d'Israël.
\VS{15}Or Hébron s'appelait autrefois Kirjath-Arba ; et Arba avait été le plus grand homme parmi les Anakim. Le pays fut en repos et sans guerre.
\Chap{15}
\TextTitle{Le territoire de Juda}
\VerseOne{} Ce sont ici la part échue par le sort à la tribu des enfants de Juda, selon leurs familles ; à la frontière d'Edom, au désert de Tsin, vers le midi, fut la dernière extrémité de leurs pays vers le midi ;
\VS{2}tellement que leur frontière, du côté du midi, fut la dernière extrémité de la mer Salée, depuis le bras qui regarde vers le midi. 
\VS{3}Et elle devait sortir vers le midi de la montée d'Akrabbim, et passer vers Tsin ; et montant du midi de Kadès-Barnéa, passer à Hetsron ; puis montant vers Addar, se tourner vers Karkaa ; 
\VS{4}puis, passant vers Atsmon, sortir au torrent d'Egypte ; tellement que les extrémités de cette frontière devaient se rendre à la mer. Ce sera là, dit Josué, votre frontière, du côté du midi.
\VS{5}Et la frontière vers l'orient était la mer salée jusqu'à l'embouchure du Jourdain. La frontière du côté du nord sera depuis la langue de mer, qui est à l'embouchure du Jourdain.
\VS{6}Et cette frontière montera jusqu'à Beth-Hogla, et passera du côté du nord de Beth-Araba ; et cette frontière montera jusqu'à la pierre de Bohan, fils de Ruben.
\VS{7}Puis cette frontière montera vers Debir, depuis la vallée d'Acor, et même vers le nord, du côté de Guilgal, qui est vis-à-vis de la montée d'Adummim, au sud du torrent. Puis cette frontière passera près des eaux d'En-Schémesch, et ses extrémités se prolongeront à En-Roguel.
\VS{8}Puis cette frontière montera de là par la vallée de Ben-Hinnom, au côté du midi de Jebus, qui est Jérusalem, puis cette frontière montera jusqu'au sommet de la montagne, qui est vis-à-vis de la vallée de Hinnom, à l'occident, et à l'extrémité de la vallée des Rephaïm, au nord.
\VS{9}Et cette frontière s'alignera, depuis le sommet de la montagne jusqu'à la source des eaux de Nephthoach, et continuera vers les villes de la montagne d'Ephron, puis cette frontière s'alignera à Baala, qui est Kirjath-Jearim.
\VS{10}Et cette frontière se tournera depuis Baala, vers l'occident, jusqu'à la montagne de Séir, puis elle traversera le côté nord de la montagne de Jearim, à Kesalon, puis descendait à Beth-Schémesch, et passera par Thimna.
\VS{11}Et cette frontière sortira jusqu'au côté d'Ekron, vers le nord et cette frontière s'alignera vers Schicron, puis ayant passé la montagne de Baala, elle se sortira jusqu'à Jabneel ; tellement que les extrémités de cette frontière se rendront à la mer. 
\VS{12}Or la frontière du côté de l'occident sera ce qui est vers la grande mer et ses limites. Telles furent de tous les côtés les frontières des fils de Juda, selon leurs familles.
\VS{13}Au reste, on donna à Caleb, fils de Jephunné, une part au milieu des fils de Juda, comme Yahweh l'avait ordonné à Josué ;savoir, Kirjath-Arba, or Arba était père d'Anak ; et Kirjath-Arba c'est Hébron.
\VS{14}Et Caleb chassa de là les trois fils d'Anak : Schéschaï, Ahiman, et Talmaï, fils d'Anak.
\VS{15}Et de là il monta contre les habitants de Debir ; Debir s'appelait autrefois Kirjath-Sépher.
\VS{16}Et Caleb dit : Je donnerai ma fille Acsa pour femme à celui qui battra Kirjath-Sépher, et la prendra\FTNT{Jg. 1:12-14.}.
\VS{17}Et Othniel, fils de Kenaz, frère de Caleb, la prit ; et Caleb lui donna sa fille Acsa pour femme.
\VS{18}Et il arriva que comme elle s'en allait, elle l'incita à demander à son père un champ ; puis elle descendit impétueusement de dessus son âne, et Caleb lui dit : Qu'as-tu ? 
\VS{19}Elle répondit : Donne-moi un présent, puisque tu m'as donné une terre du sud, donne-moi aussi des sources d'eau. Et il lui donna les sources supérieures et les sources inférieures.
\VS{20}Tel fut l'héritage de la tribu des fils de Juda, selon leurs familles.
\VS{21}Les villes situées dans la contrée du midi, à l'extrémité de la tribu des fils de Juda, près de la frontière d'Edom, étaient : Kabtseel, Eder, Jagur,
\VS{22}Kina, Dimona, Adada,
\VS{23}Kédesch, Hatsor, Ithnan,
\VS{24}Ziph, Thélem, Bealoth,
\VS{25}Hatsor-Hadattha, Kerijoth-Hetsron qui est Hatsor,
\VS{26}Amam, Schema, Molada,
\VS{27}Hatsar-Gadda, Heschmon, Beth-Paleth,
\VS{28}Hatsar-Schual, Beer-Schéba, Bizjothja,
\VS{29}Baala, Ijjim, Atsem,
\VS{30}Eltholad, Kesil, Horma,
\VS{31}Tsiklag, Madmanna, Sansanna,
\VS{32}Lebaoth, Schilhim, Aïn et Rimmon. Total des villes : Vingt-neuf villes, et leurs villages.
\VS{33}Dans la plaine : Eschthaol, Tsorea, Aschna,
\VS{34}Zanoach, En-Gannim, Tappuach, Enam,
\VS{35}Jarmuth, Adullam, Soco, Azéka,
\VS{36}Schaaraïm, Adithaïm, Guedéra et Guedérothaïm ; quatorze villes, et leurs villages.
\VS{37}Tsenan, Hadascha, Migdal-Gad,
\VS{38}Dilean, Mitspé, Joktheel,
\VS{39}Lakis, Botskath, Eglon,
\VS{40}Cabbon, Lachmas, Kithlisch,
\VS{41}Guedéroth, Beth-Dagon, Naama, et Makkéda ; seize villes, et leurs villages.
\VS{42}Libna, Ether, Aschan,
\VS{43}Jiphtach, Aschna, Netsib,
\VS{44}Keïla, Aczib et Maréscha ; neuf villes, et leurs villages.
\VS{45}Ekron, et les villes de son ressort, et ses villages.
\VS{46}Depuis Ekron et à l'occident, toutes les villes près d'Asdod, et leurs villages.
\VS{47}Asdod, les villes de son ressort, et ses villages, Gaza, les villes de son ressort, et ses villages, jusqu'au torrent d'Egypte, et à la grande mer, qui sert de limite.
\VS{48}Dans la montagne : Schamir, Jatthir, Soco,
\VS{49}Danna, Kirjath-Sanna, qui est Debir,
\VS{50}Anab, Eschthemo, Anim,
\VS{51}Gosen, Holon, et Guilo ; onze villes et leurs villages.
\VS{52}Arab, Duma, Eschean,
\VS{53}Janum, Beth-Tappuach, Aphéka,
\VS{54}Humta, Kirjath-Arba, qui est Hébron, et Tsior ; neuf villes, et leurs villages.
\VS{55}Maon, Carmel, Ziph, Juta,
\VS{56}Jizreel, Jokdeam, Zanoach,
\VS{57}Kaïn, Guibea, et Thimna ; dix villes, et leurs villages.
\VS{58}Halhul, Beth-Tsur, Guedor,
\VS{59}Maarath, Beth-Anoth, et Elthekon ; six villes, et leurs villages.
\VS{60}Kirjath-Baal, qui est Kirjath-Jearim, et Rabba ; deux villes, et leurs villages.
\VS{61}Au désert : Beth-Araba, Middin, Secaca,
\VS{62}Nibschan, Ir-Hammélach, et En-Guédi : Six villes et leurs villages.
\VS{63}Au reste, les fils de Juda ne purent pas chasser les Jébusiens qui habitaient à Jérusalem, c'est pourquoi les Jébusiens ont habité avec les fils de Juda à Jérusalem jusqu'à ce jour.
\Chap{16}
\TextTitle{Le territoire d'Ephraïm}
\VerseOne{}La part échue par le sort aux fils de Joseph depuis le Jourdain près de Jéricho, aux eaux de Jéricho, vers l'orient qui est le désert ; montant de Jéricho par la montagne jusqu'à Béthel.
\VS{2}Et cette frontière devait sortir de Béthel à Luz, puis passer vers la frontière des Arkiens jusqu'à Atharoth.
\VS{3}Et elle devait descendre tirant vers l'occident, vers la frontière des Japhléthiens, jusqu'à celle de Beth-Horon la basse et jusqu'à Guézer, de sorte que ses extrémités aboutissent à la mer.
\VS{4}Ainsi les fils de Joseph, savoir, Manassé et Ephraïm, reçurent leur héritage.
\VS{5}Or la frontière des fils d'Ephraïm, selon leurs familles, la frontière de leur héritage était à l'orient, Atharoth-Addar, jusqu'à Beth-Horon la haute.
\VS{6}Et cette frontière devait sortir vers la mer à Micmethath, du côté du nord ; et cette frontière devait se tourner vers l'orient jusqu'à Thaanath-Silo, et passant du côté d'orient, se rendre à Janoach.
\VS{7}Puis descendre de Janoach à Atharoth et à Naaratha, se rencontrer à Jéricho, et sortir au Jourdain.
\VS{8}Et cette frontière devait aller de Tappuach, vers l'occident, jusqu'au torrent de Kana, tellement que ses extrémités devaient se rendre à la mer. Ce fut là l'héritage de la tribu des fils d'Ephraïm, selon leurs familles.
\VS{9}Les fils d'Ephraïm avaient aussi des villes séparées au milieu de l'héritage des fils de Manassé, toutes ces villes, avec leurs villages.
\VS{10}Or ils ne chassèrent point les Cananéens qui habitaient à Guézer, c'est pourquoi les Cananéens ont habité parmi Ephraïm jusqu'à ce jour, mais ils furent réduits à la servitude et assujettis à un tribut\FTNT{Jg. 1:29 ; 1 R. 9:16.}.
\Chap{17}
\TextTitle{Le territoire de Manassé}
\VerseOne{}Il y eut aussi une part échut par le sort à la tribu de Manassé qui était le premier-né de Joseph. Quant à Makir, premier-né de Manassé, et père de Galaad, il avait eu Galaad et Basan parce qu'il était un homme de guerre.
\VS{2}Puis on jeta donc le sort pour les autres enfants de Manassé, selon ses familles ; aux fils d'Abiézer, aux fils de Hélek, aux fils d'Asriel, aux fils de Sichem, aux fils de Hépher, et aux fils de Schemida. Ce sont là les enfants mâles de Manassé fils de Joseph, selon leurs familles.
\VS{3}Or Tselophchad, fils de Hépher, fils de Galaad, fils de Makir, fils de Manassé, n'eut point de fils, mais il eut des filles dont voici les noms : Machla, Noa, Hogla, Milca et Thirtsa.
\VS{4}Elles vinrent se présenter devant le prêtre Eléazar, devant Josué, fils de Nun, et devant les princes, en disant : Yahweh a ordonné à Moïse de nous donner un héritage parmi nos frères. C'est pourquoi on leur donna un héritage parmi les frères de leur père, selon l'ordre de Yahweh\FTNT{No. 27:7 ; No. 36:2.}.
\VS{5}Et dix portions échurent à Manassé, outre le pays de Galaad et de Basan, qui est de l'autre côté du Jourdain.
\VS{6}Car les filles de Manassé eurent un héritage parmi ses fils, et le pays de Galaad fut pour les autres des fils de Manassé.
\VS{7}Or la frontière de Manassé fut du côté d'Aser, venant à Micmethath, qui est près de Sichem ; puis cette frontière devait aller à main droite vers les habitants d'En-Tappuach.
\VS{8}Or le pays de Tappuach appartenait à Manassé, mais Tappuach qui était près de la frontière de Manassé, appartenait aux fils d'Ephraïm.
\VS{9}De là, cette frontière devait descendre au torrent de Kana, au midi du torrent. Ces villes étaient à Ephraïm parmi les villes de Manassé. La frontière de Manassé était au côté du nord du torrent, et ses extrémités devaient se rendre à la mer.
\VS{10}Ce qui était vers le midi était à Ephraïm, et celui qui était vers le nord était à Manassé, et la mer leur servait de frontière ; et du côté du nord, les frontières se rencontraient à Aser, à Issacar, vers l'orient.
\VS{11}Car Manassé possédait dans Issacar et dans Aser : Beth-Schean et les villes de son ressort, Jibleam et les villes de son ressort, les habitants de Dor et les villes de son ressort, les habitants d'En-Dor, et les villes de son ressort, les habitants de Thaanac et les villes de son ressort, les habitants de Meguiddo et les villes de son ressort, qui sont trois contrées.
\VS{12}Au reste, les fils de Manassé ne purent pas chasser les habitants de ces villes, et les Cananéens voulurent rester dans le même pays.
\VS{13}Mais lorsque les fils d'Israël furent assez forts, ils assujettirent les Cananéens à un tribut, mais ils ne les chassèrent pas entièrement.
\VS{14}Or les fils de Joseph parlèrent à Josué, et dirent : Pourquoi nous as-tu donné en héritage un seul lot, et une seule part, vu que nous sommes un peuple nombreux, et que Yahweh nous a bénis jusqu'à présent ?
\VS{15}Et Josué leur dit : Si vous êtes un peuple nombreux, montez à la forêt, et vous l'abattrez, pour vous y faire de la place dans le pays des Phéréziens et des Rephaïm, si la montagne d'Ephraïm est trop étroite pour vous.
\VS{16}Et les fils de Joseph répondirent : Cette montagne ne sera pas suffisante pour nous, et tous les Cananéens qui habitent la vallée ont des chars de fer, et ceux qui sont à Beth-Schean, et dans les villes de son ressort, et ceux qui habitent dans la vallée de Jizreel\FTNT{Jg. 1:19 ; Jg. 4:3.}.
\VS{17}Donc Josué parla à la maison de Joseph, à Ephraïm et à Manassé, et dit : Vous êtes un peuple nombreux, et vous avez de grandes forces, vous n'aurez pas qu'une seule part.
\VS{18}Mais vous aurez la montagne, car c'est une forêt que vous abattrez et dont les extrémités vous appartiendront, et vous chasserez les Cananéens, quoiqu'ils aient des chars de fer, et qu'ils soient puissants.
\Chap{18}
\TextTitle{La tente d'assignation à Silo}
\VerseOne{}Or toute l'assemblée des enfants d'Israël s'assembla à Silo\FTNT{Silo fut pendant la période des Juges le centre religieux d'Israël car c'est dans cette ville que l'on avait déposé l'arche jusqu'à ce que le roi David l'amène à Jérusalem (Jos. 18:1 ; 2 S. 6 ; 1 Ch. 15:3). Durant le schisme, Silo, située en Samarie, fit office de capitale du royaume de sud. La ville fut finalement détruite par les Philistins aux alentours de 1050 av. J.-C.}, et ils y posèrent la tente d'assignation, après que le pays leur ait été assujetti. 
\VS{2}Mais il restait sept tribus des enfants d'Israël qui n'avaient pas encore reçu leur héritage.
\VS{3}Josué dit aux enfants d'Israël : Jusqu'à quand négligerez-vous de prendre possession du pays que Yahweh, le Dieu de vos pères, vous a donné ?
\VS{4}Prenez trois hommes de chaque tribu, que j'enverrai. Ils se lèveront, traverseront le pays, traceront un plan en vue de l'héritage, puis ils reviendront auprès de moi.
\VS{5}Ils le diviseront en sept parts ; Juda restera dans ses limites au midi, et la maison de Joseph restera dans ses limites au nord.
\VS{6}Vous donc faites-vous un plan du pays en sept parts, et apportez-le-moi ici. Puis je jetterai pour vous le sort devant Yahweh, notre Dieu.
\VS{7}Et il n'y aura point de part pour les Lévites au milieu de vous, parce que le sacerdoce de Yahweh est leur héritage. Quant à Gad et à Ruben, et à la demi-tribu de Manassé, ils ont reçu leur héritage de l'autre côté du Jourdain, vers l'orient, que Moïse, serviteur de Yahweh, leur a donné.
\VS{8}Ces hommes-là donc se levèrent et s'en allèrent pour tracer un plan du pays, Josué leur donna cet ordre en disant : Allez et traversez le pays, et tracez-en un plan, puis revenez auprès de moi, et je jetterai ici le sort pour vous devant Yahweh, à Silo.
\VS{9}Ces hommes-là donc s'en allèrent, parcoururent le pays, et en tracèrent un plan dans un livre en sept parts selon les villes ; puis ils revinrent auprès de Josué dans le camp à Silo.
\VS{10}Et Josué jeta le sort pour eux à Silo devant Yahweh, et Josué fit le partage du pays entre les enfants d'Israël, selon leurs parts.
\TextTitle{Le territoire de Benjamin}
\VS{11}Et le sort tomba sur la tribu des fils de Benjamin selon leurs familles, et la part qui leur échut par le sort avait ses frontières entre les fils de Juda et les fils de Joseph.
\VS{12}Et leur frontière du côté du nord fut depuis le Jourdain ; et cette frontière devait monter à côté de Jéricho vers le nord, puis monter en la montagne tirant vers l'Occident ; de sorte que ses extrémités devaient se rendre au désert de Beth-aven. 
\VS{13}Puis cette frontière devait passer de là vers Luz, à côté de Luz, qui est Béthel tirant vers le midi ; et cette frontière devait descendre à Hatroth-addar, près de la montagne qui est du côté du midi de Beth-horon la basse. 
\VS{14}Et cette frontière devait s'aligner et tourner du côté occidental qui regarde vers le midi, depuis la montagne qui est vis-à-vis de Beth-horon, vers le midi ; tellement que ses extrémités devaient se rendre à Kirjath Baal, qui est Kirjath Jearim, ville des enfants de Juda. C'est là le côté d'occident. 
\VS{15}Mais le côté méridional est l'extrémité de Kirjath Jearim ; et cette frontière devait sortir vers l'Occident, puis elle devait sortir à la fontaine des eaux de Nephtoah. 
\VS{16}Et cette frontière devait descendre à l'extrémité de la montagne qui est vis-à-vis de la vallée de Ben-Hinnom, dans la vallée des Rephaïm, vers le nord, et descendre par la vallée de Hinnom, sur le côté méridional des Jébusiens, puis descendre jusqu'à En-Roguel.
\VS{17}Et elle devait s'aligner vers le nord, et sortir à En-Schémesch, de là à Gueliloth, qui est vis-à-vis de la montée d'Adummim, et descendre à la pierre de Bohan, fils de Ruben,
\VS{18}et passer sur le côté nord en face d'Araba, et descendre à Araba,
\VS{19}puis cette frontière devait passer à côté de Beth-Hogla vers le nord ; de sorte que les extrémités de cette frontière aboutissent à la langue de la mer salée vers le nord, à l'embouchure du Jourdain vers le midi. C'était la frontière du midi.
\VS{20}Et le Jourdain devait borner du côté de l'orient. Ce fut là l'héritage des fils de Benjamin avec ses frontières tout autour, selon leurs familles.
\VS{21}Les villes de la tribu des fils de Benjamin, selon leurs familles, étaient : Jéricho, Beth-Hogla, Emek-Ketsits,
\VS{22}Beth-Araba, Tsemaraïm, Béthel,
\VS{23}Avvim, Para, Ophra,
\VS{24}Kephar-Ammonaï, Ophni et Guéba ; douze villes et leurs villages.
\VS{25}Gabaon, Rama, Beéroth,
\VS{26}Mitspé, Kephira, Motsa,
\VS{27}Rékem, Jirpeel, Thareala,
\VS{28}Tséla, Eleph, Jebus, qui est Jérusalem, Guibeath et Kirjath ; quatorze villes et leurs villages. Tel fut l'héritage des fils de Benjamin selon leurs familles.
\Chap{19}
\TextTitle{Le territoire de Siméon}
\VerseOne{}La deuxième part échut par le sort à Siméon, pour la tribu des fils de Siméon, selon leurs familles. Leur héritage était parmi l'héritage des fils de Juda\FTNT{Ge. 49:5-7.}.
\VS{2}Ils eurent dans leur héritage Beer-Schéba, Schéba, Molada,
\VS{3}Hatsar-Schual, Bala, Atsem,
\VS{4}Eltholad, Bethul, Horma,
\VS{5}Tsiklag, Beth-Marcaboth, Hatsar-Susa,
\VS{6}Beth-Lebaoth et Scharuchen ; treize villes et leurs villages.
\VS{7}Aïn, Rimmon, Ether, et Aschan ; quatre villes et leurs villages ;
\VS{8}et tous les villages qui étaient autour de ces villes-là jusqu'à Baalath-Beer, qui est Ramath du midi. Tel fut l'héritage de la tribu des fils de Siméon, selon leurs familles.
\VS{9}L'héritage des fils de Siméon fut pris sur la portion des fils de Juda ; car la portion des fils de Juda était trop grande pour eux ; c'est pourquoi les fils de Siméon reçurent leur héritage parmi le leur.
\TextTitle{Le territoire de Zabulon}
\VS{10}La troisième part échut par le sort aux fils de Zabulon, selon leurs familles.
\VS{11}Et leur frontière devait monter vers le quartier devers la mer, même jusqu'à Mareala, puis se rencontrer à Dabbéscheth, et de là au torrent qui est vis-à-vis de Jokneam.
\VS{12}Or cette frontière devait retourner vers Sarid à l'orient, vers le soleil levant, jusqu'à la frontière de Kisloth-Thabor, puis continuer à Dabrath, et monter à Japhia.
\VS{13}De là passer à l'orient, par Guittha-Hépher, par Ittha-Katsin, puis continuer à Rimmon, jusqu'à Néa.
\VS{14}Puis cette frontière devait tourner du côté du nord vers Hannathon, et ses extrémités devaient se rendre à la vallée de Jiphthach-El.
\VS{15}Avec Katthath, Nahalal, Schimron, Jideala, et Bethléhem ; il y avait douze villes et leurs villages.
\VS{16}Tel fut l'héritage des fils de Zabulon selon leurs familles, ces villes-là, et leurs villages.
\TextTitle{Le territoire d'Issacar}
\VS{17}La quatrième part échut par le sort à Issacar, aux fils d'Issacar, selon leurs familles.
\VS{18}Et leur frontière devaient passer par Jizreel, Kesulloth, Sunem,
\VS{19}Hapharaïm, Schion, Anacharath,
\VS{20}Rabbith, Kischjon, Abets,
\VS{21}Rémeth, En-Gannim, En-Hadda et Beth-Patsets ;
\VS{22}elle devait se rencontrer à Thabor, et vers Schachatsima et Beth-Schémesch, et les extrémités de leur frontière devaient se rendre au Jourdain. Seize villes et leurs villages.
\VS{23}Tel fut l'héritage de la tribu des fils d'Issacar, selon leurs familles, ces villes-là et leurs villages.
\TextTitle{Le territoire d'Aser}
\VS{24}La cinquième part échut par le sort à la tribu des fils d'Aser, selon leurs familles.
\VS{25}Et leur frontière fut Helkath, Hali, Béthen, Acschaph,
\VS{26}Allammélec, Amead et Mischeal ; et elle devait se rencontrer à Carmel, au quartier vers la mer, et à Schichor-Libnath.
\VS{27}Puis elle devait retourner vers l'orient, à Beth-Dagon, et se rencontrer à Zabulon, et à la vallée de Jiphthach-El, vers le nord de Beth-Emek et de Neïel, puis sortir vers Cabul, à gauche,
\VS{28}et vers Ebron, Rehob, Hammon et Kana, jusqu'à Sidon la grande.
\VS{29}Puis la frontière devait retourner à Rama, jusqu'à la ville forte de Tyr, et cette frontière devait retourner à Hosa ; de sorte que ses extrémités se rencontrent au quartier qui est vers la mer, par la contrée d'Aczib.
\VS{30}Avec Umma, Aphek et Rehob ; vingt-deux villes et leurs villages.
\VS{31}Tel fut l'héritage de la tribu des fils d'Aser, selon leurs familles ; ces villes-là et leurs villages.
\TextTitle{Le territoire de Nephthali}
\VS{32}La sixième part échut par le sort aux fils de Nephthali, selon leurs familles.
\VS{33}Leur frontière fut depuis Héleph, depuis Allon par Tsaanannim, Adami-Nékeb et Jabneel, jusqu'à Lakkum, et ses extrémités devaient se rendre au Jourdain.
\VS{34}Puis cette frontière devait retourner du côté d'occident, vers Aznoth-Thabor, et sortir de là à Hukkok ; de sorte que du côté du midi elle devait se rencontrer à Zabulon, et du côté d'occident elle devait se rencontrer à Aser et à Juda ; le Jourdain était du côté au soleil levant.
\VS{35}Au reste, les villes fortifiées étaient : Tsiddim, Tser, Hammath, Rakkath, Kinnéreth,
\VS{36}Adama, Rama, Hatsor,
\VS{37}Kédesch, Edréï, En-Hatsor,
\VS{38}Jireon, Migdal-El, Horem, Beth-Anath et Beth-Schémesch ; dix-neuf villes et leurs villages.
\VS{39}Tel fut l'héritage de la tribu des fils de Nephthali, selon leurs familles ; ces villes-là, et leurs villages.
\TextTitle{Le territoire de Dan}
\VS{40}La septième part échut par le sort à la tribu des fils de Dan selon leurs familles.
\VS{41}La limite de leur héritage fut, Tsorea, Eschthaol, Ir-Schémesch,
\VS{42}Schaalabbin, Ajalon, Jithla,
\VS{43}Elon, Thimnatha, Ekron,
\VS{44}Eltheké, Guibbethon, Baalath,
\VS{45}Jehud, Bené-Berak, Gath-Rimmon,
\VS{46}Mé-Jarkon et Rakkon, avec le territoire qui est vis-à-vis de Japho.
\VS{47}Le territoire échu aux fils de Dan était trop petit pour eux. C'est pourquoi les fils de Dan montèrent, et combattirent contre Léschem ; ils s'en emparèrent et la frappèrent du tranchant de l'épée ; ils en prirent possession, s'y établirent, et l'appelèrent Léschem, Dan, du nom de Dan leur père.
\VS{48}Tel fut l'héritage de la tribu des fils de Dan selon leurs familles ; ces villes-là et leurs villages.
\TextTitle{Josué reçoit Thimnath-Sérach}
\VS{49}Après qu'on eut achevé de partager le pays selon ses frontières, les enfants d'Israël donnèrent à Josué, fils de Nun, une possession au milieu d'eux.
\VS{50}Selon l'ordre de Yahweh, ils lui donnèrent la ville qu'il demanda, Thimnath-Sérach, dans la montagne d'Ephraïm. Il rebâtit la ville, et y habita.
\VS{51}Ce sont là les héritages que le prêtre Eléazar, Josué, fils de Nun, et les chefs de pères des tribus des enfants d'Israël partagèrent par le sort à Silo, devant Yahweh, à l'entrée de la tente d'assignation, et ils achevèrent ainsi le partage du pays.
\Chap{20}
\TextTitle{Les six villes de refuge\FTNTT{No. 35.}}
\VerseOne{}Puis Yahweh parla à Josué et dit :
\VS{2}Parle aux enfants d'Israël et dis : Etablissez-vous des villes de refuge comme je vous l'ai ordonné par le moyen de Moïse,
\VS{3}où pourra s'enfuir le meurtrier qui aura tué quelqu'un involontairement, sans intention, et elles vous serviront de refuge devant celui qui a le droit de venger le sang.
\VS{4}Et le meurtrier s'enfuira dans l'une de ces villes, s'arrêtera à l'entrée de la porte de la ville, et il exposera son affaire aux anciens de cette ville-là, ils l'écouteront, et le recevront chez eux dans la ville, et lui donneront une demeure, afin qu'il habite avec eux.
\VS{5}Et quand celui qui a le droit de venger le sang le poursuivra, ils ne livreront pas le meurtrier entre ses mains ; puisque c'est involontairement qu'il a tué son prochain, et qu'il ne le haïssait point auparavant.
\VS{6}Mais il demeurera dans cette ville-là, jusqu'à ce qu'il comparaisse devant l'assemblée pour être jugé, jusqu'à la mort du grand prêtre qui sera en fonction en ce temps-là. Alors le meurtrier s'en retournera, et reviendra dans sa ville et dans sa maison, dans la ville d'où il s'était enfui\FTNT{Ex. 21:13 ; No. 35:9-34 ; De. 19.}.
\VS{7}Ils consacrèrent donc Kédesch, en Galilée, dans la montagne de Nephthali ; Sichem dans la montagne d'Ephraïm ; et Kirjath-Arba, qui est Hébron, dans la montagne de Juda.
\VS{8}Et de l'autre côté du Jourdain, à l'orient de Jéricho, ils choisirent Betser, dans la tribu de Ruben, dans le désert, dans la plaine ; Ramoth en Galaad, dans la tribu de Gad ; et Golan en Basan, dans la tribu de Manassé\FTNT{De. 4:43.}.
\VS{9}Telles furent les villes désignées pour tous les enfants d'Israël et pour l'étranger en séjour au milieu d'eux, afin que quiconque aurait tué quelqu'un involontairement puisse s'y réfugier, et qu'il ne meure pas de la main de celui qui a le droit de venger le sang, avant d'avoir comparu devant l'assemblée.
\Chap{21}
\TextTitle{Les quarante-huit villes des Lévites}
\VerseOne{}Or les chefs des pères de famille des Lévites s'approchèrent d'Eléazar, le prêtre, de Josué, fils de Nun, et des chefs des pères de famille des tribus des enfants d'Israël.
\VS{2}Et leur parlèrent à Silo, dans le pays de Canaan, et dirent : Yahweh a ordonné par Moïse qu'on nous donne des villes pour habiter, et leurs faubourgs pour nos bêtes\FTNT{No. 35:2-3.}.
\VS{3}Alors les enfants d'Israël donnèrent aux Lévites, sur leur héritage, les villes suivantes et leurs faubourgs, d'après l'ordre de Yahweh.
\VS{4}Et on tira au sort pour les familles des Kehathites ; et les Lévites, fils d'Aaron, le prêtre eurent par le sort treize villes de la tribu de Juda, de la tribu de Siméon, et de la tribu de Benjamin.
\VS{5}Les autres fils de Kehath eurent par le sort dix villes des familles de la tribu d'Ephraïm, de la tribu de Dan, et de la demi-tribu de Manassé.
\VS{6}Et les fils de Guerschon eurent par le sort treize villes, des familles de la tribu d'Issacar, de la tribu d'Aser, de la tribu de Nephthali, et de la demi-tribu de Manassé en Basan.
\VS{7}Et les fils de Merari selon leurs familles, eurent douze villes, de la tribu de Ruben, de la tribu de Gad, et de la tribu de Zabulon.
\VS{8}Les enfants d'Israël donnèrent donc par le sort aux Lévites ces villes-là avec leurs faubourgs, comme Yahweh l'avait ordonné par Moïse.
\VS{9}Ils donnèrent donc de la tribu des fils de Juda et de la tribu des fils de Siméon, ces villes, qui vont être nommées par leurs noms,
\VS{10}et qui furent pour les fils d'Aaron, qui étaient des familles des Kehathites, et des fils de Lévi, car le sort les avait indiqués les premiers.
\VS{11}Ils leur donnèrent Kirjath-Arba, qui est Hébron, dans la montagne de Juda, avec ses faubourgs tout autour : Arba était le père d'Anak.
\VS{12}Mais quant au territoire de la ville, et à ses villages, on les donna à Caleb, fils de Jephunné, pour sa possession.
\VS{13}Ils donnèrent donc aux fils d'Aaron, le prêtre, les villes de refuge pour les meurtriers, Hébron, avec ses faubourgs, et Libna avec ses faubourgs.
\VS{14}Jatthir, avec ses faubourgs, Eschthemoa, avec ses faubourgs,
\VS{15}Holon, avec ses faubourgs, Debir, avec ses faubourgs,
\VS{16}Aïn, avec ses faubourgs, Jutta, avec ses faubourgs ; et Beth-Schémesch, avec ses faubourgs ; neuf villes de ces deux tribus-là ;
\VS{17}et de la tribu de Benjamin, Gabaon, avec ses faubourgs, et Guéba, avec ses faubourgs,
\VS{18}Anathoth, avec ses faubourgs, et Almon, avec ses faubourgs ; quatre villes.
\VS{19}Toutes les villes des prêtres, fils d'Aaron, furent treize villes, avec leurs faubourgs.
\VS{20}Quant aux Lévites, appartenant aux familles des autres fils de Kehath, ils eurent par le sort des villes de la tribu d'Ephraïm.
\VS{21}On leur donna donc les villes de refuge pour les meurtriers, Sichem, avec ses faubourgs, dans la montagne d'Ephraïm, et Guézer avec ses faubourgs ;
\VS{22}Kibtsaïm, avec ses faubourgs, et Beth-Horon, avec ses faubourgs ; quatre villes ;
\VS{23}et de la tribu de Dan, Eltheké, avec ses faubourgs ; Guibbethon, avec ses faubourgs,
\VS{24}Ajalon, avec ses faubourgs, Gath-Rimmon, avec ses faubourgs ; quatre villes.
\VS{25}Et de la demi-tribu de Manassé, Thaanac, avec ses faubourgs ; et Gath-Rimmon, avec ses faubourgs, deux villes.
\VS{26}Total des villes : Dix villes avec leurs faubourgs, pour les familles des autres fils de Kehath.
\VS{27}On donna aussi aux fils de Guerschon, d'entre les familles des Lévites : De la demi-tribu de Manassé les villes de refuge pour les meurtriers, Golan en Basan, avec ses faubourgs, et Beeschthra, avec ses faubourgs ; deux villes ;
\VS{28}et de la tribu d'Issacar, Kischjon, avec ses faubourgs, Dabrath, avec ses faubourgs,
\VS{29}Jarmuth, avec ses faubourgs, En-Gannim, avec ses faubourgs ; quatre villes ;
\VS{30}et de la tribu d'Aser, Mischeal, avec ses faubourgs, Abdon, avec ses faubourgs,
\VS{31}Helkath, avec ses faubourgs, et Rehob, avec ses faubourgs ; quatre villes ;
\VS{32}et de la tribu de Nephthali, les villes de refuge pour les meurtriers, Kédesch en Galilée avec ses faubourgs, Hammoth-Dor, avec ses faubourgs, et Karthan, avec ses faubourgs ; trois villes.
\VS{33}Total des villes des Guerschonites, selon leurs familles : Treize villes, et leurs faubourgs.
\VS{34}On donna aussi au reste des Lévites, qui appartenaient aux familles des fils de Merari : De la tribu de Zabulon, Jokneam, avec ses faubourgs, Kartha, avec ses faubourgs,
\VS{35}Dimna, avec ses faubourgs, et Nahalal, avec ses faubourgs ; quatre villes ;
\VS{36}et de la tribu de Ruben, Betser, avec ses faubourgs, et Jahtsa, avec ses faubourgs ;
\VS{37}Kedémoth, avec ses faubourgs, et Méphaath, avec ses faubourgs ; quatre villes ;
\VS{38}et de la tribu de Gad, les villes de refuge pour les meurtriers, Ramoth en Galaad, avec ses faubourgs, et Mahanaïm, avec ses faubourgs,
\VS{39}Hesbon, avec ses faubourgs, et Jaezer, avec ses faubourgs ; en tout quatre villes.
\VS{40}Total des villes qui échurent par le sort aux fils de Merari, selon leurs familles, formant le reste des familles des Lévites : Douze villes.
\VS{41}Total des villes des Lévites qui étaient parmi la possession des enfants d'Israël : Quarante-huit villes, et leurs faubourgs.
\VS{42}Chacune de ces villes avait ses faubourgs autour d'elle ; il en était ainsi de toutes ces villes-là.
\TextTitle{Yahweh accomplit sa promesse}
\VS{43}Yahweh donna donc à Israël tout le pays qu'il avait juré de donner à leurs pères ; ils le possédèrent, et y habitèrent\FTNT{Dieu accomplit toujours ses promesses (Jé. 1:12).}.
\VS{44}Yahweh leur accorda un parfait repos tout autour, selon tout ce qu'il avait juré à leurs pères ; aucun de leurs ennemis ne put leur résister, car Yahweh les livra entre leurs mains.
\VS{45}Il ne tomba pas un seul mot de toutes les bonnes paroles que Yahweh avait dites à la maison d'Israël : Toutes s'accomplirent.
\Chap{22}
\TextTitle{Ruben, Gad et la demi-tribu de Manassé retournent sur leurs terres}
\VerseOne{}Alors Josué appela les Rubénites, les Gadites et la demi-tribu de Manassé.
\VS{2}Et il leur dit : Vous avez gardé tout ce que Moïse, serviteur de Yahweh, vous a prescrit, et vous avez obéi à ma voix dans tout ce que je vous ai ordonné.
\VS{3}Vous n'avez pas abandonné vos frères, depuis une très longue période jusqu'à ce jour ; et vous avez gardé les ordres, les commandements de Yahweh votre Dieu.
\VS{4}Maintenant que Yahweh, votre Dieu, a donné du repos à vos frères, comme il le leur avait dit, retournez et allez dans vos tentes, dans le pays qui vous appartient, et que Moïse, serviteur de Yahweh, vous a donné de l'autre côté du Jourdain\FTNT{No. 32:33 ; De. 3:13 ; De. 29:8.}.
\VS{5}Prenez seulement bien garde d'observer les ordonnances et les lois que Moïse, serviteur de Yahweh, vous a prescrites : Aimez Yahweh votre Dieu, marchez dans toutes ses voies, gardez ses commandements, attachez-vous à lui, et servez-le de tout votre cœur et de toute votre âme\FTNT{De. 10:12.}.
\VS{6}Puis Josué les bénit et les renvoya ; et ils s'en allèrent vers leurs tentes.
\VS{7}Moïse avait donné à la moitié de la tribu de Manassé son héritage en Basan ; et Josué donna à l'autre moitié son héritage avec leurs frères de l'autre côté du Jourdain vers l'occident. Josué les renvoya dans leurs tentes, et les bénit.
\VS{8}Et il leur parla et dit : Vous retournez à vos tentes avec de grandes richesses, une très nombreuse quantité de bétail, avec une quantité considérable d'argent, d'or, d'airain, de fer, et de vêtements. Partagez avec vos frères le butin de vos ennemis.
\VS{9}Ainsi donc les fils de Ruben, les fils de Gad, et la demi-tribu de Manassé s'en retournèrent, et partirent de Silo, dans le pays de Canaan, après avoir quitté les enfants d'Israël, pour s'en aller dans le pays de Galaad, sur la terre de leur possession et où ils s'établirent, suivant ce que Yahweh avait ordonné par Moïse.
\TextTitle{L'autel Ed, sujet d'incompréhension}
\VS{10}Quand ils furent arrivés aux frontières du Jourdain, qui appartiennent au pays de Canaan, les fils de Ruben, les fils de Gad, et la demi-tribu de Manassé y bâtirent un autel, près du Jourdain, un autel dont la grandeur frappait les regards.
\VS{11}Les enfants d'Israël apprirent que l'on disait : Voici, les fils de Ruben, les fils de Gad, et la demi-tribu de Manassé ont bâti un autel en face du pays de Canaan, sur les frontières du Jourdain, du côté des enfants d'Israël.
\VS{12}Lorsque les enfants d'Israël entendirent cela, toute l'assemblée des enfants d'Israël se réunit à Silo, pour monter en guerre contre eux.
\VS{13}Cependant les enfants d'Israël envoyèrent vers les fils de Ruben, vers les fils de Gad, et vers la demi-tribu de Manassé, au pays de Galaad, Phinées, fils du prêtre Eléazar,
\VS{14}et avec lui dix princes, un prince par maison paternelle pour chacune des tribus d'Israël ; tous étaient chefs de maison paternelle parmi les milliers d'Israël.
\VS{15}Ils se rendirent auprès des fils de Ruben, des fils de Gad et de la demi-tribu de Manassé au pays de Galaad, et leur parlèrent, en disant :
\VS{16}Ainsi parle toute l'assemblée de Yahweh : Quelle est cette infidélité que vous avez commise contre le Dieu d'Israël, et pourquoi vous détournez-vous aujourd'hui de Yahweh, en vous bâtissant un autel, pour vous rebeller aujourd'hui contre Yahweh ?
\VS{17}Regardons-nous comme peu de chose l'iniquité de Peor\FTNT{Peor : No. 25:1-9.}, dont nous ne nous sommes pas encore bien purifiés jusqu'à présent, malgré la plaie qu'il attira sur l'assemblée de Yahweh ?
\VS{18}Et vous vous détournez aujourd'hui de Yahweh ! Si vous vous rebellez aujourd'hui contre Yahweh, demain il s'irritera contre toute l'assemblée d'Israël.
\VS{19}Si vous tenez pour impure la terre qui est votre propriété, passez sur la terre qui est la possession de Yahweh, où est fixé le tabernacle de Yahweh, ayez votre possession parmi nous, mais ne vous révoltez point contre Yahweh, et ne soyez point rebelles contre nous, en vous bâtissant un autel, outre l'autel de Yahweh notre Dieu.
\VS{20}Acan\FTNT{Acan : Jos. 7:1-26.}, fils de Zérach, ne commit-il pas une infidélité en prenant des choses dévouées par le moyen de l'interdit, et la colère de Yahweh ne s'enflamma-t-elle pas contre toute l'assemblée d'Israël ? Cependant, cet homme ne fut pas le seul qui périt à cause de son iniquité.
\VS{21}Mais les fils de Ruben, les fils de Gad, et la demi-tribu de Manassé répondirent, et dirent aux chefs des milliers d'Israël :
\VS{22}Dieu\FTNT{Dieu : de l'hébreu « El » : puissant, etc.}, Dieu\FTNT{Dieu : de l'hébreu « elohim » : juge, ange.} Yahweh, Dieu\FTNT{Dieu : de l'hébreu « El » : puissant, etc.}, Dieu\FTNT{Dieu : de l'hébreu « elohim » : juge, ange.}Yahweh, le sait, et Israël lui-même le saura ! Si c'est par rébellion et par infidélité envers Yahweh, alors qu'il ne nous vienne point en aide aujourd'hui.
\VS{23}Si nous nous sommes bâti un autel pour nous détourner de Yahweh, si c'est pour y offrir des holocaustes, ou des offrandes, ou si c'est pour y faire des sacrifices d'offrande de paix, que Yahweh lui-même nous en demande compte !
\VS{24}C'est bien plutôt par une sorte d'inquiétude que nous avons fait cela, en pensant que vos fils pourraient un jour parler à nos fils et leur dire : Qu'y a-t-il de commun entre vous et Yahweh, le Dieu d'Israël ?
\VS{25}Puisque Yahweh a mis le Jourdain pour frontière entre nous et vous, fils de Ruben, et fils de Gad ; vous n'avez point de part à Yahweh ! Et ainsi vos fils feraient qu'un jour nos fils cesseraient de craindre Yahweh\FTNT{Né. 2:20 ; Ac. 8:21.}.
\VS{26}C'est pourquoi nous avons dit : Mettons-nous maintenant à bâtir un autel, non pour des holocaustes ni pour des sacrifices ;
\VS{27}mais afin qu'il serve de témoignage entre nous et vous, et entre nos descendants et les vôtres, que nous voulons servir Yahweh devant sa face par nos holocaustes et nos sacrifices d'expiation et d'offrande de paix, afin que vos fils ne disent pas un jour à nos fils : Vous n'avez point de part à Yahweh\FTNT{Ge. 31:48.} !
\VS{28}C'est pourquoi nous avons dit : Lorsqu'ils nous tiendront ce discours, ou à nos descendants, nous leur dirons : Voyez la forme de l'autel de Yahweh qu'ont fait nos pères, non pour des holocaustes, ni pour des sacrifices, mais afin qu'il soit témoin entre nous et vous.
\VS{29}A Dieu ne plaise que nous nous révoltions contre Yahweh et que nous nous détournions aujourd'hui de Yahweh, en bâtissant un autel pour des holocaustes, pour des offrandes, et pour des sacrifices, outre l'autel de Yahweh notre Dieu, qui est devant son tabernacle !
\VS{30}Or, après que le prêtre Phinées, et les princes de l'assemblée, les chefs des milliers d'Israël qui étaient avec lui, eurent entendu les paroles que les fils de Ruben, les fils de Gad, et les fils de Manassé leur dirent, ils furent satisfaits.
\VS{31}Et Phinées, fils du prêtre Eléazar, dit aux fils de Ruben, aux fils de Gad, et aux fils de Manassé : Nous reconnaissons aujourd'hui que Yahweh est au milieu de nous, puisque vous n'avez point commis cette infidélité contre Yahweh ; vous avez ainsi délivré les enfants d'Israël de la main de Yahweh.
\VS{32}Ainsi Phinées, fils du prêtre Eléazar, et les princes, quittèrent les fils de Ruben, les fils de Gad, et revinrent du pays de Galaad dans le pays de Canaan, auprès des enfants d'Israël, auxquels ils firent un rapport.
\VS{33}Et la chose plut aux enfants d'Israël ; ils bénirent Dieu, et ne parlèrent plus de monter en armes contre eux pour détruire le pays où habitaient les fils de Ruben, et les fils de Gad.
\VS{34}Les fils de Ruben, et les fils de Gad appelèrent l'autel Ed ; car, dirent-ils, il est témoin entre nous que Yahweh est Dieu.
\Chap{23}
\TextTitle{Avertissements de Josué}
\VerseOne{}Or il arriva, plusieurs jours après, que Yahweh ayant donné du repos à Israël de tous les ennemis qui l'entouraient, Josué était vieux, fort avancé en âge.
\VS{2}Et Josué convoqua tout Israël, ses anciens, ses chefs, ses juges, ses officiers, et leur dit : Je suis devenu vieux, fort avancé en âge.
\VS{3}Vous avez vu tout ce que Yahweh, votre Dieu, a fait à toutes ces nations devant vous ; car Yahweh, votre Dieu, est celui qui combat pour vous.
\VS{4}Voyez, je vous ai donné en héritage par le sort, selon vos tribus, ces nations qui sont restées, depuis le Jourdain, et toutes les nations que j'ai exterminées, jusqu'à la grande mer vers le soleil couchant.
\VS{5}Yahweh, votre Dieu, les repoussera devant vous et les chassera ; et vous posséderez leur pays en héritage, comme Yahweh, votre Dieu, vous l'a dit\FTNT{Ex. 14:14 ; Ex. 23:27 ; No. 33:53 ; De. 6:18-19.}.
\VS{6}Appliquez-vous avec force à observer et à mettre en pratique tout ce qui est écrit dans le livre de la loi de Moïse, sans vous en détourner ni à droite ni à gauche\FTNT{De. 5:32 ; De. 28:14.}.
\VS{7}Ne vous mêlez point avec ces nations qui sont restées parmi vous ; et ne faites point mention du nom de leurs dieux, et ne faites jurer personne par eux, ne les servez point, et ne vous prosternez point devant eux\FTNT{Ex. 23:13 ; De. 12:3 ; Jé. 5:7 ; De. 6:14.}.
\VS{8}Mais attachez-vous à Yahweh, votre Dieu, comme vous l'avez fait jusqu'à ce jour\FTNT{De. 11:22.}.
\VS{9}C'est pour cela que Yahweh a chassé devant vous des nations grandes et puissantes ; nul n'a pu vous résister jusqu'à ce jour.
\VS{10}Un seul homme d'entre vous en poursuivait mille ; car Yahweh votre Dieu est celui qui combat pour vous, comme il vous l'a dit\FTNT{Lé. 26:8 ; De. 32:30.}.
\VS{11}Veillez donc attentivement sur vos âmes, afin d'aimer Yahweh, votre Dieu.
\VS{12}Autrement, si vous vous détournez et que vous vous attachez au reste de ces nations qui sont demeurées parmi vous, si vous faites alliance par des mariages avec elles, et si vous formez ensemble des relations,
\VS{13}sachez certainenement que Yahweh, votre Dieu, ne continuera pas à chasser ces nations devant vous ; mais elles seront pour vous un piège et un filet, un fouet dans vos côtés et des épines dans vos yeux, jusqu'à ce que vous ayez péri de dessus cette bonne terre que Yahweh, votre Dieu, vous a donnée\FTNT{Ex. 23:33 ; De. 7:16 ; Jg. 2:3.}.
\VS{14}Voici, je m'en vais aujourd'hui par le chemin de toute la terre. Reconnaissez de tout votre cœur et de toute votre âme qu'aucune de toutes les bonnes paroles prononcées sur vous par Yahweh, votre Dieu, n'est restée sans effet ; toutes se sont accomplies pour vous, aucune n'est restée sans effet\FTNT{Jos. 21:45 ; 2 R. 10:10.}.
\VS{15}Et il arrivera que comme toutes les bonnes paroles que Yahweh, votre Dieu, vous a dites vous sont arrivées ; ainsi Yahweh fera venir sur vous toutes les paroles mauvaises, jusqu'à ce qu'il vous ait exterminés de dessus cette bonne terre que Yahweh, votre Dieu, vous a donnée.
\VS{16}Si vous transgressez l'alliance que Yahweh, votre Dieu, vous a prescrite, et si vous allez servir d'autres dieux et vous prosterner devant eux, la colère de Yahweh s'enflammera contre vous, et vous périrez promptement de dessus cette bonne terre qu'il vous a donnée.
\Chap{24}
\TextTitle{Josué rappelle à Israël son histoire}
\VerseOne{}Josué assembla toutes les tribus d'Israël à Sichem, et il convoqua les anciens d'Israël, ses chefs, ses juges, et ses officiers, qui se présentèrent devant Dieu.
\VS{2}Et Josué dit à tout le peuple : Ainsi parle Yahweh, le Dieu d'Israël : Vos pères, Térach père d'Abraham, et père de Nachor, ont anciennement habité de l'autre côté du fleuve, où ils servaient d'autres dieux.
\VS{3}Mais j'ai pris votre père Abraham de l'autre côté du fleuve, je lui fis parcourir tout le pays de Canaan, je multipliai sa postérité, et lui donnai Isaac\FTNT{Ge. 12 ; Ge. 21:2.}.
\VS{4}Je donnai à Isaac, Jacob et Esaü ; et je donnai à Esaü le mont de Séir, pour le posséder ; mais Jacob et ses fils descendirent en Egypte\FTNT{Ge. 25:24 ; Ge. 36:6.}.
\VS{5}Puis j'envoyai Moïse et Aaron, et je frappai l'Egypte, par les prodiges que j'opérai au milieu d'elle ; puis je vous en fis sortir\FTNT{Ex. 3:10.}.
\VS{6}Je fis donc sortir vos pères hors de l'Egypte, et vous arrivâtes à la mer. Les Egyptiens poursuivirent vos pères avec des chars et des cavaliers, jusqu'à la Mer Rouge\FTNT{Ex. 14:9.}.
\VS{7}Alors ils crièrent à Yahweh. Et il mit des ténèbres entre vous et les Egyptiens, et ramena sur eux la mer, qui les couvrit. Vos yeux ont vu ce que j'ai fait aux Egyptiens. Puis vous restâtes longtemps dans le désert.
\VS{8}Ensuite je vous conduisis dans le pays des Amoréens, qui habitaient de l'autre côté du Jourdain, et ils combattirent contre vous. Mais je les livrai entre vos mains ; vous prîtes possession de leur pays, et je les détruisis devant vous.
\VS{9}Balak\FTNT{Balaak : Voir No. 22:2-14.} aussi, fils de Tsippor, roi de Moab, se leva, et fit la guerre à Israël. Il fit appeler Balaam\FTNT{Balaam : Voir No. 22.}, fils de Beor, pour qu'il vous maudisse.
\VS{10}Mais je ne voulus point écouter Balaam ; il s'agenouilla et vous bénit, et je vous délivrai de la main de Balak.
\VS{11}Et vous passâtes le Jourdain, et arrivâtes près de Jéricho. Les habitants de Jéricho, les Amoréens, les Phéréziens, les Cananéens, les Héthiens, les Guirgasiens, les Héviens et les Jébusiens vous firent la guerre. Je les livrai entre vos mains,
\VS{12}et j'envoyai devant vous des frelons qui les chassèrent loin de votre face, comme les deux rois des Amoréens : Ce ne fut ni par ton épée, ni par ton arc\FTNT{Ex. 23:28 ; De. 7:20.}.
\VS{13}Je vous donnai une terre que vous n'aviez point cultivée, des villes que vous n'aviez point bâties, et que vous habitez, et vous mangez les fruits des vignes et des oliviers que vous n'avez point plantés\FTNT{De. 6:10 ; Ps. 105:44 ; Né. 9:25.}.
\TextTitle{Le peuple choisit de servir Yahweh}
\VS{14}Maintenant, craignez Yahweh, et servez-le avec intégrité et avec fidélité. Ôtez les dieux que vos pères ont servis de l'autre côté du fleuve et en Egypte, et servez Yahweh\FTNT{1 S. 12:23-24 ; Ez. 20:7-44.}.
\VS{15}Et s'il vous déplaît de servir Yahweh, choisissez aujourd'hui qui vous voulez servir, ou les dieux que servaient vos pères au-delà du fleuve, ou les dieux des Amoréens dans le pays desquels vous habitez. Mais moi et ma maison, nous servirons Yahweh.
\VS{16}Alors le peuple répondit, et dit : Que Dieu nous garde d'abandonner Yahweh pour servir d'autres dieux !
\VS{17}Car Yahweh, notre Dieu, est celui qui nous a fait monter, nous et nos pères, hors du pays d'Egypte, de la maison de servitude, qui a fait devant nos yeux ces grands signes, qui nous a gardés dans tout le chemin par lequel nous avons marché, et entre tous les peuples parmi lesquels nous avons passé.
\VS{18}Yahweh a chassé devant nous tous les peuples, et même les Amoréens qui habitaient ce pays. Nous servirons aussi Yahweh, car il est notre Dieu.
\VS{19}Josué dit au peuple : Vous ne pourrez pas servir Yahweh, car c'est un Dieu Saint, qui est jaloux, il ne pardonnera point votre rébellion et vos péchés.
\VS{20}Lorsque vous abandonnerez Yahweh et que vous servirez les dieux des étrangers, il reviendra vous faire du mal, et il vous consumera après vous avoir fait du bien.
\VS{21}Le peuple dit à Josué : Non ! Car nous servirons Yahweh.
\VS{22}Et Josué dit au peuple : Vous êtes témoins contre vous-mêmes que c'est vous qui avez choisi Yahweh pour le servir. Et ils répondirent : Nous en sommes témoins.
\VS{23}Maintenant donc ôtez les dieux étrangers qui sont au milieu de vous, et tournez votre cœur vers Yahweh, le Dieu d'Israël.
\VS{24}Et le peuple répondit à Josué : Nous servirons Yahweh notre Dieu et nous obéirons à sa voix.
\VS{25}Ce jour-là, Josué traita alliance avec le peuple, et lui donna des lois et des ordonnances à Sichem.
\VS{26}Josué écrivit ces paroles dans le livre de la loi de Dieu. Il prit aussi une grande pierre\FTNT{Cette Pierre entend selon Josué, elle est également appelée « témoin ». Jésus-Christ, la Pierre angulaire (Es. 8:13-16) est le témoin fidèle (Ap. 19:11). Cette Pierre suivait les Hébreux dans le désert (1 Co. 10:1-3).}, qu'il dressa là sous le chêne qui était dans le lieu consacré à Yahweh.
\VS{27}Josué dit à tout le peuple : Voici, cette pierre servira de témoin contre nous, car elle a entendu toutes les paroles que Yahweh nous a déclarées ; elle servira de témoin contre vous, afin que vous ne reniiez pas votre Dieu.
\VS{28}Puis Josué renvoya le peuple, chacun dans son héritage.
\TextTitle{Mort de Josué et d'Eléazar ; ensevelissement des os de Joseph (Ge. 50 :26)}
\VS{29}Or il arriva, après ces choses, que Josué, fils de Nun, serviteur de Yahweh, mourut, âgé de cent dix ans.
\VS{30}Et on l'ensevelit dans le territoire de son héritage, à Thimnath-Sérach, dans la montagne d'Ephraïm, du côté du nord de la montagne de Gaasch.
\VS{31}Et Israël servit Yahweh tout le temps de Josué, et tout le temps des anciens qui survécurent à Josué, qui avaient connu toutes les œuvres que Yahweh avait faites pour Israël.
\VS{32}Les os de Joseph\FTNT{(Ge. 50:25 ; Ex. 13:19 ; Hé. 11:22).}, que les enfants d'Israël avaient rapportés d'Egypte, furent ensevelis à Sichem, dans la portion du champ que Jacob avait achetée des fils de Hamor, père de Sichem, pour cent kesita, et qui appartint à l'héritage des fils de Joseph.
\VS{33}Et Eléazar, fils d'Aaron, mourut, on l'enterra à Guibeath-Phinées, qui avait été donnée à son fils Phinées, dans la montagne d'Ephraïm.
\PPE{}
\end{multicols}

%\clearpage\ShortTitle{Juges}\BookTitle{Juges}\BFont
\noindent\hrulefill
{\footnotesize
\textit{
\bigskip
{\centering{}
\\Auteur : Inconnu
\\(Heb. : Shoftim)
\\Signification : Être juge, prononcer, punir
\\Thème : Défaites et délivrances
\\Date de rédaction : Environ 1100 av. J.-C.\\}
}
%\bigskip
\textit{
\\A la mort de Josué et des anciens, il s’éleva en Israël une nouvelle génération qui n'avait pas connu l’expérience du désert. Elle fit ce qui est mal aux yeux de Dieu, l’abandonna et tomba dans l’idolâtrie. Ainsi, la colère de Yahweh s’abattit sur Israël et il livra le peuple entre les mains de ses ennemis. Dans ces temps de troubles, Dieu suscita des juges - douze hommes et une femme - pour délivrer Israël de ses oppresseurs. Aussi longtemps que le juge était en vie, Israël était en paix. Mais dès qu’il venait à mourir, le peuple se corrompait de nouveau et ses oppressions recommençaient.\bigskip
}
}
\par\nobreak\noindent\hrulefill
\begin{multicols}{2}
\Chap{1}
\TextTitle{Poursuite de la conquête de Canaan}
\VerseOne{}Or il arriva qu'après la mort de Josué, les enfants d'Israël consultèrent Yahweh, en disant : Qui de nous montera le premier contre les Cananéens pour leur faire la guerre ?
\VS{2}Et Yahweh répondit : Juda montera ; voici, j'ai livré le pays entre ses mains.
\VS{3}Juda dit à Siméon son frère : Monte avec moi dans mon lot et nous ferons la guerre aux Cananéens ; et j'irai aussi avec toi dans ton lot. Ainsi Siméon alla avec lui.
\TextTitle{Victoires de Juda ; Caleb prend possession d’Hébron}
\VS{4}Juda monta, et Yahweh livra les Cananéens et les Phéréziens entre leurs mains ; ils battirent dix mille hommes à Bézek.
\VS{5}Et ils trouvèrent Adoni-Bézek à Bézek ; ils l'attaquèrent et frappèrent les Cananéens et les Phéréziens.
\VS{6}Adoni-Bézek s'enfuit mais ils le poursuivirent ; et l'ayant pris, ils lui coupèrent les pouces des mains et des pieds.
\VS{7}Alors Adoni-Bézek dit : Soixante-dix rois, dont les pouces des mains et des pieds avaient été coupés, ramassaient du pain sous ma table ; Dieu me rend ce que j’ai fait. On l’amena à Jérusalem et il y mourut\FTNT{Es. 33:1}.
\VS{8}Les fils de Juda firent la guerre contre Jérusalem et la prirent, ils frappèrent ses habitants du tranchant de l'épée et mirent le feu à la ville.
\VS{9}Puis les fils de Juda descendirent pour faire la guerre aux Cananéens, qui habitaient la montagne, la contrée du midi et la plaine.
\VS{10}Juda marcha contre les Cananéens qui habitaient à Hébron ; or le nom d'Hébron était auparavant Kirjath-Arba ; et il battit Schéschaï, Ahiman et Talmaï\FTNT{Jos. 15:14.}.
\VS{11}De là, il marcha contre les habitants de Debir ; Debir s’appelait auparavant Kirjath-Sépher\FTNT{Jos. 15:15.}.
\VS{12}Caleb dit : Je donnerai ma fille Acsa pour femme à celui qui frappera Kirjath-Sépher et qui la prendra\FTNT{Jos. 15:16.}.
\VS{13}Othniel, fils de Kenaz, frère cadet de Caleb, s’en empara ; et Caleb lui donna sa fille Acsa pour femme.
\VS{14}Et il arriva que comme elle s'en allait, elle l'incita à demander à son père un champ. Puis elle descendit  impétueusement de dessus son âne ; et Caleb lui dit : Qu'as-tu ?\FTNT{Jos. 15:18.}
\VS{15}Elle lui répondit : Donne-moi un présent, puisque tu m'as donné une terre du midi ; donne-moi aussi des sources d'eau. Et Caleb lui donna les sources supérieures et les sources inférieures.
\VS{16}Les fils du Kénien, beau-père de Moïse, montèrent de la ville des palmiers avec les fils de Juda, dans le désert de Juda, qui est au midi d'Arad, et ils allèrent et demeurèrent avec le peuple\FTNT{Jg. 4:11}.
\VS{17}Puis Juda se mit en marche avec Siméon son frère et ils frappèrent les Cananéens qui habitaient à Tsephath ; et ils détruisirent la ville par le moyen de l'interdit, c'est pourquoi on appela la ville du nom de Horma.
\VS{18}Juda prit aussi Gaza avec ses territoires ; Askalon avec ses territoires ; et Ekron avec ses territoires.
\TextTitle{Des victoires en demi-teintes}
\VS{19}Yahweh fut avec Juda et il se rendit maître de la montagne, mais il ne pût chasser les habitants de la vallée, parce qu'ils avaient des chars de fer.
\VS{20}On donna Hébron à Caleb, comme Moïse l'avait dit ; et il en chassa les trois fils d'Anak\FTNT{No. 14:24}.
\VS{21}Quant aux fils de Benjamin, ils ne chassèrent pas les Jébusiens qui habitaient à Jérusalem ; c'est pourquoi les Jébusiens ont habité avec les fils de Benjamin à Jérusalem jusqu'à ce jour.
\VS{22}Ceux de la maison de Joseph montèrent aussi contre Béthel, et Yahweh fut avec eux.
\VS{23}Ceux de la maison de Joseph firent explorer Béthel, dont le nom était auparavant Luz.
\VS{24}Les espions virent un homme qui sortait de la ville, et ils dirent : Nous te prions de nous montrer un endroit par où l’on puisse entrer dans la ville, et nous te ferons grâce.
\VS{25}Il leur montra par où ils pourraient entrer dans la ville. Et ils frappèrent la ville du tranchant de l'épée ; mais ils laissèrent aller cet homme et toute sa famille.
\VS{26}Puis cet homme se rendit dans le pays des Héthiens ; il bâtit une ville et lui donna le nom de Luz, nom qu’elle a porté jusqu'à ce jour.
\VS{27}Manassé ne chassa pas les habitants de Beth-Schean et des villes de son ressort, de Thaanac et des villes de son ressort, de Dor et des villes de son ressort, les habitants de Jibleam et des villes de son ressort, les habitants de Meguiddo et des villes de son ressort ; et les Cananéens persistèrent à habiter dans ce pays-là.
\VS{28}Il est vrai qu’il arriva que quand Israël fut devenu plus fort, il assujettit les Cananéens à un tribut mais il ne les chassa pas entièrement.
\VS{29}Ephraïm ne chassa pas les Cananéens qui habitaient à Guézer, et les Cananéens habitèrent avec lui à Guézer.
\VS{30}Zabulon ne chassa pas les habitants de Kitron, ni les habitants de Nahalol ; et les Cananéens habitèrent avec lui et lui furent assujettis à un tribut.
\VS{31}Aser ne chassa pas les habitants d’Acco, ni les habitants de Sidon, ni ceux d’Achlal, ni d'Aczib, ni d'Helba, ni d'Aphik, ni de Rehob ;
\VS{32}Mais ceux d'Aser habitèrent parmi les Cananéens, habitants du pays ; car ils ne les chassèrent pas.
\VS{33}Nephthali ne chassa pas les habitants de Beth-Schémesch, ni les habitants de Beth-Anath, mais il habita parmi les Cananéens habitants du pays ; et les habitants de Beth-Schémesch, et de Beth-Anath lui furent assujettis au tribut.
\VS{34}Les Amoréens repoussèrent les enfants de Dan dans la montagne et ne les laissèrent pas descendre dans la vallée.
\VS{35}Les Amoréens voulurent encore habiter à Har-Hérès, à Ajalon et à Schaalbim ; mais la main de la maison de Joseph étant devenue plus forte, ils furent assujettis au tribut.
\VS{36}Le territoire des Amoréens s'étendait depuis la montée d’Akrabbim, depuis Séla et en dessus.
\Chap{2}
\TextTitle{Le peuple repris pour sa désobéissance}
\VerseOne{}Or l'Ange de Yahweh monta de Guilgal à Bokim, et dit : Je vous ai fait monter hors d'Egypte, et je vous ai fait entrer dans le pays que j’avais juré à vos pères, et j’ai dit : Je n’enfreindrai jamais mon alliance que j’ai traitée avec vous\FTNT{Ge. 17:7.} ;
\VS{2}Et vous aussi vous ne traiterez pas alliance avec les habitants de ce pays, vous démolirez leurs autels. Mais vous n'avez pas obéi à ma voix. Pourquoi avez-vous fait cela\FTNT{Ex. 23:32 ; De. 7:2 ; De. 12:3.} ?
\VS{3}J’ai dit alors : Je ne les chasserai pas devant vous, mais ils seront à vos côtés, et leurs dieux vous seront un piège\FTNT{Ex. 23:33 ; Jos. 23:13.}.
\VS{4}Et il arriva que, comme l'Ange de Yahweh disait ces paroles à tous les enfants d'Israël, le peuple éleva la voix et pleura.
\VS{5}C'est pourquoi ils appelèrent ce lieu Bokim et ils y offrirent des sacrifices à Yahweh.
\VS{6}Josué renvoya le peuple, et les enfants d'Israël allèrent chacun dans son héritage pour prendre possession du pays\FTNT{Jos. 24:28–32.}.
\VS{7}Le peuple servit Yahweh tout le temps de Josué, et tout le temps des anciens qui survécurent à Josué et qui avaient vu toutes les grandes œuvres que Yahweh avait faites en faveur d’Israël\FTNT{Jos. 24:31.}.
\VS{8}Puis Josué, fils de Nun, serviteur de Yahweh, mourut, âgé de cent dix ans\FTNT{Jos. 24:29.}.
\VS{9}On l’ensevelit dans le territoire qu’il avait eu en partage à Thimnath-Hérès, dans la montagne d'Ephraïm, au nord de la montagne de Gaasch\FTNT{Jos. 24:30.}.
\TextTitle{La nouvelle génération abandonne Yahweh}
\VS{10}Toute cette génération fut recueillie auprès de ses pères, puis il s’éleva après elle une autre génération, qui ne connaissait pas Yahweh ni les œuvres qu'il avait faites en faveur d’Israël.
\VS{11}Les enfants d'Israël firent alors ce qui est mal aux yeux de Yahweh et ils servirent les Baals\FTNT{Baal est un dieu phénicien qui, sous les Ramessides, était assimilé dans la mythologie égyptienne à Seth et à Montou. Baal est un dieu d’origine sémite. Il est le dieu de la pluie. Son nom – «~le maître~» ou «~l’époux~»- se retrouve partout dans le Moyen-Orient, depuis les zones peuplées par les sémites jusqu’aux colonies phéniciennes, dont Carthage. Il était invariablement accompagné d’une divinité féminine (Astarté, Ishtar, Tanit...). Voir Jg. 3:7 ; Jg. 8:33 ; Jg. 10:6.}.
\VS{12}Et ils abandonnèrent Yahweh, le Dieu de leurs pères, qui les avait fait sortir du pays d’Égypte, ils allèrent après d'autres dieux, d'entre les dieux des peuples qui les entouraient ; et ils se prosternèrent devant eux, irritant ainsi Yahweh.
\VS{13}Ils abandonnèrent donc Yahweh, et servirent Baal et les Astartés\FTNT{Astarté ou Ashtart en punico-phénicien, ou Ishtar, dérivé de la déesse de Babylone, était  généralement assimilée à la déesse Mésopotamienne Innana. Déesse phénicienne présentant un caractère belliqueux, elle était souvent représentée à califourchon sur son cheval, accompagnant et protégeant le souverain. Elément féminin du couple suprême qu’elle formait avec Baal, celle-ci assumait des fonctions variées : protectrice du souverain et de sa dynastie ou encore des marins.  Comme pour la plupart des divinités féminines primordiales de l’antiquité (et de la proto-histoire), son culte était  lié à la fertilité et à la fécondité. Parfois vénérée sous le nom de Tanit, elle sera assimilée à Vénus par les Romains sous le nom officiel de Venere Ericina.}.
\VS{14}La colère de Yahweh s'enflamma contre Israël. Il les livra entre les mains de pillards\FTNT{Lorsqu’un enfant de Dieu ouvre la porte au péché, il s’expose aux pillards, c’est-à-dire à Satan et ses démons (Jn. 10:10).} qui les pillèrent, il les vendit entre les mains de leurs ennemis d'alentour, de sorte qu'ils ne purent plus résister face à leurs ennemis\FTNT{Ps. 44:12-13 ; Es. 50:1.}.
\VS{15}Partout où ils allaient, la main de Yahweh était contre eux pour leur faire du mal, comme Yahweh l’avait dit et leur avait juré. Ils furent dans une grande détresse\FTNT{Lé. 26:25 ; De. 28:25.}.
\TextTitle{Yahweh suscite des libérateurs : Les juges}
\VS{16}Yahweh leur suscita des juges\FTNT{Les Juges étaient principalement des libérateurs de l’oppression des ennemis d’Israël.} et ils les délivrèrent de la main de ceux qui les pillaient.
\VS{17}Mais ils ne voulurent pas écouter leurs juges, ils se prostituèrent auprès d'autres dieux, se prosternèrent devant eux. Ils se détournèrent promptement du chemin qu’avaient suivi leurs pères et ils n’obéirent pas comme eux aux commandements de Yahweh.
\VS{18}Quand Yahweh leur suscitait des juges, Yahweh était avec le juge, et il les délivrait de la main de leurs ennemis pendant tout le temps de la vie du juge ; car Yahweh se repentait à cause de leurs gémissements contre ceux qui les opprimaient et les tourmentaient.
\VS{19}Puis il arrivait que quand le juge mourrait, ils se corrompaient de nouveau plus que leurs pères en allant après d'autres dieux pour les servir et se prosterner devant eux, et ils persévéraient dans la même conduite et dans la même voie obstinée\FTNT{Jg. 3:12.}.
\TextTitle{IYahweh éprouve Israël et ne chasse pas ses ennemis}
\VS{20}C'est pourquoi la colère de Yahweh s'enflamma contre Israël, et il dit : Puisque cette nation a transgressé mon alliance que j'avais prescrite à leurs pères et puisqu’ils n'ont pas obéi à ma voix,
\VS{21}aussi je ne chasserai plus devant eux aucune des nations que Josué laissa quand il mourut\FTNT{Jos. 23:13.},
\VS{22}afin d'éprouver par elles Israël, pour savoir s'ils prendront garde ou non de suivre la voie de Yahweh, comme leurs pères y ont pris garde.
\VS{23}Yahweh laissa en repos ces nations qu'il n'avait pas livrées entre les mains de Josué et il ne se hâta pas de les chasser\FTNT{Jg. 3:1-3.}.
\Chap{3}
\VerseOne{}Voici les nations que Yahweh laissa pour éprouver par elles Israël, tous ceux qui n'avaient pas connu toutes les guerres de Canaan\FTNT{Jg. 2:21-23.} ; 
\VS{2}afin qu’au moins les générations des enfants d'Israël connaissent et apprennent la guerre, ceux qui ne l’avaient pas connue auparavant.
\VS{3}Ces nations étaient : Les cinq princes des Philistins, tous les Cananéens, les Sidoniens et les Héviens qui habitaient la montagne du Liban depuis la montagne de Baal-Hermon, jusqu'à l'entrée de Hamath\FTNT{No. 13:22.}.
\VS{4}Ces nations, dis-je, servirent à éprouver Israël pour voir s'ils obéiraient aux commandements que Yahweh avait donnés à leurs pères par le moyen de Moïse.
\TextTitle{Israël se mélange aux nations païennes}
\VS{5}Ainsi les enfants d'Israël habitèrent parmi les Cananéens, les Héthiens, les Amoréens, les Phéréziens, les Héviens et les Jébusiens.
\VS{6}Ils prirent leurs filles pour femmes, ils donnèrent leurs filles à leurs fils et servirent leurs dieux.
\VS{7}Les enfants d'Israël firent ce qui est mal aux yeux de Yahweh, ils oublièrent Yahweh et servirent les Baals et les Astartés\FTNT{Jg. 2:11.}.
\TextTitle{Othniel, premier juge suscité par Yahweh}
\VS{8}C'est pourquoi la colère de Yahweh s'enflamma contre Israël, et il les vendit entre la main de Cuschan-Rischeathaïm, roi de Mésopotamie. Et les enfants d'Israël furent asservis à Cuschan-Rischeathaïm durant huit ans.
\VS{9}Puis les enfants d'Israël crièrent à Yahweh, et Yahweh leur suscita un libérateur qui les délivra, Othniel, fils de Kenaz, frère cadet de Caleb.
\VS{10}L’Esprit de Yahweh fut sur lui. Il devint juge en Israël, et il sortit pour la guerre. Yahweh livra entre ses mains Cuschan-Rischeathaïm, roi de Mésopotamie ; et sa main fut puissante contre Cuschan-Rischeathaïm.
\VS{11}Le pays fut en repos pendant quarante ans. Puis Othniel, fils de Kenaz, mourut.
\TextTitle{Ehud, juge en Israël}
\VS{12}Les enfants d'Israël firent encore ce qui est mal aux yeux de Yahweh ; et Yahweh fortifia Eglon, roi de Moab, contre Israël, parce qu'ils avaient fait ce qui est mauvais aux yeux de Yahweh.
\VS{13}Eglon réunit auprès de lui les fils d'Ammon et les Amalécites et il se mit en marche. Il battit Israël et ils s'emparèrent de la ville des palmiers\FTNT{Palmiers: un autre nom de Jéricho.}.
\VS{14}Et les enfants d'Israël furent asservis à Eglon, roi de Moab, durant dix-huit ans.
\VS{15}Puis les enfants d'Israël crièrent à Yahweh, et Yahweh leur suscita un libérateur, Ehud, fils de Guéra, Benjamite, qui ne se servait pas de sa main droite. Les enfants d'Israël envoyèrent par lui un présent à Eglon, roi de Moab.
\VS{16}Ehud se fit une épée à deux tranchants, de la longueur d'une coudée\FTNT{Une coudée correspond environ à 45 cm.} et il la ceignit sous ses vêtements, sur sa cuisse droite.
\VS{17}Il offrit le présent à Eglon, roi de Moab ; et Eglon était un homme fort gras.
\VS{18}Or il arriva que lorsqu’il eut achevé d’offrir le présent, il renvoya le peuple qui avait apporté le présent.
\VS{19}Mais Ehud revint depuis les idoles de pierre, qui étaient près de Guilgal et il dit : Ô roi ! J’ai quelque chose de secret à te dire. Et il lui répondit : Tais-toi ! Et tous ceux qui étaient auprès de lui sortirent de là.
\VS{20}Ehud s'approcha de lui, comme il était assis seul dans sa chambre d'été, et il dit : J'ai un mot à te dire de la part de Dieu, alors le roi se leva du trône.
\VS{21}Et Ehud avança sa main gauche, tira l'épée de son côté droit et la lui enfonça dans le ventre.
\VS{22}Et la poignée entra après la lame, et la graisse serra tellement la lame, qu’il ne pouvait retirer l’épée du ventre, et il en sortit de l’excrément.
\VS{23}Après cela, Ehud sortit par le portique, ferma après lui les portes de la chambre et tira le verrou.
\VS{24}Quand il fut sorti, les serviteurs d'Eglon vinrent et regardèrent ; et voici, les portes de la chambre étaient fermées au verrou. Ils dirent : Sans doute il se couvre les pieds dans sa chambre d’été.
\VS{25}Et ils attendirent tant qu'ils en furent déconcertés ; et voyant qu'il n'ouvrait pas les portes de la chambre, ils prirent la clef et ouvrirent ; et voici, leur maître était mort, étendu à terre.
\VS{26}Mais Ehud s'échappa pendant qu’ils hésitaient ; et il dépassa les carrières de pierre et se sauva à Seïra.
\VS{27}Dès qu’il fut arrivé, il sonna du shofar dans la montagne d'Ephraïm. Les enfants d'Israël descendirent avec lui de la montagne et il marchait à leur tête.
\VS{28}Il leur dit : Suivez-moi, car Yahweh a livré entre vos mains les Moabites, vos ennemis. Ainsi ils descendirent après lui, s’emparèrent des passages du Jourdain vis-à-vis de Moab et ne laissèrent passer personne.
\VS{29}Ils battirent dans ce temps-là environ dix mille hommes de Moab, tous robustes, tous vaillants et il n'en échappa aucun.
\VS{30}En ce jour, Moab fut humilié sous la main d'Israël. Et le pays fut en repos pendant quatre-vingts ans.
\TextTitle{Schamgar, juge en Israël}
\VS{31}Après lui, il y eut Schamgar, fils d'Anath. Il battit six cents Philistins avec un aiguillon à bœufs et délivra Israël.
\Chap{4}
\TextTitle{Débora et Barak, juges en Israël}
\VerseOne{}Mais les enfants d'Israël firent encore ce qui est mal aux yeux de Yahweh après qu'Ehud fut mort.
\VS{2}C'est pourquoi Yahweh les vendit entre la main de Jabin, roi de Canaan, qui régnait à Hatsor. Le chef de son armée était Sisera, qui habitait à Haroscheth-Goïm\FTNT{Jg. 3:8-16 ; Jos. 11:11-13 ; 1 S. 12:9.}.
\VS{3}Les enfants d'Israël crièrent à Yahweh car Jabin avait neuf cents chars de fer, et il avait violemment opprimé les enfants d'Israël durant vingt ans\FTNT{Jg. 1:19.}.
\VS{4}Dans ce temps-là, Débora, prophétesse, femme de Lappidoth, était juge en Israël.
\VS{5}Débora se tenait sous un palmier, entre Rama et Béthel, dans la montagne d'Ephraïm ; et les enfants d'Israël montaient vers elle pour être jugés.
\VS{6}Elle envoya appeler Barak, fils d'Abinoam, de Kédesch-Nephthali et elle lui dit : Yahweh, le Dieu d'Israël, n'a-t-il pas donné cet ordre ? En disant : Va, et dirige-toi sur la montagne de Thabor et prends avec toi dix mille hommes des enfants de Nephthali, et des enfants de Zabulon\FTNT{Hé. 11:32.} ;
\VS{7}J’attirerai vers toi, au torrent de Kison, Sisera, chef de l'armée de Jabin, avec ses chars et ses troupes et je le livrerai entre tes mains\FTNT{Ps. 83:9-10.}.
\VS{8}Barak lui dit : Si tu viens avec moi, j'irai ; mais si tu ne viens pas avec moi, je n’irai pas.
\VS{9}Elle répondit : J'irai, j'irai avec toi, mais tu n'auras pas d'honneur sur le chemin où tu marches ; car Yahweh livrera Sisera entre les mains d'une femme. Débora se leva et elle alla avec Barak à Kédesch.
\VS{10}Barak convoqua Zabulon et Nephthali à Kédesch ; dix mille hommes marchèrent à sa suite ; et Débora monta avec lui.
\VS{11}Héber, le Kénien, s’était séparé des fils de Hobab, beau-père de Moïse et il avait dressé ses tentes jusqu'au chêne de Tsaannaïm, près de Kédesch\FTNT{No. 10:29.}.
\TextTitle{Yahweh accorde la victoire à Israël}
\VS{12}On rapporta à Sisera que Barak, fils d'Abinoam, s’était dirigé sur la montagne de Thabor.
\VS{13}Et Sisera rassembla tous ses chars, neuf cents chars de fer, et tout le peuple qui était avec lui, depuis Haroscheth-Goïm, jusqu'au torrent de Kison.
\VS{14}Alors Débora dit à Barak : Lève-toi, car voici le jour où Yahweh livre Sisera entre tes mains. Yahweh ne marche-t-il pas devant toi ? Barak descendit de la montagne de Thabor, ayant dix mille hommes à sa suite.
\VS{15}Yahweh mit en déroute devant Barak, Sisera, tous ses chars et toute l'armée, par le tranchant de l'épée. Sisera descendit du char et s'enfuit à pied\FTNT{Ps. 83:9-10.}.
\VS{16}Barak poursuivit les chars et l'armée jusqu'à Haroscheth-Goïm ; et toute l'armée de Sisera fut passée au fil de l'épée ; il n'en resta pas un seul.
\VS{17}Sisera se sauva à pied dans la tente de Jaël, femme de Héber, le Kénien ; car il y avait paix entre Jabin, roi de Hatsor et la maison de Héber, le Kénien.
\VS{18}Jaël étant sortie au-devant de Sisera, lui dit : Entre, mon seigneur, entre chez moi, ne crains pas. Il entra donc chez elle dans la tente et elle le cacha sous une couverture.
\VS{19}Puis il lui dit : Je te prie, donne-moi un peu d'eau à boire, car j'ai soif. Et elle ouvrit une outre de lait, lui donna à boire et le couvrit\FTNT{Jg. 5:25.}.
\VS{20}Il lui dit encore : Tiens-toi à l'entrée de la tente et si l’on vient t’interroger, en disant : Y a-t-il ici quelqu'un ? Alors tu répondras : Non.
\VS{21}Jaël, femme de Héber, saisit un pieu de la tente, prit en sa main un marteau, s’approcha de lui doucement, et lui enfonça dans la tempe le pieu, qui pénétra en terre, pendant qu'il dormait profondément, car il était accablé de fatigue. Et ainsi il mourut.
\VS{22}Et voici, Barak poursuivait Sisera, Jaël sortit au-devant de lui et lui dit : Viens, et je te montrerai l'homme que tu cherches. Barak entra chez elle, et voici, Sisera était étendu mort, et le pieu était dans sa tempe.
\VS{23}En ce jour-là, Dieu humilia Jabin, roi de Canaan, devant les enfants d'Israël.
\VS{24}Et la main des enfants d'Israël s’appesantit et se renforça de plus en plus sur Jabin, roi de Canaan, jusqu'à ce qu'ils aient exterminé Jabin, roi de Canaan.
\Chap{5}
\TextTitle{Cantique à la gloire de Yahweh, le Dieu qui délivre}
\VerseOne{}En ce jour-là, Débora chanta ce cantique avec Barak, fils d'Abinoam, en disant :
\VS{2}Bénissez Yahweh de ce qu’il a fait de telles vengeances en Israël et de ce que le peuple s’est offert volontairement.
\VS{3}Vous, rois, écoutez ! Vous, princes, prêtez l'oreille ! Moi, je chanterai à Yahweh, je chanterai un hymne à Yahweh, le Dieu d'Israël.
\VS{4}Ô Yahweh ! Quand tu sortis de Séir, quand tu t’avanças des champs d'Edom, la terre trembla, les cieux se fondirent, les nuées fondirent en eaux ;
\VS{5}Les montagnes s'ébranlèrent devant Yahweh, ce Sinaï devant Yahweh, le Dieu d'Israël\FTNT{Ps. 68:8-9}.
\VS{6}Aux jours de Schamgar, fils d’Anath, aux jours de Jaël, les grandes routes étaient délaissées, et ceux qui voyageaient prenaient des chemins détournés.
\VS{7}Les villes non murées n’étaient plus habitées en Israël, elles n’étaient point habitées, jusqu’à ce que je me suis levée, moi Débora, jusqu’à ce que je me suis levée pour être mère en Israël.
\VS{8}Israël choisissait-il des dieux nouveaux aussitôt la guerre était aux portes. On ne voyait ni bouclier ni lance chez quarante milliers en Israël.
\VS{9}J’ai mon cœur vers les chefs d'Israël, qui se sont portés volontairement d’entre le peuple. Bénissez Yahweh !
\VS{10}Vous qui montez sur les ânesses blanches, vous qui avez pour sièges des tapis et vous qui marchez sur le chemin, méditez !
\VS{11}Le bruit des archers ayant cessé dans les abreuvoirs, qu’on s’y entretienne des justices de Yahweh et des justices de ses villes non murées en Israël ; alors le peuple de Dieu descendra aux portes.
\VS{12}Réveille-toi, réveille-toi, Débora !  Réveille-toi, réveille-toi, dit le cantique, lève-toi Barak et emmène en captivité ceux que tu as faits captifs, toi fils d'Abinoam\FTNT{Jg. 4:6.}.
\VS{13}Yahweh a fait dominer un reste du peuple sur les puissants ; Yahweh m'a fait dominer sur les héros.
\VS{14}Leur racine est depuis Ephraïm jusqu’à Amalek. A ta suite marcha Benjamin parmi ta troupe. De Makir descendirent les chefs, et de Zabulon ceux qui manient la plume du scribe.
\VS{15}Et les chefs d’Issacar ont été avec Débora, et Issacar ainsi que Barak ; il a été envoyé avec sa suite dans la vallée ; il y a eu aux ruisseaux de Ruben, de grandes considérations dans leur cœur.
\VS{16}Pourquoi es-tu resté entre les barres des étables, à écouter le bêlement des troupeaux ? Aux ruisseaux de Ruben, grandes furent les résolutions du cœur !
\VS{17}Galaad est resté au-delà du Jourdain ; et pourquoi Dan est-il resté sur ses navires ? Aser s'est tenu sur le rivage de la mer, et s’est reposé dans ses ports.
\VS{18}Mais pour Zabulon, c'est un peuple qui a exposé son âme à la mort ; et Nephthali de même, sur les hauteurs des champs.
\VS{19}Les rois vinrent, ils combattirent. Alors combattirent les rois de Canaan, à Thaanac, près des eaux de Meguiddo ; mais ils ne remportèrent nul butin, nul argent.
\VS{20}On a combattu des cieux, les étoiles, dis-je, ont combattu du lieu de leur cours contre Sisera\FTNT{Jg. 4:7.}.
\VS{21}Le torrent de Kison les a emportés, le torrent des anciens temps, le torrent de Kison. Mon âme tu as foulé aux pieds les héros.
\VS{22}Alors les talons des chevaux battirent le sol à cause de la course rapide, de la course rapide de ses puissants chevaux.
\VS{23}Maudissez Méroz, dit l'Ange de Yahweh ; maudissez, maudissez ses habitants, car ils ne sont pas venus au secours de Yahweh, au secours de Yahweh, avec les héros.
\VS{24}Bénie soit par-dessus toutes les femmes Jaël, femme de Héber, le Kénien ! Qu'elle soit bénie entre les femmes qui habitent sous les tentes !
\VS{25}Il demanda de l'eau, elle lui a donné du lait ; elle lui a présenté de la crème dans la coupe des chefs.
\VS{26}Elle a saisi de sa main gauche le pieu et de sa main droite le marteau des ouvriers ; elle a frappé Sisera et lui a fendu la tête ; elle a fracassé et transpercé ses tempes.
\VS{27}Il s'est affaissé aux pieds de Jaël, il est tombé, il s’est couché aux pieds de Jaël ; il s'est affaissé, il est tombé ; là où il s'est affaissé, il est tombé là tout défiguré.
\VS{28}La mère de Sisera regardait par la fenêtre et s'écriait en regardant par les treillis : Pourquoi son char tarde-t-il à venir ? Pourquoi ses chars vont-ils si lentement ?
\VS{29}Les plus sages de ses dames lui répondent, et elle se répond à elle-même :
\VS{30}N’ont-ils pas trouvé ? ils partagent le butin ; une fille, deux filles à chacun par tête. Le butin des vêtements de couleurs est à Sisera, le butin de couleurs de broderie ; couleur de broderie à deux endroits, autour du cou de ceux du butin.
\VS{31}Périssent ainsi, tous tes ennemis ô Yahweh ! Et que ceux qui t'aiment soient comme le soleil quand il sort dans sa force. Et le pays fut en repos pendant quarante ans.
\Chap{6}
\TextTitle{Israël assujetti par Madian}
\VerseOne{}Or, les enfants d'Israël firent ce qui est mal aux yeux de Yahweh ; et Yahweh les livra entre les mains de Madian pendant sept ans.
\VS{2}La main de Madian fut puissante contre Israël. Pour échapper aux Madianites, les enfants d'Israël se retiraient dans les ravins des montagnes, dans des cavernes et sur les rochers fortifiés.
\VS{3}Car il arrivait que quand Israël avait semé, Madian montait avec Amalek et les fils de l’orient, et ils montaient contre lui.
\VS{4}Ils faisaient un camp contre lui, ravageaient les fruits du pays jusqu'à Gaza et ne laissaient en Israël ni vivres, ni brebis, ni bœufs, ni ânes.
\VS{5}Car ils montaient avec leurs troupeaux et leurs tentes, ils arrivaient comme une multitude de sauterelles, ils étaient innombrables, eux et leurs chameaux et ils venaient dans le pays pour le ravager.
\VS{6}Israël fut très appauvri par Madian, et les enfants d'Israël crièrent à Yahweh.
\VS{7}Lorsque les enfants d'Israël crièrent à Yahweh au sujet de Madian,
\VS{8}Yahweh envoya un prophète aux enfants d'Israël, qui leur dit : Ainsi parle Yahweh, le Dieu d'Israël : Je vous ai fait monter hors d’Égypte et je vous ai retirés de la maison de servitude.
\VS{9}Je vous ai délivrés de la main des Egyptiens et de la main de tous ceux qui vous opprimaient ; je les ai chassés devant vous et je vous ai donné leur pays.
\VS{10}Je vous ai dit : Je suis Yahweh, votre Dieu ; vous ne craindrez pas les dieux des Amoréens, dans le pays desquels vous habitez. Mais vous n'avez pas obéi à ma voix.
\TextTitle{Gédéon rencontre l’Ange de Yahweh}
\VS{11}Puis l'Ange de Yahweh vint et s'assit sous le térébinthe d’Ophra, qui appartenait à Joas, de la famille d'Abiézer. Gédéon, son fils, battait du froment au pressoir pour le mettre à l'abri de Madian.
\VS{12}Alors l'Ange de Yahweh lui apparut et lui dit : Très fort et vaillant héros, Yahweh est avec toi !
\VS{13}Gédéon lui répondit : Hélas mon Seigneur ! Est-il possible que Yahweh soit avec nous ? Pourquoi donc toutes ces choses nous sont-elles arrivées ? Et où sont tous ces prodiges que nos pères nous ont racontés, en disant : Yahweh ne nous a-t-il pas fait monter hors d'Egypte ? Car maintenant Yahweh nous a abandonnés et nous a livrés entre les mains des Madianites.
\VS{14}Yahweh le regarda et lui dit : Va avec cette force que tu as et tu délivreras Israël de la main des Madianites ; ne t'ai-je pas envoyé\FTNT{Hé.11:32}?
\VS{15}Et il lui répondit : Hélas, mon Seigneur ! Avec quoi délivrerai-je Israël ? Voici, mon millier de bétail est le plus pauvre en Manassé et je suis le plus petit de la maison de mon père\FTNT{1 S. 9:21 ; 1 S. 16:11.}.
\VS{16}Yahweh lui dit : Parce que je serai avec toi, tu frapperas les Madianites comme s'ils n'étaient qu'un seul homme.
\VS{17}Et il lui répondit : Je te prie, si j'ai trouvé grâce à tes yeux, donne-moi un signe pour montrer que c'est toi qui me parles.
\VS{18}Je te prie, ne t’éloigne pas d’ici jusqu'à ce que je revienne auprès de toi, que j'apporte mon offrande et que je la dépose devant toi. Yahweh dit : Je resterai jusqu'à ce que tu reviennes.
\VS{19}Alors Gédéon rentra et apprêta un chevreau de lait, et fit avec un épha de farine des pains sans levain. Il mit la chair dans un panier, le jus dans un pot et il les lui apporta sous le térébinthe, et les présenta.
\VS{20}L'Ange de Dieu lui dit : Prends la chair et les pains sans levain et pose-les sur ce rocher\FTNT{Voir commentaire en  Es. 8:13-17} et répands le jus. Et il fit ainsi.
\VS{21}Alors l'Ange de Yahweh avança l’extrémité du bâton qu'il avait à la main, et toucha la chair et les pains sans levain. Le feu monta du rocher, et consuma la chair et les pains sans levain. Puis l'Ange de Yahweh disparut à ses yeux.
\VS{22}Gédéon, voyant que c'était l'Ange de Yahweh, dit : Ah, malheur à moi, Seigneur Yahweh ! Car j'ai vu l’Ange de Yahweh face à face.
\VS{23}Et Yahweh lui dit : Sois en paix, ne crains pas, tu ne mourras pas.
\VS{24}Gédéon bâtit là un autel à Yahweh, et lui donna pour nom Yahweh-Shalom. Cet autel, qui appartenait à la famille d'Abiézer, existe encore aujourd'hui à Ophra.
\TextTitle{Gédéon détruit les idole ; Yahweh lui confirme sa mission}
\VS{25}Or il arriva dans cette nuit-là que Yahweh lui dit : Prends un jeune taureau d'entre les bœufs qui sont à ton père et un deuxième taureau de sept ans ; et démolis l'autel de Baal qui est à ton père, et abats l’idole d'Astarté qui est dessus.
\VS{26}Tu bâtiras ensuite et tu disposeras, sur le haut de ce rocher, un autel à Yahweh, ton Dieu. Tu prendras ce deuxième taureau, et tu l'offriras en holocauste avec le bois de l’emblème d’Astarté que tu auras démoli.
\VS{27}Gédéon ayant pris dix hommes parmi ses serviteurs, fit comme Yahweh lui avait dit ; et parce qu'il craignait la maison de son père et les gens de la ville, il l’exécuta de nuit et non de jour.
\VS{28}Lorsque les gens de la ville se levèrent de bon matin, voici, l'autel de Baal avait été démoli, et l'idole d'Astarté qui est dessus était abattue, et le deuxième taureau était offert en holocauste sur l'autel qui avait été bâti.
\VS{29}Ils se dirent les uns aux autres : Qui a fait cela ? Et ils s’informèrent et firent des recherches. On leur dit : C’est Gédéon, fils de Joas, qui a fait cela.
\VS{30}Puis les gens de la ville dirent à Joas : Fais sortir ton fils et qu'il meure ; car il a démoli l'autel de Baal et abattu l'idole d'Astarté qui est dessus.
\VS{31}Joas répondit à tous ceux qui s'adressèrent à lui : Est-ce à vous de prendre parti pour Baal, est-ce à vous de venir à son secours ? Quiconque prendra parti pour Baal sera mis à mort avant le matin. Si Baal est un dieu, qu'il défende lui-même sa cause puisqu'on a démoli son autel.
\VS{32}Et en ce jour on donna à Gédéon le nom de Jerubbaal, en disant : Que Baal défende sa cause, puisque Gédéon a démoli son autel.
\VS{33}Tout Madian, Amalek, et les fils de l’orient se rassemblèrent ;  ils passèrent le Jourdain et campèrent dans la vallée de Jizréel.
\VS{34}Gédéon fut revêtu de l'Esprit de Yahweh ; il sonna du shofar et Abiézer fut convoqué pour marcher à sa suite\FTNT{Jg. 11:29 ; Jg. 13:25.}.
\VS{35}Il envoya des messagers dans tout Manassé qui fut aussi convoqué pour marcher à sa suite. Puis il envoya des messagers dans Aser, dans Zabulon et dans Nephthali, qui montèrent à leur rencontre.
\VS{36}Gédéon dit à Dieu : Si tu veux délivrer Israël par ma main, comme tu l'as dit,
\VS{37}voici, je vais mettre une toison de laine dans l'aire de battage ; si la toison seule se couvre de rosée et que tout le terrain reste sec, je connaîtrai que tu délivreras Israël par ma main, comme tu l’as dit.
\VS{38}Et il arriva ainsi. Le jour suivant, il se leva de bon matin, pressa la toison et en fit sortir la rosée qui donna de l’eau plein une coupe.
\VS{39}Gédéon dit encore à Dieu : Que ta colère ne s'enflamme pas contre moi, et je ne parlerai plus que cette fois : Je te prie, je voudrais seulement faire encore une épreuve avec la toison : Que la toison seule reste sèche et que tout le terrain se couvre de rosée.
\VS{40}Et Dieu fit ainsi cette nuit-là. La toison seule resta sèche, et tout le terrain se couvrit de rosée.
\Chap{7}
\TextTitle{Yahweh sélectionne un petit nombre pour le combat}
\VerseOne{}Jerubbaal qui est Gédéon, et tout le peuple qui était avec lui, se levèrent de bon matin et campèrent près de la source de Harod. Le camp de Madian était au nord, vers la colline de Moré, dans la vallée.
\VS{2}Yahweh dit à Gédéon : Le peuple qui est avec toi est trop nombreux pour que je livre Madian entre ses mains, de peur qu'Israël ne se glorifie contre moi, en disant : C’est ma main qui m'a délivré.
\VS{3}Maintenant donc fais plublier ceci aux oreilles du peuple, et qu'on dise : Que celui qui est craintif et qui a peur s’en retourne et s’éloigne de la montagne de Galaad. Vingt-deux mille hommes parmi le peuple s'en retournèrent et il en resta dix mille\FTNT{De. 20:8.}.
\VS{4}Yahweh dit à Gédéon : Le peuple est encore trop nombreux. Fais-les descendre vers l'eau et là je les épurerai\FTNT{C’est Dieu qui qualifie ses ouvriers, il les éprouve et les épure pour les rendre inébranlables. Voir le test de l’épreuve des Hébreux dans le désert de Sinaï (De. 8).} ; et celui dont je te dirai : Que celui-ci aille avec toi, ira avec toi ; et celui dont je te dirai : Que celui-ci n’aille pas avec toi, n’ira pas avec toi.
\VS{5}Il fit donc descendre le peuple vers l'eau ; et Yahweh dit à Gédéon : Tous ceux qui laperont l'eau avec la langue comme lape le chien, tu les sépareras de tous ceux qui se mettront à genoux pour boire\FTNT{Ps. 110:7.}.
\VS{6}Ceux qui lapèrent l’eau en la portant à la bouche avec leur main furent au nombre de trois cents hommes et tout le reste du peuple se mit à genoux pour boire.
\VS{7}Alors Yahweh dit à Gédéon : C’est par les trois cents hommes qui ont lapé, que je vous délivrerai et que je livrerai Madian entre tes mains. Que tout le reste du peuple s'en aille donc chacun chez soi.
\VS{8}Ainsi le peuple prit entre ses mains des provisions et ses shofars. Gédéon renvoya tous les hommes d'Israël chacun dans sa tente et il retint les trois cents hommes. Or le camp de Madian était au-dessous de lui, dans la vallée.
\TextTitle{Victoire de Gédéon sur Madian}
\VS{9}Et il arriva cette nuit-là que Yahweh lui dit : Lève-toi, descends au camp car je l'ai livré entre tes mains.
\VS{10}Si tu crains de descendre, descends-y avec Pura, ton serviteur.
\VS{11}Tu écouteras ce qu'ils diront et après cela, tes mains seront fortifiées ; descends donc au camp. Il descendit avec Pura, son serviteur, jusqu'aux avant-postes du camp.
\VS{12}Or Madian, Amalek et tous les fils de l'orient étaient répandus dans la vallée comme des sauterelles, tant il y en avait, et leurs chameaux étaient sans nombre, comme le sable qui est sur le bord de la mer, tant il y en avait\FTNT{Jg. 6:3-33.}.
\VS{13}Gédéon arriva ; et voici, un homme racontait à son compagnon un songe. Il lui disait : Voici, j'ai eu un songe ; il me semblait qu'un gâteau de pain d'orge roulait dans le camp de Madian ; et il est venu heurter jusqu’à la tente et elle est tombée ; il l’a retournée sens dessus dessous et elle a été renversée.
\VS{14}Alors son compagnon répondit et dit : Ce n'est pas autre chose que l'épée de Gédéon, fils de Joas, homme d'Israël ; Dieu a livré Madian et tout le camp entre ses mains.
\VS{15}Lorsque Gédéon eut entendu le récit du songe et son interprétation, il se prosterna, revint au camp d'Israël et dit : Levez-vous car Yahweh a livré le camp de Madian entre vos mains.
\VS{16}Puis il divisa les trois cents hommes en trois corps et il leur donna à chacun des shofars à la main et des cruches vides, avec des flambeaux dans les cruches.
\VS{17}Il leur dit : Regardez-moi et faites comme je ferai. Dès que je serai arrivé à l’extrémité du camp, vous ferez comme je ferai.
\VS{18}Quand je sonnerai du shofar, moi et tous ceux qui sont avec moi, alors vous sonnerez aussi du shofar tout autour du camp et vous direz : Pour Yahweh et pour Gédéon !
\VS{19}Gédéon et les cent hommes qui étaient avec lui arrivèrent à l’extrémité du camp, au commencement de la veille de la nuit, comme on venait de placer les gardes. Ils sonnèrent du shofar et  brisèrent les cruches qu'ils avaient à la main.
\VS{20}Ainsi les trois corps sonnèrent du shofar, et brisèrent les cruches ; ils saisirent de la main gauche les flambeaux et de la main droite les shofars pour sonner et ils s’écrièrent : L'épée de Yahweh et de Gédéon !
\VS{21}Ils restèrent chacun à sa place autour du camp, et tout le camp se mit à courir ça et là, à pousser des cris et à prendre la fuite.
\VS{22}Car comme les trois cents hommes sonnèrent encore du shofar, Yahweh leur fit tourner l'épée les uns contre les autres. Le camp s'enfuit jusqu'à Beth-Schitta, vers Tseréra, jusqu'au bord d'Abel-Mehola, près de Tabbath\FTNT{1 S. 14:20 ; Ez. 38:21.}.
\VS{23}Les hommes d'Israël, à savoir ceux de Nephthali, d'Aser et de tout Manassé, se rassemblèrent, et ils poursuivirent Madian.
\VS{24}Alors Gédéon envoya des messagers dans toute la montagne d'Ephraïm, pour leur dire : Descendez pour aller à la rencontre de Madian, et coupez-leur les premiers le passage des eaux jusqu'à Beth-Bara et celui du Jourdain. Tous les hommes d'Ephraïm se rassemblèrent, et ils s’emparèrent du passage des eaux jusqu’à Beth-Bara et de celui du Jourdain.
\VS{25}Ils saisirent deux des chefs de Madian, Oreb et Zeeb ; ils tuèrent Oreb au rocher d’Oreb, et ils tuèrent Zeeb au pressoir de Zeeb. Ils poursuivirent Madian, et ils apportèrent les têtes de Oreb et de Zeeb à Gédéon, de l’autre côté du Jourdain\FTNT{Ps. 83:11 ; Es. 10:26.}.
\Chap{8}
\TextTitle{Poursuite de Zébach et Tsalmunna ; exécution des rois de Madian}
\VerseOne{}Alors les hommes d'Ephraïm dirent à Gédéon : Que signifie cette manière d’agir envers nous ? Pourquoi ne pas nous avoir appelés quand tu es allé à la guerre contre Madian ? Et ils s'emportèrent fortement contre lui\FTNT{Jg. 12:1.}.
\VS{2}Et il leur répondit : Qu'ai-je fait maintenant au prix de ce que vous avez fait ? Les grappillages d'Ephraïm ne sont-ils pas meilleurs que la vendange d'Abiézer ?
\VS{3}Dieu a livré entre vos mains les chefs de Madian, Oreb et Zeeb. Qu'ai-je pu faire au prix de ce que vous avez fait ? Et leur esprit fut apaisé envers lui lorsqu’il eut ainsi parlé.
\VS{4}Gédéon arriva au Jourdain, et il le passa, lui et les trois cents hommes qui étaient avec lui, fatigués, mais poursuivant toujours l'ennemi.
\VS{5}C'est pourquoi il dit aux gens de Succoth : Donnez, je vous prie, quelques pains aux hommes qui m’accompagnent, car ils sont fatigués, et ainsi je poursuivrai Zébach et Tsalmunna, rois de Madian.
\VS{6}Mais les chefs de Succoth répondirent : La main de Zébach et celle de Tsalmunna sont-elles déjà en ton pouvoir, pour que nous donnions du pain à ton armée ?
\VS{7}Et Gédéon dit : Eh bien ! Quand Yahweh aura livré Zébach et Tsalmunna entre mes mains, je foulerai au pied votre chair avec des épines du désert et avec des chardons.
\VS{8}Puis de là il monta à Penuel, et il fit la même demande aux gens de Penuel. Les gens de Penuel lui répondirent comme avaient répondu ceux de Succoth.
\VS{9}Et il dit aussi aux gens de Penuel : Quand je reviendrai en paix, je démolirai cette tour.
\VS{10}Zébach et Tsalmunna étaient à Karkor et leurs armées avec eux, environ quinze mille hommes, tous ceux qui étaient restés de l'armée entière des fils de l’orient ; cent vingt mille hommes tirant l’épée avaient été tués.
\VS{11}Gédéon monta par le chemin de ceux qui habitent sous les tentes, à l’orient de Nobach et de Jogbeha, et il battit l'armée, qui se croyait en sûreté.
\VS{12}Et comme Zébach et Tsalmunna s'enfuyaient, il les poursuivit, et prit les deux rois de Madian, Zébach et Tsalmunna, et mit en déroute toute l'armée\FTNT{Ps. 83:11.}.
\TextTitle{Vengeance sur Succoth et Penuel ; exécution de Zébach et Tsalmunna}
\VS{13}Puis Gédéon, fils de Joas, revint de la bataille par la montée de Hérès.
\VS{14}Il saisit un garçon d’entre les hommes de Succoth, il l'interrogea, et ce garçon lui donna par écrit le nom des chefs et des anciens de Succoth, au nombre de soixante-dix-sept hommes.
\VS{15}Et il vint auprès de gens de Succoth, et leur dit : Voici Zébach et Tsalmunna, au sujet desquels vous m'avez insulté, en disant : La main de Zébach et celle de Tsalmunna sont-elles déjà en ton pouvoir, pour que nous donnions du pain à tes hommes fatigués ?
\VS{16}Il prit donc les anciens de la ville et châtia les hommes de Succoth avec des épines du désert et des chardons.
\VS{17}Il démolit la tour de Penuel, et tua les gens de la ville.
\VS{18}Puis il dit à Zébach et à Tsalmunna : Comment étaient les hommes que vous avez tués à Thabor ? Ils répondirent : Ils étaient entièrement comme toi, chacun d'eux avait l'air d'un fils de roi.
\VS{19}Il leur dit : C'étaient mes frères, fils de ma mère. Yahweh est vivant, si vous les aviez laissés vivre, je ne vous tuerais pas.
\VS{20}Puis il dit à Jéther, son premier-né : Lève-toi, tue-les ! Mais le jeune garçon ne tira pas son épée, car il avait peur, car il était encore un enfant.
\VS{21}Et Zébach et Tsalmunna dirent : Lève-toi toi-même, et jette-toi sur nous ! Car tel est l'homme, telle est sa force. Et Gédéon se leva, et tua Zébach et Tsalmunna. Il prit ensuite les croissants qui étaient aux cous de leurs chameaux.
\TextTitle{Gédéon recommande au peuple le règne de Yahweh}
\VS{22}Les hommes d'Israël dirent tous d'un commun accord à Gédéon : Domine sur nous, tant toi que ton fils, et le fils de ton fils, car tu nous as délivrés de la main de Madian.
\VS{23}Gédéon leur répondit : Je ne dominerai pas sur vous, et mon fils ne dominera pas sur vous ; c'est Yahweh qui dominera sur vous\FTNT{De. 17:15.}.
\TextTitle{Gédéon introduit une occasion de chute en Israël}
\VS{24}Mais Gédéon leur dit : J’ai une demande à vous faire : Donnez-moi chacun les anneaux que vous avez eus pour butin. Les ennemis avaient des anneaux d'or, car ils étaient Ismaélites.
\VS{25}Ils répondirent : Nous les donnerons volontiers. Et ils étendirent un manteau sur lequel chacun jeta les anneaux de son butin.
\VS{26}Le poids des anneaux d'or que Gédéon demanda fut de mille sept cents sicles d'or, sans les croissants, les pendants d'oreilles, et les vêtements de pourpre que portaient les rois de Madian, et sans les colliers qui étaient aux cous de leurs chameaux.
\VS{27}Puis Gédéon en fit un  éphod\FTNT{Sous Moïse, il y avait deux sortes d'éphods, le premier était de simple lin pour les sacrificateurs, et le deuxième de broderie pour le souverain sacrificateur. Comme celui des simples sacrificateurs n'avait rien de particulier, Moïse ne s'est pas arrêté à le décrire. Mais il décrit longuement celui du souverain sacrificateur. (Ex. 28:6-9). Il était composé d'or, d'hyacinthe, de pourpre, de cramoisi, de coton retors ; c'était un tissu de différentes couleurs. Il y avait à l'endroit de l'éphod qui venait sur les deux épaules du souverain sacrificateur, deux grosses pierres précieuses, qui étaient chargées du nom des douze tribus d’Israël, six noms sur chaque pierre. A l'endroit où l'éphod se croisait sur la poitrine du grand prêtre, il y avait un ornement carré, nommé le rational, en hébreu «~choschen~», dans lequel étaient enchâssées douze pierres précieuses, où l'on avait gravé les noms des douze tribus d'Israël ; un sur chacune des pierres.}, et le mit dans sa ville, à Ophra, où il devint un objet de prostitution pour tout Israël ; il fut un piège pour Gédéon et pour sa maison.
\TextTitle{Fin de la vie de Gédéon ; rechute d’Israël après sa mort}
\VS{28}Ainsi Madian fut humilié devant les enfants d'Israël, et il ne leva plus la tête. Le pays fut en repos pendant quarante ans, durant les jours de Gédéon.
\VS{29}Jerubbaal, fils de Joas s’en retourna dans sa ville, et demeura dans sa maison.
\VS{30}Gédéon eut soixante-dix fils, issus de ses reins, car il eut plusieurs femmes.
\VS{31}Sa concubine, qui était à Sichem, lui enfanta aussi un fils, et il lui donna le nom d’Abimélec.
\VS{32}Puis Gédéon, fils de Joas, mourut après une heureuse vieillesse ; et il fut enseveli dans le sépulcre de Joas, son père, à Ophra, qui appartenait à la famille d’Abiézer.
\TextTitle{Rechute dans l'idolâtrie}
\VS{33}Et il arriva après que Gédéon fut mort, que les enfants d'Israël se détournèrent et se prostituèrent aux Baals, et ils établirent Baal-Berith pour leur dieu\FTNT{Jg. 2:11-17 ; 10:6.}.
\VS{34}Ainsi les enfants d'Israël ne se souvinrent pas de Yahweh, leur Dieu, qui les avait délivrés de la main de tous leurs ennemis qui les entouraient.
\VS{35}Et ils n'usèrent d'aucune loyauté envers la maison de Jerubbaal, de Gédéon, après tout le bien qu'il avait fait à Israël.
\Chap{9}
\TextTitle{Conspiration d’Abimélec pour régner sur Israël}
\VerseOne{}Et Abimélec, fils de Jerubbaal, s'en alla à Sichem vers les frères de sa mère, et leur parla, ainsi qu'à toute la maison du père de sa mère :
\VS{2}Je vous prie, faites entendre ces paroles à tous les seigneurs de Sichem : Lequel vous semble le meilleur, que soixante-dix hommes, tous fils de Jerubbaal, dominent sur vous, ou qu'un seul homme domine sur vous ? Et souvenez-vous que je suis votre os et votre chair\FTNT{Ge. 29:14.}.
\VS{3}Les frères de sa mère dirent de sa part toutes ces paroles aux oreilles de tous les seigneurs de Sichem, et leur cœur se tourna après Abimélec, car ils disaient : C'est notre frère.
\VS{4}Ils lui donnèrent soixante-dix sicles d'argent de la maison de Baal-Berith. Abimélec s'en servit pour acheter des hommes misérables et turbulents, qui allèrent après lui.
\VS{5}Et il vint dans la maison de son père à Ophra, et tua sur une seule pierre ses frères, fils de Jerubbaal, qui étaient soixante-dix hommes. Il ne resta que Jotham, le plus jeune fils de Jerubbaal, parce qu'il s'était caché.
\VS{6}Et tous les seigneurs de Sichem s'assemblèrent avec toute la maison de Millo ; ils vinrent, et firent d'Abimélec leur roi près du chêne à Sichem.
\VS{7}On le rapporta à Jotham, qui alla se tenir au sommet de la montagne de Garizim, et les appelant, il dit en élevant la voix : Écoutez-moi, seigneurs de Sichem, et que Dieu vous entende !
\VS{8}Les arbres allèrent pour oindre un roi, et ils dirent à l'olivier : Règne sur nous.
\VS{9}Mais l'olivier leur répondit : Renoncerai-je à mon huile, par laquelle Dieu et les hommes sont honorés, pour aller m'agiter sur les arbres\FTNT{Ps. 104:15.} ?
\VS{10}Puis les arbres dirent au figuier : Viens, toi, règne sur nous.
\VS{11}Mais le figuier leur répondit : Renoncerai-je à ma douceur, et à mon bon fruit, pour aller m'agiter sur les arbres ?
\VS{12}Puis les arbres dirent à la vigne : Viens, toi, et règne sur nous.
\VS{13}Mais la vigne répondit : Renoncerai-je à mon vin, qui réjouit Dieu et les hommes, pour aller m'agiter sur les arbres ?
\VS{14}Alors tous les arbres dirent à l'épine : Viens, toi, et règne sur nous.
\VS{15}Et l'épine répondit aux arbres : Si c'est en vérité que vous m'oignez pour roi, venez, et réfugiez-vous sous mon ombrage ; sinon, que le feu sorte de l'épine, et qu'il dévore les cèdres du Liban.
\VS{16}Maintenant donc, est-ce en vérité et avec intégrité que vous avez agi en établissant Abimélec pour roi ? Avez-vous bien fait envers Jerubbaal et sa maison ? L'avez-vous fait selon les bienfaits qu'il a rendus de sa main ?
\VS{17}Car mon père a combattu pour vous, il a exposé sa vie devant vous, et vous a délivrés de la main de Madian ;
\VS{18}Mais vous vous êtes levés aujourd'hui contre la maison de mon père, et avez tué sur une pierre ses fils, soixante-dix hommes, et avez établi pour roi Abimélec, fils de sa servante, sur les habitants de Sichem, parce qu'il est votre frère.
\VS{19}Si, dis-je, vous avez agi aujourd'hui en vérité et avec intégrité envers Jerubbaal, et sa maison, réjouissez-vous d'Abimélec, et qu'il se réjouisse aussi de vous !
\VS{20}Sinon, que le feu sorte d'Abimélec et qu'il dévore les seigneurs de Sichem, et la maison de Millo ; et que le feu sorte des seigneurs de Sichem, et de la maison de Millo, et qu'il dévore Abimélec !
\VS{21}Puis Jotham s'enfuit rapidement ; il s'en alla à Beer, où il demeura loin d'Abimélec, son frère.
\TextTitle{Sichem se retourne contre Abimélec}
\VS{22}Abimélec gouverna sur Israël durant trois ans.
\VS{23}Alors Dieu envoya un mauvais esprit entre Abimélec et les seigneurs de Sichem, et les seigneurs de Sichem furent infidèles à Abimélec.
\VS{24}Afin que la violence faite aux soixante-dix fils de Jerubbaal vienne et que leur sang se tourne contre Abimélec, leur frère, qui les avait tués, et sur les seigneurs de Sichem, qui l'avaient aidé par leur main à tuer ses frères.
\VS{25}Les seigneurs de Sichem mirent des embûches sur le sommet des montagnes, des gens pillaient tous ceux qui passaient près d'eux sur le chemin. Cela fut rapporté à Abimélec.
\VS{26}Alors Gaal, fils d'Ebed, vint avec ses frères, et ils passèrent à Sichem. Les seigneurs de Sichem eurent confiance en lui.
\VS{27}Puis étant sortis aux champs, ils vendangèrent leurs vignes, foulèrent les raisins, et se donnèrent à des réjouissances ; ils entrèrent dans la maison de leur dieu, ils mangèrent et burent, et ils maudirent Abimélec.
\VS{28}Alors Gaal, fils d'Ebed, dit : Qui est Abimélec, et qui est Sichem pour que nous servions Abimélec ? N'est-il pas le fils de Jerubbaal et Zebul n'est-il pas son commissaire ? Servez plutôt les hommes de Hamor, père de Sichem ; mais pour quelle raison servirions-nous Abimélec ?
\VS{29}Plaise à Dieu ! Qu'on mette ce peuple sous mon pouvoir, et je chasserais Abimélec. Et il disait d'Abimélec : Multiplie ton armée, et sors !
\VS{30}Zebul, gouverneur de la ville, entendit les paroles de Gaal, fils d'Ebed, et sa colère s'enflamma.
\VS{31}Puis il envoya astucieusement des messagers vers Abimélec, pour lui dire : Voici, Gaal fils d'Ebed, et ses frères, sont entrés dans Sichem, et voici, ils assiègent la ville contre toi.
\VS{32}Maintenant donc, lève-toi de nuit, toi et le peuple qui est avec toi, et mets-toi en embuscade dans les champs.
\VS{33}Et le matin, au lever du soleil, tu te lèveras et tu te jetteras sur la ville. Gaal et le peuple qui est avec lui sortiront contre toi, ta main lui fera selon les forces que tu trouveras.
\VS{34}Abimélec et tout le peuple qui était avec lui se levèrent de nuit, et ils se mirent en embuscade contre Sichem, divisés en quatre bandes.
\VS{35}Alors Gaal, fils d'Ebed, sortit, et il se tint à l'entrée de la porte de la ville. Abimélec et tout le peuple qui était avec lui se levèrent de l'embuscade.
\VS{36}Gaal voyant le peuple, dit à Zebul : Voici un peuple qui descend du sommet des montagnes. Zebul lui dit : Tu vois l'ombre des montagnes comme des hommes.
\VS{37}Gaal, parla encore, et dit : C'est bien un peuple qui descend des hauteurs du pays, et une bande vient du chemin du chêne des devins.
\VS{38}Et Zebul lui dit : Où est donc ta bouche, toi qui disais : Qui est Abimélec, pour que nous le servions ? N'est-ce pas ici ce peuple que tu méprisais ? Sors maintenant, je te prie, et combats !
\VS{39}Alors, Gaal sortit conduisant les seigneurs de Sichem, et combattit contre Abimélec.
\VS{40}Abimélec le poursuivit, et il s'enfuit de devant lui, et plusieurs tombèrent morts jusqu'à l'entrée de la porte.
\VS{41}Abimélec s'arrêta à Aruma. Zebul repoussa Gaal et ses frères, afin qu'ils ne restent plus à Sichem.
\VS{42}Et il arriva, dès le lendemain, que le peuple sortit aux champs. Cela fut rapporté à Abimélec,
\VS{43}qui prit son peuple, et le divisa en trois bandes, et les mit en embuscade dans les champs. Ayant vu que le peuple sortait de la ville, il se leva contre eux, et les battit.
\VS{44}Abimélec et la bande qui était avec lui se répandirent, et se tinrent à l'entrée de la porte de la ville ; mais les deux autres bandes se jetèrent sur tous ceux qui étaient aux champs, et les battirent.
\VS{45}Ainsi Abimélec combattit contre la ville toute la journée ; il prit la ville, et tua le peuple qui y était. Il la rasa, et y sema du sel.
\VS{46}Ayant appris cela, tous les seigneurs de la tour de Sichem entrèrent dans la forteresse de la maison du dieu Baal-Berith\FTNT{Jg. 9:4 ; 8:33.}.
\VS{47}On rapporta à Abimélec que tous les seigneurs de la tour de Sichem s'étaient assemblés dans la forteresse.
\VS{48}Alors Abimélec monta sur la montagne de Tsalmon, lui et tout le peuple qui était avec lui. Il prit en main une hache, coupa une branche d'arbre, et l'ayant mise sur son épaule, la porta, et dit au peuple qui était avec lui : Avez-vous vu ce que j'ai fait ? Hâtez-vous de faire comme moi.
\VS{49}Chacun donc de tout le peuple coupa une branche, et ils marchèrent derrière Abimélec ; ils mirent ces branches tout autour de la forteresse, et y mirent le feu. Il brûlèrent la forteresse, et toutes les personnes de la tour de Sichem moururent ; au nombre d'environ mille, tant hommes que femmes.
\TextTitle{Abimélec meurt}
\VS{50}Puis Abimélec marcha contre Thébets, y mit son camp, et la prit.
\VS{51}Il y avait au milieu de la ville une forte tour, où s'enfuirent tous les hommes et toutes les femmes, et tous les seigneurs de la ville, et ayant fermé les portes après eux, ils montèrent sur le toit de la Tour.
\VS{52}Alors Abimélec alla jusqu'à la tour, l'attaqua, et s'approcha jusqu'à la porte pour la brûler par le feu.
\VS{53}Mais une femme jeta une pièce de meule de moulin sur la tête d'Abimélec, et lui brisa le crâne\FTNT{2 S. 11:21.}.
\VS{54}Rapidement, il appela le garçon qui portait ses armes, et lui dit : Tire ton épée, et tue-moi, de peur qu'on ne dise de moi : C'est une femme qui l'a tué. Le garçon le transperça, et il mourut\FTNT{1 S. 31:4.}.
\VS{55}Quand les hommes d'Israël virent qu'Abimélec était mort, ils s'en allèrent chacun en son lieu.
\VS{56}Ainsi Dieu rendit à Abimélec le mal qu'il avait fait contre son père, en tuant ses soixante-dix frères,
\VS{57}Et toute la méchanceté des hommes de Sichem ; Dieu, dis-je, la fit retourner sur leurs têtes ; et ainsi la malédiction de Jotham, fils de Jérubbaal, vint sur eux.
\Chap{10}
\TextTitle{Thola, juge en Israël}
\VerseOne{}Après Abimélec, Thola fils de Pua, fils de Dodo, homme d'Issacar, se leva pour délivrer Israël ; il habitait à Schamir, dans la montagne d'Ephraïm.
\VS{2}Il fut juge en Israël pendant vingt-trois ans ; puis il mourut, et fut enterré à Schamir.
\TextTitle{Jaïr, juge en Israël}
\VS{3}Après lui se leva Jaïr, le Galaadite, qui fut juge en Israël pendant vingt-deux ans.
\VS{4}Il avait trente fils, qui montaient sur trente ânons, et qui avaient trente villes, qu'on appelle jusqu'à ce jour bourgs de Jaïr, lesquelles sont situées au pays de Galaad\FTNT{Jg. 5:10.}.
\VS{5}Et Jaïr mourut, et fut enterré à Kamon.
\TextTitle{Idolâtrie d’Israël et oppression par ses ennemis}
\VS{6}Puis les enfants d'Israël firent encore ce qui est mal aux yeux de Yahweh ; et servirent les Baals et les Astartés, les dieux de Syrie, les dieux de Sidon, les dieux de Moab, les dieux des fils d'Ammon, et les dieux des Philistins, et ils abandonnèrent Yahweh, et ne le servirent plus\FTNT{Jg. 2:11 ; 3:7 ; 8:33.}.
\VS{7}Alors la colère de Yahweh s'enflamma contre Israël, et il les vendit entre les mains des Philistins, et des fils d'Ammon.
\VS{8}Ils opprimèrent et écrasèrent les enfants d'Israël cette année-là, et pendant dix-huit ans tous les enfants d'Israël qui étaient au-delà du Jourdain, au pays des Amoréens en Galaad.
\VS{9}Même les fils d'Ammon passèrent le Jourdain pour combattre contre Juda, contre Benjamin, et contre la maison d'Ephraïm. Israël fut dans une grande détresse.
\VS{10}Alors les enfants d'Israël crièrent à Yahweh, en disant : Nous avons péché contre toi, et certes, nous avons abandonné notre Dieu et nous avons servi les Baals.
\VS{11}Mais Yahweh répondit aux enfants d'Israël : N'avez-vous pas été opprimés par les Egyptiens, les Amoréens, les fils d'Ammon et les Philistins ?
\VS{12}Et lorsque les Sidoniens, Amalek et Maon, vous opprimèrent, et que vous criâtes à moi, ne vous ai-je pas délivrés de leurs mains ?
\VS{13}Mais vous, vous m'avez abandonné, et vous avez servi d'autres dieux. C'est pourquoi je ne vous délivrerai plus.
\VS{14}Allez et criez vers les dieux que vous avez choisis ; qu'ils vous délivrent au temps de votre détresse !
\VS{15}Mais les enfants d'Israël répondirent à Yahweh : Nous avons péché ; traite-nous comme tu le trouveras bon. Nous te prions seulement que tu nous délivres aujourd'hui !
\VS{16}Alors ils ôtèrent du milieu d'eux les dieux des étrangers, et servirent Yahweh, qui fut affligé des souffrances d'Israël.
\VS{17}Les fils d'Ammon se rassemblèrent et campèrent en Galaad, et les enfants d'Israël se rassemblèrent et campèrent à Mitspa.
\VS{18}Le peuple, les chefs de Galaad se dirent l'un à l'autre : Qui sera l'homme qui commencera à combattre contre les fils d'Ammon ? Il sera chef de tous les habitants de Galaad.
\Chap{11}
\TextTitle{Jephté, juge en Israël}
\VerseOne{}Or Jephthé, le Galaadite, était un fort et vaillant homme. Il était le fils d'une femme prostituée ; et c'est Galaad qui l'avait engendré.
\VS{2}La femme de Galaad lui enfanta des fils ; et quand les fils de cette femme furent grands, ils chassèrent Jephthé, en lui disant : Tu n'auras pas d'héritage dans la maison de notre père, car tu es fils d'une autre femme.
\VS{3}Jephthé s'enfuit donc de devant ses frères, et habita au pays de Tob. Des misérables se rassemblèrent auprès de Jephthé, et ils sortirent dehors avec lui\FTNT{Jg. 9:4 ; 1 S. 22:2 ; 1 S 10:6-8.}.
\VS{4}Et il arriva, quelque temps après, les fils d'Ammon firent la guerre à Israël.
\VS{5}Et comme les fils d'Ammon faisaient la guerre à Israël, les anciens de Galaad s'en allèrent pour emmener Jephthé du pays de Tob.
\VS{6}Ils dirent à Jephthé : Viens, et sois notre chef, afin que nous combattions contre les fils d'Ammon.
\VS{7}Jephthé répondit aux anciens de Galaad : N'est-ce pas vous qui m'avez haï et chassé de la maison de mon père ? Pourquoi êtes-vous venus à moi maintenant que vous êtes dans la détresse ?
\VS{8}Alors les anciens de Galaad dirent à Jephthé : La raison pour laquelle nous retournons à toi maintenant, c'est afin que tu viennes avec nous, que tu combattes contre les fils d'Ammon, et que tu sois notre chef, celui de tous les habitants de Galaad.
\VS{9}Jephthé répondit aux anciens de Galaad : Si vous me ramenez pour combattre contre les fils d'Ammon, et que Yahweh les livre devant moi, je serai votre chef.
\VS{10}Les anciens de Galaad dirent à Jephthé : Que Yahweh nous entende, et qu'il juge, si nous ne faisons pas ce que tu dis.
\VS{11}Jephthé donc s'en alla avec les anciens de Galaad. Le peuple le mit à sa tête et l'établit pour chef, et Jephthé déclara devant Yahweh, à Mitspa, toutes les paroles qu'il avait dites.
\VS{12}Puis Jephthé envoya des messagers au roi des fils d'Ammon, pour lui dire : Qu'y a-t-il entre toi et moi, que tu viennes contre moi pour faire la guerre à mon pays ?
\VS{13}Le roi des fils d'Ammon répondit aux messagers de Jephthé : C'est parce qu'Israël a pris mon pays quand il est monté d'Egypte, depuis l'Arnon jusqu'à Jabbok, et même jusqu'au Jourdain. Maintenant rends-le de bon gré.
\VS{14}Mais Jephthé envoya encore des messagers au roi des fils d'Ammon,
\VS{15}qui lui dirent : Ainsi parle Jephthé : Israël n'a rien pris du pays de Moab, ni du pays des fils d'Ammon.
\VS{16}Mais lorsqu’Israël est monté d'Egypte, il est venu par le désert jusqu'à la Mer Rouge et il a atteint Kadès.
\VS{17}Alors Israël envoya des messagers au roi d'Edom, pour lui dire : Que je passe, je te prie, par ton pays. Le roi d'Edom ne voulut pas l'entendre. Il en envoya aussi au roi de Moab, qui ne voulut pas non plus l'entendre. Et Israël demeura à Kadès.
\VS{18}Puis il marcha par le désert, tourna le pays d'Edom et le pays de Moab, et vint à l'orient du pays de Moab ; il campa au-delà de l'Arnon, et n'entra pas sur les frontières de Moab, car l'Arnon est la frontière de Moab.
\VS{19}Mais Israël envoya des messagers à Sihon, roi des Amoréens, roi de Hesbon, auquel Israël dit : Laisse-nous passer par ton pays jusqu'au lieu où nous allons.
\VS{20}Mais Sihon n'eut pas assez confiance en Israël pour le laisser passer sur son territoire ; il rassembla tout son peuple, ils campèrent vers Jahats, et combattirent contre Israël.
\VS{21}Et Yahweh, le Dieu d'Israël, livra Sihon et tout son peuple entre les mains d'Israël, qui les battit. Israël prit possession de tout le pays des Amoréens qui habitaient cette terre.
\VS{22}Ils conquirent donc tout le pays des Amoréens, depuis l'Arnon jusqu'à Jabbok, et depuis le désert jusqu'au Jourdain.
\VS{23}Et maintenant que Yahweh, le Dieu d'Israël, a dépossédé les Amoréens de devant son peuple d'Israël, aurais-tu la possession de leur pays ?
\VS{24}Ce que ton dieu Kemosch te donne à posséder, ne le posséderais-tu pas ? Et tout ce que Yahweh, notre Dieu, a mis en notre possession devant nous, nous ne le posséderions pas !
\VS{25}Or maintenant vaux-tu mieux en quelque sorte que ce soit que Balak, fils de Tsippor, roi de Moab ? A-t-il contesté et combattu contre Israël ?
\VS{26}Voilà trois cents ans qu'Israël demeure à Hesbon, et dans les villes de son ressort, à Aroër, et dans les villes de son ressort, et dans toutes les villes qui sont le long de l'Arnon : Pourquoi ne les avez-vous pas saisies pendant ce temps-là ?
\VS{27}Je ne t'ai pas offensé, mais tu fais mal de me faire la guerre. Que Yahweh, qui est le juge, juge aujourd'hui entre les enfants d'Israël et les fils d'Ammon !
\VS{28}Le roi des fils d'Ammon n'écouta pas les paroles que Jephthé lui fit dire.
\VS{29}L'Esprit de Yahweh fut sur Jephthé. Il passa au travers de Galaad et de Manassé ; il passa jusqu'à Mitspé de Galaad, et de Mitspé de Galaad, il passa jusqu'aux fils d'Ammon.
\TextTitle{Jephté fait un vœu ; Ammon livré entre ses mains}
\VS{30}Jephthé fit un vœu à Yahweh, et dit : Si tu livres les fils d'Ammon entre mes mains,
\VS{31}alors tout ce qui sortira des portes de ma maison au-devant de moi, quand je retournerai en paix chez les fils d'Ammon, sera consacré à Yahweh, et je l'offrirai en holocauste.
\VS{32}Jephthé passa jusqu'où étaient les fils d'Ammon, et Yahweh les livra entre ses mains.
\VS{33}Il les battit par une grande défaite, depuis Aroër jusqu'à Minnith, espace qui renfermait vingt villes, et jusqu'à Abel-Keramim. Et les fils d'Ammon furent humiliés devant les fils d'Israël.
\VS{34}Puis comme Jephthé retourna à Mitspa dans sa maison, voici, sa fille, qui était seule et unique, sans qu'il eût d'autres fils ou filles, sortit au-devant de lui avec des tambourins et des danses.
\VS{35}Et il arriva qu'aussitôt qu'il l'eut aperçue, il déchira ses vêtements, et dit : Ha ! Ma fille ! Tu m'as entièrement abaissé, tu es du nombre de ceux qui me troublent ! J'ai ouvert ma bouche à Yahweh, et je ne puis le révoquer.
\VS{36}Elle répondit : Mon père, si tu as ouvert ta bouche à Yahweh, fais-moi selon ce qui est sorti de ta bouche, puisque Yahweh t'a fait vengeance de tes ennemis, des fils d'Ammon.
\VS{37}Toutefois, elle dit à son père : Que ceci me soit fait : Laisse-moi pendant deux mois ! Je m'en irai, je descendrai par les montagnes, et je pleurerai ma virginité, avec mes compagnes.
\VS{38}Il répondit : Va ! Et il la laissa aller pour deux mois. Elle s'en alla donc avec ses compagnes, et pleura sa virginité dans les montagnes.
\VS{39}Et au bout de deux mois, elle retourna vers son père ; et il lui fit selon le vœu qu'il avait fait\FTNT{Yahweh interdit les sacrifices humains (Lé. 20:2-5 ; Lé. 21 ; De. 12:31 ; De. 18:10).}. Elle n'avait pas connu d'homme. Dès lors, ce fut une coutume en Israël,
\VS{40}tous les ans les filles d'Israël allaient pour célébrer la fille de Jephthé, le Galaadite, quatre jours par an.
\Chap{12}
\TextTitle{Querelle entre Jephté et Ephraïm}
\VerseOne{}Or les hommes d'Ephraïm se rassemblèrent, passèrent par le nord, et dirent à Jephthé : Pourquoi es-tu passé pour combattre contre les enfants d'Ammon, sans nous avoir appelés pour aller avec toi ? Nous brûlerons ta maison, et toi aussi\FTNT{Jg. 8:1.}.
\VS{2}Et Jephthé leur dit : J’ai eu un grand différend avec les enfants de Ammon, moi et mon peuple, et quand je vous ai appelés, vous ne m’avez point délivré de leurs mains.
\VS{3}Voyant que vous ne me délivriez pas, j'ai exposé ma vie, et je suis passé jusqu'où étaient les fils d'Ammon. Yahweh les a livrés entre mes mains. Pourquoi donc aujourd'hui montez-vous vers moi pour me faire la guerre ?
\VS{4}Puis Jephthé assembla tous les hommes de Galaad, et combattit contre Ephraïm. Les hommes de Galaad battirent Ephraïm, parce qu'ils disaient : Vous êtes des fugitifs d'Ephraïm ! Galaad est au milieu d'Ephraïm, au milieu de Manassé !
\VS{5}Les Galaadites se saisirent des gués du Jourdain du côté d'Ephraïm. Et quand l'un des fuyards d'Ephraïm disait : Que je passe ! Les hommes de Galaad lui disaient : Es-tu Ephraïmite ? Il répondait : Non.
\VS{6}Alors ils lui disaient : dis un peu « Schibboleth ». Et il disait « Sibboleth », car il ne pouvait pas le prononcer. Sur quoi, se saisissant de lui, ils le tuaient aux gués du Jourdain. En ce temps-là quarante-deux mille hommes d'Ephraïm périrent.
\VS{7}Jephthé fut juge en Israël pendant six ans ; puis Jephthé le Galaadite mourut, et fut enterré dans l'une des villes de Galaad.
\TextTitle{Ibstan, juge en Israël}
\VS{8}Après lui, Ibtsan de Bethléhem fut juge en Israël.
\VS{9}Il eut trente fils, il envoya trente filles au-dehors, et il fit venir du dehors trente filles pour ses fils. Il fut juge en Israël pendant sept ans.
\VS{10}Puis Ibtsan mourut, et fut enterré à Bethléhem.
\TextTitle{Le juge Elon}
\VS{11}Après lui, Elon de Zabulon fut juge en Israël pendant dix ans.
\VS{12}Puis Elon de Zabulon mourut, et fut enterré à Ajalon, dans le pays de Zabulon.
\TextTitle{Le juge Abdon}
\VS{13}Après lui, Abdon, fils d'Hillel, le Pirathonite, fut juge en Israël.
\VS{14}Il eut quarante fils et trente petits-fils, qui montaient sur soixante-dix ânons. Il fut juge en  Israël pendant huit ans\FTNT{Jg. 10:4. }.
\VS{15}Puis Abdon, fils d'Hillel, le Pirathonite, mourut, et fut enterré à Pirathon, dans le pays d'Ephraïm, sur la montagne des Amalécites.
\Chap{13}
\TextTitle{Israël à nouveau asservi pas les Philistins}
\VerseOne{}Et les enfants d’Israël recommencèrent à faire ce qui est mauvais aux yeux de Yahweh ; et Yahweh les livra entre les mains des Philistins, pendant quarante ans.
\TextTitle{Naissance du juge Samson}
\VS{2}Or il y avait un homme de Tsorea, de la famille des Danites, dont le nom était Manoach. Sa femme était stérile, et n'enfantait pas.
\VS{3}L’Ange de Yahweh apparut à la femme, et lui dit : Voici, tu es stérile, et tu n'as jamais eu d'enfants ; mais tu concevras, et tu enfanteras un fils.
\VS{4}Prends donc bien garde dès maintenant de ne boire ni vin ni liqueur forte, et de ne manger aucune chose impure.
\VS{5}Car voici tu vas être enceinte et tu enfanteras un fils. Le rasoir ne s'élèvera pas sur sa tête, parce que l'enfant sera Naziréen\FTNT{Naziréen vient du mot  «~nazir~» qui signifie «~consacré~» ou «~séparé~». Voir No. 6.} pour Dieu dès le ventre de sa mère ; et ce sera lui qui commencera à délivrer Israël de la main des Philistins.
\VS{6}Et La femme vint, et parla à son mari en disant : Un homme de Dieu est venu vers moi et il avait l'aspect d'un ange de Dieu, un aspect fort redoutable. Je ne lui ai pas demandé d'où il était, et il ne m'a pas déclaré son nom.
\VS{7}Mais il m'a dit : Tu vas être enceinte, et tu enfanteras un fils ; maintenant donc ne bois ni vin ni liqueur forte, et ne mange aucune chose impure, car cet enfant sera Naziréen pour Dieu dès le ventre de sa mère jusqu'au jour de sa mort.
\TextTitle{Prière de Manoach}
\VS{8}Et Manoach pria instamment Yahweh, et dit : Ah ! Seigneur, que l'homme de Dieu que tu as envoyé vienne encore vers nous, et qu'il nous enseigne ce que nous devons faire à l'enfant quand il naîtra !
\VS{9}Et Dieu exauça la prière de Manoach, et l'Ange de Dieu vint encore vers la femme lorsqu'elle était assise dans un champ ; mais Manoach, son mari, n'était pas avec elle.
\VS{10}Et la femme courut vite le rapporter à son mari, en lui disant : Voici, l'homme qui était venu vers moi l'autre jour m'est apparu.
\VS{11}Manoach se leva, suivit sa femme, et venant vers l'homme, il lui dit : Es-tu cet homme qui a parlé à cette femme ? Il répondit : C'est moi.
\VS{12}Manoach dit : Tout ce que tu as dit arrivera, quelle conduite faudra-t-il tenir envers l'enfant, et que lui faudra-t-il faire ?
\VS{13}L'Ange de Yahweh répondit à Manoach : La femme se gardera de tout ce que je lui ai dit.
\VS{14}Elle ne mangera rien qui sorte de la vigne, elle ne boira ni vin ni liqueur forte, et ne mangera aucune chose impure ; elle prendra garde à tout ce que je lui ai ordonné.
\VS{15}Alors Manoach dit à l'Ange de Yahweh : Permets que nous te retenions, et que nous apprêtions un chevreau en ta présence.
\VS{16}Et l'Ange de Yahweh répondit à Manoach : Quand tu me retiendrais, je ne mangerai pas de ton mets ; mais si tu fais un holocauste, tu l'offriras à Yahweh. Manoach ne savait pas que ce fût un Ange de Yahweh.
\VS{17}Et Manoach dit à l'Ange de Yahweh : Quel est ton nom, afin que nous te rendions les honneurs lorsque ta parole viendra ?
\VS{18}Et l'Ange de Yahweh lui répondit : Pourquoi demandes-tu mon nom ? Il est merveilleux.
\VS{19}Alors Manoach prit un chevreau, et une offrande, et les offrit à Yahweh sur le rocher. Il se produisit une chose merveilleuse à la vue de Manoach et de sa femme.
\VS{20}Comme la flamme montait de dessus l'autel vers les cieux, l'Ange de Yahweh monta aussi avec la flamme de l'autel. A cette vue, Manoach et sa femme tombèrent la face contre terre.
\VS{21}L'Ange de Yahweh n'apparut plus à Manoach ni à sa femme. Alors Manoach sut que c'était l'Ange de Yahweh.
\VS{22}Et Manoach dit à sa femme : Certainement nous mourrons, car nous avons vu Dieu.
\VS{23}Mais sa femme lui répondit : Si Yahweh avait voulu nous faire mourir, il n'aurait pas pris de nos mains l'holocauste ni l'offrande, il ne nous aurait pas fait voir toutes ces choses ni fait entendre les choses que nous avons entendues.
\VS{24}Puis cette femme enfanta un fils, et elle l'appela du nom de Samson. L'enfant devint grand, et Yahweh le bénit.
\VS{25}Et l'Esprit de Yahweh commença à l'agiter à Machané-Dan, entre Tsorea et Eschthaol.
\Chap{14}
\TextTitle{Yahweh, le maître des évènements}
\VerseOne{}Samson descendit à Thimna, et il y vit une femme d'entre les filles des Philistins.
\VS{2}Etant remonté dans sa maison, il le déclara à son père et à sa mère, en disant : J'ai vu une femme à Thimna d'entre les filles des Philistins ; prenez-la maintenant, afin qu'elle soit ma femme.
\VS{3}Son père et sa mère lui dirent : N'y a-t-il pas de femme parmi les filles de tes frères et parmi tout notre peuple, pour que tu ailles prendre une femme d'entre les Philistins, ces incirconcis ? Et Samson dit à son père : Prenez-la pour moi, car elle est droite à mes yeux.
\VS{4}Mais son père et sa mère ne savaient pas que cela venait de Yahweh : Car Samson cherchait une occasion de dispute de la part des Philistins. Or en ce temps-là, les Philistins dominaient sur Israël.
\TextTitle{L'énigme de Samson}
\VS{5}Samson descendit avec son père et sa mère à Thimna. Ils allèrent jusqu'aux vignes de Thimna, et voici, un jeune lion rugissant vint à sa rencontre.
\VS{6}Et l'Esprit de Yahweh saisit Samson ; sans avoir rien en sa main, il déchira le lion comme on déchire un chevreau. Il ne déclara pas à son père ni à sa mère ce qu'il avait fait\FTNT{1 S. 17:34-35.}.
\VS{7}Il descendit et parla à la femme, et elle fut trouvée droite à ses yeux.
\VS{8}Puis quelque temps après, il retourna à Thimna pour la prendre, et se détourna pour voir la carcasse du lion. Et voici, il y avait dans la carcasse du lion un essaim d'abeilles et du miel.
\VS{9}Il en prit entre ses mains, et s'en alla en mangeant ; et lorsqu'il fut arrivé vers son père et sa mère, il leur en donna, et ils en mangèrent. Mais il ne leur déclara pas qu'il avait pris ce miel dans la carcasse du lion.
\VS{10}Son père descendit chez la femme. Samson fit là un festin ; car c'est ainsi que les jeunes gens faisaient.
\VS{11}Dès qu'on le vit, on prit trente compagnons qui furent avec lui.
\VS{12}Samson leur dit : Je vous propose une énigme. Si vous me l'expliquez au cours des sept jours du festin, et si vous la trouvez, je vous donnerai trente chemises et trente vêtements de rechange.
\VS{13}Mais si vous ne pouvez pas me l'expliquer, vous me donnerez trente chemises et trente vêtements de rechange. Ils lui répondirent : Propose ton énigme, et nous l'écouterons.
\VS{14}Et il leur dit : De celui qui mange est sorti ce qui se mange, et du fort est sorti le doux. Pendant trois jours, ils ne purent pas expliquer l'énigme.
\VS{15}Et au septième jour, ils dirent à la femme de Samson : Persuade ton mari de nous expliquer l'énigme ; de peur que nous ne te brûlions au feu, toi et la maison de ton père. C'est pour nous déposséder que vous nous avez appelés ici, n'est-ce pas ?
\VS{16}La femme de Samson pleurait auprès de lui, et disait : Certainement tu me hais, et tu ne m'aimes pas ; tu as proposé une énigme aux enfants de mon peuple, et tu ne me l'as pas expliquée ! Et il lui répondait : Je ne l'ai expliquée ni à mon père ni à ma mère ; est-ce à toi que je l'expliquerais ?
\VS{17}Elle pleura ainsi auprès de lui durant les sept jours du festin ; mais au septième jour, il la lui expliqua, parce qu'elle le tourmentait. Puis elle l'expliqua aux enfants de son peuple.
\VS{18}Les gens de la ville lui dirent au septième jour, avant le coucher du soleil : Qu'y a-t-il de plus doux que le miel, et qu'y a-t-il de plus fort que le lion ? Et il leur dit : Si vous n'aviez pas labouré avec ma génisse vous n'auriez pas trouvé mon énigme.
\VS{19}L'Esprit de Yahweh le saisit, et il descendit à Askalon. Il tua trente hommes, il prit leurs dépouilles, et donna les vêtements de rechange à ceux qui avaient expliqué l'énigme. Sa colère s'enflamma, et il monta à la maison de son père.
\VS{20}Et la femme de Samson fut donnée à son compagnon, avec lequel il était lié.
\Chap{15}
\TextTitle{Samson utilisé pour le jugement des Philistins}
\VerseOne{}Et il arriva quelque jours après, au jour de la moisson des blés, que Samson alla visiter sa femme, et lui porta un chevreau. Il dit : J'entrerai vers ma femme dans sa chambre. Mais le père de sa femme ne lui permit pas d'y entrer.
\VS{2}Car il lui dit : J'ai cru que tu avais de la haine pour elle, c'est pourquoi je l'ai donnée à ton compagnon. Sa jeune sœur n'est-elle pas plus belle qu'elle ? Prends-la donc à sa place.
\VS{3}Samson leur dit : Cette fois je serai innocent à l'égard des Philistins si je leur fais du mal.
\VS{4}Samson s'en alla donc. Il prit trois cents renards, il prit aussi des torches ; puis il tourna les renards queue contre queue, et mit une torche entre les deux queues, au milieu.
\VS{5}Puis il mit le feu aux torches, et lâcha les renards dans les blés des Philistins, et brûla le tas de gerbes, le blé sur pied, jusqu'aux plantations d'oliviers.
\VS{6}Les Philistins dirent : Qui a fait cela ? On répondit : Samson, le gendre du Thimnien, parce qu'il lui a pris sa femme et l'a donnée à son compagnon. Les Philistins montèrent, et ils la brûlèrent au feu, elle avec son père.
\VS{7}Alors Samson leur dit : Est-ce donc ainsi que vous faites ? Je ne cesserai qu'après m'être vengé de vous.
\VS{8}Il les battit par une grande défaite, dos et ventre ; puis il descendit, et demeura dans une caverne du rocher d'Etam.
\VS{9}Alors les Philistins montèrent, campèrent en Juda, et s'étendirent jusqu'à Léchi.
\VS{10}Les hommes de Juda dirent : Pourquoi êtes-vous montés contre nous ? Ils répondirent : Nous sommes montés pour lier Samson, afin que nous lui fassions comme il nous a fait.
\VS{11}Alors trois mille hommes de Juda descendirent à la caverne du rocher d'Etam, et dirent à Samson : Ne sais-tu pas que les Philistins dominent sur nous ? Que nous as-tu donc fait ? Il leur répondit : Je leur ai fait comme ils m'ont fait.
\VS{12}Ils lui dirent : Nous sommes descendus pour te lier, afin de te livrer entre les mains des Philistins. Samson leur dit : Jurez-moi que vous ne me tuerez pas.
\VS{13}Ils lui répondirent, en disant : Non ; mais nous te lierons, afin de te livrer entre leurs mains, mais nous ne te tuerons pas. Ils le lièrent avec deux cordes neuves, et le firent monter hors du rocher.
\VS{14}Lorsqu'il entra à Léchi, les Philistins poussèrent des cris de joie à sa rencontre. Alors l'Esprit de Yahweh le saisit. Les cordes qui étaient sur ses bras devinrent comme du lin brûlé par le feu, et les liens tombèrent de ses mains.
\VS{15}Il trouva une mâchoire d'âne fraîche, il étendit sa main, la prit, et il en tua mille hommes.
\VS{16}Puis Samson dit : Avec une mâchoire d'âne, un monceau, deux monceaux ; avec une mâchoire d'âne, j'ai tué mille hommes.
\VS{17}Quand il cessa de parler, il jeta de sa main la mâchoire. On appela ce lieu Ramath-Léchi.
\VS{18}Il eut extrêmement soif, et invoqua Yahweh en disant : Tu as accordé par la main de ton serviteur cette grande délivrance ; et maintenant mourrais-je de soif, et tomberais-je entre les mains des incirconcis\FTNT{1 S. 17:26.} ?
\VS{19}Alors Dieu fendit la cavité du rocher qui est à Léchi, et il en sortit de l'eau. Samson but, l'Esprit lui revint, et il reprit vie. C'est pourquoi on a appelé cette source du nom d'En-Hakkoré ; elle existe encore aujourd'hui  à Léchi.
\VS{20}Samson fut juge en Israël, au temps des Philistins, pendant vingt ans\FTNT{Jg. 16:31.}.
\Chap{16}
\TextTitle{Faiblesse de Samson}
\VerseOne{}Or Samson s'en alla à Gaza ; il y vit une femme prostituée, et il entra chez elle.
\VS{2}On dit aux gens de Gaza : Samson est venu ici. Ils l'entourèrent, et se tinrent en embuscade toute la nuit à la porte de la ville. Ils restèrent tranquilles toute la nuit, en disant : Au point du jour, nous le tuerons.
\VS{3}Samson demeura couché jusqu'à minuit. Au milieu de la nuit, il se leva ; et il saisit les battants des portes de la ville et les deux poteaux, les retira avec la barre, les mit sur ses épaules, et les porta sur le sommet de la montagne qui est en face d'Hébron.
\VS{4}Après cela, il aima une femme dans la vallée de Sorek. Elle se nommait Delila.
\VS{5}Les princes des Philistins montèrent vers elle, et lui dirent : Séduis-le, jusqu'à ce que tu saches de lui en quoi consiste sa grande force, et comment pourrions-nous le vaincre ; afin que nous le lions pour l'abattre, et nous te donnerons chacun mille cent sicles d'argent.
\VS{6}Delila dit à Samson : Dis-moi, je te prie, en quoi consiste ta grande force, et avec quoi il faudrait te lier pour t'abattre.
\VS{7}Samson lui répondit : Si on me liait avec sept cordes fraîches, qui ne soient pas encore sèches, je deviendrais faible et je serais comme un autre homme.
\VS{8}Les princes des Philistins emmenèrent à Delila sept cordes fraîches, qui n'étaient pas encore sèches. Et elle le lia.
\VS{9}Or il y avait chez elle, dans une chambre, des gens qui se tenaient en embuscade. Elle lui dit : Les Philistins sont sur toi, Samson ! Alors il rompit les cordes comme se romprait un cordon d'étoupe dès qu'il sent le feu. Et l'on ne connut pas d'où lui venait sa force.
\VS{10}Puis Delila dit à Samson : Voici, tu t'es moqué de moi, car tu m'as dit des mensonges. Je te prie, déclare-moi maintenant avec quoi il faut te lier.
\VS{11}Il lui répondit : Si on me liait avec des cordes neuves, dont on ne se serait jamais servi pour un quelconque ouvrage, je deviendrais faible, et je serais comme un autre homme.
\VS{12}Delila prit des cordes neuves avec lesquelles elle le lia. Puis elle lui dit : Les Philistins sont sur toi, Samson ! Or il y avait des gens en embuscade dans une chambre. Et il rompit les cordes comme un fil.
\VS{13}Puis Delila dit à Samson : Tu t'es moqué de moi, jusqu'ici tu m'as dit des mensonges. Déclare-moi avec quoi il faut te lier. Il lui dit : Tu n'as qu'à tresser les sept tresses de ma tête avec la chaîne du tissu.
\VS{14}Et elle les fixa par la cheville. Puis elle dit : Les Philistins sont sur toi, Samson ! Alors il se réveilla de son sommeil, et il retira la chaîne du tissu.
\TextTitle{Samson révèle son secret}
\VS{15}Alors elle lui dit : Comment peux-tu dire : Je t'aime ! Puisque ton cœur n'est pas avec moi ? Tu t'es moqué de moi par trois fois, et tu ne m'as pas déclaré en quoi consiste ta grande force.
\VS{16}Comme elle le tourmentait et l'importunait tous les jours par ses paroles, son âme en fut affligée jusqu'à la mort,
\VS{17}alors il lui ouvrit tout son cœur, et lui dit : Le rasoir n'est jamais passé sur ma tête, car je suis Naziréen de Dieu dès le ventre de ma mère. Si j'étais rasé, ma force partirait, je me trouverais faible, et je serais comme tous les autres hommes.
\VS{18}Delila, voyant qu'il lui avait ouvert tout son cœur, envoya appeler les princes des Philistins, et leur fit dire : Montez cette fois, car il m'a ouvert tout son cœur. Les princes des Philistins montèrent vers elle, et emmenèrent l'argent dans leurs mains.
\VS{19}Elle l'endormit sur ses genoux. Et ayant appelé un homme, elle rasa les sept tresses de la tête de Samson, et commença à le dompter. Sa force partit.
\VS{20}Alors elle dit : Les Philistins sont sur toi, Samson ! Et il se réveilla de son sommeil, et dit : Je m'en sortirai comme les autres fois, et je me dégagerai. Mais il ne savait pas que Yahweh s'était retiré de lui\FTNT{L’immoralité sexuelle de Samson et sa désobéissance à Yahweh, dues à son manque de caractère, ont ruiné à jamais son ministère et compromis l’avenir du peuple d’Israël qu’il devait diriger (Jg. 16). Cet homme avait reçu un appel puissant dès le sein de sa mère, mais il ne vivait pas dans la crainte de Dieu. Le manque de discernement de Samson lui coûta ainsi toutes les grâces que le Seigneur lui avait accordées : La sainteté symbolisée par ses sept tresses, la force ou l’onction,  la vision, la liberté (Jg. 16:21).}.
\VS{21}Les Philistins donc le saisirent, et lui crevèrent les yeux ; ils le descendirent à Gaza, et le lièrent de deux chaînes d'airain. Il tournait la meule dans la prison\FTNT{2 S. 3:34.}.
\VS{22}Les cheveux de sa tête commencèrent à repousser, depuis qu'il avait été rasé.
\TextTitle{Samson achève le jugement des Philistins}
\VS{23}Or les princes des Philistins s'assemblèrent pour offrir un grand sacrifice à Dagon, leur dieu, et pour se réjouir. Ils disaient : Notre dieu a livré en nos mains Samson, notre ennemi.
\VS{24}Et quand le peuple le vit, il loua son dieu, en disant : Notre dieu a livré entre nos mains notre ennemi, celui qui ravageait notre pays, et qui multipliait nos morts.
\VS{25}Comme ils avaient le cœur joyeux, ils dirent : Qu'on appelle Samson, afin qu'il nous fasse rire ! Ils appelèrent Samson et le tirèrent de la prison ; et il joua devant eux. Ils le firent tenir entre les colonnes.
\VS{26}Alors Samson dit au garçon qui le tenait par la main : Laisse-moi afin que je puisse toucher les colonnes sur lesquelles repose la maison pour que je m'y appuie.
\VS{27}Or la maison était remplie d'hommes et de femmes ; tous les princes des Philistins y étaient, et il y avait même sur le toit près de trois mille personnes, hommes et femmes, qui regardaient Samson jouer.
\VS{28}Alors Samson invoqua Yahweh, et dit : Seigneur Yahweh ! Je te prie, souviens-toi de moi ; ô Dieu ! Fortifie-moi seulement cette fois, et que par un coup je me venge des Philistins pour mes deux yeux\FTNT{Hé. 11:32.} !
\VS{29}Samson embrassa les deux colonnes du milieu sur lesquelles reposait la maison, et il s'appuya contre elles ; l'une à sa droite, et l'autre à sa gauche.
\VS{30}Et il dit : Que mon âme meure avec les Philistins ! Il se pencha donc de toute sa force, et la maison tomba sur les princes et sur tout le peuple qui y était. Et il fit mourir beaucoup plus de gens à sa mort, qu'il n'en avait fait mourir pendant sa vie.
\VS{31}Ensuite ses frères et toute la maison de son père descendirent, et le transportèrent. Lorsqu'ils furent montés, ils l'enterrèrent entre Tsorea et Eschthaol dans le sépulcre de Manoach, son père. Il avait été juge en Israël pendant vingt ans\FTNT{Jg. 13:2.}.
\Chap{17}
\TextTitle{Confusion en Israël}
\VerseOne{}Il y avait un homme de la montagne d'Ephraïm, nommé Mica.
\VS{2}Il dit à sa mère : Les mille cent sicles d'argent qu'on t'a pris, et pour lesquels tu as fait des imprécations même à mes oreilles, voici, j'ai cet argent, c'est moi qui l'avais pris. Alors sa mère dit : Béni soit mon fils par Yahweh !
\VS{3}Et il rendit à sa mère les mille cent sicles d'argent ; sa mère dit : Je consacre de ma main cet argent à Yahweh, afin d'en faire pour mon fils une image taillée, et une image en métal fondu ; et c'est ainsi que je te le rendrai.
\VS{4}Et il rendit l'argent à sa mère. Elle prit deux cents sicles d'argent et les donna au fondeur, qui en fit une image taillée, et une image en métal fondu.  On les plaça dans la maison de Mica.
\VS{5}Ainsi cet homme, savoir Mica, avait une maison de Dieu ; il fit un éphod et des téraphim, et il consacra par sa main l'un de ses fils, qui lui servit de sacrificateur.
\VS{6}En ce temps-là, il n'y avait pas de roi en Israël. Chacun faisait ce qui lui semblait être droit à ses yeux\FTNT{Jg. 18:1}.
\VS{7}Or il y avait un jeune homme de Bethléhem de Juda, de la famille de la tribu de Juda ; il était Lévite, et il séjournait là.
\VS{8}Cet homme partit de la ville de Bethléhem de Juda, pour trouver une demeure qui lui convienne.  En chemin, il entra dans la montagne d'Ephraïm jusqu'à la maison de Mica.
\VS{9}Mica lui dit : D'où viens-tu ? Il lui répondit : Je suis Lévite, de Bethléhem de Juda, et je voyage pour trouver une demeure qui me convienne.
\VS{10}Mica lui dit : Demeure avec moi ; tu me serviras de père et de sacrificateur, et je te donnerai dix sicles d'argent par an, les vêtements d'ordre dont tu auras besoin, et ton entretien. Et le Lévite vint\FTNT{Jg. 18:19.}.
\VS{11}Ainsi le Lévite convint de demeurer avec cet homme, qui regarda le jeune homme comme l'un de ses fils.
\VS{12}Mica consacra\FTNT{"Consacrer" signifie litteralement "remplir la main".} le Lévite, qui lui servit de sacrificateur, et qui demeura dans sa maison.
\VS{13}Mica dit : Maintenant je sais que Yahweh me fera du bien, parce que j'ai un Lévite pour sacrificateur.
\Chap{18}
\TextTitle{Dan recherche un territoire}
\VerseOne{}En ce temps-là, il n'y avait pas de roi en Israël ; et en ce même temps la tribu des Danites cherchait un héritage afin de pouvoir s'établir, car jusqu'à ce jour il ne lui était pas échu d'héritage au milieu des tribus d'Israël\FTNT{Jg. 17:6.}.
\VS{2}C'est pourquoi les fils de Dan envoyèrent de leur famille cinq hommes vaillants, de Tsorea et d'Eschthaol, pour explorer le pays et l'examiner. Ils leur dirent : Allez examiner le pays. Ils entrèrent dans la montagne d'Ephraïm jusqu'à la maison de Mica, et ils y passèrent la nuit.
\VS{3}Comme ils étaient près de la maison de Mica, ils reconnurent la voix du jeune homme Lévite et lui dirent : Qui t'a amené ici ? Qu'y fais-tu ? Que fais-tu ici ?
\VS{4}Il leur répondit : Mica fait pour moi telle et telle chose, il me donne un salaire, et je lui sers de sacrificateur.
\VS{5}Ils lui dirent : Nous te prions consulte Dieu, afin que nous sachions si le voyage que nous entreprenons prospérera.
\VS{6}Et le sacrificateur leur répondit : Allez en paix ; Yahweh a sous ses yeux le voyage que vous mènerez.
\VS{7}Ces cinq hommes s'en allèrent, et entrèrent à Laïs. Ils virent le peuple qui y habitait en sécurité selon les coutumes des Sidoniens, tranquille et en confiance ; il n'y avait personne au pays qui les humiliait en quelque chose en dominant sur eux ; ils étaient éloignés des Sidoniens, et ils n'avaient aucune affaire avec d'autres hommes.
\VS{8}Puis ils vinrent auprès de leurs frères à Tsorea et à Eschthaol, et leurs frères leur dirent : Quelle nouvelle rapportez-vous ?
\VS{9}Et ils répondirent : Allons ! Montons contre eux ; car nous avons vu le pays, et nous l'avons trouvé très bon. Quoi ! Vous restez sans rien faire ? Ne soyez pas paresseux pour aller posséder ce pays.
\VS{10}Quand vous y entrerez, vous irez vers un peuple en sécurité. Le pays est vaste, Dieu l'a livré entre vos mains ; c'est un lieu où il ne manque rien de tout ce qui est sur la terre.
\VS{11}Il partit de Tsorea et d'Eschthaol, six cents hommes de la famille de Dan, munis de leurs armes de guerre.
\VS{12}Ils montèrent, et campèrent à Kirjath-Jearim en Juda ; c'est pourquoi on a appelé ce lieu qui est derrière Kirjath-Jearim jusqu'à ce jour, Machané-Dan.
\VS{13}Puis ils passèrent par la montagne d'Ephraïm, et ils entrèrent dans la maison de Mica.
\TextTitle{Campagnes de la tribu de Dan}
\VS{14}Alors les cinq hommes qui étaient allés explorer le pays de Laïs prirent la parole et dirent à leurs frères : Savez-vous qu'il y a dans ces maisons-là un éphod, des théraphim, une image taillée et une image en métal fondu ? Voyez maintenant ce que vous avez à faire.
\VS{15}Alors ils se détournèrent de ce lieu, et entrèrent dans la maison où était le jeune homme Lévite, dans la maison de Mica, et lui demandèrent comment il se portait.
\VS{16}Et les six cents hommes d'entre les fils de Dan, qui étaient munis de leurs armes de guerre, se tenaient à l'entrée de la porte.
\VS{17}Mais les cinq hommes, qui étaient allés explorer le pays, montèrent et entrèrent dans la maison ; ils prirent l'image taillée, l'éphod, les théraphim, et l'image en métal fondu, pendant que le sacrificateur était à l'entrée de la porte avec les six cents hommes munis de leurs armes de guerre.
\VS{18}Etant entrés dans la maison de Mica, ils prirent l'image taillée, l'éphod, les théraphim, et l'image en métal fondu. Le sacrificateur leur dit : Que faites-vous ?
\VS{19}Ils lui répondirent : Tais-toi, mets ta main sur ta bouche, et viens avec nous ; sois pour nous un père et un sacrificateur. Vaut-il mieux que tu serves de sacrificateur à la maison d'un homme seul, ou que tu serves de sacrificateur à une tribu et à une famille en Israël\FTNT{Jg. 17:10.} ?
\VS{20}Le sacrificateur eut de la joie dans son cœur ; il prit l'éphod, les théraphim, et l'image taillée, et vint au milieu du peuple.
\VS{21}Après quoi ils se retournèrent et marchèrent, en mettant devant eux les petits enfants, le bétail, et les bagages.
\VS{22}Comme ils étaient loin de la maison de Mica, les gens qui  habitaient les maisons voisines de celle de Mica furent assemblés à grand cri ; et poursuivirent les fils de Dan.
\VS{23}Et ils crièrent aux fils de Dan, qui se tournèrent de face et dirent à Mica : Qu’as-tu, que tu te sois ainsi écrié pour rassembler ces gens ?
\VS{24}Il répondit : Vous avez enlevé mes dieux que j'avais faits, vous avez pris le sacrificateur, et vous vous en êtes allés : Que me reste-t-il ? Comment pouvez-vous me dire : Qu'as-tu\FTNT{Ge. 31:30.} ?
\VS{25}Les fils de Dan lui dirent : Ne fais pas entendre ta voix après nous, de peur que des hommes exaspérés ne se jettent sur vous, et que vous n’y laissiez la vie, toi, et tous ceux de ta famille.
\VS{26}Les fils de Dan firent leur chemin. Mica, voyant qu'ils étaient plus forts que lui, s'en retourna et revint dans sa maison.
\VS{27}Ainsi ils prirent les choses que Mica avait faites, et le sacrificateur qu'il avait, et ils entrèrent à Laïs, vers un peuple tranquille et en sécurité ; ils les firent passer au fil de l'épée, et ils brûlèrent la ville.
\VS{28}Et il n'y eut personne qui la délivrât, car elle était éloignée de Sidon, et ses habitants n'avaient pas d'affaires avec les autres hommes : Elle était située dans la vallée qui appartenait au pays de Beth-Rehob. Les fils de Dan rebâtirent la ville, et y demeurèrent.
\VS{29}Ils appelèrent la ville Dan, selon le nom de Dan, leur père qui était né à Israël ; mais la ville s'appelait auparavant Laïs\FTNT{Jos. 19:47.}.
\VS{30}Et les fils de Dan dressèrent l'image taillée ; et Jonathan, fils de Guerschom, fils de Manassé, lui et ses fils, furent sacrificateurs pour la tribu des Danites, jusqu'au jour de la captivité du pays.
\VS{31}Ils y dressèrent donc l'image taillée que Mica avait faite, pendant tout le temps que la maison de Dieu fut à Silo.
\Chap{19}
\TextTitle{Dégradation morale}
\VerseOne{}Il arriva aussi en ce temps-là, où il n'y avait pas de roi en Israël, qu'un Lévite qui habitait aux côtés de la montagne d'Ephraïm, prit pour concubine une femme de Bethléhem de Juda\FTNT{Jg. 17:6 ; 21:25.}.
\VS{2}Mais sa concubine se prostitua chez lui, et elle s'en alla pour aller dans la maison de son père à Bethléhem de Juda, où elle resta pendant quatre mois.
\VS{3}Puis son mari se leva et alla après elle, pour parler à son cœur, et la ramener. Il avait avec lui son serviteur et deux ânes. Elle le fit entrer dans la maison de son père ; et quand le père de la jeune femme le vit, il s'approcha avec joie.
\VS{4}Son beau-père, le père de la jeune femme, le retint avec grande instance, de sorte qu'il demeura trois jours avec lui. Ils mangèrent et burent, et logèrent là.
\VS{5}Le quatrième jour, ils se levèrent de bon matin, et le Lévite se levait pour s'en aller. Mais le père de la jeune femme dit à son gendre : Fortifie ton cœur avec un morceau de pain, et vous partirez ensuite.
\VS{6}Ils s'assirent, et ils mangèrent et burent eux deux ensemble. Puis le père de la jeune femme dit au mari : Je te prie consens à passer encore ici cette nuit, et que ton cœur se réjouisse.
\VS{7}Le mari se levait pour s'en aller ; mais son beau-père le pressa tellement, qu'il s'en retourna, et y passa encore la nuit.
\VS{8}Le cinquième jour, il se leva de bon matin pour s'en aller. Alors le père de la jeune femme dit : Fortifie ton cœur ; et attendez le déclin du jour. Et ils mangèrent eux deux.
\VS{9}Puis le mari se levait pour s'en aller, avec sa concubine et son serviteur ; mais son beau-père, le père de la jeune femme, lui dit : Voici, maintenant le jour baisse, il se fait tard, je vous prie passez ici la nuit ; voici le jour est sur son déclin, passe ici la nuit, et que ton cœur se réjouisse ; demain matin vous vous mettrez en route, et tu t'en iras à ta tente.
\VS{10}Mais le mari ne voulut pas y passer la nuit, il se leva, et s'en alla.  Il vint jusque vis-à-vis de Jébus, qui est Jérusalem, avec les deux ânes bâtés et sa concubine.
\VS{11}Comme ils étaient près de Jébus, le jour avait beaucoup baissé. Le serviteur dit à son maître : Allons, détournons-nous vers cette ville des Jébusiens, afin que nous y passions la nuit.
\VS{12}Son maître lui répondit : Nous ne nous détournerons pas vers une ville d'étrangers, où il n'y a pas d'enfants d'Israël, mais nous passerons par Guibea.
\VS{13}Il dit aussi à son serviteur : Allons, approchons-nous de l'un de ces lieux, Guibea ou Rama, et passons-y la nuit.
\VS{14}Ils continuèrent à marcher, et le soleil se coucha quand ils furent près de Guibea, qui appartient à Benjamin.
\VS{15}Alors ils se détournèrent vers Guibea, et y entrèrent pour passer la nuit. Le Lévite entra, et il s'assit sur la place de la ville. Il n'y eut aucun homme qui les reçut dans sa maison afin qu'ils y passent la nuit.
\VS{16}Et voici, sur le soir, un vieil homme venait de travailler dans les champs ; cet homme était de la montagne d'Ephraïm, il séjournait à Guibea, et les gens du lieu étaient Benjamites.
\VS{17}Et levant ses yeux, il vit le voyageur sur la place de la ville. Le vieil homme lui dit : Où vas-tu, et d'où viens-tu ?
\VS{18}Il lui répondit : Nous passons de Bethléhem de Juda vers les côtés de la montagne d'Ephraïm, d'où je suis. J'étais allé jusqu'à Bethléhem de Juda, mais maintenant je m'en vais à la maison de Yahweh. Mais il n'y a aucun homme qui me reçoive dans sa maison.
\VS{19}Nous avons pourtant de la paille et du fourrage pour nos ânes ; du pain et du vin pour moi,  pour ta servante, et pour le garçon qui est avec tes serviteurs. Nous n'avons besoin d'aucune chose.
\VS{20}Le vieil homme dit : Pourvu que la paix soit ! Quoi qu'il en soit, je me charge de tous tes besoins, je te prie seulement de ne pas passer la nuit sur la place.
\VS{21}Alors il les fit entrer dans sa maison, et il donna du fourrage aux ânes. Les voyageurs se lavèrent les pieds ; puis ils mangèrent et burent\FTNT{Ge. 43:24.}.
\VS{22}Comme ils se réjouissaient, voici, les hommes de la ville, fils d'hommes pervers, environnèrent la maison, frappèrent à la porte, et dirent au vieil homme, maître de la maison : Fais sortir l'homme qui est entré dans ta maison, afin que nous le connaissions\FTNT{Jg. 20:13 ; Os. 9:9 ; 10:9 ; Ge. 19:4.}.
\VS{23}Mais cet homme, savoir le maître de la maison, sortit vers eux, et leur dit : Non, mes frères, ne lui faites pas de mal, je vous prie ; puisque cet homme est entré dans ma maison, ne faites pas une telle infamie.
\VS{24}Voici, j'ai une fille vierge, et cet homme a une concubine ; je vous les amènerai dehors ; vous les déshonorerez, et vous ferez d'elles comme il semblera bon à vos yeux. Mais ne faites pas cette action infâme à l'égard de cet homme.
\VS{25}Mais ces gens ne voulurent pas l'écouter. C'est pourquoi l'homme saisit sa concubine, et la leur amena dehors. Ils la connurent, et abusèrent d'elle toute la nuit jusqu'au matin ; puis ils la renvoyèrent au lever de l'aurore.
\VS{26}Vers le matin, cette femme alla tomber à la porte de la maison de l'homme où était son mari, et elle y demeura jusqu'au jour.
\VS{27}Et le matin, son mari se leva, et ayant ouvert la porte de la maison, il sortit pour poursuivre son chemin. Mais voici, la femme concubine était tombée à la porte de la maison, et avait les mains sur le seuil.
\VS{28}Il lui dit : Lève-toi, et allons-nous-en. Mais elle ne répondit pas. Alors il l'emmena sur un âne, se mit en chemin, et s'en alla dans sa demeure.
\VS{29}En entrant en sa maison, il prit un couteau, et saisissant sa concubine, il la coupa avec ses os en douze morceaux, qu'il envoya dans tout le territoire d'Israël.
\VS{30}Et il arriva que tous ceux qui virent cela dirent : Une telle chose n'a été faite ni vue depuis le jour où les enfants d'Israël sont montés hors du pays d'Egypte, jusqu'à ce jour ; prenez la chose à cœur, consultez-vous, et parlez !
\Chap{20}
\TextTitle{Israël devant Yahweh à Mitspa}
\VerseOne{}Alors tous les fils d'Israël sortirent, et toute l'assemblée se réunit comme un seul homme, depuis Dan jusqu'à Beer-Schéba et jusqu'au pays de Galaad, devant Yahweh, à Mitspa.
\VS{2}Les chefs de tout le peuple, toutes les tribus d'Israël, se présentèrent à l'assemblée du peuple de Dieu, au nombre de quatre cent mille hommes de pied, tirant l'épée.
\VS{3}Les fils de Benjamin entendirent que les fils d'Israël étaient montés à Mitspa. Les fils d'Israël dirent : Parlez, comment ce mal est arrivé ?
\VS{4}Alors le Lévite, mari de la femme tuée, répondit, et dit : J'étais venu à Guibea de Benjamin, avec ma concubine, pour y passer la nuit.
\VS{5}Les seigneurs de Guibea se sont élevés contre moi, et ont encerclé de nuit la maison où j'étais. Ils avaient l'intention de me tuer, et ils ont tellement violé ma concubine qu'elle en est morte.
\VS{6}C'est pourquoi j'ai saisi ma concubine, je l'ai coupée en morceaux, et je les ai envoyés dans tout le territoire de l'héritage d'Israël ; car ils ont fait un crime et une infamie en Israël.
\VS{7}Vous voici tous, fils d'Israël ; consultez-vous sur la question, et prenez ici une décision !
\VS{8}Tout le peuple se leva comme un seul homme, et ils dirent : Aucun homme n'ira dans sa tente, et aucun homme ne se retirera dans sa maison.
\VS{9}Et maintenant voici ce que nous ferons à Guibea : Nous marcherons contre elle d'après le sort.
\VS{10}Nous prendrons dans toutes les tribus d'Israël dix hommes sur cent, cent sur mille, et mille sur dix mille ; nous prendrons des provisions pour le peuple, afin qu'en entrant à Guibea de Benjamin, on leur fasse selon toute l'infamie qu'elle a commise en Israël.
\VS{11}Ainsi tous les hommes d'Israël s'assemblèrent contre la ville, unis comme un seul homme.
\VS{12}Alors les tribus d'Israël envoyèrent des hommes vers la maison de Benjamin, pour dire : Quelle méchanceté a été faite parmi vous ?
\VS{13}Maintenant donc livrez-nous les fils des hommes pervers qui sont à Guibea, afin que nous les fassions mourir et que nous ôtions le mal du milieu d'Israël. Mais les fils de Benjamin ne voulurent pas écouter la voix de leurs frères, les enfants d'Israël.
\VS{14}Et les fils de Benjamin s'assemblèrent à Guibea pour sortir en guerre contre les fils d'Israël.
\VS{15}En ce jour-là, on fit le dénombrement des fils de Benjamin qui étaient dans ces villes, et il se trouva vingt-six mille hommes, tirant l'épée, sans compter les habitants de Guibea formant sept cents hommes d'élite.
\VS{16}De tout ce peuple, il y avait sept cents hommes d'élite qui ne se servaient pas de la main droite ; tous tirant la pierre avec la fronde,  à un cheveu près,  ils n'y manquaient pas.
\VS{17}On fit aussi le dénombrement des hommes d'Israël, excepté ceux de Benjamin, et l'on en trouva quatre cent mille hommes tirant l'épée, tous gens de guerre.
\TextTitle{Coalition pour monter contre Benjamin}
\VS{18}Et les fils d'Israël se levèrent, montèrent vers Dieu à Béthel pour le consulter, en disant : Qui d'entre nous montera le premier pour faire la guerre aux fils de Benjamin ? Yahweh répondit : Juda montera le premier.
\VS{19}Puis les fils d'Israël se levèrent de bon matin, et campèrent près de Guibea.
\VS{20}Et les hommes d'Israël sortirent pour combattre ceux de Benjamin, et se rangèrent en bataille près de Guibea.
\VS{21}Les fils de Benjamin sortirent de Guibea, et ils tuèrent ce jour-là vingt-deux mille hommes d'Israël.
\VS{22}Toutefois le peuple, les hommes d'Israël, se fortifièrent et se rangèrent de nouveau en bataille au lieu où ils s'étaient rangés le premier jour.
\VS{23}Et les fils d'Israël montèrent, et ils pleurèrent devant Yahweh jusqu'au soir ; ils consultèrent Yahweh en disant : M'approcherai-je encore pour combattre contre les fils de Benjamin, mon frère ? Yahweh dit : Montez contre lui.
\VS{24}Le second jour, les fils d'Israël s'approchèrent des fils de Benjamin.
\VS{25}Ce même jour, les Benjamites sortirent de Guibea à leur rencontre, et ils tuèrent encore dix-huit mille hommes des fils d'Israël, tous tirant l'épée.
\VS{26}Alors tous les fils d'Israël et tout le peuple montèrent et vinrent vers Dieu à Béthel ; ils pleurèrent, et restèrent là devant Yahweh. Ce jour-là ils jeûnèrent jusqu'au soir, et ils offrirent des holocaustes, et des sacrifices de paix devant Yahweh.
\VS{27}Ensuite les fils d'Israël consultèrent Yahweh, c'était là que se trouvait l'arche de l'alliance de Dieu ;
\VS{28}et Phinées, fils d'Eléazar, fils d'Aaron, se tenait devant Yahweh en ce temps-là en disant : Sortirai-je encore en guerre contre les fils de Benjamin, mon frère, ou dois-je m'en abstenir ? Yahweh répondit : Montez, car demain je les livrerai entre vos mains.
\VS{29}Alors Israël mit une embuscade autour de Guibea.
\VS{30}Le troisième jour, les fils d'Israël montèrent contre les fils de Benjamin, et ils se rangèrent en bataille contre Guibea, comme les autres fois.
\VS{31}Alors les fils de Benjamin sortirent à la rencontre du peuple, et ils furent attirés hors de la ville. Ils commencèrent à frapper à mort quelques-uns du peuple comme les autres fois, environ trente hommes d'Israël, sur les routes dont l'une monte à Béthel et l'autre à Guibea, par les champs.
\VS{32}Les fils de Benjamin disaient : Ils tombent battus devant nous, comme la première fois ! Mais les fils d'Israël disaient : Fuyons, et attirons-les hors de la ville dans les chemins.
\VS{33}Tous les hommes d'Israël se levant de leur lieu, se rangèrent à Baal-Thamar ; et l'embuscade sortit du lieu où ils étaient, de Maaré-Guibea.
\VS{34}Dix mille hommes choisis sur tout Israël vinrent contre Guibea. La bataille fut rude, et les Benjamites ne surent pas que le mal les atteindrait.
\VS{35}Yahweh battit Benjamin devant Israël, et les fils d'Israël tuèrent ce jour-là vingt-cinq mille cent hommes de Benjamin, tous tirant l'épée.
\VS{36}Les fils de Benjamin regardaient comme battus les hommes d'Israël, qui cédaient du terrain à Benjamin et se reposaient sur l'embuscade qu'ils avaient mise près de Guibea.
\VS{37}Ceux qui étaient en embuscade se jetèrent promptement sur Guibea, ils se portèrent en avant et frappèrent toute la ville au tranchant de l'épée.
\VS{38}Et le signal convenu entre les hommes d’Israël et l’embuscade était qu’ils fassent monter beaucoup de fumée de la ville.
\VS{39}Les hommes d’Israël avaient donc tourné le dos dans la bataille, et les Benjamites avaient commencé de frapper et de blesser à mort environ trente hommes de ceux d’Israël ; et ils disaient : Certainement ils tombent devant nous comme à la première bataille !
\VS{40}Mais quand l'épaisse colonne de fumée commençait à monter de la ville, les Benjamites se tournèrent ; et voici, derrière eux toute la ville disparaissait montant en feu vers le ciel.
\VS{41}Les hommes d'Israël tournèrent le visage ; et ceux de Benjamin furent épouvantés en voyant le mal qui allait les  atteindre.
\VS{42}Ils tournèrent le dos devant les hommes d'Israël par le chemin du désert. Mais les assaillants s'attachaient à leurs pas, et ils détruisirent ceux qui étaient sortis des villes.
\VS{43}Ils environnèrent Benjamin, le poursuivirent, l'écrasèrent dès qu'il voulut se reposer jusqu'en face de Guibea, du côté du soleil levant.
\VS{44}Il tomba dix-huit mille hommes de Benjamin, tous des vaillants hommes.
\VS{45}Et parmi ceux de Benjamin qui tournèrent le dos pour s'enfuir vers le désert au rocher de Rimmon, les hommes d'Israël en firent périr cinq mille hommes sur les routes ; et les poursuivant de près jusqu'à Guideom, ils frappèrent deux mille hommes.
\TextTitle{La tribu de Benjamin décimée}
\VS{46}En ce jour-là, le nombre de Benjamites qui tombèrent fut de vingt-cinq mille hommes tirant l'épée, et tous étaient des vaillants hommes.
\VS{47}Et il y eut six cents hommes de ceux qui avaient tourné le dos, qui s'échappèrent vers le désert au rocher de Rimmon, et qui demeurèrent au rocher de Rimmon pendant quatre mois.
\VS{48}Les hommes d'Israël retournèrent vers les fils de Benjamin, et ils les frappèrent du tranchant de l'épée, depuis les hommes des villes jusqu'aux bêtes, et tout ce qui s'y trouva. Ils brûlèrent toutes les villes qu'ils trouvaient.
\Chap{21}
\TextTitle{Deuil national}
\VerseOne{}Les hommes d'Israël avaient juré à Mitspa, en disant : Aucun homme ne donnera sa fille pour femme à un Benjamite.
\VS{2}Puis le peuple vint vers Dieu à Béthel, jusqu'au soir. Ils élevèrent leurs voix, et pleurèrent grandement,
\VS{3}Et ils dirent : Ô Yahweh, Dieu d'Israël, pourquoi est-il arrivé en Israël qu'une tribu d'Israël ait été aujourd'hui punie ?
\VS{4}Le lendemain, le peuple se leva de bon matin ; ils bâtirent là un autel, et ils offrirent des holocaustes et des sacrifices d'offrande de paix.
\VS{5}Alors les fils d'Israël dirent : Quel est celui d'entre toutes les tribus d'Israël qui n'est pas monté à l'assemblée vers Yahweh ? Car on avait fait un grand serment contre tout homme qui ne monterait pas vers Yahweh à Mitspa, en disant : Il sera puni de mort.
\VS{6}Les fils d'Israël se repentaient de ce qui était arrivé à Benjamin, leur frère, et ils disaient : Aujourd'hui une tribu a été retranchée d'Israël.
\VS{7}Comment ferons-nous pour donner des femmes à ceux qui ont survécu, puisque nous avons juré par Yahweh que nous ne leur donnerions pas nos filles pour femmes ?
\TextTitle{Avenir de la tribu de Benjamin}
\VS{8}Ils dirent donc : Y a-t-il quelqu'un d'entre les tribus d'Israël qui ne soit pas monté vers Yahweh à Mitspa ? Et voici, aucun homme de Jabès en Galaad n'était venu au camp, à l'assemblée.
\VS{9}Quand on fit le dénombrement du peuple, il n'y avait aucun des hommes habitant à Jabès en Galaad.
\VS{10}C'est pourquoi l'assemblée envoya contre eux douze mille hommes des fils vaillants, en leur donnant cet ordre : Allez, et frappez du tranchant de l'épée les habitants de Jabès en Galaad, tant les femmes que les enfants.
\VS{11}Voici les choses que vous ferez : Vous détruirez par le moyen de l'interdit tout mâle et toute femme qui a connu la couche d'un homme.
\VS{12}Ils trouvèrent parmi les habitants de Jabès en Galaad quatre cents filles vierges, qui n'avaient pas connu d'homme en couchant avec lui, et ils les amenèrent au camp de Silo, qui est sur la terre de Canaan.
\VS{13}Alors toute l'assemblée envoya parler aux fils de Benjamin qui étaient au rocher de Rimmon,  pour leur proclamer la paix.
\VS{14}En ce temps-là, les Benjamites revinrent, et on leur donna pour femmes celles qui avaient été conservées en vie d'entre les femmes de Jabès en Galaad. Mais ils n’en trouvèrent pas assez pour eux.
\VS{15}Le peuple se repentit de ce qui avait été fait à Benjamin, car Yahweh avait fait une brèche dans les tribus d'Israël.
\VS{16}Les anciens de l'assemblée dirent : Comment ferons-nous pour donner des femmes à ceux qui restent, car les femmes de Benjamin ont été détruites ?
\VS{17}Et ils dirent : Que ceux qui sont réchappés de Benjamin possèdent leur héritage, afin qu'une tribu d'Israël ne soit pas effacée.
\VS{18}Cependant, nous ne pouvons pas leur donner des femmes d'entre nos filles, car les fils d'Israël ont juré, en disant : Maudit soit celui qui donnera une femme à un Benjamite !
\VS{19}Et ils dirent : Voici, il y a chaque année une fête de Yahweh à Silo, qui est au nord de Béthel, à l'orient qui monte à Béthel, à Sichem, et au midi de Lebona.
\VS{20}Puis ils ordonnèrent aux fils de Benjamin : Allez, et placez-vous en embuscade dans les vignes.
\VS{21}Vous verrez, et voici, lorsque les filles de Silo sortiront pour danser, alors vous sortirez des vignes, vous enlèverez chacun une des filles de Silo pour en faire votre femme, et vous vous en irez dans le pays de Benjamin.
\VS{22}Si leurs pères ou leurs frères viennent se plaindre auprès de nous, nous leur dirons : Accordez-nous cette faveur, puisque nous n'avons pas pris de femmes pour chaque homme dans cette guerre.  Ce n'est pas vous qui les leur avez données ; sinon vous en seriez coupables en ce temps.
\VS{23}Les fils de Benjamin firent ainsi ; ils prirent des femmes selon leur nombre, parmi les danseuses qu'ils saisirent, puis ils s'en allèrent et retournèrent dans leur héritage ; ils rebâtirent les villes, et y habitèrent.
\VS{24}Ainsi en ce temps-là chacun des enfants d’Israël s’en alla de là dans sa tribu, et dans sa famille, et ils se retirèrent de là chacun dans son héritage.
\VS{25}En ce temps-là, il n'y avait pas de roi en Israël. L'homme faisait ce qui lui semblait être droit à ses yeux.
\PPE{}
\end{multicols}

%\clearpage\ShortTitle{1 Samuel}\BookTitle{1 Samuel}\BFont
\noindent\hrulefill
{\footnotesize
\textit{
\bigskip
{\centering{}
\\(Shemouel)
\\Signifie : Entendu, Exaucé de Dieu
\\Thème : Samuel, Saül et David
\\Auteur : Inconnu
\\Date de rédaction : 10ème siècle av. J.-C.\\}
}
%\bigskip
\textit{
\\Samuel naquit de l’union entre Elkana, de la montagne d’Ephraïm, et Anne.  Sa mère, que Yahweh avait rendu stérile, fit une alliance avec Dieu et lui promit de lui consacrer son premier fils. Ainsi, Samuel fut dès son plus jeune âge amené à la maison de Dieu où il grandit aux côtés d’Eli, le sacrificateur. A la mort de ce dernier, Samuel exerça les fonctions de juge, sacrificateur et prophète sur Israël. C’est en son temps qu’Israël manifesta le désir d’avoir un roi, marquant ainsi la fin de l’ère des juges et le début de la monarchie en Israël.
%\bigskip
\\Ce livre relate l’histoire de Saül, premier roi de l’histoire d’Israël, à qui Yahweh accorda de puissantes victoires notamment sur les philistins, grand ennemi du peuple de Dieu. Le parcours de Saul ne fut pas sans erreur, aussi Yahweh le disqualifia et choisit pour lui succéder sur le trône un homme de la tribu de Juda, David fils d’Isaï. L’accès à la royauté de ce dernier ne fut pas immédiat comme en témoignent ces écrits. David dut faire preuve de patience, de courage et de confiance en Dieu au milieu de nombreuses persécutions.
%\bigskip
\\Au travers de la vie des deux premiers rois d’Israël, est mise en évidence l’importance de l’obéissance à Dieu ; au travers de la vie de Samuel est mis en exergue l’impact de la prière dans une vie, une nation.\bigskip
}
}
\par\nobreak\noindent\hrulefill
\begin{multicols}{2}
\TextTitle{[Stérilité d'Anne la mère de Samuel]}
\Chap{1}
\VerseOne{}Il y avait un homme de Ramathaïm-Tsophim, de la montagne d'Ephraïm, nommé Elkana, fils de Jeroham, fils d'Elihu, fils de Thohu, fils de Tsuph, Ephratien.
\VS{2}Il avait deux femmes, dont l'une s'appelait Anne, et l'autre Peninna. Peninna avait des enfants, mais Anne n'en avait pas.
\VS{3}Or cet homme-là montait tous les ans, de sa ville à Silo\FTNT{Jos. 18:1.}, pour adorer Yahweh des armées, et lui offrir des sacrifices. Là étaient les deux fils d’Eli, Hophni et Phinées, sacrificateurs de Yahweh.
\VS{4}Le jour où Elkana offrait son sacrifice, il donnait des portions à Peninna, sa femme, à tous les fils et à toutes les filles qu'il avait d'elle.
\VS{5}Mais il donnait à Anne une portion double ; car il aimait Anne, mais Yahweh avait fermé sa matrice\FTNT{Dieu est celui qui ferme et ouvre les portes des bénédictions.}.
\VS{6}Sa rivale lui portait envie et la mortifiait fort aigrement afin de l’irriter, car Yahweh avait fermé sa matrice.
\VS{7}Et Elkana faisait donc ainsi tous les ans. Mais quand Anne montait à la maison de Yahweh, Peninna la mortifiait de la même manière, et Anne pleurait et ne mangeait pas.
\VS{8}Elkana, son mari, lui disait : Anne, pourquoi pleures-tu ? Et pourquoi ne manges-tu pas ? Pourquoi ton cœur est-il triste ? Est-ce que je ne vaux pas pour toi mieux que dix fils ?
\TextTitle{[Prière et voeu d'Anne à Yahweh]}
\VS{9}Anne se leva, après avoir mangé et bu à Silo. Et le sacrificateur Eli était assis sur un siège, près de l’un des poteaux du temple de Yahweh.
\VS{10}Elle donc, ayant le coeur rempli d'amertume, pria Yahweh en pleurant abondamment.
\VS{11}Et elle fit un vœu, en disant : Yahweh des armées ! Si tu regardes attentivement l'affliction de ta servante, et si tu te souviens de moi, et n'oublies pas ta servante, et que tu donnes à ta servante un enfant mâle, je le donnerai à Yahweh pour tous les jours de sa vie ; et aucun rasoir ne passera sur sa tête.
\VS{12}Il arriva, comme elle continuait à prier devant Yahweh, Eli observait sa bouche.
\VS{13}Or Anne parlait dans son cœur, elle ne faisait que remuer ses lèvres et on n'entendait pas sa voix. C’est pourquoi Eli estima qu'elle était ivre,
\VS{14}et Eli lui dit : Jusqu'à quand seras-tu ivre ? Eloigne-toi du vin.
\VS{15}Mais Anne répondit et dit : Je ne suis pas ivre, mon seigneur, je suis une femme affligée en son esprit, je n'ai bu ni vin ni boisson forte, mais je répandais mon âme devant Yahweh.
\VS{16}Ne mets pas ta servante au rang d'une femme pervertie, car c'est l’excès de ma douleur et de mon affliction qui m’a fait parler jusqu'à présent.
\VS{17}Alors Eli répondit et dit : Va en paix, et que le Dieu d'Israël veuille t’accorder la demande que tu lui as faite.
\VS{18}Et elle dit : Que ta servante trouve grâce à tes yeux ! Puis cette femme poursuivit son voyage. Elle mangea, et son visage ne fut plus le même.
\TextTitle{[Naissance de Samuel]}
\VS{19}Après cela, ils se levèrent de bon matin, et se prosternèrent devant Yahweh, puis ils s'en retournèrent et revinrent dans leur maison à Rama. Elkana connut Anne, sa femme, et Yahweh se souvint d'elle.
\VS{20}Il arriva donc, quelque temps après, qu'Anne conçut et enfanta un fils ; elle le nomma Samuel, parce que dit-elle, je l'ai demandé à Yahweh.
\VS{21}Puis Elkana, son mari, monta avec toute sa maison, pour offrir à Yahweh le sacrifice annuel et son vœu.
\VS{22}Mais Anne n'y monta pas, car elle dit à son mari : Je n’irai pas jusqu'à ce que le petit enfant soit sevré, et alors je le mènerai afin qu'il soit présenté devant Yahweh et qu'il demeure toujours-là.
\VS{23}Elkana, son mari, lui dit : Fais ce qui te semblera bon, reste jusqu'à ce que tu l'aies sevré. Seulement que Yahweh accomplisse sa parole. Ainsi cette femme resta et allaita son fils, jusqu'à ce qu'elle l’ait sevré.
\TextTitle{[Samuel chez Elie, Anne accomplie son voeu]}
\VS{24}Et dès qu'elle l'eut sevré, elle le fit monter avec elle, et ayant pris trois taureaux, un épha de farine et une outre de vin, elle le mena dans la maison de Yahweh à Silo ; l'enfant était très jeune.
\VS{25}Puis ils égorgèrent le veau, et ils amenèrent l'enfant à Eli.
\VS{26}Elle dit : Pardon, mon seigneur ! Aussi vrai que ton âme vit, mon seigneur, je suis cette femme qui me tenais en ta présence pour prier Yahweh.
\VS{27}J'ai prié pour avoir cet enfant, et Yahweh m’a accordé la demande que je lui ai faite.
\VS{28}C'est pourquoi je le prête à Yahweh ; il sera prêté à Yahweh pour tous les jours de sa vie. Et ils se prosternèrent là devant Yahweh.
\TextTitle{[Prière et prophésie d'Anne]}
\Chap{2}
\VerseOne{}Alors Anne pria, et dit : Mon cœur se réjouit en Yahweh ; ma force a été relevée par Yahweh ; ma bouche s'est ouverte contre mes ennemis, parce que je me suis réjouie de ton salut\FTNT{Le mot « salut » vient  de l’hébreu « yeshuw`ah » c’est-à-dire « Jésus ». Voir commentaire en Es. 26:1.}.
\VS{2}Nul n’est saint comme Yahweh ; car il n'y en a pas d'autre que toi, et il n'y a pas de rocher\FTNT{Voir commentaire en Es. 8:13-17.} tel que notre Dieu.
\VS{3}Ne proférez pas tant de paroles hautaines ; qu'il ne sorte pas de votre bouche des paroles arrogantes ; car Yahweh est le Dieu qui sait tout ; c'est lui qui pèse toutes les actions.
\VS{4}L'arc des puissants est brisé, mais ceux qui chancèlent ont la force pour ceinture.
\VS{5}Ceux qui étaient rassasiés, se louent pour du pain, mais les affamés ont cessé de l'être ; même la stérile en a enfanté sept et celle qui avait beaucoup de fils est devenue languissante.
\VS{6}Yahweh est celui qui fait mourir et qui fait vivre, qui fait descendre au scheol et qui en fait remonter.
\VS{7}Yahweh appauvrit et il enrichit, il abaisse et il élève.
\VS{8}De la poussière il retire le pauvre, du fumier il relève l’indigent, pour le faire asseoir avec les nobles, avec les nobles de son peuple, et il leur donne en héritage un trône de gloire ; car les colonnes de la terre sont à Yahweh, et il a posé le monde sur elles.
\VS{9}Il gardera les pieds de ses bien-aimés, et les méchants se tairont dans les ténèbres. Car l'homme ne triomphera pas par sa force.
\VS{10}Ceux qui contestent contre Yahweh seront effrayés ; des cieux il lancera son tonnerre sur chacun d'eux ; Yahweh jugera les extrémités de la terre ; et il donnera la force à son Roi\FTNT{Le Roi dont il est question ici est le Seigneur Jésus-Christ, le Roi des rois (Za. 14:9 ; Ap. 19:16). }, et élèvera la corne de son Messie\FTNT{Anne a annoncé la glorification ou la résurrection du Seigneur Jésus, le Messie (Jn. 3:14).}.
\VS{11}Puis Elkana s'en alla à Rama dans sa maison, et le jeune garçon vaquait au service de Yahweh, en présence du sacrificateur Eli.
\TextTitle{[Corruption des fils d'Eli]}
\VS{12}Or les fils d'Eli\FTNT{Les fils d’Eli, Hophni et Phinées étaient corrompus. Ils volaient les offrandes de Dieu, couchaient avec les femmes qui venaient adorer Dieu. L’esprit qui animait ces sacrificateurs  n’a pas disparu après leur mort, mais il opère  encore dans beuacoup d’institutions religieuses actuelles. Beaucoup de dirigeants d’églises continuent à s’aproprier ce qui appartient à Dieu (l’adoration, les âmes... ) Ils ne craignent pas Yahweh. Ils abusent de leur position et de leur autorité pour contraindre leurs fidèles à leur donner la dîme et toutes sortes d’offrandes. Ils font payer les entretiens, les prières, et les divers dons qu’ils peuvent avoir. Non seulement l'esprit qui animait les fils d'Eli existe encore, mais il s'est accru en ces temps actuels.} étaient des fils de Bélial\FTNT{Les fils d’Eli étaient qualifiés de « fils de Bélial ». Ce mot vient de l’hébreu « beliya`al » qui signifie « indigne », « bon à rien », « méchant », « ruine », « destruction ». Il est à noter que Bélial est aussi  un nom de Satan. (2 Co. 6:15). Les fils d’Eli servaient Dieu sans le connaître. En fait, ils étaient au service de Satan. Ce terme est également utilisé au sujet des méchants qui incitèrent les Israélites à servir les dieux étrangers (De. 13:14), les hommes iniques de Guivéa (Jg. 19:22 ;  Jg. 20:13), les deux vauriens qui accusèrent Naboth (1 R. 21:10-13) et les individus qui s’opposèrent à la monarchie (1 S. 10:27 ; 2 S. 20:1 ; 2 Ch. 13:7). Voir aussi  De. 13:13 ; De. 15:9 ; Job. 34:18 ; Ps. 18:4 ;  Ps. 34:8 ; Ps. 111:3 ; Pr. 6:12 ; Pr. 16:27 ; Pr. 19:28 ; Na. 1:11 ; Na. 1:18.} et ils ne connaissaient pas Yahweh,
\VS{13}et voici la coutume de ces sacrificateurs envers le peuple : Lorsque quelqu'un faisait quelque sacrifice, le serviteur du sacrificateur venait lorsqu'on faisait bouillir la chair, ayant à la main une fourchette à trois dents,
\VS{14}avec laquelle il piquait dans la chaudière, dans le chaudron, dans la marmite, dans le pot ; et le sacrificateur prenait pour lui tout ce que la fourchette enlevait. C’est ainsi qu’ils agissaient envers tous ceux d'Israël qui venaient à Silo.
\VS{15}Même avant qu'on fasse brûler la graisse, le serviteur du sacrificateur venait et disait à l'homme qui sacrifiait : Donne-moi de la chair à rôtir pour le sacrificateur ; car il ne prendra pas de toi de chair bouillie, mais de la chair crue.
\VS{16}Et si l'homme lui répondait : On va d’abord faire brûler la graisse, et après cela tu prendras ce que ton âme souhaitera, alors le serviteur lui disait : Quoi qu'il en soit, tu en donneras maintenant, sinon j'en prendrai de force.
\VS{17}Et le péché de ces jeunes hommes fut très grand devant Yahweh, car ils méprisaient l’offrande de Yahweh.
\TextTitle{[Samuel au service de Yahweh]}
\VS{18}Samuel faisait le service en présence de Yahweh, étant jeune garçon, vêtu d'un éphod de lin.
\VS{19}Sa mère lui faisait une petite tunique, qu'elle lui apportait tous les ans, quand elle montait avec son mari pour offrir le sacrifice annuel.
\VS{20}Eli bénit Elkana, et sa femme, et dit : Que Yahweh te donne des enfants de cette femme, pour le prêt qu’elle a fait à Yahweh. Et ils s'en retournèrent chez eux.
\VS{21}Et Yahweh visita Anne, elle conçut et enfanta trois fils et deux filles ; et le jeune garçon Samuel grandissait en présence de Yahweh.
\TextTitle{[Eli avertit ses fils]}
\VS{22}Or Eli était très vieux, il apprit tout ce que faisaient ses fils à tout Israël, et qu'ils couchaient avec les femmes qui s'assemblaient à la porte de la tente d'assignation.
\VS{23}Et il leur dit : Pourquoi commettez-vous de telles choses ? Car j'apprends vos méchantes actions de tout le peuple.
\VS{24}Ne faites pas ainsi, mes fils, car ce que j'entends dire de vous n'est pas bon ; vous faites pécher le peuple de Yahweh.
\VS{25}Si un homme a péché contre un autre homme, le juge interviendra ; mais si quelqu'un pèche contre Yahweh, qui interviendra pour lui ? Mais ils n'obéirent pas à la voix de leur père parce que Yahweh voulait les faire mourir.
\VS{26}Cependant le jeune garçon Samuel croissait et il était agréable à Yahweh et aux hommes.
\TextTitle{[Yahweh annonce un jugement sur la maison d'Eli]}
\VS{27}Or un homme de Dieu vint auprès d’Eli, et lui dit : Ainsi parle Yahweh : Ne me suis-je pas clairement manifesté à la maison de ton père, quand ils étaient en Egypte, dans la maison de Pharaon ?
\VS{28}Je l'ai choisie parmi toutes les tribus d'Israël pour être mon sacrificateur, afin d'offrir sur mon autel, et faire brûler les parfums, et porter l'éphod devant moi, et j'ai donné à la maison de ton père toutes les offrandes des enfants d'Israël consumées par le feu.
\VS{29}Pourquoi avez-vous foulé aux pieds mes sacrifices et mes offrandes que j'ai ordonné de faire dans ma demeure ? Et pourquoi as-tu honoré tes fils plus que moi, afin de vous engraisser du meilleur de toutes les offrandes d'Israël mon peuple ?
\VS{30}C'est pourquoi voici ce que dit Yahweh, le Dieu d'Israël : J'avais dit et promis que ta maison et la maison de ton père marcheraient devant moi éternellement. Et maintenant, dit Yahweh : Il n’en sera pas ainsi ; car j'honorerai ceux qui m'honorent, mais ceux qui me méprisent seront méprisés.
\VS{31}Voici, les jours viennent où je couperai ton bras, et le bras de la maison de ton père, de telle sorte qu'il n'y ait plus de vieillard dans ta maison.
\VS{32}Et tu verras un adversaire dans ma demeure, au temps où Dieu enverra toutes sortes de biens à Israël ; et il n'y aura plus jamais de vieillard dans ta maison.
\VS{33}Celui de tes descendants que je n'aurai pas retranché d'auprès de mon autel, subsistera pour consumer tes yeux et affliger ton âme ; et tous les enfants de ta maison mourront dans la fleur de l'âge.
\VS{34}Et ceci sera pour toi un signe, à savoir ce qui arrivera à tes deux fils, Hophni et Phinées, ils mourront tous les deux le même jour.
\VS{35}Et je m'établirai un sacrificateur fidèle\FTNT{Hé. 2:17 ; Hé 7:26-28). }, qui agira selon mon cœur, et selon mon âme ; et je lui édifierai une maison stable\FTNT{La maison stable fait premièrement allusion à Israël (Mi. 4) et ensuite à l’Eglise (Mt. 16:18). Cette prophétie sera pleinement réalisée lors du millénium (Za. 14).}, et il marchera à toujours devant mon Messie.
\VS{36}Et quiconque restera de ta maison, viendra se prosterner devant lui pour avoir une pièce d'argent et un morceau de pain et dira : Attache-moi, je te prie, à l’une des fonctions du sacerdoce pour manger un morceau de pain.
\TextTitle{[Yahweh appelle Samuel]}
\Chap{3}
\VerseOne{}Le jeune garçon Samuel servait Yahweh en présence d'Eli. La parole de Yahweh était rare en ce temps-là, et les visions n’étaient pas fréquentes.
\VS{2}Il arriva en ce temps qu'Eli était couché à sa place, ses yeux commençaient à se ternir et il ne pouvait plus voir.
\VS{3}Et avant que les lampes\FTNT{Le chandelier d'or à sept branches du tabernacle et du temple de Jérusalem a été décrit avec une extrême minutie dans plusieurs passages de la Bible. Il a été réalisé selon le modèle imposé par Dieu à Moïse au Sinaï (Ex. 25:31-40 ; Ex. 37:17-24 ; No. 8:4).} de Dieu soient éteintes, Samuel était aussi couché dans le temple de Yahweh, où était l'arche de Dieu.
\VS{4}Yahweh appela Samuel. Et il répondit : Me voici !
\VS{5}Et il courut vers Eli, et lui dit : Me voici, car tu m'as appelé ; mais Eli dit : Je ne t'ai pas appelé, retourne te coucher. Et il s'en alla, et se coucha.
\VS{6}Yahweh appela encore Samuel. Et Samuel se leva, et s'en alla vers Eli, et lui dit : Me voici, car tu m'as appelé ! Et Eli dit : Mon fils, je ne t'ai pas appelé, retourne, et couche-toi.
\VS{7}Or Samuel ne connaissait pas encore Yahweh, et la parole de Yahweh ne lui avait pas encore été révélée.
\VS{8}Et Yahweh appela encore Samuel pour la troisième fois ; et Samuel se leva, et s'en alla vers Eli, et dit : Me voici, car tu m'as appelé. Eli reconnut que Yahweh appelait ce jeune garçon.
\VS{9}Alors Eli dit à Samuel : Va et couche-toi ; et si on t'appelle, tu diras : Parle Yahweh , car ton serviteur écoute. Samuel donc s'en alla, et se coucha à sa place.
\VS{10}Yahweh donc vint, et se tint là ; et appela comme les autres fois : Samuel, Samuel ! Et Samuel dit : Parle, car ton serviteur écoute.
\TextTitle{[Autre avertissement de Yahweh à Eli par Samuel]}
\VS{11}Alors Yahweh dit à Samuel : Voici, je vais faire une chose en Israël, qui étourdira les oreilles de quiconque l’entendra.
\VS{12}En ce jour-là, j’accomplirai sur Eli tout ce que j’ai déclaré contre sa maison ; je commencerai, et j’achèverai.
\VS{13}Car je l'ai averti que je vais punir sa maison à perpétuité, à cause de l'iniquité dont il a connaissance, par laquelle ses fils se sont rendus infâmes, sans qu’ils les ait réprimés.
\VS{14}C'est pourquoi j'ai juré contre la maison d'Eli que jamais l’iniquité de la la maison d'Eli, ne sera expiée ni par des sacrifices ni par des offrandes.
\VS{15}Et Samuel resta couché jusqu'au matin, puis il ouvrit les portes de la maison de Yahweh. Or Samuel craignait de rapporter cette vision à Eli.
\VS{16}Mais Eli appela Samuel, et lui dit : Samuel mon fils ! Il répondit : Me voici !
\VS{17}Et Eli dit : Quelle est la parole qui t'a été adressée ? Je te prie ne me la cache pas. Que Dieu te traite avec rigueur, si tu me caches un seul mot de tout ce qui t'a été dit.
\VS{18}Samuel lui déclara donc toutes ces paroles, et ne lui en cacha rien. Et Eli répondit : C'est Yahweh, qu'il fasse ce qui lui semblera bon !
\TextTitle{[Samuel, prophète de Yahweh]}
\VS{19}Samuel grandissait. Et Yahweh était avec lui, il ne laissa pas tomber à terre une seule de ses paroles.
\VS{20}Tout Israël, depuis Dan jusqu'à Beer-Schéba, reconnut que Samuel était établi prophète de Yahweh.
\VS{21}Yahweh continuait de se manifester dans Silo ; car Yahweh se manifestait à Samuel dans Silo par la parole de Yahweh.
\TextTitle{[Les philistins prennent l'arche de Yahweh, jugement sur la maison d'Eli]}
\Chap{4}
\VerseOne{}La parole de Samuel s’adressait à tout Israël. Car Israël sortit en bataille pour aller à la rencontre des Philistins. Ils campèrent près d'Eben-Ezer, et les Philistins campaient à Aphek.
\VS{2}Les Philistins se rangèrent en bataille contre d'Israël, et le combat s’engagea, Israël fut battu par les Philistins, qui en tuèrent environ quatre mille hommes sur le champ de bataille.
\VS{3}Quand le peuple rentra au camp, les anciens d'Israël dirent : Pourquoi Yahweh nous a-t-il laissé battre aujourd'hui par les Philistins ? Ramenons de Silo l'arche de l'alliance de Yahweh, et qu'elle vienne au milieu de nous, et nous délivre de la main de nos ennemis.
\VS{4}Le peuple envoya donc à Silo, d’où l’on apporta l'arche de l'alliance de Yahweh des armées, qui habite entre les chérubins. Les deux fils d’Eli, Hophni et Phinées étaient là, avec l'arche de l'alliance de Dieu.
\VS{5}Et comme l'arche de Yahweh entrait dans le camp, tout Israël poussa de grands cris de joie et la terre en fut ébranlée.
\VS{6}Les Philistins entendirent le bruit de ces cris de joie, et ils dirent : Que veut dire ce bruit, et que signifient ces grands cris de joie dans le camp de ces Hébreux ? Et ils apprirent que l'arche de Yahweh était arrivée dans le camp.
\VS{7}Les Philistins eurent peur, car ils disaient : Dieu est entré dans le camp. Et ils dirent : Malheur à nous ! Car il n’en a pas été ainsi auparavant.
\VS{8}Malheur à nous ! Qui nous délivrera de la main de ces dieux puissants\FTNT{Le terme hébreu « elohim », généralement traduit par « dieu » ou « dieux », signifie également « dirigeants », « juges » ou encore « anges ».  Dans les textes bibliques, « Elohim » est employé pour désigner Moïse, qui a été fait « dieu » (« Elohim ») pour Pharaon (Ex. 7:1), ainsi que pour  les dieux païens  Baal, Kemosh et Dagaon (Jg. 6:31 ; Jg. 11:24 ; 1 S. 5:7)  Les Philistins avaient une vision polythéiste de la divinité et n’avaient pas la révélation du Dieu des hébreux qui est Un (Dt. 6:4).} ? C’est le Dieu qui a frappé les Egyptiens de toutes sortes de plaies dans le désert.
\VS{9}Philistins prenez courage, et agissez en hommes, de peur que vous ne soyez esclaves des Hébreux, comme ils vous ont été asservis ; agissez en hommes, et combattez !
\VS{10}Les Philistins donc combattirent, et Israël fut battu. Et chacun s’enfuit dans sa tente. La défaite fut très grande, trente mille hommes de pied d'Israël périrent .
\VS{11}L'arche de Dieu fut prise, et les deux fils d'Eli, Hophni et Phinées moururent.
\VS{12}Un homme de Benjamin s'enfuit de la bataille, et arriva à Silo ce même jour, ayant ses vêtements déchirés et la tête recouverte de terre.
\VS{13}Au moment où il arriva, Eli était dans l’attente, assis sur un siège au bord du chemin ; car son cœur tremblait à cause de l'arche de Dieu. Cet homme entra donc dans la ville, et donna les nouvelles , et toute la ville se mit à crier.
\VS{14}Eli, entendant les cris, dit : Que veut dire ce grand tumulte ? Et aussitôt cet homme vint à Eli, et lui raconta tout.
\VS{15}Or Eli était âgé de quatre-vingt-dix-huit ans, ses yeux étaient fixes, il ne pouvait plus voir.
\VS{16}L’homme dit à Eli : Je viens de la bataille, car je me suis enfui aujourd'hui de la bataille. Et Eli dit : Qu'est-il arrivé, mon fils ?
\VS{17}Celui qui apportait les nouvelles répondit : Israël a fui devant les Philistins, e il y a eu une grande défaite du peuple ; tes deux fils, Hophni et Phinées sont morts et l'arche de Dieu a été prise.
\VS{18}Et dès qu'il eut fait mention de l'arche de Dieu, Eli tomba à la renverse, de dessus son siège, à côté de la porte, se rompit le cou et mourut ; car cet homme était vieux et pesant. Il avait été juge en Israël pendant quarante ans.
\VS{19}Sa belle-fille, femme de Phinées, qui était enceinte, et sur le point d'accoucher. Lorsqu’elle apprit la nouvelle de la prise de l'arche de Dieu, de la mort de son beau-père et de son mari, elle se coucha et enfanta, car les douleurs la surprirent.
\VS{20}Comme elle mourait, celles qui l'assistaient lui dirent : Ne crains pas, car tu as enfanté un fils ; mais elle ne répondit rien, et n'en tint pas compte.
\VS{21}Mais elle appela l'enfant I-Kabod, en disant : La gloire s’en est allée d'Israël parce que l'arche de Yahweh était prise à cause de son beau-père et de son mari.
\VS{22}Elle dit donc : La gloire s’en allée d'Israël, car l'arche de Dieu est prise !
\TextTitle{[Jugements de Yahweh sur les philistins]}
\Chap{5}
\VerseOne{}Les Philistins prirent l'arche de Dieu, et l'emmenèrent d'Eben-Ezer à Asdod.
\VS{2}Les Philistins donc prirent l'arche de Dieu, et l'emmenèrent dans la maison de Dagon\FTNT{L'étymologie du nom Dagon  avait justifié la représentation qu’on faisait de ce dieu : une sorte de sirène mâle ou un homme avec queue de poisson. En effet, «  dâg », en hébreu signifie « poisson ». Il était le dieu des semences et de l'agriculture chez les peuples d’origine sémites, mais également l’un des principaux dieux des Philistins.}, et la posèrent auprès de Dagon.
\VS{3}Le lendemain les Asdodiens s'étant levés de bon matin, trouvèrent Dagon le visage contre terre, devant l'arche de Yahweh ; mais ils le prirent et le remirent à sa place.
\VS{4}Ils se levèrent encore le lendemain de bon matin, et voici, Dagon était tombé le visage contre terre, devant l'arche de Yahweh ; la tête de Dagon et les deux paumes de ses mains découpées étaient sur le seuil, et il ne lui restait que le tronc.
\VS{5}C'est pour cela que les sacrificateurs de Dagon, et tous ceux qui entrent dans la maison de Dagon, à Asdod, ne marchent pas sur le seuil jusqu'à aujourd'hui.
\VS{6}Puis la main de Yahweh s'appesantit sur le Asdodiens et les dévasta ; et il les frappa d’hémorroïdes à Asdod et dans tout son territoire.
\VS{7}Ceux donc d'Asdod, voyant qu'il en allait ainsi, dirent : L'arche du Dieu d'Israël ne demeurera pas chez nous ; car sa main s’est appesantie sur nous, et sur Dagon, notre dieu.
\VS{8}Et ils firent appeler et assemblèrent auprès d’eux tous les princes des Philistins, et dirent : Que ferons-nous de l'arche du Dieu d'Israël ? Et ils répondirent : Qu'on transporte à Gath l'arche du Dieu d'Israël. Ainsi on transporta l'arche du Dieu d'Israël.
\VS{9}Mais il arriva après qu'on l'eut transportée, la main de Yahweh fut sur la ville et il y eut une très grande terreur ; et il frappa les gens de la ville depuis le plus petit jusqu'au plus grand, par une éruption d’hémorroïdes.
\VS{10}Ils envoyèrent donc l'arche de Dieu à Ekron. Or comme l'arche de Dieu entrait à Ekron, ceux d’Ekron s'écrièrent, en disant : Ils ont transporté vers nous l'arche du Dieu d'Israël, pour nous faire mourir, nous et notre peuple !
\VS{11}C'est pourquoi ils firent appeler, et assemblèrent tous les princes des Philistins, en disant : Renvoyez l'arche du Dieu d'Israël, et qu'elle retourne en son lieu, afin qu'elle ne nous fasse pas mourir, nous et notre peuple. Car il y régnait une terreur mortelle dans toute la ville, et la main de Dieu s’y appesantissait fortement.
\VS{12}Les hommes qui n'en mouraient pas étaient frappés d’hémorroïdes, de sorte que le cri de la ville montait jusqu'au ciel.
\TextTitle{[L'arche de Yahweh revient en Israël]}
\Chap{6}
\VerseOne{}L'arche de Yahweh ayant été pendant sept mois dans le pays des Philistins.
\VS{2}Les Philistins appelèrent les sacrificateurs et les devins, et leur dirent : Que ferons-nous de l'arche de Yahweh ? Dites-nous comment nous devons la renvoyer en son lieu.
\VS{3}Ils répondirent : Si vous renvoyez l'arche du Dieu d'Israël, ne la renvoyez pas à vide, et n’oubliez pas de lui payer une offrande de culpabilité ; alors vous serez guéris, et vous saurez pourquoi sa main ne s’est pas retirée de dessus vous.
\VS{4}Et ils dirent : Quelle offrande lui payerons-nous pour le péché ? Et ils répondirent : Selon le nombre des princes des Philistins, vous donnerez cinq hémorroïdes d'or, et cinq souris d'or ; car une même plaie a été sur vous tous, et sur vos princes.
\VS{5}Vous ferez donc des figures de vos hémorroïdes, et des figures des souris qui ravagent le pays, et vous donnerez gloire au Dieu d'Israël. Peut-être retirera-t-il sa main de dessus vous, et de dessus vos dieux, et de dessus votre pays.
\VS{6}Et pourquoi endurciriez-vous votre cœur, comme l'Egypte et Pharaon ont endurci leur cœur ? Après qu'il eut fait de merveilleux exploits parmi eux, ne les laissèrent-ils pas partir et s’en aller ?
\VS{7}Maintenant, donc prenez de quoi faire un char tout neuf, et deux jeunes vaches qui allaitent leurs veaux et qui n’aient point porté le joug ; et attelez au char les deux jeunes vaches, et ramenez leurs petits à la maison.
\VS{8}Vous prendrez l'arche de Yahweh et vous la mettrez sur le char ; et vous déposerez dans un coffre, à côté de l’arche, les objets d'or que vous donnez à Yahweh en offrande pour le péché ; vous la renverrez, et elle s'en ira.
\VS{9}Et vous observerez ; si l'arche monte vers Beth-Schémesch, par le chemin de sa frontière, c'est Yahweh qui nous a fait tout ce grand mal ; si elle n'y va pas, nous saurons alors que sa main ne nous a pas touchés, mais que ceci nous est arrivé par hasard.
\VS{10}Ces gens firent ainsi. Ils prirent donc deux jeunes vaches qui allaitaient, ils les attelèrent au char, et ils enfermèrent leurs petits dans l'étable.
\VS{11}Ils mirent sur le char l'arche de Yahweh, et le coffre avec les souris d'or, et les figures de leurs hémorroïdes.
\VS{12}Alors les jeunes vaches prirent tout droit le chemin de Beth-Schémesch, elles suivirent toujours le même chemin en marchant et en mugissant ; et elles ne se détournèrent ni à droite ni à gauche. Les princes des Philistins allèrent après elles jusqu'à la frontière de Beth-Schémesch.
\VS{13}Or ceux de Beth-Schémesch, moissonnaient les blés dans la vallée ; et ayant élevé leurs yeux, ils virent l'arche, et se réjouirent en la voyant.
\VS{14}Le char arriva dans le champ de Josué de Beth-Schémesch, et s'arrêta là. Or il y avait là une grande pierre, et on fendit le bois du char, et on offrit les jeunes vaches en holocauste à Yahweh.
\VS{15}Les Lévites descendirent l'arche de Yahweh, et le coffre dans lequel étaient les objets d'or, et ils les mirent sur cette grande pierre. En ce même jour, ceux de Beth-Schémesch offrirent des holocaustes et des sacrifices à Yahweh.
\VS{16}Les cinq princes des Philistins, après avoir vu cela, retournèrent le même jour à Ekron.
\VS{17}Voici les hémorroïdes d'or que les Philistins donnèrent à Yahweh en offrande pour le péché ; un pour Asdod, un pour Gaza, un pour Askalon, un pour Gath, un pour Ekron.
\VS{18}Les souris d’or, selon le nombre de toutes les villes des Philistins, appartenant aux cinq princes, tant des villes fortifiées, que des villages sans murailles. Et ils les amenèrent jusqu'à la grande pierre sur laquelle on posa l'arche de Yahweh, et qui jusqu'à ce jour est dans le champ de Josué de Beth-Schémesch.
\VS{19}Yahweh frappa des gens de Beth-Schémesch parce qu'ils avaient regardé dans l'arche de Yahweh ; il frappa (cinquante mille) et soixante-dix hommes\FTNT{Ce nombre est généralement considéré comme une erreur des copistes.} et le peuple mena le deuil parce que Yahweh l'avait frappé d'une grande plaie.
\VS{20}Alors ceux de Beth-Schémesch dirent : Qui pourrait subsister en présence de Yahweh, ce Dieu Saint ? Et vers qui montera-t-il en s'éloignant de nous ?
\VS{21}Et ils envoyèrent des messagers aux habitants de Kirjath-Jearim, en disant : Les Philistins ont ramené l'arche de Yahweh ; descendez, et faites-la monter vers vous.
\TextTitle{[Un réveil après l'apostasie]}
\Chap{7}
\VerseOne{}Ceux donc de Kirjath-Jearim vinrent et firent monter l'arche de Yahweh, et la mirent dans la maison d'Abinadab sur la colline ; et ils consacrèrent Eléazar, son fils, pour garder l'arche de Yahweh.
\VS{2}Il s’écoula un long moment, depuis le jour où l'arche de Yahweh fut déposée à Kirjath-Jearim. Vingt années s’étaient écoulées. Toute la maison d'Israël soupira après Yahweh.
\VS{3}Et Samuel parla à toute la maison d'Israël, en disant : Si vous revenez à Yahweh de tout votre cœur, ôtez du milieu de vous les dieux étrangers, et les Astartés, dirigez votre cœur vers Yahweh, et servez-le lui seul ; et il vous délivrera de la main des Philistins.
\VS{4}Alors les enfants d'Israël ôtèrent les Baals, et les Astartés, et ils servirent Yahweh seul\FTNT{Jg. 2:13.}.
\VS{5}Samuel dit : Assemblez tout Israël à Mitspa, et je prierai Yahweh pour vous.
\VS{6}Ils s'assemblèrent donc à Mitspa ; ils puisèrent de l'eau qu'ils répandirent devant Yahweh et ils jeûnèrent ce jour-là, en disant : Nous avons péché contre Yahweh ! Et Samuel jugea les enfants d'Israël à Mitspa.
\VS{7}Or quand les Philistins eurent appris que les enfants d'Israël étaient assemblés à Mitspa, les princes des Philistins montèrent contre Israël. Les enfants d'Israël l’apprirent et ils eurent peur des Philistins.
\VS{8}Les enfants d'Israël dirent à Samuel : Ne cesse pas de crier pour nous à Yahweh, notre Dieu, afin qu'il nous délivre de la main des Philistins.
\TextTitle{[Victoire d'Israël contre les Philistins]}
\VS{9}Alors Samuel prit un agneau de lait, et l'offrit tout entier à Yahweh en holocauste. Et Samuel cria à Yahweh pour Israël, et Yahweh l'exauça.
\VS{10}Comme Samuel offrait l'holocauste, les Philistins s'approchèrent pour combattre contre Israël, mais Yahweh fit gronder, en ce jour-là, un grand tonnerre sur les Philistins, et les mit en déroute, et ils furent battus devant Israël.
\VS{11}Les hommes d'Israël sortirent de Mitspa, et poursuivirent les Philistins, et les frappèrent jusqu'au-dessous de Beth-Car.
\VS{12}Alors Samuel prit une pierre, et la mit entre Mitspa et Schen, et il appela ce lieu Eben-Ezer, en disant : Yahweh nous a secourus jusqu'en ce lieu-ci.
\VS{13}Les Philistins furent humiliés, et ils ne vinrent plus sur le territoire d'Israël. La main de Yahweh fut contre les Philistins durant la vie de Samuel.
\VS{14}Les villes que les Philistins avaient prises sur Israël, retournèrent à Israël, depuis Ekron jusqu'à Gath, avec leurs territoires. Israël les délivra donc de la main des Philistins, et il y eut paix entre Israël et les Amoréens.
\VS{15}Samuel fut juge en Israël tous les jours de sa vie.
\VS{16}Il allait tous les ans faire le tour de Béthel, de Guilgal et de Mitspa, et il jugeait Israël dans tous ces lieux.
\VS{17}Puis il revenait à Rama, où était sa maison ; et là il jugeait Israël, et il y bâtit un autel à Yahweh.
\TextTitle{[Israël veut un roi]}
\Chap{8}
\VerseOne{}Lorsque Samuel devint vieux, il établit ses fils juges sur Israël.
\VS{2}Son fils premier-né s’appelait Joël, et le second Abija ; ils jugeaient à Beer-Schéba.
\VS{3}Mais ses fils ne marchèrent pas dans ses voies, ils s’en détournèrent pour les profits acquis par la violence ; ils recevaient des présents et violaient la justice.
\VS{4}C'est pourquoi tous les anciens d'Israël s'assemblèrent, et vinrent auprès de Samuel à Rama.
\VS{5}Ils lui dirent : Voici, tu es devenu vieux, et tes fils ne suivent pas tes voies ; maintenant, établis sur nous un roi pour nous juger comme il y en a chez toutes les nations.
\TextTitle{[Prière de Samuel et réponse de Yahweh]}
\VS{6}Samuel fut affligé de ce qu'ils lui avaient dit : Etablis sur nous un roi pour nous juger. Et Samuel pria Yahweh.
\VS{7}Yahweh dit à Samuel : Obéis à la voix du peuple dans tout ce qu'il te dira, car ce n'est pas toi qu'ils ont rejeté, mais c'est moi qu'ils ont rejeté, afin que je ne règne plus sur eux.
\VS{8}Ils agissent à ton égard comme ils ont agi depuis le jour où je les ai fait monter hors d'Egypte jusqu’à ce jour ; ils m’ont abandonné, pour servir d'autres dieux.
\VS{9}Maintenant donc, obéis à leur voix ; mais ne manque pas de les avertir, en leur déclarant comment le roi qui régnera sur eux, les traitera.
\TextTitle{[Avertissements aux enfants d'Israël qui demandent un roi]}
\VS{10}Ainsi Samuel dit toutes les paroles de Yahweh, au peuple qui lui avait demandé un roi.
\VS{11}Il leur dit donc : Voici comment vous traitera le roi qui régnera sur vous. Il prendra vos fils et les mettra sur ses chars et parmi ses cavaliers, afin qu’ils courent devant son char ;
\VS{12}il en établira des chefs de mille, et des chefs de cinquante, pour labourer ses terres, pour récolter ses moissons, et pour fabriquer ses armes de guerre et l’équipement de ses chars.
\VS{13}Il prendra aussi vos filles pour en faire des parfumeuses, des cuisinières, et des boulangères.
\VS{14}Il prendra ce qu’il y a de meilleur parmi vos champs, vos vignes et vos oliviers, et il les donnera à ses serviteurs.
\VS{15}Il prélèvera la dîme de ce que vous aurez semé et de ce que vous aurez vendangé, et il la donnera à ses eunuques, et à ses serviteurs.
\VS{16}Il prendra vos serviteurs et vos servantes, l'élite de vos jeunes gens, vos ânes, et les emploiera à ses ouvrages.
\VS{17}Il prélèvera la dîme de vos troupeaux, et vous serez ses esclaves.
\VS{18}En ce jour-là, vous crierez à cause du roi que vous vous serez choisi, mais Yahweh ne vous exaucera pas.
\VS{19}Mais le peuple refusa d’écouter la voix de Samuel, et ils dirent : Non ! Mais il y aura un roi sur nous.
\VS{20}Nous serons aussi comme toutes les nations ; et notre roi nous jugera, il sortira devant nous, et il conduira nos guerres.
\VS{21}Samuel entendit donc toutes les paroles du peuple, et les rapporta à Yahweh.
\VS{22}Et Yahweh dit à Samuel : Obéis à leur voix, et établis un roi sur eux. Et Samuel dit aux hommes d'Israël : Allez-vous-en chacun dans sa ville.
\TextTitle{[Dieu leur donne un roi : Saül]}
\Chap{9}
\VerseOne{}Il y avait un homme de Benjamin, nommé Kis, fort et vaillant, fils d Abiel, fils de Tseror, fils de Becorath, fils d'Aphiach, fils d'un Benjamite.
\VS{2}Il avait un fils nommé Saül, jeune et beau, et aucun des enfants d'Israël n’était plus beau que lui, des épaules en haut, il dépassait tout le peuple.
\VS{3}Les ânesses de Kis, père de Saül, s’égarèrent; et Kis dit à Saül, son fils : Prends maintenant avec toi un des serviteurs et lève-toi, et va chercher les ânesses.
\VS{4}Il passa donc par la montagne d'Ephraïm et traversa le pays de Schalischa ; mais ils ne les trouvèrent pas ; puis ils passèrent par le pays de Schaalim, mais elles n'y étaient pas ; ils passèrent ensuite par le pays de Benjamin, mais ils ne les trouvèrent pas.
\VS{5}Quand ils furent arrivés dans le pays de Tsuph, Saül dit à son serviteur qui était avec lui : Viens, et retournons, de peur que mon père oublie les ânesses, et s’inquiète pour nous.
\VS{6}Le serviteur lui dit : Voici, je te prie, il y a dans cette ville un homme de Dieu, qui est un homme très honoré ; tout ce qu'il déclare ne manque pas d’arriver ; allons y maintenant, peut-être nous renseignera-t-il sur le chemin que nous devons prendre.
\VS{7}Et Saül dit à son serviteur : Mais si nous y allons, que porterons-nous à l'homme de Dieu, nous n’avons plus de provisions, et nous n'avons aucun présent pour l'homme de Dieu ? Qu’est-ce que nous avons ?
\VS{8}Le serviteur reprit la parole et dit à Saül : Voici j'ai encore entre mes mains le quart d'un sicle d'argent, et je le donnerai à l'homme de Dieu, et il nous indiquera notre chemin.
\VS{9}Autrefois en Israël quand on allait consulter Dieu, on se disait l'un à l'autre : Venez, allons vers le voyant ! Car le prophète, s'appelait autrefois le voyant.
\VS{10}Saül dit à son serviteur : Tu as bien dit ; viens, allons ! Et ils s'en allèrent dans la ville où était l'homme de Dieu.
\VS{11}Et comme ils montaient à la ville, ils trouvèrent de jeunes filles qui sortaient pour puiser de l'eau, et ils leur dirent : Le voyant n'est-il pas ici ?
\VS{12}Elles leur répondirent, et dirent : Il y est, le voilà devant toi ; hâte-toi maintenant, car il est venu aujourd'hui à la ville, parce qu'il y a aujourd'hui un sacrifice pour le peuple sur le haut lieu.
\VS{13}Quand vous entrerez dans la ville, vous le trouverez avant qu'il monte au haut lieu pour manger ; car le peuple ne mangera pas jusqu'à ce qu'il soit venu, parce qu'il doit bénir le sacrifice ; après quoi, les conviés mangeront. Montez donc maintenant, car vous le trouverez aujourd'hui.
\VS{14}Ils montèrent donc à la ville. Comme ils entraient dans la ville, Samuel, qui sortait pour monter au haut lieu, les rencontra.
\VS{15}Or, un jour avant l’arrivée de Saül, Yahweh avait fait une révélation à Samuel, en disant :
\VS{16}Demain, à cette même heure, je t'enverrai un homme du pays de Benjamin, et tu l'oindras pour être le conducteur de mon peuple d'Israël. Il délivrera mon peuple de la main des Philistins ; car j'ai regardé mon peuple parce que son cri est venu jusqu'à moi.
\VS{17}Et dès que Samuel eut aperçu Saül, Yahweh lui dit : Voici l'homme dont je t'ai parlé ; c'est lui qui dominera sur mon peuple.
\VS{18}Et Saül s'approcha de Samuel au milieu de la porte, et dit : Indique-moi je te prie, où est la maison du voyant.
\VS{19}Et Samuel répondit à Saül, et dit : Je suis le voyant. Monte devant moi au haut lieu, et vous mangerez aujourd'hui avec moi. Je te laisserai partir demain, et je te dirai tout ce que tu as sur le cœur.
\VS{20}Mais quant aux ânesses que tu as perdues il y a trois jours, ne t'en inquiète pas, parce qu'elles ont été retrouvées. Et vers qui tend tout le désir d’Israël ? N’est-ce pas vers toi, et vers toute la maison de ton père ?
\VS{21}Saül répondit : Ne suis-je pas de Benjamin, l’une des moindres tribus d'Israël, et ma famille n'est-elle pas la plus petite de toutes tribus de Benjamin ? Pourquoi m’as-tu tenu de tels discours ?
\VS{22}Samuel prit Saül et son serviteur, et les fit entrer dans la salle, et les plaça à la tête des conviés, qui étaient environ trente hommes.
\VS{23}Et Samuel dit au cuisinier : Apporte la portion que je t'ai donnée, en te disant : Mets-la à part.
\VS{24}Le cuisinier prit l’épaule, et ce qui l’entoure, et il la servit à Saül. Et Samuel dit : Voici ce qui a été réservé, mets-le devant toi, et mange, car il t’a été gardé expressément pour cette heure, lorsque j'ai résolu de convier le peuple ; et Saül mangea avec Samuel ce jour-là.
\VS{25}Puis ils descendirent du haut lieu dans la ville, et Samuel parla avec Saül sur le toit.
\VS{26}Puis ils se levèrent de bon matin ; et, dès l’aurore, Samuel appela Saül sur le toit, et lui dit : Lève-toi, et je te laisserai aller. Saül donc se leva, et ils sortirent tous deux dehors, lui et Samuel.
\VS{27}Et comme ils descendaient à l’extrémité de la ville, Samuel dit à Saül : Dis au serviteur de passer devant nous, et le serviteur passa devant. Arrête-toi maintenant, afin que je te fasse entendre la parole de Dieu.
\TextTitle{[Samuel oint Saül comme roi]}
\Chap{10}
\VerseOne{}Or, Samuel prit une fiole d'huile, qu’il répandit sur la tête de Saül. Il l’embrassa, et lui dit : Yahweh ne t'a-t-il pas oint, pour être le conducteur de son héritage?
\VS{2}Aujourd’hui, après m’avoir quitté, tu trouveras deux hommes près du sépulcre de Rachel, sur la frontière de Benjamin à Tseltsach, qui te diront : Les ânesses que tu étais allé chercher sont retrouvées ; et voici, ton père ne pense plus aux ânesses, mais il s’inquiète pour vous, disant : Que dois-je faire à propos de mon fils ?
\VS{3}En allant plus loin, tu arriveras au chêne de Thabor, où tu seras rencontré par trois hommes qui montent vers Dieu, à Béthel, et l'un porte trois chevreaux, l'autre trois pains, et l'autre une outre de vin.
\VS{4}Ils te demanderont comment tu te portes, et ils te donneront deux pains, que tu recevras de leurs mains.
\VS{5}Après cela tu arriveras à Guibea-Elohim, où se trouve une garnison des Philistins. Et il arrivera qu’en entrant dans la ville, tu rencontreras une troupe de prophètes descendant du haut lieu, précédés du luth, du tambourin, de la flûte, et de la harpe, et qui prophétisent.
\VS{6}Alors l'Esprit de Yahweh te saisira, et tu prophétiseras avec eux, et tu seras changé en un autre homme.
\VS{7}Et quand ces signes te seront arrivés, fais avec force ce que tu trouveras, car Dieu est avec toi.
\VS{8}Puis tu descendras devant moi à Guilgal, et voici, je descendrai vers toi pour offrir des holocaustes, et des sacrifices d’offrande de paix, tu m'attendras là sept jours, jusqu'à ce que je vienne, et que je te déclare ce que tu devras faire.
\VS{9}Aussitôt que Saül eut tourné le dos pour se séparer de Samuel, Dieu changea son cœur, et tous ces signes s’accomplir le même jour.
\VS{10}Quand ils arrivèrent à Guibea, voici une troupe de prophètes vint à sa rencontre. L'Esprit de Dieu le saisit, et il prophétisa au milieu d'eux.
\VS{11}Tous ceux qui le connaissaient de longue date, le virent prophétiser avec les prophètes. Il se dirent l'un à l'autre : Qu'est-il arrivé au fils de Kis ? Saül est-il aussi parmi les prophètes ?
\VS{12}Un homme répondit : Et qui est leur père ? De là le proverbe : Saül est-il aussi parmi les prophètes ?
\VS{13}Lorsqu’il eut cessé de prophétiser, il se rendit au haut lieu.
\VS{14}L'oncle de Saül dit à Saül et à son serviteur : Où êtes-vous allés ? Et il répondit : Chercher les ânesses, mais ne les trouvant pas nous sommes allés vers Samuel.
\VS{15}Et l’oncle de Saül dit : Déclare-moi, je te prie, ce que vous a dit Samuel.
\VS{16}Saül répondit à son oncle : Il nous a assuré que les ânesses étaient retrouvées ; mais il ne lui déclara rien concernant la royauté dont Samuel lui avait parlé.
\VS{17}Samuel convoqua le peuple devant Yahweh, à Mitspa.
\VS{18}Et il dit aux enfants d'Israël : Ainsi parle Yahweh, le Dieu d'Israël : J'ai fait monter Israël hors d'Egypte, et je vous ai délivrés de la main des Egyptiens, et de la main de tous les royaumes qui vous opprimaient.
\VS{19}Mais aujourd'hui, vous avez rejeté votre Dieu, celui qui vous a délivrés de tous vos malheurs, et de vos afflictions, et vous avez dit : Non, établis-nous un roi. Présentez-vous donc maintenant, devant Yahweh, par tribus, et par familles.
\VS{20}Ainsi Samuel fit approcher toutes les tribus d'Israël ; et la tribu de Benjamin fut désignée.
\VS{21}Après il fit approcher la tribu de Benjamin selon ses familles ; et la famille de Matri fut désignée; puis Saül fils de Kis fut désigné, on le chercha, mais on ne le trouva pas.
\VS{22}On consulta de nouveau Yahweh : Est-il encore venu quelqu’un ici ? Yahweh répondit : Il est caché parmi les bagages.
\VS{23}Ils coururent donc le chercher, et il se présenta au milieu du peuple, et il était plus grand que tout le peuple, depuis les épaules en haut.
\VS{24}Et Samuel dit à tout le peuple : Voyez-vous celui que Yahweh a choisi, il n'y a personne dans tout le peuple qui soit semblable à lui. Et le peuple poussa des cris de joie, et dit : Vive le roi !
\VS{25}Alors Samuel fit connaître au peuple les règles de la royauté, et les écrivit dans un livre, qu’il déposa devant Yahweh. Puis Samuel renvoya le peuple, chacun dans sa maison.
\VS{26}Saül aussi s'en alla chez lui à Guibea. Il fut accompagné par des vaillants hommes dont Dieu avait touché le cœur.
\VS{27}Mais il y eut des fils de Bélial\FTNT{1 S. 2:12.} qui dirent : Comment celui-ci nous délivrerait-il ? Et ils le méprisèrent, et ne lui apportèrent pas de présent. Mais Saül fit le sourd.
\TextTitle{[Saül vainqueur des Ammonites]}
\Chap{11}
\VerseOne{}Nachasch, l’Ammonite, vint et assiégea Jabès, en Galaad. Les habitants de Jabès dirent à Nachasch : Traite alliance avec nous et nous te servirons.
\VS{2}Mais Nachasch, l’Ammonite, leur répondit : Je traiterai avec vous à la condition que je vous crève à tous l’œil droit, et que je mette cet opprobre sur tout Israël.
\VS{3}Les anciens de Jabès lui dirent : Donne-nous sept jours de trêve, et nous enverrons des messagers dans tout le territoire d'Israël, et s'il n'y a personne qui nous délivre, nous nous rendrons à toi.
\VS{4}Les messagers arrivèrent à Guibea de Saül, et dirent ces paroles devant le peuple. Tout le peuple éleva sa voix, et pleura.
\VS{5}Et voici, Saül revenait des champs derrière ses bœufs, et il dit : Qu'est-ce qu'a ce peuple pour pleurer ainsi ? Et on lui raconta ce qu'avaient dit ceux de Jabès.
\VS{6}Et l'Esprit de Dieu saisit Saül, lorsqu'il entendit ces paroles, et sa colère s’enflamma fortement.
\VS{7}Il prit une paire de bœufs, et les coupa en morceaux qu’il envoya dans tous le territoire d'Israël, par des messagers, en disant : Les boeufs de tous ceux qui ne sortiront pas pour suivre Saül et Samuel, seront traités de la même manière. Et la frayeur de Yahweh tomba sur le peuple, et ils sortirent comme un seul homme.
\VS{8}Saül en fit la revue à Bézek ; les enfants d'Israël étaient trois cents mille et ceux de Juda trente mille.
\VS{9}Puis, ils dirent aux messagers qui étaient venus : Vous parlerez ainsi à ceux de Jabès en Galaad : Vous serez délivrés demain, quand le soleil sera dans sa force. Les messagers rapportèrent donc cela à ceux de Jabès, qui s'en réjouirent ;
\VS{10}et ils dirent aux Ammonites : Demain nous nous rendrons à vous, et vous nous traiterez selon votre bon plaisir.
\VS{11}Le lendemain, Saül disposa le peuple en trois corps. Ils entrèrent dans le camp des Ammonites à la veille du matin, et ils les battirent jusqu’à la chaleur du jour. Ceux qui échappèrent furent dispersés si bien qu'il n'en resta pas deux ensemble.
\TextTitle{[Le peuple reconnait Saül comme roi]}
\VS{12}Le peuple dit à Samuel : Qui est-ce qui dit : Saül régnera-t-il sur nous ? Donnez-nous ces hommes-là, et nous les ferons mourir.
\VS{13}Saül répondit : Personne ne sera mis à mort en ce jour, car Yahweh a délivré Israël aujourd'hui .
\VS{14}Et Samuel dit au peuple : Venez, allons à Guilgal, et nous y renouvellerons la royauté.
\VS{15}Et tout le peuple se rendit à Guilgal, et là, ils établirent Saül pour roi devant Yahweh, à Guilgal. Et ils offrirent des sacrifices d’offrande de paix, devant Yahweh ; Saül et tous ceux d'Israël se réjouirent beaucoup.
\TextTitle{[Le peuple atteste l'intégrité de Samuel]}
\Chap{12}
\VerseOne{}Alors Samuel dit à tout Israël : Voici, j'ai obéi à votre voix dans tout ce que vous m'avez dit, et j'ai établi un roi sur vous.
\VS{2}Et maintenant, voici le roi qui marchera devant vous. Car moi, je suis vieux et tout blanc, et voici, mes fils aussi sont avec vous ; pour moi j'ai marché devant vous, depuis ma jeunesse jusqu’à ce jour.
\VS{3}Me voici, témoignez contre moi, devant Yahweh, et devant son oint. De qui ai-je pris le bœuf ? Et de qui ai-je pris l'âne ? Qui ai-je opprimé ? Qui ai-je traité durement ? Et de la main de qui ai-je reçu des présents, afin de fermer les yeux sur lui ? Et je vous le rendrai.
\VS{4}Et ils répondirent : Tu ne nous as pas opprimés, tu ne nous as pas traités durement et tu n'as rien reçu de la main de personne.
\VS{5}Il leur dit encore : Yahweh est témoin contre vous, et son oint aussi est témoin aujourd'hui, que vous n'avez rien trouvé entre mes mains. Et ils répondirent : Il en est témoin.
\TextTitle{[Exhortation de Samuel]}
\VS{6}Alors Samuel dit au peuple : Yahweh est celui qui a établi Moïse et Aaron, et qui a fait monter vos pères hors du pays d'Egypte.
\VS{7}Maintenant donc, présentez-vous, et je vous jugerai devant Yahweh sur tous les bienfaits que Yahweh vous a accordés, à vous et à vos pères.
\VS{8}Après que Jacob fut entré en Egypte, vos pères crièrent à Yahweh, et Yahweh envoya Moïse et Aaron qui firent sortir vos pères hors d'Egypte, et les firent habiter en ce lieu.
\VS{9}Mais ils oublièrent Yahweh, leur Dieu, et il les livra entre les mains de Sisera, chef de l'armée de Hatsor, et entre les mains des Philistins, et entre les mains du roi de Moab, qui leur firent la guerre.
\VS{10}Ils crièrent encore à Yahweh, et dirent : Nous avons péché ; car nous avons abandonné Yahweh, et nous avons servi les Baals et les Astartés. Maintenant donc, délivre-nous de la main de nos ennemis, et nous te servirons.
\VS{11}Et Yahweh envoya Jerubbaal, Bedan, Jephthé et Samuel, et il vous délivra de la main de tous vos ennemis d'alentour, et vous demeurâtes en sécurité.
\VS{12}Mais voyant que Nachasch, roi des fils d’Ammon, marchait contre vous, vous m'avez dit : Non ! Mais un roi régnera sur nous. Alors que Yahweh, votre Dieu, était votre Roi.
\VS{13}Maintenant donc, voici le roi que vous avez choisi, que vous avez demandé, et voici Yahweh l'a établi roi sur vous.
\VS{14}Si vous craignez Yahweh, si vous le servez, et obéissez à sa voix, et que vous n’êtes pas rebelles au commandement de Yahweh, alors vous et votre roi qui règne sur vous, vous serez sous la conduite de Yahweh, votre Dieu.
\VS{15}Mais si vous n'obéissez pas à la voix de Yahweh, et si vous êtes rebelles au commandement de Yahweh, la main de Yahweh sera aussi contre vous, comme elle a été contre vos pères.
\VS{16}Maintenant, préparez-vous, et voyez cette grande chose que Yahweh va opérer sous vos yeux.
\VS{17}N'est-ce pas aujourd'hui la moisson des blés ? Je crierai à Yahweh, et il enverra des tonnerres et de la pluie. Sachez alors et voyez combien vous avez mal agi aux yeux de Yahweh en demandant un roi.
\VS{18}Alors Samuel cria à Yahweh, et Yahweh envoya des tonnerres et de la pluie ce même jour. Tout le peuple eut une grande crainte de Yahweh, et de Samuel.
\VS{19}Et tout le peuple dit à Samuel : Prie Yahweh, ton Dieu, pour tes serviteurs, afin que nous ne mourions pas ; car nous avons ajouté à nos péchés, celui d'avoir demandé un roi.
\VS{20}Alors Samuel dit au peuple : Ne craignez pas ! Vous avez fait tout ce mal, néanmoins ne vous détournez pas de Yahweh, mais servez Yahweh de tout votre cœur.
\VS{21}Ne vous en détournez pas car vous iriez après des choses de néant, qui ne vous apportent ni profit ni délivrance ; puisque ce sont des choses de néant.
\VS{22}Car Yahweh n’abandonne pas son peuple, pour l'amour de son grand Nom, car Yahweh a résolu de faire de vous son peuple.
\VS{23}Et pour moi, Dieu me garde de pécher contre Yahweh, et de cesser de prier pour vous ! Je vous enseignerai le bon et le droit chemin.
\VS{24}Craignez seulement Yahweh, et servez-le en vérité, de tout votre cœur ; car vous avez vu les choses magnifiques qu'il a faites pour vous.
\VS{25}Mais si vous persévérez à faire le mal, vous serez détruits vous et votre roi.
\TextTitle{[Saül pèche contre Yahweh en offrant l'holocauste]}
\Chap{13}
\VerseOne{}Saül régna un an sur Israël et après deux années, 
\VS{2} Saül choisit trois mille hommes d'Israël, deux mille avec lui à Micmasch, et sur la montagne de Béthel, et mille étaient avec Jonathan à Guibea de Benjamin. Il renvoya le reste du peuple, chacun à sa tente.
\VS{3}Et Jonathan battit le poste des Philistins qui était à Guéba, et les Philistins en furent informés ; et Saül fit sonner le shofar dans tout le pays, en disant : Que les Hébreux écoutent !
\VS{4}Tout Israël apprit donc que Saül avait battu le poste des Philistins, et Israël se rendit odieux aux Philistins. Et le peuple fut convoqué auprès de Saül, à Guilgal.
\VS{5}Les Philistins s'assemblèrent pour combattre Israël, ayant trente mille chars et six mille cavaliers ; et le peuple était aussi nombreux que le sable au bord de la mer, tant il était en grand nombre ; ils allèrent prendre position à Micmasch, à l'orient de Beth-Aven.
\VS{6}Les hommes d'Israël furent pris d’une grande angoisse ; car ils étaient oppressés, c'est pourquoi le peuple se cacha dans les cavernes, dans les buissons, dans les rochers, dans les tours et dans des citernes.
\VS{7}Les Hébreux passèrent le Jourdain pour aller au pays de Gad, et de Galaad. Saül était encore à Guilgal, aussi tout le peuple effrayé le rejoignit.
\VS{8}Il attendit sept jours selon le terme fixé par Samuel ; mais Samuel ne venait pas à Guilgal et le peuple se dispersait.
\VS{9}Et Saül dit : Amenez-moi un holocauste et des sacrifices d’offrande de paix, et il offrit l'holocauste.
\VS{10}Comme il achevait d'offrir l'holocauste, Samuel arriva, et Saül sortit au-devant de lui pour le saluer.
\VS{11}Et Samuel lui dit : Qu'as-tu fait ? Saül répondit : Lorsque j’ai vu que le peuple se dispersait, que tu ne venais pas au jour fixé, et que les Philistins étaient assemblés à Micmasch ;
\VS{12}J'ai dit : Les Philistins descendront maintenant contre moi à Guilgal, et je n'ai pas supplié Yahweh ! Je me suis maîtrisé un temps, mais j'ai fini par offrir l'holocauste.
\VS{13}Samuel répondit à Saül : C’est en insensé que tu as agi, car tu n'as pas gardé le commandement que Yahweh, ton Dieu t'avait donné ; car Yahweh aurait maintenu à jamais ta royauté sur Israël.
\VS{14}Et maintenant ta royauté ne subsistera pas ; Yahweh s'est choisi un homme selon son cœur, et Yahweh l’a destiné à être le chef de son peuple parce que tu n'as pas respecté le commandement de Yahweh.
\VS{15}Puis Samuel se leva, et monta de Guilgal à Guibea de Benjamin. Et Saül passa en revue le peuple qui se trouvait avec lui, qui fut d'environ six cents hommes.
\VS{16}Or Saül vint s’établir avec son fils Jonathan, et le peuple qui était sous ses ordres à Guibea de Benjamin, et les Philistins étaient campés à Micmasch.
\VS{17}Les Philistins sortirent du camp en trois divisions pour ravager ; l'une de ces divisions prit le chemin d’Ophra, vers le pays de Schual ;
\VS{18}l'autre division prit le chemin de Beth-Horon ; et la troisième prit le chemin de la frontière qui regarde vers la vallée de Tseboïm, du côté du désert.
\VS{19}Or dans tout le pays d'Israël, il ne se trouvait aucun forgeron ; car les Philistins avaient dit : Empêchons les Hébreux de faire des épées ou des lances.
\VS{20}C'est pourquoi chaque homme descendait vers les Philistins, pour aiguiser son soc, son hoyau, sa hache, et sa bêche ;
\VS{21}lorsque le tranchant des bêches, des hoyaux, des tridents, et des haches était émoussé, même pour redresser un aiguillon.
\VS{22}De sorte qu’il arriva qu’au jour du combat, nul n’avait d’ épée ni de lance dans toute l’armée qui était avec Saül et Jonathan ; si ce n’est Saül lui-même et Jonathan, son fils.
\VS{23}Un poste de Philistins s’établit au passage de Micmasch.
\TextTitle{[Courage de Jonathan]}
\Chap{14}
\VerseOne{}Jonathan, fils de Saül, dit un jour au garçon qui portait ses armes : Viens et allons jusqu’au poste de garde des Philistins qui est au-delà de ce lieu-là ; mais il ne dit rien à son père.
\VS{2}Saül se tenait à l'extrémité de Guibea sous un grenadier, à Migron, entouré d'environ six cents hommes.
\VS{3}Achija, fils d'Achithub, frère d'I-Kabnod, fils de Phinées, fils d'Eli, sacrificateur de Yahweh à Silo, portait l'éphod ; et le peuple ignorait que Jonathan s'en était allé.
\VS{4}Or entre les passages par lesquels Jonathan voulait arriver au poste de garde des Philistins, il y avait une dent de rocher d’un côté, et une dent de rocher de l’autre ; l'une s’appelait Botsets et l'autre Séné.
\VS{5}L'une de ces dents était située du côté nord vis-à-vis de Micmasch ; et l'autre, du côté sud vis-à-vis de Guéba.
\VS{6}Jonathan dit au garçon qui portait ses armes : Viens, poursuivons jusqu’au poste de garde de ces incirconcis ; peut-être que Yahweh agira-t-il pour nous : car on ne saurait empêcher Yahweh de délivrer avec peu ou beaucoup de gens.
\VS{7}Et celui qui portait ses armes lui dit : Fais tout ce que tu as dans le cœur, vas-y , voici je serai avec toi où tu voudras.
\VS{8}Et Jonathan lui dit : Allons vers ces hommes, et montrerons-nous à eux.
\VS{9}S'ils nous disent : Attendez jusqu'à ce que nous venions à vous, alors nous resterons sur place, et nous ne monterons pas vers eux.
\VS{10}Mais s'ils disent : Montez vers nous, nous irons ; car Yahweh les aura livrés entre nos mains. Que cela soit pour nous un signe.
\VS{11}Ils se montrèrent donc tous deux au poste de garde des Philistins, et les Philistins dirent : Voici, les Hébreux sortent des trous où ils s'étaient cachés.
\VS{12}Et ceux du poste de garde dirent à Jonathan, et à celui qui portait ses armes : Montez vers nous, nous avons quelque chose à vous apprendre. Alors Jonathan dit à celui qui portait ses armes : Monte avec moi ; car Yahweh les a livrés entre les mains d'Israël.
\VS{13}Et Jonathan monta en s’aidant des mains et des pieds ; celui qui portait ses armes le suivit. Puis ceux du poste de garde tombèrent sous les coups de Jonathan, et celui qui portait ses armes les tuait à sa suite.
\VS{14}Dans cette première victoire, Jonathan et celui qui portait ses armes, tuèrent environ vingt hommes, dans un espace d'environ une moitié d’un arpent de terre.
\VS{15}Et il y eut un grand effroi au camp, à la campagne, et parmi tout le peuple ; le poste de garde aussi, et ceux qui avaient ravagé furent effrayés et le pays fut tellement troublé que cela fut comme une frayeur de Dieu.
\TextTitle{[Victoire d'Israël]}
\VS{16}Les sentinelles de Saül qui étaient à Guibea de Benjamin virent que la multitude se dispersait et courait éperdue.
\VS{17}Alors Saül dit au peuple qui était avec lui : Faites donc la revue et voyez qui s’en est allé du milieu de nous. Ils firent donc la revue, et voici Jonathan n'y était pas, ni celui qui portait ses armes.
\VS{18}Et Saül dit à Achija : Fais approcher l'arche de Dieu ; - car l'arche de Dieu était en ce jour-là avec les enfants d'Israël.-
\VS{19}Pendant que Saül parlait au sacrificateur, le tumulte venant du camp des Philistins augmentait de plus en plus ; et Saül dit au sacrificateur : Retire ta main !
\VS{20}Saül et tout le peuple se rassemblèrent ; et vinrent au champ de bataille, les Philistins tournaient les épées les uns contre les autres, la confusion était extrême.
\VS{21}Les Hébreux, qui étaient montés auparavant dans le camp des Philistins et qui étaient dispersés, se joignirent aux Israëlites qui étaient avec Saül et Jonathan.
\VS{22}Et tous les Israëlites qui s'étaient cachés dans la montagne d'Ephraïm, ayant appris que les Philistins s'enfuyaient, les poursuivirent aussi pour les combattre.
\VS{23}Ce jour-là, Yahweh délivra Israël, et le combat s’étendit jusqu'à Beth-Aven.
\TextTitle{[Jonathan épargné des conséquences du vœu de Saül]}
\VS{24}Les hommes d'Israël furent épuisés cette journée-là. Mais Saül avait fait jurer le peuple, en disant : Maudit soit l'homme qui prendra de la nourriture avant le soir, avant que je me sois vengé de mes ennemis ! Et le peuple n’avait pris de pain.
\VS{25}Tout le peuple arriva dans une forêt, où il y avait du miel à la surface du sol.
\VS{26}Lorsque le peuple entra dans la forêt, il vit le miel qui coulait, mais nul ne porta la main à sa bouche ; car le peuple craignait le serment.
\VS{27}Or Jonathan n'avait pas entendu son père lorsqu'il avait fait faire le serment au peuple, il étendit le bout du bâton qu'il avait à la main, le trempa dans un rayon de miel et porta sa main à sa bouche et ses yeux furent éclaircis.
\VS{28}Alors quelqu'un du peuple lui dit : Ton père a fait jurer le peuple en disant : Maudit soit l'homme qui mangera aujourd'hui quelque chose ; quoique le peuple soit très fatigué.
\VS{29}Et Jonathan dit : Mon père trouble le peuple ; voyez comment mes yeux sont éclaircis après avoir goûté un peu de ce miel ;
\VS{30}combien plus si le peuple s’était aujourd'hui restauré du butin de ses ennemis ; la défaite des Philistins n'en aurait-elle pas été plus considérable ?
\VS{31}En ce jour-là donc ils frappèrent les Philistins de Micmasch à Ajalon. Le peuple était très fatigué.
\VS{32}Puis il se jeta sur le butin, il prit des brebis, des bœufs, et des veaux, et les égorgea sur la terre ; et le peuple les mangeait avec le sang.
\VS{33}On le rapporta à Saül, en disant : Voici, le peuple pèche contre Yahweh, en mangeant, avec le sang ; et il dit : Vous avez péché, roulez-moi ici une grosse pierre.
\VS{34}Allez parmi le peuple, ajouta-t-il, et dites à chacun d’amener son bœuf et ses brebis ; vous les égorgerez ici, vous les mangerez, et vous ne pécherez plus contre Yahweh, en mangeant avec le sang. Et chacun amena cette nuit-là son bœuf à la main, et ils les égorgèrent.
\VS{35}Saül bâtit un autel à Yahweh ; ce fut le premier autel qu'il bâtit à Yahweh.
\VS{36}Puis Saül dit : Descendons et poursuivons de nuit les Philistins, afin de les piller jusqu'au matin, et n’en laissons pas un homme de reste. Ils lui répondirent : Fais tout ce qui te semble bon ; mais le sacrificateur dit : Approchons-nous d’abord de Dieu.
\VS{37}Saül consulta donc Dieu : Descendrai-je à la poursuite des Philistins ? Les livreras-tu entre les mains d'Israël ? Mais il ne lui répondit pas.
\VS{38}Et Saül dit : Approchez ici, vous tous les chefs du peuple, recherchez et voyez par qui ce péché est arrivé aujourd'hui.
\VS{39}Car Yahweh est vivant, lui qui délivre Israël, quand il s’agirait de mon fils Jonathan, il en mourrait. Mais du peuple, personne ne répondit.
\VS{40}Puis il dit à tout Israël : Mettez-vous d'un côté, et nous serons de l'autre, moi et mon fils, Jonathan. Le peuple répondit à Saül : Fais ce qui te semble bon.
\VS{41}Et Saül dit à Yahweh, le Dieu d'Israël : Fais connaître la vérité. Jonathan et Saül furent désignés; et le peuple fut écarté.
\VS{42}Et Saül dit : Jetez le sort entre moi et Jonathan, mon fils. Et Jonathan fut désigné.
\VS{43}Alors Saül dit à Jonathan : Déclare-moi ce que tu as fait. Et Jonathan lui déclara et dit : Il est vrai que j'ai goûté un peu de miel avec le bout de mon bâton que j'avais à la main ; me voici, je mourrai.
\VS{44}Et Saül dit : Que Dieu agisse à mon égard comme il le veut, si tu ne meurs pas, Jonathan.
\VS{45}Mais le peuple dit à Saül : Jonathan qui a accompli cette grande délivrance en Israël, mourrait-il ? Garde-toi bien ! Yahweh est vivant, il ne tombera pas à terre un seul des cheveux de sa tête ; car c’est avec Dieu qu’il a agi en ce jour. Le peuple délivra Jonathan de la mort.
\VS{46}Saül renonça à poursuivre les Philistins, qui regagnèrent leur pays.
\TextTitle{[Les guerres sous le règne de Saül]}
\VS{47}Après que Saül eut pris possession de la royauté sur Israël, il fit la guerre de tous côtés contre ses ennemis, Moab, les enfants d’Ammon, Edom, les rois de Tsoba et les Philistins ; partout où il se tournait, il était vainqueur.
\VS{48}Il manifesta sa puissance en frappant Amalek et délivra Israël de la main de ceux qui le pillaient.
\VS{49}Les fils de Saül étaient Jonathan, Jischvi et Malkischua ; et quant aux noms de ses deux filles, le nom de l'aînée était Mérab, et la plus jeune, Mical.
\VS{50}Et le nom de la femme de Saül était Achinoam, fille d'Achimaats ; et le nom du chef de son armée était Abner, fils de Ner, oncle de Saül.
\VS{51}Kis, père de Saül, et Ner père d'Abner étaient fils d'Abiel.
\VS{52}La guerre contre les Philistins fut violente durant toute la vie de Saül ; et chaque fois que Saül remarquait un homme fort et vaillant, il le prenait auprès de lui.
\TextTitle{[Saül désobéit à Yahweh]}
\Chap{15}
\VerseOne{}Samuel dit à Saül : Yahweh m'a envoyé pour t'oindre afin que tu sois roi sur son peuple, sur Israël ; maintenant donc, écoute les paroles de Yahweh.
\VS{2}Ainsi parle Yahweh des armées : Je me rappelle de ce qu'Amalek a fait à Israël, comment il s'opposa à lui sur le chemin, à sa sortie d'Egypte.
\VS{3}Va maintenant, et frappe Amalek, et dévouez par interdit tout ce qui lui appartient ; ne l’épargne pas, mais fais mourir hommes et femmes, enfants et nourrissons, bœufs et menu bétail, chameaux et ânes.
\VS{4}Saül donc convoqua le peuple, et en fit la revue à Thelaïm, il y avait deux cent mille hommes de pied, et de dix mille hommes de Juda.
\VS{5}Et Saül marcha jusqu'à la ville d’Amalek, et mit une embuscade dans la vallée.
\VS{6}Et Saül dit aux Kéniens : Allez retirez-vous, séparez-vous des Amalécites, de peur que je ne vous détruise avec eux ; car vous avez agi avec bonté envers tous les enfants d'Israël, quand ils montèrent d'Egypte. Et les Kéniens se séparèrent des Amalécites.
\VS{7}Et Saül frappa les Amalécites depuis Havila jusqu'à Schur, qui est face à l'Egypte.
\VS{8}Il fit passer tout le peuple au fil de l'épée, le dévouant par interdit ; mais il épargna Agag, roi d'Amalek.
\VS{9}Saül et le peuple épargnèrent Agag, les meilleures brebis, les meilleurs boeufs, les bêtes grasses, les agneaux, ce qu’il y avait de meilleur ; ils ne voulurent pas les dévouer par interdit ; détruisant seulement tout ce qui est chétif et méprisable.
\VS{10}Alors la parole de Yahweh fut adressée à Samuel en disant :
\VS{11}Je me repens d'avoir établi Saül pour roi car il s’est détourné de moi et n'a pas exécuté mes paroles. Samuel fut très irrité, et il cria à Yahweh toute la nuit.
\TextTitle{[Samuel annonce à Saül que Yahweh le rejette]}
\VS{12}Puis Samuel se leva de bon matin pour aller rencontrer Saül. On lui rapporta que Saül venu à Carmel, s'est érigé un monument, puis s'en est retourné, pour enfin descendre à Guilgal.
\VS{13}Samuel se rendit auprès de Saül, et Saül lui dit : Sois béni de Yahweh ! J’ai exécuté la parole de Yahweh.
\VS{14}Samuel dit : Quel est donc ce bêlement de brebis qui parvient à mes oreilles, et ce mugissement de bœufs que j'entends ?
\VS{15}Et Saül répondit : Ils les ont amenés de chez les Amalécites ; car le peuple a épargné les meilleures brebis et les meilleurs bœufs, pour les sacrifier à Yahweh, ton Dieu ; et nous avons détruit le reste, nous l’avons dévoué par interdit.
\VS{16}Samuel dit à Saül : Laisse-moi te déclarerai ce que Yahweh m'a dit cette nuit ; et il lui répondit : Parle !
\VS{17}Samuel dit : N'est-il pas vrai que, quand tu étais petit à tes yeux, tu as été fait chef des tribus d'Israël, et Yahweh t'a oint pour roi sur Israël ?
\VS{18}Yahweh t'avait envoyé dans cette expédition, et t'avait dit : Va, et détruis ces pécheurs, les Amalécites, et fais-leur la guerre, jusqu'à ce qu'ils soient exterminés.
\VS{19}Pourquoi n'as-tu pas obéi à la voix de Yahweh, tu t'es jeté sur le butin, et as fait ce qui déplaît à Yahweh ?
\VS{20}Et Saül répondit à Samuel : J'ai pourtant obéi à la voix de Yahweh, et je suis allé par le chemin par lequel Yahweh m'a envoyé, et j'ai amené Agag, roi des Amalécites, et j'ai dévoué les Amalécites, par interdit.
\VS{21}Mais le peuple a pris des brebis, des bœufs, du butin, comme prémices de ce qui devait être dévoué, pour le sacrifier à Yahweh, ton Dieu à Guilgal.
\VS{22}Samuel répondit : Yahweh prend-il plaisir aux holocaustes et aux sacrifices, autant qu’à l’obéissance à sa voix ? Voici, l'obéissance vaut mieux que les sacrifices, et l’observation de sa parole vaut mieux que la graisse des béliers.
\VS{23}Car la rébellion est un péché autant que la divination, et la résistance ne l’est pas moins que l’idolâtrie et les théraphim. Puisque tu as rejeté la parole de Yahweh, il te rejette aussi afin que tu ne sois plus roi.
\VS{24}Et Saül répondit à Samuel : J'ai péché parce que j'ai transgressé le commandement de Yahweh, ainsi que tes paroles ; car je craignais le peuple et j'ai obéi à sa voix.
\VS{25}Mais maintenant, je te prie, pardonne-moi mon péché, et reviens avec moi, que je me prosterne devant Yahweh.
\VS{26}Et Samuel dit à Saül : Je n’irai pas avec toi ; parce que tu as rejeté la parole de Yahweh, Yahweh te rejette afin que tu sois plus roi d’Israël.
\VS{27}Comme Samuel se détournait pour s'en aller, Saül le saisit par le pan de son manteau qui se déchira.
\VS{28}Alors Samuel lui dit : Yahweh déchire aujourd'hui le royaume d'Israël de dessus toi, et le donne à un autre, qui est meilleur que toi.
\VS{29}En effet, le Puissant d'Israël ne ment pas, il ne se repent pas ; car il n'est pas un homme pour se repentir.
\VS{30}Et Saül répondit : J'ai péché ; mais honore-moi maintenant, je te prie, en présence des anciens de mon peuple, et en présence d'Israël, et reviens avec moi, et je me prosternerai devant Yahweh ton Dieu.
\VS{31}Samuel retourna et suivit Saül ; et Saül se prosterna devant Yahweh.
\VS{32}Puis Samuel dit : Amenez-moi Agag, roi d'Amalek. Et Agag s’avança vers lui, faisant le gracieux ; car Agag disait : certainement l'amertume de la mort est passée.
\VS{33}Mais Samuel dit : Comme ton épée a privé les femmes de leurs enfants, ainsi ta mère entre les femmes sera privée d'enfants. Et Samuel mit Agag en pièces devant Yahweh à Guilgal.
\VS{34}Puis il s'en alla à Rama ; et Saül monta dans sa maison à Guibea de Saül.
\VS{35}Et Samuel n'alla plus voir Saül jusqu'au jour de sa mort ; car Samuel pleurait sur Saül, de ce que Yahweh s'était repenti d'avoir établi Saül, roi sur Israël.
\TextTitle{[Yahweh envoie Samuel à Bethléhem pour oindre David]}
\Chap{16}
\VerseOne{}Yahweh dit à Samuel : Jusqu'à quand mèneras-tu deuil sur Saül, vu que je l'ai rejeté, afin qu'il ne règne plus sur Israël ? Remplis ta corne d'huile, et viens ; je t’enverrai chez Isaï, Bethléhémite ; car je me suis pourvu d'un de ses fils pour roi.
\VS{2}Et Samuel dit : Comment irai-je ? Car Saül l’apprendra et il me tuera. Et Yahweh répondit : Tu emmèneras avec toi une jeune vache du troupeau ; et tu diras : Je suis venu pour sacrifier à Yahweh.
\VS{3}Et tu inviteras Isaï au sacrifice, et je te ferai savoir ce que tu auras à faire, et tu m'oindras celui que je te dirai.
\VS{4}Samuel fit donc comme Yahweh lui avait dit, et il alla à Bethléhem. Les anciens de la ville tout effrayés accoururent au-devant de lui et lui dirent : Ton arrivée annonce-t-elle la paix ?
\VS{5}Et il répondit : Soyez en paix ; je suis venu pour sacrifier à Yahweh, sanctifiez-vous, et venez avec moi au sacrifice. Il fit sanctifier aussi Isaï et ses fils, et les invita au sacrifice.
\VS{6}A son entrée, il remarqua Eliab, et se dit : L'oint de Yahweh est certainement devant lui.
\VS{7}Mais Yahweh dit à Samuel : Ne prête pas attention à son apparence, ni à la hauteur de sa taille, car je l'ai rejeté ; Yahweh ne considère pas ce que l'homme considère ; car l'homme considère ce que voient ses yeux ; mais Yahweh regarde au cœur.
\VS{8}Isaï appela Abinadab, et le fit passer devant Samuel, et Samuel dit : Yahweh n'a pas non plus choisi celui-ci.
\VS{9}Isaï fit passer Schamma, et Samuel dit : Yahweh n'a pas non plus choisi celui-ci.
\VS{10}Ainsi Isaï fit passer ses sept fils devant Samuel et Samuel dit à Isaï : Yahweh n’a pas choisi ceux-ci.
\VS{11}Puis Samuel dit à Isaï : Sont-ce là tous tes garçons? Et il dit : Il reste encore le plus jeune, seulement, il fait paître les brebis. Alors Samuel dit à Isaï : Envoie-le chercher ; car nous ne retournerons pas avant qu’il ne soit venu ici.
\VS{12}Il le fit donc venir. Il était roux, avec de beaux yeux et une belle apparence. Et Yahweh dit à Samuel : Lève-toi, et oins-le ; car c'est lui !
\VS{13}Alors Samuel prit la corne d'huile, et l'oignit au milieu de ses frères ; et depuis ce jour-là l'Esprit de Yahweh saisit David. Et Samuel se leva, et s'en alla à Rama.
\TextTitle{[David chez Saül]}
\VS{14}L'Esprit de Yahweh se retira de Saül, et un mauvais esprit\FTNT{Saül a été frappé d’un esprit d’égarement (2 Th. 2:9-12)} envoyé par Yahweh le terrifiait.
\VS{15}Les serviteurs de Saül lui dirent : Voici, un mauvais esprit envoyé de Dieu te tourmente.
\VS{16}Que le roi notre seigneur parle ! Tes serviteurs sont devant toi. Ils chercheront un homme qui sache jouer de la harpe ; et quand le mauvais esprit envoyé par Dieu sera sur toi, il jouera de sa main, et tu seras soulagé.
\VS{17}Saül répondit à ses serviteurs : Trouvez-moi un homme qui sache bien jouer et amenez-le-moi.
\VS{18}L'un des serviteurs répondit : Voici, j'ai vu l’un des fils d'Isaï, le Bethléhémite, qui sait jouer des instruments, il est fort et vaillant, c’est un guerrier qui parle bien, bel homme, et Yahweh est avec lui.
\VS{19}Alors Saül envoya des messagers à Isaï, pour lui dire : Envoie-moi David, ton fils, qui est avec les brebis.
\VS{20}Isaï prit un âne, qu’il chargea de pain, et une outre de vin, et un jeune chevreau, et les envoya par David, son fils, à Saül.
\VS{21}David arrivé chez Saül, se présenta devant lui ; et Saül l'aima beaucoup, et il lui servit à porter ses armes.
\VS{22}Saül fit dire à Isaï : Je te prie que David demeure à mon service ; car il a trouvé grâce devant moi.
\VS{23}Il arrivait donc que quand le mauvais esprit envoyé de Dieu, était sur Saül, David prenait la harpe, et en jouait de sa main ; et Saül en était soulagé, parce que le mauvais esprit se retirait de lui.
\TextTitle{[Goliath défie Israël]}
\Chap{17}
\VerseOne{}Les Philistins réunirent leurs armées pour faire la guerre, et ils se rassemblèrent à Soco, qui est de Juda ; et ils campèrent entre Soco et Azéka, à Ephès-Dammim.
\VS{2}Saül et ceux d'Israël se rassemblèrent aussi ; et ils campèrent dans la vallée du chêne, et ils se mirent en ordre de bataille contre les Philistins.
\VS{3}Les Philistins étaient sur une montagne d’un côté, et les Israëlites sur une montagne de l’autre côté ; de sorte que la vallée les séparait.
\VS{4}Il sortit du camp des Philistins un homme qui se présentait entre les deux armées, il s’appelait Goliath, de la ville de Gath, haut de six coudées et d'un empan.
\VS{5}Il avait un casque d'airain sur sa tête, et était armé d'une cuirasse à écailles pesant cinq mille sicles d'airain.
\VS{6}Il avait aussi des jambières d'airain, et un javelot d'airain entre ses épaules.
\VS{7}Le bois de sa lance était comme une ensouple d'un tisserand, et le fer de sa lance pesait six cents sicles de fer. Celui qui portait son bouclier marchait devant lui.
\VS{8}Il se présenta donc, et cria aux troupes d'Israël rangées en bataille, il leur disait : Pourquoi sortez-vous pour vous ranger en bataille ? Ne suis-je pas Philistin, et n'êtes-vous pas esclaves de Saül ? Choisissez l'un d'entre vous, et qu'il descende contre moi.
\VS{9}S’il peut me battre et qu'il me tue, nous serons vos esclaves ; mais si j'ai l'avantage sur lui, et que je le tue, vous serez nos esclaves, et vous nous serez asservis.
\VS{10}Le Philistin disait : Je jette un défi en ce jour aux troupes rangées d'Israël : Donnez-moi un homme, et nous combattrons ensemble.
\VS{11}Saül et tous les Israëlites ayant entendu les paroles du Philistin furent épouvantés et saisis d’une grande frayeur.
\VS{12}David, était le fils d'un homme Ephratien, de Bethléhem de Juda, nommé Isaï, qui avait huit fils, et qui du temps de Saül, était vieux et mis au rang de personnes de qualité.
\VS{13}Et les trois fils aînés d'Isaï avaient suivi Saül à la guerre. Les noms de ses trois fils qui s'en étaient allés à la guerre, étaient Eliab, le premier-né ; Abinadab, le second ; et Schamma, le troisième.
\VS{14}David était le plus jeune, et les trois plus grands suivaient Saül.
\VS{15}David allait et revenait d'auprès de Saül, pour paître les brebis de son père à Bethléhem.
\VS{16}Et le Philistin s'approchant le matin et le soir, se présenta pendant quarante jours.
\TextTitle{[David veut combattre contre Goliath]}
\VS{17}Isaï dit à David, son fils : Prends maintenant pour tes frères un épha de ce blé rôti, et ces dix pains, et porte-les promptement au camp, à tes frères.
\VS{18}Tu porteras aussi ces dix fromages au chef de leur millier, tu t’informeras du bien-être de tes frères et tu m'en apporteras des nouvelles sûres.
\VS{19}Or Saül était avec eux et les hommes d'Israël, combattant les Philistins dans la vallée du chêne.
\VS{20}David se leva de bon matin, et laissa les brebis aux soins d’un gardien ; puis ayant pris sa charge, s'en alla, comme son père Isaï le lui avait ordonné. Lorsqu’il arriva au lieu où était le camp, l'armée sortait pour se ranger en bataille, et on poussait des cris de guerre.
\VS{21}Car les Israëlites et les Philistins se rangèrent armée contre armée.
\VS{22}Alors David se déchargea de son bagage, le laissant entre les mains de celui qui gardait le bagage, et courut vers les rangs de l’armée. Aussitôt arrivé, il demanda à ses frères s'ils se portaient bien.
\VS{23}Et comme il parlait avec eux, le Philistin de Gath, nommé Goliath, sortit des rangs de l'armée des Philistins, se présenta entre les deux armées et proféra les mêmes paroles qu'il avait proférées auparavant et David les entendit.
\VS{24}A la vue de cet homme, tous ceux d'Israël s'enfuirent devant lui, saisis d’une grande frayeur.
\VS{25}Et les Israélites disaient : Avez-vous vu s’avancer cet homme ? Il est monté pour jeter un défi à Israël, mais si quelqu'un le tue, le roi le comblera de richesses, et lui donnera sa fille, et affranchira la maison de son père en Israël.
\VS{26}Alors David parla aux personnes qui étaient là avec lui, en disant : Quel bien fera-t-on à l'homme qui frappera ce Philistin, et qui ôtera l'opprobre de dessus Israël ? Car qui est ce Philistin, cet incirconcis, pour insulter l’armée du Dieu vivant ?
\VS{27}Et le peuple lui répéta ces mêmes paroles et lui dit : C'est le bien qu'on fera à l'homme qui l'aura tué.
\VS{28}Et quand Eliab son frère aîné entendit qu'il parlait à ces personnes, sa colère s'enflamma contre David, et il lui dit : Pourquoi es-tu descendu, et à qui as-tu laissé ce peu de brebis au désert ? Je connais ton orgueil et la malice de ton cœur, car tu es descendu pour voir la bataille.
\VS{29}Et David répondit : Qu'ai-je donc fait ? Ne puis-je pas parler ainsi ?
\VS{30}Puis il se détourna de lui vers un autre, et lui posa les mêmes questions ; et le peuple lui répondit comme la première fois.
\VS{31}Les paroles que David avait dites furent entendues et rapportées devant Saül qui le fit venir.
\VS{32}David dit à Saül : Que personne ne perde courage à cause de ce Philistin ! Ton serviteur ira et se battra contre lui.
\VS{33}Mais Saül dit à David : Tu ne peux aller te battre contre ce Philistin, car tu n'es qu'un enfant, et il est un homme de guerre depuis sa jeunesse.
\VS{34}David répondit à Saül : Ton serviteur faisait paître les brebis de son père, quand un lion ou un ours venait emporter une brebis du troupeau,
\VS{35}je le poursuivais, je le frappais, et j’arrachais la brebis de la gueule, s’il se jetait sur moi, je le saisissais par la mâchoire, je le frappais, et je le tuais.
\VS{36}Ton serviteur a tué et le lion, et l’ours ; et ce Philistin, cet incirconcis, sera comme l'un d'eux ; car il a déshonoré l’armée du Dieu vivant.
\VS{37}David dit encore : Yahweh qui m'a délivré de la griffe du lion, et de la patte de l'ours, me délivrera de la main de ce Philistin. Alors Saül dit à David : Va, et que Yahweh soit avec toi.
\TextTitle{[David tue Goliath]}
\VS{38}Saül fit revêtir David de ses vêtements, et lui mit son casque d'airain sur sa tête, et lui fit endosser une cuirasse.
\VS{39}Puis David ceignit l'épée par-dessus ses vêtements, et voulut marcher, car il n’avait pas encore essayé. Et David dit à Saül : Je ne saurais marcher ainsi, je ne l’ai jamais essayé. Et il s’en débarrassa.
\VS{40}Alors il prit en main son bâton, et se choisit dans le torrent cinq pierres bien polies, et les mit dans sa mallette de berger et dans sa poche, puis sa fronde en main, il s'approcha du Philistin.
\VS{41}Le Philistin aussi s'approcha lentement de David, précédé de l'homme qui portait son bouclier.
\VS{42}Le Philistin regarda, et lorsqu’il vit David, il le méprisa, car ce n'était qu'un jeune garçon, roux et beau de figure.
\VS{43}Le Philistin dit à David : Suis-je un chien, pour que tu viennes contre moi avec des bâtons ? Et le Philistin maudit David par ses dieux.
\VS{44}Le Philistin ajouta : Viens vers moi et je donnerai ta chair aux oiseaux du ciel, et aux bêtes des champs.
\VS{45}Et David dit au Philistin : Tu marches contre moi avec l'épée, la lance, et le javelot ; mais moi, je marche contre toi au Nom de Yahweh des armées, le Dieu de l’armée d'Israël, que tu as blasphémé.
\VS{46}Aujourd'hui Yahweh te livrera entre mes mains, je t’abattrai, je te couperai la tête ; aujourd'hui je donnerai les cadavres du camp des Philistins aux oiseaux du ciel, et aux animaux de la terre ; et toute la terre saura qu'Israël a un Dieu.
\VS{47}Et toute cette assemblée saura que Yahweh ne délivre pas par l'épée ni par la lance ; car la victoire est à Yahweh, qui vous livrera entre nos mains.
\VS{48}Voyant le Philistin se mettre en mouvement et s'approcher de lui, David s’élança, et courut au milieu du champ de bataille en direction du Philistin.
\VS{49}Il mit la main à sa mallette, prit une pierre, et la lança avec sa fronde ; il frappa le Philistin au front, tellement que la pierre s'enfonça dans son front, il tomba le visage contre terre.
\VS{50}Ainsi avec une fronde et une pierre, David fut plus fort que le Philistin, il le frappa, et le tua, sans avoir une épée à la main.
\VS{51}Alors David courut, se jeta sur le Philistin, prit son épée, la tira de son fourreau, le tua, et lui coupa la tête. Les Philistins, voyant que leur héros était mort, prirent la fuite.
\VS{52}Alors les hommes d'Israël et de Juda se levèrent, et poussèrent des cris de joie, et poursuivirent les Philistins, jusqu'à la vallée, et jusqu'aux portes d’Ekron. Les Philistins blessés à mort tombèrent dans le chemin de Schaaraïm, jusqu'à Gath, et jusqu'à Ekron.
\VS{53}Et les enfants d'Israël revinrent de la poursuite des Philistins, et pillèrent leurs camps.
\VS{54}David prit la tête du Philistin et la porta à Jérusalem, et il mit aussi dans sa tente les armes du Philistin.
\VS{55}Quand Saül vit David sortant à la rencontre du Philistin, il dit à Abner, chef de l'armée : Abner, de qui ce jeune homme est-il le fils ? Abner répondit : Que ton âme vive, ô roi! Je n'en sais rien.
\VS{56}Le roi lui dit : Informe-toi de qui ce jeune garçon est fils.
\VS{57}Et quand David fut de retour après avoir tué le Philistin, Abner le prit, et le mena devant Saül. David avait la tête du Philistin à la main.
\VS{58}Et Saül lui dit : Jeune garçon, de qui es-tu fils ? David répondit : Je suis fils d'Isaï Bethléhémite, ton serviteur.
\TextTitle{[Jonathan et David font alliance]}
\Chap{18}
\VerseOne{}Dès que David eut achevé de parler à Saül, l'âme de Jonathan fut attachée à l'âme de David, et Jonathan l'aima comme son âme.
\VS{2}Ce jour-là donc Saül le retint, et ne lui permit plus de retourner à la maison de son père.
\VS{3}Alors Jonathan fit alliance avec David, parce qu'il l'aimait comme son âme.
\VS{4}Jonathan se dépouilla du manteau qu'il portait, et le donna à David, avec ses habits, jusqu'à son épée, son arc, et sa ceinture.
\TextTitle{[Saül jaloux veut tuer David]}
\VS{5}David envoyé par Saül, réussissait partout où il allait, de sorte que Saül l'établit sur son armée, et il plaisait à tout le peuple, même aux serviteurs de Saül.
\VS{6}Or quand ils rentraient, lors du retour de David après qu’il eut tué le Philistin, des femmes sortirent de toutes les villes d'Israël, en chantant et dansant devant le roi Saül, avec des tambourins, des triangles et en poussant des cris de joie.
\VS{7}Les femmes chantaient, se répondant les unes aux autres, en disant : Saül a frappé ses mille, et David ses dix mille.
\VS{8}Saül fut très irrité, car cette parole lui déplut. Il dit : Elles en ont donné dix mille à David, et à moi, mille ! Il ne lui manque plus que le royaume.
\VS{9}Depuis ce jour-là, Saül regardait David d’un mauvais œil.
\VS{10}Dès le lendemain, le mauvais esprit envoyé de Dieu saisit Saül qui prophétisait dans sa maison, et David joua de sa main, comme les autres jours, et Saül avait une lance à la main.
\VS{11}Saül jeta sa lance, se disant : Je frapperai David, contre le mur ; mais David l’évita deux fois.
\VS{12}Saül craignait la présence de David, parce que Yahweh était avec David, et qu'il s'était retiré de Saül.
\VS{13}C'est pourquoi Saül éloigna David de lui, et l'établit chef de mille ; et David allait et venait devant le peuple.
\VS{14}David réussissait dans tout ce qu'il entreprenait, car Yahweh était avec lui.
\VS{15}Saül, voyant que David réussissait beaucoup, avait peur de sa présence.
\VS{16}Mais tout Israël et Juda aimaient David, parce qu'il allait et venait devant eux.
\TextTitle{[David épouse Mérab]}
\VS{17}Saül dit à David : Voici, je te donnerai Mérab, ma fille aînée pour femme ; sois pour moi un fils vaillant, et conduis les guerres de Yahweh ; car Saül disait : Que ma main ne le touche pas, mais que ce soit celle des Philistins.
\VS{18}David répondit à Saül : Qui suis-je, et quelle est ma vie, et la famille de mon père en Israël, pour que je devienne gendre du roi ?
\VS{19}Or, au temps où l’on devait donner Mérab fille de Saül à David, elle fut donnée pour femme à Adriel de Mehola.
\VS{20}Mais Mical, fille de Saül, aima David ; ce qu'on rapporta à Saül, et la chose lui plut.
\VS{21}Et Saül dit : Je la lui donnerai afin qu'elle soit pour lui un piège, et que par ce moyen la main des Philistins l’atteigne. Saül donc dit à David pour la seconde fois : Tu seras aujourd'hui mon gendre.
\VS{22}Et Saül ordonna à ses serviteurs de parler à David en secret, et de lui dire : Voici, le roi prend plaisir en toi, et tous ses serviteurs t'aiment ; sois donc maintenant gendre du roi.
\VS{23}Les serviteurs de Saül répétèrent toutes ces paroles à David, et David répondit : Pensez-vous qu’il soit facile de devenir le gendre du roi, moi qui suis un homme pauvre, et peu important ?
\VS{24}Et les serviteurs de Saül lui rapportèrent ce que David avait répondu.
\VS{25}Saül dit : Vous parlerez ainsi à David : Le roi ne désire pas de dot, mais cent prépuces de Philistins, afin d’être vengé de ses ennemis. Or Saül avait pour but de faire tomber David aux mains des Philistins.
\VS{26}Les serviteurs de Saül rapportèrent tous ces discours à David à qui il plut de devenir gendre du roi. Le temps n’était pas encore écoulé,
\VS{27}que David se leva, et s'en alla, lui et ses gens, et tua deux cents hommes parmi les Philistins ; il apporta leurs prépuces, et on les livra au complet au roi, afin qu'il devienne gendre du roi. Alors Saül lui donna pour femme Mical, sa fille.
\VS{28}Saül vit et comprit que Yahweh était avec David et Mical, fille de Saül l'aimait.
\VS{29}Saül craignait David de plus en plus, et devint son ennemi toute sa vie durant.
\VS{30}Les chefs des Philistins firent des incursions, mais chaque fois qu’ils sortaient, David remportait du succès mieux que tous les serviteurs de Saül et son nom devint célèbre.
\TextTitle{[David échappe aux assauts de Saül]}
\Chap{19}
\VerseOne{}Saül parla à Jonathan, son fils, et à tous ses serviteurs de faire mourir David.
\VS{2}Mais Jonathan, fils de Saül, avait une grande affection pour David. C'est pourquoi Jonathan le fit savoir à David, et lui dit : Saül, mon père, cherche à te faire mourir ; maintenant donc, tiens-toi sur tes gardes jusqu'au matin, demeure dans un lieu secret, et cache-toi.
\VS{3}Je me tiendrai auprès de mon père, je sortirai dans le champ où tu seras ; car je parlerai de toi à mon père ; je verrai ce qu'il en sera, et je te le rapporterai.
\VS{4}Jonathan parla favorablement de David à Saül, son père, et lui dit : Que le roi ne pèche pas contre son serviteur David, car il n'a pas péché contre toi ; au contraire, il a agi pour ton bien.
\VS{5}Car il a exposé sa vie, il a tué le Philistin, et Yahweh a opéré une grande délivrance pour tout Israël, tu l'as vu, et tu t'en es réjoui ; pourquoi donc pécherais-tu contre le sang innocent en faisant mourir David sans cause ?
\VS{6}Saül écouta la voix de Jonathan et jura : Yahweh est vivant, il ne mourra pas.
\VS{7}Alors Jonathan appela David, et lui répéta toutes ces choses. Jonathan l’introduisit auprès de Saül, et il fut à son service comme auparavant.
\VS{8}La guerre ayant recommencé, David se mit en campagne et frappa les Philistins, et leur infligea une grande défaite, de sorte qu'ils prirent la fuite.
\VS{9}Le mauvais esprit envoyé de Yahweh fut sur Saül, comme il était assis dans sa maison, ayant sa lance à la main, et David jouait de sa main.
\VS{10}Saül voulut frapper David avec sa lance contre le mur ; mais il se glissa de devant Saül, qui frappa le mur de la lance, David s'enfuit et s'échappa cette nuit-là.
\VS{11}Saül envoya des messagers à la maison de David pour le garder, et le faire mourir au matin. Mical, femme de David l’en informa, en disant : Si tu ne te sauves pas demain on te fera mourir.
\VS{12}Mical fit descendre David par une fenêtre, et ainsi il s'en alla et s'enfuit.
\VS{13}Ensuite Mical prit un théraphim, qu’elle plaça dans le lit ; elle mit une peau de chèvre à son chevet et l’enveloppa d'une couverture.
\VS{14}Lorsque Saül envoya des gens pour prendre David, elle dit : Il est malade.
\VS{15}Saül envoya encore des gens pour prendre David, en leur disant : Apportez-le-moi dans son lit, afin que je le fasse mourir.
\VS{16}Ces gens donc vinrent, et voici, un théraphim était au lit, et la peau de chèvre à son chevet.
\VS{17}Saül dit à Mical : Pourquoi m'as-tu trompé de la sorte, et as-tu laissé aller mon ennemi, de sorte qu'il s’est échappé ? Et Mical, répondit à Saül : Il m'a dit : Laisse-moi aller ou je te tue !
\VS{18}C’est ainsi que David prit la fuite et qu’il s’échappa. Il se rendit auprès de Samuel à Rama, et lui raconta tout ce que Saül lui avait fait. Puis il s'en alla avec Samuel, et ils demeurèrent à Najoth.
\VS{19}On le rapporta à Saül, en lui disant : Voici, David est à Najoth près de Rama.
\VS{20}Alors Saül envoya des gens pour s’emparer de David. Ils virent une assemblée de prophètes qui prophétisaient, et Samuel à leur tête, se tenait là. L'Esprit de Dieu saisit les envoyés de Saül, qui prophétisèrent aussi.
\VS{21}On le rapporta à Saül, qui envoya d'autres gens, et eux aussi prophétisèrent. Saül en envoya encore pour la troisième fois et ils prophétisèrent également.
\VS{22}Alors il alla lui-même à Rama. Arrivé à la grande citerne qui est à Sécou, il s'informa disant : Où sont Samuel et David ? Et on lui répondit : Ils sont à Najoth, près de Rama.
\VS{23}Il se dirigea vers Najoth près de Rama ; et l'Esprit de Dieu le saisit à son tour, et il continua son chemin en prophétisant, jusqu'à son arrivée à Najoth près de Rama.
\VS{24}Il se dépouilla lui aussi de ses vêtements et prophétisa devant Samuel ; et il se jeta à terre nu, tout ce jour-là et toute la nuit. C'est pourquoi on dit : Saül est-il aussi parmi les prophètes ?
\TextTitle{[David et Jonathan renouvellent leur serment]}
\Chap{20}
\VerseOne{}David s'enfuit de Najoth près de Rama. Il alla voir Jonathan et lui dit : Qu'ai-je fait ? Quelle est mon iniquité, et quel est mon péché devant ton père, pour qu'il en veuille à ma vie ?
\VS{2}Jonathan lui dit : Loin de là ! Tu ne mourras pas. Voici, mon père ne fait aucune chose, ni grande, ni petite, qu'il ne m’en informe ; pourquoi mon père me cacherait-il cette chose-là ? Il n'en est rien.
\VS{3}Alors David jurant, dit encore : Ton père sait certainement que j’ai trouvé grâce à tes yeux, et il aura dit : Que Jonathan ne sache rien de ceci, de peur qu'il n'en soit attristé ; mais Yahweh est vivant, et ton âme vit ! Qu'il n'y a qu'un pas entre moi et la mort.
\VS{4}Alors Jonathan dit à David : Que désires-tu que je fasse ? Et je le ferai pour toi.
\VS{5}Et David dit à Jonathan : Voici, c'est demain la nouvelle lune, et je devrais m'asseoir auprès du roi pour manger, laisse-moi donc aller et je me cacherai aux champs, jusqu'au troisième soir.
\VS{6}Si ton père me cherche, tu lui répondras : David m'a demandé la permission de courir Bethléhem sa ville, parce que toute sa famille fait un sacrifice annuel.
\VS{7}S'il dit ainsi : C’est bien ! Ton serviteur n’a rien à craindre. Mais s'il se met en colère, sache qu’il a résolu mon malheur.
\VS{8}Use donc de bonté envers ton serviteur, puisque tu as conclu une alliance avec ton serviteur devant Yahweh. S'il y a de l’iniquité en moi, tue-moi toi-même ; car pourquoi me mènerais-tu jusqu’à ton père ?
\VS{9}Jonathan lui dit : Loin de toi cette pensée ! Si je savais ta perte arrêtée dans la pensée de mon père, ne t’en informerais-je pas ?
\VS{10}David répondit à Jonathan : Qui m’avertira si la réponse que t'aura faite ton père est sévère ?
\VS{11}Et Jonathan dit à David : Viens et sortons dans les champs. Ils sortirent donc eux deux dans les champs.
\VS{12}Alors Jonathan dit à David : Par Yahweh, le Dieu d'Israël, je sonderai mon père demain, environ à cette heure ou après demain, et s’il est favorable envers David, et que je n'envoie personne vers toi pour t’en informer,
\VS{13}que Yahweh traite Jonathan dans toute sa rigueur ! Si mon père a résolu de te faire du mal, je t’en informerai, et je te laisserai aller, et tu t'en iras en paix, de sorte que Yahweh sera avec toi comme il a été avec mon père.
\VS{14}Si je vis encore, tu useras de la bonté de Yahweh envers moi, en sorte que je ne meure pas.
\VS{15}Ne retire jamais ta bonté de ma maison, pas même quand Yahweh retranchera tous les ennemis de David de dessus la surface de la terre.
\VS{16}Ainsi Jonathan traita alliance avec la maison de David, en disant : Que Yahweh tire vengeance des ennemis de David.
\VS{17}Jonathan se lia encore par serment à David pour l'amour qu'il lui portait ; car il l'aimait comme son âme.
\VS{18}Puis Jonathan lui dit : C'est demain la nouvelle lune, et on s'informera sur toi ; car ta place sera vide.
\VS{19}Le troisième jour au soir, tu descendras en hâte, jusqu’au fond du lieu où tu t’étais caché le jour de l’affaire et tu resteras près de la pierre d'Ezel.
\VS{20}Je tirerai trois flèches à côté de cette pierre, comme si je visais un but.
\VS{21}Et voici, j'enverrai un jeune homme, et je lui dirai : Va, trouve les flèches. Si je dis au jeune homme : Voici, les flèches sont au deçà de toi, prends-les ! Alors viens, car la paix est avec toi et tu n’as rien à craindre ; Yahweh est vivant.
\VS{22}Mais si je dis ainsi au jeune homme : Voici, les flèches sont au-delà de toi ; va-t'en, car Yahweh te renvoie.
\VS{23}Et quant à la parole que nous nous sommes donnée toi et moi ; voici, Yahweh est entre moi et toi à jamais.
\TextTitle{[Saül en colère contre Jonathan]}
\VS{24}David donc se cacha dans le champ. La nouvelle lune étant venue, le roi s'assit pour prendre son repas.
\VS{25}Et le roi s’assit à sa place, comme à l’ordinaire, sur son siège près du mur, Jonathan se leva, et Abner s'assit à côté de Saül ; mais la place de David resta vide.
\VS{26}Saül ne dit rien ce jour-là, car il se disait : Il lui est arrivé quelque chose ; il n'est pas pur, certainement il n'est pas pur.
\VS{27}Mais le lendemain, le second jour de la nouvelle lune, la place de David était encore vide. Et Saül dit à Jonathan, son fils : Pourquoi le fils d'Isaï n'a-t-il été ni hier ni aujourd'hui au repas ?
\VS{28}Et Jonathan répondit à Saül : David m'a instamment demandé la permission d’aller à Bethléhem.
\VS{29}Même il m'a dit : Je te prie, laisse-moi aller ; car notre famille fait un sacrifice dans la ville, et mon frère m'a ordonné de m'y trouver ; maintenant donc si je suis dans tes bonnes grâces, je te prie que j'y aille, afin que je voie mes frères. C'est pour cela qu'il n'est pas venu à la table du roi.
\VS{30}Alors la colère de Saül s'enflamma contre Jonathan et il lui dit : Fils perfide et rebelle, ne sais-je pas que tu as choisi le fils d'Isaï à ta honte et à la honte de ta mère ?
\VS{31}Car aussi longtemps que le fils d'Isaï sera vivant sur la terre, tu ne seras pas stable, ni toi, ni ta royauté ; c'est pourquoi maintenant amène-le-moi, car il est digne de mort.
\VS{32}Et Jonathan répondit à Saül son père, et lui dit : Pourquoi le ferait-on mourir ? Qu'a-t-il fait ?
\VS{33}Et Saül lança sa lance contre lui pour le frapper. Alors Jonathan reconnut que son père avait résolu la mort de David.
\VS{34}Jonathan se leva de table dans une ardente colère, et ne mangea pas le pain le deuxième jour de la nouvelle lune ; car il était affligé à cause de David, parce que son père l'avait insulté.
\VS{35}Le matin venu, Jonathan sortit dans les champs, au lieu convenu avec David, et il amena avec lui un petit garçon.
\VS{36}Et il dit à son garçon : Cours, trouve maintenant les flèches que je m'en vais tirer. Et le garçon courut, et Jonathan tira une flèche qui le dépassa.
\VS{37}Lorsque le garçon arriva au lieu où était la flèche que Jonathan avait tirée, Jonathan cria après lui, et lui dit : La flèche n'est-elle pas plus loin de toi ?
\VS{38}Jonathan cria encore après le garçon : Hâte-toi, ne t'arrête pas ; et le garçon ramassa les flèches, et revint vers son maître.
\VS{39}Le garçon ne savait rien de cette affaire ; seuls David et Jonathan le savaient.
\VS{40}Jonathan remit ses armes au garçon et lui dit : Va, porte-les à la ville.
\VS{41}Le garçon parti, David se leva du côté du midi, se jeta le visage contre terre et se prosterna à trois reprises. Ils s’embrassèrent et pleurèrent ensemble, David versa d’abondantes larmes.
\VS{42}Jonathan dit à David : Va en paix ; comme nous l’avons juré au nom de Yahweh, en disant : Que Yahweh soit entre moi et toi, entre ma postérité et ta postérité.
\VS{43}David donc se leva, s'en alla et Jonathan rentra dans la ville.
\TextTitle{[David s'enfuit]}
\Chap{21}
\VerseOne{}David se rendit à Nob, vers Achimélec le sacrificateur, qui tout effrayé courut au-devant de David, et lui dit : Pourquoi es-tu seul et n'y a-t-il personne avec toi ?
\VS{2}David répondit au sacrificateur Achimélec : Le roi m'a donné un ordre et m'a dit : Que personne ne sache rien de l'affaire pour laquelle je t'envoie, ni de l’ordre que je t'ai donné. J’ai donné rendez-vous à mes hommes en un certain lieu.
\VS{3}Maintenant donc qu'as-tu sous la main ? Donne-moi cinq pains ou ce qui se trouvera.
\VS{4}Le sacrificateur répondit à David et dit : Je n'ai pas de pain ordinaire sous la main, mais du pain sacré\FTNT{Mt. 12:4.} ; pourvu que tes gens se soient abstenus de femmes !
\VS{5}David répondit au sacrificateur : Il est vrai que depuis que je suis parti, il y a trois jours, les femmes ont été éloignées de nous, et les vases des serviteurs sont restés purs, et si c’est là un acte profane, à plus forte raison, il sera aujourd’hui sanctifié par les vases.
\VS{6}Alors le sacrificateur lui donna du pain sacré, car il n'y avait pas là d'autre pain que les pains de proposition qui avaient été ôtés de devant Yahweh, pour le remplacer par du pain chaud le jour où on l’avait pris.
\VS{7}Or il y avait là un homme d'entre les serviteurs de Saül, retenu ce jour-là devant Yahweh ; il s’appelait Doëg, un Edomite, le plus puissant des bergers de Saül.
\VS{8}David dit à Achimélec : Mais n'as-tu pas ici sous la main quelque lance, ou quelque épée ? Car je n’ai pas pris mon épée ni mes armes sur moi, parce que l’ordre du roi était pressant.
\VS{9}Et le sacrificateur dit : Voici l'épée de Goliath, le Philistin, que tu as tué dans la vallée du chêne, elle est enveloppée d'un drap, derrière l'éphod ; si tu veux la prendre pour toi, prends-la ; car il n'y en a pas ici d'autre que celle-là. Et David dit : Il n'y en a pas de pareille ; donne-la-moi.
\TextTitle{[David se rend à Gad]}
\VS{10}Alors David se leva, et s'enfuit ce jour-là, loin de Saül, et s'en alla vers Akisch, roi de Gath.
\VS{11}Et les serviteurs d'Akisch lui dirent : N'est-ce pas là David, roi du pays ? N’est-ce pas celui duquel on chantait et répondait en dansant : Saül a tué ses mille, et David ses dix mille ?
\VS{12}David mit ces paroles dans son cœur, et eut une grande crainte d'Akisch, roi de Gath.
\VS{13}Il se montra comme un insensé à leurs yeux, il agit devant eux comme un fou ; et il faisait des marques sur les battants des portes, et laissait couler sa salive sur sa barbe.
\VS{14}Et Akisch dit à ses serviteurs : Vous voyez que cet homme a perdu la raison. Pourquoi me l'avez-vous amené ?
\VS{15}Est-ce que je manque de fous, pour que vous m’ameniez celui-ci pour faire l'insensé devant moi ? Faudrait-il qu’il entre dans ma maison ?
\TextTitle{[David se réfugie dans la caverne d'Adullam]
\\(1 Ch. 12:16-18)}
\Chap{22}
\VerseOne{}David partit de là, et se sauva dans la caverne d'Adullam. Ses frères et toute la maison de son père l’ayant appris, ils descendirent vers lui.
\VS{2}Tous ceux qui étaient dans la détresse, qui avaient des créanciers, et qui avaient le cœur rempli d'amertume, se rassemblèrent auprès de lui, et il devint leur chef. Ainsi se joignirent à lui environ quatre cents hommes.
\VS{3}David s'en alla de là à Mitspé dans le pays de Moab. Il dit au roi de Moab : Permets, je te prie à mon père et ma mère de se retirer chez vous jusqu'à ce que je sache ce que Dieu fera de moi.
\VS{4}Il les amena devant le roi de Moab, et ils demeurèrent chez lui, tout le temps que David fut dans cette forteresse.
\VS{5}Gad, le prophète, dit à David : Ne demeure pas dans cette forteresse, va-t'en, et entre dans le pays de Juda. David donc s'en alla, et vint dans la forêt de Héreth.
\TextTitle{[Saül tue les sacrificateurs]}
\VS{6}Saül apprit qu'on avait découvert David et ses gens. Or Saül était assis sous le tamaris, à Guibea, sur la hauteur ; il avait sa lance à la main, et tous ses serviteurs se tenaient devant lui.
\VS{7}Saül dit à ses serviteurs qui se tenaient près de lui : Ecoutez Benjamites ! Le fils d'Isaï vous donnera-t-il à vous tous des champs et des vignes ? Vous établira-t-il tous chefs de mille, et chefs de cent ?
\VS{8}Pourquoi avez-vous tous conspiré contre moi, et n'y a-t-il personne qui m’informe de l’alliance que mon fils a faite avec le fils d'Isaï ? Pourquoi n’y a-t-il personne de vous qui souffre à mon sujet et qui m'avertisse que mon fils a suscité mon serviteur contre moi pour me dresser des embûches, comme il le fait aujourd'hui.
\VS{9}Alors Doëg, l’Edomite, qui était établi sur les serviteurs de Saül, répondit et dit : J'ai vu le fils d'Isaï venir à Nob, auprès d’Achimélec, fils d'Achithub.
\VS{10}Il a consulté Yahweh pour lui, il lui a donné des vivres ainsi que l'épée de Goliath, le Philistin.
\VS{11}Alors le roi envoya appeler Achimélec, le sacrificateur, fils d'Achithub, la maison de son père, et les sacrificateurs qui étaient à Nob ; et ils vinrent tous vers le roi.
\VS{12}Saül dit : Ecoute, fils d'Achithub ! Il répondit : Me voici, mon seigneur.
\VS{13}Saül lui dit : Pourquoi avez-vous conspiré contre moi, toi et le fils d'Isaï ? Pourquoi lui as-tu donné du pain et une épée, et as-tu consulté Dieu pour lui, pour qu'il s'élève contre moi comme il le fait aujourd'hui, pour me dresser des embûches ?
\VS{14}Achimélec répondit au roi et dit : Entre tous tes serviteurs y en a-t-il un comme David, fidèle et gendre du roi, qui est parti sur ton commandement, et honoré dans ta maison ?
\VS{15}Est-ce d’aujourd'hui que j’ai commencé à consulter Dieu pour lui ? Loin de moi ! Que le roi n’impute aucun tort à son serviteur, à personne de la maison de mon père ; car ton serviteur ne sait rien de tout cela, petite ou grande.
\VS{16}Le roi lui dit : Tu mourras, Achimélec, toi et toute la maison de ton père.
\VS{17}Alors le roi dit aux coureurs qui se tenaient devant lui : Approchez-vous et mettez à mort les sacrificateurs de Yahweh ; car leur main est avec David, parce qu'ils savaient qu'il s'enfuyait, et qu'ils ne m'ont pas averti. Mais les serviteurs du roi ne voulurent pas étendre la main pour frapper les sacrificateurs de Yahweh.
\VS{18}Le roi dit à Doëg : Approche-toi, et frappe les sacrificateurs. Et Doëg, l’Edomite se tourna, et frappa les sacrificateurs ; il tua en ce jour-là quatre-vingt-cinq hommes qui portaient l'éphod de lin.
\VS{19}Il frappa encore du tranchant de l’épée Nob, ville des sacrificateurs ; hommes et femmes, enfants et nourrissons, bœufs, ânes, et brebis, tombèrent sous le tranchant de l'épée.
\VS{20}Toutefois un des fils d'Achimélec, fils d'Achithub, qui s’appelait Abiathar, se sauva, et s'enfuit auprès de David.
\VS{21}Abiathar rapporta à David que Saül avait tué les sacrificateurs de Yahweh.
\VS{22}David dit à Abiathar : Je savais bien ce jour-là, que Doëg, l’Edomite qui était présent, ne manquerait pas d’informer Saül. Je suis la cause de la mort de toutes les personnes de la maison de ton père.
\VS{23}Reste avec moi, ne crains rien, car celui qui cherche ma vie, cherche la tienne ; avec moi, tu seras bien gardé.
\TextTitle{[David libère Kéïla]}
\Chap{23}
\VerseOne{}On fit ce rapport à David, en disant : Voici, les Philistins font la guerre à Keïla, et pillent les aires.
\VS{2}David consulta Yahweh\FTNT{La clé du succès de David était Yahweh. Il consultait régulièrement Dieu avant de s’engager dans une guerre (Ps. 60:14).} en disant : Irai-je, et frapperai-je ces Philistins ? Et Yahweh répondit à David : Va, et tu frapperas les Philistins, et tu délivreras Keïla.
\VS{3}Les gens de David lui dirent : Voici, nous avons peur ici en Juda ; que sera-ce donc quand nous irons à Keïla contre les troupes des Philistins ?
\VS{4}C'est pourquoi David consulta encore Yahweh, et Yahweh lui répondit et dit : Lève-toi, descends à Keïla, car je livre les Philistins entre tes mains.
\VS{5}Alors David s'en alla avec ses gens à Keïla, et combattit contre les Philistins, et emmena leur bétail, et fit un grand carnage ; ainsi David délivra les habitants de Keïla.
\VS{6}Lorsque Abiathar, fils d'Achimélec, s'était enfui vers David à Keïla, il avait en main l'éphod.
\VS{7}On rapporta à Saül que David était venu à Keïla ; Saül dit : Dieu l'a livré entre mes mains car il s'est enfermé en entrant dans une ville qui a des portes et des barres.
\VS{8}Saül convoqua tout le peuple pour aller à la guerre, afin de descendre à Keïla, et d'assiéger David et ses gens.
\VS{9}David ayant eu connaissance des mauvais desseins de Saül à son égard, dit au sacrificateur Abiathar : Apporte l'éphod.
\VS{10}Puis David dit : Yahweh, Dieu d'Israël ! Ton serviteur apprend que Saül cherche à venir à Keïla, pour détruire la ville à cause de moi.
\VS{11}Les chefs de Keïla me livreront-ils entre ses mains ? Saül descendra-t-il comme ton serviteur l'a entendu dire ? Yahweh, Dieu d'Israël ! Je te prie, révèle-le à ton serviteur. Et Yahweh répondit : Il descendra.
\VS{12}David dit encore : Les chefs de Keïla me livreront-ils, moi et mes gens, entre les mains de Saül ? Et Yahweh répondit : Ils te livreront.
\TextTitle{[David échappe encore à Saül]}
\VS{13}Alors David se leva avec ses gens au nombre d’environ six cents hommes ; et ils sortirent de Keïla, et s'en allèrent où ils purent. On rapporta à Saül que David s'était sauvé de Keïla, c'est pourquoi il cessa sa marche.
\VS{14}David resta au désert, dans des lieux forts, et il se tint sur la montagne au désert de Ziph. Et Saül le cherchait tous les jours, mais Dieu ne le livra pas entre ses mains.
\VS{15}David sachant que Saül était sorti pour attenter à sa vie, se tint au désert de Ziph, dans la forêt.
\VS{16}Alors Jonathan, fils de Saül, se leva, et s'en alla dans la forêt vers David, et fortifia son autorité en Dieu.
\VS{17}Et lui dit : Ne crains pas, car Saül mon père ne t’atteindra pas, mais tu régneras sur Israël, et moi je serai le second après toi ; et même Saül mon père le sait bien.
\VS{18}Ils firent tous les deux, alliance devant Yahweh ; et David resta dans la forêt, mais Jonathan retourna dans sa maison.
\VS{19}Or les Ziphiens montèrent auprès de Saül à Guibea, et lui dirent : David ne se tient-il pas caché parmi nous dans des lieux forts, dans la forêt, sur la colline de Hakila, qui est au midi du désert ?
\VS{20}Maintenant donc, ô roi ! Puisque tout le désir de ton âme est de descendre, descends, et ce sera à nous de le livrer entre les mains du roi.
\VS{21}Et Saül dit : Que Yahweh vous bénisse de ce que vous avez eu pitié de moi !
\VS{22}Allez donc, je vous prie, assurez-vous encore davantage pour savoir et trouver le lieu où il a dirigé ses pas et qui l’a vu ; car m’a-t-on dit, il est fort rusé.
\VS{23}Examinez donc et reconnaissez tous les lieux où il se tient caché, puis retournez vers moi quand vous en serez assurés, et j'irai avec vous. S'il est dans le pays, je le chercherai parmi tous les milliers de Juda.
\VS{24}Ils se levèrent donc et s'en allèrent à Ziph avant Saül. David et ses gens étaient dans le désert de Maon, dans la plaine, au midi du désert.
\VS{25}Saül et ses gens partirent à la recherche de David. Et l’on en informa David, qui descendit le rocher, et resta dans le désert de Maon. Saül l’ayant appris, poursuivit David au désert de Maon.
\VS{26}Saül marchait d’un côté de la montagne, et David et ses gens de l'autre côté de la montagne. David fuyait précipitamment pour échapper à Saül. Mais Saül et ses gens entouraient David et ses gens pour s’emparer d’eux.
\VS{27}Lorsqu’un messager vint à Saül, en disant : Hâte-toi de venir, car les Philistins envahissent le pays.
\VS{28}Alors Saül cessa de poursuivre David, et s'en retourna au-devant des Philistins : C'est pourquoi on appela ce lieu Séla-Hammachlekoth.
\TextTitle{[David épargne la vie de Saül à En-Guédi]}
\Chap{24}
\VerseOne{}Puis David monta de là et demeura dans les lieux forts d'En-Guédi.
\VS{2}Lorsque Saül fut revenu de la poursuite des Philistins, on lui fit ce rapport disant : David est dans le désert d’En-Guédi.
\VS{3}Saül prit trois mille hommes d'élite de tout Israël, et il s'en alla chercher David et ses gens jusque sur le rocher des boucs sauvages.
\VS{4}Saül arriva à des parcs de brebis qui étaient près du chemin, où il y avait une caverne dans laquelle il entra pour se couvrir les pieds. David et ses gens se tenaient au fond de la caverne.
\VS{5}Et les gens de David lui dirent : Voici le jour où Yahweh te dit : Je te livre ton ennemi entre tes mains, afin que tu lui fasses selon ce qu'il te semblera bon. David se leva et coupa tout doucement le pan du manteau de Saül.
\VS{6}Après cela, le cœur de David battit, parce qu'il avait coupé le pan du manteau de Saül.
\VS{7}Et il dit à ses gens : Que Yahweh me garde de commettre une telle action contre mon seigneur, l'oint de Yahweh, en mettant ma main sur lui ; car il est l'oint de Yahweh\FTNT{David épargne Saül parce qu'il fait confiance à Yahweh. David laisse Dieu agir plutôt que d'agir lui-même. C'est ce que Paul, apôtre du Seigneur Jésus-Christ, a écrit en Ro. 11 :17. }.
\VS{8}Ainsi David détourna ses gens par ses paroles, et il ne leur permit pas de s'élever contre Saül. Puis Saül se leva de la caverne et poursuivit son chemin.
\VS{9}Après cela, David se leva, sortit de la caverne, et cria après Saül, en disant : Mon seigneur le roi ! Saül regarda derrière lui, et David s'inclina le visage contre terre et se prosterna.
\VS{10}David dit à Saül : Pourquoi écouterais-tu les paroles des gens qui te disent : Voici, David cherche ton malheur ?
\VS{11}Aujourd'hui, tes yeux ont vu que Yahweh t'avait livré entre mes mains dans la caverne, et on m'a dit de te tuer ; mais je t'ai épargné, et j'ai dit : Je ne porterai pas la main sur mon seigneur ; car il est l'oint de Yahweh.
\VS{12}Regarde donc, mon père, regarde le pan de ton manteau dans ma main. Car, j’ai coupé le pan de ton manteau et je ne t'ai pas tué. Sache et reconnais qu'il n'y a ni mal ni injustice dans ma conduite ; et que je n'ai pas péché contre toi. Mais cependant tu me dresses des embûches pour me tuer.
\VS{13}Yahweh sera juge entre moi et toi, et Yahweh me vengera de toi, mais ma main ne sera pas sur toi.
\VS{14}Des méchants vient la méchanceté, dit l’ancien proverbe. C'est pourquoi je ne porterai pas la main sur toi.
\VS{15}Contre qui est sorti le roi d'Israël ? Qui poursuis-tu ? Un chien mort, une puce ?
\VS{16}Yahweh sera donc juge, et jugera entre moi et toi ; il regardera et plaidera ma cause, il me rendra justice en me délivrant de ta main.
\VS{17}Dès que David eut achevé d’adresser ces paroles à Saül, Saül dit : N'est-ce pas là ta voix, mon fils David ? Et Saül éleva la voix, et pleura.
\VS{18}Et il dit à David : Tu es plus juste que moi ; car tu m'as rendu le bien pour le mal que je t'ai fait,
\VS{19}et tu m'as fait connaître aujourd'hui comment tu as usé de bonté envers moi, car Yahweh m'avait livré entre tes mains, et cependant tu ne m'as pas tué.
\VS{20}Si quelqu’un rencontre son ennemi le laisse-t-il poursuivre tranquillement son chemin ? Que Yahweh donc te récompense pour la grâce que tu m'as faite aujourd'hui !
\VS{21}Et maintenant voici, je sais que tu régneras certainement et que le royaume d'Israël restera entre tes mains.
\VS{22}C'est pourquoi maintenant, jure-moi par Yahweh, que tu ne détruiras pas ma race après moi, et que tu n'extermineras pas mon nom de la maison de mon père.
\VS{23}Et David le jura à Saül. Puis Saül s'en alla dans sa maison, et David et ses gens montèrent au lieu fort.
\TextTitle{[Israël pleure la mort de Samuel]}
\Chap{25}
\VerseOne{}Samuel mourut, et tout Israël s'assembla, et le pleura, et on l'enterra dans sa maison à Rama. David se leva, et descendit au désert de Paran.
\TextTitle{[Ingratitude de Nabal, Abigail une femme de bon sens]}
\VS{2}Il y avait à Maon un homme qui avait ses biens à Carmel, et cet homme-là était très puissant, il avait trois mille brebis, et mille chèvres ; et il se trouvait à Carmel quand on tondait ses brebis.
\VS{3}Cet homme s’appelait Nabal, sa femme Abigaïl, elle était une femme de bon sens, et belle de visage, mais l’homme était cruel et méchant dans toutes ses actions. Il était de la race de Caleb.
\VS{4}David apprit au désert, que Nabal tondait ses brebis.
\VS{5}Il envoya dix jeunes gens, et leur dit : Montez à Carmel, et rendez-vous auprès de Nabal. Vous le saluerez en mon nom,
\VS{6}et vous lui direz : Puisses-tu faire autant l’année prochaine à la même saison, et que la paix soit avec ta maison et tout ce qui est à toi.
\VS{7}Et maintenant j'ai appris que tu as les tondeurs. Or tes bergers ont été avec nous, et nous ne leur avons fait aucune injure, et ils n’ont subi aucune perte pendant tout le temps qu'ils ont été à Carmel.
\VS{8}Demande-le à tes serviteurs, et ils te le diront. Que ces jeunes gens trouvent donc grâce à tes yeux, puisque nous venons dans un jour favorable. Nous te prions de donner à tes serviteurs, et à David, ton fils, ce que tu trouveras sous ta main.
\VS{9}Les gens de David arrivèrent et dirent à Nabal, au nom de David, toutes ces paroles ; puis ils se turent.
\VS{10}Nabal répondit aux serviteurs de David, et dit : Qui est David, et qui est le fils d'Isaï ? Aujourd'hui le nombre des serviteurs qui s’échappent de leurs maîtres se multiplie.
\VS{11}Et prendrais-je mon pain, mon eau, et la viande que j'ai apprêtée pour mes tondeurs, afin de les donner à des gens qui viennent je ne sais d'où ?
\VS{12}Ainsi les gens de David rebroussèrent chemin. Ils s'en retournèrent, et firent leur rapport à David.
\VS{13}Et David dit à ses gens : Que chacun de vous ceigne son épée. Et ils ceignirent chacun leur épée. David aussi ceignit son épée et environ quatre cents hommes montèrent avec David. Il en resta deux cents près des bagages.
\VS{14}Or un des serviteurs de Nabal fit ce rapport à Abigaïl, femme de Nabal, et lui dit : Voici, David a envoyé du désert des messagers pour saluer notre maître, qui les a traités rudement.
\VS{15}Cependant ces hommes ont été très bon envers nous, et ne nous ont fait aucune injure, et rien ne nous été enlevé, tout le temps que nous avons été avec eux lorsque nous étions dans les champs.
\VS{16}Ils nous ont servi de muraille nuit et jour, tout le temps que nous avons été avec eux, faisant paître les troupeaux.
\VS{17}Sache maintenant, et vois ce que tu as à faire, car le mal est résolu contre notre maître, et contre toute sa maison, et il est si méchant qu'on n'ose lui parler.
\VS{18}Abigaïl se hâta donc, et prit deux cents pains, deux outres de vin, cinq pièces de menu bétail, cinq mesures de grain rôti, cent paquets de raisins secs, deux cents de figues sèches, et les mit sur des ânes.
\VS{19}Puis elle dit à ses gens : Passez devant moi, je vais vous suivre. Elle n'en dit rien à Nabal, son mari.
\VS{20}Et étant montée sur un âne, elle descendait de la montagne par un chemin couvert ; voici, David et ses gens descendaient en face d’elle, et elle les rencontra.
\VS{21}David avait dit : C'est en vain que j'ai gardé tout ce que cet homme a dans le désert, en sorte qu'il ne s'est rien perdu de tout ce qu’il possède ; il m'a rendu le mal pour le bien.
\VS{22}Que Dieu traite son serviteur David dans toute sa rigueur, si d'ici au matin je laisse subsister de tout ce qui appartient à Nabal.
\VS{23}Lorsque Abigaïl aperçut David, elle se hâta de descendre de son âne, et tomba sur sa face devant David, et se prosterna contre terre.
\VS{24}Elle se jeta donc à ses pieds et lui dit : A moi la faute, mon seigneur ! Permets à ta servante de parler devant toi, et écoute les paroles de ta servante.
\VS{25}Que mon seigneur ne prenne pas garde à ce méchant homme, à Nabal, car il est comme son nom ; Nabal est son nom, et il y a de la folie chez lui. Et moi, ta servante, je n'ai pas vu les gens que mon seigneur a envoyés.
\VS{26}Maintenant, mon seigneur, aussi vrai que Yahweh est vivant, et que ton âme vit, Yahweh t'a empêché d'en venir au sang, et il a retenu ta main. Or que tes ennemis, et ceux qui cherchent à nuire à mon seigneur, soient comme Nabal.
\VS{27}Voici un présent, que ta servante a apporté à mon seigneur, afin qu'on le donne aux gens qui sont à la suite de mon seigneur.
\VS{28}Pardonne, je te prie, le crime de ta servante ; vu que Yahweh ne manquera pas d'établir une maison ferme à mon seigneur ; car mon seigneur conduit les batailles de Yahweh, et il ne s'est trouvé en toi aucun mal pendant toute ta vie.
\VS{29}Si les hommes se lèvent pour te persécuter, et pour chercher ton âme, l'âme de mon seigneur sera liée au faisceau des vivants auprès de Yahweh ton Dieu ; mais il lancera au loin, avec la fronde, l'âme de tes ennemis.
\VS{30}Lorsque Yahweh fera à mon seigneur selon tout le bien qu'il t'a prédit, et qu’il t'établira conducteur d'Israël,
\VS{31}ceci ne sera pas un obstacle, ni un sujet de regret dans l'âme de mon seigneur, pour avoir répandu le sang inutilement, et pour s'être vengé lui-même. Aussi lorsque Yahweh aura fait du bien à mon seigneur, tu te souviendras de ta servante.
\VS{32}Alors David dit à Abigaïl : Béni soit Yahweh, le Dieu d'Israël, qui t'a aujourd'hui envoyée à ma rencontre !
\VS{33}Et béni soit ton bon sens, et bénie sois-tu, toi qui m'as aujourd'hui empêché d'en venir au sang, et qui as retenu ma main !
\VS{34}Car Yahweh, le Dieu d'Israël qui m'a empêché de te faire du mal, est vivant ! Si tu ne t’étais hâtée de venir à ma rencontre, il ne serait resté qui que ce soit à Nabal d'ici au matin.
\VS{35}David prit donc de sa main ce qu'elle lui avait apporté, et lui dit : Remonte en paix dans ta maison ; regarde, j'ai écouté ta voix, et j'ai répondu favorablement à ta demande.
\TextTitle{[Mort de Nabal]}
\VS{36}Alors Abigaïl revint auprès de Nabal ; et voici, il faisait un festin dans sa maison, comme un festin de roi ; et Nabal avait le cœur joyeux, et il était complètement ivre ; c'est pourquoi elle ne lui dit aucune chose petite ou grande, jusqu'au matin.
\VS{37}Mais le matin, l’ivresse de Nabal étant dissipée, sa femme lui raconta toutes ces choses. Le cœur de Nabal reçut un coup mortel, de sorte qu'il devint comme une pierre.
\VS{38}Environ dix jours après, Yahweh frappa Nabal, et il mourut.
\VS{39}Lorsque David apprit que Nabal était mort, il dit : Béni soit Yahweh, qui m'a vengé de l'outrage que j'avais reçu de la main de Nabal, et qui a préservé son serviteur de faire du mal, et a fait retomber le mal de Nabal sur sa tête ! Puis David envoya des gens pour parler à Abigaïl, afin de la prendre pour sa femme.
\VS{40}Les serviteurs de David vinrent auprès d'Abigaïl à Carmel, et lui parlèrent, en disant : David nous a envoyés vers toi, afin de te prendre pour femme.
\VS{41}Alors elle se leva, et se prosterna le visage contre terre, et dit : Voici, ta servante sera à ton service afin de laver les pieds des serviteurs de mon seigneur.
\VS{42}Aussitôt, Abigaïl se leva et monta sur un âne, accompagnée de cinq jeunes filles ; elle suivit les messagers de David, et fut sa femme.
\VS{43}Or David avait pris aussi Achinoam, de Jizreel, et toutes les deux furent ses femmes.
\VS{44}Et Saül avait donné Mical, sa fille, femme de David, à Palthi, fils de Laïsch, qui était de Gallim.
\TextTitle{[David épargne encore la vie de Saül]}
\Chap{26}
\VerseOne{}Les Ziphiens allèrent encore auprès de Saül à Guibea, en disant : David ne se tient-il pas caché sur la colline de Hakila, en face du désert ?
\VS{2}Saül se leva, et descendit au désert de Ziph, avec trois mille hommes de l'élite d'Israël, pour chercher David dans le désert de Ziph.
\VS{3}Saül campa sur la colline de Hakila, en face du désert, près du chemin. David se tenait dans le désert, et il aperçut que Saül marchait à sa poursuite au désert,
\VS{4}alors il envoya des espions et apprit avec certitude que Saül était arrivé.
\VS{5}Alors David se leva, et alla au lieu où Saül campait, et David vit la place où couchait Saül, avec Abner, fils de Ner, chef de son armée. Saül couchait au milieu du camp, et le peuple campait autour de lui.
\VS{6}David prit la parole et dit à Achimélec, Héthien, et à Abischaï, fils de Tseruja et frère de Joab, il dit : Qui veut descendre avec moi dans le camp vers Saül ? Et Abischaï répondit : J'y descendrai avec toi.
\VS{7}David et Abischaï allèrent de nuit vers le peuple, et voici, Saül dormait étant couché au milieu du camp, et sa lance était plantée en terre à son chevet ; et Abner, et le peuple étaient couchés autour de lui.
\VS{8}Alors Abischaï dit à David : Aujourd'hui, Dieu a livré ton ennemi entre tes mains ; laisse-moi donc le frapper avec la lance, jusqu'en terre d'un seul coup, et je n'y retournerai pas une seconde fois.
\VS{9}Et David dit à Abischaï : Ne le tues pas ! Car qui porterait impunément sa main sur l'oint de Yahweh ?
\VS{10}David dit encore : Yahweh est vivant ! C’est Yahweh seul qui le frappera, soit que son jour vienne, soit qu'il descende au combat et qu'il y périsse.
\VS{11}Que Yahweh me garde de mettre ma main sur l'oint de Yahweh ; mais prends maintenant la lance qui est à son chevet et la cruche d’eau, et allons-nous-en.
\VS{12}David donc prit la lance et la cruche d’eau qui étaient au chevet de Saül, puis ils s'en allèrent. Personne ne les vit, ni ne s’aperçut de rien, ni ne se réveilla ; car ils dormaient tous d’un profond sommeil dans lequel Yahweh les avait plongés.
\VS{13}David passa de l'autre côté, et s'arrêta au loin sur le sommet de la montagne, et il y avait une grande distance entre eux.
\VS{14}Et il cria au peuple, et à Abner, fils de Ner, en disant : Ne répondras-tu pas, Abner ? Abner répondit, et dit : Qui es-tu toi qui cries vers le roi ?
\VS{15}Alors David dit à Abner : N'es-tu pas un vaillant homme ? Qui est semblable à toi en Israël ? Pourquoi donc n'as-tu pas gardé le roi ton seigneur ? Car quelqu'un du peuple est venu pour tuer le roi ton seigneur.
\VS{16}Ce que tu as fait n’est pas bien ; Yahweh est vivant ! Vous méritez la mort, pour avoir si mal gardé votre seigneur, l'oint de Yahweh. Et maintenant regarde où sont la lance du roi et la cruche d’eau qui était à son chevet.
\VS{17}Alors Saül reconnut la voix de David, et dit : N'est-ce pas là ta voix, mon fils David ? Et David dit : C'est ma voix, ô roi mon seigneur.
\VS{18}Il dit encore : Pourquoi mon seigneur poursuit-il son serviteur ? Car qu'ai-je fait, de quoi suis-je coupable ?
\VS{19}Maintenant donc je te prie, que le roi mon seigneur écoute les paroles de son serviteur. Si c'est Yahweh qui te pousse contre moi, que ton offrande lui soit agréable ; mais si ce sont les hommes, qu’ils soient maudits devant Yahweh ; car aujourd'hui ils m'ont chassé, afin que je ne puisse me joindre à l'héritage de Yahweh, et ils m'ont dit : Va, sers les dieux étrangers.
\VS{20}Que mon sang ne tombe pas en terre loin de la face de Yahweh ! Car le roi d’Israël est sorti pour chercher une puce, comme on poursuivrait une perdrix dans les montagnes.
\TextTitle{[Saül se repend devant David]}
\VS{21}Saül dit : J'ai péché, reviens mon fils David ; car je ne te ferai plus de mal, parce qu'aujourd'hui ma vie t'a été précieuse. Voici, j'ai agi en insensé, et j'ai commis une très grande faute.
\VS{22}David répondit et dit : Voici la lance du roi que l'un de tes gens vienne la prendre.
\VS{23}Que Yahweh rende à chacun selon sa justice et selon sa fidélité ; car il t'avait livré aujourd'hui entre mes mains, mais je n'ai pas voulu mettre ma main sur l'oint de Yahweh.
\VS{24}Voici, comme ta vie a été aujourd'hui de grand prix à mes yeux, ainsi ma vie sera de grand prix aux yeux de Yahweh, et il me délivrera de toutes les angoisses.
\VS{25}Saül dit à David : Béni sois-tu, mon fils David ! Tu auras du succès dans tes entreprises. Alors David continua son chemin, et Saül s'en retourna chez lui.
\TextTitle{[David se réfugie dans le pays des Philistins]}
\Chap{27}
\VerseOne{}David dit en son cœur : Certes je périrai un jour par les mains de Saül ; ne vaut-il pas mieux que je me sauve en hâte au pays des Philistins, afin que Saül renonce à me chercher encore dans tout le territoire d'Israël ? Ainsi j’échapperai à sa main.
\VS{2}David se leva, lui et les six cents hommes qui étaient avec lui, et il passa chez Akisch, fils de Maoc, roi de Gath.
\VS{3}David et ses gens restèrent à Gath auprès d’Akisch ; ils avaient chacun leur famille, David et ses deux femmes, Achinoam de Jizreel, et Abigaïl, femme de Nabal, qui était de Carmel.
\VS{4}Alors on informa Saül que David s'était enfui à Gath ; et il cessa de le chercher.
\VS{5}David dit à Akisch : Si j'ai trouvé grâce à tes yeux, qu'on me donne dans l'une des villes du pays, un lieu où je puisse habiter. Car pourquoi ton serviteur habiterait-il dans la ville royale avec toi ?
\VS{6}Akisch lui donna ce même jour, Tsiklag. C'est pourquoi Tsiklag appartient aux rois de Juda jusqu'à ce jour.
\VS{7}Le temps que David demeura dans le pays des Philistins fut d’un an et quatre mois.
\VS{8}David montait avec ses gens faire des incursions chez les Gueschuriens, les Guirziens, et les Amalécites ; car ces nations habitaient dans le territoire dès les temps anciens, depuis Schur jusqu'au pays d'Egypte.
\VS{9}David ravageait ce territoire, il ne laissait en vie ni homme ni femme, et il prenait les brebis, les bœufs, les ânes, les chameaux, et les vêtements, puis il s'en retournait, et allait chez Akisch.
\VS{10}Akisch disait : Où avez-vous fait vos incursions aujourd'hui ? Et David répondait : Vers le midi de Juda, vers le midi des Jerachmeélites, et vers le midi des Kéniens.
\VS{11}Mais David ne laissait en vie ni homme ni femme pour les amener à Gath, de peur, disait-il, qu'ils ne rapportent quelque chose contre nous, disant : Ainsi a fait David. Et il agit ainsi tout le temps qu'il demeura dans le pays des Philistins.
\VS{12}Akisch croyait David, et il disait : Il se rend odieux à Israël, son peuple ; c'est pourquoi il sera mon serviteur à jamais.
\TextTitle{[Les Philistins vont en guerre contre Saül]}
\Chap{28}
\VerseOne{}En ces temps-là, les Philistins rassemblèrent leurs armées pour faire la guerre, pour combattre Israël. Akisch dit à David : Sache certainement que vous viendrez avec moi au camp, toi et tes gens.
\VS{2}David répondit à Akisch : Certainement tu verras ce que ton serviteur fera. Et Akisch dit à David : C'est pour cela que je te confierai toujours la garde de ma personne.
\VS{3}Or Samuel était mort, et tout Israël avait fait le deuil, et on l'avait enseveli à Rama qui était sa ville. Saül avait ôté du pays ceux qui avaient des esprits de Python\FTNT{L’esprit de python possède les faux prophètes (Ac. 16:16-19).}, et les médiums.
\VS{4}Les Philistins se rassemblèrent et vinrent camper à Sunem ; Saül aussi rassembla tout Israël, et ils campèrent à Guilboa.
\VS{5}A la vue du camp des Philistins, Saül eut peur, et son cœur fut saisi de crainte.
\VS{6}Saül consulta Yahweh ; mais Yahweh ne lui répondit rien, ni par des songes, ni par l'urim, ni par les prophètes.
\TextTitle{[Saül consulte une femme qui évoque les morts]}
\VS{7}Saül dit à ses serviteurs : Cherchez-moi une femme qui ait un esprit de Python, et j'irai vers elle, et je la consulterai. Ses serviteurs lui dirent : Voilà, il y a une femme à En-Dor qui évoque les morts.
\VS{8}Alors Saül se déguisa, prit d'autres vêtements, et il partit avec deux hommes. Ils arrivèrent de nuit chez cette femme et Saül lui dit : Je te prie devine-moi par l’esprit de Python, et fais-moi monter vers moi celui que je te dirai.
\VS{9}Mais la femme lui répondit : Voici, tu sais ce que Saül a fait, et comment il a exterminé du pays ceux qui ont l'esprit de Python et les médiums ; pourquoi donc dresses-tu un piège à mon âme pour me faire mourir ?
\VS{10}Saül lui jura par Yahweh, et lui dit : Yahweh est vivant ! Il ne t’arrivera pas de mal pour cela.
\VS{11}Alors la femme dit : Qui veux-tu que je te fasse monter ? Et il répondit : Fais-moi monter Samuel.
\VS{12}Et la femme voyant Samuel s'écria à haute voix, en disant à Saül : Pourquoi m'as-tu trompée ? Car tu es Saül.
\VS{13}Et le roi lui répondit : Ne crains pas, mais que vois-tu ? La femme dit à Saül : Je vois un dieu qui monte de la terre.
\VS{14}Il lui dit encore : Comment est-il fait ? Elle répondit : C'est un vieillard qui monte, et il est couvert d'un manteau. Et Saül comprit que c'était Samuel, il s’inclina le visage contre terre et se prosterna.
\VS{15}Samuel dit à Saül : Pourquoi m'as-tu troublé en me faisant monter ? Et Saül répondit : Je suis dans une grande angoisse ; car les Philistins me font la guerre, et Dieu s'est retiré de moi, et ne m'a plus répondu ni par les prophètes ni par des songes ; c'est pourquoi je t'ai invoqué\FTNT{}, afin que tu me fasses entendre ce que j'aurai à faire.
\VS{16}Samuel dit : Pourquoi donc me consultes-tu, puisque Yahweh s'est retiré de toi, et qu'il est devenu ton ennemi ?
\VS{17}Yahweh te traite comme je te l’avais annoncé de sa part ; car Yahweh a déchiré le royaume d'entre tes mains, et l'a donné à un autre, à David.
\VS{18}Parce que tu n'as pas obéi à la voix de Yahweh, et que tu n'as pas exécuté l'ardeur de sa colère contre Amalek, à cause de cela, Yahweh te traite de cette manière aujourd'hui.
\VS{19}Yahweh livrera Israël avec toi entre les mains des Philistins, et vous serez demain avec moi, toi et tes fils; Yahweh livrera aussi le camp d'Israël entre les mains des Philistins.
\VS{20}Saül s’écroula à terre tout étendu, très effrayé des paroles de Samuel, les forces lui manquèrent parce qu'il n'avait rien mangé ce jour, ni toute cette nuit.
\VS{21}Alors la femme vint auprès de Saül, et voyant qu'il avait été très effrayé, elle lui dit : Voici, ta servante a obéi à ta voix, j'ai exposé ma vie, et j'ai obéi aux paroles que tu m'as dites.
\VS{22}Maintenant, je te prie, écoute toi aussi ce que ta servante te dira : Laisse-moi de servir avant un morceau de pain, afin que tu manges pour avoir la force de te remettre en route.
\VS{23}Et il le refusa et dit : Je ne mangerai pas. Mais ses serviteurs et la femme aussi le pressèrent tellement qu'il écouta leur voix. Il se leva de terre, et s'assit sur un lit.
\VS{24}Cette femme avait dans sa maison un veau qu'elle engraissait ; et elle se hâta de le tuer, puis elle prit de la farine, et la pétrit, et en cuisit des pains sans levain.
\VS{25}Elle les mit devant Saül et devant ses serviteurs. Et ils mangèrent. Puis s'étant levés, ils s'en allèrent cette nuit-là.
\TextTitle{[Les philistins refusent que David combattent contre Israël]}
\Chap{29}
\VerseOne{}Les Philistins rassemblèrent toutes leurs armées à Aphek, et Israël campa près de la fontaine de Jizreel.
\VS{2}Les princes des Philistins s’avancèrent avec leurs centaines et leurs milliers, et David et ses gens marchèrent à l'arrière-garde avec Akisch.
\VS{3}Les princes des Philistins dirent : Que font ici ces Hébreux ? Et Akisch répondit aux princes des Philistins : N'est-ce pas David, serviteur de Saül, roi d'Israël, il y a longtemps qu’il est avec moi, même quelques années, et je n'ai pas trouvé quelque chose à lui reprocher depuis son arrivée, jusqu'à ce jour.
\VS{4}Mais les princes des Philistins se mirent en colère contre lui, et lui dirent : Renvoie cet homme, et qu'il retourne dans le lieu où tu l'as établi, et qu'il ne descende pas avec nous dans la bataille, de peur qu'il ne se tourne contre nous dans la bataille ; car comment pourrait-il se remettre en grâce auprès de son maître ? Ne serait-ce pas par le moyen des têtes de nos hommes ?
\VS{5}N'est-ce pas ce David, pour qui l’on chantait et répondait en dansant : Saül a frappé ses mille, et David ses dix mille ?
\VS{6}Akisch appela David, et lui dit : Yahweh est vivant ! Tu es certainement un homme droit, et ta conduite dans le camp m'a paru bonne, car je n'ai pas trouvé de mal en toi, depuis le jour où tu es arrivé auprès de moi jusqu'à ce jour ; mais tu ne plais pas aux princes.
\VS{7}Maintenant retourne, et va-t'en en paix, afin que tu ne fasses aucune chose qui déplaise aux princes des Philistins.
\VS{8}David dit à Akisch : Mais qu'ai-je fait ? Et qu'as-tu trouvé en ton serviteur depuis que je suis avec toi jusqu'à ce jour, pour que je n'aille pas combattre contre les ennemis du roi, mon seigneur ?
\VS{9}Akisch répondit et dit à David : Je le sais, car tu es agréable à mes yeux, comme un ange de Dieu ; mais c'est seulement les chefs des Philistins qui disent : Il ne montera pas avec nous dans la bataille.
\VS{10}C'est pourquoi lève-toi de bon matin, avec les serviteurs de ton maître qui sont venus avec toi ; levez-vous de bon matin, et partez dès que vous verrez le jour, allez-vous-en.
\VS{11}Ainsi David se leva de bonne heure, lui et ses gens, pour partir dès le matin, et retourner dans le pays des Philistins. Et les Philistins montèrent à Jizreel.
\TextTitle{[David libère Tsiklag]}
\Chap{30}
\VerseOne{}Lorsque David et ses gens arrivèrent à Tsiklag, le troisième jour, les Amalécites avaient fait une invasion dans le midi, et à Tsiklag, et ils avaient frappé et brûlé Tsiklag.
\VS{2}Après avoir fait prisonniers les femmes et tous ceux qui étaient là, petits et grands. Ils n’avaient tué personne, mais ils les avaient emmenés, et s’étaient remis en chemin.
\VS{3}David et ses gens revinrent dans la ville et voici, elle était brûlée, et leurs femmes, leurs fils, et leurs filles avaient été faits prisonniers.
\VS{4}C’est pourquoi David et le peuple qui était avec lui élevèrent leur voix, et pleurèrent tellement qu’il n’y avait plus en eux de force pour pleurer.
\VS{5}Les deux femmes de David avaient été emmenées, Achinoam, de Jizreel, et Abigaïl, de Carmel, femme de Nabal.
\VS{6}David fut dans une grande angoisse, parce que le peuple parlait de le lapider ; car tout le peuple avait de l’amertume dans l’âme à cause de leurs fils et de leurs filles ; toutefois David se fortifia en Yahweh, son Dieu.
\VS{7}Et il dit au sacrificateur Abiathar, fils d’Achimélec : Apporte-moi, je te prie, l’éphod ! Abiathar apporta l'éphod à David.
\VS{8}Et David consulta Yahweh, en disant : Poursuivrai-je cette troupe ? L’atteindrai-je ? Et il lui répondit : Poursuis, car tu l’atteindras, et tu délivreras.
\VS{9}David s’en alla avec les six cents hommes qui étaient avec lui, et ils arrivèrent au torrent de Besor, où s’arrêtèrent ceux qui restaient en arrière.
\VS{10}Ainsi David et quatre cents hommes, continuèrent la poursuite, mais deux cents hommes s’arrêtèrent, trop fatigués pour pouvoir passer le torrent de Besor.
\VS{11}Ayant trouvé un homme Egyptien dans les champs, ils l’amenèrent à David, et lui donnèrent du pain, il mangea, puis ils lui donnèrent de l’eau à boire.
\VS{12}Ils lui donnèrent aussi quelques figues sèches, et deux grappes de raisins secs, et il mangea, et le coeur lui revint ; car cela faisait trois jours et trois nuits qu’il n’avait pas mangé de pain, ni bu d’eau.
\VS{13}Et David lui dit : A qui es-tu ? Et d’où es-tu ? Et il répondit : Je suis un garçon égyptien, serviteur d’un homme amalécite ; et mon maître m’a abandonné, parce que j’étais malade il y a trois jours.
\VS{14}Nous avons envahi le midi des Kéréthiens, et sur ce qui est à Juda, et sur le midi de Caleb, et nous avons mis le feu et brûlé Tsiklag.
\VS{15}David lui dit : Me conduiras-tu vers cette troupe ? Et il répondit : Jure-moi par le Nom de Dieu que tu ne me feras point mourir, et que tu ne me livreras point entre les mains de mon maître, et je te conduirai vers cette troupe.
\VS{16}Et il le conduisit. Et voici, ils étaient dispersés sur toute la contrée, mangeant, buvant, et dansant, à cause de ce grand butin qu’ils avaient pris au pays des Philistins, et au pays de Juda.
\VS{17}Et David les frappa depuis l’aube du jour, jusqu’au soir du lendemain, et il n’en échappa aucun d’eux, hormis quatre cents jeunes hommes qui montèrent sur des chameaux, et s’enfuirent.
\VS{18}David recouvra tout ce que les Amalécites avaient emporté ; il délivra aussi ses deux femmes.
\VS{19}Il ne leur manqua personne, depuis le plus petit jusqu’au plus grand, ni fils ni filles, ni butin, ni rien de ce qu’ils leur avaient emporté ; David ramena tout.
\VS{20}David reprit aussi tout le gros et menu bétail, qu’on mena devant les troupeaux ; et on disait : C’est ici le butin de David.
\TextTitle{[David partage le butin]}
\VS{21}Puis David arriva auprès de deux cents hommes qui avaient été tellement fatigués qu’ils n’avaient pu suivre David, et qu’on avait laissés au torrent de Besor. Ils sortirent au-devant de David, et au-devant du peuple qui était avec lui. David s’étant approché du peuple, il les salua aimablement.
\VS{22}Mais tous les mauvais et méchants hommes qui étaient allés avec David, prirent la parole, et dirent : Puisqu’ils ne sont point venus avec nous, nous ne leur donnerons rien du butin que nous avons récupéré, sinon à chacun sa femme et ses enfants, et qu’ils les emmènent, et s’en aillent.
\VS{23}Mais David dit : Mes frères, n’agissez pas ainsi au sujet de ce que Yahweh nous a donné, il nous a gardés, et a livré entre nos mains la troupe qui était venue contre nous.
\VS{24}Qui vous écouterait dans cette affaire ? Car celui qui est resté près des bagages doit avoir autant que celui qui est descendu sur le champ de bataille ; ils partageront ensemble.
\VS{25}Il en fut ainsi depuis ce jour et dans la suite, il en fut fait de même une ordonnance et une loi en Israël.
\VS{26}David revint à Tsiklag, et envoya une partie du butin aux anciens de Juda, à ses amis, en disant : Voici, un présent pour vous, du butin des ennemis de Yahweh.
\VS{27}Il en envoya à ceux de Béthel, à ceux qui étaient à Ramoth du midi, à ceux de Jatthir,
\VS{28}à ceux d’Aroër, à ceux de Siphmoth, à ceux de Eschthemoa,
\VS{29}à ceux de Racal, à ceux des villes des Jerachmeélites, à ceux des villes des Kéniens,
\VS{30}à ceux d’Horma, à ceux de Cor-Aschan, à ceux d’Athac,
\VS{31}à ceux d’Hébron, et dans tous les lieux où David avait demeuré, lui et ses gens.
\TextTitle{[Mort de Jonathan et Saül à Guilboa]
\\(1 Ch. 10:1-14)}
\Chap{31}
\VerseOne{}Les Philistins livrèrent bataille à Israël, et les hommes d'Israël s'enfuirent devant les Philistins, et furent tués sur la montagne de Guilboa.
\VS{2}Les Philistins atteignirent Saül et ses fils, et tuèrent Jonathan, Abinadab et Malkischua, fils de Saül.
\VS{3}L’effort du combat se porta sur Saül, et les archers l’atteignirent et le blessèrent grièvement.
\VS{4}Alors Saül dit à celui qui portait ses armes : Tire ton épée, et transperce-moi, de peur que ces incirconcis ne viennent, ne me transpercent, et ne m’outragent. Mais celui qui portait ses armes refusa, parce qu'il était saisi de crainte. Saül prit l'épée, et se jeta dessus.
\VS{5}Alors celui qui portait les armes de Saül, voyant que Saül était mort, se jeta aussi sur son épée, et mourut avec lui.
\VS{6}Ainsi périrent en ce jour, Saül et ses trois fils, celui qui portait ses armes, et tous ses gens.
\VS{7}Ceux d'Israël qui étaient de ce côté de la vallée et de ce côté du Jourdain, ayant vu que les Israëlites s'étaient enfuis que Saül et ses fils étaient morts, abandonnèrent les villes et s'enfuirent ; de sorte que les Philistins y entrèrent et s’y établirent.
\VS{8}Le lendemain, les Philistins vinrent pour dépouiller les morts, et ils trouvèrent Saül et ses trois fils, étendus sur la montagne de Guilboa.
\VS{9}Ils coupèrent la tête de Saül et le dépouillèrent de ses armes. Ils firent annoncer ces bonnes nouvelles par tout le pays des Philistins, dans les maisons de leurs idoles et parmi le peuple.
\VS{10}Ils déposèrent les armes de Saül dans le temple d’Astarté, et ils attachèrent son cadavre sur les murs de Beth-Schan.
\VS{11}Lorsque les habitants de Jabès en Galaad apprirent ce que les Philistins avaient fait à Saül,
\VS{12}tous les vaillants hommes, se levèrent et marchèrent toute la nuit, et ils enlevèrent des murs de Beth-Schan le cadavre de Saül et les cadavres de ses fils. Ils revinrent à Jabès, où ils les brûlèrent.
\VS{13}Puis ils prirent leurs os, les ensevelirent sous un tamaris près de Jabès, et ils jeûnèrent sept jours.
\PPE{}
\end{multicols}

%\clearpage\ShortTitle{2 Samuel}\BookTitle{2 Samuel}\BFont
\noindent\hrulefill
{\footnotesize
\textit{
\bigskip
{\centering{}
\\Auteur : Inconnu
\\(Heb. : Shemuw'el)
\\Signification : Entendu, exaucé de Dieu
\\Thème : Le règne de David
\\Date de rédaction : 10\up{ème} siècle av. J.-C.\\}
}
%\bigskip
\textit{
\\Suite du premier livre de Samuel, ce livre commence par le récit de la mort de Saül et l'accession progressive à la royauté de David. La faveur de Dieu dans sa vie lui donna du succès et lui permit d'étendre son royaume jusqu'au nord de Damas. Au détriment de sa piété et de son alliance avec Dieu, David commit de lourdes erreurs. Il s'en repentit sincèrement, mais il dut en assumer les conséquences…\bigskip
}
}
\par\nobreak\noindent\hrulefill
\begin{multicols}{2}
\Chap{1}
\TextTitle{Attitude de David à la mort de Saül}
\VerseOne{}Et il arriva qu'après la mort de Saül, David, qui était revenu vainqueur des Amalécites, resta deux jours à Tsiklag.
\VS{2}Le troisième jour, un homme arriva du camp de Saül, avec ses vêtements déchirés et de la terre sur sa tête. Il se présenta à David, se jeta par terre et se prosterna.
\VS{3}David lui dit : D'où viens-tu ? Il lui répondit : Je me suis échappé du camp d'Israël.
\VS{4}David lui dit : Qu'est-il arrivé ? Je te prie, raconte-le-moi! Il répondit : Le peuple s'est enfui de la bataille, et il y a eu beaucoup du peuple qui sont tombés morts ; Saül aussi et Jonathan, son fils, sont morts.
\VS{5}David dit à ce jeune garçon qui lui disait ces nouvelles : Comment sais-tu que Saül et Jonathan, son fils, sont morts ?
\VS{6}Le jeune garçon qui lui disait ces nouvelles lui répondit : Je me trouvais par hasard sur la montagne de Guilboa ; et voici, Saül s'appuyait sur sa lance, et voici, les chars et quelques chefs des cavaliers le poursuivaient.
\VS{7}S'étant retourné, il m'aperçut et m'appela. Je lui répondis : Me voici !
\VS{8}Il me dit : Qui es-tu ? Je lui répondis : Je suis Amalécite.
\VS{9}Et il dit : Approche-toi de moi et tue-moi ; car je suis dans une grande angoisse, et ma vie est encore toute en moi.
\VS{10}Je m'approchai de lui et je lui donnai la mort\FTNT{Cet homme Amalécite a peut-être menti afin de gagner la faveur de David (1 S. 31:3-5).}, sachant bien qu'il ne survivrait après s'être jeté sur sa lance. J'ai pris la couronne qu'il avait sur sa tête, et le bracelet qui était à son bras, et je les apporte ici à mon seigneur.
\VS{11}Alors David saisit ses vêtements et les déchira, et tous les hommes qui étaient avec lui firent de même.
\VS{12}Ils furent dans le deuil, ils pleurèrent et ils jeûnèrent jusqu'au soir, à cause de Saül, de Jonathan, son fils, à cause du peuple de Yahweh, et de la maison d'Israël, parce qu'ils étaient tombés par l'épée.
\VS{13}David dit au jeune garçon qui lui avait apporté ces nouvelles : D'où es-tu ? Et il répondit : Je suis fils d'un homme étranger, d'un Amalécite.
\VS{14}David lui dit : Comment n'as-tu pas craint d'avancer ta main pour tuer l'oint de Yahweh ?
\VS{15}Et David appela l'un de ses serviteurs et lui dit : Approche-toi, jette-toi sur lui ! Ce dernier le frappa et il mourut.
\VS{16}Et David lui dit : Que ton sang retombe sur ta tête, car ta bouche a témoigné contre toi, en disant : J'ai fait mourir l'oint de Yahweh !
\TextTitle{Chant funèbre de David}
\VS{17}Alors David composa sur Saül et sur Jonathan, son fils, un chant funèbre,
\VS{18}qu'il ordonna d'enseigner aux fils de Juda. C'est le cantique de l'arc : Il est écrit dans le livre du Juste.
\VS{19}L'élite d'Israël a succombé sur tes collines ! Comment des héros sont-ils tombés ?
\VS{20}Ne l'annoncez pas dans Gath et n'en publiez point la nouvelle dans les rues d'Askalon, de peur que les filles des Philistins ne se réjouissent, de peur que les filles des incirconcis n'en tressaillent de joie.
\VS{21}Montagnes de Guilboa ! Qu'il n'y ait sur vous ni rosée, ni pluie, ni champs qui donnent des prémices pour les offrandes ! Car là ont été jetés les boucliers des héros, le bouclier de Saül ; l'huile a cessé de les oindre.
\VS{22}L'arc de Jonathan ne revenait jamais sans être teint du sang des blessés et de la graisse des hommes forts ; et l'épée de Saül ne retournait jamais sans effet.
\VS{23}Saül et Jonathan, aimables et agréables pendant leur vie, n'ont point été séparés dans leur mort ; ils étaient plus légers que les aigles, ils étaient plus forts que des lions.
\VS{24}Filles d'Israël ! Pleurez sur Saül, qui vous revêtait magnifiquement de cramoisi, qui mettait des ornements d'or à vos habits.
\VS{25}Comment les héros sont-ils tombés au milieu du combat ? Comment Jonathan a-t-il été tué sur tes collines ?
\VS{26}Jonathan, mon frère, je suis dans la douleur à cause de toi ! Tu faisais tout mon plaisir ; l'amour que j'avais pour toi était plus grand que celui qu'on a pour les femmes.
\VS{27}Comment sont tombés les héros ? Comment se sont perdus les instruments de guerre ?
\Chap{2}
\TextTitle{David oint roi de Juda}
\VerseOne{}Et il arriva après cela que David consulta Yahweh, en disant : Monterai-je dans l'une des villes de Juda ? Yahweh lui répondit : Monte. David dit : Où monterai-je ? Yahweh répondit : A Hébron.
\VS{2}David y monta, avec ses deux femmes, Achinoam de Jizreel, et Abigaïl de Carmel, qui avait été femme de Nabal.
\VS{3}David fit monter aussi les hommes qui étaient avec lui, chacun avec sa famille ; et ils habitèrent dans les villes d'Hébron.
\VS{4}Les hommes de Juda vinrent, et là ils oignirent David pour roi sur la maison de Juda. On fit un rapport à David en disant : Les gens de Jabès en Galaad ont enseveli Saül.
\VS{5}Alors David envoya des messagers vers les gens de Jabès en Galaad, pour leur dire : Soyez bénis de Yahweh, puisque vous avez montré de la bienveillance envers Saül, votre seigneur, et que vous l'avez enseveli.
\VS{6}Maintenant donc que Yahweh use envers vous de bonté et de fidélité. Moi aussi je vous ferai du bien, parce que vous avez agi de la sorte.
\VS{7}Que maintenant vos mains se fortifient, et soyez des vaillants hommes ; car Saül, votre maître, est mort, et la maison de Juda m'a oint pour être roi sur elle.
\TextTitle{Isch-Boscheth établi roi d'Israël}
\VS{8}Cependant Abner, fils de Ner, chef de l'armée de Saül, prit Isch-Boscheth, fils de Saül, et le fit passer à Mahanaïm.
\VS{9}Il l'établit roi sur Galaad, sur les Gueschuriens, sur Jizreel, sur Ephraïm, sur Benjamin, et sur tout Israël.
\VS{10}Isch-Boscheth, fils de Saül, était âgé de quarante ans quand il commença à régner sur Israël, et il régna deux ans. Il n'y eut que la maison de Juda qui suivit David.
\VS{11}Le nombre de jours pendant lesquels David régna à Hébron sur la maison de Juda fut de sept ans et six mois.
\TextTitle{Guerre entre Juda et Israël}
\VS{12}Abner, fils de Ner, et les gens d'Isch-Boscheth, fils de Saül, sortirent de Mahanaïm pour marcher vers Gabaon.
\VS{13}Joab, fils de Tseruja, et les gens de David, sortirent aussi. Ils se rencontrèrent ensemble près de l'étang de Gabaon, et les uns se tinrent d'un côté de l'étang, et les autres du côté opposé de l'étang.
\VS{14}Alors Abner dit à Joab : Que ces jeunes gens se lèvent maintenant, et qu'ils se battent devant nous ! Et Joab répondit : Qu'ils se lèvent !
\VS{15}Ils se levèrent donc, et s'avancèrent en nombre égal, douze pour Benjamin et pour Isch-Boscheth, fils de Saül, et douze des gens de David.
\VS{16}Alors, chacun saisissant son adversaire par la tête, lui enfonça son épée dans le flanc, et ils tombèrent tous ensemble. Et l'on donna à ce lieu, qui est près de Gabaon, le nom de Helkath-Hatsurim.
\VS{17}Il y eut ce jour-là un combat très rude, dans lequel Abner et les hommes d'Israël furent battus par les gens de David.
\VS{18}Les trois fils de Tseruja, Joab, Abischaï et Asaël étaient là. Asaël avait les pieds légers comme une gazelle des champs.
\VS{19}Asaël poursuivit Abner, sans se détourner de lui ni à droite ni à gauche.
\VS{20}Abner regarda derrière lui, et dit : Est-ce toi, Asaël ? Et il répondit : C'est moi.
\VS{21}Abner lui dit : Détourne-toi à droite ou à gauche ; saisis-toi de l'un de ces jeunes gens, et prends sa dépouille. Mais Asaël ne voulut point se détourner de lui.
\VS{22}Et Abner dit encore à Asaël : Détourne-toi de moi ; pourquoi te frapperais-je et t'abattrais-je à terre ? Comment ensuite lèverais-je le visage devant ton frère Joab ?
\VS{23}Mais Asaël refusa de se détourner. Abner le frappa au ventre avec l'extrémité inférieure de sa lance, qui sortit par-derrière. Il tomba là, raide mort sur place. Tous ceux qui arrivaient au lieu où Asaël était tombé mort, s'y arrêtaient.
\VS{24}Joab et Abischaï poursuivirent Abner, et le soleil se couchait quand ils arrivèrent au coteau d'Amma, qui est en face de Guiach, sur le chemin du désert de Gabaon.
\VS{25}Les fils de Benjamin s'assemblèrent auprès d'Abner et formèrent un corps de troupe, et ils s'arrêtèrent sur le sommet d'une colline.
\VS{26}Alors Abner appela Joab, et dit : L'épée dévorera-t-elle sans cesse ? Ne sais-tu pas qu'il y aura de l'amertume à la fin ? Jusqu'à quand tarderas-tu à dire au peuple qu'il cesse de poursuivre ses frères ?
\VS{27}Joab répondit : Dieu est vivant ! Si tu n'avais parlé ainsi, le peuple n'aurait pas cessé avant le matin de poursuivre ses frères.
\VS{28}Joab sonna du shofar, et tout le peuple s'arrêta ; ils ne poursuivirent plus Israël, et ils ne continuèrent plus à se battre.
\VS{29}Ainsi, Abner et ses gens marchèrent toute la nuit dans la plaine ; ils passèrent le Jourdain, traversèrent tout le Bithron, et arrivèrent à Mahanaïm.
\VS{30}Joab aussi revint de la poursuite d'Abner, et rassembla tout le peuple ; il manquait dix-neuf hommes des gens de David, et Asaël.
\VS{31}Mais les gens de David avaient frappé à mort trois cent soixante hommes de Benjamin, et des gens d'Abner.
\VS{32}Ils emportèrent Asaël, et l'ensevelirent dans le sépulcre de son père à Bethléhem. Joab et ses gens marchèrent toute la nuit et arrivèrent à Hébron au point du jour.
\Chap{3}
\TextTitle{L'autorité de David s'accroît\FTNTT{1 Ch. 3:1-4.}}
\VerseOne{}Or il y eut une longue guerre entre la maison de Saül et la maison de David. David devenait de plus en plus fort, et la maison de Saül allait en s'affaiblissant.
\VS{2}Il naquit à David des fils à Hébron. Son premier-né fut Amnon, d'Achinoam de Jizreel ;
\VS{3}le second, Kileab, d'Abigaïl de Carmel, femme de Nabal ; le troisième, Absalom, fils de Maaca, fille de Talmaï, roi de Gueschur ;
\VS{4}le quatrième, Adonija, fils de Haggith ; le cinquième, Schephathia, fils d'Abithal ;
\VS{5}et le sixième, Jithream, d'Egla, femme de David. Ce sont là ceux qui naquirent à David à Hébron.
\TextTitle{Abner fait alliance avec David}
\VS{6}Et il arriva que pendant la guerre entre la maison de Saül et la maison de David, Abner tint ferme pour la maison de Saül.
\VS{7}Or Saül avait eu une concubine, nommée Ritspa, fille d'Ajja. Et Isch-Boscheth dit à Abner : Pourquoi es-tu venu vers la concubine de mon père ?
\VS{8}Abner fut très irrité à cause du discours d'Isch-Boscheth, et il lui dit : Suis-je une tête de chien, au service de Juda ? Je fais aujourd'hui preuve de bienveillance envers la maison de Saül, ton père, envers ses frères et ses amis, je ne t'ai pas livré entre les mains de David, et c'est aujourd'hui que tu me reproches une faute avec cette femme ?
\VS{9}Que Dieu punisse sévèrement Abner, si je n'agis pas avec David selon ce que Yahweh a juré à David,
\VS{10}en disant qu'il ferait passer la royauté de la maison de Saül à la sienne, et qu'il établirait le trône de David sur Israël et sur Juda depuis Dan jusqu'à Beer-Schéba.
\VS{11}Isch-Boscheth n'osa pas répondre un seul mot à Abner, parce qu'il le craignait.
\VS{12}Abner envoya des messagers à David pour lui dire de sa part : A qui est le pays ? Fais alliance avec moi, et voici, ma main sera avec toi, pour tourner vers toi tout Israël.
\VS{13}David répondit : Je le veux bien ! Je ferai alliance avec toi ; je te demande seulement une chose, c'est que tu ne voies point ma face, à moins que tu n'amènes d'abord Mical, fille de Saül, quand tu viendras me voir.
\VS{14}Et David envoya des messagers à Isch-Boscheth, fils de Saül, pour lui dire : Rends-moi ma femme Mical, que j'ai épousée pour cent prépuces des Philistins.
\VS{15}Isch-Boscheth envoya et l'ôta à son mari Palthiel, fils de Laïsch.
\VS{16}Et son mari la suivit, marchant et pleurant continuellement après elle jusqu'à Bachurim. Alors Abner lui dit : Va, retourne-t'en ! Et il s'en retourna.
\VS{17}Abner parla aux anciens d'Israël, et leur dit : Vous désiriez autrefois avoir David pour roi ;
\VS{18}établissez-le maintenant, car Yahweh a parlé de David et a dit : C'est par David, mon serviteur, que je délivrerai mon peuple d'Israël de la main des Philistins et de la main de tous ses ennemis.
\VS{19}Abner parla aussi aux oreilles de ceux de Benjamin, puis il alla faire entendre expressément à David, qui était à Hébron, ce qui semblait bon aux yeux d'Israël et aux yeux de toute la maison de Benjamin.
\VS{20}Abner vint donc vers David à Hébron, accompagné de vingt hommes ; et David fit un festin à Abner et aux hommes qui étaient avec lui.
\VS{21}Abner dit à David : Je me lèverai, et je partirai pour rassembler tout Israël auprès du roi, mon seigneur ; ils feront alliance avec toi, et tu régneras selon le désir de ton âme. David renvoya Abner, qui s'en alla en paix.
\VS{22}Voici, les gens de David et Joab revinrent d'une excursion, et amenèrent avec eux un grand butin. Abner n'était plus avec David à Hébron, car David l'avait renvoyé, et il s'en était allé en paix.
\VS{23}Lorsque Joab et toute l'armée qui était avec lui revinrent, on fit ce rapport à Joab en ces mots : Abner, fils de Ner, est venu auprès du roi, qui l'a renvoyé, et il s'en est allé en paix.
\VS{24}Joab vint vers le roi, et dit : Qu'as-tu fait ? Voici, Abner est venu vers toi ; pourquoi l'as-tu ainsi renvoyé, en sorte qu'il s'en est allé ?
\VS{25}Tu connais Abner, fils de Ner ! C'est pour te tromper qu'il est venu, pour épier tes démarches, tes allées et venues, et pour savoir tout ce que tu fais.
\VS{26}Puis Joab, après avoir quitté David, envoya sur les traces d'Abner des messagers, qui le ramenèrent de la fosse de Sira, sans que David n'en sache rien.
\TextTitle{Mort d'Abner}
\VS{27}Lorsque Abner revint à Hébron, Joab le tira à l'écart au milieu de la porte, comme pour lui parler en secret ; et là il le frappa à la cinquième côte ; et ainsi Abner mourut à cause du sang d'Asaël, frère de Joab.
\VS{28}David apprit ce qui était arrivé et dit : Je suis à jamais innocent, mon royaume et moi, devant Yahweh, du sang d'Abner, fils de Ner.
\VS{29}Que ce sang retombe sur la tête de Joab, et sur toute la maison de son père ! Que soit retranchée la maison de Joab, qu'il y ait toujours un homme qui soit atteint d'un flux ou de la lèpre, ou qui s'appuie sur un bâton, ou qui tombe par l'épée, ou qui manque de pain !
\VS{30}Ainsi Joab et Abischaï, son frère, tuèrent Abner, parce qu'il avait tué Asaël, leur frère, à Gabaon, dans la bataille.
\VS{31}David dit à Joab et à tout le peuple qui était avec lui : Déchirez vos vêtements, ceignez-vous de sacs, et menez le deuil en marchant devant Abner ! Et le roi David marcha derrière le cercueil.
\VS{32}On ensevelit Abner à Hébron. Le roi éleva la voix et pleura sur la tombe d'Abner, et tout le peuple pleura.
\VS{33}Le roi fit une complainte sur Abner, et dit : Abner devait-il mourir comme meurt un insensé ?
\VS{34}Tes mains n'étaient pas liées, et tes pieds n'étaient pas mis dans des chaînes ! Tu es tombé comme on tombe devant les méchants. Et tout le peuple recommença à pleurer sur Abner.
\VS{35}Puis tout le peuple vint pour faire prendre quelque nourriture à David, pendant qu'il était encore jour ; mais David jura, en disant : Que Dieu me punisse sévèrement, si je goûte du pain ou quelque chose d'autre avant le coucher du soleil !
\VS{36}Tout le peuple l'entendit, et l'approuva, et tout le peuple trouva bon tout ce qu'avait fait le roi.
\VS{37}En ce jour, tout le peuple et tout Israël surent que ce n'était pas par ordre du roi qu'Abner, fils de Ner, avait été tué.
\VS{38}Le roi dit à ses serviteurs : Ne savez-vous pas qu'un chef, un grand homme, est tombé aujourd'hui en Israël ?
\VS{39}Je suis encore faible aujourd'hui, bien que j'aie été oint roi ; et ces gens, les fils de Tseruja, sont trop puissants pour moi. Que Yahweh rende à celui qui fait le mal selon sa méchanceté !
\Chap{4}
\TextTitle{Mort d'Ish-Boscheth}
\VerseOne{}Quand le fils de Saül apprit qu'Abner était mort à Hébron, ses mains restèrent sans force, et tout Israël fut dans l'épouvante.
\VS{2}Le fils de Saül avait deux chefs de bandes, dont l'un s'appelait Baana et l'autre Récab ; ils étaient fils de Rimmon de Beéroth, d'entre les fils de Benjamin. Car Beéroth était regardée comme appartenant à Benjamin,
\VS{3}et les Beérothiens s'étaient enfuis à Guitthaïm, où ils y ont habité jusqu'à ce jour.
\VS{4}Jonathan, fils de Saül, avait un fils perclus des pieds ; il était âgé de cinq ans lorsque la nouvelle de la mort de Saül et de Jonathan arriva de Jizreel ; sa nourrice le prit et s'enfuit, et comme elle se hâtait de fuir, il tomba et devint boiteux ; son nom était Mephiboscheth.
\VS{5}Les fils de Rimmon de Beéroth, Récab et Baana, se rendirent pendant la chaleur du jour à la maison d'Isch-Boscheth, qui était couché pour son repos du midi.
\VS{6}Ils pénétrèrent jusqu'au milieu de la maison, comme pour y prendre du froment, et ils le frappèrent à la cinquième côte ; puis Récab et Baana, son frère, se sauvèrent.
\VS{7}Ils entrèrent donc dans la maison lorsqu'Isch-Boscheth était couché sur son lit dans la chambre à coucher où il dormait, ils le frappèrent et le tuèrent, puis ils lui coupèrent la tête. Ils prirent sa tête, et ils marchèrent toute la nuit au travers de la plaine.
\VS{8}Ils apportèrent la tête d'Isch-Boscheth à David dans Hébron, et ils dirent au roi : Voici la tête d'Isch-Boscheth, fils de Saül, ton ennemi, qui en voulait à ta vie ; Yahweh venge aujourd'hui le roi mon seigneur de Saül et de sa race.
\VS{9}Mais David répondit à Récab et à Baana, son frère, fils de Rimmon de Beéroth, et leur dit : Yahweh, qui a délivré mon âme de toute angoisse est vivant !
\VS{10}J'ai saisi celui qui est venu m'annoncer et me dire : Voilà, Saül est mort, et qui pensait m'apprendre de bonnes nouvelles, je l'ai fait saisir et tuer à Tsiklag, pour lui donner le salaire de ses bonnes nouvelles ;
\VS{11}combien plus, quand des méchants ont tué un homme juste dans sa maison et sur sa couche, ne redemanderai-je pas maintenant son sang de vos mains et ne vous exterminerai-je pas de la terre ?
\VS{12}David ordonna à ses gens de les tuer ; ils leur coupèrent les mains et les pieds, et les pendirent près de l'étang d'Hébron. Ils prirent la tête d'Isch-Boscheth, et l'ensevelirent dans le sépulcre d'Abner à Hébron.
\Chap{5}
\TextTitle{David oint roi sur tout Israël\FTNTT{1 Ch. 11:1-3.}}
\VerseOne{}Alors toutes les tribus d'Israël vinrent auprès de David, à Hébron, et dirent : Voici, nous sommes tes os et ta chair.
\VS{2}Autrefois déjà, quand Saül était roi sur nous, c'est toi qui conduisais et qui ramenais Israël. Yahweh t'a dit : Tu paîtras mon peuple d'Israël, et tu seras le chef d'Israël.
\VS{3}Tous les anciens d'Israël vinrent donc vers le roi à Hébron, et le roi David fit alliance avec eux à Hébron, devant Yahweh. Ils oignirent David pour roi sur Israël.
\VS{4}David était âgé de trente ans lorsqu'il commença à régner ; il régna quarante ans.
\VS{5}Il régna sur Juda à Hébron sept ans et six mois, puis il régna trente-trois ans à Jérusalem sur tout Israël et Juda.
\TextTitle{Jérusalem, capitale de tout Israël\FTNTT{1 Ch. 11:4-9.}}
\VS{6}Le roi marcha avec ses gens sur Jérusalem contre les Jébusiens qui habitaient ce pays. Ils dirent à David : Tu n'entreras point ici, car les aveugles mêmes et les boiteux te repousseront ! Ce qui voulait dire : David n'entrera point ici.
\VS{7}Mais David s'empara de la forteresse de Sion : C'est la cité de David.
\VS{8}David avait dit en ce jour-là : Quiconque battra les Jébusiens et atteindra le canal, ces aveugles et ces boiteux, qui sont haïs de l'âme de David, sera récompensé… C'est pourquoi l'on dit : Aucun aveugle ni boiteux n'entrera dans cette maison.
\VS{9}Et David habita dans la forteresse, et l'appela la cité de David. Il bâtit tout autour, depuis Millo jusqu'au-dedans.
\VS{10}David devenait de plus en plus grand, et Yahweh, le Dieu des armées, était avec lui.
\TextTitle{Yahweh affermit le règne de David}
\VS{11}Hiram, roi de Tyr, envoya des messagers à David, du bois de cèdre, des charpentiers et des tailleurs de pierres à bâtir, et ils bâtirent la maison de David.
\VS{12}David reconnut que Yahweh l'affermissait comme roi sur Israël, et qu'il élevait son royaume à cause de son peuple d'Israël.
\TextTitle{Fils de David nés à Jérusalem\FTNTT{2 S. 3:2-5 ; 1 Ch. 3:1-4.}}
\VS{13}David prit encore des concubines et des femmes de Jérusalem, après qu'il fut venu d'Hébron, et il lui naquit encore des fils et des filles.
\VS{14}Voici les noms de ceux qui lui naquirent à Jérusalem : Schammua, Schobad, Nathan, Salomon,
\VS{15}Jibhar, Elischua, Népheg, Japhia,
\VS{16}Elischama, Eliada et Eliphéleth.
\TextTitle{Yahweh livre les Philistins à David\FTNTT{2 S. 23:13-17 ; 1 Ch. 14:8-17 ; 11:15-19 ; 12:8-15.}}
\VS{17}Or quand les Philistins apprirent qu'on avait oint David pour roi sur Israël, ils montèrent tous pour chercher David. Et David l'ayant appris, il descendit vers la forteresse.
\VS{18}Les Philistins arrivèrent et se répandirent dans la vallée des Rephaïm.
\VS{19}Alors David consulta Yahweh, en disant : Monterai-je contre les Philistins ? Les livreras-tu entre mes mains ? Et Yahweh parla à David : Monte, car certainement je livrerai les Philistins entre tes mains.
\VS{20}Alors David vint à Baal-Peratsim, où il les battit. Puis il dit : Yahweh a dispersé mes ennemis devant moi, comme des eaux qui s'écoulent. C'est pourquoi il nomma ce lieu-là Baal-Peratsim.
\VS{21}Ils laissèrent là leurs faux dieux que David et ses gens emportèrent.
\VS{22}Les Philistins montèrent encore une autre fois, et se répandirent dans la vallée des Rephaïm.
\VS{23}David consulta Yahweh. Et Yahweh dit : Tu ne monteras pas ; contourne-les par-derrière, et tu les atteindras vis-à-vis des mûriers.
\VS{24}Quand tu entendras un bruit comme des gens qui marchent au sommet des mûriers, alors hâte-toi, car c'est Yahweh qui sort devant toi pour battre l'armée des Philistins.
\VS{25}David fit ce que Yahweh lui avait ordonné, et il battit les Philistins depuis Guéba jusqu'à Guézer.
\Chap{6}
\TextTitle{Désobéissance dans le transport de l'arche\FTNTT{1 Ch. 13:1-14.}}
\VerseOne{}David rassembla encore toute l'élite d'Israël, au nombre de trente mille hommes.
\VS{2}Puis David se leva, ainsi que tout le peuple qui était avec lui, et se mit en marche de Baalé-Juda, pour faire monter de là l'arche de Dieu, qui est appelée du Nom, du Nom de Yahweh des armées, qui siège entre les chérubins.
\VS{3}Ils mirent l'arche\FTNT{L'arche ne devait être portée que par les Lévites. Les meilleures intentions pour le service de Yahweh ne suffisent pas pour que le Seigneur nous agrée. Nous devons nous conformer à la Parole de Dieu (1 R. 18:36-39).} de Dieu sur un char tout neuf, et l'emmenèrent de la maison d'Abinadab qui était sur la colline ; Uzza et Achjo, fils d'Abinadab, conduisaient le char neuf.
\VS{4}Ils l'emportèrent donc de la maison d'Abinadab sur la colline ; et Achjo allait devant l'arche.
\VS{5}David et toute la maison d'Israël jouaient devant Yahweh de toutes sortes d'instruments faits de bois de cyprès, de harpes, de luths, de tambourins, de sistres et de cymbales.
\VS{6}Quand ils furent arrivés à l'aire de Nacon, Uzza étendit la main vers l'arche de Dieu et la saisit parce que les bœufs la faisaient pencher.
\VS{7}La colère de Yahweh s'enflamma contre Uzza et Dieu le frappa là à cause de sa faute. Il mourut là, près de l'arche de Dieu.
\VS{8}David fut irrité de ce que Yahweh avait fait une brèche en la personne d'Uzza. C'est pourquoi on a appelé ce lieu jusqu'à ce jour Pérets-Uzza.
\VS{9}David eut peur de Yahweh en ce jour-là, et il dit : Comment l'arche de Yahweh entrerait-elle chez moi ?
\VS{10}David ne voulut pas déposer l'arche de Yahweh chez lui dans la cité de David, mais il la fit conduire dans la maison d'Obed-Edom de Gath.
\VS{11}L'arche de Yahweh resta trois mois dans la maison d'Obed-Edom de Gath, et Yahweh bénit Obed-Edom et toute sa maison.
\TextTitle{Accueil de l'arche à Jérusalem\FTNTT{1 Ch. 15:26-16:1.}}
\VS{12}Puis on vint dire au roi David : Yahweh a béni la maison d'Obed-Edom et tout ce qui lui appartient, pour l'amour de l'arche de Dieu. Alors David s'y rendit, et il fit monter l'arche de Dieu depuis la maison d'Obed-Edom jusqu'à la cité de David, au milieu des réjouissances.
\VS{13}Et il arriva que quand ceux qui portaient l'arche de Dieu eurent fait six pas, on sacrifia des taureaux et des béliers gras.
\VS{14}David dansait de toute sa force devant Yahweh, et il était ceint d'un éphod de lin.
\VS{15}Ainsi, David et toute la maison d'Israël firent monter l'arche de Yahweh avec des cris de joie et au son du shofar.
\VS{16}Comme l'arche de Yahweh entrait dans la cité de David, Mical, fille de Saül, regardait par la fenêtre, et voyant le roi David sauter et danser devant Yahweh, elle le méprisa en son cœur.
\VS{17}Ils amenèrent l'arche de Yahweh, et la posèrent au milieu de la tente que David avait dressée pour elle ; et David offrit des holocaustes et des sacrifices d'offrande de paix\FTNT{Voir commentaire en Lé. 3:1.} devant Yahweh.
\VS{18}Quand David eut achevé d'offrir des holocaustes et des sacrifices d'offrande de paix, il bénit le peuple au Nom de Yahweh des armées.
\VS{19}Et il partagea à tout le peuple, à toute la multitude d'Israël, tant aux hommes qu'aux femmes, à chacun un pain, une portion de viande, un gâteau de raisins, et une ration de vin. Puis tout le peuple s'en alla, chacun dans sa maison.
\VS{20}David s'en retourna pour bénir aussi sa maison, et Mical, fille de Saül, sortit à sa rencontre. Elle dit : Quel honneur s'est fait aujourd'hui le roi d'Israël, en se découvrant aux yeux des servantes et de ses serviteurs, comme se découvrirait un homme de néant sans en avoir honte !
\VS{21}David répondit à Mical : C'est devant Yahweh, qui m'a choisi plutôt que ton père et toute sa maison pour m'établir chef sur le peuple de Yahweh, sur Israël, c'est devant Yahweh que je me suis réjoui.
\VS{22}Je me rendrai encore plus insignifiant que je n'ai été cette fois, et je m'estimerai encore moins à mes propres yeux ; malgré cela, je serai en honneur auprès des servantes dont tu parles.
\VS{23}Or Mical, fille de Saül, n'eut point d'enfants jusqu'au jour de sa mort.
\Chap{7}
\TextTitle{David veut construire une maison à Yahweh\FTNTT{1 Ch. 17:1-2.}}
\VerseOne{}Et il arriva, lorsque le roi fut établi dans sa maison, et que Yahweh lui eut donné du repos de tous ses ennemis qui l'entouraient,
\VS{2}qu'il dit à Nathan le prophète : Regarde maintenant ! J'habite dans une maison de cèdres, et l'arche de Dieu habite sous des tapis\FTNT{Dans la plupart des versions, on a traduit ce mot par « tente », alors que le terme hébreu est « yeriy'ah », ce qui signifie « rideau », « drap », « tapis ». Voir Ex. 26.}.
\VS{3}Alors Nathan répondit au roi : Va, fais tout ce qui est dans ton cœur, car Yahweh est avec toi.
\TextTitle{Yahweh traite alliance avec David et sa postérité\FTNTT{1 Ch. 17:3-15.}}
\VS{4}Mais il arriva cette nuit-là que la parole de Yahweh fut adressée à Nathan, en disant :
\VS{5}Va, et dis à David, mon serviteur : Ainsi parle Yahweh : Me bâtirais-tu une maison afin que j'y habite ?
\VS{6}Puisque je n'ai point habité dans une maison depuis le jour où j'ai fait monter les enfants d'Israël hors d'Egypte jusqu'à ce jour ; mais j'ai marché ça et là sous une tente et dans un tabernacle.
\VS{7}Partout où j'ai marché avec tous les enfants d'Israël, ai-je dit un seul mot à quelqu'une des tribus d'Israël à qui j'avais ordonné de paître mon peuple d'Israël, ai-je dit : Pourquoi ne me bâtissez-vous pas une maison de cèdres ?
\VS{8}Maintenant tu diras à David, mon serviteur : Ainsi parle Yahweh des armées : Je t'ai pris d'une cabane, d'auprès des brebis, afin que tu sois le conducteur de mon peuple, Israël ;
\VS{9}j'ai été avec toi partout où tu as marché, j'ai exterminé tous tes ennemis devant toi, et j'ai rendu ton nom grand, comme le nom des grands qui sont sur la terre ;
\VS{10}j'ai établi une demeure à mon peuple, à Israël, et je l'ai planté pour qu'il y habite et ne soit plus agité, pour que les méchants ne l'affligent plus comme auparavant,
\VS{11}et comme du temps où j'avais établi des juges sur mon peuple d'Israël. Je t'ai accordé du repos face à tous tes ennemis. Et Yahweh t'annonce qu'il te bâtira une maison.
\VS{12}Quand tu seras endormi avec tes pères, je susciterai après toi, ton fils, qui sera sorti de tes entrailles, et j'affermirai son règne.
\VS{13}Ce sera lui qui bâtira une maison à mon Nom, et j'affermirai pour toujours le trône de son règne\FTNT{Le royaume millénaire était promis à David et à sa postérité. Il fut proclamé par Jean-Baptiste (Mt. 3:1-12), le Messie (Mt. 4:17) et les apôtres (Mt. 10:5-7) comme étant proche. Présentement, le royaume de Dieu se manifeste par la vie sanctifiée des saints en Christ (Lu. 17:20 ; Jn. 3:1-8 ; Ro. 14:17). Il n'apparaîtra pas de manière visible avant la «moisson», c'est-à-dire le jugement des nations (Mt. 13:39-50). En effet, ce n'est qu'après cette moisson que le royaume sera installé ici-bas, lorsque le Messie rétablira la monarchie et la dynastie de David en sa propre personne. Il rassemblera alors les enfants d'Israël dispersés dans le monde entier et établira sa domination sur toute la terre pendant mille ans. Ce royaume sera remis au Père par le Messie après avoir vaincu le dernier ennemi, c'est-à-dire la mort (1 Co. 15:24-26). De ce fait, personne ne mourra pendant le millénium. Toutes les nations monteront tous les ans à Jérusalem pour adorer Yahweh et célébrer la fête des tabernacles qui sera restaurée (Za. 14). Le gouvernement théocratique en Israël sera alors restauré (Es. 1:26).}.
\VS{14}Je serai pour lui un père, et il sera pour moi un fils. S'il fait le mal, je le châtierai avec une verge d'hommes et avec des plaies des fils des hommes ;
\VS{15}mais ma grâce ne se retirera point de lui, comme je l'ai retirée de Saül, que j'ai ôté de devant toi.
\VS{16}Ainsi ta maison et ton règne seront assurés à jamais devant tes yeux, et ton trône sera pour toujours affermi.
\VS{17}Nathan rapporta à David toutes ces paroles et toute cette vision.
\TextTitle{Louange et reconnaissance de David envers Yahweh\FTNTT{1 Ch. 17:16-27.}}
\VS{18}Alors le roi David alla se présenter devant Yahweh, et dit : Qui suis-je, Seigneur Yahweh, et quelle est ma maison, que tu m'aies fait arriver au point où je suis ?
\VS{19}C'est encore peu de choses à tes yeux, ô Seigneur Yahweh ! Car tu as même parlé sur la maison de ton serviteur pour les temps éloignés. Est-ce là la manière d'agir des hommes, ô Seigneur Yahweh ?
\VS{20}Et que pourrait dire de plus David ? Car, Seigneur Yahweh, tu connais ton serviteur !
\VS{21}Tu as fait toutes ces grandes choses pour l'amour de ta parole, et selon ton cœur, pour les révéler à ton serviteur.
\VS{22}C'est pourquoi tu t'es montré grand, ô Yahweh Dieu ! Car nul n'est semblable à toi, et il n'y a point d'autre Dieu que toi, d'après tout ce que nous avons entendu de nos oreilles.
\VS{23}Et qui est comme ton peuple, comme Israël, la seule nation de la terre que Dieu est venu racheter pour en faire son peuple, y mettre son Nom et pour accomplir dans ton pays, devant ton peuple que tu t'es racheté d'Egypte, des choses grandes et terribles contre les nations et contre leurs dieux ?
\VS{24}Tu as affermi ton peuple d'Israël pour qu'il soit ton peuple pour toujours ; et toi, Yahweh, tu es devenu son Dieu.
\VS{25}Maintenant donc, ô Yahweh Dieu, confirme pour toujours la parole que tu as prononcée sur ton serviteur et sur sa maison, et agis selon ta parole.
\VS{26}Que ton Nom soit à jamais glorifié, et que l'on dise : Yahweh des armées est le Dieu d'Israël ! Et que la maison de David, ton serviteur, demeure stable devant toi !
\VS{27}Car toi, Yahweh des armées, Dieu d'Israël, tu as révélé ces choses à l'oreille de ton serviteur, en disant : Je te bâtirai une maison ! C'est pourquoi ton serviteur a pris courage pour t'adresser cette prière.
\VS{28}Maintenant, Seigneur Yahweh, tu es Dieu, tes paroles sont vérité, et tu as promis cette grâce à ton serviteur.
\VS{29}Veuille donc bénir la maison de ton serviteur, afin qu'elle soit éternellement devant toi ! Car c'est toi, Seigneur Yahweh, qui a parlé, et par ta bénédiction la maison de ton serviteur sera comblée de bénédictions éternellement.
\Chap{8}
\TextTitle{Yahweh donne à David la victoire sur ses ennemis\FTNTT{1 Ch. 18:1-17.}}
\VerseOne{}Après cela, il arriva que David battit les Philistins et les humilia, et il prit Métheg-Amma de la main des Philistins.
\VS{2}Il battit aussi les Moabites, et les mesura au cordeau, en les faisant coucher par terre ; il en mesura deux cordeaux pour les faire mourir, et un plein cordeau pour leur laisser la vie. Et les Moabites furent assujettis à David, et lui payèrent un tribut.
\VS{3}David battit aussi Hadadézer, fils de Rehob, roi de Tsoba, lorsqu'il alla rétablir sa domination sur le fleuve de l'Euphrate.
\VS{4}David lui prit mille sept cents cavaliers, et vingt mille hommes de pied ; il coupa les jarrets aux chevaux de tous les chars, et ne conserva que cent attelages.
\VS{5}Les Syriens de Damas vinrent au secours d'Hadadézer, roi de Tsoba, et David battit vingt-deux mille Syriens.
\VS{6}David mit des garnisons dans la Syrie de Damas. Et les Syriens furent assujettis à David, et lui payèrent un tribut. Yahweh protégeait David partout où il allait.
\VS{7}Et David prit les boucliers d'or qui étaient aux serviteurs d'Hadadézer, et les apporta à Jérusalem.
\VS{8}Le roi David emporta aussi une grande quantité d'airain de Béthach, et de Bérothaï, villes d'Hadadézer.
\VS{9}Thoï, roi de Hamath, apprit que David avait battu toute l'armée d'Hadadézer,
\VS{10}et il envoya Joram, son fils, vers le roi David, pour le saluer et pour le féliciter d'avoir fait la guerre contre Hadadézer et de l'avoir battu. Car Hadadézer était continuellement en guerre avec Thoï. Joram apporta des vases d'argent, des vases d'or, et des vases d'airain.
\VS{11}Le roi David les consacra à Yahweh, avec l'argent et l'or qu'il avait déjà consacrés du butin de toutes les nations qu'il s'était assujetties,
\VS{12}de la Syrie, de Moab, des fils d'Ammon, des Philistins, d'Amalek, et du butin d'Hadadézer, fils de Rehob, roi de Tsoba.
\VS{13}Au retour de la défaite des Syriens, David se fit encore un nom, en battant dans la vallée du sel dix-huit mille Edomites.
\VS{14}Il mit des garnisons dans Edom, il mit des garnisons dans tout Edom. Et tout Edom fut assujetti à David. Yahweh protégeait David partout où il allait.
\VS{15}Ainsi David régna sur tout Israël, et il faisait droit et justice à tout son peuple.
\VS{16}Joab, fils de Tseruja, commandait l'armée ; Josaphat, fils d'Achilud, était archiviste ;
\VS{17}Tsadok, fils d'Achithub, et Achimélec, fils d'Abiathar, étaient sacrificateurs ; Seraja était secrétaire ;
\VS{18}Benaja, fils de Jehojada, était chef des Kéréthiens et des Péléthiens ; et les fils de David étaient ministres d'Etat.
\Chap{9}
\TextTitle{Mephiboscheth à la table de David}
\VerseOne{}Alors David dit : Ne reste-t-il donc personne de la maison de Saül, afin que je lui fasse du bien pour l'amour de Jonathan ?
\VS{2}Il y avait dans la maison de Saül un serviteur nommé Tsiba, que l'on fit venir auprès de David. Le roi lui dit : Es-tu Tsiba ? Et il répondit : Je suis ton serviteur !
\VS{3}Le roi dit : N'y a-t-il plus personne de la maison de Saül, pour que j'use envers lui de la bonté de Dieu ? Tsiba répondit au roi : Il y a encore un des fils de Jonathan, qui est perclus des pieds.
\VS{4}Le roi lui dit : Où est-il ? Et Tsiba répondit au roi : Il est dans la maison de Makir, fils d'Ammiel, à Lodebar.
\VS{5}Alors le roi David l'envoya chercher dans la maison de Makir, fils d'Ammiel, à Lodebar.
\VS{6}Quand Mephiboscheth, fils de Jonathan, fils de Saül, vint auprès de David, il tomba sur sa face et se prosterna. David dit : Mephiboscheth ! Et il répondit : Voici ton serviteur.
\VS{7}David lui dit : Ne crains point, car certainement je te ferai du bien pour l'amour de Jonathan, ton père. Je te restituerai toutes les terres de Saül, ton père, et tu mangeras toujours du pain à ma table.
\VS{8}Il se prosterna, et dit : Qui suis-je, moi ton serviteur, pour que tu regardes un chien mort tel que moi ?
\VS{9}Le roi appela Tsiba, serviteur de Saül, et lui dit : Je donne au fils de ton maître tout ce qui appartenait à Saül et à toute sa maison.
\VS{10}Tu cultiveras pour lui ces terres, toi, tes fils, et tes serviteurs, et tu en recueilleras les fruits, afin que le fils de ton maître ait du pain à manger ; et Mephiboscheth, fils de ton maître, mangera toujours du pain à ma table. Or Tsiba avait quinze fils et vingt serviteurs.
\VS{11}Tsiba dit au roi : Ton serviteur fera tout ce que le roi, mon seigneur, ordonne à son serviteur. Et Mephiboscheth mangea à la table de David comme l'un des fils du roi.
\VS{12}Mephiboscheth avait un jeune fils, nommé Mica, et tous ceux qui demeuraient dans la maison de Tsiba étaient serviteurs de Mephiboscheth.
\VS{13}Mephiboscheth habitait à Jérusalem parce qu'il mangeait toujours à la table du roi. Il était boiteux des deux pieds.
\Chap{10}
\TextTitle{Double bataille contre les Ammonites et les Syriens}
\VerseOne{}Or il arriva après cela que le roi des fils d'Ammon mourût, et Hanun, son fils, régna à sa place.
\VS{2}Et David dit : J'userai de bonté envers Hanun, fils de Nachasch, comme son père en a usé envers moi. Ainsi David lui envoya ses serviteurs pour le consoler au sujet de son père. Lorsque les serviteurs de David arrivèrent dans le pays des fils d'Ammon,
\VS{3}les chefs des fils d'Ammon dirent à Hanun, leur maître : Penses-tu que ce soit pour honorer ton père que David t'envoie des consolateurs ? N'est-ce pas pour reconnaître exactement la ville et pour l'épier, afin de la détruire, que David envoie ses serviteurs auprès de toi ?
\VS{4}Alors Hanun saisit les serviteurs de David, et fit raser la moitié de leur barbe, et couper la moitié de leurs habits jusqu'aux hanches. Puis il les renvoya.
\VS{5}David en fut informé et envoya des gens à leur rencontre, car ces hommes étaient accablés de honte ; et le roi leur fit dire : Restez à Jéricho jusqu'à ce que votre barbe ait repoussé, et revenez ensuite.
\VS{6}Les fils d'Ammon, voyant qu'ils s'étaient rendus odieux à David, firent enrôler à leur solde vingt mille hommes de pied chez les Syriens de Beth-Rehob, et chez les Syriens de Tsoba, mille hommes chez le roi de Maaca, et douze mille hommes chez les gens de Tob.
\VS{7}David l'ayant appris, envoya Joab et toute l'armée, les hommes les plus vaillants.
\VS{8}Les fils d'Ammon sortirent et se rangèrent en bataille à l'entrée de la porte ; les Syriens de Tsoba de Rehob, et les hommes de Tob et de Maaca étaient à part dans la campagne.
\VS{9}Joab, voyant que leur armée était tournée contre lui devant et derrière, choisit alors des hommes d'élite parmi tous ceux d'Israël, et les rangea contre les Syriens ;
\VS{10}et il donna le commandement du reste du peuple à Abischaï, son frère, pour le ranger en bataille contre les fils d'Ammon.
\VS{11}Il dit : Si les Syriens sont plus forts que moi, tu viendras à mon secours ; et si les fils d'Ammon sont plus forts que toi, j'irai te secourir.
\VS{12}Sois vaillant, et portons-nous vaillamment pour notre peuple et pour les villes de notre Dieu, et que Yahweh fasse ce qu'il lui semblera bon !
\VS{13}Alors Joab et le peuple qui était avec lui s'approchèrent pour livrer bataille aux Syriens, et ils s'enfuirent devant lui.
\VS{14}Quand les fils d'Ammon virent que les Syriens avaient pris la fuite, ils s'enfuirent aussi devant Abischaï et rentrèrent dans la ville. Joab s'éloigna des fils d'Ammon et revint à Jérusalem.
\VS{15}Les Syriens, voyant qu'ils avaient été battus par Israël, se rassemblèrent.
\VS{16}Hadarézer envoya chercher les Syriens qui étaient de l'autre côté du fleuve ; et ils arrivèrent à Hélam, et Schobac, chef de l'armée d'Hadarézer, les conduisait.
\VS{17}Cela fut rapporté à David, qui assembla tout Israël, passa le Jourdain, et vint à Hélam. Les Syriens se rangèrent en bataille contre David, et combattirent contre lui.
\VS{18}Mais les Syriens s'enfuirent devant Israël. Et David défit sept cents chars des Syriens et quarante mille cavaliers ; il frappa aussi Schobac, le chef de leur armée, qui mourut sur place.
\VS{19}Tous les rois soumis à Hadarézer, se voyant battus par Israël, firent la paix avec Israël et lui furent assujettis. Et les Syriens craignirent désormais de secourir les fils d'Ammon.
\Chap{11}
\TextTitle{Péché de David avec Bath-Schéba}
\VerseOne{}Et il arriva, l'année suivante, au temps où les rois partaient en guerre, que David envoya Joab, avec ses serviteurs et tout Israël, pour détruire les fils d'Ammon et assiéger Rabba. Mais David resta à Jérusalem\FTNT{Au lieu d'aller en guerre et de diriger les troupes, David resta à Jérusalem. Cette négligence l'a conduit à la convoitise, à l'adultère et au meurtre d'Urie. La distraction peut conduire à la mort. Il y a un temps pour toutes choses (Ec. 3).}.
\VS{2}Et il arriva, sur le soir, que David se leva de sa couche ; et comme il se promenait sur le toit de la maison royale, il aperçut de là une femme qui se baignait, et cette femme était très belle de figure.
\VS{3}David envoya demander qui était cette femme, et on lui dit : N'est-ce pas Bath-Schéba, fille d'Eliam, femme d'Urie, le Héthien ?
\VS{4}Et David envoya des messagers pour la chercher. Elle vint vers lui, et il coucha avec elle. Après s'être purifiée de sa souillure, elle retourna dans sa maison.
\VS{5}Cette femme devint enceinte, et elle fit dire à David : Je suis enceinte.
\VS{6}Alors David envoya dire à Joab : Envoie-moi Urie, le Héthien. Et Joab envoya Urie à David.
\VS{7}Urie se rendit auprès de David, qui l'interrogea sur l'état de Joab, sur l'état du peuple, et sur l'état de la guerre.
\VS{8}Puis David dit à Urie : Descends dans ta maison, et lave tes pieds. Urie sortit de la maison du roi, et on fit porter après lui un présent royal.
\VS{9}Mais Urie se coucha à la porte de la maison du roi, avec tous les serviteurs de son maître, et il ne descendit point dans sa maison.
\VS{10}On le rapporta à David, et on lui dit : Urie n'est pas descendu dans sa maison. David dit à Urie : N'arrives-tu pas de voyage ? Pourquoi n'es-tu pas descendu dans ta maison ?
\VS{11}Urie répondit à David : L'arche et Israël et Juda habitent sous des tentes, mon seigneur Joab et les serviteurs de mon seigneur campent aux champs, et moi j'entrerais dans ma maison pour manger et boire et pour coucher avec ma femme ! Tu es vivant, et ton âme est vivante, je ne ferai point une telle chose.
\VS{12}David dit à Urie : Reste ici encore aujourd'hui, et demain je te renverrai. Urie resta donc ce jour-là et le lendemain à Jérusalem.
\VS{13}David l'invita à manger et à boire en sa présence, et il l'enivra ; néanmoins le soir, Urie sortit pour dormir sur sa couche, avec tous les serviteurs de son maître, et il ne descendit point dans sa maison.
\VS{14}Le lendemain matin, David écrivit une lettre à Joab, et l'envoya par la main d'Urie.
\VS{15}Il écrivit en ces termes : Placez Urie à l'endroit où sera le plus fort de la bataille et éloignez-vous de lui, afin qu'il soit frappé et qu'il meure.
\VS{16}Joab, en observant la ville, plaça Urie à l'endroit qu'il savait défendu par de vaillants soldats.
\VS{17}Les hommes de la ville sortirent et combattirent contre Joab ; et quelques-uns du peuple qui étaient des serviteurs de David moururent, et Urie, le Héthien, mourut aussi.
\VS{18}Alors Joab envoya un messager à David pour lui faire savoir tout ce qui était arrivé dans ce combat.
\VS{19}Il donna cet ordre au messager : Quand tu auras achevé de raconter au roi tout ce qui est arrivé au combat,
\VS{20}peut-être se mettra-t-il en fureur et te dira : Pourquoi vous êtes-vous approchés de la ville pour combattre ? Ne savez-vous pas bien qu'on tire de dessus la muraille ?
\VS{21}Qui a tué Abimélec, fils de Jerubbéscheth ? N'est-ce pas une femme qui lança sur lui de dessus la muraille une pièce de meule de moulin, et n'en est-il pas mort à Thébets ? Pourquoi vous êtes-vous approchés de la muraille ? Alors tu lui diras : Ton serviteur Urie, le Héthien, est mort aussi.
\VS{22}Le messager partit. A son arrivée, il fit savoir à David tout ce pourquoi Joab l'avait envoyé.
\VS{23}Le messager dit à David : Ces gens ont été plus forts que nous; ils avaient fait une sortie contre nous dans les champs, mais nous les avons repoussés jusqu'à l'entrée de la porte ;
\VS{24}les archers ont tiré sur tes serviteurs du haut de la muraille, et plusieurs des serviteurs du roi ont été tués, ton serviteur Urie, le Héthien, est mort aussi.
\VS{25}David dit au messager : Tu diras ainsi à Joab : Ne sois point peiné de cette affaire, car l'épée dévore tantôt l'un, tantôt l'autre ; attaque vigoureusement la ville, et détruis-la. Et toi, encourage-le !
\VS{26}La femme d'Urie apprit qu'Urie, son mari, était mort, et elle pleura son mari.
\VS{27}Quand le deuil fut passé, David l'envoya chercher et la recueillit dans sa maison. Elle devint sa femme, et lui enfanta un fils. Ce que David avait fait était mal aux yeux de Yahweh.
\Chap{12}
\TextTitle{Le prophète Nathan envoyé pour reprendre David}
\VerseOne{}Yahweh envoya Nathan vers David. Nathan vint à lui, et lui dit : Il y avait deux hommes dans une ville, l'un riche et l'autre pauvre.
\VS{2}Le riche avait des brebis et des bœufs en très grand nombre.
\VS{3}Le pauvre n'avait rien du tout sauf une petite brebis, qu'il avait achetée ; il la nourrissait et elle grandissait chez lui avec ses enfants ; elle mangeait de son pain, buvait dans sa coupe, dormait sur son sein et elle était comme sa fille.
\VS{4}Un voyageur arriva chez l'homme riche. Ce riche a épargné ses brebis et ses bœufs, pour préparer un repas au voyageur qui était venu chez lui ; il a pris la brebis du pauvre homme, et l'a apprêtée pour l'homme qui était venu chez lui.
\VS{5}Alors la colère de David s'enflamma violemment contre cet homme, et il dit à Nathan : Yahweh est vivant ! L'homme qui a fait cela mérite la mort.
\VS{6}Parce qu'il a fait cela et qu'il n'a pas épargné cette brebis, pour une brebis il en rendra quatre.
\VS{7}Alors Nathan dit à David : Tu es cet homme-là ! Ainsi parle Yahweh, le Dieu d'Israël : Je t'ai oint pour roi sur Israël, et je t'ai délivré de la main de Saül ;
\VS{8}je t'ai même donné la maison de ton maître, et les femmes de ton maître dans ton sein, et je t'ai donné la maison d'Israël, et de Juda. Et si cela avait été peu, j'y aurais encore ajouté.
\VS{9}Pourquoi donc as-tu méprisé la parole de Yahweh, en faisant ce qui est mal à ses yeux ? Tu as frappé de l'épée Urie, le Héthien ; tu as pris sa femme pour en faire ta femme, et tu l'as tué par l'épée des fils d'Ammon.
\VS{10}Maintenant, l'épée ne s'éloignera jamais de ta maison, parce que tu m'as méprisé, et que tu as pris la femme d'Urie, le Héthien, pour en faire ta femme.
\VS{11}Ainsi parle Yahweh : Voici, je vais faire sortir de ta propre maison le malheur contre toi, et je vais prendre sous tes yeux tes propres femmes pour les donner à un homme de ta maison, qui couchera avec elles à la vue de ce soleil.
\VS{12}Car tu as agi en secret ; mais moi, je le ferai en présence de tout Israël et à la face du soleil.
\TextTitle{Repentance de David}
\VS{13}David dit à Nathan : J'ai péché contre Yahweh ! Et Nathan dit à David : Yahweh passe par-dessus ton péché, tu ne mourras point.
\VS{14}Toutefois, parce qu'en commettant cela, tu as donné l'occasion aux ennemis de Yahweh de le blasphémer, à cause de cela le fils qui t'est né mourra certainement.
\VS{15}Et Nathan retourna dans sa maison. Yahweh frappa l'enfant que la femme d'Urie avait enfanté à David, et il devint gravement malade.
\VS{16}David pria Dieu pour l'enfant, et David jeûna ; et quand il rentra, il passa la nuit couché par terre.
\VS{17}Les anciens de sa maison se levèrent et vinrent vers lui pour le faire lever de terre ; mais il ne voulut point, et il ne mangea rien avec eux.
\VS{18}Et il arriva que l'enfant mourut le septième jour. Les serviteurs de David craignaient de lui annoncer que l'enfant était mort. Car ils disaient : Voici, quand l'enfant vivait encore, nous lui avons parlé, et il n'a pas écouté notre voix ; comment donc lui dirions-nous : L'enfant est mort ? Il s'affligera bien davantage.
\VS{19}David vit que ses serviteurs parlaient à voix basse, et il comprit que l'enfant était mort. David dit à ses serviteurs : L'enfant est-il mort ? Ils répondirent : Il est mort.
\VS{20}Alors David se leva de terre. Il se lava, s'oignit, et changea de vêtements ; il alla dans la maison de Yahweh, et se prosterna. De retour chez lui, il demanda à manger ; on mit de la viande devant lui et il mangea.
\VS{21}Ses serviteurs lui dirent : Qu'est-ce que tu fais ? Tu jeûnais et pleurais pour l'amour de l'enfant lorsqu'il vivait encore ; et maintenant que l'enfant est mort, tu te lèves et tu manges !
\VS{22}Mais il répondit : Quand l'enfant vivait encore, je jeûnais et pleurais, car je disais : Qui sait si Yahweh n'aura pas pitié de moi et si l'enfant ne vivra pas ?
\VS{23}Maintenant qu'il est mort, pourquoi jeûnerais-je ? Puis-je le faire revenir ? J'irai vers lui, mais il ne reviendra pas vers moi.
\TextTitle{Naissance de Salomon}
\VS{24}David consola sa femme Bath-Schéba, et il alla auprès d'elle et coucha avec elle. Elle lui enfanta un fils qu'il nomma Salomon et qui fut aimé de Yahweh.
\VS{25}Il le remit entre les mains de Nathan, le prophète, qui lui donna le nom de Jedidja, à cause de Yahweh.
\TextTitle{Le pays et le roi de Rabba livrés à Joab et David (1 Ch. 20:1-3)}
\VS{26}Joab combattait contre Rabba, qui appartenait aux fils d'Ammon, il s'empara de la ville royale,
\VS{27}et envoya des messagers à David pour lui dire : J'ai attaqué Rabba, et j'ai pris la ville des eaux ;
\VS{28}rassemble maintenant le reste du peuple, campe contre la ville, et prends-la, de peur que je ne m'en empare et que la gloire m'en soit attribuée.
\VS{29}David rassembla tout le peuple et marcha contre Rabba ; il l'attaqua et la prit.
\VS{30}Il enleva la couronne de dessus la tête de son roi ; elle pesait un talent d'or et était garnie de pierres précieuses. On la mit sur la tête de David, qui emporta de la ville un très grand butin.
\VS{31}Il fit sortir aussi le peuple qui s'y trouvait, et il les plaça sous des scies, des herses de fer, des haches de fer et le fit passer par un fourneau où l'on cuit les briques ; il traita ainsi toutes les villes des fils d'Ammon. Puis David retourna avec tout le peuple à Jérusalem.
\Chap{13}
\TextTitle{David subit les conséquences de son péché}
\VerseOne{}Or il arriva après cela qu'Absalom, fils de David, avait une sœur qui était belle et qui se nommait Tamar ; et Amnon, fils de David, l'aima.
\TextTitle{Inceste au sein de la famille royale}
\VS{2}Et Amnon fut si tourmenté qu'il tomba malade à cause de Tamar sa sœur, car elle était vierge ; et il paraissait trop difficile à Amnon d'obtenir la moindre chose d'elle.
\VS{3}Amnon avait un ami, nommé Jonadab, fils de Schimea, frère de David, et Jonadab était un homme très rusé.
\VS{4}Il lui dit : Fils de roi, pourquoi maigris-tu ainsi de jour en jour ? Ne veux-tu pas me le dire ? Amnon lui dit : J'aime Tamar, la sœur de mon frère, Absalom.
\VS{5}Jonadab lui dit : Couche-toi dans ton lit et fais le malade. Quand ton père viendra te voir, tu lui diras : Permets à Tamar, ma sœur, de venir pour me donner à manger ; qu'elle prépare un mets sous mes yeux, afin que je le voie et que je le prenne de sa main.
\VS{6}Amnon se coucha et fit le malade. Le roi vint le voir, et Amnon dit au roi : Je te prie, que ma sœur Tamar vienne faire deux beignets sous mes yeux, et que je les mange de sa main.
\VS{7}David envoya dire à Tamar dans la maison : Va dans la maison de ton frère Amnon, et prépare-lui quelque chose d'appétissant.
\VS{8}Tamar alla dans la maison de son frère Amnon, qui était couché. Elle prit de la pâte, la pétrit, et en fit devant lui des beignets et les fit cuire.
\VS{9}Puis elle prit la poêle, et elle les versa devant lui. Mais Amnon refusa d'en manger. Il dit : Faites sortir tous ceux qui sont auprès de moi. Et tout le monde se retira.
\VS{10}Alors Amnon dit à Tamar : Apporte-moi le mets dans la chambre, et que je le mange de ta main. Tamar prit les beignets qu'elle avait faits, et les apporta à Amnon, son frère dans la chambre.
\VS{11}Comme elle les lui présentait pour qu'il en mange, il se saisit d'elle et lui dit : Viens, couche avec moi, ma sœur !
\VS{12}Elle lui répondit : Non, mon frère, ne me déshonore pas, car cela ne se fait point en Israël ; ne commets pas cette infamie.
\VS{13}Et moi, où irais-je avec mon opprobre ? Et toi, tu serais comme l'un des infâmes en Israël. Maintenant, je te prie, parle au roi, et il ne s'opposera pas à ce que je sois ta femme.
\VS{14}Mais il ne voulut pas écouter sa parole ; il fut plus fort qu'elle, lui fit violence et coucha avec elle\FTNT{Le viol et l'inceste que commit Amnon, fils de David, sur Tamar, sa demi-sœur, furent les conséquences du péché de David avec Bath-Schéba.}.
\VS{15}Après cela, Amnon eut pour elle une très grande haine, en sorte que la haine qu'il lui portait était plus grande que l'amour qu'il avait eu pour elle. Ainsi, Amnon lui dit : Lève-toi, va-t'en !
\VS{16}Elle lui répondit : Tu n'as aucune raison de me faire ce mal, que de me chasser, ce mal est plus grand que l'autre que tu m'as fait.
\VS{17}Mais il ne voulut point l'écouter, et appelant le garçon qui le servait, il dit : Qu'on chasse cette femme loin de moi, qu'on la mette dehors. Et ferme la porte après elle !
\VS{18}Elle était habillée d'une tunique de couleurs ; car les filles du roi, qui étaient encore vierges, s'habillaient ainsi. Le serviteur d'Amnon la mit dehors, et ferma la porte après elle.
\VS{19}Alors Tamar répandit de la cendre sur sa tête, et déchira sa tunique de couleurs ; elle mit la main sur sa tête, et s'en alla en poussant des cris.
\VS{20}Et son frère Absalom lui dit : Ton frère, Amnon, a-t-il été avec toi ? Maintenant, ma sœur, tais-toi, c'est ton frère ; ne prends pas cette affaire à cœur. Et Tamar, désolée, demeura dans la maison d'Absalom, son frère.
\VS{21}Quand le roi David eut appris toutes ces choses, il fut très irrité.
\VS{22}Absalom ne parla ni en bien ni en mal avec Amnon ; mais il le prit en haine, parce qu'il avait déshonoré Tamar, sa sœur.
\TextTitle{Vengeance d'Absalom sur Amnon}
\VS{23}Et il arriva au bout de deux années entières, qu'Absalom ayant les tondeurs à Baal-Hatsor, près d'Ephraïm, invita tous les fils du roi.
\VS{24}Absalom alla vers le roi, et dit : Voici, ton serviteur a les tondeurs ; je te prie que le roi et ses serviteurs viennent avec ton serviteur.
\VS{25}Et le roi dit à Absalom : Non, mon fils, nous n'irons pas tous, de peur que nous ne te soyons à charge. Absalom le pressa ; mais le roi ne voulut point aller, et il le bénit.
\VS{26}Absalom dit : Permets au moins à Amnon, mon frère, de venir avec nous. Le roi lui répondit : Pourquoi irait-il ?
\VS{27}Absalom le pressa tellement qu'il laissa aller Amnon et tous les fils du roi avec lui.
\VS{28}Or Absalom avait donné cet ordre à ses serviteurs, en disant : Prenez bien garde, je vous prie, quand le cœur d'Amnon sera égayé par le vin et que je vous dirai : Frappez Amnon ! Tuez-le ; ne craignez point, n'est-ce pas moi qui vous l'ordonne ? Fortifiez-vous et portez-vous en vaillants hommes !
\VS{29}Les serviteurs d'Absalom traitèrent Amnon comme Absalom l'avait ordonné. Et tous les fils du roi se levèrent, montèrent chacun sur son mulet, et s'enfuirent.
\VS{30}Et il arriva, comme ils étaient en chemin, que le bruit parvint à David qu'Absalom avait tué tous les fils du roi, et qu'il n'en était pas resté un seul d'entre eux.
\VS{31}Le roi se leva, déchira ses vêtements, et se coucha par terre ; et tous ses serviteurs étaient là, avec leurs vêtements déchirés.
\VS{32}Jonadab, fils de Schimea, frère de David, prit la parole, et dit : Que mon seigneur ne dise point que tous les jeunes hommes, fils du roi, ont été tués, car seul Amnon est mort ; car c'était là le dessein d'Absalom, depuis le jour où Amnon a violé Tamar, sa sœur ; car il a été exécuté selon son commandement.
\VS{33}Maintenant donc, que le roi mon seigneur ne prenne point la chose à cœur, en disant que tous les fils du roi sont morts, car Amnon seul est mort.
\VS{34}Absalom prit la fuite. Or le jeune homme placé en sentinelle leva les yeux et regarda. Et voici, un grand peuple venait par le chemin qui était derrière lui, du côté de la montagne.
\VS{35}Jonadab dit au roi : Voici les fils du roi qui arrivent ! Ainsi se confirme ce que disait ton serviteur.
\VS{36}Comme il achevait de parler, voici, les fils du roi arrivèrent. Ils élevèrent la voix et pleurèrent ; le roi aussi et tous ses serviteurs versèrent d'abondantes larmes.
\TextTitle{Absalom s'enfuit loin de son père}
\VS{37}Absalom s'était enfui, et il alla chez Talmaï, fils d'Ammihur, roi de Gueschur\FTNT{Absalom s'était réfugié chez Talmaï, roi de Gueschur (Transjordanie, au nord de la Syrie), qui était le père de Maaca, sa mère (2 S. 3:3). Il est donc allé chez son grand-père maternel.}. Et David pleurait tous les jours son fils.
\VS{38}Absalom resta trois ans à Gueschur, où il était allé, après avoir pris la fuite.
\VS{39}Le roi David cessa de poursuivre Absalom, car il était consolé de la mort d'Amnon.
\Chap{14}
\TextTitle{Joab convainc le roi de faire revenir Absalom}
\VerseOne{}Alors Joab, fils de Tseruja, s'aperçut que le cœur du roi était pour Absalom.
\VS{2}Il envoya chercher à Tekoa une femme habile, et il lui dit : Fais semblant de te lamenter, et revêts des habits de deuil ; ne t'oins pas d'huile, mais sois comme une femme qui depuis longtemps pleure un mort.
\VS{3}Ensuite va vers le roi, et tu lui parleras de cette manière. Joab lui mit dans la bouche ce qu'elle devait dire.
\VS{4}La femme de Tekoa alla parler au roi. Elle tomba la face contre terre, se prosterna et dit : Ô roi, sauve-moi !
\VS{5}Le roi lui dit : Qu'as-tu ? Elle répondit : Certainement, je suis une femme veuve, et mon mari est mort !
\VS{6}Or ta servante avait deux fils ; ils se sont tous deux querellés dans les champs, et il n'y avait personne pour les séparer ; l'un a frappé l'autre et l'a tué.
\VS{7}Et voici, toute la famille s'est élevée contre ta servante, en disant : Donne-nous le meurtrier de son frère ! Nous voulons le faire mourir, pour la vie de son frère qu'il a tué ; et que nous exterminions même l'héritier ! Ils veulent ainsi éteindre le charbon vif qui me restait, pour ne laisser à mon mari ni nom ni survivant sur la face de la terre.
\VS{8}Le roi dit à la femme : Va-t-en dans ta maison, et je donnerai des ordres en ta faveur.
\VS{9}Alors la femme de Tekoa dit au roi : Mon seigneur et mon roi ! Que l'iniquité soit sur moi et sur la maison de mon père, et que le roi et son trône en soient innocents.
\VS{10}Et le roi répondit : Si quelqu'un parle contre toi, amène-le-moi, et jamais il ne lui arrivera de te toucher.
\VS{11}Et elle dit : Je te prie, que le roi se souvienne de Yahweh, son Dieu, afin que le vengeur de sang n'augmente pas la ruine et qu'on ne fasse pas périr mon fils. Et il répondit : Yahweh est vivant ! Il ne tombera pas à terre un seul des cheveux de ton fils.
\VS{12}La femme dit : Je te prie que ta servante dise un mot au roi, mon seigneur. Et il répondit : Parle !
\VS{13}La femme dit : Mais pourquoi as-tu pensé une chose comme celle-ci contre le peuple de Dieu ? Puisqu'en tenant ce discours, le roi se déclare coupable en ce qu'il n'a pas fait revenir celui qu'il a banni ?
\VS{14}Car nous mourrons certainement, et nous sommes comme l'eau versée sur la terre qu'on ne peut recueillir. Dieu n'ôte pas la vie, mais il médite les moyens de ne pas repousser loin de lui celui qui est banni de sa présence.
\VS{15}Maintenant, si je suis venue pour tenir ce discours au roi, mon seigneur, c'est parce que le peuple m'a effrayée. Et ta servante a dit : Je veux parler maintenant au roi ; peut-être que le roi fera ce que sa servante lui dira.
\VS{16}Oui, car le roi écoutera sa servante pour la délivrer de la main de celui qui veut nous exterminer, moi et mon fils, de l'héritage de Dieu.
\VS{17}Ta servante a dit : Que la parole du roi, mon seigneur, nous apporte du repos. Car le roi mon seigneur est comme un ange de Dieu, pour entendre le bien et le mal. Que Yahweh, ton Dieu, soit avec toi !
\VS{18}Le roi répondit, et dit à la femme : Je te prie, ne me cache rien de ce que je vais te demander. Et la femme dit : Que le roi mon seigneur parle !
\VS{19}Et le roi dit : La main de Joab n'est-elle pas avec toi dans tout ceci ? Et la femme répondit et dit : Ton âme vit, ô mon seigneur, qu'on ne saurait se détourner ni à droite ni à gauche de tout ce que dit le roi mon seigneur. C'est en effet ton serviteur Joab qui m'a donné des ordres et qui a mis dans la bouche de ta servante toutes ces paroles.
\VS{20}C'est ton serviteur Joab qui a fait que j'ai ainsi tourné ce discours. Mais mon seigneur est sage comme un ange de Dieu, pour savoir tout ce qui se passe sur la terre.
\TextTitle{Retour d'Absalom à Jérusalem}
\VS{21}Alors le roi dit à Joab : Voici, maintenant c'est toi qui as conduit cette affaire ; va donc, et fais revenir le jeune homme Absalom.
\VS{22}Et Joab tomba la face contre terre et se prosterna, et il bénit le roi. Puis il dit : Aujourd'hui, ton serviteur sait qu'il a trouvé grâce à tes yeux, ô roi mon seigneur, puisque le roi agit selon ce que son serviteur lui a dit.
\VS{23}Joab se leva et partit pour Gueschur, et il ramena Absalom à Jérusalem.
\VS{24}Mais le roi dit : Qu'il se retire dans sa maison, et qu'il ne voie point ma face. Et Absalom se retira dans sa maison, et ne vit point la face du roi.
\VS{25}Il n'y avait point d'homme dans tout Israël aussi renommé qu'Absalom pour sa beauté ; depuis la plante des pieds jusqu'au sommet de la tête, il n'y avait point en lui de défaut.
\VS{26}Et quand il faisait couper ses cheveux, or il arrivait tous les ans qu'il les faisait couper, parce que sa chevelure lui pesait trop, le poids de sa chevelure était de deux cents sicles, poids du roi.
\VS{27}Il naquit à Absalom trois fils, et une fille nommée Tamar, qui était une femme belle de figure.
\VS{28}Et Absalom demeura deux ans entiers à Jérusalem, sans voir la face du roi.
\VS{29}Absalom fit demander Joab, pour l'envoyer vers le roi ; mais Joab ne voulut pas venir vers lui ; il le fit demander encore pour la seconde fois ; mais Joab ne voulut point venir.
\VS{30}Absalom dit alors à ses serviteurs : Voyez le champ de Joab qui est à côté du mien ; il y a de l'orge ; allez et mettez-y le feu. Et les serviteurs d'Absalom mirent le feu au champ.
\VS{31}Alors Joab se leva et vint vers Absalom dans sa maison. Il lui dit : Pourquoi tes serviteurs ont-ils mis le feu à mon champ?
\VS{32}Et Absalom répondit à Joab : Voici, je t'ai fait dire : Viens ici, et je t'enverrai vers le roi, afin que tu lui dises : Pourquoi suis-je revenu de Gueschur ? Il vaudrait mieux pour moi que j'y fusse encore. Je désire maintenant voir la face du roi ; et s'il y a de l'iniquité en moi, qu'il me fasse mourir.
\VS{33}Joab alla vers le roi, et lui rapporta cela. Et le roi appela Absalom, qui vint vers lui et se prosterna le visage contre terre devant le roi. Le roi embrassa Absalom.
\Chap{15}
\TextTitle{Mauvaises intentions d'Absalom}
\VerseOne{}Or il arriva qu'après cela, Absalom se procura des chars et des chevaux, et il avait cinquante hommes qui couraient devant lui\FTNT{La révolte d'Absalom était une autre conséquence du péché de David avec Bath-Schéba.}.
\VS{2}Absalom se levait de bon matin et se tenait au bord du chemin de la porte. Et chaque fois qu'un homme ayant une contestation se rendait auprès du roi pour obtenir justice, Absalom l'appelait, et lui disait : De quelle ville es-tu ? Et il répondait : Ton serviteur est de l'une des tribus d'Israël.
\VS{3}Absalom lui disait : Vois, ta cause est bonne et droite ; mais personne de chez le roi ne t'écoutera.
\VS{4}Absalom disait encore : Qui m'établira juge dans le pays ? Tout homme qui aurait une contestation et un procès viendrait vers moi, et je lui ferais justice.
\VS{5}Et il arrivait aussi que quand quelqu'un s'approchait de lui pour se prosterner, il lui tendait sa main, le saisissait, et l'embrassait.
\VS{6}Absalom faisait ainsi à tous ceux d'Israël qui venaient vers le roi pour demander justice. Et Absalom gagnait les cœurs des hommes d'Israël.
\TextTitle{Conspiration d'Absalom}
\VS{7}Et il arriva qu'au bout de quarante ans, Absalom dit au roi : Permets que j'aille à Hébron, pour accomplir le vœu que j'ai fait à Yahweh.
\VS{8}Car quand ton serviteur demeurait à Gueschur en Syrie, il fit un vœu, en disant : Si Yahweh me ramène à Jérusalem, j'en témoignerai ma reconnaissance à Yahweh.
\VS{9}Et le roi lui répondit : Va en paix. Et Absalom se leva et s'en alla à Hébron.
\VS{10}Absalom envoya des espions dans toutes les tribus d'Israël, pour dire : Aussitôt que vous entendrez le son du shofar, vous direz : Absalom est établi roi à Hébron !
\VS{11}Deux cents hommes de Jérusalem, qui avaient été invités, s'en allèrent avec Absalom ; ils y allèrent en toute simplicité de coeur, ne sachant rien de cette affaire.
\VS{12}Pendant qu'Absalom offrait les sacrifices, il envoya chercher à la ville de Guilo, Achitophel, le Guilonite, conseiller de David. Il se forma une puissante conspiration, parce que le peuple était de plus en plus nombreux auprès d'Absalom.
\TextTitle{David fuit son fils Absalom}
\VS{13}Un messager se rendit auprès de David, et lui dit : Le cœur des hommes d'Israël s'est tourné vers Absalom.
\VS{14}Et David dit à tous ses serviteurs qui étaient avec lui à Jérusalem : Levez-vous, fuyons, car nous ne pourrons échapper à Absalom. Hâtez-vous de partir ; sinon, il ne tarderait pas à nous atteindre, et il nous précipiterait dans le malheur et frapperait la ville du tranchant de l'épée.
\VS{15}Les serviteurs du roi lui répondirent : Tes serviteurs feront tout ce que le roi, notre seigneur, voudra.
\VS{16}Le roi sortit, et toute sa maison le suivait, mais le roi laissa dix femmes, des concubines, pour garder la maison.
\VS{17}Le roi sortit, et tout le peuple le suivait, et ils s'arrêtèrent à Beth-Merkhak.
\VS{18}Tous ses serviteurs marchaient à côté de lui ; tous les Kéréthiens, tous les Péléthiens, et tous les Gathiens, qui étaient six cents hommes venus de Gath, pour être à sa suite, marchaient devant le roi.
\VS{19}Mais le roi dit à Ittaï de Gath : Pourquoi viendrais-tu aussi avec nous ? Retourne et reste avec le roi, car tu es étranger, et même tu vas retourner bientôt en ton lieu.
\VS{20}Tu es arrivé hier, et te ferais-je aujourd'hui errer çà et là avec nous ? Quant à moi, je m'en vais où je pourrai ! Retourne et emmène tes frères avec toi. Que la bonté et la vérité t'accompagnent !
\VS{21}Mais Ittaï répondit au roi, et dit : Yahweh est vivant, et le roi mon seigneur est vivant ! Quel que soit le lieu où le roi mon seigneur sera, soit pour mourir, soit pour vivre, ton serviteur y sera aussi.
\VS{22}David donc dit à Ittaï : Viens, et marche ! Alors Ittaï de Gath marcha avec tous ses gens et tous les enfants qui étaient avec lui.
\VS{23}Et tout le pays pleurait à grands cris et tout le peuple passait plus avant. Puis le roi passa le torrent de Cédron, et tout le peuple passa en face du chemin qui mène au désert.
\TextTitle{L'arche de l'alliance à Jérusalem}
\VS{24}Tsadok était aussi là, et avec lui tous les Lévites portant l'arche de l'alliance de Dieu ; et ils posèrent là l'arche de Dieu, et Abiathar montait, pendant que tout le peuple achevait de sortir de la ville.
\VS{25}Le roi dit à Tsadok : Rapporte l'arche de Dieu dans la ville. Si je trouve grâce aux yeux de Yahweh, il me ramènera, et il me fera voir l'arche et sa demeure.
\VS{26}Mais s'il dit : Je ne prends point de plaisir en toi ! Me voici, qu'il fasse de moi ce qui lui semblera bon.
\VS{27}Le roi dit encore au sacrificateur Tsadok : N'es-tu pas le voyant ? Retourne en paix dans la ville, avec Achimaats, ton fils, et Jonathan, fils d'Abiathar, vos deux fils.
\VS{28}Voyez, j'attendrai dans les plaines du désert, jusqu'à ce qu'on vienne m'apporter des nouvelles de votre part.
\VS{29}Ainsi, Tsadok et Abiathar rapportèrent l'arche de Dieu à Jérusalem, et ils y restèrent.
\VS{30}David monta par la montée des oliviers. Il montait en pleurant, la tête couverte, et marchait pieds nus ; tout le peuple qui était avec lui se couvrit aussi la tête, et il montait en pleurant.
\VS{31}Alors on vint dire à David : Achitophel est parmi ceux qui ont conspiré avec Absalom. Et David dit : Je te prie, ô Yahweh, abolis les conseils d'Achitophel !
\TextTitle{Huschaï, espion pour David dans la cour d'Absalom}
\VS{32}Et il arriva que quand David fut arrivé au sommet de la montagne, où il se prosterna devant Dieu, Huschaï, l'Arkien, vint au-devant de lui, la tunique déchirée et de la terre sur sa tête.
\VS{33}David lui dit : Tu me seras à charge si tu viens avec moi.
\VS{34}Et au contraire, tu anéantiras en ma faveur les conseils d'Achitophel, si tu retournes à la ville, et que tu dis à Absalom : Ô roi, je serai ton serviteur, comme je fus autrefois le serviteur de ton père ; mais maintenant je serai ton serviteur.
\VS{35}Les sacrificateurs Tsadok et Abiathar ne seront-ils pas là avec toi ? Tout ce que tu entendras de la maison du roi, tu le rapporteras aux sacrificateurs Tsadok et Abiathar.
\VS{36}Voici, ils ont là avec eux leurs deux fils, Achimaats, fils de Tsadok, et Jonathan, fils d'Abiathar ; c'est par eux que vous me ferez savoir tout ce que vous aurez entendu.
\VS{37}Huschaï, l'ami de David, retourna donc dans la ville, et Absalom entra à Jérusalem.
\Chap{16}
\TextTitle{Tsiba retrouve David en fuite}
\VerseOne{}Quand David eut un peu dépassé le sommet, voici, Tsiba, serviteur de Mephiboscheth, vint au-devant de lui avec deux ânes bâtés, sur lesquels il y avait deux cents pains, cent paquets de raisins secs, cent de fruits d'été, et une outre de vin.
\VS{2}Le roi dit à Tsiba : Que veux-tu faire de cela ? Et Tsiba répondit : Les ânes serviront de montures pour la maison du roi, le pain et les autres fruits d'été sont pour nourrir les jeunes gens, et le vin pour désaltérer ceux qui se seront fatigués dans le désert.
\VS{3}Le roi lui dit : Mais où est le fils de ton maître ? Et Tsiba répondit au roi : Voici, il est resté à Jérusalem, car il a dit : Aujourd'hui, la maison d'Israël me rendra le royaume de mon père.
\VS{4}Alors le roi dit à Tsiba : Voici, tout ce qui est à Mephiboscheth est à toi. Et Tsiba dit : Je me prosterne ! Que je trouve grâce à tes yeux, ô roi, mon seigneur !
\TextTitle{Schimeï maudit le roi David}
\VS{5}Le roi David était arrivé jusqu'à Bachurim. Et voici, il sortit de là un homme de la famille et de la maison de Saül, nommé Schimeï, fils de Guéra. Il s'avança en prononçant des malédictions,
\VS{6}il jeta des pierres contre David, contre tous ses serviteurs, et contre tout le peuple ; tous les hommes vaillants étaient à la droite et à la gauche du roi.
\VS{7}Schimeï parlait ainsi en le maudissant : Sors, sors, homme de sang, méchant homme !
\VS{8}Yahweh fait retomber sur toi tout le sang de la maison de Saül, à la place duquel tu régnais, et Yahweh a mis le royaume entre les mains de ton fils, Absalom ; et voilà, tu souffres le mal que tu as fait, parce que tu es un homme de sang !
\VS{9}Alors Abischaï, fils de Tseruja, dit au roi : Pourquoi ce chien mort maudit-il le roi, mon seigneur ? Permets que je m'avance et que je lui ôte la tête.
\VS{10}Mais le roi répondit : Qu'ai-je à faire avec vous, fils de Tseruja ? S'il maudit, c'est que Yahweh lui a dit : Maudis David ! Qui donc lui dira : Pourquoi agis-tu ainsi ?
\VS{11}Et David dit à Abischaï et à tous ses serviteurs : Voici, mon propre fils, qui est sorti de mes entrailles, en veut à ma vie ; à plus forte raison ce Benjamite ! Laissez-le, et qu'il maudisse, car Yahweh lui a parlé.
\VS{12}Peut-être Yahweh regardera mon affliction, et que Yahweh me rendra le bien au lieu des malédictions d'aujourd'hui.
\VS{13}David donc, et ses gens, continuèrent leur chemin. Et Schimeï marchait sur le flanc de la montagne vis-à-vis de lui, continuant à maudire, jetant des pierres contre lui et de la poussière en l'air.
\VS{14}Le roi David et tout le peuple qui était avec lui arrivèrent fatigués et là ils se rafraîchirent.
\TextTitle{Abominations d'Absalom à Jérusalem}
\VS{15}Absalom, et tout le peuple, les hommes d'Israël, étaient entrés dans Jérusalem ; et Achitophel était avec lui.
\VS{16}Quand Huschaï, l'Arkien, ami de David, fut arrivé auprès d'Absalom, il lui dit : Vive le roi ! Vive le roi !
\VS{17}Et Absalom dit à Huschaï : Est-ce donc là l'affection que tu as pour ton ami ? Pourquoi n'es-tu pas allé avec ton ami ?
\VS{18}Huschaï répondit à Absalom : Non, mais je serai à celui qui a été choisi par Yahweh, par ce peuple et par tous les hommes d'Israël, et je demeurerai avec lui.
\VS{19}D'ailleurs, qui servirai-je ? Ne sera-ce pas son fils ? Je serai ton serviteur, comme j'ai été le serviteur de ton père.
\VS{20}Absalom dit à Achitophel : Donnez un conseil sur ce que nous ferons.
\VS{21}Achitophel dit à Absalom : Va vers les concubines que ton père a laissées pour garder la maison ; ainsi tout Israël saura que tu t'es rendu odieux envers ton père, et les mains de tous ceux qui sont avec toi se fortifieront.
\VS{22}On dressa une tente pour Absalom sur le toit et Absalom alla vers les concubines de son père, aux yeux de tout Israël\FTNT{2 S. 12:11-12.}.
\VS{23}Les conseils que donnait Achitophel en ce temps-là étaient autant estimés que si l'on eût demandé la parole de Dieu. C'est ainsi qu'on considérait tous les conseils qu'Achitophel donnait, tant à David qu'à Absalom.
\Chap{17}
\TextTitle{Schimeï maudit le roi David}
\VerseOne{}Après cela, Achitophel dit à Absalom : Je choisirai maintenant douze mille hommes, et je me lèverai, et je poursuivrai David cette nuit.
\VS{2}Je l'atteindrai pendant qu'il est fatigué, et que ses mains sont affaiblies ; je l'épouvanterai tellement que tout le peuple qui est avec lui s'enfuira, et je frapperai seulement le roi ;
\VS{3}et je ramènerai à toi tout le peuple ; car l'homme que tu cherches vaut autant que si tous retournaient à toi ; ainsi tout le peuple sera en paix.
\VS{4}Cette parole plut à Absalom, et à tous les anciens d'Israël.
\VS{5}Cependant Absalom dit : Qu'on appelle maintenant aussi Huschaï, l'Arkien, et que nous entendions aussi son avis.
\VS{6}Huschaï vint vers Absalom et Absalom lui dit : Achitophel a donné un tel avis ; devons-nous faire ce qu'il a dit ou non ? Parle, toi aussi.
\VS{7}Alors Huschaï dit à Absalom : Cette fois, le conseil qu'Achitophel a donné n'est pas bon.
\VS{8}Huschaï dit encore : Tu connais ton père et ses gens, ce sont des hommes forts, et ils ont l'amertume dans l'âme comme une ourse des champs privée de ses petits. Ton père est un homme de guerre, il ne passera pas la nuit avec le peuple.
\VS{9}Voici, il est maintenant caché dans quelque fosse, ou dans quelque autre lieu ; et si, dès le commencement, il en est qui tombent sous leurs coups, on ne tardera pas à l'apprendre et l'on dira : Il y a une défaite parmi le peuple qui suit Absalom !
\VS{10}Alors le plus vaillant, celui-là même qui avait le cœur comme un lion, se découragera ; car tout Israël sait que ton père est un homme de cœur, et que ceux qui sont avec lui sont vaillants.
\VS{11}Je conseille donc que tout Israël se rassemble auprès de toi, depuis Dan jusqu'à Beer-Schéba, multitude pareille au sable qui est sur le bord de la mer, et qu'en personne tu marches au combat.
\VS{12}Alors nous viendrons à lui en quelque lieu que nous le trouvions, et nous nous jetterons sur lui, comme la rosée tombe sur la terre ; et il ne lui restera aucun de tous les hommes qui sont avec lui.
\VS{13}S'il se retire dans une ville, tout Israël portera des cordes vers cette ville-là, et nous la traînerons jusqu'au torrent, jusqu'à ce qu'on n'en trouve plus une pierre.
\VS{14}Alors Absalom et tous les hommes d'Israël dirent : Le conseil de Huschaï, l'Arkien, est meilleur que le conseil d'Achitophel. Car Yahweh avait résolu de dissiper le conseil d'Achitophel, qui était bon, afin de faire venir le mal sur Absalom.
\TextTitle{Huschaï avertit David du danger}
\VS{15}Alors Huschaï dit aux sacrificateurs Tsadok et Abiathar : Achitophel a donné tel et tel conseil à Absalom, et aux anciens d'Israël ; mais moi, j'ai conseillé telle et telle chose.
\VS{16}Maintenant donc, envoyez tout de suite informer David, en disant : Ne passe point la nuit dans les plaines du désert, mais va plus loin, de peur que le roi et tout le peuple qui est avec lui ne soient exposés au péril.
\VS{17}Jonathan et Achimaats se tenaient à En-Roguel (la fontaine du foulon). Une servante vint leur dire d'aller informer le roi David ; car ils n'osaient pas se montrer et entrer dans la ville.
\VS{18}Mais un garçon les aperçut, et le rapporta à Absalom. Et ils partirent tous deux en hâte et ils arrivèrent à Bachurim, à la maison d'un homme qui avait un puits dans sa cour, dans lequel ils descendirent.
\VS{19}La femme de cet homme prit une couverture, qu'elle étendit sur l'ouverture du puits, et y répandit dessus du grain pilé en sorte qu'on ne s'aperçut de rien.
\VS{20}Les serviteurs d'Absalom entrèrent dans la maison auprès de cette femme, et lui dirent : Où sont Achimaats et Jonathan ? La femme leur répondit : Ils ont passé le ruisseau. Ils cherchèrent, et ne les trouvant pas, ils retournèrent à Jérusalem.
\VS{21}Après leur départ, Achimaats et Jonathan remontèrent du puits et allèrent informer le roi David. Ils lui dirent : Levez-vous, et hâtez-vous de passer l'eau, car Achitophel a conseillé telle chose contre vous.
\VS{22}Alors David et tout le peuple qui était avec lui se levèrent et ils passèrent le Jourdain ; à la lumière du matin, il n'en manqua pas un qui n'eût passé le Jourdain.
\VS{23}Or Achitophel voyant qu'on n'avait point fait ce qu'il avait conseillé, fit seller son âne, se leva, et s'en alla en sa maison, dans sa ville. Après avoir donné des ordres à sa maison, il s'étrangla et mourut. On l'enterra dans le sépulcre de son père.
\TextTitle{Absalom et Israël en marche contre David}
\VS{24}David arriva à Mahanaïm. Et Absalom passa le Jourdain, lui et tous les hommes d'Israël avec lui.
\VS{25}Absalom établit Amasa sur l'armée, à la place de Joab. Or Amasa était fils d'un homme nommé Jithra, l'Israélite, qui était allé vers Abigaïl, fille de Nachasch, et soeur de Tseruja, mère de Joab.
\VS{26}Israël et Absalom campèrent dans le pays de Galaad.
\TextTitle{Mahanaïm bienveillant envers David}
\VS{27}Or il arriva qu'aussitôt que David fut arrivé à Mahanaïm, Schobi, fils de Nachasch de Rabba, des fils d'Ammon, Makir, fils d'Ammiel de Lodebar, et Barzillaï, le Galaadite de Roguelim,
\VS{28}apportèrent des lits, des bassins, des vases de terre, du froment, de l'orge, de la farine, du grain rôti, des fèves, des lentilles, des pois rôtis,
\VS{29}du miel, de la crème, des brebis, et des fromages de vache. Ils apportèrent ces choses à David et au peuple qui était avec lui, afin qu'ils mangent, car ils disaient : Ce peuple a dû souffrir de la faim, de la fatigue et de la soif dans le désert.
\Chap{18}
\TextTitle{Bataille dans la forêt d'Ephraïm ; instructions de David sur Absalom}
\VerseOne{}David fit le dénombrement du peuple qui était avec lui, et il établit sur eux des chefs de milliers et des chefs de centaines.
\VS{2}David envoya le peuple, un tiers sous le commandement de Joab, un tiers sous le commandement d'Abischaï, fils de Tseruja, frère de Joab, et un tiers sous le commandement d'Ittaï, de Gath. Et le roi dit au peuple : Moi aussi, je veux sortir avec vous.
\VS{3}Mais le peuple lui dit : Tu ne sortiras point ! Car si nous prenons la fuite, ce n'est pas sur nous que l'attention se portera ; et même quand la moitié d'entre nous y serait tuée, on n'y ferait pas attention ; mais toi, tu es comme dix mille de nous, et maintenant il vaut mieux que de la ville tu puisses venir à notre secours.
\VS{4}Le roi leur répondit : Je ferai ce qui est bon à vos yeux. Le roi s'arrêta donc à la place de la porte, pendant que tout le peuple sortait par centaines et par milliers.
\VS{5}Le roi donna cet ordre à Joab, à Abischaï, et à Ittaï, et dit : Epargnez-moi le jeune homme Absalom ! Et tout le peuple entendit ce que le roi commandait à tous les chefs au sujet d'Absalom.
\VS{6}Ainsi le peuple sortit dans les champs à la rencontre d'Israël, et la bataille eut lieu dans la forêt d'Ephraïm.
\VS{7}Là, le peuple d'Israël fut battu par les serviteurs de David, et il y eut en ce jour-là dans ce même lieu, une grande défaite de vingt mille hommes.
\VS{8}La bataille s'étendit sur toute la contrée, et la forêt dévora ce jour-là beaucoup plus de peuple que l'épée.
\TextTitle{Joab tue Absalom}
\VS{9}Absalom se retrouva devant les serviteurs de David. Il était monté sur un mulet. Le mulet entra sous les branches entrelacées d'un grand chêne, et la tête d'Absalom fut prise dans le chêne ; il demeura suspendu entre le ciel et la terre, et le mulet qui était sous lui passa outre.
\VS{10}Un homme ayant vu cela, le rapporta à Joab, et lui dit : Voici, j'ai vu Absalom suspendu à un chêne.
\VS{11}Et Joab répondit à l'homme qui lui rapportait cela : Tu l'as vu ! Pourquoi ne l'as-tu pas tué là, le jetant par terre ? Je t'aurais donné dix sicles d'argent et une ceinture.
\VS{12}Mais cet homme dit à Joab : Quand je pèserais dans ma main mille pièces d'argent, je ne mettrais pas ma main sur le fils du roi ; car nous avons entendu ce que le roi vous a ordonné, à toi, à Abischaï et à Ittaï, en disant : Prenez garde chacun au jeune homme Absalom !
\VS{13}Autrement j'aurais commis une lâcheté au péril de ma vie, car rien ne serait caché au roi, et toi-même tu te lèverais contre moi.
\VS{14}Joab répondit : Je ne m'attarderai pas auprès de toi ! Et il prit en sa main trois javelots, et les enfonça dans le cœur d'Absalom qui était encore vivant au milieu du chêne.
\VS{15}Puis dix jeunes hommes, qui portaient les armes de Joab, entourèrent Absalom, le frappèrent et le firent mourir\FTNT{La mort d'Absalom fut une conséquence du péché de David avec Bath-Schéba. Le péché a donc des conséquences graves et cause beaucoup de souffrances.}.
\VS{16}Alors Joab fit sonner la trompette ; et le peuple cessa de poursuivre Israël, parce que Joab le retint.
\VS{17}Ils prirent Absalom, le jetèrent dans la forêt dans une grande fosse, et mirent sur lui un très grand monceau de pierres. Tout Israël s'enfuit, chacun dans sa tente.
\VS{18}Or Absalom s'était fait ériger, de son vivant, un monument dans la vallée du roi ; car il disait : Je n'ai point de fils pour conserver la mémoire de mon nom. Et il donna son propre nom au monument, qu'on appelle encore aujourd'hui la place d'Absalom.
\TextTitle{David apprend la mort d'Absalom}
\VS{19}Et Achimaats, fils de Tsadok, dit : Laisse-moi courir, et porter au roi la bonne nouvelle que Yahweh lui a rendu justice en jugeant ses ennemis.
\VS{20}Joab lui répondit : Tu ne seras pas aujourd'hui porteur de bonnes nouvelles ; tu le seras un autre jour ; car aujourd'hui tu ne porterais pas de bonnes nouvelles, puisque le fils du roi est mort.
\VS{21}Et Joab dit à Cuschi : Va, et annonce au roi ce que tu as vu. Cuschi se prosterna devant Joab, puis il se mit à courir.
\VS{22}Achimaats, fils de Tsadok, dit encore à Joab : Quoi qu'il arrive, laisse-moi courir après Cuschi. Joab lui dit : Pourquoi veux-tu courir, mon fils, puisque tu n'as pas de bonnes nouvelles à apporter ?
\VS{23}Quoiqu'il arrive, je veux courir, reprit Achimaats. Et Joab lui dit : Cours ! Achimaats courut par le chemin de la plaine, et il devança Cuschi.
\VS{24}David était assis entre les deux portes. La sentinelle alla sur le toit de la porte vers la muraille ; elle leva les yeux et elle regarda. Et voici un homme qui courait tout seul.
\VS{25}Alors la sentinelle cria, et avertit le roi. Le roi dit : S'il est seul, il apporte des bonnes nouvelles. Et cet homme marchait incessamment et approchait.
\VS{26}Puis la sentinelle vit un autre homme qui courait ; et elle cria au portier : Voici un homme qui court tout seul. Le roi dit : Il apporte aussi des bonnes nouvelles.
\VS{27}La sentinelle dit : La manière de courir du premier me paraît celle d'Achimaats, fils de Tsadok. Et le roi dit : C'est un homme de bien, il vient quand il y a des bonnes nouvelles.
\VS{28}Achimaats cria, et il dit au roi : Tout va bien ! Et il se prosterna devant le roi, le visage contre terre, et dit : Béni soit Yahweh, ton Dieu, qui a livré les hommes qui levaient leurs mains contre le roi, mon seigneur !
\VS{29}Le roi dit : Le jeune homme Absalom se porte-t-il bien ? Achimaats lui répondit : J'ai vu s'élever un grand tumulte au moment où Joab envoya le serviteur du roi et moi ton serviteur ; mais je ne sais pas exactement ce que c'était.
\VS{30}Et le roi lui dit : Mets-toi là de côté. Et Achimaats se tint de côté.
\VS{31}Aussitôt arriva Cuschi. Et il dit : Que le roi, mon seigneur apprenne, ces bonnes nouvelles ! Aujourd'hui, Yahweh t'a rendu justice en jugeant tous ceux qui s'élevaient contre toi.
\VS{32}Le roi dit à Cuschi : Le jeune homme Absalom se porte-t-il bien ? Et Cuschi lui répondit : Que les ennemis du roi, mon seigneur, et tous ceux qui s'élèvent contre toi pour te faire du mal soient comme ce jeune homme !
\VS{33}Alors le roi, saisi d'émotion, monta à la chambre haute de la porte, et alla pleurer. Il disait ainsi en marchant : Mon fils Absalom ! Mon fils, mon fils Absalom ! Plaise à Dieu que je sois moi-même mort à ta place ! Absalom, mon fils, mon fils !
\Chap{19}
\TextTitle{Souffrance de David ; indignation de Joab}
\VerseOne{}Et on fit ce rapport à Joab : Voici, le roi pleure et se lamente à cause d'Absalom.
\VS{2}Ainsi, la victoire fut en ce jour-là changée en deuil pour tout le peuple, car en ce jour-là le peuple entendait dire : Le roi est affligé à cause de son fils.
\VS{3}Ce même jour, le peuple rentra dans la ville à la dérobée, comme l'auraient fait des gens honteux d'avoir pris la fuite dans la bataille.
\VS{4}Le roi s'était couvert le visage, et il criait à haute voix : Mon fils Absalom ! Absalom, mon fils, mon fils !
\VS{5}Joab entra dans la chambre où était le roi, et lui dit : Tu couvres aujourd'hui de confusion les faces de tous tes serviteurs, qui ont en ce jour sauvé ta vie, celle de tes fils et de tes filles, celle de tes femmes et de tes concubines.
\VS{6}Tu aimes ceux qui te haïssent, et tu hais ceux qui t'aiment, car tu montres aujourd'hui que tes chefs et tes serviteurs ne te sont rien ; et je sais maintenant que si Absalom vivait, et que nous tous fussions morts aujourd'hui, cela serait agréable à tes yeux.
\VS{7}Maintenant donc lève-toi, sors, et parle selon le coeur de tes serviteurs ! Car je jure par Yahweh que si tu ne sors pas, il ne restera pas un seul homme avec toi cette nuit ; et ce mal sera pire que tous ceux qui te sont arrivés depuis ta jeunesse jusqu'à présent.
\TextTitle{Retour du roi David à Jérusalem}
\VS{8}Alors le roi se leva et s'assit à la porte. On fit dire à tout le peuple : Voici, le roi est assis à la porte. Et tout le peuple vint devant le roi. Cependant, Israël s'était enfui, chacun dans sa tente.
\VS{9}Et dans toutes les tribus d'Israël, tout le peuple était en contestation, disant : Le roi nous a délivrés de la main de nos ennemis, c'est lui qui nous a sauvés de la main des Philistins, et maintenant il a dû fuir du pays devant Absalom.
\VS{10}Or Absalom, que nous avions oint pour roi sur nous, est mort dans la bataille. Maintenant donc, pourquoi ne parlez-vous pas de faire revenir le roi ?
\VS{11}Le roi David envoya dire aux sacrificateurs Tsadok et Abiathar : Parlez aux anciens de Juda, et dites-leur : Pourquoi seriez-vous les derniers à ramener le roi en sa maison ? Car les discours que tout Israël avait tenus étaient parvenus jusqu'au roi dans sa maison.
\VS{12}Vous êtes mes frères, vous êtes mes os et ma chair ; pourquoi seriez-vous les derniers à ramener le roi ?
\VS{13}Dites même à Amasa : N'es-tu pas mon os et ma chair ? Que Dieu me traite dans toute sa rigueur si tu ne deviens pas devant moi pour toujours chef de l'armée à la place de Joab !
\VS{14}Ainsi David fléchit le cœur de tous les hommes de Juda, comme s'ils n'eussent été qu'un seul homme ; et ils envoyèrent dire au roi : Reviens, toi, et tous tes serviteurs.
\VS{15}Le roi revint et arriva jusqu'au Jourdain ; et Juda se rendit jusqu'à Guilgal, pour aller à la rencontre du roi afin de lui faire repasser le Jourdain.
\VS{16}Et Schimeï, fils de Guéra, Benjamite, qui était de Bachurim, se hâta de descendre avec les hommes de Juda à la rencontre du roi David.
\VS{17}Il avait avec lui mille hommes de Benjamin, et Tsiba, serviteur de la maison de Saül, ses quinze enfants, et ses vingt serviteurs étaient aussi avec lui. Ils passèrent le Jourdain en présence du roi.
\VS{18}Le bateau, mis à la disposition du roi, faisait la traversée pour transporter sa maison ; et au moment où le roi allait passer le Jourdain, Schimeï, fils de Guéra, se prosterna devant lui.
\VS{19}Et il dit au roi : Que mon seigneur ne m'impute pas mon iniquité, et ne se souvienne pas de ce que ton serviteur a fait de mal le jour où le roi mon seigneur sortait de Jérusalem, et que le roi ne le prenne point à cœur !
\VS{20}Car ton serviteur sait qu'il a péché. Et voici, je viens aujourd'hui le premier de toute la maison de Joseph à la rencontre du roi, mon seigneur.
\VS{21}Mais Abischaï, fils de Tseruja, répondit et dit : A cause de cela, ne fera-t-on pas mourir Schimeï, puisqu'il a maudit l'oint de Yahweh ?
\VS{22}Et David dit : Qu'ai-je à faire avec vous, fils de Tseruja ? Et pourquoi vous montrez-vous aujourd'hui mes adversaires ? Ferait-on mourir aujourd'hui quelqu'un en Israël ? Ne sais-je donc pas que je règne aujourd'hui sur Israël ?
\VS{23}Et le roi dit à Schimeï : Tu ne mourras point ! Et le roi le lui jura.
\VS{24}Après cela, Mephiboscheth, fils de Saül, descendit aussi à la rencontre du roi. Il n'avait point lavé ses pieds, ni fait sa barbe, ni lavé ses vêtements, depuis que le roi s'en était allé, jusqu'au jour où il revenait en paix.
\VS{25}Il se trouva donc au-devant du roi comme il entrait dans Jérusalem, et le roi lui dit : Pourquoi n'es-tu pas venu avec moi, Mephiboscheth ?
\VS{26}Et il lui répondit : Ô roi, mon seigneur, mon serviteur m'a trompé, car ton serviteur qui est boiteux avait dit : Je ferai seller mon âne, je monterai dessus, et j'irai avec le roi.
\VS{27}Et il a calomnié ton serviteur auprès du roi, mon seigneur. Mais le roi mon seigneur est comme un ange de Dieu. Fais donc ce qui semblera bon à tes yeux.
\VS{28}Car bien que tous ceux de la maison de mon père n'ont été que des gens dignes de mort devant le roi mon seigneur ; cependant tu as mis ton serviteur parmi ceux qui mangent à ta table. Quel droit puis-je encore avoir, pour me plaindre encore au roi ?
\VS{29}Et le roi lui dit : Pourquoi toutes ces paroles ? Je l'ai dit : Toi et Tsiba, vous partagerez les terres.
\VS{30}Et Mephiboscheth dit au roi : Qu'il prenne même tout, puisque le roi mon seigneur rentre en paix dans sa maison.
\VS{31}Barzillaï, le Galaadite, descendit de Roguelim, et passa le Jourdain avec le roi, pour l'accompagner jusqu'au-delà du Jourdain.
\VS{32}Barzillaï était très vieux, âgé de quatre-vingts ans. Il avait nourri le roi pendant qu'il avait séjourné à Mahanaïm, car c'était un homme fort riche.
\VS{33}Le roi dit à Barzillaï : Viens avec moi, je te nourrirai chez moi à Jérusalem.
\VS{34}Mais Barzillaï répondit au roi : Combien d'années vivrai-je encore pour que je monte avec le roi à Jérusalem ?
\VS{35}Je suis aujourd'hui âgé de quatre-vingts ans. Puis-je encore discerner ce qui est bon de ce qui est mauvais ? Ton serviteur peut-il savourer ce qu'il mange et ce qu'il boit ? Puis-je encore entendre la voix des chanteurs et des chanteuses ? Et pourquoi ton serviteur serait-il encore à charge à mon seigneur, le roi ?
\VS{36}Ton serviteur ira un peu au-delà du Jourdain avec le roi. Pourquoi le roi voudrait-il me donner une telle récompense ?
\VS{37}Je te prie que ton serviteur s'en retourne, et que je meure dans ma ville, près du sépulcre de mon père et de ma mère ! Mais voici ton serviteur Kimham, passera avec le roi mon seigneur ; fais-lui ce qui semblera bon à tes yeux.
\VS{38}Le roi dit : Que Kimham passe avec moi, et je lui ferai ce qui sera bon à tes yeux ; et tout ce que tu voudras de moi, je te l'accorderai.
\VS{39}Tout le peuple passa donc le Jourdain avec le roi. Puis le roi embrassa Barzillaï et le bénit. Et Barzillaï retourna dans sa demeure.
\TextTitle{Juda et Israël se disputent le roi}
\VS{40}De là, le roi passa à Guilgal, et Kimham passa avec lui. Ainsi, tout le peuple de Juda, et même la moitié du peuple d'Israël ramenèrent le roi.
\VS{41}Mais voici, tous les hommes d'Israël vinrent vers le roi, et lui dirent : Pourquoi nos frères, les hommes de Juda, t'ont-ils enlevé, et ont-ils fait passer le Jourdain au roi et à sa maison, et à tous les gens de David ?
\VS{42}Alors tous les hommes de Juda répondirent aux hommes d'Israël : Parce que le roi nous est plus proche ; pourquoi vous fâchez-vous de cela ? Avons-nous vécu aux dépens du roi ? Nous a-t-il fait des présents ?
\VS{43}Les hommes d'Israël répondirent aux hommes de Juda, et dirent : Le roi nous appartient dix fois autant, et David même plus qu'à vous. Pourquoi nous avez-vous méprisés ? N'avons-nous pas parlé les premiers de ramener notre roi ? Mais les hommes de Juda parlèrent avec plus de violence que les hommes d'Israël.
\Chap{20}
\TextTitle{Juda reste fidèle au roi David}
\VerseOne{}Et il se trouvait là un méchant\FTNT{Littéralement « beliya`al » : « méchant, pervers », « ruine, destruction ». Voir commentaire en 1 S. 2:12.} homme, nommé Schéba, fils de Bicri, Benjamite. Il sonna du shofar et dit : Nous n'avons point de part avec David ni d'héritage avec le fils d'Isaï ! Israël, chacun à ses tentes !
\VS{2}Ainsi tous les hommes d'Israël se séparèrent de David, et suivirent Schéba, fils de Bicri. Mais les hommes de Juda s'attachèrent à leur roi, et l'accompagnèrent depuis le Jourdain jusqu'à Jérusalem.
\VS{3}David rentra dans sa maison à Jérusalem. Il prit les dix femmes concubines qu'il avait laissées pour garder sa maison, et les mit en un lieu où elles étaient gardées ; il pourvut à leur entretien, mais il n'alla point vers elles. Ainsi, elles furent enfermées jusqu'au jour de leur mort, vivant dans le veuvage.
\TextTitle{Bataille contre Schéba ; Joab tue Amasa}
\VS{4}Puis le roi dit à Amasa : Rassemble-moi dans trois jours les hommes de Juda ; et toi, sois ici présent.
\VS{5}Amasa donc s'en alla pour rassembler Juda ; mais il tarda au-delà du temps que le roi lui avait fixé.
\VS{6}Alors David dit à Abischaï : Maintenant Schéba, fils de Bicri, nous fera plus de mal qu'Absalom. Prends toi-même les serviteurs de ton maître et poursuis-le, de peur qu'il ne trouve des villes fortes, et que nous ne le perdions de vue.
\VS{7}Et Abischaï partit, suivi des gens de Joab, des Kéréthiens et des Péléthiens, et de tous les hommes forts ; ils sortirent de Jérusalem, pour poursuivre Schéba, fils de Bicri.
\VS{8}Et comme ils furent près de la grande pierre qui est à Gabaon, Amasa vint au-devant d'eux. Joab était ceint d'une épée par-dessus les habits dont il était revêtu ; elle était attachée à ses reins dans le fourreau, et comme il s'avançait, elle tomba.
\VS{9}Joab dit à Amasa : Te portes-tu bien, mon frère ? Puis Joab prit de sa main droite la barbe d'Amasa pour l'embrasser.
\VS{10}Amasa ne prit point garde à l'épée qui était dans la main de Joab ; et Joab l'en frappa au ventre et répandit ses entrailles à terre, sans le frapper une seconde fois. Et il mourut. Après cela, Joab et Abischaï, son frère, poursuivirent Schéba, fils de Bicri.
\VS{11}Un des serviteurs de Joab resta près d'Amasa, et il disait : Qui aime Joab et qui est pour David ? Qu'il suive Joab !
\VS{12}Amasa était vautré dans son sang au milieu de la route ; et cet homme-là, ayant vu que tout le peuple s'arrêtait, poussa Amasa hors de la route dans un champ, et jeta un vêtement sur lui, lorsqu'il vit que tous ceux qui arrivaient près de lui s'arrêtaient.
\VS{13}Quand il fut ôté de la route, tous les hommes qui suivaient Joab passaient au-delà, afin de poursuivre Schéba, fils de Bicri.
\TextTitle{La révolte de Schéba}
\VS{14}Joab passa par toutes les tribus d'Israël jusqu'à Abel-Beth-Maaca, avec tous les Bériens, qui s'étaient assemblés et qui l'avaient suivi.
\VS{15}Les gens donc de Joab vinrent assiéger Schéba dans Abel-Beth-Maaca, et ils élevèrent contre la ville une terrasse qui atteignait le rempart. Tout le peuple qui était avec Joab rompait la muraille pour la faire tomber.
\VS{16}Lorsqu'une femme sage de la ville se mit à crier : Ecoutez, écoutez ! Dites, je vous prie, à Joab : Approche jusqu'ici, je veux te parler !
\VS{17}Il s'approcha d'elle, et la femme dit : Es-tu Joab ? Il répondit : Je le suis. Elle lui dit : Ecoute les paroles de ta servante. Il répondit : J'écoute.
\VS{18}Et elle dit : Autrefois on avait coutume de dire : Que l'on consulte Abel ! Et tout se terminait ainsi.
\VS{19}Je suis une des cités paisibles et fidèles en Israël ; tu cherches à détruire une ville qui est une mère en Israël ! Pourquoi détruirais-tu l'héritage de Yahweh ?
\VS{20}Joab lui répondit : A Dieu ne plaise, à Dieu ne plaise que je détruise et que je ruine !
\VS{21}La chose n'est pas ainsi. Mais un homme de la montagne d'Ephraïm, nommé Schéba, fils de Bicri, a levé sa main contre le roi David ; livrez-le, lui seul, et je m'éloignerai de la ville. La femme dit à Joab : Voici, sa tête te sera jetée par-dessus la muraille.
\VS{22}Et la femme alla vers tout le peuple, et leur parla sagement ; et ils coupèrent la tête de Schéba, fils de Bicri, et la jetèrent à Joab. Alors il sonna du shofar ; et on se dispersa loin de la ville, et chacun s'en alla dans sa tente. Puis Joab retourna vers le roi à Jérusalem.
\VS{23}Joab était le chef de toute l'armée d'Israël ; Benaja, fils de Jehojada, était à la tête des Kéréthiens et des Péléthiens ;
\VS{24}et Adoram était préposé aux impôts ; Josaphat, fils d'Achilud, était archiviste.
\VS{25}Scheja était le secrétaire ; Tsadok et Abiathar étaient les sacrificateurs ;
\VS{26}et Ira de Jaïr était ministre d'Etat de David.
\Chap{21}
\TextTitle{Vengeance des Gabaonites sur la maison de Saül}
\VerseOne{}Or il y eut du temps de David, une famine qui dura trois ans de suite. David chercha la face de Yahweh, et Yahweh lui répondit : C'est à cause de Saül et de sa maison sanguinaire, parce qu'il a fait mourir les Gabaonites.
\VS{2}Alors le roi appela les Gabaonites pour leur parler. Or les Gabaonites n'étaient point des enfants d'Israël, mais un reste des Amoréens ; les enfants d'Israël leur avaient juré de les laisser vivre\FTNT{Jos. 9.}, mais Saül dans son zèle pour les enfants d'Israël et de Juda, avait cherché à les faire mourir.
\VS{3}Et David dit aux Gabaonites : Que ferais-je pour vous, et par quel moyen vous apaiserai-je, afin que vous bénissiez l'héritage de Yahweh ?
\VS{4}Les Gabaonites lui répondirent : Il ne s'agit pas pour nous d'argent ou d'or avec Saül et avec sa maison, et ce n'est pas à nous de faire mourir un homme en Israël. Le roi leur dit : Que voulez-vous donc que je fasse pour vous ?
\VS{5}Ils répondirent au roi : Puisque cet homme nous a consumés, et qu'il avait résolu de nous exterminer pour nous faire disparaître de tout le territoire d'Israël,
\VS{6}qu'on nous livre sept hommes d'entre ses fils, et nous les pendrons devant Yahweh à Guibea de Saül, l'élu de Yahweh. Et le roi dit : Je vous les livrerai.
\VS{7}Le roi épargna Mephiboscheth, fils de Jonathan, fils de Saül, à cause du serment que David et Jonathan, fils de Saül, avaient fait entre eux, devant Yahweh.
\VS{8}Mais le roi prit les deux fils que Ritspa, fille d'Ajja, avait enfantés à Saül, Armoni et Mephiboscheth, et les cinq fils que Mérab, fille de Saül, avait enfantés à Adriel de Mehola, fils de Barzillaï,
\VS{9}et il les livra entre les mains des Gabaonites, qui les pendirent sur la montagne, devant Yahweh. Tous les sept furent tués ensemble ; on les fit mourir dans les premiers jours de la moisson, au commencement de la moisson des orges.
\VS{10}Alors Ritspa, fille d'Ajja, prit un sac et l'étendit sous elle au-dessus d'un rocher, depuis le commencement de la moisson jusqu'à ce que l'eau du ciel tombât sur eux ; et elle ne permit pas aux oiseaux du ciel de s'approcher d'eux pendant le jour, ni aux bêtes des champs pendant la nuit.
\VS{11}On informa David de ce qu'avait fait Ritspa, fille d'Ajja, concubine de Saül.
\VS{12}Et David alla prendre les os de Saül et les os de Jonathan, son fils, chez les habitants de Jabès en Galaad, qui les avaient enlevés de la place de Beth-Schan, où les Philistins les avaient pendus lorsqu'ils tuèrent Saül à Guilboa.
\VS{13}Il emporta de là les os de Saül et les os de Jonathan, son fils ; on recueillit aussi les os de ceux qui avaient été pendus.
\VS{14}On les enterra avec les os de Saül et de Jonathan, son fils, au pays de Benjamin, à Tséla, dans le sépulcre de Kis, père de Saül. Et l'on fit tout ce que le roi avait ordonné. Après cela, Dieu fut apaisé envers le pays.
\TextTitle{Nouvelles batailles contre les Philistins}
\VS{15}Il y eut encore une guerre entre les Philistins et Israël. David y était allé, et ses serviteurs avec lui, et ils combattirent tellement contre les Philistins que David défaillait.
\VS{16}Et Jischbi-Benob, qui était un des enfants de Rapha, eut l'intention de tuer David ; il avait une lance dont le fer pesait trois cents sicles d'airain, et il était ceint d'une armure neuve.
\VS{17}Mais Abischaï, fils de Tseruja, vint au secours de David, frappa le Philistin, et le tua. Alors les gens de David jurèrent, en disant : Tu ne sortiras plus avec nous à la bataille, de peur que tu n'éteignes la lampe d'Israël.
\VS{18}Après cela, il y eut encore une autre guerre à Gob avec les Philistins. Sibbecaï, le Huschatite, tua Saph, qui était un des enfants de Rapha.
\VS{19}Il y eut encore une autre guerre à Gob avec les Philistins. Et Elchanan, fils de Jaaré-Oreguim, de Bethléhem, tua Goliath de Gath, qui avait une lance dont le bois était comme une ensouple de tisserand.
\VS{20}Il y eut encore une guerre à Gath. Il s'y trouva un homme de haute taille, qui avait six doigts à chaque main, et six orteils à chaque pied, en tout vingt-quatre, lequel était aussi issu de Rapha.
\VS{21}Il jeta un défi à Israël ; et Jonathan, fils de Schimea, frère de David, le tua.
\VS{22}Ces quatre-là étaient nés à Gath, de la race de Rapha. Ils moururent par les mains de David, ou par les mains de ses serviteurs.
\Chap{22}
\TextTitle{Louange à Yahweh, le Dieu qui délivre}
\VerseOne{}Après cela, David adressa à Yahweh les paroles de ce cantique, le jour où Yahweh l'eut délivré de la main de tous ses ennemis, et de la main de Saül.
\VS{2}Il dit : Yahweh est mon rocher, ma forteresse, mon libérateur.
\VS{3}Dieu est mon rocher, où je trouve un abri, mon bouclier et la force qui me sauve, ma haute retraite et mon refuge. Ô mon Sauveur ! Tu me délivres de la violence.
\VS{4}Je m'écrie : Loué soit Yahweh! Et je suis délivré de mes ennemis\FTNT{Ps. 18:4.}.
\VS{5}Car les flots de la mort m'avaient environné, les torrents des méchants m'avaient épouvanté ;
\VS{6}les liens du scheol m'avaient entouré, les filets de la mort m'avaient surpris.
\VS{7}Dans ma détresse, j'ai invoqué Yahweh, j'ai crié à mon Dieu ; de son palais, il a entendu ma voix, et mon cri est parvenu à ses oreilles.
\VS{8}Alors la terre fut ébranlée et trembla, les fondements des cieux s'agitèrent, et ils furent ébranlés, parce qu'il était irrité.
\VS{9}Une fumée montait de ses narines, et de sa bouche sortait un feu dévorant : Il en jaillissait des charbons embrasés.
\VS{10}Il abaissa les cieux, et descendit : Il y avait une épaisse nuée sous ses pieds.
\VS{11}Il était monté sur un chérubin, et il volait, il paraissait sur les ailes du vent.
\VS{12}Il mit autour de lui les ténèbres pour tabernacle, des amas d'eaux, des nuées épaisses.
\VS{13}Des charbons de feu étaient embrasés de la splendeur qui le précédait.
\VS{14}Yahweh tonna des cieux, et le Très-Haut fit retentir sa voix ;
\VS{15}il lança des flèches, et dispersa mes ennemis ;  il lança des éclairs, et les mit en déroute.
\VS{16}Alors le fond de la mer apparut, et les fondements de la terre habitable furent mis à découvert, par la menace de Yahweh, par le souffle du vent de sa colère.
\VS{17}Il étendit sa main d'en haut, il me saisit, il me retira des grandes eaux ;
\VS{18}il me délivra de mon ennemi puissant, de ceux qui me haïssaient, car ils étaient plus forts que moi.
\VS{19}Ils m'avaient surpris au jour de ma détresse, mais Yahweh fut mon appui.
\VS{20}Il m'a mis au large, il m'a sauvé, parce qu'il a pris son plaisir en moi.
\VS{21}Yahweh m'a traité selon ma droiture, il m'a rendu selon la pureté de mes mains ;
\VS{22}parce que j'ai gardé les voies de Yahweh, et que je ne me suis point détourné de mon Dieu.
\VS{23}Toutes ses ordonnances ont été devant moi, et je ne me suis point écarté de ses lois.
\VS{24}J'ai été intègre envers lui, et je me suis gardé de mon iniquité.
\VS{25}Yahweh donc m'a rendu selon ma droiture, selon ma pureté devant ses yeux.
\VS{26}Avec celui qui est bon tu es bon, avec l'homme intègre tu es intègre,
\VS{27}avec celui qui est pur tu te montres pur, mais avec le pervers tu agis selon sa perversité.
\VS{28}Tu sauves le peuple qui s'humilie, et de ton regard, tu abaisses les orgueilleux.
\VS{29}Tu es ma lampe, ô Yahweh ! Et Yahweh éclaire mes ténèbres.
\VS{30}Avec toi je me précipite sur une troupe en armes, avec mon Dieu je franchis une muraille.
\VS{31}La voie de Dieu est parfaite, la parole de Yahweh est éprouvée ; il est le bouclier de tous ceux qui se confient en lui.
\VS{32}Car qui est Dieu, si ce n'est Yahweh ? Et qui est un rocher, si ce n'est notre Dieu ?
\VS{33}C'est Dieu qui est ma puissante forteresse, et qui me conduit dans la voie droite.
\VS{34}Il a rendu mes pieds semblables à ceux des biches, et il me fait tenir debout sur mes lieux élevés.
\VS{35}Il exerce mes mains au combat, et mes bras tendent l'arc d'airain.
\VS{36}Tu me donnes le bouclier de ton salut, et ta bonté me fait devenir plus grand.
\VS{37}Tu élargis le chemin sous mes pas, et mes pieds ne chancellent point.
\VS{38}Je poursuis mes ennemis, et je les détruis ; je ne reviens qu'après les avoir exterminés.
\VS{39}Je les anéantis, je les transperce, et ils ne se relèvent plus ; ils tombent sous mes pieds.
\VS{40}Tu me ceins de force pour le combat, tu fais plier sous moi mes adversaires.
\VS{41}Tu fais tourner le dos à mes ennemis devant moi, et j'extermine ceux qui me haïssent.
\VS{42}Ils regardent autour d'eux, et il n'y a point de sauveur ! Ils crient à Yahweh, mais il ne leur répond pas !
\VS{43}Je les broie comme la poussière de la terre, je les écrase, je les foule, comme la boue des rues.
\VS{44}Tu me délivres des dissensions de mon peuple ; tu me gardes pour être chef des nations ; un peuple que je ne connaissais pas m'est asservi.
\VS{45}Les fils de l'étranger me flattent, dès qu'ils ont entendu parler de moi, ils se sont rendus obéissants.
\VS{46}Les fils de l'étranger défaillent, et sortent tremblants de leurs forteresses.
\VS{47}Yahweh est vivant, et béni soit mon rocher ! Que Dieu, le rocher de mon salut, soit exalté,
\VS{48}le Dieu qui me donne vengeance, qui m'assujettit les peuples,
\VS{49}et qui me fait échapper à mes ennemis ! Tu m'élèves au-dessus de mes adversaires, tu me délivres de l'homme violent.
\VS{50}C'est pourquoi, ô Yahweh, je te louerai parmi les nations, et je chanterai des psaumes à ton Nom.
\VS{51}C'est lui qui est la tour de délivrance de son roi, et qui fait miséricorde à son oint, à David, et à sa postérité, à jamais.
\Chap{23}
\TextTitle{Paroles prophétiques de David}
\VerseOne{}Voici les dernières paroles de David. Parole de David, fils d'Isaï, parole de l'homme qui a été élevé, de l'oint du Dieu de Jacob, du chantre agréable d'Israël :
\VS{2}L'Esprit de Yahweh parle par moi, et sa parole est sur ma langue.
\VS{3}Le Dieu d'Israël a parlé, le Rocher\FTNT{Voir commentaire Es. 8 :13-17.} d'Israël m'a dit : Celui qui règne parmi les hommes avec justice, celui qui règne dans la crainte de Dieu,
\VS{4}est comme la lumière du matin quand le soleil se lève, un matin sans nuage ; son éclat fait germer de la terre la verdure après la pluie.
\VS{5}N'en est-il pas ainsi de ma maison devant Dieu, puisqu'il a traité avec moi une alliance éternelle, bien ordonnée, et gardée ? Tout mon salut et tout mon plaisir, ne les fera-t-il pas germer ?
\VS{6}Mais les méchants sont tous comme des épines que l'on jette au loin, parce qu'on ne les prend pas avec la main ;
\VS{7}celui qui les touche, s'arme du fer ou du bois d'une lance, et on les brûle au feu sur place.
\TextTitle{Les vaillants hommes de David\FTNTT{1 Ch. 11:10-47.}}
\VS{8}Voici les noms des vaillants hommes qui étaient au service de David. Joscheb-Basschébeth, le Tachkemonite, était l'un des principaux chefs. C'était Hadino le Hetsnite, qui eut le dessus sur huit cents hommes qu'il tua en une seule fois.
\VS{9}Après lui, Eléazar, fils de Dodo, fils d'Achochi. Il était l'un des trois vaillants hommes qui étaient avec David lorsqu'ils défièrent les Philistins rassemblés pour combattre, tandis que les hommes d'Israël se retiraient.
\VS{10}Il se leva, et frappa les Philistins jusqu'à ce que sa main fut lasse et qu'elle restât attachée à l'épée. Ce jour-là, Yahweh opéra une grande délivrance. Le peuple revint après Eléazar, seulement pour prendre les dépouilles.
\VS{11}Après lui, Schamma, fils d'Agué d'Harar. Les Philistins s'étaient rassemblés en troupe. Il y avait là une parcelle de champ pleine de lentilles ; et le peuple fuyait devant les Philistins.
\VS{12}Schamma se mit au milieu de cette parcelle, la défendit, et frappa les Philistins. Et Yahweh opéra une grande délivrance.
\VS{13}Trois des trente chefs descendirent au temps de la moisson et vinrent vers David, dans la caverne d'Adullam, lorsqu'une troupe de Philistins était campée dans la vallée des Rephaïm.
\VS{14}David était alors dans la forteresse, et la garnison des Philistins était en ce temps-là à Bethléhem.
\VS{15}Et David eut un désir, et dit : Qui est-ce qui me fera boire de l'eau de la citerne qui est à la porte de Bethléhem ?
\VS{16}Alors ces trois vaillants hommes passèrent au travers du camp des Philistins, et puisèrent de l'eau de la citerne qui est à la porte de Bethléhem. Ils l'apportèrent, et ils la présentèrent à David ; mais il ne voulut pas la boire, et il la répandit devant Yahweh.
\VS{17}Car il dit : Loin de moi, ô Yahweh, de faire une telle chose ! N'est-ce pas le sang de ces hommes qui sont allés au péril de leur vie ? Il ne voulut pas la boire. Voilà ce que firent ces trois vaillants hommes.
\VS{18}Il y avait aussi Abischaï, frère de Joab, fils de Tseruja, qui était le chef des trois. Il brandit sa lance sur trois cents hommes, les blessa à mort ; et il eut du renom parmi les trois.
\VS{19}Il était le plus considéré des trois, et il fut leur chef ; cependant il n'égala point les trois premiers.
\VS{20}Benaja, fils de Jehojada, fils d'un vaillant homme de Kabtseel, rempli de force, avait fait de grands exploits. Il frappa deux des plus puissants hommes de Moab. Il descendit au milieu d'une fosse, où il frappa un lion, un jour de neige.
\VS{21}Il frappa aussi un Egyptien d'un aspect formidable et ayant une lance à la main ; Benaja descendit contre lui avec un bâton, arracha la lance de la main de l'Egyptien, et s'en servit pour le tuer.
\VS{22}Benaja, fils de Jehojada, fit ces choses-là ; et fut illustre parmi les trois vaillants hommes.
\VS{23}Il était le plus considéré des trente ; mais il n'égala pas les trois premiers. C'est pourquoi David l'établit dans son conseil secret.
\VS{24}Asaël, frère de Joab, était des trente. Elchanan, fils de Dodo, de Bethléhem.
\VS{25}Schamma, de Harod. Elika, de Harod.
\VS{26}Hélets, de Péleth. Ira, fils d'Ikkesch, de Tekoa.
\VS{27}Abiézer, d'Anathoth. Mebunnaï, de Huscha.
\VS{28}Tsalmon, d'Achoach. Maharaï, de Nethopha.
\VS{29}Héleb, fils de Baana, de Nethopha. Ittaï, fils de Ribaï, de Guibea des fils de Benjamin.
\VS{30}Benaja, de Pirathon. Hiddaï, de Nachalé-Gaasch.
\VS{31}Abi-Albon, d'Araba. Azmaveth, de Barchum.
\VS{32}Eliachba, de Schaalbon. Bené-Jaschen. Jonathan.
\VS{33}Schamma, d'Harar. Achiam, fils de Scharar, d'Arar.
\VS{34}Eliphéleth, fils d'Achasbaï, fils d'un Maacathien. Eliam, fils d'Achitophel, de Guilo.
\VS{35}Hetsro, de Carmel. Paaraï, d'Arab.
\VS{36}Jigueal, fils de Nathan, de Tsoba. Bani, de Gad.
\VS{37}Tsélek, l'Ammonite. Naharaï, de Beéroth, qui portait les armes de guerre de Joab, fils de Tseruja.
\VS{38}Ira, de Jéther. Gareb, de Jéther.
\VS{39}Urie, le Héthien. En tout, trente-sept.
\Chap{24}
\TextTitle{Péché de David ; plaie mortelle sur Israël\FTNTT{1 Ch. 21:1-17.}}
\VerseOne{}La colère de Yahweh s'enflamma encore contre Israël, parce que David fut incité contre eux, en disant : Va, fais le dénombrement d'Israël et de Juda\FTNT{1 Ch. 21.}.
\VS{2}Le roi dit donc à Joab, chef de l'armée qui se trouvait près de lui : Parcours toutes les tribus d'Israël, depuis Dan jusqu'à Beer-Schéba ; et dénombre le peuple, afin que je sache le nombre du peuple.
\VS{3}Joab dit au roi : Que Yahweh, ton Dieu, veuille augmenter ton peuple cent fois plus, et que les yeux du roi mon seigneur le voient ! Mais pourquoi le roi mon seigneur prend-il plaisir à cela ?
\VS{4}Néanmoins, la parole du roi l'emporta sur Joab, et sur les chefs de l'armée ; et Joab et les chefs de l'armée sortirent de la présence du roi pour dénombrer le peuple d'Israël.
\VS{5}Ils passèrent le Jourdain, et ils campèrent à Aroër, à droite de la ville qui est au milieu de la vallée du torrent de Gad, et vers Jaezer.
\VS{6}Ils allèrent en Galaad et dans le territoire de ceux qui habitent vers le bas du pays de Thachthim-Hodschi. Ils allèrent à Dan-Jaan, et aux environs de Sidon.
\VS{7}Ils vinrent jusqu'à la forteresse de Tyr, et dans toutes les villes des Héviens et des Cananéens. Ils sortirent vers le midi de Juda à Beer-Schéba.
\VS{8}Ainsi ils parcoururent tout le pays, et arrivèrent à Jérusalem au bout de neuf mois et vingt jours.
\VS{9}Et Joab donna au roi le rôle du dénombrement du peuple : Il y avait en Israël huit cent mille hommes de guerre tirant l'épée, et en Juda cinq cent mille hommes.
\VS{10}Alors David sentit battre son cœur, après qu'il eut fait ainsi dénombrer le peuple. Et David dit à Yahweh : J'ai commis un grand péché en faisant cela ! Mais, je te prie, ô Yahweh, de pardonner l'iniquité de ton serviteur, car j'ai agi en insensé !
\VS{11}Après cela, David se leva dès le matin, et la parole de Yahweh fut adressée à Gad le prophète, qui était le voyant de David :
\VS{12}Va dire à David : Ainsi parle Yahweh : J'apporte trois choses contre toi ; choisis l'une d'elles afin que je te la fasse.
\VS{13}Gad alla vers David, et lui rapporta cela en disant : Que veux-tu qu'il t'arrive : Sept ans de famine sur ton pays, ou que durant trois mois tu fuies devant tes ennemis qui te poursuivront, ou que durant trois jours la peste soit dans ton pays ? Choisis maintenant, et regarde ce que tu veux que je réponde à celui qui m'a envoyé.
\VS{14}David répondit à Gad : Je suis dans une très grande détresse ! Tombons entre les mains de Yahweh, car ses compassions sont en grand nombre ; mais que je ne tombe pas entre les mains des hommes !
\VS{15}Yahweh envoya donc la peste en Israël, depuis le matin jusqu'au temps fixé ; et depuis Dan jusqu'à Beer-Schéba, il mourut soixante-dix mille hommes parmi le peuple.
\VS{16}Mais quand l'ange étendait sa main sur Jérusalem pour la ravager, Yahweh se repentit de ce mal et dit à l'ange qui ravageait le peuple : C'est assez ! Retire maintenant ta main. Or l'Ange de Yahweh était près de l'aire d'Aravna, le Jébusien.
\VS{17}Car David voyant l'ange qui frappait le peuple, parla à Yahweh, et dit : Voici, c'est moi qui ai péché ! C'est moi qui ai commis l'iniquité ; mais ces brebis, qu'ont-elles fait ? Je te prie que ta main soit contre moi et contre la maison de mon père !
\TextTitle{Sacrifice de David ; Yahweh met fin à la plaie\FTNTT{1 Ch. 21:18-30.}}
\VS{18}Ce jour-là, Gad vint vers David, et lui dit : Monte, et dresse un autel à Yahweh dans l'aire d'Aravna, le Jébusien.
\VS{19}Et David monta, selon la parole de Gad, comme Yahweh l'avait ordonné.
\VS{20}Aravna regarda, et vit le roi et ses serviteurs qui venaient vers lui ; et Aravna sortit, et se prosterna devant le roi, le visage contre terre.
\VS{21}Aravna dit : Pourquoi le roi mon seigneur vient-il vers son serviteur ? Et David répondit : Pour acheter ton aire, et y bâtir un autel à Yahweh, afin que cette plaie se retire de dessus le peuple.
\VS{22}Aravna dit à David : Que le roi mon seigneur prenne et offre ce qu'il lui plaira ; vois les bœufs seront pour l'holocauste, et les chars avec l'attelage de bœufs serviront de bois.
\VS{23}Aravna donna tout cela au roi. Et Aravna dit au roi : Que Yahweh, ton Dieu, te soit favorable !
\VS{24}Mais le roi répondit à Aravna : Non ! Je veux l'acheter de toi pour un certain prix, et je n'offrirai point à Yahweh, mon Dieu, des holocaustes qui ne me coûtent rien. Ainsi, David acheta l'aire et les bœufs pour cinquante sicles d'argent.
\VS{25}David bâtit là un autel à Yahweh, et offrit des holocaustes et des sacrifices d'offrande de paix. Alors Yahweh fut apaisé envers le pays, et la plaie se retira d'Israël.
\PPE{}
\end{multicols}

%\clearpage\ShortTitle{1 Rois}\BookTitle{1 Rois}\BFont
\noindent\hrulefill
{\footnotesize
\textit{
\bigskip
{\centering{}
\\Auteur : Inconnu
\\(Heb. : Melakhim)
\\Signification : Roi, Règne
\\Thème : Unité du royaume après le schisme
\\Date de rédaction : 6ème siècle av. J.-C.\\}
}
%\bigskip
\textit{
\\Ce livre relate la vie de Salomon : son accession à la royauté après la mort de son père David, son alliance avec Dieu qui lui accorda une sagesse exceptionnelle ainsi que la construction du temple de Yahweh et du palais royal.
%\bigskip
\\Les premières années du règne de Salomon furent exemplaires. Malheureusement, il ne fit pas preuve de la même piété
que son père et développa une affection particulière pour les femmes étrangères qui l’entrainèrent dans l’idolâtrie. A sa
mort, son fils Roboam accéda au pouvoir et provoqua la division du royaume en deux : d’un côté les dix tribus du nord qui gardèrent le nom d’Israël, gouvernées par Jéroboam, et de l’autre côté les deux tribus du sud, Juda et Benjamin, qui demeurèrent sous l’autorité de Roboam.
%\bigskip
\\Ce livre raconte également le règne et la conduite parfois abominable des rois d’Israël et de Juda jusqu’à Achab et Josaphat.
Il présente la puissance de l’appel prophétique d’Elie, le Tischbite, que Dieu suscita pour ramener son peuple à lui et
montrer sa souveraineté.\bigskip
}
}
\par\nobreak\noindent\hrulefill
\begin{multicols}{2}
\Chap{1}
\TextTitle{Fin de la vie de David}
\VerseOne{}Le roi David était vieux et avancé en âge ; on le couvrait de vêtements parce qu’il ne parvenait point à se réchauffer.
\VS{2}Ses serviteurs lui dirent : Que l'on cherche pour le roi notre seigneur, une jeune fille vierge ; qu’elle se tienne devant le roi, qu’elle le soigne et qu'elle dorme en son sein, afin que le roi notre seigneur, se réchauffe.
\VS{3}On chercha donc dans toutes les contrées d'Israël une jeune et belle femme, et on trouva Abischag la Sunamite, que l’on amena auprès du roi.
\VS{4}Cette jeune femme était fort belle. Elle prit soin du roi et le servit, mais le roi ne la connut point.
\TextTitle{Conspiration d’Adonija pour régner sur Israël}
\VS{5}Alors Adonija, fils de Haggith, se laissa emporter par l’orgueil en disant : Je suis le roi ! Il se procura un char, des cavaliers et cinquante hommes qui couraient devant lui.
\VS{6}Son père ne lui avait jamais fait un reproche jusqu’à ce jour-là, en disant : Pourquoi agis-tu ainsi ? Adonija était très beau de figure, il était né après Absalom.
\VS{7}Il s’entendit avec Joab, fils de Tseruja, et avec le sacrificateur Abiathar, qui embrassèrent son parti.
\VS{8}Mais le sacrificateur Tsadok, Benaja fils de Jehojada, Nathan le prophète, Schimeï, Reï et les vaillants hommes de David ne furent point du parti d'Adonija.
\VS{9}Or, Adonija fit tuer des brebis, des bœufs et des veaux gras près de la pierre de Zohéleth, qui est auprès d’En-Roguel ; il invita tous ses frères, fils du roi, et tous les hommes de Juda qui étaient au service du roi.
\VS{10}Mais il ne convia point Nathan le prophète, ni Benaja, ni les vaillants hommes, ni Salomon, son frère.
\TextTitle{Opposition de Nathan et Bath-Schéba}
\VS{11}Alors Nathan parla à Bath-Schéba, mère de Salomon, en disant : N'as-tu pas entendu qu'Adonija, fils de Haggith, a été fait roi ? Et David notre Seigneur n'en sait rien.
\VS{12}Maintenant donc viens, je t’en donne le conseil afin que tu sauves ta vie et la vie de ton fils Salomon.
\VS{13}Va, entre chez le roi David et dis-lui : Ô roi, mon seigneur, n'as-tu pas fait serment à ta servante, en disant : ton fils Salomon régnera après moi et sera assis sur mon trône ? Pourquoi donc Adonija règne-t-il ?
\VS{14}Et voici, lorsque tu seras encore là et que tu parleras avec le roi, je viendrai après toi et je confirmerai tes dires.
\VS{15}Bath-Schéba se rendit dans la chambre du roi. Or, le roi était très vieux et Abischag, la Sunamite, le servait.
\VS{16}Bath-Schéba s'inclina et se prosterna devant le roi. Et le roi lui dit : Qu'as-tu ?
\VS{17}Et elle lui répondit : Mon seigneur, tu as juré par Yahweh, ton Dieu à ta servante, en lui disant : Ton fils Salomon régnera après moi et s’assiéra sur mon trône.
\VS{18}Mais maintenant voici, Adonija est proclamé roi ! Et tu ne le sais pas, ô roi, mon seigneur !
\VS{19}Il a fait tuer des bœufs, des veaux gras et des brebis en grand nombre, il a convié tous les fils du roi, avec Abiathar, le sacrificateur, et Joab, chef de l'armée, mais il n'a point convié ton serviteur Salomon.
\VS{20}Ô roi mon seigneur ! Les yeux de tout Israël sont sur toi, afin que tu lui fasses connaître qui s’assiéra sur le trône du roi mon seigneur après lui.
\VS{21}Aussi, lorsque le roi mon seigneur sera endormi avec ses pères, nous serons traités comme des coupables, moi et mon fils Salomon.
\VS{22}Tandis qu’elle parlait encore avec le roi, Nathan le prophète se présenta.
\VS{23}On l’annonça au roi en disant : Voici Nathan le prophète ! Il se présenta devant le roi et se prosterna devant lui, le visage contre terre.
\VS{24}Et Nathan dit : Ô roi mon seigneur ! Tu as dit : Adonija régnera après moi et sera assis sur mon trône !
\VS{25}Car il est descendu aujourd'hui, il a sacrifié des bœufs, des veaux gras et des brebis en grand nombre. Il a convié tous les fils du roi, les chefs de l'armée et le sacrificateur Abiathar. Et voici, ils mangent et boivent devant lui ; ils disent : Vive le roi Adonija !
\VS{26}Mais il n'a convié ni moi, ton serviteur, ni le sacrificateur Tsadok, ni Benaja, fils de Jehojada, ni Salomon ton serviteur.
\VS{27}Est-ce bien par ordre de mon seigneur le roi que cette chose a lieu et sans que tu aies fait connaître à ton serviteur quel est celui qui doit s'asseoir sur le trône du roi mon seigneur après lui ?
\VS{28}Et le roi David répondit, en disant : Appelez-moi Bath-Schéba ; elle entra et se présenta devant le roi.
\VS{29}Alors le roi jura et dit : Yahweh, qui m'a délivré de toute détresse, est vivant !
\VS{30}Comme je te l'ai juré par Yahweh, le Dieu d'Israël, en disant : Ton fils Salomon régnera après moi et sera assis sur mon trône à ma place ; ainsi ferai-je aujourd'hui.
\VS{31}Alors Bath-Schéba s'inclina le visage contre terre et se prosterna devant le roi en disant : Que le roi David mon seigneur vive éternellement !
\VS{32}Et le roi David dit : Appelez-moi le sacrificateur Tsadok, le prophète Nathan et Benaja, fils de Jehojada ; et ils se présentèrent devant le roi.
\VS{33}Le roi leur dit : Prenez avec vous les serviteurs de votre seigneur, faites monter mon fils Salomon sur ma mule, et faites-le descendre à Guihon.
\VS{34}Que Tsadok le sacrificateur et Nathan le prophète, l'oignent en ce lieu-là pour roi sur Israël, puis vous sonnerez du shofar et vous direz : Vive le roi Salomon !
\VS{35}Vous monterez après lui et il viendra, il s'assiéra sur mon trône et il régnera à ma place ; car j'ai ordonné qu'il soit le chef d'Israël et de Juda.
\VS{36}Et Benaja fils de Jehojada répondit au roi : Amen ! Ainsi parle Yahweh, le Dieu de mon seigneur le roi !
\VS{37}Comme Yahweh a été avec mon seigneur le roi, qu'il soit aussi avec Salomon, et qu'il élève son trône encore plus que le trône du roi David mon seigneur !
\TextTitle{Salomon oint roi d’Israël par Tsadok\FTNTT{cp. 1 Ch. 29:22}}
\VS{38}Puis Tsadok le sacrificateur descendit avec Nathan le prophète et Benaja, fils de Jehojada, les Kéréthiens et les Péléthiens ; ils firent monter Salomon sur la mule du roi David et le menèrent à Guihon.
\VS{39}Tsadok le sacrificateur prit du tabernacle une corne d'huile dont il oignit Salomon. On sonna du shofar et tout le peuple dit : Vive le roi Salomon !
\VS{40}Et tout le monde monta après lui et le peuple jouait de la flûte, en se livrant à une grande joie, au point que la terre se fendait par leurs cris.
\VS{41}Ce bruit fut entendu d’Adonija et de tous les conviés qui étaient avec lui comme ils achevaient de manger ; et Joab entendant le son du shofar, dit : Pourquoi ce bruit de la ville en tumulte ?
\VS{42}Et comme il parlait encore, voici Jonathan, fils du sacrificateur Abiathar, arriva et Adonija lui dit : Entre, car tu es un vaillant homme et tu apportes de bonnes nouvelles.
\VS{43}Oui ! répondit Jonathan à Adonija : Le roi David, notre seigneur, a établi Salomon roi.
\VS{44}Et le roi a envoyé avec lui Tsadok le sacrificateur, Nathan le prophète, Benaja, fils de Jehojada, les Kéréthiens, et les Péléthiens, et ils l'ont fait monter sur la mule du roi.
\VS{45}Tsadok le sacrificateur, et Nathan le prophète l'ont oint pour roi à Guihon, d'où ils sont remontés avec joie, et la ville est ainsi émue ; c'est là le bruit que vous avez entendu.
\VS{46}Salomon s'est même assis sur le trône royal.
\VS{47}Et les serviteurs du roi sont venus pour bénir le roi David notre seigneur, en disant : Que ton Dieu rende le nom de Salomon encore plus grand que ton nom, et qu'il élève son trône encore plus que ton trône ! Et le roi s'est prosterné sur son lit.
\VS{48}Le roi a ainsi parlé : Béni soit Yahweh, le Dieu d'Israël, qui a aujourd’hui établi sur mon trône un successeur, et qui m’a permis de le voir !
\VS{49}Alors tous les conviés d’Adonija furent saisis de frayeur, ils se levèrent et s'en allèrent chacun son chemin.
\VS{50}Adonija eut peur de Salomon ; il se leva aussi et s'en alla empoigner les cornes de l'autel.
\VS{51}On vint l’apprendre à Salomon, en disant : Voici Adonija a peur du roi Salomon et il a saisi les cornes de l'autel, en disant : Que le roi Salomon me jure aujourd'hui qu'il ne fera point mourir son serviteur par l'épée.
\VS{52}Et Salomon dit : A l’avenir, s’il se comporte en homme de bien il ne tombera pas un seul de ses cheveux à terre ; mais s'il se trouve du mal en lui, il mourra.
\VS{53}Alors le roi Salomon envoya des personnes qui le firent descendre de l'autel. Il vint et se prosterna devant le roi Salomon, et Salomon lui dit : Va dans ta maison.
\Chap{2}
\TextTitle{Dernières paroles de David à Salomon}
\VerseOne{}David approchait du moment de sa mort, et il donna ses ordres à Salomon, son fils, en disant :
\VS{2}Je m'en vais par le chemin de toute la terre, fortifie-toi et comporte-toi en homme.
\VS{3}Observe les commandements de Yahweh, ton Dieu, en marchant dans ses voies, en gardant ses statuts, ses commandements, ses ordonnances et ses préceptes, selon ce qui est écrit dans la loi de Moïse, afin que tu réussisses dans tout ce que tu feras et dans tout ce que tu entreprendras ;
\VS{4}et afin que s’accomplisse cette parole de Yahweh déclarée sur moi : Si tes fils prennent garde à leur voie, pour marcher devant moi dans la vérité, de tout leur cœur et de toute leur âme, tu ne manqueras jamais de successeur sur le trône d'Israël.
\VS{5}Tu sais ce que m'a fait Joab, fils de Tseruja et ce qu'il a fait aux deux chefs des armées d'Israël, Abner, fils de Ner, et à Amasa, fils de Jéther, qu'il a tués, en versant pendant la paix le sang de la guerre ; il a mis de ce sang sur la ceinture qu'il avait sur ses reins et sur les chaussures qu'il avait aux pieds.
\VS{6}Tu agiras selon ta sagesse, en sorte que tu ne laisseras point ses cheveux blancs descendre en paix dans le scheol.
\VS{7}Tu traiteras avec bienveillance les fils de Barzillaï, le Galaadite, et ils seront du nombre de ceux qui mangent à ta table ; car ils se sont approchés de moi quand je fuyais Absalom, ton frère.
\VS{8}Voici, tu as avec toi Schimeï, fils de Guéra, le benjamite de Bachurim, qui proféra contre moi des malédictions violentes le jour où je m'en allais à Mahanaïm. Mais il descendit au-devant de moi vers le Jourdain et je lui jurai par Yahweh, en disant : Je ne te ferai point mourir par l'épée.
\VS{9}Maintenant donc tu ne le laisseras point impuni, car tu es sage, pour savoir comment tu dois le traiter ; et tu feras descendre ses cheveux blancs ensanglantés au scheol.
\TextTitle{Mort de David ; début du règne de Salomon\FTNTT{1 Ch. 29:23-30}}
\VS{10}Ainsi David se coucha avec ses pères, il fut enseveli dans la cité de David.
\VS{11}Et le temps que David régna sur Israël fut quarante ans. Il régna sept ans à Hébron et il régna trente-trois ans à Jérusalem.
\VS{12}Et Salomon s'assit sur le trône de David, son père, et son règne fut très affermi.
\TextTitle{Mort d'Adonija}
\VS{13}Alors Adonija, fils de Haggith, vint vers Bath-Schéba, mère de Salomon et elle dit : Amènes-tu la paix ? Et il répondit : Je viens en paix.
\VS{14}Il ajouta : J'ai un mot à te dire. Elle répondit : Parle !
\VS{15}Et il dit : Tu sais bien que le royaume m'appartenait et que tout Israël s'attendait à ce que je règne. Mais la royauté s’est détournée de moi, elle est échue à mon frère parce que Yahweh la lui a donnée.
\VS{16}Maintenant donc je te demande une chose, ne me la refuse point. Elle lui répondit : Parle !
\VS{17}Et il dit : Je te prie, dis au roi Salomon, car il ne te refusera rien, qu'il me donne Abischag, la Sunamite, pour femme.
\VS{18}Bath-Schéba répondit : Et bien, je parlerai pour toi au roi.
\VS{19}Bath-Schéba se rendit auprès du roi Salomon pour lui parler en faveur d’Adonija ; et le roi se leva pour aller au-devant d’elle, il se prosterna devant elle, puis il s'assit sur son trône. On plaça un siège pour la mère du roi, et elle s'assit à sa droite.
\VS{20}Elle dit alors : J'ai une petite demande à te faire : ne me la refuse pas ! Et le roi lui répondit : Demande, ma mère, car je ne te la refuserai point.
\VS{21}Et elle dit : Qu'on donne Abischag, la Sunamite, pour femme à Adonija, ton frère.
\VS{22}Mais le roi Salomon répondit à sa mère et dit : Et pourquoi demandes-tu Abischag, la Sunamite, pour Adonija ? Demande plutôt le royaume pour lui, parce qu'il est mon frère aîné ; demande-le pour lui, pour Abiathar, le sacrificateur, et pour Joab, fils de Tseruja !
\VS{23}Alors le roi Salomon jura par Yahweh, en disant : Que Dieu me traite dans toute sa rigueur, si Adonija n'a dit cette parole contre sa propre vie !
\VS{24}Maintenant Yahweh est vivant, lui qui m'a établi, qui m'a fait asseoir sur le trône de David, mon père, et qui m'a donné une maison, selon sa promesse ! Aujourd’hui Adonija mourra.
\VS{25}Et le roi Salomon envoya Benaja, fils de Jehojada, qui le frappa, et Adonija mourut.
\TextTitle{Abiathar dépouillé de ses fonctions au temple}
\VS{26}Puis le roi dit à Abiathar, le sacrificateur : Va-t'en à Anathoth sur tes terres, car tu mérites la mort ; toutefois je ne te ferai point mourir aujourd'hui, parce que tu as porté l'arche du Seigneur Yahweh devant David, mon père ; et parce que tu as eu part à toutes les afflictions de mon père.
\VS{27}Ainsi Salomon dépouilla Abiathar de ses fonctions, afin qu'il ne fût plus sacrificateur de Yahweh pour accomplir la parole de Yahweh, qu'il avait prononcée à Silo contre la maison d'Eli.
\TextTitle{Mort de Joab ; Benaja à la tête de l’armé}
\VS{28}Le bruit en parvint à Joab, qui avait suivi le parti d’Adonija, quoiqu'il n’eût pas suivi le parti d’Absalom. Joab s'enfuit au tabernacle de Yahweh et empoigna les cornes de l'autel.
\VS{29}On alla l’apprendre au roi Salomon, en disant : Joab s'en est enfui dans la tente de Yahweh et il est auprès de l'autel. Salomon envoya Benaja, fils de Jehojada, et lui dit : Va et frappe-le.
\VS{30}Benaja entra dans la tente de Yahweh et dit à Joab : Ainsi a parlé le roi : Sors de là ! Mais il répondit : Non ! Je veux mourir ici. Et Benaja rapporta la chose au roi en disant : Joab m'a parlé ainsi et c’est ainsi qu’il m’a répondu.
\VS{31}Et le roi dit à Benaja : Fais comme il t'a dit, frappe-le et enterre-le ; tu ôteras ainsi de dessus moi et de dessus la maison de mon père le sang que Joab a répandu sans cause.
\VS{32}Et Yahweh fera retomber son sang sur sa tête, car il a frappé deux hommes plus justes et meilleurs que lui et les a tués par l'épée, sans que mon père David n’en sût rien : Abner, fils de Ner, chef de l'armée d'Israël, et Amassa, fils de Jéther, chef de l'armée de Juda.
\VS{33}Leur sang retombera sur la tête de Joab et sur la tête de sa postérité à perpétuité ; mais il y aura paix à toujours de par Yahweh, pour David, pour sa postérité, pour sa maison et pour son trône.
\VS{34}Donc Benaja, fils de Jehojada, monta, et il frappa Joab à mort. On l'ensevelit dans sa maison, dans le désert.
\VS{35}Alors le roi établit Benaja, fils de Jehojada, sur l'armée à la place de Joab ; le roi établit aussi Tsadok sacrificateur à la place d'Abiathar.
\TextTitle{Mort de Schimeï}
\VS{36}Puis le roi fit appeler Schimeï et lui dit : Bâtis-toi une maison à Jérusalem, et demeures-y, et n'en sors point pour aller de côté ou d'autre.
\VS{37}Car sache que le jour où tu en sortiras et que tu passeras le torrent de Cédron, tu mourras certainement ; ton sang sera sur ta tête.
\VS{38}Schimeï répondit au roi : Cette parole est bonne ! Ton serviteur fera tout ce que le roi mon Seigneur a dit. Ainsi Schimeï demeura à Jérusalem plusieurs jours.
\VS{39}Mais il arriva qu'au bout de trois ans, deux serviteurs de Schimeï s'enfuirent vers Akisch, fils de Maaca, roi de Gath, et on le rapporta à Schimeï en disant : Voilà tes serviteurs sont à Gath.
\VS{40}Alors Schimeï se leva, sella son âne, et s'en alla à Gath vers Akisch pour chercher ses serviteurs. Schimeï s'en alla donc et ramena de Gath ses serviteurs.
\VS{41}On rapporta à Salomon que Schimeï était allé de Jérusalem à Gath, et qu'il était de retour.
\VS{42}Et le roi envoya appeler Schimeï, et lui dit : Ne t'avais-je pas fait jurer par Yahweh, et ne t'avais-je pas fait cette déclaration formelle : Sache-le, sache bien que le jour que tu sortiras pour aller de côté ou d’autre, tu mourras ? Et ne me répondis-tu pas : La parole que j'ai entendue est bonne ?
\VS{43}Pourquoi donc n'as-tu pas observé le serment que tu as fait par Yahweh et le commandement que je t'avais donné ?
\VS{44}Le roi dit aussi à Schimeï : Tu sais en ton cœur tout le mal que tu as fait à David, mon père ; c'est pourquoi Yahweh a fait retomber ta méchanceté sur ta tête.
\VS{45}Mais le roi Salomon sera béni, et le trône de David sera affermi devant Yahweh à jamais.
\VS{46}Et le roi donna commission à Benaja fils de Jehojada, qui sortit, et frappa Schimeï et Schimeï mourut. La royauté fut ainsi affermie entre les mains de Salomon.
\Chap{3}
\TextTitle{Salomon s'allie à Pharaon}
\VerseOne{}Or, Salomon s'allia avec Pharaon roi d'Egypte. Il prit pour femme la fille de Pharaon, et l'amena en la cité de David, jusqu'à ce qu'il eût achevé de bâtir sa maison, la maison de Yahweh, et la muraille de Jérusalem tout alentour.
\VS{2}Seulement le peuple sacrifiait dans les hauts lieux, parce que jusqu’alors on n’avait pas bâti de maison au nom de Yahweh.
\Chap{3}
\TextTitle{Salomon demande la sagesse à Yahweh\FTNTT{2 Ch. 1:2-10}}
\VS{3}Salomon aimait Yahweh, il marchait selon les ordonnances de David son père. Seulement, c’était sur les hauts lieux qu’il offrait des sacrifices et des parfums.
\VS{4}Le roi se rendit à Gabaon pour y sacrifier, car c'était le plus grand des hauts lieux. Et Salomon offrit mille holocaustes sur cet autel.
\VS{5}Et Yahweh apparut de nuit à Salomon à Gabaon dans un songe, et Dieu lui dit : Demande ce que tu veux que je te donne.
\VS{6}Et Salomon répondit : Tu as usé d'une grande bienveillance envers ton serviteur David, mon père, parce qu’il a marché devant toi fidèlement, dans la justice, et dans la droiture de cœur envers toi. Tu as gardé cette grande bienveillance envers lui en lui donnant un fils qui est assis sur son trône, comme on le voit aujourd'hui.
\VS{7}Or, maintenant, ô Yahweh mon Dieu ! Tu as fait régner ton serviteur à la place de David, mon père, et je ne suis qu'un jeune homme, je ne sais comment me conduire.
\VS{8}Ton serviteur est parmi ce peuple que tu as choisi, un peuple nombreux qui ne peut être compté ni dénombré à cause de sa multitude.
\VS{9}Accorde donc à ton serviteur un cœur intelligent pour juger ton peuple, pour discerner le bien du mal ! Car qui pourrait juger ce peuple si grand ?
\TextTitle{Yahweh exauce Salomon\FTNTT{2 Ch. 1:11-13}}
\VS{10}Cette demande de Salomon plut à Yahweh.
\VS{11}Et Dieu lui dit : Puisque c’est là ta demande et que tu n'as point demandé une longue vie, ni les richesses, ni la mort de tes ennemis, mais que tu as demandé de l'intelligence pour rendre justice,
\VS{12}voici, je fais selon ta parole. Voici, je te donne un cœur sage et intelligent, de sorte qu'il n'y aura eu personne de semblable avant toi et qu’il n'y en aura jamais de semblable après toi.
\VS{13}Et même, je te donne ce que tu n'as point demandé, les richesses et la gloire, de sorte qu'il n'y aura point de roi semblable à toi entre les rois, tant que tu vivras.
\VS{14}Et si tu marches dans mes voies pour garder mes ordonnances et mes commandements, comme David, ton père, je prolongerai tes jours.
\VS{15}Salomon s’éveilla. Et voilà le songe. Puis il s'en retourna à Jérusalem et se tint devant l'Arche de l'alliance de Yahweh. Là, il offrit des holocaustes et des offrandes de paix et fit un festin à tous ses serviteurs.
\VS{16}Alors deux femmes prostituées vinrent au roi et se présentèrent devant lui.
\VS{17}Et l'une de ces femmes dit : Hélas, mon Seigneur ! Nous demeurions cette femme-ci et moi dans une même maison et j'ai accouché près d’elle dans cette maison-là.
\VS{18}Trois jours après, cette femme a aussi accouché. Et nous étions ensemble, il n'y avait aucun étranger avec nous dans cette maison, il n’y avait que nous deux.
\VS{19}Or, l'enfant de cette femme est mort la nuit, parce qu'elle s'était couchée sur lui.
\VS{20}Elle s'est levée au milieu de la nuit, et a pris mon fils à mes côtés pendant que ta servante dormait, et l'a couché dans son sein. Et son fils mort, elle l’a couché dans mon sein.
\VS{21}Le matin, je me suis levée pour allaiter mon fils. Et voici, il était mort. Je l’ai regardé attentivement ce matin-là ; et voici, ce n'était point mon fils que j'avais enfanté.
\VS{22}L’autre femme dit : Non, c’est mon fils qui est vivant, et c’est ton fils qui est mort. Mais la première répliqua : Nullement ! Celui qui est mort est ton fils, et c’est mon fils qui vit. Elles parlaient ainsi devant le roi.
\VS{23}Et le roi dit : L’une dit : C’est mon fils qui est vivant, et c’est ton fils qui est mort ; l’autre dit : Nullement ! C’est ton fils qui est mort, et c’est mon fils qui est vivant.
\VS{24}Alors le roi dit : Apportez-moi une épée ! Et on apporta une épée devant le roi.
\VS{25}Puis le roi dit : Partagez en deux l'enfant qui vit, et donnez-en la moitié à l'une et la moitié à l'autre.
\VS{26}Alors la femme dont le fils était vivant sentit ses entrailles s’émouvoir pour son fils, et elle dit au roi : Ah ! Mon seigneur, qu'on donne à celle-ci l'enfant qui vit et qu'on ne le fasse pas mourir ! Mais l'autre dit : Il ne sera ni à moi ni à toi ; qu'on le partage.
\VS{27}Alors le roi répondit et dit : Donnez à la première l'enfant qui vit, et ne le faites pas mourir. C’est elle qui est sa mère.
\VS{28}Tout Israël entendit parler du jugement que le roi avait prononcé. Et l’on craignit le roi, car l’on reconnut que la sagesse divine était en lui pour rendre justice.
\Chap{4}
\TextTitle{Salomon établit onze chefs et douze intendants}
\VerseOne{}Le roi Salomon était roi sur tout Israël.
\VS{2}Voici les chefs qu’il avait à son service. Azaria, fils du sacrificateur Tsadok,
\VS{3}Elihoreph et Achija, enfants de Schischa, secrétaires ; Josaphat, fils d'Achilud, archiviste ;
\VS{4}Benaja, fils de Jehojada, commandait l'armée ; Tsadok et Abiathar étaient sacrificateurs ;
\VS{5}Azaria, fils de Nathan était chef des intendants ; Zabud, fils de Nathan, était le ministre d’état, favori du roi ;
\VS{6}Achischar, chef de la maison du roi ; et Adoniram, fils d’Abda, préposé sur les impôts.
\VS{7}Or, Salomon avait douze intendants sur tout Israël, qui veillaient à l’entretien du roi et de sa maison ; et chacun pendant un mois de l'année.
\VS{8}Voici leurs noms : Le fils de Hur, sur la montagne d'Ephraïm.
\VS{9}Le fils de Déker, sur Makats, sur Saalbim, sur Beth-Schémesch, à Elon de Beth-Hanan.
\VS{10}Le fils de Hésed, à Arubboth ; il avait Soco et tout le pays de Hépher.
\VS{11}Le fils d'Abinadab avait toute la contrée de Dor ; il avait Thaphath, fille de Salomon, pour femme.
\VS{12}Baana, fils d'Achilud, avait Thaanac et Meguiddo, et tout le pays de Beth-Schean qui est près de Tsarthan au-dessous de Jizreel, depuis Beth-Schean jusqu'à Abel-Mehola et jusqu'au-delà de Jokmeam.
\VS{13}Le fils de Guéber, à Ramoth en Galaad ; il avait les bourgs de Jaïr, fils de Manassé, en Galaad ; il avait aussi toute la contrée d'Argob en Basan, soixante grandes villes à murailles et garnies de barres d'airain.
\VS{14}Achinadab, fils d’Iddo, à Mahanaïm.
\VS{15}Achimaats, qui avait pour femme Basmath, fille de Salomon, en Nephthali.
\VS{16}Baana, fils de Huschaï, en Aser et sur Bealoth.
\VS{17}Josaphat, fils de Paruach, à Issacar.
\VS{18}Schimeï, fils d'Ela, en Benjamin.
\VS{19}Guéber, fils d'Uri, dans le pays de Galaad, le pays de Sihon, roi des Amoréens, et d’Og, roi de Basan ; et il était seul intendant de ce pays-là.
\TextTitle{L'étendue de la domination du royaume}
\VS{20}Juda et Israël étaient en grand nombre, semblable au sable sur le bord de la mer ; ils mangeaient, buvaient et se réjouissaient.
\VS{21}Et Salomon dominait sur tous les royaumes depuis le fleuve jusqu'au pays des Philistins et jusqu'à la frontière d'Egypte ; ils apportaient des présents, et lui furent assujettis pendant toute sa vie.
\VS{22}Or, les vivres de Salomon pour chaque jour étaient de trente cors de fine farine et soixante d'autre farine,
\VS{23}dix bœufs gras, vingt bœufs de pâturages, et cent moutons, outre les cerfs, les daims et les volailles engraissées.
\VS{24}Il dominait sur toutes les contrées de l’autre côté du fleuve, depuis Thiphsach jusqu'à Gaza, sur tous les rois qui étaient de l’autre côté du fleuve. Il était en paix avec tous les pays alentour.
\VS{25}Juda et Israël habitèrent en sécurité chacun sous sa vigne et sous son figuier, depuis Dan jusqu'à Beer-Schéba, durant toute la vie de Salomon.
\VS{26}Salomon avait aussi quarante mille crèches pour les chevaux destinés à ses chars et douze mille hommes de cheval.
\VS{27}Or, les intendants pourvoyaient à l’entretien du roi Salomon et de tous ceux qui s'approchaient de sa table, chacun en son mois ; ils ne les laissaient manquer de rien.
\VS{28}Ils faisaient aussi venir de l'orge et de la paille pour les chevaux et les coursiers dans le lieu où se trouvait le roi, chacun selon les ordres qu'il avait reçus.
\TextTitle{La sagesse de Salomon connue de toute la terre}
\VS{29}Dieu donna à Salomon de la sagesse, une très grande intelligence, et des connaissances multipliées comme le sable qui est sur le bord de la mer.
\VS{30}La sagesse de Salomon surpassait la sagesse de tous les fils de l’orient et toute la sagesse des égyptiens.
\VS{31}Il était plus sage qu’aucun homme, plus qu'Ethan, l’Ezrachite, plus qu'Héman, Calcol et Darda, les fils de Machol ; et sa renommée était répandue parmi toutes les nations d'alentour.
\VS{32}Il a prononcé trois mille paraboles et composa cinq mille cantiques.
\VS{33}Il a aussi parlé des arbres, depuis le cèdre du Liban jusqu'à l'hysope qui sort de la muraille ; il a aussi parlé sur les animaux, sur les oiseaux, sur les reptiles et sur les poissons.
\VS{34}Il venait des gens d'entre tous les peuples pour entendre la sagesse de Salomon, de la part de tous les rois de la terre qui avaient entendu parler de sa sagesse.
\Chap{5}
\TextTitle{Salomon prépare la construction du temple\FTNTT{2 Ch. 2:1 ; 13:16}}
\VerseOne{}Hiram, roi de Tyr, envoya ses serviteurs vers Salomon, car il apprit qu'on l'avait oint pour roi à la place de son père, car Hiram avait toujours aimé David.
\VS{2}Et Salomon fit dire à Hiram :
\VS{3}Tu sais que David, mon père, n'a pu bâtir une maison à Yahweh, son Dieu, à cause des guerriers qui l'ont encerclé, jusqu'à ce que Yahweh les ait mis sous la plante de ses pieds.
\VS{4}Maintenant Yahweh, mon Dieu, m'a donné du repos de toutes parts, et je n'ai plus d’adversaires, plus de calamités !
\VS{5}Voici donc j’ai l’intention de bâtir une maison au nom de Yahweh, mon Dieu, comme Yahweh l’a promis à David, mon père, en disant : Ton fils que je mettrai à ta place sur ton trône sera celui qui bâtira une maison à mon nom.
\VS{6}Ordonne maintenant que l’on coupe des cèdres du Liban pour moi. Mes serviteurs seront avec les tiens, et je donnerai pour tes serviteurs le salaire que tu auras fixé ; car tu sais qu'il n'y a personne parmi nous qui sache couper le bois comme les Sidoniens.
\VS{7}Lorsque Hiram eut entendu les paroles de Salomon, il eut une grande joie et il dit : Béni soit aujourd'hui Yahweh, qui a donné à David un fils sage pour chef de ce grand peuple !
\VS{8}Hiram fit répondre à Salomon : J'ai entendu ce que tu m'as envoyé dire et je ferai tout ce qui te plaira au sujet des bois de cèdre et des bois de cyprès.
\VS{9}Mes serviteurs les descendront du Liban à la mer, puis je les expédierai sur la mer par radeaux jusqu'au lieu que tu m'auras indiqué ; là je les ferai délier, et tu les prendras. Ce que je désire en retour, c’est que tu fournisses des vivres à ma maison.
\VS{10}Hiram donna du bois de cèdre et du bois de cyprès à Salomon autant qu'il en voulait.
\VS{11}Et Salomon donna à Hiram vingt mille cors de froment pour la nourriture de sa maison et vingt cors d'huile d’olives concassées ; Salomon en donna autant à Hiram chaque année.
\VS{12}Et Yahweh donna de la sagesse à Salomon, comme il le lui avait promis ; et il y eut paix entre Hiram et Salomon, et ils firent alliance ensemble.
\TextTitle{Les hommes de corvée\FTNTT{2 Ch. 2:2 ; 17:18}}
\VS{13}Le roi Salomon leva sur tout Israël des hommes de corvée ; ils étaient au nombre de trente mille hommes.
\VS{14}Il en envoya dix mille au Liban chaque mois, tour à tour, ils étaient un mois au Liban, et deux mois chez eux. Adoniram était préposé sur les hommes de corvée.
\VS{15}Salomon avait aussi soixante-dix mille hommes qui portaient les fardeaux et quatre-vingt mille qui taillaient les pierres dans la montagne,
\VS{16}sans compter les chefs au nombre de trois mille trois cents, préposés par Salomon sur le suivi des travaux, et chargés de surveiller les ouvriers.
\VS{17}Le roi ordonna d’extraire de grandes et précieuses pierres, pour faire le fondement de la maison, qui soient toutes taillées,
\VS{18}de sorte que les maçons de Salomon et ceux d'Hiram, taillèrent les pierres et préparèrent le bois et les pierres pour bâtir la maison.
\Chap{6}
\TextTitle{Construction du temple de Yahweh\FTNTT{2 Ch. 3:1-14}}
\VerseOne{}Ce fut la quatre cent quatre-vingtième année après la sortie des enfants d'Israël du pays d'Egypte que Salomon bâtit la maison de Yahweh\FTNTT{Voir les annexes «~Le temple de Salomon~»}, la quatrième année du règne de Salomon sur Israël, au mois de Ziv, qui est le second mois.
\VS{2}La maison que le roi Salomon bâtit à Yahweh avait soixante coudées de long, vingt de large, et trente de haut.
\VS{3}Le portique devant le temple de la maison avait vingt coudées de longueur, répondant à la largeur de la maison, et il avait dix coudées de profondeur sur le devant de la maison.
\VS{4}Il fit placer des fenêtres à la maison, fenêtres solidement grillées.
\VS{5}Il bâtit contre la muraille de la maison, à l’entour, des étages qui entouraient les murs de la maison, le temple et le sanctuaire ainsi il fit des chambres latérales tout autour.
\VS{6}L’étage inférieur était large de cinq coudées, celui du milieu de six coudées et le troisième de sept coudées ; car il avait aménagé des retraites à la maison tout autour en dehors, afin que la charpente n'entrât pas dans les murailles de la maison.
\VS{7}Pour bâtir la maison, on se servit de pierres déjà taillées, de sorte qu'en bâtissant la maison on n'entendit ni marteau, ni hache, ni aucun outil de fer.
\VS{8}L'entrée des chambres de l’étage inférieur était au côté droit de la maison, et on montait à l’étage du milieu par un escalier tournant, et de l’étage du milieu au troisième.
\VS{9}Après avoir achevé de bâtir la maison, Salomon couvrit la maison de planches et de poutres de cèdre.
\VS{10}Et il bâtit les étages joignant toute la maison, avec chacun cinq coudées de haut, et il les lia à la maison par des bois de cèdre.
\VS{11}Alors la parole de Yahweh fut adressée à Salomon, en ces termes :
\VS{12}Quant à cette maison que tu bâtis, si tu marches dans mes statuts, si tu pratiques mes ordonnances et que tu gardes tous mes commandements pour y marcher, j’accomplirai en ta faveur la parole que j'ai dite à David, ton père.
\VS{13}Et j'habiterai au milieu des enfants d'Israël, et je n'abandonnerai point mon peuple d'Israël.
\VS{14}Ainsi Salomon bâtit la maison et l'acheva.
\VS{15}Il revêtit de cèdre les murs de la maison, depuis le sol jusqu'au plafond ; il revêtit ainsi de bois l’intérieur, et il couvrit le sol de la maison de planches de cyprès.
\VS{16}Il revêtit aussi l'espace de vingt coudées de planches de cèdre à partir du fond de la maison, depuis le sol jusqu'au haut des murailles, et il bâtit cet espace au dedans pour en faire le sanctuaire, le saint des saints.
\VS{17}Les quarante coudées sur le devant formaient la maison, c’est-à-dire le temple.
\VS{18}Le bois de cèdre à l’intérieur de la maison était sculpté en coloquintes et en fleurs épanouies ; tout l’intérieur était de cèdre, on ne voyait aucune pierre.
\VS{19}Salomon disposa aussi le sanctuaire, au dedans de la maison vers le fond, pour y mettre l'arche de l'alliance de Yahweh.
\VS{20}Le sanctuaire avait par devant vingt coudées de long, vingt coudées de large, et vingt coudées de haut, et on le couvrit d’or pur ; on en couvrit aussi l'autel, fait de planches de bois de cèdre.
\VS{21}Salomon couvrit d’or pur l’intérieur de la maison, et fit passer un voile avec des chaînes d'or au-devant du sanctuaire, qu’il couvrit également d'or.
\VS{22}Ainsi il couvrit d'or la maison tout entière. Il couvrit aussi d'or tout l'autel qui était devant le sanctuaire.
\VS{23}Et il fit dans le sanctuaire deux chérubins de bois d'olivier sauvage, qui avaient chacun dix coudées de haut.
\VS{24}Chacune des ailes de l'un des chérubins avait cinq coudées et les ailes de l’autre chérubin avaient aussi cinq coudées ; depuis le bout d'une aile jusqu'au bout de l'autre aile il y avait donc dix coudées.
\VS{25}Le second chérubin était aussi de dix coudées. Les deux chérubins étaient d'une même mesure et taillés l'un comme l'autre.
\VS{26}La hauteur de chacun des deux chérubins était de dix coudées.
\VS{27}Salomon plaça les chérubins à l’intérieur, au milieu de la maison. Les ailes des chérubins étaient déployées : l'aile de l'un touchait à l’un des murs, l'aile de l'autre chérubin touchait à l'autre mur ; et leurs autres ailes se rencontraient par l’extrémité au milieu de la maison.
\VS{28}Salomon couvrit d'or les chérubins.
\VS{29}Il fit sculpter sur tout le pourtour des murs de la maison, à l’intérieur et à l’extérieur, des sculptures en relief de chérubins, des palmes et des fleurs épanouies.
\VS{30}Il couvrit aussi d'or le sol de la maison, tant à l’intérieur qu’au-dehors.
\VS{31}A l'entrée du sanctuaire, il fit une porte à deux battants de bois d'olivier sauvage, dont les linteaux avec les poteaux équivalaient à un cinquième du mur.
\VS{32}Les deux battants étaient de bois d'olivier sauvage. Il y fit sculpter des chérubins, des palmes et des fleurs épanouies qu’il couvrit d'or, étendant également l'or sur les chérubins et sur les palmes.
\VS{33}Il fit aussi, à l'entrée du temple, des poteaux de bois d'olivier sauvage, du quart de la dimension du mur.
\VS{34}Les deux battants étaient de bois de sapin ; chacun des battants était formé de deux planches brisées.
\VS{35}Il y fit sculpter des chérubins, des palmes et des fleurs épanouies, et les couvrit d'or, proprement posé sur la sculpture.
\VS{36}Il bâtit aussi le parvis de l’intérieur de trois rangées de pierres de taille et d'une rangée de poutres de cèdre.
\VS{37}La quatrième année, au mois de Ziv, les fondements de la maison de Yahweh furent posés.
\VS{38}Et la onzième année, au mois de Bul, qui est le huitième mois, la maison fut achevée dans toutes ses parties et telle qu’elle devait être. Salomon la construisit en l’espace de sept années.
\Chap{7}
\TextTitle{Construction du palais royal}
\VerseOne{}Salomon bâtit aussi sa maison, et l'acheva complètement en treize ans.
\VS{2}Il bâtit d’abord la maison de la forêt du Liban, de cent coudées de long, de cinquante coudées de large, et de trente coudées de haut, sur quatre rangées de colonnes de cèdre ; et sur les colonnes il y avait des poutres de cèdre.
\VS{3}On couvrit de bois de cèdre les chambres qui portaient sur les colonnes qui étaient au nombre de quarante-cinq, quinze par étages.
\VS{4}Et il y avait trois rangées de fenêtrages ; et une fenêtre répondait à l'autre en trois endroits.
\VS{5}Toutes les portes et tous les poteaux étaient formés de poutres carrées, avec les fenêtres ; et à chacun des trois étages, les ouvertures étaient en vis-à-vis les unes des autres.
\VS{6}Il fit aussi le portique de colonnes, long de cinquante coudées, et large de trente coudées ; et un autre portique en avant avec des colonnes et des degrés sur leur front.
\VS{7}Il fit aussi le portique du trône sur lequel il rendait ses jugements, appelé le portique du jugement ; on le couvrit de cèdre depuis un bout du sol jusqu'à l'autre.
\VS{8}La maison où il demeurait fut construite de la même manière, dans une autre cour, derrière le portique. Salomon fit une maison bâtie comme ce portique à la fille de Pharaon, qu'il avait prise pour femme.
\VS{9}Toutes ces constructions étaient de pierres de prix, taillées d’après des mesures, sciées à la scie, en dedans et en dehors, depuis les fondements jusqu'aux corniches, et par dehors jusqu'au grand parvis.
\VS{10}Le fondement était en pierres magnifiques et de grand prix, de grandes pierres, des pierres de dix coudées et des pierres de huit coudées.
\VS{11}Et par-dessus il y avait des pierres de prix, taillées d’après des mesures, et du bois de cèdre.
\VS{12}Et le grand parvis avait aussi tout alentour trois rangées de pierres de taille et une rangée de poutres de cèdre, comme le parvis intérieur de la maison de Yahweh, et le portique de la maison.
\TextTitle{Hiram, artisan spécialiste en airain\FTNTT{2 Ch. 2:12-13}}
\VS{13}Or, le roi Salomon fit venir de Tyr Hiram ;
\VS{14}fils d'une femme veuve de la tribu de Nephthali, et d’un père tyrien, Hiram travaillait le cuivre ; fort expert, intelligent et savant pour faire toutes sortes d'ouvrages d'airain ; il arriva auprès du roi Salomon, et il fit tout son ouvrage.
\TextTitle{Les colonnes du temple\FTNTT{2 Ch. 3:15-17}}
\VS{15}Il fit les deux colonnes d'airain, la première avait dix-huit coudées de hauteur ; et un cordon de douze coudées mesurait le tour de la seconde.
\VS{16}Il fit aussi deux chapiteaux d'airain fondu pour mettre sur les sommets des colonnes ; le premier chapiteau était de cinq coudées de hauteur, le second était aussi de cinq coudées.
\VS{17}Il fit des treillis en forme de maillages, des festons façonnés en forme de chaînes, pour les chapiteaux qui étaient sur le sommet des colonnes, sept pour le premier des chapiteaux, et sept pour le second.
\VS{18}Il fit deux rangs de grenades autour de l’un des treillis, pour couvrir le chapiteau qui était sur le sommet d'une des colonnes ; et il fit de même pour l'autre chapiteau.
\VS{19}Dans le portique, les chapiteaux qui étaient sur le sommet des colonnes figuraient des fleurs de lis hautes de quatre coudées au porche.
\VS{20}Ces chapiteaux placés sur les deux colonnes étaient entourés de deux cents grenades, en haut, depuis le renflement qui était au-delà du treillis ; il y avait aussi deux cents grenades, disposées par rangs, autour du second chapiteau.
\VS{21}Il dressa donc les colonnes au portique du temple. Il dressa la colonne de droite qu’il nomma Jakin ; puis il dressa la colonne de gauche qu’il nomma Boaz.
\VS{22}Et l’on mit sur le chapiteau des colonnes l'ouvrage figurant des fleurs de lis ; ainsi l'ouvrage des colonnes fut achevé.
\TextTitle{La mer de fonte\FTNTT{2 Ch. 4:2-5}}
\VS{23}Il fit aussi la mer de fonte. Elle avait dix coudées d'un bord à l'autre, ronde tout autour, avec cinq coudées de haut ; et un cordon de trente coudées en mesurait le tour.
\VS{24}Au-dessous de son bord, des coloquintes l'environnaient, dix à chaque coudée, lesquelles faisaient tout le tour de la mer. Il y avait deux rangées de coloquintes, jetées en fonte.
\VS{25}Et elle était posée sur douze bœufs, dont trois regardaient le nord et trois regardaient l'occident, trois regardaient le sud et trois regardaient l'orient. La mer était sur eux et toute la partie postérieure de leur corps était tournée en dedans.
\VS{26}Son épaisseur était d'une paume, et son bord était comme le bord d'une coupe en fleur de lis ; elle contenait deux mille baths.
\TextTitle{Les dix socles d'airain}
\VS{27}Il fit aussi dix socles d'airain, ayant chacun quatre coudées de long, quatre coudées de large et trois coudées de haut.
\VS{28}Ces socles étaient réalisés de telle manière qu'il y avait des panneaux enchâssés entre leurs bordures.
\VS{29}Sur les panneaux qui étaient entre les bordures, il y avait des lions, des bœufs et des chérubins. Et sur les bordures, au-dessus et en dessous des lions et des bœufs, il y avait des ornements qui pendaient en festons.
\VS{30}Chaque socle avait quatre roues d'airain avec des essieux d'airain. Ses quatre pieds leur servaient d’appuis. Ces appuis étaient fondus au-dessous de la cuve, et au-dessus étaient les festons.
\VS{31}Le couronnement offrait à son intérieur une ouverture avec un prolongement d'une coudée vers le haut ; cette ouverture était arrondie comme pour les ouvrages de ce genre et elle avait une coudée et demie de largeur. Il s’y trouvait aussi des sculptures ; les panneaux étaient carrés, et non arrondis.
\VS{32}Les quatre roues étaient sous les panneaux, et les essieux des roues fixés à la base ; chaque roue était haute d'une coudée et demie.
\VS{33}Les roues étaient faites comme les roues de chars ; leurs essieux, leurs jantes, leurs rais et leurs moyeux étaient tous de fonte.
\VS{34}Il y avait aux quatre angles de chaque socle quatre consoles d’une même pièce que la base.
\VS{35}La partie supérieure de la base se terminait par un cercle d’une demi-coudée de hauteur, et elle avait ses appuis et ses panneaux de la même pièce.
\VS{36}Puis, on sculpta sur la surface de ses appuis et sur ses panneaux, des chérubins, des lions et des palmes, selon les espaces libres, et des ornements tout autour.
\VS{37}Ainsi les dix socles étaient tous d’une même fonte, d’une même mesure et d’une même forme.
\TextTitle{Les dix cuves d'airain\FTNTT{2 Ch. 4:6}}
\VS{38}Il fit aussi dix cuves d'airain, dont chacune contenait quarante baths, et chaque cuve était de quatre coudées, chaque cuve était sur l’un des dix socles.
\VS{39}Il mit cinq socles au côté droit de la maison, et cinq au côté gauche de la maison ; quant à la mer, il l’a mis au côté droit de la maison, vers l'orient du côté sud.
\TextTitle{Totalité de l’œuvre d’Hiram}
\VS{40}Ainsi Hiram fit les cuves, les pelles et les bassins, et il acheva tout l'ouvrage qu'il faisait au roi Salomon pour la maison de Yahweh.
\VS{41}Savoir, deux colonnes avec les deux chapiteaux qui étaient sur le sommet des colonnes ; et deux maillages pour couvrir les deux bourrelets des chapiteaux qui étaient sur le sommet des colonnes ;
\VS{42}les quatre cents grenades pour les deux maillages, deux rangs de grenades pour chaque réseau, pour couvrir les deux renflements des chapiteaux, qui étaient sur les colonnes ;
\VS{43}les dix socles ; et les dix cuves pour mettre sur les socles ;
\VS{44}la mer avec les douze bœufs sous la mer ;
\VS{45}les pots, les pelles et les bassins. Tous ces ustensiles que Hiram fit au roi Salomon pour la maison de Yahweh étaient d'airain poli.
\VS{46}Le roi les fit fondre dans la plaine du Jourdain, dans un sol argileux, entre Succoth et Tsarthan.
\VS{47}Et Salomon ne pesa aucun de ces ustensiles, parce qu'ils étaient en trop grand nombre, de sorte qu'on ne rechercha point le poids de l’airain.
\TextTitle{Divers ustensiles d’or pour la maison de Yahweh}
\VS{48}Salomon fit aussi tous les ustensiles pour la maison de Yahweh, savoir l'autel d'or, et les tables d'or, sur lesquelles étaient les pains de proposition ;
\VS{49}les chandeliers d’or pur, cinq à droite et cinq à gauche devant le sanctuaire, avec les fleurs, les lampes et les mouchettes d'or ;
\VS{50}les coupes, les couteaux, les bassins, les tasses et les brasiers d’or pur. Les gonds, même des portes de la maison, à l’entrée du saint des saints, à la porte de la maison et à l’entrée du temple, étaient d'or.
\VS{51}Ainsi fut achevé tout l'ouvrage que le roi Salomon fit pour la maison de Yahweh ; puis il y fit apporter l'or, l’argent et les ustensiles que David, son père, avait consacrés ; il les mit dans les trésors de la maison de Yahweh.
\Chap{8}
\TextTitle{L’arche de l’alliance placée dans le saint des saints ; la gloire de Yahweh remplit le temple \FTNTT{2 Ch. 5:2-14}}
\VerseOne{}Alors le roi Salomon convoqua près de lui à Jérusalem les anciens d'Israël, tous les chefs des tribus et les chefs de famille des fils d'Israël, pour transporter l'arche de l'alliance de Yahweh de la cité de David, qui est Sion.
\VS{2}Tous les hommes d'Israël s’assemblèrent auprès du roi Salomon, au mois d'Ethanim, qui est le septième mois, pendant la fête.
\VS{3}Une fois tous les anciens d'Israël arrivés, les sacrificateurs portèrent l'arche.
\VS{4}Ils transportèrent l'arche de Yahweh, la tente d'assignation, et tous les ustensiles qui étaient dans le tabernacle ; les sacrificateurs et les Lévites les emportèrent.
\VS{5}Le roi Salomon et toute l'assemblée d'Israël convoquée auprès de lui se tinrent devant l'arche. Ils sacrifièrent du gros et du menu bétail en si grand nombre, qu'on ne pouvait ni nombrer ni compter.
\VS{6}Et les sacrificateurs portèrent l'arche de l'alliance de Yahweh à sa place, dans le sanctuaire de la maison, dans le saint des saints, sous les ailes des chérubins.
\VS{7}Car les chérubins avaient les ailes étendues sur l’emplacement de l'arche, et ils couvraient l'arche et ses barres par-dessus.
\VS{8}On avait donné aux barres une longueur telle que leurs extrémités se voyaient du lieu saint devant le sanctuaire, mais elles ne se voyaient point du dehors. Elles sont demeurées là jusqu'à ce jour.
\VS{9}Il n'y avait rien dans l'arche que les deux tables de pierre que Moïse y déposa en Horeb, lorsque Yahweh fit alliance avec les enfants d'Israël à leur sortie du pays d'Egypte.
\VS{10}Au moment où les sacrificateurs sortirent du lieu saint, la nuée remplit la maison de Yahweh.
\VS{11}Les sacrificateurs ne purent pas y rester pour faire le service, à cause de la nuée ; car la gloire de Yahweh remplissait la maison de Yahweh.
\TextTitle{Discours de Salomon\FTNTT{2 Ch. 6:1-11}}
\VS{12}Alors Salomon dit : Yahweh veut habiter dans l'obscurité !
\VS{13}J'ai achevé de bâtir une maison pour ta demeure ô Yahweh ! Ce sera une demeure, un lieu où tu résideras éternellement.
\VS{14}Le roi tourna son visage, et bénit toute l'assemblée d'Israël ; car toute l'assemblée d'Israël se tenait là debout.
\VS{15}Et il dit : Béni soit Yahweh, le Dieu d'Israël, qui a parlé de sa propre bouche à David, mon père, et qui a accompli par sa puissance ce qu’il avait déclaré en disant :
\VS{16}Depuis le jour où je fis sortir mon peuple d'Israël hors d'Egypte, je n'ai choisi aucune ville d'entre toutes les tribus d'Israël pour y bâtir une maison afin que mon nom y fût, mais j'ai choisi David pour qu’il règne sur mon peuple d'Israël.
\VS{17}David, mon père, avait à cœur de bâtir une maison au nom de Yahweh, le Dieu d'Israël.
\VS{18}Et Yahweh dit à David, mon père : Puisque tu as eu à cœur de bâtir une maison à mon nom, tu as bien fait d’avoir eu cette intention.
\VS{19}Néanmoins, tu ne bâtiras point cette maison, mais ton fils qui sortira de tes entrailles sera celui qui bâtira cette maison à mon Nom.
\VS{20}Yahweh a donc accompli la parole qu'il avait prononcée. Je me suis élevé à la place de David, mon père, et me suis assis sur le trône d'Israël, comme Yahweh l’avait annoncé, et j'ai bâti cette maison au Nom de Yahweh, le Dieu d'Israël.
\VS{21}J'y ai établi ici un lieu pour l'arche, dans lequel est l'alliance de Yahweh, qu'il traita avec nos pères quand il les fit sortir hors du pays d'Egypte.
\TextTitle{Prière de Salomon\FTNTT{2 Ch. 6:12-42}}
\VS{22}Ensuite Salomon se tint devant l'autel de Yahweh en la présence de toute l'assemblée d'Israël, et étendant ses mains vers les cieux,
\VS{23}il dit : Ô Yahweh, Dieu d'Israël ! Il n'y a point de Dieu semblable à toi en haut dans les cieux, ni en bas sur la terre ; tu gardes l'alliance et la miséricorde envers tes serviteurs qui marchent devant ta face de tout cœur !
\VS{24}Ainsi tu as tenu parole à ton serviteur David, mon père, car ce que tu as déclaré de ta bouche, tu l'as accompli en ce jour par ta main puissante.
\VS{25}Maintenant donc, ô Yahweh, Dieu d'Israël, prête attention à la promesse faite à ton serviteur David, mon père, en lui disant : Tu ne manqueras jamais devant moi d’un successeur assis sur le trône d'Israël, pourvu seulement que tes fils prennent garde à leur voie et qu’ils marchent devant ma face, comme tu y as marché.
\VS{26}Et maintenant, ô Dieu d'Israël ! Je te prie, que s’accomplisse la promesse que tu as faite à ton serviteur David, mon père.
\VS{27}Mais Dieu habiterait-il véritablement sur la terre ? Voilà, les cieux, même les cieux des cieux ne peuvent te contenir ; combien moins cette maison que j'ai bâtie !
\VS{28}Toutefois, ô Yahweh, mon Dieu, sois attentif à la prière que t’adresse ton serviteur et à sa supplication, pour entendre le cri et la prière que ton serviteur t’adresse aujourd'hui.
\VS{29}Que tes yeux soient ouverts jour et nuit sur cette maison, sur le lieu dont tu as dit : Là sera mon Nom ! Ecoute la prière que ton serviteur fait en ce lieu.
\VS{30}Daigne exaucer la supplication de ton serviteur et de ton peuple d'Israël lorsqu’ils te prieront en ce lieu ; exauce du lieu de ta demeure. Des cieux, exauce, et pardonne !
\VS{31}Si quelqu'un pèche contre son prochain et qu’on lui impose un serment pour le faire jurer, et que le serment aura été fait devant ton autel dans cette maison ;
\VS{32}écoute-le des cieux, et agis. Juge tes serviteurs, condamne le coupable en lui rendant selon sa conduite ; rends justice à l’innocent, et traite-le selon son innocence !
\VS{33}Quand ton peuple d'Israël sera battu par l'ennemi, pour avoir péché contre toi, s’il revient à toi et rend gloire à ton Nom, en t’adressant des prières et des supplications dans cette maison,
\VS{34}exauce-le des cieux, et pardonne le péché de ton peuple d'Israël, et ramène-le dans la terre que tu as donnée à leurs pères.
\VS{35}Quand les cieux seront fermés et qu'il n'y aura point de pluie, à cause de ses péchés contre toi, s'il te fait une prière en ce lieu-ci, qu’il loue ton Nom, et s’il se détourne de ses péchés, parce que tu les auras affligés,
\VS{36}exauce-le des cieux, pardonne le péché de tes serviteurs et de ton peuple d'Israël, à qui tu enseigneras quel est le chemin par lequel ils doivent marcher et envoie-leur la pluie sur la terre que tu as donnée à ton peuple pour héritage !
\VS{37}Quand il y aura dans le pays, famine, peste, jaunisse, nielle, sauterelles d’une espèce ou d’une autre, même quand les ennemis assiégeront ton peuple dans son propre pays, quand il y aura un fléau ou une maladie quelconque ;
\VS{38}si un homme, si tout ton peuple d'Israël fait entendre des prières et des supplications, que chacun reconnaisse la plaie de son cœur et étende les mains vers cette maison,
\VS{39}exauce-le des cieux, du lieu de ta demeure, pardonne, et agis. Rends à chacun selon toutes ses voies, parce que tu auras connu leurs cœurs ; car toi seul connais le cœur de tous les fils des hommes ;
\VS{40}et ils te craindront toute leur vie dans le pays que tu as donné à nos pères !
\VS{41}Et même lorsque l'étranger, qui n’est pas de ton peuple d'Israël, viendra d'un pays éloigné à cause de ton Nom,
\VS{42}car on saura que ton Nom est grand, ta main puissante et ton bras étendu, quand il viendra prier dans cette maison,
\VS{43}exauce-le des cieux, du lieu de ta demeure, et fais à cet étranger selon ce qu’il t’aura demandé, afin que tous les peuples de la terre connaissent ton Nom pour te craindre, comme ton peuple d'Israël ; et pour connaître que ton Nom est invoqué sur cette maison que j'ai bâtie !
\VS{44}Quand ton peuple sortira pour combattre son ennemi, par la voie par laquelle tu l’auras envoyé, s'ils prient Yahweh en regardant vers cette ville que tu as choisie et vers cette maison que j'ai bâtie à ton Nom !
\VS{45}Exauce des cieux leurs prières et leurs supplications, et fais leur justice !
\VS{46}Quand ils pécheront contre toi, car il n'y a point d'homme qui ne pèche, et que tu seras irrité contre eux et que tu les auras livrés à leurs ennemis, qui les emmènera captifs dans un pays ennemi, lointain ou proche ;
\VS{47}si dans le pays où ils auront été menés captifs, ils reviennent à toi et t’adressent des supplications, se repentent et te prient au pays de ceux qui les auront emmenés captifs, en disant : Nous avons péché, nous avons commis l’iniquité, nous avons fait le mal !
\VS{48}S'ils reviennent à toi de tout leur cœur et de toute leur âme, dans le pays de leurs ennemis, qui les auront emmenés captifs, et s'ils t'adressent leurs prières, les regards tournés vers le pays que tu as donné à leurs pères, vers la ville que tu as choisie, vers la maison que j'ai bâtie à ton Nom,
\VS{49}exauce des cieux, du lieu de ta demeure, leurs prières et leurs supplications, et fais-leur justice.
\VS{50}Pardonne à ton peuple ses offenses et ses péchés envers toi, et fais que ceux qui les auront emmenés captifs aient pitié d'eux et leur fassent grâce,
\VS{51}car ils sont ton peuple et ton héritage, et tu les as fait sortir hors d'Egypte, du milieu d'une fournaise de fer !
\VS{52}Que tes yeux donc soient ouverts sur la supplication de ton serviteur et celle de ton peuple d'Israël, pour les exaucer dans tout ce pourquoi ils crieront à toi !
\VS{53}Car tu les as séparés de tous les autres peuples de la terre pour être ton héritage, comme tu l’as déclaré par Moïse, ton serviteur, quand tu fis sortir nos pères hors d'Egypte, ô Seigneur Yahweh !
\TextTitle{Bénédictions et réjouissances\FTNTT{2 Ch. 7:4-10}}
\VS{54}Lorsque Salomon eut achevé de faire cette prière et cette supplication à Yahweh, il se leva de devant l'autel de Yahweh où il était agenouillé et les mains étendues vers les cieux.
\VS{55}Il se tint debout, et bénit toute l'assemblée d'Israël à haute voix, en disant :
\VS{56}Béni soit Yahweh, qui a donné du repos à son peuple d'Israël, comme il l’avait annoncé ! De toutes les paroles qu'il avait prononcées par le moyen de Moïse, son serviteur, aucune n’est restée sans effet.
\VS{57}Que Yahweh, notre Dieu, soit avec nous, comme il a été avec nos pères ; qu'il ne nous abandonne point et qu'il ne nous délaisse point,
\VS{58}mais qu'il incline nos cœurs vers lui, afin que nous marchions dans toutes ses voies, et que nous observions ses commandements, ses statuts et ses ordonnances, qu'il a prescrits à nos pères !
\VS{59}Que ces paroles, par lesquelles j'ai fait supplication à Yahweh, soient présentes devant Yahweh, notre Dieu, jour et nuit ; afin qu'il fasse justice à son serviteur et à son peuple d’Israël en tout temps,
\VS{60}afin que tous les peuples de la terre reconnaissent que c'est Yahweh qui est Dieu et qu'il n'y en a point d'autre !
\VS{61}Que votre cœur soit intègre envers Yahweh, notre Dieu, comme aujourd'hui, pour marcher dans ses statuts et pour garder ses commandements.
\VS{62}Le roi et tout Israël avec lui offrirent des sacrifices devant Yahweh.
\VS{63}Salomon offrit un sacrifice d'offrande de paix à Yahweh, savoir vingt-deux mille bœufs et cent vingt mille brebis. Ainsi le roi et tous les enfants d'Israël firent la dédicace de la maison de Yahweh.
\VS{64}En ce jour-là, le roi consacra le milieu du parvis, qui est devant la maison de Yahweh ; car il offrit là les holocaustes, les offrandes et les graisses des sacrifices d’offrandes de paix, parce que l'autel d'airain qui est devant Yahweh, était trop petit pour contenir les holocaustes, les offrandes et les graisses des offrandes de paix.
\VS{65}Et en ce temps-là, Salomon célébra une fête solennelle ; et tout Israël avec lui, venu en grande multitude depuis les environs de Hamath jusqu'au torrent d'Egypte, devant Yahweh, notre Dieu, pendant sept jours, et sept autres jours, soit quatorze jours.
\VS{66}Le huitième jour, il renvoya le peuple. Et ils bénirent le roi, et s'en allèrent dans leurs demeures, en se réjouissant, et le cœur heureux pour tout le bien que Yahweh avait fait à David, son serviteur, et à Israël, son peuple.
\Chap{9}
\TextTitle{Yahweh apparaît à Salomon une seconde fois\FTNTT{2 Ch. 7:11-22}}
\VerseOne{}Lorsque Salomon eut achevé de bâtir la maison de Yahweh, la maison royale, et tout ce que Salomon prit plaisir à faire,
\VS{2}Yahweh apparut à Salomon une seconde fois, comme il lui était apparu à Gabaon.
\VS{3}Et Yahweh lui dit : J'exauce ta prière, et la supplication que tu as faite devant moi, j'ai sanctifié cette maison que tu as bâtie pour y mettre mon Nom à jamais, et mes yeux et mon cœur seront toujours là.
\VS{4}Quant à toi, si tu marches devant moi comme David, ton père, a marché, avec intégrité et de cœur et avec droiture, en faisant tout ce que je t'ai commandé, et si tu gardes mes statuts et mes ordonnances,
\VS{5}j’affermirai le trône de ton royaume sur Israël à jamais, comme je l’ai déclaré à David, ton père, en disant : Tu ne manqueras jamais d’un successeur sur le trône d'Israël.
\VS{6}Mais si vous et vos fils, vous vous détournez de moi et que vous ne gardiez pas mes commandements, mes lois que je vous ai prescrites, et si vous allez servir d'autres dieux et vous prosterner devant eux,
\VS{7}je retrancherai Israël de la terre que je lui ai donnée, je rejetterai loin de moi cette maison que j'ai consacrée à mon Nom et Israël sera un sujet de sarcasme et de moquerie parmi tous les peuples.
\VS{8}Et si haut placée qu’ait été cette maison, quiconque passera auprès d'elle sera étonné et sifflera. Et on dira : Pourquoi Yahweh a-t-il ainsi traité ce pays et cette maison ?
\VS{9}Et on répondra : Parce qu'ils ont abandonné Yahweh, leur Dieu, qui avait tiré leurs pères hors du pays d'Egypte, qu'ils se sont attachés à d'autres dieux, se sont prosternés devant eux et les ont servis, voilà pourquoi Yahweh a fait venir sur eux tous ces maux.
\TextTitle{Les réalisations de Salomon\FTNTT{2 Ch. 8:1-18}}
\VS{10}Au bout de vingt ans, Salomon avait bâti les deux maisons, la maison de Yahweh et la maison royale.
\VS{11}Hiram, roi de Tyr, avait fourni à Salomon du bois de cèdre, du bois de sapin et de l'or, autant qu'il en avait voulu, le roi Salomon donna à Hiram vingt villes dans le pays de Galilée.
\VS{12}Hiram sortit de Tyr, pour voir les villes que Salomon lui avait données. Mais elles ne lui plurent point,
\VS{13}et il dit : Quelles villes m'as-tu assignées, mon frère ? Et il les appela, pays de Cabul, nom qu’elles ont conservé jusqu'à ce jour.
\VS{14}Hiram avait aussi envoyé au roi cent vingt talents d'or.
\VS{15}Voici ce qui concerne les hommes de corvée que le roi Salomon leva pour bâtir la maison de Yahweh, sa maison, Millo, la muraille de Jérusalem, Hatsor, Meguiddo et Guézer.
\VS{16}Pharaon, roi d'Egypte, était venu s’emparer de Guézer et l'avait incendiée, il avait tué les Cananéens qui habitaient dans la ville. Puis il la donna pour dot à sa fille, femme de Salomon.
\VS{17}Salomon donc bâtit Guézer, et Beth-Horon la basse,
\VS{18}Baalath et Thadmor, dans le désert qui est au pays,
\VS{19}toutes les villes servant de magasins et lui appartenant, les villes pour les chars et les villes pour la cavalerie, et tout ce qu’il plut à Salomon de bâtir à Jérusalem, au Liban, et dans tout le pays dont il était le souverain.
\VS{20}Tout le peuple qui était resté des Amoréens, des Héthiens, des Phéréziens, des Héviens et des Jébusiens ne faisaient point partie des fils d'Israël,
\VS{21}leurs descendants qui étaient demeurés après eux dans le pays et que les fils d'Israël n'avaient pu dévouer par le moyen de l'interdit, Salomon les fit placer à son service comme gens de corvée à toujours.
\VS{22}Mais Salomon n’employa aucun des fils d'Israël comme esclaves ; car ils étaient ses hommes de guerre, ses serviteurs, ses chefs, ses officiers, les chefs de ses chars et ses hommes d'armes.
\VS{23}Les chefs préposés aux travaux par Salomon étaient au nombre de cinq cent cinquante, lesquels géraient l'intendance des ouvriers.
\VS{24}La fille de Pharaon monta de la cité de David dans la maison que Salomon lui avait bâtie. Ce fut alors qu’il bâtit Millo.
\VS{25}Trois fois par an, Salomon offrait des holocaustes et des offrandes de paix sur l'autel qu'il avait bâti à Yahweh, et il brûlait des parfums sur celui qui était devant Yahweh. Et il acheva la maison.
\VS{26}Le roi Salomon construisit des navires à Etsjon-Guéber, près d'Eloth, sur le rivage de la Mer Rouge, au pays d'Edom.
\VS{27}Et Hiram envoya sur ces navires, auprès des serviteurs de Salomon, ses propres serviteurs, des hommes connaissant la mer.
\VS{28}Ils allèrent en Ophir, et ils prirent de là quatre cent vingt talents d'or qu’ils apportèrent au roi Salomon.
\Chap{10}
\TextTitle{La reine de Séba chez Salomon\FTNTT{2 Ch. 7:1-12}}
\VerseOne{}Or, la reine de Séba ayant appris la renommée de Salomon, à cause du Nom de Yahweh, vint l’éprouver par des énigmes.
\VS{2}Elle entra dans Jérusalem avec une suite fort nombreuse, et avec des chameaux qui portaient des aromates, une grande quantité d'or, et des pierres précieuses. Elle se rendit auprès de Salomon, et lui parla de tout ce qu'elle avait dans le cœur.
\VS{3}Salomon répondit à toutes ses questions, et il n’y eut aucune parole à laquelle le roi ne put fournir une explication.
\VS{4}La reine de Séba vit toute la sagesse de Salomon et la maison qu'il avait bâtie,
\VS{5}les mets de sa table, la demeure de ses serviteurs, l’ordre de service, leurs vêtements, ses échansons, et les holocaustes qu'il offrait dans la maison de Yahweh.
\VS{6}Elle fut toute ravie en elle-même, elle parla ainsi au roi : Ce que j'ai entendu dire dans mon pays au sujet de ta sagesse était donc vrai !
\VS{7}Je ne croyais pas ce qu’on en disait avant d’être venue et que mes yeux ne l'aient vu. Et voici, on ne m'en avait point rapporté la moitié. Ta sagesse et ta prospérité surpassent tout ce que j'en avais entendu.
\VS{8}Heureux sont tes gens ! Heureux tes serviteurs qui se tiennent continuellement devant toi, et qui entendent ta sagesse !
\VS{9}Béni soit Yahweh, ton Dieu, qui t’a accordé la faveur de t’établir sur le trône d'Israël ! Car Yahweh a aimé Israël à toujours ; et t'a établi roi pour faire droit et justice.
\VS{10}Puis elle donna au roi cent vingt talents d'or, une très grande quantité d’aromates et des pierres précieuses. Il ne vint jamais depuis une aussi grande abondance d’aromates que la reine de Séba en donna au roi Salomon.
\VS{11}Et les navires de Hiram, qui amenèrent de l'or d'Ophir, amenèrent aussi d’Ophir une grande quantité de bois de santal et de pierres précieuses.
\VS{12}Le roi fit des supports de ce bois de santal pour la maison de Yahweh et pour la maison royale ; il en fit aussi des harpes et des luths pour les chantres ; il ne vint plus de ce bois de santal et on n’en a plus vu jusqu'à ce jour-là.
\VS{13}Le roi Salomon donna à la reine de Séba tout ce qu'elle désira et répondit à tout et ce qu'elle lui demanda. Il lui fit en outre des présents dignes d'un roi tel que Salomon. Puis elle s'en retourna et alla dans son pays, elle et ses serviteurs.
\TextTitle{Les richesses de Salomon\FTNTT{2 Ch. 9:13-28}}
\VS{14}Le poids de l'or qui revenait à Salomon chaque année, était de six cent soixante-six talents d'or,
\VS{15}outre ce qui lui revenait des négociants, du trafic des marchands, de tous les rois d'Arabie, et des gouverneurs de ce pays-là.
\VS{16}Le roi Salomon fit aussi deux cents grands boucliers d'or battu au marteau, employant six cents sicles d'or pour chaque bouclier,
\VS{17}et trois cents autres boucliers d'or battu au marteau, pour chacun desquels il employa trois mines d'or ; et le roi les mit dans la maison de la forêt du Liban.
\VS{18}Le roi fit aussi un grand trône d'ivoire, qu'il couvrit d’or pur.
\VS{19}Ce trône avait six degrés, et la partie supérieure, le haut du trône était arrondi par derrière. Il y avait des accoudoirs de chaque côté du siège et deux lions se tenaient auprès des accoudoirs.
\VS{20}Il y avait aussi douze lions sur les six degrés du trône, de part et d'autre. Il ne s'est rien fait de tel dans aucun royaume.
\VS{21}Toute la vaisselle du buffet du roi Salomon était d'or, et toutes les coupes de la maison de la forêt du Liban étaient d’or pur. Il n'y en avait point en argent ; on n’en faisait aucun cas du temps de Salomon.
\VS{22}Car le roi avait en mer des navires de Tarsis avec la flotte d'Hiram ; et tous les trois ans la flotte de Tarsis revenait, apportant de l'or, de l'argent, de l'ivoire, des singes et des paons.
\VS{23}Le roi Salomon fut plus grand que tous les rois de la terre, tant en richesses qu'en sagesse.
\VS{24}Tous les habitants de la terre cherchaient à voir la face de Salomon, pour écouter la sagesse que Dieu avait mise en son cœur.
\VS{25}Et chacun d'eux lui apportait son présent, des vases d’or et d'argent, des vêtements, des armes, des aromates, des chevaux et des mulets, tous les ans.
\VS{26}Salomon rassembla ses chars et sa cavalerie ; il y avait mille quatre cents chars et douze mille chevaliers, qu'il plaça dans les villes où il tenait ses chars et à Jérusalem près du roi.
\VS{27}Le roi rendit l'argent aussi commun à Jérusalem que les pierres ; et les cèdres que les sycomores qui croissent dans les plaines, tant il y en avait.
\VS{28}C’est d’Egypte que provenaient les chevaux de Salomon ; une caravane de marchands du roi allait les chercher par troupes, à un prix fixe :
\VS{29}Un char montait et sortait d'Egypte pour six cents sicles d'argent et chaque cheval pour cent cinquante sicles ; ils en amenaient de même avec eux pour tous les rois des Héthiens et pour les rois de Syrie.
\Chap{11}
\TextTitle{Salomon détourne son cœur de Yahweh}
\VerseOne{}Le roi Salomon aima plusieurs femmes étrangères, outre la fille de Pharaon ; savoir des Moabites, des Ammonites, des Edomites, des Sidoniennes et des Héthiennes.
\VS{2}Elles étaient d'entre les nations dont Yahweh avait dit aux enfants d'Israël : Vous n'irez point vers elles, et elles ne viendront point vers vous ; car certainement elles feraient détourner vos cœurs pour suivre leurs dieux. Salomon s'attacha à elles et les aima.
\VS{3}Il eut donc pour femmes sept cents princesses et trois cents concubines ; et ses femmes détournèrent son cœur.
\VS{4}Au temps de la vieillesse de Salomon, ses femmes firent détourner son cœur vers d'autres dieux ; et son cœur ne fut point intègre devant Yahweh, son Dieu, comme David, son père.
\VS{5}Salomon alla après Astarté, la divinité des Sidoniens, et après Milcom, l'abomination des Ammonites.
\VS{6}Ainsi Salomon fit ce qui est mal aux yeux de Yahweh, et il ne persévéra point à suivre Yahweh, comme David, son père.
\VS{7}Et Salomon bâtit un haut lieu à Kemosch, l'abomination des Moabites, sur la montagne qui est vis-à-vis de Jérusalem ; et à Moloc, l'abomination des fils d’Ammon.
\VS{8}Il en fit de même pour toutes ses femmes étrangères, qui offraient des parfums et des sacrifices à leurs dieux.
\VS{9}C'est pourquoi Yahweh fut irrité contre Salomon, parce qu'il avait détourné son cœur de Yahweh, le Dieu d'Israël, qui lui était apparu deux fois.
\VS{10}Il lui avait donné cet ordre de ne point aller après d'autres dieux ; mais il ne garda point ce que Yahweh lui avait ordonné.
\VS{11}Et Yahweh dit à Salomon : Puisque tu as agi de la sorte, et que tu n'as pas observé l’alliance et les ordonnances que je t'avais prescrites, je déchirerai le royaume afin qu'il ne soit plus à toi et je le donnerai à ton serviteur.
\VS{12}Toutefois je ne le ferai point en ton temps, pour l’amour de David, ton père. Ce sera d'entre les mains de ton fils que je déchirerai le royaume.
\VS{13}Néanmoins je ne déchirerai pas tout le royaume, j'en donnerai une tribu à ton fils, pour l'amour de David, mon serviteur, et pour l'amour de Jérusalem, que j'ai choisie.
\TextTitle{Dieu suscite des ennemis à Salomon}
\VS{14}Yahweh donc suscita un ennemi à Salomon, savoir Hadad, l’Edomite, qui était de la race royale d'Edom.
\VS{15}Car il était arrivé qu'au temps que David était en Edom, Joab, chef de l'armée, étant monté pour ensevelir les morts, tua tous les mâles qui étaient en Edom ;
\VS{16}Joab demeura là six mois avec tout Israël, jusqu'à ce qu'il eût exterminé tous les mâles d'Edom.
\VS{17}Ce fut alors qu’Hadad prit la fuite avec des Edomites d'entre les serviteurs de son père, pour se retirer en Egypte. Hadad était alors un jeune garçon.
\VS{18}Une fois partis de Madian, ils allèrent à Paran, prirent avec eux des hommes de Paran, et arrivèrent en Egypte auprès de Pharaon, roi d'Egypte, qui lui donna une maison, pourvut à sa subsistance et lui donna aussi une terre.
\VS{19}Et Hadad trouva grâce aux yeux de Pharaon, de sorte que Pharaon lui donna pour femme la sœur de sa propre femme, la sœur de la reine Thachpenès.
\VS{20}Et la sœur de Thachpenès lui enfanta son fils Guenubath. Thachpenès le sevra dans la maison de Pharaon. Ainsi Guenubath fut dans la maison de Pharaon, parmi les fils de Pharaon.
\VS{21}Lorsque Hadad apprit en Egypte que David s'était endormi avec ses pères, et que Joab, chef de l'armée, était mort, il dit à Pharaon : Laisse-moi partir dans mon pays.
\VS{22}Et Pharaon lui répondit : Que te manque-t-il auprès de moi, pour désirer ainsi t'en aller dans ton pays ? Et il répondit : Je n’ai besoin de rien, mais cependant laisse-moi partir.
\VS{23}Dieu suscita aussi un autre ennemi à Salomon, savoir Rezon, fils d'Eliada, qui s'était enfui de chez son maître Hadadézer, roi de Tsoba,
\VS{24}Il avait rassemblé des gens auprès de lui, et était devenu chef de bandes, lorsque David les fit périr ; et ils s'en allèrent à Damas, s’y établirent et y régnèrent.
\VS{25}Rezon fut ennemi d'Israël au temps de Salomon, en même temps qu’Hadad le mettait à mal, il avait en aversion Israël et il régna sur la Syrie.
\VS{26}Jéroboam aussi, serviteur de Salomon, s'éleva également contre le roi. Il était fils de Nebath, Ephratien, de Tseréda, dont la mère s’appelait Tserua, femme veuve.
\VS{27}Voici à quelle occasion il s'éleva contre le roi. Salomon bâtissait Millo, et fermait la brèche de la cité de David, son père.
\VS{28}Jéroboam était un homme fort et vaillant ; et Salomon, voyant ce jeune homme à l’ouvrage, lui assigna la charge de toute la maison de Joseph.
\VS{29}Dans ce même temps, Jéroboam, étant sorti de Jérusalem, rencontra en chemin le prophète Achija de Silo, revêtu d'un manteau neuf, et ils étaient eux deux tout seuls dans les champs.
\VS{30}Et Achija prit le manteau neuf qu'il avait sur lui et le déchira en douze morceaux,
\VS{31}et il dit à Jéroboam : Prends-en pour toi dix morceaux ! Car ainsi parle Yahweh, le Dieu d'Israël : Voici, je vais arracher le royaume d'entre les mains de Salomon, et je t'en donnerai dix tribus.
\VS{32}Mais il aura une tribu, pour l'amour de David, mon serviteur, et pour l'amour de Jérusalem, qui est la ville que j'ai choisie d'entre toutes les tribus d'Israël.
\VS{33}Parce qu'ils m'ont abandonné, et se sont prosternés devant Astarté, la déesse des Sidoniens, devant Kemosch, dieu de Moab, et devant Milcom, le dieu des fils d’Ammon, et qu'ils n'ont point marché dans mes voies, pour faire ce qui est droit à mes yeux et garder mes statuts, et mes ordonnances, comme l’a fait David, père de Salomon.
\VS{34}Toutefois, je n'ôterai pas de sa main tout le royaume, car pendant toute sa vie je le maintiendrai prince, pour l'amour de David, mon serviteur, que j'ai choisi et qui a observé mes commandements et mes lois.
\VS{35}Mais j'ôterai le royaume d'entre les mains de son fils, je t'en donnerai dix tribus ;
\VS{36}j'en donnerai une tribu à son fils, afin que David, mon serviteur, ait une lampe à toujours devant moi dans Jérusalem, qui est la ville que j'ai choisie pour y mettre mon Nom.
\VS{37}Je te prendrai donc, tu régneras sur tout ce que ton âme désirera, tu seras roi sur Israël.
\VS{38}Et il arrivera que si tu m'obéis en tout ce que je te commanderai, que tu marches dans mes voies, en faisant tout ce qui est droit à mes yeux, en gardant mes statuts et mes commandements, comme l’a fait David, mon serviteur, je serai avec toi, je te bâtirai une maison qui sera stable, comme j'en ai bâti une à David, et je te donnerai Israël.
\VS{39}Ainsi j’humilierai la postérité de David à cause de cela, mais non pas à toujours.
\VS{40}Salomon chercha à faire mourir Jéroboam, mais Jéroboam se leva et s'enfuit en Egypte vers Schischak, roi d'Egypte ; et il demeura en Egypte jusqu'à la mort de Salomon.
\TextTitle{Mort de Salomon\FTNTT{2 Ch. 9:29-31}}
\VS{41}Or, le reste des faits de Salomon, tout ce qu'il a fait et sa sagesse, cela n'est-il pas écrit dans le livre des actes de Salomon ?
\VS{42}Salomon régna à Jérusalem sur tout Israël pendant quarante ans.
\VS{43}Ainsi Salomon s'endormit avec ses pères, il fut enseveli dans la cité de David, son père. Et Roboam, son fils, régna en sa place.
\Chap{12}
\TextTitle{Règne de Roboam\FTNTT{2 Ch. 10:1 ; cp. Ec. 2:18-19}}
\VerseOne{}Roboam se rendit à Sichem, parce que tout Israël était venu à Sichem pour l'établir roi.
\VS{2}Or, Jéroboam, fils de Nebath, était encore en Egypte, où il s'était enfui de devant le roi Salomon, quand il l'apprit, et c’était en Egypte qu’il habitait.
\VS{3}On l'envoya appeler. Ainsi Jéroboam et toute l'assemblée d'Israël vinrent, ils parlèrent à Roboam, en disant :
\VS{4}Ton père a mis sur nous un pesant joug ; mais toi allège maintenant cette rude servitude de ton père et ce pesant joug qu'il a mis sur nous ; et nous te servirons.
\VS{5}Il leur répondit : Allez, et dans trois jours revenez vers moi. Et le peuple s'en alla.
\VS{6}Le roi Roboam consulta les vieillards qui avaient été auprès de Salomon, son père, pendant sa vie et leur dit : Que me conseillez-vous de répondre à ce peuple ?
\VS{7}Et ils lui répondirent, en disant : Si aujourd'hui tu rends service à ce peuple et que tu leur cèdes, et si tu leur réponds avec des paroles bienveillantes, ils seront tes serviteurs à toujours.
\VS{8}Mais Roboam laissa le conseil que les vieillards lui avaient donné et consulta les jeunes gens qui avaient grandi avec lui et qui se tenaient près de lui.
\VS{9}Il leur dit : Que me conseillez-vous de répondre à ce peuple qui m'a parlé, en disant : Allège le joug que ton père a mis sur nous ?
\VS{10}Alors les jeunes gens qui avaient grandi avec lui, lui dirent : Tu parleras ainsi à ce peuple qui t'est venu dire : Ton père a mis sur nous un pesant joug, mais toi allège-le-nous ! Tu leur parleras ainsi : Mon petit doigt est plus gros que les reins de mon père.
\VS{11}Or, mon père a mis sur vous un pesant joug, mais moi je rendrai votre joug encore plus pesant ; mon père vous a châtiés avec des fouets, mais moi je vous châtierai avec des scorpions.
\VS{12}Or, trois jours après, Jéroboam avec tout le peuple vint vers Roboam, selon que le roi leur avait dit : Retournez vers moi dans trois jours.
\VS{13}Mais le roi répondit durement au peuple, laissant le conseil que les anciens lui avaient donné.
\VS{14}Il leur parla selon le conseil des jeunes gens, en leur disant : Mon père a mis sur vous un pesant joug, mais moi, je rendrai votre joug plus pesant encore ; mon père vous a châtiés avec des fouets, mais moi, je vous châtierai avec des scorpions.
\VS{15}Le roi donc n'écouta point le peuple ; car cela était ainsi conduit par Yahweh, en vue d’accomplir la parole qu'il avait prononcée par le ministère d'Achija de Silo, à Jéroboam, fils de Nebath.
\TextTitle{Schisme du royaume ; Jéroboam devient roi d’Israël\FTNTT{2 Ch. 10:12-19 ; 11:1-4}}
\VS{16}Et quand tout Israël vit que le roi ne les avait point écoutés, le peuple fit cette réponse au roi, en disant : Quelle part avons-nous avec David ? Nous n'avons point de propriété avec le fils d'Isaï ! A tes tentes, Israël ! Et toi David, pourvois maintenant à ta maison ! Ainsi Israël s'en alla dans ses tentes.
\VS{17}Les fils d'Israël qui habitaient dans les villes de Juda furent les seuls sur qui Roboam régna.
\VS{18}Or, le roi Roboam envoya Adoram, qui était préposé aux impôts, mais tout Israël le lapida, et il mourut. Alors le roi Roboam se hâta de monter sur un char pour s'enfuir à Jérusalem.
\VS{19}C’est ainsi qu’Israël s’est détaché de la maison de David jusqu'à ce jour.
\VS{20}Tout Israël apprit que Jéroboam était de retour, ils l'envoyèrent appeler dans l'assemblée, et l'établirent roi sur tout Israël. La tribu de Juda fut la seule qui suivit la maison de David.
\VS{21}Roboam arriva à Jérusalem, il rassembla toute la maison de Juda et la tribu de Benjamin, savoir cent quatre-vingt mille hommes d’élite choisis et disposés à faire la guerre, pour combattre contre la maison d'Israël, et ramener la domination à Roboam, fils de Salomon.
\VS{22}Mais la parole de Dieu fut ainsi adressée à Schemaeja, homme de Dieu, disant :
\VS{23}Parle à Roboam, fils de Salomon, roi de Juda, et à toute la maison de Juda, et de Benjamin, et au reste du peuple, en disant :
\VS{24}Ainsi parle Yahweh : Vous ne monterez point et vous ne combattrez point contre vos frères, les fils d'Israël ! Que chacun de vous retourne dans sa maison, car ceci a été fait de par moi. Ils obéirent à la parole de Yahweh, et s'en retournèrent, selon la parole de Yahweh.
\TextTitle{Idolâtrie de Jéroboam}
\VS{25}Or, Jéroboam bâtit Sichem sur la montagne d'Ephraïm, et y demeura, puis il en sortit et bâtit Penuel.
\VS{26}Et Jéroboam dit en son cœur : Maintenant le royaume pourrait bien retourner à la maison de David.
\VS{27}Si ce peuple monte à Jérusalem pour faire des sacrifices dans la maison de Yahweh, le cœur de ce peuple se tournera vers son seigneur, Roboam, roi de Juda, et ils me tueront, et ils retourneront à Roboam, roi de Juda.
\VS{28}Sur quoi le roi ayant pris conseil, fit deux veaux d'or et dit au peuple : Vous êtes longtemps montés à Jérusalem ! Voici ton dieu, ô Israël, qui t'a fait sortir hors du pays d'Egypte.
\VS{29}Il plaça un de ces veaux à Béthel, et il mit l'autre à Dan.
\VS{30}Et cela fut une occasion de péché, car le peuple allait jusqu'à Dan, pour se prosterner devant l'un des veaux.
\VS{31}Il fit aussi des maisons dans les hauts lieux, et établit des sacrificateurs pris parmi tout le peuple, qui n'étaient point des enfants de Lévi.
\VS{32}Jéroboam ordonna aussi une fête solennelle au huitième mois, le quinzième jour du mois, à l'imitation de la fête solennelle qu'on célébrait en Juda, et il offrait des sacrifices sur un autel. Il fit ainsi à Béthel, sacrifiant aux veaux qu'il avait faits, et il établit à Béthel des sacrificateurs des hauts lieux qu'il avait élevés.
\VS{33}Or, le quinzième jour du huitième mois, savoir au mois qu'il avait choisi lui-même, il monta sur l'autel qu'il avait fait à Béthel, et célébra cette fête solennelle pour les enfants d'Israël ; et fit brûler des parfums sur l'autel.
\Chap{13}
\TextTitle{Un homme de Dieu envoyé vers Jéroboam}
\VerseOne{}Et voici, un homme de Dieu vint de Juda à Béthel avec la parole de Yahweh, pendant que Jéroboam se tenait près de l'autel pour brûler des parfums.
\VS{2}Et il cria contre l'autel selon la parole de Yahweh, et dit : Autel ! Autel ! Ainsi parle Yahweh : Voici, un fils naîtra à la maison de David, qui aura pour nom Josias ; il immolera sur toi les sacrificateurs des hauts lieux qui brûlent des parfums sur toi, et on brûlera sur toi des ossements d’hommes !
\VS{3}Le même jour il donna un signe, en disant : C'est ici le signe dont Yahweh a parlé : Voici, l'autel se fendra, et la cendre qui est dessus sera répandue.
\VS{4}Lorsque le roi entendit la parole que l'homme de Dieu avait criée contre l'autel de Béthel, Jéroboam étendit sa main de l'autel, en disant : Saisissez-le ! Et la main qu'il étendit contre lui devint sèche, et il ne put la ramener à lui.
\VS{5}L'autel aussi se fendit, et la cendre qui était sur l'autel fut répandue, selon le signe que l'homme de Dieu avait donné par la parole de Yahweh.
\VS{6}Alors le roi prit la parole et dit à l'homme de Dieu : Implore Yahweh, ton Dieu, et prie pour moi, afin que ma main revienne à moi. L'homme de Dieu implora Yahweh, et la main du roi put revenir à lui et elle fut comme auparavant.
\VS{7}Alors le roi dit à l'homme de Dieu : Entre avec moi dans la maison, tu prendras quelque nourriture et je te donnerai un présent.
\VS{8}Mais l'homme de Dieu répondit au roi : Quand tu me donnerais la moitié de ta maison, je n'entrerais point chez toi, je ne mangerais point de pain, ni ne boirais d'eau en ce lieu.
\VS{9}Car cela m'a été ordonné par Yahweh, qui m'a dit : Tu ne mangeras point de pain, tu ne boiras point d'eau et tu ne t'en retourneras point par le chemin par lequel tu y seras allé.
\VS{10}Il s'en alla donc par un autre chemin, et ne s'en retourna point par le chemin par lequel il était venu à Béthel.
\TextTitle{L’homme de Dieu séduit par un vieux prophète}
\VS{11}Or, il y avait un vieux prophète qui demeurait à Béthel. Ses fils vinrent raconter toutes les choses que l'homme de Dieu avait faites ce jour-là à Béthel, et les paroles qu'il avait dites au roi ; et comme les fils de ce prophète les rapportaient à leur père,
\VS{12}il leur demanda : Par quel chemin s'en est-il allé ? Or, ses fils avaient vu le chemin par lequel l'homme de Dieu qui était venu de Juda s'en était allé.
\VS{13}Et il dit à ses fils : Sellez-moi un âne. Ils lui sellèrent, puis il monta dessus.
\VS{14}Et il s'en alla après l'homme de Dieu, et le trouva assis sous un chêne. Et il lui dit : Es-tu l'homme de Dieu qui est venu de Juda ? Et il lui répondit : C'est moi.
\VS{15}Alors il lui dit : Viens avec moi dans la maison, et tu prendras de quoi te nourrir.
\VS{16}Mais il répondit : Je ne puis retourner avec toi, ni entrer chez toi et je ne mangerai point de pain, ni ne boirai d'eau avec toi en ce lieu ;
\VS{17}Car il m'a été dit de la part de Yahweh : Tu ne mangeras point de pain, tu ne boiras point d'eau, et tu ne t'en retourneras point par le chemin par lequel tu seras allé.
\VS{18}Et il lui dit : Et moi aussi je suis prophète comme toi ; et un ange m'a parlé de la part de Yahweh, en disant : Ramène-le avec toi dans ta maison, qu'il mange du pain, et qu'il boive de l'eau ; mais il lui mentait.
\VS{19}Il s'en retourna donc avec lui, il mangea du pain et but de l'eau dans sa maison.
\VS{20}Et il arriva que comme ils étaient assis à table, la parole de Yahweh fut adressée au prophète qui l'avait ramené.
\VS{21}Et il cria à l'homme de Dieu qui était venu de Juda, en disant : Ainsi a parlé Yahweh : Parce que tu as été rebelle au commandement de Yahweh et que tu n'as point gardé l’ordre que Yahweh, ton Dieu, t'avait donné ;
\VS{22}mais tu t'en es retourné, tu as mangé du pain et bu de l'eau dans le lieu dont Yahweh t'avait dit : N'y mange point de pain et n'y bois point d'eau, ton cadavre n'entrera point au sépulcre de tes pères.
\VS{23}Et quand le prophète qu'il avait ramené eut mangé du pain et bu de l’eau, il sella l’âne pour lui.
\VS{24}L’homme de Dieu s'en alla, et un lion le rencontra dans le chemin, et le tua. Son corps était étendu dans le chemin, l'âne resta auprès du corps, et le lion aussi resta à côté du cadavre.
\VS{25}Et voici des passants virent le corps étendu dans le chemin et le lion qui se tenait auprès du corps ; et ils vinrent le dire dans la ville où le vieux prophète demeurait.
\VS{26}Et le prophète qui avait ramené du chemin l'homme de Dieu, l'ayant appris, dit : C'est l'homme de Dieu qui a été rebelle au commandement de Yahweh, c'est pourquoi Yahweh l'a livré au lion, qui l'aura déchiré après l'avoir tué, selon la parole que Yahweh avait dite à ce prophète.
\VS{27}Et il parla à ses fils, en disant : Sellez-moi un âne. Ils le lui sellèrent,
\VS{28}Et il s'en alla et trouva le corps de l'homme de Dieu étendu dans le chemin, l'âne et le lion qui se tenaient auprès du corps. Le lion n'avait pas dévoré le cadavre, ni déchiré l'âne.
\VS{29}Alors le prophète leva le corps de l'homme de Dieu, le plaça sur l'âne et le ramena ; et ce vieux prophète revint dans la ville pour le pleurer et l'enterrer.
\VS{30}Il mit le corps de ce prophète dans le sépulcre, et il pleura sur lui, en disant : Hélas, mon frère !
\VS{31}Après l’avoir enterré, il parla à ses fils, en disant : Quand je serai mort, enterrez-moi au sépulcre où est enterré l'homme de Dieu, et vous déposerez mes os à côté de ses os.
\VS{32}Car elle s’accomplira, la parole qu’il a criée de la part de Yahweh, contre l'autel qui est à Béthel et contre toutes les maisons des hauts lieux qui sont dans les villes de Samarie.
\TextTitle{Jéroboam continue dans le mal}
\VS{33}Néanmoins, Jéroboam ne se détourna point de sa mauvaise voie, mais il établit de nouveau des sacrificateurs de hauts lieux pris parmi tout le peuple ; quiconque le voulait, Jéroboam le consacrait sacrificateur des hauts lieux.
\VS{34}Cela fut une occasion de péché pour la maison de Jéroboam, qui fut effacée et exterminée de dessus la terre.
\Chap{14}
\TextTitle{Maladie et mort du fils de Jéroboam}
\VerseOne{}En ce temps-là, Abija, fils de Jéroboam, devint malade.
\VS{2}Et Jéroboam dit à sa femme : Lève-toi maintenant et déguise-toi, en sorte qu'on ne reconnaisse point que tu es la femme de Jéroboam, et va à Silo. Voici, là est Achija, le prophète, qui m'a dit que je serais roi sur ce peuple.
\VS{3}Emmène avec toi dix pains, des gâteaux et un vase de miel, et entre chez lui ; il te dira ce qui arrivera à l’enfant.
\VS{4}La femme de Jéroboam fit donc ainsi ; elle se leva et s'en alla à Silo puis elle entra dans la maison d'Achija. Or, Achija ne pouvait plus voir, parce qu’il avait les yeux figés à cause de sa vieillesse.
\VS{5}Et Yahweh dit à Achija : Voici, la femme de Jéroboam, qui vient te consulter concernant l’état de son fils, parce qu'il est malade. Tu lui parleras de telle et de telle manière. Quand elle arrivera, elle se sera déguisée.
\VS{6}Lorsque Achija eut entendu le bruit de ses pas, comme elle franchissait la porte, il dit : Entre, femme de Jéroboam. Pourquoi fais-tu semblant d'être quelqu’un d’autre ? Je suis chargé de t’annoncer des choses dures.
\VS{7}Va, dis à Jéroboam : Ainsi parle Yahweh, le Dieu d'Israël : Parce que je t'ai élevé du milieu du peuple et que je t'ai établi pour chef sur mon peuple d'Israël,
\VS{8}j'ai arraché le royaume de la maison de David et je te l'ai donné ; mais parce que tu n'as point été comme David, mon serviteur, qui a gardé mes commandements et qui a marché après moi de tout son cœur, ne faisant que ce qui est droit à mes yeux.
\VS{9}Tu as fait pire que tous ceux qui ont été devant toi, tu es allé te faire d'autres dieux et des images de fonte, pour m'irriter, et tu m'as rejeté derrière ton dos !
\VS{10}A cause de cela, voici, je vais faire venir le malheur sur la maison de Jéroboam ; je retrancherai ce qui appartient à Jéroboam, ce qu’il détient et ce qu’il néglige en Israël, et je brûlerai la maison de Jéroboam, comme on brûle les ordures, jusqu'à ce qu'il n'en reste plus.
\VS{11}Celui de la maison de Jéroboam qui mourra dans la ville, les chiens le mangeront, et celui qui mourra aux champs, les oiseaux du ciel le mangeront. Car Yahweh a parlé.
\VS{12}Toi donc lève-toi, va dans ta maison. Dès que tes pieds entreront dans la ville, l'enfant mourra.
\VS{13}Tout Israël le pleurera et on l’enterrera ; car lui seul de la famille de Jéroboam entrera au sépulcre, parce que Yahweh, le Dieu d'Israël, a trouvé quelque chose de bon en lui seul dans toute la maison de Jéroboam.
\VS{14}Yahweh s'établira un roi sur Israël qui retranchera la maison de Jéroboam. Ce jour-là, n’est-ce pas déjà ce qui arrive ?
\VS{15}Yahweh frappera Israël, l'agitant comme le roseau est agité dans l'eau ; et il arrachera Israël de ce bon pays qu'il a donné à leurs pères, et les dispersera au-delà du fleuve, parce qu'ils se sont fait des idoles, irritant Yahweh.
\VS{16}Il livrera Israël à cause des péchés que Jéroboam a commis et qu’il a fait commettre à Israël.
\VS{17}Alors la femme de Jéroboam se leva et s'en alla, elle vint à Thirtsa : et comme elle franchit le seuil de la maison, le jeune garçon mourut.
\VS{18}Il fut enseveli et tout Israël le pleura, selon la parole de Yahweh, proférée par son serviteur Achija, le prophète.
\TextTitle{Règne de Nadab sur Israël\FTNTT{cp. 2 Ch. 13:20}}
\VS{19}Quant au reste des faits de Jéroboam, comment il a fait la guerre et comment il a régné, cela est écrit dans le livre des Chroniques des rois d'Israël.
\VS{20}Jéroboam régna vingt-deux ans, puis il s'endormit avec ses pères. Et Nadab, son fils, régna à sa place.
\TextTitle{Juda dans l'apostasie\FTNTT{2 Ch. 12:1}}
\VS{21}Roboam, fils de Salomon, régna en Juda. Il avait quarante et un ans quand il devint roi, et il régna dix-sept ans à Jérusalem, la ville que Yahweh avait choisie d'entre toutes les tribus d'Israël pour y mettre son nom. Sa mère s’appelait Naama, l’Ammonite.
\VS{22}Juda fit ce qui est mal aux yeux de Yahweh ; et par les péchés qu'ils commirent, ils excitèrent sa jalousie plus que leurs pères ne l'avaient jamais fait.
\VS{23}Ils se bâtirent, eux aussi, des hauts lieux avec des statues et des idoles sur toute colline élevée, et sous tout arbre verdoyant.
\VS{24}Il y avait dans le pays des prostitués. Et ils firent selon toutes les abominations des nations que Yahweh avait chassées devant les enfants d'Israël.
\TextTitle{Le roi d’Egypte emporte les trésors de Juda ; mort de Roboam\FTNTT{2 Ch. 12:2-16}}
\VS{25}La cinquième année du roi Roboam, Schischak, roi d'Egypte, monta contre Jérusalem.
\VS{26}Il prit les trésors de la maison de Yahweh et les trésors de la maison royale, et il emporta tout. Il prit aussi tous les boucliers d'or que Salomon avait faits.
\VS{27}Le roi Roboam fit des boucliers d'airain au lieu de ceux-là, et les mit entre les mains des chefs des coureurs, qui gardaient l’entrée de la maison du roi.
\VS{28}Toutes les fois où le roi entrait dans la maison de Yahweh, les coureurs les portaient, et ensuite ils les rapportaient dans la chambre des coureurs.
\VS{29}Le reste des actions de Roboam, et tout ce qu'il a fait, n'est-il pas écrit au livre des Chroniques des rois de Juda ?
\VS{30}Il y eut toujours guerre entre Roboam et Jéroboam.
\VS{31}Roboam s'endormit avec ses pères et fut enseveli avec eux dans la cité de David. Sa mère avait pour nom Naama, l’Ammonite. Et Abijam, son fils, régna à sa place.
\Chap{15}
\TextTitle{Règne d'Abijam (ou Abija) sur Juda\FTNTT{2 Ch. 13:1-2}}
\VerseOne{}La dix-huitième année du roi Jéroboam, fils de Nebath, Abijam commença à régner sur Juda.
\VS{2}Il régna trois ans à Jérusalem. Sa mère s’appelait Maaca et était fille d'Abisalom.
\VS{3}Il marcha dans tous les péchés que son père avait commis avant lui ; son cœur ne fut point intègre envers Yahweh, son Dieu, comme l'avait été le cœur de David, son père.
\VS{4}Mais pour l'amour de David, Yahweh, son Dieu, lui donna une lampe dans Jérusalem, lui suscitant son fils après lui et laissant subsister Jérusalem ;
\VS{5}Parce que David avait fait ce qui est droit devant Yahweh, et que pendant toute sa vie il ne s'était point détourné d’aucun de ses commandements, hormis dans l'affaire d'Urie, le Héthien.
\VS{6}Or, il y eut toujours guerre entre Roboam et Jéroboam, pendant toute la vie de Roboam.
\VS{7}Le reste des actions d'Abijam, et même tout ce qu'il fit, n'est-il pas écrit au livre des Chroniques des rois de Juda ? Il y eut aussi guerre entre Abijam et Jéroboam.
\VS{8}Ainsi Abijam s'endormit avec ses pères, et on l'enterra dans la cité de David. Et Asa, son fils, régna à sa place.
\TextTitle{Règne d’Asa sur Juda\FTNTT{2 Ch. 14:1-5 ; 15:1-19}}
\VS{9}La vingtième année de Jéroboam, roi d'Israël, Asa commença à régner sur Juda.
\VS{10}Il régna quarante et un ans à Jérusalem. Sa mère avait pour nom Maaca, elle était fille d'Abisalom.
\VS{11}Asa fit ce qui est droit devant Yahweh, comme David, son père.
\VS{12}Il ôta du pays les prostitués, et ôta toutes les idoles que ses pères avaient faites.
\VS{13}Et même il ôta la dignité de reine à sa mère Maaca, parce qu'elle avait fait une idole pour Astarté. Asa mit en pièces l’idole qu'elle avait faite, et la brûla au torrent de Cédron.
\VS{14}Mais les hauts lieux ne furent point ôtés. Néanmoins, le cœur d'Asa fut intègre envers Yahweh pendant toute sa vie.
\TextTitle{Guerre entre Juda et Israël ; Asa s’allie avec la Syrie\FTNTT{1 Ch. 14:6-15 ; 16:1-10}}
\VS{15}Il remit dans la maison de Yahweh les choses qui avaient été consacrées par son père et par lui-même, de l'argent, de l'or et les ustensiles.
\VS{16}Or, il y eut guerre entre Asa et Baescha, roi d'Israël, pendant toute leur vie.
\VS{17}Baescha, roi d'Israël, monta contre Juda, et bâtit Rama, pour empêcher quiconque de sortir et entrer vers Asa, roi de Juda.
\VS{18}Asa prit tout l'argent et l'or qui était resté dans les trésors de Yahweh et dans les trésors de la maison royale, et les donna à ses serviteurs ; le roi Asa les envoya vers Ben-Hadad, fils de Thabrimmon, fils de Hezjon, roi de Syrie, qui demeurait à Damas, pour lui dire :
\VS{19}Qu’il y ait alliance entre moi et toi, comme entre mon père et le tien. Voici, je t'envoie un présent en argent et en or. Va, romps l'alliance que tu as avec Baescha, roi d'Israël, afin qu'il se retire de moi.
\VS{20}Et Ben-Hadad écouta le roi Asa ; il envoya les chefs de son armée contre les villes d'Israël, et il battit Ijjon, Dan, Abel-Beth-Maaca, tout Kinneroth, et tout le pays de Nephthali.
\VS{21}Lorsque Baescha l’apprit, il cessa de bâtir Rama et demeura à Thirtsa.
\VS{22}Alors le roi Asa fit publier par tout Juda que tous, sans en excepter aucun, eussent à emporter les pierres et le bois de Rama, que Baescha faisait bâtir, et le roi Asa s’en servit pour bâtir Guéba de Benjamin, et Mitspa.
\TextTitle{Mort d’Asa ; Josaphat règne sur Juda\FTNTT{1 Ch. 16:11-17:1}}
\VS{23}Le reste de toutes les actions d'Asa, tous ses exploits, tout ce qu'il fit, et les villes qu'il a bâties, cela n'est-il pas écrit au livre des Chroniques des rois de Juda ? Au reste, il fut malade de ses pieds au temps de sa vieillesse.
\VS{24}Et Asa s'endormit avec ses pères, avec lesquels il fut enseveli en la cité de David, son père. Et, son fils, Josaphat, régna à sa place.
\TextTitle{Baescha tue Nadab et devient roi d'Israël}
\VS{25}Or, Nadab, fils de Jéroboam, régna sur Israël la seconde année d'Asa, roi de Juda, et il régna deux ans sur Israël.
\VS{26}Il fit ce qui est mal aux yeux de Yahweh ; et il marcha dans la voie de son père, se livrant aux péchés que son père avait fait commettre à Israël.
\VS{27}Et Baescha, fils d'Achija, de la maison d'Issacar, fit une conspiration contre lui. Il le tua devant Guibbethon, qui était aux Philistins, lorsque Nadab et tout Israël assiégeaient Guibbethon.
\VS{28}Baescha le fit donc mourir la troisième année d'Asa, roi de Juda, et il régna à sa place.
\VS{29}Une fois proclamé roi, il frappa toute la maison de Jéroboam et ne laissa échapper aucune âme vivante, il détruisit tout ce qui respirait, selon la parole de Yahweh qu'il avait proférée par son serviteur Achija, Silonite,
\VS{30}A cause des péchés que Jéroboam avait commis et fait commettre à Israël, irritant ainsi Yahweh, le Dieu d'Israël.
\VS{31}Le reste des faits de Nadab, et même tout ce qu'il a fait, n'est-il pas écrit au livre des Chroniques des rois d'Israël ?
\VS{32}Or, il y eut guerre entre Asa et Baescha, roi d'Israël, pendant toute leur vie.
\VS{33}La troisième année d'Asa, roi de Juda, Baescha, fils d'Achija, commença à régner sur tout Israël à Thirtsa, il régna vingt-quatre ans.
\VS{34}Et il fit ce qui est mal aux yeux de Yahweh, et marcha dans la voie de Jéroboam, en se livrant aux péchés que Jéroboam avait fait commettre à Israël.
\Chap{16}
\TextTitle{Yahweh avertit Baescha avant sa mort}
\VerseOne{}Alors la parole de Yahweh fut adressée à Jéhu, fils de Hanani, contre Baescha, en ces mots :
\VS{2}Je t'ai élevé de la poussière et je t'ai établi chef de mon peuple d'Israël ; malgré cela tu as suivi la voie de Jéroboam et fait pécher mon peuple d'Israël, pour m'irriter par leurs péchés.
\VS{3}Voici, je m'en vais entièrement consumer Baescha et sa maison, et je rendrai ta maison semblable à la maison de Jéroboam, fils de Nebath.
\VS{4}Celui de la maison de Baescha qui mourra dans la ville, les chiens le mangeront, et celui des siens qui mourra aux champs, les oiseaux du ciel le mangeront.
\VS{5}Le reste des faits de Baescha, ce qu'il a fait et ses exploits, n'est-il pas écrit au livre des Chroniques des rois d'Israël ?
\VS{6}Ainsi Baescha s'endormit avec ses pères et fut enseveli à Thirtsa. Ela, son fils, régna à sa place.
\VS{7}La parole de Yahweh fut aussi adressée par le moyen de Jéhu, fils d'Hanani, le prophète, contre Baescha et contre sa maison, tant à cause de tout le mal qu'il avait fait devant Yahweh, en l'irritant par l'œuvre de ses mains et en devenant comme la maison de Jéroboam, que parce qu'il l'avait détruite.
\TextTitle{Ela puis Zimri règnent sur Israël}
\VS{8}La vingt-sixième année d'Asa, roi de Juda, Ela, fils de Baescha, commença à régner sur Israël et il régna deux ans à Thirtsa.
\VS{9}Son serviteur, Zimri, capitaine de la moitié des chars, fit une conspiration contre Ela, lorsqu'il était à Thirtsa, buvant et s'enivrant dans la maison d'Artsa, chef de la maison du roi à Thirtsa.
\VS{10}Alors, Zimri vint, le frappa et le tua, la vingt-septième année d'Asa, roi de Juda, et il régna à sa place.
\VS{11}Dès qu’il fut roi et qu'il fut assis sur son trône, il frappa toute la maison de Baescha, il n'en laissa échapper personne qui lui appartint, ni parent, ni ami.
\VS{12}Ainsi Zimri extermina toute la maison de Baescha, selon la parole que Yahweh avait proférée contre Baescha, par Jéhu, le prophète,
\VS{13}A cause de tous les péchés de Baescha, et des péchés d'Ela, son fils, qu’ils avaient commis et qu’ils avaient fait commettre à Israël, irritant Yahweh, le Dieu d'Israël, par leurs idoles.
\VS{14}Le reste des faits d'Ela, et même tout ce qu'il a fait, n'est-il pas écrit au livre des Chroniques des rois d'Israël ?
\VS{15}La vingt-septième année d'Asa, roi de Juda, Zimri régna sept jours à Thirtsa. Or, le peuple était campé contre Guibbethon qui appartenait aux Philistins.
\VS{16}Et le peuple qui était campé là entendit que l'on disait : Zimri a fait une conspiration, et il a même tué le roi ! En ce même jour, tout Israël établit dans le camp pour roi d’Israël Omri, chef de l'armée d'Israël.
\VS{17}Omri et tout Israël avec lui partirent de Guibbethon, et assiégèrent Thirtsa.
\VS{18}Mais dès que Zimri vit que la ville était prise, il entra au palais de la maison royale et brûla sur lui la maison royale, il mourut ainsi,
\VS{19}A cause des péchés qu’il avait commis, faisant ce qui est mal aux yeux de Yahweh, en suivant la voie de Jéroboam et le péché qu'il avait fait commettre à Israël.
\VS{20}Le reste des actions de Zimri et la conspiration qu'il forma, cela n’est-il pas écrit dans le livre des Chroniques des rois d'Israël ?
\TextTitle{Omri règne sur Israël}
\VS{21}Alors le peuple d'Israël se divisa en deux partis : la moitié du peuple voulait faire roi Thibni, fils de Guinath ; et l'autre moitié suivait Omri.
\VS{22}Mais le peuple qui suivait Omri, fut plus fort que le peuple qui suivait Thibni, fils de Guinath. Thibni mourut et Omri régna.
\VS{23}La trente et unième année d'Asa, roi de Juda, Omri commença à régner sur Israël et il régna douze ans après avoir régné six ans à Thirtsa.
\VS{24}Puis il acheta de Schémer la montagne de Samarie, deux talents d'argent ; il bâtit une ville sur cette montagne et nomma la ville qu'il bâtit, du nom de Schémer, seigneur de la montagne.
\VS{25}Omri fit ce qui est mal aux yeux de Yahweh ; il agit même plus mal que tous ceux qui avaient été avant lui.
\VS{26}Il marcha dans la voie de Jéroboam, fils de Nebath, et se livra aux péchés que Jéroboam avait fait commettre à Israël, irritant Yahweh, le Dieu d'Israël, par leurs idoles.
\VS{27}Le reste des actions d’Omri, tout ce qu'il a fait et ses exploits, cela n’est-il pas écrit au livre des Chroniques des rois d'Israël ?
\VS{28}Ainsi Omri s'endormit avec ses pères et fut enseveli à Samarie. Achab, son fils, régna à sa place.
\TextTitle{Achab règne sur Israël et épouse Jézabel}
\VS{29}Achab, fils d’Omri, régna sur Israël la trente-huitième année d'Asa, roi de Juda. Et Achab, fils d’Omri, régna sur Israël à Samarie vingt-deux ans.
\VS{30}Et Achab, fils d’Omri, fit ce qui est mal aux yeux de Yahweh, plus que tous ceux qui avaient été avant lui.
\VS{31}Et il arriva que, comme si ce lui eût été peu de chose de marcher dans les péchés de Jéroboam, fils de Nebath, il prit pour femme Jézabel, fille d'Ethbaal, roi des Sidoniens, puis il alla servir Baal et se prosterna devant lui.
\VS{32}Il dressa un autel à Baal, dans la maison de Baal, qu'il bâtit à Samarie.
\VS{33}Et Achab fit une idole d’Astarté. De sorte qu'Achab fit plus encore que tous les rois d'Israël qui avaient été avant lui, pour irriter Yahweh, le Dieu d'Israël.
\VS{34}En son temps, Hiel de Béthel bâtit Jéricho ; il en jeta les fondements au prix d’Abiram, son premier-né, et posa ses portes sur Segub, son plus jeune fils, selon la parole que Yahweh avait proférée par le moyen de Josué, fils de Nun.
\Chap{17}
\TextTitle{Elie annonce trois ans de sécheresse\FTNTT{1 R. 17-2 R. 1}}
\VerseOne{}Alors Elie, Thischbite, l’un des habitants de Galaad, dit à Achab : Yahweh, le Dieu d'Israël, en la présence duquel je me tiens, est vivant ! Il n'y aura ces années-ci ni rosée ni pluie, sinon à ma parole.
\TextTitle{Elie au torrent de Kerith}
\VS{2}Puis la parole de Yahweh fut adressée à Elie, en disant :
\VS{3}Va-t'en d'ici et tourne-toi vers l'orient ; cache-toi près du torrent de Kerith, qui est en face du Jourdain.
\VS{4}Tu boiras de l’eau du torrent, et j'ai commandé aux corbeaux de t'y nourrir.
\VS{5}Il partit donc et fit selon la parole de Yahweh, il s'en alla et demeura au torrent de Kerith, vis-à-vis du Jourdain.
\VS{6}Les corbeaux lui apportaient du pain et de la viande le matin, et du pain et de la viande le soir, et il buvait de l’eau du torrent.
\VS{7}Mais il arriva qu'au bout d’un certain temps le torrent tarit, parce qu'il n'y avait point eu de pluie dans le pays.
\TextTitle{Elie chez la veuve de Sarepta}
\VS{8}Alors la parole de Yahweh lui fut adressée, en ces mots :
\VS{9}Lève-toi, va à Sarepta, qui appartient à Sidon, et demeure-là. Voici, j'ai commandé là à une femme veuve de t'y nourrir.
\VS{10}Il se leva donc et s'en alla à Sarepta. Et comme il fut arrivé à l’entrée de la ville, voici, une femme veuve était là, qui ramassait du bois. Et il l'appela et lui dit : Apporte-moi, je te prie, un peu d'eau dans un vase et que je boive.
\VS{11}Elle alla en chercher. Il l’appela de nouveau et dit : Apporte-moi, je te prie, un morceau de pain de ta main.
\VS{12}Mais elle répondit : Yahweh, ton Dieu, est vivant ! Je n'ai rien de cuit, je n'ai qu’une poignée de farine dans un pot et un peu d'huile dans une cruche. Et voici, j'amasse deux morceaux de bois, puis je rentrerai, je l'apprêterai pour moi et pour mon fils, nous le mangerons, après quoi nous mourrons.
\VS{13}Et Elie lui dit : Ne crains point, va, fais comme tu dis. Seulement, fais-moi d’abord avec cela un petit gâteau et tu me l’apporteras, tu en feras ensuite pour toi et pour ton fils.
\VS{14}Car ainsi parle Yahweh, le Dieu d'Israël : La farine qui est dans le pot ne finira point et l'huile qui est dans la cruche ne diminuera point, jusqu'à ce que Yahweh donne de la pluie sur la terre.
\VS{15}Elle s'en alla donc, et fit selon la parole d'Elie. Et elle eut à manger, elle et sa famille, ainsi qu’Elie pendant plusieurs jours.
\VS{16}La farine du pot ne finit point, et l'huile de la cruche ne diminua point, selon la parole que Yahweh avait prononcée par le moyen d'Elie.
\TextTitle{Résurrection du fils de la veuve de Sarepta}
\VS{17}Après ces choses, il arriva que le fils de la femme, maîtresse de la maison, devint malade ; et la maladie fut si forte, qu'il expira.
\VS{18}Et elle dit à Elie : Qu'y a-t-il entre moi et toi, homme de Dieu ? Es-tu venu chez moi pour rappeler le souvenir de mon iniquité, et pour faire mourir mon fils ?
\VS{19}Et il lui dit : Donne-moi ton fils. Et il le prit du sein de cette femme, le porta dans la chambre haute où il demeurait, et le coucha sur son lit.
\VS{20}Puis il cria à Yahweh, et dit : Yahweh, mon Dieu ! Affligeras-tu cette veuve au point de faire mourir son fils, elle qui a m’a reçu comme un hôte ?
\VS{21}Et il s'étendit sur l'enfant par trois fois, et cria à Yahweh, en disant : Yahweh, mon Dieu ! Je te prie que l'âme de cet enfant revienne au-dedans de lui.
\VS{22}Et Yahweh écouta la voix d'Elie, l'âme de l'enfant revint au-dedans de lui, et il fut rendu à la vie.
\VS{23}Elie prit l'enfant, le descendit de la chambre haute dans la maison et le donna à sa mère, en lui disant : Regarde, ton fils est vivant.
\VS{24}Et la femme dit à Elie : Je reconnais maintenant, que tu es un homme de Dieu et que la parole de Yahweh, qui est dans ta bouche, est vérité.
\Chap{18}
\TextTitle{Elie à la rencontre d’Abdias puis d’Achab}
\VerseOne{}Et il arriva, après bien des jours, que la parole de Yahweh fut adressée à Elie, dans la troisième année, en disant : Va, montre-toi à Achab et je ferai tomber de la pluie sur la terre.
\VS{2}Et Elie s'en alla pour se présenter devant Achab. Il y avait alors une grande famine en Samarie.
\VS{3}Achab avait appelé Abdias, chef de sa maison ; or, Abdias craignait beaucoup Yahweh ;
\VS{4}quand Jézabel exterminait les prophètes de Yahweh, Abdias prit cent prophètes et les cacha, cinquante dans une caverne et cinquante dans une autre, et il les y nourrit de pain et d'eau.
\VS{5}Achab dit alors à Abdias : Va par le pays vers toutes les sources d'eaux et vers tous les torrents ; peut-être que nous trouverons de l'herbe, nous garderons ainsi en vie les chevaux et les mulets, et nous n’aurons pas besoin d’abattre du bétail.
\VS{6}Ils se partagèrent donc entre eux le pays pour le parcourir ; Achab allait seul par un chemin et Abdias allait seul par un autre chemin.
\VS{7}Comme Abdias était en chemin, voici, Elie le rencontra. Abdias reconnut Elie, il tomba sur son visage et lui dit : N'es-tu pas mon seigneur Elie ?
\VS{8}Il lui répondit : C'est moi ; va et dis à ton seigneur : Voici Elie !
\VS{9}Et Abdias dit : Quel péché ai-je commis, pour que tu livres ton serviteur entre les mains d'Achab pour me faire mourir ?
\VS{10}Yahweh, ton Dieu, est vivant ! Il n'y a ni nation, ni royaume, où mon seigneur n'ait envoyé pour te chercher ; et quand on répondait que tu n'y étais pas, il faisait jurer aux rois et au peuple que l'on ne t’avait pas trouvé.
\VS{11}Et maintenant tu dis : Va, dis à ton seigneur, voici Elie !
\VS{12}Puis, lorsque je t’aurai quitté, l'Esprit de Yahweh te transportera je ne sais où et j’irai informer Achab qui ne te trouvera pas et qui me tuera. Or, ton serviteur craint Yahweh dès sa jeunesse.
\VS{13}N'a-t-on point dit à mon seigneur ce que je fis quand Jézabel tuait les prophètes de Yahweh, comment j'en cachai cent, cinquante dans une caverne et cinquante dans une autre et les y ai nourris de pain et d'eau ?
\VS{14}Et maintenant tu dis : Va, dis à ton seigneur : Voici Elie ! Il me tuera !
\VS{15}Mais Elie lui répondit : Yahweh des armées, devant lequel je me tiens, est vivant ! Aujourd'hui, je me montrerai à Achab.
\VS{16}Abdias étant allé à la rencontre d'Achab, l’informa de la chose ; puis Achab alla au-devant d'Elie.
\VS{17}Et aussitôt qu'Achab eut vu Elie, il lui dit : Est-ce toi qui jettes le trouble en Israël ?
\VS{18}Et Elie lui répondit : Je n'ai point troublé Israël ; c'est toi et la maison de ton père, puisque vous avez abandonné les commandements de Yahweh et que vous êtes allés après les Baals.
\VS{19}Fais maintenant se rassembler tout Israël auprès de moi, sur le mont Carmel, les quatre cent cinquante prophètes de Baal et les quatre cents prophètes d’Astarté qui mangent à la table de Jézabel.
\TextTitle{Confrontation entre Elie et les prophètes de Baal sur le mont Carmel}
\VS{20}Ainsi Achab envoya des messagers vers tous les fils d'Israël et, il rassembla les prophètes sur le mont Carmel.
\VS{21}Alors Elie s'approcha de tout le peuple et dit : Jusqu'à quand clocherez-vous des deux côtés ? Si Yahweh est Dieu, suivez-le ; mais si Baal est dieu, suivez-le. Et le peuple ne lui répondit pas un seul mot.
\VS{22}Alors Elie dit au peuple : Je suis demeuré seul prophète de Yahweh ; et voici quatre cent cinquante prophètes de Baal.
\VS{23}Que l’on nous donne deux veaux, qu'ils en choisissent l'un pour eux, qu'ils le coupent en pièces et qu'ils le mettent sur du bois ; mais qu'ils n'y mettent point de feu ; et je préparerai l'autre veau, je le mettrai sur du bois, sans y mettre le feu.
\VS{24}Puis invoquez le nom de vos dieux, et moi j'invoquerai le nom de Yahweh ; que le dieu qui répondra par le feu, soit reconnu pour être Dieu. Et tout le peuple répondit et dit : C'est bien !
\VS{25}Et Elie dit aux prophètes de Baal : Choisissez un veau et préparez-le les premiers, car vous êtes en plus grand nombre et invoquez le nom de vos dieux ; mais n'y mettez point de feu.
\VS{26}Ils prirent donc un veau qu'on leur donna, ils l'apprêtèrent et ils invoquèrent le nom de Baal depuis le matin jusqu'à midi, en disant : Baal exauce-nous ! Mais il n'y avait ni voix ni réponse et ils sautaient devant l'autel qu'ils avaient fait.
\VS{27}A midi, Elie se moqua d'eux et dit : Criez à haute voix, puisqu’il est dieu ; mais il pense à quelque chose, ou il est occupé, ou il est en voyage ; peut-être qu'il dort et il se réveillera.
\VS{28}Ils criaient donc à haute voix ; ils se faisaient des incisions avec des couteaux et des lances, selon leur coutume, en sorte que le sang coulait sur eux.
\VS{29}Lorsque midi fut passé et qu'ils eurent fait les prophètes jusqu'au temps où l’on offre l'oblation, sans qu'il y eût ni voix, ni réponse, ni signe d’attention.
\VS{30}Elie dit alors à tout le peuple : Approchez-vous de moi ! Et tout le peuple s'approcha de lui et il répara l'autel de Yahweh, qui avait été renversé.
\VS{31}Puis Elie prit douze pierres, selon le nombre des tribus des fils de Jacob, auquel la parole de Yahweh avait été adressée, en disant : Israël sera ton nom.
\VS{32}Et il rebâtit de ces pierres l'autel au nom de Yahweh. Puis il fit un fossé de la capacité de deux mesures de semence autour de l'autel.
\VS{33}Il rangea le bois, il coupa le veau en pièces, et il le plaça sur le bois.
\VS{34}Puis il dit : Remplissez quatre cruches d'eau, puis versez-les sur l'holocauste et sur le bois. Puis il dit : Faites-le encore une seconde fois. Et ils le firent une seconde fois. Il dit : Faites-le une troisième fois. Et ils le firent pour la troisième fois ;
\VS{35}de sorte que les eaux allaient à l'entour de l'autel ; et il remplit aussi d’eau le fossé.
\VS{36}Et au moment de la présentation de l’offrande, Elie, le prophète, s'approcha et dit : Ô Yahweh ! Dieu d'Abraham, d'Isaac et d'Israël ! Que l’on sache aujourd'hui que tu es Dieu en Israël et que je suis ton serviteur ; et que j'ai fait toutes ces choses par ta parole !
\VS{37}Réponds-moi, Ô Yahweh ! Réponds-moi, afin que ce peuple connaisse que c’est toi, Yahweh, qui es Dieu et que c'est toi qui ramènes leur cœur.
\VS{38}Alors le feu de Yahweh tomba et consuma l'holocauste, le bois, les pierres et la terre, et il absorba toute l'eau qui était dans le fossé.
\VS{39}Quand tout le peuple vit cela, ils tombèrent sur leur visage et dirent : C'est Yahweh qui est Dieu ! C'est Yahweh qui est Dieu !
\VS{40}Et Elie leur dit : Saisissez les prophètes de Baal et qu'il n'en échappe aucun ! Ils les saisirent. Elie les fit descendre au torrent de Kison, où il les fit égorger là.
\TextTitle{Retour de la pluie selon la parole d’Elie\FTNTT{Ja. 5:17-18}}
\VS{41}Puis Elie dit à Achab : Monte, mange et bois ; car il se fait un bruit qui annonce la pluie.
\VS{42}Ainsi Achab monta pour manger et pour boire tandis qu’Elie monta au sommet du Carmel ; et, se penchant contre terre, il mit son visage entre ses genoux ;
\VS{43}Et il dit à son serviteur : Monte maintenant et regarde vers la mer. Le serviteur monta, il regarda et dit : Il n'y a rien. Elie dit par sept fois : Retournes-y.
\VS{44}A la septième fois, il dit : Voici un petit nuage qui s’élève de la mer et qui est comme la paume de la main d'un homme, laquelle monte de la mer. Elie dit : Monte et dis à Achab : Attelle ton char et descends de peur que la pluie ne t’arrête.
\VS{45}Ici et là, les cieux s'obscurcirent de nuages accompagnés de vent et il y eut une forte pluie. Achab monta sur son char et partit pour Jizreel.
\VS{46}Et la main de Yahweh fut sur Elie, qui se ceignit les reins et courut devant Achab, jusqu'à l'entrée de Jizreel.
\Chap{19}
\TextTitle{Fuite d’Elie devant les menaces de Jézabel}
\VerseOne{}Achab rapporta à Jézabel tout ce qu'Elie avait fait, et comment il avait tué par l'épée tous les prophètes.
\VS{2}Et Jézabel envoya un messager vers Elie, pour lui dire : Que les dieux me traitent dans toute leur rigueur, si demain, à cette heure-ci, je ne fais de ta vie ce que tu as fait de la vie de chacun d'eux !
\VS{3}Elie, voyant cela, se leva et s'en alla pour sauver sa vie. Il arriva à Beer-Schéba, qui appartient à Juda ; et il laissa là son serviteur.
\TextTitle{L’ange de Yahweh fortifie Elie}
\VS{4}Mais lui s'en alla dans le désert où, après une journée de marche, il s'assit sous un genêt et demanda la mort, en disant : C'en est assez, Ô Yahweh ! Prends mon âme, car je ne suis pas meilleur que mes pères.
\VS{5}Puis il se coucha et s'endormit sous un genêt. Voici un ange le toucha et lui dit : Lève-toi, mange.
\VS{6}Et il regarda, et voici à son chevet, un gâteau cuit sur des pierres chauffées et une cruche d'eau. Il mangea et but, puis se recoucha.
\VS{7}Et l'ange de Yahweh vint une seconde fois, le toucha et lui dit : Lève-toi, mange, car le chemin est trop long pour toi.
\TextTitle{Elie à Horeb, visitation et instructions de Yahweh}
\VS{8}Il se leva donc, mangea et but ; puis avec la force que lui donna cette nourriture, il marcha quarante jours et quarante nuits jusqu'à Horeb, la montagne de Dieu.
\VS{9}Et là, il entra dans une caverne et y passa la nuit. Et voici, la parole de Yahweh lui fut adressée en ces mots : Que fais-tu ici, Elie ?
\VS{10}Et il répondit : J'ai déployé mon zèle pour Yahweh, le Dieu des armées, parce que les enfants d'Israël ont abandonné ton alliance, ils ont renversé tes autels, ils ont tué tes prophètes par l'épée ; je suis resté, moi seul et ils me cherchent pour m'ôter la vie.
\VS{11}Yahweh lui dit : Sors et tiens-toi sur la montagne devant Yahweh. Et voici, Yahweh passa. Et devant Yahweh, il y eut un grand vent impétueux qui déchirait les montagnes et brisait les rochers, mais Yahweh n'était point dans ce vent. Après le vent, ce fut un tremblement de terre ; mais Yahweh n'était point dans ce tremblement de terre.
\VS{12}Après le tremblement de terre, un feu ; mais Yahweh n'était pas dans le feu. Et après le feu vint un murmure doux et léger.
\VS{13}Quand Elie l'entendit, il s’enveloppa le visage de son manteau, il sortit et se tint à l'entrée de la caverne. Et voici, une voix lui fit entendre ces paroles : Que fais-tu ici Elie ?
\VS{14}Et il répondit : J'ai déployé mon zèle pour Yahweh, le Dieu des armées, parce que les enfants d'Israël ont abandonné ton alliance, ils ont renversé tes autels, ils ont tué par l'épée tes prophètes ; je suis resté moi seul, et ils cherchent ma vie pour me l'ôter.
\VS{15}Yahweh lui dit : Va, retourne-t'en par ton chemin vers le désert de Damas ; et quand tu seras arrivé, tu oindras Hazaël pour roi de Syrie.
\VS{16}Tu oindras aussi Jéhu, fils de Nimschi, pour roi d’Israël ; et tu oindras Elisée, fils de Schaphath, d'Abel-Mehola, pour prophète à ta place.
\VS{17}Et il arrivera que quiconque échappera de l'épée de Hazaël, Jéhu le fera mourir ; et quiconque échappera de l'épée de Jéhu, Elisée le fera mourir.
\VS{18}Mais je me suis réservé sept mille hommes de reste en Israël, tous ceux qui n'ont point fléchi les genoux devant Baal, et dont la bouche ne l'a point baisé.
\TextTitle{Elisée devient disciple d’Elie}
\VS{19}Elie partit donc de là, et il trouva Elisée, fils de Schaphath, qui labourait. Il y avait douze paires de bœufs devant soi et il était avec la douzième. Quand Elie passa près de lui, il jeta sur lui son manteau.
\VS{20}Elisée laissa ses bœufs et courut après Elie, en disant : Je t’en prie, laisse-moi embrasser mon père et ma mère, et je te suivrai. Elie lui répondit : Va, et reviens ; car pense à ce que je t'ai fait.
\VS{21}Après s’être éloigné d’Elie, il revint prendre une paire de bœufs qu’il offrit en sacrifice ; et avec l'attelage des bœufs, il en fit bouillir la chair, et la donna au peuple ; ils mangèrent ; puis il se leva et suivit Elie. Dès lors, il fut à son service.
\Chap{20}
\TextTitle{Achab monte contre Ben-Hadad}
\VerseOne{}Alors Ben-Hadad, roi de Syrie rassembla toute son armée ; il avait avec lui trente-deux rois, des chevaux et des chars. Puis il monta, assiégea Samarie et il lui fit la guerre.
\VS{2}Il envoya des messagers à Achab, roi d'Israël, dans la ville ;
\VS{3}Et il lui fit dire : Ainsi parle Ben-Hadad : Ton argent et ton or sont à moi, tes femmes aussi et tes beaux enfants sont à moi.
\VS{4}Et le roi d'Israël répondit, et dit : Mon seigneur, je suis à toi, comme tu le dis, avec tout ce que j'ai.
\VS{5}Ensuite les messagers retournèrent, et dirent : Ainsi parle Ben-Hadad : Puisque je t'ai envoyé dire : Donne-moi ton argent et ton or, ta femme et tes enfants ;
\VS{6}A la même heure demain, j'enverrai chez toi mes serviteurs, ils fouilleront ta maison et les maisons de tes serviteurs, et se saisiront de tout ce que tu as de précieux, et ils l'emporteront.
\VS{7}Alors le roi d'Israël appela tous les anciens du pays, et il dit : Sachez et considérez, je vous prie, combien cet homme nous veut du mal ; car il m’a envoyé demander mes femmes, mes enfants, mon argent et mon or, et je ne lui avais rien refusé.
\VS{8}Et tous les anciens et tout le peuple lui dirent : Ne l'écoute point et ne consens pas.
\VS{9}Il répondit donc aux messagers de Ben-Hadad : Dites au roi, mon seigneur : Je ferai tout ce que tu as envoyé demander la première fois à ton serviteur, mais je ne pourrai faire ceci. Les messagers s'en allèrent et lui rapportèrent cette réponse.
\VS{10}Et Ben-Hadad envoya dire à Achab : Que les dieux me traitent dans toute leur rigueur, si la poudre de Samarie suffit pour remplir le creux de la main de tout le peuple qui me suit.
\VS{11}Mais le roi d'Israël répondit, et dit : Dites-lui : Que celui qui revêt une armure ne se glorifie point comme celui qui la dépose.
\VS{12}Lorsque Ben-Hadad entendit cette réponse, il était à boire avec les rois sous les tentes et il dit à ses serviteurs : Rangez-vous en bataille ! Et ils se rangèrent en bataille contre la ville.
\TextTitle{Victoire Achab}
\VS{13}Alors voici, un prophète s’approcha d’Achab, roi d'Israël et lui dit : Ainsi parle Yahweh : N'as-tu pas vu cette grande multitude ? Voilà, je m'en vais la livrer aujourd'hui entre tes mains, et tu sauras que je suis Yahweh.
\VS{14}Et Achab dit : Par qui ? Et il lui répondit : Ainsi parle Yahweh : Ce sera par les serviteurs des chefs des provinces. Et Achab dit : Qui engagera le combat ? Et il lui répondit : Toi.
\VS{15}Alors il passa en revue les serviteurs des chefs des provinces, qui furent deux cent trente-deux ; et après eux, il dénombra tout le peuple de tous les enfants d'Israël qui furent sept mille.
\VS{16}Ils firent une sortie en plein midi, lorsque Ben-Hadad buvait et s'enivrait dans les tentes, lui et les trente-deux rois qui étaient ses auxiliaires.
\VS{17}Les serviteurs des chefs des provinces sortirent les premiers et Ben-Hadad envoya quelques-uns qui le lui rapportèrent en disant : Des hommes sont sortis de Samarie.
\VS{18}Et il dit : Qu’ils soient sortis pour la paix, ou qu'ils soient sortis pour faire la guerre, saisissez-les tous vivants.
\VS{19}Les serviteurs des chefs de province sortirent de la ville puis l'armée qui était après eux.
\VS{20}Chacun d'eux frappa son homme, de sorte que les Syriens s'enfuirent et Israël les poursuivit. Ben-Hadad, roi de Syrie, se sauva sur un cheval, avec des cavaliers.
\VS{21}Et le roi d'Israël sortit et frappa les chevaux et les chars, en sorte qu'il fit éprouver une grande défaite aux Syriens.
\TextTitle{Achab monte de nouveau contre les Syriens}
\VS{22}Alors le prophète s’approcha du roi d'Israël, et lui dit : Va, fortifie-toi ; considère et vois ce que tu auras à faire ; car l’année révolue, le roi de Syrie montera contre toi.
\VS{23}Or, les serviteurs du roi de Syrie lui dirent : Leur dieu est un dieu de montagnes, c'est pourquoi ils ont été plus forts que nous. Mais combattons contre eux dans la plaine, et certainement, nous serons plus forts qu'eux.
\VS{24}Fais donc ceci : Ote chacun de ces rois de sa place, et remplace-les par des chefs ;
\VS{25}Puis lève une armée pareille à celle que tu as perdue, avec autant de chevaux et de chars, puis nous les combattrons dans la plaine et l’on verra si nous ne sommes pas plus forts qu'eux. Il les écouta, et fit ainsi.
\VS{26}L’année suivante, Ben-Hadad dénombra les Syriens et monta à Aphek pour combattre contre Israël.
\VS{27}On fit aussi le dénombrement des enfants d'Israël ; ils reçurent des vivres, et ils marchèrent à la rencontre des Syriens. Les enfants d'Israël campèrent vis-à-vis d'eux ; semblables à deux petits troupeaux de chèvres, tandis que les Syriens remplissaient le pays.
\VS{28}Alors l'homme de Dieu vint, et dit au roi d'Israël : Ainsi parle Yahweh : Parce que les Syriens ont dit : Yahweh est un dieu des montagnes et non un dieu des vallées, je livrerai entre tes mains toute cette grande multitude, et vous saurez que je suis Yahweh.
\VS{29}Sept jours durant ils campèrent vis-à-vis les uns des autres. Le septième jour, ils entrèrent en bataille, et les enfants d'Israël tuèrent en un seul jour cent mille hommes de pied des Syriens.
\VS{30}Le reste s'enfuit à la ville d'Aphek, où la muraille tomba sur vingt-sept mille hommes demeurés de reste. Ben-Hadad s'était réfugié dans la ville où il allait de chambre en chambre.
\TextTitle{Faute d'Achab qui épargne Ben-Hadad}
\VS{31}Ses serviteurs lui dirent : Voici maintenant, nous avons appris que les rois de la maison d'Israël sont des rois miséricordieux ; maintenant donc mettons des sacs sur nos reins et des cordes à nos têtes, sortons vers le roi d'Israël, peut-être qu'il te laissera la vie sauve.
\VS{32}Ils se mirent donc des sacs autour des reins et des cordes autour de leurs têtes. Ils allèrent auprès du roi d'Israël. Ils lui dirent : Ton serviteur Ben-Hadad dit : Laisse-moi la vie ! Achab répondit : Est-il encore vivant ? Il est mon frère.
\VS{33}Ces hommes tirèrent de là un bon augure, ils se hâtèrent de le prendre au mot et ils dirent : Ben-Hadad est-il ton frère ! Et il répondit : Allez, amenez-le. Ben-Hadad vint vers lui, et il le fit monter sur son char.
\VS{34}Et Ben-Hadad lui dit : Je te rendrai les villes que mon père avait prises à ton père ; et tu te feras des rues en Damas comme mon père avait fait en Samarie. Et moi, répondit Achab, je te laisserai aller en faisant alliance. Il traita donc alliance avec lui, et le laissa aller.
\VS{35}Alors un homme d'entre les fils des prophètes dit à son compagnon, sur l’ordre de Yahweh : Frappe-moi, je te prie ! Mais celui-là refusa de le frapper.
\VS{36}Et il lui dit : Parce que tu n'as point obéi à la parole de Yahweh, voilà, quand tu m’auras quitté, un lion te frappera. Quand il se fut séparé de lui, un lion survint et le frappa.
\VS{37}Puis il trouva un autre homme, et lui dit : Frappe-moi, je te prie. Cet homme-là le frappa et il le blessa.
\VS{38}Après cela le prophète s'en alla, et se plaça sur le chemin du roi ; il se déguisa avec un bandeau sur ses yeux.
\VS{39}Lorsque le roi passa, il cria vers lui, et dit : Ton serviteur était allé au milieu de la bataille ; et voici quelqu'un s'étant retiré, m'a amené un homme, en disant : Garde cet homme, s'il vient à s'échapper, ta vie en répondra, ou tu paieras un talent d'argent.
\VS{40}Et pendant que ton serviteur faisait quelques affaires çà et là, cet homme a disparu. Et le roi d'Israël lui répondit : Telle est ta condamnation, tu l’as toi-même prononcée.
\VS{41}Alors le prophète ôta promptement le bandeau de dessus ses yeux et le roi d'Israël reconnut que c'était l’un des prophètes.
\VS{42}Et il dit : Ainsi parle Yahweh : Parce que tu as laissé échapper de tes mains l'homme que j'avais dévoué par la voie de l'interdit, ta vie répondra de sa vie, et ton peuple de son peuple.
\VS{43}Mais le roi d'Israël se retira en sa maison, triste et irrité ; et il arriva en Samarie.
\Chap{21}
\TextTitle{Achab convoite la vigne de Naboth}
\VerseOne{}Après ces choses, voici ce qui arriva. Naboth de Jizreel, avait une vigne à Jizreel, près du palais d'Achab, roi de Samarie.
\VS{2}Achab parla à Naboth et lui dit : Cède-moi ta vigne, afin que j'en fasse un jardin potager, car elle est proche de ma maison et je te donnerai à la place une vigne meilleure ; ou, si cela te semble bon, je te paierai l'argent qu'elle vaut.
\VS{3}Mais Naboth répondit à Achab : Que Yahweh me garde de te donner l'héritage de mes pères !
\VS{4}Et Achab vint en sa maison tout triste et irrité, à cause de cette parole que lui avait dite Naboth de Jizreel, en disant : Je ne te donnerai point l'héritage de mes pères ! Il se coucha sur son lit, détourna son visage, et ne mangea rien.
\TextTitle{Manigance meurtrière de Jézabel}
\VS{5}Alors Jézabel, sa femme, vint auprès de lui, et lui dit : D'où vient que ton esprit est si triste ? Et pourquoi ne manges-tu point ?
\VS{6}Et il lui répondit : J’ai parlé à Naboth de Jizreel, et je lui ai dit : Donne-moi ta vigne pour de l'argent, ou si tu le désires, je te donnerai une autre vigne pour celle-là, mais il m'a dit : Je ne te céderai point ma vigne !
\VS{7}Alors Jézabel, sa femme, lui dit : Est-ce bien toi maintenant qui exerces la royauté sur Israël ? Lève-toi, prends un repas et que ton cœur se réjouisse ; je te ferai avoir la vigne de Naboth de Jizreel.
\VS{8}Et elle écrivit au nom d'Achab des lettres qu’elle scella du sceau du roi, et elle envoya aux anciens et magistrats qui habitaient avec Naboth, dans sa ville.
\VS{9}Voici ce qu’elle écrivit dans ces lettres : Publiez un jeûne et placez Naboth à la tête du peuple.
\VS{10}Mettez face à lui deux méchants hommes et qu'ils témoignent contre lui, en disant : Tu as maudit Dieu et le roi ! Puis vous le mènerez dehors et le lapiderez afin qu'il meure.
\VS{11}Les gens donc de la ville de Naboth, les anciens et les magistrats qui habitaient dans sa ville, agirent comme Jézabel le leur avait dit, et d’après ce qui était écrit dans les lettres qu'elle leur avait envoyées.
\VS{12}Ils publièrent un jeûne et ils placèrent Naboth à la tête du peuple.
\VS{13}Les deux méchants hommes vinrent et se mirent face à lui, et ces méchants hommes déclarèrent contre Naboth en la présence du peuple : Naboth a maudit Dieu et le roi ! Puis ils le menèrent hors de la ville, ils le lapidèrent, et il mourut.
\VS{14}Après cela, ils envoyèrent dire à Jézabel : Naboth a été lapidé, et il est mort.
\VS{15}Lorsque Jézabel apprit que Naboth avait été lapidé et qu'il était mort, elle dit à Achab : Lève-toi, mets-toi en possession de la vigne de Naboth de Jizreel, qu’il avait refusé de te donner pour de l'argent ; car Naboth n'est plus en vie, il est mort.
\VS{16}Ainsi dès qu'Achab eut entendu que Naboth était mort, il se leva pour descendre à la vigne de Jizreel et pour s'en mettre en possession.
\TextTitle{Jugement d’Achab et de Jézabel ; Achab s’humilie devant Dieu}
\VS{17}Alors la parole de Yahweh fut adressée à Elie, le Thischbite, en ces mots :
\VS{18}Lève-toi, descends au-devant d'Achab, roi d'Israël, lorsqu'il sera à Samarie. Le voilà dans la vigne de Naboth, où il est descendu pour en prendre possession.
\VS{19}Et tu lui diras : Ainsi parle Yahweh : N’es-tu pas un meurtrier et un voleur ? Puis tu lui diras : Ainsi parle Yahweh : Comme les chiens ont léché le sang de Naboth, les chiens lécheront aussi ton propre sang.
\VS{20}Et Achab dit à Elie : M'as-tu trouvé mon ennemi ? Mais il lui répondit : Oui, je t'ai trouvé, parce que tu t'es vendu pour faire ce qui est mal aux yeux de Yahweh.
\VS{21}Voici je vais faire venir le malheur sur toi, et je te consumerai, j’exterminerai quiconque appartient à Achab, tant celui qui est esclave, que celui qui est libre en Israël.
\VS{22}Je rendrai ta maison semblable à la maison de Jéroboam, fils de Nebath, et la maison de Baescha, fils d'Achija, parce que tu m'as irrité et fait pécher Israël.
\VS{23}Yahweh parla aussi contre Jézabel en disant : Les chiens mangeront Jézabel près du rempart de Jizreel.
\VS{24}Celui de la maison d’Achab qui mourra dans la ville, les chiens le mangeront, et celui qui mourra aux champs, les oiseaux des cieux le mangeront.
\VS{25}En effet, il n'y en avait point eu de personne comme Achab, qui se soit vendu pour faire ce qui est mal aux yeux de Yahweh, et sa femme Jézabel l’y excitait ;
\VS{26}de sorte qu'il se rendit fort abominable, allant après les idoles, comme l'avaient fait les Amoréens, que Yahweh avait chassés de devant les enfants d'Israël.
\VS{27}Après avoir entendu les paroles d’Elie, Achab déchira ses vêtements, il mit un sac sur son corps, et jeûna. Il se tenait couché avec ce sac, et il marchait lentement.
\VS{28}Et la parole de Yahweh fut adressée à Elie, le Thischbite, en disant :
\VS{29}As-tu vu comment Achab s'est humilié devant moi ? Parce qu'il s'est humilié devant moi, je ne ferai pas venir le malheur pendant sa vie, ce sera aux jours de son fils que je ferai venir le malheur sur sa maison.
\Chap{22}
\TextTitle{Josaphat aide Achab contre les Syriens}
\VerseOne{}Et on resta trois ans sans qu'il y eût guerre entre la Syrie et Israël.
\VS{2}Puis il arriva, dans la troisième année, que Josaphat, roi de Juda, descendit vers le roi d'Israël.
\VS{3}Le roi d'Israël dit à ses serviteurs : Ne savez-vous pas que Ramoth de Galaad nous appartient ? Et nous ne nous inquiétons pas de la reprendre des mains du roi de Syrie !
\VS{4}Puis il dit à Josaphat : Viendras-tu avec moi à la guerre contre Ramoth de Galaad ? Et Josaphat répondit au roi d'Israël : Nous irons, moi comme toi, mon peuple comme ton peuple, et mes chevaux comme tes chevaux.
\TextTitle{Les prophètes de mensonge\FTNTT{2 Ch. 18:4-5, 9-11}}
\VS{5}Josaphat dit encore au roi d'Israël : Consulte aujourd'hui, je te prie, la parole de Yahweh.
\VS{6}Et le roi d'Israël assembla les prophètes, au nombre de quatre cents environ, auxquels il dit : Irai-je à la guerre contre Ramoth de Galaad, ou dois-je y renoncer ? Et ils répondirent : Monte, car le Seigneur la livrera entre les mains du roi.
\VS{7}Mais Josaphat dit : N'y a-t-il point ici encore quelque prophète de Yahweh, afin que nous le consultions ?
\VS{8}Et le roi d'Israël dit à Josaphat : Il y a encore un homme par qui l’on puisse consulter Yahweh, mais je le hais, car il ne prophétise rien de bon, mais seulement du mal, c'est Michée, fils de Jimla. Josaphat dit : Que le roi ne parle point ainsi !
\VS{9}Alors le roi d'Israël appela un eunuque auquel il dit : Fais venir promptement Michée, fils de Jimla.
\VS{10}Or, le roi d'Israël et Josaphat, roi de Juda, étaient assis chacun sur son trône, revêtus de leurs habits, dans la place, vers l'entrée de la porte de Samarie ; et tous les prophètes prophétisaient en leur présence.
\VS{11}Sédécias, fils de Kenaana, s'était fait des cornes de fer et il dit : Ainsi parle Yahweh : De ces cornes-ci tu heurteras les Syriens, jusqu'à les détruire.
\VS{12}Et tous les prophètes prophétisaient de même, en disant : Monte à Ramoth de Galaad et tu réussiras ; et Yahweh la livrera entre les mains du roi.
\TextTitle{Michée annonce la défaite et la mort d'Achab\FTNTT{2 Ch. 18:6-8, 12-27, 28-34}}
\VS{13}Le messager qui était allé appeler Michée, lui parla ainsi : Voici, les prophètes parlent d'un commun accord au sujet du roi ; je te prie que ta parole soit semblable à celle de chacun d’eux ! Annonce du bien !
\VS{14}Mais Michée lui répondit : Yahweh est vivant ! J’annoncerai ce que Yahweh me dira.
\VS{15}Il vint donc vers le roi, et le roi lui dit : Michée, irons-nous à la guerre contre Ramoth de Galaad, ou devons-nous y renoncer ? Et il lui dit : Monte et tu réussiras, et Yahweh la livrera entre les mains du roi.
\VS{16}Et le roi lui dit : Jusqu'à combien de fois te conjurerai-je de ne me dire que la vérité au nom de Yahweh ?
\VS{17}Et il répondit : J'ai vu tout Israël dispersé par les montagnes, comme un troupeau de brebis qui n'a point de berger ; et Yahweh a dit : Ces gens n’ont point de maître, que chacun retourne en paix dans sa maison !
\VS{18}Alors le roi d'Israël dit à Josaphat : Ne t'ai-je pas bien dit que quand il est question de moi il ne prophétise rien de bon, mais seulement du mal ?
\VS{19}Et Michée lui dit : Ecoute néanmoins la parole de Yahweh ! J'ai vu Yahweh assis sur son trône, et toute l'armée des cieux se tenant devant lui, à sa droite et à sa gauche.
\VS{20}Et Yahweh a dit : Quel est celui qui séduira Achab, afin qu'il monte et qu'il périsse en Ramoth de Galaad ? Et ils répondaient, l'un parlait d'une manière et l'autre d'une autre.
\VS{21}Alors un esprit s'avança et se tint devant Yahweh, il déclara : Je le séduirai. Et Yahweh lui dit : Comment ?
\VS{22}Et il répondit : Je sortirai et je serai un esprit de mensonge dans la bouche de tous ses prophètes. Et Yahweh dit : Tu le séduiras et même tu en viendras à bout ; sors et fais ainsi !
\VS{23}Et maintenant, voici, Yahweh a mis un esprit de mensonge dans la bouche de tous tes prophètes que voilà et Yahweh a prononcé du mal contre toi.
\VS{24}Alors Sédécias, fils de Kenaana, s'approcha et frappa Michée sur la joue et dit : Par où l'Esprit de Yahweh est-il sorti de moi pour s'adresser à toi ?
\VS{25}Et Michée répondit : Voici, tu le verras le jour où tu iras de chambre en chambre pour te cacher.
\VS{26}Alors le roi d'Israël dit : Qu'on prenne Michée et qu'on le mène vers Amon, capitaine de la ville et vers Joas, le fils du roi.
\VS{27}Et tu diras : Ainsi a parlé le roi : Mettez cet homme en prison, nourrissez-le de pain et de l’eau d’affliction, jusqu'à ce que je revienne en paix.
\VS{28}Et Michée répondit : Si tu reviens en paix, Yahweh n'a point parlé par moi. Il dit aussi : Vous tous, peuples, entendez !
\VS{29}Le roi d'Israël monta avec Josaphat, roi de Juda, contre Ramoth de Galaad.
\VS{30}Et le roi d'Israël dit à Josaphat : Que je me déguise et que j'aille à la bataille ; mais toi, revêts-toi de tes habits. Le roi d'Israël donc se déguisa et alla au combat.
\VS{31}Or, le roi de Syrie avait donné un ordre aux trente-deux chefs de ses chars, en disant : Vous n’attaquerez ni petits ni grands, mais seulement contre le roi d'Israël.
\VS{32}Quand les chefs des chars aperçurent Josaphat, ils dirent : C'est certainement le roi d'Israël. Et ils s’approchèrent de lui pour le combattre, mais Josaphat s'écria.
\VS{33}Et quand les chefs des chars virent que ce n'était pas le roi d'Israël, ils se détournèrent de lui.
\TextTitle{Mort d’Achab}
\VS{34}Alors un homme tira de son arc au hasard, et frappa le roi d'Israël entre les jointures de la cuirasse. Et le roi dit à son conducteur de char : Tourne et fais-moi sortir du champ de bataille, car je suis blessé.
\VS{35}Or, le combat devint acharné ce jour-là. Le roi d'Israël fut arrêté dans son char en face des Syriens et il mourut sur le soir. Le sang de sa blessure coulait à l’intérieur du char.
\VS{36}Au coucher du soleil, on cria par tout le camp, en disant : Que chacun se retire en sa ville et chacun en son pays !
\VS{37}Ainsi mourut le roi, qui fut ramené à Samarie ; et l’on enterra le roi à Samarie.
\VS{38}Lorsqu’on lava le char à l’étang de Samarie, les chiens léchèrent le sang d’Achab, et les prostituées s’y baignèrent, selon la parole que Yahweh avait prononcée.
\VS{39}Le reste des actions d'Achab, tout ce qu’il a fait, la maison d'ivoire qu'il construisit et toutes les villes qu'il a bâties, toutes ces choses ne sont-elles pas écrites au livre des Chroniques des rois d'Israël ?
\TextTitle{Règne de Josaphat sur Juda\FTNTT{2 Ch. 17:19-20}}
\VS{40}Ainsi Achab se coucha avec ses pères. Et Achazia, son fils, régna à sa place.
\VS{41}Josaphat, fils d'Asa, régna sur Juda, la quatrième année d'Achab, roi d'Israël.
\VS{42}Josaphat avait trente-cinq ans lorsqu’il devint roi, et il régna vingt-cinq ans à Jérusalem. Sa mère s’appelait Azuba, fille de Schilchi.
\VS{43}Il suivit entièrement la voie d'Asa, son père, et ne s'en détourna point, faisant tout ce qui est droit aux yeux de Yahweh.
\VS{44}Toutefois les hauts lieux ne disparurent pas ; le peuple offrait encore des sacrifices et offrait encore des parfums sur les hauts lieux.
\VS{45}Josaphat fit aussi la paix avec le roi d'Israël.
\VS{46}Le reste des actions de Josaphat, ses exploits et les guerres qu'il mena ne sont-elles pas écrites au livre des Chroniques des rois de Juda ?
\VS{47}Il extermina du pays le reste des prostitués, qui étaient demeurés là depuis le temps d'Asa, son père.
\VS{48}Il n'y avait point alors de roi en Edom : c’était un intendant qui gouvernait.
\VS{49}Josaphat construisit des navires de Tarsis pour aller chercher de l'or à Ophir ; mais il n'y alla point, parce que les navires se brisèrent à Etsjon-Guéber.
\VS{50}Alors Achazia, fils d'Achab, dit à Josaphat : Que mes serviteurs aillent sur les navires avec les tiens, mais Josaphat ne le voulut point.
\TextTitle{Joram règne sur Juda\FTNTT{2 Ch. 21:1}}
\VS{51}Et Josaphat s’endormit avec ses pères et fut enterré avec eux en la cité de David, son père. Et Joram, son fils, régna à sa place.
\TextTitle{Achazia règne sur Israël}
\VS{52}Achazia, fils d'Achab, régna sur Israël à Samarie, la dix-septième année de Josaphat, roi de Juda. Et il régna deux ans sur Israël.
\VS{53}Il fit ce qui est mal aux yeux de Yahweh : il marcha dans la voie de son père, de sa mère et celle de Jéroboam, fils de Nebath, qui avait fait pécher Israël.
\VS{54}Il servit Baal, il se prosterna devant lui et il irrita Yahweh, le Dieu d'Israël, comme l’avait fait son père.
\PPE{}
\end{multicols}

%\clearpage\ShortTitle{2 R.}\BookTitle{2 Rois}\BFont
\noindent\hrulefill
{\footnotesize
\textit{
\bigskip
{\centering{}
\\Auteur~: Inconnu
\\(Heb.~: Melakhim)
\\Signification~: Roi, Règne
\\Thème~: Suite de l'histoire d'Israël et de Juda
\\Date de rédaction~: 6\up{ème} siècle av. J.-C.\\}
}
\textit{
\\Le second livre des rois s'articule autour de la vie d'Elisée, serviteur d'Elie, devenu dorénavant son successeur. On y découvre le service prophétique au travers duquel Dieu se révéla comme le Tout-Puissant, le Dieu compatissant, le Maître des temps et des circonstances, le Libérateur, le Dieu de la résurrection, le Puissant Guerrier et aussi le Juge.
\\Ce livre relate l'histoire des derniers rois, la chute d'Israël et sa captivité, la destruction de Jérusalem par Nebudcanetsar, roi de Babylone, en 586 av. J.-C., et la captivité de Juda.\bigskip
}
}
\par\nobreak\noindent\hrulefill
\begin{multicols}{2}
\Chap{1}
\TextTitle{Jugement de Yahweh sur Achazia, roi d'Israël}
\VerseOne{}Or après la mort d'Achab, Moab se révolta contre Israël.
\VS{2}Or Achazia tomba par le treillis de sa chambre haute qui était à Samarie, et il en fut malade. Il envoya des messagers et leur dit~: Allez, consultez Baal-Zebub\FTNT{Baal-Zebub était une divinité des Philistins adorée à Ekron qui se nommait aussi Béelzébul (Mt. 10:25).}, dieu d'Ekron, pour savoir si je guérirai de cette maladie.
\VS{3}Mais l'Ange de Yahweh dit à Elie\FTNT{Elie~: Voir 1 R. 17.}, le Thischbite~: Lève-toi, monte à la rencontre des messagers du roi de Samarie, et dis-leur~: N'y a-t-il point de Dieu en Israël pour que vous alliez consulter Baal-Zebub, dieu d'Ekron~?
\VS{4}C'est pourquoi ainsi parle Yahweh~: Tu ne descendras pas du lit sur lequel tu es monté, mais tu mourras, tu mourras\FTNT{Voir en Gn. 2:16.}. Et Elie s'en alla.
\VS{5}Les messagers retournèrent vers Achazia. Et il leur dit~: Pourquoi revenez-vous~?
\VS{6}Ils lui répondirent~: Un homme est monté à notre rencontre et nous a dit~: Allez, retournez vers le roi qui vous a envoyés et dites-lui~: Ainsi parle Yahweh~: N'y a-t-il point de Dieu en Israël, pour que tu envoies consulter Baal-Zebub, dieu d'Ekron~? A cause de cela, tu ne descendras pas du lit sur lequel tu es monté, mais certainement tu mourras.
\VS{7}Achazia leur dit~: Comment était cet homme qui est monté à votre rencontre et qui vous a dit ces paroles~?
\VS{8}Ils lui répondirent~: C'était un homme vêtu de poil, ayant une ceinture de cuir, ceinte sur ses reins. Et Achazia dit~: C'est Elie, le Thischbite.
\TextTitle{Affirmation de l'autorité d'Elie}
\VS{9}Alors il envoya vers lui un chef de cinquante avec ses cinquante hommes. Ce chef monta auprès d'Elie, qui demeurait au sommet d'une montagne, et il lui dit~: Homme de Dieu, le roi a dit~: Descends~!
\VS{10}Mais Elie répondit et dit au chef de cinquante~: Si je suis un homme de Dieu, que le feu descende du ciel et te consume, toi et tes cinquante hommes~! Et le feu descendit du ciel et le consuma, lui et ses cinquante hommes.
\VS{11}Achazia envoya encore un autre chef de cinquante avec ses cinquante hommes. Ce chef prit la parole et dit à Elie~: Homme de Dieu, ainsi parle le roi~: Hâte-toi de descendre~!
\VS{12}Mais Elie répondit, et leur dit~: Si je suis un homme de Dieu, que le feu descende du ciel et te consume, toi et tes cinquante hommes~! Et le feu de Dieu descendit du ciel et le consuma, lui et ses cinquante hommes.
\VS{13}Achazia envoya encore un troisième chef de cinquante avec ses cinquante hommes. Ce troisième chef de cinquante hommes monta, et vint se mettre à genoux devant Elie, le suppliant, en disant~: Homme de Dieu, je te prie, que ma vie et la vie de ces cinquante hommes, tes serviteurs, soit précieuse à tes yeux~!
\VS{14}Voici, le feu est descendu du ciel et a consumé les deux premiers chefs de cinquante, avec leurs cinquante hommes~; mais maintenant, je te prie, que ma vie soit précieuse à tes yeux~!
\VS{15}Et l'Ange de Yahweh dit à Elie~: Descends avec lui, n'aie pas peur de lui. Elie se leva donc et descendit avec lui vers le roi.
\VS{16}Il lui dit~: Ainsi parle Yahweh~: Parce que tu as envoyé des messagers pour consulter Baal-Zebub, dieu d'Ekron, comme s'il n'y avait point de Dieu en Israël, pour consulter sa parole, tu ne descendras pas du lit sur lequel tu es monté, mais certainement tu mourras.
\TextTitle{Mort d'Achazia~; Joram règne sur Israël}
\VS{17}Achazia mourut, selon la parole de Yahweh prononcée par Elie. Et Joram régna à sa place, la seconde année de Joram, fils de Josaphat, roi de Juda, parce qu'Achazia n'avait point de fils.
\VS{18}Le reste des actions d'Achazia et ce qu'il a fait, cela n'est-il pas écrit dans le livre des Chroniques des rois d'Israël~?
\Chap{2}
\TextTitle{Enlèvement d'Elie au ciel}
\VerseOne{}Or il arriva lorsque Yahweh enleva Elie au ciel dans un tourbillon, Elie et Elisée partaient de Guilgal.
\VS{2}Elie dit à Elisée~: Je te prie, reste ici, car Yahweh m'envoie jusqu'à Béthel. Mais Elisée répondit Yahweh est vivant et ton âme est vivante~! Je ne te quitterai pas~! Ainsi ils descendirent à Béthel.
\VS{3}Les fils des prophètes qui étaient à Béthel sortirent vers Elisée, et lui dirent~: Ne sais-tu pas qu'aujourd'hui Yahweh va enlever ton maître au-dessus de ta tête~? Et il répondit~: Je le sais aussi~; taisez-vous~!
\VS{4}Elie lui dit~: Elisée, je te prie, reste ici, car Yahweh m'envoie à Jéricho. Mais Elisée lui répondit~: Yahweh est vivant et ton âme est vivante~! Je ne te quitterai pas~! Ainsi, ils arrivèrent à Jéricho.
\VS{5}Les fils des prophètes qui étaient à Jéricho s'approchèrent d'Elisée, et lui dirent~: Ne sais-tu pas qu'aujourd'hui Yahweh va enlever ton maître au-dessus de ta tête~? Et il répondit~: Je le sais aussi~; taisez-vous~!
\VS{6}Elie lui dit~: Elisée, je te prie demeure ici, car Yahweh m'envoie jusqu'au Jourdain. Mais Elisée répondit~: Yahweh est vivant et ton âme est vivante~! Je ne te quitterai pas~! Ainsi, ils s'en allèrent tous les deux.
\VS{7}Cinquante hommes d'entre les fils des prophètes arrivèrent et s'arrêtèrent à distance vis-à-vis d'eux, et eux deux s'arrêtèrent au bord du Jourdain.
\VS{8}Alors Elie prit son manteau, le roula et en frappa les eaux, qui se divisèrent çà et là, et ils passèrent tous deux à sec.
\VS{9}Quand ils furent passés, Elie dit à Elisée~: Demande ce que tu veux que je fasse pour toi, avant que je sois enlevé d'avec toi. Elisée répondit~: Je te prie, que j'aie, une double portion\FTNT{Le fils aîné recevait une double portion par rapport aux autres fils (De. 21:15-17).} de ton esprit~!
\VS{10}Elie lui dit~: Tu demandes une chose difficile. Mais si tu me vois pendant que je serai enlevé d'avec toi, cela te sera accordé~; mais si tu ne me vois pas, cela ne te sera pas accordé.
\VS{11}Comme ils continuaient à marcher en parlant, voici, un char de feu et des chevaux de feu les séparèrent l'un de l'autre, et Elie monta au ciel dans un tourbillon.
\TextTitle{La double portion de l'esprit d'Elie sur Elisée}
\VS{12}Elisée le regardait et criait~: Mon père~! Mon père~! Char d'Israël et sa cavalerie~! Et il ne le vit plus. Puis saisissant ses vêtements, il les déchira en deux morceaux.
\VS{13}Il releva le manteau qu'Elie avait laissé tomber. Puis il retourna et s'arrêta sur le bord du Jourdain.
\VS{14}Ensuite il prit le manteau qu'Elie avait laissé tomber et il en frappa les eaux, et dit~: Où est Yahweh, le Dieu d'Elie, Yahweh lui-même~? Lui aussi frappa les eaux qui se divisèrent en deux~; et Elisée passa.
\TextTitle{Le service d'Elisée est reconnu par les hommes}
\VS{15}Quand les fils des prophètes qui étaient à Jéricho, vis-à-vis, l'eurent vu, ils dirent~: L'esprit d'Elie repose sur Elisée~! Ils vinrent à sa rencontre et se prosternèrent contre terre devant lui.
\VS{16}Ils lui dirent~: Voici, il y a parmi tes serviteurs cinquante hommes vaillants~; veux-tu qu'ils aillent chercher ton maître, de peur que l'Esprit de Yahweh ne l'ait enlevé et ne l'ait jeté sur quelque montagne ou dans quelque vallée~? Elisée répondit~: Ne les envoyez pas.
\VS{17}Mais ils le pressèrent tant par leurs paroles, qu'il en était embarrassé. Il leur dit donc~: Envoyez-les. Ils envoyèrent cinquante hommes, qui pendant trois jours cherchèrent Elie, mais ils ne le trouvèrent point.
\VS{18}Puis ils retournèrent vers Elisée, qui était à Jéricho, et il leur dit~: Ne vous avais-je pas dit~: N'y allez pas~?
\VS{19}Les gens de la ville dirent à Elisée~: Voici, le séjour dans cette ville est bon, comme mon seigneur le voit~; mais les eaux sont mauvaises et le pays est stérile.
\VS{20}Il dit~: Apportez-moi un vase neuf et mettez-y du sel. Et ils le lui apportèrent.
\VS{21}Puis il alla vers la source des eaux, et il y jeta le sel, et dit~: Ainsi parle Yahweh~: J'assainis ces eaux~; elles ne causeront plus ni mort ni stérilité.
\VS{22}Les eaux furent assainies, jusqu'à ce jour, selon la parole qu'Elisée avait prononcée.
\TextTitle{Jugement des moqueurs}
\VS{23}Elisée monta de là à Béthel~; et comme il montait par le chemin, des petits garçons sortirent de la ville et se moquèrent de lui. Ils lui disaient~: Monte chauve~! Monte chauve~!
\VS{24}Il se retourna pour les regarder, et il les maudit au nom de Yahweh. Alors deux ours sortirent de la forêt et déchirèrent quarante-deux de ces enfants.
\VS{25}De là il alla sur la montagne de Carmel, d'où il retourna à Samarie.
\Chap{3}
\TextTitle{Joram règne sur Israël}
\VerseOne{}La dix-huitième année de Josaphat, roi de Juda, Joram, fils d'Achab, régna sur Israël à Samarie. Il régna douze ans.
\VS{2}Il fit ce qui est mal aux yeux de Yahweh, non pas toutefois comme son père et sa mère, car il ôta la statue de Baal que son père avait faite~;
\VS{3}mais il s'attacha aux péchés de Jéroboam, fils de Nebath, qui avait fait pécher Israël et il ne s'en détourna point.
\TextTitle{Rébellion de Moab~; Israël et Juda s'allient pour combattre}
\VS{4}Or Méscha, roi de Moab, possédait des troupeaux, et il payait au roi d'Israël un tribut de cent mille agneaux et cent mille béliers avec leur laine.
\VS{5}Mais aussitôt qu'Achab mourut, le roi de Moab se révolta contre le roi d'Israël.
\VS{6}C'est pourquoi le roi Joram sortit ce jour-là de Samarie et passa en revue tout Israël.
\VS{7}Il se mit en marche et fit dire à Josaphat, roi de Juda~: Le roi de Moab s'est rebellé contre moi~; veux-tu venir avec moi faire la guerre à Moab~? Josaphat répondit~: Je monterai, moi comme toi, mon peuple comme ton peuple, mes chevaux comme tes chevaux.
\VS{8}Ensuite il dit~: Par quel chemin monterons-nous~? Joram répondit~: Par le chemin du désert d'Edom.
\TextTitle{Les rois d'Israël, de Juda et d'Edom en marche~; ils consultent Elisée}
\VS{9}Ainsi, le roi d'Israël, le roi de Juda et le roi d'Edom, partirent~; ils firent un détour, et après une marche de sept jours, ils manquèrent d'eau pour l'armée et pour les bêtes qui la suivaient.
\VS{10}Alors le roi d'Israël dit~: Hélas~! Yahweh a appelé ces trois rois pour les livrer entre les mains de Moab.
\VS{11}Et Josaphat dit~: N'y a-t-il ici aucun prophète de Yahweh, par qui nous puissions consulter Yahweh~? Et un des serviteurs du roi d'Israël répondit, et dit~: Il y a ici Elisée, fils de Schaphath, qui versait de l'eau sur les mains d'Elie.
\VS{12}Alors Josaphat dit~: La parole de Yahweh est avec lui. Le roi d'Israël, Josaphat et le roi d'Edom descendirent vers lui.
\VS{13}Mais Elisée dit au roi d'Israël~: Qu'y a-t-il entre moi et toi~? Va-t'en vers les prophètes de ton père et vers les prophètes de ta mère. Et le roi d'Israël lui répondit~: Non~! Car Yahweh a appelé ces trois rois pour les livrer entre les mains de Moab.
\VS{14}Elisée dit~: Yahweh des armées, devant lequel je me tiens, est vivant~! Si je n'avais de la considération pour Josaphat, roi de Juda, je ne ferais aucune attention à toi et je ne te regarderais même pas.
\VS{15}Mais maintenant, amenez-moi un joueur d'instruments à cordes. Et comme le joueur jouait des instruments à cordes, la main de Yahweh fut sur Elisée.
\TextTitle{Prophétie sur la défaite de Moab}
\VS{16}Et il dit~: Ainsi parle Yahweh~: Faites des tranchées dans toute cette vallée.
\VS{17}Car ainsi parle Yahweh~: Vous ne verrez ni vent, ni pluie, et néanmoins cette vallée sera remplie d'eaux, et vous boirez, vous et vos bêtes.
\VS{18}Mais cela est peu de chose aux yeux de Yahweh. Il livrera Moab entre vos mains~;
\VS{19}Vous frapperez toutes les villes fortes et toutes les villes d'élite, vous abattrez tous les bons arbres, vous boucherez toutes les sources d'eau et vous ruinerez avec des pierres tous les meilleurs champs.
\VS{20}Il arriva donc au matin, environ à l'heure de l'offrande, que l'eau arriva du chemin d'Edom, en sorte que ce pays fut rempli d'eau.
\VS{21}Cependant, tous les Moabites ayant appris que ces rois étaient montés pour leur faire la guerre s'étaient assemblés. On convoqua tous ceux qui étaient en âge de porter les armes, et même au-dessus, et ils se tinrent sur la frontière.
\VS{22}Et le lendemain, ils se levèrent de bon matin, et comme le soleil se levait sur les eaux, les Moabites virent en face d'eux les eaux rouges comme du sang.
\VS{23}Ils dirent~: C'est du sang~! Certainement, ces rois-là se sont entretués, et chacun a frappé son compagnon~; maintenant, Moabites, au butin~!
\VS{24}Ainsi ils marchèrent contre le camp d'Israël. Mais Israël se leva et frappa Moab, qui prit la fuite devant eux. Puis ils pénétrèrent dans le pays et frappèrent Moab.
\VS{25}Ils détruisirent les villes, et chacun jetait des pierres dans les meilleurs champs, de sorte qu'ils les en remplirent, ils bouchèrent toutes les sources d'eaux et abattirent tous les bons arbres~; et les frondeurs entourèrent et frappèrent Kir-Haréseth, dont on ne laissa que les pierres.
\VS{26}Le roi de Moab, voyant qu'il n'était pas le plus fort dans la bataille, prit avec lui sept cents hommes tirant l'épée pour se frayer un passage jusqu'au roi d'Edom~; mais ils ne purent pas.
\VS{27}Alors il prit son fils premier-né, qui devait régner à sa place, et l'offrit en holocauste sur la muraille. Et il y eut une grande indignation en Israël~; ainsi ils se retirèrent du roi de Moab et retournèrent dans leur pays.
\Chap{4}
\TextTitle{Miracle~: Le vase d'huile de la veuve}
\VerseOne{}Or une femme d'un des fils des prophètes cria à Elisée, en disant~: Ton serviteur mon mari est mort, et tu sais que ton serviteur craignait Yahweh~; or son créancier est venu pour prendre mes deux enfants, afin qu'ils soient ses esclaves.
\VS{2}Elisée lui répondit~: Que puis-je faire pour toi~? Dis-moi ce que tu as à la maison. Et elle dit~: Ta servante n'a rien dans toute la maison qu'un vase d'huile.
\VS{3}Alors il lui dit~: Va, demande des vases dans la rue à tous tes voisins, des vases vides, et n'en demande pas un petit nombre.
\VS{4}Puis rentre et ferme la porte sur toi et sur tes enfants, et verse dans tous ces vases, et tu mettras de côté ceux qui seront pleins.
\VS{5}Alors elle le quitta. Ayant fermé la porte sur elle et sur ses enfants~; ils lui présentaient les vases, et elle versait.
\VS{6}Lorsqu'elle eut rempli les vases, elle dit à son fils~: Présente-moi encore un vase. Mais il répondit~: Il n'y a plus de vase. Et l'huile s'arrêta.
\VS{7}Elle alla le raconter à l'homme de Dieu, qui lui dit~: Va, vends l'huile, et paye ta dette~; et vous vivrez, toi et tes fils, de ce qui restera.
\TextTitle{Yahweh se souvient de la Sunamite}
\VS{8}Et il arriva un jour qu'Elisée passait par Sunem, où il y avait une femme importante~; elle le retint avec grande instance à manger du pain chez elle. Et toutes les fois qu'il passait, il s'y retirait pour manger du pain.
\VS{9}Elle dit à son mari~: Voilà, je sais que cet homme qui passe souvent chez nous est un saint homme de Dieu.
\VS{10}Faisons-lui, je te prie, une petite chambre haute avec des murs, et mettons-y pour lui un lit, une table, un siège et un chandelier, afin que quand il viendra chez nous, il s'y retire.
\VS{11}Un jour, Elisée étant revenu à Sunem, il se retira dans cette chambre haute et s'y coucha.
\VS{12}Puis il dit à Guéhazi, son serviteur~: Appelle cette Sunamite. Guéhazi l'appela, et elle se présenta devant lui.
\VS{13}Et Elisée dit à Guéhazi~: Dis maintenant à cette femme~: Voici, tu nous as montré tout cet empressement~; que pourrait-on faire pour toi~? Faut-il parler pour toi au roi ou au chef de l'armée~? Elle répondit~: J'habite au milieu de mon peuple.
\VS{14}Et il dit~: Que faudrait-il faire pour elle~? Guéhazi répondit~: Mais elle n'a point de fils et son mari est vieux.
\VS{15}Et il dit~: Appelle-la. Guéhazi l'appela, et elle se présenta à la porte.
\VS{16}Elisée lui dit~: L'année prochaine, à cette même époque, tu embrasseras un fils. Elle répondit~: Mon seigneur, homme de Dieu, ne trompe pas, ne trompe pas ta servante~!
\VS{17}Cette femme devint enceinte et enfanta un fils un an après, à la même époque, comme Elisée lui avait dit.
\TextTitle{Foi de la Sunamite, résurrection de son fils}
\VS{18}L'enfant grandit. Il sortit un jour pour aller trouver son père vers les moissonneurs.
\VS{19}Et il dit à son père~: Ma tête~! Ma tête~! Et le père dit au serviteur~: Porte-le à sa mère.
\VS{20}Il le porta donc et l'amena à sa mère. Et l'enfant resta sur les genoux de sa mère jusqu'à midi, puis il mourut.
\VS{21}Elle monta et le coucha sur le lit de l'homme de Dieu~; et ayant fermé la porte sur lui, elle sortit.
\VS{22}Elle appela son mari et dit~: Je te prie envoie-moi un des serviteurs et une ânesse~; j'irai chez l'homme de Dieu et je reviendrai.
\VS{23}Et il dit~: Pourquoi vas-tu vers lui aujourd'hui~? Ce n'est point la nouvelle lune ni le sabbat. Elle répondit~: Tout va bien~!
\VS{24}Elle fit donc seller l'ânesse, et dit à son serviteur~: Conduis-moi et ne m'arrête pas en route sans que je te le dise.
\VS{25}Ainsi elle s'en alla et se rendit vers l'homme de Dieu sur la montagne de Carmel. L'homme de Dieu, l'ayant aperçue, dit à Guéhazi son serviteur~: Voilà la Sunamite~!
\VS{26}Va, cours à sa rencontre et dis-lui~: Te portes-tu bien~? Ton mari se porte-t-il bien~? L'enfant se porte-t-il bien~? Et elle répondit~: Nous nous portons bien.
\VS{27}Dès qu'elle fut arrivée auprès de l'homme de Dieu sur la montagne, elle embrassa ses pieds. Guéhazi s'approcha pour la repousser, mais l'homme de Dieu lui dit~: Laisse-la, car son âme est dans l'amertume, et Yahweh me l'a caché, et ne me l'a pas révélé.
\VS{28}Alors elle dit~: Ai-je demandé un fils à mon seigneur~? N'ai-je pas dit~: Ne me trompe pas~?
\VS{29}Et Elisée dit à Guéhazi~: Ceins tes reins, prends mon bâton dans ta main, et pars. Si tu rencontres quelqu'un, ne le salue pas~; et si quelqu'un te salue, ne lui réponds pas. Tu mettras mon bâton sur le visage de l'enfant.
\VS{30}Mais la mère de l'enfant dit~: Yahweh est vivant, et ton âme est vivante~! Je ne te quitterai point. Il se leva donc et la suivit.
\VS{31}Or Guéhazi les avait devancés et il avait mis le bâton sur le visage de l'enfant~; mais il n'y eut ni voix ni signe d'attention. Guéhazi retourna à la rencontre d'Elisée et l'en informa en disant~: L'enfant ne s'est pas réveillé.
\VS{32}Lorsqu'Elisée entra dans la maison, l'enfant, mort, était couché sur son lit.
\VS{33}Il ferma la porte sur eux deux et pria Yahweh.
\VS{34}Puis, il monta et se coucha sur l'enfant~; il mit sa bouche sur la bouche de l'enfant, ses yeux sur ses yeux, ses mains sur ses mains, et il s'étendit sur lui. La chair de l'enfant se réchauffa.
\VS{35}Puis il s'éloigna et marcha dans la maison, tantôt dans un lieu, tantôt dans un autre, et il remonta et s'étendit encore sur lui. L'enfant éternua sept fois et ouvrit ses yeux.
\VS{36}Alors, Elisée appela Guéhazi, et lui dit~: Appelle cette Sunamite. Guéhazi l'appela, et elle vint vers Elisée qui lui dit~: Prends ton fils~!
\VS{37}Elle se jeta à ses pieds et se prosterna contre terre. Puis elle prit son fils et sortit.
\TextTitle{Les coloquintes sauvages}
\VS{38}Après cela, Elisée revint à Guilgal. Or il y avait une famine\FTNT{Par le passé, Israël a connu plusieurs famines, dont celle relatée en 2 R. 4:38-41.. Dans ce passage, l'un des fils des prophètes trouva une vigne sauvage dans un champ et y cueillit des coloquintes sauvages. Il les ajouta au potage qui mijotait dans un pot, ne sachant pas que c'était du poison. Le pot est l'image des églises de Laodicée dans lesquelles il y a un mélange mortel de fausses doctrines et de préceptes mondains qui viennent altérer la vérité de la parole de Dieu. Ce mélange impur est absorbé par des millions de personnes ignorantes à travers le monde. Celles-ci se rendent compte qu'elles ont été empoisonnées spirituellement, et une fois le mélange ingéré, elles constatent les effets pervers et dévastateurs souvent tardivement. Le champ tout comme la vigne sauvage, selon Mt. 13:38 et Ro. 11:17, symbolise le monde. Il est par ailleurs intéressant de noter que le mot herbe, «~owrah~» en hébreu, signifie aussi lumière (Ps. 139:12). Cette histoire n'est pas sans nous rappeler le feu étranger introduit par les fils d'Aaron dans le tabernacle, et ce, malgré l'interdiction formelle de Yahweh (Ex. 30:9~; Lé. 10:1-5). C'est exactement ce qui se passe de nos jours. Les églises importent de plus en plus en leur sein la lumière luciférienne du monde (musique, marketing, philosophie, etc.). Beaucoup de pasteurs et de musiciens cherchent malheureusement leur inspiration dans le monde à cause de la famine qui sévit dans les églises. Ce feu étranger représente la plupart des doctrines et pratiques promues par l'église de Laodicée.} dans le pays, et les fils des prophètes étaient assis devant lui~; et il dit à son serviteur~: Mets le grand pot et fais cuire du potage pour les fils des prophètes.
\VS{39}Mais quelqu'un étant sorti dans les champs pour cueillir des herbes, trouva de la vigne sauvage, et cueillit des coloquintes sauvages plein sa robe, et étant revenu, il les coupa en morceaux dans le pot où était le potage, car on ne savait pas ce que c'était.
\VS{40}Et on servit à manger de ce potage à quelques-uns~; mais aussitôt qu'ils eurent mangé de ce potage, ils s'écrièrent et dirent~: Homme de Dieu, la mort est dans le pot~! Et ils ne purent en manger.
\VS{41}Et il dit~: Apportez-moi de la farine~; et il en jeta dans le pot, puis il dit~: Qu'on en verse à ce peuple, afin qu'il mange~; et il n'y avait plus rien de mauvais dans le pot.
\TextTitle{Multiplication de pains}
\VS{42}Un homme venant de Baal-Schalischa apporta à l'homme de Dieu du pain des prémices, à savoir vingt pains d'orge et des épis nouveaux. Elisée dit~: Donne cela à ces gens, et qu'ils mangent.
\VS{43}Son serviteur répondit~: Comment pourrais-je en donner à cent hommes~? Mais Elisée lui répondit~: Donne-les à ces gens, et qu'ils mangent~; car ainsi parle Yahweh~: Ils mangeront et il en restera encore.
\VS{44}Il mit donc les pains devant eux. Ils mangèrent et en eurent de reste, selon la parole de Yahweh.
\Chap{5}
\TextTitle{Guérison miraculeuse de Naaman}
\VerseOne{}Or Naaman, chef de l'armée du roi de Syrie, était un homme puissant et très considéré aux yeux de son maître~; car c'était par lui que Yahweh avait délivré les Syriens. Mais cet homme fort et vaillant était lépreux.
\VS{2}Et les Syriens étaient sortis par troupes, et ils avaient emmené prisonnière une petite fille du pays d'Israël, qui était au service de la femme de Naaman.
\VS{3}Elle dit à sa maîtresse~: Oh~! Si mon seigneur se présentait devant le prophète qui est à Samarie, il le guérirait de sa lèpre~!
\VS{4}Naaman le rapporta à son maître, en disant~: La fille qui est du pays d'Israël a dit telle et telle chose.
\VS{5}Et le roi de Syrie dit à Naaman~: Va, rends-toi à Samarie et j'enverrai une lettre au roi d'Israël. Naaman donc s'en alla et prit avec lui dix talents d'argent et six mille pièces d'or, et dix vêtements de rechange.
\VS{6}Il porta au roi d'Israël la lettre, où il était dit~: Dès que cette lettre te sera parvenue, sache que je t'ai envoyé Naaman, mon serviteur, afin que tu le guérisses de sa lèpre.
\VS{7}Et dès que le roi d'Israël eut lu la lettre, il déchira ses vêtements et dit~: Suis-je Dieu pour faire mourir et pour rendre la vie, pour qu'il s'adresse à moi afin que je guérisse un homme de sa lèpre~? Voyez et comprenez qu'il cherche certainement une occasion de dispute avec moi.
\VS{8}Et il arriva qu'aussitôt qu'Elisée, homme de Dieu, apprit que le roi d'Israël avait déchiré ses vêtements, il envoya dire au roi~: Pourquoi as-tu déchiré tes vêtements~? Laisse-le venir vers moi et il saura qu'il y a un prophète en Israël.
\VS{9}Naaman vint avec ses chevaux et son char, et il s'arrêta à la porte de la maison d'Elisée.
\VS{10}Elisée envoya un messager vers lui, pour lui dire~: Va, et lave-toi sept fois dans le Jourdain, et ta chair redeviendra saine, et tu seras pur.
\VS{11}Mais Naaman se mit dans une grande colère, et s'en alla en disant~: Voilà, je me disais~: Il sortira et viendra vers moi, il se présentera lui-même, il invoquera le Nom de Yahweh, son Dieu, puis, il agitera sa main sur la plaie, et guérira le lépreux.
\VS{12}Les fleuves de Damas, l'Abana et le Parpar ne sont-ils pas meilleurs que toutes les eaux d'Israël~? Ne pourrais-je pas m'y laver et devenir pur~? Ainsi donc, il s'en retourna et s'en alla furieux.
\VS{13}Mais ses serviteurs s'approchèrent et lui parlèrent en disant~: Mon père, si le prophète t'avait imposé quelque chose de difficile, ne l'aurais-tu pas fait~? Combien plus dois-tu faire ce qu'il t'a dit~: Lave-toi, et tu deviendras pur~!
\VS{14}Alors il descendit et se plongea sept fois dans le Jourdain, selon la parole de l'homme de Dieu~; et sa chair redevint comme la chair d'un petit enfant~; et il fut pur.
\VS{15}Il retourna vers l'homme de Dieu, lui et tout son camp, et il vint se présenter devant lui et dit~: Voici, maintenant je sais qu'il n'y a point d'autre Dieu sur toute la terre, si ce n'est en Israël. Maintenant donc, je te prie, accepte ce présent de ton serviteur.
\VS{16}Elisée répondit~: Yahweh, devant lequel je me tiens, est vivant~! Je ne l'accepterai pas~! Naaman le pressa fort de l'accepter, mais Elisée refusa~!
\VS{17}Alors, Naaman dit~: Je te prie, permets que l'on donne de la terre à ton serviteur, une charge de deux mulets~; car ton serviteur ne fera plus d'holocauste ni de sacrifice à d'autres dieux, mais seulement à Yahweh.
\VS{18}Voici toutefois, que Yahweh pardonne ceci à ton serviteur. Quand mon maître entre dans la maison de Rimmon pour s'y prosterner et qu'il s'appuie sur ma main, je me prosterne aussi dans la maison de Rimmon~: Que Yahweh me pardonne, quand je me prosternerai dans la maison de Rimmon.
\TextTitle{Convoitise et mensonge de Guéhazi~; jugement de Dieu}
\VS{19}Elisée lui dit~: Va en paix. Lorsque Naaman eut quitté Elisée et qu'il fut à une certaine distance,
\VS{20}Guéhazi\FTNT{Guéhazi, dont le nom hébreu signifie «~vallée de la vision~».}, le serviteur d'Elisée, homme de Dieu, se dit en lui-même~: Voici, mon maître a ménagé Naaman, ce Syrien, et n'a pas accepté de sa main ce qu'il avait apporté~; Yahweh est vivant~! Je vais courir après lui et j'en obtiendrai quelque chose.
\VS{21}Et Guéhazi courut après Naaman. Naaman, le voyant courir après lui, descendit de son char pour aller à sa rencontre. Il dit~: Tout va bien~?
\VS{22}Guéhazi répondit~: Tout va bien. Mon maître m'envoie te dire~: Voici, il vient d'arriver chez moi deux jeunes hommes de la montagne d'Ephraïm, d'entre les fils des prophètes. Je te prie donne-leur un talent d'argent et deux vêtements de rechange.
\VS{23}Et Naaman dit~: Consens à prendre deux talents. Il insista, puis il serra deux talents d'argent dans deux sacs avec deux vêtements de rechange et les fit porter devant Guéhazi par deux de ses serviteurs.
\VS{24}Et quand il fut arrivé dans un lieu secret, il les prit de leurs mains, et les déposa dans la maison, et il renvoya ces gens qui s'en allèrent.
\VS{25}Puis il entra et se présenta devant son maître. Elisée lui dit~: D'où viens-tu, Guéhazi~? Et il répondit~: Ton serviteur n'est allé nulle part.
\VS{26}Mais Elisée lui dit~: Mon cœur n'est-il pas allé là, lorsque cet homme a quitté son char pour venir à ta rencontre~? Est-ce le temps de prendre de l'argent, de prendre des vêtements, des oliviers, des vignes, du menu et du gros bétail, des serviteurs et des servantes~?
\VS{27}C'est pourquoi la lèpre de Naaman s'attachera à toi et à ta postérité à jamais. Et Guéhazi sortit de la présence d'Elisée avec une lèpre comme de la neige.
\Chap{6}
\TextTitle{Miracle du fer de hache}
\VerseOne{}Les fils des prophètes dirent à Elisée~: Voici, le lieu où nous sommes assis devant toi est trop étroit pour nous.
\VS{2}Allons jusqu'au Jourdain~; nous prendrons là chacun une poutre et nous y ferons un lieu d'habitation. Elisée répondit~: Allez~!
\VS{3}Et l'un d'eux dit~: Veuille, je te prie, venir avec tes serviteurs. Il répondit~: J'irai.
\VS{4}Il partit donc avec eux. Arrivés au Jourdain, ils coupèrent du bois.
\VS{5}Mais il arriva que comme l'un d'eux abattait une poutre, le fer de sa cognée tomba dans l'eau. Il s'écria et dit~: Ah~! Mon seigneur~! Je l'avais emprunté~!
\VS{6}L'homme de Dieu dit~: Où est-il tombé~? Et il lui montra l'endroit. Alors Elisée coupa un morceau de bois, le jeta au même endroit, et fit surnager le fer.
\VS{7}Et il dit~: Retire-le~! Et cet homme étendit sa main et le prit.
\TextTitle{Yahweh révèle à Elisée les plans militaires des Syriens}
\VS{8}Le roi de Syrie était en guerre avec Israël, et, dans un conseil qu'il tint avec ses serviteurs, il dit~: Mon camp sera dans un tel lieu.
\VS{9}L'homme de Dieu envoya dire au roi d'Israël~: Garde-toi de passer dans ce lieu, car les Syriens y descendent.
\VS{10}Et le roi d'Israël envoya des gens, pour s'y tenir en observation, vers le lieu que l'homme de Dieu lui avait mentionné et signalé. Et il y était sur ses gardes. Et cela n'arriva pas seulement une fois ni deux fois.
\VS{11}Le roi de Syrie en eut le cœur troublé~; et il appela ses serviteurs et leur dit~: Ne voulez-vous pas me déclarer lequel de vous est pour le roi d'Israël~?
\VS{12}Et l'un de ses serviteurs répondit~: Personne~! Ô roi, mon seigneur~! Mais Elisée, le prophète qui est en Israël, révèle au roi d'Israël les paroles même que tu déclares dans ta chambre à coucher.
\VS{13}Et il dit~: Allez et voyez où il est, et je le ferai prendre. On vint lui dire~: Voici, il est à Dothan.
\VS{14}Il envoya là des chevaux et des chars, et une grande armée, qui arrivèrent de nuit, et qui entourèrent la ville.
\TextTitle{L'armée de Yahweh plus grande que celle des Syriens}
\VS{15}Le serviteur de l'homme de Dieu se leva de grand matin et sortit~; et voici, une armée, entourait la ville, avec des chevaux et des chars. Le serviteur dit à l'homme de Dieu~: Ah~! Mon seigneur, comment ferons-nous~?
\VS{16}Il lui répondit~: Ne crains point, car ceux qui sont avec nous sont en plus grand nombre que ceux qui sont avec eux.
\VS{17}Elisée pria et dit~: Je te prie, ô Yahweh~! Ouvre ses yeux, afin qu'il voie. Et Yahweh ouvrit les yeux du serviteur et il vit. Et voici la montagne était pleine de chevaux et de chars de feu autour d'Elisée.
\TextTitle{Dieu aveugle les Syriens à la prière d'Elisée}
\VS{18}Les Syriens descendirent vers Elisée. Il adressa alors cette prière à Yahweh~: Je te prie, frappe ces gens d'aveuglement~! Et Dieu les frappa d'aveuglement, selon la parole d'Elisée.
\VS{19}Elisée leur dit~: Ce n'est pas ici le chemin, et ce n'est pas ici la ville~; suivez-moi et je vous conduirai vers l'homme que vous cherchez. Et il les conduisit à Samarie.
\VS{20}Et il arriva qu'aussitôt qu'ils furent entrés dans Samarie, Elisée dit~: Ô Yahweh ouvre leurs yeux afin qu'ils voient. Et Yahweh ouvrit leurs yeux et ils virent qu'ils étaient au milieu de Samarie.
\VS{21}Et dès que le roi d'Israël le vit, il dit à Elisée~: Frapperai-je, frapperai-je, mon père~?
\VS{22}Et Elisée répondit~: Tu ne frapperas point~; frapperais-tu de ton épée et de ton arc ceux que tu as fait prisonniers~? Sers-leur du pain et de l'eau afin qu'ils mangent et boivent~; et après cela, qu'ils s'en aillent vers leur maître.
\VS{23}Le roi d'Israël leur fit servir un grand repas et ils mangèrent et burent~; puis il les renvoya et ils s'en allèrent vers leur maître. Alors, les armées de Syrie ne revinrent plus au pays d'Israël.
\TextTitle{Siège des Syriens et famine en Samarie}
\VS{24}Et il arriva après cela que Ben-Hadad, roi de Syrie, rassembla toute son armée, monta et assiégea Samarie.
\VS{25}Il y eut une grande famine\FTNT{Cette histoire est riche en enseignements pour notre génération. Le siège de la Samarie par les étrangers, la famine qui frappait les Hébreux, le cannibalisme de certaines femmes, la cherté des produits alimentaires, la consommation d'excréments d'animaux à cause de la famine, sont des conséquences du péché. Aujourd'hui, beaucoup d'églises sont assiégées par les choses du monde, les démons, les fausses doctrines, etc.} dans Samarie~; ils l'assiégèrent tellement qu'une tête d'âne se vendait quatre-vingts pièces d'argent, et le quart d'un kab de fiente de pigeon cinq pièces d'argent.
\VS{26}Et comme le roi d'Israël passait sur la muraille, une femme lui cria~: Ô roi, mon seigneur~! Sauve-moi.
\VS{27}Il répondit~: Si Yahweh ne te sauve pas, comment pourrais-je te sauver~? Serait-ce avec le produit de l'aire ou de la cuve~?
\VS{28}Il lui dit encore~: Qu'as-tu~? Elle répondit~: Cette femme-là m'a dit~: Donne ton fils, et mangeons-le aujourd'hui, et nous mangerons mon fils demain\FTNT{Lé. 26:29~; De. 28:53-57.}.
\VS{29}Ainsi nous avons fait bouillir mon fils et l'avons mangé. Et le jour suivant, je lui ai dit~: Donne ton fils et nous le mangerons. Mais elle a caché son fils.
\VS{30}Dès que le roi entendit les paroles de cette femme, il déchira ses vêtements et passa sur la muraille. Le peuple vit qu'il avait en dessous un sac sur son corps.
\VS{31}C'est pourquoi le roi dit~: Que Dieu me traite dans toute sa rigueur, si aujourd'hui la tête d'Elisée, fils de Schaphath, reste sur lui.
\VS{32}Or Elisée était assis dans sa maison, et les anciens étaient assis avec lui. Le roi envoya un homme devant lui. Mais avant que le messager soit arrivé, Elisée dit aux anciens~: Ne voyez-vous pas que le fils de ce meurtrier envoie quelqu'un pour m'ôter la tête~? Lorsque le messager viendra, fermez la porte et repoussez-le avec la porte. N'entendez-vous pas le bruit des pas de son maître derrière lui~?
\VS{33}Et comme il parlait encore avec eux, voici le messager descendit vers lui et dit~: Voici, ce mal vient de Yahweh~; qu'ai-je à espérer encore de Yahweh~?
\Chap{7}
\TextTitle{Prophétie d'Elisée~; les lépreux dans le camp des Syriens}
\VerseOne{}Alors Elisée dit~: Ecoutez la parole de Yahweh~! Ainsi parle Yahweh~: Demain, à cette heure, on aura une mesure de fleur de farine pour un sicle, et deux mesures d'orge pour un sicle, à la porte de Samarie.
\VS{2}Mais l'officier sur la main duquel le roi s'appuyait répondit à l'homme de Dieu et dit~: Quand Yahweh ferait des fenêtres au ciel, cela arriverait-il~? Et Elisée dit~: Tu le verras de tes yeux, mais tu n'en mangeras pas.
\VS{3}Or il y avait à l'entrée de la porte quatre hommes lépreux\FTNT{Dieu s'est servi de ces quatre lépreux comme messagers de bonnes nouvelles. Le Seigneur utilise souvent les personnes rejetées et déconsidérées (1 Co. 1:26-31).}, et ils se dirent l'un à l'autre~: Pourquoi resterions-nous ici jusqu'à ce que nous mourions~?
\VS{4}Si nous pensons à entrer dans la ville, la famine est dans la ville et nous y mourrons~; et si nous restons ici, nous mourrons également. Allons-nous jeter dans le camp des Syriens~; s'ils nous laissent vivre, nous vivrons, et s'ils nous font mourir, nous mourrons.
\VS{5}Ils se levèrent donc au crépuscule pour entrer au camp des Syriens. Lorsqu'ils furent arrivés à l'extrémité du camp, voici, il n'y avait personne.
\VS{6}Car le Seigneur avait fait entendre dans le camp des Syriens un bruit de chars, et un bruit de chevaux, et un bruit d'une grande armée~; de sorte qu'ils s'étaient dit l'un à l'autre~: Voici, le roi d'Israël a payé les rois des Héthiens et les rois des Egyptiens pour venir contre nous.
\VS{7}C'est pourquoi ils s'étaient levés au crépuscule et s'étaient enfuis. Ils avaient abandonné leurs tentes, leurs chevaux, leurs ânes, et le camp tel qu'il était, et ils s'étaient enfuis pour sauver leur vie.
\VS{8}Les lépreux donc arrivèrent jusqu'à l'extrémité du camp. Ils entrèrent dans une tente, mangèrent, burent, emportèrent de l'argent, de l'or, des vêtements, et ils s'en allèrent et les cachèrent. Ils revinrent et entrèrent dans une autre tente et emportèrent de là aussi des objets, s'en allèrent et les cachèrent.
\VS{9}Alors ils se dirent l'un à l'autre~: Nous n'agissons pas bien~! Ce jour est un jour de bonnes nouvelles~; si nous gardons le silence et si nous attendons jusqu'à lumière du matin, le châtiment nous atteindra. Venez maintenant et allons informer la maison du roi.
\VS{10}Ils partirent et appelèrent les portiers de la ville, et leur racontèrent, en disant~: Nous sommes entrés dans le camp des Syriens, et voici, il n'y a personne. On n'y entend aucune voix d'homme~; il n'y a que des chevaux attachés, des ânes attachés et les tentes sont comme elles étaient.
\VS{11}Alors les portiers crièrent et transmirent ce rapport à la maison du roi.
\TextTitle{Accomplissement de la prophétie d'Elisée}
\VS{12}Le roi se leva de nuit et dit à ses serviteurs~: Je veux vous dire ce que les Syriens ont préparé contre nous. Ils savent que nous sommes affamés et ils sont sortis du camp pour se cacher dans les champs, disant~: Quand ils sortiront hors de la ville, nous les saisirons vivants et nous entrerons dans la ville.
\VS{13}L'un des serviteurs du roi répondit et dit~: Qu'on prenne cinq des chevaux qui restent encore dans la ville~; c'est presque tout ce qui est resté du grand nombre des chevaux d'Israël~; ils sont comme toute la multitude d'Israël, qui est consumée. Envoyons voir ce qui se passe.
\VS{14}Ils prirent donc deux chars avec les chevaux, et le roi envoya des messagers après l'armée des Syriens, en disant~: Allez et voyez.
\VS{15}Et ils allèrent après eux jusqu'au Jourdain~; et voici, le chemin était plein de vêtements et d'objets que les Syriens avaient jetés dans leur précipitation. Les messagers revinrent et le rapportèrent au roi.
\VS{16}Alors le peuple sortit et pilla le camp des Syriens, de sorte qu'il eut une mesure de fleur de farine pour un sicle, et deux mesures d'orge pour un sicle, selon la parole de Yahweh.
\VS{17}Le roi donna à l'officier, sur la main duquel il s'appuyait, la charge de garder la porte. Mais cet officier fut écrasé à la porte par le peuple et il en mourut selon la parole qu'avait prononcée l'homme de Dieu, quand le roi était descendu vers lui.
\VS{18}Car lorsque l'homme de Dieu avait parlé au roi, en disant~: Demain matin, à cette heure-ci, on donnera à la porte de Samarie deux mesures d'orge pour un sicle et une mesure de fleur de farine pour un sicle~;
\VS{19}cet officier avait répondu à l'homme de Dieu~: Quand Yahweh ferait des fenêtres au ciel, ce que tu dis pourrait-il arriver~? Et l'homme de Dieu avait dit~: Voici, tu le verras de tes yeux, mais tu n'en mangeras pas.
\VS{20}C'est en effet ce qui lui arriva~; car le peuple l'écrasa à la porte et il mourut.
\Chap{8}
\TextTitle{Elisée annonce une famine de sept ans}
\VerseOne{}Elisée parla à la femme dont il avait fait revivre le fils, en disant~: Lève-toi et va-t'en, toi et ta famille, et séjourne où tu pourras~; car Yahweh a appelé la famine, et même elle vient sur le pays pour sept ans.
\VS{2}La femme se leva et elle fit selon la parole de l'homme de Dieu. Elle s'en alla, elle et sa famille, et séjourna sept ans au pays des Philistins.
\TextTitle{La Sunamite retrouve ses terres}
\VS{3}Mais il arriva qu'au bout des sept ans, la femme revint du pays des Philistins, et alla implorer le roi au sujet de sa maison et de ses champs.
\VS{4}Le roi parlait à Guéhazi\FTNT{Voir 2 R. 5.}, serviteur de l'homme de Dieu, en disant~: Je te prie raconte-moi toutes les grandes choses qu'Elisée a faites.
\VS{5}Et il arriva que comme il racontait au roi comment Elisée avait rendu la vie à un mort, la femme dont Elisée avait fait revivre le fils vint implorer le roi au sujet de sa maison et de ses champs. Guéhazi dit~: Ô roi, mon seigneur, voici la femme et voici son fils, à qui Elisée a rendu la vie.
\VS{6}Alors le roi interrogea la femme, et elle lui raconta ce qui s'était passé. Le roi lui donna un eunuque, auquel il dit~: Fais restituer tout ce qui lui appartenait, même tous les revenus de ses champs, depuis le jour où elle a quitté le pays jusqu'à maintenant.
\TextTitle{Prophétie sur le règne d'Hazaël sur la Syrie}
\VS{7}Elisée se rendit à Damas. Ben-Hadad, roi de Syrie, était malade et on lui fit ce rapport~: L'homme de Dieu est venu ici.
\VS{8}Le roi dit à Hazaël~: Prends avec toi un présent et va au-devant de l'homme de Dieu, et consulte par lui Yahweh, en disant~: Guérirai-je de cette maladie~?
\VS{9}Et Hazaël s'en alla au-devant d'Elisée, ayant pris avec lui un présent, à savoir quarante chameaux chargés de tout ce qu'il y avait de meilleur à Damas. Il vint se présenter devant Elisée et dit~: Ton fils, Ben-Hadad, roi de Syrie, m'a envoyé vers toi, pour te dire~: Guérirai-je de cette maladie~?
\VS{10}Et Elisée lui répondit~: Va, dis-lui~: Tu guériras~! Tu guériras~! Toutefois, Yahweh m'a révélé qu'il mourra, qu'il mourra.
\VS{11}L'homme de Dieu arrêta son regard sur Hazaël et le fixa longtemps, puis il pleura.
\VS{12}Hazaël dit~: Pourquoi mon seigneur pleure-t-il~? Et il répondit~: Parce que je sais le mal que tu feras aux enfants d'Israël~; tu mettras le feu à leurs villes fortes, tu tueras avec l'épée leurs jeunes gens, tu écraseras leurs petits-enfants et tu fendras le ventre de leurs femmes enceintes.
\VS{13}Hazaël dit~: Mais qu'est-ce que ton serviteur, ce chien, pour faire de si grandes choses~? Et Elisée répondit~: Yahweh m'a révélé que tu seras roi de Syrie.
\VS{14}Alors Hazaël quitta Elisée et revint vers son maître, qui lui demanda~: Que t'a dit Elisée~? Et il répondit~: Il m'a dit que tu guériras~! Tu guériras~!
\VS{15}Mais le lendemain, Hazaël prit une couverture et l'ayant plongé dans l'eau, il l'étendit sur le visage de Ben-Hadad, qui mourut. Et Hazaël régna à sa place.
\TextTitle{Joram règne sur Juda\FTNTT{2 Ch. 21:1-7.}}
\VS{16}La cinquième année de Joram, fils d'Achab, roi d'Israël, Josaphat était encore roi de Juda et Joram, fils de Josaphat, roi de Juda, commença à régner sur Juda.
\VS{17}Il était âgé de trente-deux ans lorsqu'il commença à régner. Il régna huit ans à Jérusalem.
\VS{18}Il marcha dans la voie des rois d'Israël comme avait fait la maison d'Achab, car il avait pour femme la fille d'Achab\FTNT{Le mariage de Joram, fils de Josaphat, avec Athalie, fille d'Achab, était une grande erreur. Cette union qui était contractée dans le but de favoriser la paix entre les deux royaumes, entraîna le déclin de Juda~; or Dieu est contre les alliances contre nature. Voir Es. 30-31.}, et il fit ce qui est mal aux yeux de Yahweh.
\VS{19}Mais Yahweh ne voulut point détruire Juda, par amour pour David, son serviteur, selon la promesse qu'il lui avait faite de lui donner toujours une lampe parmi ses fils.
\TextTitle{Révoltes contre l'autorité de Juda}
\VS{20}De son temps, Edom se révolta contre l'autorité de Juda et se donna un roi.
\VS{21}Joram passa à Tsaïr, avec tous ses chars~; il se leva de nuit, et frappa les Edomites qui l'entouraient, et les chefs des chars, mais le peuple s'enfuit dans ses tentes.
\VS{22}Néanmoins, les Edomites ont été rebelles à Juda jusqu'à ce jour. En ce même temps, Libna aussi se révolta.
\TextTitle{Achazia règne sur Juda\FTNTT{2 Ch. 21:18-22:4.}}
\VS{23}Le reste des actions de Joram et tout ce qu'il a fait, cela n'est-il pas écrit dans le livre des Chroniques des rois de Juda~?
\VS{24}Joram se coucha avec ses pères et il fut enterré avec ses pères dans la cité de David. Et Achazia, son fils, régna à sa place.
\VS{25}La douzième année de Joram, fils d'Achab, roi d'Israël, Achazia, fils de Joram, roi de Juda, commença à régner.
\VS{26}Achazia était âgé de vingt-deux ans lorsqu'il commença à régner. Il régna un an à Jérusalem. Sa mère s'appelait Athalie, fille d'Omri, roi d'Israël.
\VS{27}Il marcha dans la voie de la maison d'Achab et il fit ce qui est mal aux yeux de Yahweh, comme avait fait la maison d'Achab, car il était gendre de la maison d'Achab.
\VS{28}Il alla avec Joram, fils d'Achab, à la guerre contre Hazaël, roi de Syrie, à Ramoth en Galaad. Et les Syriens blessèrent Joram.
\VS{29}Le roi Joram s'en retourna pour se faire guérir à Jizreel des blessures que les Syriens lui avaient faites à Rama, lorsqu'il se battait contre Hazaël, roi de Syrie. Achazia, fils de Joram, roi de Juda, descendit pour voir Joram, fils d'Achab, à Jizreel, parce qu'il était malade.
\Chap{9}
\TextTitle{Jéhu oint roi d'Israël}
\VerseOne{}Alors Elisée, le prophète, appela l'un des fils des prophètes et lui dit~: Ceins tes reins, prends cette fiole d'huile dans ta main, et va à Ramoth en Galaad.
\VS{2}Quand tu y seras entré, vois Jéhu, fils de Josaphat, fils de Nimschi. Tu iras le faire lever du milieu de ses frères et tu le conduiras dans une chambre secrète.
\VS{3}Tu prendras la fiole d'huile, tu la verseras sur sa tête et tu diras~: Ainsi parle Yahweh~: Je t'ai oint pour être roi sur Israël. Après quoi tu ouvriras la porte, tu t'enfuiras et tu ne t'arrêteras pas.
\VS{4}Le jeune homme, serviteur du prophète, s'en alla à Ramoth en Galaad.
\VS{5}Quand il arriva, voici, les chefs de l'armée étaient là assis. Il dit~: Chef, j'ai à te parler. Et Jéhu répondit~: Auquel de nous parles-tu~? Et il répondit~: A toi, chef.
\VS{6}Alors Jéhu se leva, et entra dans la maison, et le jeune homme répandit l'huile sur la tête, et lui dit~: Ainsi parle Yahweh, le Dieu d'Israël~: Je t'ai oint pour être roi sur Israël, le peuple de Yahweh.
\VS{7}Tu frapperas la maison d'Achab, ton maître, et je vengerai sur Jézabel\FTNT{1 R. 16:31~; 1 R. 17,18,19.} le sang de mes serviteurs les prophètes, et le sang de tous les serviteurs de Yahweh.
\VS{8}Et toute la maison d'Achab périra, et je retrancherai quiconque appartient à Achab, celui qui est esclave et celui qui est libre en Israël.
\VS{9}Je rendrai la maison d'Achab semblable à la maison de Jéroboam, fils de Nebath, et à la maison de Baescha, fils d'Achija.
\VS{10}Les chiens mangeront Jézabel dans le champ de Jizreel, et il n'y aura personne pour l'enterrer. Puis il ouvrit la porte et s'enfuit.
\VS{11}Jéhu sortit pour rejoindre les serviteurs de son maître et on lui dit~: Tout va bien~? Pourquoi ce fou est-il venu vers toi~? Jéhu leur répondit~: Vous connaissez l'homme et ses rêveries.
\VS{12}Mais ils répliquèrent~: Mensonge~! Réponds-nous donc. Et il dit~: Il m'a parlé de telle et telle manière, disant~: Ainsi parle Yahweh, je t'ai oint pour être roi sur Israël.
\VS{13}Alors ils se hâtèrent, et prirent chacun leurs vêtements, et les mirent sous lui au plus haut des degrés. Ils sonnèrent du shofar et dirent~: Jéhu a été fait roi~!
\TextTitle{Mort de Joram}
\VS{14}Ainsi Jéhu, fils de Josaphat, fils de Nimschi, forma une conspiration contre Joram. Or Joram et tout Israël défendaient Ramoth en Galaad contre Hazaël, roi de Syrie.
\VS{15}Le roi Joram s'en était retourné pour se faire guérir à Jizreel des blessures que les Syriens lui avaient faites, lorsqu'il se battait contre Hazaël, roi de Syrie. Jéhu dit~: Si vous le trouvez bon, que personne ne sorte ni ne s'échappe de la ville pour aller porter cette nouvelle à Jizreel.
\VS{16}Alors, Jéhu monta à cheval et s'en alla à Jizreel, car Joram était là, malade, et Achazia, roi de Juda, y était descendu pour le visiter.
\VS{17}Or il y avait une sentinelle sur une tour à Jizreel, qui voyant venir la troupe de Jéhu dit~: Je vois une troupe de gens. Et Joram dit~: Prends un cavalier et envoie-le à leur rencontre, et qu'il dise~: Est-ce la paix~?
\VS{18}Le cavalier s'en alla à sa rencontre, et dit~: Ainsi parle le roi~: Est-ce la paix~? Et Jéhu répondit~: Qu'as-tu à faire de la paix~? Mets-toi derrière moi. La sentinelle le rapporta, en disant~: Le messager est allé jusqu'à eux et il ne revient pas.
\VS{19}Joram envoya un second cavalier, qui arriva jusqu'à eux et dit~: Ainsi parle le roi~: Est-ce la paix~? Et Jéhu répondit~: Qu'as-tu à faire de la paix~? Mets-toi derrière moi.
\VS{20}La sentinelle le rapporta et dit~: Il est arrivé jusqu'à eux et il ne revient pas~; mais la manière de conduire le char est comme celle de Jéhu, fils de Nimschi~; car il le conduit avec furie.
\VS{21}Alors Joram dit~: Attelle~! Et on attela son char. Ainsi Joram, roi d'Israël, sortit avec Achazia, roi de Juda, chacun dans son char, et ils allèrent à la rencontre de Jéhu, et ils le trouvèrent dans le champ de Naboth de Jizreel\FTNT{1 R. 21.}.
\VS{22}Dès que Joram vit Jéhu, il dit~: Est-ce la paix, Jéhu~? Jéhu répondit~: Quelle paix~! Tant que durent les prostitutions de Jézabel, ta mère, et la multitude de ses enchantements~!
\VS{23}Alors Joram tourna sa main et s'enfuit, et il dit à Achazia~: Trahison, Achazia~!
\VS{24}Mais Jéhu saisit l'arc de sa main, et il frappa Joram entre ses épaules, de sorte que la flèche transperça son cœur, et il tomba sur ses genoux dans son char.
\VS{25}Jéhu dit à Bidkar, son officier~: Prends-le et jette-le dans le champ de Naboth de Jizreel~; car souviens-toi, lorsque nous étions à cheval moi et toi, ensemble, derrière Achab, son père, Yahweh prononça cette sentence contre lui~:
\VS{26}N'ai-je pas vu hier le sang de Naboth et le sang de ses fils, dit Yahweh~? Et je te le rendrai dans ce champ-ci, dit Yahweh~! C'est pourquoi prends-le donc, et jette-le dans ce champ, selon la parole de Yahweh.
\TextTitle{Mort d'Achazia\FTNTT{2 Ch. 22:7,9.}}
\VS{27}Achazia, roi de Juda, ayant vu cela, s'enfuit par le chemin de la maison du jardin~; mais Jéhu le poursuivit et dit~: Frappez-le sur le char~! Et on le frappa à la montée de Gur, près de Jibleam. Puis il se réfugia à Meguiddo, et il y mourut.
\VS{28}Ses serviteurs le transportèrent sur un char à Jérusalem, et ils l'enterrèrent dans son sépulcre avec ses pères, dans la cité de David.
\VS{29}Achazia avait commencé à régner sur Juda la onzième année de Joram, fils d'Achab.
\TextTitle{Mort de Jézabel}
\VS{30}Jéhu entra dans Jizreel. Jézabel, l'ayant appris, mit du fard à ses yeux, orna sa tête et regarda par la fenêtre.
\VS{31}Comme Jéhu franchissait la porte, elle dit~: Est-ce la paix, Zimri, assassin de son maître~?
\VS{32}Il leva sa tête vers la fenêtre et dit~: Qui est avec moi~? Qui~? Alors deux ou trois des eunuques regardèrent vers lui.
\VS{33}Et il leur dit~: Jetez-la en bas~! Et ils la jetèrent, de sorte qu'il rejaillit de son sang sur la muraille et sur les chevaux. Jéhu la foula aux pieds~;
\VS{34}puis il entra, mangea et but, et il dit~: Allez voir maintenant cette maudite et enterrez-la, car elle est fille de roi.
\VS{35}Ils allèrent donc pour l'enterrer~; mais ils ne trouvèrent d'elle que le crâne, les pieds et les paumes des mains.
\VS{36}Ils retournèrent l'annoncer à Jéhu, qui dit~: C'est la parole que Yahweh avait déclarée par son serviteur Elie\FTNT{1 R. 21:23.}, le Thischbite, en disant~: Dans le champ de Jizreel les chiens mangeront la chair de Jézabel~;
\VS{37}et le cadavre de Jézabel sera comme du fumier sur la face des champs, dans le champ de Jizreel, de sorte qu'on ne pourra dire~: C'est Jézabel.
\Chap{10}
\TextTitle{Accomplissement du jugement de Dieu sur la maison d'Achab}
\VerseOne{}Achab avait soixante-dix fils dans Samarie. Jéhu écrivit des lettres qu'il envoya à Samarie aux chefs de Jizreel, aux anciens et aux gouverneurs d'Achab. Il y était dit~:
\VS{2}Dès que cette lettre vous sera parvenue, puisque vous avez avec vous les fils de votre maître, avec vous les chars et les chevaux, la ville forte et les armes,
\VS{3}choisissez qui est le plus considérable et le plus sincère parmi les fils de votre maître, mettez-le sur le trône de son père et combattez pour la maison de votre maître.
\VS{4}Ils eurent une très grande peur et ils dirent~: Voici, deux rois n'ont point pu tenir contre lui, comment donc résisterions-nous~?
\VS{5}Et le chef de la maison, le chef de la ville, les anciens et les gouverneurs envoyèrent dire à Jéhu~: Nous sommes tes serviteurs, nous ferons tout ce que tu nous diras~; nous n'établirons personne roi, fais ce qui te semblera bon.
\VS{6}Jéhu leur écrivit une seconde lettre, où il était dit~: Si vous êtes pour moi et si vous obéissez à ma voix, prenez les têtes des fils de votre maître et venez auprès de moi demain à cette heure-ci, à Jizreel. Or les soixante-dix hommes, fils du roi, étaient avec les plus grands de la ville qui les élevaient.
\VS{7}Aussitôt que la lettre leur fut parvenue, ils prirent les fils du roi et ils égorgèrent ces soixante-dix hommes~; et ayant mis leurs têtes dans des corbeilles, ils les envoyèrent à Jéhu, à Jizreel.
\VS{8}Un messager vint l'en informer, en disant~: Ils ont apporté les têtes des fils du roi. Et il répondit~: Mettez-les en deux tas à l'entrée de la porte, jusqu'au matin.
\VS{9}Le matin, il sortit~; et se présentant à tout le peuple, il dit~: Vous êtes justes~! Voici, j'ai conspiré contre mon maître et je l'ai tué~; mais qui a frappé tous ceux-ci~?
\VS{10}Sachez maintenant qu'il ne tombera rien à terre de la parole de Yahweh\FTNT{1 R. 21:19-24.}, de la parole que Yahweh a prononcée contre la maison d'Achab~; Yahweh accomplit ce qu'il avait déclaré par son serviteur Elie.
\VS{11}Jéhu tua aussi tous ceux qui restaient de la maison d'Achab à Jizreel, tous ses grands, ses familiers et ses prêtres, sans en laisser échapper un seul.
\TextTitle{Mise à mort des frères d'Achazia et de la lignée d'Achab\FTNTT{2 Ch. 22:8.}}
\VS{12}Puis il se leva et partit pour aller à Samarie. Et comme il était près d'une maison de bergers sur le chemin,
\VS{13}Jéhu trouva les frères d'Achazia, roi de Juda, et leur dit~: Qui êtes-vous~? Ils répondirent~: Nous sommes les frères d'Achazia et nous sommes descendus pour saluer les fils du roi et les fils de la reine.
\VS{14}Jéhu dit~: Saisissez-les vivants. Ils les saisirent vivants et les égorgèrent, à savoir quarante-deux hommes, auprès du puits de la maison des bergers, sans en laisser échapper un seul.
\VS{15}Jéhu étant parti de là, il rencontra Jonadab, fils de Récab, qui venait au-devant de lui. Il le salua, et lui dit~: Ton cœur est-il aussi droit envers moi comme mon cœur l'est à ton égard~? Et Jonadab répondit~: Il l'est. Donne-moi ta main répliqua Jéhu. Et Jonadab lui donna sa main, et Jéhu le fit monter auprès de lui dans son char.
\VS{16}Puis il dit~: Viens avec moi et tu verras le zèle que j'ai pour Yahweh. Il l'emmena ainsi dans son char.
\VS{17}Et quand Jéhu fut arrivé à Samarie, il tua tous ceux qui restaient de la maison d'Achab à Samarie, et il les extermina entièrement, selon la parole que Yahweh avait dite à Elie.
\TextTitle{Mise à mort de tous les prophètes de Baal}
\VS{18}Puis Jéhu assembla tout le peuple, et leur dit~: Achab a peu servi Baal\FTNT{Jg. 2:13.}, mais Jéhu le servira beaucoup.
\VS{19}Maintenant donc, convoquez-moi tous les prophètes de Baal, tous ses serviteurs, et tous ses prêtres, sans qu'il en manque un seul, car je veux offrir un grand sacrifice à Baal~: Quiconque manquera ne vivra pas. Jéhu agissait avec ruse, pour faire périr les serviteurs de Baal.
\VS{20}Jéhu dit~: Publiez une fête solennelle en l'honneur de Baal. Et ils la publièrent.
\VS{21}Jéhu envoya des messagers dans tout Israël~; et tous les serviteurs de Baal arrivèrent, il n'y en eut pas un qui ne vînt~; et ils entrèrent dans le temple de Baal, qui fut rempli d'un bout à l'autre.
\VS{22}Alors Jéhu dit à celui qui avait la charge du vestiaire~: Sors des vêtements pour tous les serviteurs de Baal. Et cet homme sortit des vêtements.
\VS{23}Alors Jéhu, et Jonadab, fils de Récab, entrèrent dans le temple de Baal, et Jéhu dit aux serviteurs de Baal~: Cherchez et regardez afin qu'il n'y ait pas ici de serviteurs de Yahweh. Prenez garde qu'il n'y ait seulement que les serviteurs de Baal.
\VS{24}Ils entrèrent donc pour offrir des sacrifices et des holocaustes. Or Jéhu avait placé dehors quatre-vingts hommes, et leur avait dit~: Celui qui laissera échapper un de ces hommes que je remets entre vos mains, sa vie répondra de la sienne.
\VS{25}Et il arriva que dès qu'on eut achevé d'offrir l'holocauste, Jéhu dit aux gardes et aux officiers~: Entrez, tuez-les, et que nul n'échappe. Les gardes et les officiers les frappèrent du tranchant de l'épée, et les jetèrent là~; puis ils allèrent jusqu'à la ville du temple de Baal.
\VS{26}Ils tirèrent dehors les statues de la maison de Baal, et les brûlèrent.
\VS{27}Et ils démolirent la statue de Baal. Ils démolirent aussi la maison de Baal, et ils en firent un cloaque qui subsiste jusqu'à ce jour.
\VS{28}Ainsi Jéhu extermina Baal d'Israël.
\TextTitle{L'idolâtrie dans la vie de Jéhu}
\VS{29}Toutefois, Jéhu ne se détourna point des péchés que Jéroboam, fils de Nebath, avait fait commettre à Israël, à savoir les veaux d'or\FTNT{1 R. 12:28-29.} qui étaient à Béthel et à Dan.
\VS{30}Yahweh dit à Jéhu~: Parce que tu as fort bien exécuté ce qui était droit à mes yeux, et que tu as fait à la maison d'Achab tout ce qui était conforme à ma volonté, tes fils seront assis sur le trône d'Israël jusqu'à la quatrième génération.
\VS{31}Mais Jéhu ne prit point garde à marcher de tout son cœur dans la loi de Yahweh, le Dieu d'Israël~; il ne se détourna point des péchés que Jéroboam avait fait commettre à Israël.
\TextTitle{Hazaël règne sur la Syrie}
\VS{32}Dans ce temps-là, Yahweh commença à entamer le territoire d'Israël, et Hazaël battit les Israélites sur toutes les frontières.
\VS{33}Depuis le Jourdain, jusqu'au soleil levant, il battit tout le pays de Galaad, les Gadites, les Rubénites et ceux de Manassé, depuis Aroër sur le torrent de l'Arnon, jusqu'à Galaad et à Basan.
\TextTitle{Joachaz règne sur Israël}
\VS{34}Le reste des actions de Jéhu, tout ce qu'il a fait, et tous ses exploits, ne sont-ils pas écrits dans le livre des Chroniques des rois d'Israël~?
\VS{35}Jéhu se coucha avec ses pères, et on l'enterra à Samarie. Et Joachaz, son fils, régna à sa place.
\VS{36}Jéhu avait régné vingt-huit ans sur Israël à Samarie.
\Chap{11}
\TextTitle{Athalie fait périr la race royale de Juda\FTNTT{2 Ch. 22:9-12.}}
\VerseOne{}Athalie, mère d'Achazia, ayant vu que son fils était mort, se leva et extermina toute la race royale.
\VS{2}Mais Joschéba, fille du roi Joram, sœur d'Achazia, prit Joas, fils d'Achazia, et l'enleva du milieu des fils du roi, quand on les fit mourir~: Elle le mit avec sa nourrice dans la chambre des lits. Il fut ainsi dérobé aux regards d'Athalie, de sorte qu'on ne le fit point mourir.
\VS{3}Il resta caché six ans avec Joschéba dans la maison de Yahweh. Cependant Athalie régnait sur le pays.
\TextTitle{Joas devient roi de Juda\FTNTT{2 Ch. 23:1-11.}}
\VS{4}La septième année, Jehojada envoya chercher les chefs de centaines des Kéréthiens et des archers, et il les fit venir auprès de lui dans la maison de Yahweh. Il traita alliance avec eux, les fit jurer dans la maison de Yahweh, et leur montra le fils du roi.
\VS{5}Puis il leur donna cet ordre, en disant~: Voici ce que vous ferez. Parmi ceux d'entre vous qui entrent en service le jour du sabbat, un tiers doit monter la garde à la maison du roi,
\VS{6}un tiers sera à la porte de Sur, et un tiers à la porte derrière les archers~; ainsi vous veillerez à la garde de la maison, afin que personne n'y entre par force.
\VS{7}Vos deux autres compagnies, tous ceux qui sortent de service le jour du sabbat feront la garde de la maison de Yahweh, auprès du roi~:
\VS{8}Et vous entourerez le roi de toutes parts, chacun ayant ses armes à la main, et l'on mettra à mort quiconque s'avancera dans les rangs~; vous serez avec le roi quand il sortira et quand il entrera.
\VS{9}Les chefs de centaines firent donc tout ce que Jehojada, le prêtre, avait ordonné. Ils prirent chacun leurs gens, ceux qui entraient en service et ceux qui sortaient de service le jour du sabbat, et ils se rendirent vers le prêtre Jehojada.
\VS{10}Le prêtre donna aux chefs de centaine les lances et les boucliers qui provenaient du roi David, et qui étaient dans la maison de Yahweh.
\VS{11}Les archers, chacun les armes à la main, entourèrent le roi, en se plaçant depuis le côté droit de la maison, jusqu'au côté gauche, près de l'autel et près de la maison.
\VS{12}Jehojada fit amener le fils du roi, et il mit sur lui la couronne\FTNT{Couronne ou consacrer.} et le témoignage. Ils l'établirent roi et l'oignirent, et frappant des mains, ils dirent~: Vive le roi~!
\TextTitle{Mort d'Athalie\FTNTT{2 Ch. 23:12-15,21.}}
\VS{13}Athalie entendit le bruit des archers et du peuple, et elle vint vers le peuple à la maison de Yahweh.
\VS{14}Elle regarda. Et voici, le roi se tenait sur l'estrade, selon la coutume des rois. Les chefs et les trompettes étaient près du roi~: Tout le peuple du pays éclatait de joie, et on sonnait des trompettes. Alors Athalie déchira ses vêtements, et cria~: Conspiration~! Conspiration~!
\VS{15}Alors le prêtre Jehojada donna cet ordre aux chefs de centaines, qui avaient la charge de l'armée~: Faites-la sortir hors des rangs, et que celui qui la suivra soit mis à mort par l'épée. Car le prêtre avait dit~: Qu'elle ne soit pas mise à mort dans la maison de Yahweh~!
\VS{16}Ils lui firent donc place, et elle retourna dans la maison du roi par le chemin de l'entrée des chevaux~: C'est là qu'elle fut tuée.
\TextTitle{Alliance entre Jehojada, Yahweh et le peuple~; réveil sous le règne de Joas\FTNTT{2 Ch. 23:16-21.}}
\VS{17}Jehojada traita entre Yahweh, le roi et le peuple l'alliance par laquelle ils devaient être le peuple de Yahweh~; il traita aussi l'alliance entre le roi et le peuple.
\VS{18}Alors tout le peuple du pays entra dans la maison de Baal, et ils la démolirent avec ses autels~; et ils brisèrent entièrement ses images~; ils tuèrent aussi Matthan, prêtre de Baal, devant les autels. Le prêtre Jehojada établit des gardes dans la maison de Yahweh.
\VS{19}Il prit les chefs de centaines, les Kéréthiens et les archers, et tout le peuple du pays~; et ils firent descendre le roi de la maison de Yahweh, et ils entrèrent dans la maison du roi par le chemin de la porte des archers, et Joas s'assit sur le trône des rois.
\VS{20}Tout le peuple du pays fut dans la joie, et la ville fut en repos, après qu'on eût mis à mort Athalie par l'épée dans la maison du roi.
\VS{21}Joas était âgé de sept ans lorsqu'il commença à régner.
\Chap{12}
\TextTitle{Joas ordonne des réparations dans le temple\FTNTT{2 Ch. 24:2.}}
\VerseOne{}La septième année de Jéhu, Joas, commença à régner. Il régna quarante ans à Jérusalem. Sa mère s'appelait Tsibja, elle était de Beer-Schéba.
\VS{2}Joas fit ce qui est droit aux yeux de Yahweh pendant tout le temps qu'il suivit les instructions de Jehojada, le prêtre.
\VS{3}Toutefois, les hauts lieux ne disparurent point~; le peuple offrait encore des sacrifices et des parfums sur les hauts lieux.
\VS{4}Joas dit aux prêtres~: Tout l'argent consacré qu'on apporte dans la maison de Yahweh, l'argent ayant cours, à savoir l'argent pour l'évaluation des personnes d'après l'estimation qui en est faite, et tout l'argent que chacun apporte volontairement à la maison de Yahweh,
\VS{5}que les prêtres le prennent, chacun de la part des gens de sa connaissance, et qu'ils l'emploient à réparer ce qui est à réparer dans la maison, partout où l'on trouvera quelque chose à réparer.
\VS{6}Mais il arriva que, la vingt-troisième année du roi Joas, les prêtres n'avaient point encore réparé les brèches de la maison.
\VS{7}Le roi Joas appela le prêtre Jehojada et les autres prêtres, et il leur dit~: Pourquoi n'avez-vous pas réparé ce qui était à réparer dans la maison~? Maintenant, vous ne prendrez plus l'argent de vos connaissances, mais vous le livrerez pour les réparations de la maison.
\VS{8}Les prêtres convinrent de ne plus prendre l'argent du peuple et de ne pas être chargés des réparations de la maison.
\TextTitle{Offrandes volontaires pour réparer le temple\FTNTT{2 Ch. 24:8-14.}}
\VS{9}Alors le prêtre Jehojada prit un coffre, et le perça dans son couvercle, et le plaça à côté de l'autel, à droite, à l'endroit par lequel on entrait à la maison de Yahweh. Les prêtres qui avaient la garde du seuil y mettaient tout l'argent qu'on apportait à la maison de Yahweh.
\VS{10}Et dès qu'ils voyaient qu'il y avait beaucoup d'argent dans le coffre, le secrétaire du roi montait avec le grand-prêtre, et ils mettaient dans des sacs l'argent qui se trouvait dans la maison de Yahweh, puis ils le comptaient.
\VS{11}Ils remettaient cet argent bien compté entre les mains de ceux qui étaient chargés de faire exécuter l'ouvrage dans la maison de Yahweh. Et l'on employait cet argent pour les charpentiers et pour les architectes qui travaillaient à la maison de Yahweh,
\VS{12}pour les maçons et les tailleurs de pierres, pour acheter du bois et des pierres de taille, afin de réparer les brèches de la maison de Yahweh, et pour acheter tout ce qu'il fallait pour la réparation de la maison.
\VS{13}Mais, avec l'argent qu'on apportait dans la maison de Yahweh, on ne fit pour la maison de Yahweh ni bassins d'argent ni de couteaux, ni coupes, ni trompettes, ni aucun autre ustensile d'or, ou ustensile d'argent~;
\VS{14}on le distribuait à ceux qui avaient la charge de l'ouvrage et qui réparaient la maison de Yahweh.
\VS{15}On ne demandait pas de comptes aux hommes entre les mains desquels on remettait l'argent pour qu'ils le donnent à ceux qui faisaient l'ouvrage, car ils le faisaient fidèlement.
\VS{16}L'argent des sacrifices pour la culpabilité et l'argent des sacrifices pour les expiations n'était point apporté dans la maison de Yahweh~: Car il était pour les prêtres.
\TextTitle{Invasion syrienne évitée~; mort de Joas}
\VS{17}Alors Hazaël\FTNT{Hazaël envahit Juda à deux reprises. Ce passage fait mention de la première invasion~; la deuxième invasion est relatée en 2 Ch. 24:23.}, roi de Syrie, monta et fit la guerre à Gath, dont il s'empara. Hazaël avait l'intention de monter contre Jérusalem.
\VS{18}Mais Joas, roi de Juda, prit tout ce qui était consacré, que Josaphat, Joram, et Achazia, ses pères, rois de Juda, avaient consacré, tout ce que lui-même avait consacré, tout l'or qui se trouva dans les trésors de la maison de Yahweh et de la maison du roi~; et il envoya le tout à Hazaël, roi de Syrie, qui ne monta pas contre Jérusalem.
\VS{19}Le reste des actions de Joas, tout ce qu'il a fait, cela n'est-il pas écrit dans le livre des Chroniques des rois de Juda~?
\VS{20}Ses serviteurs se soulevèrent et se liguèrent~; ils frappèrent Joas dans la maison de Millo, qui est à la descente de Silla.
\VS{21}Jozacar, fils de Schimeath, et Jozabad fils de Schomer, ses serviteurs, le frappèrent, et il mourut. On l'enterra avec ses pères dans la cité de David. Et Amatsia, son fils, régna à sa place.
\Chap{13}
\TextTitle{Joachaz règne sur Israël}
\VerseOne{}La vingt-troisième année de Joas, fils d'Achazia, roi de Juda, Joachaz, fils de Jéhu, commença à régner sur Israël à Samarie. Il régna dix-sept ans.
\VS{2}Il fit ce qui est mal aux yeux de Yahweh~; car il suivit les péchés de Jéroboam, fils de Nebath, par lesquels il avait fait pécher Israël, et il ne s'en détourna point.
\TextTitle{L'idolâtrie perdure dans le pays}
\VS{3}La colère de Yahweh s'enflamma contre Israël, et il les livra entre les mains de Hazaël, roi de Syrie, et entre les mains de Ben-Hadad, fils de Hazaël, tout le temps que ces rois vécurent.
\VS{4}Mais Joachaz implora Yahweh. Et Yahweh l'exauça, parce qu'il vit l'oppression sous laquelle le roi de Syrie tenait Israël.
\VS{5}Yahweh donna donc un libérateur à Israël, et ils échappèrent aux mains des Syriens~; ainsi les enfants d'Israël habitèrent dans leurs tentes comme auparavant.
\VS{6}Mais ils ne se détournèrent point des péchés de la maison de Jéroboam, par lesquels il avait fait pécher Israël~; ils s'y livrèrent, et même l'idole d'Asherah\FTNT{Voir commentaire Jg. 2:13.} resta debout à Samarie.
\VS{7}De tout le peuple de Joachaz, Dieu ne lui avait laissé que cinquante cavaliers, dix chars, et dix mille hommes de pied~; car le roi de Syrie les avait fait périr et les avait rendus semblables à la poussière qu'on foule aux pieds.
\TextTitle{Mort de Joachaz~; Joas règne sur Israël}
\VS{8}Le reste des actions de Joachaz, tout ce qu'il a fait, et ses exploits, cela n'est-il pas écrit dans le livre des Chroniques des rois d'Israël~?
\VS{9}Ainsi Joachaz se coucha avec ses pères, et on l'ensevelit à Samarie. Et Joas, son fils, régna à sa place.
\VS{10}La trente-septième année de Joas, roi de Juda, Joas, fils de Joachaz, commença à régner sur Israël à Samarie. Il régna seize ans.
\VS{11}Et il fit ce qui est mal aux yeux de Yahweh~; il ne se détourna d'aucun des péchés de Jéroboam, fils de Nebath, par lesquels il avait fait pécher Israël, il s'y livra comme lui.
\TextTitle{Mort de Joas}
\VS{12}Le reste des actions de Joas, tout ce qu'il a fait, ses exploits, et la guerre qu'il eut avec Amatsia, roi de Juda, tout cela n'est-il pas écrit dans le livre des Chroniques des rois d'Israël~?
\VS{13}Joas se coucha avec ses pères, et Jéroboam s'assit sur son trône. Joas fut enterré à Samarie avec les rois d'Israël.
\TextTitle{Fin de la vie d'Elisée~; récit de la visite de Joas roi d'Israël}
\VS{14}Elisée était atteint de la maladie dont il mourut~; et Joas, roi d'Israël, descendit vers lui, pleura sur son visage, en disant~: Mon père~! Mon père~! Char d'Israël et sa cavalerie~!
\VS{15}Elisée lui dit~: Prends un arc et des flèches. Il prit donc un arc et des flèches.
\VS{16}Puis Elisée dit au roi d'Israël~: Bande l'arc avec ta main. Mets ta main sur l'arc. Et quand il y eut mis sa main, Elisée mit ses mains sur les mains du roi,
\VS{17}et il lui dit~: Ouvre la fenêtre à l'orient. Et il l'ouvrit. Elisée lui dit~: Tire. Après qu'il eut tiré, il lui dit~: C'est la flèche de la délivrance de la part de Yahweh, la flèche de la délivrance contre les Syriens~; tu frapperas les Syriens à Aphek, jusqu'à leur extermination.
\VS{18}Elisée lui dit encore~: Prends les flèches. Et il les prit. Elisée dit au roi d'Israël~: Frappe contre terre. Et le roi frappa trois fois, puis il s'arrêta.
\VS{19}Et l'homme de Dieu se mit dans une très grande colère contre lui, et lui dit~: Il fallait frapper cinq ou six fois~; alors tu aurais battu les Syriens jusqu'à leur extermination~; mais maintenant tu ne les frapperas que trois fois.
\TextTitle{Mort d'Elisée~; ses os rendent la vie à un mort}
\VS{20}Elisée mourut, et on l'ensevelit. L'année suivante, quelques troupes de Moabites entrèrent dans le pays.
\VS{21}Et comme on enterrait un homme, voici, on aperçut l'une des troupes de soldats, et l'on jeta l'homme dans le sépulcre d'Elisée. L'homme alla toucher les os d'Elisée, il reprit vie et se leva sur ses pieds.
\TextTitle{Fin de l'oppression syrienne}
\VS{22}Pendant toute la vie de Joachaz, Hazaël, roi de Syrie, avait opprimé Israël.
\VS{23}Mais Yahweh eut compassion d'eux, leur fit miséricorde, il tourna sa face vers eux par amour pour son alliance avec Abraham, Isaac et Jacob, de sorte qu'il ne voulut point les exterminer, et il ne les rejeta pas de sa face, jusqu'à maintenant.
\VS{24}Puis Hazaël, roi de Syrie, mourut, et Ben-Hadad, son fils, régna à sa place.
\VS{25}Joas, fils de Joachaz, reprit des mains de Ben-Hadad, fils d'Hazaël, les villes enlevées par Hazaël, à Joachaz, son père, pendant la guerre. Joas le battit trois fois et recouvra les villes d'Israël.
\Chap{14}
\TextTitle{Amatsia règne sur Juda\FTNTT{2 Ch. 25:1-4.}}
\VerseOne{}La deuxième année de Joas, fils de Joachaz, roi d'Israël, Amatsia, fils de Joas, roi de Juda, commença à régner.
\VS{2}Il était âgé de vingt-cinq ans lorsqu'il commença à régner, et il régna vingt-neuf ans à Jérusalem. Sa mère s'appelait Joaddan, elle était de Jérusalem.
\VS{3}Il fit ce qui est droit aux yeux de Yahweh, non pas toutefois comme David, son père~; il agit entièrement comme avait agi Joas, son père.
\VS{4}Seulement, les hauts lieux ne furent point ôtés~; le peuple offrait encore des sacrifices et des parfums sur les hauts lieux.
\VS{5}Et il arriva que dès que le royaume fut affermi entre ses mains, il frappa ses serviteurs qui avaient tué le roi, son père.
\VS{6}Mais il ne fit point mourir les fils des meurtriers, suivant ce qui est écrit dans le livre de la loi de Moïse, où Yahweh donne ce commandement~: On ne fera point mourir les pères pour les enfants, et l'on ne fera pas mourir les enfants pour les pères~; mais on fera mourir chacun pour son péché\FTNT{De. 24:16~; Ez. 18:4,20.}.
\VS{7}Il frappa dix mille hommes d'Edom dans la vallée du sel~; et il prit Séla durant la guerre, et l'appela Joktheel, nom qu'elle a conservé jusqu'à ce jour.
\VS{8}Alors Amatsia envoya des messagers vers Joas, fils de Joachaz, fils de Jéhu, roi d'Israël, pour lui dire~: Viens, voyons-nous en face~!
\VS{9}Et Joas, roi d'Israël, envoya dire à Amatsia, roi de Juda~: L'épine du Liban envoya dire au cèdre du Liban~: Donne ta fille en mariage à mon fils~! Et les bêtes sauvages qui sont au Liban passèrent et foulèrent l'épine.
\VS{10}Parce que tu as frappé et ravagé Edom, ton cœur s'est élevé. Contente-toi de ta gloire et reste dans ta maison. Pourquoi exciterais-tu le mal par lequel tu tomberas, toi et Juda avec toi~?
\VS{11}Mais Amatsia ne l'écouta pas. Et Joas, roi d'Israël, monta~: Et ils s'affrontèrent, lui et Amatsia, roi de Juda, à Beth-Schémesch, qui est à Juda.
\VS{12}Juda fut battu par Israël, et ils s'enfuirent chacun dans leurs tentes.
\VS{13}Joas, roi d'Israël, prit Amatsia, roi de Juda, fils de Joas, fils d'Achazia, à Beth-Schémesch. Puis il vint à Jérusalem et fit une brèche de quatre cents coudées dans la muraille de Jérusalem, depuis la porte d'Ephraïm, jusqu'à la porte de l'angle.
\VS{14}Il prit tout l'or et tout l'argent et tous les vases qui se trouvaient dans la maison de Yahweh et dans les trésors de la maison royale~; il prit aussi des enfants en otages, et il retourna à Samarie.
\TextTitle{Jéroboam II règne sur Israël}
\VS{15}Le reste des actions de Joas, ses exploits, et comment il combattit contre Amatsia, tout cela n'est-il pas écrit dans le livre des Chroniques des rois d'Israël~?
\VS{16}Et Joas se coucha avec ses pères et fut enseveli à Samarie avec les rois d'Israël. Et Jéroboam, son fils, régna à sa place.
\TextTitle{Mort d'Amatsia~; Azaria (Ozias) règne sur Juda (2 Ch. 25:26-28)}
\VS{17}Amatsia, fils de Joas, roi de Juda, vécut quinze ans après la mort de Joas, fils de Joachaz, roi d'Israël.
\VS{18}Le reste des actions d'Amatsia n'est-il pas écrit dans le livre des Chroniques des rois de Juda~?
\VS{19}On forma une conspiration contre lui à Jérusalem, et il s'enfuit à Lakis~; mais on le poursuivit à Lakis, où on le fit mourir.
\VS{20}On le transporta sur des chevaux, et il fut enseveli à Jérusalem avec ses pères, dans la cité de David.
\VS{21}Alors tout le peuple de Juda prit Azaria, âgé de seize ans, et ils l'établirent roi à la place d'Amatsia, son père.
\VS{22}Azaria bâtit Elath et la fit rentrer sous la puissance de Juda, après que le roi se coucha avec ses pères.
\TextTitle{Prophétie de Jonas accomplie par Jéroboam II}
\VS{23}La quinzième année d'Amatsia, fils de Joas, roi de Juda, Jéroboam, fils de Joas, commença à régner sur Israël à Samarie, et il régna quarante et un ans.
\VS{24}Il fit ce qui est mal aux yeux de Yahweh, et ne se détourna d'aucun des péchés de Jéroboam, fils de Nebath, par lesquels il avait fait pécher Israël.
\VS{25}Il rétablit les frontières d'Israël depuis l'entrée de Hamath, jusqu'à la mer de la plaine, selon la parole de Yahweh, le Dieu d'Israël, qu'il avait prononcée par son serviteur Jonas\FTNT{Jon. 1:1.}, fils d'Amitthaï, le prophète, de Gath-Hépher.
\VS{26}Car Yahweh vit que l'affliction d'Israël était à son comble, et l'extrémité à laquelle se trouvaient réduits esclaves et hommes libres, sans qu'il n'y ait personne pour venir au secours d'Israël.
\VS{27}Or Yahweh n'avait point résolu d'effacer le nom d'Israël de dessous les cieux, à cause de cela, il les délivra par les mains de Jéroboam, fils de Joas.
\TextTitle{Zacharie règne sur Israël}
\VS{28}Le reste des actions de Jéroboam, tout ce qu'il a fait, ses exploits de guerre, et comment il reconquit pour Israël, Damas et Hamath qui avaient appartenu à Juda, cela n'est-il pas écrit dans le livre des Chroniques des rois d'Israël~?
\VS{29}Puis Jéroboam se coucha avec ses pères, avec les rois d'Israël. Et Zacharie, son fils, régna à sa place.
\Chap{15}
\TextTitle{Juda demeure dans l'idolâtrie sous le règne d'Azaria (Ozias)\FTNTT{2 R. 14:21-22~; 2 Ch. 26:1-15.}}
\VerseOne{}La vingt-septième année de Jéroboam, roi d'Israël, Azaria\FTNT{Azaria (Ozias, selon 2 Ch. 26:1-15~; à ne pas confondre avec le prophète du même nom que son grand-père avait fait assassiner) fut couronné à l'âge de seize ans et mourut à l'âge de soixante-huit ans.}, fils d'Amatsia, roi de Juda, régna.
\VS{2}Il était âgé de seize ans lorsqu'il commença à régner, et il régna cinquante-deux ans à Jérusalem. Sa mère s'appelait Jecolia, elle était de Jérusalem.
\VS{3}Il fit ce qui est droit aux yeux de Yahweh, entièrement comme avait fait Amatsia, son père.
\VS{4}Seulement, les hauts lieux ne disparurent pas~; le peuple offrait encore des sacrifices et des parfums sur les hauts lieux.
\TextTitle{Jugement de Yahweh sur Ozias par la lèpre\FTNTT{2 Ch. 26:16-21.}}
\VS{5}Alors Yahweh frappa le roi, qui fut lépreux jusqu'au jour de sa mort, et il demeura dans une maison à l'écart. Et Jotham, fils du roi, avait la charge de la maison, jugeant le peuple du pays.
\VS{6}Le reste des actions d'Azaria, tout ce qu'il a fait, cela n'est-il pas écrit dans le livre des Chroniques des rois de Juda~?
\VS{7}Azaria se coucha avec ses pères, et fut enseveli avec ses pères dans la cité de David, et Jotham, son fils, régna à sa place.
\TextTitle{Conspiration de Schallum contre Zacharie, roi d'Israël}
\VS{8}La trente-huitième année d'Azaria, roi de Juda, Zacharie, fils de Jéroboam, commença à régner sur Israël à Samarie, et il régna six mois.
\VS{9}Il fit ce qui est mal aux yeux de Yahweh, comme avaient fait ses pères~; il ne se détourna point des péchés de Jéroboam, fils de Nebath, par lesquels il avait fait pécher Israël.
\VS{10}Schallum, fils de Jabesch, fit une conspiration contre lui, et le frappa devant le peuple. Il le tua, et régna à sa place.
\VS{11}Quant au reste des actions de Zacharie, voilà, elles sont écrites dans le livre des Chroniques des rois d'Israël.
\VS{12}Ainsi s'accomplit la parole que Yahweh avait déclarée à Jéhu, en disant~: Tes fils seront assis sur le trône d'Israël jusqu'à la quatrième génération, et il en fut ainsi\FTNT{2 R. 10:30.}
\TextTitle{Schallum règne sur Israël~; sa mort}
\VS{13}Schallum, fils de Jabesch, commença à régner la trente-neuvième année d'Ozias, roi de Juda. Il régna pendant un mois à Samarie.
\VS{14}Menahem, fils de Gadi, monta de Thirtsa et vint dans Samarie, et frappa à Samarie, Schallum, fils de Jabesch, et le fit mourir~; et il régna à sa place.
\VS{15}Le reste des actions de Schallum, et la conspiration qu'il forma, cela est écrit dans le livre des Chroniques des rois d'Israël.
\TextTitle{Menahem règne sur Israël}
\VS{16}Alors Menahem frappa Thiphsach et tous ceux qui y étaient, avec son territoire depuis Thirtsa~; il la frappa parce qu'elle ne lui avait point ouvert ses portes. Il fendit le ventre de toutes les femmes enceintes.
\VS{17}La trente-neuvième année d'Azaria, roi de Juda, Menahem, fils de Gadi, commença à régner sur Israël. Il régna dix ans à Samarie.
\VS{18}Il fit ce qui est mal aux yeux de Yahweh~; il ne se détourna point des péchés de Jéroboam, fils de Nebath, par lesquels il avait fait pécher Israël.
\TextTitle{Invasion d'Israël par le roi d'Assyrie\FTNTT{1 Ch. 5:26.}}
\VS{19}Alors Pul, roi d'Assyrie, vint contre le pays~; et Menahem donna mille talents d'argent à Pul, afin qu'il l'aide à affermir son royaume entre ses mains.
\VS{20}Menahem leva cet argent sur tous ceux d'Israël qui avaient de la richesse pour le donner au roi d'Assyrie~; chacun cinquante sicles d'argent. Ainsi, le roi d'Assyrie s'en retourna, et ne s'arrêta point dans le pays.
\VS{21}Le reste des actions de Menahem, tout ce qu'il a fait, cela n'est-il pas écrit dans le livre des Chroniques des rois d'Israël~?
\TextTitle{Mort de Menahem~; Pekachia règne sur Israël}
\VS{22}Menahem se coucha avec ses pères, et Pekachia, son fils, régna à sa place.
\VS{23}La cinquantième année d'Azaria, roi de Juda, Pekachia, fils de Menahem, commença à régner sur Israël à Samarie. Il régna deux ans.
\VS{24}Il fit ce qui est mal aux yeux de Yahweh~; il ne se détourna point des péchés de Jéroboam, fils de Nebath, par lesquels il avait fait pécher Israël.
\TextTitle{Pékach tue Pekachia et devient roi d'Israël}
\VS{25}Pékach, fils de Remalia, son officier, conspira contre lui~; il le frappa à Samarie, dans le palais de la maison royale, de même qu'Argob et Arié~; il avait avec lui cinquante hommes d'entre les fils des Galaadites. Il fit ainsi mourir Pekachia, et il régna à sa place.
\VS{26}Le reste des actions de Pekachia tout ce qu'il a fait, cela est écrit dans le livre des Chroniques des rois d'Israël.
\VS{27}La cinquante-deuxième année d'Azaria, roi de Juda, Pékach, fils de Remalia, commença à régner sur Israël à Samarie. Il régna vingt ans.
\VS{28}Il fit ce qui est mal aux yeux de Yahweh et ne se détourna point des péchés de Jéroboam, fils de Nebath, par lesquels il avait fait pécher Israël.
\VS{29}Du temps de Pékach, roi d'Israël, Tiglath-Piléser, roi d'Assyrie, vint et prit Ijjon, Abel-Beth-Maaca, Janoach, Kédesch, Hatsor, Galaad et la Galilée, et même tout le pays de Nephthali, et il emmena captifs les habitants en Assyrie.
\TextTitle{Osée conspire contre Pékach et règne sur Israël}
\VS{30}Osée, fils d'Ela, forma une conspiration contre Pékach, fils de Remalia, le frappa et le fit mourir. Il régna à sa place la vingtième année de Jotham, fils d'Ozias.
\VS{31}Le reste des actions de Pékach, tout ce qu'il a fait, cela est écrit dans le livre des Chroniques des rois d'Israël.
\TextTitle{Jotham règne sur Juda~; sa mort\FTNTT{2 R. 15:2~; 2 Ch. 26:23~; 27:1-9.}}
\VS{32}La seconde année de Pékach, fils de Remalia, roi d'Israël, Jotham, fils d'Ozias, roi de Juda, commença à régner.
\VS{33}Il était âgé de vingt-cinq ans lorsqu'il commença à régner. Il régna seize ans à Jérusalem. Sa mère s'appelait Jeruscha, fille de Tsadok.
\VS{34}Il fit ce qui est droit aux yeux de Yahweh~; il agit entièrement comme avait agi Ozias, son père.
\VS{35}Seulement, les hauts lieux ne disparurent point~; et le peuple offrait encore des sacrifices et des parfums sur les hauts lieux. Jotham bâtit la porte supérieure de la maison de Yahweh.
\VS{36}Le reste des actions de Jotham, tout ce qu'il a fait, cela n'est-il pas écrit dans le livre des Chroniques des rois de Juda~?
\VS{37}Dans ce temps-là, Yahweh commença à envoyer contre Juda, Retsin, roi de Syrie, et Pékach, fils de Remalia.
\VS{38}Jotham se coucha avec ses pères, et il fut enseveli dans la cité de David, son père. Et Achaz, son fils, régna à sa place.
\Chap{16}
\TextTitle{Achaz règne sur Juda\FTNTT{2 R. 15:38~; 2 Ch. 28:1-4.}}
\VerseOne{}La dix-septième année de Pékach, fils de Remalia, Achaz, fils de Jotham, roi de Juda, commença à régner.
\VS{2}Achaz était âgé de vingt ans lorsqu'il commença à régner. Il régna seize ans à Jérusalem. Il ne fit point ce qui est droit aux yeux de Yahweh, son Dieu, comme avait fait David, son père.
\VS{3}Mais il suivit la voie des rois d'Israël et il fit même passer son fils par le feu, selon les abominations des nations que Yahweh avait chassées devant les enfants d'Israël.
\VS{4}Il offrait aussi des sacrifices et des parfums sur les hauts lieux, sur les coteaux et sous tout arbre vert.
\TextTitle{Juda envahi par les rois d'Assyrie et d'Israël\FTNTT{2 Ch. 28:5-19.}}
\VS{5}Alors Retsin, roi de Syrie, et Pékach, fils de Remalia, roi d'Israël, montèrent contre Jérusalem pour lui faire la guerre. Ils assiégèrent Achaz~; mais ne purent en venir à bout par les armes.
\VS{6}Dans ce même temps, Retsin, roi de Syrie, fit rentrer Elath au pouvoir des Syriens~; il expulsa les Juifs d'Elath, et les Syriens vinrent à Elath, où ils ont demeuré jusqu'à ce jour.
\TextTitle{Le roi d'Assyrie vient en aide à Achaz et s'empare de Damas\FTNTT{2 Ch. 28:16-25.}}
\VS{7}Achaz envoya des messagers à Tiglath-Piléser, roi d'Assyrie, pour lui dire~: Je suis ton serviteur et ton fils~; monte et délivre-moi de la main du roi des Syriens, et de la main du roi d'Israël, qui s'élèvent contre moi.
\VS{8}Alors Achaz prit l'argent et l'or qui se trouvaient dans la maison de Yahweh, et dans les trésors de la maison royale, et il les envoya en présent au roi d'Assyrie.
\VS{9}Le roi d'Assyrie l'écouta~; il monta contre Damas, la prit, emmena les habitants en captivité à Kir et fit mourir Retsin.
\VS{10}Alors le roi Achaz s'en alla à la rencontre de Tiglath-Piléser, roi d'Assyrie, à Damas. Et ayant vu l'autel\FTNT{Achaz, roi de Juda, se rendit chez le roi d'Assyrie et il fut fasciné par l'autel de son dieu au point de le convoiter. Il demanda au prêtre Urie de fabriquer un autel identique, dont le modèle n'était pas celui que Yahweh avait décrit à Moïse. Il introduisit un objet de culte d'origine païenne dans le temple de Jérusalem, sous prétexte d'honorer Yahweh. Certains «~Pères de l'Eglise~», comme les empereurs Constantin I (285-337) et Théodose I (347-395), se sont comportés exactement comme Achaz en adoptant les pratiques païennes. Les historiens s'accordent pour dire que la diffusion de la Parole de Dieu sous l'empire de Constantin I (285-337), empereur de Rome, avait des fins strictement politiques. Cette politique a eu deux conséquences essentielles concernant l'influence de l'Eglise chrétienne et son fonctionnement de plus en plus éloigné de la Parole de Dieu~:\\- Les peuples païens ont introduit leurs rites idolâtres au sein de l'Eglise. En effet, les dogmes de l'institution devaient plaire à la majorité.\\- L'Eglise chrétienne cessant d'être persécutée, son fonctionnement intimiste fondé sur l'implication de chaque croyant et l'exercice de la prêtrise universelle des chrétiens, a changé à cause de l'effet de masse. Devenant numériquement très importante, il a fallu imposer une autorité capable de contenir un nombre de fidèles de plus en plus élevé. Mais à cause de cette augmentation numérique et de la présence de «~faux convertis~» lié au fait que l'adhésion au christianisme (religion chrétienne fondée par les hommes) devenait une obligation, l'étude de la Parole, la fraction du pain et la prière ne pouvaient plus perdurer. C'est ainsi que beaucoup d'églises ont commencé à subir l'influence du monde.} qui était à Damas, le roi Achaz envoya au prêtre Urie, la forme et le modèle exact de cet autel.
\VS{11}Le prêtre Urie construisit un autel entièrement d'après le modèle envoyé de Damas par le roi Achaz, et le prêtre Urie le fit avant que le roi Achaz soit de retour de Damas.
\VS{12}Quand le roi Achaz revint de Damas et vit l'autel, il s'en approcha et y monta~;
\VS{13}Il fit brûler son holocauste et son sacrifice, versa ses libations et répandit sur l'autel le sang de ses sacrifices d'offrande de paix\FTNT{Voir commentaire en Lé. 3:1.}.
\VS{14}Il éloigna de la face de la maison l'autel d'airain qui était devant Yahweh, afin qu'il ne soit pas entre le nouvel autel et la maison de Yahweh~; et il le plaça à côté du nouvel autel, vers le nord.
\VS{15}Et le roi Achaz donna cet ordre au prêtre Urie~: Fais brûler l'holocauste du matin et l'offrande du soir, l'holocauste du roi et son offrande, les holocaustes de tout le peuple du pays et leurs offrandes, verses-y leurs libations, et répands-y tout le sang des holocaustes et tout le sang des sacrifices~; mais pour ce qui concerne l'autel d'airain, je m'en occuperai.
\VS{16}Le prêtre Urie, exécuta tout ce que le roi Achaz lui avait ordonné.
\VS{17}Le roi Achaz brisa les panneaux des bases et en ôta les cuves qui étaient dessus. Il descendit la mer de dessus les bœufs d'airain qui étaient sous elle et il la posa sur un pavé de pierre.
\VS{18}Il changea aussi dans la maison de Yahweh, à cause du roi d'Assyrie, le portique du sabbat qu'on y avait bâti et l'entrée extérieure du roi.
\TextTitle{Mort d'Achaz~; Ezéchias devient roi de Juda\FTNTT{2 Ch. 28:26-27.}}
\VS{19}Le reste des actions d'Achaz et tout ce qu'il a fait, cela n'est-il pas écrit dans le livre des Chroniques des rois de Juda~?
\VS{20}Achaz se coucha avec ses pères, et il fut enseveli avec ses pères dans la cité de David. Et Ezéchias, son fils, régna à sa place.
\Chap{17}
\TextTitle{Osée devient le dernier roi d'Israël}
\VerseOne{}La douzième année d'Achaz, roi de Juda, Osée, fils d'Ela, régna à Samarie sur Israël. Il régna neuf ans.
\VS{2}Il fit ce qui est mal aux yeux de Yahweh, non pas toutefois comme les rois d'Israël qui avaient été avant lui.
\TextTitle{Osée tente de s'affranchir du joug de l'Assyrie}
\VS{3}Salmanasar\FTNT{Le royaume d'Israël a été détruit en 722 av. J.-C., par l'empereur assyrien Salmanasar V (règne~: 727-722 av. J.-C.), après avoir assiégé trois ans le roi Osée (règne~: 732-722 av. J.-C.) dans sa capitale Samarie. Celui-ci ne payait plus le tribut et essayait d'obtenir l'appui de l'Egypte pour retrouver l'indépendance. Le royaume d'Israël a disparu au début du 8ème siècle av. J.-C., provoquant la dispersion dans le monde de plusieurs juifs issus des dix tribus. L'origine des Samaritains remonte à cette déportation, après que le royaume du Nord soit tombé aux mains de Salmanasar, roi d'Assyrie. Malgré les déportations, les Assyriens n'avaient pas laissé déserte cette région appelée «~Samarie~»~; plusieurs Israélites y étaient restés et des colons d'autres provinces assyriennes vinrent s'y établir. Les Samaritains sont issus du mélange de ces populations, et leur religion est un mélange entre le culte à Yahweh avec celui des dieux étrangers.}, roi d'Assyrie, monta contre lui~; et Osée lui fut assujetti et lui paya un tribut.
\VS{4}Mais le roi d'Assyrie découvrit une conspiration chez Osée, qui avait envoyé des messagers vers So, roi d'Egypte, et qui ne payait plus le tribut tous les ans au roi d'Assyrie. C'est pourquoi le roi d'Assyrie le fit enfermer et enchaîner dans une prison.
\TextTitle{Siège de Samarie par le roi d'Assyrie}
\VS{5}Le roi d'Assyrie parcourut tout le pays et monta contre Samarie qu'il assiégea pendant trois ans.
\TextTitle{Les causes de la captivité d’Israël par l'Assyrie}
\VS{6}La neuvième année d'Osée, le roi d'Assyrie prit Samarie et emmena captifs les Israélites en Assyrie. Il les fit habiter à Chalach, et sur le Chabor, fleuve de Gozan, et dans les villes des Mèdes.
\VS{7}Cela arriva parce que les enfants d'Israël péchèrent contre Yahweh, leur Dieu, qui les avait fait monter hors du pays d'Egypte, de dessous la main de Pharaon, roi d'Egypte, et parce qu'ils craignirent d'autres dieux.
\VS{8}Ils suivirent les coutumes des nations que Yahweh avait chassées devant les enfants d'Israël, et celles des rois d'Israël qu'ils avaient établis.
\VS{9}Les enfants d'Israël firent en secret des choses qui n'étaient point droites, contre Yahweh, leur Dieu. Ils se bâtirent des hauts lieux dans toutes leurs villes, depuis la tour des gardes jusqu'aux villes fortes.
\VS{10}Ils se dressèrent des statues et des Asherah sur toutes les hautes collines et sous tout arbre vert.
\VS{11}Et là, ils brûlèrent des parfums sur tous les hauts lieux, comme les nations que Yahweh avait chassées devant eux, et ils firent des choses mauvaises pour irriter Yahweh.
\VS{12}Ils servirent les idoles, au sujet desquelles Yahweh leur avait dit~: Vous ne ferez pas cela\FTNT{1 R. 12:28.}.
\VS{13}Yahweh fit avertir Israël et Juda par tous ses prophètes, tous les voyants, en disant~: Détournez-vous de toutes vos mauvaises voies, revenez, et gardez mes commandements et mes ordonnances, en suivant entièrement la loi que j'ai prescrite à vos pères et que je vous ai envoyée par mes serviteurs les prophètes.
\VS{14}Mais ils n'écoutèrent point et raidirent leur cou, comme leurs pères avaient raidi leur cou, et n'avaient pas cru en Yahweh, leur Dieu.
\VS{15}Ils rejetèrent ses lois, et son alliance qu'il avait traitée avec leurs pères, et ses avertissements, qu'il leur avait adressés. Ils allèrent après des choses de néant et ne furent eux-mêmes que néant, après les nations qui les entouraient et que Yahweh leur avait défendu d'imiter.
\VS{16}Ils abandonnèrent tous les commandements de Yahweh, leur Dieu, ils firent deux veaux en métal fondu, ils fabriquèrent des idoles d'Asherah\FTNT{Voir commentaire en Jg. 2:13.}, ils se prosternèrent devant toute l'armée des cieux et ils servirent Baal.
\VS{17}Ils firent aussi passer leurs fils et leurs filles par le feu, ils s'adonnèrent à la divination et aux enchantements, et ils se vendirent pour faire ce qui est mal aux yeux de Yahweh afin de l'irriter.
\VS{18}C'est pourquoi, Yahweh fut très irrité contre Israël et il les rejeta~; il n'est resté que la seule tribu de Juda.
\VS{19}Même Juda n'avait pas gardé les commandements de Yahweh, son Dieu, mais ils avaient suivi les ordonnances qu'Israël avait établies.
\VS{20}C'est pourquoi Yahweh rejeta toute la race d'Israël~; il les a humiliés, il les a livrés entre les mains des pillards, il a fini par les chasser loin de sa face.
\VS{21}Car Israël s'était détaché de la maison de David, et avait établi roi Jéroboam, fils de Nebath. Jéroboam avait détourné Israël de Yahweh, afin qu'il ne le suive plus, et lui avait fait commettre un grand péché.
\VS{22}C'est pourquoi les enfants d'Israël s'étaient livrés à tous les péchés que Jéroboam avait commis~; ils ne s'en détournèrent pas,
\VS{23}jusqu'à ce que Yahweh ait chassé Israël de devant sa face, comme il l'avait annoncé par tous ses serviteurs les prophètes. Et Israël fut emmené captif loin de son pays en Assyrie, jusqu'à ce jour.
\TextTitle{Jugement sur les étrangers occupant les villes d'Israël}
\VS{24}Le roi d'Assyrie fit venir des gens de Babylone, de Cutha, d'Avva, de Hamath et de Sepharvaïm. Il les fit habiter dans les villes de Samarie, à la place des enfants d'Israël. Ils prirent possession de la Samarie et habitèrent dans ses villes.
\VS{25}Lorsqu'ils commencèrent à y habiter, ils ne craignirent point Yahweh, et Yahweh envoya contre eux des lions, qui les tuaient.
\VS{26}Et on dit au roi d'Assyrie~: Les nations que tu as transportées et fait habiter dans les villes de Samarie ne connaissent pas la manière de servir le dieu du pays, c'est pourquoi il a envoyé contre eux des lions, et voilà, ces lions les tuent, parce qu'ils ne connaissent pas la manière de servir le dieu du pays.
\TextTitle{L'idolâtrie dans les villes occupées}
\VS{27}Alors le roi d'Assyrie donna cet ordre, en disant~: Faites-y aller quelqu'un des prêtres que vous avez emmenés de là en captivité~; qu'il parte pour s'y établir et qu'il leur enseigne la manière de servir le dieu du pays.
\VS{28}Alors l'un des prêtres, qui avaient été emmenés captifs de Samarie vint s'établir à Béthel et leur enseigna comment ils devaient craindre Yahweh.
\VS{29}Mais les nations firent chacune leurs dieux dans les villes qu'elles habitaient et les placèrent dans les maisons des hauts lieux bâties par les Samaritains.
\VS{30}Les gens de Babylone firent Succoth-Benoth, les gens de Cuth firent Nergal, et les gens de Hamath firent Aschima.
\VS{31}Ceux d'Avva firent Nibchaz et Thartak~; ceux de Sepharvaïm brûlaient leurs enfants par le feu à Adrammélec et Anammélec, les dieux de Sepharvaïm.
\VS{32}Toutefois, ils redoutaient Yahweh et ils établirent des prêtres des hauts lieux pris parmi tout le peuple~; ces prêtres offraient pour eux des sacrifices dans les maisons des hauts lieux.
\VS{33}Ils redoutaient Yahweh et en même temps, ils servaient leurs dieux à la manière des nations d'où on les avait transportés.
\VS{34}Et jusqu'à ce jour, ils font encore selon leurs premières coutumes~: Ils ne craignirent point Yahweh, et ils ne se conforment ni à leurs lois et à leurs ordonnances, ni à la loi et aux commandements prescrits par Yahweh Dieu aux enfants de Jacob, qu'il appela du nom d'Israël.
\VS{35}Yahweh avait traité alliance avec eux et leur avait donné cet ordre, en disant~: Vous ne craindrez point d'autres dieux~; et ne vous prosternerez point devant eux~; vous ne les servirez point et vous ne leur offrirez point de sacrifices.
\VS{36}Mais vous craindrez Yahweh, qui vous a fait monter hors du pays d'Egypte avec une grande puissance et à bras étendu~; et vous vous prosternerez devant lui, et vous lui offrirez des sacrifices.
\VS{37}Vous observerez et mettrez toujours en pratique les statuts, les ordonnances, la loi et les commandements, qu'il a écrits pour vous, et vous ne craindrez pas d'autres dieux.
\VS{38}Vous n'oublierez pas l'alliance que j'ai traitée avec vous et vous ne craindrez point d'autres dieux.
\VS{39}Mais vous craindrez Yahweh, votre Dieu, et il vous délivrera de la main de tous vos ennemis.
\VS{40}Ils n'écoutèrent pas et ils firent selon leurs premières coutumes.
\VS{41}Ainsi ces nations-là redoutaient Yahweh et servaient leurs images~; leurs enfants et les enfants de leurs enfants font jusqu'à ce jour ce que firent leurs pères.
\Chap{18}
\TextTitle{Ezéchias règne sur Juda\FTNTT{2 R. 16:20~; 2 Ch. 29:1-31:21.}}
\VerseOne{}La troisième année d'Osée, fils d'Ela, roi d'Israël, Ezéchias, fils d'Achaz, roi de Juda, commença à régner.
\VS{2}Il était âgé de vingt-cinq ans lorsqu'il commença à régner, il régna vingt-neuf ans à Jérusalem. Sa mère s'appelait Abi, fille de Zacharie.
\VS{3}Il fit ce qui est droit aux yeux de Yahweh, entièrement comme avait fait David, son père.
\TextTitle{Mouvement de réveil sous Ezéchias\FTNTT{2 Ch. 29:3-31:21.}}
\VS{4}Il fit disparaître les hauts lieux, mit en pièces les statues, abattit les idoles d'Asherah, et il brisa le serpent d'airain que Moïse avait fait, car les enfants d'Israël avaient jusqu'alors brûlé des parfums devant lui~; ils l'appelaient Nehuschtan.
\VS{5}Il se confia en Yahweh, le Dieu d'Israël~; et parmi tous les rois de Juda qui vinrent après ou qui le précédèrent, il n'y en eut point de semblable à lui.
\VS{6}Il s'attacha à Yahweh, il ne se détourna point de lui et il observa les commandements que Yahweh avait prescrits à Moïse.
\TextTitle{Révolte contre l'Assyrie~; victoire sur les Philistins}
\VS{7}Et Yahweh fut avec Ezéchias, qui réussit dans toutes ses entreprises. Il se révolta contre le roi d'Assyrie et ne lui fut plus assujetti.
\VS{8}Il frappa les Philistins jusqu'à Gaza et ravagea leur territoire depuis les tours des gardes jusqu'aux villes fortes.
\TextTitle{Captivité d'Israël par l'Assyrie\FTNTT{2 R. 17:4-6.}}
\VS{9}La quatrième année du roi Ezéchias, qui était la septième du règne d'Osée, fils d'Ela, roi d'Israël, Salmanasa, roi d'Assyrie, monta contre Samarie et l'assiégea.
\VS{10}Il la prit au bout de trois ans~; la sixième année du règne d'Ezéchias, qui était la neuvième d'Osée, roi d'Israël, Samarie fut prise.
\VS{11}Le roi d'Assyrie emmena Israël en Assyrie et il les établit à Chalach, sur le Chabor, fleuve de Gozan, et dans les villes des Mèdes,
\VS{12}parce qu'ils n'avaient point obéi à la voix de Yahweh, leur Dieu, et qu'ils avaient transgressé son alliance, parce qu'ils n'avaient ni écouté ni mis en pratique tout ce qu'avait ordonné Moïse, serviteur de Yahweh.
\TextTitle{Invasion de Juda par Sanchérib\FTNTT{2 Ch. 32:1-15,30~; Es. 36:1-10.}}
\VS{13}La quatorzième année du roi Ezéchias, Sanchérib, roi d'Assyrie, monta contre toutes les villes fortes de Juda et les prit.
\VS{14}Ezéchias, roi de Juda, envoya dire au roi d'Assyrie à Lakis~: J'ai commis une faute~! Eloigne-toi de moi. Je payerai tout ce que tu m'imposeras. Et le roi d'Assyrie imposa à Ezéchias, roi de Juda, trois cents talents d'argent et trente talents d'or.
\VS{15}Ezéchias donna tout l'argent qui se trouvait dans la maison de Yahweh et dans les trésors de la maison royale.
\VS{16}En ce temps-là, Ezéchias enleva les lames d'or dont il avait couvert les portes et les linteaux du temple de Yahweh, pour les livrer au roi d'Assyrie.
\VS{17}Puis le roi d'Assyrie envoya de Lakis à Jérusalem, vers le roi Ezéchias, Tharthan, Rab-Saris et Rabschaké avec une puissante armée. Ils montèrent et arrivèrent à Jérusalem. Lorsqu'ils furent montés et arrivés~; ils s'arrêtèrent à l'aqueduc de l'étang supérieur, qui est sur le chemin du champ du foulon.
\VS{18}Ils appelèrent le roi tout haut~; alors Eliakim, fils de Hilkija, chef de la maison du roi, Schebna, le secrétaire et Joach, fils d'Asaph, l'archiviste, se rendirent auprès d'eux.
\VS{19}Rabschaké leur dit~: Dites maintenant à Ezéchias~: Ainsi parle le grand roi, le roi d'Assyrie~: Quelle est cette confiance sur laquelle tu t'appuies~?
\VS{20}Tu as dit~: Il faut pour la guerre le conseil et la force. Mais ce ne sont que des paroles. Mais en qui donc as-tu placé ta confiance, pour te rebeller contre moi~?
\VS{21}Voici maintenant, tu l'as placée dans l'Egypte, dans ce roseau cassé, qui pénètre et perce la main de quiconque s'appuie dessus~: Tel est Pharaon, roi d'Egypte, pour tous ceux qui se confient en lui.
\VS{22}Peut-être me direz-vous~: Nous nous confions en Yahweh, notre Dieu, mais n'est-ce pas celui dont Ezéchias a détruit les hauts lieux et les autels, en disant à Juda et à Jérusalem~: Vous vous prosternerez devant cet autel à Jérusalem~?
\VS{23}Maintenant, donne des otages au roi d'Assyrie, mon maître, et je te donnerai deux mille chevaux, si tu peux donner autant de cavaliers pour les monter.
\VS{24}Comment donc repousserais-tu un seul gouverneur d'entre les serviteurs de mon maître~? Mais tu mets ta confiance dans l'Egypte, à cause des chars et des cavaliers.
\VS{25}D'ailleurs, est-ce sans l'ordre de Yahweh que je suis monté contre ce lieu, pour le détruire~? Yahweh m'a dit~: Monte contre ce pays et détruis-le.
\TextTitle{Menaces de Rabschaké\FTNTT{2 Ch. 32:16,18-19~; Es. 36:11-21.}}
\VS{26}Alors Eliakim, fils de Hilkija, Schebna et Joach dirent à Rabschaké~: Nous te prions de parler en araméen à tes serviteurs, car nous le comprenons~; et ne nous parle pas en langue judaïque, aux oreilles du peuple qui est sur la muraille.
\VS{27}Rabschaké leur répondit~: Est-ce à ton maître et à toi que mon maître m'a envoyé dire ces paroles~? Ne m'a-t-il pas envoyé vers les hommes qui se tiennent sur la muraille pour leur dire qu'ils mangeront leurs propres excréments et qu'ils boiront leur urine avec vous~?
\VS{28}Rabschaké, s'étant avancé, cria à haute voix en langue judaïque, il parla et dit~: Ecoutez la parole du grand roi, le roi d'Assyrie.
\VS{29}Ainsi parle le roi~: Qu'Ezéchias ne vous trompe pas, car il ne pourra pas vous délivrer de ma main.
\VS{30}Qu'Ezéchias ne vous amène pas à vous confier en Yahweh, en disant~: Yahweh nous délivrera certainement et cette ville ne sera pas livrée entre les mains du roi d'Assyrie.
\VS{31}N'écoutez pas Ezéchias~; car ainsi parle le roi d'Assyrie~: Faites la paix avec moi et rendez-vous à moi~; et chacun de vous mangera de sa vigne, et de son figuier, et chacun boira de l'eau de sa citerne,
\VS{32}jusqu'à ce que je vienne et que je vous emmène dans un pays comme le vôtre, dans un pays de blé et de bon vin, un pays de pain et de vignes, un pays d'oliviers qui portent de l'huile, et de miel, et vous vivrez et vous ne mourrez pas. Mais n'écoutez pas Ezéchias~; car il pourrait vous séduire, en disant~: Yahweh nous délivrera.
\VS{33}Les dieux des nations ont-ils délivré chacun leur pays de la main du roi d'Assyrie~?
\VS{34}Où sont les dieux de Hamath et d'Arpad~? Où sont les dieux de Sépharvaïm, d'Héna et d'Ivva~? Et même ont-ils délivré Samarie de ma main~?
\VS{35}Parmi tous les dieux de ces pays, quels sont ceux qui ont délivré leur pays de ma main, pour dire que Yahweh délivrera Jérusalem de ma main~?
\VS{36}Le peuple se tut et on ne lui répondit pas un mot~; car le roi avait donné cet ordre~: Vous ne lui répondrez point.
\VS{37}Après cela, Eliakim, fils de Hilkija, chef de la maison du roi, et Schebna le secrétaire, et Joach, fils d'Asaph, l'archiviste, vinrent auprès d'Ezéchias, les vêtements déchirés, et ils lui rapportèrent les paroles de Rabschaké.
\Chap{19}
\TextTitle{Ezéchias demande à Esaïe de consulter Yahweh\FTNTT{2 Ch. 32:20-22~; Es. 36:22-37:5.}}
\VerseOne{}Et il arriva qu'aussitôt que le roi Ezéchias entendit ces choses, il déchira ses vêtements, se couvrit d'un sac et entra dans la maison de Yahweh.
\VS{2}Puis il envoya Eliakim, chef de la maison du roi, et Schebna le secrétaire, et les plus anciens des prêtres, couverts de sacs, vers Esaïe, le prophète, fils d'Amots.
\VS{3}Ils lui dirent~: Ainsi parle Ezéchias~: Ce jour est un jour d'angoisse, de châtiment et d'opprobre~; car les enfants sont près du sein maternel, mais il n'y a point de force pour enfanter.
\VS{4}Peut-être Yahweh, ton Dieu, a-t-il entendu toutes les paroles de Rabschaké, que le roi d'Assyrie, son maître, a envoyé pour blasphémer le Dieu vivant, et peut-être Yahweh, ton Dieu, exercera-t-il ses châtiments à cause des paroles qu'il a entendues. Fais donc une prière pour le reste qui subsiste encore.
\VS{5}Les serviteurs du roi Ezéchias vinrent donc vers Esaïe.
\TextTitle{Réponse de Yahweh\FTNTT{Es. 37:6-7.}}
\VS{6}Et Esaïe leur dit~: Voici ce que vous direz à votre maître~: Ainsi parle Yahweh~: Ne t'effraie point des paroles que tu as entendues, par lesquelles les serviteurs du roi d'Assyrie m'ont blasphémé.
\VS{7}Voici, je vais mettre en lui un esprit, tel que sur une nouvelle qu'il recevra, il retournera dans son pays~; et je le ferai tomber par l'épée dans son pays.
\TextTitle{Défi du roi d'Assyrie au Dieu d'Israël\FTNTT{2 Ch. 32:17~; Es. 37:8-13.}}
\VS{8}Rabschaké s'étant retiré, trouva le roi d'Assyrie qui attaquait Libna, car il avait appris qu'il était parti de Lakis.
\VS{9}Le roi d'Assyrie reçut une nouvelle au sujet de Tirhaka, roi d'Ethiopie~; on lui dit~: Voici, il est sorti pour te combattre. C'est pourquoi le roi d'Assyrie retourna dans son pays, mais il envoya des messagers à Ezéchias, en leur disant~:
\VS{10}Vous parlerez ainsi à Ezéchias, roi de Juda, et lui direz~: Que ton Dieu, en qui tu te confies, ne t'abuse pas en te disant~: Jérusalem ne sera point livrée entre les mains du roi d'Assyrie.
\VS{11}Voici, tu as entendu ce que les rois d'Assyrie ont fait à tous les pays, et comment ils les ont détruits entièrement~; et tu échapperais~?
\VS{12}Les dieux des nations que mes ancêtres ont détruites, savoir de Gozan, de Charan, de Retseph, et des fils d'Eden, qui sont en Telassar, les ont-ils délivrées~?
\VS{13}Où sont le roi de Hamath, le roi d'Arpad, et le roi de la ville de Sepharvaïm, d'Héna et d'Ivva~?
\TextTitle{Ezéchias dans le temple, sa prière à Yawheh\FTNTT{2 Ch. 32:20~; Es. 37:14-20.}}
\VS{14}Quand Ezéchias reçut la lettre de la main des messagers, il la lut. Puis il monta à la maison de Yahweh et la déploya devant Yahweh~;
\VS{15}puis Ezéchias lui adressa cette prière et dit~: Ô Yahweh, Dieu d'Israël~! Qui est assis entre les chérubins, c'est toi qui es le seul Dieu de tous les royaumes de la terre, c'est toi qui as fait les cieux et la terre.
\VS{16}Ô Yahweh~! Incline ton oreille et écoute. Ouvre tes yeux et regarde. Ecoute les paroles de Sanchérib, et de celui qu’il a envoyé pour blasphémer le Dieu vivant.
\VS{17}Il est vrai, ô Yahweh~! Que les rois d'Assyrie ont détruit ces nations et ravagé leurs pays,
\VS{18}et qu'ils ont jeté dans le feu leurs dieux~; mais ils n'étaient pas des dieux, mais des ouvrages de mains d'homme, du bois, et de la pierre, c'est pourquoi ils les ont détruits.
\VS{19}Maintenant donc, ô Yahweh, notre Dieu~! Je te prie, délivre-nous de la main de Sanchérib, afin que tous les royaumes de la terre sachent que c'est toi, ô Yahweh, qui es le seul Dieu.
\TextTitle{Yahweh répond au travers d'Esaïe\FTNTT{Es. 37:21-35.}}
\VS{20}Alors Esaïe, fils d'Amots, envoya dire à Ezéchias~: Ainsi parle Yahweh, le Dieu d'Israël~: Je t'ai exaucé dans ce que tu m'as demandé au sujet de Sanchérib, roi d'Assyrie.
\VS{21}Voici la parole que Yahweh a prononcée contre lui~: Elle te méprise, elle se moque de toi, la fille, vierge de Sion~; elle hoche la tête après toi, la fille de Jérusalem.
\VS{22}Qui as-tu outragé et blasphémé~? Contre qui as-tu élevé la voix~? Tu as porté tes yeux en haut, vers le Saint d'Israël~!
\VS{23}Tu as insulté le Seigneur par le moyen de tes messagers et tu as dit~: J'ai gravi le sommet des montagnes avec la multitude de mes chars, les extrémités du Liban~; je couperai les plus hauts de ses cèdres et les plus beaux de ses cyprès, et j'atteindrai sa dernière cime, la forêt de son verger.
\VS{24}J'ai creusé des sources, après avoir bu les eaux étrangères et je tarirai avec la plante de mes pieds tous les fleuves de l'Egypte.
\VS{25}N'as-tu pas appris que j'ai préparé cette ville déjà dès longtemps, et que dès les temps anciens je l'ai ainsi formée~? Et maintenant l'aurais-je conservée pour être réduite en désolation, et les villes fortes, en monceaux de ruines~?
\VS{26}Il est vrai que leurs habitants sont impuissants, épouvantés et confus~; ils sont devenus comme l'herbe des champs et la tendre verdure, comme le gazon des toits et le blé brûlé avant la formation de sa tige.
\VS{27}Mais je connais ta demeure, ta sortie et ton entrée, et comment tu es furieux contre moi.
\VS{28}Parce que tu es furieux contre moi et que ton insolence est montée à mes oreilles, je mettrai ma boucle à tes narines, et mon mors entre tes lèvres, et je te ferai retourner par le chemin par lequel tu es venu.
\VS{29}Que ceci soit un signe pour toi, ô Ezéchias~: On mangera cette année le produit du grain tombé, et la deuxième année, ce qui croît de soi-même~; mais la troisième année, vous sèmerez et vous moissonnerez, vous planterez des vignes et vous en mangerez le fruit.
\VS{30}Ce qui aura été épargné de la maison de Juda, ce qui sera resté poussera encore des racines par-dessous et produira du fruit par-dessus.
\VS{31}Car il sortira de Jérusalem un reste, et de la montagne de Sion des réchappés. Voilà ce que fera le zèle de Yahweh des armées.
\VS{32}C'est pourquoi ainsi parle Yahweh, sur le roi d'Assyrie~: Il n'entrera point dans cette ville, il n'y lancera aucune flèche, il ne se présentera point contre elle avec le bouclier et il n'élèvera point des retranchements contre elle.
\VS{33}Il s'en retournera par le chemin par lequel il est venu et il n'entrera point dans cette ville, dit Yahweh.
\VS{34}Car je protègerai cette ville, afin de la délivrer, par amour pour moi et par amour pour David, mon serviteur.
\TextTitle{L'ange de Yahweh dans le camp des Assyriens\FTNTT{Es. 37:36-38.}}
\VS{35}Il arriva nuit-là que l'Ange de Yahweh sortit et frappa cent quatre-vingt-cinq mille hommes dans le camp des Assyriens. Et quand on se leva de bon matin, voici, ils étaient tous morts.
\TextTitle{Mort de Sanchérib, roi d'Assyrie\FTNTT{Es. 37:37-38~; 2 Ch. 32:21.}}
\VS{36}Alors Sanchérib, roi d'Assyrie, leva son camp, partit et s'en retourna~; et il resta à Ninive.
\VS{37}Il arriva, comme il était prosterné dans la maison de Nisroc, son dieu, qu'Adrammélec et Scharetser, ses fils, le tuèrent avec l'épée, puis ils se sauvèrent au pays d'Ararat~; et Esar-Haddon, son fils, régna à sa place.
\Chap{20}
\TextTitle{Ezéchias malade puis guéri par Yahweh\FTNTT{2 Ch. 32:24~; Es. 38.}}
\VerseOne{}En ce temps-là, Ezéchias fut malade à la mort. Le prophète Esaïe, fils d'Amots, vint auprès de lui, et lui dit~: Ainsi parle Yahweh~: Donne tes ordres à ta maison, car tu vas mourir et tu ne vivras plus.
\VS{2}Alors Ezéchias tourna son visage contre le mur et fit sa prière à Yahweh, en disant~:
\VS{3}Je te prie, ô Yahweh~! Souviens-toi que j'ai marché devant toi avec fidélité et intégrité de cœur, et que j'ai fait ce qui est agréable à tes yeux~! Et Ezéchias pleura abondamment.
\VS{4}Esaïe n'était pas encore sorti de la cour du milieu, que la parole de Yahweh lui fut adressée, en disant~:
\VS{5}Retourne et dis à Ezéchias, chef de mon peuple~: Ainsi parle Yahweh, le Dieu de David, ton père~: J'ai exaucé ta prière, j'ai vu tes larmes. Voici je te guérirai~; dans trois jours tu monteras à la maison de Yahweh.
\VS{6}J'ajouterai quinze ans à tes jours, je te délivrerai, toi et cette ville, de la main du roi d'Assyrie~; et je protégerai cette ville, par amour pour moi et par amour pour David, mon serviteur.
\VS{7}Puis Esaïe dit~: Prenez une masse de figues sèches. Et ils la prirent et l'appliquèrent sur l'ulcère. Et Ezéchias fut guéri.
\VS{8}Ezéchias avait dit à Esaïe~: A quel signe connaîtrai-je que Yahweh me guérira et qu'au troisième jour, je monterai à la maison de Yahweh~?
\VS{9}Esaïe répondit~: Voici, de la part de Yahweh, le signe auquel tu connaîtras que Yahweh accomplira la parole qu'il a prononcée~: L'ombre s'avancera-t-elle de dix degrés, ou reculera-t-elle en arrière de dix degrés~?
\VS{10}Ezéchias dit~: C'est peu de chose que l'ombre s'avance de dix degrés~; mais plutôt que l'ombre recule en arrière de dix degrés.
\VS{11}Alors Esaïe, le prophète, invoqua Yahweh, qui fit reculer l'ombre de dix degrés sur les degrés d'Achaz, où elle était descendue.
\TextTitle{Visite des ambassadeurs babyloniens~; prophétie sur la captivité babylonienne\FTNTT{2 Ch. 32:25-31~; Es. 39.}}
\VS{12}En ce temps-là, Berodac-Baladan, fils de Baladan, roi de Babylone, envoya une lettre avec un présent à Ezéchias, parce qu'il avait appris la maladie d'Ezéchias.
\VS{13}Et Ezéchias, donna audience aux envoyés et il leur montra tous les lieux où étaient ses objets les plus précieux, l'argent, l'or, les aromates, l'huile précieuse, tout son arsenal et tout ce qui se trouvait dans ses trésors. Il n'y eut rien qu'Ezéchias ne leur montra dans sa maison et dans tous ses domaines.
\VS{14}Esaïe, le prophète, vint ensuite auprès du roi Ezéchias, et lui dit~: Qu'ont dit ces gens-là~? Et d'où sont-ils venus vers toi~? Ezéchias répondit~: Ils sont venus d'un pays très éloigné, ils sont venus de Babylone.
\VS{15}Esaïe dit~: Qu'ont-ils vu dans ta maison~? Et Ezéchias répondit~: Ils ont vu tout ce qui est dans ma maison~; il n'y a rien dans mes trésors que je ne leur aie montré.
\VS{16}Alors Esaïe dit à Ezéchias~: Ecoute la parole de Yahweh~:
\VS{17}Voici, les jours viendront où tout ce qui est dans ta maison et ce que tes pères ont amassé dans leurs trésors jusqu'à ce jour, sera emporté à Babylone~; il n'en restera rien dit Yahweh\FTNT{La déportation des juifs à Babylone~: Voir 2 R. 24-25.}.
\VS{18}On prendra même de tes fils\FTNT{2 R. 24:12~; 2 Ch. 33:11~; Da. 1.} qui seront sortis de toi, que tu auras engendrés, afin qu'ils soient eunuques dans le palais du roi de Babylone.
\VS{19}Ezéchias répondit à Esaïe~: La parole de Yahweh que tu as prononcée est bonne. Et il ajouta~: N'y aura-t-il pas paix et sécurité pendant mes jours~?
\TextTitle{Mort d'Ezéchias~; Manassé règne sur Juda\FTNTT{2 Ch. 32:32-33.}}
\VS{20}Le reste des actions d'Ezéchias, tous ses exploits, et comment il fit l'étang et l'aqueduc par lequel il fit entrer les eaux dans la ville, cela n'est-il pas écrit dans le livre des Chroniques des rois de Juda~?
\VS{21}Ezéchias se coucha avec ses pères. Et Manassé, son fils, régna à sa place.
\Chap{21}
\TextTitle{Abominations et idolâtrie de Manassé\FTNTT{2 Ch. 33:1-9.}}
\VerseOne{}Manassé était âgé de douze ans, lorsqu'il commença à régner. Il régna cinquante-cinq ans à Jérusalem. Sa mère s'appelait Hephtsiba.
\VS{2}Il fit ce qui est mal aux yeux de Yahweh, selon les abominations des nations que Yahweh avait chassées devant les enfants d'Israël.
\VS{3}Car il rebâtit les hauts lieux qu'Ezéchias, son père, avait détruits, et redressa des autels à Baal, il fit une idole d'Asherah\FTNT{Voir commentaire en Jg. 2:13.}, comme avait fait Achab, roi d'Israël, il se prosterna devant toute l'armée des cieux et les servit.
\VS{4}Il bâtit aussi des autels dans la maison de Yahweh, quoique Yahweh ait dit~: C'est dans Jérusalem que j'établirai mon nom.
\VS{5}Il bâtit des autels à toute l'armée des cieux dans les deux parvis de la maison de Yahweh.
\VS{6}Il fit aussi passer son fils par le feu, il pratiquait l'astrologie et la divination, il établit des gens qui évoquaient les esprits des morts et qui prédisaient l'avenir. Il fit de plus en plus ce qui est mal aux yeux de Yahweh pour l'irriter.
\VS{7}Il plaça aussi l'idole d'Asherah qu'il avait faite, dans la maison de laquelle Yahweh avait dit à David, et à Salomon, son fils~: C'est dans cette maison, et c'est dans Jérusalem, que j'ai choisie parmi toutes les tribus d'Israël, que je veux à toujours établir mon nom.
\VS{8}Je ne ferai plus errer le pied d'Israël hors de cette terre que j'ai donnée à leurs pères, pourvu seulement qu'ils aient soin de mettre en pratique tout ce que je leur ai ordonné et toute la loi que Moïse, mon serviteur, leur a prescrite.
\VS{9}Mais ils n'obéirent point~; car Manassé les fit s'égarer, jusqu'à faire le mal plus que les nations que Yahweh avait exterminées devant les enfants d'Israël.
\TextTitle{Jugement de Yahweh contre Juda~; mort d'Ezéchias\FTNTT{2 Ch. 33:10-20.}}
\VS{10}Alors Yahweh parla par ses serviteurs les prophètes, en disant~:
\VS{11}Parce que Manassé, roi de Juda, a commis ces abominations, parce qu'il a fait pire que tout ce qu'avaient fait avant lui les Amoréens, et parce qu'il a aussi fait pécher Juda par ses idoles,
\VS{12}à cause de cela, Yahweh, le Dieu d'Israël, dit~: Voici, je vais faire venir sur Jérusalem et sur Juda des malheurs qui étourdiront les oreilles de quiconque en entendra parler.
\VS{13}Car j'étendrai sur Jérusalem le cordeau de Samarie, et le niveau de la maison d'Achab, et je nettoierai Jérusalem comme un plat qu'on nettoie, et qu’on renverse sur son fond après l'avoir nettoyé.
\VS{14}J'abandonnerai le reste de mon héritage, et je les livrerai entre les mains de leurs ennemis~; et ils seront le butin et la proie de tous leurs ennemis~;
\VS{15}parce qu'ils ont fait ce qui est mal à mes yeux, et qu'ils m'ont irrité depuis le jour où leurs pères sont sortis d'Egypte, jusqu'à ce jour.
\TextTitle{Meurtres de Manassé~; sa mort\FTNTT{2 Ch. 33:11-20.}}
\VS{16}Manassé répandit aussi beaucoup de sang innocent, jusqu'à en remplir Jérusalem d'un bout à l'autre, outre son péché par lequel il fit pécher Juda en faisant ce qui est mal aux yeux de Yahweh.
\VS{17}Le reste des actions de Manassé, tout ce qu'il a fait~; et les péchés auxquels il se livra, cela n'est-il pas écrit dans le livre des Chroniques des rois de Juda~?
\VS{18}Manassé se coucha avec ses pères, et il fut enseveli dans le jardin de sa maison, dans le jardin d'Uzza. Amon, son fils, régna à sa place.
\TextTitle{Amon règne sur Juda~; sa mort\FTNTT{2 Ch. 33:20-25.}}
\VS{19}Amon était âgé de vingt-deux ans lorsqu'il commença à régner. Il régna deux ans à Jérusalem. Sa mère s'appelait Meschullémeth, fille de Haruts, de Jotba.
\VS{20}Il fit ce qui est mal aux yeux de Yahweh, comme avait fait Manassé, son père.
\VS{21}Car il marcha dans toute la voie où avait marché son père, il servit les idoles que son père avait servies et se prosterna devant elles.
\VS{22}Il abandonna Yahweh, le Dieu de ses pères et il ne marcha point dans la voie de Yahweh.
\TextTitle{Josias, roi de Juda\FTNTT{2 Ch. 33:24-25.}}
\VS{23}Les serviteurs d'Amon firent une conspiration contre lui et le tuèrent dans sa maison.
\VS{24}Mais le peuple du pays frappa tous ceux qui avaient conspiré contre le roi Amon~; et ils établirent Josias, son fils, roi à sa place.
\VS{25}Le reste des actions d'Amon, ce qu'il a fait, cela n'est-il pas écrit dans le livre des Chroniques des rois de Juda~?
\VS{26}On l'ensevelit dans son sépulcre, dans le jardin d'Uzza. Et Josias, son fils, régna à sa place.
\Chap{22}
\TextTitle{Droiture de Josias~; réparations dans le temple\FTNTT{2 Ch. 34:2-13.}}
\VerseOne{}Josias était âgé de huit ans lorsqu'il commença à régner. Il régna trente et un ans à Jérusalem. Sa mère s'appelait Jedida, fille d'Adaja, de Botskath.
\VS{2}Il fit ce qui est droit aux yeux de Yahweh et il marcha dans toute la voie de David, son père~; il ne s'en détourna ni à droite ni à gauche.
\VS{3}La dix-huitième année du roi Josias, le roi envoya dans la maison de Yahweh, Schaphan, le secrétaire, fils d'Atsalia, fils de Meschullam.
\VS{4}Il lui dit~: Monte vers Hilkija, le grand-prêtre, et dis-lui d'amasser l'argent qui a été apporté dans la maison de Yahweh et que ceux qui ont la garde du seuil ont recueilli du peuple.
\VS{5}On remettra cet argent entre les mains de ceux qui sont chargés de faire exécuter l'ouvrage dans la maison de Yahweh. Et ils l'emploieront pour ceux qui travaillent dans la maison de Yahweh, pour réparer les brèches de la maison,
\VS{6}pour les charpentiers, les architectes et les maçons, pour les achats du bois et des pierres de taille pour réparer la maison.
\VS{7}Mais on ne leur demandera pas de comptes pour l'argent remis entre leurs mains, parce qu'ils agissent fidèlement.
\TextTitle{Découverte et lecture du livre de la loi\FTNTT{2 Ch. 34:14-19.}}
\VS{8}Alors Hilkija, le grand-prêtre, dit à Schaphan, le secrétaire~: J'ai trouvé le livre de la loi dans la maison de Yahweh. Et Hilkija donna ce livre à Schaphan qui le lut.
\VS{9}Schaphan, le secrétaire, alla vers le roi et lui rapporta la chose, et dit~: Tes serviteurs ont amassé l'argent qui se trouvait dans la maison et l'ont remis entre les mains de ceux qui sont chargés de faire l'ouvrage dans la maison de Yahweh.
\VS{10}Schaphan, le secrétaire, dit aussi au roi~: Le prêtre Hilkija m'a donné un livre. Et Schaphan le lut devant le roi.
\VS{11}Lorsque le roi eut entendu les paroles du livre de la loi, il déchira ses vêtements.
\TextTitle{Annonce du jugement de Yahweh par Hulda\FTNTT{2 Ch. 34:20-28.}}
\VS{12}Il donna cet ordre au prêtre Hilkija, à Achikam, fils de Schaphan, à Acbor, fils de Michée, à Schaphan, le secrétaire, et à Asaja, serviteur du roi~:
\VS{13}Allez, consultez Yahweh pour moi, pour le peuple et pour tout Juda, au sujet des paroles de ce livre qui a été trouvé~; car grande est la colère de Yahweh, qui s'est enflammée contre nous, parce que nos pères n'ont point obéi aux paroles de ce livre et n'ont pas mis en pratique tout ce qui nous y est prescrit.
\VS{14}Le prêtre Hilkija, Achikam, Acbor, Schaphan et Asaja, allèrent auprès de la prophétesse Hulda, femme de Schallum, fils de Thikva, fils de Harhas, gardien des vêtements. Elle habitait dans un autre quartier de Jérusalem.
\TextTitle{Yahweh rassure Josias par la prophétesse Hulda\FTNTT{2 Ch. 34:22-28.}}
\VS{15}Après qu'ils eurent parlé avec elle, elle leur répondit~: Ainsi parle Yahweh, le Dieu d'Israël~: Dites à l'homme qui vous a envoyé vers moi~:
\VS{16}Ainsi parle Yahweh~: Voici, je vais faire venir le malheur sur cette ville et sur ses habitants, selon toutes les paroles du livre que le roi de Juda a lu.
\VS{17}Parce qu'ils m'ont abandonné et qu'ils ont offert des parfums à d'autres dieux, pour m'irriter par toutes les actions de leurs mains, ma colère s'est enflammée contre cette ville et elle ne s'éteindra point.
\VS{18}Mais quant au roi de Juda qui vous a envoyé pour consulter Yahweh, vous lui direz~: Ainsi parle Yahweh, le Dieu d'Israël, au sujet des paroles que tu as entendues~:
\VS{19}Parce que ton cœur a été touché, et que tu t'es humilié devant Yahweh en entendant ce que j'ai prononcé contre cette ville et contre ses habitants, qui seront un objet d'épouvante et de malédiction, et parce que tu as déchiré tes vêtements, et que tu as pleuré devant moi, je t'ai exaucé, dit Yahweh.
\VS{20}C'est pourquoi voici, je vais te recueillir auprès de tes pères, et tu seras recueilli dans ton sépulcre en paix, et tes yeux ne verront point tout ce mal que je vais faire venir sur cette ville. Ils rapportèrent toutes ces paroles au roi.
\Chap{23}
\TextTitle{Le livre de la loi lu au peuple\FTNTT{2 Ch. 34:29-30.}}
\VerseOne{}Alors, le roi Josias, fit assembler auprès de lui tous les anciens de Juda et de Jérusalem.
\VS{2}Le roi monta à la maison de Yahweh, avec tous les hommes de Juda, tous les habitants de Jérusalem, les prêtres, les prophètes, et tout le peuple, depuis le plus petit jusqu'au plus grand. Il lut devant eux toutes les paroles du livre de l'alliance, qui avait été trouvé dans la maison de Yahweh.
\TextTitle{Engagement de Josias et du peuple à suivre la loi de Yahweh\FTNTT{2 Ch. 34:31-32.}}
\VS{3}Le roi se tenait sur l'estrade et il traita alliance devant Yahweh, s'engageant à suivre Yahweh, à observer ses ordonnances, ses préceptes et ses lois, de tout son cœur, à persévérer dans les paroles de cette alliance, écrites dans ce livre. Et tout le peuple entra dans cette alliance.
\TextTitle{Josias débarrasse Juda de tous ses faux dieux\FTNTT{2 Ch. 34:33.}}
\VS{4}Alors le roi donna cet ordre à Hilkija, le grand-prêtre, aux prêtres du second ordre et à ceux qui gardaient le seuil, de sortir hors du temple de Yahweh tous les ustensiles qui avaient été faits pour Baal\FTNT{Voir commentaire en Jg. 2:12.}, pour Asherah\FTNT{Voir commentaire en Jg. 2:13.}, et pour toute l'armée des cieux~; et il les brûla hors de Jérusalem, dans les champs de Cédron, et en fit porter la poussière à Béthel.
\VS{5}Il chassa les prêtres des idoles, que les rois de Juda avaient établis pour brûler des parfums sur les hauts lieux, dans les villes de Juda et aux environs de Jérusalem, et ceux qui offraient des parfums à Baal, au soleil, à la lune, au zodiaque et à toute l'armée des cieux.
\VS{6}Il sortit de la maison de Yahweh l'idole d'Asherah, qu'il transporta hors de Jérusalem vers le torrent de Cédron~; il la brûla au torrent de Cédron et la réduisit en poudre, et il en jeta la poussière sur le sépulcre des fils du peuple.
\VS{7}Ensuite, il démolit les maisons des prostituées qui étaient dans la maison de Yahweh, où les femmes tissaient des tentes pour Asherah.
\VS{8}Il fit venir des villes de Juda tous les prêtres~; il profana les hauts lieux où les prêtres brûlaient des parfums, depuis Guéba jusqu'à Beer-Schéba~; il renversa les hauts lieux des portes, celui qui était à l'entrée de la porte de Josué, chef de la ville, et celui qui était à gauche de la porte de la ville.
\VS{9}Toutefois, les prêtres des hauts lieux ne montaient pas à l'autel de Yahweh à Jérusalem, mais ils mangeaient des pains sans levain parmi leurs frères.
\VS{10}Le roi profana aussi Topheth, dans la vallée des fils de Hinnom, afin que personne ne fasse plus passer son fils ou sa fille par le feu, en l'honneur de Moloc\FTNT{Lé. 20:2-3.}.
\VS{11}Il fit disparaître de l'entrée de la maison de Yahweh les chevaux que les rois de Juda avaient consacrés au soleil, près de la chambre de l'eunuque Nethan-Mélec, situé à Parvarim, et il brûla au feu les chars du soleil.
\VS{12}Le roi démolit les autels qui étaient sur le toit de la chambre haute d'Achaz, que les rois de Juda avaient faits et les autels que Manassé avait faits dans les deux parvis de la maison de Yahweh~; après les avoir brisés et enlevés de là, il en jeta la poussière dans le torrent de Cédron.
\VS{13}Le roi profana aussi les hauts lieux qui étaient en face de Jérusalem, sur la droite de la montagne de perdition, que Salomon, roi d'Israël, avait bâtis à Astarté, l'abomination des Sidoniens, à Kemosch, l'abomination des Moabites, et à Milcom, l'abomination des fils d'Ammon.
\VS{14}Il brisa aussi les statues, et abattit les Asherah, et il remplit d'ossements d'hommes les lieux où elles étaient.
\VS{15}Il renversa l'autel qui était à Béthel et le haut lieu qu'avait fait Jéroboam, fils de Nebath, qui avait fait pécher Israël~; il brûla le haut lieu et le réduisit en poudre, et il brûla l'Asherah.
\VS{16}Josias s'étant tourné et ayant vu les sépulcres qui étaient là dans la montagne, envoya prendre les ossements des sépulcres, et il les brûla sur l'autel et le profana, selon la parole de Yahweh prononcée à haute voix par l'homme de Dieu.
\VS{17}Le roi dit~: Quel est ce monument que je vois~? Et les hommes de la ville lui répondirent~: C'est le sépulcre de l'homme de Dieu qui est venu de Juda qui a crié contre l'autel de Béthel ces choses que tu as accomplies.
\VS{18}Et il dit~: Laissez-le~; que personne ne remue ses os~! Ils conservèrent ainsi ses os, avec les os du prophète qui était venu de Samarie.
\VS{19}Josias fit encore disparaître toutes les maisons des hauts lieux, qui étaient dans les villes de Samarie, et qu'avaient faites les rois d'Israël pour irriter Yahweh~; et il fit à leur égard entièrement comme il avait fait à Béthel.
\VS{20}Il immola sur les autels tous les prêtres des hauts lieux qui étaient là, et il y brûla des ossements d'hommes. Puis il retourna à Jérusalem.
\TextTitle{Josias rétablit la fête de la Pâque\FTNTT{2 Ch. 35:1-19.}}
\VS{21}Alors le roi donna cet ordre à tout le peuple, en disant~: Célébrez la Pâque en l'honneur de Yahweh, votre Dieu, comme il est écrit dans le livre de cette alliance\FTNT{Jésus-Christ est notre Pâque. Voir Ex. 12 et 1 Co. 5:7.}.
\VS{22}Aucune Pâque pareille à celle-ci n'avait été célébrée depuis le temps où les juges jugeaient Israël et pendant tous les jours des rois d'Israël et des rois de Juda.
\VS{23}Ce fut la dix-huitième année du roi Josias qu'on célébra cette Pâque en l'honneur de Yahweh à Jérusalem.
\VS{24}Josias, extermina aussi, ceux qui évoquaient les esprits des morts et les devins, les théraphim, les idoles, et toutes les abominations qui se voyaient dans le pays de Juda et à Jérusalem, afin de mettre en pratique les paroles de la loi, écrites dans le livre que Hilkija, le prêtre, avait trouvé dans la maison de Yahweh.
\TextTitle{Témoignage de Josias~; confirmation du jugement de Yahweh}
\VS{25}Avant Josias, il n'y eut point de roi qui, comme lui, revienne à Yahweh de tout son cœur, de toute son âme et de toute sa force, selon toute la loi de Moïse~; et après lui, il n'en a point paru de semblable.
\VS{26}Toutefois, Yahweh ne se détourna point de l'ardeur de sa grande colère dont il était enflammé contre Juda, à cause de tout ce que Manassé avait fait pour l'irriter.
\VS{27}Et Yahweh dit~: J'ôterai Juda de devant ma face, comme j'ai ôté Israël, et je rejetterai cette ville de Jérusalem que j'avais choisie, et la maison de laquelle j'avais dit~: Là sera mon Nom.
\VS{28}Le reste des actions de Josias, tout ce qu'il a fait, cela n'est-il pas écrit dans le livre des Chroniques des rois de Juda~?
\TextTitle{Mort de Josias~; Joachaz règne sur Juda\FTNTT{2 Ch. 35:20-27~; 2 Ch. 36:1-2.}}
\VS{29}De son temps, Pharaon Néco, roi d'Egypte, monta contre le roi d'Assyrie, vers le fleuve d'Euphrate. Le roi Josias s'en alla au-devant de lui~; mais dès que pharaon le vit, il le tua à Meguiddo.
\VS{30}Ses serviteurs l'emportèrent mort sur un char~; ils l'amenèrent de Meguiddo à Jérusalem et l'ensevelirent dans son sépulcre. Et le peuple du pays prit Joachaz, fils de Josias, ils l'oignirent et l'établirent roi à la place de son père.
\TextTitle{Joachaz mis en prison par Pharaon\FTNTT{2 Ch. 36:3.}}
\VS{31}Joachaz était âgé de vingt-trois ans, lorsqu'il commença à régner. Il régna trois mois à Jérusalem. Sa mère s'appelait Hamuthal, fille de Jérémie, de Libna.
\VS{32}Il fit ce qui est mal aux yeux de Yahweh, entièrement comme avaient fait ses pères.
\VS{33}Et pharaon Néco l'emprisonna à Ribla, dans le pays de Hamath, afin qu'il ne règne plus à Jérusalem~; et il imposa sur le pays un tribut de cent talents d'argent et d'un talent d'or.
\TextTitle{Pharaon établit Jojakim roi de Juda\FTNTT{2 Ch. 36:4-5.}}
\VS{34}Puis pharaon Néco établit roi Eliakim, fils de Josias, à la place de Josias, son père, et il changea son nom en celui de Jojakim. Il prit Joachaz, qui alla en Egypte, où il mourut.
\VS{35}Jojakim donna cet argent et cet or à pharaon~; mais il taxa le pays pour fournir cet argent, selon l'ordre de pharaon~; il détermina la part de chacun et exigea du peuple du pays l'argent et l'or qu'il devait livrer à pharaon Néco.
\VS{36}Jojakim était âgé de vingt-cinq ans lorsqu'il commença à régner. Il régna onze ans à Jérusalem. Sa mère s'appelait Zebudda, fille de Pedaja, de Ruma.
\VS{37}Il fit ce qui est mal aux yeux de Yahweh, entièrement comme avaient fait ses pères.
\Chap{24}
\TextTitle{Asservissement de Jojakim au roi de Babylone~; destruction de Juda\FTNTT{2 Ch. 36:6-7.}}
\VerseOne{}De son temps, Nebucadnetsar, roi de Babylone, monta contre Jojakim, et Jojakim lui fut asservi pendant trois ans~; mais il se révolta de nouveau contre lui.
\VS{2}Alors Yahweh envoya contre Jojakim des troupes de Chaldéens, des armées de Syriens, des troupes de Moabites et des troupes des fils d'Ammon~; il les envoya contre Juda, pour le détruire, selon la parole que Yahweh avait prononcée par ses serviteurs les prophètes.
\VS{3}Cela arriva uniquement sur l'ordre de Yahweh, qui voulait ôter Juda de devant sa face, à cause de tous les péchés commis par Manassé,
\VS{4}et à cause aussi du sang innocent qu'il avait répandu, et dont il avait rempli Jérusalem. C'est pourquoi Yahweh ne voulut point lui pardonner.
\TextTitle{Mort de Jojakim~; Jojakin règne sur Juda\FTNTT{2 Ch. 36:8-9.}}
\VS{5}Le reste des actions de Jojakim et tout ce qu'il a fait, cela n'est-il pas écrit dans le livre des Chroniques des rois de Juda~?
\VS{6}Ainsi Jojakim se coucha avec ses pères. Et Jojakin, son fils, régna à sa place.
\VS{7}Le roi d'Egypte ne sortit plus de son pays, parce que le roi de Babylone avait pris tout ce qui était au roi d'Egypte, depuis le torrent d'Egypte jusqu'au fleuve d'Euphrate.
\VS{8}Jojakin était âgé de dix-huit ans lorsqu'il commença à régner. Il régna trois mois à Jérusalem. Sa mère s'appelait Nehuschtha, fille d'Elnathan, de Jérusalem.
\VS{9}Il fit ce qui est mal aux yeux de Yahweh, entièrement comme avait fait son père.
\TextTitle{Jérusalem et son roi en captivité à Babylone~; les pauvres restent\FTNTT{2 Ch. 36:10.}}
\VS{10}En ce temps-là, les serviteurs de Nebucadnetsar, roi de Babylone, montèrent contre Jérusalem, et la ville fut assiégée.
\VS{11}Nebucadnetsar, roi de Babylone, arriva devant la ville, pendant que ses serviteurs l'assiégeaient.
\VS{12}Alors Jojakin, roi de Juda, se rendit vers le roi de Babylone, avec sa mère, ses serviteurs, ses chefs et ses eunuques. Et le roi de Babylone le fit prisonnier, la huitième année de son règne.
\VS{13}Il emporta de là, tous les trésors de la maison de Yahweh et les trésors de la maison royale~; et il mit en pièces tous les ustensiles d'or que Salomon, roi d'Israël, avait faits pour le temple de Yahweh, comme Yahweh l'avait ordonné.
\VS{14}Il emmena en captivité tout Jérusalem\FTNT{Première déportation~: 2 R. 24:1-4 et 2 Ch. 36:6-7. La première déportation eut lieu en 597 av. J.-C. pendant le règne de Jojakim, roi de Juda. Les premiers exilés furent installés dans la région du fleuve Kebar (Ez. 1:1-3), un canal de 90 km de long reliant l'Euphrate au nord de Babylone au même fleuve au sud d'Ur en Chaldée. Jérémie savait que leur séjour à l'étranger serait long. Il avait prophétisé qu'il durerait soixante-dix ans (Jé. 25:1~; Jé. 25:11-12) et leur conseilla de se construire des maisons, de cultiver des jardins et de se multiplier (Jé. 29). Daniel et ses compagnons furent déportés à Babylone lors de la première déportation (Da. 1). Daniel fut déporté environ huit ans avant Ezéchiel.}, à savoir, tous les chefs, et tous les vaillants hommes de guerre, au nombre de dix mille captifs, avec les charpentiers et les serruriers, de sorte qu'il ne resta plus que le peuple pauvre du pays.
\VS{15}Ainsi il transporta Jojakin à Babylone, avec la mère du roi, les femmes du roi et ses eunuques. Il emmena captifs à Babylone tous les grands du pays, de Jérusalem à Babylone,
\VS{16}avec tous les guerriers au nombre de sept mille, les charpentiers, les serruriers au nombre de mille, tous les hommes vaillants et propres à la guerre. Le roi de Babylone les emmena captifs à Babylone.
\TextTitle{Nebucadnetsar établit Sédécias roi de Juda\FTNTT{2 Ch. 36:10-12.}}
\VS{17}Et le roi de Babylone établit roi, à la place de Jojakin, Matthania, son oncle, et il changea son nom en celui de Sédécias.
\VS{18}Sédécias était âgé de vingt et un ans lorsqu'il commença à régner. Il régna onze ans à Jérusalem. Sa mère s'appelait Hamuthal, fille de Jérémie, de Libna.
\VS{19}Il fit ce qui est mal aux yeux de Yahweh, entièrement comme avait fait Jojakim.
\TextTitle{Sédécias se révolte\FTNTT{2 Ch. 36:13-16.}}
\VS{20}Cela arriva à cause de la colère de Yahweh contre Jérusalem et contre Juda, qu'il voulait rejeter de devant sa face. Et Sédécias se révolta contre le roi de Babylone.
\Chap{25}
\TextTitle{Siège de Jérusalem\FTNTT{Jé. 39:1.}}
\VerseOne{}Et il arriva dans la neuvième année du règne de Sédécias, le dixième jour du dixième mois, que Nebucadnetsar\FTNT{Jérusalem fut assiégée pendant deux ans. Lors de ce siège, des femmes juives faisaient cuire leurs enfants pour les consommer (La. 2:20~; La. 4:10).}, roi de Babylone, vint avec toute son armée contre Jérusalem~; il campa devant elle et éleva des retranchements tout autour.
\VS{2}La ville fut assiégée jusqu'à la onzième année du roi Sédécias.
\VS{3}Le neuvième jour du 4ème mois, la famine\FTNT{La. 4:10.} augmenta dans la ville, de sorte qu'il n'y avait pas de pain pour le peuple du pays.
\TextTitle{Sédécias lié et emmené à Babylone\FTNTT{Jé. 39:2-7.}}
\VS{4}Alors la brèche fut faite à la ville~; et tous les gens de guerre s'enfuirent de nuit par le chemin de la porte entre les deux murailles près du jardin du roi, pendant que les Chaldéens environnaient la ville. Les fuyards et le roi prirent le chemin de la plaine.
\VS{5}Mais l'armée des Chaldéens poursuivit le roi et l'atteignit dans les plaines de Jéricho, et toute son armée se dispersa loin de lui.
\VS{6}Ils saisirent donc le roi et le firent monter vers le roi de Babylone à Ribla~; et l'on prononça contre lui un jugement.
\VS{7}Et on égorgea les fils de Sédécias en sa présence~; puis on creva les yeux à Sédécias, et on le lia de doubles chaînes d'airain, et on le mena à Babylone.
\TextTitle{Destruction de Jérusalem, du temple et des murailles\FTNTT{2 Ch. 36:17-21~; Jé. 39~:8-10.}}
\VS{8}Le septième jour du cinquième mois, c'était la dix-neuvième année du roi Nebucadnetsar, roi de Babylone, Nebuzaradan, chef des gardes, serviteur du roi de Babylone,\FTNT{Troisième déportation~: Le temple fut brûlé, la ville de Jérusalem fut totalement rasée et ses habitants furent déportés (De. 28:49-68). Contrairement à ce que l'on pense, il y a eu d'autres déportations. Voir Jé. 52.} entra dans Jérusalem.
\VS{9}Il brûla la maison de Yahweh, la maison royale et toutes les maisons de Jérusalem~; il brûla par le feu toutes les grandes maisons.
\VS{10}Toute l'armée des Chaldéens, qui était avec le chef des gardes, démolit les murailles qui entouraient Jérusalem.
\VS{11}Et Nebuzaradan, chef des gardes, emmena captifs le reste du peuple, ceux qui étaient restés dans la ville, ceux qui s'étaient rendus au roi de Babylone et le reste de la multitude.
\VS{12}Cependant le chef des gardes laissa quelques-uns des plus pauvres du pays comme vignerons et comme laboureurs.
\VS{13}Les Chaldéens brisèrent les colonnes d'airain qui étaient dans la maison de Yahweh, les bases, la mer d'airain qui était dans la maison de Yahweh, et ils en emportèrent l'airain à Babylone.
\VS{14}Ils prirent aussi les cendriers, les pelles, les couteaux, les tasses et tous les ustensiles d'airain avec lesquels on faisait le service.
\VS{15}Le chef des gardes emporta aussi les encensoirs et les coupes, ce qui était d'or et ce qui était d'argent.
\VS{16}Les deux colonnes, la mer et les bases, que Salomon avait faits pour la maison de Yahweh, tous ces ustensiles d'airain avaient un poids inconnu.
\VS{17}La hauteur d'une colonne était de dix-huit coudées, et il y avait au-dessus un chapiteau d'airain dont la hauteur était de trois coudées~; autour du chapiteau il y avait un treillis et des grenades, le tout d'airain~; il en était de même pour la seconde colonne avec le treillis.
\VS{18}Le chef des gardes emmena aussi Seraja, le premier prêtre, et Sophonie, le second prêtre, et les trois gardiens du seuil.
\VS{19}Et dans la ville, il prit un eunuque qui avait sous son commandement des hommes de guerre, cinq hommes de ceux qui voyaient la face du roi et qui furent trouvés dans la ville, il prit aussi le secrétaire du chef de l'armée qui était chargé d'enrôler le peuple du pays, et soixante hommes du peuple du pays qui se trouvaient dans la ville.
\VS{20}Nebuzaradan, chef des gardes, les prit et les conduisit vers le roi de Babylone à Ribla.
\VS{21}Le roi de Babylone les frappa, et les fit mourir à Ribla, dans le pays de Hamath. Ainsi Juda fut transporté captif hors de sa terre.
\TextTitle{Guedalia nommé gouverneur de Juda\FTNTT{Jé. 40:7-11.}}
\VS{22}Nebucadnetsar, roi de Babylone, plaça le reste du peuple, qu'il laissa dans le pays de Juda, sous le commandement de Guedalia, fils d'Achikam, fils de Schaphan.
\VS{23}Lorsque tous les chefs des troupes et leurs hommes, eurent appris que le roi de Babylone avait établi Guedalia pour gouverneur, ils allèrent trouver Guedalia à Mitspa, à savoir Ismaël, fils de Nethania, Jochanan, fils de Karéach, Seraja, fils de Thanhumeth, de Nethopha, Jaazania, fils du Maacathien, eux et leurs hommes.
\VS{24}Guedalia leur jura, à eux et à leurs hommes, et leur dit~: Ne craignez pas d'être serviteurs des Chaldéens~; demeurez dans le pays et servez le roi de Babylone, et vous vous en trouverez bien.
\TextTitle{Fuite du peuple en Egypte\FTNTT{Jé. 41:1-3~; Jé. 43:4-7.}}
\VS{25}Mais il arriva au septième mois, qu'Ismaël, fils de Nethania, fils d'Elischama, qui était de race royale, vint, accompagné de dix hommes, et ils frappèrent mortellement Guedalia, ainsi que les Juifs et les Chaldéens qui étaient avec lui à Mitspa.
\VS{26}Alors tout le peuple, depuis le plus petit jusqu'au plus grand, avec les chefs des troupes, se levèrent et s'en allèrent en Egypte, parce qu'ils avaient peur des Chaldéens.
\TextTitle{Jojakin à la table du roi de Babylone\FTNTT{Jé. 52:31-34.}}
\VS{27}La trente-septième année de la captivité de Jojakin, roi de Juda, le vingt-septième jour du douzième mois, Evil-Merodac, roi de Babylone, dans la première année de son règne, releva la tête de Jojakin, roi de Juda et le tira de prison.
\VS{28}Il lui parla avec bonté et il mit son trône au-dessus du trône des rois qui étaient avec lui à Babylone.
\VS{29}Il lui fit changer ses vêtements de prison, et Jojakin mangea du pain tout le temps de sa vie en sa présence.
\VS{30}Et quant à son entretien, un entretien perpétuel, lui fut accordé par le roi pour chaque jour, tous les jours de sa vie.
\PPE{}
\end{multicols}

%\clearpage
\ShortTitle{Esaïe}\BookTitle{Esaïe}\BFont
\noindent\hrulefill
{\footnotesize
\textit{
\bigskip
{\centering{}
\\Auteur : Esaïe
\\(Heb. : Yesha'yah)
\\Signification : YAHWEH a sauvé
\\Thème : Le Messie d'Israël
\\Date de rédaction : 8\up{ème} siècle av. J.-C.\\}
}
%\bigskip
\textit{
\\Prophète en Israël, Esaïe fut une figure marquante en raison du contenu et de l'impact de son message. Véritable porte-parole de Dieu, il parla de la ruine morale d'Israël, de la déportation à Babylone et des jugements de Dieu sur son peuple. Il prophétisa également sur le retour de l'exil, la restauration finale et la reconstruction de Jérusalem. Plus qu'aucun autre livre, les écrits d'Esaïe annoncent clairement la naissance du Messie, son service, sa mission rédemptrice, son sacrifice et son futur règne millénaire. 
%\bigskip
\\L'autorité et l'exactitude de ses prophéties ont été une source d'édification au fil des siècles.\bigskip
}
}
\par\nobreak\noindent\hrulefill
\begin{multicols}{2}
\Chap{1}
\TextTitle{Prophéties concernant Juda}
\VerseOne{}La vision d'Esaïe, fils d'Amots, qu'il a vue touchant Juda et Jérusalem, au jour d'Ozias, de Jotham, d'Achaz, et d'Ezéchias, rois de Juda.
\VS{2}Cieux, écoutez ! Et toi, terre, prête l'oreille ! Car Yahweh parle. J'ai nourri des enfants, je les ai élevés, mais ils se sont rebellés contre moi.
\VS{3}Le bœuf connaît son possesseur, et l'âne la crèche de son maître, mais Israël n'a point de connaissance, mon peuple n'a point d'intelligence.
\VS{4}Ah! Nation pécheresse, peuple chargé d'iniquités, race de gens méchants, enfants qui ne font que se corrompre ! Ils ont abandonné Yahweh, ils ont irrité par leur mépris le Saint d'Israël, ils se sont retirés en arrière.
\VS{5}Pourquoi serez-vous encore frappés ? Vous ajouterez la révolte ! La tête entière est malade, et tout le cœur est languissant.
\VS{6}Depuis la plante du pied jusqu'à la tête, il n'y a rien de sain en lui : Il n'y a que blessures, meurtrissures et plaies pourries, qui n'ont été ni nettoyées, ni bandées, et dont aucune n'a été adoucie par l'huile.
\VS{7}Votre pays n'est que désolation, et vos villes sont en feu ; des étrangers dévorent votre terre sous vos yeux, et cette désolation est comme un bouleversement fait par des étrangers.
\VS{8}Car la fille de Sion est restée comme une cabane dans une vigne, comme une cabane dans un champ de concombres, comme une ville assiégée.
\VS{9}Si Yahweh des armées ne nous avait pas laissé un petit reste, qui est même bien peu, nous serions comme Sodome, nous ressemblerions à Gomorrhe.
\TextTitle{Yahweh rejette la religiosité et recherche la justice}
\VS{10}Ecoutez la parole de Yahweh, chefs de Sodome, prêtez l'oreille à la loi de notre Dieu, peuple de Gomorrhe !
\VS{11}Qu'ai-je à faire, dit Yahweh, de la multitude de vos sacrifices ? Je suis rassasié des holocaustes de béliers et de la graisse des veaux ; je ne prends point plaisir au sang des taureaux, ni des agneaux, ni des boucs\FTNT{1 S. 15:22 ; Os. 8:13 ; Mt. 9:13.}.
\VS{12}Quand vous entrez pour vous présenter devant ma face, qui a requis cela de votre main, que vous fouliez de vos pieds mes parvis ?
\VS{13}Ne continuez plus à m'apporter de vaines offrandes : Le parfum m'est en abomination, quant aux nouvelles lunes, aux sabbats et à la publication de vos convocations ; je ne puis plus supporter votre méchanceté ni vos assemblées solennelles.
\VS{14}Mon âme hait vos nouvelles lunes et vos fêtes solennelles ; elles me sont fâcheuses, je suis las de les supporter.
\VS{15}C'est pourquoi, quand vous étendez vos mains, je cache mes yeux de vous ; quand vous multipliez vos prières, je ne les exauce pas ; vos mains sont pleines de sang\FTNT{Es. 59:1-3 ;Mi. 3:4.}.
\VS{16}Lavez-vous, purifiez-vous, ôtez de devant mes yeux la méchanceté de vos actions ; cessez de faire le mal.
\VS{17}Apprenez à bien faire, recherchez la droiture, redressez celui qui est foulé ; faites justice à l'orphelin, défendez la cause de la veuve.
\TextTitle{Mise en garde ; appel à la justice de Yahweh}
\VS{18}Venez maintenant, dit Yahweh, et débattons nos droits. Si vos péchés sont comme l'écarlate, ils seront blanchis comme la neige ; s'ils sont rouges comme le vermillon ils seront blanchis comme la laine.
\VS{19}Si vous obéissez volontairement, vous mangerez le meilleur du pays.
\VS{20}Mais si vous refusez d'obéir et si vous êtes rebelles, vous serez dévorés par l'épée, car la bouche de Yahweh a parlé.
\VS{21}Comment la cité fidèle est-elle devenue une prostituée ? Elle était pleine de droiture et la justice y habitait ; mais maintenant elle est pleine de meurtriers !
\VS{22}Ton argent s'est changé en scories; ton breuvage est mêlé d'eau. 
\VS{23}Les chefs de ton peuple sont rebelles et compagnons des voleurs ; chacun d'eux aime les présents, ils courent après les récompenses ; ils ne font point droit à l'orphelin, et la cause de la veuve ne vient point devant eux. 
\VS{24}C'est pourquoi le Seigneur, Yahweh des armées, le Puissant d'Israël dit : Ah ! Je me satisferai en punissant mes adversaires, et je me vengerai de mes ennemis. 
\VS{25}Et je remettrai ma main sur toi, je refondrai tes scories comme avec la potasse, et j'ôterai tout ton étain ;
\VS{26}mais je rétablirai tes juges, tels qu'ils étaient autrefois, et tes conseillers, tels qu'ils étaient au commencement\FTNT{Dans le royaume messianique, le gouvernement théocratique sera restauré et la fonction des juges sera rétablie (voir livre des Juges ; Mt. 19:28 ; 1 Co. 6:2-3).}. Après cela, on t'appellera cité de la justice, ville fidèle.
\VS{27}Sion sera rachetée par la droiture et ceux qui s'y convertiront seront rachetés par la justice.
\VS{28}Mais les rebelles et les pécheurs seront détruits ensemble, et ceux qui abandonnent Yahweh seront consumés.
\VS{29}Car on sera honteux à cause des térébinthes que vous avez désirés, et vous rougirez à cause des jardins que vous avez choisis\FTNT{Des cultes idolâtres avaient lieu autour des térébinthes et dans des jardins (De. 16:21 ; Es. 57:4-5 ; Es. 65:3 ; Jé. 2:20 ; Ez. 20:28 ; Os. 4:13).}.
\VS{30}Car vous serez comme le térébinthe dont le feuillage tombe, et comme un jardin qui n'a pas d'eau.
\VS{31}Et le fort sera de l'étoupe, et son œuvre une étincelle ; et tous deux brûleront ensemble, et il n'y aura personne pour éteindre le feu.
\Chap{2}
\TextTitle{Vision du règne messianique}
\VerseOne{}La parole qu'Esaïe, fils d'Amots a vue touchant Juda et Jérusalem.
\VS{2}Or il arrivera, dans les derniers jours\FTNT{Voir Ge. 49:1-2.}, que la montagne de la maison de Yahweh sera affermie au  sommet des montagnes, qu'elle sera élevée par-dessus les collines et que toutes les nations y afflueront.
\VS{3}Et plusieurs peuples iront et diront : Venez, et montons à la montagne de Yahweh, à la maison du Dieu de Jacob ; et il nous instruira ses voies, et nous marcherons dans ses sentiers ; car la loi sortira de Sion, et la parole de Yahweh sortira de Jérusalem. 
\VS{4}Il exercera le jugement parmi les nations, et reprendra plusieurs peuples. De leurs épées ils forgeront des hoyaux, et de leurs lances des serpes ; une nation ne lèvera plus l'épée contre une autre et ils ne s'adonneront plus à la guerre.
\VS{5}Venez, ô maison de Jacob, et marchons dans la lumière de Yahweh.
\TextTitle{L'orgueilleux abaissé au jour de Yahweh}
\VS{6}Certes tu as rejeté ton peuple, la maison de Jacob, parce qu'ils se sont remplis d'orient et adonnés à la divination comme les Philistins, et parce qu'ils s'allient aux enfants des étrangers\FTNT{De. 18:8-13 ; Os. 13:2 ; Mi. 5:11-13.}.
\VS{7}Son pays est rempli d'argent et d'or, et il n'y a pas de fin à ses trésors ; son pays est rempli de chevaux, et il n'y a pas de fin à ses chars.
\VS{8}Son pays est rempli d'idoles ; ils se prosternent devant l'ouvrage de leurs mains et devant ce que leurs doigts ont fabriqué.
\VS{9}Et ceux du commun sont abattus, et les personnes de qualité sont abaissées ; ne leur pardonne donc point.
\VS{10}Entre dans les rochers et cache-toi dans la poussière, à cause de la frayeur de Yahweh, et à cause de la gloire de sa majesté\FTNT{Ap. 6:15-16.}.
\VS{11}Les yeux hautains des hommes seront abaissés et les hommes qui s'élèvent seront humiliés, Yahweh sera seul haut élevé en ce jour-là.
\VS{12}Car il y a un jour assigné par Yahweh des armées contre tout homme orgueilleux et hautain, et contre tout homme qui s'élève, afin qu'il soit abaissé ;
\VS{13}contre tous les cèdres du Liban, hauts et élevés, et contre tous les chênes de Basan ;
\VS{14}contre toutes les hautes montagnes, et contre toutes les collines élevées ;
\VS{15}contre toutes les hautes tours, et contre toutes les murailles fortes ;
\VS{16}contre tous les navires de Tarsis, et contre toutes les peintures de plaisance.
\VS{17}Et l'arrogance des hommes sera humiliée, et les hommes qui s'élèvent seront abaissés :
\VS{18}Yahweh seul sera élevé en ce jour-là. Quant aux idoles, elles tomberont toutes.
\VS{19}Et les hommes entreront dans les cavernes des rochers et dans les trous de la terre, à cause de la frayeur de Yahweh et à cause de sa gloire magnifique, lorsqu'il se lèvera pour faire trembler la terre.
\VS{20}En ce jour-là, les hommes jetteront aux taupes et aux chauves-souris leurs idoles d'argent et leurs idoles d'or, qu'ils s'étaient faites pour se prosterner devant elles ;
\VS{21}et ils entreront dans les fentes des rochers et dans les creux des rochers, à cause de la frayeur de Yahweh, et à cause de sa gloire magnifique, quand il se lèvera pour punir la terre.
\VS{22}Retirez-vous de l'homme, dans les narines duquel il n'y a qu'un souffle : Car quel cas mérite-t-il qu'on en fasse ?
\Chap{3}
\TextTitle{Le péché, cause de dissolution nationale}
\VerseOne{}Car voici, le Seigneur, Yahweh des armées, va ôter de Jérusalem et de Juda tout appui et toute ressource, toute ressource de pain et toute ressource d'eau.
\VS{2}L'homme fort et l'homme de guerre, le juge et le prophète, le devin et l'ancien,
\VS{3}le chef de cinquante et l'homme d'autorité, le conseiller, l'expert d'entre les artisans et l'habile enchanteur.
\VS{4}Et je leur donnerai de jeunes gens pour chefs, et des enfants domineront sur eux.
\VS{5}Le peuple sera opprimé ; l'un opprimera l'autre, chacun son prochain. Le jeune homme se portera arrogamment contre le vieillard, et l'homme de rien contre l'honorable.
\VS{6}Même un homme ira jusqu'à saisir son frère dans la maison paternelle et lui dira : Tu as un manteau, sois notre chef ! Et prends en main ces ruines !
\VS{7}Ce jour même il répondra : Je ne suis pas médecin, et dans ma maison il n'y a ni pain ni manteau ; ne m'établissez donc pas chef du peuple.
\VS{8}Certes Jérusalem est renversée, et Juda est tombée, parce que leurs langues et leurs actions sont contre Yahweh, pour braver les regards de sa gloire.
\VS{9}L'aspect de leur visage témoigne contre eux, ils publient leur péché comme Sodome, ils ne le cachent pas. Malheur à leur âme, car ils ont attiré le mal sur eux !
\VS{10}Dites au juste que du bien lui arrivera, car il mangera le fruit de ses œuvres.
\VS{11}Malheur au méchant qui ne cherche qu'à faire le mal, car la rétribution de ses mains lui sera rendue.
\VS{12}Quant à mon peuple, il a pour oppresseur des enfants, et des femmes dominent sur lui. Mon peuple, ceux qui te conduisent t'égarent, ils corrompent le chemin dans lequel tu marches.
\VS{13}Yahweh se présente pour plaider, il se tient debout pour juger les peuples.
\VS{14}Yahweh entre en jugement avec les anciens de son peuple et avec ses chefs ; car vous avez brouté la vigne, et ce que vous avez ravi au pauvre est dans vos maisons.
\VS{15}Que vous revient-il de fouler mon peuple, et d'écraser le visage des affligés ? Dit le Seigneur, Yahweh des armées.
\TextTitle{Les filles hautaines de Sion}
\VS{16}Yahweh dit aussi : Parce que les filles de Sion sont hautaines, et qu'elles marchent le cou tendu et les yeux pleins de convoitise, parce qu'elles marchent avec une fière démarche faisant du bruit avec leurs pieds,
\VS{17}Yahweh rendra chauve le sommet de la tête des filles de Sion, Yahweh découvrira leur nudité.
\VS{18}En ce temps-là, le Seigneur ôtera l'ornement de leurs anneaux de cheville, et les filets et les croissants ;
\VS{19}les pendants d'oreilles, les bracelets et les voiles ;
\VS{20}les parures de la tête, les chaînettes des pieds et les ceintures, les boîtes à parfum et les amulettes ;
\VS{21}les anneaux et les bagues qui leur pendent sur le nez ;
\VS{22}les vêtements de fête et les larges tuniques, les manteaux et les gibecières ;
\VS{23}les miroirs et les chemises fines, les tiares et les voiles légers.
\VS{24}Et il arrivera qu'au lieu du parfum, il y aura de la puanteur ; au lieu de ceintures, des cordes ; au lieu de cheveux bouclés, des têtes chauves ; au lieu de robes flottantes, des sacs étroits ; et au lieu d'un beau teint, un teint tout hâlé.
\VS{25}Tes hommes tomberont par l'épée et ta force par la guerre.
\VS{26}Et ses portes gémiront et mèneront deuil ; désolée, elle s'assiéra par terre.
\Chap{4}
\TextTitle{Vision du règne messianique\FTNTT{Es. 11:1-16.}}
\VerseOne{}Et en ce jour sept femmes saisiront un seul homme, et diront : Nous mangerons notre pain, et nous nous vêtirons de nos habits ; seulement fais-nous porter ton nom ; ôte notre opprobre.
\VS{2}En ce temps-là, le germe de Yahweh\FTNT{Jésus est le « germe » de Yahweh (Es. 4:2) et le germe de David (Jé. 23:5 ; Za. 3:8 ; Za. 6:12). Ce germe a été placé par la vertu du Saint-Esprit dans le sein d'une vierge (Es. 7:14 ; Lu. 1:34-35) et l'enfant qui naquit d'elle fut appelé « Fils de Dieu » tout en étant le Dieu Tout-Puissant. Il existe de toute éternité en forme de Dieu (Jn. 1:1 ; Es. 9:5), mais il a été fait chair pour nous sauver (Jn. 1:14. 1 Ti. 3:16).
C'est le plus grand des miracles et la démonstration de sa divinité, de sa sagesse et de son amour envers les hommes.
} sera plein de noblesse et de gloire, et le fruit de la terre plein de grandeur et d'excellence pour les réchappés d'Israël.
\VS{3}Et il arrivera que les restes de Sion, et les restes de Jérusalem, seront appelés saints; et ceux de Jérusalem seront inscrits parmi les vivants\FTNT{Es. 10:20-22 ; Ro. 9:27 ; Ro. 11:5 ;}.
\VS{4}Quand le Seigneur aura lavé la souillure des filles de Sion, et purifié Jérusalem du sang qui est au milieu d'elle, par l'esprit de jugement et par l'esprit qui consume;
\VS{5}aussi Yahweh créera, sur toute l'étendue du mont Sion et sur ses assemblées, une nuée avec une fumée pendant le jour, et une splendeur de feu flamboyant pendant la nuit, car la gloire se répandra partout.
\VS{6}Et il y aura un tabernacle pour donner de l'ombre contre la chaleur du jour, pour servir de refuge et d'asile contre la tempête et la pluie\FTNT{Ap. 21:3.}.
\Chap{5}
\TextTitle{Israël, vigne de Yahweh}
\VerseOne{}Je chanterai maintenant pour mon bien-aimé le cantique de mon bien-aimé sur sa vigne. Mon bien-aimé avait une vigne sur un coteau fertile.
\VS{2}Il l'environna d'une haie, en ôta les pierres, et y planta des ceps exquis ; il bâtit une tour au milieu d'elle, et il y creusa aussi une cuve. Puis il espéra qu'elle produirait des raisins, mais elle a produit des grappes sauvages\FTNT{Lu. 13:6-9.}.
\VS{3}Maintenant donc, vous habitants de Jérusalem et vous hommes de Juda, jugez, je vous prie, entre moi et ma vigne.
\VS{4}Qu'y avait-il encore à faire à ma vigne que je ne lui aie fait ? Pourquoi, quand j'ai attendu qu'elle produirait des raisins, a-t-elle produit des grappes sauvages ?
\VS{5}Maintenant donc je vous dirai ce que je vais faire à ma vigne : J'ôterai sa haie, et elle sera broutée ; je romprai sa clôture et elle sera foulée.
\VS{6}Et je la réduirai en désert, elle ne sera plus taillée, ni cultivée ; les ronces et les épines y croîtront ; et je commanderai aux nuées qu'elles ne laissent plus tomber de pluie sur elle.
\VS{7}Or la maison d'Israël est la vigne de Yahweh des armées, et les hommes de Juda sont la plante en laquelle il prenait plaisir. Il en attendait de la droiture, et voici du saccagement ! De la justice, et voici des cris de détresse !
\TextTitle{Six malheurs en punition de l'infidélité d'Israël}
\VS{8}Malheur à ceux qui ajoutent maison à maison, et qui joignent champ à champ, jusqu'à ce qu'il n'y ait plus d'espace et qu'ils habitent seuls au milieu du pays.
\VS{9}Yahweh des armées m'a fait entendre : Certainement, ces maisons nombreuses seront réduites en désolation, ces maisons grandes et belles seront sans habitants.
\VS{10}Même dix arpents de vigne ne produiront qu'un bath, et un homer de semence ne produira qu'un épha.
\VS{11}Malheur à ceux qui se lèvent de bon matin, qui recherchent les boissons fortes, qui demeurent jusqu'au soir, et jusqu'à ce que le vin les échauffe !
\VS{12}La harpe et le luth, le tambourin, la flûte et le vin sont dans leurs festins ; mais ils ne regardent pas l'œuvre de Yahweh, et ils ne voient pas l'ouvrage de ses mains.
\VS{13}C'est pourquoi mon peuple sera emmené captif, parce qu'il n'a pas de connaissance\FTNT{2 R. 24:14-16 ; Os. 4:6.} ; et les plus honorables parmi eux seront des pauvres qui mourront de faim, et leur multitude sera asséchée par la soif.
\VS{14}C'est pourquoi le scheol s'élargit, il ouvre sa gueule outre mesure ; et sa magnificence y descend, sa multitude, sa pompe et tous ceux qui s'y réjouissent.
\VS{15}Ceux du commun seront abattus, les personnes de qualité seront humiliées, et les yeux des hautains seront humiliés.
\VS{16}Et Yahweh des armées sera haut élevé en jugement, et le Dieu saint sera sanctifié dans la justice.
\VS{17}Les agneaux paîtront selon qu'ils seront parqués, et les étrangers dévoreront les champs désolés des riches.
\VS{18}Malheur à ceux qui tirent l'iniquité avec des cordes de vanité, et le péché avec les traits d'un char,
\VS{19}et qui disent : Qu'il hâte et qu'il fasse venir son œuvre bientôt, afin que nous la voyions ! Que le conseil du Saint d'Israël s'avance et vienne, afin que nous le connaissions !
\VS{20}Malheur à ceux qui appellent le mal bien et le bien mal\FTNT{Mi. 7:2.} ; qui font les ténèbres lumière, et la lumière ténèbres ; qui font l'amertume douceur, et la douceur amertume.
\VS{21}Malheur à ceux qui sont sages à leurs yeux, en se considérant eux-mêmes intelligents !
\VS{22}Malheur à ceux qui sont forts pour boire le vin et vaillants pour mêler des boissons fortes ;
\VS{23}qui justifient le méchant pour des présents, et qui ôtent à chacun des justes sa justice.
\VS{24}C'est pourquoi, comme le flambeau de feu consume le chaume, et la flamme consume l'herbe sèche, ainsi leur racine sera comme la pourriture, et leur fleur sera détruite comme la poussière ; parce qu'ils ont rejeté la loi de Yahweh des armées, et ils ont méprisé la parole du Saint d'Israël.
\VS{25}C'est pourquoi la colère de Yahweh s'enflamme contre son peuple, il étend sa main sur lui, et il le frappe ; les montagnes tremblent, et leurs cadavres ont été mis en pièces au milieu des rues. Malgré tout cela, sa colère ne se détourne pas, mais sa main est encore étendue.
\VS{26}Il élève une bannière pour les nations éloignées, et il siffle à chacune d'elles depuis les extrémités de la terre ; et voici chacune viendra promptement et légèrement.
\VS{27}Nul n'est fatigué, nul ne chancelle de lassitude, personne ne sommeille ni ne dort ; et la ceinture de leurs reins ne sera point déliée, et la courroie de leurs souliers ne sera point rompue.
\VS{28}Leurs flèches sont aiguës et tous leurs arcs tendus ; les sabots de leurs chevaux ressemblent à des cailloux, et les roues de leurs chars à un tourbillon.
\VS{29}Leur rugissement est comme celui d'un vieux lion ; ils rugissent comme des lionceaux ; ils grondent et saisissent la proie, il l'emportent et personne ne vient à son secours.
\VS{30}En ce jour-là, on mènera un bruit sur lui, semblable au mugissement de la mer ; en regardant la terre, on ne verra que ténèbres et détresse ; la lumière sera obscurcie dans le ciel.
\Chap{6}
\TextTitle{Révélation de Yahweh à Esaïe}
\VerseOne{}L'année de la mort du roi Ozias, je vis le Seigneur assis sur un trône haut et élevé, et les pans de sa robe remplissaient le temple\FTNT{2 Ch. 26:23.}.
\VS{2}Les séraphins se tenaient au-dessus de lui ; et chacun d'eux avait six ailes ; deux dont ils se couvraient la face, deux dont ils se couvraient les pieds et deux dont ils se servaient pour voler.
\VS{3}Et ils criaient l'un à l'autre, et disaient : Saint, saint, saint est Yahweh des armées ! Toute la terre est pleine de sa gloire !
\VS{4}Et les poteaux des seuils furent ébranlés dans leurs fondements par la voix de celui qui criait ; et la maison fut remplie de fumée.
\VS{5}Alors je dis : Malheur à moi ! Je suis perdu, car je suis un homme dont les lèvres sont impures, j'habite au milieu d'un peuple dont les lèvres sont impures et mes yeux ont vu le Roi, Yahweh des armées\FTNT{Jg. 13:21-22.}.
\VS{6}Mais l'un des séraphins vola vers moi, tenant à la main un charbon ardent, qu'il avait pris sur l'autel avec des pincettes.
\VS{7}Il en toucha ma bouche, et dit : Voici, ceci a touché tes lèvres, c'est pourquoi ton iniquité est ôtée, et la propitiation est faite pour ton péché.
\VS{8}Puis j'entendis la voix du Seigneur, disant : Qui enverrai-je et qui marchera pour nous ? Je répondis : Me voici, envoie-moi.
\TextTitle{Mission d'Esaïe}
\VS{9}Et il dit : Va et dis à ce peuple : En entendant vous entendrez, mais vous ne comprendrez point ; et en voyant vous verrez, mais vous n'apercevrez point.
\VS{10}Engraisse le cœur de ce peuple, et rends ses oreilles pesantes, et bouche-lui les yeux ; de peur qu'il ne voie de ses yeux, et qu'il n'entende de ses oreilles, et que son cœur ne comprenne, et qu'il ne se convertisse, et qu'il ne recouvre la santé\FTNT{Mt. 13:15 ; Mc. 4:12 ; Jn. 12:40 ; Ac. 28:27.}.
\VS{11}Je dis : Jusqu'à quand, Seigneur ? Et il répondit : Jusqu'à ce que les villes soient dévastées, jusqu'à ce qu'il n'y ait plus d'habitants, ni d'hommes dans les maisons, et que la terre soit mise en entière désolation ;
\VS{12} et que Yahweh ait dispersé au loin les hommes, et que l'abandon ait été grand au milieu du pays.
\VS{13}Toutefois s'il y reste un dixième des habitants, ils reviendront pour être la proie des flammes. Mais comme le térébinthe et le chêne conservent leur tronc quand ils sont abattus, une sainte postérité renaîtra de ce peuple\FTNT{Ro. 11:17-25.}.
\Chap{7}
\TextTitle{Retsin et Pékach complote contre Juda}
\VerseOne{}Or il arriva du temps d'Achaz, fils de Jotham, fils d'Ozias, roi de Juda, que Retsin, roi de Syrie, et Pékach, fils de Remalia, roi d'Israël, montèrent contre Jérusalem pour lui faire la guerre ; mais ils ne purent l'assiéger.
\VS{2}Et on rapporta à la maison de David : La Syrie s'est reposée sur Ephraïm. Et le cœur d'Achaz, et le cœur de son peuple furent ébranlés comme les arbres des forêts qui sont ébranlés par le vent.
\VS{3}Alors Yahweh dit à Esaïe : Sors maintenant au devant d'Achaz, toi et Schear-Jaschub, ton fils, vers l'extrémité de l'aqueduc de l'étang supérieur, sur la route du champ du foulon.
\VS{4}Et dis-lui : Prends garde à toi, et demeure tranquille, ne crains point, et que ton cœur ne devienne point lâche à cause des deux queues de ces tisons fumants, à cause de l'ardeur, dis-je, de la colère de Retsin et de la Syrie, et du fils de Remalia,
\VS{5}de ce que la Syrie délibère avec Ephraïm et le fils de Remalia de te faire du mal, en disant :
\VS{6}Montons contre Juda, assiégeons la ville, battons-la en brèche, et établissons pour roi le fils de Tabeel au milieu d'elle.
\VS{7}Ainsi parle le Seigneur, Yahweh : Cela n'aura point d'effet, et cela ne se fera point.
\VS{8}Car la tête de la Syrie c'est Damas, et le chef de Damas c'est Retsin. Encore soixante-cinq ans, Ephraïm sera froissé pour n'être plus un peuple.
\VS{9}Et la tête d'Ephraïm c'est la Samarie, et le chef de la Samarie c'est le fils de Remalia. Si vous ne croyez pas, certainement vous ne serez point affermis.
\TextTitle{Annonce de la naissance d'Emmanuel}
\VS{10}Et Yahweh parla de nouveau à Achaz, en disant :
\VS{11}Demande pour toi un signe à Yahweh ton Dieu, demande-le, soit dans les bas lieux, soit dans les lieux élevés.
\VS{12}Et Achaz répondit : Je ne demanderai rien, et je ne tenterai point Yahweh.
\VS{13}Alors Esaïe dit : Ecoutez maintenant, ô maison de David ! Est-ce trop peu pour vous de lasser les hommes, que vous lassiez aussi mon Dieu ?
\VS{14}C'est pourquoi le Seigneur lui-même vous donnera un signe : Voici, une vierge sera enceinte, et elle enfantera un fils, et elle lui donnera le nom d'Emmanuel\FTNT{Le nom « Emmanuel » est dérivé de l'hébreu « Immanuw'el » qui signifie « Dieu est avec nous ». Jésus a dit aux disciples dans Mt. 28:20 : « Et moi, je suis avec vous tous les jours jusqu'à la fin des temps ». Jésus est Emmanuel, Dieu avec nous jusqu'à la fin des temps.}.
\VS{15}Il mangera du lait et du miel, jusqu'à ce qu'il sache rejeter le mal et choisir le bien.
\VS{16}Mais avant que l'enfant sache rejeter le mal et choisir le bien, la terre que tu as en détestation sera abandonnée par ses deux rois.
\TextTitle{Prophétie sur l'imminente invasion de Juda\FTNTT{2 Ch. 28:1-20.}}
\VS{17}Yahweh fera venir sur toi, sur ton peuple et sur la maison de ton père, par le roi d'Assyrie, des jours tels qu'il n'y en a point eu de semblable depuis le jour où Ephraïm s'est séparé de Juda.
\VS{18}Et il arrivera qu'en ce jour-là, Yahweh sifflera aux mouches qui sont à l'extrémité des ruisseaux d'Egypte, et aux abeilles qui sont au pays d'Assyrie.
\VS{19}Elles viendront, et se poseront dans toutes les vallées désertes, et dans les fentes des rochers, et par tous les buissons, et par tous les halliers.
\VS{20}En ce jour-là, le Seigneur rasera avec le rasoir pris à louage au-delà du fleuve, avec le roi d'Assyrie, la tête et les poils des pieds, et il enlèvera aussi la barbe\FTNT{2 R. 16:5-9.}.
\VS{21}Et il arrivera, en ce jour-là, qu'un homme nourrira une jeune vache et deux brebis.
\VS{22}Et il arrivera que de l'abondance du lait qu'elles rendront, il mangera du beurre ; car tous ceux qui seront restés dans le pays mangeront du beurre et du miel.
\VS{23}Et il arrivera, en ce jour-là, que tout lieu où il y aura mille vignes, valant mille sicles d'argent, sera réduit en ronces et en épines.
\VS{24}On y entrera avec des flèches et avec l'arc, car tout le pays ne sera que ronces et épines.
\VS{25}Et dans toutes les montagnes que l'on cultivait avec la bêche, on ne craindra plus de voir des ronces et des épines ; mais on y lâchera les bœufs, et la brebis en foulera le sol.
\Chap{8}
\TextTitle{Annonce de la défaite de Damas et de la Samarie}
\VerseOne{}Et Yahweh me dit : Prends un grand rouleau et écris dessus en grosses lettres : Qu'on se dépêche de butiner, qu'on se hâte de piller.
\VS{2}Et je pris avec moi des témoins fidèles : Urie, le sacrificateur, et Zacharie, fils de Bérékia.
\VS{3}Puis je m'étais approché de la prophétesse ; elle conçut et elle enfanta un fils. Et Yahweh me dit : Donne-lui pour nom Maher-Schalal-Chasch-Baz\FTNT{« Maher-Schalal-Chasch-Baz » signifie « rapide au butin, rapide sur la proie ».}.
\VS{4}Car avant que l'enfant sache dire : Mon père ! Ma mère ! On enlèvera la puissance de Damas et le butin de Samarie, devant le roi d'Assyrie.
\VS{5}Et Yahweh continua encore de me parler, en disant :
\VS{6}Parce que ce peuple a rejeté les eaux de Siloé qui coulent doucement, et qu'il s'est réjoui au sujet de Retsin, et du fils de Remalia,
\VS{7}à cause de cela, voici, le Seigneur va faire monter contre eux les puissantes et grandes eaux du fleuve : Le roi d'Assyrie et toute sa gloire. Il s'élèvera partout au-dessus de son lit, et il se répandra sur toutes ses rives.
\VS{8}Et il pénétrera dans Juda, il débordera et inondera, il atteindra jusqu'au cou. Et les étendues de ses ailes rempliront la largeur de ton pays, ô Emmanuel !
\TextTitle{Exhortation aux disciples de Yahweh à rester fidèles}
\VS{9}Alliez-vous, peuples ! Et vous serez brisés ; prêtez l'oreille, vous tous qui êtes d'un pays éloigné ! Equipez-vous, et vous serez brisés ; équipez-vous, et vous serez brisés.
\VS{10}Prenez conseil, et il sera dissipé ; dites la parole, et elle sera sans effet : Car Dieu est avec nous.
\VS{11}Car ainsi m'a parlé Yahweh, avec une main forte, et il m'instruisit de ne point aller par le chemin de ce peuple-ci, en me disant :
\VS{12}Ne dites point : Conjuration, toutes les fois que ce peuple dit conjuration ; ne craignez point ce qu'il craint, et ne vous en épouvantez point.
\VS{13}Sanctifiez Yahweh des armées, lui-même, c'est lui que vous devez craindre et redouter.
\VS{14}Et il sera un sanctuaire, mais aussi une pierre d'achoppement\FTNT{Yahweh s'est présenté comme une pierre d'achoppement et un rocher de scandale. En Es. 44:8 il affirme d'ailleurs ne pas connaître d'autre rocher que lui. Esaïe n'est pas le seul prophète à qui le Seigneur s'est révélé comme étant une pierre et un rocher. Dans le Ps. 118:22-23, il est dit : « La pierre qu'ont rejetée ceux qui bâtissaient est devenue la principale de l'angle ». Daniel et Zacharie ont également prophétisé au sujet de cette pierre : « Tu regardais, lorsqu'une pierre se détacha sans le secours d'aucune main, frappa les pieds de fer et d'argile de la statue, et les mit en pièces. Mais la pierre qui avait frappé la statue devint une grande montagne, et remplit toute la terre » (Da. 2:34-35). « Car voici, pour ce qui est de la pierre que j'ai placée devant Josué, il y a sept yeux sur cette seule pierre ; voici, je graverai moi-même ce qui doit y être gravé, dit Yahweh des armées; et j'enlèverai l'iniquité de ce pays, en un jour » (Za. 3:9). Ces prophéties se sont accomplies en Jésus-Christ, l'Agneau de Dieu qui ôte le péché du monde (Jn. 1:29). Le Seigneur s'est d'ailleurs clairement identifié à la pierre angulaire, affirmant ainsi sa divinité (Lu. 20:17-19). En Mt. 16:18, il s'est présenté comme le rocher inébranlable sur lequel il allait bâtir son Eglise. De plus, il est à noter que dans le livre de l'Apocalypse, l'Agneau possède sept yeux comme la pierre vue par Zacharie (Ap. 5:6). Ces sept yeux sont aussi les sept lampes du chandelier d'or que Zacharie et Jean avaient également vues (Za. 4:2 ; Ap. 4:5 ). Or le chiffre sept symbolise la plénitude et la perfection divines. Esaïe prophétisa encore en ces termes : « Voici, j'ai mis pour fondement en Sion une pierre, une pierre éprouvée, une pierre angulaire de prix, solidement posée; celui qui la prendra pour appui n'aura point hâte de fuir » (Es. 28:16). Les écrits de la Nouvelle Alliance attestent l'accomplissement de cette prophétie en Jésus-Christ, notamment par la bouche de Paul et de Pierre : « Vous avez été édifiés sur le fondement des apôtres et des prophètes, Jésus-Christ lui-même étant la pierre angulaire » (Ep. 2:20). « Car personne ne peut poser un autre fondement que celui qui a été posé, savoir Jésus-Christ » (1 Co. 3:11). « Approchez-vous de lui, pierre vivante, rejetée par les hommes, mais choisie et précieuse devant Dieu ; et vous-mêmes, comme des pierres vivantes, édifiez-vous pour former une maison spirituelle, un saint sacerdoce, afin d'offrir des victimes spirituelles, agréables à Dieu, par Jésus-Christ ». (1 Pi. 2:4-5).}, un rocher de scandale pour les deux maisons d'Israël, un filet et un piège pour les habitants de Jérusalem.
\VS{15}Plusieurs d'entre eux trébucheront, ils tomberont et se briseront, ils seront enlacés et pris.
\VS{16}Enveloppe ce témoignage, scelle cette loi\FTNTT{"towrah" ou "torah" en hébreu.} parmi mes disciples.
\VS{17}Je m'attends à Yahweh, qui cache sa face à la maison de Jacob, et je regarde à lui.
\VS{18}Me voici, avec les enfants que Yahweh m'a donnés, pour être un signe et un miracle en Israël, de la part de Yahweh des armées, qui habite sur la montagne de Sion.
\VS{19}Si l'on vous dit : Consultez ceux qui évoquent les morts et les diseurs de bonne aventure, qui poussent des sifflements et des soupirs, répondez : Un peuple ne consultera-t-il pas son Dieu ? S'adressera-t-il aux morts en faveur des vivants ?
\VS{20}A la loi et au témoignage ! Si l'on ne parle pas ainsi, il n'y aura certainement point d'aurore pour le peuple.
\VS{21}Et il sera errant dans le pays, accablé et affamé ; et il arrivera que dans sa faim, il s'irritera, maudira son roi et son Dieu, et tournera les yeux en haut ;
\VS{22}puis il regardera vers la terre, et voici, il n'y aura que détresse, ténèbres et de sombres angoisses : Il sera enfoncé dans l'obscurité.
\VS{23}Mais l'obscurité ne sera pas autant qu'elle avait été dans son humiliation ; quand au commencement, il affligea légèrement le pays de Zabulon et le pays de Nephthali, et ensuite, l'affligea plus sévèrement près de la mer, au-delà du Jourdain, dans la Galilée des Gentils.
\Chap{9}
\TextTitle{Annonce de la naissance et du règne du Messie}
\VerseOne{}Le peuple qui marchait dans les ténèbres voit une grande lumière, et la lumière resplendit sur ceux qui habitaient le pays de l'ombre de la mort\FTNT{Mt. 4:15-16.}.
\VS{2}Tu multiplies la nation, tu lui accordes de grandes joies, ils se réjouissent devant toi, comme on se réjouit à la moisson, comme on s'égaye quand on partage le butin.
\VS{3}Car tu as mis en pièces le joug dont il était chargé, et le bâton dont on lui battait ordinairement les épaules, et la verge de celui qui l'opprimait, comme au jour de Madian.
\VS{4}Parce que toute bataille de guerrier se fait dans un bruit confus, et que le vêtement est vautré dans le sang ; mais ceci sera comme un embrasement, quand le feu dévore quelque chose.
\VS{5}Car un enfant nous est né, un Fils nous a été donné\FTNT{Jésus-Christ est 100\% Dieu  et 100\% homme. Il  existe depuis toute éternité en tant que Dieu. Il est devenu homme au moment de son incarnation (Ph. 2 :5-7).}, et l'empire reposera sur son épaule : On l'appellera l'Admirable, le Conseiller, le Dieu Puissant, le Père d'éternité\FTNT{Philippe, disciple de Jésus-Christ voulait rencontrer le Père. Il posa au Seigneur cette question « Seigneur, montre-nous le Père, et cela nous suffit » (Jn. 14:8). Jésus lui répondit : « Il y a si longtemps que je suis avec vous, et tu ne m'as pas connu, Philippe ! » (Jn. 14:9).}, le Prince de paix,
\VS{6}pour accroître l'empire, et une paix sans fin au trône de David et à son royaume, pour l'affermir et le soutenir par le droit et par la justice, dès maintenant et à toujours\FTNT{Lu. 1:32-33.}. Voilà ce que fera le zèle de Yahweh des armées.
\TextTitle{Jugement sur le royaume du nord}
\VS{7}Le Seigneur envoie une parole à Jacob, et elle tombe sur Israël\FTNT{Ge. 32:28.}.
\VS{8}Et tout le peuple en aura connaissance, Ephraïm et les habitants de Samarie, qui disent avec orgueil et avec un cœur hautain :
\VS{9}Des briques sont tombées, mais nous bâtirons en pierres de taille ; des sycomores ont été coupés, mais nous les changerons en cèdres.
\VS{10}Yahweh élèvera contre eux les ennemis de Retsin, et il armera les ennemis d'Israël ;
\VS{11}la Syrie à l'orient, et les Philistins à l'occident ; et ils dévoreront Israël à gueule ouverte. Malgré tout cela, sa colère ne s'apaise point, et sa main est encore étendue.
\VS{12}Parce que le peuple ne revient pas à celui qui le frappe, et il ne cherche pas Yahweh des armées.
\VS{13}A cause de cela Yahweh retranchera d'Israël en un seul jour la tête et la queue, la branche de palmier et le roseau.
\VS{14}L'ancien et le magistrat, c'est la tête ; et le prophète qui enseigne le mensonge, c'est la queue.
\VS{15}Ceux donc qui font croire à ce peuple qu'il est heureux sont des séducteurs\FTNT{1 Ti. 4:1 ; Tit. 1:10.}; et ceux qui se laissent diriger par eux se perdent.
\VS{16}C'est pourquoi le Seigneur ne saurait prendre plaisir à leurs jeunes hommes ni avoir pitié de leurs orphelins et de leurs veuves, car tous sont des hypocrites et des méchants, et toute bouche ne profère que des infamies. Malgré tout cela, sa colère ne s'apaise point et sa main est encore étendue.
\VS{17}Car la méchanceté consume comme un feu, elle dévore les ronces et les épines ; elle embrase l'épaisseur de la forêt, d'où s'élèvent des colonnes de fumée.
\VS{18}A cause de la fureur de Yahweh des armées, la terre est obscurcie, et le peuple est comme la proie du feu ; nul n'a compassion de son frère.
\VS{19}On pille à droite, et l'on a faim ; on dévore à gauche, et l'on n'est pas rassasié ; chacun mange la chair de son bras.
\VS{20}Manassé dévore Ephraïm, Ephraïm dévore Manassé, et ensemble ils fondent sur Juda. Malgré tout cela, sa colère ne s'apaise point, et sa main est encore étendue.
\Chap{10}
\VerseOne{}Malheur à ceux qui décrètent des ordonnances iniques, et à ceux qui écrivent pour ordonner l'oppression,
\VS{2}pour refuser la justice aux pauvres et ravir leur droit aux malheureux de mon peuple, afin d'avoir les veuves pour leur butin, et de piller les orphelins !
\VS{3}Et que ferez-vous au jour de la visitation, et de la ruine éclatante qui viendra de loin ? Vers qui fuirez-vous pour avoir du secours et où laisserez-vous votre gloire\FTNT{Os. 9:7 ; Mt. 24:17-21 ; Lu. 19:41-44.} ?
\VS{4}Les uns seront courbés parmi les prisonniers, les autres tomberont parmi les morts. Malgré tout cela, sa colère ne s'apaise point, et sa main est encore étendue.
\TextTitle{Jugement sur l'Assyrie}
\VS{5}Malheur à l'Assyrie, verge de ma colère ! La verge dans leur main c'est l'instrument de ma colère.
\VS{6}Je l'ai envoyé contre une nation impie, et je l'ai fait marcher contre le peuple de ma fureur, afin qu'il se livre au pillage et fasse du butin, pour qu'il le foule aux pieds comme la boue des rues.
\VS{7}Mais il n'en juge pas ainsi, et ce n'est pas là la pensée de son cœur ; il ne songe qu'à détruire, qu'à exterminer beaucoup de nations.
\VS{8}Car il dit : Mes princes ne sont-ils pas autant de rois ?
\VS{9}Calno n'est-elle pas comme Carkemisch ? Hamath n'est-elle pas comme Arpad ? Et Samarie n'est-elle pas comme Damas ?
\VS{10}Puisque ma main a soumis les royaumes qui avaient des idoles, où il y avait plus d'images taillées qu'à Jérusalem et à Samarie,
\VS{11}ne ferai-je pas aussi à Jérusalem et à ses dieux, comme j'ai fait à Samarie et à ses idoles ?
\VS{12}Mais il arrivera que, quand le Seigneur aura achevé toute son œuvre sur la montagne de Sion et à Jérusalem, je punirai le roi d'Assyrie pour le fruit de son cœur orgueilleux, et pour la gloire de ses regards hautains.
\VS{13}Parce qu'il dit : C'est par la force de ma main que j'ai agi, c'est par ma sagesse, car je suis intelligent ; j'ai reculé les bornes des peuples, et j'ai pillé ce qu'ils avaient de plus précieux ; et comme un homme vaillant, j'ai fait descendre ceux qui étaient assis.
\VS{14}Ma main a trouvé les richesses des peuples, comme on trouve un nid ; comme on rassemble des œufs délaissés, ainsi ai-je rassemblé toute la terre ; nul n'a remué l'aile, ni ouvert le bec, ni poussé un cri.
\VS{15}La hache se glorifie-t-elle envers celui qui s'en sert ? Ou la scie s'élève-t-elle au-dessus de celui qui la manie ? Comme si la verge faisait mouvoir celui qui la lève, et que le bâton se levait comme s'il n'était pas du bois !
\VS{16}C'est pourquoi le Seigneur, Yahweh des armées, enverra la maigreur sur ses hommes gras ; et sous sa gloire éclatera l'embrasement d'un feu.
\VS{17}Car la lumière d'Israël deviendra un feu, et son Saint une flamme qui embrasera et consumera ses épines et ses ronces tout en un jour ;
\VS{18}et il consumera la gloire de sa forêt et de ses campagnes, depuis l'âme jusqu'à la chair. Il en sera comme quand celui qui porte l'étendard est défait.
\VS{19}Le reste des arbres de sa forêt pourra être compté, et un enfant en écrirait le nombre.
\TextTitle{Conversion et délivrance du reste d'Israël}
\VS{20}Et il arrivera en ce jour-là, que le reste d'Israël et les réchappés de la maison de Jacob ne s'appuieront plus sur celui qui les frappait, mais ils s'appuieront avec confiance sur Yahweh, le Saint d'Israël.
\VS{21}Le reste se convertira, le reste, dis-je, de Jacob se convertira au Dieu puissant.
\VS{22}Car quand ton peuple, ô Israël, serait comme le sable de la mer, un reste seulement se convertira ; la destruction est résolue, elle fera déborder la justice.
\VS{23}Car la destruction qu'il a résolue, le Seigneur, Yahweh des armées, va l'exécuter au milieu de toute la terre.
\VS{24}C'est pourquoi ainsi parle le Seigneur, Yahweh des armées : Mon peuple qui habites en Sion, ne crains pas le roi d'Assyrie ; il te frappe de la verge, et il lève son bâton sur toi comme faisait l'Egypte.
\VS{25}Mais encore un peu de temps, un peu de temps, et le châtiment cessera, puis ma colère se tournera contre lui pour l'exterminer.
\VS{26}Et Yahweh des armées lèvera le fouet contre lui, comme il frappa Madian au rocher d'Oreb ; et de même qu'il leva son bâton sur la mer, il le lèvera aussi comme contre les Egyptiens.
\VS{27}En ce jour-là, son fardeau sera ôté de dessus ton épaule et son joug de dessus ton cou ; et l'onction fera rompre le joug.
\TextTitle{Défaite des Assyriens\FTNTT{Es. 35-36 ; 37.7.}}
\VS{28}Il marche sur Ajjath, traverse Migron et il met ses bagages à Micmasch.
\VS{29}Ils passent le défilé, ils couchent à Guéba ; Rama est effrayée ; Guibea de Saül prend la fuite.
\VS{30}Pousse des cris, fille de Gallim ! Malheur à toi Anathoth ! Prends garde Laïs !
\VS{31}Madména se disperse, les habitants de Guébim se sauvent en foule.
\VS{32}Encore un jour d'arrêt à Nob, et il menace de sa main la montagne de la fille de Sion, la colline de Jérusalem.
\VS{33}Voici, le Seigneur, Yahweh des armées, brise les rameaux avec force ; et ceux qui sont les plus hauts élevés sont coupés, et les hauts montés sont abaissés.
\VS{34}Et il taille avec le fer les lieux les plus épais de la forêt, et le Liban tombe sous le Puissant.
\Chap{11}
\TextTitle{Rétablissement du règne de David par le Messie}
\VerseOne{}Mais il sortira un rameau du tronc d'Isaï, et un rejeton naîtra de ses racines\FTNT{Mt. 1:6-16 ; Lu. 1:31-32 ; Ro. 15:12 ; Ap. 5:5.}.
\VS{2}L'Esprit de Yahweh reposera sur lui, Esprit de sagesse et d'intelligence, Esprit de conseil et de force, Esprit de connaissance et de crainte de Yahweh\FTNT{Es. 61:1 ; Lu. 4:18}.
\VS{3}Il respirera la crainte de Yahweh, il ne jugera point sur l'apparence et il ne reprendra point sur un ouï-dire\FTNT{Jé. 11:20 ; Mt. 22:16 ; Ap. 2:23.}.
\VS{4}Mais il jugera les pauvres avec justice, et il prononcera avec droiture un jugement sur les malheureux de la terre, et il frappera la terre par la verge de sa bouche, et il fera mourir le méchant par le souffle de ses lèvres\FTNT{Job. 4:9 ; Job. 15:30 ; 2 Thess. 2:8.}.
\VS{5}La justice sera la ceinture de ses reins, et la fidélité, la ceinture de ses flancs\FTNT{Ep. 6:14.}.
\VS{6}Le loup habitera avec l'agneau, et le léopard se couchera avec le chevreau ; le veau, et le lionceau, et le bétail qu'on engraisse seront ensemble, et un petit enfant les conduira.
\VS{7}La jeune vache paîtra avec l'ourse, leurs petits auront un même gîte, et le lion, comme le bœuf, mangera de la paille\FTNT{Es. 65:25.}.
\VS{8}Le nourrisson s'ébattra sur l'antre de l'aspic, et l'enfant sevré mettra sa main dans la caverne du vipère.
\VS{9}Il ne se fera ni tort ni dommage sur toute ma montagne sainte, car la terre sera remplie de la connaissance de Yahweh, comme le fond de la mer des eaux qui le couvrent.
\VS{10}En ce jour-là, les nations rechercheront le rejeton d'Isaï qui sera comme une bannière\FTNT{Jésus, notre bannière doit être élevé afin que les pécheurs soient sauvés (Jn. 3:14-15). Le livre du Cantique des cantiques est une superbe image de l'amour de Dieu manifesté en Jésus-Christ, pour nous son Église, qui sommes sa bien-aimée. « Il m'a fait entrer dans la maison du vin ; et la bannière qu'il déploie sur moi, c'est l'amour » (Ca. 2:4). Le vin dont il est question c'est le Saint-Esprit que le Seigneur déverse sur nous jour après jour et qui nous désaltère spirituellement. « Moïse bâtit un autel, et lui donna pour nom : Yahweh ma bannière » (Ex. 17:15). Autrefois, lors des combats, les différentes armées portaient bien haut leur bannière en tête des troupes pour témoigner de leur appartenance et pour indiquer pour quel pays elles combattaient. En portant en nous Jésus, nous proclamons notre appartenance à Dieu et à son Royaume. Le Ps. 20:6 dit ceci : « Nous nous réjouirons de ton salut, nous lèverons l'étendard au nom de notre Dieu ; Yahweh exaucera tous tes vœux ». Et dans le Ps. 60:6 il est dit : « Tu as donné à ceux qui te craignent une bannière pour qu'elle s'élève à cause de la vérité ». La vérité se trouve en Jésus-Christ qui est le chemin, la vérité et la vie (Jn. 14:6). En élevant Jésus comme notre bannière, nous proclamons l'œuvre parfaite accomplie à la croix. Jésus, notre bannière, est le point de rassemblement des chrétiens de tous horizons. Ce rassemblement forme le corps de Christ, l'Eglise dont nous sommes les membres. Tout comme les douze tribus d'Israël se réunissaient pour combattre, nous nous réunissons tous sous la même bannière. Jésus, notre bannière, est le signe de la victoire contre les puissances des ténèbres. En élevant le nom de Jésus comme une bannière, nous faisons fuir toute l'armée de Satan. Dans Jn. 12:32 le Seigneur nous dit : « Et moi, quand j'aurai été élevé de la terre, j'attirerai tous les hommes à moi ». Jésus a été élevé comme étant la bannière qui a réconcilié Dieu avec les pécheurs. Lorsque cette bannière est élevée, les pécheurs sont attirés vers Dieu, ils passent des ténèbres à la lumière, de la mort à la vie.} pour les peuples, et son séjour ne   sera que gloire.
\TextTitle{Etablissement du règne du Messie}
\VS{11}Et il arrivera en ce jour-là, que le Seigneur mettra encore sa main une seconde fois pour acquérir le reste de son peuple dispersé en Assyrie, en Egypte, à Pathros, en Ethiopie, à Elam, à Schinear, à Hamath et dans les îles de la mer.
\VS{12}Il élèvera une bannière parmi les nations, il rassemblera les exilés d'Israël qui auront été chassés, et il recueillera les dispersés de Juda des quatre extrémités de la terre.
\VS{13}Et la jalousie d'Ephraïm sera ôtée, et les oppresseurs de Juda seront retranchés ; Ephraïm ne sera plus jaloux de Juda, et Juda n'opprimera plus Ephraïm.
\VS{14}Mais ils voleront sur l'épaule des Philistins vers la mer ; ils pilleront ensemble les fils de l'orient ; Edom et Moab seront la proie de leurs mains et les enfants d'Ammon leur obéiront.
\VS{15}Yahweh exterminera aussi à la façon de l'interdit la langue de la mer d'Egypte, et il lèvera sa main contre le fleuve par la force de son vent, et il le frappera sur les sept rivières, et fera qu'on y marche avec des souliers.
\VS{16}Et il y aura un chemin pour le reste de son peuple, qui sera échappé de l'Assyrie, comme il y en eut un pour Israël le jour où il remonta du pays d'Egypte.
\Chap{12}
\TextTitle{Louange au sein du royaume}
\VerseOne{}Tu diras en ce jour-là : Je te loue, ô Yahweh ! Car tu as été irrité contre moi, ta colère s'est apaisée, et tu m'as consolé.
\VS{2}Voici, Dieu est ma délivrance, j'aurai confiance et je ne craindrai rien ; car Yahweh, Yahweh est ma force et ma louange ; il est mon Sauveur.
\VS{3}Et vous puiserez de l'eau avec joie aux sources du salut\FTNT{Jn. 4:10-14.},
\VS{4}et vous direz en ce jour-là : Louez Yahweh, invoquez son Nom, publiez ses œuvres parmi les peuples, rappelez que son Nom est une haute retraite !
\VS{5}Psalmodiez à Yahweh car il a fait des choses magnifiques : Cela est connu dans toute la terre !
\VS{6}Habitante de Sion, égaye-toi, et réjouis-toi avec chant de triomphe ! Car le Saint d'Israël est grand au milieu de toi.
\Chap{13}
\TextTitle{Yahweh lève une armée}
\VerseOne{}Prophétie sur Babylone, révélé à Esaïe, fils d'Amots.
\VS{2}Elevez la bannière sur la haute montagne, élevez la voix vers eux, faites des signes avec la main, et qu'on entre dans les portes des magnifiques !
\VS{3}C'est moi qui ai donné des ordres à ceux qui me sont consacrés, j'ai appelé mes hommes forts pour exécuter ma colère, ceux qui se réjouissent de ma grandeur.
\VS{4}Il y a sur les montagnes un bruit d'une multitude, comme celui d'un grand peuple ; on entend un tumulte de royaumes, de nations rassemblées : Yahweh des armées passe en revue l'armée pour le combat.
\VS{5}D'un pays éloigné, de l'extrémité des cieux, Yahweh vient avec les instruments de sa colère pour détruire tout le pays.
\TextTitle{Jugement de Yahweh sur Babylone}
\VS{6}Hurlez, car le jour de Yahweh est proche, il vient comme un ravage du Tout-Puissant.
\VS{7}C'est pourquoi toutes les mains deviennent lâches, et tout cœur d'homme se fond.
\VS{8}Ils sont épouvantés ; les détresses et les douleurs les saisissent ; ils sont en travail comme celle qui enfante ; ils se regardent les uns les autres avec stupeur, leurs visages sont comme des visages enflammés.
\VS{9}Voici, le jour de Yahweh arrive, jour cruel, jour de colère et d'ardente fureur\FTNT{Mal. 4:1 ; Ap. 19:15}, qui réduira le pays en désolation, et en exterminera les pécheurs.
\VS{10}Même les étoiles des cieux et leurs astres ne feront plus briller leur lumière ; le soleil s'obscurcira dès son lever, et la lune ne fera plus resplendir sa lueur\FTNT{Joë. 2:31 ; Mt. 24:29 ; Mc. 13:24.}.
\VS{11}Je punirai le monde habitable à cause de sa malice, et les méchants à cause de leur iniquité ; je ferai cesser l'orgueil des hautains et j'abaisserai l'arrogance des tyrans.
\VS{12}Je ferai qu'un homme sera plus précieux que l'or fin, et une personne plus que l'or d'Ophir.
\VS{13}C'est pourquoi j'ébranlerai les cieux, et la terre sera secouée de sa base\FTNT{Ag. 2:6}, à cause de la fureur de Yahweh des armées, et à cause du jour de son ardente colère.
\VS{14}Et chacun sera comme un chevreuil qui est chassé, et comme une brebis que personne ne retire, chacun se tournera vers son peuple, chacun fuira vers son pays.
\VS{15}Quiconque sera trouvé, sera transpercé ; et quiconque s'y sera joint, tombera par l'épée.
\VS{16}Et leurs petits enfants seront écrasés sous leurs yeux\FTNT{Na. 3:10.}, leurs maisons seront pillées, et leurs femmes violées.
\TextTitle{Yahweh envoie les Mèdes contre Babylone}
\VS{17}Voici, je vais susciter contre eux les Mèdes, qui ne font point cas de l'argent, et qui ne convoitent point l'or.
\VS{18}Leurs arcs écraseront les jeunes gens, et ils seront sans pitié pour le fruit des entrailles, leur œil n'épargnera point les enfants.
\VS{19}Ainsi Babylone, l'ornement des royaumes, la parure et l'orgueil des Chaldéens, sera comme Sodome et Gomorrhe que Dieu détruisit.
\VS{20}Elle ne sera plus jamais habitée, elle ne sera point habitée de génération en génération ; même les Arabes n'y dresseront point leurs tentes, et les bergers n'y feront plus reposer leurs troupeaux.
\VS{21}Mais les bêtes sauvages des déserts y prendront leur gîte, et les hiboux rempliront ses maisons, les autruches en feront leur demeure, et les boucs y sauteront.
\VS{22}Les chacals hurleront dans ses palais, et les dragons dans ses maisons de plaisance. Son temps est près d'arriver et ses jours ne se prolongeront pas.
\Chap{14}
\TextTitle{Chant d'Israël après la chute de Babylone}
\VerseOne{}Car Yahweh aura pitié de Jacob, il choisira encore Israël, et il les rétablira dans leur terre ; les étrangers se joindront à eux et s'attacheront à la maison de Jacob.
\VS{2}Et les peuples les prendront, et les ramèneront à leur demeure, et la maison d'Israël les possédera en droit d'héritage sur la terre de Yahweh, comme serviteurs et comme servantes ; ils retiendront captifs ceux qui les avaient tenus captifs, et ils domineront sur leurs oppresseurs.
\VS{3}Et il arrivera qu'au jour que Yahweh fera cesser ton travail, ton tourment, et la dure servitude qui te fut imposée,
\VS{4}alors tu prononceras ce proverbe sur le roi de Babylone, et tu diras : Comment a-t-il fini le tyran ? Comment se repose celle qui était si avide de richesses ?
\VS{5}Yahweh a brisé le bâton des méchants, et la verge des dominateurs.
\VS{6}Celui qui frappait avec fureur les peuples de coups qu'on ne pouvait point détourner, qui dominait sur les nations avec colère, est poursuivi sans ménagement.
\VS{7}Toute la terre jouit du repos et de la paix ; on éclate en chants de triomphe à gorge déployée.
\VS{8}Même les cyprès et les cèdres du Liban se réjouissent de toi en disant : Depuis que tu es tombé, personne n'est monté pour nous abattre.
\TextTitle{Le roi de Babylone dépouillé de sa gloire}
\VS{9}Le scheol s'émeut jusque dans ses profondeurs, pour t'accueillir à ton arrivée ; il réveille à cause de toi les morts, et il fait lever de leurs sièges tous les principaux de la terre.
\VS{10}Tous prennent la parole pour te dire : Toi aussi, tu es sans force comme nous, tu es devenu semblable à nous !
\VS{11}Ta hauteur est descendue dans le scheol, avec le son de tes luths ; tu es couché sur une couche de vers, et la vermine est ta couverture.
\TextTitle{Orgueil, rébellion et chute de Satan}
\VS{12}Comment es-tu tombé du ciel, astre brillant, fils de l'aurore ? Toi qui foulais les nations, tu es abattu jusqu'à terre !
\VS{13}Tu disais en ton cœur : Je monterai aux cieux, je placerai mon trône au-dessus des étoiles de Dieu ; je m'assiérai sur la montagne de l'assemblée, du côté d'Aquilon\FTNT{Aquilon est un dieu des vents septentrionaux, froids et violents, dans la mythologie romaine.} ;
\VS{14}je monterai au dessus des hauts lieux des nuées, je serai semblable au Très-Haut.
\VS{15}Et cependant tu as été précipité dans le scheol, dans les profondeurs de la fosse\FTNT{Voir commentaire Ge. 1:1-2.}.
\VS{16}Ceux qui te voient fixent sur toi leurs regards, ils te considèrent attentivement, en disant : N'est-ce pas celui qui faisait trembler la terre, qui ébranlait les royaumes,
\VS{17}qui réduisait le monde habitable en désert, qui détruisait les villes, et ne relâchait pas ses prisonniers, ni ne les renvoyait chez eux ?
\TextTitle{Babylone anéantie}
\VS{18}Tous les rois des nations, oui, tous, reposent avec honneur, chacun dans sa maison.
\VS{19}Mais toi, tu as été jeté loin de ton sépulcre, comme un rejeton pourri, comme une dépouille de gens tués, transpercés avec l'épée, qu'on jette sous les pierres d'une fosse, comme un cadavre foulé aux pieds.
\VS{20}Tu ne seras point rangé comme eux dans le sépulcre, car tu as ravagé ta terre, tu as tué ton peuple. La race des méchants ne sera point renommée à toujours.
\VS{21}Préparez la tuerie pour ses enfants, à cause de l'iniquité de leurs pères ; afin qu'ils ne se relèvent point, et qu'ils n'héritent point la terre, et ne remplissent point de villes le dessus de la terre habitable.
\VS{22}Je m'élèverai contre eux, dit Yahweh des armées, et je retrancherai à Babylone le nom et le reste qu'elle a, ses descendants et sa postérité\FTNT{Ap. 14:8 ; Ap. 18:2.}, dit Yahweh.
\VS{23}J'en ferai l'habitation du butor et un marécage, et je la balayerai avec le balai de la destruction, dit Yahweh des armées.
\TextTitle{Jugement sur le roi d'Assyrie}
\VS{24}Yahweh des armées l'a juré, en disant : Certainement ce que j'ai décidé arrivera, ce que j'ai résolu s'accomplira.
\VS{25}Je briserai le roi d'Assyrie dans ma terre, je le foulerai aux pieds sur mes montagnes ; et son joug leur sera ôté, et son fardeau sera ôté de dessus leurs épaules.
\VS{26}C'est là le conseil arrêté contre toute la terre, c'est là la main étendue sur toutes les nations.
\TextTitle{Jugement sur le pays des Philistins}
\VS{27}Car Yahweh des armées l'a arrêté en son conseil : Qui l'empêchera ? Sa main est étendue : Qui la détournera\FTNT{Ec. 7:13.} ?
\VS{28}L'année de la mort du roi Achaz, cette prophétie fut prononcée : 
\VS{29}Ne te réjouis pas, toi pays des Philistins, de ce que la verge de celui qui te frappait est brisée ! Car de la racine du serpent sortira un vipère, et son fruit sera un serpent brûlant qui vole.
\VS{30}Alors les plus misérables seront repus, et les pauvres reposeront en assurance ; mais je ferai mourir de faim ta racine, et ce qui restera de toi sera tué.
\VS{31}Porte, hurle ! Ville, crie ! Tremble, pays tout entier des Philistins ! Car d'Aquilon, vient une fumée, et il ne restera pas un homme dans ses habitations.
\VS{32}Et que répondra-t-on aux envoyés de cette nation ? On répondra que Yahweh a fondé Sion, et que les affligés de son peuple y trouvent un refuge.
\Chap{15}
\TextTitle{Jugement sur Moab}
\VerseOne{}Prophétie sur Moab. La nuit même où elle est ravagée, Ar-Moab est détruite ! La nuit même où elle est saccagée, Kir-Moab est détruite !
\VS{2}Il monte à Bajith et à Dibon, dans les hauts lieux, pour pleurer ; Moab est en lamentations sur Nebo et sur Médeba : Toutes les têtes sont rasées et toutes les barbes sont coupées.
\VS{3}On sera couvert de sacs dans les rues ; chacun hurle, fondant en larmes sur ses toits et dans ses places\FTNT{Jé. 48:38.}.
\VS{4}Hesbon et Elealé poussent des cris, et l'on entend leur voix jusqu'à Jahats ; c'est pourquoi les guerriers de Moab se lamentent, ils ont l'effroi dans l'âme.
\VS{5}Mon cœur crie à cause de Moab, dont les fugitifs s'enfuient jusqu'à Tsoar, comme une génisse de trois ans ; car ils montent par la montée de Luchith avec des pleurs, et ils jettent des cris de détresse sur le chemin de Choronaïm.
\VS{6}Même les eaux de Nimrim ne sont que désolations, même le foin est déjà séché, l'herbe est consumée, et il n'y a point de verdure.
\VS{7}C'est pourquoi ils surveillent les richesses abondantes qu'ils ont acquises, afin que ce qu'ils ont réservé soit porté dans la vallée des saules.
\VS{8}Car les cris environnent les frontières de Moab, ses lamentations retentissent jusqu'à Eglaïm, ses lamentations retentissent jusqu'à Beer-Elim.
\VS{9}Même les eaux de Dimon sont pleines de sang ; car j'ajouterai un surcroît sur Dimon : Des lions contre les réchappés de Moab, et le reste du pays.
\Chap{16}
\TextTitle{Lamentation sur Moab}
\VerseOne{}Envoyez l'agneau au souverain du pays, envoyez-le du rocher du désert, à la montagne de la fille de Sion.
\VS{2}Car il arrivera que les filles de Moab seront au passage de l'Arnon, comme un oiseau volant ça et là, comme une nichée chassée de son nid.
\VS{3}Mets en avant le conseil, fais l'ordonnance, sers d'ombre comme une nuit au milieu de midi ; cache ceux qui ont été chassés, et ne trahis pas ceux qui sont errants.
\VS{4}Que ceux de mon peuple qui ont été chassés séjournent chez toi, ô Moab ! Sois pour eux un refuge contre le dévastateur ! Car celui qui use d'extorsion cessera, la dévastation finira, celui qui foule le pays sera consumé de dessus la terre.
\VS{5}Et le trône s'affermira par la clémence ; et sur ce trône sera assis en vérité, dans le tabernacle de David, un juge recherchant le droit, et se hâtant de faire justice\FTNT{Mi. 4:7 ; Da. 7:14 ; Lu. 1:33 ; Ap. 11:15}.
\VS{6}Nous avons entendu l'orgueil de Moab, le peuple extrêmement orgueilleux, sa fierté, son orgueil, son arrogance et ses vains discours.
\VS{7}C'est pourquoi Moab gémit sur Moab, chacun gémit ; vous soupirez pour les fondements de Kir-Haréseth, il n'y aura que des gens blessés à mort.
\VS{8}Car les campagnes de Hesbon et le vignoble de Sibma languissent ; les maîtres des nations ont foulé ses meilleurs ceps, qui s'étendaient jusqu'à Jaezer, qui couraient ça et là par le désert ; ses rameaux s'étendaient et passaient au-delà de la mer.
\VS{9}C'est pourquoi je pleure sur la vigne de Sibma, comme sur Jaezer ; je vous arrose de mes larmes, ô Hesbon et Elealé ! Car l'ennemi avec des cris s'est jeté sur tes fruits d'été et sur ta moisson.
\VS{10}Et la joie et l'allégresse se sont retirées du champ fertile ; on ne se réjouit plus et on ne s'égaye plus dans les vignes, le vendangeur ne foule plus dans les cuves, j'ai fait cesser la chanson de la vendange\FTNT{Jé. 48:31-34.}.
\VS{11}C'est pourquoi mes entrailles gémissent sur Moab, comme une harpe, et mon intérieur sur Kir-Harès.
\VS{12}Et on voit Moab qui se fatigue sur les hauts lieux ; il entre dans son sanctuaire pour prier mais il ne peut rien obtenir.
\VS{13}Telle est la parole que Yahweh a prononcée depuis longtemps sur Moab.
\VS{14}Et maintenant Yahweh a parlé, en disant : Dans trois ans, comme les années d'un mercenaire, la gloire de Moab sera avilie, avec toute cette grande multitude ; et le reste sera petit, ce sera peu de chose, ce ne sera rien de considérable.
\Chap{17}
\TextTitle{Prophétie sur la chute de Damas et de ses alliés}
\VerseOne{}Prophétie sur Damas. Voici, Damas est détruite pour ne plus être une ville, et elle ne sera qu'un monceau de ruines\FTNT{Jé. 49:23-27.}.
\VS{2}Les villes d'Aroër sont abandonnées, elles sont livrées aux troupeaux qui s'y reposent, et il n'y a personne qui les effraie.
\VS{3}Il n'y aura plus de forteresse en Ephraïm, ni de royaume à Damas et dans le reste de la Syrie ; ils seront comme la gloire des enfants d'Israël, dit Yahweh des armées.
\VS{4}Et il arrivera en ce jour-là que la gloire de Jacob sera affaiblie et la graisse de sa chair sera fondue.
\VS{5}Il en sera comme quand le moissonneur cueille les blés, et qu'il moissonne les épis avec son bras\FTNT{Joë. 3:13 ; Mt. 13:24-30.} ; comme quand on ramasse les épis dans la vallée de Rephaïm.
\VS{6}Mais il en restera quelques grappillages, comme quand on secoue l'olivier, et qu'il reste deux ou trois olives en haut de la cime, et qu'il y en a quatre ou cinq que l'olivier a produites dans ses branches fruitières, dit Yahweh, le Dieu d'Israël.
\VS{7}En ce jour-là, l'homme regardera vers celui qui l'a fait, et ses yeux se tourneront vers le Saint d'Israël.
\VS{8}Et il ne regardera plus vers les autels, qui sont l'ouvrage de ses mains, et il ne regardera plus ce que ses doigts ont fabriqué, ni les images d'Asherah, ni les statues du soleil.
\VS{9}En ce jour-là, ses villes fortes seront abandonnées à cause des enfants d'Israël, ils seront comme un bois taillis et des rameaux abandonnés, et ce sera un désert.
\VS{10}Parce que tu as oublié le Dieu de ton salut, et que tu ne t'es pas souvenue du rocher\FTNT{Voir commentaire Es. 8:13-14.} de ta force, à cause de cela tu as transplanté des plantes de plaisance, et tu as planté des ceps étrangers.
\VS{11}De jour tu as fais croître ce que tu as planté, et le matin tu as fait levé ta semence; mais la moisson a été enlevée au jour que l'on voulait en jouir, et il y a eu une douleur désespérée.
\VS{12}Malheur à la multitude de peuples nombreux, qui font un bruit comme le bruit des mers ; et à la tempête éclatante des nations, qui font du bruit comme une tempête éclatante d'eaux impétueuses !
\VS{13}Les nations font un bruit comme une tempête éclatante de grosses eaux, mais il les menace, et elles s'enfuient ; elles seront poursuivies comme la balle des montagnes chassée par le vent, et comme une boule poussée par un tourbillon.
\VS{14}Au temps du soir, voici une terreur soudaine ; mais avant le matin, ils ne sont plus ! C'est là le partage de ceux qui nous dépouillent, et le lot de ceux qui nous pillent.
\Chap{18}
\TextTitle{Jugement sur l'Ethiopie}
\VerseOne{}Malheur à la terre qui fait ombre avec des ailes, qui est au-delà des fleuves de l'Ethiopie ;
\VS{2}qui envoie par mer des messagers, dans des navires de jonc, voguant à la surface des eaux ! Allez, messagers rapides, vers la nation robuste et vigoureuse, vers le peuple redoutable, depuis là où il est et par delà ; nation puissante et qui écrase tout, et dont les fleuves ravagent son pays.
\VS{3}Vous tous, habitants du monde, et vous qui habitez dans le pays, quand la bannière sera élevée sur les montagnes, regardez ; et quand le shofar sonnera, écoutez !
\VS{4}Car ainsi m'a parlé Yahweh : Je me tiens tranquillement, et je regarde de ma demeure, par la chaleur de la lumière, et par la vapeur de la rosée, au temps de la chaude moisson.
\VS{5}Car avant la moisson, quand le bourgeon vient en sa perfection, et que la fleur devient un raisin qui mûrit, il coupe les sarments avec des serpes, il enlève les sarments, les ayant retranchés.
\VS{6}Ils seront tous ensemble abandonnés aux oiseaux de proie qui demeurent dans les montagnes, et aux bêtes de la terre ; les oiseaux de proie seront sur eux tout le long de l'été, et toutes les bêtes de la terre y passeront l'hiver.
\VS{7}En ce temps-là, un présent sera apporté à Yahweh des armées, par le peuple robuste et vigoureux, de la part, dis-je, du peuple terrible depuis là où il est et au-delà, nation puissante et qui écrase tout, et dont le pays est ravagé par ses fleuves ; il sera apporté dans la demeure du Nom de Yahweh des armées, sur la montagne de Sion.
\Chap{19}
\TextTitle{Chute de l'Egypte}
\VerseOne{}Prophétie sur l'Egypte. Voici, Yahweh est monté sur une nuée rapide, il entre en Egypte ; et les idoles d'Egypte s'enfuient de toutes parts devant sa face, et le cœur des Egyptiens se fond au milieu d'elle\FTNT{Jé. 43:12.}.
\VS{2}Et je ferai venir pêle-mêle l'Egyptien contre l'Egyptien, et chacun fera la guerre contre son frère, et chacun contre son ami, ville contre ville, et royaume contre royaume.
\VS{3}L'esprit de l'Egypte disparaîtra du milieu d'elle, et je dissiperai son conseil ; et ils consulteront les idoles et les enchanteurs, ceux qui évoquent les morts et ceux qui prédisent l'avenir.
\VS{4}Et je livrerai l'Egypte entre les mains d'un maître sévère ; et un roi cruel dominera sur eux, dit le Seigneur, Yahweh des armées.
\VS{5}Les eaux de la mer tariront, le fleuve séchera et tarira\FTNT{Jé. 51:36.}.
\VS{6}Et on fera détourner les fleuves ; les ruisseaux des digues s'abaisseront et sécheront ; les roseaux et les joncs seront coupés.
\VS{7}Les prairies qui sont près des ruisseaux, et sur l'embouchure du fleuve, tout ce qui aura été semé le long des ruisseaux, séchera, sera jeté au loin, et ne sera plus.
\VS{8}Et les pêcheurs gémiront, tous ceux qui jettent l'hameçon dans le fleuve mèneront deuil, et ceux qui étendent des filets sur les eaux languiront.
\VS{9}Ceux qui travaillent en fin lin et en fin crêpe, et ceux qui tissent les filets seront confus.
\VS{10}Les fondements du pays seront rompus, et tous ceux qui font des écluses de viviers auront l'âme attristée.
\VS{11}Certes les chefs de Tsoan ne sont que des insensés, les sages d'entre les conseillers de Pharaon forment un conseil stupide. Comment osez-vous dire à Pharaon : Je suis fils des sages, fils des anciens rois ?
\VS{12}Où sont-ils maintenant? Où sont, dis-je, tes sages ? Qu'ils t'annoncent, je te prie, s'ils le savent, ce que Yahweh des armées a décrété contre l'Egypte.
\VS{13}Les chefs de Tsoan sont devenus insensés, les chefs de Noph se sont trompés, les chefs des tribus font égarer l'Egypte.
\VS{14}Yahweh a versé au milieu d'elle un esprit de vertige\FTNT{1 R. 22:18-22.}, pour qu'ils fassent chanceler les Egyptiens dans toutes leurs actions, comme un homme ivre se vautre dans son vomissement.
\VS{15}Et l'Egypte sera hors d'état de faire ce que font la tête et la queue, la branche de palmier et le roseau.
\TextTitle{L'Egypte et l'Assyrie dans le royaume du Messie}
\VS{16}En ce jour-là, l'Egypte sera comme des femmes : Elle sera étonnée et épouvantée à cause de la main de Yahweh des armées, quand il élèvera la main contre elle.
\VS{17}Et la terre de Juda sera pour l'Egypte un objet d'effroi ; quiconque fera mention d'elle, en sera épouvanté en lui-même, à cause du conseil décrété contre elle par Yahweh des armées.
\VS{18}En ce jour-là, il y aura cinq villes au pays d'Egypte, qui parleront la langue de Canaan, et qui jureront par Yahweh des armées ; l'une sera appelée ville de la destruction.
\VS{19}En ce jour-là, il y aura un autel à Yahweh au milieu du pays d'Egypte, et un monument dressé à Yahweh sur la frontière.
\VS{20}Et ce sera un signe et un témoignage pour Yahweh des armées dans le pays d'Egypte ; car ils crieront à Yahweh à cause des oppresseurs, et il leur enverra un sauveur, quelqu'un de grand, et il les délivrera\FTNT{Es. 43:11.}.
\VS{21}Et Yahweh se fera connaître aux Egyptiens, et les Egyptiens connaîtront Yahweh en ce jour-là ; ils le serviront, ils offriront des sacrifices et des offrandes, et ils feront des vœux à Yahweh et les accompliront.
\VS{22}Ainsi Yahweh frappera les Egyptiens, il les frappera, mais il les guérira ; et ils retourneront à Yahweh, qui les exaucera et les guérira.
\VS{23}En ce jour-là, il y aura un chemin battu de l'Egypte en Assyrie ; et l'Assyrie viendra en Egypte, et l'Egypte en Assyrie, et l'Egypte servira avec l'Assyrie.
\VS{24}En ce même temps, Israël sera, lui troisième, uni à l'Egypte et à l'Assyrie, et la bénédiction sera au milieu de la terre.
\VS{25}Yahweh des armées les bénira, en disant : Bénis soit l'Egypte mon peuple, et l'Assyrie œuvre de mes mains, et Israël mon héritage !
\Chap{20}
\TextTitle{Conquête de l'Egypte et de l'Ethiopie}
\VerseOne{}L'année où Tharthan, envoyé par Sargon, roi d'Assyrie, vint et combattit contre Asdod, et la prit.
\VS{2}En ce temps-là, Yahweh parla par Esaïe, fils d'Amots, et lui dit : Va, délie le sac de dessus tes reins et ôte tes souliers de tes pieds. Il fit ainsi, marchant nu et déchaussé.
\VS{3}Puis Yahweh dit : De même que mon serviteur Esaïe marche nu et déchaussé, ce qui sera dans trois ans un signe et un prodige contre l'Egypte et contre l'Ethiopie,
\VS{4}de même le roi d'Assyrie emmènera de l'Egypte et de l'Ethiopie prisonniers et captifs les jeunes et les vieux, nus et déchaussés, ayant les hanches découvertes, ce qui sera l'opprobre de l'Egypte\FTNT{2 S. 10:4 ; Es. 3:17 ; Jé. 13:22-26.}.
\VS{5}Ils seront effrayés, et ils seront honteux à cause de l'Ethiopie à qui ils s'attendaient, et à cause de l'Egypte dont ils se glorifiaient.
\VS{6}Et les habitants de cette côte diront en ce jour-là : Voilà ce qu'est devenu le peuple à qui nous nous attendions, celui vers qui nous courions chercher du secours, afin d'être délivrés du roi d'Assyrie ! Comment pourrons-nous échapper ?
\Chap{21}
\TextTitle{Annonce de la conquête de Babylone}
\VerseOne{}Prophétie sur le désert de la mer. Il vient du désert, de la terre redoutable, comme des tourbillons qui s'élèvent au pays du midi pour traverser.
\VS{2}Une vision terrible m'a été révélée. Le traître demeure traître, celui qui saccage, saccage toujours. Monte, Elam ! Assiège, Médie ! Je fais cesser tous les soupirs.
\VS{3}C'est pourquoi mes reins sont remplis de douleur ; les angoisses me saisissent comme les douleurs de celle qui enfante ; je suis tourmenté à cause de ce que j'ai entendu, et j'ai été tout troublé à cause de ce que j'ai vu.
\VS{4}Mon cœur est agité de toutes parts, la terreur s'empare de moi ; la nuit de mes plaisirs devient une nuit de crainte.
\VS{5}Qu'on dresse la table, que la sentinelle veille, qu'on mange, qu'on boive! Levez-vous, chefs ! Oignez le bouclier !
\VS{6}Car ainsi m'a parlé le Seigneur : Va, place la sentinelle, et qu'elle rapporte ce qu'elle verra\FTNT{Ez. 33:1-19.}.
\VS{7}Et elle vit un char, un couple de cavaliers, un char tiré par des ânes, un char tiré par des chameaux ; et elle les considéra fort attentivement.
\VS{8}Et elle s'écria : C'est un lion ! Seigneur, je me tiens en sentinelle toute la journée et je suis à mon poste toutes les nuits ;
\VS{9}et voici venir le char d'un homme et un couple de cavaliers ! Alors elle parla et dit : Elle est tombée, elle est tombée, Babylone\FTNT{Prophétie sur la chute de Babylone. Voir Jé. 50 et 51 ; Ap.18.}, et toutes les images taillées de ses dieux sont brisées par terre.
\VS{10}C'est ce que j'ai foulé, et le grain que j'ai battu dans mon aire. Je vous ai annoncé ce que j'ai entendu de Yahweh des armées, du Dieu d'Israël.
\VS{11}Prophétie sur Duma. On me crie de Séir : Ô sentinelle ! Qu'en est-il de la nuit ? Ô sentinelle ! Qu'en est-il de la nuit ?
\VS{12}La sentinelle répond : Le matin vient et la nuit aussi. Si vous demandez, demandez. Retournez, venez.
\TextTitle{Jugement sur l'Arabie}
\VS{13}Prophétie contre l'Arabie. Vous passerez pêle-mêle la nuit dans la forêt, caravanes de Dedan !
\VS{14}Les habitants du pays de Théma portent de l'eau à ceux qui ont soif ; ils  viennent au-devant du fugitif avec du pain pour lui.
\VS{15}Car ils fuient devant les épées, devant l'épée dégainée, devant l'arc tendu, devant le fort de la bataille.
\VS{16}Car ainsi m'a parlé le Seigneur : Encore une année, comme les années d'un mercenaire, et toute la gloire de Kédar prendra fin.
\VS{17}Et le reste du nombre des forts archers des fils de Kédar sera diminué, car Yahweh, le Dieu d'Israël, a parlé.
\Chap{22}
\TextTitle{Malédiction sur la vallée des visions, Jérusalem}
\VerseOne{}Prophétie sur la vallée des visions. Qu'as-tu maintenant, que tu sois toute montée sur les toits ?
\VS{2}Toi ville bruyante, pleine de tumulte, ville joyeuse ! Tes blessés à morts ne seront pas blessés à mort par l'épée, et ils ne mourront pas par la guerre.
\VS{3}Tous tes chefs fuient ensemble, ils sont liés par les archers ; tous ceux des tiens qui sont trouvés sont liés ensemble tandis qu'ils s'enfuient au loin.
\VS{4}C'est pourquoi je dis : Détournez de moi vos regards, que je pleure amèrement. Ne vous empressez pas pour me consoler du désastre de la fille de mon peuple.
\VS{5}Car c'est le jour de trouble, d'oppression et de confusion\FTNT{Lam. 1:5 ; Lam. 2:2.}, envoyé par le Seigneur, Yahweh des armées, dans la vallée des visions. Il démolit la muraille et les cris retentissent jusqu'à la montagne.
\VS{6}Même Elam prend son carquois, il y a des hommes montés sur des chars et des cavaliers ; Kir découvre le bouclier.
\VS{7}Et tes plus belles vallées sont remplies de chars, et les cavaliers se rangent tous en bataille à tes portes.
\VS{8}Et on découvre ce qui couvrait Juda, et en ce jour là tu regardes vers les armes de la maison de la forêt.
\VS{9}Vous voyez que les brèches de la cité de David sont nombreuses ; et vous assemblez les eaux de l'étang inférieur.
\VS{10}Vous faites le dénombrement des maisons de Jérusalem, et vous démolissez les maisons pour fortifier la muraille.
\VS{11}Et vous faites aussi un réservoir d'eau entre les deux murailles, pour les eaux de l'ancien étang. Mais vous ne regardez pas à celui qui a fait ces choses, qui les a formées il y a longtemps.
\VS{12}Le Seigneur, Yahweh des armées, vous appelle ce jour-là aux pleurs et au deuil, à vous raser la tête, et à ceindre le sac\FTNT{Ez. 7:18 ; Joë 1:13}.
\VS{13}Et voici il y a de la joie et de l'allégresse ! On égorge des bœufs et l'on tue des moutons, on mange la viande et l'on boit du vin ; puis on dit : Mangeons et buvons, car demain nous mourrons\FTNT{Es. 56:12 ; 1 Co. 15:32.} !
\VS{14}Or il m'a été révélé à l'oreille, par Yahweh des armées : Sûrement cette iniquité ne vous sera pas pardonnée jusqu'à ce que vous mouriez, a dit le Seigneur, Yahweh des armées.
\TextTitle{Eliakim succède à Schebna}
\VS{15}Ainsi parle le Seigneur, Yahweh des armées : Va, entre chez ce trésorier, chez Schebna, gouverneur du palais et dis-lui :
\VS{16}Qu'as-tu à faire ici, et qu'as-tu ici qui t'appartienne, que tu te tailles ici un sépulcre ? Il taille un sépulcre en hauteur, il se taille une demeure dans le rocher.
\VS{17}Voici, ô homme ! Yahweh te chassera au loin d'un bras vigoureux ; il t'enveloppera entièrement.
\VS{18}Il te fera rouler fort vite, comme une balle sur une terre large et spacieuse ; là tu mourras, là seront les chars de ta gloire, ô toi qui es la honte de la maison de ton Seigneur !
\VS{19}Je te jetterai hors de ton rang, et on t'arrachera de ton service.
\VS{20}Et il arrivera en ce jour-là que j'appellerai mon serviteur Eliakim, fils de Hilkija.
\VS{21}Je le revêtirai de ta tunique, je le ceindrai de ta ceinture, et je remettrai ton autorité entre ses mains, il sera un père pour les habitants de Jérusalem et pour la maison de Juda.
\VS{22}Et je mettrai la clef de la maison de David sur son épaule ; et il ouvrira, et il n'y aura personne qui ferme ; et il fermera, et il n'y aura personne qui ouvre\FTNT{La clé de David est le symbole de l'autorité du Messie (Es. 9:5 ; Mt. 28:18 ; Ap. 3:7-8)}.
\VS{23}Je l'enfoncerai comme un clou dans un lieu sûr, et il sera un trône de gloire pour la maison de son père.
\VS{24}Et on y pendra toute la gloire de la maison de son père, de ses parents et de celles qui lui appartiennent ; tous les ustensiles des plus petites choses,  des bassins comme des vases. 
\VS{25}En ce jour-là, dit Yahweh des armées, le clou enfoncé dans un lieu sûr sera ôté ; et étant retranché il tombera, et le fardeau qui était sur lui sera retranché, car Yahweh a parlé.
\Chap{23}
\TextTitle{Effondrement de Tyr}
\VerseOne{}Prophétie sur Tyr. Hurlez, navires de Tarsis ! Car elle est détruite, il n'y a plus de maisons, on n'y entre plus ! Ceci leur a été révélé du pays de Kittim.
\VS{2}Vous qui habitez dans l'île, taisez-vous ! Toi qui étais remplie de marchands de Sidon, et de ceux qui traversaient la mer !
\VS{3}A travers les grandes eaux, les grains de Shichor, la moisson du Nil était pour elle son revenu ; elle était le marché des nations\FTNT{Ez. 27.}.
\VS{4}Sois honteuse, ô Sidon ! Car la mer, la forteresse de la mer, a parlé en disant : Je n'ai point eu de douleurs, je n'ai point enfanté, je n'ai point nourri de jeunes gens ni élevé aucune vierge.
\VS{5}Selon la nouvelle qui a été touchant l'Egypte, ainsi sera-t-on en travail quand on entendra la nouvelle touchant Tyr.
\VS{6}Passez à Tarsis, hurlez, vous qui habitez dans l'île !
\VS{7}N'est-ce pas ici votre ville joyeuse ? Elle avait une origine antique et ses propres pieds la mènent séjourner dans un pays étranger.
\VS{8}Qui a pris ce conseil contre Tyr, celle qui couronnait les siens, dont les marchands étaient des princes, et dont les trafiquants étaient les plus honorables de la terre\FTNT{Ap. 18:9-18.} ?
\VS{9}Yahweh des armées a pris ce conseil, pour flétrir l'orgueil de toute la noblesse, et pour avilir tous les honorables de la terre.
\VS{10}Traverse ton pays, comme une rivière, ô fille de Tarsis ! Il n'y a plus de ceinture.
\VS{11}Il a étendu sa main sur la mer, il a fait trembler les royaumes ; Yahweh a ordonné la destruction des forteresses de Canaan.
\VS{12}Il a dit : Tu ne te livreras plus à la joie, vierge opprimée, fille de Sidon ! Lève-toi, passe au pays de Kittim ! Même là, il n'y aura pas de repos pour toi.
\VS{13}Voilà le pays des Chaldéens ; ce peuple-là n'était pas autrefois ; Assur\FTNT{Assur : Le second fils de Sem (Ge. 10:22). L'ancêtre des Assyriens.} l'a fondé pour les gens du désert ; on a dressé ses forteresses, on a élevé ses palais, et il l'a mis en ruines.
\VS{14}Hurlez, navires de Tarsis ! Car votre force est détruite !
\VS{15}Et il arrivera en ce jour-là que Tyr tombera dans l'oubli durant soixante-dix ans, selon les jours d'un roi. Mais au bout de soixante-dix ans\FTNT{Jé. 25 : 11-12.}, on chantera une chanson à Tyr comme à une femme prostituée :
\VS{16}Prends la harpe, fais le tour de la ville, ô prostituée qu'on oublie ! Sonne avec force, chante et rechante, afin qu'on se ressouvienne de toi!
\VS{17}Et il arrivera au bout de soixante-dix ans que Yahweh visitera Tyr, mais elle retournera au salaire de sa prostitution, et elle se prostituera avec tous les royaumes de la terre, sur le dessus de la terre.
\VS{18}Mais son trafic et son salaire seront sanctifiés à Yahweh ; il n'en sera rien réservé, ni serré ; car son trafic sera pour ceux qui habitent dans la présence de Yahweh, pour en manger à satiété, et pour avoir des vêtements durables.
\Chap{24}
\TextTitle{Désastre après l'invasion babylonienne}
\VerseOne{}Voici, Yahweh s'en va rendre le pays vide et l'épuiser, il en renverse le dessus, et disperse ses habitants\FTNT{Ge. 11:1-8.}.
\VS{2}Et il en est du sacrificateur comme du peuple, du maître comme de son serviteur, de la dame comme de sa servante, du vendeur comme de l'acheteur, de celui qui prête comme de celui qui emprunte, du créancier comme du débiteur.
\VS{3}Le pays est entièrement vidé et entièrement pillé, car Yahweh a prononcé cet arrêt.
\VS{4}La terre mène le deuil, elle est déchue ; le pays habité est devenu languissant, il est déchu ; les plus distingués du peuple de la terre sont languissants.
\VS{5}Le pays était profané par ses habitants qui marchent sur lui ; car ils ont transgressé les lois, ils ont changé les ordonnances et ont enfreint l'alliance éternelle\FTNT{Da. 7:25.}.
\VS{6}C'est pourquoi la malédiction dévore le pays, et ses habitants portent la peine de leurs crimes ; c'est pourquoi les habitants du pays sont brûlés et il n'en reste qu'un petit nombre.
\VS{7}Le vin excellent pleure, la vigne languit, et tous ceux qui avaient le cœur joyeux soupirent.
\VS{8}La joie des tambours a cessé ; le bruit de ceux qui s'égayent a pris fin, la joie de la harpe a cessé.
\VS{9}On ne boit plus de vin en chantant ; les boissons fortes sont amères à ceux qui les boivent.
\VS{10}La ville confuse est en ruines ; toutes les maisons sont fermées, on n'y entre plus.
\VS{11}On crie dans les rues parce que le vin manque ; toute la joie est tournée en obscurité, l'allégresse du pays s'en est allée.
\VS{12}La désolation est restée dans la ville et la porte est frappée d'une ruine éclatante.
\VS{13}Car il arrivera au milieu de la terre et parmi les peuples, comme quand on secoue l'olivier, et comme quand on grappille après la vendange.
\TextTitle{Un reste de rescapés célèbre Yahweh}
\VS{14}Ils élèvent leur voix, ils se réjouissent avec chant de triomphe ; et s'égayent du côté de la mer, ils célèbrent la majesté de Yahweh.
\VS{15}C'est pourquoi glorifiez Yahweh dans les vallées, le Nom de Yahweh, le Dieu d'Israël, dans les îles de la mer !
\VS{16}De l'extrémité de la terre, nous entendons des cantiques à la gloire du Juste ; mais moi je dis : Maigreur sur moi ! Maigreur sur moi ! Malheur à moi ! Les perfides ont agi perfidement ; et ils ont imité la mauvaise foi des perfides.
\TextTitle{Manifestation des jugements de Yahweh}
\VS{17}La frayeur, la fosse, et le piège sont sur toi, habitant du pays !
\VS{18}Et il arrivera que celui qui fuit à cause du bruit de la frayeur tombe dans la fosse, et celui qui remonte hors de la fosse se prend au filet ; car les écluses d'en haut s'ouvrent et les fondements de la terre tremblent.
\VS{19}La terre est entièrement brisée, la terre s'écrase entièrement, la terre se remue de sa place.
\VS{20}La terre chancelle entièrement comme un homme ivre, elle est transportée comme une cabane ; son péché pèse sur elle, elle tombe et ne se relève plus.
\VS{21}Et il arrivera en ce jour là, que Yahweh punira dans le lieu élevé l'armée d'en haut, et sur la terre les rois de la terre.
\VS{22}Ils seront assemblés en troupes comme des prisonniers dans une fosse, et ils seront enfermés dans une prison, et après plusieurs jours ils seront visités.
\VS{23}La lune rougira et le soleil sera honteux quand Yahweh des armées régnera sur la montagne de Sion et à Jérusalem, resplendissant de gloire en présence de ses anciens\FTNT{Mt. 24:29-30 ; 2 Pi. 3:10-12 ; Ap. 6:12.}.
\Chap{25}
\TextTitle{Le royaume de Yahweh}
\VerseOne{}Ô Yahweh, tu es mon Dieu ; je t'exalterai, je célébrerai ton nom, car tu as fait des choses merveilleuses ; tes conseils conçus d'avance sont fidèlement accomplis.
\VS{2}Car tu as fait de la ville un monceau de pierres, et de la cité forte une ruine ; le palais des étrangers qui était dans la ville ne sera jamais rebâti.
\VS{3}C'est pourquoi le peuple fort te glorifie, la ville des nations redoutables te révère.
\VS{4}Parce que tu as été la force du faible, la force du misérable dans sa détresse, le refuge contre la tempête, l'ombrage contre la chaleur ; car le souffle des tyrans est comme la tempête qui abat une muraille.
\VS{5}Tu as rabaissé la tempête éclatante des étrangers ; comme la chaleur, dis-je, dans un pays sec, comme la chaleur par l'ombre d'une nuée, le branchage des tyrans sera abattu.
\VS{6}Et Yahweh des armées prépare à tous les peuples sur cette montagne un banquet de choses grasses, un banquet de vins vieux, un banquet, dis-je, de choses grasses et moelleuses, et de vins vieux bien purifiés\FTNT{Mt. 22:2 ; Ap. 3:20.}.
\VS{7}Et il détruit sur cette montagne l'enveloppe redoublée qu'on voit sur tous les peuples, et la couverture qui est étendue sur toutes les nations.
\VS{8}Il détruit la mort par sa victoire\FTNT{1 Co. 15:54.} ; et le Seigneur Yahweh essuie les larmes de tous les visages\FTNT{Ap. 7:17.}, et il ôte l'opprobre de son peuple de toute la terre\FTNT{Lu. 1:25.}, car Yahweh a parlé.
\VS{9}Et l'on dira en ce jour-là : Voici, c'est ici notre Dieu, auquel nous nous attendons, aussi c'est lui qui nous sauve ; c'est ici Yahweh, auquel nous nous attendons ; soyons dans l'allégresse, et réjouissons-nous de son salut !
\VS{10}Car la main de Yahweh repose sur cette montagne ; mais Moab est foulé aux pieds sous lui, comme on foule la paille pour en faire du fumier.
\VS{11}Et il étend ses mains au milieu d'eux, comme le nageur étend ses mains pour nager ; et Yahweh abat son orgueil, ainsi que l'artifice de ses mains.
\VS{12}Il abaisse la forteresse des plus hautes retraites de tes murailles, il les renverse, il les fait crouler à terre, et les réduit en poussière.
\Chap{26}
\TextTitle{Adoration à Yahweh}
\VerseOne{}En ce jour-là, ce cantique sera chanté dans le pays de Juda : Nous avons une ville forte ; le salut\FTNT{Le mot salut vient du mot « Yeshuw'ah ». Cette même racine a donné le prénom Jésus qui signifie Yahweh sauve. Jésus est notre muraille et notre rempart. Dans Ex. 15:2, Moïse identifie Yahweh à « Yeshuw'ah » c'est-à-dire à Jésus. Dans 1 Ch. 16:23, il est dit que « Yeshuw'ah » doit être annoncé tous les jours. Dans Ps. 62:2, il est présenté comme Dieu et le Rocher. Dans Es. 12:2, il est le Dieu qui sauve. Jacob et David avaient mis en lui leur espoir (Ge. 49:18 ; Ps. 119:166). Dans Es. 49:6, il est dit que le salut (« Yeshuw'ah » ou Jésus) doit être annoncé aux extrémités de la terre, et cela est répété et confirmé en Mt. 28:18-20. Es. 56:1 nous apprend que celui qui vient s'appelle « Yeshuw'ah ». Es. 59:17 le présente comme notre casque, ce qui fait écho au casque du salut en Ep. 6:17. Les murs de la Nouvelle Jérusalem portent son Nom (Es. 60:18). Ha. 3:8 nous dit que « Yeshuw'ah » montera sur ses chevaux, corroborant le récit de son retour en gloire dans Ap. 19:11-20. « Yeshuw'ah » est notre flambeau selon Es. 62:1 et Ap. 21:23.} y sera mis pour muraille et pour rempart.
\VS{2}Ouvrez les portes, et la nation juste, celle qui garde la fidélité, y entrera.
\VS{3}Tu gardes dans une paix parfaite celui dont l'esprit s'appuie sur toi, parce qu'il se confie en toi\FTNT{Es. 57:19 ; Ph. 4:6-7.}.
\VS{4}Confiez-vous en Yahweh à perpétuité, car le Rocher \FTNT{Voir commentaire en Es. 8:13-14. } des siècles est en Yahweh Dieu.
\VS{5}Car il a abaissé ceux qui habitaient aux lieux haut élevés, il a renversé la ville de haute retraite, il l'a renversée jusqu'à terre, il l'a réduite jusqu'à la poussière.
\VS{6}Le pied marchera dessus ; les pieds, dis-je, des pauvres, les plantes des misérables marcheront dessus.
\VS{7}Le sentier du juste est la droiture ; toi qui est juste, tu dresses au niveau le chemin du juste.
\VS{8}Aussi t'avons-nous attendu, ô Yahweh, dans le sentier de tes jugements ! Ton Nom et ton souvenir sont le désir de notre âme.
\VS{9}De nuit, je te désire de mon âme, et dès le point du jour, mon esprit qui est en moi te recherche ; car lorsque tes jugements s'exercent sur la terre, les habitants du monde apprennent la justice.
\VS{10}Est-il fait grâce au méchant ? Il n'en apprend point la justice, mais il agit méchamment sur la terre de la droiture, et il ne regarde pas à la majesté de Yahweh.
\VS{11}Yahweh, quand ta main est élevée, ils ne le voient pas. Mais ils verront et seront honteux à cause de leur jalousie pour ton peuple ; et le feu dont tu punis tes ennemis les dévorera.
\VS{12}Yahweh, tu ordonnes la paix pour nous, car aussi tout ce que nous faisons, c'est toi qui l'accomplis en nous.
\VS{13}Yahweh, notre Dieu, d'autres seigneurs que toi nous ont maîtrisés, mais c'est par toi seul que nous pouvons faire mention de ton Nom.
\VS{14}Ils sont morts, ils ne revivront plus, ils sont trépassés, ils ne se relèveront pas ; car tu les as châtiés et exterminés, et tu as fait périr toute mémoire d'eux\FTNT{Ec. 9:5.}.
\VS{15}Yahweh, tu avais accru la nation, tu avais accru la nation, tu as été glorifié, mais tu les as jetés loin dans toutes les extrémités de la terre.
\TextTitle{Un reste épargné de la colère de Yahweh}
\VS{16}Yahweh, étant en détresse ils se sont rendus auprès de toi ; ils se sont répandus en prières quand ton châtiment a été sur eux.
\VS{17}Comme celle qui est enceinte est en travail, et crie dans ses tranchées, lorsqu'elle est prête d'enfanter, ainsi avons-nous été devant ta face ô Yahweh !
\VS{18}Nous avons conçu et nous avons éprouvé des douleurs et nous avons comme enfanté du vent. Nous ne saurions en aucune manière délivrer le pays et les habitants de la terre habitable ne tomberaient point par notre force.
\VS{19}Tes morts vivront ! Même mon corps mort vivra ! Ils se relèveront. Réveillez-vous et réjouissez-vous avec des chants de triomphe, vous, habitants de la poussière ; car ta rosée est comme la rosée des herbes, et la terre jettera dehors les morts\FTNT{Os. 13:14 ; Da. 12:2 ; 1 Co. 15:52.}.
\VS{20}Va, mon peuple, entre dans tes cabinets et ferme ta porte derrière toi\FTNT{Mt. 6:6.} ; cache-toi pour un petit moment, jusqu'à ce que l'indignation soit passée.
\VS{21}Car voici, Yahweh s'en va sortir de son lieu pour visiter l'iniquité des habitants de la terre, commise contre lui ; alors la terre découvrira le sang qu'elle aura reçu et ne couvrira plus ceux qu'on a mis à mort.
\Chap{27}
\TextTitle{Israël rétabli}
\VerseOne{}En ce jour-là, Yahweh frappera de sa dure, grande et forte épée le Léviathan\FTNT{Ps. 104:26 ; Job. 40:20}, le serpent fuyard, le Léviathan, dis-je, le serpent tortueux, et il tuera le monstre qui est dans la mer.
\VS{2}En ce jour-là, chantez sur la vigne désirable\FTNT{Esaïe annonce ici le rétablissement d'Israël. Voir également Ro. 11:1-24.}.
\VS{3}C'est moi Yahweh qui la garde, je l'arrose à chaque instant, je la garde nuit et jour, afin que personne ne lui fasse du mal.
\VS{4}Il n'y a point de fureur en moi ; qu'on me donne des ronces, des épines pour les combattre ! Je marcherai contre elles, je les brulerai toutes ensemble.
\VS{5}Ou bien, qu'il saisisse ma force, qu'il fasse la paix avec moi, qu'il fasse la paix avec moi.
\VS{6}Il fera que Jacob prendra racine, Israël  fleurira, et s'épanouira ; et il remplira de fruits le dessus de la terre habitable.
\VS{7}L'a-t-il frappé comme il a frappé celui qui le frappaient ? L'a-t-il tué comme il a tué ceux qui le tuaient ?
\VS{8}Tu as plaidé avec elle modérément, quand tu l'as renvoyée ; en l'emportant par le vent rude au jour du vent d'orient.
\VS{9}C'est pourquoi l'expiation de l'iniquité de Jacob sera faite par ce moyen, et ceci en sera le fruit entier, que son péché sera ôté ; quand il aura transformé toutes les pierres des autels comme des pierres de chaux réduites en poussière ; et lorsque les idoles d'Asherah et les statues consacrées au soleil ne seront plus debout.
\VS{10}Car la ville fortifiée est désolée, la demeure agréable est abandonnée et délaissée comme le désert. Là pâture le veau, il y gîte et broute les branches.
\VS{11}Quand son branchage est sec, il est brisé ; et les femmes y venant en allument un feu. Car c'est un peuple sans intelligence\FTNT{De. 32:28 ; Es. 1:3.},  c'est pourquoi celui qui l'a fait n'a point eu pitié de lui, et celui qui l'a formé ne lui a point fait grâce.
\VS{12}Il arrivera en ce jour-là que Yahweh secouera, depuis le cours du fleuve jusqu'au torrent d'Egypte ; mais vous serez glanés un à un, ô enfants d'Israël.
\VS{13}Et il arrivera en ce jour-là qu'on sonnera du grand shofar, et ceux qui étaient exilés au pays d'Assyrie, et ceux qui avaient été chassés au pays d'Egypte, reviendront et se prosterneront devant Yahweh, sur la sainte montagne, à Jérusalem.
\Chap{28}
\TextTitle{Malheur et captivité d'Ephraïm en Assyrie}
\VerseOne{}Malheur à la couronne de fierté des ivrognes d'Ephraïm, la noblesse de la gloire qui n'est qu'une fleur qui tombe ; ceux qui sont sur le sommet de la grasse vallée sont étourdis de vin !
\VS{2}Voici, le Seigneur a dans sa main un homme fort et puissant, semblable à une tempête de grêle, à un tourbillon destructeur, à une tempête de grosses eaux débordées ; il la fera tomber à terre avec la main.
\VS{3}Elle seront foulées aux pieds, la couronne de fierté et les ivrognes d'Ephraïm.
\VS{4}Et la noblesse de sa gloire qui est sur le sommet de la fertile vallée, ne sera qu'une fleur qui tombe ; ils seront comme les fruits précoces avant l'été, aussitôt que celui qui regarde les voit, à peine ils sont dans sa main, il les dévore.  
\VS{5}En ce jour-là, Yahweh des armées sera une couronne de noblesse et un diadème de gloire pour le reste de son peuple ;
\VS{6}et un esprit de jugement pour celui qui sera assis au siège de jugement, et une force à ceux qui dans le combat repousseront l'ennemi jusqu'à la porte.
\VS{7}Mais eux aussi, s'oublient dans le vin, et se fourvoient dans les boissons fortes ; le sacrificateur et le prophète s'oublient dans les boissons fortes ; ils sont engloutis par le vin, ils se fourvoient à cause des boissons fortes ; ils s'oublient dans la vision, ils vacillent dans le jugement.
\VS{8}Car toutes leurs tables sont couvertes de vomissements et d'ordures ; aussi il n'y a plus de place !
\VS{9}A qui enseigne-t-on la connaissance ? A qui fait-on comprendre l'enseignement ? Est-ce à ceux qu'on vient de sevrer et de retirer de la mamelle ?
\VS{10}Car il faut leur donner précepte après précepte, précepte après précepte, règle après règle, règle après règle, un peu ici, un peu là\FTNT{Hé. 5:12.}.
\VS{11}C'est pourquoi, il parlera à ce peuple par des lèvres qui balbutient et une langue étrangère.
\VS{12}Il leur disait : Voici le repos, donnez du repos à celui qui est fatigué ;  voici le soulagement ! Mais ils n'ont point voulu écouter.
\VS{13}Ainsi la parole de Yahweh sera pour eux précepte après précepte, précepte après précepte, règle après règle, règle après règle, un peu ici, un peu là ; afin qu'ils aillent et tombent à la renverse, et qu'ils soient brisés, et afin qu'ils tombent dans le piège et qu'ils soient pris.
\TextTitle{Yahweh rompt le pacte du scheol par une pierre angulaire}
\VS{14}C'est pourquoi écoutez la parole de Yahweh, vous hommes moqueurs, qui dominez sur ce peuple qui est à Jérusalem !
\VS{15}Car vous dites : Nous avons fait un pacte avec la mort, et nous avons un accord avec le scheol ; quand le fléau débordé passera, il ne viendra pas sur nous, car nous avons le mensonge pour refuge et nous nous sommes cachés sous la fausseté.
\VS{16}C'est pourquoi ainsi parle le Seigneur Yahweh : Voici, je mettrai pour fondement en Sion une pierre\FTNT{Voir commentaire en Es. 8:13-16.}, une pierre éprouvée, la pierre angulaire la plus précieuse, pour être un fondement solide ; celui qui croira ne se hâtera point.
\VS{17}Et je mettrai le jugement à l'équerre, et la justice au niveau ; et la grêle détruira le refuge du mensonge, et les eaux inonderont le lieu où l'on se retirait.
\VS{18}Et votre pacte avec la mort sera détruit, votre accord avec le scheol ne tiendra pas ; quand le fléau débordé passera, vous en serez foulés.
\VS{19}Dès qu'il passera, il vous emportera. Or il passera tous les matins, le jour et la nuit ; et dès qu'on en entendra le bruit, il n'y aura que terreur.
\VS{20}Car le lit sera trop court, et on ne pourra pas s'y étendre, et la couverture trop étroite pour s'en envelopper.
\VS{21}Car Yahweh se lèvera comme à la montagne de Peratsim, et il sera ému comme dans la vallée de Gabaon, pour faire son œuvre, son œuvre extraordinaire, et pour faire son travail, son travail non accoutumé.
\VS{22}Maintenant donc, ne vous moquez plus, de peur que vos liens ne soient renforcés, car j'ai entendu de par le Seigneur, Yahweh des armées, que la destruction est déterminée sur tout le pays. 
\VS{23}Prêtez l'oreille, et écoutez ma voix ; soyez attentifs, et écoutez mon discours !
\VS{24}Celui qui laboure pour semer, laboure-t-il tous les jours ? Ne casse-t-il pas et ne rompt-il pas les mottes de sa terre ? 
\VS{25}Quand il en aura aplani la surface, ne sèmera-t-il pas la vesce\FTNT{La vesce est un genre de plante herbacées de la famille des légumineuse} ; ne répandra-t-il pas le cumin, ne mettra-t-il pas le froment au meilleur endroit, et l'orge en son lieu assigné, et l'épeautre\FTNT{L'épeautre est une espèce de blé} en son quartier ?
\VS{26}Parce que son Dieu l'a instruit, et lui a enseigné ce qu'il faut faire.
\VS{27}Car on ne foule pas la vesce avec la herse\FTNT{La herse est un instrument agricole permettant de travailler la terre en surface}, et on ne tourne point la roue du chariot sur le cumin ; mais on bat la vesce avec la verge, et le cumin avec le bâton.
\VS{28}Le blé avec lequel on fait le pain se menuise, car le laboureur ne le foule pas entièrement ; et quoiqu'il l'écrase avec la roue de son chariot, néanmoins il ne le menuisera pas avec ses chevaux.
\VS{29}Cela aussi vient de Yahweh des armées qui est admirable en conseil et magnifique en moyens.
\Chap{29}
\TextTitle{Avertissement d'un châtiment imminent}
\VerseOne{}Malheur à Ariel\FTNT{Ariel : Lion de Dieu, nom appliqué à Jérusalem.}, à Ariel, la ville dont David fit sa demeure ! Ajoutez année à année, qu'on égorge des victimes pour les fêtes.
\VS{2}Mais je mettrai Ariel à l'étroit, il n'y aura que tristesse et deuil ; et elle sera pour moi comme Ariel.
\VS{3}Car je camperai en rond contre toi, et je t'assiégerai avec des tours, et je dresserai contre toi des retranchements.
\VS{4}Et tu seras abaissée, et tu parleras depuis la terre, et ta parole sortira étouffée par la poussière ; et ta voix sortira de terre comme celle d'un esprit de Python , et ta parole marmottera comme si elle sortait de la poussière.
\VS{5}La multitude de tes étrangers sera comme une fine poussière ; et la multitude des guerriers sera comme la balle qui passe, et cela sera pour un petit moment.
\VS{6}Elle sera visitée par Yahweh des armées avec des tonnerres, des tremblements de terre, et un grand bruit\FTNT{Za. 14:13-14 ; Ap. 16:18-19.} ; avec la tempête, le tourbillon, et avec la flamme d'un feu dévorant.
\VS{7}Et la multitude de toutes les nations qui feront la guerre à Ariel, et tous ceux qui la combattront, et ceux qui la serreront de près seront comme un songe d'une vision de nuit.
\VS{8}Et il arrivera que comme celui qui a faim rêve qu'il mange, mais quand il se réveille son âme est vide ; et comme celui qui a soif rêve qu'il boit, mais quand il se réveille il est épuisé, et son âme est altérée ; ainsi sera-t-il de la multitude de toutes les nations qui combattront contre la montagne de Sion.
\TextTitle{Yahweh donne les raisons du châtiment}
\VS{9}Arrêtez-vous et soyez étonnés ! Ecriez-vous et criez ! Ils sont ivres, mais non de vin ; ils chancellent, mais non pas à cause des boissons fortes.
\VS{10}Car Yahweh a répandu sur vous un esprit d'un profond sommeil\FTNT{Ro. 11:8.} ; il a fermé vos yeux, il a bandé ceux de vos prophètes et de vos principaux voyants.
\VS{11}Et toute vision est pour vous comme les paroles d'un livre cacheté que l'on donne à un homme de lettres en lui disant : Nous te prions, lis donc cela ! Et qui répond : Je ne le puis, car il est cacheté ;
\VS{12}puis si on le donne à quelqu'un qui n'est pas un homme de lettres, en lui disant : Nous te prions, lis donc cela ! Et qui répond : Je ne sais pas lire.
\VS{13}C'est pourquoi le Seigneur dit : Parce que ce peuple s'approche de moi de sa bouche et qu'il m'honore de ses lèvres, mais que son cœur est éloigné de moi ; et parce que la crainte qu'il a de moi lui a été enseigné par un commandement d'hommes\FTNT{Mt. 15:8-9 ; Mc. 7:6-7.}.
\VS{14}A cause de cela, voici, je continuerai de faire à l'égard de ce peuple-ci des merveilles et des prodiges étranges ; et la sagesse de ses sages périra, et l'intelligence de ses hommes intelligents disparaîtra.
\VS{15}Malheur à ceux qui cachent profondément leurs desseins, pour les dissimuler à Yahweh, et dont les œuvres sont dans les ténèbres, et qui disent : Qui nous voit, et qui nous connaît\FTNT{Es. 47:10 ; Ez. 8:12 ; Ps. 10:11 ; Ps. 94:7.} ?
\VS{16}Ce que vous renversez ne sera-t-il pas réputé comme l'argile d'un potier ? Même l'ouvrage dira-t-il de celui qui l'a fait : Il ne m'a point fait ? Et la chose formée dira-t-elle de celui qui l'a formée : Il n'a point d'intelligence\FTNT{Ps. 100:3.} ?
\TextTitle{Yahweh rachète Jacob}
\VS{17}Le Liban ne sera-t-il pas encore dans très peu de temps changé en un Carmel ? Et Carmel ne sera-t-il pas considéré comme une forêt ?
\VS{18}En ce jour-là, les sourds entendront les paroles du livre, et les yeux des aveugles, étant délivrés de l'obscurité et des ténèbres, verront\FTNT{Mt. 11:5 ; Lu. 7:22.}.
\VS{19}Les humbles auront joie sur joie en Yahweh, et les pauvres d'entre les hommes se réjouiront dans le Saint d'Israël\FTNT{Mt. 5:3-11.}.
\VS{20}Car l'oppresseur prendra fin, le moqueur sera consumé, et tous ceux qui veillaient pour commettre l'iniquité seront retranchés\FTNT{Ap. 20:10.},
\VS{21}ceux qui rendaient coupable les hommes pour une parole, qui tendaient des pièges à celui qui les reprenait à la porte, et qui faisaient tomber le juste en confusion. 
\VS{22}C'est pourquoi ainsi parle Yahweh, lui qui a racheté Abraham, à la maison de Jacob : Jacob ne sera plus honteux, et sa face ne pâlira plus.
\VS{23}Car quand il verra ses fils, ouvrage de mes mains, au milieu de lui, ils sanctifieront mon Nom ; ils sanctifieront, dis-je, le Saint de Jacob, et ils craindront le Dieu d'Israël.
\VS{24}Et ceux dont l'esprit s'était fourvoyé deviendront intelligents, et ceux qui murmuraient apprendront la doctrine.
\Chap{30}
\TextTitle{Mise en garde contre les alliances étrangères}
\VerseOne{}Malheur aux enfants rebelles, dit Yahweh, qui prennent des conseils, et non pas de moi, et qui se forgent des idoles de métal où mon esprit n'est point, afin d'ajouter péché sur péché.
\VS{2}Qui sans avoir interrogé ma bouche, marchent pour descendre en Egypte, afin de se fortifier de la force de Pharaon et se retirer sous l'ombre de l'Egypte\FTNT{Jé. 42:19}.
\VS{3}Car la force de Pharaon sera pour vous une honte, et le refuge sous l'ombre de l'Egypte votre confusion.
\VS{4}Car ses princes sont à Tsoan, et ses messagers ont atteint Hanès.
\VS{5}Tous seront rendus honteux par un peuple qui ne leur profitera de rien, ils n'en recevront aucun secours ni aucun avantage, il sera leur honte et leur opprobre.
\VS{6}Les bêtes sont chargées pour aller au midi, ils portent leurs richesses sur les dos des ânons, et leurs trésors sur la bosse des chameaux, vers le peuple qui ne leur profitera point dans le pays de détresse et d'angoisse, d'où viennent le vieux lion et le lion, la vipère et le serpent volant ; .
\VS{7}Car le secours de l'Egypte n'est que vanité et néant ; c'est pourquoi je crie ceci : Leur force est de se tenir tranquille.
\VS{8}Va maintenant, et écris-le en leur présence sur une table, et rédige-le par écrit dans un livre, afin que cela demeure pour le temps à venir, à perpétuité, à jamais ;
\VS{9}que c'est ici un peuple rebelle, des enfants menteurs, des enfants qui ne veulent point écouter la loi de Yahweh\FTNT{No. 20: 3-5 ; De. 9:7 ; Ac. 7:51.} ;
\VS{10}qui disent aux voyants : Ne voyez pas ! Et aux prophètes : Ne nous prophétisez pas des choses droites, mais dites-nous des choses agréables, voyez des choses trompeuses\FTNT{2 Ti. 4:3-4 ; Mi. 2:6.} !
\VS{11}Retirez-vous du chemin, détournez-vous du sentier, éloignez de notre présence le Saint d'Israël\FTNT{Jn. 14:6.}.
\VS{12}C'est pourquoi ainsi dit le Saint d'Israël : Parce que vous rejetez cette parole et que vous vous confiez dans l'oppression et dans les détours, et que vous vous êtes appuyés sur ces choses,
\VS{13}à cause de cela, cette iniquité sera pour vous comme la fente d'une muraille qui va tomber, un renflement dans un mur élevé, dont la ruine vient soudainement, et en un instant.
\VS{14}Il la brise donc comme on brise un vase de terre, que l'on n'épargne point, et de ses pièces, il ne se trouve pas un tesson pour prendre du feu au foyer, ou pour puiser de l'eau à la citerne.
\TextTitle{La confiance en Yahweh, la vraie force}
\VS{15}Car ainsi a parlé le Seigneur Yahweh, le Saint d'Israël : En vous tenant tranquille et en repos vous serez sauvés ; votre force sera en vous tenant en repos et en espérance. Mais vous ne l'avez point voulu.
\VS{16}Et vous avez dit : Non, mais nous nous enfuirons sur des chevaux ; à cause de cela vous vous enfuirez. Et vous avez dit : Nous monterons sur des chevaux rapides ; à cause de cela ceux qui vous poursuivront seront rapides.
\VS{17}Mille d'entre vous s'enfuiront à la menace d'un seul ; vous vous enfuirez à la menace de cinq ; jusqu'à ce que vous soyez abandonnés comme un arbre tout ébranché au sommet d'une montagne, et comme un étendard sur la colline.
\VS{18}Cependant Yahweh attend pour vous faire grâce, et ainsi il sera exalté pour vous faire miséricorde ; car Yahweh est le Dieu de jugement : Ô bienheureux sont tous ceux qui se confient en lui !
\VS{19}Car le peuple demeurera dans Sion et dans Jérusalem. Tu ne pleureras point ! Certes, il te fera grâce dès qu'il entendra ton cri ; dès qu'il aura entendu, il t'exaucera.
\VS{20}Le Seigneur vous donnera du pain de détresse, et de l'eau d'angoisse, mais tes enseignants ne s'envoleront plus, et tes yeux verront tes enseignants.
\VS{21}Et tes oreilles entendront la parole de celui qui sera derrière toi, disant : Voici le chemin, marchez-y ,soit que vous tiriez à droite, soit que vous tiriez à gauche !
\VS{22}Et vous tiendrez pour souillés les chapiteaux des images taillées faites d'argent, et les ornements faits d'or fondu ; tu les jetteras au loin comme un sang impur, et tu leur diras : Hors d'ici ! 
\VS{23}Alors il donnera la pluie sur la semence que tu auras semées en terre, et le grain du revenu de la terre sera abondant et bien nourri ; en ce jour-là, ton bétail paîtra dans un pâturage spacieux\FTNT{Jn. 14:6.}.
\VS{24}Les bœufs et les ânes qui labourent la terre mangeront le pur fourrage de ce qui aura été vanné avec la pelle et le van.
\VS{25}Et il y aura des ruisseaux d'eau courante sur toute haute montagne, et sur toute colline haut élevée, au jour de la grande tuerie, quand les tours tomberont.
\VS{26}Et la lumière de la lune sera comme la lumière du soleil ; et la lumière du soleil sera sept fois plus grande, comme si c'était la lumière de sept jours, le jour où Yahweh bandera la blessure de son peuple, et qu'il guérira la blessure de sa plaie.
\TextTitle{Jugement de Yahweh sur les Assyriens}
\VS{27}Voici, le Nom de Yahweh vient de loin, sa colère est ardente, et une pesante charge ; ses lèvres sont pleines d'indignation, et sa langue est comme un feu dévorant.
\VS{28}Son Esprit est comme un torrent qui déborde et atteint jusqu'au milieu du cou, pour disperser les nations d'une telle dispersion qu'elles seront réduites à néant, et il est comme une bride aux mâchoires des peuples, qui les fera errer.
\VS{29}Vous aurez un cantique comme la nuit où l'on célèbre une fête solennelle ; vous aurez le cœur joyeux comme celui qui marche au son de la flûte, pour aller à la montagne de Yahweh, vers le Rocher d'Israël.
\VS{30}Et Yahweh fera entendre sa voix, pleine de majesté, et il montrera où aura assené son bras dans l'indignation de sa colère, avec une flamme de feu dévorant, avec éclat, tempête, et pierres de grêle.
\VS{31}Car l'Assyrien, qui frappait du bâton, sera effrayé par la voix de Yahweh.
\VS{32}Et partout où passe le bâton dont Yahweh l'a assené, et par lequel il combattra dans les batailles à bras élevé, on entendra les tambourins et les harpes.
\VS{33}Car Topheth\FTNT{Topheth : Lieu pour brûler. Un lieu à l'extrémité sud-est de la vallée de Hinnom au sud de Jérusalem.} est déjà préparée, et même elle est apprêtée pour le roi ; on a fait son bûcher profond et large ; son bûcher c'est du feu et du bois en abondance ; le souffle de Yahweh l'allume comme un torrent de soufre.
\Chap{31}
\TextTitle{Le secours de Yahweh préférable à celui de l'Egypte}
\VerseOne{}Malheur à ceux qui descendent en Egypte pour avoir de l'aide, et qui s'appuient sur les chevaux, et qui mettent leur confiance dans leurs chars parce qu'ils sont nombreux, et en leurs cavaliers quand ils sont bien forts, mais qui ne regardent pas vers le Saint d'Israël, et ne recherchent pas Yahweh.
\VS{2}Et cependant, c'est lui qui est sage, et il fait venir le malheur et ne révoque point sa parole ; il s'élève contre la maison des méchants et contre ceux qui aident les ouvriers d'iniquité.
\VS{3}Or les Egyptiens sont des hommes et non Dieu ; et leurs chevaux sont chair et non esprit. Quand Yahweh étendra sa main, et celui qui donne du secours sera renversé ; et celui à qui le secours est donné tombera ; et eux tous ensemble seront consumés.
\VS{4}Mais ainsi m'a dit Yahweh : Comme le lion, comme le lionceau rugit sur sa proie, et quoiqu'on appelle contre lui un grand nombre de bergers, il ne se laisse ni effrayer par leur cri, ni abaisser par leur bruit ; ainsi Yahweh des armées descendra pour combattre en faveur de la montagne de Sion et de sa colline.
\VS{5}Comme les oiseaux volent, ainsi Yahweh des armées défendra Jérusalem, la défendant et la délivrant, passant outre et la sauvant\FTNT{De. 32:11 ; Ps. 91:4 ; Mt. 23:37.}.
\VS{6}Retournez vers celui de qui les enfants d'Israël se sont étrangement éloignés.
\VS{7}Car en ce jour-là, chacun rejettera ses idoles d'argent et ses idoles d'or que vos propres mains ont fabriquées pour vous faire pécher.
\VS{8}Et l'Assyrien tombera par l'épée qui n'est pas celle d'un vaillant homme, et l'épée qui n'est pas celle d'un homme le dévorera ; et il s'enfuira devant l'épée, et ses jeunes hommes seront rendus tributaires.
\VS{9}Et saisi de frayeur, il s'enfuira à sa forteresse, et ses chefs seront effrayés à cause de la bannière, dit Yahweh, qui a son feu dans Sion et son fourneau dans Jérusalem.
\Chap{32}
\TextTitle{La venue de l'Esprit annonce la paix et la justice}
\VerseOne{}Voici, un roi régnera selon la justice, et les princes gouverneront avec équité.
\VS{2}Et un homme sera comme le lieu où l'on se cache du vent et comme un asile contre la tempête ; comme des ruisseaux d'eau dans un pays sec, et l'ombre d'un grand rocher dans une terre altérée.
\VS{3}Alors les yeux de ceux qui voient ne seront point retenus, et les oreilles de ceux qui entendent seront attentives.
\VS{4}Et le cœur des étourdis entendra la science, et la langue de ceux qui balbutient parlera aisément et nettement.
\VS{5}Le chiche ne sera plus appelé libéral, et l'avare trompeur ne sera plus nommé magnifique.
\VS{6}Car l'homme vil dira des choses viles, et son cœur ne machine qu'iniquité, pour exécuter son hypocrisie et pour proférer des faussetés contre Yahweh, pour rendre vide l'âme de celui qui a faim, et faire tarir la boisson de celui qui a soif\FTNT{Jn. 10:10.}.
\VS{7}Les instruments de l'avare sont pernicieux ; il prend des conseils pleins de machinations, pour attraper par des paroles de mensonge les affligés, même quand la cause du pauvre est juste\FTNT{2 Pi. 2:3.}.
\VS{8}Mais le libéral forme des conseils de libéralité et se lève pour user de libéralité.
\VS{9}Femmes qui êtes à votre aise, levez-vous, écoutez ma voix ! Filles qui vous tenez assurées, prêtez l'oreille à ma parole !
\VS{10}Dans un an et quelques jours, vous qui vous tenez assurées serez troublées ; car la vendange a manqué, la récolte n'arrivera plus.
\VS{11}Vous qui êtes à votre aise, tremblez ! Vous qui vous tenez assurées, soyez troublées ! Dépouillez-vous, quittez vos habits et ceignez de sacs vos reins !
\VS{12}On se frappe la poitrine à cause de la vigne abondante en fruits.
\VS{13}Les épines et les ronces montent sur la terre de mon peuple, même sur toutes les maisons où il y a de la joie et sur la ville joyeuse.
\VS{14}Car le palais est abandonné, la multitude de la cité est délaissée ; les lieux inaccessibles du pays et les forteresses serviront de cavernes à toujours ; les ânes sauvages y joueront, et les troupeaux y paîtront,
\VS{15}jusqu'à ce que l'Esprit soit répandu d'en haut sur nous\FTNT{Joë. 2:28 ; Za.12:10 ; Ac. 2:17-18.}, et que le désert devienne un Carmel et que Carmel  soit considéré comme une forêt.
\VS{16}Le jugement habitera dans le désert et la justice se tiendra en Carmel.
\VS{17}La justice produira de la paix, et le fruit de la justice sera le repos et la sécurité pour toujours.
\VS{18}Mon peuple habitera dans une demeure paisible, et dans des habitations assurées, et dans un repos fort tranquille.
\VS{19}Mais la grêle tombera sur la forêt, et la ville sera entièrement abaissée.
\VS{20}Heureux vous qui semez sur toutes les eaux, et qui laissez sans entraves le pied du bœuf et de l'âne !
\Chap{33}
\TextTitle{Yahweh se lève}
\VerseOne{}Malheur à toi qui dépouilles et qui n'as pas été dépouillé ! Qui pilles et qu'on n'a pas encore pillé ! Quand tu auras fini de dépouiller, tu seras dépouillé ; et quand tu auras achevé de piller, on te pillera.
\VS{2}Yahweh, aie pitié de nous ! Nous nous attendons à toi ! Sois leur bras dès le matin et notre délivrance au temps de la détresse !
\VS{3}Au son du tumulte, les peuples s'enfuient ; quand tu te lèves, les nations se dispersent.
\VS{4}Et votre butin est recueilli comme on rassemble les sauterelles ; on saute  dessus comme sautellent les sauterelles.
\VS{5}Yahweh est élevé, car il habite dans les lieux élevés ; il remplit Sion de jugement et de justice\FTNT{Ps. 97:9.}.
\VS{6}Et la sagesse et la science seront la certitude de ta durée, et la force de ton salut ; la crainte de Yahweh est son trésor.
\VS{7}Voici, leurs hérauts poussent des cris au-dehors, et les messagers de paix pleurent amèrement.
\VS{8}Les routes sont réduites en désolation, les passants n'y passent plus. Il a rompu l'alliance, il rejette les villes, il ne fait plus cas des hommes.
\VS{9}On mène le deuil, la terre languit. Le Liban est honteux et flétri. Le Saron est comme un désert. Le Basan et le Carmel secouent leur feuillage.
\VS{10}Maintenant je me lèverai, dit Yahweh, maintenant je serai exalté, maintenant je serai élevé.
\VS{11}Vous avez conçu du foin, et vous enfanterez de la paille ; votre souffle vous dévorera comme le feu.
\VS{12}Et les peuples seront des fourneaux de chaux ; ils seront brûlés au feu comme des épines coupées.
\VS{13}Vous qui êtes loin, écoutez ce que j'ai fait ! Et vous qui êtes près, connaissez ma force !
\TextTitle{Yahweh assure la paix aux justes}
\VS{14}Les pécheurs sont effrayés dans Sion, et le tremblement saisit les hypocrites, tellement qu'ils disent : Qui de nous pourra séjourner avec le feu dévorant\FTNT{Hé. 12:29.} ? Qui de nous pourra séjourner avec les flammes éternelles ?
\VS{15}Celui qui observe la justice et qui profère des choses droites ; celui qui rejette le gain déshonnête d'extorsion, et qui secoue ses mains pour ne pas accepter un présent ; celui qui bouche ses oreilles pour ne pas entendre des propos sanguinaires, et qui ferme ses yeux pour ne pas voir le mal,
\VS{16}Celui-là habitera dans des lieux élevés, des forteresses assises sur des rochers seront sa haute retraite ; son pain lui sera donné, et ses eaux ne lui manqueront point\FTNT{Jn. 4:14 ; Jn. 6:33-35 ; Ap 21:6.}.
\VS{17}Tes yeux contempleront le roi dans sa beauté ; et ils regarderont la terre éloignée.
\VS{18}Ton cœur méditera-il la frayeur, en disant : Où est le secrétaire, où est le trésorier ? Où est celui qui tient le compte des tours ?
\VS{19}Tu ne verras plus le peuple fier, le peuple au langage inconnu qu'on n'entend pas, et de langue bégayante qu'on ne comprend pas.
\VS{20}Regarde Sion, la ville de nos fêtes solennelles ! Que tes yeux voient Jérusalem, séjour tranquille, tabernacle qui ne sera pas transportée, et dont les pieux ne seront jamais ôtés, et dont les cordages ne seront point rompus\FTNT{Ap. 21:2.}.
\VS{21}C'est là que Yahweh nous est glorieux ; c'est le lieu de fleuves, de vastes rivières, où n'ira pas de navire à rame et où aucun gros navire passera.
\VS{22}Parce que Yahweh est notre Juge, Yahweh est notre Législateur, Yahweh est notre Roi\FTNT{Jésus-Christ exerce toutes les fonctions gouvernementales : législatives, exécutives et judiciaires.} ; c'est lui qui vous sauvera.
\VS{23}Tes cordages sont lâchés ; et ainsi ils ne tiennent point ferme leur mât et on n'étendra point la voile. Alors la dépouille d'un grand butin est partagé ; même les boiteux pillent le butin.
\VS{24}Et celui qui fait sa demeure dans la maison ne dit point : Je suis malade ! Le peuple qui habite en elle reçoit le pardon de ses iniquités.
\Chap{34}
\TextTitle{Le jugement des nations\FTNTT{Ap. 19:17-21.}}
\VerseOne{}Approchez-vous nations, pour écouter ! Et vous peuples, soyez attentifs ! Que la terre et tout ce qui la remplit écoute ! Que le monde habitable et tout ce qui y est produit écoute !
\VS{2}Car l'indignation de Yahweh est sur toutes les nations, et sa fureur sur toute leur armée ; il les voue à l'interdit, il les livre pour être tuées.
\VS{3}Leurs blessés à morts sont jetés là, et la puanteur de leurs corps morts se répand et les montagnes découlent de leur sang.
\VS{4}Et toute l'armée des cieux se fond ; les cieux sont roulés comme un livre\FTNT{Ap. 6:14.}, et toute leur armée tombe, comme tombe la feuille de la vigne, et comme tombe celle du figuier\FTNT{Mt. 24:28 ; Mc. 13:25.}.
\VS{5}Parce que mon épée s'est enivrée dans les cieux, voici, elle va descendre en jugement contre Edom, et contre le peuple que j'ai voué à l'interdit.
\VS{6}L'épée de Yahweh est pleine de sang ; engraissée de graisse, et du sang des agneaux et des boucs, et de la graisse des reins de béliers ; car il y a des sacrifices de Yahweh à Botsra, et une grande tuerie dans le pays d'Edom.
\VS{7}Les licornes descendent avec eux, et les bœufs avec les taureaux ; leur terre est enivrée de sang, et leur poussière engraissée de graisse.
\VS{8}Car c'est un jour de vengeance pour Yahweh, une année de rétribution pour maintenir la cause de Sion\FTNT{Jé. 46:10 ; Joë. 2:2 ; So. 1:15.}.
\VS{9}Et ces torrents d'Edom seront changés en poix, et sa poussière en soufre, et sa terre deviendra de la poix ardente.
\VS{10}Elle ne sera point éteinte ni jour ni nuit ; sa fumée montera éternellement, elle sera désolée de génération en génération ; il n'y aura personne qui passe par elle à jamais.
\VS{11}Le pélican et le hérisson la posséderont, la chouette et le corbeau  y habiteront ; et on étendra sur elle la ligne de la désolation et le niveau de désordre.
\VS{12}Ses magistrats crieront qu'il n'y a plus là de royaume, et tous ses princes seront réduits à néant.
\VS{13}Les épines croîtront dans ses palais, les chardons et les buissons dans ses forteresses; elle sera la demeure des dragons, et le parvis des hiboux.
\VS{14}Les bêtes sauvages des déserts rencontreront les bêtes sauvages des îles ; et les boucs s'y appelleront les uns les autres ; là aussi, la Lilith\FTNT{Lilith est le nom d'une déesse de la nuit connue pour être un démon nocturne qui hantait les lieux déserts d'Edom.} aura sa demeure et trouvera son lieu de repos ;
\VS{15}là le martinet fera son nid, déposera ses œufs, les couvera, et recueillera ses petits à son ombre ; et là aussi se rassembleront tous les vautours.
\VS{16}Consultez le livre de Yahweh et lisez : Il n'en manquera pas un seul point ; ni l'un ni l'autre ne manqueront ; car c'est ma bouche qui l'a ordonné, et son Esprit qui les rassemblera.
\VS{17}Car il leur a jeté le sort, et sa main leur a partagé cette terre au cordeau, ils la posséderont toujours, ils l'habiteront d'âge en âge.
\Chap{35}
\TextTitle{Yahweh se révèle et sauve son peuple}
\VerseOne{}Le désert et le lieu aride seront dans la joie ; le lieu solitaire se réjouira et fleurira comme une rose.
\VS{2}Il fleurira abondamment, et se réjouira, se réjouissant même et chantant en triomphe. La gloire du Liban lui est donnée, avec la magnificence de Carmel et de Saron ; ils verront la gloire de Yahweh et la magnificence de notre Dieu.
\VS{3}Renforcez les mains lâches, et fortifiez les genoux tremblants\FTNT{Hé. 12:12.}.
\VS{4}Dites à ceux qui ont le cœur troublé : Prenez courage et ne craignez plus\FTNT{Jn. 14:1 ; Jn. 16:33.} ; voici votre Dieu, la vengeance viendra, la rétribution de Dieu ; il viendra lui-même et vous délivrera.
\VS{5}Alors les yeux des aveugles seront ouverts, et les oreilles des sourds seront débouchées.
\VS{6}Alors le boiteux sautera comme un cerf, et la langue du muet chantera en triomphe\FTNT{Esaïe a annoncé la venue de Yahweh lui-même. Cette prophétie s'est parfaitement accomplie en Jésus-Christ qui a réalisé tout ce qui avait été prédit. « Allez rapporter à Jean ce que vous entendez et ce que vous voyez : Les aveugles voient, les boiteux marchent, les lépreux sont purifiés, les sourds entendent, les morts ressuscitent, et l'Evangile est annoncé aux pauvres » (Mt. 11:4-5).}. Car des eaux jailliront dans le désert, et des torrents dans le lieu solitaire.
\VS{7}Et les lieux secs deviendront des étangs, et la terre desséchée deviendra des sources d'eaux ; et dans les repaires où des dragons faisaient leur gîte, il y aura un parvis à roseaux et à joncs.
\VS{8}Il y aura là un sentier et un chemin, qu'on appellera le chemin de sainteté ; celui qui est souillé n'y passera point, mais il sera pour ceux-là ; celui qui va son chemin, et les insensés ne s'y égareront point\FTNT{Mt. 7:13-14 ; Jn. 14:6.}.
\VS{9}Là il n'y aura point de lion ; et aucune des bêtes qui ravissent les autres, n'y montera, et ne s'y trouvera ; mais les rachetés y marcheront.
\VS{10}Ceux dont Yahweh a payé la rançon\FTNT{Jésus-Christ est Yahweh qui a payé notre rançon (Mc. 10:45).}, retourneront, et viendront en Sion avec chant de triomphe, et une joie éternelle sera sur leur tête ; ils obtiendront la joie et l'allégresse ; la douleur et le gémissement s'enfuiront.
\Chap{36}
\TextTitle{Invasion de Sanchérib, menaces de Rabschaké\FTNTT{2 R. 18:9-37 ; 2 Ch. 32:1-19.}}
\VerseOne{}La quatorzième année du roi Ezéchias, Sanchérib, roi d'Assyrie, monta contre toutes les villes fortes de Juda et les prit\FTNT{2 R. 18:17.}.
\VS{2}Puis le roi d'Assyrie envoya de Lakis à Jérusalem, vers le roi Ezéchias, Rabschaké avec une puissante armée. Rabschaké s'arrêta à l'aqueduc de l'étang supérieur, sur le chemin du champ du foulon.
\VS{3}Alors Eliakim, fils de Hilkija, chef de la maison du roi, Schebna, le secrétaire, et Joach, fils d'Asaph, l'archiviste, sortirent vers lui.
\VS{4}Rabschaké leur dit : Dites maintenant à Ezéchias : Ainsi parle le grand roi, le roi d'Assyrie : Quelle est cette confiance que tu as ?
\VS{5}Je te le dis, ce ne sont là que des paroles ; mais il faut pour la guerre de la prudence et de la force. Or maintenant en qui t'es tu confié pour t'être rebellé contre moi ?
\VS{6}Voici, tu t'es confié sur ce bâton qui n'est qu'un roseau cassé, sur l'Egypte, qui perce et traverse la main de celui qui s'appuie dessus ; tel est Pharaon, roi d'Egypte, à tous ceux qui se confient en lui.
\VS{7}Que si tu me dis : Nous nous confions en Yahweh, notre Dieu. Mais n'est-ce pas lui dont Ezéchias a ôté les hauts lieux et les autels, en disant à Juda et à Jérusalem : Vous vous prosternerez devant cet autel-ci ?
\VS{8}Maintenant donc, donne des otages au roi d'Assyrie, mon maître ; et je te donnerai deux mille chevaux, si tu peux donner autant d'hommes pour monter dessus.
\VS{9}Et comment ferais-tu tourner le visage à un seul gouverneur d'entre les moindres serviteurs de mon maître ? Mais tu te confies en l'Egypte pour les chars et pour les cavaliers.
\VS{10}Mais suis-je monté sans Yahweh dans ce pays pour le détruire ? Yahweh m'a dit : Monte contre ce pays et détruis-le.
\VS{11}Alors Eliakim, Schebna et Joach dirent à Rabschaké : Nous te prions de parler en langue araméenne à tes serviteurs, car nous la comprenons ; mais ne parle pas en langue judaïque, pendant que le peuple qui est sur la muraille l'écoute.
\VS{12}Et Rabschaké répondit : Mon maître m'a-t-il envoyé vers ton maître ou vers toi, pour dire ces paroles là ? Ne m'a-t-il pas envoyé vers les hommes qui se tiennent sur la muraille, pour leur dire qu'ils mangeront leur propre fiente, et qu'ils boiront leur urine avec vous ?
\VS{13}Puis Rabschaké se dressa et s'écria à haute voix en langue judaïque, et dit : Ecoutez les paroles du grand roi, du roi d'Assyrie !
\VS{14}Ainsi parle le roi : Qu'Ezéchias ne vous séduise pas, car il ne pourra pas vous délivrer.
\VS{15}Qu'Ezéchias ne vous fasse pas confier en Yahweh, en disant : Yahweh nous délivrera certainement ; cette ville ne sera point livrée entre les mains du roi d'Assyrie.
\VS{16}N'écoutez point Ezéchias ; car ainsi parle le roi d'Assyrie : Faites un accord avec moi pour votre bien, et sortez vers moi, et vous mangerez chacun  de sa vigne, et chacun de son figuier, et vous boirez chacun de l'eau de sa citerne,
\VS{17}jusqu'à ce que je vienne, et que je vous emmène dans un pays qui est comme votre pays, un pays de blé et de bon vin, un pays de pain et de vignes.
\VS{18}Qu'Ezéchias donc ne vous séduise point, en disant : Yahweh nous délivrera. Les dieux des nations ont-ils délivré chacun leur pays de la main du roi d'Assyrie ?
\VS{19}Où sont les dieux de Hamath et d'Arpad ? Où sont les dieux de Sepharvaïm ? Ont-ils délivré Samarie de ma main ?
\VS{20}Qui sont ceux d'entre tous les dieux de ces pays qui aient délivré leur pays de ma main, pour que Yahweh délivre Jérusalem de ma main ?
\VS{21}Mais ils se turent et ne lui répondirent pas un mot ; car le roi avait donné cet ordre, disant : Vous ne lui répondrez pas.
\TextTitle{Ezéchias informé des menaces}
\VS{22}Après cela, Eliakim fils de Hilkija, chef de la maison du roi, Schebna, le secrétaire, et Joach, fils d'Asaph l'archiviste, s'en revinrent auprès d'Ezéchias, les vêtements déchirés, et lui rapportèrent les paroles de Rabschaké.
\Chap{37}
\TextTitle{Ezéchias recherche Yahweh auprès d'Esaïe\FTNTT{2 R. 19:1-7 ; 2 Ch. 32:20.}}
\VerseOne{}Et il arriva qu'aussitôt que le roi Ezéchias eut entendu ces choses, il déchira ses vêtements, se couvrit d'un sac, et entra dans la maison de Yahweh\FTNT{2 R. 19:1-7 ; 2 Ch. 32:20.}.
\VS{2}Puis il envoya Eliakim, chef de la maison du roi, et Schebna, le secrétaire, et les plus anciens des sacrificateurs couverts de sacs, vers Esaïe, le prophète, fils d'Amots.
\VS{3}Et ils lui dirent : Ainsi parle Ezéchias : Ce jour est un jour d'angoisse, de répréhension et de blasphème ; car les enfants sont près de sortir du sein maternel, mais il n'y a point de force pour enfanter.
\VS{4}Peut-être que Yahweh, ton Dieu, a-t-il entendu les paroles de Rabschaké, que le roi d'Assyrie, son maître, a envoyé pour blasphémer le Dieu vivant et lui faire outrage ; selon les paroles que Yahweh, ton Dieu, a entendues ; fais donc requête pour le reste qui subsiste encore.
\VS{5}Les serviteurs du roi Ezéchias vinrent vers Esaïe.
\VS{6}Et Esaïe leur dit : Voici ce que vous direz à votre maître : Ainsi parle Yahweh : Ne crains point pour les paroles que tu as entendues, par lesquelles les serviteurs du roi d'Assyrie m'ont blasphémé.
\VS{7}Voici, je vais mettre en lui un esprit tel qu'ayant entendu une certaine rumeur, il retournera dans son pays, et je le ferai tomber par l'épée dans son pays.
\TextTitle{Provocation et menace de Sanchérib\FTNTT{2 R. 19:8-13 ; 2 Ch. 32:17-19.}}
\VS{8}Or quand Rabschaké s'en fut retourné, il alla trouver le roi d'Assyrie qui attaquait Libna, car il avait appris qu'il était parti de Lakis.
\VS{9}Alors le roi d'Assyrie ayant entendu dire au sujet de Tirhaka, roi d'Ethiopie : Il est sorti pour te faire la guerre. Dès qu'il eut entendu cela, il envoya des messagers à Ezéchias, en leur disant :
\VS{10}Vous parlerez ainsi à Ezéchias, roi de Juda : Que ton Dieu, auquel tu te confies, ne te séduise point, en disant : Jérusalem ne sera point livrée entre les mains du roi d'Assyrie.
\VS{11}Voilà, tu as entendu ce que les rois d'Assyrie ont fait à tous les pays, en les détruisant entièrement ; et toi, tu échapperais ?
\VS{12}Les dieux des nations que mes ancêtres ont détruites, à savoir Gozan, Charan, Retseph, et les fils d'Eden, qui sont à Telassar, les ont-ils délivrées ?
\VS{13}Où sont le roi de Hamath, le roi d'Arpad, et le roi de la ville de Sepharvaïm, d'Héna et d'Ivva ?
\TextTitle{Prière d'Ezéchias à Yahweh\FTNTT{2 R. 19:14-19 ; 2 Ch. 32:20.}}
\VS{14}Et quand Ezéchias reçut les lettres de la main des messagers et les lut, il monta à la maison de Yahweh, et Ezéchias les déploya devant Yahweh.
\VS{15}Puis Ezéchias fit sa prière à Yahweh, en disant :
\VS{16}Ô Yahweh des armées ! Dieu d'Israël qui es assis entre les chérubins ! C'est toi qui es le seul Dieu de tous les royaumes de la terre, c'est toi qui as fait les cieux et la terre.
\VS{17}Ô Yahweh ! Incline ton oreille et écoute ! Ô Yahweh ! Ouvre tes yeux et regarde ! Ecoute les paroles de Sanchérib, qu'il m'a envoyé dire pour blasphémer le Dieu vivant.
\VS{18}Il est bien vrai, ô Yahweh, que les rois d'Assyrie ont détruit tous les pays et leurs contrées ;
\VS{19}et qu'ils ont jeté dans le feu leurs dieux ; mais ce n'étaient point des dieux, mais un ouvrage de mains d'homme, du bois et de la pierre ; c'est pourquoi ils les ont détruits.
\VS{20}Maintenant donc, ô Yahweh notre Dieu ! Délivre-nous de la main de Sanchérib, afin que tous les royaumes de la terre sachent que toi seul es Yahweh.
\TextTitle{Esaïe transmet la réponse de Yahweh\FTNTT{2 R. 19:20-34.}}
\VS{21}Alors Esaïe, fils d'Amots, envoya dire à Ezéchias : Ainsi parle Yahweh, le Dieu d'Israël : J'ai entendu la prière que tu m'as faite au sujet de Sanchérib, roi d'Assyrie.
\VS{22}C'est ici la parole que Yahweh a prononcée contre lui : La vierge, fille de Sion, te méprise et se moque de toi ; la fille de Jérusalem hoche la tête après toi.
\VS{23}Contre qui as-tu élevé ta voix, et levé tes yeux en haut ? C'est contre le Saint d'Israël.
\VS{24}Tu as outragé le Seigneur par le moyen de tes serviteurs, et tu as dit : Je suis monté avec la multitude de mes chars sur le haut des montagnes, aux côtés du Liban, je couperai les plus hauts cèdres, et les plus beaux cyprès qui y soient, et j'entrerai jusqu'en son plus haut bout, et en la forêt de son Carmel.
\VS{25}J'ai creusé des sources, et j'en ai bu les eaux, et je tarirai avec la plante de mes pieds tous les fleuves de l'Egypte.
\VS{26}N'as-tu pas appris, qu'il y a déjà longtemps, j'ai fait cette ville, et que dès les temps anciens je l'ai ainsi formée ? Et maintenant l'aurais-je conservée pour être réduite en désolation, et les villes fortes en monceaux de ruines ?
\VS{27}Or leurs habitants, étant dénués de force, ont été épouvantés et confus ; ils sont devenus comme l'herbe des champs ; et l'herbe verte, comme le foin des toits, et le blé brûlé avant la formation de sa tige.
\VS{28}Mais je sais quand tu t'assieds, quand tu sors et quand tu entres, et comment tu es furieux contre moi\FTNT{Ps. 139:2.}.
\VS{29}Parce que tu es furieux contre moi, et que ton insolence est montée à mes oreilles, je mettrai ma boucle à tes narines, et mon mors en ta bouche, et je te ferai retourner par le chemin par lequel tu es venu.
\VS{30}Et ceci te sera pour signe, ô Ezéchias, c'est qu'on mangera cette année ce qui viendra de soi-même aux champs ; et en la deuxième année ce qui croîtra encore sans semer ; mais la troisième année, vous sèmerez, vous moissonnerez, vous planterez des vignes, et vous en mangerez le fruit.
\VS{31}Et ce qui est réchappé, et demeuré de reste dans la maison de Juda, étendra sa racine par-dessous, et elle produira du fruit par-dessus.
\VS{32}Car il sortira de Jérusalem un reste, et de la montagne de Sion quelques réchappés, la jalousie de Yahweh des armées fera cela.
\VS{33}C'est pourquoi ainsi parle Yahweh sur le roi d'Assyrie : Il n'entrera point dans cette ville, il n'y jettera aucune flèche, il ne se présentera point contre elle avec le bouclier, et il ne dressera point de retranchements contre elle.
\VS{34}Il s'en retournera par le chemin par lequel il est venu, et il n'entrera point dans cette ville, dit Yahweh.
\VS{35}Car je protégerai cette ville pour la délivrer pour l'amour de moi, et pour l'amour de David, mon serviteur.
\TextTitle{Yahweh frappe Sanchérib\FTNTT{2 R. 19:35-37 ; 2 Ch. 32:21.}}
\VS{36}L'ange de Yahweh\FTNT{Ge. 16:7.} sortit et frappa cent quatre-vingt-cinq mille hommes dans le camp des Assyriens. Et quand on se leva le matin, voici, ils étaient tous morts.
\VS{37}Alors Sanchérib, roi d'Assyrie, partit de là ; il s'en alla et s'en retourna, et il se tint à Ninive.
\VS{38}Et il arriva qu'étant prosterné dans la maison de Nisroc\FTNT{Le nom Nisroc signifie « le grand aigle ». C'était une idole de Ninive adorée par Sanchérib, symbolisée par un aigle à figure humaine.}, son dieu, Adrammélec et Scharetser, ses fils, le tuèrent avec l'épée ; puis ils s'enfuirent au pays d'Ararat. Et Esar-Haddon, son fils, régna à sa place.
\Chap{38}
\TextTitle{Maladie et guérison d'Ezéchias\FTNTT{2 R. 20:1-11 ; 2 Ch. 32:24-30.}}
\VerseOne{}En ces jours-là, Ezéchias fut malade à la mort\FTNT{2 R. 20:1-11 ; 2 Ch. 32:24-30.}. Et Esaïe le prophète, fils d'Amots, vint auprès de lui, et lui dit : Ainsi parle Yahweh : Donne tes ordres à ta maison, car tu vas mourir et tu ne vivras plus.
\VS{2}Alors Ezéchias tourna sa face contre la muraille et fit sa prière à Yahweh,
\VS{3}et dit : Ô Yahweh, souviens-toi maintenant je te prie que j'ai marché devant toi en vérité et en intégrité de cœur, et que j'ai fait ce qui est agréable à tes yeux ! Et Ezéchias pleura abondamment.
\VS{4}Puis la parole de Yahweh fut adressée à Esaïe, en disant :
\VS{5}Va, et dis à Ezéchias ainsi parle Yahweh, le Dieu de David, ton père : J'ai exaucé ta prière, j'ai vu tes larmes. Voici, j'ajouterai à tes jours quinze années.
\VS{6}Et je te délivrerai de la main du roi d'Assyrie, toi et cette ville, et je défendrai cette ville.
\VS{7}Et ce signe t'est donné par Yahweh, pour voir que Yahweh accomplira la parole qu'il a prononcée.
\VS{8}Voici, je ferai retourner de dix degrés en arrière avec le soleil l'ombre des degrés qui est descendue sur les degrés d'Achaz. Et le soleil retourna de dix degrés par les degrés par lesquels il était descendu.
\VS{9}Or c'est ici l'écrit d'Ezéchias, roi de Juda, sur sa maladie et sur son rétablissement.
\VS{10}J'avais dit dans le retranchement de mes jours : Je m'en irai aux portes du scheol, je suis privé de ce qui restait de mes années.
\VS{11}Je disais : Je ne contemplerai plus Yahweh, Yahweh sur la terre des vivants ; je ne verrai plus aucun homme parmi les habitants du monde !
\VS{12}Ma durée s'en est allée, et a été transportée loin de moi, comme une cabane de berger ; ma vie est coupée je suis retranché comme la toile que le tisserand détache de sa trame. Du matin au soir tu m'auras enlevé\FTNT{Aux versets 12 et 13, le mot qui a été traduit par « enlevé » est « shalam » : « être dans une alliance de paix, être en paix ».} !
\VS{13}Je pensais en moi-même jusqu'au matin ; comme un lion, qui briserait ainsi tous mes os ; du matin au soir tu m'auras enlevé !
\VS{14}Je grommelais comme la grue et l'hirondelle ; je gémissais comme la colombe ; mes yeux défaillaient à force de regarder en haut : Ô Yahweh, je suis opprimé, sois mon garant !
\VS{15}Que dirai-je ? Il m'a parlé et lui-même l'a fait. Je m'en irai tout doucement tous les ans de ma vie, dans l'amertume de mon âme.
\VS{16}Seigneur, par ces choses-là on a la vie, et dans toutes ces choses est la vie de mon esprit. Ainsi tu me rétabliras et me feras revivre.
\VS{17}Voici, dans ma paix, une grande amertume m'est survenue, mais tu as embrassé mon âme afin qu'elle ne tombe pas dans la fosse de la pourriture, car tu as jeté tous mes péchés derrière ton dos.
\VS{18}Car le scheol ne te loue point,  la mort ne te célèbre point ; ceux qui sont descendus dans la fosse ne s'attendent plus à ta vérité\FTNT{Ps. 115:17.}.
\VS{19}Mais le vivant, le vivant est celui qui te célèbre, comme moi aujourd'hui ; le père conduira ses enfants à la connaissance de ta vérité\FTNT{Pr. 22:6 ; Ep. 6:4.}.
\VS{20}Yahweh est venu me délivrer, et à cause de cela, nous jouerons sur les instruments mes cantiques, tous les jours de notre vie dans la maison de Yahweh.
\VS{21}Or Esaïe avait dit : Qu'on prenne une masse de figues sèches et qu'on en fasse un emplâtre sur l'ulcère ; et Ezéchias guérira.
\VS{22}Et Ezéchias avait dit : Quel est le signe que je monterai à la maison de Yahweh ?
\Chap{39}
\TextTitle{Ezéchias montre toutes ses richesses aux Babyloniens\FTNTT{2 R. 20:12-19}}
\VerseOne{}En ce temps-là\FTNT{2 R. 20:12-19.}, Mérodac-Baladan, fils de Baladan, roi de Babylone, envoya des lettres avec un présent à Ezéchias, parce qu'il avait entendu qu'il avait été malade, et qu'il était guéri.
\VS{2}Et Ezéchias en eut de la joie, et il leur montra les cabinets où étaient ses choses précieuses, l'argent, l'or, et les aromates, et l'huile précieuse, tout son arsenal, et tout ce qui se trouvait dans ses trésors ; il n'y eut rien qu'Ezéchias ne leur montra dans sa maison et dans tous ses domaines.
\VS{3}Puis le prophète Esaïe vint vers le roi Ezéchias, et lui dit : Qu'ont dit ces hommes-là, et d'où sont-ils venus vers toi ? Et Ezéchias répondit : Ils sont venus vers moi d'un pays éloigné, de Babylone.
\VS{4}Puis Esaïe dit : Qu'ont-ils vu dans ta maison ? Ezéchias répondit : Ils ont vu tout ce qui est dans ma maison ; il n'y a rien dans mes trésors que je ne leur aie montré.
\VS{5}Et Esaïe dit à Ezéchias : Ecoute la parole de Yahweh des armées :
\VS{6}Voici, les jours viennent où l'on emportera à Babylone tout ce qui est dans ta maison, et ce que tes pères ont amassé dans leurs trésors jusqu'à aujourd'hui ; il n'en restera rien, dit Yahweh\FTNT{2 R. 24:13 ; 2 R. 25:13-15 ; Jé. 20:5.}.
\VS{7}Même on prendra de tes fils qui sortiront de toi, et que tu auras engendrés afin qu'ils soient eunuques dans le palais du roi de Babylone\FTNT{Da. 1:3-4.}.
\VS{8}Et Ezéchias répondit à Esaïe : La parole de Yahweh, que tu as prononcée, est bonne ; et, il ajouta, au moins qu'il y ait paix et sécurité pendant mes jours.
\Chap{40}
\TextTitle{Un nouveau message pour Esaïe}
\VerseOne{}Consolez, consolez mon peuple, dit votre Dieu.
\VS{2}Parlez à Jérusalem selon son cœur, et criez-lui que son temps marqué est accompli, que son iniquité est tenue pour acquittée, qu'elle a reçu de la main de Yahweh le double pour tous ses péchés.
\TextTitle{Mission de Jean-Baptiste\FTNTT{Mt. 3:3.}}
\VS{3}La voix de celui qui crie au désert\FTNT{L'accomplissement de cette prophétie se trouve en Mt 3:3, où il nous est dit que la voix qui devait crier ces choses était celle de Jean-Baptiste (voir aussi Mal. 3:1 ; Mal. 4:5-6 ; Mt. 17:10-13).} est : Préparez le chemin de Yahweh\FTNT{Les évangiles nous enseignent que Jean-Baptiste a été envoyé pour préparer le chemin du Seigneur Jésus (Jn. 1:19-27 ; Jn. 1:29-34 ; Jn. 3:28-31).}, aplanissez parmi les lieux arides un chemin pour notre Dieu.
\VS{4}Toute vallée sera comblée, toute montagne et toute colline seront abaissées, et les lieux tortueux seront redressés, et les lieux raboteux seront aplanis.
\VS{5}Alors la gloire de Yahweh sera manifestée, et toute chair en même temps la verra, car la bouche de Yahweh a parlé.
\TextTitle{La grandeur de Dieu échappe à l'homme}
\VS{6}La voix dit : Crie ! Et on a répondu : Que crierai-je ? Toute chair est comme l'herbe, et toute sa grâce est comme la fleur d'un champ\FTNT{Ja. 1:10 ; 1 Pi. 1:24-25.}.
\VS{7}L'herbe sèche, et la fleur tombe, parce que le vent de Yahweh souffle dessus. Certainement le peuple est comme l'herbe.
\VS{8}L'herbe sèche, et la fleur tombe, mais la parole de notre Dieu demeure éternellement.
\VS{9}Sion, qui annonce de bonnes nouvelles, monte sur une haute montagne ; Jérusalem, qui annonce de bonnes nouvelles, élève ta voix avec force ; élève-la, ne crains point ; dis aux villes de Juda : Voici votre Dieu !
\VS{10}Voici, le Seigneur Yahweh\FTNT{Jésus-Christ est Yahweh qui vient (Es. 35:4 ; Es. 40:10-11 ; Es. 60:1 ; Es. 62:11-12 ; Es. 66:15-16 ; Za. 14:1-7 ; Mt. 24 ; Jn. 14:1-3; Ac. 1:10-12 ; Ap. 3:11 ; Ap. 19:11-12 ; Ap. 22:7 ; Ap. 22:12 ; Ap. 22:20).} viendra contre le fort, et son bras dominera sur lui ; voici son salaire est avec lui, et ses rétributions sont devant lui.
\VS{11}Il paîtra son troupeau comme un berger, il rassemblera les agneaux dans ses bras, il les placera dans son sein ; il conduira celles qui allaitent\FTNT{Jn. 10.}.
\VS{12}Qui est celui qui a mesuré les eaux avec le creux de sa main, et qui a pris les dimensions des cieux avec la paume, qui a rassemblé toute la poussière de la terre dans un boisseau, et qui a pesé au crochet les montagnes et les collines à la balance ?
\VS{13}Qui a dirigé l'Esprit de Yahweh, ou qui a été son conseiller pour l'enseigner\FTNT{1 Co. 2:16 ; Ro. 11:34.} ?
\VS{14}Avec qui a-t-il pris conseil, et qui l'a instruit, et lui a enseigné le sentier de jugement ? Qui lui a enseigné la science, et lui a montré le chemin de l'intelligence ?
\VS{15}Voilà, les nations sont comme une goutte qui tombe d'un seau, et elles sont réputée comme la menue poussière d'une balance ; voila, il a jeté çà et là les îles comme de la poudre.
\VS{16}Et le Liban ne suffirait pas pour faire le feu, et les bêtes qui y sont ne seraient pas suffisantes pour l'holocauste.
\VS{17}Toutes les nations sont devant lui comme un rien, et il ne les considère que comme de la poussière, et comme un néant.
\VS{18}A qui donc ferez-vous ressembler Dieu ? Et à quelle ressemblance l'égalerez-vous ?
\VS{19}L'ouvrier fond l'image, et l'orfèvre la couvre d'or, et y soude des chaînettes d'argent.
\VS{20}Celui qui est si pauvre qu'il n'a pas de quoi faire une offrande, choisit un bois qui ne pourrisse point ; il se cherche un habile ouvrier pour faire une image taillée qui ne bouge pas\FTNT{Es. 44:9-20.}.
\VS{21}Ne le savez-vous pas ? Ne l'avez-vous pas entendu ? Cela ne vous a-t-il pas été déclaré dès le commencement ? Ne l'avez-vous pas entendu dès les fondements de la terre ?
\VS{22}C'est lui qui est assis au-dessus du globe de la terre, et à qui ses habitants sont comme des sauterelles ; c'est lui qui étend les cieux comme un voile, il les déploie même comme une tente pour y demeurer.
\VS{23}C'est lui qui réduit les princes à rien, et qui fait des chefs de la terre une chose de néant.
\VS{24}Ils ne sont pas même plantés, pas même semés, même leur tronc n'a point de racine en terre ; il souffle sur eux, et ils sèchent, et le tourbillon les emporte comme de la paille.
\VS{25}A qui donc me ferez-vous ressembler, et à qui serais-je égalé ? Dit le Saint.
\VS{26}Elevez vos yeux en haut et regardez ! Qui a créé ces choses ? C'est lui qui fait sortir leur armée par ordre, et qui les appelle toutes par leur nom ; il n'y en a pas une qui fait défaut, à cause de la grandeur de sa force, et parce qu'il excelle en puissance.
\VS{27}Pourquoi donc dis-tu, ô Jacob, pourquoi dis-tu, ô Israël : Ma voie est cachée à Yahweh, et mon jugement passe inaperçu devant mon Dieu ?
\VS{28}Ne sais-tu pas ? N'as-tu pas entendu que le Dieu d'éternité, Yahweh, a créé les extrémités de la terre ; il ne se fatigue point, il ne se lasse point, et il n'y a pas moyen de sonder son intelligence.
\VS{29}C'est lui qui donne de la force à celui qui est las, et il multiplie la force de celui qui n'a aucune vigueur.
\VS{30}Les jeunes gens se lassent et se fatiguent, même les jeunes hommes tombent sans force.
\VS{31}Mais ceux qui s'attendent à Yahweh renouvellent leur force. Ils s'élèvent avec des ailes, comme des aigles ; ils courent, et ne se fatiguent point ; ils marchent, et ne se lassent point.
\Chap{41}
\TextTitle{Dénonciation des idoles}
\VerseOne{}Iles, faites moi silence ! Que les peuples renouvellent leurs forces ; qu'ils s'approchent et qu'alors ils parlent ; allons ensemble en jugement.
\VS{2}Qui a fait levé l'homme droit de l'orient ? Qui l'a appelé à sa suite ? Qui a soumis à son commandement les nations ? Qui lui a donné la domination sur les rois ? Qui les a livrés à son épée comme de la poussière, et à son arc comme de la paille poussée par le vent ?
\VS{3}Il les a poursuivis, il est passé en paix par le chemin que son pied n'avait jamais foulé.
\VS{4}Qui est celui qui a opéré et fait ces choses ? C'est celui qui a appelé les âges dès le commencement. Moi, Yahweh, JE SUIS le premier, et JE SUIS avec les derniers\FTNT{Ap. 1:8 ; Ap. 21:6 ; Ap. 22:13.}.
\VS{5}Les îles voient, et sont dans la crainte, les extrémités de la terre sont effrayées, ils s'approchent, ils viennent.
\VS{6}Chacun aide son prochain, et chacun dit à son frère : Fortifie-toi.
\VS{7}L'ouvrier encourage le fondeur ; celui qui frappe doucement du marteau encourage celui qui frappe sur l'enclume, et il dit : Cela est bon pour souder, puis il fixe l'idole avec des clous, afin qu'elle ne bouge pas.
\VS{8}Mais toi, Israël, tu es mon serviteur, et toi, Jacob, tu es celui que j'ai élu, la race d'Abraham qui m'a aimé !
\VS{9}Car je t'ai pris aux extrémités de la terre, je t'ai appelé en te préférant aux plus excellents qui sont en elle, et je t'ai dit : C'est toi qui est mon serviteur, je t'ai élu, et je ne te rejette point\FTNT{De. 7:6 ; Ps. 77:8.}.
\VS{10}Ne crains rien, car je suis avec toi ; ne sois pas étonné, car je suis ton Dieu ; je te fortifie, et je t'aide, même je te soutiens par la droite de ma justice.
\VS{11}Voici, tous ceux qui sont indignés contre toi seront honteux et confus ; ils seront réduits à néant, et les hommes qui ont querelle avec toi périront.
\VS{12}Tu les chercheras, et tu ne les trouveras plus, ceux qui te suscitaient querelle ; ils seront réduits à néant, et ceux qui te font la guerre seront comme ce qui n'est plus.
\VS{13}Car je suis Yahweh, ton Dieu, qui soutient ta main droite, et te dis : Ne crains rien, c'est moi qui te secours.
\VS{14}Ne crains point, vermisseau de Jacob, hommes mortels d'Israël ; je viens à ton secours dit Yahweh, et ton défenseur, le Saint d'Israël.
\VS{15}Voici, je fais de toi un traîneau aigu, tout neuf, ayant des dents ; tu fouleras les montagnes et les menuiseras, et tu rendras les collines semblables à de la balle.
\VS{16}Tu les vanneras, et le vent les emportera, et le tourbillon les dispersera. Mais toi, tu te réjouiras en Yahweh, tu te glorifiera au Saint d'Israël.
\VS{17}Quant aux affligés et aux misérables qui cherchent des eaux, et n'en ont point ; dont la langue est tellement altérée qu'elle n'en peut plus ; moi, Yahweh, je les exaucerai ; moi, le Dieu d'Israël, je ne les abandonnerai pas\FTNT{Ge. 28:15 ; Jos. 1:5 ; Hé. 13:5.}.
\VS{18}Je ferai jaillir des fleuves sur les hauteurs et des fontaines au milieu des vallées ; et je ferai du désert des étang d'eaux et de la terre sèche des sources d'eaux.
\VS{19}Je ferai croître au désert le cèdre, l'acacia, le myrte et l'olivier ; je mettrai dans les lieux stériles le cyprès, l'orme et le buis ensemble,
\VS{20}afin qu'on voit, qu'on sache, qu'on pense, et qu'on comprenne que la main de Yahweh a fait cela, et que le Saint d'Israël a créé cela.
\VS{21}Plaidez votre cause, dit Yahweh ; et mettez en avant les fondements de votre cause, dit le Roi de Jacob.
\VS{22}Qu'ils les amènent et qu'ils nous déclarent ce qui doit arriver. Déclarez-nous que veulent dire les choses qui ont été auparavant et nous y prendrons garde, et nous saurons leur issue, ou faites-nous entendre ce qui est prêt à arriver. 
\VS{23}Déclarez les choses qui doivent arriver dorénavant, et nous saurons que vous êtes des dieux ; faites aussi du bien ou du mal, et nous en serons tout étonnés puis nous regarderons ensemble.
\VS{24}Voici, vous n'êtes rien, et votre œuvre est le néant ; celui qui vous choisit n'est qu'abomination.
\VS{25}Je l'ai suscité du nord, et il est venu ; il invoque mon Nom de devant le soleil levant ; et marche sur les princes comme sur le mortier, et les foule comme le potier foule la boue.
\VS{26}Qui est celui qui a manifesté ces choses dès le commencement, afin que nous le connaissions ? Et longtemps d'avance, que nous puissions dire : Il est juste. Mais il n'y a personne qui les annonce, même il n'y a personne qui les donne à entendre, même il n'y a personne qui entende vos paroles.
\VS{27}Le premier sera pour Sion, disant : Voici, les voici ! Et je donnerai quelqu'un à Jérusalem qui annoncera de bonnes nouvelles\FTNT{Es. 52:7 ; Ap. 14:6.}.
\VS{28}Je regarde, et il n'y a point d'homme même entre ceux-là, et il n'y a aucun homme de conseil ; je les interroge aussi afin qu'il réponde quelque chose. 
\VS{29}Voici, quant à eux tous, leurs œuvres ne sont que vanité, leurs idoles de fonte sont du vent et de la confusion.
\Chap{42}
\TextTitle{Le messie, serviteur de Yahweh}
\VerseOne{}Voici mon serviteur, que je soutiens, c'est mon élu, en qui mon âme prend son bon plaisir ; j'ai mis mon Esprit sur lui, il manifestera le jugement aux nations\FTNT{Mt. 3:17 ; Mt. 17:5 ; Mc. 9:7.}.
\VS{2}Il ne criera point, et il ne haussera, ni ne fera entendre sa voix dans les rues.
\VS{3}Il ne brisera point le roseau cassé, et il n'éteindra point le lumignon qui fume\FTNT{Mt. 12:18-20.} ; il mettra en avant le jugement en vérité.
\VS{4}Il ne se retirera point et ne s'affaiblira point, jusqu'à ce qu'il ait établi la justice sur la terre, et que les îles s'attendent à sa loi.
\VS{5}Ainsi parle Dieu, Yahweh, qui a créé les cieux, et qui les a étendus, qui a aplani la terre avec ce qu'elle produit, qui donne la respiration au peuple qui est sur elle, et l'esprit à ceux qui y marchent.
\VS{6}Moi Yahweh, je t'ai appelé en justice, et je prendrai ta main et te garderai, et je te ferai être l'alliance du peuple et la lumière des nations\FTNT{Voir commentaire en Ge. 1:3-5.},
\VS{7}afin d'ouvrir les yeux des aveugles, et de faire sortir les prisonniers hors du lieu où on les tient enfermés, et ceux qui habitent dans les ténèbres hors de la prison.
\TextTitle{Israël n'a pas été attentif à Yahweh}
\VS{8}Je suis Yahweh, c'est là mon Nom ; et je ne donnerai pas ma gloire à un autre, ni ma louange aux images taillées\FTNT{Es. 48:11.}.
\VS{9}Voici, les choses qui ont été prédites auparavant se sont accomplies. Et je vous en annonce de nouvelles ; et je vous les fait entendre avant qu'elles arrivent.
\VS{10}Chantez à Yahweh un cantique nouveau, et que sa louange éclate aux extrémités de la terre, vous qui descendez en la mer, et tout ce qui est en elle,  les îles et leurs habitants !
\VS{11}Que le désert et ses villes élèvent la voix ! Que les villages où habite Kédar et ceux qui habitent dans les rochers éclatent en chant de triomphe ! Qu'ils s'écrient du sommet des montagnes !
\VS{12}Qu'on donne gloire à Yahweh, et qu'on publie sa louange dans les îles !
\VS{13}Yahweh sort comme un homme vaillant, il réveille sa jalousie comme un homme de guerre, il jette, dis-je, des cris de joie, il jette de grands cris, et il prévaut sur ses ennemis.
\VS{14}Je me suis tu dès longtemps ; me tiendrais-je en repos ? Me retiendrais-je ? Je crierai comme celle qui enfante, je détruirai, et j'engloutirai tout à la fois.
\VS{15}Je réduirai les montagnes et les collines en désert, et j'en dessécherai toute la verdure, je réduirai les fleuves en îles, et je ferai tarir les étangs.
\VS{16}Je conduirai les aveugles sur un chemin qu'ils ne connaissent pas, je les ferai marcher par des sentiers qu'ils ne connaissent pas ; je réduirai devant eux les ténèbres en lumière, et les choses tortues en choses droites ; voilà ce que je ferai, et je ne les abandonnerai point.
\VS{17}Ils se retireront en arrière, et ils seront tout honteux, ceux qui se confient aux images taillées, et qui disent aux images de fonte : Vous êtes nos dieux !
\VS{18}Sourds, écoutez ! Et vous aveugles, regardez et voyez !
\VS{19}Qui, dis-je, est aveugle, sinon mon serviteur ? Et qui est sourd, comme mon messager que j'envoie ? Qui est aveugle, comme celui que j'ai comblé de grâces ? Qui est aveugle, comme le serviteur de Yahweh ?
\VS{20}Vous voyez beaucoup de choses, mais vous ne prenez garde à rien ; vous avez les oreilles ouvertes, mais vous n'entendez rien.
\VS{21}Yahweh a plaisir en lui à cause de sa justice ; il a magnifié la loi et l'a  rendu honorable. 
\VS{22}Mais c'est ici un peuple pillé et dépouillé ! Ils sont enlacés dans les cavernes, et sont cachés dans des prisons ; ils sont un butin, et il n'y a personne qui les délivre ; une proie, et il n'y a personne qui dise : Restituez !
\VS{23}Qui est celui d'entre vous qui prêtera l'oreille à ces choses ? Qui s'y rendra attentif et l'écoutera à l'avenir ?
\VS{24}Qui est-ce qui a livré Jacob au pillage, et Israël aux pillards\FTNT{Jg. 2:13-16.} ? N'est-ce pas Yahweh, contre lequel nous avons péché ? Car on n'a point voulu marcher dans ses voies et on n'a point obéi à sa loi.
\VS{25}C'est pourquoi il a répandu sur lui la fureur de sa colère, et une forte guerre ; et il l'a embrasé tout alentour, mais Israël ne l'a point connu ; et il l'a brûlé, mais il n'y a point pris garde.
\Chap{43}
\TextTitle{Yahweh veut racheter Israël}
\VerseOne{}Mais maintenant ainsi parle Yahweh, qui t'a créé, ô Jacob ! Celui qui t'a formé, ô Israël ! Ne crains point, car je te rachète, je t'appelle par ton nom, tu es à moi !
\VS{2}Si tu passes par les eaux, je serai avec toi ; et si tu passes par les fleuves, ils ne te noieront pas ; si tu marches dans le feu, tu ne seras pas brûlé, et la flamme ne t'embrasera pas.
\VS{3}Car je suis Yahweh, ton Dieu, le Saint d'Israël, ton Sauveur. Je donne l'Egypte pour ta rançon, l'Ethiopie et Saba à ta place.
\VS{4}Parce que tu es précieux à mes yeux, tu es rendu honorable et je t'aime, je donne des hommes à ta place, et des peuples pour ta vie.
\VS{5}Ne crains point, car je suis avec toi ; je ferai venir ta postérité de l'orient, et je t'assemblerai de l'occident.
\VS{6}Je dirai au nord : Donne ! Et au midi : Ne retiens point ! Fais venir mes fils de loin, et mes filles du bout de la terre,
\VS{7}savoir tous ceux qui s'appellent de mon Nom\FTNT{Dans les Ecritures, le Nom de Dieu le plus cité est YHWH. Jésus, dont le nom signifie « YHWH est salut » correspond au nom et à l'identité que Dieu a révélé à tous ceux qui l'ont rencontré quand il était sur cette terre. Dans sa dernière prière à Gethsémané, Jésus dit : « J'ai fait connaître ton Nom » (Jn. 17:6), et « Je leur ai fait connaître ton Nom » (Jn. 17:26). Ce nom n'est autre que le sien puisque Jésus (YHWH est salut) était et est le Nom de Dieu. Moïse n'avait pas reçu la révélation de ce Nom (Ex. 3:13-14) car cette révélation était réservée à l'Eglise. En tant qu'épouse de Christ, l'Eglise porte le Nom du Seigneur et bénéficie de l'autorité qu'il confère. Ainsi, Jésus est le seul Nom par lequel nous pouvons être sauvés (Ac. 4:12). C'est aussi en son Nom que nous devons être baptisés (Ac. 8:16 ; Ac. 19:5), que nous recevons l'exaucement de nos prières (Jn. 14:13-14 ; Jn. 16:24), que nous sommes délivrés de l'ennemi et que nous obtenons la victoire sur le camp de l'ennemi (Mc. 16:17 ; Ph. 2:9-11).} ; car je les ai créés pour ma gloire ; je les ai formés et les ai faits.
\TextTitle{Yahweh appelle ses témoins}
\VS{8}Amène dehors le peuple aveugle qui a des yeux, et les sourds qui ont des oreilles.
\VS{9}Que toutes les nations soient ramassées ensemble, et que les peuples soient assemblés. Lequel d'entre eux a annoncé ces choses-là ? Et qui sont ceux qui nous ont fait entendre les choses qui ont été ci-devant ? Qu'ils produisent leurs témoins et qu'ils se justifient ; qu'on les entende et qu'on dise : C'est vrai !
\VS{10}Vous êtes mes témoins\FTNT{Ac. 1:8.}, dit Yahweh, et mon serviteur que j'ai élu, afin que vous connaissiez, que vous me croyiez et que vous compreniez que JE SUIS. Avant moi il n'a pas été formé de Dieu, et il n'y en aura point après moi.
\VS{11}Moi, JE SUIS Yahweh, et à part moi il n'y a point de Sauveur\FTNT{Yahweh dit qu'à part lui, il n'y a pas d'autres sauveurs. Or les écrits de la nouvelle alliance affirment que Jésus-Christ est le seul Sauveur (Lu. 1:67-80 ; Ac. 4:11-12).}.
\VS{12}C'est moi qui ai prédit ce qui devait arriver, qui vous ai sauvés, et qui vous ai fait entendre l'avenir, quand il n'y avait point de dieu étranger parmi vous ; et vous êtes mes témoins, dit Yahweh, que je suis Dieu.
\VS{13}Et même avant que le jour fût, JE SUIS, et il n'y a personne qui puisse délivrer de ma main ; je ferai l'œuvre, qui m'en empêchera ?
\TextTitle{Yahweh fera une chose nouvelle car Jacob ne l'a pas honoré}
\VS{14}Ainsi parle Yahweh, votre Rédempteur\FTNT{Es. 60:16 ; 1 Co. 1:30 ; Ro. 3:24 ; Ep. 1:7.}, le Saint d'Israël : J'envoie pour l'amour de vous contre Babylone, et je les fais descendre tous fugitifs, et le cri des Chaldéens sera dans les navires.
\VS{15}Je suis Yahweh, votre Saint, le Créateur d'Israël, votre Roi.
\VS{16}Ainsi parle Yahweh, qui fraya un chemin dans la mer, et un sentier parmi les eaux impétueuses ;
\VS{17}qui amena des chars et des chevaux, et de grandes forces ; ils ont été étendus ensemble, et ils ne se relèveront point, ils ont été étouffés, ils ont été éteints comme un lumignon :
\VS{18}Ne pensez plus aux choses passées, et ne considérez point les choses anciennes.
\VS{19}Voici, je m'en vais faire une chose nouvelle\FTNT{2 Co. 5:17.}, qui paraîtra bientôt, ne la connaîtrez-vous pas ? Je mettrai un chemin dans le désert, et des fleuves dans le lieu de désolation.
\VS{20}Les bêtes des champs me glorifieront, les serpents et les autruches, parce que j'aurai mis des eaux dans le désert, et des fleuves dans la solitude, pour abreuver mon peuple que j'ai élu.
\VS{21}Ce peuple que je me suis formé racontera mes louanges.
\VS{22}Mais toi, Jacob, tu ne m'as pas invoqué, car tu t'es lassé de moi, ô Israël !
\VS{23}Tu ne m'as pas offert le menu bétail de tes holocaustes, et tu ne m'as pas glorifié dans tes sacrifices ; je ne t'ai point asservi pour me faire des offrandes, et je ne t'ai point fatigué pour de l'encens.
\VS{24}Tu ne m'as pas acheté à prix d'argent du roseau aromatique, et tu ne m'as pas rassasié de la graisse de tes sacrifices ; mais tu m'as asservi par tes péchés, et tu m'as peiné par tes iniquités.
\VS{25}Moi, JE SUIS celui qui efface tes transgressions pour l'amour de moi, et je ne me souviendrai plus de tes péchés.
\VS{26}Réveille ma mémoire, et plaidons ensemble ; toi, déclare pour que tu puisses être justifié.
\VS{27}Ton premier père a péché, et tes docteurs se sont rebellés contre moi.
\VS{28}C'est pourquoi j'ai profané les chefs du lieu saint, et j'ai livré Jacob à la destruction, et Israël à l'opprobre.
\Chap{44}
\TextTitle{Promesse de l'Esprit, folie de l'idolâtrie}
\VerseOne{}Ecoute maintenant, ô Jacob, mon serviteur, et toi Israël que j'ai choisi !
\VS{2}Ainsi parle Yahweh, qui t'a fait et formé dès le ventre, celui qui te soutient : Ne crains point, ô Jacob, mon serviteur ! Et toi Jeshurun que j'ai élu.
\VS{3}Car je répandrai des eaux sur celui qui est altéré, et des rivières sur la terre sèche ; je répandrai mon esprit sur ta postérité, et ma bénédiction sur ta descendance.
\VS{4}Et ils germeront comme au milieu de l'herbe, comme les saules auprès des courants d'eau.
\VS{5}L'un dira : Je suis à Yahweh ; et l'autre se réclamera du nom de Jacob ; et un autre écrira de sa main : Je suis à Yahweh, et se nommera du nom d'Israël.
\VS{6}Ainsi parle Yahweh, le Roi d'Israël et son Rédempteur, Yahweh des armées : Je suis le premier, et je suis le dernier ; et à part moi il n'y a point de Dieu.
\VS{7}Et qui, comme moi, a appelé, déclaré et ordonné cela, depuis que j'ai établi le peuple ancien ? Qu'ils déclarent les choses à venir, les choses qui arriveront ci-après !
\VS{8}Ne soyez point effrayés et ne soyez point troublés ; ne te l'ai-je pas fait entendre et déclarer dès ce temps-là ? Vous êtes mes témoins ; y a-t-il un autre Dieu que moi ? Certes il n'y a pas d'autre Rocher\FTNT{Yahweh dit qu'il ne connaît pas d'autre rocher. Jésus-Christ est ce rocher qui suivait les Hébreux dans le désert (Mt. 16:18 ; 1 Co. 10:1-4. Voir aussi commentaire en Es. 8:14). }, je n'en connais pas.
\VS{9}Les ouvriers d'images taillées ne sont tous que vanité, et leurs choses les plus désirables ne sont d'aucun profit ; elles le témoignent elles-mêmes, elles ne voient point, et ne connaissent point, afin qu'ils soient honteux.
\VS{10}Mais qui est-ce qui fabrique un dieu, ou fond une image taillée, pour n'en avoir aucun profit ?
\VS{11}Voici, tous ses compagnons seront honteux, car ces ouvriers-là sont d'entre les hommes. Qu'ils s'assemblent tous, qu'ils se tiennent là ! Ils seront effrayés et rendus honteux tous ensemble.
\VS{12}Le forgeron fait une hache, et il travaille avec le charbon, et il le forme à coups de marteau ; il le fait à force de bras, même il a faim et il est sans force, il ne boit point d'eau, et il est tout fatigué.
\VS{13}Le charpentier étend sa règle, il trace sa forme au crayon avec de la craie ; il le fait avec des équerres, et le forme au compas, et le fait à la ressemblance d'un homme, selon la beauté d'un homme, afin qu'il demeure dans la maison.
\VS{14}Il se coupe des cèdres, et prend un cyprès, ou un chêne, qu'il a laissé croître parmi les arbres de la forêt ; il plante des pins, et la pluie les fait croître.
\VS{15}Ces arbres servent à l'homme pour brûler, car il en prend et il s'en chauffe. Il en fait du feu, dis-je, et en cuit du pain ; et il en fait aussi un dieu et se prosterne devant lui ; il en fait une image taillée et l'adore.
\VS{16}Il en brûle au feu une partie, et d'une autre partie il mange sa chair, laquelle il rôtit, et s'en rassasie ; il s'en chauffe aussi, et il dit : Ah ! Ah ! Je me chauffe, je vois la flamme !
\VS{17}Puis avec le reste il fait un dieu pour être son image taillée ; il se prosterne devant elle, il l'adore, il lui fait sa requête et dit : Délivre-moi, car tu es mon dieu !
\VS{18}Ils ne savent et n'entendent rien, car on leur a plâtré les yeux afin qu'ils ne voient point, et les cœurs pour qu'ils ne comprennent point.
\VS{19}Nul ne rentre en lui-même\FTNT{So. 2:1 ; 2 Co. 13:5.}, et il n'a ni la connaissance ni l'intelligence pour dire : J'en ai brûlé une partie au feu, et même j'ai cuit du pain sur les charbons, j'ai rôti de la viande et je l'ai mangée ; et avec le reste ferais-je une abomination ? Adorerais-je une branche de bois ?
\VS{20}Il se repaît de cendres, et son cœur abusé l'égare, et il ne délivrera point son âme, et ne dira point : N'est-ce pas du mensonge que j'ai dans ma main droite ?
\TextTitle{Yahweh rachète son peuple}
\VS{21}Souviens-toi de ces choses, ô Jacob ! Ô Israël, car tu es mon serviteur ; je t'ai formé, tu es mon serviteur, ô Israël ! Je ne t'oublierai pas.
\VS{22}J'efface tes transgressions comme une nuée épaisse, et tes péchés comme une nuée ; reviens à moi, car je t'ai racheté.
\VS{23}Ô cieux ! Réjouissez-vous avec chants de triomphe, car Yahweh a opéré ; profondeurs de la terre, jetez des cris de réjouissance ! Montagnes, éclatez de joie avec chant de triomphe ! Et vous aussi forêts et tous les arbres qui êtes en elles ! Parce que Yahweh a racheté Jacob, et s'est manifesté glorieusement en Israël.
\VS{24}Ainsi parle Yahweh, ton Rédempteur, celui qui t'a formé dès le ventre : Je suis Yahweh qui ai fait toutes choses, qui seul ai étendu les cieux, et qui ai par moi-même étendu la terre ;
\VS{25}qui dissipe les signes des menteurs, qui rends insensés les devins ; qui renverse l'esprit des sages, et qui fait que leur science devient une folie.
\VS{26}C'est lui qui confirme la parole de son serviteur, et accomplit le conseil  de ses messagers ; qui dit à Jérusalem : Tu seras encore habitée ! Et aux villes de Juda : Vous serez rebâties ! Et je redresserai ses lieux déserts.
\VS{27}Qui dit à l'abîme : Sois asséchée, et je tarirai tes fleuves.
\TextTitle{Prophétie sur le rétablissement d'Israël par Cyrus}
\VS{28}Qui dit de Cyrus\FTNT{Esaïe prophétisa la destruction de Babylone deux siècles avant la réalisation de cet événement  le 5 octobre 539 av. J.-C. Fait remarquable : il précisa même le nom du commandant Cyrus qui dompta le lion babylonien. L’historien Hérodote donnera  par ailleurs raison au prophète sur le déroulement de la prise de Babylone.} : Il est mon berger, et il accomplira tout mon bon plaisir ; disant même à Jérusalem : Tu seras rebâtie ! Et au temple : Tu seras fondé.
\Chap{45}
\TextTitle{Cyrus suscité par Yahweh}
\VerseOne{}Ainsi parle Yahweh à son oint, à Cyrus\FTNT{Cyrus le Grand (580 av. J.-C. - 530 av. J.-C.). Voir Esd. 1.},
\VS{2}que je tiens par la main droite, pour terrasser les nations devant lui, et pour délier les ceintures des rois, pour ouvrir devant lui les portes, afin qu'elles ne soient point fermées.
\VS{3}J'irai devant toi, et j'aplanirai les lieux tortueux ; je romprai les portes d'airain, et je mettrai en pièces les barres de fer. Et je te donnerai des trésors cachés, et des richesses le plus secrètement gardées, afin que tu saches que je suis Yahweh, le Dieu d'Israël, qui t'appelle par ton nom.
\VS{4}Pour l'amour de Jacob, mon serviteur, et d'Israël mon élu ; je t'ai, dis-je, appelé par ton nom, et je t'ai surnommé avant que tu me connaisses.
\TextTitle{Yahweh, le seul Dieu}
\VS{5}Je suis Yahweh, et il n'y en a point d'autre ; à part moi, il n'y a point de Dieu. Je t'ai ceint avant que tu me connaisses,
\VS{6}afin que l'on sache, du soleil levant au soleil couchant, qu'à part moi, il n'y a point de Dieu. Je suis Yahweh, et il n'y en a point d'autre.
\VS{7}Je forme la lumière, et je crée les ténèbres ; je fais la paix et je crée l'adversité ; moi, Yahweh, je fais toutes ces choses.
\VS{8}Ô cieux ! Répandez la rosée d'en haut, et que les nuées laissent couler la justice ! Que la terre s'ouvre, qu'elle produise le salut, qu'elle fasse également germer la justice ! Moi, Yahweh, je crée ces choses.
\VS{9}Malheur à celui qui conteste avec celui qui l'a façonné ! Vase parmi des vases de terre ! L'argile dit-elle à celui qui la façonne : Que fais-tu ? Et l'œuvre dit-elle à l'ouvrier : Tu n'as point de mains\FTNT{Jé. 18:6 ; Ro. 9:21.} ?
\VS{10}Malheur à celui qui dit à son père : Qu'engendres-tu ? Et à sa mère : Qu'enfantes-tu ?
\VS{11}Ainsi parle Yahweh, le Saint d'Israël, qui est son Créateur : Interrogez-moi sur les choses à venir, mes fils ; me commanderez-vous sur l'œuvre de mes mains ?
\VS{12}C'est moi qui ai fait la terre et qui ai créé l'homme sur elle ; c'est moi qui ai étendu les cieux de mes mains, et qui ai donné la loi à toute leur armée.
\VS{13}C'est moi qui ai suscité Cyrus dans ma justice, et j'aplanirai toutes ses voies ; il rebâtira ma ville, et libérera mes captifs\FTNT{Cyrus le grand libéra les Juifs après 70 ans de captivité (Esd. 1).}, sans rançon ni présents, dit Yahweh des armées.
\TextTitle{Les autres peuples reconnaîtront la main de Yahweh sur Israël}
\VS{14}Ainsi parle Yahweh : Le travail de l'Egypte, et le trafic de l'Ethiopie, et ceux des Sabéens, gens de grande stature, passeront chez toi Jérusalem, et seront à toi ; ils marcheront à ta suite, ils passeront enchaînés, ils se prosterneront devant toi, ils te diront en suppliant : Certainement, Dieu est au milieu de toi, et il n'y a point d'autre Dieu que lui.
\VS{15}En vérité, tu es le Dieu qui te caches, le Dieu d'Israël, le Sauveur.
\VS{16}Ils sont tous honteux et confus, ils s'en vont tous avec ignominie, les fabricants d'idoles.
\VS{17}Mais Israël a été sauvé par Yahweh, d'un salut éternel ; vous ne serez ni honteux ni confus jusque dans l'éternité.
\VS{18}Car ainsi parle Yahweh qui a créé les cieux, Dieu lui-même qui a formé la terre, qui l'a faite et qui l'a affermie ; qui l'a créée pour qu'elle ne soit pas informe\FTNT{Informe : de l'hébreu « tohuw » qui signifie « informe, confusion, solitude, désert, néant ». On retrouve ce mot dès Ge. 1:2.}, qui l'a formée pour qu'elle soit habitée ; je suis Yahweh, et il n'y en a point d'autre.
\VS{19}Je n'ai point parlé en secret ni dans quelque lieu ténébreux de la terre ; je n'ai point dit à la postérité de Jacob : Cherchez-moi vainement ! Je suis Yahweh, qui prononce ce qui est juste, qui déclare ce qui est droit.
\VS{20}Assemblez-vous et venez, approchez-vous ensemble, vous les réchappés des nations ! Ceux qui portent le bois de leur image taillée ne savent rien, et invoquent un dieu qui ne sauve pas.
\VS{21}Déclarez-le, et faites-les approcher ! Qu'ils prennent conseil ensemble ! Qui a fait entendre ces choses dès l'origine, et les a déclarées dès longtemps ? N'est-ce pas moi, Yahweh ? Or il n'y a point d'autre Dieu à part moi ; un Dieu juste et un Sauveur, il n'y en a pas d'autre à part moi.
\VS{22}Vous tous qui êtes aux extrémités de la terre, regardez vers moi, et soyez sauvés ; car je suis Dieu, et il n'y en a point d'autre.
\VS{23}Je le jure par moi-même, la parole sort en justice de ma bouche, et elle ne sera point révoquée : Tout genou fléchira devant moi, et toute langue jurera par moi\FTNT{Ph. 2:9-11.}.
\VS{24}Certainement, on dira de moi : En Yahweh seul sont la justice et la force ; à lui viendront, pour être confondus, tous ceux qui étaient irrités contre lui.
\VS{25}Toute la postérité d'Israël sera justifiée, et elle se glorifiera en Yahweh.
\Chap{46}
\TextTitle{La puissance de Yahweh, l'incapacité des idoles}
\VerseOne{}Bel s'incline sur ses genoux, Nebo est renversé ; leurs faux dieux sont mis sur leurs bêtes et leur bétail ; les idoles que vous portiez, ont été chargées, elles sont un fardeau pour la bête fatiguée !
\VS{2}Elles se sont courbées, elles se sont inclinées ensemble sur leurs genoux, et ne peuvent échapper au fardeau, et elles-mêmes s'en vont en captivité.
\VS{3}Ecoutez-moi, maison de Jacob, et vous tous, tout le reste de la maison d'Israël, dont je me suis chargé dès le ventre, et que j'ai porté dès le sein maternel.
\VS{4}Jusqu'à votre vieillesse, JE SUIS ; et je vous chargerai sur moi jusqu'à votre blanche vieillesse, je l'ai fait, et je vous porterai encore, je vous chargerai sur moi et vous sauverai.
\VS{5}A qui me ferez-vous ressembler, et à qui m'égalerez-vous ? A qui me comparerez-vous pour que nous soyons semblable ? 
\VS{6}Ils tirent l'or de la bourse, et pèsent l'argent à la balance, et ils engagent un orfèvre pour en faire un dieu ; ils l'adorent, et se prosternent devant lui.
\VS{7}On le porte sur les épaules, on s'en charge ; on le pose en sa place où il se tient debout et ne bouge point de son lieu, puis on crie à lui, mais il ne répond pas, et il ne délivre pas de la détresse ceux qui crient vers lui.
\VS{8}Souvenez-vous de cela, et montrez-vous des hommes ; rappelez-le à votre pensée, ô vous transgresseurs !
\VS{9}Souvenez-vous des premières choses d'autrefois ; je suis Dieu, et il n'y en a point d'autre, je suis Dieu et il n'y en a point comme moi ;
\VS{10}qui déclare dès le commencement ce qui doit arriver à la fin, et longtemps auparavant, les choses qui n'ont pas encore été faites ; qui dis : Mon conseil tiendra, et j'exécuterai tout mon bon plaisir ;
\VS{11}qui appelle de l'orient l'oiseau de proie, et d'une terre éloignée un homme pour exécuter mon conseil. Oui, j'ai parlé, aussi je ferai venir la chose ; je l'ai formé, aussi je l'accomplirai. 
\VS{12}Ecoutez-moi, vous qui avez le cœur endurci et qui êtes éloignés de la justice.
\VS{13}Je fais approcher ma justice, elle ne s'éloignera point loin ; et mon salut, il ne tardera pas. Je mettrai le salut en Sion pour Israël, qui est ma gloire.
\Chap{47}
\TextTitle{Jugement sur Babylone}
\VerseOne{}Descends, et assieds-toi dans la poussière, vierge, fille de Babylone ! Assieds-toi à terre, il n'y a plus de trône pour la fille des Chaldéens ! Car tu ne te feras plus appeler la délicate et la voluptueuse.
\VS{2}Mets la main aux meules, et fais moudre la farine ; délie tes tresses, déchausse-toi, découvre tes jambes et traverse les fleuves !
\VS{3}Ta honte sera découverte et ton opprobre sera vue ; je prendrai vengeance, je n'irai point contre toi en homme.
\VS{4}Quant à notre Rédempteur, son Nom est Yahweh des armées, le Saint d'Israël.
\VS{5}Assieds-toi sans dire mot, et entre dans les ténèbres, fille des Chaldéens, car tu ne te feras plus appeler la dame des royaumes.
\VS{6}J'ai été embrasé de colère contre mon peuple, j'ai profané mon héritage, c'est pourquoi je les ai livrés entre tes mains, mais tu n'as point usé de miséricorde envers eux, tu as durement appesanti ton joug sur le vieillard.
\VS{7}Et tu as dit : Je serai dame à toujours ! De sorte que tu n'as point mis ces choses-là dans ton cœur, tu ne t'es point souvenue ce qu'en serait la fin.
\VS{8}Maintenant donc écoute ceci, toi voluptueuse qui habite avec assurance, et qui dis en ton cœur : C'est moi, et il n'y en a point d'autre que moi ; je ne deviendrai point veuve, et je ne saurai point ce que c'est que d'être privée d'enfants.
\VS{9}Mais ces deux choses t'arriveront en un moment, en un même jour, la privation d'enfants et le veuvage ; elles viendront sur toi dans leur perfection, pour le  grand nombre de tes sortilèges, et pour la grande abondance de tes enchantements\FTNT{Ap. 18:7-8.}.
\VS{10}Et tu t'es confiée dans ta méchanceté, et disais : Personne ne me voit ! Ta sagesse et ta science t'ont pervertie, et tu disais en ton cœur : C'est moi, et il n'y en a point d'autre que moi.
\VS{11} C'est pourquoi le mal viendra sur toi, et tu ne sauras pas quand il sera près d'arriver, et le malheur qui tombera sur toi sera tel, que tu ne pourras pas le détourner ; et la ruine éclatante que tu n'as pas soupçonnée viendra sur toi subitement.
\VS{12}Tiens-toi maintenant avec tes enchantements, et avec le grand nombre de tes sortilèges, après lesquels tu as travaillé dès ta jeunesse ; peut-être pourras-tu en tirer quelque profit ; peut-être en seras-tu renforcé.
\VS{13}Tu t'es lassée à force de demander des conseils. Que les spectateurs des cieux qui contemplent les étoiles, et qui font leurs prédictions selon les lunes, comparaissent maintenant, et qu'ils te délivrent des choses qui viendront sur toi.
\VS{14}Voici, ils sont devenus comme de la paille, le feu les consume, ils ne délivreront pas leur vie du pouvoir de la flamme ; il n'y a point de charbon pour se chauffer et il n'y a point de lueur de feu pour s'asseoir vis-à-vis. 
\VS{15}Tels te sont devenus ceux avec lesquels tu as travaillé et avec lesquels tu as trafiqué dès ta jeunesse, chacun s'en est fui en son quartier comme un vagabon ; il n'y a personne pour te sauver.
\Chap{48}
\TextTitle{Yahweh rappelle ses promesses}
\VerseOne{}Ecoutez ceci, maison de Jacob, qui êtes appelés du nom d'Israël, et qui êtes sortis des eaux de Juda ; qui jurez par le nom de Yahweh, et qui faites mention du Dieu d'Israël, mais non pas conformément à la vérité et à la justice\FTNT{Jé. 5:2.}.
\VS{2}Car ils prennent leur nom de la sainte cité, et ils s'appuient sur le Dieu d'Israël, dont le nom est Yahweh des armées\FTNT{Ex. 20:7.}.
\VS{3}J'ai déclaré les premières choses dès le commencement, elles sont sorties de ma bouche et je les ai publiées ; je les ai faites subitement et elles se sont accomplies.
\VS{4}Parce que j'ai connu que tu es obstiné, que ton cou est une barre de fer, et que ton front est d'airain,
\VS{5}je t'ai déclaré ces choses dès lors, et je les ai faites entendre avant qu'elles arrivent, de peur que tu ne dises : Mes dieux ont fait ces choses ; mon image taillée, et mon image de fonte les ont ordonnées.
\VS{6}Tu l'entends ! Vois tout ceci ! Et vous, ne l'annoncerez-vous pas ? Je te fais entendre dès maintenant des choses nouvelles, et qui étaient en réserve et que tu ne savais pas.
\VS{7}Elles sont créées maintenant, et non pas depuis le commencement ; et avant ce jour-ci tu n'en avais rien entendu, afin que tu ne dises pas : Voici, je les savais bien.
\VS{8}Oui, tu n'en avais pas entendu parler, oui, tu ne savais pas ; oui, depuis ce temps ton oreille n'a pas été ouverte ; car j'ai connu que tu agirais perfidement; aussi tu as été appelé transgresseur dès le ventre.
\VS{9}Pour l'amour de mon Nom, je diffère ma colère ; et pour l'amour de ma louange, je retiens mon courroux contre toi, afin de ne pas te retrancher.
\VS{10}Voici, je t'ai épuré, mais non pas comme on épure l'argent ; je t'ai éprouvé au creuset de l'affliction.
\VS{11}Pour l'amour de moi, pour l'amour de moi, je le ferai, car comment mon Nom serait-il profané ? Certes, je ne donnerai pas ma gloire à un autre
\VS{12}Ecoute-moi, Jacob ! Et toi Israël, mon appelé ; moi, JE SUIS le premier, JE SUIS aussi le dernier.
\VS{13}Ma main aussi a fondé la terre et ma droite a étendu les cieux ; quand je les appelle, ils comparaissent ensemble.
\VS{14}Vous tous, assemblez-vous et écoutez ! Lequel parmi eux a déclaré ces choses ? Yahweh l'aime et exécutera son bon plaisir contre Babylone, et son bras sera contre sur les Chaldéens.
\VS{15}Moi, JE SUIS celui qui ai parlé, je l'ai aussi appelé, je l'ai amené, et ses desseins réussiront.
\VS{16}Approchez-vous de moi et écoutez ceci ! Dès le commencement, je n'ai point parlé en secret, depuis l'origine de ces choses, JE SUIS. Or maintenant, le Seigneur, Yahweh, et son Esprit m'ont envoyé.
\VS{17}Ainsi parle Yahweh, ton Rédempteur, le Saint d'Israël : Je suis Yahweh, ton Dieu, qui t'enseigne pour ton profit, et qui te guide dans le chemin où tu dois marcher.
\VS{18}Ô ! Si tu étais attentif à mes commandements, ta paix serait comme un fleuve, et ta justice comme les flots de la mer\FTNT{Jos. 1:8 ; Ps. 1:2 ; Jn. 14:21 ; Ja. 1:22.},
\VS{19}ta postérité serait comme le sable, et ceux qui sortent de tes entrailles comme les grains de sable\FTNT{Ge. 15:5 ; Ge. 22:17 ; Ge. 32:12.} ; son nom ne serait point retranché ni effacé de devant ma face.
\VS{20}Sortez de Babylone, fuyez loin des Chaldéens ! Publiez ceci avec une voix de chant de triomphe, annoncez le, portez ceci jusqu'aux extrémités de la terre, dites : Yahweh a racheté son serviteur Jacob !
\VS{21}Et ils n'auront pas soif quand il les fera marcher dans les déserts ; il fera découler pour eux l'eau hors du rocher, même il leur fendra le rocher, et les eaux couleront.
\VS{22}Il n'y a point de paix pour les méchants, dit Yahweh.
\Chap{49}
\TextTitle{Le Messie, la lumière de tous les peuples}
\VerseOne{}Iles, écoutez-moi ! Soyez attentifs, vous peuples éloignés ! Yahweh m'a appelé dès le ventre, il a fait mention de mon nom dès les entrailles de ma mère\FTNT{Jé. 1:5 ; Ps. 139:16.}.
\VS{2}Et il a rendu ma bouche semblable à une épée aiguë ; il m'a caché dans l'ombre de sa main, et m'a rendu semblable à une flèche bien polie, il m'a serré dans son carquois.
\VS{3}Et il m'a dit : Tu es mon serviteur, ô Israël, en qui je serai glorifié.
\VS{4}Et moi j'ai dit : J'ai travaillé en vain, j'ai consumé ma force pour néant et sans fruit ; toutefois mon jugement est auprès de Yahweh, et ma récompense est auprès de mon Dieu.
\VS{5}Maintenant donc, Yahweh, qui m'a formé dès le ventre pour être à son service, m'a dit que je lui ramène Jacob, mais Israël ne se rassemble point ; toutefois je serai honoré aux yeux de Yahweh, et mon Dieu sera ma force.
\VS{6}Il me dit : C'est peu de chose que tu sois serviteur pour relever les tribus de Jacob et pour ramener les restes d'Israël ; c'est pourquoi je te donne pour lumière aux nations, afin que tu sois mon salut jusqu'aux extrémités de la terre.
\VS{7}Ainsi parle Yahweh, le Rédempteur, le Saint d'Israël, à celui qu'on méprise, à celui qui est abominable au peuple, au serviteur de ceux qui dominent ; les rois le verront, et se lèveront, et les princes aussi, et ils se prosterneront devant lui, pour l'amour de Yahweh, qui est fidèle, et du Saint d'Israël qui t'a élu.
\VS{8}Ainsi parle Yahweh : Je t'ai exaucé au temps de la bienveillance, et je t'ai aidé au jour du salut ; je te garderai, et je te donnerai pour être l'alliance du peuple, pour relever la terre, afin que tu possèdes les héritages désolés ;
\VS{9}disant à ceux qui sont emprisonnés : Sortez ! Et à ceux qui sont dans les ténèbres : Montrez-vous ! Ils paîtront sur les chemins, et leurs pâturages seront sur tous les lieux élevés.
\VS{10}Ils n'auront pas faim et ils n'auront pas soif ; la chaleur et le soleil ne les frapperont plus, car celui qui a pitié d'eux sera leur guide, et les conduira vers des sources d'eaux\FTNT{Ps. 121:6 ; Lu. 1:67-79.}.
\VS{11}Et je réduirai toutes mes montagnes en chemins, et mes sentiers seront relevés.
\VS{12}Voici, ceux-ci viennent de loin, et voici ceux-là viennent du nord et de l'occident, et les autres du pays de Sinim.
\VS{13}Ô cieux, réjouissez-vous avec des chants de triomphe! Et toi, ô terre, sois dans l'allégresse ! Et vous, ô montagnes, éclatez de joie avec des chants de triomphe ! Car Yahweh console son peuple, il a compassion de ceux qu'il a affligés.
\VS{14}Mais Sion disait : Yahweh me délaisse, le Seigneur m'oublie !
\VS{15}Une femme peut-elle oublier son enfant qu'elle allaite de sorte qu'elle n'ait pas pitié du fils de ses entrailles ? Mais quand les femmes les oublieraient, moi je ne t'oublierai point.
\VS{16}Voici, je t'ai gravé sur les paumes de mes mains ; tes murs sont continuellement devant moi.
\VS{17}Tes enfants viennent à grande hâtent, mais ceux qui te détruisaient et ceux qui te réduisaient en désert, sortiront du milieu de toi.
\VS{18}Elève tes yeux autour de toi, et regarde : Tous ceux-ci s'assemblent, ils viennent à toi. Je suis vivant, dit Yahweh, tu te revêtiras de tous comme d'une parure, et tu t'en orneras comme une épouse.
\VS{19}Car tes déserts, tes ruines, et ton pays détruit seront désormais trop étroits pour ses habitants, et ceux qui t'engloutissaient s'éloigneront.
\VS{20}Les enfants que tu auras après avoir perdu les autres diront encore, à tes oreilles : Le lieu est trop étroit pour moi, fais-moi de la place pour que je puisse y demeurer.
\VS{21}Et tu diras en ton cœur : Qui m'a engendré ceux-ci vu que j'avais perdu mes enfants et que j'étais stérile, emmenée en captivité et agitée ? Et qui m'a nourri ceux-ci ? Voici, j'étais restée toute seule, et ceux-ci où étaient-ils ?
\VS{22}Ainsi parle le Seigneur Yahweh : Voici, je lèverai ma main vers les nations et je dresserai ma bannière vers les peuples ; et ils ramèneront tes fils entre leurs bras, et ils porteront tes filles sur les épaules.
\VS{23}Et les rois seront tes nourriciers et leurs princesses, leurs femmes, tes nourrices ; ils se prosterneront devant toi le visage contre terre, et ils lécheront la poussière de tes pieds ; et tu sauras que je suis Yahweh, et que ceux qui se confient en moi ne seront point confus\FTNT{Ps. 22:5-6 ; Ps. 69:7 ; Ro. 9:33 ; 1 Pi. 2:6.}.
\VS{24}Le butin sera-t-il ôté à l'homme puissant ? Et les captifs du juste seront-ils délivrés ?
\VS{25}Car ainsi parle Yahweh : Même les captifs pris par l'homme puissant lui seront ôtés, et le butin de l'homme fort lui sera enlevé ; car je plaiderai moi-même avec ceux qui plaident contre toi, et je délivrerai tes enfants.
\VS{26}Et je ferai manger leur propre chair à ceux qui t'oppriment ; et ils s'enivreront de leur sang comme du moût, et toute chair connaîtra que je suis Yahweh, ton Sauveur, ton Rédempteur, le Puissant de Jacob.
\Chap{50}
\TextTitle{Avertissements de Yahweh par son serviteur}
\VerseOne{}Ainsi parle Yahweh : Où est la lettre de divorce par laquelle j'ai répudié votre mère\FTNT{De. 24:1 ; Jé. 3:8 ; Mt. 5:31.} ? Ou bien, auquel de mes créanciers vous ai-je vendus ? Voici, vous avez été vendus à cause de vos iniquités, et votre mère a été répudiée à cause de vos transgressions.
\VS{2}Je suis venu : Pourquoi ne s'est-il trouvé personne ? J'ai appelé : Pourquoi personne n'a-t-il répondu ? Ma main est-elle trop courte pour racheter\FTNT{No. 11:23 ; Es. 59:1.} ? Ou n'y a-t-il plus de force en moi pour délivrer ? Voici, par ma menace, je dessèche la mer, je réduis les fleuves en désert ; leurs poissons se corrompent faute d'eau, et ils meurent de soif.
\VS{3}Je revêts les cieux de noirceur, et je fais d'un sac leur couverture.
\VS{4}Le Seigneur, Yahweh, m'a donné la langue des savants, pour que je sache soutenir par la parole celui qui est accablé de maux\FTNT{Job. 6:14 ; 1 Th. 5:14. } ; chaque matin il me réveille soigneusement afin que je prête l'oreille aux discours des sages.
\VS{5}Le Seigneur Yahweh m'a ouvert l'oreille et je n'ai pas été rebelle, et je ne me suis pas retiré en arrière.
\VS{6}J'ai exposé mon dos à ceux qui me frappaient et mes joues à ceux qui me tiraient le poil ; je n'ai pas caché mon visage aux opprobres et aux crachats\FTNT{Mt. 5:39 ; Mt. 26:67 ; Lu. 6:29 ; Lu. 18:32.}.
\VS{7}Mais le Seigneur, Yahweh m'a aidé, c'est pourquoi je n'ai point été confus, et ainsi j'ai rendu mon visage semblable à un caillou\FTNT{Ez. 3:8-9.}, car je sais que je ne serais point rendu honteux.
\VS{8}Celui qui me justifie est proche ; qui plaidera contre moi ? Comparaissons ensemble ! Qui est mon adversaire ? Qu'il s'approche de moi.
\VS{9}Voici, le Seigneur, Yahweh m'aidera, qui est celui qui me condamnera ? Voici, tous seront usés comme un vêtement, la teigne les dévorera.
\VS{10}Qui est celui d'entre vous qui craint Yahweh, et qui obéit à la voix de son serviteur ! Que celui qui marche dans les ténèbres, et qui n'a pas de clarté, se confie dans le Nom de Yahweh, et qu'il s'appuie sur son Dieu.
\VS{11}Voici, vous tous qui allumez le feu, et qui vous ceignez d'étincelles, marchez à la lueur de votre feu et des étincelles que vous avez embrasées ; voici ce que vous aurez de ma main ; vous vous coucherez dans les tourments.
\Chap{51}
\TextTitle{Exhortation à ceux qui recherchent Yahweh}
\VerseOne{}Ecoutez-moi, vous qui poursuivez la justice et qui cherchez Yahweh ! Regardez au rocher d'où vous avez été taillés, et au creux de la citerne dont vous avez été tirés.
\VS{2}Regardez à Abraham, votre père, et à Sara qui vous a enfantés ; car lui seul je l'ai appelé, je l'ai béni et multiplié\FTNT{Ro. 4:1-16 ; Hé. 11:8-12.}.
\VS{3}Car Yahweh console Sion, il console de toutes ses désolations, il rendra son désert semblable à Eden, et sa terre aride à un jardin de Yahweh. En elle sera trouvée la joie et l'allégresse, la reconnaissance et la voie de mélodie.
\VS{4}Ecoutez-moi donc attentivement, mon peuple, et prêtez-moi l'oreille, vous ma nation ; car la loi sortira de moi, et j'établirai mon jugement pour être la lumière des peuples.
\VS{5}Ma justice est proche, mon salut va paraître, et mes bras jugeront les peuples ; les îles espéreront en moi, elles se confieront en mon bras.
\VS{6}Levez les yeux vers les cieux et regardez en bas sur la terre ! Car les cieux s'évanouiront comme la fumée, et la terre tombera en lambeaux comme un vêtement, et ses habitants périront pareillement ; mais mon salut demeurera éternellement, et ma justice ne sera point anéantie.
\VS{7}Ecoutez-moi, vous qui connaissez la justice, peuple dans le cœur duquel est ma loi ! Ne craignez point l'opprobre des hommes et ne soyez point effrayés devant leurs outrages.
\VS{8}Car la teigne les rongera comme un vêtement\FTNT{Mt. 6:19 ; Lu. 12:33 ; Ja. 5:2.}, et la gerce les dévorera comme de la laine ; mais ma justice demeurera toujours, et mon salut d'âge en âge.
\VS{9}Réveille-toi, réveille-toi, revêts-toi de force, bras de Yahweh ! Réveille-toi comme aux jours anciens, aux siècles passés. N'es-tu pas celui qui tailla en pièce l'Egypte, et qui blessa mortellement le dragon ?
\VS{10}N'est-ce pas toi qui fis tarir la mer, les eaux du grand abîme ? Qui réduisit les lieux les plus profonds de la mer en un chemin afin que les rachetés y passent ?
\VS{11}Ainsi ceux dont Yahweh aura payé la rançon, retourneront, ils iront à Sion avec chants de triomphe ; et une allégresse éternelle couronnera leurs têtes ; ils obtiendront la joie et l'allégresse ; la douleur et le gémissement s'enfuiront.
\VS{12}C'est moi qui suis celui qui vous console. Qui es-tu pour avoir peur de l'homme mortel qui mourra, et du fils de l'homme qui deviendra comme du foin ?
\VS{13}Et tu oublierais Yahweh qui t'a fait, qui a étendu les cieux et fondé la terre ; et chaque jour tu tremblerais continuellement à cause de la fureur de ton oppresseur parce qu'il s'apprête à détruire ! Et où est maintenant la fureur de ton oppresseur ?
\VS{14}Il se hâtera de faire que celui qui aura été transporté d'un lieu à l'autre, soit mis en liberté, afin qu'il ne meure point dans la fosse, et que son pain ne lui manque pas.
\VS{15}Car je suis Yahweh, ton Dieu, qui fend la mer, et les flots rugissants. Yahweh des armées est son Nom.
\VS{16}Or je mets mes paroles dans ta bouche, et je te couvre de l'ombre de ma main, afin que j'affermisse les cieux, que je fonde la terre, et que je dise à Sion : Tu es mon peuple !
\VS{17}Réveille-toi, réveille-toi ! Lève-toi, Jérusalem, qui as bu de la main de Yahweh la coupe de sa fureur ; tu as bu, tu as sucé la lie de la coupe d'étourdissement\FTNT{Ps. 60:5 ; Ap. 14:10.} !
\VS{18}Il n'y a pas un de tous les enfants qu'elle a enfantés qui te conduise, et de tous les enfants qu'elle a nourris, il n'y en a pas un qui la prenne par la main.
\VS{19}Ces deux choses te sont arrivées ; qui te plaindra ? Le ravage et la ruine, la famine et l'épée ; par qui te consolerai-je ?
\VS{20}Tes enfants en défaillance gisaient aux carrefours de toutes les rues, comme un bœuf sauvage pris dans les filets, pleins de la fureur de Yahweh, de la répréhension de ton Dieu.
\VS{21}C'est pourquoi, écoute maintenant ceci, ô affligée, ivre, mais non pas de vin.
\VS{22}Ainsi parle Yahweh, ton Seigneur et ton Dieu, qui plaide la cause de son peuple : Voici, je prends de la main la coupe d'étourdissement, la lie de la coupe de ma fureur, tu n'en boiras plus désormais !
\VS{23}Car je la mettrai dans la main de ceux qui t'ont affligée, et qui disaient à ton âme : Courbe-toi, et nous passerons ! C'est pourquoi tu as exposé ton corps  comme la terre, comme une rue pour les passants.
\Chap{52}
\TextTitle{Le réveil de Jérusalem, la ville sainte}
\VerseOne{}Réveille-toi, réveille-toi, Sion ! Revêts-toi de ta force ! Jérusalem, ville sainte ! Revêts-toi de tes vêtements magnifiques ! Car l'incirconcis et le souillé ne passeront plus désormais parmi toi. 
\VS{2}Jérusalem, secoue ta poussière, lève-toi, et assieds-toi ! Détache les liens de ton cou, captive, fille de Sion !
\VS{3}Car ainsi parle Yahweh : Vous avez été vendus pour rien, et vous serez aussi rachetés sans argent.
\VS{4}Car ainsi parle le Seigneur, Yahweh : Mon peuple descendit jadis en Egypte pour y séjourner ; mais les Assyriens l'opprimèrent sans cause.
\VS{5}Et maintenant, qu'ai-je à faire ici, dit Yahweh, quand mon peuple a été enlevé pour rien ? Ceux qui dominent sur lui le font hurler, dit Yahweh, et mon Nom est blasphémé continuellement chaque jour.
\VS{6}C'est pourquoi mon peuple connaîtra mon Nom ; c'est pourquoi il saura, en ce jour-là, que JE SUIS parle : Voici JE SUIS !
\VS{7}Combien sont beaux sur les montagnes les pieds de celui qui apporte de bonnes nouvelles, qui publie la paix\FTNT{Na. 2:1 ; Ro. 10:15.}, qui apporte de bonnes nouvelles concernant le bien, qui publie le salut, qui dit à Sion : Ton Dieu règne !
\VS{8}Tes sentinelles élèvent leurs voix, elles se réjouissent ensemble avec chants de triomphe ; car de leurs propres yeux elles voient comment Yahweh ramène Sion.
\VS{9}Déserts de Jérusalem, éclatez, réjouissez-vous ensemble avec chants de triomphe ! Car Yahweh console son peuple, il rachète Jérusalem.
\VS{10}Yahweh manifeste le bras de sa sainteté aux yeux de toutes les nations\FTNT{Es. 53:1.}, et toutes les extrémités de la terre verront le salut\FTNT{Toutes les extrémités de la terre verront le salut de Yahweh, c'est-à-dire Jésus (Mt. 28:18-20). } de notre Dieu.
\VS{11}Retirez-vous, retirez-vous, sortez de là ! Ne touchez rien d'impur ! Sortez du milieu d'elle\FTNT{Jé. 51:45 ; 2 Co. 6:17 ; Ap. 18:4.} ! Nettoyez-vous, vous qui portez les vases de Yahweh.
\VS{12}Car vous ne sortirez pas en hâte, et vous ne marcherez pas en fuyant, car Yahweh ira devant vous, et le Dieu d'Israël sera votre arrière-garde.
\TextTitle{Le serviteur de Yahweh}
\VS{13}Voici, mon serviteur prospérera, il sera fort exalté, élevé et glorifié.
\VS{14}Comme plusieurs ont été étonnés en te voyant, son visage était défiguré plus que celui d'aucun homme, et son apparence plus que celle d'aucun fils d'homme ;
\VS{15}ainsi, il aspergera plusieurs nations, et les rois fermeront la bouche sur lui ; car ceux auxquels on n'en avait point parlé le verront ; et ceux qui ne l'avait point entendu l'entendront.
\Chap{53}
\TextTitle{Le sacrifice du Messie, serviteur de Yahweh}
\VerseOne{}Qui a cru à notre prédication ? Et à qui le bras de Yahweh\FTNT{Jésus-Christ homme est le bras de Yahweh. Le bras de Yahweh est le symbole de la puissance divine. Cette puissance s'est manifestée dans l'œuvre du Messie accomplissant le salut du monde. Le prophète est transporté au moment où le peuple juif, après avoir rejeté son Messie, ouvrira enfin les yeux et acceptera celui qu'il a percé (Za. 12:10 ; Ap. 1:7). Voir aussi Jé. 27:4-5 ; Jé. 32:17.} a-t-il été révélé ?
\VS{2}Toutefois il s'est élevé devant lui comme une jeune plante, comme un rejeton qui sort d'une terre desséchée ; il n'y avait en lui ni beauté, ni splendeur, quand nous le regardions, ni apparence qui nous le fasse désirer.
\VS{3}Il était le méprisé et le rejeté des hommes\FTNT{Ps. 22:6-7 ; Mt. 27:27-31 ; Mc. 9:12 ; Jn. 16:32.}, homme de douleur, et sachant ce que c'est que la maladie ; et nous avons comme caché notre visage arrière de lui, tant il était méprisé ; et nous ne l'avons pas estimé.
\VS{4}En vérité, il a porté nos maladies, et il s'est chargé de nos douleurs\FTNT{Mt. 8:17 ; 1 Pi. 2:24.} ; et nous l'avons considéré comme frappé, battu par Dieu et humilié.
\VS{5}Mais il était transpercé pour nos péchés, brisé pour nos iniquités, le châtiment qui nous apporte la paix est tombé sur lui, et c'est par ses meurtrissures que nous avons la guérison.
\VS{6}Nous avons tous été errants\FTNT{Pierre, apôtre de l'Agneau, confirme que le Messie est bel et bien le Bon Berger (1 Pi. 2:25).} comme des brebis, nous nous sommes détournés, chacun suivait son propre chemin, et Yahweh a fait venir sur lui l'iniquité de nous tous.
\VS{7}Opprimé et humilié, il n'a point ouvert sa bouche\FTNT{Mt. 26:62-63 ; Mc. 15:3-5 ; Jn. 19:9 ; Ac. 8:32-33.}, semblable à un agneau qu'on mène à la boucherie, à une brebis muette devant celui qui la tond, et il n'a point ouvert sa bouche.
\VS{8}Il a été enlevé de la force de l'angoisse et de la condamnation ; mais qui racontera sa durée ? Car il a été retranché de la terre des vivants, et la plaie lui a été faite pour les péchés de mon peuple.
\VS{9}On a mis son sépulcre parmi les méchants, et dans sa mort, il a été avec le riche, quoiqu'il n'ait point commis de violence, et qu'il n'y ait point eu de fraude dans sa bouche\FTNT{Mc. 15:28 ; Lu. 23:32-33.}.
\VS{10}Toutefois il a plu à Yahweh de le briser ; il l'a mis dans la souffrance. Après avoir mis son âme en sacrifice pour le péché, il verra une postérité et prolongera ses jours ; et le bon plaisir de Yahweh prospérera en sa main\FTNT{Jé. 23:5.}.
\VS{11}Il jouira du travail de son âme et en sera rassasié ; mon serviteur juste justifiera beaucoup d'hommes par la connaissance qu'ils auront de lui ; et lui-même portera leurs iniquités.
\VS{12}C'est pourquoi je lui donnerai sa part parmi les grands ; il partagera le butin avec les puissants, parce qu'il a livré son âme à la mort, qu'il a été mis au rang des transgresseurs, et que lui-même a porté les péchés de plusieurs, et qu'il a intercédé pour les transgresseurs.
\Chap{54}
\TextTitle{Yahweh réhabilite Israël la délaissée}
\VerseOne{}Réjouis-toi avec chants de triomphe, stérile, toi qui n'enfantes point, toi qui n'a pas connu les douleurs de l'accouchement! Eclate de joie avec chant de triomphe et réjouis-toi  ! Car les enfants de la délaissée seront plus nombreux que les enfants de celle qui est mariée, dit Yahweh.
\VS{2}Elargis l'espace de ta tente, et qu'on étende les couvertures de ton tabernacle : Ne retiens rien ! Allonge tes cordages et affermis tes pieux !
\VS{3}Car tu te répandras à droite et à gauche, et ta postérité possédera les nations et peuplera les villes désertes.
\VS{4}Ne crains pas, car tu ne seras point honteuse, ni confuse, et tu ne rougiras pas ; mais tu oublieras la honte de ta jeunesse, et tu ne te souviendras plus de l'opprobre de ton veuvage.
\VS{5}Car ton Créateur est ton époux : Yahweh des armées est son Nom ; et ton Rédempteur est le Saint d'Israël : Il sera appelé le Dieu de toute la terre.
\VS{6}Car Yahweh t'appelle comme une femme délaissée et à l'esprit affligé, comme une femme qu'on a épousée dans la jeunesse, et qui a été répudiée, dit ton Dieu.
\VS{7}Je t'avais délaissée pour un petit moment, mais je te rassemblerai avec de grandes compassions.
\VS{8}Dans une courte colère, je t'avais un moment caché ma face, mais j'aurai compassion de toi avec une bonté éternelle, dit Yahweh, ton Rédempteur.
\VS{9}Car il en sera pour moi comme les eaux de Noé : De même que j'avais juré que les eaux de Noé ne se répandraient plus sur la terre\FTNT{Ge. 9:11 ; Ge. 8:21.} ; je jure de ne plus m'irriter contre toi, et de ne plus te menacer.
\VS{10}Car quand les montagnes s'en iraient, quand les collines chancelleraient, ma bonté ne s'en ira point de toi, et mon alliance de paix ne chancellera point, dit Yahweh, qui a compassion de toi.
\VS{11}Ô affligée, agitée de la tempête, dénuée de consolation, voici, je coucherai tes pierres d'antimoine, et je te fonderai sur des saphirs ;
\VS{12}et je ferai tes fenêtrages d'agates, et tes portes de rubis, et toute ton enceinte de pierres précieuses.
\VS{13}Aussi tous tes enfants seront enseignés de Yahweh, et grande sera la paix de tes fils.
\VS{14}Tu seras établie en justice, tu seras loin de l'oppression, et tu ne craindras rien ; tu seras, dis-je, loin de la frayeur, car elle n'approchera pas de toi.
\VS{15}Voici, on ne manquera pas de comploter contre toi, cela ne viendra pas de moi ; quiconque complotera contre toi tombera pour l'amour de toi\FTNT{Ps. 91:7 ; Ge. 37.}.
\VS{16}Voici, c'est moi qui ai créé le forgeron soufflant le charbon au feu, et formant un instrument pour son travail, et j'ai créé aussi le destructeur pour détruire.
\VS{17}Aucune arme forgée contre toi ne réussira, et toute langue qui se lèvera en jugement contre toi, tu la condamneras\FTNT{Ps. 23:4.}. Tel est l'héritage des serviteurs de Yahweh, et telle est la justice qui leur viendra de moi, dit Yahweh.
\Chap{55}
\TextTitle{Le salut gratuit par la grâce de Dieu}
\VerseOne{}Vous tous qui avez soif, venez aux eaux, et vous qui n'avez pas d'argent, venez, achetez et mangez ; venez, dis-je, achetez du vin et du lait sans argent, et sans rien payer !
\VS{2}Pourquoi dépensez-vous de l'argent pour ce qui ne nourrit pas ? Pourquoi travaillez-vous pour ce qui ne rassasie pas\FTNT{Ro. 14:17.} ? Ecoutez-moi attentivement, et vous mangerez de ce qui est bon, et votre âme se délectera de la graisse.
\VS{3}Inclinez l'oreille, et venez à moi\FTNT{Mt. 11:28.}, écoutez, et votre âme vivra ; et je traiterai avec vous une alliance éternelle, les miséricordes immuables promises à David.
\VS{4}Voici, je l'ai donné comme témoin auprès des peuples, comme chef et dominateur des peuples.
\VS{5}Voici, tu appelleras des nations que tu ne connais pas, et les nations qui ne te connaissent pas accourront vers toi, à cause de Yahweh, ton Dieu, et du Saint d'Israël, qui t'auras glorifié.
\VS{6}Cherchez Yahweh pendant qu'il se trouve, invoquez-le tandis qu'il est près.
\VS{7}Que le méchant abandonne sa voie, et l'homme injuste ses pensées ; et qu'il retourne à Yahweh, qui aura pitié de lui, et à notre Dieu qui pardonne abondamment\FTNT{Jé. 18:11 ; Ez. 33:11 ; Jon. 3:10 ; 1 Ti. 2:1-4 ; 2 Pi. 3:9.}.
\VS{8}Car mes pensées ne sont pas vos pensées, et mes voies ne sont pas vos voies, dit Yahweh.
\VS{9}Mais autant les cieux sont élevés au-dessus de la terre, autant mes voies sont élevées au-dessus de vos voies, et mes pensées au-dessus de vos pensées.
\VS{10}Car comme la pluie et la neige descendent des cieux et n'y retournent plus, mais arrosent la terre, et la font produire et germer, afin de donner de la semence au semeur, et du pain à celui qui mange,
\VS{11}ainsi en est-il de ma parole qui sort de ma bouche, elle ne retourne point vers moi sans effet, mais elle fait tout ce en quoi je prends plaisir, et prospérera dans l'œuvre pour laquelle je l'ai envoyée.
\VS{12}Car vous sortirez avec joie, et vous serez conduits en paix ; les montagnes et les collines éclateront de joie avec chants de triomphe devant vous, et tous les arbres des champs battront des mains.
\VS{13}Au lieu de l'épine s'élèvera le cyprès, au lieu de la ronce croîtra le myrte ; et ceci fera connaître le nom de Yahweh, et ce sera un signe perpétuel, qui ne sera jamais retranché.
\Chap{56}
\TextTitle{Exhortation à s'attacher à Yahweh}
\VerseOne{}Ainsi parle Yahweh : Observez le jugement, faites ce qui est juste, car mon salut ne tardera pas à venir, et ma justice à être révélée.
\VS{2}Bienheureux l'homme qui fait cela, et le fils de l'homme qui s'y tient, observant le sabbat pour ne pas le profaner, et gardant ses mains pour ne faire aucun mal.
\VS{3}Et que l'enfant de l'étranger qui se joint à Yahweh ne parle pas en disant : Yahweh me séparera entièrement de son peuple ! Et que l'eunuque ne dise pas : Voici, je suis un arbre sec.
\VS{4}Car ainsi parle Yahweh touchant les eunuques : Ceux qui garderont mes sabbats, et qui choisiront ce en quoi je prends plaisir, et qui tiendront dans mon alliance,
\VS{5}je leur donnerai dans ma maison et dans mes murailles une place et un nom meilleur que le nom de fils ou de filles ; je leur donnerai à chacun un nom éternel qui ne périra jamais\FTNT{Ap. 2:17.}.
\VS{6}Et les enfants des étrangers qui se joindront à Yahweh pour le servir, pour aimer le Nom de Yahweh, pour être ses serviteurs, savoir tous ceux qui garderont le sabbat pour ne pas le profaner et qui tiendront dans mon alliance\FTNT{Ex. 31:14.},
\VS{7}je les amènerai sur ma montagne sainte, et je les réjouirai dans ma maison de prière ; leurs holocaustes et leurs sacrifices seront agréés sur mon autel, car ma maison sera appelée une maison de prière\FTNT{Mt. 21:13 ; Mc. 11:17 ; Lu. 19:46.} pour tous les peuples.
\VS{8}Le Seigneur, Yahweh, parle, lui qui rassemble les exilés d'Israël : Je réunirai d'autres peuples à lui, outre ceux déjà rassemblés.
\VS{9}Bêtes des champs, bêtes des forêts, venez toutes pour manger !
\VS{10}Toutes ses sentinelles sont aveugles, elles ne connaissent rien ; ce sont tous des chiens muets, qui ne peuvent aboyer, dormant et demeurant couchés, et aimant à sommeiller.
\VS{11}Ce sont des chiens voraces et insatiables ; ce sont des pasteurs qui ne savent rien comprendre ; tous suivent leur propre voie, chacun à son gain injuste dans son quartier, en disant\FTNT{Mt. 23:24 ; Tit. 1:7-11 ; 1 Pi. 5:2.} :
\VS{12}Venez, je vais chercher du vin, et nous nous enivrerons de boissons fortes ! Nous en ferons autant demain, et même beaucoup plus encore !
\Chap{57}
\TextTitle{Yahweh expose la fausseté et défend le juste}
\VerseOne{}Le juste périt, et nul ne le prend à cœur ; et les gens de bien sont recueillis, sans qu'on y soit attentif, sans qu'on considère que le juste a été recueilli devant le mal\FTNT{Mi. 7:2 ; Ec. 7:15.}.
\VS{2}Il entrera en paix, il reposera sur sa couche, celui qui aura marché dans la droiture\FTNT{Mt. 25:23 ; Lu. 19:17.}.
\VS{3}Mais vous, approchez ici, enfants de l'enchanteresse, race de l'adultère et de la prostituée !
\VS{4}De qui vous êtes-vous moqués ? Contre qui avez-vous ouvert la bouche et tirez-vous la langue ? N'êtes-vous pas des enfants de rébellion, une race de mensonge ?
\VS{5}S'échauffant près des faux dieux, sous tout arbre vert ; égorgeant les enfants dans les vallées, sous les fentes des rochers\FTNT{Lé. 18:21 ; 1 R. 14:23 ; Jé. 2:20 ; Jé. 32:35.}.
\VS{6}Parmi les pierres polies des torrents est ta portion, ce sont elles, ce sont elles qui sont ton lot ; tu leur a aussi répandu ton aspersion, tu leur as aussi offert des offrandes ; puis-je être content de ces choses ?
\VS{7}Tu dresses ta couche sur les montagnes hautes et élevées ; c'est aussi là que tu montes pour offrir des sacrifices.
\VS{8}Et tu mets ton souvenir derrière la porte et les poteaux ; car tu te découvres loin de moi et tu montes, tu élargis ta couche, et tu te l'est taillé plus grande que n'ont fait ceux-là ; tu as aimé leur couche, tu as pris garde aux belles places.
\VS{9}Tu voyages vers le roi avec de l'huile précieuse, et tu ajoutes parfums sur parfums ; tu envoies au loin tes ambassades, tu t'abaisses jusqu'au scheol.
\VS{10}Tu te fatigues par la longueur du chemin, et tu ne dis pas : C'est sans espoir ! Tu trouves encore de la vigueur dans ta main ; c'est pourquoi tu n'as pas été languissante.
\VS{11}Et qui redoutais-tu, qui craignais-tu pour que tu me mentes, pour ne pas te souvenir et te soucier de moi ? N'ai-je pas gardé le silence, et même depuis longtemps, et tu ne me crains pas.
\VS{12}Je vais déclarer ta justice et tes œuvres, qui ne te profiteront pas.
\VS{13}Quand tu crieras, que ceux que tu assembles te délivrent ! Mais le vent les emmènera tous, la vanité les enlèvera ; mais celui qui met sa confiance en moi, héritera la terre et possédera ma montagne sainte\FTNT{Es. 2:3 ; Ps. 2:6 ;  Hé. 12:22.}.
\VS{14}On dira : Frayez, frayez, préparez le chemin, enlevez tout obstacle loin du chemin de mon peuple !
\TextTitle{Yahweh aime l'homme contrit}
\VS{15}Car ainsi parle celui qui est haut et élevé, qui habite dans l'éternité et dont le nom est le Saint : J'habiterai dans les lieux hauts et saints, avec celui qui a le cœur brisé et qui est humble d'esprit, afin de vivifier l'esprit des humbles, et afin de vivifier ceux qui ont le cœur brisé\FTNT{Ps. 34:19 ; Ps. 51:19.}.
\VS{16}Parce que je ne veux pas contester à toujours, et que je ne serai pas irrité à jamais ;  car devant moi tombent en défaillance les esprits, et les âmes que j'ai faites\FTNT{Mi. 7:18 ; Ps. 85:6 ; Ps. 103:9.}.
\VS{17}A cause de l'iniquité de ses gains déshonnêtes, je me suis irrité et je l'ai frappé, je me suis caché ma dans ma colère ; et le rebelle a suivi la voie de son cœur.
\VS{18}J'ai vu ses voies, et toutefois je le guérirai ; je le conduirai et je le restaurerai, lui et ceux qui mènent deuil avec lui.
\VS{19}Je crée les fruits des lèvres. Paix, paix à celui qui est loin et à celui qui est près ! dit Yahweh, car je le guérirai.
\VS{20}Mais les méchants sont comme la mer agitée, quand elle ne peut se calmer, et dont les eaux rejettent la boue et le bourbier.
\VS{21}Il n'y a point de paix pour les méchants, dit mon Dieu.
\Chap{58}
\TextTitle{Le vrai et le faux jeûne}
\VerseOne{}Crie à plein gosier, ne te retiens pas, élève ta voix comme un shofar, et annonce à mon peuple ses iniquités et à la maison de Jacob ses péchés !
\VS{2}Car ils me cherchent tous les jours, ils prennent plaisir à connaître mes voies ; comme une nation qui aurait pratiqué la justice, et qui n'aurait pas abandonné les ordonnances de son Dieu ; ils me demandent des jugements justes, ils prennent plaisir à s'approcher de Dieu, et puis ils disent :
\VS{3}Pourquoi jeûnons-nous, et tu ne le vois pas ? Pourquoi affligeons-nous nos âmes, si tu n'y as point connaissance ? Voici, le jour de votre jeûne, vous trouvez votre plaisir, et vous oppressez tous vos travailleurs.
\VS{4}Voici, vous jeûnez pour faire des querelles et vous disputer, et pour frapper du poing méchamment ; vous ne jeûnez pas comme le veut ce jour, pour que votre voix soit exaucée d'en haut.
\VS{5}Est-ce là le jeûne que j'ai choisi, que l'homme afflige son âme un jour ? Est-ce en courbant sa tête comme le jonc et en étendant le sac et la cendre ? Appelleras-tu cela un jeûne et un jour agréable à Yahweh ?
\VS{6}N'est-ce pas plutôt ici le jeûne que j'ai choisi : Que tu détaches les liens de la méchanceté, que tu délies les cordages du joug, que tu laisses aller libres les opprimés, et que l'on rompe toute espèce de joug ?
\VS{7}N'est ce-pas que tu partages ton pain avec celui qui a faim ? Et que tu fasses venir dans ta maison les affligés errants ? Quand tu vois un homme nu, que tu le couvres, et que tu ne te caches pas de ta propre chair ?
\TextTitle{Bénédiction pour ceux qui pratiquent le bien}
\VS{8}Alors ta lumière éclatera comme l'aurore, et ta guérison germera rapidement ; ta justice ira devant toi, et la gloire de Yahweh sera ton arrière-garde.
\VS{9}Alors tu prieras, et Yahweh t'exaucera ; tu crieras, et il dira : Me voici ! Si tu ôtes du milieu de toi le joug, si tu cesses de lever le doigt et de dire des outrages ;
\VS{10}si tu ouvres ton âme à celui qui a faim, si tu rassasies l'âme affligée ; ta lumière se lèvera sur les ténèbres, et l'obscurité sera comme le midi.
\VS{11}Et Yahweh te conduira continuellement, il rassasiera ton âme dans les grandes sécheresses, il fortifiera tes os, et tu seras comme un jardin arrosé, et comme une source dont les eaux ne tarissent pas\FTNT{Jn. 4:14 ; Ap. 21:6.}.
\VS{12}Et ceux qui sortiront de toi rebâtiront les lieux déserts depuis longtemps, tu rétabliras les fondements ruinés depuis plusieurs générations ; et on t'appellera le réparateur des brèches et le restaurateur des chemins, afin qu'on habite au pays.
\VS{13}Si tu détournes ton pied pendant le sabbat pour ne pas faire ta volonté en mon saint jour ; si tu appelles le sabbat tes délices, et honorable ce qui est saint à Yahweh, et si tu l'honores en ne suivant point tes voies, en ne te livrant pas à tes désirs et à des vains discours,
\VS{14}alors tu prendras plaisir en Yahweh, et je te ferai monter comme à cheval par-dessus les lieux haut élevés de la terre, et je te donnerai à manger de l'héritage de Jacob, ton père ; car la bouche de Yahweh a parlé.
\Chap{59}
\TextTitle{Le péché sépare de Yahweh}
\VerseOne{}Voici, la main de Yahweh n'est pas trop courte pour pouvoir sauver, ni son oreille trop pesante pour pouvoir entendre.
\VS{2}Mais ce sont vos iniquités qui mettent une séparation entre vous et votre Dieu ; ce sont vos péchés qui vous cachent sa face, afin qu'il ne vous entende point\FTNT{De. 31:17-18 ; Ez. 39:23-24.}.
\VS{3}Car vos mains sont souillées de sang, et vos doigts d'iniquité ; vos lèvres profèrent le mensonge, et votre langue déclare la perversité.
\VS{4}Nul ne crie pour la justice, nul ne plaide pour la vérité ; ils s'appuient sur des choses vaines et disent des faussetés, ils conçoivent le mal et enfantent l'iniquité.
\VS{5}Ils font éclore des œufs de vipère, et ils tissent des toiles d'araignée ; celui qui mange de leurs œufs meurt ; et si on les écrase, il en sort une vipère.
\VS{6}Leurs toiles ne servent point à faire des vêtements, et on ne se couvre pas de leurs ouvrages ; car leurs ouvrages sont des ouvrages d'iniquité, et il y a en leurs mains des actions de violence.
\VS{7}Leurs pieds courent au mal, et se hâtent pour répandre le sang innocent ; leurs pensées sont des pensées d'iniquité ; le ravage et la ruine sont sur leurs voies.
\VS{8}Ils ne connaissent point le chemin de la paix, et il n'y a point de jugement dans leurs voies, ils se sont pervertis dans leurs sentiers, tous ceux qui y marchent ignorent la paix\FTNT{Pr. 1:16 ; Pr. 6:16-19.}.
\VS{9}C'est pourquoi le jugement s'est éloigné de nous, et la justice ne parvient pas jusqu'à nous ; nous attendions la lumière, et voici les ténèbres, la clarté, et nous marchons dans l'obscurité.
\VS{10}Nous tâtonnons comme des aveugles le long du mur, nous tâtonnons comme ceux qui sont sans yeux ; nous chancelons en plein midi comme la nuit, et nous sommes dans les lieux abondants comme y sont des morts.
\VS{11}Nous rugissons tous comme des ours, et nous ne cessons de gémir comme des colombes ; nous attendons le jugement, et il n'y en a point, la délivrance, et elle est éloignée de nous.
\VS{12}Car nos transgressions se sont multipliées devant toi, et chacun de nos péchés témoignent contre nous ; parce que nos transgressions sont avec nous, et nous connaissons nos iniquités ;
\VS{13}qui sont de pécher et de mentir contre Yahweh, de s'éloigner de notre Dieu, de proférer l'oppression et la révolte, de concevoir et prononcer du cœur des paroles de mensonge.
\VS{14}C'est pourquoi le jugement s'est éloigné et la justice se tient éloignée ; car la vérité est tombée par les rues, et la droiture ne peut y entrer.
\VS{15}Même la vérité a disparu, et quiconque se retire du mal est exposé au pillage ; Yahweh voit, et cela lui a déplu, parce qu'il n'y a plus de droiture.
\TextTitle{Yahweh cherche un homme, il suscite le Messie}
\VS{16}Il voit aussi qu'il n'y a aucun homme, il s'étonne que personne ne se tienne à la brèche ; c'est pourquoi son bras lui vient en aide, et sa propre justice lui sert d'appui\FTNT{Es. 53:1 ; Es. 63:5 ; Ps. 77:15-16 ; Ac. 13:17.}.
\VS{17}Car il se revêt de la justice comme d'une cuirasse, et le casque du salut est sur sa tête\FTNT{Ep. 6:14-17.} ; il se revêt de la vengeance comme d'un vêtement, et se couvre de la jalousie comme d'un manteau.
\VS{18}Selon leurs actes, il rendra à chacun la pareille\FTNT{Jé. 17:10 ; Job. 34:11 ; Mt. 16:27 ; Ap. 2:23 ; Ap. 20:13.}, la fureur à ses adversaires, la rétribution à ses ennemis ; il rendra ainsi la rétribution aux îles.
\VS{19}Et on craindra le Nom de Yahweh depuis l'occident, et sa gloire depuis le soleil levant ; car l'ennemi viendra comme un fleuve, mais l'Esprit de Yahweh lèvera la bannière\FTNT{En hébreu « Yahweh Nissi », c'est-à-dire « Yahweh est ma bannière ». C'est le nom donné par Moïse à l'autel qu'il construisit pour célébrer la défaite d'Amalek (Ex. 17:15). En No. 21:8-9, Moïse éleva une bannière sur laquelle il avait fixé un serpent d'airain pour la guérison des malades. } contre lui.
\VS{20}Et le Rédempteur\FTNT{Le Rédempteur qui viendra pour Sion est le Seigneur Jésus-Christ (Ro. 11:26). Voir aussi Es. 60 : 16. } viendra en Sion, et vers ceux de Jacob qui se convertiront de leur péché, dit Yahweh.
\VS{21}Et quant à moi, c'est ici mon alliance que je ferai avec eux, dit Yahweh : Mon Esprit qui est sur toi, et mes paroles que j'ai mises dans ta bouche, ne se retireront point de ta bouche, ni de la bouche de ta postérité, ni de la bouche de la postérité de ta postérité, dit Yahweh, dès maintenant et à jamais.
\Chap{60}
\TextTitle{La gloire de Yahweh se lèvera sa gloire sur Sion}
\VerseOne{}Lève-toi, sois illuminée, car ta lumière arrive, et la gloire de Yahweh se lève sur toi.
\VS{2}Car voici, les ténèbres couvrent la terre, et l'obscurité couvre les peuples ; mais Yahweh se lève sur toi, et sa gloire apparaît sur toi.
\VS{3}Des nations marchent à ta lumière, et des rois à la splendeur qui se lève sur toi\FTNT{Ap. 21:24.}.
\VS{4}Elève tes yeux alentour, et regarde : Tous ceux-ci s'assemblent, ils viennent vers toi ; tes fils viennent de loin, et tes filles sont nourries par des nourriciers, étant portées sur les côtés.
\VS{5}Alors tu verras et tu seras éclairée, et ton cœur s'étonnera et s'épanouira de joie, quand l'abondance de la mer se sera tournée vers toi, et que la puissance des nations sera venue chez toi.
\VS{6}Tu seras couverte d'une foule de chameaux, des dromadaires de Madian et d'Epha ; et tous ceux de Séba viendront, ils apporteront de l'or et de l'encens, et publieront les louanges de Yahweh.
\VS{7}Toutes les brebis de Kédar seront assemblées vers toi, les béliers de Nebajoth seront à ton service ; ils seront agréables étant offerts sur mon autel, et je rendrai magnifique la maison de ma gloire.
\VS{8}Qui sont ceux-là qui volent comme des nuées, comme des colombes vers leur colombier ?
\VS{9}Car les îles s'attendent à moi, et les navires de Tarsis les premiers, afin d'amener de loin tes enfants, avec leur argent et leur or, à cause du Nom de Yahweh, ton Dieu, et du Saint d'Israël qui te glorifie.
\VS{10}Les fils des étrangers rebâtiront tes murailles, et leurs rois seront employés à ton service ; car je t'ai frappée dans ma colère, mais j'ai eu pitié de toi au temps de mon bon plaisir.
\VS{11}Tes portes seront continuellement ouvertes, elles ne seront fermées ni nuit ni jour, afin que les forces des nations te soient amenées et que leur roi y soient conduits\FTNT{Ap. 21:25-26.}.
\VS{12}Car la nation et le royaume qui ne te serviront pas périront, et ces nations-là seront réduites en une entière désolation.
\VS{13}La gloire du Liban viendra vers toi, le cyprès, l'orme, et le buis, tous ensemble pour rendre honorable le lieu de mon sanctuaire ; et je rendrais glorieux le lieu de mes pieds.
\VS{14}Mais les enfants de tes oppresseurs viendront vers toi en se courbant, et tous ceux qui te méprisaient se prosterneront à tes pieds et t'appelleront la ville de Yahweh, la Sion du Saint d'Israël.
\VS{15}Au lieu d'avoir été délaissée et haïe, si bien que personne ne passait par toi, je te mettrai dans une élévation éternelle et dans une joie qui sera de génération en génération.
\VS{16}Et tu suceras le lait des nations, et tu suceras la mamelle des rois, et tu sauras que je suis Yahweh, ton Sauveur, ton Rédempteur\FTNT{Le verbe « ga'al » et le nom correspondant « go'el », ont été traduits respectivement en français par « racheter » et « rédempteur ». Selon la loi de Moïse, si quelqu'un perdait son héritage à cause d'une dette ou s'il se vendait comme esclave, lui et ses biens pouvaient être rachetés par un proche parent qui devait payer le prix de la rédemption (Lé. 25:23-55). Yahweh se présente comme le Rédempteur par excellence (Es. 49:26 ; Es. 60:16 ; Ps. 78:35 ; Ps. 130:7; Job. 19:25). Or Jésus-Christ «[…] a été fait pour nous sagesse, justice, sanctification et rédemption » (1 Co. 1:30). Les épîtres nous révèlent la rédemption qu'il a acquise pour nous : « nous avons la rédemption par son sang » (Ep. 1:7). La rédemption est le paiement d'une rançon, or il est écrit : « Jésus-Christ s'est donné en rançon pour nous tous » (1 Ti. 2:6). « Vous avez été rachetés à grand prix » (1 Co. 6:20).}, le Puissant de Jacob.
\VS{17}Je ferai venir de l'or au lieu de l'airain, et de l'argent au lieu du fer, et de l'airain au lieu du bois, et du fer au lieu des pierres ; et je ferai régner la paix et dominer la justice.
\VS{18}On n'entendra plus parler de violence dans ton pays ni de ravage et de ruine dans ton territoire ; mais tu appelleras tes murailles : Salut ; et tes portes : Louange.
\VS{19}Tu n'auras plus le soleil pour la lumière du jour, et la lueur de la lune ne t'éclairera plus, mais Yahweh sera pour toi la lumière éternelle\FTNT{Voir le commentaire en Ge 1:3.}, et ton Dieu sera ta gloire.
\VS{20}Ton soleil ne se couchera plus, et ta lune ne se retirera plus, car Yahweh te sera pour lumière perpétuelle, et les jours de ton deuil seront finis.
\VS{21}Quant à ton peuple, ils seront tous justes, ils posséderont la terre à toujours ; savoir le germe de mes plantes, l'œuvre de mes mains pour y être glorifié\FTNT{Es. 11:1 ; Ro. 15:12 ; Ap. 5:5 ; Ap. 22:16.}.
\VS{22}La petite famille deviendra un millier de personnes, et la moindre deviendra une nation puissante. Je suis Yahweh, je hâterai ces choses en leur temps.
\Chap{61}
\TextTitle{La mission du Messie}
\VerseOne{}L'Esprit du Seigneur Yahweh est sur moi, car Yahweh m'a oint pour évangéliser les malheureux ; il m'a envoyé pour guérir ceux qui ont le cœur brisé, pour proclamer aux captifs la liberté, et aux prisonniers l'ouverture de la prison ;
\VS{2}pour publier une année de grâce de Yahweh, et le jour de vengeance de notre Dieu ; pour consoler tous ceux qui mènent deuil\FTNT{Lu. 4:14-19.} ;
\VS{3}pour annoncer à ceux de Sion qui mènent deuil, que la magnificence leur sera donnée au lieu de la cendre, une huile de joie au lieu du deuil, un manteau de louange au lieu d'un esprit abattu\FTNT{Job. 29:14 ; Ja. 1:12 ; 1 Co.9:25 ; 2 Ti. 4:8.}, afin qu'on les appelle des térébinthes de la justice, une plantation de Yahweh, pour servir à sa gloire.
\VS{4}Et ils rebâtiront les ruines antiques, ils relèveront les lieux qui étaient auparavant désolés, et ils renouvelleront des villes ravagées, et les choses désolées d'âge en âge.
\VS{5}Et des étrangers s'y tiendront là et feront paître vos troupeaux, et les enfants de l'étranger seront vos laboureurs et vos vignerons.
\VS{6}Mais vous, vous serez appelés sacrificateurs de Yahweh, et on vous nommera serviteurs de notre Dieu\FTNT{Ap. 1:6 ; Ap. 5:10.} ; vous mangerez les richesses des nations, et vous vous glorifierez de leur gloire.
\VS{7}Au lieu de la honte que vous avez eue, les nations en auront le double, et elles crieront tout haut que la confusion est leur portion ; c'est pourquoi ils posséderont le double dans leur pays, et leur joie sera éternelle.
\VS{8}Car je suis Yahweh qui aime le jugement et qui hait la rapine pour l'holocauste ; j'établirai leur œuvre dans la vérité et je traiterai avec eux une alliance éternelle.
\VS{9}Et leur race sera connue parmi les nations, et ceux qui seront sortis d'eux seront connus parmi les peuples ; tous ceux qui les verront connaîtront qu'ils sont la race que Yahweh aura bénie.
\VS{10}Je me réjouirai extrêmement en Yahweh, et mon âme se réjouira en mon Dieu ; car il m'a revêtu des vêtements du salut, il m'a couvert du manteau de la justice, comme un époux qui se pare de magnificence, et comme une épouse qui s'orne de ses joyaux\FTNT{Os. 2:21-22 ; Ap. 19:7-8.}.
\VS{11}Car comme la terre fait éclore son germe, et comme un jardin fait germer ses semences, ainsi le Seigneur Yahweh fera germer la justice, et la louange en présence de toutes les nations.
\Chap{62}
\TextTitle{Yahweh proclamme la restauration d'Israël}
\VerseOne{}Pour l'amour de Sion, je ne me tiendrai pas tranquille, et pour l'amour de Jérusalem je ne prendrai point de repos, jusqu'à ce que sa justice sorte dehors comme une splendeur, et que sa délivrance ne soit allumée comme une lampe.
\VS{2}Alors les nations verront ta justice, et tous les rois ta gloire ; et on t'appellera d'un nouveau nom\FTNT{Ap. 2:17.}, que la bouche de Yahweh aura expressément déclaré.
\VS{3}Tu seras une couronne de gloire dans la main de Yahweh, un turban royal dans la main de ton Dieu.
\VS{4}On ne te nommera plus la délaissée, et on ne nommera plus ta terre la désolation ; mais on t'appellera mon bon plaisir en elle ; et on appellera ta terre l'épouse ; car Yahweh prend son bon plaisir en toi, et ta terre aura un époux.
\VS{5}Car comme le jeune homme épouse la vierge, comme tes enfants se marient chez toi, ainsi ton Dieu se réjouira en toi, de la joie qu'un époux a de son épouse.
\VS{6}Jérusalem, j'ai placé des gardes sur tes murailles tout le jour et toute la nuit, et ils ne se tairont point. Vous qui faites mention de Yahweh, ne gardez point le silence !
\VS{7}Et ne vous arrêtez pas de l'invoquer jusqu'à ce qu'il rétablisse Jérusalem et lui rende sa renommée sur la terre.
\VS{8}Yahweh l'a juré par sa droite et par son bras puissant : Je ne donnerai plus ton froment pour nourriture à tes ennemis, et les enfants des étrangers ne boiront plus ton vin excellent pour lequel tu as travaillé.
\VS{9}Mais ceux qui auront amassé le froment le mangeront et loueront Yahweh, et ceux qui auront récolté le vin le boiront dans les parvis de ma sainteté.
\VS{10}Passez, passez les portes ! Disant : Préparez le chemin du peuple ! Frayez, frayez la route, et ôtez-en les pierres ! Elevez une bannière vers les peuples.
\VS{11}Voici ce que Yahweh proclame aux extrémités de la terre : Dites à la fille de Sion : Voici, ton Sauveur vient\FTNT{De nombreux passages, notamment dans le livre d'Esaïe, présentent Dieu comme le sauveur, le seul sauveur (Es. 43:3 ; Es. 43:11 ; Os. 13:4) qui viendra pour délivrer son peuple (Es. 35:4 ; Es. 60:1 ; Za. 14:1-7). Jésus-Christ a accompli en tous points les prophéties relatives à la venue de Yahweh. Dieu est bel et bien venu sur terre il y a plus de 2000 ans et ce même Dieu revient bientôt (Ac. 1:11 ; Ap. 1:7).} ; voici, son salaire est avec lui, et sa récompense marche devant lui.
\VS{12}Et on les appellera le peuple saint, les rachetés de Yahweh\FTNT{1 Pi. 2:9 ; Ap. 5:9.} ; et toi, on t'appellera la recherchée, la ville non abandonnée.
\Chap{63}
\TextTitle{Le jour de vengeance du Messie\FTNTT{Es. 2:10-22 ; Ap. 19:11-21.}}
\VerseOne{}Qui est celui-ci qui vient d'Edom, de Botsra, en habits rouges, magnifiquement paré en son vêtement, marchant selon la grandeur de sa force ? C'est moi qui parle en justice et qui ai tout pouvoir de sauver.
\VS{2}Pourquoi tes vêtements sont-ils rouges, et pourquoi tes habits sont comme les habits de ceux qui foulent dans la cuve ?
\VS{3}J'ai été seul à fouler au pressoir, et nul homme d'entre les peuples n'était avec moi. Cependant, j'ai marché sur eux dans ma colère, et je les ai foulés dans ma fureur ; et leur sang a rejailli sur mes vêtements, et j'ai souillé tous mes habits.
\VS{4}Car le jour de la vengeance était dans mon cœur, et l'année de mes rachetés est venue.
\VS{5}Je regardais donc, il n'y avait personne pour m'aider ; et j'étais étonné, et  il n'y avait personne pour me soutenir ; mais mon bras m'a sauvé et ma fureur m'a soutenu.
\VS{6}Ainsi j'ai foulé des peuples dans ma colère, et je les ai enivrés dans ma fureur ; et j'ai abattu leur force par terre.
\TextTitle{Esaïe confesse les péchés du peuple}
\VS{7}Je ferai mention des bontés de Yahweh, qui sont les louanges de Yahweh, pour tous les bienfaits que Yahweh nous a faits ; car grande est la bonté envers la maison d'Israël, qu'il a traitée selon ses compassions et la richesse de sa miséricorde.
\VS{8}Car il a dit : Certainement, ils sont mon peuple, des enfants qui ne tricheront pas ! Et il a été pour eux un Sauveur.
\VS{9}Et dans toutes leurs détresses, il a été en détresse, et l'ange qui est devant sa face les a délivrés\FTNT{Ge. 16:7-10 ; Jg. 6:11-14 ; Za.1:11.} ; lui-même les a rachetés dans son amour et sa miséricorde, et il les a soutenus et portés, tous les jours d'autrefois.
\VS{10}Mais ils ont été rebelles, et ils ont attristé son Esprit saint\FTNT{Ep. 4:30.}, c'est pourquoi il est devenu leur ennemi, et il a lui-même combattu contre eux.
\VS{11}Et on se souvint des anciens jours de Moïse et de son peuple. Où est celui, a-t-on dit, qui les fit monter de la mer, avec les pasteurs de son troupeau ? Où est celui qui mit au milieu d'eux son Esprit saint ;
\VS{12}qui les dirigea par la droite de Moïse et par son bras glorieux ; qui fendit les eaux devant eux pour se faire un nom éternel ?
\VS{13}Qui les dirigea à travers les flots, comme un cheval dans le désert, sans qu'ils ne bronchent ?
\VS{14}L'Esprit de Yahweh les a menés au repos comme on mène une bête qui descend dans la vallée. C'est ainsi que tu as conduit ton peuple, afin de t'acquérir un nom glorieux.
\VS{15}Regarde du ciel et vois de ta demeure sainte et glorieuse : Où sont ton zèle et ta puissance ? Le son de tes entrailles et de tes compassions se retiennent-ils envers moi ?
\VS{16}Certes tu es notre Père, encore qu'Abraham ne nous connaisse pas, et qu'Israël ne nous reconnaisse pas ; Yahweh, c'est toi qui es notre Père, et ton Nom est notre Rédempteur de tout temps.
\VS{17}Pourquoi nous as-tu fait égarer loin de tes voies, ô Yahweh, et endurcis-tu notre cœur contre ta crainte ? Reviens, pour l'amour de tes serviteurs, des tribus de ton héritage !
\VS{18}Ton peuple saint n'a possédé le pays que peu de temps ; nos ennemis ont foulé ton sanctuaire.
\TextTitle{Prière du reste d'Israël à Yahweh pour sa délivrance}
\VS{19}Nous sommes comme ceux sur lesquels tu ne domines pas depuis longtemps, et sur lesquels ton Nom n'est point réclamé. Ô ! Si tu fendais les cieux, et si tu descendais, les montagnes s'ébranleraient devant toi !
\Chap{64}
\VerseOne{}Comme un feu de fonte est ardent, le feu fait bouillir l'eau, afin de faire connaître ton Nom à tes ennemis, et que les nations tremblent en ta présence.
\VS{2}Lorsque tu fis les choses redoutables que nous n'attendions pas, tu descendis et les montagnes tremblèrent devant toi.
\VS{3}Jamais on n'a appris ni entendu dire, et jamais l'œil n'a vu qu'un autre dieu que toi fît de telles choses pour ceux qui s'attendent à lui\FTNT{1 Co. 2:9.}.
\VS{4}Tu viens à la rencontre de celui qui se réjouit et qui agit avec justice, et  se souviennent de toi dans tes voies. Voici tu as été irrité parce que nous avons péché ; tes compassions sont éternelles, c'est pourquoi nous serons sauvés.
\VS{5}Or nous sommes tous devenus comme une chose souillée, et toute notre justice est comme le linge le plus souillé\FTNT{Ap. 19:8.} ; nous sommes tous flétris comme la feuille, et nos iniquités nous emportent comme le vent.
\VS{6}Il n'y a personne qui invoque ton Nom, qui se réveille pour s'attacher fortement à toi ; c'est pourquoi tu nous as caché ta face, et tu nous fais fondre par l'effet de nos iniquités.
\VS{7}Cependant, ô Yahweh, tu es notre Père ; nous sommes l'argile, et c'est toi qui nous as formés, et nous sommes tous l'ouvrage de ta main\FTNT{Es. 29:16 ; Es. 45:9 ; Jé. 18:6 ; Ro. 9:20-21.}.
\VS{8}Ne t'irrite pas à l'extrême, ô Yahweh, et ne te souviens pas à toujours de notre iniquité. Voici, regarde, nous te prions, nous sommes tous ton peuple.
\VS{9}Tes villes saintes sont devenues un désert ; Sion est devenue un désert, et Jérusalem une désolation.
\VS{10}Notre maison sainte et glorieuse, où nos pères te louaient, a été brûlée par le feu ; tout ce que nous avions de précieux a été dévasté.
\VS{11}Après cela, ô Yahweh, ne te retiendras-tu pas ? Ne cesseras-tu pas, et nous affligeras-tu à l'excès ?
\Chap{65}
\TextTitle{Réponse de Yahweh}
\VerseOne{}Je me suis fait recherché de ceux qui ne me demandaient point, et je me suis laissé trouver par ceux qui ne me cherchaient pas\FTNT{Mt. 7:7 ; Lu. 11:9.} ; j'ai dit à la nation qui ne s'appelait pas de mon Nom : Me voici, me voici !
\VS{2}J'ai tendu mes mains tous les jours vers un peuple rebelle, à ceux qui marche dans une mauvaise voie, au gré de ses pensées ;
\VS{3}vers un peuple qui m'irrite continuellement en face, qui sacrifie dans les jardins, et qui fait des parfums sur les autels de briques,
\VS{4}qui habite les sépulcres et passe la nuit dans les lieux désolés, qui mangent la chair de porc, et ayant dans ses vases le jus des choses abominables.
\VS{5}Qui dit : Retire-toi, ne m'approche pas, car je suis plus saint que toi ! Ceux-là sont une fumée dans mes narines, un feu ardent tout le jour.
\VS{6}Voici, ceci est écrit devant moi, je ne me tairai point, mais je leur ferai porter la peine, oui je leur ferai porter la peine
\VS{7}de vos iniquités, dit Yahweh, et les iniquités de vos pères ensemble, qui ont brûlé de l'encens sur les montagnes, et qui m'ont blasphémé sur les collines ; c'est pourquoi je leur mesurerai aussi dans leur sein le salaire de ce qu'ils ont fait au commencement.
\VS{8}Ainsi parle Yahweh : Comme quand on trouve du vin dans une grappe, on dit : Ne la détruis pas, car il y a là une bénédiction ! J'agirai de même à cause de mes serviteurs, afin de ne pas tous les détruire.
\VS{9}Je ferai sortir de Jacob une postérité, et de Juda celui qui héritera de mes montagnes ; et mes élus hériteront le pays, et mes serviteurs y habiteront.
\VS{10}Et Saron servira de pâturage au menu bétail, et la vallée d'Acor sera le gîte du gros bétail, pour mon peuple qui m'aura recherché.
\VS{11}Mais vous, qui abandonnez Yahweh et qui oubliez ma montagne sainte, qui dressez la table pour Gad\FTNT{Gad : Dieu de la fortune.}, et qui remplissez une coupe pour Meni\FTNT{Meni : divinité païenne assimilée à la lune et dont le nom signifie « destin, sort ou fortune ».},
\VS{12}je vous destine aussi à l'épée, et vous serez tous courbés pour être égorgés ; parce que j'ai appelé, et vous n'avez point répondu ; j'ai parlé, et vous n'avez point écouté ; mais vous avez fait ce qui me déplaît, et vous avez choisi les choses auxquelles je ne prends pas plaisir.
\VS{13}C'est pourquoi, ainsi parle le Seigneur, Yahweh : Voici, mes serviteurs mangeront, et vous aurez faim ; voici, mes serviteurs boiront, et vous aurez soif ; voici mes serviteurs se réjouiront, et vous serez honteux.
\VS{14}Voici, mes serviteurs se réjouiront avec chants de triomphe pour la joie qu'ils auront au cœur ; mais vous, vous crierez pour la douleur que vous aurez au cœur, et vous crierez à cause de l'accablement de votre esprit.
\VS{15}Et vous laisserez votre nom à mes élus comme malédiction ; et le Seigneur Yahweh vous fera mourir ; et il donnera à ses serviteurs un autre nom.
\VS{16}Celui qui se bénira sur la terre, se bénira par le Dieu de vérité ; et celui qui jurera sur la terre jurera par le Dieu de vérité ; car les détresses du passé seront oubliées, et même elles seront cachées devant mes yeux.
\TextTitle{De nouveaux cieux et une nouvelle terre}
\VS{17}Car voici, je vais créer de nouveaux cieux et une nouvelle terre\FTNT{Es. 66:22 ; 2 Pi. 3:13 ; Ap. 21:1.} ; et on ne se souviendra plus des choses précédentes, elles ne reviendront plus au cœur.
\VS{18}Réjouissez-vous plutôt et soyez à toujours dans l'allégresse, à cause de ce que je vais créer ; car voici je vais créer Jérusalem pour n'être que joie, et son peuple pour n'être qu'allégresse.
\VS{19}Je ferai de Jérusalem mon allégresse, et de mon peuple ma joie ; on n'y entendra plus le bruit des pleurs et le bruit des clameurs.
\VS{20}Il n'y aura plus désormais ni nourrisson ni vieillard qui n'accomplissent leurs jours ; car celui qui mourra âgé de cent ans sera encore jeune ; mais le pécheur âgé de cent ans sera maudit.
\VS{21}Ils bâtiront des maisons et y habiteront ; ils planteront des vignes et ils en mangeront le fruit.
\VS{22}Ils ne bâtiront pas des maisons pour qu'un autre y habite ; ils ne planteront pas des vignes pour qu'un autre en mange le fruit ; car les jours de mon peuple seront comme les jours des arbres ; et mes élus jouiront de l'œuvre de leurs mains.
\VS{23}Ils ne travailleront plus en vain, et ils n'engendreront plus des enfants pour être exposés à la frayeur ; car ils seront la postérité des bénis de Yahweh, et ceux qui sortiront d'eux seront avec eux.
\VS{24}Et il arrivera qu'avant qu'ils crient, je les exaucerai ; et lorsqu'encore ils parleront, je les aurai déjà entendus.
\VS{25}Le loup et l'agneau paîtront ensemble, le lion comme le bœuf mangeront de la paille, et la poussière sera la nourriture du serpent\FTNT{Es. 2:4 ; Es. 11:6-7.}. On ne nuira point et on ne fera aucun dommage sur toute ma montagne sainte, dit Yahweh.
\Chap{66}
\TextTitle{Yahweh réprouve l'hypocrisie et agrée ceux qui le craignent}
\VerseOne{}Ainsi parle Yahweh : Le ciel est mon trône, et la terre est le marchepied de mes pieds\FTNT{Mt. 5:34-35 ; Ac. 7:49.}. Quelle maison me bâtiriez-vous, et quel serait le lieu de mon repos ?
\VS{2}Car ma main a fait toutes ces choses, et c'est par moi que toutes ces choses ont eu leur être, dit Yahweh. Mais à qui regarderai-je ? A celui qui est affligé, qui a l'esprit abattu, et qui tremble à ma parole.
\VS{3}Celui qui égorge un bœuf est comme celui qui tuerait un homme ; celui qui sacrifie une brebis est comme celui qui romprait la nuque à un chien ; celui qui présente une offrande est comme celui qui offrirait le sang d'un pourceau ; celui qui fait un parfum d'encens est comme celui qui bénirait une idole ; tous ceux-là ont choisi leurs voies, et leur âme trouve du plaisir dans leurs abominations.
\VS{4}Moi aussi je ferai attention à leurs tromperies, et je ferai venir sur eux les choses qu'ils craignent ; parce que j'ai appelé, et personne n'a répondu, parce que j'ai parlé, et qu'ils n'ont point écouté ; mais ils ont fait ce qui est mal à mes yeux, et ils ont choisi les choses auxquelles je ne prends pas de plaisir. 
\VS{5}Ecoutez la parole de Yahweh, vous qui tremblez à sa parole ; vos frères, qui vous haïssent et qui vous repoussent comme une chose abominable, à cause de mon Nom disent : Que Yahweh montre sa gloire ! Il sera donc vu à votre joie mais eux seront honteux. 
\VS{6}Un son éclatant sort de la ville, un son sort du temple, le son de Yahweh, qui rend à ses ennemis selon leurs œuvres.
\TextTitle{Israël renaît en un jour}
\VS{7}Elle a enfanté, avant d'éprouver les douleurs de l'enfantement ; elle a donné naissance à un enfant mâle, avant que les souffrances lui viennent.
\VS{8}Qui a jamais entendu une telle chose ? Qui en a jamais vu de semblable ? Ferait-on qu'un pays naisse en un jour ? Ou une nation naîtrait-elle d'un seul coup\FTNT{Cette prophétie fait allusion à la création de l'Etat d'Israël le 14 mai 1948.} ? Car dès que Sion a été en travail, elle a enfanté ses enfants !
\VS{9}Moi qui fais enfanter les autres, ne ferais-je point enfanter Sion ? Dit Yahweh. Moi qui donne de la postérité aux autres, l'empêcherais-je d'enfanter ? Dit ton Dieu.
\TextTitle{Réjouissance à Jérusalem et consolation}
\VS{10}Réjouissez-vous avec Jérusalem, faites d'elle le sujet de votre allégresse, vous tous qui l'aimez ; vous tous qui menez deuil sur elle, réjouissez-vous avec elle d'une grande joie ;
\VS{11}afin que vous soyez allaités et rassasiés de la mamelle de ses consolations, afin que vous suciez le lait et que vous jouissiez à plaisir de la plénitude de sa gloire.
\VS{12}Car ainsi parle Yahweh : Voici, je ferai couler vers elle la paix comme un fleuve, et la gloire des nations comme un torrent débordé, et vous serez allaités, vous serez portés sur les côtés et caressés sur les genoux.
\VS{13}Je vous consolerai pour vous apaiser, comme quelqu'un que sa mère caresse pour l'apaiser, vous serez consolés dans Jérusalem.
\VS{14}Vous le verrez et votre cœur se réjouira, et vos os germeront comme l'herbe ; et la main de Yahweh sera connue de ses serviteurs ; mais il sera indigné contre ses ennemis.
\TextTitle{Jugement de Yahweh}
\VS{15}Car voici, Yahweh viendra avec le feu, et ses chars seront comme la tempête ; afin qu'il tourne sa colère en fureur, et sa menace en flamme de feu.
\VS{16}Car Yahweh exercera jugement contre toute chair par le feu et avec son épée ; et le nombre de ceux qui seront mis à mort par Yahweh sera grand.
\VS{17}Ceux qui se sanctifient et se purifient au milieu des jardins, l'un après l'autre, qui mangent de la chair de porc et des choses abominables, comme des souris, seront ensemble consumés, dit Yahweh.
\VS{18}Mais pour moi, voyant leurs œuvres et leurs pensées, le temps est venu de rassembler toutes les nations et les langues ; ils viendront et verront ma gloire.
\TextTitle{Toutes les nations adoreront Yahweh}
\VS{19}Car je mettrai un signe en eux, et j'enverrai ceux d'entre eux qui seront réchappés, vers les nations, à Tarsis, à Pul, à Lud, gens tirant de l'arc, à Tubal et à Javan, et vers les îles lointaines, qui n'ont point entendu ma renommée, et qui n'ont pas vu ma gloire ; et ils annonceront ma gloire parmi les nations.
\VS{20}Et ils amèneront tous vos frères d'entre toutes les nations, sur des chevaux, sur des chars et dans des litières, sur des mulets et sur des dromadaires, en offrande à Yahweh, à la montagne sainte, à Jérusalem, dit Yahweh, comme lorsque les enfants d'Israël apportent l'offrande dans un vase pur, à la maison de Yahweh.
\VS{21}Et même je prendrai aussi parmi eux des sacrificateurs, des Lévites, dit Yahweh.
\VS{22}Car comme les nouveaux cieux et la nouvelle terre que je vais faire subsisteront devant moi, dit Yahweh, ainsi subsistera votre postérité et votre nom.
\VS{23}Et il arrivera que de nouvelle lune en nouvelle lune, et de sabbat en sabbat, toute chair viendra se prosterner devant ma face, dit Yahweh.
\VS{24}Et quand ils sortiront dehors, ils verront les cadavres des hommes qui se sont rebellés contre moi ; car leur ver ne mourra point, et leur feu ne s'éteindra point\FTNT{Mc. 9:48.} ; et ils seront méprisés de tout le monde.
\PPE{}
\end{multicols}

%\clearpage\ShortTitle{Jérémie}\BookTitle{Jérémie}\BFont
\noindent\hrulefill
{\footnotesize
\textit{
\bigskip
{\centering{}
\\Auteur : Jérémie
\\(Heb. : Yirmeyah)
\\Signification : Celui que Yahweh a désigné
\\Thème : Avertissements et jugements
\\Date de rédaction : 7\up{ème} siècle av J.C\\}
}
%\bigskip
\textit{
\\Issu d'une famille de sacrificateurs, Jérémie fut appelé dès son plus jeune âge au service de Yahweh et exerça un ministère prophétique avant et pendant les premières années de déportation. Outre son message à Israël et aux nations, le livre de Jérémie révèle sa personnalité. On découvre alors que l'opposition de ses pairs fut l'une de ses expériences les plus douloureuses. En effet, ce récit raconte ses combats contre les faux prophètes et met en évidence les signes accompagnant les prophètes authentiques, à savoir la souffrance, la solitude, l'incompréhension et le rejet.
%\bigskip
\\Son message annonçait le jugement imminent de Dieu et invitait le peuple à la repentance pour éviter le châtiment de Yahweh. Après la chute de Jérusalem, alors que Nebucadnetsar lui avait laissé le choix, Jérémie décida de rester avec les plus pauvres plutôt que de partir pour Babylone. Cependant, des Israélites décidèrent de s'expatrier en Egypte et l'entraînèrent avec eux de force. En terre étrangère, Jérémie continua de porter le fardeau de son peuple, l'exhortant à réformer ses voies. 
%\bigskip
\\Parmi les prophéties de Jérémie, figure le retour du peuple d'Israël sur la terre promise avant la seconde venue de Christ.\bigskip
}
}
\par\nobreak\noindent\hrulefill
\begin{multicols}{2}
\Chap{1}
\TextTitle{Yahweh appelle Jérémie à son service}
\VerseOne{}Les Paroles de Jérémie, fils de Hilkija, d'entre les sacrificateurs qui étaient à Anathoth, dans le pays de Benjamin;
\VS{2}auquel fut adressée la parole de Yahweh aux jours de Josias, fils d'Amon, roi de Juda, la treizième année de son règne,
\VS{3}laquelle lui fut aussi adressée aux jours de Jojakim, fils de Josias, roi de Juda, jusqu'à la fin de la onzième année de Sédécias, fils de Josias, roi de Juda; savoir jusqu'au temps où  Jérusalem fut transportée, ce qui arriva au cinquième mois.
\VS{4}La parole de Yahweh me fut adressée, en disant :
\VS{5}Avant que je t'aie formé dans le ventre de ta mère, je te connaissais, et avant que tu sois sorti de son sein, je t'avais consacré, je t'avais établi prophète pour les nations\FTNT{Es. 49:5 ; Ga. 1:15.}.
\VS{6}Je répondis : Ah ! Seigneur Yahweh ! Voici, je ne sais pas parler, car je suis un enfant\FTNT{Ex. 4:10-11.}.
\VS{7}Et Yahweh me dit : Ne dis pas : Je suis un enfant. Car tu iras partout où je t'enverrai, et tu diras tout ce que je t'ordonnerai.
\VS{8}Ne crains pas de te montrer devant eux, car je suis avec toi pour te délivrer, dit Yahweh.
\VS{9}Puis Yahweh avança sa main et toucha ma bouche ; et Yahweh me dit : Voici, je mets mes paroles dans ta bouche.
\VS{10}Regarde, je t'établis aujourd'hui sur les nations et sur les royaumes, pour que tu arraches et que tu démolisses, pour que tu ruines et que tu détruises, pour que tu bâtisses et que tu plantes\FTNT{Jérémie devait d'abord arracher, démolir, ruiner et détruire avant de bâtir et de planter. Il y avait dans le temple de Jérusalem les autels de Baal et le pieu d'Asherah (2 R. 21). De même, avant de planter la Parole de Dieu qui est une semence plantée dans les cœurs (Mc. 4 : 3-17), il est nécessaire au préalable d'arracher et de renverser les fausses doctrines et le péché en les dénonçant.}.
\TextTitle{Yahweh confirme la mission de Jérémie et l'établit sur Juda}
\VS{11}Puis la parole de Yahweh me fut adressée, en disant : Que vois-tu, Jérémie ? Et je répondis : Je vois une branche d'amandier.
\VS{12}Et Yahweh me dit : Tu as bien vu ; car je me hâte d'exécuter ma parole.
\VS{13}La parole de Yahweh me fut adressée pour la seconde fois, en disant : Que vois-tu ? Et je répondis : Je vois un pot bouillant dont le devant est tourné vers le nord.
\VS{14}Et Yahweh me dit : le mal se découvrira du côté du nord sur tous les habitants de ce pays-ci.
\VS{15}Car voici, je vais appeler toutes les familles des royaumes du nord, dit Yahweh ; elles viendront et mettront chacune leur trône à l'entrée des portes de Jérusalem, contre toutes ses murailles à l'entour, et contre toutes les villes de Juda.
\VS{16}Et je prononcerai mes jugements contre eux, à cause de toute leur méchanceté, par laquelle ils m'ont délaissé, et ont fait des parfums à d'autres dieux, et se sont prosternés devant l'ouvrage de leurs mains. 
\VS{17}Toi donc, ceins tes reins, lève-toi, et dis-leur tout ce que je t'ordonnerai. Ne crains pas de te montrer devant eux, de peur que je ne te mette en pièces en leur présence.
\VS{18}Car voici, je t'établis aujourd'hui sur tout le pays comme une ville forte, une colonne de fer, et un mur d'airain, contre les rois de Juda, contre les chefs du pays, contre ses sacrificateurs, et contre le peuple du pays.
\VS{19}Et ils combattront contre toi, mais ils ne seront pas plus forts que toi ; car je suis avec toi, dit Yahweh, pour te délivrer.
\Chap{2}
\TextTitle{Yahweh dénonce l'attitude d'Israël et l'avertit}
\VerseOne{}La parole de Yahweh me fut adressée, en disant :
\VS{2}Va et crie aux oreilles de Jérusalem, et dis : Ainsi parle Yahweh : Je me souviens de la fidélité de ta jeunesse, de l'amour de tes fiançailles, quand tu me suivais au désert, dans une terre qu'on n'ensemence pas. 
\VS{3}Israël était une chose sainte à Yahweh, il était les prémices de son revenu\FTNT{Lé. 23:20 ; Pr. 3:9 ; Né. 10:35.}; tous ceux qui le dévoraient étaient coupables, il leur en arrivait du mal dit Yahweh.
\VS{4}Ecoutez la parole de Yahweh, maison de Jacob, et vous toutes, familles de la maison d'Israël !
\VS{5}Ainsi parle Yahweh : Quelle iniquité vos pères ont-ils trouvée en moi, pour qu’ils se soient éloignés de moi, et qu’ils aient marché après la vanité et soient devenus vains ?
\VS{6}Ils n'ont pas dit : Où est Yahweh qui nous a fait remonter du pays d'Egypte, qui nous a conduits par un désert, par un pays de landes et montagneux, par un pays aride et d'ombre de mort, par un pays où aucun homme n'avait passé, et où personne n'avait habité ? 
\VS{7}Je vous ai fait entrer dans un pays de verger, pour que vous en mangiez les fruits et les biens ; mais sitôt vous y êtes entrés, vous avez souillé mon pays, et vous avez rendu abominable mon héritage.
\VS{8}Les sacrificateurs n'ont pas dit : Où est Yahweh ? Les dépositaires de la loi ne m'ont pas connu, les pasteurs se sont révoltés contre moi, les prophètes ont prophétisé par Baal\FTNT{Baal. Voir Jg. 2:13.}, et sont allés après ce qui n'est d'aucun profit.
\VS{9}A cause de cela, je veux encore contester avec vous, dit Yahweh, je veux contester avec les fils de vos fils.
\VS{10}Passez par les îles de Kittim et voyez ! Envoyez quelqu'un à Kédar ; observez bien, et voyez s'il n'y a rien de semblable !
\VS{11}Y a-t-il une nation qui change ses dieux, quoiqu'ils ne soient pas des dieux ? Et mon peuple a changé sa gloire contre ce qui n'est d'aucun profit\FTNT{Ro. 1:23.} !
\VS{12}Cieux, soyez étonnés de cela ; frémissez d'horreur et soyez stupéfaits ! dit Yahweh.
\VS{13}Car mon peuple a commis doublement le mal : Ils m'ont abandonné, moi qui suis la source d'eaux vives\FTNT{Yahweh est la Source d'eaux vives. Jésus-Christ se présente aussi comme la Source d'eau vive (Jn. 4:13-14 ; Ap. 21:6).}, pour se creuser des citernes, des citernes crevassées qui ne peuvent pas retenir l'eau.
\VS{14}Israël est-il un esclave, ou un esclave né dans la maison ? Pourquoi donc est-il mis au pillage ?
\VS{15}Les lionceaux rugissent, poussent leurs cris contre lui, et ils mettent son pays en désolation ; ses villes sont brûlées, de sorte que personne n'y habite.
\VS{16}Même les fils de Noph et de Tachpanès te casseront le sommet de la tête.
\VS{17}Cela ne t'arrive-t-il pas parce que tu as abandonné Yahweh, ton Dieu, à l'époque où il te conduisait par le chemin ?
\VS{18}Et maintenant, qu'as-tu à faire d'aller en Egypte, pour boire l'eau du Schichor\FTNT{Schichor : Sombre, noir, boueux. Le Nil, une rivière ou un canal affluent du fleuve. Les Israélites préféraient ces eaux à Yahweh.} ? Qu'as-tu à faire d'aller en Assyrie, pour boire l'eau du fleuve ?
\VS{19}Ta méchanceté te châtiera, et tes débauches te jugeront, tu sauras et tu verras que c'est une chose mauvaise et amère d'abandonner Yahweh, ton Dieu, et de n'avoir de moi aucune crainte, dit le Seigneur, Yahweh des armées.
\VS{20}Tu as dès longtemps brisé ton joug, rompu tes liens, et tu as dit : Je ne veux plus être dans la servitude ! Mais sur toute haute colline et sous tout arbre vert tu t'es incliné, tu t'es prostitué.
\VS{21}Je t'avais moi-même plantée comme une vigne exquise, dont tout le plant était franc ; comment t'es-tu changée en sarments d'une vigne étrangère ?
\VS{22}Quand tu te laverais avec du nitre, et que tu prendrais beaucoup de savon, ton iniquité resterait encore marquée devant moi, dit le Seigneur, Yahweh.
\VS{23}Comment dirais-tu : Je ne me suis pas souillée, je ne suis pas allée après les Baals ? Regarde tes pas dans la vallée, reconnais ce que tu as fait, dromadaire à la course légère et vagabonde !
\VS{24}Anesse sauvage, accoutumée au désert, humant le vent à son plaisir. Qui l'arrêtera dans son ardeur ? Tous ceux qui la cherchent n'ont pas à se fatiguer ; ils la trouvent pendant son mois.
\VS{25}Garde ton pied de se déchausser, ton gosier d'avoir soif ! Mais tu dis : C'est en vain, non ! Car j'aime les dieux étrangers, et j'irai après eux.
\VS{26}Comme un voleur est confus quand il est surpris, ainsi seront confus ceux de la maison d'Israël, eux, leurs rois, leurs chefs, leurs sacrificateurs et leurs prophètes.
\VS{27}Ils disent au bois : Tu es mon père ! Et à la pierre : Tu m'as enfanté ! Car ils me tournent le dos, et non la face. Et ils disent dans le temps de leur malheur : Lève-toi, et sauve-nous !
\VS{28}Où donc sont tes dieux que tu t'es faits ? Qu'ils se lèvent, s'ils peuvent te sauver au temps de ton malheur ! Car tu as autant de dieux que de villes, ô Juda !
\VS{29}Pourquoi contesteriez-vous avec moi ? Vous vous êtes tous rebellés contre moi, dit Yahweh.
\VS{30}En vain ai-je frappé vos fils ; ils n'ont pas reçu d'instruction ; votre épée a dévoré vos prophètes comme un lion destructeur.
\VS{31}Hommes de cette génération, considérez la parole de Yahweh ! Ai-je été un désert pour Israël, ou un pays de ténèbres ? Pourquoi mon peuple dit-il : Nous sommes libres, nous ne viendrons plus à toi ?
\VS{32}La vierge oublie-t-elle ses ornements, la fiancée sa ceinture ? Mais mon peuple m'a oublié depuis des jours sans nombre.
\VS{33}Comme tu es habile dans tes voies pour chercher ce que tu aimes ! C'est pourquoi aussi tu accoutumes tes voies aux crimes.
\VS{34}Même sur les pans de ta robe se trouve le sang des pauvres, des innocents que tu n'as pas trouvés en effraction.
\VS{35}Malgré cela, tu dis : Oui, je suis innocent ! Certainement sa colère s'est détournée de moi ! Voici, je vais entrer en jugement avec toi, sur ce que tu as dit : Je n'ai pas péché.
\VS{36}Pourquoi tant te précipiter pour changer ton chemin ? Tu auras autant de confusion de l'Egypte que tu en as eu de l'Assyrie.
\VS{37}Tu sortiras même d'ici, ayant tes mains sur la tête ; car Yahweh rejette ceux en qui tu te confies, et tu n'auras aucune prospérité par eux.
\Chap{3}
\TextTitle{Israël comparé à une prostituée}
\VerseOne{}Il dit : Si un homme répudie sa femme, qu'elle le quitte et se joigne à un autre, cet homme retourne-t-il encore vers elle\FTNT{Lé. 21:7 ; De. 24:2.} ? Le pays même n'en serait-il pas entièrement souillé ? Or toi, tu t'es prostituée à plusieurs amants, et tu reviendrais à moi ! dit Yahweh.
\VS{2}Lève tes yeux vers les lieux élevés et regarde ! Où ne t'es-tu pas prostituée ! Tu te tenais sur les chemins, comme un Arabe dans le désert, et tu as souillé le pays par tes prostitutions et par ta méchanceté.
\VS{3}Aussi les pluies ont été retenues, et il n'y a pas eu de pluie de l'arrière-saison ; mais tu as eu le front d'une femme prostituée, tu n'as pas voulu avoir honte.
\VS{4}Maintenant, n'est-ce pas ? Tu cries vers moi : Mon père ! Tu as été l'ami de ma jeunesse !
\VS{5}Gardera-t-il à toujours sa colère ? La conservera-t-il à jamais\FTNT{Es. 57:16 ; Ps.103:9.} ? Voici, tu as ainsi parlé, tu as fait ces maux-là autant que tu as pu.
\TextTitle{Yahweh appelle Israël à la repentance}
\VS{6}Yahweh me dit au temps du roi Josias : As-tu vu ce qu'a fait Israël, l'infidèle ? Elle est allée sur toute haute colline et sous tout arbre vert, et elle s'y est prostituée.
\VS{7}Je disais : Après avoir fait toutes ces choses, elle reviendra à moi. Mais elle n'est pas revenue. Et sa sœur Juda, la perfide, l'a vu.
\VS{8}Quoique j'aie répudié Israël, l'infidèle, à cause de tous ses adultères, et que je lui aie donné sa lettre de divorce, j'ai vu que la perfide Juda, sa sœur, n'a pas eu de crainte, mais elle s'en est allée et s'est aussi prostituée.
\VS{9}Par le bruit de sa prostitution, elle a souillé le pays, elle a commis un adultère avec la pierre et le bois.
\VS{10}Malgré tout cela, sa sœur Juda, la perfide, n'est pas revenue à moi de tout son cœur ; c'est avec fausseté qu'elle l'a fait, dit Yahweh.
\VS{11}Et Yahweh me dit : Israël, l'infidèle, se montre plus juste que Juda, la perfide.
\VS{12}Va, crie ces paroles vers le nord, et dis : Reviens, Israël, l'infidèle, dit Yahweh. Je ne jetterai pas sur vous un regard sévère ; car je suis miséricordieux, dit Yahweh, je ne garde pas ma colère à toujours.
\VS{13}Reconnais seulement ton iniquité, que tu t'es rebellée contre Yahweh, ton Dieu, que tu as tourné çà et là tes pas vers les étrangers, sous tout arbre vert, et que tu n'as pas écouté ma voix, dit Yahweh.
\VS{14}Fils rebelles, convertissez-vous, dit Yahweh, car je suis votre maître. Je vous prendrai, un d'une ville, deux d'une famille, et je vous ferai entrer dans Sion.
\VS{15}Je vous donnerai des pasteurs selon mon cœur, qui vous paîtront avec intelligence et avec sagesse\FTNT{Jé. 23:5.}.
\VS{16}Lorsque vous aurez multiplié et fructifié dans le pays, en ces jours-là, dit Yahweh, on ne parlera plus de l'arche de l'alliance de Yahweh, elle ne viendra plus à la pensée ; on ne s'en souviendra plus, on ne s'apercevra plus de son absence, et l'on n'en fera pas une autre.
\VS{17}En ce temps-là, on appellera Jérusalem le trône de Yahweh ; toutes les nations s'assembleront à Jérusalem, au Nom de Yahweh, et elles ne marcheront plus suivant les penchants de leur mauvais cœur.
\VS{18}En ces jours-là, la maison de Juda marchera avec la maison d'Israël ; elles viendront ensemble du pays du nord au pays que j'ai donné en héritage à vos pères.
\VS{19}Je disais : Comment te mettrai-je parmi mes fils et te donnerai-je un pays désirable, le plus bel héritage des armées des nations ? Je disais : Tu m'appelleras : Mon père ! Et tu ne te détourneras pas de moi.
\VS{20}Mais, comme une femme est infidèle à son compagnon, ainsi vous m'avez été infidèles, maison d'Israël, dit Yahweh.
\VS{21}Une voix se fait entendre sur les lieux élevés ; ce sont les pleurs, les supplications des fils d'Israël ; car ils ont perverti leur voie, ils ont oublié Yahweh, leur Dieu.
\VS{22}Fils rebelles, convertissez-vous, je guérirai vos infidélités. Nous voici, nous venons à toi, car tu es Yahweh, notre Dieu.
\VS{23}Certainement, on s'attend en vain aux collines et à la multitude des montagnes ; mais c'est en Yahweh, notre Dieu, qu'est la délivrance d'Israël.
\VS{24}Car la honte a dévoré dès notre jeunesse le travail de nos pères, leurs brebis et leurs bœufs, leurs fils et leurs filles.
\VS{25}Nous serons gisants dans notre honte, et notre ignominie nous couvrira ; parce que nous avons péché contre Yahweh, notre Dieu, nous et nos pères, dès notre jeunesse jusqu'à ce jour, et nous n'avons pas obéi à la voix de Yahweh, notre Dieu.
\Chap{4}
\TextTitle{Prophétie sur l'invasion du pays}
\VerseOne{}Israël, si tu reviens, dit Yahweh, si tu reviens à moi, si tu ôtes tes abominations de devant moi, tu ne seras plus errant ça et là.
\VS{2}Alors tu jureras avec vérité, avec droiture et avec justice : Yahweh est vivant ! Et les nations seront bénies en lui, et se glorifieront en lui.
\VS{3}Car ainsi parle Yahweh aux hommes de Juda et de Jérusalem : Labourez pour vous une terre arable et ne semez pas parmi les épines\FTNT{Mt. 13:7 ; Mt. 13:22 ; Mc. 4:7 ; Mc. 4:18 ; Lu. 8:14.}.
\VS{4}Hommes de Juda, et vous habitants de Jérusalem, circoncisez-vous pour Yahweh, circoncisez vos cœurs\FTNT{Ro. 2:29.}, de peur que ma fureur ne sorte comme un feu et qu'elle ne brûle sans qu'on puisse l'éteindre, à cause de la méchanceté de vos actions.
\VS{5}Annoncez en Juda, publiez dans Jérusalem, et dites : Sonnez du shofar dans le pays ! Criez à pleine voix et dites : Assemblez-vous et nous entrerons dans les villes fortes !
\VS{6}Elevez une bannière vers Sion, fuyez, ne vous arrêtez pas ! Car je fais venir du nord le malheur et une grande calamité.
\VS{7}Le lion\FTNT{Lion est ici une allusion à Nebucadnetsar, roi de Babylone. Voir 2 R. 24 et 25 ; Da. 7:4.} est sorti de la caverne, le destructeur des nations est en marche, il est sorti de son lieu, pour réduire ton pays en désert ; tes villes seront ruinées, il n'y aura personne pour y habiter.
\VS{8}C'est pourquoi ceignez-vous de sacs, lamentez-vous et gémissez ; car l'ardeur de la colère de Yahweh ne se détourne pas de nous.
\VS{9}Et il arrivera ce jour-là, dit Yahweh, que le cœur du roi et le cœur des chefs seront épouvantés et que les sacrificateurs seront étonnés, et que les prophètes seront stupéfaits.
\VS{10}C'est pourquoi je dis : Ah ! Seigneur Yahweh ! Oui certainement tu as abusé ce peuple et Jérusalem, en disant : Vous aurez la paix ! Et cependant l'épée est venue jusqu'à l'âme.
\VS{11}En ce temps-là on dira à ce peuple et à Jérusalem : Un vent brûlant souffle des lieux élevés du désert sur le chemin de la fille de mon peuple, non pas pour vanner ni pour nettoyer.
\VS{12}C'est un vent impétueux qui vient de là jusqu'à moi et je leur ferai maintenant leur procès. 
\VS{13}Voici, il monte comme des nuées ; ses chars sont comme un tourbillon, ses chevaux sont plus légers que les aigles. Malheur à nous, car nous sommes détruits !
\VS{14}Jérusalem, lave ton cœur du mal afin que tu sois délivrées ! Jusqu'à quand séjourneront-tu au-dedans de toi les pensées de ton injustice ?
\VS{15}Car une voix apporte des nouvelles de Dan, elle publie depuis la montagne d'Ephraïm le tourment.
\VS{16}Rappelez-le aux nations, faites-le entendre à Jérusalem : Des observateurs viennent d'un pays éloigné ; ils poussent des cris contre les villes de Juda.
\VS{17}Ils se sont mis tout autour d'elle comme ceux qui gardent un champ, parce qu'elle s'est rebellée contre moi, dit Yahweh.
\VS{18}Ta conduite et tes actions t'ont produit ces choses, telle a été ta méchanceté, parce que cela a été une chose amère, certainement elle t'atteindra jusqu'à ton cœur.
\VS{19}Mes entrailles ! Mes entrailles : Je suis dans la douleur au-dedans de mon cœur, mon cœur bat, je ne puis me taire ; car, ô mon âme, tu entends le son du shofar, la clameur de la guerre.
\VS{20}On annonce brèche sur brèche, car tout le pays est dévasté ; mes tentes sont détruites tout à coup, mes pavillons en un moment.
\VS{21}Jusqu'à quand verrai-je la bannière et entendrai-je le son du shofar ?
\VS{22}Car mon peuple est insensé ; ils ne m'ont pas reconnu, ce sont des enfants insensés qui n'ont pas d'intelligence ; ils sont habiles pour faire le mal, et ils ne savent pas faire le bien.
\VS{23}Je regarde la terre, et voici, elle est informe et vide\FTNT{Dieu n'a pas créé la terre informe et vide, mais elle l'est devenue à cause du péché. Voir le commentaire en Gn. 1:2.} ; les cieux et leur lumière ne sont plus.
\VS{24}Je regarde les montagnes, et voici, elles sont ébranlées ; et toutes les collines sont renversées.
\VS{25}Je regarde, et voici, il n'y a pas un seul homme et tous les oiseaux des cieux se sont enfuis.
\VS{26}Je regarde, et voici, le Carmel est un désert ; et toutes ses villes sont détruites, devant Yahweh, devant l'ardeur de sa colère.
\VS{27}Car ainsi parle Yahweh : Tout le pays sera dévasté, mais je ne ferai pas une entière destruction.
\VS{28}C'est pourquoi le pays mènera deuil et les cieux en haut seront obscurcis, parce que je l'ai dit, je l'ai résolu, et je ne m'en repentirai pas et je le révoquerai pas.
\VS{29}Toute la ville s'enfuit à cause du bruit des cavaliers et des archers ; ils entrent dans les bois fourrés et montent sur les rochers ; toute la ville est abandonnée, et aucun homme n'y habite.
\VS{30}Et quand tu auras été détruite que fais-tu ? Quoique tu te revêtes de pourpre, que tu te pares d'ornements d'or, et que tu bordes tes yeux de fard, tu t'embellis en vain: tes amants t'ont méprisée; c'est ta vie qu'ils cherchent.
\VS{31}Car j'entends un cri comme celui d'une femme qui est en travail, et une angoisse comme celle d'une femme qui est en travail de son premier-né ; c'est le cri de la fille de Sion ; elle soupire, elle étend ses mains, en disant : Malheur maintenant à moi, car mon âme a défailli à cause des meurtriers. 
\Chap{5}
\TextTitle{Raisons du jugement de Yahweh}
\VerseOne{}Parcourez les rues de Jérusalem et regardez maintenant, sachez et cherchez dans les places, si vous y trouvez un homme de bien, s'il y a quelqu'un qui fasse ce qui est droit, qui cherche la vérité, et je pardonne à Jérusalem\FTNT{Es. 59:15 ; Mi. 7:2 ; Pr. 20:6.}.
\VS{2}Même s'ils disent : Yahweh est vivant ! En cela, ils jurent faussement.
\VS{3}Yahweh, tes yeux ne regardent-ils pas à la fidélité ? Tu les frappes, et ils ne sentent pas de douleur ; tu les consumes, et ils refusent de recevoir l'instruction ; ils endurcissent leurs faces plus qu'un rocher, ils refusent de se convertir.
\VS{4}Je disais : Certainement ce ne sont que les plus petits ; ils se montrent insensés parce qu'ils ne connaissent pas la voie de Yahweh, le droit de leur Dieu.
\VS{5}J'irai donc vers les plus grands, et je leur parlerai ; car cela connaissent la voie de Yahweh, le droit de leur Dieu ; mais ceux-là même ont brisés le joug et ont rompu les liens.
\VS{6}C'est pourquoi le lion de la forêt les tue, le loup du soir les détruit, et le léopard est aux aguets contre leurs villes ; quiconque en sortira sera déchiré ; car leurs transgressions sont nombreuses, et leurs infidélités se sont renforcées.
\VS{7}Comment te pardonnerais-je en cela ? Tes fils m'ont abandonné, et ils jurent par ce qui ne sont pas dieux. Je les ai rassasiés, mais ils commettent l'adultère et ils se pressent en foule dans la maison de la prostituée.
\VS{8}Ils sont comme des chevaux bien nourris, quand ils se lèvent le matin, chacun hennit après la femme de son prochain.
\VS{9}Ne punirais-je pas ces choses-là, dit Yahweh ? Et mon âme ne se vengerait-elle pas d'une telle nation ?
\VS{10}Montez sur ses murailles et détruisez-les, mais ne les achevez pas entièrement ! Otez ses sarments car ils ne sont pas à Yahweh\FTNT{Jn. 15:5.} !
\VS{11}Car la maison d'Israël et la maison de Juda m'ont été infidèles, dit Yahweh.
\VS{12}Ils démentent Yahweh, et disent : Cela n'arrivera pas, et le malheur ne viendra pas sur nous, nous ne verrons ni l'épée ni la famine.
\VS{13}Et les prophètes sont légers comme le vent, et la parole n'est pas en eux. Qu'il leur soit fait ainsi !
\VS{14}C'est pourquoi ainsi parle Yahweh, le Dieu des armées : Parce que vous avez prononcé cette parole-là, voici, je vais mettre mes paroles dans ta bouche pour y être comme une feu, et ce peuple sera comme le bois, et ce feu les consumera.
\VS{15}Maison d'Israël, voici, je fais venir contre vous une nation d'un pays éloigné\FTNT{Il s'agit de Babylone. Voir 2 R. 24 et 25.}, dit Yahweh, une nation puissante, une nation ancienne, une nation dont tu ne connais pas la langue, et dont tu ne comprendras pas ce qu'elle dira.
\VS{16}Son carquois est comme un sépulcre ouvert, et ils sont tous des hommes vaillants.
\VS{17}Et elle dévorera ta moisson et ton pain, que tes fils et tes filles devaient manger; elle dévorera tes brebis et tes bœufs; elle dévorera les fruits ta vigne et ton figuier et réduira à la pauvreté par l'épée tes villes fortes dans lesquelles tu te confies.
\VS{18}Toutefois en ces jours-là, dit Yahweh, je ne vous achèverai pas entièrement.
\VS{19}Et il arrivera que vous direz : Pourquoi Yahweh, notre Dieu, nous a-t-il fait toutes ces choses ? Tu leur diras ainsi : Comme vous m'avez abandonné et que vous avez servi les dieux étrangers dans votre pays, ainsi vous servirez des étrangers dans un pays qui n'est pas le vôtre.
\VS{20}Annoncez ceci dans la maison de Jacob, et publiez-le dans Juda, en disant :
\VS{21}Ecoutez maintenant ceci, peuple insensé, et qui n'avez pas d'intelligence; qui avez des yeux et ne voyez pas; et qui avez des oreilles et n'entendez pas\FTNT{Ez. 12:2 ; Jn. 12:40.}.
\VS{22}Ne me craindrez-vous pas, dit Yahweh, ne tremblerez-vous pas devant ma face ? C'est moi qui ai mis le sable pour limite à la mer, par une ordonnance perpétuelle et qui ne passera pas ; ses vagues s'agitent, mais elles sont impuissantes ; elles grondent, mais elles ne la passent pas\FTNT{Pr. 8:29 ; Job. 38:8.}.
\VS{23}Mais ce peuple-ci a un cœur indocile et rebelle ; ils reculent en arrière et s'en vont.
\VS{24}Et ils ne disent pas dans leur cœur : Craignons maintenant Yahweh, notre Dieu, qui nous donne la pluie en son temps, de la première et de l'arrière-saison, et qui nous réserve les semaines ordonnées pour la moisson.
\VS{25}Vos iniquités ont détourné ces choses, vos péchés retiennent loin de vous le bien.
\VS{26}Car il se trouve parmi mon peuple des méchants ; ils épient comme l'oiseleur qui dresse des pièges, ils tendent des filets et prennent des hommes\FTNT{Ps. 91:3 ; Ps. 124:7.}.
\VS{27}Comme la cage est remplie d'oiseaux, ainsi leurs maisons sont remplies de fraude ; c'est par ce moyen qu'ils deviennent grands et riches.
\VS{28}Ils s'engraissent, ils sont brillants ; ils surpassent les actions des méchants, ils ne jugent pas la cause, la cause de l'orphelin, et ils prospèrent ; ils ne font pas droit aux pauvres.
\VS{29}Ne punirais-je pas ces choses-là, dit Yahweh ? Et mon âme ne se vengerait-elle pas d'une telle nation ?
\VS{30}Il est arrivé dans le pays une chose étonnante et horrible :
\VS{31}C'est que les prophètes prophétisent le mensonge, et les sacrificateurs dominent par leur moyen, et mon peuple prend plaisir à cela. Que ferez-vous donc quand elle prendra fin ?
\Chap{6}
\TextTitle{Jérusalem dans la confusion}
\VerseOne{}Fils de Benjamin, fuyez par troupes du milieu de Jérusalem, et sonnez du shofar à Tekoa, et élevez un signal de feu à Beth-Hakkérem ! Car on voit venir du nord un malheur et une grande ruine.
\VS{2}La belle et la délicate, la fille de Sion, je la détruis !
\VS{3}Les pasteurs avec leurs troupeaux viennent contre elle ; ils plantent leurs tentes autour d'elle, chacun paîtra en son quartier.
\VS{4}Préparez le combat contre elle ! Levez-vous, et montons en plein midi !… Malheur à nous, car le jour décline, les ombres du soir s'étendent.
\VS{5}Levez-vous ! Montons de nuit, et ruinons ses palais !
\VS{6}Car ainsi parle Yahweh des armées : Coupez des arbres, élevez des terrasses contre Jérusalem ! C'est la ville qui doit être visitée ; tout est oppression au milieu d'elle.
\VS{7}Comme le puits fait jaillir ses eaux, ainsi elle fait jaillir sa méchanceté ; on n'entend continuellement en elle, devant moi, que violence et ruine, avec des maladies et des plaies.
\VS{8}Jérusalem, reçois l'instruction, de peur que mon âme ne se retire de toi, et que je ne fasse de toi un désert, et une terre inhabitée !
\VS{9}Ainsi parle Yahweh des armées : On grappillera entièrement comme une vigne les restes d'Israël. Remets ta main dans les paniers, comme un vendangeur.
\VS{10}A qui parlerai-je, et qui prendrai-je à témoin, pour qu'ils écoutent ? Voici, leur oreille est incirconcise, et ils ne peuvent entendre ; voici, la parole de Yahweh leur est en opprobre, ils n'y prennent point de plaisir.
\VS{11}C'est pourquoi je suis plein de la fureur de Yahweh, et je suis las de la contenir. Répands-la sur les enfants dans la rue, et sur les assemblées des jeunes gens. Car tant le mari que la femme seront pris, le vieillard et celui qui est chargé de jours.
\VS{12}Et leurs maisons passeront à d'autres, les champs et les femmes aussi, quand j'étendrai ma main sur les habitants du pays, dit Yahweh.
\VS{13}Car depuis le plus petit d'entre eux jusqu'au plus grand, chacun s'adonne au gain déshonnête, tant le prophète que le sacrificateur, tous se agissent faussement.
\VS{14}Et ils pansent à la légère la plaie de la fille de mon peuple, disant : Paix ! Paix ! et il n'y a pas de paix\FTNT{1 Th. 5:3.}.
\VS{15}Sont-ils confus d'avoir commis des abominations ? Ils n'en ont même aucune honte, et ils ne savent pas ce que c'est que de rougir ; c'est pourquoi ils tomberont parmi ceux qui tombent, ils seront renversés au temps où je les visiterai, dit Yahweh.
\VS{16}Ainsi parle Yahweh : Tenez-vous sur les chemins, regardez et enquérez-vous des sentiers des siècles passés, quel est le bon chemin ; et marchez-y, et vous trouverez le repos de vos âmes ! Et ils répondent : Nous n'y marcherons pas.
\VS{17}J'ai aussi établi sur vous des sentinelles\FTNT{Es. 21:6 ; Ez. 33:1-19.} qui disent : Soyez attentifs au son du shofar ! Mais ils répondent : Nous n'y serons pas attentifs.
\VS{18}Vous donc, nations, écoutez, et toi assemblée, connais ce qui est entre eux.
\VS{19}Ecoute, terre ! Voici, je fais venir un mal sur ce peuple, à savoir le fruit de leurs pensées ; car ils n'ont pas été attentifs à mes paroles, et qu'ils ont rejeté ma loi.
\VS{20}Pourquoi m'offrir de l'encens venu de Séba, et le bon roseau aromatique du pays éloigné ? Vos holocaustes ne me plaisent pas, et vos sacrifices ne me sont pas agréables.
\VS{21}C'est pourquoi ainsi parle Yahweh : Voici, je mettrai devant ce peuple des pierres d'achoppement, auxquels les pères et les fils, le voisin et son compagnon, se heurteront ensemble et ils périront.
\VS{22}Ainsi parle Yahweh : Voici, un peuple vient du pays du nord, et une grande nation se réveille des extrémités de la terre.
\VS{23}Ils prendront l'arc et le javelot ; ils sont cruels et n'ont pas de pitié ; leur voix gronde comme la mer ; ils sont montés sur des chevaux, ils sont rangés comme un seul homme en bataille contre toi, fille de Sion !
\VS{24}Nous en entendons le bruit, nos mains en deviennent lâches, l'angoisse nous saisit, et une douleur comme celle d'une femme qui enfante.
\VS{25}Ne sortez pas dans les champs, n'allez pas par les chemins ; car l'épée de l'ennemi, la terreur est partout.
\VS{26}Fille de mon peuple, ceins-toi d'un sac et roule-toi dans la cendre, prends le deuil comme pour un fils unique, fais une lamentation très amère ! Car le dévastateur vient subitement sur nous.
\VS{27}Je t'avais établi en observateur au milieu de mon peuple, comme une forteresse, pour que tu connaisses et que tu éprouves leur voie.
\VS{28}Ils sont tous rebelles et plus que rebelles, des calomniateurs, ils sont comme de l'airain et du fer ; ils sont tous corrompus.
\VS{29}Le soufflet est brûlant, le plomb est consumé par le feu ; c'est en vain que l'on fond et refond, car les mauvais ne sont pas séparés.
\VS{30}On les appelle de l'argent réprouvé, car Yahweh les a réprouvés.
\Chap{7}
\TextTitle{Hypocrisie de Juda}
\VerseOne{}La parole fut adressée à Jérémie de la part de Yahweh, en disant :
\VS{2}Tiens-toi debout à la porte de la maison de Yahweh, et là, crie cette parole, et dis : Ecoutez la parole de Yahweh, vous tous, hommes de Juda, qui entrez par ces portes, pour vous prosterner devant Yahweh !
\VS{3}Ainsi parle Yahweh des armées, le Dieu d'Israël : Amendez vos voies et vos actions, et je vous ferai habiter en ce lieu-ci.
\VS{4}Ne vous confiez pas en des paroles trompeuses, en disant : C'est ici le temple de Yahweh, le temple de Yahweh, le temple de Yahweh !
\VS{5}Mais amendez sérieusement vos voies et vos actions, et appliquez-vous à faire droit à ceux qui plaident l'un contre l'autre,
\VS{6}et ne faites pas de tort à l'étranger, ni à l'orphelin, ni à la veuve, et ne répandez pas en ce lieu-ci le sang innocent, et ne marchez pas après les dieux étrangers, pour votre malheur.
\VS{7}Et je vous ferai habiter depuis un siècle jusqu'à l'autre siècle en ce lieu-ci, dans le pays que j'ai donné à vos pères.
\VS{8}Voici, vous vous confiez en des paroles trompeuses, sans aucun profit.
\VS{9}Ne dérobez-vous pas ? Ne tuez-vous pas ? Ne commettez-vous pas adultère ? Ne jurez-vous pas faussement ? Ne faites-vous pas des encensements à Baal ? N'allez-vous pas après les dieux étrangers, que vous ne connaissez point ?
\VS{10}Toutefois vous venez et vous vous présentez devant moi, dans cette maison sur laquelle mon Nom est invoqué, et vous dites : Nous sommes délivrés !… Pour faire toutes ces abominations !
\VS{11}N'est-elle plus à vos yeux qu'une caverne de voleurs\FTNT{Mt. 21:13 ; Mc. 11:17 ; Lu. 19:46.}, cette maison sur laquelle mon Nom est invoqué ? Et voici, moi-même je le vois, dit Yahweh.
\VS{12}Mais allez maintenant à mon lieu qui était à Silo, où j'avais fait demeurer mon Nom au commencement. Et regardez ce que je lui ai fait, à cause de la méchanceté de mon peuple d'Israël.
\VS{13}Maintenant donc, puisque vous avez fait toutes ces actions, dit Yahweh, puisque je vous ai parlé, parlé dès le matin, et que vous n'avez pas écouté, puisque je vous ai appelés et que vous n'avez pas répondu ;
\VS{14}je ferai à cette maison sur laquelle mon Nom est invoqué, et sur laquelle vous vous confiez, et à ce lieu que je vous ai donné à vous et à vos pères, comme j'ai fait à Silo ;
\VS{15}et je vous chasserai de devant ma face, comme j'ai chassé tous vos frères, avec toute la postérité d'Ephraïm.
\VS{16}Toi donc ne prie pas pour ce peuple, et n'élève pour eux ni cri ni prière, et n'intercède pas auprès de moi\FTNT{Ez. 3:26-27.} ; car je ne t'écouterai pas.
\VS{17}Ne vois-tu pas ce qu'ils font dans les villes de Juda et dans les rues de Jérusalem ?
\VS{18}Les fils ramassent le bois, et les pères allument le feu, et les femmes pétrissent la pâte pour faire des gâteaux à la reine des cieux\FTNT{La reine des cieux est une déesse qui change de nom en fonction des pays. Asherah, Astarté, Isis, Junon, Cybèle, Diane ou encore la vierge Marie, proclamée mère de Dieu en 431 au concile d'Ephèse. Voir De. 16:2-3.}, et pour faire des libations aux dieux étrangers, afin de m'irriter.
\VS{19}Est-ce moi qu'ils irritent ? dit Yahweh ; n'est-ce pas contre eux-mêmes, à la confusion de leurs faces ?
\VS{20}C'est pourquoi ainsi parle le Seigneur Yahweh : Voici, ma colère et ma fureur se répandent sur ce lieu-ci, sur les hommes et sur les bêtes, sur les arbres des champs et sur le fruit de la terre ; ma colère brûlera et ne s'éteindra pas.
\VS{21}Ainsi parle Yahweh des armées, le Dieu d'Israël : Ajoutez vos holocaustes à vos sacrifices, et mangez-en la chair !
\VS{22}Car je n'ai pas parlé avec vos pères et je ne leur ai pas donné d'ordre au sujet des holocaustes et des sacrifices, le jour où je les ai fait sortir du pays d'Egypte.
\VS{23}Mais voici la parole que je leur ai commandée, disant : Ecoutez ma voix, et je serai votre Dieu, et vous serez mon peuple ; marchez dans toutes les voies que je vous ordonne, afin que vous soyez heureux\FTNT{Ex. 15:26.}.
\VS{24}Mais ils n'ont pas écouté, et n'ont pas prêté l'oreille ; mais ils ont suivi d'autres conseils, les penchants de leur mauvais cœur ; ils se sont éloignés et ne sont pas revenus à moi.
\VS{25}Depuis le jour où vos pères sont sortis du pays d'Egypte, jusqu'à ce jour, je vous ai envoyé tous mes serviteurs les prophètes, je les ai envoyés chaque jour, dès le matin.
\VS{26}Mais ils ne m'ont pas écouté, et ils n'ont pas prêté l'oreille ; mais ils ont raidi leur cou, ils ont fait le mal plus que leurs pères.
\VS{27}Tu leur diras toutes ces paroles, mais ils ne t'écouteront pas ; et tu crieras après eux, mais ils ne te répondront pas.
\VS{28}C'est pourquoi tu leur diras : C'est ici la nation qui n'écoute pas la voix de Yahweh, son Dieu, et qui ne reçoit pas d'instruction ; la vérité a disparu, elle s'est retirée de leur bouche.
\VS{29}Coupe ta chevelure, ô Jérusalem ! Et jette-la au loin, et prononce à haute voix ta complainte sur les lieux élevés ! Car Yahweh rejette et abandonne la génération qui a provoqué sa fureur.
\VS{30}Car les fils de Juda ont fait ce qui est mal à mes yeux, dit Yahweh ; ils ont mis leurs abominations dans cette maison sur laquelle mon Nom est invoqué, afin de la souiller.
\VS{31}Et ils ont bâti les hauts lieux de Topheth, qui est dans la vallée de Ben-Hinnom\FTNT{Voir commentaire en Ap. 16:16.}, pour brûler au feu leurs fils et leurs filles\FTNT{Lé. 18:21. Voir commentaire en Lé. 20:2.} : Ce que je n'avais pas ordonné, et à quoi je n'ai jamais pensé.
\VS{32}C'est pourquoi voici, les jours viennent, dit Yahweh, qu'elle ne sera plus appelée Topheth, ni la vallée de Ben-Hinnom, mais la vallée de la tuerie ; et on enterrera les morts à Topheth, à cause qu'il n'y aura plus d'autre lieu.
\VS{33}Et les cadavres de ce peuple seront la pâture des oiseaux des cieux et des bêtes de la terre ; sans qu'il n'y ait personne qui les effraye.
\VS{34}Je ferai aussi cesser dans les villes de Juda et dans les rues de Jérusalem les cris de joie et les cris d'allégresse, la voix de l'époux et la voix de l'épouse ; car le pays sera un désert.
\Chap{8}
\TextTitle{Juda dans l'égarement}
\VerseOne{}En ce temps-là, dit Yahweh, on sortira les os des rois de Juda, et les os de ses chefs, les os des sacrificateurs, et les os des prophètes, et les os des habitants de Jérusalem, hors de leurs sépulcres.
\VS{2}Et on les étendra devant le soleil, et devant la lune, et devant toute l'armée des cieux, qui sont des choses qu'ils ont aimées, qu'ils ont servies et après lesquelles ils ont marché ; des choses qu'ils ont recherchées, et devant lesquelles ils se sont prosternés ; ils ne seront pas recueillis ni ensevelis, ils seront comme du fumier sur la face du sol.
\VS{3}Et la mort sera plus désirable que la vie pour tous ceux qui resteront de cette race mauvaise, ceux, dis-je, qui seront restés dans tous les lieux où je les aurai chassés, dit Yahweh des armées.
\VS{4}Dis-leur donc : Ainsi parle Yahweh : Si on tombe, ne se relève-t-on pas ? Et si on se détourne, ne revient-on pas ?
\VS{5}Pourquoi donc ce peuple de Jérusalem s'abandonne-t-il à de perpétuels égarements ? Ils tiennent ferme à la tromperie, et ils refusent de convertir.
\VS{6}Je suis attentif et j'écoute, mais nul ne parlent selon la justice ; il n'y a personne qui se repente de sa méchanceté, disant : Qu'ai-je fait ? Ils retournent tous vers les objets qui les entraînent, comme le cheval qui se jette avec impétuosité parmi la bataille.
\VS{7}Même la cigogne connaît dans les cieux ses saisons ; la tourterelle et l'hirondelle, et la grue observent le temps où elles doivent venir ; mais mon peuple ne connaît pas les ordonnances de Yahweh.
\VS{8}Comment dites-vous : Nous sommes les sages, et la loi de Yahweh est avec nous ? Voilà, certes on a agi faussement, et la plume des scribes est une plume de fausseté.
\VS{9}Les sages sont confus, ils sont épouvantés et pris ; car ils ont rejeté la parole de Yahweh, et quelle sagesse ont-ils ?
\VS{10}C'est pourquoi je donnerai leurs femmes à d'autres, et leurs champs à des gens qui les posséderont en héritage. Car depuis le plus petit jusqu'au plus grand, chacun s'adonne au gain déshonnête, tant le prophète que le sacrificateur, tous agissent faussement.
\VS{11}Ils pansent à la légère la plaie de la fille de mon peuple, en disant : Paix ! Paix ! Et il n'y a pas de paix.
\VS{12}Sont-ils confus d'avoir commis des abominations ? Ils n'en ont même aucune honte, et ils ne savent pas ce que c'est que de rougir ; c'est pourquoi ils tomberont parmi ceux qui tombent, ils seront renversés au temps où je les visiterai, dit Yahweh.
\VS{13}Je les ramasserai, j'en finirai avec eux, dit Yahweh ; il n'y aura plus de raisins à la vigne, et il n'y aura plus de figues au figuier, les feuilles se flétriront ; et ce que je leur avais donné sera transporté avec eux.
\VS{14}Pourquoi restons-nous assis ? Assemblez-vous et entrons dans les villes fortes, et nous serons là en repos ! Car Yahweh, notre Dieu, nous réduit au silence, et il nous fait boire des eaux empoisonnées, parce que nous avons péché contre Yahweh.
\VS{15}On attendait la paix, et il n'y a rien de bon ; on attend le temps de guérison, et voici la terreur !
\VS{16}Le hennissement de ses chevaux se fait entendre de Dan, et tout le pays tremble au bruit des hennissements de ses puissants chevaux ; ils viennent et dévorent le pays et ce qu'il contient, la ville et ceux qui l'habitent.
\VS{17}Qui plus est, voici, j'envoie contre vous des serpents, des basilics, contre lesquels il n'y a pas d'enchantement, et ils vous mordront, dit Yahweh.
\VS{18}J'ai voulu prendre des forces pour soutenir la douleur, mais mon cœur est languissant au dedans de moi.
\VS{19}Voici la voix du cri de la fille de mon peuple, qui crie d'un pays éloigné : Yahweh n'est-il plus à Sion ? Son Roi n'est-il plus au milieu d'elle ? Pourquoi m'ont-ils irrité par leurs images taillées, par les vanités\FTNT{Idoles que Dieu appelle vanité, vapeur ou souffle} (2) étrangères ?
\VS{20}La moisson est passée, l'été est fini, et nous ne sommes pas sauvés !
\VS{21}Je suis brisé par la blessure de la fille de mon peuple, je suis sombre, l'épouvante me saisit.
\VS{22}N'y a-t-il pas de baume en Galaad ? N'y a-t-il pas là de médecin ? Pourquoi donc la guérison de la fille de mon peuple ne s'opère-t-elle pas ?
\Chap{9}
\TextTitle{Jérémie pleure sur son peuple}
\VerseOne{}Plaise à Dieu que ma tête soit comme un réservoir d'eau, et que mes yeux soient une vive fontaine de larmes, et je pleurerais jour et nuit les blessés à mort de la fille de mon peuple !
\VS{2}Plaise à Dieu que j'aie au désert une cabane de voyageurs, j'abandonnerais mon peuple, je m'en irais loin de lui ! Car ils sont tous des adultères, et une assemblée de perfides.
\VS{3}Ils ont tendu leur langue, qui a été comme leur arc pour décrocher le mensonge\FTNT{Ps. 64:3-4.} ; et ils se sont renforcés dans la terre contre la fidélité ; car ils sont allés de méchanceté en méchanceté, et ne m'ont pas reconnu, dit Yahweh.
\VS{4}Gardez-vous chacun de son intime ami, et ne vous confiez en aucun frère\FTNT{Mi. 7:5.} ; car tout frère fait métier de supplanter, et tout intime ami marche dans la calomnie.
\VS{5}Et chacun se moque de son intime ami, et on ne parle pas selon la vérité ; ils ont instruit leur langue à dire le mensonge, ils se tourmentent extrêmement pour faire le mal.
\VS{6}Ta demeure est au milieu de la tromperie ; ils refusent, à cause de la tromperie, de me connaître, dit Yahweh.
\VS{7}C'est pourquoi, ainsi parle Yahweh des armées : Voici, je vais les fondre, je les éprouverai\FTNT{Mal. 3:3.}. Car comment en agirais-je autrement à l'égard de la fille de mon peuple ?
\VS{8}Leur langue est une flèche meurtrière, elle profère des tromperies ; chacun de sa bouche parle de la paix avec son ami, mais au-dedans il lui dresse des embûches\FTNT{Ps. 12:3; Ps. 28:3.}.
\VS{9}Ne les punirais-je pas pour ces choses-là, dit Yahweh ? Mon âme ne se vengerait-elle pas d'une telle nation ?
\VS{10}J'élèverai ma voix avec larmes, et je prononcerai à haute voix une lamentation à cause des montagnes, et une complainte à cause des cabanes du désert, parce qu'elles sont brûlées, de sorte que personne n'y passe et qu'on n'y entend plus la voix des troupeaux ; les oiseaux des cieux et le bétail ont fui, ils s'en sont allés.
\VS{11}Et je ferai de Jérusalem des monceaux de ruines, elle sera un repaire de serpents, et je ferai des villes de Juda un désert sans habitants.
\VS{12}Qui est l'homme sage qui comprenne ceci ? Qui est celui à qui la bouche de Yahweh a parlé ? Qu'il le déclare et qu'il dise pourquoi le pays est-il détruit, brûlé comme un désert, sans que personne y passe ?
\VS{13}Yahweh donc dit : Parce qu'ils ont abandonné ma loi que j'avais mise devant eux ; parce qu'ils n'ont pas écouté ma voix, et qu'ils n'ont pas marché selon elle ;
\VS{14}mais parce qu'ils ont marché suivant les penchants de leur cœur, et après les Baals, comme leurs pères le leur ont enseigné.
\VS{15}C'est pourquoi, ainsi parle Yahweh des armées, le Dieu d'Israël : Voici, je vais faire manger de l'absinthe à ce peuple-ci, et je leur ferai boire des eaux empoisonnées.
\VS{16}Je les disperserai parmi les nations que n'ont connues ni eux ni leurs pères, et j'enverrai après eux l'épée, jusqu'à ce que je les aie exterminés.
\VS{17}Ainsi parle Yahweh des armées : Considérez, et appelez des pleureuses, afin qu'elles viennent, et mandez les femmes sages, et qu'elles viennent !
\VS{18}Qu'elles se hâtent, et qu'elles prononcent à haute voix une lamentation sur nous ! Et que nos larmes tombent de nos yeux et que l'eau coule de nos paupières !
\VS{19}Car une voix de lamentation se fait entendre de Sion, disant : Eh quoi ! Nous sommes dévastés ! Nous sommes couverts de honte ! Car nous avons abandonné le pays, car nos demeures nous ont jetés dehors !
\VS{20}C'est pourquoi, vous, femmes, écoutez la parole de Yahweh, et que votre oreille reçoive la parole de sa bouche ! Enseignez vos filles à se lamenter, et chacune sa compagne à faire des complaintes !
\VS{21}Car la mort est montée par nos fenêtres, elle est entrée dans nos palais, pour exterminer les enfants dans les rues, et les jeunes hommes dans les places.
\VS{22}Dis : Ainsi parle Yahweh : Même les cadavres des hommes tomberont comme du fumier sur le dessus des champs, et comme une gerbe après le moissonneur, sans que personne les ramasse !
\VS{23}Ainsi parle Yahweh : Que le sage ne se glorifie pas de sa sagesse, que le fort ne se glorifie pas de sa force, et que le riche ne se glorifie pas de sa richesse.
\VS{24}Mais que celui qui se glorifie, se glorifie d'avoir de l'intelligence et de me connaître, car je suis Yahweh, qui fais miséricorde, droit et justice sur la terre ; car je prends plaisir en ces choses-là, dit Yahweh\FTNT{Ps. 62:10 ; 1 Co. 1:31 ; 2 Co. 10:17 ; 1 Ti. 6:17.}.
\VS{25}Voici, les jours viennent, dit Yahweh, où je punirai tout circoncis incirconcis,
\VS{26}l'Egypte, Juda, Edom, les fils d'Ammon, Moab, et tous ceux qui se coupent les coins de leur barbe et qui habitent dans le désert ; car toutes les nations sont incirconcises, et toute la maison d'Israël a le cœur incirconcis.
\Chap{10}
\TextTitle{Dénonciation de l'idolâtrie en Israël}
\VerseOne{}Ecoutez la parole que Yahweh vous adresse, maison d'Israël !
\VS{2}Ainsi parle Yahweh : N'apprenez pas les façons de faire des nations\FTNT{Lé. 18:3 ; De. 12:30.}, et ne craignez pas les signes des cieux, parce que les nations les craignent.
\VS{3}Car les lois des peuples ne sont que vanité\FTNT{Les lois des peuples, ou encore statuts, coutumes, ordonnances ne sont que vanité. Nous devons nous soumettre aux lois des nations tant que celles-ci ne s'opposent pas à la Loi de Dieu (1 Pi. 2 :13). Quand celles-ci sont contraires aux règles morales établies par le Seigneur, nous devons obéir à Dieu, car il vaut mieux obéir à Dieu plutôt qu'aux hommes (Ac. 4:19 ; Ac. 5:29).}. On coupe le bois dans la forêt ; la main de l'ouvrier le travaille avec la hache\FTNT{Es. 40 :20 ; Es. 44 :12-18.} ;
\VS{4}on l'embellit avec de l'argent et de l'or, on le fait tenir avec des clous et à coups de marteau, afin qu'il ne vacille pas.
\VS{5}Ils sont façonnés tout droits comme des colonnes massives, et ils ne parlent pas ; on les porte par nécessité, parce qu'ils ne peuvent pas marcher. Ne les craignez pas, car ils ne sauraient faire aucun mal, et aussi ils sont incapables de faire du bien.
\VS{6}Nul n'est semblable à toi, ô Yahweh ! Tu es grand, et ton Nom est grand par ta puissance.
\VS{7}Qui ne te craindrait, Roi des nations ? Car cela t'est dû ; car, parmi tous les sages des nations et dans tous leurs royaumes, nul n'est semblable à toi\FTNT{Ap. 15:4.}.
\VS{8}Et ils sont tous ensemble stupides et insensés ; le bois ne leur enseigne que des vanités\FTNT{Ha. 2:18.}.
\VS{9}L'argent qui est étendu en plaques est apporté de Tarsis, et l'or d'Uphaz, pour être mis en œuvre par l'ouvrier et par les mains du fondeur ; et la pourpre et l'écarlate sont leur vêtement ; toutes ces choses sont l'ouvrage de gens habiles.
\VS{10}Mais Yahweh est le Dieu de vérité, c'est le Dieu vivant et le Roi éternel ; la terre tremble devant sa colère, et les nations ne supportent pas sa fureur.
\VS{11}Vous leur parlerez ainsi : Les dieux qui n'ont pas fait les cieux et la terre périront de la terre et de dessous les cieux.
\VS{12}Mais Yahweh est celui qui a fait la terre par sa puissance, qui a fondé le monde habitable par sa sagesse, et qui a étendu les cieux par son intelligence.
\VS{13}Sitôt qu'il fait retentir sa voix, il y a un tumulte d'eaux dans les cieux ; il fait monter les vapeurs des extrémités de la terre, il fait les éclairs et la pluie, et il fait sortir le vent de ses réservoirs.
\VS{14}Tout homme devient stupide par sa connaissance, tout fondeur est honteux par les images taillées ; car les idoles en métal fondu ne sont que mensonge, il n'y a pas de souffle en elles ;
\VS{15}elles ne sont que vanité, une œuvre de tromperie ; elles périront au temps de leur châtiment.
\VS{16}La portion de Jacob n'est pas comme ces choses-là ; car c'est lui qui a tout formé, et Israël est la tribu de son héritage. Son Nom est Yahweh des armées.
\VS{17}Toi qui es assise dans la détresse, rassemble du pays tes paquets !
\VS{18}Car ainsi parle Yahweh : Voici, cette fois je vais lancer au loin, comme avec une fronde, les habitants du pays ; je vais les mettre à l'étroit, afin qu'on les atteigne.
\VS{19}Malheur à moi, diront-ils, à cause de ma blessure ! Ma plaie est douloureuse ! Mais moi, je dis : Quoi qu'il en soit, c'est une maladie qu'il faut que je supporte.
\VS{20}Ma tente est dévastée, tous mes cordages sont rompus ; mes fils m'ont quittée, et ils ne sont plus ; il n'y a plus personne qui dresse ma tente, qui relève mes pavillons.
\VS{21}Car les pasteurs ont été stupides, ils n'ont pas cherché Yahweh ; c'est pour cela qu'ils n'ont pas réussi et que tous leurs troupeaux s'éparpillent.
\VS{22}Voici, une rumeur se fait entendre ; avec une grande secousse qui vient du pays du nord, pour faire des villes de Juda un désert, un repaire de serpents.
\VS{23}Yahweh ! Je sais que la voie de l'homme ne dépend pas de lui\FTNT{Pr. 16:1.}, et qu'il n'est pas au pouvoir de l'homme qui marche de diriger ses pas.
\VS{24}Ô Yahweh ! Châtie-moi, mais avec équité, et non dans ta colère, de peur que tu ne me réduises à rien\FTNT{Es. 27:8 ; Ps. 38:2.}.
\VS{25}Répands ta fureur sur les nations qui ne te connaissent pas, et sur les familles qui n'invoquent pas ton Nom ! Car ils ont dévoré Jacob, ils l'ont, dis-je, dévoré et consumé, et ils ont mis en désolation son agréable demeure.
\Chap{11}
\TextTitle{Yahweh dénonce la prostitution de Juda}
\VerseOne{}La parole fut adressée à Jérémie de la part de Yahweh, en disant :
\VS{2}Ecoutez les paroles de cette alliance, et parlez aux hommes de Juda et aux habitants de Jérusalem !
\VS{3}Dis-leur : Ainsi parle Yahweh, le Dieu d'Israël : Maudit soit l'homme qui n'écoute pas les paroles de cette alliance\FTNT{De. 27:26 ; Ga. 3:10.},
\VS{4}que j'ai ordonnée à vos pères, le jour où je les ai fait sortir du pays d'Egypte, de la fournaise de fer, en disant : Ecoutez ma voix et faites toutes les choses que je vous ordonnerai ; alors vous serez mon peuple, et je serai votre Dieu\FTNT{Lé. 26:12 ; De. 4:20.},
\VS{5}afin que j'accomplisse le serment que j'ai juré à vos pères, de leur donner un pays où coulent le lait et le miel, comme vous le voyez aujourd'hui. Et je répondis et dis : Amen ! Ô Yahweh !
\VS{6}Puis Yahweh me dit : Crie toutes ces paroles dans les villes de Juda et dans les rues de Jérusalem, en disant : Ecoutez les paroles de cette alliance et observez-les !
\VS{7}Car j'ai averti vos pères, depuis le jour où je les ai fait monter du pays d'Egypte jusqu'à ce jour, je les ai avertis dès le matin, en disant : Ecoutez ma voix !
\VS{8}Mais ils n'ont pas écouté, ils n'ont pas prêté l'oreille, ils ont marché chacun suivant les penchants de leur mauvais cœur ; c'est pourquoi j'ai fait venir sur eux toutes les paroles de cette alliance, que je leur avais donné l'ordre d'observer, et qu'ils n'ont pas observée.
\VS{9}Yahweh me dit : Il y a une conspiration entre les hommes de Juda et entre les habitants de Jérusalem.
\VS{10}Ils sont retournés aux iniquités de leurs premiers pères, qui ont refusé d'écouter mes paroles, et ils sont allés après d'autres dieux pour les servir. La maison d'Israël et la maison de Juda ont rompu mon alliance, que j'avais faite avec leurs pères.
\VS{11}C'est pourquoi ainsi parle Yahweh : Voici, je fais venir sur eux un mal dont ils ne pourront sortir. Ils crieront vers moi, et je ne les écouterai pas\FTNT{Es. 1:15 ; Ez. 8:18 ; Mi. 3:4 ; Pr. 1:28.}.
\VS{12}Et les villes de Juda et les habitants de Jérusalem s'en iront et crieront vers les dieux auxquels ils brûlent de l'encens, mais ces dieux-là ne les sauveront pas au temps de leur malheur.
\VS{13}Car, ô Juda ! Tu as eu autant de dieux que de villes ; et toi, Jérusalem, tu as dressé autant d'autels aux choses honteuses que tu as de rues, des autels, dis-je, pour brûler de l'encens à Baal\FTNT{Ez. 16:24-31 ; Ac. 17:23.}…
\VS{14}Toi donc, n'intercède pas pour ce peuple, et n'élève pour eux ni cri ni prière ; car je ne les écouterai pas au temps où ils crieront vers moi dans leur malheur.
\VS{15}Qu'est-ce que mon bien-aimé a à faire dans ma maison, que tant de gens se servent d'elle pour y faire leurs complots? la chair sainte est transportée loin de toi, et encore quand tu fais le mal, c'est alors que tu triomphes !
\VS{16}Yahweh avait appelé ton nom Olivier verdoyant et beau par la forme de ton fruit ; mais au bruit d'un grand fracas, il y a mis le feu, et ses rameaux sont brisés.
\VS{17}Yahweh des armées, qui t'a plantée, prononce le mal contre toi, à cause de la méchanceté de la maison d'Israël et de la maison de Juda, qui ont agi pour m'irriter, en brûlant de l'encens à Baal.
\TextTitle{Jugement des ennemis de Jérémie}
\VS{18}Et Yahweh me l'a fait savoir, et je l'ai su ; alors tu m'as fait voir leurs actions.
\VS{19}Mais moi, comme un agneau, ou comme un bœuf qu'on mène pour être égorgé, je ne savais pas qu'ils projetaient de mauvais desseins contre moi, en disant : Détruisons l'arbre avec son fruit ! Exterminons-le de la terre des vivants, et qu'on ne se souvienne plus de son nom !
\VS{20}Mais toi, Yahweh des armées, qui juges justement, et qui éprouve les reins et le cœur ! Fais que je voie ta vengeance s'exercer contre eux, car je t'ai découvert ma cause\FTNT{1 S. 16:7 ; Ps. 26:2 ; 1 Ch. 28:9 ; Ap. 2:23.}.
\VS{21}C'est pourquoi ainsi parle Yahweh contre les gens d'Anathoth, qui cherchent ta vie et qui disent : Ne prophétise plus au Nom de Yahweh, et tu ne mourras pas par nos mains\FTNT{Es. 30:10 ; Mi. 2:6.} !
\VS{22}C'est pourquoi donc ainsi parle Yahweh des armées : Voici, je vais les punir ; les jeunes hommes mourront par l'épée, leurs fils et leurs filles mourront par la famine.
\VS{23}Et il ne restera rien d'eux ; car je ferai venir le mal sur les gens d'Anathoth, l'année de leur châtiment.
\Chap{12}
\TextTitle{Prière de Jérémie et réponse de Yahweh}
\VerseOne{}Yahweh, quand je contesterai avec toi, tu seras trouvé juste ; mais toutefois j'entrerai en contestation avec toi : Pourquoi la voie des méchants est-elle prospère ? Pourquoi tous les perfides vivent-ils en paix\FTNT{Job. 21:7-9 ; Ro. 3:4.} ?
\VS{2}Tu les as plantés, et ils ont pris racine, ils s'avancent, et ils portent du fruit. Tu es près de leur bouche, mais tu es loin de leurs cœurs\FTNT{Es. 29:13 ; Job. 21:7-8.}.
\VS{3}Mais, ô Yahweh, tu me connais, tu me vois, tu éprouves mon cœur qui est avec toi. Traîne-les comme des brebis qu'on mène pour être égorgées, et mets-les à part pour le jour de la tuerie !
\VS{4}Jusqu'à quand le pays mènera-t-il deuil, et l'herbe de tous les champs séchera-t-elle à cause de la méchanceté des habitants qui sont en la terre ? Les bêtes et les oiseaux ont été consumés par la disette, parce que ces méchants ont dit : On ne verra pas notre dernière fin. 
\VS{5}Si tu cours avec des piétons et qu'ils te fatiguent, comment lutteras-tu avec les chevaux ? Et si tu te crois en sûreté dans une terre de paix, que feras-tu devant l'orgueil du Jourdain ?
\VS{6} Certainement, mêmes tes frères et la maison de ton père, ceux-là mêmes ont agi perfidement contre toi, eux-mêmes ont crié après toi à plein gosier ; ne les crois point, quoiqu'ils te parlent amicalement\FTNT{Pr. 26:25.}.
\VS{7}J'ai abandonné ma maison, j'ai quitté mon héritage, ce que mon âme aimait le plus je l'ai livré aux mains de ses ennemis.
\VS{8}Mon héritage a été pour moi comme un lion dans la forêt, il a poussé contre moi ses rugissements ; c'est pourquoi je l'ai pris en haine.
\VS{9}Mon héritage a-t-il donc été pour moi comme un oiseau de proie tacheté ? Les oiseaux de proie ne sont-ils pas autour de lui ? Venez, assemblez-vous, vous tous les animaux des champs, venez pour le dévorer\FTNT{Es. 56:9.} !
\VS{10}Plusieurs pasteurs ravagent ma vigne, ils foulent mon champ ; ils réduisent le champ de mes délices en un désert, en une désolation.
\VS{11}Ils le réduisent en un désert ; il est en deuil, il est désolé devant moi. Tout le pays est ravagé, car nul n'y prend garde.
\VS{12}Les destructeurs viennent sur tous les lieux élevés du désert, car l'épée de Yahweh dévore le pays d'un bout à l'autre ; il n'y a de paix pour aucune chair.
\VS{13}Ils ont semé du froment, et ils moissonnent des épines, ils se sont fatigués sans profit. Soyez honteux de vos récoltes, à cause de l'ardeur de la colère de Yahweh\FTNT{Lé. 26:16.}.
\VS{14}Ainsi parle Yahweh contre tous mes mauvais voisins, qui mettent la main sur l'héritage que j'ai donné à mon peuple d'Israël : Voici, je les arracherai de leur pays, et j'arracherai la maison de Juda du milieu d'eux.
\VS{15}Mais il arrivera qu'après que je les avoir arrachés, j'aurai encore compassion d'eux, et je les ramènerai chacun dans son héritage, chacun dans son pays\FTNT{De. 30:3.}.
\VS{16}Et il arrivera que s'ils apprennent bien les voies de mon peuple, pour jurer par mon Nom, en disant : Yahweh est vivant ! Comme ils ont enseigné à mon peuple à jurer par Baal, ils seront édifiés au milieu de mon peuple.
\VS{17}Mais s'ils n'écoutent pas, j'arracherai entièrement une telle nation, et je la ferai périr, dit Yahweh\FTNT{Es. 60:12.}.
\Chap{13}
\TextTitle{La ceinture pourrie, illustration du jugement}
\VerseOne{}Ainsi m'a parlé Yahweh : Va, et achète-toi une ceinture de lin et mets-la sur tes reins ; et ne la mets pas dans l'eau.
\VS{2}J'achetai donc une ceinture, selon la parole de Yahweh, et je la mis sur mes reins.
\VS{3}Et la parole de Yahweh me fut adressée pour la seconde fois, en disant :
\VS{4}Prends la ceinture que tu as achetée et qui est sur tes reins ; lève-toi, va-t'en vers l'Euphrate, et là, cache-la dans la fente d'un rocher.
\VS{5}J'allai donc et je la cachai près de l'Euphrate, comme Yahweh me l'avait ordonné.
\VS{6}Et il arriva que plusieurs jours après Yahweh me dit : Lève-toi, va vers l'Euphrate et reprends la ceinture que je t'avais ordonné d'y cacher.
\VS{7}Et j'allai vers l'Euphrate, je creusai, et je pris la ceinture dans le lieu où je l'avais cachée ; mais voici, la ceinture était pourrie, elle n'était plus bonne à rien.
\VS{8}Alors la parole de Yahweh me fut adressée, en disant :
\VS{9}Ainsi parle Yahweh : Je ferai ainsi pourrir l'orgueil de Juda et le grand orgueil de Jérusalem.
\VS{10}L'orgueil de ce peuple très méchant, qui refuse d'écouter mes paroles, qui marche selon les penchants de son cœur, et qui va après d'autres dieux, pour les servir et pour se prosterner devant eux, qu'il devienne comme cette ceinture qui n'est plus bonne à rien !
\VS{11}Car comme une ceinture est attachée aux reins d'un homme, ainsi je m'étais attaché toute la maison d'Israël et toute la maison de Juda, dit Yahweh, afin qu'elles soient mon peuple, mon Nom, ma louange, et ma gloire. Mais ils ne m'ont pas écouté.
\VS{12}Tu leur diras donc cette parole-ci : Ainsi parle Yahweh, le Dieu d'Israël : Toute outre sera remplie de vin. Et ils te diront : Ne savons-nous pas que toute outre sera remplie de vin ?
\VS{13}Mais tu leur diras : Ainsi parle Yahweh : Voici, je vais remplir d'ivresse tous les habitants de ce pays, les rois qui sont assis sur le trône de David, les sacrificateurs, les prophètes, et tous les habitants de Jérusalem.
\VS{14}Et je les briserai les uns contre les autres, les pères et les fils ensemble, dit Yahweh\FTNT{Es. 51:17-20 ; Ps. 60:5.} ; je n'aurai pas de compassion, je n'épargnerai pas, et je n'aurai pas de miséricorde ; rien ne m'empêchera de les détruire.
\VS{15}Écoutez et prêtez l'oreille ! Ne vous élevez pas ! Car Yahweh parle.
\VS{16}Donnez gloire à Yahweh, votre Dieu, avant qu'il fasse venir les ténèbres, avant que vos pieds se heurtent contre les montagnes du crépuscule ; vous attendrez la lumière, et il la changera en ombre de la mort, il la réduira en obscurité profonde\FTNT{Es. 59:9 ; Jn. 12:35.}.
\VS{17}Que si vous n'écoutez pas ceci, mon âme pleurera en secret, à cause de votre orgueil ; mes yeux verseront des larmes en abondance, ils se fondront en larmes, parce que le troupeau de Yahweh sera emmené captif\FTNT{La. 1:2-16.}.
\VS{18}Dis au roi et à la reine : Humiliez-vous et asseyez-vous sur la cendre ! Car elle est tombée de vos têtes, la couronne de votre gloire.
\VS{19}Les villes du midi sont fermées, il n'y a personne qui les ouvre ; tout Juda est transporté en captivité, il est transporté entièrement.
\VS{20}Levez vos yeux et voyez ceux qui viennent du nord. Où est le troupeau qui t'avait été donné, le troupeau qui faisait ta gloire ?
\VS{21}Que diras-tu quand il te punit ? Car tu les as enseignés à dominer en maîtres sur toi. Les douleurs ne te saisiront-elles pas, comme elles saisissent une femme qui enfante ?
\VS{22}Que si tu dis en ton cœur : Pourquoi cela m'arrive-t-il ? C'est à cause de la multitude de tes iniquités que les pans de ta robe sont relevés, et que tes talons sont violemment mis à nu\FTNT{Es. 47:2-3.}.
\VS{23}L'éthiopien peut-il changer sa peau et le léopard ses taches ? Pourriez-vous, aussi, faire quelque bien, vous qui êtes accoutumés à faire le mal ?
\VS{24}C'est pourquoi je les disperserai, comme du chaume, qui est emporté çà et là par le vent du désert.
\VS{25}Voilà ton sort, la portion que je te mesure, dit Yahweh, parce que tu m'as oublié, et que tu as mis ta confiance dans le mensonge.
\VS{26}A cause de cela, je relèverai les pans de ta robe sur ton visage, et ta honte se verra.
\VS{27}Tes adultères et tes hennissements, l'énormité de tes prostitutions sur les collines et dans les champs, tes abominations, je les ai vues. Malheur à toi, Jérusalem ! Ne seras-tu pas purifiée ? Jusqu'à quand cela durera-t-il ?
\Chap{14}
\TextTitle{Le pays frappé par la sécheresse}
\VerseOne{}La parole de Yahweh, qui fut adressée à Jérémie, à l'occasion de la sécheresse.
\VS{2}Juda est dans le deuil, et ses portes sont dans un état pitoyable. Ils sont tous en deuil, gisant par terre ; et les cris de Jérusalem montent au ciel.
\VS{3}Et les personnes distinguées envoient les petits chercher de l'eau, et les petits vont aux citernes, ne trouvent pas d'eau, et reviennent leurs vases vides ; ils sont honteux et confus, ils couvrent leur tête.
\VS{4}Parce que la terre est crevassée, parce qu'il n'y a pas eu de pluie dans le pays, les laboureurs sont honteux, ils se couvrent la tête.
\VS{5}Même la biche met bas son faon dans le champ et l'abandonne, parce qu'il n'y a pas d'herbe.
\VS{6}Et les ânes sauvages se tiennent sur les lieux élevés, humant l'air comme des serpents ; leurs yeux se consument, parce qu'il n'y a pas d'herbe.
\VS{7}Si nos iniquités témoignent contre nous, agis à cause de ton Nom, ô Yahweh\FTNT{Es. 59:12.} ! Car nos infidélités sont nombreuses, c'est contre toi que nous avons péché.
\VS{8}Toi qui es l'espérance d'Israël, son sauveur au temps de la détresse, pourquoi serais-tu dans le pays comme un étranger, comme un voyageur qui se détourne pour passer la nuit ?
\VS{9}Pourquoi serais-tu comme un homme stupéfait, et comme un héros qui ne peut sauver ? Or tu es au milieu de nous, ô Yahweh, et ton Nom est invoqué sur nous : Ne nous abandonne pas !
\VS{10}Voici ce que Yahweh dit de ce peuple : Parce qu'ils aiment à errer ainsi çà et là, et qu'ils ne savent retenir leurs pieds, Yahweh ne prend pas plaisir en eux, il se souvient maintenant de leurs iniquités, et il punit leurs péchés\FTNT{Os. 8:13.}.
\VS{11}Puis Yahweh me dit : N'intercède pas en faveur de ce peuple.
\VS{12}Quand ils jeûnent, je n'écouterai pas leurs cris ; et quand ils offrent des holocaustes et des offrandes, je n'y prendrai pas plaisir ; mais je les consumerai par l'épée, par la famine et par la peste.
\VS{13}Et je répondis : Ah ! ah ! Seigneur Yahweh ! Voici, les prophètes leur disent : Vous ne verrez pas l'épée, et vous n'aurez pas de famine ; mais je vous donnerai dans ce lieu-ci une paix assurée.
\VS{14}Et Yahweh me dit : C'est le mensonge ce que ces prophètes prophétisent en mon Nom ; je ne les ai pas envoyés, je ne leur ai pas donné d'ordre, je ne leur ai pas parlé ; ils vous prophétisent des visions de mensonge, des divinations, de l'idolâtrie et des tromperies de leur cœur\FTNT{De. 18:20-22 ; Ez. 13:2-3.}.
\VS{15}C'est pourquoi ainsi parle Yahweh sur les prophètes qui prophétisent en mon Nom, sans que je les ai envoyés, et qui disent : Il n'y aura ni épée ni la famine dans ce pays : Ces prophètes-là seront consumés par l'épée et par la famine.
\VS{16}Et le peuple à qui ils prophétisent sera jeté dans les rues de Jérusalem à cause de la famine et de l'épée ; et il n'y aura personne pour les enterrer, ni eux, ni leurs femmes, ni leurs fils, ni leurs filles ; je répandrai sur eux leur méchanceté.
\VS{17}Tu leur diras donc cette parole-ci : Que mes yeux se fondent en larmes nuit et jour, et qu'ils ne cessent pas\FTNT{La. 1:16.} ; car la vierge, fille de mon peuple, a été frappée d'un grand coup, d'une plaie très douloureuse.
\VS{18}Si je sors dans les champs, voici les gens tués par l'épée ; si j'entre dans la ville, voici les gens consumés par la faim ; même le prophète et le sacrificateur parcourent le pays, sans savoir où ils vont.
\VS{19}As-tu entièrement rejeté Juda, et ton âme a-t-elle Sion en horreur ? Pourquoi nous frappes-tu sans qu'il y ait pour nous de guérison ? On attend la paix, mais il n'y a rien de bon, un temps de guérison, et voici la terreur !
\VS{20}Yahweh, nous reconnaissons notre méchanceté, l'iniquité de nos pères ; car nous avons péché contre toi\FTNT{Ps. 106:6 ; Da. 9:8.}.
\VS{21}Ne nous rejette pas, à cause de ton Nom, et ne déshonore pas le trône de ta gloire ! Souviens-toi de ton alliance avec nous, et ne la romps pas !
\VS{22}Parmi les vanités\FTNT{Ce terme veut aussi dire « idole ».} des nations, y en a-t-il qui fassent pleuvoir, et les cieux donnent-ils des ondées\FTNT{Es. 30:23 ; Ac. 14:17.} ? N'est-ce pas toi, ô Yahweh, notre Dieu ? C'est pourquoi nous nous attendons à toi, car c'est toi qui as fait toutes ces choses.
\Chap{15}
\TextTitle{Yahweh fermement décidé à juger son peuple}
\VerseOne{}Et Yahweh me dit : Quand Moïse et Samuel se tiendraient devant moi, je n'aurais pourtant point d'affection pour ce peuple ; chasse-les de devant ma face, et qu'ils sortent.
\VS{2}Que s'ils te disent : Où irons-nous ? Tu leur répondras : Ainsi parle Yahweh : Ceux qui sont destinés à la mort iront à la mort ; et ceux qui sont destinés à l’épée iront à l’épée ; et ceux qui sont destinés à la famine, iront à la famine ; et ceux qui sont destinés à la captivité iront en captivité\FTNT{Za. 11:9.} !
\VS{3}J'établirai aussi sur eux quatre espèces de punitions, dit Yahweh, l'épée pour tuer, et les chiens pour traîner, et les oiseaux des cieux, et les bêtes de la terre pour dévorer et pour détruire. 
\VS{4}Et je les livrerai à être agités par tous les Royaumes de la terre, à cause de Manassé, fils d'Ezéchias, Roi de Juda, pour les choses qu'il a faites dans Jérusalem. 
\VS{5}Car qui aurait compassion de toi, Jérusalem, ou qui te plaindrait ? Ou qui se détournerait pour s'informer de ta paix ? 
\VS{6}Tu m'as abandonné, dit Yahweh, et tu t'en es allée en arrière ; c'est pourquoi j'étends ma main sur toi, et je te détruis, je suis las d'avoir compassion.
\VS{7}Je les vanne avec un van aux portes du pays\FTNT{Mt. 3:12.} ; je prive d'enfants, je fais périr mon peuple, et ils ne se sont pas détournés de leurs voies.
\VS{8}Je multiplie ses veuves plus que le sable de la mer ; je fais venir sur eux, sur la mère du jeune homme, le dévastateur en plein midi ; je fais tomber subitement sur elle l'angoisse et les frayeurs.
\VS{9}Celle qui en avait enfanté sept languit, elle rend l'âme ; son soleil se couche pendant qu'il est encore jour\FTNT{Am. 8:9.} ; elle est confuse, couverte de honte. Ceux qui restent, je les livre à l'épée devant leurs ennemis, dit Yahweh.
\VS{10}Malheur à moi, ô ma mère, de ce que tu m'as enfanté\FTNT{Job. 3:1-2.} pour être un homme de contestation et un homme de dispute pour tout le pays ! Je n'emprunte ni ne prête, et néanmoins tous me maudissent et me méprisent.
\VS{11}Alors Yahweh dit : En vérité tout ira bien pour ton reste ; en vérité je ferai que l’ennemi te traite bien au temps du malheur, et au temps de la détresse.
\VS{12}Le fer brisera-t-il le fer du nord et l'airain ?
\VS{13}Je livre au pillage, sans en faire le prix, tes richesses et tes trésors, et cela à cause de tous tes péchés, sur tout ton territoire.
\VS{14}Je te fais passer avec tes ennemis dans un pays que tu ne connais pas, car le feu de ma colère s'est allumé, il brûle sur vous\FTNT{De. 32:22.}.
\TextTitle{La mise à part de Jérémie}
\VS{15}Yahweh ! Tu sais tout, souviens-toi de moi, visite-moi, venge-moi de ceux qui me persécutent\FTNT{Ps. 106:4.} ! Ne m'enlève pas, tandis que tu te montres lent à la colère ! Sache que je supporte l'opprobre à cause de toi.
\VS{16}J'ai trouvé tes paroles, je les ai aussitôt dévorées\FTNT{Ez. 3:3 ; Ap. 10:9.} ; tes paroles ont fait la joie et l'allégresse de mon cœur ; car ton Nom est invoqué sur moi, ô Yahweh, Dieu des armées !
\VS{17}Je ne me suis pas assis dans l'assemblée des moqueurs, et je ne m'y suis pas réjoui ; mais je me suis assis tout seul à cause de ta main, car tu me remplissais d'indignation.
\VS{18}Pourquoi ma douleur est-elle continuelle ? Pourquoi ma plaie est-elle incurable et refuse-t-elle d'être guérie ? Serais-tu pour moi comme une source trompeuse, comme des eaux qui ne durent pas ?
\VS{19}C'est pourquoi ainsi parle Yahweh : Si tu reviens, je te ramènerai, et tu te tiendras devant moi ; et si tu sépares la chose précieuse de la méprisable, tu seras comme ma bouche. Qu'ils reviennent vers toi, mais toi, ne retourne pas vers eux.
\VS{20}Je ferai que tu sois pour ce peuple une muraille d'airain bien forte ; ils combattront contre toi, mais il n'auront pas le dessus contre toi ; car je suis avec toi pour te sauver et te délivrer, dit Yahweh.
\VS{21}Et je te délivrerai de la main des malins, et te rachèterai de la main des méchants.
\Chap{16}
\TextTitle{Célibat de Jérémie, illustration du jugement sur Juda}
\VerseOne{}Puis la parole de Yahweh me fut adressée, en disant :
\VS{2}Tu ne prendras pas de femme, et tu n'auras pas de fils ni de filles dans ce lieu-ci.
\VS{3}Car ainsi parle Yahweh sur les fils et les filles qui naîtront en ce lieu-ci, sur leurs mères qui les auront enfantés, et sur leurs pères qui les auront engendrés dans ce pays :
\VS{4}Ils mourront de maladie mortelle ; ils ne seront ni pleurés ni enterrés ; ils seront comme du fumier sur la face du sol ; ils seront consumés par l'épée et par la famine ; et leurs cadavres seront la pâture des oiseaux des cieux et des bêtes de la terre.
\VS{5}Car ainsi parle Yahweh : N'entre pas dans une maison de deuil, ne vas pas te lamenter ni te plaindre avec eux ; car j'ai retiré de ce peuple dit Yahweh, ma paix, ma miséricorde et mes compassions.
\VS{6}Et les grands et petits mourront dans ce pays ; ils ne seront pas enterrés ; on ne les pleurera pas, on ne se fera pas d'incision, et on ne se rasera pas pour eux\FTNT{Lé. 19:28 ; De. 14:1 ; Ez. 7:11}.
\VS{7}On ne rompra pas le pain dans le deuil pour consoler quelqu'un au sujet d'un mort, et on ne leur donnera pas à boire de la coupe de consolation pour leur père ou pour leur mère.
\VS{8}Aussi n'entre pas non plus dans une maison de festin pour t'asseoir avec eux, pour manger et pour boire.
\VS{9}Car ainsi parle Yahweh des armées, le Dieu d'Israël : Voici, je vais faire cesser dans ce lieu-ci, devant vos yeux et en vos jours, les cris de joie et les cris d'allégresse, la voix de l'époux et la voix de l'épouse.
\VS{10}Et il arrivera que quand tu annonceras à ce peuple toutes ces paroles-là, ils te diront : Pourquoi Yahweh parle-t-il de tout ce grand mal contre nous ? Quelle est notre iniquité ? Quel est le péché que nous avons commis contre Yahweh, notre Dieu ?
\VS{11}Et tu leur diras : Parce que vos pères m'ont abandonné, dit Yahweh, et sont allés après d'autres dieux, et les ont servis, et se sont prosternés devant eux, et m'ont abandonné et n'ont pas gardé ma loi ; 
\VS{12}et que vous avez fait le mal plus encore que vos pères. Car voici, chacun de vous marche selon les penchants de son mauvais cœur pour ne pas m'écouter.
\VS{13}A cause de cela, je vous jetterai de ce pays dans un pays que vous n'avez pas connu, ni vous ni vos pères ; et là, vous servirez jour et nuit les autres dieux, car je ne vous aurai pas fait grâce\FTNT{De. 28:64-65.}.
\VS{14}Néanmoins voici, les jours viennent, dit Yahweh, qu'on ne dira plus : Yahweh est vivant, lui qui a fait monter les fils d'Israël du pays d'Egypte !
\VS{15}Mais on dira : Yahweh est vivant, lui qui a fait monter les fils d'Israël du pays du nord et de tous les pays où il les avait chassés ; après que je les aurai ramenés dans leur pays, que j'avais donné à leurs pères.
\VS{16}Voici, j'envoie plusieurs pêcheurs, dit Yahweh, et ils les pêcheront ; et ensuite, j'enverrai plusieurs chasseurs, et ils les chasseront de toutes les montagnes et de toutes les collines, et des fentes des rochers.
\VS{17}Car mes yeux sont sur toutes leurs voies, elles ne sont pas cachées devant ma face, et leur iniquité n'est pas couverte devant mes yeux\FTNT{Pr. 5:21 ; Job. 34:21.}.
\VS{18}Mais premièrement je leur rendrai le double de leur iniquité et de leur péché, parce qu'ils ont souillé mon pays par les cadavres de leurs idoles, et parce qu'ils ont rempli mon héritage de leurs abominations.
\VS{19}Yahweh, qui est ma force et ma forteresse, et mon refuge au jour de la détresse ! Les nations viendront à toi des extrémités de la terre, et diront : Certes nos pères ont hérité le mensonge et la vanité, et les choses auxquelles il n'y a pas de profit.
\VS{20}L'homme se fera-t-il lui-même des dieux, qui ne sont pas dieux ?
\VS{21}C'est pourquoi voici, je leur fais connaître, cette fois, je leur fais connaître ma main et ma force ; et ils sauront que mon Nom est Yahweh.
\Chap{17}
\TextTitle{Le caractère sinueux du cœur}
\VerseOne{}Le péché de Juda est écrit avec un burin de fer, et avec une pointe de diamant ; il est gravé sur la table de leur cœur, et sur les cornes de leurs autels.
\VS{2}De sorte que leurs fils se souviennent de leurs autels, et de leurs poteaux d'Asherah, auprès des arbres verts sur les hautes collines.
\VS{3}Ma montagne, je livre par les champs tes richesses et tous tes trésors au pillage ; tes hauts lieux sont pleins de péché sur tout ton territoire.
\VS{4}Et toi, et ceux qui sont avec toi, vous laisserez vacant l'héritage que je t'avais donné ; et je t'asservirai à tes ennemis dans un pays que tu ne connais pas ; car vous avez allumé le feu de ma colère, et il brûlera à toujours.
\VS{5}Ainsi parle Yahweh : Maudit soit l'homme qui se confie dans l'homme, et qui fait de la chair sa force, et dont le cœur se retire de Yahweh !
\VS{6}Car il sera comme la bruyère dans le désert, et il ne voit pas venir le bien ; mais il demeure dans des lieux brûlés du désert, dans une terre salée et inhabitable.
\VS{7}Béni soit l'homme qui se confie en Yahweh, et dont Yahweh est l'espérance !
\VS{8}Il est comme un arbre planté près des eaux\FTNT{Ps. 23.}, et qui étend ses racines le long d'une eau courante ; quand la chaleur vient, il ne s'en aperçoit pas, et sa feuille reste verte ; il n'est pas en peine dans l'année de la sécheresse, et ne cesse de porter du fruit.
\VS{9}Le cœur est rusé et désespérément malin par-dessus tout : Qui peut le connaître\FTNT{Ps. 64:7.} ?
\VS{10}Je suis Yahweh, qui sonde le cœur, et qui éprouve les reins ; même pour rendre à chacun selon sa voie, et selon le fruit de ses actions.
\VS{11}Celui qui acquiert des richesses, sans observer la justice, est une perdrix qui couve ce qu'elle n'a pas pondu ; il les laissera au milieu de ses jours, et à la fin il sera trouvé insensé\FTNT{Ec. 4:8}.
\VS{12}Le lieu de notre sanctuaire est un trône de gloire, un lieu haut élevé dès le commencement.
\VS{13}Yahweh, qui es l'espérance d'Israël ! Tous ceux qui t'abandonnent seront honteux : Ceux qui se détournent de moi seront écrits sur la terre, car ils abandonnent la source des eaux vives, Yahweh\FTNT{Es. 1:28 ; Ps. 73:28.}.
\VS{14}Yahweh, guéris-moi, et je serai guéri ; sauve-moi, et je serai sauvé ; car tu es ma louange.
\VS{15}Voici, ceux-ci me disent : Où est la parole de Yahweh ? Qu'elle vienne présentement\FTNT{Es. 5:19 ; Ez. 12:23 ; 2 Pi. 3:3-4.} !
\VS{16}Mais je ne me suis pas avancé plus qu’un pasteur après toi, je n'ai pas non plus désiré le jour du malheur, tu le sais ; et ce qui est sorti de mes lèvres est présent devant toi.
\VS{17}Ne sois pas pour moi un sujet d'effroi, toi, mon refuge au jour du malheur !
\VS{18}Que ceux qui me persécutent soient honteux, mais que je ne sois pas honteux ; qu'ils soient brisés, mais que je ne sois pas brisé ! Fais venir sur eux le jour du malheur, frappe-les d'une double plaie !
\TextTitle{Message à propos du sabbat}
\VS{19}Ainsi m'a parlé Yahweh : Va, et tiens-toi debout à la porte des fils du peuple, par laquelle les rois de Juda entrent et par laquelle ils sortent, et à toutes les portes de Jérusalem.
\VS{20}Tu leur diras : Ecoutez la parole de Yahweh, rois de Juda, et vous tous homme de Juda, et vous tous habitants de Jérusalem qui entrez par ces portes !
\VS{21}Ainsi parle Yahweh : Prenez garde à vos âmes ; ne portez aucun fardeau le jour du sabbat, et ne les faites pas passer par les portes de Jérusalem\FTNT{Né. 13:19.}.
\VS{22}Ne faites sortir de vos maisons aucun fardeau le jour du sabbat, et ne faites aucune œuvre ; mais sanctifiez le jour du sabbat, comme je l'ai ordonné à vos pères\FTNT{Ex. 20:8 ; Ex. 23:12.}.
\VS{23}Mais ils n'ont pas écouté, ils n'ont pas prêté l'oreille ; ils ont raidi leur cou, pour ne pas écouter et ne pas recevoir d'instruction.
\VS{24}Il arrivera donc, si vous m'écoutez attentivement, dit Yahweh, pour ne faire passer aucun fardeau par les portes de cette ville le jour du sabbat, et si vous sanctifiez le jour du sabbat, en ne faisant aucune œuvre ce jour-là,
\VS{25}que les rois et les chefs, ceux qui sont assis sur le trône de David, montés sur des chars et sur des chevaux, eux et les chefs d'entre eux, les hommes de Juda et les habitants de Jérusalem, entreront par les portes de cette ville, et cette ville sera habitée à toujours.
\VS{26}On viendra aussi des villes de Juda et des environs de Jérusalem, et du pays de Benjamin, et du bas pays, des montagnes et du midi, pour apporter des holocaustes, des sacrifices, des offrandes et de l'encens ; pour apporter aussi des sacrifices de louanges dans la maison de Yahweh.
\VS{27}Mais si vous ne m'écoutez pas pour sanctifier le jour du sabbat, pour ne porter aucun fardeau, et n'en faire entrer aucun par les portes de Jérusalem le jour du sabbat, je mettrai le feu à ses portes, et il consumera les palais de Jérusalem et ne s'éteindra pas\FTNT{2 R. 25:9.}.
\Chap{18}
\TextTitle{La maison du potier ; appel à la repentance et avertissement}
\VerseOne{}Cette parole fut adressée à Jérémie de la part de Yahweh, disant :
\VS{2}Lève-toi et descends dans la maison d'un potier ; et là, je te ferai entendre mes paroles.
\VS{3}Je descendis donc dans la maison d'un potier, et voici, il faisait son ouvrage, assis sur sa selle.
\VS{4}Et le vase qu'il faisait avec l'argile qu'il tenait dans sa main, fut gâté ; et il en fit encore un autre vase, comme il lui sembla bon de le faire.
\VS{5}Alors la parole de Yahweh me fut adressée, en disant :
\VS{6}Maison d'Israël, ne puis-je pas faire de vous comme a fait ce potier ? Dit Yahweh. Voici, comme l'argile est dans la main d'un potier, ainsi vous êtes dans ma main, maison d'Israël !
\VS{7}En un instant je parle contre une nation et contre un royaume, pour arracher, pour démolir, et pour détruire ;
\VS{8}mais si cette nation, contre laquelle j'ai parlé, revient de sa méchanceté, je me repentirai aussi du mal que j'avais pensé de lui faire\FTNT{Jon. 3:6-10.}.
\VS{9}Et si en un instant je parle d'une nation et d'un royaume, pour l'édifier et pour le planter ;
\VS{10} et que cette nation fasse ce qui est mal à mes yeux, en sorte qu'elle n'écoute pas ma voix, je me repentirai aussi du bien que j'avais dit que je lui ferais.
\VS{11}Or donc, parle maintenant aux hommes de Juda et aux habitants de Jérusalem, en disant : Ainsi parle Yahweh : Voici, je projette du mal contre vous, et je forme un dessein contre vous. Détournez-vous donc chacun de votre mauvaise voie, et amendez votre voie et vos actions !
\VS{12}Et ils répondent : Il n'y a plus d'espérance ; c'est pourquoi nous suivrons nos pensées, chacun de nous fera selon les penchants de son mauvais cœur.
\VS{13}C'est pourquoi ainsi parle Yahweh : Demandez maintenant aux nations ! Qui a entendu de telles choses ? La vierge d'Israël a fait une chose très horrible\FTNT{1 Co. 5:1.}.
\VS{14}La neige du Liban abandonnerait-elle le rocher du champ ? Ou les eaux fraîches et ruisselantes qui viennent de loin tariraient-elles ?
\VS{15}Mais mon peuple m'a oublié, et il brûle de l'encens à ce qui n'est que vanité, et qui les a fait chanceler leurs voies, pour les faire retirer des anciens sentiers, afin de marcher dans les sentiers d'un chemin non frayé ;
\VS{16}pour faire venir sur leur pays une désolation et un opprobre perpétuel ; quiconque passera par là, en sera étonné et secouera la tête.
\VS{17}Je les disperserai devant l'ennemi, comme par le vent d'orient ; je leur tournerai le dos, et non pas la face, au jour de leur calamité\FTNT{Es. 27:8.}.
\VS{18}Et ils ont dit : Venez, et faisons des complots contre Jérémie ! Car la loi ne périra pas chez le sacrificateur, ni le conseil chez le sage, ni la parole chez le prophète. Venez, et tuons-le avec la langue, et ne soyons pas attentifs à ses discours !
\VS{19}Yahweh ! Fais attention à moi, et écoute la voix de ceux qui contestent avec moi !
\VS{20}Le mal sera-t-il rendu pour le bien\FTNT{Ps. 35:12 ; Ps. 109:5.} ? Car ils ont creusé une fosse pour mon âme. Souviens-toi que je me suis tenu devant toi, afin de parler pour leur bien, et afin de détourner d'eux ta grande colère.
\VS{21}C'est pourquoi livre leurs fils à la famine, et fais couler leur sang à coups d’épée ; que leurs femmes soient privées d'enfants, et deviennent veuves, et que leurs maris soient enlevés par la mort ; et leurs jeunes gens frappés par l'épée dans la bataille\FTNT{Ps. 109:9-13.} !
\VS{22}Qu'on entende le cri de leurs maisons, quand tu feras venir subitement des troupes contre eux ! Car ils ont creusé une fosse pour me prendre, et ils ont caché des pièges pour mes pieds.
\VS{23}Or tu sais, ô Yahweh ! Que tout leur conseil est contre moi pour me mettre à mort ; ne sois pas apaisé à l'égard de leur iniquité, et n'efface pas leur péché de devant ta face, mais qu'on les fasse tomber en ta présence ; agis contre eux au temps de ta colère.
\Chap{19}
\TextTitle{Le vase brisé : Image de Juda}
\VerseOne{}Ainsi a parlé Yahweh : Va, et achète un vase de terre d'un potier, et prends avec toi des anciens du peuple et des anciens des sacrificateurs.
\VS{2}Et sors à la vallée de Ben-Hinnom, qui est auprès de l'entrée de la porte de la poterie, et crie là les paroles que je te dirai.
\VS{3}Dis donc : Rois de Juda, et vous, habitants de Jérusalem, écoutez la parole de Yahweh ! Ainsi parle Yahweh des armées, le Dieu d'Israël : Voici, je vais faire venir sur ce lieu-ci un mal, tel que quiconque l'entendra, les oreilles lui tinteront\FTNT{1 S. 3:11 ; 2 R. 21:12.} ; 
\VS{4}parce qu'ils m'ont abandonné, et qu'ils ont profané ce lieu, et y ont brûlé de l'encens à d'autres dieux, que ni eux, ni leurs pères, ni les rois de Juda n'ont connus, et parce qu'ils ont rempli ce lieu du sang des innocents ;
\VS{5}et qu'ils ont bâti des hauts lieux à Baal, afin de brûler au feu leurs fils pour en faire des holocaustes à Baal : Ce que je n'avais pas ordonné, et dont je n'avais pas parlé, et qui ne m'était pas monté à cœur.
\VS{6}A cause de cela, voici les jours viennent, dit Yahweh, que ce lieu-ci ne sera plus appelé Topheth, ni la vallée de Ben-Hinnom, mais la vallée de la tuerie.
\VS{7}Et j'anéantirai dans ce lieu-ci le conseil de Juda et de Jérusalem ; et je les ferai tomber par l'épée devant leurs ennemis et par la main de ceux qui cherchent leur vie ; et je donnerai leurs cadavres en pâture aux oiseaux des cieux et aux bêtes de la terre.
\VS{8}Je ferai de cette ville un objet de désolation et de moquerie ; quiconque passera près d'elle sera étonné et sifflera à cause de toutes ses plaies.
\VS{9}Et je leur ferai manger la chair de leurs fils et la chair de leurs filles ; et chacun mangera la chair de son compagnon durant le siège, et dans la détresse où les réduiront leurs ennemis et ceux qui cherchent leur vie\FTNT{Lé. 26:29 ; De. 28:53 ; La. 2:20.}.
\VS{10}Puis tu briseras le vase, sous les yeux des hommes qui seront allés avec toi.
\VS{11}Et tu leur diras : Ainsi parle Yahweh des armées : Je briserai ce peuple et cette ville, de même qu'on brise un vase de potier, qui ne peut être réparé. Et ils seront enterrés à Topheth parce qu'il n'y aura plus d'autre lieu pour les enterrer.
\VS{12}Je ferai ainsi à ce lieu-ci, dit Yahweh, et à ses habitants, et je rendrai cette ville semblable à Topheth ;
\VS{13}et les maisons de Jérusalem, et les maisons des rois de Juda, seront impures comme le lieu de Topheth, à cause de toutes les maisons sur les toits desquelles ils brûlaient de l'encens à toute l'armée des cieux, et faisaient des libations à d'autres dieux.
\VS{14}Puis Jérémie revint de Topheth, là où Yahweh l'avait envoyé pour prophétiser. Et il se tint debout dans le parvis de la maison de Yahweh, et il dit à tout le peuple :
\VS{15}Ainsi parle Yahweh des armées, le Dieu d'Israël : Voici, je vais faire venir sur cette ville et sur toutes ses villes tout le mal que j'ai prononcés contre elle, parce qu'ils ont raidi leur cou pour ne pas écouter mes paroles.
\Chap{20}
\TextTitle{Paschhur outrage Jérémie}
\VerseOne{}Alors Paschhur, fils d'Immer, qui était sacrificateur et inspecteur en chef dans la maison de Yahweh, entendit Jérémie qui prophétisait ces choses.
\VS{2}Et Paschhur frappa le prophète Jérémie, et le mit dans la prison qui était à la porte supérieure de Benjamin, dans la maison de Yahweh.
\VS{3}Et il arriva que dès le lendemain, que Paschhur tira Jérémie hors de la prison. Et Jérémie lui dit : Yahweh ne t'appelle pas du nom de Paschhur, mais Magor-Missabib\FTNT{« Magor-Missabib » veut dire « terreur de chaque côté ».}.
\VS{4}Car ainsi parle Yahweh : Voici, je vais te livrer à la terreur, toi et tous tes amis qui tomberont par l'épée de leurs ennemis, et tes yeux le verront. Je livrerai tous ceux de Juda entre les mains du roi de Babylone, qui les transportera à Babylone et les frappera de l'épée.
\VS{5}Et je livrerai toutes les richesses de cette ville, et tout son travail, et tout ce qu'elle a de précieux, je livrerai, dis-je, tous les trésors des rois de Juda entre les mains de leurs ennemis, qui les pilleront, les enlèveront et les conduiront à Babylone.
\VS{6}Et toi, Paschhur, et tous ceux qui demeurent dans ta maison, vous irez en captivité ; tu iras à Babylone, tu y mourras, et y seras enterré, toi et tous tes amis auxquels tu as prophétisé le mensonge.
\TextTitle{Jérémie gémit auprès de Yahweh}
\VS{7}Ô Yahweh ! Tu m'as persuadé, et je me suis laissé persuader ; tu m'as saisi, et tu m'as vaincu. Je suis un objet de moquerie chaque jour, chacun se moque de moi.
\VS{8}Car depuis que je parle, je crie, je crie violence et dévastation ! Et la parole de Yahweh est pour moi un sujet d'opprobre et de moquerie chaque jour\FTNT{Es. 57:4.}.
\VS{9}C'est pourquoi j'ai dit : Je ne ferai plus mention de lui, je ne parlerai plus en son Nom, mais il y a eu dans mon cœur comme un feu ardent, renfermé dans mes os ; je me fatigue à le contenir, et je ne le puis.
\VS{10}Car j'entends les mauvais propos de plusieurs, la frayeur m'a saisi de tous côtés ; rapportez, disent-ils, et nous le rapporterons ! Tous ceux qui étaient en paix avec moi observent si je bronche, et disent : Peut-être se laissera-t-il séduire, et nous le vaincrons, nous tirerons vengeance de lui !
\VS{11}Mais Yahweh est avec moi comme un héros puissant ; c'est pourquoi ceux qui me persécutent seront renversés, ils ne me vaincront pas ; ils seront honteux, car ils n'ont pas réussi : Ce sera une honte éternelle qui ne s'oubliera jamais.
\VS{12}Yahweh des armées qui éprouve les justes, qui voit les reins et les cœurs, fais que je voie ta vengeance s'exercer contre eux, car je t'ai découvert ma cause.
\VS{13}Chantez à Yahweh, louez Yahweh ! Car il délivre l'âme des pauvres de la main des méchants.
\VS{14}Maudit soit le jour où je suis né ! Que le jour où ma mère m'a enfanté ne soit pas béni !
\VS{15}Maudit soit l'homme qui porta cette nouvelle à mon père, en lui disant : Un fils mâle t'est né, et qui le combla de joie !
\VS{16}Que cet homme-là soit comme les villes que Yahweh a renversées sans s'en repentir ! Qu'il entende la clameur le matin, et le cri de guerre au temps du midi\FTNT{Ge. 19:24-25 ; So. 2:4.} !
\VS{17}Que ne m'a-t-on fait mourir dans le sein de ma mère ! Pourquoi ma mère ne m'a-t-elle pas servi de sépulcre ? Et pourquoi n'est-elle pas restée éternellement enceinte ?
\VS{18}Pourquoi suis-je sorti de son sein pour ne voir que peine et douleur, et pour consumer mes jours dans la honte ?
\Chap{21}
\TextTitle{Prophétie sur les rois de Juda : Sédécias}
\VerseOne{}La parole qui fut adressée à Jérémie de la part de Yahweh, lorsque le roi Sédécias envoya vers lui Paschhur, fils de Malkija, et Sophonie, fils de Maaséja, le sacrificateur, pour lui dire :
\VS{2}Consulte maintenant Yahweh pour nous ; car Nebucadnetsar, roi de Babylone, combat contre nous ; peut-être que Yahweh fera-t-il en notre faveur un de ses miracles, afin qu'il se retire de nous.
\VS{3}Et Jérémie leur dit : Vous direz ainsi à Sédécias :
\VS{4}Ainsi parle Yahweh, le Dieu d'Israël : Voici, je vais détourner les armes de guerre qui sont dans vos mains, et avec lesquelles vous combattez en dehors des murailles contre le roi de Babylone et contre les Chaldéens qui vous assiègent, et je les rassemblerai au milieu de cette ville.
\VS{5}Et je combattrai contre vous, avec une main étendue, et avec un bras puissant, avec colère, avec fureur, et avec une grande indignation.
\VS{6}Et je frapperai les habitants de cette ville, les hommes, et les bêtes ; et ils mourront d'une grande peste.
\VS{7}Et après cela, dit Yahweh, je livrerai Sédécias, roi de Juda, et ses serviteurs, et le peuple, et ceux qui dans cette ville survivront à la peste, à l'épée et à la famine, entre les mains de Nebucadnetsar\FTNT{2 R. 24 et 25. }, roi de Babylone, et entre les mains de leurs ennemis, et entre les mains de ceux qui cherchent leur vie ; et il les frappera au tranchant de l'épée, il ne les épargnera pas, il n'en aura pas de compassion, il n'en aura pas de pitié.
\VS{8}Tu diras aussi à ce peuple : Ainsi parle Yahweh : Voici, je mets devant vous le chemin de la vie et le chemin de la mort\FTNT{De. 30:19.}.
\VS{9}Quiconque restera dans cette ville mourra par l'épée, ou par la famine, ou par la peste, mais celui qui en sortira, et se rendra aux Chaldéens qui vous assiègent vivra et aura sa vie pour butin.
\VS{10}Car je dresse ma face en mal et non en bien contre cette ville, dit Yahweh ; elle sera livrée entre les mains du roi de Babylone, et il la brûlera par le feu.
\VS{11}Et quant à la maison du roi de Juda : Ecoutez la parole de Yahweh !
\VS{12}Maison de David ! Ainsi parle Yahweh : Rendez la justice dès le matin, et délivrez celui qui aura été pillé d'entre les mains de l'oppresseur, de peur que ma fureur ne sorte comme un feu, et qu'elle ne brûle sans qu'on puisse l'éteindre, à cause de la méchanceté de vos actions.
\VS{13}Voici, j'en veux à toi qui habites dans la vallée, et qui es le rocher de la plaine, dit Yahweh ; à vous qui dites : Qui descendra contre nous, et qui entrera dans nos demeures ?
\VS{14}Et je vous punirai selon le fruit de vos actions, dit Yahweh ; et je mettrai le feu dans sa forêt qui consumera tout ce qui est autour d'elle\FTNT{Ez. 21:2-3.}.
\Chap{22}
\TextTitle{Sédécias averti de la destruction de Jérusalem}
\VerseOne{}Ainsi parle Yahweh : Descends dans la maison du roi de Juda, et là prononce cette parole.
\VS{2}Tu diras donc : Ecoute la parole de Yahweh, ô roi de Juda qui es assis sur le trône de David, toi et tes serviteurs, et ton peuple, qui entrez par ces portes !
\VS{3}Ainsi parle Yahweh : Faites droit et justice ; et délivrez celui qui aura été pillé d'entre les mains de l'oppresseur ; ne maltraitez pas l'orphelin, ni l'étranger, ni la veuve ; et n'usez d'aucune violence, et ne répandez pas le sang innocent dans ce lieu-ci.
\VS{4}Car si vous mettez exactement en effet cette parole, alors les rois qui sont assis à la place de David sur son trône, montés sur des chars et sur des chevaux, entreront par les portes de cette maison, eux et leurs serviteurs, et leur peuple.
\VS{5}Mais si vous n'écoutez pas ces paroles, je le jure par moi-même\FTNT{Es. 45:23 ; Hé. 6:13.}, dit Yahweh, que cette maison deviendra une ruine.
\VS{6}Car ainsi parle Yahweh sur la maison du roi de Juda : Tu es pour moi un Galaad, et le sommet du Liban ; mais certainement, je ferai de toi un désert, une ville sans habitants.
\VS{7}Je prépare contre toi des destructeurs, chacun avec ses armes, qui couperont tes cèdres de choix, et les jetteront au feu.
\VS{8}Et plusieurs nations passeront près de cette ville, et chacun dira à son compagnon : Pourquoi Yahweh a-t-il fait ainsi à cette grande ville\FTNT{De. 29:24-28 ; 1 R. 9:8.} ?
\VS{9}Et on dira : C'est parce qu'ils ont abandonné l'alliance de Yahweh, leur Dieu, et qu'ils se sont prosternés devant d'autres dieux et les ont servis.
\TextTitle{Prophétie sur les rois de Juda : Joachaz (Schallum)}
\VS{10}Ne pleurez pas celui qui est mort, et ne vous lamentez pas sur lui ; mais pleurez amèrement celui qui s'en va, car il ne reviendra plus, il ne reverra plus le pays de sa naissance.
\VS{11}Car ainsi parle Yahweh sur Schallum, fils de Josias, roi de Juda, qui régnait à la place de Josias, son père, et qui est sorti de ce lieu : Il n'y reviendra plus ;
\VS{12}mais il mourra dans le lieu où on l'a transporté, et ne verra plus ce pays.
\TextTitle{Prophétie sur les rois de Juda : Jojakim}
\VS{13}Malheur à celui qui bâtit sa maison par l'injustice, et ses chambres hautes sans droiture ; qui fait travailler son prochain pour rien, sans lui donner le salaire de son travail\FTNT{Lé. 19:13 ; De. 24:14-15 ; Ha. 2:9.}.
\VS{14}Qui dit : Je me bâtirai une grande maison et des chambres spacieuses, et qui s'y fait percer des fenêtres ; elle est lambrissée de cèdre, et peinte de vermillon.
\VS{15}Régneras-tu, parce que tu t’enfermes dans du cèdre ? Ton père n'a-t-il pas mangé et bu ? Quand il a fait jugement et justice, alors il a prospéré.
\VS{16}Il jugeait la cause du pauvre et de l'indigent, alors il a prospéré. N'est-ce pas là me connaître ? Dit Yahweh.
\VS{17}Mais tes yeux et ton cœur ne sont adonnés qu'à ton gain déshonnête, qu'à répandre le sang innocent, qu'à faire du tord et qu'à opprimer.
\VS{18}C'est pourquoi ainsi parle Yahweh sur Jojakim, fils de Josias, roi de Juda : On ne le pleurera pas en disant : Hélas, mon frère ! et hélas, ma sœur ! On ne le pleurera pas en disant : Hélas, seigneur ! Et hélas, sa majesté !
\VS{19}Il sera enterré de la sépulture d'un âne, étant traîné et jeté hors des portes de Jérusalem.
\TextTitle{Prophétie sur les rois de Juda : Jojakin}
\VS{20}Monte sur le Liban, et crie ! Donne de la voix sur le Basan ! Crie du haut d'Abarim ! A cause que tous ceux qui t'aiment sont brisés.
\VS{21}Je t'ai parlé durant ta grande prospérité, mais tu disais : Je n'écouterai pas ; telle est ta voie depuis ta jeunesse, que tu n'as pas écouté ma voix.
\VS{22}Tous tes pasteurs seront la pâture du vent, et ceux qui t'aiment iront en captivité ; certainement tu seras alors honteuse et confuse, à cause de toute ta méchanceté.
\VS{23}Toi qui habites sur le Liban, et qui fais ton nid dans les cèdres, que tu seras à plaindre quand les douleurs t'atteindront, les douleurs comme celles d'une femme qui enfante.
\VS{24}Je suis vivant, dit Yahweh, que quand Jéconia, fils de Jojakim, roi de Juda, serait une bague à ma main droite, je t'arracherais de là.
\VS{25}Je te livrerai entre les mains de ceux qui cherchent ta vie, et entre les mains devant qui tu es craintif, et entre les mains de Nebucadnetsar, roi de Babylone, et entre les mains des Chaldéens\FTNT{2 R. 24:14 ; Ez. 17:12 ; 2 Ch. 36:10.}.
\VS{26}Et je te jetterai, toi et ta mère qui t'a enfanté, dans un autre pays où vous n'êtes pas nés, et vous y mourrez.
\VS{27}Et quant au pays qu'ils désirent pour y retourner, ils n'y retourneront pas.
\VS{28}Cet homme, Jéconia, est-il un vase méprisé et brisé ? Est-il un objet qui ne fait plus plaisir ? Pourquoi sont-ils jetés là, lui et sa postérité, lancés, dis-je, dans un pays qu'ils ne connaissent pas\FTNT{Os. 8:8.} ?
\VS{29}Ô terre, terre, terre ! Ecoute la parole de Yahweh !
\VS{30}Ainsi parle Yahweh : Ecrivez que cet homme-là est privé d'enfant, que c'est un homme qui ne prospérera pas pendant ses jours, et que même il n'y aura pas d'homme de sa postérité qui prospère, et qui soit assis sur le trône de David, ni qui domine plus en dominer sur Juda\FTNT{2 R. 24:8-16.}.
\Chap{23}
\TextTitle{Israël sera rassemblé par le Messie}
\VerseOne{}Malheur aux pasteurs qui détruisent et dispersent le troupeau de mon pâturage ! Dit Yahweh.
\VS{2}C'est pourquoi ainsi parle Yahweh, le Dieu d'Israël, sur les pasteurs qui paissent mon peuple : Vous avez dispersé\FTNT{Les brebis du Seigneur sont dispersées ou éparpillées par les faux pasteurs (Ez. 34).} mes brebis, et vous les avez chassées, et ne vous en êtes pas occupés ; voici, je vous punirai à cause de la méchanceté de vos actions, dit Yahweh.
\VS{3}Mais je rassemblerai le reste de mes brebis de tous les pays où je les ai chassées ; et je les ramènerai à leur pâturage, et elles seront fécondes et multiplieront.
\VS{4}Je susciterai aussi sur elles des pasteurs qui les paîtront, et elles n'auront plus de peur, et ne s'épouvanteront plus, et il n'en manquera aucune, dit Yahweh.
\VS{5}Voici, les jours viennent, dit Yahweh, où je susciterai à David un Germe juste, qui régnera en Roi ; il prospérera, et exercera le droit et la justice dans le pays\FTNT{Es. 4:2 ; Za. 6:12-13 ; Ps. 96:13 ; Lu. 1:32-33.}.
\VS{6}En son temps, Juda sera sauvé, Israël demeurera en sécurité ; et c'est ici le nom dont on l'appellera : Yahweh notre justice.
\VS{7}C'est pourquoi, voici, les jours viennent, dit Yahweh, qu'on ne dira plus : Yahweh est vivant, lui qui a fait monter les fils d'Israël du pays d'Egypte !
\VS{8}Mais : Yahweh est vivant, lui qui a fait monter et qui a ramené la postérité de la maison d'Israël, du pays du nord et de tous les pays où je les avais chassés, et ils habiteront dans leur pays.
\TextTitle{Jugement sur les faux prophètes}
\VS{9}A cause des prophètes mon cœur est brisé au-dedans de moi, tous mes os se relâchent ; je suis comme un homme ivre, et comme un homme que le vin a surmonté, à cause de Yahweh, et à cause des paroles de sa sainteté.
\VS{10}Car le pays est rempli d'hommes qui commettent l'adultère ; et le pays est en deuil à cause de la malédiction : Les pâturages du désert sont desséchés, leur course ne va qu'au mal, et leur force à ce qui n'est pas droit.
\VS{11}Car le prophète et le sacrificateur sont corrompus ; j'ai même trouvé dans ma maison leur méchanceté, dit Yahweh.
\VS{12}C'est pourquoi leur chemin sera comme des lieux glissants dans l'obscurité, ils y seront poussés et ils tomberont\FTNT{Ps. 35:6 ; Pr. 4:19.} ; car je ferai venir du mal sur eux, dans l'année de leur châtiment, dit Yahweh.
\VS{13}Or j'ai vu de la folie dans les prophètes de Samarie, car ils prophétisaient par Baal, et faisaient égarer mon peuple Israël.
\VS{14}Mais j'ai vu des choses horribles dans les prophètes de Jérusalem car ils commettent des adultères, et ils marchent dans le mensonge ; ils fortifient les mains de ceux qui font le mal, afin qu'aucun ne se détourne de sa méchanceté ; ils me sont tous comme Sodome, et les habitants de la ville comme Gomorrhe\FTNT{Es. 1:9.}.
\VS{15}C'est pourquoi, ainsi parle Yahweh des armées sur les prophètes : Voici, je vais leur faire manger de l'absinthe, et leur ferai boire des eaux empoisonnées ; car c'est par les prophètes que la profanation est venue dans tout le pays.
\VS{16}Ainsi parle Yahweh des armées : N'écoutez pas les paroles des prophètes qui vous prophétisent ! Ils vous font devenir vains, ils disent les visions de leur cœur, et ils ne les tiennent pas de la bouche de Yahweh.
\VS{17}Ils ne cessent de dire à ceux qui me méprisent : Yahweh a dit : Vous aurez la paix ; et ils disent à tous ceux qui marchent suivant les penchants de leur cœur : Il ne vous arrivera aucun mal\FTNT{Ez. 13:10.}.
\VS{18}Car qui s'est trouvé au conseil secret de Yahweh ? Et qui a aperçu et entendu sa parole\FTNT{Es. 40:13 ; Job. 15:8 ; 1 Co. 2:16.} ? Qui a été attentif à sa parole, et l'a entendue ?
\VS{19}Voici la tempête de Yahweh, sa fureur va se montrer, et le tourbillon prêt à fondre tombera sur la tête des méchants.
\VS{20}La colère de Yahweh ne se détournera pas jusqu'à ce qu'il ait accompli, exécuté les desseins de son cœur. Vous aurez une claire intelligence de ceci dans les derniers jours\FTNT{Gn. 49:1-2.}.
\VS{21}Je n'ai pas envoyé ces prophètes-là, et ils ont couru ; je ne leur ai pas parlé, et ils ont prophétisé.
\VS{22}S'ils s'étaient trouvés dans mon conseil secret, ils auraient aussi fait entendre mes paroles à mon peuple, et ils les auraient ramenés de leur mauvaise voie, de la méchanceté de leurs actions.
\VS{23}Suis-je un Dieu de près, dit Yahweh, et ne suis-je pas aussi un Dieu de loin ?
\VS{24}Quelqu'un se cachera-t-il dans un lieu secret sans que je ne le voie ? Dit Yahweh. Ne remplis-je pas, moi, les cieux et la terre ? Dit Yahweh\FTNT{Ps. 139:7-8 ; Am. 9:2-3.}.
\VS{25}J'ai entendu ce que les prophètes disent, prophétisant le mensonge en mon nom, et disant : J'ai eu un songe ! J'ai eu un songe !
\VS{26}Jusqu'à quand ceci sera-t-il au cœur des prophètes qui prophétisent le mensonge, et qui prophétisent la tromperie de leur cœur ?
\VS{27}Qui pensent comment ils feront oublier mon nom à mon peuple, par les songes que chacun d'eux raconte à son compagnon, comme leurs pères ont oublié mon Nom pour Baal\FTNT{Jg. 2:13.}.
\VS{28}Que le prophète qui a eu un songe raconte ce songe ; et que celui qui a ma parole proclame ma parole en vérité. Quelle convenance y a-t-il entre la paille et le froment ? Dit Yahweh.
\VS{29}Ma parole n'est-elle pas comme un feu, dit Yahweh, et comme un marteau qui brise le roc ?
\VS{30}C'est pourquoi voici, dit Yahweh, j'en veux aux prophètes qui se dérobent mes paroles l'un à l'autre.
\VS{31}Voici, dit Yahweh, j'en veux aux prophètes qui accommodent leurs langues, et qui disent : Il dit.
\VS{32}Voici, dit Yahweh, j'en veux à ceux qui prophétisent des songes faux, et qui les racontent, et font égarer mon peuple par leurs mensonges, et par leur témérité, quoique je ne les ai pas envoyés, et que je ne leur aie pas donné d'ordre ; c'est pourquoi ils ne sont d'aucune utilité à ce peuple, dit Yahweh\FTNT{So. 3:4.}.
\VS{33}Si donc ce peuple t'interroge, ou qu'il interroge le prophète, ou le sacrificateur, en disant : Quel est l'oracle de Yahweh ? Tu leur diras : Quel est cet oracle ? Je vous abandonnerai, dit Yahweh.
\VS{34}Et quant au prophète, et au sacrificateur, et au peuple qui dira : Oracle de Yahweh ; je punirai cet homme-là et sa maison.
\VS{35}Vous direz ainsi chacun à son compagnon, et chacun à son frère : Qu'a répondu Yahweh ? Qu'a dit Yahweh ?
\VS{36}Et vous ne mentionnerez plus : Oracle de Yahweh ; car la parole de chacun sera pour lui un oracle ; parce que vous tordez les paroles du Dieu vivant\FTNT{2 Pi. 3:15-16.}, les paroles de Yahweh des armées, notre Dieu.
\VS{37}Tu diras au prophète : Que t'a répondu Yahweh et que t'a dit Yahweh ?
\VS{38}Et si vous dites : Oracle de Yahweh ; à cause de cela, parle Yahweh, parce que vous dites cette parole : Oracle de Yahweh ; et que j'ai envoyé vers vous pour dire : Ne dites plus : Oracle de Yahweh !
\VS{39}A cause de cela, me voici, et je vous oublierai entièrement, et je vous rejetterai loin de ma présence, vous et la ville que j'ai donnée à vous et à vos pères.
\VS{40}Et je mettrai sur vous un opprobre éternel et une honte éternelle, qui ne s'oublieront pas.
\Chap{24}
\TextTitle{Bonnes figues: Futur retour en Juda des captifs de Babylone ; mauvaises figues: Jugement sur Jérusalem}
\VerseOne{}Yahweh me fit voir une vision, et voici deux paniers de figues posés devant le temple de Yahweh, après que Nebucadnetsar, roi de Babylone, eut transporté de Jérusalem, Jéconia, fils de Jojakim, roi de Juda, les chefs de Juda, avec les charpentiers et les serruriers, et les eut conduits à Babylone.
\VS{2}L'un des paniers avait de très bonnes figues, comme les figues de la première récolte ; et l'autre panier avait de très mauvaises figues, qu'on ne pouvait manger à cause de leur mauvaise qualité.
\VS{3}Et Yahweh me dit : Que vois-tu, Jérémie ? Et je répondis : Des figues. Les bonnes figues sont très bonnes, et les mauvaises sont très mauvaises et ne peuvent être mangées à cause de leur mauvaise qualité.
\VS{4}Alors la parole de Yahweh me fut adressée, en disant :
\VS{5}Ainsi parle Yahweh, le Dieu d'Israël : Comme tu distingues ces bonnes figues, ainsi je me souviendrai, pour leur faire du bien, des captifs de Juda, que j'ai envoyés hors de ce lieu dans le pays des Chaldéens.
\VS{6}Et je les regarderai d'un œil favorable, et je les ramènerai dans ce pays, je les y rétablirai et je ne les détruirai plus, je les planterai et je ne les arracherai pas.
\VS{7}Et je leur donnerai un cœur pour me connaître, pour connaître, dis-je, que je suis Yahweh ; et ils seront mon peuple, et je serai leur Dieu : Car ils reviendront à moi de tout leur cœur\FTNT{De. 30:6 ; Ez. 11:19.}.
\VS{8}Et comme les mauvaises figues, qui ne peuvent être mangées à cause de leur mauvaise qualité ; ainsi certainement, dit Yahweh, je ferai devenir Sédécias, roi de Juda, ses chefs, et le reste de Jérusalem, ceux qui sont restés dans ce pays, et ceux qui habitent dans le pays d'Egypte.
\VS{9}Et je les livrerai pour être agités pour leur malheur par tous les royaumes de la terre, et pour être en opprobre, en de proverbe, en raillerie, et en malédiction, par tous les lieux où je les aurai chassé\FTNT{De. 28:37.}.
\VS{10}Et j'enverrai sur eux l'épée, la famine et la peste, jusqu'à ce qu'ils soient consumés du pays que j'avais donné à eux et à leurs pères.
\Chap{25}
\TextTitle{Prophétie sur les soixante-dix ans de captivité babylonienne\FTNTT{Da.9:2.}}
\VerseOne{}La parole qui fut adressée à Jérémie touchant tout le peuple de Juda, la quatrième année de Jojakim, fils de Josias, roi de Juda, qui est la première année de Nebucadnetsar, roi de Babylone,
\VS{2}parole que Jérémie, le prophète, prononça à tout le peuple de Juda, et à tous les habitants de Jérusalem, en disant :
\VS{3}Depuis la treizième année de Josias, fils d'Amon, roi de Juda, jusqu'à ce jour, qui est la vingt-troisième année, la parole de Yahweh m'a été adressée ; et je vous ai parlé, me levant dès le matin et parlant, et vous n'avez pas écouté.
\VS{4}Et Yahweh vous a envoyé tous ses serviteurs, les prophètes, se levant dès le matin et les envoyant ; et vous ne les avez pas écoutés, vous n'avez pas prêté l'oreille pour écouter.
\VS{5}Lorsqu'ils disaient : Détournez-vous maintenant chacun de sa mauvaise voie et de la méchanceté de vos actions, et vous habiterez d'un siècle à l'autre dans le pays que Yahweh a donné à vous et à vos pères\FTNT{Jon. 3:8 ; 2 R. 17:13.} ;
\VS{6}et n'allez pas après d'autres dieux pour les servir et pour vous prosterner devant eux, ne m'irritez pas par les œuvres de vos mains, et je ne vous ferai aucun mal.
\VS{7}Mais vous ne m'avez désobéi, dit Yahweh, pour m'irriter par les œuvres de vos mains, pour votre malheur.
\VS{8}C'est pourquoi ainsi parle Yahweh des armées : Parce que vous n'avez pas écouté mes paroles,
\VS{9}voici j'enverrai et j'assemblerai toutes les familles du nord, dit Yahweh, et j'enverrai, dis-je, vers Nebucadnetsar, roi de Babylone, mon serviteur ; et je les ferai venir contre ce pays et contre ses habitants, et contre toutes ces nations d'alentour, je les détruirai à la façon de l'interdit, je les mettrai en désolation, en opprobre et en ruines éternelles\FTNT{De. 28:37 ; Es. 10:6.}.
\VS{10}Et je ferai cesser parmi eux les cris de joie et les cris d'allégresse, la voix de l'époux et la voix de l'épouse, le bruit des meules et la lumière des lampes\FTNT{Es. 24:7 ; Ez. 26:13.}.
\VS{11}Et tout ce pays sera une ruine jusqu'à s'en étonner, et ces nations seront asservies au roi de Babylone pendant soixante-dix ans\FTNT{Voir Jé. 29 : 10. Les soixante-dix ans se rapportent également au temps de la domination mondiale babylonienne. Le peuple avait une dette envers Yahweh de 70 ans de sabbats (Lé. 26 34-43 ; 2 Ch. 36:21).}.
\TextTitle{Jugement sur Babylone et les nations impies}
\VS{12}Et il arrivera que quand ces soixante-dix ans seront accomplis, je punirai le roi de Babylone et cette nation, dit Yahweh, à cause de leurs iniquités ; je punirai le pays des Chaldéens, que je mettrai en désolations éternelles\FTNT{Da. 9:2.}.
\VS{13}Et je ferai venir sur ce pays-là toutes mes paroles que j'ai prononcées contre lui, toutes les choses qui sont écrites dans ce livre, ce que Jérémie a prophétisé contre toutes ces nations.
\VS{14}Car de grands rois aussi et de grandes nations se serviront d'eux, et je leur rendrai selon leurs actions et selon l'œuvre de leurs mains.
\VS{15}Car ainsi m'a parlé Yahweh, le Dieu d'Israël : Prends de ma main cette coupe du vin, savoir de cette fureur-ci, et fais-la boire à toutes les nations vers lesquelles je t'enverrai\FTNT{Ab. 16.}.
\VS{16}Ils en boiront, et ils chancelleront et seront comme fous, à cause de l'épée que j'enverrai parmi eux.
\VS{17}Je pris donc la coupe de la main de Yahweh, et je la fis boire à toutes les nations vers lesquelles Yahweh m'envoyait :
\VS{18}Savoir : A Jérusalem et aux villes de Juda, à ses rois et à ses chefs, pour les mettre en désolation, en étonnement, en opprobre et en malédiction, comme il paraît aujourd'hui ;
\VS{19}à Pharaon, roi d'Egypte, à ses serviteurs, à ses chefs, et à tout son peuple ;
\VS{20}et à tout le mélange des peuples d'Arabie, à tous les rois du pays d'Uts, à tous les rois du pays des Philistins, à Askalon, à Gaza, à Ekron, et au reste d'Asdod ;
\VS{21}à Edom, à Moab, et aux fils d'Ammon ;
\VS{22}à tous les rois de Tyr, à tous les rois de Sidon, et aux rois des îles qui sont au-delà de la mer ;
\VS{23}à Dedan, à Théma, à Buz, et à tous ceux qui se coupent les coins de la barbe ;
\VS{24}à tous les rois d'Arabie, et à tous les rois des Arabes qui habitent au désert ;
\VS{25}à tous les rois de Zimri, à tous les rois d'Elam, et à tous les rois de Médie ;
\VS{26}à tous les rois du nord, tant proches qu'éloignés l'un de l'autre, et à tous les royaumes du monde qui sont sur la face de la terre. Et le roi de Schéschac boira après eux.
\VS{27}Et tu leur diras : Ainsi parle Yahweh des armées, le Dieu d'Israël : Buvez et soyez enivrés, même vomissez, et tombez sans vous relever, à cause de l'épée que j'enverrai parmi vous !
\VS{28}Or il arrivera qu'ils refuseront de prendre la coupe de ta main pour boire; mais tu leur diras : Ainsi parle Yahweh des armées : Vous en boirez certainement !
\VS{29}Car voici, je commence à envoyer du mal sur la ville sur laquelle mon nom est invoqué ; et vous, en seriez exempts en quelque sorte ? Vous n'en serez pas exempts ; car je m'en vais appeler l'épée sur tous les habitants de la terre, dit Yahweh des armées\FTNT{1 Pi. 4:17-18.}.
\VS{30}Tu prophétiseras donc contre eux toutes ces paroles-là, et tu leur diras : Yahweh rugira d'en haut ; il fera entendre sa voix de la demeure de sa sainteté ; il rugira, il rugira contre son son agréable demeure ; il poussera un cri  contre tous les habitants de la terre\FTNT{Joë. 3:16 ; Am. 1:2.}, comme ceux qui foulent au pressoir,
\VS{31}le son éclatant est parvenu jusqu'à l'extrémité de la terre ; car Yahweh plaide avec les nations, et il conteste contre toute chair. On livrera les méchants à l'épée, dit Yahweh.
\VS{32}Ainsi parle Yahweh des armées : Voici, le mal va sortir d'une nation à l'autre, et une grande tempête se se lèvera des extrémités de la terre.
\VS{33}Et en ce jour-là, ceux qui auront été mis à mort par Yahweh seront étendus d'un bout de la terre à l'autre bout ; ils ne seront ni pleurés, ni recueillis, ni enterrés, mais ils seront comme du fumier sur la face du sol.
\VS{34}Vous, pasteurs, hurlez et criez ! Et vous, les nobles du troupeau, roulez-vous dans la cendre ; car les jours pour vous massacrer sont accomplis. Je vous disperserai et vous tomberez comme un vase précieux.
\VS{35}Et les pasteurs n'auront aucun moyen de s'enfuir, ni les nobles d'échapper. 
\VS{36}Il y aura la voix du cri des bergers, les hurlements des nobles du troupeau ; parce que Yahweh s'en va ravager leur pâturage.
\VS{37}Et les demeures paisibles seront abattues, à cause de l'ardeur de la colère de Yahweh.
\VS{38}Il a abandonné son tabernacle comme un lionceau ; car leur pays est réduit en désert, à cause de l'ardeur du destructeur et à cause, dis-je, de l'ardeur de sa colère.
\Chap{26}
\TextTitle{Avertissement dans le parvis du temple}
\VerseOne{}Au commencement du règne de Jojakim, fils de Josias, roi de Juda, cette parole fut adressée à Jérémie part Yahweh, en disant :
\VS{2}Ainsi parle Yahweh : Tiens-toi debout dans le parvis de la maison de Yahweh, et prononce à toutes les villes de Juda qui viennent pour se prosterner dans la maison de Yahweh toutes les paroles que je t'ordonne de leur dire ; n'en retranche pas un mot.
\VS{3}Peut-être qu'ils écouteront et qu'ils se détourneront chacun de sa mauvaise voie ; et je me repentirai du mal que j'avais pensé leur faire à cause de la méchanceté de leurs actions.
\VS{4}Tu leur diras donc : Ainsi parle Yahweh : Si vous ne m'écoutez pas pour marcher selon ma loi que je vous ai proposée,
\VS{5}pour obéir aux paroles des prophètes, mes serviteurs, que je vous envoie, me levant dès le matin, et les envoyant, lesquels que vous n'avez pas écoutés,
\VS{6}je mettrai cette maison dans le même état que Silo, et je livrerai cette ville à la malédiction, à toutes les nations de la terre.
\VS{7}Or les sacrificateurs, les prophètes, et tout le peuple, entendirent Jérémie prononcer ces paroles dans la maison de Yahweh.
\TextTitle{Jérémie menacé de mort par les sacrificateurs et les prophètes}
\VS{8}Et il arrivera qu'aussitôt que Jérémie eut achevé prononcer tout ce que Yahweh lui avait ordonné de dire à tout le peuple, les sacrificateurs, les prophètes, et tout le peuple, le saisirent en disant : Tu mourras, tu mourras\FTNT{Gn. 2:17.} !
\VS{9}Pourquoi as-tu prophétisé au nom de Yahweh, en disant : Cette maison sera comme Silo, et cette ville sera déserte tellement que personne n'y habitera ? Et tout le peuple s'assembla autour de Jérémie dans la maison de Yahweh.
\VS{10}Et les les chefs de Juda ayant entendu toutes ces choses, montèrent de la maison du roi à la maison de Yahweh, et s'assirent à l'entrée de la porte neuve de la maison de Yahweh.
\VS{11}Et les sacrificateurs et les prophètes parlèrent aux chefs et à tout le peuple, en disant : Cet homme mérite d'être condamné à la mort ; car il a prophétisé contre cette ville, comme vous l'avez entendu de vos oreilles.
\VS{12}Et Jérémie parla à tous les chefs et à tout le peuple, en disant : Yahweh m'a envoyé pour prophétiser contre cette maison et contre cette ville toutes les paroles que vous avez entendues.
\VS{13}Maintenant donc, amendez votre conduite et vos actions, écoutez la voix de Yahweh, votre Dieu, et Yahweh se repentira du mal qu'il a prononcé contre vous.
\VS{14}Pour moi, me voici entre vos mains ; faites-moi ce qui vous semblera bon et juste.
\VS{15}Mais sachez comme une chose certaine, que si vous me faites mourir, vous mettrez du sang innocent sur vous, sur cette ville et sur ses habitants ; car en vérité Yahweh m'a envoyé vers vous pour prononcer à vos oreilles toutes ces paroles.
\VS{16}Alors les chefs et tout le peuple dirent aux sacrificateurs et aux prophètes : Cet homme ne mérite pas d'être condamné à la mort, car il nous a parlé au nom de Yahweh, notre Dieu.
\VS{17}Et quelques-uns des anciens du pays se levèrent et parlèrent à toute l'assemblée du peuple, en disant :
\VS{18}Michée, de Moréscheth, prophétisait aux jours d'Ezéchias, roi de Juda, et il parlait à tout le peuple de Juda, en disant : Ainsi parle Yahweh des armées : Sion sera labourée comme un champ, Jérusalem sera réduite en un monceau de pierres, et la montagne du temple en des hauts lieux d'une forêt\FTNT{Mi. 1:1 ; Mi. 3:12.}.
\VS{19}Ezéchias, roi de Juda, et tous ceux de Juda l'ont-ils fait mourir ? Ezéchias ne craignit-il pas Yahweh ? N'implora-t-il pas Yahweh ? Et Yahweh se repentit du mal qu'il avait prononcé contre eux. Et nous, nous ferions donc un grand mal contre nos âmes\FTNT{2 Ch. 32:26.} !
\VS{20}Mais aussi dirent les autres, y eut aussi un homme qui prophétisait au nom de Yahweh, savoir, Urie, fils de Schemaeja, de Kirjath-Jearim. Il prophétisa contre cette même ville et contre ce même pays, de la même manière que Jérémie.
\VS{21}Et le roi Jojakim, tous ses vaillants hommes, et tous ses chefs entendirent ses paroles, et le roi chercha à le faire mourir. Urie, qui en fut informé, eut peur, prit la fuite, et se retira en Egypte.
\VS{22}Et le roi Jojakim envoya des hommes en Egypte, savoir, Elnathan, fils d'Acbor, et quelques hommes avec lui, qui allèrent en Egypte.
\VS{23}Et ils firent sortir d'Egypte Urie et l'amenèrent au roi Jojakim qui le frappa avec l'épée et jeta son cadavre sur les sépulcres des fils du peuple.
\VS{24}Toutefois la main d'Achikam, fils de Schaphan, fut avec Jérémie, et empêcha qu'il ne soit livré au peuple pour être mis à mort.
\Chap{27}
\TextTitle{Prophétie : Les nations seront asservies à Nebucadnetsar}
\VerseOne{}Au commencement du règne de Jojakim\FTNT{Il est probable que ce soit une erreur de copiste, car bien que l'hébreu dit « Jojakim », le contexte se rapporte à Sédécias. Voir Jé. 27:3 ; Jé. 27:12 ; Jé. 27:20 ; Jé 28:1.}, fils de Josias, roi de Juda, cette parole fut adressée à Jérémie de la part de Yahweh, en disant :
\VS{2}Ainsi m'a parlé Yahweh : Fais-toi des liens et des jougs, et mets-les sur ton cou\FTNT{Ez. 7:23.}.
\VS{3}Et envoie-les au roi d'Edom, et au roi de Moab, et au roi des fils d'Ammon, et au roi de Tyr et au roi de Sidon, par les mains des messagers qui sont venus à Jérusalem vers Sédécias, roi de Juda ;
\VS{4}et tu leur donneras mes ordres pour leurs maîtres, en disant : Ainsi parle Yahweh des armées, le Dieu d'Israël : Vous direz ainsi à vos maîtres :
\VS{5}J'ai fait la terre, les hommes et les bêtes qui sont sur la terre, par ma grande force et par mon bras étendu, et je la donne à qui cela me plaît\FTNT{De. 32:8.}.
\VS{6}Et maintenant j'ai livré tous ces pays entre les mains de Nebucadnetsar, roi de Babylone, mon serviteur ; et même je lui ai donné les bêtes des champs pour qu'elles lui soient asservies\FTNT{Da. 2:38.}.
\VS{7}Et toutes les nations lui seront asservies, à lui, à son fils, et au fils de son fils, jusqu'à ce que le temps de son pays vienne aussi, et que plusieurs nations et de grands rois l'asservissent.
\VS{8}Et il arrivera que la nation ou le royaume qui ne se soumettra pas à lui, à Nebucadnetsar, roi de Babylone, et qui ne soumettra pas son cou au joug du roi de Babylone, je punirai cette nation par l'épée, par la famine et par la peste, dit Yahweh, jusqu'à ce que je les aie consumés par sa main.
\VS{9}Vous donc, n'écoutez pas vos prophètes, ni vos devins, ni vos songeurs, ni vos augures, ni vos magiciens, qui vous parlent, en disant : Vous ne serez pas asservis au roi de Babylone.
\VS{10}Car ils vous prophétisent le mensonge pour vous faire aller loin de votre pays, afin que je vous chasse et que vous périssiez.
\VS{11}Mais la nation qui livrera son cou au joug du roi de Babylone, et qui le servira, je la laisserai dans son pays, dit Yahweh, pour qu'elle le cultive et qu'elle y demeure.
\VS{12}Puis j'ai parlé à Sédécias, roi de Juda, selon toutes ces paroles-là, en disant : Soumettez votre cou au joug du roi de Babylone, et rendez-vous sujets, à lui et son peuple, et vous vivrez.
\VS{13}Pourquoi mourriez-vous, toi et ton peuple, par l'épée, par la famine et par la peste, selon que Yahweh a parlé contre la nation qui ne sera pas soumise au roi de Babylone ?
\VS{14}N'écoutez donc pas les paroles des prophètes qui vous parlent en disant : Vous ne serez pas asservis au roi de Babylone ! Car ils vous prophétisent le mensonge.
\VS{15}Même je ne les ai pas envoyés, dit Yahweh, et ils vous prophétisent faussement en mon nom, afin que je vous rejette et que vous périssiez, vous et les prophètes qui vous prophétisent.
\VS{16}J'ai aussi parlé aussi aux sacrificateurs et à tout ce peuple, en disant : Ainsi parle Yahweh : N'écoutez pas les paroles de vos prophètes qui vous prophétisent, en disant : Voici, les ustensiles de la maison de Yahweh seront bientôt rapportés de Babylone ! Car ils vous prophétisent le mensonge.
\VS{17}Ne les écoutez donc pas, rendez-vous sujets au roi de Babylone, et vous vivrez. Pourquoi cette ville serait-elle réduite en un désert ?
\VS{18}Et s'ils sont prophètes et si la parole de Yahweh est en eux, qu'ils intercèdent maintenant auprès de Yahweh des armées, afin que les ustensiles qui restent dans la maison de Yahweh, dans la maison du roi de Juda, et dans Jérusalem, n'aillent pas à Babylone.
\VS{19}Car ainsi parle Yahweh des armées au sujet des colonnes, de la mer, des bases, et des autres ustensiles qui sont restés dans cette ville,
\VS{20}que Nebucadnetsar, roi de Babylone, n'a pas emportés, quand il a transporté de Jérusalem à Babylone, Jéconia, fils de Jojakim, roi de Juda, et tous les nobles de Juda et de Jérusalem,
\VS{21}Yahweh, dis-je, des armées le Dieu d'Israël, parle ainsi au sujet des ustensiles qui restent dans la maison de Yahweh, dans la maison du roi de Juda et dans Jérusalem :
\VS{22}Ils seront emportés à Babylone, et ils y demeureront jusqu'au jour où je les visiterai, dit Yahweh, et où je les ferai remonter et revenir dans ce lieu\FTNT{2 R. 24:14-15 ; Esd. 1:7-11 ; 2 Ch. 25:13-16 ; 2 Ch. 36:18.}.
\Chap{28}
\TextTitle{Hanania meurt suite à sa prophétie mensongère}
\VerseOne{}Il arriva aussi, en cette même année, au commencement du règne de Sédécias, roi de Juda, savoir, au cinquième mois de la quatrième année, que Hanania, fils d'Azzur, prophète de Gabaon, me parla dans la maison de Yahweh, aux yeux des sacrificateurs et de tout le peuple, en disant :
\VS{2}Ainsi parle Yahweh des armées, le Dieu d'Israël : Je romps le joug du roi de Babylone !
\VS{3}Dans deux années accomplis, et je ferai rapporter dans ce lieu tous les ustensiles de la maison de Yahweh, que Nebucadnetsar, roi de Babylone, a pris de ce lieu, et qu'il a transportés à Babylone.
\VS{4}Et je ferai revenir dans ce lieu, dit Yahweh, Jéconia, fils de Jojakim, roi de Juda, et tous les captifs de Juda qui sont allés à Babylone ; car je romprai le joug du roi de Babylone.
\VS{5}Alors Jérémie, le prophète, répondit à Hanania, le prophète, aux yeux des sacrificateurs, et aux yeux de tout le peuple qui se tenait dans la maison de Yahweh.
\VS{6}Et Jérémie, le prophète, dit : Ainsi soit-il ! Que Yahweh fasse ainsi ! Que Yahweh accomplisse les paroles que tu as prophétisées, et qu'il fasse revenir de Babylone dans ce lieu-ci les ustensiles de la maison de Yahweh, et tous les captifs de Babylone !
\VS{7}Toutefois, écoute maintenant cette parole que je prononce, à tes oreilles et aux oreilles de tout le peuple :
\VS{8}Les prophètes qui ont été avant moi et avant toi, dès les temps anciens, ont prophétisé contre plusieurs pays et de grands royaumes, la guerre, le malheur et la peste ;
\VS{9}Le prophète qui aura prophétisé la paix, quand la parole de ce prophète sera accomplie, ce prophète-là sera reconnu pour avoir été véritablement envoyé par Yahweh.
\VS{10}Alors Hanania, le prophète, prit le joug de dessus le cou de Jérémie, le prophète, et le rompit.
\VS{11}Puis Hanania parla aux yeux de tout le peuple, en disant : Ainsi parle Yahweh : C'est ainsi que dans deux années, je romprai le joug de Nebucadnetsar, roi de Babylone, de dessus le cou de toutes les nations. Et Jérémie, le prophète, alla au loin par la route.
\VS{12}Mais la parole de Yahweh fut adressée à Jérémie, après que Hanania, le prophète, eut rompu le joug de dessus le cou de Jérémie, le prophète, en disant :
\VS{13}Va, et parle à Hanania, en disant : Ainsi parle Yahweh : Tu as rompu les jougs de bois, et tu auras à la place un joug de fer.
\VS{14}Car ainsi parle Yahweh des armées, le Dieu d'Israël : Je mets un joug de fer sur le cou de toutes ces nations, afin qu'elles servent Nebucadnetsar, roi de Babylone, et elles le serviront ; et je lui donne aussi les bêtes des champs\FTNT{De. 28:48.}.
\VS{15}Puis Jérémie, le prophète, dit à Hanania, le prophète : Ecoute maintenant, ô Hanania ! Yahweh ne t'a pas envoyé, et tu as fait que ce peuple se confie au mensonge\FTNT{Ez. 13:3-9.}.
\VS{16}C'est pourquoi ainsi parle Yahweh : Voici, je te chasse de la face de la terre ; et tu mourras cette année ; car tu as parlé de révolte contre Yahweh.
\VS{17}Et Hanania, le prophète, mourut cette année-là, dans le septième mois.
\Chap{29}
\TextTitle{Message à l'attention des Juifs captifs à Babylone}
\VerseOne{}Or ce sont ici les paroles de la lettre que Jérémie, le prophète, envoya de Jérusalem au reste des anciens en captivité, aux sacrificateurs et aux prophètes, et à tout le peuple, que Nebucadnetsar avait transportés de Jérusalem à Babylone,
\VS{2}après que le roi Jéconia fut sorti de Jérusalem, avec la reine, et les eunuques, et les chefs de Juda et de Jérusalem, et les charpentiers et les serruriers\FTNT{2 R. 24:12.}.
\VS{3}C'est par la main d'Eleasa, fils de Schaphan, et Guemaria, fils de Hilkija, que Sédécias, roi de Juda, l'envoya à Babylone vers Nebucadnetsar, roi de Babylone. La lettre disait :
\VS{4}Ainsi parle Yahweh des armées, le Dieu d'Israël, à tous les captifs que j'ai fait transporter de Jérusalem à Babylone.
\VS{5}Bâtissez des maisons, et habitez-les ; plantez des jardins, et mangez-en les fruits.
\VS{6}Prenez des femmes, et engendrez des fils et des filles ; prenez aussi des femmes pour vos fils, et donnez vos filles à des hommes, afin qu'elles enfantent des fils et des filles ; multipliez-vous là, et ne diminuez pas.
\VS{7}Et cherchez la paix de la ville où je vous ai transporté, et priez Yahweh pour elle ; parce que dans sa paix vous aurez la paix.
\VS{8}Car ainsi parle Yahweh des armées, le Dieu d'Israël : Que vos prophètes qui sont au milieu de vous, et vos devins, ne vous séduisent pas, et n'écoutez pas vos songes que vous vous songez\FTNT{Tous les songes ne viennent pas toujours du Seigneur. Les visions et les songes doivent être en accord avec la Parole de Dieu.}.
\VS{9}Parce qu'ils vous prophétisent faussement en mon nom. Je ne les ai pas envoyés, dit Yahweh.
\VS{10}Car ainsi parle Yahweh : Lorsque les soixante-dix ans seront accomplis pour Babylone, je vous visiterai, et j'accomplirai ma bonne parole à votre égard, pour vous faire revenir dans ce lieu.
\VS{11}Car je sais que les pensées que j'ai pour vous, dit Yahweh, sont des pensées de paix et non pas d'adversité, pour vous donner une fin telle que vous espérez\FTNT{Jos. 1:8.}.
\VS{12}Alors vous m'invoquerez, et vous partirez ; vous me prierez, et je vous exaucerai\FTNT{Os. 5:15.}.
\VS{13}Vous me chercherez, et vous me trouverez, après que vous m'aurez recherché de tout votre cœur\FTNT{Mt. 7:7.}.
\VS{14}Car je me laisserai trouver par vous, dit Yahweh, je ramènerai vos captifs ; et je vous rassemblerai d'entre toutes les nations et de tous les lieux où je vous ai chassés, dit Yahweh, et je vous ramènerai dans le lieu d'où je vous ai transportés.
\VS{15}Cependant si vous dites : Yahweh nous a suscité des prophètes à Babylone !
\VS{16} A cause de cela, ainsi parle Yahweh sur le roi qui est assis sur le trône de David, sur tout le peuple qui habite dans cette ville, sur vos frères qui ne sont pas allés avec vous en captivité ;
\VS{17}ainsi parle Yahweh des armées : Voici, je vais envoyer sur eux l'épée, la famine, et la peste, et je les ferai devenir comme des figues affreuses qui ne peuvent être mangées à cause de leur mauvaise qualité.
\VS{18}Et je les poursuivrai par l'épée, par la famine et par la peste, je les abandonnerai pour être agités par tous les royaumes de la terre, et pour être une malédiction, un étonnement, une moquerie et un opprobre parmi toutes les nations où je les chasserai\FTNT{De. 28:25-37.},
\VS{19}parce qu'ils n'ont pas écouté mes paroles, dit Yahweh, eux à qui j'ai envoyé mes serviteurs, les prophètes, en me levant dès le matin ; et ils n'ont pas écouté, dit Yahweh.
\VS{20}Vous tous donc, écoutez la parole de Yahweh, vous les captifs que j'ai envoyés de Jérusalem à Babylone !
\VS{21}Ainsi parle Yahweh des armées, le Dieu d'Israël sur Achab, fils de Kolaja, et sur Sédécias, fils de Maaséja, qui vous prophétisent faussement en mon nom : Voici, je vais les livrer entre les mains de Nebucadnetsar, roi de Babylone ; et il les frappera sous vos yeux.
\VS{22}Et on se servira d'eux comme une formule de malédiction, parmi tous les captifs de Juda qui sont à Babylone, en disant : Que Yahweh te mette dans un tel état, comme Sédécias et comme Achab, que le roi de Babylone a fait rôtir au feu !
\VS{23}parce qu'ils ont commis des impuretés en Israël, parce qu'ils ont commis l'adultère avec les femmes de leurs prochains, et qu'ils ont dit en mon nom des paroles fausses, alors que je ne leur avais pas commandées. Je le sais, et j'en suis témoin, dit Yahweh.
\VS{24}Parle aussi à Schemaeja, Néchélamite, en disant :
\VS{25}Ainsi parle Yahweh des armées, le Dieu d'Israël : Tu as envoyé en ton nom une lettre à tout le peuple de Jérusalem, à Sophonie, fils de Maaséja, le sacrificateur, et à tous les sacrificateurs, en disant :
\VS{26}Yahweh t'a établi sacrificateur à la place de Jehojada, le sacrificateur, afin qu'il y ait dans la maison de Yahweh des inspecteurs pour surveiller tout homme qui est fou et se donne pour prophète, et afin que tu le mettes en prison et dans les fers.
\VS{27}Et maintenant, pourquoi n'as-tu pas réprimé pas Jérémie d'Anathoth, qui prophétise parmi vous,
\VS{28}car à cause de cela il nous a envoyé dire à Babylone : La captivité sera longue ; bâtissez des maisons, et habitez-les ; plantez des jardins, et mangez-en les fruits !
\VS{29}Or Sophonie, le sacrificateur, lut cette lettre aux oreilles de Jérémie, le prophète.
\VS{30}C'est pourquoi la parole de Yahweh fut adressée à Jérémie, en disant :
\VS{31}Envoie dire à tous les captifs : Ainsi parle Yahweh sur Schemaeja, Néchélamite : Parce que Schemaeja vous prophétise, quoique que je ne l'aie envoyé, et qu'il vous a fait vous confier dans le mensonge,
\VS{32}à cause de cela, dit Yahweh : Je vais punir Schemaeja, Néchélamite, et sa postérité ; il n'y aura personne de sa race qui  habite au milieu de ce peuple, et il ne verra pas le bien que je ferai à mon peuple, dit Yahweh ; car il a parlé de révolte contre Yahweh.
\Chap{30}
\TextTitle{Le jour de Yahweh}
\VerseOne{}La parole qui fut adressée à Jérémie de la part de Yahweh, en disant :
\VS{2}Ainsi parle Yahweh, le Dieu d'Israël : Ecris pour toi dans un livre toutes les paroles que je t'ai dites.
\VS{3}Car voici, les jours viennent, dit Yahweh, où je ramènerai les captifs de mon peuple d'Israël et de Juda, dit Yahweh ; je les ramènerai dans le pays que j'ai donné à leurs pères, et ils le posséderont.
\VS{4}Ce sont ici les paroles que Yahweh a prononcées sur Israël et Juda.
\VS{5}Ainsi parle Yahweh : Nous entendons des cris d'effroi et de terreur, il n'y a pas de paix.
\VS{6}Informez-vous, je vous prie et voyez si un mâle enfante ! Pourquoi vois-je les hommes les mains sur leurs reins, comme une femme qui enfante ? Pourquoi tous les visages sont-ils devenus pâles ?
\VS{7}Malheur ! Que ce jour est grand ; il n'y en a pas eu de semblable. Il sera un temps de détresse pour Jacob ; mais il en sera pourtant délivré\FTNT{Joë. 2:11 ; So. 1:15 ; Da. 12:1 ; Mt. 24:21.}.
\VS{8}Et il arrivera en ce jour-là, dit Yahweh des armées, que je briserai son joug de dessus ton cou, je romprai tes liens, et les étrangers ne t'asserviront plus.
\VS{9}Mais ils serviront Yahweh, leur Dieu, et David, leur roi, que je leur susciterai\FTNT{Ez. 34:23-24.}.
\VS{10}Toi donc, mon serviteur Jacob, ne crains pas, dit Yahweh, et ne t'épouvante pas, ô Israël ! Car, voici, je te délivrerai du pays éloigné, et ta postérité du pays de leur captivité ; et Jacob reviendra, il sera en repos et sera en paix, et il n'y aura personne qui lui fasse peur\FTNT{Es. 41:13.}.
\VS{11}Car je suis avec toi, dit Yahweh, pour te délivrer ; et même je consumerai entièrement toutes les nations parmi lesquelles je t'ai dispersé, mais quand à toi, je ne te consumerai pas entièrement; je te châtierai avec équité, je ne te tiens pas entièrement pour innocent\FTNT{Es. 27:7-8.}.
\VS{12}Ainsi parle Yahweh : Ta blessure est incurable, ta plaie est très douloureuse\FTNT{Mi. 1:9 ; 2 Ch. 36:16.}.
\VS{13}Il n'y a personne qui défende ta cause, pour panser ta plaie ; il n'y a pour toi aucun remède, aucun moyen de guérison.
\VS{14}Tous tes amoureux t'oublient, ils ne te recherchent pas ; car je t'ai frappée d'une plaie d'ennemi, d'un châtiment d'homme cruel, à cause de la multitude de tes iniquités, tes péchés se sont renforcés\FTNT{La. 1:2.}.
\VS{15}Pourquoi cries-tu à cause de ta plaie ? Ta douleur est hors d'espérance ; je t'ai fait ces choses à cause de la grandeur de tes iniquités, du grand nombre de tes péchés.
\VS{16}Néanmoins tous ceux qui te dévorent seront dévorés, et tous ceux qui te mettent dans la détresse, iront en captivité ; ceux qui te dépouillent seront dépouillés, et je livrerai au pillage tous ceux qui te pillent\FTNT{Es. 41:11 ; Ab. 15.}.
\VS{17}Même je guérirai tes plaies, et je te guérirai tes blessures, dit Yahweh. Car ils t'appellent la repoussée, cette Sion que personne ne recherche.
\TextTitle{Israël délivré par Yahweh}
\VS{18}Ainsi parle Yahweh : Voici, je ramène les captifs des tentes de Jacob, j'ai compassion de ses demeures ; la ville sera rebâtie sur le monceau de ses ruines et le palais sera rétabli comme il était.
\VS{19}Et il en sortira des remerciements et des cris de joie ; et je les multiplierai, et ils ne diminueront pas ; je les honorerai, et ils ne seront pas amoindris.
\VS{20}Et ses enfants seront comme autrefois, son assemblée sera affermie devant moi, et je punirai tous ceux qui l'oppriment.
\VS{21}Et son chef sera tiré de son sein, son dominateur sortira du milieu de lui ; je le ferai approcher, et il viendra vers moi ; car qui disposerait son cœur pour venir vers moi ? dit Yahweh.
\VS{22}Et vous serez mon peuple, et je serai votre Dieu.
\VS{23}Voici, la tempête de Yahweh, la fureur éclate, un tourbillon qui s'entasse ; il tombera sur la tête des méchants. 
\VS{24}L'ardeur de la colère de Yahweh ne se détournera pas, jusqu'à ce qu'il ait exécuté, accompli les desseins de son cœur ; vous le comprendrez dans les derniers jours.
\Chap{31}
\TextTitle{Communion retrouvée : La paix et et la joie}
\VerseOne{}En ce temps-là, dit Yahweh, je serai le Dieu de toutes les familles d'Israël, et ils seront mon peuple.
\VS{2}Ainsi parle Yahweh : Le peuple survivant à l'épée a trouvé grâce dans le désert ; Israël marche vers son lieu de repos.
\VS{3}De loin Yahweh m'est apparu, et m'a dit : Je t'aime d'un amour éternel, c'est pourquoi j'ai prolongé ma bonté envers toi.
\VS{4}Je te rétablirai encore, et tu seras rétablie, ô vierge d'Israël ! Tu te pareras encore de tes tambours, et tu sortiras au milieu des danses joyeuses.
\VS{5}Tu planteras encore des vignes sur les montagnes de Samarie ; les vignerons planteront et recueilleront les fruits pour leur usage\FTNT{Es. 65:21.}.
\VS{6}Car il y a un jour où les gardes crieront sur la montagne d'Ephraïm : Levez-vous, et montons à Sion, vers Yahweh, notre Dieu !
\VS{7}Car ainsi parle Yahweh : Réjouissez-vous avec chant de triomphe, et avec allégresse à cause de Jacob, et vous égayez à cause du chef des nations ! Faites-le entendre, chantez des louanges, et dites : Yahweh, délivre ton peuple, le reste d’Israël !
\VS{8}Voici, je vais les faire venir du pays du nord, et je les rassemblerai des extrémités de la terre ; l'aveugle et le boiteux, la femme enceinte et celle qui enfante seront ensemble parmi eux ; une grande assemblée qui reviendra ici.
\VS{9}Ils y seront allés en pleurant, mais je les ferai retourner avec des supplications, et je les conduirai aux torrents d'eaux, et par un droit chemin, où ils ne broncheront pas ; car je suis un père pour Israël, et Ephraïm est mon premier-né\FTNT{Ex. 4:22.}.
\VS{10}Nations, écoutez la parole de Yahweh, et annoncez-la aux îles éloignées ! Dites : Celui qui a dispersé Israël le rassemblera, et il le gardera comme un berger garde son troupeau.
\VS{11}Car Yahweh rachète Jacob, et le retire de la main d'un ennemi plus fort que lui.
\VS{12}Ils viendront donc, et se réjouiront avec des chants de triomphe sur les hauteurs de Sion ; ils afflueront vers les biens de Yahweh, le blé, le vin, l'huile, et le fruit du gros et du menu bétail ; et leur âme sera comme un jardin arrosé, et ils ne seront plus dans la souffrance\FTNT{Es. 61:11.}.
\VS{13}Alors la vierge se réjouira à la danse, les jeunes hommes et les anciens ensemble ; je changerai leur deuil en joie, et je les consolerai ; et je les réjouirai en les délivrant de leur douleur.
\VS{14}Je rassasierai aussi de graisse l'âme des sacrificateurs, et mon peuple sera rassasié de mes biens, dit Yahweh.
\VS{15}Ainsi parle Yahweh : On entend des cris à Rama, des lamentations, des larmes amères ; Rachel pleure ses fils ; elle refuse d'être consolée sur ses fils, car ils ne sont plus\FTNT{Mt. 2:17-18.}.
\VS{16}Ainsi parle Yahweh : Retiens ta voix de pleurer, et tes yeux de verser des larmes, car ton œuvre aura son salaire, dit Yahweh ; et ils reviendront des terres de l'ennemi.
\VS{17}Et il y a de l'espérance pour tes derniers jours, dit Yahweh ; et tes fils reviendront dans leur territoire.
\VS{18}J'ai très bien entendu Ephraïm se plaignant, et disant : Tu m'as châtié, et j'ai été châtié comme un veau qui n'est pas dompté. Fais-moi revenir, et je reviendrai, car tu es Yahweh, mon Dieu\FTNT{Ps. 119:67-71.}.
\VS{19}Certes, après m'être détourné, je me repens ; et après avoir reconnu mes fautes, je frappe sur ma cuisse ; je suis honteux et confus, car je porte l'opprobre de ma jeunesse\FTNT{Ez. 21:17.}.
\VS{20}Ephraïm est-il donc pour moi un cher fils, un fils qui fait mes délices ? Car plus je parle de lui, plus encore son souvenir est en moi ; aussi mes entrailles sont émues en sa faveur : J'aurai certainement pitié de lui, dit Yahweh\FTNT{Es. 5:7.}.
\VS{21}Dresse-toi des signes sur les chemins, place des poteaux, prends garde à la route, au chemin par lequel tu es venue… Reviens, vierge d'Israël, reviens dans tes villes !
\VS{22}Jusqu'à quand seras-tu errante, fille rebelle ? Car Yahweh crée une chose nouvelle sur la terre : La femme entourera l'homme.
\VS{23}Ainsi parle Yahweh des armées, le Dieu d'Israël : On dira encore cette parole-ci dans le pays de Juda et dans ses villes, quand j'aurai ramené leurs captifs : Que Yahweh te bénisse, ô agréable demeure de la justice, montagne de sainteté !
\VS{24}Juda et toutes ses villes ensemble, les laboureurs, et ceux qui conduisent les troupeaux, y habiteront.
\VS{25}Car j'abreuverai l'âme épuisée par le travail, et je remplirai toute âme languissante.
\VS{26}C'est pourquoi je me suis réveillé, et j'ai regardé ; mon sommeil m'avait été agréable.
\TextTitle{Promesse d'une nouvelle alliance}
\VS{27}Voici, les jours viennent, dit Yahweh, où j'ensemencerai la maison d'Israël et la maison de Juda d'une semence d'hommes et d'une semence de bêtes.
\VS{28}Et il arrivera que comme j'ai veillé sur eux pour arracher et démolir, pour détruire, pour perdre et pour faire du mal ; ainsi je veillerai sur eux pour bâtir et pour planter, dit Yahweh.
\VS{29}En ces jours-là, on ne dira plus : Les pères ont mangé des raisins verts, et les dents des fils en ont été agacées\FTNT{Ez. 18:2-3.}.
\VS{30}Mais chacun mourra pour son iniquité ; tout homme qui mangera des raisins verts, ses dents en seront agacées.
\VS{31}Voici, les jours viennent, dit Yahweh, où je traiterai une nouvelle alliance\FTNT{Il s'agit de l'alliance du sang que Jésus, notre Messie, est venu inaugurer en prenant sur lui tous nos péchés et en mourant sur la croix à notre place (Mt. 26:27-29 ; Hé. 8:7-13).} avec la maison d'Israël et avec la maison de Juda,
\VS{32}non comme l'alliance que je traitai avec leurs pères, le jour où je les pris par la main, pour les faire sortir du pays d'Egypte, mon alliance qu'ils ont violée ; et toutefois j’avais été pour eux un mari, dit Yahweh.
\VS{33}Car c'est ici l'alliance que je traiterai avec la maison d'Israël, après ces jours-là, dit Yahweh, je mettrai ma loi au-dedans d'eux, je l'écrirai dans leur cœur ; et je serai leur Dieu, et ils seront mon peuple.
\VS{34}Aucun homme parmi eux n'enseignera plus son prochain, ni personne son frère, en disant : Connaissez Yahweh ! Car tous me connaîtront, depuis le plus petit jusqu'au plus grand, dit Yahweh ; parce que je pardonnerai leur iniquité, et que je ne me souviendrai plus de leur péché\FTNT{Es. 54:13 ; Ha. 2:14 ; Jn. 6:45.}.
\VS{35}Ainsi parle Yahweh, qui a donné le soleil pour être la lumière du jour, et qui a réglé la lune et les étoiles pour être la lumière de la nuit, qui remue la mer, et fait gronder ses flots, lui dont le nom est Yahweh des armées\FTNT{Ge. 1:16 ; Es. 51:15.} :
\VS{36}Si ces lois\FTNT{Les lois de l'univers ont été établies par Yahweh. Ces lois sont : la loi de la gravité, la loi de l'attraction et la loi de la résonance (voir Ps. 148:5-6 ; Job. 38:33).} viennent à cesser devant moi, dit Yahweh, la race d'Israël aussi cessera d'être à jamais une nation devant moi.
\VS{37}Ainsi parle Yahweh : Si les cieux en haut peuvent être mesurés, si les fondements de la terre en bas peuvent être sondés, alors je rejetterai toute la race d'Israël, à cause de toutes les choses qu'ils ont faites, dit Yahweh.
\VS{38}Voici, les jours viennent, dit Yahweh, où cette ville sera rebâtie à Yahweh, depuis la tour de Hananeel, jusqu'à la porte de l'angle\FTNT{Za. 14:10 ; Né. 3:1 ; 2 Ch. 26:9.}.
\VS{39}Le cordeau à mesurer sera encore tiré vis-à-vis d'elle, sur la colline de Gareb, et tournera vers Goath.
\VS{40}Et toute la vallée des cadavres et des cendres, et tous les champs jusqu'au torrent de Cédron, jusqu'à l'angle de la porte des chevaux à l'orient, seront consacrés à Yahweh, et ne seront plus jamais arrachés ni détruits.
\Chap{32}
\TextTitle{Le champ de Hanameel : La pérennité d'Israël}
\VerseOne{}La parole qui fut adressée à Jérémie de la part de Yahweh, la dixième année de Sédécias, roi de Juda. C'était la dix-huitième année de Nebucadnetsar.
\VS{2}Or l'armée du roi de Babylone assiégeait alors Jérusalem ; et Jérémie le prophète était enfermé dans la cour de la prison, qui était dans la maison du roi de Juda ;
\VS{3}car Sédécias, roi de Juda, l'avait fait enfermer, et lui avait dit : Pourquoi prophétises-tu, en disant : Ainsi parle Yahweh : Voici, je vais livrer cette ville entre les mains du roi de Babylone, et il la prendra ;
\VS{4}et Sédécias, roi de Juda, n'échappera pas aux mains des Chaldéens ; mais il sera livré entre les mains du roi de Babylone, et lui parlera bouche à bouche, et ses yeux verront les yeux de ce roi ;
\VS{5}il emmènera Sédécias à Babylone, qui y demeurera jusqu'à ce que je le visite, dit Yahweh ; si vous combattez contre les Chaldéens, vous ne prospérerez pas.
\VS{6}Jérémie donc dit : La parole de Yahweh m'a été adressée, en disant :
\VS{7}Voici Hanameel, fils de Schallum, ton oncle, qui vient vers toi pour te dire : Achète mon champ qui est à Anathoth, car tu as le droit de rachat pour l'acquérir\FTNT{Lé. 25:48 ; Ru. 3:12.}.
\VS{8}Hanameel donc, fils de mon oncle, vint à moi, selon la parole de Yahweh, dans la cour de la prison, et me dit : Achète, je te prie, mon champ, qui est à Anathoth, dans le pays de Benjamin, car tu as le droit d'héritage et de rachat, achète-le ! Et je connus alors que c'était la parole de Yahweh.
\VS{9}Ainsi j'achetai de Hanameel, fils de mon oncle, le champ qui est à Anathoth, et je lui pesai l'argent, qui fut dix-sept sicles d'argent.
\VS{10}Puis j'écrivis le contrat, que je cachetai, je pris des témoins après avoir pesé l'argent sur la balance.
\VS{11}Et je pris le contrat d'acquisition, celui qui était cacheté, selon les ordonnances et les statuts, et celui qui était ouvert ;
\VS{12}Et je remis le contrat d'acquisition à Baruc, fils de Nérija, fils de Machséja, sous les yeux de Hanameel, fils de mon oncle, des témoins qui avaient signé le contrat d'acquisition, et sous les yeux de tous les juifs qui étaient assis dans la cour de la prison.
\VS{13}Puis je donnai sous leurs yeux cet ordre à Baruc, en disant :
\VS{14}Ainsi parle Yahweh des armées, le Dieu d'Israël : Prends ces contrats-ci, à savoir, ce contrat d'acquisition, celui qui est scellé, et celui qui est ouvert, et mets-les dans un vase de terre, afin qu'ils se conservent longtemps.
\VS{15}Car ainsi parle Yahweh des armées, le Dieu d'Israël : On achètera encore des maisons, des champs et des vignes, dans ce pays.
\TextTitle{Promesse du retour des Juifs en Israël}
\VS{16}Et après que j'eus donné à Baruc, fils de Nérija, le contrat d'acquisition, je fis cette prière à Yahweh, en disant :
\VS{17}Ah ! Ah ! Seigneur Yahweh, voici, tu as fait les cieux et la terre par ta grande puissance et par ton bras étendu : Aucune chose n'est étonnante de ta part.
\VS{18}Tu fais miséricorde jusqu'à la millième génération, et tu punis l'iniquité des pères dans le sein de leurs fils après eux\FTNT{Ex. 34:7 ; Es. 65:7 ; Ps. 79:12.}. Tu es le Dieu, le Grand, le Puissant, dont le nom est Yahweh des armées.
\VS{19}Tu es grand en conseil et puissant en actions ; tes yeux sont ouverts sur toutes les voies des fils des hommes, pour rendre à chacun selon ses voies, et selon le fruit de ses œuvres.
\VS{20}Tu as fait dans le pays d'Egypte des miracles et des prodiges qui sont connus jusqu'à ce jour, et en Israël et parmi les hommes, tu t'es fait un nom tel qu'il est aujourd'hui.
\VS{21}Car tu as fait sortir du pays d'Egypte ton peuple d'Israël, avec des miracles et des prodiges, et avec une main forte, et avec un bras étendu, et en répandant partout une grande terreur ;
\VS{22}Et tu leur as donné ce pays, que tu avais juré à leurs pères de leur donner, pays où coulent le lait et le miel.
\VS{23}Et ils y sont entrés, ils l'ont possédé ; mais ils n'ont pas obéi à ta voix, et n'ont pas marché dans ta loi, et n'ont pas fait tout ce que tu leur avais ordonné de faire. C'est pourquoi tu as fait arriver sur eux tout ce mal-ci !
\VS{24}Voilà, les terrasses sont élevées, on est venu contre la ville pour la prendre, et à cause de l’épée, de la famine, et de la peste, la ville est livrée entre la main des Chaldéens qui combattent contre elle ; et ce que tu as dit est arrivé, et voici, tu le vois.
\VS{25}Et cependant, Seigneur Yahweh ! Tu m'as dit : Achète-toi ce champ à prix d'argent, et prends-en des témoins, quoique la ville soit livrée entre les mains des Chaldéens.
\VS{26}Mais la parole de Yahweh fut adressée à Jérémie, en disant :
\VS{27}Voici, je suis Yahweh, le Dieu de toute chair. Y a-t-il quelque chose d'étonnant de ma part ?
\VS{28}C'est pourquoi ainsi parle Yahweh : Voici, je vais livrer cette ville entre les mains des Chaldéens, et entre les mains de Nebucadnetsar, roi de Babylone, qui la prendra.
\VS{29}Et les Chaldéens qui combattent contre cette ville, y entreront, et mettront le feu à cette ville, et la brûleront, avec les maisons sur les toits desquelles on a brûlé de l'encens à Baal, et où l'on a fait des libations à d'autres dieux pour m'irriter.
\VS{30}Car les fils d'Israël et les fils de Juda n'ont fait, dès leur jeunesse, que ce qui est mal à mes yeux ; les fils d'Israël n'ont fait que m'irriter par les œuvres de leurs mains, dit Yahweh.
\VS{31}Car cette ville a été portée à provoquer ma colère et ma fureur, depuis le jour qu'ils l'ont bâtie, jusqu'à ce jour, afin que je l'ôte de devant ma face ;
\VS{32}à cause de tout le mal que les fils d'Israël et les fils de Juda ont fait pour m'irriter, eux, leurs rois, leurs chefs, leurs sacrificateurs et leurs prophètes, les hommes de Juda et les habitants de Jérusalem.
\VS{33}Ils m'ont tourné le dos, et non la face ; je les ai enseignés, je les ai enseignés dès le matin, mais ils n'ont pas écouté pour recevoir l'instruction.
\VS{34}Mais ils ont mis leurs abominations dans la maison sur laquelle mon Nom est invoqué, pour la souiller.
\VS{35}Et ils ont bâti les hauts lieux de Baal, qui sont dans la vallée de Ben-Hinnom, pour faire passer par le feu leurs fils et leurs filles à Moloc\FTNT{Voir commentaire en Lé. 20:2.} ; ce que je ne leur avais pas ordonné, et il ne m'était pas monté à la pensée qu'ils feraient cette abomination pour faire pécher Juda.
\VS{36}Et maintenant, à cause de cela Yahweh, le Dieu d'Israël, ainsi parle sur cette ville dont vous dites qu’elle est livrée entre les mains du roi de Babylone, à cause que l’épée, la famine, et la peste sont en elle :
\VS{37}Voici, je vais les rassembler de tous les pays où je les ai chassés, dans ma colère, dans ma fureur et dans mon grand courroux ; et je les ramènerai dans ce lieu-ci, et je les y ferai habiter en sécurité.
\VS{38}Et Ils seront mon peuple, et je serai leur Dieu.
\VS{39}Et je leur donnerai un même cœur et une même voie, afin qu'ils me craignent à toujours, pour leur bien et celui de leurs fils après eux.
\VS{40}Et je traiterai avec eux une alliance éternelle, à savoir, que je ne me détournerai plus d'eux pour leur faire du bien ; et je mettrai ma crainte dans leur cœur, afin qu'ils ne se détournent pas de moi\FTNT{Es. 54:10.}.
\VS{41}Et je me réjouirai à leur faire du bien, et je les planterai dans ce pays-ci solidement, de tout mon cœur, et de toute mon âme.
\VS{42}Car ainsi parle Yahweh : Comme j'ai fait venir tous ce grand mal sur ce peuple, ainsi je ferai venir sur eux tout le bien que je prononce en leur faveur.
\VS{43}Et on achètera des champs dans ce pays, duquel vous dites que ce n'est que désolation, sans hommes ni bêtes, et qui est livré entre les mains des Chaldéens.
\VS{44}On achètera, dis-je, des champs à prix d'argent, et on en écrira les contrats, et on les cachettera, et on en prendra des témoins dans le pays de Benjamin, et aux environs de Jérusalem, dans les villes de Juda, tant dans les villes des montagnes, que dans les villes de la plaine, et dans les villes du midi. Car je ramènerai leurs captifs, dit Yahweh.
\Chap{33}
\TextTitle{Jésus, le Germe appelé à régner\FTNTT{Voir 2 S. 7:8-16.}}
\VerseOne{}Et la parole de Yahweh fut adressée une seconde fois à Jérémie, quand il était encore enfermé dans la cour de la prison, en disant :
\VS{2}Ainsi parle Yahweh, qui fait ces choses, Yahweh qui les forme et les établit, lui dont le nom est Yahweh :
\VS{3}Crie vers moi\FTNT{Yahweh, qui demandait qu'on l'invoque, n'est autre que Jésus-Christ, notre Seigneur (Joë. 2:32 ; 1 Co. 1:2 ; Ro. 10:13).}, je te répondrai, et je t'annoncerai des choses grandes, des choses cachées, que tu ne connais pas.
\VS{4}Car ainsi parle Yahweh, le Dieu d'Israël, touchant les maisons de cette ville-ci et les maisons des rois de Juda ; elles seront abattues par les terrasses et par l'épée.
\VS{5}Ils sont venus pour combattre contre les Chaldéens, mais ça été pour remplir leurs maisons des cadavres des hommes que j'ai frappé dans ma colère et dans ma fureur, et parce que j'ai caché ma face de cette ville à cause toute leur méchanceté.
\VS{6}Voici, je vais lui donner la santé et la guérison, je les guérirai, et je leur ferai découvrir une abondance de paix et de fidélité\FTNT{Ap. 22:1-2.}.
\VS{7}Et je ramènerai les captifs de Juda, et les captifs d'Israël, et je les rétablirai comme autrefois.
\VS{8}Et je les purifierai de toute leur iniquité, par laquelle ils ont péché contre moi ; et je pardonnerai toutes leurs iniquités par lesquelles ils ont péché contre moi, et par lesquelles ils se sont révoltés contre moi\FTNT{Ez. 37:23.}.
\VS{9}Et cette ville sera pour moi un sujet de joie, de louange et de gloire, parmi toutes les nations de la terre qui entendront parler de tout le bien que je leur ferai, et elles seront dans la crainte et trembleront à cause de tout le bien et de toute la prospérité que je vais lui donner.
\VS{10}Ainsi parle Yahweh : Dans ce lieu-ci duquel vous dites : Il est désert, il n'y a plus d'hommes, plus de bêtes, dans les villes de Juda, et dans les rues de Jérusalem, qui sont désolées, privées d'hommes, d'habitants, de bêtes,
\VS{11}on y entendra encore les cris de joie et les cris d'allégresse, la voix de l'époux et la voix de l'épouse, et la voix de ceux qui disent : Louez Yahweh des armées ; car Yahweh est bon, parce que sa miséricorde demeure à toujours, lorsqu'ils offriront des offrandes de reconnaissance dans la maison de Yahweh ; car je ramènerai les captifs de ce pays, et je les rétablirai comme autrefois, dit Yahweh.
\VS{12}Ainsi parle Yahweh des armées : Dans ce lieu désert, où il n'y a ni hommes ni bêtes, et dans toutes ses villes, il y aura encore des demeures de bergers qui y feront reposer leurs troupeaux ;
\VS{13}dans les villes des montagnes, et dans les villes de la plaine, dans les villes du midi, dans le pays de Benjamin et aux environs de Jérusalem, et dans les villes de Juda ; les brebis passeront encore sous les mains de celui qui les compte, dit Yahweh.
\VS{14}Voici, les jours viennent, dit Yahweh, où j'accomplirai la bonne parole que j'ai prononcée sur la maison d'Israël et la maison de Juda.
\VS{15}En ces jours et en ce temps-là, je ferai germer à David le Germe de justice, qui exercera le jugement et la justice dans le pays.
\VS{16}En ces jours-là, Juda sera sauvé, Jérusalem habitera en sécurité ; et voici comment on l'appellera : Yahweh notre justice.
\VS{17}Car ainsi parle Yahweh : David ne manquera jamais d'un successeur assis sur le trône de la maison d'Israël ;
\VS{18}et d'entre les sacrificateurs Lévites, il ne manquera jamais d'y avoir devant moi d'homme offrant des holocaustes, brûlant de l'encens avec les offrandes, et faisant des sacrifices tous les jours.
\VS{19}La parole de Yahweh fut encore adressée à Jérémie, en disant :
\VS{20}Ainsi parle Yahweh : Si vous pouvez rompre mon alliance avec le jour et mon alliance avec la nuit, de sorte que le jour et la nuit ne soient plus en leur temps,
\VS{21}alors aussi mon alliance avec David, mon serviteur, sera rompue ; de sorte qu'il n'aura plus de fils régnant sur son trône ; et avec les Lévites sacrificateurs, faisant mon service.
\VS{22}Car comme on ne peut compter l'armée des cieux, ni mesurer le sable de la mer, ainsi je multiplierai la postérité de David mon serviteur, et les Lévites qui font mon service\FTNT{Ge. 2:1 ; Ge. 15:5.}.
\VS{23}La parole de Yahweh fut encore adressée à Jérémie, en disant :
\VS{24}N'as-tu pas vu ce que ce peuple prononce, en disant : Yahweh a rejeté les deux familles qu'il avait élues ? Ainsi ils méprisent mon peuple, ils ne sont plus une nation devant eux.
\VS{25}Ainsi parle Yahweh : Si je n'ai pas fait mon alliance avec le jour et la nuit, et si je n'ai pas établi les ordonnances des cieux et de la terre ;
\VS{26}aussi rejetterai-je la postérité de Jacob, et celle de David mon serviteur, pour ne plus prendre de sa postérité des gens qui dominent sur les descendants d'Abraham, d'Isaac et de Jacob ; car je ramènerai leurs captifs, et j'aurai compassion d'eux.
\Chap{34}
\TextTitle{Désobéissance du peuple : Jérusalem dévastée}
\VerseOne{}La parole qui fut adressée à Jérémie de la part de Yahweh, lorsque Nebucadnetsar, roi de Babylone, et toute son armée, et tous les royaumes de la terre, et tous les peuples qui étaient sous la puissance de sa main, combattaient contre Jérusalem, et contre toutes ses villes, en disant\FTNT{2 R. 25:1-2.} :
\VS{2}Ainsi parle Yahweh, le Dieu d'Israël : Va, et parle à Sédécias, roi de Juda, et dis-lui : Ainsi parle Yahweh : Voici, je vais livrer cette ville entre les mains du roi de Babylone, et il la brûlera par le feu.
\VS{3}Et tu n'échapperas pas de sa main, car certainement tu seras pris et tu seras livré entre ses mains, et tes yeux verront les yeux du roi de Babylone, et il te parlera bouche à bouche, et tu iras à Babylone.
\VS{4}Toutefois écoute la parole de Yahweh, ô Sédécias, roi de Juda ! Ainsi parle Yahweh sur toi : Tu ne mourras pas par l'épée ;
\VS{5}mais tu mourras en paix, et on brûlera pour toi des parfums aromatiques, comme on en a brûlé pour tes pères, les rois précédents qui ont été avant toi ; et on te pleurera, en disant : Hélas, Seigneur ! Car j'ai prononcé cette parole, dit Yahweh\FTNT{2 Ch. 16:14.}.
\VS{6}Jérémie, le prophète, dit toutes ces paroles à Sédécias, roi de Juda, à Jérusalem.
\VS{7}Et l'armée du roi de Babylone combattait contre Jérusalem et contre toutes les villes de Juda qui restaient, à savoir, contre Lakis et contre Azéka, car c'étaient les seules villes fortifiées qui restaient parmi les villes de Juda\FTNT{2 R. 18:13.}.
\TextTitle{Jérusalem deviendra une désolation à cause de la désobéissance}
\VS{8}La parole fut adressée à Jérémie de la part de Yahweh, après que le roi Sédécias eut traité une alliance avec tout le peuple de Jérusalem, pour proclamer la liberté,
\VS{9}afin que chacun renvoie libre son esclave et chacun sa servante, l'hébreu ou la femme de l'hébreu, et qu'aucun juif ne soit l'esclave de son frère.
\VS{10}Tous les chefs et tout le peuple, qui étaient entrés dans cette alliance, entendirent que chacun devait renvoyer libre son serviteur et chacun sa servante, sans plus les asservir ; ils obéirent et les renvoyèrent.
\VS{11}Mais ensuite, ils changèrent d'avis ; ils firent revenir leurs esclaves et leurs servantes, qu'ils avaient renvoyés libres, et les assujettirent pour être leurs esclaves et leurs servantes.
\VS{12}Et la parole de Yahweh fut adressée à Jérémie en disant :
\VS{13}Ainsi parle Yahweh, le Dieu d'Israël : J'ai traité une alliance avec vos pères, le jour où je les ai fait sortir du pays d'Egypte, de la maison de servitude, en disant :
\VS{14}A la fin de la septième année, chacun renverra libre son frère hébreu qui aura été vendu ; il te servira six années, puis tu le renverras libre de chez toi. Mais vos pères ne m'ont pas écouté, ils n'ont pas prêté l'oreille\FTNT{Ex. 21:2 ; Lé. 25:10-15 ; De. 15:12.}.
\VS{15}Et vous, qui aujourd'hui étiez revenus à vous-mêmes, et vous aviez fait ce qui était droit à mes yeux, en publiant la liberté chacun pour son prochain, vous aviez traité une alliance devant moi, dans la maison sur laquelle mon Nom est invoqué.
\VS{16}Mais vous êtes revenus en arrière, et vous avez souillé mon Nom ; vous avez fait revenir chacun ses esclaves et ses servantes, que vous aviez renvoyés libres, rendus à eux-mêmes, et vous les avez assujettis, afin qu'ils soient pour vous des serviteurs et des servantes.
\VS{17}C'est pourquoi ainsi parle Yahweh : Vous ne m'avez pas obéi, en publiant la liberté chacun à son frère, et chacun à son prochain. Voici, je vais publier contre vous, dit Yahweh, la liberté contre vous à l'épée, à la peste, et à la famine ;  et je vous livrerai pour être transportés par tous les royaumes de la terre.
\VS{18}Et je livrerai les hommes qui ont transgressé mon alliance, et qui n'ont pas observé les paroles de l'alliance qu'ils avaient traitée devant moi, lorsqu'ils sont passés entre les morceaux du veau qu'ils ont coupé en deux ;
\VS{19}les chefs de Juda, et les chefs de Jérusalem, les eunuques, et les sacrificateurs, et tout le peuple du pays, qui sont passés au travers des morceaux du veau ;
\VS{20}je les livrerai, dis-je, entre les mains de leurs ennemis, entre les mains de ceux qui cherchent leur vie ; et leurs cadavres seront la pâture des oiseaux des cieux et des bêtes de la terre.
\VS{21}Je livrerai aussi Sédécias, roi de Juda, et les chefs de sa cour, entre les mains de leurs ennemis, entre les mains de ceux qui cherchent leur vie, entre les mains de l'armée du roi de Babylone, qui s'est retiré de devant vous.
\VS{22}Voici, je vais leur donner mes ordres, dit Yahweh, et je les ramènerai contre cette ville-ci ; et ils combattront contre elle, et la prendront, et la brûleront au feu ; et je ferai des villes de Juda un désert sans habitants.
\Chap{35}
\TextTitle{L'obéissance des Récabites}
\VerseOne{}C'est ici la parole qui fut adressée à Jérémie de la part de Yahweh, au temps de Jojakim, fils de Josias, roi de Juda, en disant :
\VS{2}Va à la maison des Récabites, et parle-leur, et fais les venir à la maison de Yahweh, dans l'une des chambres, et présente-leur du vin à boire\FTNT{2 S. 4:2 ; 1 Ch. 2:55.}.
\VS{3}Je pris donc Jaazania, fils de Jérémie, fils de Habazinia, et ses frères, et tous ses fils, et toute la maison des Récabites,
\VS{4}et je les fis venir dans la maison de Yahweh, dans la chambre des fils de Hanan, fils de Jigdalia, homme de Dieu, qui était près de la chambre des chefs, au-dessus de la chambre de Maaséja, fils de Schallum, garde du seuil.
\VS{5}Et je mis devant les fils de la maison des Récabites des coupes pleines de vin, et des calices, et je leur dis : Buvez du vin !
\VS{6}Et ils répondirent : Nous ne buvons pas de vin ; car Jonadab, fils de Récab, notre père, nous a donné cet ordre en disant : Vous ne boirez jamais de vin, ni vous ni vos fils\FTNT{Lé. 10:9 ; No. 6:2-4.} ;
\VS{7}vous ne bâtirez aucune maison, vous ne sèmerez aucune semence, vous ne planterez aucune vigne, et vous n'en aurez pas ; mais vous habiterez sous des tentes toute votre vie, afin que vous viviez longtemps sur la terre où vous êtes étrangers.
\VS{8}Nous avons donc obéi à la voix de Jonadab, fils de Récab, notre père dans toutes les choses qu'il nous a ordonnées, de sorte que nous n'avons pas bu de vin tous les jours de notre vie, ni nous, ni nos femmes, ni nos fils, ni nos filles.
\VS{9}Nous n'avons bâti aucune maison pour notre demeure, et nous n'avons eu ni vigne, ni champ, ni semence.
\VS{10}Mais nous avons habité sous des tentes, et nous avons obéi, et nous avons fait selon toutes les choses que Jonadab, notre père, nous a ordonnées.
\VS{11}Mais il est arrivé que quand Nebucadnetsar, roi de Babylone, est monté au pays, nous avons dit : Venez, et entrons dans Jérusalem, pour fuir de devant l'armée des Chaldéens, et de devant l'armée de Syrie. C'est ainsi que nous habitons à Jérusalem.
\VS{12}Alors la parole de Yahweh fut adressée à Jérémie, en disant :
\VS{13}Ainsi parle Yahweh des armées, le Dieu d'Israël : Va, et dis aux hommes de Juda, et aux habitants de Jérusalem : Ne recevrez-vous pas d'instruction pour obéir à mes paroles ? Dit Yahweh.
\VS{14}Toutes les paroles de Jonadab, fils de Récab, qui a ordonné à ses fils de ne pas boire de vin, ont été observées, et ils n'en ont pas bu jusqu'à ce jour ; mais ils ont obéi au commandement de leur père ; mais moi, je vous ai parlé, je vous ai parlé dès le matin, et vous ne m'avez pas obéi.
\VS{15}Car je vous ai envoyé tous les prophètes, mes serviteurs, je les ai envoyés dès le matin, pour vous dire : Revenez maintenant chacun de votre mauvaise voie, et amendez vos actions, et n'allez pas après d'autres dieux pour les servir, afin que vous demeureriez dans le pays que j'ai donné à vous et à vos pères. Mais vous n'avez pas prêté l'oreille, et vous ne m'avez pas écouté.
\VS{16}Parce que les fils de Jonadab, fils de Récab, ont observé le commandement que leur avait donné leur père, et que ce peuple ne m'écoute pas ;
\VS{17}à cause de cela, Yahweh le Dieu des armées, le Dieu d'Israël, parle ainsi : Voici, je vais faire venir sur Juda et sur tous les habitants de Jérusalem tout le mal que j'ai prononcés contre eux ; parce que je leur ai parlé, et ils n'ont pas écouté ; et que je les ai appelés, et ils n'ont pas répondu.
\VS{18}Et Jérémie dit à la maison des Récabites: Ainsi parle Yahweh des armées, le Dieu d'Israël : Parce que vous avez obéi au commandement de Jonadab, votre père, et que vous avez gardé tous ses commandements, et avez fait selon tout ce qu'il vous a ordonné\FTNT{Les Récabites furent bénis parce qu'ils obéirent aux commandements de leur père (Ep. 6:1-3)} ;
\VS{19}c'est pourquoi, ainsi parle Yahweh des armées, le Dieu d'Israël : Jonadab, fils de Récab, ne manquera jamais de descendants qui se tiennent debout devant moi.
\Chap{36}
\TextTitle{Le roi Jojakim brûle le manuscrit de Jérémie}
\VerseOne{}Or il arriva, dans la quatrième année de Jojakim, fils de Josias, roi de Juda, que cette parole fut adressée à Jérémie de la part de Yahweh, en disant :
\VS{2}Prends-toi un rouleau de livre, et tu y écriras toutes les paroles que je t'ai dites contre Israël et contre Juda, et contre toutes les nations, depuis le jour où je t'ai parlé, c'est à dire, depuis le jours de Josias, jusqu'à ce jour.
\VS{3}Peut-être que la maison de Juda entendra tout le mal que je pense de leur faire, afin que chaque homme se détourne de sa mauvaise voie, et que je leur pardonne leur iniquité, et leur péché.
\VS{4}Jérémie donc appela Baruc, fils de Nérija, et Baruc écrivit, sous la dictée de Jérémie, dans le rouleau de livre, toutes les paroles que Yahweh lui avait dites.
\VS{5}Puis Jérémie donna cet ordre à Baruc, en disant : Je suis retenu, et je ne peux pas entrer dans la maison de Yahweh.
\VS{6}Tu y entreras donc et tu liras dans le rouleau que tu as écrit sous ma dictée, toutes les paroles de Yahweh, aux oreilles du peuple dans la maison de Yahweh le jour du jeûne ; tu les liras, dis-je, aussi aux oreilles de tous ceux de Juda qui seront venus de leurs villes.
\VS{7}Peut-être que Yahweh écoutera leur supplication et qu'ils reviendront chacun de leur mauvaise voie ; car grande est la colère, la fureur que Yahweh a déclarée contre ce peuple.
\VS{8}Baruc donc, fils de Nérija, fit selon tout ce que lui avait ordonné Jérémie le prophète, lisant dans le rouleau les paroles de Yahweh, dans la maison de Yahweh.
\VS{9}Or il arriva dans la cinquième année de Jojakim, fils de Josias, roi de Juda, le neuvième mois, qu'on publia le jeûne devant Yahweh à tout le peuple de Jérusalem et à tout le peuple venu des villes de Juda à Jérusalem.
\VS{10}Et Baruc lut dans le livre les paroles de Jérémie, aux oreilles de tout le peuple, dans la maison de Yahweh, dans la chambre de Guemaria, fils de Schaphan, le secrétaire, dans le parvis supérieur, à l'entrée de la porte neuve de la maison de Yahweh.
\VS{11}Et quand Michée fils de Guemaria, fils de Schaphan, eut entendu toutes les paroles de Yahweh contenues dans le livre ;
\VS{12}il descendit dans la maison du roi, vers la chambre du secrétaire, et voici tous les chefs y étaient assis, à savoir, Elischama le secrétaire, Delaja, fils de Schemaeja, Elnathan, fils de Acbor, et Guemaria, fils de Schaphan, et Sédécias, fils de Hanania, et tous les chefs.
\VS{13}Et Michée leur rapporta toutes les paroles qu'il avait entendues, quand Baruc lisait dans le livre, aux oreilles du peuple.
\VS{14}C'est pourquoi tous les chefs envoyèrent vers Baruc, Jehudi, fils de Nethania, fils de Schélémia, fils de Cuschi, pour lui dire : Prends en ta main le rouleau que tu as lu aux oreilles du peuple, et viens ici ! Baruc donc, fils de Nérija, prit le rouleau en sa main, et vint vers eux.
\VS{15}Et ils lui dirent : Assieds-toi maintenant, et lis-le à nos oreilles ; et Baruc le lut à leurs oreilles.
\VS{16}Et il arriva que sitôt qu'ils eurent entendu toutes les paroles, ils furent effrayés entre eux, et dirent à Baruc : Nous ne manquerons pas de rapporter au roi toutes ces paroles.
\VS{17}Et ils interrogèrent Baruc, en disant : Dis-nous comment tu as écrit toutes ces paroles sous sa dictée.
\VS{18}Et Baruc leur dit : Il me dictait de sa bouche toutes ces paroles, et je les écrivais avec de l'encre dans le livre.
\VS{19}Alors les chefs dirent à Baruc : Va, et cache-toi, ainsi que Jérémie, et que personne ne sache où vous serez.
\VS{20}Puis ils s'en allèrent vers le roi dans la cour, mais ils prirent soin de laisser le rouleau dans la chambre d'Elischama le secrétaire ; et ils racontèrent toutes ces paroles aux oreilles du roi.
\VS{21}Et le roi envoya Jehudi pour prendre le rouleau ; et quand Jehudi l'eut prit de la chambre d'Elischama le secrétaire, et il le lut aux oreilles du roi et de tous les chefs qui étaient autour de lui.
\VS{22}Or le roi était assis dans la maison d'hiver, au neuvième mois, et un brasier était allumé devant lui.
\VS{23}Et il arriva qu'aussitôt que Jehudi en eut lu trois ou quatre feuilles, le roi déchira le rouleau avec le canif du secrétaire, et le jeta au feu du brasier, jusqu'à ce que tout le rouleau fut consumé au feu du brasier.
\VS{24}Et ni le roi ni tous ses serviteurs qui entendirent toutes ces paroles, n'en furent pas effrayés, et ne déchirèrent pas leurs vêtements.
\VS{25}Toutefois Elnathan, et Delaja et Guemaria intercédèrent envers le roi, afin qu'il ne brûle pas le rouleau ; mais il ne les écouta pas.
\VS{26}Même le roi ordonna à Jerachmeel, fils de Hammélec, et à Seraja, fils d'Azriel, et à Schélémia, fils de Abdeel, de saisir Baruc, le secrétaire, et Jérémie le prophète ; mais Yahweh les cacha.
\TextTitle{Remplacement du manuscrit brûlé ; jugement sur Jojakim}
\VS{27}Et la parole de Yahweh fut adressée à Jérémie, après que le roi eut brûlé le rouleau contenant les paroles que Baruc avait écrites sous la dictée de Jérémie, en disant :
\VS{28}Prends encore un autre rouleau, et tu y écriras toutes les premières paroles qui étaient dans le premier rouleau que Jojakim, roi de Juda, a brûlé.
\VS{29}Et tu diras à Jojakim, roi de Juda : Ainsi parle Yahweh : Tu as brûlé ce rouleau, et tu as dit : Pourquoi y as-tu écrit ces paroles : Le roi de Babylone viendra certainement, il détruira ce pays, et il exterminera les hommes et les bêtes ?
\VS{30}C'est pourquoi ainsi parle Yahweh sur Jojakim, roi de Juda : Aucun des siens ne sera assis sur le trône de David, et son cadavre sera jeté de jour à la chaleur et de nuit à la gelée.
\VS{31}Je le punirai, lui, sa postérité, et ses serviteurs, à cause de leur iniquité ; et je ferai venir sur eux, et sur les habitants de Jérusalem, et sur les hommes de Juda, tout le mal que je leur ai prononcé, et qu'ils n'ont pas écouter.
\VS{32}Jérémie donc prit un autre rouleau, et le donna à Baruc, fils de Nérija secrétaire, lequel y écrivit, sous la dictée de Jérémie, toutes les paroles du rouleau que Jojakim, roi de Juda, avait brûlé au feu. Beaucoup de paroles semblables y furent encore ajoutées.
\Chap{37}
\TextTitle{Sédécias sollicite l'intercession de Jérémie}
\VerseOne{}Or le roi Sédécias, fils de Josias, régna à la place de Jéconia, fils de Jojakim, et il fut établi roi dans le pays de Juda par Nebucadnetsar, roi de Babylone.
\VS{2}Mais, ni lui, ni ses serviteurs, ni le peuple du pays, n'obéirent pas aux paroles que Yahweh prononça par Jérémie le prophète.
\VS{3}Toutefois le roi Sédécias envoya Jucal, fils de Schélémia, et Sophonie, fils de Maaséja sacrificateur, vers Jérémie le prophète, pour lui dire : Intercède pour nous auprès de Yahweh, notre Dieu.
\VS{4}Car Jérémie allait et venait parmi le peuple, parce qu'on ne l'avait pas encore mis en prison.
\VS{5}Alors l'armée de Pharaon sortit d'Egypte, et quand les Chaldéens qui assiégeaient Jérusalem en entendirent cette nouvelle, ils se retirèrent de devant Jérusalem.
\VS{6}Et la parole de Yahweh fut adressée à Jérémie le prophète, en disant :
\VS{7}Ainsi parle Yahweh, le Dieu d'Israël : Vous direz ainsi au roi de Juda, qui vous a envoyés me consulter : Voici, l'armée de Pharaon, qui était sortie à votre secours, retourne dans son pays, en Egypte ;
\VS{8}et les Chaldéens reviendront, et combattront contre cette ville, et la prendront, et la brûleront au feu.
\VS{9}Ainsi parle Yahweh : Ne vous abusez pas vous-mêmes, en disant : Les Chaldéens s'en iront loin de nous ; car ils ne s'en iront pas.
\VS{10}Même quand vous auriez battu toute l'armée des Chaldéens qui combattent contre vous, et qu'il n'y aurait de reste entre eux que des hommes percés de blessures, ils se relèveront pourtant chacun dans sa tente, et brûleront cette ville au feu.
\TextTitle{Jérémie calomnié et emprisonné}
\VS{11}Or il arriva que quand l'armée des Chaldéens se fut retirée de Jérusalem, à cause de l'armée de Pharaon,
\VS{12}Jérémie sortit de Jérusalem, pour s'en aller dans le pays de Benjamin, se glissant hors de là au milieu du peuple.
\VS{13}Mais quand il fut à la porte de Benjamin, il y avait là un commandant de la garde, nommé Jireija, fils de Schélémia, fils de Hanania, qui saisit Jérémie le prophète, en lui disant : Tu vas te rendre aux Chaldéens !
\VS{14}Et Jérémie répondit : C'est un mensonge ! Je ne vais pas me rendre aux Chaldéens. Mais il ne l'écouta pas, et Jireija prit Jérémie, et l'amena vers les chefs.
\VS{15}Et les chefs se mirent en colère contre Jérémie, et le frappèrent et le mirent en prison dans la maison de Jonathan le secrétaire, car ils en avaient fait une prison.
\VS{16}Et ainsi Jérémie entra dans la fosse de la maison et dans les cachots ; et Jérémie y demeura plusieurs jours.
\VS{17}Mais le roi Sédécias y envoya, et l'en tira, et il l'interrogea en secret dans sa maison, et lui dit : Y a-t-il une parole de la part de Yahweh ? Et Jérémie répondit : Il y en a une ; Et lui dit : Tu seras livré entre les mains du roi de Babylone.
\VS{18}Puis Jérémie dit au roi Sédécias : Quel péché ai-je commis contre toi, contre tes serviteurs, et contre ce peuple, pour que vous m'ayez mis en prison ?
\VS{19}Mais où sont vos prophètes qui vous prophétisaient, en disant : Le roi de Babylone ne reviendra pas contre vous, ni contre ce pays ?
\VS{20}Or écoute maintenant, je te prie, ô roi, mon seigneur ! Et que maintenant ma supplication soit reçue devant ta face, et ne me renvoie pas dans la maison de Jonathan le secrétaire, de peur que je n'y meure !
\VS{21}C'est pourquoi le roi Sédécias ordonna qu'on garde Jérémie dans la cour de la prison, et qu'on lui donne chaque jour un pain de la rue des boulangers, jusqu'à ce que tout le pain de la ville soit épuisé. Ainsi Jérémie demeura dans la cour de la prison.
\Chap{38}
\TextTitle{Jérémie jeté dans la fosse puis délivré par Ebed-Mélec l'éthiopien}
\VerseOne{}Mais Schephathia, fils de Matthan, et Guedalia, fils de Paschhur, et Jucal, fils de Schélémia, et Paschhur, fils de Malkija, entendirent les paroles que Jérémie prononçait à tout le peuple, en disant :
\VS{2}Ainsi parle Yahweh : Celui qui restera dans cette ville mourra par l'épée, par la famine, ou par la peste ; mais celui qui sortira vers les Chaldéens vivra, et sa vie sera son butin, et il vivra.
\VS{3}Ainsi parle Yahweh : Cette ville sera livrée certainement aux mains de l'armée du roi de Babylone, qui la prendra.
\VS{4}Et les chefs dirent au roi : Qu'on fasse mourir cet homme ! Car il décourage les mains des hommes de guerre qui restent dans cette ville, et les mains de tout le peuple, en leur disant de telles paroles ; parce que cet homme ne cherche pas le bien de ce peuple, mais le mal.
\VS{5}Et le roi Sédécias dit : Voici, il est entre vos mains ; car le roi ne peut rien contre vous.
\VS{6}Ils prirent donc Jérémie, et le jetèrent dans la fosse de Malkija, fils de Hammélec, laquelle était dans la cour de la prison, et ils descendirent Jérémie avec des cordes dans cette fosse où Il n'y avait pas d'eau mais de la boue ; et ainsi Jérémie enfonça dans la boue.
\VS{7}Mais Ebed-Mélec l'éthiopien, eunuque, qui était dans la maison du roi, apprit qu'ils avaient mis Jérémie dans cette fosse ; et le roi était assis à la porte de Benjamin.
\VS{8}Et Ebed-Mélec sortit de la maison du roi, et parla au roi, en disant :
\VS{9}Ô roi, mon seigneur ! Ces hommes-là ont mal fait dans tout ce qu'ils ont fait contre Jérémie le prophète, en le jetant dans la fosse, car il serait déjà mort de faim dans le lieu où il était parce qu'il n'y a plus de pain dans la ville.
\VS{10}C'est pourquoi le roi donna cet ordre à Ebed-Mélec l'éthiopien, en disant : Prends ici trente hommes avec toi, et fais remonter hors de la fosse Jérémie le prophète, avant qu'il meure.
\VS{11}Ebed-Mélec donc prit des hommes avec lui, et entra dans la maison du roi, dans un lieu au-dessous du trésor, d'où il prit de vieux lambeaux et de vieux chiffons, et les descendit avec des cordes à Jérémie dans la fosse.
\VS{12}Et Ebed-Mélec l'éthiopien dit à Jérémie : Mets ces vieux lambeaux et ces chiffons sous les aisselles de tes bras, au-dessous des cordes. Et Jérémie fit ainsi.
\VS{13}Ainsi ils tirèrent Jérémie dehors avec les cordes, et le firent remonter hors de la fosse ; et Jérémie demeura dans la cour de la prison.
\TextTitle{Jérémie appelle Sédécias à la repentance}
\VS{14}Et le roi Sédécias envoya chercher Jérémie le prophète, et le fit amener vers lui à la troisième entrée qui était dans la maison de Yahweh. Et le roi dit à Jérémie : Je vais te demander une chose, ne me cache rien.
\VS{15}Et Jérémie répondit à Sédécias : Quand je te l'aurais déclarée, n'est-il pas vrai que tu me feras mourir ? Et quand je t'aurai donné conseil, tu ne m'écouteras pas.
\VS{16}Alors le roi Sédécias jura secrètement à Jérémie, en disant : Yahweh est vivant, qui nous a fait cette âme, je ne te ferai pas mourir, et que je ne te livrerai pas entre les mains de ces hommes qui cherchent ta vie.
\VS{17}Alors Jérémie dit à Sédécias : Ainsi parle Yahweh, le Dieu des armées, le Dieu d'Israël : Si tu sors volontairement pour aller vers les chefs du roi de Babylone, tu auras la vie, et cette ville ne sera pas brûlée par le feu ; et tu vivras toi et ta maison.
\VS{18}Mais si tu ne sors pas vers les chefs du roi de Babylone, cette ville sera livrée entre les mains des Chaldéens, qui la brûleront par le feu ; et tu n'échapperas pas à leurs mains.
\VS{19}Et le roi Sédécias dit à Jérémie : Je crains à cause des Juifs qui se sont rendus aux Chaldéens, je crains qu'on ne me livre entre leurs mains et qu’ils ne se moquent de moi.
\VS{20}Et Jérémie lui répondit : On ne te livrera pas à eux. Je te prie, écoute la voix de Yahweh dans ce que je te dis ; tu t'en trouveras bien, et tu auras la vie.
\VS{21}Que si tu refuses de sortir, voici ce que Yahweh m'a fait voir :
\VS{22}C'est que, voici toutes les femmes qui restent dans la maison du roi de Juda seront menées aux chefs du roi de Babylone, et elles diront : Tu as été séduit, vaincu, par les hommes qui te prédisaient la paix ; et quand tes pieds sont enfoncés dans la boue, ils se sont retirés en arrière.
\VS{23}Toutes tes femmes et tes fils seront menés dehors aux Chaldéens ; et tu n'échapperas pas à leurs mains, mais tu seras pris, pour être livré entre les mains du roi de Babylone, et à cause de toi, cette ville sera brûlée par le feu.
\VS{24}Alors Sédécias dit à Jérémie : Que personne ne sache rien de ces paroles, et tu ne mourras pas.
\VS{25}Et si les chefs entendent que je t'ai parlé, et qu'ils viennent vers toi, et te dise : Déclare-nous maintenant ce que tu as dit au roi, et ce que le roi t'a dit, ne nous en cache rien, et nous ne te ferons pas mourir ;
\VS{26}tu leur diras : J'ai présenté ma supplication devant le roi afin qu'il ne me renvoie pas dans la maison de Jonathan, pour y mourir.
\VS{27}Tous les chefs donc vinrent vers Jérémie, et l'interrogèrent ; mais il leur répondit exactement comme le roi lui avait ordonné ; et ils gardèrent le silence, car l'affaire n'avait pas été divulguée.
\VS{28}Ainsi Jérémie demeura dans la cour de la prison, jusqu'au jour où Jérusalem fut prise, et il y était lorsque Jérusalem fut prise.
\Chap{39}
\TextTitle{Prise de Jérusalem ; Sédécias déporté à Babylone\FTNTT{2 R. 25:1-7; Jé. 52:4-17 ; 2 Ch. 36:17-21.}}
\VerseOne{}La neuvième année de Sédécias, roi de Juda, au dixième mois, Nebucadnetsar, roi de Babylone, vint avec toute son armée contre Jérusalem, et ils l'assiégèrent.
\VS{2}Et la onzième année de Sédécias, le neuvième jour du quatrième mois, une brèche fut faite à la ville.
\VS{3}Et tous les chefs du roi de Babylone y entrèrent, et s'assirent à la porte du milieu, à savoir, Nergal-Scharetser, Samgar-Nebu, Sarsekim, chef des eunuques, Nergal-Scharetser, chef des devins et tous les autres chefs du roi de Babylone.
\VS{4}Or il arriva qu'aussitôt que Sédécias, roi de Juda, et tous les hommes de guerre les eurent vus, ils s'enfuirent et sortirent de nuit hors de la ville, par le chemin du jardin du roi, par la porte entre les deux murailles, et ils s'en allèrent par le chemin de la plaine.
\VS{5}Mais l'armée des Chaldéens les poursuivit et atteignit Sédécias dans les plaines de Jéricho. Ils le prirent, et le firent monter vers Nebucadnetsar, roi de Babylone, à Ribla, dans le pays de Hamath, où il prononça contre lui une sentence.
\VS{6}Et le roi de Babylone fit égorger à Ribla les fils de Sédécias sous ses yeux ; le roi de Babylone fit aussi égorger tous les nobles de Juda.
\VS{7}Puis il fit crever les yeux à Sédécias, et le fit lier de doubles chaînes d'airain, pour le conduire à Babylone.
\VS{8}Les Chaldéens brûlèrent par le feu la maison royale et les maisons du peuple, et démolirent\FTNT{Ici débute le « temps des nations » (587-586 av J.-C.), Jérusalem est foulée aux pieds par les nations. Voir aussi 2 R. 25:8-24 ; 2 Ch. 36:17-21.} les murailles de Jérusalem.
\VS{9}Et Nebuzaradan, chef des gardes, transporta à Babylone le reste du peuple qui était resté dans la ville, et ceux qui s'étaient rendus à lui, le reste, dis-je, du peuple qui avait été épargné.
\VS{10}Mais Nebuzaradan, chefs des gardes, laissa dans le pays de Juda les plus pauvres du peuple qui n'avaient rien ; et en ce jour-là, il leur donna des vignes et des champs.
\TextTitle{Jérémie libéré de prison}
\VS{11}Or Nebucadnetsar, roi de Babylone, avait donné cet ordre au sujet de Jérémie, à Nebuzaradan, chef des gardes, en disant :
\VS{12}Prends cet homme et veille sur lui ; ne lui fais aucun mal, mais fais pour lui tout ce qu'il te dira.
\VS{13}Nebuzaradan donc chefs des gardes, envoya, et aussi Nebuschazban, Rabsaris, chef des eunuques, Nergal-Scharetser, Rabmag, chef des devins, et tous les chefs du roi de Babylone ;
\VS{14}ils envoyèrent, dis-je, chercher Jérémie dans la cour de la prison, et le remirent à Guedalia, fils d'Achikam, fils de Schaphan, pour qu'il le conduise dans sa maison. Ainsi il demeura au milieu du peuple.
\TextTitle{Yahweh épargne Ebed-Mélec}
\VS{15}Or la parole de Yahweh fut adressée à Jérémie pendant qu'il était enfermé dans la cour de la prison, en disant :
\VS{16}Va, et parle à Ebed-Mélec l'éthiopien, et dis-lui : Ainsi parle Yahweh des armées, le Dieu d'Israël : Voici, je vais faire venir mes paroles sur cette ville pour son malheur et non pas pour son bien, et elles s'accompliront en ce jour-là devant toi.
\VS{17}Mais je te délivrerai en ce jour-là, dit Yahweh, et tu ne seras pas livré entre les mains des hommes que tu crains.
\VS{18}Car certainement je te ferai échapper, et tu ne tomberas pas sous l'épée ; mais ta vie sera ton butin, parce que tu as eu confiance en moi, dit Yahweh.
\Chap{40}
\TextTitle{Assassinat de Guedalia et meurtres en série d'Ismaël}
\VerseOne{}La parole qui fut adressée à Jérémie de la part de Yahweh, quand Nebuzaradan, chef des gardes, l'eut renvoyé de Rama, après l'avoir pris lorsqu'il était lié de chaînes parmi tous les captifs de Jérusalem et de Juda qu'on transportait à Babylone.
\VS{2}Quand donc le chef des gardes prit Jérémie, et il lui dit : Yahweh, ton Dieu, a prononcé ce mal contre ce lieu-ci ;
\VS{3}et Yahweh l'a fait venir et a fait comme il avait dit, parce que vous avez péché contre Yahweh, et que vous n'avez pas écouté sa voix, à cause de cela ceci vous est arrivé.
\VS{4}Maintenant donc voici, je t'affranchis aujourd'hui des chaînes que tu as aux mains ; s'il est bon à tes yeux de venir avec moi à Babylone, viens, et j'aurai les yeux sur toi ; mais s'il est mauvais de venir avec moi à Babylone, ne viens pas ; regarde, tout le pays est à ta disposition, va où il te semblera bon et convenable d'aller.
\VS{5}Or Guedalia ne retournera plus ici ; retourne, dit-il, vers Guedalia, fils d'Achikam, fils de Schaphan, que le roi de Babylone a établi sur les villes de Juda, et demeure avec lui parmi le peuple ; ou bien, va partout où il conviendra à tes yeux d'aller. Et le chef des gardes lui donna des vivres et quelques présents, et le renvoya.
\VS{6}Jérémie donc alla vers Guedalia, fils d'Achikam, à Mitspa, et demeura avec lui parmi le peuple qui était resté dans le pays.
\VS{7}Et tous les chefs des armées qui étaient dans les champs, eux et leurs hommes, entendirent que le roi de Babylone avait établi Guedalia, fils d'Achikam, sur le pays, et qu'il lui avait commis les hommes, et les femmes, et les enfants, et ceux-là d'entre les plus pauvres du pays, à savoir, de ceux qui n'avaient pas été transportés à Babylone.
\VS{8}Alors ils allèrent vers Guedalia à Mitspa ; à savoir, Ismaël fils de Nethania, et Jochanan et Jonathan fils de Karéach, et Seraja fils de Thanhumeth, et les fils d'Ephaï de Nethopha, et Jezania fils du Maacatite, eux et leurs hommes.
\VS{9}Et Guedalia, fils d'Achikam, fils de Schaphan, leur jura, à eux et à leurs hommes, en disant : Ne craignez pas de servir les Chaldéens ; demeurez dans le pays, et servez le roi de Babylone, et vous vous en trouverez bien.
\VS{10}Et pour moi, voici, je resterai à Mitspa, pour me tenir prêt à recevoir les ordres des Chaldéens qui viendront vers nous ; mais vous, recueillez le vin, les fruits d'été et l'huile, et mettez-les dans vos vases, et demeurez dans vos villes que vous avez prises pour votre demeure.
\VS{11}Pareillement aussi tous les Juifs qui étaient au pays de Moab, et parmi les Ammonites, et au pays d'Edom, et dans toutes ces contrées, quand il eurent entendu que le roi de Babylone avait laissé quelque reste à Juda, et qu'il avait établi sur eux Guedalia, fils d'Achikam, fils de Schaphan ;
\VS{12}tous ces juifs-là retournèrent de tous les lieux où ils avaient été chassés, et vinrent dans le pays de Juda vers Guedalia à Mitspa, et recueillirent du vin et des fruits d'été en grande abondance.
\VS{13}Mais Jochanan, fils de Karéach, et tous les chefs des armées qui étaient dans les champs, vinrent vers Guedalia à Mitspa,
\VS{14}et lui dirent : Ne sais-tu pas certainement que Baalis, roi des Ammonites, a envoyé Ismaël, le fils de Nethania, pour t'ôter la vie ? Mais Guedalia, fils d'Achikam, ne les crut pas.
\VS{15}Et Jochanan, fils de Karéach parla en secret à Guedalia à Mitspa, en disant : Laisse-moi aller et frapper Ismaël, fils de Nethania, et personne ne le saura. Pourquoi t'ôterait-il la vie, afin que tous les Juifs qui se sont rassemblés vers toi soient dissipés, et que les restes de Juda périssent ?
\VS{16}Mais Guedalia, fils d'Achikam, dit à Jochanan, fils de Karéach : Ne fais pas cela, car tu parles faussement d'Ismaël.
\Chap{41}
\TextTitle{Assassinat de Guedalia}
\VerseOne{}Or il arriva, au septième mois, qu'Ismaël, fils de Nethania, fils d'Elischama, de la race royale, et l'un des grands du roi et dix hommes avec lui, vinrent vers Guedalia, fils d'Achikam, à Mitspa ; et ils mangèrent là du pain ensemble à Mitspa\FTNT{2 R. 25:25.}.
\VS{2}Mais Ismaël, fils de Nethania, se leva, et les dix hommes qui étaient avec lui, et ils frappèrent avec l'épée Guedalia, fils d'Achikam, fils de Schaphan, et on le fit mourir, lui que le roi de Babylone avait établi sur le pays.
\VS{3}Ismaël frappa aussi tous les juifs qui étaient avec Guedalia à Mitspa, et les Chaldéens, gens de guerre, qui se trouvaient là.
\VS{4}Et il arriva que le second jour après après qu'on eut fait mourir Guedalia, avant que personne le sût,
\VS{5}quelques hommes de Sichem, de Silo et de Samarie, au nombre de quatre-vingts hommes, ayant la barbe rasée et les vêtements déchirés, et s'étant fait des incisions, vinrent avec des dons et de l'encens dans leurs mains pour les apporter dans la maison de Yahweh.
\VS{6}Alors Ismaël, fils de Nethania, sortit de Mitspa au-devant d'eux, et il marchait en pleurant, et quand il les rencontra, il leur dit : Venez vers Guedalia, fils d'Achikam.
\VS{7}Mais sitôt qu'ils arrivèrent au milieu de la ville, Ismaël, fils de Nethania, accompagné des hommes qui étaient avec lui, les égorgea et les jeta dans une fosse.
\VS{8}Mais il se trouva parmi eux dix hommes, qui dirent à Ismaël : Ne nous fais pas mourir, car nous avons dans les champs des provisions cachées de froment, d'orge, d'huile et de miel ; et il les laissa, et ne les fit pas mourir avec leurs frères.
\VS{9} Et la fosse dans laquelle Ismaël jeta les cadavres des hommes qu'il tua, à l'occasion de Guedalia, est celle que le roi Asa avait faite, lorsqu'il craignait Baescha, roi d'Israël ; et Ismaël, fils de Nethania, la remplit de gens tués\FTNT{1 R. 15:22.}.
\VS{10}Et Ismaël emmena captif tout le reste du peuple qui était à Mitspa, les filles du roi et tous ceux du peuple qui demeuraient à Mitspa, que Nebuzaradan, chef des gardes, avait commis à Guedalia, fils d'Achikam ; Ismaël, fils de Nethania, les emmena captifs, et s'en alla pour passer vers les Ammonites.
\TextTitle{Jochanan délivre le peuple ; fuite d'Ismaël}
\VS{11}Mais Jochanan, fils de Karéach, et tous les chefs des armées qui étaient avec lui, entendirent tout le mal qu'Ismaël, fils de Nethania, avait fait ;
\VS{12}et ils prirent tous les hommes, et s'en allèrent pour combattre contre Ismaël, fils de Nethania. Ils le trouvèrent près des grandes eaux qui sont à Gabaon.
\VS{13}Et il arriva qu'aussitôt que tout le peuple qui était avec Ismaël vit Jochanan, fils de Karéach, et tous les chefs des armées qui étaient avec lui, ils s'en réjouirent ;
\VS{14}et tout le peuple qu'Ismaël avait emmené captif de Mitspa tourna visage, et revenant sur leur pas, il s'en alla vers Jochanan, fils de Karéach.
\VS{15}Mais Ismaël, fils de Nethania, échappa avec huit hommes devant Jochanan, et s'en alla vers les Ammonites.
\VS{16}Et Jochanan, fils de Karéach, et tous les chefs des armées qui étaient avec lui, prirent tout le reste du peuple qu'ils avaient retiré des mains d'Ismaël, fils de Nethania, qu'il emmenait captif de Mitspa, après avoir tué Guedalia, fils d'Achikam, à savoir, les vaillants hommes de guerre, et les femmes, et les enfants et les eunuques ; et les ramenèrent de Gabaon.
\VS{17}Et ils s'en allèrent et demeurèrent à l'hôtellerie de Kimham, près de Bethléhem, pour se retirer ensuite en Egypte,
\VS{18}à cause des Chaldéens ; car ils avaient peur d'eux, parce qu'Ismaël, fils de Nethania, avait tué Guedalia, fils d'Achikam, qui avait été établi sur le pays par le roi de Babylone.
\Chap{42}
\TextTitle{Yahweh défend au reste du peuple de se réfugier en Egypte }
\VerseOne{}Alors tous les chefs des armées, et Jochanan, fils de Karéach, et Jezania, fils d'Hosée, et tout le peuple, depuis le plus petit jusqu'au plus grand, s'approchèrent,
\VS{2}et dirent à Jérémie le prophète : Que notre supplication soit favorable devant toi ! Intercède auprès de Yahweh, ton Dieu, pour nous, à savoir, pour tout ce reste-ci ; car de beaucoup de monde que nous étions, nous sommes restés peu, comme tes yeux nous voient ;
\VS{3}et que Yahweh, ton Dieu, nous déclare le chemin par lequel nous aurons à marcher, et ce que nous avons à faire !
\VS{4}Et Jérémie le prophète, leur répondit : J'ai entendu votre demande ; voici, je vais prier Yahweh, votre Dieu, selon vos paroles ; et il arrivera que je vous déclarerai tout ce que Yahweh vous répondra, et je ne vous en cacherai pas un mot.
\VS{5}Et ils dirent à Jérémie : Yahweh soit entre nous un témoin véritable et fidèle, si nous ne faisons pas selon toutes les paroles que Yahweh, ton Dieu, t'enverra vers nous !
\VS{6}Soit bien, soit mal, nous obéirons à la voix de Yahweh, notre Dieu, vers qui nous t'envoyons, afin qu'il nous arrive du bien, quand nous aurons obéi à la voix de Yahweh, notre Dieu.
\VS{7}Et il arriva, au bout de dix jours, que la parole de Yahweh fut adressée à Jérémie.
\VS{8}Et il appela Jochanan, fils de Karéach, tous les chefs des armées qui étaient avec lui, et tout le peuple, depuis le plus petit jusqu'au plus grand ;
\VS{9}et leur dit : Ainsi parle Yahweh, le Dieu d'Israël, vers qui vous m'avez envoyé, pour présenter votre supplication devant lui :
\VS{10}Si vous continuez à demeurer dans ce pays, je vous rétablirai et je ne vous détruirai pas ; je vous y planterai et je ne vous arracherai pas, car je me repens du mal que je vous ai fait.
\VS{11}Ne craignez pas le roi de Babylone, dont vous avez peur, ne craignez pas, dit Yahweh, car je suis avec vous pour vous sauver et pour vous délivrer de sa main.
\VS{12}Même je vous ferai obtenir miséricorde, tellement qu'il aura pitié de vous, et vous fera retourner dans votre pays.
\VS{13}Que si vous dites : Nous ne demeurerons pas dans ce pays, et nous n'écouterons pas la voix de Yahweh, notre Dieu,
\VS{14}en disant : Non ; mais nous irons au pays d'Egypte, afin que nous ne voyons pas de guerre, et que nous n'entendions pas le son du shofar, et que nous ne manquions pas de pain, et nous demeurerons là.
\VS{15}A cause de cela écoutez maintenant la parole de Yahweh, vous les restes de Juda ! Ainsi parle Yahweh des armées, le Dieu d'Israël : Si vous tournez le visage pour aller en Egypte, et que vous y entriez pour y demeurer ;
\VS{16}il arrivera que l'épée dont vous avez peur vous attrapera là au pays d'Egypte ; et la famine que vous craignez si fort vous suivra en Egypte, et vous y mourrez\FTNT{Ez. 30:9-11.}.
\VS{17}Et il arrivera que tous les hommes qui tourneront le visage pour aller en Egypte afin d'y demeurer, mourront par l'épée, par la famine et par la peste ; et il n'y aura ni survivant ni réchappé devant le mal que je vais faire venir sur eux.
\VS{18}Car ainsi parle Yahweh des armées, le Dieu d'Israël : Comme ma colère et ma fureur se sont répandues sur les habitants de Jérusalem, ainsi ma fureur sera versée sur vous, quand vous serez entrés en Egypte ; et vous serez un sujet d'exécration, d'épouvante, de malédiction et d'opprobre, et vous ne verrez plus ce lieu-ci.
\VS{19}Vous, les restes de Juda, Yahweh dit contre vous : N'allez pas en Egypte ! Sachez certainement que je vous ai avertis aujourd'hui.
\VS{20}Car vous vous êtes séduits vous-mêmes dans vos âmes, quand vous m'avez envoyé vers Yahweh, votre Dieu, en me disant : Intercède pour nous auprès de Yahweh, notre Dieu, et déclare-nous tout ce que Yahweh, notre Dieu, te dira, et nous le ferons.
\VS{21}Et je vous l'ai déclaré aujourd'hui ; mais vous n'écoutez pas la voix de Yahweh, votre Dieu, ni rien de tout ce pour quoi il m'a envoyé vers vous.
\VS{22}Maintenant donc sachez certainement que vous mourrez par l'épée, par la famine et par la peste, dans le lieu où vous avez désiré d’aller pour y demeurer.
\Chap{43}
\TextTitle{Désobéissance des Hébreux ; jugement sur l'Egypte}
\VerseOne{}Or il arriva qu'aussitôt que Jérémie eut achevé de prononcer à tout le peuple toutes les paroles de Yahweh, leur Dieu, pour lesquelles Yahweh, leur Dieu, l'avait envoyé vers eux, à savoir, toutes ces choses-là ;
\VS{2}Azaria, fils d'Hosée, et Jochanan, fils de Karéach, et tous ces hommes orgueilleux, dirent à Jérémie : Tu dis un mensonge ; Yahweh, notre Dieu, ne t'a pas envoyé nous dire : N'allez pas en Egypte pour y demeurer.
\VS{3}Mais Baruc, fils de Nérija, t'incite contre nous, afin de nous livrer entre les mains des Chaldéens, pour nous faire mourir, et pour nous faire transporter à Babylone.
\VS{4}Ainsi Jochanan, fils de Karéach, et tous les chefs des armées, et tout le peuple, n'obéirent pas à la voix de Yahweh, pour demeurer dans le pays de Juda.
\VS{5}Car Jochanan, fils de Karéach, et tous les chefs des armées, prirent tous les restes de Juda qui étaient revenus de toutes les nations, parmi lesquelles ils avaient été chassés, pour demeurer dans le pays de Juda ;
\VS{6}les hommes, et les femmes, et les enfants, et les filles du roi, et toutes les personnes que Nebuzaradan, chef des gardes, avait laissées avec Guedalia, fils d'Achikam, fils de Schaphan ; ils prirent aussi Jérémie le prophète et Baruc, fils de Nérija.
\VS{7}Et ils entrèrent dans le pays d'Egypte, car ils n'obéirent pas à la voix de Yahweh, et ils vinrent jusqu'à Tachpanès.
\VS{8}Alors la parole de Yahweh fut adressée à Jérémie, à Tachpanès, en disant :
\VS{9}Prends dans ta main de grandes pierres, et cache-les dans l'argile, dans le four à briques qui est à l'entrée de la maison de Pharaon à Tachpanès, sous les yeux des Juifs ;
\VS{10}et dis-leur : Ainsi parle Yahweh des armées, le Dieu d'Israël : Voici, j'enverrai chercher Nebucadnetsar, roi de Babylone, mon serviteur, et je mettrai son trône sur ces pierres que j'ai cachées, et il étendra son dais sur elles ;
\VS{11}et Il viendra et frappera le pays d'Egypte. Ceux qui sont destinés à la mort, iront à la mort ; et ceux qui sont destinés à la captivité, iront en captivité ; et ceux qui sont destinés à l’épée, seront livrés à l’épée\FTNT{Ez. 29:9.} !
\VS{12}Et j'allumerai le feu dans les maisons des dieux d'Egypte, Nebucadnetsar les brûlera, et il emmènera captifs ceux d'Egypte, et il se parera des richesses du pays d'Egypte, comme le pasteur s'enveloppe de son vêtement, et il en sortira en paix\FTNT{Es. 19:1 ; Ez. 30:13.}.
\VS{13}Il brisera aussi les statues de Beth-Schémesch, qui est au pays d'Egypte, et il brûlera par le feu les maisons des dieux d'Egypte.
\Chap{44}
\TextTitle{Yahweh avertit les Juifs d'Egypte\FTNTT{Jé. 43:8-13}}
\VerseOne{}La parole qui fut adressée à Jérémie sur tous les Juifs qui demeuraient au pays d'Egypte, qui habitaient à Migdol, à Tachpanès, à Noph, et au pays de Pathros, en disant :
\VS{2}Ainsi parle Yahweh des armées, le Dieu d'Israël : Vous avez vu tous les malheurs que j'ai fait venir sur Jérusalem et sur toutes les villes de Juda : Voici, elles ne sont plus aujourd'hui que des ruines, et personne n'y habite,
\VS{3}à cause des méchancetés qu'ils ont faites pour m'irriter, en allant brûler de l'encens pour servir d'autres dieux, qu'ils n'ont pas connu, ni eux, ni vous, ni vos pères.
\VS{4}Et je vous ai envoyé tous mes serviteurs, les prophètes, me levant dès le matin, et les envoyant, pour vous dire : Ne commettez pas maintenant cette chose abominable que je hais.
\VS{5}Mais ils n'ont pas écouté, ils n'ont pas prêté l'oreille pour se détourner de leur méchanceté, afin de ne pas faire brûler de l'encens à d'autres dieux.
\VS{6}C'est pourquoi ma fureur et ma colère se sont répandues sur eux, et ont embrasé les villes de Juda et les rues de Jérusalem, qui ne sont réduites en désert et en désolation, comme il paraît aujourd'hui.
\VS{7}Maintenant donc, ainsi parle Yahweh, le Dieu des armées, le Dieu d'Israël : Pourquoi faites-vous ce grand mal contre vos âmes, pour vous faire exterminer du milieu de Juda, hommes et femmes, petits enfants et ceux qui tètent, afin qu'on ne vous laisse aucun reste ?
\VS{8}En m'irritant par les œuvres de vos mains, en brûlant de l'encens à d'autres dieux au pays d'Egypte, où vous êtes venus pour y demeurer, afin de vous faire exterminer et d'être un objet de malédiction et d'opprobre parmi toutes les nations de la terre ?
\VS{9}Avez-vous oublié les crimes de vos pères, les crimes des rois de Juda, les crimes de leurs femmes, vos propres crimes et les crimes de vos femmes, commis dans le pays de Juda et dans les rues de Jérusalem ?
\VS{10}Jusqu'à ce jour, ils ne se sont pas humiliés, ils n'ont pas eu de crainte, ils n'ont pas marché dans ma loi ni dans mes ordonnances, que j'ai mises devant vous et devant vos pères.
\VS{11}C'est pourquoi ainsi parle Yahweh des armées, le Dieu d'Israël : Voici, je tourne ma face contre vous pour vous nuire et vous retrancher tout Juda\FTNT{Am. 9:4.}.
\VS{12}Et je prendrai les restes de ceux de Juda qui ont tourné le visage pour aller au pays d'Egypte afin d'y demeurer ; ils seront tous consumés, ils tomberont dans le pays d'Egypte ; ils seront consumés par l'épée, par la famine, depuis le plus petit jusqu'au plus grand ; ils mourront par l'épée et par la famine ; et ils seront en exécration, en étonnement, en malédiction et en opprobre.
\VS{13}Et je punirai ceux qui demeurent au pays d'Egypte, comme j'ai puni Jérusalem, par l'épée, par la famine, et par la peste.
\VS{14}Il n'y aura personne du reste de Juda qui est venu dans le pays d’Égypte pour y séjourner, échappera ou restera, pour retourner dans le pays de Juda, où ils aspirent à retourner pour y demeurer ; car pas un ne retournera, sinon les rescapés.
\VS{15}Mais les tous hommes qui savaient que leurs femmes brûlaient de l'encens à d'autres dieux, toutes les femmes qui se tenaient là en grande compagnie, et tout le peuple qui demeurait au pays d'Egypte, à Pathros, répondirent à Jérémie, en disant :
\VS{16}Quant à la parole que tu nous as dite au nom de Yahweh, nous ne t'écouterons pas.
\VS{17}Mais nous ferons assurément selon toute parole qui est sortie de notre bouche, brûler de l'encens à la reine des cieux\FTNT{Voir commentaire en Jé. 7:18.}, et lui faire des libations, comme nous l'avons fait, nous et nos pères, nos rois et nos chefs, dans les villes de Juda et dans les rues de Jérusalem. Alors nous étions rassasiés de pain, nous étions heureux, et nous ne voyions pas le malheur\FTNT{Ez. 16:24 ; Ez. 20:32.}.
\VS{18}Mais depuis le temps que nous avons cessé de brûler de l'encens à la reine des cieux et de lui faire des libations, nous avons manqué de tout, et nous avons été consumés par l'épée et par la famine…
\VS{19}Quand nous brûlions de l'encens à la reine des cieux et que nous lui faisions des libations, est-ce à l'insu de nos maris que nous lui faisons des gâteaux sur lesquels elle est représentée et que nous lui faisons des libations ?
\VS{20}Alors Jérémie parla à tout le peuple, aux hommes, aux femmes, et à tous ceux qui lui avaient donné cette réponse, et leur dit :
\VS{21}Yahweh ne s'est-il pas souvenu, ne lui est-il pas monté à cœur l'encens que vous avez brûlé dans les villes de Juda et dans les rues de Jérusalem, vous et vos pères, vos rois et vos chefs, et le peuple du pays ?
\VS{22}Yahweh n'a pas pu le supporter davantage, à cause de la méchanceté de vos actions, à cause des abominations que vous avez faites ; et votre pays est devenu une ruine, un désert et un objet de malédiction, sans que personne y habite, comme on le voit aujourd'hui.
\VS{23}C'est parce que vous avez brûlé de l'encens et que vous avez péché contre Yahweh, parce que vous n'avez pas écouté la voix de Yahweh, et que vous n'avez pas marché dans sa loi, ni dans ses ordonnances, ni dans ses témoignages, c'est pour cela que ces malheurs vous sont arrivés, comme on le voit aujourd'hui.
\VS{24}Jérémie dit à tout le peuple et à toutes les femmes : Vous tous de Juda, qui êtes au pays d'Egypte, écoutez la parole de Yahweh !
\VS{25}Ainsi parle Yahweh des armées, le Dieu d'Israël, en disant : Vous et vos femmes, vous avez parlé de vos bouches et accompli de vos mains, en disant : Certainement, nous accomplirons nos vœux que nous avons faits, brûler de l'encens à la reine des cieux, et lui faire des libations. Vous avez entièrement accompli vos vœux, vous les avez effectués très exactement.
\VS{26}C'est pourquoi, écoutez la parole de Yahweh, vous tous de Juda, qui demeurez au pays d'Egypte ! Voici, je le jure par mon grand Nom, dit Yahweh, mon Nom ne sera plus invoqué par la bouche d'aucun homme de Juda, et dans tout le pays d'Egypte aucun ne dira : Le Seigneur Yahweh est vivant !
\VS{27}Voici, je veille sur eux pour leur mal et non pour leur bien ; et tous les hommes de Juda qui sont dans le pays d'Egypte seront consumés par l'épée et par la famine, jusqu'à ce qu'ils soient exterminés\FTNT{Da. 9:14.}.
\VS{28}Et ceux qui seront échappés à l'épée, retourneront du pays d'Egypte au pays de Juda en fort petit nombre. Mais tout le reste de Juda, tous ceux qui sont venus dans le pays d'Egypte pour y demeurer, sauront quelle est la parole qui s'accomplira, la mienne ou la leur.
\VS{29}Et ceci sera pour vous le signe, dit Yahweh, que je vous punirai dans ce lieu, afin que vous sachiez que mes paroles s'accompliront infailliblement pour votre malheur.
\VS{30}Ainsi parle Yahweh : Voici, je livrerai Pharaon Hophra, roi d'Egypte, entre les mains de ses ennemis, entre les mains de ceux qui cherchent sa vie, comme j'ai livré Sédécias, roi de Juda, entre les mains de Nebucadnetsar, roi de Babylone, son ennemi, et qui cherchait sa vie.
\Chap{45}
\TextTitle{Yahweh explique son dessein à Baruc}
\VerseOne{}La parole que Jérémie, le prophète, adressa à Baruc, fils de Nérija, quand il écrivit dans un livre ces paroles, sous la dictée de Jérémie, la quatrième année de Jojakim, fils de Josias, roi de Juda. Il dit :
\VS{2}Ainsi parle Yahweh, le Dieu d'Israël, sur toi, Baruc :
\VS{3}Tu dis : Malheur à moi ! Car Yahweh ajoute la tristesse à ma douleur ; je me suis lassé dans mon gémissement, et je ne trouve pas de repos.
\VS{4}Tu lui diras : Ainsi parle Yahweh : Voici, je vais détruire ce que j'ai bâti, et arracher ce que j'ai planté, à savoir tout ce pays.
\VS{5}Et toi, chercherais-tu de grandes choses ? Ne les cherche pas ! Car voici, je vais faire venir du mal sur toute chair, dit Yahweh ; je te donnerai ta vie pour butin, dans tous les lieux où tu iras.
\Chap{46}
\TextTitle{Prophétie contre l'Egypte}
\VerseOne{}La parole de Yahweh qui fut adressée à Jérémie, le prophète, sur les nations.
\VS{2}A l'égard de l'Egypte, contre l'armée de Pharaon Neco, roi d'Egypte, qui était près du fleuve de l'Euphrate, à Carkemisch, et qui fut battue par Nebucadnetsar, roi de Babylone, la quatrième année de Jojakim, fils de Josias, roi de Juda\FTNT{2 R. 24:7.}.
\VS{3}Préparez le bouclier et l'écu et approchez-vous pour la bataille !
\VS{4}Attelez les chevaux, montez, cavaliers ! Présentez-vous avec vos casques, polissez vos lances, revêtez l'armure !
\VS{5}D'où vient que je vois ceci ? Ils sont effrayés, ils reviennent en arrière ; leurs hommes vaillants sont battus ; ils s'enfuient avec précipitation sans regarder derrière eux… La frayeur les environne, dit Yahweh.
\VS{6}Que l'homme léger à la course ne s'enfuie pas, et que le fort ne se sauve pas\FTNT{Am. 2:14-16.} ! Ils sont renversés et tombés vers le nord, auprès du rivage du fleuve de l'Euphrate.
\VS{7}Qui est celui-ci qui s'élève comme le Nil et dont les eaux sont agitées comme les fleuves ?
\VS{8}C'est l'Egypte. Elle s'élève comme le Nil et ses eaux agitées comme les fleuves ; et elle dit : Je m'élèverai et je couvrirai la terre ; je détruirai la ville et ceux qui y habitent.
\VS{9}Montez, chevaux ! Agissez en insensés, chars ! Que les hommes vaillants sortent, ceux d'Ethiopie et de Puth qui manient le bouclier, et ceux de Lud qui manient et tendent l'arc\FTNT{Ez. 30:5-9 ; Na. 3:9-10.} !
\VS{10}Car c'est le jour du Seigneur,  Yahweh des armées ; c'est un jour de vengeance, où il se venge de ses ennemis. L'épée dévore, elle se rassasie, elle s'enivre de leur sang. Car il y a des sacrifices pour le Seigneur, Yahweh des armées, dans le pays du nord, sur le fleuve de l'Euphrate\FTNT{Es. 34:5-6 ; Ez. 39:17 ; So. 1:7.}.
\VS{11}Monte en Galaad, prends du baume, vierge, fille de l'Egypte ! En vain tu multiplies les remèdes, il n'y a pas de guérison pour toi\FTNT{Ez. 30:21-25 ; Na. 3:19.}.
\VS{12}Les nations apprennent ta honte, et tes cris remplissent la terre, car les hommes forts chancellent l'un sur l'autre, ils tombent tous deux ensemble.
\VS{13}La parole que Yahweh prononça à Jérémie, le prophète, sur la venue de Nebucadnetsar, roi de Babylone, pour frapper le pays d'Egypte :
\VS{14}Déclarez-le en Egypte, et publiez-le à Migdol, à Noph, et à Tachpanès et Dites : Présente-toi, tiens-toi prêt car l'épée dévore ce qui est autour de toi !
\VS{15}Pourquoi tes vaillants hommes sont-ils emportés ? Ils ne tiennent pas ferme, parce que Yahweh les pousse.
\VS{16}Il en a terrassé un grand nombre, et même chacun tombe sur son compagnon, et ils disent : Levons-nous, retournons vers notre peuple, au pays de notre naissance, loin de l'épée de l'oppresseur !
\VS{17}Là, ils s'écrient : Pharaon, roi d'Egypte, n'est qu'un bruit ; il a laissé passer le temps fixé.
\VS{18}Je suis vivant ! dit le Roi, dont le Nom est Yahweh des armées ; comme le Thabor entre les montagnes, comme le Carmel qui s'avance dans la mer, ainsi viendra-t-il.
\VS{19}O fille, habitante de l'Egypte, fais tes bagages pour la captivité ! Car Noph sera un désert, elle sera brûlée, elle n'aura plus d'habitants.
\VS{20}L'Egypte est une très belle génisse… la destruction vient, elle vient du nord.
\VS{21}Memes les mercenaires aussi sont au milieu d'elle comme des veaux engraissés. Et eux aussi tournent le dos, ils fuient tous sans résister. Car le jour de leur malheur, le temps de leur châtiment est venu sur eux.
\VS{22}Elle sifflera comme un serpent ; car ils marcheront avec une puissante armée, ils viendront contre elle avec des haches, comme des bûcherons.
\VS{23}Ils couperont sa forêt, dit Yahweh, quoiqu'elle soit impénétrable ; parce que leur armée est en plus grand nombre que les sauterelles, on ne saurait la compter.
\VS{24}La fille de l'Egypte est confuse, elle est livrée entre les mains du peuple du nord.
\VS{25}Yahweh des armées, le Dieu d'Israël, dit : Voici, je vais punir Amon de No, Pharaon, l'Egypte, ses dieux et ses rois, Pharaon et ceux qui se confient en lui.
\VS{26}Et je les livrerai entre les mains de ceux qui cherchent leur vie, entre les mains, dis-je, de Nebucadnetsar, roi de Babylone, et entre les mains de ses serviteurs ; mais après cela, l'Egypte sera habitée comme aux temps passés, dit Yahweh.
\VS{27}Et toi, Jacob, mon serviteur, ne crains pas ; ne t'épouvante pas Israël ! Car voici, je te sauverai de la terre lointaine, je sauverai ta postérité du pays de leur captivité ; Jacob reviendra, il sera en repos et en paix, et il n'y aura personne qui lui fasse peur.
\VS{28}Toi donc,Jacob, mon serviteur, ne crains pas ! dit Yahweh ; car je suis avec toi. Et même je consumerai entièrement  toutes les nations parmi lesquelles je t'ai chassé, mais je ne te consumerai pas entièrement ; et je te châtierai avec justice, je ne te tiendrai pas tout à fait pour innocent.
\Chap{47}
\TextTitle{Prophétie contre la Philistie et la Phénicie}
\VerseOne{}La parole de Yahweh, qui fut adressée à Jérémie, le prophète, contre les Philistins, avant que Pharaon frappe Gaza.
\VS{2}Ainsi parle Yahweh : Voici des eaux montent du nord, elles sont comme un torrent qui déborde ; elles inondent le pays et ce qu'il contient, les villes et leurs habitants. Les hommes poussent des cris, et tous les habitants du pays se lamentent,
\VS{3}à cause du bruit des battements de sabots de ses puissants chevaux, du bruit de ses chars et au son de ses roues ; les pères ne se tournent pas vers leurs fils, tant les mains sont affaiblies,
\VS{4}parce que le jour vient où seront détruits tous les Philistins, exterminés tout le reste de ceux qui servaient de secours à Tyr et à Sidon ; car Yahweh va détruire les Philistins, les restes de l'île de Caphtor.
\VS{5}Gaza est devenue chauve, Askalon est perdue, le reste de leur plaine aussi. Jusqu'à quand te feras-tu des incisions ?
\VS{6}Ah ! Epée de Yahweh, quand te reposeras-tu ? Rentre dans ton fourreau, repose-toi, et sois tranquille !
\VS{7}Mais comment te reposerais-tu ? Car Yahweh lui donne ses ordres, il l'a assignée contre Askalon et contre le rivage de la mer.
\Chap{48}
\TextTitle{Prophétie sur Moab}
\VerseOne{}Sur Moab. Ainsi parle Yahweh des armées, le Dieu d'Israël : Malheur à Nebo, car elle est dévastée ! Kirjathaïm est honteuse, elle est prise ; Misgab est honteuse et brisée.
\VS{2} Moab ne se glorifiera plus à Hesbon, car on a machiné du mal contre elle en disant : Allons, exterminons-la, qu'elle ne soit plus une nation ! Toi aussi, Madmen, tu seras détruite ; l'épée te poursuivra.
\VS{3}Il y a un bruit de clameur qui vient de Choronaïm ; c'est un ravage, une grande ruine.
\VS{4}Moab est brisé ! On entend les cris des plus jeunes.
\VS{5}Pleurs sur pleurs s’élèveront à la montée de Luchith, car on entendra à la descente de Choronaïm\FTNT{Es. 15:5.}. ceux qui crieront à cause des plaies que les ennemis leur auront faites.
\VS{6}Fuyez, dira-t-on, sauvez vos vies, et soyez comme un misérable dans le désert !
\VS{7}Car, parce que tu as eu confiance dans tes ouvrages, et dans tes trésors, tu seras pris, et Kemosch sortira pour être transporté, avec ses sacrificateurs et ses chefs\FTNT{Es. 46:1-7.}.
\VS{8}Et le dévastateur entrera dans toutes les villes, et aucune ville n'échappera ; la vallée périra et la plaine sera détruite, comme Yahweh l'a dit.
\VS{9}Donnez des ailes à Moab, et qu'il parte en volant ! Ses villes seront réduites en désert, elles n'auront plus d'habitants.
\VS{10}Maudit soit celui qui fait l'œuvre de Yahweh avec paresse, maudit soit celui qui garde son épée pour répandre le sang !
\VS{11}Moab était tranquille depuis sa jeunesse, il reposait sur sa lie, il n'était pas vidé de vase en vase, et il n'allait pas en captivité. C'est pourquoi son goût lui est resté, et son odeur ne s'est pas changée.
\VS{12}Mais voici, les jours viennent, dit Yahweh, où je lui enverrai des gens qui le transvaseront, qui videront ses vases, et qui briseront ses outres.
\VS{13}Moab aura honte à cause de Kemosch, comme la maison d'Israël a eu honte à cause de Béthel, qui était sa confiance.
\VS{14}Comment dites-vous : Nous sommes de vaillants hommes, des soldats prêts à combattre ?
\VS{15}Moab est dévasté, et chacune de ses villes monte en fumée, l'élite de sa jeunesse est descendue pour être égorgée, dit le Roi, dont le nom est Yahweh des armées.
\VS{16}La calamité de Moab est proche, son malheur avance à grands pas.
\VS{17}Vous tous qui êtes autour de lui, soyez-en émus à compassion, et vous tous qui connaissez son nom, dites : Comment a été rompue cette forte verge, et ce sceptre d'honneur ? 
\VS{18}Toi qui te tiens chez la fille de Dibon, descends de ta gloire, et assieds-toi dans un lieu desséché ! Car le dévastateur de Moab monte contre toi, il détruit tes forteresses.
\VS{19}Habitante d'Aroër, tiens-toi sur le chemin, et regarde ! Interroge celui qui s'enfuit, celui qui s'échappe, et dis : Qu'est-il arrivé ?
\VS{20}Moab est rendu honteux, car il est brisé. Poussez des gémissements et des cris\FTNT{Es. 15:5 ; Es. 16:7.} ! Rapportez dans Arnon que Moab est dévasté !
\VS{21}Et que jugement est venu sur le pays de la plaine, sur Holon, sur Jahats, sur Méphaath,
\VS{22}Et sur Dibon, sur Nebo, sur Beth-Diblathaïm,
\VS{23}Et sur Kirjathaïm, sur Beth-Gamul, sur Beth-Meon,
\VS{24}Et sur Kerijoth, sur Botsra, sur toutes les villes du pays de Moab, éloignées et proches.
\VS{25}La force de Moab est abattue, et son bras est brisé, dit Yahweh.
\VS{26}Enivrez-le, car il s'est élevé contre Yahweh ! Moab se vautrera dans le vin qu'il aura rendu et deviendra aussi un sujet de moquerie !
\VS{27}Car ô Moab! Israël n'a-t-il pas été pour toi un objet de moquerie ? Avait-il été trouvé parmi les voleurs, pour que tu ne dises des paroles qu'en secouant la tête ?
\VS{28}Habitants de Moab, quittez les villes, et demeurez dans les rochers ! Soyez comme les colombes qui font leur nid aux côtés de l'entrée des cavernes !
\VS{29}Nous avons appris l'extrême orgueil de Moab, son arrogance, sa fierté, et son cœur hautain\FTNT{Es. 16:6 ; So. 2:9-10.}.
\VS{30}J'ai connu son orgueil, dit Yahweh ; mais il n'en sera pas ainsi ; j'ai connu ceux sur lesquels il s'appuie ; ils n'ont rien fait de droit. 
\VS{31}Je hurlerai donc à cause de Moab, même je crierai à cause de Moab tout entier ; on gémira sur les gens de Kir-Hérès.
\VS{32}O vignoble de Sibma, je pleurerai sur toi du pleur de Jaezer ; tes rameaux allaient au-delà de la mer, ils atteignaient la mer de Jaezer ; le dévastateur s'est jeté sur tes fruits d'été et sur ta vendange.
\VS{33}L'allégresse ausi et la et la joie se sont retirées loin des campagnes et du pays de Moab ; j'ai fait cesser le vin des cuves ; on ne foule plus gaîment au pressoir ; il y a des cris de guerre, et non des cris de joie\FTNT{Es. 16:10.}.
\VS{34}A cause de cris de Hesbon qui est parvenu jusqu'à Elealé, et ils font entendre leurs cris jusqu'à Jahats, même depuis Tsoar jusqu'à Choronaïm, jusqu'à Eglath-Schelischija ; car les eaux de Nimrim seront aussi réduites en désolation.
\VS{35}Je ferai cesser en Moab, dit Yahweh, celui qui offre sur les hauts lieux et celui qui brûle de l’encens à ses dieux. 
\VS{36}C’est pourquoi mon cœur mènera un bruit sur Moab, comme des flûtes ; mon cœur mènera un bruit comme des flûtes sur les hommes de Kir-Hérès, parce que tous les biens qu'ils ont acquis ont péri.
\VS{37}Car toutes les têtes sont chauves, toutes les barbes sont coupées ; et il y a des incisions sur toutes les mains, et sur les reins des sacs.
\VS{38}Il y aura des lamentations sur tous les toits de Moab et dans ses places, parce que j'aurai brisé Moab comme un vase auquel on ne prend nul plaisir, dit Yahweh.
\VS{39}Hurlez, en disant : Comment a-t-il été mis en pièces ? brisé ! Comment Moab a-t-il tourné honteusement le dos ! Car Moab sera un objet de moquerie et de frayeur pour tous ceux qui sont autour de lui.
\VS{40}Car ainsi parle Yahweh : Voici, il volera comme un aigle, et il étendra ses ailes sur Moab.
\VS{41}Kerijoth est prise, les forteresses sont saisies, et le cœur des hommes forts de Moab est en ce jour comme le cœur d'une femme qui est en travail.
\VS{42}Et Moab sera exterminé, il ne sera plus un peuple, parce qu'il s'est élevé contre Yahweh.
\VS{43}Habitant de Moab, la frayeur, la fosse, et le filet sont sur toi ! dit Yahweh.
\VS{44}Celui qui s'enfuira à cause de la frayeur tombera dans la fosse, et celui qui remontera de la fosse sera au filet ; car je fera venir sur lui, sur Moab, l'année de son châtiment, dit Yahweh\FTNT{Es. 24:18.}.
\VS{45}Ils se sont arrêtés à l'ombre de Hesbon, voulant éviter la force ; mais le feu sort de Hesbon, une flamme du milieu de Sihon ; elle dévore les flancs de Moab, et le sommet de la tête des fils du tumulte\FTNT{No. 21:28.}.
\VS{46}Malheur à toi, Moab ! Le peuple de Kemosch est perdu ! Car tes fils sont enlevés et emmenés captifs, et tes filles ont été emmenées captives.
\VS{47}Toutefoi je ramènerai et mettrai en repos les captifs de Moab, aux derniers jours\FTNT{Gn. 49:1-2.}, dit Yahweh. Là est le jugement de Moab.
\Chap{49}
\TextTitle{Prophétie sur Ammon}
\VerseOne{}Sur les fils d'Ammon. Ainsi parle Yahweh : Israël n'a-t-il pas de fils ? N'a-t-il pas d'héritier ? Pourquoi donc Malcom hérite-t-il de Gad, et pourquoi son peuple demeure-t-il dans ses villes ?
\VS{2}C'est pourquoi, les jours viennent, dit Yahweh, où je ferai entendre le cri de guerre contre Rabbath des fils d'Ammon ; elle sera réduite en un monceau de ruines, et les villes de son ressort seront brûlées par le feu ; Israël possédera ceux qui l'auront possédé, dit Yahweh.
\VS{3}Hurle, ô Hesbon, car Aï est dévastée ! Poussez des cris, filles de Rabba, ceignez-vous de sacs, lamentez-vous, courez ça et là le long des murailles ! Car Malcom s'en va en captivité avec ses sacrificateurs et ses chefs.
\VS{4}Pourquoi te glorifies-tu de tes vallées ? Ta vallée se fond, fille rebelle, qui te confiais dans tes trésors : Qui viendra contre moi ?
\VS{5}Voici, je fais venir sur toi la terreur, dit le Seigneur, Yahweh des armées, de tous les alentours ; vous serez chassés chacun çà et là, et il n'y aura personne qui rassemblera les fuyards.
\VS{6}Mais après cela, je ramènerai les captifs des fils d'Ammon, dit Yahweh.
\TextTitle{Prophétie sur Edom}
\VS{7}Sur Edom. Ainsi parle Yahweh des armées : N'y a-t-il plus de sagesse dans Théman ? Le conseil a-t-il manqué aux hommes intelligents ? Leur sagesse s'est-elle évanouie\FTNT{Ab. 1:8.} ?
\VS{8}Fuyez, tournez le dos, demeurez dans les cavernes, habitants de Dedan ! Car je fais venir la détresse sur Esaü, le temps de son châtiment.
\VS{9}Si des vendangeurs entrent chez toi, ne laissent-ils rien à grappiller ? Si des voleurs viennent de nuit, ils ne pillent que ce qu'ils peuvent.
\VS{10}Mais je dépouillerai Esaü, je découvrirai ses lieux secrets, il ne pourra se cacher ; sa postérité, ses frères, et ses voisins, périront, et il ne sera plus.
\VS{11}Laisse tes orphelins, je les ferai vivre, et que tes veuves se confient en moi !
\VS{12}Car ainsi parle Yahweh : Voici, ceux dont le jugement n'était pas de boire la coupe, la boiront certainement ; et toi, tu resterais impuni ! Tu ne resteras pas impuni, tu la boiras.
\VS{13}Car je le jure par moi-même, dit Yahweh, que Botsra sera un objet de désolation, d'opprobre, de dévastation, et de malédiction, et que toutes ses villes deviendront des ruines éternelles.
\VS{14}J'ai entendu de Yahweh une nouvelle, et un messager a été envoyé parmi les nations : Assemblez-vous, et venez contre elle ! Levez-vous pour la guerre !
\VS{15}Car voici, je te rendrai petit entre les nations, méprisé entre les hommes.
\VS{16}Mais ta présomption, l'orgueil de ton cœur t'a séduit, toi qui habites dans le creux des rochers, et qui occupes le sommet des collines. Quand tu aurais élevé ton nid comme l'aigle, je t'en ferai descendre, dit Yahweh.
\VS{17}Edom sera un objet de désolation ; quiconque passera près de lui sera étonné, et sifflera à cause de toutes ses plaies.
\VS{18}Comme Sodome et Gomorrhe et les villes voisines qui furent détruites, dit Yahweh, il ne sera plus habité par des hommes, il ne sera le séjour d'aucun fils d'homme\FTNT{Ge. 19:25 ; Am. 4:11.}…
\VS{19}Voici, il monte comme un lion des rives orgueilleuses du Jourdain, vers la demeure forte ; soudain, j'en ferai fuir Edom, et j'établirai sur elle celui que j'ai choisi. Car qui est semblable à moi ? Qui me donnera des ordres ? Et quel est le chef qui me résistera en face\FTNT{Job. 41:1.} ?
\VS{20}C'est pourquoi écoutez le conseil que Yahweh a donné contre Edom, et les desseins qu'il a projetés contre les habitants de Théman ! Certainement, on les traînera comme les plus petits du troupeau, certainement on dévastera leur demeure.
\VS{21}La terre tremble au bruit de leur chute ; le bruit de leur cri se fait entendre jusqu'à la Mer Rouge…
\VS{22}Voici, il monte comme un aigle, il vole, il étend ses ailes sur Botsra, et le cœur des hommes forts d'Edom est en ce jour comme le cœur d'une femme en travail.
\TextTitle{Prophétie sur Damas}
\VS{23}Sur Damas. Hamath et Arpad sont honteuses parce qu'elles ont entendu des mauvaises nouvelles, elles tremblent ; il y a une tourmente dans la mer qui ne peut se calmer.
\VS{24}Damas est défaillante, elle se tourne pour fuir, et la panique la saisit ; l'angoisse et les douleurs la saisissent comme une femme qui enfante.
\VS{25}Ah ! Elle n'est pas abandonnée, la ville glorieuse, ma ville de plaisance !
\VS{26}C'est pourquoi ses jeunes gens tomberont dans les places, et tous ses hommes de guerre périront en ce jour, dit Yahweh des armées.
\VS{27}Je mettrai le feu à la muraille de Damas, qui dévorera les palais de Ben-Hadad.
\TextTitle{Prophétie sur Kédar (les Arabes) et Hatsor}
\VS{28}Sur Kédar et les royaumes de Hatsor, que Nebucadnetsar, roi de Babylone, frappa. Ainsi parle Yahweh : Levez-vous, montez vers Kédar, et détruisez les fils d'orient !
\VS{29}On prendra leurs tentes et leurs troupeaux, on prendra leurs tentes, tous leurs bagages et leurs chameaux, et l'on jettera de toutes parts contre eux des cris de terreur.
\VS{30}Fuyez, fuyez de toutes vos forces, cherchez une demeure dans les cavernes, vous habitants de Hatsor ! dit Yahweh ; car Nebucadnetsar, roi de Babylone, a pris une résolution contre vous, il a imaginé un plan contre vous.
\VS{31}Levez-vous, montez vers la nation tranquille qui habite en sécurité, dit Yahweh ; elle n'a ni portes ni barres, elle habite seule\FTNT{Ez. 38:11.}.
\VS{32}Leurs chameaux seront au pillage, et la multitude de leur bétail sera une proie ; je les disperserai à tout vent, vers ceux qui se coupent le coin de la barbe, et je ferai venir de tous les côtés leur détresse, dit Yahweh.
\VS{33}Hatsor sera le repaire des serpents, un désert pour toujours ; personne n'y habitera, et aucun fils d'homme n'y séjournera.
\TextTitle{Prophétie sur Elam}
\VS{34}La parole de Yahweh fut adressée à Jérémie, le prophète, sur Elam, au commencement du règne de Sédécias, roi de Juda, en disant :
\VS{35}Ainsi parle Yahweh des armées : Voici, je vais briser l'arc d'Elam, qui est sa principale force\FTNT{Ez. 32:24-27.}.
\VS{36}Je ferai venir contre Elam les quatre vents des quatre extrémités des cieux, je les disperserai par tous ces vents ; et il n'y aura pas une nation où ne viennent ceux qui seront chassés d'Elam éternellement.
\VS{37}Je ferai trembler les habitants d'Elam devant leurs ennemis, et devant ceux qui cherchent leur vie, je ferai venir le malheur sur eux, l'ardeur de ma colère, dit Yahweh, et j'enverrai l'épée après eux, jusqu'à ce que je les aie consumés.
\VS{38}Je mettrai mon trône dans Elam, et j'en détruirai les rois et les chefs, dit Yahweh.
\VS{39}Mais dans les derniers jours\FTNT{Gn. 49:1-2.}, je ramènerai les captifs d'Elam, dit Yahweh.
\Chap{50}
\TextTitle{Prophétie sur Babylone}
\VerseOne{}La parole que Yahweh prononça sur Babylone, sur le pays des Chaldéens, par le moyen de Jérémie, le prophète :
\VS{2}Annoncez-le parmi les nations, entendez-le, levez une bannière ! Entendez-le, ne le cachez pas ! Dites : Babylone est prise ! Bel est confus, Merodac est brisé ! Ses idoles sont confuses et brisées\FTNT{Es. 46:1.} !
\VS{3}Car une nation monte contre elle du nord, qui mettra son pays en ruines, il n'y aura plus personne qui y habite ; les hommes et les bêtes fuient, ils s'en vont.
\VS{4}En ces jours, et en ce temps-là, dit Yahweh, les fils d'Israël et les fils de Juda viendront ensemble ; ils marcheront en pleurant, et en cherchant Yahweh, leur Dieu.
\VS{5}Ils demanderont la route de Sion, ils tourneront leur visage vers elle : Venez, attachez-vous à Yahweh, par une alliance éternelle qui ne soit jamais oubliée !
\VS{6}Mon peuple était comme un troupeau de brebis perdues ; leurs bergers les égaraient, les rendaient errantes par les montagnes ; elles allaient de montagne en colline, oubliant leur bercail\FTNT{Ez. 34:5-6 ; Za. 10:2 ; Mt. 9:36.}.
\VS{7}Tous ceux qui les trouvaient les dévoraient, et leurs ennemis disaient : Nous ne sommes coupables d'aucun mal, parce qu'ils ont péché contre Yahweh, contre la demeure de la justice, contre Yahweh, l'espérance de leurs pères.
\VS{8}Fuyez du milieu de Babylone, sortez du pays des Chaldéens, et soyez comme les boucs qui vont devant le troupeau\FTNT{Es. 48:20 ; 2 Co. 6:17 ; Ap. 18:4.} !
\VS{9}Car voici, je vais susciter et faire monter contre Babylone une multitude de grandes nations du pays du nord ; elles se rangeront en bataille contre elle, de sorte qu'elle sera prise ; leurs flèches font des ravages comme celles d'un habile guerrier qui ne retourne pas à vide\FTNT{Es. 13:18.}.
\VS{10}Et la Chaldée sera abandonnée au pillage ; tous ceux qui la pilleront seront rassasiés, dit Yahweh.
\VS{11}Oui, réjouissez-vous, soyez dans l'allégresse, vous qui avez pillé mon héritage ! Oui, bondissez comme une génisse qui est dans l'herbe, hennissez comme de puissants chevaux !
\VS{12}Votre mère est fort honteuse, celle qui vous a enfantés rougit de honte ; voici, elle est la dernière entre les nations, c'est un désert, un pays sec et aride.
\VS{13}Elle ne sera plus habitée à cause de la colère de Yahweh, elle ne sera plus qu'une désolation. Quiconque passera près de Babylone sera étonné et sifflera à cause de toutes ses plaies.
\VS{14}Rangez-vous en bataille contre Babylone, mettez-vous tout alentour vous tous qui tendez l'arc ! Tirez contre elle, n'épargnez pas les flèches ! Car elle a péché contre Yahweh.
\VS{15}Poussez des cris de guerre contre elle tout alentour ! Elle tend les mains ; ses fondements tombent ; ses murs sont renversés. Car c'est ici la vengeance de Yahweh. Vengez-vous sur elle ! Faites-lui comme elle a fait\FTNT{Ab. 1:15 ; Ps. 137:8 ; Lu. 6:38.} !
\VS{16}Retranchez de Babylone le semeur, et celui qui manie la faucille au temps de la moisson ! Devant l'épée de l'oppresseur, que chacun se tourne vers son peuple, que chacun s'enfuie vers son pays.
\VS{17}Israël est comme une brebis égarée que les lions ont chassée ; le roi d'Assyrie l'a dévorée le premier ; mais ce dernier, Nebucadnetsar, roi de Babylone, lui a brisé les os.
\VS{18}C'est pourquoi ainsi parle Yahweh des armées, le Dieu d'Israël : Voici, je punirai le roi de Babylone et son pays, comme j'ai puni le roi d'Assyrie\FTNT{Es. 37:36 ; 2 R. 19:35.}.
\VS{19}Je ramènerai Israël dans sa demeure ; il paîtra au Carmel et au Basan, et son âme se rassasiera sur la montagne d'Ephraïm et de Galaad.
\VS{20}En ces jours, et en ce temps-là, dit Yahweh, on cherchera l'iniquité d'Israël, mais il n'y en aura pas, les péchés de Juda ne seront pas trouvés ; car je pardonnerai au reste que j'aurai fait demeurer.
\VS{21}Monte contre ce pays doublement rebelle, contre les habitants, et châtie-les ! Massacre, extermine-les ! dit Yahweh, fais selon toutes les choses que je t'ai ordonnées.
\VS{22}Des cris de guerre retentissent dans le pays, et la ruine est grande.
\VS{23}Eh quoi ! Il est rompu, brisé, le marteau de toute la terre ! Babylone est réduite en une désolation parmi les nations !
\VS{24}Je t'ai tendu un piège, et tu as été prise, Babylone, et tu n'en savais rien ; tu as été trouvée, et même attrapée, parce que tu as lutté contre Yahweh.
\VS{25}Yahweh a ouvert son arsenal et en a sorti les armes de sa colère ; c'est là une œuvre du Seigneur, de Yahweh des armées, dans le pays des Chaldéens.
\VS{26}Venez de toutes parts dans Babylone, ouvrez ses greniers, faites-y des monceaux, comme des tas de gerbes, et détruisez-la ! Qu'il ne reste plus rien d'elle !
\VS{27}Tuez tous ses taureaux et qu'ils descendent à l'abattage ! Malheur à eux ! Car le jour est venu, le temps de leur châtiment.
\VS{28}Ecoutez la voix de ceux qui s'enfuient, de ceux qui sont échappés du pays de Babylone pour annoncer dans Sion la vengeance de Yahweh, notre Dieu, la vengeance de son temple !
\VS{29}Appelez les archers contre Babylone, vous tous qui tendez l'arc ! Campez-vous contre elle tout alentour, que personne n'échappe, rendez-lui selon ses œuvres, faites-lui selon tout ce qu'elle a fait ! Car elle s'est fièrement élevée contre Yahweh, contre le Saint d'Israël\FTNT{Es. 13:11 ; Joë. 3:4-9 ; La. 1:22.}.
\VS{30}C'est pourquoi ses jeunes gens tomberont dans les places, et tous ses hommes de guerre périront en ce jour, dit Yahweh.
\VS{31}Voici, j'en veux à toi, orgueilleuse ! dit le Seigneur, Yahweh des armées ; car ton jour est venu, le temps de ton châtiment.
\VS{32}L'orgueilleuse chancellera et tombera, et il n'y aura personne pour la relever ; je mettrai le feu à ses villes, et il dévorera tous ses environs.
\VS{33}Ainsi parle Yahweh des armées : Les fils d'Israël et les fils de Juda sont ensemble opprimés ; tous ceux qui les ont emmenés captifs les retiennent, et refusent de les laisser aller.
\VS{34}Leur Rédempteur est fort, son nom est Yahweh des armées ; il défendra certainement leur cause, pour donner du repos au pays, et pour faire trembler les habitants de Babylone.
\VS{35}L'épée est sur les Chaldéens ! dit Yahweh, sur les habitants de Babylone, ses chefs, et ses sages !
\VS{36}L'épée est tirée contre ses devins de mensonges ! Qu'ils soient comme des insensés ! L'épée contre ses hommes forts ! Qu'ils soient épouvantés !
\VS{37}L'épée est sur ses chevaux et sur ses chars ! Contre les foules de toute espèce qui sont au milieu d'elle ! Qu'ils deviennent comme des femmes ! L'épée est sur ses trésors ! Qu'ils soient pillés !
\VS{38}La sécheresse est sur ses eaux ! Qu'elles soient mises à sec ! Parce que c'est un pays d'idoles ; ils agissent en insensés à l'égard de leurs idoles\FTNT{Es. 2:8.}.
\VS{39}C'est pourquoi les animaux du désert y habiteront avec les chacals, et les autruches y habiteront aussi ; elle ne sera plus jamais habitée, et on n'y demeurera plus jamais.
\VS{40}Comme Sodome et Gomorrhe, et les villes voisines que Dieu détruisit, dit Yahweh, elle ne sera plus habitée par des hommes, elle ne sera le séjour d'aucun fils d'homme.
\VS{41}Voici, un peuple vient du nord, une grande nation et plusieurs rois se lèvent des extrémités de la terre.
\VS{42}Ils saisissent l'arc et le javelot ; ils sont cruels, et ils n'ont pas de compassion ; leur voix mugit comme la mer ; ils sont montés sur des chevaux, chacun d'eux est rangé en bataille comme un seul homme, contre toi, fille de Babylone !
\VS{43}Le roi de Babylone entend la nouvelle, et ses mains s'affaiblissent, l'angoisse le saisit comme la douleur de celle qui enfante…
\VS{44}Voici, il monte comme un lion des rives orgueilleuses du Jourdain vers la demeure forte ; soudain, je les ferai courir, et je désignerai sur elle celui que j'ai choisi. Car qui est semblable à moi ? Qui me donnera des ordres ? Et quel est le chef qui me résistera en face ?
\VS{45}C'est pourquoi écoutez le conseil que Yahweh a donné contre Babylone, et les desseins qu'il a projetés contre le pays des Chaldéens ! Certainement, on les traînera comme les plus petits du troupeau, certainement on dévastera leur demeure.
\VS{46}La terre tremble au bruit de la prise de Babylone, et le cri se fait entendre parmi les nations.
\Chap{51}
\TextTitle{Le jugement de Babylone par Yahweh}
\VerseOne{}Ainsi parle Yahweh : Voici, je fais lever un vent destructeur contre Babylone et contre ceux qui habitent au cœur du royaume.
\VS{2}J'envoie contre Babylone des vanneurs qui la vanneront, qui videront son pays ; car de tous côtés ils seront contre elle, au jour du malheur.
\VS{3}Qu'on bande l'arc contre celui qui bande son arc, contre celui qui s'élève dans son armure ! N'épargnez pas ses jeunes hommes ! Exterminez toute son armée !
\VS{4}Qu'ils tombent les blessés à mort dans le pays des Chaldéens, percés de coups dans les rues de Babylone !
\VS{5}Car Israël et Juda ne sont pas abandonnés de leur Dieu, de Yahweh des armées, quoique leur pays ai été trouvé par le Saint d'Israël plein de crimes.
\VS{6}Fuyez hors de Babylone, et que chacun sauve sa vie, ne soyez point exterminés dans son iniquité ; car c'est le temps de la vengeance de Yahweh ; il lui rend ce qu'elle a mérité.
\VS{7}Babylone était comme une coupe d'or dans la main de Yahweh, enivrant toute la terre ; les nations ont bu de son vin : C'est pourquoi les nations ont agi comme des insensées.
\VS{8}Babylone est tombée\FTNT{Ap. 18.} en un instant, elle est brisée ! Gémissez sur elle, prenez du baume pour sa douleur : Peut-être qu'elle guérira.
\VS{9}Nous avons pansé Babylone, mais elle n'a pas guéri. Laissons-la et allons-nous-en chacun dans son pays ; car son jugement atteint les cieux et s'élève jusqu'aux nues.
\VS{10}Yahweh a rendu la justice de notre cause ; venez et racontons dans Sion l'œuvre de Yahweh, notre Dieu.
\VS{11}Aiguisez les flèches, remplissez vos mains avec les boucliers ! Yahweh a réveillé l'esprit des rois de Médie, car sa pensée est de détruire Babylone ; c'est ici la vengeance de Yahweh, la vengeance de son temple.
\VS{12}Elevez une bannière contre les murs de Babylone ! Fortifiez les postes, levez des gardes, préparez des embuscades ! Car Yahweh a formé un projet, il fait ce qu'il a dit contre les habitants de Babylone.
\VS{13}Toi qui habites près des grandes eaux, abondantes en trésors, ta fin est venue, ta cupidité est à son terme !
\VS{14}Yahweh des armées a juré par lui-même, en disant : Je te remplirai d'hommes comme de sauterelles, et ils pousseront contre toi des cris de guerre.
\VS{15}C'est lui qui a fait la terre par sa puissance, qui a fondé le monde habitable par sa sagesse, et qui a étendu les cieux par son intelligence\FTNT{Ge. 1:1 ; Es. 40:22 ; Ps. 104:2; Job. 9:8.}.
\VS{16}Lorsqu'il donne de la voix, il y a un tumulte d'eaux dans les cieux, il fait monter les vapeurs des extrémités de la terre, il fait les éclairs et la pluie, il fait sortir le vent de ses réservoirs.
\VS{17}Tout homme devient stupide par sa connaissance, tout fondeur est honteux par les images taillées ; car ses idoles en métal fondu ne sont que mensonge, il n'y a pas de souffle en elles.
\VS{18}Elles ne sont que vanité, une œuvre de tromperie ; elles périront au temps de leur châtiment.
\VS{19}La portion de Jacob n'est pas comme ces choses-là ; car c'est lui qui a tout formé, et Israël est la tribu de son héritage. Son nom est Yahweh des armées.
\VS{20}Tu as été pour moi un marteau, un instrument de guerre. Par toi j'ai brisé des nations, par toi j'ai détruit des royaumes.
\VS{21}Par toi j'ai brisé le cheval et son cavalier ; par toi j'ai brisé le char et celui qui était monté dessus.
\VS{22}Par toi j'ai brisé l'homme et la femme ; par toi j'ai brisé le vieillard et le jeune garçon ; par toi j'ai brisé le jeune homme et la jeune fille.
\VS{23}Par toi j'ai brisé le berger et son troupeau ; par toi j'ai brisé le laboureur et ses bœufs ; par toi j'ai brisé les gouverneurs et les chefs.
\VS{24}Mais je rendrai à Babylone et à tous les habitants de la Chaldée tout le mal qu'ils ont fait à Sion sous vos yeux, dit Yahweh\FTNT{La. 1:21.}.
\VS{25}Voici, j'en veux à toi, montagne de destruction, dit Yahweh, à toi qui détruisais toute la terre ! J'étendrai ma main sur toi, je te roulerai du haut des rochers, je ferai de toi une montagne embrasée.
\VS{26}On ne pourra prendre de toi aucune pierre pour la placer à l'angle de l'édifice ni aucune pierre pour servir de fondement ; car tu seras une ruine éternelle, dit Yahweh\FTNT{Es. 13:19-20.}…
\VS{27}Elevez une bannière dans le pays ! Sonnez du shofar parmi les nations ! Préparez les nations contre elle, appelez contre elle les royaumes d'Ararat, de Minni et d'Aschkenaz ! Établissez contre elle des chefs ! Faites monter ses chevaux comme des sauterelles hérissées !
\VS{28}Préparez contre elle les nations, les rois de Médie, ses gouverneurs et tous ses chefs, et tout le pays sous leur domination\FTNT{Es. 13:17.} !
\VS{29}La terre tremble, elle se tord ; car la pensée de Yahweh se dresse contre Babylone ; il va faire du pays de Babylone un désert sans habitants\FTNT{Es. 13:14 ; Joë. 3:16.}.
\VS{30}Les hommes forts de Babylone cessent de combattre, ils demeurent dans les forteresses ; leur force est épuisée, ils sont comme des femmes. On met le feu aux demeures, on brise les barres.
\VS{31}Le courrier rencontre le courrier, et le messager rencontre le messager, pour annoncer au roi de Babylone que sa ville est prise par tous les côtés,
\VS{32}que les gués sont saisis, les marais brûlés par le feu, et les hommes de guerre épouvantés.
\VS{33}Car ainsi parle Yahweh des armées, le Dieu d'Israël : La fille de Babylone est comme une aire dans le temps où on la foule ; encore un peu de temps, et le moment de la moisson sera venu pour elle.
\VS{34}Nebucadnetsar, roi de Babylone, m'a dévorée, m'a détruite ; il a fait de moi un vase vide ; il m'a engloutie tel un dragon, il a rempli son ventre de mes délices ; il m'a chassée au loin.
\VS{35}Que la violence envers moi et ma chair déchirée retombe sur Babylone ! dit l'habitante de Sion. Que mon sang retombe sur les habitants de la Chaldée ! dit Jérusalem.
\VS{36}C'est pourquoi ainsi parle Yahweh : Voici, je défendrai ta cause, je te vengerai ! Je dessécherai la mer de Babylone, et je ferai tarir sa source.
\VS{37}Babylone sera un monceau de ruines, un repaire de serpents, un objet d'épouvante et de moquerie ; sans que personne n'y habite.
\VS{38}Ils rugiront ensemble comme des lions, ils pousseront des cris comme des lionceaux.
\VS{39}Quand ils seront échauffés, je les ferai boire, et les enivrerai, afin qu'ils se réjouissent et qu'ils dorment d'un sommeil éternel et qu'ils ne se réveillent plus, dit Yahweh.
\VS{40}Je les ferai descendre comme des agneaux à la boucherie, comme des béliers et des boucs.
\VS{41}Eh quoi ! Schéschac est prise ! Celle dont la louange remplissait toute la terre est conquise ! Eh quoi ! Babylone est réduite en désolation parmi les nations !
\VS{42}La mer est montée sur Babylone, elle a été couverte de la multitude de ses flots\FTNT{Es. 8:8 ; Ez. 26:3-19 ; Lu. 21:25.}.
\VS{43}Ses villes sont en ruines, une terre sèche et déserte ; c'est un pays où personne ne demeure, et où il ne passe aucun fils d'homme.
\VS{44}Je punirai aussi Bel à Babylone, je sortirai de sa bouche ce qu'il a englouti, et les nations n'aborderont plus vers lui. Le mur même de Babylone est tombé !
\VS{45}Mon peuple, sortez du milieu d'elle, et que chacun sauve sa vie de l'ardeur de la colère de Yahweh.
\VS{46}Que votre cœur ne se trouble pas, et ne craignez pas les nouvelles qu'on entendra dans tout le pays ; car cette année viendra une nouvelle, et l'année d'après une autre nouvelle, et il y aura violence dans le pays, et un dominateur s'élèvera contre un autre dominateur.
\VS{47}C'est pourquoi voici, les jours viennent où je punirai les idoles de Babylone, et tout son pays sera honteux ; tous ses morts tomberont au milieu d'elle.
\VS{48}Les cieux, la terre, et tout ce qui y est, pousseront des cris de joie contre Babylone ; car du nord les dévastateurs viendront contre elle, dit Yahweh.
\VS{49}Babylone tombera, ô morts d'Israël, comme Babylone a fait tomber les morts de tout le pays.
\VS{50}Vous qui avez échappé à l'épée, allez, ne vous arrêtez pas ! Souvenez-vous de Yahweh dans ces pays éloignés, et que Jérusalem revienne à vos cœurs !
\VS{51}Nous étions honteux des reproches que nous entendions ; la honte couvrait nos visages, quand les étrangers sont venus dans le sanctuaire de la maison de Yahweh.
\VS{52}C'est pourquoi, voici, les jours viennent, dit Yahweh, où je châtierai ses idoles ; et les blessés gémiront dans tout son pays.
\VS{53}Quand Babylone monterait jusqu'aux cieux et qu'elle rendrait inaccessible le plus haut de sa forteresse, alors les dévastateurs viendront contre elle, dit Yahweh\FTNT{Am. 9:2 ; Ab. 1:4.}…
\VS{54}Un grand cri s'entend de Babylone, et la ruine est grande dans le pays des Chaldéens.
\VS{55}Parce que Yahweh dévaste Babylone, il en fait échapper de grands cris ; les flots des dévastateurs mugissent comme de grandes eaux, le bruit du mugissement s'étend.
\VS{56}Car le destructeur est venu contre elle, contre Babylone ; ses hommes forts sont pris, leurs arcs sont brisés. Car Yahweh est un Dieu qui rend à chacun selon ses œuvres, qui paie à chacun son salaire.
\VS{57}J'enivrerai ses princes et ses sages, ses gouverneurs, ses chefs, et ses hommes forts ; ils dormiront d'un sommeil éternel, et ils ne se réveilleront plus, dit le Roi dont le nom est Yahweh des armées.
\VS{58}Ainsi parle Yahweh des armées : Les larges murs de Babylone seront renversés, ses portes qui sont si hautes, seront brûlées par le feu ; ainsi les peuples auront travaillé en vain, les nations se seront fatiguées pour le feu.
\VS{59}C'est ici l'ordre que Jérémie, le prophète, donna à Seraja, fils de Nérija, fils de Machséja, lorsqu'il alla avec Sédécias, roi de Juda, la quatrième année du règne de Sédécias. Or Seraja était premier chambellan.
\VS{60}Jérémie écrivit dans un livre tous les malheurs qui devaient venir sur Babylone, toutes ces paroles qui sont écrites contre Babylone.
\VS{61}Jérémie dit à Seraja : Lorsque tu seras venu à Babylone, et que tu auras vu, tu liras toutes ces paroles,
\VS{62}et tu diras : Yahweh, c'est toi qui as déclaré que ce lieu serait exterminé, en sorte qu'il n'y ait aucun habitant, depuis l'homme jusqu'à la bête, mais qu'il deviendrait un désert pour toujours.
\VS{63}Et quand tu auras achevé de lire ce livre, tu le lieras à une pierre et tu le jetteras dans l'Euphrate,
\VS{64}et tu diras : Ainsi Babylone sera submergée, elle ne se lèvera pas des malheurs que je ferai venir sur elle ; ils seront épuisés. Jusqu'ici sont les paroles de Jérémie.
\Chap{52}
\TextTitle{Chute de Jérusalem et destruction du temple ; Juda déporté à Babylone\FTNTT{2 R. 25:1-26; Jé. 39:1-10.}}
\VerseOne{}Sédécias avait vingt et un ans lorsqu'il devint roi, et il régna onze ans à Jérusalem. Sa mère se nommait Hamuthal, fille de Jérémie, de Libna\FTNT{2 R. 24 et 25.}.
\VS{2}Il fit ce qui est mal aux yeux de Yahweh, comme avait fait Jojakim.
\VS{3}Et cela arriva à cause de la colère de Yahweh contre Jérusalem et Juda, qu'il voulait chasser de devant sa face. Sédécias se rebella contre le roi de Babylone.
\VS{4}La neuvième année de son règne, le dixième jour du dixième mois, Nebucadnetsar, roi de Babylone, vint contre Jérusalem, lui et toute son armée ; ils campèrent devant elle et construisirent des retranchements tout alentour.
\VS{5}La ville fut assiégée jusqu'à la onzième année du roi Sédécias.
\VS{6}Le neuvième jour du quatrième mois, la famine était forte dans la ville, et il n'y avait pas de pain pour le peuple du pays\FTNT{La. 2:11-12.}.
\VS{7}Alors la brèche fut faite à la ville ; et tous les gens de guerre s'enfuirent et sortirent de nuit hors de la ville par le chemin de la porte entre les deux murailles près du jardin du roi, tandis que les Chaldéens entouraient la ville. Ils s'en allèrent par le chemin de la plaine.
\VS{8}Mais l'armée des Chaldéens poursuivit le roi, et ils atteignirent Sédécias dans les plaines de Jéricho ; et toute son armée se dispersa loin de lui.
\VS{9}Ils prirent le roi, et le firent monter vers le roi de Babylone à Ribla, dans le pays de Hamath ; et il prononça contre lui une sentence.
\VS{10}Le roi de Babylone fit égorger les fils de Sédécias sous ses yeux ; il fit aussi égorger tous les chefs de Juda à Ribla.
\VS{11}Puis il fit crever les yeux à Sédécias, et le fit lier avec des chaînes d'airain ; le roi de Babylone l'emmena à Babylone, et le mit en prison jusqu'au jour de sa mort.
\VS{12}Le dixième jour du cinquième mois, c'était la dix-neuvième année du règne de Nebucadnetsar, roi de Babylone, Nebuzaradan, chef des gardes, qui se tenait devant le roi de Babylone, entra dans Jérusalem.
\VS{13}Il brûla la maison de Yahweh, la maison du roi, et toutes les maisons de Jérusalem ; il brûla toutes les maisons des personnes considérées.
\VS{14}Toute l'armée des Chaldéens, qui était avec le chef des gardes, renversa toutes les murailles qui entouraient Jérusalem.
\VS{15}Nebuzaradan, chef des gardes, transporta à Babylone une partie des plus pauvres du peuple, le reste du peuple qui était demeuré dans la ville, ceux qui s'étaient rendus au roi de Babylone, et le reste de la multitude.
\VS{16}Toutefois, Nebuzaradan, chef des gardes, laissa quelques-uns des plus pauvres du pays pour être vignerons et laboureur.
\VS{17}Les Chaldéens brisèrent les colonnes d'airain qui étaient dans la maison de Yahweh, les bases, la mer d'airain qui était dans la maison de Yahweh, et ils emmenèrent tout l'airain à Babylone.
\VS{18}Ils prirent les cendriers, les pelles, les couteaux, les coupes, les tasses, et tous les ustensiles d'airain avec lesquels on faisait le service.
\VS{19}Le chef des gardes prit aussi les coupes, les encensoirs, les cendriers, les chandeliers, les tasses et les calices, ce qui était d'or et ce qui était d'argent.
\VS{20}Les deux colonnes, la mer, et les douze bœufs d'airain qui servaient de base, et que le roi Salomon avait faits pour la maison de Yahweh, tous ces ustensiles d'airain ne pouvaient être pesés.
\VS{21}La hauteur de l'une des colonnes était de dix-huit coudées, et un cordon de douze coudées l'entourait ; elle était épaisse de quatre doigts, et creuse;
\VS{22}il y avait par-dessus un chapiteau d'airain, et la hauteur d'un des chapiteaux était de cinq coudées ; il y avait aussi un treillis et des grenades tout autour du chapiteau, le tout d'airain ; il en était de même pour la seconde colonne avec des grenades\FTNT{1 R. 7:15-20.}.
\VS{23}Il y avait quatre-vingt-seize grenades de chaque côté, et les grenades qui étaient autour du treillis étaient au nombre de cent.
\VS{24}Le chef des gardes prit Seraja, qui était le premier sacrificateur, Sophonie, qui était le second sacrificateur, et les trois gardiens du seuil.
\VS{25}Il prit de la ville un eunuque qui avait sous son commandement des hommes de guerre, sept hommes de ceux qui voyaient la face du roi et qui furent trouvés dans la ville, le secrétaire du chef de l'armée qui enrôlait le peuple du pays, et soixante hommes du peuple du pays, qui se trouvèrent dans la ville.
\VS{26}Nebuzaradan, chef des gardes, les prit et les emmena vers le roi de Babylone à Ribla.
\VS{27}Le roi de Babylone les frappa et les fit mourir à Ribla, dans le pays de Hamath. Ainsi Juda fut transporté hors de son pays.
\VS{28}Et c'est ici le peuple que Nebucadnetsar emmena en captivité : La septième année, trois mille vingt-trois juifs ;
\VS{29}la dix-huitième année de Nebucadnetsar, il emmena de Jérusalem huit cent trente-deux personnes ;
\VS{30}La vingt-troisième année de Nebucadnetsar, Nebuzaradan, chef des gardes, transporta en exil sept cent quarante-cinq personnes des Juifs ; en tout quatre mille six cents personnes.
\VS{31}La trente-septième année de la captivité de Jojakin, roi de Juda, le vingt-cinquième jour du douzième mois, Evil-Merodac, roi de Babylone, dans la première année de son règne, leva la tête de Jojakin, roi de Juda, et le fit sortir de prison.
\VS{32}Il lui parla avec bonté, et mit son trône au-dessus du trône des autres rois qui étaient avec lui à Babylone.
\VS{33}Il fit changer ses vêtements de prison, il mangea du pain tous les jours de sa vie en présence du roi.
\VS{34}Le roi de Babylone lui donna continuellement des vivres pour chaque jour, jusqu'au jour de sa mort, tout le temps de sa vie.
\PPE{}
\end{multicols}

%\clearpage\ShortTitle{Ezéchiel}\BookTitle{Ezéchiel}\BFont
\noindent\hrulefill
{\footnotesize
\textit{
\bigskip
{\centering{}
\\Auteur : Ezéchiel
\\(Heb. : Yechezqe'l)
\\Signification : Dieu fortifie
\\Thème : Jugements et gloire
\\Date de rédaction : 6\up{ème} siècle av. J.-C.\\}
}
%\bigskip
\textit{
\\Déporté à Babylone alors qu'il remplissait la fonction de sacrificateur, Ezéchiel eut la particularité d'exercer le ministère prophétique hors de la terre d'Israël. Sa mission était en partie d'affermir la foi des déportés par la promesse du jugement de leurs ennemis et du rétablissement de la nation. Il leur rappela aussi que les péchés de leurs pères étaient la raison de leur captivité et que l'occasion leur était donnée de réformer leurs voies.
%\bigskip
\\Ezéchiel reçut aussi des oracles concernant ceux qui étaient restés à Jérusalem et dont la condition n'était guère meilleure que celle des exilés et pour lesquels le pire était à venir. Ses prophéties s'exprimaient en songes et en visions, c'est donc ainsi qu'il vit la gloire de Yahweh quittant le temple de Jérusalem. Il reçut plusieurs prophéties sur les derniers temps, notamment la promesse d'un cœur nouveau en vue de la conversion, le retour de la gloire de Dieu lors du règne millénaire et aussi le rétablissement total d'Israël.\bigskip
}
}
\par\nobreak\noindent\hrulefill
\begin{multicols}{2}
\Chap{1}
\TextTitle{Vision de la gloire de Yahweh}
\VerseOne{}Or il arriva en la trentième année, au cinquième jour du quatrième mois, comme j'étais parmi ceux qui avaient été transportés sur le fleuve de Kebar, que les cieux furent ouverts, et je vis des visions de Dieu.
\VS{2}Au cinquième jour du mois de cette année, qui fut la cinquième après que le roi Jojakin\FTNT{2 R. 24:12-16.}ait été mené en captivité,
\VS{3}la parole de Yahweh fut adressée expressément à Ezéchiel, le sacrificateur, fils de Buzi, dans le pays des Chaldéens\FTNT{Ezéchiel était en exil à Babylone.}, sur le fleuve de Kebar, et la main de Yahweh fut là sur lui.
\VS{4}Je regardai donc, et voici, un vent impétueux vint du nord, une grosse nuée, et un feu qui prenait de tous côtés. Il y avait autour de la nuée une splendeur, et au milieu de la nuée paraissait comme de l'airain poli, lorsqu'il sort du milieu du feu.
\VS{5}Et du milieu aussi paraissait une ressemblance de quatre animaux\FTNT{Les quatre animaux représentent quatre aspects de Jésus. La face de l'homme correspond à l'humanité du Seigneur mise en exergue dans l'évangile de Luc. La face de lion symbolise la royauté de Christ, mise en évidence dans l'évangile de Matthieu. La face de bœuf fait écho à l'évangile de Marc où le Seigneur y est présenté comme serviteur. La face d'aigle symbolise la divinité du Messie mise en évidence dans l'évangile de Jean. Le Seigneur y est présenté comme le Fils de Dieu et le Dieu véritable.} et voici leur forme: Ils avaient une ressemblance humaine.
\VS{6}Et chacun d'eux avait quatre faces, et chacun avait quatre ailes.
\VS{7}Et leurs pieds étaient des pieds droits, et la plante de leurs pieds était comme la plante d'un pied de veau, ils étincelaient comme la couleur d'un airain poli.
\VS{8}Et il y avait des mains d'homme sous leurs ailes à leurs quatre côtés ; et tous quatre avaient leurs faces et leurs ailes.
\VS{9}Leurs ailes étaient jointes l'une à l'autre ; ils ne se tournaient point quand ils marchaient, mais chacun marchait droit devant soi.
\VS{10}Leurs faces ressemblaient à la face d'un homme, à la face de lion à la main droite, à la face de bœuf à la gauche des quatre, et à la face d'aigle à tous les quatre.
\VS{11}Leurs faces et leurs ailes étaient divisées par le haut ; chacun avait des ailes qui se joignaient l'une à l'autre, et deux couvraient leurs corps.
\VS{12}Chacun marchait droit devant soi ; ils allaient partout où l'Esprit les poussait à aller, et ils ne se tournaient point lorsqu'ils marchaient.
\VS{13}Et quant à l'aspect des animaux, leur regard était comme des charbons de feu ardent, comme des torches ; le feu courait parmi les animaux ; et le feu avait une splendeur, et de ce feu sortait un éclair.
\VS{14}Et les animaux couraient et revenaient selon que l'éclair paraissait.
\VS{15}Je regardais les animaux, et voici, une roue apparut sur la terre auprès des animaux, devant leurs quatre faces.
\VS{16}Et l'aspect et la forme des roues étaient comme la couleur d'un chrysolithe, et toutes les quatre avaient un même aspect ; leur aspect et leur structure étaient comme si chaque roue avait été au milieu d'une autre roue.
\VS{17}En marchant, elles allaient de leurs quatre côtés, et elles ne se tournaient point quand elles marchaient.
\VS{18}Elles avaient des jantes, elles étaient si hautes qu'elles faisaient peur, et leurs jantes étaient pleines d'yeux tout autour des quatre roues.
\VS{19}Quand ils marchaient, elles marchaient auprès d'eux ; et quand ils s'élevaient au-dessus de la terre, elles aussi s'élevaient.
\VS{20}Ils allaient partout où l'Esprit les poussait à aller ; l'Esprit tendait-il là, ils y allaient, et les roues s'élevaient avec eux, car l'esprit des animaux était dans les roues.
\VS{21}Quand ils marchaient, elles marchaient ; et quand ils s'arrêtaient, elles s'arrêtaient ; et quand ils s'élevaient au-dessus de la terre, les roues aussi s'élevaient avec eux, car l'esprit des animaux était dans les roues.
\VS{22}L'aspect de ce qui était au-dessus des têtes des animaux, était une étendue semblable à un cristal étincelant et terrible à voir, laquelle s'étendait au-dessus de leurs têtes.
\VS{23}Sous l'étendue, leurs ailes se tenaient droites l'une contre l'autre ; ils avaient chacun deux ailes dont ils se couvraient, chacun, dis-je, en avait deux qui couvraient leurs corps.
\VS{24}Puis j'entendis le bruit que faisaient leurs ailes quand ils marchaient, semblable au bruit des grandes eaux, et au bruit du Tout-Puissant, un bruit éclatant comme le bruit d'une armée ; quand ils s'arrêtaient, ils baissaient leurs ailes.
\VS{25}Et lorsqu'ils s'arrêtaient et laissaient tomber leurs ailes, il se faisait un bruit au-dessus de l'étendue qui était sur leurs têtes.
\VS{26}Et au-dessus de cette étendue, qui était sur leurs têtes, il y avait quelque chose de semblable à une pierre de saphir, en forme de trône ; et sur cette forme de trône apparaissait comme une figure d'homme\FTNT{Il est question ici de la manifestation du Messie} placée dessus en hauteur.
\VS{27}Je vis encore comme de l'airain poli semblable à un feu, au dedans duquel était cet homme, et qui l'environnait ; depuis la forme de ses reins jusqu'en haut et depuis la forme de ses reins jusqu'en bas, je vis comme du feu, et il y avait une lumière éclatante autour de lui.
\VS{28}Et la splendeur qui se voyait autour de lui, était comme l'arc qui se fait dans la nuée en un jour de pluie. C'est là la vision de la représentation de la gloire de Yahweh. A sa vue je tombai sur ma face, et j'entendis une voix qui parlait.
\Chap{2}
\TextTitle{Mandat d'Ezechiel}
\VerseOne{}Il me dit : Fils de l'homme, tiens-toi sur tes pieds, et je parlerai avec toi.
\VS{2}Alors l'Esprit entra en moi, après qu'il m'eut parlé, et il me releva sur mes pieds, et j'entendis celui qui me parlait.
\VS{3}Il me dit : Fils de l'homme, je t'envoie vers les fils d'Israël, vers des nations rebelles qui se sont rebellées contre moi. Eux et leurs pères ont péché contre moi jusqu'à ce jour même\FTNT{Jé. 3:25.}.
\VS{4}Ce sont des enfants à la face dure et au cœur obstiné, vers lesquels je t'envoie vers eux ; c'est pourquoi tu leur diras que le Seigneur Yahweh a ainsi parlé.
\VS{5}Et soit qu'ils écoutent, ou qu'ils n'en fassent rien, car ils sont une maison rebelle ; ils sauront pourtant qu'il y aura eu un prophète parmi eux\FTNT{Es. 6:9-10.}.
\VS{6}Mais toi, fils de l'homme, ne les crains point, et ne crains point leurs paroles ; quoique des gens rebelles et dont les langues sont perçantes comme des épines soient avec toi, et que tu habites parmi des scorpions ; ne crains point leurs paroles, et ne t'effraie point à cause d'eux, quoiqu'ils soient une maison rebelle\FTNT{Jé. 1:8 ; 1 Pi. 3:14.}.
\VS{7}Tu leur prononceras mes paroles, qu'ils écoutent ou qu'ils n'en fassent rien, car ils ne sont que rébellion.
\VS{8}Mais toi, fils de l'homme, écoute ce que je te dis, et ne sois point rebelle comme cette maison rebelle ; ouvre ta bouche et mange ce que je vais te donner\FTNT{Ap. 10:9 ; Jé. 15:16.}.
\VS{9}Alors je regardai, et voici, une main fut envoyée vers moi, et voici, elle avait un livre en rouleau.
\VS{10}Et elle l'ouvrit devant moi, et voici, il était écrit dedans et dehors ; des lamentations, des soupirs, et des gémissements y étaient écrits.
\Chap{3}
\TextTitle{Yahweh établit Ezéchiel comme sentinelle}
\VerseOne{}Puis il me dit : Fils de l'homme, mange ce que tu trouveras, mange ce rouleau, et va, parle à la maison d'Israël !
\VS{2}J'ouvris donc ma bouche, et il me fit manger ce rouleau.
\VS{3}Il me dit : Fils de l'homme, nourris ton ventre et remplis tes entrailles de ce rouleau que je te donne ! Je le mangeai, et il fut doux dans ma bouche comme du miel\FTNT{Ps. 119:103.}.
\VS{4}Puis il me dit : Fils de l'homme, lève-toi et va vers la maison d'Israël, et prononce-leur mes paroles !
\VS{5}Car tu n'es point envoyé vers un peuple au langage inconnu, ou à la langue barbare ; c'est vers la maison d'Israël ;
\VS{6}ni vers plusieurs peuples ayant un langage inconnu ou une langue barbare, dont tu ne puisses comprendre les paroles. Si je t'envoyais vers eux, ils t'écouteraient.
\VS{7}Mais la maison d'Israël ne voudra pas t'écouter, parce qu'ils ne veulent point m'écouter ; car toute la maison d'Israël a le front dur et le cœur obstiné.
\VS{8}Voici, j'endurcirai ta face contre leurs faces, et j'endurcirai ton front contre leurs fronts\FTNT{Jé. 1:18 ; Mi. 3:8.}.
\VS{9}Et j'ai rendu ton front semblable à un diamant, plus dur que le roc. Ne les crains donc point, et ne t'effraie point à cause d'eux, quoiqu'ils soient une maison rebelle.
\VS{10}Puis il me dit : Fils de l'homme, reçois dans ton cœur et écoute de tes oreilles toutes les paroles que je te dirai.
\VS{11}Lève-toi donc, va vers ceux qui ont été emmenés captifs, vers les enfants de ton peuple, parle-leur et dis-leur que le Seigneur Yahweh a ainsi parlé ; soit qu'ils écoutent ou qu'ils n'en fassent rien.
\VS{12}Puis l'Esprit m'enleva, et j'entendis derrière moi le bruit d'un grand tremblement, disant : Bénie soit la gloire de Yahweh du lieu de sa demeure !
\VS{13}Et j'entendis le bruit des ailes des animaux, qui s'entre-touchaient les unes les autres, et le bruit des roues auprès d'eux, et le bruit d'un grand tremblement.
\VS{14}L'Esprit donc m'enleva, et me prit, et j'allai, l'esprit rempli d'amertume et de colère, mais la main de Yahweh me fortifia.
\VS{15}Je vins donc vers ceux qui avaient été transportés à Thel-Abib, vers ceux qui demeuraient auprès du fleuve de Kebar ; et je me tins là où ils se tenaient, même je me tins là parmi eux sept jours, tout étonné.
\VS{16}Et au bout de sept jours, la parole de Yahweh me fut adressée, en disant :
\VS{17}Fils de l'homme, je t'établis pour être sentinelle sur la maison d'Israël ; tu écouteras donc la parole de ma bouche, et tu les avertiras de ma part\FTNT{Es. 52:8 ; Es. 62:6 ; Jé. 6:17.}.
\VS{18}Quand je dirai au méchant : Tu mourras, tu mourras ! Si tu ne l'avertis pas, et si tu ne parles pas pour l'avertir de se détourner de ses mauvaises voies, afin de lui sauver la vie ; ce méchant-là mourra dans son iniquité, mais je redemanderai son sang de ta main.
\VS{19}Et si tu avertis le méchant, et qu'il ne se détourne pas de sa méchanceté ni de ses mauvaises voies, il mourra dans son iniquité, mais toi, tu sauveras ton âme\FTNT{Ez. 18:23-24 ; Ez. 33:6.}.
\VS{20}Pareillement, si le juste se détourne de sa justice et commet l'iniquité, lorsque j'aurai mis quelque obstacle devant lui, il mourra ; parce que tu ne l'auras point averti, il mourra dans son péché, et il ne sera point fait mention de ses justices qu'il aura faites ; mais je te redemanderai son sang de ta main.
\VS{21}Et si tu avertis le juste de ne point pécher, et qu'il ne pèche point, il vivra, il vivra parce qu'il aura été averti, et toi pareillement tu sauveras ton âme.
\VS{22}Et la main de Yahweh fut sur moi, et il me dit : Lève-toi, et sors vers la vallée, et là je te parlerai.
\VS{23}Je me levai donc, et sortis dans la vallée ; voici, la gloire de Yahweh se tenait là, telle que je l'avais vue près du fleuve de Kebar, et je tombai sur ma face.
\VS{24}Alors l'Esprit entra en moi et me releva sur mes pieds ; il me parla et me dit : Entre, et enferme-toi dans ta maison.
\VS{25}Fils de l'homme, voici, on mettra des cordes sur toi, on te liera, et tu ne sortiras point pour aller parmi eux.
\VS{26} Et j'attacherai ta langue à ton palais, tu seras muet, et tu ne les reprendras point ; parce qu'ils sont une maison  rebelle\FTNT{Jn. 1:20-22.}.
\VS{27}Mais quand je te parlerai, j'ouvrirai ta bouche, et tu leur diras : Ainsi parle le Seigneur Yahweh : Que celui qui écoute, écoute ; et que celui qui n'écoute pas, n'écoute pas ; car ils sont une maison rebelle.
\Chap{4}
\TextTitle{Signes annonciateurs du jugement de Jérusalem : 
\\La brique, la plaque de fer et les cordes}
\VerseOne{}Toi, fils de l'homme, prends une brique et place-la devant toi, et traces-y la ville de Jérusalem.
\VS{2}Puis tu mettras le siège contre elle, tu bâtiras contre elle des retranchements, tu élèveras contre elle des terrasses, tu mettras des camps contre elle, et tu mettras autour d'elle des béliers pour la battre\FTNT{2 R. 25:1.}.
\VS{3}Tu prendras aussi une plaque de fer, et tu la mettras comme un mur de fer entre toi et la ville ; tu dresseras ta face contre elle, elle sera assiégée, et tu l'assiégeras ; ce sera un signe pour la maison d'Israël.
\VS{4}Après, tu dormiras sur ton côté gauche, mets-y l'iniquité de la maison d'Israël, et tu porteras leur iniquité autant de jours que tu seras couché sur ce côté.
\VS{5}Et je t'ai assigné un nombre de jours égal à celui des années de leur iniquité : Trois cent quatre-vingt-dix jours ; ainsi tu porteras l'iniquité de la maison d'Israël.
\VS{6}Et quand tu auras accompli ces jours-là, tu dormiras une seconde fois sur ton côté droit, et tu porteras l'iniquité de la maison de Juda pendant quarante jours ; un jour pour chaque année, car je t'ai assigné un jour pour chaque année.
\VS{7}Tu tourneras ta face et ton bras nu vers Jérusalem assiégée, et tu prophétiseras contre elle.
\VS{8}Et voici, j'ai mis sur toi des cordes, afin que tu ne puisses pas te tourner d'un côté sur l'autre, jusqu'à ce que tu aies accompli les jours de ton siège.
\TextTitle{Le pain impur}
\VS{9}Tu prendras aussi du froment, de l'orge, des fèves, des lentilles, du millet, et de l'épeautre ; tu les mettras dans un vase, et tu en feras du pain autant de jours que tu seras couché sur ton côté ; tu en mangeras pendant trois cent quatre-vingt-dix jours.
\VS{10}La viande que tu mangeras sera du poids de vingt sicles par jour ; et tu la mangeras de temps à autre.
\VS{11}Et tu boiras de l'eau par mesure; savoir la sixième de hin ; tu la boiras de temps à autre.
\VS{12}Et tu mangeras aussi des gâteaux d'orge, que tu feras cuire avec des excréments humains en leur présence.
\VS{13}Puis Yahweh dit : Les fils d'Israël mangeront ainsi leur pain souillé parmi les nations vers lesquelles je les chasserai\FTNT{Os. 9:3 ; Da. 1:8.}.
\VS{14}Et je dis : Ah ! Seigneur Yahweh, voici, mon âme n'a point été souillée, et je n'ai mangé d'aucune bête morte d'elle-même, ou déchirée par les bêtes sauvages, depuis ma jeunesse jusqu'à présent, et aucune chair impure n'est entrée dans ma bouche\FTNT{Lé 17:15 ; De. 14:3 ; Ac. 10:14.}.
\VS{15}Il me répondit : Voici, je te donne des excréments de bœuf au lieu d'excréments humains, et tu y feras cuire ton pain.
\VS{16}Puis il me dit : Fils de l'homme, voici, je m'en vais rompre le bâton du pain dans Jérusalem ; et ils mangeront leur pain au poids et avec chagrin ; et ils boiront de l'eau par mesure et avec horreur\FTNT{Lé. 26:26 ; Es. 3:1 ; Ps. 105:16 ; La. 5:4.}.
\VS{17}Parce que le pain et l'eau leur manqueront, ils seront épouvantés, se regardant les uns les autres, et ils se décomposeront à cause de leur iniquité.
\Chap{5}
\TextTitle{Les cheveux coupés et divisés en trois}
\VerseOne{}Et toi, fils de l'homme, prends un couteau tranchant, prends un rasoir de barbier, et fais-le passer sur ta tête et sur ta barbe. Puis, tu prendras une balance à peser, et tu partageras ce que tu auras rasé\FTNT{Lé. 21:5 ; Ez. 44:20.}.
\VS{2}Brûles-en un tiers dans le feu, au milieu de la ville, lorsque les jours du siège seront accomplis ; prends-en un tiers, et frappe-le avec l'épée tout autour de la ville ; disperses-en un tiers au vent, car je tirerai l'épée derrière eux\FTNT{Lé. 26:25 ; La. 1:20.}.
\VS{3}Tu en prendras une petite quantité que tu serreras aux pans de ton manteau.
\VS{4}De ceux-là, tu en prendras encore, les jetteras au milieu du feu, et les brûleras au feu. De là sortira un feu contre toute la maison d'Israël.
\VS{5}Ainsi parle le Seigneur Yahweh : C'est là cette Jérusalem que j'avais placée au milieu des nations et des pays qui sont autour d'elle.
\VS{6}Elle a changé mes ordonnances et s'est rendue plus coupable que les nations et les pays d'alentour ; car ils ont rejeté mes ordonnances, et n'ont point marché dans mes ordonnances.
\VS{7}C'est pourquoi ainsi parle le Seigneur Yahweh : Parce que vous avez multiplié vos méchancetés plus que les nations qui vous entourent, et que vous n'avez point suivi mes ordonnances et observé mes lois, et que vous n'avez pas agi selon les ordonnances des nations qui vous entourent ;
\VS{8}à cause de cela, ainsi parle le Seigneur Yahweh : Voici, j'en veux à toi et j'exécuterai au milieu de toi mes jugements, sous les yeux des nations.
\VS{9}Je te ferai, à cause de toutes tes abominations, des choses que je n'ai jamais faites, et ce que je ne ferai jamais\FTNT{Da. 9:12 ; Mt. 24:21.}.
\VS{10}Des pères mangeront leurs fils au milieu de toi, et des fils mangeront leurs pères ; j'exécuterai mes jugements sur toi, et je disperserai à tous les vents tout ce qui restera de toi\FTNT{Lé. 26:33 ; De. 28:64 ; Jé. 9:16 ; Za. 2:6.}.
\VS{11}Je suis vivant, dit le Seigneur Yahweh, parce que tu as souillé mon lieu saint par toutes tes infamies, et par toutes tes abominations, moi-même je te raserai, et mon oeil ne t'épargnera point, et je n'aurai point de compassion\FTNT{Jé. 7:9-11.}.
\VS{12}Un tiers d'entre vous mourra de la peste, et sera consumé par la famine au milieu de toi ; un tiers tombera par l'épée autour de toi ; et je disperserai à tous les vents l'autre tiers, je tirerai l'épée derrière eux.
\VS{13}Car ma colère sera portée à son comble, je ferai reposer ma fureur sur eux, et je me donnerai satisfaction ; ils sauront que moi, Yahweh, j'ai parlé dans ma jalousie, quand j'aurai consumé ma fureur sur eux.
\VS{14}Je ferai de toi un désert, un sujet d'opprobre parmi les nations qui sont autour de toi, aux yeux de tous les passants\FTNT{Lé. 26:31-32 ; Né. 2:17.}.
\VS{15}Tu seras en opprobre, en ignominie, un exemple et un sujet d'étonnement pour les nations qui t'entourent, quand j'aurai exécuté mes jugements sur toi, avec colère, avec fureur, et par des châtiments pleins de fureur ; moi, Yahweh, j'ai parlé\FTNT{De. 28:37 ; 1 R. 9:7 ; Ps. 79:4 ; Jé. 24:9 ; Es. 26:9.}.
\VS{16}Quand je lancerai sur eux les flèches douloureuses de la famine, qui seront mortelles, quand je les enverrai pour vous détruire, j'ajouterai la famine sur vous, et romprai pour vous le bâton du pain\FTNT{De. 32:24.}.
\VS{17}Je vous enverrai la famine, et des bêtes féroces, qui te priveront d'enfants ; la peste et le sang passeront au milieu de toi, et je ferai venir l'épée sur toi. Moi,Yahweh, j'ai parlé.
\Chap{6}
\TextTitle{Grâce de Yahweh pour quelques réchappés d'Israël}
\VerseOne{}La parole de Yahweh me fut encore adressée, en disant :
\VS{2}Fils de l'homme, tourne ta face contre les montagnes d'Israël, et prophétise contre elles !
\VS{3}Et dis : Montagnes d'Israël, écoutez la parole du Seigneur Yahweh. Ainsi parle le Seigneur Yahweh aux montagnes et aux collines, aux cours des rivières, et aux vallées : Me voici, je vais faire venir l'épée sur vous, et je détruirai vos hauts lieux\FTNT{Lé. 26:30.}.
\VS{4}Vos autels seront dévastés, vos autels d'encens seront brisés, et je ferai tomber vos morts devant vos idoles.
\VS{5}Car je mettrai les cadavres des fils d'Israël devant leurs idoles, et je disperserai vos os autour de vos autels\FTNT{2 R. 23:14-20.}.
\VS{6}Les villes seront désertes, là où sont vos demeures, et les hauts lieux seront dévastés, vos autels seront délaissés et abandonnés, et vos idoles seront brisées et ne seront plus ; vos autels d'encens abattus, et vos ouvrages seront nettoyés.
\VS{7}Les tués tomberont parmi vous ; et vous saurez que je suis Yahweh.
\VS{8}Mais je laisserai quelques restes d'entre vous, afin que vous ayez quelques réchappés de l'épée parmi les nations, quand vous serez dispersés parmi les pays.
\VS{9}Vos réchappés se souviendront de moi\FTNT{Jé. 51:50.} parmi les nations où ils seront captifs, parce que j'aurai brisé leur cœur adonné à la fornication, qui s'est détourné de moi, et à cause de leurs yeux qui se sont livrés à la prostitution après leurs idoles ; ils se prendront eux-mêmes en dégoût, à cause du mal qu'ils ont commis, à cause de leurs abominations.
\VS{10}Ils sauront que je suis Yahweh, que ce n'est point en vain que je les ai menacés.
\TextTitle{Sentence envers les idolâtres}
\VS{11}Ainsi parle le Seigneur Yahweh : Frappe de ta main et bats de ton pied, et dis : Hélas ! A cause de toutes les abominations, des maux de la maison d'Israël ; car ils tomberont par l'épée, par la famine, et par la peste.
\VS{12}Celui qui sera loin mourra de la peste, et celui qui sera près tombera par l'épée ; et celui qui restera et sera assiégé, mourra par la famine, ainsi je consumerai ma fureur sur eux\FTNT{Am. 4:10.}.
\VS{13}Vous saurez que je suis Yahweh quand les blessés à morts seront au milieu de leurs idoles, autour de leurs autels, sur toute colline élevée, sur tous les sommets des montagnes, sous tout arbre vert, et sous tout chêne touffu, là où ils offraient des parfums de bonne odeur à toutes leurs idoles\FTNT{Os. 4:13.}.
\VS{14}J'étendrai donc ma main sur eux, et je rendrai leur pays désolé et désert dans toutes leurs demeures, plus que le désert qui est vers Dibla. Et ils sauront que je suis Yahweh.
\Chap{7}
\TextTitle{Attaque babylonienne imminente}
\VerseOne{}Puis la parole de Yahweh me fut adressée, en disant :
\VS{2}Et toi, fils de l'homme, écoute : Ainsi parle le Seigneur Yahweh à la terre d'Israël : La fin, la fin vient sur les quatre coins de la terre !
\VS{3}Maintenant la fin vient sur toi, j'enverrai sur toi ma colère, et je te jugerai selon ta voie, et je mettrai sur toi toutes tes abominations\FTNT{Ro. 2:6.}.
\VS{4}Et mon œil ne t'épargnera point, et je n'aurai point de compassion ; mais je te chargerai de tes voies, et tes abominations seront au milieu de toi ; et vous saurez que je suis Yahweh.
\VS{5}Ainsi parle le Seigneur Yahweh : Voici un mal, un seul mal qui vient !
\VS{6}La fin vient, la fin vient, elle se réveille contre toi ; voici, le mal vient !
\VS{7}Ton tour arrive, habitant du pays ! Le temps vient, le jour est près de toi, il ne sera que frayeur, et non pas une invitation des montagnes\FTNT{So. 1:14-15.} à s'entre-réjouir.
\VS{8}Maintenant, je répandrai bientôt ma fureur sur toi, et je consumerai ma colère sur toi ; je te jugerai selon ta voie, je mettrai sur toi toutes tes abominations.
\VS{9}Mon œil ne t'épargnera point, et je n'aurai point de compassion, je te punirai selon ta voie, et tes abominations seront au milieu de toi ; et vous saurez que je suis Yahweh qui frappe.
\VS{10}Voici le jour, voici il vient, le matin paraît, la verge fleurit, l'orgueil bourgeonne.
\VS{11}La violence s'élève pour servir de verge à la méchanceté ; il ne restera rien d'eux, ni de leur multitude, ni de leur tumulte, et on ne se lamentera point sur eux.
\VS{12}Le temps vient, le jour est tout proche : Que celui donc qui achète ne se réjouisse point, et que celui qui vend ne se lamente point ; car il y a une ardente colère sur toute leur multitude.
\VS{13}Car le vendeur ne recouvrera pas ce qu'il a vendu, serait-il encore parmi les vivants ; car la vision touchant toute leur multitude ne sera point révoquée ; et à cause de son iniquité, nul ne conservera sa vie.
\VS{14}On sonne de la trompette, tout est prêt, mais il n'y a personne pour aller au combat, parce que l'ardeur de ma colère est sur toute leur multitude.
\VS{15}L'épée est au-dehors, la peste et la famine au-dedans ! Celui qui est aux champs mourra par l'épée ; et celui qui est dans la ville, la famine et la peste le dévoreront.
\VS{16}Les réchappés s'enfuiront et seront sur les montagnes comme les pigeons des vallées, tous gémissant, chacun sur son iniquité.
\VS{17}Toutes les mains deviendront lâches, et tous les genoux se fondront en eau\FTNT{Es. 13:7 ; Jé. 6:24.}.
\VS{18}Ils se ceindront de sacs, et le tremblement les couvrira, la confusion sera sur tous leurs visages, et leurs têtes deviendront chauves\FTNT{Es. 3:24 ; Jé. 48:37 ; Am. 8:10.}.
\VS{19}Ils jetteront leur argent par les rues, et leur or s'en ira au loin ; leur argent et leur or ne pourront pas les délivrer au jour de la grande colère de Yahweh\FTNT{Pr. 11:4 ; So. 1:18.} ; ils ne rassasieront point leurs âmes, et ne rempliront point leurs entrailles, parce que leur iniquité aura été leur ruine.
\TextTitle{Violation du temple}
\VS{20}Ils étaient fiers de leur magnifique parure ; mais ils y ont placé des images de leurs abominations et de leurs infamies, c'est pourquoi je la rendrai pour eux un objet d'horreur.
\VS{21}Je l'ai livrée au pillage dans la main des étrangers, et en proie aux méchants de la terre qui la profaneront\FTNT{Jé. 20:5.}.
\VS{22}Je détournerai aussi ma face d'eux, et on violera mon lieu secret, et des furieux entreront et le profaneront.
\VS{23}Fais une chaîne ! Car le pays est plein de crimes, de meurtre, et la ville est pleine de violence.
\VS{24}C'est pourquoi je ferai venir les plus méchants des nations, qui possèderont leurs maisons, et je ferai cesser l'orgueil des puissants, et leurs saints lieux seront profanés.
\VS{25}La destruction vient, et ils chercheront la paix, mais il n'y en aura point.
\VS{26}Il viendra malheur sur malheur, et il y aura rumeur sur rumeur ; ils demanderont la vision aux prophètes\FTNT{La. 2:9.} ; la loi périra chez le sacrificateur, et le conseil chez les anciens.
\VS{27}Le roi se lamentera, les princes se vêtiront de désolation, et les mains du peuple du pays tomberont de frayeur. Je les traiterai selon leur voie, je les jugerai comme ils le méritent et ils sauront que je suis Yahweh.
\Chap{8}
\TextTitle{Visions divines}
\VerseOne{}Puis il arriva dans la sixième année, au cinquième jour du sixième mois, comme j'étais assis dans ma maison, et que les anciens de Juda étaient assis devant moi, que la main du Seigneur Yahweh tomba là sur moi.
\VS{2}Je regardai, et voici c'était une figure ayant l'aspect d'un feu qui frappe les regards ; depuis ses reins jusqu'en bas c'était du feu, et depuis ses reins jusqu'en haut, c'était d'un aspect brillant comme de l'airain poli.
\VS{3}Il étendit une forme de main et me prit par les cheveux de ma tête. L'Esprit m'enleva entre la terre et le ciel et me transporta à Jérusalem, dans des visions de Dieu, à l'entrée de la porte intérieure, du côté nord, où était posée l'idole de jalousie\FTNT{L'idole de la jalousie : Dans le temple de Jérusalem à l'époque d'Ezéchiel, l'idolâtrie s'y développait sans retenue (2 R. 21, 22 et 23). Il y avait dans ce temple les idoles d'Astarté et les autels de Baal. Le temple était souillé.} qui provoque la jalousie.
\VS{4}Voici, la gloire du Dieu d'Israël était là, telle que je l'avais vue en vision dans la vallée.
\TextTitle{Abominations dans le temple}
\VS{5}Il me dit : Fils de l'homme, lève maintenant tes yeux vers le chemin qui tend vers le nord ! J'élevai mes yeux vers le chemin qui tend vers le nord, et voici du côté nord, à la porte de l'autel, était cette idole de jalousie, à l'entrée.
\VS{6}Il me dit : Fils de l'homme, ne vois-tu pas ce qu'ils font, les grandes abominations que la maison d'Israël commet ici, pour que je me retire de mon lieu saint ? Mais tourne-toi encore, tu verras de grandes abominations.
\VS{7}Il me conduisit donc à l'entrée du parvis. Je regardai, et voici, il y avait un trou dans le mur.
\VS{8}Il me dit : Fils de l'homme, perce maintenant le mur ; et quand je perçai le mur, il y avait une porte.
\VS{9}Puis il me dit : Entre et regarde les méchantes abominations qu'ils commettent ici.
\VS{10}J'entrai donc et je regardai ; et voici, toutes sortes de figures de reptiles et de bêtes abominables, et toutes les idoles de la maison d'Israël étaient peintes sur le mur tout autour\FTNT{Ex. 20:4 ; De. 4:16-18 ; Ro. 1:23.}.
\VS{11}Soixante-dix hommes des anciens de la maison d'Israël, au milieu desquels était Jaazania, fils de Schaphan, se tenaient debout devant ces idoles, chacun l'encensoir à la main, d'où s'élevait une épaisse nuée d'encens.
\VS{12}Alors il me dit : Fils de l'homme, n'as-tu pas vu ce que les anciens de la maison d'Israël font dans les ténèbres, chacun dans sa chambre pleine de figures ? Car ils disent : Yahweh ne nous voit point, Yahweh a abandonné le pays\FTNT{Es. 29:15.}.
\VS{13}Puis il me dit : Tourne-toi encore, et tu verras les grandes abominations qu'ils commettent.
\VS{14}Il me conduisit donc à l'entrée de la porte de la maison de Yahweh qui est vers le nord. Et voici, il y avait là des femmes assises qui pleuraient Thammuz\FTNT{Thammuz ou Adonis.}.
\VS{15}Il me dit : Fils de l'homme, n'as-tu pas vu ? Tourne-toi encore, et tu verras des abominations plus grandes que celles-ci.
\VS{16}Il me fit donc entrer dans le parvis intérieur de la maison de Yahweh. Et voici, à l'entrée du temple de Yahweh, entre le portique et l'autel, environ vingt-cinq hommes avaient le dos tourné contre le temple de Yahweh, leurs visages tournés vers l'orient ; et ils se prosternaient vers l'orient, devant le soleil\FTNT{De. 4:19.}.
\VS{17}Alors il me dit : Fils de l'homme, n'as-tu pas vu ? Est-ce une chose légère à la maison de Juda de commettre ces abominations qu'ils commettent ici ? Car ils ont rempli le pays de violence, et ils se sont ainsi tournés pour m'irriter ; mais voici ils approchent le rameau de leurs nez.
\VS{18}Et moi, j'agirai dans ma fureur ; mon œil ne les épargnera point, et je n'aurai point de compassion ; quand ils crieront à haute voix à mes oreilles, je ne les exaucerai point\FTNT{Pr. 1:28 ; Es. 1:15 ; Jé. 11:11 ; Mi. 3:4 ; Za. 7:13.}.
\Chap{9}
\TextTitle{Marque de Yahweh sur les justes ; extermination des impies}
\VerseOne{}Puis il cria d'une voix forte à mes oreilles : Faites approcher ceux qui châtient la ville, chacun avec son instrument de destruction à la main !
\VS{2}Et voici, six hommes venaient par le chemin de la haute porte qui regarde vers le nord, et chacun avait dans sa main son instrument de destruction. Il y avait au milieu d'eux un homme vêtu de lin, qui avait une écritoire sur ses reins ; ils entrèrent et se tinrent près de l'autel d'airain.
\VS{3}Alors la gloire du Dieu d'Israël s'éleva du chérubin sur lequel elle était, et vint sur le seuil de la maison. Il cria à l'homme qui était vêtu de lin et qui avait l'écritoire sur ses reins.
\VS{4}Yahweh lui dit : Passe par le milieu de la ville, par le milieu de Jérusalem, et marque la lettre Thau sur les fronts des hommes qui gémissent et qui soupirent à cause de toutes les abominations qui s'y commettent\FTNT{Ap. 7:3 ; Ap. 9:4 ; Ap. 13:16-17 ; Ap. 20:4 ; Ex. 12:7-23.}.
\VS{5}Et s'adressant aux autres en ma présence, il dit : Passez dans la ville après lui, et frappez ; que votre oeil soit sans pitié et n'ayez point de compassion !
\VS{6}Tuez-les tous, les vieillards, les jeunes gens, les vierges, les enfants et les femmes\FTNT{2 Ch. 36:17.} ; mais n'approchez pas de ceux qui ont la lettre Thau\FTNT{La lettre Thau ou Tav est la marque, le signe, le symbole ou le sceau Divin. La lettre Tav est formée par la réunion des lettres Daleth et Nun. Ces deux lettres forment le mot « dan » qui veut dire « juge ». Selon la Bible, la marque des chrétiens est représentée par : Le Saint-Esprit, le nom de Jésus-Christ (Ep. 1:13-14 ; Ep. 4:30 ; Ap. 14:1), le nom de la Nouvelle Jérusalem (Ap. 3 : 12) et le nom du Père. Les chrétiens fidèles à Dieu sont marqués par l'Esprit de Dieu qui est notre sceau. Le Saint-Esprit est saint, la sainteté est donc la marque des chrétiens (1 Pi. 1:2). Il est aussi l'Esprit de vérité, donc la vérité est aussi la marque des chrétiens (Jn. 16:13). Il est aussi amour, l'amour étant également la marque distinctive des véritables chrétiens (Ro. 5:5).}, et commencez par mon lieu saint\FTNT{Le jugement commence par la maison de Dieu (1 Pi. 4:17-18).}. Ils commencèrent donc par les vieillards qui étaient devant la maison.
\VS{7}Il leur dit : Profanez la maison, et remplissez de morts les parvis !… Sortez !… Et ils sortirent, et ils frappèrent dans la ville.
\VS{8}Or il arriva que comme ils frappaient, je restai là, et m'étant prosterné le visage contre terre, je criai et dis : Ah ! Seigneur Yahweh ! Vas-tu donc détruire tous les restes d'Israël en répandant ta fureur sur Jérusalem ?
\VS{9}Il me dit : L'iniquité de la maison d'Israël et de Juda est excessivement grande, le pays est rempli de meurtres et la ville remplie de crimes ; car ils ont dit : Yahweh a abandonné le pays, Yahweh ne nous voit point.
\VS{10}Quant à moi, mon oeil aussi ne les épargnera point, et je n'en aurai point compassion ; je mettrai leur voie sur leur tête.
\VS{11}Et voici, l'homme vêtu de lin, qui avait une écritoire sur ses reins, rapporta ce qui avait été fait, et il dit : J'ai fait comme tu m'as ordonné.
\Chap{10}
\TextTitle{La gloire de Yahweh quitte le temple}
\VerseOne{}Je regardai, et voici, sur l'étendue qui était au-dessus de la tête des chérubins, parut comme une pierre de saphir ; on voyait au-dessus d'eux quelque chose de semblable à un trône.
\VS{2}On parla à l'homme vêtu de lin, et on lui dit : Va entre les roues, sous les chérubins, et remplis tes mains de charbons ardents que tu prendras entre les chérubins, et répands-les sur la ville\FTNT{Es. 6:6 ; Ap. 8:5.} ; il y entra devant mes yeux.
\VS{3}Les chérubins étaient à la droite de la maison quand l'homme entra ; et une nuée remplit le parvis intérieur\FTNT{1 R. 8:10-11.}.
\VS{4}Puis la gloire de Yahweh s'éleva de dessus les chérubins pour venir sur le seuil de la maison, et la maison fut remplie d'une nuée, et le parvis fut rempli de la splendeur de la gloire de Yahweh.
\VS{5}On entendit le bruit des ailes des chérubins jusqu'au parvis extérieur, pareil à la voix du Dieu Tout-Puissant lorsqu'il parle.
\VS{6}Ainsi Yahweh donna cet ordre à l'homme qui était vêtu de lin : Prends du feu d'entre les roues des chérubins ; il entra et se tint auprès des roues.
\VS{7}L'un des chérubins étendit sa main entre les chérubins, vers le feu qui était entre les chérubins ; il en prit et le mit entre les mains de l'homme vêtu de lin. Et cet homme le prit et sortit.
\VS{8}On voyait aux chérubins une forme de main d'homme sous leurs ailes.
\VS{9}Puis je regardai, et voici, il y avait quatre roues près des chérubins, une roue près de chaque chérubin ; et ces roues avaient l'aspect d'une pierre de chrysolithe.
\VS{10}A leur aspect, toutes les quatre avaient la même forme ; chaque roue paraissait être au milieu d'une autre roue.
\VS{11}Quand elles marchaient, elles allaient de leurs quatre côtés, et elles ne se tournaient point dans leur marche ; mais elles allaient dans la direction de la tête, sans se tourner dans leur marche.
\VS{12}Tout le corps des chérubins, leur dos, leurs mains, leurs ailes, étaient remplis d'yeux, aussi bien que les roues tout autour, les quatre roues\FTNT{Ap. 4:6-8.}.
\VS{13}J'entendis qu'on appela les roues tourbillon.
\VS{14}Chaque animal avait quatre faces : La première face était la face d'un chérubin ; la seconde face était la face d'un homme ; la troisième était la face d'un lion ; et la quatrième la face d'un aigle\FTNT{Ez. 1 ; Ap. 4:7.}.
\VS{15}Puis les chérubins s'élevèrent. Ce sont là les animaux que j'avais vus près du fleuve de Kebar.
\VS{16}Lorsque les chérubins marchaient, les roues aussi marchaient à côté d'eux ; et quand les chérubins élevaient leurs ailes pour s'élever de terre, les roues ne se détournaient point d'eux.
\VS{17}Lorsqu'ils s'arrêtaient, elles s'arrêtaient ; et lorsqu'ils s'élevaient, elles s'élevaient ; car l'esprit des animaux était dans les roues.
\VS{18}Puis la gloire de Yahweh se retira de dessus le seuil de la maison, et se tint au-dessus des chérubins.
\VS{19}Les chérubins élevant leurs ailes, s'élevèrent de terre sous mes yeux quand ils partirent ; les roues s'élevèrent aussi. Et chacun d'eux s'arrêta à l'entrée de la porte orientale de la maison de Yahweh ; la gloire du Dieu d'Israël était sur eux en haut.
\VS{20}C'étaient les animaux que j'avais vus sous le Dieu d'Israël près du fleuve de Kebar ; et je reconnus que c'étaient des chérubins.
\VS{21}Chacun avait quatre faces, et chacun quatre ailes, une forme de main d'homme était sous leurs ailes.
\VS{22}Quant à l'aspect de leurs faces, c'étaient les faces que j'avais vues près du fleuve de Kebar, c'était le même aspect, c'étaient eux-mêmes. Et chacun marchait droit devant soi.
\Chap{11}
\TextTitle{Sentences sur les princes infidèles}
\VerseOne{}Puis l'Esprit m'enleva et me transporta à la porte orientale de la maison de Yahweh, à celle qui regarde vers l'orient. Et il y avait vingt-cinq hommes à l'entrée de la porte, et je vis au milieu d'eux Jaazania, fils d'Azzur, et Pelathia, fils de Benaja, les princes du peuple.
\VS{2}Il me dit : Fils de l'homme, ce sont les hommes qui ont des pensées d'iniquité, et qui donnent un mauvais conseil dans cette ville\FTNT{Mi. 2:1.}.
\VS{3}Ils disent : Ce n'est pas le moment ! Bâtissons des maisons ! La ville est la chaudière et nous sommes la viande.
\VS{4}C'est pourquoi prophétise contre eux, prophétise, fils de l'homme !
\VS{5}L'Esprit de Yahweh tomba sur moi. Et il me dit : Ainsi parle Yahweh : Vous parlez de la sorte, maison d'Israël, et je connais toutes les pensées de votre esprit.
\VS{6}Vous avez multiplié les meurtres dans cette ville, et vous avez rempli ses rues de gens que vous avez tués.
\VS{7}C'est pourquoi, ainsi parle le Seigneur Yahweh : Les gens que vous avez tués, et que vous avez mis au milieu d'elle, sont la viande, et elle est la chaudière, mais je vous tirerai hors du milieu d'elle\FTNT{Mi. 3:3.}.
\VS{8}Vous avez eu peur de l'épée, mais je ferai venir l'épée sur vous, dit le Seigneur Yahweh\FTNT{Jé. 42:16.}.
\VS{9}Je vous tirerai hors de la ville, je vous livrerai entre les mains des étrangers, et j'exécuterai mes jugements contre vous.
\VS{10}Vous tomberez par l'épée ; je vous jugerai dans le pays d'Israël, et vous saurez que je suis Yahweh.
\VS{11}Elle ne sera point une chaudière pour vous, et vous ne serez point au dedans d'elle comme la viande ; je vous jugerai dans le pays d'Israël.
\VS{12}Et vous saurez que je suis Yahweh ; car vous n'avez point suivi mes ordonnances, et vous n'avez pas observé mes lois, mais vous avez agi selon les ordonnances des nations qui sont autour de vous.
\VS{13}Or il arriva comme je prophétisais, que Pelathia, fils de Benaja, mourut. Alors je me prosternai sur mon visage, je criai à haute voix, et dis : Ah ! Seigneur Yahweh ! Vas-tu consumer entièrement le reste d'Israël ?
\TextTitle{Restauration d'Israël et de ses exilés}
\VS{14}La parole de Yahweh me fut adressée, en disant :
\VS{15}Fils de l'homme, tes frères, tes frères, les hommes de ta parenté, et la maison d'Israël tout entière, à qui les habitants de Jérusalem ont dit : Eloignez-vous de Yahweh, la terre nous a été donnée en héritage.
\VS{16}C'est pourquoi dis-leur : Ainsi parle le Seigneur Yahweh : Quoique je les aie éloignés des nations, et que je les aie dispersés dans divers pays, je serai pour eux quelque temps un lieu saint\FTNT{Alors que le lieu saint ou maison terrestre était souillée, Yahweh se présente comme le Lieu Sacré pour son peuple.} dans les pays où ils sont venus.
\VS{17}C'est pourquoi dis-leur : Ainsi parle le Seigneur Yahweh : Je vous rassemblerai du milieu des peuples, et je vous recueillerai des pays auxquels vous avez été dispersés, et je vous donnerai la terre d'Israël\FTNT{Es. 11:11-16 ; Jé. 24:6 ; Ez. 28:25 ; Ez. 34:13 ; Ez. 36:24.}.
\VS{18}C'est là qu'ils iront, et ils ôteront hors d'elle toutes ses infamies et toutes ses abominations.
\VS{19}Je leur donnerai un même cœur, et je mettrai en eux un esprit nouveau ; j'ôterai de leur corps le cœur de pierre, et je leur donnerai un cœur de chair\FTNT{Il s'agit d'une allusion à la nouvelle alliance (Jé. 31:31-34 ; Hé. 8).},
\VS{20}afin qu'ils suivent mes ordonnances, et qu'ils gardent et observent mes lois ; ils seront mon peuple, et je serai leur Dieu.
\VS{21}Quant à ceux dont le cœur se plaît à leurs idoles et à leurs abominations, quant à ceux-là, je ferai tomber sur leur tête les peines que mérite leur conduite, dit le Seigneur Yahweh.
\TextTitle{La gloire de Dieu en mouvement vers le Mont des Oliviers\FTNTT{Cp. Ez. 43:1-4.}}
\VS{22}Puis les chérubins élevèrent leurs ailes, accompagnés des roues ; et la gloire du Dieu d'Israël était sur eux, en haut.
\VS{23}La gloire de Yahweh s'éleva du milieu de la ville\FTNT{Le départ de la gloire de Dieu du temple de Jérusalem marque la fin de la théocratie (règne de Dieu) en Israël. Cet événement, comparable au retrait de l'Esprit de Dieu en Ge. 6:3, fut consécutif à la décadence morale d'Israël (voir Ez. 8) qui fut désormais livré aux nations. Certains estiment que la théocratie a cessé au moment où les israélites ont demandé un roi (voir 1 S. 8). Or bien que cette demande déplut à Yahweh, il continua néanmoins à diriger Israël au travers des souverains tels que David, qu'il établissait à la tête de son peuple. Les Hébreux avaient déjà reçu un sérieux avertissement avec la destruction du temple lors de la première déportation babylonienne (2 R. 24). Cet événement, bien que traumatisant pour beaucoup, n'avait cependant pas provoqué une réelle repentance, c'est pourquoi les israélites retombèrent rapidement dans leurs travers. Ainsi, comme en témoigne Mal. 2 :17 qui rapporte les propos de certains juifs : « Où est le Dieu de la justice ? », en dépit de la reconstruction du temple sous Néhémie et Esdras, la gloire de Dieu ne s'y manifestait plus depuis longtemps. Ezéchiel ne fait donc qu'assister à la conséquence de plusieurs siècles d'infidélité des juifs à l'égard de leur Dieu.}, et s'arrêta sur la montagne qui est à l'orient de la ville.
\VS{24}Puis l'Esprit m'enleva et me transporta en Chaldée, vers ceux qui avaient été emmenés captifs, le tout en vision par l'Esprit de Dieu ; et la vision que j'avais vue disparut au-dessus de moi.
\VS{25}Alors je dis à ceux qui avaient été emmenés captifs toutes les paroles que Yahweh m'avait révélées.
\Chap{12}
\TextTitle{Fuite d'Ezéchiel, un signe pour Israël}
\VerseOne{}La parole de Yahweh me fut encore adressée en ces mots :
\VS{2}Fils de l'homme : Tu habites au milieu d'une maison rebelle, au milieu de gens qui ont des yeux pour voir, et ne voient point ; et qui ont des oreilles pour entendre, et n'entendent point ; parce qu'ils sont une maison de rebelles\FTNT{Es. 6:9 ; Es. 49:19-20 ; Jé. 5:21 ; Ac. 28:26.}.
\VS{3}Toi donc fils d’homme, fais-toi des bagages d’un homme qui s'exile et pars en exil de jour, sous leurs yeux, pars en exil, dis-je de ton lieu pour aller dans un autre lieu, sous leurs yeux. Peut-être qu'ils y prendront garde, quoi qu’ils soient une maison rebelle.
\VS{4}Tu mettras donc dehors pendant le jour tes bagages comme les bagages d'un homme qui s'exile sous leurs yeux, et le soir, tu sortiras sous leurs yeux, comme quand on sort pour s'exiler.
\VS{5}Perce-toi le mur sous leurs yeux et sors par là tes bagages.
\VS{6}Tu les porteras sur tes épaules, sous leurs yeux, et tu sortiras tes bagages pendant l'obscurité. Tu couvriras aussi ton visage, afin que tu ne voies point la terre ; car je t'ai mis pour être un signe pour la maison d'Israël.
\VS{7}Je fis donc ce qui m'avait été ordonné : Je portai dehors pendant le jour mes bagages comme des bagages d'exil ; le soir je perçai le mur avec la main et je les sortis pendant l'obscurité, je les portai sur l'épaule, sous leurs yeux.
\VS{8}Au matin, la parole de Yahweh me fut adressée en ces mots :
\VS{9}Fils de l'homme, la maison d'Israël, maison rebelles, ne t'a-t-elle pas dit : Qu'est-ce que tu fais ?
\VS{10}Dis-leur : Ainsi parle le Seigneur Yahweh : Cet ordre dont je suis chargé s'adresse au prince qui est à Jérusalem et à toute la maison d'Israël qui s'y trouve.
\VS{11}Dis : Je suis pour vous un signe ; comme j'ai fait, ainsi il leur sera fait ; ils iront en exil, en captivité.
\VS{12}Et le prince qui est parmi eux, mettra son bagage sur l'épaule et sortira ; on percera le mur pour le tirer dehors ; il couvrira son visage, afin qu'il ne voie point de ses yeux la terre\FTNT{2 R. 25:4.}.
\VS{13}J'étendrai mon rets sur lui, et il sera pris dans mes filets ; je le ferai entrer dans Babylone, au pays des Chaldéens, mais il ne la verra point, et il y mourra.
\VS{14}Je disperserai à tout vent tout ce qui est autour de lui, son secours, et tous ses corps d'armées ; et je tirerai l'épée sur eux.
\VS{15}Ils sauront que je suis Yahweh, quand je les aurai répandus parmi les nations, et que je les aurai dispersés dans divers pays.
\VS{16}Je laisserai un reste d'entre eux, quelques hommes, préservés de l'épée, de la famine, et de la peste, afin qu'ils racontent toutes leurs abominations parmi les nations où ils iront ; et ils sauront que je suis Yahweh.
\TextTitle{La captivité du peuple imminente\FTNTT{Cp. 2 R. 25:1-10.}}
\VS{17}Puis la parole de Yahweh me fut adressée en ces mots :
\VS{18}Fils de l'homme, mange ton pain dans l'agitation, et bois ton eau en tremblant et avec inquiétude.
\VS{19}Puis tu diras au peuple du pays : Ainsi parle le Seigneur Yahweh, sur les habitants de Jérusalem, à la terre d'Israël : Ils mangeront leur pain avec chagrin, et ils boiront leur eau avec frayeur, parce que son pays sera désolé, étant privé de son abondance, à cause de la violence de tous ceux qui y habitent.
\VS{20}Les villes peuplées seront désertes, et le pays ne sera que désolation ; et vous saurez que je suis Yahweh.
\VS{21}La parole de Yahweh me fut encore adressée en ces mots :
\VS{22}Fils de l'homme, que signifient ces discours moqueurs que vous tenez sur la terre d'Israël, en disant : Les jours seront prolongés, et toute vision périra\FTNT{Es. 5:19 ; Am. 6:3 ; 2 Pi. 3:3.} ?
\VS{23}C'est pourquoi dis-leur : Ainsi parle le Seigneur Yahweh : Je ferai cesser ce proverbe, et on ne s'en servira plus comme proverbe en Israël ; et dis-leur : Les jours approchent, et toutes les visions s'accompliront.
\VS{24}Car il n'y aura plus désormais aucune vision de vanité ni aucune divination de flatteur, au milieu de la maison d'Israël.
\VS{25}Car moi, Yahweh, je parlerai, et la parole que j'aurai prononcée sera mis en exécution, elle ne sera plus différée ; mais ô maison rebelle! Je prononcerai en vos jours la parole, et je l'exécuterai dit le Seigneur Yahweh.
\VS{26}La parole de Yahweh me fut encore adressée en ces mots :
\VS{27}Fils de l'homme, voici, ceux de la maison d'Israël disent : La vision que celui-ci voit n'arrivera pas avant longtemps, et il prophétise pour des temps qui sont encore éloignés.
\VS{28}C'est pourquoi dis-leur : Ainsi parle le Seigneur Yahweh : Aucune de mes paroles ne sera plus différée, mais la parole que j'aurai prononcée sera exécutée incessament, dit le Seigneur Yahweh.
\Chap{13}
\TextTitle{Jugement sur ceux qui égarent le peuple de Dieu}
\VerseOne{}La parole de Yahweh me fut encore adressée en ces mots :
\VS{2}Fils de l'homme, prophétise contre les prophètes d'Israël qui prophétisent, et dis à ces prophètes qui prophétisent selon leur propre cœur : Ecoutez la parole de Yahweh !
\VS{3}Ainsi parle le Seigneur Yahweh : Malheur aux prophètes insensés qui suivent leur propre esprit, et qui n'ont point eu de vision.
\VS{4}Israël, tes prophètes ont été comme des renards dans les déserts.
\VS{5}Vous n'êtes point montés devant les brèches, et vous n'avez point réparé les murs pour la maison d'Israël, afin de vous tenir debout pour le combat au jour de Yahweh.
\VS{6}Ils ont eu des visions vaines et des divinations de mensonge, ils disent : Yahweh a dit ; et toutefois Yahweh ne les a point envoyés ; et ils font espérer que leur parole s'accomplira\FTNT{Les faux prophètes (Jé. 23) ; Jé. 14:14 ; Jé. 28:15.}.
\VS{7}N'avez-vous pas vu des visions de vanité, et prononcé des divinations de mensonge ? Cependant vous dites : Yahweh a parlé ; et je n'ai point parlé.
\VS{8}C'est pourquoi ainsi parle le Seigneur Yahweh : Parce que vous avez prononcé des choses vaines, et que vous avez eu des visions de mensonge, à cause de cela j'en veux à vous, dit le Seigneur Yahweh.
\VS{9}Et ma main sera sur les prophètes qui ont des visions de vanité et des divinations de mensonge ; ils ne seront plus admis dans le conseil de mon peuple, ils ne seront plus écrits dans les registres de la maison d'Israël, ils n'entreront plus dans la terre d'Israël ; et vous saurez que je suis le Seigneur Yahweh.
\VS{10}Parce, oui parce qu'ils ont abusé mon peuple, en disant : Paix ! Et il n'y avait point de paix\FTNT{Jé. 6:14 ; Jé. 8:11.}. L'un bâtissait le mur, et les autres l'induissaient de mortier mal lié.
\VS{11}Dis à ceux qui enduisent le mur de mortier mal lié, qu'il tombera ; il y aura une pluie débordante, et vous, pierres de grêle, vous tomberez sur lui, et un vent de tempête le fendra.
\VS{12}Et voici, le mur est tombé ; ne vous sera-t-il donc pas dit : Où est l'enduit dont vous l'avez couvert ?
\VS{13}C'est pourquoi, ainsi parle le Seigneur Yahweh : Je ferai dans ma fureur éclater un vent impétueux, et dans ma colère, il surviendra une pluie débordante et des pierres de grêle dans ma fureur, pour détruire entièrement.
\VS{14}Je démolirai le mur que vous avez enduit de mortier mal lié, je le jetterai par terre, tellement que son fondement sera découvert, et il tombera ; vous serez consumés au milieu de lui, et vous saurez que je suis Yahweh.
\VS{15}Ainsi j'accomplirai ma colère contre le mur, et contre ceux qui l'enduisent de mortier mal lié ; et je vous dirai : Le mur n'est plus ni ceux qui l'ont enduit ;
\VS{16}à savoir les prophètes d'Israël, qui prophétisent sur Jérusalem et qui voient pour elle des visions de paix ; et néanmoins il n'y a point de paix, dit le Seigneur Yahweh.
\VS{17}Aussi, toi, fils de l'homme, tourne ta face contre les filles de ton peuple qui prophétisent selon leur propre cœur, prophétise contre elles !
\VS{18} Et dis : ainsi parle le Seigneur Yahweh : Malheur à celles qui cousent des coussins\FTNT{Le mot « coussin » vient du terme hébreu « keceth », et signifie : bande, filet, faux phylactères,  tissu utilisé par les fausses prophétesses en Israël pour étayer leurs plans démoniaques de diseuses de bonne aventure. } pour s'accouder le long du bras jusqu'aux mains, et qui font des voiles pour mettre sur la tête des personnes de toute taille, pour séduire les âmes. Séduiriez-vous les âmes de mon peuple\FTNT{Ge. 10:9.} ; et conserveriez-vous vos âmes ?
\VS{19}Et me profaneriez-vous envers mon peuple pour des poignées d'orge et pour des morceaux de pain, en faisant mourir les âmes qui ne devaient point mourir et en faisant vivre les âmes qui ne devaient point vivre, en mentant à mon peuple qui écoute le mensonge ?
\VS{20}C'est pourquoi ainsi parle le Seigneur Yahweh : Voici, j'en veux à vos coussins, par lesquels vous séduisez les âmes pour les faire voler vers vous ; et je déchirerai ces coussins de vos bras, et je ferai échapper les âmes que vous avez attirées afin qu'elles volent vers vous\FTNT{Ap. 18:11-13 ; 1 Co. 6:10 ; 2 Pi. 2:14}.
\VS{21}Je déchirerai aussi vos voiles, et je délivrerai mon peuple d'entre vos mains, et ils ne seront plus entre vos mains pour en faire votre proie ; et vous saurez que je suis Yahweh.
\VS{22}Parce que vous avez affligé sans raison le cœur du juste, quand moi-même je ne l'ai point attristé, et que vous avez renforcé les mains du méchant, afin qu'il ne se détourne point de son mauvais chemin, et que je lui sauve la vie.
\VS{23}C'est pourquoi, vous n'aurez plus aucune vision de vanité ni aucune divination, mais je délivrerai mon peuple d'entre vos mains ; et vous saurez que je suis Yahweh.
\Chap{14}
\TextTitle{Jugement sur les anciens idolâtres}
\VerseOne{}Or quelques-uns des anciens d'Israël vinrent auprès de moi et s'assirent devant moi.
\VS{2}Et la parole de Yahweh me fut adressée en ces mots :
\VS{3}Fils de l'homme, ces gens élèvent leurs idoles dans leurs cœurs, et ils attachent les regards sur ce qui les fait tomber dans l'iniquité. Serais-je consulté par eux sérieusement ?
\VS{4}C'est pourquoi parle-leur et dis-leur : Ainsi parle le Seigneur Yahweh. Quiconque de la maison d'Israël aura élevé ses idoles dans son cœur, et aura mis devant sa face ce qui l'a fait tomber dans son iniquité, s'il vient vers le prophète, je suis Yahweh, je lui répondrai puisqu'il vient avec la multitude de ses idoles,
\VS{5}afin que je saisisse la maison d'Israël par leur propre cœur; car eux tous se sont éloignés de moi par leurs idoles.
\VS{6}C'est pourquoi dis à la maison d'Israël : Ainsi parle le Seigneur Yahweh : Revenez, et détournez-vous de vos idoles, détournez les regards de toutes vos abominations\FTNT{Es. 55:6-7.}.
\VS{7}Car quiconque de la maison d'Israël, ou des étrangers qui séjournent en Israël, qui s'est séparé de moi, qui éleve ses idoles dans son cœur, et attache ses regards sur ce qui l'a fait tomber dans l'iniquité, s'il vient vers le prophète pour me consulter par de lui, je suis Yahweh, on lui répondra tout ce qu'on a à lui répondre.
\VS{8}Je me tournerai contre cet homme\FTNT{Lé. 17:10 ; Lé. 20:3-6 ; Jé. 44:11.}, et je ferai de lui un signe, et un sujet de sarcasme\FTNT{No. 26:10 ; De. 28:37}. Je le retrancherai du milieu de mon peuple ; et vous saurez que je suis Yahweh.
\VS{9}S'il arrive que le prophète soit séduit, et qu'il profère quelque parole, moi, Yahweh, je séduirai ce prophète-là\FTNT{1 R. 22:23 ; Job. 12:16 ; 2 Th. 2:11.} ; et j'étendrai ma main sur lui, et je l'exterminerai du milieu de mon peuple d'Israël.
\VS{10}Et ils porteront la peine de leur iniquité ; la peine de l'iniquité du prophète sera comme la peine de celui qui l'aura interrogé ;
\VS{11}afin que la maison d'Israël ne s'éloigne plus de moi, et qu'ils ne se souillent plus par tous leurs crimes\FTNT{Jé. 31:18-19 ; Hé. 12:11 ; Ja. 1:1-3.} ; alors ils seront mon peuple, et je serai leur Dieu, dit le Seigneur Yahweh.
\TextTitle{Châtiments d'Israël ; Yahweh épargne un reste}
\VS{12}Puis la parole de Yahweh me fut adressée en ces mots :
\VS{13}Fils de l'homme, lorsqu'un pays aura péché contre moi, en commettant une infidélité, et que j'aurai étendu ma main contre lui, et que je lui aurai rompu le bâton du pain, envoyé la famine et retranché du milieu de lui tant les hommes que les bêtes,
\VS{14}et que ces trois hommes, Noé, Daniel et Job s'y trouvent, ils sauveraient leurs âmes par leur justice, dit le Seigneur Yahweh.
\VS{15}Si je fais passer les bêtes féroces par ce pays-là et qu'elles le privent d'enfants, tellement qu'il soit devenu un désert où personne ne passe à cause des bêtes,
\VS{16}et que ces trois hommes-là s'y trouvent, je suis vivant, dit le Seigneur Yahweh, ils ne sauveraient ni fils ni filles, eux seulement seraient sauvés, et le pays sera un désert.
\VS{17}Si je faisais venir l'épée sur ce pays-là et si je disais : Que l'épée passe par le pays, et qu'elle en retranche les hommes et les bêtes !
\VS{18}Si ces trois hommes-là se trouvent au milieu du pays, je suis vivant, dit le Seigneur, ils ne sauveraient ni fils ni filles ; mais eux seulement seraient sauvés.
\VS{19}Ou si j'envoyais la peste dans ce pays, et que je répandais ma colère contre lui jusqu'à faire ruisseler le sang, au point de retrancher du milieu de lui les hommes et les bêtes,
\VS{20}et que Noé\FTNT{Ge. 6:8.}, Daniel\FTNT{Da. 1:8-12.} et Job\FTNT{Job. 1:8.}, s'y trouvent, je suis vivant, dit le Seigneur Yahweh, ils ne sauveraient ni fils ni filles ; mais ils sauveraient leurs âmes par leur justice.
\VS{21}Car ainsi parle le Seigneur Yahweh : J'envoie mes quatre plaies mortelles, l'épée, la famine, les bêtes féroces, et la peste, contre Jérusalem, pour en retrancher les hommes et les bêtes\FTNT{Jé. 15:2-3.} ;
\VS{22}Et toutefois, il y aura un reste qui échappera, qui en sortira, des fils et des filles. Voici, ils viennent vers vous, et vous verrez leur conduite et leurs actions, et vous serez consolés du malheur que je fais venir contre Jérusalem, de tout ce que je fais venir sur elle.
\VS{23}Vous serez consolés, lorsque vous verrez leur conduite et leurs actions ; et vous reconnaîtrez que ce n'est pas sans raison que je fais tout ce que je lui fais, dit le Seigneur Yahweh\FTNT{Jé. 22:8-9.}.
\Chap{15}
\TextTitle{Infidélités d'Israël\FTNTT{Cp. Es. 5:1-24.}}
\VerseOne{}La parole de Yahweh me fut encore adressée, en disant :
\VS{2}Fils de l'homme, que vaut le bois de la vigne de plus que les autres bois ? Et les sarments de plus que les branches des arbres d'une forêt ?
\VS{3}Et prendra-t-on du bois pour en faire quelque ouvrage ? Ou prendra-t-on un clou pour y pendre quelque chose ?
\VS{4}Voici, on le met au feu pour être consumé ; le feu consume aussitôt ses deux bouts, et le milieu est en feu ; serait-il bon pour quelque ouvrage ?
\VS{5}Voici, quand il est entier, on n'en fait aucun ouvrage ; à plus forte raison quand le feu l'aura consumé et qu'il sera brûlé, sera-t-il bon pour quelque ouvrage ?
\VS{6}C'est pourquoi ainsi parle le Seigneur Yahweh : Comme le bois de la vigne est parmi les arbres d'une forêt, que j'ai assigné au feu pour être consumé, ainsi je livrerai les habitants de Jérusalem.
\VS{7}Je me tournerai contre eux, seront-ils sortis du feu ? Encore le feu les consumera ; et vous saurez que je suis Yahweh, quand je tournerai ma face contre eux.
\VS{8}Je ferai de ce pays une désolation, parce qu'ils ont commis une infidélité, dit le Seigneur Yahweh.
\Chap{16}
\TextTitle{Bonté de Yahweh, prostitutions d'Israël}
\VerseOne{}La parole de Yahweh me fut aussi adressée en ces mots :
\VS{2}Fils de l'homme, fais connaître à Jérusalem ses abominations.
\VS{3}Et dis : Ainsi parle le Seigneur Yahweh à Jérusalem : Tu as tiré ton origine et ta naissance du pays de Canaan ; ton père était Amoréen, et ta mère Héthienne.
\VS{4}Quant à ta naissance, le jour où tu naquis, ton cordon ombilical n'a pas été coupé, tu n'as pas été lavée dans l'eau pour être nettoyée ; tu n'as pas été salée de sel ni emmaillotée.
\VS{5}Il n'y a pas eu d'œil qui ait eu pitié de toi pour te faire une seule de ces choses, en ayant compassion pour toi ; mais tu as été jetée sur la face des champs le jour de ta naissance, parce qu'on avait horreur de toi.
\VS{6}Et passant près de toi, je te vis gisante par terre, dans ton sang, et je te dis : Vis dans ton sang ! Et je te redis encore : Vis dans ton sang !
\VS{7}Je t'ai fait croître par millions comme l'herbe des champs. Et tu pris de l'accroissement et tu devins grande, tu parvins à une parfaite bauté, tes seins se formèrent, ta chevelure poussa, tu devins nubile; mais tu étais abandonnée et sans habits.
\VS{8}Je passai près de toi, je te regardai, et voici, le temps était là, le temps des amours. J'étendis sur toi le pan de ma robe, et je couvris ta nudité. Je te jurai, j'entrai en alliance avec toi, dit le Seigneur Yahweh, et tu devins mienne.
\VS{9}Je te lavai dans l'eau en t'y plongeant, j'ôtai le sang de dessus toi, et je t'oignis d'huile.
\VS{10}Je te revêtis de vêtements brodés, je te chaussai de fourrure, je te ceignis de fin lin, et je te couvris de soie.
\VS{11}Je te parai d'ornements : Je mis des bracelets sur tes mains, et un collier à ton cou.
\VS{12}Je mis un anneau à ton nez, des pendants à tes oreilles, et une couronne de gloire sur ta tête.
\VS{13}Tu fus donc parée d'or et d'argent, et ton vêtement était de fin lin, de soie, et de broderie ; tu mangeas la fleur de farine, le miel, et l'huile ; tu devins extrêmement belle, et tu prospéras jusqu'à régner.
\VS{14}Ta renommée se répandit parmi les nations à cause de ta beauté, car elle était parfaite, à cause de ma gloire que j'avais mise sur toi, dit le Seigneur Yahweh.
\VS{15}Mais tu t'es confiée dans ta beauté, et tu t'es prostituée à cause de ta renommée, tu t'es abandonnée à tous les passants\FTNT{Es. 1:21 ; Jé. 2:20 ; Jé. 3:2-6 ; Os. 1:2.}.
\VS{16}Tu as pris tes vêtements pour t'en faire des hauts lieux de diverses couleurs, tels qu'il n'y en a point eu et n'en aura jamais, et tu t'y es prostituée.
\VS{17}Tu as pris ta magnifique parure d'or et d'argent, que je t'avais donnée, et tu t'en es fait des images d'hommes, tu as commis la fornication avec elles.
\VS{18}Tu as pris tes vêtements brodés, tu les en as couvertes, et tu as mis mon huile et mon encens devant elles.
\VS{19}Mon pain que je t'avais donné, la fleur de farine, l'huile, et le miel que je t'avais donné à manger, tu as mis cela devant elle en sacrifice de bonne odeur ; il a été fait ainsi, dit le Seigneur Yahweh.
\VS{20}Tu as aussi pris tes fils et tes filles que tu m'avais enfantés, et tu les as sacrifiés pour être mangés\FTNT{Lé. 18:21 ; Lé. 20:2 ; Es. 57:5 ; Jé. 19:5, Jé. 32:35.}. N'était-ce pas assez de tes prostitutions ?
\VS{21}Tu as égorgé mes fils, et tu les as livrés pour les faire passer par le feu, en l'honneur de ces idoles\FTNT{2 R. 17:17.}.
\VS{22}Et parmi toutes tes abominations et tes adultères, tu ne t'es point souvenue du temps de ta jeunesse, quand tu étais sans habits et toute nue, gisante par terre dans ton sang.
\VS{23}Après toutes tes méchancetés, malheur, malheur à toi ! dit le Seigneur Yahweh.
\VS{24}Tu t'es bâti un lieu éminent, et tu t'es fait des hauts lieux dans toutes les places.
\VS{25}A l'entrée de chaque chemin tu as bâti un haut lieu, et tu as rendu ta beauté abominable, tu t'es prostituée à tous les passants, tu as multiplié tes adultères.
\VS{26}Tu t'es abandonnée aux fils d'Egypte, tes voisins au corps avantageux ; tu as multiplié tes adultères pour m'irriter.
\VS{27}Et voici, j'ai étendu ma main sur toi, j'ai diminué la portion que je t'avais prescrite, et je t'ai abandonnée à la volonté de celles qui te haïssaient, des filles des Philistins, lesquelles ont honte de tes voies qui ne sont que méchanceté.
\VS{28}Tu t'es aussi abandonnée aux fils des Assyriens\FTNT{2 R. 16:7-10 ; Jé. 2:18-36.}, parce que tu n'étais pas encore rassasiée ; et après avoir commis l'adultère avec eux, tu n'as point encore été rassasiée.
\VS{29}Tu as multiplié tes adultères dans le pays de Canaan jusqu'en Chaldée, et avec cela tu n'as pas encore été rassasiée.
\VS{30}Quelle faiblesse de cœur tu as eue, dit le Seigneur, Yahweh, d'avoir fait toutes ces choses-là, qui sont les actions d'une femme qui se prostitue avec arrogance.
\VS{31}De t'être bâti un lieu éminent à chaque entrée de chemin, et d'avoir fait ton haut lieu dans toutes les places. Et tu n'as pas été comme la prostituée, car tu n'as point tenu compte du salaire.
\VS{32}Femme adultère, tu prends des étrangers au lieu de ton mari.
\VS{33}On donne un salaire à toutes les prostituées, mais toi tu as donné à tous tes amants des présents\FTNT{Es. 57:8-9 ; Os. 8:9-10.}, tu les as gagnés par des présents, afin que de toutes parts ils viennent vers toi, pour se plonger avec toi dans le crime.
\VS{34}Tu as été le contraire des autres prostituées, parce qu'on ne te recherchait pas ; et en donnant un salaire au lieu d'en recevoir un, tu as été le contraire des autres.
\TextTitle{Conséquences de l'infidélité de Jérusalem}
\VS{35}C'est pourquoi, ô adultère, écoute la parole de Yahweh !
\VS{36}Ainsi parle le Seigneur, Yahweh : Parce que ton venin s'est répandu, et que dans tes excès tu t'es abandonnée à ceux que tu aimais, à tes abominables idoles, et que tu as mis à mort tes fils que tu leur as donnés ;
\VS{37}à cause de cela, voici, je vais rassembler tous tes amants, avec lesquels tu te plaisais, et tous ceux que tu as aimés, avec tous ceux que tu as haïs ; je les assemblerai de toutes parts contre toi, et je découvrirai ta honte à leurs yeux et ils verront ton infamie.
\VS{38}Et je te jugerai comme on juge les femmes adultères, et celles qui répandent le sang\FTNT{Lé. 20:10 ; De. 22:22-30.} ; je te livrerai pour être mise à mort selon ma fureur et ma jalousie.
\VS{39}Je te livrerai, dis-je, entre leurs mains ; et ils détruiront tes maisons de prostitution, et ils détruiront tes hauts lieux ; ils te dépouilleront de tes vêtements, emporteront ta magnifique parure, et ils te laisseront sans habits et entièrement nue.
\VS{40}Et on fera monter contre toi une foule de gens qui te lapideront de pierres, et qui te perceront avec leurs épées.
\VS{41}Puis ils brûleront tes maisons, et feront justice de toi aux yeux d'un grand nombre de femmes, je ferai cesser tes prostitutions et tu ne donneras plus de salaires.
\VS{42}J'abandonnerai alors ma colère contre toi, et ma jalousie se retirera de toi ; je serai en repos, et je ne m'irriterai plus.
\VS{43}Parce que tu ne t'es point souvenue du temps de ta jeunesse, et que tu m'as provoqué par toutes ces choses-là ; à cause de cela, voici, j'ai fait tomber la peine de tes crimes sur ta tête, dit le Seigneur,Yahweh ; et tu ne feras plus de mauvais projets avec toutes tes abominations.
\VS{44}Voici, tous ceux qui usent de proverbes feront un proverbe de toi, en disant : Telle mère, telle fille !
\VS{45}Tu es la fille de ta mère, qui a dédaigné son mari et ses fils ; et tu es la sœur de chacune de tes sœurs, qui ont dédaigné leurs maris et leurs fils. Votre mère était Héthienne, et votre père était Amoréen.
\VS{46}Ta grande sœur qui demeure à ta gauche, c'est Samarie avec ses filles ; et ta petite sœur qui demeure à ta droite, c'est Sodome avec ses filles.
\VS{47}Et tu n'as pas seulement marché dans leurs voies et fait selon leurs abominations, c'était fort peu ; mais tu t'es corrompue plus qu'elles dans toutes tes voies.
\VS{48}Je suis vivant ! dit le Seigneur Yahweh, Sodome, ta sœur et tes filles, n'ont point fait comme tu as fait, toi et tes filles.
\VS{49}Voici quel a été le crime de Sodome, ta sœur : Elle avait de l'orgueil, elle vivait dans l'abondance de pain, et dans une insouciante tranquillité, elle et ses filles, elle ne fortifiait pas la main du pauvre et de l'indigent.
\VS{50}Elles se sont élevées, elles ont commis des abominations devant moi, et je me suis détourné quand j'ai vu cela.
\VS{51}Quant à Samarie, elle n'a pas commis la moitié de tes péchés ; car tu as multiplié tes abominations plus qu'elle, et tu as justifié tes sœurs par toutes les abominations que tu as commises.
\VS{52}Porte ta honte, toi qui as jugé chacune de tes sœurs, à cause de tes péchés, par lesquels tu as été rendue plus abominable qu'elles ; elles sont plus justes que toi ; c'est pourquoi sois honteuse, et porte ta confusion, vu que tu as justifié tes sœurs.
\VS{53}Quand je ramènerai leurs captifs, les captifs, dis-je, de Sodome et des villes de son ressort; et les captifs de Samarie et de villes de son ressort; je ramenèrai aussi les captifs de la captivité parmi elles!
\VS{54}afin que tu portes ta honte, et que tu sois confuse à cause de tout ce que tu as fait, et que tu les consoles.
\VS{55}Quand ta sœur Sodome, et les villes de son ressort retourneront à leur état précédent ; Samarie et les villes de son ressort retourneront à leur état précédent ; toi aussi, et les villes de ton ressort retournerez à votre état précédent.
\VS{56}Or ta bouche n'a point fait mention de ta sœur, Sodome, au jour de tes fiertés,
\VS{57}avant que ta méchanceté soit découverte ; lorsque tu as reçu les outrages des filles de Syrie, et de tous ses alentours, des filles des Philistins, qui te pillèrent de tous côtés !
\VS{58}Tu portes sur toi tes méchancetés et tes abominations, dit Yahweh.
\VS{59}Car ainsi parle le Seigneur Yahweh : Je te ferai comme tu as fait, quand tu as méprisé le serment en rompant l'alliance.
\TextTitle{Fidélité de Yahweh à son alliance}
\VS{60}Mais je me souviendrai de l'alliance que j'ai traitée avec toi dans les jours de ta jeunesse, et j'établirai avec toi une alliance éternelle\FTNT{Lé. 26:42-45 ; Ps. 106:45.}.
\VS{61}Et tu te souviendras de tes voies, et en seras confuse, lorsque tu recevras tes sœurs, tant tes plus grandes que tes plus petites, et je te les donnerai pour filles ; mais pas selon ton alliance.
\VS{62}Car j'établirai mon alliance avec toi, et tu sauras que je suis Yahweh,
\VS{63}afin que tu te souviennes de ta vie passée, que tu en sois honteuse, et que tu n'ouvres plus la bouche, à cause de ta confusion, après que j'aurai été apaisé envers toi, pour tout ce que tu auras fait, dit le Seigneur Yahweh.
\Chap{17}
\TextTitle{Enigme de Yahweh}
\VerseOne{}La parole de Yahweh me fut adressée en ces mots :
\VS{2}Fils de l'homme, propose une énigme, une parabole à la maison d'Israël.
\VS{3}Tu diras : Ainsi parle le Seigneur Yahweh : Un grand aigle à grandes ailes, aux ailes déployées, couvert de plumes de toutes les couleurs, vint au Liban, et enleva la cime d'un cèdre.
\VS{4}Il arracha la tête de ses rameaux, l'emmena dans un pays de commerce, et la mit dans une ville marchande.
\VS{5}Il prit de la semence du pays, et la mit dans un champ propre à semer, l'apporta près des grosses eaux, la planta comme un saule.
\VS{6}Cette semence poussa et devint un cep de vigne étendu, mais de peu d'élévation ; ses rameaux étaient tournés vers l'aigle, et ses racines étaient sous lui ; il devint une vigne, donna des jets, et produisit des branches.
\VS{7}Mais il y avait un autre grand aigle, aux longues ailes, et au plumage épais. Et voici, cette vigne serra vers lui ses racines, et étendit ses rameaux vers lui, afin qu'il l'arrose des eaux qui coulent des terrasses.
\VS{8}Elle était donc plantée dans une bonne terre, près des grosses eaux, en sorte qu'il y sortait des sarments et portait du fruit\FTNT{Mt. 13:8-23 ; Mc. 4:8-20 ; Lu. 8:8-15.}. Elle était devenue une vigne magnifique.
\VS{9}Dis : Ainsi parle le Seigneur Yahweh, prospèrera-t-elle ? N'arrachera-t-il pas ses racines, et ne coupera-t-il pas ses fruits pour qu'ils deviennent secs ? Tous les sarments qu'il a jetés sécheront, et il ne faudra pas un grand effort et beaucoup de monde pour l'enlever de dessus ses racines.
\VS{10}Mais voici, quoique plantée, prospèrera-t-elle ? Quand le vent d'orient l'aura touchée, ne séchera-t-elle pas entièrement ? Elle séchera sur le terrain où elle était plantée.
\TextTitle{Jugement de Dieu sur Sédécias\FTNTT{2 R. 24:17-20 ; 25:1-10.}}
\VS{11}Puis la parole de Yahweh me fut adressée en ces mots :
\VS{12}Parle maintenant à la maison rebelle : Ne savez-vous pas ce que veulent dire ces choses ? Dis : Voici, le roi de Babylone est venu à Jérusalem. Il a pris le roi, et les princes, et les a emmenés avec lui à Babylone.
\VS{13}Il a pris un de la race royale, il a traité alliance avec lui, il lui a fait prêter serment, et il a retenu les puissants du pays,
\VS{14}afin que le royaume soit tenu dans l'abaissement, et qu'il ne s'élève point, mais qu'en gardant son alliance, il subsiste.
\VS{15}Mais celui-ci s'est rebellé contre lui, envoyant ses messagers en Egypte, pour qu'on lui donne des chevaux et un grand peuple. Celui qui fait de telles choses prospérera-t-il, échappera-t-il ? Ayant enfreint l'alliance, échappera-t-il ?
\VS{16}Je suis vivant, dit le Seigneur Yahweh, c'est dans le pays du roi qui l'a établi pour roi, envers qui il a violé son serment et dont il a rompu l'alliance, c'est près de lui, au milieu de Babylone, qu'il mourra\FTNT{Référence à Nebucadnetsar. Sédécias eut les yeux crevés avant d'être emmené captif (2 R. 25:7 ; Jé. 34:3 ; Jé. 52:11).}.
\VS{17}Pharaon n'ira pas avec une grande armée et un peuple nombreux pour le secourir dans cette guerre, lorsque l'ennemi élèvera des terrasses et fera des retranchements pour exterminer beaucoup d'âmes.
\VS{18}Car il a méprisé le serment en violant l'alliance ; car voici, après avoir donné sa main, il a fait néanmoins toutes ces choses-là ; il n'échappera point !
\VS{19}C'est pourquoi ainsi parle le Seigneur Yahweh : Je suis vivant, si je ne fais tomber sur sa tête mon serment qu'il a méprisé, et mon alliance qu'il a enfreinte.
\VS{20}Et j'étendrai mon rets sur lui, et il sera pris dans mes filets, je le ferai entrer dans Babylone, et là j'entrerai en jugement contre lui pour le crime qu'il a commis contre moi.
\VS{21}Et tous ses fugitifs avec toutes ses troupes tomberont par l'épée, et ceux qui resteront seront dispersés à tout vent ; et vous saurez que moi, Yahweh, j'ai parlé.
\VS{22}Ainsi parle le Seigneur Yahweh : Je prendrai aussi un rameau de la cime de ce haut cèdre, et je le planterai ; je couperai, dis-je, du bout de ses jeunes branches, un tendre rameau, et je le planterai sur une montagne haute et éminente.
\VS{23}Je le planterai sur la haute montagne d'Israël, et là il produira des branches et produira du fruit, et il deviendra un excellent cèdre ; et des oiseaux de tout plumage demeureront sous lui, et habiteront sous l'ombre de ses branches.
\VS{24}Et tous les bois des champs connaîtront que moi, Yahweh, j'aurai abaissé le grand arbre, et élevé le petit arbre, fait sécher le bois vert, et fait reverdir le bois sec ; moi, Yahweh, j'ai parlé, et je le ferai.
\Chap{18}
\TextTitle{Chacun responsable de son péché}
\VerseOne{}La parole de Yahweh me fut encore adressée, en disant :
\VS{2}Que voulez-vous dire, vous qui usez ordinairement de ce proverbe touchant le pays d'Israël, en disant : Les pères ont mangé des raisins verts et les dents des enfants ont été agacées\FTNT{Jé. 31:29 ; La. 5:7.} ?
\VS{3}Je suis vivant, dit le Seigneur Yahweh et vous n'userez plus de ce proverbe en Israël.
\VS{4}Voici, toutes les âmes sont à moi ; l'âme du fils est à moi comme l'âme du père ; l'âme qui pèche sera celle qui mourra.
\VS{5}Mais l'homme qui est juste, et qui pratique la droiture et la justice,
\VS{6}qui ne mange pas sur les montagnes, et qui ne lève pas ses yeux vers les idoles de la maison d'Israël, et qui ne souille pas la femme de son prochain et ne s'approche pas de la femme dans son état d'impureté\FTNT{Lé 18:18 ; Lé. 20:18.},
\VS{7}qui n'opprime personne, qui rend le gage à son débiteur\FTNT{Ex. 22:26 ; De. 24:12-13.}, qui ne ravit pas le bien d'autrui, qui donne son pain à celui qui a faim et qui couvre d'un vêtement celui qui est nu\FTNT{De. 15:11 ; Es. 58:7.},
\VS{8}qui ne prête pas à intérêt, et ne tire pas d'usure, qui détourne sa main de l'iniquité et qui juge selon la vérité entre les parties qui plaident ensemble\FTNT{Ex. 22:25 ; Lé. 25:35-37 ; De. 23:19.},
\VS{9}qui suit mes lois et garde mes ordonnances pour agir avec fidélité, celui-là est juste, certainement il vivra, dit le Seigneur Yahweh.
\VS{10}Et s'il a engendré un fils qui soit un meurtrier, répandant le sang, et commettant des choses semblables ;
\VS{11}et qui ne fasse aucune de ces choses que j'ai ordonnées, s'il mange sur les montagnes, s'il déshonore la femme de son prochain,
\VS{12}s'il opprime le malheureux et le pauvre, s'il ravit le bien d'autrui, s'il ne rend pas le gage, s'il lève ses yeux vers les idoles et commet des abominations,
\VS{13}s'il prête à intérêt, et tire une usure, ce fils-là, vivrait ? Il ne vivra pas, quand il aura commis toutes ces abominations, on le fera mourir, et son sang retombera sur lui.
\VS{14}Mais s'il engendre un fils qui voie tous les péchés que commet son père, qui les voie et n'agisse pas de la même manière ;
\VS{15}S'il ne mange pas sur les montagnes et qu'il ne lève point ses yeux vers les idoles de la maison d'Israël, s'il ne déshonore pas la femme de son prochain,
\VS{16}s'il n'opprime personne, s'il ne prend point de gages, s'il ne ravit point le bien d'autrui, s'il donne de son pain à celui qui a faim et couvre celui qui est nu,
\VS{17}s'il retire sa main du pauvre, s'il n'exige ni usure ni intérêt, s'il garde mes ordonnances, et s'il suit mes lois ; il ne mourra point pour l'iniquité de son père, mais certainement il vivra.
\VS{18}Mais son père, parce qu'il a usé de fraude, et qu'il a ravi ce qui était à son frère, et fait parmi son peuple ce qui n'est pas bon, voici, il mourra pour son iniquité.
\VS{19}Mais, direz-vous : Pourquoi le fils ne porte-t-il pas l'iniquité de son père\FTNT{Ex. 20:5 ; De. 5:9.} ? Parce que le fils a fait ce qui était juste et droit, et qu'il a gardé toutes mes lois et les a observées, certainement il vivra.
\VS{20}L'âme qui pèche est celle qui mourra. Le fils ne portera point l'iniquité du père, et le père ne portera point l'iniquité du fils. La justice du juste sera sur le juste, et la méchanceté du méchant sera sur le méchant.
\VS{21}Si le méchant se détourne de tous ses péchés qu'il aura commis, et qu'il garde toutes mes lois, et fasse ce qui est juste et droit, certainement il vivra, il ne mourra point.
\VS{22}Il ne lui sera point fait mention de tous ses crimes qu'il aura commis, mais il vivra pour sa justice, à laquelle il se sera adonné.
\VS{23}Ce que je désire, est-ce que le méchant meure ? dit le Seigneur Yahweh. N'est-ce pas qu'il se détourne de ses mauvaises voies et qu'il vive ?
\VS{24}Mais si le juste se détourne de sa justice, et commet l'iniquité, selon toutes les abominations que le méchant a l'habitude de commettre, vivra-t-il ? Il ne sera point fait mention de toutes ses justices qu'il aura faites, à cause de son crime qu'il aura commis, et à cause de son péché qu'il aura fait ; il mourra à cause de ces choses-là.
\VS{25}Et vous, vous dites : La voie du Seigneur n'est pas bien réglée. Ecoutez maintenant maison d'Israël, ma voie n'est-elle pas bien réglée ? Ne sont-ce pas plutôt vos voies qui ne sont pas bien réglées ?
\VS{26}Si le juste se détourne de sa justice, et commet l'iniquité, il mourra à cause de ces choses-là ; il mourra à cause de son iniquité qu'il aura commise.
\VS{27}Si le méchant se détourne de sa méchanceté qu'il aura commise, et pratique ce qui est juste et droit, il fera vivre son âme.
\VS{28}Ayant donc considéré sa conduite, et s'étant détourné de tous ses crimes qu'il aura commis, certainement il vivra, il ne mourra point.
\VS{29}La maison d'Israël dit : La voie du Seigneur Yahweh n'est pas bien réglée. Ô maison d'Israël ! Mes voies ne sont-elles pas bien réglées ? Ne sont-ce pas plutôt vos voies qui ne sont pas bien réglées ?
\VS{30}C'est pourquoi je jugerai chacun de vous selon ses voies, ô maison d'Israël ! dit le Seigneur. Revenez, et détournez-vous de tous vos péchés, et l'iniquité ne vous ruinera pas.
\VS{31}Rejetez loin de vous tous les crimes par lesquels vous avez péché ; et faites-vous un nouveau cœur et un esprit nouveau ; pourquoi mourriez-vous, ô maison d'Israël ?
\VS{32}Car je ne désire pas la mort de celui qui meurt, dit le Seigneur Yahweh. Convertissez-vous donc, et vivez\FTNT{Ac. 3:19-20.}.
\Chap{19}
\TextTitle{Complaintes sur les dirigeants d'Israël}
\VerseOne{}Et toi, prononce à haute voix une complainte touchant les princes d'Israël.
\VS{2}Et dis : Ta mère, qu'était-ce ? C'était une lionne couchée parmi les lions, et qui a élevé ses petits parmi les jeunes lions.
\VS{3}Elle fit croître un de ses petits, qui devint un jeune lion, et qui apprit à déchirer la proie et a dévorer les hommes.
\VS{4}Les nations en entendirent parler, il fut attrapé dans leur fosse ; et elles l'emmenèrent avec des boucles au pays d'Egypte\FTNT{2 R. 23 : 33-34.}.
\VS{5}Puis ayant vu qu'elle attendait en vain, qu'elle était trompée dans son espérance, elle prit un autre de ses petits, et en fit un jeune lion.
\VS{6}Il marcha parmi les lions et devint un jeune lion, il apprit à déchirer la proie et a dévorer les hommes.
\VS{7}Il désola leurs palais, il ravagea leurs villes, de sorte que le pays, et tout ce qui y est, fut épouvanté par le cri de son rugissement.
\VS{8}Les nations s'armèrent contre lui de toutes les provinces, elles étendirent leurs rets contre lui, et il fut attrapé dans leur fosse\FTNT{2 R. 24:2.}.
\VS{9}Puis ils l'enfermèrent et l'enchaînèrent, pour l'amener au roi de Babylone, et le mettre dans une forteresse, afin que sa voix ne soit plus entendue sur les montagnes d'Israël.
\VS{10}Ta mère était comme une vigne dans ton sang plantée auprès des eaux, et elle est devenue chargée de fruits et de rameaux, à cause des grandes eaux.
\VS{11}Elle avait de puissantes branches pour en faire des sceptres de souverains ; son tronc s'était élevé jusqu'à ses branches touffues, et on la voyait dans sa hauteur avec la multitude de ses rameaux.
\VS{12}Mais elle a été arrachée avec fureur, et jetée par terre ; et le vent d'orient a séché son fruit ; ses puissantes branches se sont rompues et ont séché ; le feu les a consumées.
\VS{13}Maintenant elle est plantée dans le désert, dans une terre sèche et aride.
\VS{14}Le feu est sorti de ses branches, et a consumé son fruit ; et il n'y a plus en elle de puissantes branches pour un sceptre de souverain. C'est là une complainte, et cela servira de complainte.
\Chap{20}
\TextTitle{Compassions de Yahweh face aux infidélités d'Israël}
\VerseOne{}Or il arriva la septième année, au dixième jour du cinquième mois, que quelques-uns des anciens d'Israël vinrent pour consulter Yahweh, et s'assirent devant moi.
\VS{2}La parole de Yahweh me fut adressée en ces mots :
\VS{3}Fils de l'homme, parle aux anciens d'Israël, et dis-leur : Ainsi parle le Seigneur Yahweh : Est-ce pour me consulter que vous venez ? Je suis vivant, dit le Seigneur Yahweh, si vous me consultez.
\VS{4}Ne les jugeras-tu pas, ne les jugeras-tu pas, fils de l'homme ? Donne-leur à connaître les abominations de leurs pères.
\VS{5}Et dis-leur : Ainsi parle le Seigneur Yahweh : Le jour où j'ai choisi Israël, j'ai levé ma main vers la postérité de la maison de Jacob, et je me suis fait connaître à eux dans le pays d'Egypte, et j'ai levé ma main vers eux, en disant : Je suis Yahweh, votre Dieu.
\VS{6}En ce jour, j'ai levé ma main vers eux, pour les faire sortir du pays d'Egypte, pour les amener dans un pays que j'avais cherché pour eux, pays où coulent le lait et le miel, et qui est la noblesse de tous les pays\FTNT{Ex. 3:8 ; Ex. 6:7.}.
\VS{7}Alors je leur dis : Que chacun de vous rejette les abominations qui attirent ses regards, et ne vous souillez point par les idoles d'Egypte ! Je suis Yahweh, votre Dieu\FTNT{Jos. 24:14-23.}.
\VS{8}Mais ils se rebellèrent contre moi, et ils ne voulurent point m'écouter. Aucun ne rejeta les abominations qui attiraient ses regards, et ils n'abandonnèrent point les idoles de l'Egypte. Et je dis que je répandrais ma fureur sur eux, que je consumerais ma colère sur eux au milieu du pays d'Egypte.
\VS{9}Mais je les ai tirés hors du pays d'Egypte, je l'ai fait pour l'amour de mon Nom, afin qu'il ne soit point profané aux yeux des nations parmi lesquelles ils se trouvaient, et aux yeux desquelles je m'étais fait connaître à eux, pour les faire sortir du pays d'Egypte.
\VS{10}Je les fit donc sortir du pays d'Egypte, et je les conduisis dans le désert.
\VS{11}Je leur donnai mes lois et leur fis connaître mes ordonnances, que l'homme doit mettre en pratique, afin de vivre par elles\FTNT{Lé. 18:5 ; Ro. 10:5 ; Ga. 3:12.}.
\VS{12}Je leur donnai aussi mes sabbats, pour être un signe entre moi et eux, afin qu'ils sachent que je suis Yahweh qui les sanctifie\FTNT{Ex. 20:8 ; Ex. 31:13.}.
\VS{13}Mais ceux de la maison d'Israël se rebellèrent contre moi dans le désert. Ils ne suivirent point mes lois, et ils rejetèrent mes ordonnances que l'homme doit mettre en pratique, afin de vivre par elles, et ils profanèrent à l'excès mes sabbats. C'est pourquoi je dis que je répandrais sur eux ma fureur dans le désert pour les consumer\FTNT{Ex. 16:28.}.
\VS{14}Je l'ai fait pour l'amour de mon Nom, afin qu'il ne soit point profané devant les nations, en présence desquelles je les avais fait sortir d'Egypte\FTNT{Ex. 32:12 ; No. 14:13-14 ; De. 9:28 ; Jos. 7:9.}.
\VS{15}Je levai ma main vers eux dans le désert pour ne pas les amener dans le pays que je leur avais donné, pays où coulent le lait et le miel, et qui est la noblesse de tous les pays,
\VS{16}parce qu'ils ont rejeté mes ordonnances, qu'ils n'ont point suivi mes lois, et qu'ils ont profané mes sabbats, car leur cœur ne s'est pas éloigné de leurs idoles.
\VS{17}Toutefois, j'eus pour eux un regard de pitié pour ne point les détruire, et je ne les consumai point entièrement dans le désert.
\VS{18}Mais je dis à leurs fils dans le désert : Ne marchez point dans les statuts de vos pères, et ne gardez point leurs ordonnances, et ne vous souillez point par leurs idoles.
\VS{19}Je suis Yahweh, votre Dieu. Marchez dans mes statuts, et gardez mes ordonnances et accomplissez-les.
\VS{20}Sanctifiez mes sabbats, et ils seront un signe entre moi et vous, afin que vous reconnaissiez que je suis Yahweh, votre Dieu.
\VS{21}Mais les fils se rebellèrent aussi contre moi, et ils ne marchèrent point dans mes statuts, et ne gardèrent point mes ordonnances pour les faire ; ce que l'homme doit accomplir, pour vivre par elles. Ils profanèrent mes sabbats ; c'est pourquoi je dis que je répandrais ma fureur sur eux, et que je consumerais ma colère sur eux dans le désert.
\VS{22}Toutefois, je retirai ma main, et je le fis pour l'amour de mon Nom, afin qu'il ne soit point profané devant les nations, en présence desquelles je les avais sortis d'Egypte.
\VS{23}Néanmoins, je levai ma main vers eux dans le désert, pour les répandre parmi les nations, et les disperser dans les pays\FTNT{Lé. 26:13-33.},
\VS{24}parce qu'ils n'ont point accompli mes ordonnances, et qu'ils ont rejeté mes statuts, profané mes sabbats, et que leurs yeux se sont attachés aux idoles de leurs pères.
\VS{25}A cause de cela, je leur donnai des statuts qui n'étaient pas bons, et des ordonnances par lesquelles ils ne vivraient point.
\VS{26}Je les souillai par leurs dons, quand ils firent passer par le feu tous les premiers-nés, afin de les punir, et que l'on sache que je suis Yahweh.
\VS{27}C'est pourquoi, toi fils de l'homme, parle à la maison d'Israël, et dis-leur : Ainsi parle le Seigneur Yahweh : Vos pères m'ont encore outragé, car ils ont commis un crime contre moi.
\VS{28}Je les ai conduits dans le pays que j'avais juré de leur donner, et ils ont regardé toute haute colline, et tout arbre touffu, ils y ont fait leurs sacrifices, ils y ont posé leur oblation pour m'irriter, ils y ont mis leurs parfums, et ils y ont répandu leurs aspersions.
\VS{29}Je leur ai dit : Que veulent dire ces hauts lieux où vous allez ? Et le nom de hauts lieux leur a été donné jusqu'à ce jour.
\VS{30}C'est pourquoi dis à la maison d'Israël : Ainsi parle le Seigneur Yahweh : Ne vous souillez-vous pas selon les voies de vos pères, et ne vous prostituez-vous point à leurs idoles abominables,
\VS{31}en offrant vos dons, en faisant passer vos fils par le feu, en vous souillant par toutes vos idoles jusqu'à ce jour ? Est-ce ainsi que vous me consultez, ô maison d'Israël ? Je suis vivant, dit le Seigneur Yahweh, vous ne me consultez point.
\VS{32}Ce que vous pensez n'arrivera nullement, quand vous dites : Nous serons comme les nations, et comme les familles des pays, en servant le bois et la pierre.
\TextTitle{Restauration future d'Israël}
\VS{33}Je suis vivant ! dit le Seigneur Yahweh. Je règnerai sur vous avec une main forte, et un bras étendu, et avec effusion de colère.
\VS{34}Je vous sortirai du milieu des peuples, et vous rassemblerai hors des pays dans lesquels vous êtes dispersés, avec une main forte, et un bras étendu et avec effusion de colère.
\VS{35}Je vous ferai venir dans le désert des peuples, et je contesterai là contre vous, face à face,
\VS{36}comme j'ai contesté contre vos pères dans le désert du pays d'Egypte, ainsi je contesterai contre vous, dit le Seigneur Yahweh.
\VS{37}Je vous ferai passer sous la verge, et vous ramènerai au lieu de l'alliance\FTNT{Es. 65:12.}.
\VS{38}Je séparerai de vous les rebelles, et ceux qui se révoltent contre moi ; je les ferai sortir du pays dans lequel ils séjournent, mais ils n'entreront point dans la terre d'Israël ; et vous saurez que je suis Yahweh.
\VS{39}Vous donc, ô maison d'Israël, ainsi parle le Seigneur Yahweh : Allez, servez chacun vos idoles, puisque vous ne voulez pas m'écouter ! Ainsi vous ne profanerez plus mon saint Nom par vos dons et par vos idoles.
\VS{40}Mais ce sera sur ma sainte montagne, sur la haute montagne d'Israël, dit le Seigneur Yahweh, que toute la maison d'Israël me servira, dans le pays\FTNT{Jn. 4:21-24.}. Là, je prendrai plaisir en eux, et là je demanderai vos offrandes et les prémices de vos dons, et tout ce que vous me consacrerez.
\VS{41}Je prendrai plaisir en vous par vos parfums d'une agréable odeur, quand je vous aurai fait sortir du milieu des peuples, et que je vous aurai rassemblés des pays dans lesquels vous êtes dispersés ; je serai sanctifié par vous, aux yeux des nations.
\VS{42}Vous saurez que je suis Yahweh, quand je vous aurai fait revenir dans le pays d'Israël, dans le pays où j'ai levé ma main pour le donner à vos pères.
\VS{43}Et là, vous vous souviendrez de vos voies, et de toutes vos actions, par lesquelles vous vous êtes souillés ; et vous vous prendrez vous-mêmes en dégoût à cause de tous vos maux que vous aurez faits.
\VS{44}Vous saurez que je suis Yahweh, par tout ce que j'aurai fait pour vous, à cause de mon Nom, et non pas selon vos méchantes voies et vos actions corrompues, ô maison d'Israël ! dit le Seigneur Yahweh.
\Chap{21}
\TextTitle{L'épée de Yahweh}
\VerseOne{}La parole de Yahweh me fut encore adressée en ces mots :
\VS{2}Fils de l'homme, tourne ta face vers Jérusalem, parle en direction du sud, et prophétise contre la forêt du champ du sud.
\VS{3}Dis à la forêt du sud : Ecoute la parole de Yahweh. Ainsi parle le Seigneur Yahweh : Voici, je m'en vais allumer au dedans de toi un feu qui consumera tout bois vert et tout bois sec au dedans de toi ; la flamme de l'embrasement ne s'éteindra point, et tout le dessus en sera brûlé, depuis le sud jusqu'au nord\FTNT{Jé. 21:14 ; Jé. 22:7 ; Jé. 46:23 ; Lu. 23:31.}.
\VS{4}Toute chair verra que moi, Yahweh, j'ai allumé le feu ; et il ne s'éteindra point.
\VS{5}Je dis : Ah ! Seigneur Yahweh, ils disent de moi : N'est-il pas vrai que celui-ci ne fait que mettre en avant des similitudes ?
\TextTitle{Parabole de l'épée de Yahweh}
\VS{6}La parole de Yahweh me fut adressée en ces mots :
\VS{7}Fils de l'homme, tourne ta face vers Jérusalem, et parle en direction du lieu saint, et prophétise contre la terre d'Israël.
\VS{8}Dis à la terre d'Israël : Ainsi parle Yahweh : Voici, j'en veux à toi, je tirerai mon épée de son fourreau, et je retrancherai du milieu de toi le juste et le méchant.
\VS{9}Parce que je retrancherai du milieu de toi le juste et le méchant, à cause de cela mon épée sortira de son fourreau contre toute chair, depuis le sud jusqu'au nord.
\VS{10}Toute chair saura que moi, Yahweh, j'ai tiré mon épée de son fourreau, et elle n'y retournera plus.
\VS{11}Aussi, toi, fils de l'homme, gémis en te rompant les reins de douleur, et soupire avec amertume dans leur présence.
\VS{12}Quand ils te diront : Pourquoi gémis-tu ? Alors tu répondras : C'est à cause d'une nouvelle, car elle vient, et tout cœur se fondra, et toutes les mains seront baissées, tout esprit sera affaibli, et tous les genoux se fondront en eau ; voici, elle vient, elle arrive, dit le Seigneur Yahweh\FTNT{Jé. 6:24 ; Jé. 49:23.}.
\VS{13}Puis la parole de Yahweh me fut adressée en ces mots :
\VS{14}Fils de l'homme, prophétise, et dis : Ainsi parle Yahweh : Dis : L'épée ! L'épée a été aiguisée, elle est polie !
\VS{15}Elle a été aiguisée pour faire un grand carnage, elle a été polie afin qu'elle brille… Nous réjouirons-nous ? C'est la verge de mon fils, elle dédaigne tout bois.
\VS{16}Yahweh l'a donnée à polir, afin qu'on la tienne à la main ; l'épée a été aiguisée, et elle a été polie pour la mettre dans la main du destructeur.
\VS{17}Crie et hurle, fils de l'homme, car elle est contre mon peuple, elle est contre tous les princes d'Israël ; ils sont livrés à l'épée à cause de mon peuple. C'est pourquoi frappe sur ta cuisse !
\VS{18}Oui l'épreuve sera faite ; et que sera-ce si ce sceptre qui méprise tout est anéanti ? dit le Seigneur Yahweh.
\VS{19}Toi donc, fils de l'homme, prophétise, et frappe d'une main contre l'autre, que les coups de l'épée soient doublés, soient triplés, c'est l'épée du carnage, l'épée du grand carnage, l'épée qui doit les poursuivre.
\VS{20}J'ai mis à toutes leurs portes l'épée étincelante, afin que le cœur se fonde, et que les ruines soient multipliées. Ah ! Elle est faite pour briller et réservée pour tuer.
\VS{21}Joins-toi épée, frappe à la droite ! Avance-toi, frappe à la gauche, à tous côtés que tu rencontres !
\VS{22}Je frapperai aussi d'une main contre l'autre, je donnerai du repos à ma colère. Moi, Yahweh, j'ai parlé.
\VS{23}La parole de Yahweh me fut adressée en ces mots :
\VS{24}Toi, fils de l'homme, pose deux chemins où l'épée du roi de Babylone pourrait venir ; que les deux chemins sortent d'un même pays, et forme-les, forme-les de ta main à l'endroit où commence le chemin de la ville.
\VS{25}Tu poseras le chemin par lequel l'épée doit venir contre Rabbath des fils d'Ammon, et le chemin qui va en Judée, et à Jérusalem, ville forte.
\VS{26}Car le roi de Babylone se tient au carrefour, à l'entrée des deux chemins, pour consulter les devins ; il aiguise les flèches, il interroge les théraphim, il examine le foie.
\VS{27}Dans sa main droite est la divination contre Jérusalem, pour y dresser des béliers, pour publier le carnage, pour pousser des cris de guerre, pour ranger les béliers contre les portes, pour élever des terrasses et construire des remparts.
\VS{28}Mais ce sera pour eux, à leurs yeux, une divination vaine ; il y a de grands serments entre eux. Mais lui, il se souvient de leur iniquité, en sorte qu'ils seront pris.
\VS{29}C'est pourquoi ainsi parle le Seigneur Yahweh : Parce que vous avez fait revenir le souvenir de votre iniquité, lorsque vos crimes se sont découverts, au point de voir vos péchés dans toutes vos actions ; parce que vous avez fait qu'on se souvienne de vous, vous serez saisis par sa main.
\TextTitle{Quand l'iniquité arrive à son terme\FTNTT{Ap. 19:11-20:6.}}
\VS{30}Et toi, profane, méchant, prince d'Israël, dont le jour arrive au temps où l'iniquité est à son terme !
\VS{31}Ainsi parle le Seigneur Yahweh : Qu'on ôte cette tiare, et qu'on enlève cette couronne. Ce ne sera plus celle-ci ; j'élèverai ce qui est bas, et j'abaisserai ce qui est haut\FTNT{Job. 5:11 ; 1 Co. 1:27.}.
\VS{32}J'en ferai une ruine, une ruine, une ruine, et elle ne sera plus. Mais cela n'aura lieu qu'à la venue de celui à qui appartient le jugement et à qui je le lui donnerai.
\VS{33}Toi, fils de l'homme, prophétise, et dis : Ainsi parle le Seigneur Yahweh, au sujet des fils d'Ammon, et de leur opprobre : Dis donc, épée, épée dégainée, polie pour le massacre, pour dévorer avec son éclat !
\VS{34}Au milieu de tes visions vaines et de tes oracles menteurs, elle te fera tomber parmi les cadavres des méchants, dont le jour arrive au temps où l'iniquité est à son terme.
\VS{35}La remettrait-on dans son fourreau ? Je te jugerai sur le lieu où tu as été créé, au pays de ta naissance.
\VS{36}Je répandrai ma colère sur toi, j'allumerai sur toi le feu de ma fureur, et je te livrerai entre les mains d'hommes brutaux, qui ne travaillent qu'à détruire\FTNT{Jé. 25:11 ; Jé. 52:30.}.
\VS{37}Tu seras destiné au feu pour être dévoré ; ton sang sera au milieu de la terre : On ne se souviendra plus de toi, car c'est moi, Yahweh, qui parle.
\Chap{22}
\TextTitle{Les péchés d'Israël}
\VerseOne{}La parole de Yahweh me fut encore adressée en ces mots :
\VS{2}Et toi, fils de l'homme, ne jugeras-tu pas, ne jugeras-tu pas la ville sanguinaire, et ne lui donneras-tu pas à connaître toutes ses abominations\FTNT{Na. 3:1-4 ; Ha. 1:13 ; Ez. 24:6-9.} ?
\VS{3}Tu diras donc, ainsi parle le Seigneur Yahweh : Ville qui répands le sang au milieu de toi, afin que ton temps vienne, et qui as fait des idoles à ton préjudice, pour en être souillée.
\VS{4}Tu t'es rendue coupable par ton sang que tu as répandu, et tu t'es souillée par tes idoles que tu as faites ; tu as fait approcher tes jours, et tu es venue au terme de tes années ; c'est pourquoi je t'ai exposée en opprobre aux nations, et en dérision dans tous les pays\FTNT{2 R. 21:16 ; Jé. 26:21-23.}.
\VS{5}Celles qui sont près de toi, et celles qui en sont loin, se moqueront de toi, infâme de réputation, et remplie de troubles.
\VS{6}Voici, les princes d'Israël ont contribué au dedans de toi, chacun selon sa force, à répandre le sang.
\VS{7}Au dedans de toi, on méprise père et mère, on use de tromperie à l'égard de l'étranger, on opprime l'orphelin et la veuve.
\VS{8}Tu méprises ma sainteté, et profanes mes sabbats.
\VS{9}Des gens médisants sont au milieu de toi pour répandre le sang, ceux qui sont chez toi mangent sur les montagnes ; on commet des actions énormes au milieu de toi\FTNT{Es. 57:7 ; Jé. 2:20.}.
\VS{10}L'enfant découvre la nudité du père au milieu de toi, et on humilie au milieu de toi la femme dans le temps de son impureté\FTNT{Lé. 18:6-9 ; Ge. 9:22-23.}.
\VS{11}L'un commet l'abomination avec la femme de son prochain ; et l'autre se souille par l'inceste avec sa belle-fille ; chacun humilie sa sœur, fille de son père\FTNT{Ge. 19:32-36 ; Lé. 18:15-20 ; Jé. 5:8.}.
\VS{12}Chez toi, on reçoit des présents pour répandre le sang ; tu exiges un intérêt et une usure, tu dépouilles ton prochain par l'extorsion, et tu m'oublies, dit le Seigneur Yahweh\FTNT{Ex. 23:8 ; De. 27:25.}.
\VS{13}Voici, je frappe de mes mains l'une contre l'autre à cause de ton gain déshonnête que tu fais, et à cause de ton sang qui se répand au milieu de toi.
\VS{14}Ton cœur pourra-t-il tenir ferme, tes mains seront-elles fortes dans les jours où j'agirai contre toi ? Moi, Yahweh, j'ai parlé, et je le ferai.
\VS{15}Je te disperserai parmi les nations, je t'éparpillerai en divers pays, et je consumerai ta souillure, jusqu'à ce qu'il n'y en ait plus en toi.
\VS{16}Tu seras souillée par toi-même aux yeux des nations, et tu sauras que je suis Yahweh.
\TextTitle{La fureur de Yahweh}
\VS{17}Puis la parole de Yahweh me fut adressée en ces mots :
\VS{18}Fils de l'homme, la maison d'Israël m'est devenue comme de l'écume ; eux tous sont de l'airain, de l'étain, du fer et du plomb dans un creuset ; ils sont devenus comme une écume d'argent.
\VS{19}C'est pourquoi ainsi parle le Seigneur Yahweh : Parce que vous êtes tous devenus comme de l'écume, voici, je vais à cause de cela vous rassembler au milieu de Jérusalem,
\VS{20}comme on assemble de l'argent, de l'airain, du fer, du plomb, et de l'étain dans un creuset, afin d'y souffler le feu pour les fondre ; je vous rassemblerai ainsi dans ma colère et dans ma fureur, et je vous fondrai.
\VS{21}Je vous assemblerai, je soufflerai contre vous le feu de ma fureur, et vous serez fondus au milieu de Jérusalem.
\VS{22}Comme l'argent se fond dans le creuset, ainsi vous serez fondus au milieu d'elle, et vous saurez que moi,Yahweh, j'ai répandu ma fureur sur vous.
\TextTitle{Se tenir à la brèche devant Yahweh}
\VS{23}La parole de Yahweh me fut encore adressée en ces mots :
\VS{24}Fils de l'homme, dis-lui : Tu es une terre qui n'est pas purifiée ni arrosée de pluie au jour de la colère.
\VS{25}Il y a un complot de ses prophètes au milieu d'elle ; ils seront comme des lions rugissants, qui ravissent la proie : Ils dévorent les âmes, ils emportent les richesses et la gloire, ils multiplient les veuves au milieu d'elle\FTNT{Mt. 23:13 ; 1 Pi. 5:8.}.
\VS{26}Ses sacrificateurs ont fait violence à ma loi et ont profané mes choses saintes ; ils ne font pas de différence entre la chose sainte et profane; ils ne donnent pas à connaître la diffrence qu'il y a entre la chose impure et la pure, et ils cachent leurs yeux de mes sabbats, et je suis profané au milieu d'eux.
\VS{27}Ses princes sont au milieu d'elle comme des loups qui ravissent la proie, pour répandre le sang et pour détruire les âmes, pour s'adonner au gain déshonnête\FTNT{2 Pi. 2:16 ; Mt. 10:16 ; Mi. 3:11.}.
\VS{28}Ses prophètes ont pour eux des enduits de plâtre, des visions fausses, et des oracles menteurs, en disant : Ainsi parle le Seigneur Yahweh ; et cependant Yahweh n'a point parlé.
\VS{29}Le peuple du pays use de tromperies, ravit le bien d'autrui, opprime l'affligé et le pauvre, et foule l'étranger contre tout droit.
\VS{30}Je cherche parmi eux un homme\FTNT{Dieu n'a pas besoin d'une foule de gens avant d'agir. Une seule personne suffit. } qui élève un mur, et qui se tient à la brèche devant moi pour le pays, afin que je ne le détruise point ; mais je n'en trouve pas.
\VS{31}C'est pourquoi je répandrai sur eux ma colère, et je les consumerai par le feu de ma fureur ; je mettrai leur voie sur leur tête, dit le Seigneur Yahweh.
\Chap{23}
\TextTitle{Prostitutions d'Israël et de Juda}
\VerseOne{}La parole de Yahweh me fut encore adressée en ces mots :
\VS{2}Fils de l'homme, il y a eu deux femmes, filles d'une même mère,
\VS{3}qui se sont prostituées en Egypte, elles se sont prostituées dans leur jeunesse. Là, leur sein fut déshonoré et leur virginité touchée.
\VS{4}Et c'était ici leurs noms, celui de la plus grande était Ohola, et celui de sa sœur Oholiba\FTNT{Ohola signifie « sa propre tente », et Oholiba « la femme de la tente ». 2 R. 17:23-24.} ; elles étaient à moi, et elles ont enfanté des fils et des filles ; leurs noms donc étaient Ohola, qui était Samarie, et Oholiba, qui est Jérusalem.
\VS{5}Or Ohola a commis adultère étant ma femme, et s'est rendue amoureuse de ses amoureux, c'est-à-dire, des Assyriens ses voisins,
\VS{6}vêtus de pourpre, gouverneurs et magistrats, tous jeunes et aimables, tous cavaliers, montés sur des chevaux.
\VS{7}Elle a commis ses adultères avec toute l'élite des fils des Assyriens, et avec tous ceux pour qui elle s'était enflammée, et s'est souillée avec toutes leurs idoles.
\VS{8}Elle n'a pas abandonné ses fornications d'Egypte, car ils avaient couché avec elle dans sa jeunesse, ils avaient déshonoré sa virginité et s'étaient livrés à l'impureté avec elle\FTNT{Ac. 7:42.}.
\VS{9}C'est pourquoi je l'ai livrée entre les mains de ses amoureux, entre les mains des fils des Assyriens, dont elle s'était rendue amoureuse.
\VS{10}Ils l'ont couverte d'opprobre, ils ont enlevé ses fils et ses filles, et l'ont tuée elle-même avec l'épée ; elle a été en renom parmi les femmes, après avoir exercé des jugements sur elle.
\VS{11}Quand sa sœur Oholiba a vu cela, et fut plus déréglée qu'elle dans ses passions ; ses prostitutions dépassèrent celles de sa sœur.
\VS{12}Elle s'enflamma pour les fils des Assyriens, des gouverneurs et des magistrats, ses voisins, vêtus magnifiquement, et des cavaliers montés sur des chevaux, tous jeunes et bien faits.
\VS{13}J'ai vu qu'elle s'était souillée, et que l'une et l'autre avaient suivi la même voie.
\VS{14}Elle alla même plus loin dans ses prostitutions. Elle aperçut contre des murailles des peintures d'hommes, des images de Chaldéens peints en couleur rouge,
\VS{15}avec des ceintures autour des reins, avec des turbans de couleurs variées flottant sur la tête. Tous ayant l'apparence de princes, et figurant des fils de Babylone.
\VS{16}Elle s'enflamma pour eux au premier regard, et leur envoya des messagers en Chaldée.
\VS{17}Les fils de Babylone vinrent vers elle au lit de ses prostitutions, et la souillèrent par leurs adultères ; elle s'est aussi souillée avec eux, et après cela son cœur s'est détaché d'eux.
\VS{18}Elle a manifesté ses fornications et fait connaître son opprobre ; et mon cœur s'est détaché d'elle, comme mon cœur s'était détaché de sa sœur.
\VS{19}Car elle a multiplié ses adultères, jusqu'à rappeler le souvenir des jours de sa jeunesse, lorsqu'elle s'était abandonnée au pays d'Egypte.
\VS{20}Elle s'est enflammée pour des impudiques dont la chair était comme celle des ânes, et dont la force égale celle des chevaux.
\VS{21}Tu as donc repris les méchancetés de ta jeunesse, lorsque tu as été déshonorée, depuis que tu étais en Egypte, à cause du sein de ta jeunesse.
\VS{22}C'est pourquoi, Oholiba, ainsi parle le Seigneur Yahweh : Voici, je m'en vais réveiller contre toi tous tes amants, ceux dont ton cœur s'est détaché, et je les amènerai contre toi de toutes parts.
\VS{23}Les fils de Babylone, et tous les Chaldéens, Pekod, Shoa, Koa, et tous les Assyriens avec eux, tous jeunes gens d'élite, gouverneurs et magistrats, grands seigneurs et renommés, tous montant à cheval.
\VS{24}Ils viendront contre toi avec des armes, des chars, et des roues, avec une multitude de peuples, avec le grand bouclier et le petit bouclier, avec les casques, et je leur mettrai le jugement en main, ils te jugeront selon leur jugement.
\VS{25}Je mettrai ma jalousie contre toi, et ils agiront contre toi avec fureur ; ils te retrancheront le nez et les oreilles, et ce qui restera de toi tombera par l'épée. Ils enlèveront tes fils et tes filles, et ce qui restera de toi sera dévoré par le feu.
\VS{26}Ils te dépouilleront de tes vêtements, et t'enlèveront les ornements dont tu te pares.
\VS{27}Je mettrai un terme à tes méchancetés et tes prostitutions du pays d'Egypte ; tu ne lèveras plus tes yeux vers eux, et tu ne te souviendras plus de l'Egypte.
\VS{28}Car ainsi parle le Seigneur Yahweh : Voici, je te livre entre les mains de ceux que tu hais, entre les mains de ceux dont ton cœur s'est détaché.
\VS{29}Ils te traiteront avec haine ; ils enlèveront tout ton travail, et te laisseront sans habits et découverte ; et la turpitude de tes adultères, de ton énormité, et de tes fornications, sera découverte.
\VS{30}On te fera ces choses-là parce que tu t'es prostituée aux nations, avec lesquelles tu t'es souillée par leurs idoles.
\VS{31}Tu as marché dans la voie de ta sœur, c'est pourquoi je mets sa coupe dans ta main.
\VS{32}Ainsi parle le Seigneur Yahweh : Tu boiras la coupe profonde et large de ta sœur ; elle sera une coupe d'une grande mesure ; tu seras un sujet de risée et de moquerie\FTNT{Ps. 75:9 ; Es. 51:17 ; Jé. 25:15.}.
\VS{33}Tu seras remplie d'ivresse et de douleur, par la coupe de désolation et de dégât, qui est la coupe de ta sœur Samarie.
\VS{34}Tu la boiras et la videras, tu briseras ce pot de terre et tu déchireras ton sein. Car j'ai parlé, dit le Seigneur Yahweh.
\VS{35}C'est pourquoi ainsi parle le Seigneur Yahweh : Parce que tu m'as oublié, et que tu m'as jeté derrière ton dos, aussi porteras-tu la peine de ta méchanceté, et de tes prostitutions.
\TextTitle{Jugement sur Israël et Juda}
\VS{36}Puis Yahweh me dit : Fils de l'homme, ne jugeras-tu pas Ohola et Oholiba ? Déclare-leur donc leurs abominations.
\VS{37}Déclare-leur comment elles ont commis l'adultère et comment il y a du sang dans leurs mains ; comment, dis-je, elles ont commis l'adultère avec leurs idoles, et ont même fait passer par le feu leurs fils pour les consumer, ces enfants qu'elles m'avaient enfantés.
\VS{38}Voici encore ce qu'elles m'ont fait : Elles ont souillé mon lieu saint ce même jour, et ont profané mes sabbats.
\VS{39}Car après avoir égorgé leurs fils à leurs idoles, elles sont entrées ce même jour-là dans mon lieu saint pour le profaner ; et voilà, comment elles ont fait au milieu de ma maison\FTNT{2 R. 21:4.}.
\VS{40}Et qui plus est, elles ont fait chercher des hommes venant de loin, elles leur ont envoyé des messagers, et voici, ils sont venus. Pour eux tu t'es lavée, tu as fardé ton visage, et t'es parée d'ornements.
\VS{41}Tu t'es assise sur un lit magnifique, devant lequel a été apprêtée une table, sur laquelle tu as mis mon encens et mon huile.
\VS{42}On entendait le bruit d'une multitude tranquille ; et parmi cette foule d'hommes, on a fait venir du désert des Sabéens, qui ont mis des bracelets aux mains des deux sœurs, et de superbes couronnes sur leurs têtes.
\VS{43}J'ai dit au sujet de celle qui avait vieilli dans l'adultère : Maintenant ses impudicités prendront fin, et elle aussi.
\VS{44}Toutefois on est venu vers elle comme on vient vers une femme prostituée ; ils sont ainsi venus vers Ohola, et vers Oholiba, femmes pleines de méchanceté.
\VS{45}Les hommes justes donc les jugeront comme on juge les femmes adultères, et comme on juge celles qui répandent le sang ; car elles sont adultères, et le sang est dans leurs mains.
\VS{46}C'est pourquoi ainsi parle le Seigneur Yahweh : Qu'on fasse monter l'assemblée contre elles, et qu'elles soient abandonnées au tumulte et au pillage.
\VS{47}Que l'assemblée les lapide de pierres et les taille en pièces avec leurs épées ; qu'ils tuent leurs fils et leurs filles, et qu'ils brûlent au feu leurs maisons.
\VS{48}Et ainsi je ferai cesser la méchanceté dans le pays, et toutes les femmes seront enseignées à ne point faire selon votre méchanceté.
\VS{49}On mettra votre méchanceté sur vous, et vous porterez les péchés de vos idoles ; et vous saurez que je suis le Seigneur Yahweh.
\Chap{24}
\TextTitle{Malheur à la ville sanguinaire}
\VerseOne{}La neuvième année, au dixième jour du dixième mois, la parole de Yahweh me fut adressée en ces mots :
\VS{2}Fils de l'homme, mets par écrit la date de ce jour, de ce jour-ci ! Car en ce même jour le roi de Babylone s'approche contre Jérusalem\FTNT{2 R. 25:1.}.
\VS{3}Propose une parabole à la famille de rebelles, et dis-leur : Ainsi parle le Seigneur Yahweh : Mets, mets la chaudière, et verse de l'eau dedans.
\VS{4}Mets-y les morceaux, tous les bons morceaux, la cuisse, et l'épaule, et remplis-la des meilleurs os.
\VS{5}Prends la meilleure bête du troupeau, et fais brûler des os sous la chaudière, fais-la bouillir à gros bouillons, et que les os cuisent au-dedans.
\VS{6}Car ainsi parle le Seigneur Yahweh : Malheur à la ville sanguinaire, à la chaudière pleine de rouille, et de laquelle la rouille n'est point sortie ! Vide-la morceau par morceau, et que le sort ne soit point jeté sur elle.
\VS{7}Parce que son sang est au milieu d'elle, qu'elle l'a mis sur le rocher brillant, et qu'elle ne l'a point répandu sur la terre pour le couvrir de poussière,
\VS{8}j'ai mis son sang sur un rocher brillant, afin qu'il ne soit point couvert, pour faire monter la fureur, et pour me venger.
\VS{9}C'est pourquoi ainsi parle le Seigneur Yahweh : Malheur à la ville sanguinaire ! J'en ferai aussi un grand tas de bois à brûler !
\VS{10}Amasse beaucoup de bois, allume le feu, fais cuire la chair entièrement, et fais-la consumer, et que les os soient brûlés.
\VS{11}Puis mets sur les charbons ardents la chaudière toute vide, afin qu'elle s'échauffe et que son airain se brûle, et que sa souillure soit fondue à l'intérieur, et que sa rouille soit consumée.
\VS{12}Les efforts sont inutiles, sa rouille dont elle est pleine n'est point sortie d'elle ; sa rouille ne s'en ira que par le feu.
\VS{13}L'impureté est dans ta souillure ; car je t'avais purifiée, et tu n'as point été pure ; tu ne seras pas encore nettoyée de ta souillure, jusqu'à ce que j'aie assouvi sur toi ma fureur.
\VS{14}Moi, Yahweh, j'ai parlé, cela arrivera, et je le ferai ; et je ne me retirerai point en arrière, je n'épargnerai point, et je ne serai point apaisé. On t'a jugée selon tes voies et selon tes actions, dit le Seigneur Yahweh.
\TextTitle{La vie d'Ezéchiel, un signe pour Israël}
\VS{15}La parole de Yahweh me fut adressée en ces mots :
\VS{16}Fils de l'homme, voici, je vais t'ôter par une plaie ce que tes yeux voient avec le plus de plaisir. Ne mène point de deuil, ne pleure point, ne fais point couler tes larmes\FTNT{Jé. 16:6-7.}.
\VS{17}Garde-toi de gémir, et ne fais pas le deuil des morts ; attache ton turban sur ta tête, mets tes souliers à tes pieds, ne te couvre pas la barbe, et ne mange pas le pain des autres\FTNT{Lé. 10:6.}.
\VS{18}Je parlai au peuple le matin, et ma femme mourut le soir ; le lendemain matin je fis comme il m'avait été ordonné.
\VS{19}Le peuple me dit : Ne nous déclareras-tu point ce que nous signifient ces choses-là que tu fais ?
\VS{20}Je leur répondis : La parole de Yahweh m'a été adressée en ces mots :
\VS{21}Parle à la maison d'Israël : Ainsi parle le Seigneur Yahweh : Voici, je m'en vais profaner mon lieu saint, la magnificence de votre force, ce qui est le plus agréable à vos yeux, ce que vous voudriez épargner sur toutes choses ; et vos fils et vos filles, que vous aurez laissés, tomberont par l'épée.
\VS{22}Vous ferez alors comme j'ai fait ; vous ne couvrirez point vos barbes, et vous ne mangerez point le pain des autres.
\VS{23}Vos turbans seront sur vos têtes, et vos souliers à vos pieds ; vous ne mènerez point de deuil ni ne pleurerez ; mais vous pourrirez à cause de vos iniquités, et vous gémirez les uns avec les autres.
\VS{24}Ezéchiel sera pour vous un signe ; vous ferez selon toutes les choses qu'il a faites ; et quand cela sera arrivé, vous saurez que je suis le Seigneur Yahweh.
\VS{25}Quant à toi, fils de l'homme, au jour que je leur ôterai leur force, la joie de leur ornement, l'objet le plus agréable à leurs yeux, et l'objet de leurs cœurs, leurs fils et leurs filles,
\VS{26}ce jour-là un fuyard ne viendra-t-il pas vers toi pour te le raconter ?
\VS{27}En ce jour-là ta bouche sera ouverte envers celui qui sera échappé, et tu parleras, et ne seras plus muet ; ainsi tu seras pour eux un signe, et ils sauront que je suis Yahweh.
\Chap{25}
\TextTitle{Jugement de Dieu sur Ammon}
\VerseOne{}Puis la parole de Yahweh me fut adressée en ces mots :
\VS{2}Fils de l'homme, tourne ta face vers les fils d'Ammon, et prophétise contre eux\FTNT{Jé. 49:1.}.
\VS{3}Dis aux fils d'Ammon : Ecoutez la parole du Seigneur Yahweh : Parce que vous avez dit : Ah ! Ah ! contre mon lieu saint, parce qu'il était profané ; et contre la terre d'Israël, parce qu'elle était désolée ; et contre la maison de Juda, parce qu'ils allaient en captivité\FTNT{Am. 1:13 ; So. 2:8.};
\VS{4}à cause de cela, voici, je m'en vais te donner en héritage aux fils d'orient, et ils bâtiront des palais dans tes villes, et ils demeureront chez toi ; ils mangeront tes fruits et boiront ton lait.
\VS{5}Je livrerai Rabba pour être le repaire des chameaux, et le pays des fils d'Ammon pour être le gîte des brebis, et vous saurez que je suis Yahweh.
\VS{6}Car ainsi parle le Seigneur Yahweh : Parce que tu as frappé des mains, que tu as battu des pieds, et que tu t'es réjoui de bon cœur avec tout le mépris que tu as eu pour la terre d'Israël,
\VS{7}à cause de cela voici, j'ai étendu ma main sur toi, et je te livrerai pour être pillée par les nations, et je te retrancherai du milieu des peuples, je te ferai périr d'entre les pays ; je te détruirai ; et tu sauras que je suis Yahweh.
\TextTitle{Jugement sur Moab}
\VS{8}Ainsi parle le Seigneur Yahweh : Parce que Moab et Séir ont dit : Voici, la maison de Juda est comme toutes les autres nations ;
\VS{9}à cause de cela voici, j'ouvre le territoire de Moab du côté des villes, de ses villes frontières, la beauté du pays de Beth-Jeschimoth, de Baal-Meon et de Kirjathaïm\FTNT{Jos. 12:3 ; No. 32:38.},
\VS{10}je l'ouvre aux fils d'orient, qui sont au-delà du pays des fils d'Ammon, je leur donne en possession, afin qu'on ne se souvienne plus des fils d'Ammon parmi les nations.
\VS{11}J'exercerai aussi des jugements contre Moab, et ils sauront que je suis Yahweh.
\TextTitle{Jugement sur Edom}
\VS{12}Ainsi parle le Seigneur Yahweh : A cause de ce qu'Edom a fait quand il s'est vengé de la maison de Juda, et parce qu'il s'en est rendu coupable en se vengeant d'eux\FTNT{Ps. 137:7.},
\VS{13}à cause de cela, le Seigneur Yahweh dit : J'étendrai ma main sur Edom, j'en retrancherai les hommes et les bêtes, et j'en ferai un désert ; depuis Théman à Dedan ils tomberont par l'épée\FTNT{Jé. 49:7-9 ; Am. 1:12 ; Ab. 1:9.}.
\VS{14}J'exercerai ma vengeance sur Edom à cause de mon peuple d'Israël, et on traitera Edom selon ma colère, et selon ma fureur, et ils reconnaîtront ma vengeance, dit le Seigneur Yahweh.
\TextTitle{Jugement sur les Philistins}
\VS{15}Ainsi parle le Seigneur Yahweh : Puisque les Philistins ont agi par vengeance, et qu'ils se sont vengés avec mépris et du fond de leur âme, voulant tout détruire dans leur haine éternelle ;
\VS{16}à cause de cela le Seigneur Yahweh dit : Voici, je m'en vais étendre ma main sur les Philistins, j'exterminerai les Kéréthiens, et je ferai périr le reste sur le rivage de la mer.
\VS{17}J'exercerai sur eux de grandes vengeances par des châtiments de fureur ; et ils sauront que je suis Yahweh, quand j'aurai exécuté sur eux ma vengeance\FTNT{Es. 14:29 ; Jé. 25:20 ; So. 2:7.}.
\Chap{26}
\TextTitle{Jugement sur Tyr}
\VerseOne{}Il arriva dans la onzième année, le premier jour du mois, que la parole de Yahweh me fut adressée en ces mots :
\VS{2}Fils de l'homme, parce que Tyr a dit au sujet de Jérusalem : Ah ! Ah ! Celle qui était la porte des peuples a été rompue, elle s'est réfugiée chez moi, je serai remplie parce qu'elle a été rendue déserte\FTNT{Am. 1:9 ; Za. 9:2-3.} !
\VS{3}A cause de cela, ainsi parle le Seigneur Yahweh : Voici, j'en veux à toi, Tyr, et je ferai monter contre toi plusieurs nations, comme la mer fait monter ses flots\FTNT{Jé. 51:42.}.
\VS{4}Elles détruiront les murailles de Tyr, et démoliront ses tours ; j'en raclerai sa poussière, et la rendrai semblable à un rocher nu\FTNT{ Es. 23:15.}.
\VS{5}Elle servira à étendre les filets au milieu de la mer ; car j'ai parlé, dit le Seigneur Yahweh, et elle sera en pillage aux nations.
\VS{6}Ses filles sur sa terre seront tuées par l'épée, et elles sauront que je suis Yahweh.
\VS{7}Car ainsi parle le Seigneur Yahweh : Voici, je m'en vais faire venir du nord contre Tyr, Nebucadnetsar, roi de Babylone, le roi des rois, avec des chevaux, des chars, des cavaliers, et un grand peuple assemblé de toutes parts.
\VS{8}Il tuera par l'épée tes filles sur ta terre, il fera des remparts contre toi, il dressera des terrasses contre toi, et il lèvera les boucliers contre toi.
\VS{9}Il donnera des coups de béliers contre tes murs, et renversera tes tours avec ses épées.
\VS{10}La multitude de ses chevaux te couvrira de poussière, tes murs trembleront au bruit des cavaliers, des roues, et des chars, quand il entrera par tes portes, comme on entre dans une ville qu'on a divisée.
\VS{11}Il foulera toutes tes rues avec les sabots de ses chevaux, il tuera ton peuple avec l'épée, et les trophées de ta force tomberont par terre\FTNT{Jé. 47:3 ; Es. 5:8.}.
\VS{12}Puis ils retireront tes biens, et pilleront ta marchandise ; ils renverseront tes murs, et renverseront tes maisons de plaisance ; et ils mettront tes pierres, ton bois et ta poussière au milieu des eaux.
\VS{13}Je ferai cesser le bruit de tes chansons, et le son de tes harpes ne sera plus entendu.
\VS{14}Je te rendrai semblable à un rocher nu ; tu seras un lieu pour étendre les filets, et tu ne seras plus rebâtie, parce que moi,Yahweh, j'ai parlé, dit le Seigneur Yahweh.
\VS{15}Ainsi parle le Seigneur Yahweh, à Tyr : Les îles ne trembleront-elles pas du bruit de ta ruine, quand ceux qui seront blessés à mort gémiront, quand le carnage se fera au milieu de toi ?
\VS{16}Tous les princes de la mer descendront de leurs trônes, ôteront leurs manteaux, dépouilleront leurs vêtements brodés, et s'envelopperont de frayeur ; ils s'assiéront sur la terre, ils seront effrayés à chaque instant, et seront désolés à cause de toi.
\VS{17}Ils prononceront à haute voix une complainte sur toi, et te diront : Comment as-tu péri, toi qui étais fréquentée par ceux qui vont sur la mer, ville renommée, qui étais forte dans la mer, toi et tes habitants qui inspiraient la terreur à tous ceux qui habitent chez elle\FTNT{Es. 23:15-16 ; Ap. 18:9.}?
\VS{18}Maintenant les îles seront effrayées au jour de ta ruine, et les îles qui sont dans la mer seront terrifiées à cause de ta fuite.
\VS{19}Car ainsi parle le Seigneur Yahweh : Quand je ferai de toi une ville désolée, comme sont les villes qui ne sont point habitées, quand j'aurai fait tomber sur toi l'abîme, et que les grosses eaux t'auront couverte ;
\VS{20}alors je te ferai descendre avec ceux qui descendent dans la fosse, vers le peuple d'autrefois, et je te placerai aux lieux les plus bas de la terre, aux endroits désolés depuis longtemps, avec ceux qui descendent dans la fosse, afin que tu ne sois plus habitée, mais je donnerai la gloire pour la terre des vivants.
\VS{21}Je ferai qu'on sera épouvanté à cause de toi, de ce que tu n'es plus ; et quand on te cherchera, on ne te trouvera plus jamais, dit le Seigneur Yahweh.
\Chap{27}
\TextTitle{Lamentation sur Tyr\FTNTT{Cp. Ap. 18:1-24.}}
\VerseOne{}La parole de Yahweh me fut encore adressée en ces mots :
\VS{2}Toi donc, fils de l'homme, prononce à haute voix une complainte sur Tyr.
\VS{3}Tu diras à Tyr : Toi qui demeures au bord de la mer, qui trafiques avec les peuples dans plusieurs îles ; ainsi parle le Seigneur Yahweh : Tyr, tu disais : Je suis parfaite en beauté !
\VS{4}Ton territoire est au cœur de la mer, ceux qui t'ont bâtie t'ont rendue parfaite en beauté.
\VS{5}Ils t'ont bâti de tous les côtés des navires de sapins de Senir ; ils ont pris les cèdres du Liban pour te faire des mâts.
\VS{6}Ils ont fait tes rames de chênes de Basan, et la troupe des Assyriens a fait tes bancs d'ivoire, apporté des îles de Kittim.
\VS{7}Le fin lin d'Egypte, avec des broderies, te servait de voiles et de pavillon ; des étoffes teintes en bleu et en pourpre des îles d'Elischa formaient tes couvertures.
\VS{8}Les habitants de Sidon et d'Arvad étaient tes rameurs, ô Tyr ! Les plus sages du milieu de toi étaient tes pilotes.
\VS{9}Les anciens de Guebal et ses hommes experts furent parmi toi, réparant tes brèches ; tous les navires de la mer, et leurs mariniers étaient chez toi, pour faire l'échange de tes marchandises.
\VS{10}Ceux de Perse, de Lud, et de Puth servaient dans ton armée. C'étaient des hommes de guerre, ils suspendaient chez toi le bouclier et le casque ; ils t'ont rendue magnifique.
\VS{11}Les fils d'Arvad avec ton armée étaient autour de tes murs, et des hommes braves étaient dans tes tours ; ils ont suspendu leurs boucliers à tous tes murs, ils ont achevé de te rendre parfaite en beauté.
\VS{12}Ceux de Tarsis ont trafiqué avec toi de toutes sortes de richesses, d'argent, de fer, d'étain et de plomb.
\VS{13}Javan, Tubal, et Méschec trafiquaient avec toi ; ils donnaient des personnes et des ustensiles d'airain en échange de tes marchandises.
\VS{14}Ceux de la maison de Togarma pourvoyaient tes marchés de chevaux, de cavaliers, et de mulets.
\VS{15}Les fils de Dedan trafiquaient avec toi ; tu avais dans ta main le commerce de plusieurs îles ; et on t'a rendu en échange des dents d'ivoire et de l'ébène.
\VS{16}La Syrie trafiquait avec toi, en quantité d'ouvrages faits pour toi ; elle pourvoyait tes marchés d'escarboucles, d'écarlate, de broderie, de fin lin, de corail, et d'agate.
\VS{17}Juda et le pays d'Israël trafiquaient avec toi, faisant valoir ton commerce en blé de Minnith, en pâtisseries, en miel, en huile, et en baume.
\VS{18}Damas trafiquait avec toi en quantité d'ouvrages faits pour toi, en toutes sortes de richesses, en vin de Helbon, et en laine blanche.
\VS{19}Vedan, et Javan depuis Uzal, pourvoyaient tes marchés ; le fer luisant, la casse et le roseau aromatique furent dans ton commerce.
\VS{20}Dedan trafiquait avec toi en couvertures pour s'asseoir à cheval.
\VS{21}Les arabes, et tous les princes de Kédar, étaient des marchands dans ta main, trafiquant avec toi en agneaux, en moutons, et en boucs.
\VS{22}Les marchands de Séba et de Raema trafiquaient avec toi de tous les meilleurs aromates, de toute sorte de pierres précieuses et d'or.
\VS{23}Charan, Canné, et Eden, les marchands de Séba, d'Assyrie, de Kilmad, trafiquaient avec toi.
\VS{24}Ils trafiquaient avec toi toutes sortes de belles choses, des manteaux teints en bleu, en broderie, en riches étoffes contenues dans des coffres attachés avec des cordes, faits en bois de cèdre, et amenés sur tes marchés.
\VS{25}Les navires de Tarsis naviguaient pour ton commerce ; tu étais au comble de la force et de la richesse, au cœur des mers.
\VS{26}Tes rameurs t'ont amenée dans de grosses eaux, le vent d'orient t'a brisée au cœur de la mer.
\VS{27}Tes richesses, tes marchés et tes marchandises, tes mariniers et tes pilotes, ceux qui réparaient tes brèches, et ceux qui s'occupaient de ton commerce, tous tes hommes de guerre qui étaient chez toi, et toute ta multitude au milieu de toi, tomberont dans le cœur de la mer au jour de ta ruine\FTNT{Ap. 18:9.}.
\VS{28}Les faubourgs trembleront au bruit du cri de tes pilotes.
\VS{29}Tous ceux qui manient la rame descendront de leurs navires, les mariniers, et tous les pilotes de la mer ; ils se tiendront sur la terre ;
\VS{30}ils feront entendre leur voix, et crieront amèrement ; ils jetteront de la poussière sur leurs têtes, et se vautreront dans la cendre ;
\VS{31}ils arracheront leurs cheveux, et rendront leur tête chauve à cause de toi, ils se ceindront de sacs, et te pleureront avec l'amertume dans leur âme, en menant un deuil amer.
\VS{32}Ils prononceront à haute voix sur toi une complainte dans leur lamentation, et feront leur complainte sur toi, en disant : Qui fut jamais comme Tyr, comme cette ville détruite au cœur de la mer ?
\VS{33}Tu as rassasié plusieurs peuples par la traite des marchandises qu'on apportait de tes marchés au-delà des mers ; et tu as enrichi les rois de la terre par la multitude de tes richesses et de ton commerce.
\VS{34}Quand tu as été brisée par la mer au fond des eaux, ton commerce et toute ta multitude sont tombés avec toi.
\VS{35}Tous les habitants des îles sont désolés à cause de toi ; et leurs rois sont saisis d'épouvante, et leur visage pâlit.
\VS{36}Les marchands parmi les peuples t'insultent, tu es réduite au néant, tu ne seras plus à jamais !
\Chap{28}
\TextTitle{Yahweh réprime l'arrogance du roi de Tyr}
\VerseOne{}La parole de Yahweh me fut encore adressée en ces mots :
\VS{2}Fils de l'homme, dis au prince de Tyr : Ainsi parle le Seigneur Yahweh : Parce que ton cœur s'est élevé et que tu as dit : Je suis Dieu, je suis assis sur le siège de Dieu, au cœur de la mer, quoique tu sois un homme, et non Dieu, et parce que tu as élevé ton cœur comme si tu étais un Dieu.
\VS{3}Voici, tu es plus sage que Daniel, rien de caché ne t'a été rendu obscur.
\VS{4}Tu t'es acquis de la puissance par ta sagesse et par ton intelligence ; et tu as amassé de l'or et de l'argent dans tes trésors\FTNT{Za. 9:2-3.}.
\VS{5}Tu as multiplié ta puissance par la grandeur de ta sagesse dans ton commerce, puis ton cœur s'est élevé à cause de ta puissance.
\VS{6}C'est pourquoi ainsi parle le Seigneur Yahweh : Parce que tu as élevé ton cœur, comme si tu étais un Dieu,
\VS{7}à cause de cela voici, je m'en vais faire venir contre toi des étrangers, les plus terribles parmi les nations, qui tireront leurs épées sur la beauté de ta sagesse, et souilleront ta splendeur.
\VS{8}Ils te feront descendre dans la fosse, et tu mourras comme ceux qui tombent percés de coups, au milieu de la mer.
\VS{9}En face de ton meurtrier diras-tu : Je suis Dieu ? Tu seras homme et non Dieu sous la main de celui qui te tuera.
\VS{10}Tu mourras de la mort des incirconcis par la main des étrangers ; car j'ai parlé, dit le Seigneur Yahweh.
\TextTitle{Chute du roi de Tyr représentant satan\FTNTT{Cp. Es. 14:12-17.}}
\VS{11}La parole de Yahweh me fut encore adressée en ces mots :
\VS{12}Fils de l'homme, prononce à haute voix une complainte sur le roi de Tyr, et dis-lui : Ainsi parle le Seigneur Yahweh : Toi à qui rien ne manquait, plein de sagesse, et parfait en beauté ;
\VS{13}tu étais en Eden, le jardin de Dieu ; ta couverture était de pierres précieuses de toutes sortes, de sardoine, de topaze, de diamant, de chrysolithe, d'onyx, de jaspe, de saphir, d'escarboucle, d'émeraude, et d'or ; tes tambourins et tes flûtes étaient à ton service ; préparés pour le jour où tu fus créé.
\VS{14}Tu étais un chérubin, oint pour servir de protection ; je t'avais établi, et tu étais sur la sainte montagne de Dieu ; tu marchais entre les pierres éclatantes.
\VS{15}Tu étais parfait dans tes voies dès le jour où tu fus créé, jusqu'à celui où l'injustice fut trouvée en toi.
\VS{16}Selon la grandeur de ton trafic\FTNT{Satan est le premier commerçant. Il avait transformé ses sanctuaires célestes en un lieu de trafic, en un marché. Il avait reçu gratuitement du Seigneur, notre Dieu, plusieurs dons : la beauté, des pierres précieuses, des instruments de musique, la sagesse, un sanctuaire. Au lieu de les utiliser pour la gloire de Dieu, il en fit un trafic pour son propre profit égoïste. Il est le père de tous ceux qui vendent les dons de Dieu pour s'enrichir, de tous ceux qui font du commerce avec l'Evangile. Or Jésus nous a donné cet ordre formel : « Vous avez reçu gratuitement, donnez gratuitement » (Mt. 10 : 8). De même que le temple de Dieu était devenu une caverne de voleurs, plusieurs pasteurs ont transformé les bâtiments de leurs églises en véritables boutiques pour vendre toutes sortes de produits dérivés qui ne servent pas à l'avancement du Royaume de Dieu mais à enrichir des dirigeants cupides esclaves du dieu Mammon (Jn. 2:13-17 ; Mt. 6:24 ; Lu. 16:13 ; 1 Ti. 6:10 ; Hé. 13:5).}, tu as été rempli de violence, et tu as péché ; c'est pourquoi je te jette comme une chose souillée hors de la montagne de Dieu\FTNT{Ap. 12:1-12.}, et je te détruis d'entre les pierres éclatantes, ô chérubin protecteur !
\VS{17}Ton cœur s'est élevé à cause de ta beauté, tu as corrompu ta sagesse à cause de ton éclat ; je te jette par terre, je te donne en spectacle aux rois, afin qu'ils te regardent.
\VS{18}Tu as profané tes sanctuaires par la multitude de tes iniquités, par l'injustice de ton commerce ; et je fais sortir du milieu de toi un feu qui te consume, je te réduis en cendres sur la terre, dans la présence de tous ceux qui te regardent.
\VS{19}Tous ceux qui te connaissent parmi les peuples sont désolés à cause de toi ; tu es réduit à néant, tu ne seras plus à jamais.
\TextTitle{Jugement sur Sidon}
\VS{20}Puis la parole de Yahweh me fut adressée en ces mots :
\VS{21}Fils de l'homme, tourne ta face vers Sidon, et prophétise contre elle.
\VS{22}Tu diras : Ainsi parle le Seigneur Yahweh : Voici j'en veux à toi, Sidon ! Je serai glorifié au milieu de toi ; et on saura que je suis Yahweh, quand j'aurai exercé des jugements contre elle et que je serai sanctifié.
\VS{23}J'enverrai la peste dans son sein, je ferai couler le sang dans ses rues. Les morts tomberont au milieu d'elle par l'épée qui viendra de toutes parts sur elle ; et ils sauront que je suis Yahweh.
\VS{24}Elle ne sera plus pour la maison d'Israël une épine qui blesse, une ronce déchirante, parmi tous ceux qui l'entourent et qui la méprisent. Et ils sauront que je suis le Seigneur Yahweh.
\TextTitle{Rétablissement d'Israël}
\VS{25}Ainsi parle le Seigneur Yahweh : Quand j'aurai rassemblé la maison d'Israël d'entre les peuples parmi lesquels ils auront été dispersés, je manifesterai en elle ma sainteté, aux yeux des nations, et ils habiteront sur leur terre que j'ai donnée à mon serviteur Jacob.
\VS{26}Ils y habiteront en sûreté, ils bâtiront des maisons, ils planteront des vignes ; ils y habiteront, dis-je, en sûreté, lorsque j'aurai exercé des jugements contre ceux qui les auront pillés de toutes parts ; et ils sauront que je suis Yahweh leur Dieu.
\Chap{29}
\TextTitle{Jugement sur l'Egypte}
\VerseOne{}La dixième année, au douzième jour du dixième mois, la parole de Yahweh me fut adressée en ces mots :
\VS{2}Fils de l'homme, tourne ta face contre Pharaon, roi d'Egypte, prophétise contre lui, et contre toute l'Egypte\FTNT{Jé. 43:8-11.}.
\VS{3}Parle, et dis : Ainsi parle le Seigneur Yahweh : Voici, j'en veux à toi, Pharaon, roi d'Egypte, grand serpent couché au milieu de tes fleuves, qui dis : Mes fleuves sont à moi, et je me les suis faits\FTNT{Ps. 74:13-14 ; Es. 27:1.} !
\VS{4}C'est pourquoi je mettrai des crocs dans ta mâchoire, j'attacherai à tes écailles les poissons de tes fleuves ; je te tirerai hors de tes fleuves, avec tous les poissons de tes fleuves, qui seront attachés à tes écailles.
\VS{5}Et t'ayant tiré dans le désert, je te laisserai là, toi, et tous les poissons de tes fleuves ; tu tomberas sur la face des champs, tu ne seras point recueilli ni ramassé ; je te livrerai aux bêtes de la terre, et aux oiseaux des cieux, pour en être dévoré.
\VS{6}Et tous les habitants d'Egypte sauront que je suis Yahweh ; parce qu'ils ont été un soutien de roseau pour la maison d'Israël\FTNT{2 R. 18:21 ; Es. 36:6.}.
\VS{7}Quand ils t'ont pris par la main, tu t'es rompu, et tu leur as percé toute l'épaule ; et quand ils se sont appuyés sur toi, tu t'es cassé, et tu les as fait tomber à la renverse.
\VS{8}C'est pourquoi ainsi parle le Seigneur Yahweh : Voici, je m'en vais faire venir l'épée sur toi, et j'exterminerai du milieu de toi les hommes et les bêtes.
\VS{9}Le pays d'Egypte sera dans la désolation et dans le désert, et ils sauront que je suis Yahweh, parce que le roi d'Egypte a dit : Les fleuves sont à moi, et je les ai faits !
\VS{10}C'est pourquoi voici, j'en veux à toi, et à tes fleuves, et je réduirai le pays d'Egypte en désert de sécheresse et de désolation, depuis Migdol jusqu'à Syène, et aux frontières de l'Ethiopie.
\VS{11}Nul pied d'homme ne passera par là, et il n'y passera non plus aucun pied d'animal, elle sera quarante ans sans être habitée.
\VS{12}Car je réduirai le pays d'Egypte en désolation entre les pays désolés, et ses villes entre les villes réduites en désert ; elles seront en désolation durant quarante ans, je disperserai les Egyptiens parmi les nations, et je les répandrai parmi les pays.
\VS{13}Toutefois, ainsi parle le Seigneur Yahweh : Au bout de quarante ans, je ramasserai les Egyptiens d'entre les peuples parmi lesquels ils auront été dispersés ;
\VS{14}je ramènerai les captifs d'Egypte, et les ferai retourner au pays de Pathros, au pays de leur origine, mais ils seront là un royaume rabaissé.
\VS{15}Il sera le plus bas des royaumes, et il ne s'élèvera plus au-dessus des nations, je le diminuerai, afin qu'il ne domine point sur les nations.
\VS{16}Ce royaume ne sera plus pour la main d'Israël un sujet de confiance ; il lui rappellera son iniquité, quand elle se tournait vers eux ; et ils sauront que je suis le Seigneur Yahweh.
\VS{17}Il arriva la vingt-septième année, au premier jour du premier mois, que la parole de Yahweh me fut adressée en ces mots :
\VS{18}Fils de l'homme, Nebucadnetsar, roi de Babylone, a fait servir son armée dans un service pénible contre Tyr ; toute tête en est devenue chauve, et toute épaule en a été foulée, mais il n'a point eu de salaire, ni lui ni son armée, à cause de Tyr, pour le service qu'il a fait contre elle.
\VS{19}C'est pourquoi ainsi parle le Seigneur Yahweh : Voici, je m'en vais donner à Nebucadnetsar, roi de Babylone, le pays d'Egypte ; il enlèvera la multitude, il emportera le butin et fera le pillage ; ce sera là le salaire de son armée.
\VS{20}Pour prix du service qu'il a fait contre Tyr, je lui ai donné le pays d'Egypte, parce qu'ils ont travaillé pour moi, dit le Seigneur Yahweh.
\VS{21}En ce jour-là, je ferai germer la corne de la maison d'Israël, et j'ouvrirai ta bouche au milieu d'eux, et ils sauront que je suis Yahweh.
\Chap{30}
\TextTitle{Disgrâce de l'Egypte}
\VerseOne{}La parole de Yahweh me fut encore adressée en ces mots :
\VS{2}Fils de l'homme, prophétise, et dis : Ainsi parle le Seigneur Yahweh : Hurlez, et dites : Malheureux jour !
\VS{3}Car le jour est proche, oui le jour de Yahweh est proche, c'est un jour ténébreux ; ce sera le temps des nations.
\VS{4}L'épée viendra sur l'Egypte, et il y aura de l'effroi en Ethiopie, quand ceux qui seront blessés à mort tomberont dans l'Egypte, et quand on enlèvera la multitude de son peuple, et que ses fondements seront ruinés.
\VS{5}L'Ethiopie, Puth, Lud, toute l'Arabie, Cub, et les fils du pays allié tomberont par l'épée avec eux\FTNT{Jé. 46:9 ; Na. 3:9-10.}.
\VS{6}Ainsi parle Yahweh : Ceux qui soutiendront l'Egypte, tomberont ; et l'orgueil de sa force sera renversé ; ils tomberont par l'épée de Migdol à Syène, dit le Seigneur Yahweh.
\VS{7}Ils seront désolés au milieu des pays désolés, et ses villes seront au milieu des villes désertes.
\VS{8}Ils sauront que je suis Yahweh, quand j'aurai mis le feu en Egypte ; et tous ceux qui lui donneront du secours, seront brisés.
\VS{9}En ce jour-là, des messagers sortiront de ma part sur des navires pour effrayer l'Ethiopie dans sa sécurité, et il y aura entre eux un tourment au jour de l'Egypte ; car voici, il vient.
\VS{10}Ainsi parle le Seigneur Yahweh : Je ferai périr la multitude d'Egypte par la puissance de Nebucadnetsar, roi de Babylone.
\VS{11}Lui et son peuple avec lui, les plus terribles d'entre les nations, seront amenés pour ruiner le pays, et ils tireront leurs épées contre les Egyptiens, et rempliront la terre de morts.
\VS{12}Je mettrai à sec les fleuves et je livrerai le pays entre les mains des méchants ; je désolerai le pays, et tout ce qui y est, par la puissance des étrangers ; moi, Yahweh, j'ai parlé.
\VS{13}Ainsi parle le Seigneur Yahweh : Je détruirai aussi les idoles, j'anéantirai les faux dieux de Noph, et il n'y aura point de prince qui soit du pays d'Egypte ; je mettrai la frayeur dans le pays d'Egypte\FTNT{Es. 19:1-13 ; Jé. 43:12 ; Jé. 46:13.}.
\VS{14}Je désolerai Pathros, je mettrai le feu à Tsoan, et j'exercerai mes jugements sur No\FTNT{Jé. 44:1.}.
\VS{15}Je répandrai ma fureur sur Sin, qui est la place forte de l'Egypte, et j'exterminerai la multitude qui est à No.
\VS{16}Quand je mettrai le feu en Egypte, Sin sera grièvement tourmentée, et No sera rompue par diverses brèches, et il n'y aura à Noph que détresses en plein jour.
\VS{17}Les jeunes hommes d'On et de Pi-Béseth tomberont par l'épée, et ces villes iront en captivité.
\VS{18}Le jour s'obscurcira à Tachpanès, lorsque j'y romprai le joug de l'Egypte, et que l'orgueil de sa force aura cessé ; un nuage la couvrira, et les villes de son ressort iront en captivité.
\VS{19}J'exercerai des jugements en Egypte ; et ils sauront que je suis Yahweh.
\TextTitle{Chute et dispersion de l'Egypte}
\VS{20}Dans la onzième année, au septième jour du premier mois, la parole de Yahweh me fût adressée en ces mots :
\VS{21}Fils de l'homme, j'ai rompu le bras de Pharaon, roi d'Egypte ; et voici on ne l'a point bandé pour le guérir, on ne lui a point mis de linges pour le bander, et pour le fortifier, afin qu'il puisse manier l'épée.
\VS{22}C'est pourquoi ainsi parle le Seigneur Yahweh : Voici, j'en veux à Pharaon, roi d'Egypte, et je romprai ses bras, tant celui qui est fort que celui qui est rompu, et je ferai tomber l'épée de sa main.
\VS{23}Je disperserai les Egyptiens parmi les nations, et les répandrai parmi les pays.
\VS{24}Je fortifierai les bras du roi de Babylone, je lui mettrai mon épée dans la main ; mais je romprai les bras de Pharaon, et il gémira devant lui comme gémissent les mourants.
\VS{25}Je fortifierai donc les bras du roi de Babylone, mais les bras de Pharaon tomberont ; et on saura que je suis Yahweh, quand j'aurai mis mon épée dans la main du roi de Babylone, et qu'il l'a tournera contre le pays d'Egypte.
\VS{26}Je disperserai les Egyptiens parmi les nations, les répandrai parmi les pays ; et ils sauront que je suis Yahweh.
\Chap{31}
\TextTitle{Avertissement contre l'arrogance de Pharaon}
\VerseOne{}Il arriva aussi dans la onzième année, au premier jour du troisième mois, que la parole de Yahweh me fut adressée en ces mots :
\VS{2}Fils de l'homme, parle à Pharaon, roi d'Egypte, et à la multitude de son peuple : A qui ressembles-tu dans ta grandeur ?
\VS{3}Voici, le roi d'Assyrie a été comme un cèdre du Liban, ayant de belles branches, et des rameaux qui faisaient une grande ombre, et qui étaient d'une grande hauteur ; sa cime était fort touffue.
\VS{4}Les eaux l'ont fait croître, l'abîme l'a fait pousser en hauteur, ses fleuves ont coulé autour de ses plantes, et il a envoyé ses eaux abondantes vers tous les arbres des champs.
\VS{5}C'est pourquoi il s'est élevé au-dessus de tous les autres arbres des champs, ses branches se sont multipliées, et ses rameaux croissaient par les grandes eaux qui faisaient pousser ses branches.
\VS{6}Tous les oiseaux des cieux ont fait leurs nids dans ses branches, toutes les bêtes des champs ont fait leurs petits sous ses rameaux, et toutes les grandes nations ont habité sous son ombre.
\VS{7}Il était beau par sa grandeur, et par l'étendue de ses branches, parce que sa racine était sur de grandes eaux.
\VS{8}Les cèdres du jardin de Dieu ne le surpassaient point ; les cyprès n'égalaient point ses branches, et les platanes n'égalaient point comme ses rameaux ; aucun arbre du jardin de Dieu ne lui était comparable en beauté.
\VS{9}Je l'avais embelli par la multitude de ses rameaux, au point que tous les arbres d'Eden, qui étaient dans le jardin de Dieu, lui portaient envie.
\VS{10}C'est pourquoi le Seigneur Yahweh dit : Parce qu'il s'est élevé, parce qu'il lançait sa cime au milieu d'épais rameaux et que son cœur était fier de sa hauteur,
\VS{11}je l'ai livré entre les mains du plus fort des nations, qui l'a traité comme il fallait, et je l'ai chassé à cause de sa méchanceté.
\VS{12}Les étrangers les plus terrifiants parmi les nations l'ont coupé et l'ont laissé là, ses branches sont tombées sur les montagnes et sur toutes les vallées ; ses rameaux se sont rompus dans tous les ravins de la terre, et tous les peuples de la terre se sont retirés de dessous son ombre, et l'ont laissé là.
\VS{13}Tous les oiseaux des cieux se sont tenus sur ses ruines, et toutes les bêtes des champs se sont retirées vers ses rameaux,
\VS{14}afin que tous les arbres près des eaux n'élèvent plus leur hauteur, et qu'ils ne lancent plus leur cime au milieu d'épais rameaux, afin que tous les chênes arrosés d'eau ne gardent plus leur hauteur ; car tous sont livrés à la mort, aux profondeurs de la terre, parmi les fils des hommes, avec ceux qui descendent dans la fosse.
\VS{15}Ainsi parle le Seigneur Yahweh : Le jour qu'il descendit dans le scheol, j'ai répandu deuil sur lui, j'ai couvert l'abîme devant lui, j'ai empêché ses fleuves de couler, et les grosses eaux ont été retenues ; j'ai fait que le Liban soit en deuil à cause de lui, et tous les arbres des champs ont été desséchés.
\VS{16}J'ai ébranlé les nations par le bruit de sa ruine, quand je l'ai fait descendre dans le scheol, avec ceux qui descendent dans la fosse\FTNT{Es. 14:9.} ; et tous les arbres d'Eden, les plus beaux et les plus agréables du Liban, tous arrosés par les eaux, ont été consolés dans les profondeurs de la terre.
\VS{17}Eux aussi sont descendus avec lui dans le scheol, vers ceux qui ont péri par l'épée ; ils étaient son bras et ils habitaient sous son ombre parmi les nations.
\VS{18}A qui ressembles-tu ainsi en gloire et en grandeur parmi les arbres d'Eden ? Tu seras précipité avec les arbres d'Eden dans les profondeurs de la terre, tu seras gisant au milieu des incirconcis, avec ceux qui ont péri par l'épée. Voilà Pharaon et toute sa multitude ! dit le Seigneur Yahweh.
\Chap{32}
\TextTitle{Lamentation sur le pays d'Egypte}
\VerseOne{}Dans la douzième année, le premier jour du douzième mois, la parole de Yahweh me fut adressée en ces mots :
\VS{2}Fils de l'homme, prononce à haute voix une complainte sur Pharaon, roi d'Egypte, et dis-lui : Tu as été parmi les nations semblable à un lionceau, et comme un serpent dans les mers ; tu t'élançais dans tes fleuves, et tu troublais les eaux avec tes pieds, et remplissais de bourbe leurs fleuves.
\VS{3}Ainsi parle le Seigneur Yahweh : J'étendrai mon rets sur toi dans une assemblée nombreuse de peuples qui te tireront dans mes filets.
\VS{4}Je te laisserai à l'abandon sur la terre ; je te jetterai sur le dessus des champs, et je ferai demeurer sur toi tous les oiseaux des cieux, et rassasierai de toi les bêtes de toute la terre.
\VS{5}Car je mettrai ta chair sur les montagnes, et je remplirai les vallées de tes débris.
\VS{6}J'arroserai de ton sang jusqu'aux montagnes, la terre où tu nages, et les lits des eaux seront remplis de toi.
\VS{7}Quand je t'éteindrai, je couvrirai les cieux et j'obscurcirai leurs étoiles, je couvrirai le soleil de nuages, et la lune ne donnera plus sa lumière\FTNT{Es. 13:10 ; Joë. 2:31 ; Mt. 24:29.}.
\VS{8}J'obscurcirai à cause de toi tous les luminaires des cieux, et je répandrai les ténèbres sur ton pays, dit le Seigneur Yahweh.
\VS{9}J'affligerai le cœur de beaucoup de peuples, quand j'annoncerai ta ruine parmi les nations, à des pays que tu ne connaissais pas.
\VS{10}Je frapperai de stupeur beaucoup de peuples à cause de toi, et leurs rois seront saisis d'épouvante à cause de toi, quand je ferai luire mon épée à leurs yeux ; ils seront effrayés à chaque instant, chacun pour sa vie, au jour de ta ruine.
\VS{11}Car ainsi parle le Seigneur Yahweh : L'épée du roi de Babylone viendra sur toi.
\VS{12}J'abattrai ta multitude par les épées des hommes forts, qui tous sont les plus terribles d'entre les nations ; ils détruiront l'orgueil de l'Egypte, et toute la multitude de son peuple sera ruinée.
\VS{13}Je ferai périr tout son bétail près des grandes eaux, et aucun pied d'homme ne les troublera plus, ni aucun pied d'animaux ne les agitera plus.
\VS{14}Alors je rendrai profondes leurs eaux, et je ferai couler leurs fleuves comme de l'huile, dit le Seigneur Yahweh.
\VS{15}Quand j'aurai réduit le pays d'Egypte en désolation, et que le pays sera dénué des choses dont il était rempli ; quand je frapperai tous ceux qui y habitent, ils sauront alors que je suis Yahweh.
\VS{16}C'est ici la complainte qu'on fera sur elle, les filles des nations feront cette complainte sur elle ; elles feront cette complainte sur l'Egypte et sur toute la multitude de son peuple, dit le Seigneur.
\VS{17}Il arriva aussi dans la douzième année, le quinzième jour du mois, que la parole de Yahweh me fut adressée en ces mots :
\VS{18}Fils de l'homme, dresse une lamentation sur la multitude d'Egypte, et fais-la descendre, elle et les filles des nations magnifiques, aux plus bas lieux de la terre, avec ceux qui descendent dans la fosse\FTNT{Jé. 1:10 ; Jé. 18:7.}.
\VS{19}Qui surpasses-tu en beauté ? Descends, et couche-toi avec les incirconcis !
\VS{20}Ils tomberont au milieu de ceux qui seront tués par l'épée. L'épée a déjà été donnée : Entraînez l'Egypte et toute sa multitude !
\VS{21}Les plus forts d'entre les puissants lui parleront du milieu du scheol, avec ceux qui lui donnaient du secours, et diront : Ils sont descendus, ils sont couchés, les incirconcis, tués par l'épée.
\VS{22}Là est l'Assyrien, et toute son assemblée ; ses sépulcres sont autour de lui, eux tous, mis à mort, sont tombés par l'épée.
\VS{23}Car ses sépulcres sont posés au fond de la fosse et son assemblée autour de sa sépulture ; eux tous qui avaient répandu leur terreur sur la terre des vivants sont tombés morts par l'épée.
\VS{24}Là est Elam et toute sa multitude autour de son sépulcre ; eux tous sont tombés morts par l'épée, ils sont descendus incirconcis dans les plus bas lieux de la terre ; et après avoir répandu leur terreur sur la terre des vivants, ils ont porté leur ignominie avec ceux qui descendent dans la fosse.
\VS{25}On a mis sa couche parmi ceux qui ont été tués, avec toute sa multitude ; ses sépulcres sont autour de lui ; eux tous incirconcis, tués par l'épée, quoiqu'ils aient répandu leur terreur sur la terre des vivants, toutefois ils ont porté leur ignominie avec ceux qui descendent dans la fosse ; ils ont été placés parmi les morts.
\VS{26}Là est Méschec, Tubal, et toute leur multitude ; leurs sépulcres sont autour d'eux ; eux tous incirconcis, tués par l'épée, quoiqu'ils aient répandu leur terreur sur la terre des vivants.
\VS{27}Ils ne se sont point couchés avec les hommes vaillants qui sont tombés d'entre les incirconcis, lesquels sont descendus dans le scheol avec leurs armes de guerre, dont on a mis les épées sous leurs têtes, et dont les iniquités ont reposé sur leurs os ; parce que la terreur des hommes forts est dans la terre des vivants.
\VS{28}Toi aussi tu seras brisé au milieu des incirconcis, et tu seras couché avec ceux qui sont tués par l'épée.
\VS{29}Là est Edom, ses rois, et tous ses princes, qui ont été placés malgré leur force avec ceux qui sont tués par l'épée ; ils seront couchés avec les incirconcis, et avec ceux qui sont descendus dans la fosse.
\VS{30}Là sont tous les princes du nord, et tous les Sidoniens, qui sont descendus avec ceux qui sont tués, malgré la terreur qu'inspirait leur force ; ils sont couchés incirconcis avec ceux qui sont tués par l'épée ; ils ont porté leur ignominie avec ceux qui sont descendus dans la fosse.
\VS{31}Pharaon les verra, et il se consolera au sujet de toute la multitude de son peuple ; Pharaon, dit le Seigneur Yahweh, verra les blessés par l'épée et toute son armée.
\VS{32}Car je mettrai ma terreur dans la terre des vivants, c'est pourquoi Pharaon avec toute la multitude de son peuple se couchera au milieu des incirconcis, avec ceux qui sont tués par l'épée, dit le Seigneur Yahweh.
\Chap{33}
\TextTitle{Ezéchiel établi comme sentinelle pour avertir le pécheur}
\VerseOne{}La parole de Yahweh me fut encore adressée en ces mots :
\VS{2}Fils de l'homme, parle aux fils de ton peuple, et dis-leur : Quand je ferai venir l'épée sur un pays, et que le peuple du pays aura choisi quelqu'un d'entre eux, et l'aura établi pour leur servir de sentinelle,
\VS{3}et que voyant venir l'épée sur le pays, il sonnera du shofar et avertira le peuple,
\VS{4}si le peuple ayant bien entendu le son du shofar, ne se tient pas sur ses gardes, et qu'ensuite l'épée vienne le prendre, son sang sera sur sa tête.
\VS{5}Car il a entendu le son du shofar, et ne s'est point tenu sur ses gardes ; son sang sera sur lui ; mais s'il se tient sur ses gardes, il sauvera sa vie.
\VS{6}Si la sentinelle voit venir l'épée, et qu'elle ne sonne point du shofar, en sorte que le peuple ne se tienne point sur ses gardes, et qu'ensuite l'épée survienne et ôte la vie à l'un d'entre eux, celui-ci sera emmené en captivité à cause de son iniquité, mais je redemanderai son sang de la main de la sentinelle.
\VS{7}Toi donc, fils de l'homme, je t'ai établi pour sentinelle sur la maison d'Israël ; tu écouteras donc la parole qui sort de ma bouche, et tu les avertiras de ma part.
\VS{8}Quand j'aurai dit au méchant : Méchant, tu mourras ! et que tu n'auras point parlé au méchant pour l'avertir de se détourner de sa voie, ce méchant mourra dans son iniquité ; mais je redemanderai son sang de ta main.
\VS{9}Mais si tu as averti le méchant de se détourner de sa voie, et qu'il ne se détourne pas de sa voie, il mourra dans son iniquité ; mais toi tu auras délivré ton âme.
\VS{10}Toi donc, fils de l'homme, dis à la maison d'Israël : Vous avez parlé ainsi, en disant : Puisque nos crimes et nos péchés sont sur nous, et que nous périssons à cause d'eux, comment pourrions-nous vivre\FTNT{Lé. 26:39.} ?
\VS{11}Dis-leur : Je suis vivant, dit le Seigneur Yahweh, je ne prends point plaisir dans la mort du méchant, mais que le méchant se détourne de sa voie et qu'il vive. Détournez-vous, détournez-vous de votre méchante voie ! Pourquoi mourriez-vous, maison d'Israël ?
\VS{12}Toi donc, fils de l'homme, dis aux fils de ton peuple : La justice du juste ne le délivrera point au jour de son péché, le méchant ne tombera point par sa méchanceté au jour où il s'en détournera ; et le juste ne pourra pas vivre par sa justice au jour de son péché.
\VS{13}Quand j'aurai dit au juste qu'il vivra certainement, et que lui, se confiant sur sa justice, aura commis l'iniquité, on ne se souviendra plus d'aucune de ses justices, mais il mourra dans son iniquité qu'il aura commise.
\VS{14}Aussi quand j'aurai dit au méchant : Tu mourras ! s'il se détourne de son péché, et qu'il fasse ce qui est juste et droit ;
\VS{15}si le méchant rend le gage et qu'il restitue ce qu'il aura ravi, et qu'il marche dans les statuts de la vie, sans commettre d'iniquité, certainement il vivra, il ne mourra point.
\VS{16}On ne se souviendra plus des péchés qu'il aura commis ; il a fait ce qui est juste et droit ; certainement il vivra.
\VS{17}Or les fils de ton peuple ont dit : La voie du Seigneur n'est pas bien réglée ; mais c'est plutôt leur voie qui n'est pas bien réglée.
\VS{18}Quand le juste se détournera de sa justice, et qu'il commettra l'iniquité, il mourra à cause de cela.
\VS{19}Quand le méchant se détournera de sa méchanceté, et qu'il fera ce qui est juste et droit, il vivra à cause de cela.
\VS{20}Vous avez dit : La voie du Seigneur n'est pas bien réglée ! Je vous jugerai, maison d'Israël, chacun selon sa voie.
\TextTitle{Exécution du jugement de Yahweh}
\VS{21}Or il arriva dans la douzième année de notre captivité, au cinquième jour du dixième mois, qu'un homme qui s'était échappé de Jérusalem vint vers moi, en disant : La ville est prise !
\VS{22}La main de Yahweh fut sur moi le soir, avant l'arrivée du fugitif, et Yahweh ouvrit ma bouche lorsqu'il vint auprès de moi le matin. Ma bouche était ouverte et je n'étais plus muet.
\TextTitle{Ne pas se contenter d'écouter la Parole de Dieu}
\VS{23}La parole de Yahweh me fut adressée en ces mots :
\VS{24}Fils de l'homme, ceux qui habitent dans ces ruines, sur la terre d'Israël, discourent en disant : Abraham était seul, et il a possédé le pays\FTNT{Ge. 15:7.}; mais nous sommes un grand nombre de gens, et le pays nous a été donné en héritage.
\VS{25}C'est pourquoi tu leur diras : Ainsi parle le Seigneur Yahweh : Vous mangez la chair avec le sang, et vous levez vos yeux vers vos idoles, vous répandez le sang ; et vous posséderiez le pays\FTNT{Ge. 9:4 ; Lé. 3:17 ; Lé. 17:10.} ?
\VS{26}Vous vous appuyez sur votre épée, vous commettez des abominations, vous souillez chacun de vous la femme de son prochain ; et vous posséderiez le pays ?
\VS{27}Tu leur diras : Ainsi parle le Seigneur Yahweh : Je suis vivant, ceux qui sont dans ces ruines tomberont par l'épée, et je livrerai aux bêtes celui qui est dans les champs, afin qu'elles le mangent ; et ceux qui sont dans les forteresses et dans les cavernes mourront par la peste.
\VS{28}Ainsi je réduirai le pays en désolation et en désert, l'orgueil de sa force sera aboli, et les montagnes d'Israël seront désolées, en sorte qu'il n'y passera plus personne.
\VS{29}Ils sauront que je suis Yahweh, quand j'aurai réduit leur pays en désolation et en désert, à cause de toutes leurs abominations qu'ils ont commises.
\VS{30}Quant à toi, fils de l'homme, les fils de ton peuple parlent de toi près des murs et aux entrées des maisons, et parlent l'un à l'autre, chacun avec son prochain, en disant : Venez maintenant, et écoutez la parole qui vient de Yahweh.
\VS{31}Ils viennent vers toi en foule, et mon peuple s'assied devant toi, ils écoutent tes paroles, mais ils ne les mettent point en pratique ; ils les répètent comme si c'était une chanson profane, mais leur cœur marche toujours après leur gain déshonnête.
\VS{32}Voici tu es pour eux comme un homme qui leur chante une chanson profane avec une belle voix, qui résonne bien ; car ils écoutent bien tes paroles, mais ils ne les mettent point en pratique.
\VS{33}Mais quand ces choses arriveront, et voici, elles arrivent, ils sauront qu'il y avait un prophète au milieu d'eux.
\Chap{34}
\TextTitle{Jugement de Dieu sur les faux bergers}
\VerseOne{}La parole de Yahweh me fut encore adressée en ces mots :
\VS{2}Fils de l'homme, prophétise contre les pasteurs d'Israël\FTNT{Les faux pasteurs prennent en otage les brebis du Seigneur (Jé. 23). La véritable fonction pastorale consiste au service envers les frères et sœurs et non le contraire. Un vrai pasteur sert les autres, il n'aime pas être servi comme un roi. Il ne dit pas aux autres de faire les choses, mais il les fait et les autres l'imitent (Jn. 10).} ! Prophétise, et dis à ces pasteurs : Ainsi parle le Seigneur Yahweh : Malheur aux pasteurs d'Israël ! Qui ne paissent qu'eux-mêmes ! Les pasteurs ne paissent-ils pas le troupeau ?
\VS{3}Vous en mangez la graisse, et vous vous habillez de laine ; vous tuez ce qui est gras, vous ne paissez point le troupeau !
\VS{4}Vous n'avez point fortifié les brebis languissantes, vous n'avez point donné de remède à celle qui était malade, vous n'avez point bandé la plaie de celle qui avait la jambe rompue, vous n'avez point ramené celle qui était chassée, et vous n'avez point cherché celle qui était perdue\FTNT{Lu. 15:4-6 ; 1 Pi. 5:1-3.} ; mais vous les avez maîtrisées avec dureté et rigueur.
\VS{5}Elles se sont dispersées, parce qu'elles n'avaient pas de pasteurs, et elles se sont exposées à toutes les bêtes des champs, pour en être dévorées, étant dispersées.
\VS{6}Mes brebis sont errantes sur toutes les montagnes, et sur toutes les collines élevées ; mes brebis sont dispersées sur toute la surface de la terre ; et il n'y a personne qui les recherche, et il n'y a personne pour s'en soucier\FTNT{Mc. 14:27 ; Za. 13:7 ; Mt. 26:31.}.
\VS{7}C'est pourquoi pasteurs, écoutez la parole de Yahweh :
\VS{8}Je suis vivant, dit le Seigneur Yahweh, parce que mes brebis sont pillées, et que mes brebis sont la nourriture de toutes les bêtes des champs, parce qu'elles n'ont point de pasteur ; car mes pasteurs n'ont point recherché mes brebis, mais les pasteurs se sont nourris simplement eux-mêmes, et n'ont point fait paître mes brebis.
\VS{9}C'est pourquoi pasteurs, écoutez la parole de Yahweh !
\VS{10}Ainsi parle le Seigneur Yahweh : Voici, j'en veux à ces pasteurs-là, et je redemanderai mes brebis de leur main ; ils cesseront de paître les brebis, et les pasteurs ne se repaîtront plus eux-mêmes, mais je délivrerai mes brebis de leur bouche, et elles ne seront plus dévorées par eux.
\TextTitle{Yahweh, le bon berger qui restaure son troupeau\FTNT{Jn. 10:1-18}}
\VS{11}Car ainsi parle le Seigneur Yahweh : Me voici, je redemanderai mes brebis, et je les rechercherai.
\VS{12}Comme le pasteur prend soin de son troupeau quand il est au milieu de ses brebis dispersées, ainsi je rechercherai mes brebis, et les retirerai de tous les lieux où elles auront été dispersées au jour des nuages et de l'obscurité.
\VS{13}Je les retirerai d'entre les peuples et les rassemblerai des territoires, les ramènerai dans leur terre, et les nourrirai sur les montagnes d'Israël, auprès des cours d'eau et dans toutes les demeures du pays.
\VS{14}Je les paîtrai dans de bons pâturages, et leur demeure sera sur les hautes montagnes d'Israël ; et là elles coucheront dans une agréable demeure, et paîtront dans de gras pâturages, sur les montagnes d'Israël.
\VS{15}Moi-même je paîtrai mes brebis et les ferai reposer, dit le Seigneur Yahweh\FTNT{Ps. 23.}.
\VS{16}Je chercherai celle qui était perdue, et je ramènerai celle qui était chassée, je banderai la plaie de celle qui a la jambe rompue, et je fortifierai celle qui est malade ; mais je détruirai la grasse et la forte ; je les paîtrai avec justice.
\VS{17}Quant à vous, mes brebis, ainsi parle le Seigneur Yahweh : Voici, je m'en vais mettre à part les brebis, les béliers, et les boucs.
\VS{18}Et vous, est-ce peu de chose de vous faire paître dans de bons pâturages, pour que vous fouliez de vos pieds le reste de votre pâture ? Et de boire des eaux claires, pour que vous troubliez le reste avec vos pieds ?
\VS{19}Mais mes brebis sont nourries du pâturage que vous foulez de vos pieds, et boivent ce que vos pieds ont troublé.
\VS{20}C'est pourquoi le Seigneur Yahweh leur dit : Me voici, je mettrai moi-même à part la brebis grasse et la brebis maigre.
\VS{21}Parce que vous poussez du côté et de l'épaule, et que vous heurtez de vos cornes toutes celles qui sont languissantes, jusqu'à ce que vous les ayez chassées dehors,
\VS{22}je sauverai mes brebis, au point qu'elles ne seront plus au pillage. Voici, je jugerai entre brebis et brebis.
\VS{23}Je susciterai sur elles un pasteur qui les paîtra, mon serviteur David ; il les paîtra, et lui-même sera leur pasteur.
\VS{24}Moi, Yahweh, je serai leur Dieu, et mon serviteur David sera prince au milieu d'elles ; moi,Yahweh, j'ai parlé.
\VS{25}Je traiterai avec elles une alliance de paix ; et je détruirai dans le pays les mauvaises bêtes ; les brebis habiteront dans le désert en sécurité, et dormiront dans les forêts.
\VS{26}Je les comblerai de bénédictions, elles, et tous les environs de mes collines ; je ferai tomber la pluie en sa saison ; ce seront des pluies de bénédiction.
\VS{27}Les arbres des champs produiront leur fruit, et la terre rapportera son revenu ; elles seront dans leur terre en sécurité, et sauront que je suis Yahweh, quand j'aurai rompu les bois de leur joug, et que je les aurai délivrées de la main de ceux qui se les asservissent.
\VS{28}Elles ne seront plus au pillage parmi les nations, et les bêtes de la terre ne les dévoreront plus ; mais elles habiteront en sécurité, et il n'y aura personne pour les effrayer.
\VS{29}Je leur susciterai une plantation de renom ; elles ne mourront plus de faim sur la terre, et ne porteront plus l'opprobre des nations.
\VS{30}Ils sauront que moi, Yahweh, leur Dieu, suis avec eux, et qu'eux, la maison d'Israël, sont mon peuple, dit le Seigneur Yahweh.
\VS{31}Or vous êtes mes brebis, vous hommes, les brebis de mon pâturage, et je suis votre Dieu, dit le Seigneur Yahweh.
\Chap{35}
\TextTitle{Jugement sur Edom}
\VerseOne{}La parole de Yahweh me fut encore adressée en ces mots :
\VS{2}Fils de l'homme, tourne ta face contre la montagne de Séir, et prophétise contre elle\FTNT{Am. 1:11.}.
\VS{3}Dis-lui : Ainsi parle le Seigneur Yahweh : Voici, j'en veux à toi, montagne de Séir, et j'étendrai ma main contre toi, et te réduirai en désolation et en désert.
\VS{4}Je réduirai tes villes en désert, tu ne seras que désolation, et tu sauras que je suis Yahweh.
\VS{5}Parce que tu as eu une inimitié immortelle, et que tu as fait couler le sang des fils d'Israël à coups d'épée, au temps de leur détresse, au temps où l'iniquité était à son terme\FTNT{Ps. 137:7.}.
\VS{6}C'est pourquoi je suis vivant, dit le Seigneur Yahweh, je te mettrai à sang, et le sang te poursuivra ; parce que tu n'as point haï le sang, le sang aussi te poursuivra.
\VS{7}Je réduirai la montagne de Séir en désolation et en désert, et j'en éloignerai tous ceux qui la fréquentaient.
\VS{8}Je remplirai de morts ses montagnes ; tes hommes tués par l'épée tomberont sur tes collines, dans tes vallées, et dans tous tes courants d'eau.
\VS{9}Je te réduirai en désolations éternelles, et tes villes ne seront plus habitées ; vous saurez que je suis Yahweh.
\VS{10}Parce que tu as dit : Les deux nations, et les deux pays seront à moi, nous les posséderons, quand même Yahweh était là ;
\VS{11}à cause de cela, je suis vivant, dit le Seigneur Yahweh, j'agirai avec la colère et la jalousie que tu as montrées dans ta haine contre eux ; et je me ferai connaître au milieu d'eux, quand je te jugerai.
\VS{12}Tu sauras que moi, Yahweh, j'ai entendu toutes les paroles insultantes que tu as prononcées contre les montagnes d'Israël, en disant : Elles sont dévastées, elles nous sont livrées comme une proie.
\VS{13}Vous m'avez bravé par vos discours, et vous avez multiplié vos paroles contre moi ; je l'ai entendu.
\VS{14}Ainsi parle le Seigneur Yahweh : Quand toute la terre se réjouira, je te réduirai en désolation.
\VS{15}Comme tu t'es réjouie sur l'héritage de la maison d'Israël et de sa désolation, j'en ferai de même envers toi ; tu ne seras que désolation, ô montagne de Séir ! Ainsi qu'Edom tout entier ; et ils sauront que je suis Yahweh\FTNT{Ab. 1:11-16.}.
\Chap{36}
\TextTitle{Yahweh rétablit Israël}
\VerseOne{}Toi, fils de l'homme, prophétise sur les montagnes d'Israël, et dis : Montagnes d'Israël, écoutez la parole de Yahweh !
\VS{2}Ainsi parle le Seigneur Yahweh : Parce que l'ennemi a dit contre vous : Ah ! Ah ! Tous ces hauts lieux éternels sont devenus notre possession !
\VS{3}Prophétise, et dis : Ainsi parle le Seigneur Yahweh : Oui, parce qu'on vous a réduites en désolation, et que de toutes parts, on vous a englouties pour que vous soyez la propriété des autres nations, et qu'on vous a exposées à la langue et aux insultes des nations,
\VS{4}à cause de cela, montagnes d'Israël, écoutez la parole du Seigneur Yahweh : Ainsi parle le Seigneur Yahweh, aux montagnes, aux collines, aux courants d'eau, aux vallées, aux lieux détruits et désolés, et aux villes abandonnées qui sont pillées et sont un sujet de moquerie aux autres nations d'alentour ;
\VS{5}à cause de cela, ainsi parle le Seigneur Yahweh : Je parle dans le feu de ma jalousie contre les autres nations, et contre tous ceux d'Edom qui se sont attribués ma terre en possession, avec toute la joie de leur cœur et le mépris de leur âme, afin d'en piller le butin\FTNT{Lé. 25:23 ; Es. 14:2 ; Jé. 2:7.}.
\VS{6}C'est pourquoi prophétise sur la terre d'Israël, et dis aux montagnes et aux collines, aux courants d'eau et aux vallées : Ainsi parle le Seigneur Yahweh : Voici, j'ai parlé avec jalousie, et avec fureur, parce que vous avez porté l'ignominie des nations.
\VS{7}C'est pourquoi ainsi parle le Seigneur Yahweh : J'ai levé ma main, si les nations qui sont tout autour de vous ne portent leur ignominie.
\VS{8}Mais vous, montagnes d'Israël, vous pousserez vos branches, et vous porterez votre fruit pour mon peuple d'Israël ; car ils sont prêts à venir.
\VS{9}Car me voici, je viens à vous, et je retournerai vers vous, et vous serez labourées et semées.
\VS{10}Je mettrai sur vous des hommes en grand nombre, la maison d'Israël tout entière, et les villes seront habitées, les lieux déserts seront rebâtis.
\VS{11}Je multiplierai sur vous les hommes et les animaux, ils multiplieront et seront féconds ; je veux que vous soyez habitées comme auparavant, et je vous ferai plus de bien que vous n'en avez eu au commencement ; et vous saurez que je suis Yahweh.
\VS{12}Je ferai marcher sur vous des hommes, mon peuple d'Israël, qui vous posséderont, vous serez leur héritage, et vous ne les consumerez plus.
\VS{13}Ainsi parle le Seigneur Yahweh : Parce qu'on dit de vous : Tu es un pays qui dévore les hommes, et tu as consumé tes habitants ;
\VS{14}à cause de cela, tu ne dévoreras plus les hommes et ne consumeras plus tes habitants, dit le Seigneur Yahweh.
\VS{15}Je ne te ferai plus entendre l'ignominie des nations, tu ne porteras plus l'opprobre des peuples ; et tu ne feras plus périr tes habitants, dit le Seigneur Yahweh.
\VS{16}Puis la parole de Yahweh me fut adressée en ces mots :
\VS{17}Fils de l'homme, ceux de la maison d'Israël habitant sur leur terre l'ont souillée par leur voie et par leurs actions ; leur voie est devenue devant moi comme la souillure d'une femme pendant son impureté\FTNT{Lé. 12:2 ; Lé. 15:19.} ;
\VS{18}j'ai répandu ma fureur sur eux à cause du sang qu'ils ont répandu sur le pays, et parce qu'ils l'ont souillé par leurs idoles.
\VS{19}Je les ai dispersés parmi les nations, et ils ont été disséminés en divers pays ; je les ai jugés selon leur voie, et selon leurs actions.
\VS{20}Ils sont arrivés chez les nations où ils allaient, ils ont profané mon saint Nom en sorte qu'on disait d'eux : Ceux-ci sont le peuple de Yahweh, c'est de son pays qu'ils sont sortis\FTNT{Ro. 2:24.}.
\VS{21}Mais j'ai épargné mon saint Nom, que la maison d'Israël avait profané parmi les nations où elle est allée.
\VS{22}C'est pourquoi dis à la maison d'Israël : Ainsi parle le Seigneur Yahweh : Je ne le fais point à cause de vous, ô maison d'Israël ! Mais à cause de mon saint Nom, que vous avez profané parmi les nations où vous êtes allés\FTNT{De. 7:7 ; De. 9:5 ; Ps. 25:11 ; Es. 43:25.}.
\VS{23}Je sanctifierai mon grand nom, qui a été profané parmi les nations, et que vous avez profané au milieu d'elles ; et les nations sauront que je suis Yahweh, dit le Seigneur Yahweh, quand je serai sanctifié par vous, sous leurs yeux.
\VS{24}Je vous retirerai d'entre les nations, je vous rassemblerai de tous les pays, et je vous ramènerai dans votre terre.
\VS{25}Je répandrai sur vous une eau pure\FTNT{Il est question ici de la Nouvelle Alliance (Jé. 31:31-34 ; Hé. 8:7-13).}, et vous serez nettoyés ; je vous nettoierai de toutes vos souillures et de toutes vos idoles.
\TextTitle{Prophétie sur la naissance d'en haut}
\VS{26}Je vous donnerai un nouveau cœur, je mettrai au dedans de vous un Esprit nouveau ; j'ôterai de votre chair le cœur de pierre, et je vous donnerai un cœur de chair\FTNT{Jé. 32:39 ; 2 Co. 3:3 ; Ez. 11:19.}.
\VS{27}Je mettrai mon Esprit au dedans de vous, je ferai en sorte que vous suiviez mes ordonnances, et que vous observiez et pratiquiez mes lois.
\VS{28}Vous habiterez le pays que j'ai donné à vos pères, vous serez mon peuple, et je serai votre Dieu.
\VS{29}Je vous délivrerai de toutes vos souillures, j'appellerai le blé, je le multiplierai, et je ne vous enverrai plus la famine.
\VS{30}Je multiplierai le fruit des arbres et le revenu des champs, afin que vous ne portiez plus l'opprobre de la famine parmi les nations.
\VS{31}Vous vous souviendrez de votre mauvaise voie et de vos actions, qui n'étaient pas bonnes, et vous prendrez vous-mêmes en dégoût vos iniquités et vos abominations.
\VS{32}Je ne le fais point par amour pour vous, dit le Seigneur Yahweh ; sachez-le ! Soyez honteux et confus à cause de votre voie, ô maison d'Israël !
\VS{33}Ainsi parle le Seigneur Yahweh : Le jour où je vous aurai purifiés de toutes vos iniquités, je vous ferai habiter dans des villes, et les lieux déserts seront rebâtis.
\VS{34}La terre désolée sera cultivée, tandis qu'elle n'était que désolation aux yeux de tous les passants.
\VS{35}On dira : Cette terre-ci qui était désolée est devenue comme le jardin d'Eden ; et ces villes qui étaient désertes, désolées, et détruites, sont fortifiées et habitées\FTNT{Jé. 22:8-9 ; Es. 33:20.}.
\VS{36}Les nations qui resteront autour de vous sauront que moi, Yahweh, j'ai rebâti les lieux détruits et planté le pays désolé ; moi, Yahweh, j'ai parlé, et je le ferai.
\VS{37}Ainsi parle le Seigneur Yahweh : Je me laisserai rechercher par la maison d'Israël. Voici ce que je ferai pour eux : Je multiplierai les hommes comme un troupeau de brebis.
\VS{38}Les villes qui sont désertes seront remplies de troupeaux d'hommes, pareils aux troupeaux consacrés, aux troupeaux qu'on amène à Jérusalem pendant ses fêtes solennelles ; et ils sauront que je suis Yahweh.
\Chap{37}
\TextTitle{Vision des ossements desséchés, image de la restauration d'Israël}
\VerseOne{}La main de Yahweh fut sur moi, et Yahweh me transporta par son Esprit et me déposa au milieu d'une vallée remplie d'ossements\FTNT{Les ossements desséchés représentent les Israélites dispersés dans les nations.}.
\VS{2}Il me fit passer auprès d'eux, tout autour ; et voici, ils étaient fort nombreux à la surface de cette vallée et complètement secs.
\VS{3}Puis il me dit : Fils de l'homme, ces os pourront-ils revivre ? Et je répondis : Seigneur Yahweh, tu le sais.
\VS{4}Alors il me dit : Prophétise sur ces os, et dis-leur : Ossements desséchés, écoutez la parole de Yahweh !
\VS{5}Ainsi parle le Seigneur Yahweh à ces os : Voici, je ferai entrer un esprit en vous, et vous vivrez\FTNT{Ro. 8:11 ; Ps. 71:20.};
\VS{6}je mettrai des nerfs sur vous, je ferai croître de la chair sur vous, et j'étendrai la peau sur vous ; puis je mettrai un esprit en vous, et vous vivrez. Et vous saurez que je suis Yahweh.
\VS{7}Alors je prophétisai selon l'ordre que j'avais reçu. Et comme je prophétisais, il se fit un bruit, et voici, il se fit un mouvement, et ces os s'approchèrent les uns des autres.
\VS{8}Puis je regardai, et voici, il vint des nerfs sur eux, et il y crût de la chair, la peau fut étendue par dessus ; mais il n'y avait pas en eux d'esprit.
\VS{9}Alors il me dit : Prophétise à l'Esprit ! Prophétise, fils de l'homme ! Et dis à l'Esprit : Ainsi parle le Seigneur Yahweh : Esprit, viens des quatre vents, et souffle sur ces morts, et qu'ils revivent !
\VS{10}Je prophétisai donc selon l'ordre qu'il m'avait donné. Et l'Esprit entra en eux, ils reprirent vie, et se tinrent sur leurs pieds ; c'était une armée extrêmement grande.
\VS{11}Alors il me dit : Fils de l'homme, ces os sont toute la maison d'Israël ; voici, ils disent : Nos os sont desséchés, et notre attente est perdue, c'en est fait de nous !
\VS{12}C'est pourquoi prophétise, et dis-leur : Ainsi parle le Seigneur Yahweh : Mon peuple, voici, je m'en vais ouvrir vos sépulcres, je vous tirerai hors de vos sépulcres, et vous ferai entrer dans la terre d'Israël\FTNT{Les sépulcres représentent les nations dans lesquelles les Israélites se sont établis. Dieu annonce le retour de son peuple sur la terre d'Israël. Es. 26:19 ; Os. 13:14.}.
\VS{13}Et vous, mon peuple, vous saurez que je suis Yahweh quand j'aurai ouvert vos sépulcres, et que je vous aurai tirés hors de vos sépulcres.
\VS{14}Je mettrai mon Esprit en vous, et vous vivrez, je vous rétablirai sur votre terre ; et vous saurez que moi, Yahweh, j'ai parlé et que je l'ai fait, dit Yahweh.
\TextTitle{Prophétie sur l'unité d'Israël}
\VS{15}Puis la parole de Yahweh me fut adressée en ces mots :
\VS{16}Et toi, fils de l'homme, prends un bois et écris dessus : Pour Juda, et pour les fils d'Israël ses compagnons. Prends encore un autre bois, et écris dessus : Le bois d'Ephraïm et de toute la maison d'Israël, ses compagnons, pour Joseph.
\VS{17}Puis tu les joindras l'un à l'autre pour ne former qu'un même bois, ils seront unis dans ta main.
\VS{18}Quand les fils de ton peuple demanderont, en disant : Ne nous déclareras-tu pas ce que tu veux dire par ces choses ?
\VS{19}Dis-leur : Ainsi parle le Seigneur Yahweh : Voici, je m'en vais prendre le bois de Joseph qui est dans la main d'Ephraïm, et des tribus d'Israël, ses compagnons ; je les joindrai au bois de Juda, et j'en formerai un seul bois, ils ne seront qu'un seul bois dans ma main.
\VS{20}Ainsi les bois sur lesquels tu écriras seront dans ta main, sous leurs yeux.
\VS{21}Dis-leur : Ainsi parle le Seigneur Yahweh : Voici, je m'en vais prendre les fils d'Israël d'entre les nations parmi lesquelles ils sont allés, je les rassemblerai de toutes parts, et je les ferai entrer dans leur terre.
\VS{22}Je ferai d'eux une seule nation dans le pays, sur les montagnes d'Israël ; un seul roi sera leur roi à tous, ils ne seront plus deux nations, et ils ne seront plus divisés en deux royaumes\FTNT{Os. 2:2 ; Es. 11:12-13 ; Jn. 10:16.}.
\VS{23}Ils ne se souilleront plus par leurs idoles, ni par leurs infamies, ni par tous leurs crimes, et je les retirerai de toutes leurs demeures dans lesquelles ils ont péché, et je les purifierai ; ils seront mon peuple, et je serai leur Dieu\FTNT{Es. 1:18 ; Jé. 33:8 ; Jé. 24:7 ; Jé. 32:38 ; Za. 8:8 ; 2 Co. 6:16.}.
\VS{24}David, mon serviteur, sera leur roi, et ils auront tous un seul pasteur ; ils suivront mes ordonnances, ils garderont mes lois et les mettront en pratique.
\VS{25}Ils habiteront dans le pays que j'ai donné à Jacob, mon serviteur, dans lequel vos pères ont habité ; ils y habiteront, dis-je, eux, et leurs fils, et les fils de leurs fils, pour toujours ; et David mon serviteur sera leur prince pour toujours.
\VS{26}Je traiterai avec eux une alliance de paix, et il y aura une alliance éternelle avec eux ; je les établirai, et les multiplierai, je mettrai mon lieu saint au milieu d'eux pour toujours.
\VS{27}Ma demeure sera parmi eux ; je serai leur Dieu, et ils seront mon peuple.
\VS{28}Les nations sauront que je suis Yahweh qui sanctifie Israël, quand mon lieu saint sera au milieu d'eux pour toujours.
\Chap{38}
\TextTitle{Jugement sur Gog}
\VerseOne{}La parole de Yahweh me fut encore adressée en ces mots :
\VS{2}Fils de l'homme, tourne ta face vers Gog au pays de Magog\FTNT{Gog est un prince et Magog le pays. Ce chapitre doit être mis en parallèle avec Za. 12:1-4 ; Za. 14:1-9 ; Mt. 24:14-30 ; Ap. 14:14-20 ; Ap. 20:8.}, vers le prince de Rosch, de Méschec et de Tubal, et prophétise contre lui !
\VS{3}Tu diras : Ainsi parle le Seigneur Yahweh : Voici, j'en veux à toi, Gog, prince des chefs de Méschec et de Tubal !
\VS{4}Je te ferai retourner en arrière, et je mettrai des boucles dans tes mâchoires, et te ferai sortir avec toute ton armée, avec les chevaux, et les cavaliers, tous parfaitement bien équipés, une grande multitude portant le grand et le petit bouclier, et tous maniant l'épée ;
\VS{5}ceux de Perse, d'Ethiopie, et de Puth avec eux, qui tous ont des boucliers et des casques.
\VS{6}Gomer et toutes ses troupes, la maison de Togarma à l'extrême nord, avec toutes ses troupes, et plusieurs peuples avec toi.
\VS{7}Apprête-toi, tiens-toi prêt, toi, et toute la multitude assemblée autour de toi ! Sois leur chef !
\VS{8}Après plusieurs jours, tu seras à leur tête, et dans la suite des années, tu marcheras contre le pays dont les habitants, délivrés de l'épée, auront été rassemblés d'entre plusieurs peuples sur les montagnes d'Israël longtemps désertes ; retirés du milieu des peuples, ils seront en sécurité dans leurs demeures.
\VS{9}Tu monteras, tu viendras comme une dévastation, tu seras comme une nuée pour couvrir la terre, toi, toutes tes troupes, et plusieurs peuples avec toi\FTNT{Da. 11:40.}.
\VS{10}Ainsi parle le Seigneur Yahweh : Il arrivera dans ces jours-là que des pensées s'élèveront dans ton cœur, et que tu formeras un dessein pernicieux.
\VS{11}Car tu diras : Je monterai contre le pays dont les villes sont sans murailles ; j'envahirai ceux qui sont en repos, qui habitent en sécurité, qui demeurent tous dans des villes sans murs, lesquelles n'ont ni barres ni portes\FTNT{Jé. 49:31.} ;
\VS{12}pour enlever un grand butin et faire un grand pillage ; pour remettre ta main sur les déserts qui de nouveau étaient habités, et sur le peuple rassemblé d'entre les nations, ayant des troupeaux et des biens, et occupant les lieux élevés du pays.
\VS{13}Séba, et Dedan, les marchands de Tarsis, et tous ses lionceaux, te diront : Ne vas-tu pas pour faire du butin, et n'as-tu pas assemblé ta multitude pour faire un grand pillage, pour emporter de l'argent et de l'or, pour prendre le bétail et les biens, pour enlever un grand butin ?
\VS{14}Toi donc, fils de l'homme, prophétise, et dis à Gog : Ainsi parle le Seigneur Yahweh : En ce jour-là, quand mon peuple d'Israël habitera en sécurité, ne le sauras-tu pas ?
\VS{15}Ne viendras-tu pas de ton lieu, de l'extrême nord, toi, et plusieurs peuples avec toi, tous montés sur des chevaux, une grande multitude, et une grosse armée ?
\VS{16}Ne monteras-tu pas contre mon peuple d'Israël, comme une nuée pour couvrir la terre ? Dans la suite de ces jours, je te ferai venir sur ma terre, afin que les nations me connaissent, quand je serai sanctifié par toi sous leurs yeux, ô Gog !
\VS{17}Ainsi parle le Seigneur Yahweh : N'est-ce pas de toi que j'ai parlé autrefois par le ministère de mes serviteurs, les prophètes d'Israël, qui ont prophétisé dans ces jours-là pendant plusieurs années, qu'on te ferait venir contre eux ?
\VS{18}Mais il arrivera dans ce jour-là, au jour de la venue de Gog sur la terre d'Israël, dit le Seigneur Yahweh, que ma colère éclatera.
\VS{19}Je le déclare, dans ma jalousie, dans l'ardeur de ma fureur, en ce jour-là, il y aura une grande agitation sur la terre d'Israël.
\VS{20}Les poissons de la mer, les oiseaux des cieux, et les bêtes des champs, et tous les reptiles qui rampent sur la terre, et tous les hommes qui sont sur la surface de la terre seront épouvantés par ma présence ; les montagnes seront renversées, les parois des rochers tomberont, et tous les murs chuteront par terre.
\VS{21}J'appellerai contre lui l'épée sur toutes mes montagnes, dit le Seigneur Yahweh ; l'épée de chacun d'eux sera contre son frère.
\VS{22}J'entrerai en jugement avec lui par la peste, et par le sang ; je ferai pleuvoir sur lui, sur ses troupes, et sur les grands peuples qui seront avec lui, des torrents d'eau, des pierres de grêle, du feu et du soufre\FTNT{Ap. 8:7 ; Ps. 11:6 ; Ap. 16:21 ; Ap. 11:19.}.
\VS{23}Je me glorifierai, je me sanctifierai, je serai connu aux yeux de plusieurs nations ; et elles sauront que je suis Yahweh.
\Chap{39}
\TextTitle{Jugement sur Gog, suite}
\VerseOne{}Toi donc, fils de l'homme, prophétise contre Gog, et dis : Ainsi parle le Seigneur Yahweh : Voici, j'en veux à toi, Gog, prince des chefs de Méschec et de Tubal !
\VS{2}Je te ferai retourner en arrière, je te conduirai, je te ferai monter de l'extrême nord, et je t'amènerai sur les montagnes d'Israël.
\VS{3}Car je frapperai ton arc dans ta main gauche, et je ferai tomber tes flèches de ta main droite.
\VS{4}Tu tomberas sur les montagnes d'Israël, toi et toutes tes troupes, et les peuples qui seront avec toi ; je te livrerai aux oiseaux de proie, à tout ce qui a des ailes, et aux bêtes des champs, pour en être dévoré.
\VS{5}Tu tomberas sur la face des champs, parce que j'ai parlé, dit le Seigneur Yahweh.
\VS{6}Je mettrai le feu dans Magog, et parmi ceux qui demeurent en sécurité dans les îles ; et ils sauront que je suis Yahweh.
\VS{7}Je ferai connaître mon saint Nom au milieu de mon peuple d'Israël ; et je ne profanerai plus mon saint Nom ; les nations sauront que je suis Yahweh, le Saint d'Israël.
\VS{8}Voici, cela arrive et sera fait, dit le Seigneur Yahweh ; c'est ici le jour dont j'ai parlé.
\VS{9}Les habitants des villes d'Israël sortiront, allumeront le feu, brûleront les armes, les petits et les grands boucliers, les arcs, les flèches, les bâtons qu'on lance de la main, et les javelots ; ils en feront du feu pendant sept ans.
\VS{10}On n'apportera point du bois des champs, et on n'en coupera point dans les forêts, parce qu'ils feront du feu de ces armes, lorsqu'ils dépouilleront ceux qui les avaient dépouillés, et qu'ils pilleront ceux qui les avaient pillés, dit le Seigneur Yahweh.
\VS{11}Il arrivera ce jour-là que je donnerai à Gog dans ces quartiers-là un lieu pour sépulcre en Israël, à savoir la vallée des passants, qui est au-devant de la mer ; elle réduira les passants au silence ; on enterrera là Gog, et toute la multitude de son peuple, et on l'appellera la vallée d'Hamon-Gog\FTNT{La vallée d'Hamon-Gog : La vallée de la multitude de Gog.}.
\VS{12}Ceux de la maison d'Israël les enterreront, et cela durera sept mois, afin de purifier le pays.
\VS{13}Tout le peuple du pays les enterrera, et il en aura du renom, le jour où je serai glorifié, dit le Seigneur Yahweh.
\VS{14}Ils mettront à part des gens qui ne feront autre chose que parcourir le pays, et qui enterreront, avec l'aide des passants, les corps restés à la surface de la terre, pour la purifier, et ils seront à la recherche pendant sept mois.
\VS{15}Ils parcourront le pays, et celui qui verra l'os d'un homme, dressera auprès de lui un signal ; jusqu'à ce que les fossoyeurs l'aient enterré dans la vallée d'Hamon-Gog.
\VS{16}Il y aura aussi une ville nommée Hamona\FTNT{Hamona signifie « multitude ».}, et on nettoiera le pays.
\VS{17}Toi donc, fils de l'homme, ainsi parle le Seigneur Yahweh : Dis aux oiseaux de toutes espèces, et à toutes les bêtes des champs : Assemblez-vous et venez ; amassez-vous de toutes parts vers mon sacrifice que je fais pour vous, qui est un grand sacrifice sur les montagnes d'Israël ! Vous mangerez de la chair, et vous boirez du sang.
\VS{18}Vous mangerez la chair des hommes puissants, et vous boirez le sang des princes de la terre, le sang des moutons, des agneaux, des boucs, et des veaux engraissés sur le Basan\FTNT{Es. 34:6 ; Jé. 46:10 ; So. 1:7 ; Mt. 24:28 ; Job. 39:33.}.
\VS{19}Vous mangerez de la graisse jusqu'à en être rassasiés, et vous boirez du sang jusqu'à en être ivres, de la graisse et du sang de mon sacrifice, que j'aurai sacrifié pour vous.
\VS{20}Vous serez rassasiés à ma table, de chevaux et de bêtes d'attelage, d'hommes forts, et de tous hommes de guerre, dit le Seigneur Yahweh.
\VS{21}Je mettrai ma gloire parmi les nations, et toutes les nations verront mon jugement que j'aurai exercé, et comment j'aurai mis ma main sur eux.
\VS{22}La maison d'Israël connaîtra dès ce jour-là, et dans la suite, que je suis Yahweh, leur Dieu.
\VS{23}Les nations sauront que la maison d'Israël avait été emmenée en captivité à cause de son iniquité, parce qu'ils avaient péché contre moi, et que je leur avais caché ma face ; aussi je les avais livrés entre les mains de leurs ennemis pour qu'ils périssent par l'épée\FTNT{De. 31:17-18 ; Ps. 13:2.}.
\VS{24}Je leur avais fait selon leurs souillures, et selon leurs crimes, et je leur avais caché ma face.
\TextTitle{Rétablissement et conversion d'Israël}
\VS{25}C'est pourquoi, ainsi parle le Seigneur Yahweh : Maintenant, je ramènerai la captivité de Jacob, et j'aurai pitié de toute la maison d'Israël, et je serai jaloux de mon saint Nom,
\VS{26}après avoir porté leur ignominie, et tout leur crime, lorsqu'ils avaient péché contre moi, quand ils demeuraient en sûreté dans leur terre, sans qu'il y eût personne pour les effrayer.
\VS{27}Parce que je les ramènerai d'entre les peuples, que je les rassemblerai des pays de leurs ennemis, et que je serai sanctifié par eux, sous les yeux de plusieurs nations.
\VS{28}Ils sauront que je suis Yahweh, leur Dieu, lorsqu'après les avoir enlevés parmi les nations, je les rassemblerai sur leurs terres, et que je n'en laisserai chez elles aucun d'eux.
\VS{29}Je ne leur cacherai plus ma face, car je répandrai mon Esprit sur la maison d'Israël, dit le Seigneur Yahweh\FTNT{Joë. 2:28 ; Ac. 2:17.}.
\Chap{40}
\TextTitle{Mesures du futur temple}
\VerseOne{}Dans la vingt-cinquième année de notre captivité, au commencement de l'année, au dixième jour du mois, la quatorzième année après que la ville fut prise, en ce même jour, la main de Yahweh fut sur moi, et il m'amena là.
\VS{2}Il m'amena par des visions de Dieu, au pays d'Israël, et me posa sur une montagne fort élevée, sur laquelle du côté sud il y avait comme une ville construite.
\VS{3}Après qu'il m'y fît entrer, voici un homme, dont l'aspect était comme de l'airain, qui avait dans sa main un cordeau de lin, et une canne à mesurer, et qui se tenait debout à la porte.
\VS{4}Cet homme me parla ainsi : Fils de l'homme, regarde de tes yeux, écoute de tes oreilles, et applique ton cœur à toutes les choses que je m'en vais te faire voir, car tu as été amené ici afin que je te les fasse voir, et que tu fasses savoir à la maison d'Israël toutes les choses que tu vas voir.
\VS{5}Voici, un mur extérieur entourait la maison. Cet homme avait dans la main une canne à mesurer longue de six coudées, chaque coudée étant d'une coudée normale et une largeur de main en plus. Il mesura la largeur de ce mur bâti, laquelle était d'une canne, et sa hauteur d'une autre canne.
\VS{6}Puis il vint vers la porte orientale, et monta par ses étages. Il mesura l'un des poteaux de la porte d'une canne en largeur, et l'autre poteau d'une autre canne en largeur.
\VS{7}Puis il mesura chaque chambre d'une canne en longueur, et d'une canne en largeur. L'espace entre les deux chambres était de cinq coudées. Il mesura d'une canne chacun des poteaux de la porte près du vestibule qui menait à la porte la plus intérieure.
\VS{8}Puis il mesura d'une canne le vestibule qui menait à la porte la plus intérieure.
\VS{9}Il mesura de huit coudées le vestibule de la porte et ses poteaux, le vestibule de la porte était en dedans.
\VS{10}Les chambres de la porte orientale étaient au nombre de trois d'un côté et de trois de l'autre, toutes les trois avaient la même mesure.
\VS{11}Puis il mesura de dix coudées la largeur de l'ouverture de la première porte, et de treize coudées la longueur de la même porte.
\VS{12}Ensuite, il mesura d'un côté un espace limité au-devant des chambres d'une coudée, et une autre coudée d'espace limité de l'autre côté ; chaque chambre avait six coudées d'un côté, et six coudées de l'autre.
\VS{13}Après cela, il mesura le portail depuis le toit d'une chambre jusqu'au toit de l'autre, de la largeur de vingt-cinq coudées entre les deux ouvertures opposées.
\VS{14}Il compta soixante coudées pour les poteaux, près desquels était une cour, autour de la porte.
\VS{15}L'espace entre la porte d'entrée et le vestibule de la porte intérieure était de cinquante coudées.
\VS{16}Il y avait des fenêtres closes aux chambres et à leurs poteaux, à l'intérieur de la porte tout autour. Il y avait aussi des fenêtres dans les vestibules tout autour intérieurement, des palmes étaient sculptées sur les poteaux.
\VS{17}Il me mena dans le parvis extérieur, où se trouvaient des chambres et un pavé tout autour. Il y avait trente chambres sur ce pavé.
\VS{18}Le pavé était au côté des portes et répondait à la longueur des portes ; c'était le pavé inférieur.
\VS{19}Il mesura la largeur du parvis depuis la porte qui menait vers le bas jusqu'au parvis intérieur en dehors. Il y avait cent coudées à l'orient et au nord.
\VS{20}Après cela, il mesura la longueur et la largeur de la porte nord du parvis extérieur.
\VS{21}Quant aux chambres, au nombre de trois d'un côté et trois de l'autre, ses poteaux et ses vestibules avaient la même mesure que la première porte, cinquante coudées en longueur, et vingt-cinq coudées en largeur.
\VS{22}Ses fenêtres, son vestibule, et ses palmes avaient la même mesure que la porte orientale ; on y montait par sept étages, devant lesquels étaient son vestibule.
\VS{23}La porte du parvis intérieur était vis-à-vis de la première porte du nord, et vis-à-vis de la porte orientale. Il mesura depuis une porte jusqu'à l'autre cent coudées.
\VS{24}Après cela, il me conduisit du côté sud, où se trouvait la porte méridionale, il en mesura les poteaux et les vestibules qui avaient la même mesure.
\VS{25}Cette porte et ses vestibules avaient des fenêtres tout autour, comme les autres fenêtres, cinquante coudées de long, et vingt-cinq coudées de large.
\VS{26}On y montait par sept étages, devant lesquels était son vestibule ; il avait de chaque côté des palmes sur ses poteaux.
\VS{27}Pareillement, le parvis intérieur avait sa porte du côté sud ; il mesura d'une porte à l'autre au sud cent coudées.
\VS{28}Après cela il me fit entrer dans le parvis intérieur par la porte sud, et il mesura la porte sud, selon les mesures précédentes.
\VS{29}Ses chambres, ses poteaux et ses vestibules avaient la même mesure. Cette porte et ses vestibules avaient des fenêtres tout autour, cinquante coudées de long, et vingt-cinq coudées de large.
\VS{30}Il y avait tout autour des vestibules de vingt-cinq coudées de long, et cinq coudées de large.
\VS{31}Les vestibules de la porte aboutissaient au parvis extérieur ; il y avait des palmes sur ses poteaux, et huit étages pour y monter.
\VS{32}Il me conduisit dans le parvis intérieur, par l'entrée orientale. Il mesura la porte, qui avait la même mesure.
\VS{33}Ses chambres, ses poteaux et ses vestibules avaient la même mesure. Cette porte et ses vestibules avaient des fenêtres tout autour, cinquante coudées de long, et vingt-cinq de large.
\VS{34}Ses vestibules aboutissaient au parvis extérieur ; il y avait de chaque côté des palmes sur ses poteaux, et huit étages pour y monter.
\VS{35}Il me conduisit vers la porte nord. Il la mesura et trouva la même mesure.
\VS{36}Ainsi qu'à ses chambres, à ses poteaux et à ses vestibules ; elle avait des fenêtres tout autour, cinquante coudées de long, et vingt-cinq coudées de large.
\VS{37}Ses vestibules aboutissaient au parvis extérieur ; il y avait de chaque côté des palmes sur ses poteaux, et huit étages pour y monter.
\VS{38}Il y avait une chambre qui s'ouvrait vers les poteaux des portes, et où l'on devait laver les holocaustes.
\VS{39}Il y avait aussi dans le vestibule de la porte de chaque côté deux tables, pour y égorger les bêtes qu'on sacrifierait pour l'holocauste, et le sacrifice pour l'expiation et le sacrifice pour la culpabilité.
\VS{40}Vers l'un des côtés de la porte, au dehors, vers le lieu où l'on montait, à l'entrée de la porte nord, il y avait deux tables, et de l'autre côté, vers le vestibule de la porte, deux autres tables.
\VS{41}Il se trouvait ainsi, aux côtés de la porte, quatre tables d'une part, et quatre tables de l'autre, en tout huit tables, sur lesquelles on devait abattre les victimes.
\VS{42}Les quatre tables qui étaient pour l'offrande entièrement consumée, étaient en pierres de taille, de la longueur d'une coudée et demie, et de la largeur d'une coudée et demie, et de la hauteur d'une coudée ; et même on devait poser sur elles les instruments avec lesquels on tuait les victimes pour les offrandes entièrement consumées, et les autres sacrifices.
\VS{43}Il y avait aussi à l'intérieur de la maison tout autour, des chevilles pour accrocher, larges d'une paume, bien adaptées, d'où l'on apportait la chair des sacrifices sur les tables.
\TextTitle{Répartition des pièces du futur temple}
\VS{44}En dehors de la porte intérieure, il y avait des chambres pour les chantres dans le parvis intérieur, l'une était à côté de la porte nord et avait la face au sud, l'autre était à côté de la porte orientale et avait la face au nord.
\VS{45}Il me dit : Ces chambres, dont la face est au sud, sont pour les sacrificateurs qui ont la charge de la maison.
\VS{46}Mais ces chambres, dont la face est au nord, sont pour les sacrificateurs qui ont la charge de l'autel, qui sont les fils de Tsadok, qui, parmi les fils de Lévi, s'approchent de Yahweh pour faire son service.
\VS{47}Puis il mesura un parvis de la longueur et de la largeur de cent coudées, en carré ; et l'autel était devant la maison.
\VS{48}Ensuite, il me fit entrer dans le vestibule de la maison ; et il mesura les poteaux du vestibule de cinq coudées d'un côté, et de cinq coudées de l'autre, puis la largeur de la porte de trois coudées d'un côté, et de trois coudées de l'autre.
\VS{49}Le vestibule avait une longueur de vingt coudées, et une largeur de onze coudées ; on y montait par des étages. Il y avait des colonnes près des poteaux, l'une d'un côté, et l'autre de l'autre.
\Chap{41}
\TextTitle{Description du temple}
\VerseOne{}Puis il me fit entrer dans le temple, et il mesura des poteaux de six coudées de largeur d'un côté, et de six coudées de largeur de l'autre côté, largeur de la tente.
\VS{2}Ensuite il mesura la largeur de l'ouverture de la porte qui était de dix coudées, et les côtés de l'ouverture de cinq coudées, d'une part, et de cinq coudées de l'autre part. Puis il mesura la longueur du temple, quarante coudées, et la largeur, vingt coudées.
\VS{3}Il entra à l'intérieur, et il mesura un poteau d'une ouverture de porte, deux coudées, la hauteur de cette ouverture, six coudées, et la largeur de cette ouverture, sept coudées.
\VS{4}Puis il mesura une longueur de vingt coudées, et une largeur de vingt coudées en face du temple ; et il me dit : C'est ici le Saint des saints.
\VS{5}Il mesura l'épaisseur du mur de la maison, qui fut de six coudées, et la largeur des chambres qui étaient tout autour de la maison, de quatre coudées.
\VS{6}Les chambres latérales étaient les unes à côté des autres, au nombre de trente, et il y avait trois poutres ; elles entraient dans un mur construit pour ces chambres tout autour de la maison, elles y étaient appuyées sans entrer dans le mur même de la maison.
\VS{7}Les chambres occupaient plus d'espace, à mesure qu'elles s'élevaient, et l'on allait en tournant, car on montait autour de la maison par un escalier tournant. Il y avait plus d'espace dans le haut de la maison, et l'on montait de l'étage inférieur à l'étage supérieur par celui du milieu.
\VS{8}Je considérai la hauteur autour de la maison. Les chambres latérales, à partir de leur fondement avaient une canne pleine, six grandes coudées.
\VS{9}La largeur du mur extérieur des chambres latérales était de cinq coudées ; l'espace libre entre les chambres latérales de la maison,
\VS{10}et les chambres autour de la maison avait une largeur de vingt coudées.
\VS{11}L'ouverture des chambres latérales donnait sur l'espace libre, une ouverture au nord, et une autre ouverture au sud ; la largeur de l'espace libre était de cinq coudées tout autour.
\VS{12}Le bâtiment qui était devant la place vide, du côté de l'occident, avait une largeur de soixante-dix coudées, un mur de cinq coudées de largeur tout autour, et une longueur de quatre-vingt-dix coudées.
\VS{13}Il mesura la maison, qui avait cent coudées de longueur ; la place vide, le bâtiment et les murs avaient une longueur de cent coudées.
\VS{14}La largeur de la face de la maison et de la place vide, du côté oriental, était de cent coudées.
\VS{15}Il mesura la longueur du bâtiment devant la place vide, sur le derrière, et ses galeries de chaque côté : Il y avait cent coudées.
\VS{16}Les seuils, les fenêtres closes, les galeries du pourtour aux trois étages, en face des seuils, étaient recouverts de bois tout autour. Depuis le sol jusqu'aux fenêtres fermées,
\VS{17}jusqu'au-dessus des ouvertures, et jusqu'à la maison au-dedans comme au dehors, tout le mur du pourtour, à l'intérieur et à l'extérieur, tout était d'après la mesure,
\VS{18}et fait de chérubins et de palmes. Il y avait une palme entre deux chérubins, et chaque chérubin avait deux faces.
\VS{19}Une face d'homme était tournée vers la palme d'un côté, et une face de jeune lion était tournée vers la palme de l'autre côté ; il en était ainsi tout autour de la maison.
\VS{20}Depuis le sol jusqu'au-dessus des ouvertures il y avait des chérubins et des palmes et aussi sur le mur du temple.
\VS{21}Les poteaux du temple étaient carrés ; et la face du lieu saint avait la même apparence.
\VS{22}L'autel était de bois, de la hauteur de trois coudées, et de deux coudées de longueur ; ses angles, ses pieds et ses côtés étaient de bois. Puis il me dit : C'est ici la table qui est devant Yahweh.
\VS{23}Le temple et le lieu saint avaient deux portes.
\VS{24}Il y avait deux portes, deux battants, qui tous deux tournaient sur les portes, deux battants pour une porte et deux pour l'autre.
\VS{25}Il y avait aussi des chérubins et des palmes façonnés sur les portes du temple, comme sur les murs. Un entablement en bois était sur le front du vestibule en dehors.
\VS{26}Il y avait des fenêtres fermées, et des palmes de part et d'autre, ainsi qu'aux côtés du vestibule, aux chambres latérales de la maison, et aux entablements.
\Chap{42}
\TextTitle{Mesures supplémentaires du temple}
\VerseOne{}Après cela, il me fit sortir vers le parvis extérieur, du côté nord ; et il me conduisit vers les chambres qui étaient vis-à-vis de la place vide et vis-à-vis du bâtiment, au nord.
\VS{2}Sur la face où se trouvait une ouverture au nord, il y avait une longueur de cent coudées, et la largeur était de cinquante coudées.
\VS{3}C'était vis-à-vis des vingt coudées du parvis intérieur, et vis-à-vis du pavé extérieur, là où se trouvaient les galeries des trois étages.
\VS{4}Devant les chambres, il y avait une promenade large de dix coudées, et une voie d'une coudée ; leurs ouvertures donnaient au nord.
\VS{5}Les chambres supérieures étaient plus étroites que les inférieures et que celles du milieu du bâtiment parce que les galeries leur ôtaient de la place.
\VS{6}Car elles étaient à trois étages, et n'avaient point de colonnes, comme les colonnes des parvis ; c'est pourquoi à partir du sol, les chambres du haut étaient plus étroites que celles du bas et du milieu.
\VS{7}Le mur extérieur parallèle aux chambres, du côté du parvis extérieur devant les chambres, avait cinquante coudées de long.
\VS{8}Car la longueur des chambres du côté du parvis extérieur était de cinquante coudées. Mais sur la face du temple, il y avait cent coudées.
\VS{9}Au bas de ces chambres était l'entrée orientale quand on y venait du parvis extérieur.
\VS{10}Il y avait encore des chambres sur la largeur du mur du parvis du côté oriental, vis-à-vis de la place vide et vis-à-vis du bâtiment.
\VS{11}Devant elles, il y avait un chemin, comme devant les chambres qui étaient du côté nord. La longueur et la largeur étaient les mêmes ; leurs issues, leur disposition et leurs ouvertures étaient semblables.
\VS{12}Il en était de même pour les ouvertures des chambres du côté sud. Il y avait une ouverture à la tête du chemin, du chemin qui se trouvait droit devant le mur du côté oriental par où l'on y entrait.
\VS{13}Après cela, il me dit : Les chambres du parvis nord et les chambres du parvis sud, qui sont devant la place vide, ce sont les chambres du lieu saint où les sacrificateurs qui s'approchent de Yahweh, mangeront les choses très saintes. Ils déposeront là les choses très saintes, savoir les gâteaux, les offrandes pour l'expiation et les offrandes pour la culpabilité ; car ce lieu est saint.
\VS{14}Quand les sacrificateurs seront entrés, ils ne sortiront point du lieu saint pour venir au parvis extérieur, mais ils déposeront là leurs vêtements avec lesquels ils font le service ; car ces vêtements sont saints ; ils en mettront d'autres pour s'approcher du peuple.
\VS{15}Lorsqu'il eut achevé de mesurer la maison intérieure, il me fit sortir par la porte qui était du côté oriental, puis il mesura l'enceinte tout autour.
\VS{16}Il mesura le côté oriental avec la canne qui servait de mesure, et il y avait tout autour cinq cents cannes.
\VS{17}Ensuite il mesura le côté nord, avec la canne qui servait de mesure, et il y avait tout autour cinq cents cannes.
\VS{18}Puis il mesura le côté sud avec la canne qui servait de mesure, et il y avait cinq cents cannes.
\VS{19}Il se tourna du côté occidental, et mesura cinq cents cannes avec la canne qui servait de mesure.
\VS{20}Il mesura des quatre côtés le mur formant l'enceinte de la maison ; la longueur était de cinq cents cannes, et la largeur de cinq cents cannes, ce mur marquait la séparation entre le saint et le profane.
\Chap{43}
\TextTitle{La gloire de Yahweh remplit la maison\FTNTT{Cp. Ez. 11:22-24.}}
\VerseOne{}Puis il me ramena à la porte, à la porte qui était du côté oriental.
\VS{2}Et voici, la gloire du Dieu d'Israël s'avançait de l'orient, sa voix était pareille au bruit des grandes eaux, et la terre resplendissait de sa gloire\FTNT{Ap. 1:15.}.
\VS{3}La vision que j'eus alors était semblable à celle que j'avais vue lorsque j'étais venu pour détruire la ville, ces visions étaient comme la vision que j'avais vue sur le fleuve de Kebar ; et je me prosternai le visage contre terre.
\VS{4}Puis la gloire de Yahweh entra dans la maison par la porte qui était du côté oriental.
\VS{5}L'Esprit m'enleva et me fit entrer dans le parvis intérieur, et voici la gloire de Yahweh remplissait la maison.
\TextTitle{Le trône de Yahweh}
\VS{6}Je l'entendis s'adressant à moi depuis la maison, et l'homme qui me conduisait était debout près de moi.
\VS{7}Yahweh me dit : Fils de l'homme, c'est ici le lieu de mon trône, et le lieu des plantes de mes pieds, dans lequel je ferai ma demeure éternellement parmi les fils d'Israël ; et la maison d'Israël ne souillera plus mon saint Nom, ni eux, ni leurs rois, par leurs fornications ; mais ils souilleront leurs hauts lieux par les cadavres de leurs rois.
\VS{8}Car ils ont mis leur seuil près de mon seuil, et leur poteau près de mon poteau, il y avait un mur entre moi et eux ; ils ont souillé mon saint Nom par leurs abominations qu'ils ont faites, c'est pourquoi je les ai consumés dans ma colère.
\VS{9}Maintenant ils rejetteront loin de moi leurs adultères et les cadavres de leurs rois, et je ferai ma demeure éternellement parmi eux.
\VS{10}Toi donc, fils de l'homme, montre ce temple à la maison d'Israël ; et qu'ils soient confus à cause de leurs iniquités ; et qu'ils aient honte de leur iniquité.
\VS{11}S'ils rougissent de tout ce qu'ils ont fait, fais-leur connaître la forme de ce temple, sa disposition, avec ses sorties et ses entrées, toutes ses figures et toutes ses ordonnances, toutes ses formes, toutes ses lois, et écris-les sous leurs yeux, afin qu'ils gardent toutes ses formes, et toutes les ordonnances, et qu'ils les pratiquent.
\VS{12}Tel est la loi de la maison. Sur le sommet de la montagne, tout le territoire sera un lieu très saint tout autour. Voilà donc la loi de la maison.
\TextTitle{L'autel pour les holocaustes et les sacrifices}
\VS{13}Voici les mesures de l'autel, d'après les coudées dont chacune était d'une largeur de main plus longue que la coudée ordinaire. Le fond avait une coudée de hauteur et une coudée de largeur, et le rebord qui terminait son contour avait un empan de largeur ; c'était le dos de l'autel.
\VS{14}Depuis le fond sur le sol jusqu'à l'encadrement inférieur, il y avait deux coudées, et une coudée de largeur, et depuis le petit jusqu'au grand encadrement, il y avait quatre coudées et une coudée de largeur.
\VS{15}L'autel avait quatre coudées ; et quatre cornes s'élevaient de l'autel.
\VS{16}L'autel avait douze coudées de longueur, douze coudées de largeur, et formait un carré par ses quatre côtés.
\VS{17}L'encadrement avait quatorze coudées de longueur sur quatorze coudées de largeur à ses quatre côtés, le rebord qui terminait son contour avait une demi-coudée, le fond avait une coudée tout autour, et les étages étaient tournés vers l'orient.
\VS{18}Il me dit : Fils de l'homme, ainsi parle le Seigneur Yahweh : Ce sont ici les lois au sujet de l'autel pour le jour où on le fera, afin qu'on y offre l'holocauste, et qu'on y répande le sang.
\VS{19}C'est que tu donneras aux sacrificateurs, aux Lévites, qui sont de la race de Tsadok, et qui s'approchent de moi, dit le Seigneur Yahweh, afin qu'ils y fassent mon service, un jeune veau en sacrifice pour le péché.
\VS{20}Et tu prendras de son sang, et en mettras sur les quatre cornes de l'autel, et sur les quatre angles de l'encadrement et sur le rebord qui l'entoure, ainsi tu purifieras l'autel, et tu feras propitiation pour lui\FTNT{Ex. 29:36-39.}.
\VS{21}Tu prendras le jeune taureau expiatoire, et on le brûlera dans un lieu réservé de la maison, en dehors du lieu saint.
\VS{22}Le second jour, tu offriras en expiation un bouc, sans défaut, et on purifiera l'autel comme on l'aura purifié avec le jeune taureau.
\VS{23}Quand tu auras achevé de purifier l'autel, tu offriras un jeune taureau sans défaut, et un bélier du troupeau sans défaut.
\VS{24}Tu les offriras devant Yahweh, et les sacrificateurs jetteront du sel par dessus, et les offriront en holocauste à Yahweh\FTNT{Lé. 2:13.}.
\VS{25}Durant sept jours, tu sacrifieras chaque jour un bouc comme victime expiatoire, et les sacrificateurs sacrifieront un jeune taureau et un bélier du troupeau sans défaut.
\VS{26}Pendant sept jours, les sacrificateurs feront la propitiation pour l'autel, on le purifiera et chacun d'eux sera consacré\FTNT{Le terme « consacré » veut dire littéralement « remplir sa main ». Voir aussi Jg.17:5 et 17:12}
\VS{27}Lorsque ces jours seront accomplis, dès le huitième jour, et à l'avenir, les sacrificateurs offriront sur cet autel vos holocaustes et vos sacrifices d'offrande de paix. Et je serai apaisé envers vous, dit le Seigneur Yahweh.
\Chap{44}
\TextTitle{La porte fermée du sanctuaire}
\VerseOne{}Puis il me ramena vers la porte extérieure du lieu saint, du côté oriental, mais elle était fermée.
\VS{2}Yahweh me dit : Cette porte-ci sera fermée, et ne sera point ouverte, personne n'y passera, parce que Yahweh, le Dieu d'Israël, est entré par cette porte ; elle sera donc fermée\FTNT{Ap. 3:8.}.
\VS{3}Elle sera pour le prince ; le prince sera le seul qui s'y assiéra pour manger le pain devant Yahweh ; il entrera par le chemin du vestibule de la porte, et sortira par le même chemin.
\TextTitle{[La gloire dans la maison de Yahweh]}
\VS{4}Il me fit revenir par le chemin de la porte nord jusque sur le devant de la maison, je regardai, et voici, la gloire de Yahweh avait rempli la maison de Yahweh, et je me prosternai sur ma face.
\VS{5}Alors Yahweh me dit : Fils de l'homme, applique ton cœur, et regarde de tes yeux, écoute de tes oreilles tout ce dont je vais te parler, concernant toutes les ordonnances et toutes les lois qui concernent la maison de Yahweh. Applique ton cœur en ce qui concerne l'entrée de la maison et toutes les sorties du lieu saint.
\VS{6}Tu diras aux rebelles, à la maison d'Israël : Ainsi parle le Seigneur Yahweh : Maison d'Israël ! Assez de toutes vos abominations !
\VS{7}Vous avez fait entrer les fils de l'étranger, incirconcis de cœur et incirconcis de chair, pour être dans mon lieu saint, pour profaner ma maison. Vous avez offert mon pain, la graisse et le sang, à toutes vos abominations, vous avez enfreint mon alliance\FTNT{Lé. 3:11-16 ; Lé. 22:25 ; No. 28:2.}.
\VS{8}Vous n'avez pas observé l'office de mon lieu saint, mais vous les avez mis à votre place pour faire l'office dans mon lieu saint.
\TextTitle{Recommandations aux sacrificateurs du futur temple}
\VS{9}Ainsi parle le Seigneur Yahweh : Pas un de tous ceux qui seront fils d'étranger, incirconcis de cœur et incirconcis de chair, n'entrera dans mon lieu saint, pas même un d'entre tous les fils d'étrangers qui seront parmi les fils d'Israël.
\VS{10}Mais les Lévites qui se sont éloignés de moi, lorsque Israël s'est égaré, et qui se sont égarés de moi pour suivre leurs idoles, porteront la peine de leur iniquité.
\VS{11}Toutefois, ils seront employés dans mon lieu saint aux charges qui sont vers les portes de la maison, et ils feront le service de la maison ; ils égorgeront pour le peuple les bêtes pour l'holocauste, et pour les autres sacrifices, et se tiendront prêts devant lui pour le servir.
\VS{12}Parce qu'ils l'ont servi se présentant devant leurs idoles, et qu'ils ont fait tomber dans l'iniquité la maison d'Israël, à cause de cela j'ai levé ma main en jurant contre eux, dit le Seigneur Yahweh, qu'ils porteront la peine de leur iniquité.
\VS{13}Ils n'approcheront plus de moi pour exercer la sacrificature, ni pour approcher mes sanctuaires, mes lieux très saints ; mais ils porteront leur confusion et leurs abominations qu'ils ont commises.
\VS{14}C'est pourquoi je les établirai pour avoir la garde de la maison pour tout son service, et pour tout ce qui s'y fait.
\VS{15}Mais quant aux sacrificateurs et aux Lévites, fils de Tsadok, qui ont soigneusement administré ce qu'il fallait faire dans mon lieu saint, lorsque les fils d'Israël se sont éloignés de moi, ceux-là s'approcheront de moi pour faire mon service, et se tiendront devant moi pour m'offrir la graisse et le sang, dit le Seigneur Yahweh.
\VS{16}Ceux-là entreront dans mon lieu saint, et s'approcheront de ma table, pour faire mon service, et ils administreront soigneusement ce que j'ai ordonné de faire.
\VS{17}Lorsqu'ils franchiront les portes des parvis intérieurs, ils se vêtiront de robes de lin ; et il n'y aura point de laine sur eux pendant qu'ils feront le service aux portes des parvis intérieurs et dans la maison.
\VS{18}Ils auront des ornements de lin sur leur tête, et des caleçons de lin sur leurs reins, et ne se ceindront point de manière à provoquer la sueur.
\VS{19}Quand ils sortiront pour aller dans le parvis extérieur, dans le parvis extérieur, vers le peuple, ils se dévêtiront de leurs habits, avec lesquels ils font le service, et les poseront dans les chambres saintes, et se revêtiront d'autres habits, afin qu'ils ne sanctifient point le peuple avec leurs habits.
\VS{20}Ils ne se raseront point la tête, ni ne laisseront point croître leurs cheveux, mais simplement ils tondront leur tête\FTNT{Lé. 19:27.}.
\VS{21}Pas un des sacrificateurs ne boira du vin quand ils entreront au parvis intérieur.
\VS{22}Ils ne prendront point pour femme une veuve, ni une répudiée ; mais ils prendront des vierges, de la race de la maison d'Israël, ou une veuve qui soit veuve d'un sacrificateur\FTNT{Lé. 21:13-14.}.
\VS{23}Ils enseigneront à mon peuple la différence qu'il y a entre le saint et le profane, et leur feront entendre la différence qu'il y a entre ce qui est souillé et ce qui est pur.
\VS{24}Quand il surviendra quelque procès, ils assisteront au jugement, et jugeront suivant les lois que j'ai données ; et ils garderont mes lois et mes statuts dans toutes mes fêtes, et ils sanctifieront mes sabbats.
\VS{25}Un sacrificateur n'ira pas vers un mort, de peur d'en être souillé, il pourra se rendre impur que pour un père, pour une mère, pour un fils, pour une fille, pour un frère, et pour une sœur qui n'aura point eu de mari\FTNT{Lé. 21:1-2.}.
\VS{26}Et après que chacun d'eux se sera purifié, on lui comptera sept jours.
\VS{27}Le jour où il entrera dans le lieu saint, dans le parvis intérieur pour faire le service dans le lieu saint, il offrira son sacrifice pour son péché, dit le Seigneur Yahweh.
\VS{28}Et cela leur sera pour héritage. Ce sera moi leur héritage, car vous ne leur donnerez aucune possession en Israël, ce sera moi leur possession\FTNT{No. 18:20 ; De. 18:1-2.}.
\VS{29}Ils mangeront donc les gâteaux et ce qui s'offrira pour l'expiation, et ce qui s'offrira pour la culpabilité ; et tout interdit en Israël leur appartiendra.
\VS{30}Les prémices de tous les fruits et toutes les offrandes que vous présenterez par élévation, appartiendront aux sacrificateurs ; vous donnerez aussi les prémices de votre pâte aux sacrificateurs, afin que la bénédiction repose sur votre maison.
\VS{31}Les sacrificateurs ne mangeront aucune créature volante, aucun animal mort ou déchiré\FTNT{Ex. 22:31 ; Lé. 22:8.}.
\Chap{45}
\TextTitle{Zone réservée à Yahweh et aux sacrificateurs}
\VerseOne{}Quand vous partagerez au sort le pays en héritage, vous prélèverez comme une offrande en élévation pour Yahweh, une portion du pays longue de vingt-cinq mille cannes, et large de dix mille ; ce sera une chose sainte dans tous ses territoires et aux environs.
\VS{2}De cette portion, vous prendrez pour le lieu saint cinq cents cannes sur cinq cents en carré, et cinquante coudées tout autour pour ses faubourgs.
\VS{3}Sur cette étendue de vingt-cinq mille en longueur, et de dix mille en largeur, tu mesureras un emplacement pour le lieu saint, pour le Saint des saints.
\VS{4}C'est la portion sainte du pays, elle appartiendra aux sacrificateurs qui font le service du lieu saint, qui s'approchent de Yahweh pour le servir ; c'est là que seront leur maison, et ce sera un lieu saint pour le lieu saint.
\VS{5}Vingt-cinq mille cannes en longueur, et dix mille en largeur, formeront la propriété des Lévites, serviteurs de la maison, avec vingt chambres.
\VS{6}Vous donnerez pour la possession de la ville la largeur de cinq mille et la longueur de vingt-cinq mille, suivant la proportion de la portion sanctifiée, qui aura été levée pour toute la maison d'Israël.
\TextTitle{Zone réservée au prince}
\VS{7}Pour le prince vous réserverez un espace aux deux côtés de la portion sainte et de la propriété de la ville, le long de la portion sainte et le long de la propriété de la ville, du côté de l'occident vers l'occident, et du côté de l'orient vers l'orient, sur une longueur parallèle à l'une des parts, depuis la limite de l'occident jusqu'à la limite de l'orient.
\VS{8}Ce sera sa terre, sa propriété en Israël ; et mes princes que j'établirai ne fouleront plus mon peuple, mais ils distribueront le pays à la maison d'Israël, selon leurs tribus.
\TextTitle{Le prince, exemple au milieu du peuple ; prescriptions sur les offrandes}
\VS{9}Ainsi parle le Seigneur Yahweh : Assez, princes d'Israël ! Otez la violence et le pillage, et jugez avec justice ; ôtez vos extorsions de dessus mon peuple ! dit le Seigneur Yahweh.
\VS{10}Ayez la balance juste, l'épha juste, et le bath juste\FTNT{Lé. 19:35-36.}.
\VS{11}L'épha et le bath seront de même mesure ; on prendra un bath pour la dixième partie d'un homer, et l'épha sera la dixième partie d'un homer, la mesure de l'un et de l'autre se rapportera à l'homer.
\VS{12}Le sicle sera de vingt guéras ; vingt sicles, vingt-cinq sicles et quinze sicles feront la mine\FTNT{Ex. 30:13 ; Lé. 27:25.}.
\VS{13}Voici l'offrande que vous élèverez en offrande : La sixième partie d'un épha d'un homer de blé ; et vous donnerez la sixième partie d'un épha d'un homer d'orge.
\VS{14}Le bath est la mesure pour l'huile, l'offrande ordonnée pour l'huile sera la dixième partie d'un bath sur un cor, qui est égal à un homer de dix baths ; car dix baths feront un homer.
\VS{15}Pareillement l'offrande ordonnée des bêtes du menu bétail sera de deux cents l'une, même des meilleurs pâturages d'Israël ; toute cette offrande sera employée en gâteaux et en holocaustes, et en offrandes de paix, afin de faire propitiation pour vous, dit le Seigneur Yahweh.
\VS{16}Tout le peuple du pays sera tenu à cette offrande élevée, pour celui qui sera prince en Israël.
\VS{17}Mais le prince sera tenu de fournir les holocaustes, les offrandes et les libations qu'il faudra offrir aux fêtes solennelles, aux nouvelles lunes et aux sabbats, et dans toutes les solennités de la maison d'Israël. Il tiendra prêtes les bêtes qu'on sacrifiera pour l'expiation, et les gâteaux, et les bêtes qu'on sacrifiera pour l'holocauste, et les bêtes qu'on sacrifiera pour les offrandes de paix, afin de faire propitiation pour la maison d'Israël.
\VS{18}Ainsi parle le Seigneur Yahweh : Au premier mois, au premier jour du mois, tu prendras un jeune taureau sans défaut, et tu feras l'expiation du lieu saint.
\VS{19}Le sacrificateur prendra du sang de ce sacrifice offert pour le péché, et en mettra sur les poteaux de la maison, et sur les quatre angles de l'encadrement de l'autel, et sur les poteaux de la porte du parvis intérieur.
\VS{20}Tu en feras ainsi au septième jour du même mois, à cause des hommes qui pèchent involontairement et à cause des hommes simples ; et vous ferez ainsi propitiation pour la maison.
\VS{21}Au premier mois, au quatorzième jour du mois, vous aurez la Pâque, fête solennelle qui durera sept jours, pendant lesquels on mangera des pains sans levain\FTNT{Lé. 25:5 ; No. 9:3 ; Ex. 12.}.
\VS{22}En ce jour-là, le prince offrira un taureau pour le sacrifice d'expiation, tant pour lui que pour tout le peuple du pays.
\VS{23}Pendant les sept jours de cette fête solennelle, il offrira chaque jour sept taureaux et sept béliers sans défaut, pour l'holocauste qu'on offrira à Yahweh, et un bouc en sacrifice d'expiation, chaque jour.
\VS{24}Il offrira un épha pour chaque taureau, et un épha pour chaque bélier, avec un hin d'huile par épha.
\VS{25}Au septième mois, le quinzième jour du mois, à la fête solennelle, il offrira durant sept jours les mêmes choses, le même sacrifice expiatoire, le même holocauste, et la même offrande avec l'huile.
\Chap{46}
\TextTitle{Le service le jour du sabbat et les jours de fêtes}
\VerseOne{}Ainsi parle le Seigneur Yahweh : La porte du parvis intérieur, du côté oriental, sera fermée les six jours ouvrables, mais elle sera ouverte le jour du sabbat, elle sera aussi ouverte le jour de la nouvelle lune.
\VS{2}Et le prince y entrera par le chemin du vestibule de la porte du parvis extérieure, et se tiendra près de l'un des poteaux de l'autre porte, les sacrificateurs prépareront son holocauste et ses sacrifices d'offrande de paix ; il se prosternera sur le seuil de cette porte, et ensuite il sortira ; et la porte ne sera point fermée jusqu'au soir.
\VS{3}Tellement que le peuple du pays se prosternera devant Yahweh à l'entrée de cette porte, les jours de sabbat et des nouvelles lunes.
\VS{4}L'holocauste que le prince offrira à Yahweh le jour du sabbat sera de six agneaux sans défaut, et d'un bélier sans défaut.
\VS{5}L'offrande pour le bélier sera d'un épha, et l'offrande pour chacun des agneaux sera selon ce qu'il pourra donner ; mais il y aura un hin d'huile pour chaque épha.
\VS{6}Au jour de la nouvelle lune, son holocauste sera d'un jeune taureau, sans défaut, de six agneaux et d'un bélier, aussi sans défaut.
\VS{7}Son offrande pour le taureau sera d'un épha, pour l'offrande du bélier, un autre épha, et pour chacun des agneaux selon ce qu'il pourra donner ; mais il y aura un hin d'huile pour chaque épha.
\VS{8}Lorsque le prince entrera, il entrera par le chemin du vestibule de la porte, et il sortira par le même chemin.
\VS{9}Quand le peuple du pays entrera pour se présenter devant Yahweh, aux fêtes solennelles, celui qui y entrera par le chemin de la porte nord pour y adorer Yahweh, sortira par le chemin de la porte sud ; et celui qui y entrera par le chemin de la porte sud, sortira par le chemin de la porte nord ; personne ne retournera par le chemin de la porte par laquelle il sera entré, mais il sortira par celle qui lui est opposée.
\VS{10}Alors le prince entrera parmi eux, quand ils entreront ; et quand ils sortiront, ils sortiront ensemble.
\VS{11}Or,dans ces fêtes solennelles et dans ces solennités, l'offrande d'un taureau sera d'un épha, et l'offrande d'un bélier d'un autre épha, l'offrande de chacun des agneaux sera selon ce que le prince pourra donner, et il y aura un hin d'huile pour chaque épha.
\VS{12}Et si le prince offre un sacrifice volontaire, quelque un holocauste, soit quelques un sacrifices d'offrande de paix en offrande à Yahweh, on lui ouvrira la porte qui est du côté oriental, et il offrira son holocauste et ses sacrifices d'offrande de paix comme il les offre le jour du sabbat, puis il sortira, et après qu'il sera sorti, on fermera cette porte.
\VS{13}Tu sacrifieras chaque jour en holocauste à Yahweh un agneau d'un an sans défaut, tu le sacrifieras tous les matins.
\VS{14}Tu lui offriras tous les matins l'offrande, faite de la sixième partie d'un épha, et de la troisième d'un hin d'huile pour pétrir la farine ; c'est l'offrande à Yahweh qu'il faut offrir par ordonnances perpétuelles.
\VS{15}Ainsi on offrira tous les matins en holocauste perpétuel cet agneau et l'offrande avec cette huile.
\VS{16}Ainsi a dit le Seigneur Yahweh : quand le Prince aura fait un don de quelque pièce de son héritage à quelqu'un de ses fils, ce don appartiendra à ses fils ; parce qu'ils ont droit de possession en l'héritage.
\VS{17}Mais s'il fait un don pris de son héritage à l'un de ses serviteurs, le don lui appartiendra bien, mais seulement jusqu'à l'année de la liberté, puis il retournera au prince ; ses fils seuls posséderont ce qu'il leur donnera de son héritage\FTNT{Lé. 25:10.}.
\VS{18}Et le prince ne prendra pas de l'héritage du peuple en les opprimant, les chassant de leur possession : c'est de sa propre possession qu'il fera hériter ses fils, afin qu'aucun de mon peuple ne soit pas dispersé loin de sa possession.
\VS{19}Puis il me mena par l'entrée qui était vers le côté de la porte, aux chambres saintes qui appartenaient aux sacrificateurs, vers le nord. Et voici, il y avait un certain lieu dans le fond du côté occidental.
\VS{20}Il me dit : C'est là le lieu où les sacrificateurs feront bouillir le reste de la bête qu'on aura sacrifiée pour la culpabilité, et le reste de la bête qu'on aura sacrifiée pour l'expiation, et où ils feront cuire les offrandes, afin qu'ils ne les emportent point au parvis extérieur de manière à sanctifier le peuple\FTNT{No. 18:9.}.
\VS{21}Puis il me fit sortir vers le parvis extérieur, et me fit traverser vers les quatre angles du parvis, et voici, il y avait une cour à chacun des angles du parvis.
\VS{22}Aux quatre angles de ce parvis, il y avait des cours voûtées, longues de quarante coudées, et larges de trente ; et toutes les quatre avaient la même mesure dans les angles.
\VS{23}Un mur les entourait toutes les quatre, et des foyers étaient faits au bas du campement tout autour.
\VS{24}Et il me dit : Ce sont ici les cuisines, où ceux qui font le service de la maison cuiront les sacrifices du peuple.
\Chap{47}
\TextTitle{Les eaux pures du sanctuaire\FTNTT{Cp. Za. 14:8-9 ; Ap. 22:1-2.}}
\VerseOne{}Puis il me ramena vers l'entrée de la maison, et voici, des eaux sortaient sous le seuil de la maison, vers l'orient, car la face de la maison était vers l'orient ; et ces eaux-là descendaient du côté droit de la maison, du côté sud de l'autel\FTNT{Ps. 46:5 ; Joë. 3:18 ; Za. 13:1 ; Za. 14:8 ; Ap. 22:1.}.
\VS{2}Puis il me fit sortir par le chemin de la porte nord, et me fit faire le tour par dehors, jusqu'à la porte extérieure, du côté de l'orient, et voici, les eaux coulaient du côté droit.
\VS{3}Quand cet homme s'avança vers l'orient, il avait dans sa main un cordeau ; et il mesura mille coudées, puis il me fit traverser ces eaux-là, et j'avais de l'eau jusqu'aux chevilles.
\VS{4}Puis il mesura mille autres coudées, et me fit traverser les eaux, j'avais de l'eau jusqu'aux genoux ; puis il mesura mille autres coudées, et me fit traverser, et j'avais de l'eau jusqu' aux reins.
\VS{5}Il mesura mille autres coudées ; mais ces eaux-là étaient déjà un torrent que je ne pouvais traverser ; car ces eaux-là étaient si profondes qu'il fallait y traverser à la nage, c'était un torrent que l'on ne pouvait traverser.
\VS{6}Alors il me dit : Fils de l'homme, as-tu vu ? Puis il me fit aller et revenir vers le bord du torrent.
\VS{7}Quand je revins, il y avait un grand nombre d'arbres sur les deux bords du torrent.
\VS{8}Il me dit : Ces eaux couleront vers la Galilée orientale, et elles descendront à la campagne, puis elles entreront dans la mer, et quand elles se seront jetées dans la mer, les eaux deviendront saines.
\VS{9}Il arrivera que tout être vivant qui se meut vivra partout où les deux torrents couleront, et il y aura une grande quantité de poissons ; car là où ces eaux entreront, les eaux deviendront saines, et tout vivra là où ce torrent parviendra.
\VS{10}Il arrivera que des pêcheurs se tiendront le long de cette mer, depuis En-Guédi jusqu'à En-Eglaïm ; on étendra les filets, il y aura des poissons de diverses espèces, comme les poissons de la grande mer, et ils seront très nombreux.
\VS{11}Ses marais et ses fosses ne seront pas assainis, ils seront abandonnés au sel.
\VS{12}Auprès de ce torrent et sur ses deux bords, il croîtra des arbres fruitiers de toutes sortes. Leur feuillage ne se flétrira point, et l'on trouvera toujours du fruit. Tous les mois, ils produiront des fruits mûrs, parce que les eaux de ce torrent sortent du lieu saint, et à cause de cela leur fruit sera bon à manger, et leur feuillage servira de remède\FTNT{Ap. 22:2.}.
\TextTitle{Délimitations du pays\FTNTT{Cp. Ge. 15:18-21.}}
\VS{13}Ainsi parle le Seigneur Yahweh : Voici les frontières du pays que vous aurez en héritage, selon les douze tribus d'Israël. Joseph aura deux portions.
\VS{14}Vous en aurez la possession l'un comme l'autre de ce pays. J'ai levé ma main de le donner à vos pères ; et ce pays-là vous sera donc échu en héritage\FTNT{Ge. 12:7 ; Ge. 17:8.}.
\VS{15}Voici la frontière du pays, du côté nord, depuis la grande mer, le chemin de Hethlon jusqu'à Tsedad,
\VS{16}Hamath, Bérotha, et Sibraïm, entre la frontière de Damas et la frontière de Hamath, Hatzer-Hatthicon, vers la frontière de Havran.
\VS{17}La frontière sera depuis la mer, Hatsar-Enon ; la frontière de Damas, Tsaphon au nord et la frontière de Hamath : Ce sera le côté nord.
\VS{18}Le côté oriental sera le Jourdain entre Havran, Damas et Galaad, et le pays d'Israël ; vous mesurerez depuis la frontière nord jusqu'à la mer orientale : Ce sera le côté oriental.
\VS{19}Le côté méridional, au midi, ira depuis Thamar jusqu'aux eaux de Meriba à Kadès, jusqu'au torrent vers la grande mer : Ce sera le côté méridional.
\VS{20}Le côté occidental sera la grande mer, depuis la frontière jusque vis-à-vis de Hamath : Ce sera le côté occidental.
\VS{21}Vous partagerez ce pays entre vous, selon les tribus d'Israël.
\VS{22}Vous le diviserez en héritage par le sort pour vous et pour les étrangers qui séjourneront au milieu de vous, qui engendreront des fils au milieu de vous ; vous les regarderez comme natifs des fils d'Israël ; ils partageront au sort l'héritage avec vous parmi les tribus d'Israël.
\VS{23}Vous donnerez à l'étranger son héritage dans la tribu où il séjournera, dit le Seigneur Yahweh.
\Chap{48}
\TextTitle{Héritage de sept tribus\FTNTT{Cp. Jos. 13:1-19:51.}}
\VerseOne{}Voici les noms des tribus. Depuis l'extrémité qui regarde vers le nord, le long de la contrée du chemin de Hethlon du quartier par lequel on entre à Hamath, jusqu'à Hatsar-Enon, qui est la frontière de Damas, du côté qui regarde vers le nord, le long de la contrée de Hamath, tellement que cette extrémité est le canton de l'orient et celui de l'occident: il y aura une portion pour Dan.
\VS{2}Et tout joignant les frontières de Dan, depuis le canton de l'orient jusqu'au canton qui regarde vers l'occident : il y aura une partion pour Aser.
\VS{3}Et tout joignant les frontières d'Aser, depuis le canton qui regarde vers l'orient jusqu'au canton qui regarde vers l'occident : il y aura une partion pour Nephthali.
\VS{4}Et tout les frontières de Nephthali, depuis le canton qui regarde vers l'orient jusqu'au canton qui regarde vers  l'occident : il y aura une partion pour Manassé. 
\VS{5}Et tout joignant les frontières de Manassé, depuis le canton qui regarde vers de l'occident jusqu'au canton qui regarde vers  l'orient : il y aura une part pour Ephraïm. 
\VS{6}Et tout joignant les frontières d'Ephraïm, encore depuis le canton de l'orient,jusqu'au canton qui regarde vers l'occident: il y aura une partion pour Ruben. 
\VS{7}Et joignant les frontièrs de Ruben, depuis le canton de l'orient jusqu'au canton qui regarde vers l'occident : il y aura une partion pour Juda.
\VS{8}Et tout le long des frontières de Juda, depuis le canton de l'orient, jusqu'au canton qui regarde vers l'occident; il y aura une portion que vous prélèverez sur toute la masse du pays, comme une offrande élevée et elle aura vingt-cinq mille cannes de largueur et de longueur, autant que l'une des autres partions depuis le canton qui regarde vers l'orient jusqu'au canton qui regarde vers l'occident; de sorte que le sanctuaire sera au milieu.
\VS{9}La portion que vous lèverez pour Yahweh et offerte en offrande élevée, aura vingt-cinq mille cannes de longueur, et dix mille de largeur.
\TextTitle{Territoire réservé aux sacrificateurs et aux Lévites}
\VS{10}Et cette portion sainte sera pour les sacrificateurs, vingt-cinq mille cannes de longueur au nord, et dix mille de largeur, à l'occident, dix mille en largeur à l'orient, et vingt-cinq mille en longueur au sud, et le sanctuaire de Yahweh sera au milieu.
\VS{11}Elle sera pour les sacrificateurs, et quiconque aura été sanctifié d'entre les fils de Tsadok, qui ont fait ce que j'ai ordonné, et qui ne se sont point égarés quand les fils d'Israël se sont égarés, comme se sont égarés les autres Lévites,
\VS{12}Ceux-là auront une portion ainsi levée sur l'autre très sainte, prélevée sur la portion du pays qui aura été prélevée, à côté de la frontière des Lévites.
\VS{13}Les Lévites auront, parallèlement à la frontière des sacrificateurs, vingt-cinq mille cannes en longueur et dix mille de largeur, vingt-cinq mille pour toute la longueur et dix mille pour toute la largeur.
\VS{14}Ils n'en pourront ni vendre, ni échanger, et les prémices du pays ne seront point transgressées car elles sont mises à part pour Yahweh.
\VS{15}Les cinq mille cannes qui resteront en largeur sur les vingt-cinq mille cannes seront destinées à la ville, pour les habitations et le faubourg, et la ville sera au milieu.
\VS{16}En voici les mesures : Du côté nord, quatre mille cinq cents cannes, du côté sud, quatre mille cinq cents, du côté oriental, quatre mille cinq cents, et du côté occidental, quatre mille cinq cents.
\VS{17}Puis il y aura des faubourgs pour la ville, vers le nord. La ville aura un faubourg au nord de deux cent cinquante cannes, de deux cent cinquante au sud, de deux cent cinquante à l'orient, et de deux cent cinquante à l'occident.
\VS{18}Quant à ce qui sera de reste sur la longueur et qui sera tout joignant à la portion sanctifiée, et qui aura dix mille cannes à l'orient, et dix mille autres cannes à l'occident, parallèlement à la portion sanctifiée, le revenu qu'on en tirera sera pour nourriture de ceux qui feront le service qu'il faut dans la ville.
\VS{19}Le sol sera travaillé par ceux de toutes les tribus d'Israël qui travailleront pour la ville.
\VS{20}Toute la portion prélevée sera de vingt-cinq mille cannes en longueur sur vingt-cinq mille en largeur ; vous en séparerez un carré pour la propriété de la ville.
\VS{21}PUis le reste sera pour le prince aux deux côtés de la portion sainte et de la possession de la ville, des vingt-cinq mille coudées de la portion prélevée jusqu'à la frontière de l'orient, et à l'occident, des vingt-cinq mille coudées jusqu'à la frontière de l'occident, le long des parts, pour le prince, et la portion sainte et le sanctuaire de la Maison seront au milieu de tout le pays. 
\VS{22}Ce qui sera donc pour le prince sera l'espace compris depuis la possession des Lévites, et depuis la possession de la ville ; ce qui sera entre ces possessions là et la frontière de Juda, et la frontière de Benjamin, sera pour le prince.
\TextTitle{Héritage de cinq tribus}
\VS{23}Or ce qui sera de reste sera pour les autres tribus; depuis le canton de ce qui regarde vers l'orient, jusqu'au canton de ce qui regarde vers l'occident, il y aura une portion pour Benjamin.
\VS{24}Puis tout joignant les frontières de Benjamin, depuis le canton de ce qui regarde vers l'orient, jusqu'au canton de ce qui regarde vers l'occident, il y aura une autre portion pour Siméon.
\VS{25}Puis tout joignant les frontières de Siméon, depuis le canton de ce qui regarde vers l'orient, jusqu'au canton de ce qui regarde vers l'occident, il y aura une autre portion pour Issacar.
\VS{26}Puis tout joignant sur les frontières d'Issacar, depuis le canton de ce qui regarde vers l'orient, jusqu'au canton de ce qui regarde vers l'occident, il y aura une autre portion pour Zabulon.
\VS{27}Puis joignant les frontières de Zabulon, depuis le canton de ce qui regarde vers l'orient, jusqu'au canton de ce qui regarde vers l'occident, il y aura une autre portion pour Gad.
\VS{28}Or ce qui appartient au côté du midi qui regarde proprement le vent d'autant, et sur la frontière de Gad, et cette frontière sera depuis Thamar jusqu'aux eaux de contestation, à Kadès, le long du torrent jusqu'à la grande mer.
\VS{29}C'est là le pays que vous partagerez par le sort en héritage aux tribus d'Israël, et ce sont là leurs portions, dit le Seigneur Yahweh.
\VS{30}Et ce sont ici les sorties de la ville : Du côté du nord, il y aura quatre mille cinq cents mesures.
\VS{31}Et les portes de la ville seront selon les noms des tribus d'Israël : Trois portes vers le nord, une porte de Ruben, une porte de Juda, une porte de Lévi.
\VS{32}Et du côté de l'orient, quatre mille cinq cents mesures, et trois portes : Une porte de Joseph, une porte de Benjamin, une porte de Dan.
\VS{33}Et du côté sud, quatre mille cinq cents mesures, et trois portes : Une porte de Simeon, une porte d'Issacar, une porte de Zabulon.
\VS{34}Et du côté ouest, quatre mille cinq cents mesures, avec leurs trois portes : Une porte de Gad, une porte d'Aser, une porte de Nephthali.
\VS{35}Ainsi le circuit de la ville sera de dix-huit mille mesures ; et le nom de la ville depuis ce jour-là sera : Yahweh est ici.
\PPE{}
\end{multicols}

%\clearpage\ShortTitle{Osée}\BookTitle{Osée}\BFont
\noindent\hrulefill
{\footnotesize
\textit{
\bigskip
{\centering{}
\\Auteur : Osée
\\(Heb. : Hoswhéa')
\\Signification : Salut, Sauve
\\Thème : Israël sera rejeté à cause de son apostasie. D'autres nations seront appelées à sa place.
\\Date de rédaction : 8\up{ème} siècle av. J.-C.\\}
}
%\bigskip
\textit{
\\Osée, fils de Béeri, exerça son service dans le royaume du nord au temps de Joas, roi d'Israël. Il était contemporain des prophètes Amos, Michée et Esaïe.
%\bigskip
\\Yahweh demanda à Osée d'épouser une prostituée pour que le prophète puisse partager plus profondément son fardeau
et la tristesse qu'il subissait en raison de l'infidélité du peuple qu'il aimait tant. Malgré la rébellion et les mauvais
agissements d'Israël, Yahweh manifesta une fois de plus sa patience et utilisa Osée pour avertir et inviter les fils de Jacob à la repentance.\bigskip
}
}
\par\nobreak\noindent\hrulefill
\begin{multicols}{2}
\Chap{1}
\VerseOne{}La parole de Yahweh qui fut adressée à Osée fils de Béeri, au temps d'Ozias, de Jotham, d'Achaz et d'Ezéchias, rois de Juda, et au temps de Jéroboam, fils de Joas, roi d'Israël.
\TextTitle{Mariage d'Osée et naissance de Jizreel}
\VS{2}La première fois que Yahweh parla à Osée, Yahweh dit à Osée : Va, prends une femme prostituée, et aie d'elle des enfants de prostitution ; car le pays s'est entièrement prostitué en abandonnant Yahweh !
\VS{3}Il alla, et il prit Gomer, fille de Diblaïm. Elle conçut, et lui enfanta un fils.
\VS{4}Et Yahweh lui dit : Donne-lui le nom de Jizreel ; car encore un peu de temps, et je punirai la maison de Jéhu pour le sang versé à Jizreel, et je ferai cesser le royaume de la maison d'Israël\FTNT{Cette prophétie s'est accomplie en 722 av. J-C. Voir 2 R.17.}.
\VS{5}Et il arrivera qu'en ce jour-là, je briserai l'arc d'Israël dans la vallée de Jizreel.
\TextTitle{Naissance de Lo-Ruchama}
\VS{6}Elle conçut encore, et enfanta une fille. Et Yahweh lui dit : Donne-lui le nom de Lo-Ruchama\FTNT{Tout au long de ce livre, certains mots sont symboliquement utilisées pour nommer Israël. A savoir : 
\\- « ruchama » : miséricorde ;
\\- « Lo-Ruchama » : celle dont on ne fait pas miséricorde ; 
\\- « ammi » : mon peuple ;
\\- « Lo-Ammi » : pas mon peuple.
\\Ces noms illustrent ainsi l'infidélité d'Israël envers Yahweh.} ; car je ne continuerai plus à faire miséricorde à la maison d'Israël, mais je les enlèverai entièrement.
\VS{7}Mais je ferai miséricorde à la maison de Juda; et je les délivrerai par Yahweh, leur Dieu, et je ne les délivrerai ni par l'arc, ni par l'épée, ni par les combats, ni par les chevaux, ni par les cavaliers.
\TextTitle{Naissance de Lo-Ammi}
\VS{8}Puis quand elle sevra Lo-Ruchama ; elle conçut, et enfanta un fils.
\VS{9}Et Yahweh dit : Appelle-le du nom de Lo-Ammi ; car vous n'êtes point mon peuple, et je ne suis pas votre Dieu.
\Chap{2}
\TextTitle{Futur rétablissement d'Israël}
\VerseOne{}Cependant le nombre des fils d'Israël sera comme le sable de la mer, qui ne peut ni se mesurer ni se compter ; et dans la ville où il leur est dit : Vous n'êtes pas mon peuple ! On leur dira : Vous êtes les fils du Dieu vivant !
\VS{2}Aussi les fils de Juda et les fils d'Israël se rassembleront, et ils s'établiront un chef, et monteront hors du pays ; car la journée de Jizreel sera grande.
\VS{3}Dites à vos frères : Ammi ! Et à vos sœurs : Ruchama !
\TextTitle{Châtiment d'Israël, la prostituée\FTNTT{2 R. 17:1-18.}}
\VS{4}Plaidez, plaidez contre votre mère, car elle n'est point ma femme, et je ne suis point son mari ! Qu'elle ôte ses prostitutions de son visage, et ses adultères de son sein !
\VS{5}De peur que je ne la dépouille à nu, et que je ne l'expose comme au jour de sa naissance, et que je ne la rende semblable à un désert, à une terre aride, et ne la fasse mourir de soif ;
\VS{6}et je n'aurai point de miséricorde pour ses enfants, car ce sont des enfants de prostitution.
\VS{7}Car leur mère s'est prostituée, celle qui les a conçus s'est déshonorée, car elle a dit : Je m'en irai après mes amants, qui me donnent mon pain et mes eaux, ma laine et mon lin, mon huile, et mes boissons.
\VS{8}C'est pourquoi voici, je vais fermer ton chemin avec des épines, j'y élèverai un mur, afin qu'elle ne trouve plus ses sentiers.
\VS{9}Elle poursuivra ses amants, mais ne les atteindra pas ; elle les cherchera, mais elle ne les trouvera point. Puis elle dira : Je m'en irai, et je retournerai vers mon premier mari, car alors j'étais plus heureuse que maintenant.
\VS{10}Mais elle n'a pas reconnu que c'était moi qui lui donnais le blé, le vin et l'huile ; et l'on a fait des offrandes à Baal\FTNT{Baal : Voir commentaire en Jg. 2:13.} avec l'argent et l'or que je lui prodiguais.
\VS{11}C'est pourquoi je reprendrai mon froment en son temps, mon vin en sa saison, et je retirerai ma laine et mon lin qui couvraient sa nudité.
\VS{12}Et maintenant je découvrirai sa honte aux yeux de ses amants, et personne ne la délivrera de ma main.
\VS{13}Je ferai cesser toute sa joie, ses fêtes, ses nouvelles lunes, ses sabbats, et toutes ses solennités.
\VS{14}Je ravagerai ses vignes et ses figuiers, dont elle disait : Voici le salaire que mes amants m'ont donné ! Je les réduirai en une forêt, et les bêtes des champs les dévoreront.
\VS{15}Je la punirai pour les jours où elle encensait les Baals, où elle se parait de ses anneaux et de ses colliers, et s'en allait après ses amants, et m'oubliait, dit Yahweh.
\TextTitle{La femme adultère revient dans son foyer : Israël revient à Yahweh}
\VS{16}Néanmoins, voici, je veux l'attirer et la mener au désert, là je parlerai à son cœur.
\VS{17}Là, je lui accorderai ses vignes et la vallée d'Acor, telle une porte d'espérance, et là, elle chantera comme au temps de sa jeunesse, et comme au jour où elle remonta du pays d'Egypte.
\VS{18} Et il arrivera en ce jour-là, dit Yahweh, tu m'appelleras: Mon Mari ! Et tu ne m'appelleras plus: Mon Maître\FTNT{Littéralement : Baal.} !
\VS{19}Car j'ôterai de sa bouche les noms des Baals, et on ne fera plus mention de leurs noms.
\VS{20}Aussi en ce temps-là, je traiterai pour eux une alliance avec les bêtes des champs, avec les oiseaux du ciel, et avec les reptiles de la terre ; je briserai et j'ôterai du pays l'arc, l'épée et la guerre, et je les ferai se reposer en sécurité.
\VS{21}Et je te fiancerai pour moi à toujours ; je te fiancerai, dis-je, pour moi, par la justice, la droiture, la grâce et la miséricorde;
\VS{22}je serai ton fiancé par la fidélité, et tu reconnaîtras Yahweh.
\VS{23}Et il arrivera en ce jour-là, j'exaucerai, dit Yahweh, je témoignerai aux cieux, et les cieux exauceront la terre;
\VS{24}la terre exaucera le blé, le bon vin et l'huile, et ils exauceront Jizreel. 
\VS{25}Puis, je la sèmerai pour moi dans ce pays, et je ferai miséricorde à Lo-Ruchama ; je dirai à Lo-Ammi : Tu es mon peuple ! Et il me répondra : Mon Dieu !
\Chap{3}
\TextTitle{Soumission d'Israël à Yahweh}
\VerseOne{}Après cela Yahweh me dit : Va encore, et aime une femme aimée d'un ami, et adultère; aime-la comme Yahweh aime les enfants d'Israël, qui se tournent toutefois vers d'autres dieux et aiment les gâteaux de raisins.
\VS{2}J'achetai donc cette femme pour quinze pièces d'argent, un homer et demi d'orge\FTNT{Voir dans les annexes les tableaux des poids et mesures.}.
\VS{3}Et je lui dis : Assieds-toi avec moi pendant plusieurs jours, ne t'abandonne plus à la prostitution, ne sois à aucun homme, et je serai fidèle envers toi.
\VS{4}Car les enfants d'Israël demeureront plusieurs jours sans roi, sans chef, sans sacrifice, sans statue, sans éphod, et sans théraphim\FTNT{Cette prophétie s'est accomplie d'une manière extraordinaire à travers l'histoire du peuple d'Israël depuis la première venue de Jésus-Christ. Les Israélites étaient dispersés, sans unité politique faute de roi, empêchés d'offrir des sacrifices depuis la destruction du temple par Titus (39-81), fils de l'empereur romain Vespasien (9-79), en l'an 70.}.
\VS{5}Mais après cela, les enfants d'Israël se repentiront\FTNT{La repentance et la conversion nationale d'Israël auront lieu lors du retour du Messie (Es. 59:20-21 ; Ro. 11:26-27). Dieu n'a pas abandonné son peuple, il viendra lui-même le délivrer.}; et rechercheront Yahweh, leur Dieu, et David, leur roi; ils seront dans la crainte à la vue de Yahweh et de sa bonté, dans les derniers jours\FTNT{Les derniers jours : Voir commentaire en Ge. 49:1.}.
\Chap{4}
\TextTitle{Israël, la nation pécheresse}
\VerseOne{}Ecoutez la parole de Yahweh, fils d'Israël ! Car Yahweh a un procès avec les habitants du pays, parce qu'il n'y a ni de vérité, ni de miséricorde, ni de connaissance de Dieu dans le pays.
\VS{2}Il n'y a que parjures et mensonges, meurtres, vols et adultères ; on use de violence, et un meurtre touche l'autre.
\VS{3}C'est pourquoi le pays sera dans le deuil, et tous ceux qui l'habitent seront languissants, et avec eux toutes les bêtes des champs et tous les oiseaux du ciel ; même les poissons de la mer périront.
\VS{4}Mais que nul ne conteste, et que nul ne reprenne ; car ton peuple est comme ceux qui disputent avec le sacrificateur.
\VS{5}Tu tomberas donc en plein jour, et le prophète aussi tombera avec toi de nuit, et j'exterminerai ta mère.
\TextTitle{Israël dans l'ignorance}
\VS{6}Mon peuple est détruit, parce qu'il lui manque la connaissance\FTNT{Le verbe « détruire » vient de l'hébreu « damah » qui signifie aussi « égorger ». Satan est celui qui vient égorger, dérober et détruire, notamment avec ses faux prophètes (Jn. 10:10). Chaque disciple de Jésus-Christ doit avoir une vie de prière et de méditation quotidienne afin de résister aux attaques de l'ennemi.}. Parce que tu as rejeté la connaissance, je te rejetterai, afin que tu n'exerces plus la sacrificature; puisque tu as oublié la loi de ton Dieu, moi aussi j'oublierai tes enfants.
\VS{7}Plus ils se sont multipliés, plus ils ont péché contre moi : Je changerai leur gloire en ignominie.
\VS{8}Ils se nourrissent des péchés de mon peuple, leur âme soutient leur iniquité.
\VS{9}C'est pourquoi le sacrificateur sera traité comme le peuple; je le châtierai selon ses voies, et je lui rendrai selon ses œuvres.
\VS{10}Et ils mangeront mais ils ne seront point rassasiés, ils se prostitueront mais ils ne multiplieront point, parce qu'ils ont cessé de prendre garde à Yahweh.
\VS{11}La prostitution, le vin et le moût, font perdre l'entendement.
\TextTitle{Israël dans l'idolâtrie}
\VS{12}Mon peuple consulte son bois, et c'est son bâton qui lui répond ; car l'esprit de prostitution égare, et ils se prostituent loin de leur Dieu.
\VS{13}Ils sacrifient sur le sommet des montagnes, ils brûlent de l'encens sur les collines, sous les chênes, sous les peupliers, et les térébinthes, parce que leur ombrage est agréable. C'est pourquoi vos filles se prostituent, et vos belles-filles commettent l'adultère.
\VS{14}Je ne punirai pas vos filles parce qu'elles se prostituent, ni vos belles-filles parce qu'elles commettent l'adultère, car eux-mêmes se retirent avec des prostituées, et sacrifient avec des femmes débauchées. Ainsi le peuple qui est sans intelligence sera ruiné.
\VS{15}Si tu te prostitues, ô Israël, au moins que Juda ne se rende point coupable ! N'entrez donc point dans Guilgal, et ne montez pas à Beth-Aven, et ne jurez point : Yahweh est vivant !
\VS{16}Parce qu'Israël se révolte comme une vache indomptable, maintenant Yahweh les fera paître comme des agneaux dans de vastes plaines.
\VS{17}Ephraïm s'est associé aux idoles ; abandonne-le !
\VS{18}Leur breuvage est devenu aigre ; ils n'ont fait que se prostituer ; ils n'aiment qu'à dire : apportez ; ce n'est qu'ignominie que ses protecteurs.
\VS{19}Le vent l'a enfermé dans ses ailes, et ils auront honte de leurs sacrifices.
\Chap{5}
\TextTitle{Yahweh abandonne son peuple}
\VerseOne{}Ecoutez ceci, sacrificateurs ! Maison d'Israël, sois attentive ! Maison du roi, tendez l'oreille ! Car c'est à vous que s'adresse le jugement, parce que vous avez été un piège à Mitspa, et un filet tendu sur le Thabor.
\VS{2}Les infidèles s'enfoncent dans le crime. Et moi, je les châtierai tous.
\VS{3}Je connais Ephraïm, et Israël ne m'est point caché; car maintenant, Ephraïm, tu t'es prostitué, et Israël est souillé.
\VS{4}Leurs œuvres ne leur permettent pas de revenir à leur Dieu, parce que l'esprit de prostitution est au milieu d'eux, et parce qu'ils ne connaissent point Yahweh.
\VS{5}L'orgueil d'Israël témoigne contre lui; Israël et Ephraïm tomberont par leur iniquité ; Juda aussi tombera avec eux.
\VS{6}Ils iront avec leurs brebis et leurs bœufs chercher Yahweh, mais ils ne le trouveront point, il s'est retiré du milieu d'eux.
\VS{7}Ils se sont montrés infidèles envers Yahweh, car ils ont engendré des fils étrangers ; maintenant un mois suffira pour les dévorer avec leurs biens.
\VS{8}Sonnez du shofar à Guibea. Sonnez de la trompette à Rama ! Poussez des cris de guerre à Beth-Aven ! Derrière toi, Benjamin !
\VS{9}Ephraïm sera un sujet d'épouvante au jour du châtiment ; je le fais savoir parmi les tribus d'Israël comme une chose certaine.
\VS{10}Les chefs de Juda sont comme ceux qui déplacent les bornes; je répandrai sur eux ma fureur comme un torrent.
\VS{11}Ephraïm est opprimé, brisé par le jugement, car il a vécu selon les préceptes qui lui plaisaient.
\VS{12}Je serai comme une teigne pour Ephraïm, comme de la pourriture pour la maison de Juda.
\VS{13}Ephraïm voit sa maladie, et Juda ses plaies ; Ephraïm s'en est allé vers le roi d'Assyrie, et s'est adressé au roi Jareb. Mais ce roi ne pourra ni vous guérir, ni panser vos plaies.
\VS{14}Je serai comme un lion pour Ephraïm, comme un lionceau pour la maison de Juda. Moi, moi je déchirerai, puis je m'en irai, j'emporterai la proie, et nul ne me l'enlèvera.
\TextTitle{Israël revient à Yahweh}
\VS{15}Je m'en irai, je reviendrai dans ma demeure, jusqu'à ce qu'ils se reconnaissent coupables, et qu'ils cherchent ma face. Quand ils seront dans la détresse, dans leur angoisse, ils me chercheront.
\Chap{6}
\VerseOne{}Venez, retournons à Yahweh ! Car il a déchiré, mais il nous guérira ; il a frappé, mais il bandera nos plaies.
\VS{2}Il nous rendra la vie dans deux jours; et le troisième jour il nous relèvera, et nous vivrons en sa présence.
\VS{3}Car nous connaîtrons Yahweh, et nous continuerons à le connaître ; sa venue\FTNT{Il est question ici du retour du Seigneur Jésus-Christ : Voir  commentaire en Za. 14:1.} est aussi certaine que celle de l'aurore. Il viendra pour nous comme la pluie, comme la pluie de l'arrière-saison\FTNT{Voir commentaire en Joë. 2:23.} qui arrose la terre.
\TextTitle{Yahweh dénonce le péché d'Ephraïm}
\VS{4}Que te ferai-je, Ephraïm ? Que te ferai-je, Juda ? Votre piété est comme la nuée du matin, comme la rosée qui se dissipe dès le matin.
\VS{5}C'est pourquoi je les taillerai en pièces par mes prophètes, je les tuerai par les paroles de ma bouche\FTNT{Hé. 4:12 ; Ap. 1:16 ; Ap. 19:15.}, et mes jugements apporteront la lumière.
\VS{6}Car je prends plaisir à la miséricorde et non aux sacrifices, et à la connaissance de Dieu plus qu'aux holocaustes.
\VS{7}Mais ils ont transgressé l'alliance, comme si elle avait été d'un homme, en quoi ils se sont portés perfidement contre moi.
\VS{8}Galaad est une ville d'ouvriers d'iniquité, couverte de traces de sang.
\VS{9}Et comme les bandes des voleurs attendent quelqu'un, ainsi les sacrificateurs, après avoir comploté, tuent les gens sur le chemin du côté de Sichem ; car ils exécutent leurs méchants desseins.
\VS{10}J'ai vu des choses infâmes dans la maison d'Israël: Là Ephraïm se prostitue, Israël en est souillé.
\VS{11}A toi aussi Juda, une moisson est préparée, quand je ramènerai les captifs de mon peuple.
\Chap{7}
\TextTitle{Transgression d'Ephraïm}
\VerseOne{}Lorsque je guérissais Israël, l'iniquité d'Ephraïm et la méchanceté de Samarie se sont révélées, car ils ont agi frauduleusement ; le voleur vient tandis que la bande dépouille au-dehors.
\VS{2}Ils n'ont point pensé dans leur cœur que je me souviens de toute leur méchanceté ; maintenant leurs œuvres les entourent, elles sont devant ma face.
\VS{3}Ils réjouissent le roi par leur méchanceté, et les chefs par leurs mensonges.
\VS{4}Ils sont tous adultères, comme un four allumé par le boulanger : Il cesse d'attiser le feu depuis qu'il a pétri la pâte jusqu'à ce qu'elle soit levée.
\VS{5}Au jour de notre roi, les chefs se rendent malades par les excès de vin ; il tend la main aux moqueurs.
\VS{6}Lorsqu'ils dressent des embuscades, leur cœur s'embrase comme un four ; leur boulanger dort toute la nuit, le matin le four est embrasé comme un feu accompagné de flammes.
\VS{7}Ils sont tous ardents comme un four, et ils dévorent leurs chefs ; tous leurs rois tombent, et il n'y a aucun d'entre eux qui crie à moi.
\VS{8}Ephraïm se mêle avec les peuples, Ephraïm est un gâteau qui n'a pas été retourné.
\VS{9}Les étrangers ont dévoré sa force, et il ne s'en doute pas ; les cheveux gris sont aussi parsemés sur lui, et il ne s'en doute pas.
\VS{10}L'orgueil d'Israël rendra témoignage contre lui ; car ils ne reviennent pas à Yahweh, leur Dieu, et ils ne le recherchent pas malgré tout cela. 
\VS{11}Ephraïm est comme une colombe troublée, sans intelligence ; car ils appellent l'Egypte, et s'en vont vers le roi d'Assyrie.
\VS{12}Quand ils s'en iront, j'étendrai mon filet sur eux, et je les précipiterai comme les oiseaux du ciel ; je les châtierai, comme ils en ont été avertis au sein de leurs assemblées.
\VS{13}Malheur à eux, parce qu'ils me fuient ! Ruine sur eux, car ils se révoltent contre moi ! Je voudrais les sauver, mais ils profèrent contre moi des paroles mensongères.
\VS{14}Ils ne crient pas vers moi dans leur cœur, mais ils gémissent sur leurs couches ; ils se rassemblent pour le froment et le bon vin, et ils s'éloignent de moi.
\VS{15}Je les ai châtiés, et j'ai fortifié leurs bras, mais ils méditent le mal contre moi.
\VS{16}Ce n'est pas au Très-Haut qu'ils retournent ; ils sont comme un arc trompeur. Leurs chefs tomberont par l'épée, à cause de l'insolence de leur langue. C'est ce qui en fera un objet de moquerie dans le pays d'Egypte.
\Chap{8}
\TextTitle{Conséquences de la désobéissance}
\VerseOne{}Crie comme si tu avais un shofar dans ta bouche ! Il vient comme un aigle contre la maison de Yahweh, parce qu'ils ont transgressé mon alliance, et qu'ils ont agi méchamment contre ma loi.
\VS{2}Ils crieront à moi : Mon Dieu, nous te connaissons, dira Israël !
\VS{3}Israël a rejeté le bien ; l'ennemi le poursuivra.
\VS{4}Ils ont fait régner, mais non pas de ma part, ils ont établi des chefs, et je n'en ai rien su ; ils se sont fait des idoles avec leur argent et leur or ; c'est pourquoi ils seront retranchés.
\VS{5}Samarie, ton veau t'a chassée loin ! Ma colère s'est embrasée contre eux. Jusqu'à quand ne pourront-ils pas s'adonner à l'innocence ?
\VS{6}Car il vient d'Israël, c'est un orfèvre qui l'a fait, et il n'est pas Dieu ; c'est pourquoi le veau de Samarie sera mis en pièces.
\VS{7}Parce qu'ils ont semé du vent, ils moissonneront la tempête ; ils n'auront pas un épi de blé ; le grain qui poussera ne donnera point de farine, et s'il en faisait, les étrangers la dévoreraient.
\VS{8}Israël est dévoré ! Il est maintenant parmi les nations comme un vase dont on ne se soucie pas.
\VS{9}Car ils sont montés vers le roi d'Assyrie, qui est un âne sauvage se tenant seul à part ; Ephraïm a fait des présents à ceux qui l'aimait.
\VS{10}Et parce qu'ils ont fait des présents aux nations, je les rassemblerai maintenant ; et ils commenceront à être amoindris à cause de l'impôt pour le roi des princes.
\VS{11}Parce qu'Ephraïm a fait plusieurs autels pour pécher, ils auront des autels pour pécher.
\VS{12}Je lui ai écrit les grandes choses de ma loi, mais elles sont estimées comme des lois étrangères.
\VS{13}Quant aux sacrifices qui me sont offerts, ils sacrifient de la chair, et la mangent ; mais Yahweh ne les accepte point. Et maintenant il se souviendra de leur iniquité, et punira leurs péchés ; ils retourneront en Egypte.
\VS{14}Israël a oublié celui qui l'a fait, et il a bâti des palais ; et Juda a multiplié les villes fortes ; c'est pourquoi j'enverrai le feu dans les villes de celui-ci, quand il aura dévoré les palais de celui-là.
\Chap{9}
\TextTitle{Ephraïm châtié et rejeté}
\VerseOne{}Israël, ne te réjouis point, ne sois pas dans l'allégresse, comme les autres peuples, de ce que tu t'es prostitué en abandonnant ton Dieu, de ce que tu as obtenu un salaire de tes amants dans toutes les aires à blé !
\VS{2}L'aire et la cuve ne les nourriront pas, et le vin doux les trompera.
\VS{3}Ils ne resteront pas dans le pays de Yahweh; Ephraïm retournera en Egypte, et ils mangeront en Assyrie ce qui est impur.
\VS{4}Ils ne feront pas d'aspersions de vin à Yahweh : Elles ne lui seraient point agréables. Leurs sacrifices seront pour eux comme le pain de deuil ; tous ceux qui en mangeront se rendront impurs ; car leur pain ne sera que pour eux, il n'entrera point dans la maison de Yahweh.
\VS{5}Que ferez-vous aux jours des fêtes solennelles, aux jours des fêtes de Yahweh ?
\VS{6}Car voici, ils partent à cause de la dévastation ; l'Egypte les recueillera, Moph les enterrera ; ce qu'ils ont de précieux, leur argent, sera la proie des ronces, et l'épine sera dans leurs tentes.
\VS{7}Les jours du châtiment sont venus, les jours de la rétribution sont venus, et Israël le saura ! Les prophètes sont fous, les hommes de révélation sont insensés, à cause de la grandeur de ton iniquité, et de ta grande aversion.
\VS{8}Ephraïm est une sentinelle avec mon Dieu ; mais le prophète est un filet d'oiseleur sur toutes ses voies, en rébellion contre la maison de son Dieu.
\VS{9}Ils se sont profondément corrompus, comme aux jours de Guibea ; Yahweh se souviendra de leur iniquité, il punira leurs péchés.
\VS{10}J'ai, dira-t-il, trouvé Israël comme des raisins dans le désert ; j'ai vu vos pères comme les premiers fruits d'un figuier ; mais ils sont allés vers Baal-Peor\FTNT{Baal-Peor, « seigneur de la brèche », était une divinité adorée à Peor avec des rites licencieux (No. 23:28 ; No. 25:1-3 ; Ps. 106:28-29).}, ils se sont consacrés à l'infâme idole, et ils sont devenus abominables comme ce qu'ils ont aimé.
\VS{11}La gloire d'Ephraïm s'envolera comme un oiseau : Point d'enfantement, point de grossesse, point de conception.
\VS{12}Que s'ils élèvent leurs enfants, je les en priverai tellement, que pas un d'entre eux ne deviendra homme ; malheur à eux, quand je me retirerai d'eux !
\VS{13}Ephraïm était comme j'ai vu Tyr, plantée dans un lieu agréable ; mais Ephraïm mènera ses fils à celui qui les tuera.
\VS{14}Ô Yahweh, donne-leur ! Mais que leur donnerais-tu ? Donne-leur un sein qui avorte et des mamelles desséchées.
\VS{15}Toute leur méchanceté s'est manifestée à Guilgal ; c'est là que je les ai pris en aversion. Je les chasserai de ma maison à cause de la malice de leurs actions. Je ne les aimerai plus ; tous leurs chefs sont des rebelles.
\VS{16}Ephraïm est frappé, sa racine est devenue sèche ; ils ne porteront plus de fruit ; et s'ils engendrent des enfants, je mettrai à mort les fruits désirables de leur ventre.
\VS{17}Mon Dieu les rejettera, parce qu'ils ne l'ont point écouté, et ils seront vagabonds parmi les nations.
\Chap{10}
\TextTitle{Yahweh annonce la destruction du royaume d'Israël}
\VerseOne{}Israël était une vigne dévastée, elle ne fait de fruit que pour elle-même. Selon l'abondance de son fruit, il a multiplié les autels ; selon la beauté de son pays, il a rendu belles ses statues.
\VS{2}Leur cœur est partagé. Ils vont être déclarés coupables. Yahweh renversera leurs autels, il détruira leurs statues.
\VS{3}Car bientôt ils diront : Nous n'avons point de roi, parce que nous n'avons point craint Yahweh ; et le roi, que pourrait-il faire pour nous ?
\VS{4}Ils prononcent des paroles vaines, des faux serments, lorsqu'ils concluent une alliance. C'est pourquoi le châtiment germera dans les sillons des champs, comme une plante vénéneuse.
\VS{5}Les habitants de Samarie seront épouvantés à cause des jeunes vaches de Beth-Aven; car le peuple mènera deuil sur son idole ; et les prêtres de ses idoles, qui s'en étaient réjouis, mèneront deuil parce que sa gloire est transportée loin d'elle.
\VS{6}Elle sera transportée en Assyrie, pour en faire un présent au roi Jareb. Ephraïm sera dans la confusion, et Israël aura honte de ses desseins.
\VS{7}C'en est fait de Samarie, et de son roi, qui sera retranché comme l'écume qui est à la surface des eaux.
\VS{8}Les hauts lieux de Beth-Aven, qui sont le péché d'Israël, seront détruits ; l'épine et la ronce croîtront sur leurs autels. Et on dira aux montagnes : Couvrez-nous ! Et aux collines : Tombez sur nous !
\VS{9}Israël, tu as péché dès les jours de Guibea ! Là ils restèrent debout, la guerre contre les fils d'iniquité ne les atteignit pas à Guibea.
\VS{10}Je les châtierai selon ma volonté, et les peuples s'assembleront contre eux, lorsqu'on les enchaînera pour leur double iniquité.
\VS{11}Ephraïm est une génisse bien dressée, qui aime à fouler le blé, mais je m'approcherai de son superbe cou ; j'attellerai Ephraïm, Juda labourera, Jacob brisera ses mottes.
\VS{12}Semez selon la justice, moissonnez selon la miséricorde, défrichez-vous un champ nouveau ! Car il est temps de chercher Yahweh, jusqu'à ce qu'il vienne, et répande sur vous sa justice.
\VS{13}Vous avez cultivé la méchanceté, et vous avez moissonné l'iniquité, vous avez mangé le fruit du mensonge; parce que vous avez eu confiance dans vos voies, dans la multitude de vos vaillants hommes.
\VS{14}Il s'élèvera un tumulte parmi ton peuple, et on détruira toutes tes forteresses, comme Schalman a détruit Beth-Arbel, au jour de la bataille, où la mère fut écrasée avec les enfants.
\VS{15}Béthel vous fera de même, à cause de votre extrême méchanceté ; le roi d'Israël sera entièrement exterminé dès l'aurore.
\Chap{11}
\TextTitle{L'amour de Yahweh pour Israël}
\VerseOne{}Quand Israël était jeune enfant, je l'aimais, et j'appelai mon fils hors d'Egypte\FTNT{La sortie des Hébreux de l'Egypte sous Moïse était une préfiguration de celle de Jésus-Christ lorsqu'il fuyait le massacre décrété par Hérode (Mt. 2:15).}.
\VS{2}Lorsqu'on les appelait ils se sont éloignés ; ils ont sacrifié aux Baals, et offert de l'encens aux idoles.
\VS{3}J'appris à Ephraïm à marcher en le prenant par les bras; et ils n'ont pas vu que je les guérissais.
\VS{4}Je les tirai avec des liens d'humanité, et avec des cordages d'amour, et je fus pour eux comme ceux qui enlèveraient le joug de dessus leur mâchoire, et je leur présentai de la nourriture.
\VS{5}Ils ne retourneront pas au pays d'Egypte ; mais le roi d'Assyrie sera leur roi, parce qu'ils n'ont point voulu revenir à moi.
\VS{6}L'épée fondra sur leurs villes, les réduira à néant, consumera leurs forces, et les dévorera, à cause des desseins qu'ils ont eus.
\VS{7}Mon peuple tient à se détourner de moi ; on les appelle vers le Très-Haut, mais aucun d'eux ne l'exalte.
\VS{8}Que ferai-je de toi, Ephraïm ? Te livrerais-je, Israël ? Te traiterai-je comme Adma ? Te rendrai-je semblable à Tseboïm ? Mon cœur s'agite au-dedans de moi, mes compassions sont émues.
\VS{9}Je n'exécuterai pas l'ardeur de ma colère, je ne reviendrai pas pour détruire Ephraïm ; car je suis Dieu, et non pas un homme, je suis le Saint au milieu de toi; et je n'entrerai point dans la ville.
\VS{10}Ils marcheront après Yahweh, qui rugira comme un lion\FTNT{Yahweh rugit comme un lion : Jésus-Christ est le lion de la tribu de Juda, car selon la chair, il est issu de la postérité de Juda (Lu. 3:23-38 ; Ap. 5:5). Le lion est le roi des animaux, or Jacob fut le premier a avoir annoncé la venue du Schilo, c'est-à-dire celui à qui appartient le sceptre (Ge. 49:8-12).}, et quand il rugira, les enfants accourront en hâte de la mer.
\VS{11}Ils accourront en hâte hors d'Egypte, comme des oiseaux, et hors du pays d'Assyrie, comme des colombes. Et je les ferai habiter dans leurs maisons, dit Yahweh.
\Chap{12}
\TextTitle{Dénonciation du péché d'Ephraïm}
\VerseOne{}Ephraïm m'entoure avec des mensonges, et la maison d'Israël avec des tromperies ; lorsque Juda erre sans frein vis-à-vis du Dieu Puissant, vis-à-vis du Saint fidèle.
\VS{2}Ephraïm se repaît de vent, et poursuit le vent d'orient ; il multiplie chaque jour le mensonge et la violence, et il traite alliance avec l'Assyrie, et l'on porte des huiles de senteur en Egypte.
\VS{3}Yahweh a aussi un procès avec Juda, et il punira Jacob pour sa conduite, il lui rendra selon ses œuvres.
\VS{4}Dans le ventre Jacob saisit son frère par le talon\FTNT{Ge. 25:26}, puis dans sa vigueur, il lutta avec Dieu\FTNT{Ge. 32:24-28.}.
\VS{5}Il lutta avec l'Ange, et il fut vainqueur, il pleura, et lui demanda grâce. Jacob l'avait rencontré à Béthel, et c'est là que Dieu nous a parlé.
\VS{6}Yahweh est le Dieu des armées ; Yahweh est son mémorial.
\VS{7}Et toi donc, reviens à ton Dieu, garde la miséricorde et la justice, et espère toujours en ton Dieu.
\VS{8}Ephraïm est un marchand\FTNT{Ephraïm est appelé « marchand », littéralement « Canaan ». Notez que l'ange de Laodicée s'exprime comme Ephraïm : « Je suis riche, je me suis enrichi… » (Ap. 3:14-19).}, qui a dans sa main des balances fausses, il aime à frauder.
\VS{9}Et Ephraïm dit : Quoi qu'il en soit, je suis devenu riche ; je me suis acquis des richesses ; c'est entièrement le produit de mon travail ; on ne trouvera en moi aucune iniquité, rien qui soit un péché.
\VS{10}Et moi, je suis Yahweh, ton Dieu, dès le pays d'Egypte ; je te ferai encore habiter dans des tentes, comme aux jours des fêtes solennelles.
\VS{11}Je parlerai par les prophètes, et je multiplierai les visions, et par les prophètes, je proposerai des paraboles.
\VS{12}Certainement Galaad n'est qu'iniquité, certainement ils ne seront que vanité. Ils sacrifient des bœufs dans Guilgal ; même leurs autels seront comme des monceaux de pierres sur les sillons des champs.
\VS{13}Jacob s'enfuit au pays de Syrie, et Israël servit pour une femme, et pour une femme il garda les troupeaux.
\VS{14}Par un prophète, Yahweh fit monter Israël hors d'Egypte, et par un prophète, Israël fut gardé.
\VS{15}Mais Ephraïm a provoqué Yahweh à une amère colère ; son Seigneur laissera sur lui le sang qu'il a répandu, et lui rendra ses mépris.
\Chap{13}
\TextTitle{Ephraïm persiste dans sa méchanceté}
\VerseOne{}Quand Ephraïm parlait, c'était une terreur ; il s'éleva en Israël. Mais il se rendit coupable par Baal, et mourut.
\VS{2}Et maintenant ils continuent de pécher, et se sont fait avec leur argent des images de fonte, des idoles selon leurs pensées; toutes sont un travail d'artisans, desquelles ils disent : Que les hommes qui sacrifient embrassent\FTNT{C'est une expression d'hommage.} les veaux !
\VS{3}C'est pourquoi ils seront comme la nuée du matin, et comme la rosée qui bientôt disparaît ; comme la balle qui est emportée par le vent hors de l'aire, comme la fumée sortant de la cheminée.
\VS{4}Et moi, je suis Yahweh, ton Dieu, dès le pays d'Egypte. Et tu ne devrais reconnaître d'autre dieu que moi, et il n'y a pas d'autre Sauveur que moi\FTNT{Yahweh dit qu'il n'y a pas d'autre sauveur que lui (Es. 43:11). Et les Ecrits de la Nouvelle Alliance nous présentent clairement notre Sauveur : Jésus-Christ (Mt.1:21; Ac.13:23 ; 2 Ti. 1:10 ; Tit : 1:4).}.
\VS{5}Je t'ai connu dans le désert, dans une terre aride.
\VS{6}Ils se sont rassasiés dans leurs pâturages ; ils se sont rassasiés, et leur cœur s'est enflé ; alors ils m'ont oublié.
\VS{7}Je serai pour eux comme un lion ; je les épierai sur la route comme un léopard.
\VS{8}Je les attaquerai, comme une ourse à qui on a enlevé ses petits, et je déchirerai l'enveloppe de leur cœur ; et là, je les dévorerai comme un lion ; les bêtes des champs les mettront en pièces.
\TextTitle{Châtiment d'Ephraïm}
\VS{9}Ta ruine, ô Israël, c'est que tu as été contre moi, alors que moi seul pouvais te secourir !
\VS{10}Où donc est ton roi ? Qu'il te délivre dans toutes tes villes ! Où sont tes juges, au sujet desquels tu as dit : Donne-moi un roi et des princes ?
\VS{11}Je t'ai donné un roi\FTNT{Ce passage concerne Saül, premier roi d'Israël (1 S. 8, 9 et 10)} dans ma colère, et je l'ôterai dans ma fureur.
\VS{12}L'iniquité d'Ephraïm est enveloppée, et son péché est mis en réserve.
\VS{13}Les douleurs comme de celle qui enfante le surprendront ; c'est un enfant qui n'est pas sage, qui, au temps marqué, ne sort pas du sein maternel.
\VS{14}Je les rachèterai de la puissance du scheol, je les délivrerai de la mort\FTNT{L'auteur de l'épître aux Hébreux applique ce passage à la victoire que le Seigneur Jésus-Christ a remportée face à la mort lors de sa résurrection (Hé. 2 :14-18). Depuis la chute d'Adam, les hommes ont toujours eu peur de la mort. Cette peur est d'autant plus forte de nos jours, car la plupart des gens sont angoissés par son aspect imprévisible, inévitable et par son non-sens. Et bien que beaucoup ne croient pas à l'existence de la vie après la mort (au paradis ou à l'enfer), la mort associée à l'annihilation, au non-être, apparaît d'autant plus monstrueuse et insupportable. Or notre Seigneur Jésus-Christ a vaincu la mort et il promet la vie éternelle à ceux qui croient en lui (Jn. 3:16 ; Jn. 5:24-29 ; Ap. 1:18). En plaçant notre foi en lui, nous avons non seulement la victoire sur la mort, mais aussi sur l'angoisse qu'elle produit dans le cœur de tout homme.} ; ô mort, où est ta peste ? Scheol, où est ta destruction\FTNT{1 Co. 15:55-57} ? Mais le repentir se cache à mes yeux !
\VS{15}Ephraïm a beau être fertile au milieu de ses frères, le vent d'orient, le vent de Yahweh s'élèvera du désert, viendra, desséchera ses sources et tarira ses fontaines. On pillera le trésor de tous ses objets précieux.
\VS{16}Samarie sera châtiée, car elle s'est rebellée contre son Dieu. Ils tomberont par l'épée; leurs petits enfants seront écrasés, et l'on fendra le ventre de leurs femmes enceintes.
\Chap{14}
\TextTitle{Bénédiction future d'Israël}
\VerseOne{}Israël, reviens à Yahweh ton Dieu ; car tu es tombé par ton iniquité.
\VS{2}Apportez avec vous des paroles, et revenez à Yahweh. Dites-lui : Pardonne toutes nos iniquités, et reçois le bien, pour le mettre à sa place! Et nous t'offrirons pour sacrifices la louange de nos lèvres.
\VS{3}L'Assyrie ne nous sauvera pas, nous ne monterons pas sur des chevaux, et nous ne dirons plus à l'ouvrage de nos mains : Notre dieu ! Car c'est auprès de toi que l'orphelin trouve de la compassion.
\VS{4}Je guérirai leur rébellion, et les aimerai volontairement ; parce que ma colère s'est détournée d'eux.
\VS{5}Je serai comme la rosée pour Israël ; il fleurira comme le lis, et il poussera ses racines comme le Liban.
\VS{6}Ses branches s'étendront, et sa magnificence sera comme celle de l'olivier, avec un parfum comme celui du Liban.
\VS{7}Ils reviendront s'asseoir à son ombre, et ils redonneront la vie au froment, et ils fleuriront comme la vigne ; et l'odeur de chacun d'eux sera comme celle du vin du Liban.
\VS{8}Ephraïm dira : Qu'ai-je à faire encore avec les idoles ? Je l'exaucerai, je le regarderai, je serai pour lui comme un cyprès verdoyant. C'est de moi que tu recevras ton fruit.
\VS{9}Qui est celui qui est sage ? Qu'il entende ces choses ! Et qui est celui qui est prudent ? Qu'il les connaisse ! Car les voies de Yahweh sont droites ; aussi les justes y marcheront, mais les rebelles y tomberont.
\PPE{}
\end{multicols}

%\clearpage\ShortTitle{Joël}\BookTitle{Joël}\BFont
\noindent\hrulefill
{\footnotesize
\textit{
\bigskip
{\centering{}
\\Auteur : Joël
\\(Heb. : Yow'el)
\\Signification : Yahweh est Dieu
\\Thème : Le jour de Yahweh
\\Date de rédaction : 9ème ou 8ème siècle av. J.-C.\\}
}
%\bigskip
\textit{
\\Joël, fils de Pethuel, exerça son ministère dans le royaume de Juda. Son message faisait suite à deux fléaux qui s’étaient abattus sur Juda, à savoir une invasion de sauterelles et la sécheresse. Il s’agissait d’un avertissement de Yahweh qui appelait le peuple à revenir à lui avec la promesse de le restaurer dans tout ce qu’il avait perdu. Joël annonça en outre l’effusion de l’Esprit sur toute chair dans un avenir lointain, prophétie ayant trouvé son accomplissement à la naissance de l’Eglise lors de la Pentecôte.\bigskip
}
}
\par\nobreak\noindent\hrulefill
\begin{multicols}{2}
\Chap{1}
\VerseOne{}La parole de Yahweh qui fut adressée à Joël, fils de Pethuel.
\VS{2}Anciens écoutez ceci ! Et vous, tous les habitants du pays, prêtez l'oreille ! Rien de pareil est-il arrivé de votre temps, ou même du temps de vos pères ?
\VS{3}Racontez-le à vos enfants, et que vos enfants le racontent à leurs enfants, et leurs enfants à la génération suivante !
\TextTitle{Désolation  après  l'invasion des sauterelles}
\VS{4}La sauterelle a dévoré les restes du gazam, le jélek a dévoré les restes de la sauterelle, et le hasil a dévoré les restes du jélek.
\VS{5}Ivrognes, réveillez-vous, et pleurez ; et vous tous buveurs de vin hurlez à cause du vin nouveau, parce qu’il est retranché de votre bouche.
\VS{6}Car une nation puissante et innombrable est montée contre mon pays. Elle a les dents d’un lion et les mâchoires d’un vieux lion.
\VS{7}Elle a réduit ma vigne en désert ; et a ôté l’écorce de mes figuiers ; elle les a entièrement dépouillés, et les a abattus, leurs branches en sont devenues blanches.
\VS{8}Lamente-toi, comme une jeune fille qui se revêt d'un sac pour pleurer le mari de sa jeunesse !
\VS{9}L'offrande et la libation sont retranchées de la maison de Yahweh, et les sacrificateurs qui font le service de Yahweh mènent deuil.
\VS{10}Les champs sont ravagés, la terre est dans le deuil ; parce que le blé est détruit, le moût est tari, l'huile est desséchée.
\VS{11}Les laboureurs sont confus, les vignerons gémissent, à cause du froment et de l'orge, car la moisson des champs est perdue.
\VS{12}La vigne est desséchée, le figuier languissant ; le grenadier, le palmier, le pommier, tous les arbres des champs ont séché, c'est pourquoi la joie a cessé parmi les fils de l’homme !
\VS{13}Sacrificateurs, ceignez-vous et pleurez ! Poussez des gémissements, vous qui faites le service de l’autel, hurlez, vous qui faites le service de mon Dieu ; entrez, passez la nuit vêtus de sacs car il est défendu à l'offrande et à la libation d’entrer dans la maison de votre Dieu.
\TextTitle{Désolation après la sécheresse et la famine}
\VS{14}Sanctifiez le jeûne, publiez l’assemblée solennelle, assemblez les anciens, et tous les habitants du pays dans la maison de Yahweh votre Dieu, et criez à Yahweh en disant :
\VS{15}Hélas ! Quel jour ! Car le jour de Yahweh\FTNT{Jour de Yahweh : Voir commentaire en Za. 14:1.} est proche : Il vient comme un ravage fait par le Tout-Puissant.
\VS{16}La nourriture n’est-elle pas retranchée sous nos yeux ? Et la joie et l'allégresse de la maison de notre Dieu ?
\VS{17}Les semances sont pourries sous leurs mottes, les magasins sont dévastés, les greniers sont renversés parce que le blé a manqué.
\VS{18}Ô combien ont gémi les bêtes, et dans quelle peine ont été les troupeaux de bœufs, parce qu’ils n’ont point de pâturage ! Aussi les troupeaux de brebis sont dévastés.
\VS{19}Ô Yahweh, je crierai à toi, car le feu a consumé les pâturages du désert, et la flamme a brûlé tous les arbres des champs.
\VS{20}Même toutes les bêtes des champs crient aussi vers toi ; car les torrents d’eau sont à sec, et le feu a consumé les pâturages du désert.
\Chap{2}
\TextTitle{Le jour de Yahweh, invasion future}
\VerseOne{}Sonnez du shofar en Sion, et sonnez avec un retentissement bruyant dans la montagne de ma sainteté ; que tous les habitants du pays tremblent ; car le jour de Yahweh vient ; car il est proche,
\VS{2}jour de ténèbres et d'obscurité, jour de nuées et de brouillards, il vient comme l'aurore s'étend sur les montagnes. Voici un peuple nombreux et puissant, tel qu’il n’y en a jamais eu, et qu’il n’y en aura jamais dans la suite des siècles.
\VS{3}Devant lui est un feu dévorant, et derrière lui la flamme brûle ; le pays était, avant sa venue, comme le jardin d’Eden, et après qu’il sera parti il sera comme un désert affreux ; et même il n’y aura rien qui lui échappe.
\VS{4}Leur aspect est comme l’aspect des chevaux, et ils courent comme des cavaliers.
\VS{5}C’est comme le bruit de chariots, quand ils sautent au sommet des montagnes, comme le bruit d’une flamme de feu, qui dévore le chaume, comme un peuple puissant rangé en bataille.
\VS{6}Les peuples tremblent en le voyant ; tous les visages en deviennent pâles et livides.
\VS{7}Ils courent comme des hommes vaillants, et montent sur les murailles comme des gens de guerre ; chacun va son chemin, sans se détourner de son chemin.
\VS{8}Ils ne se pressent point les uns les autres, chacun va son chemin ; ils se jettent au travers des épées sans être blessés.
\VS{9}Ils courent çà et là dans la ville, se précipitent sur les murailles, montent sur les maisons, entrent par les fenêtres comme le voleur.
\VS{10}La terre tremble devant eux, les cieux sont ébranlés, le soleil et la lune s’obscurcissent, et les étoiles retirent leur éclat.
\VS{11}Aussi Yahweh fait entendre sa voix devant son armée ; parce que son camp est très grand, car l'exécuteur de sa parole est puissant. Certainement le jour de Yahweh est grand et terrible. Qui peut le supporter ?
\TextTitle{Repentance et miséricorde}
\VS{12}Maintenant encore, dit Yahweh, revenez à moi de tout votre cœur, avec des jeûnes, avec des pleurs et des lamentations !
\VS{13}Déchirez vos cœurs et non vos vêtements, et revenez à Yahweh, votre Dieu ; car il est compatissant et miséricordieux, lent à la colère et riche en bonté, et il se repent d’avoir affligé.
\VS{14}Qui sait si Yahweh, votre Dieu, ne reviendra pas et ne se repentira pas, et s'il ne laissera point après lui la bénédiction, des offrandes et des libations ?
\VS{15}Sonnez du shofar en Sion ! Sanctifiez le jeûne, publiez l'assemblée solennelle !
\VS{16}Assemblez le peuple, sanctifiez la congrégation ! Réunissez les anciens, assemblez les enfants, même les nourrissons à la mamelle ! Que l’époux sorte de sa demeure, et l’épouse de sa chambre nuptiale !
\VS{17}Que les sacrificateurs qui font le service de Yahweh pleurent entre le portique et l'autel, et qu'ils disent : Yahweh ! Epargne ton peuple ! N’expose pas ton héritage à l'opprobre, que les nations n’en fassent pas un sujet de railleries ! Pourquoi dirait-on parmi les peuples : Où est leur Dieu ?
\TextTitle{Promesse de restauration}
\VS{18}Or Yahweh est jaloux pour son pays, et il est ému de compassion envers son peuple.
\VS{19}Yahweh répond et il dit à son peuple : Voici, je vous enverrai du blé, du moût, et de l'huile, et vous en serez rassasiés ; et je ne vous exposerai plus à l'opprobre parmi les nations.
\VS{20}J'éloignerai de vous l’armée venue du nord, je la chasserai vers une terre aride et déserte, son avant-garde dans la mer orientale, son arrière-garde dans la mer occidentale ; et sa puanteur montera, et son infection s’élèvera, après avoir fait de grandes choses.
\VS{21}Terre, ne crains pas, sois dans l’allégresse et réjouis-toi, car Yahweh fait de grandes choses !
\VS{22}Ne craignez point, bêtes des champs, car les pâturages du désert ont poussé leur jet, et même les arbres portent leur fruit ; le figuier et la vigne ont poussé avec vigueur.
\VS{23}Et vous, enfants de Sion, soyez dans l’allégresse et réjouissez-vous en Yahweh, votre Dieu, car il vous donnera la pluie selon sa justice, il vous enverra la pluie de la première\FTNT{La pluie de la première saison : En Orient, la première pluie tombe au moment des semailles d’automne. Elle est nécessaire afin que la semence puisse germer. Sous l'influence des pluies fertilisantes, les tendres pousses sortent du sol.} et de l’arrière-saison\FTNT{La pluie de l’arrière-saison : Elle tombe vers la fin de la saison, mûrit le grain et le prépare pour la moisson. C’est la pluie du printemps. Voir Jé. 5:24 ; Os. 6:1-3 ; Za. 10:1.}, au premier mois.
\VS{24}Et les aires se rempliront de blé, et les cuves regorgeront de moût et d'huile.
\VS{25}Ainsi je vous rendrai les fruits des années qu'ont dévoré la sauterelle, le jélek, le hasil et le gazam, ma grande armée que j’avais envoyée contre vous.
\VS{26}Vous aurez donc abondamment de quoi manger et être rassasiés, et vous louerez le Nom de Yahweh votre Dieu, qui aura fait pour vous des choses merveilleuses ; et mon peuple ne sera plus jamais dans la confusion.
\VS{27}Et vous saurez que je suis au milieu d'Israël, que je suis Yahweh, votre Dieu, et qu'il n'y en a point d'autre, et mon peuple ne sera plus jamais dans la confusion.
\TextTitle{La promesse de l'Esprit}
\VS{28}Et il arrivera après cela, que je répandrai mon Esprit sur toute chair\FTNT{Cette promesse s’est réalisée dans Actes 2. Elle se réalise encore aujourd’hui dans la vie de chaque enfant de Dieu. Enfin, elle sera pleinement réalisée lors du retour du Messie en Israël (Za. 12:10-14) puisque cette prophétie annonce la repentance nationale d’Israël (Ro. 11:26-27).} ; et vos fils et vos filles prophétiseront ; vos vieillards songeront des songes, et vos jeunes gens verront des visions.
\VS{29}Et même en ces jours-là, je répandrai mon Esprit sur les serviteurs et sur les servantes.
\TextTitle{Prodiges précédant le jour de Yahweh\FTNTT{Es. 13:9-10 ; 24:21-23 ; Ez. 32:7-10 ; Mt. 24:29-30.}}
\VS{30}Je ferai des prodiges dans les cieux et sur la terre, du sang, et du feu, et des colonnes de fumée ;
\VS{31}Le soleil se changera en ténèbres, et la lune en sang, avant que le grand et terrible jour de Yahweh vienne.
\VS{32}Et il arrivera que quiconque invoquera le Nom de Yahweh\FTNT{Quiconque invoquera le Nom de Yahweh sera sauvé. Ce passage nous confirme que Jésus-Christ est vraiment Yahweh. En effet, Paul, apôtre des païens, attribue le Nom de Yahweh et cette prophétie à Jésus-Christ (Ro. 10:9-13). C’est bien le Nom de Jésus-Christ qu'il faut invoquer pour être sauvé (Ac. 4:12 ; Ac. 9:21 ; 1 Co. 1:2). Les éditeurs de la Traduction du Monde Nouveau (bible des témoins de Jéhovah) se sont permis de « restituer » le Nom divin YHWH qui apparaît près de 6000 fois dans le Tanakh, en 237 endroits dans les écrits de la nouvelle alliance, alors qu’aucun ancien manuscrit de la nouvelle alliance (testament de Jésus) ne le contient. Ils affirment, sur la base d’éléments de preuves indirectes, que les scribes du IIème siècle remplaçaient le Nom divin dans la Nouvelle Alliance par « Seigneur » ou « Dieu ». Pour restituer ce Nom (YHWH), ils se basent sur les citations du Tanakh où celui-ci figure et sur des versions hébraïques de la nouvelle alliance dont la plus ancienne date du XIVème siècle pour la plupart des copies de textes plus anciens. On constate cependant qu’ils n’ont pas restitué le Nom divin en 1 Pierre 2:3 qui est pourtant une citation du Psaumes 34:8. Pourquoi ? 
Parce que l’application de ce texte à Jésus-Christ, la pierre rejetée, est évidente. Si ce texte du Tanakh mentionnant Yahweh est appliqué à Jésus que penser des autres ? Jésus-Christ est vraiment Yahweh qui s’est incarné pour nous sauver. D'ailleurs, le Nom de Jésus veut dire « YHWH est Sauveur » (Es. 7:14 ; Es. 9:5 ; Mt. 1 ; Lu. 1 ; 1 Ti. 3:16).} sera sauvé ; car le salut sera sur la montagne de Sion et dans Jérusalem, comme l’a dit Yahweh, et parmi les réchappés que Yahweh appellera.
\Chap{3}
\TextTitle{Rétablissement d'Israël\FTNTT{Es. 11:10-12 ; Jé. 23:5-8 ; Ez. 37:21-28 ; Ac. 15:15-17.}}
\VerseOne{}Car voici, en ces jours-là, et en ce temps-là, quand je ramènerai les captifs de Juda et de Jérusalem,
\TextTitle{Jugements des nations étrangères\FTNTT{Za. 12:2-3.}}
\VS{2}J'assemblerai toutes les nations\FTNT{Dieu rassemblera les nations dans la vallée de Josaphat (de l'hébreu « Yehowshaphat », « Yahweh a jugé ») pour leur jugement. Cette vallée est peut-être celle où le roi Josaphat remporta une grande victoire, avec beaucoup de facilité, sur les Moabites, les Ammonites et les Maonites (2 Ch. 20). Cette vallée s'étend à l'orient de Jérusalem, entre la ville et le Mont des Oliviers, et traverse le torrent de Cédron.}, et je les ferai descendre dans la vallée de Josaphat ; là, j'entrerai en jugement avec elles, à cause de mon peuple, et d'Israël, mon héritage, lequel ils ont dispersé parmi les nations, et parce qu’ils ont partagé entre eux mon pays ;
\VS{3}et qu'ils ont tiré mon peuple au sort ; ils ont donné l’enfant pour une prostituée, ils ont vendu la jeune fille pour du vin, et ils ont bu.
\VS{4}Et qu’ai-je aussi affaire de vous, Tyr et Sidon, et de vous, toutes les limites de la Palestine, me rendrez-vous ma récompense, ou voulez-vous m'irriter ? Je vous rendrai promptement et sans délai votre récompense sur votre tête.
\VS{5}Car vous avez pris mon argent et mon or ; et vous avez emporté dans vos temples ce que j’avais de plus précieux et de plus beau.
\VS{6}Vous avez vendu les enfants de Juda et de Jérusalem aux enfants des Grecs, afin de les éloigner de leur territoire.
\VS{7}Voici, je les ferai lever\FTNT{Le verbe lever vient de l'hébreu «'uwr » qui signifie « se réveiller », « éveiller », « être éveillé » , « inciter », « veiller »,  « se lever », « sortir de l’assoupissement », «prendre courage». Yahweh annonce le réveil des hébreux depuis les nations, d'où ils sont établis. Ce réveil est une prise de conscience qui aboutira au retour à la terre sainte.} du lieu où ils ont été transportés après que vous les avez vendus ; et je ferai retourner votre récompense sur votre tête.
\VS{8}Je vendrai donc vos fils et vos filles entre les mains des enfants de Juda, et ils les vendront à ceux de Séba, qui les transporteront vers une nation éloignée ; car Yahweh a parlé.
\VS{9}Publiez ceci parmi les nations ! Préparez la guerre ! Réveillez les hommes vaillants ! Qu’ils s’approchent, et qu’ils montent, tous les hommes de guerre !
\VS{10}Forgez des épées de vos hoyaux, et des lances de vos serpes ! Et que le faible dise : Je suis fort !
\VS{11}Hâtez-vous et venez, vous toutes les nations d'alentour, et rassemblez-vous ! Là, ô Yahweh, fais descendre tes hommes vaillants !
\VS{12}Que les nations se réveillent, et qu'elles montent à la vallée de Josaphat ! Car là je siégerai pour juger toutes les nations d'alentour.
\VS{13}Saisissez la faucille, car la moisson est mûre ! Venez, et descendez, car le pressoir est plein, les cuves regorgent ! Car leur méchanceté est grande,
\VS{14}Des multitudes, des multitudes, dans la vallée du jugement ; car le jour de Yahweh est proche, dans la vallée du jugement.
\VS{15}Le soleil et la lune s’obscurcissent, et les étoiles retirent leur éclat.
\VS{16}De Sion Yahweh rugit, de Jérusalem il fait entendre sa voix ; les cieux et la terre sont ébranlés. Mais Yahweh est le refuge pour son peuple, et la forteresse\FTNT{Jésus-Christ est notre rocher (commentaire Es. 8:14 ; Ps. 78:35 ; 1 Co. 10:4).} pour les enfants d’Israël.
\VS{17}Et vous saurez que je suis Yahweh, votre Dieu, qui habite à Sion, ma sainte montagne. Jérusalem sera sainte, et les étrangers n'y passeront plus.
\TextTitle{Restauration finale et pleine bénédiction du royaume}
\VS{18}Et il arrivera en ce jour-là, le moût ruissellera des montagnes, le lait coulera des collines, il y aura de l’eau dans tous les torrents de Juda ; et une source\FTNT{Jésus est celui qui fait jaillir en nous une source d’eau qui étanche notre soif à jamais et nous donne la vie éternelle (Jé. 2:13 ; Jé. 17:13 ; Ez. 47:1-12 ; Za. 14:8 ; Jn. 4:14 ; Ap. 22:1).} sortira de la maison de Yahweh, et arrosera la vallée de Sittim.
\VS{19}L'Egypte sera dévastée, Edom sera réduit en désert de désolation, à cause de la violence faite aux enfants de Juda, dont ils ont répandu le sang innocent dans leur pays.
\VS{20}Mais là, Judas sera éternellement habitée, et Jérusalem, d’âge en âge.
\VS{21}Et je nettoierai leur sang que je n’avais point nettoyé ; car Yahweh habite en Sion.

\PPE{}
\end{multicols}

%\clearpage\ShortTitle{Amos}\BookTitle{Amos}\BFont
\noindent\hrulefill
{\footnotesize
\textit{
\bigskip
{\centering{}
\\Auteur : Amos
\\(Heb. : Amowc)
\\Signification : Fardeau, porteur de fardeau
\\Thème : Jugement sur le péché
\\Date de rédaction : 8ème siècle av. J.-C.\\}
}
%\bigskip
\textit{
\\Originaire de Tekoa, Amos exerça son ministère dans le royaume du nord, au temps d’Ozias,  roi de Juda, et Jéroboam II, roi d’Israël. Il fut aussi le contemporain des prophètes Osée, Michée, Jonas et Esaïe.
%\bigskip
\\Alors que le peuple juif jouissait d’une certaine prospérité, l’immoralité et les sacrilèges prirent place dans le royaume. Amos avertit le peuple de son péché et du jugement qu'il encourait. Il lui rappela la bonté de Dieu et l’invita à revenir à Yahweh et à lui rester fidèle.\bigskip
}
}
\par\nobreak\noindent\hrulefill
\begin{multicols}{2}
\Chap{1}
\VerseOne{}Paroles d'Amos, berger de Tekoa, qui prophétisa sur Israël, du temps d’Ozias, roi de Juda, et de Jéroboam, fils de Joas, roi d'Israël, deux ans avant le tremblement de terre\FTNT{Za. 14:5}.
\VS{2}Il dit : Yahweh rugit de Sion, et fait entendre sa voix de Jérusalem. Les habitations des bergers sont en deuil, et le sommet du Carmel est desséché\FTNT{Jé. 25:30 ; Joë 3:16}.
\TextTitle{Yahweh annonce ses jugements sur les villes et les pays d'alentour}
\VS{3}Ainsi parle Yahweh : A cause de trois crimes de Damas, et même de quatre, je ne rappellerai point cela, mais je le ferai\FTNT{Il est question du jugement de Dieu.}, parce qu'ils ont foulé Galaad avec des herses de fer\FTNT{Es. 17:1}.
\VS{4}J'enverrai le feu dans la maison de Hazaël, et il dévorera le palais de Ben-Hadad.
\VS{5}Je briserai aussi les verrous de Damas, j'exterminerai de Bikath-Aven ses habitants, et de Beth-Eden celui qui tient le sceptre. Et le peuple de Syrie sera mené captif à Kir, dit Yahweh.
\VS{6}Ainsi parle Yahweh : A cause de trois crimes de Gaza, et même de quatre, je ne rappellerai point cela, mais je le ferai\FTNT{Il est question du jugement de Dieu.} parce qu'ils ont emmené des captifs en grand nombre pour les livrer à Edom\FTNT{Ez. 25:13-17}.
\VS{7}J'enverrai le feu dans les murs de Gaza, et il dévorera ses palais.
\VS{8}J'exterminerai d'Asdod les habitants, et d'Askalon celui qui tient le sceptre ; je tournerai ma main contre Ekron, et le reste des Philistins périra, dit le Seigneur, Yahweh.
\VS{9}Ainsi parle Yahweh : A cause de trois crimes de Tyr, et même de quatre, je ne rappellerai point cela, mais je le ferai\FTNT{Il est question du jugement de Dieu.}, parce qu'ils ont livré à Edom des captifs en grand nombre sans se souvenir de l'alliance fraternelle\FTNT{Ez. 26:2}.
\VS{10}J'enverrai le feu dans les murs de Tyr, et il dévorera ses palais.
\VS{11}Ainsi parle Yahweh : A cause de trois crimes d'Edom, et même de quatre, je ne rappellerai point cela,, mais je le ferai\FTNT{Il est question du jugement de Dieu.}, parce qu'il a poursuivi son frère avec l'épée, refoulant toute compassion, parce que sa colère déchire continuellement et qu'il garde sa fureur éternellement.
\VS{12}J'enverrai le feu dans Théman, et il dévorera les palais de Botsra\FTNT{Jé. 49:7 ; Abd. 1:9}.
\VS{13}Ainsi parle Yahweh : A cause de trois crimes des enfants d'Ammon, et même de quatre, je ne rappellerai point cela, mais je le ferai\FTNT{Il est question du jugement de Dieu.}, parce qu’ils ont fendu le ventre des femmes enceintes de Galaad pour étendre leurs frontières\FTNT{Ez. 21:33 ; So. 2:8}.
\VS{14}J'allumerai le feu dans les murs de Rabba, et il dévorera les palais, au bruit des cris de guerre au jour du combat, et au milieu de l’ouragan au jour de la tempête.
\VS{15}Et leur roi ira en captivité, lui et ses chefs, dit Yahweh.
\Chap{2}
\TextTitle{Suite des jugements prononcés sur les villes et les pays d'alentour}
\VerseOne{}Ainsi parle Yahweh : A cause de trois crimes de Moab, et même de quatre, je ne rappellerai point cela, mais je le ferai, parce qu'il a brûlé les os du roi d'Edom jusqu’à les calciner.
\VS{2}J'enverrai le feu dans Moab, et il dévorera les palais de Kerijoth ; et Moab périra dans le tumulte, au milieu des cris de guerre et du bruit du shofar\FTNT{Ez. 25:8-9}.
\VS{3}J'exterminerai les juges de son pays, et je tuerai tous ses chefs, dit Yahweh.
\TextTitle{Juda et Israël jugés à cause de leurs iniquités}
\VS{4}Ainsi parle Yahweh : A cause de trois crimes de Juda, et même de quatre, je ne rappellerai point cela, mais je le ferai, parce qu'ils ont rejeté la loi de Yahweh et n'ont point gardé ses ordonnances ; parce qu’ils ont été égarés par les mensonges après lesquels leurs pères ont marché.
\VS{5}J'enverrai le feu dans Juda, et il dévorera les palais de Jérusalem.
\VS{6}Ainsi parle Yahweh : A cause de trois crimes d'Israël, et même de quatre, je ne rappellerai point cela, mais je le ferai, parce qu'ils ont vendu le juste pour de l'argent, et le pauvre pour une paire de souliers.
\VS{7}Ils aspirent à voir la poussière de la terre sur la tête des misérables, et ils pervertissent la voie des pauvres. Le fils et le père vont vers la même jeune fille, pour profaner mon Saint Nom.
\VS{8}Ils se couchent près de chaque autel, sur les vêtements qu'ils ont pris en gage, et boivent dans la maison de leurs dieux le vin de ceux qu’ils châtient.
\VS{9}Pourtant j'ai détruit devant eux les Amoréens qui étaient hauts comme les cèdres et forts comme les chênes ; j'ai détruit son fruit en haut, et ses racines en bas\FTNT{No. 21:24 ; Jos. 24:8}.
\VS{10}Je vous ai fait monter du pays d'Egypte et je vous ai conduits dans le désert quarante ans pour que vous possédiez le pays des Amoréens.
\VS{11}J'ai suscité quelques-uns d'entre vos fils pour être prophètes, et quelques-uns d'entre vos jeunes gens pour être nazaréens\FTNT{Le mot nazaréen vient de l'hébreu nâzîr, de la racine nâzar qui signifie séparer. Il y avait deux types de nazaréens. Premièrement ceux qui étaient appelés par Dieu. Par exemple : Samson Jg. 13:1-7 ; Samuel 1 S. 1:11 ; Jean-Baptiste Lu. 1:15. Deuxièmement les personnes qui voulaient se consacrer à Dieu. No. 6:13.}. N'en est-il pas ainsi, enfants d'Israël ? dit Yahweh.
\VS{12}Mais vous avez fait boire du vin aux nazaréens, et vous avez donné cet ordre aux prophètes disant : Ne prophétisez pas\FTNT{Es. 30:10 ; Jé. 11:21 ; Mi. 2:6} !
\VS{13}Voici, je m’en vais fouler le lieu où vous habitez, comme un chariot plein de gerbes foule tout par où il passe.
\VS{14}Tellement que l’homme agile ne pourra pas fuir, et le fort ne pourra pas se servir de sa vigueur, et l’homme vaillant ne sauvera pas sa vie\FTNT{Jé. 46:6}.
\VS{15}Celui qui manie l'arc, ne pourra pas tenir ferme, et celui qui a les pieds légers n'échappera pas, et le cavalier ne sauvera pas sa vie.
\VS{16}Le plus courageux d’entre les hommes vaillants s'enfuira tout nu en ce jour-là, dit Yahweh.
\Chap{3}
\TextTitle{La maison de Jacob coupable devant Yahweh}
\VerseOne{}Enfants d’Israël écoutez la parole que Yahweh prononce contre vous, contre toutes les familles que j'ai fait monter du pays d'Egypte.
\VS{2}Je vous ai connu vous seuls d'entre toutes les familles de la terre ; c'est pourquoi je vous châtierai pour toutes vos iniquités\FTNT{Ex. 19:5-6 ; Ps. 147:19-20}.
\VS{3}Deux hommes marchent-ils ensemble s’ils ne sont pas accordés ?
\VS{4}Le lion rugit-il dans la forêt sans qu’il n'ait de proie ? Le lionceau jette-t-il son cri de sa tanière sans qu’il n'ait rien attrapé ?
\VS{5}L'oiseau tombe-t-il dans le filet posé à terre sans que ce ne soit un piège ? Le filet est-il ramassé par terre sans qu’il n’y ait rien de capturé ?
\VS{6}Le shofar sonne-t-il dans une ville sans que le peuple en étant tout effrayé s’assemble ? Arrive-t-il un malheur dans une ville sans que Yahweh ne l’ait causé\FTNT{Es. 45:7 ; La. 3:37-38} ?
\VS{7}Car le Seigneur, Yahweh, ne fait aucune chose qu'il n'ait révélé son secret aux prophètes ses serviteurs.
\VS{8}Le lion rugit, qui ne serait pas effrayé ? Le Seigneur, Yahweh, parle, qui ne prophétiserait\FTNT{Lorsque Yahweh parle, des prophètes sont suscités : Jé. 20:9 ; Mi. 3:8 ; Ac. 4:20.} ?
\VS{9}Faites entendre votre voix dans les palais d'Asdod, et dans les palais du pays d'Egypte, et dites : Assemblez-vous sur les montagnes de Samarie, et voyez l’important tumulte interne et quelles oppressions dans son sein !
\VS{10}Ils ne savent pas faire ce qui est droit, dit Yahweh, ils amassent la violence et la rapine dans leurs palais.
\VS{11}C'est pourquoi ainsi parle le Seigneur, Yahweh : L'ennemi viendra, il cernera le pays, il t'ôtera ta force et tes palais seront pillés.
\VS{12}Ainsi parle Yahweh : Comme un berger arrache de la gueule d'un lion deux jambes ou un bout d'oreille, ainsi les enfants d'Israël qui habitent dans Samarie seront arrachés de l’angle d’un lit et de l’asile de Damas.
\VS{13}Ecoutez et soyez mes témoins contre la maison de Jacob, dit le Seigneur, Yahweh, le Dieu des armées :
\VS{14}Le jour où je punirai Israël pour ses péchés, j’exercerai mon châtiment sur les autels de Béthel ; les cornes de l'autel seront brisées, et tomberont à terre.
\VS{15}J’abattrai la maison d'hiver et la maison d'été ; les maisons d'ivoire seront détruites, et un grand nombre de maisons disparaîtront, dit Yahweh.
\Chap{4}
\TextTitle{Yahweh condamne les sacrifices du peuple}
\VerseOne{}Ecoutez cette parole, vaches de Basan, qui êtes sur la montagne de Samarie, vous qui opprimez les faibles, qui maltraitez les pauvres, qui dites à leurs maîtres : Apportez, et que nous buvions !
\VS{2}Le Seigneur, Yahweh, l’a juré par sa sainteté : Voici, les jours viennent sur vous, où l’on vous enlèvera avec des hameçons, et votre postérité avec des crochets de pêche\FTNT{Jé. 16:16 ; Ha. 1.14-16}.
\VS{3}Vous sortirez dehors par les brèches, chacune devant soi, et vous serez jetées dans la forteresse, dit Yahweh.
\VS{4}Entrez dans Béthel, et péchez ! Multipliez vos péchés dans Guilgal ! Amenez vos sacrifices dès le matin, et vos dîmes tous les trois ans\FTNT{Voir commentaires en Mal. 3:10 et No. 18:21.} !
\VS{5}Brûlez de l’encens avec du pain levé pour l’offrande de remerciement ; proclamez et publiez les offrandes volontaires ; car c’est là ce que vous aimez, enfants d'Israël, dit le Seigneur, Yahweh\FTNT{Lé. 2:1}.
\TextTitle{Endurcissement du peuple malgré les châtiments de Yahweh}
\VS{6}C'est pourquoi je vous ai envoyé la famine dans toutes vos villes, et la disette de pain dans toutes vos demeures ; mais malgré cela, vous n’êtes pas revenus vers moi, dit Yahweh.
\VS{7}Je vous ai aussi privés de pluie, alors qu’il restait encore trois mois jusqu'à la moisson ; j'ai fait pleuvoir sur une ville et je n'ai pas fait pleuvoir sur une autre ville ; une parcelle a été arrosée par la pluie, et l'autre parcelle, sur laquelle il n'a pas plu, est desséchée\FTNT{1 R. 8:35 ; 1 R. 17:1 ; Es. 5:6 ; Ag. 1:11}.
\VS{8}Et deux, même trois villes sont allées vers une autre ville pour boire de l'eau et n'ont pas été désaltérées, mais vous n’êtes pas revenus vers moi, dit Yahweh.
\VS{9}Je vous ai frappés par la rouille et par la nielle, et la sauterelle a brouté autant de jardins et de vignes, de figuiers et d'oliviers que vous aviez, mais vous n’êtes pas revenus vers moi, dit Yahweh\FTNT{De. 28:22-39 ; 1 R. 8:37 ; Ag. 2:17 ; 2 Ch. 6:28}.
\VS{10}J’ai envoyé parmi vous la peste comme celle en Egypte ; j'ai tué par l'épée vos jeunes hommes et vos chevaux en captivité ; j'ai fait remonter, jusque dans votre nez, la puanteur de vos camps ; mais vous n’êtes pas revenus vers moi, dit Yahweh\FTNT{Ez. 14:19}.
\VS{11}Je vous ai détruits comme Dieu détruisit Sodome et Gomorrhe, et vous avez été comme un tison arraché du feu, mais vous n’êtes pas revenus vers moi, dit Yahweh\FTNT{Ge. 19:24 ; Jé. 49:18 ; Za. 3:2},
\VS{12}C'est pourquoi je te traiterai de la même manière ô Israël ; et parce que je te traiterai ainsi, prépare-toi à la rencontre de ton Dieu, ô Israël !
\VS{13}Car voici celui qui a formé les montagnes et créé le vent, et qui déclare à l'homme quelle est sa pensée, qui fait l'aube et l'obscurité, et qui marche sur les hauteurs de la terre ; son nom est Yahweh, le Dieu des armées.
\Chap{5}
\TextTitle{Israël invité à revenir entièrement à Yahweh}
\VerseOne{}Ecoutez cette parole, cette complainte que je prononce sur vous, maison d'Israël !
\VS{2}Elle est tombée, elle ne se relèvera plus, la vierge d'Israël ; elle est couchée par terre, et personne ne la relève.
\VS{3}Car ainsi a parlé le Seigneur, Yahweh : La ville qui mettait en campagne mille hommes n'en aura de reste que cent ; et celle qui mettait en campagne cent hommes n'en aura de reste que dix dans la maison d’Israël.
\VS{4}Car ainsi a parlé Yahweh à la maison d'Israël : Cherchez-moi, et vous vivrez !
\VS{5}Ne cherchez pas Béthel, et n'allez pas à Guilgal, et ne passez point à Beer-Schéba. Car Guilgal sera transportée en captivité, et Béthel sera détruite\FTNT{Os. 4:15}.
\VS{6}Cherchez Yahweh, et vous vivrez, de peur qu'il ne saisisse comme un feu la maison de Joseph, et que ce feu ne la consume, sans qu'il y ait personne à Béthel pour l’éteindre.
\VS{7}Ils changent le jugement en absinthe, et ils foulent à terre la justice\FTNT{Es. 5:26-28 ; Ha. 1:1-3}.
\VS{8}Celui qui a créé les Pléiades et l'Orion, qui change les profonds ténèbres en aurore, et qui obscurcit le jour en nuit, qui appelle les eaux de la mer, et les répand sur la surface de la Terre, Yahweh est son nom\FTNT{Es. 58:8-10 ; Job 9:9 ; Job 38:31}.
\VS{9}Il fait éclater la ruine sur les puissants, et la ruine vient sur les forteresses.
\VS{10}Ils haïssent à la porte ceux qui les reprennent, et ils ont en abomination celui qui parle en intégrité.
\VS{11}C'est pourquoi, puisque vous opprimez le pauvre, et que vous prenez de lui du blé en présent, vous avez bâti des maisons en pierres de taille, mais vous n'y habiterez pas ; vous avez planté des vignes délicieuses, mais vous n'en boirez pas le vin.
\VS{12}Car j'ai connu vos crimes, ils sont en grand nombre, et vos péchés se sont multipliés : Vous opprimez le juste, vous recevez des présents, et vous violez à la porte le droit des pauvres.
\VS{13}C'est pourquoi, en ce temps-ci, le sage se tait, car les temps sont mauvais.
\VS{14}Recherchez le bien et non le mal, afin que vous viviez ; et qu’ainsi Yahweh, le Dieu des armées, soit avec vous, comme vous l'avez dit.
\VS{15}Haïssez le mal, et aimez le bien, faites régner la justice à la porte ; peut-être Yahweh, le Dieu des armées, aura pitié des restes de Joseph.
\TextTitle{Le jour de Yahweh}
\VS{16}C'est pourquoi ainsi parle Yahweh, le Dieu des armées, le Seigneur, parle ainsi : Dans toutes les places on se lamentera, dans toutes les rues on dira : Hélas ! Hélas ! On appellera au deuil le laboureur, et à la lamentation ceux qui en savent le métier.
\VS{17}Dans toutes les vignes on se lamentera, quand je passerai au milieu de toi, dit Yahweh.
\VS{18}Malheur à ceux qui désirent le jour de Yahweh\FTNT{L’expression «~le jour du Seigneur~» ou «~le jour de Yahweh~» est une période durant laquelle Jésus-Christ interviendra ouvertement dans les affaires des hommes. Elle est utilisé dix-neuf fois dans le Tanakh (Es. 2:12 ; Es. 13:6-9 ; Ez. 13:5 ; Ez. 30:3 ; Joë. 1:15 ; Joë. 2:1, 11, 31 ; Joë. 3:14 ; Am. 5:18, 20 ; Ab. 1:15 ; So. 1:7, 14 ; Za. 14:1 ; Mal. 4:5) et quatre fois dans le Testament de Jésus (Ac. 2:20 ; 2 Th. 2:2 ; 2 Pi. 3:10). On y fait également allusion dans d’autres passages (Ap. 6:17 ; Ap. 16:14).}. Qu’attendez-vous du jour de Yahweh ? Ce sont des ténèbres, et non pas une lumière.
\VS{19}Vous serez comme un homme qui fuit devant un lion et qui rencontre un ours, ou qui entre dans sa maison, appuie sa main sur le mur et un serpent le mord.
\TextTitle{Mépris du droit et de la justice}
\VS{20}Le jour de Yahweh n’est-il pas ténèbres et non lumière ? Obscurité et non clarté ?
\VS{21}Je hais, je méprise vos fêtes, je ne prends pas plaisir à vos assemblées solennelles.
\VS{22}Si vous me présentez des holocaustes, je n’agréerai pas vos offrandes, je ne regarderai pas les bêtes grasses de vos offrandes de paix.
\VS{23}Eloigne de moi le bruit de tes cantiques ; je n’écouterai pas la mélodie de tes luths.
\VS{24}Mais le jugement roule comme de l'eau, et la justice comme un torrent intarissable.
\VS{25}Est-ce à moi, maison d'Israël, que vous avez offert des sacrifices et des gâteaux dans le désert pendant quarante ans ? 
\VS{26}Au contraire, vous avez porté la tente de votre roi, et de vos idoles Kijun\FTNT{«~Kijun~» : Probablement une statue d'un dieu Assyro-Babylonien de la planète Saturne et utilisé pour symboliser l'apostasie d'Israél}, l'étoile de votre dieu que vous vous êtes fabriqué.
\VS{27}C'est pourquoi je vous transporterai au-delà de Damas, dit Yahweh, dont le nom est le Dieu des armées.
\Chap{6}
\TextTitle{Ceux qui prospèrent seront emmenés captifs}
\VerseOne{}Hélas vous qui êtes à votre aise en Sion, et qui vous confiez en la montagne de Samarie, lieux les plus renommés d’entre les principaux des nations, auprès desquels va la maison d'Israël.
\VS{2}Passez à Calné, et regardez ; allez de là à Hamath la grande, puis descendez à Gath chez les Philistins. Ces villes sont-elles plus prospères que vos deux royaumes, ou leur pays n'est-il pas plus étendu que votre pays ?
\VS{3}Vous qui éloignez le jour du malheur, et qui approchez le règne de la violence.
\VS{4}Vous qui vous couchez sur des lits d'ivoire, et qui sont étendus sur vos coussins ; qui mangez les agneaux du troupeau, et les veaux pris du lieu où on les engraisse ;
\VS{5}qui fredonnez au son du luth ; qui inventez des instruments de musique comme David.
\VS{6}Qui buvez le vin dans de grandes coupes, et qui parfumez des parfums les plus exquis, et qui n’êtes pas affligés pour la plaie de Joseph.
\VS{7}A cause de cela ils vont être emmenés à la tête des captifs, et les cris de joie de ces personnes voluptueuses prendront fin.
\VS{8}Le Seigneur, Yahweh, l’a juré par lui-même. Yahweh, Dieu des armées, dit : J'ai en détestation l'orgueil de Jacob, et j'ai en haine ses palais, c'est pourquoi je livrerai la ville, et tout ce qui est en elle.
\VS{9}Et si il reste dix hommes dans une maison, ils mourront.
\VS{10}Un proche parent prendra un mort et le brûlera pour emporter les os hors de la maison ; il dira à celui qui est au fond de la maison : Y a-t-il encore quelqu'un avec toi ? Et il répondra : Il n’y a plus personne. Puis il dira : Silence ! Ce n’est pas le moment de prononcer le nom de Yahweh.
\VS{11}Car voici, Yahweh ordonne : Et il frappera les grandes maisons par des débordements d'eau, et la petite maison en débris.
\VS{12}Les chevaux courent-ils sur les rochers, y laboure-t-on avec des bœufs, pour que vous ayez changé la droiture en poison, et le fruit de la justice en absinthe ?
\VS{13}Vous vous réjouissez de choses qui ne sont que néant, vous dites : N’est-ce pas par notre force que nous avons acquis de la puissance ?
\VS{14}Voici, je ferai lever contre vous, maison d’Israël, dit Yahweh, le Dieu des armées, une nation qui vous opprimera depuis l'entrée de Hamath jusqu'au torrent du désert.
\Chap{7}
\TextTitle{Avertissement\FTNTT{Am. 8:1 ; 9:10}}
\VerseOne{}Le Seigneur, Yahweh, me fit voir cette vision : Voici, il formait des sauterelles au temps où le regain commençait à croître ; et voici le regain poussait après les récoltes du roi.
\VS{2}Et quand elles eurent achevé de dévorer l'herbe de la terre, je dis : Seigneur Yahweh, pardonne, je te prie ! Comment Jacob subsistera-t-il ? Car il est faible.
\VS{3}Yahweh se repentit de cela. Cela n'arrivera pas, dit Yahweh.
\VS{4}Le Seigneur, Yahweh, me fit voir cette vision : Voici, le Seigneur, Yahweh, proclamait le jugement par le feu. Et le feu dévorait le grand abîme et dévorait les champs.
\VS{5}Et je dis : Seigneur Yahweh ! Arrête, je te prie ! Comment Jacob subsistera-t-il ? Car il est faible.
\VS{6}Yahweh se repentit de cela. Cela non plus n'arrivera pas, dit le Seigneur, Yahweh.
\VS{7}Il me fit voir cette vision : Voici, le Seigneur se tenait debout sur un mur fait au niveau, et il avait un niveau dans la main.
\VS{8}Et Yahweh me dit : Que vois-tu, Amos ? Et je répondis : Un niveau. Et le Seigneur me dit : Je mettrai le niveau au milieu de mon peuple d'Israël, je ne lui pardonnerai plus.
\VS{9}Et les hauts lieux d'Isaac seront ravagés, et les sanctuaires d'Israël seront détruits ; et je me lèverai contre la maison de Jéroboam avec l'épée.
\TextTitle{Amatsia accuse Amos devant Jéroboam}
\VS{10}Alors Amatsia, sacrificateur de Béthel, fit dire à Jéroboam roi d'Israël : Amos conspire contre toi au milieu de la maison d'Israël ; le pays ne saurait supporter toutes ses paroles.
\VS{11}Car voici ce que dit Amos : Jéroboam mourra par l'épée, et Israël sera emmené captif hors de son pays.
\VS{12}Et Amatsia dit à Amos : Voyant\FTNT{Voyant ou prophète.}, va-t’en, fuis dans le pays de Juda, et manges-y ton pain, et là tu prophétiseras.
\VS{13}Mais ne continue pas à prophétiser à Béthel\FTNT{Bethel, qui signifie «~maison de Dieu~», était devenue le sanctuaire du roi Jéroboam. De même aujourd’hui des églises du Seigneur sont devenues la propriété des hommes et les brebis de Dieu sont devenues la propriété des pasteurs.}, car c'est le sanctuaire du roi, et c'est une maison royale.
\TextTitle{Amos répond}
\VS{14}Amos répondit à Amatsia : Je n'étais ni prophète ni fils de prophète ; j'étais un berger, et je cueillais des figues sauvages.
\VS{15}Or Yahweh m’a pris derrière le troupeau, et Yahweh m’a dit : Va, prophétise à mon peuple d'Israël.
\VS{16}Ecoute maintenant la parole de Yahweh, tu dis : Ne prophétise pas contre Israël, et ne parle pas contre la maison d'Isaac.
\VS{17}C'est pourquoi ainsi parle Yahweh : Ta femme se prostituera dans la ville, tes fils et tes filles tomberont par l'épée, ton champ sera partagé au cordeau, et toi, tu mourras sur une terre souillée, et Israël sera emmené captif hors de son pays.
\Chap{8}
\TextTitle{Vision du panier de fruit, la fin pour le peuple d'Israël}
\VerseOne{}Le Seigneur, Yahweh, me fit voir cette vision : Voici, je vis un panier de fruits d'été.
\VS{2}Il dit : Que vois-tu, Amos ? Et je répondis : Un panier de fruits. Et Yahweh me dit : La fin est venue pour mon peuple d'Israël, je ne continuerai plus à lui pardonner.
\VS{3}En ce jour-là, les chants du palais seront des gémissements, dit le Seigneur, Yahweh ; en tout lieu, il y aura beaucoup de cadavres que l'on jettera en silence.
\VS{4}Ecoutez ceci vous qui dévorez les pauvres et qui faites périr les pauvres misérables du pays,
\VS{5}et qui dites : Quand la nouvelle lune sera-t-elle passée pour que nous vendions du blé ? Quand finira le sabbat pour que nous ouvrions les greniers ? Nous diminuerons l’épha, nous augmenterons le sicle, nous falsifierons les balances pour tromper,
\VS{6}nous achèterons les faibles pour de l’argent, et le pauvre pour une paire de souliers, et nous vendrons la criblure du froment.
\VS{7}Yahweh l’a juré par la gloire de Jacob : Jamais je n’oublierai toutes leurs actions !
\VS{8}La terre ne sera-t-elle point émue d'une telle chose, et tous ses habitants ne se lamenteront-ils point ? Le pays tout entier montera comme le fleuve. Il se soulèvera et s’affaissera comme le fleuve d'Egypte.
\VS{9}Il arrivera en ce jour-là, dit le Seigneur, Yahweh, que je ferai coucher le soleil à midi, et que j’obscurcirai la terre en plein jour.
\VS{10}Je changerai vos fêtes en deuil, et tous vos chants en lamentations ; je couvrirai de sacs tous les reins, et je rendrai chauves toutes les têtes ; je mettrai le pays dans le deuil comme pour un fils unique, et sa fin sera un jour d’amertume.
\VS{11}Voici, les jours viennent, dit le Seigneur, Yahweh, où j'enverrai la famine\FTNT{Nous sommes dans une époque où la Parole de Dieu, l’Evangile véritable a presque disparu au profit de l’évangile de prospérité. L’esprit de commerce a pris place au sein de beaucoup d’églises. C’est le temps de l’église de Laodicée, une église qui fait l’apologie de la richesse matérielle.} dans le pays ; non une famine de pain, ni une soif d'eau, mais d’entendre les paroles de Yahweh.
\VS{12}Ils erreront d’une mer jusqu'à l'autre, du nord à l'orient, ils iront çà et là pour chercher la parole de Yahweh, et ils ne la trouveront pas.
\VS{13}En ce jour-là, les belles vierges et les jeunes hommes mourront de soif.
\VS{14}Ceux qui jurent par le péché de Samarie disent : Vive ton Dieu, ô Dan ! Vive la voie de Beer-Schéba ! Mais ils tomberont et ne se relèveront plus.
\Chap{9}
\TextTitle{Prophétie annonçant la destruction\FTNTT{De. 28:63-68.}}
\VerseOne{}Je vis le Seigneur qui se tenait debout sur l'autel. Et il dit : Frappe le chapiteau et que les seuils s’ébranlent ; et brise-les sur leurs têtes à tous ! Je tuerai par l'épée ce qui restera d'eux. Il ne s’enfuira pas un fugitif, il ne s’échappera pas un fuyard.
\VS{2}S’ils pénètrent dans le séjour des morts, ma main les enlèvera de là ; s’ils montent aux cieux, je les en ferai descendre.
\VS{3}S’ils se cachent au sommet du Carmel, je les y rechercherai et je les enlèverai de là ; s’ils se dérobent à mes yeux dans le fond de la mer, là j’ordonnerai au serpent de les mordre.
\VS{4}Lorsqu'ils s'en iront en captivité devant leurs ennemis, là j’ordonnerai à l’épée de les tuer ; je fixerai mon regard sur eux pour leur faire du mal et non du bien.
\VS{5}Le Seigneur, Yahweh des armées, touche la terre, et elle tremble, et tous ses habitants sont dans le deuil ; elle monte tout entière comme le fleuve, et elle s’affaisse comme le fleuve d'Egypte.
\VS{6}Il a bâti sa demeure dans les cieux, et fondé sa voûte sur la terre ; il appelle les eaux de la mer, et les répand sur la surface de la Terre. Son nom est Yahweh.
\VS{7}N'êtes-vous pas pour moi comme les enfants des Ethiopiens, enfants d'Israël ? dit Yahweh. N'ai-je pas fait monter Israël du pays d'Egypte, les Philistins de Caphtor et les Syriens de Kir ?
\VS{8}Voici, les yeux du Seigneur, Yahweh, sont sur ce royaume pécheur. Je le détruirai de dessus la surface de la terre. Cependant, je ne détruirai pas entièrement la maison de Jacob, dit Yahweh.
\VS{9}Car voici, je donnerai mes ordres, et je secouerai la maison d'Israël parmi toutes les nations, comme on secoue le blé dans le crible, sans qu'il en tombe un grain à terre.
\VS{10}Tous les pécheurs de mon peuple mourront par l'épée, ceux qui disent : Le mal n'approchera pas, il ne nous atteindra pas.
\TextTitle{Yahweh relève la maison de David}
\VS{11}En ce temps-là, je relèverai le tabernacle de David qui est tombé, j’en réparerai les brèches, j’en redresserai les ruines, et je le rebâtirai comme il était autrefois,
\VS{12}afin qu'ils possèdent le reste d’Edom et toutes les nations sur lesquelles mon nom a été invoqué, dit Yahweh, qui accomplira cela.
\TextTitle{Restauration d'Isarël}
\VS{13}Voici, les jours viennent, dit Yahweh, où le laboureur suivra de près le moissonneur, et celui qui foule les raisins atteindra celui qui répand la semence ; et le moût ruissellera des montagnes et découlera de toutes les collines.
\VS{14}Je ramènerai les captifs de mon peuple d'Israël ; ils rebâtiront les villes dévastées, et y habiteront, ils planteront des vignes, et en boiront le vin ; ils feront des jardins et en mangeront les fruits.
\VS{15}Je les planterai sur leur terre, et ils ne seront plus arrachés du pays que je leur ai donné\FTNT{Cette prophétie annonce la restauration de la maison de David. Personne ne chassera Israël de sa terre, aucune nation n’a le pouvoir de le déloger, car c’est le Seigneur qui l’a établi.}, dit Yahweh, ton Dieu.
\PPE{}
\end{multicols}

%\clearpage\ShortTitle{Abdias}\BookTitle{Abdias}\BFont
\noindent\hrulefill
{\footnotesize
\textit{
\bigskip
{\centering{}
\\Auteur : Abdias
\\(Heb. : Obadyah)
\\Signification : Adorateur ou serviteur de Yahweh
\\Thème : Condamnation d'Edom
\\Date de rédaction : 6ème siècle av. J.-C.\\}
}
%\bigskip
\textit{
\\Prophète ayant exercé son ministère en Juda, Abdias reçut un message court, mais clair sur le jour de Yahweh, et plus particulièrement sur le jugement d’Edom à la suite de ses violences envers Israël.\bigskip
}
}
\par\nobreak\noindent\hrulefill
\begin{multicols}{2}
\TextTitle{[I. Malédiction d'Edom : Introduction]}
\Chap{1}
\VerseOne{}Vision d'Abdias. Ainsi parle le Seigneur, Yahweh, sur Edom : Nous l’avons entendu de la part de Yahweh, et un messager a été envoyé parmi les nations : Courage, levons-nous contre lui pour le combattre\FTNT{Jé. 49:14} !
\VS{2}Voici, je te rendrai petit parmi les nations, tu seras fort méprisé.
\VS{3}L'orgueil de ton coeur t'a égaré, toi qui habites dans le creux des rochers, qui sont ta haute demeure, et qui dis en toi-même : Qui me précipitera jusqu’à terre ?
\VS{4}Quand tu élèverais ton nid comme l'aigle, et quand bien même tu le mettrais entre les étoiles, je te précipiterai de là, dit Yahweh.
\VS{5}Si des voleurs entraient chez toi, ou des pillards de nuit, comme te voilà ruiné ! Mais ils ne prendraient que ce qui leur suffit. Si des vendangeurs entraient chez toi, ne laisseraient-ils pas des grappillages ?
\VS{6}Comme Esaü a été fouillé ! Comme ses trésors cachés ont été découverts !
\VS{7}Tous tes alliés t'ont chassé jusqu'à la frontière, ceux qui étaient en paix avec toi t'ont trompé et ont eu le dessus sur toi, ceux qui mangeaient ton pain t'ont tendu des pièges, et tu ne t’en es pas aperçu.
\VS{8}N’est-ce pas en ce jour-là, dit Yahweh, que je ferai périr les sages d'Edom, et l’intelligence de la montagne d'Esaü ?
\VS{9}Tes guerriers seront effrayés, ô Théman ! Afin qu’ils soient tous retranchés de la montagne d'Esaü par le carnage\FTNT{Ez. 25:13 ; Mi. 7:8 ; So. 2:8}.
\TextTitle{II. Causes de la malédiction}
\VS{10}A cause de la violence que tu as faite à ton frère Jacob, la honte te couvrira, et tu seras retranché à jamais.
\VS{11}Le jour où tu te tenais en face de lui, le jour où des étrangers emmenaient captive son armée, où des inconnus entraient dans ses portes et jetaient le sort sur Jérusalem, toi aussi, tu étais comme l'un d'eux.
\VS{12}Ne considère pas avec joie le jour de ton frère, le jour de son malheur, ne te réjouis pas sur les enfants de Juda au jour de leur ruine, et n’ouvre pas une grande bouche au jour de la détresse.
\VS{13}N’entre pas dans les portes de mon peuple au jour de sa ruine, ne considère pas avec joie son malheur, au jour de sa ruine, et que tes mains ne se portent pas sur ses richesses, au jour de sa ruine !
\VS{14}Ne te tiens pas aux carrefours pour exterminer ses fugitifs, et ne livre pas ses fuyards au jour de la détresse.
\TextTitle{III. Edom au jour de Yahweh}
\VS{15}Car le jour de Yahweh est proche pour toutes les nations ; on te fera comme tu as fait, tes actes retomberont sur ta tête\FTNT{Jé. 50:15-29 ; Ez. 35:15}.
\VS{16}Car comme vous avez bu sur ma montagne sainte, ainsi toutes les nations boiront continuellement ; elles boiront, elles avaleront, et elles seront comme si elles n'avaient jamais été\FTNT{Jé. 25:15-28}.
\TextTitle{Délivrance future de Jacob et jugement sur Edom}
\VS{17}Mais le salut\FTNT{Le salut sera sur la montagne de Sion. Cette prophétie fait allusion au Royaume messianique. Voir Ro. 11:26.} sera sur la montagne de Sion, elle sera sainte, et la maison de Jacob possédera ses possessions.
\VS{18}La maison de Jacob sera un feu, et la maison de Joseph une flamme, et la maison d'Esaü du chaume ; ils l'allumeront et la consumeront ; et il ne restera rien de la maison d'Esaü, car Yahweh a parlé.
\VS{19}Ceux du midi possèderont la montagne d'Esaü, ceux de la plaine le pays des Philistins ; ils posséderont le territoire d'Ephraïm et celui de Samarie ; et Benjamin possédera Galaad.
\VS{20}Les captifs de cette armée des enfants d'Israël posséderont le pays des Cananéens jusqu'à Sarepta, et ceux qui auront été transportés de Jérusalem, qui sont à Sepharad, posséderont les villes du midi.
\VS{21}Des libérateurs monteront sur la montagne de Sion, pour juger la montagne d'Esaü ; et la royauté sera à Yahweh.
\PPE{}
\end{multicols}

%\clearpage\ShortTitle{Jonas}\BookTitle{Jonas}\BFont
\begin{multicols}{2}
\TextTitle{[I. Désobéissance et fuite de Jonas
\\Introduction]}
\Chap{1}
\VerseOne{}La parole de Yahweh fut adressée à Jonas, fils d'Amitthaï, en ces mots :
\TextTitle{[Jonas fuit la face de Yahweh]}
\VS{2}Lève-toi, va à Ninive (1), la grande ville, et crie contre elle ! Car leur malice est montée jusqu'à moi.
\VS{3}Mais Jonas se leva pour s'enfuir à Tarsis, loin de la face de Yahweh. Il descendit à Japho, où il trouva un navire qui allait à Tarsis ; il paya le prix du transport et y entra pour aller à Tarsis, loin de la face de Yahweh.
\VS{4}Mais Yahweh fit lever un grand vent sur la mer, et il y eut une grande tempête sur la mer, de sorte que le navire semblait se briser.
\VS{5}Et les mariniers eurent peur, et ils crièrent chacun à son dieu, et jetèrent dans la mer les objets qui étaient dans le navire, pour l’alléger. Jonas descendit au fond du navire, se coucha et s’endormit profondément.
\VS{6}Le chef des marins s'approcha de lui, et lui dit : Qu’as-tu dormeur ? Lève-toi, invoque ton Dieu ! Peut-être ton Dieu pensera à nous et nous ne périrons pas.
\VS{7}Puis ils se dirent l'un à l'autre : Venez, tirons au sort pour savoir qui est la cause de ce malheur. Ils tirèrent au sort, et le sort tomba sur Jonas.
\VS{8}Alors ils lui dirent : Dis-nous quelle est la cause de ce malheur. Quel est ton métier, et d'où viens-tu ? Quel est ton pays, et de quel peuple es-tu ?
\VS{9}Il leur répondit : Je suis Hébreu, et je crains Yahweh, le Dieu des cieux, qui a fait la mer et la terre sèche.
\VS{10}Alors ces hommes furent saisis d'une grande crainte, et lui dirent : Pourquoi as-tu fait cela ? Car ces hommes savaient qu’il fuyait loin de la face de Yahweh, parce qu'il le leur avait déclaré.
\VS{11}Ils lui dirent : Que te ferons-nous pour que la mer se calme ? Car la mer était de plus en plus agitée.
\TextTitle{[II. Jonas et le poisson
\\Jonas englouti par le poisson]}
\VS{12}Il leur répondit : Prenez-moi, et jetez-moi dans la mer, et la mer se calmera ; car je sais que c’est moi la cause de cette grande tempête.
\VS{13}Ces hommes ramaient pour revenir sur la terre sèche, mais ils ne le purent, car la mer s'agitait toujours plus contre eux.
\VS{14}Alors ils invoquèrent Yahweh, et dirent : O Yahweh, ne nous fais pas périr à cause de la vie de cet homme, et ne mets pas sur nous le sang innocent ! Car toi, Yahweh, tu fais comme il te plait (2).
\VS{15}Alors ils prirent Jonas, et le jetèrent dans la mer. Et la fureur de la mer s'arrêta.
\VS{16}Ces hommes furent saisis d’une grande crainte envers Yahweh, et ils offrirent des sacrifices à Yahweh, et firent des vœux.
\VS{17}Yahweh ordonna à un grand poisson d’engloutir Jonas, et Jonas fut dans le ventre du poisson trois jours et trois nuits.
\Chap{2}
\VerseOne{}Jonas pria Yahweh, son Dieu, dans le ventre du poisson.
\TextTitle{[Prière de Jonas et l'exaucement de Yahweh]}
\VS{2}Il dit : Dans ma détresse j’ai invoqué Yahweh, et il m'a exaucé ; du sein du scheol j’ai crié, et tu as entendu ma voix (1).
\VS{3}Tu m'as jeté dans les profondeurs, au cœur de la mer, et le courant m'a environné ; tous tes flots et toutes tes vagues ont passé sur moi.
\VS{4}Je disais : Je suis chassé loin de tes yeux ! Cependant je verrai encore le temple de ta sainteté.
\VS{5}Les eaux m'ont environné jusqu'à l'âme. L'abîme m'a enveloppé, les roseaux ont lié ma tête.
\VS{6}Je suis descendu jusqu'aux bases des montagnes, la terre fermait sur moi ses barres pour toujours ; mais tu m’as fait remonter vivant de la fosse, Yahweh, mon Dieu !
\VS{7}Quand mon âme s’était affaiblie en moi, je me suis souvenu de Yahweh, et ma prière est parvenue jusqu’à toi, dans le temple de ta sainteté.
\VS{8}Ceux qui s’adonnent à des vanités mensongères abandonnent ta miséricorde.
\VS{9}Mais moi, je t’offrirai des sacrifices avec un cri de louange, j’accomplirai les vœux que j’ai faits : Car le salut vient de Yahweh (2).
\VS{10}Alors Yahweh parla au poisson, et le poisson vomit Jonas sur la terre sèche.
\TextTitle{[III. Le plus grand réveil de l'histoire
\\Ninive se repent et elle est épargnée]}
\Chap{3}
\VerseOne{}La parole de Yahweh fut adressée à Jonas une seconde fois, en disant :
\VS{2}Lève-toi, va à Ninive, la grande ville, et proclames-y à haute voix ce que je t'ordonne !
\VS{3}Jonas se leva, et alla à Ninive, suivant la parole de Yahweh. Or Ninive était une grande ville devant Dieu, de trois jours de marche.
\VS{4}Jonas commença dans la ville le chemin d'une journée de marche ; il criait et disait : Encore quarante jours, et Ninive sera renversée !
\VS{5}Les hommes de Ninive crurent à Dieu, ils publièrent un jeûne, et se vêtirent de sacs, depuis le plus grand d'entre eux jusqu'au plus petit.
\VS{6}Cette parole parvint au roi de Ninive ; il se leva de son trône, ôta de dessus lui son manteau, se couvrit d'un sac, et s'assit sur la cendre.
\VS{7}Puis il fit faire une proclamation, et publier dans Ninive par décret du roi et de ses grands : Que les hommes, les bêtes, les bœufs et les brebis, ne goûtent de rien, ne paissent point, et ne boivent point d'eau !
\VS{8}Que les hommes et les bêtes soient couverts de sacs, qu'ils crient à Dieu avec force, et que chacun revienne de sa mauvaise voie, et des actions violentes que ses mains ont commises !
\VS{9}Qui sait si Dieu ne reviendra pas et ne se repentira pas, et s'il ne se détournera pas de son ardente colère, en sorte que nous ne périssions point ?
\VS{10}Dieu vit ce qu’ils faisaient et comment ils revenaient de leur mauvaise voie. Alors Dieu se repentit du mal qu'il avait déclaré de leur faire, et il ne le fit point.
\TextTitle{[IV. Miséricorde infinie de Dieu
\\Mécontentement de Jonas]}
\Chap{4}
\VerseOne{}Mais cela déplut fortement à Jonas, et il fut furieux.
\VS{2}Il pria Yahweh, et dit : Oh ! Yahweh, n'est-ce pas là ce que je disais quand j'étais encore dans mon pays ? C'est pourquoi j'ai voulu m'enfuir à Tarsis. Car je savais que tu es un Dieu compatissant, miséricordieux, lent à la colère et riche en bonté, et qui te repens du mal (1).
\VS{3}Maintenant, Yahweh, prends-moi donc la vie, car la mort m'est meilleure que la vie.
\TextTitle{[Reproches de Yahweh à Jonas]}
\VS{4}Et Yahweh répondit : Fais-tu bien de te mettre en colère ?
\VS{5}Alors Jonas sortit de la ville, et s'assit à l'orient de la ville, là il se fit une cabane, et y resta à l'ombre, jusqu'à ce qu'il vît ce qui arriverait à la ville.
\VS{6}Yahweh Dieu ordonna à un ricin de croître au-dessus de Jonas, pour donner de l’ombre sur sa tête et pour le délivrer de son mal. Jonas éprouva une grande joie à cause de ce ricin.
\VS{7}Mais le lendemain, à l’aurore, Dieu ordonna à un ver d’attaquer le ricin, et le ricin sécha.
\VS{8}Au lever du soleil, Dieu ordonna à un vent chaud d’orient de souffler, et le soleil frappa la tête de Jonas, au point qu’il s’évanouit. Il demanda la mort, et dit : La mort m'est meilleure que la vie.
\VS{9}Dieu dit à Jonas : Fais-tu bien de te mettre en colère à cause du ricin ? Et il répondit : Je fais bien de m’irriter jusqu’à la mort.
\VS{10}Et Yahweh dit : Tu as pitié du ricin pour lequel tu n'as point travaillé et que tu n'as point fait croître, qui est né dans une nuit et qui a péri dans une nuit.
\VS{11}Et moi, je n’aurais pas pitié de Ninive, la grande ville, dans laquelle il y a plus de cent vingt mille personnes qui ne savent point distinguer leur main droite de leur main gauche, et des animaux en grand nombre !
\PPE{}
\end{multicols}

%\clearpage\ShortTitle{Michée}\BookTitle{Michée}\BFont
\noindent\hrulefill
{\footnotesize
\textit{
\bigskip
{\centering{}
\\(Mikha)
\\Signifie : Qui est semblable à Dieu ?
\\Thème : Le jugement et le royaume
\\Auteur : Michée 
\\Date de rédaction : 8ème siècle av. J.-C.\\}
}
%\bigskip
\textit{
\\Originaire de Moréscheth, Michée exerça son ministère dans le royaume du sud au temps d’Ezéchias, roi de Juda et fut contemporain d’Osée, Amos et Esaïe. Alors que la corruption et l’idolâtrie régnaient en Samarie et à Jérusalem, Michée appela le peuple à se détourner de ses iniquités et les prévint du danger qui les menaçait. Il prophétisa également le rétablissement final de la nation juive et mit en exergue la miséricorde divine.\bigskip
}
}
\par\nobreak\noindent\hrulefill
\begin{multicols}{2}
\TextTitle{[Jugement de Yahweh sur Israël infidèle]}
\Chap{1}
\VerseOne{}La Parole de Yahweh qui fut adressée à Michée, de Moréscheth, au temps de Jotham, Achaz, et Ezéchias, rois de Juda, laquelle lui fut adressée dans une vision contre Samarie et Jérusalem.
\VS{2}Vous tous, peuples, écoutez ! Et toi, terre, et tout ce qui est en elle, soyez attentifs ! Et que le Seigneur, Yahweh, soit témoin contre vous, le Seigneur, sortant du palais de sa sainteté.
\VS{3}Car voici, Yahweh sortira de son lieu, il descendra, et marchera sur les hauts lieux de la terre\FTNT{Cette prophétie annonce à la fois la destruction du royaume du nord par Salmanasar V (régna de 727-722 av. J.-C.) en 722 av. J.-C. (2 R. 17:1-23), l’invasion de Sanchérib (régna de 705 à 681 av. J.-C.), (2 R. 18:13 à 2 R. 19:37) ainsi que celle de Nebucadnetsar (régna de 604 à 562 av. J.-C.), (2 R. 24 et 25).} ;
\VS{4}et les montagnes se fondront sous lui, et les vallées se fendront, elles seront comme de la cire devant le feu et comme des eaux qui coulent sur une pente.
\VS{5}Tout ceci arrivera à cause du crime de Jacob, et à cause des péchés de la maison d'Israël. Or quel est le crime de Jacob ? N'est-ce pas Samarie ? Et quels sont les hauts lieux de Juda ? N'est-ce pas Jérusalem ?
\TextTitle{[Chutes futures de Samarie et Jérusalem]}
\VS{6}C'est pourquoi je réduirai Samarie en un monceau de pierres dans les champs, un lieu où l'on plante des vignes ; et je ferai rouler ses pierres dans la vallée, et je découvrirai ses fondements.
\VS{7}Toutes ses images taillées seront brisées, tous ses salaires de prostitution seront brûlés au feu, et je mettrai tous ses faux dieux en désolation ; parce qu'elle les a entassés par le moyen du salaire de sa prostitution, ils serviront de salaire à une prostituée.
\VS{8}C'est pourquoi je me plaindrai, et je hurlerai ; je m'en irai dépouillé et nu ; je ferai une lamentation comme celle des dragons, et je mènerai le deuil comme celui des autruches.
\VS{9}Car sa plaie est incurable, elle est venue jusqu'en Juda, et est parvenue jusqu'à la porte de mon peuple, jusqu'à Jérusalem.
\VS{10}Ne l'annoncez point dans Gath, et ne pleurez nullement ! Vautre-toi dans la poussière à Beth-Leaphra.
\VS{11}Passez habitants de Schaphir, dans la nudité et la honte ! Les habitants de Tsaanan ne sont point sortis ; le deuil de Beth-Haëtsel vous prive de son abri.
\VS{12}L’habitante de Maroth est dans l'angoisse à cause de son bien ; parce que le mal est descendu de par Yahweh sur la porte de Jérusalem.
\VS{13}Attelle le cheval au char, habitante de Lakisch ! Toi qui es le commencement du péché de la fille de Sion ; car en toi ont été trouvés les crimes d'Israël.
\VS{14}C'est pourquoi donne des présents à cause de Moréschet-Gath ; les maisons d'Aczib mentiront aux rois d'Israël.
\VS{15}Je t’amènerai un autre héritier, habitante de Maréscha ; et la gloire d'Israël s'en ira jusqu'à Adullam.
\VS{16}Arrache tes cheveux et fais-toi tondre, à cause de tes fils qui font tes délices ; arrache tout le poil de ton corps, comme un aigle qui mue, car ils sont emmenés prisonniers loin de toi.
\TextTitle{[Causes du jugement de Dieu sur Israël]}
\Chap{2}
\VerseOne{}Malheur à ceux qui pensent à faire outrage, qui forgent le mal sur leurs lits, et qui l'exécutent dès le point du jour, parce qu'ils ont le pouvoir en main.
\VS{2}S'ils convoitent des possessions, ils les ont aussitôt ravies, des maisons, ils les ont aussitôt prises ; ainsi ils oppriment l'homme et sa maison, l'homme et son héritage.
\VS{3}C'est pourquoi ainsi parle Yahweh : Voici, je médite contre cette famille-ci un mal duquel vous ne pourrez point préserver votre cou, et vous ne marcherez point la tête levée, car ce temps est mauvais.
\VS{4}En ce temps-là on fera de vous un proverbe lugubre, et l'on gémira d'un gémissement lamentable, en disant : Nous sommes entièrement détruits ; la part de mon peuple, il la change de mains ! Comment nous enlève-t-il et partage-t-il notre terre à l'infidèle ?
\VS{5}C'est pourquoi il n'y aura personne qui jettera le cordeau pour ton lot, dans l'assemblée de Yahweh.
\VS{6}Ne prophétisez point, disent-ils, ne prophétisez point de telles choses ; l'opprobre ne s'éloignera point.
\VS{7}Or toi qui es appelée maison de Jacob, l'Esprit de Yahweh est-il amoindri ? Sont-ce là ses actes ? Mes paroles ne sont-elles pas bonnes pour celui qui marche droitement ?
\VS{8}Mais celui qui était hier mon peuple, s'élève à la manière d'un ennemi ; vous dépouillez le manteau avec le vêtement à ceux qui passent en assurance, de retour de la guerre.
\VS{9}Vous chassez les femmes de mon peuple hors des maisons de leurs délices ; vous ôtez pour toujours ma gloire de dessus leurs enfants.
\VS{10}Levez-vous et marchez, car ce pays n'est plus pour vous un lieu de repos ; à cause de la souillure, il vous détruira d’une violente destruction.
\VS{11}S'il y a quelque homme qui court après le vent et le mensonge, et qui mente en disant : Je te prophétiserai sur du vin et sur les boissons fortes, ce sera le prophète de ce peuple.
\TextTitle{[Yahweh, le Dieu qui rassemble son peuple]}
\VS{12}Mais je t'assemblerai tout entier, ô Jacob ! Et je ramasserai entièrement le reste d'Israël, et le mettrai tout ensemble comme les brebis d'une bergerie, comme un troupeau au milieu de son pâturage ; il y aura un grand bruit pour la foule des hommes.
\VS{13}Celui qui fera la brèche\FTNT{«~Celui qui fera la brèche~» est Jean-Baptiste, qui fut suscité à une époque où la gloire de Dieu ne se manifestait plus depuis quatre cents ans. En effet, Dieu n’avait plus suscité de ministère prophétique ou de messagers pour parler à son peuple. Le ciel était comme fermé. Il est donc venu ouvrir une brèche, c’est-à-dire préparer le chemin du Seigneur selon Es. 40:3-5, Mal. 3:1, Mal. 4:5-6.} montera  devant eux, on brisera, et on passera outre, et ils sortiront par la porte ; et leur Roi marchera devant eux\FTNT{Le Roi qui marchera devant eux est bien Jésus-Christ. Le message de Jean-Baptiste était clair : «~Repentez-vous car le Royaume des cieux arrive~»,  or ce royaume est celui du Messie (Mt. 3:1-3).}, Yahweh sera à leur tête.
\TextTitle{[Corruption et méchanceté des chefs]}
\Chap{3}
\VerseOne{}C'est pourquoi je dis : Ecoutez, chefs de Jacob, et vous conducteurs de la maison d'Israël ! N'est-ce point à vous de connaître ce qui est juste ?
\VS{2}Ils haïssent le bien, et aiment le mal ; ils leur arrachent la peau et la chair de dessus les os.
\VS{3}Ils dévorent la chair de mon peuple, lui arrachent la peau, et lui brisent les os ; ils le mettent en pièces comme dans un pot, comme de la viande dans une chaudière.
\VS{4}Alors ils crieront à Yahweh, mais il ne les exaucera point, et il leur cachera sa face en ce temps-là, parce qu’ils se sont mal conduits dans leurs actions
\VS{5}Ainsi parle Yahweh contre les prophètes qui égarent mon peuple, qui annoncent la paix si leurs dents ont quelque chose à mordre, et qui publient la guerre si on ne leur met rien dans la bouche :
\VS{6}C'est pourquoi la nuit sera sur vous, afin que vous n'ayez plus de vision ; et elle s'obscurcira, afin que vous ne deviniez plus ; le soleil se couchera sur ces prophètes-là, et le jour leur sera ténébreux.
\VS{7}Les voyants seront honteux, et les devins seront confondus ; eux tous se couvriront la barbe, parce qu'il n'y aura aucune réponse de Dieu.
\VS{8}Mais moi, je suis rempli de force, de justice, et de courage, par l'Esprit de Yahweh, pour déclarer à Jacob son crime, et à Israël son péché.
\TextTitle{[Future destruction de Jérusalem]}
\VS{9}Ecoutez maintenant ceci, chefs de la maison de Jacob, et vous conducteurs de la maison d'Israël, qui avez la justice en abomination, et qui pervertissez tout ce qui est droit,
\VS{10}vous qui bâtissez Sion avec le sang, et Jérusalem avec l'injustice.
\VS{11}Ses chefs jugent pour des présents, ses sacrificateurs enseignent pour un salaire, et ses prophètes devinent pour de l'argent\FTNT{Ce passage est encore d’actualité de nos jours. En effet, nombreux sont les dirigeants chrétiens qui exigent un salaire, notamment par le moyen de la dîme pour la plupart, en échange de leurs prières, enseignements,  conseils, formations bibliques… Le même constat se fait avec les chantres qui font des concerts payants alors qu’ils ont reçu leurs grâces du Seigneur gratuitement. Voir commentaire en Mt. 10:8.}, puis ils s'appuient sur Yahweh, en disant : Yahweh n'est-il pas parmi nous ? Le mal ne nous atteindra pas.
\VS{12}C'est pourquoi, à cause de vous, Sion sera labourée comme un champ, et Jérusalem sera réduite en ruines, et la montagne du temple en hauts lieux de forêt.
\TextTitle{[Marcher au nom de Yahweh]}
\Chap{4}
\VerseOne{}Mais il arrivera dans les derniers jours\FTNT{Voir commentaire en Ge. 49:1-2.}, que  la montagne de la maison de Yahweh\FTNT{Dans les Ecritures, les montagnes symbolisent parfois une grande puissance terrestre, et les collines celles de moindre importance. Cette prophétie confirme l’établissement du Royaume messianique dont la capitale sera Jérusalem (2 S. 7:14-16). Esaïe avait reçu la même prophétie que l’on peut découvrir au chapitre 2 de son livre.} sera affermie au sommet des montagnes, et sera élevée par-dessus les collines ; les peuples y afflueront.
\VS{2}Et des nations nombreuses iront et diront : Venez, et montons à la montagne de Yahweh, à la maison du Dieu de Jacob ; il nous enseignera ses voies, et nous marcherons dans ses sentiers ; car la loi sortira de Sion, et la parole de Yahweh de Jérusalem.
\VS{3}Il exercera le jugement parmi des peuples nombreux, et il sera l'arbitre de nations puissantes et lointaines ; et de leurs épées elles forgeront des hoyaux ; et de leurs hallebardes, des serpes ; une nation ne lèvera plus l'épée contre une autre, et on n'apprendra plus la guerre.
\VS{4}Mais chacun demeurera sous sa vigne et sous son figuier, et il n'y aura personne qui les épouvante ; car la bouche de Yahweh des armées aura parlé.
\VS{5}Les peuples marchent chacun au nom de leur dieu ; mais nous, nous marchons au Nom de Yahweh, notre Dieu, à toujours et à perpétuité.
\TextTitle{[Future restauration d'Israël]}
\VS{6}En ce jour-là, dit Yahweh, j'assemblerai les boiteux, je recueillerai ceux que j'avais chassés et ceux que j'avais maltraités.
\VS{7}Je ferai de ceux qui boitent un reste, et de ceux qui étaient éloignés une nation robuste ; Yahweh régnera sur eux, à la montagne de Sion, dès lors et à toujours.
\VS{8}Et toi, tour du troupeau, citadelle de la fille de Sion, jusqu'à toi viendra, à toi arrivera la souveraineté première ; le royaume sera à la fille de Jérusalem.
\TextTitle{[Yahweh, le Dieu qui rachète son peuple]}
\VS{9}Pourquoi maintenant pousses-tu des cris ? N'y a-t-il point de roi au milieu de toi ? Ou ton conseiller est-il mort, que la douleur t'ait saisie comme celle qui enfante ?
\VS{10}Souffre et gémis, fille de Sion, comme celle qui enfante ; car tu sortiras bientôt de la ville, tu demeureras aux champs, et tu iras jusqu'à Babylone ; là tu seras délivrée ; là Yahweh te rachètera de la main de tes ennemis.
\TextTitle{[Nations rassemblés pour l'Harmaguedon]}
\VS{11}Maintenant plusieurs nations se sont rassemblées contre toi\FTNT{Il est ici question de la guerre d’Harmaguédon. Voir commentaire en Ap. 16:12-16.} et disent : Qu'elle soit profanée ! Et que notre œil voie en Sion ce qu’il y voudrait voir\FTNT{Jérusalem est une horloge de Dieu. Le Seigneur a fait en sorte que les nations aient les yeux tournés vers ce bout de terre, car c'est là que débutera la troisième guerre mondiale, l’Harmaguédon (Mt. 24:15-28 ; Ap. 16:12-16 ; Ap. 19:11-21). Là aura lieu le jugement des nations, dans la vallée de Josaphat (Joë. 3:2-12). Le Messie reviendra (Es. 59:20-21 ; Za. 14:1-8 ; Ac. 1:10-11) et gouvernera le monde depuis Jérusalem (Za. 14:9-21). L'actuel conflit israélo-palestinien nous confirme bien ces prophéties. Il ne se passe pas un jour sans que les informations nous rapportent des événements venant de cet endroit du monde. D'ailleurs le Seigneur lui-même nous invite à suivre de près ce qu'il s'y passe (Mt. 24:32-34).}.
\VS{12}Mais ils ne connaissent point les pensées de Yahweh, et ne comprennent pas ses desseins ; car il les a assemblées comme des gerbes dans l'aire.
\VS{13}Lève-toi, et foule, fille de Sion ! Car je te ferai une corne de fer, et te mettrai des ongles d'airain ; et tu écraseras des peuples nombreux, et tu consacreras par interdit leurs profits à Yahweh, et leurs richesses au Seigneur de toute la terre.
\VS{14}Maintenant, fille de troupes, rassemble tes troupes ; on a mis le siège contre nous, on frappera le juge d'Israël avec la verge sur la joue.
\TextTitle{[Naissance du roi: le Messie]
\\(cp. Mt. 2:1-6 ; 27:24-37)}
\Chap{5}
\VerseOne{}Mais toi, Bethléhem Ephrata, petite pour être entre les milliers de Juda, de toi sortira quelqu’un pour être dominateur en Israël, dont l'origine remonte aux temps anciens, aux jours de l'éternité\FTNT{Il est question ici de Jésus-Christ. Ce passage nous parle de sa préexistence éternelle. Les pharisiens, scribes et principaux sacrificateurs avaient la connaissance de cette prophétie concernant le Messie (Mt. 2:1-6).}.
\VS{2}C'est pourquoi il les livrera jusqu'au temps où enfantera celle qui doit enfanter ; et le reste de ses frères retournera avec les enfants d'Israël.
\VS{3}Et il se maintiendra et gouvernera par la force de Yahweh, avec la magnificence du Nom de Yahweh, son Dieu ; et ils auront une demeure assurée, car dès lors il sera élevé jusqu'aux extrémités de la terre.
\VS{4}C'est lui qui sera la paix. Lorsque l'Assyrien sera entré dans notre pays, et qu'il aura mis le pied dans nos palais, nous élèverons contre lui sept pasteurs et huit princes du peuple.
\VS{5}Ils ravageront le pays d'Assyrie avec l'épée, et le pays de Nimrod à ses portes. Il nous délivrera ainsi des Assyriens, quand ils seront entrés dans notre pays, et qu'ils auront mis le pied dans nos quartiers.
\VS{6}Et le reste de Jacob sera au milieu de peuples nombreux, comme une rosée qui vient de Yahweh, et comme une pluie qui tombe sur l'herbe, qui ne s’attend à aucun homme, et qui n'espère pas des enfants des hommes.
\VS{7}Aussi le reste de Jacob sera parmi les nations, et au milieu de peuples nombreux, comme un lion parmi les bêtes de la forêt, et comme un lionceau parmi les troupeaux de brebis ; qui, en passant, foule et déchire, sans que personne ne puisse les sauver.
\TextTitle{[Jugement de Dieu sur les ennemis d'Isrël]}
\VS{8}Ta main se lèvera sur tes adversaires, et tous tes ennemis seront exterminés.
\VS{9}Et il arrivera en ce temps-là, dit Yahweh, que j'exterminerai du milieu de toi tes chevaux, et ferai périr tes chars.
\VS{10}J'exterminerai les villes de ton pays et renverserai toutes tes forteresses.
\VS{11}J'exterminerai aussi de ta main les sorcelleries, et tu n'auras plus aucun devin.
\VS{12}Et j'exterminerai du milieu de toi tes idoles et tes statues, et tu ne te prosterneras plus devant l'ouvrage de tes mains.
\VS{13}J'arracherai aussi du milieu de toi les poteaux d'Asherah\FTNT{Ex. 34:13.}, et détruirai tes ennemis.
\VS{14}Et j'exercerai ma vengeance avec colère et avec fureur contre toutes les nations qui ne m'auront pas écouté.
\TextTitle{[Yahweh appelle son peuple à l'humilité]}
\Chap{6}
\VerseOne{}Ecoutez maintenant ce que dit Yahweh : Lève-toi, plaide devant les montagnes, et que les collines entendent ta voix !
\VS{2}Ecoutez, montagnes, le procès de Yahweh, vous solides fondements de la terre ! Car Yahweh a un procès avec son peuple, et il plaidera avec Israël.
\VS{3}Mon peuple, que t'ai-je fait, ou en quoi t'ai-je causé de la peine ? Réponds-moi !
\VS{4}Car je t'ai fait sortir du pays d'Égypte et t'ai délivré de la maison de servitude, et j'ai envoyé devant toi Moïse, Aaron et Marie.
\VS{5}Mon peuple, rappelle-toi quel conseil Balak, roi de Moab, avait pris contre toi, et de ce que Balaam, fils de Beor, lui répondit ; et de ce que j'ai fait depuis Sittim jusqu'à Guilgal, afin que tu connaisses les justices de Yahweh.
\TextTitle{[Pratiquer la justice]}
\VS{6}Avec quoi me présenterai-je devant Yahweh, et me prosternerai-je devant le Dieu Très-Haut ? Me présenterai-je avec des holocaustes, et avec des veaux d'un an ?
\VS{7}Yahweh prendra-t-il plaisir à des milliers de béliers ou à des myriades de torrents d'huile ? Donnerai-je pour mon crime mon premier-né\FTNT{Selon la loi (Ex. 13 : 2 ; Ex. 3 : 12), les premiers-nés de l’homme et des animaux appartenaient au Seigneur. Ceux des animaux étaient offerts en sacrifice alors que le sacrifice des enfants était formellement interdit sous peine de mort (Lé. 18:21 ; Lé. 20:2-5 ; De. 12:31 ; De. 18:10).}, le fruit de mes entrailles pour le péché de mon âme ?
\VS{8}Ô homme ! Il t'a fait connaître ce qui est bon, et ce que Yahweh exige de toi : Que tu fasses ce qui est juste, que tu aimes la miséricorde, et que tu marches en toute humilité avec ton Dieu.
\VS{9}La voix de Yahweh crie à la ville, et le sage reconnaît son Nom. Écoutez la verge, et celui qui la dirige !
\VS{10}Y a-t-il encore dans la maison du méchant des trésors iniques, et un épha court et détestable ?
\VS{11}Tiendrai-je pour pur celui qui a de fausses balances et de faux poids dans son sac ?
\VS{12}Ses riches sont pleins de violence, ses habitants usent de mensonge, et ils ont une langue trompeuse dans leur bouche.
\VS{13}C'est pourquoi je te rendrai languissante en te frappant, et te ravagerai à cause de tes péchés.
\VS{14}Tu mangeras, mais tu ne seras pas rassasiée, et la faim sera au-dedans de toi-même ; tu mettras de côté, mais tu ne sauveras point, et ce que tu auras sauvé je le livrerai à l'épée.
\VS{15}Tu sèmeras, mais tu ne moissonneras point ; tu presseras l'olive, mais tu ne feras pas d'onctions d'huile ; et tu presseras le moût, mais tu ne boiras pas le vin.
\VS{16}Car tu as gardé les ordonnances d'Omri, et toutes les œuvres de la maison d'Achab, et tu as marché dans leurs conseils. C'est pourquoi je te livrerai à la désolation, je ferai de tes habitants un objet de raillerie, et vous porterez l'opprobre de mon peuple.
\TextTitle{[Le mal appelé bien, et le bien appelé mal]}
\Chap{7}
\VerseOne{}Malheur à moi ! Car je suis comme quand on a cueilli les fruits d'été et les grappillages de la vendange : Il n'y a ni grappe pour manger ni les premiers fruits que mon âme désirait.
\VS{2}Le fidèle est exterminé du pays, et il n'y a plus de juste entre les hommes ; ils sont tous en embûche pour verser le sang, chacun chasse son frère avec des filets.
\VS{3}Leurs mains sont habiles à faire le mal : Le gouverneur exige, le juge demande un salaire, le grand déclare ce qu'il convoite, et ils s'unissent.
\VS{4}Le meilleur d'entre eux est comme une ronce, et le plus juste est pire qu'une haie d'épines. Le jour annoncé par tes sentinelles, ton châtiment arrive. C'est alors qu'ils seront dans la confusion.
\VS{5}Ne crois pas à ton ami intime, et ne te confie pas en tes conducteurs ; garde-toi d'ouvrir ta bouche devant la femme qui dort dans ton sein.
\VS{6}Car le fils déshonore le père, la fille s'élève contre sa mère, la belle-fille contre sa belle-mère, et chacun a pour ennemis les gens de sa maison.
\TextTitle{[Espérance en Yahweh, le Dieu de notre salut]}
\VS{7}Mais moi, je regarderai vers Yahweh, je m'attendrai au Dieu de mon salut ; mon Dieu m'exaucera.
\VS{8}Toi, mon ennemie, ne te réjouis pas sur moi ; si je suis tombée, je me relèverai ; si j'ai été gisante dans les ténèbres, Yahweh m'éclairera.
\VS{9}Je supporterai la colère de Yahweh, car j'ai péché contre lui, jusqu'à ce qu'il défende ma cause, et qu'il me fasse justice ; il me conduira à la lumière, je verrai sa justice.
\VS{10}Et mon ennemie le verra, et la honte la couvrira ; elle qui me disait : Où est Yahweh, ton Dieu ? Mes yeux la verront, et alors elle sera foulée aux pieds comme la boue des rues.
\VS{11}Le jour où il rebâtira tes murs, en ce jour-là tes limites seront reculées.
\VS{12}En ce jour-là on viendra jusqu'à toi d'Assyrie et des villes d'Égypte, et depuis les villes d'Égypte jusqu'au fleuve, et depuis une mer jusqu'à l'autre mer, et depuis une montagne jusqu'à l'autre montagne ;
\VS{13}après que le pays aura été en désolation à cause de ses habitants, et du fruit de leurs actions.
\VS{14}Pais ton peuple avec ta houlette, le troupeau de ton héritage, qui demeure seul dans les forêts au milieu de Carmel ! Et fais qu'ils paissent en Basan et en Galaad, comme aux temps anciens.
\VS{15}Je lui ferai voir des choses merveilleuses, comme au jour où tu sortis du pays d'Égypte.
\VS{16}Les nations le verront, et elles seront honteuses avec toute leur force ; elles mettront la main sur la bouche, et leurs oreilles seront sourdes.
\VS{17}Elles lécheront la poussière comme le serpent, comme les reptiles de la terre ; elles trembleront dans leurs forteresses et accourront toutes effrayées vers Yahweh, notre Dieu, et te craindront.
\VS{18}Quel dieu est semblable à toi, qui est un Dieu qui pardonne l'iniquité, et qui passe par-dessus les péchés du reste de son héritage ? Il ne garde pas à toujours sa colère, parce qu'il prend plaisir à la miséricorde.
\VS{19}Il aura encore compassion de nous ; il effacera nos iniquités, et jettera tous nos péchés au fond de la mer.
\VS{20}Tu feras voir ta fidélité à Jacob, et ta miséricorde à Abraham, comme tu l’as juré à nos pères dès les temps anciens\FTNT{Les versets 18 à 20 de Mi. 7 sont lus chaque année dans les synagogues le jour des expiations.}.
\PPE{}
\end{multicols}

%\clearpage\ShortTitle{Na.}\BookTitle{Nahum}\BFont
\noindent\hrulefill
{\footnotesize
\textit{
\bigskip
{\centering{}
\\Auteur~: Nahum
\\(Heb.~: Nachuwm)
\\Signification~: Consolation, qui a compassion
\\Thème~: La ruine de Ninive
\\Date de rédaction~: 7\up{ème} siècle av. J.-C.\\}
}
\textit{
\\Nahum d'Elkosch, contemporain d'Habakuk, exerça son service dans le royaume de Juda. Il fut chargé d'annoncer la chute de Ninive, qui s'était repentie quelques décennies plus tôt, suite à la prédication de Jonas, mais elle multiplia de nouveau ses actes d'injustice et de violences au point de terroriser tous les peuples des alentours.\bigskip
}
}
\par\nobreak\noindent\hrulefill
\begin{multicols}{2}
\Chap{1}
\VerseOne{}Prophétie sur Ninive\FTNT{Ninive était la capitale de l'ancien empire assyrien. Voir Jon. 1:1-2.}, qui est le livre de la vision de Nahum d'Elkosch.
\TextTitle{Jugement annoncé sur Ninive}
\VS{2}Yahweh est un Dieu jaloux, il se venge, Yahweh se venge, il est plein de fureur~; Yahweh se venge de ses adversaires, et il garde sa colère contre ses ennemis.
\VS{3}Yahweh est lent à la colère, et grand par sa force, mais il ne tient nullement le coupable pour innocent. Yahweh marche parmi les tourbillons et les tempêtes, les nuées sont la poussière de ses pieds.
\VS{4}Il réprimande la mer et la fait tarir, il dessèche tous les fleuves~; le Basan et le Carmel languissent, la fleur du Liban languit.
\VS{5}Les montagnes tremblent à cause de lui, et les collines se fondent~; la terre se soulève devant sa présence, dis-je, et tous ceux qui y habitent.
\VS{6}Qui subsistera devant son indignation~? Et qui demeurera ferme dans l'ardeur de sa colère~? Sa fureur se répand comme un feu, et les rochers se brisent devant lui.
\VS{7}Yahweh est bon, il est une forteresse au jour de la détresse, et il connaît ceux qui se confient en lui.
\VS{8}Il s'en va passer comme un débordement d'eaux~; il réduira son lieu à néant, et fera que les ténèbres poursuivront ses ennemis.
\TextTitle{Jugement contre les ennemis de Yahweh}
\VS{9}Que projetez-vous contre Yahweh~? C'est lui qui réduit à néant~; la détresse ne se lèvera pas deux fois~;
\VS{10}car entrelacés comme des épines, et ivres de leur vin, ils seront consumés entièrement comme la paille sèche.
\VS{11}De toi est sorti celui qui méditait du mal contre Yahweh, et qui avait de mauvais desseins.
\VS{12}Ainsi parle Yahweh~: Bien qu'ils soient en paix et en grand nombre, ils seront certainement retranchés, et on passera outre. Or je t'ai affligé, mais je ne t'affligerai plus.
\VS{13}Maintenant je briserai son joug de dessus toi, et je détacherai tes liens.
\VS{14}Voici ce qu'a ordonné Yahweh contre toi~: Tu n'auras plus de semence qui porte ton nom~; je retrancherai de la maison de tes dieux les images taillées et en fonte~; j'en ferai ta tombe parce que tu es insignifiant.
\Chap{2}
\TextTitle{Délivrance et réjouissance}
\VerseOne{}Voici sur les montagnes les pieds de celui qui apporte de bonnes nouvelles\FTNT{Es. 40:9~; 52:7~; Ro. 10:15.}, qui publie la paix~! Toi Juda, célèbre tes fêtes solennelles, accomplis tes vœux~; car à l'avenir les hommes violents ne passeront plus au milieu de toi, ils sont entièrement retranchés.
\TextTitle{Récit de la destruction de Ninive}
\VS{2}Le destructeur est monté contre toi~; garde la forteresse~! Veille sur la route~! Affermis tes reins~! Consolide toute ta force~!
\VS{3}Car Yahweh se détourne de la majesté de Jacob et d'Israël~; parce que les dévastateurs les ont vidés, et qu'ils ont détruit leurs sarments.
\VS{4}Le bouclier de ses hommes forts est rouge~; ses hommes puissants sont teints de pourpre~; le fer des chars étincelle au jour qu'il a fixé pour la bataille, et les lances sont agitées.
\VS{5}Les chars s'élancent avec rapidité dans les rues, ils se précipitent sur les places, ils sont comme des flambeaux, et courent comme des éclairs.
\VS{6}Il se souvient de ses hommes vaillants, mais ils chancellent dans leur marche. Ils se hâtent vers les murs, et ils se préparent à la défense.
\VS{7}Les portes des fleuves sont ouvertes, et le palais se fond.
\VS{8}C'est fixé~: Elle est découverte et emportée~; ses servantes gémissent de leur voix comme des colombes, frappant leurs poitrines comme un tambour.
\VS{9}Or Ninive, depuis qu'elle a été bâtie, a été comme un vivier d'eaux~; mais ils s'enfuient… Arrêtez-vous~! Arrêtez-vous~! Mais il n'y a personne qui tourne le visage…
\VS{10}Pillez l'argent~! Pillez l'or~! Il y a des dispositions sans fin, des richesses en objets précieux.
\VS{11}On pille, on dévaste, on ravage~! Et les cœurs se fondent, leurs genoux se heurtent l'un contre l'autre. Que le tourment soit dans les reins de tous~! Et que leurs visages deviennent noirs comme un pot qui a été mis sur le feu~!
\VS{12}Où est le repaire des lions, le pâturage des lionceaux, dans lequel se retiraient les lions, et où se tenaient le lion et la lionne, et le petit du lion, sans qu'aucun ne les effraie~?
\VS{13}Les lions ravissaient tout ce qu'il fallait pour leurs petits, et étranglaient les bêtes pour leurs lionnes, ils remplissaient leurs tanières de proies, et leurs repaires de dépouilles.
\VS{14}Voici, j'en veux à toi, dit Yahweh des armées, je brûlerai tes chars, et ils s'en iront en fumée, l'épée consumera tes lionceaux. Je retrancherai de la terre ta proie, et la voix de tes messagers ne sera plus entendue.
\Chap{3}
\TextTitle{Méchanceté de Ninive}
\VerseOne{}Malheur à la ville sanguinaire qui est pleine de mensonge, pleine de violence~; la rapine ne s'en retirera point,
\VS{2}ni le bruit du fouet, ni le bruit impétueux des roues, ni le galop des chevaux, ni le saut des chars~;
\VS{3}ni les cavaliers montant leurs chevaux, ni l'épée flamboyante, ni la lance étincelante, ni la multitude des blessés, ni le grand nombre de cadavres. Des corps morts à l'infini, on trébuche sur un grand nombre de corps morts~!
\VS{4}à cause de la multitude des prostitutions de cette prostituée, pleine de charmes, experte en sortilèges, qui vendait les nations par ses prostitutions, et les familles par ses enchantements.
\VS{5}Voici, j'en veux à toi, dit Yahweh des armées, je relèverai ta robe jusqu'à ton visage~; je manifesterai ta nudité aux nations, et ton ignominie aux royaumes.
\VS{6}Je ferai tomber sur ta tête la peine de tes abominations, je te consumerai et je te donnerai en spectacle.
\VS{7}Et il arrivera que quiconque te verra, s'éloignera de toi et dira~: Ninive est détruite~! Qui aura compassion d'elle~? D'où te chercherai-je des consolateurs~?
\VS{8}Vaux-tu mieux que No-Amon, qui est assise au milieu des fleuves, qui a des eaux autour d'elle, dont la mer est le rempart, et à qui la mer sert de murailles~?
\VS{9}L'Ethiopie et l'Egypte étaient sa force, et une infinité d'autres peuples~; Puth et les Lybiens sont allés à son secours.
\VS{10}Elle-même aussi est transportée hors de sa terre, elle s'en est allée en captivité~; ses enfants ont été écrasés aux carrefours de toutes les rues, et on a jeté le sort sur ses gens honorables, et tous ses grands ont été liés de chaînes.
\VS{11}Toi aussi, tu seras enivrée, tu te tiendras cachée, et tu chercheras un refuge contre l'ennemi.
\VS{12}Toutes tes forteresses seront comme des figues, et comme des premiers fruits qui étant secoués, tombent dans la bouche de celui qui veut les manger.
\VS{13}Voici, ton peuple sera comme autant de femmes au milieu de toi~; les portes de ton pays seront toutes ouvertes, elles seront ouvertes à tes ennemis~; le feu consumera tes verrous.
\VS{14}Puise-toi de l'eau pour le siège~! Fortifie tes remparts~! Enfonce le pied dans la boue, foule l'argile~! Et fortifie le four à brique~!
\VS{15}Là, le feu te consumera, l'épée te retranchera, elle te dévorera comme la sauterelle dévore les arbres. Multiplie-toi comme les sauterelles~! Multiplie-toi comme les sauterelles~!
\VS{16}Tu as multiplié le nombre de tes marchands plus que les étoiles des cieux~; les sauterelles s'étant répandues, ont tout ravagé, et puis se sont envolées.
\VS{17}Tes princes et leurs scribes sont comme des sauterelles qui campent dans les murs au temps de la froidure~: Le soleil paraît, elles s'envolent et on ne reconnaît plus le lieu où elles étaient.
\VS{18}Tes bergers se sont endormis, ô roi d'Assyrie~! Tes grands hommes se tiennent dans leurs tentes~; ton peuple est dispersé par les montagnes, et il n'y a personne qui le rassemble.
\VS{19}Il n'y a point de remède à ta blessure, ta plaie est douloureuse~; tous ceux qui entendront parler de toi battront des mains sur toi~; car qui n'a pas continuellement éprouvé les effets de ta méchanceté~?
\PPE{}
\end{multicols}

%\clearpage\ShortTitle{Habakuk}\BookTitle{Habakuk}\BFont
\noindent\hrulefill
{\footnotesize
\textit{
\bigskip
{\centering{}
\\Signifie : Embrasser, amour
\\Thème : Du doute à la foi
\\Auteur : Habakuk
\\Date de rédaction : 7ème siècle av. J.-C.\\}
}
%\bigskip
\textit{
\\Habakuk, contemporain de Nahum, Sophonie et Jérémie, exerça son ministère dans le royaume de Juda. Véritable sentinelle, il fut chargé d’annoncer le châtiment de Juda par les chaldéens. Ce récit, qui est en partie un dialogue entre Dieu et Habakuk, témoigne de la relation qui les liait. Il est aussi une invitation à la patience et la foi en Yahweh.\bigskip
}
}
\par\nobreak\noindent\hrulefill
\begin{multicols}{2}
\Chap{1}
\TextTitle{[Quand la méchanceté semble triompher de la justice]}
\VerseOne{}Oracle qu’Habakuk, le prophète a vu.
\VS{2}Ô Yahweh ! Jusqu’à quand crierai-je sans que tu m'écoutes ? Jusqu'à quand crierai-je vers toi ? On me traite avec violence sans que tu me délivres !
\VS{3}Pourquoi me fais-tu voir la méchanceté\FTNT{La perplexité d’Habakuk était la même que celle de Job (Job. 21:7), d’Asaph (Ps. 73) et de Jérémie (Jé. 12:1-2). Les méchants semblent prospérer tandis que les justes pleurent et sont persécutés (Mal. 3 : 12-15).}, et vois-tu la perversité ? Pourquoi y a-t-il de l’oppression et de la violence devant moi, et des gens qui excitent des procès et des querelles ?
\VS{4}Parce que la loi est sans force, et que la justice ne se fait jamais, à cause de cela le méchant environne le juste, et à cause de cela on rend des jugements corrompus\FTNT{Jé. 5:26 ; Am. 5:7.}.
\TextTitle{[La réponse de Yahweh]}
\VS{5}Regardez parmi les nations, et voyez, et soyez étonnés et stupéfaits ! Car je vais faire en vos jours une œuvre que vous ne croiriez pas si on vous la racontait\FTNT{Ac. 13:41.}.
\VS{6}Car voici, je vais susciter les Chaldéens, ce peuple cruel et impétueux, marchant sur l'étendue de la terre, pour posséder des demeures qui ne lui appartiennent pas.
\VS{7}Il est redoutable et terrible, son gouvernement et son autorité viennent de lui-même .
\VS{8}Ses chevaux sont plus légers que les léopards, et ils ont la vue plus aiguë que les loups du soir ; et ses cavaliers se répandront çà et là, même ses cavaliers viendront de loin ; ils voleront comme un aigle qui fond sur sa proie\FTNT{Jé. 5:6 ; So. 3:3.}.
\VS{9}Ils viendront tous pour la violence ; ce qu'ils engloutiront de leurs regards sera porté vers l'orient, et ils amasseront les prisonniers comme du sable.
\VS{10}Ce peuple se moque des rois, et les princes sont l’objet de ses railleries ; il se rit de toutes les forteresses ; il amoncelle de la terre, et il s’en empare.
\VS{11}Alors il traverse comme le vent, il passe outre et se rend coupable, car sa force est son dieu.
\TextTitle{[La souveraineté de Dieu]}
\VS{12}N'es-tu pas de toute éternité, ô Yahweh ! Mon Dieu, mon Saint ? Nous ne mourrons point ! Ô Yahweh, tu l'as établi pour exécuter tes jugements ; et toi, mon rocher\FTNT{Voir commentaire en  Es. 8:13-17.}, tu l'as fondé pour punir.
\VS{13}Tu as les yeux trop purs pour voir le mal, et tu ne saurais prendre plaisir à regarder le mal qu'on fait à autrui. Pourquoi regarderais-tu les perfides, et te tairais-tu quand le méchant dévore son prochain qui est plus juste que lui ?
\VS{14}Or tu as fait les hommes comme les poissons de la mer, et comme le reptile qui n'a point de maître.
\VS{15}Il a tout enlevé avec l'hameçon ; il l'a amassé avec son filet, et l'a assemblé dans son rets ; c'est pourquoi il se réjouira et s'égayera\FTNT{Am. 4:2.}.
\VS{16}A cause de cela, il sacrifie à son filet, et il offre de l’encens à ses rets, parce qu'il aura eu par leur moyen une grasse portion, et que sa viande est une chose moelleuse.
\VS{17}Videra-t-il à cause de cela son filet ? Et ne cessera-t-il jamais de faire le carnage des nations ?
\TextTitle{[S'attendre à Yahweh]}
\Chap{2}
\VerseOne{}Je me tenais en sentinelle, j'étais debout dans la forteresse et je faisais le guet, pour voir ce qu’il me dira, et ce que je répondrais après ma plainte\FTNT{Jé. 6:17 ; Es. 21:1-6 ; Ez. 33:1-19.}.
\TextTitle{[Le juste vivra par la foi]}
\VS{2}Et Yahweh m'a répondu et m'a dit : Ecris la vision, et grave-la sur des tablettes, afin qu'on la lise couramment.
\VS{3}Car la vision est encore différée jusqu'à un certain temps, et Yahweh parlera de ce qui arrivera à la fin, et il ne mentira point. S'il tarde, attends-le, car il ne manquera point de venir, et il ne tardera point\FTNT{Hé. 10:37.}.
\VS{4}Voici, l'âme de celui qui s'élève n'est point droite en lui ; mais le juste vivra de sa foi\FTNT{Ro. 1:17 ; Hé. 10:38.}.
\VS{5}Et combien plus l'homme adonné au vin est-il perfide, et l'homme puissant est-il orgueilleux, ne se tenant point tranquille chez lui ; il élargit son âme comme le scheol, et il est insatiable comme la mort, il rassemble vers lui toutes les nations, et réunit à lui tous les peuples.
\VS{6}Tous ceux-là ne feront-ils pas de lui un sujet de raillerie et d’énigmes ? Et ne dira-t-on pas : Malheur à celui qui accumule ce qui ne lui appartient point ; jusqu'à quand le fera-t-il, et entassera-t-il sur lui de la boue épaisse ?
\VS{7}Ne se lèveront-ils pas soudain, ceux qui le mordront ? Ne se réveilleront-ils pas pour te tourmenter ? Et tu deviendras leur proie.
\VS{8}Parce que tu as pillé beaucoup de nations, tout le reste des peuples te pillera, et à cause aussi des meurtres des hommes, et de la violence faite dans le pays, contre la ville, et contre tous ses habitants\FTNT{Es. 33:1 ; Na. 3:1.}.
\VS{9}Malheur à celui qui amasse pour sa maison des gains injustes, afin de placer son nid dans un lieu élevé, pour échapper à l’atteinte de la calamité !
\VS{10}C’est pour la confusion de ta maison que tu as pris conseil, en détruisant beaucoup de peuples, et c’est contre ton âme que tu as péché.
\VS{11}Car la pierre crie du milieu de la muraille, et de la charpente la poutre lui répond.
\VS{12}Malheur à celui qui bâtit des villes avec le sang et qui fonde des cités sur l'iniquité.
\VS{13}Voici, n'est-ce pas la volonté de Yahweh des armées que les peuples travaillent pour le feu, et que les peuples se lassent pour le néant ?
\VS{14}Car la terre sera remplie de la connaissance de la gloire de Yahweh\FTNT{Es. 11:9.}, comme le fond de la mer par les eaux qui le couvrent.
\VS{15}Malheur à celui qui fait boire son compagnon en lui approchant sa bouteille, et qui l’enivre afin qu'on voie sa nudité\FTNT{Es. 5:22 ; Ge. 9:21-24.}.
\VS{16}Tu seras rassasié de honte plutôt que de gloire ; toi aussi bois, et découvre-toi. La coupe de la droite de Yahweh fera le tour jusqu’à toi, et l'ignominie sera répandue sur ta gloire.
\VS{17}Car la violence faite au Liban retombera sur toi ; et les ravages des bêtes t’effrayeront, parce que tu as répandu le sang des hommes, et commis  des violences dans le pays, contre la ville et tous ses habitants.
\VS{18}A quoi sert l'image taillée  pour qu’un ouvrier la taille ? A quoi sert l’image de fonte, docteur de mensonge, a quoi sert-elle pour que l'ouvrier qui l’a faite place en elle sa confiance en fabriquant des idoles muettes ?
\VS{19}Malheur à ceux qui disent au bois : Réveille-toi ! Et à la pierre muette : Réveille-toi ! Enseignera-t-elle ? Voici, elle est couverte d'or et d'argent, et il n'y a aucun esprit au-dedans d’elle.
\VS{20}Mais Yahweh est dans le temple de sa sainteté. Que toute la terre fasse silence devant lui !
\TextTitle{[Habakuk reconnait et accepte la volonté de Dieu]}
\Chap{3}
\VerseOne{}Prière d'Habakuk, le prophète, sur le mode des chants lyriques.
\VS{2}Yahweh, j'ai entendu ce que tu m'as fait entendre, et j'ai été saisi de crainte, ô Yahweh ! Dans le cours des années, ravive ton œuvre ; dans le cours des années, fais-la connaître; dans ta colère souviens-toi de tes compassions.
\VS{3}Dieu vient de Théman, et le Saint vient du mont de Paran. Sélah. Sa majesté couvre les cieux, et la terre est remplie de sa louange.
\VS{4}Sa splendeur est comme la lumière même, et des rayons sortent de sa main ; c'est là où réside sa force.
\VS{5}La peste marche devant lui, et une  flamme ardente sort sous ses pieds.
\VS{6}Il s'arrête et mesure la terre ; il regarde et met en déroute les nations ; les montagnes antiques sont  brisées en éclats,  et les collines éternelles s’affaissent. Ses voies sont les voies anciennes.
\VS{7}Je vois les tentes de Cuschan accablées sous la punition ; les pavillons du pays de Madian sont ébranlés.
\VS{8}Est-ce contre les fleuves que s’irrite Yahweh ? Ta colère est-elle contre les fleuves, et ta fureur contre la mer, que tu sois monté sur tes chevaux et sur tes chars de délivrance ?
\VS{9}Ton arc est mis à nu et tire toutes les flèches, selon le serment fait aux tribus, à savoir ta parole. Sélah. Tu fends la terre et tu en fais sortir des fleuves\FTNT{Ps. 78:15-16 ; Ps. 105:41.}.
\VS{10}Les montagnes te voient et elles tremblent\FTNT{Ps. 114:4-7.}; des torrents d’eau se précipitent, l'abîme fait retentir sa voix de la profondeur, il élève ses mains en haut.
\VS{11}Le soleil et la lune s'arrêtent dans leur habitation\FTNT{Jos. 10:12 ; Ap. 22:5.}, ils marchent à la lueur de tes flèches, et à la splendeur de l'éclat de ta lance étincelante.
\VS{12}Tu marches sur la terre avec indignation, et foules les nations avec colère.
\VS{13}Tu sors pour la délivrance de ton peuple, tu sors avec ton Oint pour la délivrance ; tu transperces le chef, afin qu'il n'y en ait plus dans la maison du méchant, tu en découvres le fondement  jusqu’au fond. Sélah.
\VS{14}Tu perces avec ses flèches  la tête de ses chefs, quand ils viennent comme une tempête pour me dissiper ; ils s'égaient comme pour dévorer l'affligé dans sa retraite.
\VS{15}Tu marches avec tes chevaux par la mer, les grandes eaux ayant été amoncelées.
\VS{16}J'ai entendu ce que tu m'as déclaré, et mes entrailles en sont émues ; à ta voix le tremblement saisit mes lèvres ; la pourriture entre dans mes os, et je tremble en moi-même, car je serai en repos au jour de la détresse, lorsque montant vers le peuple, il le mettra en pièces.
\VS{17}Car le figuier ne fleurira pas, et il n'y aura point de fruit dans les vignes ; ce que l'olivier produit mentira, et aucun champ ne produira rien à manger ; les brebis seront retranchées du parc, et il n'y aura point de bœufs dans les étables.
\VS{18}Mais moi, je me réjouis en Yahweh, et je me réjouis dans le Dieu de ma délivrance.
\VS{19}Yahweh, le Seigneur, est ma force, et il rend mes pieds semblables à ceux des biches, et me fait marcher sur mes lieux élevés\FTNT{Ps. 18:33-34 ; De. 32:13.}. Au chef des chantres avec instruments à cordes.
\PPE{}
\end{multicols}

%\clearpage\ShortTitle{So.}\BookTitle{Sophonie}\BFont
\noindent\hrulefill
{\footnotesize
\textit{
\bigskip
{\centering{}
\\Auteur~: Sophonie
\\(Heb.~: Tsephanyah)
\\Signification~: Yahweh a caché, protégé
\\Thème~: Le jour de Yahweh
\\Date de rédaction~: 7\up{ème} siècle av. J.-C.\\}
}
\textit{
\\De lignée royale, Sophonie exerça son service dans le royaume de Juda au temps du roi Josias et fut contemporain de Jérémie, Habakuk, Ezéchiel et Abdias. A une époque où l'iniquité s'était accrue au point où les quelques personnes fidèles à Dieu étaient persécutées, Sophonie fut suscité par Yahweh pour annoncer le jugement de Juda, d'Israël et de quelques nations païennes.\bigskip
}
}
\par\nobreak\noindent\hrulefill
\begin{multicols}{2}
\Chap{1}
\TextTitle{Yahweh annonce son jugement sur Juda, conséquence de son idolâtrie}
\VerseOne{}C'est ici la parole de Yahweh qui fut adressée à Sophonie, fils de Cuschi, fils de Guedalia, fils d'Amaria, fils d'Ezéchias, du temps de Josias, fils d'Amon, roi de Juda.
\VS{2}Je ferai entièrement périr toutes choses de dessus cette terre, dit Yahweh.
\VS{3}Je ferai périr l'homme et le bétail~; je consumerai les oiseaux des cieux et les poissons de la mer~; et la ruine arrivera aux méchants, et je retrancherai les hommes de dessus cette terre, dit Yahweh.
\VS{4}J'étendrai ma main sur Juda, et sur tous les habitants de Jérusalem~; je retrancherai de ce lieu-ci le reste de Baal\FTNT{Voir commentaire en Jg. 2:13.}, les noms des prêtres des faux dieux, les prêtres,
\VS{5}ceux qui se prosternent sur les toits devant l'armée des cieux, ceux qui se prosternent devant Yahweh, qui jurent par lui, et qui jurent aussi par Malcom\FTNT{2 R. 17:33~; 2 R. 23:11-12~; Jé. 19:13.},
\VS{6}ceux qui se détournent de Yahweh, ceux qui n'ont point cherché Yahweh, qui ne l'ont point consulté.
\VS{7}Silence, à cause de la présence du Seigneur Yahweh, car le jour de Yahweh est proche\FTNT{Voir commentaire en Za. 14:1.}~; Yahweh a préparé le sacrifice, il a invité ses conviés.
\VS{8}Et il arrivera au jour du sacrifice de Yahweh que je punirai les chefs, et les enfants du roi, et tous ceux qui portent des vêtements étrangers.
\VS{9}Et je punirai, en ce jour-là, tous ceux qui sautent par-dessus le seuil, et ceux qui remplissent de violence et de fraude la maison de leurs maîtres.
\VS{10}Et en ce jour-là dit Yahweh, il y aura de grands cris vers la porte des poissons, et des hurlements vers la seconde partie de la ville, et une grande désolation sur les collines.
\VS{11}Vous qui habitez dans Macthesch\FTNT{Macthesch était un bas-quartier de Jérusalem où se trouvaient les marchés.}, hurlez~! Car tous ceux qui trafiquaient ont été détruits, et tous ceux qui apportaient de l'argent ont été retranchés.
\VS{12}Et il arrivera en ce temps-là que je fouillerai Jérusalem avec des lampes, que je punirai les hommes qui sont figés sur leurs lies, et qui disent dans leurs cœurs~: Yahweh ne nous fera ni bien ni mal.
\VS{13}Leurs biens seront au pillage et leurs maisons en désolation~; et ils auront bâti des maisons, mais ils ne les habiteront pas~; ils auront planté des vignes, mais ils n'en boiront pas le vin.
\VS{14}Le grand jour de Yahweh est proche, il est proche, et il se hâte beaucoup~; le jour de Yahweh n'est que bruit~; celui qui est dans l'amertume, crie de toute sa force. Là sont les hommes vaillants\FTNT{Jé. 30:7~; Joë. 2:11~; Am. 5:18.}.
\VS{15}Ce jour est un jour de fureur, un jour de détresse et d'angoisse, un jour de bruit éclatant et effrayant, un jour de ténèbres et d'obscurité, un jour de nuées et de brouillards~;
\VS{16}un jour de shofar et de cris de guerre contre les villes fortifiées, et contre les hautes tours.
\VS{17}Je mettrai les hommes dans la détresse, et ils marcheront comme des aveugles, parce qu'ils ont péché contre Yahweh~; et leur sang sera répandu comme de la poussière, et leur chair comme des ordures.
\VS{18}Ni leur argent ni leur or ne pourront les délivrer au jour de la fureur de Yahweh~; et tout ce pays sera dévoré par le feu de sa jalousie, car il se hâtera de consumer tous les habitants de ce pays\FTNT{Ez. 7:19~; Pr. 11:4.}.
\Chap{2}
\TextTitle{Yahweh invite Israël à la repentance}
\VerseOne{}Examinez-vous, examinez-vous avec soin ô nation non désirée\FTNT{1 Th. 5:21~; 2 Co. 13:5~; Ep. 5:10.}~!
\VS{2}Avant que le décret enfante, et que le jour passe comme la balle~; avant que l'ardeur de la colère de Yahweh vienne sur vous, avant que le jour de la colère de Yahweh vienne sur vous~!
\VS{3}Vous, tous les pauvres du pays, qui faites ce qu'il ordonne, cherchez Yahweh, cherchez la justice, cherchez l'humilité~; peut-être serez-vous protégés au jour de la colère de Yahweh\FTNT{Am. 5:15.}.
\VS{4}Mais Gaza sera abandonnée, et Askalon sera en désolation~; on chassera les habitants d'Asdod en plein midi, et Ekron sera arrachée\FTNT{Am. 8:9~; Za. 9:5.}.
\VS{5}Malheur aux habitants de la contrée maritime, à la nation des Kéréthiens~! La parole de Yahweh est contre vous~; Canaan, qui est le pays des Philistins, je te détruirai, si bien que, personne n'y habitera.
\VS{6}Et la contrée maritime sera des pâturages, des demeures pour les bergers, et des parcs pour les troupeaux.
\VS{7}Et cette contrée sera pour le reste de la maison de Juda~; ils paîtront dans ces lieux-là, et le soir ils feront leur gîte dans les maisons d'Askalon~; car Yahweh, leur Dieu, les visitera, et il ramènera leurs captifs.
\VS{8}J'ai entendu les insultes de Moab, et les outrages des fils d'Ammon, quand ils ont diffamé mon peuple, et l'ont bravé sur leur frontière\FTNT{Ez. 25:3-6.}.
\VS{9}C'est pourquoi, je suis vivant, dit Yahweh des armées, le Dieu d'Israël, Moab sera comme Sodome, et les fils d'Ammon comme Gomorrhe, un lieu couvert d'orties, et une carrière de sel et de désolation à jamais~; les restes de mon peuple les pilleront, et les restes de ma nation les posséderont.
\VS{10}Ceci leur arrivera en échange de leur orgueil, parce qu'ils ont usé d'insultes et d'arrogance, contre le peuple de Yahweh des armées\FTNT{Es. 16:6~; Jé. 48:29.}.
\VS{11}Yahweh sera terrible contre eux, car il anéantira tous les dieux du pays~; et on se prosternera devant lui, chacun de son lieu, même dans toutes les îles des nations\FTNT{Mal. 1:11~; Jn. 4:21.}.
\VS{12}Vous aussi, habitants de Cusch, vous serez blessés à mort par mon épée.
\VS{13}Il étendra aussi sa main sur le nord, et il détruira l'Assyrie, et il fera de Ninive une désolation, dans un lieu aride comme un désert.
\VS{14}Et les troupeaux feront leur gîte au milieu d'elle, et toutes les bêtes des nations, même le pélican et le hérisson, habiteront parmi les chapiteaux de ses colonnes~; la voix des oiseaux retentira à la fenêtre, la désolation sera au seuil, parce qu'il en aura abattu les cèdres\FTNT{Es. 14:23~; Es. 34:11.}.
\VS{15}C'est là cette ville remplie de joie, qui se tenait assurée, et qui disait en son cœur~: C'est moi, et il n'y en a point d'autre que moi~! Comment a-t-elle été réduite en désert, pour être le repère des bêtes~? Quiconque passera près d'elle sifflera et secouera sa main.
\Chap{3}
\TextTitle{Israël persiste dans l'immoralité}
\VerseOne{}Malheur à la ville immonde et souillée et qui ne fait qu'opprimer~!
\VS{2}Elle n'a point écouté la voix, elle n'a point reçu d'instruction, elle ne s'est point confiée en Yahweh, elle ne s'est point approchée de son Dieu.
\VS{3}Ses chefs au milieu d'elle sont des lions rugissants, et ses juges sont des loups du soir, qui ne gardent pas les os pour les ronger le matin\FTNT{Ez. 22:27~; Pr. 28:15.}.
\VS{4}Ses prophètes sont des téméraires, et des hommes infidèles~; ses prêtres ont souillé les choses saintes, ils ont fait violence à la loi\FTNT{Jé. 23:11-32.}.
\VS{5}Yahweh est juste au milieu d'elle, il ne commet point d'iniquité\FTNT{De. 32:4.}. Chaque matin il met en lumière son jugement, il n'y manque pas~; mais celui qui est inique ne sait ce que c'est que d'avoir honte.
\VS{6}J'ai exterminé les nations, et leurs forteresses ont été désolées~; j'ai rendu désertes leurs places, si bien que personne n'y passe~; leurs villes ont été détruites, sans qu'il y soit resté un seul homme, et sans qu'il y ait aucun habitant.
\VS{7}Et je disais~: Au moins tu me craindras, tu recevras instruction, et sa demeure ne sera pas retranchée, quelque soit la punition que je lui envoie. Mais ils se sont levés de bon matin, ils ont corrompu toutes leurs actions.
\VS{8}C'est pourquoi attendez-moi, dit Yahweh, au jour où je me lèverai pour le butin~; car j'ai résolu de rassembler les nations et de réunir les royaumes, pour répandre sur eux mon indignation, et toute l'ardeur de ma colère~; car tout le pays sera dévoré par le feu de ma jalousie.
\TextTitle{Un reste trouve refuge en Yahweh}
\VS{9}Alors je transformerai les langues\FTNT{Il est question ici de la conversion des peuples issus des nations (Ap. 7:9-17).} des nations en des langues pures, afin qu'elles invoquent toutes le Nom de Yahweh, pour qu'elles le servent d'un commun accord.
\VS{10}Mes adorateurs qui sont au-delà des fleuves de Cusch, à savoir la fille de mes dispersés, m'apporteront mes offrandes\FTNT{Es. 19:21~; Es. 27:13~; Ps. 68:31-32~; Ps. 72:10-11.}.
\VS{11}En ce jour-là, tu ne seras plus confuse à cause de toutes tes actions, par lesquelles tu as péché contre moi~; parce qu'alors j'aurai ôté du milieu de toi ceux qui se réjouissent de ton orgueil, et désormais tu ne t'enorgueilliras plus de la montagne de ma sainteté.
\VS{12}Et je laisserai au milieu de toi un peuple humble et faible, et il mettra sa confiance dans le Nom de Yahweh.
\VS{13}Les restes d'Israël ne commettront point d'iniquité, et ne proféreront point de mensonge, et il n'y aura point dans leur bouche de langue trompeuse~; aussi ils paîtront et se reposeront, et il n'y aura personne qui les épouvante.
\TextTitle{Israël délivré et restauré}
\VS{14}Réjouis-toi avec chant de triomphe, fille de Sion~! Pousse des cris de réjouissance, ô Israël~! Réjouis-toi et triomphe de tout ton cœur, fille de Jérusalem~!
\VS{15}Yahweh a aboli ta condamnation, il a éloigné ton ennemi. Le Roi d'Israël, Yahweh, est au milieu de toi~; tu ne verras plus de mal\FTNT{Ps. 46:5-6~; Col. 2:14.}.
\VS{16}En ce temps-là, on dira à Jérusalem~: Ne crains point Sion, que tes mains ne défaillent point~!
\VS{17}Yahweh, ton Dieu, est au milieu de toi comme le Puissant qui sauve~; il se réjouira à cause de toi d'une grande joie~; il se taira à cause de son amour, et se réjouira à cause de toi avec chant de triomphe.
\VS{18}Je rassemblerai ceux qui sont tristes à cause de l'assemblée solennelle, ils sont sortis de toi~; sur eux pèse l'opprobre.
\VS{19}Voici, je détruirai en ce temps-là tous ceux qui t'auront affligé~; je sauverai la boiteuse, je recueillerai celle qui avait été chassée, et je les ferai louer et devenir célèbres, dans tous les pays où ils auront été couverts de honte.
\VS{20}En ce temps-là, je vous ramènerai, et en ce temps-là je vous rassemblerai~; car je vous rendrai célèbres et un sujet de louange parmi tous les peuples de la terre, quand je ramènerai vos captifs sous vos yeux, dit Yahweh.
\PPE{}
\end{multicols}

%\clearpage\ShortTitle{Ag.}\BookTitle{Aggée}\BFont
\noindent\hrulefill
{\footnotesize
\textit{
\bigskip
{\centering{}
\\Auteur~: Aggée
\\(Heb.~: Chaggay)
\\Signification~: En fête, né un jour de fête
\\Thème~: Reconstruction du temple
\\Date de rédaction~: 6\up{ème} siècle av. J.-C.\\}
}
\textit{
\\Aggée, contemporain de Zacharie, exerça son service dans le royaume de Juda après le retour de l'exil. Alors que la reconstruction du temple était négligée, Aggée reçut un message rappelant au peuple quelles devaient être ses priorités et redéfinissant les exigences de Yahweh en matière de sainteté. Ce récit montre la bénédiction accompagnant celui qui oublie ses propres intérêts et qui prend véritablement à cœur l'œuvre de Dieu.\bigskip
}
}
\par\nobreak\noindent\hrulefill
\begin{multicols}{2}
\Chap{1}
\TextTitle{Israël coupable de négligence}
\VerseOne{}La seconde année du roi Darius, le premier jour du sixième mois, la parole de Yahweh vint par le moyen d'Aggée, le prophète, à Zorobabel, fils de Schealthiel, gouverneur de Juda, et à Josué, fils de Jotsadak, le grand-prêtre, en ces mots\FTNT{Esd. 4:24.}~:
\VS{2}Ainsi parle Yahweh des armées, en disant~:~Ce peuple dit : Le temps n'est pas encore venu, le temps de rebâtir la maison de Yahweh.
\VS{3}C'est pourquoi la parole de Yahweh a été adressée par le moyen d'Aggée, le prophète, en disant~:
\VS{4}Est-il temps pour vous d'habiter dans vos maisons lambrissées pendant que cette maison est en ruine~?
\VS{5}Maintenant donc, ainsi parle Yahweh des armées~: Considérez attentivement votre conduite~!
\VS{6}Vous avez semé beaucoup, mais vous avez récolté peu. Vous avez mangé, mais non pas jusqu'à être rassasiés. Vous avez bu, mais vous n'avez pas eu de quoi boire abondamment. Vous avez été vêtus, mais non pas jusqu'à en être échauffés. Et celui qui se loue, se loue pour mettre son salaire dans un sac percé\FTNT{Mi. 6:14-15.}.
\VS{7}Ainsi parle Yahweh des armées~: Considérez attentivement vos chemins~!
\VS{8}Montez à la montagne, apportez du bois, et bâtissez cette maison~; et j'y prendrai mon plaisir et je serai glorifié, a dit Yahweh.
\VS{9}Vous comptiez sur beaucoup, et voici, il y a eu peu~; vous l'avez apporté à la maison et j'ai soufflé dessus. Pourquoi~? A cause de ma maison, dit Yahweh des armées, parce qu'elle est en ruine pendant que vous vous empressez chacun pour sa maison.
\VS{10}A cause de cela, les cieux au-dessus de vous retiennent la rosée, et la terre a retenu ses fruits\FTNT{Lé. 26:19~; De. 28:23.}.
\VS{11}Et j'ai appelé la sécheresse sur la terre, et sur les montagnes, et sur le blé, et sur le moût, et sur l'huile, et sur tout ce que la terre produit, et sur les hommes et sur les bêtes, et sur tout le travail des mains\FTNT{Am. 4:7~; Ps. 105:16.}.
\TextTitle{Yahweh réveille son peuple}
\VS{12}Zorobabel donc, fils de Schealthiel, et Josué, fils de Jotsadak, le grand-prêtre, et tout le reste du peuple, entendirent la voix de Yahweh, leur Dieu, et les paroles d'Aggée, le prophète, ainsi que Yahweh, leur Dieu, l'avait envoyé~; et le peuple eut de la crainte devant Yahweh.
\VS{13}Et Aggée, messager de Yahweh, parla au peuple, suivant le message de Yahweh, en disant~: Je suis avec vous, dit Yahweh.
\VS{14}Et Yahweh réveilla l'esprit de Zorobabel, fils de Schealthiel, gouverneur de Juda, et l'esprit de Josué, fils de Jotsadak, le grand-prêtre, et l'esprit de tout le reste du peuple. Et ils vinrent et travaillèrent à la maison de Yahweh, leur Dieu,
\VS{15}le vingt-quatrième jour du sixième mois, de la seconde année du roi Darius.
\Chap{2}
\TextTitle{Encouragements à poursuivre la construction}
\VerseOne{}Le vingt et unième jour du septième mois, la parole de Yahweh vint par le moyen d'Aggée, le prophète, en disant~:
\VS{2}Parle maintenant à Zorobabel, fils de Schealthiel, gouverneur de Juda, et à Josué, fils de Jotsadak, le grand-prêtre, et au reste du peuple, en disant~:
\VS{3}Quel est parmi vous le survivant qui ait vu cette maison dans sa première gloire~? Et comment la voyez-vous maintenant~? N'est-elle pas comme un rien devant vos yeux, au prix de celle-là\FTNT{Esd. 3:12.}~?
\VS{4}Maintenant donc Zorobabel, fortifie-toi~! dit Yahweh. Toi aussi, Josué, fils de Jotsadak, grand-prêtre, fortifie-toi~! Vous aussi, tout le peuple du pays, fortifiez-vous~! dit Yahweh. Et travaillez, car je suis avec vous, dit Yahweh des armées.
\VS{5}La parole de l'Alliance que je traitai avec vous, quand vous sortîtes d'Egypte, et mon Esprit, demeurent au milieu de vous~; ne craignez point\FTNT{Za. 4:6.}~!
\VS{6}Car ainsi parle Yahweh des armées~: Encore un peu de temps, et j'ébranlerai les cieux et la terre, la mer et le sec\FTNT{Hé. 12:26.}~;
\VS{7}j'ébranlerai toutes les nations~; et les trésors de toutes les nations viendront, et je remplirai de gloire cette maison, dit Yahweh des armées.
\VS{8}L'argent est à moi, et l'or est à moi, dit Yahweh des armées.
\VS{9}La gloire de cette dernière maison sera plus grande que celle de la première, dit Yahweh des armées~; et je mettrai la paix en ce lieu, dit Yahweh des armées.
\TextTitle{Purification et sanctification du peuple}
\VS{10}Le vingt-quatrième jour du neuvième mois de la seconde année de Darius, la parole de Yahweh vint par le moyen d'Aggée le prophète, en disant~:
\VS{11}Ainsi parle Yahweh des armées~: Interroge maintenant les prêtres sur la loi en ces mots~:
\VS{12}Si quelqu'un porte de la chair consacrée dans le pan de son vêtement, et que ce vêtement touche du pain, ou un mets cuit, ou du vin, ou de l'huile, ou un aliment quelconque, cela devient-il sanctifié~? Et les prêtres répondirent et dirent~: Non~!
\VS{13}Alors Aggée dit~: Si celui qui est souillé pour un mort touche toutes ces choses-là, ne seront-elles pas souillées~? Et les prêtres répondirent et dirent~: Elles seront souillées\FTNT{Lé. 17:15~; No. 19:22~; Tit. 1:15.}.
\VS{14}Alors Aggée répondit et dit~: Tel est ce peuple et telle est cette nation devant ma face, dit Yahweh~; et telles sont toutes les œuvres de leurs mains~; même ce qu'ils offrent ici est souillé.
\VS{15}Maintenant donc mettez ceci, je vous prie, dans votre cœur, depuis ce jour et par la suite, avant qu'on ait mis pierre sur pierre au temple de Yahweh~!
\VS{16}Avant cela, dis-je, quand on venait à un monceau de blé, au lieu de vingt mesures, il n'y en avait que dix~; et quand on  venait au pressoir, au lieu de puiser de la cuve cinquante mesures, il n'y en avait que vingt.
\VS{17}Je vous ai frappés de brûlure, de rouille, de grêle, dans tout le travail de vos mains. Et vous n'êtes point revenus à moi, dit Yahweh\FTNT{De. 28:22~; 1 R. 8:37~; Am. 4:9~; 2 Ch. 6:28.}.
\VS{18}Mettez maintenant ceci dans votre cœur~; depuis ce jour-ci et dans la suite~; depuis, dis-je, le vingt-quatrième jour du neuvième mois, depuis le jour où les fondements du temple de Yahweh ont été posés, mettez ceci dans votre cœur~!
\VS{19}Ya-t-il encore de la semence dans les greniers~? Même jusqu'à la vigne, au figuier, au grenadier, et à l'olivier, rien n'a rapporté~; mais depuis ce jour-ci, je donnerai la bénédiction.
\TextTitle{Destruction des royaumes des nations}
\VS{20}Et la parole de Yahweh vint pour la seconde fois à Aggée, le vingt-quatrième jour du mois, en disant~:
\VS{21}Parle à Zorobabel, gouverneur de Juda, et dis-lui~: J'ébranlerai les cieux et la terre~;
\VS{22}je renverserai le trône des royaumes, je détruirai la force des royaumes des nations, je renverserai les chars et ceux qui les montent~; et les chevaux et ceux qui les montent seront abattus, chacun par l'épée de son frère.
\VS{23}En ce jour-là, dit Yahweh des armées, je te prendrai, ô Zorobabel, fils de Schealthiel, mon serviteur, dit Yahweh~; et je te mettrai comme un sceau\FTNT{Le sceau est un symbole d'autorité.}, car je t'ai choisi, dit Yahweh des armées.
\PPE{}
\end{multicols}

%\clearpage\ShortTitle{Zacharie}\BookTitle{Zacharie}\BFont
\noindent\hrulefill
{\footnotesize
\textit{
\bigskip
{\centering{}
\\Signifie : Yahweh se souvient
\\Thème : Les deux avènements du Messie
\\Auteur : Zacharie
\\Date de rédaction : 6ème siècle av. J.-C.\\}
}
%\bigskip
\textit{
\\Zacharie, contemporain d’Aggée, exerça son ministère en Juda au retour des exilés de Babylone, où il était né. Il annonça la venue du Messie et raconta de manière très précise différents épisodes de sa vie, également retrouvés dans le récit des évangiles. Il dévoila également quelques-uns des attributs du Sauveur, premièrement rejeté mais finalement accepté par le peuple juif pendant le millenium. On y découvre ainsi le Christ en tant que souverain sacrificateur, germe, serviteur, ange de l’Yahweh, roi de paix, fils de David…\bigskip
}
}
\par\nobreak\noindent\hrulefill
\begin{multicols}{2}
\Chap{1}
\TextTitle{Yahweh avertit son peuple}
\VerseOne{}Le huitième mois de la deuxième année de Darius, la parole de Yahweh fut adressée à Zacharie, le prophète, fils de Bérékia, fils d’Iddo, en ces mots :
\VS{2}Yahweh a été extrêmement irrité contre vos pères.
\VS{3}C'est pourquoi tu leur diras : Ainsi parle Yahweh des armées : Revenez à moi, dit Yahweh des armées, et je reviendrai à vous, dit Yahweh des armées\FTNT{Joë. 2:12 ; Es. 31:6 ; Jé. 3:12.}.
\VS{4}Ne soyez point comme vos pères, auxquels s’adressaient les premiers prophètes, en disant : Ainsi a dit Yahweh des armées : Détournez-vous maintenant de vos mauvaises voies et de vos mauvaises actions ! Mais ils n’écoutèrent pas, ils ne furent pas attentifs à ce que je leur disais, dit Yahweh\FTNT{2 Ch. 29:6 ; Esd. 9:7 ; Né. 9:16 ; La. 5:7.}.
\VS{5}Vos pères où sont-ils ? Et ces prophètes-là pouvaient-ils vivre éternellement ?
\VS{6}Cependant mes paroles et mes ordonnances que j'avais données aux prophètes, mes serviteurs, n'ont-elles pas atteint vos pères ? De sorte qu’étant revenus, ils ont dit : Yahweh des armées nous a traités comme il avait résolu de le faire, selon nos voies et nos actions.
\TextTitle{Le cavalier sur le cheval roux}
\VS{7}Le vingt-quatrième jour du onzième mois, qui est le mois de Schebat, la deuxième année de Darius, la parole de Yahweh fut adressée à Zacharie, le prophète, fils de Bérékia, fils d’Iddo, en ces mots :
\VS{8}Je voyais de nuit une vision, et voici, un homme était monté sur un cheval roux, et il se tenait parmi des myrtes qui étaient dans un lieu creux ; il y avait derrière lui des chevaux roux, fauves et blancs\FTNT{Ap. 6:2-4.}.
\VS{9}Je dis : Mon Seigneur ! Que signifient ces choses ? Et l’Ange qui me parlait me dit : Je te montrerai ce que signifient ces choses.
\VS{10}L’homme qui se tenait parmi les myrtes répondit et dit : Ce sont ceux que Yahweh a envoyés pour parcourir la terre.
\VS{11}Et ils répondirent à l'Ange de Yahweh\FTNT{Voir commentaire en Ge. 16:7.} qui se tenait parmi les myrtes, et dirent : Nous avons parcouru la terre ; et voici, toute la terre est en repos et tranquille.
\TextTitle{La compassion de Yahweh pour Jérusalem}
\VS{12}Alors l'Ange de Yahweh répondit et dit : Yahweh des armées, jusqu'à quand n'auras-tu pas compassion de Jérusalem et des villes de Juda, contre lesquelles tu es irrité depuis soixante-dix ans\FTNT{Jérémie prophétisa que la captivité babylonienne durerait soixante-dix ans (Jé. 25:11-12 ; Jé. 29:10). Les soixante-dix ans commencèrent à la déportation de la famille royale à Babylone en 605 av. J.-C. (2 R. 24 ; Da. 1) et se terminèrent avec la première vague de retours conduite par Zorobabel (Esd. 1). Les Israélites furent emmenés en captivité en plusieurs vagues. Le livre d’Esdras raconte les deux premières. En 538 av. J.-C., Zorobabel mena la première vague et fut nommé gouverneur (Ag. 1:1). Le sacrificateur Josué  (Esd. 3:2) et les prophètes Aggée et Zacharie (Es. 5:1-2) le secondaient. Leur plus grand défi fut de rebâtir le temple. Puisque la seule tribu à retourner en masse fut celle de Juda, dès lors, le reste du peuple fut appelé «~les Juifs~» (Esd. 4:23).} ?
\VS{13}Yahweh répondit à l'Ange qui me parlait, par de bonnes paroles, par des paroles de consolation.
\VS{14}Puis l'Ange qui me parlait me dit : Crie, en disant : Ainsi parle Yahweh des armées : Je suis ému d'une grande jalousie pour Jérusalem et pour Sion,
\VS{15}et je suis extrêmement irrité contre les nations qui sont à leur aise ; car je n’étais que peu irrité, mais elles ont contribué au mal.
\VS{16}C'est pourquoi ainsi parle Yahweh : Je reviens à Jérusalem avec compassion, et ma maison y sera rebâtie, dit Yahweh des armées ; et le cordeau sera étendu sur Jérusalem.
\VS{17}Crie encore, et dis : Ainsi parle Yahweh des armées : Mes villes regorgeront encore de biens, et Yahweh consolera encore Sion, il choisira encore Jérusalem.
\TextTitle{Les quatre cornes et les quatre forgerons}
\VS{18}Puis je levai les yeux et je regardai ; et voici, quatre cornes\FTNT{Da. 7:7-11 ; Da. 8:22 ; Ap. 13:1-11.}.
\VS{19}Alors je dis à l'Ange qui me parlait : Que veulent dire ces choses ? Et il me répondit : Ce sont les cornes qui ont dispersé Juda, Israël et Jérusalem.
\VS{20}Puis Yahweh me fit voir quatre forgerons.
\VS{21}Je dis : Que viennent-ils faire ? Et il répondit et dit : Ce sont les cornes qui ont dispersé Juda, au point que personne ne lève la tête ; et ces forgerons sont venus pour les effrayer, et pour abattre les cornes des nations qui ont levé la corne contre le pays de Juda, pour le disperser.
\Chap{2}
\TextTitle{L'homme tenant dans sa main le cordeau pour mésurer}
\VerseOne{}Je levai encore mes yeux et je regardai, et voici, il y avait un homme tenant dans la main un cordeau pour mesurer,
\VS{2}auquel je dis : Où vas-tu ? Et il me répondit : Je vais mesurer Jérusalem, pour voir quelle est sa largeur et quelle est sa longueur.
\VS{3}Et voici, l'Ange qui me parlait s’avança, et un autre ange sortit à sa rencontre.
\TextTitle{Yahweh, la gloire de Jérusalem}
\VS{4}Il lui dit : Cours, et parle à ce jeune homme, et dis : Jérusalem sera habitée comme les villes sans murailles, à cause de la multitude d'hommes et de bêtes qui seront au milieu d'elle\FTNT{Né. 1:3 ; Né. 2:13.}.
\VS{5}Mais je serai pour elle, dit Yahweh, une muraille de feu tout autour, et je serai sa gloire au milieu d'elle\FTNT{Es. 60:19.}.
\VS{6}Ha ! Fuyez, fuyez hors du pays du nord ! dit Yahweh. Car je vous ai dispersés aux quatre vents des cieux, dit Yahweh.
\VS{7}Ha ! Sauve-toi, Sion, toi qui habites chez la fille de Babylone\FTNT{Jé. 50:8 ; Es. 48:20 ; Es. 52:11 ; Jé. 51:6.} !
\VS{8}Car ainsi parle Yahweh des armées, lequel après la gloire, il m'a envoyé vers les nations qui ont fait de vous leur proie ; car celui qui vous touche, touche à la prunelle de son œil\FTNT{De. 32:10 ; Ps. 17:8.}.
\VS{9}Car voici, je vais lever ma main contre elles, et elles seront la proie de ceux qui leur étaient asservis. Et vous saurez que Yahweh des armées m'a envoyé.
\VS{10}Pousse des cris d’allégresse et réjouis-toi, fille de Sion ! Car voici, je viens\FTNT{Jésus-Christ est Yahweh qui vient ! C’est la seconde venue de Jésus-Christ qui est évoquée ici (Es. 40:10-11 ; Za. 12:10-14 ; Za. 14:1-10 ; Ac. 1:1-11 ; Ap. 1:7-8 ; Ap. 22:12-17).}, et j'habiterai au milieu de toi, dit Yahweh.
\VS{11}Beaucoup de nations se joindront à Yahweh en ce jour-là, et deviendront mon peuple ; et j'habiterai au milieu de toi ; et tu sauras que Yahweh des armées m'a envoyé vers toi.
\VS{12}Yahweh possédera Juda comme sa part dans la terre sainte, et il choisira encore Jérusalem.
\VS{13}Que toute chair fasse silence devant la face de Yahweh ! Car il s'est réveillé de sa demeure sainte.
\Chap{3}
\TextTitle{Yahweh enlève l'iniquité du pays}
\VerseOne{}Puis Yahweh me fit voir Josué, le souverain sacrificateur, se tenant debout devant l'Ange de Yahweh\FTNT{Voir commentaire en Ge. 16:7.}, et Satan qui se tenait debout à sa droite, pour l’accuser.
\VS{2}Yahweh dit à Satan : Que Yahweh te réprime, ô Satan ! Que Yahweh, dis-je, qui a choisi Jérusalem, te réprime ! N’est-ce pas là un tison qui a été retiré du feu\FTNT{Jud. 1:9 ; Am. 4:11.} ?
\VS{3}Or Josué était vêtu de vêtements sales, et il se tenait debout devant l'Ange.
\VS{4}L’Ange prit la parole et dit à ceux qui étaient debout devant lui : Otez-lui ces vêtements sales ! Et il dit à Josué : Regarde, je t’enlève ton iniquité, et je te revêts d’habits de fête.
\VS{5}Je dis : Qu'on mette sur sa tête un turban pur ! Et ils mirent un turban pur sur sa tête, puis ils lui mirent des vêtements\FTNT{Ap. 19:8.}. L’Ange de Yahweh était présent.
\VS{6}Alors l'Ange de Yahweh fit à Josué cette déclaration, en disant :
\VS{7}Ainsi parle Yahweh des armées : Si tu marches dans mes voies, et si tu observes mes commandements, tu jugeras ma maison, tu garderas mes parvis, et je te donnerai libre accès parmi ceux qui se tiennent devant moi.
\VS{8}Ecoute maintenant, Josué, souverain sacrificateur, toi, et tes compagnons qui sont assis devant toi ! Car ce sont des hommes qui serviront de signes. Certainement voici je ferai venir mon serviteur, le Germe\FTNT{Le Germe est un autre nom de Jésus-Christ, notre Seigneur (Es. 4:2).}.
\VS{9}Car voici, quant à la pierre\FTNT{Jésus-Christ est le Rocher des âges (Es. 8:13-17; Ap. 5:1-7).} que j'ai mise devant Josué, sur cette pierre, qui n'est qu'une\FTNT{Cette pierre est UNE (~E’had~) c’est-à-dire indivisible (De. 6:4).}, il y a sept yeux. Voici, je graverai moi-même ce qui doit y être gravé, dit Yahweh des armées ; et j'ôterai en un jour l'iniquité de ce pays.
\VS{10}En ce jour-là, dit Yahweh des armées, chacun de vous appellera son prochain sous la vigne et sous le figuier.
\Chap{4}
\TextTitle{Le peuple de Yahweh peut tout par son Esprit}
\VerseOne{}Puis l'Ange qui me parlait revint, et il me réveilla comme un homme que l’on réveille de son sommeil.
\VS{2}Il me dit : Que vois-tu ? Et je répondis : Je regarde, et voici, il y a un chandelier tout en or, surmonté d’un vase et portant ses sept lampes, avec sept conduits pour les sept lampes qui sont au sommet du chandelier\FTNT{Ap. 1:12-13.} ;
\VS{3}et il y a deux oliviers près de lui, l'un à la droite du vase, et l'autre à sa gauche.
\VS{4}Alors je pris la parole et je dis à l'Ange qui me parlait : Mon Seigneur, que signifient ces choses ?
\VS{5}L'Ange qui me parlait répondit et me dit : Ne sais-tu pas ce que signifient ces choses ? Je dis : Non, mon Seigneur !
\VS{6}Alors il reprit et me dit : C'est ici la parole que Yahweh adresse à Zorobabel : Ce n'est point par la puissance ni par la force, mais par mon Esprit, dit Yahweh des armées.
\VS{7}Qui es-tu, grande montagne, devant Zorobabel ? Tu seras aplanie. Il fera sortir la pierre principale ; il y aura des sons éclatants : Grâce, grâce pour elle !
\TextTitle{Yahweh encourage son peuple à achever l'oeuvre commencée}
\VS{8}Aussi la parole de Yahweh me fut adressée en ces mots :
\VS{9}Les mains de Zorobabel ont fondé cette maison, et ses mains l'achèveront ; et tu sauras que Yahweh des armées m'a envoyé vers vous.
\VS{10}Car qui est-ce qui a méprisé le jour des faibles commencements ? Ils se réjouiront en voyant le niveau dans la main de Zorobabel.  Ces sept\FTNT{Les sept yeux de Yahweh sont aussi les sept yeux de l’Agneau (Ap. 5 : 6). Ces yeux représentent l’omniscience et l’omniprésence de Jésus-Christ (Za. 3:9 ; Za 4:10. ; Za. 14:7 ; Jn. 16:30 ; Ac. 1:24 ; Ap. 21:17).} sont les yeux de Yahweh qui parcourent toute la terre.
\VS{11}Je pris la parole et je lui dis : Que signifient ces deux oliviers\FTNT{L’identité de ces deux individus est inconnue.  Selon Ap. 11:3, ces deux hommes recevront des pouvoirs incroyables pour les trois années et demi de la grande tribulation qui précéderont le retour du Christ. Si quiconque tente de leur faire du mal ou d’interférer dans leur ministère et leur témoignage, «~… du feu sortirait de leur bouche et dévorerait leurs ennemis~», (Ap. 11:5). Ils auront aussi le pouvoir de provoquer la sécheresse et la famine sur la terre, tout comme l’avait fait Elie (1 R. 17:1-7 ; 2 R. 1:9-15 ; Lu. 4:25). Ils auront également le pouvoir de frapper la Terre par des plaies diverses, semblables à celles provoquées par Moïse (Chapitres 7, 8, 9, 10, 11 d’Exode ; Ap. 11:6).}, à la droite et à la gauche du chandelier ?
\VS{12}Je pris la parole pour la seconde fois et je lui dis : Que signifient ces deux branches d'olivier qui sont près des deux conduits d'or, d’où l'or découle ?
\VS{13}Il me répondit et dit : Ne sais-tu pas ce que signifient ces choses ? Et je dis : Non, mon Seigneur.
\VS{14}Et il dit : Ce sont les  deux fils oints, qui se tiennent devant le Seigneur de toute la terre.
\Chap{5}
\TextTitle{La malédiction se répands sur Israël}
\VerseOne{}Puis je me retournai, et levai mes yeux pour regarder ; et voici, un rouleau qui volait.
\VS{2}Alors il me dit : Que vois-tu ? Je répondis : Je vois un rouleau qui vole, dont la longueur est de vingt coudées, et la largeur de dix coudées.
\VS{3}Et il me dit : C'est l’exécration du serment qui sort sur la face de tout le pays ; car selon elle, quiconque d'entre ce peuple-ci vole, sera puni comme elle ; et selon elle, quiconque d'entre ce peuple parjure, sera puni comme elle.
\VS{4}Je déploierai cette exécration dit Yahweh des armées, et elle entrera dans la maison du voleur, et dans la maison de celui qui jure faussement en mon Nom, et elle logera au milieu de leur maison, et la consumera avec son bois et ses pierres.
\TextTitle{L'épha au pays de Schinear}
\VS{5}L’Ange qui me parlait sortit, et me dit : Lève maintenant tes yeux, et regarde ce qui sort là.
\VS{6}Et je dis : Qu'est-ce ? Et il répondit : C'est l’épha\FTNT{L'épha était une unité de mesure utilisée dans le commerce des céréales, souvent à des fins frauduleuses (De. 25:14 ; Mi. 6:10 ; Am. 8:5).} qui sort dehors. Puis il dit : C'est ici leur aspect dans tout le pays.
\VS{7}Et voici, on portait une masse de plomb, et une femme était assise au milieu de l'épha\FTNT{Zacharie voit une femme assise au milieu de l'épha. L'ange déclare : «~C'est la méchanceté ou l’iniquité~». Elle représente la grande prostituée décrite en Ap. 17, avec sa coupe d'or pleine de ses abominations et des impuretés de sa fornication (v. 4). Cette femme est la figure du «~mystère de l'iniquité qui opère déjà~» (2 Th. 2:7).}.
\VS{8}Il dit : C'est là l’iniquité\FTNT{L’iniquité ou la méchanceté.} ; puis il la repoussa dans l'épha, et il jeta la masse de plomb sur l’ouverture.
\VS{9}Je levai les yeux et je regardai, et voici deux femmes sortirent\FTNT{Les deux femmes ayant «~des ailes de cigogne~» apparaissent portées par le vent. Sous Moïse, cet oiseau devait être considéré comme impur (Lé. 11 : 19). Dans les Ecritures, le vent est constamment en relation avec le jugement (Job. 27:20-22 ; Job. 30:22 ; Es. 7:2 ; Es. 26:6 ; Es. 41:16). Elles soulèvent l'épha et l'emportent dans son lieu d'origine, le pays de Schinear, c’est-à-dire Babylone, pour lui bâtir une maison, au siège même de l'idolâtrie et de la révolte contre Dieu. (Ge. 11:2-9 ; 2 R. 17:24).}. Le vent soufflait dans leurs ailes : Elles avaient des ailes comme les ailes de la cigogne. Et elles enlevèrent l'épha entre la terre et le ciel.
\VS{10}Je dis à l'Ange qui me  parlait : Où emportent-elles l'épha ?
\VS{11}Il me répondit : C'est pour lui bâtir une maison dans le pays de Schinear\FTNT{Schinear ou Babylone (Ge. 10:6-12).} ; et quand elle sera prête, il sera déposé là, sur sa base.
\Chap{6}
\TextTitle{Les quatre vents des cieux}
\VerseOne{}Je levai encore les yeux et je regardai, et voici quatre chars\FTNT{Dans les Ecritures, les chars et les chevaux représentent souvent la puissance de Dieu exerçant un jugement sur la terre (Jé. 46:9-10 ; Joë. 2:3-11). Ce jugement concerne le monde entier (Ap. 6:1-8).} sortaient d'entre deux montagnes ; et ces montagnes étaient des montagnes d'airain.
\VS{2}Au premier char, il y avait des chevaux roux ; au deuxième char, des chevaux noirs,
\VS{3}au troisième char, des chevaux blancs, et au quatrième char, des chevaux tachetés, rouges.
\VS{4}Je pris la parole et je dis à l'Ange qui me parlait : Mon Seigneur, que veulent dire ces choses ?
\VS{5}L’Ange répondit et me dit : Ce sont les quatre vents des cieux, qui sortent du lieu où ils se tenaient devant le Seigneur de toute la terre.
\VS{6}Quant au char où sont les chevaux noirs, ils se dirigent vers le pays du nord, et les blancs sortent après eux ; les tachetés se dirigent vers le pays du midi.
\VS{7}Ensuite les rouges sortirent et demandèrent à aller parcourir la terre. L’Ange leur dit : Allez, et parcourez la terre ! Et ils parcoururent la terre.
\VS{8}Puis il m'appela, et me parla, en disant : Voici, ceux qui se dirigent vers le pays du nord ont apaisé mon Esprit dans le pays du nord.
\TextTitle{Prophétie sur le règne du germe de Yahweh}
\VS{9}La parole de Yahweh me fut adressée en ces mots :
\VS{10}Tu recevras les dons de ceux qui sont de retour de la captivité : Heldaï, Tobija et Jedaeja. Et tu iras toi-même ce même jour-là, et tu iras dans la maison de Josias, fils de Sophonie, où ils se sont rendus en arrivant de Babylone.
\VS{11}Tu prendras de l'argent et de l'or, et tu en feras des couronnes que tu mettras sur la tête de Josué, fils de Jotsadak, le souverain sacrificateur.
\VS{12}Tu lui diras : Ainsi parle Yahweh des armées : Voici un homme, dont le nom est Germe\FTNT{Es. 4:2.}, germera dans son lieu, et bâtira le temple de Yahweh\FTNT{C’est Yahweh, c’est-à-dire Jésus-Christ lui-même, qui bâtit son temple (Ps. 127:1-2 ; Mt. 16:18).}.
\VS{13}Oui, lui-même bâtira le temple de Yahweh ; et lui-même sera rempli de majesté. Il s’assiéra et dominera sur son trône, il sera Sacrificateur\FTNT{Jésus-Christ est Souverain Sacrificateur (Hé. 6:20 ; Hé. 7:1-28).}, étant sur son trône ; et il y aura un conseil de paix entre les deux.
\VS{14}Les couronnes seront pour Hélem, Tobija et Jedaeja, et pour Hen, fils de Sophonie, un souvenir dans le temple de Yahweh.
\VS{15}Ceux qui sont éloignés viendront, et travailleront au temple de Yahweh ; et vous saurez que Yahweh des armées m'a envoyé vers vous. Cela arrivera, si vous écoutez attentivement la voix de Yahweh, votre Dieu.
\Chap{7}
\TextTitle{Yahweh dénonce le jeûne formaliste}
\VerseOne{}La quatrième année du roi Darius, la parole de Yahweh fut adressée à Zacharie, le quatrième jour du neuvième mois, qui est le mois de Kisleu.
\VS{2}On avait envoyé à Béthel Scharetser et Réguem-Mélec avec ses gens, pour supplier Yahweh,
\VS{3}et pour parler aux sacrificateurs de la maison de Yahweh des armées, et aux prophètes, en disant : Dois-je pleurer au cinquième mois, et faire abstinence, comme j'ai déjà fait pendant plusieurs années ?
\VS{4}La parole de Yahweh des armées me fut adressée en ces mots :
\VS{5}Parle à tout le peuple du pays et aux sacrificateurs, et dis-leur : Quand vous avez jeûné et pleuré au cinquième mois et au septième, et cela depuis soixante-dix ans, avez-vous célébré ce jeûne par amour pour moi ?
\VS{6}Et quand vous buvez et mangez, n'est-ce pas vous qui mangez et vous qui buvez\FTNT{Es. 58:3-4.} ?
\VS{7}Ne connaissez-vous pas les paroles qu’a proclamées Yahweh par les premiers prophètes, lorsque Jérusalem était habitée et paisible avec ses villes à l’entour, et que le midi et la plaine étaient habités ?
\TextTitle{Yahweh n'exauce pas les pécheurs}
\VS{8}Puis la parole de Yahweh fut adressée à Zacharie en ces mots :
\VS{9}Ainsi parlait Yahweh des armées, en disant : Rendez véritablement la justice, et exercez la miséricorde et la compassion chacun envers son frère.
\VS{10}N’opprimez pas la veuve et l'orphelin, l'étranger et le pauvre, et ne méditez aucun mal dans vos cœurs chacun contre son frère\FTNT{Ex. 22:21 ; Es. 1:23 ; Jé. 5:28 ; Pr. 22:22-23.}.
\VS{11}Mais ils refusèrent d’être attentifs, ils eurent l'épaule rebelle, et ils endurcirent leurs oreilles pour ne pas entendre.
\VS{12}Ils rendirent leur cœur dur comme le diamant, pour ne pas écouter la loi et les paroles que Yahweh des armées adressait par son Esprit, par les premiers prophètes. C’est pourquoi  Yahweh des armées s’enflamma d’une grande colère.
\VS{13}Quand il appelait, ils n'ont pas écouté. Aussi n’ai-je pas écouté, quand ils ont appelé, dit Yahweh des armées\FTNT{Pr. 1:28 ; Es. 1:15 ; Jé. 11:11.}.
\VS{14}Je les ai dispersés comme par un tourbillon parmi toutes les nations qu'ils ne connaissaient pas ; le pays a été dévasté derrière eux, il n’y a plus eu ni allants ni venants ; et d’un pays de délices ils ont fait un désert.
\Chap{8}
\TextTitle{Futur royaume d'Israël rétabli dans la justice}
\VerseOne{}La parole de Yahweh des armées me fut encore adressée en ces mots :
\VS{2}Ainsi parle Yahweh des armées : Je suis jaloux pour Sion d'une grande jalousie, et je suis jaloux pour elle d’une grande fureur.
\VS{3}Ainsi parle Yahweh : Je retourne à Sion, et j'habiterai au milieu de Jérusalem ; et Jérusalem sera appelée ville fidèle ; et la montagne de Yahweh des armées sera appelée montagne sainte\FTNT{Es. 1:26.}.
\VS{4}Ainsi parle Yahweh des armées : Il y aura encore des vieillards et des femmes âgées, assis dans les rues de Jérusalem, et chacun aura son bâton à la main, à cause du grand nombre de leurs jours.
\VS{5}Les rues de la ville seront remplies de fils et de filles, jouant dans les rues.
\VS{6}Ainsi parle Yahweh des armées : S’il semble difficile aux yeux du reste de ce peuple que cela arrive, en ces jours-là, sera-t-il de même difficile à mes yeux ? dit Yahweh des armées.
\VS{7}Ainsi parle Yahweh des armées : Voici, je délivre mon peuple du pays de l'orient et du pays du soleil couchant.
\VS{8}Je les ramènerai, et ils habiteront au milieu de Jérusalem ; ils seront mon peuple, et je serai leur Dieu avec vérité et droiture.
\TextTitle{Juger selon la vérité}
\VS{9}Ainsi parle Yahweh des armées : Que vos mains soient fortifiées, vous qui entendez aujourd’hui ces paroles de la bouche des prophètes qui parurent au jour où la maison de Yahweh fut fondée, et où le temple allait être bâti\FTNT{Ag. 2:4.}.
\VS{10}Car avant ces jours-là, il n'y avait pas de salaire pour l'homme ni de salaire pour la bête ; et il n'y avait pas de paix pour ceux qui entraient et sortaient, à cause de la détresse ; et je lâchais tous les hommes les uns contre les autres.
\VS{11}Mais maintenant je ne serai pas pour le reste de ce peuple comme les premiers jours,  dit Yahweh des armées.
\VS{12}Car les semailles prospéreront, la semence de paix sera là ; la vigne rendra son fruit, et la terre donnera ses produits ; les cieux donneront leur rosée, et je ferai hériter toutes ces choses au reste de ce peuple.
\VS{13}De même que vous avez été en malédiction parmi les nations, ô maison de Juda et maison d'Israël, de même je vous délivrerai, et vous serez en bénédiction. Ne craignez pas, mais que vos mains soient fortifiées\FTNT{Ge. 1:11 ; Ap. 22:2.}.
\VS{14}Car ainsi parle Yahweh des armées : Comme j'ai eu la pensée de vous affliger, lorsque vos pères ont provoqué ma colère, dit Yahweh des armées, et que je ne m'en suis point repenti,
\VS{15}ainsi je reviens en arrière et j’ai résolu en ces jours de faire du bien à Jérusalem, et à la maison de Juda. Ne craignez pas !
\VS{16}Voici les choses que vous devez faire : Que chacun dise la vérité à son prochain ; jugez selon la vérité et prononcez un jugement en vue de la paix dans vos portes\FTNT{Ep. 4:25 ; Ex. 20:16 ; Mt. 19:18 ; Lu. 18:20.} ;
\VS{17}que personne ne projette du mal dans son cœur contre son prochain ; et n'aimez point le faux serment, car ce sont là des choses que je hais, dit Yahweh\FTNT{Ps. 5:5 ; Ps. 11:5 ; Pr. 6:16-19.}.
\VS{18}Puis la parole de Yahweh des armées me fut adressée en ces mots :
\VS{19}Ainsi parle Yahweh des armées : Le jeûne du quatrième mois, le jeûne du cinquième, le jeûne du septième et le jeûne du dixième seront changés pour la maison de Juda en joie et en allégresse, et en fêtes solennelles de réjouissance. Aimez donc la vérité et la paix\FTNT{Ep. 4:15.}.
\TextTitle{Les nations reconnaissent que Yahweh est le seul Dieu}
\VS{20}Ainsi parle Yahweh des armées : Il viendra encore des peuples et des habitants de plusieurs villes.
\VS{21}Les habitants d’une ville  iront à l'autre, en disant : Allons, allons implorer Yahweh et chercher Yahweh des armées ! Nous irons aussi !
\VS{22}Et beaucoup de peuples et de puissantes nations viendront rechercher Yahweh des armées à Jérusalem\FTNT{Jérusalem est appelée à devenir le centre d’adoration de la terre et la capitale du monde à cause de la présence de Dieu (Es. 66:23 ; Za. 14:16-21).}, et implorer Yahweh.
\VS{23}Ainsi parle Yahweh des armées : En ce jour-là, dix hommes de toutes les langues des nations saisiront le pan de la robe d'un homme Juif, et diront : Nous irons avec vous, car nous avons entendu que Dieu est avec vous.
\Chap{9}
\TextTitle{Le jugement de Yahweh sur les nations}
\VerseOne{}Oracle, parole de Yahweh sur le pays de Hadrac. Elle s’arrête sur Damas, car Yahweh a l’œil sur les hommes et sur toutes les tribus d'Israël.
\VS{2}Il s’arrête aussi sur Hamath, à la frontière de Damas, sur Tyr, et Sidon, quoique chacune d'elles soit fort sage.
\VS{3}Car Tyr s'est bâti une forteresse ; elle a amassé l'argent comme la poussière, et l’or fin comme la boue des rues\FTNT{Ez. 28:3-17.}.
\VS{4}Voici, le Seigneur l'appauvrira, et en la frappant, il jettera sa puissance dans la mer, et elle sera consumée par le feu\FTNT{Ez. 26:3-4.}.
\VS{5}Askalon le verra, et elle sera dans la crainte ; Gaza aussi le verra, et un violent tremblement la saisira ; Ekron aussi, car son espoir sera confondu. Et il n'y aura plus de roi à Gaza, et Askelon ne sera plus habitée\FTNT{So. 2:4.}.
\VS{6}Et le bâtard habitera à Asdod ; et j’abattrai l'orgueil des Philistins.
\VS{7}J'ôterai le sang de la bouche de chacun d'eux, et leurs abominations d'entre leurs dents ; et lui aussi restera pour notre Dieu, il sera comme un chef en Juda, et Ekron sera comme le Jébusien.
\VS{8}Je camperai autour de ma maison, pour la défendre contre une armée, contre les allants et les venants, et l’oppresseur ne passera plus près d’eux ; car maintenant mes yeux sont fixés sur elle.
\TextTitle{Prophétie sur la première venue du Messie}
\VS{9}Sois transportée d’allégresse, fille de Sion ! Pousse des cris de joie, fille de Jérusalem ! Voici, ton Roi vient à toi ; il est juste et vainqueur, il est monté sur un âne, sur un âne, le petit d'une ânesse\FTNT{Cette prophétie s’est accomplie 500 ans après. Jésus est effectivement entré à Jérusalem monté sur un âne (Mt. 21:1-11 ; Lu. 19:28-40 ; Jn. 12:12-19).}.
\TextTitle{La vision du Messie pour Israël}
\VS{10}Je détruirai les chars d'Ephraïm, et les chevaux de Jérusalem ; et les arcs de guerre seront aussi retranchés.  Et le Roi parlera de paix aux nations ; et sa domination s'étendra d’une mer à l’autre, depuis le fleuve jusqu'aux extrémités de la terre\FTNT{Es. 57:19 ; Ps. 2:8 ; Ps. 72:8.}.
\VS{11}Quant à toi, à cause de ton alliance scellée par le sang, je retirerai tes captifs de la fosse où il n'y a pas d'eau.
\VS{12}Retournez à la forteresse, captifs pleins d’espérance ! Aujourd'hui même je le déclare, je te rendrai le double.
\VS{13}Car je bande Juda comme un arc, je m’arme d’Ephraïm comme d’un arc, et j’exciterai tes enfants, ô Sion, contre tes enfants, ô Javan ! Je te rendrai pareille à l’épée d’un vaillant homme.
\VS{14}Alors Yahweh au-dessus d’eux apparaîtra, et ses dards partiront comme l'éclair, et le Seigneur, Yahweh, sonnera du shofar, il s’avancera dans le tourbillon du midi.
\VS{15}Yahweh des armées sera leur protecteur ; ils dévoreront, après avoir subjugué ceux qui tirent les pierres de fronde ; ils boiront, ils seront bruyants comme des hommes ivres, ils se rempliront de vin comme un bassin, et comme les coins de l'autel.
\VS{16}Yahweh, leur Dieu, les sauvera en ce jour-là, comme le troupeau de son peuple ; car ils sont les pierres d’une couronne  qui brilleront dans son pays.
\VS{17}Car combien est grande sa bonté ! Quelle beauté ! Le froment fera croître les jeunes hommes, et le vin doux rendra ses vierges éloquentes.
\Chap{10}
\TextTitle{Yahweh rassemblera son peuple}
\VerseOne{}Demandez à Yahweh la pluie\FTNT{Les pluies en Israël : En Israël, la saison des pluies commence généralement vers la fin du mois d’octobre avec de légères pluies qui ramollissent la terre (Ps. 65:10), et se poursuit ensuite par de fortes précipitations intermittentes durant deux ou trois jours, tout au long des mois de novembre et de décembre. Ces fortes précipitations étaient appelées dans les écritures «~la pluie de la première saison~» (en hébreu «~yoreb~» ou «~moreh~). Les fermiers dépendaient de la pluie de la première saison pour que la terre dure comme le roc soit rendue apte au labour et à l’ensemencement. Quand ces fortes précipitations s’achèvent, des pluies plus fines continuent encore de façon intermittente. Toutefois, à l’approche de la moisson,  la forte pluie revenait gonfler le grain et le fruit en préparation. Celle-ci était connue comme étant «~la pluie de l’arrière-saison~» (Jé. 5:24; Joë. 2:23-24 ; Os. 6:3).}, la pluie au temps de la dernière saison ! Yahweh produira des éclairs, et il vous donnera une abondante pluie, il donnera à chacun de l'herbe dans son champ.
\VS{2}Car les théraphim ont des paroles vaines, et les devins prophétisent le mensonge, ils profèrent des songes vains et consolent par la vanité. C'est pourquoi ils sont errants comme des brebis, ils sont malheureux, parce qu'il n'y a point de pasteur\FTNT{Mt. 9:36 ; Ez. 34:2 ; Jé. 23:21-30.}.
\VS{3}Ma colère s'est enflammée contre ces pasteurs, et je châtierai ces boucs ; car Yahweh des armées visite son troupeau, la maison de Juda ; et il les a rangés en bataille comme son cheval d'honneur.
\VS{4}De lui sortira l’Angle\FTNT{De lui (Juda) sortira l’Angle ou la pierre angulaire (Jésus-Christ), (1 Pi. 2:7 ; Es. 8:13-17).}, de lui sortira le clou, de lui sortira l'arc de bataille, et de lui sortiront tous les chefs ensemble.
\VS{5}Ils seront comme des vaillants hommes foulant la boue des rues dans la bataille, et ils combattront, parce que Yahweh sera avec eux ; et les cavaliers seront confus.
\VS{6}Car je fortifierai la maison de Juda, et je sauverai la maison de Joseph ; je les ramènerai, et je les ferai habiter en repos, parce que j'aurai compassion d'eux, et ils seront comme si je ne les avais point rejetés ; car je suis Yahweh, leur Dieu, et je les exaucerai.
\VS{7}Et ceux d'Ephraïm seront comme un héros, et leur cœur se réjouira comme par le vin ; leurs fils le verront, et se réjouiront ; leur cœur se réjouira en Yahweh.
\VS{8}Je les sifflerai et les rassemblerai, car je les rachète ; et ils seront multipliés comme ils l'ont été auparavant.
\VS{9}Et après que je les aurai dispersés parmi les peuples, ils se souviendront de moi dans les pays éloignés, et ils vivront avec leurs enfants, et ils reviendront.
\VS{10}Ainsi je les ramènerai du pays d'Egypte, je les rassemblerai de l'Assyrie, je les ferai venir au pays de Galaad, et au Liban, et il n'y aura point assez d'espace pour eux.
\VS{11}Il passera la mer de détresse, et il frappera les flots de la mer ; et toutes les profondeurs du fleuve seront desséchées ; l'orgueil de l'Assyrie sera abattu, et le sceptre d'Egypte sera ôté.
\VS{12}Je les fortifierai en Yahweh, et ils marcheront en son Nom, dit Yahweh.
\Chap{11}
\TextTitle{Les houlettes du vrai berger}
\VerseOne{}Liban, ouvre tes portes, et que le feu dévore tes cèdres !
\VS{2}Cyprès, gémis, car le cèdre est tombé, parce que les choses magnifiques ont été ravagées ! Chênes de Basan, gémissez, car la forêt inaccessible est coupée !
\VS{3}Les pasteurs poussent des cris de lamentations, parce que leur magnificence est ravagée ; on entend le rugissement des lionceaux, parce que l'orgueil du Jourdain est abattu.
\VS{4}Ainsi parle Yahweh, mon Dieu : Pais les brebis exposées au carnage !
\VS{5}Leurs possesseurs les égorgent, sans qu'on les tienne pour coupables, et celui qui les vend dit : Béni soit Yahweh, car je m’enrichis !  Et leurs pasteurs ne les épargnent pas.
\VS{6}Car je n’ai plus de pitié pour les habitants du pays, dit Yahweh ; et voici, je livre les hommes aux mains les uns des autres et aux mains de leur roi ; ils ravageront le pays, et je ne le délivrerai pas de leur main.
\VS{7}Alors je me mis donc à paître les brebis exposées au carnage, qui sont véritablement les plus misérables du troupeau. Puis je pris deux verges : J'appelai l'une Grâce, et l'autre Cordon ; et je me mis à paître les brebis.
\VS{8}Et je supprimai les trois pasteurs en un mois ; car mon âme était impatiente à leur sujet, et leur âme aussi avait pour moi du dégoût.
\VS{9}Et je dis : Je ne vous paîtrai plus ; que celle qui va mourir meure, et que celle qui va périr périsse, et que celles qui restent se dévorent la chair les unes les autres.
\VS{10}Puis je pris ma verge, appelée Grâce, et je la brisai, pour rompre mon alliance que j'avais traitée avec tous ces peuples.
\VS{11}Elle fut rompue en ce jour-là ; et les plus malheureuses brebis\FTNT{Les plus malheureuses des brebis sont le reste d’Israël.}, qui prirent garde à moi, reconnurent ainsi que c'était la parole de Yahweh.
\VS{12}Je leur dis : S'il vous semble bon, donnez-moi mon salaire ; sinon, ne me le donnez pas.  Alors ils pesèrent\FTNT{Mt. 26:15 ; Mt. 27:9-10.}  pour mon salaire trente pièces d'argent\FTNT{Selon la loi de Moïse, pour racheter un mâle de 20 à 60 ans, ayant fait un vœu, il fallait payer cinquante sicles d'argent (Lé. 27:3). Pour dédommager un préjudice causé par un bœuf ayant frappé un esclave, on devait donner trente sicles d'argent au maître de l'esclave et lapider le bœuf (Ex. 21:32). Or le prix du Seigneur a été estimé à trente sicles d’argent, comme pour les esclaves.}.
\VS{13}Yahweh me dit : Jette-le au potier, ce prix honorable auquel ils m’ont estimé ! Alors je pris les trente pièces d'argent, et les jetai dans la maison de Yahweh, pour le potier.
\VS{14}Puis je brisai ma seconde verge, appelée Cordon, pour rompre la fraternité entre Juda et Israël.
\TextTitle{Caractéristiques du faux berger}
\VS{15}Yahweh me dit : Prends-toi encore l'équipage d'un berger insensé.
\VS{16}Car voici, je susciterai dans le pays un pasteur, qui ne visitera pas les brebis qui périssent ; il ne cherchera pas celles qui s’égarent, il ne guérira pas celles qui sont blessées, et il ne soutiendra pas celles qui sont saines, mais il dévorera la chair des plus grasses, et il déchirera jusqu’aux cornes de leurs pieds.
\VS{17}Malheur au pasteur inutile qui abandonne les brebis ! Que l’épée fonde sur son bras et sur son œil droit ! Que son bras se dessèche, et que son œil droit s’éteigne entièrement !
\Chap{12}
\TextTitle{Jérusalem, une coupe d'étourdissement pour les nations}
\VerseOne{}Oracle, parole de Yahweh, sur Israël. Ainsi parle Yahweh, qui a étendu les cieux et fondé la terre, et qui a formé l'esprit de l'homme au-dedans de lui :
\VS{2}Voici, je ferai de Jérusalem une coupe d'étourdissement pour tous les peuples d'alentour ; et aussi pour Juda dans le siège de Jérusalem\FTNT{Ap. 16:12-16.}.
\VS{3}En ce jour-là, je ferai de Jérusalem une pierre pesante pour tous les peuples ; tous ceux qui en porteront le poids seront entièrement écrasés, car toutes les nations de la terre s'assembleront contre elle.
\VS{4}En ce temps-là, dit Yahweh, je frapperai d'étourdissement tous les chevaux, et de folie ceux qui les monteront ; mais j’aurai les yeux ouverts sur la maison de Juda, et je frapperai d'aveuglement tous les chevaux des peuples.
\VS{5}Les chefs de Juda diront en leur cœur : Les habitants de Jérusalem sont notre force, par Yahweh des armées, leur Dieu.
\VS{6}En ce jour-là, je ferai des chefs de Juda comme un foyer de feu parmi du bois, et comme une torche enflammée parmi des gerbes ; ils dévoreront à droite et à gauche tous les peuples d'alentour ; et Jérusalem sera encore habitée à sa place, à Jérusalem.
\VS{7}Yahweh sauvera premièrement les tentes de Juda, afin que la gloire de la maison de David, la gloire des habitants de Jérusalem, ne s'élève point au-dessus de Juda.
\VS{8}En ce jour-là, Yahweh sera le protecteur des habitants de Jérusalem ; et le plus faible parmi eux sera en ce jour-là comme David ; la maison de David sera comme Dieu, comme l'Ange de Yahweh devant leur face.
\VS{9}En ce jour-là, je chercherai à détruire toutes les nations qui viendront contre Jérusalem.
\TextTitle{Repentance et délivrance d'Israël}
\VS{10}Et je répandrai sur la maison de David, et sur les habitants de Jérusalem, l'Esprit de grâce et de supplications\FTNT{Joë. 2:28-30.}, et ils regarderont vers moi\FTNT{Au retour du Messie, il y aura une repentance et une conversion nationale d’Israël (Ro. 11:26).}, celui qu’ils ont percé, et ils pleureront sur lui\FTNT{Celui qu’ils ont percé : Il est question ici du Seigneur Jésus, le Messie (Ap. 1:7).}, comme on pleure sur un fils unique, et ils pleureront amèrement sur lui, comme quand on pleure sur un premier-né.
\VS{11}En ce jour-là, il y aura un grand deuil à Jérusalem, comme le deuil d'Hadadrimmon dans la vallée de Meguiddon.
\VS{12}Le pays sera dans le deuil, chaque famille à part : La famille de la maison de David à part, et les femmes de cette maison-là à part ; la famille de la maison de Nathan à part, et les femmes de cette maison-là à part.
\VS{13}La famille de la maison de Lévi à part, et les femmes de cette maison-là à part ; la famille de Schimeï à part, et ses femmes à part.
\VS{14}Toutes les autres  familles, chaque famille à part, et leurs femmes à part.
\Chap{13}
\TextTitle{Dieu frappe les faux prophètes}
\VerseOne{}En ce jour-là, il y aura une source ouverte en faveur de la maison de David et des habitants de Jérusalem, pour le péché et pour la souillure.
\VS{2}En ce jour-là, dit Yahweh des armées, je retrancherai du pays les noms des faux dieux, et on n'en fera plus mention. J'ôterai aussi du pays les faux prophètes et l'esprit d'impureté.
\VS{3}Et il arrivera que si quelqu'un prophétise encore, son père et sa mère qui l’ont engendré, lui diront : Tu ne vivras plus ; car tu as prononcé des mensonges au Nom de Yahweh ; et son père et sa mère qui l’ont engendré, le transperceront quand il prophétisera.
\VS{4}En ce jour-là, les prophètes seront confus de leurs visions, quand ils prophétiseront ; et ils ne revêtiront plus un manteau de poil pour mentir.
\VS{5}Chacun d’eux dira : Je ne suis pas prophète, mais je suis laboureur, car on m'a appris à gouverner du bétail dès ma jeunesse.
\VS{6}Et si on lui demande : Que veulent donc dire ces blessures que tu as aux mains ? Et il répondra : C’est dans la maison de mes amis qu’on me les a faites.
\TextTitle{Prophétie sur le vrai berger, le Messie}
\VS{7}Epée, réveille-toi contre mon Berger\FTNT{Mon Berger : Il est question de Jésus-Christ, le Bon Berger (Ps. 23 ; Jn. 10:1-17).}, et sur l'homme qui est mon compagnon ! dit Yahweh des armées frappe le Berger, et les brebis seront dispersées\FTNT{Frappe le Berger : Cette prophétie fait référence à la crucifixion du Seigneur Jésus-Christ (Ge. 3:15 ; Mt. 26:31 ; Mc. 14:27 ; Mc. 14:50 ; Mc. 15:19).} ; et je tournerai ma main vers les faibles.
\TextTitle{Le reste de Yahweh épuré à travers l'épreuve}
\VS{8}Dans tout le pays, dit Yahweh, les deux tiers seront retranchées et périront, et l’autre tiers restera.
\VS{9}Je mettrai ce tiers dans le feu, et je le purifierai comme on purifie l'argent, je les éprouverai comme on éprouve l'or. Il invoquera mon Nom, et je l'exaucerai ; je dirai : C'est ici mon peuple ! Et il dira : Yahweh est mon Dieu\FTNT{1 Pi. 1:6-7 ; Ps. 50:15 ; Ps. 91:15 ; Ps. 144:15.} !
\Chap{14}
\TextTitle{Imminence du jour de Yahweh}
\VerseOne{}Voici, le jour de Yahweh\FTNT{L’expression «~le jour du Seigneur~» ou «~le jour de Yahweh~» est utilisée dix-neuf fois dans le Tanakh (Es. 2:12 ;  Es. 13:6 ; Es. 13:9 ; Ez. 13:5 ; Ez. 30:3 ; Joë. 1:15 ; Joë. 2:1 ; Joë. 2:11 ; Joë. 2:31 ;  Joë. 3:14 ; Am. 5:18-20 ; Ab. 1:15 ; So. 1:7 ; So. 1:14 ; Za. 14:1 ; Mal. 4:5) et quatre fois dans les textes de la nouvelle alliance (Ac. 2:20 ; 2 Th. 2:2 ; 2 Pi. 3:10 ; Ap. 6:17 ; Ap. 16:14). Cette expression désigne habituellement des événements qui se déroulent à la fin des temps (Es. 7:18-25). Elle désigne un espace de temps au cours duquel Dieu va intervenir personnellement dans l’histoire des hommes. Appelé  «~jour de colère~», «~jour de  visitation~», et «~grand jour du Dieu Tout-Puissant~» ; il se réfère ainsi à un accomplissement encore futur, quand la colère de Dieu viendra s’abattre sur l’Israël qui n’aura pas cru (Es. 22 ; Jé. 30:1-17 ; Joë. 1 et 2 ; Am. 5 ; So. 1) et sur tous les incrédules du monde (Ez. 38 et 39 ; Za. 14). Ce jour sera aussi un temps de salut puisque Dieu va délivrer «~le reste~» d’Israël, accomplissant ainsi sa promesse selon laquelle «~tout Israël sera sauvé~» (Ro. 11:26) : il pardonnera leurs péchés et restaurera le peuple qu’Il s’est choisi sur la terre promise à Abraham (Es. 10:27 ; Jé. 30:19-31 ; Mi. 4 ; Za. 13).} arrive, et tes dépouilles seront partagées au milieu de toi, Jérusalem.
\VS{2}Je rassemblerai toutes les nations à Jérusalem pour qu’elles lui fassent la guerre\FTNT{Joë. 3 ; Ap. 16:12-16.} ; la ville sera prise, les maisons pillées, et les femmes violées ; la moitié de la ville ira en captivité, mais le reste du peuple ne sera pas retranché de la ville.
\VS{3}Yahweh sortira, et il combattra contre ces nations, comme il a combattu au jour de la bataille.
\TextTitle{Retour visible et en gloire du Seigneur}
\VS{4}Ses pieds se poseront en ce jour sur la Montagne des Oliviers\FTNT{Ce sont les pieds de Jésus-Christ (Ac. 1:10-11).}, qui est vis-à-vis de Jérusalem, du côté de l’orient ; et la Montagne des Oliviers se fendra par le milieu, à l'orient et à l'occident, de sorte qu'il y aura une très grande vallée ; une moitié de la montagne reculera vers le nord, et l'autre moitié vers le midi.
\VS{5}Vous fuirez alors dans la vallée de mes montagnes ; car la vallée des montagnes s’étendra jusqu'à Atzel ; et vous fuirez comme vous avez fui devant le tremblement de terre, aux jours d’Ozias, roi de Juda. Alors Yahweh, mon Dieu, viendra, et tous les saints seront avec lui\FTNT{Ce passage confirme clairement que Jésus-Christ est Yahweh (1 Th. 3:13 ; Jud. 14-15 ; Es. 34:5 ; Es. 40:10-11 ; Es. 62:11-15).}.
\VS{6}Et il arrivera qu'en ce jour-là, la lumière précieuse ne sera pas mêlée de ténèbres.
\VS{7}Ce sera un jour unique, connu de Yahweh, et qui ne sera ni jour ni nuit ; mais au temps du soir il y aura de la lumière.
\VS{8}Et il arrivera qu'en ce jour-là, des eaux vives\FTNT{Ez. 47:1-12 ;  Ap. 22:1-2.} sortiront de Jérusalem, la moitié d'elles coulera vers la mer orientale, et l'autre moitié, vers la mer occidentale ; il en sera ainsi été et hiver.
\TextTitle{Le royaume messianique}
\VS{9}Yahweh sera Roi sur toute la terre ; en ce jour-là, Yahweh sera Un, et son nom sera Un\FTNT{Littéralement «~E’had~». Le jour du Seigneur est un comme le jour un de Ge. 1:5. Yahweh est Un et non trois (De. 6:4). Son Nom est Un (Ac. 4:12).}.
\VS{10}Toute la terre deviendra comme la plaine, depuis Guéba jusqu'à Rimmon, au midi de Jérusalem ; et Jérusalem sera exaltée et restera à sa place, depuis la porte de Benjamin, jusqu'à l'endroit de la première porte, jusqu'à la porte des angles, et depuis la tour de Hananeel, jusqu'aux pressoirs du roi.
\VS{11}On habitera dans son sein, et il n'y aura plus d'interdit, mais Jérusalem sera habitée en sûreté.
\VS{12}Voici la plaie dont Yahweh frappera tous les peuples qui auront fait la guerre contre Jérusalem ; il fera que la chair de chacun tombera en pourriture tandis qu’ils seront sur leurs pieds, leurs yeux tomberont en pourriture dans leurs orbites, et leur langue tombera en pourriture dans leur bouche.
\VS{13}Et il arrivera en ce jour-là que Yahweh produira un grand trouble parmi eux ; car chacun saisira la main de son prochain, et la main de l'un s'élèvera contre la main de l'autre.
\VS{14}Juda combattra aussi dans Jérusalem, et les richesses de toutes les nations d'alentour y seront amassées : L'or, l'argent, et des vêtements en très grand nombre.
\VS{15}Et la même plaie sera sur les chevaux, les mulets, les chameaux, les ânes et sur toutes les bêtes qui seront dans ces camps, cette plaie sera semblable à l’autre.
\TextTitle{Adoration de Yahweh des armées dans le royaume}
\VS{16}Et il arrivera que tous ceux qui resteront de toutes les nations venues contre Jérusalem, monteront en foule chaque année pour adorer le Roi, Yahweh des armées, et pour célébrer la fête des tabernacles.
\VS{17}S’il y a des familles de la terre qui ne montent pas à Jérusalem, pour adorer le Roi, Yahweh des armées, la pluie ne tombera pas sur elles.
\VS{18}Si la famille d'Egypte ne monte pas, si elle ne vient pas, la pluie ne tombera pas sur elle ; elle sera frappée de la plaie dont Yahweh frappera les nations qui ne monteront pas pour célébrer la fête des tabernacles.
\VS{19}Ce sera la peine du péché de l’Egypte, et du péché de toutes les nations qui ne monteront pas pour célébrer la fête des tabernacles.
\VS{20}En ce jour-là, il sera écrit sur les clochettes des chevaux : Sainteté à Yahweh ! Et les chaudières dans la maison de Yahweh seront comme les coupes devant l'autel.
\VS{21}Toute chaudière qui sera à Jérusalem et dans Juda, sera consacrée à Yahweh des armées ; et tous ceux qui offriront des sacrifices viendront, et s’en serviront pour cuire  les viandes ; et il n'y aura plus de marchands dans la maison de Yahweh des armées, en ce jour-là.
\PPE{}
\end{multicols}

%\clearpage\ShortTitle{Malachie}\BookTitle{Malachie}\BFont
\noindent\hrulefill
\textit{
\bigskip
{\centering{}
\\(Malakhi)
\\Signifie : Mon messager, mon ange
\\Thème : Message final de l'ancienne alliance à une nation désobéissante
\\Auteur : Malachie
\\Date de rédaction : Vème siècle av. J.-C.\\}
}
%\bigskip
\textit{
\\Dernier prophète de l’ancienne alliance, Malachie exerça son ministère en Juda après la reconstruction du temple et la reprise des cultes. Il annonça la venue du Messie et du messager qui devait le précéder, le nouvel Elie que Jésus-Christ reconnut en Jean-Baptiste. Ses écrits mettent en évidence l’importance de l’obéissance à la loi de Yahweh et la justice divine.
\bigskip
\\message. 
\bigskip
\\message.\bigskip
}
\par\nobreak\noindent\hrulefill
\begin{multicols}{2}
\TextTitle{[L'amour de Yahweh pour son peuple]}
\Chap{1}
\VerseOne{}Oracle, parole de Yahweh contre Israël, par le moyen de Malachie.
\VS{2}Je vous ai aimés, dit Yahweh ; et vous dites : En quoi nous as-tu aimés ? Esaü n'était-il pas frère de Jacob ? dit Yahweh. Or j'ai aimé Jacob,
\VS{3}Mais j'ai eu de la haine pour Esaü, et j'ai fait de ses montagnes une solitude, j’ai livré son héritage aux chacals du désert.
\VS{4}Si Edom dit : Nous sommes détruits, nous rebâtirons les lieux ruinés ! Ainsi parle Yahweh des armées : Ils rebâtiront, mais je détruirai, et on les appellera pays de méchanceté, peuple contre lequel Yahweh est irrité pour toujours.
\VS{5}Vos yeux le verront, et vous direz : Yahweh est grand par-delà les frontières d'Israël !
\TextTitle{[Le péché des sacrificateurs après le retour d'exil]}
\VS{6}Un fils honore son père, et un serviteur son maître. Si donc je suis Père, où est l'honneur qui m'appartient ? Si je suis maître, où est la crainte qu'on a de moi ? dit Yahweh des armées, à vous sacrificateurs, qui méprisez mon Nom, et qui dites : En quoi avons-nous méprisé ton Nom ?
\VS{7}Vous offrez sur mon autel des aliments souillés, et vous dites : En quoi t'avons-nous profané ? C'est en disant : La table de Yahweh est méprisable !
\VS{8}Et quand vous amenez une bête aveugle pour la sacrifier, n'y a-t-il point de mal en cela ? Quand vous en offrez une boiteuse ou malade, n’est-ce pas mal ? Offre-la à ton gouverneur ! T’agréera-t-il, te recevra-t-il favorablement ? dit Yahweh des armées.
\VS{9}Maintenant, donc suppliez Dieu, pour qu'il ait pitié de nous ! Cela vient de vos mains : Vous recevra-t-il favorablement ? dit Yahweh des armées.
\VS{10}Lequel de vous fermera les portes pour que vous n’allumiez pas en vain le feu sur mon autel ? Je ne prends aucun plaisir en vous, dit Yahweh des armées, et je n’agrée pas l'offrande de vos mains.
\VS{11}Car depuis le soleil levant jusqu'au soleil couchant, mon Nom est grand parmi les nations, et en tous lieux on brûle de l’encens en l'honneur de mon Nom, et des offrandes pures ; car mon Nom est grand parmi les nations, dit Yahweh des armées.
\VS{12}Mais vous, vous le profanez, en disant : La table de Yahweh est souillée, et ce qu’elle rapporte est un aliment méprisable.
\VS{13}Vous dites aussi : Quelle fatigue ! Et vous le dédaignez, dit Yahweh des armées ; vous amenez ce qui a été dérobé, ce qui est boiteux, et malade, ce sont là les offrandes que vous faites ! Accepterai-je cela de vos mains ? dit Yahweh.
\VS{14}C'est pourquoi, maudit soit l'homme trompeur, qui a dans son troupeau un mâle, et qui voue et sacrifie à Yahweh ce qui est corrompu ! Car je suis un grand roi, dit Yahweh des armées, et mon Nom est redoutable parmi les nations.
\TextTitle{[Mise en garde de Yahweh aux sacrificateurs]}
\Chap{2}
\VerseOne{}Maintenant, c’est à vous, sacrificateurs que s'adresse ce commandement :
\VS{2}Si vous n'écoutez pas, et que vous ne preniez point pas à cœur de donner gloire à mon Nom, dit Yahweh des armées, j'enverrai sur vous la malédiction, et je maudirai vos bénédictions ; et déjà même je les ai maudites, parce que vous ne prenez pas cela à cœur.
\VS{3}Voici, je vais détruire vos semences, et je répandrai les excréments de vos victimes sur vos visages, les excréments, dis-je, de vos solennités, et on vous emportera avec eux.
\VS{4}Alors vous saurez que je vous ai adressé ce commandement, afin que mon alliance avec Lévi subsiste, dit Yahweh des armées.
\VS{5}Mon alliance avec lui était la vie et la paix, c’est ce que je lui accordai pour qu’il me craigne ; il a eu pour moi de la crainte, et il a tremblé devant mon Nom.
\VS{6}La loi de la vérité était dans sa bouche, et il ne s'est point trouvé de perversité sur ses lèvres ; il a marché avec moi dans la paix et dans la droiture, et il en a détourné beaucoup de l'iniquité.
\VS{7}Car les lèvres du sacrificateur doivent garder la science, et c’est de sa bouche qu’on demande la loi, parce qu'il est un messager de Yahweh des armées.
\VS{8}Mais vous, vous vous êtes écartés de la voie, vous avez fait de la loi une occasion de chute pour beaucoup, et vous avez corrompu l'alliance de Lévi, dit Yahweh des armées.
\TextTitle{[Infidélités envers des frères et envers Yahweh]}
\VS{9}C'est pourquoi je vous rendrai méprisables et abjects aux yeux de tout le peuple. Parce que vous n’avez pas gardé mes voies, et vous avez égard à l'apparence des personnes quand vous enseignez la loi.
\VS{10}N'avons-nous pas tous un seul Père ? N’est-ce pas un seul Dieu qui nous a créés ? Pourquoi donc agissons-nous avec perfidie l’un avec l’autre, en violant l'alliance de nos pères ?
\VS{11}Juda s’est montré infidèle, et une abomination a été commise en Israël et à Jérusalem ; car Juda a profané ce qui est consacré à Yahweh, ce qu’il aime, il s'est marié à la fille d'un dieu étranger.
\VS{12}Yahweh retranchera l’homme qui fait cela, celui qui veille et qui répond, il le retranchera des tentes de Jacob, et il retranchera celui qui présente une offrande à Yahweh des armées.
\VS{13}Voici une autre chose que vous faites : Vous couvrez l'autel de Yahweh de larmes, de plaintes et de gémissements, en sorte qu’il n’a plus égard aux offrandes et qu’il ne peut rien agréer de vos mains.
\VS{14}Et vous dites : Pourquoi ?... C'est parce que Yahweh est intervenu comme témoin entre toi et la femme de ta jeunesse, envers laquelle tu as été infidèle, bien qu’elle soit ta compagne et la femme de ton alliance.
\VS{15}Nul n’a fait cela, avec un reste de bon esprit. Un seul l’a fait, et pourquoi ? Parce qu'il cherchait une postérité de Dieu. Prenez donc garde en votre esprit, et qu’aucun ne soit infidèle à la femme de sa jeunesse !
\VS{16}Car je hais la répudiation, dit Yahweh, le Dieu d'Israël, et celui qui couvre de violence son vêtement, dit Yahweh des armées. Prenez donc garde en votre esprit, et ne soyez pas infidèles !
\TextTitle{[Fausse profession religieuse]}
\VS{17}Vous fatiguez Yahweh par vos paroles, et vous dites : En quoi l'avons-nous fatigué ? C'est quand vous dites : Quiconque fait le mal plaît à Yahweh, et il prend plaisir à de tels gens ! Autrement : Où est le Dieu du jugement ?
\TextTitle{[Venue du précurseur du messie]}
\Chap{3}
\VerseOne{}Voici, j'enverrai mon messager\FTNT{Ce messager, ou Elie le prophète, est Jean-Baptiste (Es. 40:1-3 ; Mal. 3:1 ; Mt. 3:1-15 ; Mt. 11:14 ; Mt. 17:10-13 ; Mc. 1:1-11 ; Mc. 9:11-13 ; Lu. 1:17 ; Lu. 3:1-5).} ; il préparera le chemin devant moi. Et soudain entrera dans son temple le Seigneur que vous cherchez ; l’ange de l'alliance\FTNT{L’ange de l’alliance est le Seigneur Jésus-Christ. Voir aussi commentaire en Mt. 1:20}, que vous désirez, voici, il vient, dit Yahweh des armées.
\VS{2}Mais qui pourra soutenir le jour de sa venue ? Qui pourra subsister quand il paraîtra ? Car il sera comme le feu du fondeur et comme la potasse des foulons.
\VS{3}Et il sera assis comme celui qui raffine et purifie l'argent ; il nettoiera les fils de Lévi, il les épurera comme l’or et l'argent, et ils présenteront à Yahweh des offrandes avec justice.
\VS{4}Alors l’offrande de Juda et de Jérusalem sera agréable à Yahweh, comme aux anciens jours, comme aux années d'autrefois.
\VS{5}Je m'approcherai de vous pour le jugement, et je me hâterai de témoigner contre les enchanteurs et les adultères, contre ceux qui jurent faussement, et contre ceux qui retiennent le salaire du mercenaire, qui oppriment la veuve et l'orphelin, qui font tort à l'étranger, et qui ne me craignent point, dit Yahweh des armées.
\VS{6}Parce que je suis Yahweh et que je n'ai point changé ; à cause de cela, enfants de Jacob, vous n'avez point été consumés.
\TextTitle{[Le peuple infidèle qui vole Yahweh]}
\VS{7}Depuis le temps de vos pères, vous vous êtes écartés de mes ordonnances, vous ne les avez point observées. Revenez à moi, et je reviendrai à vous, dit Yahweh des armées. Et vous dites : En quoi nous convertirons-nous ?
\VS{8}L'homme pillera-t-il Dieu, que vous me pilliez ? Et vous dites : En quoi t'avons-nous pillé ? Vous l'avez fait dans les dîmes et dans les offrandes.
\VS{9}Vous êtes certainement maudits, parce que vous me pillez, vous, toute la nation !
\VS{10}Apportez toutes les dîmes\FTNT{Il est question ici de la dîme de la dîme que les levites donnaient aux sacrificateurs. Cette dîme était rapportée aux magasins, ou greniers (Né. 10:35-39), là aussi était stocké toute sorte de trésor. Pour les autres dîmes, voir le commentaire dans Dt. 14:22-29.} aux magasins, afin qu'il y ait provision dans ma maison ; et dès maintenant éprouvez moi en cela, a dit Yahweh des armées, si je ne vous ouvre pas les écluses des cieux, et si je ne répands pas en votre faveur la bénédiction, jusqu'à ce qu'il n'y ait plus assez de place.
\VS{11}Et je réprimerai pour l'amour de vous le dévorateur, et il ne vous ravagera pas les fruits de la terre, et vos vignes ne seront pas stériles dans vos campagnes, a dit Yahweh des armées.
\VS{12}Toutes les nations vous diront heureux, car vous serez un pays de délices, dit Yahweh des armées.
\VS{13}Vos paroles sont rudes contre moi, a dit Yahweh. Et vous dites : Qu'avons-nous donc dit contre toi ?
\VS{14}Vous avez dit : C'est en vain que l’on sert Dieu ; et qu'avons-nous gagné à observer ses ordonnances, et à marcher en pauvre état pour l'amour de Yahweh des armées ?
\VS{15}Et maintenant nous tenons pour heureux les orgueilleux ; et même ceux qui commettent la mechanceté, sont avancés ; et s'ils ont tenté Dieu, ils ont été délivrés !
\TextTitle{[Le "reste d'Israël" demeure fidèle à Yahweh"]}
\VS{16}Alors ceux qui craignent Yahweh se parlèrent l'un à l’autre ; et Yahweh fut attentif, et il écouta ; et un livre de souvenir fut écrit devant lui pour ceux qui craignent Yahweh et qui pensent à son Nom.
\VS{17}Ils seront à moi, a dit Yahweh des armées, le jour où je mettrai à part mes plus précieux joyaux, et je leur pardonnerai comme un homme pardonne à son fils qui le sert.
\VS{18}Convertissez-vous donc, et vous verrez la différence qu'il y a entre le juste et le méchant, entre celui qui sert Dieu et celui qui ne le sert pas.
\TextTitle{[Avènement du jour de Yahweh]}
\Chap{4}
\VerseOne{}Car voici, le jour vient, ardent comme une fournaise. Tous les orgueilleux et tous les méchants seront comme du chaume ; et ce jour qui vient, dit Yahweh des armées, les embrasera, il ne leur laissera ni racine ni rameau.
\VS{2}Mais pour vous qui craignez mon Nom, se lèvera le Soleil de justice\FTNT{Le Soleil de justice : Jésus-Christ est notre Soleil (Lu. 1:78-79). Cet aspect de Jésus-Christ nous parle de la grâce de Dieu : « il fait lever son soleil sur les méchants, et sur les gens de bien » (Mt. 5:45). Le soleil évoque aussi le jugement de Dieu. Ainsi, en plein midi, il est le feu de la justice et de la colère de Dieu. Son pardon et son amour pour nous sont alors comparés à une ombre fraîche qui nous sauve de sa chaleur ardente (Ps. 121 ; Es. 25:4). Dans Ps. 19:6, le soleil est comparé à un époux. Or Jésus-Christ est notre époux et le soleil qui nous apporte la guérison.}, et la guérison sera sous ses ailes ; vous sortirez, et bondirez comme les veaux d’une étable.
\VS{3}Et vous foulerez les méchants, car ils seront comme de la cendre sous les plantes de vos pieds, au jour où je ferai mon œuvre, dit Yahweh des armées.
\VS{4}Souvenez-vous de la loi de Moïse, mon serviteur, auquel j’ai prescrit en Horeb, pour tout Israël, des statuts et des ordonnances.
\TextTitle{[Retour d'Elie avant le jour de Yahweh]}
\VS{5}Voici, je vous enverrai Elie, le prophète\FTNT{Voir commentaire en Mal. 3:1.}, avant que le jour grand et redoutable de Yahweh vienne.
\VS{6}Il ramènera le cœur des pères à leurs enfants, et le cœur des enfants à leurs pères, de peur que je ne vienne et que je ne frappe la terre d’interdit.
\PPE{}
\end{multicols}

%\addcontentsline{toc}{section}{Ketouvim (Écrits)}\clearpage
%\clearpage\ShortTitle{Psaumes}\BookTitle{Psaumes}\BFont
\noindent\hrulefill
{\footnotesize
\textit{
\bigskip
{\centering{}
\\Auteurs : David essentiellement et d'autres écrivains
\\(Heb. : Tehilim)
\\Signification : Louanges 
\\Thème : La louange et l'adoration
\\Date de rédaction : A compter du 10\up{ème} siècle et au-delà\\}
}
%\bigskip
\textit{
\\Le terme « psaume » désigne un poème chanté avec l'accompagnement d'un instrument. C'est ainsi que furent initialement contés les récits de la création divine, la captivité ou encore la gloire de Jérusalem. Expressions de joie, de reconnaissance, de repentance, d'angoisse ou de vulnérabilité de l'homme, ces hymnes étaient des prières adressées à Dieu.
%\bigskip
\\Prophétiques, certains psaumes annoncent les événements de la fin des temps, notamment les souffrances de Christ. Utilisé comme recueil de chants, le livre des Psaumes exalte la grandeur de Dieu, sa souveraineté, sa miséricorde et son omniscience. Il est le fruit d'une grande variété d'expériences spirituelles du fait de la diversité de ses auteurs. De plus, il contient une richesse de styles considérable, ce qui en fait le chef-d'œuvre de la poésie hébraïque.\bigskip
}
}
\par\nobreak\noindent\hrulefill
\begin{multicols}{2}
\Chap{1}
\TextTitle{La voie du juste et du pécheur}
\VerseOne{}Heureux l'homme qui ne marche pas selon le conseil des méchants, et qui ne s'arrête pas sur la voie des pécheurs, qui ne s'assied pas dans l'assemblée des moqueurs\FTNT{Jé. 15:17 ; 1 Co. 15 : 33 ; Ep. 5:11.},
\VS{2}mais qui prend plaisir dans la loi de Yahweh, et qui médite sa loi jour et nuit\FTNT{De. 6:6 ; De. 17:19 ; Jos. 1:8.}.
\VS{3}Il est comme un arbre planté près des ruisseaux d'eaux, qui rend son fruit en sa saison, et dont le feuillage ne se flétrit point\FTNT{Jé. 17:7-8 ; Ez. 47:12 ; Jn. 15:8 ; Ap. 22:2.}. Et ainsi tout ce qu'il fera réussira.
\VS{4}Il n'en est pas ainsi des méchants : Ils sont comme la balle que le vent chasse au loin\FTNT{Job. 21:17-18 ; Os. 13:3.}.
\VS{5}C'est pourquoi les méchants ne résistent pas dans le jugement, ni les pécheurs dans l'assemblée des justes.
\VS{6}Car Yahweh connaît la voie des justes, mais la voie des méchants périra.
\Chap{2}
\TextTitle{Complot des nations contre le Messie}
\VerseOne{}Pourquoi cette agitation parmi les nations, et pourquoi les peuples projettent-ils des choses vaines ?
\VS{2}Pourquoi les rois de la terre se lèvent-ils en personne, et les princes se liguent-ils avec eux contre Yahweh, et contre son Messie\FTNT{Cette prophétie concerne le complot des Juifs, de Pilate et d'Hérode contre Jésus-Christ, notre Seigneur. Il est également question du gouvernement mondial dirigé par Satan. Mt. 12:14 ; Mt. 26:3-4 ; Mt. 26:59-66. ; Mt. 27:1-2 ; Mc. 3:6 ; Mc. 11:18 ; Ac. 4:23-29.}?
\VS{3}Rompons leurs liens et jetons loin de nous leurs cordes!
\VS{4}Celui qui habite dans les cieux se rit d'eux, le Seigneur se moque d'eux.
\VS{5}Il leur parle dans sa colère, et il les remplira de terreur par la grandeur de son courroux\FTNT{Pr. 1:26.}:
\VS{6}C'est moi qui ai consacré mon Roi sur Sion, la montagne de ma sainteté\FTNT{Mi. 4:7.}!
\VS{7}Je vous réciterai cette ordonnance ; Yahweh m'a dit : Tu es mon Fils ! Je t'ai engendré aujourd'hui\FTNT{Ac. 13:33 ; Hé. 1:5 ; Hé. 5:5.}.
\VS{8}Demande-moi, et je te donnerai les nations pour héritage, et les extrémités de la terre pour possession.
\VS{9}Tu les briseras avec un sceptre de fer et tu les mettras en pièces comme un vase de potier\FTNT{Da. 2:44 ; Ap. 2:27.}.
\VS{10}Maintenant donc, rois, ayez de l'intelligence ! Juges de la terre, recevez instruction !
\VS{11}Servez Yahweh avec crainte, et réjouissez-vous avec tremblement\FTNT{Ps. 19:10.}.
\VS{12}Embrassez le Fils, de peur qu'il ne s'irrite et que vous ne périssiez dans cette conduite, quand sa colère s'embrasera promptement. Heureux sont tous ceux qui se confient en lui !
\Chap{3}
\TextTitle{Yahweh, le véritable secours}
\VerseOne{}Psaume de David au sujet de sa fuite devant Absalom, son fils.
\VS{2}Ô Yahweh, que mes adversaires sont nombreux ! Beaucoup de gens se lèvent contre moi!
\VS{3}Plusieurs disent à mon âme : Plus de salut pour lui auprès de Dieu! Sélah\FTNT{Le mot hébreu « Sélah » signifie « élever, exalter ». Il peut aussi traduire une pause dans le cantique ou le texte. C'est sûrement un terme technique musical montrant probablement une accentuation, une pause, une interruption.}.
\VS{4}Mais toi, ô Yahweh ! Tu es un bouclier autour de moi, tu es ma gloire, et tu relèves ma tête.
\VS{5}De ma voix je crie à Yahweh, et il me répond de sa sainte montagne. Sélah.
\VS{6}Je me couche, je m'endors, je me réveille, car Yahweh me soutient\FTNT{Lé. 26:6.}.
\VS{7}Je ne crains pas les myriades de peuples quand ils se rangent contre moi de toutes parts.
\VS{8}Lève-toi, Yahweh, mon Dieu ! Délivre-moi ! Car tu frappes à la joue tous mes ennemis, tu brises les dents des méchants.
\VS{9}La délivrance vient de Yahweh\FTNT{Es. 43:11 ; Jé. 3:23 ; Pr. 21:31 ; Ap. 7:10.} ! Que ta bénédiction soit sur ton peuple! Sélah.
\Chap{4}
\TextTitle{Yahweh, la joie et la paix du juste}
\VerseOne{}Psaume de David, donné au chef des chantres pour le chanter sur Neguinoth.
\VS{2}Ô Dieu de ma justice, puisque je crie, réponds-moi ! Quand j'étais à l'étroit, tu m'as mis au large ! Aie pitié de moi, et exauce ma prière\FTNT{Ps. 28:1-2.} !
\VS{3}Fils des hommes, jusqu'à quand ma gloire sera-t-elle diffamée ? Jusqu'à quand aimerez-vous la vanité et chercherez-vous le mensonge ? Sélah.
\VS{4}Or sachez que Yahweh s'est choisi un bien-aimé. Yahweh m'exauce quand je crie à lui\FTNT{1 Jn. 5:14.}.
\VS{5}Tremblez et ne péchez point ; parlez en vos cœurs sur votre couche et taisez-vous. Sélah.
\VS{6}Offrez des sacrifices de justice\FTNT{Ps. 51:19.} et confiez-vous en Yahweh.
\VS{7}Plusieurs disent : Qui nous fera voir le bonheur ? Lève sur nous la lumière de ta face, ô Yahweh !
\VS{8}Tu mets plus de joie dans mon cœur qu'ils n'en ont, quand abondent leur froment et leur vin.
\VS{9}Je me couche et je m'endors en paix, car toi seul, ô Yahweh ! Tu me fais reposer en sécurité\FTNT{Pr. 3:24.}.
\Chap{5}
\TextTitle{Recours à la protection de Yahweh}
\VerseOne{}Psaume de David, donné au chef des chantres, pour le chanter sur Nehiloth.
\VS{2}Yahweh, prête l'oreille à mes paroles ! Ecoute ma méditation !
\VS{3}Mon Roi et mon Dieu ! Sois attentif à la voix de mon cri ; car c'est à toi que j'adresse ma requête.
\VS{4}Yahweh, le matin tu entends ma voix, dès le matin je me tourne vers toi, et je veille.
\VS{5}Car tu n'es point un Dieu qui prenne plaisir au mal ; le méchant n'a point sa demeure auprès de toi.
\VS{6}Les orgueilleux ne subsistent pas devant tes yeux ; tu hais tous ceux qui commettent l'iniquité\FTNT{Ps. 1:5 ; Ha. 1:13.}.
\VS{7}Tu fais périr les menteurs ; Yahweh a en abomination l'homme sanguinaire et le trompeur.
\VS{8}Mais moi, comblé de tes bienfaits, j'entrerai dans ta maison, je me prosternerai dans le palais de ta sainteté avec les sentiments d'une crainte respectueuse.
\VS{9}Yahweh, conduis-moi dans ta justice, à cause de mes ennemis, aplanis ta voie sous mes pas\FTNT{Ps. 25:4-5 ; Ps. 27:11.}.
\VS{10}Car il n'y a rien de droit dans leur bouche; leur cœur est rempli de malice, leur gosier est un sépulcre ouvert, ils flattent de leur langue\FTNT{Ps. 10:7 ; Ps. 12:3 ; Ro. 3:13.}.
\VS{11}Ô Dieu ! Fais-leur leur procès, qu'ils échouent dans leurs entreprises ! Chasse-les au loin, à cause du grand nombre de leurs transgressions ! Car ils se sont rebellés contre toi.
\VS{12}Mais que tous ceux qui se confient en toi se réjouiront, qu'ils soient dans la joie perpétuellement, et que tu sois leur protecteur ; et que ceux qui aiment ton Nom s'égayent en toi !
\VS{13}Car tu bénis le juste, ô Yahweh ! Et tu l'entoures de ta bienveillance comme d'un bouclier.
\Chap{6}
\TextTitle{La miséricorde de Yahweh}
\VerseOne{}Psaume de David, donné au chef des chantres, pour le chanter pour le chanter en Neguinoth, sur Sheminith.
\VS{2}Yahweh, ne me punis pas dans ta colère et ne me châtie pas dans ta fureur\FTNT{Jé. 10:24.}.
\VS{3}Yahweh, aie pitié de moi ! Car je suis sans aucune force. Guéris-moi, ô Yahweh ! Car mes os sont épouvantés.
\VS{4}Même mon âme est fort troublée ; et toi, ô Yahweh ! Jusqu'à quand ?
\VS{5}Reviens, Yahweh ! Délivre mon âme. Sauve-moi, à cause de ta miséricorde.
\VS{6}Car celui qui meurt n'a plus ton souvenir ; qui te célébrera dans le scheol\FTNT{Es. 38 : 18 ; Ps. 88 : 11 ; Ps. 115 : 17.} ?
\VS{7}Je m'épuise à force de gémir; chaque nuit ma couche est baignée de mes larmes\FTNT{Job 7 : 3-4.}, mon lit est arrosé de mes pleurs.
\VS{8}J'ai le visage usé par le chagrin\FTNT{Ps. 31 : 10.}; il vieillit à cause de tous ceux qui m'oppriment.
\VS{9}Retirez-vous loin de moi, vous tous ouvriers d'iniquité\FTNT{Mt. 7:23 ; Mt. 25 : 41 ; Lu. 13 : 27.}! Car Yahweh a entendu la voix de mes pleurs.
\VS{10}Yahweh a entendu ma supplication, Yahweh a reçu ma prière.
\VS{11}Tous mes ennemis sont confondus, saisis d'épouvante; ils reculent soudain, honteux.
\Chap{7}
\TextTitle{La délivrance se trouve auprès de Yahweh}
\VerseOne{}Shiggaïon de David, chantée à Yahweh, au sujet de Cusch, le Benjamite.
\VS{2}Yahweh, mon Dieu ! Je cherche en toi mon refuge. Sauve-moi de tous mes persécuteurs et délivre-moi,
\VS{3}afin qu'ils ne me déchirent pas, comme un lion qui dévore sans qu'il n'y ait personne qui me secoure.
\VS{4}Yahweh, mon Dieu ! Si j'ai commis une telle action, s'il y a de l'iniquité dans mes mains,
\VS{5}si j'ai rendu le mal à celui qui était paisible envers moi, si j'ai dépouillé celui qui m'opprimait sans cause,
\VS{6}que l'ennemi me poursuive et m'atteigne, qu'il foule à terre ma vie, et qu'il couche ma gloire dans la poussière ! Sélah.
\VS{7}Lève-toi, ô Yahweh ! Dans ta colère, lève-toi contre la fureur de mes adversaires. Réveille-toi pour me secourir, ordonne un jugement !
\VS{8}Que l'assemblée des peuples t'environne ! Monte au-dessus d'elle vers les lieux élevés !
\VS{9}Yahweh juge les peuples : Rends-moi justice, ô Yahweh\FTNT{Ps. 9 : 5.} ! Selon ma droiture et selon mon intégrité !
\VS{10}Que la malice des méchants prenne fin, et affermis le juste, toi qui sondes les cœurs et les reins\FTNT{Jé. 11:20 ; Jé. 17:10.}, ô Dieu juste !
\VS{11}Mon bouclier est en Dieu, qui délivre ceux qui sont droits de cœur.
\VS{12}Dieu est un juste juge, Dieu s'irrite en tout temps.
\VS{13}Si le méchant ne se convertit pas, Dieu aiguise son épée\FTNT{De. 32 : 41.}, il bande son arc, et vise.
\VS{14}Il dirige sur lui des traits meurtriers, il rend ses flèches brûlantes.
\VS{15}Voici, le méchant prépare le mal, il conçoit l'iniquité, et il enfante le mensonge\FTNT{Ja. 1:15.}.
\VS{16}Il fait une fosse, il la creuse, et il tombe dans la fosse qu'il a faite\FTNT{Ps. 9 : 16.}.
\VS{17}Son travail retourne sur sa tête, et sa violence redescend sur son front.
\VS{18}Je célébrerai Yahweh à cause de sa justice, je psalmodierai le Nom de Yahweh, du Très-Haut.
\Chap{8}
\TextTitle{Magnificence de Dieu et vanité de l'homme}
\VerseOne{}Psaume de David, donné au chef des chantres, pour le chanter sur guitthith.
\VS{2}Yahweh, notre Seigneur ! Que ton Nom est magnifique sur toute la terre ! Ta majesté s'élève au-dessus des cieux\FTNT{Es. 6 : 3.}.
\VS{3}Par la bouche des petits enfants et de ceux qui tètent\FTNT{Mt. 21 : 16.}, tu as fondé ta puissance, à cause de tes adversaires, afin de faire cesser l'ennemi et le vindicatif.
\VS{4}Quand je regarde tes cieux, l'ouvrage de tes doigts, la lune et les étoiles que tu as fixées:
\VS{5}Qu'est-ce que l'homme, pour que tu te souviennes de lui ? Et le fils de l'homme, pour que tu le visites\FTNT{Dans ce passage, il est question de l'incarnation de Yahweh afin de nous sauver (1 Co. 15:45-49 ; 1 Ti. 3:16 ; Hé. 2:14). Jésus-Christ s'est lui-même nommé « fils de l'homme » (Lu. 9:22-26), littéralement « fils d'Adam », d'ailleurs cette expression apparaît dans les évangiles plus de quatre-vingt fois.} ?
\VS{6}Tu l'as fait de peu inférieur aux anges, et tu l'as couronné de gloire et d'honneur.
\VS{7}Tu lui as donné la domination sur les œuvres de tes mains, tu as tout mis sous ses pieds\FTNT{1 Co. 15:27.},
\VS{8}les brebis comme les bœufs, les animaux des champs,
\VS{9}les oiseaux du ciel et les poissons de la mer, tout ce qui parcourt les sentiers des mers.
\VS{10}Yahweh, notre Seigneur ! Que ton Nom est magnifique sur toute la terre !
\Chap{9}
\TextTitle{Louange à Yahweh, l'auteur de nos victoires}
\VerseOne{}Psaume de David, donné au chef des chantres, pour le chanter sur Muth-Labben.
\VS{2}Je célébrerai de tout mon cœur Yahweh, je raconterai toutes tes merveilles.
\VS{3}Je me réjouirai et je m'égaierai en toi, je chanterai ton Nom, ô Très-Haut !
\VS{4}Mes ennemis reculent, ils trébuchent, ils périssent devant ta face.
\VS{5}Car tu soutiens mon droit et ma cause, tu sièges sur ton trône en juste juge.
\VS{6}Tu châties les nations, tu détruis le méchant, tu effaces leur nom pour toujours, et à perpétuité.
\VS{7}Plus d'ennemis ! Les désolations ont-elles pris fin ? As-tu aussi rasé les villes pour toujours ? Leur mémoire est perdue avec elles.
\VS{8}Mais Yahweh sera assis éternellement, il a établi son trône pour juger.
\VS{9}Il juge le monde avec justice, il juge les peuples avec droiture\FTNT{Ps. 96:13 ; Ps. 98:9.}.
\VS{10}Yahweh est un refuge pour l'opprimé, un refuge au temps de la détresse\FTNT{Ps. 37:39 ; Ps. 46:2 ; Ps. 91:2.}.
\VS{11}Ceux qui connaissent ton Nom se confient en toi\FTNT{Pr. 3:5.}. Car tu n'abandonnes point ceux qui te cherchent, ô Yahweh !
\VS{12}Chantez à Yahweh qui habite en Sion, annoncez ses exploits parmi les peuples !
\VS{13}Lorsqu'il recherche le sang versé, il se souvient des malheureux? il n'oublie pas le cris des affligés.
\VS{14}Aie pitié de moi, Yahweh ! Vois la misère où me réduisent mes ennemis, enlève-moi des portes de la mort,
\VS{15}afin que je raconte toutes tes louanges, dans les portes de la fille de Sion. Je me réjouirai de la délivrance\FTNT{Voir commentaire en Es. 26:1.} que tu m'auras donnée.
\VS{16}Les nations tombent dans la fosse qu'elles ont faite\FTNT{Ps. 10:2 ; Ps. 35:7.}, leur pied se prend au filet qu'elles ont caché.
\VS{17}Yahweh se fait connaître, il fait justice, le méchant est enlacé dans l'ouvrage de ses mains. Jeu d'instruments. Sélah.
\VS{18}Les méchants retournent dans le scheol, toutes les nations qui oublient Dieu.
\VS{19}Car le pauvre n'est point oublié à jamais, l'espérance des affligés ne périt pas à toujours.
\VS{20}Lève-toi, ô Yahweh ! Que l'homme mortel ne triomphe point ! Que les nations soient jugées devant ta face !
\VS{21}Frappe-les de terreur, ô Yahweh ! Que les peuples sachent qu'ils ne sont que des hommes mortels\FTNT{Es. 51:12.}! Sélah.
\Chap{10}
\TextTitle{Appel au jugement de Dieu sur les méchants}
\VerseOne{}Pourquoi, ô Yahweh ! Te tiens-tu éloigné ? Pourquoi te caches-tu au temps où nous sommes dans la détresse\FTNT{Ps. 13:2 ; Ps. 44:24.} ?
\VS{2}Le méchant par son orgueil poursuit ardemment les affligés, mais ils seront pris par les machinations qu'ils ont préméditées\FTNT{Ps. 7:15-16 ; Ps. 9:16 ; Ps. 35:8.}.
\VS{3}Car le méchant se glorifie du désir de son âme, il bénit l'avare et il méprise Yahweh.
\VS{4}Le méchant dit avec arrogance : Il ne fera pas d'enquête ! Il n'y a point de Dieu\FTNT{Ps. 14:1 ; Ps. 53:2.} ! Voilà toutes ses pensées.
\VS{5}Ses voies réussissent en tout temps; tes jugements sont éloignés de lui, il souffle contre tous ses adversaires.
\VS{6}Il dit en son cœur : Je ne chancelle pas, je suis pour toujours à l'abri du malheur !
\VS{7}Sa bouche est pleine de malédictions, de tromperies et de fraudes ; il n'y a sous sa langue qu'oppression et outrage\FTNT{Ps. 59:7-8 ; Ps. 64:3-4 ; Job. 20:12.}.
\VS{8}Il se tient aux embûches dans des villages, il tue l'innocent dans des lieux cachés, ses yeux épient le malheureux.
\VS{9}Il se tient aux aguets dans un lieu caché, comme un lion dans sa tanière, il se tient aux aguets pour attraper l'affligé ; il attrape l'affligé, l'attirant dans son filet.
\VS{10}Il se courbe, il se baisse, et les malheureux tombent dans ses griffes.
\VS{11}Il dit en son cœur : Dieu oublie ! Il cache sa face, il ne le verra jamais\FTNT{Ps 94:7.}!
\VS{12}Lève-toi, ô Yahweh ! Lève ta main ! N'oublie pas les malheureux !
\VS{13}Pourquoi le méchant méprise-t-il Dieu ? Il dit en son cœur que tu ne punis pas.
\VS{14}Tu regardes cependant, car tu vois la peine et la souffrance, pour prendre en main leur cause ; c'est toi qui viens en aide à l'orphelin.
\VS{15}Brise le bras du méchant, punis ses iniquités et qu'il disparaisse à tes yeux !
\VS{16}Yahweh est Roi à toujours et à perpétuité\FTNT{Ps. 29:10 ; Ps. 145:13 ; Ps. 146:10 ; La. 5:19.} ; les nations sont exterminées de sa terre.
\VS{17}Tu entends les vœux de ceux qui souffrent, ô Yahweh ! Tu affermis leur cœur; tu prêtes l'oreille
\VS{18}pour rendre justice à l'orphelin et à l'opprimé, afin que l'homme mortel tiré de la terre cesse d'inspirer l'effroi.
\Chap{11}
\TextTitle{Yahweh, le refuge des hommes droits}
\VerseOne{}Psaume de David, donné au chef des chantres. C'est en Yahweh que je cherche un refuge. Comment un homme peut-il dire à mon âme : Fuis dans vos montagnes, comme un oiseau ?
\VS{2}En effet, les méchants bandent l'arc\FTNT{Ps. 37:14.}, ils ajustent leur flèche sur la corde, pour tirer dans l'ombre sur ceux dont le cœur est droit.
\VS{3}Quand les fondements sont renversés, que fera le juste ?
\VS{4}Yahweh est dans son saint temple, Yahweh a son trône dans les cieux ; ses yeux contemplent, ses paupières sondent les fils des hommes.
\VS{5}Yahweh sonde le juste et le méchant ; et son âme hait celui qui aime la violence.
\VS{6}Il fait pleuvoir sur les méchants des charbons, du feu et du soufre\FTNT{Ez. 38:22.} ; un vent brûlant, c'est le calice qu'ils ont en partage.
\VS{7}Car Yahweh est juste, il aime la justice ; les hommes droits contemplent sa face.
\Chap{12}
\TextTitle{Le langage des lèvres arrogantes}
\VerseOne{}Psaume de David, donné au chef des chantres pour le chanter sur Sheminith.
\VS{2}Sauve, ô Yahweh ! Car les hommes pieux s'en vont, les fidèles disparaissent parmi les fils de l'homme.
\VS{3}Chacun dit des faussetés à son compagnon avec des lèvres flatteuses et ils parlent avec un cœur double.
\VS{4}Que Yahweh retranche toutes les lèvres flatteuses, la langue qui parle fièrement\FTNT{Ps. 17:10.},
\VS{5}parce qu'ils disent : Nous sommes puissants par nos langues, nous avons nos lèvres avec nous ; qui serait notre maître ?
\VS{6}A cause du mauvais traitement que l'on fait aux malheureux, à cause du gémissement des pauvres, je me lèverai maintenant, dit Yahweh, je mettrai en sûreté celui à qui l'on tend des pièges.
\VS{7}Les paroles de Yahweh sont des paroles pures, c'est un argent éprouvé sur terre au creuset\FTNT{Ps. 19:10 ; Ps. 119:140 ; Pr. 30:5.}, et sept fois épuré.
\VS{8}Toi, Yahweh ! Garde-les, préserve cette race à jamais.
\VS{9}Les méchants se promènent de toutes parts, tandis que des gens abjects sont élevés parmi les fils des hommes.
\Chap{13}
\TextTitle{Savoir attendre le secours de Dieu}
\VerseOne{}Psaume de David, donné au chef des chantres.
\VS{2}Yahweh, jusqu'à quand m'oublieras-tu ? Sera-ce pour toujours ? Jusqu'à quand me cacheras-tu ta face\FTNT{Ps. 10:1 ; Ps. 27:9.} ?
\VS{3}Jusqu'à quand consulterai-je mon âme, affligerai-je mon cœur tous les jours ? Jusqu'à quand mon ennemi s'élèvera-t-il contre moi ?
\VS{4}Yahweh, mon Dieu ! Regarde, exauce-moi, illumine mes yeux, de peur que je ne dorme du sommeil de la mort,
\VS{5}de peur que mon ennemi ne dise : J'ai eu le dessus! Que mes adversaires ne se réjouissent, si je venais à tomber\FTNT{Ps. 25:2.}.
\VS{6}Mais moi, je me confie en ta bonté, mon cœur se réjouira de la délivrance que tu m'auras donnée ; je chanterai à Yahweh, parce qu'il m'a fait du bien.
\Chap{14}
\TextTitle{L'insensé ne cherche pas Dieu}
\VerseOne{}Psaume de David, donné au chef des chantres. L'insensé dit en son cœur : Il n'y a point de Dieu\FTNT{Ceux qui ne croient pas en l'existence de Dieu sont appelés insensés. En effet, la création révèle l'existence du Créateur (Ro. 1:19-20).} ! Ils se sont corrompus, ils ont commis des actions abominables ; il n'y a personne qui fasse le bien.
\VS{2}Yahweh regarde des cieux les fils de l'homme, pour voir s'il y a quelqu'un qui soit intelligent, qui cherche Dieu\FTNT{Ps. 33:13 ; Job. 28:24.}.
\VS{3}Ils se sont tous égarés, ils se sont tous ensemble rendus odieux, il n'y a personne qui fasse le bien, pas même un seul\FTNT{Tous les hommes naissent pécheurs (Ro. 3:10-23).}.
\VS{4}Tous ces ouvriers d'iniquité n'ont-ils point de connaissance ? Ils dévorent mon peuple, ils le prennent pour nourriture ; ils n'invoquent point Yahweh.
\VS{5}Là, ils seront saisis d'une grande frayeur, car Dieu est avec la race des justes.
\VS{6}Jetez l'opprobre sur l'espérance du malheureux…Yahweh est son refuge.
\VS{7}Oh ! Qui fera partir de Sion la délivrance d'Israël\FTNT{C'est le Messie qui délivrera Israël (Ro. 11:25-27).} ? Quand Yahweh ramenera son peuple captif, Jacob se réjouira, Israël se réjouira.
\Chap{15}
\TextTitle{L'homme que Yahweh agrée}
\VerseOne{}Psaume de David. Yahweh, qui séjournera dans ta tente ? Qui demeurera sur ta montagne sainte\FTNT{Ps. 24:3-4.} ?
\VS{2}Celui qui marche dans l'intégrité, qui fait ce qui est juste, et qui profère la vérité telle qu'elle est dans son cœur,
\VS{3}qui ne calomnie point avec sa langue, qui ne fait point de mal à son ami, qui ne diffame point son prochain.
\VS{4}Il regarde avec dédain celui qui est méprisable, mais il honore ceux qui craignent Yahweh; il ne se rétracte point s'il fait un serment à son préjudice.
\VS{5}Il n'exige point d'intérêt de son argent, et il n'accepte point de présent contre l'innocent\FTNT{Lé. 25:36 ; De. 16:19 ; De. 27:25.}. Celui qui fait ces choses ne sera jamais ébranlé.
\Chap{16}
\TextTitle{Yahweh, la source de la vie}
\VerseOne{}Mictam de David. Garde-moi, ô Dieu ! Car je cherche en toi mon refuge.
\VS{2}Je dis à Yahweh : Tu es mon Seigneur, tu es mon bonheur!
\VS{3}Les saints qui sont dans le pays, les hommes pieux sont l'objet de toute mon affection.
\VS{4}On multiplie les peines, on court après les dieux étrangers : Je ne répands pas leurs libations de sang et je ne mets pas leurs noms sur mes lèvres.
\VS{5}Yahweh est la part de mon héritage et ma coupe ; tu maintiens mon lot;
\VS{6}un héritage délicieux m'est échu, une belle possession m'est accordée.
\VS{7}Je bénirai Yahweh qui me donne conseille; je le bénirai même durant les nuits dans lesquelles mes reins m'enseignent.
\VS{8}J'ai constamment Yahweh sous mes yeux ; quand il est à ma droite, je ne chancelle pas\FTNT{Ps. 109:31 ; Ps. 110:5 ; Ac. 2:25.}.
\VS{9}C'est pourquoi mon cœur se réjouit, mon esprit se réjouit et mon corps repose en sécurité.
\VS{10}Car tu n'abandonneras point mon âme au scheol, tu ne permettras point que ton bien-aimé voie la corruption\FTNT{Le roi David prophétise ici la résurrection du Messie.}.
\VS{11}Tu me feras connaître le chemin de la vie ; il y a d'abondantes joies devant ta face, des délices éternels à ta droite.
\Chap{17}
\TextTitle{L'assurance en Dieu}
\VerseOne{}Prière de David. Yahweh, écoute la droiture, sois attentif à mon cri, prête l'oreille à ma prière faite avec des lèvres sans tromperie !
\VS{2}Que ma justice paraisse devant ta face, que tes yeux contemplent mon intégrité !
\VS{3}Tu as sondé mon cœur\FTNT{Ps. 139:1 ; Jé. 12:3.}, tu l'as visité de nuit, tu m'as examiné, tu n'as rien trouvé : Ma pensée ne va point au-delà de ma parole.
\VS{4}Quant aux actions des hommes, selon la parole de tes lèvres, je me tiens en garde contre la voie du violent.
\VS{5}Mes pas sont fermes dans tes sentiers, mes pieds ne chancellent point.
\VS{6}Je t'invoque, car tu m'exauces, ô Dieu ! Incline ton oreille vers moi, écoute mes paroles !
\VS{7}Signale ta bonté, toi qui sauves ceux qui cherchent un refuge, et qui par ta droite les délivres de leurs adversaires !
\VS{8}Garde-moi comme la prunelle de l'œil, cache-moi à l'ombre de tes ailes\FTNT{Mt. 23:37.},
\VS{9}contre les méchants qui me traitent violemment, mes ardents ennemis qui m'entourent.
\VS{10}Ils sont enfermés dans leur propre graisse, leur bouche parle avec orgueil.
\VS{11}Maintenant, ils nous environnent à chaque pas que nous faisons ; ils jettent leur regard pour nous étendre par terre.
\VS{12}Ils ressemblent au lion qui ne demande qu'à déchirer, et au lionceau qui se tient dans les lieux cachés.
\VS{13}Lève-toi, ô Yahweh, devance-les, renverse-les ! Délivre mon âme du méchant par ton épée.
\VS{14}Yahweh, délivre-moi par ta main de ces gens, des gens de ce monde ! Leur part est dans cette vie et tu remplis leur ventre de tes biens ; leurs enfants sont rassasiés et ils laissent leurs restes à leurs petits-enfants.
\VS{15}Mais moi, dans mon innocence, je verrai ta face\FTNT{Job. 19:26-27 ; Ps. 16:10-11.}, et je me rassasierai de ton image, dès mon réveil.
\Chap{18}
\TextTitle{Louange à Dieu, le bouclier des saints}
\VerseOne{}Psaume de David, serviteur de Yahweh, qui adressa à Yahweh les paroles de ce cantique le jour où Yahweh l'eut délivré de la main de Saül. Au chef des chantres.
\VS{2}Il dit donc : Je t'aime, ô Yahweh, ma force !
\VS{3}Yahweh est mon rocher\FTNT{Yahweh est le rocher sur lequel s'appuyait David. Paul enseigne que ce rocher était Jésus-Christ (1 Co. 10:1-4). Voir commentaire en Es. 8:13-17.}, ma forteresse et mon libérateur ! Mon Dieu, mon rocher où je trouve un refuge ! Mon bouclier, la force qui me sauve, ma haute retraite !
\VS{4}Je crie : Loué soit Yahweh ! Et je suis délivré de mes ennemis.
\VS{5}Les liens de la mort m'avaient environné et des torrents de destruction m'avaient épouvanté.
\VS{6}Les liens du scheol m'avaient entouré, les filets de la mort m'avaient surpris\FTNT{Ps. 116:3.}.
\VS{7}Dans ma détresse, j'ai invoqué Yahweh, j'ai crié à mon Dieu ; il a entendu ma voix de son palais, mon cri est parvenu devant lui à ses oreilles.
\VS{8}La terre fut ébranlée et trembla, les fondements des montagnes croulèrent\FTNT{Es. 5:25 ; Es. 64:1-3 ; Jé. 4:24 ; Ps. 104:32.}, et ils furent ébranlés, parce qu'il était irrité.
\VS{9}Une fumée montait de ses narines, et de sa bouche sortait un feu dévorant, des charbons embrasés.
\VS{10}Il abaissa les cieux et descendit : Il y avait une épaisse nuée sous ses pieds.
\VS{11}Il était monté sur un chérubin, et il volait, il était porté sur les ailes du vent\FTNT{Ps. 104:3.}.
\VS{12}Il faisait des ténèbres sa demeure secrète, autour de lui était sa tente, il était enveloppé des eaux obscures et de sombres nuages.
\VS{13}De la splendeur qui le précédait s'échappaient les nuées, lançant de la grêle et des charbons de feu.
\VS{14}Yahweh tonna dans les cieux, le Très-Haut fit retentir sa voix avec de la grêle et des charbons de feu.
\VS{15}Il tira ses flèches, et écarta mes ennemis, il lança des éclairs et les mit en déroute\FTNT{Ps. 77:18.}.
\VS{16}Le fond des eaux parut, les fondements du monde furent découverts, par ta menace, ô Yahweh ! Par le souffle du vent de tes narines.
\VS{17}Il étendit la main d'en haut, il m'enleva et me retira des grandes eaux\FTNT{2 S. 22:17.} ;
\VS{18}il me délivra de mon puissant ennemi, et de ceux qui me haïssaient, car ils étaient plus forts que moi.
\VS{19}Ils m'avaient surpris au jour de ma détresse, mais Yahweh, fut mon appui.
\VS{20}Il m'a mis au large, il m'a délivré, parce qu'il m'aime.
\VS{21}Yahweh m'a rendu selon ma justice, il m'a traité selon la pureté de mes mains\FTNT{Ps. 18:25 ; Ps. 7:9.},
\VS{22}car j'ai observé les voies de Yahweh et je n'ai point été coupable envers mon Dieu.
\VS{23}Car j'ai eu devant moi toutes ses ordonnances et je ne me suis point écarté de ses lois.
\VS{24}J'ai été intègre envers lui, et je me suis tenu en garde contre mon iniquité.
\VS{25}Aussi Yahweh m'a rendu selon ma justice, selon la pureté de mes mains devant ses yeux,
\VS{26}avec celui qui est bon, tu te montres bon, avec l'homme droit tu agis selon la droiture.
\VS{27}Avec celui qui est pur, tu te montres pur, et avec le pervers tu agis selon sa perversité.
\VS{28}Car tu sauves le peuple affligé et tu abaisses les yeux hautains\FTNT{Es. 2:11 ; Es. 5:15.}.
\VS{29}Tu fais briller ma lumière ; Yahweh, mon Dieu, éclaire mes ténèbres.
\VS{30}Avec toi, je me précipite sur un corps d'armée, avec mon Dieu je franchis la muraille.
\VS{31}Les voies de Dieu sont sans défaut ; la parole de Yahweh est éprouvée\FTNT{De. 32:4 ; Ps. 19:8-9 ; Da. 4:37.} ; il est un bouclier pour tous ceux qui se confient en lui.
\VS{32}Car qui est Dieu, si ce n'est Yahweh ? Et qui est un rocher, si ce n'est notre Dieu ?\FTNT{1 S. 2:2 ; 2 S. 22:32.}
\VS{33}C'est le Dieu qui me ceint de force, et qui me conduit dans la voie droite.
\VS{34}Il rend mes pieds semblables à ceux des biches\FTNT{2 S. 2:18.}, et il me place sur mes lieux élevés.
\VS{35}Il exerce mes mains au combat, tellement qu'un arc d'airain a été rompu avec mes bras.\FTNT{Job. 20:24.}.
\VS{36}Tu me donnes le bouclier de ton salut, ta droite me soutient, et je deviens puissant par ta bonté.
\VS{37}Tu élargis le chemin sous mes pas, et mes pieds ne chancellent point.
\VS{38}Je poursuis mes ennemis, je les atteints, et je ne reviens pas avant de les avoir anéantis.
\VS{39}Je les brise et ils ne peuvent se relever ; ils tombent sous mes pieds.
\VS{40}Car tu m'as ceint de force pour le combat, tu fais plier sous moi ceux qui s'élevaient contre moi.
\VS{41}Tu fais tourner le dos à mes ennemis devant moi, et j'extermine ceux qui me haïssaient.
\VS{42}Ils crient, mais il n'y a point de libérateur! Ils crient à Yahweh, mais il ne leur répond point!
\VS{43}Je les brise comme la poussière qui est dispersée par le vent et je les foule comme la boue des rues.
\VS{44}Tu me délivres des séditions du peuple, tu m'établis chef des nations. Un peuple que je ne connais point m'est asservi.
\VS{45}Ils m'obéissent au premier ordre, les fils de l'étranger me flattent.
\VS{46}Les étrangers s'enfuient et ils tremblent de peur dans leurs forteresses.
\VS{47}Yahweh est vivant, et béni soit mon rocher ! Que le Dieu de mon salut soit exalté !
\VS{48}Le Dieu qui est mon vengeur et qui m'assujettit les peuples,
\VS{49}c'est lui qui me délivre de mes ennemis ! Tu m'élèves au-dessus de mes adversaires, tu me sauves de l'homme violent.
\VS{50}C'est pourquoi, ô Yahweh, je te célébrerai parmi les nations ! Et je chanterai des psaumes à ton Nom.
\VS{51}Il accorde de grandes délivrances à son roi, et il fait miséricorde à son oint, à David, et à sa postérité, pour toujours.
\Chap{19}
\TextTitle{La création exalte la grandeur de Dieu}
\VerseOne{}Psaume de David, donné au chef des chantres.
\VS{2}Les cieux racontent la gloire de Dieu, et l'étendue met en évidence l'oeuvre de ses mains.
\VS{3}Un jour en instruit un autre jour, et une nuit fait connaître sa science à l'autre nuit.
\VS{4}Ce n'est pas un langage, ce ne sont pas des paroles dont le cri ne soit point entendu :
\VS{5}Leur retentissement couvre toute la terre, et leur voix est allée jusqu'aux extrémités du monde\FTNT{Ro. 10:18.}. Il a dressé une tente pour le soleil.
\VS{6}Et le soleil est semblable à un époux sortant de sa chambre ; il s'élance sur le sentier avec la joie d'un homme vaillant;
\VS{7}il se lève à l'extrémité des cieux et achève sa course à l'autre extrémité\FTNT{Ec. 1:5.}: Rien ne se dérobe à sa chaleur.
\VS{8}La loi de Yahweh est parfaite, elle restaure l'âme ; le témoignage de Yahweh est fidèle, il donne la sagesse au simple\FTNT{2 S. 22:31 ; Ps. 18:31 ; Ps. 119:130.}.
\VS{9}Les ordonnances de Yahweh sont droites, elles réjouissent le cœur ; les commandements de Yahweh sont purs, ils éclairent les yeux.
\VS{10}La crainte de Yahweh est pure, elle subsiste à toujours ; les jugements de Yahweh sont vrais, et ils sont tous justes.
\VS{11}Ils sont plus précieux que l'or, que beaucoup d'or fin ; et plus doux que le miel, que celui qui coule des rayons de miel\FTNT{Ps. 119:103.}.
\VS{12}Ton serviteur aussi en reçoit l'éclairage ; pour qui les observe la récompense est grande.
\VS{13}Qui connaît ses fautes commises par erreur ? Purifie-moi de mes fautes cachées.
\VS{14}Eloigne aussi ton serviteur des actions commises par fierté, en sorte qu'elles ne dominent point sur moi, qu'elles cessent et que je sois nettoyé de mes grands péchés!
\VS{15}Que les propos de ma bouche et la méditation de mon cœur te soient agréables, ô Yahweh ! Mon rocher et mon rédempteur\FTNT{Voir commentaire en Es. 60:16.}!
\Chap{20}
\TextTitle{Recours à l'intervention de Dieu}
\VerseOne{}Psaume de David, donné au chef des chantres.
\VS{2}Que Yahweh te réponde au jour de la détresse, que le Nom du Dieu de Jacob te protège !
\VS{3}Qu'il envoie ton secours du saint lieu, et qu'il te soutienne de Sion !
\VS{4}Qu'il se souvienne de toutes tes offrandes, qu'il réduise en cendres ton holocauste ! Sélah.
\VS{5}Qu'il te donne ce que ton cœur désire, et qu'il fasse réussir tes desseins !
\VS{6}Nous triompherons dans ton salut, nous lèverons la bannière au Nom de notre Dieu ; Yahweh exaucera tous tes vœux.
\VS{7}Je sais déjà que Yahweh sauve son oint ; il l'exaucera des cieux, de sa sainte demeure, par le secours puissant de sa droite.
\VS{8}Les uns se vantent de leurs chars, et les autres de leurs chevaux ; mais nous, nous glorifierons le Nom de Yahweh, notre Dieu.
\VS{9}Eux ils plient, et ils tombent ; nous, nous tenons ferme, et restons debout.
\VS{10}Yahweh, sauve le roi ! Qu'il nous réponde quand nous crions à lui !
\Chap{21}
\TextTitle{La protection de Dieu sur le roi}
\VerseOne{}Psaume de David, donné au chef des chantres.
\VS{2}Yahweh, le roi se réjouit de ta puissance, ton secours le remplit d'allégresse !
\VS{3}Tu lui as donné ce que désirait son cœur et tu n'as point refusé ce que demandaient ses lèvres. Sélah.
\VS{4}Car tu l'as prévenu par les bénédictions de ta bonté, et tu as mis sur sa tête une couronne d'or pur.
\VS{5}Il t'avait demandé la vie, et tu la lui as donnée, une vie longue pour toujours et à perpétuité.
\VS{6}Sa gloire est grande à cause de ton salut, tu l'as couvert de majesté et d'honneur.
\VS{7}Tu le rends à jamais un objet de bénédictions, tu le combles de joie devant ta face\FTNT{Ps. 16:11.}.
\VS{8}Le roi se confie en Yahweh, et par la bonté du Très-Haut, il ne chancelle pas\FTNT{Ps. 16:8.}.
\VS{9}Ta main trouvera tous tes ennemis, ta droite trouvera tous ceux qui te haïssent.
\VS{10}Tu les rendras tels qu'une fournaise ardente le jour où l'on verra ta face ; Yahweh les engloutira dans sa colère, et le feu les consumera.
\VS{11}Tu feras périr leur fruit de la terre et leur race du milieu des fils des hommes.
\VS{12}Car ils ont projeté du mal contre toi et ils ont conçu de mauvais desseins dont ils ne pourront venir à bout.
\VS{13}Parce que tu leur feras tourner le dos, et avec ton arc tu tireras sur eux.
\VS{14}Elève-toi, Yahweh, par ta force ! Nous chanterons et célébrerons ta puissance.
\Chap{22}
\TextTitle{Les souffrances du Messie}
\VerseOne{}Psaume de David, donné au chef des chantres, pour le chanter sur Ajéleth-Hashakhar.
\VS{2}Mon Dieu ! Mon Dieu ! Pourquoi m'as-tu abandonné\FTNT{Le Ps. 22 est une description détaillée de la mort par crucifixion du Seigneur Jésus-Christ (Mt. 27:45-46).}, et t'éloignes-tu sans me secourir, sans écouter mes plaintes ?
\VS{3}Mon Dieu ! Je crie le jour, mais tu ne réponds point ; la nuit, et je n'ai point de repos.
\VS{4}Pourtant tu es le Saint, tu habites au milieu des louanges d'Israël.
\VS{5}Nos pères se sont confiés en toi ; ils se sont confiés, et tu les as délivrés.
\VS{6}Ils ont crié vers toi, et ils ont été délivrés ; ils se sont appuyés sur toi, et ils n'ont point été confus\FTNT{Es. 49:23 ; Ps. 25:3 ; Ps. 31:2.}.
\VS{7}Et moi, je suis un ver et non un homme, l'opprobre des hommes et le méprisé du peuple\FTNT{Es. 53:2-3.}.
\VS{8}Tous ceux qui me voient se moquent de moi, ils ouvrent les lèvres, secouent la tête\FTNT{Ps. 109:25 ; Mt. 27:39.} :
\VS{9}Recommande-toi à Yahweh ! Qu'il te délivre, et qu'il te sauve, puisqu'il prend plaisir en toi\FTNT{Mt. 27:43.} !
\VS{10}Cependant, c'est toi qui m'as tiré hors du ventre de ma mère, qui m'as mis en sûreté lorsque j'étais sur les mamelles de ma mère.
\VS{11}J'ai été sous ta garde, dès le sein maternel, tu as été mon Dieu dès le ventre de ma mère\FTNT{Es. 49:1.}.
\VS{12}Ne t'éloigne point de moi, car la détresse est près de moi, et il n'y a personne qui me secoure\FTNT{Ps. 69:21.} !
\VS{13}Plusieurs taureaux sont autour de moi, de puissants taureaux de Basan m'entourent.
\VS{14}Ils ouvrent leur gueule contre moi, comme un lion qui déchire et rugit.
\VS{15}Je suis comme de l'eau qui s'écoule, et tous mes os se séparent ; mon cœur est comme de la cire, il se fond dans mes entrailles.
\VS{16}Ma force se dessèche comme l'argile, et ma langue s'attache à mon palais ; tu me réduis à la poussière de la mort.
\VS{17}Car des chiens m'environnent, une assemblée de méchants m'entoure, ils ont percé mes mains et mes pieds.
\VS{18}Je pourrais compter tous mes os un par un. Eux, ils m'examinent, ils me regardent.
\VS{19}Ils se partagent mes vêtements et tirent au sort ma tunique\FTNT{Mt. 27:35 ; Mc. 15:24 ; Lu. 23:33.}.
\VS{20}Et toi, Yahweh, ne t'éloigne point ! Ma force, hâte-toi de me secourir !
\VS{21}Délivre ma vie de l'épée, ma vie contre le pouvoir des chiens !
\VS{22}Sauve-moi de la gueule du lion, délivre-moi des cornes du buffle !
\VS{23}Je déclarerai ton Nom à mes frères, je te louerai au milieu de l'assemblée\FTNT{Hé. 2:12.}.
\VS{24}Vous qui craignez Yahweh, louez-le ! Toute la race de Jacob, glorifiez-le ! Toute la race d'Israël, redoutez-le !
\VS{25}Car il n'a ni mépris ni dédain pour les peines du misérable, et il ne lui cache point sa face, mais il l'écoute quand il crie à lui.
\VS{26}Tu seras l'objet de mes louanges dans la grande assemblée ; j'accomplirai mes vœux en présence de ceux qui te craignent\FTNT{Ps. 56:13.}.
\VS{27}Les malheureux mangeront et seront rassasiés, ceux qui cherchent Yahweh le loueront. Votre cœur vivra à perpétuité !
\VS{28}Toutes les extrémités de la terre s'en souviendront, ils se convertiront à Yahweh, et toutes les familles des nations se prosterneront devant toi\FTNT{Ps. 72:8-11 ; Ps. 86:9.}.
\VS{29}Car le règne appartient à Yahweh : Il domine sur les nations.
\VS{30}Tous les gens de la terre mangeront et se prosterneront devant lui ; tous ceux qui descendent dans la poussière s'inclineront, même celui qui ne peut conserver sa vie.
\VS{31}La postérité le servira, on parlera du Seigneur de génération en génération\FTNT{Es. 59:21 ; Es. 65:23 ; Ps. 110:3.}.
\VS{32}Ils viendront et ils publieront sa justice au peuple qui naîtra, parce qu'il aura fait ces choses.
\Chap{23}
\TextTitle{Le bon Berger}
\VerseOne{}Psaume de David. Yahweh est mon berger\FTNT{Yahweh, le bon berger, est notre Seigneur Jésus-Christ. Es. 40:11 ; Jé. 23:4 ; Jn. 10:11.}: Je ne manquerai de rien.
\VS{2}Il me fait reposer dans de verts pâturages, il me dirige près des eaux paisibles.
\VS{3}Il restaure mon âme, et me conduit dans les sentiers de la justice, à cause de son Nom.
\VS{4}Quand je marche dans la vallée de l'ombre de la mort, je ne crains aucun mal\FTNT{Ps. 118:6.}, car tu es avec moi : Ton bâton et ta houlette me consolent.
\VS{5}Tu dresses devant moi une table, en face de mes adversaires; tu oins d'huile ma tête et ma coupe déborde.
\VS{6}Le bonheur et la grâce m'accompagneront tous les jours de ma vie, et j'habiterai dans la maison de Yahweh jusqu'à la fin de mes jours.
\Chap{24}
\TextTitle{Accueil de Yahweh, le Roi de gloire}
\VerseOne{}Psaume de David. La terre appartient à Yahweh, avec tout ce qui est en elle\FTNT{Ex. 19:5 ; De. 10:14 ; Ps. 50:12 ; Job. 41:2 ; 1 Co. 10:26.}, le monde et ceux qui y habitent!
\VS{2}Car il l'a fondée sur les mers, et affermie sur les fleuves.
\VS{3}Qui pourra monter à la montagne de Yahweh ? Qui s'élèvera jusqu'à son lieu saint\FTNT{Ps. 15:1-2 ; Ps. 118:19.} ?
\VS{4}Celui qui a les mains pures et le cœur pur, qui ne livre point son âme au mensonge, et qui ne jure pas pour tromper.
\VS{5}Il obtiendra la bénédiction de Yahweh et la justice du Dieu de son salut.
\VS{6}Voilà le partage de la génération qui l'invoque, de ceux qui cherchent ta face de Jacob ! Sélah.
\VS{7}Portes, élevez vos linteaux; élevez-vous portes éternelles ! Que le Roi de gloire fasse son entrée !
\VS{8}Qui est ce Roi de gloire ? C'est Yahweh fort et puissant, Yahweh puissant dans les combats.
\VS{9}Portes, élevez vos linteaux; élevez-les aussi, vous portes éternelles! Que le Roi de gloire fasse son entrée !
\VS{10}Qui est ce Roi de gloire ? Yahweh des armées : Voilà le Roi de gloire! Sélah.
\Chap{25}
\TextTitle{La crainte de Dieu mène à la voie de Yahweh}
\VerseOne{}Psaume de David. [Aleph.] Yahweh, j'élève mon âme à toi.
\VS{2}[Beth.] Mon Dieu ! Je me confie en toi : Que je ne sois point honteux\FTNT{Ps. 22:5 ; Ps. 31:2.} ! Que mes ennemis ne triomphent point de moi!
\VS{3}[Guimel.] Tous ceux qui espèrent en toi ne seront point confus\FTNT{Ro. 10:11.} ; ceux qui agissent avec tromperie sans cause seront honteux.
\VS{4}[Daleth.] Yahweh ! Fais-moi connaître tes voies, enseigne-moi tes sentiers\FTNT{Ps. 27:11 ; Ps. 86:11 ; Ps. 143:10.}.
\VS{5}[He. Vav.] Fais-moi marcher selon la vérité, et instruis-moi, car tu es le Dieu de ma délivrance, je m'attends à toi tous les jours.
\VS{6}[Zayin.] Yahweh ! Souviens-toi de ta miséricorde et de ta bonté, car elles sont éternelles\FTNT{Jé. 33:11 ; Ps. 103:17 ; Ps. 106:1 ; Ps. 107:1 ; Ps. 117:2 ; Ps. 136:1-2.}.
\VS{7}[Heth.] Ne te souviens point des péchés de ma jeunesse ni de mes transgressions ; souviens-toi de moi selon ta miséricorde, à cause de ta bonté, ô Yahweh !
\VS{8}[Teth.] Yahweh est bon et droit : C'est pourquoi il enseigne aux pécheurs la voie.
\VS{9}[Yod.] Il conduit les humbles dans la justice, et il leur enseigne sa voie.
\VS{10}[Kaf.] Tous les sentiers de Yahweh sont miséricorde et fidélité, pour ceux qui gardent son alliance et son témoignage.
\VS{11}[Lamed.] Pour l'amour de ton Nom, ô Yahweh ! Tu me pardonneras mon iniquité, car elle est grande\FTNT{2 S. 24:10.}.
\VS{12}[Mem.] Qui est l'homme qui craint Yahweh ? Yahweh lui enseignera la voie qu'il doit choisir.
\VS{13}[Nun.] Son âme demeurera dans le bonheur, et sa postérité possédera la terre en héritage.
\VS{14}[Samech.] Le secret de Yahweh est pour ceux qui le craignent, et son alliance leur donne le savoir.
\VS{15}[Ayin.] Mes yeux sont continuellement sur Yahweh, car c'est lui qui sortira mes pieds du filet.
\VS{16}[Pe.] Tourne ta face vers moi, et aie pitié de moi, car je suis seul et affligé.
\VS{17}[Tsade.] Les angoisses de mon cœur augmentent ; sors-moi de ma détresse.
\VS{18}[Resh.] Vois ma misère et ma peine, et pardonne tous mes péchés.
\VS{19}[Resh.] Vois combien mes ennemis sont nombreux, et me haïssent d'une haine pleine de violence\FTNT{Jn. 15:25.}.
\VS{20}[Shin.] Garde mon âme et délivre-moi ! Que je ne sois point confus, car je me suis réfugié en toi!
\VS{21}[Tav.] Que l'innocence et la droiture me protègent, car je m'attends à toi!
\VS{22}[Pe.] Ô Dieu ! Rachète Israël de toutes ses détresses !
\Chap{26}
\TextTitle{Demeurer dans l'intégrité}
\VerseOne{}Psaume de David. Yahweh, rends-moi justice\FTNT{Ps. 43:1 ; Ps. 54:3.} ! Car je marche dans l'intégrité, je me confie en Yahweh, je ne chancelle pas.
\VS{2}Sonde-moi et éprouve-moi\FTNT{Ps. 11:4-5 ; Ps. 17:3 ; Ps. 139:23.}, Yahweh ! Fais passer au creuset mes reins et mon cœur ;
\VS{3}car ta grâce est devant mes yeux, et je marche dans ta vérité.
\VS{4}Je ne m'assieds pas avec les hommes faux\FTNT{Ps. 1:1 ; 1 Co. 5:9-11 ; 1 Co. 15:33.}, et je ne vais point avec les gens dissimulés.
\VS{5}Je hais la compagnie de ceux qui font le mal\FTNT{Ps. 101:2-5 ; Ps. 119:113.}, et je ne m'assieds pas avec les méchants.
\VS{6}Je lave mes mains dans l'innocence et je fais le tour de ton autel\FTNT{Ps. 73:13.}, ô Yahweh !
\VS{7}Pour faire entendre le cri de reconnaissance, et pour raconter toutes tes merveilles.
\VS{8}Yahweh, j'aime la demeure de ta maison, le lieu dans lequel est le tabernacle de ta gloire.
\VS{9}N'enlève pas mon âme avec les pécheurs, ma vie avec les hommes de sang,
\VS{10}dont les mains sont criminelles, et la droite pleine de présents.
\VS{11}Moi, je marche dans l'intégrité ; délivre-moi et aie pitié de moi !
\VS{12}Mon pied se tient dans la droiture ; je bénirai Yahweh dans les assemblées.
\Chap{27}
\TextTitle{La foi qui triomphe des épreuves}
\VerseOne{}Psaume de David. Yahweh est ma lumière\FTNT{Es. 60:19-20 ; Mi. 7:8 ; Ps. 118:6 ; Jn. 8:12 ; Ap. 21:23.} et mon salut : De qui aurai-je peur ? Yahweh est le soutien de ma vie : De qui aurai-je peur ?
\VS{2}Lorsque les méchants s'avancent contre moi pour dévorer ma chair, ce sont mes adversaires et mes ennemis qui chancellent et tombent.
\VS{3}Si toute une armée campait contre moi, mon cœur ne craindrait point ; si une guerre s'élevait contre moi, je serai plein de confiance.
\VS{4}Je demande une chose à Yahweh, que je désire ardemment : C'est d'habiter dans la maison de Yahweh tous les jours de ma vie, pour contempler la beauté de Yahweh et pour admirer son temple.
\VS{5}Car il me cachera dans son tabernacle au jour du malheur, il me tiendra caché sous l'abri de sa tente; il m'élèvera sur un rocher.
\VS{6}Même maintenant ma tête s'élève par-dessus mes ennemis qui m'entourent ; et j'offrirai des sacrifices dans sa tente, au son de la trompette; je chanterai et célèbrerai Yahweh.
\VS{7}Yahweh ! Ecoute ma voix, je t'invoque : Aie pitié de moi et exauce-moi !
\VS{8}Mon cœur dit de ta part : Cherche ma face ! Je chercherai ta face, ô Yahweh !
\VS{9}Ne me cache point ta face, ne rejette point avec colère ton serviteur ! Tu es mon secours, ne me laisse pas, ne m'abandonne pas, Dieu de mon salut !
\VS{10}Car mon père et ma mère m'abandonnent, mais Yahweh me recueillera\FTNT{Es. 49:15.}.
\VS{11}Yahweh, enseigne-moi ta voie, et conduis-moi dans le sentier de la droiture, à cause de mes ennemis\FTNT{Ps. 5:9 ; Ps. 25:4-5.}.
\VS{12}Ne me livre pas au désir de mes adversaires, car s'élèvent contre moi de faux témoins et des gens qui ne respirent que la violence.
\VS{13}Oh ! Si je n'étais pas sûr de voir la bonté de Yahweh sur la terre des vivants…
\VS{14}Espère en Yahweh ! Fortifie-toi et que ton cœur s'affermisse\FTNT{Es. 33:2 ; Ps. 31:25.} ! Espère en Yahweh !
\Chap{28}
\TextTitle{Louange à Yahweh, le Rocher de son peuple}
\VerseOne{}Psaume de David. Je crie à toi, ô Yahweh ! Mon rocher! Ne te rends point sourd envers moi, de peur que si tu ne me réponds pas, je ne sois semblable à ceux qui descendent dans la fosse\FTNT{Ps. 4:2 ; Ps. 143:7. 
Voir commentaire en Es. 8:13-17.}.
\VS{2}Ecoute la voix de mes supplications, lorsque je crie à toi, quand j'élève mes mains vers ton saint sanctuaire.
\VS{3}Ne m'emporte pas avec les méchants ni avec les ouvriers d'iniquité, qui parlent de paix avec leur prochain pendant que la malice est dans leur cœur\FTNT{Jé. 9:8 ; Ps. 26:9.}.
\VS{4}Traite-les selon leurs œuvres et selon la malice de leurs actions, traite-les selon l'ouvrage de leurs mains, rends-leur ce qu'ils ont mérité\FTNT{2 Ti. 4:14.}.
\VS{5}Parce qu'ils ne prennent point garde aux œuvres de Yahweh, à l'œuvre de ses mains. Qu'il les renverse et ne les édifie point !
\VS{6}Béni soit Yahweh ! Car il exauce la voix de mes supplications.
\VS{7}Yahweh est ma force et mon bouclier ; mon cœur se confie en lui, et je suis secouru ; mon cœur se réjouit, c'est pourquoi je le loue par mes chants.
\VS{8}Yahweh est la force de son peuple, il est le refuge des délivrances de son oint.
\VS{9}Sauve ton peuple et bénis ton héritage ! Nourris-les et élève-les éternellement.
\Chap{29}
\TextTitle{La suprématie de Dieu}
\VerseOne{}Psaume de David. Fils de Dieu, rendez à Yahweh, rendez à Yahweh la gloire et la force\FTNT{Ps. 96:7-8.} !
\VS{2}Rendez à Yahweh la gloire due à son Nom ! Prosternez-vous devant Yahweh avec des ornements sacrés !
\VS{3}La voix de Yahweh est sur les eaux, le Dieu de gloire fait tonner ; Yahweh est sur les grandes eaux.
\VS{4}La voix de Yahweh est forte, la voix de Yahweh est majestueuse.
\VS{5}La voix de Yahweh brise les cèdres, Yahweh brise les cèdres du Liban,
\VS{6}il les fait sauter comme un veau, le Liban et le Sirion comme de jeunes buffles.
\VS{7}La voix de Yahweh fait jaillir des flammes de feu.
\VS{8}La voix de Yahweh fait trembler le désert; Yahweh fait trembler le désert de Kadès.
\VS{9}La voix de Yahweh fait naître les biches, et dépouille les forêts. Dans son palais tout s'écrie : Gloire !
\VS{10}Yahweh était assis lors du déluge ; Yahweh est assis comme roi éternellement\FTNT{Ps. 146:10.}.
\VS{11}Yahweh donne de la force à son peuple ; Yahweh bénit son peuple en paix.
\Chap{30}
\TextTitle{De la délivrance découle la louange}
\VerseOne{}Psaume. Cantique pour la dédicace de la maison de David.
\VS{2}Yahweh, je t'exalte parce que tu m'as relevé, tu n'as pas voulu que mes ennemis se réjouissent à mon sujet.
\VS{3}Yahweh, mon Dieu ! J'ai crié à toi, et tu m'as guéri.
\VS{4}Yahweh ! Tu as fait remonter mon âme du scheol, tu m'as rendu la vie, afin que je ne descende point dans la fosse.
\VS{5}Chantez à Yahweh, vous ses bien-aimés, et célébrez la mémoire de sa sainteté\FTNT{Ps. 97:12.} !
\VS{6}Car sa colère dure un instant, mais sa grâce toute la vie. Le soir arrivent les pleurs, et le matin les cris de louange.
\VS{7}Dans ma sécurité, je disais : Je ne serai jamais ébranlé\FTNT{Ps. 10:6.} !
\VS{8}Yahweh ! Par ta faveur tu avais affermi ma montagne… Tu cachas ta face, et je fus terrifié\FTNT{Ps. 13:2 ; Ps. 88:15 ; Ps. 102:3 ; Ps. 143:7.}.
\VS{9}Yahweh, j'ai crié à toi, j'ai présenté ma supplication à Yahweh :
\VS{10}Que gagnes-tu à verser mon sang si je descends dans la fosse ? La poussière te célébrera-t-elle ? Racontera-t-elle ta fidélité\FTNT{Es. 38:18.} ?
\VS{11}Yahweh, écoute, et aie pitié de moi ! Yahweh, secours-moi !
\VS{12}Tu as changé mon deuil en allégresse, tu as détaché mon sac, et tu m'as ceint de joie,
\VS{13}afin que ma langue te loue\FTNT{Ps. 57:10.} et ne se taise point. Yahweh, mon Dieu ! Je te célébrerai toujours.
\Chap{31}
\TextTitle{Recherche de la protection divine}
\VerseOne{}Psaume de David. Au chef des chantres.
\VS{2}Yahweh ! Tu es mon refuge : Que je ne sois jamais confus ! Délivre-moi par ta justice\FTNT{Ps. 25:2-20 ; Ps. 71:1-2.} !
\VS{3}Incline ton oreille vers moi, hâte-toi de me délivrer ! Sois pour moi un rocher protecteur, une forteresse, afin que je puisse m'y sauver !
\VS{4}Car tu es mon rocher, ma forteresse ; tu me dirigeras et tu me donneras du repos, à cause de ton Nom.
\VS{5}Tire-moi du filet qu'ils m'ont tendu en secret, car tu es ma vigueur.
\VS{6}Je remets mon esprit entre tes mains\FTNT{Lu. 23:46.} ; tu me rachèteras, Yahweh, Dieu de vérité !
\VS{7}Je hais ceux qui s'adonnent aux vanités trompeuses, et je me confie en Yahweh.
\VS{8}Je serai par ta bonté dans l'allégresse et dans la joie ; car tu vois mon affliction, tu sais les angoisses de mon âme,
\VS{9}tu ne m'as pas livré entre les mains de l'ennemi, mais tu feras tenir mes pieds au large.
\VS{10}Yahweh, aie pitié de moi! Car je suis dans la détresse. Mes yeux, mon âme et mon corps dépérissent de chagrin\FTNT{Ps. 6:8 ; Ps. 88:10.}.
\VS{11}Ma vie se consume dans la douleur, et mes années dans les soupirs ; ma force chancelle à cause de mon iniquité, et mes os sont consumés.
\VS{12}J'ai été un objet d'opprobre à cause de tous mes adversaires, de grand opprobre pour mes voisins, et de terreur pour ceux qui me connaissent ; ceux qui me voient dehors s'enfuient loin de moi\FTNT{Ps. 38:12 ; Job. 19:13-14.}.
\VS{13}Je suis oublié des cœurs comme un mort, je suis comme un vase détruit.
\VS{14}J'entends les calomnies de plusieurs, la crainte m'environne, quand ils se concertent unis contre moi : Ils projettent de m'ôter la vie\FTNT{Jé. 20:10.}.
\VS{15}Toutefois, je me confie en toi, ô Yahweh ! Je dis : Tu es mon Dieu !
\VS{16}Ma destinée est entre tes mains ; délivre-moi de la main de mes ennemis et de ceux qui me poursuivent !
\VS{17}Fais luire ta face sur ton serviteur\FTNT{Ps. 4:7 ; Ps. 67:2.}, délivre-moi par ta bonté !
\VS{18}Yahweh, que je ne sois point confus puisque je t'ai invoqué. Que les méchants soient confus, qu'ils soient couchés dans le scheol !
\VS{19}Que les lèvres menteuses soient muettes, elles profèrent des paroles dures contre le juste, avec orgueil et avec mépris!
\VS{20}Que ta bonté est grande\FTNT{Ps. 36:6.} ! Toi qui la réserves pour ceux qui te craignent, tu leur fais un refuge à la vue des fils de l'homme !
\VS{21}Tu les caches sous l'abri de ta face, loin du complot des hommes, tu les caches sous ton abri contre les langues querelleuses.
\VS{22}Béni soit Yahweh ! Car il a rendu merveilleuse sa bonté envers moi, comme si j'avais été dans une ville retranchée.
\VS{23}Je disais dans ma précipitation : Je suis retranché loin de ton regard ! Mais tu as entendu la voix de mes supplications quand j'ai crié vers toi.
\VS{24}Aimez Yahweh, vous tous, ses bien-aimés! Yahweh garde les fidèles, et il punit sévèrement les orgueilleux.
\VS{25}Fortifiez-vous et que votre esprit s'affermisse, espérez en Yahweh\FTNT{Ps. 27:14.} !
\Chap{32}
\TextTitle{La puissance du pardon}
\VerseOne{}Cantique de David. Heureux celui à qui la transgression est pardonnée, et dont le péché est couvert !
\VS{2}Heureux l'homme à qui Yahweh n'impute point son iniquité\FTNT{Ro. 4:6-8.}, et dans l'esprit duquel il n'y a point de fraude !
\VS{3}Quand je me suis tu, mes os se sont consumés, je n'ai fait que gémir tout le jour;
\VS{4}parce que jour et nuit ta main s'appesantissait sur moi\FTNT{Ps. 38:3.}, ma vigueur s'est changée en une sécheresse d'été. Sélah.
\VS{5}Je t'ai fait connaître mon péché et je n'ai point caché mon iniquité; j'ai dit : J'avouerai mes transgressions à Yahweh\FTNT{Pr. 28:13 ; 1 Jn 1:9.} ! Et tu as porté la peine de mon péché. Sélah.
\VS{6}Que tout fidèle te prie au temps convenable\FTNT{Es. 55:6 ; So. 2:3 ; Ps. 69:14.}! Si de grandes eaux débordent, elles ne l'atteindront point.
\VS{7}Tu es mon asile, tu me gardes de la détresse, tu m'environnes de chants de triomphe à cause de ta délivrance. Sélah.
\VS{8}Je te rendrai intelligent, je t'enseignerai la voie dans laquelle tu dois marcher ; je te guiderai, mon oeil sera sur toi.
\VS{9}Ne soyez pas comme le cheval ni comme le mulet qui sont sans intelligence ; il faut brider leur bouche avec un mors et un frein, de peur qu'ils ne s'approchent de toi\FTNT{Ja. 3:3.}.
\VS{10}Beaucoup de douleurs atteindront le méchant\FTNT{Pr. 19:29.}, mais la bonté environne l'homme qui se confie en Yahweh.
\VS{11}Vous justes, réjouissez-vous en Yahweh, soyez dans l'allégresse ! Criez de joie, vous tous qui êtes droits de cœur\FTNT{Ps. 33:1 ; Ps. 64:11.} !
\Chap{33}
\TextTitle{Louanges à Yahweh, le Dieu fidèle}
\VerseOne{}Vous justes, poussez un cri de joie à cause de Yahweh\FTNT{Ps. 32:11 ; Ps. 97:12 ; Ps. 147:1.} ! Sa louange sied aux hommes droits.
\VS{2}Célébrez Yahweh avec la harpe, chantez-le sur le luth à dix cordes!
\VS{3}Chantez-lui un cantique nouveau\FTNT{Ps. 40:4 ; Ps. 96:1 ; Ps. 98:1 ; Ps. 144:9 ; Ap. 5:9 ; Ap. 14:3.} ! Jouez de vos instruments avec un cri de réjouissance !
\VS{4}Car la parole de Yahweh est droite, et toutes ses œuvres s'accomplissent avec fidélité ;
\VS{5}il aime la justice et la droiture\FTNT{Ps. 45:8 ; Hé. 1:9.} ; la terre est remplie de la bonté de Yahweh.
\VS{6}Les cieux ont été faits par la parole de Yahweh, et toute leur armée par le souffle de sa bouche\FTNT{Ge. 2:1-2.}.
\VS{7}Il amoncelle en un tas les eaux de la mer, il met les abîmes dans des réservoirs.
\VS{8}Que toute la terre craigne Yahweh ! Que tous les habitants du monde le redoutent !
\VS{9}Car il dit, et la chose arrive ; il ordonne, et la chose se présente.
\VS{10}Yahweh rompt le conseil des nations, il anéantit les desseins des peuples ;
\VS{11}mais le conseil de Yahweh subsiste à toujours, les desseins de son cœur subsistent d'âge en âge\FTNT{Pr. 19:21.}.
\VS{12}Heureuse la nation dont Yahweh est le Dieu\FTNT{Ps. 144:15.} et le peuple qu'il s'est choisi pour héritage !
\VS{13}Yahweh regarde des cieux, il voit tous les fils des hommes\FTNT{Job. 28:24.};
\VS{14}du lieu de sa demeure, il observe tous les habitants de la terre.
\VS{15}C'est lui qui forme également leur cœur et qui prend garde à toutes leurs actions.
\VS{16}Le roi n'est point sauvé par une grande armée, l'homme puissant n'échappe point par sa grande force;
\VS{17}le cheval est impuissant pour sauver, et ne délivre point par la grandeur de sa force\FTNT{Ps. 147:10.}.
\VS{18}Voici, l'œil de Yahweh est sur ceux qui le craignent\FTNT{Ps. 34:16 ; 1 Pi. 3:12.}, sur ceux qui s'attendent à sa bonté,
\VS{19}afin qu'il les délivre de la mort, et les fasse vivre durant la famine.
\VS{20}Notre âme espère en Yahweh; il est notre aide et notre bouclier.
\VS{21}Notre cœur se réjouit en lui, car nous avons confiance en son saint Nom.
\VS{22}Que ta bonté soit sur nous, ô Yahweh ! Nous nous attendons à toi!
\Chap{34}
\TextTitle{Yahweh libère les siens}
\VerseOne{}Psaume de David, lorsqu'il contrefît l'insensé en présence d'Abimélec, qui s'en alla, chassé par lui.
\VS{2}[Aleph.] Je bénirai Yahweh en tout temps, sa louange sera continuellement dans ma bouche.
\VS{3}[Beth.] Mon âme se glorifie en Yahweh ! Que les pauvres écoutent et se réjouissent.
\VS{4}[Guimel.] Glorifiez Yahweh avec moi ! Elevons son Nom tous ensemble !
\VS{5}[Daleth.] J'ai cherché Yahweh et il m'a répondu ; il m'a délivré de toutes mes frayeurs.
\VS{6}[He. Vav.] Quand on le regarde, on est illuminé, et la face n'est point confuse.
\VS{7}[Zayin.] Cet affligé a crié et Yahweh l'a exaucé, et l'a délivré de toutes ses détresses.
\VS{8}[Heth.] L'ange de Yahweh campe tout autour de ceux qui le craignent, et les équipe.
\VS{9}[Teth.] Goûtez et voyez combien Yahweh est bon ! Heureux l'homme qui se confie en lui !
\VS{10}[Yod.] Craignez Yahweh vous ses saints ! Car rien ne manque à ceux qui le craignent.
\VS{11}[Kaf.] Les lionceaux éprouvent la disette et la faim, mais ceux qui cherchent Yahweh ne manquent d'aucun bien.
\VS{12}[Lamed.] Venez, mes fils, écoutez-moi ! Je vous enseignerai la crainte de Yahweh.
\VS{13}[Mem.] Qui est l'homme qui prend plaisir à la vie, qui aime la prolonger pour jouir du bonheur ?
\VS{14}[Nun.] Garde ta langue du mal et tes lèvres des paroles trompeuses\FTNT{1 Pi. 3:10.} ;
\VS{15}[Samech.] détourne-toi du mal et fais-le bien ; cherche la paix et poursuis-la\FTNT{Hé. 12:14.}.
\VS{16}[Ayin.] Les yeux de Yahweh sont sur les justes et ses oreilles sont attentives à leur cri.
\VS{17}[Pe.] La face de Yahweh est contre ceux qui font le mal, pour retrancher de la terre leur mémoire\FTNT{Jé. 44:11 ; Lé. 17:10.}.
\VS{18}[Tsade.] Quand les justes crient, Yahweh les exauce et il les délivre de toutes leurs détresses.
\VS{19}[Qof.] Yahweh est près de ceux qui ont le cœur déchiré par la douleur, et il délivre ceux qui ont l'esprit abattu.
\VS{20}[Resh.] Le juste a des maux en grand nombre, mais Yahweh le délivre de tous\FTNT{2 Ti. 3:11.}.
\VS{21}[Shin.] Il garde tous ses os, aucun d'eux n'est brisé.
\VS{22}[Tav.] Le mauvais tue le méchant, et ceux qui haïssent le juste sont détruits.
\VS{23}[Pe.] Yahweh rachète l'âme de ses serviteurs, et aucun de ceux qui se confient en lui ne sera détruit.
\Chap{35}
\TextTitle{Prière au juste Juge}
\VerseOne{}Psaume de David. Yahweh, défends-moi contre mes adversaires, combats ceux qui me combattent !
\VS{2}Prends le petit et le grand bouclier, et lève-toi pour me secourir !
\VS{3}Brandis la lance et le javelot contre mes persécuteurs ! Dis à mon âme : Je suis ta délivrance !
\VS{4}Que ceux qui en veulent à ma vie soient honteux et confus\FTNT{Jé. 17:18 ; Ps. 40:15 ; Ps. 70:3.} ! Que ceux qui méditent ma perte reculent et rougissent !
\VS{5}Qu'ils soient comme la balle emportée par le vent\FTNT{Es. 29:5 ; Os. 13:3.}, et que l'Ange de Yahweh les chasse !
\VS{6}Que leur chemin soit ténébreux et glissant, et que l'Ange de Yahweh les poursuive.
\VS{7}Car sans cause ils m'ont tendu leur filet sur une fosse, sans cause ils l'ont creusée pour m'ôter la vie\FTNT{Jé. 18:20 ; Ps. 57:7 ; Ps. 140:5 ; Ps. 141:9.}.
\VS{8}Que la ruine les atteigne sans qu'ils le sachent, qu'ils soient capturés dans le filet qu'ils ont caché. Qu'ils y tombent et soient ravagés !
\VS{9}Mon âme aura de la joie en Yahweh, de l'allégresse en sa délivrance.
\VS{10}Tous mes os diront : Yahweh ! Qui est semblable à toi ? Qui délivre l'affligé de la main de celui qui est plus fort que lui ? L'affligé et le pauvre de celui qui le pille ?
\VS{11}De faux témoins s'élèvent contre moi : On m'interroge sur ce que j'ignore.
\VS{12}Ils me rendent le mal pour le bien, tâchant de m'ôter la vie\FTNT{Ps. 38:21 ; Ps. 109:5.}.
\VS{13}Mais moi, quand ils étaient malades, je me couvrais d'un sac, j'affligeais mon âme par le jeûne, je priais dans mon sein,
\VS{14}comme pour un ami, pour un frère, j'étais abattu, en pleurs, comme pour le deuil d'une mère.
\VS{15}Mais quand je chancelle, ils se réjouissent et s'assemblent, ils s'assemblent contre moi sans que je le sache pour me frapper, ils me déchirent pour que je sois silencieux ;
\VS{16}avec les hypocrites d’entre les railleurs qui suivent les bonnes tables, et ils ont grincé les dents contre moi.
\VS{17}Seigneur ! Jusqu'à quand le verras-tu ? Détourne mon âme de leurs tempêtes, mon unique des lionceaux.
\VS{18}Je te célébrerai dans la grande assemblée, je te louerai parmi un peuple nombreux\FTNT{Ps. 111:1.}.
\VS{19}Que ceux qui sont mes ennemis par leur mensonge ne se réjouissent point de moi, que ceux qui me haïssent sans cause ne m'insultent point par leurs regards\FTNT{Jn. 15:25.}.
\VS{20}Car ils ne parlent point de paix, mais ils préméditent des choses pleines de fraudes contre les gens tranquilles de la terre.
\VS{21}Ils ont ouvert leur bouche autant qu'ils ont pu contre moi, et ont dit : Ah ! Ah ! Nos yeux l'ont vu !
\VS{22}Yahweh ! Tu le vois : Ne te tais point\FTNT{Ps. 83:2.} ! Seigneur, ne t'éloigne point de moi !
\VS{23}Réveille-toi, réveille-toi pour me rendre justice\FTNT{Ps. 44:24.} ! Mon Dieu et mon Seigneur, défends ma cause !
\VS{24}Juge-moi selon ta justice, Yahweh mon Dieu ! et qu'ils ne se réjouissent point de moi !
\VS{25}Qu'ils ne disent point dans leur cœur : Ah ! Notre âme ! Et qu'ils ne disent point : Nous l'avons englouti !
\VS{26}Que ceux qui se réjouissent de mon mal soient honteux et rougissent tous ensemble ! Que ceux qui s'élèvent contre moi soient couverts de honte et de confusion !
\VS{27}Mais que ceux qui prennent plaisir à ma justice se réjouissent avec des chants de triomphe, qu'ils disent sans cesse : Grand est Yahweh qui désire la paix de son serviteur !
\VS{28}Alors ma langue publiera ta justice et ta louange tous les jours.
\Chap{36}
\TextTitle{Opposition : Les justes et les méchants}
\VerseOne{}Psaume de David, serviteur de Yahweh, donné au chef des chantres.
\VS{2}La transgression du méchant me dit, au dedans de mon cœur, qu'il n'y a point de crainte de Dieu devant ses yeux.
\VS{3}Car il se flatte à ses propres yeux pour consumer, pour assouvir sa haine.
\VS{4}Les paroles de sa bouche ne sont que méchanceté et tromperie, il cesse d'être sage et de faire le bien.
\VS{5}Il projette le malheur sur sa couche, il se tient sur un chemin qui n'est pas bon, il ne rejette pas le mal.
\VS{6}Yahweh ! Ta bonté atteint jusqu'aux cieux, ta fidélité jusqu'aux nues\FTNT{Ps. 57:11 ; Ps. 108:5.}.
\VS{7}Ta justice est comme les montagnes de Dieu, tes jugements sont un grand abîme. Yahweh ! Tu sauves les hommes et les bêtes.
\VS{8}Ô Dieu ! Combien est précieuse ta bonté ! Aussi les fils des hommes se retirent à l'ombre de tes ailes\FTNT{Ps. 17:8 ; Ps. 57:2.}.
\VS{9}Ils seront abondamment rassasiés de la graisse de ta maison et tu les abreuveras au fleuve de tes délices.
\VS{10}Car la source de la vie est auprès de toi, et par ta lumière nous voyons la lumière.
\VS{11}Etends ta bonté sur ceux qui te connaissent, et ta justice sur ceux qui ont le cœur droit !
\VS{12}Que le pied de l'orgueilleux ne s'avance point sur moi, et que la main des méchants ne m'ébranle point !
\VS{13}Là sont tombés les ouvriers d'iniquité ; ils sont renversés et ne peuvent se relever.
\Chap{37}
\TextTitle{Se confier en la justice de Yahweh}
\VerseOne{}Psaume de David. [Aleph.] Ne t'irrite pas contre les méchants, ne jalouse pas ceux qui s'adonnent à la perversité\FTNT{Pr. 23:17 ; Pr. 24:19.}.
\VS{2}Car ils seront soudainement retranchés comme le foin, et ils se faneront comme l'herbe verte.
\VS{3}[Beth.] Confie-toi en Yahweh, et fais ce qui est bon ; aie le pays pour demeure et la fidélité pour pâture.
\VS{4}Fais de Yahweh tes délices et il t'accordera ce que ton cœur désire.
\VS{5}[Guimel.] Recommande tes voies à Yahweh, confie-toi en lui et il agira\FTNT{Ps. 22:9 ; Ps. 55:23 ; Pr. 16:3.}.
\VS{6}Il manifestera ta justice comme la lumière et ton droit comme le soleil à son midi\FTNT{Pr. 4:18.}.
\VS{7}[Daleth.] Garde le silence devant Yahweh et tremble devant lui ; ne t'irrite point contre celui qui réussit dans ses voies, contre celui qui vient à bout de ses mauvais desseins.
\VS{8}[He.] Laisse la colère et abandonne la rage\FTNT{Ep. 4:26.} ; ne t'irrite pas pour faire le mal.
\VS{9}Car les méchants seront retranchés, mais ceux qui se confient en Yahweh hériteront la terre.
\VS{10}[Vav.] Encore un peu de temps et le méchant ne sera plus ; tu regardes le lieu où il était et il n'y est plus\FTNT{Job. 7:10 ; Job. 20:9.}.
\VS{11}Les pauvres prennent possession du pays et jouissent abondamment de la paix.
\VS{12}[Zayin.] Le méchant complote contre le juste et grince ses dents contre lui.
\VS{13}Le Seigneur se rit de lui, car il voit que son jour approche.
\VS{14}[Heth.] Les méchants tirent leur épée et bandent leur arc pour faire tomber le malheureux et le pauvre, pour massacrer ceux qui marchent dans la droiture\FTNT{Ps. 11:2.}.
\VS{15}Mais leur épée entre dans leur propre cœur, et leurs arcs se brisent.
\VS{16}[Teth.] Mieux vaut au juste le peu qu'il a, que l'abondance de beaucoup de méchants\FTNT{Pr. 15:16-17 ; Ec. 4:6.} ;
\VS{17}car les bras des méchants seront brisés, mais Yahweh soutient les justes.
\VS{18}[Yod.] Yahweh connaît les jours de ceux qui sont intègres, et leur héritage demeure à jamais.
\VS{19}Ils ne sont pas honteux au jour du malheur, mais ils sont rassasiés au jour de la famine.
\VS{20}[Kaf.] Mais les méchants périssent, et les ennemis de Yahweh, comme les beaux pâturages, s'évanouissent, ils s'évanouissent en fumée.
\VS{21}[Lamed.] Le méchant emprunte et ne rend point ; mais le juste a compassion et donne.
\VS{22}Car les bénis de Yahweh hériteront la terre, mais ceux qu'il a maudits seront retranchés.
\VS{23}[Mem.] Yahweh affermit les pas de l'homme, et il prend plaisir à ses voies.
\VS{24}S'il tombe, il ne sera pas entièrement abattu, car Yahweh le soutient de sa main.
\VS{25}[Nun.] J'ai été jeune et j'ai vieilli ; et je n'ai point vu le juste abandonné, ni sa postérité mendiant son pain.
\VS{26}Il est compatissant tout le temps, et il prête ; et sa postérité est bénie.
\VS{27}[Samech.] Retire-toi du mal et fais le bien ; et tu auras une demeure éternelle.
\VS{28}Car Yahweh aime ce qui est juste, et il n'abandonne point ses fidèles ; c'est pourquoi ils sont sous sa garde pour toujours, mais la postérité des méchants est retranchée.
\VS{29}[Ayin.] Les justes hériteront la terre et y habiteront à perpétuité.
\VS{30}[Pe.] La bouche du juste prononce la sagesse et sa langue déclare la justice.
\VS{31}La loi de son Dieu est dans son cœur\FTNT{Ps. 40:8-9.}, aucun de ses pas ne chancellera.
\VS{32}[Tsade.] Le méchant épie le juste et cherche à le faire mourir.
\VS{33}Yahweh ne l'abandonne point entre ses mains et ne le laisse point condamner quand on le juge.
\VS{34}[Qof.] Espère en Yahweh et garde sa voie, et il t'élèvera pour que tu hérites la terre ; tu verras les méchants retranchés.
\VS{35}[Resh.] J'ai vu le méchant dans toute sa puissance, il s'étendait comme un arbre verdoyant.
\VS{36}Il a passé, et voici, il n'est plus ; je le cherche et il ne se trouve plus.
\VS{37}[Shin.] Observe l'homme intègre et considère l'homme droit, car il y a une issue pour l'homme de paix.
\VS{38}Mais les rebelles seront tous détruits et ce qui sera resté des méchants sera retranché.
\VS{39}[Tav.] Mais la délivrance des justes viendra de Yahweh, il sera leur force au temps de la détresse.
\VS{40}Yahweh les secourt et les délivre ; il les délivre des méchants et les sauve, parce qu'ils se confient en lui.
\Chap{38}
\TextTitle{La tristesse du péché mène à la repentance}
\VerseOne{}Psaume de David. Pour souvenir.
\VS{2}Yahweh ! Ne me juge pas dans ta colère et ne me châtie pas dans ta fureur.
\VS{3}Car tes flèches m'ont atteint et ta main s'est appesantie sur moi.
\VS{4}Il n'y a rien de sain dans ma chair, à cause de ta colère, ni de paix dans mes os, à cause de mon péché.
\VS{5}Car mes iniquités s'élèvent au-dessus de ma tête, elles se sont appesanties comme un pesant fardeau, au-delà de mes forces\FTNT{Ps. 40:13.}.
\VS{6}Mes plaies ont une mauvaise odeur et sont purulentes à cause de ma folie.
\VS{7}Je suis courbé et abattu outre mesure ; je marche en pleurs tout le jour.
\VS{8}Car un mal brûlant remplit mes reins, et dans ma chair il n'y a rien de sain.
\VS{9}Je suis affaibli et brisé, je rougis le cœur troublé.
\VS{10}Seigneur, tout mon désir est devant toi, et mon soupir ne t'est point caché.
\VS{11}Mon cœur est agité çà et là, ma force m'abandonne, et la lumière de mes yeux n'est plus avec moi.
\VS{12}Ceux qui m'aiment, et même mes amis intimes, se tiennent loin de ma plaie, et mes proches se tiennent loin de moi\FTNT{Job. 19:13-14.}.
\VS{13}Ceux qui en veulent à ma vie me tendent des pièges ; ceux qui cherchent ma perte parlent de calamités et méditent des tromperies tous les jours.
\VS{14}Mais moi je suis comme un sourd, comme un muet qui n'ouvre point sa bouche.
\VS{15}Je suis, dis-je, comme un homme qui n'entend pas et qui n'a point de réplique dans sa bouche.
\VS{16}Car je m'attends à toi, ô Yahweh ! Tu me répondras, Seigneur mon Dieu !
\VS{17}Je dis : Il faut prendre garde qu'ils ne triomphent de moi ; quand mon pied chancelle, ils s'élèvent contre moi\FTNT{Ps. 94:18.} !
\VS{18}Car je suis près de tomber et ma douleur est continuellement devant moi.
\VS{19}Car je reconnais mon iniquité et je suis dans la crainte à cause de mon péché.
\VS{20}Cependant mes ennemis qui sont vivants se renforcent, et ceux qui me haïssent à tort se multiplient.
\VS{21}Ceux qui me rendent le mal pour le bien sont mes adversaires, parce que je recherche le bien\FTNT{Ps. 109:5 ; Jé. 18:20.}.
\VS{22}Ne m'abandonne pas Yahweh ! Mon Dieu, ne t'éloigne pas de moi !
\VS{23}Hâte-toi de venir à mon secours, Seigneur, tu es ma délivrance !
\Chap{39}
\TextTitle{La faiblesse de l'homme}
\VerseOne{}Psaume de David, donné au chef des chantres, à Jeduthun.
\VS{2}J'ai dit : Je prends garde à mes voies, de peur de pécher par ma langue ; je mettrai un frein à ma bouche tant que le méchant sera devant moi.
\VS{3}Je suis resté muet, dans le silence ; je me suis tu, quoique malheureux ; et ma douleur n'était pas moins vive.
\VS{4}Mon cœur brûlait au-dedans de moi, un feu intérieur me consumait, et la parole est venue sur ma langue.
\VS{5}Yahweh ! Dis-moi quel est le terme de ma vie et quelle est la mesure de mes jours\FTNT{Ps. 119:84.} ; que je sache combien je suis fragile.
\VS{6}Voici, tu as réduit mes jours à la largeur de ma main, et ma vie est comme un rien devant toi. Oui, tout homme debout n'est qu'un souffle\FTNT{Ja. 4:14.}. Sélah.
\VS{7}Oui, l'homme se promène comme une ombre, il s'agite inutilement ; il amasse des biens et il ne sait pas qui les recueillera.
\VS{8}Maintenant que puis-je espérer, Seigneur ? Mon espérance est en toi.
\VS{9}Délivre-moi de toutes mes transgressions ! Ne permets pas l'opprobre des insensés.
\VS{10}Je me suis tu et je n'ai point ouvert ma bouche, parce que c'est toi qui agis.
\VS{11}Détourne de moi tes coups ! Je suis consumé par les attaques de ta main.
\VS{12}Aussitôt que tu châties quelqu'un, en le punissant à cause de son iniquité, tu détruis comme la teigne ce qu'il a de plus cher. Oui, tout homme est une vapeur. Sélah.
\VS{13}Yahweh, écoute ma prière et prête l'oreille à mon cri ! Ne sois point sourd à mes larmes ! Car je suis un voyageur et un étranger chez toi, comme tous mes pères\FTNT{Lé. 25:23 ; Ps. 119:19 ; 1 Pi. 2:11 ; Hé. 11:13.}.
\VS{14}Détourne ton regard de moi, afin que je reprenne mes forces, avant que je m'en aille et que je ne sois plus.
\Chap{40}
\TextTitle{Un cantique nouveau à Yahweh}
\VerseOne{}Psaume de David, donné au chef des chantres.
\VS{2}J'ai attendu patiemment Yahweh, et il s'est tourné vers moi et a entendu mon cri.
\VS{3}Il m'a retiré de la fosse de destruction, du fond de la boue ; il a mis mes pieds sur un roc et a assuré mes pas.
\VS{4}Il a mis dans ma bouche un cantique nouveau, qui est la louange de notre Dieu ; plusieurs verront cela et ils craindront et se confieront en Yahweh.
\VS{5}Heureux l'homme qui place sa confiance en Yahweh et qui ne se tourne pas vers les orgueilleux et les menteurs !
\VS{6}Yahweh, mon Dieu ! Tu as multiplié tes merveilles et tes desseins envers nous ; nul n'est comparable à toi ; je voudrais les annoncer et les déclarer, mais leur nombre est trop grand pour que je les raconte.
\VS{7}Tu ne désires ni sacrifice ni offrande. Tu m'as percé les oreilles ; tu ne demandes ni holocauste ni victime expiatoire pour le péché\FTNT{Hé. 10:5.}.
\VS{8}Alors je dis : Voici, je viens avec le rouleau du livre écrit pour moi.
\VS{9}Mon Dieu, je prends plaisir à faire ta volonté, et ta loi est au fond de mes entrailles\FTNT{Ps. 37:31 ; Es. 51:7.}.
\VS{10}J'annonce ta justice dans la grande assemblée ; voilà, je ne ferme pas mes lèvres, Yahweh, tu le sais !
\VS{11}Je ne cache pas ta justice, qui est dans mon cœur ; je déclare ta fidélité et ta délivrance ; je ne cache pas ta bonté ni ta vérité dans la grande assemblée.
\VS{12}Toi, Yahweh ! Ne m'épargne point tes compassions, que ta bonté et ta vérité me gardent continuellement.
\VS{13}Car des maux sans nombre m'environnent ; mes iniquités m'atteignent et je ne supporte pas leur vue ; elles surpassent en nombre les cheveux de ma tête, et mon cœur m'abandonne.
\VS{14}Yahweh, veuille me délivrer ! Yahweh, hâte-toi de venir à mon secours !
\VS{15}Que tous ensemble ils soient honteux et confus, ceux qui cherchent mon âme pour la perdre ; et que ceux qui prennent plaisir à mon malheur retournent en arrière et rougissent.
\VS{16}Que ceux qui disent de moi : Ah ! Ah ! Soient consumés, en récompense de la honte qu'ils m'ont faite.
\VS{17}Que tous ceux qui te cherchent soient dans l'allégresse et se réjouissent en toi\FTNT{Ps. 70:5.} ! Que ceux qui aiment ta délivrance disent continuellement : Grand est Yahweh !
\VS{18}Moi, je suis affligé et misérable, mais le Seigneur prend soin de moi. Tu es mon secours et mon libérateur : Mon Dieu ne tarde point\FTNT{Ps. 70:6.} !
\Chap{41}
\TextTitle{Secours de Yahweh dans le malheur}
\VerseOne{}Psaume de David, donné au chef des chantres.
\VS{2}Heureux celui qui s'intéresse au pauvre ! Yahweh le délivrera au jour du malheur ;
\VS{3}Yahweh le garde et lui conserve la vie. Il est heureux sur la terre, et tu ne le livres pas au bon plaisir de ses ennemis.
\VS{4}Yahweh le soutient sur son lit de douleur ; tu le soulages dans toutes ses maladies.
\VS{5}Je dis : Yahweh ! Aie pitié de moi, guéris mon âme, car j'ai péché contre toi.
\VS{6}Mes ennemis disent du mal de moi : Quand mourra-t-il ? Et quand périra son nom ?
\VS{7}Si quelqu'un vient me voir, il dit des mensonges, il recueille de mauvais desseins\FTNT{Ps. 5:10 ; Ps. 10:7 ; Ps. 12:3.}, il s'en va et il parle au-dehors.
\VS{8}Tous ceux qui m'ont en haine murmurent sourdement ensemble contre moi, et machinent du mal contre moi.
\VS{9}Quelque action criminelle\FTNT{Le mot « criminelle » donne en hébreu « Belial »} pèse sur lui ; le voilà couché, disent–ils, il ne se relèvera plus !
\VS{10}Même celui qui était en paix avec moi, qui avait ma confiance et qui mangeait mon pain, a levé le talon contre moi\FTNT{Il est question ici de la trahison du Messie par Judas (Jn. 13:18-19).}.
\VS{11}Mais toi, ô Yahweh ! Aie pitié de moi et relève-moi ! Et je leur rendrai ce qui leur est dû.
\VS{12}Je connaîtrai que tu prends plaisir en moi, si mon ennemi ne triomphe pas de moi.
\VS{13}Pour moi, tu m'as soutenu à cause de mon intégrité, et tu m'as établi pour toujours en ta présence.
\VS{14}Béni soit Yahweh, le Dieu d'Israël, d'éternité en éternité. Amen ! Amen !
\Chap{42}
\TextTitle{Avoir soif de Dieu}
\VerseOne{}Cantique des fils de Koré, donné au chef des chantres.
\VS{2}Comme une biche soupire après des courants d'eau, ainsi mon âme soupire ardemment après toi, ô Dieu !
\VS{3}Mon âme a soif de Dieu, du Dieu vivant\FTNT{Ps. 63:2 ; Ps. 84:3.} : Ô quand entrerai-je et me présenterai-je devant la face de Dieu ?
\VS{4}Mes larmes sont ma nourriture jour et nuit, quand on me dit chaque jour : Où est ton Dieu\FTNT{Ps. 80:6 ; Ps. 115:2.} ?
\VS{5}Je rappelais ces choses dans mon souvenir, en répandant mon âme au-dedans de moi, savoir que je marchais dans la foule, et que je m'en allais tout doucement en leur compagnie, avec une voix de triomphe et de louange, jusqu'à la maison de Dieu, et qu'une grande multitude de gens sautait alors de joie.
\VS{6}Mon âme, pourquoi t'abats-tu et murmures-tu au-dedans de moi ? Attends-toi à Dieu, car je le célébrerai encore ; sa face est la délivrance même.
\VS{7}Mon Dieu ! Mon âme est abattue au-dedans de moi, aussi je me souviens de toi depuis la terre du Jourdain, depuis l'Hermon, et depuis la montagne de Mitsear.
\VS{8}Un flot appelle un autre flot au bruit de tes ondées ; toutes tes vagues et tes flots passent sur moi.
\VS{9}Toutefois, Yahweh enverra sa bonté compatissante de jour, et de nuit son cantique sera avec moi, et ma prière sera au Dieu qui est ma vie.
\VS{10}Je dis à Dieu, mon rocher : Pourquoi m'oublies-tu ? Pourquoi marcherai-je dans la tristesse à cause de l'oppression de l'ennemi ?
\VS{11}Comme avec une épée dans mes os, mes ennemis m'outragent, tandis qu'ils me disent chaque jour : Où est ton Dieu ?
\VS{12}Mon âme, pourquoi t'abats-tu et pourquoi murmures-tu au-dedans de moi ? Attends-toi à Dieu, car je le célébrerai encore, il est ma délivrance et mon Dieu.
\Chap{43}
\TextTitle{Espérer dans la délivrance de Dieu}
\VerseOne{}Fais-moi justice, ô Dieu ! Et défends ma cause contre une nation infidèle\FTNT{Ps. 26:1 ; Ps. 35:1.} ! Délivre-moi de l'homme trompeur et pervers.
\VS{2}Toi, mon Dieu protecteur, pourquoi me repousses-tu ? Pourquoi marcherai-je dans la tristesse à cause de l'oppression de l'ennemi ?
\VS{3}Envoie ta lumière et ta vérité, afin qu'elles me conduisent et m'introduisent dans ta sainte montagne, et dans tes demeures.
\VS{4}Alors je viendrai à l'autel de Dieu, au Dieu de ma joie et mon allégresse, et je te célébrerai sur la harpe, ô Dieu ! Mon Dieu !
\VS{5}Mon âme, pourquoi t'abats-tu et pourquoi murmures-tu au-dedans de moi ? Attends-toi à Dieu, car je le célébrerai encore ; il est ma délivrance et mon Dieu\FTNT{Ps. 42:6.}.
\Chap{44}
\TextTitle{Prière des affligés}
\VerseOne{}Cantique des fils de Koré, donné au chef des chantres.
\VS{2}Ô Dieu ! Nous avons entendu de nos oreilles et nos pères nous ont raconté les exploits que tu as faits de leur temps, aux jours d'autrefois\FTNT{Jg. 6:13 ; Ps. 77:12.}.
\VS{3}Tu as de ta main chassé les nations, et tu as affermi nos pères, tu as affligé les peuples, et tu as fait prospérer nos pères.
\VS{4}Car ce n'est point par leur épée qu'ils ont conquis le pays, et ce n'est pas leur bras qui les a délivrés, mais c'est ta droite, c'est ton bras, c'est la lumière de ta face, parce que tu les aimais.
\VS{5}Ô Dieu ! Tu es mon Roi : Ordonne la délivrance de Jacob !
\VS{6}Avec toi nous battrons nos adversaires, par ton Nom nous foulerons ceux qui s'élèvent contre nous.
\VS{7}Car je ne me confie point en mon arc, et ce n'est pas mon épée qui me délivrera.
\VS{8}Mais tu nous délivreras de nos adversaires, et tu rendras confus ceux qui nous haïssent.
\VS{9}Nous nous glorifierons en Dieu chaque jour et nous célébrerons à jamais ton Nom. Sélah.
\VS{10}Mais tu nous rejettes, tu nous confonds, et tu ne sors plus avec nos armées.
\VS{11}Tu nous fais reculer devant l'adversaire, et ceux qui nous haïssent enlèvent nos dépouilles.
\VS{12}Tu nous livres comme des brebis destinées à être dévorées, et tu nous as dispersés parmi les nations.
\VS{13}Tu as vendu ton peuple pour rien, et tu ne l'estimes pas d'une grande valeur\FTNT{Es. 52:3 ; Jé. 15:13.}.
\VS{14}Tu nous as mis en opprobre chez nos voisins, en dérision, et en sujet de moquerie auprès de ceux qui habitent autour de nous\FTNT{Jé. 24:9 ; Ps. 79:4.}.
\VS{15}Tu fais de nous un objet de sarcasmes parmi les nations, et de hochement de tête parmi les peuples.
\VS{16}Ma confusion est tout le jour devant moi, et la honte couvre ma face,
\VS{17}à cause des discours de celui qui nous fait des reproches et qui nous injurie, et à cause de l'ennemi et du vindicatif.
\VS{18}Tout cela nous est arrivé, et cependant nous ne t'avons point oublié, et nous n'avons point violé ton alliance.
\VS{19}Notre cœur ne s'est point détourné, nos pas ne se sont point éloignés de tes sentiers,
\VS{20}pour que tu nous écrases dans le lieu du serpent, et que tu nous couvres de l'ombre de la mort\FTNT{Ps. 23:4.}.
\VS{21}Si nous avions oublié le Nom de notre Dieu et étendu nos mains vers un dieu étranger,
\VS{22}Dieu ne le saurait-il pas, lui qui connaît les secrets du cœur ?
\VS{23}Mais nous sommes tous les jours mis à mort pour l'amour de toi, nous sommes regardés comme des brebis destinées à la boucherie\FTNT{Es. 53:7.}.
\VS{24}Lève-toi, pourquoi dors-tu Seigneur ? Réveille-toi ! Ne nous rejette point à jamais !
\VS{25}Pourquoi caches-tu ta face, pourquoi oublies-tu notre affliction et notre oppression ?
\VS{26}Car notre âme est abattue dans la poussière et notre ventre est attaché à la terre.
\VS{27}Lève-toi pour nous secourir ! Délivre-nous à cause de ta bonté.
\Chap{45}
\TextTitle{La beauté du Roi}
\VerseOne{}Cantique des fils de Koré, qui est un chant nuptial donné au chef des chantres pour le chanter sur Shoshannim.
\VS{2}Des paroles agréables bouillonnent dans mon cœur, et j'ai dit : Mon œuvre est pour le roi ! Ma langue sera comme la plume d'un habile écrivain !\FTNT{Ca. 5:13 ; Ca. 5:16.}
\VS{3}Tu es le plus beau des fils de l'homme, la grâce est répandue sur tes lèvres : C'est pourquoi Dieu t'a béni éternellement.
\VS{4}Ô héros, ceins ton épée sur ta cuisse, ta majesté et ta magnificence,
\VS{5}et prospère dans ta magnificence. Sois porté sur la parole de vérité, de douceur, et de justice, et ta droite versera des choses terribles !
\VS{6}Tes flèches sont aiguës, les peuples tomberont sous toi, elles perceront le cœur des ennemis du roi.
\VS{7}Ô Dieu, ton trône est à toujours et à perpétuité ! Le sceptre de ton règne est un sceptre d'équité.
\VS{8}Tu aimes la justice et tu hais la méchanceté : C'est pourquoi, ô Dieu, ton Dieu t'a oint d'une huile de joie par privilège sur tes compagnons\FTNT{Hé. 1:8-9.}.
\VS{9}Tous tes vêtements sont parfumés de myrrhe, d'aloès et de casse. Dans les palais d'ivoire les instruments à cordes te réjouissent.
\VS{10}Des filles de rois sont parmi tes bien-aimées ; la reine est à ta droite, parée d'or d'Ophir.
\VS{11}Ecoute, jeune fille, vois et prête l'oreille ; oublie ton peuple et la maison de ton père.
\VS{12}Le roi porte ses désirs sur ta beauté ; puisqu'il est ton Seigneur, prosterne-toi devant lui.
\VS{13}La fille de Tyr et les plus riches des peuples te supplieront avec des présents.
\VS{14}La fille du roi est intérieurement pleine de gloire. Elle porte un vêtement tissé d'or.
\VS{15}Elle sera présentée au roi en vêtements de broderie, et les filles qui viennent après elle, et qui sont ses compagnes, seront amenées vers toi.
\VS{16}Elles te seront présentées avec réjouissance et allégresse, et elles entreront au palais du roi.
\VS{17}Tes fils seront au lieu de tes pères, tu les établiras pour princes sur toute la terre.
\VS{18}Je rendrai ton Nom mémorable dans tous les âges, et à cause de cela les peuples te célébreront pour toujours et à perpétuité\FTNT{Ps. 67:3-5.}.
\Chap{46}
\TextTitle{L'assurance du peuple de Dieu}
\VerseOne{}Cantique des fils de Koré, donné au chef des chantres pour le chanter sur Alamoth. Cantique.
\VS{2}Dieu est notre retraite, notre force, et notre secours qui ne manque jamais dans les détresses\FTNT{Ps. 9:10.}.
\VS{3}C'est pourquoi nous ne craindrons point quand la terre est bouleversée et que les montagnes chancellent au cœur des mers\FTNT{Es. 54:10.},
\VS{4}quand ses eaux mugissent et écument, se soulèvent jusqu'à faire trembler les montagnes. Sélah.
\VS{5}Il est un fleuve dont les courants réjouissent la cité de Dieu, le lieu saint des demeures du Très-Haut\FTNT{Ez. 47:1-2 ; Za. 14:8-9 ; Jn. 7:38 ; Ap. 22:1-2.}.
\VS{6}Dieu est au milieu d'elle : Elle n'est point ébranlée. Dieu la secourt dès le point du jour\FTNT{So. 3:16-17.}.
\VS{7}Les nations murmurent, les royaumes s'ébranlent ; il a fait entendre sa voix et la terre se fond.
\VS{8}Yahweh des armées est avec nous, le Dieu de Jacob est pour nous une haute retraite. Sélah.
\VS{9}Venez, contemplez les œuvres de Yahweh et voyez quels ravages il a faits sur la terre.
\VS{10}Il a fait cesser les guerres jusqu'au bout de la terre, il a brisé l'arc et rompu la lance, il a consumé par le feu les chars de guerre\FTNT{Es. 2:4.}.
\VS{11}Arrêtez, et sachez que je suis Dieu : Je suis élevé parmi les nations, je suis élevé sur toute la terre.
\VS{12}Yahweh des armées est avec nous, le Dieu de Jacob est pour nous une haute retraite. Sélah.
\Chap{47}
\TextTitle{Yahweh, le Dieu élevé}
\VerseOne{}Psaume des fils de Koré, donné au chef des chantres.
\VS{2}Peuples battez tous des mains ! Poussez vers Dieu des cris de joie avec une voix de triomphe !
\VS{3}Car Yahweh, le Très-Haut, est terrible. Il est un grand Roi sur toute la terre.
\VS{4}Il nous assujettit des peuples et des nations sous nos pieds.
\VS{5}Il nous choisit notre héritage, la gloire de Jacob qu'il aime. Sélah.
\VS{6}Dieu est monté avec un cri de réjouissance, Yahweh monte au son du shofar.
\VS{7}Chantez à Dieu, chantez ! Chantez à notre Roi, chantez !
\VS{8}Car Dieu est le Roi de toute la terre : Chantez un cantique !
\VS{9}Dieu règne sur les nations, Dieu est assis sur son saint trône.
\VS{10}Les princes des peuples se rassemblent vers le peuple du Dieu d'Abraham, car les boucliers de la terre sont à Dieu : Il est puissamment élevé.
\Chap{48}
\TextTitle{Sion, splendeur du grand Roi}
\VerseOne{}Cantique. Psaume des fils de Koré.
\VS{2}Yahweh est grand, il est l'objet de toutes les louanges dans la ville de notre Dieu, sur sa montagne sainte.
\VS{3}Belle est la colline, joie de toute la terre, la montagne de Sion, le côté nord, c'est la ville du grand Roi.
\VS{4}Dieu est connu dans ses palais pour une haute retraite.
\VS{5}Ils l'ont vue, et aussitôt ils ont été émerveillés ; ils ont été troublés et se sont enfuis à la hâte.
\VS{6}Ils ont regardé tout stupéfaits, ils ont eu peur et ont pris la fuite.
\VS{7}Là un tremblement les a saisis, une douleur comme celle de l'enfantement\FTNT{Es. 13:8.}.
\VS{8}Ils ont été chassés comme par le vent d'orient qui brise les navires de Tarsis.
\VS{9}Comme nous l'avions entendu, ainsi l'avons-nous vu dans la ville de Yahweh des armées, dans la ville de notre Dieu : Dieu l'établira à toujours. Sélah.
\VS{10}Ô Dieu ! Nous pensons à ta bonté au milieu de ton temple.
\VS{11}Ô Dieu ! Comme ton Nom, ta louange retentit jusqu'aux extrémités de la terre ; ta droite est pleine de justice.
\VS{12}La montagne de Sion se réjouit, et les filles de Juda sont dans la joie, à cause de tes jugements.
\VS{13}Entourez Sion, faites-en le tour, comptez ses tours.
\VS{14}Observez son rempart, examinez ses palais pour le raconter à la génération future.
\VS{15}Car ce Dieu-là est notre Dieu éternellement et à jamais ; il nous accompagnera jusqu'à la mort.
\Chap{49}
\TextTitle{Vanité des richesses terrestres}
\VerseOne{}Psaume des fils de Koré, au chef des chantres.
\VS{2}Vous tous peuples, entendez ceci, vous habitants du monde, prêtez l'oreille,
\VS{3}petits et grands, riches et pauvres !
\VS{4}Ma bouche prononcera des discours pleins de sagesse, et les pensées de mon cœur sont pleines de sens.
\VS{5}Je prête l'oreille aux sentences qui me sont inspirées, je poserai mes questions au son de la harpe.
\VS{6}Pourquoi craindrai-je au jour du malheur, quand l'iniquité de mes adversaires m'entoure ?
\VS{7}Ils mettent leur confiance dans leurs biens et se glorifient de l'abondance de leurs richesses.
\VS{8}Ils ne peuvent se racheter l'un l'autre ni donner à Dieu le prix du rachat\FTNT{Mt. 16:26 ; Mc. 8:36-37 ; Lu. 12:15-21.}.
\VS{9}Car le rachat de leur âme est trop considérable, et il ne se fera jamais ;
\VS{10}ils ne vivront pas toujours et n'éviteront pas la vue de la fosse.
\VS{11}Car on voit que les sages meurent, l'insensé et le stupide périssent également, et ils laissent à d'autres leurs biens\FTNT{Ec. 2:21 ; Ec. 6:2.}.
\VS{12}Leur intention est que leurs maisons durent éternellement, et que leurs habitations demeurent d'âge en âge, ils ont donné leurs noms à leurs terres.
\VS{13}Mais l'homme qui est en honneur n'a point de durée, il est semblable aux bêtes que l'on égorge.
\VS{14}Tel est leur chemin, leur folie, et ceux qui les suivent se plaisent à leurs discours. Sélah.
\VS{15}Ils seront mis dans le scheol comme des brebis, la mort en fait sa pâture, et au matin les hommes droits les foulent aux pieds, leur beau rocher s'use, le scheol est leur résidence\FTNT{Job. 24:19.}.
\VS{16}Mais Dieu rachètera mon âme du pouvoir du scheol, quand il m'enlèvera de sa captivité\FTNT{Ps. 68:19 ; Ep. 4:8-9.}. Sélah.
\VS{17}Ne crains point quand tu verras quelqu'un s'enrichir et quand les trésors de sa maison se multiplient.
\VS{18}Car lorsqu'il mourra, il n'emportera rien, ses trésors ne descendront point après lui\FTNT{Job. 27:16-19 ; 1 Ti. 6:7.}.
\VS{19}Il aura beau s'estimer heureux pendant sa vie, on aura beau te louer des jouissances que tu te donnes,
\VS{20}tu iras néanmoins au séjour de tes pères, qui jamais ne reverront la lumière.
\VS{21}L'homme qui est en honneur, qui n'a pas d'intelligence, est semblable aux bêtes que l'on égorge.
\Chap{50}
\TextTitle{Yahweh, le juste Juge}
\VerseOne{}Psaume d'Asaph. Le Dieu puissant, Dieu, Yahweh a parlé et il a appelé toute la terre, depuis le soleil levant jusqu'au soleil couchant.
\VS{2}De Sion, Dieu a fait luire sa splendeur qui est d'une beauté parfaite,
\VS{3}notre Dieu viendra, il ne se taira point : Il y aura devant lui un feu dévorant, et tout autour de lui une grosse tempête.
\VS{4}Il appellera les cieux d'en haut, et la terre pour juger son peuple :
\VS{5}Rassemblez-moi mes bien-aimés qui ont traité alliance avec moi par le sacrifice\FTNT{Mt. 24:29-31.}.
\VS{6}Les cieux aussi annonceront sa justice parce que Dieu est le juge. Sélah.
\VS{7}Ecoute, ô mon peuple ! Et je parlerai. Entends, Israël ! Et je t'avertirai. Moi je suis Dieu, ton Dieu.
\VS{8}Je ne te réprimande pas pour tes sacrifices, tes holocaustes sont continuellement devant moi.
\VS{9}Je ne prendrai point de taureaux de ta maison ni de boucs de tes bergeries\FTNT{Ps. 40:7.}.
\VS{10}Car tous les animaux des forêts sont à moi, toutes les bêtes qui paissent sur mille montagnes.
\VS{11}Je connais tous les oiseaux des montagnes, et tout ce qui se meut dans les champs m'appartient.
\VS{12}Si j'avais faim, je ne t'en dirais rien, car le monde est à moi et tout ce qu'il renferme.
\VS{13}Mangerais-je la chair des gros taureaux ? Et boirais-je le sang des boucs ?
\VS{14}Offre à Dieu la reconnaissance et accomplis tes vœux envers le Très-Haut.
\VS{15}Invoque-moi au jour de ta détresse, je te délivrerai, et tu me glorifieras\FTNT{Ps. 37:5.}.
\VS{16}Dieu dit au méchant : Quoi donc ? Tu énumères mes lois ! Et tu as mon alliance dans ta bouche !
\VS{17}Toi qui hais la correction, et qui jettes mes paroles derrière toi !
\VS{18}Si tu vois un voleur, tu te plais avec lui, et ta part est avec les adultères.
\VS{19}Tu livres ta bouche au mal, et ta langue est un tissu de tromperies.
\VS{20}Tu t'assieds et parles contre ton frère, tu couvres d'opprobre le fils de ta mère.
\VS{21}Tu as fait ces choses-là, et je me suis tu. Tu as estimé que je te ressemble, mais je vais te reprendre et tout mettre sous tes yeux.
\VS{22}Comprenez cela maintenant, vous qui oubliez Dieu, de peur que je ne déchire sans que personne ne vous délivre.
\VS{23}Celui qui offre la louange me glorifie, et à celui qui veille sur sa voie, je lui montrerai le salut de Dieu.
\Chap{51}
\TextTitle{Le cœur repentant, sacrifice agréable à Dieu}
\VerseOne{}Psaume de David, au chef des chantres.
\VS{2}Lorsque Nathan le prophète vint à lui, après que David fut allé vers Bath-Schéba\FTNT{2 S. 11 ; 2 S. 12.}.
\VS{3}Ô Dieu ! Aie pitié de moi dans ta bonté, selon ta grande miséricorde, efface mes transgressions ;
\VS{4}lave-moi parfaitement de mon iniquité et purifie-moi de mon péché.
\VS{5}Car je reconnais mes transgressions, et mon péché est continuellement devant moi\FTNT{Es. 59:12.}.
\VS{6}J'ai péché contre toi, contre toi seul, et j'ai fait ce qui déplaît à tes yeux : En sorte que tu seras juste dans ta sentence, sans reproche dans ton jugement.
\VS{7}Voici, je suis né dans l'iniquité, et ma mère m'a conçu dans le péché.
\VS{8}Mais tu prends plaisir à la vérité au fond du cœur, et tu me fais connaître la sagesse au-dedans de moi.
\VS{9}Purifie-moi de mon péché avec de l'hysope, et je serai pur ; lave-moi, et je serai plus blanc que la neige.
\VS{10}Fais-moi entendre la joie et l'allégresse, et les os que tu as brisés se réjouiront.
\VS{11}Détourne ta face de mes péchés, et efface toutes mes iniquités.
\VS{12}Ô Dieu ! Crée en moi un cœur pur et renouvelle en moi un esprit ferme\FTNT{Mt. 5:8.}.
\VS{13}Ne me rejette pas loin de ta face et ne m'ôte pas ton Esprit Saint.
\VS{14}Rends-moi la joie de ton salut et qu'un esprit bien disposé me soutienne.
\VS{15}J'enseignerai tes voies aux transgresseurs et les pécheurs reviendront à toi.
\VS{16}Ô Dieu, Dieu de mon salut ! Délivre-moi de tant de sang, et ma langue chantera hautement ta justice.
\VS{17}Seigneur, ouvre mes lèvres, et ma bouche annoncera ta louange.
\VS{18}Car tu ne prends point plaisir aux sacrifices, autrement je t'en donnerais ; l'holocauste ne t'est point agréable.
\VS{19}Les sacrifices à Dieu, c'est un esprit brisé. Ô Dieu ! Tu ne méprises point un cœur brisé et contrit.
\VS{20}Répands par ta grâce, tes bienfaits sur Sion, édifie les murs de Jérusalem.
\VS{21}Alors tu prendras plaisir aux sacrifices de justice, à l'holocauste, et aux sacrifices qui se consument entièrement par le feu ; alors on offrira des taureaux sur ton autel.
\Chap{52}
\TextTitle{Sort de l'homme qui se confie en ses richesses}
\VerseOne{}Cantique de David, donné au chef des chantres.
\VS{2}A l'occasion du rapport que Doëg, l'Edomite, vint faire à Saül, en lui disant : David s'est rendu dans la maison d'Achimélec.
\VS{3}Pourquoi te vantes-tu du mal, vaillant homme ? La bonté de Dieu dure à toujours.
\VS{4}Ta langue trame des méchancetés, elle est comme un rasoir affilé qui trompe.
\VS{5}Tu aimes plus le mal que le bien, le mensonge plutôt que de dire la vérité. Sélah.
\VS{6}Tu aimes tous les discours pernicieux, le langage trompeur.
\VS{7}Aussi Dieu te détruira pour toujours, il t'enlèvera et t'arrachera de ta tente ; il te déracinera de la terre des vivants. Sélah.
\VS{8}Les justes le verront et auront de la crainte, et ils se riront d'un tel homme, disant :
\VS{9}Voilà cet homme qui ne tenait point Dieu pour sa protection, mais qui se confiait en ses grandes richesses et qui mettait sa force dans ses mauvais désirs\FTNT{Es. 47:10 ; Lu. 12:15-21.}.
\VS{10}Mais moi, je serai dans la maison de Dieu comme un olivier verdoyant. Je me confie dans la bonté de Dieu pour toujours et à jamais.
\VS{11}Je te célébrerai à jamais, car tu agis ; et je mettrai mon espérance en ton Nom, parce qu'il est bon envers tes fidèles.
\Chap{53}
\TextTitle{Egarement des impies}
\VerseOne{}Cantique de David, donné au chef des chantres, pour le chanter sur la flûte.
\VS{2}L'insensé dit en son cœur : Il n'y a point de Dieu ! Ils se sont corrompus, ils ont commis des injustices abominables ; il n'y a personne qui fasse le bien\FTNT{Ps. 10:4 ; Ro. 1:20-21 ; Ro. 3:12.}.
\VS{3}Dieu a regardé des cieux les fils des hommes, pour voir s'il y a quelqu'un qui soit intelligent, qui cherche Dieu.
\VS{4}Ils se sont tous détournés et se sont tous rendus odieux. Il n'y a personne qui fasse le bien, pas même un seul.
\VS{5}Les ouvriers d'iniquité n'ont-ils point de connaissance ? Ils mangent mon peuple comme s'ils mangeaient du pain. Ils n'invoquent point Dieu.
\VS{6}Ils seront épouvantés sans qu'il y ait sujet d'épouvante, car Dieu a dispersé les os de celui qui campe contre toi. Tu les confondras, car Dieu les a rejetés.
\VS{7}Oh ! Qui fera partir de Sion les délivrances d'Israël ? Quand Dieu aura ramené son peuple captif, Jacob s'égayera, Israël se réjouira.
\Chap{54}
\TextTitle{La délivrance vient de Yahweh}
\VerseOne{}Cantique de David, donné au chef des chantres, pour le chanter avec instruments à cordes.
\VS{2}Lorsque les Ziphiens vinrent dire à Saül : David n'est-il pas caché parmi nous\FTNT{1 S. 23:19 ; 1 S. 26:1.} ?
\VS{3}Ô Dieu ! Délivre-moi par ton Nom et fais-moi justice par ta puissance.
\VS{4}Ô Dieu ! Ecoute ma prière, prête l'oreille aux paroles de ma bouche !
\VS{5}Car des étrangers se sont élevés contre moi, et des gens terribles qui ne mettent pas Dieu devant eux en veulent à ma vie. Sélah.
\VS{6}Voilà, Dieu m'accorde son secours, le Seigneur est de ceux qui soutiennent mon âme.
\VS{7}Il fera retourner le mal sur ceux qui m'épient ; détruis-les selon ta vérité.
\VS{8}Je t'offrirai de bon cœur des sacrifices ; Yahweh ! Je célébrerai ton Nom parce qu'il est bon.
\VS{9}Car il m'a délivré de toute détresse ; et mes yeux se réjouissent à la vue de mes ennemis.
\Chap{55}
\TextTitle{Se garder des méchants}
\VerseOne{}Cantique de David, donné au chef des chantres, pour le chanter avec instruments à cordes.
\VS{2}Ô Dieu ! Prête l'oreille à ma prière et ne te cache pas de mes supplications !
\VS{3}Ecoute-moi et réponds-moi ! J'erre çà et là dans ma méditation et je suis agité
\VS{4}à cause du bruit que fait l'ennemi, à cause de l'oppression du méchant ; car ils font tomber sur moi les outrages, et ils me haïssent jusqu'à la fureur.
\VS{5}Mon cœur tremble au-dedans de moi et les terreurs de la mort tombent sur moi.
\VS{6}La crainte et l'épouvante m'atteignent et le frisson m'habille.
\VS{7}Je dis : Qui me donnera des ailes de colombe ? Je m'envolerais et je trouverais ma demeure.
\VS{8}Voilà, je m'enfuirais bien loin et je me tiendrais au désert. Sélah.
\VS{9}Je m'échapperais en toute hâte, plus rapide que le vent impétueux, que la tempête.
\VS{10}Seigneur, réduis à néant, divise leur langue, car j'ai vu la violence et les querelles dans la ville.
\VS{11}Elles font jour et nuit le tour sur les murailles ; l'iniquité et la malice sont dans son sein.
\VS{12}Les calamités sont au milieu d'elle, et la tromperie et la fraude ne partent point de ses places.
\VS{13}Car ce n'est pas mon ennemi qui m'a diffamé, je le supporterais ; ce n'est point celui qui m'a en haine qui s'élève contre moi, je me cacherais de lui.
\VS{14}Mais c'est toi, ô homme ! Que j'estimais mon égal, mon confident et mon ami\FTNT{Ps. 41:10.} !
\VS{15}Nous prenions plaisir à communiquer nos secrets ensemble, nous allions avec la multitude dans la maison de Dieu.
\VS{16}Que la mort les séduise ! Qu'ils descendent vivants dans le scheol ! Car le mal est dans leur demeure, parmi eux dans leur assemblée.
\VS{17}Mais moi je crie à Dieu, et Yahweh me délivrera.
\VS{18}Le soir, le matin, et à midi je me plains et je gémis, et il entendra ma voix.
\VS{19}Il délivrera mon âme de la guerre et me rendra la paix ; car ils sont nombreux contre moi.
\VS{20}Dieu entendra et témoignera en ma faveur. Lui qui de toute éternité est assis sur son trône. Sélah. Car il n'y a point de changement en eux, et ils ne craignent point Dieu.
\VS{21}Chacun d'eux porte la main sur ceux qui vivaient en paix avec lui, et viole son alliance.
\VS{22}Les paroles de sa bouche sont plus douces que la crème, mais la guerre est dans son cœur ; ses paroles sont plus douces que l'huile, néanmoins elles sont tout autant d'épées nues.
\VS{23}Remets ton sort à Yahweh et il te soulagera, il ne permettra jamais que le juste tombe.
\VS{24}Mais toi, ô Dieu ! Tu les précipiteras au puits de la perdition ; les hommes sanguinaires et trompeurs ne parviendront point à la moitié de leurs jours. C'est en toi que je me confie.
\Chap{56}
\TextTitle{Se glorifier en la Parole de Yahweh}
\VerseOne{}Hymne de David, donné au chef des chantres, pour le chanter sur « Colombe des térébinthes lointains ». Lorsque les Philistins le saisirent à Gath\FTNT{1 S. 21:10-14.}.
\VS{2}Dieu ! Aie pitié de moi, car des hommes m'écrasent et m'oppriment, me faisant tout le jour la guerre, ils m'oppressent.
\VS{3}Mes adversaires me piétinent tout le jour ; car, ô Très-Haut, plusieurs me font la guerre comme des hautains.
\VS{4}Le jour où j'aurai peur, je me confierai en toi.
\VS{5}Je me glorifierai en Dieu, en sa parole ; je me confie en Dieu, je ne craindrai rien. Que peuvent me faire les hommes\FTNT{Ps. 118:6 ; Hé. 13:6.} ?
\VS{6}Tout le jour ils tordent mes propos, et toutes leurs pensées tendent à me nuire.
\VS{7}Ils s'assemblent, ils se tiennent cachés, ils observent mes pas, s'attendant à m'ôter la vie.
\VS{8}C'est par l'iniquité qu'ils espèrent échapper. Dans ta colère, ô Dieu, précipite les peuples !
\VS{9}Tu comptes mes allées et venues ; recueille mes larmes dans tes outres : Ne sont-elles pas écrites dans ton livre ?
\VS{10}Le jour où je crierai à toi, mes ennemis reculeront ; je sais que Dieu est pour moi.
\VS{11}Je me glorifierai en Dieu, en sa parole, je me glorifierai en Yahweh, en sa parole.
\VS{12}Je me confie en Dieu, je ne craindrai rien : Que me fera l'homme ?
\VS{13}Ô Dieu ! Les vœux que je t'ai fait s'accompliront, je te louerai.
\VS{14}Car tu as délivré mon âme de la mort, tu as garanti mes pieds de la chute, afin que je marche devant Dieu, à la lumière des vivants.
\Chap{57}
\TextTitle{Avoir confiance en Dieu dans les difficultés}
\VerseOne{}Hymne de David, donné au chef des chantres, pour le chanter sur Al-Thasheth\FTNT{Al-Thasheth signifie « Ne détruis pas »}. Lorsqu'il se réfugia dans la caverne, poursuivi par Saül\FTNT{1 S. 22:1.}.
\VS{2}Aie pitié de moi, ô Dieu, aie pitié de moi ! Car mon âme cherche un refuge ; je cherche un refuge à l'ombre de tes ailes, jusqu'à ce que les calamités soient passées\FTNT{Ps. 17:8.}.
\VS{3}Je crie au Dieu Très-Haut, au Dieu qui accomplit son œuvre pour moi.
\VS{4}Il m'enverra des cieux la délivrance, il rendra honteux celui qui veut me dévorer. Sélah. Dieu enverra sa bonté et sa vérité.
\VS{5}Mon âme est parmi des lions ; je suis couché au milieu de gens qui vomissent la flamme, parmi des hommes dont les dents sont des lances et des flèches, et dont la langue est une épée aiguë\FTNT{Ps. 59:8 ; Ps. 64:4 ; Ja. 3:5-12.}.
\VS{6}Ô Dieu, élève-toi sur les cieux ! Que ta gloire soit sur toute la terre !
\VS{7}Ils avaient tendu un filet sous mes pas : Mon âme se courbait. Ils avaient creusé une fosse devant moi, mais ils y sont tombés. Sélah.
\VS{8}Mon cœur est affermi, ô Dieu ! Mon cœur est affermi, je chanterai et je ferai retentir mes instruments.
\VS{9}Réveille-toi ma gloire ! Réveillez-vous mon luth et ma harpe ! Je me réveillerai à l'aube du jour.
\VS{10}Seigneur, je te célébrerai parmi les peuples, je te chanterai parmi les nations.
\VS{11}Car ta bonté est grande jusqu'aux cieux, et ta vérité jusqu'aux nues\FTNT{Ps. 118:4-5.}.
\VS{12}Ô Dieu ! Elève-toi sur les cieux ! Que ta gloire soit sur toute la terre !
\Chap{58}
\TextTitle{Yahweh rend justice sur la terre}
\VerseOne{}Hymne de David, donné au chef des chantres, pour le chanter sur Al-Thasheth « Ne détruis pas ».
\VS{2}En vérité, vous, gens de l'assemblée, prononcez-vous ce qui est juste ? Vous, fils des hommes, jugez-vous avec droiture ?
\VS{3}Au contraire, vous tramez des injustices dans votre cœur. Sur la terre, c'est la violence de vos mains que vous placez sur la balance.
\VS{4}Les méchants se sont égarés dès le sein maternel, ils ont erré dès le ventre de leur mère, en parlant faussement.
\VS{5}Ils ont un venin semblable au venin du serpent, ils sont comme l'aspic sourd, qui ferme son oreille,
\VS{6}qui n'entend pas la voix des enchanteurs, du magicien le plus sage.
\VS{7}Ô Dieu, brise-leur les dents dans leur bouche ! Yahweh, brise les mâchoires des lionceaux !
\VS{8}Qu'ils s'écoulent comme de l'eau, et qu'ils se fondent ! Que chacun d'eux bande son arc, mais que ses flèches soient comme si elles étaient rompues !
\VS{9}Qu'ils s'en aillent comme un limaçon qui se fond ! Qu'ils ne voient point le soleil comme l'avorton d'une femme !
\VS{10}Avant que vos chaudières aient senti le feu des épines, l'ardeur de la colère, semblable à un tourbillon, enlèvera chacun d'eux comme de la chair crue..
\VS{11}Le juste se réjouira quand il aura vu la vengeance, il lavera ses pieds avec le sang du méchant.
\VS{12}Et chacun dira : Quoi qu'il en soit, il y a une récompense pour le juste ; quoi qu'il en soit, il y a un Dieu qui juge sur la terre.
\Chap{59}
\TextTitle{Intervention divine}
\VerseOne{}Hymne de David, donné au chef des chantres, pour le chanter sur Al-Thasheth « Ne détruis pas ». Lorsque Saül envoya des gens qui épièrent sa maison afin de le tuer\FTNT{1 S. 19:11.}.
\VS{2}Mon Dieu ! Délivre-moi de mes ennemis, protège-moi de ceux qui s'élèvent contre moi !
\VS{3}Délivre-moi des ouvriers d'iniquité et garde-moi des hommes sanguinaires !
\VS{4}Car voici, ils m'ont dressé des embûches, et des gens robustes se sont assemblés contre moi, bien qu'il n'y ait point en moi de transgression ni de péché, ô Yahweh !
\VS{5}Ils courent çà et là, et se mettent en ordre, bien qu'il n'y ait point d'iniquité en moi. Réveille-toi pour venir au-devant de moi ! Et regarde !
\VS{6}Toi donc, ô Yahweh ! Dieu des armées, Dieu d'Israël, réveille-toi pour visiter toutes les nations ! Ne fais point de grâce à aucun de ceux qui me trahissent ! Sélah.
\VS{7}Ils reviennent chaque soir, ils hurlent comme des chiens, ils font le tour de la ville.
\VS{8}Voici, de leur bouche ils font jaillir le mal, il y a des épées sur leurs lèvres\FTNT{Ja. 3:5-12.} ; car, disent-ils, qui nous entend ?
\VS{9}Mais toi, Yahweh ! Tu te riras d'eux, tu te moqueras de toutes les nations\FTNT{Ps. 2:4.}.
\VS{10}Quelle que soit leur force, je m'attends à toi, car Dieu est ma haute retraite.
\VS{11}Dieu qui me favorise me préviendra, Dieu me fera voir mes adversaires\FTNT{Ps. 118:7.}.
\VS{12}Ne les tue pas, de peur que mon peuple ne l'oublie ; fais-les errer par ta puissance et abats-les ; Seigneur, notre bouclier !
\VS{13}Leur bouche pèche à chaque parole de leurs lèvres ; qu'ils soient pris par leur orgueil ! Ils ne tiennent que des discours de malédiction et de mensonge.
\VS{14}Consume-les avec fureur, consume-les de sorte qu'ils ne soient plus ! Qu'on sache que Dieu domine sur Jacob et jusqu'aux extrémités de la terre ! Sélah.
\VS{15}Qu'ils reviennent le soir, et qu'ils hurlent comme des chiens, et qu'ils fassent le tour de la ville.
\VS{16}Qu'ils errent çà et là cherchant leur nourriture, et qu'ils passent la nuit sans être rassasiés.
\VS{17}Mais moi je chanterai ta force, je louerai dès le matin à haute voix ta bonté\FTNT{Ps. 88:14.}. Car tu es pour moi une haute retraite, et mon asile au jour de ma détresse.
\VS{18}Ma force ! Je te chanterai ; car Dieu est ma haute retraite, le Dieu qui me favorise.
\Chap{60}
\TextTitle{Yahweh, le meilleur secours}
\VerseOne{}Hymne de David, pour enseigner, donné au chef des chantres, pour le chanter sur le lis lyrique,
\VS{2}lorsqu'il fit la guerre contre les Syriens de Mésopotamie, et contre les Syriens de Tsoba, et que Joab revint et défit douze mille Edomites dans la vallée du sel\FTNT{2 S. 8:3-13 ; 1 Ch. 18:3-12.}.
\VS{3}Ô Dieu ! Tu nous as rejetés, tu nous as dispersés, tu t'es irrité : Reviens vers nous !
\VS{4}Tu as ébranlé la terre et l'as mise en pièces ; répare ses brèches, car elle chancelle !
\VS{5}Tu as fait voir à ton peuple des choses dures, tu nous as abreuvés d'un vin d'étourdissement\FTNT{Es. 51:17-21 ; Ap. 14:10.}.
\VS{6}Mais tu as donné une bannière à ceux qui te craignent, afin de l'élever bien haut pour l'amour de ta vérité. Sélah.
\VS{7}Afin que ceux que tu aimes soient délivrés ; sauve-moi par ta droite et exauce-moi\FTNT{Ps. 108:6.}.
\VS{8}Dieu a parlé dans son lieu saint : Je me réjouirai, je partagerai Sichem, je mesurerai la vallée de Succoth ;
\VS{9}Galaad est à moi, Manassé aussi est à moi, et Ephraïm est la protection de ma tête et Juda mon sceptre.
\VS{10}Moab est le bassin où je me lave ; je jette mon soulier sur Edom ; pays des Philistins, pousse des cris de guerre à mon sujet\FTNT{2 S. 8:2 ; 1 Ch. 18:2.}.
\VS{11}Qui me conduira dans la ville forte ? Qui me conduira jusqu'en Edom ?
\VS{12}Ne sera-ce pas toi, ô Dieu, qui nous avais rejetés, et qui ne sortais plus, ô Dieu, avec nos armées ?
\VS{13}Donne-nous du secours pour sortir de la détresse ! Car la délivrance qu'on attend de l'homme est vanité\FTNT{Jé. 17:5 ; Ps. 118:8.}.
\VS{14}Avec le secours de Dieu, nous ferons des exploits, et il foulera nos ennemis.
\Chap{61}
\TextTitle{Dieu, le parfait Refuge}
\VerseOne{}Psaume de David, donné au chef des chantres, pour le chanter sur instruments à cordes.
\VS{2}Ô Dieu, je crie à toi, sois attentif à ma prière !
\VS{3}Je crie à toi du bout de la terre, le cœur abattu ; conduis-moi sur le rocher qui est trop haut pour moi !
\VS{4}Car tu es mon refuge, une tour forte au-devant de l'ennemi.
\VS{5}Je séjournerai éternellement dans ta tente, je me retirerai à l'ombre de tes ailes. Sélah.
\VS{6}Car, ô Dieu ! Tu exauces mes vœux, et tu me donnes l'héritage de ceux qui craignent ton Nom.
\VS{7}Tu ajoutes des jours aux jours du roi ; que ses années se prolongent à jamais !
\VS{8}Qu'il demeure toujours dans la présence de Dieu ! Que la bonté et la vérité le gardent !
\VS{9}Ainsi je chanterai ton Nom à perpétuité, en rendant mes vœux chaque jour.
\Chap{62}
\TextTitle{La confiance en Dieu}
\VerseOne{}Psaume de David, donné au chef des chantres, d'après Jeduthun.
\VS{2}Quoiqu'il en soit, mon âme se repose en Dieu ; c'est de lui que vient ma délivrance.
\VS{3}Quoiqu'il en soit, il est mon rocher, ma délivrance, et ma haute retraite ; je ne serai pas entièrement ébranlé.
\VS{4}Jusqu'à quand accablerez-vous de maux un homme ? Vous serez tous mis à mort, et vous serez comme le mur qui penche, comme une cloison qui a été ébranlée.
\VS{5}Ils ne font que consulter pour le faire déchoir de son élévation ; ils prennent plaisir au mensonge ; ils bénissent de leur bouche, mais au-dedans ils maudissent. Sélah.
\VS{6}Mais toi mon âme, demeure tranquille, regarde à Dieu, car mon espérance est en lui.
\VS{7}Quoiqu'il en soit, il est mon rocher, ma délivrance, et ma haute retraite ; je ne serai point ébranlé.
\VS{8}En Dieu est ma délivrance et ma gloire ; en Dieu est le rocher de ma force et ma retraite.
\VS{9}Peuples, confiez-vous en lui en tout temps, déchargez votre cœur sur lui ! Dieu est notre retraite. Sélah.
\VS{10}Oui, vanité, les fils de l'homme ! Mensonge, les fils de l'homme ! Dans une balance, ils monteraient tous ensemble, plus légers qu'un souffle.
\VS{11}Ne vous confiez pas dans la violence ni dans la rapine ; ne devenez point vains ; quand les richesses abonderont, n'y mettez point votre cœur.
\VS{12}Dieu a parlé une fois, j'ai entendu cela deux fois : C'est que la force est à Dieu.
\VS{13}Et c'est à toi, Seigneur, qu'appartient la bonté ; certainement tu rendras à chacun selon son œuvre\FTNT{Jé. 32:19 ; Pr. 24:12 ; Job. 34:11 ; Mt. 16:27 ; Ro. 2:6.}.
\Chap{63}
\TextTitle{Soif de la présence de Dieu}
\VerseOne{}Psaume de David, lorsqu'il était dans le désert de Juda\FTNT{1 S. 22:5 ; 1 S. 23:14-15.}.
\VS{2}Ô Dieu ! Tu es mon Dieu, je te cherche au point du jour ; mon âme a soif de toi, mon corps soupire après toi sur cette terre aride, desséchée, et sans eau\FTNT{Ps. 42:2 ; Ps. 84:3 ; Ps. 143:6.}.
\VS{3}Ainsi je te contemple dans ton lieu saint pour voir ta force et ta gloire.
\VS{4}Car ta bonté vaut mieux que la vie, mes lèvres te louent.
\VS{5}Et ainsi je te bénirai donc toute ma vie\FTNT{Ps. 104:33.}, j'élèverai mes mains en ton Nom.
\VS{6}Mon âme est rassasiée comme de mets gras et succulents, et ma bouche te loue avec un chant de réjouissance.
\VS{7}Quand je me souviens de toi dans mon lit, je médite sur toi durant les veilles de la nuit\FTNT{Ps. 16:7 ; Ps. 119:55.}.
\VS{8}Car tu m'as secouru, je me réjouirai à l'ombre de tes ailes.
\VS{9}Mon âme s'est attachée à toi pour te suivre, ta droite me soutient.
\VS{10}Mais ceux-ci qui demandent que mon âme tombe en ruine, entreront au plus bas de la terre.
\VS{11}On les détruira à coups d'épée, ils seront la proie des chacals.
\VS{12}Mais le roi se réjouira en Dieu ; quiconque jure par lui s'en glorifiera, car la bouche de ceux qui mentent sera fermée\FTNT{Ps. 107:42 ; Job. 5:16.}.
\Chap{64}
\TextTitle{Yahweh, le seul abri}
\VerseOne{}Psaume de David, donné au chef des chantres.
\VS{2}Ô Dieu ! Ecoute ma voix quand je m'écrie. Protège ma vie contre l'ennemi que je crains !
\VS{3}Cache-moi des complots des méchants, de l'assemblée tumultueuse des ouvriers d'iniquité !
\VS{4}Ils aiguisent leur langue comme une épée\FTNT{Jé. 9:3 ; Ps. 11:2 ; Ps. 59:8.}, ils tirent comme des flèches leurs paroles amères,
\VS{5}afin de tirer sur l'innocent dans sa cachette ; ils tirent soudainement sur lui et n'ont aucune crainte.
\VS{6}Ils se fortifient dans leur méchanceté, tiennent des discours pour tendre des pièges, ils disent : Qui les verra\FTNT{Job. 24:15.} ?
\VS{7}Ils cherchent curieusement des méchancetés ; ils ont sondé tout ce qui se peut sonder, même ce qui peut être au–dedans de l'homme, et au cœur le plus profond.
\VS{8}Mais Dieu lance contre eux ses traits, soudain les voilà frappés.
\VS{9}Leur langue a causé leur chute ; tous ceux qui les voient secouent leur tête.
\VS{10}Et tous les hommes craindront et raconteront l'œuvre de Dieu, et considéreront ce qu'il aura fait.
\VS{11}Le juste se réjouira en Yahweh, et se retirera vers lui, et tous ceux qui sont droits de cœur s'en glorifieront\FTNT{Ps. 63:12 ; Ps. 97:12.}.
\Chap{65}
\TextTitle{Le règne de Yahweh sur la nature}
\VerseOne{}Psaume de David. Cantique. Donné au chef des chantres.
\VS{2}Ô Dieu ! Dans le calme, on te louera dans Sion, et l'on accomplira nos vœux\FTNT{Ps. 50:14 ; Ps. 66:13.}.
\VS{3}Tu entends nos prières, toute chair viendra jusqu'à toi.
\VS{4}Les iniquités prévalent sur moi, mais tu feras la propitiation de nos transgressions.
\VS{5}Heureux celui que tu choisis et que tu admets dans ta présence pour qu'il habite dans tes parvis ! Nous serons rassasiés des biens de ta maison, des biens du saint lieu de ton temple.
\VS{6}Dans ta justice, tu nous réponds par des choses terribles, ô Dieu de notre salut, espoir de toutes les extrémités lointaines de la terre et de la mer.
\VS{7}Il affermit les montagnes par sa force, il est ceint de puissance.
\VS{8}Il apaise le mugissement de la mer, le mugissement de leurs flots, et le tumulte des peuples.
\VS{9}Ceux qui habitent aux extrémités de la terre ont peur de tes prodiges ; tu réjouis l'orient et l'occident.
\VS{10}Tu visites la terre, tu lui donnes l'abondance, tu la combles de richesses ; le ruisseau de Dieu est plein d'eau ; tu prépares le blé, quand tu l'établis ainsi.
\VS{11}Tu arroses ses sillons, et tu aplanis ses mottes ; tu l'amollis par la pluie, et tu bénis son germe\FTNT{Es. 55:10 ; Ps. 104:13-14.}.
\VS{12}Tu couronnes l'année de tes biens, et tes voies versent l'abondance.
\VS{13}Les plaines du désert sont abreuvées et les collines sont ceintes de joie.
\VS{14}Les pâturages se couvrent de brebis, et les vallées se revêtent de froments ; les cris de joie et les chants retentissent.
\Chap{66}
\TextTitle{Louange au Dieu de grâces}
\VerseOne{}Cantique. Psaume, donné au chef des chantres. Vous tous habitants de toute la terre, poussez des cris de triomphe à Dieu.
\VS{2}Chantez la gloire de son Nom, faites éclater sa gloire par vos louanges.
\VS{3}Dites à Dieu : Que tes œuvres sont redoutables ! Tes ennemis te mentiront à cause de la grandeur de ta force.
\VS{4}Toute la terre se prosterne devant toi et te chante ; elle chante ton Nom. Sélah.
\VS{5}Venez et voyez les œuvres de Dieu : Il est redoutable quand il agit sur les fils des hommes.
\VS{6}Il a fait de la mer une terre sèche ; on a passé le fleuve à pied sec ; là, nous nous sommes réjouis en lui\FTNT{Ex. 14:21 ; Jos. 3:14-17.}.
\VS{7}Il domine par sa puissance éternellement ; ses yeux prennent garde sur les nations\FTNT{Ps. 14:2 ; Ps. 33:13 ; Job. 28:24.} ; les rebelles ne pourront point s'élever. Sélah.
\VS{8}Peuples, bénissez notre Dieu, et faites retentir le son de sa louange.
\VS{9}C'est lui qui a remis notre âme en vie, et qui n'a point permis que nos pieds chancellent.
\VS{10}Car, ô Dieu, tu nous as éprouvés ! Tu nous a fait passer au creuset comme l'argent.
\VS{11}Tu nous as amenés dans le filet, tu as mis sur nos reins un pesant fardeau.
\VS{12}Tu as fait monter des hommes sur notre tête, et nous avons passé par le feu et par l'eau. Mais tu nous as fait entrer dans un lieu d'abondance.
\VS{13}J'entrerai dans ta maison avec des holocaustes, j'accomplirai mes vœux envers toi\FTNT{Ps. 22:26 ; Ps. 76:12 ; Ps. 116:14.}.
\VS{14}Pour eux, mes lèvres se sont ouvertes et ma bouche les a prononcés dans ma détresse.
\VS{15}Je t'offrirai en holocauste des brebis grasses, avec la graisse des béliers, je te sacrifierai des taureaux et des boucs. Sélah.
\VS{16}Vous tous qui craignez Dieu, venez, écoutez, et je raconterai ce qu'il a fait à mon âme.
\VS{17}Je l'ai invoqué de ma bouche, et la louange a été sur ma langue.
\VS{18}Si j'avais conçu l'iniquité dans mon cœur, le Seigneur ne m'aurait pas écouté\FTNT{Jn. 9:31.}.
\VS{19}Mais certainement Dieu m'a écouté, il a été attentif à la voix de ma prière.
\VS{20}Béni soit Dieu qui n'a point rejeté ma prière, et qui n'a point éloigné de moi sa bonté.
\Chap{67}
\TextTitle{Louange des peuples}
\VerseOne{}Psaume. Cantique donné au chef des chantres, pour le chanter avec instruments à cordes.
\VS{2}Que Dieu ait pitié de nous et qu'il nous bénisse, qu'il fasse luire sa face sur nous\FTNT{No. 6:25 ; Ps. 4:7 ; Ps. 31:17 ; Ps. 119:135.}. Sélah.
\VS{3}Afin que ta voie soit connue sur la terre et ta délivrance parmi toutes les nations.
\VS{4}Les peuples te célébreront, ô Dieu ! Tous les peuples te célébreront\FTNT{Ps. 22:27 ; Ps. 68:33.} !
\VS{5}Les peuples se réjouissent et chantent de joie, car tu juges les peuples avec droiture et tu conduis les nations sur la terre\FTNT{Ps. 96:10.}. Sélah.
\VS{6}Les peuples te célébreront, ô Dieu ! Tous les peuples te célébreront !
\VS{7}La terre produira son fruit ; Dieu, notre Dieu, nous bénira.
\VS{8}Dieu nous bénira, et toutes les extrémités de la terre le craindront.
\Chap{68}
\TextTitle{Yahweh, le Dieu glorieux}
\VerseOne{}Psaume. Cantique de David, donné au chef des chantres.
\VS{2}Que Dieu se lève, et ses ennemis seront dispersés, et ceux qui le haïssent s'enfuiront devant lui\FTNT{No. 10:35.}.
\VS{3}Tu les chasseras comme la fumée est chassée par le vent ; comme la cire se fond devant le feu, ainsi les méchants périront devant Dieu\FTNT{Ps. 37:20 ; Ps. 97:5.}.
\VS{4}Mais les justes se réjouiront et s'égayeront devant Dieu, et tressailliront de joie\FTNT{Ps. 67:4-5.}.
\VS{5}Chantez à Dieu, célébrez son Nom ! Exaltez celui qui est monté sur les cieux ! Son Nom est Yahweh ! Réjouissez-vous dans sa présence.
\VS{6}Il est le père des orphelins et le juge des veuves ; Dieu est dans sa demeure sainte\FTNT{Ps. 146:9}.
\VS{7}Dieu donne une famille à ceux qui étaient abandonnés, il délivre ceux qui étaient enchaînés, mais les rebelles habitent sur une terre déserte.
\VS{8}Ô Dieu ! Quand tu sortis devant ton peuple, quand tu marchais dans le désert ! Sélah.
\VS{9}La terre trembla et les cieux répandirent leurs eaux à cause de la présence de Dieu, le mont Sinaï trembla à cause de la présence de Dieu, du Dieu d'Israël\FTNT{Ex. 19:18 ; Jg. 5:5.}.
\VS{10}Ô Dieu ! Tu as fait tomber une pluie abondante sur ton héritage, et quand il était épuisé, tu l'as rétabli.
\VS{11}Ton troupeau établit sa demeure dans le pays, que par ta bonté tu avais préparé pour les malheureux, ô Dieu !
\VS{12}Le Seigneur donne une parole, et les messagères de bonnes nouvelles sont une grande armée.
\VS{13}Les rois des armées se sont enfuis, ils se sont enfuis, et celle qui se tenait à la maison a partagé le butin\FTNT{1 S. 30:16.}.
\VS{14}Tandis que vous vous couchez dans les étables, les ailes de la colombe sont couvertes d'argent, et son plumage est d'un jaune d'or.
\VS{15}Quand le Tout-Puissant dispersa les rois dans le pays, il devint blanc comme la neige du Tsalmon.
\VS{16}La montagne de Dieu est un mont de Basan ; une montagne élevée, un mont de Basan.
\VS{17}Pourquoi l'insultez-vous, montagnes dont le sommet est élevé ? Dieu a désiré cette montagne pour y habiter, et Yahweh y demeurera à jamais.
\VS{18}Les chars de Dieu se comptent par vingt-mille, par milliers et par milliers ; le Seigneur est au milieu d'eux ; le Sinaï est dans le sanctuaire.
\VS{19}Tu es monté dans les hauteurs, tu as emmené des captifs, tu as pris des dons pour les distribuer parmi les hommes, et même parmi les rebelles, afin qu'ils habitent dans le lieu de Yahweh Dieu\FTNT{Ep. 4:8-10. Cette prophétie concerne la résurrection du Seigneur Jésus-Christ.}.
\VS{20}Béni soit le Seigneur, qui tous les jours nous comble de ses biens ; Dieu est notre délivrance. Sélah.
\VS{21}Dieu est pour nous le Dieu de délivrance, et les issues de la mort sont à Yahweh le Seigneur.
\VS{22}Certainement, Dieu écrasera la tête de ses ennemis\FTNT{Ge. 3:15.}, le sommet de la tête chevelue de celui qui marche dans ses péchés.
\VS{23}Le Seigneur dit : Je les ramènerai de Basan\FTNT{No. 21:33-35.}, je les ramènerai du fond de la mer.
\VS{24}Afin que tu plonges ton pied dans le sang\FTNT{Ps. 58:11.}, et que la langue de tes chiens ait sa part de tes ennemis.
\VS{25}Ils voient ta marche, ô Dieu ! Ils ont vu ta marche dans le lieu saint, la marche de mon Dieu, mon Roi.
\VS{26}Les chantres allaient devant, ensuite les joueurs d'instruments, et au milieu les jeunes filles jouant du tambour\FTNT{Ex. 15:20 ; 1 S. 18:6.}.
\VS{27}Bénissez Dieu dans les assemblées, bénissez le Seigneur, vous qui êtes descendants d'Israël.
\VS{28}Là sont Benjamin, le plus jeune qui domine sur eux, les chefs de Juda et leur corps d'armée, les chefs de Zabulon, et les chefs de Nephthali.
\VS{29}Ton Dieu ordonne que tu sois puissant. Affermis, ô Dieu, ce que tu as fait pour nous.
\VS{30}Dans ton temple, à Jérusalem, les rois t'amèneront des présents\FTNT{1 R. 10:10 ; Ps. 72:10 ; 2 Ch. 32:23.}.
\VS{31}Epouvante les bêtes sauvages des roseaux, la troupe des taureaux, et les veaux des peuples, et ceux qui se prosternent avec des pièces d'argent. Disperse les peuples qui prennent plaisir à la guerre.
\VS{32}De grands seigneurs viendront d'Egypte ; l'Ethiopie se hâtera d'étendre ses mains vers Dieu.
\VS{33}Royaumes de la terre, chantez à Dieu, célébrez le Seigneur ! Sélah.
\VS{34}Chantez celui qui est monté dans les cieux des cieux, les cieux éternels ; voilà, il fait retentir de sa voix un son puissant.
\VS{35}Attribuez la force à Dieu ; sa majesté est sur Israël, et sa force est dans les nuées.
\VS{36}Dieu ! Tu es redouté à cause de ton lieu saint. Le Dieu d'Israël est celui qui donne la force et la puissance à son peuple. Béni soit Dieu !
\Chap{69}
\TextTitle{Dieu attentif à la prière de ceux qui s'humilient}
\VerseOne{}Psaume de David, donné au chef des chantres, pour le chanter sur les lis.
\VS{2}Délivre-moi, ô Dieu, car les eaux menacent ma vie\FTNT{Ps. 124:4 ; Ps. 144:7.}.
\VS{3}Je suis enfoncé dans un bourbier profond, sans appui ; je suis entré au plus profond des eaux, et les courants d'eau me submergent.
\VS{4}Je suis las de crier, mon gosier se dessèche, mes yeux se consument pendant que je m'attends à Dieu.
\VS{5}Ceux qui me haïssent sans cause\FTNT{Jn. 15:25.} dépassent en nombre les cheveux de ma tête ; ceux qui tâchent de me ruiner et qui sont mes ennemis à tort se sont renforcés ; je dois rendre ce que je n'avais point ravi.
\VS{6}Ô Dieu ! Tu connais ma folie et mes fautes ne te sont point cachées.
\VS{7}Ô Seigneur Yahweh des armées ! Que ceux qui se confient en toi ne soient point honteux à cause de moi ; et que ceux qui te cherchent ne soient point humiliés à cause de moi, ô Dieu d'Israël !
\VS{8}Car pour l'amour de toi j'ai souffert l'opprobre, la honte a couvert mon visage.
\VS{9}Je suis devenu un étranger pour mes frères, et un homme de dehors pour les fils de ma mère\FTNT{Ge. 31:14-15 ; Jn. 7:3-5}.
\VS{10}Car le zèle de ta maison me dévore\FTNT{Jn. 2:17 ; Ro. 15:3.}, et les outrages de ceux qui t'insultaient sont tombés sur moi.
\VS{11}Je pleure et je jeûne : C'est ce qui m'attire l'opprobre.
\VS{12}Je prends un sac pour vêtement, et je suis l'objet de leurs discours moqueurs.
\VS{13}Ceux qui sont assis à la porte parlent de moi, et les buveurs de boissons fortes me mettent en chanson\FTNT{Job. 30:9 ; La. 3:14.}.
\VS{14}Mais je t'adresse ma prière, ô Yahweh\FTNT{Ps. 102:2.} ! Que ce soit le temps favorable, ô Dieu ! Par ta grande bonté. Réponds-moi en m'assurant ta délivrance.
\VS{15}Délivre-moi de la boue, que je ne m'y enfonce point\FTNT{Ps. 40:3.}, et que je sois délivré de ceux qui me haïssent, et des eaux profondes.
\VS{16}Que les courants d'eau ne me submergent plus, que l'abîme ne m'engloutisse point, et que le puits ne ferme point sa bouche sur moi.
\VS{17}Yahweh ! Exauce-moi, car ta bonté est agréable ; dans tes grandes compassions, tourne ta face vers moi ;
\VS{18}et ne cache point ta face à ton serviteur, car je suis en détresse. Hâte-toi, exauce-moi !
\VS{19}Approche-toi de mon âme, rachète-la ; délivre-moi à cause de mes ennemis.
\VS{20}Tu connais toi-même mon opprobre, et ma honte, et mon ignominie ; tous mes ennemis sont devant toi.
\VS{21}L'opprobre m'a brisé le cœur, et je suis languissant ; j'ai attendu que quelqu'un ait compassion de moi, mais il n'y en a point eu. J'ai attendu des consolateurs, mais je n'en ai point trouvé.
\VS{22}Ils m'ont au contraire donné du fiel\FTNT{Mt. 27:34 ; Mt. 27:48.} pour mon repas ; et dans ma soif, ils m'ont abreuvé de vinaigre.
\VS{23}Que leur table soit pour eux un piège et un appât au sein de leur perfection.
\VS{24}Que leurs yeux soient tellement obscurcis, qu'ils ne puissent point voir ; et fais continuellement chanceler leurs reins.
\VS{25}Répands ton indignation sur eux, et que l'ardeur de ta colère les saisisse.
\VS{26}Que leur campement soit désolé, et qu'il n'y ait personne qui habite dans leurs tentes.
\VS{27}Car ils persécutent celui que tu avais frappé, et racontent les souffrances de ceux que tu blesses.
\VS{28}Mets des iniquités à leurs iniquités ; et qu'ils n'entrent point dans ta justice.
\VS{29}Qu'ils soient effacés du livre de vie, et qu'ils ne soient point inscrits avec les justes.
\VS{30}Mais pour moi, qui suis affligé et dans la douleur, ta délivrance, ô Dieu, m'élèvera en une haute retraite.
\VS{31}Je louerai le Nom de Dieu par des cantiques et je le glorifierai par des louanges.
\VS{32}Cela est agréable à Yahweh plus qu'un taureau avec des cornes et des sabots fendus.
\VS{33}Les malheureux le voient et ils se réjouissent ; que votre cœur vive, vous qui cherchez Dieu.
\VS{34}Car Yahweh exauce les misérables et ne méprise point ses prisonniers.
\VS{35}Que les cieux et la terre le louent ; que la mer et tout ce qui s'y meut le louent aussi\FTNT{Ps. 96:11.}.
\VS{36}Car Dieu délivrera Sion et bâtira les villes de Juda ; on y habitera et on la possèdera.
\VS{37}Et la postérité de ses serviteurs en fera son héritage, et ceux qui aiment son Nom y auront leur demeure.
\Chap{70}
\TextTitle{Le pauvre et l'indigent}
\VerseOne{}Psaume de David, pour souvenir, donné au chef des chantres.
\VS{2}Dieu ! Hâte-toi de me délivrer, ô Dieu ! Hâte-toi de venir à mon secours\FTNT{Ps. 40:14 ; Ps. 71:12.}.
\VS{3}Que ceux qui cherchent mon âme soient honteux et rougissent\FTNT{Ps. 35:4 ; Ps. 71:13.} ; et que ceux qui prennent plaisir à mon mal soient repoussés en arrière et soient confus.
\VS{4}Que ceux qui disent : Aha ! Aha ! Retournent en arrière par l'effet de leur honte.
\VS{5}Que tous ceux qui te cherchent exultent et se réjouissent en toi ; et que ceux qui aiment ta délivrance disent toujours : Glorifié soit Dieu !
\VS{6}Moi, je suis affligé et misérable, ô Dieu ! Hâte-toi de venir vers moi ; tu es mon secours et mon libérateur, ô Yahweh ! Ne tarde point.
\Chap{71}
\TextTitle{Demeurer en Dieu jusqu'au bout}
\VerseOne{}Yahweh ! Je cherche en toi mon refuge : Que je ne sois jamais confus !
\VS{2}Délivre-moi par ta justice et sauve-moi. Incline ton oreille vers moi, mets-moi en sûreté.
\VS{3}Sois pour moi le rocher de mon refuge, afin que je puisse toujours m'y retirer ; tu as donné l'ordre de me mettre en sûreté, car tu es mon rocher et ma forteresse.
\VS{4}Mon Dieu ! Délivre-moi de la main du méchant, de la main du pervers et de l'oppresseur.
\VS{5}Car tu es mon espérance, Seigneur Yahweh ! Ma confiance dès ma jeunesse.
\VS{6}Je m'appuie sur toi dès le ventre de ma mère ; c'est toi qui m'as tiré hors des entrailles de ma mère\FTNT{Ps. 22:10-11.} ; tu es le sujet continuel de mes louanges.
\VS{7}Je suis pour plusieurs comme un miracle, mais tu es mon puissant refuge.
\VS{8}Que ma bouche soit remplie de ta louange et de ta gloire chaque jour.
\VS{9}Ne me rejette point au temps de ma vieillesse ; ne m'abandonne point maintenant que ma force est consumée.
\VS{10}Car mes ennemis ont parlé de moi, et ceux qui épient mon âme ont pris conseil ensemble,
\VS{11}disant : Dieu l'a abandonné. Poursuivez-le et saisissez-le, car il n'y a personne qui le délivre.
\VS{12}Dieu, ne t'éloigne point de moi ! Mon Dieu hâte-toi de venir à mon secours !
\VS{13}Que ceux qui sont les ennemis de mon âme soient honteux et défaits ; et que ceux qui cherchent mon malheur soient enveloppés d'opprobre et de honte.
\VS{14}Mais moi, j'espèrerai toujours et je te louerai tous les jours davantage.
\VS{15}Ma bouche racontera chaque jour ta justice et ta délivrance, bien que je n'en sache point le nombre.
\VS{16}Je marcherai par la force du Seigneur Yahweh ; je raconterai ta seule justice.
\VS{17}Ô Dieu ! Tu m'as enseigné dès ma jeunesse et j'ai annoncé jusqu'à présent tes merveilles.
\VS{18}Ô Dieu ! Ne m'abandonne pas, même dans la blanche vieillesse. Afin que j'annonce ta force à cette génération présente, ta puissance à la génération à venir.
\VS{19}Car ta justice, ô Dieu, est haut élevée, car tu as fait de grandes choses. Ô Dieu, qui est semblable à toi ?
\VS{20}Tu m'as fait éprouver bien des détresses et des malheurs, mais tu me redonneras la vie et tu me feras remonter hors des abîmes de la terre.
\VS{21}Relève ma grandeur et console-moi encore.
\VS{22}Je te louerai au son du luth, je chanterai ta fidélité, mon Dieu, je te célèbrerai avec la harpe, Saint d'Israël !
\VS{23}Mes lèvres et mon âme, que tu as rachetée, pousseront des cris de joie quand je te chanterai.
\VS{24}Ma langue aussi publiera chaque jour ta justice, car ceux qui cherchent mon malheur seront honteux et rougiront.
\Chap{72}
\TextTitle{Le royaume messianique}
\VerseOne{}De Salomon. Ô Dieu, donne tes jugements au roi et ta justice au fils du roi.
\VS{2}Qu'il juge avec justice ton peuple, et tes malheureux avec équité.
\VS{3}Que les montagnes portent la paix pour le peuple, et que les collines la portent en justice.
\VS{4}Qu'il fasse droit aux malheureux du peuple, qu'il délivre les fils du misérable, et qu'il écrase l'oppresseur !
\VS{5}Ils te craindront tant que le soleil et la lune dureront d'âge en âge.
\VS{6}Il descendra comme la pluie sur l'herbe fauchée, comme les ondées qui arrosent la terre.
\VS{7}En son temps, le juste fleurira, et il y aura abondance de paix jusqu'à ce qu'il n'y ait plus de lune.
\VS{8}Il dominera depuis une mer jusqu'à l'autre, et depuis le fleuve jusqu'aux extrémités de la terre.
\VS{9}Les habitants des déserts se courberont devant lui, et ses ennemis lécheront la poussière.
\VS{10}Les rois de Tarsis et des îles lui rapporteront des dons ; les rois de Saba et de Séba lui apporteront des présents.
\VS{11}Tous les rois aussi se prosterneront devant lui, toutes les nations le serviront.
\VS{12}Car il délivrera le pauvre qui crie vers lui, l'affligé et celui qui n'a personne qui l'aide\FTNT{Ps. 34:18 ; Job. 29:12.}.
\VS{13}Il aura compassion du pauvre et du misérable, et il sauvera les âmes des misérables.
\VS{14}Il garantira leur âme de la fraude et de la violence, et leur sang sera précieux devant ses yeux.
\VS{15}Il vivra donc, et on lui donnera de l'or de Séba, et on fera des prières pour lui continuellement ; on le bénira chaque jour.
\VS{16}Les blés abonderont dans le pays, au sommet des montagnes, et leurs épis s'agiteront comme les arbres du Liban ; les hommes fleuriront dans les villes comme l'herbe de la terre.
\VS{17}Sa renommée durera à toujours ; sa renommée ira de père en fils tant que le soleil durera ; et on se bénira en lui ; toutes les nations le diront heureux.
\VS{18}Béni soit Yahweh Dieu, le Dieu d'Israël, qui seul fait des choses merveilleuses !
\VS{19}Béni soit éternellement son Nom glorieux, et que toute la terre soit remplie de sa gloire. Amen ! Oui, amen !
\VS{20}Fin des prières de David, fils d'Isaï.
\Chap{73}
\TextTitle{L'orgueil des méchants}
\VerseOne{}Psaume d'Asaph. Quoi qu'il en soit, Dieu est bon pour Israël, pour ceux qui ont le cœur pur\FTNT{Mt. 5:8.}.
\VS{2}Toutefois, mes pieds allaient fléchir, mes pas étaient sur le point de glisser.
\VS{3}Car j'ai porté envie aux insensés en voyant la prospérité des méchants.
\VS{4}Rien ne les tourmente jusqu'à leur mort, et leur corps est gras.
\VS{5}Ils n'ont point de part aux peines des humains, et ils ne sont point frappés avec les autres hommes.
\VS{6}C'est pourquoi l'orgueil les environne comme un collier, et un vêtement de violence les couvre.
\VS{7}Les yeux leur sortent dehors à force de graisse ; ils surpassent les desseins de leur cœur.
\VS{8}Ils sont pernicieux, et parlent méchamment d'opprimer ; ils parlent d'une manière hautaine.
\VS{9}Ils élèvent leur bouche jusqu'aux cieux et leur langue parcourt la terre.
\VS{10}C'est pourquoi son peuple se tourne de leur côté, il avale l'eau abondamment.
\VS{11}Ils disent : Comment Dieu saurait-il ? Comment le Très-Haut connaîtrait-il\FTNT{Es. 29:15 ; Ez. 8:12 ; Ps. 94:7 ; Job. 22:12-13.} ?
\VS{12}Voilà, ceux-ci sont méchants, ils prospèrent toujours dans ce monde et acquièrent de plus en plus de richesses.
\VS{13}Quoi qu'il en soit, c'est donc en vain que j'ai purifié mon cœur et que j'ai lavé mes mains dans l'innocence\FTNT{Mal. 3:14 ; Job. 35:3}.
\VS{14}Je suis frappé tous les jours, et tous les matins mon châtiment est là.
\VS{15}Si je disais : Je veux parler comme eux, voici je trahirais la génération de tes fils.
\VS{16}Toutefois, j'ai tâché de connaître cela, mais cela m'a paru fort difficile,
\VS{17}jusqu'à ce que je sois entré dans le sanctuaire de Dieu et que j'aie considéré la fin de telles gens.
\VS{18}Quoi qu'il en soit, tu les as mis sur des voies glissantes, tu les fais tomber dans des précipices.
\VS{19}Comment ont-ils été ainsi détruits en un instant ? Ont-ils défailli ? Ont-ils été consumés d'épouvante ?
\VS{20}Ils sont comme un songe lorsqu'on s'est réveillé. Seigneur, tu méprises leur image à ton réveil.
\VS{21}Quand mon cœur s'aigrissait et que je me sentais percé dans les entrailles,
\VS{22}j'étais alors stupide, et je n'avais aucune connaissance ; j'étais comme une bête dans ta présence.
\VS{23}Je serai donc toujours avec toi ; tu m'as pris par la main droite.
\VS{24}Tu me conduiras par ton conseil, et tu me recevras dans la gloire.
\VS{25}Quel autre ai-je au ciel ? Or sur la terre je ne prends plaisir qu'en toi seul.
\VS{26}Ma chair et mon cœur étaient consumés, mais Dieu est le rocher de mon cœur, et mon partage pour toujours.
\VS{27}Car voilà, ceux qui s'éloignent de toi périront ; tu retrancheras tous ceux qui se détournent de toi.
\VS{28}Mais pour moi, m'approcher de Dieu c'est mon bien ; j'ai mis toute mon espérance dans le Seigneur Yahweh, afin de raconter toutes tes œuvres.
\Chap{74}
\TextTitle{Appel au secours du peuple de Dieu}
\VerseOne{}Cantique d'Asaph. Ô Dieu, pourquoi nous as-tu rejetés pour toujours ? Et pourquoi ta colère fume-t-elle contre le troupeau de ton pâturage\FTNT{Ps. 79:5.} ?
\VS{2}Souviens-toi de ton assemblée que tu as acquise autrefois. Tu t'es approprié cette montagne de Sion, sur laquelle tu habitais, afin qu'elle soit la portion de ton héritage.
\VS{3}Elève tes pas vers les lieux constamment dévastés ; l'ennemi a tout renversé dans le lieu saint.
\VS{4}Tes adversaires ont rugi au milieu de ton assemblée ; ils ont mis leurs signes pour signes.
\VS{5}On les a vus pareils à celui qui lève la cognée dans une épaisse forêt.
\VS{6}Et maintenant, avec des haches et des marteaux, ils brisent les sculptures.
\VS{7}Ils ont mis le feu à ton lieu saint. Ils ont abattu à terre et profané la demeure dédiée à ton Nom\FTNT{2 R. 25:9.}.
\VS{8}Ils ont dit en leur cœur : Saccageons-les tous ensemble ! Ils ont brûlé dans le pays tous les lieux saints de Dieu.
\VS{9}Nous ne voyons plus nos signes ; il n'y a plus de prophètes ; et personne parmi nous qui sache jusqu'à quand\FTNT{La. 2:9-10.}.
\VS{10}Ô Dieu ! Jusqu'à quand l'adversaire te couvrira-t-il d'opprobres et l'ennemi méprisera-t-il ton Nom à jamais ?
\VS{11}Pourquoi retires-tu ta main, même ta droite ? Consume-les en la tirant du milieu de ton sein !
\VS{12}Or Dieu est mon Roi dès les temps anciens, faisant des délivrances au milieu de la terre.
\VS{13}Tu as fendu la mer par ta force ; tu as brisé les têtes des serpents sur les eaux.
\VS{14}Tu as brisé les têtes du léviathan, tu l'as donné pour nourriture au peuple du désert.
\VS{15}Tu as ouvert la fontaine et le torrent, tu as desséché les grosses rivières.
\VS{16}A toi est le jour, à toi aussi est la nuit ; tu as établi la lumière et le soleil.
\VS{17}Tu as posé toutes les limites de la terre ; tu as formé l'été et l'hiver.
\VS{18}Souviens-toi de ceci : Que l'ennemi a blasphémé Yahweh et qu'un peuple insensé a outragé ton Nom.
\VS{19}Ne livre pas aux vivants l'âme de la tourterelle, n'oublie pas à toujours la vie de tes affligés.
\VS{20}Regarde à ton alliance, car les lieux ténébreux de la terre sont remplis d'habitations de violence.
\VS{21}Ne permets pas que celui qui est foulé s'en retourne tout confus. Que l'affligé et le pauvre louent ton Nom !
\VS{22}Ô Dieu ! Lève-toi, défends ta cause, souviens-toi de l'opprobre qui t'est fait tous les jours par l'insensé !
\VS{23}N'oublie pas le cri de tes adversaires, le bruit de ceux qui s'élèvent contre toi monte continuellement !
\Chap{75}
\TextTitle{L'élevation vient de Yahweh}
\VerseOne{}Psaume d'Asaph. Cantique donné au chef des chantres, pour le chanter sur Al-Thasheth\FTNT{Voir Ps. 57:1.}.
\VS{2}Nous te célébrons, ô Dieu ! Nous te célébrons et ton Nom est près de nous ; nous racontons tes merveilles.
\VS{3}Au temps que j'aurai fixé, je jugerai avec droiture.
\VS{4}La terre se dissout avec tous ceux qui y habitent, mais j'affermis ses piliers. Sélah.
\VS{5}Je dis aux insensés : N'agissez point follement ; et aux méchants : N'élevez pas la tête.
\VS{6}N'élevez pas si haut votre tête, et ne parlez point avec fierté.
\VS{7}Car l'élévation ne vient point d'orient, ni d'occident ni du désert.
\VS{8}Car c'est Dieu qui gouverne ; il abaisse l'un, et élève l'autre\FTNT{1 S. 2:7.}.
\VS{9}Il y a une coupe dans la main de Yahweh\FTNT{Es. 51:17-22 ; Jé. 25:27-28 ; Ap. 14:10 ; Ap. 16:19.}, et le vin rougit dedans ; il est plein de mélange, et Dieu en verse ; certainement, tous les méchants de la terre en suceront et en boiront jusqu'à la lie.
\VS{10}Mais moi, je raconterai ces choses à jamais, je chanterai au Dieu de Jacob.
\VS{11}J'humilierai tous les méchants, mais les justes seront élevés.
\Chap{76}
\TextTitle{La Puissance du Dieu redoutable}
\VerseOne{}Psaume d'Asaph. Cantique donné au chef des chantres, pour le chanter avec instruments à cordes.
\VS{2}Dieu est connu en Judée, sa renommée est grande en Israël ;
\VS{3}sa tente est à Salem et sa demeure à Sion.
\VS{4}Là il a brisé les arcs étincelants, le bouclier, l'épée et les armes de guerre. Sélah.
\VS{5}Tu es resplendissant, plus magnifique que les montagnes des ravisseurs.
\VS{6}Les plus courageux sont étourdis, ils sont dans un profond assoupissement, et aucun de ces hommes vaillants n'a trouvé ses mains.
\VS{7}Ô Dieu de Jacob, les cavaliers et les chevaux se sont endormis quand tu les as menacés.
\VS{8}Tu es redoutable, toi. Qui peut se tenir devant toi quand ta colère éclate ?
\VS{9}Tu fais entendre des cieux le jugement ; la terre en a eu peur et s'est tenue dans le silence.
\VS{10}Quand tu te lèves, ô Dieu, pour faire jugement, pour délivrer tous les malheureux de la terre ! Sélah.
\VS{11}L'homme te célèbre, même dans sa fureur, quand tu te ceins de toute ta colère.
\VS{12}Faites vos vœux à Yahweh votre Dieu et accomplissez-les ! Que tous ceux qui l'environnent apportent des dons au Dieu terrible !
\VS{13}Il coupe le souffle des princes ; il est redoutable aux rois de la terre.
\Chap{77}
\TextTitle{Se souvenir des prodiges de Yahweh}
\VerseOne{}Psaume d'Asaph, donné au chef des chantres, d'après Jeduthun.
\VS{2}Ma voix s'élève à Dieu, et je crie ; ma voix s'adresse à Dieu, et il m'écoutera.
\VS{3}Je cherche le Seigneur au jour de ma détresse ; sans cesse mes mains s'étendent durant la nuit ; mon âme refuse d'être consolée.
\VS{4}Je me souviens de Dieu, et je gémis ; je médite, et mon esprit est affaibli. Sélah.
\VS{5}Tu empêches mes yeux de dormir ; je suis troublé, et ne peux parler.
\VS{6}Je pense aux jours d'autrefois et aux années des siècles passées\FTNT{Ps. 143:5.}.
\VS{7}Je me souviens de mes chants pendant la nuit, je médite en mon cœur, et mon esprit cherche diligemment.
\VS{8}Le Seigneur m'a-t-il rejeté pour toujours ? Ne me sera-t-il plus favorable ?
\VS{9}Sa bonté est-elle disparue pour toujours ? Sa parole a-t-elle pris fin pour l'éternité ?
\VS{10}Dieu a-t-il oublié d'avoir compassion ? A-t-il dans sa colère retiré sa miséricorde ? Sélah.
\VS{11}Je dis : Ce qui me fait devenir malade, je me souviendrai des années de la droite du Très–Haut.
\VS{12}Je me souviens des exploits de Yahweh ; je me suis, dis-je, souvenu de tes merveilles d'autrefois.
\VS{13}Je méditerai toutes tes œuvres, et je parlerai de tes œuvres.
\VS{14}Ô Dieu ! Tes voies sont saintes. Quel dieu est grand comme Dieu ?
\VS{15}Tu es le Dieu qui fait des merveilles ! Tu as fait connaître ta force parmi les peuples.
\VS{16}Tu as délivré par ton bras ton peuple, les fils de Jacob et de Joseph. Sélah.
\VS{17}Les eaux t'ont vu, ô Dieu ! Les eaux t'ont vu et ont tremblé, même les abîmes en ont été émus.
\VS{18}Les nuées ont versé un déluge d'eau, les nuées ont fait retentir leur son ; tes flèches ont volé de toutes parts.
\VS{19}La voix de ton tonnerre était dans le tourbillon, les éclairs ont éclairé le monde, la terre en a été émue et en a tremblé.
\VS{20}Tu te frayas un chemin par la mer, un sentier par les grosses eaux ; et tes traces ne furent plus reconnues.
\VS{21}Tu as mené ton peuple comme un corps d'armée sous la conduite de Moïse et d'Aaron\FTNT{Mi. 6:4.}.
\Chap{78}
\TextTitle{Les œuvres de Dieu dans l'histoire d'Israël}
\VerseOne{}Cantique d'Asaph. Mon peuple, écoute ma loi, prêtez vos oreilles aux paroles de ma bouche.
\VS{2}J'ouvrirai ma bouche en une parabole ; je proférerai les énigmes cachées des temps anciens\FTNT{Mt. 13:35.}.
\VS{3}Ce que nous avons entendu et connu, et que nos pères nous ont raconté\FTNT{Ps. 44:2.},
\VS{4}nous ne le cacherons point à leurs fils. Ils raconteront à la génération à venir les louanges de Yahweh, sa puissance et ses merveilles qu'il a faites.
\VS{5}Car il a établi le témoignage en Jacob, et il a mis la loi en Israël ; il a donné cet ordre à nos pères de la faire connaître à leurs fils\FTNT{De. 4:9.},
\VS{6}pour qu'elle soit connue de la génération future, des fils qui naîtraient, et pour que lorsqu'ils seront grands, ils la relatent à leurs fils,
\VS{7}afin qu'ils mettent leur confiance en Dieu, et qu'ils n'oublient point les œuvres de Dieu, et qu'ils gardent ses commandements.
\VS{8}Afin qu'ils ne soient point comme leurs pères, une génération revêche et rebelle, une génération insoumise de cœur, dont l'esprit est infidèle à Dieu\FTNT{Ex. 32:9 ; Ac. 7:51.}.
\VS{9}Les fils d'Ephraïm, armés et tirant de l'arc, tournèrent le dos le jour de la bataille.
\VS{10}Ils ne gardèrent point l'alliance de Dieu et refusèrent de marcher selon sa loi.
\VS{11}Ils oublièrent ses œuvres et ses merveilles qu'il leur avait fait voir.
\VS{12}Il avait fait des miracles en présence de leurs pères, dans le pays d'Egypte, dans le champ de Tsoan.
\VS{13}Il fendit la mer et les fit passer au travers ; et il fit arrêter les eaux comme un monceau de pierres.
\VS{14}Il les conduisit de jour par la nuée, et toute la nuit par une lumière de feu\FTNT{Ex. 13:21.}.
\VS{15}Il fendit les rochers au désert, et leur donna à boire d'abondantes eaux, comme s'il eût puisé des abîmes.
\VS{16}Il fit sortir des ruisseaux de la roche\FTNT{Ex. 17:6 ; No. 20:11 ; 1 Co. 10:4.} et fit couler des eaux comme des rivières.
\VS{17}Toutefois, ils continuèrent à pécher contre lui, irritant le Très-Haut dans le désert.
\VS{18}Ils tentèrent Dieu dans leurs cœurs, en demandant de la viande selon leur désir.
\VS{19}Ils parlèrent contre Dieu, disant : Dieu pourrait-il dresser une table dans ce désert\FTNT{No. 11:4.} ?
\VS{20}Voilà, dirent-ils, il a frappé le rocher, et les eaux ont coulé et des torrents ont débordé ; mais pourrait-il aussi nous donner du pain ? Fournirait-il de la viande à son peuple ?
\VS{21}C'est pourquoi, Yahweh les ayant entendus, se mit dans une grande colère, et le feu s'embrasa contre Jacob, et sa colère s'excita contre Israël.
\VS{22}Parce qu'ils n'avaient point cru en Dieu et ne s'étaient point confiés en sa délivrance.
\VS{23}Il ordonna aux nuées d'en haut et il ouvrit les portes des cieux ;
\VS{24}il fit pleuvoir la manne sur eux pour leur nourriture et il leur donna le blé du ciel\FTNT{Ex. 16:14 ; Jn. 6:31.}.
\VS{25}Ils mangèrent tous le pain des grands. Il leur envoya de la viande pour s'en rassasier.
\VS{26}Il excita dans les cieux le vent d'orient et il amena par sa puissance le vent du sud.
\VS{27}Il fit pleuvoir sur eux de la viande comme de la poussière, et comme le sable des mers des oiseaux ailées.
\VS{28}Il les fit tomber au milieu du camp, autour de leurs demeures.
\VS{29}Ils en mangèrent et en furent pleinement rassasiés, car il leur donna selon leur désir.
\VS{30}Mais ils ne furent pas encore dégoûtés de leur désir, et leur viande était encore dans leur bouche
\VS{31}quand la colère de Dieu s'excita contre eux, et qu'il mit à mort les plus gras d'entre eux, et abattit les gens d'élite d'Israël\FTNT{1 Co. 10:5.}.
\VS{32}Malgré cela, ils péchèrent encore et ne crurent point à ses prodiges\FTNT{No. 14:2.}.
\VS{33}C'est pourquoi il consuma leurs jours par la vanité et leurs années par une fin soudaine.
\VS{34}Quand il les mettait à mort, alors ils le recherchaient ; ils se repentaient et ils cherchaient Dieu dès le matin.
\VS{35}Ils se souvenaient que Dieu était leur rocher, et Dieu, le Très-Haut, était leur libérateur.
\VS{36}Mais ils le trompaient de leur bouche et ils lui mentaient de leur langue\FTNT{Es. 29:13 ; Jé. 12:2 ; Mt. 15:8.} ;
\VS{37}car leur cœur n'était point droit envers lui, et ils ne furent point fidèles à son alliance.
\VS{38}Toutefois, comme il est compatissant, il pardonna leur iniquité, au point qu'il ne les détruisit pas ; mais il détourna souvent sa colère et ne réveilla pas toute sa fureur.
\VS{39}Il se souvint qu'ils n'étaient que chair, qu'un vent qui passe et qui ne revient point.
\VS{40}Combien de fois l'ont–ils irrité au désert, et combien de fois l'ont–ils attristé dans ce lieu inhabitable ?
\VS{41}Ils ne cessèrent de tenter Dieu et de provoquer le Saint d'Israël.
\VS{42}Ils ne se souvinrent point de sa puissance, du jour où il les délivra de la main de l'ennemi,
\VS{43}des miracles qu'il accomplit en Egypte, et de ses merveilles dans les champs de Tsoan.
\VS{44}Il changea en sang leurs fleuves et leurs ruisseaux et ils ne purent en boire les eaux\FTNT{Ex. 7:20.}.
\VS{45}Il envoya contre eux des mouches qui les dévorèrent et des grenouilles qui les détruisirent\FTNT{Ex. 8:6-24.}.
\VS{46}Il livra leurs récoltes aux sauterelles, le produit de leur travail aux sauterelles\FTNT{Ex. 10:13.}.
\VS{47}Il détruisit leurs vignes par la grêle, et leurs sycomores par les orages\FTNT{Ex. 9:23.}.
\VS{48}Il livra leur bétail à la grêle, et leurs troupeaux aux foudres étincelantes.
\VS{49}Il envoya sur eux l'ardeur de sa colère, la fureur, la rage et la détresse, un corps d'armée de messagers de malheur.
\VS{50}Il donna libre cours à sa colère, et ne retira point leur âme de la mort ; il livra leur vie à la peste\FTNT{Ex. 9:6.}.
\VS{51}Il frappa tout premier-né en Egypte, les prémices de la vigueur dans les tentes de Cham\FTNT{Ex. 12:29.}.
\VS{52}Il fit partir son peuple comme des brebis, il les mena comme un corps d'armée dans le désert.
\VS{53}Il les conduisit sûrement, et sans qu'ils eussent aucune frayeur, là où la mer couvrit leurs ennemis.
\VS{54}Il les amena vers sa frontière sainte, vers cette montagne que sa droite a acquise\FTNT{Ex. 15:17.}.
\VS{55}Il chassa devant eux les nations, leur distribua le pays en héritage, et fit habiter les tribus d'Israël dans les tentes de ces nations.
\VS{56}Mais ils tentèrent et irritèrent le Dieu Très-Haut, et ne gardèrent point ses préceptes.
\VS{57}Et ils se retirèrent en arrière et furent infidèles comme leurs pères ; ils tournèrent comme un arc trompeur.
\VS{58}Ils le provoquèrent à la colère par leurs hauts lieux, et l'émurent à la jalousie par leurs images taillées\FTNT{De. 32:16-21.}.
\VS{59}Dieu l'entendit et se mit dans une grande colère, et il méprisa fortement Israël.
\VS{60}Il abandonna la demeure de Silo, la tente où il habitait parmi les hommes.
\VS{61}Il livra en captivité sa force et son ornement entre les mains de l'ennemi.
\VS{62}Il livra son peuple à l'épée et se mit dans une grande colère contre son héritage.
\VS{63}Le feu consuma leurs gens d'élite, et leurs vierges ne furent point louées.
\VS{64}Leurs sacrificateurs tombèrent par l'épée, et leurs veuves ne les pleurèrent point.
\VS{65}Puis le Seigneur se réveilla comme un homme qui se serait endormi, et comme un puissant homme qui s'écrie ayant encore le vin dans la tête.
\VS{66}Il frappa ses adversaires par derrière et les mit en opprobre perpétuel.
\VS{67}Mais il dédaigna la tente de Joseph, et ne choisit point la tribu d'Ephraïm.
\VS{68}Mais il choisit la tribu de Juda, la montagne de Sion, celle qu'il aime.
\VS{69}Il bâtit son lieu saint dans les lieux élevés, et l'établit comme la terre qu'il a fondée pour toujours.
\VS{70}Il choisit David, son serviteur, et le prit de la bergerie\FTNT{1 S. 16:11 ; 2 S. 7:8.} ;
\VS{71}il le prit derrière les brebis qui allaitent et l'amena pour paître Jacob, son peuple, et Israël, son héritage.
\VS{72}Aussi il les dirigea selon l'intégrité de son cœur, et les conduisit avec des mains intelligentes.
\Chap{79}
\TextTitle{Appel au jugement de Dieu}
\VerseOne{}Psaume d'Asaph. Ô Dieu ! Les nations sont entrées dans ton héritage ; on a profané ton saint temple, on a mis Jérusalem en monceaux de pierres.
\VS{2}On a livré les cadavres de tes serviteurs pour viande aux oiseaux du ciel, et la chair de tes fidèles aux bêtes de la terre.
\VS{3}On a répandu leur sang comme de l'eau autour de Jérusalem, et il n'y a eu personne pour les enterrer.
\VS{4}Nous sommes un sujet d'opprobre à nos voisins, de moquerie et de risée à ceux qui habitent autour de nous\FTNT{Ps. 44:14 ; Ps. 80:7.}.
\VS{5}Jusqu'à quand, ô Yahweh, t'irriteras-tu sans cesse et ta jalousie s'embrasera-t-elle comme un feu\FTNT{Ps. 89:47.} ?
\VS{6}Répands ta fureur sur les nations qui ne te connaissent point et sur les royaumes qui n'invoquent point ton Nom\FTNT{Jé. 10:25.}.
\VS{7}Car on a dévoré Jacob et on a ravagé ses demeures.
\VS{8}Ne rappelle point devant nous les iniquités passées. Que tes compassions viennent en hâte au-devant de nous, car nous sommes dans une extrême détresse.
\VS{9}Ô Dieu de notre délivrance ! Aide-nous pour l'amour de la gloire de ton Nom, et délivre-nous ! Pardonne-nous nos péchés pour l'amour de ton Nom !
\VS{10}Pourquoi les nations diraient-elles : Où est leur Dieu ? Que la vengeance du sang de tes serviteurs, qui a été répandu, soit manifestée parmi les nations en notre présence.
\VS{11}Que le gémissement des captifs parviennent jusqu'à toi. Par ton bras puissant sauve tes fils, ceux qui vont périr !
\VS{12}Et rends à nos voisins, dans leur sein, sept fois au double l'opprobre qu'ils t'ont fait, ô Yahweh !
\VS{13}Mais nous, ton peuple, et le troupeau de ton pâturage, nous te louerons pour toujours, et de génération en génération nous publierons tes louanges.
\Chap{80}
\TextTitle{Implorer Yahweh}
\VerseOne{}Psaume d'Asaph, donné au chef des chantres, pour le chanter Sosannim-héduth.
\VS{2}Toi qui pais Israël, prête l'oreille ! Toi qui mènes Joseph comme un troupeau, toi qui es assis entre les chérubins\FTNT{2 S. 6:2 ; Es. 37:16 ; Ps. 99:1.}, fais briller ta splendeur !
\VS{3}Réveille ta puissance au-devant d'Ephraïm, de Benjamin et de Manassé ; et viens pour notre délivrance !
\VS{4}Dieu, ramène-nous et fais briller ta face ! Et nous serons délivrés !
\VS{5}Ô Yahweh, Dieu des armées, jusqu'à quand seras-tu irrité contre la prière de ton peuple ?
\VS{6}Tu les nourris de pain de larmes et tu les abreuves de larmes à pleine mesure.
\VS{7}Tu fais de nous un sujet de dispute entre nos voisins, et nos ennemis se moquent de nous.
\VS{8}Ô Dieu des armées, ramène-nous et fais briller ta face ! Et nous serons délivrés.
\VS{9}Tu avais retiré une vigne hors d'Egypte, tu as chassé les nations, et tu l'as plantée\FTNT{Es. 5:1-7 ; Os. 10:1 ; Mt. 20:1 ; Mt. 21:28-33.}.
\VS{10}Tu as préparé une place devant elle, tu lui as fait prendre racine, et elle a rempli la terre.
\VS{11}Les montagnes étaient couvertes de son ombre, et ses rameaux étaient comme de hauts cèdres de Dieu.
\VS{12}Elle étendait ses branches jusqu'à la mer, et ses rejetons jusqu'au fleuve.
\VS{13}Pourquoi as-tu rompu ses clôtures, de sorte que tous les passants sur la route cueillent ses raisins ?
\VS{14}Les sangliers de la forêt l'ont détruite, et toutes les bêtes des champs en font leur pâture.
\VS{15}Ô Dieu des armées, reviens ! Regarde des cieux, vois, et visite cette vigne ;
\VS{16}et le plant que ta droite avait planté, et le fils que tu t'es choisi.
\VS{17}Elle est brûlée par le feu, elle est coupée ; ils périssent devant ta face menaçante.
\VS{18}Que ta main soit sur l'homme de ta droite, sur le fils de l'homme que tu t'es choisi.
\VS{19}Et nous ne nous éloignerons plus de toi. Rends-nous la vie, et nous invoquerons ton Nom.
\VS{20}Ô Yahweh ! Dieu des armées, ramène-nous, fais briller ta face, et nous serons délivrés !
\Chap{81}
\TextTitle{Se débarasser des dieux étrangers}
\VerseOne{}Psaume d'Asaph, donné au chef des chantres, pour le chanter sur la Guitthith.
\VS{2}Chantez avec allégresse à notre Dieu, notre force ! Poussez des cris de joie en l'honneur du Dieu de Jacob.
\VS{3}Sonnez du shofar, prenez le tambour, la harpe mélodieuse et le luth.
\VS{4}Sonnez du shofar à la nouvelle lune, à la pleine lune, au jour de notre fête\FTNT{No. 10:10.}.
\VS{5}Car c'est une loi pour Israël, une ordonnance du Dieu de Jacob.
\VS{6}Il établit un statut à Joseph, lorsqu'il marcha contre le pays d'Egypte, où j'entendis un langage que je ne connaissais pas.
\VS{7}J'ai retiré son épaule du fardeau, et ses mains ont lâché les corbeilles.
\VS{8}Tu as crié dans la détresse, et je t'ai sauvé ; je t'ai répondu dans le lieu caché du tonnerre ; je t'ai éprouvé auprès des eaux de Mériba. Sélah.
\VS{9}Ecoute mon peuple, je te relèverai. Israël, si tu m'écoutais !
\VS{10}Qu'il n'y ait point de dieu étranger au milieu de toi, et ne te prosterne point devant les dieux des étrangers.
\VS{11}Je suis Yahweh, ton Dieu, qui t'ai fait monter hors du pays d'Egypte. Ouvre ta bouche et je la remplirai.
\VS{12}Mais mon peuple n'a point écouté ma voix, et Israël ne m'a point obéi.
\VS{13}C'est pourquoi je les ai abandonnés aux penchants de leur cœur, et ils ont suivi leurs propres conseils\FTNT{Es. 63:17 ; Es. 65:2 ; 2 Pi. 3:3.}.
\VS{14}Ô si mon peuple m'écoutait ! Si Israël marchait dans mes voies !
\VS{15}J'abattrais en un instant leurs ennemis et je tournerais ma main contre leurs adversaires.
\VS{16}Ceux qui haïssent Yahweh le flatteraient, et le bonheur de mon peuple durerait toujours.
\VS{17}Dieu le nourrirait du meilleur froment ; et je le rassasierais du miel du rocher.
\Chap{82}
\TextTitle{Dieu dénonce l'injustice des hommes}
\VerseOne{}Psaume d'Asaph. Dieu se tient dans l'assemblée de Dieu, il juge au milieu des juges.
\VS{2}Jusqu'à quand jugerez-vous injustement et aurez-vous égard à l'apparence de la personne des méchants\FTNT{Ps. 58:2.} ? Sélah.
\VS{3}Faites droit à celui qu'on opprime et à l'orphelin ; faites justice à l'affligé et au pauvre ;
\VS{4}délivrez celui qu'on maltraite et le misérable, retirez-le de la main des méchants.
\VS{5}Ils ne connaissent ni n'entendent rien ; ils marchent dans les ténèbres, tous les fondements de la terre sont ébranlés.
\VS{6}J'ai dit : Vous êtes des dieux\FTNT{Jn. 10:34.}, et vous êtes tous fils du Très-Haut.
\VS{7}Toutefois, vous mourrez comme des hommes, et vous les princes vous tomberez comme les autres.
\VS{8}Ô Dieu ! Lève-toi, juge la terre ; car tu auras en héritage toutes les nations\FTNT{Ps. 2:8 ; Hé. 1:2.}.
\Chap{83}
\TextTitle{Dessein et confusion des ennemis d'Israël}
\VerseOne{}Cantique et psaume d'Asaph.
\VS{2}Ô Dieu ! Ne garde point le silence, ne te tais point, et ne te tiens point en repos, ô Dieu\FTNT{Ps. 35:22.} !
\VS{3}Car voici, tes ennemis s'agitent, et ceux qui te haïssent ont levé la tête.
\VS{4}Ils ont consulté finement en secret contre ton peuple, et ils ont tenu conseil contre ceux qui se sont retirés vers toi pour se cacher\FTNT{Ps. 2:2.}.
\VS{5}Ils disent : Venez et détruisons-les, en sorte qu'ils ne soient plus une nation, et qu'on ne fasse plus mention du nom d'Israël\FTNT{Ce passage fait allusion aux désirs qu'ont certaines nations de voir Israël détruite Mi. 4:11 ; Ap. 11:1-2.}.
\VS{6}Car ils consultent ensemble d'un même esprit ; ils font alliance contre toi.
\VS{7}Les tentes d'Edom et des Ismaélites, des Moabites et des Hagaréniens ;
\VS{8}de Guebal, d'Ammon, d'Amalek, les Philistins avec les habitants de Tyr.
\VS{9}L'Assyrie aussi se joint à eux ; ils ont servi de bras aux fils de Lot. Sélah.
\VS{10}Fais-leur comme tu fis à Madian\FTNT{Jg. 7:15.}, comme à Sisera\FTNT{Jg. 4:15.}, et comme à Jabin, auprès du torrent de Kison !
\VS{11}Ils furent détruits à En-Dor et servirent de fumier à la terre.
\VS{12}Que leurs chefs soient traités comme Oreb et comme Zeeb ; et que tous leurs princes soient comme Zébach et Tsalmunna\FTNT{Jg. 7:25.} ;
\VS{13}parce qu'ils ont dit : Prenons possession des habitations agréables de Dieu.
\VS{14}Mon Dieu ! Rends-les semblables au tourbillon et au chaume chassé par le vent,
\VS{15}comme le feu brûle une forêt, et comme la flamme embrase les montagnes.
\VS{16}Poursuis-les ainsi par ta tempête et épouvante-les par ton tourbillon !
\VS{17}Couvre leurs visages d'ignominie afin qu'on cherche ton Nom, ô Yahweh !
\VS{18}Qu'ils soient honteux et épouvantés à jamais, qu'ils rougissent, et qu'ils périssent ;
\VS{19}afin qu'on sache que toi seul, dont le nom est Yahweh, tu es le Très-Haut sur toute la terre.
\Chap{84}
\TextTitle{Délices pour ceux qui ont Yahweh comme appui}
\VerseOne{}Psaume des fils de Koré, donné au chef des chantres, pour le chanter sur la Guitthith.
\VS{2}Yahweh des armées, que tes demeures sont aimables !
\VS{3}Mon âme soupire et languit après les parvis de Yahweh ; mon cœur et ma chair poussent des cris de joie vers le Dieu vivant.
\VS{4}Le passereau même trouve sa maison, et l'hirondelle son nid où elle a mis ses petits… Tes autels, ô Yahweh des armées ! Mon Roi et mon Dieu !
\VS{5}Heureux ceux qui habitent ta maison et qui te louent sans cesse ! Sélah.
\VS{6}Heureux l'homme dont la force est en toi, ils trouvent dans leur cœur des chemins tout tracés !
\VS{7}Passant par la vallée de Baca, ils la réduisent en fontaine ; la pluie la couvre de bénédictions.
\VS{8}Ils marchent avec force pour se présenter devant Dieu à Sion.
\VS{9}Yahweh Dieu des armées, écoute ma prière, Dieu de Jacob, prête l'oreille. Sélah.
\VS{10}Ô Dieu, notre bouclier, vois et regarde la face de ton oint !
\VS{11}Car mieux vaut un jour dans tes parvis, que mille ailleurs. J'aimerais mieux me tenir à la porte dans la maison de mon Dieu, que de demeurer dans les tentes des méchants.
\VS{12}Car Yahweh Dieu est un soleil et un bouclier\FTNT{Ge. 15:1 ; Ps. 89:19 ; Ps. 144:2.} ; Yahweh donne la grâce et la gloire, et il ne refuse aucun bien à ceux qui marchent dans l'intégrité.
\VS{13}Yahweh des armées, heureux l'homme qui se confie en toi\FTNT{Ps. 2:12.} !
\Chap{85}
\TextTitle{Supplication des rescapés de l'exil}
\VerseOne{}Psaume des fils de Koré, donné au chef des chantres.
\VS{2}Yahweh, tu as été favorable à ta terre, tu as ramené et mis en repos les prisonniers de Jacob.
\VS{3}Tu as pardonné l'iniquité de ton peuple, tu as couvert tous leurs péchés. Sélah.
\VS{4}Tu as retiré toute ta colère, tu es revenu de l'ardeur de ton indignation.
\VS{5}Ô Dieu de notre délivrance, rétablis-nous et fais cesser la colère que tu as contre nous.
\VS{6}Seras-tu irrité à jamais contre nous ? Feras-tu durer ta colère de génération en génération ?
\VS{7}Ne reviendras–tu pas nous rendre la vie\FTNT{Ps. 71:20.}, afin que ton peuple se réjouisse en toi ?
\VS{8}Yahweh, fais-nous voir ta miséricorde et accorde-nous ta délivrance !
\VS{9}J'écouterai ce que dira Dieu, Yahweh ; car il parlera de paix à son peuple et à ses bien-aimés, pourvu que jamais ils ne retournent à leur folie.
\VS{10}Certainement sa délivrance est proche de ceux qui le craignent, la gloire habite dans notre pays.
\VS{11}La bonté et la vérité se rencontrent ; la justice et la paix s'embrassent\FTNT{Hé. 7:2.}.
\VS{12}La vérité germe de la terre et la justice regarde des cieux.
\VS{13}Yahweh aussi donne le bien et notre terre rendra son fruit.
\VS{14}La justice marchera devant lui, et il la mettra partout où il passera.
\Chap{86}
\TextTitle{Coeur disposé à la crainte de Dieu}
\VerseOne{}Prière de David. Yahweh, écoute, réponds-moi, car je suis affligé et misérable.
\VS{2}Garde mon âme, car je suis un de tes bien-aimés ; ô toi mon Dieu, délivre ton serviteur qui se confie en toi !
\VS{3}Seigneur, aie pitié de moi, car je crie à toi tout le jour.
\VS{4}Réjouis l'âme de ton serviteur, car j'élève mon âme à toi, Seigneur.
\VS{5}Yahweh ! Tu es bon et clément, et d'une grande bonté envers tous ceux qui t'invoquent\FTNT{Joë. 2:13.}.
\VS{6}Yahweh, prête l'oreille à ma prière, et sois attentif à la voix de mes supplications.
\VS{7}Je t'invoque au jour de ma détresse, car tu m'exauces\FTNT{Ps. 50:15.}.
\VS{8}Seigneur, nul n'est comme toi parmi les dieux, et rien ne ressemble à tes œuvres\FTNT{De. 3:24 ; Ps. 95:3.}.
\VS{9}Seigneur, toutes les nations que tu as faites viendront et se prosterneront devant toi, et glorifieront ton Nom,
\VS{10}car tu es grand, et tu fais des choses merveilleuses ; tu es Dieu, toi seul.
\VS{11}Yahweh ! Enseigne-moi tes voies et je marcherai dans ta vérité\FTNT{Ps. 25:4 ; Ps. 27:11.} ; lie mon cœur à la crainte de ton Nom.
\VS{12}Seigneur, mon Dieu, je te célébrerai de tout mon cœur, et je glorifierai ton Nom à toujours.
\VS{13}Car ta bonté est grande envers moi, et tu as retiré mon âme du profond scheol.
\VS{14}Ô Dieu ! Des gens orgueilleux se sont élevés contre moi, et un corps d'armée de méchants en veut à ma vie ; ils ne portent pas leurs pensées sur toi.
\VS{15}Mais toi, Seigneur, tu es le Dieu compatissant, miséricordieux, lent à la colère, riche en bonté et en vérité.
\VS{16}Tourne-toi vers moi, et aie pitié de moi ! Donne ta force à ton serviteur, délivre le fils de ta servante !
\VS{17}Accorde–moi un signe de ta faveur, et que ceux qui me haïssent le voient et soient honteux, parce que tu m'aideras, ô Yahweh ! Tu me consoleras !
\Chap{87}
\TextTitle{Sion, la cité de Dieu}
\VerseOne{}Psaume. Cantique des fils de Koré. Elle est fondée sur les montagnes saintes.
\VS{2}Yahweh aime les portes de Sion, plus que toutes les demeures de Jacob.
\VS{3}Ce qui se dit de toi, cité de Dieu, sont des choses glorieuses. Sélah.
\VS{4}Je ferai mention de Rahab et de Babylone parmi ceux qui me connaissent ; voici le pays des philistins, et Tyr avec l'Ethiopie. C'est dans Sion qu'ils sont nés.
\VS{5}Et de Sion il est dit : Un homme y est né ; le Très-Haut lui-même l'établira.
\VS{6}Yahweh compte en inscrivant les peuples : C'est là qu'ils sont nés. Sélah.
\VS{7}Et les chantres, de même que les joueurs de flûte, toutes mes sources sont en toi.
\Chap{88}
\TextTitle{Lamentation dans l'affliction}
\VerseOne{}Cantique. Psaume des fils de Koré, donné au chef des chantres. Pour chanter sur la flûte. Cantique d'Héman, l'Ezrahite.
\VS{2}Yahweh ! Dieu de ma délivrance, je crie jour et nuit devant toi\FTNT{Lu. 18:7.}.
\VS{3}Que ma prière parvienne en ta présence ; étends ton oreille à mon cri.
\VS{4}Car mon âme est rassasiée de maux, et ma vie atteint le scheol\FTNT{Lu. 16:23.}.
\VS{5}On m'a mis au rang de ceux qui descendent dans la fosse\FTNT{Ps. 28:1 ; Ps. 31:13.} ; je suis devenu comme un homme qui n'a plus de vigueur.
\VS{6}Je suis étendu parmi les morts, semblable à ceux qui sont tués et couchés dans la tombe, à ceux dont tu n'as plus le souvenir, et qui sont séparés par ta main.
\VS{7}Tu m'as jeté dans une fosse profonde, dans les ténèbres, dans les abîmes.
\VS{8}Ta fureur se pose sur moi, et tu m'as accablé de tous tes flots. Sélah.
\VS{9}Tu as éloigné de moi ceux de qui j'étais connu, tu m'as mis en abomination devant eux ; je suis enfermé et je ne peux sortir.
\VS{10}Mes yeux se consument dans la souffrance ; Yahweh ! Je crie à toi tout le jour ! J'étends mes mains vers toi\FTNT{Ex. 9:29 ; 1 R. 8:22. ; Job. 17:7} !
\VS{11}Est-ce pour les morts que tu fais des miracles ? Les morts se relèveront-ils pour te célébrer\FTNT{1 Th. 4:16 ; 1 Co. 15:12-13.} ? Sélah.
\VS{12}Parle-t-on de ta bonté dans le sépulcre, de ta fidélité dans le tombeau\FTNT{Ep. 4:9-10 ; 1 Pi. 3:18-20.} ?
\VS{13}Connaîtra-t-on tes merveilles dans les ténèbres et ta justice dans la terre de l'oubli ?
\VS{14}Mais moi, ô Yahweh ! J'implore ton secours, ma prière s'élève dès le matin.
\VS{15}Yahweh ! Pourquoi rejettes-tu mon âme, pourquoi me caches-tu ta face\FTNT{Mt. 27:46 ; Mc. 15:34.} ?
\VS{16}Je suis malheureux et moribond dès ma jeunesse ; j'ai été exposé à tes terreurs, et je ne sais pas où j'en suis.
\VS{17}Les ardeurs de ta colère sont passées sur moi et tes terreurs m'anéantissent\FTNT{Es. 53:5.}.
\VS{18}Elles m'environnent tout le jour comme des eaux, elles m'enveloppent toutes à la fois.
\VS{19}Tu as éloigné de moi mon ami et mon compagnon, mes connaissances ont disparu\FTNT{Mt. 26:56.}.
\Chap{89}
\TextTitle{«Heureux le peuple qui connaît le son de la trompette »}
\VerseOne{}Cantique d'Ethan, l'Ezrachite.
\VS{2}Je chanterai toujours les bontés de Yahweh ; je ferai connaître de ma bouche ta fidélité de génération en génération.
\VS{3}Car je dis : Ta bonté a des fondements éternels, tu établis ta fidélité dans les cieux quand tu dis :
\VS{4}J'ai traité alliance avec mon élu, j'ai fait serment à David mon serviteur :
\VS{5}J'affermirai ta postérité pour toujours, et j'établirai ton trône de génération en génération\FTNT{2 S. 7:8-16.}. Sélah.
\VS{6}Les cieux célèbrent tes merveilles, ô Yahweh ! Ta fidélité aussi est célébrée dans l'assemblée des saints.
\VS{7}Car qui dans le ciel peut se comparer à Yahweh ? Qui est semblable à Yahweh parmi les fils de Dieu ?
\VS{8}Dieu se rend extrêmement terrible dans le conseil secret des saints, il est plus redouté que tous ceux qui sont à l'entour de lui.
\VS{9}Ô Yahweh Dieu des armées ! Qui est semblable à toi, puissant Yahweh ? Aussi ta fidélité t'environne.
\VS{10}Tu domines l'élévation des flots de la mer ; quand ses vagues s'élèvent, tu les calmes\FTNT{Job. 26:12 ; Job. 38:8-12.}.
\VS{11}Tu écrasas Rahab\FTNT{Ce terme hébreu fait référence au nom emblèmatique de l'Egypte, il signifie « largeur », « arrogance ».} comme un homme blessé à mort ; tu dispersas tes ennemis par le bras de ta force.
\VS{12}A toi sont les cieux, à toi aussi est la terre ; tu as fondé le monde, et tout ce qui est en lui.
\VS{13}Tu as créé le nord et le sud ; le Thabor et l'Hermon se réjouissent en ton Nom.
\VS{14}Ton bras est puissant, ta main est forte, ta droite est haut élevée.
\VS{15}La justice et l'équité sont la base de ton trône ; la bonté et la vérité marchent devant ta face.
\VS{16}Heureux le peuple qui connaît le son de la trompette\FTNT{1 Co. 15:52 ; Ap. 10:7.} ! Il marche, ô Yahweh ! A la clarté de ta face.
\VS{17}Il se réjouit chaque jour en ton Nom, et il se glorifie de ta justice.
\VS{18}Parce que tu es la gloire de leur force ; et notre pouvoir est distingué par ta faveur.
\VS{19}Car notre bouclier est Yahweh, et notre Roi est le Saint d'Israël.
\VS{20}Tu as autrefois parlé en vision touchant ton bien-aimé, et tu as dit : J'ai ordonné mon secours en faveur d'un homme vaillant ; j'ai élevé l'élu du milieu du peuple.
\VS{21}J'ai trouvé David mon serviteur, je l'ai oint de ma sainte huile\FTNT{1 S. 16:13 ; Ac. 13:22.} ;
\VS{22}ma main sera ferme avec lui, et mon bras le renforcera.
\VS{23}L'ennemi ne le surprendra point, et l'inique ne l'affligera point ;
\VS{24}mais j'écraserai devant lui ses adversaires, et je détruirai ceux qui le haïssent.
\VS{25}Ma fidélité et ma bonté seront avec lui, et sa gloire sera élevée en mon Nom.
\VS{26}Je mettrai sa main sur la mer, et sa droite sur les fleuves.
\VS{27}Il m'invoquera : Tu es mon Père, mon Dieu, et le rocher\FTNT{Voir commentaire en Es. 8:14.} de ma délivrance.
\VS{28}Aussi je ferai de lui le premier-né\FTNT{Col. 1:15.}, le plus élevé des rois de la terre.
\VS{29}Je lui garderai ma bonté à toujours, et mon alliance lui sera assurée.
\VS{30}Je rendrai éternelle sa postérité, et son trône comme les jours des cieux.
\VS{31}Mais si ses fils abandonnent ma loi, et ne marchent point selon mes ordonnances ;
\VS{32}s'ils violent mes statuts, et qu'ils ne gardent point mes commandements ;
\VS{33}je punirai de la verge leur transgression, et de plaie leur iniquité.
\VS{34}Mais je ne retirerai point de lui ma bonté, et je ne lui trahirai point ma fidélité.
\VS{35}Je ne violerai point mon alliance, et je ne changerai point ce qui est sorti de mes lèvres.
\VS{36}J'ai une fois juré par ma sainteté : Mentirais-je à David\FTNT{Hé. 6:13.} ?
\VS{37}Sa postérité sera à toujours, et son trône sera devant moi comme le soleil.
\VS{38}Il aura une durée éternelle comme la lune ; le témoin qui est dans le ciel est fidèle. Sélah.
\VS{39}Néanmoins, tu l'as rejeté et dédaigné\FTNT{Es. 53:3.} ; tu t'es mis en grande colère contre ton oint.
\VS{40}Tu as rejeté l'alliance faite avec ton serviteur ; tu as souillé sa couronne en la jetant par terre.
\VS{41}Tu as rompu toutes ses murailles ; tu as mis en ruines ses forteresses.
\VS{42}Tous ceux qui passaient par le chemin l'ont pillé ; il a été mis en opprobre à ses voisins.
\VS{43}Tu as élevé la droite de ses adversaires, tu as réjoui tous ses ennemis.
\VS{44}Tu as fait reculer le tranchant de son épée, et tu ne l'as point élevé dans le combat.
\VS{45}Tu as fait cesser sa splendeur, et tu as jeté par terre son trône.
\VS{46}Tu as abrégé les jours de sa jeunesse et l'as couvert de honte. Sélah.
\VS{47}Jusqu'à quand, ô Yahweh ? Te cacheras-tu à jamais ? Ta fureur s'embrasera-t-elle comme un feu ?
\VS{48}Souviens-toi quelle est la durée de ma vie ; pourquoi aurais-tu créé en vain tous les fils des hommes ?
\VS{49}Qui est l'homme qui vivra et ne verra point la mort, et qui garantira son âme de la main du scheol\FTNT{1 Co. 15:54-57.} ? Sélah.
\VS{50}Seigneur, où sont tes bontés premières que tu juras à David dans ta fidélité ?
\VS{51}Seigneur ! Souviens-toi de l'opprobre de tes serviteurs, et comment je porte dans mon sein l'opprobre qui nous a été fait par tous les peuples nombreux.
\VS{52}Souviens-toi des outrages de tes ennemis, ô Yahweh ! Des outrages contre les pas de ton oint.
\VS{53}Béni soit à toujours Yahweh ; amen ! Oui, amen !
\Chap{90}
\TextTitle{Mortalité de l'homme}
\VerseOne{}Prière de Moïse, homme de Dieu\FTNT{De. 33:1.}. Seigneur ! Tu as été pour nous un refuge de génération en génération.
\VS{2}Avant que les montagnes soient nées et que tu aies formé la terre et le monde, d'éternité en éternité, tu es Dieu\FTNT{Ge. 17:1 ; Es. 40:28.}.
\VS{3}Tu fais revenir l'homme à la poussière, et tu dis : Fils des hommes, retournez\FTNT{Ge. 3:19 ; Ec. 12:7.} !
\VS{4}Car mille ans sont à tes yeux comme le jour d'hier qui est passé, et comme une veille de la nuit\FTNT{Ps. 39:5 ; 2 Pi. 3:8.}.
\VS{5}Tu les emportes semblables à un songe qui, le matin, passe comme l'herbe :
\VS{6}Elle fleurit au matin et reverdit ; le soir on la coupe et elle se fane\FTNT{1 Pi. 1:24.}.
\VS{7}Car nous sommes consumés par ta colère et nous sommes troublés par ta fureur.
\VS{8}Tu as mis devant toi nos iniquités, et à la lumière de ta face nos fautes cachées.
\VS{9}Car tous nos jours s'en vont par ta grande colère, et nos années se consument dans un soupir.
\VS{10}Les jours de nos années reviennent à soixante-dix ans, et pour les plus forts, à quatre-vingts ans ; l'orgueil qu'ils en tirent n'est que peine et misère ; car il passe vite, et nous nous envolons.
\VS{11}Qui connaît, selon ta crainte, la force de ton indignation et de ta grande colère ?
\VS{12}Enseigne-nous à compter nos jours, afin que nous puissions avoir un cœur rempli de sagesse.
\VS{13}Yahweh ! Reviens ! Jusqu'à quand ? Sois apaisé envers tes serviteurs.
\VS{14}Rassasie-nous chaque matin de ta bonté, afin que nous nous réjouissions et que nous soyons joyeux tout le long de nos jours.
\VS{15}Réjouis-nous autant de jours que tu nous as affligés, autant d'années que nous avons vu le malheur.
\VS{16}Que ton œuvre se voie sur tes serviteurs, et ta gloire sur leurs fils.
\VS{17}Que la grâce de Yahweh, notre Dieu, soit sur nous, et affermis l'œuvre de nos mains ; oui, affermis l'œuvre de nos mains !
\Chap{91}
\TextTitle{La sécurité et la fidélité de Yahweh}
\VerseOne{}Celui qui demeure sous la couverture\FTNT{Jésus est notre couverture spirituelle.} du Très-Haut, repose à l'ombre du Tout-Puissant.
\VS{2}Je dis à Yahweh : Tu es ma retraite et ma forteresse, tu es mon Dieu en qui je me confie.
\VS{3}Certes, il te délivre du filet de l'oiseleur, de la peste et de ses ravages.
\VS{4}Il te couvrira de ses plumes et tu trouveras un refuge sous ses ailes ; sa fidélité est un bouclier et une cuirasse.
\VS{5}Tu ne craindras ni les terreurs de la nuit, ni la flèche qui vole le jour\FTNT{Pr. 3:23-24.},
\VS{6}ni la peste qui marche dans les ténèbres, ni la destruction qui frappe en plein midi.
\VS{7}Que mille tombent à ton côté et dix mille à ta droite, tu ne seras pas atteint.
\VS{8}De tes yeux tu regarderas et tu verras la rétribution des méchants.
\VS{9}Car tu es mon refuge, ô Yahweh ! Tu fais du Très-Haut ta demeure.
\VS{10}Aucun malheur ne s'approchera de toi, aucun fléau n'approchera de ta tente\FTNT{Ex. 8:18-19 ; Ps. 121:6-8.}.
\VS{11}Car il ordonnera à ses anges de te garder dans toutes tes voies.
\VS{12}Ils te porteront sur les mains, de peur que ton pied ne heurte contre une pierre\FTNT{Mt. 4:5-6 ; Lu. 4:9-11.}.
\VS{13}Tu marcheras sur le lion et sur l'aspic, tu piétineras le lionceau et le dragon.
\VS{14}Puisqu'il m'aime, je le délivrerai ; je le mettrai sur les hauteurs, parce qu'il connaît mon Nom.
\VS{15}Il m'invoquera et je l'exaucerai ; je serai avec lui dans la détresse, je le délivrerai et le glorifierai.
\VS{16}Je le rassasierai de jours et je lui ferai voir ma délivrance.
\Chap{92}
\TextTitle{Proclamer la louange de Dieu}
\VerseOne{}Psaume. Cantique pour le jour du sabbat.
\VS{2}C'est une belle chose que de célébrer Yahweh, et de chanter ton Nom, ô Très-Haut\FTNT{Ps. 147:1.} !
\VS{3}Afin d'annoncer chaque matin ta bonté et ta fidélité toutes les nuits\FTNT{Ps. 59:17 ; Ps. 88:14 ; Ps. 89:2.}.
\VS{4}Sur l'instrument à dix cordes, sur le luth, et par un cantique prémédité sur la harpe.
\VS{5}Car, ô Yahweh ! Tu me réjouis par tes œuvres, je me réjouis des œuvres de tes mains.
\VS{6}Ô Yahweh ! Que tes œuvres sont magnifiques ! Tes pensées sont merveilleusement profondes\FTNT{Es. 55:8-9 ; Job. 5:9.}.
\VS{7}L'homme stupide n'y connaît rien et le fou n'y prend point garde\FTNT{Es. 5:12 ; Ro. 1:21.}.
\VS{8}Les méchants croissent comme l'herbe, et tous les ouvriers d'iniquité fleurissent pour être exterminés éternellement\FTNT{Jé. 12:1-2 ; Mal. 3:15 ; Ps. 37:2 ; Ps. 73:1-20.}.
\VS{9}Mais toi, ô Yahweh ! Tu es élevé à toujours.
\VS{10}Car voici tes ennemis, ô Yahweh ! Car voici, tes ennemis périssent, tous les ouvriers d'iniquité sont dispersés.
\VS{11}Mais tu élèveras ma corne comme celle d'un buffle, je serai oint d'une huile fraîche\FTNT{Ps. 23:5 ; Hé. 1:9.}.
\VS{12}Mes yeux se plaisent à regarder ceux qui m'épient, et mes oreilles à entendre les méchants qui s'élèvent contre moi.
\VS{13}Le juste fleurit comme le palmier, il croît comme le cèdre au Liban.
\VS{14}Etant plantés dans la maison de Yahweh, ils fleurissent dans les parvis de notre Dieu.
\VS{15}Ils portent encore des fruits dans la blanche vieillesse ; ils sont gras et verdoyants\FTNT{Os. 14:6 ; Ps. 1:3.},
\VS{16}afin d'annoncer que Yahweh est droit ; c'est mon rocher, et il n'y a point d'injustice en lui.
\Chap{93}
\TextTitle{Majesté et puissance de Yahweh}
\VerseOne{}Yahweh règne, il est revêtu de majesté ; Yahweh est revêtu de force, il s'en est ceint ; aussi le monde est ferme, tellement qu'il ne sera point ébranlé.
\VS{2}Ton trône est établi dès lors, tu es de toute éternité\FTNT{Ps. 9:8 ; Hé. 1:8.}.
\VS{3}Les fleuves élevés, ô Yahweh ! Les fleuves augmentent leur bruit, les fleuves élèvent leurs flots\FTNT{Ps. 46:4 ; Ps. 65:7-8.} ;
\VS{4}Yahweh, qui est dans les lieux élevés, est plus puissant que le bruit des grandes eaux, et que les fortes vagues de la mer\FTNT{Es. 57:15 ; Ac. 7:49.}.
\VS{5}Tes préceptes sont entièrement fidèles. Yahweh ! La sainteté orne ta maison pour une longue durée.
\Chap{94}
\TextTitle{A Dieu seul la vengeance}
\VerseOne{}Ô Yahweh ! Dieu des vengeances, Dieu des vengeances, fais briller ta splendeur !
\VS{2}Toi, juge de la terre, élève-toi ! Rends aux orgueilleux selon leurs œuvres.
\VS{3}Jusqu'à quand les méchants, ô Yahweh ! Jusqu'à quand les méchants se réjouiront-ils ?
\VS{4}Jusqu'à quand tous les ouvriers d'iniquité discourront-ils et diront-ils des paroles rudes et se vanteront-ils ?
\VS{5}Yahweh, ils écrasent ton peuple et affligent ton héritage.
\VS{6}Ils tuent la veuve et l'étranger, et ils mettent à mort les orphelins.
\VS{7}Ils disent : Yahweh ne le voit point, le Dieu de Jacob n'entend rien.
\VS{8}Vous les plus abrutis d'entre les peuples, prenez garde à ceci ; et vous insensés, quand serez-vous intelligents ?
\VS{9}Celui qui a planté l'oreille, n'entendrait-il point ? Celui qui a formé l'œil, ne verrait-t-il point\FTNT{Ex. 4:11 ; Pr. 20:12.} ?
\VS{10}Celui qui châtie les nations, celui qui enseigne la science aux hommes, ne réprimanderait-il point\FTNT{Ap. 19:15.} ?
\VS{11}Yahweh connaît les pensées des hommes qui ne sont que vanité.
\VS{12}Heureux l'homme que tu châties, ô Yahweh\FTNT{Hé. 12:6.} ! Que tu instruis par ta loi,
\VS{13}afin qu'il soit dans la paix aux jours du malheur, jusqu'à ce que la fosse soit creusée pour le méchant !
\VS{14}Car Yahweh ne délaisse point son peuple et n'abandonne point son héritage\FTNT{Es. 49:15 ; Ro. 11:2.}.
\VS{15}C'est pourquoi le jugement s'unira à la justice, et tous ceux qui sont droits de cœur le suivront.
\VS{16}Qui se lèvera pour moi contre les méchants\FTNT{Job. 19:25 ; Ro. 8:31.} ? Qui m'assistera contre les ouvriers d'iniquité ?
\VS{17}Si Yahweh n'était pas mon secours, mon âme serait bien vite dans la demeure du silence.
\VS{18}Quand je dis : Mon pied chancelle, ta bonté me soutient, ô Yahweh !
\VS{19}Quand j'ai beaucoup de pensées au-dedans de moi, tes consolations font les délices de mon âme.
\VS{20}Serais–tu l'allié du trône de méchanceté, qui forge des injustices contre les règles de la justice ?
\VS{21}Ils se rassemblent contre l'âme du juste et condamnent le sang innocent\FTNT{Mt. 27:1-4 ; Mt. 27:24.}.
\VS{22}Or Yahweh est pour moi une haute retraite ; mon Dieu est le rocher de mon refuge.
\VS{23}Il fera retourner sur eux leur iniquité et les détruira par leur propre méchanceté. Yahweh notre Dieu les détruira\FTNT{Mt. 13:30 ; Ap. 20:14-15.}.
\Chap{95}
\TextTitle{Adoration à Yahweh}
\VerseOne{}Venez, chantons à Yahweh, poussons des cris de réjouissance au rocher de notre salut.
\VS{2}Allons au-devant de lui en lui présentant nos louanges ; et poussons devant lui des cris de réjouissance en chantant des psaumes.
\VS{3}Car Yahweh est un grand Dieu, et il est un grand Roi au-dessus de tous les dieux.
\VS{4}Les lieux les plus profonds de la terre sont dans sa main, et les sommets des montagnes sont à lui.
\VS{5}C'est à lui qu'appartient la mer, car lui-même l'a faite, et ses mains ont formé la terre.
\VS{6}Venez, prosternons-nous, inclinons-nous, et mettons-nous à genoux devant Yahweh qui nous a faits\FTNT{Ps. 96:9 ; Ph. 2:10-11.}.
\VS{7}Car il est notre Dieu, et nous sommes le peuple de son pâturage, et les brebis que sa main conduit\FTNT{Ps. 23:1 ; Ps. 100:3 ; Jn. 10:11.}. Si vous entendez aujourd'hui sa voix,
\VS{8}n'endurcissez point votre cœur\FTNT{Hé. 3:8 ; Hé. 4:7.}, comme à Meriba, comme à la journée de Massa, au désert ;
\VS{9}là où vos pères m'ont tenté et éprouvé bien qu'ils virent mes œuvres\FTNT{Ex. 17:7.}.
\VS{10}J'ai eu cette génération en dégoût durant quarante ans, et j'ai dit : C'est un peuple dont le cœur s'égare ; et ils n'ont point connu mes voies ;
\VS{11}c'est pourquoi j'ai juré dans ma colère, ils n'entreront pas dans mon repos\FTNT{No. 14:22-23 ; Hé. 3:15-19 ; Hé. 4:3.}.
\Chap{96}
\TextTitle{La grandeur et la gloire de Dieu}
\VerseOne{}Chantez à Yahweh un cantique nouveau\FTNT{Es. 42:10 ; Ps. 98:1 ; Ap. 5:9 ; Ap. 14:3.} ! Vous tous habitants de la terre chantez à Yahweh !
\VS{2}Chantez à Yahweh, bénissez son Nom ! Prêchez de jour en jour sa délivrance !
\VS{3}Racontez sa gloire parmi les nations, ses merveilles parmi tous les peuples\FTNT{Ps. 67:5.}.
\VS{4}Car Yahweh est grand et digne d'être loué ; il est redoutable au-dessus de tous les dieux\FTNT{Ph. 2:9 ; Ap. 5:9.} ;
\VS{5}car tous les dieux des peuples ne sont que des idoles, mais Yahweh a fait les cieux.
\VS{6}La splendeur et la magnificence marchent devant lui, la force et la beauté sont dans son lieu saint.
\VS{7}Familles des peuples, rendez à Yahweh, rendez à Yahweh la gloire et la puissance !
\VS{8}Rendez à Yahweh la gloire due à son Nom ! Apportez des offrandes et entrez dans ses parvis !
\VS{9}Prosternez–vous devant Yahweh avec des ornements sacrés ; tremblez devant lui, vous toute la terre !
\VS{10}Dites parmi les nations : Yahweh règne ; même le monde est affermi, il ne sera point ébranlé ; il jugera les peuples avec équité.
\VS{11}Que les cieux se réjouissent et que la terre soit dans l'allégresse ! Que la mer tonne avec tout ce qui la remplit !
\VS{12}Que les champs s'égayent avec tout ce qui est en eux. Alors tous les arbres de la forêt chanteront de joie
\VS{13}devant Yahweh, car il vient ! Car il vient pour juger la terre ; il jugera avec justice le monde habitable et les peuples selon sa fidélité.
\Chap{97}
\TextTitle{Aimer Dieu, c'est haïr le mal}
\VerseOne{}Yahweh règne, que la terre soit dans l'allégresse et que les îles nombreuses s'en réjouissent\FTNT{Es. 42:10 ; Ps. 86:9 ; Ps. 93:1 ; Ps. 99:1} !
\VS{2}La nuée et l'obscurité sont autour de lui ; la justice et le jugement sont la base de son trône.
\VS{3}Le feu marche devant lui et embrase tout autour ses adversaires.
\VS{4}Ses éclairs éclairent le monde, et la terre le voit et tremble tout étonnée\FTNT{Job. 38:35 ; Ap. 4:5.}.
\VS{5}Les montagnes se fondent comme de la cire\FTNT{Mi. 1:4.}, à cause de la présence de Yahweh, à cause de la présence du Seigneur de toute la terre.
\VS{6}Les cieux annoncent sa justice et tous les peuples voient sa gloire.
\VS{7}Que tous ceux qui servent les images et qui se glorifient des idoles soient confus\FTNT{De. 4:25-26 ; 1 S. 5:1-5.} ; vous dieux, prosternez-vous tous devant lui.
\VS{8}Sion l'a entendu et s'en est réjouie ; et les filles de Juda se sont égayées pour l'amour de tes jugements, ô Yahweh !
\VS{9}Yahweh, tu es le Très-Haut sur toute la terre ; tu es élevé au-dessus de tous les dieux.
\VS{10}Vous qui aimez Yahweh, haïssez le mal\FTNT{Am. 5:14-15 ; Ro. 12:9.} ! Il garde les âmes de ses bien-aimés et les délivre de la main des méchants\FTNT{Ps. 34:8 ; Jn. 10:28-29.}.
\VS{11}La lumière est faite pour le juste\FTNT{Mt. 5:15-16.} et la joie pour ceux qui sont droits de cœur.
\VS{12}Justes, réjouissez-vous en Yahweh et célébrez la mémoire de sa sainteté.
\Chap{98}
\TextTitle{Invitation à la louange}
\VerseOne{}Psaume. Chantez à Yahweh un cantique nouveau, car il a fait des choses merveilleuses ; sa droite et le bras de sa sainteté l'ont délivré\FTNT{Es. 52:10 ; Es 53:1 ; Es. 63:3-5.}.
\VS{2}Yahweh a fait connaître son salut\FTNT{Il est question de la révélation de Jésus. Voir commentaire en Es. 26:1.}, il a révélé sa justice devant les yeux des nations.
\VS{3}Il s'est souvenu de sa bonté et de sa fidélité envers la maison d'Israël ; toutes les extrémités de la terre ont vu la délivrance de notre Dieu\FTNT{Es. 49:6 ; Lu. 1:72 ; Ac. 13:47.}.
\VS{4}Vous tous habitants de la terre, poussez des cris de réjouissance à Yahweh ! Faites retentir vos cris et chantez de joie !
\VS{5}Chantez à Yahweh avec la harpe, avec la harpe et avec une voix mélodieuse !
\VS{6}Poussez des cris de réjouissance avec le shofar au son du cor devant le Roi, Yahweh !
\VS{7}Que la mer tonne avec tout ce qu'elle contient, que la terre et ceux qui y habitent fassent éclater leurs cris !
\VS{8}Que les fleuves frappent des mains et que les montagnes chantent de joie
\VS{9}devant Yahweh ! Car il vient pour juger la terre\FTNT{Yahweh, qui vient pour juger la terre, est Jésus-Christ ( 
Za. 14:1-7 ; 2 Ti. 4:1 ; Ap. 19:15).} ; il jugera le monde habitable avec justice et les peuples avec équité.
\Chap{99}
\TextTitle{Grandeur, justice, et sainteté de Dieu}
\VerseOne{}Yahweh règne, que les peuples tremblent ; il est assis entre les chérubins, que la terre soit ébranlée\FTNT{Ex. 25:22 ; Es. 37:16.}.
\VS{2}Yahweh est grand en Sion, et il est élevé au-dessus de tous les peuples.
\VS{3}Ils célébreront ton Nom, grand et terrible, car il est saint.
\VS{4}Qu'on célèbre la force du roi qui aime la justice ! Tu as ordonné l'équité, tu as prononcé des jugements justes en Jacob.
\VS{5}Exaltez Yahweh notre Dieu et prosternez-vous devant son marchepied ! Il est saint !
\VS{6}Moïse et Aaron étaient parmi ses sacrificateurs\FTNT{Ex. 31:10 ; Lé. 2:2.} ; et Samuel parmi ceux qui invoquaient son Nom ; ils invoquaient Yahweh et il leur répondait\FTNT{1 S. 12:18-19.}.
\VS{7}Il leur parlait de la colonne de nuée ; ils ont gardé ses préceptes et l'ordonnance qu'il leur avait donnée.
\VS{8}Ô Yahweh, mon Dieu ! Tu les as exaucés, tu as été pour eux un Dieu qui pardonne\FTNT{Hé. 10:16-17.}, mais tu les as punis de leurs fautes.
\VS{9}Exaltez Yahweh notre Dieu ! Prosternez-vous sur la montagne de sa sainteté ! Car Yahweh, notre Dieu est saint !
\Chap{100}
\TextTitle{Célébrer et bénir le nom de Yahweh}
\VerseOne{}Psaume de louange. Vous tous habitants de la terre, poussez des cris de réjouissance à Yahweh !
\VS{2}Servez Yahweh avec allégresse, venez devant lui avec un chant de joie !
\VS{3}Sachez que Yahweh est Dieu. C'est lui qui nous a faits, ce n'est pas nous qui nous sommes faits ; nous sommes son peuple et le troupeau de son pâturage\FTNT{Ps. 79:13 ; Ps. 80:2 ; Ps. 95:6 ; Ps. 119:73.}.
\VS{4}Entrez dans ses portes avec des louanges ; et dans ses parvis, avec des cantiques. Célébrez-le, bénissez son Nom !
\VS{5}Car Yahweh est bon ; sa bonté demeure à toujours et sa fidélité de génération en génération.
\Chap{101}
\TextTitle{Appel à l'intégrité}
\VerseOne{}Psaume de David. Je chanterai la miséricorde et la justice ; Yahweh ! Je te chanterai.
\VS{2}Je me rendrai attentif à une conduite pure jusqu'à ce que tu viennes à moi ; je marcherai dans l'intégrité de mon cœur au milieu de ma maison.
\VS{3}Je ne mettrai point devant mes yeux des choses de Bélial\FTNT{Ps. 26:5 ; Ps. 119:115.}; j'ai en haine les actions de ceux qui se détournent ; elles ne s'attacheront pas à moi.
\VS{4}Le cœur mauvais s'éloignera de moi ; je ne connaîtrai pas le méchant.
\VS{5}Je retrancherai celui qui calomnie en secret son prochain ; je ne supporterai pas celui qui a les yeux élevés et le cœur enflé\FTNT{Pr. 6:16-17.}.
\VS{6}Je prendrai garde aux gens de bien du pays afin qu'ils demeurent avec moi ; celui qui marche dans la voie de l'intégrité me servira.
\VS{7}Celui qui usera de tromperie ne demeurera point dans ma maison ; celui qui profèrera des mensonges ne sera point affermi devant mes yeux.
\VS{8}Je retrancherai chaque matin tous les méchants du pays, afin d'exterminer de la cité de Yahweh tous les ouvriers d'iniquité.
\Chap{102}
\TextTitle{Yahweh, le Dieu immuable}
\VerseOne{}Prière de l'affligé étant dans l'angoisse et répandant sa plainte devant Yahweh.
\VS{2}Yahweh ! Ecoute ma prière, et que mon cri parvienne jusqu'à toi\FTNT{Ps. 69:14.}.
\VS{3}Ne cache pas ta face arrière de moi ; au jour où je suis en détresse, prête l'oreille à ma prière ; au jour où je t'invoque, hâte-toi de me répondre.
\VS{4}Car mes jours se sont évanouis comme la fumée et mes os brûlent comme dans un foyer.
\VS{5}Mon cœur est frappé et se dessèche comme l'herbe, car j'ai oublié de manger mon pain\FTNT{Mt. 4:4 ; Lu. 4:4.}.
\VS{6}Le gémissement de ma voix est tel que mes os s'attachent à ma chair\FTNT{Job. 19:20.}.
\VS{7}Je suis devenu semblable au pélican du désert ; et je suis comme la chouette des lieux sauvages.
\VS{8}Je veille et je suis semblable au passereau solitaire sur le toit.
\VS{9}Mes ennemis m'outragent tous les jours, et ceux qui sont furieux contre moi jurent contre moi.
\VS{10}Car j'ai mangé la cendre comme le pain et j'ai mêlé des larmes à ma boisson,
\VS{11}à cause de ta colère et de ta fureur ; car après m'avoir soulevé, tu m'as jeté par terre.
\VS{12}Mes jours sont comme l'ombre qui décline et je deviens sec comme l'herbe.
\VS{13}Mais toi, ô Yahweh ! Tu demeures éternellement, et ta mémoire est de génération en génération.
\VS{14}Tu te lèveras, tu auras compassion de Sion ; car il est temps d'en avoir pitié, parce que le temps assigné est échu.
\VS{15}Car tes serviteurs aiment ses pierres et chérissent sa poussière.
\VS{16}Alors les nations redouteront le Nom de Yahweh, et tous les rois de la terre ta gloire.
\VS{17}Quand Yahweh aura édifié Sion, quand il aura été vu dans sa gloire,
\VS{18}quand il aura eu égard à la prière du désolé et qu'il n'aura point méprisé leur supplication.
\VS{19}Cela sera enregistré pour la génération à venir, le peuple qui sera créé louera Yahweh !
\VS{20}Car il regarde du lieu élevé de sa sainteté. Du haut des cieux, Yahweh regarde la terre,
\VS{21}pour entendre le gémissement des prisonniers, pour délier ceux qui étaient voués à la mort\FTNT{Es. 42:6-7 ; Es. 61:1 ; Lu. 4:18-19.},
\VS{22}afin qu'on annonce le Nom de Yahweh dans Sion et sa louange dans Jérusalem,
\VS{23}quand les peuples se seront joints ensemble, et les royaumes aussi, pour servir Yahweh.
\VS{24}Il a brisé ma force en chemin, il a abrégé mes jours.
\VS{25}Je dis : Mon Dieu, ne m'enlève point au milieu de mes jours dont les années durent éternellement.
\VS{26}Tu as jadis fondé la terre, et les cieux sont l'ouvrage de tes mains.
\VS{27}Ils périront, mais tu subsisteras, ils s'useront tous comme un vêtement ; tu les changeras comme un habit, et ils seront changés.
\VS{28}Mais toi, tu es toujours le même et tes années ne seront jamais achevées.
\VS{29}Les fils de tes serviteurs habiteront près de toi et leur postérité sera établie devant toi.
\Chap{103}
\TextTitle{Yahweh, le Dieu miséricordieux et compatissant}
\VerseOne{}Psaume de David. Mon âme, bénis Yahweh, et que tout ce qui est en moi bénisse son saint Nom.
\VS{2}Mon âme, bénis Yahweh, et n'oublie pas un de ses bienfaits\FTNT{De. 6:12.}.
\VS{3}C'est lui qui pardonne toutes tes iniquités, qui guérit toutes tes infirmités\FTNT{Es. 33:24 ; Es. 53:5 ; Jé. 17:14 ; Ps. 130:3-4 ; Mt. 9:6 ; Lu. 7:47.} ;
\VS{4}qui garantit ta vie de la fosse\FTNT{Es. 59:20 ; Ps. 106:10.}, qui te couronne de bonté et de compassions ;
\VS{5}qui rassasie ta bouche de biens ; ta jeunesse est renouvelée comme celle de l'aigle\FTNT{Es. 40:31.}.
\VS{6}Yahweh fait justice et droit à tous les opprimés\FTNT{Ps. 146:7.}.
\VS{7}Il a fait connaître ses voies à Moïse, et ses exploits aux fils d'Israël\FTNT{Ex. 33:12-17.}.
\VS{8}Yahweh est compatissant, miséricordieux, lent à la colère, et riche en bonté.
\VS{9}Il ne conteste pas éternellement, et il ne garde point à toujours sa colère\FTNT{Es. 57:16 ; Jé. 3:5 ; Mi. 7:18.}.
\VS{10}Il ne nous traite pas selon nos péchés, et ne nous rend point selon nos iniquités\FTNT{Esd. 9:13.}.
\VS{11}Car autant les cieux sont élevés au-dessus de la terre, autant sa bonté est grande sur ceux qui le craignent.
\VS{12}Il éloigne de nous nos transgressions, autant que l'orient est éloigné de l'occident\FTNT{Es. 38:17.}.
\VS{13}Comme un père a compassion de ses fils, Yahweh a compassion de ceux qui le craignent\FTNT{Mal. 3:17 ; Lu. 11:11-13.}.
\VS{14}Car il sait bien de quoi nous sommes faits, se souvenant que nous ne sommes que poussière.
\VS{15}L'homme ! Ses jours sont comme l'herbe, il fleurit comme la fleur d'un champ.
\VS{16}Car le vent étant passé par-dessus, elle n'est plus, et son lieu ne la reconnaît plus.
\VS{17}Mais la miséricorde de Yahweh est de tout temps, et elle sera pour toujours en faveur de ceux qui le craignent ; et sa justice en faveur des fils de leurs fils ;
\VS{18}pour ceux qui gardent son alliance, et qui se souviennent de ses commandements pour les faire\FTNT{De. 7:9.}.
\VS{19}Yahweh a établi son trône dans les cieux, et son règne domine sur tout.
\VS{20}Bénissez Yahweh, vous ses anges puissants en force, qui faites ses affaires, en obéissant à la voix de sa parole.
\VS{21}Bénissez Yahweh, vous toutes ses armées, qui êtes ses serviteurs faisant sa volonté.
\VS{22}Bénissez Yahweh, vous toutes ses œuvres, par tous les lieux de sa domination. Mon âme, bénis Yahweh !
\Chap{104}
\TextTitle{Yahweh, le Dieu de toute la création}
\VerseOne{}Mon âme, bénis Yahweh. Ô Yahweh mon Dieu, tu es merveilleusement grand, tu es revêtu de majesté et de splendeur.
\VS{2}Il s'enveloppe de lumière comme d'un vêtement, il étend les cieux comme un voile\FTNT{Es. 40:22 ; Job. 9:8 ; 1 Ti. 6:16.}.
\VS{3}Avec les eaux, il va à la rencontre de sa demeure ; il fait des grosses nuées son char, il se promène sur les ailes du vent\FTNT{Es. 19:1 ; Ps. 18:10 ; Ap. 14:14.}.
\VS{4}Il fait des vents ses messagers, et des flammes de feu ses serviteurs\FTNT{Ps. 148:8 ; Hé. 1:7 ; Jn. 3:8.}.
\VS{5}Il a fondé la terre sur ses bases, elle ne sera jamais ébranlée\FTNT{Ps. 24:1-2 ; Ps. 78:69 ; Ps. 93:1 ; Job. 26:7 ; Job. 38:4-6.}.
\VS{6}Tu l'avais couverte de l'abîme comme d'un vêtement, les eaux se tenaient sur les montagnes\FTNT{Ge. 1:2.}.
\VS{7}Elles s'enfuirent à ta menace et se mirent promptement en fuite au son de ton tonnerre.
\VS{8}Les montagnes s'élevèrent et les vallées s'abaissèrent au même lieu que tu leur avais fixé.
\VS{9}Tu as posé une limite que les eaux ne doivent point franchir, afin qu'elles ne reviennent plus couvrir la terre\FTNT{Ge. 1:9 ; Jé. 5:2 ; Pr. 8:29 ; Job. 26:10.}.
\VS{10}C'est lui qui conduit les sources par les vallées, elles se promènent entre les monts.
\VS{11}Elles abreuvent toutes les bêtes des champs, les ânes sauvages y étanchent leur soif.
\VS{12}Les oiseaux des cieux se tiennent auprès d'elles, et font résonner leur voix parmi les rameaux.
\VS{13}Il abreuve les montagnes de ses chambres hautes ; la terre est rassasiée du fruit de tes œuvres.
\VS{14}Il fait germer l'herbe pour le bétail, et les plantes pour le besoin de l'homme, faisant sortir le pain de la terre,
\VS{15}et le vin qui réjouit le cœur de l'homme\FTNT{Jg. 9:11 ; Pr. 31:6-7.}, qui fait resplendir son visage avec l'huile, et qui soutient le cœur de l'homme avec le pain.
\VS{16}Les hauts arbres de Yahweh en sont rassasiés, ainsi que les cèdres du Liban qu'il a plantés,
\VS{17}afin que les oiseaux y fassent leurs nids. Quant à la cigogne, les sapins sont sa demeure.
\VS{18}Les hautes montagnes sont pour les chamois, et les rochers sont la retraite des lapins.
\VS{19}Il a fait la lune pour les saisons, et le soleil sait quand il doit se coucher\FTNT{Ge. 1:16.}.
\VS{20}Tu amènes les ténèbres, et il fait nuit ; alors toutes les bêtes de la forêt sont en mouvement.
\VS{21}Les lionceaux rugissent après la proie pour demander à Dieu leur nourriture.
\VS{22}Le soleil se lève-t-il ? Ils se retirent et se couchent dans leurs tanières.
\VS{23}Alors l'homme sort pour se rendre à son ouvrage, et à son travail jusqu'au soir.
\VS{24}Ô Yahweh, que tes œuvres sont en grand nombre ! Tu les as toutes faites avec sagesse ; la terre est pleine de tes richesses.
\VS{25}Cette mer grande et spacieuse, là où des animaux sans nombre se meuvent, des petites bêtes avec des grandes !
\VS{26}Là se promènent les navires, et ce léviathan que tu as formé pour jouer dans les flots.
\VS{27}Ils s'attendent tous à toi afin que tu leur donnes la nourriture en leur temps.
\VS{28}Quand tu la leur donnes, ils la recueillent, et quand tu ouvres ta main, ils sont rassasiés de biens.
\VS{29}Caches-tu ta face ? Ils sont troublés ; retires-tu leur souffle ? Ils défaillent et retournent dans leur poussière.
\VS{30}Tu envoies ton souffle, ils sont créés ; et tu renouvelles la face de la terre.
\VS{31}Que la gloire de Yahweh subsiste à toujours, que Yahweh se réjouisse dans ses œuvres !
\VS{32}Il jette son regard sur la terre et elle tremble ; il touche les montagnes et elles fument.
\VS{33}Je chanterai à Yahweh durant ma vie ; je chanterai à mon Dieu tant que j'existerai.
\VS{34}Ma méditation lui sera agréable, et je me réjouirai en Yahweh.
\VS{35}Que les pécheurs soient consumés de dessus la terre et qu'il n'y ait plus de méchants ! Mon âme, bénis Yahweh ! Louez Yahweh !
\Chap{105}
\TextTitle{Yahweh, le Dieu fidèle}
\VerseOne{}Célébrez Yahweh, invoquez son Nom, faites connaître parmi les peuples ses exploits.
\VS{2}Chantez-le, chantez-le, parlez de toutes ses merveilles !
\VS{3}Glorifiez-vous de son saint Nom, et que le cœur de ceux qui cherchent Yahweh se réjouisse.
\VS{4}Recherchez Yahweh et sa puissance ; cherchez continuellement sa face.
\VS{5}Souvenez-vous de ses merveilles qu'il a faites, de ses miracles, et des jugements de sa bouche.
\VS{6}La postérité d'Abraham sont ses serviteurs ; les enfants de Jacob sont ses élus !
\VS{7}Il est Yahweh notre Dieu, ses jugements sont sur toute la terre.
\VS{8}Il s'est souvenu pour toujours de son alliance, de la parole qu'il a ordonnée pour mille générations,
\VS{9}du traité qu'il a fait avec Abraham et du serment qu'il a fait à Isaac\FTNT{Ge. 17:2 ; Ge. 22:16 ; Ge. 26:3 ; Ge. 28:13 ; Ge. 33:11 ; Lu. 1:73.}.
\VS{10}Il l'a érigé pour être une ordonnance à Jacob, et à Israël pour être une alliance éternelle,
\VS{11}en disant : Je te donnerai le pays de Canaan, comme héritage qui vous est échu\FTNT{Ge. 13:15 ; Ge. 15:18.}.
\VS{12}Ils étaient alors un petit nombre de gens, très peu nombreux, et étrangers dans le pays.
\VS{13}Car ils allaient de nation en nation, et d'un royaume vers un autre peuple.
\VS{14}Il ne permit à personne de les opprimer, il châtia des rois à cause d'eux\FTNT{Ge. 35:5.},
\VS{15}disant : Ne touchez point à mes oints et ne faites point de mal à mes prophètes\FTNT{1 Ch. 16:22.} !
\VS{16}Il appela aussi la famine sur la terre, rompit le bâton du pain\FTNT{Lé. 26:26 ; Es. 3:1 ; Ez. 4:16.}.
\VS{17}Il envoya un homme devant eux ; Joseph fut vendu pour esclave\FTNT{Ge. 37:28-36.}.
\VS{18}On serra ses pieds dans des ceps, sa personne fut mise aux fers.
\VS{19}Jusqu'au temps où arriva ce qu'il avait annoncé, et où la parole de Yahweh l'éprouva.
\VS{20}Le roi le relâcha et le laissa aller ; le dominateur des peuples le délivra.
\VS{21}Il l'établit pour maître sur sa maison, et pour gouverneur sur tout son domaine\FTNT{Ge. 41:40.} ;
\VS{22}pour soumettre les princes à ses désirs, et pour instruire ses anciens.
\VS{23}Puis Israël entra en Egypte, et Jacob séjourna dans le pays de Cham\FTNT{Ge. 46:6 ; Ps. 78:51.}.
\VS{24}Yahweh rendit son peuple très fécond et le rendit plus puissant que ceux qui l'opprimaient.
\VS{25}Il changea leur cœur, au point qu'ils haïrent son peuple jusqu'à conspirer contre ses serviteurs\FTNT{Ex. 1:7-12.}.
\VS{26}Il envoya Moïse son serviteur, et Aaron, qu'il avait élu\FTNT{Ex. 4:14.}.
\VS{27}Ils accomplirent au milieu d'eux des prodiges et des miracles qu'ils avaient eu la charge de faire dans le pays de Cham.
\VS{28}Il envoya les ténèbres et fit venir l'obscurité ; et ils ne furent point rebelles à sa parole.
\VS{29}Il changea leurs eaux en sang et fit mourir leurs poissons.
\VS{30}Leur terre produisit en abondance des grenouilles jusque dans les chambres de leurs rois.
\VS{31}Il dit, et des mouches vinrent, des poux sur tout leur pays.
\VS{32}Il leur donna pour pluie de la grêle, et un feu flamboyant sur la terre.
\VS{33}Il frappa leurs vignes et leurs figuiers, et il brisa les arbres du pays.
\VS{34}Il ordonna et les sauterelles vinrent, des jeunes sauterelles sans nombre
\VS{35}qui dévorèrent toute l'herbe du pays, et qui dévorèrent le fruit de leur terroir.
\VS{36}Il frappa tous les premiers-nés du pays, les prémices de toute leur vigueur\FTNT{Lire Ex. 7 à 12.}.
\VS{37}Puis il les fit sortir avec de l'or et de l'argent, et nul ne chancela parmi ses tribus.
\VS{38}Les Egyptiens se réjouirent à leur départ, car la peur qu'ils avaient d'eux les avait saisis.
\VS{39}Il étendit la nuée pour couverture, et le feu pour éclairer la nuit.
\VS{40}Le peuple demanda et il fit venir des cailles ; et il les rassasia du pain des cieux\FTNT{Ex. 16:12-13.}.
\VS{41}Il ouvrit le rocher et les eaux en coulèrent ; elles se répandirent comme un fleuve dans les lieux arides\FTNT{Ex. 17:6.}.
\VS{42}Car il se souvint de sa parole sainte qu'il avait donnée à Abraham son serviteur\FTNT{Ge. 15:13-16.}.
\VS{43}Il fit sortir son peuple dans l'allégresse, ses élus au milieu des cris retentissants\FTNT{Ex. 15:1.}.
\VS{44}Il leur donna les terres des nations et ils possédèrent le fruit du travail des peuples,
\VS{45}afin qu'ils gardent ses statuts et qu'ils observent ses lois. Louez Yahweh !
\Chap{106}
\TextTitle{L'infidélité d'Israël}
\VerseOne{}Louez Yahweh ! Célébrez Yahweh car il est bon, car sa bonté demeure à toujours !
\VS{2}Qui pourrait réciter les exploits de Yahweh ? Qui pourrait faire retentir toute sa louange ?
\VS{3}Heureux ceux qui observent la justice, qui font en tout temps ce qui est juste !
\VS{4}Yahweh, souviens-toi de moi selon la bienveillance que tu portes à ton peuple, aie soin de moi selon ta délivrance !
\VS{5}Afin que je voie le bien de tes élus, que je me réjouisse dans la joie de ta nation, que je me glorifie avec ton héritage.
\VS{6}Nous avons péché avec nos pères, nous avons agi dans l'iniquité, nous avons fait le mal\FTNT{Da. 9:16 ; Esd. 9:7 ; Né. 1:6.}.
\VS{7}Nos pères n'ont point été attentifs à tes merveilles en Egypte ; ils ne se sont point souvenus de la multitude de tes faveurs ; mais ils furent rebelles près de la mer, vers la Mer Rouge\FTNT{Ex. 14:11.}.
\VS{8}Toutefois, il les délivra pour l'amour de son Nom, afin de faire connaître sa puissance.
\VS{9}Il menaça la Mer Rouge et elle se sécha ; et il les conduisit à travers les profondeurs de la mer comme un désert ;
\VS{10}il les délivra de la main de ceux qui les haïssaient et les racheta de la main de l'ennemi.
\VS{11}Les eaux couvrirent leurs oppresseurs, il n'en resta pas un seul\FTNT{Ex. 14:27.}.
\VS{12}Alors ils crurent à ses paroles et ils chantèrent sa louange.
\VS{13}Mais ils oublièrent vite ses œuvres et ne s'attendirent point à son conseil.
\VS{14}Ils furent épris de convoitise au désert et ils tentèrent Dieu dans le désert.
\VS{15}Alors il leur donna ce qu'ils avaient demandé, toutefois il leur envoya le dépérissement dans leur corps.
\VS{16}Ils jalousèrent dans le camp Moïse et Aaron, le saint de Yahweh.
\VS{17}La terre s'ouvrit et engloutit Dathan ; et recouvrit de terre Abiram\FTNT{No. 16.}.
\VS{18}Le feu s'alluma au milieu de leur assemblée, la flamme brûla les méchants.
\VS{19}Ils firent un veau en Horeb, et se prosternèrent devant une image de métal fondu\FTNT{Ex. 32.}.
\VS{20}Ils changèrent leur gloire contre la figure d'un bœuf qui mange l'herbe.
\VS{21}Ils oublièrent Dieu, leur libérateur, qui avait fait de grandes choses en Egypte,
\VS{22}des choses merveilleuses dans le pays de Cham, et des prodiges sur la Mer Rouge.
\VS{23}C'est pourquoi il dit qu'il les détruirait ; mais Moïse, son élu, se tint à la brèche devant lui pour détourner sa fureur, afin qu'il ne les détruisît point\FTNT{Ex. 32:11.}.
\VS{24}Ils méprisèrent le pays désirable et ne crurent point à sa parole.
\VS{25}Ils murmurèrent dans leurs tentes et n'obéirent point à la voix de Yahweh.
\VS{26}C'est pourquoi il leur jura la main levée de les faire tomber dans le désert,
\VS{27}d'accabler leur postérité parmi les nations, et de les disperser au milieu des pays\FTNT{No. 14:22.}.
\VS{28}Ils se joignirent aux adorateurs de Baal-Peor et mangèrent des victimes sacrifiées aux morts.
\VS{29}Ils irritèrent Dieu par leurs actions, au point qu'une plaie fit une brèche parmi eux.
\VS{30}Mais Phinées se présenta et fit justice ; et la plaie fut arrêtée.
\VS{31}Cela lui fut imputé à justice de génération en génération, pour toujours\FTNT{No. 25:3-8.}.
\VS{32}Ils excitèrent aussi sa colère près des eaux de Meriba, et Moïse fut puni à cause d'eux.
\VS{33}Car ils aigrirent son esprit et il parla avec légèreté de ses lèvres\FTNT{No. 20:12.}.
\VS{34}Ils ne détruisirent point les peuples que Yahweh leur avait dit de détruire,
\VS{35}mais ils se mêlèrent parmi ces nations et apprirent leurs manières de faire.
\VS{36}Ils servirent leurs faux dieux qui furent un piège pour eux.
\VS{37}Car ils sacrifièrent leurs fils et leurs filles aux démons\FTNT{Lé. 18:21 ; De. 12:31 ; 2 R. 16:3 ; Ez. 20:26.}.
\VS{38}Ils répandirent le sang innocent, le sang de leurs fils et de leurs filles, ils sacrifièrent aux faux dieux de Canaan ; et le pays fut souillé de sang\FTNT{No. 35:33.}.
\VS{39}Ils se souillèrent par leurs œuvres et se prostituèrent par leurs actions.
\VS{40}C'est pourquoi la colère de Yahweh s'embrasa contre son peuple et il eut en abomination son héritage.
\VS{41}Il les livra entre les mains des nations, et ceux qui les haïssaient dominèrent sur eux.
\VS{42}Leurs ennemis les opprimèrent et ils furent humiliés sous leur main.
\VS{43}Il les délivra souvent, mais ils se montrèrent rebelles dans leurs desseins et furent humiliés par leur iniquité.
\VS{44}Toutefois, il vit leur détresse lorsqu'il entendit leurs supplications.
\VS{45}Il se souvint en leur faveur de son alliance et se repentit selon la grandeur de ses compassions.
\VS{46}Il fit que ceux qui les avaient emmenés captifs eurent pitié d'eux.
\VS{47}Yahweh notre Dieu, délivre-nous et rassemble-nous du milieu des nations ! Afin que nous célébrions ton saint Nom et que nous mettions notre gloire à te louer !
\VS{48}Béni soit Yahweh, le Dieu d'Israël, d'éternité en éternité ! Et que tout le peuple dise amen ! Louez Yahweh.
\Chap{107}
\TextTitle{La grâce de Yahweh pour ses rachetés}
\VerseOne{}Célébrez Yahweh car il est bon, parce que sa bonté demeure à toujours.
\VS{2}Qu'ainsi disent les rachetés de Yahweh, ceux qu'il a rachetés de la main de l'oppresseur,
\VS{3}et qu'il a rassemblés de tous les pays, de l'orient et de l'occident, du nord et de la mer.
\VS{4}Ils erraient dans le désert, ils marchaient dans la solitude, sans trouver une ville où ils puissent habiter.
\VS{5}Ils étaient affamés et assoiffés, leur âme était languissante.
\VS{6}Alors ils crièrent vers Yahweh dans leur détresse et il les délivra de leurs angoisses ;
\VS{7}il les conduisit sur le droit chemin pour aller dans une ville habitée.
\VS{8}Qu'ils célèbrent Yahweh pour sa bonté et ses merveilles envers les fils des hommes !
\VS{9}Parce qu'il a désaltéré l'âme altérée et rassasié de ses biens l'âme affamée\FTNT{Ps. 146:7 ; Lu. 1:53.}.
\VS{10}Ceux qui avaient pour demeure les ténèbres et l'ombre de la mort, vivaient captifs dans l'affliction et dans les chaînes,
\VS{11}parce qu'ils furent rebelles aux paroles de Dieu, et parce qu'ils avaient rejeté le conseil du Très-Haut\FTNT{De. 31:20 ; La. 3:42.}.
\VS{12}Il humilia leur cœur par la souffrance, ils furent abattus ; et personne ne les secourut.
\VS{13}Alors ils crièrent vers Yahweh dans leur détresse, et il les délivra de leurs angoisses.
\VS{14}Il les fit sortir hors des ténèbres et de l'ombre de la mort ; et il rompit leurs liens\FTNT{Ps. 68:19 ; Ep. 4:8 ; Col. 1:12-13.}.
\VS{15}Qu'ils célèbrent Yahweh pour sa bonté et ses merveilles envers les fils des hommes !
\VS{16}Parce qu'il a brisé les portes d'airain et cassé les barreaux de fer.
\VS{17}Les insensés sont affligés à cause de leurs transgressions et à cause de leurs iniquités.
\VS{18}Leur âme avait en horreur toute nourriture, et ils touchaient aux portes de la mort.
\VS{19}Alors ils crièrent vers Yahweh dans leur détresse, et il les délivra de leurs angoisses\FTNT{Ps. 50:15 ; Os. 5:15.}.
\VS{20}Il envoya sa parole et les guérit ; et il les délivra de leurs tombeaux.
\VS{21}Qu'ils célèbrent Yahweh pour sa bonté et ses merveilles envers les fils des hommes !
\VS{22}Qu'ils offrent des sacrifices de remerciements, et qu'ils racontent ses œuvres avec des cris de joie.
\VS{23}Ceux qui descendaient sur la mer dans des navires, faisant commerce sur les grandes eaux,
\VS{24}ceux-là virent les œuvres de Yahweh et ses merveilles dans les lieux profonds,
\VS{25}car il dit, et il fit paraître la tempête qui souleva les vagues de la mer.
\VS{26}Ils montaient vers les cieux, ils descendaient dans l'abîme ; leur âme se fondait d'angoisse.
\VS{27}Saisis de vertiges, ils chancelaient comme un homme ivre ; et toute leur sagesse était anéantie\FTNT{Es. 51:17-21 ; Jé. 13:13.}.
\VS{28}Alors ils crièrent vers Yahweh dans leur détresse, et il les tira hors de leurs angoisses.
\VS{29}Il arrêta la tempête, la changeant en calme, et les ondes se turent.
\VS{30}Puis ils se réjouirent de ce qu'elles s'étaient apaisées, et il les conduisit au port qu'ils désiraient.
\VS{31}Qu'ils célèbrent Yahweh pour sa bonté et ses merveilles envers les fils des hommes !
\VS{32}Et qu'ils l'exaltent dans l'assemblée du peuple et le louent dans l'assemblée des anciens.
\VS{33}Il réduit les fleuves en désert, et les sources d'eaux en sécheresse ;
\VS{34}la terre fertile en terre salée, à cause de la méchanceté de ses habitants\FTNT{Jé. 12:4 ; Jé. 17:6.}.
\VS{35}Il transforme le désert en étangs d'eaux, et la terre sèche en des sources d'eaux\FTNT{Es. 41:18.} ;
\VS{36}il y établit ceux qui sont affamés, ils bâtissent des villes pour l'habiter.
\VS{37}Ils ensemencent des champs et plantent des vignes qui rendent du fruit tous les ans.
\VS{38}Il les bénit et ils se multiplient extrêmement ; et il ne laisse point diminuer leur bétail.
\VS{39}Puis ils sont amoindris et humiliés par l'oppression, le malheur et la souffrance.
\VS{40}Il répand le mépris sur les princes et les fait errer dans des lieux déserts sans chemin.
\VS{41}Mais il relève le pauvre et le délivre de la misère, il établit les familles comme des troupeaux\FTNT{1 S. 2:8 ; Ps. 113:7.}.
\VS{42}Les hommes droits le voient et se réjouissent, mais toute iniquité a la bouche fermée.
\VS{43}Quiconque est sage prendra garde à ces choses, afin qu'on considère les bontés de Yahweh.
\Chap{108}
\TextTitle{Yahweh, le secours}
\VerseOne{}Cantique. Psaume de David. Mon cœur est affermi, ô Dieu ! Je chante et je joue de mes instruments, c'est ma gloire !
\VS{2}Réveillez-vous, mon luth et ma harpe ! Je me réveillerai à l'aube du jour.
\VS{3}Yahweh, je te célébrerai parmi les peuples et je te chanterai parmi les nations.
\VS{4}Car ta bonté est grande par-dessus les cieux, et ta vérité atteint jusqu'aux nues.
\VS{5}Ô Dieu ! Elève-toi sur les cieux, et que ta gloire soit sur toute la terre !
\VS{6}Afin que ceux que tu aimes soient délivrés ; sauve-moi par ta droite et exauce-moi !
\VS{7}Dieu a dit dans sa sainteté : Je me réjouirai, je partagerai Sichem et mesurerai la vallée de Succoth.
\VS{8}Galaad sera à moi, Manassé sera à moi, et Ephraïm sera le sommet de ma forteresse, Juda mon législateur.
\VS{9}Moab sera le bassin où je me laverai, je jetterai mon soulier sur Edom, je triompherai des Philistins. 
\VS{10}Qui me conduira dans la ville forte ? Qui me conduira jusqu'en Edom ?
\VS{11}N'est-ce pas toi, ô Dieu, qui nous avais rejetés, et qui ne sortais plus, ô Dieu, avec nos armées ?
\VS{12}Donne–nous du secours pour sortir de la détresse ! Car la délivrance qu'on attend de l'homme est vaine.
\VS{13}Avec Dieu, nous ferons des exploits ; il foulera nos ennemis\FTNT{Ps. 60:5-14.}.
\Chap{109}
\TextTitle{La méchanceté de l'homme}
\VerseOne{}Psaume de David ; Donné au chef des chantres\FTNT{Les Psaumes d'imprécations (Ps. 35 ; 52 ; 55 ; 58, 59 ; 79 ; 109 ; 137) sont des demandes faites à Dieu pour qu'il punisse les méchants. Le Seigneur Jésus-Christ nous demande aujourd'hui de bénir nos ennemis (Lu. 6:27-37).}. Dieu de ma louange, ne te tais point !
\VS{2}Car la bouche du méchant et la bouche remplie de fraude se sont ouvertes contre moi ; ils parlent contre moi avec une langue mensongère !
\VS{3}Ils m'entourent de paroles pleines de haine et ils me font la guerre sans cause !
\VS{4}Tandis que je les aime, ils sont mes ennemis ; mais moi, je n'ai fait que prier en leur faveur !
\VS{5}Ils me rendent le mal pour le bien, et la haine pour l'amour que je leur porte.
\VS{6}Etablis le méchant sur lui, et que Satan se tienne à sa droite !
\VS{7}Quand il sera jugé, fais qu'il soit déclaré méchant, et que sa prière soit regardée comme un crime !
\VS{8}Que sa vie soit courte\FTNT{Il est question ici du suicide de Judas (Mt. 27:3-5).} et qu'un autre prenne sa charge\FTNT{Ce passage parle de Judas (Ac. 1:20).} !
\VS{9}Que ses enfants soient orphelins et sa femme veuve !
\VS{10}Que ses enfants soient entièrement vagabonds, et qu'ils mendient et quêtent en sortant de leurs maisons détruites\FTNT{Job. 20:10.} !
\VS{11}Que le créancier usant d'exaction attrape tout ce qui est à lui et que les étrangers butinent tout son travail !
\VS{12}Qu'il n'y ait personne qui étende sa compassion sur lui, et qu'il n'y ait personne qui ait pitié de ses orphelins !
\VS{13}Que sa postérité soit exposée à être retranchée ; que leur nom soit effacé dans la génération qui le suivra !
\VS{14}Que l'iniquité de ses pères revienne en mémoire à Yahweh, et que le péché de sa mère ne soit point effacé !
\VS{15}Qu'ils soient continuellement devant Yahweh, et qu'il retranche leur mémoire de la terre\FTNT{Ps. 34:17.},
\VS{16}parce qu'il ne s'est point souvenu d'user de miséricorde, mais il a persécuté l'homme affligé et misérable, dont le cœur est brisé, et cela pour le faire mourir !
\VS{17}Puisqu'il aime la malédiction, que la malédiction tombe sur lui ! Puisqu'il ne prend pas plaisir à la bénédiction, que la bénédiction aussi s'éloigne de lui !
\VS{18}Et qu'il soit revêtu de la malédiction comme de sa robe ; qu'elle entre dans son corps comme de l'eau, et dans ses os comme de l'huile !
\VS{19}Qu'elle lui soit comme un vêtement dont il se couvre, et comme une ceinture dont il se ceigne continuellement !
\VS{20}Telle soit, de la part de Yahweh, la récompense de mes adversaires, et de ceux qui parlent mal de moi !
\VS{21}Mais toi, Yahweh, Seigneur, agis avec moi pour l'amour de ton Nom ! Et parce que ta miséricorde est grande, délivre-moi !
\VS{22}Car je suis affligé et misérable, et mon cœur est blessé au-dedans de moi.
\VS{23}Je m'en vais comme l'ombre quand elle décline, et je suis chassé comme une sauterelle.
\VS{24}Mes genoux sont affaiblis par le jeûne, et mon corps est épuisé de maigreur au lieu d'être gras.
\VS{25}Je suis pour eux un objet d'opprobre ; quand ils me voient, ils secouent la tête.
\VS{26}Yahweh, mon Dieu ! Aide-moi, délivre-moi selon ta miséricorde.
\VS{27}Afin qu'on sache que c'est ta main, que c'est toi, ô Yahweh, qui l'as fait.
\VS{28}Ils maudiront, mais tu béniras ; ils s'élèveront, mais ils seront confus ; et ton serviteur se réjouira.
\VS{29}Que mes adversaires soient revêtus de confusion et couverts de leur honte comme d'un manteau.
\VS{30}Je célébrerai hautement de ma bouche Yahweh, et je le louerai au milieu de plusieurs nations.
\VS{31}De ce qu'il se tient à la droite du misérable pour le délivrer de ceux qui condamnent son âme.
\Chap{110}
\TextTitle{Yahweh, le Roi et le Sacrificateur}
\VerseOne{}Psaume de David. Yahweh a dit à mon Seigneur : Assieds-toi à ma droite, jusqu'à ce que je fasse de tes ennemis le marchepied de tes pieds\FTNT{Ce psaume affirme la divinité de Jésus-Christ (Mt. 22:41-46 ; Mc. 12:35-37 ; Lu. 20:41-44 ; Ac. 2:34-35 ; Hé. 1:13 ; Hé. 10:12-13).}.
\VS{2}Yahweh étendra de Sion le sceptre de ta puissance, en disant : Domine au milieu de tes ennemis\FTNT{Es. 2:2-3 ; Da. 7:14.} !
\VS{3}Ton peuple est plein d'ardeur quand tu rassembles ton armée ; avec des ornements sacrés, du sein de l'aurore, ta jeunesse vient à toi comme une rosée.
\VS{4}Yahweh l'a juré, et il ne s'en repentira point que tu es sacrificateur éternellement, à la manière de Melchisédek\FTNT{Hé. 5:6 ; Hé. 6:20 ; Hé. 7:17 ; Ge. 14:18.}.
\VS{5}Le Seigneur est à ta droite, il brisera les rois au jour de sa colère.
\VS{6}Il exercera le jugement sur les nations, il remplira tout de cadavres ; il brisera le chef qui domine sur un grand pays\FTNT{Ap. 14 ; Ap. 16.}.
\VS{7}Il boit au torrent pendant la marche : C'est pourquoi il lève haut la tête.
\Chap{111}
\TextTitle{Les oeuvres magnifiques de Dieu}
\VerseOne{}Louez Yahweh. [Aleph.] Je célébrerai Yahweh de tout mon cœur, [Beth.] dans la compagnie des hommes droits et dans l'assemblée.
\VS{2}[Guimel.] Les œuvres de Yahweh sont grandes, [Daleth.] elles sont recherchées par tous ceux qui y prennent plaisir.
\VS{3}[He.] Son œuvre n'est que majesté et magnificence, [Vav.] et sa justice demeure à perpétuité.
\VS{4}[Zayin.] Il a rendu ses merveilles mémorables. [Heth.] Yahweh est miséricordieux et compatissant.
\VS{5}[Teth.] Il a donné de la nourriture à ceux qui le craignent ; [Yod.] il s'est souvenu pour toujours de son alliance.
\VS{6}[Kaf.] Il a manifesté à son peuple la puissance de ses œuvres, [Lamed.] en leur donnant l'héritage des nations.
\VS{7}[Mem.] Les œuvres de ses mains ne sont que vérité et équité. [Nun.] Tous ses commandements sont véritables,
\VS{8}[Samech.] appuyés à perpétuité, éternellement, [Ayin.] faits avec fidélité et droiture.
\VS{9}[Pe.] Il a envoyé la rédemption à son peuple\FTNT{Ex. 6:6 ; Jn. 3:16.} ; [Tsade.] il lui a donné une alliance éternelle ; [Qof.] son nom est saint et redoutable.
\VS{10}[Resh.] Le commencement de la sagesse c'est la crainte de Yahweh : [Shin.] Tous ceux qui s'adonnent à faire ce qu'elle prescrit sont sages\FTNT{Pr. 1:7 ; Pr. 9:10 ; Pr. 8:13 ; De. 4:6.}. [Tav.] Sa louange demeure à perpétuité.
\Chap{112}
\TextTitle{La crainte de Yahweh enrichit et donne de l'assurance}
\VerseOne{}Louez Yahweh. [Aleph.] Heureux l'homme qui craint Yahweh [Beth.] et qui prend un grand plaisir à ses commandements !
\VS{2}[Guimel.] Sa postérité sera puissante sur la terre, [Daleth.] la génération des hommes droits sera bénie\FTNT{Pr. 20:7.}.
\VS{3}[He.] Il y aura des biens et des richesses dans sa maison ; [Vav.] et sa justice demeure à perpétuité.
\VS{4}[Zayin.] La lumière s'est levée dans les ténèbres sur ceux qui sont justes\FTNT{Pr. 4:18 ; Ps. 37:6.} ; [Heth.] il est compatissant, miséricordieux et juste.
\VS{5}[Teth.] Heureux l'homme de bien qui exerce la miséricorde et prête, [Yod.] qui règle ses actions avec justice.
\VS{6}[Kaf.] Il ne chancelle jamais. [Lamed.] La mémoire du juste dure toujours\FTNT{Pr. 10:7.}.
\VS{7}[Mem.] Il ne craint pas les mauvaises nouvelles ; [Nun.] son cœur est ferme, confiant en Yahweh.
\VS{8}[Samech.] Son cœur est bien affermi, il ne craint pas, [Ayin.] jusqu'à ce qu'il mette son plaisir à regarder ses adversaires.
\VS{9}[Pe.] Il fait des largesses, il donne aux pauvres ; [Tsade.] sa justice demeure à perpétuité ; [Qof.] sa corne s'élève en gloire.
\VS{10}[Resh.] Le méchant le voit et s'irrite. [Shin.] Il grince des dents et se consume ; [Tav.] les désirs des méchants périssent.
\Chap{113}
\TextTitle{Yahweh, le Dieu élevé au-dessus de tout}
\VerseOne{}Louez Yahweh\FTNT{Les psaumes disant « Alléluia » sont les Ps. 104 à 106, 111 à 113, 115 à 117, 135 à 136, 146 à 150. Parmi eux, les psaumes 135 et 146 à 150, étaient chantés durant le service quotidien d'adoration dans la synagogue. Les psaumes 115 à 118, appelés « le grand Hallel », étaient chantés lors des fêtes de Pâque. Alléluia veut dire « Louez Yahweh » (Ap. 19:1).} ! Louez, vous serviteurs de Yahweh, louez le Nom de Yahweh !
\VS{2}Que le Nom de Yahweh soit béni dès maintenant et à toujours !
\VS{3}Le Nom de Yahweh est digne de louanges depuis le soleil levant jusqu'au soleil couchant.
\VS{4}Yahweh est élevé par-dessus toutes les nations, sa gloire est au-dessus des cieux.
\VS{5}Qui est semblable à Yahweh notre Dieu, qui habite dans les lieux très hauts ?
\VS{6}Il s'abaisse pour regarder sur le ciel et sur la terre,
\VS{7}Il relève l'affligé de la poussière, et retire le pauvre\FTNT{1 S. 2:8 ; Ps. 107:41.} de dessus le fumier,
\VS{8}pour le faire asseoir avec les nobles, avec les nobles de son peuple\FTNT{Job. 36:7.}.
\VS{9}Il donne une maison à la femme stérile, il en fait une mère joyeuse au milieu de ses fils\FTNT{Ge. 17:17-21 ; 1 S. 2:5 ; Ps. 68:6.}. Louez Yahweh !
\Chap{114}
\TextTitle{La création tremble devant le Tout-Puissant}
\VerseOne{}Quand Israël sortit d'Egypte, quand la maison de Jacob s'éloigna d'un peuple barbare,
\VS{2}Juda devint son lieu saint, Israël son domaine\FTNT{Jé. 2:2-3.}.
\VS{3}La mer le vit et s'enfuit, le Jourdain retourna en arrière\FTNT{Jos. 3:13-16 ; Ps. 77:17.}.
\VS{4}Les montagnes sautèrent comme des béliers, les collines comme des agneaux\FTNT{Jg. 5:5 ; Ha. 3:10 ; Ps. 68:9.}.
\VS{5}Ô Mer ! Qu'avais-tu pour t'enfuir ? Jourdain, pour retourner en arrière ?
\VS{6}Et vous montagnes, pour sauter comme des béliers ? Et vous collines, comme des agneaux ?
\VS{7}Ô Terre ! Tremble devant la présence du Seigneur, devant la présence du Dieu de Jacob,
\VS{8}qui a changé le rocher en un étang d'eaux, la pierre très dure en une source d'eaux.
\Chap{115}
\TextTitle{Louange au Dieu de gloire}
\VerseOne{}Non point à nous, ô Yahweh ! Non point à nous, mais à ton Nom donne gloire, à cause de ta bonté, à cause de ta fidélité !
\VS{2}Pourquoi les nations diraient-elles : Où est maintenant leur Dieu ?
\VS{3}Certes notre Dieu est au ciel, il fait tout ce qu'il veut\FTNT{Ps. 135:6 ; Job. 23:13.}.
\VS{4}Leurs idoles sont des dieux d'or et d'argent, elles sont l'ouvrage de mains d'homme.
\VS{5}Elles ont une bouche, et ne parlent point ; elles ont des yeux, et ne voient point ;
\VS{6}elles ont des oreilles, et n'entendent point ; elles ont un nez, et ne sentent point ;
\VS{7}elles ont des mains, et elles ne touchent point ; elles ont des pieds, et elles ne marchent point ; et elles ne rendent aucun son de leur gosier\FTNT{Ex. 32:2-8 ; 1 R. 18:25-26 ; Es. 44:9 ; Ez. 8:8-12.}.
\VS{8}Ils leur ressemblent, ceux qui les fabriquent, tous ceux qui se confient en eux.
\VS{9}Israël confie-toi en Yahweh ; il est leur secours et leur bouclier de ceux qui se confient en lui.
\VS{10}Maison d'Aaron, confie-toi en Yahweh ; il est leur secours et leur bouclier.
\VS{11}Vous qui craignez Yahweh, confiez-vous en Yahweh ; il est leur secours et leur bouclier.
\VS{12}Yahweh s'est souvenu de nous, il bénira, il bénira la maison d'Israël, il bénira la maison d'Aaron.
\VS{13}Il bénira ceux qui craignent Yahweh, tant les petits que les grands.
\VS{14}Yahweh vous multipliera ses bénédictions, à vous et à vos fils.
\VS{15}Vous êtes bénis de Yahweh, qui a fait les cieux et la terre.
\VS{16}Les cieux, sont les cieux de Yahweh, mais il a donné la terre aux fils des hommes.
\VS{17}Ce ne sont pas les morts qui célèbrent Yahweh, ce n'est aucun de ceux qui descendent dans le lieu du silence\FTNT{Ps. 88:11 ; Es. 38:18-19 ; Ps. 6:6.}.
\VS{18}Mais nous, nous bénirons Yahweh dès maintenant et pour toujours. Louez Yahweh !
\Chap{116}
\TextTitle{Psaume des rachetés}
\VerseOne{}J'aime Yahweh, car il a entendu ma voix et mes supplications.
\VS{2}Car il a incliné son oreille vers moi, c'est pourquoi je l'invoquerai durant mes jours.
\VS{3}Les liens de la mort m'avaient environné, et les angoisses du scheol m'avaient trouvé\FTNT{2 S. 22:5 ; Ps. 18:5.} ; j'avais trouvé la détresse et la douleur.
\VS{4}Mais j'invoquai le Nom de Yahweh en disant : Je te prie, délivre mon âme, ô Yahweh !
\VS{5}Yahweh est compatissant et juste, et notre Dieu fait miséricorde.
\VS{6}Yahweh garde les simples ; j'étais devenu misérable et il m'a sauvé.
\VS{7}Mon âme, retourne dans ton repos, car Yahweh t'a fait du bien.
\VS{8}Parce que tu as retiré mon âme de la mort, mes yeux des larmes et mes pieds de la chute,
\VS{9}je marcherai dans la présence de Yahweh, sur la terre des vivants.
\VS{10}J'ai cru, c'est pourquoi j'ai parlé\FTNT{2 Co. 4:13.} ; j'ai été fort affligé.
\VS{11}Je disais dans ma précipitation : Tout homme est menteur\FTNT{Ro. 3:4.}.
\VS{12}Que rendrai-je à Yahweh ? Tous ses bienfaits envers moi ?
\VS{13}J'élèverai la coupe des délivrances et j'invoquerai le Nom de Yahweh.
\VS{14}J'accomplirai maintenant mes vœux à Yahweh, devant tout son peuple.
\VS{15}Elle a du prix aux yeux de Yahweh, la mort de ceux qu'il aime.
\VS{16}Ecoute-moi, ô Yahweh ! Car je suis ton serviteur, je suis ton serviteur, fils de ta servante. Tu as délié mes liens.
\VS{17}Je t'offrirai le sacrifice de remerciement et j'invoquerai le Nom de Yahweh.
\VS{18}J'accomplirai maintenant mes vœux à Yahweh, devant tout son peuple,
\VS{19}dans les parvis de la maison de Yahweh, au milieu de toi, Jérusalem ! Louez Yahweh !
\Chap{117}
\TextTitle{Toutes les nations louent Yahweh}
\VerseOne{}Toutes les nations, louez Yahweh ! Tous les peuples, célébrez-le !
\VS{2}Car sa miséricorde est grande envers nous, et sa fidélité dure à toujours. Louez Yahweh !
\Chap{118}
\TextTitle{Yahweh, le Dieu de mon secours}
\VerseOne{}Célébrez Yahweh, car il est bon, parce que sa bonté dure à toujours !
\VS{2}Qu'Israël dise maintenant : Car sa bonté dure à toujours !
\VS{3}Que la maison d'Aaron dise maintenant : Car sa bonté dure à toujours !
\VS{4}Que ceux qui craignent Yahweh disent maintenant : Car sa bonté dure à toujours !
\VS{5}Me trouvant dans la détresse, j'ai invoqué Yahweh\FTNT{Ps. 120:1.} ; et Yahweh m'a répondu et m'a mis au large.
\VS{6}Yahweh est pour moi, je ne craindrai point. Que me ferait l'homme ?
\VS{7}Yahweh est pour moi parmi ceux qui me secourent, c'est pourquoi je verrai en ceux qui me haïssent ce que je désire.
\VS{8}Mieux vaut se confier en Yahweh que se confier en l'homme\FTNT{Es. 2:22 ; Jé. 17:5 ; Ps. 62:9.}.
\VS{9}Mieux vaut se confier en Yahweh que se reposer sur les grands d'entre les peuples.
\VS{10}Toutes les nations m'avaient environné, mais au Nom de Yahweh je les taille en pièces.
\VS{11}Elles m'avaient environné, elles m'avaient, dis-je, environné ; mais au Nom de Yahweh je les taille en pièces.
\VS{12}Elles m'avaient environné comme des abeilles, elles s'éteignent comme un feu d'épines\FTNT{De. 1:44.}, car au Nom de Yahweh je les taille en pièces.
\VS{13}Tu me poussais violemment pour me faire tomber, mais Yahweh m'a secouru.
\VS{14}Yahweh est ma force et le sujet de mes louanges, et il a été ma délivrance\FTNT{Ex. 15:2 ; Es. 12:2.}.
\VS{15}Une voix de chant de triomphe et de délivrance retentit dans les tentes des justes : La droite de Yahweh exerce la puissance !
\VS{16}La droite de Yahweh est élevée, la droite de Yahweh exerce sa puissance.
\VS{17}Je ne mourrai pas, je vivrai et je raconterai les œuvres de Yahweh.
\VS{18}Yahweh m'a châtié sévèrement, mais il ne m'a point livré à la mort.
\VS{19}Ouvrez-moi les portes de la justice ; j'y entrerai et je célébrerai Yahweh.
\VS{20}C'est ici la porte de Yahweh, les justes y entreront.
\VS{21}Je te célébrerai parce que tu m'as exaucé et tu as été mon libérateur.
\VS{22}La Pierre que les architectes avaient rejetée, est devenue la principale de l'angle\FTNT{Le Messie est présenté comme la pierre ou le rocher. (cp. Es. 8:13-17 ; 1 Pi. 2:7).}.
\VS{23}Ceci a été fait par Yahweh, c'est un prodige à nos yeux.
\VS{24}C'est ici la journée que Yahweh a faite, qu'elle soit pour nous un sujet d'allégresse et de joie.
\VS{25}Yahweh, je te prie, délivre maintenant. Yahweh, je te prie, donne maintenant la prospérité !
\VS{26}Béni soit celui qui vient au Nom de Yahweh ! Nous vous bénissons de la maison de Yahweh.
\VS{27}Yahweh est Dieu, et il nous a éclairés. Liez avec des cordes la bête du sacrifice, et amenez-la jusqu'aux cornes de l'autel.
\VS{28}Tu es mon Dieu, c'est pourquoi je te célébrerai. Tu es mon Dieu, je t'exalterai.
\VS{29}Célébrez Yahweh car il est bon, parce que sa miséricorde demeure à toujours !
\Chap{119}
\TextTitle{La Parole de Yahweh éclaire}
\VerseOne{}[Aleph.] Heureux ceux qui sont intègres dans leur voie, qui marchent selon la loi de Yahweh.
\VS{2}Heureux sont ceux qui gardent ses préceptes et qui le cherchent de tout leur cœur\FTNT{Jos. 1:8.} ;
\VS{3}qui ne font point d'iniquité, qui marchent dans ses voies\FTNT{1 Jn. 3:9 ; 1 Jn. 5:18.}.
\VS{4}Tu as donné tes commandements afin qu'on les garde soigneusement.
\VS{5}Oh ! Que mes voies soient bien établies pour garder tes statuts !
\VS{6}Et je ne rougirai point de honte quand je regarderai à tous tes commandements.
\VS{7}Je te célébrerai avec droiture de cœur quand j'aurai appris les ordonnances de ta justice.
\VS{8}Je veux garder tes statuts, ne me délaisse point entièrement.
\VS{9}[Beth.] Par quel moyen le jeune homme rendra-t-il pure sa voie ? Ce sera en y prenant garde selon ta parole.
\VS{10}Je te recherche de tout mon cœur, ne me laisse pas m'égarer loin de tes commandements.
\VS{11}Je serre ta parole dans mon cœur afin de ne pas pécher contre toi.
\VS{12}Yahweh ! Tu es béni ; enseigne-moi tes statuts.
\VS{13}De mes lèvres je raconte toutes les ordonnances de ta bouche.
\VS{14}Je me réjouis dans le chemin de tes préceptes comme si je possédais toutes les richesses du monde.
\VS{15}Je médite tes commandements et j'observe tes voies.
\VS{16}Je prends plaisir à tes statuts et je n'oublie pas tes paroles.
\VS{17}[Guimel.] Fais du bien à ton serviteur afin que je vive, et je garderai ta parole\FTNT{Ps. 116:7.}.
\VS{18}Ouvre mes yeux afin que je regarde aux merveilles de ta loi\FTNT{Ep. 1:18.} !
\VS{19}Je suis voyageur sur la terre, ne me cache pas tes commandements.
\VS{20}Mon âme est brisée par le désir qui toujours me porte vers tes ordonnances.
\VS{21}Tu réprimandes les orgueilleux, ces maudits, qui se détournent de tes commandements.
\VS{22}Décharge-moi de l'opprobre et du mépris, car j'ai gardé tes préceptes\FTNT{Ps. 3:9.}.
\VS{23}Même les princes s'assoient et parlent contre moi pendant que ton serviteur médite tes statuts.
\VS{24}Tes préceptes font mes délices, ce sont mes conseillers.
\VS{25}[Daleth.] Mon âme est attachée à la poussière, fais-moi revivre selon ta parole\FTNT{Ps. 44:26 ; Ps. 143:11.}.
\VS{26}Je te raconte mes voies et tu me réponds ; enseigne-moi tes statuts.
\VS{27}Fais-moi entendre la voie de tes commandements, et je parlerai de tes merveilles\FTNT{Ps. 145:6.}.
\VS{28}Mon âme pleure de chagrin, relève-moi selon tes paroles.
\VS{29}Eloigne de moi la voie du mensonge et accorde–moi la grâce d'observer ta loi\FTNT{Le mot « loi » vient de l'hebreu « towrah » qui donne « Thora » en français}.
\VS{30}Je choisis la voie de la vérité et je place tes ordonnances sous mes yeux.
\VS{31}Je m'attache à tes préceptes, ô Yahweh ! Ne me fais point rougir de honte.
\VS{32}Je cours dans la voie de tes commandements car tu élargis mon cœur.
\VS{33}[He.] Yahweh, enseigne-moi la voie de tes statuts, et je la garderai jusqu'au bout.
\VS{34}Donne-moi de l'intelligence ; je garderai ta loi et je l'observerai de tout mon cœur\FTNT{Pr. 2:6 ; Ja. 1:5.}.
\VS{35}Fais-moi marcher sur le sentier de tes commandements car j'y prends plaisir.
\VS{36}Incline mon cœur à tes préceptes et non point au profit\FTNT{Ez. 33:31 ; Mc 7:21-22 ; Hé. 13:5.}.
\VS{37}Détourne mes yeux de la vue des choses vaines ; fais-moi vivre dans ta voie.
\VS{38}Accomplis ta parole envers ton serviteur, parole qui est pour ceux qui te craignent.
\VS{39}Eloigne de moi l'opprobre que je redoute, car tes ordonnances sont bonnes.
\VS{40}Voici, je désire pratiquer tes commandements, fais-moi vivre dans ta justice.
\VS{41}[Vav.] Que ta miséricorde vienne sur moi, ô Yahweh ! Et ta délivrance aussi, selon ta promesse !
\VS{42}Et je pourrai répondre à celui qui m'outrage, car je me confie en ta parole.
\VS{43}N'arrache pas de ma bouche la parole de vérité, car j'espère en tes jugements.
\VS{44}Je garderai continuellement ta loi, à toujours et à perpétuité.
\VS{45}Je marcherai au large parce que je recherche tes commandements.
\VS{46}Je parlerai de tes préceptes devant les rois et je ne rougirai pas de honte\FTNT{Ps. 138:1-4 ; Mt. 10:18-19 ; Ac. 26.}.
\VS{47}Je fais mes délices de tes commandements que j'aime ;
\VS{48}j'étends mes mains vers tes commandements que j'aime ; et je médite tes statuts.
\VS{49}[Zayin.] Souviens-toi de la parole donnée à ton serviteur, sur laquelle tu m'as fait espérer.
\VS{50}C'est ici ma consolation dans mon affliction, car ta parole me rend la vie.
\VS{51}Les orgueilleux se sont fort moqués de moi, mais je ne me suis pas détourné de ta loi.
\VS{52}Yahweh, je me souviens de tes jugements anciens et je me suis consolé en eux.
\VS{53}L'horreur me saisit à cause des méchants qui abandonnent ta loi.
\VS{54}Tes statuts sont le sujet de mes cantiques dans la maison où je suis étranger.
\VS{55}Yahweh, je me souviens de ton Nom pendant la nuit et je garde ta loi.
\VS{56}Cela m'arrive parce que je garde tes commandements.
\VS{57}[Heth.] Ô Yahweh ! J'en conclus que ma part est de garder tes paroles.
\VS{58}Je te supplie de tout mon cœur : Aie pitié de moi selon ta parole.
\VS{59}Je fais le compte de mes voies et je rebrousse chemin vers tes préceptes\FTNT{Os. 6:3 ; La. 3:40.}.
\VS{60}Je me hâte, je ne diffère point de garder tes commandements.
\VS{61}Une compagnie de méchants me pille, mais je n'oublie pas ta loi.
\VS{62}Je me lève au milieu de la nuit pour te célébrer à cause des ordonnances de ta justice.
\VS{63}Je suis l'ami de tous ceux qui te craignent et qui gardent tes commandements.
\VS{64}Yahweh, la terre est pleine de ta bonté ; enseigne-moi tes statuts.
\VS{65}[Teth.] Yahweh, tu fais du bien à ton serviteur selon ta parole.
\VS{66}Enseigne-moi le bon sens et la connaissance car je crois à tes commandements.
\VS{67}Avant d'avoir été humilié, je m'égarais, mais maintenant j'observe ta parole.
\VS{68}Tu es bon et bienfaisant, enseigne-moi tes statuts.
\VS{69}Les orgueilleux imaginent des faussetés contre moi, mais je garde de tout mon cœur tes commandements.
\VS{70}Leur cœur est insensible comme la graisse, mais moi, je prends plaisir dans ta loi\FTNT{De. 32:15 ; Jé. 5:28.}.
\VS{71}Il est bon que je sois humilié afin que j'apprenne tes statuts.
\VS{72}La loi que tu as prononcée de ta bouche m'est plus précieuse que mille pièces d'or ou d'argent\FTNT{Ps. 19:10-11 ; Job. 22:2.}.
\VS{73}[Yod.] Tes mains m'ont façonné, elles m'ont formé\FTNT{Jé. 1:5 ; Job. 10:9.} ; donne-moi l'intelligence afin que j'apprenne tes commandements.
\VS{74}Ceux qui te craignent me verront et se réjouiront, parce que j'espère en tes promesses.
\VS{75}Je reconnais, ô Yahweh, que tes jugements sont justes, et que tu m'as humilié par ta fidélité\FTNT{Hé. 12:10.}.
\VS{76}Que ta bonté soit ma consolation, comme tu l'as promis à ton serviteur.
\VS{77}Que tes compassions viennent sur moi et je vivrai ; car ta loi fait mes délices.
\VS{78}Que les orgueilleux rougissent de honte, de ce qu'ils m'oppriment sans cause ; mais moi, je médite sur tes ordonnances.
\VS{79}Que ceux qui te craignent et ceux qui connaissent tes préceptes reviennent vers moi.
\VS{80}Que mon cœur soit intègre dans tes statuts afin que je ne sois pas couvert de honte.
\VS{81}[Kaf.] Mon âme se consume en attendant ta délivrance ; j'espère en ta promesse.
\VS{82}Mes yeux s'épuisent en attendant ta promesse, lorsque je dis : Quand me consoleras-tu ?
\VS{83}Car je suis comme une outre dans la fumée, je n'oublie pas tes statuts.
\VS{84}Quel est le nombre de jours de ton serviteur ? Quand jugeras-tu ceux qui me poursuivent\FTNT{Ap. 6:10.} ?
\VS{85}Les orgueilleux me creusent des fosses, ils n'agissent pas selon ta loi.
\VS{86}Tous tes commandements ne sont que fidélité ; on me persécute sans cause, aide-moi\FTNT{Mt. 5:10.} !
\VS{87}On m'a presque réduit à rien et mis par terre ; mais je n'ai point abandonné tes commandements.
\VS{88}Fais-moi revivre selon ta miséricorde et je garderai les préceptes de ta bouche.
\VS{89}[Lamed.] Ô Yahweh ! Ta parole subsiste à toujours dans les cieux.
\VS{90}Ta fidélité dure d'âge en âge ; tu as établi la terre, et elle demeure ferme\FTNT{Pr. 1:4.}.
\VS{91}Ces choses subsistent aujourd'hui selon tes ordonnances, car toutes choses te servent.
\VS{92}Si ta loi n'avait pas fait mes délices, j'aurais déjà péri dans mon affliction.
\VS{93}Je n'oublierai jamais tes commandements car c'est par eux que tu m'as fait revivre.
\VS{94}Je suis à toi, sauve-moi ; car je recherche tes commandements.
\VS{95}Les méchants m'attendent pour me faire périr, mais je suis attentif à tes préceptes.
\VS{96}Je vois des bornes à tout ce qui est parfait, mais tes commandements n'ont point de limites.
\VS{97}[Mem.] Combien j'aime ta loi\FTNT{Ps. 1:2.} ! Elle est tout le jour l'objet de ma méditation.
\VS{98}Par tes commandements, tu m'as rendu plus sage que mes ennemis, parce que tes commandements sont toujours avec moi.
\VS{99}J'ai surpassé en prudence tous ceux qui m'avaient enseigné parce que tes préceptes sont l'objet de ma méditation.
\VS{100}Je suis devenu plus intelligent que les vieillards parce que j'observe tes commandements.
\VS{101}Je garde mes pieds de toute mauvaise voie afin d'observer ta parole.
\VS{102}Je ne me suis point détourné de tes ordonnances parce que tu me les enseignes.
\VS{103}Que ta parole est douce à mon palais ! Plus douce que le miel à ma bouche.
\VS{104}Je suis devenu intelligent par tes commandements, c'est pourquoi je hais toute voie de mensonge.
\VS{105}[Nun.] Ta parole est une lampe à mes pieds et une lumière sur mon sentier\FTNT{Pr. 6:23 ; 2 Pi. 1:19.}.
\VS{106}J'ai juré et je le tiendrai, d'observer les lois de ta justice\FTNT{Né. 10:29.}.
\VS{107}Yahweh, je suis extrêmement affligé, fais-moi revivre selon ta parole.
\VS{108}Yahweh, je te prie, agrée les sentiments que ma bouche exprime, et enseigne-moi tes ordonnances\FTNT{Os. 14:2 ; Hé. 13:15.}.
\VS{109}Ma vie est continuellement en danger, toutefois je n'oublie pas ta loi.
\VS{110}Les méchants m'ont tendu des pièges, toutefois je ne me suis point égaré de tes commandements.
\VS{111}J'ai pris pour héritage perpétuel tes préceptes car ils sont la joie de mon cœur.
\VS{112}J'ai incliné mon cœur à accomplir toujours tes statuts jusqu'au bout.
\VS{113}[Samech.] Je hais les hommes indécis\FTNT{1 R. 18:21 ; Ja. 1:6 ; Ja. 4:8.}, mais j'aime ta loi.
\VS{114}Tu es mon refuge et mon bouclier, je m'attends à ta parole.
\VS{115}Méchants, retirez-vous de moi\FTNT{Mt. 7:23 ; Ps. 6:9.} ! Et je garderai les commandements de mon Dieu.
\VS{116}Soutiens-moi suivant ta parole, et je vivrai ; et ne me fais point rougir de honte en me refusant ce que j'espérais.
\VS{117}Soutiens-moi, et je serai en sûreté ; et j'aurai continuellement les yeux sur tes statuts.
\VS{118}Tu as foulé aux pieds tous ceux qui se détournent de tes statuts, car le mensonge est le moyen dont ils se servent pour tromper.
\VS{119}Tu réduis à néant tous les méchants de la terre, comme de l'écume ; c'est pourquoi j'aime tes préceptes.
\VS{120}Ma chair frissonne de l'effroi que tu m'inspires et je crains tes jugements\FTNT{Ha. 3:16.}.
\VS{121}[Ayin.] J'ai exercé le jugement et la justice, ne m'abandonne pas à ceux qui me font tort.
\VS{122}Prends sous ta garantie le bien de ton serviteur et ne permets pas que je sois opprimé par les orgueilleux.
\VS{123}Mes yeux s'épuisent en attendant ta délivrance et la parole de ta justice.
\VS{124}Agis envers ton serviteur suivant ta miséricorde et enseigne-moi tes statuts.
\VS{125}Je suis ton serviteur, donne-moi l'intelligence, et je connaîtrai tes préceptes\FTNT{Pr. 1:4 ; Pr. 6:23.}.
\VS{126}Il est temps que Yahweh opère ; ils ont aboli ta loi.
\VS{127}C'est pourquoi j'aime tes commandements, plus que l'or et l'or fin.
\VS{128}C'est pourquoi je trouve justes tous tes commandements, je hais toute voie de mensonge.
\VS{129}[Pe.] Tes préceptes sont merveilleux, c'est pourquoi mon âme les garde.
\VS{130}La révélation de tes paroles éclaire, elle donne de l'intelligence aux simples.
\VS{131}J'ouvre ma bouche et je soupire, car je désire tes commandements.
\VS{132}Regarde-moi, et aie pitié de moi, selon tes jugements à l'égard de ceux qui aiment ton Nom.
\VS{133}Affermis mes pas sur ta parole, et que l'iniquité n'ait point d'emprise sur moi.
\VS{134}Délivre-moi de l'oppression des hommes afin que je garde tes commandements.
\VS{135}Fais luire ta face sur ton serviteur et enseigne-moi tes statuts.
\VS{136}Mes yeux répandent des torrents d'eau parce qu'on n'observe point ta loi.
\VS{137}[Tsade.] Tu es juste, ô Yahweh, et droit dans tes jugements.
\VS{138}Tu ordonnes tes préceptes avec justice et grande fidélité.
\VS{139}Mon zèle me consume parce que mes adversaires oublient tes paroles.
\VS{140}Ta parole est entièrement éprouvée, c'est pourquoi ton serviteur l'aime.
\VS{141}Je suis petit et méprisé, toutefois je n'oublie point tes commandements.
\VS{142}Ta justice est une justice éternelle, et ta loi est la vérité.
\VS{143}La détresse et l'angoisse m'atteignent, mais tes commandements font mes délices.
\VS{144}Tes préceptes ne sont que justice éternelle ; donne-moi l'intelligence afin que je vive.
\VS{145}[Qof.] Je crie de tout mon cœur, réponds-moi, ô Yahweh ! Je garde tes statuts.
\VS{146}Je crie vers toi, sauve-moi afin que j'observe tes préceptes.
\VS{147}Je devance l'aurore et je crie ; je m'attends à ta parole.
\VS{148}Mes yeux ont devancé les veilles de la nuit pour méditer ta parole.
\VS{149}Ecoute ma voix selon ta miséricorde, ô Yahweh ! Fais-moi revivre selon ton ordonnance.
\VS{150}Ceux qui poursuivent le crime s'approchent de moi, et ils s'éloignent de ta loi.
\VS{151}Yahweh, tu es près de moi ; et tous tes commandements ne sont que vérité.
\VS{152}Depuis longtemps, je sais par tes préceptes, que tu les as établis pour toujours.
\VS{153}[Resh.] Regarde mon affliction et sauve-moi, car je n'oublie pas ta loi.
\VS{154}Soutiens ma cause et rachète-moi ; fais-moi revivre selon ta parole.
\VS{155}La délivrance est loin des méchants parce qu'ils ne recherchent pas tes statuts.
\VS{156}Tes compassions sont en grand nombre, ô Yahweh ! Fais-moi revivre selon tes ordonnances.
\VS{157}Ceux qui me persécutent et qui me pressent sont en grand nombre, toutefois je ne me détourne pas de tes préceptes.
\VS{158}Je vois avec dégoût les traîtres et je suis rempli de tristesse car ils n'observent pas ta parole.
\VS{159}Regarde combien j'aime tes commandements, Yahweh ! Fais-moi revivre selon ta miséricorde !
\VS{160}Le fondement de ta parole est la vérité, et toutes les lois de ta justice sont éternelles.
\VS{161}[Shin.] Les princes du peuple me persécutent sans cause, mais mon cœur tremble à cause de ta parole.
\VS{162}Je me réjouis de ta parole comme ferait celui qui aurait trouvé un grand butin.
\VS{163}J'ai en haine et en abomination le mensonge ; j'aime ta loi.
\VS{164}Sept fois le jour je te loue à cause des ordonnances de ta justice.
\VS{165}Il y a une grande paix pour ceux qui aiment ta loi, et rien ne peut les renverser\FTNT{Es. 32:17 ; Ph. 4:7.}.
\VS{166}Yahweh, j'espère en ta délivrance et je pratique tes commandements.
\VS{167}Mon âme observe tes préceptes, je les aime beaucoup.
\VS{168}J'observe tes commandements et tes préceptes, car toutes mes voies sont devant toi.
\VS{169}[Tav.] Yahweh, que mon cri parvienne jusqu'à toi, donne-moi l'intelligence selon ta parole.
\VS{170}Que ma supplication vienne devant toi, délivre-moi selon ta parole.
\VS{171}Mes lèvres publieront ta louange quand tu m'auras enseigné tes statuts.
\VS{172}Ma langue ne s'entretiendra que de ta parole, parce que tous tes commandements ne sont que justice.
\VS{173}Que ta main me soit en aide, parce que j'ai choisi tes commandements.
\VS{174}Yahweh, je souhaite ta délivrance, et ta loi fait mes délices.
\VS{175}Que mon âme vive afin qu'elle te loue, et que tes ordonnances me soient en aide !
\VS{176}Je suis errant comme une brebis perdue\FTNT{Es. 53:6 ; Lu. 15:4 ; 1 Pi. 2:25.} ; cherche ton serviteur, car je n'oublie pas tes commandements.
\Chap{120}
\TextTitle{Cri de détresse}
\VerseOne{}Cantique des degrés\FTNT{Les psaumes 120 à 134 sont appelés « psaumes des degrés » ou de « l'ascension ». Ces psaumes furent chantés par les Israélites montant à Jérusalem au retour de la captivité de Babylone.}. J'ai invoqué Yahweh dans ma grande détresse, et il m'a exaucé.
\VS{2}Yahweh, délivre mon âme des lèvres mensongères et de la langue trompeuse.
\VS{3}Que te donne, que te rapporte la langue trompeuse ?
\VS{4}Ce sont des flèches aiguës tirées par un homme puissant et des charbons ardents du genêt\FTNT{Jé. 9:3 ; Ja. 3:5-6.}.
\VS{5}Malheureux que je suis de séjourner à Méschec et de demeurer aux tentes de Kédar !
\VS{6}Assez longtemps mon âme a demeuré auprès de ceux qui haïssent la paix !
\VS{7}Je ne cherche que la paix, mais lorsque j'en parle, ils sont pour la guerre.
\Chap{121}
\TextTitle{Yahweh ne dort ni ne sommeille}
\VerseOne{}Cantique des degrés. J'élève mes yeux vers les montagnes, d'où me viendra le secours.
\VS{2}Mon secours vient de Yahweh qui a fait les cieux et la terre\FTNT{Ps. 124:8.}.
\VS{3}Il ne permettra point que ton pied chancelle, celui qui te garde ne sommeillera point\FTNT{Es. 27:3 ; Pr. 3:23.}.
\VS{4}Voici, il ne sommeille ni ne dort celui qui garde Israël.
\VS{5}Yahweh est celui qui te garde, Yahweh est ton ombre à ta main droite\FTNT{Es. 25:4.}.
\VS{6}Pendant le jour, le soleil ne te frappera point, ni la lune pendant la nuit\FTNT{Es. 49:10. Ap. 7:16.}.
\VS{7}Yahweh te gardera de tout mal, il gardera ton âme.
\VS{8}Yahweh gardera ton départ et ton arrivée, dès maintenant et à jamais\FTNT{De. 28:6.}.
\Chap{122}
\TextTitle{Jérusalem, la ville de Yahweh}
\VerseOne{}Cantique des degrés de David. Je me réjouis à cause de ceux qui me disent : Allons à la maison de Yahweh\FTNT{Ps. 84:1-5.} !
\VS{2}Nos pieds s'arrêtent dans tes portes, ô Jérusalem !
\VS{3}Jérusalem, qui est bâtie comme une ville dont les édifices sont joints ensemble,
\VS{4}à laquelle montent les tribus, les tribus de Yahweh, selon le témoignage d'Israël, pour célébrer le Nom de Yahweh.
\VS{5}Car c'est là qu'ont été posés les trônes pour juger\FTNT{Mt. 19:28.}. Les trônes de la maison de David.
\VS{6}Demandez la paix de Jérusalem ; que ceux qui t'aiment jouissent du repos.
\VS{7}Que la paix soit dans tes murs et la tranquillité dans tes palais.
\VS{8}Pour l'amour de mes frères et de mes amis, je prie maintenant pour ta paix.
\VS{9}A cause de la maison de Yahweh notre Dieu, je fais une requête pour ton bonheur.
\Chap{123}
\TextTitle{Les regards fixés sur Yahweh}
\VerseOne{}Cantique des degrés. J'élève mes yeux vers toi qui habites dans les cieux.
\VS{2}Voici, comme les yeux des serviteurs regardent la main de leurs maîtres, comme les yeux de la servante regardent la main de sa maîtresse, ainsi nos yeux regardent à Yahweh notre Dieu, jusqu'à ce qu'il ait pitié de nous\FTNT{Ps. 25:15.}.
\VS{3}Aie pitié de nous, ô Yahweh ! Aie pitié de nous ! Car nous sommes assez rassasiés de mépris !
\VS{4}Notre âme est assez rassasiée des moqueries des orgueilleux, du mépris des hautains.
\Chap{124}
\TextTitle{Yahweh, le Dieu qui secours et protège son peuple}
\VerseOne{}Cantique des degrés, de David. Sans Yahweh, qui nous protégea, qu'Israël le dise !
\VS{2}Sans Yahweh, qui nous protégea, quand les hommes s'élevèrent contre nous ?
\VS{3}Ils nous auraient engloutis tous vivants quand leur colère s'enflamma contre nous.
\VS{4}Alors les eaux nous auraient submergés, les torrents auraient passé sur notre âme.
\VS{5}Alors les flots impétueux auraient passé sur notre âme.
\VS{6}Béni soit Yahweh qui ne nous a point livrés en proie à leurs dents !
\VS{7}Notre âme s'est échappée comme l'oiseau du filet des oiseleurs ; le filet a été rompu, et nous nous sommes échappés\FTNT{Pr. 6:5.}.
\VS{8}Notre secours est dans le Nom de Yahweh\FTNT{Ac. 4:11-12.} qui a fait les cieux et la terre.
\Chap{125}
\TextTitle{Yahweh entoure tous ceux qui se confient en lui}
\VerseOne{}Cantique des degrés. Ceux qui se confient en Yahweh sont comme la montagne de Sion : Elle ne chancelle point et est affermie pour toujours.
\VS{2}Quant à Jérusalem, il y a des montagnes autour d'elle, ainsi Yahweh entoure son peuple, dès maintenant et à jamais.
\VS{3}Car la verge de la méchanceté ne restera pas sur le lot des justes, de peur que les justes n'étendent leurs mains vers l'iniquité\FTNT{Es. 14:5.}.
\VS{4}Yahweh, répands tes bienfaits sur les bons et sur ceux dont le cœur est droit.
\VS{5}Mais ceux qui s'engagent dans des voies détournées, que Yahweh les fasse marcher avec les ouvriers d'iniquité\FTNT{Mt. 7:23.}. La paix sera sur Israël.
\Chap{126}
\TextTitle{Yahweh, le libérateur}
\VerseOne{}Cantique des degrés. Quand Yahweh ramena les captifs de Sion, nous étions comme ceux qui font un rêve.
\VS{2}Alors notre bouche était remplie de joie, et notre langue de chants de triomphe, alors on disait parmi les nations : Yahweh a fait de grandes choses pour eux !
\VS{3}Yahweh a fait de grandes choses pour nous ; nous sommes dans la joie.
\VS{4}Ô Yahweh ! Ramène nos captifs, comme des ruisseaux dans le midi\FTNT{Os. 6:11 ; Joë. 3:11.} !
\VS{5}Ceux qui sèment avec larmes moissonneront avec chants d'allégresse\FTNT{Ga. 6:9.}.
\VS{6}Celui qui marche en pleurant quand il porte la semence pour la mettre en terre, revient avec des chants d'allégresse quand il porte ses gerbes.
\Chap{127}
\TextTitle{Yahweh, le plus grand architecte}
\VerseOne{}Cantique des degrés, de Salomon. Si Yahweh ne bâtit la maison, ceux qui la bâtissent travaillent en vain ; si Yahweh ne garde la ville, celui qui la garde fait le guet en vain.
\VS{2}C'est en vain que vous vous levez de grand matin, que vous vous couchez tard, et que vous mangez le pain de douleurs ; certes c'est Dieu qui donne du repos à celui qu'il aime\FTNT{Ez. 20:20 ; Mc. 2:27.}.
\VS{3}Voici, les fils sont un héritage donné par Yahweh et le fruit du ventre est une récompense de Dieu\FTNT{Ps. 113:9 ; Ps. 128:3-6.}.
\VS{4}Telles sont les flèches dans la main d'un homme puissant, tels sont les fils de la jeunesse.
\VS{5}Heureux l'homme qui en a rempli son carquois ! Ils ne seront pas honteux quand ils parleront avec leurs ennemis à la porte.
\Chap{128}
\TextTitle{Yahweh assure la paix à celui qui le craint}
\VerseOne{}Cantique des degrés. Heureux tout homme qui craint Yahweh et marche dans ses voies !
\VS{2}Tu jouis du travail de tes mains ; tu es heureux et tu prospères\FTNT{Es. 3:10.}.
\VS{3}Ta femme est dans ta maison comme une vigne qui porte du fruit ; tes fils sont autour de ta table comme des plants d'oliviers.
\VS{4}Voici, certainement ainsi sera béni l'homme qui craint Yahweh.
\VS{5}Yahweh te bénira de Sion et tu verras le bien de Jérusalem tous les jours de ta vie.
\VS{6}Tu verras les fils de tes fils. La paix sera sur Israël.
\Chap{129}
\TextTitle{L'opprimé plus que vainqueur en Yahweh}
\VerseOne{}Cantique des degrés. Qu'Israël dise maintenant : Ils m'ont souvent tourmenté dès ma jeunesse.
\VS{2}Ils m'ont assez opprimé dès ma jeunesse, mais ils ne m'ont pas vaincu.
\VS{3}Des laboureurs ont labouré mon dos, ils y ont tracé de longs sillons.
\VS{4}Yahweh est juste, il a coupé les cordes des méchants.
\VS{5}Qu'ils soient honteux et qu'ils reculent, tous ceux qui haïssent Sion !
\VS{6}Qu'ils soient comme l'herbe des toits qui sèche avant qu'on l'arrache !
\VS{7}Le moissonneur n'en remplit point sa main, ni celui qui lie les gerbes n'en remplit point ses bras ;
\VS{8}et les passants ne disent pas : Que la bénédiction de Yahweh soit sur vous ! Nous vous bénissons au nom de Yahweh !
\Chap{130}
\TextTitle{La rédemption en abondance auprès de Yahweh}
\VerseOne{}Cantique des degrés. Ô Yahweh ! Je t'invoque du fond de l'abîme.
\VS{2}Seigneur, écoute ma voix ! Que tes oreilles soient attentives à la voix de mes supplications !
\VS{3}Yahweh ! si tu prends garde aux iniquités, Seigneur, qui subsistera ?
\VS{4}Mais le pardon se trouve auprès de toi, afin qu'on te craigne\FTNT{Mt. 26:28 ; Ro. 3:24 ; Col. 1:12-14.}.
\VS{5}J'espère en Yahweh, mon âme espère, et j'attends sa parole.
\VS{6}Mon âme attend le Seigneur plus que les sentinelles n'attendent le matin, plus que les sentinelles n'attendent le matin.
\VS{7}Israël, attends-toi à Yahweh, car Yahweh est miséricordieux et la rédemption est auprès de lui en abondance.
\VS{8}Lui-même rachètera Israël de toutes ses iniquités.
\Chap{131}
\TextTitle{Mettre son espoir en Yahweh seul}
\VerseOne{}Cantique des degrés, de David. Ô Yahweh ! Je n'ai ni un cœur qui s'élève ni un regard hautain\FTNT{Pr. 16:5 ; Pr. 6:17.} ; je ne m'occupe pas de choses trop grandes et trop extraordinaires pour moi.
\VS{2}J'ai l'âme calme et tranquille comme un enfant sevré de sa mère ; j'ai l'âme comme un enfant sevré.
\VS{3}Israël attends-toi à Yahweh dès maintenant et à jamais !
\Chap{132}
\TextTitle{Sion, le trône de Yahweh}
\VerseOne{}Cantique des degrés. Ô Yahweh ! Souviens-toi de David et de toute son affliction !
\VS{2}Il a juré à Yahweh et fait ce vœu au puissant de Jacob :
\VS{3}Je n'entrerai pas dans la tente où j'habite, je ne monterai pas sur le lit où je couche,
\VS{4}je ne donnerai pas du sommeil à mes yeux, je ne laisserai pas sommeiller mes paupières,
\VS{5}jusqu'à ce que j'aie trouvé un lieu pour Yahweh, une demeure pour le puissant de Jacob\FTNT{1 Ch. 15:1.}.
\VS{6}Voici, nous avons entendu parler d'elle à Ephrata, nous l'avons trouvée dans les champs de Jaar.
\VS{7}Entrons dans sa demeure, prosternons-nous devant son marchepied.
\VS{8}Lève-toi, ô Yahweh, pour venir à ton lieu de repos, toi et l'arche de ta force\FTNT{No. 10:35-36 ; 2 Ch. 6:41.}.
\VS{9}Que tes sacrificateurs soient revêtus de justice et que tes bien-aimés chantent de joie\FTNT{Es. 11:5 ; Ap. 19:8.} !
\VS{10}Pour l'amour de David, ton serviteur, ne permets pas que ton oint retourne en arrière !
\VS{11}Yahweh a juré la vérité à David, et il ne se rétractera pas, disant : Je mettrai le fruit de tes entrailles\FTNT{2 S. 7:12 ; 1 R. 8:25 ; 2 Ch. 6:16 ; Lu. 1:69 ; Ac. 2:30.} sur ton trône.
\VS{12}Si tes fils gardent mon alliance et mon témoignage que je leur enseignerai, leurs fils aussi seront assis à perpétuité sur ton trône.
\VS{13}Car Yahweh a choisi Sion, il l'a préférée pour être son trône :
\VS{14}Elle est mon lieu de repos à perpétuité, j'y habiterai parce que je l'ai désirée.
\VS{15}Je bénirai abondamment sa nourriture, je rassasierai de pain ses pauvres.
\VS{16}Je revêtirai de salut ses sacrificateurs, et ses bien-aimés chanteront avec des cris de joie.
\VS{17}Je ferai qu'en elle germera une corne à David ; je préparerai une lampe à mon oint,
\VS{18}je revêtirai de honte ses ennemis, et sur lui fleurira son diadème.
\Chap{133}
\TextTitle{La bénédiction dans la communion fraternelle}
\VerseOne{}Cantique des degrés. De David. Voici, oh ! Que c'est une chose bonne et que c'est une chose agréable que des frères demeurent unis ensemble\FTNT{Hé. 13:1 ; Ac. 2:46.} !
\VS{2}C'est comme cette huile précieuse, répandue sur la tête, qui coule sur la barbe d'Aaron\FTNT{Ex. 30:22-30.}, sur le bord de ses vêtements ;
\VS{3}comme la rosée de l'Hermon, celle qui descend sur les montagnes de Sion. Car c'est là que Yahweh a ordonné la bénédiction et la vie, pour l'éternité.
\Chap{134}
\TextTitle{Bénissez Yahweh, vous tous ses serviteurs}
\VerseOne{}Cantique des degrés. Voici, bénissez Yahweh ! Vous tous les serviteurs de Yahweh ! Qui vous tenez toutes les nuits dans la maison de Yahweh !
\VS{2}Elevez vos mains vers le lieu saint ! Et bénissez Yahweh !
\VS{3}Que Yahweh, qui a fait les cieux et la terre, te bénisse de Sion !
\Chap{135}
\TextTitle{La souveraineté de Dieu}
\VerseOne{}Louez le Nom de Yahweh ! Vous serviteurs de Yahweh ! Louez-le !
\VS{2}Vous qui vous tenez dans la maison de Yahweh, dans les parvis de la maison de notre Dieu,
\VS{3}louez Yahweh, car Yahweh est bon ! Chantez son Nom, car il est agréable !
\VS{4}Car Yahweh s'est choisi Jacob et Israël pour sa possession\FTNT{Ex. 19:5 ; De. 7:6 ; Tit. 2:14 ; 1 Pi. 2:9.}.
\VS{5}Certainement, je sais que Yahweh est grand et que notre Seigneur est au-dessus de tous les dieux.
\VS{6}Yahweh fait tout ce qu'il lui plaît, dans les cieux et sur la terre, dans la mer et dans tous les abîmes.
\VS{7}C'est lui qui fait monter les vapeurs des extrémités de la terre ; il fait les éclairs et la pluie ; il tire le vent hors de ses trésors.
\VS{8}C'est lui qui a frappé les premiers-nés d'Egypte, tant des hommes que des bêtes ;
\VS{9}qui a envoyé des prodiges et des miracles au milieu de toi, ô Egypte ! Contre Pharaon et contre tous ses serviteurs ;
\VS{10}qui a frappé plusieurs nations et tué les puissants rois ;
\VS{11}Sihon, roi des Amoréens, et Og, roi de Basan, et ceux de tous les royaumes de Canaan\FTNT{No. 21:33-35 ; De. 3:11.} ;
\VS{12}qui a donné leur pays en héritage, en héritage à Israël son peuple.
\VS{13}Yahweh, ton Nom est pour toujours ! Yahweh, ta mémoire de génération en génération !
\VS{14}Car Yahweh jugera son peuple et se repentira à l'égard de ses serviteurs.
\VS{15}Les dieux des nations ne sont que de l'or et de l'argent, un ouvrage de mains d'homme.
\VS{16}Ils ont une bouche, et ne parlent point ; ils ont des yeux, et ne voient point ;
\VS{17}ils ont des oreilles, et n'entendent point ; il n'y a point de souffle dans leur bouche.
\VS{18}Ils leur ressemblent ceux qui les font, et tous ceux qui s'y confient.
\VS{19}Maison d'Israël, bénissez Yahweh ! Maison d'Aaron, bénissez Yahweh !
\VS{20}Maison des Lévites, bénissez Yahweh ! Vous qui craignez Yahweh, bénissez Yahweh !
\VS{21}Béni soit de Sion Yahweh qui habite dans Jérusalem ! Louez Yahweh !
\Chap{136}
\TextTitle{La bonté de Yahweh demeure à toujours}
\VerseOne{}Célébrez Yahweh, car il est bon, car sa bonté demeure à toujours !
\VS{2}Célébrez le Dieu des dieux, car sa bonté demeure à toujours !
\VS{3}Célébrez le Seigneur des seigneurs, car sa bonté demeure à toujours !
\VS{4}Célébrez celui qui seul fait de grandes merveilles, car sa bonté demeure à toujours !
\VS{5}Celui qui a fait avec intelligence les cieux, car sa bonté demeure à toujours !
\VS{6}Celui qui a étendu la terre sur les eaux, car sa bonté demeure à toujours !
\VS{7}Celui qui a fait les grands luminaires, car sa bonté demeure à toujours !
\VS{8}Le soleil pour dominer sur le jour, car sa bonté demeure à toujours !
\VS{9}La lune et les étoiles pour dominer la nuit, car sa bonté demeure à toujours !
\VS{10}Celui qui a frappé l'Egypte dans leurs premiers-nés, car sa bonté demeure à toujours !
\VS{11}Qui a fait sortir Israël du milieu d'eux, car sa bonté demeure à toujours.
\VS{12}Et cela avec main forte et bras étendu, car sa bonté demeure à toujours !
\VS{13}Il a fendu la Mer Rouge en deux, car sa bonté demeure à toujours !
\VS{14}Il a fait passer Israël par le milieu d'elle, car sa bonté demeure à toujours !
\VS{15}Il a renversé Pharaon et son armée dans la Mer Rouge, car sa bonté demeure à toujours !
\VS{16}Il a conduit son peuple dans le désert, car sa bonté demeure à toujours !
\VS{17}Il a frappé les grands rois, car sa bonté demeure à toujours !
\VS{18}Qui a tué des grands rois, car sa bonté demeure à toujours !
\VS{19}Sihon, roi des Amoréens, car sa bonté demeure à toujours !
\VS{20}Og, roi de Basan, car sa bonté demeure à toujours !
\VS{21}Il a donné leur pays en héritage, car sa bonté demeure à toujours\FTNT{Jos. 12:7.} !
\VS{22}En héritage à Israël son serviteur, car sa bonté demeure à toujours !
\VS{23}Et qui, lorsque nous étions humiliés, s'est souvenu de nous, car sa bonté demeure à toujours !
\VS{24}Il nous a délivrés de la main de nos adversaires, car sa bonté demeure à toujours !
\VS{25}Il donne la nourriture à toute chair, car sa bonté demeure à toujours\FTNT{Ps. 104:21 ; Mt. 6:26 ; Ps. 147:9.} !
\VS{26}Célébrez le Dieu des cieux, car sa bonté demeure à toujours !
\Chap{137}
\TextTitle{Le coeur des captifs}
\VerseOne{}Sur les bords des fleuves de Babylone, nous étions assis et nous pleurions en nous souvenant de Sion.
\VS{2}Nous avions suspendu nos harpes au milieu des saules.
\VS{3}Là, ceux qui nous avaient emmenés en captivité, nous ont demandé des paroles de chants, et nos oppresseurs de la joie, en nous disant : Chantez-nous quelques cantiques de Sion ! Nous avons répondu :
\VS{4}Comment chanterions-nous les cantiques de Yahweh sur une terre étrangère ?
\VS{5}Si je t'oublie, Jérusalem, que ma droite s'oublie elle-même.
\VS{6}Que ma langue soit attachée à mon palais\FTNT{Ez. 3:26.}, si je ne me souviens pas de toi, si je ne fais pas de Jérusalem le sujet de ma réjouissance.
\VS{7}Ô Yahweh, souviens-toi des fils d'Edom, qui dans la journée de Jérusalem disaient : Rasez, rasez jusqu'à ses fondements\FTNT{Jé. 25:15-21 ; Jé. 49:7-8 ; Ez. 25:12 ; La. 4:21 ; Am. 1:11.} !
\VS{8}Fille de Babylone, qui va être détruite, heureux celui qui te rend la pareille de ce que tu nous as fait\FTNT{Jé. 50:15-29 ; Ap. 18:6.} !
\VS{9}Heureux celui qui saisit tes petits enfants et qui les écrase contre le rocher\FTNT{Es. 13:16.} !
\Chap{138}
\TextTitle{La renommée de Yahweh dans les nations}
\VerseOne{}Psaume de David. Je te célèbre de tout mon cœur, je te chante des louanges dans la présence de Dieu.
\VS{2}Je me prosterne dans ton saint temple, et je célèbre ton Nom à cause de ta bonté et de ta fidélité ; car ta renommée s'est accrue par l'accomplissement de ta promesse.
\VS{3}Le jour où je t'ai invoqué, tu m'as exaucé, tu m'as rassuré, tu m'as fortifié d'une nouvelle force en mon âme.
\VS{4}Yahweh ! Tous les rois de la terre te célèbrent, quand ils entendent les paroles de ta bouche.
\VS{5}Ils chantent les voies de Yahweh, car la gloire de Yahweh est grande.
\VS{6}Car Yahweh est haut élevé, il voit les humbles et il reconnaît de loin les orgueilleux.
\VS{7}Quand je marche au milieu de l'adversité, tu me rends la vie, tu avances ta main contre la colère de mes ennemis, et ta droite me délivre.
\VS{8}Yahweh achèvera ce qui me concerne. Yahweh, ta bonté demeure toujours ; tu n'abandonnes pas l'œuvre de tes mains\FTNT{Ph. 1:6.}.
\Chap{139}
\TextTitle{L'omniscience de Yahweh}
\VerseOne{}Psaume de David, donné au chef des chantres. Yahweh, tu me sondes et tu me connais\FTNT{Jé. 12:3 ; Ps. 17:3.}.
\VS{2}Tu sais quand je m'assieds et quand je me lève ; tu discernes de loin ma pensée.
\VS{3}Tu sais quand je marche et quand je me couche ; tu connais parfaitement toutes mes voies.
\VS{4}Avant que la parole soit sur ma langue, voici, ô Yahweh, tu la connais déjà !
\VS{5}Tu m'entoures par derrière et par devant, et tu mets ta main sur moi.
\VS{6}Ta science est trop merveilleuse pour moi, elle est si haut élevée que je ne saurais l'atteindre\FTNT{Job. 42:3 ; Ps. 92:6 ; Ro. 11:33.}.
\VS{7}Où irai-je loin de ton Esprit, et où fuirai-je loin de ta face\FTNT{Jé. 23:24 ; Am. 9:2-4 ; Jon. 1:3.} ?
\VS{8}Si je monte aux cieux, tu y es ; si je me couche dans le scheol, t'y voilà.
\VS{9}Si je prends les ailes de l'aurore et que je demeure à l'extrémité de la mer,
\VS{10}là aussi ta main me conduira et ta droite me saisira.
\VS{11}Si je dis : Au moins les ténèbres me couvriront, la nuit même sera une lumière tout autour de moi.
\VS{12}Même les ténèbres ne me cacheront point de toi, et la nuit resplendira comme le jour, et les ténèbres comme la lumière.
\VS{13}Tu as créé mes reins, tu me couvres du sein de ma mère.
\VS{14}Je te célèbre de ce que je suis une créature redoutée et merveilleuse ; tes œuvres sont merveilleuses, et mon âme le reconnaît très bien.
\VS{15}Mon corps n'était pas caché devant toi lorsque j'ai été fait dans un lieu secret et brodé dans les profondeurs de la terre\FTNT{Ps. 119:73 ; Ec. 11:5.}.
\VS{16}Tes yeux me voyaient quand je n'étais qu'un embryon, et sur ton livre étaient inscrits tous les jours qui m'étaient destinés\FTNT{Ph. 4:3 ; Ap. 3:5 ; Ap. 20:15.}.
\VS{17}Dieu ! Que tes pensées sont précieuses ! Que le nombre en est grand !
\VS{18}Si je les compte, elles sont plus nombreuses que les grains de sable. Je m'éveille et je suis encore avec toi.
\VS{19}Ô Dieu ! Ne tueras-tu pas le méchant ? C'est pourquoi, hommes sanguinaires, retirez-vous loin de moi !
\VS{20}Car ils ont parlé de toi en pensant à quelque méchanceté ; ils ont élevé tes ennemis en mentant.
\VS{21}Yahweh, n'aurais-je point en haine ceux qui te haïssent ; et ne serais-je point irrité contre ceux qui s'élèvent contre toi ?
\VS{22}Je les hais d'une parfaite haine ; ils sont pour moi des ennemis.
\VS{23}Ô Dieu ! Sonde-moi et considère mon cœur ! Eprouve-moi et considère mes discours !
\VS{24}Et regarde si je suis sur une mauvaise voie ; conduis-moi sur la voie de l'éternité.
\Chap{140}
\TextTitle{Yahweh, le protecteur}
\VerseOne{}Psaume de David, donné au chef des chantres. Yahweh, délivre-moi de l'homme méchant, garde-moi de l'homme violent.
\VS{2}Ils méditent des méchancetés dans leur cœur, tous les jours ils complotent des guerres.
\VS{3}Ils aiguisent leur langue comme un serpent, il y a du venin de vipère sous leurs lèvres. Sélah.
\VS{4}Yahweh, garde-moi de la main du méchant, préserve-moi de l'homme violent, de ceux qui méditent de me faire tomber.
\VS{5}Les orgueilleux me tendent un piège et des filets, et ils étendent des rets le long du chemin, ils me dressent des embûches. Sélah.
\VS{6}Je dis à Yahweh : Tu es mon Dieu, Yahweh ! Prête l'oreille à la voix de mes supplications !
\VS{7}Ô Yahweh ! Seigneur ! La force de mon salut ! Tu couvres ma tête au jour de la bataille.
\VS{8}Yahweh n'accorde point au méchant ses désirs ; qu'il n'apporte pas ses méchants desseins, ils s'élèveraient. Sélah.
\VS{9}Quant à la tête de ceux qui m'environnent, que la méchanceté de leurs lèvres les recouvre.
\VS{10}Que des charbons ardents soient jetés sur eux ! Qu'ils tombent sur eux ! Qu'il les fasse tomber dans le feu, et dans des fosses profondes, sans qu'ils se relèvent\FTNT{Pr. 25:21-22 ; Ro. 12:20.} !
\VS{11}Que l'homme à la langue méchante ne soit point affermi sur la terre ; quant à l'homme violent et mauvais, qu'on le chasse jusqu'à ce qu'il soit exterminé.
\VS{12}Je sais que Yahweh fera justice au malheureux et droit aux indigents.
\VS{13}Quoi qu'il en soit, les justes célébreront ton Nom, les hommes droits habiteront devant ta face.
\Chap{141}
\TextTitle{Yahweh, garde-moi du mal !}
\VerseOne{}Psaume de David. Yahweh, je t'invoque, hâte-toi de venir vers moi ; prête l'oreille à ma voix lorsque je crie à toi.
\VS{2}Que ma prière te soit agréable comme l'encens, et l'élévation de mes mains comme l'offrande du soir\FTNT{Ex. 30:1 ; Ap. 5:8 ; Ap. 8:3.}.
\VS{3}Yahweh, mets une garde à ma bouche, garde l'entrée de mes lèvres.
\VS{4}N'incline point mon cœur à des choses mauvaises, au point que je commette quelques méchantes actions par malice, avec les hommes qui font le mal ; et que je ne mange point de leurs délices.
\VS{5}Que le juste me frappe, ce me sera une faveur ; et qu'il me réprimande, ce sera pour moi un baume excellent\FTNT{Pr. 27:6 ; Ec. 7:5.} ; il ne blessera point ma tête ; car ma prière sera pour eux leur calamité.
\VS{6}Que leurs juges soient précipités le long des rochers, et l'on écoutera mes paroles, car elles sont agréables.
\VS{7}Nos os sont dispersés dans la bouche du scheol comme quand on laboure la terre et on fend le bois.
\VS{8}C'est pourquoi, ô Yahweh, Seigneur, mes yeux sont sur toi, je me suis retiré vers toi, n'abandonne point mon âme !
\VS{9}Garde-moi du piège qu'ils m'ont tendu et des filets de ceux qui font le mal.
\VS{10}Que tous les méchants tombent dans leurs filets, jusqu'à ce que je sois passé.
\Chap{142}
\TextTitle{Yahweh, mon refuge}
\VerseOne{}Cantique de David. Prière qu'il fit lorsqu'il était dans la caverne\FTNT{1 S. 24:4.}.
\VS{2}Je crie de ma voix à Yahweh, je supplie de ma voix Yahweh.
\VS{3}Je répands devant lui ma complainte, je déclare mon angoisse devant lui\FTNT{1 S. 1:15 ; La. 2:19.}.
\VS{4}Quand mon esprit est abattu en moi, toi, tu connais mon sentier. Ils me tendent un piège sur le chemin par lequel je marche.
\VS{5}Je contemple à ma droite et je regarde, et il n'y a personne qui me reconnaît ; tout refuge s'évanouit devant moi, il n'y a personne qui prend soin de mon âme.
\VS{6}Yahweh, je crie vers toi ; je dis : Tu es mon refuge, ma part sur la terre des vivants.
\VS{7}Sois attentif à mon cri car je suis devenu très affaibli. Délivre-moi de ceux qui me poursuivent car ils sont plus puissants que moi.
\VS{8}Retire mon âme de sa prison afin que je célèbre ton Nom ! Les justes viendront m'entourer quand tu m'auras fait du bien.
\Chap{143}
\TextTitle{Yahweh, enseigne-moi à faire ta volonté}
\VerseOne{}Psaume de David. Yahweh, écoute ma requête, prête l'oreille à mes supplications ! Exauce-moi dans ta fidélité, réponds-moi à cause de ta justice !
\VS{2}N'entre point en jugement avec ton serviteur, car aucun homme vivant n'est juste devant toi.
\VS{3}Car l'ennemi poursuit mon âme, il foule ma vie par terre ; il me fait habiter dans les ténèbres comme ceux qui sont morts depuis longtemps.
\VS{4}Et mon esprit est abattu au-dedans de moi, mon cœur est épouvanté en mon sein.
\VS{5}Je me souviens des jours anciens, je médite sur toutes tes œuvres, je médite sur l'ouvrage de tes mains\FTNT{Ps. 77:11-13.}.
\VS{6}J'étends mes mains vers toi ; mon âme s'adresse à toi comme une terre desséchée\FTNT{Ps. 28:1 ; Ps. 42:1-3.}. Sélah.
\VS{7}Ô Yahweh, hâte-toi, réponds-moi ! Mon esprit se consume ! Ne me cache point ta face au point que je devienne semblable à ceux qui descendent dans la fosse !
\VS{8}Fais-moi entendre dès le matin ta miséricorde, car je me confie en toi ; fais-moi connaître le chemin par lequel je dois marcher, car j'ai élevé mon cœur vers toi\FTNT{Ps. 25:1.}.
\VS{9}Yahweh, délivre-moi de mes ennemis, car je me suis réfugié auprès de toi !
\VS{10}Enseigne-moi à faire ta volonté, car tu es mon Dieu ! Que ton bon Esprit me conduise sur la voie de la droiture\FTNT{Jn. 16:13.} !
\VS{11}Yahweh, rends-moi la vie pour l'amour de ton Nom ! Retire mon âme de la détresse à cause de ta justice !
\VS{12}Et selon la bonté que tu as pour moi, retranche mes ennemis ! Détruis tous ceux qui tiennent mon âme oppressée, parce que je suis ton serviteur !
\Chap{144}
\TextTitle{Se confier en Yahweh, le Rocher}
\VerseOne{}Psaume de David. Béni soit Yahweh, mon rocher\FTNT{Voir commentaire en Es. 8:13-17.} qui exerce mes mains au combat et mes doigts à la bataille,
\VS{2}qui déploie sa bonté envers moi, qui est ma forteresse, ma haute retraite, mon libérateur\FTNT{Es. 59:20-21 ; Ro. 11:26.}, mon bouclier\FTNT{Ep. 6:16.}, mon refuge\FTNT{Ps. 91 ; Mt. 11:28-30.}, qui m'assujettit mon peuple.
\VS{3}Ô Yahweh ! Qu'est-ce que l'homme pour que tu aies soin de lui\FTNT{Ps. 8:5 ; Job. 7:17 ; Hé. 2:6-7.} ? Le fils de l'homme mortel pour que tu prennes garde à lui ?
\VS{4}L'homme est semblable à la vanité, ses jours sont comme une ombre qui passe\FTNT{Ps. 102:12 ; Job. 14:1-2 ; Ec. 6:12.}.
\VS{5}Yahweh abaisse tes cieux et descends ! Touche les montagnes et qu'elles soient fumantes\FTNT{Es. 63:19 ; Ps. 18:7-8.}.
\VS{6}Lance les éclairs et disperse mes ennemis ! Lance tes flèches et mets-les en déroute !
\VS{7}Etends tes mains d'en haut ; sauve-moi et délivre-moi des grandes eaux, de la main des fils de l'étranger,
\VS{8}dont la bouche profère le mensonge, et dont la droite est une droite trompeuse !
\VS{9}Ô Dieu ! Je chanterai un cantique nouveau ! Je te célèbrerai sur le luth à dix cordes !
\VS{10}Toi qui donnes la délivrance aux rois et qui délivres de l'épée meurtrière David, ton serviteur.
\VS{11}Retire-moi et délivre-moi de la main des fils de l'étranger, dont la bouche profère le mensonge et dont la droite est une droite trompeuse ;
\VS{12}afin que nos fils soient comme des plantes qui croissent dans leur jeunesse et nos filles comme des pierres angulaires taillées pour l'ornement d'un palais.
\VS{13}Que nos greniers soient pleins, fournissant toute espèce de provision ; que nos troupeaux multiplient par milliers, même par dix milliers dans nos rues.
\VS{14}Que nos bœufs soient chargés de graisse. Qu'il n'y ait ni brèche, ni sortie dans nos murailles, ni cri dans nos places.
\VS{15}Heureux le peuple pour qui il en est ainsi ! Heureux le peuple dont Yahweh est le Dieu !
\Chap{145}
\TextTitle{Louange à Yahweh pour tout ce qu'il est}
\VerseOne{}Psaume de louange, composé par David. [Aleph.] Mon Dieu, mon roi, je t'exalterai et je bénirai ton Nom à toujours, et à perpétuité !
\VS{2}[Beth.] Je te bénirai chaque jour, et je louerai ton Nom à toujours, et à perpétuité !
\VS{3}[Guimel.] Yahweh est grand et très digne de louanges, il n'est pas possible de sonder sa grandeur.
\VS{4}[Daleth.] Que chaque génération célèbre tes œuvres et publie tes hauts faits !
\VS{5}[He.] Je dirai la splendeur glorieuse de ta majesté et de tes faits merveilleux.
\VS{6}[Vav.] On parlera de ta puissance redoutable, et je raconterai ta grandeur.
\VS{7}[Zayin.] Ils proclameront le souvenir de ton immense bonté, et ils raconteront avec chants de triomphe ta justice.
\VS{8}[Heth.] Yahweh est miséricordieux et compatissant, lent à la colère et grand en bonté.
\VS{9}[Teth.] Yahweh est bon envers tous et ses compassions sont au-dessus de toutes ses œuvres.
\VS{10}[Yod.] Yahweh, toutes tes œuvres te célébreront, et tes fidèles te béniront.
\VS{11}[Kaf.] Ils diront la gloire de ton règne, et ils proclameront ta puissance
\VS{12}[Lamed.] pour faire connaître aux fils de l'homme ta puissance et la splendeur glorieuse de ton règne.
\VS{13}[Mem.] Ton règne est un règne de tous les siècles et ta domination subsiste dans tous les âges.
\VS{14}[Samech.] Yahweh soutient tous ceux qui tombent et redresse tous ceux qui sont courbés\FTNT{Ps. 146:8.}.
\VS{15}[Ayin.] Les yeux de tous les animaux s'attendent à toi et tu leur donnes leur nourriture en leur temps.
\VS{16}[Pe.] Tu ouvres ta main et tu rassasies à souhait toute créature vivante.
\VS{17}[Tsade.] Yahweh est juste dans toutes ses voies et plein de bonté dans toutes ses œuvres\FTNT{Da. 4:37.}.
\VS{18}[Qof.] Yahweh est près de tous ceux qui l'invoquent, de tous ceux qui l'invoquent avec vérité\FTNT{Ps. 34:18.}.
\VS{19}[Resh.] Il accomplit le désir de ceux qui le craignent, il entend leur cri et les délivre.
\VS{20}[Shin.] Yahweh garde tous ceux qui l'aiment, mais il exterminera tous les méchants.
\VS{21}[Tav.] Ma bouche racontera la louange de Yahweh, et toute chair bénira le Nom de sa sainteté à toujours, et à perpétuité\FTNT{Ps. 103:1.}.
\Chap{146}
\TextTitle{La fidélité de Yahweh dure à toujours}
\VerseOne{}Louez Yahweh ! Mon âme, loue Yahweh !
\VS{2}Je louerai Yahweh durant ma vie, je chanterai mon Dieu tant que je vivrai !
\VS{3}Ne vous confiez pas aux grands, ni en aucun fils de l'homme qui ne peuvent délivrer.
\VS{4}Son esprit s'en va et l'homme retourne dans sa terre, et ce même jour ses desseins périssent.
\VS{5}Heureux celui qui a pour secours le Dieu de Jacob, qui met son espoir en Yahweh, son Dieu !
\VS{6}Il a fait les cieux et la terre, la mer et tout ce qui s'y trouve. Il garde la vérité à toujours !
\VS{7}Il fait droit aux opprimés, il donne du pain aux affamés ; Yahweh délie ceux qui sont liés\FTNT{Jn. 11:43-44.}.
\VS{8}Yahweh ouvre les yeux des aveugles\FTNT{Les miracles de Jésus-Christ confirment sa divinité (Es. 35:4-6 ; Lu. 7:19-23).} ; Yahweh redresse ceux qui sont courbés\FTNT{Lu. 13:11-13.} ; Yahweh aime les justes.
\VS{9}Yahweh protège les étrangers, il soutient l'orphelin et la veuve, mais il renverse la voie des méchants.
\VS{10}Yahweh règne éternellement. Ô Sion ! Ton Dieu subsiste d'âge en âge. Louez Yahweh !
\Chap{147}
\TextTitle{Yahweh aime ceux qui le craignent et qui s'attendent à sa bonté}
\VerseOne{}Louez Yahweh ! Car il est beau de chanter à notre Dieu ! Car il est doux et bienséant de le louer !
\VS{2}Yahweh est celui qui bâtit Jérusalem ; il rassemblera ceux d'Israël qui sont dispersés çà et là.
\VS{3}Il guérit ceux qui ont le cœur brisé et il bande leurs plaies\FTNT{Ex. 15:26 ; De. 32:39 ; Job. 5:18.}.
\VS{4}Il compte le nombre des étoiles, il les appelle toutes par leur nom.
\VS{5}Notre Seigneur est grand, puissant par sa force, son intelligence n'a point de limites.
\VS{6}Yahweh soutient les malheureux, mais il abaisse les méchants jusqu'à terre.
\VS{7}Chantez à Yahweh avec reconnaissance ! Célébrez notre Dieu avec la harpe !
\VS{8}Il couvre les cieux de nuées, il prépare la pluie pour la terre ; il fait germer l'herbe sur les montagnes.
\VS{9}Il donne la nourriture au bétail et aux petits du corbeau qui crient.
\VS{10}Il ne prend point plaisir dans la force du cheval ; il ne fait point cas des jambes de l'homme.
\VS{11}Yahweh aime ceux qui le craignent, ceux qui s'attendent à sa bonté.
\VS{12}Jérusalem, loue Yahweh ! Sion, loue ton Dieu !
\VS{13}Car il a affermi les barres de tes portes, il a béni tes fils au milieu de toi.
\VS{14}Il rend la paix à son territoire et te rassasie du meilleur froment.
\VS{15}C'est lui qui envoie ses ordres sur la terre, sa parole court avec rapidité\FTNT{Es. 55:10-11.}.
\VS{16}C'est lui qui donne la neige comme des flocons de laine et qui répand la gelée blanche comme de la cendre.
\VS{17}C'est lui qui lance sa glace comme par morceaux, qui peut résister devant son froid ?
\VS{18}Il envoie sa parole, et il les fond ; il fait souffler son vent, et les eaux coulent\FTNT{Ps. 135:7.}.
\VS{19}Il déclare ses paroles à Jacob, ses statuts et ses ordonnances à Israël\FTNT{Ps. 78:5.}.
\VS{20}Il n'a pas agi de même pour toutes les nations, c'est pourquoi elles ne connaissent point ses ordonnances. Louez Yahweh !
\Chap{148}
\TextTitle{La création loue son Dieu}
\VerseOne{}Louez Yahweh ! Louez des cieux Yahweh ! Louez-le dans les lieux élevés !
\VS{2}Louez-le, vous tous anges ! Louez-le, vous toutes ses armées !
\VS{3}Louez-le, vous, soleil et lune ! Louez-le, vous toutes, étoiles lumineuses !
\VS{4}Louez-le, vous, cieux des cieux ! Et vous, eaux qui êtes au-dessus des cieux !
\VS{5}Qu'ils louent le Nom de Yahweh ! Car il a commandé et ils ont été créés\FTNT{Ge. 1:3-6 ; Jé. 31:35.}.
\VS{6}Il les a établis à perpétuité et à toujours ; il a donné des lois, et il ne les violera pas\FTNT{Ps. 104:5 ; Ps. 119:91 ; Job. 14:5.}.
\VS{7}De la terre, louez Yahweh ! Louez-le, monstres marins et tous les abîmes !
\VS{8}Feu et grêle, neige et brouillard, vent impétueux qui exécutez ses ordres,
\VS{9}montagnes et toutes les collines, arbres fruitiers et tous les cèdres,
\VS{10}bêtes sauvages et tout le bétail, reptiles et oiseaux ailés,
\VS{11}rois de la terre et tous les peuples, princes et tous les juges de la terre,
\VS{12}ceux qui sont à la fleur de leur âge, et les vierges aussi, les vieillards, et les jeunes gens !
\VS{13}Qu'ils louent le Nom de Yahweh ! Car son Nom seul est haut élevé ! Sa majesté est au-dessus de la terre et des cieux.
\VS{14}Il a relevé la force de son peuple, sujet de louange pour tous ses fidèles, pour les fils d'Israël, du peuple qui est près de lui. Louez Yahweh !
\Chap{149}
\TextTitle{Adorons Yahweh}
\VerseOne{}Louez Yahweh ! Chantez à Yahweh un cantique nouveau et louez-le dans l'assemblée de ses fidèles !
\VS{2}Qu'Israël se réjouisse en celui qui l'a fait ! Et que les fils de Sion soient dans l'allégresse à cause de leur Roi\FTNT{Ps. 100:3 ; Za. 9:9 ; Mt. 21:5.} !
\VS{3}Qu'ils louent son Nom avec des danses ! Qu'ils le chantent avec le tambourin et la harpe !
\VS{4}Car Yahweh prend plaisir à son peuple, il glorifie les pauvres en les délivrant.
\VS{5}Que les fidèles se réjouissent dans la gloire, qu'ils poussent des cris de joie sur leur couche.
\VS{6}Les louanges de Dieu sont dans leur bouche et les épées affilées à deux tranchants dans leur main,
\VS{7}pour se venger des nations, pour châtier les peuples,
\VS{8}pour lier leurs rois avec des chaînes, et les plus honorables parmi eux avec des ceps de fer,
\VS{9}pour exercer sur eux le jugement qui est écrit ! Cet honneur est pour tous ses fidèles. Louez Yahweh !
\Chap{150}
\TextTitle{Que tout ce qui respire loue Yahwe !}
\VerseOne{}Louez Yahweh ! Louez Dieu à cause de sa sainteté ! Louez-le dans l'étendue de toute sa puissance !
\VS{2}Louez-le pour ses hauts faits ! Louez-le selon la grandeur de sa magnificence !
\VS{3}Louez-le au son du shofar ! Louez-le avec le luth et la harpe !
\VS{4}Louez-le avec le tambour et avec des danses ! Louez-le avec des instruments à cordes et le chalumeau !
\VS{5}Louez-le avec les cymbales sonores ! Louez-le avec les cymbales de cri de joie !
\VS{6}Que tout ce qui respire loue Yahweh ! Louez Yahweh !
\PPE{}
\end{multicols}

%\clearpage\ShortTitle{Proverbes}\BookTitle{Proverbes}\BFont
\noindent\hrulefill
{\footnotesize
\textit{
\bigskip
{\centering{}
\\Auteurs : Salomon, Agur et Lemuel
\\(Heb. : Mishlei)
\\Signification : Paraboles
\\Thème : La sagesse
\\Date de rédaction : 10\up{ème} siècle av J.-C.\\}
}
%\bigskip
\textit{
\\Le mot « proverbe » désigne un genre littéraire appliqué à une sentence, une énigme, une comparaison, un oracle, une
parabole ou une parole de sagesse. Le livre des proverbes est donc un recueil de sentences dont la majeure partie
est attribuée à Salomon. Véritable collection de maximes morales et spirituelles, la sagesse, la crainte de Dieu et la
tempérance en sont les thèmes principaux.
%\bigskip
\\Ce livre met en évidence l'opposition entre la voie du méchant et celle du juste, entre la femme étrangère et la femme
vertueuse, entre l'orgueil et l'humilité, entre la sagesse et la folie, entre le chemin de la vie et celui de la mort. Comme il était coutume au Moyen-Orient, ces écrits s'adressaient particulièrement aux jeunes gens en vue de leur instruction.\bigskip
}
}
\par\nobreak\noindent\hrulefill
\begin{multicols}{2}
\Chap{1}
\TextTitle{[But du livre : connaître la sagesse}
\VerseOne{}Les Proverbes de Salomon, fils de David et roi d'Israël.
\VS{2}Pour connaître la sagesse et l'instruction, pour discerner les paroles d'intelligence ;
\VS{3}pour recevoir une leçon de bon sens, de justice, de jugement et d'équité.
\VS{4}Pour donner du discernement aux simples, aux jeunes gens de la connaissance et de la réflexion.
\VS{5}Le sage écoutera, et il augmentera son savoir, et l'homme intelligent acquerra de la prudence ;
\VS{6}afin d'entendre les paraboles et les énigmes ; les discours des sages et leurs énigmes.
\TextTitle{Le fondement de la sagesse : la crainte de Dieu}
\VS{7}La crainte de Yahweh\FTNT{Pr. 8:13.} est la principale de la science ; mais les fous méprisent la sagesse et l'instruction.
\VS{8}Mon fils, écoute l'instruction de ton père, et n'abandonne pas l'enseignement\FTNT{Loi vient de Torah (instruction, enseignement, direction etc.).} de ta mère.
\VS{9}Car se sont des grâces enfilées ensemble autour de ta tête, et des colliers autour de ton cou.
\VS{10}Mon fils, si les pécheurs veulent t'attirer, ne t'y accorde pas.
\VS{11}S'ils disent : Viens avec nous, dressons des embûches pour tuer; épions secrètement l'innocent, quoiqu'il ne nous en ai point donné de sujet aucune raison.
\VS{12}Engloutissons-les tout vifs, comme le scheol ; et tout entiers, comme ceux qui descendent dans la fosse ;
\VS{13}nous trouverons toutes sortes de biens précieux, nous remplirons nos maisons de butin ;
\VS{14}tu auras ta part avec nous, il n'y aura qu'une bourse pour nous tous.
\VS{15}Mon fils, ne te mets point en chemin avec eux ; retires ton pied de leur sentier ;
\VS{16}parce que leurs pieds courent au mal, et se hâtent pour répandre le sang\FTNT{Es. 59:7.}.
\VS{17}Car c'est en vain qu'on jette le filet devant les yeux de tout Baal ailé\FTNT{Ec. 10:20.} ;
\VS{18}ainsi ceux-ci dressent des embûches contre le sang de ceux-là et épient secrètement leurs vies.
\VS{19}Tel est le train de tout homme convoiteux de gain déshonnête, qui ôte la vie à ceux qui y sont adonnés.
\TextTitle{La sagesse crie}
\VS{20}La souveraine sagesse crie hautement au-dehors, elle fait retentir sa voix dans les rues.
\VS{21}Elle crie dans les carrefours, là où on fait le plus de bruit, aux entrées des portes, elle prononce ses paroles dans la ville :
\VS{22}Stupides, dit-elle, jusqu'à quand aimerez-vous la stupidité ? Et jusqu'à quand les moqueurs prendront-ils plaisir à la moquerie, et les haïront-ils la connaissance ?
\VS{23}Etant repris par moi, convertissez-vous ; voici, je vous donnerai de mon Esprit en abondance, et je vous ferai connaître mes paroles.
\VS{24}Parce que je crie, et que vous refusez d'entendre ; parce que j'étends ma main, et que personne n'y prend garde ;
\VS{25}et parce que vous rejetez tout mon conseil, et que vous n'avez accepté je vous reprenne ;
\VS{26}moi aussi je rirai quand vous serez dans le malheur, je me moquerai quand la terreur viendra sur vous.
\VS{27}Quand votre effroi surviendra comme une ruine, et que votre calamité viendra comme un tourbillon; vous enveloppera comme un tourbillon ; quand la détresse et l'angoisse viendront  sur vous ;
\VS{28}alors on criera vers moi, mais je ne répondrai point ; on me cherchera de grand matin, mais on ne me trouvera pas\FTNT{De. 31:18 ; Job. 35:12.}.
\VS{29}Parce qu'ils auront haï la connaissance, et qu'ils n'auront point choisi la crainte de Yahweh.
\VS{30}Ils n'ont point aimé mon conseil ; ils ont rejeté toutes mes réprimandes.
\VS{31}Qu'ils mangent donc le fruit de leur voie, et qu'ils se rassasient de leurs conseils.
\VS{32}Car l'égarement des sots les tue, et la prospérité des insensés les perd.
\VS{33}Mais celui qui m'écoute habitera en sécurité et sera tranquille, sans être effrayé d'aucun mal.
\Chap{2}
\TextTitle{La sagesse nous libère du mal}
\VerseOne{}Mon fils, si tu reçois mes paroles, et que tu gardes précieusement en toi mes commandements,
\VS{2}si tu rends ton oreille attentive à la sagesse, et que tu inclines ton cœur à l'intelligence ;
\VS{3}si tu appelles à toi la sagesse, et que tu adresses ta voix à l'intelligence,
\VS{4}si tu la cherches comme de l'argent, et si tu la recherches soigneusement comme des trésors,
\VS{5}alors tu connaîtras la crainte de Yahweh, et tu trouveras la connaissance de Dieu.
\VS{6}Car Yahweh donne la sagesse, et de sa bouche procède la connaissance et l'intelligence.
\VS{7}Il réserve le salut pour ceux qui sont droits, et il est le bouclier de ceux qui marchent dans l'intégrité,
\VS{8}pour garder les sentiers de la justice ; il gardera la voie de ses bien-aimés.
\VS{9}Alors tu comprendras la justice, le jugement, l'équité, et tout bon chemin.
\VS{10}Si la sagesse vient dans ton coeur, et si la connaissance est agréable à ton âme ;
\VS{11}la réflexion veillera sur toi, et l'intelligence te gardera,
\VS{12}pour te délivrer du mauvais chemin, et de l'homme qui tient de mauvais discours ,
\VS{13}de ceux qui abandonnent les voies de la droiture pour marcher dans les chemins ténébreux,
\VS{14}qui sont joyeux de mal faire, et qui se réjouissent dans la perversité des méchants.
\VS{15}Eux dont les sentiers sont tortueux, et qui dans leur conduite vont de travers.
\VS{16}Afin qu'il te délivre de la femme étrangère\FTNT{La femme étrangère est la prostituée ou l'esprit de Jézabel qui séduit les hommes. Voir Pr. 6:24 ; Pr. 7:5.}, et de la femme d'autrui, dont les paroles sont flatteuses ;
\VS{17}qui abandonne l'ami de sa jeunesse et qui oublie l'alliance de son Dieu.
\VS{18}Car sa maison penche vers la mort, et son chemin mène vers les morts.
\VS{19}Pas un de ceux qui vont vers elle n'en retourne, ni ne reprend les sentiers de la vie.
\VS{20}Ainsi tu marcheras dans la voie des gens de bien, et tu garderas les sentiers des justes.
\VS{21}Car ceux qui sont droits habiteront la terre, les hommes intègres y demeureront.
\VS{22}Mais les méchants seront retranchés de la terre, et ceux qui agissent perfidement seront arrachés.
\Chap{3}
\TextTitle{La sagesse bénit et protège}
\VerseOne{}Mon fils, ne mets pas en oubli mon enseignement, et que ton cœur garde mes commandements.
\VS{2}Car ils t'apportent de longs jours et des années de vie et de paix.
\VS{3}Que la bonté et la vérité ne t'abandonnent pas : Lie-les à ton cou, et écris-les sur la table de ton coeur ;
\VS{4}et tu trouveras la grâce et la prudence au yeux de Dieu et des hommes.
\VS{5}Confie-toi de tout ton coeur en Yahweh et ne t'appuie point sur ton intelligence.
\VS{6}Considère-le dans toutes tes voies et il dirigera tes sentiers.
\VS{7}Ne sois point sage à tes yeux ; crains Yahweh, et détourne-toi du mal.
\VS{8}Ce sera la guérison de ton nombril et un rafraîchissement pour tes os.
\VS{9}Honore Yahweh avec tes biens et les prémices de tout ton revenu\FTNT{De. 12:6.} :
\VS{10}Alors tes greniers seront remplis d'abondance, et tes cuves regorgeront de vin nouveau.
\VS{11}Mon fils, ne rebute pas l'instruction de Yahweh, et ne te fâche pas de ce qu'il te reprend.
\VS{12}Car Yahweh châtie\FTNT{Hé. 12:4-11.} celui qu'il aime, comme un père le fils auquel il prend plaisir.
\VS{13}Heureux l'homme qui a trouvé la sagesse, et l'homme qui possède l'intelligence !
\VS{14}Car le trafic qu'on peut faire d'elle est meilleur que le trafic l'argent; et le profit qu'on en tire est meilleur que l'or fin.
\VS{15}Elle est plus précieuse que les perles, et toutes tes choses désirables ne la valent point.
\VS{16}Il y a de longs jours dans sa main droite, des richesses et de la gloire en sa gauche.
\VS{17}Ses voies sont des voies agréables, et tous ses sentiers ne sont que paix.
\VS{18}Elle est l'arbre de vie pour ceux qui l'embrassent ; et tous ceux qui la tiennent sont heureux\FTNT{Ge. 2:9 ; Ap. 22:2.}.
\VS{19}Yahweh a fondé la terre par la sagesse, il a disposé les cieux par l'intelligence.
\VS{20}C'est par sa science que les abîmes se sont ouverts, et que les nuages distillent la rosée.
\VS{21}Mon fils, que ces enseignements ne s'écartent point de devant tes yeux ; garde la sagesse et la réflexion :
\VS{22}Elles seront la vie de ton âme et l'ornement de ton cou.
\VS{23}Alors tu marcheras avec assurance dans ton chemin, et ton pied ne bronchera pas.
\VS{24}Si tu te couches, tu seras sans crainte, et quand tu seras couché ton sommeil sera doux.
\VS{25}Ne crains ni une terreur soudaine, ni la ruine des méchants, quand elle arrivera.
\VS{26}Car Yahweh sera ton assurance, et il gardera ton pied de toute embûche.
\VS{27}Ne retiens pas le bien à ceux à qui il est dû, quand il est au pouvoir de ta main de le faire\FTNT{Ga. 6:10.}.
\VS{28}Ne dis pas à ton prochain : Va, et reviens, demain je te donnerai ! Quand tu as de quoi donner.
\VS{29}Ne médite pas le mal contre ton prochain, lorsqu'il demeure tranquillement près de toi.
\VS{30}Ne conteste pas sans motif avec quelqu'un, à moins qu'il ne t'ait causé quelque tort\FTNT{Ro. 12:18.}.
\VS{31}Ne porte pas envie à l'homme violent, et ne choisis aucune de ses voies.
\VS{32}Car celui qui va de travers est en abomination à Yahweh ; mais son intimité est pour ceux qui sont justes.
\VS{33}La malédiction de Yahweh est dans la maison du méchant ; mais il bénit la demeure des justes.
\VS{34}Certes il se moque des moqueurs, mais il fait grâce à ceux qui s'humilient.
\VS{35}Les sages hériteront la gloire ; mais la honte élève des insensés.
\Chap{4}
\TextTitle{Instructions et conseils d'un père}
\VerseOne{}Ecoutez, mes fils, l'instruction du père, et soyez attentifs pour connaître l'intelligence.
\VS{2}Car je vous donne une bonne doctrine, ne rejetez donc pas mon enseignement.
\VS{3}J'ai été un fils pour mon père. Un fils tendre et unique auprès de ma mère.
\VS{4}Il m'a enseigné, et m'a dit : Que ton coeur retienne mes paroles ; garde mes commandements et tu vivras.
\VS{5}Acquiers la sagesse, acquiers l'intelligence ; n'oublie pas les paroles de ma bouche, et ne t'en détourne pas.
\VS{6}Ne l'abandonne point, et elle te gardera ; aime-la, et elle te protégera.
\VS{7}La principale chose c'est la sagesse ; donc acquiers la sagesse ; et sur toutes tes acuisitions, acquiers la prudence.
\VS{8}Exalte-la, et elle t'élèvera ; elle te glorifiera quand tu l'auras embrassée.
\VS{9}Elle posera sur ta tête une couronne de grâce, et elle t'ornera d'un magnifique diadème.
\VS{10}Ecoute, mon fils, et reçois mes paroles, ainsi les années de ta vie te seront multipliées.
\VS{11}Je t'ai enseigné le chemin de la sagesse, et je t'ai conduit dans les sentiers de la droiture\FTNT{Ps. 23:3.}.
\VS{12}Quand tu y marcheras, ton pas ne sera pas gêné ; et si tu cours, tu ne chancelleras pas\FTNT{Ps. 121:3.}.
\VS{13}Embrasse l'instruction, ne la lâche pas ; garde-la ; car elle est ta vie.
\VS{14}N'entre pas dans le sentier des méchants, et ne marche pas dans la voie des hommes mauvais.
\VS{15}Détourne-t'en, ne passe pas par là, détourne-t'en, et passe outre.
\VS{16}Car ils ne dormiraient pas, s'ils n'avaient fait quelque mal, et le sommeil leur serait ôté, s'ils n'avaient fait tomber quelqu'un.
\VS{17}Parce qu'ils mangent le pain de méchanceté, et qu'ils boivent le vin de la violence.
\VS{18}Mais le sentier des justes est comme la lumière resplendissante, dont l'éclat augmente jusqu'à ce que le jour soit dans sa perfection.
\VS{19}La voie des méchants est comme l'obscurité ; ils n'aperçoivent pas ce qui les fera tomber.
\VS{20}Mon fils, sois attentif à mes paroles, incline ton oreille à mes discours.
\VS{21}Qu'ils ne s'écarte pas de tes yeux ; garde-les dans le fond de ton coeur.
\VS{22}Car ils sont la vie pour ceux qui les trouvent, et la santé de tout le corps de chacun d'eux.
\VS{23}Garde ton coeur de tout ce dont il faut se garder ; car de lui procèdent les sources de la vie\FTNT{Mt 12:35 ; Mt. 15:18-19.}.
\VS{24}Eloigne de toi la perversité de la bouche et la dépravation des lèvres.
\VS{25}Que tes yeux regardent droit et que tes paupières dirigent ton chemin devant toi.
\VS{26}Pèse le chemin de tes pieds, et que toutes tes voies soient bien stables.
\VS{27}Ne te tourne ni à droite ni à gauche ; détourne ton pied du mal.
\Chap{5}
\TextTitle{[Se garder de l'immoralité]}
\VerseOne{}Mon fils, sois attentif à ma sagesse, incline ton oreille à mon intelligence ;
\VS{2}afin que tu gardes mes avis, et que tes lèvres conservent la connaissance.
\VS{3}Car les lèvres de l'étrangère distillent des rayons de miel, et son palais est plus doux que l'huile.
\VS{4}Mais ce qui en provient est amère comme de l'absinthe, et aigu comme une épée à deux tranchants.
\VS{5}Ses pieds descendent à la mort, ses pas atteignent le scheol.
\VS{6}Afin que tu ne balance pas sur le chemin de la vie car, ses chemins en sont écartés; tu ne le connaîtras pas.
\VS{7}Maintenant donc, fils, écoutez-moi, et ne vous détournez pas des paroles de ma bouche.
\VS{8}Eloigne ton chemin de la femme étrangère et n'approche pas de l'entrée de sa maison.
\VS{9}De peur que tu ne donnes ton honneur à d'autres, et tes années à un homme cruel.
\VS{10}De peur que les étrangers ne se rassasient de tes biens, et que le fruit de ton travail ne soit dans la maison d'un étranger.
\VS{11}De peur que tu ne gémisses quand tu seras près de ta fin, quand ta chair et ton corps seront consumés ;
\VS{12}et que tu ne dises : Comment donc ai-je pu haïr la correction, et comment mon coeur a-t-il dédaigné les réprimandes ?
\VS{13}Et comment n'ai-je point obéi à la voix de ceux qui m'instruisaient, et n'ai-je point incliné mon oreille à ceux qui m'enseignaient ?
\VS{14}Peu s'en est fallu que je n'aie été dans toute sorte de mal, au milieu du peuple et de l'assemblée.
\VS{15}Bois des eaux de ta citerne et de ce qui coule du milieu de ton puits ;
\VS{16}que tes sources se répandent dehors, et les ruisseaux d'eau sur les rues ;
\VS{17}qu'elles soient à toi seul, et non aux étrangers avec toi.
\VS{18}Que ta source soit bénie, et réjouis-toi de la femme de ta jeunesse,
\VS{19}comme d'une biche des amours, et d'une chevrette gracieuse ; que ses mamelles te rassasient en tout temps, et sois continuellement épris de son amour.
\VS{20}Et pourquoi, mon fils, irais-tu errant après l'étrangère et embrasserais-tu le sein de l'inconnue ?
\VS{21}Vu que les voies de l'homme sont devant les yeux de Yahweh et qu'il pèse toutes ses voies\FTNT{Jé 16:17 ; Hé. 4:13.}.
\VS{22}Les iniquités du méchant l'attraperont, et il sera retenu par les cordes de son péché.
\VS{23}Il mourra faute d'instruction et il s'égarera par l'excès de sa folie.
\Chap{6}
\TextTitle{Recommandations diverses}
\VerseOne{}Mon fils, si tu t'es porté caution pour ton prochain, si tu as engagé ta main pour un étranger,
\VS{2}tu es enlacé par les paroles de ta bouche, tu es pris par les paroles de ta bouche.
\VS{3}Mon fils, fais maintenant ceci, et dégage-toi, puisque tu es tombé entre les mains de ton intime ami, va, prosterne-toi, et importune tes amis.
\VS{4}Ne donne point de sommeil à tes yeux et ne laisse point sommeiller tes paupières.
\VS{5}Dégage-toi comme la gazelle de la main du chasseur, et comme l'oiseau de la main de l'oiseleur.
\VS{6}Va, paresseux, vers la fourmi, regarde ses voies, et sois sage.
\VS{7}Elle n'a ni chef, ni directeur, ni gouverneur,
\VS{8}et cependant elle prépare en été son pain, et amasse durant la moisson de quoi manger.
\VS{9}Paresseux, jusqu'à quand resteras-tu couché ? Quand te lèveras-tu de ton sommeil ?
\VS{10}Un peu de sommeil, dis-tu, un peu d'assoupissement, un peu croiser les mains afin de rester couché ;
\VS{11}et ta pauvreté viendra comme un voyageur, et ta disette comme un soldat.
\VS{12}Celui qui marche, la fausseté dans sa bouche, est un homme de Bélial\FTNT{1 S. 2:12.}, un homme inique.
\VS{13}Il cligne des yeux, parle du pied, enseigne de ses doigts.
\VS{14}Il y a la perversité dans son cœur, il machine du mal en tout temps, il fait naître des querelles.
\VS{15}C'est pourquoi sa calamité viendra subitement, il sera subitement brisé, il n'y aura point de guérison.
\VS{16}Il y a six choses que Yahweh hait, et il y en a sept qui sont en abomination à son âme ;
\VS{17}savoir, les yeux hautains\FTNT{Ps. 101:5.}, la langue mensongère\FTNT{Ps. 120:2-3.}, les mains qui répandent le sang innocent\FTNT{Es. 1:15.},
\VS{18}le coeur qui médite des projets iniques\FTNT{Ps. 36:5.}, les pieds qui se hâtent de courir au mal\FTNT{Es. 59:7.},
\VS{19}le faux témoin qui profère des mensonges\FTNT{Ps. 27:12.}, et celui qui sème des querelles entre les frères\FTNT{Jud. 1:16-19.}.
\VS{20}Mon fils, garde le commandement de ton père, et n'abandonne pas l'enseignement de ta mère ;
\VS{21}attache-les continuellement à ton coeur, lie-les à ton cou.
\VS{22}Quand tu marcheras, il te conduira ; et quand tu te coucheras, il te gardera ; et quand tu te réveilleras, il s'entretiendra avec toi.
\VS{23}Car le commandement est une lampe ; et l'enseignement une lumière\FTNT{Ps. 119:105.} ; et les réprimandes propres à instruire sont le chemin de la vie.
\VS{24}Ils te préserveront de la mauvaise femme, de la langue doucereuse de l'étrangère.
\VS{25}Ne la convoite pas dans ton coeur pour sa beauté, et ne te laisse pas prendre par ses yeux\FTNT{Mt. 5:28.}.
\VS{26}Car pour l'amour de la femme prostituée on est réduit à un morceau de pain, et la femme adultère chasse après l'âme précieuse de l'homme.
\VS{27}Un homme peut-il prendre du feu dans son sein, sans que ses habits brûlent ?
\VS{28}Un homme marchera-t-il sur des charbons ardents, sans que ses pieds en soient brûlés ?
\VS{29}Il en est de même pour celui qui va vers la femme de son prochain ; quiconque la touchera ne restera pas impuni.
\VS{30}On ne méprise pas un voleur, s'il vole pour satisfaire son âme quand il a faim ;
\VS{31}si on le trouve, il rendra sept fois autant, il donnera tout ce qu'il a dans sa maison.
\VS{32}Mais celui qui commet un adultère avec une femme est dépourvu de sens ; et celui qui le fera, détruira son âme.
\VS{33}Il trouvera des plaies et de l'ignominie, et son opprobre ne sera pas effacé.
\VS{34}Car la jalousie d'un mari est une fureur, il n'épargnera pas l'adultère au jour de la vengeance.
\VS{35}Il n'aura égard à aucune rançon, et il n'acceptera rien, quand même tu multiplierais les présents.
\Chap{7}
\TextTitle{Mise en garde contre la femme prostituée}
\VerseOne{}Mon fils, observe mes paroles, et garde avec toi mes commandements.
\VS{2}Garde mes commandements, et tu vivras, garde mes enseignements comme la prunelle de tes yeux\FTNT{Lé. 18:5.}.
\VS{3}Lie-les sur tes doigts, écris-les sur la table de ton coeur.
\VS{4}Dis à la sagesse : Tu es ma soeur ; et appelle l'intelligence, ton amie.
\VS{5}Afin qu'elles te préservent de la femme étrangère, de l'étrangère qui emploie des paroles doucereuses.
\VS{6}Comme je regardais de la fenêtre de ma maison à travers mon treillis,
\VS{7}je vis parmi les stupides, et je remarquai parmi les jeunes gens un jeune homme dépourvu de sens.
\VS{8}Il passait dans la rue, près de l'angle où se tenait une de ces femmes, et qui suivait le chemin de sa maison,
\VS{9}au crépuscule, au soir du jour, au milieu de la nuit et de l'obscurité.
\VS{10}Et voici, il fut abordé par une femme, vêtue en tenue de prostituée, et pleine de ruse dans le cœur.
\VS{11}Elle était bruyante et rebelle, ses pieds ne restaient point dans sa maison ;
\VS{12}tantôt dehors, tantôt sur les places, elle était aux aguets à chaque coin de rue.
\VS{13}Elle le saisit, et l'embrassa ; et avec un visage effronté, lui dit :
\VS{14}J'ai chez moi des sacrifices d'offrande de paix ; j'ai aujourd'hui accompli mes voeux.
\VS{15}C'est pourquoi je suis sortie à ta rencontre pour chercher ton visage, et je t'ai trouvé.
\VS{16}J'ai orné mon lit de couvertures, d'étoffes de fil d'Egypte.
\VS{17}J'ai parfumé ma couche de myrrhe, d'aloès et de cinnamome.
\VS{18}Viens, enivrons-nous de plaisir jusqu'au matin, réjouissons-nous en amours.
\VS{19}Car mon mari n'est point à la maison, il est parti pour un voyage lointain.
\VS{20}Il a pris un sac d'argent dans sa main, il ne reviendra à la maison qu'à la nouvelle lune.
\VS{21}Elle l'a fait détourner par beaucoup de douces paroles, et l'a attiré par la flatterie de ses lèvres.
\VS{22}Il s'en alla aussitôt après elle, comme un boeuf qui va à la boucherie, comme le fou qu'on lie pour être châtié ;
\VS{23}jusqu'à ce que la flèche lui ait transpercé le foie ; comme l'oiseau qui se hâte vers le filet, sans savoir que c'est au prix de sa vie.
\VS{24}Maintenant donc, fils, écoutez-moi, et soyez attentifs aux paroles de ma bouche.
\VS{25}Que ton coeur ne se détourne pas vers les voies d'une telle femme, ne t' égare pas dans ses sentiers.
\VS{26}Car elle a fait tomber plusieurs blessés à mort, et tous ceux qu'elle a tués sont nombreux.
\VS{27}Sa maison est le chemin du scheol, qui descend vers les demeures de la mort.
\Chap{8}
\TextTitle{[La sagesse préférable aux richesses]}
\VerseOne{}La sagesse ne crie-t-elle pas ? Et l'intelligence ne fait-elle pas entendre sa voix ?
\VS{2}Elle s'est présentée sur le sommet des lieux élevés, sur le chemin, aux carrefours.
\VS{3}Elle crie près des portes, devant la ville, à l'entrée des portes,
\VS{4}ô vous ! Hommes de qualité, je vous appelle ; et ma voix s'adresse aussi aux fils des hommes.
\VS{5}Vous stupides, apprenez le discernement, et vous tous, devenez intelligents de coeur.
\VS{6}Écoutez, car je dirai des choses importantes : Et j'ouvrirai mes lèvres pour enseigner des choses droites.
\VS{7}Parce que ma bouche proclame la vérité, et mes lèvres ont en horreur le mensonge.
\VS{8}Tous les discours de ma bouche sont selon la justice, il n'y a rien en eux de faux, ni de déformé.
\VS{9}Ils sont tous clairs à l'homme intelligent, et droits pour ceux qui ont trouvé la connaissance.
\VS{10}Recevez mon instruction plutôt que de l'argent, la connaissance à l'or le plus précieux.
\VS{11}Car la sagesse vaut mieux que les perles, et tout ce qu'on pourrait souhaiter ne la vaut pas\FTNT{Ps. 19:11 ; Ps. 119:127 ; Job. 28:18.}.
\VS{12}Moi, la Sagesse, j'habite avec le discernement, et je possède la connaissance de la réflexion.
\VS{13}La crainte de Yahweh c'est la haine du mal. Je hais l'orgueil et l'arrogance, la voie du mal, et la bouche perverse.
\VS{14}A moi appartiennent le conseil et le succès ; je suis l'intelligence, à moi appartient la force.
\VS{15}Par moi règnent les rois, et par moi les princes décrètent ce qui est juste.
\VS{16}Par moi gouvernent les seigneurs, les princes, et tous les juges de la terre.
\VS{17}J'aime ceux qui m'aiment ; et ceux qui me cherchent soigneusement me trouveront\FTNT{Mt. 7:7 ; Lu. 11:9 ; Jn 14:23-24.}.
\VS{18}Avec moi sont la richesse et la gloire, les biens durables et la justice.
\VS{19}Mon fruit est meilleur que le fin or, même que l'or raffiné ; et mon revenu est meilleur que l'argent choisi.
\VS{20}Je marche dans le chemin de la justice, au milieu des sentiers de la droiture ;
\VS{21}pour donner des biens en héritage à ceux qui m'aiment, et pour remplir leurs trésors.
\VS{22}Yahweh m'a acquise dès le commencement de ses voies, avant ses œuvres les plus anciennes.
\VS{23}J'ai été déclarée princesse depuis l'éternité, dès le commencement, avant l'origine de la terre.
\VS{24}J'ai été engendrée lorsqu'il n'y avait point encore d'abîmes, ni de sources chargées d'eaux.
\VS{25}Avant que les montagnes soient affermies, avant que les collines existent, j'ai été engendrée.
\VS{26}Lorsqu'il n'avait pas encore fait la terre et les campagnes, et le commencement de la poussière du monde habitable.
\VS{27}Lorsqu'il disposa les cieux, j'étais là ; lorsqu'il traça un cercle à la surface de l'abîme ;
\VS{28}lorsqu'il fixa les nuages en haut ; et que les sources de l'abîme jaillirent avec force ;
\VS{29}lorsqu'il donna une limite à la mer, pour que les eaux ne franchissent pas les bords ; lorsqu'il posa les fondements de la terre,
\VS{30}j'étais à l'œuvre auprès de lui, je faisais ses délices tous les jours, et toujours j'étais en joie en sa présence.
\VS{31}Je me réjouissais dans la partie habitable de sa terre, trouvant mes délices avec les fils de l'homme.
\VS{32}Maintenant donc, mes fils, écoutez-moi : Heureux sont ceux qui observent mes voies.
\VS{33}Ecoutez l'instruction, et soyez sages, et ne la rejetez point.
\VS{34}Ô ! Heureux est l'homme qui m'écoute, qui veille chaque jour à mes portes. Et qui monte la garde aux montants de mes portes !
\VS{35}Car celui qui me trouve a trouvé la vie, et obtient la faveur de Yahweh.
\VS{36}Mais celui qui pèche contre moi nuit à son âme ; tous ceux qui me haïssent aiment la mort.
\Chap{9}
\TextTitle{[La sagesse, source de vie]}
\VerseOne{}La Souveraine Sagesse a bâti sa maison, elle a taillé ses sept colonnes.
\VS{2}Elle a apprêté sa viande, elle a mêlé son vin ; elle a aussi dressé sa table.
\VS{3}Elle a envoyé ses servantes, elle crie du haut des lieux les plus élevés de la ville, disant : 
\VS{4}Que celui qui est stupide, entre ici ; et elle dit à ceux qui sont dépourvus de sens :
\VS{5}Venez, mangez de mon pain, et buvez du vin que j'ai mêlé.
\VS{6}Abandonnez la stupidité, et vous vivrez ; et marchez droit dans la voie de l'intelligence.
\VS{7}Celui qui instruit le moqueur, en reçoit de l'ignominie ; et celui qui reprend le méchant en reçoit une tache.
\VS{8}Ne reprends point le moqueur, de crainte qu'il ne te haïsse ; reprends le sage, et il t'aimera\FTNT{Ps. 141:5.}.
\VS{9}Donne l'instruction au sage, et il deviendra encore plus sage ; enseigne le juste, et il croîtra en science.
\VS{10}Le commencement de la sagesse est la crainte de Yahweh\FTNT{Ps. 19:10.} ; et la connaissance des saints, c'est l'intelligence.
\VS{11}Car tes jours se multiplieront par moi, et les années de vie augmenteront.
\VS{12}Si tu es sage, tu es sage pour toi-même ; si tu es moqueur, tu en porteras seul la peine.
\VS{13}La femme folle est bruyante, stupide et elle ne connaît rien.
\VS{14}Et elle s'assied à la porte de sa maison sur un siège, dans les lieux élevés de la ville ;
\VS{15}pour appeler les passants qui vont droit leur chemin, disant :
\VS{16}Que celui qui est stupide entre ici ! Elle dit à celui qui est dépourvu de sens :
\VS{17}Les eaux dérobées sont douces, et le pain pris en secret est agréable.
\VS{18}Et il ne sait pas que là sont les défunts, et que ceux qu'elle a conviés sont dans le scheol.
\Chap{10}
\TextTitle{[La justice s'oppose à la méchanceté]}
\VerseOne{}Proverbes de Salomon. Le fils sage réjouit son père, mais le fils insensé est l'ennui de sa mère.
\VS{2}Les trésors de méchanceté ne profitent pas, mais la justice délivre de la mort.
\VS{3}Yahweh ne laisse pas l'âme du juste avoir faim, mais il repousse au loin l'avidité des méchants.
\VS{4}Celui qui agit d'une main nonchalante s'appauvrit, mais la main des diligents enrichit.
\VS{5}L'enfant prudent amasse en été, mais celui qui dort durant la moisson est un enfant qui fait honte.
\VS{6}Les bénédictions seront sur la tête du juste, mais la violence couvrira la bouche des méchants.
\VS{7}La mémoire du juste est en bénédiction\FTNT{Ps 112:6.}, mais la réputation des méchants tombe en pourriture.
\VS{8}Celui qui est sage de coeur reçoit les commandements, mais celui qui est insensé des lèvres, tombera.
\VS{9}Celui qui marche dans l'intégrité marche avec assurance, mais celui qui pervertit ses voies, sera connu.
\VS{10}Celui qui cligne de l'oeil cause du chagrin, et celui qui a les lèvres insensées sera renversé.
\VS{11}La bouche du juste est une source de vie, mais la cruauté couvre la bouche des méchants.
\VS{12}La haine excite les querelles, mais la charité couvre toutes les fautes\FTNT{1 Pi 4:8.}.
\VS{13}La sagesse se trouve sur les lèvres de l'homme intelligent, mais la verge est pour le dos de celui qui est dépourvu de sens.
\VS{14}Les sages tiennent la connaissance en réserve, mais la bouche de l'insensé est une ruine prochaine.
\VS{15}Les biens du riche sont la ville de sa force, mais la pauvreté des misérables est leur ruine.
\VS{16}L'oeuvre du juste est pour la vie, mais le revenu du méchant est pour le péché.
\VS{17}Celui qui garde l'instruction est dans le chemin de la vie, mais celui qui néglige la correction s'y égare.
\VS{18}Celui qui dissimule la haine a des lèvres menteuses, et celui qui répand la calomnie est un insensé.
\VS{19}Dans la multitude de paroles le péché ne manque pas, mais celui qui retient ses lèvres est prudent.
\VS{20}La langue du juste est un argent de choix, mais le coeur des méchants est bien peu de chose.
\VS{21}Les lèvres du juste en instruisent plusieurs, mais les insensés mourront faute de sens.
\VS{22}La bénédiction de Yahweh est celle qui enrichit, et il n'y ajoute aucune peine.
\VS{23}C'est comme un jeu à un insensé de pratiquer l'infamie, mais la sagesse appartient à l'homme intelligent.
\VS{24}Ce que redoute le méchant, c'est ce qui lui arrive ; mais Dieu accorde aux justes ce qu'ils désirent.
\VS{25}Comme le tourbillon passe, ainsi le méchant n'est plus ; mais le juste est un fondement perpétuel.
\VS{26}Ce qu'est le vinaigre aux dents et la fumée aux yeux, tel est le paresseux à ceux qui l'envoient.
\VS{27}La crainte de Yahweh augmente les jours, mais les années des méchants sont raccourcies.
\VS{28}L'espérance des justes n'est que joie, mais l'espérance des méchants périra.
\VS{29}La voie de Yahweh est le refuge de l'homme intègre, mais elle est la ruine pour ceux qui pratiquent l'iniquité.
\VS{30}Le juste ne sera jamais ébranlé, mais les méchants ne demeureront pas sur la terre.
\VS{31}La bouche du juste produit la sagesse, mais la langue perverse sera retranchée.
\VS{32}Les lèvres du juste connaissent ce qui est agréable ; mais la bouche des méchants n'est que perversité.
\Chap{11}
\TextTitle{[La justice s'oppose à la méchanceté (suite)]}
\VerseOne{}La fausse balance est une abomination à Yahweh, mais le poids juste lui est agréable\FTNT{Lé 19:35-36 ; De. 25:13-16.}.
\VS{2}Quand l'orgueil vient, la honte vient aussi ; mais la sagesse est avec ceux qui sont modestes.
\VS{3}L'intégrité des hommes droits les conduit, mais la perversité des perfides les détruit.
\VS{4}Les richesses ne servent à rien au jour de la colère, mais la justice délivre de la mort.
\VS{5}La justice de l'homme intègre rend droite sa voie, mais le méchant tombe par sa méchanceté.
\VS{6}La justice des hommes droits les délivre, mais les perfides sont pris par leur méchanceté.
\VS{7}Quand l'homme méchant meurt, son espoir périt ; et l'espérance des hommes iniques périt.
\VS{8}Le juste est délivré de la détresse, et le méchant y entre à sa place.
\VS{9}Par sa bouche l'impie corrompt son prochain, mais les justes en sont délivrés par la connaissance.
\VS{10}La ville se réjouit quand les justes sont heureux, et quand les méchants périssent, c'est un triomphe.
\VS{11}La ville est élevée par la bénédiction des hommes droits, mais elle est renversée par la bouche des méchants.
\VS{12}Celui qui méprise son prochain est dépourvu de sens, mais l'homme prudent se tait.
\VS{13}Celui qui va rapportant, révèle les secrets, mais celui qui a l'esprit qui supporte les paroles, les couvre.
\VS{14}Le peuple tombe par faute de prudence, mais la délivrance est dans la multitude de conseillers.
\VS{15}Celui qui se porte garant pour un étranger en souffrira, et celui qui hait le cautionnement est assuré.
\VS{16}La femme gracieuse obtient de l'honneur, et les hommes robustes obtiennent les richesses.
\VS{17}L'homme doux fait du bien à son âme, mais le cruel trouble sa chair.
\VS{18}Le méchant fait une oeuvre qui le trompe, mais la récompense est assurée à celui qui sème la justice\FTNT{Os. 10:12.}.
\VS{19}Ainsi la justice conduit à la vie, mais celui qui poursuit le mal aboutit à sa mort.
\VS{20}Ceux qui ont le cœur pervers sont en abomination à Yahweh, mais ceux qui sont intègres dans leurs voies lui sont agréables.
\VS{21}De main en main le méchant ne demeurera point impuni, mais la race des justes sera délivrée.
\VS{22}Une belle femme qui se détourne de la raison est comme un anneau d'or au nez d'un pourceau.
\VS{23}Le souhait des justes n'est que le bien, mais l'attente des méchants c'est l'indignation.
\VS{24}Tel, qui donne libéralement, devient plus riche ; et tel qui épargne à l'excès ne fait que s'appauvrir.
\VS{25}Celui qui bénit sera engraisssé ; et celui qui arrose abondamment sera lui-même arrosé.
\VS{26}Sera maudit du peuple, celui qui cache le froment, mais la bénédiction est sur la tête de celui qui le vend.
\VS{27}Qui recherche le bien cherche la faveur, mais le mal arrive à qui le recherche.
\VS{28}Celui qui se confie dans ses richesses tombera, mais les justes verdiront comme le feuillage\FTNT{Ps. 1:3 ; Jé 17: 8.}.
\VS{29}Celui qui ne gouverne pas sa maison avec ordre, aura le vent pour héritage, et le fou sera le serviteur de celui qui a le coeur sage.
\VS{30}Le fruit du juste est un arbre de vie, et celui qui gagne les âmes est sage.
\VS{31}Voici, le juste reçoit sur la terre sa rétribution, combien plus le méchant et le pécheur la recevront-ils ?
\Chap{12}
\TextTitle{[La justice s'oppose à la méchanceté (suite)]}
\VerseOne{}Celui qui aime la correction aime la connaissance, mais celui qui hait la réprimande est un stupide.
\VS{2}L'homme de bien obtient la faveur de Yahweh, mais Yahweh condamne l'homme qui a des mauvaises pensées.
\VS{3}L'homme ne sera point affermi par la méchanceté, mais la racine des justes ne sera point ébranlée.
\VS{4}La femme vertueuse est la couronne de son mari\FTNT{Pr. 31:10.}, mais celle qui fait honte est comme la pourriture dans ses os.
\VS{5}Les pensées des justes ne sont que jugement, mais les conseils des méchants ne sont que fraude.
\VS{6}Les paroles des méchants ne tendent qu'à dresser des embûches pour répandre le sang, mais la bouche des hommes droits les délivrera.
\VS{7}Les méchants sont renversés, et ils ne sont plus, mais la maison des justes se maintiendra.
\VS{8}L'homme est estimé en raison de sa prudence, mais celui qui a le coeur pervers est l'objet du mépris.
\VS{9}Mieux vaut l'homme qui ne fait pas cas de lui-même, bien qu'il ait des serviteurs, que celui qui se glorifie, et qui manque de pain.
\VS{10}Le juste a égard à la vie de sa bête, mais les entrailles des méchants sont cruelles.
\VS{11}Celui qui cultive son champ sera rassasié de pain, mais celui qui court après des futilités est dépourvu de sens.
\VS{12}Ce que le méchant désire, est un filet des hommes mauvais, mais la racine des justes donnera son fruit.
\VS{13}Il y a dans le péché des lèvres un piège pernicieux, mais le juste sortira de la détresse.
\VS{14}L'homme sera rassasié de biens par le fruit de sa bouche, et on rendra à l'homme la rétribution de ses mains.
\VS{15}La voie de l'insensé est droite à son opinion, mais celui qui écoute le conseil est sage.
\VS{16}Quand à l'insensé, sa colère est révélée le jour même, mais l'homme bien avisé couvre son ignominie.
\VS{17}Celui qui prononce des choses véritables rend un témoignage juste, mais le faux témoin fait des rapports trompeurs.
\VS{18}Il y a tel homme dont les paroles blessent comme des pointes d'épée, mais la langue des sages apporte la guérison.
\VS{19}La lèvre véridique est affermie pour toujours, mais la fausse langue n'est que pour un moment\FTNT{Ps. 52: 6-7.}.
\VS{20}Il y a de la tromperie dans le coeur de ceux qui méditent le mal, mais il y a de la joie pour ceux qui conseillent la paix.
\VS{21}Il n'arrivera aucun outrage aux justes, mais les méchants seront remplis de mal.
\VS{22}Les fausses lèvres sont une abomination à Yahweh\FTNT{Ap. 22:15.}, mais ceux qui agissent fidèlement lui sont agréables.
\VS{23}L'homme bien avisé cache sa connaissance, mais le coeur des insensés publie la folie.
\VS{24}La main des diligents dominera, mais la main paresseuse sera tributaire.
\VS{25}Le chagrin qui est au cœur de l'homme, l'accable ; mais la bonne parole le réjouit.
\VS{26}Le juste a plus de reste que son voisin, mais la voie des méchants les égare.
\VS{27}L'homme paresseux ne rôtit point son gibier ; mais les biens précieux de l'homme sont au diligent.
\VS{28}La vie est dans le chemin de la justice, et la voie de son sentier ne tend point à la mort.
\Chap{13}
\TextTitle{[La justice s'oppose à la méchanceté (suite)]}
\VerseOne{}Un fils sage écoute l'instruction de son père, mais le moqueur n'écoute pas la réprimande\FTNT{Ps. 1:1.}.
\VS{2}L'homme mange du bien par le fruit de sa bouche, mais l'âme de ceux qui agissent perfidement mangent l'injustice.
\VS{3}Celui qui garde sa bouche, garde son âme ; mais celui qui ouvre à tout propos ses lèvres, tombera en ruine\FTNT{Ps. 39:2.}.
\VS{4}L'âme du paresseux a des désirs qu'il ne peut satisfaire, mais l'âme des diligents sera engraissée.
\VS{5}Le juste hait la parole mensongère, mais elle rend le méchant odieux et le fait tomber dans la confusion.
\VS{6}La justice garde celui qui est intègre dans sa voie, mais la méchanceté renversera celui qui s'égare.
\VS{7}Tel fait le riche et n'a rien du tout, tel fait le pauvre et a de grandes fortunes.
\VS{8}Les richesses d'un homme servent de rançon pour sa vie, mais le pauvre n'entend pas des réprimandes.
\VS{9}La lumière des justes remplit de joie, mais la lampe des méchants s'éteint.
\VS{10}L'orgueil ne produit que querelle, mais la sagesse est avec ceux qui écoutent les conseils.
\VS{11}Les richesses provenues de la fraude seront diminuées, mais celui qui amasse peu à peu les augmentera.
\VS{12}Un espoir différé fait languir le cœur, mais un désir accompli est comme un arbre de vie.
\VS{13}Celui qui méprise la parole périra à cause d'elle, mais celui qui craint le commandement en sera récompensé.
\VS{14}L'enseignement du sage est une source de vie, pour se détourner des pièges de la mort.
\VS{15}Le bon sens donne de la grâce ; mais la voie de ceux qui agissent perfidement est raboteuse.
\VS{16}Tout homme bien avisé agira avec connaissance, mais l'insensé fera l'étalage de sa folie\FTNT{Da.11:32}.
\VS{17}Le méchant messager tombe dans le mal, mais l'ambassadeur fidèle apporte la guérison.
\VS{18}La pauvreté et l'ignominie arrivent à celui qui rejette l'instruction, mais celui qui garde la réprimande est honoré.
\VS{19}Le souhait accompli est une chose douce à l'âme, mais se détourner du mal est une abomination aux insensés.
\VS{20}Celui qui marche avec les sages deviendra sage, mais le compagnon des insensés sera accablé.
\VS{21}Le mal poursuit les pécheurs, mais le bien sera rendu aux justes.
\VS{22}L'homme de bien laissera de quoi hériter aux fils de ses fils, mais les richesses du pécheur sont réservées aux justes.
\VS{23}Il y a beaucoup à manger dans les terres défrichées des pauvres, mais il y a tel qui est consumé faute de règles.
\VS{24}Celui qui épargne sa verge hait son fils, mais celui qui l'aime se hâte de le châtier.
\VS{25}Le juste mangera jusqu'à être rassasié à son souhait, mais le ventre des méchants aura la disette.
\Chap{14}
\TextTitle{[La justice s'oppose à la méchanceté (suite)]}
\VerseOne{}Toute femme sage bâtit sa maison, mais la folle la ruine de ses mains.
\VS{2}Celui qui marche dans la droiture craint Yahweh, mais celui dont les voies sont perverses le méprise.
\VS{3}La verge d'orgueil est dans la bouche de l'insensé, mais les lèvres des sages les garderont.
\VS{4}Où il n'y a point de boeuf, la grange est vide ; et l'abondance du revenu provient de la force du boeuf.
\VS{5}Le témoin véritable ne ment jamais, mais le faux témoin avance volontiers des mensonges.
\VS{6}Le moqueur cherche la sagesse et ne la trouve pas, mais la connaissance est aisée à trouver pour l'homme intelligent.
\VS{7}Eloigne-toi de l'homme insensé, puisque tu n'as pas trouvé sur ses lèvres la connaissance.
\VS{8}La sagesse d'un homme avisé est de connaître les règles de sa voie, mais la folie des insensés est la tromperie.
\VS{9}Les insensés se moquent du péché, mais parmi les hommes droits se trouve la bienveillance.
\VS{10}Le cœur d'un chacun connaît l'amertume de son âme, et un autre ne saurait partager sa joie.
\VS{11}La maison des méchants sera abolie, mais la tente des hommes droits fleurira.
\VS{12}Il y a telle voie qui semble droite à l'homme, mais dont l'issue sont les voies de la mort.
\VS{13}Même en riant le coeur sera triste, et la joie finit par l'ennui.
\VS{14}Celui qui a un cœur hypocrite, sera rassasié de ses voies ; mais l'homme de bien de ce qui est en lui.
\VS{15}Le simple croit à toute parole ; mais l'homme bien avisé considère ses pas.
\VS{16}Le sage craint et se retire du mal, mais l'insensé se met en colère et est confiant.
\VS{17}Celui qui est prompt à la colère agit follement\FTNT{Ps. 37:8.}, et l'homme plein de ruse est haï.
\VS{18}Les naïfs hériteront la folie ; mais les prudents seront couronnés de connaissance.
\VS{19}Les malins seront humiliés devant les bons, et les méchants, devant les portes du juste.
\VS{20}Le pauvre est haï même de son ami, mais les amis du riche sont en grand nombre.
\VS{21}Celui qui méprise son prochain commet un péché, mais celui qui a pitié des pauvres affligés est heureux.
\VS{22}Ceux qui méditent le mal ne s'égarent-ils pas ? Mais la bonté et la vérité sont pour ceux qui méditent le bien.
\VS{23}En tout travail il y a quelque profit, mais les vains discours ne tournent qu'à la disette.
\VS{24}Les richesses des sages leur sont comme une couronne, mais la stupidité des insensés est toujours stupidité.
\VS{25}Le témoin fidèle délivre les âmes, mais celui qui prononce des mensonges est trompeur.
\VS{26}En la crainte de Yahweh il y a une ferme assurance, et une retraite pour ses fils.
\VS{27}La crainte de Yahweh est une source de vie pour se détourner des pièges de la mort.
\VS{28}La gloire d'un roi, c'est la multitude du peuple, mais quand le peuple manque, c'est la ruine du prince.
\VS{29}Celui qui est lent à la colère a une grande intelligence, mais celui qui est prompt à s'emporter excite la folie.
\VS{30}Un coeur sain est la vie de la chair, mais l'envie est la pourriture des os.
\VS{31}Celui qui fait tort au pauvre déshonore celui qui l'a fait, mais celui qui a pitié de l'indigent honore Yahweh\FTNT{De. 24:11 ; Ps. 107:41.}.
\VS{32}Le méchant est chassé par sa malice, mais le juste trouve un refuge même dans sa mort.
\VS{33}La sagesse repose au coeur de l'homme intelligent, et elle est même reconnue au milieu des insensés.
\VS{34}La justice élève une nation, mais le péché est l'ignominie des peuples.
\VS{35}Le roi prend plaisir au serviteur prudent, mais son indignation sera contre celui qui lui fait honte.
\Chap{15}
\TextTitle{[La justice s'oppose à la méchanceté (suite)]}
\VerseOne{}La réponse douce apaise la fureur ; mais la parole douloureuse excite la colère
\VS{2}La langue des sages se réjouit de la connaissance, mais la bouche des insensés profère la sottise.
\VS{3}Les yeux de Yahweh sont en tous lieux, observant les méchants et les bons.
\VS{4}La langue qui corrige le prochain est comme l'arbre de vie, mais celle où il y a de la perversité est comme une brèche dans l'esprit.
\VS{5}L'insensé méprise l'instruction de son père, mais celui qui prend garde à la réprimande agit avec prudence.
\VS{6}Il y a un grand trésor dans la maison du juste, mais il y a du trouble dans les revenus du méchant.
\VS{7}Les lèvres des sages répandent partout la connaissance, mais le coeur des insensés ne fait pas ainsi.
\VS{8}Le sacrifice des méchants est en abomination à Yahweh, mais la requête des hommes droits lui est agréable.
\VS{9}La voie du méchant est en abomination à Yahweh, mais il aime celui qui poursuit soigneusement la justice.
\VS{10}Le châtiment est fâcheux à celui qui quitte le droit chemin, mais celui qui hait d'être repris, mourra.
\VS{11}Le schéol et le gouffre sont devant Yahweh ; combien plus les coeurs des fils des hommes !
\VS{12}Le moqueur n'aime pas qu'on le reprenne, et il ne va pas vers les sages.
\VS{13}Le cœur joyeux rend le visage beau, mais l'esprit est abattu par l'ennui du cœur.
\VS{14}Le cœur de l'homme prudent cherche la science ; mais la bouche des insensés se repaît de folie.
\VS{15}Tous les jours de l'affligé sont mauvais, mais quand on a le coeur gai, c'est un festin perpétuel.
\VS{16}Un peu de bien vaut mieux avec la crainte de Yahweh, qu'un grand trésor avec lequel il y a du trouble\FTNT{Ps. 37:16.}.
\VS{17}Mieux vaut un repas d'herbes où il y a de l'amitié, qu'un repas de boeuf bien gras où il y a de la haine.
\VS{18}L'homme furieux excite la querelle, mais l'homme lent à la colère apaise la dispute.
\VS{19}La voie du paresseux est comme une haie d'épines, mais le chemin des hommes droits est aplani.
\VS{20}Un fils sage réjouit le père, et un homme insensé méprise sa mère.
\VS{21}La stupidité est la joie de celui qui est dépourvu de sens, mais un homme prudent dresse ses pas au chemin de la droiture.
\VS{22} Les résolutions deviennent inutiles où il n'y a point de conseil ; mais il y a de la fermeté dans la multitude des conseillers.
\VS{23}L'homme a de la joie dans les réponses de sa bouche ; et combien est bonne une parole dite en son temps !
\VS{24}Le chemin de la vie élève l'homme prudent, afin qu'il se détourne du scheol qui est en bas.
\VS{25}Yahweh renverse la maison des orgueilleux, mais il affermit la borne de la veuve.
\VS{26}Les pensées du malin sont en abomination à Yahweh, mais celles de ceux qui sont purs sont des paroles agréables à ses yeux.
\VS{27}Celui qui est entièrement adonné au gain déshonnête trouble sa maison, mais celui qui hait les présents vivra.
\VS{28}Le coeur du juste médite ce qu'il doit répondre, mais la bouche des méchants profère des choses mauvaises.
\VS{29}Yahweh est loin des méchants, mais il exauce la requête des justes.
\VS{30}La clarté des yeux réjouit le coeur ; et la bonne renommée fortifie les os.
\VS{31}L'oreille qui écoute la correction qui donne la vie habite parmi les sages.
\VS{32}Celui qui rejette l'instruction a en dédain son âme, mais celui qui écoute la réprimande s'acquiert du sens.
\VS{33}La crainte de Yahweh enseigne la sagesse, et l'humilité précède la gloire\FTNT{Ps. 19:10.}.
\Chap{16}
\TextTitle{[La justice s'oppose à la méchanceté (suite)]}
\VerseOne{}Les préparations du cœur sont à l'homme, mais le discours réponse de la langue est de par Yahweh.
\VS{2}Chacune des voies de l'homme lui semble pure à ses yeux; mais Yahweh pèse les esprits.
\VS{3}Recommande tes affaires à Yahweh, et tes pensées seront bien ordonnées.
\VS{4}Yahweh a fait toutes choses pour lui-même ; et même le méchant pour le jour de l'affliction.
\VS{5}Yahweh a en abomination tout homme hautain de coeur ; assurément, il ne demeurera pas impuni.
\VS{6}Il y aura propitiation de l'iniquité par la miséricorde et la vérité ; on se détourne du mal par la crainte de Yahweh. 
\VS{7}Quand Yahweh prend plaisir aux voies d'un homme, il apaise\FTNT{Apaiser vient de shalom qui signifie : être dans une alliance de paix, être en paix, apaiser, vivre dans la paix etc.} envers lui même ses ennemis.
\VS{8}Il vaut mieux un peu de bien avec justice, qu'un gros revenu là où on n'a pas de droit.
\VS{9}Le cœur de l'homme médite sur sa voie, mais Yahweh conduit ses pas.
\VS{10}La divination est sur les lèvres du roi : Sa bouche ne doit pas s'égarer du droit.
\VS{11}La balance et le poids justes sont à Yahweh, tous les poids du sachet sont aussi son oeuvre.
\VS{12}Commettre une injustice doit être en abomination aux rois, parce que le trône est affermi par la justice.
\VS{13}Les rois doivent prendre plaisir aux lèvres de justice, et aimer celui qui profère des paroles justes.
\VS{14}Ce sont autant de messagers de mort que la colère du roi, mais l'homme sage l'apaisera.
\VS{15}Le visage serein du roi c'est la vie, et sa faveur est comme la nuée portant la pluie de la dernière saison.
\VS{16}Combien est-il plus précieux que l'or fin, d'acquérir de la sagesse! Et combien est-il plus excellent que l'argent, d'acquérir de la prudence ! 
\VS{17}Le chemin aplani des hommes droits, c'est de se détourner du mal ; celui qui prend garde de sa voie garde son âme.
\VS{18}L'orgueil va devant l'écrasement, et la fierté d'esprit devant la ruine.
\VS{19}Mieux vaut être humilié d'esprit avec les débonnaires, que de partager le butin avec les orgueilleux.
\VS{20}Celui qui prend garde à la parole trouvera le bien, et celui qui se confie en Yahweh est heureux\FTNT{Ps. 2:12.}.
\VS{21}On appellera prudent le sage de cœur, et la douceur des lèvres augmente l'instruction.
\VS{22}La prudence est à ceux qui la possèdent une source de vie ; mais le l'instruction des fous c'est leur folie.
\VS{23}Celui qui est sage de coeur conduit prudemment sa bouche, et ajoute l'instruction sur ses lèvres.
\VS{24}Les paroles agréables sont des rayons de miel, douces à l'âme et santé pour les os.
\VS{25} II y a telle voie qui semble droite à l'homme, mais dont la fin sont les voies de la mort.
\VS{26}Celui qui travaille, travaille pour lui-même, parce que sa bouche se courbe devant lui\FTNT{Ec. 6:7.}.
\VS{27}L'homme méchant creuse le mal, et il y a comme un feu brûlant sur ses lèvres.
\VS{28}L'homme qui use de perversité sème des querelles, et le rapporteur divise les grands amis.
\VS{29}L'homme violent attire son compagnon et le fait marcher dans une voie qui n'est pas bonne.
\VS{30}Il fait signe des yeux pour méditer des choses perverses, et remuant ses lèvres il exécute le mal.
\VS{31}Les cheveux blancs sont une couronne d'honneur ; elle se trouvera dans la voie de la justice.
\VS{32}Celui qui est lent à la colère vaut mieux que l'homme fort, et celui qui est maître de son cœur, vaut mieux que celui qui prend des villes.
\VS{33}On jette le sort dans le pan de la robe, mais tout ce qui doit arriver est de part Yahweh.
\Chap{17}
\TextTitle{[La justice s'oppose à la méchanceté (suite)]}
\VerseOne{}Mieux vaut un morceau de pain sec là où il y a la paix, qu'une maison pleine de viandes, là où il y a des querelles.
\VS{2}Le serviteur prudent sera maître sur l'enfant qui fait honte, et il partagera l'héritage entre les frères.
\VS{3}Le creuset est pour éprouver l'argent, et le fourneau l'or ; mais Yahweh éprouve les coeurs\FTNT{Jé. 17:10 ; Mal. 3:3 ; Ps. 26:2.}.
\VS{4}L'homme mauvais est attentif à la lèvre trompeuse, et le menteur écoute la mauvaise langue.
\VS{5}Celui qui se moque du pauvre déshonore celui qui l'a fait ; et celui qui se réjouit de l'affliction ne demeurera pas impuni.
\VS{6}Les petits-fils sont la couronne des vieillards\FTNT{Ps. 127:3 ; Ps. 128:3.}, et les pères sont la gloire de leurs fils.
\VS{7}La parole distinguée ne convient pas à un fou ; combien moins aux principaux du peuple des paroles de mensonge!
\VS{8}Le présent est comme une pierre précieuse aux yeux de ceux qui y sont adonnés ; de quelque côté qu'ils se tournent, ils réussissent.
\VS{9}Celui qui couvre les fautes cherche l'amitié, mais celui qui rapporte la chose divise les plus grands amis.
\VS{1}La répréhension se fait mieux sentir sur l'homme prudent que cent coups au fou.
\VS{11}Le méchant ne cherche que rébellion, mais le messager cruel sera envoyé contre lui.
\VS{12}Que l'homme rencontre plutôt une ourse qui a perdu ses petits qu'un fou dans sa folie.
\VS{13}Le mal ne partira point de la maison de celui qui rend le mal pour le bien.
\VS{14}Le commencement d'une querelle est comme quand on lâche une l'eau; mais avant qu'on en vienne à la dispute, retire-toi.
\VS{15}Celui qui déclare juste le méchant et celui qui déclare méchant le juste, sont tous deux en abomination à Yahweh\FTNT{Ex. 23:7 ; Es. 5:23.}.
\VS{16}A quoi sert le prix dans la main du fou pour acheter la sagesse, vu qu'il n'a pas de sens?
\VS{17}L'ami intime aime en tout temps, et il naît comme un frère dans la détresse.
\VS{18}Celui là est dépourvu de sens qui touche à la main et se rend caution pour son ami.
\VS{19}Celui qui aime les querelles aime le péché ; celui qui élève sa porte cherche sa ruine.
\VS{20}Celui qui est pervers de coeur ne trouve pas le bien; et l'hypocrite tombe dans le malheur.
\VS{21}Celui qui engendre un sot en aura de l'ennui, et le père du sot ne se réjouira pas.
\VS{22}Le coeur joyeux est un remède, mais l'esprit abattu dessèche les os.
\VS{23}Le méchant rend les présents en secret, pour pervertir les voies du jugement.
\VS{24}La sagesse est en présence de l'homme prudent; mais les yeux du fou sont à l'extrémité de la terre.
\VS{25}Le fils fou est l'ennui de son père, et l'amertume de celle qui l'a enfanté.
\VS{26}Il n'est pas bon de condamner l'innocent à l'amende, ni que les principaux frappent quelqu'un pur avoir agi avec droiture.
\VS{27}L'homme retenu dans ses paroles sait ce qu'est la connaissance, et l'homme qui est d'un esprit calme est un homme intelligent.
\VS{28}Même le fou, quand il se tait, est réputé sage ; et celui qui ferme ses lèvres est réputé intelligent.
\Chap{18}
\TextTitle{[La justice s'oppose à la méchanceté (suite)]}
\VerseOne{}Celui qui se sépare cherche ce qui lui fait plaisir, et se mêle de savoir comment tout doit aller.
\VS{2}Le fou ne prend pas plaisir à l'intelligence, mais à ce que son cœur soit manifesté.
\VS{3}Quand le méchant vient, le mépris vient aussi, et le reproche avec l'ignominie.
\VS{4}Les paroles de la bouche d'un homme sont des eaux profondes ; et la source de la sagesse est un torrent qui bouillonne\FTNT{Jn. 4:14.}.
\VS{5}Il n'est pas bon d'avoir égard à l'apparence de la personne du méchant, pour renverser le juste en jugement.
\VS{6}La bouche du fou entrent en querelles, et sa bouche appelle les combats.
\VS{7}La bouche du fou lui est une ruine, et ses lèvres sont un piège à son âme.
\VS{8}Les paroles du flatteur sont de ceux qui font semblant d'y toucher ; mais elles pénètrent jusqu'au-dedans des entrailles.
\VS{9}Celui qui se relâche dans son ouvrage est frère de celui qui dissipe ce qu'il a.
\VS{10}Le Nom de Yahweh est une tour forte, le juste y court et y trouve une haute retraite.
\VS{11}Les biens du riche sont sa ville forte et comme une haute muraille de retraite, selon son imagination.
\VS{12}Le coeur de l'homme s'élève avant que la ruine arrive, mais l'humilité précède la gloire.
\VS{13}Celui qui répond à quelque propos avant de l'avoir entendu, agit en fou et s'attire le reproche.
\VS{14}L'esprit d'un homme fort soutiendra dans son infirmité ; mais l'esprit abattu, qui le relèvera ?
\VS{15}Le coeur de l'homme intelligent acquiert la connaissance, et l'oreille des sages cherche la connaissance.
\VS{16}Le présent d'un homme lui fait faire place, et le conduit devant les grands.
\VS{17}Celui qui plaide le premier paraît juste; mais sa partie adverse vient, et examine le tout.
\VS{18}Le sort fait cesser les procès et fait les partages entre les puissants.
\VS{19}Un frère offensé se rend plus difficile qu'une ville forte, et les discordes entre frères sont comme les verrous d'un palais.
\VS{20}Le ventre de chacun est rassasié du fruit de sa bouche, il se rassasie du revenu de ses lèvres.
\VS{21}La mort et la vie sont au pouvoir de la langue\FTNT{Mt. 12:37.}, et celui qui aime à parler mangera de ses fruits.
\VS{22}Celui qui trouve une femme vertueuse trouve le bonheur et il obtient une faveur de Yahweh.
\VS{23}Le pauvre ne prononce que des supplications, mais le riche ne répond que des paroles dures.
\VS{24}L'homme qui a des intimes amis se tiennent à leur amitié parce qu'il y a tel ami qui est plus attaché que le frère.
\Chap{19}
\TextTitle{[La justice s'oppose à la méchanceté (suite)]}
\VerseOne{}Le pauvre qui marche dans son intégrité, vaut mieux que celui qui pervertit ses lèvres et qui est fou.
\VS{2}La vie même sans connaissance n'est pas une bonne personne ; et celui qui hâte ses pas dans le péché, s'égare.
\VS{3}La folie de l'homme renverse son chemin ; et cependant, c'est contre Yahweh que son coeur s'irrite.
\VS{4}Les richesses attirent un grand nombre d'amis, mais celui qui est pauvre est abandonné même par son ami.
\VS{5}Le faux témoin ne restera pas impuni, et celui qui profère des mensonges n'échappera pas.
\VS{6}Plusieurs supplient celui qui est en état de faire du bien, et chacun est ami de celui qui donne.
\VS{7}Tous les frères du pauvre le haïssent ; combien plus ses amis se retirent-ils de lui ! Il les supplie, mais il n'y a que des paroles pour lui.
\VS{8}Celui qui acquiert du sens aime son âme, et celui qui prend garde à l'intelligence c'est pour trouve le bonheur.
\VS{9}Le faux témoin ne restera pas impuni, et celui qui profère des mensonges périra.
\VS{10}Il ne sied pas à un fou de vivre dans les délices ; combien moins sied-il à un esclave de dominer sur les personnes de distinction !
\VS{11}La prudence de l'homme retient à la colère ; c'est un honneur pour lui de passer par dessus le tort qu'on lui fait.
\VS{12}La colère du roi est comme le rugissement d'un jeune lion, mais sa faveur est comme la rosée sur l'herbe.
\VS{13}Un fils insensé est un grand malheur pour son père, et les querelles d'une femme sont une gouttière continuelle.
\VS{14}On peut hériter de ses pères une maison et des richesses, mais la femme prudente est un don de Yahweh.
\VS{15}La paresse fait venir le sommeil, et l'âme paresseuse a faim.
\VS{16}Celui qui garde le commandement garde son âme, mais celui qui méprise ses voies mourra.
\VS{17}Celui qui a pitié du pauvre prête à Yahweh, qui lui rendra son bienfait.
\VS{18}Châtie ton fils tandis qu'il y a de l'espérance, mais ne va pas jusqu'à le faire mourir.
\VS{19}Celui qui est de grande colère en porte la peine ; et si tu l'en retires, tu y ajoute davantage.
\VS{20}Ecoute le conseil et reçois l'instruction, afin que tu deviennes sage en ton dernier temps.
\VS{21}Il y a dans le cœur de l'homme plusieurs pensées, mais le conseil de Yahweh est\FTNT{Es. 46:10 ; Ps. 33:11.}.
\VS{22}Ce que l'homme doit désirer, c'est d'exercer la miséricorde ; et le pauvre vaut mieux qu'un menteur.
\VS{23}La crainte de Yahweh conduit à la vie, et celui qui l'a, passe la nuit étant rassasié, sans qu'il soit visité par aucun mal.
\VS{24}Le paresseux cache sa main dans le sein, et il ne daigne même pas la ramener à sa bouche.
\VS{25}Si tu bats le moqueur, le sot en rend garde ; et si tu reprends l'homme intelligent, il discernera ce qu'il faut savoir.
\VS{26}L'enfant qui fait honte et sème la confusion, détruit le père et met en fuite sa mère.
\VS{27}Mon fils, cesse d'écouter ce qui pourrait t'apprendre à te détourner des paroles de la connaissance.
\VS{28}Le témoin indigne\FTNT{Le mot « pervers » vient de l'hébreu « beliya'al » : « sans valeur », « vaurien » (Jg. 19:22 ; 1. S. 2:12). Bélial est aussi un autre nom de Satan (2 Co. 6:15).} se moque de la justice, et la bouche des méchants avale l'iniquité.
\VS{29}Les jugements sont préparés pour les moqueurs, et les grands coups pour le dos des fous.
\Chap{20}
\TextTitle{[La justice s'oppose à la méchanceté (suite)]}
\VerseOne{}Le vin est moqueur et les boissons fortes sont tumultueuses, quiconque en fait excès, n'est pas sage.
\VS{2}La terreur du roi est comme le rugissement d'un jeune lion, celui qui l'irrite pèche contre sa propre âme.
\VS{3}C'est une gloire à l'homme de s'abstenir des disputes, mais tout insensé s'y engage.
\VS{4}Le paresseux ne labourera pas à cause de l'hiver, lors de la moisson il mendiera et n'aura rien.
\VS{5}Les conseils dans le coeur d'un homme sage sont comme des eaux profondes, et l'homme intelligent sait y puiser.
\VS{6}Beaucoup de gens vantent leur bonté ; mais l'homme fidèle, qui le trouvera ?
\VS{7}Ô, que les fils du juste qui marchent dans son intégrité seront heureux après lui !
\VS{8}Le roi assis sur le trône de justice dissipe tout mal par son regard.
\VS{9}Qui est-ce qui peut dire : J'ai purifié mon cœur, je suis net de mon péché ?
\VS{10}Le double poids et la double mesure sont tous deux en abomination à Yahweh.
\VS{11}Un jeune enfant fait connaître par ses actions si son oeuvre sera pure et droite.
\VS{12}L'oreille qui entend et l'oeil qui voit, Yahweh les a faits tous les deux.
\VS{13}N'aime point le sommeil, de peur que tu ne deviennes pauvre ; ouvre tes yeux, et tu auras suffisamment de pain.
\VS{14}Il est mauvais, il est mauvais, dit l'acheteur ; puis il s'en va, et se vante.
\VS{15}Il y a de l'or et beaucoup de perles ; mais les lèvres qui gardent la connaissance sont un vase précieux.
\VS{16}Quand quelqu'un se porte garant pour l'étranger, prends son vêtement ; exige de lui des gages pour cet étranger.
\VS{17}Le pain acquis par la tromperie est doux à l'homme, mais ensuite sa bouche sera remplie de gravier.
\VS{18}Les projets s'affermissent par le conseil ; fais donc la guerre avec prudence.
\VS{19}Celui qui médit révèle les secrets ; ne te mêle donc pas avec celui qui séduit par ses lèvres.
\VS{20}Celui qui traite avec mépris son père ou sa mère, sa lampe s'éteindra au milieu des ténèbres les plus noires\FTNT{Ex. 21:17 ; Lé. 20:9 ; Mt. 15:4.}.
\VS{21}L'héritage pour lequel on s'est trop hâté dès l'origine, ne sera pas béni à la fin.
\VS{22}Ne dis point : Je rendrai le mal ; espère en Yahweh, et il te délivrera.
\VS{23}Le double poids est en horreur à Yahweh, et la balance fausse n'est pas une chose bonne.
\VS{24}Les pas de l'homme sont dirigés par Yahweh, comment donc l'homme comprendrait-il sa voie ?
\VS{25}C'est un piège à l'homme que prendre à la légère un engagement sacré, et de ne réfléchir qu'après avoir fait un vœu.
\VS{26}Un roi sage disperse les méchants et ramène la roue sur eux.
\VS{27}L'esprit de l'homme est une lampe de Yahweh, il pénètre jusqu'au fond des entrailles.
\VS{28}La bienveillance et la vérité protègent le roi, et il soutient son trône par la bienveillance.
\VS{29}La force est la gloire des jeunes gens, et les cheveux blancs sont l'honneur des vieillards.
\VS{30}Les meurtrissures et les plaies nettoient le mal, de même les coups qui pénètrent jusqu'au fond des entrailles.
\Chap{21}
\TextTitle{[La justice s'oppose à la méchanceté (suite)]}
\VerseOne{}Le coeur du roi est un courant d'eau dans la main de Yahweh ; il l'incline partout où il veut.
\VS{2}Toutes les voies de l'homme sont droites à ses yeux, mais c'est Yahweh qui pèse les coeurs.
\VS{3}Faire ce qui est juste et droit est une chose que Yahweh préfère aux sacrifices.
\VS{4}Des regards hautains et le coeur qui s'enfle sont la lampe des méchants, ce n'est que péché.
\VS{5}Les projets de l'homme diligent ne mènent qu'à l'abondance, mais celui qui agit avec précipitation ne court qu'à l'indigence.
\VS{6}Des trésors acquis par une langue mensongère, c'est une vanité qu'on ne peut retenir, un signe avant-coureur de la mort.
\VS{7}La violence des méchants les emporte, parce qu'ils refusent de faire ce qui est droit.
\VS{8}La voie d'un homme coupable est détournée, mais l'oeuvre de celui qui est innocent est droite.
\VS{9}Il vaut mieux habiter à l'angle d'un toit qu'avec une femme querelleuse dans une grande maison.
\VS{10}L'âme du méchant désire le mal, son prochain ne trouve pas de grâce à ses yeux.
\VS{11}Quand on punit le moqueur, le sot devient sage ; et quand on instruit le sage, il reçoit la connaissance.
\VS{12}Il y a un juste qui considère attentivement la maison du méchant, Yahweh renverse les méchants dans le malheur.
\VS{13}Celui qui bouche son oreille pour ne pas entendre le cri du pauvre, criera aussi lui-même, et on ne lui répondra point.
\VS{14}Un don fait en secret apaise la colère, et un présent fait en cachette calme une fureur violente.
\VS{15}C'est une joie pour le juste de pratiquer la justice, mais c'est la ruine pour les ouvriers d'iniquité.
\VS{16}L'homme qui s'écarte du chemin de la sagesse aura sa demeure dans l'assemblée des morts.
\VS{17}Celui qui aime les réjouissances reste dans l'indigence ; et celui qui aime le vin et l'huile ne s'enrichira pas.
\VS{18}Le méchant sert de rançon pour le juste, et le déloyal pour les hommes intègres.
\VS{19}Il vaut mieux habiter dans une terre déserte qu'avec une femme querelleuse et qui se dépite.
\VS{20}Des précieux trésors et l'huile sont dans la demeure du sage, mais l'homme insensé les engloutit.
\VS{21}Celui qui poursuit la justice et la bonté, trouve la vie, la justice et la gloire.
\VS{22}Le sage entre dans la ville des forts et il abat la force qui lui donnait de l'assurance.
\VS{23}Celui qui veille sur sa bouche et sur sa langue préserve son âme des angoisses.
\VS{24}On appelle moqueur un superbe arrogant, qui agit avec colère et orgueil.
\VS{25}Les désirs du paresseux le tuent, parce que ses mains refusent de travailler.
\VS{26}Tout le jour il désire avidement, mais le juste donne sans parcimonie.
\VS{27}Le sacrifice des méchants est une abomination ; combien plus quand ils l'apportent avec des mauvaises intentions\FTNT{1 S. 15:22.} ?
\VS{28}Le témoin menteur périra, mais l'homme qui écoute parlera avec gain de cause.
\VS{29}L'homme méchant prend un air effronté, mais l'homme droit règle sa conduite.
\VS{30}Il n'y a ni sagesse, ni intelligence, ni conseil, contre Yahweh.
\VS{31}Le cheval est équipé pour le jour de la bataille, mais la délivrance vient de Yahweh.
\Chap{22}
\TextTitle{[La justice s'oppose à la méchanceté (suite)]}
\VerseOne{}La renommée est préférable aux grandes richesses\FTNT{Ec.7:1.}, et la bonne grâce plus que l'argent et l'or.
\VS{2}Le riche et le pauvre se rencontrent ; celui qui les a faits l'un et l'autre, c'est Yahweh\FTNT{Lu. 16.}.
\VS{3}L'homme prudent voit le mal et se cache, mais les stupides passent et en portent la peine.
\VS{4}Les fruits de l'humilité et de la crainte de Yahweh sont les richesses, la gloire et la vie.
\VS{5}Il y a des épines et des pièges dans la voie de l'homme pervers ; celui qui aime son âme s'en retirera loin.
\VS{6}Instruis le jeune enfant selon la voie qu'il doit suivre, et quand il sera vieux, il ne s'en détournera pas.
\VS{7}Le riche domine sur les pauvres\FTNT{Ja. 2:6.}, et celui qui emprunte est l'esclave de celui qui prête.
\VS{8}Celui qui sème l'injustice moissonne le malheur\FTNT{Job. 4:8 ; Ga. 6:7.}, et la verge de sa fureur prendra fin.
\VS{9}Celui qui a l'oeil bienveillant sera béni, parce qu'il aura donné de son pain au pauvre.
\VS{10}Chasse le moqueur, et la querelle prendra fin ; les disputes et l'ignominie cesseront.
\VS{11}Le roi est ami de celui qui aime la pureté de coeur, et qui a de la grâce dans ses paroles.
\VS{12}Les yeux de Yahweh veillent sur la connaissance, mais il confond les paroles du perfide.
\VS{13}Le paresseux dit : Il y a un lion dehors ! Je serais tué dans les rues !
\VS{14}La bouche des courtisanes est une fosse profonde, celui contre qui Yahweh est irrité y tombera.
\VS{15}La folie est liée au coeur du jeune enfant, mais la verge de la correction l'éloignera de lui.
\VS{16}Celui qui fait tort au pauvre pour s'enrichir et pour donner au riche, ne peut manquer de tomber dans l'indigence.
\VS{17}Prête ton oreille et écoute les paroles des sages, et applique ton coeur à ma connaissance.
\VS{18}Car ce sera une chose agréable pour toi si tu les gardes au-dedans de toi, et qu'elles soient toutes présentes sur tes lèvres.
\VS{19}Je te les ai fait connaître à toi aujourd'hui, dis-je, afin que ta confiance soit en Yahweh.
\VS{20}N'ai-je pas déjà pour toi mis par écrit des choses qui conviennent à ceux qui gouvernent, des conseils et des réflexions,
\VS{21}pour te faire connaître la certitude des paroles vraies, afin que tu répondes par des paroles vraies à celui qui t'envoie ?
\VS{22}Ne dépouille pas le pauvre, parce qu'il est pauvre ; et n'opprime pas le malheureux à la porte.
\VS{23}Car Yahweh défendra leur cause et privera de la vie ceux qui les auront volés.
\VS{24}Ne fréquente pas quelqu'un de coléreux, ne va pas avec l'homme violent ;
\VS{25}de peur que tu n'apprennes ses manières, et qu'ils ne deviennent un piège pour ton âme.
\VS{26}Ne sois pas parmi ceux qui prennent des engagements ni de ceux qui cautionnent les dettes.
\VS{27}Si tu n'as pas de quoi payer, pourquoi prendrait-on ton lit de dessous toi ?
\VS{28}Ne déplace pas la borne ancienne, que tes pères ont posée.
\VS{29}As-tu vu un homme habile en son travail ? Il sera au service des rois, il ne se tiendra pas devant des gens obscurs.
\Chap{23}
\TextTitle{[La justice s'oppose à la méchanceté (suite)]}
\VerseOne{}Quand tu t'assieds pour manger avec un gouverneur, considère avec attention celui qui est devant toi.
\VS{2}Autrement tu te mettras le couteau à la gorge, si ton appétit te domine.
\VS{3}Ne convoite pas ses friandises, car c'est un pain trompeur.
\VS{4}Ne travaille pas en vue d'acquérir des richesses ; désiste-toi de la résolution que tu as prise.
\VS{5}Jetteras-tu tes yeux sur ce qui bientôt n'est plus ? Car certainement, il se fera des ailes, il s'envolera comme un aigle dans les cieux.
\VS{6}Ne mange pas le pain de celui dont le regard est envieux, et ne désire pas ses friandises.
\VS{7}Car il est tel qu'il pense dans son âme. Il te dira bien : Mange et bois, mais son coeur n'est pas avec toi.
\VS{8}Tu voudrais vomir le morceau que tu auras mangé, et tu auras perdu tes paroles agréables.
\VS{9}Ne parle pas aux oreilles de l'insensé, car il méprise le bon sens de ton discours.
\VS{10}Ne déplace pas la borne ancienne et n'entre pas dans les champs des orphelins :
\VS{11}Car leur vengeur est puissant, il défendra leur cause contre toi.
\VS{12}Applique ton coeur à l'instruction, et tes oreilles aux paroles de la connaissance.
\VS{13}Ne te retiens pas de corriger le jeune enfant ; quand tu l'auras frappé de la verge, il n'en mourra pas.
\VS{14}En le frappant de la verge, tu préserves son âme du scheol.
\VS{15}Mon fils, si ton coeur est sage, mon coeur s'en réjouira, oui, moi-même.
\VS{16}Certes, mes reins tressailliront de joie quand tes lèvres proféreront ce qui est droit.
\VS{17}Que ton coeur ne porte pas d'envie aux pécheurs, mais adonne-toi tout le jour à la crainte de Yahweh.
\VS{18}Car il y a véritablement un avenir, et ton espérance ne sera pas retranchée.
\VS{19}Toi, mon fils, écoute et sois sage, et dirige ton coeur dans la bonne voie.
\VS{20}Ne fréquente point les ivrognes ni les gourmands\FTNT{Ro. 13:13 ; Ep. 5:18 ; Ga. 5:18-21.}.
\VS{21}Car l'ivrogne et le gourmand s'appauvrissent ; et l'assoupissement fait porter des vêtements déchirés.
\VS{22}Ecoute ton père, c'est celui qui t'a engendré ; et ne méprise pas ta mère, quand elle est devenue vieille.
\VS{23}Acquiers la vérité, et ne la vends point, acquiers la sagesse, l'instruction et l'intelligence.
\VS{24}Le père du juste aura beaucoup de joie, et celui qui donne naissance à un sage se réjouira en lui.
\VS{25}Que ton père et ta mère se réjouissent, que celle qui t'a enfanté soit dans l'allégresse !
\VS{26}Mon fils, donne-moi ton coeur, et que tes yeux prennent plaisir à mes voies.
\VS{27}Car la femme prostituée est une fosse profonde, et la courtisane un puits de détresse.
\VS{28}Aussi se tient-elle en embûche comme un voleur, et elle augmente parmi les hommes le nombre des infidèles.
\VS{29}Pour qui les « ah » ? Pour qui les « malheur à moi ! » Pour qui les disputes ? Pour qui les plaintes ? Pour qui les blessures sans raison ? Pour qui les yeux rouges ?
\VS{30}Pour ceux qui s'attardent auprès du vin, pour ceux qui vont chercher des vins mélangés.
\VS{31}Ne regarde pas le vin parce qu'il est d'un beau rouge, qu'il donne son éclat dans la coupe, et qu'il coule aisément.
\VS{32}Il finit par mordre par derrière comme un serpent, et par piquer comme un basilic.
\VS{33}Ensuite tes yeux regarderont les femmes étrangères, et ton coeur parlera d'une manière perverse.
\VS{34}Tu seras comme un homme qui dort au milieu de la mer, et comme un homme couché sur le sommet d'un mât.
\VS{35}On m'a battu, diras-tu, et je n'en ai pas été malade, on m'a frappé, et je ne l'ai pas senti, quand me réveillerai-je ? Je me remettrai encore à chercher le vin.
\Chap{24}
\TextTitle{[La justice s'oppose à la méchanceté (suite)]}
\VerseOne{}N'envie pas les hommes qui font le mal, et ne désire pas être avec eux.
\VS{2}Car leur coeur médite la destruction, et leurs lèvres parlent d'iniquité.
\VS{3}C'est par la sagesse qu'une maison est bâtie, et par l'intelligence qu'elle s'affermit.
\VS{4}C'est par la connaissance que les chambres seront remplies de tous les biens précieux et agréables.
\VS{5}Un homme sage est accompagné de force, et celui qui a de la connaissance affermit sa vigueur.
\VS{6}Car avec de bonnes directives tu feras la guerre avantageusement, et le salut est dans le grand nombre des bons conseillers.
\VS{7}La sagesse est trop élevée pour l'insensé, il n'ouvrira pas sa bouche à la porte.
\VS{8}Celui qui médite de faire le mal s'appelle un homme plein de malice.
\VS{9}Le projet de la folie est un péché, et le moqueur est en abomination parmi les hommes.
\VS{10}Si tu perds courage au jour de la détresse, ta force n'est que détresse.
\VS{11}Ne te retiens pas de délivrer ceux qu'on traîne à la mort, ceux qu'on va égorger, agis pour qu'on les épargne !
\VS{12}Si tu dis : Ah ! nous n'en savions rien… Celui qui pèse les cœurs, lui, ne le considérera-t-il pas ? Celui qui garde ton âme, lui, le sait, et il rend à chacun selon son œuvre.
\VS{13}Mon fils, mange le miel, car il est bon ; un rayon de miel sera doux à ton palais.
\VS{14}Ainsi sera à ton âme la connaissance de la sagesse, quand tu l'auras trouvée ; il y a un avenir, et ton espérance ne sera pas anéantie.
\VS{15}Méchant, ne mets pas des embûches dans le domaine du juste, et ne dévaste pas le lieu où il se repose.
\VS{16}Car le juste tombera sept fois, et sera relevé\FTNT{Ps. 34:20. ; Job. 5:19.} ; mais les méchants trébuchent pour tomber dans le malheur.
\VS{17}Si ton ennemi tombe, ne t'en réjouis pas, et que ton coeur ne soit pas dans l'allégresse quand il chancelle,
\VS{18}de peur que Yahweh ne le voie et que cela ne lui déplaise, tellement qu'il détourne sa colère de dessus de lui sur toi.
\VS{19}Ne t'irrite pas à cause de ceux qui font le mal, n'envie pas les méchants,
\VS{20}car il n'y a pas d'avenir pour celui qui fait le mal, la lampe des méchants sera éteinte.
\VS{21}Mon fils, crains Yahweh et le roi ; et ne te mêle pas avec les gens agités.
\VS{22}Car leur ruine s'élèvera tout d'un coup ; et qui sait le malheur qui arrivera à l'un et à l'autre ?
\VS{23}Voici encore ce qui vient des sages : Il n'est pas bon d'avoir égard à l'apparence des personnes dans le jugement.
\VS{24}Celui qui dit au méchant : Tu es juste ! Les peuples le maudiront, et les nations seront indignées contre lui.
\VS{25}Mais pour ceux qui le reprennent, ils en retireront de la satisfaction. Et la bénédiction vient sur eux pour leur bonheur.
\VS{26}Celui qui répond avec justesse fait plaisir à celui qui l'écoute.
\VS{27}Prépare ton ouvrage au-dehors, et apprête ton champ, et après, tu bâtiras ta maison.
\VS{28}Ne témoigne pas sans cause contre ton prochain ; car voudrais-tu tromper de tes lèvres\FTNT{Ep. 4:25.} ?
\VS{29}Ne dis pas : Comme il m'a fait, ainsi je lui ferai ; je rendrai à cet homme selon ce qu'il m'a fait.
\VS{30}J'ai passé près du champ de l'homme de paresseux, et près de la vigne d'un homme dépourvu de sens ;
\VS{31}et voici, tout y était monté en chardons, et les ronces avaient couvert la surface, et le mur de pierres était écroulé.
\VS{32}Et j'ai regardé, j'y ai appliqué mon coeur, j'ai vu et j'en ai tiré instruction.
\VS{33}Un peu de sommeil, un peu d'assoupissement, un peu croiser les mains pour dormir ! …
\VS{34}Et la pauvreté te surprendra comme un rôdeur ; et la disette, comme un homme armé.
\Chap{25}
\TextTitle{[Avertissements et conseils]}
\VerseOne{}Ce sont ici aussi des proverbes de Salomon\FTNT{1 R. 4: 32.}, que les gens d'Ezéchias, roi de Juda, ont transcrits.
\VS{2}La gloire de Dieu est de cacher les choses, et la gloire des rois est de sonder les choses.
\VS{3}Les cieux dans leur hauteur, la terre dans sa profondeur, le coeur des rois sont impénétrables.
\VS{4}Ôte de l'argent les scories, et il en sortira un vase pour l'orfèvre.
\VS{5} De même, ôte le méchant de devant le roi, et son trône sera affermi par la justice.
\VS{6}Ne t'élève pas devant le roi et ne te tiens pas à la place des grands.
\VS{7}Car il vaut mieux qu'on te dise: Monte-ici ! Que si l'on t'abaisse devant un seigneur que tes yeux ont vu\FTNT{Lu. 14:8-11.}.
\VS{8}Ne te hâte pas d'entrer en contestation, de peur que tu ne saches que faire à la fin, lorsque ton prochain t'aura confondu.
\VS{9}Plaide ta cause contre ton prochain, mais ne révèle pas le secret d'autrui.
\VS{10}De peur que celui qui l'écoute ne te couvre de honte, et que ton opprobre ne s'efface pas.
\VS{11}Telles que sont des pommes d'or sur des ciselures d'argent, telle est une parole dite quand il faut.
\VS{12}Comme un anneau d'or ou comme un joyau d'or fin, ainsi est l'oreille obéissante pour le sage qui réprimande.
\VS{13}Le messager fidèle est à ceux qui l'envoient, comme la fraîcheur de la neige au temps de la moisson, il restaure l'âme de son maître.
\VS{14}Celui qui se glorifie faussement de ses libéralités, est comme les nuages et le vent sans pluie.
\VS{15}Un prince est fléchi par la patience, et la langue douce brise les os.
\VS{16}As-tu trouvé du miel, manges-en autant qu'il t'en faut, de peur que tu n'en sois rassasié, que tu ne le vomisses.
\VS{17}De même, mets rarement le pied dans la maison de ton prochain, de peur qu'étant rassasié de toi, il ne te haïsse.
\VS{18}Comme une massue, une épée et une flèche aiguë, ainsi est un homme qui porte un faux témoignage contre son prochain.
\VS{19}Comme une dent qui se rompt, un pied qui glisse, telle est la confiance qu'on met en un traître au jour de la détresse.
\VS{20}Celui qui chante des chansons à un coeur affligé est comme celui qui ôte sa robe dans un jour froid, et comme du vinaigre répandu sur du nitre.
\VS{21}Si celui qui te hait a faim, donne-lui à manger du pain ; et s'il a soif, donne-lui à boire de l'eau\FTNT{Mt 5:39-44.}.
\VS{22}Car ce sont des charbons ardents que tu lui mets sur sa tête, et Yahweh te le rendra.
\VS{23}Le vent du nord engendre les averses, et la langue qui médit en cachette un visage irrité.
\VS{24}Il vaut mieux habiter à l'angle d'un toit que de partager la demeure d'une femme querelleuse.
\VS{25}Comme de l'eau fraîche pour une personne fatiguée et lasse, ainsi est une bonne nouvelle venant d'une terre lointaine.
\VS{26}Le juste qui bronche devant le méchant est une fontaine troublée et une source gâtée.
\VS{27}Comme il n'est pas bon de manger beaucoup de miel, aussi n'y a-t-il pas de gloire pour les hommes de rechercher leur gloire avec ardeur.
\VS{28}L'homme qui n'est pas maître de son esprit est comme une ville où il y a une brèche, et qui est sans murailles.
\Chap{26}
\TextTitle{[Avertissements et conseils (suite)]}
\VerseOne{}Comme la neige en été et la pluie pendant la moisson, ainsi la gloire ne convient pas à un insensé.
\VS{2}Comme l'oiseau est prompt à s'échapper et l'hirondelle à s'envoler, ainsi la malédiction sans cause n'atteint pas.
\VS{3}Le fouet est pour le cheval, le mors pour l'âne, et la verge pour le dos des insensés.
\VS{4}Ne réponds pas à l'insensé selon sa folie, de peur que tu ne lui ressembles toi-même.
\VS{5}Réponds à l'insensé selon sa folie, de peur qu'il ne devienne sage à ses propres yeux.
\VS{6}Celui qui envoie des messages par l'intermédiaire d'un insensé, se coupe les pieds et boit la peine du tort qu'il s'est fait.
\VS{7}Faites marcher un homme boiteux, ainsi il en sera d'un proverbe dans la bouche des insensés.
\VS{8}Celui qui donne de la gloire à un insensé, c'est comme s'il jetait une pierre précieuse dans un monceau de pierres.
\VS{9}Comme une épine dans la main d'un homme ivre, ainsi est un proverbe dans la bouche des insensés.
\VS{10}Les puissants donnent de l'ennui à tout le monde, et prennent à leur service les insensés et les passants.
\VS{11}Comme le chien retourne à ce qu'il a vomi, ainsi l'insensé réitère sa folie\FTNT{2 Pi. 2:22.}.
\VS{12}As-tu vu un homme qui croit être sage ? Il y a plus à espérer d'un insensé que de lui.
\VS{13}Le paresseux dit : Il y a un lion rugissant sur le chemin, il y a un lion dans les rues.
\VS{14}Comme une porte tourne sur ses gonds, ainsi fait le paresseux sur son lit.
\VS{15}Le paresseux plonge sa main dans le plat, et il trouve fatigant de la ramener à sa bouche.
\VS{16}Le paresseux se croit plus sage que sept hommes qui répondent avec bon sens.
\VS{17}Celui qui en passant se met en colère pour une dispute qui ne le touche en rien, est comme celui qui prend un chien par les oreilles.
\VS{18}Tel est celui qui fait l'insensé, et qui cependant jette des feux, des flèches, et des choses propres à tuer,
\VS{19}tel est l'homme qui a trompé son ami, et qui après cela dit : N'était-ce pas pour plaisanter ?
\VS{20}Le feu s'éteint faute de bois, ainsi quand il n'y a pas de rapporteurs la querelle s'apaise.
\VS{21}Le charbon est pour faire de la braise, et le bois pour faire du feu, et l'homme querelleur pour exciter des querelles.
\VS{22}Les paroles d'un rapporteur sont comme des friandises, elles descendent jusqu'au fond des entrailles.
\VS{23}Les lèvres brûlantes de zèle et le coeur mauvais sont comme des scories d'argent appliquées sur un vase de terre.
\VS{24}Celui qui a de la haine se déguise par ses discours, mais au-dedans de lui il nourrit la trahison.
\VS{25}Lorsqu'il prend une voix douce, ne le crois pas, car il y a sept abominations dans son coeur.
\VS{26}S'il cache sa haine sous la dissimulation, sa méchanceté sera révélée dans l'assemblée.
\VS{27}Celui qui creuse la fosse y tombe ; et la pierre retourne sur celui qui la roule\FTNT{Ps. 7:16-17 ; Ps. 57:7 ; Ec. 10:8.}.
\VS{28}La fausse langue hait ceux qu'elle a écrasés ; et la bouche flatteuse prépare la ruine.
\Chap{27}
\TextTitle{[Avertissements et conseils (suite)]}
\VerseOne{}Ne te vante point du lendemain, car tu ne sais pas quelle chose le jour enfantera\FTNT{Ja. 4:13-15.}.
\VS{2}Qu'un autre te loue, et non pas ta propre bouche ; que ce soit l'étranger, et non pas tes lèvres.
\VS{3}La pierre est pesante, et le sable est lourd ; mais l'irritation de l'insensé est plus pesante que tous les deux.
\VS{4}Il y a de la cruauté dans la fureur, et du débordement dans la colère ; mais qui pourra subsister devant la jalousie ?
\VS{5}Mieux vaut une réprimande ouverte qu'une amitié cachée.
\VS{6}Les blessures d'un ami sont dignes de confiance, mais les baisers d'un ennemi sont à craindre\FTNT{Il est question ici de Judas}.
\VS{7}L'âme rassasiée foule aux pieds les rayons de miel ; mais l'âme qui a faim trouve doux même ce qui est amer.
\VS{8}Tel qu'est un oiseau errant loin de son nid, tel est l'homme qui s'écarte de son lieu.
\VS{9}L'huile et les parfums réjouissent le coeur, et il en est ainsi de la douceur d'un ami dont le fruit est un conseil qui vient du coeur.
\VS{10}Ne quitte point ton ami ni l'ami de ton père, et n'entre pas dans la maison de ton frère au jour de ta détresse ; car le voisin qui est proche vaut mieux que le frère qui est éloigné.
\VS{11}Mon fils sois sage, et réjouis mon coeur, afin que j'aie de quoi répondre à celui qui m'outrage.
\VS{12}L'homme prudent voit le malheur arriver et se cache ; mais les stupides passent outre et en payent l'amende.
\VS{13}Quand quelqu'un se portera garant pour l'étranger, prends son vêtement, exige de lui des gages, à cause des étrangers.
\VS{14}Celui qui bénit son ami à haute voix, se levant de grand matin, on le lui comptera comme une malédiction.
\VS{15}Une gouttière continuelle en un jour de grosse pluie, et une femme querelleuse, cela se ressemble.
\VS{16}Celui qui veut la retenir est comme s'il voulait arrêter le vent, et retenir dans sa main une huile qui s'écoule.
\VS{17}Comme le fer aiguise le fer, ainsi l'homme aiguise la personnalité de son prochain.
\VS{18}Comme celui qui garde le figuier mangera de son fruit, ainsi celui qui garde son maître sera honoré.
\VS{19}Comme dans l'eau le visage répond au visage, ainsi le coeur d'un homme répond à celui d'un autre homme.
\VS{20}Le scheol et le gouffre ne sont jamais rassasiés ; de même, les yeux des hommes sont insatiables\FTNT{Ec. 1:8 ; 2 Pi. 2:14.}.
\VS{21}Le fourneau est pour éprouver l'argent, et le creuset pour l'or ; mais un homme est jugé d'après sa renommée.
\VS{22}Quand tu pilerais un insensé dans un mortier, parmi du grain qu'on pile avec un pilon, sa stupidité ne se détacherait pas de lui.
\VS{23}Sois diligent à reconnaître l'état de chacune de tes brebis, et applique ton coeur aux troupeaux.
\VS{24}Car l'abondance ne dure pas à toujours, et une couronne dure-t-elle d'âge en âge ?
\VS{25}Le foin s'enlève et la verdure paraît, et les herbes des montagnes sont amassées.
\VS{26}Les agneaux sont pour te vêtir, et les boucs pour payer le champ ;
\VS{27}et l'abondance du lait des chèvres sera pour ta nourriture et celle de ta maison, et pour la subsistance de tes servantes.
\Chap{28}
\TextTitle{[Avertissements et conseils (suite)]}
\VerseOne{}Le méchant prend la fuite sans qu'on le poursuive, mais les justes seront assurés comme un jeune lion.
\VS{2}Il y a plusieurs chefs, à cause de la rébellion d'un pays, mais pour l'amour de l'homme avisé et intelligent, il y aura prolongation du même gouvernement.
\VS{3}L'homme qui est pauvre et qui opprime les pauvres, est comme une pluie violente qui cause la disette du pain.
\VS{4}Ceux qui abandonnent la loi louent le méchant, mais ceux qui gardent la loi leur font la guerre.
\VS{5}Les gens adonnés au mal n'entendent point ce qui est droit ; mais ceux qui cherchent Yahweh comprennent tout.
\VS{6}Le pauvre qui marche dans son intégrité vaut mieux que le pervers qui marche par deux chemins et qui est riche.
\VS{7}Celui qui garde la loi est un fils prudent, mais celui qui entretient les gourmands fait honte à son père.
\VS{8}Celui qui augmente ses biens par l'intérêt et l'usure, les amasse pour celui qui en fera des libéralités aux pauvres.
\VS{9}Celui qui détourne son oreille pour ne pas écouter la loi, sa prière même est une abomination\FTNT{La prière doit être faite selon la Parole de Dieu, en conformité avec sa volonté. Le Seigneur n'exauce que ceux qui obéissent à sa Parole (Mt. 6:9-10 ; Jn. 9:31 ; Jn. 15:7 ; 1 Jn. 5:14-15).}.
\VS{10}Celui qui égare les hommes droits dans le mauvais chemin tombera dans la fosse qu'il aura creusée, mais ceux qui sont intègres hériteront le bonheur.
\VS{11}L'homme riche pense être sage, mais le pauvre qui est intelligent le sondera.
\VS{12}Quand les justes se réjouissent, la gloire est grande, mais quand les méchants sont élevés, chacun se déguise.
\VS{13}Celui qui cache ses transgressions ne prospère point, mais celui qui les confesse et les délaisse, obtient miséricorde\FTNT{Ec. 1:8 ; 2 Pi. 2:14.}.
\VS{14}Heureux est l'homme qui est continuellement dans la crainte, mais celui qui endurcit son coeur tombera dans la calamité.
\VS{15}Le méchant qui domine sur un peuple pauvre est un lion rugissant et comme un ours quêtant sa proie.
\VS{16}Le conducteur qui manque d'intelligence fait beaucoup d'extorsions, mais celui qui hait le gain déshonnête prolonge ses jours.
\VS{17}L'homme chargé du sang d'une personne fuira jusqu'à la fosse sans qu'aucun ne le retienne.
\VS{18}Celui qui marche dans l'intégrité sera sauvé, mais le pervers qui suit deux chemins tombera tout à coup.
\VS{19}Celui qui laboure sa terre sera rassasié de pain, mais celui qui suit les fainéants sera accablé de misère.
\VS{20}L'homme fidèle abondera en bénédictions, mais celui qui se hâte de s'enrichir ne restera pas impuni.
\VS{21}Il n'est pas bon d'avoir égard à l'apparence des personnes, car pour un morceau de pain l'homme commet un crime.
\VS{22}L'homme qui a l'oeil malin se hâte pour avoir des richesses, et il ne sait pas que la disette lui arrivera.
\VS{23}Celui qui reprend les hommes obtient ensuite plus de faveur que celui qui flatte de sa langue.
\VS{24}Celui qui pille son père ou sa mère, et qui dit que ce n'est point un péché, est compagnon de l'homme dissipateur.
\VS{25}Celui qui a l'âme enflée excite les querelles, mais celui qui se confie en Yahweh sera rassasié.
\VS{26}Celui qui se confie dans son propre coeur est un fou, mais celui qui marche sagement sera délivré.
\VS{27}Celui qui donne au pauvre n'aura point de disette, mais celui qui en détourne ses yeux abondera en malédictions.
\VS{28}Quand les méchants s'élèvent, l'homme se cache ; mais quand ils périssent, les justes se multiplient.
\Chap{29}
\TextTitle{[Avertissements et conseils (suite)]}
\VerseOne{}L'homme qui étant repris, raidit son cou, sera subitement brisé et sans qu'il y ait de guérison.
\VS{2}Quand les justes sont nombreux, le peuple se réjouit ; mais quand le méchant domine, le peuple gémit.
\VS{3}L'homme qui aime la sagesse, réjouit son père, mais celui qui se plaît avec les femmes prostituées dissipe ses richesses.
\VS{4}Le roi affermit le pays par la justice, mais l'homme qui est adonné aux présents le ruinera.
\VS{5}L'homme qui flatte son prochain lui tend un piège sous ses pas.
\VS{6}Le péché de l'homme méchant lui tend un piège dangereux, mais le juste triomphe et se réjouit.
\VS{7}Le juste prend connaissance de la cause des pauvres, mais le méchant n'en prend pas connaissance.
\VS{8}Les hommes moqueurs troublent la ville, mais les sages apaisent la colère.
\VS{9}Un homme sage qui conteste avec un insensé, qu'il se fâche ou qu'il rie, la paix n'aura pas lieu.
\VS{10}Les hommes de sang ont en haine l'homme intègre, mais les hommes droits tiennent chère sa vie.
\VS{11}L'insensé pousse au-dehors tout ce qu'il a dans l'esprit, mais le sage le calme et le retient en arrière.
\VS{12}Tous les serviteurs d'un prince qui prêtent l'oreille à la parole de mensonge sont méchants.
\VS{13}Le pauvre et l'oppresseur se rencontrent, c'est Yahweh qui illumine les yeux de l'un et de l'autre.
\VS{14}Le trône du roi qui fait justice selon la vérité aux pauvres, sera établi à perpétuité.
\VS{15}La verge et la réprimande donnent la sagesse, mais l'enfant livré à lui-même fait honte à sa mère.
\VS{16}Quand les méchants se multiplient, les péchés s'accroissent, mais les justes verront leur ruine.
\VS{17}Corrige ton fils, et il te donnera du repos, et il procurera du plaisir à ton âme.
\VS{18}Lorsqu'il n'y a pas de vision\FTNT{Le manque de vision n'est bon pour personne. Dieu donne une vision aux personnes qu'il a appelées. La vision peut être un songe, une directive, une prophétie, etc. Il s'agit des objectifs à atteindre.}, le peuple est sans frein, mais heureux est celui qui garde la loi !
\VS{19}L'esclave ne se corrige pas par des paroles, même s'il comprend, il n'obéit pas.
\VS{20}As-tu vu un homme irréfléchi dans ses paroles ? Il y a plus à espérer d'un insensé que de lui.
\VS{21}Le serviteur qu'on a traité délicatement dès sa jeunesse finit par se croire un fils.
\VS{22}L'homme coléreux excite des querelles, et l'homme furieux commet beaucoup de péchés.
\VS{23}L'orgueil de l'homme l'abaisse, mais celui qui est humble d'esprit obtient la gloire\FTNT{Mt. 23:12 ; Lu. 14:11 ; 1 Pi. 5:5.}.
\VS{24}Celui qui partage avec un voleur hait son âme ; il entend la malédiction, et il ne révèle rien.
\VS{25}La crainte qu'on a des hommes tend un piège, mais celui qui se confie en Yahweh est élevé dans une haute retraite.
\VS{26}Plusieurs recherchent la face de celui qui domine, mais c'est de Yahweh que vient le jugement des hommes.
\VS{27}L'homme inique est en abomination aux justes, et celui dont la voie est droite est en abomination au méchant.
\Chap{30}
\TextTitle{[Proverbe d'Agur]}
\VerseOne{}Les paroles d'Agur, fils de Jaké, à savoir la sentence prononcée par cet homme pour Ithiel, pour Ithiel et Ucal.
\VS{2}Certainement je suis le plus stupide de tous les hommes, et il n'y a pas en moi l'intelligence humaine.
\VS{3}Et je n'ai pas appris la sagesse ; et je n'ai pas la connaissance des saints.
\VS{4}Qui est celui qui est monté aux cieux, ou qui en est descendu\FTNT{Jn. 3:13 ; Ro. 10:6-7.} ? Qui est celui qui a recueilli le vent dans le creux de sa main, qui a serré les eaux dans son manteau, qui a dressé toutes les bornes de la terre ? Quel est son nom, et quel est le nom de son fils, le sais-tu ?
\VS{5}Toute la parole de Dieu est éprouvée ; il est un bouclier pour ceux qui se réfugient en lui\FTNT{Ps. 18:31 ; Ps. 115:9-11.}.
\VS{6}N'ajoute rien à ses paroles, de peur qu'il ne te reprenne et que tu ne sois trouvé menteur.
\VS{7}Je te demande deux choses : Ne me les refuse pas durant ma vie.
\VS{8}Eloigne de moi la vanité et la parole mensongère ; ne me donne ni pauvreté ni richesse, nourris-moi du pain qui m'est nécessaire.
\VS{9}De peur que dans l'abondance je ne te renie, et que je ne dise : Qui est Yahweh ? Ou que dans la pauvreté, je ne dérobe et que je ne porte atteinte au Nom de mon Dieu.
\VS{10}N'accuse pas un serviteur devant son maître, de peur que ce serviteur ne te maudisse, et qu'il ne t'en arrive du mal.
\VS{11}Il est une race de gens qui maudit son père et qui ne bénit pas sa mère.
\VS{12}Il est une race de gens qui croit être pure, et qui toutefois n'est point lavée de son ordure.
\VS{13}Il est une race de gens dont les yeux sont fort hautains, et les paupières élevées.
\VS{14}Il est une race de gens dont les dents sont des épées, et les mâchoires sont des couteaux pour dévorer les malheureux sur la terre et les pauvres d'entre les hommes.
\VS{15}La sangsue a deux filles qui disent : Apporte ! Apporte ! Il y a trois choses qui sont insatiables, il y en a même quatre qui ne disent point : C'est assez !
\VS{16}Le scheol, la matrice stérile, la terre qui n'est pas rassasiée d'eau, et le feu qui ne dit jamais : C'est assez !
\VS{17}L'oeil de celui qui se moque de son père et qui méprise l'enseignement de sa mère, les corbeaux des torrents le crèveront, et les petits de l'aigle le mangeront.
\VS{18}Il y a trois choses qui sont trop merveilleuses pour moi, même quatre, que je ne connais point :
\VS{19}La trace de l'aigle dans le ciel, la trace du serpent sur un rocher, le chemin d'un navire au milieu de la mer, et la trace de l'homme chez la jeune femme.
\VS{20}Telle est la trace de la femme adultère : Elle mange, et s'essuie la bouche, puis elle dit : Je n'ai pas commis d'iniquité.
\VS{21}La terre tremble pour trois choses, même pour quatre, qu'elle ne peut supporter :
\VS{22}Pour l'esclave quand il vient à régner, pour l'insensé quand il est rassasié de pain,
\VS{23}pour la femme odieuse quand elle se marie, et pour la servante quand elle hérite de sa maîtresse.
\VS{24}Il y a quatre choses des plus petites de la terre qui toutefois sont bien sages entre les sages :
\VS{25}Les fourmis, qui sont un peuple sans force, et qui néanmoins préparent durant l'été leur nourriture ;
\VS{26}les damans, qui sont un peuple qui n'est pas puissant, et qui néanmoins font leurs maisons dans les rochers ;
\VS{27}les sauterelles, qui n'ont point de roi, et qui toutefois sortent toutes par divisions ;
\VS{28}les lézards que tu peux saisir avec les mains, et qui sont pourtant dans les palais des rois.
\VS{29}Il y a trois choses qui ont une belle allure, même quatre, qui ont une belle démarche :
\VS{30}Le lion, qui est le plus fort d'entre les animaux, et qui ne recule pas à la rencontre de qui que ce soit ;
\VS{31}le cheval, qui a les flancs bien troussés ; le bouc, et le roi devant qui personne ne résiste.
\VS{32}Si tu t'es conduit follement en t'emportant, et si tu as des mauvaises intentions, mets la main sur ta bouche.
\VS{33}Comme celui qui bat le lait en fait sortir le beurre, et comme celui qui presse le nez en fait sortir le sang, ainsi celui qui provoque la colère excite la querelle.
\Chap{31}
\TextTitle{Proverbe de Lemuel}
\VerseOne{}Les paroles du roi Lemuel et l'instruction que sa mère lui donna.
\VS{2}Quoi, mon fils ? Quoi, fils de mes entrailles ? Eh quoi, mon fils, pour lequel j'ai tant fait de voeux ?
\VS{3}Ne livre pas ta vigueur aux femmes et tes voies à celles qui perdent les rois.
\VS{4}Lemuel, ce n'est point aux rois, ce n'est point aux rois de boire le vin, ni aux princes de boire la cervoise\FTNT{La cervoise est une bière faite avec de l'orge ou d'autres céréales.}
\VS{5}de peur qu'ayant bu, ils n'oublient ce qui a été prescrit, et qu'ils n'altèrent le jugement de tous les pauvres affligés.
\VS{6}Donnez de la cervoise à celui qui va périr, et du vin à celui qui a l'amertume dans le coeur ;
\VS{7}afin qu'il en boive, et qu'il oublie sa pauvreté, et ne se souvienne plus de ses peines.
\VS{8}Ouvre ta bouche en faveur du muet, pour la cause de tous les fils délaissés qui vont périr.
\VS{9}Ouvre ta bouche, fais justice, et plaide pour le malheureux et l'indigent.
\TextTitle{[La femme vertueuse]}
\VS{10}[Aleph.] Qui est-ce qui trouvera une femme vertueuse ? Car son prix surpasse de beaucoup les perles.
\VS{11}[Beth.] Le coeur de son mari a confiance en elle, et il ne manquera point de dépouilles.
\VS{12}[Guimel.] Elle lui fera du bien tous les jours de sa vie, et jamais du mal.
\VS{13}[Daleth.] Elle cherche de la laine et du lin, et elle en fait ce qu'elle veut avec ses mains.
\VS{14}[He.] Elle est semblable aux navires d'un marchand, elle amène son pain de loin.
\VS{15}[Vav.] Elle se lève lorsqu'il est encore nuit, elle donne la nourriture nécessaire à sa maison et elle donne à ses servantes leur tâche.
\VS{16}[Zayin.] Elle pense à un champ, et l'acquiert ; et elle plante la vigne du fruit de ses mains.
\VS{17}[Heth.] Elle ceint ses reins de force, et affermit ses bras.
\VS{18}[Teth.] Elle sent que ce qu'elle gagne est bon ; sa lampe ne s'éteint pas pendant la nuit.
\VS{19}[Yod.] Elle met sa main à la quenouille, et ses doigts tiennent le fuseau.
\VS{20}[Kaf.] Elle tend sa main au malheureux, elle tend ses mains à l'indigent.
\VS{21}[Lamed.] Elle ne craint point la neige pour sa famille, car toute sa famille est vêtue de vêtements doubles.
\VS{22}[Mem.] Elle se fait des couvertures, le fin lin et l'écarlate sont ce dont elle s'habille.
\VS{23}[Nun.] Son mari est considéré aux portes, lorsqu'il siège avec les anciens du pays.
\VS{24}[Samech.] Elle fait des chemises et les vend, et elle livre des ceintures au marchand.
\VS{25}[Ayin.] Elle est revêtue de force et de gloire, elle se rit du jour à venir.
\VS{26}[Pe.] Elle ouvre sa bouche avec sagesse, et la loi de la charité est sur sa langue.
\VS{27}[Tsade.] Elle veille sur ce qui se passe dans sa maison, et elle ne mange pas le pain de la paresse.
\VS{28}[Qof.] Ses fils se lèvent et la disent bienheureuse ; son mari aussi, et il la loue, en disant :
\VS{29}[Resh.] Plusieurs filles se sont conduites vertueusement, mais toi, tu les surpasses toutes.
\VS{30}[Shin.] La grâce est trompeuse et la beauté vaine, mais la femme qui craint Yahweh est celle qui sera louée.
\VS{31}[Tav.] Récompensez-la du fruit de ses mains, et que ses oeuvres la louent aux portes.
\PPE{}
\end{multicols}

%\clearpage\ShortTitle{Job}\BookTitle{Job}\BFont
\noindent\hrulefill
{\footnotesize
\textit{
\bigskip
{\centering{}
\\Auteur : Inconnu
\\(Heb. : Iyov)
\\Signification : haï, ennemi et « je m'exclamerai »
\\Thème : La souffrance
\\Date de rédaction : Incertaine\\}
}
%\bigskip
\textit{
\\Job était un homme prospère et intègre auquel Dieu rendit témoignage. Il subit une succession de malheurs en très peu de temps en perdant tout ce qui lui était cher. Après avoir cherché à se justifier et subi les railleries de sa femme et les accusations de ses amis, Job s'humilia devant Dieu et comprit l'impuissance de sa propre justice. Cette histoire, dont on n'a aucune indication spatio-temporelle et qui pourtant parle à tous, est un encouragement pour le juste éprouvé.
%\bigskip
\\Rappelant que la souffrance peut être le moyen choisi par Dieu pour enseigner et se révéler, ce récit illustre la fidélité et la bonté de Yahweh envers ceux qui le craignent.\bigskip
}
}
\par\nobreak\noindent\hrulefill
\begin{multicols}{2}
\Chap{1}
\TextTitle{Job et sa famille}
\VerseOne{}Il y avait dans le pays d'Uts\FTNT{Ge. 36:28} un homme dont le nom était Job\FTNT{Ez. 14:14; Ja. 5:11.}. Cet homme était intègre\FTNT{1 R. 8:61.} et droit, craignant\FTNT{Ps. 19:10; Pr. 1:7.} Dieu et se détournant du mal.
\VS{2}Il eut sept fils et trois filles.
\VS{3}Et son bétail était de sept mille brebis, trois mille chameaux, cinq cents paires de bœufs, cinq cents ânesses, avec un très grand nombre de serviteurs\FTNT{Job 42:12-13.}; tellement que cet homme était le plus puissant de tous les Orientaux.
\VS{4}Or ses fils allaient et faisaient des festins les uns chez les autres chacun à son jour, et ils envoyaient appeler leurs trois sœurs pour manger et boire avec eux.
\VS{5}Quand les jours de festin étaient passés, Job envoyait chercher ses fils pour les sanctifier, et se levant de bon matin, il offrait un holocauste selon le nombre de ses enfants ; car Job disait : Peut-être mes fils ont-ils péché, et ont-ils blasphémé contre Dieu dans leurs cœurs. Job faisait toujours ainsi.\FTNT{Job 42:8.}
\VS{6}Or, il arriva un jour que les fils de Dieu\FTNT{Ps. 89:7 ; Job 38:7.} vinrent se présenter devant Yahweh, et Satan\FTNT{Es. 14:12; Ap. 12:9-10.} aussi vint au milieu d'eux.
\VS{7}Yahweh dit à Satan : D'où viens-tu ? Et Satan répondit à Yahweh : De courir çà et là sur la terre et de m'y promener\FTNT{1 Pi. 5:8.}.
\VS{8}Yahweh dit à Satan : N'as-tu point considéré mon serviteur Job, qui n'a point d'égal sur la terre ; homme intègre et droit, craignant Dieu, et se détournant du mal ?
\VS{9}Et Satan répondit à Yahweh : Est-ce en vain que Job craint Dieu ?
\VS{10}N'as-tu pas mis une haie \FTNT{La haie est une protection, une barrière végétale entretenue afin de protéger et de clôturer un terrain.} tout autour de lui, autour de sa maison, autour de tout ce qui lui appartient ? Tu as béni l'œuvre de ses mains, et ses troupeaux se répandent sur la terre.
\VS{11}Mais étends maintenant ta main, touche à tout ce qui lui appartient, et tu verras s'il ne te maudit pas en face.
\VS{12}Et Yahweh dit à Satan : Voilà, tout ce qui lui appartient est en ton pouvoir ; seulement ne porte pas la main sur lui. Et Satan sortit de devant la face de Yahweh\FTNT{1 R. 22:22.}.
\TextTitle{Première attaque de Satan}
\VS{13}Il arriva donc qu'un jour comme les fils et les filles de Job mangeaient et buvaient du vin dans la maison de leur frère aîné, un messager vint vers Job,
\VS{14}et lui dit : Les bœufs labouraient, et les ânesses paissaient à côté d'eux;
\VS{15}et ceux de Séba se sont jetés dessus, les ont pris, et ont frappé les serviteurs au fil de l'épée. Et je me suis échappé, moi seul, pour te l'annoncer.
\VS{16}Cet homme parlait encore, lorsqu'un autre vint et dit : Le feu de Dieu est tombé du ciel, il a brûlé les brebis et les serviteurs, et les a consumés\FTNT{2 R. 1:10-12.}. Et je me suis échappé moi seul, pour te l'annoncer.
\VS{17}Cet homme parlait encore, lorsqu'un autre vint et dit : Des Chaldéens\FTNT{Ge. 11:28.} ont fait trois bandes, se sont jetés sur les chameaux et les ont pris, ils ont frappé les serviteurs au fil de l'épée, et je me suis échappé moi seul, pour te l'annoncer.
\VS{18}Cet homme parlait encore, lorsqu'un autre vint et dit : Tes fils et tes filles mangeaient et buvaient du vin dans la maison de leur frère aîné ;
\VS{19}voici, un grand vent est venu de l'autre côté du désert et a frappé contre les quatre coins de la maison ; elle est tombée sur les jeunes gens, et ils sont morts. Et je me suis échappé, moi seul, pour te l'annoncer.
\VS{20}Alors Job se leva, déchira\FTNT{Job 2:12 ; Est. 4:1.} son manteau et  rasa la tête ; et se jetant par terre, se prosterna,
\VS{21}et dit : Je suis sorti nu du ventre de ma mère, et nu je retournerai dans le sein de la terre\FTNT{Ec. 5:14. ; 1 Ti. 6:7.} ; Yahweh a donné, Yahweh a enlevé\FTNT{1 S. 2:6.} ; que le nom de Yahweh soit béni !
\VS{22}En tout cela, Job ne pécha pas et n'attribua rien d'injuste à Dieu.
\Chap{2}
\TextTitle{Deuxième attaque de Satan}
\VerseOne{}Or il arriva un jour que les fils de Dieu vinrent un jour se présenter devant Yahweh, Satan\FTNT{Za. 3:1-2.} vint aussi au milieu d'eux se présenter devant Yahweh.
\VS{2}Yahweh dit à Satan : D'où viens-tu ? Satan répondit à Yahweh : De courir çà et là sur la terre et de m'y promener.
\VS{3}Yahweh dit à Satan: N'as-tu point considéré mon serviteur Job,qui n'a point d'égal sur la terre ; homme sincère et droit, craignant Dieu, et se détournant du mal ? Il demeure ferme dans son intégrité, quoique tu m'aies incité contre lui à le détruire sans cause\FTNT{Job 9:17.}.
\VS{4}Et Satan répondit à l’Eternel, en disant : Chacun donnera peau pour peau, et tout ce qu’il a, pour sa vie.
\VS{5}Mais étends maintenant ta main, et frappe à ses os et à sa chair\FTNT{Job 19:20.}, et tu verras s'il ne te maudit pas en face. 
\VS{6}Yahweh dit à Satan : Voici, il est en ta main : Seulement garde sa vie.
\VS{7}Ainsi Satan sortit de devant l’Eternel, et frappa Job d’un ulcère malin, depuis la plante de ses pieds jusqu’au sommet de la tête.
\VS{8}Job prit un tesson pour se gratter et s'assit au milieu de la cendre\FTNT{Jé. 6:26 ; Jon. 3:6.}.
\TextTitle{Réaction de Job et de sa femme}
\VS{9}Et sa femme lui dit : Conserveras-tu encore ton intégrité ?  Bénis\FTNT{Job 1:11.} Dieu, et meurs !
\VS{10}Et il lui dit : Tu parles comme une femme insensée ! Nous recevons le bien de la part de Dieu, et nous n'en recevrions pas le mal !\FTNT{Es. 45:7 ; Am. 3:6 ; La. 3:37.} En tout cela, Job ne pécha pas par ses lèvres.
\TextTitle{Job et ses trois amis}
\VS{11}Et trois des amis de Job, Eliphaz de Théman, Bildad de Schuach, et Tsophar de Naama, ayant appris tous les maux qui lui étaient arrivés, vinrent chacun du lieu de leur demeure, après s'être convenus ensemble d'un jour pour venir le plaindre et le consoler.
\VS{12}Ayant de loin levé les yeux sur lui, ils ne le reconnurent pas, alors ils élevèrent la voix et ils pleurèrent. Ils déchirèrent leurs manteaux, et jetèrent de la poussière vers le ciel au-dessus de leur tête.
\VS{13}Et ils s'assirent à terre avec lui, sept jours et sept nuits, et aucun d'eux ne lui dit une parole, car ils voyaient que sa douleur était fort grande.
\Chap{3}
\TextTitle{Lamentations de Job}
\VerseOne{}Après cela, Job ouvrit la bouche et maudit le jour de sa naissance.\FTNT{Jé. 20:14 ; Job 10:18.}
\VS{2}Car prenant la parole, il dit :
\VS{3}Périsse le jour où je suis né, et la nuit qui a dit : Un homme est conçu !
\VS{4}Que ce jour-là ne soit que ténèbres ; que Dieu ne le recherche point d'en haut, et qu'il ne soit point éclairé de la lumière ! 
\VS{5}Que les ténèbres et l'ombre de la mort\FTNT{Job 10:21-22.} s'en emparent, que les nuées demeurent sur lui, qu'il soit rendu terrible comme le jour de ceux à qui la vie est amère ! 
\VS{6}Que l'obscurité prenne cette nuit, qu'elle ne se réjouisse pas au milieu des jours de l'année, qu'elle n'entre pas dans le compte des mois !
\VS{7}Voici, que cette nuit soit stérile, et qu'aucun cri de joie n'y survienne !
\VS{8}Qu'ils la maudissent ceux qui maudissent les jours, ceux qui sont prêts à réveiller le Léviathan !
\VS{9}Que les étoiles de son crépuscule soient obscurcies ; qu'elle attende la lumière, mais qu'il n'y en ait point, et qu'elle ne voie point les rayons de l'aube du jour ! \FTNT{Job 41:9.} !
\VS{10}Parce qu'elle n'a pas fermé le sein qui me conçut ni caché la souffrance à mes yeux.
\VS{11}Pourquoi ne suis-je pas mort dans le sein de ma mère ? Pourquoi n'ai-je pas expiré aussitôt que je suis sorti de ses entrailles ?\FTNT{Job 10:18.}
\VS{12}Pourquoi des genoux m'ont-ils reçu? Pourquoi des mamelles m'ont-elles allaité ?
\VS{13}Je serais couché maintenant, je h serais tranquille, je dormirais, je me reposerais\FTNT{Job 17:16.},
\VS{14}avec les rois et les grands de la terre, qui se bâtirent des mausolées,
\VS{15}avec les princes qui possedèrent de l'or, et qui remplirent d'argent leurs maisons.
\VS{16}Ou comme l'avorton caché, je n'existerais pas\FTNT{Ps. 58:9.}, comme les petits enfants qui n'ont pas vu la lumière.
\VS{17}Là les méchants n'agitent plus personne, et là se reposent ceux qui sont fatigués. 
\VS{18}Pareillement ceux qui avaient été dans les liens, jouissent là du repos, et n'entendent plus la voix de l'oppresseur. 
\VS{19}le petit et le grand sont là, et l'esclave est délivré de son maître.
\VS{20}Pourquoi la lumière est-elle donnée au misérable, et la vie à ceux qui ont le cœur dans l'amertume ;
\VS{21}qui désirent en vain la mort, et qui la recherchent plus que le trésor,\FTNT{Ap. 9:6.}
\VS{22}qui seraient ravis de joie et seraient dans l'allégresse s'ils avaient trouvé le tombeau ?
\VS{23}Pourquoi, dis-je, la lumière est-elle donnée à l'homme à qui le chemin est caché, et que Dieu a enfermé de toutes parts\FTNT{Job 19:8 ; La. 3:7.} ?
\VS{24}Car avant que je mange, mon soupir vient, et mes cris se répandent comme de l'eau. 
\VS{25}Ce que je crains le plus, m'arrive, et ce que je redoute le plus, m'atteint. 
\VS{26}Je n'ai point eu de paix, je n'ai point eu de repos, ni de calme, depuis que ce trouble m'est arrivé. 
\Chap{4}
\TextTitle{Premier discours d'Eliphaz}
\VerseOne{}Alors Eliphaz de Théman prit la parole et dit :
\VS{2}Si l'on tente de te parler, en seras-tu peiné ? Mais qui pourrait retenir ses paroles ?
\VS{3}Voici, tu as souvent instruit les autres, et tu as fortifié les mains affaiblies\FTNT{Es. 35:3 ; Hé. 12:12.},
\VS{4}Tes paroles ont affermi ceux qui chancelaient, et tu as fortifié les genoux qui pliaient\FTNT{Job 16:5.}.
\VS{5}Et maintenant que le malheur t'arrive, tu faiblis ! Maintenant que tu es atteint, tu en es tout troublé !
\VS{6} Ta crainte de Yahweh n'a-t-elle pas été ton espérance ? Et l'intégrité de tes voies n'a-t-elle pas été ton attente ? 
\VS{7}Rappelle, je te prie, dans ton souvenir : Quel est l'innocent qui a péri ? Quels sont les justes qui ont été exterminés ?\FTNT{Job 8:20.}
\VS{8}Selon ce que j'ai vu, ceux qui labourent l'iniquité et qui sèment la peine en moissonnent les fruits ;\FTNT{Job 15:35 ; Ga. 6:7.}
\VS{9}ils périssent par le souffle de Dieu, et ils sont consumés par le vent de ses narines.\FTNT{Ex. 15:8 ; Es. 11:4 ; 30:33 ; Job 15:30 ; 2 Th. 2:8.}
\VS{10}Il étouffe le rugissement du lion, et le cri d'un grand lion, et il arrache les dents des lionceaux ;
\VS{11}le lion périt faute de proie, et les petits de la lionne sont dispersés.
\VS{12}Une parole m'est furtivement arrivée, et mon oreille en a saisi les sons légers.
\VS{13}Au moment où les visions de la nuit agitent la pensée, quand un profond sommeil tombe sur les hommes\FTNT{Job 33:15.},
\VS{14}une frayeur et un tremblement me saisirent, et tous mes os tremblèrent.
\VS{15}Un esprit passa devant moi, et mes cheveux en furent tout hérissés. 
\VS{16}Il se tint là et je ne reconnus pas son visage ; une figure était devant mes yeux. Et j'entendis un léger murmure et une voix :
\VS{17}L'homme serait-il juste devant Dieu ? L'homme serait-il pur devant celui qui l'a fait ?\FTNT{Job 25:4.}
\VS{18}Voici, il ne se fie pas à ses serviteurs, il trouve des erreurs à ses anges\FTNT{Job 15:15 ; Job 25:5 ; 2 Pi 2:4. }
\VS{19}combien plus chez ceux qui habitent des maisons d'argile, qui ont leurs fondements dans la poussière, qu'on écrase comme des vermisseaux !\FTNT{Job 25:6.}
\VS{20}Du matin au soir ils sont brisés, et, sans qu'on s'en aperçoive, ils périssent pour toujours. 
\VS{21}L'excellence qui était en eux, n'a-t-elle pas été emportée ? Ils meurent sans être sages. 
\Chap{5}
\VerseOne{}Crie maintenant ! Y aura-t-il quelqu'un qui te réponde ? Et vers quel saint te tourneras-tu ?\FTNT{Job 15:15.}
\VS{2}La colère tue l'insensé, et le fou meurt dans ses emportements.
\VS{3}J'ai vu l'insensé qui s'enracinait \FTNT{Jé. 12:1-2.}, mais j'ai aussitôt maudit sa demeure.
\VS{4}Ses fils sont loin de tout secours ; ils sont écrasés à la porte, et personne ne les délivre !\FTNT{Ps. 119:155.}
\VS{5} Sa moisson est dévorée par l'affamé, qui même la ravit d'entre les épines ; et le voleur convoite ses biens.
\VS{6}Le malheur ne sort pas de la poussière, et le travail ne germe pas de la terre ;
\VS{7}l'homme naît pour la peine\FTNT{Ge. 3:17-19 ; Job 14:1-5.}, comme l'étincelle pour voler et s'élever.
\VS{8}Mais moi, j'aurais recours à Dieu, et j'adresserais ma parole à Dieu.
\VS{9}Il fait de grandes choses qu'on ne peut sonder, de merveilleuses choses qu'on ne peut compter\FTNT{Ps. 72:18. Ps. 92:5 ; Job 9:10.}.
\VS{10}Il répand la pluie sur la face de la terre, et envoie les eaux sur les campagnes\FTNT{De. 28:12 ; Ps. 135:7 ; Job 28:26; Job 38:25-26 ; Ac. 14:17.};
\VS{11}il met en haut ceux qui sont abaissés, et délivre les affligés\FTNT{1 S. 2:7; Ez. 21:31 ; Ps. 113:7-8.} ;
\VS{12}il anéantit les projets des hommes rusés, de sorte qu'ils ne viennent pas à bout de leurs entreprises\FTNT{Es. 8:10 ; Ps. 33:10 ; Né. 4:15.} ;
\VS{13}il prend les sages dans leur propre ruse\FTNT{1 Co. 3:19.}, et les desseins des hommes pervers sont renversés :
\VS{14}De jour ils rencontrent les ténèbres, et ils marchent à tâtons en plein midi, comme dans la nuit.
\FTNT{De. 28:29.}.
\VS{15}Ainsi Dieu délivre le pauvre de l'épée de leur bouche, et le sauve de la main des puissants\FTNT{Ps. 12:3-4; Ps 52:2; Ps. 57:4.} ;
\VS{16}et l'espérance soutient le malheureux\FTNT{1S. 2:8.}, et la méchanceté a la bouche fermée\FTNT{Es. 52:15 ; Ps. 63: 11; Ps. 107: 42; Pr. 10:6.}.
\VS{17}Voici, heureux est celui que Yahweh châtie ! Ne rejette donc point le châtiment de Yahweh.
\FTNT{Ps 94:12 ; Pr.3:11-12 ; Hé. 12:5-6; Ap. 3:19.}.
\VS{18}Car c'est lui qui fait la plaie, et la bande ; il blesse et ses mains guérissent\FTNT{De. 32:39; 1S. 2: 6-7 ; Cp. Es. 30:26 ; Os. 6:1.}.
\VS{19}Six fois il te délivrera de l'angoisse, et sept fois le mal ne te touchera pas\FTNT{Ps 34:20; Ps. 91:3; Pr.24:16.}.
\VS{20}Il te sauvera de la mort pendant la famine, et du tranchant de l'épée pendant la guerre\FTNT{Ps. 33: 19; Ps. 37:19.}.
\VS{21}Tu seras à l'abri du fléau de la langue, et tu n’auras point peur de la dévastation, quand elle arrivera.
\FTNT{Ps. 31:21.}.
\VS{22}Tu riras de la dévastation et de la famine, et tu n'auras pas peur des bêtes de la terre\FTNT{Es. 65:25; Ez. 34:25; Os. 2:20.};
\VS{23}car tu feras une alliance avec les pierres des champs, et les bêtes des champs seront en paix avec toi\FTNT{Os. 2:20.}.
\VS{24}Tu jouiras en paix de la prospérité sous ta tente, tu pourvoiras à ta demeure et tu n'y seras point trompé ;
\VS{25}tu verras ta postérité s'accroître, et tes descendants se multiplier comme l'herbe de la terre\FTNT{Ps. 72:16; Ps. 127: 3-5; Ps. 128:6.}.
\VS{26}Tu entreras au tombeau dans ta vieillesse, comme une gerbe qu'on emporte en son temps\FTNT{Pr. 9:11; Pr. 10:27.}.
\VS{27}Voilà ce que nous avons examiné, voilà ce qui est ; à toi d'entendre et de choisir.
\Chap{6}
\TextTitle{Réponse de Job}
\VerseOne{}Job prit la parole et dit :
\VS{2}Oh ! si l’on pesait ma douleur, et si l’on mettait en même temps mes calamités dans la balance !
\VS{3}Car elle serait plus pesante que le sable de la mer ; c’est pourquoi mes paroles sont englouties.
\FTNT{Pr. 27:3.} !
\VS{4}Car les flèches du Tout-Puissant sont sur moi, mon âme en boit le venin ; les terreurs\FTNT{Job 30:15 ; Ps. 88:16-17.} de Dieu se rangent en bataille contre moi\FTNT{Job 19:12 ; Ps. 38:2-3.}.
\VS{5}L'âne sauvage\FTNT{Job 39:8.} brait-il auprès de l’herbe ? Le bœuf mugit-il auprès de son fourrage ?
\VS{6}Mange-t-on sans sel ce qui est fade ? Trouve-t-on du goût dans un blanc d’œuf ?
\VS{7}Ce que mon âme voudrait ne pas toucher, c'est là ma nourriture, si dégoûtante soit-elle !
\VS{8}Oh ! Puisse ma prière s'accomplir et Dieu me donner ce que j'attends !
\VS{9}Qu’il plaise à Dieu de me réduire en poussière, qu’il laisse aller sa main pour m’achever !
!\FTNT{Job 7:16; 9:21; 10:1; cp. No. 11:15; 1R. 19:4; Jon. 4:3, 8.}
\VS{10}Mais j’ai encore cette consolation, quoique la douleur me consume, et qu’elle ne m’épargne point, je n'ai pas transgressé les paroles du Saint.
\VS{11}Quelle est ma force pour que j’espère, et quelle est ma fin pour que je prenne patience ?
\VS{12}Ma force est-elle une force de pierre ? Ma chair est-elle d'airain ?
\VS{13}Ne suis-je pas sans secours, et le salut n'est-il pas loin de moi ?
\VS{14}A celui qui souffre, est due la compassion de son ami ; mais il a abandonné la crainte. \FTNT{Ps. 19:10.} du Tout-Puissant\FTNT{Pr. 17:17.}.
\VS{15}Mes frères m'ont trompé comme un torrent, comme le lit des torrents qui passent\FTNT{Ps. 38:12; Ps 41:10; Ps 69:9; Jé. 15:19.}.
\VS{16}Les glaçons en troublent le cours, la neige s'y cache ;
\VS{17}mais au temps de la sécheresse, ils tarissent, et dans les chaleurs, ils disparaissent de leur place.
\VS{18}Les caravanes se détournent de leur route, elles montent dans le désert et périssent.
\VS{19}Les caravanes de Théma\FTNT{Ge. 25:15.} fixent le regard, les voyageurs de Séba\FTNT{1R. 10:1; Ps. 72:10; Ez. 27:22-23.} s'attendent à eux;
\VS{20}ils sont honteux d'avoir eu cette confiance, ils restent confondus quand ils arrivent.
\VS{21}Certes, vous m'êtes devenus inutiles ; vous voyez mon angoisse, et vous en avez horreur !\FTNT{Job 19:13 ; Ps. 31:12.}
\VS{22}Mais vous ai-je dit : Donnez-moi quelque chose, et de vos biens, faites des présents en ma faveur ? 
\VS{23}délivrez-moi de la main de l'ennemi, et rachetez-moi de la main des violents ?
\VS{24}Instruisez-moi, et je me tairai ; faites-moi comprendre en quoi je me suis égaré.
\VS{25}Ô combien sont fortes les paroles de vérité ! Mais que veut censurer votre argumentation ?
\VS{26}Voulez-vous donc blâmer ce que j'ai dit, et ne voir que du vent dans les paroles d'un homme désespéré ?\FTNT{Ec. 9:16.}
\VS{27}Vous vous jetez même sur un orphelin, vous persécutez votre ami.
\VS{28}Regardez-moi, je vous prie ! Et voyez si je vous mens en face ?
\VS{29}Revenez\FTNT{Job 17:10.} donc, soyez sans injustice; revenez, et reconnaissez mon innocence\FTNT{Job 27:5-6 ; 34:5 ; cp. Job 23:10 ; 42:1-6.}.
\VS{30}Y a-t-il de l'injustice dans ma langue ? et mon palais ne sait-il pas discerner le mal ? 
\Chap{7}
\VerseOne{}N'y a-t-il pas un temps de guerre limité à l'homme sur la terre ? Et ses jours ne sont-ils pas comme les jours d'un mercenaire ?
\VS{2}Comme un esclave, il soupire après l'ombre, comme un mercenaire\FTNT{Es. 16:14.}, il attend son salaire\FTNT{Ps. 39:5.}.
\VS{3}Ainsi j'ai reçu en partage des mois en vain, et l'on m'a assigné des nuits de peine\FTNT{Ps. 6:6.}.
\VS{4}Si je suis couché, je dis : Quand me lèverai-je ? Quand finira la nuit ? Et je suis rassasié d'agitations jusqu'au point du jour\FTNT{De. 28:67.}.
\VS{5}Ma chair se couvre de vers et d'une croûte terreuse, ma peau se crevasse et coule.
\VS{6}Mes jours sont plus rapides que la navette du tisserand, ils se consument sans espoir !\FTNT{Es. 38:12 ; Job 9:25 ; 17:11; Ja. 4:14.}
\VS{7}Souviens-toi que ma vie est un souffle ! Et que mes yeux ne reverront plus le bonheur\FTNT{Es. 40:6 ; Ps. 78:39 ; Ps. 89:48 ; Ps. 102: 12 ; Ps. 103:15 ; Job 8:9 ; Job 14:1-2 ; 1P. 1:24.}.
\VS{8}L'œil de ceux qui me regarndent ne me verra plus ; tes yeux seront sur moi, et je ne serai plus.
\VS{9}La nuée se dissipe et s'en va, ainsi celui qui descend au scheol\FTNT{cp. Ha. 2:5 ; Lu. 16:23.} ne remontera pas\FTNT{Job 10:21-22 ; Job 14:7-14.};
\VS{10}il ne reviendra plus dans sa maison, et le lieu qu'il habitait ne le reconnaîtra plus\FTNT{Ps. 37:35-36 ; Ps. 103:16 ; Job 10:21.}.
\VS{11}C'est pourquoi, je ne retiendrai pas ma bouche, je parlerai dans l'angoisse de mon esprit, je me plaindrai dans l'amertume de mon âme\FTNT{Job 10:1.}.
\VS{12}Suis-je une mer ? Suis-je un monstre marin, pour que tu poses autour de moi des gardes ?
\VS{13}Quand je dis : Mon lit me consolera, ma couche calmera ma plainte,
\VS{14}alors tu me terrifies par des songes, et tu m'épouvantes par des visions.
\VS{15}C'est pourquoi je choisirais d'être étranglé, et de mourir, plutôt que de conserver mes os.
\VS{16}Je les méprise !… Je ne vivrai pas toujours… Laisse-moi car mes jours sont un souffle\FTNT{Job 10:20.}.
\VS{17}Qu'est-ce que l'homme pour que tu en fasses tant de cas, pour que tu poses ta main sur son cœur,\FTNT{Ps. 8:5 ; Ps. 144:3 ; Hé. 2:6.}
\VS{18}pour que tu le visites tous les matins, pour que tu l'éprouves\FTNT{Job 23:10.} à chaque instant ?
\VS{19}Quand finiras-tu de me regarder? Ne me lâcheras-tu pas, pour que j'avale ma salive ?\FTNT{Job 9:18.}
\VS{20}J'ai péché ; que te ferai-je, gardien des hommes ?  \FTNT{1 Ti. 4:10.}  Pourquoi m'as-tu mis en butte à tes coups, et pourquoi suis-je à charge à moi-même ?
\VS{21}Et pourquoi ne pardonnes-tu pas mon péché, et ne fais-tu pas passer mon iniquité ? Car je vais maintenant me coucher dans la poussière ; tu me chercheras, et je ne serai plus.
\Chap{8}
\TextTitle{Premier discours de Bildad}
\VerseOne{}Bildad de Schuach prit la parole et dit :
\VS{2}Jusqu'à quand parleras-tu ainsi, et les paroles de ta bouche seront-elles un vent impétueux ?\FTNT{Job 15:2.}
\VS{3}Dieu renverserait-il le droit, et le Tout-puissant renverserait-il la justice ? \FTNT{Cp. Ge. 18:25.} ?\FTNT{De. 32:4 ; Job 34:12 ; Da. 9:14 ; 2 Ch. 19:7.}
\VS{4}Si tes fils ont péché contre lui, il les a livrés à leur crime.
\VS{5}Mais toi si tu cherches Dieu, si tu demandes grâce au Tout-Puissant ;\FTNT{Cp. Job 5:17-27.}
\VS{6}si tu es pur et droit, il veillera certainement sur toi, il rendra le bonheur à la demeure de ta justice ;
\VS{7}tes commencements\FTNT{Za. 4:10.} auront été peu de chose, et ta fin sera bien plus grande.\FTNT{Job 42:12.}
\VS{8}Interroge ceux des générations précédentes, applique-toi à l'expérience de leurs pères.\FTNT{De. 4:32 ; De. 32:7.}
\VS{9}Car nous sommes d'hier, et nous ne savons rien, parce que nos jours sur la terre ne sont qu'une ombre.\FTNT{Ps. 102:12 ; Ps. 144:41 ; Ch. 29:15.}
\VS{10}Ils t'instruiront, ils te parleront, ils tireront de leur cœur ces discours :
\VS{11}Le roseau croît-il sans marais ? Le jonc pousse-t-il sans eau ?
\VS{12}Il est encore en sa verdure, sans qu'on le coupe, il sèche plus vite que toutes les herbes.\FTNT{Cp. Jé. 17:5-8 ; Ps. 129:6.}
\VS{13}Ainsi est la voie de tous ceux qui oublient Dieu\FTNT{Ps. 9:18.}, et l'espérance de l'impie périra\FTNT{Ps. 1:4 ; Ps. 112:10 ; Pr. 10:28 ; Job 11:20 ; Job 27:8.}.
\VS{14}Sa confiance est brisée, son soutien est une toile d'araignée.
\VS{15}Il s'appuie sur sa maison, et elle ne tient pas ; il s'y cramponne, et elle ne reste pas debout.
\VS{16}Dans toute sa vigueur, en plein soleil, il étend ses rameaux sur son jardin,
\VS{17}mais ses racines s'entrelacent parmi des monceaux de pierres, il pénètre dans les rochers.
\VS{18}S'Il l'ôte de sa place, celle-ci le renie, disant je ne t'ai pas connu ! 
\VS{19}Telle est la joie que ses voies lui procurent. Puis sur le même sol, d'autres s'élèvent après lui.
\VS{20}Dieu ne rejette pas l'homme intègre, il ne soutient pas la main des méchants.\FTNT{Job 4:7.}
\VS{21}Il remplira encore ta bouche de cris de joie, et tes lèvres de chants d'allégresse.\FTNT{Ps. 126:2.}
\VS{22}Ceux qui te haïssent seront revêtus de honte, et la tente des méchants ne sera plus. \FTNT{Ps. 35:26 ; Ps. 109:29.}
\Chap{9}
\TextTitle{Réponse de Job}
\VerseOne{}Job prit la parole et dit :
\VS{2}Certainement, je sais qu'il en est ainsi ; et comment l'homme mortel se justifierait-il devant Dieu ? \FTNT{Ha. 2:4 ; Ga. 3:11 ; Ro. 1:17 ; Hé. 10:38.} devant Dieu ?\FTNT{Ps. 25:4 ; Ps. 143:2 ; Job 15:14-16 ; Da. 9:11 ; Ro 3:19.}
\VS{3}S'il veut plaider avec lui, il ne lui répondra pas une fois sur mille. \FTNT{Es. 45:9-10.}
\VS{4} Dieu est sage de coeur, et puissant en force. Qui est-ce qui s'est opposé à lui, et s'en est bien trouvé ? \FTNT{Job 12:13 ; Job 36:5 ; Job 37:23.}
\VS{5}Il transporte les montagnes, et quand il les renverse dans sa fureur, elles n'en connaissent rien.\FTNT{Ps. 144:5.}
\VS{6}Il remue la terre de sa place, et ses piliers sont ébranlés.\FTNT{Ag. 2:6, 21 ; Hé. 12:26.}
\VS{7}Il commande au soleil, et le soleil ne se lève pas ; et il met un sceau sur les étoiles.\FTNT{Jos. 10:12.}
\VS{8} C'est lui seul qui étend les cieux\FTNT{Ge 1:6-8 ; Es. 44:24; Es. 51:13 ; Ps. 104:2.},qui marche sur les hauteurs de la mer\FTNT{Cp. Mt. 14:25.}.
\VS{9}Il a fait la grande ourse, l'orion, les pléiades, et les étoiles des régions australes.\FTNT{Ge. 1:16 ; Am. 5:8 ; Ps. 89:12 ; Job 38:31-32.}
\VS{10}Il fait de grandes choses qu'on ne peut sonder, des merveilles sans nombre.\FTNT{Ps. 86:10 ; Ps. 139:6, 17-18 ; Job 5:9 ; Job 37:5.}
\VS{11}Voici, il passe près de moi, et je ne le vois pas ; il passe encore, et je ne l'aperçois pas.\FTNT{Job 23:8-9 ; 35:14.}
\VS{12}S'il enlève, qui l'en détournera? Qui lui dira : Que fais-tu ?\FTNT{Es. 45: 9-10 ; Da. 4:35 ; Ro. 11:33-35.}
\VS{13}Dieu ne revient pas sur sa colère ; sous lui s'inclinent les appuis de l'orgueil.\FTNT{Job 26:12; Cp. Es. 30:7.}
\VS{14}Combien moins lui répondrais-je, moi et comment choisirais-je mes paroles contre lui ? 
\VS{15}Quand je serais juste, je ne répondrais pas ; je demanderais grâce à mon juge.\FTNT{Job 23:1-7.}
\VS{16}Si je l'invoque et qu'il me réponde, ne croirais-je pas qu'il ait écouté ma voix,
\VS{17}lui qui m'assaille comme par une tempête, qui multiplie mes plaies sans motif,\FTNT{Job 6:29.}
\VS{18}qui ne me permet pas de reprendre haleine ; qui me rassasie d'amertume.\FTNT{Job 7:19.}
\VS{19}S'il est question de savoir qui est le plus fort ; voilà, il est fort ; et s'il est question d'aller en justice, qui est-ce qui m'y fera comparaître ? 
\VS{20}Si je me justifie, ma propre bouche me condamnera ; si je me fais parfait, il me convaincra d'être coupable.
\VS{21}Je suis innocent ! Je ne me soucie pas de vivre, je méprise ma vie.\FTNT{Job 10:1.}
\VS{22}Tout se vaut! C'est pourquoi j'ai dit: Il détruit l'innocent comme l'impie. \FTNT{ Cp. Ez. 21:3 ; Ec. 9:2-3 ; Mt 5:45.}
\VS{23}Au moins si le fléau dont il frappe faisait mourir tout aussitôt ; mais il se rit de l'épreuve des innocents. 
\VS{24}[C'est par lui que] la terre est livrée entre les mains du méchant ; c'est lui qui couvre la face des juges de la [terre] ; et si ce n'est pas lui, qui est-ce donc ? 
\VS{25}Or mes jours vont plus vite qu'un courrier ; ils s'en fuient sans avoir vu le bonheur ;\FTNT{Job 7:6-7.}
\VS{26}ils passent comme les navires de roseaux, comme l'aigle qui fond sur sa proie.
\VS{27}i je dis : J'oublierai ma plainte, je renoncerai à ma colère, je me fortifierai; 
\VS{28}Je suis épouvanté de tous mes tourments. Je sais que tu ne me jugeras pas innocent. .\FTNT{Cp. Ps. 130:3.}
\VS{29}Je serai jugé coupable ; pourquoi travaillerais-je en vain ?
\VS{30}Quand je me laverais dans de l'eau de neige, et que je nettoierais mes mains dans la pureté, \FTNT{Jé. 2:22.}
\VS{31}tu me plongerais dans le fossé, et mes vêtements m'auraient en horreur.
\VS{32}Car il n'est pas comme moi un homme, pour que je lui réponde, [et] que nous allions ensemble en jugement. .\FTNT{Es. 45:9 ; Jé 49:19 ; Ec. 6:10; Ro. 9:20.}
\VS{33}Mais il n'y a personne qui prend connaissance de la cause qui serait entre nous, et qui pose la main sur nous deux. \FTNT{Cp. 1 S. 2:25.}
\VS{34}Qu'il ôte donc sa verge de dessus moi, et que la frayeur que j'ai de lui ne me trouble plus. ;
\VS{35}Je parlerai, et je ne le craindrai pas ; mais dans l'état où je suis je ne suis plus à moi-même. 
\Chap{10}
\VerseOne{}Mon âme a pris en dégoût la vie ! Je laisserai aller ma plainte, je parlerai dans l'amertume de mon âme.
\VS{2}Je dirai à Dieu : Ne me condamne pas ; montre-moi pourquoi tu plaides contre moi ?
\VS{3}Te plais-tu à m'opprimer, et à dédaigner l'ouvrage de tes mains, et à bénir les desseins des méchants\FTNT{Es. 64:7-8.} ?
\VS{4}As-tu des yeux de chair ? Vois-tu comme voit un homme mortel?
\VS{5}Tes jours sont-ils comme les jours de l'homme mortel ? Tes années sont-elles comme les jours de l'homme, 
\VS{6}Que tu recherches mon iniquité, et que tu t'informes de mon péché,
\VS{7}tu sais que je n'ai point commis de crime, et qu'il n'y a personne qui me délivre de ta main?
\VS{8}Tes mains m'ont formé, et elles ont rangé toutes les parties de mon corps; et tu me détruirais !\FTNT{Ge. 2:7 ; Ps. 119:73 ; Ps. 139:14-15.} !
\VS{9}Souviens-toi, je te prie, que tu m'as formé comme de la boue, et que tu me feras retourner en poudre?
\VS{10}Ne m'as-tu pas coulé comme du lait ? et ne m'as-tu pas fait cailler comme un fromage ?
\VS{11}Tu m'as revêtu de peau et de chair, et tu m'as composé d'os et de nerfs;
\VS{12}Tu m'as donné la vie, et tu as usé de miséricorde envers moi, et [par] tes soins continuels tu as gardé mon esprit.
\VS{13}Et cependant tu gardais ces choses en ton cœur ; mais je connais que cela était devant toi. 
\VS{14}Si j'ai pèche, tu m'observes, et tu ne me tiens pas pour innocent de mon iniquité.
\VS{15}Si j'agis méchamment, malheur à moi ! si je suis juste, je n'en lève pas la tête plus haut. Je suis rempli d'ignominie ; mais regarde mon affliction. 
\VS{16}Si je redresse la tête, tu me poursuis comme à un lion, et tu multiplie tes exploits contre moi; \FTNT{Zq. 38:13 ; La. 3:10.}.
\VS{17}Tu renouvelles tes témoins contre moi, et ton indignation augmente contre moi. De nouvelles troupes toutes fraîches viennent contre moi.
\VS{18}Mais pourquoi m'as-tu fait sortir du sein de ma mère? J'aurais expiré, et aucun œil ne m'aurait vu ;
\VS{19}et j'aurais été comme n'ayant jamais été, et j'aurais été porté du ventre de ma mère au tombeau.
\VS{20}Mes jours ne sont-ils pas en petit nombre ? Cesse donc et retire-toi de moi, et que je me renforce un peu. .
\VS{21}Avant que j'aille au lieu d'où je ne reviendrai plus, en la terre de ténèbres et de l'ombre de la mort,
\VS{22}terre d'une grande obscurité, comme les ténèbres de l'ombre de la mort, où il n'y a aucun ordre, et où rien ne luit que des ténèbres. 
\Chap{11}
\TextTitle{Première accusation de Tsophar}
\VerseOne{}Tsophar de Naama prit la parole et dit :
\VS{2}Ne répondra-t-on point à tant de discours, et suffira-t-il d'être un grand parleur pour être justifié ?
\VS{3}Tes vains feront-ils taire les gens ? Et quand tu te seras moqué, n'y aura-t-il personne qui te fasse honte ?
\VS{4}Car tu as dit : Ma doctrine est pure, et je suis sans tache devant tes yeux. 
\VS{5}Mais je voudrais que Dieu parle, et qu'il ouvre sa bouche pour te répondre; ,
\VS{6}Qu'il te montre les secrets de sa sagesse, de son immense sagesse; et que tu reconnaisse que Dieu oublie une partie de ton iniquité. .
\VS{7}Trouveras-tu Dieu en le sondant ? Connaîtras-tu parfaitement le Tout-puissant ? 
\VS{8}Ce sont les hauteurs des cieux : Qu'y feras-tu ? C'est plus profond que le scheol : Qu'y connaîtras-tu ?
\VS{9}Son étendue est plus longue que la terre, et plus large que la mer.
\VS{10}S'il remue, et qu'il resserre, ou qu'il rassemble, qui l'en détournera ?
\VS{11}Car il connaît les hommes vicieux, il discerne par le regard les coupables\FTNT{Ps. 10:11-14 ; Ps. 35:22.}.
\VS{12}Mais l'homme vide de sens devient intelligent, quoique l'homme naisse comme un ânon sauvage\FTNT{Ec. 3:18.}.
\VS{13}Si tu disposes ton cœur, et que tu étendes tes mains vers lui,
\VS{14}Si tu éloignes de toi l'iniquité qui est en ta main, et si tu ne permets pas que la méchanceté habite dans tes tentes ; 
\VS{15}Alors certainement tu pourras élever ton visage sans tache ; tu seras ferme et tu ne craindras rien;
\VS{16}tu oublieras tes peines, tu t'en souviendras comme des eaux écoulées.
\VS{17}La vie se lèvera pour toi plus brillante que le midi, et l'obscurité même sera comme le matin\FTNT{Ps. 37:6 ; Ps. 112:4.}.
\VS{18} Tu seras plein de confiance, parce qu'il y aura de l'espérance pour toi; tu creuseras, et tu reposeras sûrement. \FTNT{Lé. 26:6 ; Ps. 3:6 ; Pr. 3:24.}.
\VS{19}Tu te coucheras, et il n'y aura personne qui t'épouvante, et plusieurs te feront la cour. 
\VS{20}Mais les yeux des méchants seront consumés; tout refuge leur sera ôté et toute leur espérance sera de rendre l'âme !
\Chap{12}
\TextTitle{Réplique de Job}
\VerseOne{}Job reprit la parole, et dit :
\VS{2}On dirait vraiment que vous êtes tout un peuple, et qu'avec vous doit mourir la sagesse.
\VS{3}J'ai du bon sens aussi bien que vous, et je ne vous suis point inférieur ; et qui ne sait de telles choses ?
\VS{4}Je suis pour mes amis un objet de raillerie, quand je m'écrie à Dieu pour qu'il me réponde; on se moque d'un homme qui est juste et droit.
\VS{5} Mépris au malheur! telle est la pensée des heureux; le mépris est réservé à ceux dont le pied chancelle !
\VS{6}Elles sont en paix, les tentes des pillards, et toutes les sécurités sont pour ceux qui irritent Dieu, qui se font un dieu de leur bras. \FTNT{ Jé. 12:1 ; Ps. 73:12.}.
\VS{7}Mais interroge donc les bêtes, et elles t'instruiront, ou les oiseaux des cieux, et ils te l'annonceront ;
\VS{8}Ou parle à la terre, et elle t'enseignera ; même les poissons de la mer te le raconteront ; 
\VS{9}Qui est-ce qui ne sait toutes ces choses; que c'est la main de Yahweh qui a fait cela ?
\VS{10} Qu'il tient en sa main, l'âme de tout ce qui vit, et l'esprit de toute chair humaine,
\VS{11}L'oreille ne discerne-t-elle pas les discours, ainsi que le palais savoure les aliments ?
\VS{12}La sagesse est dans les vieillards, et l'intelligence est le fruit d'une longue vie.
\VS{13}Mais en Dieu est la sagesse et la force ; à lui appartient le conseil et l'intelligence. \FTNT{Da. 2:20.}.
\VS{14}Voici, il démolit, et on ne rebâtit pas ; il enferme un homme, et on ne lui ouvre pas\FTNT{Es. 22:22 ; Ap. 3:7.}.
\VS{15}Voilà, il retient les eaux, et tout devient sec ; il les lâche, et elles bouleversent la terre.
\VS{16} En lui résident la puissance et la sagesse; de lui dépendent celui qui s'égare et celui qui égare.
\VS{17} Il emmène dépouillés les conseillers, et il met hors de sens les juges. \FTNT{2 S. 15:31 ; 2 S. 17:14-23 ; Es. 19:12 ; Es. 29:14 ; 1 Co. 1:19.}.
\VS{18}Il rend impuissant le gouvernement des rois, et lie de chaînes leurs reins. 
\VS{19}Il fait marcher pieds nus les sacrificateurs ; et il renverse les puissants.
\VS{20}Il ôte la parole à ceux qui sont les plus assurés en leurs discours, et il prive de sens les anciens.
\VS{21}Il verse le mépris sur les nobles ; il relâche la ceinture des forts\FTNT{Es. 40:23.}.
\VS{22}Il met en évidence les choses qui étaient cachées dans les ténèbres, et il produit en lumière l'ombre de la mort. \FTNT{Ps. 139:11-12 ; Ec. 12:16 ; Mt. 10:26 ; 1 Co. 4:6.}.
\VS{23} Il multiplie les nations, et les fait périr ; il répand çà et là les nations, et puis il les ramène. 
\VS{24}Il ôte la raison aux Chefs des peuples de la terre, et les fait errer dans les déserts où il n'y a point de chemin;
\VS{25}ils tâtonnent dans les ténèbres, sans aucune clarté, et il les fait chanceler comme des gens ivres. 
\Chap{13}
\VerseOne{}Voici, mon œil a vu toutes ces choses, mon oreille l'a entendu et compris.
\VS{2}Comme vous les savez, je les sais aussi ; je ne vous suis pas inférieur. 
\VS{3}Mais je veux parler au Tout-Puissant, je veux plaider auprès de Dieu. .
\VS{4}Et certes vous inventez des mensonges ; vous êtes tous des médecins inutiles.
\VS{5}Plaît à Dieu que vous demeuriez entièrement dans le silence ; et cela vous sera réputé à sagesse. \FTNT{Pr. 17:28.}.
\VS{6}Ecoutez donc maintenant ma cause, et soyez attentifs à la défense de mes lèvres.
\VS{7}Tiendrez-vous des discours injustes en faveur de Dieu, et, pour le défendre, direz-vous des mensonges?
\VS{8}Ferez-vous acception de personnes en sa faveur? Prétendrez-vous plaider pour Dieu? 
\VS{9}S'il vous sonde, vous trouvera-t-il bon ? Comme on trompe un homme, le tromperez-vous? 
\VS{10}Certainement il vous reprendra, si même en secret vous faites acception de personnes.
\VS{11}Sa majesté ne vous épouvantera-t-elle pas ? Et sa frayeur ne tombera-t-elle pas sur vous ? 
\VS{12}Vos discours mémorables sont des sentences de cendre, et vos éminences sont des éminences de boue. 
\VS{13}Taisez-vous devant moi, et que je parle ; et il m'arrivera ce qui pourra. 
\VS{14}Pourquoi porterais-je ma chair entre mes dents, et tiendrais-je mon âme entre mes mains ? \FTNT{Jg. 12:3 ; 1 S. 19:5.}.
\VS{15}Voilà, qu'il me tue, je ne cesserai pas d'espérer en lui ; et je défendrai ma conduite en sa présence.
\VS{16}Et qui plus est, il sera lui-même mon salut ; mais l'hypocrite ne viendra point devant sa face. \FTNT{Ps. 1:5.}.
\VS{17}Ecoutez attentivement mes paroles, et prêtez l'oreille à ce que je vais vous déclarer. 
\VS{18}Voici, j'ai préparé ma cause. Je sais que je serai justifié.
\VS{19}Qui est-ce qui veut disputer contre moi ? car maintenant si je me tais, je mourrai. 
\VS{20}Seulement ne me fais pas ces deux choses, et alors je ne me cacherai point devant ta face :
\VS{21}Retire ta main de dessus moi, et que tes terreurs ne me troublent pas.
\VS{22}Puis appelle-moi, et je répondrai ; ou bien je parlerai, et tu me répondras. 
\VS{23}Combien ai-je d'iniquités et de péchés ? Montre-moi mon crime et mon péché. 
\VS{24}Pourquoi caches-tu ta face, et me tiens-tu pour ton ennemi ?
\VS{25}Déploieras-tu tes forces contre une feuille que le vent emporte ? Poursuivras-tu du chaume tout sec \FTNT{1 S. 24:15.} ?
\VS{26}Que tu écrives contre moi des choses amères, et que tu me fasses porter la peine des péchés de ma jeunesse? \FTNT{Ps. 25:7.} ?
\VS{27}Que tu mettes mes pieds aux ceps, et observes tous mes chemins, et que tu suives les traces de mes pieds,
\VS{28}Quand mon corps s'en va par pièces comme du bois vermoulu, et comme une robe que la teigne a rongée? 
\Chap{14}
\VerseOne{}L'homme né de la femme est de courte vie, et rassasié d'agitations. \FTNT{Ps. 102:12 ; Ps. 103:15 ; Ps. 144:4 ; Ja. 4:14.}.
\VS{2}Il sort comme une fleur, puis il est coupé, et il s'enfuit comme une ombre qui ne s'arrête pas. \FTNT{Es. 40:6 ; Ps. 90:61 ; 1 Pi. 1:24.}.
\VS{3}Cependant tu as ouvert tes yeux sur lui, et tu me conduis en justice avec toi.
\VS{4}Qui est-ce qui tirera le pur de l'impur ? Personne. \FTNT{Es. 48:8 ; Pr. 22:15.}.
\VS{5}Les jours de l'homme sont déterminés, le nombre de ses mois est entre tes mains, tu lui as prescrit ses limites, et il ne passera point au delà.
\VS{6}Retire-toi de lui, afin qu'il ait du relâche, jusqu'à ce que comme un mercenaire il ait achevé sa journée.
\VS{7}Car si un arbre est coupé, il y a de l'espérance, et il poussera encore, et ne manquera pas de rejetons ; 
\VS{8}quoique sa racine ait vieilli dans la terre, et que son tronc soit mort dans la poussière;
\VS{9}Dès qu'il sent l'eau il regerme, et produit des branches, comme un arbre nouvellement planté. 
\VS{10}Mais l'homme meurt et perd toute sa force; il expire et puis où est-il ?
\VS{11}Les eaux s'écoulent de la mer, et une rivière s'assèche, et tarit ;
\VS{12} Ainsi l'homme est couché par terre, et ne se relève plus ; jusqu'à ce qu'il n'y ait plus de cieux, ils ne se réveillera plus, et ne sera pas réveillé de son sommeil. 
\VS{13}Oh que tu me caches dans le scheol, que tu me gardes à l'abri jusqu'à ce que ta colère soit passée, que tu me donnes un temps arrêté, après lequel tu te souviendrais de moi !
\VS{14}Si un homme meurt, revivra-t-il ? Tous les jours de ma détresse, j'attendrais jusqu'à ce que mon état vînt à changer.
\VS{15} Tu appellerais, et moi je te répondrais, tu ne dédaignerais pas l'ouvrage de tes mains.
\VS{16}Mais maintenant tu comptes mes pas, et tu n'exceptes rien de mon péché. \FTNT{Ps. 56:9 ; Ps. 139:2-4 ; Pr. 5:21.} ;
\VS{17}Mes péchés sont scellés dans un sac, et tu as cousu ensemble mes iniquités. \FTNT{Os. 13:12.}.
\VS{18}Car comme une montagne s'éboule en tombant, et comme un rocher est transporté de sa place ; 
\VS{19} et comme les eaux minent les pierres, et entraînent par leur débordement la poussière de la terre, avec tout ce qu'elle a produit, tu fais ainsi périr l'attente de l'homme. 
\VS{20}Tu te montres toujours plus fort que lui, et il s'en va, et lui ayant défiguré le visage, tu le renvoies.
\VS{21}Quand ses fils sont honorés, il n'en sait rien ; et quand ils sont abaissés, il ne s'en aperçoit pas.
\VS{22}Seulement sa chair sur lui, a de la douleur, et son âme en lui s'afflige. 
\Chap{15}
\TextTitle{Deuxième discours d'Eliphaz}
\VerseOne{}Eliphaz de Théman prit la parole et dit :
\VS{2}Un homme sage profère-t-il dans ses réponses une science aussi légère que le vent, des opinions vaines ? Remplit-il son ventre du vent d'orient ?
\VS{3}Contestant avec des discours qui ne servent de rien, et avec des paroles dont on ne peut tirer aucun profit ?
\VS{4}Certainement tu abolis la crainte de Dieu, et tu anéantis peu à peu la prière qu'on doit présenter à Dieu. 
\VS{5} Car ta bouche fait connaître ton iniquité, et tu as choisi un langage trompeur. 
\VS{6}C'est ta bouche qui te condamne, et non pas moi ; et tes lèvres témoignent contre toi. 
\VS{7}Es-tu le premier homme né ? Ou as-tu été formé avant les montagnes ? \FTNT{Ps. 90:2 ; Pr. 8:25.} ?
\VS{8}As-tu été instruit dans le conseil secret de Dieu, et renfermes-tu seul la sagesse ? \FTNT{Es. 40:13 ; Jé. 23:18 ; Ro. 11:34.} ?
\VS{9}Que sais-tu que nous ne sachions pas ? Quelle connaissance as-tu que nous n'ayons pas ?
\VS{10}Parmi nous, il y a des hommes à cheveux blancs, et des gens d'une fort grande vieillesse, il y en a même de plus âgés que ton père. 
\VS{11}Les consolations du Dieu te semblent-elles trop petites ? As-tu quelque chose de caché par-devant toi ? …
\VS{12}Pourquoi ton coeur s'emporte-il et pourquoi tes yeux clignent-ils ?
\VS{13}C'est contre Dieu que tu tournes ta colère, et que tu fais sortir de ta bouche de tels discours! 
\VS{14}Qu'est-ce que de l'homme, pour qu'il soit pur, et celui qui est né de femme, pour qu'il soit juste ? \FTNT{Ps. 14:3 ; Pr. 20:9 ; Ec. 7:20.} ?
\VS{15}Si Voici, Dieu ne se fie pas à ses saints, et les cieux ne sont pas purs à ses yeux,
\VS{16}Combien plus est abominable et corrompu, l'homme qui boit l'iniquité comme l'eau !  
\VS{17}Je t'enseignerai, écoute-moi, et je te raconterai ce que j'ai vu ,
\VS{18} savoir ce que les sages ont déclaré, et qu'ils n'ont point caché ; ce qu'ils avaient [reçu] de leurs pères.
\VS{19}Eux à qui seuls la terre a été donnée, et parmi lesquels l'étranger n'est point passé.
\VS{20}Toute sa vie, le méchant est tourmenté, et un petit nombre d'années sont réservées au malfaiteur.\FTNT{Es. 48:22 ; Es. 57:21.}.
\VS{21}Un cri de frayeur est dans ses oreilles ; au milieu de la paix [il croit] que le destructeur se jette sur lui. \FTNT{1 Th. 5:3.} ;
\VS{22}Il ne croit pas pouvoir sortir des ténèbres, car il voit la menace de   l’épée;
\VS{23}il court çà et là pour chercher son pain, il sait que le jour des ténèbres lui est préparé \FTNT{Ps. 109:10.}.
\VS{24}La détresse et l'angoisse l'épouvantent, elles l'assaillent comme un roi prêt à combattre ;
\VS{25}Parce qu'il a élevé sa main contre Dieu, et qu'il s'est levé contre le Tout-puissant ;
\VS{26}Il lui a sauté au collet, et sur l'épaisseur de ses gros boucliers. 
\VS{27}Parce que la graisse a couvert son visage, et qu'elle a fait des replis sur son ventre;
\VS{28} il habite des villes détruites, des maisons désertes, tout près de n'être plus que des monceaux de pierres. 
\VS{29}Et il ne s'enrichira plus, car ses biens ne subsisteront pas, et ses richesses ne se répandront pas sur la terre. 
\VS{30}Il ne pourra pas se détourner des ténèbres, la flamme desséchera ses rejetons, et Dieu le fera disparaître par le souffle de sa bouche.
\VS{31}S'il a confiance dans la vanité, il se trompe, car la vanité sera sa récompense.
\VS{32}Ce sera fait de lui avant son temps, ses branches ne reverdiront plus. 
\VS{33}On arrachera ses fruits non mûrs, comme à une vigne; on jettera sa fleur, comme celle d'un olivier. 
\VS{34}Car la famille des hypocrites est stérile, et le feu dévore les tentes de l'homme corrompu.
\VS{35}Ils conçoivent le travail, et ils enfantent la misère, et machinent dans le cœur des fraudes. \FTNT{Es. 59:4 ; Os. 10:13.}.
\Chap{16}
\TextTitle{Réponse de Job}
\VerseOne{}Job répondit, et dit :
\VS{2}J'ai souvent entendu de pareils discours ; vous êtes tous des consolateurs fâcheux.
\VS{3}Y aura-t-il une fin à [ces] paroles de vent ? Qu'est-ce qui t'irrite, que tu répondes ?
\VS{4}Parlerais-je comme vous faites, si vous étiez en ma place ; accumulerais-je des paroles contre vous, ou secouerais-je ma tête contre vous ? 
\VS{5}Je vous fortifierais de ma bouche, et le mouvement de mes lèvres vous soulagerait.
\VS{6}Si je parle, ma douleur ne sera point soulagée. Si je me tais, en sera-t-elle diminuée?
\VS{7}Maintenant il m'a épuisé... Tu as dévasté toute ma famille, ;
\VS{8}Tu m'as tout couvert de rides, qui sont un témoignage des maux que je souffre ; et il s'est élevé en moi une maigreur qui en rend aussi témoignage sur mon visage. 
\VS{9}Sa fureur me déchire, il se déclare mon ennemi, il grince des dents contre moi, et étant devenu mon ennemi, il étincelle des yeux contre moi.
\VS{10}Ils ouvrent contre moi leur bouche; ils me frappent à la joue pour m'outrager; ils se réunissent tous ensemble contre moi. 
\VS{11}Dieu m'a livré à l'impie, et m'a jeté entre les mains des méchants. 
\VS{12}J'étais tranquille, et il m'a secoué, il m'a saisi par la nuque et m'a brisé, il m'a posé en butte à ses traits.
\VS{13}Ses archers m'ont environné, il me perce les reins, et ne m'épargne pas ; il répand mon fiel par terre. 
\VS{14}Il m'a brisé en me faisant plaie sur plaie, il a couru sur moi comme un homme fort.
\VS{15}J'ai cousu un sac sur ma peau, j'ai souillé ma tête dans la poussière\FTNT{Ps. 44 : 25 ; Ps. 119 : 25.},
\VS{16}J'ai le visage tout enflammé, à force de pleurer, et l'ombre de la mort est sur mes paupières, 
\VS{17}Quoiqu'il n'y ait point de violence dans mes mains, et que ma prière fut toujours pure.
\VS{18}Ô terre, ne cache pas mon sang, et qu'il n'y ait aucun lieu où s'arrête mon cri !
\VS{19}Mais maintenant voilà, mon témoin est aux cieux, mon témoin est dans les lieux élevés. \FTNT{Ap. 1:5 ; Ap. 3:14.}.
\VS{20}Mes amis se moquent de moi: c'est vers Dieu que mon œil se tourne en pleurant,
\VS{21}pour qu'il fasse justice entre l'homme et Dieu, entre le fils d'Adam et son semblable.
\VS{22}Car les années de mon compte arrivent à leur terme, et j'entre dans un sentier d'où je ne reviendrai plus. 
\Chap{17}
\VerseOne{}Mon souffle se perd, mes jours s'éteignent, le sépulcre m'attend.
\VS{2}Je suis environné de moqueurs, et mon œil veille toute la nuit au milieu de leurs insultes.
\VS{3}Dépose un gage, sois ma caution auprès de toi-même; car qui voudrait répondre pour moi? 
\VS{4}C'est pourquoi tu ne les élèveras pas\FTNT{De. 29:4 ; Mt. 11:25.}.
\VS{5}Celui qui trahit ses amis pour qu'ils soient pillés, les yeux de ses fils se consument.
\VS{6}On a fait de moi la fable des peuples, un être à qui l'on crache au visage.
\VS{7}Mon œil est obscurci par le chagrin, tous mes membres sont comme une ombre\FTNT{Ps. 6:7 ; Ps. 31:10.}.
\VS{8}Les hommes droits en sont consternés, et l'innocent est irrité contre l'impie. 
\VS{9}Toutefois le juste se tient ferme dans sa voie, et celui qui a les mains pures, se renforce.
\VS{10}Retournez donc vous tous, et revenez, je vous prie ; car je ne trouve pas de sages parmi vous. 
\VS{11}Mes jours sont passés; mes desseins, chers à mon cœur, sont renversés.…
\VS{12}On me change la nuit en jour, et on fait que la lumière se trouve proche des ténèbres !
\VS{13}Certes je n'ai plus à attendre que le sépulcre, qui va être ma maison ; j'ai dressé mon lit dans les ténèbres ;
\VS{14}J'ai crié à la fosse : tu es mon père ; et aux vers : vous êtes ma mère et ma Soeur. 
\VS{15}Où est donc mon espérance? Et mon espérance, qui pourrait la voir? \VS{16}Elle descendra au fond du sépulcre ; certes elle reposera avec moi dans la poussière. 
\Chap{18}
\TextTitle{Deuxième discours de Bildad}
\VerseOne{}Bildad de Schuach prit la parole et dit :
\VS{2}Quand finirez-vous ces discours ? Ecoutez, et puis nous parlerons.
\VS{3}Pourquoi sommes-nous regardés comme des bêtes, et sommes-nous stupides à vos yeux?
\VS{4}Ô toi qui déchires ton âme dans ta colère, la terre sera-t-elle abandonnée à cause de toi, et le rocher sera-t-il transporté de sa place ? 
\VS{5}Certainement, la lumière du méchant s'éteindra, et la flamme de son feu ne brillera pas\FTNT{Ps. 37:9-10.}.
\VS{6}La lumière sera ténèbres dans sa tente, et sa lampe sera éteinte au-dessus de lui. 
\VS{7}Les pas de sa force seront resserrés, et son propre conseil le renversera.
\VS{8}Car il est poussé dans le filet par ses propres pieds ; et il marche sur les mailles du filet.
\VS{9}Le piège le prend par le talon, et le filet s'empare de lui;
\VS{10}la corde est cachée dans la terre, et la trappe est sur son sentier.
\VS{11}Les terreurs l'assiègent de tous côtés, et le font courir ses pieds çà et là.\FTNT{Jé. 6:25 ; Jé. 46:5 ; Jé. 49:29.}.
\VS{12}Sa vigueur sera affamée, la détresse est à ses côtés.
\VS{13}Il dévorera les membres de son corps, il dévorera ses membres, le premier-né de la mort ! 
\VS{14} Les choses en quoi il mettait sa confiance seront arrachées de sa tente, et il sera conduit vers le Roi des épouvantements. 
\VS{15}On habitera dans sa tente, qui ne sera plus à lui; le soufre sera répandu sur sa demeure. 
\VS{16}Ses racines sèchent au dessous, et ses branches sont coupées en haut. 
\VS{17}Sa mémoire périt sur la terre, et on ne parle plus de son nom dans les places \FTNT{Ps. 109:13 ; Pr. 10:7.}.
\VS{18}Il est chassé de la lumière dans les ténèbres, et il est exterminé du monde. 
\VS{19}Il n'a ni lignée, ni descendance au milieu de son peuple, ni survivant dans ses habitations. \FTNT{Es. 14:20-22 ; Jé. 22:30 ; Ps. 37:28 ; Ps. 109:13.}. 
\VS{20}Ceux qui seront venus après lui, seront étonnés de sa ruine ; et ceux qui auront été avant lui en seront saisis d'horreur. 
\VS{21}Tel est le sort de l'injuste. Telle est la destinée de celui qui ne connaît pas Dieu. 
\Chap{19}
\TextTitle{Réponse de Job}
\VerseOne{}Job prit la parole et dit :
\VS{2}Jusqu'à quand affligerez-vous mon âme, et m'accablerez-vous de paroles ?
\VS{3} Voilà déjà dix fois que vous m'outragez: vous n'avez pas honte de me maltraiter? 
\VS{4}Vraiment si j'ai failli, ma faute demeure avec moi. 
\VS{5}Si réellement vous voulez vous élever contre moi et faire valoir mon opprobre contre moi, 
\VS{6}Sachez donc que c'est Dieu qui me renverse, et qui tend son filet autour de moi.
\VS{7}Voici je crie pour la violence qui m'est faite, et je ne suis pas exaucé ; je m'écrie, et il n'y a point de justice !
\VS{8}Il a fermé mon chemin, et je ne puis passer; il a mis des ténèbres sur mes sentiers. 
\VS{9}Il m'a dépouillé de ma gloire, il a ôté la couronne de ma tête.
\VS{10}Il m'a détruit de tous côtés, et je m'en vais ; il a arraché mon espérance comme un arbre.
\VS{11}Il s'est enflammé de colère contre moi, et m'a traité comme un de ses ennemis\FTNT{La. 2:5.}.
\VS{12}Ses troupes sont venues ensemble, et elles ont dressé leur chemin contre moi, et se sont campées autour de ma tente\FTNT{La. 2:22.}.
\VS{13}Il a éloigné de moi mes frères, et ceux qui me connaissaient se sont écartés comme des étrangers\FTNT{Ps. 88:9.} ;
\VS{14}mes proches m'ont abandonné, et ceux que je connaissais m'ont oublié.
\VS{15}Ceux qui séjournent dans ma maison et mes servantes m'ont traité comme un étranger; je suis devenu un inconnu pour eux. 
\VS{16}J'appelle mon serviteur, il ne me répond; de ma propre bouche, je le supplie en vain. 
\VS{17}Mon haleine est devenue dégoûtante à ma femme, et ma plainte aux fils de mes entrailles.
\VS{18}Je suis méprisé même par des enfants ; si je me lève, ils parlent contre moi.
\VS{19}Ceux que j'avais pour confidents m'ont en horreur, ceux que j'aimais se sont tournés contre moi\FTNT{Ps. 55:13-14.}.
\VS{20}Mes os sont attachés à ma peau et à ma chair ; et je me suis échappé avec la peau de mes dents\FTNT{La. 4:8.}.
\VS{21}Ayez pitié, ayez pitié de moi, vous, mes amis ! Car la main de Dieu m'a frappé.
\VS{22}Pourquoi, comme Dieu, me poursuivez-vous et n'êtes-vous pas rassasiés de ma chair \FTNT{Ps. 27:2.} ?
\VS{23}Oh! je voudrais que mes paroles fussent écrites quelque part, je voudrais qu'elles fussent inscrites dans un livre; 
\VS{24}Qu'avec un burin de fer et avec du plomb, elles fussent gravées sur le roc, pour toujours... 
\VS{25}Mais je sais que mon rédempteur est vivant, il demeurera le dernier sur la terre.
\VS{26}Et après que cette peau aura été détruite, hors de ma chair, je verrai Dieu \FTNT{Ps. 17:15.}.
\VS{27}Je le verrai moi-même, et mes yeux le verront, et non un autre. Mes reins se consument dans mon sein. 
\VS{28}Vous direz : Comment le poursuivrons-nous, et trouverons-nous en lui la cause de son malheur? 
\VS{29}Ayez peur de l'épée ; car la fureur [avec laquelle vous me persécutez], est [du nombre] des iniquités qui attirent l'épée ; c'est pourquoi sachez qu'il y a un jugement. 
\Chap{20}
\TextTitle{Dernier discours de Tsophar}
\VerseOne{}Tsophar de Naama prit la parole et dit :
\VS{2}C'est à cause de cela que mes pensées diverses me poussent à répondre, et que cette promptitude est en moi. 
\VS{3}J'ai entendu la correction dont tu veux me faire honte, mais mon esprit tirera de mon intelligence la réponse pour moi. 
\VS{4}Ne sais-tu pas que, de tout temps, depuis que Dieu a mis l'homme sur la terre, 
\VS{5}Le triomphe des méchants est de peu de durée, et la joie de l'hypocrite n'est que pour un moment \FTNT{Ps. 37:35-36.} ?
\VS{6}Quand son élévation monterait jusqu'aux cieux, et que sa tête atteindrait les nues,
\VS{7}il périra pour toujours comme ses ordures, et ceux qui le voyaient diront : Où est-il ?
\VS{8}Il s'envolera comme un songe, et on ne le trouvera plus ; il se  retirera comme une vision nocturne\FTNT{Ps. 73:19-20.} ;
\VS{9}l'œil qui le regardait ne le regardera plus, le lieu qu'il habitait ne le contemplera plus.
\VS{10}Ses fils rechercheront la faveur des pauvres, et ses mains restitueront ce que sa violence a ravi\FTNT{Ps. 109:10.}.
\VS{11}Ses os seront pleins de la punition, à  cause des péchés de sa jeunesse, et elle reposera avec lui dans la poussière.
\VS{12}Le mal était doux à sa bouche, il le cachait sous sa langue,
\VS{13}s'il l'épargne, et ne le rejette point, mais le retient dans son palais ; 
\VS{14}Ce qu'il mangera se changera dans ses entrailles en un fiel d'aspic.
\VS{15}Il a englouti des richesses, il les vomira ; Dieu les arrachera de son ventre.
\VS{16}Il a sucé du venin d'aspic, la langue de la vipère le tuera.
\VS{17}Il ne verra plus les ruisseaux, les fleuves, les torrents de miel et de lait.
\VS{18}Il rendra le fruit de son travail, et ne l'avalera pas; il restituera à proportion de ce qu'il aura amassé, et ne s'en réjouira pas\FTNT{So. 2:10.}.
\VS{19}Car il a opprimé, délaissé les pauvres; il a pillé des maisons et ne les a pas rebâties.
\VS{20}Certainement il ne sentira pas dans son ventre la satisfaction de son avidité, et il ne sauvera rien de ce qu'il aura tant convoité \FTNT{Ec. 5:12.}.
\VS{21}Rien n'échappait à sa voracité, mais son bonheur ne durera pas.
\VS{22}Après que la mesure de ses biens aura été remplie, il sera dans la misère ; toutes les mains de ceux qu'il aura opprimés se jetteront sur lui.
\VS{23}Il arrivera que pour lui remplir le ventre, Dieu enverra contre lui l'ardeur de sa colère; il la fera pleuvoir sur lui et entrer dans sa chair.
\VS{24}S’il s’enfuit de devant les armes de fer, l’arc d’airain le transpercera.
\VS{25}Il arrachera la flèche, et elle sortira de son corps, et le fer étincelant, de son foie; les frayeurs de la mort viendront sur lui.
\VS{26}Toutes les ténèbres sont renfermées dans ses demeures les plus secrètes ; un feu qu'on n'aura point soufflé, le consumera ; l'homme qui restera dans sa tente sera malheureux\FTNT{Ps. 12:6.}.
\VS{27}Les cieux découvriront son iniquité, et la terre s'élèvera contre lui. 
\VS{28}Le revenu de sa maison sera emporté. Tout s'écoulera au jour de la colère de Dieu.
\VS{29}C'est là la portion que Dieu réserve à l'homme méchant, et l'héritage qu'il aura de Dieu pour ses discours.
\Chap{21}
\TextTitle{Réponse de Job}
\VerseOne{}Job répondit, et dit :
\VS{2}Ecoutez, écoutez mes discours, donnez-moi seulement cette consolation.
\VS{3}Supportez-moi, et je parlerai ; et quand j'aurai parlé, tu pourras te moquer.
\VS{4}Mais est-ce contre un homme que s'adresse ma plainte ? Et pourquoi mon âme ne serait-elle pas impatiente ?
\VS{5}Regardez-moi, soyez étonnés, et mettez la main sur la bouche.
\VS{6}Quand j'y pense, cela m'épouvante, et un frisson saisit mon corps.
\VS{7}Pourquoi les méchants vivent-ils, vieillissent-ils, et croissent-ils en puissance\FTNT{Jé. 12:1 ; Ha. 1:3 ; Mal. 3:14-15.}?
\VS{8}Leur postérité s'établit avec eux et en leur présence, leurs rejetons prospèrent sous leurs yeux.
\VS{9}Dans leurs maisons règne la paix, loin de la crainte ; la verge de Dieu ne vient pas les frapper.
\VS{10}Leurs taureaux sont féconds, leurs génisses conçoivent et n'avortent pas\FTNT{Ps. 144:13-14.}.
\VS{11}Ils laissent courir leurs enfants comme un troupeau, et les enfants prennent leurs ébats.
\VS{12}Ils chantent au son du tambourin et de la harpe, ils se réjouissent au son du chalumeau.
\VS{13}Ils passent leurs jours dans le bonheur, et ils descendent en un instant au scheol.
\VS{14}Ils disaient pourtant à Dieu : Éloigne-toi de nous, nous ne voulons pas connaître tes voies.
\VS{15}Qu'est-ce que le Tout-Puissant pour que nous le servions ? Que gagnerions-nous à lui adresser nos prières\FTNT{Ex. 5:2.} ?
\VS{16}Quoi donc ! Ne sont-ils pas en possession du bonheur entre leurs mains ? Loin de moi le conseil des méchants\FTNT{Ps. 1:1-2.} !
\VS{17}Mais arrive-t-il que la lampe des méchants s'éteigne, que la ruine vienne sur eux, que Dieu leur distribue leur part dans sa colère\FTNT{Ps. 11:5-6 ; Pr. 13:9.},
\VS{18}qu'ils soient comme la paille face au vent, comme la balle enlevée par le tourbillon\FTNT{Ps. 1:4.}?
\VS{19}Dieu réservera-t-il aux enfants du méchant la punition de ses violences? Il la leur rendra, et il le connaîtra!
\VS{20}Il verra de ses propres yeux sa ruine, c'est lui qui devrait boire la colère du Tout-Puissant\FTNT{Es. 51:17-22 ; Jé. 25:15 ; Ez. 23:31-32 ; Ap. 14:10.}.
\VS{21}Car que lui importe sa maison après lui, quand le nombre de ses mois est achevé ?
\VS{22}Enseignerait-on la science à Dieu, lui qui juge les esprits élevés\FTNT{Ro. 11:34 ; 1 Co. 2:16.} ?
\VS{23}L'un meurt au sein du bien-être, tout à son aise et en joie,
\VS{24}les flancs chargés de graisse, et ses os comme abreuvés de mœlle ;
\VS{25}l'autre meurt l'amertume dans l'âme, n'ayant jamais mangé ce qui est bon.
\VS{26}Et tous deux se couchent dans la poussière, tous deux deviennent couverts de vers.
\VS{27}Je sais bien quelles sont vos pensées, quels jugements iniques vous portez sur moi.
\VS{28}Vous dites : Où est la maison de l'homme puissant ? Où est la tente, demeure des méchants ?
\VS{29}Mais quoi ! N'avez-vous pas interrogé les voyageurs, et n'avez-vous pas appris par les rapports qu'ils vous on faits ?
\VS{30}Au jour du malheur, le méchant est épargné ; au jour de la colère, il échappe\FTNT{Pr. 16:4 ; Ec. 9:12.}.
\VS{31}Qui lui dit en face sa conduite ? Qui lui rend ce qu'il a fait ?
\VS{32}Il est porté au tombeau, et il veille encore sur sa tombe.
\VS{33}Les mottes de la vallée lui sont légères ; et tous après lui suivront la même voie, comme une multitude l'a déjà suivie.
\VS{34}Pourquoi donc m'offrir de vaines consolations ? Ce qui reste de vos réponses n'est que transgression.
\Chap{22}
\TextTitle{Dernier discours d'Eliphaz}
\VerseOne{}Eliphaz de Théman prit la parole et dit :
\VS{2}Un homme peut-il être utile à Dieu ? Mais le sage n'est utile qu'à lui-même.
\VS{3}Si tu es juste, est-ce à l'avantage du Tout-Puissant ? Si tu es intègre dans tes voies, qu'y gagne-t-il ?
\VS{4}Est-ce par crainte de toi qu'il te reprend, qu'il entre en jugement avec toi?
\VS{5}Ta méchanceté n'est-elle pas grande ? Tes iniquités ne sont-elles pas sans fin ?
\VS{6}Tu a pris sans raison le gage de tes frères, tu privais de leurs vêtements ceux qui étaient nus\FTNT{Ex. 22:21.} ;
\VS{7}tu ne donnais pas d'eau à boire à l'homme altéré, tu refusais du pain à l'homme affamé.
\VS{8}Le pays était à l'homme le plus fort, et le puissant s'y établissait.
\VS{9}Tu renvoyais les veuves à vide, les bras des orphelins étaient brisés.
\VS{10}C'est pour cela que tu es entouré de pièges, et que la terreur t'a saisi tout à coup.
\VS{11}Ne vois-tu donc pas ces ténèbres, ces eaux débordées qui te couvrent ?
\VS{12}Dieu n'est-il pas là-haut dans les cieux ? Regarde le sommet des étoiles, comme il est élevé !
\VS{13}Et tu dis : Qu'est-ce que Dieu connaît ? Peut-il juger à travers l'obscurité\FTNT{So. 1:12 ; Ps. 10:11-13 ; Ps. 94:7.} ?
\VS{14}Les nuées l'enveloppent, et il ne voit rien ; il ne parcourt que la voûte des cieux.
\VS{15}Eh quoi ! N'as-tu pas pris garde à l'ancienne route qu'ont suivie les hommes d'iniquité ?
\VS{16}Ils ont été emportés avant le temps, ils ont eu la durée d'un torrent qui s'écoule.
\VS{17}Ils disaient à Dieu : Éloigne-toi de nous ; que peut faire pour nous le Tout-Puissant ?
\VS{18}Dieu cependant avait rempli leurs maisons de biens ! Loin de moi le conseil des méchants !
\VS{19}Les justes le verront, se réjouiront, et l'innocent se moquera d'eux\FTNT{Ps. 107:42.} :
\VS{20}Certainement, notre adversaire a été détruit, le feu a dévoré ce qui en restait\FTNT{Ps. 37:20 ; Ec. 8:12-13.} !
\VS{21}Attache-toi donc à Dieu, et tu auras la paix, tu atteindras ainsi le bonheur.
\VS{22}Reçois de sa bouche l'instruction, et mets ses paroles dans ton cœur\FTNT{Ps. 119:72.}.
\VS{23}Si tu reviens au Tout-Puissant, tu seras rétabli ; si tu éloignes l'iniquité de ta tente.
\VS{24}Jette l'or dans la poussière, l'or d'Ophir parmi les rochers des torrents ;
\VS{25}et le Tout-Puissant sera ton or, ton argent, ta richesse.
\VS{26}Alors tu feras du Tout-Puissant tes délices, tu élèveras vers Dieu ta face ;
\VS{27}tu le prieras, et il t'exaucera, et tu lui rendras tes vœux\FTNT{Ps. 50:14-15.}.
\VS{28}Quand tu prendras des résolutions elles s'accompliront, sur tes sentiers brillera la lumière\FTNT{Ps. 97:11.}.
\VS{29}Quand on aura abaissé quelqu'un et que tu auras dit qu'il soit élevé; alors Dieu délivrera celui qui tenait les yeux abaissés\FTNT{Pr. 29:23.}.
\VS{30}Il délivrera le coupable ; il sera délivré par la pureté de tes mains.
\Chap{23}
\TextTitle{Réponse de Job}
\VerseOne{}Job répondit, et dit :
\VS{2}Maintenant encore ma plainte est une révolte, et pourtant ma main appesantit mes soupirs.
\VS{3}Oh ! Si je savais où le trouver, j'irais jusqu'à son trône,
\VS{4}je disposerais en ordre ma cause devant lui, je remplirais ma bouche d'arguments,
\VS{5}je saurais ce qu'il peut avoir à répondre, je comprendrais ce qu'il peut avoir à me dire.
\VS{6}Contesterait-il avec moi dans la grandeur de sa force? Ne prendrait-il pas le temps de m'écouter ?
\VS{7}Ce serait un homme juste qui argumenterait avec lui, et je serais pour toujours absous par mon juge.
\VS{8}Mais, si je vais à l'orient, il n'y est pas ; si je vais à l'occident, je ne l'aperçois pas ;
\VS{9}est-il occupé au nord, je ne le vois pas ; se cache-t-il au midi, je ne l'aperçois pas.
\VS{10}Il connaît la voie que j'ai suivie ; et s'il m'éprouvait, j'en sortirai pur comme l'or\FTNT{1 Pi. 1:7.}.
\VS{11}Mon pied s'est attaché à ses pas ; j'ai gardé sa voie, et je ne m'en suis pas détourné.
\VS{12}Je n'ai pas abandonné les commandements de ses lèvres ; j'ai fait plier ma volonté aux paroles de sa bouche.
\VS{13}Mais il n'a qu'une pensée ; qui l'en fera revenir ? Ce que son âme désire, il le fait\FTNT{Ps. 115:3 ; Ps. 135:6.}.
\VS{14}Il achèvera donc ses desseins à mon égard, et il en concevra beaucoup d'autres encore.
\VS{15}C'est pourquoi je suis terrifié à cause de sa présence, et quand je le considère, je suis effrayé devant lui.
\VS{16}Dieu a brisé mon cœur, le Tout-Puissant m'a épouvanté.
\VS{17}Car ce n'est pas la présence des ténèbres qui m'anéantit, ce n'est pas l'obscurité dont ma face est couverte.
\Chap{24}
\VerseOne{}Pourquoi le Tout-Puissant ne met-il pas des temps en réserve, et pourquoi ceux qui le connaissent ne voient-ils pas ses jours ?
\VS{2}On déplace les bornes, on ravit des troupeaux, et on les fait paître\FTNT{De. 19:14 ; De. 27:17 ; Pr. 13:10 ; Pr. 22:28.} ;
\VS{3}on emmène l'âne de l'orphelin, on prend pour gage le bœuf de la veuve ;
\VS{4}on fait écarter les pauvres du chemin, on force tous les affligés du pays à se cacher.
\VS{5}Et voici, comme les ânes sauvages du désert, ils sortent le matin pour chercher de la nourriture, ils n'ont que le désert pour trouver le pain de leurs enfants ;
\VS{6}ils moissonnent le fourrage qui reste dans les champs, ils grappillent dans la vigne de l'impie ;
\VS{7}ils passent la nuit nus, sans vêtements, sans couverture contre le froid\FTNT{Lé. 19:13 ; De. 24:12-13.} ;
\VS{8}ils sont percés par la pluie des montagnes, et ils embrassent les rochers comme unique refuge.
\VS{9}On arrache l'orphelin à la mamelle, on prend des gages sur le pauvre.
\VS{10}Ils font aller sans habits l'homme qu'ils ont dépouillé; et ils enlèvent à ceux qui n'avaient pas de quoi manger, ce qu'ils avaient glâné.
\VS{11}Dans les enclos de l'impie, ils font de l'huile, ils foulent le pressoir à raisin et ils ont soif.
\VS{12}Ils font gémir les gens dans la ville, l'âme de ceux qu'ils ont fait mourir, crient; Dieu ne fait rien d'indigne de lui.
\VS{13}En voici d'autres qui se révoltent contre la lumière, ils n'en connaissent pas les voies, ils ne restent pas sur leurs sentiers.
\VS{14}Le meurtrier se lève au point du jour ; il tue le pauvre et l'indigent, et il dérobe pendant la nuit\FTNT{Ps. 10:8-9.}.
\VS{15}L'œil de l'adultère épie le crépuscule ; aucun œil ne me verra, dit-il, et il met un voile sur le visage\FTNT{Ps. 64:6 ; Pr. 7:7-10.}.
\VS{16}Ils percent durant les ténèbres les maisons, qu'ils avaient marquées le jour, ils haïssent la lumière.\FTNT{Jn. 3:20.}.
\VS{17}Pour eux, le matin c'est l'ombre de la mort ; si quelqu'un les reconnaît, ils ont des terreurs.
\VS{18}Eh quoi ! L'impie est d'un poids léger sur la surface de l'eau, il n'a sur la terre qu'un héritage maudit, il ne prend jamais le chemin des vignes !
\VS{19}Comme la sécheresse et la chaleur absorbent les eaux de la neige, ainsi le scheol engloutit ceux qui pèchent\FTNT{Ps. 49:15.} !
\VS{20}Quoi ! Le sein maternel l'oublie, les vers en font leurs délices, on ne se souvient plus de lui ! L'injuste est brisé comme du bois,
\VS{21}lui qui dépouille la femme stérile et sans enfants, lui qui ne répand aucun bien sur la veuve !…
\VS{22}Non ! Dieu par sa force prolonge les jours des violents, et les voilà s'élever quand ils ne croyaient plus en la vie.
\VS{23}Il leur donne de la sécurité et de la confiance, ses yeux sont sur leurs voies.
\VS{24}Ils se sont élevés ; et en un peu de temps ils ne sont plus, ils s'affaissent, ils meurent en chemin comme tous les hommes, ils sont coupés comme une tête d'épi.
\VS{25}S'il n'en est pas ainsi, qui me fera mentir, qui fera de mes paroles un rien ?
\Chap{25}
\TextTitle{Dernier discours de Bildad}
\VerseOne{}Bildad de Schuach prit la parole et dit :
\VS{2}La domination et la terreur appartiennent à Dieu ; il fait régner la paix dans ses hautes régions.
\VS{3}Ses armées peuvent-elles se compter ? Sur qui sa lumière ne se lève-t-elle pas\FTNT{Mt. 5:45.} ?
\VS{4}Comment l'homme serait-il juste devant Dieu ? Comment celui qui est né de la femme serait-il pur ?
\VS{5}Voici, la lune même n'est pas brillante, et les étoiles ne sont pas pures à ses yeux ;
\VS{6}combien moins l'homme qui n'est qu'un ver, le fils de l'homme qui n'est qu'un vermisseau\FTNT{Ps. 22:7.} !
\Chap{26}
\TextTitle{Réponse de Job}
\VerseOne{}Job répondit, et dit :
\VS{2}Comme tu as aidé celui qui était sans force ! Comme tu as secouru le bras sans force !
\VS{3}Quels bons conseils tu donnes à celui qui manque de sagesse ! Tu fais connaître l'abondance de ton intelligence !
\VS{4}A qui s'adressent tes paroles ? Et de qui est l'esprit qui est sorti de toi ?
\VS{5}Devant Dieu les ombres des morts tremblent au-dessous des eaux, et de leurs habitants ;
\VS{6}devant lui le scheol est nu, l'abîme est sans voile\FTNT{Ps. 139:8-12 ; Pr. 15:11 ; Hé. 4:13.}.
\VS{7}Il étend la direction nord sur le vide, il suspend la terre sur le néant.
\VS{8}Il renferme les eaux dans ses nuages, et la nuée n'éclate pas sous leur poids\FTNT{Ps. 104:2-3.}.
\VS{9}Il couvre la face de son trône, il répand sur lui sa nuée.
\VS{10}Il a tracé un cercle à la surface des eaux, comme limite entre la lumière et les ténèbres\FTNT{Ge. 1:9 ; Jé. 5:22 ; Ps. 33:7 ; Ps. 104:9 ; Pr. 8:29.}.
\VS{11}Les colonnes du ciel s'ébranlent et s'étonnent à sa menace.
\VS{12}Par sa force il soulève la mer, par son intelligence il en brise l'orgueil\FTNT{Ps. 89:10.}.
\VS{13}Il a orné les cieux par son Esprit, et  de sa main, il transperce le serpent fuyard.
\VS{14}Ce sont là les bords de ses voies, c'est le discours fait en chuchotant que nous entendons ; mais qui comprendra le tonnerre de sa puissance\FTNT{Ec. 3:10.} ?
\Chap{27}
\VerseOne{}Et Job continuant, reprit son discours sentencieux, et dit :
\VS{2}Dieu, qui met mon droit à l'écart, et le Tout-puissant qui remplit mon âme d'amertume, est vivant.
\VS{3}Aussi longtemps que j'aurai ma respiration et que l'esprit de Dieu sera dans mes narines,
\VS{4}mes lèvres ne prononceront rien d'injuste, et ma langue ne dira pas de chose fausse\FTNT{Es. 33:15 ; Ps. 15:2 ; Ps. 24:4.}.
\VS{5}Loin de moi la pensée de vous reconnaître pour justes ! Tant que je vivrai je n'abandonnerai pas mon intégrité.
\VS{6}Je conserve ma justice, et je ne l'abandonne pas ; et mon cœur ne me reproche rien en mes jours.
\VS{7}Qu'il en soit de mon ennemi comme du méchant ; et de celui qui se lève contre moi, comme de l'injuste !
\VS{8}Quelle espérance reste-t-il à l'hypocrite quand Dieu coupe le fil de sa vie, quand il lui retire son âme\FTNT{Mt. 16:26 ; Lu. 12:20.} ?
\VS{9}Est-ce que Dieu entend ses cris, quand l'angoisse vient sur lui\FTNT{ Es. 1:15 ; Jé. 14:12 ; Ez. 8:18 ; Mi. 3:4 ; Ps. 18:41 ; Pr. 1:28 ; Jn. 9:31 ; Ja. 4:3.} ?
\VS{10}Trouvera-t-il son plaisir dans le Tout-Puissant ? Invoque-t-il Dieu en tout temps ?
\VS{11}Je vous enseignerai comment la main de Dieu agit, je ne vous cacherai pas les desseins du Tout-Puissant.
\VS{12}Voilà, vous avez tous vu ces choses, et pourquoi vous laissez-vous aller à des pensées vaines ?
\VS{13}Voici la part que Dieu réserve à l'homme méchant, l'héritage que les violents reçoivent du Tout-Puissant.
\VS{14}S'il a des fils en grand nombre, c'est pour l'épée, et ses rejetons ne seront pas rassasiés de pain ;
\VS{15}ses survivants sont ensevelis par la peste, et leurs veuves ne les pleurent pas\FTNT{Ps. 78:64.}.
\VS{16}Parce qu’il entasse l'argent comme la poussière, et qu'il entasse des habits comme on amasse de la boue,
\VS{17}le riche tombe, et il n’est pas relevé ; il ouvre ses yeux, et il ne trouve rien.
\VS{18}Il se bâtit une maison comme celle de la teigne, comme la cabane que fait un gardien\FTNT{Ps. 49:18.}.
\VS{19}Il se couche riche, et il périt dépouillé ; il ouvre les yeux, et tout a disparu.
\VS{20}Les frayeurs l'atteignent comme des eaux ; le tourbillon l'enlève de nuit.
\VS{21}Le vent d'orient l'emporte, et il s'en va ; il l'arrache de sa demeure comme un tourbillon.
\VS{22}Dieu le précipite à terre et ne l'épargne pas, et le méchant voudrait fuir devant sa main.
\VS{23}On applaudit à sa chute, et on le siffle au lieu où il se tient.
\Chap{28}
\VerseOne{}Il y a pour l'argent une mine d'où on le fait sortir, et pour l'or un lieu d'où on le purifie pour l'affiner ;
\VS{2}le fer se tire de la poussière, et la pierre se fond pour produire l'airain.
\VS{3}L'homme fait cesser les ténèbres, il explore jusqu'aux extrêmes limites les pierres cachées dans l'obscurité et dans l'ombre de la mort.
\VS{4}Il creuse un puits, loin des lieux habités ; ne se souvenant plus de ses pieds, il est suspendu, balancé, loin des humains.
\VS{5}La terre, d'où sort le pain, est bouleversée dans ses entrailles comme par le feu.
\VS{6}Ses pierres sont la demeure du saphir, et l'on y trouve de la poudre d'or.
\VS{7}L'oiseau de proie n'en connaît pas le chemin, l'œil du vautour ne l'aperçoit pas ;
\VS{8}les plus jeunes et fiers animaux n'y ont pas marché, le lion n'y a jamais passé.
\VS{9}L'homme avance sa main sur le roc, il renverse les montagnes depuis la racine ;
\VS{10}il fend des tranchées dans les rochers, et son œil voit tout ce qu'il y a de précieux ;
\VS{11}il arrête l'écoulement des eaux, et il fait sortir ce qui est caché.
\VS{12}Mais la sagesse, où se trouve-t-elle ? Où est le lieu où se tient l'intelligence ?
\VS{13}L'homme n'en connaît pas le prix, elle ne se trouve pas dans la terre des vivants.
\VS{14}L'abîme dit : Elle n'est pas en moi ; et la mer dit : Elle n'est pas avec moi.
\VS{15}Elle ne se donne pas contre de l'or pur, elle ne s'achète pas au poids de l'argent\FTNT{Pr. 3:14 ; Pr. 8:11 ; Pr. 16:16.} ;
\VS{16}elle ne se pèse pas contre de l'or d'Ophir, ni contre le précieux onyx, ni contre le saphir.
\VS{17}Elle ne peut se comparer à l'or ni au verre, elle ne peut s'échanger pour un vase d'or fin.
\VS{18}On ne se souvient ni du corail ni du cristal auprès d'elle : La sagesse vaut plus que les perles.
\VS{19}On ne la compare pas avec la topaze d'Ethiopie ; on ne la met pas en balance avec l'or pur.
\VS{20}D'où vient donc la sagesse ? Où est la demeure de l'intelligence ?
\VS{21}Elle est cachée aux yeux de tous les vivants, elle est cachée aux oiseaux des cieux.
\VS{22}L'abîme et la mort disent : Nous en avons entendu parler de nos oreilles.
\VS{23}C'est Dieu qui en sait le chemin, c'est lui qui en connaît la demeure ;
\VS{24}car il regarde jusqu'aux extrémités de la terre, il voit tout sous les cieux\FTNT{Ps. 14:2 ; Ps. 33:13-14 ; Ps. 102:20.}.
\VS{25}Quand il façonna le poids du vent, et qu'il estima la mesure des eaux\FTNT{Pr. 8:29.},
\VS{26}quand il ordonna des lois à la pluie, et qu'il fit un chemin à l'éclair et au tonnerre,
\VS{27}alors il vit la sagesse et la manifesta ; il l'établit et la sonda.
\VS{28}Puis il dit à l'homme : Voici, la crainte du Seigneur, c'est la sagesse ; se détourner du mal, c'est l'intelligence\FTNT{De. 4:6 ; Jé. 9:24 ; Ps. 111:10 ; Pr. 1:7 ; Pr. 9:10 ; Ec. 12:15.}.
\Chap{29}
\TextTitle{La postérité passée de Job}
\VerseOne{}Job prit de nouveau la parole sous forme sentencieuse et dit :
\VS{2}Oh ! Que ne puis-je être comme aux mois du passé, comme aux jours où Dieu me gardait,
\VS{3}quand sa lampe brillait sur ma tête, quand je marchais à sa lumière dans les ténèbres !
\VS{4}Que ne suis-je comme aux jours de mon automne, où Dieu veillait en ami sur ma tente,
\VS{5}quand le Tout-Puissant était encore avec moi, et que mes serviteurs m'entouraient ;
\VS{6}quand je lavais mes pieds dans le lait, et que le rocher répandait près de moi des torrents d'huile\FTNT{De. 32:13.} !
\VS{7}Si je sortais pour aller à la porte de la ville, et si je me faisais préparer un siège dans la place,
\VS{8}les jeunes gens se retiraient en me voyant, les vieillards se levaient et se tenaient debout.
\VS{9}Les princes s'abstenaient de parler, et mettaient la main sur leur bouche ;
\VS{10}la voix des chefs se taisait, et leur langue s'attachait à leur palais.
\VS{11}L'oreille qui m'entendait me disait heureux, l'œil qui me voyait me rendait témoignage ;
\VS{12}car je délivrais l'affligé qui criait au secours, et l'orphelin qui n'avait personne pour le secourir\FTNT{Ps. 72:12 ; Pr. 21:13.}.
\VS{13}La bénédiction de celui qui allait périr venait sur moi ; je remplissais de joie le cœur de la veuve.
\VS{14}Je me revêtais de la justice et elle se revêtait de moi, j'avais ma droiture pour manteau et pour turban\FTNT{Es. 59:17 ; 1 Th. 5:8 ; Ep. 6:14-17.}.
\VS{15}J'étais les yeux de l'aveugle et les pieds du boiteux.
\VS{16}J'étais le père des pauvres, j'examinais la cause de l'inconnu\FTNT{Pr. 29:7.} ;
\VS{17}je brisais les mâchoires de l'injuste, et j'arrachais la proie d'entre ses dents\FTNT{Ps. 58:7.}.
\VS{18}Alors je disais : Je mourrai dans mon nid, mes jours seront aussi nombreux que le sable ;
\VS{19}l'eau pénétrera dans mes racines, la rosée passera la nuit sur mes branches\FTNT{Jé. 17:5-8 ; Ps. 1:3.} ;
\VS{20}ma gloire se renouvellera sans cesse en moi, et mon arc se renouvellera dans ma main.
\VS{21}On m'écoutait et l'on restait dans l'attente, on gardait le silence devant mes conseils.
\VS{22}Après mes discours, nul ne répliquait, et ma parole était pour tous une bienfaisante rosée ;
\VS{23}ils s'attendaient à moi comme à la pluie, ils ouvraient la bouche comme pour une pluie de printemps.
\VS{24}Je souriais quand ils perdaient confiance, et l'on ne pouvait faire tomber la sérénité de mon visage.
\VS{25}J'aimais à aller avec eux, et je m'asseyais à leur tête ; j'étais comme un roi au milieu de ses gardes, comme un consolateur auprès des affligés.
\Chap{30}
\TextTitle{Son humiliation}
\VerseOne{}Mais maintenant !… Chaque jour je suis la risée de plus jeunes que moi, de ceux dont je dédaignais de mettre les pères parmi les chiens de mon troupeau.
\VS{2}Mais à quoi me servirait la force de leurs mains ? En eux avait péri toute vigueur.
\VS{3}Desséchés par la disette et la faim, ils fuient dans les lieux arides, depuis longtemps abandonnés et déserts ;
\VS{4}ils arrachent près des buissons l'herbe sauvage, et la racine des genêts est leur nourriture.
\VS{5}On les chasse du milieu des hommes, on crie après eux comme après un voleur.
\VS{6}Ils habitent dans le creux des torrents, dans les trous de la terre et des rochers ;
\VS{7}ils hurlent parmi les buissons, ils se rassemblent sous les ronces.
\VS{8}Peuple insensé et sans nom, on les repousse du pays !
\VS{9}Et maintenant, je suis le sujet de leurs chansons, je suis en butte à leurs propos\FTNT{Ps. 69:12 ; La. 3:14.}.
\VS{10}Ils m'ont en horreur, ils s'éloignent de moi, ils ne se retiennent pas de me cracher leur salive au visage.
\VS{11}Ils n'ont aucune retenue et ils m'humilient, ils rejettent tout frein devant moi.
\VS{12}Ces misérables se lèvent à ma droite et me poussent les pieds, ils se fraient contre moi des routes pour ma ruine\FTNT{Ps. 35:15.} ;
\VS{13}ils détruisent mon propre sentier et travaillent à ma perte, eux à qui personne ne viendrait en aide ;
\VS{14}ils viennent contre moi comme par une brèche large, et ils se sont jetés sur moi à cause de ma désolation.
\VS{15}Toutes les terreurs se tournent contre moi ; ma gloire est emportée comme par le vent, mon bonheur a passé comme un nuage\FTNT{Os. 13:3.}.
\VS{16}Et maintenant, mon âme se répand en mon sein, les jours d'affliction m'ont saisi.
\VS{17}La nuit me perce et m'arrache les os, la douleur qui me ronge ne se donne aucun repos.
\VS{18}Par la violence du mal, mon vêtement se déforme, il se colle à mon corps comme ma tunique.
\VS{19}Dieu m'a jeté dans la boue, et je ressemble à la poussière et à la cendre.
\VS{20}Je crie vers toi, et tu ne me réponds pas ; je me tiens debout, et tu m'aperçois.
\VS{21}Tu deviens cruel contre moi, tu t'opposes à moi avec la force de ta main.
\VS{22}Tu me soulèves, tu me fais chevaucher sur le vent, et tu me fais fondre au bruit de la tempête.
\VS{23}Car, je le sais, tu me mènes à la mort, à la demeure fixée pour tous les vivants\FTNT{Hé. 9:27.}.
\VS{24}Mais celui qui va périr n'étend-il pas les mains ? Celui qui est dans le malheur n'implore-t-il pas du secours ?
\VS{25}Ne pleurais-je pas sur l'homme qui passait des jours difficiles ? Mon âme n'avait-elle pas pitié du pauvre\FTNT{Ro. 12:15.} ?
\VS{26}J'attendais le bonheur, et le malheur est arrivé ; j'espérais la lumière, et les ténèbres sont venues.
\VS{27}Mes entrailles bouillonnent sans repos, les jours d'affliction m'ont confronté.
\VS{28}Je marche noirci, mais non par le soleil ; je me lève en pleine assemblée, et je crie.
\VS{29}Je suis devenu le frère des serpents, le compagnon des autruches\FTNT{Ps. 102:7-8.}.
\VS{30}Ma peau noircit et tombe, mes os brûlent et se dessèchent\FTNT{La. 4:8 ; La. 5:10.}.
\VS{31}Ma harpe n'est plus qu'un instrument de deuil, et mon chalumeau ne peut rendre que des voix en pleurs.
\Chap{31}
\TextTitle{Job se justifie}
\VerseOne{}J'avais fait une alliance avec mes yeux, et je n'aurais pas regardé une vierge.
\VS{2}Quelle part Dieu m'eût-il réservée d'en haut ? Quel héritage le Tout-Puissant m'aurait-il envoyé des cieux ?
\VS{3}La ruine n'est-elle pas pour l'injuste, et le malheur pour ceux qui commettent l'iniquité ?
\VS{4}Dieu ne voit-il pas mes voies ? Ne compte-t-il pas tous mes pas\FTNT{Pr. 5:21 ; Pr. 15:3 ; 2 Ch. 16:9.} ?
\VS{5}Si j'ai marché dans le mensonge, si mon pied s'est hâté pour tromper,
\VS{6}que Dieu me pèse dans des balances justes, et il reconnaîtra mon intégrité !
\VS{7}Si mes pas se sont détournés du droit chemin, si mon cœur a suivi mes yeux, si quelque souillure s'est attachée à mes mains,
\VS{8}Que je sème et qu'un autre mange, et tout ce que j'aurais fait produire soit déraciné!
\VS{9}Si mon cœur a été séduit par une femme, si j'ai fait le guet à la porte de mon prochain\FTNT{Pr. 7.},
\VS{10}que ma femme broie le grain pour un autre, et que d'autres se penchent sur elle !
\VS{11}Car c'est un crime, une iniquité punie par les juges ;
\VS{12}c'est un feu qui dévore jusqu'à la destruction, et qui aurait détruit toutes mes récoltes dans leur racine.
\VS{13}Si j'ai méprisé le droit de mon serviteur ou de ma servante, lorsqu'ils étaient en contestation avec moi,
\VS{14}qu'ai-je à faire, quand Dieu se lève ? Qu'ai-je à répondre, quand il châtie ?
\VS{15}Celui qui m'a fait dans le ventre de ma mère ne l'a-t-il pas fait aussi ? Un même Dieu ne nous a-t-il pas formés dans le sein maternel\FTNT{Pr. 14:31 ; Pr. 17:5.} ?
\VS{16}Si j'ai refusé aux pauvres leur désir, si j'ai laissé se consumer les yeux de la veuve\FTNT{Es. 10:2 ; Lu. 18:2-3.},
\VS{17}si j'ai mangé seul mon morceau de pain, sans que l'orphelin en ait sa part,
\VS{18}moi qui l'ai dès ma jeunesse fait grandir près de moi comme un père, et qui dès le sein de ma mère, ai été le guide de la veuve ;
\VS{19}si j'ai vu le malheureux périr faute de vêtements, le pauvre manquer de couverture\FTNT{Mt. 25:41-45.},
\VS{20}sans que ses reins m'aient béni, sans qu'il ait été réchauffé par la toison de mes agneaux ;
\VS{21}si j'ai levé la main contre l'orphelin, parce que je me voyais comme un appui dans les portes\FTNT{Pr. 22:22.} ;
\VS{22}que mon épaule tombe de sa jointure, que mon bras tombe et qu'il se brise l'os !
\VS{23}Car les châtiments de Dieu m'épouvantent, et je ne pourrais pas prévaloir devant sa majesté.
\VS{24}Si j'ai mis dans l'or ma confiance, si j'ai dit à l'or fin : Tu es mon espoir\FTNT{Mc. 10:24 ; 1 Ti. 6:17.} ;
\VS{25}si je me suis réjoui de ma grande puissance, de la quantité des richesses que ma main a acquise\FTNT{Ps. 62:11.} ;
\VS{26}si j'ai regardé le soleil quand il brillait, la lune quand elle s'avançait de façon majestueuse,
\VS{27}et si mon cœur s'est laissé secrètement séduire, si ma main a envoyé des baisers de ma bouche ;
\VS{28}c'est encore une iniquité que doit punir le juge, et j'aurais renié le Dieu d'en haut !
\VS{29}Si je me suis réjoui du malheur de mon ennemi, si j'ai sauté d'allégresse quand le mal l'a atteint\FTNT{Mt. 5:43-44.},
\VS{30}moi qui n'ai pas permis à ma langue de pécher en demandant sa mort par des malédictions ;
\VS{31}si les gens de ma tente ne disaient pas : Où est celui qui n'a pas été rassasié de sa viande\FTNT{Ps. 27:2.} ?
\VS{32}Si l'étranger passait la nuit dehors, si je n'ouvrais pas ma porte au voyageur\FTNT{Ge. 19:1-2 ; De. 10:19 ; 1 Pi. 4:9 ; Hé. 13:2.} ;
\VS{33}si, comme les hommes, j'ai caché mes transgressions et mon crime dans mon sein\FTNT{Ge. 3:10-12 ; Pr. 28:13.},
\VS{34}parce que je craignais la multitude, et je craignais le mépris des familles, en sorte que je restais tranquille et n'osais franchir ma porte…
\VS{35}Oh ! Qui me fera trouver quelqu'un qui m'écoute ? Voilà ma défense toute signée : Que le Tout-Puissant me réponde ! Qui me donnera la plainte écrite par mon adversaire ?
\VS{36}Je porterai son écrit sur mon épaule, je l'attacherai sur mon front comme une couronne ;
\VS{37}je lui déclarerai le nombre de mes pas, je m'approcherai de lui comme un prince.
\VS{38}Si ma terre crie contre moi, et que ses sillons pleurent ;
\VS{39}si j'en ai mangé le produit sans l'avoir payée, et que j'aie attristé l'âme de ses anciens maîtres ;
\VS{40}Qu'elle en produise des épines au lieu du froment, et de l'ivraie au lieu de l'orge ! C'est ici la fin des paroles de Job.
\Chap{32}
\TextTitle{Discours d'Elihu : reproches à Job et à ses amis}
\VerseOne{}Ces trois hommes-là cessèrent de répondre à Job, parce qu'il se regardait comme juste.
\VS{2}Élihu, fils de Barakeel de Buz, de la famille de Ram, s'enflamma de colère contre Job, parce qu'il disait son âme juste devant Dieu.
\VS{3}Et sa colère s'enflamma contre ses trois amis, parce qu'ils ne trouvaient rien à répondre et que néanmoins ils condamnaient Job.
\VS{4}Comme ils étaient plus âgés que lui, Elihu avait attendu jusqu'à ce moment pour parler à Job.
\VS{5}Mais, voyant que ces trois hommes n'avaient plus aucune réponse à la bouche, Elihu se mit en colère.
\VS{6}Et Elihu, fils de Barakeel de Buz, prit la parole et dit : Je suis jeune et vous êtes des vieillards ; c'est pourquoi j'ai craint, j'ai eu peur de vous faire connaître mon sentiment.
\VS{7}Je disais: les jours parleront, et le grand nombre des années fera connaître la sagesse.
\VS{8}L'esprit dans l'homme, c'est l'esprit, le souffle du Tout-Puissant qui rend intelligent\FTNT{Da. 1:17 ; Da. 2:21 ; Pr. 2:6 ; Ec. 2:26.} ;
\VS{9}ce ne sont pas les aînés qui sont sages, ce ne sont pas les vieillards qui comprennent ce qui est juste.
\VS{10}C'est pourquoi je dis : Ecoute-moi ! Et je dirai aussi ma pensée.
\VS{11}J'ai attendu la fin de vos discours, j'ai écouté vos raisonnements, jusqu'à ce que vous ayez bien examiné les discours de Job.
\VS{12}J'ai pris le soin de vous écouter ; et voici, aucun de vous n'a convaincu Job, aucun n'a répondu à ses paroles.
\VS{13}Qu'il ne vous n'arrive pas de dire: nous avons trouvé la sagesse ; c'est Dieu qui le poursuit, et non pas l'homme !
\VS{14}Il n'a pas dirigé ses discours contre moi : Aussi je ne lui répondrai pas à votre manière.
\VS{15}Ils sont étonnés! Ils ne répondent plus rien! On leur a ôté la parole!
\VS{16}J'ai attendu jusqu'à ce qu'ils n'ont plus rien dit, car ils sont demeurés muets, et ils n'ont plus su que répondre.
\VS{17}A mon tour, je veux répondre pour moi, et je veux donner mon avis.
\VS{18}Car je suis rempli de discours, l'esprit qui est en mon sein me presse.
\VS{19}Mon sein est comme vin sans air, comme des outres neuves qui vont éclater\FTNT{Mt. 9:17 ; Mc. 2:22 ; Lu. 5:38.}.
\VS{20}Je parlerai pour respirer à l'aise, j'ouvrirai mes lèvres et je répondrai.
\VS{21}Je ne ferai pas acception de personnes, et je flatterai aucun homme.
\VS{22}Car je ne sais pas flatter : Mon Créateur m'enlèverait bien vite.
\Chap{33}
\TextTitle{Discours d'Elihu : la justice de Dieu}
\VerseOne{}C'est pourquoi Job, écoute mon discours, je te prie et prête l'oreille à toutes mes paroles !
\VS{2}Voici, j'ouvre la bouche, ma langue parle dans mon palais.
\VS{3}Mes paroles exprimeront la droiture de mon cœur, mes lèvres diront la vérité pure.
\VS{4}L'Esprit de Dieu m'a fait, et le souffle du Tout-Puissant me donne la vie\FTNT{Ge. 2:7.}.
\VS{5}Si tu peux, réponds-moi, dresse-toi contre moi, demeure ferme!
\VS{6}Voici, je suis pour le Dieu fort, selon que tu en as parlé; j'ai été formé de la terre tout comme toi.\FTNT{Ac. 14:15.} ;
\VS{7}Voici ma terreur ne te trouble pas, et ma main ne s'appesantit pas sur toi
\VS{8}Quoi qu'il en soit, tu as dit, moi l'entendant, j'ai entendu la voix de tes discours:
\VS{9}Je suis pur, sans péché, je suis net, il n'y a pas d'iniquité en moi.
\VS{10}Voici il cherche à rompre avec moi, il me considère comme son ennemi ;
\VS{11}il met mes pieds dans les ceps, il surveille tous mes chemins.
\VS{12}Je te répondrai qu'en cela tu n'as pas été juste, car Dieu sera toujours plus grand que l'homme.
\VS{13}Pourquoi as-tu donc plaidé contre lui ? Car il ne rend pas compte de toutes ses actions.
\VS{14}Car Dieu parle une première fois, et une seconde fois à celui qui n'aura pas pris garde à la première.
\VS{15}Par des songes, par des visions nocturnes, quand les hommes tombent dans un profond sommeil, quand ils dorment sur leur couche.
\VS{16}Alors il ouvre l'oreille de l'homme d'une mauvaise action et de rabaisser la fierté de l'homme.
\VS{17}afin de détourner l'homme de son œuvre et de le préserver de l'orgueil,
\VS{18}il garantit son âme de la fosse, et sa vie de l'épée.
\VS{19}L'homme est aussi châtié par des douleurs sur son lit, à cause d'une lutte perpétuelle en ses os\FTNT{Ps. 38:4.}.
\VS{20}Alors sa vie prend en horreur le pain et son âme les mets les plus désirés\FTNT{Ps. 107:18.} ;
\VS{21}Sa chair est tellement consumée qu'elle paraît plus, ses os sont tellement brisés, qu'on y connaît plus rien;
\VS{22}son âme s'approche de la fosse, et sa vie des messagers de la mort.
\VS{23}Mais s'il y a pour cet homme un messager qui interprète, un d'entre les mille, pour lui annoncer la voie de la droiture,
\VS{24}alors Dieu prend pitié de lui et dit : Garantis-le, afin qu'il ne descende pas dans la fosse ; j'ai trouvé la propitiation!
\VS{25}Sa chair devient plus délicate qu'elle n'était dan son enfance; il revient aux jours de sa jeunesse.
\VS{26}Il supplie Dieu par ses prières, et Dieu lui est favorable, il lui laisse voir sa face avec joie, et lui rend sa justice\FTNT{Es. 58:9.}.
\VS{27}Il regarde vers les hommes et dit : J'ai péché, j'ai violé la justice, et je n'ai pas été puni comme je le méritais ;
\VS{28}Dieu a racheté mon âme afin qu'elle ne passe pas dans la fosse, et ma vie voit encore la lumière !
\VS{29}Voilà ce que Dieu fait, deux fois, trois fois, envers l'homme\FTNT{Ps. 62:11.},
\VS{30}pour ramener son âme de la fosse, pour l'éclairer de la lumière des vivants\FTNT{Ps. 56:14.}.
\VS{31}Sois attentif, Job, écoute-moi ! Tais-toi, et je parlerai !
\VS{32}Si tu as quelque chose à dire, réponds-moi ! Parle, car je désire te justifier.
\VS{33}Sinon, écoute, tais-toi et je t'enseignerai la sagesse.
\Chap{34}
\TextTitle{Discours d'Elihu : il accuse Job de se révolter}
\VerseOne{}Elihu reprit la parole, et dit :
\VS{2}Sages, écoutez mes discours ! Vous qui avez la connaissance, prêtez-moi l'oreille !
\VS{3}Car l'oreille discerne les discours, comme le palais savoure ce qu'il mange.
\VS{4}Choisissons ce qui est juste, voyons entre nous ce qui est bon.
\VS{5}Job dit : Je suis juste, et Dieu a écarté ma justice;
\VS{6}mentirai-je à mon droit? Ma flèche est mortelle sans que j'aie commis de crime.
\VS{7}où y a-t-il un homme comme Job, qui boit le péché la moquerie comme l'eau,
\VS{8}qui marche en compagnie des ouvriers d'iniquité, et qui fréquente  avec les hommes marchant de pair avec les hommes méchants ?
\VS{9}Car il a dit : Il est inutile à l'homme de plaire à Dieu\FTNT{Mal. 3:14.}.
\VS{10} C'est pourquoi écoutez, vous qui avez de l'intelligence, écoutez-moi ! Loin de Dieu la méchanceté, loin du Tout-Puissant l'injustice\FTNT{De. 32:4 ; Ps. 92:16 ; Ro. 9:14.} !
\VS{11}Car il rend à l'homme selon son œuvre, il fait trouver à chacun selon sa voie\FTNT{Jé. 17:10 ; Jé. 32:19 ; Ez. 7:27 ; Pr. 24:12 ; Mt. 16:27 ; Ro. 2:6 ; 2 Co. 5:10 ; Ep. 6:8 ; Ap. 22:12.}.
\VS{12}Certes, Dieu ne commet pas l'injustice ; le Tout-Puissant ne renverse pas le droit.
\VS{13}Qui lui a donné la terre en charge ? Ou Qui a placé la terre habitable?
\VS{14}S'il ne pensait qu'à lui-même, s'il retirait à lui son esprit et son souffle\FTNT{Ps. 104:29.},
\VS{15}toute chair périrait ensemble, et l'homme retournerait dans la poussière\FTNT{Ge. 3:19 ; Ec. 3:20 ; Ec. 12:9.}.
\VS{16}Si donc tu as de l'intelligence, écoute ceci, prête l'oreille à ce que tu entendras de moi.
\VS{17}Comment celui qui n'aimerait pas à faire la justice jugerait-il le monde? Et condamneras-tu celui qui est souverainement juste ?
\VS{18}Dira-t-on à un roi, qu'il est un scélérat ? Et aux princes, qu'ils sont des méchants ?
\VS{19}Combien moins le dira-t-on à celui qui n'a point d'égard à la personne des grands, et qui ne connaît point les riches pour les préférer aux pauvres, parce qu'ils sont tous l'ouvrage de ses mains\FTNT{De. 10:17 ; 2 Ch. 19:7 ; Ac. 10:34 ; Ga. 2:6 ; Ro. 2:11 ; Ep. 6:9 ; Col. 3:25.} ?
\VS{20}En un moment, ils mourront ; au milieu de la nuit, un peuple est ébranlé et passe ; le puissant s'en va, sans la main d'aucun homme.
\VS{21}Car les yeux de Dieu sont sur les voies de l'homme, il regarde tous ses pas.
\VS{22}Il n'y a ni ténèbres ni ombre de la mort où puissent se cacher les ouvriers d'iniquité.
\VS{23}Dieu ne regarde pas à deux fois un homme, pour le faire aller en jugement avec lui.
\VS{24}Il brise les puissants par des voies incompréhensibles, et il n établit d'autres à leur place ;
\VS{25}car il connaît leurs œuvres. Il les renverse de nuit, et ils sont écrasés ;
\VS{26}il les frappe comme des impies, au lieu où se tiennent tous les regards.
\VS{27}Du fait qu'ils se sont détournés de lui, et qu'ils n'ont considéré aucune de ses voies.
\VS{28}ils ont fait monter à Dieu le cri du pauvre, et il a entendu le cri des affligés\FTNT{Ja. 5:4.}.
\VS{29}S'il donne le repos, qui est-ce qui causera du trouble? S'il cache sa face à quelqu'un, qui le regardera, qu'il s'agisse de toute une nation ou d'un seul un homme?
\VS{30}afin que l'hypocrite ne règne pas, de peur qu'il plus un piège pour le peuple.
\VS{31}Car a-t-il jamais dit à Dieu : J'ai été pardonné, je ne pécherai plus ;
\VS{32}montre-moi ce que je ne vois pas ; si j'ai fait le mal, je ne le ferai plus ?
\VS{33}Mais Dieu ne te le rendra-t-il pas, puisque tu as rejeté son châtiment, quand tu as fait le choix que tu as fait? Pour moi, je ne sais que dire à cela; mais toi, si tu as quelque chose à répondre, parle.
\VS{34}Les gens de bon sens diront avec moi, et tout homme sage en conviendra,
\VS{35}que Job ne parle pas avec connaissance, et ses paroles manquent d'intelligence.
\VS{36}Ah! Mon père, que Job soit éprouvé jusqu'à ce qu'il soit vaincu, puisqu'il vaincu, puisqu'il répond comme les impies.
\VS{37}Car il ajoute péché sur péché; il applaudit au milieu de nous ; il parle de plus en plus contre Dieu.
\Chap{35}
\TextTitle{Discours d'Elihu : il reproche à Job ses propos irréfléchis}
\VerseOne{}Elihu reprit la parole et dit :
\VS{2}Penses-tu avoir raison de dire : Je suis juste devant Dieu ?
\VS{3}Quand tu dis : Que me sert-il, et que gagnerais-je de plus sans pécher ?
\VS{4}Je te répondrai en ces termes, et à tes amis qui sont avec toi.
\VS{5}Regarde les cieux, et considère-les ! Vois les nuées, comme elles sont plus hautes que toi !
\VS{6}Si tu pèches, quel mal fais-tu à Dieu ? Et si tes péchés se multiplient, quel mal reçoit-il ?
\VS{7}Si tu es juste, que lui donnes-tu ? Que reçoit-il de ta main ?
\VS{8}C'est à un homme, comme tu es, que ta méchanceté peut seule nuire, et c'est au fils d'un homme que ta justice peut seule être utile.
\VS{9}On fait crier les opprimés par la grandeur des maux qu'on leur afflige; ils crient à cause de la violence des grands;
\VS{10}Et nul ne dit : Où est le Dieu qui m'a fait, qui donne de quoi chanter pendant la nuit ,
\VS{11}qui nous instruit plus que les animaux de la terre, et plus intelligent que les oiseaux des cieux ?
\VS{12}On crie donc à cause de la fierté des méchants; mais Dieu ne l'exauce pas.
\FTNT{Es. 1:15 ; Ez. 8:18 ; Mi. 3:4 ; Jn. 9:31.}.
\VS{13} Cependant que ce soit en vain; que Dieu n'écoute pas, et que le Tout-puissant n'y a pas égard.
\VS{14}Encore moins dois-tu lui dire, tu ne le vois pas; car le jugement est devant lui, attends-le donc!
\VS{15}Mais maintenant, ce n'est rien ce que sa colère exécute, et il n'a pas encore pris connaissance en profondeur toutes les choses que tu as faites.
\VS{16}Job ouvre donc sa bouche pour se plaindre, il multiplie les paroles sans intelligence.
\Chap{36}
\TextTitle{Discours d'Elihu : Dieu traite les hommes selon leurs oeuvres}
\VerseOne{}Elihu continua de parler, et dit :
\VS{2}Attends un peu, et je te montrerai qu'il y a encore d'autres raisons pour la cause de Dieu.
\VS{3}Je tirerai de loin mes raisons, et je défendrai la justice du Créateur.
\VS{4}Car certainement il n'y aura rien de faux en tout ce que je dirai, et celui qui est avec toi, est parfait dans sa connaissance.
\VS{5}Dieu est puissant, mais il ne méprise personne ; il est puissant par la force de son coeur.
\VS{6}Il ne laisse pas vivre le méchant, et il fait droit aux pauvres.
\VS{7}Il ne détourne pas ses yeux de dessus les justes, il les place sur le trône avec les rois, il les y fait asseoir pour toujours, afin qu'ils soient élevés\FTNT{Ps. 33:18 ; Ps. 34:16.}.
\VS{8}S'ils sont liés de chaînes, s'ils sont pris dans les liens de l'affliction,
\VS{9}il leur fait connaître leurs œuvres, leurs transgressions, leur orgueil.
\VS{10}Alors il ouvre leur oreille pour leur discipline, il leur dit de se détourner de l'iniquité.
\VS{11}S'ils écoutent, et s'ils le servent, ils achèvent leurs jours dans le bonheur, leurs années dans la joie.
\VS{12}S'ils n'écoutent pas, ils passent par l'épée, ils expirent dans leur aveuglement.
\VS{13}Ceux qui sont hypocrites dans leur cœur, ils ne crient pas à lui quand il les a liés ;
\VS{14}leur personne meurt dans sa jeunesse, leur vie s'éteint parmi les débauchés.
\VS{15}Mais Dieu sauve celui qui est affligé de son oppression, et c'est par la détresse qu'il lui ouvre les oreilles.
\VS{16}Il t'écartera aussi de la détresse, pour te mettre au large, loin de toute angoisse, et ta table sera chargée de viandes grasses\FTNT{Ps. 50:15 ; Ps. 63:6.}.
\VS{17}Or tu remplis le jugement du méchant, mais le jugement et le droit subsisteront.
\VS{18}Certainement Dieu et irrité; prends garde qu'il ne te  plonge dans l'affliction, car il n'y aura pas alors de rançon si grande pour te délivrer\FTNT{Ps. 49:8.} !
\VS{19}Tes cris valent-ils ton or, et même toutes les forces qui se trouvent dans tes richesses ?
\VS{20}Ne soupire pas après la nuit, qui enlève les peuples de leur place.
\VS{21}Garde-toi de te retourner vers l'iniquité, car la souffrance t'y dispose.
\VS{22}Dieu est élevé par sa puissance ; qui saurait enseigner comme lui ?
\VS{23}Qui lui prescrit le chemin qu'il devait tenir? Qui lui dit : Tu a fait une injustice ?
\VS{24}Souviens-toi de célébrer ses ouvrages, que tous les hommes voient.
\VS{25}Tout homme les voit, chacun les contemple de loin.
\VS{26}Dieu est grand, mais nous ne le connaissons pas, quant au nombre de ses années il est insondable\FTNT{Es. 63:16 ; Ps. 92:8 ; Ps. 93:2 ; Ps. 102:13 ; La. 5:19.}.
\VS{27}Parce qu'il met les eaux  en petites gouttes, elle répandent la pluie selon la vapeur d'eau qui la contient ;
\VS{28}les nuées la font dégoutter, elles coulent sur les hommes en abondance.
\VS{29}Et qui pourra comprendre l'étendue des nuages et le son éclatant de sa tente ?
\VS{30}Voici, il étend sa lumière sur elle, et il se cache jusque dans les profondeurs de la mer.
\VS{31} Or c'est par ces choses qu'il juge les peuples, qu' il donne la nourriture en abondance.
\VS{32} Il tient caché dans les paumes de ses mains le feu étincelant, et lui ordonne de frapper de ce qui se présente à sa rencontre.
\VS{33} Son bruit porte les nouvelles ; les troupeaux font connaître qu'il approche.
\Chap{37}
\TextTitle{Discours d'Elihu : conclusion}
\VerseOne{}Mon cœur même à cause de cela est tout tremblant, il sort de sa place.
\VS{2}Ecoutez attentivement et en tremblant le bruit de sa voix, le grondement qui sort de sa bouche\FTNT{Ps. 29:3-9.} !
\VS{3}Il le conduit dans toute l'étendue des cieux, et son éclair brille jusqu'aux extrémités de la terre\FTNT{Ps. 97:4.}.
\VS{4}Après lui de l'élève un grand bruit, il tonne de sa voix majestueuse; il ne tarde pas après que sa voix a été entendue\FTNT{Jé. 10:13.}.
\VS{5}Dieu tonne avec sa voix d'une manière étonnante ; il fait de grandes choses que nous ne comprenons pas.
\VS{6}Car il dit à la neige : Tombe sur la terre ! Il le dit à la pluie, même aux plus fortes pluies.
\VS{7}Il met un sceau à la main de tous les hommes pour reconnaître tous les hommes qui sont son ouvrage.
\VS{8}Les bêtes entrent dans leurs tanières, et elles demeurent dans leurs repaires.
\VS{9}L'ouragan vient du fond du sud, et le froid vient des vents du nord.
\VS{10}Par son souffle, Dieu donne la glace, et il réduit l'espace où se répondaient au large les eaux\FTNT{Ps. 147:17-18.}.
\VS{11}Il lasse les nuages à force d'arroser, il écarte les nuages par sa lumière.
\VS{12}Et ceux-ci font plusieurs tours pour faire ce qu'il a commandé, sur la face de la terre sur la face de la terre habitée ;
\VS{13}Il les fait venir pour s'en servir soit comme une verge pour la terre, soit pour répandre ses bienfaits\FTNT{Ex. 9:18-23 ; 1 S. 12:18-19.}.
\VS{14}Job, arrête-toi, prête l'oreille à ces choses ! Considère encore les merveilles de Dieu !
\VS{15}Sais-tu comment Dieu les dispose, et fait briller la lumière de ses nuages?
\VS{16}Connais-tu le balancement des nuages, les merveilles de celui dont la science est parfaite ?
\VS{17}Sais-tu pourquoi tes vêtements sont chauds quand la terre se repose par le vent du midi ?
\VS{18}Peux-tu étendre avec lui les cieux, aussi fermes qu'un miroir de fonte ?
\VS{19}Montre-nous ce que nous pouvons lui dire ; car nous ne saurions rien dire par ordre à cause de nos ténèbres. 
\VS{20}Lui racontera-t-on quand je parlerai ? S'il y a un homme qui en parle, certainement il en sera englouti ?
\VS{21}Et maintenant on ne voit pas la lumière du soleil qui resplendit dans les cieux, lorsque le vent passe et le nettoie ;
\VS{22}Le temps qui la reluit comme l'or vient du nord. Il y a en Dieu une majesté redoutable.
\VS{23}Nous ne saurions comprendre le Tout-Puissant, grand en puissance, en jugement et en abondante justice, il n'opprime personne !
\VS{24}C'est pourquoi les hommes le craignent ; mais il ne les voit pas tous sages de cœur\FTNT{Ps. 92:7 ; Ro. 1:21.}.
\Chap{38}
\TextTitle{Yahweh interroge Job}
\VerseOne{}Yahweh répondit à Job du milieu de le tourbillon et dit :
\VS{2}Qui est celui qui obscurcit mes décisions par des paroles sans connaissance ?
\VS{3}Ceins maintenant tes reins comme un vaillant homme ; je t'interrogerai, et tu me fera voir ta science.
\VS{4}Où étais-tu quand je fondais la terre ? Dis-le, si tu as de l'intelligence\FTNT{Pr. 8:29.}.
\VS{5}Qui en a réglé les mesures, le sais-tu ? Ou qui a appliqué sur elle le niveau ?
\VS{6}Sur quoi ses bases sont-elles plantées ? Ou qui en a posé la pierre angulaire pour la soutenir\FTNT{Ps. 104:5.}?
\VS{7}Quand les étoiles du matin se réjouissent ensemble, et que tous les fils de Dieu poussent des cris de joie \FTNT{Ps. 148:3.} ?
\VS{8}Qui a renfermé la mer dans ses bords, quand elle fut tirée de la matrice et qu'elle sortit? 
\VS{9}quand je lui donnai la nuée pour vêtement, et l'obscurité pour langes ;
\VS{10}que je lui imposai ma loi, et que je lui mis des barrières et des portes;
\VS{11} et quand je dis : Tu viendras jusqu'ici, tu n'iras pas plus loin ; ici s'arrêtera l'orgueil de tes flots ?
\VS{12}Depuis que tu es au monde, as-tu commandé au matin et as-tu montré à l'aube du jour le lieu où elle doit se lever,
\VS{13}pour qu'elle saisisse les extrémités de la terre, et que les méchants en soient chassés ;
\VS{14}pour que la terre prenne une forme comme l'argile qui reçoit un sceau, et qu'elle soit parée comme d'un vêtement nouveau ;
\VS{15}pour que la lumière soit ôtée aux méchants, et que le bras qui se lève soit brisé\FTNT{Ps. 10:15.} ?
\VS{16}As-tu pénétré jusqu'aux sources de la mer ? T'es-tu promené dans les profondeurs de l'abîme ?
\VS{17}Les portes de la mort se sont-elles découvertes à toi ? As-tu vu les portes de l'ombre de la mort ?
\VS{18}As-tu compris l'étendue de la terre ? Si tu sais tout cela, dis-le !
\VS{19}Où est la demeure de la lumière, et où est le lieu des ténèbres ?
\VS{20}Pour que tu les prennes à leur limite, et que tu connaisses le chemin de leur maison ?
\VS{21}Tu le sais, car alors tu étais né, et le nombre de tes jours est grand !
\VS{22}Es tu entré dans les trésors de la neige ? As-tu vu les trésors de grêle,
\VS{23}que je réserve pour les temps de détresse, pour les jours de guerre et de bataille\FTNT{Ex. 9:23 ; Jos. 10:11 ; Ap. 8 :7.} ?
\VS{24}Par quel chemin la lumière se partage la lumière, et le vent d'orient se répand-il sur la terre\FTNT{Jn. 3:8.} ?
\VS{25}Qui a ouvert un conduit aux inondations, et tracé la route de l'éclair et du tonnerre,
\VS{26}pour qu'elle pleuve sur une terre sans habitants, sur un désert sans hommes\FTNT{Ps. 104:13-14 ; PS. 147:8 ; Ac. 14:17.} ;
\VS{27}pour qu'elle abreuve les lieux solitaires et arides, et qu'elle fasse germer et sortir l'herbe ?
\VS{28}La pluie a-t-elle un père ? Qui enfante les gouttes de la rosée ?
\VS{29}De quel sein est sortie la glace ? Et qui enfante le givre du ciel,
\VS{30}pour que les eaux se cachent comme une pierre, et que le dessus de l'abîme soit enchaîné ?
\VS{31}Peux-tu resserer les liens des pléiades ou détacher les chaînes d'orient\FTNT{Am. 5:8.}?
\VS{32}Fais-tu sortir en leur temps les signes du zodiaque, et conduis-tu la Grande Ourse avec ses petits ?
\VS{33}Connais-tu les lois du ciel ? Disposes-tu de son pouvoir sur la terre\FTNT{Jé. 31:35-36 ; Ps. 104:4.} ?
\VS{34}Élèves-tu la voix jusqu'aux nuées, pour que des eaux abondantes te couvrent ?
\VS{35}Envoies-tu les éclairs ? Partent-ils ? Te disent-ils : Nous voici ?
\VS{36}Qui a mis la sagesse dans le cœur, ou qui a donné l'intelligence à l'esprit\FTNT{Ec. 2:26.} ?
\VS{37}Qui Est-ce qui peut avec intelligence compter les nuages, et pour placer les outres des cieux,
\VS{38}Quand la poussière est détrompée par les eaux qui l'arrosent, et que les mottes viennent à se joindre ?
\Chap{39}
\TextTitle{Yahweh démontre son omnipotence}
\VerseOne{}Chasses-tu de la proie pour la lionne, et apaises-tu la faim des lionceaux\FTNT{Ps. 104:21.},
\VS{2}Quand ils se tapissent dans leurs tanières et se tiennent aux aguets dans leur repaire ?
\VS{3}Qui est-ce qui apprête la nourriture au corbeau, quand ses petits crient à Dieu, et qu'ils vont errants, parce qu'ils n'ont point de quoi manger\FTNT{Ps. 104:27 ; Ps. 147:9 ; Mt. 6:26.} ?
\VS{4}Sais-tu quand les boucs de rochers mettent bas ? Observes-tu les biches de rochers quand elles font leurs petits\FTNT{Ps. 29:9.} ?
\VS{5}Comptes-tu les mois de leur gestation, et sais-tu le temps auquel elles font leurs petits ?
\VS{6}Et qu'elles se courbent pour mettre bas leurs petits et se délivrent de leurs douleurs ?
\VS{7}Leurs petits se fortifient, ils croissent en plein air, ils s’en vont et ne reviennent plus vers elles.
\VS{8}Qui a laissé aller libre l’âne sauvage ? et qui a délié les liens de l’âne farouche ?
\VS{9}Auquel j’ai donné le désert pour maison, et la terre inhabitée pour ses retraites\FTNT{Jé. 2:24.} ?
\VS{10}Il se rit du bruit des villes, il n'entend pas les cris d'un exacteur.
\VS{11}Les montagnes qu'il va épiant çà et là, sont ses pâturages, et il cherche toute sorte de verdure. 
\VS{12}Le buffle voudra-t-il te servir, ou demeurera-t-il à ta crèche ? 
\VS{13}Lies-tu le buffle avec son licou pour labourer ? ou rompra-t-il les mottes des vallées après toi ? 
\VS{14}Te fies-tu à lui parce que sa force est grande, et lui abandonnes-tu ton travail? 
\VS{15}Comptes-tu sur lui pour rentrer ta semence, et pour l'amasser sur ton aire ? 
\VS{16}As-tu donné aux paons ce plumage qui est si brillant, ou à l'autruche les ailes et les plumes ? 
\VS{17}Néanmoins elle abandonne ses oeufs à terre, et les fait échauffer sur la poussière ;
\VS{18}et elle oublie que le pied peut les écraser, ou que les bêtes des champs peuvent les fouler. 
\VS{19}Elle est dure envers ses petits, comme s’ils n’étaient pas siens. Son travail est vain, elle ne s’en inquiète pas.
\VS{20}Car Dieu l'a privée de sagesse et ne lui a pas donné l'intelligence.
\VS{21}A la première occasion elle se dresse en haut, et se moque du cheval et de celui qui le monte. 
\VS{22}As-tu donné la force au cheval ? et as-tu revêtu son cou d'un hennissement éclatant comme le tonnerre ? 
\VS{23}Fais-tu bondir le cheval comme la sauterelle ? le son magnifique de ses narines est effrayant.
\VS{24}Il creuse la terre de son pied, il s'égaie dans sa force, il va à la rencontre d'un homme armé ;
\VS{25}Il se rit de la frayeur, il ne s'épouvante de rien, et il ne se détourne point de devant l'épée.
\VS{26}Il n'a point peur des flèches qui sifflent tout autour de lui, ni du fer luisant de la lance et du javelot. 
\VS{27}Il creuse la terre, plein d'émotion et d'ardeur au son de la trompette, et il ne peut se retenir. 
\VS{28}Au son bruyant de la trompette, il dit : En avant ! En avant ! Il flaire de loin la bataille, le tonnerre des capitaines, et le cri de triomphe.
\VS{29}Est-ce par ta sagesse que l'épervier prend son vol, et qu'il étend ses ailes vers le midi ?
\VS{30}Est-ce par ton commandement que l'aigle s'élève, et qu'il place son nid sur les hauteurs\FTNT{Jé. 49:16 ; Abd. 1:4.} ?
\VS{31}Elle habite sur les rochers, et elle s'y tient ; même sur les sommets des rochers et dans des lieux forts. 
\VS{32}De là il découvre le gibier, ses yeux voient de loin.
\VS{33}Ses petits auss sucent le sang ; et là où sont des cadavres, il s'y trouve aussitôt\FTNT{Mt. 24:28 ; Lu. 17:37.}.
\TextTitle{Yahweh lui pose une question}
\VS{34}Yahweh prit encore la parole et dit à Job :
\VS{35}Celui qui conteste avec le Tout-puissant, lui apprendra-t-il quelque chose ? Que celui qui dispute avec Dieu réponde à ceci.
\TextTitle{Réponse de Job}
\VS{36}Alors Job répondit à Yahweh et dit :
\VS{37}Voici, je suis un homme vil ; que te répondrais-je ? Je mets ma main sur ma bouche\FTNT{Ps. 39:10.}.
\VS{38}J'ai parlé une fois, mais je ne répondrai plus ; j'ai même parlé deux fois mais je n'ajouterai plus.
\Chap{40}
\TextTitle{Yahweh questionne encore Job}
\VerseOne{}Et Yahweh répondit à Job du milieu d'un tourbillon, et lui dit :
\VS{2}Ceins maintenant tes reins comme un vaillant homme ; je t'interrogerai et tu m'enseigneras.
\VS{3}Anéantiras-tu mon jugement ? me condamneras-tu pour te justifier\FTNT{Ps. 51:6 ; Ro. 3:4.} ?
\VS{4}As-tu un bras comme celui de Dieu ; tonnes-tu de la voix, comme lui ?
\VS{5}Pare-toi maintenant de magnificence et de grandeur, et revêts-toi de majesté et de gloire.
\VS{6}Répands les fureurs de ta colère, d'un regard, humilie tous les orgueilleux
\VS{7}D'un regard humilie les orgueilleux, écrase sur place les méchants,
\VS{8}cache-les tous ensemble dans la poussière, enferme leur face dans les ténèbres !
\VS{9}Alors je rends hommage à mon sauveur qui me sauve par sa droite.
\VS{10}Voici le béhémoth, que j'ai façonné comme toi ! Il mange de l'herbe comme le bœuf.
\VS{11}Regarde donc, sa force est dans ses reins, et sa puissance dans les muscles de son ventre ;
\VS{12}il plie sa queue aussi ferme qu'un cèdre ; les tendons de ses cuisses sont entrelacés ;
\VS{13}ses os sont des tubes d'airain, ses membres sont comme des barres de fer.
\VS{14}C’est le chef-d’œuvre de Dieu ; celui qui l’a fait lui a donné son épée.
\VS{15}Car les montagnes lui apportent sa pâture, là où se jouent toutes les bêtes des champs.
\VS{16}Il se couche sous les lotus, caché dans les roseaux et les marécages ;
\VS{17}les lotus le couvrent de leur ombre, les saules du torrent l'enveloppent.
\VS{18}Voilà, il engloutit une rivière en buvant, et il ne s'en retire pas vite ; et il ne s'étonnerait pas quand le Jourdain se dégorgerait dans sa gueule. 
\VS{19}Il l'engloutit en le voyant, et son nez passe au travers des empêchements qu'il rencontre.  
\VS{20}Attireras-tu le léviathan à l'hameçon ? Saisiras-tu sa langue avec une corde ?
\VS{21}Mettras-tu un jonc dans ses narines ? Lui perceras-tu la mâchoire avec un crochet ?
\VS{22}Accumulera-t-il les supplications ? Te parlera-t-il d'une voix douce ?
\VS{23}Fera-t-il une alliance avec toi, pour te prendre pour toujours comme esclave ?
\VS{24}Joueras-tu avec lui comme avec un oiseau ? L'attacheras-tu pour amuser les jeunes filles ?
\VS{25}Les pêcheurs en trafiquent-ils ? Le partagent-ils entre les marchands ?
\VS{26}Couvriras-tu sa peau de dards, et sa tête de harpons ?
\VS{27}Mets ta main contre lui, et tu ne te souviendras plus de l'attaquer.
\VS{28}Voici, on est trompé dans son attente ; à sa vue n'est-on pas terrassé ?
\Chap{41}
\VerseOne{}Nul n'est assez féroce pour l'exciter ; qui donc me résisterait en face ?
\VS{2}De qui suis-je le débiteur ? Je le paierai. Sous le ciel tout m'appartient\FTNT{Ex. 19:5 ; De. 10:14 ; Ps. 24:1 ; Ps. 50:12 ; 1 Co. 10:26 ; Ro. 11:35.}.
\VS{3}Je veux encore parler de ses discours, et de sa force, et de la beauté de sa structure.
\VS{4}Qui découvrira son vêtement devant ma face ? Qui viendra freiner ses mâchoires par un mors ?
\VS{5}Qui ouvrira les portes devant sa face ? Autour du lion habite la terreur.
\VS{6}Ses magnifiques et puissants boucliers sont fermés comme un sceau ;
\VS{7}ils se serrent l'un contre l'autre, et l'air n'entrerait pas entre eux ;
\VS{8}ce sont des frères qui s'embrassent, se saisissent, demeurent inséparables.
\VS{9}Ses éternuements font briller la lumière, ses yeux sont comme les paupières de l'aurore.
\VS{10}Des flammes viennent de sa bouche, des étincelles de feu s'en échappent.
\VS{11}Une fumée sort de ses narines, comme d'un chaudron qui bout, d'une chaudière ardente.
\VS{12}Son souffle allume les charbons, de sa bouche sort la flamme.
\VS{13}La force a son cou pour demeure, et l'effroi bondit devant lui.
\VS{14}Ses parties charnues sont jointes ensemble, fondues sur lui, inébranlables.
\VS{15}Son cœur est dur comme la pierre, dur comme la meule inférieure.
\VS{16}Quand il se lève, les plus vaillants ont peur, et l'épouvante les fait quitter le droit chemin.
\VS{17}C'est en vain qu'on l'attaque avec l'épée ; la lance, le javelot, la cuirasse ne servent à rien.
\VS{18}Il regarde le fer comme de la paille, l'airain comme du bois pourri.
\VS{19}La flèche de l'arc ne le met pas en fuite, les pierres de la fronde sont pour lui changés en chaume.
\VS{20}Il ne voit dans la massue qu'un brin de paille, il rit au sifflement des dards.
\VS{21}Sous son ventre sont des pointes aiguës : On dirait une herse qu'il étend sur la boue.
\VS{22}Il fait bouillir les profondeurs de la mer comme une chaudière, il la traite comme un vase rempli de parfums.
\VS{23}Il laisse après lui un sentier lumineux ; l'abîme prend la chevelure d'un vieillard.
\VS{24}Sur la terre nul n'est son maître ; il a été façonné pour ne rien craindre.
\VS{25}Il regarde avec dédain tout ce qui est élevé, il est le roi des plus fiers animaux.
\Chap{42}
\TextTitle{Job reconnaît la souveraineté de Dieu et s'humilie}
\VerseOne{}Job répondit à Yahweh et dit :
\VS{2}Je sais que tu peux tout, et qu'on ne saurait t'empêcher de faire ce que tu penses.
\VS{3}Quel est celui qui a la folie d'obscurcir mes conseils ? Oui, j'ai parlé sans les comprendre, de merveilles qui me dépassent et que je ne connais pas\FTNT{Ps. 40:6 ; Ps. 131:1 ; Ps. 139:6.}.
\VS{4}Écoute-moi maintenant, et je parlerai ; je t'interrogerai et tu m'instruiras.
\VS{5}J'avais entendu parler de toi ; mais maintenant mon œil t'a vu.
\VS{6}C'est pourquoi je me condamne et je me repens d'avoir ainsi parlé et je m'en sur la poussière et sur la cendre.
\VS{7}Après que Yahweh eut ainsi parlé à Job, il dit à Eliphaz de Théman : Ma colère est embrasée contre toi et contre tes deux amis, parce que vous n'avez pas parlé de moi avec droiture comme Job, mon serviteur.
\VS{8} C'est pourquoi prenez maintenant sept taureaux et sept béliers, allez auprès de mon serviteur Job, et offrez un holocauste pour vous. Job, mon serviteur, priera pour vous, et certainement j'exaucerai sa prière, afin que je ne vous traite pas selon votre folie ; car vous n'avez pas parlé de moi avec droiture, comme mon serviteur Job.
\VS{9} Ainsi Eliphaz de Théman, Bildad de Schuach, et Tsophar de Naama allèrent et firent comme Yahweh leur avait commandé ; et Yahweh exauça la prière de Job.
\VS{10}Yahweh rétablit Job de sa captivité, quand il eut prié pour ses amis ; et Yahweh lui ajouta le double de tout ce qu'il avait possédé.
\VS{11}Ses frères, ses sœurs, et tous ceux qui l'avaient connu auparavant vinrent tous le visiter, et ils mangèrent avec lui dans sa maison. Ils compatirent et le consolèrent au sujet de tout le mal que Yahweh avait fait venir sur lui, et chacun lui donna une kesita et un anneau d'or.
\VS{12}Pendant ses dernières années, Job reçut de Yahweh plus de bénédictions qu'il n'en avait reçu dans les premières. Il posséda quatorze mille brebis, six mille chameaux, mille paires de bœufs, et mille ânesses.
\VS{13}Il eut aussi sept fils et trois filles :
\VS{14}Il donna à la première le nom de Jemima, à la seconde celui de Ketsia, à la troisième celui de Kéren-Happuc.
\VS{15}Et il ne se trouvait pas de femmes aussi belles que les filles de Job dans tout le pays. Leur père leur donna une part de l'héritage parmi leurs frères.
\VS{16}Job vécut, après ces choses, cent quarante ans, et il vit ses fils et les fils de ses fils jusqu'à la quatrième génération.
\VS{1} Job mourut âgé et rassasié de jours.
\PPE{}
\end{multicols}

%\clearpage\ShortTitle{Cantique des cantiques}\BookTitle{Cantique des cantiques}\BFont
\begin{multicols}{2}
\TextTitle{[La fiancée et le fiancé exprime leur amour mutuel.]}
\Chap{1}
\VerseOne{}Le Cantique des cantiques, de Salomon.
\VS{2}[La Sulamithe :] Qu'il me baise des baisers de sa bouche ! [Les filles de Jérusalem :] Car tes amours sont plus agréables que le vin,
\VS{3}à cause de l'odeur de tes excellents parfums, ton nom est comme un parfum qui se répand ; c'est pourquoi les filles t'aiment.
\VS{4}[La Sulamithe :] Entraine-moi après toi ! [Les filles de Jérusalem :] Nous courrons ! [La Sulamithe :] Le roi m'introduit dans ses appartements. [Les filles de Jérusalem :] Nous nous égaierons et nous nous réjouirons à cause de toi ; nous célébrerons ton amour plus que le vin. C’est avec raison que l’on t’aime.
\VS{5}[La Sulamithe :] Ô filles de Jérusalem, je suis noire, je suis belle. Je suis comme les tentes de Kédar, et comme les pavillons de Salomon.
\VS{6}Ne prenez pas garde à moi, de ce que je suis noire : Le soleil m'a brûlée. Les fils de ma mère se sont mis en colère contre moi, ils m'ont faite gardienne des vignes. Ma vigne à moi, je ne l’ai pas gardée.
\VS{7}Dis-moi, toi que mon âme aime, où tu fais paître ton troupeau, et où tu les fais reposer à midi ; car pourquoi serais-je comme une femme errante près des troupeaux de tes compagnons ?
\VS{8}[Salomon :] Si tu ne le sais pas, ô la plus belle des femmes, sors sur les traces du troupeau, et fais paître tes chevreaux près des demeures des bergers.
\VS{9}Ma grande amie, je te compare au plus beau couple de chevaux que j'ai aux chars de Pharaon.
\VS{10}Tes joues sont belles au milieu de tes colliers, et ton cou est beau au milieu des rangées de perles.
\VS{11}[Les filles de Jérusalem :] Nous te ferons des colliers d'or, avec des boutons d'argent.
\VS{12}[La Sulamithe :] Tandis que le roi est assis à table, mon nard répand son parfum.
\VS{13}Mon bien-aimé est pour moi comme un bouquet de myrrhe, il passe la nuit entre mes seins.
\VS{14}Mon bien-aimé m'est comme une grappe de troëne dans les vignes d'En-Guédi.
\VS{15}[Salomon :] Que tu es belle, ma grande amie, que tu es belle ! Tes yeux sont comme ceux des colombes.
\VS{16}[La Sulamithe :] Que tu es beau, mon bien-aimé, que tu es aimable ! Aussi, notre lit est verdoyant.
\VS{17}Les poutres de nos maisons sont faites de cèdre, et nos chevrons de cyprès.
\TextTitle{[l'amour entre la fiancée et le fiancé]}
\Chap{2}
\VerseOne{}[La Sulamithe :] Je suis la rose de Saron, et le lis des vallées.
\VS{2}[Salomon :] Comme un lis au milieu des épines (1), telle est ma grande amie parmi les filles.
\VS{3}[La Sulamithe :] Comme un pommier au milieu des arbres de la forêt, tel est mon bien-aimé parmi les jeunes hommes. J'ai désiré m’asseoir à son ombre, son fruit est doux à mon palais.
\VS{4}Il m'a fait entrer dans la salle du festin (2) ; et la bannière qu’il déploie sur moi c’est l’amour (3).
\VS{5}Soutenez-moi avec des gâteaux de raisins, fortifiez-moi avec des pommes ; car je suis malade d'amour.
\VS{6}Que sa main gauche soit sous ma tête et que sa droite m'embrasse !
\VS{7}[Salomon :] Filles de Jérusalem, je vous en conjure, par les gazelles et par les biches des champs, ne réveillez pas celle que j'aime, ne la réveillez pas jusqu'à ce qu'elle le veuille.
\TextTitle{[La Sulamithe parle de Salomon]}
\VS{8}[La Sulamithe :] C'est la voix de mon bien-aimé ! Le voici, il vient, sautant sur les montagnes et bondissant sur les collines.
\VS{9}Mon bien-aimé est semblable à la gazelle ou aux faons des biches. Le voici qui se tient derrière notre muraille, il regarde par les fenêtres, il se fait voir par les treillis.
\VS{10}[La Sulamithe rapporte les paroles de Salomon :] Mon bien-aimé parle et me dit : Lève-toi, ma grande amie, ma belle, et viens !
\VS{11}Car voici, l'hiver est passé ; la pluie a cessé, elle s'en est allée.
\VS{12}Les fleurs paraissent sur la terre, le temps des chansons est venu, et la voix de la tourterelle se fait déjà entendre dans notre contrée.
\VS{13}Le figuier produit ses premiers fruits, et les vignes en fleurs exhalent le parfum. Lève-toi, ma grande amie, ma belle, et viens !
\VS{14}Ma colombe qui te tiens dans les fentes du rocher (4), dans les lieux secrets (5) et escarpés, fais-moi voir ta figure, fais-moi entendre ta voix ; car ta voix est douce, et ta figure est belle.
\VS{15}Prenez-nous les renards, et les petits renards qui ravagent les vignes, car nos vignes produisent des grappes.
\VS{16}[La Sulamithe :] Mon bien-aimé est à moi, et je suis à lui ; il fait paître son troupeau parmi les lis.
\VS{17}Avant que le jour se rafraîchisse et que les ombres s'enfuient, reviens mon bien-aimé ! Sois comme la gazelle ou les faons des biches, sur les montagnes qui nous séparent.
\TextTitle{[La fiancée recherche son fiancé et le trouve.]}
\Chap{3}
\VerseOne{}[La Sulamithe :] J'ai cherché pendant les nuits, sur ma couche, celui que mon âme aime ; je l'ai cherché, mais je ne l'ai point trouvé.
\VS{2}Je me lèverai maintenant, et je ferai le tour de la ville, des carrefours et des places ; je chercherai celui que mon âme aime. Je l'ai cherché, mais je ne l'ai point trouvé.
\VS{3}Les gardes qui font la ronde dans la ville m'ont rencontrée : N'avez-vous pas vu, leur ai-je dit, celui que mon âme aime ?
\VS{4}A peine les avais-je passés, que j’ai trouvé celui que mon âme aime ; je l’ai saisi et je ne l’ai point lâché jusqu’à ce que je l'aie amené dans la maison de ma mère, dans la chambre de celle qui m'a conçue.
\VS{5}[Salomon :] Filles de Jérusalem, je vous en conjure par les gazelles et par les biches des champs, ne réveillez pas celle que j'aime, ne la réveillez pas, jusqu'à ce qu'elle le veuille.
\TextTitle{[Salomon entre avec sa fiancée dans Sion]}
\VS{6}[La Sulamithe :] Qui est celle qui monte du désert, comme des colonnes de fumée en forme de palmiers, parfumée de myrrhe et d'encens, et de tous les aromates du parfumeur ?
\VS{7}[Un officier des gardes du roi Salomon :] Voici la litière de Salomon, autour de laquelle il y a soixante vaillants hommes, des plus vaillants d'Israël.
\VS{8}Tous sont armés de l'épée et sont très bien exercés à la guerre ; chacun porte son épée sur sa hanche, à cause des frayeurs de la nuit.
\VS{9}Le roi Salomon s'est fait une litière de bois du Liban.
\VS{10}Il en a fait les piliers d’argent, le dossier d’or, le siège de pourpre ; et au milieu, il a placé un tissu que les filles de Jérusalem aiment.
\VS{11}[Les filles de Jérusalem :] Sortez, filles de Sion, et regardez le roi Salomon avec la couronne dont sa mère l'a couronné le jour de ses fiançailles, le jour de la joie de son cœur.
\TextTitle{[Le fiancé déclare son amour]}
\Chap{4}
\VerseOne{}[Salomon :] Que tu es belle, ma grande amie, que tu es belle ! Tes yeux sont comme ceux des colombes derrière ton voile. Tes cheveux sont comme le poil d'un troupeau de chèvres suspendu aux flancs de la montagne de Galaad.
\VS{2}Tes dents sont comme un troupeau de brebis tondues qui remontent de l’abreuvoir ; toutes sont des jumelles, il n’y en a pas une qui manque.
\VS{3}Tes lèvres sont comme un fil teint d’écarlate, et ta bouche est charmante ; ta joue est comme une moitié de grenade, derrière ton voile.
\VS{4}Ton cou est comme la tour de David, bâtie pour être un arsenal ; mille boucliers y sont suspendus, tous les grands boucliers des héros.
\VS{5}Tes deux seins sont comme deux faons, comme les jumeaux d'une gazelle qui paissent au milieu des lis.
\VS{6}Avant que le jour se rafraîchisse et que les ombres s'enfuient, je m'en irai à la montagne de myrrhe, et à la colline de l'encens.
\VS{7}Tu es toute belle, ma grande amie, et il n'y a point de défaut (1) en toi.
\VS{8}Viens du Liban avec moi, mon épouse, viens du Liban avec moi ! Regarde du sommet de l'Amana, du sommet du Senir et de l’Hermon, des repaires des lions et des montagnes des léopards.
\VS{9}Tu me ravis le cœur, ma sœur, mon épouse, tu me ravis le cœur, par l'un de tes regards et par l'un des colliers de ton cou.
\VS{10}Que de charmes dans ton amour, ma sœur, mon épouse ! Comme ton amour vaut mieux que le vin, et combien l'odeur de tes parfums exhale plus que tous les aromates !
\VS{11}Tes lèvres, mon épouse, distillent des rayons de miel ; le miel et le lait sont sous ta langue, et l'odeur de tes vêtements est comme l'odeur du Liban.
\VS{12}Ma sœur, mon épouse, tu es un jardin fermé, une source fermée, une fontaine scellée (1).
\VS{13}Tes jets forment un jardin, où sont des grenadiers, avec les fruits les plus excellents, les troënes avec le nard.
\VS{14}Le nard et le safran, le roseau aromatique et le cinnamome, avec tous les arbres qui donnent l'encens ; la myrrhe et l'aloès, avec tous les excellents aromates.
\VS{15}Ô fontaine des jardins ! Ô source d'eau vive ! Ruisseaux coulant du Liban !
\VS{16}Lève-toi, nord ! Et viens, vent du midi ! Soufflez sur mon jardin afin que ses parfums s’en exhalent ! [La Sulamithe :] Que mon bien-aimé entre dans son jardin et qu'il mange de ses fruits délicieux.
\Chap{5}
\VerseOne{}[Salomon :] J’entre dans mon jardin, ma sœur, mon épouse ; je cueille ma myrrhe avec mes aromates, je mange mes rayons de miel ; avec mon miel, je bois mon vin avec mon lait. Mes amis, mangez, buvez, enivrez-vous d’amour, mes bien-aimés !
\TextTitle{[La Sulamithe raconte son rêve]}
\VS{2}[La Sulamithe :] J'étais endormie, mais mon cœur veillait (1), c’est la voix de mon bien-aimé qui frappe, en disant : [La Sulamithe rapporte les propos de Salomon :] Ouvre-moi, ma sœur, ma grande amie, ma colombe, ma parfaite ! Car ma tête est pleine de rosée et mes cheveux des gouttes de la nuit.
\VS{3}[La Sulamithe :] J'ai ôté ma tunique, lui dis-je, comment la remettrais-je ? J'ai lavé mes pieds, comment les salirais-je ?
\VS{4}Mon bien-aimé a passé la main par la fenêtre, et mes entrailles se sont émues pour lui.
\VS{5}Je me suis levée pour ouvrir à mon bien-aimé, et la myrrhe a ruisselé de mes mains, et la myrrhe s’est répandue de mes doigts sur la poignée du verrou.
\VS{6}J'ai ouvert à mon bien-aimé, mais mon bien-aimé s'en était allé, il avait disparu ; mon âme était hors de moi quand il me parlait. Je l’ai cherché, mais je ne l’ai point trouvé ; je l'ai appelé, mais il ne m’a point répondu.
\VS{7}Les gardes qui font la ronde dans la ville m’ont rencontrée ; ils m’ont battue, ils m’ont blessée ; les gardes des murailles m'ont enlevé mon voile.
\VS{8}Filles de Jérusalem, je vous en conjure, si vous trouvez mon bien-aimé, que lui direz-vous ? Que je suis malade d’amour.
\VS{9}[Les filles de Jérusalem :] Qu'a ton bien-aimé de plus qu’un autre, ô la plus belle des femmes ? Qu'a ton bien-aimé de plus qu’un autre pour que tu nous conjures ainsi ?
\TextTitle{[La Sulamithe décrit Salomon]}
\VS{10}[La Sulamithe :] Mon bien-aimé est blanc et vermeil, il se distingue entre dix mille.
\VS{11}Sa tête est de l’or pur, ses cheveux sont bouclés et flottants, noirs comme le corbeau.
\VS{12}Ses yeux sont comme ceux des colombes au bord des ruisseaux, lavés dans du lait, reposant au sein de la plénitude.
\VS{13}Ses joues sont comme un parterre d’aromates, comme des fleurs parfumées ; ses lèvres sont comme des lis d’où découle la myrrhe.
\VS{14}Ses mains sont comme des anneaux d'or, garnis de chrysolithes ; son ventre est comme de l’ivoire bien poli, couvert de saphirs.
\VS{15}Ses jambes sont comme des piliers de marbre, posés sur des bases d’or fin. Son aspect est comme le Liban, il est précieux comme les cèdres.
\VS{16}Son palais n'est que douceur, et toute sa personne est pleine de charme (2). Tel est mon bien-aimé, tel est mon ami, filles de Jérusalem !
\TextTitle{[Les filles de Jérusalem aident la Sulamithe à chercher Salomon]}
\Chap{6}
\VerseOne{}[Les filles de Jérusalem :] Où est allé ton bien-aimé, ô la plus belle des femmes ? De quel côté est allé ton bien-aimé ? Nous le chercherons avec toi.
\VS{2}[La Sulamithe :] Mon bien-aimé est descendu à son jardin, au parterre d’aromates, pour faire paître son troupeau dans les jardins, et pour cueillir des lis.
\VS{3}Je suis à mon bien-aimé et mon bien-aimé est à moi ; il fait paître son troupeau parmi les lis.
\TextTitle{[Salomon à la Sulamithe]}
\VS{4}[Salomon :] Tu es belle, grande amie, comme Thirtsa ; agréable comme Jérusalem, redoutable comme des troupes sous leurs bannières.
\VS{5}Détourne de moi tes yeux, car ils me troublent. Tes cheveux sont comme un troupeau de chèvres suspendus aux flancs de Galaad.
\VS{6}Tes dents sont comme un troupeau de brebis qui remontent de l’abreuvoir ; toutes portent des jumeaux, et aucune d'elles n'est stérile.
\VS{7}Ta joue est comme une moitié de grenade, derrière ton voile.
\VS{8}Il y a soixante reines, quatre-vingts concubines, et des vierges sans nombre.
\VS{9}Une seule est ma colombe, ma parfaite ; elle est l’unique de sa mère, celle qui l'a enfantée. Les filles la voient et la disent heureuse ; les reines et les concubines la louent en disant :
\VS{10}Qui est celle qui apparaît comme l’aurore, belle comme la lune, brillante comme le soleil, redoutable comme des troupes sous leurs bannières ?
\VS{11}[La Sulamithe :] Je suis descendu au jardin des noyers pour voir les fruits de la vallée qui mûrissent, et pour voir si la vigne pousse, et si les grenadiers fleurissent.
\VS{12}Je ne me suis pas aperçue que mon affection m'a rendue semblable aux chars d’Amminadib (1).
\TextTitle{[Description de la beauté de la Sulamithe.]}
\Chap{7}
\VerseOne{}[Les filles de Jérusalem :] Reviens, reviens, ô Sulamithe ! Reviens, reviens, afin que nous te contemplions. [La Sulamithe :] Qu’avez-vous à contempler la Sulamithe comme une danse de deux armées ?
\VS{2}[Les filles de Jérusalem :] Que tes pieds sont beaux dans tes chaussures, fille de prince ! Les contours de ta hanche sont comme des colliers, œuvres des mains d'un excellent artisan.
\VS{3}Ton sein est comme une coupe arrondie, pleine d’un vin aromatisé, ton ventre est comme un tas de blé entouré de lis.
\VS{4}Tes deux seins sont comme deux faons, comme les jumeaux d'une gazelle.
\VS{5}Ton cou est comme une tour d'ivoire ; tes yeux sont comme les étangs de Hesbon, près de la porte de Bath-Rabbim ; ton nez est comme la tour du Liban qui regarde vers Damas.
\VS{6}Ta tête est élevée comme le Carmel, et les cheveux fins de ta tête sont comme le pourpre ; un roi est enchaîné par tes boucles pour te contempler.
\VS{7}[Salomon :] Que tu es belle, que tu es agréable, ô mon amour, au milieu des délices !
\VS{8}Ta taille est semblable à un palmier, et tes seins à des grappes.
\VS{9}Je me dis : Je monterai sur le palmier, je prendrai possession de ses rameaux ! Que tes seins soient comme les grappes de la vigne, l’odeur de tes narines comme celle des pommes,
\VS{10}et ta bouche comme un vin excellent… [La Sulamithe :] …qui coule droitement pour mon bien-aimé, et glisse sur les lèvres de ceux qui s’endorment.
\VS{11}Je suis à mon bien-aimé et ses désirs se portent vers moi.
\VS{12}Viens, mon bien-aimé, sortons dans les champs, passons la nuit dans les villages !
\VS{13}Levons-nous dès le matin pour aller aux vignes, nous verrons si la vigne pousse, si la fleur s’ouvre, si les grenadiers fleurissent. Là je te donnerai mes amours.
\VS{14}Les mandragores répandent leur parfum, et à nos portes il y a toutes sortes de fruits exquis, des fruits nouveaux, et des fruits anciens : Mon bien-aimé, je les ai gardés pour toi.
\Chap{8}
\VerseOne{}[La Sulamithe :] Oh ! Que n’es-tu pour moi comme un frère, allaité des seins de ma mère ! Je te rencontrerais dehors, je t’embrasserais, et on ne me mépriserait pas.
\VS{2}Je te conduirais, je t’introduirais dans la maison de ma mère ; tu m’instruiras, et je te ferai boire du vin parfumé d'aromates et du moût de mon grenadier.
\VS{3}Que sa main gauche soit sous ma tête, et que sa droite m'embrasse !
\VS{4}[La Sulamithe (citant Salomon) :] Je vous en conjure, filles de Jérusalem, ne réveillez pas celle que j'aime, ne la réveillez pas, jusqu'à ce qu'elle le veuille.
\VS{5}[Les frères de la Sulamithe :] Qui est celle qui monte du désert, mollement appuyée sur son bien-aimé ? [Salomon :] Je t'ai réveillée sous le pommier ; là où ta mère t'a enfantée, là où celle qui t'a conçue t'a donné le jour.
\VS{6}Mets-moi comme un sceau sur ton cœur (1), comme un sceau sur ton bras ; car l'amour est fort comme la mort, et la jalousie est cruelle comme le scheol ; leurs ardeurs sont des ardeurs de feu, une flamme de Yahweh.
\VS{7}[La Sulamithe (à Salomon) :] Les grandes eaux ne peuvent éteindre l’amour, même les fleuves ne pourraient le submerger ; quand un homme donnerait toutes les richesses de sa maison contre l’amour, il ne s’attirerait qu’un profond mépris.
\VS{8}[La Sulamithe (racontant ce que ses frères lui ont dit) :] Nous avons une petite sœur qui n'a pas encore de seins ; que ferons-nous à notre sœur le jour où on parlera d’elle ?
\VS{9}Si elle est comme une muraille, nous bâtirons sur elle un palais d'argent ; si elle est une porte, nous la renforcerons avec une planche de cèdre.
\VS{10}[La Sulamithe :] Je suis comme une muraille, et mes seins sont comme des tours ; j'ai été à ses yeux comme celle qui trouve la paix.
\VS{11}Salomon avait une vigne à Baal-Hamon ; il remit la vigne à des gardiens ; chacun apportait pour son fruit mille pièces d'argent.
\VS{12}Ma vigne, qui est à moi, je la garde. Ô Salomon, que les mille pièces d'argent soient à toi, et deux cents pour les gardiens du fruit de la vigne !
\VS{13}[Les frères de la Sulamithe :] Ô toi qui habites dans les jardins ! Les amis sont attentifs à ta voix. [Salomon :] Daigne me la faire entendre !
\VS{14}[La Sulamithe :] Fuis, mon bien-aimé ! Sois semblable à la gazelle ou au faon des biches, sur les montagnes des aromates !
\PPE{}
\end{multicols}

\clearpage\ShortTitle{Ruth}\BookTitle{Ruth}\BFont
\noindent\hrulefill
{\footnotesize
\textit{
\bigskip
{\centering{}
\\(Routh)
\\Signifie : Amitié, une amie
\\Thème : Les origines de la famille messianique
\\Auteur : Inconnu
\\Date de rédaction : 11ème siècle av. J.-C.\\}
}
%\bigskip
\textit{
\\Au temps des juges, tout le pays fut frappé par une famine qui poussa Elimélec, sa femme Naomi, et ses deux fils à s’installer dans le pays de Moab. Ce pays tire son nom de son fondateur Moab, né de l’inceste entre Lot et sa fille ainée.
%\bigskip
\\Ils y rencontrèrent Ruth qui devint ensuite la belle fille d’Elimélec. Après la mort de son époux, cette moabite démontra  son attachement non seulement à cette famille mais également au Dieu de cette famille qui devint aussi le sien.
%\bigskip
\\Au prix de sa détermination, son obéissance et son humilité, la destinée de cette femme fut complètement bouleversée. Image du rachat des nations, elle entra dans la lignée de Jésus-Christ homme.\bigskip
}
}
\par\nobreak\noindent\hrulefill
\begin{multicols}{2}
\Chap{1}
\TextTitle{[Famine en Juda]}
\VerseOne{}Au temps où les juges gouvernaient, il y eut une famine dans le pays. Un homme de Bethléhem de Juda s'en alla, avec sa femme et ses deux fils, pour séjourner sur la terre de Moab.
\TextTitle{[Séjour en Moab]}
\VS{2}Le nom de cet homme était Elimélec, celui de sa femme Naomi, et les noms de ses deux fils Machlon et Kiljon ; ils étaient Ephratiens, de Bethléhem de Juda. Entrés sur la terre de Moab, ils s’y établirent.
\VS{3}Elimélec, mari de Naomi, mourut, et elle resta avec ses deux fils.
\VS{4}Ils prirent pour eux des femmes Moabites, dont l'une se nommait Orpa, et la seconde Ruth\FTNT{Ruth, la Moabite, dont l’ancêtre était issu d’une relation incestueuse (Ge. 19:36-37), est devenue l’ancêtre du Messie (Mt. 1:5-6).}, et ils demeurèrent là environ dix ans.
\VS{5}Machlon et Kiljon moururent aussi tous les deux, et cette femme resta privée de ses deux fils et de son mari.
\TextTitle{[Retour en Juda]}
\VS{6}Puis elle se leva avec ses belles-filles afin de retourner de la terre de Moab, car elle entendit, au pays de Moab, que Yahweh avait visité son peuple en lui donnant du pain.
\VS{7}Elle sortit du lieu où elle habitait, avec ses deux belles-filles, et elle marcha pour revenir sur la terre de Juda.
\VS{8}Naomi dit à ses deux belles-filles : Allez, retournez chacune dans la maison de sa mère ! Que Yahweh use de bonté envers vous, comme vous l'avez fait envers ceux qui sont morts et envers moi !
\VS{9}Que Yahweh vous donne de trouver chacune du repos dans la maison d'un mari ! Et elle les embrassa. Elles levèrent leur voix, et pleurèrent ;
\VS{10}et elles lui dirent : Non, nous retournerons avec toi vers ton peuple.
\TextTitle{[Décision loyale de Ruth]}
\VS{11}Naomi dit : Retournez, mes filles ! Pourquoi viendriez-vous avec moi ? Ai-je encore dans mon sein des fils qui puissent devenir vos maris ?
\VS{12}Retournez, mes filles, allez ! Je suis trop vieille pour me remarier. Et quand je dirais : J'ai de l'espérance, quand cette nuit même je serais avec un mari, et que j'enfanterais des fils,
\VS{13}Attendriez-vous donc qu'ils aient grandi, refuseriez-vous donc des maris ? Non, mes filles ! Je suis dans une plus grande amertume que vous, car la main de Yahweh s'est éloignée de moi.
\VS{14}Et elles levèrent leur voix, et pleurèrent encore. Orpa embrassa sa belle-mère, mais Ruth s’attacha à elle.
\VS{15}Naomi dit à Ruth : Voici, ta belle-sœur est retournée vers son peuple et vers ses dieux ; retourne, après ta belle-sœur.
\VS{16}Ruth répondit : Ne me prie pas de te laisser, de me retourner et de ne pas te suivre ! Où tu iras, j'irai, où tu demeureras je demeurerai ; ton peuple sera mon peuple, et ton Dieu sera mon Dieu ;
\VS{17}Où tu mourras je mourrai, et j'y serai enterrée. Que Yahweh me traite dans toute sa rigueur, si autre chose que la mort vient à me séparer de toi !
\VS{18}Naomi, la voyant déterminée à aller avec elle, arrêta de lui parler.
\TextTitle{[Arrivée à Bethléhem]}
\VS{19}Elles marchèrent toutes deux jusqu'à ce qu'elles entrent à Bethléhem. Lorsqu'elles entrèrent dans Bethléhem, toute la ville fut agitée à cause d'elles, et les femmes disaient : Est-ce là Naomi ?
\VS{20}Elle leur dit : Ne m'appelez pas Naomi ; appelez-moi Mara, car le Tout-Puissant m'a remplie de beaucoup d'amertume.
\VS{21}J’étais dans l'abondance à mon départ, et Yahweh me ramène à vide. Pourquoi m'appelleriez-vous Naomi, après que Yahweh s'est prononcé contre moi, et que le Tout-Puissant m'a affligée ?
\VS{22}Ainsi revinrent de la terre de Moab Naomi et sa belle-fille, Ruth, la Moabite. Elles entrèrent dans Bethléhem au commencement de la moisson des orges.
\Chap{2}
\TextTitle{[Boaz félicite Ruth des soins désintéressés dont elle entoure Naomi]}
\VerseOne{}Naomi avait un parent de son mari. C'était un homme puissant et riche, de la famille d'Elimélec, et qui s’appelait Boaz.
\VS{2}Ruth la Moabite dit à Naomi : Je te prie laisse-moi aller glaner des épis dans le champ de celui aux yeux duquel je trouverai grâce. Elle lui dit : Va, ma fille.
\VS{3}Elle s'en alla et entra dans un champ, pour glaner après les moissonneurs. Et elle arriva par hasard sur une parcelle de champ qui appartenait à Boaz, qui était de la famille d'Elimélec.
\VS{4}Or voici, Boaz vint de Bethléhem, et il dit aux moissonneurs : Que Yahweh soit avec vous ! Ils lui dirent : Que Yahweh te bénisse !
\VS{5}Et Boaz dit à son serviteur qui était établi sur les moissonneurs : A qui est cette jeune fille ?
\VS{6}Le serviteur qui était établi sur les moissonneurs répondit et dit : C'est une jeune femme Moabite, qui est revenue avec Naomi de la terre de Moab.
\VS{7}Elle nous a dit : Permettez-moi de glaner et de recueillir des épis entre les gerbes, après les moissonneurs. Depuis ce matin qu'elle est venue, elle est restée jusqu'à présent, et s'est à peine assise dans la maison.
\VS{8}Boaz dit à Ruth : Ecoute, ma fille, ne va pas glaner dans un autre champ ; ne pars pas au loin, et reste avec mes servantes.
\VS{9}Regarde où l'on moissonne dans le champ, et va après elles. J'ai défendu à mes serviteurs de te toucher. Et si tu as soif, va prendre des vases, et bois de ce que les serviteurs auront puisé.
\VS{10}Alors elle tomba sur sa face, et se prosterna contre terre, et elle lui dit : Comment ai-je trouvé grâce à tes yeux, pour que tu prêtes attention à moi, moi qui suis une étrangère ?
\VS{11}Boaz lui répondit et dit : On m'a raconté tout ce que tu as fait pour ta belle-mère depuis que ton mari est mort, comment tu as laissé ton père, ta mère, et le pays de ta naissance, pour aller vers un peuple que tu ne connaissais pas auparavant.
\VS{12}Que Yahweh te récompense pour ton œuvre, et que ton salaire soit entier de la part de Yahweh le Dieu d'Israël, sous les ailes duquel tu es venue te réfugier !
\VS{13}Et elle dit : Mon seigneur, que je trouve grâce à tes yeux ! Car tu m'as consolée, et tu as parlé au cœur de ta servante. Et pourtant je ne suis pas, moi, comme l'une de tes servantes.
\VS{14}Au moment du repas, Boaz dit à Ruth : Approche-toi ici, mange du pain, et trempe ton morceau dans le vinaigre. Elle s'assit à côté des moissonneurs. On lui donna du grain rôti ; elle mangea et se rassasia, et elle garda le reste.
\VS{15}Puis elle se leva pour glaner. Boaz ordonna à ses serviteurs : Qu'elle glane même entre les gerbes, et ne lui faites pas honte.
\VS{16}Et vous retirerez même pour elle quelques poignées de gerbes, que vous lui laisserez glaner, sans la réprimander.
\VS{17}Elle glana donc dans le champ jusqu'au soir, et elle battit ce qu'elle avait glané. Il y eut environ un épha d'orge.
\VS{18}Elle l'emporta, entra dans la ville, et sa belle-mère vit ce qu'elle avait glané. Elle sortit aussi les restes de son repas, et les lui donna.
\VS{19}Sa belle-mère lui dit : Où as-tu glané aujourd'hui, et où as-tu travaillé ? Béni soit celui qui t'a reconnue ! Et Ruth raconta à sa belle-mère chez qui elle avait travaillé : L'homme chez qui j'ai travaillé aujourd'hui s’appelle Boaz.
\VS{20}Naomi dit à sa belle-fille : Qu'il soit béni de Yahweh, puisqu'il a la même bonté pour les vivants, comme il en eut pour ceux qui sont morts ! Cet homme est un proche parent, lui dit encore Naomi, il est un de ceux qui ont sur nous le droit de rachat\FTNT{Le droit de rachat : Le rédempteur est celui qui rachète une personne moyennant le paiement d'une rançon. Sous la première alliance, le rachat se faisait soit par un frère, soit par un proche parent, pour la libération de celui qui s’était fait esclave ou qui avait aliéné sa propriété ou son bien (Lé. 25:25 et 48). Sous la nouvelle alliance, Jésus-Christ est désormais notre rédempteur. Romains 3:23-24 nous dit : «~ Car tous ont péché, et sont entièrement privés de la gloire de Dieu. Et ils sont gratuitement justifiés par sa grâce, par la rédemption qui est en Jésus-Christ «~». Christ nous a rachetés de la malédiction de la loi en se donnant lui-même pour nous afin de nous délivrer de toute iniquité (Ga. 3:13 ; Ti. 2:14). Dieu s’est fait homme (Hé. 2:14-17) afin de mieux nous libérer de l'esclavage du diable par sa mort à la croix de Golgotha (Es. 60:16).}.
\VS{21}Ruth la Moabite dit : Il m'a même dit : Reste avec mes serviteurs jusqu'à ce qu'ils aient achevé toute ma moisson.
\VS{22}Et Naomi dit à Ruth, sa belle-fille : Ma fille, il est bon que tu sortes avec ses servantes, et qu'on ne te rencontre pas dans un autre champ.
\VS{23}Elle resta donc avec les servantes de Boaz, pour glaner, jusqu'à la fin de la moisson des orges et la moisson des froments. Et elle demeurait avec sa belle-mère.
\Chap{3}
\TextTitle{[Ruth dans l'obéissance de la foi]}
\VerseOne{}Naomi, sa belle-mère, lui dit : Ma fille, je voudrais chercher ton repos, afin que tu sois heureuse.
\VS{2}Maintenant Boaz, avec les servantes duquel tu as été, n'est-il pas de notre parenté ? Voici, il doit vanner cette nuit les orges qui ont été foulées dans l'aire.
\VS{3}Lave-toi et oins-toi, puis mets tes habits, et descends dans l'aire. Ne te fais pas connaître à lui, jusqu'à ce qu'il ait achevé de manger et de boire.
\VS{4}Quand il se couchera, découvre le lieu où il se couche. Ensuite, entre, découvre ses pieds, et couche-toi. Il te dira ce que tu as à faire.
\VS{5}Elle lui répondit : Je ferai tout ce que tu as dit.
\VS{6}Elle descendit à l'aire, et fit tout ce que sa belle-mère lui avait ordonné.
\VS{7}Boaz mangea et but, et son cœur était joyeux. Il vint se coucher à l'extrémité d'un tas de gerbes. Ruth vint secrètement, découvrit ses pieds, et se coucha.
\VS{8}Au milieu de la nuit, cet homme eut peur ; il se retourna et retira ses pieds, car voici, une femme était couchée à ses pieds.
\VS{9}Il dit : Qui es-tu ? Elle répondit : Je suis Ruth, ta servante ; étends le pan de ta robe sur ta servante, car tu as droit de rachat.
\VS{10}Et il dit : Ma fille, que Yahweh te bénisse ! Ce dernier trait de bonté me réjouit plus que le premier, car tu n'es pas allée après des jeunes gens, pauvres ou riches.
\VS{11}Maintenant, ma fille, ne crains pas ; je te ferai tout ce que tu me diras ; car toute la porte de mon peuple sait que tu es une femme vertueuse.
\VS{12}Il est bien vrai que j'ai droit de rachat, mais il existe un autre plus proche que moi, qui a le droit de rachat.
\VS{13}Passe ici la nuit, et demain, si cet homme veut user envers toi du droit de rachat, à la bonne heure, qu'il te rachète ; mais s'il ne lui plaît pas de te racheter, moi je te rachèterai, Yahweh est vivant ! Couche-toi jusqu'au matin.
\VS{14}Elle se coucha à ses pieds jusqu'au matin, et elle se leva avant qu'on puisse se reconnaître l'un l'autre. Boaz dit : Qu'on ne sache pas qu'une femme est entrée dans l'aire.
\VS{15}Et il dit : Donne-moi le manteau qui est sur toi, et tiens-le. Elle le tint, et il mesura six mesures d'orge, qu'il posa sur elle. Puis il entra dans la ville.
\VS{16}Ruth revint auprès de sa belle-mère, et Naomi dit : Est-ce toi ma fille ? Ruth lui raconta tout ce que cet homme avait fait pour elle.
\VS{17}Elle dit : Il m'a donné ces six mesures d'orge, en disant : Tu n'iras pas à vide vers ta belle-mère.
\VS{18}Et Naomi dit : Ma fille, assieds-toi ici jusqu'à ce que tu saches ce que l'affaire deviendra, car cet homme ne se donnera pas de repos, qu'il n'ait achevé cette affaire aujourd'hui.
\Chap{4}
\TextTitle{[Ruth comblée par le mariage]}
\VerseOne{}Boaz monta à la porte, et s'y assit. Or voici, celui qui avait le droit de rachat, et dont Boaz avait parlé, passa. Boaz lui dit : Ah ! Détourne-toi, reste ici, toi un tel. Et il se détourna, et s'assit.
\VS{2}Boaz prit dix hommes d'entre les anciens de la ville, et leur dit : Asseyez-vous ici. Et ils s'assirent.
\VS{3}Puis il dit à celui qui avait le droit de rachat : Naomi qui est revenue de la terre de Moab, a vendu la parcelle du champ qui appartenait à notre frère Elimélec.
\VS{4}J'ai parlé à tes oreilles afin de te le faire savoir et te le dire : Acquiers-la en la présence de ceux qui sont assis ici et en présence des anciens de mon peuple. Si tu veux racheter par droit de rachat, rachète-la ; mais si tu ne veux pas la racheter, déclare-le-moi, afin que je le sache. Car il n'y a pas d'autre que toi qui ait le droit de rachat, et je l'ai après toi. Et il dit : je rachèterai.
\VS{5}Boaz dit : Le jour où tu acquerras le champ de la main de Naomi, tu l'acquerras aussi de Ruth la Moabite, femme du défunt, pour maintenir le nom du défunt dans son héritage.
\VS{6}Et celui qui avait le droit de rachat dit : Je ne puis pas racheter pour mon compte, de peur de détruire mon héritage ; prends pour toi le droit de rachat, car je ne puis pas le racheter.
\VS{7}Autrefois en Israël, pour confirmer une affaire quelconque relative à un rachat ou à un échange, l'homme ôtait son soulier et le donnait à son parent : C'était là, en Israël, un témoignage qu'on cédait son droit.
\VS{8}Celui qui avait le droit de rachat dit à Boaz : Acquiers-le pour toi ! Et il ôta son soulier.
\VS{9}Alors Boaz dit aux anciens et à tout le peuple : Vous êtes aujourd'hui témoins que j'ai acquis de la main de Naomi tout ce qui appartenait à Elimélec, à Kiljon et à Machlon.
\VS{10}Et que je me suis également acquis pour femme Ruth la Moabite, femme de Machlon, pour maintenir le nom du défunt dans son héritage, et afin que le nom du défunt ne soit pas retranché d'entre ses frères et de la porte de sa ville. Vous en êtes témoins aujourd'hui !
\VS{11}Tout le peuple qui était à la porte et les anciens dirent : Nous en sommes témoins ! Que Yahweh rende la femme qui entre dans ta maison semblable à Rachel et à Léa, qui ont bâti toutes deux, la maison d'Israël ! Montre ta puissance dans Ephrata et proclame ton nom dans Bethléhem !
\VS{12}Puisse la postérité que Yahweh te donnera de cette jeune femme, rendre ta maison semblable à la maison de Pérets, que Tamar enfanta à Juda !
\VS{13}Boaz prit Ruth, qui devint sa femme, et il alla vers elle. Yahweh lui fit la grâce de concevoir, et elle enfanta un fils.
\VS{14}Les femmes dirent à Naomi : Béni soit Yahweh qui ne t'a pas laissé manquer aujourd'hui d'un homme, ayant droit de rachat, et dont le nom sera proclamé en Israël !
\VS{15}Cet enfant restaurera ton âme, et sera le soutien de ta vieillesse ; car ta belle-fille, qui t'aime, l'a enfanté, et elle vaut mieux que sept fils.
\VS{16}Naomi prit l'enfant et le posa sur son sein, et elle fut sa nourrice.
\TextTitle{[Le fils de Ruth sera le grand-père de David]}
\VS{17}Les voisines lui donnèrent un nom, en disant : Un fils est né à Naomi ! Et elles l'appelèrent du nom de Obed. Ce fut le père d'Isaï, père de David.
\VS{18}Voici la généalogie de Pérets. Pérets engendra Hetsron ;
\VS{19}Hetsron engendra Ram ; Ram engendra Amminadab ;
\VS{20}Amminadab engendra Nachschon ; Nachschon engendra Salmon ;
\VS{21}Salmon engendra Boaz ; Boaz engendra Obed ;
\VS{22}Obed engendra Isaï, et Isaï engendra David.
\PPE{}
\end{multicols}

%\clearpage\ShortTitle{Lamentations de Jérémie}\BookTitle{Lamentations de Jérémie}\BFont
\noindent\hrulefill
{\footnotesize
\textit{
\bigskip
{\centering{}
\\Auteur : Jérémie
\\(Heb. : Eikha)
\\Signification : Où ?
\\Thème : Affliction pour Jérusalem
\\Date de rédaction : 6\up{ème} siècle av. J.-C\\}
}
%\bigskip
\textit{
\\Recueil de pièces poétiques, les lamentations de Jérémie furent composées selon un procédé visant à accentuer le caractère funèbre, de façon à ce qu'elles soient récitées avec gémissements. Ses complaintes exposent la profonde désolation du prophète face au fardeau du peuple qu'il portait dans ses entrailles tout comme la douleur et la tristesse de Yahweh face à Israël.
%\bigskip
\\Très différentes des prophéties retrouvées dans le livre de Jérémie, les Lamentations reflètent l'affliction convenant à la
gravité du châtiment subi : famine, pillage et ruine du temple, déportation, cessation du culte, diverses calamités… Jérémie rappelle ainsi les conséquences de l'endurcissement du cœur face aux appels à la repentance ; il présente aussi les bontés éternelles de Yahweh.\bigskip
}
}
\par\nobreak\noindent\hrulefill
\begin{multicols}{2}
\Chap{1}
\TextTitle{Pleurs et désolation de Jérusalem}
\VerseOne{}[Aleph.] Comment est-il arrivé que la ville si peuplée se trouve si solitaire ? Que celle qui était grande entre les nations est devenue comme une veuve ? Que celle qui était noble dame entre les provinces a été rendue tributaire ?
\VS{2}[Beth.] Elle ne cesse de pleurer pendant la nuit, et ses larmes sont sur ses joues ; il n'y a pas un de tous ses amis qui la console ; ses intimes amis ont agi perfidement contre elle, ils sont devenus ses ennemis.
\VS{3}[Guimel.] Juda a été emmenée captive tant elle est affligée, et tant est grande sa servitude ; elle demeure maintenant entre les nations, et ne trouve point de repos ; tous ses persécuteurs l'ont attrapée dans sa détresse\FTNT{Jé. 52:26.}.
\VS{4}[Daleth.] Les chemins de Sion mènent deuil de ce qu'il n'y a plus personne qui vienne aux fêtes solennelles ; toutes ses portes sont désolées, ses sacrificateurs sanglotent, ses vierges sont accablées de tristesse ; elle est remplie d'amertume. 
\VS{5}[He.] Ses adversaires sont établis pour chefs, ses ennemis prospèrent ; car Yahweh l'a humiliée à cause de la multitude de ses transgressions ; ses petits enfants ont marché captifs devant l'adversaire\FTNT{Jé. 30:14.}.
\VS{6}[Vav.] Et tout l'honneur de la fille de Sion s'est retiré d'elle ; ses chefs sont devenus semblables à des cerfs qui ne trouvent pas de pâture, et qui fuient sans force devant celui qui les poursuit.
\VS{7}[Zayin.] Jérusalem dans les jours de son affliction et de son pauvre état s'est souvenue de toutes ses choses précieuses qu'elle avait depuis si longtemps, lorsque son peuple est tombé par la main de l'ennemi, sans aucun secours ; les ennemis l'ont vue, et se sont moqués de ses sabbats.
\VS{8}[Heth.] Jérusalem a grièvement péché ; c'est pourquoi elle est devenue un objet de dégoût ; tous ceux qui l'honoraient l'ont méprisée parce qu'ils ont vu son ignominie ; elle en a aussi sangloté, et s'est retournée en arrière.
\VS{9}[Teth.] Sa souillure était dans les pans de sa robe, et elle ne s'est pas souvenue de sa dernière fin ; elle a été extraordinairement abaissée, et elle n'a pas de consolateur. Vois ma misère, ô Yahweh ! Car l'ennemi s'est élevé avec orgueil !
\VS{10}[Yod.] L'ennemi a étendu sa main sur toutes ses choses désirables ; car elle a vu entrer dans son sanctuaire les nations au sujet desquelles tu avais donné cet ordre : Elles n'entreront point dans ton assemblée\FTNT{De. 23:3.}.
\VS{11}[Kaf.] Tout son peuple gémit, cherchant du pain\FTNT{Jé. 52:6.} ; ils ont donné leurs choses désirables pour des aliments, afin ranimer leur vie. Vois, ô Yahweh ! Regarde combien je suis méprisée.
\VS{12}[Lamed.] Cela ne vous touche-t-il point ? Vous tous passants, contemplez, et voyez s'il est une douleur comme ma douleur, celle dont j'ai été frappée ! Moi que Yahweh a accablée de douleur au jour de l'ardeur de sa colère.
\VS{13}[Mem.] Il a envoyé d'en haut, dans mes os, un feu qui les domine ; il a tendu un filet sous mes pieds, et m'a fait revenir en arrière ; il m'a mise dans la désolation, dans une langueur de tous les jours.
\VS{14}[Nun.] Le joug de mes iniquités est lié par sa main ; elles sont entrelacées, et appliquées sur mon cou ; il a renversé ma force ; le Seigneur m'a livrée entre les mains de ceux contre qui je ne pourrai pas me lever.
\VS{15}[Samech.] Le Seigneur a abattu tous les hommes forts que j'avais au milieu de moi ; il a appelé contre moi, au temps fixé, une armée pour détruire mes jeunes hommes ; le Seigneur a foulé au pressoir la vierge, fille de Juda.
\VS{16}[Ayin.] À cause de ces choses, je pleure, mes yeux fondent en larmes ; car le consolateur qui restaurait ma vie est loin de moi. Mes fils sont dans la désolation parce que l'ennemi a été plus fort.
\VS{17}[Pe.] Sion a étendu les mains, et personne ne l'a consolée ; Yahweh a ordonné aux ennemis de Jacob de l'entourer de toutes parts. Jérusalem a été comme une impureté au milieu d'eux.
\VS{18}[Tsade.] Yahweh est juste car j'ai été rebelle à ses ordres. Ecoutez, vous tous, peuples, et voyez ma douleur ! Mes vierges et mes jeunes hommes sont allés en captivité.
\VS{19}[Qof.] J'ai appelé mes amis, mais ils m'ont trompé. Mes sacrificateurs et mes anciens sont morts dans la ville : Ils cherchaient de la nourriture afin de restaurer leur vie.
\VS{20}[Resh.] Regarde Yahweh ! car je suis dans la détresse ; mes entrailles bouillonnent, mon coeur palpite au dedans de moi, parce que je n'ai fait qu'être rebelle ; au dehors l’épée m’a privée d’enfants ; au dedans il y a comme la mort. 
\VS{21}[Shin.] On m'a entendu sangloter et je n'ai personne qui me console ; tous mes ennemis ont appris mon malheur, et s’en sont réjouis, parce que tu l’as fait ; tu amèneras le jour que tu as assigné, et ils seront dans mon état.
\VS{22}[Tav.]Que toute leur méchanceté vienne devant toi, et traite-leur comme tu m'as traitée à cause de tous mes péchés ; car mes sanglots sont en grand nombre et mon coeur est languissant. 
\Chap{2}
\TextTitle{Le jour de la colère de Yahweh}
\VerseOne{}[Aleph.] Comment est-il arrivé que le Seigneur a couvert de sa colère la fille de Sion tout à l'entour, comme d'une nuée, et qu'il a précipité du ciel sur la terre la beauté d'Israël, et ne s'est pas souvenu du marchepied de ses pieds\FTNT{Ez. 43:7.} au jour de sa colère ?
\VS{2}[Beth.] Le Seigneur a englouti sans épargner toutes les habitations de Jacob ; il a dans sa fureur renversé les forteresses de la fille de Juda, il les a jetées par terre ; il a profané le royaume et ses chefs.
\VS{3}[Guimel.] Il a retranché toute la force d'Israël par l'ardeur de sa colère ; il a retiré sa droite en arrière devant l'ennemi ; il s'est allumé dans Jacob comme un feu flamboyant qui le consume de toutes parts.
\VS{4}[Daleth.] Il a tendu son arc comme un ennemi ; sa droite s'est dressée comme celle d'un adversaire ; il a tué tout ce qui était agréable à l'œil dans la tente de la fille de Sion ; il a répandu sa fureur comme un feu.
\VS{5}[He.] Le Seigneur a été comme un ennemi ; il a englouti Israël, il a englouti tous ses palais, il a détruit toutes ses forteresses ; il a multiplié chez la fille de Juda le deuil et les afflictions.
\VS{6}[Vav.] Il a mis en pièces avec violence sa tente comme un jardin ; il a détruit le lieu de son assemblée ; Yahweh a fait oublier dans Sion la fête solennelle et le sabbat, et dans sa violente colère, il a rejeté le roi et le sacrificateur.
\VS{7}[Zayin.] Le Seigneur a rejeté au loin son autel, il a dédaigné son sanctuaire ; il a livré entre les mains de l'ennemi les murailles de ses palais ; ils ont poussé des cris dans la maison de Yahweh, comme aux jours des fêtes solennelles.
\VS{8}[Heth.] Yahweh avait projeté de détruire les murailles de la fille de Sion ; il a étendu le cordeau, il n'a pas fait revenir sa main sans les avoir engloutis ; il a plongé dans le deuil remparts et murailles, ils ont été ruinés tous ensemble.
\VS{9}[Teth.] Ses portes sont enfoncées dans la terre ; il en a détruit et brisé les barres. Son roi et ses chefs sont parmi les nations ; la loi n'est plus. Même les prophètes ne reçoivent plus aucune vision de Yahweh\FTNT{Ez. 7:26.}.
\VS{10}[Yod.] Les anciens de la fille de Sion sont assis à terre, ils sont muets ; ils ont couvert leur tête de poussière, ils se sont ceints de sacs ; les vierges de Jérusalem baissent leurs têtes vers la terre.
\VS{11}[Kaf.] Mes yeux se consument à force de larmes, mes entrailles bouillonnent, ma bile se répand sur la terre. À cause des ruines de la fille de mon peuple, des enfants et des nourrissons qui tombent en défaillance dans les rues de la ville.
\VS{12}[Lamed.] Ils disaient à leurs mères : Où y a-t-il du blé et du vin ? Et ils tombaient comme morts dans les rues de la ville, comme un homme blessé à mort, ils rendaient l'âme sur le sein de leurs mères.
\VS{13}[Mem] Qui dois-je prendre à témoin ? À qui te comparer, fille de Jérusalem ? Qui pourrait t'égaler, et quelle consolation te donner, vierge, fille de Sion ? Car ta ruine est grande comme une mer : Qui pourrait te guérir\FTNT{Es. 51:19-20.} ?
\VS{14}[Nun.] Tes prophètes ont eu pour toi des visions vaines et insensées ; ils n'ont pas découvert ton iniquité, afin de détourner ta captivité ; ils t'ont prophétisé des oracles mensongers et trompeurs\FTNT{Jé. 2:8 ; Jé. 5:31 ; Jé. 14:14.}.
\VS{15}[Samech.] Tous les passants applaudissent sur toi, ils sifflent, ils secouent leur tête contre la fille de Jérusalem : Est-ce ici la ville de laquelle on disait : La parfaite en beauté, la joie de toute la terre\FTNT{Na. 3:19.} ?
\VS{16}[Pe.] Tous tes ennemis ouvrent la bouche contre toi, ils sifflent, ils grincent des dents, ils disent : Nous l'avons engloutie ! C'est ici le jour que nous attendions, nous l'avons atteint, nous le voyons !
\VS{17}[Ayin.] Yahweh a fait ce qu'il avait projeté, il a accompli sa parole qu'il avait ordonnée depuis longtemps, il a détruit sans épargner, il a fait de toi la joie de l'ennemi, il a donné de la force à tes adversaires.
\VS{18}[Tsade.] Leur cœur crie au Seigneur… Muraille de la fille de Sion, fais couler des larmes jour et nuit, comme un torrent\FTNT{Jé. 14:17.} ! Ne te donne pas de repos ; et que la prunelle de tes yeux ne se repose pas !
\VS{19}[Qof.] Lève-toi, pousse des cris dès le commencement des veilles de la nuit ! Répands ton cœur comme de l'eau en présence du Seigneur ! Lève tes mains vers lui pour l'âme de tes enfants qui meurent de faim aux coins de toutes les rues !
\VS{20}[Resh.] Vois, ô Yahweh ! Regarde qui tu as traité avec sévérité ! Les femmes n'ont-elles pas mangé leur fruit : leurs petits enfants objets de leur tendresse ? Le sacrificateur et le prophète n'ont-ils pas été tués dans le sanctuaire du Seigneur\FTNT{Lé. 26:29 ; De. 28:53 ; Jé. 19:9.} ?
\VS{21}[Shin.] Les jeunes gens et les vieillards sont couchés par terre dans les rues ; mes vierges et mes jeunes hommes sont tombés par l'épée ; tu as tué au jour de ta colère, tu as massacré sans épargner.
\VS{22}[Tav.] Tu as convié comme pour un jour solennel mes frayeurs de toutes parts. Au jour de la colère de Yahweh, il n'y a eu ni réchappé ni survivant. Ceux que j'avais langés et élevés, mon ennemi les a consumés.
\Chap{3}
\TextTitle{Jérémie partage l'affliction des siens}
\VerseOne{}[Aleph.] Je suis l'homme qui a vu l'affliction par la verge de sa fureur\FTNT{Jé. 15:15-18.}.
\VS{2}Il m'a conduit, mené dans les ténèbres, et non dans la lumière.
\VS{3}Certes c'est contre moi qu'il a tout le jour tourné et retourné sa main.
\VS{4}[Beth.] Il a fait vieillir ma chair et ma peau, il a brisé mes os\FTNT{Es. 38:13.}.
\VS{5}Il a bâti autour de moi, il m'a environné de venin et de peine.
\VS{6}Il me fait habiter dans les lieux ténèbreux, comme ceux qui sont morts depuis longtemps.
\VS{7}[Guimel.] Il a fait une cloison autour de moi, afin que je ne sorte point ; il a appesanti mes chaînes.
\VS{8}Même quand je crie et que j'élève ma voix, il rejette ma prière.
\VS{9}Il a fait un mur de pierres de taille pour fermer mes chemins, il a renversé mes sentiers.
\VS{10}[Daleth.] Il a été pour moi un ours en embuscade, un lion qui se tient dans un lieu caché\FTNT{Os. 13:8.}.
\VS{11}Il a détourné mes chemins, il m'a mis en pièces, il m'a mis dans la désolation.
\VS{12}Il a tendu son arc, et il m'a placé comme une cible pour sa flèche.
\VS{13}[He.] Il a fait entrer dans mes reins les flèches de son carquois.
\VS{14}Je suis la risée pour tout mon peuple, et leur chanson\FTNT{Ps. 69:13 ; Job. 30:9.} tout le jour.
\VS{15}Il m'a rassasié d'amertume, il m'a enivré d'absinthe.
\VS{16}[Vav.] Il a brisé mes dents avec du gravier, il m'a couvert de cendres.
\VS{17}Tellement que la paix s'est éloignée de mon âme, j'ai oublié ce que c'est que d'être à son aise.
\VS{18}Et j'ai dit : Ma force est perdue, et mon espérance aussi que j'avais en Yahweh.
\VS{19}[Zayin.] Souviens-toi de mon affliction, et de mon pauvre état qui n'est qu'absinthe et que fiel ;
\VS{20}Mon âme s'en souvient sans cesse, et elle est abattue au-dedans de moi.
\VS{21}Mais je rappellerai ceci en mon coeur, et c'est pourquoi j'aurai de l'espérance :
\VS{22}[Heth.] C'est une grâce de Yahweh que nous n'avons point été consumés parce que ses compassions ne sont pas épuisées\FTNT{Ps. 103:10.} ;
\VS{23}elles se renouvellent chaque matin. C'est une chose grande que ta fidélité !
\VS{24}Yahweh est ma portion, dit mon âme ; c'est pourquoi j'aurai espérance en lui\FTNT{Ps. 16:5.}.
\VS{25}[Teth.] Yahweh est bon pour ceux qui s'attendent à lui, pour l'âme qui le cherche.
\VS{26}Il est bon d'espérer et d'attendre en silence la délivrance de Yahweh.
\VS{27}Il est bon pour l'homme de porter le joug dans sa jeunesse.
\VS{28}[Yod.] Il sera assis solitaire et silencieux parce qu'on le lui impose.
\VS{29}Il mettra sa bouche dans la poussière, peut-être y aura-t-il quelque espérance ?
\VS{30}Il présentera la joue à celui qui le frappe, il se rassasiera d'opprobres.
\VS{31}[Kaf.] Car le Seigneur ne rejette pas à toujours\FTNT{Es. 57:16 ; Ps. 77:8.}.
\VS{32}Mais s'il afflige quelqu'un, il a aussi compassion selon la grandeur de sa miséricorde.
\VS{33}Car ce n'est pas sa volonté d'affliger et d'humilier les fils des hommes.
\VS{34}[Lamed.] Lorsqu'on foule aux pieds tous les prisonniers de la terre,
\VS{35}lorsqu'on pervertit la justice humaine en la présence du Très-Haut,
\VS{36}lorsqu'on fait tort à quelqu'un dans son procès, le Seigneur ne le voit-il pas ?
\VS{37}[Mem.] Qui est-ce qui dit qu'une chose est arrivée sans que le Seigneur l'ait commandé ?
\VS{38}Les maux et les biens\FTNT{Es. 45:7 ; Am. 3:6 ; Job. 1:21.} ne procèdent-ils pas de la bouche du Très-Haut ?
\VS{39}Pourquoi un homme vivant se plaindrait-il, un homme, à cause de la peine de ses péchés ?
\TextTitle{Le peuple appelé à s'examiner pour revenir à Yahweh}
\VS{40}[Nun.] Recherchons nos voies, sondons-les, et retournons à Yahweh\FTNT{Ps. 119:59 ; 2 Co. 13:5.} ;
\VS{41}élevons nos cœurs et nos mains vers Dieu qui est au ciel :
\VS{42}Nous avons péché, nous avons été rebelles ! Tu n'as pas pardonné !
\VS{43}[Samech.] Tu nous as couverts de ta colère, et tu nous as poursuivis ; tu as tué sans épargner ;
\VS{44}tu t'es couvert d'une nuée pour que les prières ne te parviennent pas.
\VS{45}Tu nous as fait être la raclure et le rebut au milieu des peuples.
\VS{46}[Pe.] Tous nos ennemis ouvrent leur bouche contre nous.
\VS{47}La frayeur et la fosse, le dégât et la calamité nous sont arrivés\FTNT{Es. 24:18 ; Jé. 48:44.}.
\VS{48}De mes yeux coulent des torrents d'eau à cause de la ruine de la fille de mon peuple.
\VS{49}[Ayin.] Mon œil fond en larmes, sans repos, sans relâche,
\VS{50}jusqu'à ce que Yahweh regarde et voie des cieux\FTNT{Ps. 80:15 ; Ps. 102:20.} ;
\VS{51}mon œil fait souffrir mon âme à cause de toutes les filles de ma ville.
\TextTitle{Yahweh, le soutien de Jérémie dans la détresse}
\VS{52}[Tsade.] Ceux qui sont mes ennemis sans cause m'ont poursuivi à outrance, comme après un oiseau.
\VS{53}Ils ont voulu anéantir ma vie dans une fosse, et ils ont jeté une pierre sur moi.
\VS{54}Les eaux ont coulé par-dessus ma tête ; je disais : Je suis retranché !
\VS{55}[Qof.] J'ai invoqué ton nom, ô Yahweh, du fond de la fosse\FTNT{Jé. 38:6.}.
\VS{56}Tu as entendu ma voix : Ne ferme pas tes oreilles à mes soupirs, à mes cris !
\VS{57}Au jour où je t'ai invoqué, tu t'es approché, et tu as dit : Ne crains rien !
\VS{58}[Resh.] Ô Seigneur, tu as plaidé la cause de mon âme, tu as racheté ma vie.
\VS{59}Tu as vu, ô Yahweh ! le tort qu'on me fait, fais-moi justice !
\VS{60}Tu as vu toutes les vengeances dont ils ont usé, et toutes leurs machinations contre moi.
\VS{61}[Shin.] Yahweh, tu as entendu leurs outrages, toutes leurs machinations contre moi,
\VS{62}les discours de ceux qui se lèvent contre moi, et leur dessein qu'ils ont contre moi tout au long du jour.
\VS{63}Considère quand ils sont assis et quand ils se lèvent, car je suis leur chanson.
\VS{64}[Tav.] Rends-leur la pareille, ô Yahweh, selon l'œuvre de leurs mains ;
\VS{65}livre-les à l'endurcissement de leur cœur, à ta malédiction.
\VS{66}Poursuis-les dans ta colère, et extermine-les de dessous les cieux, ô Yahweh !
\Chap{4}
\TextTitle{Crimes et apostasie du peuple}
\VerseOne{}[Aleph.] Comment l'or est-il devenu obscur, et le fin or s'est-il altéré ? Comment les pierres du sanctuaire sont-elles répandues aux coins de toutes les rues ?
\VS{2}[Beth.] Comment les chers fils de Sion, qui étaient estimés à l'égal de l'or pur, sont-ils reputés comme des vases de terre, ouvrage des mains du potier !
\VS{3}[Guimel.] Il y a même des monstres marins qui présentent leurs mamelles et allaitent leurs petits ; mais la fille de mon peuple est devenue cruelle comme les autruches du désert.
\VS{4}[Daleth.] La langue de celui qui têtait s'est attachée à son palais dans sa soif ; les enfants demandent du pain, et personne ne leur en donne\FTNT{Jé. 52:6.}.
\VS{5}[He.] Ceux qui mangeaient des mets délicats sont en désolation dans les rues ; ceux qui étaient nourris sur l'étoffe écarlate embrassent le fumier.
\VS{6}[Vav.] L'iniquité de la fille de mon peuple est plus grande que le péché de Sodome, renversée en un instant, sans que personne n'ait tourné la main sur elle.
\VS{7}[Zayin.] Ses naziréens étaient plus purs que la neige, plus blancs que le lait ; leur teint était plus vermeil que les pierres précieuses ; ils étaient polis comme un saphir.
\VS{8}[Heth.] Leur apparence est plus sombre que le noir ; on ne les reconnaît pas dans les rues ; ils ont la peau collée sur les os ; elle est devenue sèche comme du bois\FTNT{Job. 30:30.}.
\VS{9}[Teth.]Ceux qui ont été mis à mort par l'épée, ont été plus heureux que ceux qui sont morts par la famine, qui eux sont consumés peu à peu, transpercés par le défaut du fruit des champs.
\VS{10}[Yod.] Les mains des femmes, naturellement tendres, font cuire leurs enfants ; ils leur servent de nourriture dans la ruine de la fille de mon peuple\FTNT{De. 28:57 ; 2 R. 6:29.}.
\VS{11}[Kaf.] Yahweh a accompli sa fureur, il a répandu l'ardeur de sa colère ; il a allumé dans Sion un feu qui en dévore les fondements.
\VS{12}[Lamed.] Les rois de la terre, et tous les habitants de la terre habitable n'auraient jamais cru que l'adversaire et l'ennemi entrerait dans les portes de Jérusalem.
\VS{13}[Mem.] Cela est arrivé à cause des péchés de ses prophètes, et des iniquités de ses sacrificateurs, qui répandaient le sang des justes au milieu d'elle\FTNT{Jé. 5:29-31. Le péché des conducteurs donne accès à l'ennemi pour les détruire ainsi que les biens qui leur ont été confiés (Lu. 11:21-22).}.
\VS{14}[Nun.] Ils erraient comme des aveugles dans les rues, souillés de sang, au point qu'on ne pouvait pas toucher leurs vêtements.
\VS{15}[Samech.] On leur criait : retirez-vous, souillés, retirez-vous, retirez-vous, ne nous touchez point. Quand ils se sont enfuis, ils ont erré ça et là ; on a dit parmi les nations : Ils n'auront plus leur demeure !
\VS{16}[Pe.] La face de Yahweh les a dispersés, il ne veut plus les regarder ; ils n'ont pas eu de respect pour les sacrificateurs, et n'ont pas été miséricordieux envers les vieillards.
\VS{17}[Ayin.] Pour nous, nos yeux se consumaient après un vain secours ; nous regardions du haut de nos lieux élevés vers une nation qui ne pouvait pas délivrer\FTNT{Jé. 18:15.}.
\VS{18}[Tsade.] Ils ont épié nos pas afin de nous empêcher d'aller sur nos places ; notre fin s'approchait, nos jours étaient accomplis… Notre fin est arrivée !
\VS{19}[Qof.] Nos persécuteurs étaient plus légers que les aigles des cieux ; ils nous ont poursuivis sur les montagnes, ils ont mis des embûches contre nous dans le désert.
\VS{20}[Resh.] Le souffle de nos narines, l'oint de Yahweh\FTNT{L'oint en question est le roi Josias (2 R. 21:24 ; 22 ; 23).}, a été pris dans leurs fosses, celui de qui nous disions : Nous vivrons sous son ombre parmi les nations.
\VS{21}[Shin.] Réjouis-toi, sois dans l'allégresse, fille d'Edom, habitante du pays d'Uts ! La coupe passera aussi vers toi ; tu en seras enivrée, et tu seras mise à nu\FTNT{Jé. 25:15-18 ; Ps. 137:7.}.
\VS{22}[Tav.] Fille de Sion, ton iniquité est expiée ; il ne t'enverra plus en exil. Fille d'Edom, il châtiera ton iniquité, il découvrira tes péchés.
\Chap{5}
\TextTitle{Supplications de Jérémie à Yahweh}
\VerseOne{}Souviens-toi, ô Yahweh, de ce qui nous est arrivé ! Regarde et vois notre opprobre !
\VS{2}Notre héritage a été renversé par des étrangers, nos maisons par des inconnus.
\VS{3}Nous sommes devenus comme des orphelins qui sont sans pères, et nos mères sont comme des veuves.
\VS{4}Nous buvons notre eau à prix d'argent, et notre bois nous est vendu.
\VS{5}Ceux qui nous poursuivent sont sur notre cou ; nous sommes épuisés, nous n'avons pas de repos.
\VS{6}Nous avons étendu la main vers l'Egypte, et vers l'Assyrie pour nous rassasier de pain.
\VS{7}Nos pères ont péché, ils ne sont plus, et c'est nous qui portons la peine de leurs iniquités\FTNT{Chaque homme naît pécheur et hérite de la nature pécheresse d'Adam (Ge. 3:20 ; Ac. 17:26 ; Ro. 5:12-21). De ce fait, les péchés commis par les parents ont des conséquences sur les enfants (Ex. 20:4-5 ). Jésus-Christ nous a délivrés du péché d'Adam et de celui de nos ancêtres à la croix (Col. 1:12). Lors de notre naissance d'en haut, les péchés de notre passé et de nos origines sont expiés (2 Co. 5:17 ; Ep. 1:7 ; Ep. 2:1-15 ; Col. 1:12-14 ; Col. 2:13-15 ; 1 Pi. 1:18-19 ; 1 Jn. 1:7 ; 1 Jn. 1:9). Les péchés des ancêtres et leurs conséquences touchent les personnes qui vivent dans les péchés de leurs ancêtres, c'est-à-dire ceux qui haïssent Dieu et ses commandements (De. 24:16 ; Jé. 31:29-34 ; Ez. 18:17-20).}.
\VS{8}Les esclaves dominent sur nous, et personne ne nous délivre de leurs mains.
\VS{9}Nous amenons notre pain au péril de notre vie, à cause de l'épée du désert.
\VS{10}Notre peau est brûlante comme un four, à cause l'ardeur de la faim.
\VS{11}Ils ont déshonoré les femmes dans Sion, les vierges dans les villes de Juda.
\VS{12}Des chefs ont été pendus par leurs mains ; et ils n'ont pas honoré la personne des vieillards.
\VS{13}Ils ont pris les jeunes gens pour moudre, et les enfants sont tombés sous le bois.
\VS{14}Les vieillards ont cessé de se trouver aux portes, et les jeunes gens de chanter.
\VS{15}La joie a disparu de notre cœur, et notre danse est changée en deuil.
\VS{16}La couronne de notre tête est tombée ! Malheur à nous, parce que nous avons péché !
\VS{17}C'est pourquoi notre coeur est languissant. À cause de ces choses, nos yeux sont obscurcis.
\VS{18}À cause de la montagne de Sion qui est désolée ; les renards s'y promènent.
\VS{19}Toi, ô Yahweh, tu demeures éternellement, et ton trône subsiste de génération en génération.
\VS{20}Pourquoi nous oublierais-tu à jamais ? pourquoi nous délaisserais-tu si longtemps ?
\VS{21}Convertis-nous à toi, ô Yahweh ! et nous serons convertis ; renouvelle nos jours comme ils étaient autrefois\FTNT{Jé. 30:20 ; Jé. 31:18 ; Ps. 80:3.}.
\VS{22}Ou bien, nous aurais-tu entièrement rejetés ? Serais-tu extrêmement courroucé contre nous ?
\PPE{}
\end{multicols}

%\clearpage\ShortTitle{Ec.}\BookTitle{Ecclésiaste}\BFont
\noindent\hrulefill
{\footnotesize
\textit{
\bigskip
{\centering{}
\\Auteur~: Salomon
\\(Heb.~: Qohelet)
\\Signification~: Prédicateur
\\Thème~: Les raisonnements humains
\\Date de rédaction~: 10\up{ème} siècle av. J.-C.\\}
}
\textit{
\\Ce livre, du fait de sa date de rédaction, est généralement attribué à Salomon en raison de l'allusion faite au premier chapitre et du style adopté. L'Ecclésiaste figure d'ailleurs dans le canon des livres reconnus d'inspiration divine.
\\La problématique centrale du livre est de savoir si la vie vaut la peine d'être vécue ou non. L'auteur y répondit en connaissance de cause car il avait obtenu tout ce que l'homme pouvait désirer~: les richesses, le luxe, la volupté, la sagesse… Sans pour autant incriminer Dieu, il dressa le constat de ce qu'est l'expérience humaine. Selon lui, l'homme vit dans un cycle d'éternels recommencements où tout n'est que poursuite du vent et vanité.\bigskip
}
}
\par\nobreak\noindent\hrulefill
\begin{multicols}{2}
\Chap{1}
\TextTitle{Tout est vanité\FTNTT{Ec. 12:8.}}
\VerseOne{}Les Paroles de l'Ecclésiaste, fils de David, roi de Jérusalem.
\VS{2}Vanité des vanités, dit l'Ecclésiaste, vanité des vanités, tout est vanité.
\TextTitle{Le cycle du temps}
\VS{3}Quel avantage a l'homme de tout son travail auquel il s'occupe sous le soleil~?\FTNT{Ec. 2:22~; Ec. 3:9.}
\VS{4}Une génération passe et une autre génération vient, mais la terre demeure toujours ferme.
\VS{5}Le soleil aussi se lève et le soleil se couche~; il soupire après le lieu d'où il se lève.
\VS{6}Le vent va vers le midi, tourne vers le nord~; il va tournoyant çà et là, et il retourne après ses circuits.
\VS{7}Tous les fleuves vont à la mer, et la mer n'en est point remplie~; les fleuves retournent au lieu d'où ils étaient partis, pour revenir\FTNT{Job 38:8-11~; Ps. 104:9-10.} à la mer. 
\VS{8}Toutes les choses sont lassantes et l'homme ne peut en parler~; l'œil n'est jamais rassasié de voir\FTNT{Pr. 27:20.} et l'oreille ne se lasse pas d'entendre. 
\VS{9}Ce qui a été, c'est ce qui sera, et ce qui s'est fait, c'est ce qui se fera, il n'y a rien de nouveau sous le soleil\FTNT{Ec. 3:15.}.
\VS{10}Y~a-t-il quelque chose dont on puisse dire~: Regarde cela, il est nouveau~? Il a déjà été dans les siècles qui ont été avant nous.
\VS{11}On ne se souvient plus des choses d'autrefois~; de même on ne se souviendra point des choses à venir et ceux qui viendront n'en auront aucun souvenir. 
\TextTitle{La sagesse des hommes ne comble pas}
\VS{12}Moi l'Ecclésiaste, j'ai été roi sur Israël à Jérusalem.
\VS{13}Et j'ai appliqué mon cœur à rechercher et à sonder par la sagesse tout ce qui se fait sous les cieux~: C'est une occupation désagréable que Dieu a donnée aux hommes, afin qu'ils s'y occupent\FTNT{1 R. 4:30-34~; Ec. 7:25.}.
\VS{14}J'ai vu toutes les œuvres qui se font sous le soleil~; et voici tout est vanité et tourment d'esprit.
\VS{15}Ce qui est courbé ne peut se redresser, et ce qui manque ne peut être compté.
\VS{16}J'ai parlé en mon cœur, disant~: Voici, je suis devenu grand et j'ai surpassé en sagesse tous ceux qui ont été avant moi sur Jérusalem, et mon cœur a vu beaucoup de sagesse et de science.
\VS{17}Et j'ai appliqué mon cœur à connaître la sagesse, et à connaître les sottises et la stupidité~; et j'ai reconnu que cela aussi était un tourment d'esprit.
\VS{18}Car là où il y a beaucoup de sagesse, il y a beaucoup de chagrin, et celui qui augmente sa connaissance, augmente son chagrin.
\Chap{2}
\TextTitle{Les richesses ne comblent pas}
\VerseOne{}J'ai dit en mon cœur~: Allons, que je t'éprouve maintenant par la joie et prends du bon temps. Et voici, c'est encore une vanité\FTNT{Lu. 12:19.}.
\VS{2}J'ai dit concernant le rire~: Il est insensé~! Et concernant la joie~: A quoi sert-elle~?
\VS{3}J'ai recherché en moi-même le moyen de me traiter délicatement, de faire que mon cœur s'accoutume cependant à la sagesse, et qu'il comprenne ce que c'est que la folie, jusqu'à ce que je voie ce qu'il est bon aux hommes de faire sous les cieux, pendant les jours de leur vie. 
\VS{4}Je me suis fait des choses magnifiques~; je me suis bâti des maisons~; je me suis planté des vignes.
\VS{5}Je me suis fait des jardins et des vergers, et j'y plantai des arbres fruitiers de toutes sortes~;
\VS{6}je me suis fait des réservoirs d'eaux pour arroser la forêt où poussent les arbres.
\VS{7}J'ai acquis des hommes et des femmes esclaves~; et j'ai eu des esclaves nés dans ma maison, et j'ai eu plus de gros et de menu bétail que tous ceux qui ont été avant moi dans Jérusalem. 
\VS{8}Je me suis aussi amassé de l'argent et de l'or, et des plus précieux joyaux qui se trouvent chez les rois et dans les provinces\FTNT{1 R. 9:28~; 1 R. 10:10~; 2 Ch. 1:15.}. Je me suis acquis des chanteurs et des chanteuses, et les délices des hommes~; une harmonie d'instruments de musique, même plusieurs harmonies de toutes sortes d'instruments\FTNT{La plupart des bibles ont traduit la deuxième partie de ce verset par le mot «~femme~», or le sens du terme hébreu «~shiddah~» est incertain. Toutefois, le contexte de ce verset montre clairement que Salomon parlait des instruments de musique et des chanteurs et chanteuses qu'il a acquis, et non de ses conquêtes féminines.}. 
\VS{9}Je me suis aggrandi et me suis accru plus que tous ceux qui ont été avant moi dans Jérusalem. Et ma sagesse est demeurée avec moi.
\VS{10}Enfin je n'ai rien refusé à mes yeux de tout ce qu'ils ont demandé~; et je n'ai épargné aucune joie à mon cœur~; car mon cœur s'est réjoui de tout mon travail et c'est là tout ce que j'ai eu de tout mon travail.
\VS{11}Mais ayant considéré toutes mes œuvres que mes mains avaient faites, et tout le travail auquel je m'étais occupé en les faisant, voilà tout était vanité et tourment d'esprit~; tellement que l'homme n'a aucun avantage de ce qui est sous le soleil.
\TextTitle{Le sage et l'insensé ont le même sort}
\VS{12}Puis je me suis mis à considérer tant la sagesse que les sottises et la folie, (or qui est l'homme qui pourrait suivre le roi dans ce qui a été déjà fait~?).
\VS{13}Et j'ai vu que la sagesse a beaucoup d'avantage sur la folie, comme la lumière a beaucoup d'avantage sur les ténèbres.
\VS{14}Le sage a ses yeux à sa tête, et l'insensé marche dans les ténèbres. Mais j'ai aussi reconnu qu'un même sort leur arrive à tous\FTNT{Ps. 49:11~; Ec. 3:17~; Ec. 9:2.}.
\VS{15}C'est pourquoi j'ai dit en mon cœur~: Il m'arrivera le même sort que l'insensé~; de quoi donc me servira-t-il alors d'avoir été plus sage~? C'est pourquoi j'ai dit en mon cœur que cela aussi est une vanité. 
\VS{16}Car le souvenir du sage n'est pas plus éternel que celui de l'insensé, parce que ce qui est maintenant va être oublié dans les jours qui suivent. Le sage meurt aussi bien que l'insensé\FTNT{Ec. 8:10~; Ec. 9:5.}~!
\VS{17}C'est pourquoi j'ai haï cette vie, car les choses qui se sont faites sous le soleil m'ont déplu~; car tout est vanité, et tourment d'esprit.
\VS{18}J'ai aussi haï tout mon travail, auquel je me suis occupé sous le soleil, parce que je le laisserai à l'homme qui sera après moi\FTNT{Ec. 4:8.}.
\VS{19}Et qui sait s'il sera sage ou insensé~? Cependant il sera maître de tout mon travail, auquel je me suis occupé et de ce en quoi j'ai été sage sous le soleil. Cela aussi est une vanité.
\VS{20}C'est pourquoi j'ai fait en sorte que mon cœur perde toute espérance de tout le travail auquel je m'étais occupé sous le soleil.
\VS{21}Car il y a tel homme, dont le travail a été avec sagesse, science, et adresse, qui néanmoins le laisse à celui qui n'y a point travaillé comme étant sa part~; cela aussi est une vanité et un grand mal. 
\VS{22}Car qu'est-ce que l'homme a de tout son travail et du désir de son cœur, dont il souffre sous le soleil~?
\VS{23}Puisque tous ses jours ne sont que douleur, et son occupation n'est que chagrin~; même la nuit son cœur ne repose point. Cela aussi est une vanité\FTNT{Ps. 90:9~; Job. 14:1.}.
\VS{24}N'est-ce donc pas un bien pour l'homme de manger, et de boire, et de faire que son âme jouisse du bien dans son travail~? J'ai vu aussi que cela vient de la main de Dieu\FTNT{Ec. 3:12~; Ec. 3:22~; Ec. 5:18~; Ec. 8:15.}.
\VS{25}Car qui en mangera, qui s'est réjoui plus que moi~?
\VS{26}Parce que Dieu donne à celui qui lui est agréable, de la sagesse, de la science et de la joie~; mais il donne au pécheur de l'occupation à recueillir et à assembler, afin que cela soit donné à celui qui est agréable à Dieu. Cela aussi est une vanité et un tourment d'esprit\FTNT{Pr. 13:22~; Pr. 28:8~; Job 27:17.}.
\Chap{3}
\TextTitle{Il y a un temps pour toute chose}
\VerseOne{}A toute chose sa saison, et à toute affaire sous les cieux son temps.
\VS{2}Un temps pour naître et un temps pour mourir~; un temps pour planter et un temps pour arracher ce qui est planté~;
\VS{3}un temps pour tuer et un temps pour guérir~; un temps pour démolir et un temps pour bâtir~;
\VS{4}un temps pour pleurer et un temps pour rire~; un temps pour se lamenter et un temps pour sauter de joie~;
\VS{5}un temps pour jeter des pierres et un temps pour ramasser des pierres~; un temps pour embrasser et un temps pour s'éloigner des embrassements~;
\VS{6}un temps pour chercher et un temps pour perdre~; un temps pour garder et un temps pour jeter~;
\VS{7}un temps pour déchirer et un temps pour coudre~; un temps pour être silencieux et un temps pour parler~;
\VS{8}un temps pour aimer et un temps pour haïr~; un temps pour la guerre et un temps pour la paix.
\TextTitle{Dieu fait toute chose belle en son temps}
\VS{9}Quel avantage celui qui travaille a-t-il de sa peine~?
\VS{10}J'ai considéré cette occupation que Dieu a donnée aux fils des hommes pour s'y appliquer.
\VS{11}Il a fait que toutes choses sont belles en leur temps~; aussi a-t-il mis l'éternité dans leur cœur, sans toutefois que l'homme puisse comprendre du commencement à la fin\FTNT{Ec. 8:17.}l'œuvre que Dieu a faite. 
\VS{12}C'est pourquoi j'ai reconnu qu'il n'y a rien de meilleur aux hommes, que de se réjouir et de se faire du bien pendant leur vie. 
\VS{13}Et même si un homme mange et boit et jouit du bien-être de tout son travail, c'est un don de Dieu\FTNT{Ec. 5:18~; Ec. 8:15~; Ec. 9:7.}.
\VS{14}J'ai reconnu que tout ce que Dieu fait subsiste à toujours, il n'y a rien à y ajouter et rien à en retrancher, et Dieu le fait afin que devant lui, on le craigne.
\VS{15}Ce qui a été, est maintenant~; et ce qui doit être, a déjà été~; et Dieu rappelle ce qui est passé.
\VS{16}J'ai encore vu sous le soleil qu'au lieu établi pour juger, il y a de la méchanceté~; et qu'au lieu établi pour la justice, il y a de la méchanceté.
\VS{17}J'ai dit en mon cœur~: Dieu jugera le juste et le méchant~; car il y a là un temps pour toute chose et pour toute œuvre.
\VS{18}J'ai dit en mon cœur sur l'état des fils de l'homme, que Dieu les éprouverait, et qu'ils verraient qu'ils ne sont que des bêtes.
\VS{19}Car le sort des fils d'Adam et le sort de la bête est un même sort~; telle qu'est la mort de l'un, telle est la mort de l'autre. Tous ont un même souffle et la supériorité de l'homme sur la bête est nulle. Car tout est vanité.
\VS{20}Tout va dans un même lieu~; tout a été fait de la poussière, et tout retourne à la poussière\FTNT{Ge. 3:19~; Job 34:15~; Ec. 6:6~; Ec. 12:9.}.
\VS{21}Qui sait si l'esprit des fils de l'homme monte en haut, et si l'esprit de la bête descend en bas dans la terre\FTNT{Les animaux comme les hommes ont une âme et un esprit (Ge. 1:20~; Ez. 1:1-28). A leur mort, leurs esprits quittent leurs corps (Ja. 2:26). L'homme régénéré reçoit le Saint-Esprit, ce qui n'est pas le cas des animaux (Ro. 8:16). Les animaux, tout comme la création tout entière, attendent leur rédemption, car ils ont été soumis à la corruption à cause du péché de l'homme (Ro. 8:19-22).Dans le royaume millénaire, il y aura des animaux (Es. 11:6-9).}~?
\VS{22}J'ai donc vu qu'il n'y a rien de meilleur pour l'homme que de se réjouir de ses œuvres~: C'est là sa part. Car qui le ramènera pour voir ce qui sera après lui~?
\Chap{4}
\TextTitle{Un monde injuste}
\VerseOne{}Puis je me suis mis à regarder toutes les injustices qui se font sous le soleil~; et voici les larmes de ceux à qui on fait tort, et ils n'ont point de consolation. Et la force est du côté de ceux qui leur font tort, et ils n'ont point de consolateur. 
\VS{2}C'est pourquoi j'estime plus les morts qui sont déjà morts, que les vivants qui sont encore vivants\FTNT{Ec. 7:1.}~;
\VS{3}même j'estime celui qui n'a pas encore été, plus heureux que les uns et les autres~; car il n'a pas vu les mauvaises actions qui se font sous le soleil.
\VS{4}Puis j'ai vu que tout travail et tout succès dans le travail n'est que jalousie de l'un à l'égard de l'autre. Cela aussi est une vanité et un tourment d'esprit. 
\VS{5}L'insensé se croise les mains et dévore sa propre chair\FTNT{Pr. 6:10~; Pr. 19:24~; Pr. 24:33~; Pr. 26:15.}.
\VS{6}Mieux vaut le creux de la main pleine avec repos, que les deux mains pleines avec travail et tourment d'esprit\FTNT{Ps. 37:16~; Pr. 15:16-17~; Pr. 16:8.}. 
\VS{7}Puis je me suis mis à regarder une autre vanité sous le soleil. 
\VS{8}C'est qu'il y a tel qui est seul, et qui n'a point de second, qui aussi n'a ni fils ni frère et qui cependant ne met nulle fin à son travail~; même son œil ne voit jamais assez de richesses, et il ne se dit point en lui-même~: Pour qui est-ce que je travaille, et que je prive mon âme du bien~? Cela aussi est une vanité et une fâcheuse occupation\FTNT{Ec. 2:26~; Ps. 39:7~; Lu. 12:20.}.
\VS{9}Deux valent mieux qu'un, car ils ont un meilleur salaire de leur travail.
\VS{10}Même si l'un des deux tombe, l'autre relèvera son compagnon~; mais malheur à celui qui est seul~; parce qu'étant tombé, il n'aura personne pour le relever. 
\VS{11}Si deux aussi couchent ensemble, ils en auront plus de chaleur~; mais celui qui est seul, comment aura-t-il chaud~? 
\VS{12}Et si quelqu'un a le dessus sur l'un ou sur l'autre, les deux peuvent lui résister~; et la corde à trois cordons ne se rompt pas rapidement.
\VS{13}Un enfant pauvre et sage vaut mieux qu'un roi vieux et insensé qui ne sait ce que c'est que d'être averti.
\VS{14}Car tel qui sort de prison pour régner, et de même tel étant né roi, devient pauvre dans son royaume.
\VS{15}J'ai vu tous les vivants qui marchent sous le soleil suivre le fils qui est la seconde personne après le roi, et qui doit être à sa place. 
\VS{16}Il n'y a pas de fin à tout le peuple, à tous ceux qui ont été devant eux~; cependant ceux qui viendront après ne se réjouiront point en lui. Certainement cela aussi est une vanité, et un tourment d'esprit. 
\TextTitle{Le sacrifice des insensés}
\VS{17}Quand tu entres dans la maison de Dieu, prends garde à ton pied, et approche-toi pour écouter, plutôt que pour donner ce que donnent les insensés, car ils ne savent pas qu'ils font mal.
\Chap{5}
\VerseOne{}Ne te précipite point à parler, et que ton cœur ne se hâte point de parler devant Dieu~; car Dieu est au ciel, et toi sur la terre~; c'est pourquoi use de peu de paroles.
\VS{2}Car comme le songe vient de la multitude des occupations~; ainsi la voix des insensés sort de la multitude des paroles\FTNT{Pr. 10:19.}.
\VS{3}Quand tu as fais quelque vœu à Dieu, ne diffère point de l'accomplir~; car il ne prend point de plaisir aux insensés~; accomplis donc le vœu que tu as fait\FTNT{No. 30:3. De. 23:21}.
\VS{4}Il vaut mieux que tu ne fasses point de vœux que d'en faire et de ne pas les accomplir\FTNT{De. 23:21-22.}.
\VS{5}Ne permets pas à ta bouche de faire pécher ta chair, et ne dis point devant le messager de Dieu que c'est un péché involontaire. Pourquoi Yahweh s'irriterait-il de tes paroles, et détruirait-il l'œuvre de tes mains~?
\VS{6}Car comme dans la multitude des songes il y a des vanités, aussi y en a-t-il beaucoup dans la multitude des paroles~; mais crains Dieu\FTNT{Ec. 10:14~; Pr. 10:19.}.
\VS{7}Si tu vois dans la Province qu'on fasse tort au pauvre, et que le droit et la justice y soient violés, ne t'étonne point de cela~; car celui qui est plus élevé que les plus hauts élevés y prend garde, et il y en a de plus élevés qu'eux\FTNT{Es. 3:14-15.}.
\TextTitle{Vanité des richesses}
\VS{8}C'est un avantage pour le pays, un roi qui travaille dans les champs.
\VS{9}Celui qui aime l'argent n'est point rassasié par l'argent\FTNT{Jé. 6:13~; Pr. 22:7~; Pr. 28:16~; Mt. 6:33~; Mt. 7:7-11~; Lu. 12:13-20~; Ac. 20:33~; 2 Co. 9:5~; Ep. 4:19~; Ep. 5:5~; Col. 3:5~; 1 Ti. 6:10~; Hé. 13:5.}, et celui qui aime les richesses n'en est pas nourri~; cela aussi est une vanité. 
\VS{10}Où il y a beaucoup de bien, là il y a beaucoup de gens qui le mangent~; et quel avantage en revient-il à son maître, sinon qu'il le voit de ses yeux~? 
\VS{11}Le sommeil de celui qui travaille est doux, qu'il mange peu ou beaucoup~; mais le rassasiement du riche ne le laisse point dormir. 
\VS{12}Il y a un mal fâcheux que j'ai vu sous le soleil, c'est que des richesses sont conservées à leurs maîtres afin qu'ils en aient du mal. 
\VS{13}Ces richesses périssent par quelque fâcheux accident~; de sorte qu'on aura engendré un fils et il n'aura rien entre ses mains.
\VS{14}Et comme il est sorti du ventre de sa mère, il s'en retournera nu, s'en allant comme il était venu, et il n'emportera rien de son travail auquel il a employé ses mains\FTNT{1 Ti. 6:7.}.
\VS{15}Et c'est aussi un mal fâcheux, que comme il est venu, il s'en va de même~; et quel avantage a-t-il d'avoir travaillé pour du vent~?
\VS{16}Il mange aussi tous les jours de sa vie dans les ténèbres et se chagrine beaucoup, et son mal va jusqu'à la fureur.
\VS{17}Voilà donc ce que j'ai vu~; que c'est une chose bonne et agréable à l'homme de manger, de boire et de jouir du bien-être de tout son travail qu'il fait sous le soleil, pendant le nombre des jours de vie que Dieu lui a donnés~; car c'est là sa part.
\VS{18}Aussi ce que Dieu donne de richesses et de biens à un homme, quel qu'il soit~; ce dont il le fait maître pour en manger, pour en prendre sa part et pour se réjouir de son travail~; c'est là un don de Dieu. 
\VS{19}Car il ne se souviendra pas beaucoup des jours de sa vie, parce que Dieu lui répond par la joie de son cœur. 
\Chap{6}
\TextTitle{Vanité de la vie de l'homme}
\VerseOne{}Il y a un mal que j'ai vu sous le soleil, et qui est fréquent parmi les hommes.
\VS{2}C'est qu'il y a tel homme à qui Dieu a donné des richesses, des biens et des honneurs, en sorte qu'il ne manque rien pour son âme de tout ce qu'il peut souhaiter. Mais Dieu ne l'en fait pas le maître pour en manger, car un étranger le mangera. Cela est une vanité et un mal fâcheux. 
\VS{3}Quand un homme engendrerait cent fils, qu'il vivrait plusieurs années, en sorte que les jours de ses années se soient fort multipliés, cependant si son âme ne s'est point rassasiée de bien, et même s'il n'a point eu de sépulture, je dis qu'un avorton vaut mieux que lui.
\VS{4}Car il est venu en vain, et s'en va dans les ténèbres, et son nom est couvert de ténèbres~;
\VS{5}Il n'a même point vu le soleil~; il n'a rien connu~; il a plus de repos que cet homme-là\FTNT{Job 3:16.}.
\VS{6}Et s'il vivait deux fois mille ans, et qu'il ne jouit d'aucun bien, tous ne vont-ils pas dans un même lieu\FTNT{Ec. 3:20~; Job 3:13-19~; Job 30:23~; Ps. 89:48~; Hé. 9:27.}~?
\VS{7}Tout le travail de l'homme est pour sa bouche, et cependant son âme n'est jamais satisfaite\FTNT{Les richesses de ce monde ne peuvent jamais combler le vide de l'âme. Seul l'amour de Dieu peut réellement inonder nos âmes (Pr. 13:4).}.
\VS{8}Car qu'est-ce que le sage a de plus que l'insensé~? Ou quel avantage a le malheureux qui sait se conduire devant les vivants~?
\VS{9}Mieux vaut ce qu'on voit de ses yeux, que les grandes recherches que fait l'âme. Cela aussi est une vanité, et un tourment d'esprit\FTNT{1 Ti. 6:9.}.
\VS{10}Ce qui existe a déjà été appelé par son nom\FTNT{Ec. 1:9~; Ec. 3:15.}~; et savait-on ce que devait être l'homme, et qu'il ne pourrait plaider avec celui qui est plus fort que lui~? 
\VS{11}Quand on a beaucoup de choses, on a beaucoup de vanités. Quel avantage en a l'homme~? 
\VS{12}Car qui est-ce qui connaît ce qui est bon à l'homme dans sa vie, pendant les jours de la vie de sa vanité, lesquels il passe comme une ombre~? Et qui est-ce qui déclarera à l'homme ce qui sera après lui sous le soleil\FTNT{Ps. 144:4~; Ec. 8:7~; Ec. 8:13~; Ec. 10:14~; Ja. 4:13-14.}~?
\Chap{7}
\TextTitle{La sagesse qu'enseigne la vie de l'homme}
\VerseOne{}Une bonne réputation vaut mieux que le bon parfum, et le jour de la mort que le jour de la naissance\FTNT{Pr. 22:1.}.
\VS{2}Il vaut mieux aller dans une maison de deuil que d'aller dans une maison de festin~; car c'est là la fin de tout homme, et le vivant met cela dans son cœur.
\VS{3}Il vaut mieux le chagrin que le rire~; car par la tristesse du visage le cœur devient joyeux\FTNT{Ec. 8:1~; 2 Co. 7:10.}.
\VS{4}Le cœur des sages est dans la maison du deuil, mais le cœur des insensés est dans la maison de joie.
\VS{5}Il vaut mieux entendre la réprimande du sage, que d'entendre la chanson des hommes insensés\FTNT{Ps. 141:5~; Pr. 13:18~; Pr. 15:31-32.}.
\VS{6}Car tel qu'est le bruit des épines sous la chaudière, tel est le rire de l'insensé. Cela aussi est une vanité. 
\VS{7}Certainement l'oppression fait perdre le sens au sage~; et le don fait perdre l'entendement. 
\VS{8}Mieux vaut la fin d'une chose que son commencement. Mieux vaut l'homme qui est d'un esprit patient que l'homme qui est d'un esprit hautain. 
\VS{9}Ne te précipite point en ton esprit de t'irriter, car l'irritation repose dans le sein des insensés.
\VS{10}Ne dis point~: D'où vient que les jours passés ont été meilleurs que ceux-ci~? Car ce n'est pas par sagesse que tu demandes cela. 
\VS{11}La sagesse est bonne avec un héritage, elle est un avantage pour ceux qui voient le soleil.
\VS{12}Car on est à couvert à l'ombre de la sagesse, de même qu'à l'ombre de l'argent~; mais la science a cet avantage, que la sagesse fait vivre celui qui en est doué. 
\VS{13}Regarde l'œuvre de Dieu~: Qui pourra redresser ce qu'il a renversé~?
\VS{14}Au jour du bonheur, sois heureux, et au jour de l'adversité, prends-y garde~; car Dieu a fait l'un exactement comme l'autre, afin que l'homme ne trouve rien à redire après lui. 
\VS{15}J'ai vu tout ceci pendant les jours de ma vanité. Il y a tel juste qui périt dans sa justice, et il y a tel méchant qui prolonge ses jours dans sa méchanceté\FTNT{Ec. 8:14~; Job 21:7-8.}.
\VS{16}Ne te crois pas trop juste, et ne te fais pas plus sage qu'il ne faut~: Pourquoi t'exposer à la ruine\FTNT{Pr. 3:7~; Ro. 12:16.}~?
\VS{17}Ne sois point méchant à l'excès, et ne sois point insensé~: Pourquoi mourrais-tu avant ton temps\FTNT{Ec. 9:16.}~?
\VS{18}Il est bon que tu retiennes ceci, et que tu ne retires point ta main de cela~; car celui qui craint Dieu sort de tout.
\VS{19}La sagesse donne plus de force au sage que dix gouverneurs qui sont dans une ville.
\VS{20}Certainement il n'y a point d'homme juste sur la terre qui agisse toujours bien, et qui ne pèche point\FTNT{Ps. 14:3~; Pr. 20:9~; 2 Ch. 6:36~; Ja. 3:2~; Ro. 3:12~; 1 Jn. 1:8.}.
\VS{21}Ne mets point aussi ton cœur à toutes les paroles qu'on dira, afin que tu n'entendes pas ton serviteur médire de toi. 
\VS{22}Car ton cœur aussi a reconnu plusieurs fois que tu as pareillement mal parlé des autres. 
\VS{23}J'ai essayé tout ceci avec sagesse, et j'ai dit~: J'acquerrai de la sagesse~; mais elle s'est éloignée de moi. 
\VS{24}Ce qui est loin et ce qui est profond, qui le trouvera~?
\VS{25}Je me suis appliqué dans mon cœur à connaître, à sonder, et à chercher la sagesse et la raison de tout~; et à connaître la méchanceté de la folie, de la bêtise et des sottises. 
\VS{26}Et j'ai trouvé plus amère que la mort, la femme dont le cœur est un piège et un filet, et dont les mains sont des liens~; celui qui est agréable à Dieu lui échappera~; mais le pécheur sera pris par elle\FTNT{Pr. 5:3-4~; Pr. 6:26~; Pr. 7:13-27~; Pr. 9:13-16~; Pr. 22:14.}.
\VS{27}Voici, dit l'Ecclésiaste, ce que j'ai trouvé en cherchant la raison de toutes choses, l'une après l'autre~;
\VS{28}C'est que jusqu'à présent, mon âme a cherché, mais que je n'ai point trouvé, c'est que j'ai bien trouvé un homme entre mille~; mais pas une femme entre elles toutes. 
\VS{29}Seulement voici ce que j'ai trouvé~; c'est que Dieu a créé l'homme juste~; mais ils ont cherché beaucoup d'inventions.
\Chap{8}
\TextTitle{L'obéissance aux autorités}
\VerseOne{}Qui est tel que le sage~? Et qui sait ce que veulent dire les choses~? La sagesse de l'homme fait briller son visage, et son regard farouche en est changé\FTNT{Ec. 7:3~; Pr. 15:13.}.
\VS{2}Je te le dis~: Prends garde aux ordres du roi, et cela à cause du serment fait à Dieu.
\VS{3}Ne te précipite point de te retirer de devant sa face~; et ne persévère point dans une chose mauvaise~; car il fera tout ce qu'il lui plaira. 
\VS{4}En quelque lieu qu'est la parole du roi, là est la puissance~; et qui lui dira~: Que fais-tu~? 
\VS{5}Celui qui garde le commandement ne sentira aucun mal~; et le cœur du sage discerne le temps et ce qui est juste. 
\VS{6}Car dans toute affaire il y a un temps et un jugement, autrement mal sur mal tombe sur l'homme. 
\VS{7}Car il ne sait pas ce qui arrivera~; et même qui est-ce qui lui déclarera quand cela arrivera~? 
\VS{8}L'homme n'est point maître de son souffle\FTNT{Le souffle ou l'esprit de l'homme quitte son corps le jour de sa mort (Ps. 39:5~; Ja. 2:26).} pour pouvoir le retenir, il n'a aucune puissance sur le jour de la mort~; il n'y a point de délivrance dans ce combat, et la méchanceté ne délivrera point son maître.
\VS{9}J'ai vu tout cela, et j'ai appliqué mon cœur à toute œuvre qui se fait sous le soleil. Il y a un temps où l'homme domine sur l'autre pour son malheur.
\VS{10}Alors j'ai vu les méchants ensevelis et s'en aller~; et ceux qui avaient agi avec droiture s'en aller loin du lieu saint et être oubliés dans la ville. Cela aussi est une vanité\FTNT{Ec. 2:16~; Ec. 9:5.}.
\VS{11}Parce que la sentence contre les mauvaises œuvres ne s'exécute point promptement, à cause de cela le cœur des fils de l'homme se remplit en eux de l'envie de faire le mal\FTNT{Ec. 12:1.}.
\VS{12}Car bien que le pécheur fasse le mal cent fois, et qu'il y persévère longtemps, je sais aussi qu'il y aura du bonheur pour ceux qui craignent Dieu et qui révèrent sa face\FTNT{Job 22:21~; Pr. 1:33~; Es. 3:10.}.
\VS{13}Mais le bonheur n'est pas pour le méchant, et il ne prolongera point ses jours plus que l'ombre, parce qu'il n'a pas de crainte devant Dieu.
\VS{14}Il y a une vanité qui arrive sur la terre~: C'est qu'il y a des justes auxquels il arrive selon l'œuvre des méchants~; et il y a aussi des méchants auxquels il arrive selon l'œuvre des justes. Je dis que cela aussi est une vanité.
\VS{15}C'est pourquoi j'ai loué la joie, parce qu'il n'y a rien sous le soleil de meilleur à l'homme, que de manger et de boire et de se réjouir~; c'est aussi ce qui lui restera de son travail durant les jours de sa vie, que Dieu lui donne sous le soleil. 
\VS{16}Après avoir appliqué mon cœur à connaître la sagesse, et à regarder les occupations qu'il y a sur la terre, (car l'homme ne donne, ni jour ni nuit, de repos à ses yeux), 
\VS{17}après avoir vu, dis-je, toute l'œuvre de Dieu, j'ai vu que l'homme ne peut pas trouver l'œuvre qui se fait sous le soleil~; il a beau se fatiguer à chercher, il n'est pas capable de trouver~; et même si le sage dit la connaître, il ne peut la trouver.
\Chap{9}
\TextTitle{L'impuissance de la sagesse face à la mort}
\VerseOne{}Certainement j'ai appliqué mon cœur à tout cela~; et pour l'éclaircir à savoir que les justes, les sages et leurs actions sont dans la main de Dieu~; mais les hommes ne connaissent ni l'amour ni la haine de tout ce qui est devant eux. 
\VS{2}Tout arrive également à tous~; un même sort arrive au juste et au méchant~; au bon, au pur et au souillé~; à celui qui sacrifie et à celui qui ne sacrifie point~; le pécheur est comme l'homme de bien~; celui qui jure, comme celui qui craint de jurer. 
\VS{3}C'est un mal parmi tout ce qui se fait sous le soleil, c'est qu'il y a pour tous un même sort~; aussi le cœur des fils de l'homme est-il plein de méchanceté, et la folie est dans leur cœur pendant leur vie~; après cela, ils vont chez les morts. Qui est celui qui voudrait leur être associé\FTNT{Ec. 2:16~; Job 9:22}~?
\VS{4}Il y a de l'espérance pour tous ceux qui sont encore vivants~; et même un chien vivant vaut mieux qu'un lion mort.
\VS{5}Certainement les vivants savent qu'ils mourront, mais les morts ne savent rien, et ne gagnent plus rien~; car leur mémoire est mise en oubli. 
\VS{6}Aussi leur amour, leur haine, et leur envie ont déjà péri, et ils n'auront plus aucune part à tout ce qui se fait sous le soleil.
\VS{7}Va donc, mange ton pain avec joie, et bois gaiement ton vin~; car depuis longtemps Dieu prend plaisir à tes œuvres.
\VS{8}Que tes vêtements soient blancs en tout temps, et que le parfum ne manque point sur ta tête. 
\VS{9}Vis joyeusement tous les jours de ta vie de vanité avec la femme que tu aimes, qui t'a été donnée sous le soleil, tous les jours de ta vanité~; car c'est là ta part dans la vie, au milieu de ton travail que tu fais sous le soleil.
\VS{10}Tout ce que ta main trouve à faire, fais-le selon ton pouvoir~; car dans le scheol, où tu vas, il n'y a ni œuvre, ni pensée, ni connaissance, ni sagesse.
\VS{11}Je me suis tourné ailleurs, et j'ai vu sous le soleil que la course n'est point aux légers, ni la guerre aux héros, ni le pain aux sages, ni la richesse à ceux qui sont intelligents, ni la grâce aux savants~; mais que le temps et les circonstances décident de ce qui arrive à tous.
\VS{12}Car l'homme ne connaît pas son heure, comme les poissons qui sont pris au filet de malheur et les oiseaux qui sont pris au piège~; comme eux, les fils de l'homme sont enlacés au temps du malheur, lorsqu'il tombe subitement sur eux.
\VS{13}J'ai aussi vu cette sagesse sous le soleil, et elle m'a semblé grande.
\VS{14}Il y avait une petite ville, avec peu d'hommes dans son sein~; un roi puissant marcha contre elle, l'investit et bâtit de grands forts contre elle.
\VS{15}Mais il s'y trouvait un homme pauvre et sage qui délivra la ville par sa sagesse. Et personne ne s'est souvenu de cet homme pauvre.
\VS{16}Alors j'ai dit~: La sagesse vaut mieux que la force. Cependant, la sagesse du pauvre est méprisée, et ses paroles ne sont point écoutées.
\VS{17}Les paroles des sages doivent être écoutées plus paisiblement que le cri de celui qui domine parmi les insensés. 
\VS{18}Mieux vaut la sagesse que tous les instruments de guerre~; et un seul homme pécheur détruit beaucoup de bien.
\Chap{10}
\TextTitle{La sagesse vaut mieux que la folie}
\VerseOne{} Les mouches mortes font puer et fermenter les parfums du parfumeur~; et un peu de folie produit le même effet à l'égard de celui qui est estimé pour sa sagesse, et pour sa gloire.
\VS{2}Le cœur du sage est à sa droite, et le cœur de l'insensé est à sa gauche.
\VS{3}Et même quand l'insensé se met en chemin, le sens lui manque~; et il dit de chacun~: Il est insensé. 
\VS{4}Si l'esprit de celui qui domine s'élève contre toi, ne sors point de ta condition~; car la douceur fait pardonner de grandes fautes. 
\VS{5}Il y a un mal que j'ai vu sous le soleil, comme une erreur qui procède du prince~:
\VS{6}C'est que la folie est mise aux plus hauts lieux, et que les riches sont assis dans un lieu bas. 
\VS{7}J'ai vu des serviteurs sur des chevaux, et des princes marchant sur terre comme des serviteurs.
\VS{8}Celui qui creuse la fosse y tombera, et celui qui coupe la haie, le serpent le mordra\FTNT{Ps. 7:15~; Pr. 26:27~; Pr. 28:10.}.
\VS{9}Celui qui remue des pierres hors de leur place, en sera blessé, et celui qui fend du bois se met en danger.
\VS{10}Si le fer est émoussé, et qu'on n'en ait point aiguisé le tranchant, il devra redoubler de force~; mais la sagesse a l'avantage du succès.
\VS{11}Si le serpent mord sans faire du bruit, le médisant ne vaut pas mieux. 
\VS{12}Les paroles de la bouche du sage ne sont que grâce, mais les lèvres de l'insensé le réduisent à néant\FTNT{Pr. 10:21.}.
\VS{13}Le commencement des paroles de sa bouche est folie, et la fin de son discours est une méchante folie.
\VS{14}Or l'insensé multiplie les paroles. L'homme ne sait point ce qui arrivera, et qui lui déclarera ce qui sera après lui~?
\VS{15}Le travail de l'insensé le fatigue, parce qu'il ne sait pas aller à la ville.
\VS{16}Malheur à toi, pays dont le roi est un enfant, et dont les princes mangent dès le matin\FTNT{Es. 3:4.}~!
\VS{17}Que tu es béni, ô pays~! Si ton roi est de race illustre, et si tes gouverneurs mangent au temps convenable, pour leur réfection et non pour se livrer à la débauche~! 
\VS{18}A cause des mains paresseuses, la charpente s'affaisse~; et à cause des mains lâches, la maison a des gouttières.
\VS{19}On fait des pains pour se réjouir et le vin réjouit les vivants et l'argent répond à tout.
\VS{20}Ne maudis point le roi, même dans ta pensée, et ne maudis pas le riche dans la chambre où tu couches~; car l'oiseau du ciel emporterait ta voix, le Baal ailé\FTNT{Baal ailé~: Le terme sémitique «~baal~» (en hébreu ba'al) signifie à l'origine «~possesseur~», «~maître~» ou «~seigneur~». Le Baal ailé était une créature ailée. Utilisé au pluriel, l'expression «~baalim de flèches~» désignait des archers. Les écritures nous parlent de Baal-Zebub (seigneur des mouches), un démon adoré à Ekron, l'une des villes des Philistins (2 R. 1:1-16). Baal-Zebud à donné «~Béelzébul~» dans les Evangiles (Mt. 10:25~; Mt. 12:24~; Mt. 12:27~; Lu. 11:15-19). Ce passage nous enseigne clairement que les démons épient les enfants de Dieu et vont ensuite faire leurs rapports à Satan afin de mieux les attaquer. Ils agissent comme des espions. Ces esprits sont comme des mouches et essayent de s'infiltrer partout.} rapporterait tes paroles.
\Chap{11}
\TextTitle{L'homme travaille en tâtonnant}
\VerseOne{}Jette ton pain à la face des eaux, car avec le temps tu le retrouveras.
\VS{2}Donnes-en une part à sept et même à huit, car tu ne sais point quel mal viendra sur la terre.
\VS{3}Quand les nuages sont pleins, ils répandent la pluie sur la terre~; et quand un arbre tombe, au sud ou au nord, il reste à la place où il est tombé.
\VS{4}Celui qui prend garde au vent, ne sèmera point~; et celui qui regarde les nuées, ne moissonnera point. 
\VS{5}Comme tu ne sais point quel est le chemin du vent, ni comment se forment les os dans le ventre de celle qui est enceinte, ainsi tu ne connais pas l'œuvre de Dieu qui fait tout\FTNT{Ceux qui sont nés d'en-haut sont insaisissables comme le vent (Jn. 3:8).}.
\VS{6} Sème ta semence dès le matin, et ne laisse pas reposer tes mains le soir~; car tu ne sais point lequel sera le meilleur, ceci ou cela~; et si tous deux seront pareillement bons.
\VS{7}Il est vrai que la lumière est douce, et qu'il est agréable aux yeux de voir le soleil.
\VS{8}Mais si un homme vit de nombreuses années, qu'il se réjouisse, et qu'il se souvienne des jours de ténèbres qui seront en grand nombre, tout ce qui lui arrivera est vanité.
\Chap{12}
\TextTitle{Message à la jeunesse}
\VerseOne{}Jeune homme, réjouis-toi dans ton jeune âge, et que ton cœur te rende gai aux jours de ta jeunesse, et marche comme ton cœur te mène, et selon le regard de tes yeux~; mais sache que pour toutes ces choses Dieu t'amènera en jugement. 
\VS{2}Ôte le chagrin de ton cœur, et éloigne de toi le mal~; car le jeune âge et l'adolescence ne sont que vanité. 
\VS{3}Mais souviens-toi de ton Créateur pendant les jours de ta jeunesse, avant que les jours mauvais arrivent et que viennent les années où tu diras~: Je n'y prends point de plaisir~;
\VS{4}avant que le soleil et la lumière, la lune et les étoiles s'obscurcissent, et que les nuages reviennent après la pluie.
\VS{5}Lorsque les gardes de la maison tremblent\FTNT{«~Ceux qui gardent la maison~» représentent les mains.}, et que les hommes forts\FTNT{«~Les hommes forts~» sont les jambes.} se courbent, et que celles qui moulent\FTNT{Les dents sont celles qui moulent.} cessent de travailler parce qu'elles sont diminuées, et quand ceux qui regardent par les fenêtres\FTNT{Les yeux sont ceux qui regardent par les fenêtres.} sont obscurcis.
\VS{6}Et quand les deux battants de la porte\FTNT{Les oreilles sont les deux battants de la porte.} se ferment sur la rue quand s'abaisse le bruit de la meule, quand on se lève au chant de l'oiseau, et que toutes les chanteuses s'affaiblissent.
\VS{7}Quand aussi on craint ce qui est élevé, et qu'on tremble en chemin, quand l'amandier fleurit, et quand les cigales deviennent pesantes, et que l'appétit s'en ira, car l'homme s'en va vers sa maison éternelle\FTNT{La maison éternelle c'est la Nouvelle Jérusalem pour les chrétiens (Ap. 21.) et pour les païens le lac de feu (Ap. 20:11-15).}, et ceux qui pleurent font le tour des rues.
\VS{8}Avant que la corde d'argent\FTNT{Cette corde est comme le cordon ombilical, elle lie l'âme au corps. Lors de la mort, la corde d'argent est coupée.} se détache, que le vase d'or\FTNT{Le corps humain est comme un vase ou une tente qui renferme son esprit. Comme l'argile dans la main du potier, ainsi est l'homme dans celle de Dieu. Avec cette argile, il décide souverainement de fabriquer de la même masse un vase d'honneur et un autre pour un usage vil (Jé. 18:4-6~; Ro. 9:21~; 2 Ti. 2:20-21).} se brise, que la cruche se rompe sur la source, que la roue s'écrase sur la citerne~;
\VS{9}avant que la poussière retourne dans la terre, comme elle y avait été, et que l'esprit retourne à Dieu qui l'a donné.
\TextTitle{Conclusion}
\VS{10}Vanité des vanités, dit l'Ecclésiaste, tout est vanité.
\VS{11}Plus l'Ecclésiaste a été sage, plus il a enseigné la science au peuple~; il a fait entendre, il a recherché et mis en ordre plusieurs graves sentences. 
\VS{12}L'Ecclésiaste a cherché pour trouver des discours agréables~; mais ce qui en a été écrit ici, est la droiture même~; ce sont des paroles de vérité. 
\VS{13}Les paroles des sages sont comme des aiguillons, et les maîtres qui en ont fait des recueils, sont comme des clous plantés, et ces choses ont été données par un maître.
\VS{14}Mon fils, garde-toi de ce qui est au-delà de ceci~; car il n'y a point de fin à faire plusieurs livres, et tant d'étude n'est que travail qu'on se donne. 
\VS{15}Voici la conclusion de tout le discours qui a été entendu~: Crains Dieu, et garde ses commandements~; car c'est là le tout de l'homme.
\VS{16}Parce que Dieu amènera toute œuvre en jugement, au sujet de tout ce qui est caché, soit bien, soit mal.
\PPE{}
\end{multicols}

%\clearpage\ShortTitle{Esther}\BookTitle{Esther}\BFont
\noindent\hrulefill
{\footnotesize
\textit{
\bigskip
{\centering{}
\\(Ecter)
\\Signifie : Etoile (persan) ; Myrthe (hébreu)
\\Thème : Délivrance des juifs de l’extermination
\\Auteur : Inconnu
\\Date de rédaction : 5ème siècle av. J.C.\\}
}
%\bigskip
\textit{
\\Dernier livre à caractère historique du Tanahk, l’histoire d’Esther se déroula à Suse, capitale du royaume de Perse. En ce temps, le peuple d’Israël était dispersé et le roi Assuérus régnait sur un large territoire allant de l’Inde à l'Ethiopie.
%\bigskip
\\Ce livre raconte la vie d’Esther, son ascension au trône royal où elle succéda à la reine Vasthi et la manière dont elle fut utilisée pour éviter le génocide du peuple juif.  
%\bigskip
\\Bien que ne comportant pas le nom de Dieu ni d’allusion à une œuvre spirituelle, hormis le jeûne, ce récit met en évidence le secours divin.\bigskip
}
}
\par\nobreak\noindent\hrulefill
\begin{multicols}{2}
\TextTitle{[Un festin de sept jour au palais de Suse]}
\Chap{1}
\VerseOne{}Or il arriva qu’au temps d’Assuérus, de cet Assuérus qui régnait depuis les Indes jusqu'en Ethiopie, sur cent vingt-sept provinces ;
\VS{2}[Il arriva, dis-je], en ce temps-là, que le roi Assuérus était assis sur le trône royal à Suse, dans la capitale.
\VS{3}La troisième année de son règne, il fit un festin à tous les principaux princes de ses pays ; à ses serviteurs, à l’armée des Perses et de Mèdes, aux nobles et aux chefs des provinces qui furent réunis devant lui,
\VS{4}pour leur montrer la gloire de la richesse de son royaume et la splendeur de sa grandeur, durant plusieurs jours, pendant cent quatre-vingts jours.
\VS{5}Lorsque ces jours furent achevés, le roi fit pour tout le peuple qui se trouvait à Suse, la capitale, depuis le plus grand jusqu'au plus petit, un festin pendant sept jours, dans la cour du jardin du palais royal.
\VS{6}Des étoffes blanches, vertes et violettes, étaient attachées par des cordons de byssus et de pourpre à des anneaux d'argent et à des colonnes de marbre. Les lits étaient d'or et d'argent sur un pavé de porphyre, de marbre, de nacre, et de pierres noires.
\VS{7}On servait à boire dans des vases d'or, de différentes espèces, et il y avait du vin royal en abondance, selon la libéralité du roi.
\VS{8}On ne forçait personne à boire, car le roi avait ordonné à tous les chefs de sa maison de se conformer à la volonté de chacun.
\VS{9}La reine Vasthi fit aussi un festin aux femmes dans la maison royale du roi Assuérus.
\TextTitle{[Destitution de la reine Vasthi]}
\VS{10}Or le septième jour, le cœur du roi était réjoui par le vin, il ordonna à Mehuman, Biztha, Harbona, Bigtha, Abagtha, Zéthar, et Carcas, les sept eunuques qui servaient devant le roi Assuérus,
\VS{11}d’amener en sa présence la reine Vasthi, portant la couronne royale, afin de montrer sa beauté aux peuples et aux princes, car elle était belle de figure.
\VS{12}Les eunuques transmirent l’ordre du roi à la reine Vasthi, mais elle refusa de venir. Et le roi fut très irrité, et il s’enflamma de colère.
\VS{13}Alors le roi dit aux sages qui avaient la connaissance des temps. Car le roi traitait ainsi les affaires en présence de tous ceux qui connaissaient les lois et le droit.
\VS{14}Il avait auprès de lui, Carschena, Schéthar, Admatha, Tarsis, Mérès, Marsena, Memucan, sept princes de Perse et de Médie, qui voyaient la face du roi et qui occupaient le premier rang dans le royaume.
\VS{15}Que faut-il faire dit-il, selon les lois, à la reine Vasthi, pour n'avoir pas observé l’ordre que le roi Assuérus lui a ordonné par les eunuques ?
\VS{16}Alors Memucan répondit en présence du roi et des princes : La reine Vasthi n'a pas seulement mal agi contre le roi, mais aussi contre tous les princes et tous les peuples qui sont dans toutes les provinces du roi Assuérus.
\VS{17}Car l'action de la reine parviendra à la connaissance de toutes les femmes, et les portera à mépriser leurs maris ; elles diront : Le roi Assuérus avait ordonné qu'on fasse venir en sa présence la reine, et elle n'y est pas allée.
\VS{18}Dès ce jour, les princesses de Perse et de Médie qui auront appris l’action de la reine répondront de même à tous les princes du roi ; ce sera une marque de mépris et un sujet de colère.
\VS{19}Si le roi le trouve bon, qu'un édit royal soit publié de sa part, et qu'il soit écrit parmi les lois de Perse et de Médie, avec défense de la transgresser, que Vasthi ne vienne plus devant le roi Assuérus et le roi donnera sa royauté à une compagne, qui sera meilleure qu'elle.
\VS{20}L’édit du roi sera présenté et connu dans tout son royaume, quelque grand qu'il soit, toutes les femmes honoreront leurs maris\FTNT{Respect ou soumission de la femme à l’égard de son mari : Ep. 5 : 22 ; Col. 3 : 18 ; Ti. 2 : 5 ; 1 Pi. 3 : 1-5.}, depuis le plus grand jusqu'au plus petit.
\VS{21}Cette parole plut au roi et aux princes, et le roi fit selon la parole de Memucan.
\VS{22}Il envoya des lettres à toutes les provinces du royaume, à chaque province selon son écriture et à chaque peuple selon sa langue ; elles portaient que tout homme devait être le maître de sa maison\FTNT{L’homme, chef de la femme et maître de la maison : 1 Co. 11 : 3 ; Ep. 5 : 23.}, et qu’il parlerait la langue de son peuple.
\TextTitle{[Le roi choisit une autre reine]}
\Chap{2}
\VerseOne{}Après ces choses, quand la colère du roi Assuérus fut calmée, il se souvint de Vasthi, de ce qu'elle avait fait, et de ce qui avait été décrété à son sujet.
\VS{2}Les serviteurs qui servaient le roi dirent : Qu'on cherche pour le roi des jeunes filles, vierges, et belles de figure.
\VS{3}Que le roi désigne des commissaires dans toutes les provinces de son royaume chargés de rassembler toutes les jeunes filles, vierges et belles de figure, dans Suse, la capitale, dans la maison des femmes sous la charge d'Hégué, eunuque du roi et gardien des femmes, qu'on leur donne les parfums nécessaires pour leur toilette ;
\VS{4}et la jeune fille qui plaira au roi régnera à la place de Vasthi. Ce discours plût au roi, et il fit ainsi.
\VS{5}Or, il y avait à Suse, la capitale, un juif nommé Mardochée, fils de Jaïr, fils de Schimeï, fils de Kis, Benjamite,
\VS{6}qui avait été emmené de Jérusalem\FTNT{La captivité babylonienne : Voir 2 R. 24.}, parmi les captifs déportés avec Jeconia, roi de Juda, par Nebucadnetsar, roi de Babylone.
\VS{7}Il élevait Hadassa, qui est Esther, fille de son oncle ; car elle n'avait ni père ni mère. La jeune fille était belle de taille et très belle de figure. Après la mort de son père et de sa mère, Mardochée l'avait prise pour fille.
\VS{8}Lorsqu’on eut publié l’ordre du roi et son édit, un grand nombre de jeunes filles furent rassemblées à Suse, la capitale, sous la charge d'Hégaï. Esther fut aussi amenée dans la maison du roi, sous la charge d'Hégaï, gardien des femmes.
\VS{9}La jeune fille lui plut, et trouva grâce à ses yeux, il s’empressa de lui fournir les parfums nécessaires pour sa toilette, et pour sa subsistance, lui donna sept jeunes filles choisies, et établies dans la maison du roi, il lui fit changer d'appartement, et la logea, elle et ses servantes, dans le meilleur des appartements de la maison des femmes.
\VS{10}Esther ne fit connaître ni son peuple ni sa parenté, car Mardochée lui avait ordonné de ne rien raconter.
\VS{11}Tous les jours Mardochée allait et venait devant la cour de la maison des femmes, pour savoir comment se portait Esther, et comment on s’occupait d'elle.
\VS{12}Chaque jeune fille allait à son tour vers le roi Assuérus, après s’être conformée au décret concernant les femmes pendant douze mois\FTNT{Esther se soumit à une toilette particulière avant de rencontrer le roi. Le mot « toilette » vient de l’hébreu « tam-rook », qui signifie « grattement ». La racine de ce mot signifie « nettoyer », « purifier », « polir » (voir Lé. 6 : 28 ; Jé. 46 : 4). Ce grattage symbolise le dépouillement du vieil homme et le renoncement aux œuvres de la chair (Ep. 4 : 22).
Douze mois étaient nécessaires pour préparer Esther aux noces : Six mois avec de l’huile de myrrhe et six mois avec des aromates et des parfums. La myrrhe était l’une des composantes de l’onction sainte dont on s’est servi pour oindre notamment la tente d’assignation, l’arche du témoignage ainsi qu’Aaron et ses fils (Ex. 30 : 23-30). Cet aspect de la toilette d’Esther nous parle de la sanctification sans laquelle nul ne peut voir le Seigneur (Hé. 12 : 14). La myrrhe est par ailleurs citée à sept reprises dans le livre du Cantique des cantiques, véritable hymne de l’amour parfait qui lie Christ à son Eglise. Le parfum quant à lui symbolise les prières que nous devons faire en tout temps afin de maintenir notre communion avec Jésus, notre époux (Ap. 5 : 8 ; Ap. 8 : 4 ; Ep. 6 : 18 ; 1 Th. 5 : 17).
Ainsi, à l’instar d’Esther qui se préparait à rencontrer le roi, l’Eglise se prépare depuis deux mille ans pour les noces de l’agneau (Ap. 19 : 7-9).}. C'est ainsi que s'accomplissaient les jours de leurs préparatifs, six mois avec de l'huile de myrrhe, et six autres mois avec des aromates et des parfums en usage parmi les femmes.
\VS{13}C'est ainsi que la jeune fille entrait vers le roi ; et, quand elle passait de la maison des femmes à la maison du roi, on lui laissait prendre ce qu’elle voulait.
\VS{14}Elle y entrait le soir, et le matin elle retournait dans la seconde maison des femmes sous la charge de Schaaschgaz, eunuque du roi et gardien des concubines. Elle ne retournait plus vers le roi, à moins que le roi n’en ait le désir et qu'elle soit appelée par son nom.
\TextTitle{[Esther, reine de Suse]}
\VS{15}Quand son tour d’aller vers le roi fut arrivé, Esther, fille d'Abichaïl, oncle de Mardochée qui l’avait prise pour sa fille, ne demanda rien sinon ce qui fut ordonné par Hégaï, eunuque du roi et gardien des femmes. Esther trouva grâce aux yeux de tous ceux qui la voyaient.
\VS{16}Ainsi Esther fut amenée auprès du roi Assuérus, dans sa maison royale, le dixième mois, qui est le mois de Tébeth, la septième année de son règne.
\VS{17}Le roi aima Esther plus que toutes les autres femmes, elle obtint sa grâce et sa bienveillance plus que toutes les vierges. Il mit la couronne royale sur sa tête, et l'établit reine à la place de Vasthi.
\VS{18}Le roi fit alors un grand festin à tous les princes de ses pays, et à ses serviteurs, un festin en l’honneur d'Esther ; il donna du repos aux provinces, et fit des présents selon la puissance du roi.
\VS{19}Or pendant qu'on assemblait les vierges pour la seconde fois, Mardochée s’assit à la porte du roi.
\VS{20}Esther n’avait fait connaître ni sa parenté ni son peuple, car Mardochée le lui avait défendu. Elle faisait tout ce que lui disait Mardochée, comme à l’époque où elle était élevée par lui.
\TextTitle{[Mardochée sauve la vie du roi]}
\VS{21}En ces jours-là, Mardochée s’assit à la porte du roi, Bigthan et Théresch, deux eunuques du roi, gardes du seuil, s’irritèrent et cherchèrent à mettre la main sur le roi Assuérus.
\VS{22}Mardochée ayant eu connaissance de l’affaire, informa la reine Esther, qui le redit au roi de la part de Mardochée.
\VS{23}On vérifia l’affaire et on trouva que cela était exact, les deux eunuques furent pendus à un bois, et cela fut écrit dans le livre des chroniques en présence du roi.
\TextTitle{[Conspiration de Haman contre les Juifs]}
\Chap{3}
\VerseOne{}Après ces choses, le roi Assuérus fit de grands honneurs à Haman, fils d'Hammedatha, l’Agaguite ; il l'éleva en dignité et plaça son siège au-dessus de tous les princes qui étaient auprès de lui.
\VS{2}Tous les serviteurs du roi qui étaient à la porte du roi s'inclinaient et se prosternaient devant Haman, car le roi l’avait ainsi ordonné. Mais Mardochée ne s'inclinait pas et ne se prosternait pas devant lui.
\VS{3}Les serviteurs du roi, qui étaient à la porte du roi, disaient à Mardochée : Pourquoi transgresses-tu l’ordre du roi ?
\VS{4}Comme ils le lui répétaient chaque jour et qu'il ne les écoutait pas, ils le rapportèrent à Haman, pour voir si Mardochée tiendrait ferme dans sa résolution ; car il leur avait déclaré qu'il était juif.
\VS{5}Haman vit que Mardochée ne s'inclinait pas et ne se prosternait pas devant lui et il fut rempli de colère.
\VS{6}Mais il dédaigna de porter la main sur Mardochée seul, car on lui avait rapporté de quel peuple était Mardochée. Haman chercha à exterminer tous les juifs, le peuple de Mardochée qui se trouvait dans tout le royaume d'Assuérus.
\VS{7}Au premier mois, qui est le mois de Nissan, la douzième année du roi Assuérus, on jeta le pur, c'est-à-dire le sort, devant Haman, pour chaque jour et pour chaque mois, jusqu’au douzième mois, qui est le mois d'Adar.
\VS{8}Haman dit au roi Assuérus : Il y a un peuple dispersé dans toutes les provinces de ton royaume, qui se tient à part parmi les peuples. Leurs lois sont différentes de celles de tous les autres peuples, ils n’observent pas les lois du roi. Il n'est pas dans l’intérêt du roi de le laisser en repos.
\VS{9}S'il plaît au roi, qu'on écrive l’ordre de les faire périr, et je pèserai dix mille talents d'argent entre les mains de ceux qui s’occupent des affaires, pour les porter dans le trésor du roi.
\VS{10}Le roi ôta son anneau de sa main, et le donna à Haman fils de Hammedatha, l’Agaguite, l’adversaire des Juifs.
\VS{11}Outre cela, le roi dit à Haman : Cet argent t'est donné avec ce peuple ; fais-en ce que tu voudras.
\VS{12}Le treizième jour du premier mois, les secrétaires du roi furent appelés, et on écrivit selon l’ordre d'Haman, aux satrapes du roi, aux gouverneurs de chaque province et aux princes de chaque peuple, à chaque province selon son écriture et à chaque peuple selon sa langue. Ce fut au nom du roi Assuérus que l’on écrivit, et on scella avec l'anneau du roi.
\VS{13}Les lettres furent envoyées par des coureurs dans toutes les provinces du roi, afin qu'on extermine, qu’on tue et qu’on fasse périr tous les juifs, jeunes et vieux, petits enfants et femmes, en un seul jour, le treizième du douzième mois, qui est le mois d'Adar, et pour que leurs biens soient livrés au pillage.
\VS{14}Ces lettres qui furent écrites portaient une copie de l’édit, qui devait être publié dans chaque province, et invitaient publiquement tous les peuples, à se tenir prêts pour ce jour-là.
\VS{15}Ainsi les coureurs partirent en toute hâte d’après l’ordre du roi. L'édit fut aussi publié dans Suse, la capitale. Or le roi et Haman étaient assis pour boire, pendant que la ville de Suse était dans la confusion.
\TextTitle{[Esther avertie du complot d'Haman]}
\Chap{4}
\VerseOne{}Mardochée, ayant appris ce qui se passait, déchira ses vêtements et se couvrit d'un sac et de la cendre. Puis il alla au milieu de la ville en poussant avec force des cris amers,
\VS{2}et se rendit jusqu'à la porte du roi, or il était interdit d'entrer dans le palais du roi revêtu d'un sac.
\VS{3}Dans chaque province, partout où arrivait l’ordre du roi et son édit, il y eut une grande désolation parmi les juifs ; ils jeûnaient, pleuraient, gémissaient, et beaucoup se couchaient sur le sac et la cendre.
\VS{4}Les servantes d'Esther et ses eunuques vinrent lui raconter ces choses, et la reine fut très effrayée. Elle envoya des vêtements à Mardochée pour le couvrir et lui faire ôter son sac, mais il ne les prit pas.
\VS{5}Alors Esther appela Hathac, l'un des eunuques que le roi avait établi pour la servir, et elle le chargea de demander à Mardochée ce qui s’était passé et pourquoi il agissait ainsi.
\VS{6}Hathac sortit donc vers Mardochée sur la place de la ville, devant la porte du roi.
\VS{7}Mardochée lui raconta tout ce qui lui était arrivé, et la somme d'argent qu'Haman avait promis de payer comptant au trésor du roi, pour la destruction des juifs.
\VS{8}Il lui donna aussi une copie de l'édit publié dans Suse en vue de leur extermination, afin qu’il le montre à Esther et lui fasse tout connaître ; et il ordonna qu’Esther se rende chez le roi pour implorer sa miséricorde, et faire une requête en faveur de son peuple.
\VS{9}Hathac vint rapporter à Esther les paroles de Mardochée.
\TextTitle{[Mardochée incite Esther à risquer sa vie pour ses frères]}
\VS{10}Esther chargea Hathac de dire à Mardochée :
\VS{11}Tous les serviteurs du roi et le peuple des provinces du roi savent qu'il existe une loi prescrivant la peine de mort contre quiconque, homme ou femme, entre chez le roi, dans la cour intérieure sans avoir été appelé ; à moins que le roi ne lui tende le sceptre d'or, celui-là a la vie sauve. Or il y a déjà trente jours que je n'ai pas été appelée pour entrer chez le roi.
\VS{12}On rapporta les paroles d'Esther à Mardochée.
\VS{13}Mardochée fit cette réponse à Esther : Ne t’imagine pas que tu échapperas seule d'entre tous les juifs parce que tu es dans la maison du roi.
\VS{14}Mais si tu te tais et gardes le silence en ce temps-ci, les juifs seront secourus et délivrés par un autre moyen, mais toi et la maison de ton père vous périrez. Et qui sait si tu n'es pas arrivée à la royauté pour un temps comme celui-ci ?
\TextTitle{[Esther demande un jeûne]}
\VS{15}Esther fit cette réponse à Mardochée :
\VS{16}Va, rassemble tous les juifs qui se trouvent à Suse, et jeûnez pour moi, sans manger ni boire pendant trois jours, ni la nuit ni le jour. Moi aussi et mes servantes nous jeûnerons de même, puis j'entrerai chez le roi, malgré la loi ; et si je dois périr, je périrai.
\VS{17}Mardochée s'en alla, et fit comme Esther lui avait ordonné.
\TextTitle{[Esther se présente devant le roi]}
\Chap{5}
\VerseOne{}Le troisième jour, Esther mit des vêtements royaux et se présenta dans la cour intérieure de la maison du roi, devant la maison du roi. Le roi était assis sur le trône dans la maison royale, en face de l’entrée de la maison.
\VS{2}Dès que le roi vit la reine Esther debout dans la cour, elle trouva grâce à ses yeux ; le roi tendit à Esther le sceptre d'or qui était dans sa main. Esther s'approcha, et toucha le bout du sceptre.
\VS{3}Le roi lui dit : Qu'as-tu, reine Esther, et que demandes-tu ? Quand ce serait la moitié du royaume, elle te serait donnée.
\VS{4}Esther répondit : Si le roi le trouve bon, que le roi vienne aujourd'hui avec Haman au festin que je lui ai préparé.
\VS{5}Alors le roi dit : Qu'on fasse venir en toute hâte, Haman, pour accomplir la parole d'Esther. Le roi vint donc avec Haman au festin qu'Esther avait préparé.
\VS{6}Le roi dit à Esther, pendant qu’on buvait le vin : Quelle est ta demande ? Elle te sera accordée. Quelle est ta requête ? Quand ce serait la moitié du royaume, tu l’obtiendras.
\VS{7}Esther répondit et dit : Voici ce que je demande et ce que je désire.
\VS{8}Si j'ai trouvé grâce aux yeux du roi, et si le roi trouve bon d'accorder ma requête, que le roi et Haman viennent au festin que je leur préparerai, et je donnerai demain une réponse au roi selon sa parole.
\VS{9}Haman sortit ce jour-là, joyeux et le cœur content. Mais aussitôt qu'il vit, à la porte du roi, Mardochée, qui ne se levait ni ne tremblait devant lui, il fut rempli de colère contre Mardochée.
\VS{10}Il sut toutefois se contenir, et il alla dans sa maison. Puis il envoya chercher ses amis et Zéresch, sa femme.
\VS{11}Haman leur parla de la magnificence de ses richesses, du nombre de ses fils, et tout ce qu’avait fait le roi pour le rendre puissant, et comment il l'avait élevé au-dessus des princes et des serviteurs du roi.
\VS{12}Puis Haman ajouta : Même la reine Esther n'a fait venir que moi et le roi au festin qu'elle a fait, et je suis encore invité demain chez elle avec le roi.
\VS{13}Mais tout cela n’est d’aucun intérêt, aussi longtemps que je verrai Mardochée, le juif, assis à la porte du roi.
\VS{14}Zéresch sa femme, et tous ses amis lui répondirent : Qu'on prépare un bois haut de cinquante coudées, et demain matin dis au roi qu'on y pende Mardochée ; et tu iras joyeux au festin avec le roi. Cette parole plut à Haman, et il fit préparer le bois.
\TextTitle{[Le roi Assuérus se souvient de Mardochée]}
\Chap{6}
\VerseOne{}Cette nuit-là, le roi ne put dormir, il fit apporter le livre des annales, les chroniques. On les lut devant le roi,
\VS{2}et l’on trouva écrit ce que Mardochée avait rapporté au sujet de la conspiration de Bigthan et de Théresch, les deux eunuques du roi, gardes du seuil, qui avaient cherché à mettre la main sur le roi Assuérus.
\VS{3}Le roi dit : Quel honneur et quelle distinction a-t-on accordé à Mardochée pour cela ? Il n’a rien reçu répondirent les serviteurs du roi.
\VS{4}Le roi dit : Qui est dans la cour ? Haman était venu dans la cour extérieure de la maison du roi, pour demander au roi de pendre Mardochée au bois qu'il avait préparé.
\VS{5}Les serviteurs du roi répondirent : C’est Haman qui se tient dans la cour. Et le roi dit : Qu'il entre.
\VS{6}Haman entra, et le roi lui dit : Que faudrait-il faire à un homme que le roi désire honorer ? Haman se dit en lui-même : A qui le roi voudrait-il faire plus honneur qu'à moi ?
\VS{7}Haman répondit au roi : Pour un homme que le roi désire honorer,
\VS{8}qu’on lui apporte le vêtement royal, dont le roi se revêt, et qu'on lui amène le cheval que le roi monte, et qu'on lui mette la couronne royale sur la tête.
\VS{9}Et qu'ensuite on donne ce vêtement et ce cheval à quelqu'un des principaux et des plus grands chefs qui sont auprès du roi, et qu'on revête l'homme que le roi prend plaisir d'honorer, et qu'on le fasse aller à cheval par les rues de la ville ; et qu'on crie devant lui : C'est ainsi qu'on doit faire à l'homme que le roi prend plaisir d'honorer.
\VS{10}Alors le roi dit à Haman : Prends tout de suite le vêtement, et le cheval, comme tu l'as dit, et fais ainsi à Mardochée, le juif qui est assis à la porte du roi ; ne néglige rien de tout ce que tu as déclaré.
\VS{11}Et Haman prit le vêtement et le cheval, il revêtit Mardochée, il le promena à cheval à travers les rues de la ville, et il criait devant lui : C'est ainsi que l’on fait à l'homme que le roi désire honorer.
\VS{12}Mardochée retourna à la porte du roi, et Haman se retira en hâte dans sa maison, pleurant et ayant la tête voilée.
\VS{13}Haman raconta à Zéresch, sa femme, et à tous ses amis, tout ce qui lui était arrivé. Ses sages, et Zéresch, sa femme, lui répondirent : Si Mardochée devant lequel tu as commencé à tomber, est de la race des juifs, tu n'auras pas le dessus sur lui, mais tu tomberas certainement devant lui.
\VS{14}Comme ils parlaient encore avec lui, les eunuques du roi vinrent, et se hâtèrent d'amener Haman au festin qu'Esther avait préparé.
\TextTitle{[Esther plaide sa cause et celle de son peuple]}
\Chap{7}
\VerseOne{}Le roi et Haman allèrent au festin chez la reine Esther.
\VS{2}Le roi dit encore à Esther, ce second jour, pendant qu’on buvait le vin : Quelle est ta demande, reine Esther ? Elle te sera donnée. Que désires-tu ? Quand ce serait la moitié du royaume, cela te sera accordé.
\VS{3}Alors la reine Esther répondit, et dit : Si j'ai trouvé grâce à tes yeux, ô roi ! et si le roi le trouve bon, que ma vie me soit donnée à ma demande, et que mon peuple me soit donné à ma prière.
\VS{4}Car nous avons été vendus, mon peuple et moi, pour être détruits, tués, exterminés. Si nous avions été vendus pour être esclaves et serviteurs, j’aurais gardé le silence, bien que l'oppresseur ne saurait compenser le dommage fait au roi.
\VS{5}Le roi Assuérus parla et dit à la reine Esther : Qui est-il et où est l’homme dont le cœur est consacré à faire cela ?
\VS{6}Esther répondit : L'oppresseur, l'ennemi, c’est Haman, ce méchant ! Alors Haman fut terrifié en présence du roi et de la reine.
\TextTitle{[Haman pendu au gibet qu'il avait dressé]}
\VS{7}Le roi, dans sa colère, se leva et quitta le festin, il entra dans le jardin du palais. Haman resta pour demander grâce pour sa vie à la reine Esther, car il voyait bien que sa perte était résolue par le roi.
\VS{8}Puis le roi revint du jardin du palais dans la salle du festin, il vit Haman qui s’était précipité sur le lit où était Esther, et il dit : Serait-ce encore pour faire violence sous mes yeux à la reine dans cette maison ? Dès que la parole fut sortie de la bouche du roi, on voila le visage d'Haman.
\VS{9}Et Harbona, l'un des eunuques, dit en présence du roi : Voici, le bois préparé par Haman pour Mardochée, qui a parlé pour le bien du roi, est dressé dans la maison d'Haman, à une hauteur de cinquante coudées. Le roi dit : Qu’on y pende Haman !
\VS{10}On pendit Haman au bois qu'il avait préparé pour Mardochée. Et la colère du roi fut apaisée.
\TextTitle{[Un décret royal fait échouer le complot d'Haman]}
\Chap{8}
\VerseOne{}Ce même jour, le roi Assuérus donna à la reine Esther la maison d'Haman, l'oppresseur des juifs ; et Mardochée fut introduit devant le roi, car Esther avait déclaré quel était son lien de parenté avec elle.
\VS{2}Le roi ôta son anneau, qu'il avait repris à Haman, et le donna à Mardochée ; Esther établit Mardochée sur la maison d'Haman.
\VS{3}Esther parla encore en présence du roi. Elle se jeta à ses pieds, elle pleura, elle l’implora d’empêcher les effets de la méchanceté d'Haman, l’Agaguite, et la réussite de ses projets contre les juifs.
\VS{4}Le roi tendit le sceptre d'or à Esther qui se releva et resta debout devant le roi.
\VS{5}Elle dit : Si le roi le trouve bon, et si j'ai trouvé grâce devant lui, si mes paroles semblent convenables au roi et si je suis agréable à ses yeux, qu'on écrive pour révoquer les lettres conçues par Haman, fils d'Hammedatha, l’Agaguite, qu'il écrivit afin de détruire les juifs qui sont dans toutes les provinces du roi.
\VS{6}Car comment pourrais-je voir le mal qui atteindrait mon peuple, et comment pourrais-je voir la destruction de ma race ?
\VS{7}Le roi Assuérus dit à la reine Esther et au juif Mardochée : Voici, j'ai donné la maison d'Haman à Esther, et il a été pendu au bois pour avoir étendu sa main contre les juifs.
\VS{8}Ecrivez donc, au nom du roi, en faveur des juifs comme il vous plaira, et scellez l'écrit de l'anneau du roi ; car un édit écrit au nom du roi et scellé de l'anneau du roi ne peut être révoqué.
\VS{9}En ce temps, le vingt-troisième jour du troisième mois, qui est le mois de Sivan, les secrétaires du roi furent appelés, et on écrivit, comme Mardochée l’ordonna, aux juifs, aux satrapes, aux gouverneurs, et aux princes des cent vingt-sept provinces, de l’Inde jusqu'en Ethiopie, à chaque province selon son écriture, à chaque peuple selon sa langue, et aux juifs selon leur écriture et selon leur langue.
\VS{10}On écrivit les lettres au nom du roi Assuérus, et on les scella de l'anneau du roi. On les envoya par des coureurs, ayant pour montures des chevaux et des mulets nés de juments.
\VS{11}Par ces lettres, le roi accordait aux juifs, qui étaient dans chaque ville la permission de se rassembler et de défendre leur vie, de détruire, de tuer, et d’exterminer toute force armée du peuple et de quelque province que ce soit, qui prendraient les armes pour les attaquer, ainsi que leurs petits enfants et leurs femmes, et de piller leurs biens ;
\VS{12}et cela en un seul jour, dans toutes les provinces du roi Assuérus, le treizième jour du douzième mois, qui est le mois d'Adar.
\VS{13}Ces lettres écrites portaient une copie de l’édit qui devait être publié dans chaque province, et informaient tous les peuples que les juifs seraient prêts en ce jour à se venger de leurs ennemis.
\VS{14}Les coureurs, montés sur des chevaux et des mulets, partirent aussitôt et en toute hâte, d’après l’ordre du roi. L'édit fut aussi publié dans Suse, la capitale.
\TextTitle{[Mardochée honoré]}
\VS{15}Mardochée sortit de chez le roi, en vêtement royal violet et blanc, avec une grande couronne d'or, et une robe de byssus et de pourpre. La ville de Suse poussait des cris, et elle fut dans la joie.
\VS{16}Il eut pour les juifs du bonheur et de la joie, des réjouissances et des honneurs.
\VS{17}Dans chaque province et dans chaque ville, partout où arrivaient l’ordre du roi et son décret, il y eut pour les juifs de la joie, des réjouissances, des festins, et des fêtes. Et beaucoup de gens d'entre les peuples du pays se faisaient juifs, parce que la crainte des juifs les avait saisis.
\TextTitle{[Les juifs triomphent de leurs ennemis]}
\Chap{9}
\VerseOne{}Le douzième mois, qui est le mois d'Adar, le treizième jour du mois, où l’ordre du roi et son décret devaient être exécutés, au jour où les ennemis des juifs espéraient dominer, ce fut le contraire qui arriva, les juifs dominèrent sur leurs ennemis.
\VS{2}Les juifs se rassemblèrent dans leurs villes, dans toutes les provinces du roi Assuérus, pour mettre la main sur ceux qui cherchaient leur perte ; et personne ne put leur résister, car la crainte qu'on avait d'eux avait saisi tous les peuples.
\VS{3}Et tous les princes des provinces, les satrapes, les gouverneurs, et ceux qui s’occupaient des affaires du roi, soutenaient les juifs, à cause de la terreur que leur inspirait Mardochée.
\VS{4}Car Mardochée était puissant dans la maison du roi, et sa renommée se répandait dans toutes les provinces, parce qu’il devenait de plus en plus puissant.
\VS{5}Les juifs frappèrent tous leurs ennemis à coups d'épée, ils les tuèrent et les détruisirent ; ils traitèrent selon leurs désirs ceux qui les haïssaient.
\VS{6}Dans Suse, la capitale, les juifs tuèrent et firent périr cinq cents hommes.
\VS{7}Ils tuèrent aussi Parschandatha, Dalphon, Aspatha,
\VS{8}Poratha, Adalia, Aridatha,
\VS{9}Parmaschtha, Arizaï, Aridaï, et Vajezatha,
\VS{10}les dix fils d'Haman, fils d'Hammedatha, l'oppresseur des juifs. Mais ils ne mirent pas leurs mains au pillage.
\VS{11}Ce jour-là, on rapporta au roi le nombre de ceux qui avaient été tués dans Suse, la capitale.
\VS{12}Le roi dit à la reine Esther : Dans Suse, la capitale, les juifs ont tué et détruit cinq cents hommes, et les dix fils d'Haman, qu'auront-ils fait dans le reste des provinces du roi ? Quelle est ta demande ? Et elle te sera accordée. Que désires-tu encore ? Tu l’obtiendras.
\VS{13}Esther répondit : Si le roi le trouve bon qu'il soit permis aux juifs, qui sont à Suse, d’agir encore demain selon le décret d’aujourd'hui, et que l'on pende au bois les dix fils d'Haman.
\VS{14}Et le roi ordonna de faire ainsi. L'édit fut publié dans Suse. On pendit les dix fils d'Haman ;
\VS{15}et les juifs qui étaient dans Suse se rassemblèrent encore le quatorzième jour du mois d'Adar et tuèrent dans Suse trois cents hommes. Mais ils ne mirent pas la main au pillage.
\VS{16}Les autres juifs qui étaient dans les provinces du roi se rassemblèrent, et défendirent leur vie ; ils eurent du repos et furent délivrés de leurs ennemis, et ils tuèrent soixante-quinze mille hommes de ceux qui les haïssaient. Mais ils ne mirent pas la main au pillage.
\VS{17}Ces choses arrivèrent le treizième jour du mois d'Adar, et le quatorzième du même mois ils se reposèrent, et ils en firent un jour de festin et de joie.
\VS{18}Les juifs qui étaient dans Suse, s'assemblèrent le treizième et le quatorzième jour du même mois, et ils se reposèrent le quinzième jour, et ils en firent un jour de festin et de joie.
\VS{19}C'est pourquoi les juifs des campagnes qui habitent dans des villes sans murailles, font le quatorzième jour du mois d'Adar, un jour de réjouissance, de festin et de fête, où l’on s’envoie des portions les uns aux autres.
\TextTitle{[Esther confirme l'instauration la fête des Purim]}
\VS{20}Mardochée écrivit ces choses, et il envoya les lettres à tous les juifs qui étaient dans toutes les provinces du roi Assuérus, auprès et au loin.
\VS{21}Il leur prescrivait de célébrer chaque année le quatorzième jour et le quinzième jour du mois d'Adar.
\VS{22}Comme les jours où les juifs avaient obtenu du repos en se délivrant de leurs ennemis, de célébrer le mois où leur angoisse fut changée en joie, et leur deuil en jour heureux, et de faire de ces jours des jours de festin et de joie, où l’on s’envoie des portions les uns aux autres, et des dons aux pauvres.
\VS{23}Les juifs s’engagèrent à faire ce qu’ils avaient déjà commencé et ce que Mardochée leur prescrivit.
\VS{24}Car Haman, fils d'Hammedatha, l’Agaguite, l'oppresseur de tous les juifs, avait projeté de détruire les juifs, et il avait jeté le pur, c'est-à-dire le sort, afin de les détruire et de les tuer ;
\VS{25}mais Esther s’étant présentée devant le roi, le roi ordonna par écrit que le méchant projet qu'Haman avait imaginé contre les juifs, retombe sur sa tête, et qu'on le pende au bois, lui et ses fils.
\VS{26}C'est pourquoi on appelle ces jours-là purim, du nom de pur\FTNT{Pur ou purim : Ce terme signifie sort (Est. 3 : 7). La fête de purim a été instituée pour célébrer leur délivrance de l’extermination planifiée par Haman, à la suite de l’intervention héroïque d’Esther. Les juifs l’observent désormais chaque année le 14 du mois d’Adar (février ou mars) depuis le temps d’Esther jusqu’à ce jour.}. D’après tout le contenu de cette lettre, et selon ce qu’ils avaient eux-mêmes vu et ce qui leur était arrivé,
\VS{27}Les juifs établirent et adoptèrent pour eux, pour leur postérité, et pour tous ceux qui s’attacheraient à eux, l’engagement de ne pas manquer de célébrer chaque année ces deux jours, selon le mode prescrit et au temps fixé.
\VS{28}Ces jours devaient être rappelés et observés de génération en génération, dans chaque famille, dans chaque province et dans chaque ville ; et ces jours de Purim ne devaient jamais être abolis au milieu des juifs, ni le souvenir s’en effacer parmi leurs descendants.
\VS{29}La reine Esther, fille d'Abichaïl, écrivit aussi avec le juif Mardochée, de manière pressante pour la seconde fois, pour confirmer la lettre sur les Purim.
\VS{30}On envoya des lettres à tous les juifs, dans les cent vingt-sept provinces du royaume d'Assuérus. Elles contenaient des paroles de paix et de vérité,
\VS{31}pour établir ces jours de Purim au temps fixé, comme Mardochée le juif et la reine Esther les avaient établis pour eux, et comme ils les avaient établis pour eux-mêmes et pour leur postérité, à l’occasion de leur jeûne et de leurs cris.
\VS{32}Ainsi l'édit d'Esther confirma l’institution des Purim, et cela fut écrit dans le livre.
\TextTitle{[Mardochée établi dans la cours du roi]}
\Chap{10}
\VerseOne{}Le roi Assuérus imposa un tribut au pays, et aux îles de la mer.
\VS{2}Tous les faits concernant ses exploits, et les détails sur la grandeur à laquelle le roi éleva Mardochée, ne sont-ils pas écrits dans le livre des chroniques des rois de Médie et de Perse ?
\VS{3}Car Mardochée le juif était le premier après le roi Assuérus ; grand parmi les juifs et agréable à la multitude de ses frères, il chercha le bien-être de son peuple, et parla pour la paix de toute sa race.
\PPE{}
\end{multicols}

%\clearpage\ShortTitle{Daniel}\BookTitle{Daniel}\BFont
\noindent\hrulefill
{\footnotesize
\textit{
\bigskip
{\centering{}
\\Auteur : Daniel
\\(Heb. : Daniye'l)
\\Signification : Dieu est mon juge
\\Thème : Ascension et chute des royaumes
\\Date de rédaction : 6\up{ème} siècle av. J.-C.\\}
}
%\bigskip
\textit{
\\Issu d'une famille princière de Juda, Daniel fut déporté de Jérusalem à Babylone pendant sa jeunesse, sous le règne de Nebucadnestar. Lui et trois de ses amis – eux aussi de noble lignée - furent choisis pour être instruits selon la sagesse babylonienne en vue de servir le roi. Fervent dans sa foi en Yahweh, Daniel - imité ensuite par ses compagnons – résolut de ne point se souiller et obtint ainsi la faveur de son Dieu. Son intégrité et sa crainte de Dieu lui valurent de miraculeuses victoires, de nombreuses distinctions et une grande sagesse. Daniel avait reçu du discernement pour expliquer songes et visions et délivra plusieurs prophéties dont certaines se sont déjà accomplies, d'autres se réaliseront à la fin du temps des nations, au moment du retour de Christ.
%\bigskip
\\Dieu témoigna de la justice de Daniel au prophète Ezéchiel dont il fut contemporain.\bigskip
}
}
\par\nobreak\noindent\hrulefill
\begin{multicols}{2}
\Chap{1}
\TextTitle{Juda livré à la captivité babylonienne}
\VerseOne{}La troisième année du règne de Jojakim, roi de Juda, Nebucadnetsar, roi de Babylone, vint contre Jérusalem et l'assiégea.
\VS{2}Le Seigneur livra entre ses mains Jojakim\FTNT{En 597 av. J-C., la ville de Jérusalem tomba entre les mains des Babyloniens qui déportèrent le roi Jojakim et nommèrent comme roi à sa place son oncle Sédécias. Une petite partie de la population fut déportée à cette occasion. Cette première déportation ne concernait que l'élite administrative et sacerdotale : prêtres, scribes, hauts fonctionnaires, membres de la famille royale et artisans métallurgistes. Pour de nombreux historiens, il s'agissait moins d'une déportation que d'une constitution d'un groupe d'otages. Le roi, quelques membres de sa famille, et de diverses familles de notables, furent tenus en résidence surveillée à la cour babylonienne pour s'assurer que le royaume de Juda resterait pacifié.}, roi de Juda et une partie des vases de la maison de Dieu. Nebucadnetsar emporta les vases au pays de Schinear\FTNT{Schinear : « le pays des deux fleuves ». C'est l'ancien nom du territoire qui est devenu Babylonie ou Chaldée. C'est le pays de Nimrod (Ge. 10:6-12). C'est à Schinear qu'on tenta de construire la tour de Babel et de mettre en place le premier gouvernement mondial.}, dans la maison de son dieu, il les mit dans la maison du trésor de son dieu.
\VS{3}Le roi dit à Aschpenaz, capitaine de ses eunuques, d'amener quelques-uns des enfants d'Israël de race royale\FTNT{Daniel était de la race royale (2 R. 20:16-19 ; Es. 39:1-8).} et des principaux seigneurs,
\VS{4}quelques jeunes enfants en qui il n'y avait aucun défaut corporel, beaux de figure, instruits en toute sagesse, connaissant les sciences, pleins d'intelligence, et capables de se tenir dans le palais du roi ; et à qui l'on enseignerait les lettres et la langue des Chaldéens.
\VS{5}Le roi leur assigna pour provision chaque jour une portion de la viande royale et du vin dont il buvait, afin qu'on les nourrisse ainsi pendant trois ans au bout desquels ils se tiendraient devant le roi.
\VS{6}Il y avait parmi eux, d'entre les fils de Juda, Daniel, Hanania, Mischaël et Azaria.
\VS{7}Mais le capitaine des eunuques leur donna d'autres noms, il donna à Daniel le nom de Beltschatsar, à Hanania celui de Schadrac, à Mischaël celui de Méschac et à Azaria celui d'Abed-Nego.
\TextTitle{La fermeté de Daniel à Babylone}
\VS{8}Daniel résolut dans son cœur de ne pas se souiller par la portion de la viande du roi et par le vin dont le roi buvait; c'est pourquoi il supplia le chef des eunuques afin qu'il ne l'oblige pas à se souiller.
\VS{9}Et Dieu fit trouver à Daniel faveur et grâce auprès du chef des eunuques. 
\VS{10}Et le chef des eunuques dit à Daniel : Je crains le roi, mon seigneur, qui a fixé ce que vous devez manger et boire ; car pourquoi verrait-il vos visages plus défaits que ceux des jeunes gens de votre âge ? Vous exposeriez ma tête auprès du roi.
\VS{11}Mais Daniel dit à Meltsar, l'intendant à qui le chef des eunuques avait remis la surveillance de Daniel, Hanania, Mischaël et Azaria :
\VS{12}Eprouve, je te prie, tes serviteurs pendant dix jours, et qu'on nous donne des légumes à manger et de l'eau à boire.
\VS{13}Après cela, tu regarderas nos visages et ceux des jeunes enfants qui mangent la portion de la viande royale; puis tu feras à tes serviteurs selon ce que tu auras vu.
\VS{14}Et il les écouta dans cette affaire et les éprouva pendant dix jours.
\VS{15}Au bout des dix jours, leurs visages parurent en meilleur état et plus d'embonpoint que tous les jeunes gens qui mangeaient la portion de la viande royale.
\VS{16} Ainsi Meltsar prenait la portion de leur viande et le vin qu'ils devaient boire, et leur donnait des légumes.
\VS{17}Et Dieu donna à ces quatre jeunes gens de la science et de l'intelligence dans toutes les lettres, et de la sagesse ; et Daniel comprenait toutes les visions et tous les songes.
\VS{18}Et à la fin des jours fixés par le roi pour qu'on les lui amène, le chef des eunuques les présenta à Nebucadnetsar.
\VS{19}Le roi s'entretint avec eux ; mais entre eux tous il ne s'en trouva pas de tels que Daniel, Hanania, Mischaël et Azaria ; et ils entrèrent au service du roi.
\VS{20}Sur toutes les questions savantes qui réclamaient de la sagesse et de l'intelligence, et sur lesquelles le roi les interrogeait, il les trouva dix fois supérieurs à tous les magiciens et les astrologues qui étaient dans tout son royaume.
\VS{21}Et Daniel fut là jusqu'à la première année du roi Cyrus.
\Chap{2}
\TextTitle{Les sages de Babylone tous condamnés à mort}
\VerseOne{}La deuxième année du règne de Nebucadnetsar, Nebucadnetsar eut des songes, et son esprit fut agité, et son sommeil fut interrompu.
\VS{2}Alors le roi fit appeler les magiciens, les astrologues, les enchanteurs et les Chaldéens, pour qu'ils lui expliquent ses songes ; ils vinrent donc et se présentèrent devant le roi.
\VS{3}Le roi leur dit : J'ai eu un songe, mon esprit est agité, tâchant de connaître ce songe\FTNT{Ce songe annonce la future mise en place d'un gouvernement mondial. Voir commentaire en Da. 7:3.}.
\VS{4}Et les Chaldéens répondirent au roi en langue araméenne\FTNT{L'araméen : De Daniel 2:5 à 7:28, le livre est écrit en araméen.} : Ô roi, vis éternellement ! Dis le songe à tes serviteurs et nous en donnerons l'interprétation.
\VS{5}Mais le roi répondit et dit aux Chaldéens : La chose m'a échappé ; si vous ne me faites connaître le songe et son interprétation, vous serez mis en pièces et vos maisons seront réduites en un tas d'immondices.
\VS{6}Mais si vous me faites connaître le songe et son interprétation, vous recevrez de moi, des dons, des présents et un grand honneur. Quoi qu'il en soit faites-moi connaître le songe et son interprétation.
\VS{7}Ils répondirent pour la seconde fois et dirent : Que le roi dise le songe à ses serviteurs et nous en donnerons l'interprétation.
\VS{8}Le roi répondit et dit : Je m'aperçois en vérité, que vous ne cherchez qu'à gagner du temps, parce que vous voyez que la chose m'a échappée.
\VS{9}Mais si vous ne me faites pas connaître le songe, il y a une même sentence contre vous tous ; car vous vous êtes préparés à dire devant moi des mensonges et des faussetés en attendant que le temps soit changé. Quoi qu'il en soit, dites-moi le songe et je saurai que vous pouvez m'en donner l'interprétation.
\VS{10}Les Chaldéens répondirent au roi et dirent : Il n'y a aucun homme sur la terre qui puisse exécuter ce que le roi demande. Et aussi il n'y a ni roi, ni seigneur, ni gouverneur qui ait jamais demandé une telle chose à quelque magicien, astrologue ou Chaldéen que ce soit.
\VS{11}Car la chose que le roi demande est extrêmement difficile et il n'y a personne qui puisse le faire connaître au roi, excepté les dieux dont la demeure n'est pas parmi les hommes.
\VS{12}A cause de cela, le roi s'irrita et se mit dans une très grande colère, et ordonna qu'on fasse périr tous les sages de Babylone.
\VS{13}La sentence fut donc publiée; on mettait à mort les sages et l'on cherchait Daniel et ses compagnons pour les faire périr.
\TextTitle{Daniel implore la miséricorde de Dieu}
\VS{14}Alors Daniel détourna l'exécution du conseil et l'arrêt donné à Arjoc, chef des gardes du roi, qui était sorti pour tuer les sages de Babylone.
\VS{15}Et il demanda et dit à Arjoc, commandant du roi : Pourquoi la sentence du roi est-elle si sévère ? Arjoc exposa la chose à Daniel.
\VS{16}Et Daniel entra et pria le roi de lui accorder du temps pour donner l'interprétation au roi.
\VS{17}Alors Daniel alla dans sa maison et informa de cette affaire Hanania, Mischaël et Azaria, ses compagnons,
\VS{18}pour implorer la miséricorde du Dieu des cieux sur ce secret, afin qu'on ne mette pas à mort Daniel et ses compagnons avec le reste des sages de Babylone. 
\TextTitle{Le songe de la grande statue révélé à Daniel}
\VS{19}Et le secret fut révélé à Daniel dans une vision pendant la nuit. Et Daniel bénit le Dieu des cieux.
\VS{20}Daniel prit donc la parole et dit : Béni soit le nom de Dieu, d'éternité en éternité ! A lui appartiennent la sagesse et la force\FTNT{Job. 12:13 ; Ap. 5:12 ; Ap. 7:12.}.
\VS{21}C'est lui qui change les temps et les saisons, qui ôte et qui établit les rois, qui donne la sagesse aux sages et la connaissance à ceux qui ont de l'intelligence.
\VS{22}C'est lui qui révèle les choses profondes et cachées, il connaît les choses qui sont dans les ténèbres et la lumière demeure avec lui\FTNT{De. 29:29 ; Es. 48:6 ; Jé. 33:3 ; Lu. 12:2-3.}.
\VS{23}Ô Dieu de nos pères ! Je te glorifie et te loue de ce que tu m'as donné de la sagesse et de la force, et de ce que tu m'as maintenant fait connaître ce que nous t'avons demandé, en nous ayant fait connaître le secret du roi.
\VS{24}Après cela, Daniel alla auprès d'Arjoc, à qui le roi avait ordonné de faire périr les sages de Babylone. Il alla et lui parla ainsi : Ne fais pas périr les sages de Babylone, mais fais-moi entrer devant le roi et je donnerai au roi l'interprétation qu'il souhaite.
\VS{25}Alors Arjoc conduisit promptement Daniel devant le roi et lui parla ainsi : J'ai trouvé parmi les captifs de Juda un homme qui donnera au roi l'interprétation de son songe.
\VS{26}Le roi prit la parole et dit à Daniel, qu'on nommait Beltschatsar : Es-tu capable de me faire connaître le songe que j'ai eu et son interprétation ?
\VS{27}Daniel répondit en présence du roi et dit : Ce que le roi demande est un secret que les sages, les astrologues, les magiciens et les devins ne sont pas capables de révéler au roi.
\VS{28}Mais il y a dans les cieux un Dieu qui révèle les secrets et qui a fait connaître au roi Nebucadnetsar ce qui doit arriver dans les derniers jours\FTNT{« Les derniers jours » voir commentaires dans Ge. 49:1-2.}. Voici ton songe et les visions de ta tête que tu as eues sur ta couche.
\VS{29}Sur ta couche, ô roi, il t'est monté des pensées touchant ce qui arriverait après ce temps-ci ; et celui qui révèle les secrets t'a fait connaître ce qui doit arriver.
\VS{30}Si ce secret m'a été révélé, ce n'est point qu'il y ait en moi une sagesse supérieure à celle de tous les vivants, mais c'est afin de donner au roi l'interprétation de son songe et afin que tu connaisses les pensées de ton cœur.
\VS{31}Ô roi, tu regardais et tu voyais une grande statue\FTNT{Voir annexe « La statue de Nebucadnetsar ».} ; cette grande statue, dont la splendeur était extraordinaire, était debout devant toi et son apparence était terrible.
\VS{32}La tête de cette statue était d'un or très fin, sa poitrine et ses bras étaient d'argent ; son ventre et ses cuisses étaient d'airain ;
\VS{33}ses jambes étaient de fer et ses pieds étaient en partie de fer et en partie de terre.
\VS{34}Tu regardais cela, jusqu'à ce qu'une pierre se détacha sans main, frappa les pieds de fer et d'argile de la statue et les brisa.
\VS{35}Alors le fer, l'argile, l'airain, l'argent et l'or furent brisés ensemble et devinrent comme la paille de l'aire en été que le vent transporte çà et là ; et nulle trace n'en fut retrouvée. Mais la pierre qui avait frappé la statue devint une grande montagne et remplit toute la terre.
\TextTitle{Premier empire universel : Babylone\FTNT{Cp. Da. 7:4.}}
\VS{36}C'est là le songe. Nous en donnerons maintenant l'interprétation devant le roi.
\VS{37}Ô roi, tu es le roi des rois, parce que le Dieu des cieux t'a donné le royaume, la puissance, la force et la gloire.
\VS{38}Il a remis entre tes mains, en quelque lieu qu'ils habitent, les enfants des hommes, les bêtes des champs et les oiseaux du ciel, et il t'a fait dominer sur eux tous : C'est toi qui es la tête d'or\FTNT{Jé. 27:6-7.}.
\TextTitle{Deuxième et troisième empires : les Mèdes et les Perses\FTNTT{cp. Da. 7:5 ; 8:20} et la Grèce\FTNTT{cp. Da. 7:6 ; 8:21}}
\VS{39}Mais après toi, il s'élèvera un autre royaume, moindre que le tien ; et ensuite un troisième royaume qui sera d'airain et qui dominera sur toute la terre.
\TextTitle{Quatrième empire : Rome\FTNTT{Cp. Da. 7:7 ; 9:26.}}
\VS{40}Puis il y aura un quatrième royaume, fort comme du fer ; de même que le fer brise et rompt tout ainsi il brisera et rompra tout, comme le fer qui met tout en pièces.
\VS{41}Et quant à ce que tu as vu, que les pieds et les orteils étaient en partie d'argile de potier et en partie de fer, c'est que ce royaume sera divisé, mais il y aura en lui de la force du fer, parce que tu as vu le fer mêlé avec l'argile de potier.
\VS{42}Et comme les doigts des pieds étaient en partie de fer et en partie d'argile, ce royaume sera en partie fort et en partie fragile.
\VS{43} Quant à ce que tu as vu, le fer mêlé avec l'argile de potier, c'est qu'ils se mêleront par des alliances humaines\FTNT{Le mot « alliances » vient de l'araméen « zera » qui signifie « semence » et « descendant ».} ; mais ils ne seront point unis l'un à l'autre de même que le fer ne s'allie point avec l'argile.
\TextTitle{Le royaume du Messie}
\VS{44}Dans le temps de ces rois, le Dieu des cieux suscitera un Royaume qui ne sera jamais détruit, et ce Royaume ne passera point à un autre peuple ; il brisera et anéantira tous ces royaumes-là, et lui-même sera établi éternellement.
\VS{45}Selon que tu as vu que de la montagne une pierre a été coupée sans main et qu'elle a brisé le fer, l'airain, la terre, l'argent et l'or. Le grand Dieu a fait connaître au Roi ce qui arrivera ci-après ; or le songe est véritable et son interprétation est certaine.
\TextTitle{Yahweh, le Dieu qui revèle les secrets}
\VS{46}Alors le roi Nebucadnetsar tomba sur sa face et se prosterna devant Daniel et il ordonna qu'on lui offre des offrandes de bonne odeur et des parfums.
\VS{47}Le roi parla à Daniel et lui dit : Certainement, votre Dieu est le Dieu des dieux, et le Seigneur des rois, et il révèle les secrets, puisque tu as pu découvrir ce secret.
\VS{48}Alors le roi éleva Daniel en dignité et lui fit de nombreux et riches présents ; il l'établit gouverneur sur toute la province de Babylone et chef suprême de tous les sages de Babylone.
\VS{49}Daniel pria le roi de remettre l'intendance de la province de Babylone à Schadrac, Méschac et Abed-Négo. Et Daniel se tenait à la porte du roi.
\Chap{3}
\TextTitle{La statue d'or de Nebucadnetsar}
\VerseOne{}Le roi Nebucadnetsar fit une statue d'or\FTNT{Nebucadnetsar est un type de l'antéchrist qui s'oppose aux plans de Dieu. Contrairement à la statue composée de plusieurs métaux qu'il avait vue en songe, et où il est représenté par la tête en or (Da. 2:38), il s'est fait construire une statue entièrement en or, se déclarant ainsi symboliquement invincible et immortel. En agissant de la sorte, Nebucadnetsar se fait Dieu et exige d'être adoré (2 Th. 2:3-4). Cette statue annonçait prophétiquement la mise en place d'une religion mondiale, fruit d'un mélange entre la politique et la religion. Ces choses sont déjà bien installées, il ne manque plus que la révélation de l'impie. Nous vivons dans une époque où on oblige les chrétiens à adhérer à des organisations politiques, religieuses, sous contrôle de l'état, et cela dans le but de contrôler les individus et le message qu'ils entendent et diffusent. Ainsi, les dirigeants actuels doivent passer au préalable par des études théologiques, dont l'enseignement contredit de plus en plus la vérité biblique pour se conformer aux préceptes de ce monde. Une fois ordonnés, ils doivent affilier leurs églises à des fédérations qui sont sous contrôle de l'état. En échange des subventions, beaucoup accepteront de diluer l'évangile, privant ainsi les âmes de la vérité.}, dont la hauteur était de soixante coudées, et la largeur de six coudées. Il la dressa dans la vallée de Dura, dans la province de Babylone.
\VS{2}Puis le roi Nebucadnetsar envoya pour rassembler les satrapes, les intendants, les gouverneurs, les conseillers, les trésoriers, les jurisconsultes, les juges, et tous les magistrats des provinces, afin qu'ils se rendent à la dédicace de la statue que le roi Nebucadnetsar avait dressée.
\VS{3}Ainsi furent assemblés les satrapes, les intendants, les gouverneurs, les conseillers, les trésoriers, les jurisconsultes, les juges, et tous les magistrats des provinces, pour la dédicace de la statue que le roi Nebucadnetsar avait dressée. Ils s'assemblèrent devant la statue que le roi Nebucadnetsar avait dressée.
\VS{4}Alors un héraut cria à haute voix, en disant : On vous fait savoir, ô peuples, nations, et langues !
\VS{5}Au moment où vous entendrez le son du cor, du chalumeau, de la guitare, de la sambuque, du psaltérion, de la cornemuse, et de toutes sortes d'instruments de musique, vous vous jetterez à terre et vous adorerez la statue d'or que le roi Nebucadnetsar a dressée.
\VS{6}Quiconque ne se jettera pas à terre et n'adorera pas sera jeté à l'instant même au milieu de la fournaise de feu ardent.
\VS{7}C'est pourquoi, au moment où tous les peuples entendirent le son du cor, du chalumeau, de la guitare, de la sambuque, du psaltérion, et de toutes sortes d'instruments de musique, tous les peuples, les nations, et les hommes de toutes les langues, se prosternèrent et adorèrent la statue d'or que le roi avait dressée.
\TextTitle{Le refus de l'idolâtrie}
\VS{8}Alors à ce même moment, certains chaldéens s'approchèrent et accusèrent les Juifs.
\VS{9}Et ils parlèrent et dirent au roi Nebucadnetsar : Roi, vis éternellement !
\VS{10}Toi, ô roi, tu as donné un ordre d'après lequel tout homme qui entendrait le son du cor, du chalumeau, de la guitare, de la sambuque, du psaltérion, de la cornemuse, et de toutes sortes d'instruments de musique, devrait se prosterner et adorer la statue d'or,
\VS{11}et que quiconque ne se prosternerait pas et ne l'adorerait pas, serait jeté au milieu d'une fournaise ardente.
\VS{12}Or, il y a certains Juifs que tu as établis sur les affaires de la province de Babylone, Schadrac, Méschac, et Abed-Négo ; ces hommes-là, ô roi, ne tiennent aucun compte de toi ; ils ne servent pas tes dieux, et ils n'adorent pas la statue d'or que tu as dressée.
\VS{13}Alors le roi Nebucadnetsar, saisi de colère et de fureur, ordonna qu'on amène Schadrac, Méschac, et Abed-Négo. Et ces hommes furent amenés devant le roi.
\VS{14}Et le roi Nebucadnetsar prit la parole et leur dit : Est-il vrai, Schadrac, Méschac, et Abed-Négo, que vous ne servez pas mes dieux, et que vous ne vous prosternez pas devant la statue d'or que j'ai dressée ?
\VS{15}Maintenant si vous êtes prêts, au moment où vous entendrez le son du cor, du chalumeau, de la guitare, de la sambuque, du psaltérion, de la cornemuse, et de toutes sortes d'instruments de musique, vous vous prosternerez, et vous adorerez la statue que j'ai faite ; si vous ne l'adorez pas, vous serez jetés à l'instant au milieu de la fournaise de feu ardent. Et qui est le dieu qui vous délivrera de mes mains ?
\VS{16}Schadrac, Méschac et Abed-Négo répondirent et dirent au roi Nebucadnetsar : Nous n'avons pas besoin de te répondre sur ce sujet.
\VS{17}Voici, notre Dieu, que nous servons, peut nous délivrer de la fournaise de feu ardent, et il nous délivrera de ta main, ô roi !
\VS{18}Sinon, sache, ô roi, que nous ne servirons pas tes dieux, et que nous n'adorerons pas la statue d'or que tu as dressée.
\TextTitle{L'épreuve de la fournaise de feu ardent}
\VS{19}Alors Nebucadnetsar fut rempli de fureur, et il changea de visage en tournant ses regards contre Schadrac, Méschac, et Abed-Négo. Il prit la parole et ordonna de chauffer la fournaise sept fois plus qu'on avait coutume de la chauffer.
\VS{20}Puis il commanda aux hommes les plus forts et les plus vaillants qui étaient dans son armée de lier Schadrac, Méschac, et Abed-Négo, et de les jeter dans la fournaise de feu ardent.
\VS{21}Et en même temps ces hommes furent liés avec leurs caleçons, leurs chaussures, leurs tiares, et leurs vêtements, et furent jetés au milieu de la fournaise de feu ardent.
\VS{22}Et parce que l'ordre du roi était sévère, et que la fournaise était extraordinairement chauffée, la flamme tua les hommes qui y avaient jetés, Schadrac, Méschac, et Abed-Négo.
\VS{23}Et ces trois hommes, Schadrac, Méschac, et Abed-Négo, tombèrent tous liés au milieu de la fournaise ardente.
\TextTitle{La grandeur de Yahweh, le Dieu qui délivre}
\VS{24}Alors le roi Nebucadnetsar fut effrayé, et se leva précipitamment. Il prit la parole et il dit à ses conseillers : N'avons-nous pas jeté trois hommes liés au milieu du feu ? Ils répondirent et dirent au roi : Certainement, ô roi !
\VS{25}Il reprit et dit : Voici, je vois quatre hommes sans liens qui marchent au milieu du feu, et qui n'ont point de mal ; et la figure du quatrième est semblable à celle d'un fils de Dieu.
\VS{26}Alors Nebucadnetsar s'approcha vers la porte de la fournaise de feu ardent ; et prenant la parole, il dit : Schadrac, Méschac, et Abed-Négo, serviteurs du Dieu Très-Haut, sortez et venez ! Alors Schadrac, Méschac, et Abed-Négo sortirent du milieu du feu.
\VS{27}Puis les satrapes, les intendants, les gouverneurs, et les conseillers du roi s'assemblèrent pour contempler ces hommes-là, et le feu n'avait eu aucun pouvoir sur leurs corps, et aucun cheveu de leur tête n'était brûlé, et leurs caleçons n'étaient point endommagés, et l'odeur du feu n'avait pas passé sur eux.
\VS{28}Alors Nebucadnetsar prit la parole et dit : Béni soit le Dieu de Schadrac, Méschac, et Abed-Négo, lequel a envoyé son ange et délivré ses serviteurs qui ont eu confiance en lui, et qui ont violé l'ordre du roi et livré leur corps plutôt que de servir et d'adorer aucun autre dieu que leur Dieu\FTNT{Mt. 4:10 ; Ac. 4:19 ; Ac. 5:29.}.
\TextTitle{Schadrac, Méschac et Abed-Nego élevé par le roi}
\VS{29}Voici maintenant l'ordre que je donne : Tout homme, à quelque nation ou langue qu'il appartienne, qui parlera mal du Dieu de Schadrac, Méschac, et Abed-Négo, sera mis en pièces, et sa maison sera réduite en un tas d'immondices, parce qu'il n'y a aucun autre dieu qui puisse délivrer comme lui.
\VS{30}Alors le roi fit réussir Schadrac, Méschac, et Abed-Négo dans la province de Babylone.
\Chap{4}
\TextTitle{La suprématie de Yahweh déclarée aux nations}
\VerseOne{}Le roi Nebucadnetsar, à tous les peuples, aux nations, aux hommes de toutes langues qui habitent sur toute la terre : Que votre paix soit multipliée !
\VS{2}Il m'a semblé bon de vous déclarer les signes et les merveilles que le Dieu Très-Haut a opérés à mon égard.
\VS{3}Ô que ses signes sont grands, et ses merveilles pleines de force ! Son règne est un règne éternel, et sa domination subsiste de génération en génération\FTNT{Ps 102:12 ; La. 5:19 ; Lu. 1:33.}.
\TextTitle{La vision du grand arbre}
\VS{4}Moi, Nebucadnetsar, j'étais tranquille dans ma maison, et heureux dans mon palais.
\VS{5}J'ai eu un songe qui m'épouvanta ; et les pensées sur ma couche et les visions de ma tête me troublèrent.
\VS{6}J'ordonnai qu'on fasse venir devant moi tous les sages de Babylone, afin qu'ils me donnent l'interprétation du songe.
\VS{7}Alors vinrent les magiciens, les astrologues, les Chaldéens et les devins. Je leur dis le songe, mais ils ne purent m'en donner l'interprétation.
\VS{8}En dernier lieu, se présenta devant moi Daniel, nommé Beltschatsar, selon le nom de mon Dieu, et qui a en lui l'Esprit des dieux saints. Je lui dis le songe :
\VS{9}Beltschatsar, chef des magiciens, comme je sais que l'Esprit des dieux saints est en toi, et que nul secret ne t'est difficile, écoute les visions que j'ai eues en songe, et donne-moi son interprétation.
\VS{10}Voici les visions de ma tête, pendant que j'étais sur ma couche. Je regardais, et voici, il y avait un arbre au milieu de la terre d'une grande hauteur.
\VS{11}Cet arbre était devenu grand et fort, sa cime s'élevait jusqu'aux cieux, et on le voyait des extrémités de toute la terre.
\VS{12}Son feuillage était beau, et son fruit abondant, et il portait de la nourriture pour tous ; les bêtes des champs s'abritaient sous son ombre, les oiseaux du ciel habitaient dans ses branches, et tout être vivant tirait de lui sa nourriture.
\VS{13}Dans les visions de ma tête que j'avais sur ma couche, je regardais, et voici, un de ceux qui veillent et qui sont saints descendit des cieux.
\VS{14}Il cria à haute voix et parla ainsi : Abattez l'arbre, et coupez ses branches ! Secouez son feuillage, et dispersez son fruit ; que les bêtes s'enfuient de dessous, et les oiseaux du milieu de ses branches !
\VS{15}Mais laissez en terre le tronc où se trouvent ses racines, et liez-le avec des chaînes de fer et d'airain, qu'il soit parmi l'herbe des champs. Qu'il soit trempé de la rosée des cieux, et qu'il ait, comme les bêtes, l'herbe de la terre pour partage.
\VS{16}Que son cœur d'homme soit changé, et qu'un cœur de bête lui soit donné ; et que sept temps passent sur lui.
\VS{17}Cette sentence est le décret de ceux qui veillent, cette résolution est un ordre des saints, afin que les vivants sachent que le Très-Haut domine sur le royaume des hommes, qu'il le donne à qui il lui plaît, et qu'il y élève le plus vil des hommes\FTNT{Cette vérité est confirmée par l'apôtre Paul dans Ro. 13:1. C'est Dieu qui choisit souverainement qui il établit à la tête d'un pays. Selon les Ecritures, toute autorité vient de Dieu.}.
\VS{18}Voilà le songe que j'ai eu, moi, le roi Nebucadnetsar. Toi donc Beltschatsar, donnes-en l'interprétation, puisque tous les sages de mon royaume ne peuvent me la donner ; mais toi, tu le peux, parce l'Esprit des dieux saints est en toi.
\TextTitle{Interprétation de la vision}
\VS{19}Alors Daniel, nommé Beltschatsar, fut stupéfait environ une heure, et ses pensées le troublaient. Le roi reprit et dit : Beltschatsar, que le songe et son interprétation ne te troublent pas ! Et Beltschatsar répondit : Mon seigneur, que le songe soit pour ceux qui te haïssent, et son interprétation pour tes ennemis !
\VS{20}L'arbre que tu as vu, qui était devenu grand et fort, dont la cime s'élevait jusqu'aux cieux, et qu'on voyait de tous les points de la terre ;
\VS{21}cet arbre, dont le feuillage était beau, et les fruits abondants, qui portait de la nourriture pour tous, sous lequel s'abritaient les bêtes des champs, et parmi les branches duquel les oiseaux du ciel faisaient leur demeure,
\VS{22}c'est toi, ô roi, qui es devenu grand et fort, dont la grandeur s'est accrue et s'est élevée jusqu'aux cieux, et dont la domination s'étend jusqu'aux extrémités de la terre.
\VS{23}Le roi a vu un de ceux qui veillent et qui sont saints descendre des cieux et dire : Abattez l'arbre, et détruisez-le ! Toutefois, laissez en terre le tronc où se trouvent ses racines, et liez-le avec des chaînes de fer et d'airain, parmi l'herbe des champs, qu'il soit trempé de la rosée du ciel, et que son partage soit avec les bêtes des champs, jusqu'à ce que sept temps soient passés sur lui.
\VS{24}Voici l'interprétation, ô roi, voici le décret du Très-Haut, qui s'accomplira sur mon seigneur, le roi :
\VS{25}On te chassera du milieu des hommes, tu auras ta demeure avec les bêtes des champs, et l'on te donnera de l'herbe comme aux bœufs, et tu seras trempé de la rosée du ciel ; et sept temps passeront sur toi, jusqu'à ce que tu reconnaisses que le Très-Haut domine sur le royaume des hommes, et qu'il le donne à qui il lui plaît.
\VS{26}L'ordre de laisser le tronc où se trouvent les racines de cet arbre signifie que ton royaume te sera rendu, dès que tu auras reconnu que les cieux dominent.
\VS{27}C'est pourquoi, ô roi, que mon conseil te soit agréable : Rachète tes péchés par la justice, et tes iniquités en faisant miséricorde aux pauvres, et ta paix pourra se prolonger.
\TextTitle{Le roi déchu à cause de son orgueil}
\VS{28}Toutes ces choses se sont accomplies sur le roi Nebucadnetsar.
\VS{29}Au bout de douze mois, comme il se promenait dans le palais royal de Babylone,
\VS{30}le roi prit la parole et dit : N'est-ce pas ici Babylone la grande, que j'ai bâtie pour être la maison royale, par la puissance de ma force et pour la gloire de ma magnificence ?
\VS{31}La parole était encore dans la bouche du roi, qu'une voix descendit du ciel, disant : Roi Nebucadnetsar, on t'annonce que ton royaume va t'être ôté.
\VS{32}On te chassera du milieu des hommes, tu auras ta demeure avec les bêtes des champs ; on te donnera de l'herbe à manger comme aux bœufs ; et sept temps passeront sur toi, jusqu'à ce que tu reconnaisses que le Très-Haut domine sur le royaume des hommes, et qu'il le donne à qui il lui plaît.
\VS{33}Au même instant, la parole s'accomplit sur Nebucadnetsar. Il fut chassé du milieu des hommes, il mangea de l'herbe comme les bœufs, et son corps fut trempé de la rosée du ciel jusqu'à ce que ses cheveux croissent comme les plumes des aigles, et ses ongles comme ceux des oiseaux.
\TextTitle{Le roi est rétabli ; il loue le Dieu Très-Haut}
\VS{34}Mais à la fin de ces jours-là, moi Nebucadnetsar, je levai mes yeux vers le ciel, et la raison me revint. J'ai béni le Très-Haut, j'ai loué et glorifié celui qui vit éternellement, celui dont la domination est une domination éternelle, et dont le règne subsiste de génération en génération.
\VS{35}Tous les habitants de la terre ne sont à ses yeux que néant ; il agit comme il lui plaît avec l'armée des cieux et avec les habitants de la terre, et il n'y a personne qui empêche sa main, et qui lui dise : Que fais-tu\FTNT{Es. 45:9 ; Jé. 23:18-22 ; ps.115:3 ; Job. 9:12.} ?
\VS{36}En ce temps, la raison me revint, et je retournai à la gloire de mon royaume, ma magnificence et ma splendeur me furent rendues ; mes conseillers et mes grands me redemandèrent ; je fus rétabli dans mon royaume, et ma gloire fut augmentée.
\VS{37}Maintenant, moi, Nebucadnetsar, je loue, j'exalte, et je glorifie le Roi des cieux, dont toutes les œuvres sont véritables et ses voies justes, et qui peut abaisser ceux qui marchent avec orgueil\FTNT{De. 32:4 ; Es. 13:11 ; Ez. 17:24 ; Ps. 145:17.}.
\Chap{5}
\TextTitle{Les vases du temple souillés}
\VerseOne{}Le roi Belschatsar donna un grand festin à ses grands au nombre de mille, et il buvait le vin devant ces mille courtisans.
\VS{2}Et ayant goûté au vin, Belschatsar ordonna qu'on apporte les vases d'or et d'argent que Nebucadnetsar, son père, avait enlevés du temple de Jérusalem\FTNT{Ex. 27 ; Ex. 30 ; 2 Ch. 36:10.}, afin que le roi et ses grands, ses femmes et ses concubines, s'en servent pour boire.
\VS{3}Alors furent apportés les vases d'or qui avaient été enlevés du temple, de la maison de Dieu qui était à Jérusalem ; et le roi, et ses grands, ses femmes et ses concubines, s'en servirent pour boire.
\VS{4}Ils burent du vin, et ils louèrent leurs dieux d'or, d'argent, d'airain, de fer, de bois et de pierre.
\TextTitle{L'écriture sur la muraille}
\VS{5}Et à cette même heure-là sortirent de la muraille des doigts d'une main d'homme, qui écrivaient à l'endroit du chandelier, sur l'enduit de la muraille du palais royal ; et le Roi voyait cette partie de main qui écrivait.
\VS{6}Alors l'aspect du roi changea, et ses pensées l'effrayèrent, si bien que les jointures de ses reins se desserrèrent, et ses genoux se cognèrent l'un contre l'autre.
\VS{7}Puis le roi cria avec force qu'on fasse venir les astrologues, les Chaldéens et les devins ; et le roi prit la parole et dit aux sages de Babylone : Quiconque lira cette écriture, et m'en donnera l'interprétation, sera revêtu de pourpre, il aura un collier d'or à son cou, et sera le troisième dans le gouvernement du royaume.
\VS{8}Alors tous les sages du roi entrèrent, mais ils ne purent pas lire l'écriture et en donner au roi l'interprétation.
\VS{9}Sur quoi le roi Belschatsar fut très effrayé, il changea de couleur, et ses grands furent consternés.
\TextTitle{Interprétation de l'écriture: Division de l'empire babylonien}
\VS{10}La reine entra dans la maison du festin, à cause de ce qui était arrivé au roi et à ses grands. La reine prit la parole et dit : Ô roi, vis éternellement ! Que tes pensées ne te troublent pas, et que ton visage ne change pas de couleur !
\VS{11}Il y a dans ton royaume un homme qui a en lui l'Esprit des dieux saints ; et du temps de ton père, on trouva en lui une lumière, une intelligence, et une sagesse semblable à la sagesse des dieux. Aussi, le roi Nebucadnetsar, ton père, et le roi, ton père\FTNT{Belschatsar était le petit-fils de Nebucadnetsar qui avait régné conjointement avec son père, Nabonide, à partir de 552 av. J.-C.}, ô roi, l'établirent chef des magiciens, des astrologues, des Chaldéens et des devins,
\VS{12}parce qu'on trouva chez lui, chez Daniel, que le roi avait nommé Beltschatsar, un esprit supérieur, de la connaissance et de l'intelligence, pour interpréter les songes, pour expliquer les énigmes et résoudre les questions difficiles. Que Daniel soit donc appelé et il donnera l'interprétation que tu souhaites.
\VS{13}Alors Daniel fut introduit devant le roi. Le roi prit la parole et dit à Daniel : Es-tu ce Daniel, l'un des captifs de Juda, que le roi, mon père, a amenés de Juda ?
\VS{14}J'ai appris sur ton compte que tu as en toi l'Esprit des dieux, et qu'on trouve en toi une lumière, une intelligence et une sagesse extraordinaires.
\VS{15}On vient d'amener devant moi les sages et les astrologues, afin qu'ils lisent cette écriture et m'en donnent l'interprétation, mais ils n'ont pas pu donner l'interprétation de la chose.
\VS{16}J'ai appris que tu peux interpréter et résoudre les choses difficiles ; maintenant donc si tu peux lire cette écriture, et m'en donner l'interprétation, tu seras revêtu de pourpre, tu porteras à ton cou un collier d'or, et tu seras le troisième dans le gouvernement du royaume.
\VS{17}Alors Daniel répondit et dit en présence du roi : Que tes dons restent à toi, et donne tes présents à un autre ; toute fois je lirai l'écriture au roi, et je lui en donnerai l'interprétation.
\VS{18}Ô roi ! Le Dieu Très-Haut avait donné à Nebucadnetsar, ton père, le royaume, la magnificence, la gloire et l'honneur.
\VS{19}Et à cause de la grandeur qu'il lui avait donnée, tous les peuples, les nations, et les hommes de toutes langues tremblaient devant lui et le redoutaient. Il faisait mourir ceux qu'il voulait, et il laissait la vie à ceux qu'il voulait ; il élevait ceux qu'il voulait, et il abaissait ceux qu'il voulait.
\VS{20}Mais lorsque son cœur s'éleva et que son esprit s'endurcit jusqu'à l'arrogance, il fut renversé de son trône royal et dépouillé de sa gloire ;
\VS{21}il fut chassé du milieu des fils des hommes, son cœur fut rendu semblable à celui des bêtes, et sa demeure fut avec les ânes sauvages ; on lui donna comme aux bœufs de l'herbe à manger, et son corps fut trempé de la rosée du ciel, jusqu'à ce qu'il reconnaisse que le Dieu Très-Haut domine sur les royaumes des hommes, et qu'il y établit ceux qu'il lui plaît.
\VS{22}Et toi aussi, Belschatsar, son fils, tu n'as pas humilié ton cœur, quoique tu saches toutes ces choses.
\VS{23}Mais tu t'es élevé contre le Seigneur des cieux ; les vases de sa maison ont été apportés devant toi, et vous vous en êtes servis pour boire du vin, toi et tes grands, tes femmes et tes concubines ; tu as loué les dieux d'argent, d'or, d'airain, de fer, de bois et de pierre, qui ne voient point, qui n'entendent point, et qui ne savent rien, et tu n'as pas glorifié le Dieu dans la main duquel est ton souffle, et toutes tes voies\FTNT{Job. 12:10 et 33:4.}.
\VS{24}Alors de sa part a été envoyée cette partie de main, et cette écriture a été gravée.
\VS{25}Voici l'écriture qui a été gravée : Compté, compté, pesé et divisé.
\VS{26}Et voici l'interprétation de ces paroles. Compté : Dieu a compté ton règne, et y a mis la fin.
\VS{27}Pesé : Tu as été pesé dans la balance, et tu as été trouvé léger.
\VS{28}Mesuré : Ton royaume a été divisé, et donné aux Mèdes et aux Perses.
\VS{29}Aussitôt, Belschatsar donna des ordres, et l'on revêtit Daniel de pourpre, on lui mit un collier d'or au cou, et on publia qu'il serait le troisième dans le gouvernement du royaume.
\VS{30}Cette même nuit, Belschatsar, roi des Chaldéens, fut tué.
\VS{31}Et Darius, le Mède, reçut le royaume, étant âgé d'environ soixante-deux ans\FTNT{Es. 13:17 ; Es. 21:2 ; Jé. 51:11.}.
\Chap{6}
\TextTitle{Règne de Darius, le Mède}
\VerseOne{}Darius trouva bon d'établir sur le royaume cent vingt satrapes, qui devaient être répartis dans tout le royaume.
\VS{2}Il mit à leur tête trois chefs, au nombre desquels était Daniel, afin que ces satrapes leur rendent compte, et que le roi ne souffre aucun préjudice.
\VS{3}Daniel surpassait les autres chefs et satrapes, parce qu'il y avait en lui un Esprit supérieur ; et le roi pensait à l'établir sur tout le royaume.
\TextTitle{Daniel refuse l'idolâtrie et persévère dans la prière}
\VS{4}Alors les chefs et les satrapes cherchèrent une occasion d'accuser Daniel en ce qui concerne les affaires du royaume. Mais ils ne purent trouver en lui aucune occasion, ni aucune fausseté, parce qu'il était fidèle, et il ne se trouvait en lui ni faute ni vice.
\VS{5}Et ces hommes dirent : Nous ne trouverons aucune occasion d'accuser ce Daniel, à moins que nous n'en trouvions une dans la loi de son Dieu.
\VS{6}Alors ces chefs et ces satrapes se rendirent tumultueusement auprès du roi, et lui parlèrent ainsi : Roi Darius, vis éternellement !
\VS{7}Tous les chefs de ton royaume, les intendants, les satrapes, les conseillers, et les gouverneurs, sont d'avis d'établir un édit royal et une défense sévère, portant que quiconque, dans l'espace de trente jours, adressera des prières à quelque dieu ou à quelque homme, excepté à toi, ô roi, sera jeté dans la fosse aux lions.
\VS{8}Maintenant donc, ô roi, établis cette défense, et écris le décret afin qu'il soit irrévocable, selon la loi des Mèdes et des Perses, qui est immuable.
\VS{9}Là-dessus, le roi Darius écrivit le décret et la défense.
\VS{10}Lorsque Daniel sut que le décret était écrit, il entra dans sa maison, où les fenêtres de sa chambre étaient ouvertes dans la direction de Jérusalem ; et trois fois par jour, il se mettait à genoux, il priait, et il louait son Dieu, comme il le faisait auparavant\FTNT{1 R. 8:44 ; Ps. 55:17-18.}.
\VS{11}Alors ces hommes entrèrent tumultueusement, et ils trouvèrent Daniel qui priait et invoquait son Dieu.
\VS{12}Puis ils s'approchèrent du roi, et lui dirent au sujet de la défense royale : N'as-tu pas écrit une défense portant que tout homme dans l'espace de trente jours qui adresserait des prières à quelque dieu ou à quelque homme, excepté à toi, ô roi, serait jeté dans la fosse aux lions ? Le roi répondit : La chose est certaine, selon la loi des Mèdes et des Perses, qui est irrévocable.
\VS{13}Ils prirent de nouveau la parole et dirent au roi : Daniel, l'un des captifs de Juda, n'a tenu aucun compte de toi, ô roi, ni de la défense que tu as écrite, et il fait sa prière trois fois par jour.
\VS{14}Le roi fut très affligé quand il entendit cela ; il prit à cœur de délivrer Daniel, et jusqu'au coucher du soleil il s'efforça de le sauver.
\VS{15}Mais ces hommes se rendirent tumultueusement auprès du roi, et lui dirent : Sache, ô roi, que la loi des Mèdes et des Perses exige que toute défense ou tout décret établi par le roi soit irrévocable.
\TextTitle{Daniel demeure fidèle à Dieu face à la mort}
\VS{16}Alors le roi commanda qu'on amène Daniel, et qu'on le jette dans la fosse aux lions. Et le roi prenant la parole et dit à Daniel : Ton Dieu, lequel tu sers constamment, sera celui qui te délivrera.
\VS{17}On apporta une pierre, et on la mit sur l'ouverture de la fosse ; le roi la scella de son anneau, et de l'anneau de ses grands, afin que rien ne soit changé à l'égard de Daniel.
\TextTitle{Yahweh fait justice à Daniel}
\VS{18}Le roi se rendit ensuite dans son palais ; il passa la nuit à jeun, il ne fit point venir des danseuses\FTNT{Le mot « danseuse » vient de l'araméen « dachavah » qui signifie « divertissement », « instrument de musique », « danseuse », « concubine », « musique ».} auprès de lui, et il ne put se livrer au sommeil.
\VS{19}Puis le roi se leva au point du jour, avec l'aurore, et il alla précipitamment à la fosse aux lions.
\VS{20}En s'approchant de la fosse, il cria d'une voix triste : Daniel ! Le roi prit la parole et dit à Daniel : Daniel, serviteur du Dieu vivant, ton Dieu, que tu sers avec persévérance, a-t-il pu te délivrer des lions ?
\VS{21}Alors Daniel dit au roi : Ô roi, vis éternellement !
\VS{22}Mon Dieu a envoyé son ange, et a tellement fermé la gueule des lions, qu'ils ne m'ont fait aucun mal, parce que j'ai été trouvé innocent devant lui ; et même à ton égard, ô Roi ! je n'ai commis aucune faute. 
\VS{23}Alors le roi fut extrêmement heureux pour lui et il ordonna qu'on fasse retirer Daniel de la fosse. Ainsi Daniel fut retiré de la fosse, et on ne trouva sur lui aucune blessure, parce qu'il avait cru en son Dieu.
\VS{24}Le roi ordonna que ces hommes qui avaient accusé Daniel, soient amenés et jetés dans la fosse aux lions, eux, leurs enfants et leurs femmes, et avant qu'ils soient parvenus au fond de la fosse, les lions se saisirent d'eux, et leur brisèrent tous les os.
\TextTitle{Les merveilles de Yahweh proclamées aux nations}
\VS{25}Après cela, le roi Darius écrivit à tous les peuples, à toutes les nations, aux hommes de toutes les langues, qui habitent sur toute la terre : Que votre paix soit multipliée !
\VS{26}J'ordonne que dans toute l'étendue de mon royaume on ait de la crainte et de la frayeur pour le Dieu de Daniel, car c'est le Dieu vivant, et il subsiste éternellement ; son Royaume ne sera jamais détruit, et sa domination durera jusqu'à la fin\FTNT{Lu. 1:33 ; Es. 11.}.
\VS{27}Il sauve et délivre, il fait des prodiges et des merveilles dans les cieux et sur la terre, et il a délivré Daniel de la puissance des lions.
\VS{28}Ainsi Daniel prospéra sous le règne de Darius, et sous le règne de Cyrus, roi de Perse.
\Chap{7}
\TextTitle{Songe des quatre animaux ; Explication des visions de Daniel}
\VerseOne{}La première année de Belschatsar, roi de Babylone, Daniel eut un songe et des visions de sa tête, étant dans sa couche. Ensuite il écrivit le songe, et il relata les principales choses.
\VS{2}Daniel donc parla et dit : Je regardais dans ma vision nocturne, et voici, les quatre vents des cieux se levèrent avec impétuosité sur la grande mer.
\VS{3}Puis quatre grandes bêtes\FTNT{Les quatre bêtes représentent les quatre empires historiques. 
Le lion :
V. 4 : Le premier animal est un lion, il représente l'Empire néo-babylonien (625 – 539 av. J.-C.). Les ailes suggèrent la rapidité de la conquête babylonienne (Ha.1:6-8 ; Jé. 4:13). En 30 ans, l'Arabie, la Judée, la Syrie et la Phénicie furent conquises.
Les ailes arrachées annonçant l'arrêt des grandes conquêtes avec la mort de Nebucadnetsar. 
Le cœur d'homme donné au lion symbolise la conversion de Nebucadnetsar et le changement dans l'attitude des rois babyloniens (Da.4:30-31 ; 2 R. 25:27-30).
L'ours :
V. 5 : Le deuxième animal est un ours. Il représente l'empire médo-perse (539–331 av. J.-C.) qui succéda à l'Empire Babylonien.
Le fait que l'ours se tienne sur un côté indique que les Mèdes sont soumis aux Perses qui sont les véritables maîtres de l'empire. Les trois côtes dans la gueule de l'ours symbolisent trois grandes conquêtes médo-perses : la Lydie (546 av. J.-C.), la Babylonie (539 av. J.-C.) et l'Egypte (524 av. J.-C.).
Le léopard :
V. 6 : Le troisième animal est un léopard, qui représente l'Empire gréco-macédonien (331 – 146 av. J.-C.). En 331 av. J.-C., le coup de grâce est donné aux Médo-Perses à la bataille d'Arbèles.
Les quatre ailes symbolisent la grande rapidité des conquêtes. Quand Alexandre le Grand mourut à l'âge de 33 ans, il avait le plus grand Empire jamais vu jusqu'à l'époque. Ses conquêtes s'étendaient jusqu'en Inde !
Les quatre têtes symbolisent quatre de ses généraux qui, à la mort d'Alexandre, se partagèrent l'immense empire : Cassandre en Grèce et en Macédoine, Lysimaque en Thrace et en Asie Mineure, Séleucus en Syrie et en orient, Ptolémée en Égypte.
Très rapidement, la Palestine, qui se trouvait au croisement des routes, fut l'objet de rivalités entre les généraux et leurs successeurs. Après quelques années de stabilité, les généraux luttèrent entre eux jusqu'au maintien de deux dynasties : les Séleucides, au nord, et les Lagides, au sud, en Égypte. Cela dura jusqu'à l'apparition de l'Empire romain.
La quatrième bête, est différente des autres
V. 7, 19, 24 : La quatrième bête est extraordinaire, terrible, effrayante, elle ne porte même pas de nom ! Elle représente l'Empire romain qui succéda à l'Empire gréco-macédonien (146 av. J.-C. – 476 ap. J.-C.). En 168 av. J.-C., la Macédoine passa sous le contrôle de Rome, puis, en 146 av. J.-C., c'est au tour de la Grèce de devenir une province romaine.
 Le quatrième empire ne peut être que celui de Rome comme l'enseigne l'histoire de l'antiquité. 
Dès le quatrième siècle, l'Empire romain fut assailli par les tribus barbares venues du nord (Alamans, Wisigoths, Goths, Vandales, Burgondes, Ostrogoths, etc.) et, en 476, le dernier empereur romain d'occident, Romulus Augustule, fut chassé par le roi barbare Odoacre (Goth). L'Empire romain n'est plus.
Les orteils en partie de fer et en partie d'argile représentent les nations européennes issues de la fragmentation de l'Empire romain qui a eu lieu le 4 septembre 476.} montèrent de la mer, différentes les unes des autres.
\TextTitle{Premier empire universel: Babylone\FTNTT{Cp. Da. 2:37-38.}}
\VS{4}La première était semblable à un lion, et avait des ailes d'aigle ; je la regardai jusqu'à ce que les plumes de ses ailes furent arrachées ; elle fut enlevée de terre et dressée sur ses pieds comme un homme, et un cœur d'homme lui fut donné.
\TextTitle{Deuxième empire: Les Mèdes et les Perses\FTNTT{Cp. Da. 2:39 ; 8:20.}}
\VS{5}Et voici, une deuxième bête était semblable à un ours, et se tenait sur un côté ; il avait trois côtes dans la gueule entre ses dents ; et on lui disait ainsi : Lève-toi, mange beaucoup de chair.
\TextTitle{Troisième empire: La Grèce\FTNTT{Cp. Da. 2:39 ; 8:21-22 ; 11:2-4.}}
\VS{6}Après cela je regardai, et voici une autre bête, semblable à un léopard, qui avait sur son dos quatre ailes d'oiseau, et cette bête avait quatre têtes, et la domination lui fut donnée.
\TextTitle{Quatrième empire: Rome\FTNTT{Cp. Da. 2:40-43 ; 7:23-24 ; 9:26.}}
\VS{7}Après cela, je regardai dans mes visions nocturnes, et voici, il y avait une quatrième bête, terrible, épouvantable et extraordinairement forte ; elle avait de grandes dents de fer, elle mangeait, brisait, et elle foulait à ses pieds ce qui restait ; elle était différente de toutes les bêtes qui avaient été avant elle, et elle avait dix cornes.
\TextTitle{Les dix cornes et la petite corne\FTNTT{Da. 7:24-27.}}
\VS{8}Je considérai ses cornes, et voici, une autre petite corne sortit du milieu d'elles, et trois des premières cornes furent arrachées par elle ; et voici, elle avait des yeux comme des yeux d'homme, et une bouche qui proférait de grandes choses.
\TextTitle{Le règne de Yahweh, l'Ancien des jours\FTNTT{Cp. Mt. 24:27-30 ; 25:31-34 ; Ap. 19:11-21.}}
\VS{9}Je regardai jusqu'à ce que les trônes soient placés. Et l'Ancien des jours s'assit. Son vêtement était blanc comme la neige, et les cheveux de sa tête étaient comme de la laine pure ; son trône était des flammes de feu, et ses roues un feu ardent.
\VS{10}Un fleuve de feu coulait et sortait de devant lui. Mille milliers le servaient, et dix mille millions se tenaient en sa présence. Le jugement se tint, et les livres furent ouverts\FTNT{Ap. 5:11 ; Ps. 68:18 ; 1 R. 22:19.}.
\VS{11}Je regardai alors, à cause du bruit des paroles arrogantes que proférait la corne ; et tandis que je regardais, la bête fut tuée, et son corps fut détruit et livré pour être brûlé au feu.
\VS{12}Les autres bêtes furent dépouillées de leur domination, mais une prolongation de vie leur fut accordée jusqu'à un temps déterminé.
\TextTitle{La domination du Fils de l'homme est éternelle\FTNTT{Cp. Ap. 5:1-14.}}
\VS{13}Je regardai encore dans les visions nocturnes, et je vis, comme le Fils de l'homme, qui venait avec les nuées des cieux, et il vint jusqu'à l'Ancien des jours, et se tint devant lui\FTNT{Ap. 19:14 ; Jud. 1:14.}.
\VS{14}Et il lui donna la domination, la gloire et le règne ; et tous les peuples, les nations et les langues le serviront. Sa domination est une domination éternelle qui ne passera point, et son règne ne sera jamais détruit.
\TextTitle{Interprétation de la vision du quatrième animal}
\VS{15}Moi, Daniel, j'eus l'esprit troublé au-dedans de moi, et les visions de ma tête m'effrayèrent.
\VS{16}Je m'approchai de l'un des assistants, et lui demandai ce qu'il y avait de vrai dans toutes ces choses. Il me parla, et me donna l'interprétation de ces choses, en disant :
\VS{17}Ces quatre grandes bêtes sont quatre rois, qui s'élèveront de la terre.
\VS{18}Mais les saints du Très-Haut recevront le Royaume, et ils posséderont le Royaume éternellement, d'éternité en éternité.
\VS{19}Alors, je désirai savoir la vérité sur la quatrième bête, qui était différente de toutes les autres, extraordinairement terrible, qui avait des dents de fer et des ongles d'airain, qui mangeait, brisait, et foulait à ses pieds ce qui restait ;
\VS{20}et sur les dix cornes qu'elle avait à la tête, et sur l'autre corne qui était sortie et devant laquelle trois étaient tombées, sur cette corne qui avait une bouche parlant avec arrogance, et une plus grande apparence que celle de ses associées.
\VS{21}Je regardai comment cette corne faisait la guerre aux saints et l'emportait sur eux\FTNT{Ap. 13:2-7.},
\VS{22}jusqu'au moment où l'Ancien des jours vint donner droit aux saints du Très-Haut, et que le temps arriva où les saints furent en possession du Royaume.
\VS{23}Il me parla donc ainsi : La quatrième bête est un quatrième royaume qui sera sur la terre, différent de tous les royaumes, et qui dévorera toute la terre, la foulera, et la brisera.
\TextTitle{Règne de l'homme impie et jugement de Dieu}
\VS{24}Mais les dix cornes sont dix rois qui s'élèveront de ce royaume. Un autre s'élèvera après eux, il sera différent des premiers, et il abattra trois rois.
\VS{25}Il proférera des paroles contre le Très-Haut, il harcelera les saints du Très-Haut, et il aura l'intention de changer les temps et la loi ; et les saints seront livrés entre ses mains pendant un temps, des temps, et la moitié d'un temps.
\VS{26}Mais le jugement se tiendra, et on lui ôtera sa domination, en la détruisant et la faisant périr, jusqu'à en voir la fin..
\VS{27}Afin que le règne, la domination, et la grandeur de tous les royaumes qui sont sous les cieux, soient donnés au peuple des saints du Très-Haut. Son royaume est un royaume éternel, et tous les royaumes lui seront assujettis et lui obéiront.
\VS{28}Jusqu'ici est la fin de cette affaire. Quant à moi, Daniel, mes pensées m'effrayèrent beaucoup, et ma splendeur changea en moi, toutefois je gardais cette affaire dans mon cœur.
\Chap{8}
\TextTitle{Vision du bélier et du bouc}
\VerseOne{}La troisième année du règne du roi Belschatsar, moi, Daniel, j'eus cette vision, en plus de celle que j'avais eue auparavant.
\VS{2}Je vis cette vision, et il arriva, comme je regardais, que j'étais à Suse, la capitale, dans la province d'Élam, et dans ma vision, je me trouvais près du fleuve d'Ulaï.
\VS{3}Et je levai mes yeux, je regardai, et voici, un bélier se tenait devant le fleuve, et il avait deux cornes ; et les deux cornes étaient hautes, mais l'une était plus haute que l'autre, et la plus haute s'éleva sur la dernière.
\VS{4}Je vis ce bélier qui frappait de ses cornes à l'occident, au nord, et au midi ; aucune bête ne pouvait subsister devant lui,et il n'y avait personne qui puisse délivrer de sa puissance ; et il agissait selon sa volonté et devenait grand.
\VS{5}Comme je regardais attentivement, voici, un bouc d'entre les chèvres venait de l'occident, et parcourait toute la terre à sa surface, sans la toucher ; ce bouc avait entre les yeux une corne considérable.
\VS{6}Il arriva jusqu'au bélier qui avait deux cornes et que j'avais vu se tenant devant le fleuve, et il courut sur lui dans la fureur de sa force.
\VS{7}Je le vis qui s'approchait du bélier et s'irritait contre lui ; il frappa le bélier et lui brisa les deux cornes, et le bélier n'avait aucune force pour tenir ferme contre lui ; et quand il l'eut jeté par terre, il le foula; et il n'y eut personne pour délivrer le bélier de sa puissance.
\VS{8}Alors le bouc d'entre les chèvres grandit extrêmement ; mais lorsqu'il fut puissant, sa grande corne se brisa. Quatre grandes cornes s'élevèrent pour la remplacer, aux quatre vents des cieux.
\TextTitle{La petite corne renverse la vérité}
\VS{9}De l'une d'elles sortit une petite corne\FTNT{Antiochus IV Épiphane est le fils d'Antiochos III le Grand, né vers 215 av. J.-C. Il gouverna le royaume séleucide de 175 av. J.-C. à 164 av. J.-C., date de sa mort. Ce dernier avait profané le temple de Jérusalem en sacrifiant des porcs sur l'autel (Voir commentaire en Mt. 24:15). Cette petite corne, qui fait tomber par terre une partie de l'armée des étoiles, agit de même que Satan au ciel qui avait fait chuter un tiers des étoiles, soit des anges (Ap. 12:3-4).}, qui s'agrandit beaucoup vers le midi, et vers l'orient, et vers le pays de noblesse.
\VS{10}Elle s'éleva même jusqu'à l'armée des cieux, elle fit tomber à terre une partie de l'armée et des étoiles, et elle les foula\FTNT{Es. 14:12-15 ; Ez. 28:12-19.}.
\VS{11}Et elle s'éleva même jusqu'au chef de l'armée, lui enleva le sacrifice perpétuel, et renversa la demeure de son sanctuaire.
\VS{12}L'armée fut livrée avec le sacrifice perpétuel, à cause du péché ; la corne jeta la vérité par terre, et fit de grands exploits, et prospéra.
\VS{13}Alors j'entendis un saint qui parlait ; et un autre saint disait à celui qui parlait : Jusqu'à quand durera cette vision sur le sacrifice perpétuel et sur le péché qui cause la désolation ? Jusqu'à quand le sanctuaire et l'armée seront-ils foulés ?
\VS{14}Et il me dit : Deux mille trois cents soirs et matins ; puis le sanctuaire sera purifié.
\TextTitle{La vision du bélier et du bouc interprétée}
\VS{15}Et quand à moi, Daniel, j'avais cette vision et que je désirais la comprendre, voici, quelqu'un qui avait l'apparence d'un homme se tenait devant moi.
\VS{16}Et j'entendis la voix d'un homme au milieu du fleuve Ulaï ; il cria et dit : Gabriel, explique-lui la vision.
\VS{17}Puis Gabriel vint alors près du lieu où je me tenais ; et à son approche, je fus effrayé, et je tombai sur ma face. Il me dit : Comprends, fils de l'homme, car la vision est pour le temps de la fin.
\VS{18}Comme il me parlait, je restai frappé d'étourdissement, la face contre terre. Il me toucha, et me fit tenir debout à la place où je me trouvais.
\VS{19}Et il dit : Voici, je vais t'apprendre ce qui arrivera à la fin de la colère, car il y a un temps marqué pour la fin.
\VS{20}Le bélier que tu as vu qui avait deux cornes, ce sont les rois des Mèdes et des Perses ;
\VS{21}et le bouc velu, c'est le roi de Javan\FTNT{Javan ou Grèce.} ; et la grande corne entre ses yeux, c'est le premier roi.
\VS{22}Les quatre cornes qui se sont élevées pour remplacer cette corne brisée, ce sont quatre royaumes qui s'élèveront de cette nation, mais qui n'auront pas autant de force.
\TextTitle{Le roi impie, adversaire de Dieu ; la vision scellée}
\VS{23}A la fin de leur règne, lorsque les pécheurs seront consumés, il se lèvera un roi cruel et artificieux.
\VS{24}Sa puissance s'accroîtra, mais non par sa propre force ; il fera d'incroyables ravages, il réussira dans ses entreprises, il détruira les puissants et le peuple des saints.
\VS{25}Et par la subtilité de son esprit, il fera prospérer la fraude dans sa main. Il aura de l'arrogance dans le cœur, et fera périr beaucoup d'hommes qui vivaient dans la paix, et il s'élèvera contre le Prince des princes ; mais il sera brisé, sans l'effort d'aucune main.
\VS{26}Et la vision du soir et du matin, dont il s'agit, est véritable. Mais toi, scelle la vision, car elle se rapporte à un temps éloigné.
\VS{27}Moi Daniel, je fus tout défait et malade pendant quelques jours ; puis je me levai, et je m'occupai des affaires du roi. J'étais étonné de la vision, et personne n'en eut connaissance.
\Chap{9}
\TextTitle{Supplications de Daniel à Yahweh}
\VerseOne{}La première année de Darius, fils d'Assuérus, de la race des Mèdes, lequel était établi roi sur le royaume des Chaldéens.
\VS{2}La première année, dis-je, de son règne, moi Daniel, je discernai par les livres, que le nombre des années dont Yahweh avait parlé au prophète Jérémie\FTNT{Jé. 25:11.} pour finir les désolations de Jérusalem, était de soixante et dix ans. 
\VS{3}Et je tournai ma face vers le Seigneur Dieu, pour le chercher par la prière et des supplications, avec jeûne, et le sac et la cendre.
\VS{4}Je priai Yahweh, mon Dieu, et je lui fis ma confession : Ah ! Seigneur, Dieu grand et redoutable, toi qui gardes ton alliance et qui fais miséricorde à ceux qui t'aiment et qui gardent tes commandements !
\VS{5}Nous avons péché, nous avons commis l'iniquité, nous avons agi méchamment, nous avons été rebelles, et nous nous sommes détournés de tes commandements et de tes ordonnances.
\VS{6}Nous n'avons pas écouté tes serviteurs, les prophètes, qui ont parlé en ton Nom à nos rois, à nos chefs, à nos pères, et à tout le peuple du pays.
\VS{7}Ô Seigneur ! A toi est la justice, et à nous la confusion de face, en ce jour, aux hommes de Juda, aux habitants de Jérusalem, et à tout Israël, à ceux qui sont près et à ceux qui sont loin, dans tous les pays où tu les as dispersés, à cause des infidélités dont ils se sont rendus coupables envers toi\FTNT{Né. 9:30 ; Ps. 106:6 ; La. 3:42.}.
\VS{8}Seigneur, à nous est la confusion de face, à nos rois, à nos chefs, et à nos pères, parce que nous avons péché contre toi.
\VS{9}Auprès du Seigneur, notre Dieu, la miséricorde et le pardon, car nous avons été rebelles envers lui.
\VS{10}Nous n'avons pas écouté la voix de Yahweh, notre Dieu, pour marcher dans ses lois, qu'il a mises devant nous par le moyen de ses serviteurs, les prophètes.
\VS{11}Tout Israël a transgressé ta loi, et s'est détourné pour ne pas écouter ta voix. Alors se sont répandues sur nous les malédictions et les imprécations qui sont écrites dans la loi de Moïse, serviteur de Dieu, parce que nous avons péché contre Dieu\FTNT{Lé. 26:14-33 ; De. 27:15-33.}.
\VS{12}Il a accompli les paroles qu'il avait prononcées contre nous, et contre nos chefs qui nous ont gouvernés, et il a fait venir sur nous un grand mal, et il n'en est jamais arrivé sous le ciel entier un semblable à celui qui est arrivé à Jérusalem.
\VS{13}Comme cela est écrit dans la loi de Moïse, ce mal est venu sur nous ; et nous n'avons pas imploré Yahweh, notre Dieu, pour nous détourner de nos iniquités, et pour nous rendre attentifs à ta vérité.
\VS{14}Yahweh a veillé sur le mal que nous avons fait et il l'a fait venir sur nous ; car Yahweh, notre Dieu, est juste dans toutes les œuvres qu'il a faites, vu que nous n'avons point obéi à sa voix.
\VS{15}Or maintenant, Seigneur, notre Dieu ! Toi qui as tiré ton peuple du pays d'Egypte par ta main puissante, et qui t'es acquis un Nom comme il l'est aujourd'hui, nous avons péché, nous avons été méchants.
\VS{16}Seigneur, je te prie que selon ta justice, que ta colère et ton indignation se détournent de ta ville de Jérusalem, de la montagne de ta sainteté ; car à cause de nos péchés et des iniquités de nos pères, Jérusalem et ton peuple sont en opprobre à tous ceux qui nous entourent.
\VS{17}Maintenant donc, ô notre Dieu, écoute la prière et les supplications de ton serviteur, et pour l'amour du Seigneur, fais briller ta face sur ton sanctuaire dévasté.
\VS{18}Mon Dieu ! Prête l'oreille, et écoute ; ouvre tes yeux, et regarde nos ruines, et la ville sur laquelle ton Nom a été invoqué ; car ce n'est pas à cause de notre justice que nous te présentons nos supplications, c'est à cause de tes grandes compassions.
\VS{19}Seigneur, exauce, Seigneur pardonne, Seigneur sois attentif, et opère ; ne tarde pas, par amour pour toi, ô mon Dieu ! Car ton Nom a été invoqué sur ta ville, et sur ton peuple.
\VS{20}Je parlais encore, je priais, je confessais mon péché, et le péché de mon peuple d'Israël, et je présentais ma supplication à Yahweh, mon Dieu, en faveur de la sainte montagne de mon Dieu.
\TextTitle{Les soixante-dix semaines}
\VS{21}Je parlais encore dans ma prière, quand l'homme Gabriel, que j'avais vu précédemment dans une vision, s'approcha de moi d'un vol rapide au moment de l'offrande du soir.
\VS{22}Il m'instruisit, et s'entretint avec moi. Il me dit : Daniel, je suis venu maintenant pour ouvrir ton intelligence.
\VS{23}La parole est sortie dès le commencement de tes supplications, et je suis venu pour te la déclarer, car tu es un bien-aimé. Sois attentif à la parole, et comprends la vision.
\VS{24}Il y a soixante-dix semaines\FTNT{Le verset 24 concerne la chronologie de l'accomplissement de la prophétie de Jérémie (25 : 11). Les soixante-dix semaines auxquelles elle fait allusion représentent une période de 490 ans, conformément au principe biblique prophétique selon lequel un jour prophétique équivaut à une année (No. 14:33-34 ; Ez. 4:4-6). Dans les versets 25 et 27, les soixante-dix semaines sont divisées en trois périodes : 7 semaines (49 ans), 62 semaines (434 ans), et une semaine (7 ans). Les soixante-dix semaines devaient débuter au moment où la parole a annoncé que Jérusalem serai rebâtie (v. 25). En 445 avant notre ère, dans la vingtième année de son règne, le roi Artaxerxès publia un décret permettant à Esdras de retourner à Jérusalem pour achever la reconstruction de la ville (Esd. 7:6-10 ; Esd. 9:9 ; Né 2:5). Il est attesté par l'histoire profane que cette date est le point de départ de la soixante-dixième semaine de Daniel. Les 69 premières semaines vont jusqu'au Messie conducteur. La semaine qui reste (7 ans) concerne la période du règne de l'Antéchrist, celle-ci est divisée en deux période : trois ans et demi de fausse paix (1 Th. 5:3), et trois ans et demi concernent la Grande Tribulation (Ap. 7:9-17 ; Ap. 11:1-3 ; Ap. 12:6 ; Ap. 13:5).} fixées sur ton peuple et sur ta ville sainte, pour abolir la transgression et mettre fin aux péchés, faire la propitiation pour l'iniquité, pour amener la justice éternelle, pour mettre le sceau à la vision et à la prophétie et pour oindre le Saint des saints.
\VS{25}Tu sauras donc et tu comprendras, que depuis le moment où la parole a annoncé que Jérusalem sera rebâtie jusqu'au Messie, le Conducteur, il y a sept semaines et soixante-deux semaines ; et les places et les brèches seront rebâties, mais en des temps d'angoisse.
\VS{26}Et après ces soixante-deux semaines, le Messie sera retranché, mais non pas pour lui. Le peuple du chef qui viendra, détruira la ville et le sanctuaire, et sa fin arrivera comme par une inondation ; il est déterminé que les dévastations dureront jusqu'à la fin de la guerre.
\VS{27}Et il confirmera l'alliance à plusieurs pour une semaine, et à la moitié de cette semaine il fera cesser le sacrifice, et l'offrande ; puis par le moyen des ailes abominables, qui causeront la désolation, même jusqu'à une consomption déterminée, la désolation fondra sur le désolé.
\Chap{10}
\TextTitle{Daniel voit la gloire du Messie}
\VerseOne{}La troisième année de Cyrus, roi de Perse, une parole fut révélée à Daniel, qu'on nommait Beltschatsar. Cette parole est véritable et annonce une grande guerre. Il fut attentif à cette parole, et il eut l'intelligence de la vision.
\VS{2}En ce temps-là, moi Daniel, je fus dans le deuil pendant trois semaines entières.
\VS{3}Je ne mangeai aucun mets délicat, il n'entra ni viande ni vin dans ma bouche, et je ne m'oignis point, jusqu'à ce que ces trois semaines entières soient accomplies.
\VS{4}Le vingt-quatrième jour du premier mois, j'étais au bord du grand fleuve qui est Hiddékel.
\VS{5}Je levai les yeux, et je regardai, et voici, il y avait un homme vêtu de lin, et ayant sur les reins une ceinture d'or fin d'Uphaz.
\VS{6}Son corps était comme de chrysolithe, et son visage brillait comme l'éclair, ses yeux étaient comme des flammes de feu, ses bras et ses pieds ressemblaient à de l'airain poli, et le son de sa voix était comme le bruit d'une multitude de gens\FTNT{Ap. 1:13-15.}.
\VS{7}Moi, Daniel, je vis seul la vision, et les hommes qui étaient avec moi ne la virent point ; mais ils furent saisis d'une grande frayeur, et ils s'enfuirent pour se cacher.
\VS{8}Je restai seul, et je vis cette grande vision ; les forces me manquèrent, mon visage changea de couleur et fut tout défait, et je ne conservai aucune vigueur.
\VS{9}J'entendis le son de ses paroles ; et comme j'entendais le son de ses paroles, je tombai frappé d'étourdissement, la face contre terre.
\TextTitle{Le combat du monde spirituel}
\VS{10}Et voici, une main me toucha et me fit mettre sur mes genoux, et sur les paumes de mes mains.
\VS{11}Puis il me dit : Daniel, homme aimé de Dieu, sois attentif aux paroles que je vais te dire, et tiens-toi debout à la place où tu es ; car je suis maintenant envoyé vers toi. Lorsqu'il m'eut ainsi parlé, je me tins debout en tremblant.
\VS{12}Il me dit : Ne crains rien, Daniel, car dès le premier jour où tu as appliqué ton cœur à comprendre, et à t'humilier devant ton Dieu, tes paroles ont été exaucées, et c'est à cause de tes paroles que je viens.
\VS{13}Mais le chef du royaume de Perse m'a résisté vingt et un jours ; mais voici, Micaël, l'un des principaux chefs, est venu à mon secours, et je suis demeuré là auprès des rois de Perse.
\VS{14}Je viens maintenant pour te faire connaître ce qui doit arriver à ton peuple dans les derniers jours, car la vision s'étend jusqu'à ces temps-là.
\VS{15}Pendant qu'il m'adressait ces paroles, je mis mon visage contre terre, et je gardai le silence.
\VS{16}Et voici, quelqu'un qui avait l'apparence des fils de l'homme toucha mes lèvres. J'ouvris la bouche, je parlai, et je dis à celui qui se tenait devant moi : Mon seigneur ! La vision m'a rempli d'effroi, et j'ai perdu toute vigueur.
\VS{17}Comment le serviteur de mon seigneur pourrait-il parler avec mon seigneur ? Maintenant les forces me manquent, et je n'ai plus de souffle.
\VS{18}Alors celui qui avait l'apparence d'un homme me toucha encore, et me fortifia.
\VS{19}Puis il me dit : Ne crains rien, homme bien-aimé, que la paix soit avec toi ! Fortifie-toi, fortifie-toi ! Et comme il me parlait, je repris des forces, et je dis : Que mon seigneur parle, car tu m'as fortifié.
\VS{20}Il me dit : Ne sais-tu pas pourquoi je suis venu vers toi ? Maintenant je m'en retournerai pour combattre le chef de Perse ; et quand je partirai, voici, le chef de Javan viendra.
\VS{21}Mais je veux te faire connaître ce qui est écrit dans le livre de vérité. Et il n'y a personne qui me soutienne contre ceux-là, excepté Micaël, votre chef.
\Chap{11}
\TextTitle{Succession des monarques jusqu'à l'homme impie\FTNTT{Da. 11:1 - 12:13.}}
\VerseOne{}Et moi, dans la première année de Darius, le Mède, je me tenais auprès de lui pour l'aider et le fortifier.
\VS{2}et maintenant, je vais te faire connaître la vérité : Voici, il y aura encore trois rois en Perse. Le quatrième amassera plus de richesses que les autres ; et quand il sera puissant par ses richesses, il soulèvera tout le monde contre le royaume de Javan.
\VS{3}Mais il s'élèvera un vaillant roi\FTNT{Ce vaillant roi est Alexandre le Grand qui règna de 336 à 323 av. J.-C.}, qui dominera avec une grande puissance, et fera ce qu'il voudra.
\VS{4}Et sitôt qu'il sera élevé, son royaume sera brisé et sera divisé\FTNT{A la mort d'Alexandre le Grand, ses quatre principaux généraux se partagèrent l'empire : 
-Lysimaque régna sur l'Asie mineure. 
-Cassandre régna sur la Grèce et la Macédoine.
-Seleucos régna en Syrie, en Babylonie et sur toutes les régions à l'est jusqu'aux Indes. 
-Ptolémée régna sur l'Égypte, la Judée et une partie de la Syrie.} vers les quatre vents des cieux ; il ne passera point à ses descendants, et n'aura pas la même puissance qu'il a exercée, car son royaume sera déchiré, et il passera à d'autres qu'à eux.
\VS{5}Le roi du midi\FTNT{Le roi du midi est Ptolémée 1er Soter (règne : 323-285 av. J.-C.), le chef plus fort que lui est Séleucus 1er Nicator (règne : 305-281 av. J.-C.).} deviendra fort et puissant. Mais un de ses chefs [roi de Javan] sera plus puissant que lui et dominera ; sa domination sera puissante.
\VS{6}Au bout de quelques années, ils s'allieront, et la fille du roi du midi viendra vers le roi du nord pour redresser les affaires. Mais elle ne conservera pas la force de son bras, et il ne résistera pas, ni lui ni son bras ; elle sera livrée avec ceux qui l'auront amenée, avec son père et avec celui qui aura été son soutien dans ce temps-là.
\VS{7}Mais un rejeton de ses racines s'élèvera pour le remplacer\FTNT{Ptolémée III Evergète (règne : 246-222 av. J.-C.)} ; il viendra à l'armée, il entrera dans les forteresses du roi du nord, il en disposera à son gré, et il se rendra puissant.
\VS{8}Et même il emmènera captifs en Egypte leurs dieux, avec leurs images de fonte et avec leurs vases précieux d'argent et d'or. Puis il restera quelques années éloigné du roi du nord.
\VS{9}Et celui-ci marchera contre le royaume du roi du midi, et retournera dans son pays.
\VS{10}Ses fils\FTNT{Ses fils : Ce sont les deux rois de Syrie, Seuleucus III Ceraunus (règne : 225-223 av. J.-C.), et Antiochus III le Grand (223-187 av. J.-C.).} entreront en guerre et rassembleront une multitude nombreuse de troupes ; l'un d'eux s'avancera et se répandra comme un torrent, débordera, puis reviendra ; et il poussera la guerre jusqu'à la forteresse du roi du midi.
\VS{11}Et le roi du midi sera irrité, il sortira et combattra contre lui, savoir contre le roi du nord ; il soulèvera une grande multitude, et les troupes du roi du nord seront livrées entre les mains du roi du midi.
\VS{12}Et après avoir défait cette multitude, le cœur du roi s'élèvera ; il fera tomber des milliers, mais il ne triomphera pas.
\VS{13}Car le roi du nord reviendra et rassemblera une plus grande multitude que la première ; au bout de quelque temps, de quelques années, il viendra avec une grande armée et de grandes richesses.
\VS{14}Et en ce temps-là, plusieurs s'élèveront contre le roi du midi ; et des hommes violents parmi ton peuple se révolteront pour accomplir la vision, mais ils succomberont.
\VS{15}Le\FTNT{Le roi du nord est Séleucos IV Philopator (règne :187-175 av. J.-C.)} roi du nord viendra, il élèvera des terrasses, et prendra les villes fortes. Les bras du midi et l'élite du roi ne résisteront pas, ils manqueront de force pour résister.
\VS{16}Celui qui marchera contre lui fera ce qu'il voudra, et personne ne lui résistera ; il s'arrêtera dans le pays de noblesse, exterminant ce qui tombera sous sa main.
\VS{17}Puis il tournera sa face pour entrer avec la force de tout son royaume, et fera un accord avec le roi du midi, et il lui donnera sa fille pour femme, pour ruiner le royaume ; mais cela ne tiendra pas, et elle ne sera pas pour lui. 
\VS{18}Puis il tournera ses vues vers les îles, et il en prendra plusieurs ; mais un chef mettra fin à l'opprobre qu'il voulait lui attirer, et le fera retomber sur lui.
\VS{19}Il se dirigera ensuite vers les forteresses de son pays ; et il chancellera, il tombera, et on ne le trouvera plus.
\VS{20}Et un autre sera établi à sa place, qui fera passer un exacteur dans l'ornement du royaume, et en peu de jours il sera brisé, et ce ne sera ni par la colère ni par la guerre.
\TextTitle{Usage de la tromperie pour régner}
\VS{21}Et sa place il en sera établi un autre qui sera méprisé, auquel on ne donnera pas l'honneur royal ; mais il viendra en paix, et il s'emparera du royaume par des flatteries.
\VS{22}Les troupes qui se répandront comme un torrent seront submergées devant lui, et brisées, de même qu'un chef de l'alliance.
\VS{23}Mais après les accords faits avec lui, il usera de tromperie, et il montera, et il aura le dessus avec peu de gens.
\VS{24} Il entrera tranquillement dans les lieux les plus riches de la province, et il fera ce que n'avaient pas fait ses pères, ni les pères de ses pères ; il distribuera le butin, le pillage et les richesses ; et il formera des desseins contre les places fortes, et cela jusqu'à un certain temps.
\VS{25}Puis il réveillera sa force et son coeur contre le roi du midi avec une grande armée. Et le roi du midi s'avancera en bataille avec une très grande et très forte armée ; mais il ne résistera pas, car on formera des complots contre lui.
\VS{26}Ceux qui mangent les mets de sa table le mettront en pièces ; son armée se répandra comme un torrent, et beaucoup de gens tomberont blessés à mort.
\VS{27}Et les deux rois chercheront en leur cœur à se nuire, et à la même table ils parleront avec fausseté. Mais cela ne réussira pas ; car la fin ne viendra qu'au temps marqué.
\VS{28}Après quoi il retournera dans son pays avec de grandes richesses ; et son cœur sera contre la sainte alliance, il agira contre elle, puis retournera dans son pays.
\VS{29}Ensuite il retournera au temps fixé, et il viendra contre le midi ; mais cette dernière expédition ne sera pas comme la précédente.
\VS{30}Car les navires de Kittim viendront contre lui ; affligé, il rebroussera chemin. Puis irrité contre la sainte alliance, il agira contre elle, il retournera et s'entendra avec les apostats de la sainte alliance.
\VS{31}Et les forces seront de son côté, et on profanera le sanctuaire qui est la forteresse, et on fera cesser le sacrifice perpétuel, et on y dressera l'abomination qui causera la désolation.
\VS{32}Et il corrompra par des flatteries ceux qui agissent méchamment à l'égard de l'alliance. Mais ceux du peuple qui connaîtront leur Dieu agiront avec courage.
\VS{33}Et les plus intelligents parmi le peuple donneront instruction à plusieurs. Il en est qui succomberont pour un temps à l'épée et à la flamme, à la captivité et au pillage.
\VS{34}Dans le temps où ils succomberont, ils seront un peu secourus, et plusieurs se joindront à eux par hypocrisie.
\VS{35}Et quelques-uns des hommes intelligents succomberont, afin qu'ils soient épurés, purifiés et blanchis, jusqu'au temps de la fin, car elle n'arrivera qu'au temps marqué.
\TextTitle{Blasphème du roi contre Yahweh, le Dieu des dieux}
\VS{36}Le roi fera ce qu'il voudra, il s'élèvera, il se glorifiera au-dessus de tous les dieux ; il proférera des choses étranges contre le Dieu des dieux, il prospérera jusqu'à ce que la colère soit consommée, car ce qui est décrété sera exécuté.
\VS{37}Il n'aura égard ni aux dieux de ses pères, ni à l'objet du désir des femmes ; il n'aura égard à aucun dieu ; car il s'élèvera au-dessus de tout.
\VS{38}Mais, à la place, il honorera le dieu Mahuzzim ; ce dieu que ses pères n'ont pas connu, il rendra des hommages avec de l'or et de l'argent, et des pierres précieuses, et des objets de prix.
\VS{39}C'est avec le dieu étranger qu'il agira contre les lieux les plus fortifiés ; et il comblera d'honneurs ceux qui le reconnaîtront, il les fera dominer sur plusieurs, il leur partagera des terres à prix d'argent.
\VS{40}Au temps de la fin, le roi du midi se heurtera contre lui de ses cornes. Et le roi du nord fondra sur lui comme une tempête, avec des chars et des cavaliers, et avec de nombreux navires ; il s'avancera dans les terres, se répandra comme un torrent et débordera.
\VS{41}Il entrera dans le pays de noblesse, et plusieurs pays succomberont ; mais Edom, Moab et les principaux des enfants d'Ammon seront délivrés de sa main.
\VS{42}Il étendra sa main sur ces pays-là, et le pays d'Egypte n'échappera point.
\VS{43}Il se rendra maître des trésors d'or et d'argent, et de toutes les choses précieuses de l'Egypte ; les Libyens et les Ethiopiens seront à sa suite.
\VS{44}Mais des nouvelles de l'orient et du nord viendront le troubler, et il partira avec une grande fureur, pour détruire et exterminer beaucoup de gens.
\VS{45}Il dressera les tentes de son palais entre les mers, vers la glorieuse et sainte montagne. Puis il arrivera à la fin, et personne ne lui donnera du secours.
\Chap{12}
\TextTitle{La résurrection pour le jugement éternel}
\VerseOne{}Or, en ce temps-là Michaël, ce grand Chef qui tient ferme pour les enfants de ton peuple, tiendra ferme ; et ce sera un temps de détresse, tel qu'il n'y en a point eu de semblable depuis que les nations existent jusqu'à ce temps-là. En ce temps-là, ceux de ton peuple qui seront trouvés inscrits dans le livre seront sauvés.
\TextTitle{Les deux résurrections}
\VS{2}Plusieurs de ceux qui dorment dans la poussière de la terre se réveilleront\FTNT{Il est question ici de la résurrection. Tout d'abord, il y aura la résurrection des morts en Christ, lors du retour de Jésus-Christ (1 Th. 4:12-17). Ensuite, il y aura celle de tous les saints lors du retour de Christ avec l'Eglise (Ap. 19 et 20). Enfin, la dernière résurrection interviendra à l'issue du millénium. Il s'agit de la résurrection des impies (Ap. 20:11-15). Voir également Jn. 5 : 24-29 ; Jn. 11:25.}, les uns pour la vie éternelle, et les autres pour l'opprobre, pour l'infamie éternelle.
\VS{3}Ceux qui auront été intelligents, brilleront comme la splendeur du ciel, et ceux qui auront amené plusieurs à la justice brilleront comme les étoiles, à toujours et à perpétuité\FTNT{Mt. 13:43.}.
\TextTitle{Dernières paroles de Yahweh à Daniel ; le livre scellé jusqu'au temps de la fin}
\VS{4}Mais toi, Daniel, tiens secrètes ces paroles, et scelle le livre jusqu'au temps de la fin. Plusieurs le liront et la connaissance augmentera\FTNT{Ap. 10:4 ; Ap. 5:2.}.
\VS{5}Et moi, Daniel, je regardai, et voici, deux autres hommes se tenaient debout, l'un en deçà du bord du fleuve, et l'autre au-delà du bord du fleuve.
\VS{6}L'un d'eux dit à l'homme vêtu de lin, qui se tenait au-dessus des eaux du fleuve : Quand sera la fin de ces merveilles ?
\VS{7}Et j'entendis l'homme vêtu de lin, qui se tenait au-dessus des eaux du fleuve ; il leva sa main droite et sa main gauche vers les cieux, et il jura par celui qui vit éternellement que ce sera dans un temps, des temps, et la moitié d'un temps, et que toutes ces choses finiront quand la force du peuple saint sera entièrement brisée.
\VS{8}J'entendis, mais je ne compris pas ; et je dis : Mon seigneur, quelle sera l'issue de ces choses ?
\VS{9}Il répondit : Va, Daniel, car ces paroles sont tenues secrètes et scellées jusqu'au temps de la fin.
\VS{10}Plusieurs seront purifiés, blanchis et éprouvés ; mais les méchants agiront avec méchanceté, et aucun des méchants ne comprendra, mais les sages comprendront.
\VS{11}Depuis le temps où cessera le sacrifice perpétuel et où sera dressée l'abomination de la désolation, il y aura mille deux cent quatre-vingt-dix jours\FTNT{Mt. 24:15 ; Mc. 13:14 ; Lu. 21:20.}.
\VS{12}Heureux celui qui attendra et qui parviendra jusqu'à mille trois cent trente-cinq jours.
\VS{13}Mais toi, marche vers ta fin ; néanmoins tu te reposeras, et tu seras debout pour ton héritage à la fin des jours.
\PPE{}
\end{multicols}

%\clearpage\ShortTitle{Esdras}\BookTitle{Esdras}\BFont
\noindent\hrulefill
{\footnotesize
\textit{
\bigskip
{\centering{}
\\Auteur : Esdras
\\(Heb : Ezrah)
\\Signification : Secours
\\Thème : Edit de Cyrus et reconstruction du temple 
\\Date de rédaction : 5ème siècle av. J.-C.\\}
}
%\bigskip
\textit{
\\Conformément aux prophéties reçues par Esaïe et Jérémie, Yahweh toucha le cœur du roi Cyrus afin de renvoyer les fils d’Israël sur leur terre avec la mission de reconstruire le temple détruit quelques décennies auparavant. Ce livre montre comment Dieu ramena glorieusement son peuple à Jérusalem et retrace la reconstruction du temple ainsi que les épreuves ayant accompagné ce projet. Il traite des réformes sociales et religieuses mises en place dans le cadre d’un retour total à Yahweh.\bigskip
}
}
\par\nobreak\noindent\hrulefill
\begin{multicols}{2}
\Chap{1}
\TextTitle{Publication de Cyrus}
\VerseOne{}La première année de Cyrus\FTNT{538 av. J.-C.}, roi de Perse, afin que la parole de Yahweh prononcée par la bouche de Jérémie\FTNT{Jé. 25:12 ; 29:10 ; 33:7-10.} soit accomplie,  Yahweh réveilla l'esprit de Cyrus, roi de Perse, qui fit publier par écrit et de vive voix dans tout son royaume, en disant :
\VS{2}Ainsi parle Cyrus, roi de Perse : Yahweh, le Dieu des cieux, m'a donné tous les royaumes de la terre, et il m'a ordonné de lui bâtir une maison à Jérusalem, en Juda.
\VS{3}Qui d'entre vous est de son peuple, qui veut s'y employer ? Que son Dieu soit avec lui, qu'il monte à Jérusalem, en Juda, et qu'il rebâtisse la maison de Yahweh, le Dieu d'Israël ! C'est le Dieu qui est à Jérusalem.
\VS{4}Dans tout lieu où séjournent des restes du peuple,  les gens du lieu leur donneront de l’argent, de l’or, des biens, et du bétail, avec des offrandes volontaires pour la maison du Dieu qui est à Jérusalem.
\TextTitle{Cyrus rend les ustensiles}
\VS{5}Alors les chefs des familles de Juda et de Benjamin, les sacrificateurs et les Lévites, tous ceux dont Dieu réveilla l'esprit, se levèrent afin de monter pour rebâtir la maison de Yahweh à Jérusalem.
\VS{6}Tous ceux qui étaient autour d'eux les encouragèrent, leur fournissant des objets d'argent, d'or, des biens, du bétail, et des choses précieuses, outre toutes les offrandes volontaires.
\VS{7}Le roi Cyrus prit les ustensiles de la maison de Yahweh, que Nebucadnetsar avait emportés\FTNT{2 R. 24:13 ; 2 Ch 36:7.} de Jérusalem et mis dans la maison de son dieu.
\VS{8}Cyrus, roi de Perse, les fit sortir par Mithredath, le trésorier, qui les remit à Scheschbatsar, prince de Juda.
\VS{9}Et voici leur nombre : Trente bassins d'or, mille bassins d'argent, vingt-neuf couteaux,
\VS{10}trente coupes d'or, quatre cent dix coupes d'argent de second ordre, et d'autres ustensiles par milliers.
\VS{11}Tous les ustensiles d'or et d'argent étaient de cinq mille quatre cents. Scheschbatsar emporta le tout, lorsqu’on fit remonter de Babylone à Jérusalem ceux de la captivité.
\Chap{2}
\TextTitle{Dénombrement des Israélites revenus de captivité}
\VerseOne{}Voici ceux de la province qui revinrent de la captivité, d'entre ceux que Nebucadnetsar, roi de Babylone, avait transportés en exil à Babylone, et qui retournèrent à Jérusalem, et en Juda ; chacun dans sa ville\FTNT{Esd. 5:8 ; Né. 1:3 ; Né. 7:6.}.
\VS{2}Ils vinrent avec Zorobabel, Josué, Néhémie, Seraja, Reélaja, Mardochée, Bilschan, Mispar, Bigvaï, Rehum, Baana. Nombre des hommes du peuple d'Israël :
\VS{3}Les fils de Pareosch, deux mille cent soixante-douze\FTNT{Né. 7:8.} ;
\VS{4}les fils de Schephathia, trois cent soixante-douze ;
\VS{5}les fils d'Arach, sept cent soixante-quinze ;
\VS{6}les fils de Pachath-Moab, des fils de Josué, de Joab, deux mille huit cent douze ;
\VS{7}les fils d'Elam, mille deux cent cinquante-quatre ;
\VS{8}les fils de Zatthu, neuf cent quarante-cinq ;
\VS{9}les fils de Zaccaï, sept cent soixante ;
\VS{10}les fils de Bani, six cent quarante-deux\FTNT{Né. 7:15.} ;
\VS{11}les fils de Bébaï, six cent vingt-trois ;
\VS{12}les fils d'Hazgad, mille deux cent vingt-deux ;
\VS{13}les fils d'Adonikam, six cent soixante-six ;
\VS{14}les fils de Bigvaï, deux mille cinquante-six ;
\VS{15}les fils d’Adin, quatre cent cinquante-quatre ;
\VS{16}les fils d'Ather, de la famille d'Ezéchias, quatre-vingt-dix-huit ;
\VS{17}les fils de Betsaï, trois cent vingt-trois ;
\VS{18}les fils de Jora, cent douze ;
\VS{19}les fils de Haschum, deux cent vingt-trois ;
\VS{20}les fils de Guibbar, quatre-vingt-quinze ;
\VS{21}les fils de Bethléhem, cent vingt-trois ;
\VS{22}les gens de Netopha, cinquante-six ;
\VS{23}les gens d'Anathoth, cent vingt-huit ;
\VS{24}les fils d'Azmaveth, quarante-deux ;
\VS{25}les fils de Kirjath-Arim, de Kephira, et de Beéroth, sept cent quarante-trois ;
\VS{26}les fils de Rama et de Guéba, six cent vingt et un ;
\VS{27}les gens de Micmas, cent vingt-deux ;
\VS{28}les gens de Béthel et d’Aï, deux cent vingt-trois ;
\VS{29}les fils de Nebo, cinquante-deux ;
\VS{30}les fils de Magbisch, cent cinquante-six.
\VS{31}les fils d'un autre Elam, mille deux cent cinquante-quatre ;
\VS{32}les fils de Harim, trois cent vingt ;
\VS{33}les fils de Lod, de Hadid, et d'Ono, sept cent vingt-cinq ;
\VS{34}les fils de Jéricho, trois cent quarante-cinq ;
\VS{35}les fils de Senaa, trois mille six cent trente.
\TextTitle{Dénombrement des sacrificateurs revenus de captivité}
\VS{36}Des sacrificateurs : Les fils de Jedaeja, de la maison de Josué, neuf cent soixante-treize ;
\VS{37}les fils d'Immer, mille cinquante-deux ;
\VS{38}les fils de Paschhur, mille deux cent quarante-sept ;
\VS{39}les fils de Harim, mille dix-sept.
\TextTitle{Dénombrement des Lévites revenus de captivité}
\VS{40}Des Lévites : Les fils de Josué et de Kadmiel, d'entre les fils d’Hodavia, soixante-quatorze.
\VS{41}Des chantres : Les fils d'Asaph, cent vingt-huit.
\VS{42}Des fils des portiers : Les fils de Schallum, les fils d'Ather, les fils de Thalmon, les fils d’Akkub, les fils de Hathitha, les fils de Schobaï, en tout cent trente-neuf.
\VS{43}Des Néthiniens : Les fils de Tsicha, les fils de Hasupha, les fils de Tabbahoth\FTNT{Esd.8:17 ; Jos. 9:23.},
\VS{44}les fils de Kéros, les fils de Siaha, les fils de Padon,
\VS{45}les fils de Lebana, les fils de Hagaba, les fils d'Akkub,
\VS{46}les fils de Hagab, les fils de Schamlaï, les fils de Hanan,
\VS{47}les fils de Guiddel, les fils de Gachar, les fils de Reaja,
\VS{48}les fils de Retsin, les fils de Nekoda, les fils de Gazzam,
\VS{49}les fils d'Uzza, les fils de Paséach, les fils de Bésaï,
\VS{50}les fils d'Asna, les fils de Mehunim, les fils de Nephusim,
\VS{51}les fils de Bakbuk, les fils de Hakupha, les fils de Harhur,
\VS{52}les fils de Batsluth, les fils de Mehida, les fils de Harscha,
\VS{53}les fils de Barkos, les fils de Sisera, les fils de Thamach,
\VS{54}les fils de Netsiach, les fils de Hathipha.
\TextTitle{Dénombrement des serviteurs de Salomon revenus de captivité}
\VS{55}Des fils des serviteurs de Salomon : Les fils de Sothaï, les fils de Sophéreth, les fils de Peruda,
\VS{56}les fils de Jaala, les fils de Darkon, les fils de Guiddel,
\VS{57}les fils de Schephathia, les fils de Hatthil, les fils de Pokéreth-Hatsebaïm, les fils d'Ami.
\VS{58}Total des Néthiniens et des fils des serviteurs de Salomon : Trois cent quatre-vingt-douze.
\VS{59}Voici ceux qui montèrent de Thel-Mélach, de Thel-Harscha, de Kerub-Addan et qui ne purent pas faire connaître leur maison paternelle et leur race, pour prouver qu’ils étaient d'Israël :
\VS{60}Les fils de Delaja, les fils de Tobija, les fils de Nekoda, six cent cinquante-deux.
\TextTitle{Certains sacrificateurs rejetés de la sacrificature}
\VS{61}Des fils des sacrificateurs : Les fils de Habaja, les fils d'Hakkots, les fils de Barzillaï, qui avait pris pour femme une des filles de Barzillaï, le Galaadite, fut appelé de leur nom.
\VS{62}Ils cherchèrent leurs registres généalogiques, mais ils ne les trouvèrent point. C'est pourquoi ils furent rejetés pour ne pas souiller le sacerdoce,
\VS{63}et le gouverneur leur dit de ne pas manger des choses très saintes, en attendant qu'un sacrificateur ait consulté l'urim et le thummim.
\TextTitle{Nombre total des Israélites revenus de captivité}
\VS{64}L’assemblée tout entière était de quarante-deux mille trois cent soixante,
\VS{65}sans leurs serviteurs et leurs servantes, qui étaient sept mille trois cent trente-sept. Ils avaient deux cents chantres ou chanteuses.
\VS{66}Ils avaient sept cent trente-six chevaux, et deux cent quarante-cinq mulets,
\VS{67}quatre cent trente-cinq chameaux, et six mille sept cent vingt ânes.
\VS{68}Quelques-uns d'entre les chefs des pères, quand ils vinrent à la maison de Yahweh à Jérusalem, firent des offrandes volontaires pour la maison de Dieu, afin qu'on la rétablît sur son emplacement.
\VS{69}Ils donnèrent au trésor de l'ouvrage, selon leurs moyens, soixante et un mille drachmes d'or, et cinq mille mines d'argent, et cent tuniques de sacrificateurs.
\VS{70}Ainsi, les sacrificateurs, les Lévites, quelques-uns du peuple, les chantres, les portiers et les Néthiniens habitèrent dans leurs villes. Et tous ceux d'Israël dans leurs villes aussi.
\Chap{3}
\TextTitle{Rétablissement de l'autel et des sacrifices}
\VerseOne{}Le septième mois approcha, et les fils d'Israël étaient dans leurs villes. Le peuple s'assembla alors comme un seul homme à Jérusalem.
\VS{2}Alors\FTNT{Ag. 1:1 ; De.12:5-6.} Josué, fils de Jotsadak, avec ses frères les sacrificateurs, et Zorobabel, fils de Schealthiel, avec ses frères, se levèrent et bâtirent l'autel du Dieu d'Israël, pour y offrir des holocaustes, comme il est écrit dans la loi de Moïse, homme de Dieu.
\VS{3}Ils rétablirent l'autel de Dieu sur ses fondements, parce qu'ils avaient peur en eux-mêmes des peuples du pays, et ils y offrirent des holocaustes à Yahweh, les holocaustes du matin et du soir\FTNT{No. 28:3.}.
\VS{4}Ils célébrèrent aussi la fête des tabernacles, comme il est écrit, et ils offrirent des holocaustes, autant qu'il en fallait  chaque jour\FTNT{Lé. 23:34 ; No. 29:12.}.
\VS{5}Après cela, ils offrirent l'holocauste perpétuel, ceux des nouvelles lunes, de toutes les fêtes solennelles consacrées à Yahweh, et ceux de quiconque faisait des offrandes volontaires à Yahweh\FTNT{No. 28:11 ; Né.10:33.}.
\VS{6}Dès le premier jour du septième mois, ils commencèrent à offrir des holocaustes à Yahweh. Cependant, les fondements du temple de Yahweh n'étaient pas encore posés.
\VS{7}Ils donnèrent de l'argent aux tailleurs de pierres et aux charpentiers, et aussi de la nourriture, des boissons, de l'huile aux Sidoniens et aux Tyriens, afin qu'ils amènent du bois de cèdre du Liban par la mer de Japho, selon la permission que Cyrus, roi de Perse, leur en avait donnée.
\TextTitle{Les fondements du temple posés}
\VS{8}Et la deuxième année depuis leur arrivée à la maison de Dieu à Jérusalem, au deuxième mois, Zorobabel, fils de Schealthiel,  Josué, fils de Jotsadak, et le reste de leurs frères les sacrificateurs et les Lévites, et tous ceux qui étaient revenus de la captivité à Jérusalem, débutèrent l’œuvre et désignèrent des Lévites, depuis l'âge de vingt ans et au-dessus pour surveiller l'ouvrage de la maison de Yahweh.
\VS{9}Et Josué, avec ses fils et ses frères, Kadmiel, avec ses fils, fils de Juda, les fils de Hénadad, avec leurs fils et leurs frères les Lévites, se tenaient debout pour surveiller ceux qui faisaient l'ouvrage de la maison de Dieu.
\VS{10}Et lorsque ceux qui bâtissaient posèrent les fondements du temple de Yahweh, on fit assister les sacrificateurs revêtus de leurs habits, avec leurs trompettes, et les Lévites, fils d'Asaph, avec les cymbales, pour qu’ils célèbrent Yahweh, selon l’institution de David, roi d'Israël.
\VS{11}Et en louant et célébrant Yahweh, ils s'entre-répondaient : Il est bon, parce que sa miséricorde demeure à toujours sur Israël ! Et tout le peuple poussait de grands cris de joie en louant Yahweh, parce qu'on posait les fondements de la maison de Yahweh.
\VS{12}Mais plusieurs des sacrificateurs et des Lévites, et des chefs de familles âgés, qui avaient vu la première maison, pleuraient à grand bruit pendant qu'on posait sous leurs yeux les fondements de cette maison. Et beaucoup élevaient leur voix avec des cris de joie,
\VS{13}et le peuple ne pouvait distinguer le bruit des cris de joie  d'avec le bruit des pleurs du peuple, car le peuple poussait de grands cris de joie dont le son s’entendait de très loin.
\Chap{4}
\TextTitle{Les ennemis de Juda et de Benjamin découragent le peuple de Juda}
\VerseOne{}Les ennemis de Juda et de Benjamin entendirent que les fils de la captivité rebâtissaient un temple à Yahweh, le Dieu d'Israël.
\VS{2}Ils vinrent vers Zorobabel et vers les chefs des familles, et leur dirent : Nous bâtirons\FTNT{On ne doit jamais s’associer avec les impies pour bâtir l’œuvre du Seigneur. Satan essaie toujours de s’infiltrer dans les assemblées afin de nous éloigner de la vérité, c’est pour cela que nous devons faire preuve de discernement (2 Co. 6:14-16).} avec vous ; car nous invoquons votre Dieu comme vous ; et nous lui avons sacrifié depuis le temps d'Esar-Haddon, roi d'Assyrie, qui nous a fait monter ici.
\VS{3}Mais Zorobabel, Josué, et les autres chefs des familles d'Israël, leur répondirent : Il ne convient pas à vous de bâtir la maison de notre Dieu ; mais nous, qui sommes ici ensemble, nous la bâtirons à Yahweh, le Dieu d'Israël, comme nous l'a ordonné le roi Cyrus, roi de Perse\FTNT{Esd. 1:1,2,5.}.
\VS{4}Alors les gens du pays rendirent paresseuses les mains du peuple de Juda ; ils l’intimidèrent pour l'empêcher de bâtir,
\VS{5}ils avaient même engagé à prix d’argent des conseillers pour faire échouer leur projet, pendant toute la durée de vie de Cyrus, roi de Perse, jusqu'au règne de Darius, roi de Perse.
\VS{6}Et sous le règne d'Assuérus, au commencement de son règne, ils écrivirent une accusation contre les habitants de Juda et de Jérusalem.
\TextTitle{Lettre envoyé à Artaxerxès}
\VS{7}Et du temps d'Artaxerxès, Bischlam, Mithredath, Thabeel, et le reste de leurs collègues, écrivirent à Artaxerxès, roi de Perse. La lettre était écrite en caractères araméens, et elle était traduite en araméen.
\VS{8}Rehum, le gouverneur, et Schimschaï, le secrétaire, écrivirent au roi Artaxerxès la lettre suivante concernant Jérusalem :
\VS{9}Rehum, gouverneur, Schimschaï, secrétaire, et le reste de leurs collègues, ceux de Din, d'Arpharsathac, de Tharpel, d'Apharas, d'Erec, de Babylone, de Suse, de Déha, d'Elam,
\VS{10}et les autres peuples que le grand et illustre Osnappar a transportés et fait habiter dans la ville de Samarie et les autres régions au-delà du fleuve, à cette date.
\VS{11}Voici donc ici la copie de la lettre qu'ils envoyèrent au roi Artaxerxès : Tes serviteurs, les gens de ce côté du fleuve, à cette date.
\VS{12}Que le roi sache que les Juifs qui sont montés de chez lui et arrivés vers nous à Jérusalem rebâtissent la ville rebelle et méchante, et achèvent en finissant de poser et de réparer les fondements des murs.
\VS{13}Que le roi sache donc que si cette ville est rebâtie et si ses murs sont réparés, ils ne paieront plus de tribut, ni d’impôt, ni de droit de passage, et elle causera une grande nuisance aux revenus du roi.
\VS{14}Et parce que nous mangeons le sel du palais, il ne nous parait pas convenable de voir le roi déshonoré ; c'est pourquoi nous envoyons au roi ces informations.
\VS{15}Qu'on recherche dans le livre des mémoires de tes pères, et tu trouveras et tu apprendras dans ce livre des mémoires que cette ville est une ville rebelle, nuisible aux rois et aux provinces ; et qu’on s’y est livré à la révolte depuis toujours. Donc cette ville a été détruite à cause de cela.
\VS{16}Nous faisons donc savoir au roi que si cette ville est rebâtie et si ses murs sont relevés, il n'aura plus de possession de ce côté du fleuve.
\TextTitle{Réponse du roi Artaxerxès}
\VS{17}Et c'est ici le décret envoyé par le roi à Rehum, le gouverneur, à Schimschaï, le secrétaire, et au reste de leurs collègues demeurant à Samarie, et aux autres de l'autre côté du fleuve : Paix sur vous, à cette date.
\VS{18}La lettre que vous nous avez envoyée a été lue exactement en ma présence.
\VS{19}J'ai donné ordre de faire des recherches et l’on a trouvé  que depuis toujours cette ville s'est soulevée contre les rois, et qu'on s’y est livré à la sédition et à la révolte.
\VS{20}Il y eut aussi à Jérusalem des rois puissants, maîtres de tout le pays de l'autre côté du fleuve, et auxquels on payait  tribut, impôt et droit de passage\FTNT{2 S. 8:2,6 ; 1 R 4:21 ; 2 Ch. 17:11 ; 32:23.}.
\VS{21}A présent, donnez l’ordre de ne pas laisser continuer ces gens-là, afin que cette ville ne se rebâtisse point, jusqu’à ce que je l'ordonne par décret.
\VS{22}Gardez-vous de mettre en cela de la négligence, de peur que le mal  n’augmente au préjudice des rois.
\VS{23}Aussitôt que la copie de la lettre du roi Artaxerxès eut été lue en présence de Rehum, de Schimschaï,  le secrétaire, et de leurs collègues, ils allèrent en hâte à Jérusalem vers les Juifs, et ils les firent cesser leurs travaux avec violence et force.
\VS{24}Alors l’ouvrage de la maison de Dieu, à Jérusalem, cessa, et elle demeura dans cet état, jusqu'à la deuxième année du règne de Darius, roi de Perse\FTNT{Esd. 5:2}.
\Chap{5}
\TextTitle{Aggée et Zacharie prophétisent}
\VerseOne{}Aggée, le prophète, et Zacharie, fils d'Iddo, le prophète, prophétisèrent aux Juifs qui étaient en Juda et à Jérusalem, au nom du Dieu d'Israël, qui s’adressait à eux\FTNT{Ag. 1:4 ; Za. 1:1.}.
\VS{2}Alors Zorobabel, fils de Schealthiel, et Josué, fils de Jotsadak, se levèrent et commencèrent à rebâtir la maison de Dieu à Jérusalem. Et ils avaient avec eux les prophètes de Dieu qui les soutenaient\FTNT{Ag. 1:14 ; Esd. 6:14.}.
\VS{3}En ce temps-là, Thathnaï, gouverneur de ce côté du fleuve, et Schethar-Boznaï, et leurs collègues, vinrent à eux et leur parlèrent ainsi : Qui vous a donné l’ordre de rebâtir cette maison et de relever ces murs ?\FTNT{Esd. 5:9.}
\VS{4}Ils leur dirent alors : Quels sont les noms des hommes qui construisent cet édifice ?
\VS{5}Mais l’œil de Dieu était sur les anciens des Juifs. Et on ne les laissa continuer les travaux, pendant l’envoi d’un rapport à Darius, et jusqu'à la réception d’une lettre sur cet objet.
\TextTitle{Thathnaï, Schethar-Boznaï et leurs collègues d'Apharsac écrivent à Darius}
\VS{6}Copie de la lettre envoyée au roi Darius par Thathnaï, gouverneur de ce côté du fleuve, Schethar-Boznaï, et leurs collègues d'Apharsac, de l’autre côté du fleuve.
\VS{7}Ils lui envoyèrent un rapport ainsi écrit : Paix parfaite soit au roi Darius !
\VS{8}Que le roi sache que nous sommes allés dans la province de Juda, vers la maison du grand Dieu. Elle se bâtit avec des pierres de taille, et le bois se pose dans les murs ; ce travail se réalise complètement et prospère entre leurs mains\FTNT{Esd. 2:1.}.
\VS{9}Nous avons interrogé ces anciens, et nous leur avons parlé ainsi : Qui vous a donné l'autorisation de rebâtir cette maison et de finir ces murs ?\FTNT{Esd. 5:3.}
\VS{10}Nous leur avons aussi demandé leurs noms pour te les faire connaître, et nous avons mis par écrit les noms des hommes à leur tête.
\VS{11}Et ils nous ont répondu de cette manière, disant : Nous sommes les serviteurs du Dieu des cieux et de la terre, et nous rebâtissons la maison qui avait été bâtie il y a de nombreuses années ; un grand roi d'Israël l’avait bâtie et finie.
\VS{12}Mais après que nos pères eurent provoqué la colère du Dieu des cieux, il les livra entre les mains de Nebucadnetsar\FTNT{Voir 2 R. 24 et 25.}, roi de Babylone, Chaldéen, qui détruisit cette maison et qui emmena le peuple en exil à Babylone\FTNT{2 Ch. 36:7}.
\VS{13}Mais la première année de Cyrus, roi de Babylone, le roi Cyrus prit un décret pour rebâtir cette maison de Dieu\FTNT{Esd. 1:1-2.}.
\VS{14}Et même le roi Cyrus ôta du temple de Babylone les ustensiles d'or et d'argent de la maison de Dieu, que Nebucadnetsar avait sortis du temple qui était à Jérusalem et transportés dans le temple de Babylone, et il les fit remettre au nommé Scheschbatsar, qu’il établit gouverneur\FTNT{Esd. 1:8.},
\VS{15}et il lui dit : Prends ces ustensiles, et va les déposer dans le temple de Jérusalem ; et que la maison de Dieu soit rebâtie sur sa place.
\VS{16}Alors ce Scheschbatsar est venu, et il a posé les fondements de la maison de Dieu à Jérusalem ; et depuis ce temps-là jusqu'à présent, on la bâtit, et elle n'est point encore achevée.
\VS{17}Maintenant, s'il semble bon au roi, que l’on fasse des recherches dans la maison des trésors du roi à Babylone, pour voir s'il est vrai qu'il y a eu un ordre donné par Cyrus de rebâtir cette  maison de Dieu à Jérusalem. Puis, que le roi nous transmette sa volonté sur cet objet.
\Chap{6}
\TextTitle{Darius confirme l'édit de Cyrus}
\VerseOne{}Alors le roi Darius donna un ordre de faire des recherches dans la maison des livres où l'on déposait les  trésors à Babylone.
\VS{2}Et l’on trouva à Achmetha, dans un coffre, capitale de la province de Médie, un rouleau à l’intérieur duquel était écrit  le mémoire suivant :
\VS{3}La première année du roi Cyrus, le roi Cyrus prit un décret quant à la maison de Dieu à Jérusalem : Que cette maison soit rebâtie, afin d’être un lieu où l'on offre des sacrifices, et que ses fondements soient solides pour porter sa charge. La hauteur sera de soixante coudées, et la longueur de soixante coudées,
\VS{4}trois rangées de pierres de taille et une rangée de bois neuf.  La dépense sera payée par la maison du roi.
\VS{5}Aussi, les ustensiles d'or et d'argent de la maison de Dieu, que Nebucadnetsar avait enlevés du temple de Jérusalem et apportés à Babylone, seront remis et apportés dans le temple de Jérusalem, à leur place, et déposés dans la maison de Dieu.
\VS{6}Maintenant, Thathnaï, gouverneur de l'autre côté du fleuve, Schethar-Boznaï, et vos collègues d'Apharsac de l'autre côté du fleuve, tenez-vous loin de ce lieu.
\VS{7}Laissez le travail de cette maison de Dieu ; que le gouverneur des Juifs et les anciens des Juifs rebâtissent cette maison de Dieu à sa place.
\VS{8}En raison de ce décret pris, ce que vous aurez à exécuter, avec les anciens de ces Juifs pour rebâtir cette maison de Dieu : Sur les finances du roi provenant du tribut de l’autre côté du fleuve, les frais seront complètement payés à ces hommes, afin qu'il n'y ait pas d'interruption.
\VS{9}Et ce qui sera nécessaire pour les holocaustes du Dieu des cieux, veaux, béliers et agneaux, blé, sel, vin et huile, seront livrés, sur leur demande, aux sacrificateurs de Jérusalem, jour après jour, sans négligence,
\VS{10}afin qu'ils offrent des sacrifices de bonne odeur au Dieu des cieux et qu'ils prient pour la vie du roi et de ses fils.
\VS{11}Et voici l’ordre que je donne touchant quiconque changera cette parole : On arrachera de sa maison une pièce de bois, on la dressera, afin qu'il y soit exterminé, et l’on fera de sa maison un tas de déchets\FTNT{2 R. 10:27 ; Ez. 6:11 ; Da. 3:29.}.
\VS{12}Et que Dieu, qui fait résider en ce lieu son nom, renverse tout roi et tout peuple qui étendrait sa main pour changer et détruire cette maison de Dieu à Jérusalem ! Moi, Darius, j’ai donné cet ordre. Qu'il soit donc exécuté complètement.
\TextTitle{Achèvement et dédicace de la maison de Dieu}
\VS{13}Alors Thathnaï, gouverneur de l'autre côté du fleuve,  Schethar-Boznaï, et leurs collègues, firent exécuter ainsi complètement ce que le roi Darius leur envoya.
\VS{14}Et les anciens des Juifs bâtirent avec succès, selon les prophéties d'Aggée, le prophète, et de Zacharie, fils d’Iddo ; ils bâtirent et finirent, d'après l'ordre du Dieu d'Israël, et d'après l'ordre de Cyrus, de Darius, et d'Artaxerxès, roi de Perse.
\VS{15}Cette maison fut achevée le troisième jour du mois d'Adar, dans la sixième année du règne du roi Darius.
\VS{16}Les fils d'Israël, les sacrificateurs, les Lévites, et le reste des fils de la captivité, célébrèrent la dédicace de cette maison de Dieu avec joie.
\VS{17}Ils offrirent pour la dédicace de cette maison de Dieu, cent taureaux, deux cents béliers, quatre cents agneaux, et douze boucs comme victimes expiatoires pour tout Israël, selon le nombre des tribus d'Israël.
\VS{18}Ils établirent les sacrificateurs selon leurs classes et les Lévites selon leurs divisions, pour le service de Dieu à Jérusalem,  selon ce qui est écrit dans le livre de Moïse\FTNT{No. 3:6,32 ; No. 8:11}.
\TextTitle{Rétablissement de la Pâque}
\VS{19}Puis les fils de la captivité célébrèrent la Pâque le quatorzième jour du premier mois\FTNT{Lé. 23:5 ; No. 28:16 ; De. 16:2.}.
\VS{20}Les sacrificateurs et les Lévites s'étaient purifiés comme un seul homme, tous étaient purs ; c'est pourquoi ils immolèrent la Pâque pour tous les fils de la captivité, pour leurs frères les sacrificateurs, et pour eux-mêmes\FTNT{2 Ch. 30:15,17,21}.
\VS{21}Les fils d'Israël revenus de la captivité mangèrent la Pâque, avec tous ceux qui s'étaient séparés de l’impureté des nations du pays pour chercher Yahweh, le Dieu d'Israël.
\VS{22}Ils célébrèrent avec joie la fête des pains sans levain pendant sept jours, car Yahweh les avait réjouis en disposant  le cœur du roi d'Assyrie à fortifier leurs mains dans l’œuvre de la maison de Dieu, du Dieu d'Israël.
\Chap{7}
\TextTitle{Voyage d'Esdras jusqu'à Jérusalem}
\VerseOne{}Après ces choses, sous le règne d'Artaxerxès, roi de Perse, Esdras, fils de Seraja, fils d'Azaria, fils de Hilkija\FTNT{Esd. 6:14.},
\VS{2}fils de Schallum, fils de Tsadok, fils d'Achithub,
\VS{3}fils d'Amaria, fils d'Azaria, fils de Merajoth,
\VS{4}fils de Zerachja, fils d'Uzzi, fils de Bukki,
\VS{5}fils d'Abischua, fils de Phinées, fils d'Eléazar, fils d'Aaron, souverain sacrificateur.
\VS{6}Esdras monta de Babylone : C’était un scribe bien exercé dans la loi de Moïse, donnée par Yahweh, le Dieu d'Israël. Et comme la main de Yahweh, son Dieu, était sur lui, le roi lui accorda toute sa requête\FTNT{Vers. 9,28.}.
\VS{7}Des fils d'Israël, des sacrificateurs, des Lévites, des chantres, des portiers, et des Néthiniens, montèrent à Jérusalem, la septième année du roi Artaxerxès.
\VS{8}Il entra à Jérusalem le cinquième mois de la septième année du roi ;
\VS{9}il était parti de Babylone au premier jour du premier mois, et il entra à Jérusalem au premier jour du cinquième mois, selon que la main de son Dieu était bonne sur lui.
\VS{10}Car Esdras avait disposé son cœur à étudier la loi de Yahweh, à l’observer et à enseigner les lois et les ordonnances parmi le peuple d'Israël.
\TextTitle{Lettre d'Artaxerxès à Esdras}
\VS{11}Voici la copie de la lettre que le roi Artaxerxès donna à Esdras, sacrificateur et scribe, enseignant les paroles des commandements de Yahweh et ses ordonnances concernant Israël :
\VS{12}Artaxerxès, roi des rois, à Esdras, sacrificateur et scribe de la loi du Dieu des cieux, à cette date.
\VS{13}J’ai donné ordre de laisser aller tous ceux de mon royaume qui sont du peuple d'Israël, de ses sacrificateurs et Lévites, qui se présenteront volontairement pour aller avec toi à Jérusalem.
\VS{14}Tu es envoyé de la part du roi, et de ses sept conseillers, pour inspecter Juda et Jérusalem touchant la loi de ton Dieu, laquelle est entre tes mains,
\VS{15}et pour porter l'argent et l'or que le roi et ses conseillers ont offert volontairement au Dieu d'Israël, dont la demeure est à Jérusalem\FTNT{Esd. 8:24.},
\VS{16}tout l'argent et l'or que tu trouveras dans toute la province de Babylone, avec les offrandes volontaires du peuple et des sacrificateurs, qu'ils feront volontairement à la maison de leur Dieu à Jérusalem.
\VS{17} C'est pourquoi tu achèteras avec cet argent des taureaux, des béliers, des agneaux, avec leurs offrandes et leurs libations, et tu les offriras sur l'autel de la maison de votre Dieu à Jérusalem.
\VS{18}Vous ferez, selon la volonté de votre Dieu, ce qu'il te semblera bon à toi et à tes frères de faire du reste de l'argent et de l'or.
\VS{19}Et pour ce qui est des ustensiles qui te sont remis pour le service de la maison de ton Dieu, déposes-les en présence du Dieu de Jérusalem.
\VS{20}Quand au reste de ce qui sera nécessaire pour la maison de ton Dieu, autant qu'il t'en faudra employer, tu le prendras de la maison des trésors du roi.
\VS{21}Moi, le roi Artaxerxès, je donne l’ordre à tous les trésoriers qui sont de l'autre côté du fleuve de livrer exactement à Esdras, sacrificateur et scribe de la loi du Dieu des cieux, tout ce qu’il vous demandera,
\VS{22}jusqu'à cent talents d'argent, cent cors de froment, cent baths de vin, cent baths d'huile, et du sel sans nombre.
\VS{23}Que tout ce qui est ordonné par le Dieu des cieux se fasse exactement pour la maison du Dieu des cieux, afin que sa colère ne soit pas sur le royaume, sur le roi et sur ses fils.
\VS{24}Nous vous faisons savoir qu'on ne pourra imposer ni tribut, ni impôt, ni droit de passage sur aucun des sacrificateurs, des Lévites, des chantres, des portiers, des  Néthiniens, et des serviteurs de cette maison de Dieu.
\VS{25}Et toi, Esdras, établis des magistrats et des juges selon la sagesse de ton Dieu que tu possèdes, afin qu'ils rendent justice à tout ce peuple de l'autre côté du fleuve, à tous ceux qui connaissent les lois de ton Dieu ; afin que vous enseigniez celui qui ne les connaît point.
\VS{26}Et tous ceux qui n'observeront point la loi de ton Dieu et la loi du roi seront aussitôt jugés, soit à la mort, soit au bannissement, soit à une amende pécuniaire, ou à l'emprisonnement.
\VS{27}Béni soit Yahweh, le Dieu de nos pères, qui a mis cela au cœur du roi, pour honorer la maison de Yahweh, qui est à Jérusalem ;
\VS{28}et qui a fait que j'ai trouvé grâce devant le  roi, devant ses conseillers, et devant tous les puissants chefs ! Fortifié par la main de Yahweh, mon Dieu, qui était sur moi, j'ai rassemblé les chefs d'Israël, afin qu'ils montent avec moi.
\Chap{8}
\TextTitle{Dénombrement de ceux qui montèrent avec Esdras}
\VerseOne{}Voici les chefs des pères, avec le dénombrement fait selon les généalogies de ceux qui montèrent avec moi de Babylone, pendant le règne du roi Artaxerxès\FTNT{1 Ch. 4:33.}.
\VS{2}Des fils de Phinées, Guerschom ; des fils d'Ithamar, Daniel ; des fils de David, Hattusch ;
\VS{3}des fils de Schecania ; des fils de Pareosch, Zacharie, et avec lui, en faisant le dénombrement par leur généalogie selon les hommes, cent cinquante hommes ;
\VS{4}des fils de Pachat Moab, Eljoénaï, fils de Zerachja, et avec lui deux cents hommes;
\VS{5}des fils de Schecania, le fils de Jachaziel, et avec lui trois cents hommes;
\VS{6}des fils d'Adin, Ebed, fils de Jonathan, et avec lui cinquante hommes ;
\VS{7}des fils d'Elam, Esaïe, fils d'Athalia, et avec lui soixante-dix hommes;
\VS{8}des fils de Schephathia, Zebadia, fils de Micaël, et avec lui quatre-vingts hommes ;
\VS{9}des fils de Joab, Abdias, fils de Jehiel, et avec lui deux cent dix-huit hommes ;
\VS{10}des fils de Schelomith, le fils de Josiphia, et avec lui cent soixante hommes ;
\VS{11}des fils de Bébaï, Zacharie, fils de Bébaï, et avec lui vingt-huit hommes ;
\VS{12}des fils d'Azgad, Jochanan, fils d'Hakkathan, et avec lui cent-dix hommes ;
\VS{13}des fils d'Adonikam, les derniers, dont voici les noms: Eliphélet, Jeïel, et Schemaeja, et avec eux soixante hommes ;
\VS{14}des fils de Bigvaï, Uthaï, Zabbud, et avec eux soixante-dix hommes.
\VS{15}Je les rassemblai près du fleuve qui coule vers Ahava, et nous campâmes là trois jours. Puis je portai mon attention sur  le peuple et les sacrificateurs, et je n'y trouvai aucun des fils de Lévi.
\VS{16}Alors j'envoyai d'entre les chefs Eliézer, Ariel, Schemaeja, Elnathan, Jarib, Elnathan, Nathan, Zacharie et Meschullam, avec les docteurs Jojarib et Elnathan.
\VS{17}Je leur donnai des ordres pour le chef Iddo, demeurant à Casiphia, et je mis dans leur bouche les paroles qu'ils devaient dire à Iddo et à ses frères les Néthiniens, qui étaient à Casiphia, afin qu'ils nous amènent des serviteurs pour la maison de notre Dieu\FTNT{Esd. 2:43.}.
\VS{18}Et comme la bonne main de notre Dieu était sur nous, ils nous amenèrent Schérébia, un homme intelligent, d'entre les fils de Machli, fils de Lévi, fils d'Israël, et avec ses fils et ses frères, au nombre dix-huit\FTNT{Esd. 7:6,9,28.} ;
\VS{19}Haschabia, et avec lui Esaïe, d'entre les fils de Merari, ses frères, et leurs fils, au nombre vingt ;
\VS{20}et des Néthiniens, que David et les chefs du peuple avaient assignés pour le service des Lévites, deux cent vingt Néthiniens, tous désignés par leurs noms\FTNT{Esd. 2:43,58.}.
\TextTitle{Esdras publie un jeûne pour obtenir la protection de Dieu}
\VS{21}Et je publiai là un jeûne près de la rivière d'Ahava, afin de nous humilier devant notre Dieu, le priant de nous donner un heureux voyage, pour nos enfants, et pour tous nos biens.
\VS{22}Car j'aurais eu honte de demander au roi une armée et des cavaliers pour nous soutenir contre des ennemis pendant le chemin ; car nous avions dit au roi : La main de notre Dieu est favorable sur tous ceux qui le cherchent ; mais sa force et sa colère sont contre ceux qui l'abandonnent.
\VS{23}Nous jeûnâmes donc, et nous cherchâmes notre Dieu à cause de cela. Et il se laissa fléchir par nos prières.
\TextTitle{Trésors remis par Esdras entre les mains de douze sacrificateurs}
\VS{24}Alors je mis à part douze chefs des sacrificateurs, Schérébia, Haschabia, et dix de leurs frères.
\VS{25}Je pesai l'argent, l'or et les ustensiles donnés en offrandes pour la maison de notre Dieu par le roi, ses conseillers, ses chefs, et tous ceux d'Israël qu'on avait trouvés\FTNT{Esd. 7:14,15.}.
\VS{26}Je pesai donc, et je remis entre leurs mains six cent cinquante talents d'argent, des ustensiles d'argent pesant cent talents, cent talents d'or,
\VS{27}vingt coupes d'or valant mille drachmes, et deux ustensiles d’un bel airain poli, aussi précieux que de l'or.
\VS{28}Et je leur dis : Vous êtes consacrés à Yahweh ; et les ustensiles sont sanctifiés, et cet argent et cet or sont une offrande volontaire faite à Yahweh, le Dieu de vos pères.
\VS{29}Soyez vigilants et gardez-les, jusqu'à ce que vous les pesiez devant les chefs des sacrificateurs et les Lévites, et devant les chefs des pères d'Israël, à Jérusalem, dans les chambres de la maison de Yahweh.
\VS{30}Les sacrificateurs et les Lévites reçurent le poids de l'argent, de l'or, et des ustensiles, pour les apporter à Jérusalem, dans la maison de notre Dieu.
\TextTitle{Esdras arrive à Jérusalem}
\VS{31}Nous partîmes du fleuve d'Ahava pour aller à Jérusalem, le douzième jour du premier mois. La main de notre Dieu fut sur nous et nous délivra de la main des ennemis et des  embûches sur le chemin.
\VS{32}Puis nous arrivâmes à Jérusalem, et nous nous y reposâmes trois jours.
\VS{33}Le quatrième jour, nous pesâmes l'argent, l'or, et les ustensiles dans la maison de notre Dieu, et nous les remîmes à Merémoth, fils d'Urie, le sacrificateur - il était avec Eléazar, fils de Phinées, et avec eux les Lévites Jozabad, fils de Josué, et Noadia, fils de Binnuï-
\VS{34} selon tout le nombre et le poids de toutes ces choses, et tout le poids fut mis alors par écrit.
\VS{35}Et les fils de la captivité revenus de l’exil offrirent en holocauste au Dieu d'Israël douze taureaux, quatre-vingt-seize béliers, soixante-dix-sept agneaux, et douze boucs comme victimes expiatoires pour tout Israël, le tout en holocauste à Yahweh.
\VS{36}Ils transmirent les ordres du roi entre les mains des satrapes du roi et des gouverneurs qui étaient de ce côté du fleuve, lesquels favorisèrent le peuple et la maison de Dieu.
\Chap{9}
\TextTitle{La désobéissance}
\VerseOne{}Après que ces choses furent terminées, les chefs du peuple s'approchèrent de moi, en disant : Le peuple d'Israël,  les sacrificateurs et les Lévites ne se sont point séparés des peuples de ces pays, quant à leurs abominations, celles des Cananéens, des Héthiens, des Phéréziens, des Jébusiens, des Ammonites, des Moabites, des Egyptiens, et des Amoréens.
\VS{2}Car ils ont pris de leurs filles pour eux et pour leurs fils, et ont mêlé la semence sainte avec les peuples de ces pays ; et des chefs et des magistrats ont été les premiers à commettre ce péché\FTNT{Né. 13:3.}.
\VS{3}Lorsque j'entendis cela, je déchirai mes vêtements et mon manteau, j'arrachai les cheveux de ma tête et ma barbe, et je m'assis tout épouvanté.
\VS{4}Et tous ceux qui tremblaient aux paroles du Dieu d'Israël, s'assemblèrent auprès de moi, à cause de l’infidélité de ceux de la captivité ; et je demeurai assis tout épouvanté jusqu'à l'offrande du soir.
\TextTitle{Prière et confession d'Esdras}
\VS{5}Et au temps de l'offrande du soir, je me levai du sein de mon affliction, et ayant mes vêtements et mon manteau déchirés, je me mis à genoux, et j'étendis mes mains vers Yahweh, mon Dieu,
\VS{6}et je dis : Mon Dieu ! J’ai honte, et je suis trop confus, ô mon Dieu, pour lever ma face vers toi ; car nos iniquités se sont multipliées au-dessus de nos têtes, et notre péché s'est élevé jusqu’aux cieux.
\VS{7}Depuis les jours de nos pères jusqu'à ce jour, nous sommes grandement coupables, et c’est à cause de nos iniquités que nous avons été livrés, nous, nos rois et nos sacrificateurs entre les mains des rois des pays, à l'épée, à la captivité, au pillage, et à la honte, comme il paraît aujourd'hui.
\VS{8}Et cependant Yahweh, notre Dieu, nous a maintenant fait grâce, en épargnant un reste, et il nous a donné un clou dans son saint lieu, afin d'éclaircir nos yeux et nous donner un peu de répit dans notre servitude\FTNT{Es. 22:23.}.
\VS{9}Car nous sommes esclaves, mais notre Dieu ne nous a point abandonnés dans notre servitude. Il a incliné la bienveillance des rois de Perse pour nous accorder de préserver nos vies afin que nous puissions relever la maison de notre Dieu, et rétablir ces lieux en ruines, et pour nous donner une clôture en Juda et à Jérusalem.
\VS{10}Mais maintenant, ô notre Dieu ! Que dirons-nous après ces choses ? Car nous avons abandonné tes commandements,
\VS{11}que tu as ordonnés par tes serviteurs les prophètes, en disant : Le pays dans lequel vous entrez pour le posséder est un pays souillé par les impuretés des peuples de ces pays, à cause des abominations dont ils l'ont rempli d’un bout à l'autre par leurs impuretés\FTNT{Lé. 18:25-27.};
\VS{12}maintenant donc, ne donnez point vos filles à leurs fils, et ne prenez point leurs filles pour vos fils, ne cherchez jamais ni leur bonheur, ni leur paix, ainsi vous deviendrez forts,  vous mangerez les meilleurs productions du pays, et vous le laisserez hériter à vos fils pour toujours\FTNT{De. 7:3.}.
\VS{13}Après toutes les choses qui nous sont arrivées à cause de nos mauvaises actions et des grandes offenses que nous avons commises - quoi que tu ne nous aies pas, ô notre Dieu, punis en proportion de nos péchés et maintenant que  tu nous as conservé ces réchappés ; 
\VS{14}retournerions-nous à violer tes commandements, et à faire alliance avec ces peuples abominables ? Ne serais-tu pas en colère contre nous, jusqu'à nous exterminer, sans aucun reste ni aucun réchappé ?
\VS{15}Yahweh, Dieu d'Israël ! Tu es juste, car nous sommes aujourd'hui un reste de réchappés. Voici, nous sommes devant toi avec nos fautes, ne pouvant subsister à cause d’elles devant ta face.
\Chap{10}
\TextTitle{Confession et séparation}
\VerseOne{}Pendant qu’Esdras priait et faisait cette confession, pleurant et étant prosterné à terre devant la maison de Dieu, une grande multitude d'hommes, de femmes, et d’enfants d'Israël, s'assembla auprès de lui ; et le peuple se lamenta abondamment par des pleurs.
\VS{2}Alors Schecania, fils de Jehiel, d'entre les fils d’Elam, prit la parole, et dit à Esdras : Nous avons péché contre notre Dieu, en nous mariant avec des femmes étrangères d'entre les peuples de ce pays. Mais Israël ne reste pas pour cela sans espérance\FTNT{De. 7:22,23.}.
\VS{3}Faisons maintenant une alliance avec notre Dieu pour le renvoi de toutes ces femmes et de leurs enfants, selon le conseil de mon seigneur et de ceux qui tremblent devant les commandements de notre Dieu. Et qu'il en soit fait selon la loi\FTNT{Esd. 9:4 ; Mal. 3:16.}.
\VS{4}Lève-toi, car cette affaire te regarde. Nous serons avec toi. Prends donc courage et agis.
\VS{5}Esdras se leva, et il fit jurer aux chefs des sacrificateurs, des Lévites, et de tout Israël, de faire selon cette parole. Et ils le jurèrent.
\VS{6}Puis Esdras se retira de devant la maison de Dieu, et s'en alla dans la chambre de Jochanan, fils d'Eliaschib ; et quand il y fut entré, il ne mangea point de pain, ne but point d'eau, parce qu'il se lamentait à cause du péché de ceux de la captivité.
\VS{7}Alors on publia dans le pays de Juda et à Jérusalem que tous ceux qui étaient retournés de la captivité aient à s'assembler à Jérusalem,
\VS{8}et que quiconque ne s'y rendrait pas dans trois jours, selon l'avis des chefs et des anciens, aurait tous ses biens complètement détruits, et que lui-même serait séparé de l'assemblée de ceux de la captivité.
\VS{9}Ainsi tous ceux de Juda et de Benjamin s'assemblèrent à Jérusalem dans les trois jours. C’était le vingtième jour du neuvième mois. Tout le peuple se tenait sur la place de la maison de Dieu, tremblant au sujet de cette affaire et à cause des pluies\FTNT{1 S. 12:18.}.
\VS{10}Esdras, le sacrificateur, se leva et leur dit : Vous avez péché en vous mariant avec des femmes étrangères, de sorte que vous avez augmenté la culpabilité d'Israël\FTNT{De. 7:3.}.
\VS{11}Prononcez maintenant votre confession à Yahweh, le Dieu de vos pères, et faites sa volonté ! Séparez-vous des peuples du pays et des femmes étrangères.
\VS{12}Et toute l'assemblée répondit à haute voix : A nous de faire ce que tu as dit !
\VS{13}Mais le peuple est nombreux, le temps est pluvieux, et il n'y a pas moyen de se tenir dehors ; d’ailleurs, ce n’est pas l’affaire d’un jour ou de deux, car il y en a beaucoup parmi nous qui ont péché dans cette affaire.
\VS{14}Que tous nos chefs se présentent donc devant toute l'assemblée, et que tous ceux qui sont dans nos villes, et qui se sont mariés avec des femmes étrangères, viennent à un temps fixé, et que les anciens de chaque ville et ses juges soient avec eux, jusqu'à ce que nous détournions de nous l'ardente colère de notre Dieu à ce sujet.
\VS{15}Il n'y eut que Jonathan, fils d'Asaël, et Jachzia, fils de Thikva, qui s'opposèrent a cet avis ; et Meschullam et Schabthaï, Lévites, les appuyèrent ;
\VS{16}mais ceux qui étaient retournés de la captivité s’y conformèrent. On choisit Esdras, le sacrificateur, et des chefs de famille selon leurs maisons paternelles, tous désignés par leurs noms ; ils siégèrent le premier jour du dixième mois, pour suivre cette affaire.
\VS{17}Le premier jour du premier mois, ils en finirent avec tous les hommes qui s’étaient mariés à des femmes étrangères.
\VS{18}Parmi les fils des sacrificateurs qui s’étaient mariés à des femmes étrangères, il se trouva d'entre les fils de Josué, fils de Jotsadak et de ses frères, Maaséja, Eliézer, Jarib et Guedalia,
\VS{19}qui, en donnant leurs mains, renvoyèrent leurs femmes ; et offrirent un bélier comme sacrifice de culpabilité ;
\VS{20}des fils d'Immer, Hanani et Zebadia ;
\VS{21}des fils de Harim, Maaséja, Elie, Schemaeja, Jehiel et Ozias ;
\VS{22}des fils de Paschhur, Eljoénaï, Maaséja, Ismaël, Nethaneel, Jozabad et Eleasa.
\VS{23}Parmi les Lévites : Jozabad, Schimeï, Kélaja (ou Kelitha) Pethachja, Juda et Eliézer.
\VS{24}Parmi les chantres : Eliaschib. Et des portiers : Schallum, Thélem et Uri.
\VS{25}Parmi ceux d'Israël : Des fils de Pareosch, Ramia, Jizzija, Malkija, Mijamin, Eléazar, Malkija et Benaja ;
\VS{26}des fils d’Elam, Matthania, Zacharie, Jehiel, Abdi, Jérémoth et Elie ;
\VS{27}des fils de Zatthu, Eljoénaï, Eliaschib, Matthania, Jérémoth, Zabad et Aziza ;
\VS{28}des fils de Bébaï, Jochanan, Hanania, Zabbaï et Athlaï ;
\VS{29}des fils de Bani, Meschullam, Malluc, Adaja, Jaschub, Scheal et Ramoth ;
\VS{30}des fils de Pachath-Moab, Adna, Kelal, Benaja, Maaséja, Matthania, Betsaleel, Binnuï et Manassé ;
\VS{31}des fils de Harim, Eliézer, Jischija, Malkija, Schemaeja, Siméon,
\VS{32}Benjamin, Malluc et Schemaria ;
\VS{33}des fils de Haschum, Matthnaï, Matthattha, Zabad, Eliphéleth, Jerémaï, Manassé et Schimeï ;
\VS{34}des fils de Bani, Maadaï, Amram, Uel,
\VS{35}Benaja, Bédia, Keluhu,
\VS{36}Vania, Merémoth, Eliaschib,
\VS{37}Matthania, Matthnaï, Jaasaï,
\VS{38}Bani, Binnuï, Schimeï,
\VS{39}Schélémia, Nathan, Adaja,
\VS{40}Macnadbaï, Schaschaï, Scharaï,
\VS{41}Azareel, Schélémia, Schemaria,
\VS{42}Schallum, Amaria et Joseph ;
\VS{43}des fils de Nebo, Jeïel, Matthithia, Zabad, Zebina, Jaddaï, Joël et Benaja.
\VS{44}Tous ceux-là avaient pris des femmes étrangères ; et  quelques-uns avaient eu des fils avec ces femmes-là.
\PPE{}
\end{multicols}

%\clearpage\ShortTitle{Néhémie}\BookTitle{Néhémie}\BFont
\noindent\hrulefill
{\footnotesize
\textit{
\bigskip
{\centering{}
\\Auteur : Néhémie
\\(Heb. : Nechemyah)
\\Signification : Yahweh a consolé
\\Thème : Reconstruction des murailles de Jérusalem
\\Date de rédaction : 5\up{ème} siècle av. J.-C.\\}
}
%\bigskip
\textit{
\\En apprenant l'état de ruine dans lequel se trouvait Jérusalem, Néhémie, échanson du roi perse Artaxerxés Ier, fut profondément affecté. Après plusieurs jours dans la désolation et l'humiliation, le Seigneur toucha le cœur du roi qui lui donna l'autorisation et le matériel nécessaire pour rebâtir la muraille de Jérusalem. Malgré les nombreuses oppositions dont il fit l'objet au cours de son entreprise, Néhémie acheva l'œuvre qui lui avait été confiée. Dans le même temps, il mit en place de profondes réformes dans le cadre du retour à la loi de Yahweh.
%\bigskip
\\Complément du livre d'Esdras avec lequel il ne formait initialement qu'un ouvrage, le livre de Néhémie présente un homme de prière, un serviteur œuvrant pour, avec, et au Nom de Yahweh.\bigskip
}
}
\par\nobreak\noindent\hrulefill
\begin{multicols}{2}
\Chap{1}
\TextTitle{La détresse du peuple resté à Jérusalem est racontée à Néhémie}
\VerseOne{}Paroles de Néhémie, fils de Hacalia. Il arriva au mois de Kisleu, la vingtième année, comme j'étais à Suse, la capitale,
\VS{2}Hanani, l'un de mes frères et quelques hommes arrivèrent de Juda. Je les questionnai au sujet des Juifs réchappés qui étaient restés de la captivité et au sujet de Jérusalem.
\VS{3}Et ils me dirent : Ceux qui sont restés de la captivité sont là dans la province, dans une grande misère et dans l'opprobre ; et la muraille de Jérusalem demeure renversée et ses portes ont été consumées par le feu.
\TextTitle{Néhémie prie Yahweh et implore sa grâce}
\VS{4}Or il arriva que, dès que j'entendis ces paroles, je m'assis, je pleurai et je fus dans le deuil plusieurs jours. Je jeûnai et je priai devant le Dieu des cieux,
\VS{5} et je dis : Je te prie, ô Yahweh ! Dieu des cieux, Dieu grand et redoutable, qui garde l'alliance et la miséricorde de ceux qui t'aiment et qui observent tes commandements !
\VS{6}Je te prie que ton oreille soit attentive et que tes yeux soient ouverts pour entendre la prière que ton serviteur te présente en ce temps-ci, jour et nuit, pour tes serviteurs les enfants d'Israël, en confessant les péchés des enfants d'Israël, que nous avons commis contre toi ; même moi et la maison de mon père, nous avons péché.
\VS{7}Certainement nous sommes coupables devant toi, nous n'avons pas gardé les commandements, les lois et les ordonnances que tu prescrivis à Moïse, ton serviteur.
\VS{8}Mais, je te prie, souviens-toi de la parole que tu chargeas Moïse, ton serviteur, de dire : Vous pécherez et je vous disperserai parmi les peuples\FTNT{De. 28:63-67.} ;
\VS{9}mais si vous revenez à moi, et si vous gardez mes commandements et les observez ; et s'il y en a d'entre vous qui ont été chassés jusqu'à l'extrémité du ciel, je vous rassemblerai de là, et je vous ramènerai au lieu que j'aurai choisi pour y faire habiter mon Nom\FTNT{De. 30:1-10.}.
\VS{10}Ils sont tes serviteurs et ton peuple, que tu as rachetés par ta grande puissance et par ta main forte.
\VS{11}Je te prie donc, Seigneur, que ton oreille soit maintenant attentive à la prière de ton serviteur, et à la prière de tes serviteurs qui prennent plaisir à craindre ton Nom ! Je te prie, donne aujourd'hui du succès à ton serviteur, et fais-lui trouver grâce devant cet homme ! J'étais alors échanson du roi.
\Chap{2}
\TextTitle{Yahweh exauce Néhémie et lui donne la faveur du roi}
\VerseOne{}Et il arriva, au mois de Nisan, la vingtième année du roi Artaxerxès, comme le vin était devant lui, je pris le vin et le présentai au roi. Je n'avais jamais été triste devant lui\FTNT{Pr. 15:13.}.
\VS{2}Et le roi me dit : Pourquoi as-tu mauvais visage, puisque tu n'es point malade ? Cela ne peut être qu'une tristesse de cœur. Je fus alors saisi d'une grande crainte,
\VS{3}et je répondis au roi : Que le roi vive éternellement ! Comment n'aurais-je pas mauvais visage, puisque la ville où sont les sépulcres de mes pères demeure désolée et que ses portes ont été consumées par le feu ?
\VS{4}Et le roi dit : Que me demandes-tu ? Alors je priai le Dieu des cieux,
\VS{5}et je dis au roi : Si le roi le trouve bon, et si ton serviteur lui est agréable, envoie-moi en Juda, vers la ville des sépulcres de mes pères, pour la rebâtir\FTNT{La reconstruction de la ville de Jérusalem sous Néhémie date, selon certains, de l'an 445 av. J.-C., suite au décret d'Artaxerxès. Cette date marquerait le point de départ des soixante-dix semaines d'années annoncées par Daniel (Da. 9:24-27).}.
\VS{6}Le roi me dit, et sa femme aussi qui était assise auprès de lui : Combien ton voyage durera-t-il, et quand seras-tu de retour ? Je lui précisai le temps, et le roi trouva bon de m'envoyer.
\VS{7}Puis je dis au roi : Si le roi le trouve bon, qu'on me donne des lettres pour les gouverneurs de l'autre côté du fleuve, afin qu'ils me laissent passer, jusqu'à ce que j'arrive en Juda ;
\VS{8}et des lettres pour Asaph, le garde de la forêt du roi, afin qu'il me donne du bois pour la charpente des portes de la forteresse près de la maison, pour les murailles de la ville, et pour la maison dans laquelle j'entrerai. Et le roi me l'accorda, car la main de mon Dieu était bonne sur moi.
\TextTitle{Arrivée à Jérusalem, constat des murailles en ruines}
\VS{9}J'allai donc vers les gouverneurs qui sont de l'autre côté du fleuve et je leur donnai les lettres du roi. Le roi avait aussi envoyé avec moi des chefs de l'armée et des cavaliers.
\VS{10}Quand Sanballat, le Horonite, et Tobija, le serviteur Ammonite, l'ayant appris, ils eurent un très grand déplaisir de ce qu'il venait un homme pour procurer du bien aux enfants d'Israël.
\VS{11}Ainsi j'arrivai à Jérusalem et j'y passai trois jours.
\VS{12}Puis je me levai de nuit, avec quelques hommes ; mais je ne dis à personne ce que Dieu avait mis dans mon cœur de faire pour Jérusalem. Il n'y avait point d'autre bête avec moi que celle sur laquelle j'étais monté.
\VS{13}Je sortis donc de nuit par la porte de la vallée et me dirigeai vers la source du dragon, vers la porte du fumier ; et je considérai les murailles de Jérusalem qui étaient en ruines\FTNT{Jé. 39:8.}, et ses portes consumées par le feu.
\VS{14}Je passai près de la porte de la source et vers l'étang du roi ; et il n'y avait point de place par où je puisse passer avec ma monture.
\VS{15}Je montai de nuit par le torrent et je considérai la muraille. Puis en revenant, je rentrai par la porte de la vallée ; et ainsi je fus de retour.
\VS{16}Or les magistrats ne savaient pas où j'étais allé, ni ce que je faisais ; car je n'avais rien dit jusqu'à ce moment, ni aux Juifs, ni aux sacrificateurs, ni aux chefs, ni aux magistrats, ni au reste de ceux qui s'occupaient des affaires.
\TextTitle{Néhémie partage sa vision de rebâtir la muraille}
\VS{17}Alors je leur dis : Vous voyez la misère dans laquelle nous sommes ! Comment Jérusalem demeure désolée et ses portes brûlées par le feu ! Venez et rebâtissons les murailles de Jérusalem et nous ne serons plus dans l'opprobre.
\VS{18}Et je leur déclarai comment la main de mon Dieu avait été bonne sur moi, et quelles paroles le roi m'avait dites. Alors ils dirent : Levons-nous et bâtissons ! Ils fortifièrent leurs mains pour bien faire.
\TextTitle{Premières oppositions}
\VS{19}Mais Sanballat, le Horonite, Tobija, le serviteur Ammonite, et Guéschem, l'Arabe, l'ayant appris, se moquèrent de nous et nous méprisèrent. Ils dirent : Qu'est-ce que vous faites ? Ne vous rebellez-vous pas contre le roi ?
\VS{20}Et je leur répondis cette parole : Le Dieu des cieux lui-même nous donnera le succès ! Nous donc, qui sommes ses serviteurs, nous nous lèverons et nous bâtirons ; mais vous, vous n'avez aucune part, ni droit, ni souvenir, à Jérusalem.
\Chap{3}
\TextTitle{Les participants à la reconstruction de la muraille}
\VerseOne{}Eliaschib, le souverain sacrificateur, se leva donc avec ses frères, les sacrificateurs et ils rebâtirent la porte des brebis\FTNT{La première porte qui fut reconstruite fut la porte des brebis. Cette porte est très proche du temple, c'est par elle que l'on faisait entrer les brebis destinées aux sacrifices dans la cour du temple. Cette porte est la préfiguration de Jésus-Christ qui s'est lui-même présenté comme étant la « porte des brebis » (Jn. 10:7).}. Ils la sanctifièrent, ils y posèrent ses battants. Ils la sanctifièrent depuis la tour de Méa jusqu'à la tour de Hananeel.
\VS{2}Et les gens de Jéricho rebâtirent à son côté ; et à côté d'eux Zaccur, fils d'Imri, rebâtit aussi.
\VS{3}Les fils de Senaa rebâtirent la porte des poissons. Ils en firent la charpente et y mirent ses portes, ses serrures et ses barres.
\VS{4}Et à leur côté travailla aux réparations Merémoth, fils d'Urie, fils d'Hakkots ; et à leur côté travailla Meschullam, fils de Bérékia, fils de Meschézabeel, et à leur côté travailla Tsadok, fils de Baana.
\VS{5}A leur côté travaillèrent les Tekoïtes ; mais les chefs d'entre eux ne vinrent point au service de leur Seigneur.
\VS{6}Et Jojada, fils de Paséach, et Meschullam, fils de Besodia, réparèrent la vieille porte. Ils en firent la charpente, y mirent ses battants, ses serrures et ses barres.
\VS{7}A leur côté travaillèrent Melatia, le Gabaonite, Jadon, le Méronothite, et les hommes de Gabaon et de Mitspa, vers le siège du gouverneur de ce côté du fleuve.
\VS{8}A côté d'eux travailla Uzziel, fils de Harhaja, d'entre les orfèvres, et à côté de lui travailla Hanania, d'entre les parfumeurs. Et ainsi ils relevèrent Jérusalem jusqu'à la muraille large.
\VS{9}Et à leur côté travailla Rephaja, fils de Hur, chef d'un demi-quartier de Jérusalem.
\VS{10}Puis à leur côté travailla Jedaja, fils de Harumaph, devant sa maison ; et à son côté travailla Hattusch, fils de Haschabnia.
\VS{11}Et Malkija, fils de Harim, et Haschub, fils de Pachath-Moab, en réparèrent une seconde section, et la tour des fours.
\VS{12}Et à leur côté travailla, avec ses filles, Schallum, fils de d'Hallochesch, chef de la moitié du quartier de Jérusalem.
\VS{13}Hanun et les habitants de Zanoach réparèrent la porte de la vallée. Ils la rebâtirent et mirent ses battants, ses serrures, et ses barres, et ils bâtirent mille coudées de muraille, jusqu'à la porte du fumier.
\VS{14}Et Malkija, fils de Récab, chef du quartier de Beth-Hakkérem, répara la porte du fumier. Il la rebâtit et mit ses battants, ses serrures et ses barres.
\VS{15}Schallum, fils de Col-Hozé, chef du quartier de Mitspa, répara la porte de la source. Il la rebâtit et la couvrit, et mit ses portes, ses serrures, et ses barres. Il répara aussi la muraille de l'étang de Siloé, vers le jardin du roi, et jusqu'aux marches qui descendent de la cité de David.
\VS{16}Après lui travailla Néhémie, fils d'Azbuk, chef de la moitié du quartier de Beth-Tsur, jusqu'à l'endroit des sépulcres de David, et jusqu'à l'étang qui avait été refait, et jusqu'à la maison des hommes vaillants.
\VS{17}Après lui travaillèrent les Lévites, Rehum, fils de Bani ; et à son côté travailla Haschabia, chef de la moitié du quartier de Keïla, pour ceux de son quartier.
\VS{18}Après lui travaillèrent leurs frères, Bavvaï, fils de Hénadad, chef de la moitié du quartier de Keïla.
\VS{19}A son côté, Ezer, fils de Josué, chef de Mitspa, en répara autant, à l'endroit où l'on monte à l'arsenal, à l'angle.
\VS{20}Après lui Baruc, fils de Zabbaï, répara avec ardeur une seconde section, depuis l'angle jusqu'à la porte de la maison d'Eliaschib, le souverain sacrificateur.
\VS{21}Après lui Merémoth, fils d'Urie, fils d'Hakkots, répara une seconde section, depuis l'entrée de la maison d'Eliaschib, jusqu'à l'extrémité de la maison d'Eliaschib.
\VS{22}Et après lui travaillèrent les sacrificateurs, habitants des environs.
\VS{23}Après eux, Benjamin et Haschub travaillèrent devant leur maison. Après eux, Azaria, fils de Maaséja, fils d'Anania, travailla auprès de sa maison.
\VS{24}Après lui, Binnuï, fils de Hénadad, répara une seconde section, depuis la maison d'Azaria jusqu'à l'angle et jusqu'au coin.
\VS{25}Palal, fils d'Uzaï, travailla vis-à-vis de l'angle, et de la tour qui sort de la tour supérieure du roi, qui est auprès de la cour de la prison. Après lui travailla Pedaja, fils de Pareosch.
\VS{26}Les Néthiniens, qui demeuraient sur la colline, réparèrent vers l'orient, jusqu'à l'endroit de la porte des eaux, et vers la tour qui sort.
\VS{27}Après eux, les Tekoïtes réparèrent une seconde section, depuis l'endroit de la grande tour qui sort en dehors, jusqu'à la muraille de la colline.
\VS{28}Au-dessus de la porte des chevaux, les sacrificateurs travaillèrent, chacun devant de sa maison.
\VS{29}Après eux, Tsadok, fils d'Immer, travailla devant sa maison. Après lui répara Schemaeja, fils de Schecania, gardien de la porte orientale.
\VS{30}Après lui, Hanania, fils de Schélémia et Hanun le sixième fils de Tsalaph, en réparèrent une seconde section. Après eux, Meschullam, fils de Bérékia, travailla vis-à-vis de sa chambre.
\VS{31}Après lui, Malkija, fils de l'orfèvre, répara jusqu'à la maison des Néthiniens et des marchands, vis-à-vis de la porte de Miphkad, et jusqu'à la chambre haute du coin.
\VS{32}Et les orfèvres et les marchands travaillèrent entre la chambre haute du coin et la porte des brebis.
\Chap{4}
\TextTitle{La prière, solution pour faire face aux attaques et moqueries}
\VerseOne{}Or il arriva que Sanballat apprit que nous rebâtissions la muraille, il devint furieux et très fâché. Il se moqua des Juifs.
\VS{2}Et il dit en présence de ses frères, et des gens de guerre de Samarie : Que font ces faibles Juifs ? Les laissera-t-on faire ? Sacrifieront-ils ? Et achèveront-ils tout en un jour ? Pourront-ils faire revenir à la vie les pierres des monceaux de poussière, puisqu'elles sont brûlées ?
\VS{3}Et Tobija, l'Ammonite, qui était auprès de lui, dit : Qu'ils bâtissent encore ! Si un renard monte, il rompra leur muraille de pierre !
\VS{4}Ô notre Dieu, écoute comment nous sommes méprisés ! Fais retourner leurs insultes sur leur tête, et donne-les en pillage dans un pays de captivité.
\VS{5}Ne couvre point leur iniquité, et que leur péché ne soit point effacé de devant ta face ; car ils ont irrité les bâtisseurs.
\VS{6}Nous rebâtîmes donc la muraille, et tout le mur fut achevé jusqu'à sa moitié ; et le peuple avait le cœur au travail.
\VS{7}Mais quand Sanballat et Tobija, les Arabes, les Ammonites et les Asdodiens eurent appris que la muraille de Jérusalem avait été refaite, et qu'on avait commencé à fermer les brèches, ils s'enflammèrent de colère.
\VS{8}Et ils se liguèrent tous ensemble pour venir faire la guerre contre Jérusalem, et pour les faire échouer.
\VS{9}Alors nous priâmes notre Dieu, et ayant peur d'eux, nous établîmes une garde jour et nuit pour nous défendre contre leurs attaques.
\TextTitle{Persévérance du peuple prêt à se battre à tout moment}
\VS{10}Et Juda disait : La force des ouvriers est affaiblie, et il y a beaucoup de débris, en sorte que nous ne pourrons pas bâtir la muraille.
\VS{11}Et nos ennemis disaient : Qu'ils n'en sachent rien et qu'ils ne voient rien, jusqu'à ce que nous entrions au milieu d'eux ; nous les tuerons et ferons ainsi cesser l'ouvrage.
\VS{12}Mais il arriva que les Juifs, qui habitaient près d'eux, vinrent dix fois nous avertir, de tous les lieux d'où ils se rendaient vers nous.
\VS{13}C'est pourquoi je plaçai le peuple depuis le bas, derrière la muraille, et sur des lieux élevés, secs et lumineux, selon leurs familles, avec leurs épées, leurs lances et leurs arcs.
\VS{14}Puis je regardai et m'étant levé, je dis aux chefs, aux magistrats et au reste du peuple : N'ayez point peur d'eux ! Souvenez-vous du Seigneur, qui est grand et terrible, et combattez pour vos frères, pour vos fils et pour vos filles, pour vos femmes et pour vos maisons !
\VS{15}Et quand nos ennemis entendirent que nous étions avertis, Dieu fit échouer leur projet, et nous retournâmes tous aux murailles, chacun à son travail.
\VS{16}Depuis ce jour-là, la moitié de mes serviteurs travaillait, et l'autre moitié avait des lances, des boucliers, des arcs et des cuirasses. Les gouverneurs suivaient chaque maison de Juda.
\VS{17}Ceux qui bâtissaient la muraille, et ceux qui portaient ou chargeaient les fardeaux, travaillaient chacun d'une main, et de l'autre ils tenaient une arme.
\VS{18}Car chacun de ceux qui bâtissaient avait son épée ceinte autour des reins. Et celui qui sonnait du shofar se tenait près de moi.
\VS{19}Et je dis aux chefs, aux magistrats et au reste du peuple : L'ouvrage est grand et étendu, et nous sommes séparés sur la muraille, éloignés les uns des autres.
\VS{20}En quelque lieu donc d'où vous entendrez le son du shofar, courez-y vers nous ; notre Dieu combattra pour nous\FTNT{Ex. 14:14 ; De. 1:30 ; 2 Ch. 20:29.}.
\VS{21}C'est donc ainsi que nous accomplissions le travail ; la moitié tenait des lances, depuis le lever du jour jusqu'à l'apparition des étoiles.
\VS{22}En ce temps-là, je dis aussi au peuple : Que chacun passe la nuit dans Jérusalem avec son serviteur, afin de faire la garde la nuit et de travailler le jour.
\VS{23}Et nous ne quittions point nos vêtements, ni moi, ni mes frères, ni mes serviteurs, ni les hommes de garde qui me suivaient; chacun n'avait que ses armes et de l'eau.
\Chap{5}
\TextTitle{Cupidité des chefs dévoilée ; rétablissement de la justice}
\VerseOne{}Or il y eut un grand cri du peuple et de leurs femmes, contre les Juifs, leurs frères.
\VS{2}Les uns disaient : Nous, nos fils et nos filles, nous sommes nombreux; qu'on nous donne du blé, afin que nous mangions et que nous vivions.
\VS{3}Et d'autres disaient : Nous engageons nos champs, nos vignes et nos maisons, pour avoir du blé pendant la famine.
\VS{4}D'autres disaient : Nous avons emprunté de l'argent sur nos champs et sur nos vignes pour le tribut du roi.
\VS{5}Toutefois notre chair est comme la chair de nos frères, et nos fils sont comme leurs fils ; et voici, nous soumettons à la servitude nos fils et nos filles ; et quelques-unes de nos filles sont déjà esclaves et ne sont plus en notre pouvoir ; et nos champs et nos vignes sont à d'autres.
\VS{6}Je fus très en colère quand j'entendis leur cri et ces paroles-là.
\VS{7}Je résolus dans mon cœur de réprimander les chefs et les magistrats, et je leur dis : Vous prêtez avec intérêt à vos frères\FTNT{Ex.22:25 ; Lé. 25:36.} ! Et je fis convoquer autour d'eux une grande foule.
\VS{8}Et je leur dis : Nous avons racheté selon notre pouvoir nos frères Juifs vendus aux nations, et vous vendriez vous-mêmes vos frères, ou c'est à nous qu'ils seraient vendus ? Ils se turent, ne trouvant rien à dire.
\VS{9}Et je dis : Ce que vous faites n'est pas bien. Ne voulez-vous pas marcher dans la crainte de notre Dieu, plutôt que d'être insultés par les nations qui sont nos ennemies ?
\VS{10}Moi aussi, mes frères et mes serviteurs, nous leur avons prêté de l'argent et du blé. Abandonnons je vous prie, cette dette !
\VS{11}Rendez-leur, je vous prie, aujourd'hui leurs champs, leurs vignes, leurs oliviers et leurs maisons ; et outre cela, le centième de l'argent, du blé, du vin, et de l'huile que vous exigez d'eux.
\VS{12}Et ils répondirent : Nous les rendrons et nous ne leur demanderons rien ; nous ferons ce que tu dis. Alors j'appelai les sacrificateurs et je les fis jurer de tenir parole.
\VS{13}Et je secouai mon bras et je dis : Que Dieu secoue ainsi de sa maison et de son travail tout homme qui n'aura pas tenu parole, et qu'il soit ainsi secoué et vidé ! Et toute l'assemblée répondit : Amen ! Et ils louèrent Yahweh. Et le peuple fit selon cette parole.
\TextTitle{Néhémie, modèle de dévouement}
\VS{14}Et même, depuis le jour où le roi m'établit comme gouverneur au pays de Juda, depuis la vingtième année jusqu'à la trente-deuxième année du roi Artaxerxès, pendant douze ans, moi et mes frères, nous n'avons pas pris ce qui était assigné au gouverneur comme revenu.
\VS{15}Quoique, les premiers gouverneurs qui avaient été avant moi, chargeaient le peuple, et prenaient de lui du pain et du vin, outre quarante sicles d'argent, et leurs serviteurs tyrannisaient le peuple. Mais je n'ai point fait ainsi, à cause de la crainte de mon Dieu.
\VS{16}Et même, j'ai travaillé à la réparation d'une partie de cette muraille, et nous n'avons acheté aucun champ, et tous mes serviteurs étaient tous ensemble à l'ouvrage.
\VS{17}Et outre cela, j'avais aussi à ma table les Juifs et les magistrats, au nombre de cent cinquante hommes, et ceux qui venaient vers nous des nations d'alentour.
\VS{18}On m'apprêtait chaque jour un bœuf, six moutons choisis et aussi des volailles ; et tous les dix jours on me présentait toutes sortes de vins en abondance. Malgré cela, je n'ai point demandé le revenu qui était assigné au gouverneur ; parce que les travaux étaient à la charge de ce peuple.
\VS{19}Ô mon Dieu ! Souviens-toi de moi en bien, à cause de tout ce que j'ai fait pour ce peuple.
\Chap{6}
\TextTitle{Complot et mensonge contre Néhémie ; fermeté et confiance en Dieu}
\VerseOne{}Or il arriva que quand Sanballat, Tobija, et Guéschem l'Arabe, et le reste de nos ennemis apprirent que j'avais rebâti la muraille, et qu'il n'y restait aucune brèche. (bien que jusqu'à ce temps-là, je n'avais pas encore mis les battants aux portes.)
\VS{2}Alors Sanballat et Guéschem envoyèrent vers moi, pour dire : Viens, et ayons ensemble une rencontre dans les villages qui sont dans la vallée d'Ono. Or ils avaient comploté de me faire du mal.
\VS{3}Mais j'envoyai des messagers vers eux pour leur dire : J'ai un grand ouvrage à faire, et je ne puis descendre. Le travail serait interrompu pendant que je le quitterais pour aller vers vous.
\VS{4}Ils m'adressèrent la même chose quatre fois ; et je leur répondis la même réponse.
\VS{5}Alors Sanballat m'envoya son serviteur pour me tenir le même discours une cinquième fois ; et il avait dans sa main une lettre ouverte.
\VS{6}Il y était écrit : On entend dire parmi les nations, et Gaschmu le dit, que vous pensez, toi et les Juifs, à vous révolter, et que c'est pour cela que tu rebâtis la muraille. Et tu vas, dit-on, devenir leur roi ;
\VS{7}Même que tu as ordonné des prophètes pour te louer dans Jérusalem, et pour dire : Il est roi de Juda. Et maintenant, on fera entendre au roi ces mêmes choses. Viens donc afin que nous consultions ensemble.
\VS{8}Et je renvoyai vers lui pour lui dire : Ce que tu dis là n'est point, mais c'est toi qui l'inventes dans ton propre coeur !
\VS{9}Car tous ces gens voulaient nous effrayer, en disant : Leurs mains relâcheront le travail, de sorte qu'il ne se fera point. Maintenant donc, ô Dieu, fortifie-moi !
\VS{10}Je me rendis à la maison de Schemaeja, fils de Delaja, fils de Mehétabeel. Il s'était enfermé et il me dit : Assemblons-nous dans la maison de Dieu, au milieu du temple et fermons les portes du temple ; car ils doivent venir pour te tuer, et ils viendront pendant la nuit pour te tuer.
\VS{11}Mais je répondis : Un homme tel que moi s'enfuirait-il ? Et quel homme tel que moi pourrait entrer dans le temple pour sauver sa vie ? Je n'y entrerai point.
\VS{12}Et voilà, je reconnus bien que Dieu ne l'avait point envoyé, mais qu'il avait prononcé cette prophétie contre moi parce que Sanballat et Tobija lui avaient donné de l'argent.
\VS{13}Car il était leur pensionnaire pour m'épouvanter, et pour m'obliger à agir de la sorte, et à commettre cette faute, afin qu'ils aient quelque mauvaise chose à me reprocher.
\VS{14}Ô mon Dieu ! Souviens-toi de Tobija et de Sanballat, et de leurs actions et aussi de Noadia, la prophétesse, et du reste des prophètes qui cherchaient à m'effrayer !
\TextTitle{Achèvement de la muraille}
\VS{15}Néanmoins, la muraille fut achevée le vingt-cinquième jour du mois d'Elul, en cinquante-deux jours.
\VS{16}Quand donc tous nos ennemis l'apprirent et qu'ils la virent, toutes les nations qui étaient autour de nous furent dans la crainte ; elles éprouvèrent une grande humiliation, et ils reconnurent que cet ouvrage s'était accompli par le secours de notre Dieu.
\VS{17}Mais aussi en ce temps-là, les chefs de Juda adressaient fréquemment des lettres à Tobija, et celles de Tobija venaient à eux.
\VS{18}Car il y en avait plusieurs en Juda qui s'étaient liés à lui par serment, parce qu'il était gendre de Schecania, fils d'Arach, et que son fils Jochanan avait pris la fille de Meschullam, fils de Bérékia.
\VS{19}Ils racontaient même du bien de lui en ma présence, et lui rapportaient mes paroles. Et Tobija envoyait des lettres pour m'effrayer.
\Chap{7}
\TextTitle{Instructions spécifiques à Hanani et Hanania}
\VerseOne{}Or après que la muraille fut rebâtie, et que j'aie mis les portes, et qu'on ait fait la revue des portiers, des chantres et des Lévites ; 
\VS{2}je donnai cet ordre à Hanani, mon frère, et à Hanania, chef de la forteresse de Jérusalem ; car il était tel qu'un homme fidèle doit être, et il craignait Dieu plus que plusieurs autres ;
\VS{3}et je leur dis : Que les portes de Jérusalem ne s'ouvrent point avant la chaleur du soleil ; et pendant que les gardes seront encore là, que l'on ferme les portes, et qu'on y mette les barres ; que l'on place comme gardes les habitants de Jérusalem, chacun à son poste, et chacun devant de sa maison.
\VS{4}Or la ville était spacieuse et grande, mais il y avait peu de gens, et ses maisons n'étaient point bâties\FTNT{De. 4:27.}.
\TextTitle{Liste des familles revenues de captivité avec Zorobabel}
\VS{5}Et mon Dieu me mit à coeur d'assembler les chefs, les magistrats et le peuple, pour en faire le dénombrement selon leurs généalogies. Je trouvai le registre du dénombrement selon les généalogies de ceux qui étaient montés la première fois. Et j'y trouvai ainsi écrit :
\VS{6}Ce sont ici ceux de la province qui remontèrent de la captivité, d'entre ceux que Nebucadnetsar, roi de Babylone, avait transportés en exil, et qui retournèrent à Jérusalem et en Juda, chacun dans sa ville.
\VS{7}Ils vinrent avec Zorobabel\FTNT{Esd. 5:2.}, Josué, Néhémie, Azaria, Raamia, Nachamani, Mardochée, Bilschan, Mispéreth, Bigvaï, Nehum, et Baana. Nombre des hommes du peuple d'Israël :
\VS{8}Les fils de Pareosch, deux mille cent soixante-douze.
\VS{9}Les fils de Schephathia, trois cent soixante-douze.
\VS{10}Les fils d'Arach, six cent cinquante-deux.
\VS{11}Les fils de Pachath-Moab, des fils de Josué et de Joab, deux mille huit cent dix-huit.
\VS{12}Les fils d'Elam, mille deux cent cinquante-quatre.
\VS{13}Les fils de Zatthu, huit cent quarante-cinq.
\VS{14}Les fils de Zaccaï, sept cent soixante.
\VS{15}Les fils de Binnuï, six cent quarante-huit.
\VS{16}Les fils de Bébaï, six cent vingt-huit.
\VS{17}Les fils d'Azgad, deux mille trois cent vingt-deux.
\VS{18}Les fils d'Adonikam, six cent soixante-sept.
\VS{19}Les fils de Bigvaï, deux mille soixante-sept.
\VS{20}Les fils d'Adin, six cent cinquante-cinq.
\VS{21}Les fils d'Ather, issu d'Ezéchias, quatre-vingt-dix-huit.
\VS{22}Les fils de Haschum, trois cent vingt-huit.
\VS{23}Les fils de Betsaï, trois cent vingt-quatre.
\VS{24}Les fils de Hariph, cent douze.
\VS{25}Les fils de Gabaon, quatre-vingt-quinze.
\VS{26}Les gens de Bethléhem et de Netopha, cent quatre-vingt-huit.
\VS{27}Les gens d'Anathoth, cent vingt-huit.
\VS{28}Les gens de Beth-Azmaveth, quarante-deux.
\VS{29}Les gens de Kirjath-Jearim, de Kephira et de Beéroth, sept cent quarante-trois.
\VS{30}Les gens de Rama et de Guéba, six cent vingt et un.
\VS{31}Les gens de Micmas, cent vingt-deux.
\VS{32}Les gens de Béthel et d'Aï, cent vingt-trois.
\VS{33}Les gens de l'autre Nebo, cinquante-deux.
\VS{34}Les fils d'un autre Elam, mille deux cent cinquante-quatre.
\VS{35}Les fils de Harim, trois cent vingt.
\VS{36}Les fils de Jéricho, trois cent quarante-cinq.
\VS{37}Les fils de Lod, de Hadid et d'Ono, sept cent vingt et un.
\VS{38}Les fils de Senaa, trois mille neuf cent trente.
\TextTitle{Liste des sacrificateurs revenus de captivité}
\VS{39}Sacrificateurs : Les fils de Jedaeja, de la maison de Josué, neuf cent soixante-treize.
\VS{40}Les fils d'Immer, mille cinquante-deux.
\VS{41}Les fils de Paschhur, mille deux cent quarante-sept.
\VS{42}Les fils de Harim, mille dix-sept.
\TextTitle{Liste des Lévites revenus de captivité}
\VS{43}Lévites : Les fils de Josué et de Kadmiel, d'entre les fils de Hodva, soixante quatorze.
\VS{44}Chantres : Les fils d'Asaph, cent quarante-huit.
\VS{45}Portiers : Les fils de Schallum, les fils d'Ather, les fils de Thalmon, les fils d'Akkub, les fils de Hathitha, les fils de Schobaï, cent trente-huit.
\TextTitle{Liste des Néthiniens revenus de captivité}
\VS{46}Néthiniens : Les fils de Tsicha, les fils de Hasupha, les fils de Thabbaoth,
\VS{47}les fils de Kéros, les fils de Sia, les fils de Padon,
\VS{48}les fils de Lebana, les fils de Hagaba, les fils de Salmaï,
\VS{49}les fils de Hanan, les fils de Guiddel, les fils de Gachar,
\VS{50}les fils de Reaja, les fils de Retsin, les fils de Nekoda,
\VS{51}les fils de Gazzam, les fils d'Uzza, les fils de Paséach,
\VS{52}les fils de Bésaï, les fils de Mehunim, les fils de Nephischsim,
\VS{53}les fils de Bakbuk, les fils de Hakupha, les fils de Harhur,
\VS{54}les fils de Batslith, les fils de Mehida, les fils de Harscha,
\VS{55}les fils de Barkos, les fils de Sisera, les fils de Thamach,
\VS{56}les fils de Netsiach, les fils de Hathipha.
\TextTitle{Liste des fils des serviteurs de Salomon revenus de captivité}
\VS{57}Fils des serviteurs de Salomon : Les fils de Sothaï, les fils de Sophéreth, les fils de Perida,
\VS{58}les fils de Jaala, les fils de Darkon, les fils de Guiddel,
\VS{59}les fils de Schephathia, les fils de Hatthil, les fils de Pokéreth-Hatsebaïm, les fils d'Amon.
\VS{60}Tous les Néthiniens, et les fils des serviteurs de Salomon, étaient trois cent quatre-vingt-douze.
\VS{61}Voici ceux qui montèrent de Thel-Mélach, de Thel-Harscha, de Kerub-Addon et d'Immer, lesquels ne purent montrer la maison de leurs pères, ni leur race, pour prouver qu'ils étaient d'Israël.
\VS{62}Les fils de Delaja, les fils de Tobija, les fils de Nekoda, six cent quarante-deux.
\TextTitle{Liste des sacrificateurs exclus de la sacrificature}
\VS{63}Et les sacrificateurs : Les fils de Hobaja, les fils d'Hakkots, les fils de Barzillaï, qui prit pour femme une des filles de Barzillaï, le Galaadite, et qui fut appelé de leur nom.
\VS{64}Ils cherchèrent leur registre généalogique, mais ils n'y furent point trouvés ; c'est pourquoi ils furent exclus de la sacrificature.
\VS{65}Et le gouverneur leur dit de ne pas manger des choses très saintes, jusqu'à ce que le sacrificateur eût consulté l'urim et le thummim\FTNT{Ex. 28:30.}.
\TextTitle{Somme des Israélites revenus de captivité}
\VS{66}Toute l'assemblée réunie était de quarante-deux mille trois cent soixante ;
\VS{67}sans leurs serviteurs et leurs servantes, qui étaient sept mille trois cent trente-sept ; et ils avaient deux cent quarante-cinq chantres ou chanteuses.
\TextTitle{Dons des fils d'Israël pour le trésor}
\VS{68}Ils avaient sept cent trente-six chevaux, deux cent quarante-cinq mulets ;
\VS{69}quatre cent trente-cinq chameaux et six mille sept cent vingt ânes.
\VS{70}Or quelques-uns des chefs des pères firent des dons pour l'ouvrage. Le gouverneur donna au trésor mille drachmes d'or, cinquante coupes, cinq cent trente tuniques de sacrificateurs.
\VS{71}Quelques autres d'entre les chefs des pères donnèrent pour le trésor de l'ouvrage vingt mille drachmes d'or et deux mille deux cent mines d'argent.
\VS{72}Le reste du peuple donna vingt mille drachmes d'or, deux mille mines d'argent et soixante-sept tuniques de sacrificateurs.
\VS{73}Et ainsi les sacrificateurs, les Lévites, les portiers, les chantres, quelques-uns du peuple, les Néthiniens, et tous ceux d'Israël habitèrent dans leurs villes. Ainsi, quand le septième mois approcha, les enfants d'Israël étaient dans leurs villes.
\Chap{8}
\TextTitle{Lecture du livre de la loi, conviction de péché du peuple}
\VerseOne{}Or tout le peuple s'assembla, comme un seul homme, sur la place qui est devant la porte des eaux. Et ils dirent à Esdras, le scribe, d'apporter le livre de la loi de Moïse, que Yahweh avait ordonnée à Israël.
\VS{2}Et ainsi le premier jour du septième mois, Esdras, le sacrificateur, apporta la loi devant l'assemblée, composée d'hommes et de femmes, et de tous ceux qui étaient capables de l'entendre.
\VS{3}Et il lut dans le livre, sur la place qui est devant la porte des eaux, depuis le matin jusqu'au milieu du jour, en présence des hommes et des femmes, et de ceux qui étaient capables d'entendre. Et les oreilles de tout le peuple étaient attentives à la lecture du livre de la loi.
\VS{4}Ainsi Esdras, le scribe, était debout sur une tour bâtie de bois, qu'on avait dressée pour cela. Il avait auprès de lui, à sa droite, Matthithia, Schéma, Anaja, Urie, Hilkija et Maaséja ; et à sa gauche étaient Pedaja, Mischaël, Malkija, Haschum, Haschbaddana, Zacharie, et Meschullam.
\VS{5}Esdras ouvrit le livre devant les yeux de tout le peuple ; car il était au-dessus de tout le peuple ; et sitôt qu'il l'eut ouvert, tout le peuple se tint debout.
\VS{6}Puis Esdras bénit Yahweh, le grand Dieu ; et tout le peuple répondit en élevant leurs mains: Amen ! Amen ! Et ils s'inclinèrent et se prosternèrent devant Yahweh, le visage contre terre.
\VS{7}Aussi Josué, Bani, Schérébia, Jamin, Akkub, Schabbethaï, Hodija, Maaséja, Kelitha, Azaria, Jozabad, Hanan, Pelaja, et les Lévites, faisaient comprendre la loi au peuple, et le peuple se tenait à sa place.
\VS{8}Et ils lisaient dans le livre de la loi de Dieu, ils l'expliquaient et en donnaient l'intelligence, la faisant comprendre par l'Ecriture elle-même.
\VS{9}Or Néhémie, qui est le gouverneur, Esdras, le sacrificateur et le scribe, et les Lévites qui instruisaient le peuple dirent à tout le peuple : Ce jour est consacré à Yahweh, notre Dieu ; ne soyez pas dans les lamentations, et ne pleurez point ! Car tout le peuple pleurait en entendant les paroles de la loi.
\VS{10}Puis on leur dit : Allez, mangez des viandes grasses, et buvez du vin doux ; et envoyez-en des portions à ceux qui n'ont rien de prêt ; car ce jour est consacré à notre Seigneur. Ne soyez donc point tristes, car la joie de Yahweh est votre force.
\VS{11}Et les Lévites faisaient faire silence parmi tout le peuple, en disant : Taisez-vous, car ce jour est saint, et ne vous affligez point.
\VS{12}Ainsi tout le peuple s'en alla pour manger et pour boire, pour envoyer des portions, et pour faire une grande réjouissance, parce qu'ils avaient bien compris les paroles qu'on leur avait fait connaître.
\TextTitle{Célébration de la fête des tabernacles}
\VS{13}Et le second jour, les chefs des pères de tout le peuple, les sacrificateurs et les Lévites, s'assemblèrent auprès d'Esdras, le scribe, pour sagement comprendre les paroles de la loi.
\VS{14}Et ils trouvèrent écrit dans la loi que Yahweh avait ordonnée par Moïse, que les enfants d'Israël devaient habiter sous des tentes\FTNT{Voir les sept fêtes de Yahweh en Lé. 23.} pendant la fête solennelle au septième mois.
\VS{15}Ce qu'ils firent savoir et qu'ils publièrent dans toutes leurs villes et à Jérusalem, en disant : Allez sur la montagne, et apportez des rameaux d'oliviers, et des rameaux d'autres arbres huileux, des rameaux de myrte, des rameaux de palmier, et des rameaux d'arbres touffus, afin de faire des tentes, selon ce qui est écrit.
\VS{16}Alors le peuple alla et apporta des rameaux. Ils se firent des tentes, chacun sur son toit, dans les cours de leurs maisons, et dans les parvis de la maison de Dieu, sur la place de la porte des eaux, et sur la place de la porte d'Ephraïm.
\VS{17}Ainsi toute l'assemblée de ceux qui étaient revenus de la captivité fit des tentes, et ils habitèrent sous ces tentes. Or les enfants d'Israël n'en avaient point fait de telles depuis les jours de Josué, fils de Nun, jusqu'à ce jour ; et il y eut une très grande joie.
\VS{18}On lut dans le livre de la loi de Dieu chaque jour, depuis le premier jour jusqu'au dernier. On célébra la fête pendant sept jours, et il y eut une assemblée solennelle au huitième jour, comme cela est ordonné.
\Chap{9}
\TextTitle{Confession, jeune et prière du peuple}
\VerseOne{}Et le vingt-quatrième jour du même mois, les enfants d'Israël s'assemblèrent, jeûnant, revêtus de sacs, et ayant de la terre sur eux.
\VS{2}Et la race d'Israël se sépara de tous les étrangers, et ils se présentèrent confessant leurs péchés et les iniquités de leurs pères.
\VS{3}Ils se levèrent donc à leur place, et on lut dans le livre de la loi de Yahweh, leur Dieu, pendant un quart de la journée, et pendant un autre quart, ils faisaient confession, et se prosternaient devant Yahweh, leur Dieu.
\TextTitle{Prière des Lévites, alliance avec Yahweh}
\VS{4}Josué, Bani, Kadmiel, Schebania, Bunni, Schérébia, Bani et Kenani se levèrent sur le lieu qu'on avait élevé pour les Lévites, et crièrent à haute voix à Yahweh, leur Dieu.
\VS{5}Et les Lévites Josué, Kadmiel, Bani, Haschabnia, Schérébia, Hodija, Schebania et Pethachja, dirent : Levez-vous, bénissez Yahweh, votre Dieu, d'éternité en éternité ! Que l'on bénisse ton Nom glorieux, qui est au-dessus de toute bénédiction et de toute louange !
\VS{6}Toi seul, Yahweh, tu as fait les cieux, les cieux des cieux, et toute leur armée ; la terre, et tout ce qui y est ; les mers, et toutes les choses qui y vivent. Tu donnes la vie à toutes ces choses, et l'armée des cieux se prosterne devant toi.
\VS{7}Tu es Yahweh, notre Dieu, qui as choisi Abram, et qui l'as fait sortir d'Ur en Chaldée, et qui lui as donné le nom d'Abraham\FTNT{Ge. 11:31 ; Ge. 17:5.}.
\VS{8}Tu trouvas son coeur fidèle devant toi, et tu traitas avec lui cette alliance que tu donneras à sa postérité le pays des Cananéens, des Héthiens, des Amoréens, des Phéréziens, des Jébusiens, et des Guirgasiens. Et tu as accompli ce que tu as promis, parce que tu es juste.
\VS{9}Car tu vis l'affliction de nos pères en Egypte et tu entendis leurs cris près de la Mer Rouge\FTNT{Ex. 2:23-25.}.
\VS{10}Tu fis des miracles et des prodiges sur Pharaon et sur tous ses serviteurs, et sur tout le peuple de son pays ; parce que tu connus qu'ils s'étaient orgueilleusement élevés contre eux, et tu t'es acquis un renom, tel qu'il paraît aujourd'hui.
\VS{11}Tu fendis aussi la mer devant eux, et ils passèrent à sec au milieu de la mer ; et tu jetas dans l'abîme ceux qui les poursuivaient, comme une pierre dans les eaux violentes.
\VS{12}Tu les fis marcher de jour par la colonne de nuée, et de nuit par la colonne de feu, pour les éclairer dans le chemin par où ils devaient aller\FTNT{Ex. 13:21.}.
\VS{13}Tu descendis sur la montagne de Sinaï, tu parlas avec eux du haut des cieux, tu leur donnas des ordonnances justes et des lois de vérité, des statuts et des commandements bons.
\VS{14}Tu leur fis connaître ton saint sabbat\FTNT{Ge 2:1-3 ; Ex. 20:8-11.} ; et tu leur donnas les commandements, les statuts, et la loi par Moïse, ton serviteur.
\VS{15}Tu leur donnas aussi, du haut des cieux, du pain quand ils avaient faim, et tu fis sortir de l'eau du rocher quand ils avaient soif\FTNT{Ex. 16:13-36 ; No. 20 : 8.}. Et tu leur dis d'entrer et de posséder le pays que tu avais juré de leur donner.
\VS{16}Mais nos pères s'élevèrent orgueilleusement et raidirent leur cou. Ils n'écoutèrent point tes commandements.
\VS{17}Ils refusèrent d'écouter et ne se souvinrent point des merveilles que tu avais faites en leur faveur. Mais ils raidirent leur cou, et par leur rébellion, ils s'attribuèrent un chef pour retourner à leur servitude. Mais toi, tu es un Dieu qui pardonne, miséricordieux, compatissant, lent à la colère et abondant en bonté, et tu ne les abandonnas pas.
\VS{18}Et quand ils se firent un veau en métal fondu et qu'ils dirent : Voici ton Dieu qui t'a fait sortir hors d'Egypte, et qu'ils te firent de grands outrages\FTNT{Ex. 32:1-14.} ;
\VS{19}dans ton immense miséricorde, tu ne les abandonnas pourtant pas dans le désert ; et la colonne de nuée ne se retira point pour les conduire le jour par le chemin, ni la colonne de feu la nuit, pour les éclairer dans le chemin par lequel ils devaient aller.
\VS{20}Tu leur donnas ton bon Esprit pour les rendre sages ; tu ne retiras point ta manne de leur bouche, et tu leur donnas de l'eau pour leur soif.
\VS{21}Tu les nourris ainsi quarante ans au désert, en sorte que rien ne leur manqua. Leurs vêtements ne s'usèrent point, et leurs pieds ne s'enflèrent point.
\VS{22}Tu leur donnas les royaumes et les peuples, dont tu partageas entre eux les contrées ; et ils possédèrent le pays de Sihon, le pays du roi de Hesbon, et le pays d'Og, roi de Basan.
\VS{23}Et tu multiplias leurs fils comme les étoiles des cieux, et les fis entrer au pays dont tu avais dit à leurs pères qu'ils y entreraient pour le posséder.
\VS{24}Ainsi leurs fils y entrèrent et possédèrent le pays ; tu humilias devant eux les habitants du pays, les Cananéens, et les livras entre leurs mains, eux et leurs rois, et les peuples du pays, afin qu'ils en fissent selon leur volonté.
\VS{25}Ils prirent les villes fortifiées et la terre grasse, ils possédèrent les maisons remplies de toutes sortes de biens, les puits qu'on avait creusés, les vignes, les oliviers, et les arbres fruitiers en abondance ; ils mangèrent, ils se rassasièrent ; ils s'engraissèrent et ils vécurent dans les délices de ta grande bonté.
\VS{26}Mais ils se rebellèrent et se révoltèrent contre toi. Ils jetèrent ta loi derrière leur dos, ils tuèrent tes prophètes qui les avertissaient pour les ramener à toi, et ils te firent de grands outrages.
\VS{27}C'est pourquoi tu les donnas aux mains de leurs ennemis, qui les opprimèrent. Mais au temps de leur détresse, ils crièrent à toi, et tu les entendis des cieux ; et selon ta grande miséricorde, tu leur donnas des libérateurs qui les délivrèrent de la main de leurs ennemis.
\VS{28}Mais dès qu'ils eurent du repos, ils recommencèrent à faire le mal devant toi. Alors tu les abandonnas entre les mains de leurs ennemis, qui dominèrent sur eux. Puis ils revinrent et crièrent vers toi, et tu les entendis des cieux. Ainsi tu les délivras selon tes miséricordes, plusieurs fois, et en divers temps.
\VS{29}Et tu les exhortas à revenir à ta loi, mais ils s'élevèrent orgueilleusement et n'écoutèrent pas tes commandements ; ils péchèrent contre tes ordonnances, qui font vivre l'homme qui les observe. Ils tirèrent l'épaule en arrière, raidirent leur cou et n'écoutèrent pas.
\VS{30}Tu les supportas patiemment plusieurs années, et tu les avertissais par ton Esprit, par la main de tes prophètes ; mais ils ne prêtèrent point l'oreille. C'est pourquoi tu les livras entre les mains des peuples des pays étrangers.
\VS{31}Néanmoins, dans ta grande miséricorde, tu ne les anéantis pas et tu ne les abandonnas pas ; car tu es un Dieu compatissant et miséricordieux.
\VS{32}Et maintenant donc, ô notre Dieu ! Grand, puissant et terrifiant, qui garde ton alliance et la miséricorde ; ne regarde pas comme peu de chose cette affliction qui nous est arrivée, à nous, à nos rois, à nos chefs, à nos sacrificateurs, à nos prophètes, à nos pères et à tout ton peuple, depuis le temps des rois d'Assyrie jusqu'à aujourd'hui.
\VS{33}Tu as été juste dans toutes les choses qui nous sont arrivées ; car tu as agi avec fidélité, mais nous, nous avons agi méchamment.
\VS{34}Nos rois, nos chefs, nos sacrificateurs et nos pères n'ont point pratiqué ta loi et n'ont point été attentifs à tes commandements ni à tes témoignages par lesquels tu les as avertis.
\VS{35}Ils ne t'ont point servi durant leur règne ni durant les grands biens que tu leur as faits, même dans le pays vaste et riche que tu leur avais donné pour être à leur disposition, et ils ne se sont point détournés de leurs mauvaises oeuvres.
\VS{36}Voici, nous sommes aujourd'hui esclaves ! Sur la terre que tu as donnée à nos pères pour en manger le fruit et les biens ; voici, nous y sommes esclaves !
\VS{37}Elle rapporte ses produits en abondance pour les rois que tu as établis sur nous à cause de nos péchés, et qui dominent sur nos corps et sur nos bêtes, à leur volonté, de sorte que nous sommes dans une grande angoisse !
\VS{38}C'est pourquoi, à cause de toutes ces choses, nous contractâmes une alliance et nous l'écrivîmes ; et les chefs d'entre nous, nos Lévites et nos sacrificateurs y apposèrent leur sceau.
\Chap{10}
\TextTitle{Liste des contractants et termes de l'alliance}
\VerseOne{}Voici ceux qui apposèrent leur sceau. Néhémie, qui est le gouverneur, fils de Hacalia, et Sédécias.
\VS{2}Seraja, Azaria, Jérémie,
\VS{3}Paschhur, Amaria, Malkija,
\VS{4}Hattusch, Schebania, Malluc,
\VS{5}Harim, Merémoth, Abdias,
\VS{6}Daniel, Guinnethon, Baruc,
\VS{7}Meschullam, Abija, Mijamin,
\VS{8}Maazia, Bilgaï et Schemaeja. Ce sont les sacrificateurs.
\VS{9}Des Lévites : Josué, fils d'Azania, Binnuï d'entre les fils de Hénadad, et Kadmiel.
\VS{10}Et leurs frères, Schebania, Hodija, Kelitha, Pelaja, Hanan,
\VS{11}Michée, Rehob, Haschabia.
\VS{12}Zaccur, Schérébia, Schebania,
\VS{13}Hodija, Bani et Beninu.
\VS{14}Des chefs du peuple : Pareosch, Pachath-Moab, Elam, Zatthu, Bani,
\VS{15}Bunni, Azgad, Bébaï,
\VS{16}Adonija, Bigvaï, Adin,
\VS{17}Ather, Ezéchias, Azzur,
\VS{18}Hodija, Haschum, Betsaï,
\VS{19}Hariph, Anathoth, Nébaï,
\VS{20}Magpiasch, Meschullam, Hézir,
\VS{21}Meschézabeel, Tsadok, Jaddua,
\VS{22}Pelathia, Hanan, Anaja,
\VS{23}Hosée, Hanania, Haschub,
\VS{24}Hallochesch, Pilcha, Schobek,
\VS{25}Rehum, Haschabna, Maaséja,
\VS{26}Achija, Hanan, Anan,
\VS{27}Malluc, Harim et Baana.
\VS{28}Quant au reste du peuple, les sacrificateurs, les Lévites, les portiers, les chantres, les Néthiniens et tous ceux qui s'étaient séparés des peuples de ces pays pour suivre la loi de Dieu, leurs femmes, leurs fils et leurs filles, tous ceux qui étaient capables de connaissance et d'intelligence,
\VS{29}se joignirent à leurs frères les plus considérables d'entre eux. Ils s'engagèrent par serment et jurèrent de marcher dans la loi de Dieu, qui avait été donnée par Moïse, serviteur de Dieu ; de garder et faire tous les commandements de Yahweh, notre Seigneur, ses jugements et ses ordonnances ;
\VS{30}de ne pas donner nos filles aux peuples du pays, et de ne pas prendre leurs filles pour nos fils ;
\VS{31}de ne rien prendre le jour du sabbat, ou tel autre jour consacré, des peuples du pays qui apporteraient des marchandises et toutes sortes de denrées, le jour du sabbat, pour les vendre, d'abandonner la septième année et de faire remise de toute dette.
\VS{32}Nous fîmes aussi des ordonnances, nous chargeant de donner chaque année le tiers d'un sicle, pour le service de la maison de notre Dieu,
\VS{33}pour les pains de proposition, pour l'offrande perpétuelle et pour l'holocauste perpétuel ; pour ceux des sabbats, des nouvelles lunes et des fêtes ; pour les choses consacrées, pour les sacrifices d'expiation afin de faire propitiation pour Israël ; et pour toute l'oeuvre de la maison de notre Dieu.
\VS{34}Nous tirâmes au sort, pour l'offrande du bois, tant les sacrificateurs et les Lévites, que le peuple, afin de l'amener dans la maison de notre Dieu, selon les maisons de nos pères, et dans les temps fixés, d'année en année, pour le brûler sur l'autel de Yahweh, notre Dieu, ainsi qu'il est écrit dans la loi.
\VS{35}Nous décidâmes aussi d'apporter dans la maison de Yahweh, d'année en année, les premiers fruits de notre terre, et les prémices de tous les fruits de tous les arbres ;
\VS{36}d'amener les premiers-nés de nos fils, et de nos bêtes, comme il est écrit dans la loi ; et d'amener dans la maison de notre Dieu, aux sacrificateurs qui font le service dans la maison de notre Dieu, les premiers-nés de nos boeufs et de notre menu bétail;
\VS{37}d'apporter les prémices de notre pâte, nos offrandes, les fruits de tous les arbres, le vin, et l'huile aux sacrificateurs, dans les chambres de la maison de notre Dieu, et la dîme de notre terre aux Lévites, et que les Lévites prendraient les dîmes dans toutes les villes agricoles.
\VS{38}Le sacrificateur, fils d'Aaron, sera avec les Lévites, lorsque les Lévites paieront la dîme\FTNT{Il est question de la dîme des Lévites (No. 18:24 ; De. 14:28-29).}; et les Lévites apporteront la dîme de la dîme\FTNT{Il s'agit ici de la dîme de la dîme que les Lévites donnaient aux sacificateurs. Elle était apportée aux magasins du temple. Voir commentaires en No. 18:21 et Mal. 3:10.} à la maison de notre Dieu, dans les chambres de la maison où sont les magasins\FTNT{Le mot hébreu « owtsar » (trésor) signifie aussi magasin (Né. 12:44 ; Né. 13:12. Né. 13:13).}.
\VS{39}Car les enfants d'Israël et les fils de Lévi apporteront dans ces chambres les offrandes du blé, du vin et de l'huile ; là sont les ustensiles du sanctuaire, et les sacrificateurs qui font le service, les portiers, et les chantres. Et nous n'abandonnâmes point la maison de notre Dieu.
\Chap{11}
\TextTitle{Les habitants de Jérusalem}
\VerseOne{}Les chefs du peuple demeurèrent à Jérusalem. Mais tout le reste du peuple tira au sort, afin qu'un sur dix vînt habiter à Jérusalem, la ville sainte, et que les neuf autres parties demeurassent dans les autres villes.
\VS{2}Et le peuple bénit tous ceux qui se présentèrent volontairement pour habiter à Jérusalem.
\VS{3}Voici les chefs de la province qui habitèrent à Jérusalem ; les autres s'étant établis dans les villes de Juda, chacun dans sa propriété, selon sa ville, Israélites, sacrificateurs, Lévites, Néthiniens, et les fils des serviteurs de Salomon.
\VS{4}A Jérusalem habitèrent donc des fils de Juda et des fils de Benjamin. Des fils de Juda : Athaja, fils d'Ozias, fils de Zacharie, fils d'Amaria, fils de Schephathia, fils de Mahalaleel, d'entre les fils de Pérets,
\VS{5}et Maaséja, fils de Baruc, fils de Col-Hozé, fils de Hazaja, fils d'Adaja, fils de Jojarib, fils de Zacharie, fils de Schiloni.
\VS{6}Total des fils de Pérets, qui s'établirent à Jérusalem : Quatre cent soixante-huit vaillants hommes.
\VS{7}Voici les fils de Benjamin : Sallu, fils de Meschullam, fils de Joëd, fils de Pedaja, fils de Kolaja, fils de Maaséja, fils d'Ithiel, fils d'Esaïe,
\VS{8}et après lui, Gabbaï et Sallaï : Neuf cent vingt-huit.
\VS{9}Joël, fils de Zicri, était leur chef ; et Juda, fils de Senua, était le second chef de la ville.
\VS{10}Des sacrificateurs : Jedaeja, fils de Jojarib, Jakin,
\VS{11}Seraja, fils de Hilkija, fils de Meschullam, fils de Tsadok, fils de Merajoth, fils d'Achithub, prince de la maison de Dieu,
\VS{12}et leurs frères, faisant le service de la maison : Huit cent vingt-deux. Adaja, fils de Jerocham, fils de Pelalia, fils d'Amtsi, fils de Zacharie, fils de Paschhur, fils de Malkija,
\VS{13}et ses frères, chefs des pères : Deux cent quarante-deux ; et Amaschsaï, fils d'Azareel, fils d'Achzaï, fils de Meschillémoth, fils d'Immer,
\VS{14}et leurs frères, forts et vaillants: Cent vingt-huit. Zabdiel, fils de Guedolim, était leur chef.
\VS{15}Des Lévites : Schemaeja, fils de Haschub, fils d'Azrikam, fils de Haschabia, fils de Bunni,
\VS{16}Schabbethaï et Jozabad chargés des travaux extérieurs pour la maison de Dieu, étant d'entre les chefs des Lévites ;
\VS{17}Matthania, fils de Michée, fils de Zabdi, fils d'Asaph, était le chef qui commençait le premier à chanter les louanges dans la prière, et Bakbukia, le second parmi ses frères, puis Abda, fils de Schammua, fils de Galal, fils de Jeduthun.
\VS{18}Total des Lévites dans la ville sainte : Deux cent quatre-vingt-quatre.
\VS{19}Et les portiers : Akkub, Thalmon, et leurs frères qui gardaient les portes : Cent soixante-douze.
\TextTitle{Les habitants des autres villes}
\VS{20}Le reste d'Israël, des sacrificateurs et des Lévites, fut dans toutes les villes de Juda, chacun dans son héritage.
\VS{21}Mais les Néthiniens habitèrent sur la colline ; et Tsicha et Guischpa étaient leurs chefs.
\VS{22}Celui qui avait la charge des Lévites à Jérusalem était Uzzi, fils de Bani, fils de Haschabia, fils de Matthania, fils de Michée, d'entre les fils d'Asaph, chantres, pour l'ouvrage de la maison de Dieu ;
\VS{23}car il y avait un commandement du roi à leur égard, et il y avait chaque jour un salaire assuré pour les chantres.
\VS{24}Pethachja, fils de Meschézabeel, d'entre les fils de Zérach, fils de Juda, était commissaire du roi pour toutes les affaires du peuple.
\VS{25}Dans les villages et leurs territoires, quelques-uns des fils de Juda habitèrent à Kirjath-Arba, et dans les lieux de son ressort ; à Dibon, et dans les lieux de son ressort ; à Jekabtseel, et dans les villages de son ressort,
\VS{26}à Jéschua, à Molada, à Beth-Paleth,
\VS{27}à Hatsar-Schual, à Beer-Schéba, et dans les lieux de son ressort,
\VS{28}à Tsiklag, à Mecona, et dans les lieux de son ressort,
\VS{29}à En-Rimmon, à Tsorea, à Jarmuth,
\VS{30}à Zanoach, à Adullam, et dans leurs villages, à Lakis et dans ses territoires, à Azéka et dans les lieux de son ressort. Ils habitèrent depuis Beer-Schéba jusqu'à la vallée de Hinnom.
\VS{31}Et les fils de Benjamin habitèrent depuis Guéba à Micmasch, à Ajja, à Béthel, et dans les lieux de son ressort,
\VS{32}à Anathoth, à Nob, à Hanania,
\VS{33}à Hatsor, à Rama, à Guitthaïm,
\VS{34}à Hadid, à Tseboïm, à Neballath,
\VS{35}à Lod, et à Ono, la vallée des ouvriers.
\VS{36}D'entre les Lévites, des classes de Juda se rattachèrent à Benjamin.
\Chap{12}
\TextTitle{Les sacrificateurs et les Lévites montés avec Zorobabel}
\VerseOne{}Voici les sacrificateurs et les Lévites qui montèrent avec Zorobabel, fils de Schealthiel, et avec Josué : Seraja, Jérémie, Esdras,
\VS{2}Amaria, Malluc, Hattusch,
\VS{3}Schecania, Rehum, Merémoth,
\VS{4}Iddo, Guinnethoï, Abija,
\VS{5}Mijamin, Maadia, Bilga,
\VS{6}Schemaeja, Jojarib, Jedaeja,
\VS{7}Sallu, Amok, Hilkija, Jedaeja. Ce furent là les chefs des sacrificateurs, et de leurs frères, du temps de Josué.
\VS{8}Lévites : Josué, Binnuï, Kadmiel, Schérébia, Juda, Matthania, qui dirigeait les louanges, lui et ses frères.
\VS{9}Bakbukia et Unni, leurs frères, étaient avec eux pour la surveillance.
\TextTitle{Les fils des sacrificateurs}
\VS{10}Josué engendra Jojakim, Jojakim engendra Eliaschib, Eliaschib engendra Jojada,
\VS{11}Jojada engendra Jonathan, et Jonathan engendra Jaddua.
\VS{12}Au temps de Jojakim, étaient sacrificateurs, chefs des pères : Pour Seraja, Meraja ; pour Jérémie, Hanania ;
\VS{13}pour Esdras, Meschullam ; pour Amaria, Jochanan ;
\VS{14}pour Meluki, Jonathan ; pour Schebania, Joseph ;
\VS{15}pour Harim, Adna ; pour Merajoth, Helkaï ;
\VS{16}pour Iddo, Zacharie ; pour Guinnethon, Meschullam ;
\VS{17}pour Abija, Zicri ; pour Minjamin et Moadia, Pilthaï ;
\VS{18}pour Bilga, Schammua ; pour Schemaeja, Jonathan ;
\VS{19}pour Jojarib, Matthnaï ; pour Jedaeja, Uzzi ;
\VS{20}pour Sallaï, Kallaï ; pour Amok, Eber ;
\VS{21}pour Hilkija, Haschabia ; pour Jedaeja, Nethaneel.
\TextTitle{Les chefs des fils de Lévi}
\VS{22}Au temps d'Eliaschib, de Jojada, de Jochanan et de Jaddua, les Lévites, chefs de famille, et les sacrificateurs, furent inscrits sous le règne de Darius, le Perse.
\VS{23}Les fils de Lévi, chefs des pères, furent enregistrés dans le livre des Chroniques jusqu'au temps de Jochanan, fils d'Eliaschib.
\VS{24}Les chefs des Lévites, Haschabia, Schérébia, et Josué, fils de Kadmiel, et leurs frères, étaient vis-à-vis d'eux, pour louer et célébrer, selon l'ordre de David, homme de Dieu.
\VS{25}Matthania, Bakbukia, Abdias, Meschullam, Thalmon, et Akkub, les portiers, faisaient la garde au seuil des portes.
\VS{26}Ce fut du temps de Jojakim, fils de Josué, fils de Jotsadak, et du temps de Néhémie, le gouverneur, et d'Esdras, sacrificateur et scribe.
\TextTitle{La dédicace de la muraille de Jérusalem}
\VS{27}Lors de la dédicace de la muraille de Jérusalem, on envoya chercher les Lévites de tous les lieux où ils étaient, pour les faire venir à Jérusalem, afin de célébrer la dédicace avec joie, par des louanges, et par des chants sur des cymbales, des luths et des harpes.
\VS{28}Les fils des chantres se rassemblèrent des plaines aux alentours de Jérusalem, des villages des Nethophatiens,
\VS{29}de Beth-Guilgal, et des territoires de Guéba et d'Azmaveth ; car les chantres s'étaient bâtis des villages aux alentours de Jérusalem.
\VS{30}Les sacrificateurs et les Lévites se purifièrent, et ils purifièrent le peuple, les portes et la muraille.
\VS{31}Puis je fis monter sur la muraille les chefs de Juda, et j'établis deux grands chœurs. Le premier se mit en marche du côté droit sur la muraille, vers la porte du fumier.
\VS{32}Et après eux marchait Hosée, avec la moitié des chefs de Juda,
\VS{33}Azaria, Esdras, Meschullam,
\VS{34}Juda, Benjamin, Schemaeja et Jérémie,
\VS{35}des fils des sacrificateurs avec les trompettes, Zacharie, fils de Jonathan, fils de Schemaeja, fils de Matthania, fils de Michée, fils de Zaccur, fils d'Asaph,
\VS{36}et ses frères, Schemaeja, Azareel, Milalaï, Guilalaï, Maaï, Nethaneel, Juda, et Hanani, avec les instruments des cantiques de David, homme de Dieu. Esdras, le scribe, marchait devant eux.
\VS{37}A la porte de la source, qui était vis-à-vis d'eux, ils montèrent aux marches de la cité de David, par la montée de la muraille, depuis la maison de David, jusqu'à la porte des eaux, vers l'orient.
\VS{38}Le second choeur de ceux qui chantaient les louanges allait à l'opposé. J'allais après lui, avec l'autre moitié du peuple, allant sur la muraille. Passant par-dessus la tour des fours, jusqu'à la muraille large ;
\VS{39}puis vers la porte d'Ephraïm, vers la vieille porte, vers la porte des poissons, la tour de Hananeel, et la tour de Méa, jusqu'à la porte des brebis. Et l'on s'arrêta à la porte de la prison.
\VS{40}Les deux choeurs s'arrêtèrent dans la maison de Dieu ; moi aussi, avec les magistrats qui étaient avec moi,
\VS{41}et les sacrificateurs Eliakim, Maaséja, Minjamin, Michée, Eljoénaï, Zacharie, Hanania, avec les trompettes,
\VS{42}et Maaséja, Schemaeja, Eléazar, Uzzi, Jochanan, Malkija, Elam et Ezer. Puis les chantres, desquels Jizrachja avait la charge, se firent entendre.
\VS{43}On offrit ce jour-là de nombreux sacrifices, et on se réjouit, parce que Dieu leur avait donné un grand sujet de joie. Les femmes et les enfants se réjouirent aussi ; et la joie de Jérusalem fut entendue au loin.
\TextTitle{Les sacrificateurs et les Lévites à leur poste}
\VS{44}En ce jour-là, on établit des hommes sur les chambres des trésors, des offrandes, des prémices et des dîmes ; pour rassembler du territoire des villes les portions ordonnées par la loi aux sacrificateurs et aux Lévites. Car Juda se réjouissait de ce que les sacrificateurs et de ce que les Lévites étaient à leur poste,
\VS{45}et parce qu'ils avaient gardé la charge qui leur avait été donnée de la part de leur Dieu, et la charge de la purification. Les chantres et les portiers remplissaient aussi leurs fonctions, selon le commandement de David, et de Salomon, son fils.
\VS{46}Car autrefois, du temps de David et d'Asaph, on avait établi des chefs de chantres et des cantiques de louange et de reconnaissance à Dieu.
\VS{47}Tout Israël, du temps de Zorobabel et de Néhémie, donna les portions des chantres et des portiers, jour par jour, et les consacraient aux Lévites, et les Lévites les consacraient aux fils d'Aaron.
\Chap{13}
\TextTitle{Lecture du livre de Moïse, séparation d'avec les étrangers}
\VerseOne{}Dans ce temps-là, on lut en présence du peuple dans le livre de Moïse, et l'on y trouva écrit que les Ammonites et les Moabites ne devaient jamais entrer dans l'assemblée de Dieu,
\VS{2}parce qu'ils n'étaient pas venus au-devant des enfants d'Israël avec du pain et de l'eau ; et qu'ils avaient engagé à prix d'argent Balaam\FTNT{Balaam : voir No. 22,23 et 24.} contre eux pour qu'il les maudisse ; mais notre Dieu changea la malédiction en bénédiction.
\VS{3}Dès qu'on eut entendu la loi, on sépara d'Israël tous les étrangers.
\TextTitle{Purification des chambres du temple}
\VS{4}Avant cela, le sacrificateur Eliaschib, établi sur les chambres de la maison de notre Dieu, et parent de Tobija,
\VS{5}avait disposé pour lui une grande chambre, où on mettait auparavant les offrandes, l'encens, les ustensiles, les dîmes du blé, du vin et de l'huile, qui étaient ordonnées pour les Lévites, pour les chantres et pour les portiers, avec les contributions pour les sacrificateurs.
\VS{6}Je n'étais point à Jérusalem pendant tout cela, car j'étais retourné vers le roi la trente-deuxième année d'Artaxerxès, roi de Babylone. Et à la fin de l'année, j'obtins du roi la permission
\VS{7}de revenir à Jérusalem, et je m'aperçus du mal qu'Eliaschib avait fait, en disposant une chambre pour Tobija dans le parvis de la maison de Dieu.
\VS{8}J'en éprouvai un vif déplaisir, et je jetai tous les objets de Tobija hors de la chambre ;
\VS{9}j'ordonnai qu'on purifie les chambres, et j'y ramenai les ustensiles de la maison de Dieu, les offrandes et l'encens.
\TextTitle{Sur les portions des Lévites}
\VS{10}J'appris aussi que les portions des Lévites ne leur avaient point été données ; et que les Lévites et les chantres qui faisaient le service s'étaient enfuis chacun sur sa terre.
\VS{11}Je fis des réprimandes aux magistrats, leur disant : Pourquoi a-t-on abandonné la maison de Dieu ? Je rassemblai les Lévites et les chantres, et les rétablis à leur place.
\VS{12}Alors tous ceux de Juda apportèrent dans le trésor les dîmes du blé, du vin et de l'huile.
\VS{13}Je confiai la surveillance du trésor à Schélémia, le sacrificateur, et Tsadok, le scribe, et Pedaja, l'un des Lévites ; et pour les aider, Hanan, fils de Zaccur, fils de Matthania, parce qu'ils étaient considérés comme très fidèles. Ils furent chargés de faire les distributions à leurs frères.
\VS{14}Souviens-toi de moi, ô mon Dieu, à cause de cela et n'efface point ce que j'ai fait avec fidélité pour la maison de mon Dieu, et pour ce qu'il est ordonné d'y faire !
\TextTitle{Avertissement pour le respect du sabbat}
\VS{15}En ces jours-là, je vis quelques-uns de Juda fouler aux pressoirs le jour du sabbat, et d'autres apporter des gerbes, et charger sur des ânes du vin, des raisins, des figues, et toutes autres sortes de fardeaux, et les apporter à Jérusalem le jour du sabbat ; et je les avertis le jour où ils vendaient leurs denrées.
\VS{16}Les Tyriens, qui demeuraient aussi à Jérusalem, apportaient du poisson, et plusieurs autres marchandises, et les vendaient aux fils de Juda dans Jérusalem le jour du sabbat.
\VS{17}Je fis des réprimandes aux chefs de Juda, et leur dis : Quel mal ne faites-vous pas, en violant le jour du sabbat ?
\VS{18}Vos pères n'ont-ils pas fait la même chose, et n'est-ce pas pour cela que notre Dieu a fait venir tout ce mal sur nous et sur cette ville ? Et vous amenez de nouveau son ardente colère contre Israël, en violant le sabbat !
\VS{19}C'est pourquoi, dès que le soleil s'était retiré des portes de Jérusalem, avant le sabbat, par mon commandement, on ferma les portes ; j'ordonnai aussi qu'on ne les ouvre point jusqu'après le sabbat. Et je plaçai quelques-uns de mes serviteurs aux portes, afin d'empêcher l'entrée des fardeaux le jour du sabbat.
\VS{20}Alors les marchands et les vendeurs de toutes sortes de denrées passèrent une ou deux fois la nuit hors de Jérusalem.
\VS{21}Je les avertis et je leur dis : Pourquoi passez-vous la nuit devant la muraille ? Si vous le faites encore, je mettrai la main sur vous. Ainsi, depuis ce temps-là, ils ne vinrent plus le jour du sabbat.
\VS{22}J'ordonnai aussi aux Lévites de se purifier, et de venir garder les portes pour sanctifier le jour du sabbat. Souviens-toi de moi, ô mon Dieu, à cause de cela, et ai compassion de moi selon la grandeur de ta miséricorde !
\TextTitle{Condamnation des unions mixtes ; rétablissement des fonctions des sacrificateurs et des Lévites}
\VS{23}En ces jours-là, je vis des Juifs qui avaient pris des femmes Asdodiennes, Ammonites et Moabites.
\VS{24}La moitié de leurs fils parlaient en partie asdodien et ne savaient point parler l'hébreu ; mais ils parlaient la langue de divers peuples.
\VS{25}Je leur fis des réprimandes et les maudis ; j'en frappai même quelques-uns, leur arrachai les cheveux et les fis jurer par Dieu, qu'ils ne donneraient point leurs filles à leurs fils, et qu'ils ne prendraient point leurs filles pour leurs fils, ou pour eux.
\VS{26}Salomon, le roi d'Israël, n'avait-il point péché par ce moyen ? Il n'y avait point de roi semblable à lui parmi un grand nombre de nations, il était aimé de son Dieu, et Dieu l'avait établi pour roi sur tout Israël ; toutefois, les femmes étrangères l'amenèrent à pécher.
\VS{27}Faut-il donc apprendre que vous fassiez tout ce grand mal, de commettre ce péché contre notre Dieu, en prenant des femmes étrangères ?
\VS{28}Or un des fils de Jojada, fils d'Eliaschib, grand sacrificateur, était gendre de Sanballat, le Horonite. Je le chassai loin de moi.
\VS{29}Souviens-toi d'eux, ô mon Dieu, car ils ont souillé le sacerdoce et l'alliance contractée par les sacrificateurs et les Lévites.
\VS{30}Ainsi je les nettoyai de tous les étrangers et je rétablis les fonctions des sacrificateurs et des Lévites, chacun selon ce qu'il avait à faire,
\VS{31}et ce qui concernait l'offrande du bois aux temps fixés, de même que les prémices. Souviens-toi de moi en bien, ô mon Dieu !
\PPE{}
\end{multicols}

%\clearpage\ShortTitle{1 Chroniques}\BookTitle{1 Chroniques}\BFont
\noindent\hrulefill
{\footnotesize
\textit{
\bigskip
{\centering{}
\\Auteur : Probablement Esdras
\\(Heb. : Hayyamim dibre)
\\Signification : Actes des journées
\\Thème : Généalogies et Histoire
\\Date de rédaction : 5ème siècle av. J.-C.\\}
}
%\bigskip
\textit{
\\Les deux livres des Chroniques constituent des compléments aux livres des Rois dans la mesure où ils confirment les récits de ceux-ci.
%\bigskip
\\Après avoir établi la généalogie d’Adam à Jacob, puis une généalogie plus détaillée de la descendance de Jacob jusqu’au retour de la captivité babylonienne, le premier livre des Chroniques reprend l’histoire du roi David et met un accent particulier sur certains combats qu’il eut à mener, les rapports avec ses serviteurs, ainsi que les préparatifs de la construction du temple. Il présente aussi l’organisation du travail des sacrificateurs et des Lévites au service de
Dieu et du peuple.\bigskip
}
}
\par\nobreak\noindent\hrulefill
\begin{multicols}{2}
\Chap{1}
\TextTitle{Généalogie d'Adam à Noé\FTNTT{Ge. 5:1-32}}
\VerseOne{}Adam, Seth, Enosch\FTNT{Les généalogies se faisaient par les premiers-nés de chaque famille.}.
\VS{2}Kénan, Mahalaleel, Jéred ;
\VS{3}Hénoc, Metuschélah, Lémec.
\VS{4}Noé, Sem, Cham et Japhet\FTNT{Ge. 5:1-32.}.
\TextTitle{Les fils de Japhet\FTNTT{Ge. 10:2-5.}}
\VS{5}Les fils de Japhet furent : Gomer, Magog, Madaï, Javan, Tubal, Méschec et Tiras.
\VS{6}Les fils de Gomer furent : Aschkenaz, Diphat et Togarma.
\VS{7}Les fils de Javan furent : Elischa, Tarsisa, Kittim et Rodanim.
\TextTitle{Les fils de Cham\FTNTT{Ge. 10:6-20.}}
\VS{8}Les fils de Cham furent : Cusch, Mitsraïm, Puth et Canaan.
\VS{9}Les fils de Cusch furent  : Saba, Havila, Sabta, Raema et Sabteca. Les fils de Raema furent : Séba et Dedan.
\VS{10}Cusch engendra aussi Nimrod qui commença à être puissant sur la terre.
\VS{11}Mitsraïm engendra les Ludim, les Anamim, les Lehabim, les Naphtuhim,
\VS{12}les Patrusim, les Casluhim, desquels sont issus les Philistins et les Caphtorim.
\VS{13}Canaan engendra Sidon, son fils aîné, et Heth;
\VS{14}les Jébusiens, les Amoréens, les Guirgasiens,
\VS{15}les Héviens, les Arkiens, les Siniens,
\VS{16}les Arvadiens, les Tsemariens et les Hamathiens.
\TextTitle{Les fils de Sem\FTNTT{Ge. 10:21-31.}}
\VS{17}Les fils de Sem furent : Elam, Assur, Arpacschad, Lud, Aram, Uts, Hul, Guéter et Méschec.
\VS{18}Arpacschad engendra Schélach, et Schélach engendra Héber.
\VS{19}A Héber naquirent deux fils : L'un s'appelait Péleg, car en son temps la terre fut partagée; et son frère s’appelait Jokthan.
\VS{20}Jokthan engendra Almodad, Schéleph, Hatsarmaveth, Jérach,
\VS{21}Hadoram, Uzal, Dikla,
\VS{22}Ebal, Abimaël, Séba,
\VS{23}Ophir, Havila et Jobab; tous ceux-là furent des fils de Jokthan\FTNT{Ge. 10:2-31.}.
\TextTitle{De Sem aux fils d'Abraham\FTNTT{Ge. 11:10-26.}}
\VS{24}Sem, Arpacschad, Schélach\FTNT{Ge. 11:10-26},
\VS{25}Héber, Péleg, Rehu,
\VS{26}Serug, Nachor, Térach,
\VS{27}et Abram, qui est Abraham.
\VS{28}Les fils d’Abraham furent  Isaac et Ismaël.
\TextTitle{Les fils d’Ismaël\FTNTT{Ge. 25:12-18.}}
\VS{29}Voici leur postérité\FTNT{Ge 25:12-18} : Le premier-né d'Ismaël fut Nebajoth, puis Kédar, Adbeel, Mibsam,
\VS{30}Mischma, Duma, Massa, Hadad, Téma,
\VS{31}Jethur, Naphisch et Kedma; ce sont là les fils d'Ismaël.
\TextTitle{Les fils de Ketura\FTNTT{Ge. 25:1-4.}}
\VS{32}Quant aux fils de Ketura, concubine d'Abraham, elle enfanta Zimran, Jokschan, Medan, Madian, Jischbak et Schuach; et les fils de Jokschan furent Séba et Dedan.
\VS{33}Les fils de Madian furent  Epha, Epher, Hénoc, Abida et Eldaa. Tous ceux-là furent les fils de Ketura.
\TextTitle{Les fils d’Isaac\FTNTT{Ge. 25:19-26.}}
\VS{34}Or Abraham engendra Isaac; et les fils d'Isaac furent  Esaü et Israël.
\TextTitle{Les descendants d’Esaü \FTNTT{Ge. 36:1-14.}}
\VS{35}Les fils d’Esaü furent  Eliphaz, Reuel, Jeusch, Jaelam et Koré\FTNT{Ge. 36:1-14.}.
\VS{36}Les fils d’Eliphaz furent Théman, Omar, Tsephi, Gaetham et Kenaz; Thimna lui enfanta Amalek.
\VS{37}Les fils de Reuel furent Nahath, Zérach, Schamma et Mizza.
\VS{38}Les fils de Séir furent Lothan, Schobal, Tsibeon, Ana, Dischon, Etser et Dischan.
\VS{39}Les fils de Lothan furent  Hori et Homam; et Thimna fut la soeur de Lothan.
\VS{40}Les fils de Schobal furent Aljan, Manahath, Ebal, Schephi et Onam. Les fils de Tsibeon furent Ajja et Ana.
\VS{41}Ana eut un fils : Dischon.  Les fils de Dischon furent Hamran, Eschban, Jithran et Keran.
\VS{42}Les fils d’Etser furent Bilhan, Zaavan et Jaakan. Les fils de Dischon furent Uts et Aran.
\TextTitle{Les rois et les chefs d’Edom\FTNTT{Ge. 36:15-19, 25-43}}
\VS{43}Voici les rois qui ont régné au pays d'Edom, avant qu’un roi ne règne sur les fils d’Israël : Béla, fils de Beor, et le nom de sa ville était Dinhaba.
\VS{44}Béla mourut, et Jobab, fils de Zérach de Botsra, régna à sa place.
\VS{45}Jobab mourut, et Huscham, du pays des Thémanites, régna à sa place.
\VS{46}Huscham mourut, et Hadad, fils de Bedad, régna à sa place. C’est lui qui frappa Madian dans les champs de Moab. Le nom de sa ville était Avith.
\VS{47}Hadad mourut, et Samla de Masréka, régna à sa place.
\VS{48}Samla mourut, et Saül de Rehoboth, sur le fleuve, régna à sa place.
\VS{49}Saül mourut, et Baal-Hanan, fils de Acbor, régna à sa place.
\VS{50}Baal-Hanan mourut, et Hadad régna à sa place. Le nom de sa ville était Pahi, et le nom de sa femme Mehéthabeel, qui était fille de Mathred, et petite- fille de Mézahab.
\VS{51}Enfin Hadad mourut. Ensuite vinrent les chefs d'Edom, le chef Thimna, le chef Alja, le chef Jetheth.
\VS{52}Le chef Oholibama, le chef Ela, le chef Pinon.
\VS{53}Le chef Kenaz, le chef Théman, le chef Mibtsar.
\VS{54}Le chef Magdiel, et le chef Iram. Ce sont là les chefs d'Edom.
\Chap{2}
\TextTitle{Les douze fils de Jacob (Israël)\FTNTT{Ge. 29:31-35 ; 30:6-24 ; 35:16-18}}
\VerseOne{}Voici les fils d'Israël : Ruben, Siméon, Lévi, Juda, Issacar, Zabulon,
\VS{2}Dan, Joseph, Benjamin, Nephthali, Gad et Aser.
\TextTitle{Les descendants de Juda jusqu’aux fils d’Hetsron\FTNTT{Ge. 46:12 ; No 26:19-22}}
\VS{3}Les fils de Juda furent  Er, Onan, et Schéla. Ces trois lui naquirent de la fille de Schua, la Cananéenne. Mais Er, premier-né de Juda, fut méchant aux yeux de Yahweh, qui le fit mourir.
\VS{4}Et Tamar, belle-fille de Juda, lui enfanta Pérets et Zérach. Tous les fils de Juda furent cinq.
\VS{5}Les fils de Pérets furent Hetsron et Hamul.
\VS{6}Et les fils de Zérach furent Zimri, Ethan, Héman, Calcol et Dara, cinq en tout.
\VS{7}Carmi n'eut point d’autre fils qu'Acar qui troubla Israël et qui pécha en prenant de l'interdit.
\VS{8}Ethan eut un seul fils : Azaria.
\VS{9}Les fils qui naquirent à Hetsron furent Jerachmeel, Ram et Kelubaï.
\TextTitle{Les descendants de Ram jusqu’à David\FTNTT{Ru. 4:17-22}}
\VS{10}Ram engendra Amminadab et Amminadab engendra Nachschon, chef des fils de Juda.
\VS{11}Nachschon engendra Salma et Salma engendra Boaz.
\VS{12}Boaz engendra Obed et Obed engendra Isaï.
\VS{13}Isaï engendra son premier-né Eliab, le second Abinadab, le troisième Schimea,
\VS{14}le quatrième Nethaneel, le cinquième Raddaï,
\VS{15}le sixième Otsem, et le septième, David.
\VS{16}Tseruja et Abigaïl furent leurs soeurs. Tseruja eut trois fils : Abischaï, Joab, et Asaël.
\VS{17}Abigaïl enfanta Amasa, dont le père fut Jéther l’Ismaélite.
\TextTitle{Les descendants de Caleb}
\VS{18}Or Caleb, fils de Hetsron, eut des enfants d’Azuba sa femme, et aussi de Jerioth; et ses fils furent Jéscher, Schobab et Ardon.
\VS{19}Azuba mourut, et Caleb prit pour femme Ephrath, qui lui enfanta Hur.
\VS{20}Hur engendra Uri, et Uri engendra Betsaleel.
\VS{21}Après cela, Hetsron vint vers la fille de Makir, père de Galaad, et la prit pour sa femme ; il était âgé de soixante ans, et elle lui enfanta Segub.
\VS{22}Segub engendra Jaïr, qui eut vingt-trois villes au pays de Galaad.
\VS{23}Il prit sur Gueschur et sur la Syrie les bourgades de Jaïr, et Kenath, avec les villes de son ressort, au nombre de soixante. Tous ceux-là furent fils de Makir, père de Galaad.
\VS{24}Après la mort de Hetsron, à Caleb-Ephratha, la femme de Hetsron, Abija, lui enfanta Aschchur, père de Tekoa.
\VS{25}Les fils de Jerachmeel, premier-né de Hetsron furent : Ram, son fils aîné, puis Buna, Oren et Otsem, nés d'Achija.
\VS{26}Jerachmeel eut aussi une autre femme, dont le nom était Athara, qui fut mère d'Onam.
\VS{27}Les fils de Ram, premier-né de Jerachmeel, furent Maats, Jamin et Eker.
\VS{28}Les fils  d'Onam furent  Schammaï et Jada; et les fils de Schammaï furent  Nadab et Abischur.
\VS{29}Le nom de la femme d'Abischur fut Abichaïl, qui lui enfanta Achban et Molid.
\VS{30}Les fils de Nadab furent Séled et Appaïm; mais Séled mourut sans fils.
\VS{31}Appaïm eut un seul fils : Jischeï. Jischeï eut un seul fils : Schéschan. Schéschan n'eut qu' Achlaï.
\VS{32}Les fils de Jada, frère de Schammaï, furent Jéther et Jonathan ; mais Jéther mourut sans fils.
\VS{33}Les fils de Jonathan furent Péleth et Zara ; ce furent là les fils de Jerachmeel.
\VS{34}Schéschan n'eut point de fils, mais des filles ; or il avait un serviteur Egyptien, dont le nom était Jarcha ;
\VS{35}Schéschan donna sa fille pour femme à Jarcha, son serviteur, et elle lui enfanta Attaï.
\VS{36}Attaï engendra Nathan, et Nathan engendra Zabad ;
\VS{37}Zabad engendra Ephlal; et Ephlal engendra Obed ;
\VS{38} Obed engendra Jéhu; Jéhu engendra Azaria;
\VS{39}Azaria engendra Halets;  Halets engendra Elasa;
\VS{40}Elasa engendra Sismaï; Sismaï engendra Schallum;
\VS{41}Schallum engendra Jekamja; Jekamja engendra Elischama.
\TextTitle{Les autres fils de Caleb}
\VS{42}Les fils de Caleb, frère de Jerachmeel, furent Méscha, son premier-né, qui fut le père de Ziph, et les fils de Maréscha, père d'Hébron.
\VS{43}Les fils d'Hébron furent  Koré, Thappuach, Rékem et Schéma.
\VS{44}Schéma engendra Racham, père de Jorkeam, et Rékem engendra Schammaï.
\VS{45}Le fils de Schammaï fut Maon. Maon fut père de Beth-Tsur.
\VS{46}Et Epha, concubine de Caleb, enfanta Haran, Motsa et Gazez; Haran aussi engendra Gazez.
\VS{47}Les fils de Jahdaï furent Réguem, Jotham, Guéschan, Péleth, Epha et Schaaph.
\VS{48}Maaca, la concubine de Caleb, enfanta Schéber et Tirchana.
\VS{49}La femme de Schaaph, père de Madmanna, enfanta Scheva, père de Macbéna, et le père de Guibea, et la fille de Caleb fut Acsa.
\TextTitle{Les descendants de Hur, fils de Caleb\FTNTT{v. 19 ; cp. 1 Ch. 4:1}}
\VS{50}Ceux-ci furent les fils de Caleb, fils de Hur, premier-né d'Ephrata : Schobal, père de Kirjath-Jearim.
\VS{51}Salma, père de Bethléhem, Hareph, père de Beth-Gader.
\VS{52}Schobal, père de Kirjath-Jearim, eut des fils : Haroé et Hatsi-Hammenuhoth.
\VS{53}Les familles de Kirjath-Jearim furent les Jéthriens, les Puthiens, les Schumathiens et les Mischraïens, desquels sont sortis les Tsoreathiens et les Eschthaoliens.
\VS{54}Les fils de Salma : Bethléhem et les Nethophatiens, Athroth-Beth-Joab, Hatsi-Hammanachthi et les Tsoreïens.
\VS{55}Et les familles des scribes, qui habitaient à Jaebets : Les Thireathiens, les Schimeathiens, les Sucathiens ; ce sont les Kéniens, qui sont sortis de Hamath père de Récab.
\Chap{3}
\TextTitle{Les fils de David\FTNTT{2 S. 3:2-5 ; 5:13-16}}
\VerseOne{}Voici les fils de David, qui lui naquirent à Hébron\FTNT{2 S. 3:2-5.}. Le premier-né fut Amnon, fils d' Achinoam de Jizreel; le second Daniel, d'Abigaïl de Carmel.
\VS{2}Le troisième, Absalom, fils de Maaca, fille de Talmaï, roi de Gueschur ; le quatrième, Adonija, fils de Haggith ;
\VS{3}le cinquième, Schephatia, d'Abithal; le sixième, Jithream, d'Egla sa femme.
\VS{4}Ces six lui naquirent à Hébron, où il régna sept ans et six mois ; puis il régna trente-trois ans à Jérusalem.
\VS{5}Ceux-ci lui naquirent à Jérusalem : Schimea, Schobab, Nathan et Salomon, tous quatre de Bath-Schua, fille d'Ammiel ;
\VS{6}et Jibhar, Elischama, Eliphéleth,
\VS{7}Noga, Népheg, Japhia,
\VS{8}Elischama, Eliada et Eliphéleth, qui sont neuf.
\VS{9}Ce sont tous des fils de David, outre les fils de ses concubines. Et Tamar était leur sœur.
\TextTitle{De Salomon à Sédécias}
\VS{10}Le fils de Salomon fut  Roboam. Abija, son fils ; Asa, son fils ; Josaphat, son fils ;
\VS{11}Joram, son fils ; Achazia, son fils ; Joas, son fils ;
\VS{12}Amatsia, son fils ; Azaria, son fils ; Jotham, son fils ;
\VS{13}Achaz, son fils ; Ezéchias, son fils ; Manassé, son fils ;
\VS{14}Amon, son fils ; Josias, son fils.
\VS{15}Les fils de Josias furent Jochanan, son premier-né ; le deuxième, Jojakim ; le troisième Sédécias ; le quatrième, Schallum.
\VS{16}Les fils de Jojakim furent  Jéconias, son fils, qui eut pour fils Sédécias.
\TextTitle{Les fils de Jéconias}
\VS{17}Quant aux fils de Jéconias, Assir qui fut emmené en captivité, Schealthiel fut son fils ;
\VS{18}dont les fils furent Malkiram, Pedaja, Schénatsar, Jekamia, Hoschama et Nedabia.
\VS{19}Les fils de Pedaja furent Zorobabel et Schimeï ; et les fils de Zorobabel furent Meschullam et Hanania ; et Schelomith était leur soeur.
\VS{20}De Meschullam, Haschuba, Ohel, Bérékia, Hasadia et Juschab-Hésed, en tout cinq.
\VS{21}Les fils de Hanania furent  Pelathia et Esaïe ; les fils de Rephaja, les fils d'Arnan, les fils d’Abdias et les fils de Schecania.
\VS{22}De Schecania naquit Schemaeja ; et les fils de Schemaeja, Hattusch, Jigueal, Bariach, Nearia, Schaphath, en tout six.
\VS{23}Les fils de Nearia furent trois : Eljoénaï, Ezéchias et Azrikam.
\VS{24}Et les fils d' Eljoénaï furent  sept : Hodavia, Eliaschib, Pelaja, Akkub, Jochanan, Delaja, et Anani.
\Chap{4}
\TextTitle{Les autres fils de Hur\FTNTT{1 Ch. 2:50}}
\VerseOne{}Les fils de Juda furent Pérets, Hetsron, Carmi, Hur et Schobal.
\VS{2}Reaja, fils de Schobal, engendra Jachath ; et Jachath engendra Achumaï et Lahad. Ce sont les familles des Tsoreathiens.
\VS{3}Voici les descendants du père d’Etham : Jizreel, Jischma, et Jidbasch ; le nom de leur soeur était Hatselelponi.
\VS{4}Penuel, père de Guedor, et Ezer, père de Huscha, sont les fils de Hur, premier-né d'Ephrata, père de Bethléhem.
\TextTitle{Les descendants d’Aschchur\FTNTT{1 Ch. 2:24}}
\VS{5}Aschchur, père de Tekoa, eut deux femmes : Hélea et Naara.
\VS{6}Naara lui enfanta Achuzzam, Hépher, Thémeni et Achaschthari. Ce sont là les fils de Naara.
\VS{7}Les fils de Hélea furent Tséreth, Tsochar et Ethnan.
\VS{8}Kots engendra Anub, Hatsobéba et les familles Acharchel, fils de Harum.
\TextTitle{Jaebets invoque Dieu}
\VS{9}Jaebets était plus honoré que ses frères ; sa mère lui avait donné le nom de Jaebets, parce que, dit-elle, je l'ai enfanté avec douleur.
\VS{10}Jaebets invoqua le Dieu d'Israël, en disant : Ô, si tu me bénis abondamment et que tu étends mes limites, si ta main est avec moi, et si tu me mets à l'abri du mal, en sorte que je ne sois pas dans l’affliction !... Et Dieu lui accorda ce qu'il avait demandé.
\TextTitle{Les fils de Juda et de Caleb}
\VS{11}Kelub, frère de Schucha, engendra Mechir, qui fut père d'Eschthon.
\VS{12}Et Eschthon engendra la maison de Rapha, Paséach et Thechinna, père de la ville de Nachasch ; ce sont là les gens de Réca.
\VS{13}Les fils de Kenaz furent Othniel et Seraja. Et le fils d’Othniel, Hathath.
\VS{14}Meonothaï engendra Ophra ; et Seraja engendra Joab, père de la vallée des ouvriers ; car ils étaient ouvriers.
\VS{15}Les fils de Caleb, fils de Jephunné, furent Iru, Ela et Naam, et les fils d'Ela, Kenaz.
\VS{16}Les fils de Jehalléleel furent Ziph, Zipha, Thirja, et Asareel.
\VS{17}Les fils d'Esdras furent Jéther, Méred, Epher, et Jalon ; et la femme de Méred enfanta Miriam, Schammaï, et Jischbach, père d' Eschthemoa.
\VS{18}Sa femme, la Juive, enfanta Jéred, père de Guedor ;  Héber, père de Soco ;  Jekuthiel, père de Zanoach. Ceux-là sont les fils de Bithja, fille de Pharaon, que Méred prit pour femme.
\VS{19}Les fils de la femme de Hodija, soeur de Nacham : Le père de Kehila, le Garmien, et Eschthemoa, le Maacathien.
\VS{20}Et les fils de Simon furent Amnon, Rinna, Ben-Hanan et Thilon. Les fils de Jischeï furent Zocheth et Ben-Zocheth.
\TextTitle{Les fils de Juda par Schéla\FTNTT{1 Ch. 2:3}}
\VS{21}Les fils de Schéla, fils de Juda, furent Er, père de Léca ; Laeda, père de Maréscha ; et les familles de la maison où l'on travaille le byssus, qui sont de la maison d'Aschbéa.
\VS{22}Jokim, et les gens de Cozéba, Joas et Saraph dominèrent sur Moab, avec Jaschubi-Léchem. Mais ce sont là des choses anciennes.
\VS{23}C’étaient les potiers et les habitants des plantations  et des parcs. Ils cohabitaient là chez le roi et oeuvraient pour lui.
\TextTitle{Les descendants de Siméon ; leurs terres et leurs conquêtes}
\VS{24}Les fils de Siméon furent Nemuel, Jamin, Jarib, Zérach et Saül.
\VS{25}Schallum son fils, Mibsam son fils, et Mischma son fils.
\VS{26}Les fils de Mischma furent Hammuel son fils, Zaccur son fils, et Schimeï son fils.
\VS{27}Schimeï eut seize fils et six filles ; mais ses frères n'eurent pas beaucoup de fils, et toute leur famille ne put être aussi nombreuse que celle des fils de Juda.
\VS{28}Ils habitèrent à Beer-Schéba, à Molada, à Hatsar-Schual,
\VS{29}à Bilha, à Etsem, à Tholad,
\VS{30}à Bethuel, à Horma, à Tsiklag,
\VS{31}à Beth-Marcaboth, à Hatsar-Susim, à Beth-Bireï, et à Schaaraïm. Ce furent là leurs villes jusqu'au temps où David devint roi.
\VS{32}Leurs villages furent Etham, Aïn, Rimmon, Thoken, et Aschan, cinq villes ;
\VS{33}et tous leurs villages, qui étaient autour de ces villes-là, jusqu'à Baal. Ce sont là leurs habitations et leur généalogie :
\VS{34}Meschobab, Jamlec, Joscha fils d'Amatsia ;
\VS{35}Joël, Jéhu fils de Joschibia, fils de Seraja, fils d'Asiel ;
\VS{36}Eljoénaï, Jaakoba, Jeschochaja, Asaja, Adiel, Jesimiel, Benaja,
\VS{37}Ziza, fils de Schipheï, fils d'Allon, fils de Jedaja, fils de Schimri, fils de Schemaeja.
\VS{38}Ceux-là furent désignés pour être des chefs dans leurs familles, et les maisons de leurs pères s’étendirent abondamment.
\VS{39}Et ils allèrent pour entrer dans Guedor, jusqu'à l'orient de la vallée, cherchant des pâturages pour leurs troupeaux.
\VS{40}Ils trouvèrent des pâturages gras et bons, et un pays spacieux, paisible et fertile ; car ceux qui habitaient là auparavant étaient descendus de Cham.
\VS{41}Ceux-ci, dont les noms sont inscrits, vinrent du temps d'Ezéchias, roi de Juda, et abattirent leurs tentes ; et quant aux Maonites qui s’y trouvaient, et les détruisirent à la façon de l'interdit jusqu'à ce jour, et y habitèrent à leur place, car il y avait là des pâturages pour leurs troupeaux.
\VS{42}Cinq cents hommes d'entre eux, c'est-à-dire des fils de Siméon, s'en allèrent à la montagne de Séir, et ils avaient à leur tête Pelathia, Nearia, Rephaja, et Uziel, fils de Jischeï ;
\VS{43}ils frappèrent le reste des réchappés d'Amalek, et ils demeurèrent là jusqu'à ce jour.
\Chap{5}
\TextTitle{Les descendants de Ruben jusqu’au temps des captivités}
\VerseOne{}Les fils de Ruben, le premier-né d'Israël. - Car il était le premier-né ; mais après qu'il eut souillé le lit de son père, son droit d'aînesse fut donné aux fils de Joseph fils d'Israël ; cependant, Joseph ne fut pas enregistré  dans la généalogie selon le droit d'aînesse.
\VS{2}Car Juda fut le plus puissant parmi ses frères, et de lui est issu un chef ; mais le droit d'aînesse est à Joseph.
\VS{3}Les fils de Ruben, premier-né d'Israël, furent donc Hénoc, Pallu, Hetsron, et Carmi.-
\VS{4}Les fils de Joël furent  Schemaeja, son fils ; Gog, son fils ; Schimeï, son fils ;
\VS{5}Michée, son fils ; Reaja, son fils ; Baal,  son fils ;
\VS{6}Beéra, son fils, qui fut emmené captif par Tilgath- Pilnéser, roi d’Assyrie ; c'est lui qui était le principal chef des Rubénites.
\VS{7}Ses frères, selon leurs familles, d’après le registre généalogique et selon leurs générations, avaient pour chefs Jeïel et Zacharie.
\VS{8}Béla, fils d’Azaz, fils de Schéma, fils de Joël, habitait depuis Aroër jusqu'à Nebo et Baal-Meon.
\VS{9}Ensuite, il habita du côté de l’orient jusqu'à l'entrée du désert, depuis le fleuve d'Euphrate; car son bétail s'était multiplié dans le pays de Galaad.
\VS{10}Du temps de Saül, ils firent la guerre contre les Hagaréniens, qui tombèrent par leurs mains, et ils habitèrent dans leurs tentes, dans toute la partie orientale de Galaad.
\TextTitle{Les descendants de Gad et leurs villes}
\VS{11}Les fils de Gad habitaient près d'eux, au pays de Basan, jusqu'à Salca.
\VS{12}Joël fut le premier chef, et Schapham le deuxième après lui, puis Jaenaï, puis Schaphath en Basan.
\VS{13}Et leurs frères, selon la maison de leurs pères, furent sept : Micaël, Meschullam, Schéba, Joraï, Jaecan, Zia, et Eber.
\VS{14}Ceux-ci furent les fils d'Abichaïl, fils de Huri, fils de Jaroach, fils de Galaad, fils de Micaël, fils de Jeschischaï, fils de Jachdo, fils de Buz.
\VS{15}Achi, fils d'Abdiel, fils de Guni, fut le chef de la maison de leurs pères.
\VS{16}Ils habitèrent en Galaad, et en Basan, dans les villes de son ressort, et dans toutes les banlieues de Saron, jusqu’à leurs limites.
\VS{17}Tous ceux-ci furent inscrits dans la généalogie du temps de Jotham, roi de Juda, et du temps de Jéroboam, roi d'Israël.
\TextTitle{Captivité de Ruben, Gad et la demi-tribu de Manassé}
\VS{18}Il y eut des fils de Ruben, et de ceux de Gad, et de la demi-tribu de Manassé, d'entre les vaillants hommes, portant le bouclier et l'épée, tirant de l'arc, et exercés à la guerre, quarante-quatre mille sept cent soixante, en état d’aller à l’armée.
\VS{19}Ils firent la guerre contre les Hagaréniens, contre Jethur, Naphisch, et Nodab.
\VS{20}Et ils reçurent du secours contre eux, de sorte que les Hagaréniens, et tous ceux qui étaient avec eux furent livrés entre leurs mains, parce qu'ils crièrent à Dieu dans la bataille, et il les exauça parce qu'ils avaient mis leur confiance en lui.
\VS{21}Ainsi ils prirent leurs troupeaux, consistant en cinquante mille chameaux, deux cent cinquante mille brebis, deux mille ânes, avec cent mille personnes ;
\VS{22}car il y eut beaucoup de morts, parce que la bataille venait de Dieu.  Ils habitèrent là, à leur place, jusqu'au temps de la déportation.
\VS{23}Les fils de la demi-tribu de Manassé habitèrent aussi dans ce pays-là, et s'étendirent depuis Basan jusqu'à Baal-Hermon et à Sénir, à la montagne d’Hermon ; ils étaient nombreux.
\VS{24}Et voici les chefs de la maison de leurs pères : Epher, Jischeï, Eliel, Azriel, Jérémie, Hodavia, et Jachdiel, hommes forts et vaillants, gens de réputation, et chefs des maisons de leurs pères.
\VS{25}Mais ils péchèrent contre le Dieu de leurs pères, et se prostituèrent après les dieux des peuples du pays, que Dieu avait détruits devant eux.
\VS{26}Le Dieu d'Israël excita l'esprit de Pul, roi d’Assyrie, et l'esprit de Thilgath-Pilnéser, roi d’Assyrie,  qui emmena en captivité les Rubénites, les Gadites et la demi-tribu de Manassé, et les emmena à Chalach, à Chabor, à Hara, et au fleuve de Gozan, où ils sont restés jusqu'à ce jour.
\Chap{6}
\TextTitle{Les fils de Kehath le Lévite, jusqu'à la captivité}
\VerseOne{}Les fils de Lévi furent Guerschon, Kehath et Merari.
\VS{2}Les fils de Kehath furent  Amram, Jitsehar, Hébron, et Uziel.
\VS{3}Et les fils d'Amram furent Aaron, Moïse et Marie. Les fils d'Aaron furent  Nadab, Abihu, Eléazar et Ithamar.
\VS{4}Eléazar engendra Phinées, et Phinées engendra Abischua.
\VS{5}Abischua engendra Bukki, et Bukki engendra Uzzi.
\VS{6}Uzzi engendra Zerachja, et Zerachja engendra Merajoth.
\VS{7}Merajoth engendra Amaria, et Amaria engendra Achithub.
\VS{8}Achithub engendra Tsadok, et Tsadok engendra Achimaats.
\VS{9}Achimaats engendra Azaria, et Azaria engendra Jochanan.
\VS{10}Jochanan engendra Azaria, qui exerça la sacrificature au temple que Salomon bâtit à Jérusalem.
\VS{11}Azaria engendra Amaria, et Amaria engendra Achithub.
\VS{12}Achithub engendra Tsadok, et Tsadok engendra Schallum.
\VS{13}Schallum engendra Hilkija, et Hilkija engendra Azaria.
\VS{14}Azaria engendra Seraja, et Seraja engendra Jehotsadak,
\VS{15}Jehotsadak s'en alla, quand Yahweh emmena en exil Juda et Jérusalem par le moyen de Nebucadnetsar.
\TextTitle{Les fils de Guerschon, Kehath et Mérari}
\VS{16}Les fils de Lévi  furent donc  Guerschon, Kehath et Merari.
\VS{17}Voici les noms des fils de Guerschon : Libni et Schimeï.
\VS{18}Les fils de Kehath furent Amram, Jitsehar, Hébron et Uziel.
\VS{19}Les fils de Merari furent  Machli et Muschi. Ce sont là les familles des Lévites, selon les maisons de leurs pères.
\VS{20}De Guerschon, Libni son fils, Jachath son fils, Zimma son fils,
\VS{21}Joach son fils, Iddo son fils, Zérach son fils, Jeathraï son fils.
\VS{22}Des fils de Kehath, Amminadab son fils, Koré son fils, Assir son fils,
\VS{23}Elkana son fils, Ebjasaph son fils, Assir son fils,
\VS{24}Thachath son fils, Uriel son fils, Ozias son fils, et Saül son fils.
\VS{25}Les fils d’Elkana furent  Amasaï, Achimoth ;
\VS{26}Elkana, son fils ; les fils d’Elkana furent Elkana-Tsophaï, son fils, Nachath son fils,
\VS{27}Eliab son fils, Jerocham son fils, Elkana son fils.
\VS{28}Quant aux fils de Samuel, fils d'Elkana, son fils aîné fut Vaschni, puis Abija.
\VS{29}Les fils de Merari furent Machli, Libni son fils, Schimeï son fils, Uzza son fils,
\VS{30}Schimea son fils, Hagguija son fils, Asaja son fils.
\TextTitle{Les chefs des chantres}
\VS{31}Or voici ceux que David établit pour la direction de la musique dans la maison de Yahweh, depuis que l’arche fut en lieu de repos.
\VS{32}Ils faisaient le service comme chantres devant le tabernacle, devant la tente d'assignation, jusqu'à ce que Salomon eût bâti la maison de Yahweh à Jérusalem ; ils continuèrent dans leur service selon l'ordonnance qui était prescrite. Voici ceux qui firent le service avec leurs fils : D'entre les fils des Kehathites, Héman le chantre, fils de Joël, fils de Samuel,
\VS{34}fils d'Elkana, fils de Jerocham, fils d’Eliel, fils de Thoach,
\VS{35}fils de Tsuph, fils d'Elkana, fils de Machath, fils de Amasaï,
\VS{36}fils d'Elkana, fils de Joël, fils d’Azaria, fils de Sophonie,
\VS{37}fils de Thachath, fils d'Assir, fils de Ebjasaph, fils de Koré,
\VS{38}fils de Jitsehar, fils de Kehath, fils de Lévi, fils d'Israël.
\VS{39}Son frère Asaph, qui se tenait à sa droite. Asaph était fils de Bérékia, fils de Schimea,
\VS{40}fils de Micaël, fils de Baaséja, fils de Malkija,
\VS{41}fils d’Ethni, fils de Zérach, fils d’Adaja,
\VS{42}fils d'Ethan, fils de Zimma, fils de Schimeï,
\VS{43}fils de Jachath, fils de Guerschon, fils de Lévi.
\VS{44}Les fils de Merari, leurs frères étaient à la gauche ; à savoir Ethan, fils de Kischi, fils d’Abdi, fils de Malluc,
\VS{45}fils de Haschabia, fils d'Amatsia, fils de Hilkija,
\VS{46}fils d'Amtsi, fils de Bani, fils de Schémer,
\VS{47}fils de Machli, fils de Muschi, fils de Merari, fils de Lévi.
\VS{48}Et leurs autres frères Lévites furent ordonnés pour tout le service du tabernacle de la maison de Dieu.
\VS{49}Mais Aaron et ses fils offraient les parfums sur l'autel de l'holocauste et sur l'autel des parfums ; pour tout ce qu'il fallait faire dans le Saint des saints, et pour faire propitiation pour Israël ; comme Moïse, serviteur de Dieu, l'avait commandé.
\TextTitle{Les sacrificateurs d’Aaron à Achimaats}
\VS{50}Voici les fils d'Aaron : Eléazar son fils, Phinées son fils, Abischua son fils,
\VS{51}Bukki son fils, Uzzi son fils, Zerachja son fils,
\VS{52}Merajoth son fils, Amaria son fils, Achithub son fils,
\VS{53}Tsadok son fils, Achimaats son fils.
\TextTitle{Villes des fils d’Aaron et des Lévites}
\VS{54}Voici leurs lieux d’habitation, selon leurs demeures et leurs limites. Aux fils d'Aaron, qui appartiennent à la famille des Kehathites, désignés par le sort,
\VS{55}on leur donna Hébron dans le pays de Juda, et sa banlieue tout autour.
\VS{56}Mais on donna à Caleb, fils de Jephunné, le territoire de la ville et ses villages.
\VS{57}On donna donc aux fils d'Aaron, d'entre les villes de refuge, Hébron, Libna et sa banlieue, Jatthir et Eschthemoa, avec leurs banlieues,
\VS{58}Hilen, avec sa banlieue, Debir avec sa banlieue,
\VS{59}Aschan avec sa banlieue, et Beth-Schémesch avec sa banlieue.
\VS{60}De la tribu de Benjamin, Guéba, avec sa banlieue, Allémeth avec sa banlieue, et Anathoth avec sa banlieue. Toutes leurs villes, selon leurs familles, étaient treize en nombre.
\VS{61}On donna au reste des fils de Kehath, par le sort, dix villes des familles de la demi-tribu, c'est-à-dire de la demi-tribu de Manassé.
\VS{62}Et aux fils de Guerschon, selon leurs familles, de la tribu d'Issacar, de la tribu d'Aser, de la tribu de Nephthali, et de la tribu de Manassé en Basan, treize villes.
\VS{63}Aux fils de Merari, selon leurs familles, par le sort, douze villes, de la tribu de Ruben, de la tribu de Gad, et de la tribu de Zabulon.
\VS{64}Ainsi, les fils d’Israël donnèrent aux Lévites ces villes-là, avec leurs banlieues.
\VS{65}Et ils donnèrent, par le sort, de la tribu des fils de Juda, de la tribu des fils de Siméon, et de la tribu des fils de Benjamin, ces villes qu’ils désignèrent par leurs noms.
\VS{66}Et pour les autres familles des fils de Kehath, ils eurent pour territoire des villes de la tribu d'Ephraïm.
\VS{67}Car on leur donna entre les villes de refuge, Sichem avec sa banlieue, dans la montagne d'Ephraïm, Guézer avec sa banlieue,
\VS{68}Jokmeam avec sa banlieue, Beth-Horon avec sa banlieue,
\VS{69}Ajalon avec sa banlieue, et Gath-Rimmon avec sa banlieue.
\VS{70}De la demi-tribu de Manassé, Aner avec sa banlieue, et Bileam avec sa banlieue, on donna ces villes-là aux familles qui restaient des fils de Kehath.
\VS{71}Aux fils de Guerschon, on donna, des familles de la demi-tribu de Manassé, Golan en Basan avec sa banlieue, et Aschtaroth, avec sa banlieue.
\VS{72}De la tribu d'Issacar, Kédesch avec sa banlieue, Dobrath avec sa banlieue,
\VS{73}Ramoth avec sa banlieue, et Anem avec sa banlieue.
\VS{74}Et de la tribu d'Aser, Maschal, avec sa banlieue, Abdon, avec sa banlieue,
\VS{75}Hukok avec sa banlieue, et Rehob avec sa banlieue.
\VS{76}De la tribu de Nephthali, Kédesch en Galilée avec sa banlieue, Hammon avec sa banlieue, et Kirjathaïm avec sa banlieue.
\VS{77}Aux fils de Merari, qui étaient le reste d'entre les Lévites, on donna, de la tribu de Zabulon, Rimmono avec sa banlieue, et Thabor avec sa banlieue.
\VS{78}Au-delà du Jourdain, vis-à-vis de Jéricho, vers l’orient du Jourdain, de la tribu de Ruben, Betser au désert avec sa banlieue, Jahtsa avec sa banlieue,
\VS{79}Kedémoth avec sa banlieue, et Méphaath avec sa banlieue.
\VS{80}De la tribu de Gad, Ramoth en Galaad avec sa banlieue, Mahanaïm avec sa banlieue,
\VS{81}Hesbon avec sa banlieue, et Jaezer avec sa banlieue.
\Chap{7}
\TextTitle{Les descendants d'Issacar}
\VerseOne{}Les fils d'Issacar furent  Thola, Pua, Jaschub et Schimron, quatre.
\VS{2}Les fils de Thola furent Uzzi, Rephaja, Jeriel, Jachmaï, Jibsam et Samuel, chefs des maisons de leurs pères qui étaient de Thola, gens forts et vaillants dans leurs générations ; leur nombre, aux jours de David, était de vingt-deux mille six cents.
\VS{3}Le fils d’Uzzi : Jizrachja. Et les fils de Jizrachja : Micaël, Abdias, Joël, et Jischija, en tout cinq chefs.
\VS{4}Ils avaient avec eux, selon leurs générations, et selon les familles de leurs pères, trente-six mille hommes de troupe, armés pour la guerre, car ils eurent plusieurs femmes et plusieurs fils.
\VS{5}Leurs frères selon toutes les familles d'Issacar, hommes forts et vaillants, étant comptés tous selon leur généalogie, furent quatre-vingt-sept mille.
\TextTitle{Les descendants de Benjamin}
\VS{6}Les fils de Benjamin furent Béla, Béker et Jediaël, trois\FTNT{Benjamin avait encore d’autres fils (Ge. 46:21 ; No. 26:38-41 ; 1 Ch. 8:1-2).}.
\VS{7}Les fils de Béla furent  Etsbon, Uzzi, Uziel, Jerimoth et Iri, cinq chefs des familles de leurs pères, hommes forts et vaillants, et enregistrés dans la généalogie au nombre de vingt-deux mille trente-quatre.
\VS{8}Les fils de Béker furent  Zemira, Joasch, Eliézer, Eljoénaï, Omri, Jerémoth, Abija, Anathoth, et Alameth, tous ceux-là furent fils de Béker,
\VS{9}et enregistrés dans les généalogies, selon leurs générations, comme chefs des familles de leurs pères, hommes forts et vaillants au nombre de vingt mille deux cents.
\VS{10}Jediaël eut pour fils Bilhan. Et les fils de Bilhan furent Jeusch, Benjamin, Ehud, Kenaana, Zéthan, Tarsis, et Achischachar.
\VS{11}Tous ceux-là furent fils de Jediaël, comme chefs des familles de leurs pères, dix-sept mille deux cents hommes forts et vaillants, en état de porter les armes et d’aller à la guerre.
\VS{12}Schuppim et Huppim furent  des fils d’Ir ; et Huschim fut fils d'Acher.
\TextTitle{Les descendants de Nephtali}
\VS{13}Les fils de Nephthali furent Jahtsiel, Guni, Jetser, et Schallum, fils de Bilha.
\TextTitle{Les descendants de Manassé}
\VS{14}Les fils de Manassé : Asriel, qu’enfanta sa concubine Araméenne. Elle enfanta Makir, père de Galaad.
\VS{15}Makir prit une femme de la parenté de Huppim et de Schuppim ; car ils avaient une sœur dont le nom était Maaca. Et le nom d'un des petits-fils de Galaad fut Tselophchad ; et Tselophchad eut des filles.
\VS{16}Maaca, femme de Makir, enfanta un fils et l'appela Péresch, et le nom de son frère Schéresch, dont les fils furent Ulam et Rékem.
\VS{17}Le fils d'Ulam fut Bedan. Ce sont là les fils de Galaad, fils de Makir, fils de Manassé.
\VS{18}Mais sa soeur Hammoléketh enfanta Ischhod, Abiézer et Machla.
\VS{19}Les fils de Schemida furent Achjan, Sichem, Likchi et Aniam.
\TextTitle{Les descendants d'Ephraïm et leurs villes}
\VS{20}Or les fils d'Ephraïm furent  Schutélach ; Béred son fils, Tachath son fils, Eleada son fils, Tachath son fils.
\VS{21}Zabad son fils, Schutélach son fils, Ezer, et Elead. Mais ceux de Gath, nés dans le pays, les mirent à mort, parce qu'ils étaient descendus pour prendre leur bétail.
\VS{22}Ephraïm, leur père, fut dans le deuil plusieurs jours, et ses frères vinrent pour le consoler.
\VS{23}Puis il alla vers sa femme, qui conçut et enfanta un fils ; et elle l'appela du nom de Beria, parce que le malheur était dans sa maison.
\VS{24}Il eut pour fille Schééra, qui bâtit la basse et la haute Beth-Horon, et Uzzen-Schééra.
\VS{25}Son fils fut  Réphach, puis Réscheph, et Thélach son fils, Thachan son fils,
\VS{26}Laedan son fils, Ammihud son fils, Elischama son fils,
\VS{27}Nun son fils, Josué son fils.
\VS{28}Ils possédaient et habitaient Béthel ainsi que les villes de son ressort ; à l’orient Naaran, à l’occident Guézer, avec les villes de son ressort, et Sichem avec les villes de son ressort, jusqu'à Gaza avec les villes de son ressort.
\VS{29}Les lieux qui étaient aux fils de Manassé furent  Beth-Schean avec les villes de son ressort, Thaanac avec les villes de son ressort, Meguiddo avec les villes de son ressort, et Dor avec les villes de son ressort. Les fils de Joseph, fils d'Israël, habitèrent dans ces villes.
\TextTitle{Les descendants d'Aser}
\VS{30}Les fils d’Aser furent Jimna, Jischva, Jischvi, Beria, et Sérach leur soeur.
\VS{31}Les fils de Beria furent Héber et Malkiel, qui fut père de Birzavith.
\VS{32}Héber engendra Japhleth, Schomer, Hotham, et Schua leur soeur.
\VS{33}Les fils de Japhleth furent Pasac, Bimhal, et Aschvath. Ce sont là les fils de Japhlet.
\VS{34}Et les fils de Schamer furent Achi, Rohega, Hubba et Aram.
\VS{35}Les fils d'Hélem, son frère, furent  Tsophach, Jimna, Schélesch et Amal.
\VS{36}Les fils de Tsophach furent Suach, Harnépher, Schual, Béri, Jimra,
\VS{37}Betser, Hod, Schamma, Schilscha, Jithran, et Beéra.
\VS{38}Les fils de Jéther furent Jephunné, Pispa et Ara.
\VS{39}Les fils d'Ulla furent Arach, Hanniel et Ritsja.
\VS{40}Tous ceux-là furent fils d'Aser, chefs des maisons de leurs pères, gens d'élite, forts et vaillants, chefs des princes, enregistrés au nombre de vingt-six mille hommes, en état de porter les armes et d’aller en guerre.
\Chap{8}
\TextTitle{Les descendants de Benjamin}
\VerseOne{}Benjamin engendra Béla, qui fut son premier-né, Aschbel le deuxième, Achrach le troisième,
\VS{2}Nocha le quatrième, et Rapha le cinquième.
\VS{3}Les fils de Béla furent Addar, Guéra, Abihud,
\VS{4}Abischua, Naaman, Achoach,
\VS{5}Guéra, Schephuphan et Huram.
\VS{6}Voici les fils d'Echud, qui étaient chefs des maisons des pères des habitants de Guéba, et qui les transportèrent à Manachath :
\VS{7}Naaman, Achija, et Guéra. Guéra, qui les transporta et qui après engendra Uzza et Achichud.
\VS{8}Or Schacharaïm eut des enfants au pays de Moab, après avoir renvoyé Huschim et Baara, ses femmes.
\VS{9}Il engendra, de Hodesch sa femme, Jobab, Tsibja, Méscha, Malcam,
\VS{10}Jeuts, Schocja et Mirma. Ce sont là ses fils, chefs des pères.
\VS{11}Mais de Huschim, il engendra Abithub, Elpaal.
\VS{12}Les fils d'Elpaal furent Eber, Mischeam, et Schémer, qui bâtit Ono, Lod et les villes de son ressort.
\VS{13}Et Beria et Schéma furent chefs des pères des habitants d'Ajalon ; ils mirent en fuite les habitants de Gath.
\VS{14}Achjo, Schaschak, Jerémoth,
\VS{15}Zebadja, Arad, Eder,
\VS{16}Micaël, Jischpha, et Jocha, fils de Beria.
\VS{17}Zebadja, Meschullam, Hizki, Héber,
\VS{18}Jischmeraï, Jizlia, et Jobab, fils d' Elpaal.
\VS{19}Jakim, Zicri, Zabdi,
\VS{20}Eliénaï, Tsilthaï, Eliel,
\VS{21}Adaja, Beraja, et Schimrath, fils de Schimeï.
\VS{22}Jischpan, Eber, Eliel,
\VS{23}Abdon, Zicri, Hanan,
\VS{24}Hanania, Elam, Anthothija,
\VS{25}Jiphdeja et Penuel, fils de Schaschak.
\VS{26}Schamscheraï, Schecharia, Athalia,
\VS{27}Jaaréschia, Elija, et Zicri, fils de Jerocham.
\VS{28}Ce sont là les chefs des pères, selon leurs générations ; et ils habitèrent à Jérusalem.
\TextTitle{Les fils du père de Gabaon, ascendant de Saül}
\VS{29}Le père de Gabaon habita à Gabaon, sa femme avait pour nom Maaca.
\VS{30}Son fils premier-né fut Abdon, puis Tsur, Kis, Baal, Nadab,
\VS{31}Guedor, Achjo, et Zéker.
\VS{32}Mikloth engendra Schimea. Ils habitèrent aussi vis-à-vis de leurs frères à Jérusalem, avec leurs frères.
\VS{33}Ner engendra Kis, et Kis engendra Saül, et Saül engendra Jonathan, Malki-Schua, Abinadab, et Eschbaal.
\VS{34}Le fils de Jonathan fut  Merib-Baal ; et Merib-Baal engendra Michée.
\VS{35}Les fils de Michée furent Pithon, Mélec, Thaeréa, et Achaz.
\VS{36}Achaz engendra Jehoadda ; et Jehoadda engendra Alémeth, Azmaveth et Zimri ; Zimri engendra Motsa.
\VS{37}Motsa engendra Binea, qui eut pour fils Rapha, qui eut pour fils Eleasa, qui eut pour fils Atsel.
\VS{38}Atsel eut six fils, dont les noms sont : Azrikam, Bocru, Ismaël, Schearia, Abdias, et Hanan ; tous ceux-là furent fils d'Atsel.
\VS{39}Les fils d'Eschek, son frère, furent  Ulam son premier-né, Jéusch le second, Eliphéleth le troisième.
\VS{40}Et les fils d'Ulam furent des hommes forts et vaillants, tirant bien de l'arc, et ils eurent beaucoup de fils et de petits-fils, jusqu'à cent cinquante ; tous des fils de Benjamin.
\Chap{9}
\TextTitle{Les habitants de Jérusalem}
\VerseOne{}Ainsi, tous ceux d'Israël furent enregistrés par généalogie et inscrits dans le livre des rois d'Israël. Et ceux de Juda furent emmenés en captivité à Babylone à cause de leurs péchés\FTNT{La captivité babylonienne voir 2 R. 24 et 25.}.
\VS{2}Mais ce sont ici les premiers qui habitèrent dans leurs possessions, et dans leurs villes, tant d'Israël que des sacrificateurs, des Lévites, et des Néthiniens.
\VS{3}A Jérusalem habitaient les fils de Juda, les fils de Benjamin, et les fils d'Ephraïm et de Manassé.
\VS{4}Uthaï, fils d'Ammihud, fils d'Omri, fils d'Imri, fils de Bani, des fils de Pérets, fils de Juda.
\VS{5}Des Schilonites, Asaja le premier-né, et ses fils.
\VS{6}Des fils de Zérach, Jeuel, et ses frères, six cent quatre-vingt-dix.
\VS{7}Des fils de Benjamin, Sallu fils de Meschullam, fils de Hodavia, fils d'Assenua.
\VS{8}Jibneja, fils de Jerocham, et Ela fils d’Uzzi, fils de Micri ; et Meschullam fils de Schephathia, fils de Reuel, fils de Jibnija.
\VS{9}Leurs frères, selon leurs générations, furent neuf cent cinquante-six. Tous ces hommes-là furent chefs des pères dans les maisons de leurs  pères.
\VS{10}Des sacrificateurs : Jedaeja, Jehojarib, et Jakin.
\VS{11}Azaria fils de Hilkija, fils de Meschullam, fils de Tsadok, fils de Merajoth, fils d' Achithub, intendant de la maison de Dieu.
\VS{12}Adaja, fils de Jerocham, fils de Paschhur, fils de Malkija; et Maesaï, fils d'Adiel, fils de Jachzéra, fils de Meschullam, fils de Meschillémith, fils d'Immer.
\VS{13}Leurs frères, chefs de la maison de leurs pères, mille sept cent soixante hommes, forts et vaillants, occupés au service de la maison de Dieu.
\VS{14}Des Lévites : Schemaeja, fils de Haschub, fils d'Azrikam, fils de Haschabia, des fils de Merari,
\VS{15}Bakbakkar, Héresch, et Galal ; et Matthania, fils de Michée, fils de Zicri, fils d'Asaph,
\VS{16}Abdias fils de Schemaeja, fils de Galal, fils de Jeduthun ; et Bérékia, fils d'Asa, fils d'Elkana, qui habita dans les villages des Nethophathiens.
\VS{17}Et les portiers : Schallum, Akkub, Thalmon, et Achiman, et leurs frères ; mais Schallum était le chef.
\VS{18}Il l'a été jusqu'à maintenant, ayant la charge de la porte du roi vers l’orient. Ceux-là furent portiers pour le camp des fils de Lévi.
\VS{19}Schallum, fils de Koré, fils d'Ebiasaph, fils de Koré, et ses frères Koréites, de la maison de son père, remplissaient les fonctions de gardiens, gardant les seuils de la tente, comme leurs pères en avaient gardé l'entrée au camp de Yahweh ;
\VS{20}Phinées, fils d'Eléazar, fut établi chef sur eux en présence de Yahweh qui était avec lui.
\VS{21}Zacharie, fils de Meschélémia, était le portier de l'entrée de la tente d'assignation.
\VS{22}Ils étaient en tout deux cent douze, choisis pour être les portiers des seuils, et enregistrés selon les familles dans la généalogie, selon leurs villages ; David et Samuel, le voyant, les avaient établis dans leurs fonctions.
\VS{23}Eux, dis-je, et leurs fils furent établis sur les portes de la maison de Yahweh, qui est la maison du tabernacle, pour y faire la garde.
\VS{24}Il y avait des portiers aux quatre vents, à l'orient, à l'occident, au nord et au midi.
\VS{25}Et leurs frères, qui étaient dans leurs villages, devaient de temps à autre venir auprès d’eux pendant sept jours.
\VS{26}Car selon cette fonction, il y avait toujours quatre chefs des portiers, des Lévites, qui avaient la surveillance des chambres et des trésors de la maison de Dieu.
\VS{27}Ils se tenaient la nuit tout autour de la maison de Dieu, dont ils avaient la garde, et qu’ils devaient ouvrir tous les matins.
\VS{28}Certains d’entre eux prenaient soin des ustensiles du service; car on en faisait le compte lorsqu'on les rentrait et qu'on les sortait.
\VS{29}D’autres veillaient sur les ustensiles, sur tous les ustensiles du sanctuaire, sur la fleur de farine, sur le vin, sur l'huile, sur l'encens et sur les aromates.
\VS{30}Mais ceux qui composaient les parfums aromatiques étaient des fils de sacrificateurs.
\VS{31}Matthithia, d'entre les Lévites, premier-né de Schallum, Koréite, s’occupait des gâteaux cuits sur les plaques.
\VS{32}Et quelques-uns de leurs frères, parmi les fils des Kehathites, avaient la charge du pain de proposition\FTNT{Il y avait douze gâteaux de pain qu’on plaçait sur une table dans le tabernacle ou dans le temple et qu’on remplaçait chaque sabbat (Ex. 35:13 ;  Ex. 39:36 ; 1 R. 7:48 ; 2 Ch. 13:11 ; Né. 10:32-33). En hébreu, le pain de proposition signifie littéralement le « pain de la face ». Le mot rendu par « face » se rapporte parfois à la « présence » (2 R. 13:23). Le pain de proposition est en réalité l’image du Seigneur Jésus-Christ, notre pain de vie (Jn. 6:48-59).}  pour l'apprêter chaque sabbat.
\VS{33}Certains étaient des chantres, chefs des pères des Lévites, qui demeuraient dans les chambres, sans avoir d’autres charges, parce qu'ils devaient être en fonction le jour et la nuit.
\VS{34}Ce sont là les chefs des pères des Lévites, selon leurs familles ; ils furent chefs, et ils habitèrent à Jérusalem.
\TextTitle{De Jeïel au roi Saül, de Jonathan à Arsel\FTNTT{1 Ch. 10 ; 1 S. 1 ; 30}}
\VS{35}Or Jeïel, le père de Gabaon, habita à Gabaon ; et le nom de sa femme était Maaca.
\VS{36}Son fils premier-né, Abdon, puis Tsur, Kis, Baal, Ner, Nadab,
\VS{37}Guedor, Achjo, Zacharie, et Mikloth.
\VS{38}Mikloth engendra Schimeam ; et ils habitèrent vis-à-vis de leurs frères à Jérusalem, avec leurs frères.
\VS{39}Ner engendra Kis, et Kis engendra Saül, et Saül engendra Jonathan, Malki-Schua, Abinadab et Eschbaal.
\VS{40}Le fils de Jonathan fut  Merib-Baal; et Merib-Baal engendra Michée.
\VS{41}Et les fils de Michée furent Pithon, Mélec, Thachréa et Achaz.
\VS{42}Achaz engendra Jaera ; et Jaera engendra Alémeth, Azmaveth, et Zimri ; et Zimri engendra Motsa.
\VS{43}Motsa engendra Binea, qui eut pour fils Rephaja, qui eut pour fils Eleasa, qui eut pour fils Atsel.
\VS{44}Atsel eut six fils, dont les noms sont Azrikam, Bocru, Ismaël, Schearia, Abdias et Hanan. Ce furent là les fils d'Atsel.
\Chap{10}
\TextTitle{Mort de Saül\FTNTT{1 S. 31:1-10 ; 2 S. 1}}
\VerseOne{}Les Philistins combattirent contre Israël, et les hommes d'Israël s'enfuirent devant les Philistins, et tombèrent blessés à mort sur la montagne de Guilboa\FTNT{1 S. 31:1-10}.
\VS{2}Les Philistins poursuivirent et atteignirent Saül et ses fils, et tuèrent Jonathan, Abinadab et Malki-Schua, les fils de Saül.
\VS{3}L’effort du combat se porta sur Saül ; de sorte que les archers l'atteignirent, et il eut peur de ces archers.
\VS{4}Alors Saül dit à celui qui portait ses armes : Tire ton épée, et transperce-moi, de peur que ces incirconcis ne viennent et ne fassent de moi selon leur volonté ; mais celui qui portait ses armes ne voulut pas, parce qu’il avait très peur. Saül prit donc son épée, et se jeta dessus.
\VS{5}Alors celui qui portait les armes de Saül, ayant vu que Saül était mort, se jeta aussi sur son épée, et il mourut.
\VS{6}Ainsi mourut Saül, et ses trois fils, et toute sa maison périt avec lui.
\VS{7}Tous ceux d'Israël, qui étaient dans la vallée, ayant vu qu'on avait fui, et que Saül et ses fils étaient morts, abandonnèrent leurs villes et s'enfuirent, de sorte que les Philistins y entrèrent et y habitèrent.
\VS{8}Or il arriva que dès le lendemain, les Philistins vinrent pour dépouiller les morts, et ils trouvèrent Saül et ses fils étendus sur la montagne de Guilboa.
\VS{9}Ils le dépouillèrent et emportèrent sa tête et ses armes. Puis ils firent annoncer ces bonnes nouvelles par tout le pays des Philistins, et aux environs, pour en faire savoir les nouvelles à leurs dieux et au peuple.
\VS{10}Ils mirent ses armes dans la maison de leur dieu, et ils attachèrent sa tête dans la maison de Dagon\FTNT{1 S. 5:1-11.}.
\VS{11}Tous ceux de Jabès de Galaad, ayant appris tout ce que les Philistins avaient fait à Saül,
\VS{12}tous les vaillants hommes d’entre eux se levèrent et enlevèrent le corps de Saül et les corps de ses fils ; ils les apportèrent à Jabès, et ils ensevelirent leurs os sous un chêne à Jabès, et jeûnèrent pendant sept jours.
\VS{13}Saül mourut pour le crime qu'il avait commis contre Yahweh, en ce qu'il n'avait point gardé la parole de Yahweh, et qu'il avait même consulté ceux qui évoquent les morts\FTNT{1 S. 28:7-20.} pour savoir ce qui devait lui arriver.
\VS{14}Il ne consulta point Yahweh ; c'est pourquoi Yahweh le fit mourir, et transféra la royauté à David, fils d'Isaï.
\Chap{11}
\TextTitle{David règne sur Israël\FTNTT{2 S. 5:1-3 ; 2 S. 2-4}}
\VerseOne{}Tous ceux d'Israël s'assemblèrent auprès de David à Hébron, et lui dirent : Voici, nous sommes tes os et ta chair.
\VS{2}Autrefois déjà, quand Saül était roi, tu étais celui qui faisais sortir et qui ramenais Israël. Yahweh, ton Dieu, t'a dit : Tu paîtras mon peuple d'Israël, et tu seras le chef de mon peuple d'Israël.
\VS{3}Ainsi, tous les anciens d'Israël vinrent auprès du roi à Hébron ; et David traita alliance avec eux à Hébron, devant Yahweh. Ils oignirent David pour roi sur Israël, selon la parole de Yahweh, prononcée par Samuel\FTNT{2 S. 2, 3, 4 ; 2 S. 5:1-3.}.
\TextTitle{Jérusalem devient la cité de David\FTNTT{2 S. 5:6-10}}
\VS{4}David et tous ceux d'Israël s'en allèrent à Jérusalem, qui est Jébus. Là étaient  les Jébusiens qui habitaient le pays.
\VS{5}Ceux qui habitaient à Jébus dirent à David : Tu n'entreras point ici. Mais David prit la forteresse de Sion, qui est la cité de David.
\VS{6}Car David avait dit: Quiconque battra le premier les Jébusiens sera chef et prince. Joab, fils de Tseruja, monta le premier, et fut fait chef.
\VS{7}David s’établit dans la forteresse ; c'est pourquoi on l'appela la cité de David\FTNT{2 S. 5:6-10.}.
\VS{8}Il bâtit aussi la ville tout autour, depuis Millo et ses environs ; et Joab répara le reste de la ville.
\VS{9}David devenait de plus en plus grand, car Yahweh des armées était avec lui.
\TextTitle{Les vaillants hommes de David\FTNTT{2 S. 23:8-39}}
\VS{10}Voici les chefs des hommes vaillants qui étaient au service de David, qui l’aidèrent avec tout Israël à assurer sa royauté, afin de le faire régner selon la parole de Yahweh au sujet d’Israël.
\VS{11}Ceux-ci sont du nombre des vaillants hommes que David avait. Jaschobeam, fils de Hacmoni, chef entre les trois principaux. Il brandit sa lance contre trois cents hommes et les blessa à mort en une seule fois\FTNT{2 S. 23:8-39. }.
\VS{12}Après lui était Eléazar, fils de Dodo, l'Achochite, qui fut l’un des trois vaillants hommes.
\VS{13}Il se trouvait avec David à Pas-Dammim, lorsque les Philistins s'étaient assemblés pour combattre. Il y avait là une parcelle de terre remplie d'orge ; et le peuple fuyait devant les Philistins.
\VS{14}Ils s'arrêtèrent au milieu de cette parcelle de champ, la défendirent, et battirent les Philistins. Ainsi, Yahweh accorda une grande délivrance.
\VS{15}Il en descendit encore trois des trente chefs près du rocher, auprès de David, dans la caverne d'Adullam, lorsque l'armée des Philistins campait dans la vallée des Rephaïm.
\VS{16}David était alors dans la forteresse, et la garnison des Philistins était en ce même temps-là à Bethléhem.
\VS{17}David eut un désir, et dit : Qui est-ce qui me fera boire de l'eau du puits qui est à la porte de Bethléhem ?
\VS{18}Alors ces trois hommes passèrent au travers du camp des Philistins, et puisèrent de l'eau du puits qui était à la porte de Bethléhem ; et l'ayant apportée, la présentèrent à David, qui ne voulut point la boire, mais la répandit en l'honneur de Yahweh.
\VS{19}Car il dit : Que mon Dieu me garde de faire une telle chose ! Boirais-je le sang de ces hommes qui ont fait un tel voyage au péril de leur vie ? Car ils m'ont apporté cette eau au péril de leur vie. Ainsi, il ne voulut point la boire. Voilà ce que firent ces trois vaillants hommes.
\VS{20}Abischaï, frère de Joab, était chef des trois. Il sortit sa lance sur trois cents hommes, les blessa à mort ; et il eut du renom entre les trois.
\VS{21}Entre les trois, il fut plus honoré que les deux autres, et il fut leur chef ; cependant, il n'égala point ces trois premiers.
\VS{22}Benaja aussi, fils de Jehojada, fils d'un vaillant homme de Kabtseel, avait fait de grands exploits. Il tua deux des plus puissants hommes de Moab. Il descendit et frappa un lion au milieu d'une fosse en un jour de neige.
\VS{23}Il tua aussi un homme Egyptien qui était haut de cinq coudées. Cet Egyptien avait à la main une lance grosse comme une ensouple de tisserand ; mais il descendit contre lui avec un bâton, et arracha la lance de la main de l'Egyptien, et le tua avec sa propre lance.
\VS{24}Benaja, fils de Jehojada, fit ces choses-là, et fut célèbre entre ces trois vaillants hommes.
\VS{25}Voilà, il était le plus honoré des trente ; cependant, il n'égala point les trois premiers. David l'établit dans son conseil privé.
\VS{26}Et les plus vaillants d'entre les gens de guerre furent Asaël, frère de Joab ; et Elchanan fils de Dodo, de Bethléhem,
\VS{27}Schammoth d'Haror, Hélets de Palon,
\VS{28}Ira, fils d'Ikkesch, de Tekoa, Abiézer d'Anathoth,
\VS{29}Sibbecaï le Huschatite, Ilaï d'Achoach,
\VS{30}Maharaï de Nethopha, Héled fils de Baana de Nethopha,
\VS{31}Ithaï fils de Ribaï, de Guibea des fils de Benjamin, Benaja de Pirathon,
\VS{32}Huraï de Nachalé-Gaasch, Abiel d'Araba,
\VS{33}Azmaveth de Bacharum, Eliachba de Schaalbon,
\VS{34}Bené-Haschem de Guizon, Jonathan fils de Schagué d'Harar,
\VS{35}Achiam fils de Sacar d'Harar, Eliphal fils d'Ur,
\VS{36}Hépher de Mekéra, Achija de Palon,
\VS{37}Hetsro de Carmel, Naaraï fils d'Ezbaï,
\VS{38}Joël frère de Nathan, Mibchar fils d'Hagri,
\VS{39}Tsélek l'Ammonite, Nachraï de Béroth, qui portait les armes de Joab fils de Tseruja,
\VS{40}Ira de Jéther, Gareb de Jéther,
\VS{41}Urie le Héthien, Zabad fils d' Achlaï,
\VS{42}Adina fils de Schiza le Rubénite, chef des Rubénites, et trente avec lui.
\VS{43}Hanan fils de Maaca, et Josaphat de Mithni,
\VS{44}Ozias d'Aschtharoth, Schama et Jehiel fils de Hotham d'Aroër,
\VS{45}Jediaël fils de Schimri, et Jocha son frère, le Thitsite,
\VS{46}Eliel de Machavim, Jeribaï, et Joschavia fils d'Elnaam, et Jithma le Moabite,
\VS{47}Eliel, et Obed, et Jaasie-Metsobaja.
\Chap{12}
\TextTitle{Les guerriers venus chez David à Tsiklag\FTNTT{2 S. 5:17 ; 1 Ch. 12:8-15 ; 1 Ch. 14:8}}
\VerseOne{}Voici ceux qui allèrent trouver David à Tsiklag, lorsqu'il était encore éloigné de la présence de Saül, fils de Kis. Ils étaient parmi les vaillants hommes qui lui prêtèrent leur secours pendant la guerre.
\VS{2}Ils étaient équipés d'arcs, se servant de la main droite et de la gauche pour jeter des pierres, et pour tirer des flèches avec l'arc. Ils étaient frères de Saül, de Benjamin,
\VS{3}Achiézer, le chef, et Joas, fils de Schemaa, qui était de Guibea, Jeziel, Péleth, fils d'Azmaveth, Beraca et Jéhu d'Anathoth ;
\VS{4}Jischmaeja de Gabaon, vaillant entre les trente, et même au-dessus des trente, et Jérémie, Jachaziel, Jochanan et Jozabad de Guedéra ;
\VS{5}Eluzaï, Jerimoth, Bealia, Schemaria et Schephathia de Haroph ;
\VS{6}Elkana, Jischija, Azareel, Joézer et Jaschobeam Koréites ;
\VS{7}Joéla et Zebadia, fils de Jerocham de Guedor.
\TextTitle{Les guerriers venus chez David dans la forteresse de Moab\FTNTT{1 S. 22:2-4}}
\VS{8}Quelques-uns aussi des Gadites se retirèrent auprès de David, dans la forteresse, au désert, hommes forts et vaillants, experts à la guerre et maniant le bouclier et la lance. Leurs visages étaient comme des faces de lion, et aussi prompts que des gazelles sur les montagnes.
\VS{9}Ezer le premier, Abdias le second, Eliab le troisième ;
\VS{10}Mischmanna le quatrième, Jérémie le cinquième ;
\VS{11}Attaï le sixième ; Eliel le septième ;
\VS{12}Jochanan le huitième, Elzabad le neuvième ;
\VS{13}Jérémie le dixième, Macbannaï le onzième.
\VS{14}C’étaient des fils de Gath, qui furent chefs de l'armée ; le plus petit avait la charge de cent hommes, et le plus grand de mille.
\VS{15}Ce sont ceux qui passèrent le Jourdain au premier mois, quand il déborde sur tous ses rivages ; et ils chassèrent ceux qui demeuraient dans les vallées, vers l'orient et l'occident.
\VS{16}Il vint aussi des fils de Benjamin et de Juda vers David à la forteresse.
\VS{17}David sortit au-devant d'eux, et prenant la parole, il leur dit : Si vous êtes venus en paix vers moi pour m'aider, mon cœur s’unira à vous ; mais si c'est pour me trahir et me livrer à mes ennemis, quoique je ne sois coupable d'aucune violence, que le Dieu de nos pères le voie, et qu'il fasse justice !
\VS{18}Alors Amasaï, l’un des principaux officiers, fut revêtu de l’Esprit, et dit : Que la paix soit avec toi, ô David ! Qu'elle soit avec toi, fils d'Isaï ! Que la paix soit à ceux qui t'aident, puisque ton Dieu t'aide ! Et David les reçut, et les établit parmi les chefs de ses troupes.
\VS{19}Des hommes de Manassé se joignirent à David, lorsqu’il alla combattre Saül avec les Philistins. Mais David et ses gens ne les aidèrent pas, parce que les princes des Philistins, après en avoir délibéré entre eux, le renvoyèrent, en disant : Il se tournera vers son maître Saül, au péril de nos têtes.
\VS{20}Quand donc il retournait à Tsiklag, Adnach, Jozabad, Jediaël, Micaël, Jozabad, Elihu et Tsilthaï, chefs des milliers qui étaient en Manassé, se tournèrent vers lui.
\VS{21}Et ils aidèrent David contre la troupe des Amalécites, car ils étaient tous forts et vaillants, et ils furent faits chefs dans l'armée.
\VS{22}De jour en jour, il venait des gens auprès de David pour l'aider, de sorte qu'il eut une grande armée, comme une armée de Dieu\FTNT{1 S. 22:2-4}.
\TextTitle{Les guerriers venus chez David à Hébron\FTNTT{2 S. 5:1-3}}
\VS{23}Voici le nombre des hommes équipés pour la guerre, qui vinrent auprès de David à Hébron, afin de lui transférer la royauté de Saül, selon le commandement de Yahweh\FTNT{2 S. 5:1-3}.
\VS{24}Des fils de Juda, qui portaient le bouclier et la lance, six mille huit cents, équipés pour la guerre.
\VS{25}Des fils de Siméon, forts et vaillants pour la guerre, sept mille cent.
\VS{26}Des fils de Lévi, quatre mille six cents.
\VS{27}Et Jehojada, prince de ceux d'Aaron, et avec lui trois mille sept cents ;
\VS{28}et Tsadok, jeune homme fort et vaillant, et vingt-deux chefs de la maison de son père.
\VS{29}Des fils de Benjamin, parents de Saül, trois mille ; car jusqu'alors la plus grande partie d’entre eux soutenaient la maison de Saül.
\VS{30}Des fils d'Ephraïm, vingt mille huit cents, forts et vaillants, et hommes de renom dans la maison de leurs pères.
\VS{31}De la demi-tribu de Manassé, dix-huit mille, qui furent désignés par leur nom pour aller établir David roi.
\VS{32}Des fils d'Issacar\FTNT{Les fils d’Issacar avaient la connaissance des temps. Discerner les temps dans lesquels nous sommes n’a rien à voir avec le fait de chercher à connaître la date du retour du Seigneur. Seul le Père connaît la date du retour du Messie (Za. 14:7 ; Mt. 24:36). Comprendre les caractéristiques de notre époque nous aide à nous réveiller afin d’accomplir les oeuvres que le Seigneur nous confie. Cette prise de conscience nous aidera à éviter les pièges de Satan et à mieux nous préparer aux noces de l’Agneau. Voir Mt. 16:3 ;  Ro. 13:11-14 ;  2 Pi. 1:19.}, fort intelligents dans la connaissance des temps, pour savoir ce que devait faire Israël, deux cents de leurs chefs, et tous leurs frères sous leurs ordres.
\VS{33}De Zabulon, cinquante mille combattants, rangés en bataille avec toutes sortes d'armes, et prêts à livrer bataille d’un cœur assuré.
\VS{34}De Nephthali, mille capitaines, et avec eux trente-sept mille, portant le bouclier et la lance.
\VS{35}Des Danites, vingt-huit mille six cents, équipés pour la guerre.
\VS{36}D'Aser, quarante mille combattants, et prêts à combattre.
\VS{37}De l’autre côté du Jourdain, à savoir des Rubénites, des Gadites, et de la demi-tribu de Manassé, cent vingt mille, avec tous les instruments de guerre pour combattre.
\VS{38}Tous ces hommes, gens de guerre, prêts a combattre, vinrent tous de bon cœur à Hébron, pour établir David roi sur tout Israël. Et tout le reste d'Israël était aussi d'un même sentiment pour établir David roi.
\VS{39}Et ils furent là avec David, mangeant et buvant pendant trois jours ; car leurs frères leur avaient préparé des vivres.
\VS{40}Et même ceux qui étaient les plus proches d'eux, jusqu'à Issacar, Zabulon et Nephthali, apportaient du pain sur des ânes, sur des chameaux, sur des mulets et sur des bœufs, de la farine, des figues sèches, des raisins secs, du vin, et de l'huile ; et ils amenaient des bœufs et des brebis en abondance, car il y avait une joie en Israël.
\Chap{13}
\TextTitle{Retour de l'arche, Uzza frappé par Yahweh\FTNTT{2 S. 6:1-11}}
\VerseOne{}Or David tint conseil avec les chefs de milliers et de centaines, avec tous les princes du peuple.
\VS{2}Et il dit à toute l'assemblée d'Israël : Si vous l'approuvez, et que cela vient de Yahweh, notre Dieu, envoyons partout vers nos autres frères, qui sont dans toutes les contrées d'Israël, et avec lesquels sont les sacrificateurs et les Lévites, dans leurs villes et dans leurs banlieues, afin qu'ils se réunissent à nous,
\VS{3}et que nous ramenions auprès de nous l’arche de notre Dieu ; car nous ne nous en sommes pas occupés du temps de Saül.
\VS{4}Et toute l'assemblée répondit qu'on le fasse ainsi ; car la chose fut approuvée par tout le peuple.
\VS{5}David donc assembla tout Israël, depuis Schichor, le torrent d'Egypte, jusqu'à l'entrée du pays de Hamath, pour ramener de Kirjath-Jearim l’arche de Dieu.
\VS{6}Et David monta avec tout Israël vers Baala à Kirjath-Jearim, qui appartient à Juda, pour faire amener de là l’arche de Dieu, devant laquelle est invoqué le Nom de Yahweh, qui habite entre les chérubins.
\VS{7}Ils mirent l’arche de Dieu sur un char neuf, et l'emmenèrent de la maison d'Abinadab ; et Uzza et Achjo conduisaient le char.
\VS{8}Et David et tout Israël dansaient en présence de Dieu de toute leur force, en chantant des cantiques et en jouant sur des violons, des luths, des tambourins, des cymbales, et des trompettes.
\VS{9}Quand ils furent arrivés à l'aire de Kidon, Uzza\FTNT{L’arche devait être transportée grâce à  des barres faites spécialement à cet effet, qui ne devaient pas être enlevées (Ex 27:6-7 ;  No. 1:51). Selon la Loi, seuls les Lévites devaient préparer et déplacer tout ce qui concernait le tabernacle.  Et même parmi les Lévites, chaque famille avait une fonction spécifique (No. 3 ; No. 4). Les Kehathites n’étaient pas autorisés à toucher l’arche, leur rôle se limitait seulement à la transporter à l’aide des barres (No. 4:15). Uzza a étendu sa main sur l’arche, alors qu’il n’était certainement pas Lévite. Il était devenu trop familier avec les choses saintes et avait pris à la légère les principes de Dieu. Il a voulu aider le Seigneur. Or, il ne faut jamais chercher à servir Dieu sans être appelé par lui.}  étendit sa main pour retenir l’arche, parce que les boeufs avaient glissé.
\VS{10}Et la colère de Yahweh s'enflamma contre Uzza, et il le frappa, parce qu'il avait étendu sa main sur l’arche. Uzza mourut en présence de Dieu.
\VS{11}David fut irrité de ce que Yahweh avait fait une brèche en la personne de Uzza. On a appelé jusqu'à ce jour ce lieu-là Pérets-Uzza, brèche d'Uzza.
\VS{12}David eut peur de Dieu en ce jour-là, et il dit: Comment ferais-je entrer chez moi l’arche de Dieu ?
\VS{13}C'est pourquoi David ne la retira point chez lui, dans la cité de David, mais il la fit conduire dans la maison d'Obed- Edom de Gath.
\VS{14}Et l’arche de Dieu demeura trois mois avec la famille d'Obed-Edom, dans sa maison. Yahweh bénit la maison d'Obed-Edom, et tout ce qui lui appartenait.
\Chap{14}
\TextTitle{Rayonnement du règne de David\FTNTT{2 S. 5:11-25 ; 23:13-17 ; 1 Ch. 3:5-9 ; 11:15-19 ; 12:8-15}}
\VerseOne{}Hiram, roi de Tyr, envoya des messagers à David, et du bois de cèdre, des tailleurs de pierres et des charpentiers, pour lui bâtir une maison.
\VS{2}Alors David reconnut que Yahweh l'affermissait comme roi sur Israël, et que son règne était fort élevé, à cause de son peuple d'Israël.
\VS{3}David prit encore des femmes à Jérusalem, et il engendra encore des fils et des filles.
\VS{4}Voici les noms des fils qu'il eut à Jérusalem : Schammua, Schobab, Nathan, Salomon,
\VS{5}Jibhar, Elischua, Elphéleth,
\VS{6}Noga, Népheg, Japhia,
\VS{7}Elischama, Beéliada et Eliphéleth.
\VS{8}Or quand les Philistins apprirent que David avait été oint pour roi sur tout Israël, ils montèrent tous à sa recherche. David l'ayant appris, sortit au-devant d'eux.
\VS{9}Les Philistins vinrent et se répandirent dans la vallée des Rephaïm.
\VS{10}David consulta Dieu, en disant : Monterai-je contre les Philistins, et les livreras-tu entre mes mains? Yahweh lui répondit : Monte, et je les livrerai entre tes mains.
\VS{11}Alors ils montèrent à Baal-Peratsim\FTNT{Baal-Peratsim signifie « seigneur des brèches »}, où David les battit. Puis il dit : Dieu a fait une brèche au milieu de mes ennemis par ma main, comme une brèche faite par les eaux. C'est pourquoi on donna à ce lieu-là le nom de Baal- Peratsim.
\VS{12}Et ils laissèrent là leurs dieux, et David ordonna qu'on les brûle au feu.
\VS{13}Les Philistins se répandirent encore une autre fois dans cette même vallée.
\VS{14}David consulta encore Dieu ; et Dieu lui répondit : Tu ne monteras point vers eux, mais tu te détourneras d'eux, et tu iras contre eux vis-à-vis des mûriers.
\VS{15}Dès que tu auras entendu au sommet des mûriers un bruit comme des gens qui marchent, tu sortiras alors pour combattre, car c’est Dieu qui marche devant toi pour frapper le camp des Philistins.
\VS{16}David fit selon ce que Dieu lui avait ordonné, et on frappa le camp des Philistins, depuis Gabaon jusqu'à Guézer.
\VS{17}Ainsi, la renommée de David se répandit par tous ces pays-là, et Yahweh remplit de frayeur toutes ces nations-là, au seul nom de David.
\Chap{15}
\TextTitle{David coordonne avec minutie l’arrivé de l’arche à Jérusalem\FTNTT{2 S. 6:12}}
\VerseOne{}David se bâtit des maisons dans la cité de David ; il prépara un lieu pour l’arche de Dieu, et dressa pour elle une tente.
\VS{2}Et David dit : L’arche de Dieu ne doit être portée que par les Lévites, car Yahweh les a choisis pour porter l’arche de Dieu, et pour faire le service à toujours\FTNT{No. 4:15.}.
\VS{3}David donc assembla tous ceux d'Israël à Jérusalem, pour faire monter l’arche de Yahweh dans le lieu qu'il lui avait préparé.
\VS{4}David assembla aussi les fils d'Aaron, et les Lévites.
\VS{5}Des fils de Kehath : Uriel, le chef, et ses frères, cent vingt.
\VS{6}Des fils de Merari : Asaja, le chef, et ses frères, deux cent vingt.
\VS{7}Des fils de Guerschon : Joël, le chef, et ses frères, cent trente.
\VS{8}Des fils d'Elitsaphan : Schemaeja, le chef, et ses frères, deux cents.
\VS{9}Des fils de Hébron : Eliel, le chef, et ses frères, quatre-vingts.
\VS{10}Des fils de Uziel : Amminadab, le chef, et ses frères, cent douze.
\VS{11}David appela les sacrificateurs Tsadok et Abiathar, et les Lévites, à savoir Uriel, Asaja, Joël, Schemaeja, Eliel, et Amminadab ;
\VS{12}et il leur dit: Vous qui êtes les chefs des familles des Lévites, sanctifiez-vous, vous et vos frères ; et transportez l’arche de Yahweh, le Dieu d'Israël, au lieu que je lui ai préparé.
\VS{13}Parce que vous n'y étiez pas la première fois, Yahweh, notre Dieu, a fait une brèche parmi nous ; car nous ne l'avons pas cherché selon la loi.
\VS{14}Les sacrificateurs donc et les Lévites se sanctifièrent pour faire monter l’arche de Yahweh, le Dieu d'Israël.
\VS{15}Et les fils des Lévites portèrent l’arche de Dieu sur leurs épaules, avec les barres qu'ils avaient sur eux, comme Moïse l'avait ordonné selon la parole de Yahweh.
\VS{16}David dit aux chefs des Lévites d'établir quelques-uns de leurs frères chantres, avec des instruments de musique, des luths, des violons, et des cymbales qui feraient retentir des sons éclatants, en signe de réjouissance.
\VS{17}Les Lévites donc établirent Héman, fils de Joël, et parmi ses frères, Asaph, fils de Bérékia; et des fils de Merari, qui étaient leurs frères, Ethan, fils de Kuschaja ;
\VS{18}avec eux leurs frères pour être du second ordre : Zacharie, Ben, Jaaziel, Schemiramoth, Jehiel, Unni, Eliab, Benaja, Maaséja, Matthithia, Eliphelé, Miknéja, Obed-Edom, et Jeïel, les portiers.
\VS{19}Quant aux chantres : Héman, Asaph et Ethan, ils avaient des cymbales d'airain pour les faire retentir.
\VS{20}Zacharie, Aziel, Schemiramoth, Jehiel, Unni, Eliab, Maaséja, et Benaja jouaient des luths sur alamoth ;
\VS{21}et Matthithia, Eliphelé, Miknéja, Obed-Edom, Jeïel et Azazia jouaient des harpes à huit cordes, pour conduire le chant.
\VS{22}Mais Kenania, le chef des Lévites, avait la charge de faire porter l’arche, enseignant comment il fallait la porter, car il était un homme très intelligent.
\VS{23}Bérékia et Elkana étaient portiers de l’arche.
\VS{24}Schebania, Josaphat, Nethaneel, Amasaï, Zacharie, Benaja, Eliézer, les sacrificateurs, sonnaient des trompettes devant l’arche de Dieu, et Obed-Edom et Jechija étaient portiers de l’arche.
\TextTitle{L'arche transportée au milieu des réjouissances\FTNTT{2 S. 6:12}}
\VS{25}David et les anciens d'Israël, avec les gouverneurs de milliers, marchaient, amenant avec joie l’arche de l'alliance de Yahweh, de la maison d' Obed-Edom.
\VS{26}Dieu aidait les Lévites qui portaient l’arche de l'alliance de Yahweh, et l’on sacrifia sept veaux et sept béliers.
\VS{27}David était vêtu d'un manteau de fin lin ; et tous les Lévites aussi qui portaient l’arche, les chantres ;  et Kenania, qui avait la principale charge de faire porter l’arche, était avec les chantres ; et David avait un éphod de lin.
\VS{28}Ainsi tout Israël amena l’arche de l'alliance de Yahweh, avec de grands cris de joie, et au son du cor, des shofars et des cymbales, faisant retentir leur voix avec des luths et des harpes.
\VS{29}Mais il arriva, comme l’arche de l'alliance de Yahweh entrait dans la cité de David, que Mical, fille de Saül, regardant par la fenêtre, vit le roi David sautant et dansant, et elle le méprisa dans son coeur.
\Chap{16}
\TextTitle{L’arche placée dans une tente à Jérusalem ; sacrifices et cantiques pour Yahweh\FTNTT{2 S. 6:17-19}}
\VerseOne{}Ils amenèrent donc l’arche de Dieu et la posèrent au milieu de la tente que David avait dressée pour elle ; et on offrit devant Dieu des holocaustes et des sacrifices d’offrande de paix.
\VS{2}Quand David eut achevé d'offrir les holocaustes et les sacrifices d’offrande de paix, il bénit le peuple au Nom de Yahweh.
\VS{3}Et il distribua à chacun, tant aux hommes qu'aux femmes, un pain,  un morceau de viande et un gâteau de raisin.
\VS{4}Et il établit quelques-uns des Lévites pour faire le service devant l’arche de Yahweh, pour célébrer, remercier, et louer le Dieu d'Israël.
\VS{5}Asaph était le premier et Zacharie le second ; Jeïel, Schemiramoth, Jehiel, Matthithia, Eliab, Benaja, Obed-Edom, et Jeïel, qui avaient des instruments de musique, à savoir des luths et des harpes ; et Asaph faisait retentir sa voix avec des cymbales.
\VS{6}Benaja et Jachaziel, les sacrificateurs, étaient continuellement avec des trompettes devant l’arche de l'alliance de Dieu.
\VS{7}Et en ce même jour, David remit entre les mains d'Asaph et de ses frères, les Psaumes suivants, pour commencer à célébrer Yahweh :
\VS{8}Célébrez Yahweh, invoquez son Nom ! Faites connaître parmi les peuples ses exploits !
\VS{9}Chantez-le,  célébrez-le !  Parlez de toutes ses merveilles !
\VS{10}Glorifiez-vous de son saint Nom !  Que le cœur de ceux qui cherchent Yahweh se réjouisse !
\VS{11}Recherchez Yahweh et sa force, cherchez continuellement sa face !
\VS{12}Souvenez-vous des merveilles qu'il a faites, de ses miracles et des jugements de sa bouche.
\VS{13}Postérité d'Israël, son serviteur, fils de Jacob, ses élus !
\VS{14}Yahweh est notre Dieu ; ses jugements s’exercent sur toute la terre.
\VS{15}Souvenez-vous toujours de son alliance, de ses promesses établies pour mille générations ;
\VS{16}du traité qu'il a fait avec Abraham et du serment qu’il a fait à Isaac,
\VS{17}et qu’il a confirmé à Jacob et à Israël, pour être une loi et une alliance éternelle,
\VS{18}en disant : Je te donnerai le pays de Canaan, comme l’héritage qui vous est échu.
\VS{19}Ils étaient alors une  poignée de gens, peu nombreux, et étrangers dans le pays,
\VS{20}car ils étaient errants de nation en nation, et d'un royaume vers un autre peuple.
\VS{21}Il ne permit à personne de les opprimer ; il a même châtié des rois à cause d'eux.
\VS{22}Et il a dit : Ne touchez point à mes oints, et ne faites point de mal à mes prophètes\FTNT{L’expression « ne touchez pas à mes oints » signifie qu'il ne faut pas leur porter physiquement atteinte. C’est une expression associée à des mauvais traitements physiques. Il est donc clair que ce verset, qu’on trouve également dans le  Ps. 105 : 15, ne peut absolument pas concerner la remise en question des enseignements d'un quelconque pasteur, prophète ou apôtre. Dans le contexte de ce passage, il est question des rois, des prophètes et des sacrificateurs, car c’est sur eux que reposait l’onction. Aujourd’hui, tous les chrétiens sont oints de Dieu (Ep. 1:13 ; Ep. 4:30).} !
\VS{23}Habitants de la terre, chantez à Yahweh ! Racontez chaque jour sa délivrance.
\VS{24}Racontez sa gloire parmi les nations, et ses merveilles parmi tous les peuples !
\VS{25}Car Yahweh est grand et très digne de louanges, il est plus redoutable que tous les dieux.
\VS{26}Car tous les dieux des peuples sont des idoles\FTNT{Jésus-Christ est le seul et le véritable Dieu (1 Co. 8:6 ; 1 Jn. 5:20).}, mais Yahweh a fait les cieux.
\VS{27}La majesté et la magnificence marchent devant lui; la force et la joie sont dans le lieu où il habite.
\VS{28}Familles des peuples, donnez à Yahweh, donnez à Yahweh gloire et force !
\VS{29}Donnez à Yahweh la gloire due à son Nom ! Apportez des offrandes, et présentez-vous devant lui. Prosternez-vous devant Yahweh avec des ornements saints !
\VS{30}Tremblez, vous tous habitants de la terre tout étonnés devant sa face ! Car la terre habitable est affermie par lui, et elle ne chancelle point.
\VS{31}Que les cieux se réjouissent, que la terre soit dans l’allégresse ! Et que l’on dise parmi les nations : Yahweh règne !
\VS{32}Que la mer retentisse avec tout ce qu'elle contient ! Que la campagne se réjouisse avec tout ce qu’elle renferme !
\VS{33}Que les arbres de la forêt poussent des cris de joie au devant de Yahweh, parce qu'il vient juger la terre\FTNT{Yahweh vient juger la terre. Cette prophétie confirme de façon incontestable la divinité de Jésus-Christ. Voir Za. 14:1-7.}.
\VS{34}Célébrez Yahweh, car il est bon, car sa miséricorde demeure à jamais !
\VS{35}Et dites : Ô Dieu de notre salut, sauve-nous, et rassemble-nous, et retire-nous d'entre les nations, pour célébrer ton saint Nom, et que nous nous glorifions de ta louange !
\VS{36}Béni soit Yahweh, le Dieu d'Israël, de siècle en siècle ! Et tout le peuple dit : Amen ! Louez Yahweh !
\VS{37}On laissa donc là, devant l’arche de l'alliance de Yahweh, Asaph et ses frères, pour faire le service continuellement, remplissant leur tâche jour par jour devant l’arche.
\VS{38}On laissa Obed-Edom, et ses frères, au nombre de soixante-huit, Obed-Edom, dis-je, fils de Jeduthun, et Hosa comme portiers.
\VS{39}On établit le sacrificateur Tsadok, et les sacrificateurs ses frères, devant le tabernacle de Yahweh, dans le haut lieu qui était à Gabaon,
\VS{40}pour offrir des holocaustes à Yahweh continuellement sur l'autel de l'holocauste, matin et soir, selon tout ce qui est écrit dans la loi de Yahweh, qu’il ordonna à Israël.
\VS{41}Auprès d’eux étaient Héman et Jeduthun, et les autres qui furent choisis et désignés par leur nom, pour célébrer Yahweh, parce que sa miséricorde demeure éternellement.
\VS{42}Et Héman et Jeduthun étaient avec ceux-là; il y avait aussi des trompettes et des cymbales pour ceux qui les faisaient retentir, et des instruments pour chanter les cantiques de Dieu. Les fils de Jeduthun étaient portiers.
\VS{43}Puis tout le peuple s'en alla chacun dans sa maison, et David aussi s'en retourna pour bénir sa maison.
\Chap{17}
\TextTitle{David veut construire un temple à Yahweh\FTNTT{2 S. 7:1-3}}
\VerseOne{}Or il arriva après que David fut établi dans sa maison, qu'il dit à Nathan, le prophète : Voici,  j’habite dans une maison de cèdres, et l’arche de l'alliance de Yahweh est sous une tente.
\VS{2}Nathan dit à David : Fais tout ce que tu as dans le cœur, car Dieu est avec toi.
\TextTitle{Réponse de Yahweh à David\FTNTT{2 S. 7:4-17}}
\VS{3}Mais il arriva cette nuit-là que la parole de Dieu fut adressée à Nathan, en disant :
\VS{4}Va, et dis à David, mon serviteur : Ainsi parle Yahweh :  Tu ne me bâtiras point de maison pour y habiter.
\VS{5}Puisque je n'ai point habité dans une maison depuis le jour où j'ai fait monter les fils d’Israël hors d'Egypte jusqu'à ce jour ; mais j'ai été de tente en tente, et de tabernacle en tabernacle.
\VS{6}Partout où j’ai marché avec tout Israël, ai-je dit un mot à un seul des juges d'Israël, auxquels j'ai ordonné de paître mon peuple, ai-je dit : Pourquoi ne m'avez-vous point bâti une maison de cèdres ?
\VS{7}Maintenant donc tu diras ainsi à David, mon serviteur : Ainsi parle Yahweh des armées : Je t'ai pris d'une cabane, d'auprès des brebis, afin que tu sois le conducteur de mon peuple d'Israël ;
\VS{8}j'ai été avec toi partout où tu as marché, j'ai exterminé devant toi tous tes ennemis, et j’ai rendu ton nom semblable au nom des grands qui sont sur la terre.
\VS{9}J’ai établi un lieu pour mon peuple d'Israël, et je l’ai planté afin qu’il habite chez lui et ne soit plus agité.  Les fils d'iniquité ne le détruiront plus comme ils l’ont fait auparavant,
\VS{10}et comme à l’époque où j'ai établi des juges sur mon peuple d'Israël. J’ai humilié tous tes ennemis. Je t’informe que Yahweh te bâtira une maison.
\VS{11}Quand tes jours seront accomplis pour t'en aller avec tes pères, je ferai lever ta postérité après toi, l’un de tes fils, et j'affermirai son règne\FTNT{Cette prophétie est relative au Messie. Voir 2 S. 7:12-17.}.
\VS{12}Il me bâtira une maison, et j'affermirai son trône éternellement.
\VS{13}Je serai pour lui un père, et il sera pour moi un fils ; et je ne retirerai point de lui ma grâce, comme je l'ai retirée de celui qui a été avant toi.
\VS{14}Mais je l'établirai dans ma maison et dans mon royaume éternellement, et son trône sera affermi pour toujours.
\VS{15}Nathan récita à David toutes ces paroles, et toute cette vision.
\TextTitle{Adoration et reconnaissance de David à Yahweh\FTNTT{2 S. 7:18-29}}
\VS{16}Alors le roi David entra, et se tint devant Yahweh, et dit : Ô Yahweh Dieu ! Qui suis-je, et quelle est ma maison, que tu m'aies fait parvenir au point où je suis ?
\VS{17}Mais cela t'a semblé être peu de chose, ô Dieu ! Et tu as parlé de la maison de ton serviteur pour le temps à venir, et tu as porté les regards sur moi à la manière de l'homme, toi qui es élevé, ô Yahweh Dieu !
\VS{18}Que pourrait te dire encore David de l'honneur que tu fais à ton serviteur ? Car tu connais ton serviteur.
\VS{19}Ô Yahweh ! Pour l'amour de ton serviteur, et selon ton cœur, tu as fait toutes ces grandes choses, pour lui révéler toutes ces grandeurs.
\VS{20}Ô Yahweh ! Nul n’est semblable à toi, et il n'y a point d'autre Dieu que toi selon tout ce que nous avons entendu de nos oreilles.
\VS{21}Et qui est comme ton peuple d'Israël, la seule nation sur la terre que Dieu lui-même est venu racheter pour lui, afin qu'elle soit son peuple, et pour te faire un Nom et pour accomplir des miracles et des prodiges, en chassant les nations devant ton peuple que tu as racheté d'Egypte ?
\VS{22}Et tu as établi ton peuple d'Israël afin qu’il soit ton peuple à toujours ; et toi, ô Yahweh ! Tu as été son Dieu.
\VS{23}Maintenant donc, ô Yahweh ! Que la parole que tu as prononcée sur ton serviteur et sur sa maison, soit ferme à jamais, et agis selon ta parole !
\VS{24}Et que ton Nom subsiste et soit magnifié éternellement, de sorte qu'on dise : Yahweh des armées, le Dieu d'Israël, est Dieu pour Israël ; et que la maison de David, ton serviteur, soit affermie devant toi.
\VS{25}Car, ô mon Dieu ! Tu as révélé à ton serviteur que tu lui bâtirais une maison. C'est pourquoi ton serviteur a pris la hardiesse de te faire cette prière.
\VS{26}Maintenant, ô Yahweh ! Tu es Dieu, et tu as parlé de ce bien à ton serviteur.
\VS{27}Veuille donc maintenant bénir la maison de ton serviteur, afin qu'elle soit éternellement devant toi ; car tu l'as bénie, ô Yahweh ! Et elle sera bénie à jamais !
\Chap{18}
\TextTitle{Le règne de David affermi\FTNTT{2 S. 8:1-18}}
\VerseOne{}Et il arriva que David battit les Philistins, et les humilia, et il enleva de la main des Philistins Gath et les villes de son ressort\FTNT{2 S. 8.  }.
\VS{2}Il battit aussi les Moabites, et les Moabites furent asservis à David et lui payèrent un tribut.
\VS{3}David battit aussi Hadarézer, roi de Tsoba, vers Hamath, lorsqu’il alla établir sa domination sur le fleuve de l'Euphrate.
\VS{4}David lui prit mille chars, sept mille cavaliers, et vingt mille hommes de pied ; et il coupa les jarrets des chevaux de tous les chars, mais il réserva cent chars.
\VS{5}Les Syriens de Damas vinrent au secours d’Hadarézer, roi de Tsoba, et David battit vingt-deux mille Syriens.
\VS{6}Puis David mit des garnisons dans la Syrie de Damas. Et les Syriens furent assujettis à David et lui payèrent un tribut. Yahweh sauvait David partout où il allait.
\VS{7}Et David prit les boucliers d'or qui étaient aux serviteurs de Hadarézer, et les apporta à Jérusalem.
\VS{8}Il emporta aussi de Thibchath, et de Cun, villes de Hadarézer, une grande quantité d'airain, dont Salomon fit la mer d'airain, les colonnes et les ustensiles d'airain.
\VS{9}Thohu, roi de Hamath, apprit que David avait défait toute l'armée de Hadarézer, roi de Tsoba.
\VS{10}Et il envoya Hadoram, son fils, vers le roi David pour le saluer et le féliciter de ce qu'il avait combattu Hadarézer, et qu'il l'avait défait. Car Hadarézer était dans une guerre continuelle contre Thohu. Quant à tous les vases d'or, d'argent, et d'airain,
\VS{11}le roi David les consacra aussi à Yahweh, avec l'argent et l'or qu'il avait emporté de toutes les nations, à savoir d'Edom, de Moab, des fils d'Ammon, des Philistins, et d’Amalek.
\VS{12}Et Abischaï, fils de Tseruja battit dix-huit mille Edomites dans la vallée du sel.
\VS{13}Il mit une garnison dans Edom, et tous les Edomites furent asservis à David ; et Yahweh gardait David partout où il allait.
\VS{14}Ainsi, David régna sur tout Israël, rendant jugement et justice à tout son peuple.
\VS{15}Joab, fils de Tseruja, avait la charge de l'armée, et Josaphat, fils d'Achilud, était archiviste.
\VS{16}Tsadok, fils d'Achithub, et Abimélec, fils d'Abiathar, étaient les sacrificateurs; et Schavscha était le secrétaire.
\VS{17}Benaja, fils de Jehojada, était sur les Kéréthiens et les Péléthiens ; mais les fils de David étaient les premiers auprès du roi.
\Chap{19}
\TextTitle{David monte contre les Ammonites et les Syriens\FTNTT{2 S. 10}}
\VerseOne{}Or il arriva après cela que Nachasch, roi des fils d’Ammon, mourut ; et son fils régna à sa place.
\VS{2}David dit : J'userai de bonté envers Hanun, fils de Nachasch, car son père a usé de bonté envers moi. Ainsi, David envoya des messagers pour le consoler de la mort de son père ; et les serviteurs de David vinrent au pays des fils d’Ammon vers Hanun pour le consoler.
\VS{3}Mais les chefs d'entre les fils d’Ammon dirent à Hanun : Penses-tu que ce soit pour honorer ton père que David t'a envoyé des consolateurs ? N'est-ce pas pour examiner et épier le pays, afin de le détruire, que ses serviteurs sont venus vers toi ?
\VS{4}Alors Hanun prit les serviteurs de David, les fit raser, et les fit couper leurs habits par le milieu jusqu'aux hanches. Puis il les renvoya.
\VS{5}Ils s'en allèrent, et le firent savoir par le moyen de quelques personnes à David, qui envoya des gens à leur rencontre ; car ces hommes-là étaient fort confus. Et le roi leur fit dire : Restez à Jéricho jusqu'à ce que votre barbe ait repoussé, et revenez ensuite.
\VS{6}Or les fils d’Ammon voyant qu'ils s'étaient rendus odieux à David, Hanun et les fils d'Ammon envoyèrent mille talents d'argent pour prendre à leur solde des chars et des cavaliers de Mésopotamie, de Syrie, de Maaca et de Tsoba.
\VS{7}Ils prirent à leur solde trente-deux mille hommes et des chars, et le roi de Maaca avec son peuple, lesquels vinrent camper devant Médeba. Les fils d'Ammon aussi s'assemblèrent de leurs villes et vinrent pour combattre.
\VS{8}David l’ayant appris, envoya Joab et ceux de toute l'armée qui étaient les plus vaillants.
\VS{9}Les fils d’Ammon sortirent et rangèrent leur armée en bataille à l'entrée de la ville ; et les rois qui étaient venus étaient à part dans la campagne.
\VS{10}Joab, voyant que l'armée était tournée contre lui devant et derrière, prit de tous les gens d'élite d'Israël, et les rangea contre les Syriens.
\VS{11}Et il donna la conduite du reste du peuple à Abischaï, son frère; et on les rangea contre les fils d’Ammon.
\VS{12}Et Joab lui dit: Si les Syriens sont plus forts que moi, tu viendras me délivrer ; et si les fils d'Ammon sont plus forts que toi, je te délivrerai.
\VS{13}Sois ferme, et montrons-nous vaillants pour notre peuple, et pour les villes de notre Dieu ; et que Yahweh fasse ce qui lui semblera bon.
\VS{14}Alors Joab et le peuple qui était avec lui s'approchèrent pour livrer bataille aux Syriens qui s'enfuirent de devant lui.
\VS{15}Et les fils d'Ammon voyant que les Syriens s'étaient enfuis, eux aussi s'enfuirent devant Abischaï, frère de Joab, et rentrèrent dans la ville, et Joab revint à Jérusalem.
\VS{16}Mais les Syriens, qui avaient été battus par ceux d'Israël, envoyèrent des messagers et firent venir les Syriens qui étaient au-delà du fleuve ; et Schophach, chef de l'armée d'Hadarézer, les conduisait.
\VS{17}On le rapporta à David, qui assembla tout Israël, passa le Jourdain, alla au-devant d'eux, et se rangea en bataille contre eux. David donc rangea la bataille contre les Syriens, et ils combattirent contre lui.
\VS{18}Mais les Syriens s'enfuirent de devant Israël ; et David défit sept mille chars des Syriens et quarante mille hommes de pied ; et il tua Schophach, le chef de l'armée.
\VS{19}Alors les serviteurs d'Hadarézer, voyant qu'ils avaient été battus par ceux d'Israël, firent la paix avec David, et lui furent asservis ; et les Syriens ne voulurent plus secourir les fils d’Ammon.
\Chap{20}
\TextTitle{Conquête de Rabba\FTNTT{2 S. 11:1-12:25 ; 2 S. 12:26-31}}
\VerseOne{}L’année suivante, au temps où les rois se mettaient en campagne, Joab conduisit une forte armée et ravagea le pays des fils d’Ammon ; puis il alla assiéger Rabba, tandis que David resta à Jérusalem. Joab battit Rabba, et la détruisit\FTNT{2 S 12:26-31.}.
\VS{2}David enleva la couronne de dessus la tête de son roi, et il trouva qu'elle pesait un talent d'or : Elle était garnie de pierres précieuses. On la mit sur la tête de David, qui emmena un très grand butin de la ville.
\VS{3}Il emmena aussi le peuple qui y était, et les mit aux scies, aux pics de fer et aux haches de fer ; David traita de la sorte toutes les villes des fils d’Ammon ; puis il s'en retourna avec tout le peuple à Jérusalem.
\TextTitle{Guerre contre les Philistins\FTNTT{2 S. 21:15-22}}
\VS{4}Il arriva après cela que la guerre continua à Guézer contre les Philistins. Alors Sibbecaï, le Huschatite, frappa Sippaï, qui était des fils de Rapha, et ils furent humiliés\FTNT{2 S 21:15-22.}.
\VS{5}Il y eut encore une autre guerre contre les Philistins. Et Elchanan, fils de Jaïr, frappa Lachmi, frère de Goliath de Gath, qui avait une lance dont le bois était comme une ensouple de tisserand.
\VS{6}Il y eut encore une autre guerre à Gath, où se trouva un homme de grande stature, qui avait six doigts à chaque main, et six orteils à chaque pied, de sorte qu'il en avait en tout vingt-quatre ; et il était aussi issu de Rapha.
\VS{7}Et il défia Israël ; mais Jonathan, fils de Schimea, frère de David, le tua.
\VS{8}Ceux-là naquirent à Gath ; ils étaient des enfants de Rapha, et ils moururent par les mains de David, et par les mains de ses serviteurs.
\Chap{21}
\TextTitle{David fait le dénombrement contre la volonté de Yahweh\FTNTT{2 S. 24:1-17}}
\VerseOne{}Mais Satan s'éleva contre Israël, et il incita David à faire le dénombrement d'Israël.
\VS{2}Et David dit à Joab et aux chefs du peuple : Allez et faites le dénombrement d'Israël, depuis Beer-Schéba jusqu'à Dan, et rapportez-le-moi, afin que j’en connaisse le nombre.
\VS{3}Mais Joab répondit : Que Yahweh veuille augmenter son peuple cent fois encore plus qu'il ne l’est, ô roi, mon seigneur. Tous ne sont-ils pas serviteurs de mon seigneur ? Pourquoi mon seigneur cherche-t-il cela ? Et pourquoi cela serait-il imputé comme un crime à Israël ?
\VS{4}Mais la parole du roi l'emporta sur Joab. Et Joab partit et parcourut tout Israël ; puis il revint à Jérusalem.
\VS{5}Et Joab donna à David le rôle du dénombrement du peuple, et il se trouva dans tout Israël onze cent mille hommes tirant l'épée ; et dans Juda quatre cent soixante-dix mille hommes tirant l'épée.
\VS{6}Bien qu'il n'eût pas compté entre eux ni Lévi ni Benjamin, parce que Joab exécutait la parole du roi en l’ayant en abomination,
\VS{7}cette chose déplut à Dieu, c'est pourquoi il frappa Israël.
\VS{8}Et David dit à Dieu : J'ai commis un très grand péché d'avoir fait une telle chose ; je te prie, pardonne maintenant l'iniquité de ton serviteur, car j'ai agi en insensé.
\VS{9}Et Yahweh parla à Gad, le voyant de David, en disant :
\VS{10}Va, parle à David, et dis-lui : Ainsi parle Yahweh, je te propose trois choses ; choisis l'une d'elles, afin que je te la fasse.
\VS{11}Et Gad vint à David, et lui dit : Ainsi parle Yahweh :
\VS{12}Choisis soit la famine durant l'espace de trois ans ; soit d'être consumé durant trois mois, étant poursuivi par tes ennemis, en sorte que l'épée de tes ennemis t'atteigne ; ou que l'épée de Yahweh, la peste, soit durant trois jours sur le pays, et que l'Ange de Yahweh porte la destruction dans toutes les contrées d'Israël. Vois maintenant ce que j'aurai à répondre à celui qui m'a envoyé.
\VS{13}Alors David répondit à Gad : Je suis dans une très grande angoisse ! Que je tombe, je te prie, entre les mains de Yahweh, parce que ses compassions sont immenses ; mais que je ne tombe point entre les mains des hommes !
\VS{14}Yahweh envoya donc la peste sur Israël, et il tomba soixante-dix mille hommes d'Israël.
\VS{15}Dieu envoya aussi un ange à Jérusalem pour la détruire ; et comme il la détruisait, Yahweh regarda et se repentit de ce mal. Et il dit à l'Ange qui détruisait : C'est assez ! Retire à présent ta main. Et l'Ange de Yahweh se tenait près de l'aire d'Ornan, le Jébusien.
\VS{16}Or David leva les yeux,  et vit l'Ange de Yahweh\FTNT{Ge. 16:7.} qui était entre la terre et le ciel, ayant dans sa main son épée nue, tournée contre Jérusalem. Et David et les anciens, couverts de sacs, tombèrent sur leurs faces.
\VS{17}Et David dit à Dieu : N'est-ce pas moi qui ai ordonné qu'on fasse le dénombrement du peuple ? C'est donc moi qui ai péché et qui ai très mal agi ; mais ces brebis qu'ont-elles fait ? Yahweh, mon Dieu ! Je te prie que ta main soit contre moi, et contre la maison de mon père, mais qu'elle ne soit pas contre ton peuple, pour le détruire.
\TextTitle{Fin de la plaie après l’offrande de David\FTNTT{2 S. 24:18-25}}
\VS{18}Alors l'Ange de Yahweh ordonna à Gad de dire à David, qu'il monte pour dresser un autel à Yahweh, dans l'aire d'Ornan, le Jébusien.
\VS{19}David donc monta selon la parole que Gad lui avait dite au Nom de Yahweh.
\VS{20}Ornan s'étant retourné, et ayant vu l'Ange,  ses quatre fils se cachèrent avec lui. Or Ornan foulait du blé.
\VS{21}David vint jusqu'à Ornan, et Ornan regarda, et ayant vu David, il sortit de l'aire et se prosterna devant lui, le visage à terre.
\VS{22}Et David dit à Ornan : Donne-moi la place de cette aire, et j'y bâtirai un autel à Yahweh ; donne-la-moi pour le prix qu'elle vaut, afin que cette plaie soit arrêtée de dessus le peuple.
\VS{23}Et Ornan dit à David : Prends-la, et que le roi, mon seigneur, fasse tout ce qui lui semblera bon. Voici, je donne ces bœufs pour les holocaustes, et ces instruments à fouler du blé pour le bois, et ce blé pour l'offrande ; je donne toutes ces choses.
\VS{24}Mais le roi David lui répondit : Non, mais certainement j'achèterai tout cela au prix qu'il vaut ; car je ne présenterai point à Yahweh ce qui est à toi, et je n'offrirai point un holocauste qui ne me coûte rien.
\VS{25}David donna donc à Ornan pour cette place, six cents sicles d'or de poids.
\VS{26}Puis il bâtit là un autel à Yahweh, et il offrit des holocaustes et des sacrifices d’offrande de paix, et il invoqua Yahweh, qui l'exauça par le feu envoyé des cieux sur l'autel de l'holocauste.
\VS{27}Alors Yahweh parla à l'ange, et l'ange remit son épée dans son fourreau.
\VS{28}En ce temps-là, David, voyant que Yahweh l'avait exaucé dans l'aire d'Ornan, le Jébusien, y offrait des sacrifices.
\VS{29}Or le tabernacle de Yahweh, que Moïse avait construit au désert, et l'autel des holocaustes, étaient en ce temps-là dans le haut lieu de Gabaon.
\VS{30}Mais David ne pouvait pas aller devant cet autel pour invoquer Dieu, parce qu'il avait été épouvanté à cause de l'épée de l'Ange de Yahweh.
\Chap{22}
\TextTitle{Préparatifs de David pour la construction du temple}
\VerseOne{}Et David dit : C'est ici la maison de Yahweh Dieu, et c'est ici l'autel pour les holocaustes d'Israël.
\VS{2}David ordonna de rassembler les étrangers qui étaient dans le pays d'Israël, et il établit des tailleurs de pierres pour tailler des pierres de taille, pour la construction de la maison de Dieu.
\VS{3}David prépara aussi du fer en abondance, afin d'en faire des clous pour les battants des portes et pour les crampons,  de l’airain en quantité telle qu’il n’était pas possible de le peser,
\VS{4}et du bois de cèdre sans nombre, parce que les Sidoniens et les Tyriens amenaient à David du bois de cèdre en abondance.
\VS{5}David dit : Salomon, mon fils, est jeune et délicat, et la maison qu'il faut bâtir à Yahweh doit être magnifique en excellence, en réputation, et en gloire, dans tous les pays. Je lui préparerai donc maintenant de quoi la bâtir. Ainsi, David prépara, avant sa mort, ces choses en abondance.
\TextTitle{Recommandation de David à Salomon }
\VS{6}Puis il appela Salomon, son fils, et lui ordonna de bâtir une maison à Yahweh, le Dieu d'Israël.
\VS{7}David donc dit à Salomon : Mon fils, j’avais à cœur de bâtir une maison au Nom de Yahweh, mon Dieu.
\VS{8}Mais la parole de Yahweh m'a été adressée, en disant : Tu as répandu beaucoup de sang, et tu as fait de grandes guerres ; tu ne bâtiras point de maison à mon Nom, parce que tu as répandu beaucoup de sang sur la terre devant moi.
\VS{9}Voici, il te naîtra un fils, qui sera un homme de repos,  et à qui je donnerai du repos par rapport à tous ses ennemis tout autour, c'est pourquoi son nom sera Salomon. Et en son temps, je donnerai la paix et le repos à Israël.
\VS{10}Ce sera lui qui bâtira une maison à mon Nom ; et il sera un fils pour moi, et je serai un père pour lui ; et j'affermirai le trône de son règne sur Israël à jamais.
\VS{11}Maintenant donc, mon fils, Yahweh sera avec toi, et tu prospéreras, et tu bâtiras la maison de Yahweh, ton Dieu, ainsi qu'il l’a déclaré à ton égard.
\VS{12}Seulement, que Yahweh te donne de la sagesse et de l'intelligence, et qu'il t'instruise touchant le gouvernement d'Israël, et comment tu dois garder la loi de Yahweh, ton Dieu.
\VS{13}Tu prospéreras si tu as soin de mettre en pratique les lois et les ordonnances que Yahweh a prescrites à Moïse pour Israël. Fortifie-toi et prends courage ; ne crains point et ne t'effraie de rien.
\VS{14}Voici, selon ma petitesse, j'ai préparé pour la maison de Yahweh cent mille talents d'or et un million de talents d'argent. Quant à l'airain et au fer, il est d'un poids incalculable, car il est en abondance. J'ai aussi préparé le bois et les pierres ; et tu y ajouteras ce qu'il faudra .
\VS{15}Tu as avec toi beaucoup d'ouvriers, de maçons, de tailleurs de pierres, de charpentiers, et toutes sortes de gens experts dans toute espèce d’ouvrage.
\VS{16}Il y a de l'or et de l'argent, de l'airain et du fer sans nombre. Lève-toi et agis, et Yahweh sera avec toi.
\VS{17}David ordonna aussi à tous les chefs d'Israël d'aider Salomon, son fils ; et il leur dit :
\VS{18}Yahweh,votre Dieu, n'est-il pas avec vous, et ne vous a-t-il pas donné du repos de tous côtés ? Car il a livré entre mes mains les habitants du pays, et le pays a été soumis devant Yahweh, et devant son peuple.
\VS{19}Maintenant donc, appliquez vos cœurs et vos âmes à rechercher Yahweh, votre Dieu ; levez-vous et bâtissez le sanctuaire de Yahweh Dieu, afin d’amener l’arche de l'alliance de Yahweh, et les ustensiles consacrés à Dieu dans la maison qui doit être bâtie au Nom de Yahweh.
\Chap{23}
\TextTitle{David désigne Salomon comme son successeur\FTNTT{1 Ch. 28:1}}
\VerseOne{}David étant vieux et rassasié de jours, établit Salomon, son fils, pour roi sur Israël.
\VS{2}Et il assembla tous les principaux d'Israël, les sacrificateurs et les Lévites.
\VS{3}On fit le dénombrement des Lévites, depuis l'âge de trente ans et au-dessus ; et les mâles d' entre-eux étant comptés, chacun par tête, il y eut trente-huit mille hommes\FTNT{No. 3:25-37}.
\VS{4}Et David dit : Qu’il y en ait  parmi eux vingt-quatre mille pour vaquer ordinairement à l'œuvre de la maison de Yahweh, et six mille comme  magistrats et juges,
\VS{5}quatre mille portiers, et quatre autres mille pour louer Yahweh avec des instruments que j'ai faits pour le louer.
\TextTitle{Dénombrement des Lévites\FTNTT{No. 3:25-37}}
\VS{6}David les divisa en classes d'après les fils de Lévi, à savoir Guerschon, Kehath et Merari.
\VS{7}Des Guerschonites il y eut Laedan et Schimeï.
\VS{8}Les fils de Laedan furent ces trois : Jehiel le premier, puis Zétham, puis Joël.
\VS{9}Les fils de Schimeï furent ces trois : Schelomith, Haziel et Haran. Ce sont là les chefs des maisons paternelles de la famille de Laedan.
\VS{10}Et les fils de Schimeï furent Jachath, Zina, Jeusch et Beria. Ce sont là les quatre fils de Schimeï.
\VS{11}Jachath était le premier et Zina le second; mais Jeusch et Beria n'eurent pas beaucoup de fils, c'est pourquoi ils furent comptés pour une seule maison paternelle dans le dénombrement.
\VS{12}Des fils de Kehath il y eut Amram, Jitsehar, Hébron et Uziel, en tout quatre.
\VS{13}Les fils d’Amram furent Aaron et Moïse. Aaron fut séparé lui et ses fils à toujours, pour sanctifier le Saint des saints, pour faire brûler des parfums en présence de Yahweh, pour le servir, et pour bénir en son Nom à toujours.
\VS{14}Et quant à Moïse, homme de Dieu, ses fils devaient être comptés de la tribu de Lévi.
\VS{15}Les fils de Moïse furent Guerschom et Eliézer.
\VS{16}Des fils de Guerschom, Schebuel le premier.
\VS{17}Quant aux fils d' Eliézer, Rechabia fut le premier ; et Eliézer n'eut point d'autres fils, mais les fils de Rechabia furent très nombreux.
\VS{18}Des fils de Jitsehar, Schelomith était le premier.
\VS{19}Les fils de Hébron furent Jerija le premier, Amaria le second, Jachaziel le troisième, Jekameam le quatrième.
\VS{20}Les fils d’Uziel furent Michée le premier, Jischija le second.
\VS{21}Des fils de Merari il y eut Machli et Muschi. Les fils de Machli furent Eléazar et Kis.
\VS{22}Eléazar mourut, et n'eut point de fils, mais des filles ; et les fils de Kis, leurs frères les prirent pour femmes.
\VS{23}Les fils de Muschi furent Machli, Eder et Jerémoth, eux trois.
\TextTitle{Fonctions des Lévites\FTNTT{No. 3:5-12}}
\VS{24}Ce sont là les fils de Lévi, selon les maisons de leurs pères, chefs des maisons paternelles, selon leurs dénombrements qui furent faits en comptant leurs noms, étant comptés chacun par tête ; et ils faisaient l’œuvre du service de la maison de Yahweh, depuis l'âge de vingt ans et au-dessus.
\VS{25}Car David dit : Yahweh, le Dieu d'Israël, a donné du repos à son peuple, et il établira sa demeure dans Jérusalem  à toujours.
\VS{26}Quant aux Lévites, ils n'auront plus à porter le tabernacle ni tous les ustensiles pour son service.
\VS{27}C'est pourquoi, dans les derniers registres de David, les fils de Lévi furent dénombrés depuis l'âge de vingt ans et au-dessus.
\VS{28}Car leur charge était d'assister les fils d'Aaron pour le service de la maison de Yahweh, étant établis sur le parvis et sur les chambres, pour la purification de toutes les choses saintes, pour l'œuvre du service de la maison de Dieu,
\VS{29}pour les pains de proposition, de la fleur de farine pour l'offrande, des galettes sans levain, pour tout ce qui se cuit sur la plaque, pour tout ce qui est rissolé, et pour la petite et grande mesure,
\VS{30}pour se présenter tous les matins et tous les soirs, afin de célébrer et louer Yahweh,
\VS{31}et offrir tous les holocaustes qu'il fallait offrir à Yahweh les jours de sabbat, aux nouvelles lunes, et aux fêtes solennelles, continuellement devant Yahweh, selon le nombre et les usages prescrits.
\VS{32}Ils donnaient leurs soins à la tente d’assignation, au lieu saint, et aux fils d’Aaron, leurs frères, pour le service de la maison de Yahweh.
\Chap{24}
\TextTitle{Vingt-quatre classes de sacrificateurs}
\VerseOne{}Quant aux fils d'Aaron, voici leurs classes\FTNT{Les vingt-quatre classes de sacrificateurs qui se tenaient devant Yahweh dans le temple de Jérusalem étaient une représentation des vingt-quatre vieillards qui se tiennent devant le trône de Dieu (Ap. 4:4).}. Les fils d’Aaron furent Nadab, Abihu, Eléazar et Ithamar.
\VS{2}Mais Nadab et Abihu\FTNT{Lé. 10:1-4.} moururent en présence de leur père, et n'eurent point de fils ; et Eléazar et Ithamar exercèrent la sacrificature.
\VS{3}Or David les sépara, à savoir Tsadok, qui était des fils d'Eléazar, et Achimélec, qui était des fils d'Ithamar, en fonction de leurs charges dans le service qu'ils avaient à faire.
\VS{4}Il se trouva parmi les fils d'Eléazar plus de chefs que parmi les fils d'Ithamar, et on en fit la division ; les fils d'Eléazar avaient seize chefs, selon leurs maisons paternelles, et les fils d'Ithamar huit chefs de maisons paternelles.
\VS{5}Et on les classa par le sort, les entremêlant les uns avec les autres, car les chefs du sanctuaire et les chefs de la maison de Dieu furent tirés tant des fils d'Eléazar que des fils d'Ithamar.
\VS{6}Schemaeja, fils de Nethaneel, le scribe, qui était de la tribu de Lévi, les mit par écrit devant le roi, les princes du peuple, devant Tsadok, le sacrificateur, et Achimélec, fils d'Abiathar, et devant les chefs de maisons paternelles des sacrificateurs et des Lévites. On tira au sort une maison paternelle pour Eléazar, et une autre fut tirée pour Ithamar.
\VS{7}Le premier sort échut à Jehojarib, le second à Jedaeja,
\VS{8}le troisième à Harim, le quatrième à Seorim,
\VS{9}le cinquième à Malkija, le sixième à Mijamin,
\VS{10}le septième à Hakkots, le huitième à Abija,
\VS{11}le neuvième à Josué, le dixième à Schecania,
\VS{12}le onzième à Eliaschib, le douzième à Jakim,
\VS{13}le treizième à Huppa, le quatorzième à Jeschébeab,
\VS{14}le quinzième à Bilga, le seizième à Immer,
\VS{15}le dix-septième à Hézir, le dix-huitième à Happitsets,
\VS{16}le dix-neuvième à Pethachja, le vingtième à Ezéchiel,
\VS{17}le vingt et unième à Jakim, et le vingt-deuxième à Gamul,
\VS{18}le vingt-troisième à Delaja, le vingt-quatrième à Maazia.
\VS{19}Tel fut leur classement pour le service qu'ils avaient à faire, lorsqu'ils entraient dans la maison de Yahweh, selon qu'il leur avait été ordonné par Aaron, leur père, comme Yahweh, le Dieu d'Israël, le lui avait ordonné.
\TextTitle{Les chefs des lévites ; les fils de Kehath et de Merari}
\VS{20}Voici les chefs du reste des Lévites. Des fils de Amram : Schubaël ; et des fils de Shubaël, Jechdia.
\VS{21}De Rechabia, des fils de Rechabia, Jischija était le premier.
\VS{22}Des Jitseharites, Schelomoth ; des fils de Schelomoth, Jachath.
\VS{23}Des fils d’Hébron, Jerija, Amaria le second ; Jachaziel le troisième, Jekameam le quatrième.
\VS{24}Des fils d'Uziel, Michée ; des fils de Michée, Schamir.
\VS{25}Le frère de Michée était Jischija ; des fils de Jischija, Zacharie.
\VS{26}Des fils de Merari, Machli et Muschi. Des fils de Jaazija, son fils.
\VS{27}Des fils de Merari, de Jaazija, son fils : Schoham, Zaccur et Ibri.
\VS{28}De Machli, Eléazar, qui n'eut point de fils.
\VS{29}De Kis, les fils de Kis, Jerachmeel.
\VS{30}Et des fils de Muschi, Machli, Eder et Jerimoth. Ce sont là les fils des Lévites, selon les maisons de leurs pères.
\VS{31}Eux aussi, comme leurs frères, les fils d'Aaron, ils tirèrent au sort, devant le  roi David, Tsadok et Ahimélec, et les chefs des pères des sacrificateurs et des Lévites. Il en fut ainsi pour chaque chef de maison comme pour le moindre de ses frères.
\Chap{25}
\TextTitle{Dénombrement des musiciens et des chantres}
\VerseOne{}David et les chefs de l'armée mirent à part pour le service ceux des fils d'Asaph, d'Héman et de Jeduthun qui prophétisaient avec des harpes, des luths et des cymbales. Et voici le nombre des hommes employés pour le service qu’ils avaient à faire.
\VS{2}Des fils d'Asaph : Zaccur, Joseph, Nethania et Aschareéla, fils d'Asaph, sous la conduite d'Asaph, qui prophétisait selon les ordres du roi.
\VS{3}De Jeduthun, les six fils de Jeduthun : Guedalia, Tseri, Esaïe, Haschabia, Matthithia et Schimeï, jouaient de la harpe, sous la conduite de leur père Jeduthun, qui prophétisait en célébrant et louant Yahweh.
\VS{4}D'Héman, les fils d'Héman : Bukkija, Matthania, Uziel, Schebuel, Jerimoth, Hanania, Hanani, Eliatha, Guiddalthi, Romamthi-Ezer, Joschbekascha, Mallothi, Hothir, Machazioth.
\VS{5}Tous ceux-là étaient fils d'Héman, le voyant du roi, qui révélait les paroles de Dieu pour en exalter la puissance. Dieu donna à Héman quatorze fils et trois filles.
\VS{6}Tous ceux-là étaient employés, sous la conduite de leurs pères, aux cantiques de la maison de Yahweh, avec des cymbales, des luths, et des harpes, dans le service de la maison de Dieu, selon les ordres du roi donnés à Asaph, à Jeduthun et à Héman.
\VS{7}Et leur nombre avec leurs frères, auxquels on avait enseigné les cantiques de Yahweh, était de deux cent quatre-vingt-huit, tous très habiles.
\TextTitle{Les musiciens et chantres répartis en vingt-quatre classes}
\VS{8}Et ils tirèrent au sort pour leurs fonctions, petits et grands, maîtres et disciples.
\VS{9}Et le premier sort échut à Asaph, à savoir à Joseph. Le second à Guedalia, lui, ses frères et ses fils étaient douze.
\VS{10}Le troisième à Zaccur, lui, ses fils et ses frères étaient douze.
\VS{11}Le quatrième à Jitseri, lui, ses fils et ses frères étaient douze.
\VS{12}Le cinquième à Nethania, lui, ses fils et ses frères étaient douze.
\VS{13}Le sixième à Bukkija, lui, ses fils et ses frères étaient douze.
\VS{14}Le septième à Jesareéla, lui, ses fils et ses frères étaient douze.
\VS{15}Le huitième à Esaïe, lui, ses fils et ses frères étaient douze.
\VS{16}Le neuvième à Matthania, lui, ses fils et ses frères étaient douze.
\VS{17}Le dixième à Schimeï, lui, ses fils et ses frères étaient douze.
\VS{18}L'onzième à Azareel, lui, ses fils et ses frères étaient douze.
\VS{19}Le douzième à Haschabia, lui, ses fils, et ses frères étaient douze.
\VS{20}Le treizième à Schubaël, lui, ses fils et ses frères étaient douze.
\VS{21}Le quatorzième à Matthithia, lui, ses fils et ses frères étaient douze.
\VS{22}Le quinzième à Jerémoth, lui, ses fils et ses frères étaient douze.
\VS{23}Le seizième à Hanania, lui, ses fils et ses frères étaient douze.
\VS{24}Le dix-septième à Joschbekascha, lui, ses fils et ses frères étaient douze.
\VS{25}Le dix-huitième à Hanani, lui, ses fils et ses frères étaient douze.
\VS{26}Le dix-neuvième à Mallothi, lui, ses fils et ses frères étaient douze.
\VS{27}Le vingtième à Elijatha, lui, ses fils et ses frères étaient douze.
\VS{28}Le vingt et unième à Hothir, lui, ses fils et ses frères étaient douze.
\VS{29}Le vingt-deuxième à Guiddalthi, lui, ses fils et ses frères étaient douze.
\VS{30}Le vingt-troisième à Machazioth, lui, ses fils et ses frères étaient douze.
\VS{31}Le vingt-quatrième à Romamthi-Ezer, lui, ses fils et ses frères étaient douze.
\Chap{26}
\TextTitle{Les classes des portiers}
\VerseOne{}Et quant aux classes des portiers, il y eut pour les Koréites : Meschélémia, fils de Koré, d'entre les fils d'Asaph.
\VS{2}Les fils de Meschélémia furent Zacharie, le premier-né, Jediaël le second, Zebadia le troisième, Jathniel le quatrième,
\VS{3}Elam le cinquième, Jochanan le sixième et Eljoénaï le septième.
\VS{4}Les fils d’Obed-Edom furent Schemaeja le premier-né, Jozabad le second, Joach le troisième, Sacar le quatrième, Nethaneel le cinquième,
\VS{5}Ammiel le sixième, Issacar le septième, Peulthaï le huitième ; car Dieu l'avait béni.
\VS{6}A Schemaeja, son fils, naquirent des fils qui eurent le commandement sur la maison de leur père, parce qu'ils étaient des hommes forts et vaillants.
\VS{7}Les fils donc de Schemaeja furent Othni, Rephaël, Obed, Elzabad et ses frères, hommes vaillants, Elihu et Semaeja.
\VS{8}Tous ceux-là étaient des fils d' Obed-Edom, eux, leurs fils et leurs frères, étaient des hommes pleins de vigueur et de force pour le service ; ils étaient soixante-deux d'Obed-Edom.
\VS{9}Les fils de Meschélémia avec ses frères, vaillants hommes étaient au nombre de dix-huit.
\VS{10}Les fils de Hosa, d'entre les fils de Merari, furent  Schimri le chef, quoiqu'il ne fût pas l'aîné, néanmoins son père l'établit pour chef ;
\VS{11}Hilkija était le second, Thebalia le troisième, Zacharie le quatrième; tous les fils et frères de Hosa furent treize.
\VS{12}A ces classes de portiers, aux chefs de ces hommes et à leurs frères, fut remise la garde pour le service de la maison de Yahweh.
\VS{13}Ils tirèrent au sort pour chaque porte, autant pour le plus petit que pour le plus grand, selon leurs familles.
\VS{14}Et ainsi, le sort pour la porte vers l'orient échut à Schélémia. Puis on tira au sort pour Zacharie, son fils, qui était un sage conseiller, et la porte du côté du nord lui fut échue par le sort.
\VS{15}Le sort d'Obed-Edom lui échut pour la porte du côté du sud, et la maison des magasins échut à ses fils.
\VS{16}A Schuppim et à Hosa pour la porte vers l'occident, auprès de la porte de Schalléketh, au chemin montant ; une garde étant vis-à-vis de l'autre.
\VS{17}Il y avait vers l'orient six Lévites ; vers le nord, quatre par jour  ; vers le sud, quatre aussi par jour ; et vers la maison des magasins, deux de chaque côté ;
\VS{18}du côté de la banlieue vers l'occident, il y en avait quatre au chemin, et deux vers la banlieue.
\VS{19}Ce sont là les classes des portiers pour les fils des Koréites, et pour les fils de Merari.
\TextTitle{Les Lévites commis sur les trésors du temple}
\VS{20}Ceux-ci aussi étaient Lévites : Achija commis sur les trésors de la maison de Dieu et les trésors des choses consacrées.
\VS{21}Des fils de Laedan, qui étaient d'entre les fils des Guerschonites, du côté de Laedan, d'entre les chefs des maisons paternelles appartenant à Laedan le Guerschonite, Jehiéli.
\VS{22}D'entre les fils de Jehiéli : Zétham et Joël, son frère, commis sur les trésors de la maison de Yahweh.
\VS{23}Pour les Amramites, les Jitseharites, les Hébronites et les  Uziélites,
\VS{24}Schebuel, fils de Guerschom, fils de Moïse, était commis sur les autres trésors.
\VS{25}Et quant à ses frères issus d'Eliézer, dont Rechabia fut fils, dont le fils fut Esaïe, dont le fils fut Joram, dont le fils fut  Zicri, dont le fils fut Schelomith,
\VS{26}c’étaient Schelomith et ses frères qui gardaient tous les trésors des choses saintes que le roi David, les chefs des familles paternelles, les chefs de milliers et de centaines, et les chefs de l'armée avaient consacrées.
\VS{27}C'était le butin de guerre qu'ils avaient consacré, pour l’entretien de la maison de Yahweh.
\VS{28}Tout ce qu'avait consacré Samuel, le voyant, Saül, fils de Kis, Abner, fils de Ner et Joab, fils de Tseruja, toutes les choses consacrées étaient mises sous la main de Schelomith et de ses frères.
\TextTitle{Les magistrats et juges en Israël}
\VS{29}Parmi les Jitseharites, Kenania et ses fils étaient employés aux affaires extérieures sur Israël pour être magistrats et juges.
\VS{30}Quant aux Hébronites, Haschabia et ses frères, hommes vaillants, au nombre de mille sept cents, avaient la surveillance d'Israël de l’autre côté du Jourdain, vers l'occident, pour toute œuvre qui concernait Yahweh, et pour le service du roi.
\VS{31}Quant aux Hébronites, selon leurs générations dans les maisons paternelles, Jerija fut le chef des Hébronites. On fit une recherche au sujet des Hébronites à la quarantième année du règne de David, et on trouva parmi eux à Jaezer de Galaad, des hommes forts et vaillants.
\VS{32}Les frères de Jerija, hommes vaillants, furent deux mille sept cents, issus des maisons paternelles ; et le roi David les établit sur les Rubénites, sur les Gadites, et sur la demi-tribu de Manassé, pour toute œuvre qui concernait Dieu, et pour les affaires du roi.
\Chap{27}
\TextTitle{Les douze chefs de guerre de David}
\VerseOne{}Quant aux fils d'Israël, selon leur dénombrement, il y avait des chefs de maisons paternelles, des chefs de milliers et de centaines, et leurs officiers, qui servaient le roi pour tout ce qui concernait les divisions, leur arrivée et leur départ, mois par mois, pendant tous les mois de l'année, et chaque division était de vingt-quatre mille hommes.
\VS{2}Et Jaschobeam, fils de Zabdiel, présidait sur la première division, pour le premier mois ; et dans sa division il y avait vingt-quatre mille hommes.
\VS{3}Il était des fils de Pérets, chef de tous les capitaines de l'armée du premier mois.
\VS{4}Dodaï, l'Achochite, présidait sur la division du deuxième mois, Mikloth, l’un des chefs de sa division ; et il avait une division de vingt-quatre mille hommes.
\VS{5}Le chef de la troisième armée pour le troisième mois était Benaja, fils de Jehojada, le sacrificateur et le capitaine en chef ; et dans sa division il y avait vingt-quatre mille hommes.
\VS{6}C'est ce Benaja qui était fort entre les trente, et par dessus les trente ; et Ammizadab, son fils, était dans sa division.
\VS{7}Le quatrième pour le quatrième mois était Asaël, frère de Joab, et Zébadia son fils, après lui ; et il y avait dans sa division vingt-quatre mille hommes.
\VS{8}Le cinquième pour le cinquième mois était le capitaine Schamehuth, le Jizrachite ; et dans sa division il y avait vingt-quatre mille hommes.
\VS{9}Le sixième pour le sixième mois était Ira, fils d' Ikkesch le Tekoïte ; et dans sa division il y avait vingt-quatre mille hommes.
\VS{10}Le septième pour le septième mois était Hélets le Pelonite, des fils d'Ephraïm ; et il y avait dans sa division vingt-quatre mille hommes.
\VS{11}Le huitième pour le huitième mois était Sibbecaï le Huschatite, de la famille des Zérachites ; et il y avait dans sa division vingt-quatre mille hommes.
\VS{12}Le neuvième pour le neuvième mois était Abiézer d'Anathoth, des Benjamites ; et il y avait dans sa division vingt-quatre mille hommes.
\VS{13}Le dixième pour le dixième mois était Maharaï de Nethopha, de la famille des Zérachites ; et il y avait dans sa division vingt-quatre mille hommes.
\VS{14}Le onzième pour le onzième mois était Benaja de Pirathon, des fils d'Ephraïm; et il y avait dans sa division vingt-quatre mille hommes.
\VS{15}Le douzième pour le douzième mois était Heldaï de Nethopha, appartenant à Othniel; et il y avait dans sa division vingt-quatre mille hommes.
\TextTitle{Les douze chefs des tribus d’Israël}
\VS{16}Et ceux-ci présidaient sur les tribus d'Israël : Eliézer, fils de Zicri, était le conducteur des Rubénites. Des Siméonites : Schephathia, fils de Maaca.
\VS{17}Des Lévites, Haschabia, fils de Kemuel. De ceux d'Aaron : Tsadok.
\VS{18}De Juda : Elihu, qui était des frères de David. De ceux d'Issacar : Omri, fils de Micaël.
\VS{19}De ceux de Zabulon : Jischemaeja, fils d'Abdias. De ceux de Nephthali : Jerimoth, fils d'Azriel.
\VS{20}Des fils d'Ephraïm : Hosée, fils d'Azazia. De la demi-tribu de Manassé : Joël, fils de Pedaja.
\VS{21}De l'autre demi-tribu de Manassé en Galaad : Jiddo, fils de Zacharie. De ceux de Benjamin : Jaasiel, fils d'Abner.
\VS{22}De ceux de Dan : Azareel, fils de Jerocham. Ce sont là les chefs des tribus d'Israël.
\TextTitle{Dénombrement arrêté par Yahweh}
\VS{23}Mais David ne fit point le dénombrement des Israélites, depuis l'âge de vingt ans et au-dessous ; parce que Yahweh avait dit qu'il multiplierait Israël comme les étoiles du ciel.
\VS{24}Joab, fils de Tseruja, avait bien commencé à en faire le dénombrement, mais il n'acheva pas parce que la colère de Dieu s'était répandue à cause de cela sur Israël ; c'est pourquoi ce dénombrement ne fut point mis parmi les dénombrements enregistrés dans les Chroniques du roi David.
\TextTitle{Les gestionnaires de David}
\VS{25}Or Azmaveth, fils d'Adiel, était commis sur les finances du roi ; mais Jonathan, fils d'Ozias, était commis sur les provisions dans les champs, dans les villes, les villages et les châteaux.
\VS{26}Et Ezri, fils de Kelub, était commis sur ceux qui travaillaient dans la campagne et cultivaient la terre.
\VS{27}Et Schimeï de Rama sur les vignes, et Zabdi de Schepham sur ce qui provenait des vignes, et sur les celliers du vin.
\VS{28}Et Baal-Hanan de Guéder sur les oliviers et sur les figuiers qui étaient à la campagne ; et Joasch sur les celliers à huile.
\VS{29}Schithraï de Saron était commis sur le gros bétail qui paissait en Saron ; Schaphath, fils d'Adlaï, sur le gros bétail qui paissait dans les vallées.
\VS{30}Obil, l'Ismaélite, sur les chameaux; Jechdia de Méronoth, sur les ânesses.
\VS{31}Jaziz, l'Hagarénien, sur les troupeaux du menu bétail. Tous ceux-là avaient la charge des biens qui appartenaient au roi David.
\TextTitle{Les conseillers de David}
\VS{32}Mais Jonathan, oncle de David, était conseiller, homme très intelligent et scribe ; et Jehiel, fils de Hacmoni, était avec les fils du roi.
\VS{33}Achitophel était le conseiller du roi ; et Huschaï, l'Arkien, était l'intime ami du roi.
\VS{34}Après Achitophel était Jehojada, fils de Benaja et Abiathar; et Joab était le chef de l'armée du roi.
\Chap{28}
\TextTitle{Dernières paroles de David, la royauté remise à Salomon\FTNTT{1 Ch. 23:2}}
\VerseOne{}David convoqua à Jérusalem tous les chefs d'Israël, les chefs des tribus, et les chefs des divisions qui servaient le roi ; et les chefs de milliers et de centaines, et ceux qui avaient la charge de tous les biens du roi, et de tout ce qu’il possédait, ses fils avec ses eunuques, et les hommes puissants, et tous les  héros et tous les hommes vaillants.
\VS{2}Puis le roi David se leva sur ses pieds, et dit : Mes frères et mon peuple, écoutez-moi ! J’avais à cœur de bâtir une maison de repos pour l’arche de l'alliance de Yahweh, et pour le marchepied de notre Dieu, et j'ai fait les préparatifs pour la bâtir.
\VS{3}Mais Dieu m'a dit : Tu ne bâtiras point de maison à mon Nom, parce que tu es un homme de guerre, et que tu as répandu beaucoup de sang.
\VS{4}Or comme Yahweh, le Dieu d'Israël, m'a choisi dans toute la maison de mon père pour être roi sur Israël à toujours ; car il a choisi Juda pour conducteur, et de la maison de Juda la maison de mon père, et d'entre les fils de mon père il a pris son plaisir en moi, pour me faire régner sur tout Israël.
\VS{5}Aussi, entre tous mes fils, car Yahweh m'a donné beaucoup de fils, il a choisi Salomon, mon fils, pour le faire asseoir sur le trône du royaume de Yahweh, sur Israël.
\VS{6}Et il m'a dit : Salomon, ton fils, est celui qui bâtira ma maison et mes parvis ; car je me le suis choisi pour fils et je serai pour lui un père.
\VS{7}Et j'affermirai son règne à toujours s'il s'applique à pratiquer mes commandements et à observer mes ordonnances, comme il le fait aujourd'hui.
\VS{8}Maintenant donc, je vous somme en présence de tout Israël, qui est l'assemblée de Yahweh, et devant notre Dieu qui l'entend, que vous ayez à garder et à rechercher diligemment tous les commandements de Yahweh, votre Dieu, afin que vous possédiez ce bon pays, et que vous le fassiez hériter à vos fils après vous, à jamais.
\VS{9}Et toi, Salomon, mon fils, connais le Dieu de ton père, et sers-le avec un cœur droit et une bonne volonté ; car Yahweh sonde tous les cœurs, et connaît toutes les dispositions des pensées. Si tu le cherches, il se laissera trouver par toi ; mais si tu l'abandonnes, il te rejettera pour toujours.
\VS{10}Considère maintenant que Yahweh t'a choisi pour bâtir une maison pour son sanctuaire. Fortifie-toi donc et applique-toi à y travailler.
\VS{11}David donna à Salomon, son fils, le modèle\FTNT{David donna le modèle du temple qu’il  avait reçu de Dieu à Salomon. Beaucoup veulent servir Dieu sans modèle, tandis que d’autres vont chercher des modèles dans le monde (2 R. 16:10-18). Nous devons faire l’œuvre de Dieu uniquement selon le modèle biblique.} du portique, de ses maisons, des chambres du trésor, des chambres hautes, des chambres intérieures et du lieu du propitiatoire.
\VS{12}Il lui donna le modèle de toutes les choses qui lui avaient été inspirées par l'Esprit qui était avec lui, pour les parvis de la maison de Yahweh, pour les chambres d'alentour, pour les trésors de la maison de Yahweh et pour les trésors des choses saintes ;
\VS{13}pour les divisions des sacrificateurs et des Lévites, pour toute l'œuvre du service de la maison de Yahweh, et pour tous les ustensiles du service de la maison de Yahweh.
\VS{14}Il lui donna aussi de l'or, un certain poids, pour les choses qui devaient être d'or, à savoir pour tous les ustensiles de chaque service ; et de l'argent, un certain poids, pour tous les ustensiles d'argent, à savoir pour tous les ustensiles de chaque service.
\VS{15}Le poids des chandeliers d'or, et de leurs lampes d'or, selon le poids de chaque chandelier et de ses lampes ; et le poids des chandeliers d'argent, selon le poids de chaque chandelier et de ses lampes, selon l’usage de chaque chandelier.
\VS{16}Et le poids de l'or pesant ce qu'il fallait pour chaque table des pains de proposition; et de l'argent pour les tables d'argent.
\VS{17}Il lui donna le modèle pour les fourchettes, pour les bassins et pour les calices d’or pur ; le modèle pour les coupes d'or, selon le poids de chaque coupe,  et de l'argent pour les coupes d'argent, selon le poids de chaque coupe ;
\VS{18}et le modèle pour l'autel des parfums en or épuré, avec le poids. Il lui donna encore le modèle du char, des chérubins d’or qui étendent les ailes et qui couvrent l’arche de l'alliance de Yahweh.
\VS{19}C’est par un écrit de sa main, dit-il, que Yahweh m’a donné l'intelligence de tout cela, de tous les ouvrages de ce modèle.
\TextTitle{David demande à Salomon de  bâtir le temple}
\VS{20}C'est pourquoi David dit à Salomon, son fils : Fortifie-toi, prends courage et travaille ; ne crains point et ne t'effraie point. Car Yahweh Dieu, mon Dieu, sera avec toi, il ne te délaissera point, et il ne t'abandonnera point, jusqu'à ce que tu aies achevé tout l'ouvrage du service de la maison de Yahweh.
\VS{21}Et voici, j'ai fait les divisions des sacrificateurs et des Lévites pour tout le service de la maison de Dieu ; et il y a avec toi pour tout cet ouvrage toutes sortes de gens prompts et experts, pour toutes sortes de services ; et les chefs avec tout le peuple seront prêts pour exécuter tout ce que tu diras.
\Chap{29}
\TextTitle{Offrandes volontaires de David et de tout le peuple}
\VerseOne{}Puis le roi David dit à toute l'assemblée : Dieu a choisi un seul de mes fils, à savoir Salomon, qui est encore jeune et délicat, et l'ouvrage est considérable, car ce palais n'est point pour un homme, mais pour Yahweh Dieu.
\VS{2}Et moi, j'ai préparé de toutes mes forces pour la maison de mon Dieu, de l'or pour les choses qui doivent être d'or, de l'argent pour celles qui doivent être d'argent, de l'airain pour celles d'airain, du fer pour celles de fer, du bois pour celles de bois, des pierres d'onyx, et des pierres pour être enchâssées, des pierres d'escarboucle, et des pierres de diverses couleurs, des pierres précieuses de toutes sortes, et du marbre en abondance.
\VS{3}Et outre cela, parce que j'ai une grande affection pour la maison de mon Dieu, je donne pour la maison de mon Dieu, outre toutes les choses que j'ai préparées pour la maison du sanctuaire, l'or et l'argent que j'ai entre mes plus précieux joyaux :
\VS{4}Trois mille talents d'or, de l'or d'Ophir, et sept mille talents d'argent affiné, pour revêtir les murailles de la maison ;
\VS{5}afin qu'il y ait de l'or partout où il faut de l'or, et de l'argent partout où il faut de l'argent ; et pour tout l'ouvrage qui se fera par la main des ouvriers. Or qui est celui d'entre vous qui se disposera volontairement à offrir aujourd'hui libéralement à Yahweh ?
\VS{6}Alors les chefs des maisons paternelles, les chefs des tribus d'Israël, les chefs de milliers et de centaines et les intendants du roi offrirent volontairement.
\VS{7}Ils donnèrent pour le service de la maison de Dieu cinq mille talents et dix mille drachmes d'or, dix mille talents d'argent, dix-huit mille talents d'airain, et cent mille talents de fer.
\VS{8}Ils mirent aussi les pierres que chacun avait, pour le trésor de la maison de Yahweh, entre les mains de Jehiel, le Guerschonite.
\VS{9}Et le peuple offrait avec joie volontairement, car ils offraient de tout leur cœur leurs offrandes volontaires à Yahweh; et David en eut une très grande joie.
\TextTitle{Prières de David}
\VS{10}Puis David bénit Yahweh en présence de toute l'assemblée, et dit : Ô Yahweh, Dieu d'Israël, notre père ! Tu es béni de tout temps et à toujours.
\VS{11}Ô Yahweh ! C’est à toi qu'appartient la magnificence, la puissance, la gloire, l'éternité, et la majesté; car tout ce qui est aux cieux et sur la terre est à toi, ô Yahweh ! Le règne est à toi, et tu t'élèves en souverain au-dessus de toutes choses !
\VS{12}Les richesses et les honneurs viennent de toi, et tu as la domination sur toutes choses ; la force et la puissance sont dans ta main, et il est aussi du pouvoir de ta main d'agrandir et de fortifier toutes choses.
\VS{13}Maintenant donc, ô notre Dieu ! Nous te célébrons et nous louons ton Nom glorieux.
\VS{14}Mais qui suis-je, et qui est mon peuple, que nous ayons assez pour pouvoir t’offrir ces choses volontairement ? Car toutes choses viennent de toi, et les ayant reçues de ta main, nous te les présentons.
\VS{15}Et même nous sommes devant toi des étrangers et des habitants, comme ont été tous nos pères ; et nos jours sont comme l'ombre sur la terre, et il n'y a point d’espérance.
\VS{16}Yahweh, notre Dieu,  toute cette abondance que nous avons préparée pour bâtir une maison à ton saint Nom, est de ta main, et toutes ces choses sont à toi.
\VS{17}Et je sais, ô mon Dieu, que c'est toi qui sondes les cœurs, et que tu prends plaisir à la droiture. C'est pourquoi j'ai volontairement offert d'un cœur droit toutes ces choses, et j'ai vu maintenant avec joie que ton peuple, qui se trouve ici, t'a fait son offrande volontairement.
\VS{18}Ô Yahweh ! Dieu d'Abraham, d'Isaac et d'Israël, nos pères, conserve à toujours dans le cœur de ton peuple, ces dispositions et ces pensées, et affermis leurs cœurs en toi.
\VS{19}Donne aussi un cœur droit à Salomon, mon fils, afin qu'il garde tes commandements, tes préceptes et tes lois, et qu'il fasse tout ce qui est nécessaire et qu'il bâtisse le palais que j'ai préparé.
\TextTitle{Sacrifices en l’honneur de Yahweh ; Salomon oint roi\FTNTT{1 Ch. 23:1 ; 1 R. 2:12 ; 1 R. 1:32-37}}
\VS{20}Après cela, David dit à toute l'assemblée : Bénissez maintenant Yahweh, votre Dieu ! Et toute l'assemblée bénit Yahweh, le Dieu de leurs pères. Ils s'inclinèrent et se prosternèrent devant Yahweh et devant le roi.
\VS{21}Et le lendemain, ils offrirent des sacrifices à Yahweh, et des holocaustes ; à savoir mille veaux, mille moutons, et mille agneaux, avec leurs libations ; et des sacrifices en grand nombre pour tous ceux d'Israël.
\VS{22}Et ils mangèrent et burent ce jour-là devant Yahweh avec une grande joie ; et ils établirent roi pour la seconde fois Salomon, fils de David, et l'oignirent en l'honneur de Yahweh pour être leur conducteur, et Tsadok pour sacrificateur.
\VS{23}Salomon s'assit donc sur le trône de Yahweh pour être roi à la place de David, son père. Il prospéra, car tout Israël lui obéit.
\VS{24}Et tous les chefs et les héros, et même tous les fils du roi David consentirent d'être les sujets du roi Salomon.
\VS{25}Ainsi, Yahweh éleva souverainement Salomon, à la vue de tout Israël, et lui donna une majesté royale telle qu'aucun roi avant lui n'en avait eue en Israël.
\TextTitle{Fin du règne de David ; sa mort\FTNTT{2 S. 5:4-5 ; 1 R. 2:10-12 ; 1 Ch. 3:4}}
\VS{26}David donc, fils d'Isaï, régna sur tout Israël.
\VS{27}Et les jours qu'il régna sur Israël furent quarante ans ; il régna sept ans à Hébron et trente-trois ans à Jérusalem.
\VS{28}Puis il mourut dans une heureuse vieillesse, rassasié de jours, de richesses, et de gloire. Et Salomon, son fils, régna à sa place.
\VS{29}Les actions du roi David, tant les premières que les dernières, sont écrites dans le livre de Samuel le voyant, dans le livre de Nathan le prophète, et dans le livre de Gad le prophète,
\VS{30}avec tout son règne, ses exploits et ce qui se passa de son temps, tant sur Israël que sur tous les royaumes du territoire.
\PPE{}
\end{multicols}

%\clearpage\ShortTitle{2 Chroniques}\BookTitle{2 Chroniques}\BFont
\noindent\hrulefill
{\footnotesize
\textit{
\bigskip
{\centering{}
\\Auteur : Inconnu
\\(Heb. : Hayyamim dibre)
\\Signification : Actes des journées
\\Thème : La grandeur de Juda
\\Date de rédaction : 5\up{ème} siècle av. J.-C.\\}
}
%\bigskip
\textit{
\\Initialement, 1 et 2 Chroniques ne constituaient qu'un seul ouvrage. Ce livre raconte le règne de Salomon, la construction de la maison de Dieu et du palais. Il reprend ensuite l'histoire des royaumes d'Israël et de Juda, du schisme à la captivité babylonienne, mettant en exergue l'instabilité du peuple dont le cœur balançait entre Yahweh et les idoles.\bigskip
}
}
\par\nobreak\noindent\hrulefill
\begin{multicols}{2}
\Chap{1}
\TextTitle{Yahweh élève Salomon qui demande la sagesse\FTNTT{1 R. 2:12 ; 3:4-9 ; 1 Ch. 29:23-25}}
\VerseOne{}Or Salomon, fils de David, se fortifia dans son royaume ; Yahweh, son Dieu, fut avec lui, et l'éleva au plus haut.
\VS{2}Salomon parla à tout Israël, aux chefs de milliers et de centaines, aux juges et à tous les principaux de tout Israël, chefs des pères.
\VS{3}Salomon et toute l'assemblée avec lui allèrent au haut lieu qui était à Gabaon ; car là était la tente d'assignation de Dieu, que Moïse, serviteur de Yahweh, avait faite dans le désert.
\VS{4}Mais David avait fait monter l'arche de Dieu de Kirjath-Jearim au lieu qu'il avait préparé ; car il lui avait dressé une tente à Jérusalem.
\VS{5}L'autel d'airain que Betsaleel, fils d'Uri, fils de Hur, avait fait, était là devant le tabernacle de Yahweh. Et Salomon et l'assemblée y cherchèrent Yahweh\FTNT{Ex. 27:1-8 ; Ex. 36:1-2.}.
\VS{6}Salomon offrit là, devant Yahweh, mille holocaustes, sur l'autel d'airain qui était devant la tente d'assignation.
\VS{7}En cette nuit-là, Dieu apparut à Salomon, et lui dit : Demande ce que tu veux que je te donne.
\VS{8}Et Salomon répondit à Dieu : Tu as usé d'une grande bienveillance envers David, mon père, et tu m'as établi roi à sa place.
\VS{9}Maintenant, ô Yahweh Dieu ! Que ta parole à David, mon père, se confirme ; car tu m'as établi roi sur un peuple nombreux comme la poussière de la terre.
\VS{10}Donne-moi donc maintenant de la sagesse et de l'intelligence, afin que je sache me conduire devant ce peuple ; car qui pourrait juger ton peuple, ce peuple si grand ?
\TextTitle{Yahweh agrée la prière de Salomon et l'exauce\FTNTT{1 R. 3:10-28}}
\VS{11}Et Dieu dit à Salomon : Puisque c'est là ce qui est dans ton cœur, et que tu n'as demandé ni des richesses, ni des biens, ni de la gloire, ni la mort de ceux qui te haïssent, ni même des jours nombreux, mais que tu as demandé pour toi de la sagesse et de l'intelligence, afin de pouvoir juger mon peuple, sur lequel je t'ai établi roi,
\VS{12}la sagesse et l'intelligence te sont données. Je te donnerai aussi des richesses, des biens et de la gloire, comme n'en ont pas eu les rois qui ont été avant toi, et comme il n'en aura aucun après toi.
\VS{13}Puis Salomon s'en retourna à Jérusalem, du haut lieu qui était à Gabaon devant la tente d'assignation ; et il régna sur Israël.
\VS{14}Salomon rassembla des chars et des cavaliers ; il avait quatorze cents chars et douze mille cavaliers ; et il les plaça dans les villes où il tenait ses chars, et auprès du roi, à Jérusalem.
\VS{15}Et le roi fit que l'argent et l'or étaient aussi communs à Jérusalem que les pierres, et les cèdres que les sycomores de la plaine.
\VS{16}Le lieu d'où étaient issus les chevaux de Salomon était l'Egypte ; une caravane de marchands du roi allait les prendre par troupe à un prix convenu.
\VS{17}On faisait monter et sortir d'Egypte un char pour six cents sicles d'argent, et un cheval pour cent cinquante. On en amenait de même par eux pour tous les rois des Héthiens, et pour les rois de Syrie.
\Chap{2}
\TextTitle{La prière de Salomon exaucée\FTNTT{1 R. 5:1-18 ; 7:13,14}}
\VerseOne{}Or Salomon ordonna de bâtir une maison au Nom de Yahweh, ainsi qu'une maison royale.
\VS{2}Et il fit un dénombrement de soixante et dix mille hommes qui portaient les fardeaux, et de quatre vingt mille qui coupaient le bois sur la montagne, et de trois mille six cents qui étaient commis sur eux.
\VS{3}Puis Salomon envoya vers Huram, roi de Tyr, pour lui dire : Fais pour moi comme tu as fait pour David, mon père, à qui tu as envoyé des cèdres, pour se bâtir une maison afin d'y habiter.
\VS{4}Voici, je vais bâtir une maison au Nom de Yahweh, mon Dieu, pour la lui consacrer, pour faire brûler devant lui le parfum des aromates, pour présenter continuellement devant lui les pains de proposition, et pour offrir les holocaustes du matin et du soir, des sabbats, des nouvelles lunes, et des fêtes de Yahweh, notre Dieu, ce qui est perpétuel en Israël.
\VS{5}La maison que je vais bâtir sera grande ; car notre Dieu est plus grand que tous les dieux.
\VS{6}Mais qui aurait le pouvoir de lui bâtir une maison, puisque les cieux et les cieux des cieux ne sauraient le contenir ? Et qui suis-je pour lui bâtir une maison, si ce n'est pour faire brûler des parfums devant sa face ?
\VS{7}Maintenant, envoie-moi un homme habile pour travailler l'or, l'argent, l'airain et le fer, en écarlate, en cramoisi et en pourpre, sachant faire des sculptures, pour travailler avec les hommes habiles que j'ai avec moi en Juda et à Jérusalem, et que David, mon père, a préparés.
\VS{8}Envoie-moi aussi du Liban du bois de cèdre, de cyprès et de santal ; car je sais que tes serviteurs savent couper les bois du Liban. Voici, mes serviteurs seront avec les tiens.
\VS{9}Qu'on me prépare du bois en grande quantité ; car la maison que je vais bâtir sera grande et magnifique.
\VS{10}Et je donnerai à tes serviteurs qui couperont, qui abattront les bois, vingt mille cors de froment foulé, vingt mille cors d'orge, vingt mille baths de vin, et vingt mille baths d'huile.
\VS{11}Huram, roi de Tyr, répondit dans un écrit qu'il envoya à Salomon : C'est parce que Yahweh aime son peuple qu'il t'a établi roi sur eux.
\VS{12}Et Huram dit : Béni soit Yahweh, le Dieu d'Israël, qui a fait les cieux et la terre, de ce qu'il a donné au roi David un fils sage, prudent et intelligent, qui va bâtir une maison à Yahweh, et une maison royale !
\VS{13}Je t'envoie donc un homme habile et intelligent, Huram-Abi,
\VS{14}fils d'une femme d'entre les filles de Dan, et d'un père tyrien. Il sait travailler l'or, l'argent, l'airain et le fer, les pierres et le bois, en écarlate, en pourpre, en fin lin et en cramoisi ; il sait faire toutes sortes de sculptures et imaginer toutes sortes d'objets d'art qu'on lui donne à faire. Il travaillera avec tes hommes habiles et avec les hommes habiles de mon seigneur David, ton père.
\VS{15}Et maintenant, que mon seigneur envoie à ses serviteurs le froment, l'orge, l'huile et le vin comme il l'a dit.
\VS{16}Et nous couperons des bois du Liban autant que tu en auras besoin, et nous te les amènerons en radeaux, par la mer, jusqu'à Japho, et tu les feras monter à Jérusalem.
\VS{17}Alors Salomon compta tous les hommes étrangers qui étaient au pays d'Israël, d'après le dénombrement que David, son père, en avait fait. On en trouva cent cinquante-trois mille six cents.
\VS{18}Et il en établit soixante-dix mille qui portaient des fardeaux, quatre-vingt mille qui taillaient les pierres dans la montagne, et trois mille six cents surveillants pour faire travailler le peuple.
\Chap{3}
\TextTitle{Salomon commence la construction du temple\FTNTT{1 R. 6:1}}
\VerseOne{}Salomon commença donc à bâtir la maison de Yahweh à Jérusalem, sur la montagne de Morija, qui avait été indiquée à David, son père, au lieu même que David avait préparé dans l'aire d'Ornan, le Jébusien.
\VS{2}Il commença à bâtir, le second jour du second mois, la quatrième année de son règne.
\TextTitle{Les matériaux du temple et les dimensions\FTNTT{1 R. 6:2-38 ; 7:13-22}}
\VS{3}Or voici les fondements fixés par Salomon pour bâtir la maison de Dieu : La longueur, en coudées de l'ancienne mesure, était de soixante coudées, et la largeur de vingt coudées.
\VS{4}Le portique qui était sur le devant, et dont la longueur répondait à la largeur de la maison, avait vingt coudées, et cent vingt de hauteur. Il le revêtit intérieurement d'or pur.
\VS{5}Et il recouvrit la grande maison de bois de cyprès ; il la revêtit d'or fin, et y fit mettre des palmes et des chaînettes.
\VS{6}Il revêtit la maison de pierres précieuses, pour l'ornement ; et l'or était de l'or de Parvaïm.
\VS{7}Il revêtit d'or la maison, les poutres, les seuils, les parois et les portes ; et il fit sculpter des chérubins sur les parois.
\VS{8}Il fit le Saint des saints, dont la longueur était de vingt coudées, selon la largeur de la maison, et la largeur de vingt coudées ; et il le couvrit d'or fin, pour une valeur de six cents talents.
\VS{9}Et le poids des clous montait à cinquante sicles d'or. Il revêtit aussi d'or les chambres hautes.
\VS{10}Il fit dans le Saint des saints deux chérubins sculptés, et on les couvrit d'or ;
\VS{11}La longueur des ailes des chérubins était de vingt coudées. L'aile du premier, longue de cinq coudées, touchait la paroi de la maison, et l'autre aile, longue de cinq coudées, touchait une aile de l'autre chérubin.
\VS{12}Et une aile de l'autre chérubin, longue de cinq coudées, touchait la paroi de la maison ; et l'autre aile longue de cinq coudées, joignait l'aile de l'autre chérubin.
\VS{13}Les ailes étendues de ces chérubins faisaient vingt coudées. Ils se tenaient debout sur leurs pieds, leurs faces tournées vers la maison.
\VS{14}Il fit le voile de pourpre, d'écarlate, de cramoisi et de fin lin: Et il y représenta par dessus des chérubins.
\VS{15}Devant la maison, il fit deux colonnes de trente-cinq coudées de hauteur, et le chapiteau sur leur sommet était de cinq coudées.
\VS{16}Il fit des chaînes dans le sanctuaire ; et il en mit sur le sommet des colonnes ; et il fit cent grenades qu'il mit aux chaînes.
\VS{17}Il dressa les colonnes sur le devant du temple, l'une à droite, et l'autre à gauche ; il appela celle de droite Jakin, et celle de gauche Boaz.
\Chap{4}
\TextTitle{L'autel d'airain, la mer de fonte et les ustensiles du temple\FTNTT{1 R. 7:23-50}}
\VerseOne{}Il fit aussi un autel d'airain\FTNT{Voir l'annexe « Le temple de Salomon - extérieur »} long de vingt coudées, large de vingt coudées, et haut de dix coudées.
\VS{2}Il fit la mer de fonte de dix coudées d'un bord à l'autre, ronde tout autour, et haute de cinq coudées, et une circonférence que mesurait un cordon de trente coudées.
\VS{3}Des figures de bœufs l'entouraient en dessous, dix par coudée, faisant tout le tour de la mer ; il y avait deux rangées de bœufs fondus avec elle en une seule pièce.
\VS{4}Elle était posée sur douze bœufs, dont trois tournés vers le nord, trois tournés vers l'occident, trois tournés vers le sud, et trois tournés vers l'orient. La mer était sur eux, et toute la partie postérieure de leur corps était en dedans.
\VS{5}Son épaisseur était d'une paume ; et son bord était comme le bord d'une coupe en fleur de lis. Elle avait une contenance de trois mille baths\FTNT{Ex 25 ; Ex 27.}.
\VS{6}Il fit aussi dix cuves, et en mit cinq à droite et cinq à gauche, pour servir à la purification. On y lavait ce qui appartenait aux holocaustes, et la mer servait aux sacrificateurs pour s'y laver.
\VS{7}Il fit dix chandeliers d'or, d'après l'ordonnance, et les mit dans le temple, cinq à droite et cinq à gauche.
\VS{8}Il fit aussi dix tables, et il les mit dans le temple, cinq à droite et cinq à gauche. Il fit cent coupes d'or.
\VS{9}Il fit encore le parvis des sacrificateurs, le grand parvis et des portes pour ce parvis, et couvrit d'airain ces portes.
\VS{10}Il mit la mer du côté droit, vers l'orient, face au sud-est.
\VS{11}Et Huram fit les cuves, les pelles et les bassins. Huram acheva de faire l'ouvrage qu'il faisait pour le roi Salomon dans la maison de Dieu :
\VS{12}Deux colonnes, les bourrelets et les deux chapiteaux sur le sommet des colonnes ; les deux maillages pour couvrir les deux bourrelets des chapiteaux sur le sommet des colonnes ;
\VS{13}et les quatre cents grenades pour les deux maillages, deux rangs de grenades à chaque maille, pour couvrir les deux bourrelets des chapiteaux sur le sommet des colonnes.
\VS{14}Il fit aussi les bases, et il fit les cuves sur les bases ;
\VS{15}la mer et les douze bœufs sous elle ;
\VS{16}les pots, les pelles et les fourchettes et tous leurs ustensiles ; Huram-Abi les fit au roi Salomon, pour la maison de Yahweh, en airain poli.
\VS{17}Le roi les fit fondre dans la plaine du Jourdain, dans une terre grasse, entre Succoth et Tseréda.
\VS{18}Et Salomon fit tous ces ustensiles en si grand nombre qu'on ne rechercha point le poids de l'airain.
\VS{19}Salomon fit encore tous les ustensiles\FTNT{Voir l'annexe « Le temple de Salomon - intérieur »} qui étaient dans la maison de Yahweh : L'autel d'or, et les tables sur lesquelles on mettait le pain de proposition ;
\VS{20}les chandeliers et leurs lampes d'or fin, qu'on devait allumer devant le sanctuaire, selon l'ordonnance ;
\VS{21}les fleurs, les lampes, et les mouchettes d'or, d'un or parfaitement pur ;
\VS{22}et les mouchettes, les bassins, les tasses et les encensoirs d'or fin. Quant à l'entrée de la maison, les portes intérieures conduisant dans le Saint des saints, et les portes de la maison pour entrer au temple étaient d'or.
\Chap{5}
\TextTitle{L'arche dans le sanctuaire, Yahweh manifeste sa gloire\FTNTT{1 R. 7:51-8:11}}
\VerseOne{}Ainsi fut achevé tout l'ouvrage que Salomon fit pour la maison de Yahweh. Puis Salomon fit apporter ce que David, son père, avait consacré : L'argent, l'or et tous les ustensiles ; et il les mit dans les trésors de la maison de Dieu.
\VS{2}Alors Salomon assembla à Jérusalem les anciens d'Israël, et tous les chefs des tribus, les chefs des pères des fils d'Israël, pour transporter de la ville de David, qui est Sion, l'arche de l'alliance de Yahweh.
\VS{3}Et tous les hommes d'Israël s'assemblèrent auprès du roi pour la fête ; c'était le septième mois.
\VS{4}Tous les anciens d'Israël vinrent, et les Lévites portèrent l'arche.
\VS{5}Ils transportèrent l'arche, la tente d'assignation, et tous les ustensiles sacrés qui étaient dans la tente ; les sacrificateurs et les Lévites les emportèrent.
\VS{6}Or le roi Salomon et toute l'assemblée d'Israël réunie auprès de lui étaient devant l'arche, sacrifiant du menu et du gros bétail en si grand nombre qu'on ne pouvait ni dénombrer ni compter.
\VS{7}Les sacrificateurs portèrent l'arche de l'alliance de Yahweh à sa place, dans le sanctuaire de la maison, dans le Saint des saints, sous les ailes des chérubins.
\VS{8}Les chérubins étendaient les ailes sur l'endroit où devait être l'arche, et les chérubins couvraient l'arche et ses barres par-dessus.
\VS{9}Les barres avaient une longueur telle que leurs extrémités se voyaient en avant de l'arche, devant le sanctuaire ; mais elles ne se voyaient point du dehors. Et l'arche a été là jusqu'à ce jour.
\VS{10}Il n'y avait dans l'arche que les deux tables que Moïse y avait mises en Horeb, quand Yahweh traita alliance avec les enfants d'Israël à leur sortie d'Egypte.
\VS{11}Or il arriva que comme les sacrificateurs sortaient du lieu saint (car tous les sacrificateurs présents s'étaient sanctifiés, sans observer l'ordre des classes),
\VS{12}etque tous les Lévites qui étaient chantres, Asaph, Héman, Jeduthun, leurs fils et leurs frères, vêtus de fin lin, avec des cymbales, des luths et des harpes, se tenaient à l'orient de l'autel ; et il y avait avec eux cent vingt sacrificateurs sonnant des trompettes.
\VS{13}Il arriva, dis-je, que comme un seul homme, ceux qui sonnaient des trompettes et ceux qui chantaient firent entendre leur voix d'un même accord, pour célébrer et pour louer Yahweh, et firent retentir le son des trompettes, des cymbales et d'autres instruments de musique, et ils célébrèrent Yahweh, en disant : Car il est bon, car sa miséricorde demeure à toujours\FTNT{Jé. 33:11 ; Ps. 118:29 ;  Ps. 136} ! Il arriva que la maison de Yahweh fut remplie d'une nuée.
\VS{14}Les sacrificateurs ne purent s'y tenir pour faire le service, à cause de la nuée ; car la gloire de Yahweh remplissait la maison de Dieu.
\Chap{6}
\TextTitle{Salomon s'adresse à l'assemblée d'Israël\FTNTT{1 R. 8:12-21}}
\VerseOne{}Alors Salomon dit : Yahweh a dit qu'il habiterait dans l'obscurité\FTNT{Nous avons ici une prophétie concernant la venue du Messie. Dieu, qui est lumière, a accepté d'habiter dans les ténèbres afin de nous sauver (Mt. 4:16 ; Jn. 1:5).}.
\VS{2}Et moi, j'ai bâti une maison qui sera ta demeure, et un domicile afin que tu y résides à toujours !
\VS{3}Puis le roi tourna son visage, et bénit toute l'assemblée d'Israël ; et toute l'assemblée d'Israël était debout.
\VS{4}Et il dit : Béni soit Yahweh, le Dieu d'Israël, qui de sa bouche a parlé à David, mon père, et qui par sa main puissante accomplit ce qu'il avait déclaré en disant :
\VS{5}Depuis le jour où j'ai fait sortir mon peuple du pays d'Egypte, je n'ai point choisi de ville entre toutes les tribus d'Israël pour y bâtir une maison afin que mon Nom y réside, et je n'ai point choisi d'homme pour être chef de mon peuple d'Israël.
\VS{6}Mais j'ai choisi Jérusalem pour que mon Nom y réside, et j'ai choisi David pour qu'il règne sur mon peuple d'Israël.
\VS{7}Or David, mon père, avait à cœur de bâtir une maison au Nom de Yahweh, le Dieu d'Israël.
\VS{8}Mais Yahweh parla à David, mon père : Puisque tu as eu à cœur de bâtir une maison à mon Nom, tu as bien fait d'avoir eu cette intention.
\VS{9}Seulement, ce n'est pas toi qui bâtiras cette maison ; mais ce sera ton fils, qui sortira de tes entrailles, qui bâtira cette maison à mon Nom.
\VS{10}Yahweh a accompli la parole qu'il avait déclarée ; j'ai succédé à David, mon père, et je me suis assis sur le trône d'Israël, comme Yahweh l'avait dit, et j'ai bâti cette maison au Nom de Yahweh, le Dieu d'Israël.
\VS{11}J'y ai mis l'arche où est l'alliance de Yahweh, qu'il traita avec les enfants d'Israël.
\TextTitle{Prière de Salomon\FTNTT{1 R. 8:22-61}}
\VS{12}Puis il se plaça devant l'autel de Yahweh, en face de toute l'assemblée d'Israël, et il étendit ses mains.
\VS{13}Car Salomon avait fait une tribune d'airain, et il l'avait mise au milieu du grand parvis ; elle était longue de cinq coudées, large de cinq coudées, et haute de trois coudées. Il s'y plaça, se mit à genoux en face de toute l'assemblée d'Israël, et étendant ses mains vers les cieux, il dit :
\VS{14}Ô Yahweh, Dieu d'Israël ! Il n'y a ni dans les cieux ni sur la terre de Dieu semblable à toi, qui gardes l'alliance et la miséricorde envers tes serviteurs qui marchent de tout leur cœur devant ta face.
\VS{15}Toi qui as tenu parole à ton serviteur David, mon père. Ce que tu lui avais promis, et ce que tu as déclaré de ta bouche, tu l'as accompli de ta main puissante, comme il paraît aujourd'hui.
\VS{16}Maintenant, ô Yahweh, Dieu d'Israël ! Tiens la parole que tu as faite à ton serviteur David, mon père, en disant : Tu ne manqueras jamais devant moi d'un successeur assis sur le trône d'Israël, pourvu que tes fils prennent garde à leur voie pour marcher dans ma loi, comme tu as marché devant ma face.
\VS{17}Et maintenant, ô Yahweh, Dieu d'Israël ! Que ta parole, que tu as déclarée à David, ton serviteur, soit confirmée !
\VS{18}Mais Dieu habiterait-il véritablement sur la terre avec les hommes ? Voici, les cieux, même les cieux des cieux, ne peuvent te contenir, combien moins cette maison que j'ai bâtie !
\VS{19}Toutefois, ô Yahweh, mon Dieu, aie égard à la prière de ton serviteur et à sa supplication, pour écouter le cri et la prière que ton serviteur t'adresse.
\VS{20}Que tes yeux soient ouverts jour et nuit sur cette maison, sur le lieu où tu as promis de mettre ton Nom ! Ecoute la prière que ton serviteur te fait en ce lieu.
\VS{21}Exauce les supplications de ton serviteur et de ton peuple d'Israël, quand ils prieront en ce lieu. Exauce des cieux, du lieu de ta demeure ; exauce et pardonne !
\VS{22}Si quelqu'un pèche contre son prochain, et qu'on lui impose un serment pour le faire jurer, et qu'il vient prêter serment devant ton autel, dans cette maison ;
\VS{23}écoute-le des cieux, agis et juge tes serviteurs, en donnant au méchant son salaire, et fais retomber sa conduite sur sa tête, en justifiant le juste, et lui rendant selon sa justice.
\VS{24}Quand ton peuple d'Israël sera battu par l'ennemi, pour avoir péché contre toi ; s'ils retournent à toi, s'ils donnent gloire à ton Nom, s'ils t'adressent dans cette maison des prières et des supplications ;
\VS{25}toi, exauce-les des cieux, et pardonne le péché de ton peuple d'Israël, et ramène-les dans la terre que tu leur as donnée à eux et à leurs pères.
\VS{26}Quand les cieux seront fermés, et qu'il n'y aura point de pluie, parce qu'ils auront péché contre toi ; s'ils prient en ce lieu, s'ils donnent gloire à ton Nom, et s'ils se détournent de leurs péchés, parce que tu les auras affligés ;
\VS{27}toi, exauce-les des cieux, et pardonne le péché de tes serviteurs et de ton peuple d'Israël, après que tu leur auras enseigné le bon chemin, par lequel ils doivent marcher ; et envoie de la pluie sur la terre que tu as donnée en héritage à ton peuple.
\VS{28}Quand il y aura dans le pays la famine ou la peste, quand il y aura la rouille, la nielle, les sauterelles d'une espèce ou d'une autre, quand les ennemis les assiégeront dans leur pays, dans leurs portes, ou qu'il y aura un fléau, une maladie quelconque ;
\VS{29}si un homme, si tout ton peuple d'Israël fait entendre des prières et des supplications, et que chacun reconnaît sa plaie et sa douleur, et étend ses mains vers cette maison ;
\VS{30}exauce-le des cieux, du lieu de ta demeure, et pardonne. Rends à chacun selon toutes ses voies, toi qui connais leur cœur ; car seul tu connais le cœur des fils des hommes ;
\VS{31}afin qu'ils te craignent, pour marcher dans tes voies, tout le temps qu'ils vivront sur la terre que tu as donnée à nos pères.
\VS{32}Et l'étranger, qui ne sera pas de ton peuple d'Israël, mais qui viendra d'un pays éloigné, à cause de ton grand Nom, de ta main puissante, et de ton bras étendu ; quand il viendra prier dans cette maison,
\VS{33}exauce-le des cieux, du lieu de ta demeure, et accorde tout ce que cet étranger réclamera de toi ; afin que tous les peuples de la terre connaissent ton Nom pour te craindre comme ton peuple d'Israël, et sachent que ton Nom est invoqué sur cette maison que j'ai bâtie.
\VS{34}Quand ton peuple sortira en guerre contre ses ennemis, par la voie par laquelle tu l'auras envoyé ; s'ils te prient, en regardant vers cette ville que tu as choisie, et vers cette maison que j'ai bâtie à ton Nom,
\VS{35}exauce des cieux leur prière et leur supplication, et fais-leur droit.
\VS{36}Quand ils pécheront contre toi, car il n'y a point d'homme qui ne pèche, et qu'irrité contre eux, tu les auras livrés à leurs ennemis, et que ceux qui les auront pris les auront emmenés captifs en quelque pays, soit éloigné soit proche ;
\VS{37}si dans le pays où ils seront captifs, ils rentrent en eux-mêmes et s'ils se repentent, s'ils t'adressent des supplications dans le pays de leur captivité, en disant : Nous avons péché, nous avons commis l'iniquité, nous avons agi méchamment !
\VS{38}S'ils retournent à toi de tout leur cœur et de toute leur âme, dans le pays de leur captivité où ils ont été emmenés captifs, et s'ils t'adressent des prières, les regards tournés vers leur pays que tu as donné à leurs pères, vers cette ville que tu as choisie, et vers cette maison que j'ai bâtie à ton Nom ;
\VS{39}exauce des cieux, du lieu de ta demeure, leurs prières et leurs supplications, et fais-leur droit ; pardonne à ton peuple qui aura péché contre toi !
\VS{40}Maintenant, ô mon Dieu, que tes yeux soient ouverts et que tes oreilles soient attentives à la prière qu'on te fera en ce lieu !
\VS{41}Et maintenant, Yahweh Dieu ! Lève-toi, viens au lieu de ton repos, toi et l'arche de ta puissance. Yahweh Dieu, que tes sacrificateurs soient revêtus du salut, et que tes bien-aimés se réjouissent du bien que tu leur fais !
\VS{42}Yahweh Dieu, ne repousse la face pas ton oint ; souviens-toi des grâces accordées à David, ton serviteur.
\Chap{7}
\TextTitle{Yahweh répond par le feu : Sa gloire remplit la maison}
\VerseOne{}Lorsque Salomon eut achevé de prier, le feu descendit du ciel et consuma l'holocauste et les sacrifices\FTNT{Lé 9:24 ; 1 R 18:38.} ; et la gloire de Yahweh remplit la maison.
\VS{2}Les sacrificateurs ne pouvaient entrer dans la maison de Yahweh, parce que la gloire de Yahweh avait rempli la maison de Yahweh.
\VS{3}Tous les enfants d'Israël virent descendre le feu et la gloire de Yahweh sur la maison ; et ils se courbèrent, le visage contre terre, sur le pavé, se prosternèrent et louèrent Yahweh, en disant : Car il est bon, car sa miséricorde demeure éternellement !
\TextTitle{Salomon et le peuple offrent des sacrifices à Yahweh\FTNTT{1 R. 8:62-66}}
\VS{4}Or le roi et tout le peuple offraient des sacrifices devant Yahweh.
\VS{5}Le roi Salomon offrit un sacrifice de vingt-deux mille bœufs, et cent vingt mille brebis. Ainsi, le roi et tout le peuple firent la dédicace de la maison de Dieu.
\VS{6}Les sacrificateurs se tenaient à leurs fonctions, ainsi que les Lévites, avec les instruments de musique de Yahweh, que le roi David avait faits pour louer Yahweh en disant : Car sa miséricorde demeure éternellement ; ayant les Psaumes de David entre leurs mains. Et les sacrificateurs sonnaient des trompettes vis-à-vis d'eux, et tout Israël se tenait debout.
\VS{7}Salomon consacra le milieu du parvis, qui est devant la maison de Yahweh ; car il offrit là les holocaustes et les graisses des sacrifices d'offrande de paix\FTNT{Voir commentaire en Lé. 3:1.}, parce que l'autel d'airain que Salomon avait fait ne pouvait contenir les holocaustes, les offrandes et les graisses.
\VS{8}Ainsi Salomon célébra, en ce temps-là, la fête pendant sept jours, avec tout Israël. Il y avait une grande multitude, venue depuis l'entrée d'Hamath jusqu'au torrent d'Egypte.
\VS{9}Le huitième jour, ils firent une assemblée solennelle ; car ils firent la dédicace de l'autel pendant sept jours, et la fête pendant sept jours.
\VS{10}Le vingt-troisième jour du septième mois, il laissa aller le peuple dans ses tentes, se réjouissant et ayant le cœur plein de joie, à cause du bien que Yahweh avait fait à David, à Salomon, et à Israël, son peuple.
\TextTitle{Yahweh apparaît à Salomon\FTNTT{1 R. 9:1-9}}
\VS{11}Salomon acheva donc la maison de Yahweh et la maison du roi ; et Salomon réussit dans tout ce qui lui vint à cœur de faire dans la maison de Yahweh et dans sa maison.
\VS{12}Yahweh apparut à Salomon pendant la nuit, et lui dit : J'exauce ta prière, et je choisis ce lieu comme une maison de sacrifices.
\VS{13}Quand je fermerai les cieux, et qu'il n'y aura point de pluie, et quand j'ordonnerai aux sauterelles de consumer le pays, et quand j'enverrai la peste parmi mon peuple ;
\VS{14}si mon peuple, sur lequel mon Nom est invoqué, s'humilie, prie, et cherche ma face, et s'il se détourne de ses mauvaises voies, alors je l'exaucerai des cieux, je pardonnerai ses péchés, et je guérirai son pays.
\VS{15}Mes yeux seront désormais ouverts, et mes oreilles seront attentives à la prière faite en ce lieu.
\VS{16}Maintenant je choisis et je sanctifie cette maison, afin que mon Nom y soit à toujours ; mes yeux et mon cœur seront toujours là.
\VS{17}Et toi, si tu marches devant moi comme David, ton père, a marché, faisant tout ce que je t'ai ordonné, et si tu gardes mes lois et mes ordonnances,
\VS{18}j'affermirai le trône de ton royaume, comme je l'ai déclaré à David, ton père, en disant : Il ne te manquera point de successeur qui règne en Israël.
\VS{19}Mais si vous vous détournez, et si vous abandonnez mes lois et mes commandements que je vous ai prescrits, et si vous allez servir d'autres dieux et vous prosterner devant eux,
\VS{20}je vous arracherai de mon pays que je vous ai donné, je rejetterai loin de moi cette maison que j'ai consacrée à mon Nom, et j'en ferai un sujet de sarcasmes et de moqueries parmi tous les peuples.
\VS{21}Et quiconque passera près de cette maison qui aura été élevée, sera dans l'étonnement et dira : Pourquoi Yahweh a-t-il ainsi traité ce pays et cette maison?
\VS{22}Et on répondra : Parce qu'ils ont abandonné Yahweh, le Dieu de leurs pères, qui les a fait sortir du pays d'Egypte, et qu'ils se sont attachés à d'autres dieux, et qu'ils se sont prosternés devant eux, et les ont servis ; à cause de cela, il a fait venir sur eux tous ces maux.
\Chap{8}
\TextTitle{Les réalisations de Salomon\FTNTT{1 R. 9:15-28 ; 10:26-29}}
\VerseOne{}Au bout de vingt ans, pendant lesquels Salomon bâtit la maison de Yahweh et sa propre maison,
\VS{2}il bâtit les villes que Huram lui avait données et y fit habiter les enfants d'Israël.
\VS{3}Puis Salomon marcha contre Hamath de Tsoba, et la conquit.
\VS{4}Il bâtit Thadmor au désert, et toutes les villes servant de magasins qu'il bâtit dans le pays de Hamath.
\VS{5}Il bâtit Beth-Horon la haute, et Beth-Horon la basse, villes fortes de murailles, de portes et de barres ;
\VS{6}Baalath, et toutes les villes servant de magasins qu'avait Salomon, toutes les villes pour les chars, les villes pour la cavalerie, et tout ce que Salomon prit plaisir à bâtir à Jérusalem, au Liban, et dans tout le pays de sa domination.
\VS{7}Tout le peuple qui était resté des Héthiens, des Amoréens, des Phéréziens, des Héviens et des Jébusiens, qui n'étaient point d'Israël ;
\VS{8}leurs descendants, qui étaient restés après eux dans le pays, et que les enfants d'Israël n'avaient pas détruits, Salomon les leva comme des gens de corvée jusqu'à ce jour.
\VS{9}Salomon n'employa comme esclave pour ses travaux aucun des fils d'Israël ; car ils étaient des hommes de guerre, les chefs de ses officiers, les chefs de ses chars et de ses hommes d'armes.
\VS{10}Voici le nombre des chefs de ceux qui étaient préposés aux travaux du roi Salomon : Ils étaient deux cent cinquante, ayant autorité sur le peuple.
\VS{11}Salomon fit monter la fille de Pharaon de la cité de David dans la maison qu'il lui avait bâtie ; car il dit : Ma femme n'habitera point dans la maison de David, roi d'Israël, parce que les lieux où l'arche de Yahweh est entrée sont saints.
\VS{12}Alors Salomon offrit des holocaustes à Yahweh, sur l'autel de Yahweh qu'il avait bâti devant le portique.
\VS{13}Il offrait chaque jour ce qui était prescrit par Moïse pour les sabbats, pour les nouvelles lunes, et pour les fêtes, trois fois l'année, à la fête des pains sans levain, à la fête des semaines, et à la fête des tabernacles\FTNT{Ex. 14:17 ; Lé. 23:1-44.}.
\VS{14}Il établit, selon l'ordonnance de David, son père, les classes des sacrificateurs selon leur fonction, et les Lévites selon leurs charges, pour célébrer Yahweh et pour faire, jour par jour, le service en présence des sacrificateurs ; et les portiers, selon leurs classes, à chaque porte ; car tel était le commandement de David, homme de Dieu.
\VS{15}Et on ne s'écarta pas du commandement du roi à l'égard des Sacrificateurs et des Lévites, en aucune chose, ni à l'égard les trésors.
\VS{16}Ainsi fut préparé tout l'ouvrage de Salomon, jusqu'au jour de la fondation de la maison de Yahweh et jusqu'à ce qu'elle fut terminée. La maison de Yahweh fut donc achevée.
\VS{17}Alors Salomon alla à Etsjon-Guéber et à Eloth, sur le rivage de la mer, dans le pays d'Edom.
\VS{18}Et Huram lui envoya, sous la conduite de ses serviteurs, des navires et des serviteurs connaissant la mer. Ils allèrent avec les serviteurs de Salomon à Ophir, et ils y prirent quatre cent cinquante talents d'or, qu'ils apportèrent au roi Salomon.
\Chap{9}
\TextTitle{La reine de Séba chez Salomon\FTNTT{1 R. 10:1-13}}
\VerseOne{}Or la reine de Séba, ayant appris la renommée de Salomon, vint à Jérusalem pour éprouver Salomon par des énigmes. Elle avait une suite très nombreuse, et des chameaux portant des aromates, de l'or en grande quantité et des pierres précieuses. Elle vint auprès de Salomon, et elle lui parla de tout ce qu'elle avait dans le cœur.
\VS{2}Salomon lui expliqua tout ce qu'elle lui proposa ; il n'y eut rien que Salomon n'entendît et qu'il ne sût lui expliquer.
\VS{3}Alors, la reine de Séba vit toute la sagesse de Salomon, et la maison qu'il avait bâtie,
\VS{4}les mets de sa table, la demeure de ses serviteurs, l'ordre de service et les vêtements de ceux qui le servaient, ses échansons et leurs vêtements, et les marches par où l'on montait à la maison de Yahweh, et elle fut toute ravie hors d'elle-même.
\VS{5}Elle parla ainsi au roi : Ce que j'ai entendu dire dans mon pays de tes actions et de ta sagesse était donc vrai !
\VS{6}Je ne croyais pas ce qu'on en disait avant d'être venue et que mes yeux ne l'aient vu ; et voici, on ne m'avait pas rapporté la moitié de la grandeur de ta sagesse ; tu surpasses la rumeur que j'avais entendue.
\VS{7}Heureux tes gens ! Heureux tes serviteurs qui se tiennent continuellement devant toi, et qui entendent ta sagesse !
\VS{8}Béni soit Yahweh, ton Dieu, qui a pris plaisir en toi pour te placer sur son trône comme roi pour Yahweh, ton Dieu ! C'est parce que ton Dieu aime Israël et veut le faire subsister à jamais, qu'il t'a établi roi sur eux pour faire droit et justice.
\VS{9}Puis elle donna au roi cent vingt talents d'or, une très grande quantité d'aromates, et des pierres précieuses ; et il n'y eut plus d'aromates tels que ceux que la reine de Séba donna au roi Salomon.
\VS{10}Les serviteurs de Huram et les serviteurs de Salomon, qui amenèrent de l'or d'Ophir, amenèrent aussi du bois de santal et des pierres précieuses.
\VS{11}Le roi fit de ce bois de santal les chemins qui allaient à la maison de Yahweh et à la maison du roi, et des harpes et des luths pour les chantres. On n'en avait point vu auparavant de semblable dans le pays de Juda.
\VS{12}Le roi Salomon donna à la reine de Séba tout ce qu'elle désira, ce qu'elle demanda, plus qu'elle n'avait apporté au roi ; et elle s'en retourna, revint dans son pays, elle et ses serviteurs.
\TextTitle{Les richesses de Salomon\FTNTT{cp. 1 R. 4:1-34}}
\VS{13}Le poids de l'or qui arrivait à Salomon chaque année était de six cent soixante-six talents d'or,
\VS{14}outre ce qu'il retirait des négociants et des marchands qui en apportaient, et de tous les rois d'Arabie et des gouverneurs de ces pays-là, qui apportaient de l'or et de l'argent à Salomon.
\VS{15}Le roi Salomon fit deux cents grands boucliers d'or battu, employant six cents sicles d'or battu pour chaque bouclier ;
\VS{16}et trois cents autres boucliers plus petits d'or battu, employant trois cents sicles d'or pour chaque bouclier ; et le roi les mit dans la maison de la forêt du Liban.
\VS{17}Le roi fit aussi un grand trône d'ivoire, qu'il couvrit d'or pur.
\VS{18}Ce trône avait six marches et un marchepied d'or qui était accolé au trône ; et il avait des accoudoirs de l'un et de l'autre côté du siège ; et deux lions se tenaient auprès des accoudoirs.
\VS{19}Douze lions se tenaient là sur les six marches de part et d'autre. Rien de pareil n'avait été fait pour aucun royaume.
\VS{20}Et toutes les coupes à boire du roi Salomon étaient d'or, et toute la vaisselle de la maison de la forêt du Liban était d'or pur ; rien n'était d'argent ; on n'en faisait aucun cas du temps de Salomon.
\VS{21}Car les navires du roi allaient à Tarsis avec les serviteurs de Huram ; et une fois tous les trois ans arrivaient les navires de Tarsis, apportant de l'or, de l'argent, des dents d'éléphants, des singes et des paons.
\VS{22}Le roi Salomon fut plus grand que tous les rois de la terre, tant en richesses qu'en sagesse.
\VS{23}Tous les rois de la terre cherchaient à voir la face de Salomon, pour écouter la sagesse que Dieu avait mise dans son cœur.
\VS{24}Et chacun d'eux apportait son présent : Des ustensiles d'argent, des ustensiles d'or, des vêtements, des armes, des aromates, des chevaux et des mulets, et il en était ainsi année après année.
\VS{25}Salomon avait quatre mille écuries pour ses chevaux, avec des chars ; et douze mille cavaliers qu'il plaça dans les villes où il avait des chars et auprès du roi à Jérusalem.
\VS{26}Il dominait sur tous les rois depuis le fleuve jusqu'au pays des Philistins, et jusqu'à la frontière d'Egypte.
\VS{27}Et le roi fit que l'argent était aussi commun à Jérusalem que les pierres, et les cèdres aussi nombreux que les sycomores qui sont dans les plaines.
\VS{28}On tirait des chevaux pour Salomon de l'Egypte et de tous les pays.
\TextTitle{Mort de Salomon\FTNTT{1 R. 11:1-40}}
\VS{29}Le reste des actions de Salomon, les premières et les dernières, cela n'est-il pas écrit dans le livre de Nathan le prophète, dans la prophétie d'Achija de Silo, et dans la vision de Jéedo le voyant, touchant Jéroboam, fils de Nebath ?
\VS{30}Salomon régna quarante ans à Jérusalem sur tout Israël.
\VS{31}Puis Salomon s'endormit avec ses pères, et on l'ensevelit dans la cité de David, son père ; et Roboam, son fils, régna à sa place.
\Chap{10}
\TextTitle{Roboam règne sur Israël\FTNTT{1 R. 12:1-15}}
\VerseOne{}Roboam se rendit à Sichem, car tout Israël était venu à Sichem pour l'établir roi.
\VS{2}Quand Jéroboam, fils de Nebath, qui était en Egypte, où il s'était enfui de devant le roi Salomon, l'eut appris, il revint d'Egypte.
\VS{3}Or on l'envoya appeler. Ainsi Jéroboam et tout Israël vinrent et parlèrent à Roboam, en disant :
\VS{4}Ton père a mis sur nous un joug pesant. Allège maintenant cette rude servitude de ton père, et ce joug pesant qu'il a mis sur nous, et nous te servirons.
\VS{5}Alors il leur dit : Revenez vers moi dans trois jours. Et le peuple s'en alla.
\VS{6}Le roi Roboam demanda conseil aux vieillards qui avaient été auprès de Salomon, son père, pendant sa vie, et il leur parla ainsi : Comment, et quelle chose me conseillez-vous de répondre à ce peuple ?
\VS{7} Et ils lui répondirent en ces termes : Si tu es bon envers ce peuple, si tu es bienveillant envers eux, et que tu leur dises de bonnes paroles, ils seront tes serviteurs à toujours.
\VS{8}Mais il laissa le conseil que les vieillards lui avaient donné, et il demanda conseil aux jeunes gens qui avaient grandi avec lui, et qui se tenaient auprès de lui.
\VS{9}Et il leur dit : Que me conseillez-vous de répondre à ce peuple qui m'a parlé en disant : Allège le joug que ton père a mis sur nous ?
\VS{10}Et les jeunes gens qui avaient grandi avec lui, lui parlèrent en disant : Tu répondras en disant à ce peuple qui t'a parlé et t'a dit : Ton père a mis sur nous un joug pesant, mais toi, allège-le ; tu leur répondras donc : Mon petit doigt est plus gros que les reins de mon père.
\VS{11}Or mon père a mis sur vous un joug pesant, mais moi, je rendrai votre joug encore plus pesant. Mon père vous a châtiés avec des fouets, mais moi, je vous châtierai avec des scorpions.
\TextTitle{Roboam délaisse le conseil des anciens}
\VS{12}Trois jours après, Jéroboam, avec tout le peuple, vint vers Roboam, suivant ce qu'avait dit le roi : Revenez vers moi dans trois jours.
\VS{13}Mais le roi leur répondit durement. Le roi Roboam délaissa le conseil des anciens,
\VS{14}et leur parla suivant le conseil des jeunes gens, en disant : Mon père a mis sur vous un joug pesant ; mais moi, j'y ajouterai encore. Mon père vous a châtiés avec des fouets ; mais moi, je vous châtierai avec des scorpions.
\VS{15}Le roi n'écouta donc point le peuple ; cela était conduit par Dieu, afin que Yahweh accomplisse la parole qu'il avait déclarée par Achija de Silo, à Jéroboam, fils de Nebath.
\TextTitle{Israël se détache de la maison de David\FTNTT{1 R. 12:16-19}}
\VS{16}Quand tout Israël vit que le roi ne les écoutait pas, le peuple répondit au roi, en disant : Quelle part avons-nous avec David ? Nous n'avons point d'héritage avec le fils d'Isaï. Israël, chacun à ses tentes ! Et toi David, pourvois maintenant à ta maison. Ainsi, tout Israël s'en alla dans ses tentes.
\VS{17}Mais quant aux enfants d'Israël qui habitaient les villes de Juda, Roboam régna sur eux.
\VS{18}Alors le roi Roboam envoya Hadoram, qui était préposé aux impôts ; mais les enfants d'Israël le lapidèrent à coups de pierres et il mourut. Et le roi Roboam se hâta de monter sur un char pour s'enfuir à Jérusalem.
\VS{19}C'est ainsi qu'Israël s'est rebellé contre la maison de David, jusqu'à ce jour\FTNT{1 R. 12:16-19.}.
\Chap{11} 
\TextTitle{Yahweh interdit la guerre entre Juda et Israël\FTNTT{1 R. 12:21-24}}
\VerseOne{}Roboam, étant arrivé à Jérusalem, assembla la maison de Juda et de Benjamin, cent quatre-vingt mille hommes d'élite et de guerre, afin de combattre contre Israël, pour le ramener sous le règne de Roboam.
\VS{2}Mais la parole de Yahweh fut adressée à Schemaeja, homme de Dieu, en ces termes :
\VS{3}Parle à Roboam, fils de Salomon, roi de Juda, et à ceux d'Israël qui sont en Juda et en Benjamin, et dis-leur :
\VS{4}Ainsi parle Yahweh : Ne montez point, et ne combattez point contre vos frères. Retournez chacun dans sa maison ; c'est par moi que cette chose est arrivée. Et ils obéirent aux paroles de Yahweh, et ils s'en retournèrent sans aller contre Jéroboam\FTNT{1 R. 12:21-24.}.
\VS{5}Roboam demeura donc à Jérusalem, et il bâtit des villes fortes en Juda.
\VS{6}Il bâtit Bethléhem, Etham, Tekoa,
\VS{7}Beth-Tsur, Soco, Adullam,
\VS{8}Gath, Maréscha, Ziph,
\VS{9}Adoraïm, Lakis, Azéka,
\VS{10}Tsorea, Ajalon et Hébron, qui étaient en Juda et en Benjamin, et en fit des villes fortes.
\VS{11}Il les fortifia et y mit des gouverneurs, des provisions de vivres, d'huile et de vin.
\VS{12}Dans chacune de ces villes, il mit des boucliers et des lances, et il les rendit puissantes. Ainsi Juda et Benjamin lui furent soumis.
\TextTitle{Les sacrificateurs et les Lévites soutiennent Roboam}
\VS{13}Les sacrificateurs et les Lévites, qui étaient dans tout Israël, vinrent de toutes leurs contrées se joindre à lui.
\TextTitle{Jéroboam abandonne Yahweh\FTNTT{1 R. 12:26-30 ; 14:7-8}}
\VS{14}Car les Lévites abandonnèrent leurs faubourgs et leurs possessions et vinrent en Juda et à Jérusalem, parce que Jéroboam et ses fils les avaient rejetés des fonctions de sacrificateurs pour Yahweh.
\VS{15}Car il s'était établi des sacrificateurs pour les hauts lieux, pour les boucs, et pour les veaux qu'il avait faits.
\VS{16}Et à leur suite, ceux d'entre toutes les tribus d'Israël qui avaient appliqué leur cœur à chercher Yahweh, le Dieu d'Israël, vinrent à Jérusalem pour sacrifier à Yahweh, le Dieu de leurs pères.
\VS{17}Ils fortifièrent le royaume de Juda et affermirent Roboam, fils de Salomon, pendant trois ans ; car on suivit les voies de David et de Salomon pendant trois ans.
\TextTitle{Les femmes et les enfants de Roboam}
\VS{18}Or Roboam prit pour femme : Mahalath, fille de Jerimoth, fils de David et d'Abichaïl, fille d'Eliab, fils d'Isaï.
\VS{19}Elle lui enfanta des fils : Jeusch, Schemaria et Zaham.
\VS{20}Après elle, il prit Maaca, fille d'Absalom, qui lui enfanta Abija, Attaï, Ziza et Schelomith.
\VS{21}Roboam aima Maaca, fille d'Absalom, plus que toutes ses femmes et ses concubines. Car il prit dix-huit femmes et soixante concubines, et il engendra vingt-huit fils et soixante filles.
\VS{22}Roboam établit pour chef Abija, fils de Maaca, comme prince entre ses frères ; car il voulait le faire roi.
\VS{23}Il agit prudemment et dispersa tous ses fils dans toutes les contrées de Juda et de Benjamin, dans toutes les villes fortes ; il leur donna de quoi vivre en abondance, et demanda pour eux une multitude de femmes.
\Chap{12}
\TextTitle{Roboam affermi, il abandonne Yahweh\FTNTT{1 R. 14:21-24}}
\VerseOne{}Lorsque la royauté de Roboam fut affermie et qu'il eut acquis de la force, il abandonna la loi de Yahweh, et tout Israël avec lui\FTNT{1 R. 14:21-29.}.
\TextTitle{Yahweh veut livrer Juda à Schischak\FTNTT{1 R. 14:25-28}}
\VS{2}C'est pourquoi il arriva que la cinquième année du Roi Roboam, Schischak, roi d'Egypte, monta contre Jérusalem, parce qu'ils avaient péché contre Yahweh.
\VS{3}Il avait mille deux cents chars et soixante mille cavaliers, et le peuple qui vint avec lui d'Egypte, des Libyens, des Sukkiens et des Ethiopiens, était innombrable.
\VS{4}Il prit les villes fortes qui appartenaient à Juda, et vint jusqu'à Jérusalem.
\VS{5}Alors Schemaeja, le prophète, vint vers Roboam et les chefs de Juda, qui s'étaient assemblés à Jérusalem à cause de Schischak, et leur dit : Ainsi parle Yahweh : Vous m'avez abandonné ; moi aussi je vous abandonne aux mains de Schischak.
\VS{6}Alors les chefs d'Israël et le roi s'humilièrent, et dirent : Yahweh est juste !
\VS{7}Et quand Yahweh vit qu'ils s'humiliaient, la parole de Yahweh fut adressée à Schemaeja, et il lui dit : Ils se sont humiliés ; je ne les détruirai pas, mais je leur donnerai dans peu de temps un moyen d'échapper, et ma fureur ne se répandra point sur Jérusalem par la main de Schischak.
\VS{8}Toutefois, ils lui seront asservis, afin qu'ils sachent ce que c'est que de me servir ou de servir les royaumes de la terre.
\VS{9}Schischak, roi d'Egypte, monta donc contre Jérusalem, et prit les trésors de la maison de Yahweh et les trésors de la maison du roi ; il prit tout. Il prit les boucliers d'or que Salomon avait faits.
\VS{10}Le roi Roboam fit des boucliers d'airain à leur place, et il les mit entre les mains des chefs des coureurs qui gardaient la porte de la maison du roi.
\VS{11}Et toutes les fois que le roi entrait dans la maison de Yahweh, les coureurs venaient et les portaient ; puis ils les rapportaient dans la chambre des coureurs.
\VS{12}Ainsi comme il s'était humilié, la colère de Yahweh se détourna de lui, et ne le détruisit pas entièrement ; car il y avait encore de bonnes choses en Juda.
\TextTitle{Mort de Roboam\FTNTT{1 R. 14:21,29,31}}
\VS{13}Le roi Roboam se fortifia donc dans Jérusalem, et régna. Il avait quarante et un ans quand il devint roi, et il régna dix-sept ans à Jérusalem, la ville que Yahweh avait choisie de toutes les tribus d'Israël, pour y mettre son Nom. Sa mère s'appelait Naama, l'Ammonite.
\VS{14}Il fit le mal, car il ne disposa point son cœur pour chercher Yahweh.
\VS{15}Or les actions de Roboam, les premières et les dernières, ne sont-elles pas écrites dans les livres de Schemaeja le prophète, et d'Iddo le voyant, parmi les registres généalogiques ? Les guerres entre Roboam et Jéroboam furent continuelles.
\VS{16}Roboam s'endormit avec ses pères, et il fut enseveli dans la cité de David ; et Abija, son fils, régna à sa place.
\Chap{13}
\TextTitle{Abija règne sur Juda ; guerre entre Israël et Juda\FTNTT{1 R. 15:1-8}}
\VerseOne{}La dix-huitième année du roi Jéroboam, Abija commença à régner sur Juda.
\VS{2}Il régna trois ans à Jérusalem. Sa mère s'appelait Micaja, fille d'Uriel, de Guibea. Or il y eut guerre entre Abija et Jéroboam.
\VS{3}Abija engagea la guerre avec une armée de vaillants guerriers, quatre cent mille hommes d'élite ; et Jéroboam se rangea en bataille contre lui avec huit cent mille hommes d'élite, forts et vaillants.
\VS{4}Et Abija se leva du haut de la montagne de Tsemaraïm, parmi les montagnes d'Ephraïm, et dit : Jéroboam et tout Israël, écoutez-moi!
\VS{5}Ne savez-vous pas que Yahweh, le Dieu d'Israël, a donné pour toujours la royauté sur Israël à David, à lui et à ses fils, par une alliance de sel\FTNT{Sel : Voir commentaire en Lé. 2:13} !
\VS{6}Mais Jéroboam, fils de Nebath, serviteur de Salomon, fils de David, s'est élevé et s'est rebellé contre son seigneur.
\VS{7}Et des gens sans valeur, des fils de Belial, se sont assemblés avec lui et se sont fortifiés contre Roboam, fils de Salomon. Or Roboam était un jeune homme craintif et sans force devant eux.
\VS{8}Et maintenant, vous vous dites être forts devant la royauté de Yahweh, qui est aux mains des fils de David ; vous êtes une multitude, et vous avez avec vous les veaux d'or que Jéroboam vous a faits pour dieux.
\VS{9}N'avez-vous pas rejeté les sacrificateurs de Yahweh, les fils d'Aaron, et les Lévites ? Et ne vous êtes-vous pas faits des sacrificateurs comme les peuples des autres pays ? Quiconque venait, avec un jeune taureau et sept béliers pour être consacré, devenait sacrificateur de ce qui n'est pas Dieu.
\VS{10}Mais quant à nous, Yahweh est notre Dieu, et nous ne l'avons pas abandonné ; les sacrificateurs qui font le service de Yahweh sont fils d'Aaron, et ce sont les Lévites qui tiennent cette fonction.
\VS{11}Nous faisons brûler pour Yahweh, chaque matin et chaque soir, les holocaustes et le parfum d'aromates. Les pains de proposition sont rangés sur la table pure, et on allume le chandelier d'or avec ses lampes, chaque soir. Car nous gardons ce que Yahweh, notre Dieu,veut qu'on garde ; mais vous, vous l'avez abandonné.
\VS{12}Voici, Dieu est avec nous pour être notre chef, avec ses sacrificateurs, et les trompettes retentissantes, pour les faire sonner contre vous. Fils d'Israël, ne combattez pas contre Yahweh, le Dieu de vos pères ; car cela ne vous réussira pas.
\VS{13}Mais Jéroboam fit une embuscade par un détour, et arriva derrière eux. De sorte que les Israélites étaient en face de Juda, qui avait l'embuscade par-derrière.
\VS{14}Ceux de Juda se retournèrent et voici ils avaient la bataille par-devant et par-derrière. Alors ils crièrent à Yahweh, et les sacrificateurs sonnèrent des trompettes.
\TextTitle{Victoire de Juda sur Israël}
\VS{15}Les hommes de Juda poussèrent un cri, et au cri de guerre des hommes de Juda, Yahweh frappa Jéroboam et tout Israël devant Abija et Juda.
\VS{16}Les fils d'Israël s'enfuirent devant ceux de Juda, parce que Dieu les livra entre leurs mains.
\VS{17}Abija et son peuple leur firent un grand un carnage, et il tomba d'Israël cinq cent mille hommes d'élite blessés à mort.
\VS{18}Ainsi, les enfants d'Israël furent humiliés en ce temps-là ; et les enfants de Juda devinrent plus forts, parce qu'ils s'étaient appuyés sur Yahweh, le Dieu de leurs pères.
\VS{19} Abija poursuivit Jéroboam, et lui prit ces villes : Béthel et les villes de son ressort, Jeschana et les villes de son ressort, Ephron et les villes de son ressort.
\TextTitle{Mort de Jéroboam\FTNTT{1 R. 14:19,20}}
\VS{20}Et Jéroboam n'eut plus de force durant le temps d'Abija ; et Yahweh le frappa, et il mourut.
\TextTitle{Les femmes et les fils d'Abija\FTNTT{1 R. 15:7-8}}
\VS{21}Mais Abija se fortifia ; il prit quatorze femmes, et engendra vingt-deux fils et seize filles.
\VS{22}Le reste des actions d'Abija, sa conduite et ses paroles sont écrites dans les mémoires du prophète Iddo.
\VS{23}Abija s'endormit avec ses pères, et on l'ensevelit dans la cité de David ; et Asa, son fils, régna à sa place. De son temps, le pays fut en repos pendant dix ans.
\Chap{14}
\TextTitle{Asa règne sur Juda, il rétablit l'ordre de Yahweh\FTNTT{1 R. 15:11}}
\VerseOne{}Asa fit ce qui est bon et droit aux yeux de Yahweh, son Dieu.
\VS{2}Il ôta les autels étrangers et les hauts lieux ; il brisa les statues et mit en pièces les idoles d'Asherah.
\VS{3}Et il recommanda à Juda de rechercher Yahweh, le Dieu de leurs pères, et de pratiquer la loi et les commandements.
\VS{4}Il ôta de toutes les villes de Juda les hauts lieux et les colonnes consacrées au soleil. Et le royaume fut en repos devant lui.
\VS{5}Il bâtit des villes fortes en Juda, car le pays fut en repos. Et pendant ces années-là, il n'y eut point de guerre contre lui, parce que Yahweh lui donna du repos.
\VS{6}Et il dit à Juda : Bâtissons ces villes, et entourons-les de murailles, de tours, de portes et de barres ; le pays est encore devant nous, parce que nous avons recherché Yahweh, notre Dieu. Nous l'avons recherché, et il nous a donné du repos de toutes parts. Ainsi, ils bâtirent et prospérèrent.
\TextTitle{Asa s'appuie sur Yahweh et triomphe de Zérach\FTNTT{2 Ch. 16:1-10}}
\VS{7}Or Asa avait dans son armée trois cent mille hommes de Juda, portant le grand bouclier et la lance, et deux cent quatre-vingt mille de Benjamin, portant le bouclier et tirant de l'arc, tous vaillants guerriers.
\VS{8}Mais Zérach, l'Ethiopien, sortit contre eux avec une armée d'un million d'hommes, et de trois cents chars ; et il vint jusqu'à Maréscha.
\VS{9}Asa alla au-devant de lui, et ils se rangèrent en bataille dans la vallée de Tsephata, près de Maréscha.
\VS{10}Alors Asa cria à Yahweh, son Dieu, et dit : Yahweh ! Toi seul peux nous secourir, que l'on soit nombreux ou sans force ! Aide-nous, Yahweh, notre Dieu ! Car nous nous appuyons sur toi, et nous sommes venus en ton Nom contre cette multitude. Tu es Yahweh, notre Dieu : Que l'homme ne prévale pas contre toi !
\VS{11}Et Yahweh frappa les Ethiopiens devant Asa et devant Juda ; et les Ethiopiens s'enfuirent.
\VS{12}Asa et le peuple qui était avec lui les poursuivirent jusqu'à Guérar, et tant d'Ethiopiens tombèrent sans pouvoir sauver leur vie ; car ils furent brisés devant Yahweh et son armée, et on emporta un très grand butin.
\VS{13}Ils frappèrent aussi toutes les villes autour de Guérar, car la terreur de Yahweh était sur eux ; et ils pillèrent toutes ces villes, car il s'y trouvait un grand butin.
\VS{14}Ils frappèrent aussi les tentes des troupeaux, et emmenèrent des brebis et des chameaux en abondance ; puis ils retournèrent à Jérusalem.
\Chap{15}
\TextTitle{Azaria le prophète avertit Asa}
\VerseOne{}Alors l'Esprit de Dieu fut sur Azaria, fils d'Oded.
\VS{2}Et il sortit au-devant d'Asa, et lui dit : Asa, et tout Juda et Benjamin, écoutez-moi ! Yahweh est avec vous quand vous êtes avec lui. Si vous le cherchez, vous le trouverez ; mais si vous l'abandonnez, il vous abandonnera.
\VS{3}Pendant longtemps Israël a été sans vrai Dieu, sans sacrificateur qui l'enseignait, et sans loi.
\VS{4}Mais dans leur détresse, ils sont revenus vers Yahweh, le Dieu d'Israël ; ils l'ont cherché, et ils l'ont trouvé\FTNT{Ps. 107:19-20.}.
\VS{5}Dans ces temps-là, il n'y avait point de sûreté pour ceux qui allaient et venaient, car il y avait de grands troubles parmi tous les habitants du pays.
\VS{6}Une nation était écrasée par une autre nation, et une ville par une autre ville ; car Dieu les agitait par toutes sortes d'angoisses.
\VS{7}Mais vous, fortifiez-vous, et que vos mains ne se relâchent pas ; car il y a une récompense pour vos œuvres.
\TextTitle{Asa écoute les paroles d'Azaria\FTNTT{1 R. 15:12-15}}
\VS{8}Or dès qu'Asa eut entendu ces paroles et la prophétie d'Oded le prophète, il se fortifia ; et fit disparaître les abominations de tout le pays de Juda et de Benjamin, et des villes qu'il avait prises dans les montagnes d'Ephraïm ; et il rétablit l'autel de Yahweh, qui était devant le portique de Yahweh.
\VS{9}Puis il assembla tout Juda et Benjamin, et ceux d'Ephraïm, de Manassé et de Siméon, qui habitaient avec eux ; car un grand nombre de gens d'Israël passaient à lui, voyant que Yahweh, son Dieu, était avec lui.
\VS{10}Ils s'assemblèrent donc à Jérusalem, le troisième mois de la quinzième année du règne d'Asa ;
\VS{11}et ils sacrifièrent ce jour-là à Yahweh sept cents bœufs et sept mille brebis, du butin qu'ils avaient amené.
\VS{12}Et ils rentrèrent dans l'alliance pour chercher Yahweh, le Dieu de leurs pères, de tout leur cœur et de toute leur âme ;
\VS{13}de sorte qu'on devait faire mourir quiconque ne rechercherait pas Yahweh, le Dieu d'Israël, petit ou grand, homme ou femme.
\VS{14}Et ils jurèrent à Yahweh, à haute voix, avec des cris de joie, et au son des shofars et des cors.
\VS{15}Tout Juda se réjouit de ce serment, parce qu'ils avaient juré de tout leur cœur et qu'ils avaient recherché Yahweh de leur plein gré, et qu'ils l'avaient trouvé. Et Yahweh leur donna du repos de toutes parts.
\VS{16}Le roi Asa destitua même sa mère, Maaca, de son rang de reine, parce qu'elle avait fait une idole pour Astarté. Asa abattit l'idole, l'écrasa et la brûla près du torrent de Cédron.
\VS{17}Mais les hauts lieux ne furent point ôtés du milieu d'Israël. Néanmoins, le cœur d'Asa fut intègre tout le long de ses jours.
\VS{18}Il remit dans la maison de Dieu les choses que son père avait consacrées, avec ce qu'il avait lui-même consacré, l'argent, l'or et les ustensiles.
\VS{19}Et il n'y eut point de guerre jusqu'à la trente-cinquième année du règne d'Asa.
\Chap{16}
\TextTitle{Alliance d'Asa et du roi de Syrie contre le roi d'Israël\FTNTT{ 1 R. 15:16-22 ; cp. 1 R. 15:27 ; 16:7}}
\VerseOne{}La trente-sixième année du règne d'Asa, Baescha, roi d'Israël, monta contre Juda, et il bâtit Rama, pour empêcher quiconque de sortir et d'entrer vers Asa, roi de Juda.
\VS{2}Alors Asa sortit de l'argent et de l'or des trésors de la maison de Yahweh et de la maison royale, et il envoya dire à Ben-Hadad, roi de Syrie, qui habitait à Damas :
\VS{3}Il y a alliance entre nous, et entre mon père et ton père ; voici, je t'envoie de l'argent et de l'or ; va, romps l'alliance que tu as avec Baescha, roi d'Israël, afin qu'il s'éloigne de moi.
\VS{4}Ben-Hadad écouta le roi Asa, et il envoya les chefs de son armée contre les villes d'Israël, et ils frappèrent Ijjon, Dan, Abel-Maïm, et tous les magasins des villes de Nephthali.
\VS{5}Et aussitôt que Baescha l'apprit, il cessa de bâtir Rama et suspendit ses travaux.
\VS{6}Alors le roi Asa prit avec lui tout Juda, et ils emportèrent les pierres et le bois de Rama, que Baescha faisait bâtir ; et il en bâtit Guéba et Mitspa.
\TextTitle{Hanani condamne l'alliance d'Asa}
\VS{7}En ce temps-là, Hanani le voyant, vint vers Asa, roi de Juda, et lui dit : Parce que tu t'es appuyé sur le roi de Syrie, et que tu ne t'es point appuyé sur Yahweh, ton Dieu, l'armée du roi de Syrie a échappé de ta main.
\VS{8}Les Ethiopiens et les Libyens n'étaient-ils pas une grande armée, ayant des chars et une multitude de cavaliers ? Mais parce que tu t'étais appuyé sur Yahweh, il les livra entre tes mains.
\VS{9}Car les yeux de Yahweh parcourent toute la terre, pour soutenir ceux dont le cœur est tout entier à lui. Tu as agi follement en cela ; car désormais tu auras des guerres.
\VS{10}Asa fut irrité contre le voyant, et le mit en prison, car il était indigné contre lui à ce sujet. Asa opprima aussi, en ce temps-là, quelques-uns du peuple.
\TextTitle{Mort d'Asa\FTNTT{1 R. 15:23-24}}
\VS{11}Or voici, les actions d'Asa, les premières et les dernières, sont écrites dans le livre des rois de Juda et d'Israël.
\VS{12}Asa fut malade des pieds la trente-neuvième année de son règne, et sa maladie fut très grave. Toutefois, il ne chercha point Yahweh dans sa maladie, mais les médecins.
\VS{13}Puis Asa s'endormit avec ses pères, et il mourut la quarante et unième année de son règne.
\VS{14}On l'ensevelit dans le sépulcre qu'il s'était creusé dans la cité de David. On le coucha dans un lit qui était rempli de parfums et d'aromates, composés par le travail d'un parfumeur ; et on lui en brûla une quantité considérable.
\Chap{17}
\TextTitle{Josaphat règne sur Juda, il recherche Yahweh\FTNTT{1 R. 15:24}}
\VerseOne{}Josaphat son fils régna à sa place et se fortifia contre Israël.
\VS{2}Il mit des troupes dans toutes les villes fortes de Juda, et des garnisons dans le pays de Juda, et dans les villes d'Ephraïm qu'Asa, son père, avait prises.
\VS{3}Yahweh fut avec Josaphat, parce qu'il suivit les premières voies de David, son père, et qu'il ne rechercha point les Baals ;
\VS{4}car il rechercha le Dieu de son père, et il marcha dans ses commandements, et non pas selon ce que faisait Israël.
\VS{5}Yahweh affermit donc le royaume entre ses mains ; et tout Juda apportait des présents à Josaphat, et il eut en abondance des richesses et de la gloire.
\VS{6}Son cœur grandit dans les voies de Yahweh, et il ôta encore de Juda les hauts lieux et les idoles d'Astarté.
\VS{7}Puis, la troisième année de son règne, il envoya ses chefs Ben-Haïl, Abdias, Zacharie, Nethaneel et Michée, pour enseigner dans les villes de Juda ;
\VS{8}et avec eux les Lévites Schemaeja, Nethania, Zebadia, Asaël, Schemiramoth, Jonathan, Adonija, Tobija et Tob-Adonija, Lévites, et avec eux Elischama et Joram, les sacrificateurs.
\VS{9}Ils enseignèrent dans Juda, ayant avec eux le livre de la loi de Yahweh. Ils firent le tour de toutes les villes de Juda, et enseignèrent parmi le peuple.
\TextTitle{Affermissement du règne de Josaphat}
\VS{10}La terreur de Yahweh fut sur tous les royaumes des pays qui entouraient Juda, et ils ne firent point la guerre à Josaphat.
\VS{11}On apporta aussi à Josaphat des présents de la part des Philistins, et un impôt en argent ; et les Arabes lui amenèrent aussi du bétail, sept mille sept cents béliers et sept mille sept cents boucs.
\VS{12}Ainsi Josaphat s'élevait jusqu'au plus haut degré de gloire. Et il bâtit en Juda des châteaux et des villes pour servir de magasins.
\VS{13}Il fit de grands travaux dans les villes de Juda ; et il avait à Jérusalem des gens de guerre puissants et vaillants.
\VS{14}Voici leur dénombrement, selon les maisons de leurs pères. Les chefs de milliers de Juda furent Adna le chef, avec trois cent mille vaillants guerriers.
\VS{15}Et après lui, Jochanan le chef, avec deux cent quatre-vingt mille hommes.
\VS{16}A ses côtés, Amasia, fils de Zicri, qui s'était volontairement offert à Yahweh, avec deux cent mille vaillants guerriers.
\VS{17}De Benjamin, Eliada, vaillant guerrier, avec deux cent mille hommes, armés d'arcs et de boucliers,
\VS{18}à côté de lui Zozabad, avec cent quatre-vingt mille hommes équipés pour le combat.
\VS{19}Tels sont ceux qui étaient au service du roi, outre ceux que le roi avait placés dans toutes les villes fortes de Juda.
\Chap{18}
\TextTitle{Josaphat s'allie à Achab contre les Syriens\FTNTT{1 R. 22:2-4}}
\VerseOne{}Or Josaphat, ayant beaucoup de richesses et de gloire, s'allia par mariage avec Achab.
\VS{2}Et au bout de quelques années, il descendit vers Achab, à Samarie. Achab tua pour lui, et pour le peuple qui était avec lui, un grand nombre de brebis et de bœufs, et l'incita à monter contre Ramoth de Galaad\FTNT{1 R. 22:2-40.}.
\VS{3}Achab, roi d'Israël, dit à Josaphat, roi de Juda : Viendras-tu avec moi contre Ramoth de Galaad ? Et il lui répondit : Compte sur moi comme sur toi, et sur mon peuple comme sur ton peuple, nous irons avec toi à la guerre.
\TextTitle{Les prophètes de mensonge encouragent Achab\FTNTT{1 R. 22:5-12}}
\VS{4}Puis Josaphat dit au roi d'Israël : Consulte aujourd'hui, je te prie, la parole de Yahweh.
\VS{5}Le roi d'Israël assembla les prophètes, au nombre de quatre cents, et leur dit : Irons-nous à la guerre contre Ramoth de Galaad, ou dois-je y renoncer ? Ils répondirent : Monte, et Dieu la livrera entre les mains du roi.
\VS{6}Mais Josaphat dit : N'y a-t-il point encore ici quelque prophète de Yahweh, afin que nous l'interrogions ?
\VS{7}Le roi d'Israël dit à Josaphat : Il y a encore un homme par qui on peut consulter Yahweh ; mais je le hais parce qu'il ne me prophétise rien de bon, mais du mal ; c'est Michée, fils de Jimla. Josaphat dit : Que le roi ne parle pas ainsi !
\VS{8}Alors le roi d'Israël appela un eunuque, et dit : Fais promptement venir Michée, fils de Jimla.
\VS{9}Or le roi d'Israël et Josaphat, roi de Juda, étaient assis, chacun sur son trône, revêtus de leurs habits, et ils étaient assis dans la place, à l'entrée de la porte de Samarie ; et tous les prophètes prophétisaient en leur présence.
\VS{10}Alors Sédécias, fils de Kenaana, s'étant fait des cornes de fer, dit : Ainsi parle Yahweh : Avec ces cornes tu heurteras les Syriens jusqu'à les détruire.
\VS{11}Tous les prophètes prophétisaient de même, en disant : Monte à Ramoth de Galaad, et tu prospèreras ; Yahweh la livrera entre les mains du roi.
\TextTitle{Michée annonce la défaite et la mort d'Achab\FTNTT{1 R. 22:13-28 ; 1 R. 22:29-40}}
\VS{12}Or le messager qui était allé appeler Michée, lui parla et lui dit : Voici, tous les prophètes disent d'une même bouche du bien au roi ; je te prie que ta parole soit semblable à celle de chacun d'eux ! Annonce du bien !
\VS{13}Mais Michée répondit : Yahweh est vivant ! Je dirai ce que mon Dieu dira.
\VS{14}Il vint donc vers le roi, et le roi lui dit : Michée, irons-nous à la guerre contre Ramoth de Galaad, devons-nous y renoncer ? Et il répondit : Montez, vous prospérerez, et ils seront livrés entre vos mains.
\VS{15}Et le roi lui dit : Combien de fois devrais-je te faire jurer de ne me dire que la vérité au Nom de Yahweh ?
\VS{16}Et il répondit : J'ai vu tout Israël dispersé par les montagnes, comme un troupeau de brebis qui n'a point de berger ; et Yahweh a dit : Ces gens n'ont point de seigneur ; que chacun retourne en paix dans sa maison !
\VS{17}Alors le roi d'Israël dit à Josaphat : Ne t'ai-je pas dit qu'il ne prophétise rien de bon quand il s'agit de moi, mais seulement du mal ?
\VS{18}Et Michée dit : Ecoute la parole de Yahweh ! J'ai vu Yahweh assis sur son trône, et toute l'armée des cieux se tenant à sa droite et à sa gauche.
\VS{19}Et Yahweh dit : Qui est-ce qui séduira Achab, roi d'Israël, afin qu'il monte et qu'il tombe à Ramoth de Galaad ? Et l'un répondait d'une façon et l'autre d'une autre.
\VS{20}Alors un esprit s'avança et se tint devant Yahweh, et dit : Moi, je le séduirai. Yahweh lui dit : Comment ?
\VS{21}Il répondit : Je sortirai, dit-il, et je serai un esprit de mensonge\FTNT{Achab a été frappé de l'esprit d'égarement (2 Th. 2:9-11). Voir commentaires en Ge. 6:3 ; Mt. 12:31.} dans la bouche de tous ses prophètes. Et Yahweh dit : Tu le séduiras, et même tu en viendras à bout. Sors, et fais ainsi.
\VS{22}Maintenant voici, Yahweh a mis un esprit de mensonge dans la bouche de tes prophètes que voilà ; et Yahweh a prononcé du mal contre toi.
\VS{23}Alors Sédécias, fils de Kenaana, s'étant approché, frappa Michée sur la joue, et dit : Par quel chemin l'Esprit de Yahweh s'est-il retiré de moi pour te parler ?
\VS{24}Et Michée répondit : Voici, tu le verras au jour où tu iras de chambre en chambre pour te cacher !
\VS{25}Alors le roi d'Israël dit : Prenez Michée, et emmenez-le vers Amon, chef de la ville, et vers Joas, fils du roi.
\VS{26}Et vous direz : Ainsi parle le roi : Mettez cet homme en prison, et nourrissez-le du pain et de l'eau de l'affliction, jusqu'à ce que je revienne en paix.
\VS{27}Et Michée dit : Si jamais tu retournes et reviens en paix, Yahweh n'aura point parlé par moi. Et il dit : Entendez cela peuples, vous tous qui êtes ici !
\VS{28}Le roi d'Israël monta donc avec Josaphat, roi de Juda, à Ramoth de Galaad.
\VS{29}Le roi d'Israël dit à Josaphat : Je vais me déguiser pour aller au combat ; mais toi, revêts-toi de tes habits. Ainsi le roi d'Israël se déguisa ; et ils allèrent au combat.
\VS{30}Or le roi des Syriens avait donné cet ordre aux chefs de ses chars, disant : Vous ne combattrez ni petit ni grand, mais seulement le roi d'Israël.
\VS{31}Les chefs des chars aperçurent Josaphat, et dirent : C'est le roi d'Israël ! Et ils se tournèrent vers lui pour le combattre ; mais Josaphat poussa un cri, et Yahweh le secourut, et Dieu les éloigna de lui.
\VS{32}Quand les chefs des chars virent que ce n'était pas le roi d'Israël, ils se détournèrent de lui.
\VS{33}Alors quelqu'un tira de son arc au hasard, et frappa le roi d'Israël entre les jointures de la cuirasse ; et le roi dit à son conducteur de char : Tourne-toi, et sors-moi du camp ; car je suis blessé.
\VS{34}Or en ce jour-là, le combat fut très rude. Le roi d'Israël se posa dans son char, en face des Syriens, jusqu'au soir ; et il mourut vers le coucher du soleil.
\Chap{19}
\TextTitle{Jéhu dénonce l'alliance de Josaphat avec Achab}
\VerseOne{}Josaphat roi de Juda, revint en paix dans sa maison, à Jérusalem.
\VS{2}Mais Jéhu, fils de Hanani, le voyant, sortit au-devant du roi Josaphat, et lui dit : Faut-il donner du secours au méchant, ou aimer ceux qui haïssent Yahweh ? A cause de cela, Yahweh est irrité contre toi.
\VS{3}Mais il s'est trouvé de bonnes choses en toi, puisque tu as ôté du pays les idoles d'Astarté, et tu as appliqué ton cœur à rechercher Dieu.
\VS{4}Josaphat demeura à Jérusalem. Puis, il ressortit de nouveau parmi le peuple, depuis Beer-Schéba jusqu'à la montagne d'Ephraïm, et il les ramena à Yahweh, le Dieu de leurs pères.
\TextTitle{Josaphat organise la justice}
\VS{5}Il établit aussi des juges dans le pays, dans toutes les villes fortes de Juda, de ville en ville.
\VS{6}Et il dit aux juges : Veillez sur ce que vous ferez ; car vous n'exercez pas la justice de la part d'un homme, mais de la part de Yahweh, qui sera avec vous quand vous prononcerez les jugements.
\VS{7}Maintenant, que la crainte de Yahweh soit sur vous ; prenez garde à ce que vous ferez ; car il n'y a point d'iniquité chez Yahweh, notre Dieu, ni d'acception de personnes, ni d'acceptation de présents. 
\VS{8}Josaphat établit aussi à Jérusalem des Lévites, des sacrificateurs, et des chefs des pères d'Israël, pour le jugement de Yahweh, et pour les contestations ; car on revenait à Jérusalem.
\VS{9}Il leur donna des ordres, en disant : Vous agirez ainsi dans la crainte de Yahweh, avec fidélité et avec intégrité de cœur.
\VS{10}Dans toute contestation qui viendra devant vous, de la part de vos frères qui habitent dans leurs villes, soit d'un meurtre, d'une loi, d'un commandement, d'un statut ou d'une ordonnance, vous les instruirez, afin qu'ils ne se rendent pas coupables envers Yahweh, et que sa colère ne vienne pas sur vous et sur vos frères. Vous agirez ainsi afin de ne pas être coupables.
\VS{11}Et voici, Amaria, le souverain sacrificateur, sera au-dessus de vous pour toutes les affaires de Yahweh ; et Zebadia, fils d'Ismaël, prince de la maison de Juda, pour toutes les affaires du roi ; et pour secrétaires, vous avez devant vous les Lévites. Fortifiez-vous et faites ainsi ; et que Yahweh soit avec l'homme de bien !
\Chap{20}
\TextTitle{Menaces des ennemis de Juda, prière de Josaphat}
\VerseOne{}Après ces choses, les fils de Moab et les fils d'Ammon, et avec eux les Maonites, vinrent contre Josaphat pour lui faire la guerre.
\VS{2}On vint le rapporter à Josaphat, en disant : Il vient contre toi une grande multitude depuis l'autre bord de la mer, de Syrie ; et les voici à Hatsatson-Thamar, qui est En-Guédi.
\VS{3}Alors Josaphat craignit ; mais il se disposa à rechercher Yahweh, et publia un jeûne pour tout Juda.
\VS{4}Juda s'assembla donc pour rechercher Yahweh ; on vint même de toutes les villes de Juda pour chercher Yahweh.
\VS{5}Et Josaphat se tint au milieu de l'assemblée de Juda et de Jérusalem, dans la maison de Yahweh, devant le nouveau parvis.
\VS{6}Il dit : Yahweh, Dieu de nos pères ! N'es-tu pas Dieu dans les cieux, toi qui domines sur tous les royaumes des nations ? Ne tiens-tu pas dans ta main la force et la puissance, et à qui nul ne peut résister ?
\VS{7}N'est-ce pas toi, ô notre Dieu, qui as dépossédé les habitants de ce pays devant ton peuple d'Israël, et qui l'as donné pour toujours à la postérité d'Abraham, qui t'aimait ?
\VS{8}Ils y ont habité et t'y ont bâti un sanctuaire pour ton Nom, en disant :
\VS{9}S'il nous arrive quelque malheur, l'épée, le jugement, la peste, ou la famine, nous nous tiendrons devant cette maison, et en ta présence ; car ton Nom est en cette maison ; et nous crierons à toi dans notre détresse, et tu exauceras et tu délivreras !
\VS{10}Maintenant, voici les enfants d'Ammon et de Moab, et ceux de la montagne de Séir, chez lesquels tu ne permis pas à Israël d'entrer quand il venait du pays d'Egypte, car il se détourna d'eux, et ne les détruisit pas.
\VS{11}Voici, pour nous récompenser, ils viennent nous chasser de ton héritage, que tu nous as fait posséder.
\VS{12}Ô notre Dieu ! Ne seras-tu pas juge contre eux ? Car nous sommes sans force devant cette grande multitude qui vient contre nous, et nous ne savons que faire ; mais nos yeux sont sur toi.
\VS{13}Or tout Juda se tenait devant Yahweh, même avec leurs petits enfants, leurs femmes et leurs fils.
\TextTitle{Yahweh répond à Josaphat}
\VS{14}Alors l'Esprit de Yahweh saisit au milieu de l'assemblée Jachaziel, fils de Zacharie, fils de Benaja, fils de Jeïel, fils de Matthania, Lévite, d'entre les fils d'Asaph,
\VS{15}et il dit : Soyez attentifs, tout Juda et habitants de Jérusalem, et toi, roi Josaphat ! Ainsi parle Yahweh : Ne craignez point, et ne soyez point effrayés en face de cette grande multitude ; car ce ne sera pas à vous de combattre, mais à Dieu.
\VS{16}Descendez demain vers eux ; les voici qui montent par la montée de Tsits, et vous les trouverez à l'extrémité de la vallée, en face du désert de Jeruel.
\VS{17}Ce ne sera point à vous de combattre en cette bataille ; présentez-vous, tenez-vous là, et voyez la délivrance que Yahweh va vous donner. Juda et Jérusalem, ne craignez point, et ne soyez point effrayés ! Demain, sortez au-devant d'eux, et Yahweh sera avec vous.
\VS{18}Alors Josaphat s'inclina le visage contre terre, et tout Juda et les habitants de Jérusalem se jetèrent devant Yahweh, se prosternant devant Yahweh.
\VS{19}Et les Lévites, d'entre les fils des Kehathites et d'entre les fils des Koréites, se levèrent pour célébrer Yahweh, le Dieu d'Israël, d'une voix haute et forte.
\TextTitle{Yahweh délivre Juda des armées ennemies}
\VS{20}Puis, le matin, ils se levèrent de bonne heure et sortirent vers le désert de Tekoa. Et comme ils sortaient, Josaphat se tint debout et dit : Ecoutez-moi Juda et vous, habitants de Jérusalem ! Croyez en Yahweh, votre Dieu, et vous serez en sûreté ; croyez en ses prophètes, et vous réussirez.
\VS{21}Puis, ayant consulté le peuple, il établit des chantres de Yahweh, qui célébraient sa sainte magnificence ; et marchant devant l'armée, ils disaient : Louez Yahweh, car sa miséricorde dure à toujours\FTNT{Ps. 136} !
\VS{22}Et au moment où ils commencèrent le chant et la louange, Yahweh mit des embuscades contre les fils d'Ammon, de Moab, et ceux de la montagne de Séir, qui venaient contre Juda. Et ils furent battus.
\VS{23}Les fils d'Ammon et de Moab se levèrent contre les habitants de la montagne de Séir pour les dévouer par interdit et les exterminer ; et quand ils en eurent fini avec les habitants de Séir, ils s'aidèrent l'un l'autre à se détruire mutuellement.
\VS{24}Et quand Juda fut arrivé sur la hauteur d'où l'on voit le désert, ils regardèrent vers cette multitude, et voici, c'étaient des cadavres gisant à terre, et personne n'avait échappé.
\VS{25}Ainsi Josaphat et son peuple vinrent pour piller leurs dépouilles, et ils trouvèrent parmi les cadavres des biens en abondance, et des objets précieux ; et ils en saisirent tant qu'ils ne pouvaient tout porter ; et ils pillèrent le butin pendant trois jours, car il était considérable.
\VS{26}Le quatrième jour, ils s'assemblèrent dans la vallée de Beraca ; car ils bénirent là Yahweh ; c'est pourquoi on a appelé ce lieu, jusqu'à ce jour, la vallée de Beraca.
\VS{27}Et tous les hommes de Juda et de Jérusalem, et Josaphat à leur tête, s'en retournèrent, revenant à Jérusalem avec joie ; car Yahweh les avait réjouis au sujet de leurs ennemis. 
\VS{28}Ils entrèrent donc à Jérusalem, dans la maison de Yahweh, avec des luths, des harpes et des trompettes.
\VS{29}Et la crainte de Dieu fut sur tous les royaumes des autres pays, lorsqu'ils apprirent que Yahweh avait combattu contre les ennemis d'Israël.
\VS{30}Ainsi le royaume de Josaphat fut tranquille, et son Dieu lui donna du repos de toutes parts.
\TextTitle{Règne de Josaphat, son alliance coupable\FTNTT{1 R. 22:41-49}}
\VS{31}Josaphat régna donc sur Juda. Il était âgé de trente-cinq ans quand il devint roi, et il régna vingt-cinq ans à Jérusalem. Sa mère s'appelait Azuba, fille de Schilchi.
\VS{32}Il suivit les traces d'Asa, son père, et il ne s'en détourna point, faisant ce qui est droit aux yeux de Yahweh.
\VS{33}Seulement les hauts lieux ne furent pas ôtés, et le peuple n'avait pas encore le cœur fermement attaché au Dieu de ses pères.
\VS{34}Or le reste des actions de Josaphat, les premières et les dernières, voici, elles sont écrites dans les mémoires de Jéhu, fils de Hanani, insérées dans le livre des rois d'Israël.
\VS{35}Après cela, Josaphat, roi de Juda, s'associa avec Achazia, roi d'Israël, dont la conduite était impie.
\VS{36}Il s'associa avec lui pour faire des navires, afin d'aller à Tarsis ; et ils firent des navires à Etsjon-Guéber.
\VS{37}Alors Eliézer, fils de Dodava, de Maréscha, prophétisa contre Josaphat, en disant : Parce que tu t'es associé avec Achazia, Yahweh a détruit ton œuvre. Et les navires furent brisés, et ne purent aller à Tarsis.
\Chap{21}
\TextTitle{Joram règne sur Juda\FTNTT{1 R. 22:50 ; 2 R. 8:16-19}}
\VerseOne{}Puis Josaphat s'endormit avec ses pères, et il fut enseveli avec eux dans la cité de David. Et Joram, son fils, régna à sa place\FTNT{1 R. 22:51.}.
\VS{2}Il avait des frères, fils de Josaphat : Azaria, Jehiel, Zacharie, Azaria, Micaël et Schephathia. Tous ceux-là étaient fils de Josaphat, roi d'Israël.
\VS{3}Leur père leur avait fait de grands dons d'argent, d'or et de choses précieuses, avec des villes fortes en Juda ; mais il avait donné le royaume à Joram, parce qu'il était le premier-né.
\VS{4}Quand Joram fut élevé sur le royaume de son père, et s'y fut fortifié, il tua avec l'épée tous ses frères, et quelques-uns aussi des chefs d'Israël.
\VS{5}Joram était âgé de trente-deux ans quand il devint roi, et il régna huit ans à Jérusalem.
\VS{6}Il marcha dans la voie des rois d'Israël, comme avait fait la maison d'Achab ; car la fille d'Achab était sa femme, et il fit ce qui est mal aux yeux de Yahweh.
\VS{7}Toutefois, Yahweh, à cause de l'alliance qu'il avait traitée avec David, ne voulut pas détruire la maison de David, selon qu'il avait dit qu'il lui donnerait une lampe, à lui et à ses fils, pour toujours.
\TextTitle{Rébellion d'Edom et de Libna\FTNTT{1 R. 8:20-23}}
\VS{8}De son temps, Edom se rebella de l'autorité de Juda, et établit un roi sur lui\FTNT{2 R. 8:20-23}.
\VS{9}Joram se mit donc en marche avec ses chefs et tous ses chars ; et s'étant levé de nuit, il battit les Edomites qui l'entouraient, et tous les chefs des chars.
\VS{10}Néanmoins, Edom se rebella contre l'autorité de Juda jusqu'à ce jour. En ce même temps, Libna se rebella aussi contre son autorité, parce qu'il avait abandonné Yahweh, le Dieu de ses pères.
\VS{11}Il fit aussi des hauts lieux dans les montagnes de Juda ; il fit que les habitants de Jérusalem se prostituèrent, et il y entraîna ceux de Juda.
\TextTitle{Elie prononce un jugement sur Joram}
\VS{12}Alors il lui vint un écrit de la part d'Elie, le prophète, disant : Ainsi parle Yahweh, le Dieu de David, ton père : Parce que tu n'as point suivi le chemin de Josaphat, ton père, ni celui d'Asa, roi de Juda,
\VS{13}mais que tu as suivi les voies des rois d'Israël, et que tu as poussé à la prostitution Juda et les habitants de Jérusalem, comme s'est prostituée la maison d'Achab, et que tu as tué tes frères, meilleurs que toi, la maison même de ton père ;
\VS{14}voici, Yahweh frappera d'une grande plaie ton peuple, tes fils, tes femmes et tous tes biens.
\VS{15}Et toi, tu auras une grosse maladie, une maladie d'entrailles ; jusqu'à ce que, de jour en jour, tes entrailles sortent par la force de la maladie.
\TextTitle{Yahweh excite les Philistins et les Arabes contre Joram}
\VS{16}Yahweh souleva contre Joram l'esprit des Philistins et des Arabes, qui habitent près des Ethiopiens.
\VS{17}Ils montèrent donc contre Juda, et firent une brèche pour piller toutes les richesses qui furent trouvées dans la maison du roi ; et même, ils emmenèrent captifs ses fils et ses femmes, de sorte qu'il ne lui resta d'autre fils que Joachaz, le plus jeune de ses fils.
\TextTitle{Mort de Joram}
\VS{18}Après tout cela, Yahweh frappa ses entrailles d'une maladie sans remède.
\VS{19}Elle s'avança chaque jour, et vers la fin de la seconde année, ses entrailles sortirent par la force de son mal, et il mourut dans de grandes souffrances. Son peuple ne brûla pas sur lui de parfums, comme il l'avait fait pour ses pères.
\VS{20}Il était âgé de trente-deux ans quand il devint roi, et il régna huit ans à Jérusalem. Il s'en alla sans être regretté, et on l'ensevelit dans la cité de David, mais non dans les sépulcres des rois.
\Chap{22}
\TextTitle{Achazia règne sur Juda\FTNTT{2 R. 8:24-29}}
\VerseOne{}Les habitants de Jérusalem firent régner à sa place Achazia, le plus jeune de ses fils, parce que les troupes qui étaient venues au camp avec les Arabes avaient tué tous les plus âgés ; et Achazia, fils de Joram, roi de Juda, régna\FTNT{2 R. 8:24-29 ; 2 R. 9:16.}.
\VS{2}Achazia était âgé de quarante-deux ans quand il devint roi, et il régna un an à Jérusalem. Sa mère avait pour nom Athalie, fille d'Omri.
\VS{3}Il suivit aussi les voies de la maison d'Achab, car sa mère lui donnait des conseils impies.
\VS{4}Il fit donc ce qui est mal aux yeux de Yahweh, comme la maison d'Achab ; parce qu'ils furent ses conseillers après la mort de son père, pour sa ruine.
\TextTitle{Achazia livré aux mains de Jéhu\FTNTT{2 R. 8:28-29 ; 2 R. 9:1-30}}
\VS{5}Conduit par leurs conseils, il alla avec Joram, fils d'Achab, roi d'Israël, à la guerre à Ramoth de Galaad, contre Hazaël, roi de Syrie. Et les Syriens frappèrent Joram,
\VS{6}qui s'en retourna à Jizreel, pour guérir des blessures que les Syriens lui avaient faites à Rama, lorsqu'il faisait la guerre contre Hazaël, roi de Syrie. Azaria, fils de Joram, roi de Juda, descendit pour voir Joram, le fils d'Achab, à Jizreel, parce qu'il était malade.
\VS{7}Dieu fit pour sa ruine qu'Achazia vint auprès de Joram. En effet, quand il fut arrivé, il sortit avec Joram pour aller au-devant de Jéhu, fils de Nimschi, que Yahweh avait oint pour retrancher la maison d'Achab.
\VS{8}Et comme Jéhu faisait justice de la maison d'Achab\FTNT{2 R. 10:12-30.}, il trouva les chefs de Juda et les fils des frères d'Achazia, qui servaient Achazia, et il les tua.
\VS{9}Il chercha ensuite Achazia, qui s'était caché en Samarie. On le prit, et on l'amena vers Jéhu qui le fit mourir. Puis on l'ensevelit, car on dit : C'est le fils de Josaphat, qui cherchait Yahweh de tout son cœur. Et il n'y eut plus personne dans la maison d'Achazia qui fut capable de régner.
\TextTitle{Joas échappe au massacre de sa famille\FTNTT{2 R. 11:1-3}}
\VS{10}Or Athalie, mère d'Achazia, voyant que son fils était mort, se leva et fit périr toute la race royale de la maison de Juda\FTNT{1 R. 11:1-3.}.
\VS{11}Mais Joschabeath, fille du roi Joram, prit Joas, fils d'Achazia, en le dérobant d'entre les fils du roi qu'on faisait mourir. Elle le mit avec sa nourrice dans la salle des lits. Ainsi Joschabeath, fille du roi Joram et femme de Jehojada le sacrificateur, étant la sœur d'Achazia, le cacha de la vue d'Athalie, qui ne put le faire mourir.
\VS{12}Il fut ainsi caché avec eux dans la maison de Dieu six ans ; et c'est Athalie qui régnait sur le pays.
\Chap{23}
\TextTitle{Joas devient roi grâce à Jehojada\FTNTT{2 R. 11:4-12}}
\VerseOne{}Mais la septième année, Jehojada prit courage et traita alliance avec les chefs de centaines, Azaria, fils de Jerocham, Ismaël, fils de Jochanan, Azaria, fils d'Obed, Maaséja, fils d'Adaja, et Elischaphath, fils de Zicri.
\VS{2}Ils firent le tour de Juda, pour rassembler de toutes les villes de Juda les Lévites et les chefs des pères d'Israël ; puis ils vinrent à Jérusalem.
\VS{3}Et toute cette assemblée traita alliance avec le roi dans la maison de Dieu. Jehojada leur dit : Voici, c'est le fils du roi qui régnera, selon la parole de Yahweh au sujet des fils de David.
\VS{4}Vous ferez donc ceci : Le tiers qui parmi vous entre en service au sabbat, sacrificateurs et Lévites, fera la garde des seuils.
\VS{5}Un autre tiers se tiendra dans la maison du roi, et un tiers à la porte de Jesod ; et tout le peuple sera dans les parvis de la maison de Yahweh.
\VS{6}Que personne n'entre dans la maison de Yahweh, sauf les sacrificateurs et les Lévites de service : Ils entreront car ils sont sanctifiés ; et tout le reste du peuple gardera les ordres de Yahweh.
\VS{7}Les Lévites environneront le roi de toutes parts, tenant chacun leurs armes à la main, et donneront la mort à quiconque voudra entrer dans la maison ; vous serez avec le roi quand il entrera et quand il sortira.
\VS{8}Les Lévites et tout Juda firent tout ce que Jehojada le sacrificateur, avait ordonné. Ils prirent chacun leurs gens, tant ceux qui entraient en service que ceux qui en sortaient au sabbat ; car Jehojada, le sacrificateur, n'avait exempté aucune classe.
\VS{9}Et Jehojada le sacrificateur, donna aux chefs de centaines les lances, les grands et les petits boucliers qui provenaient du roi David, et qui étaient dans la maison de Dieu.
\VS{10}Puis il rangea tout le peuple autour du roi, chacun tenant ses armes à la main, du côté droit du temple jusqu'au côté gauche de la maison, près de l'autel et de la maison.
\VS{11}Alors ils firent sortir le fils du roi, et mirent sur lui la couronne et le témoignage. Ils l'établirent roi, et Jehojada et ses fils l'oignirent et dirent : Vive le roi !
\TextTitle{Mort d'Athalie\FTNTT{2 R. 11:13-16}}
\VS{12}Mais Athalie, entendant le bruit du peuple qui courait et célébrait le roi, vint vers le peuple, dans la maison de Yahweh.
\VS{13}Elle regarda, et voici, le roi se tenait près de la colonne, à l'entrée ; les chefs et les trompettes étaient près du roi, et tout le peuple du pays était dans la joie, et l'on sonnait des trompettes ; les chantres, avec des instruments de musique, dirigeaient les chants de louanges. Alors Athalie déchira ses vêtements et dit : Conspiration ! Conspiration !
\VS{14}Le sacrificateur Jehojada fit sortir les chefs de centaines qui étaient à la tête de l'armée, et leur dit : Faites-la sortir hors des rangs, et que celui qui la suivra soit mis à mort par l'épée ! Car le sacrificateur avait dit : Ne la mettez pas à mort dans la maison de Yahweh.
\VS{15}Ils mirent donc la main sur elle pour la faire entrer dans la maison du roi, par l'entrée de la porte des chevaux ; et là ils la firent mourir.
\TextTitle{Jehojada fait asseoir Joas sur le trône de Juda\FTNTT{2 R. 11:17-20}}
\VS{16}Puis Jehojada traita, avec tout le peuple et le roi, une alliance pour être le peuple de Yahweh.
\VS{17}Et tout le peuple entra dans la maison de Baal pour la détruire. Ils brisèrent ses autels et ses images et ils tuèrent devant les autels Matthan, sacrificateur de Baal.
\VS{18}Jehojada remit aussi les fonctions de la maison de Yahweh entre les mains des sacrificateurs, des Lévites, comme David les avait répartis dans la maison de Yahweh, afin qu'ils élèvent des holocaustes à Yahweh, comme cela est écrit dans la loi de Moïse, avec joie et avec des chants, selon les ordonnances de David.
\VS{19}Il établit aussi les portiers aux portes de la maison de Yahweh, afin qu'il n'y entrât aucune personne souillée de quelque manière que ce fût.
\VS{20}Il prit les chefs de centaines, hommes considérés, qui avaient de l'autorité parmi le peuple, et tout le peuple du pays. Il fit descendre le roi, de la maison de Yahweh à la maison du roi, en entrant par la porte supérieure ; et ils firent asseoir le roi sur le trône royal.
\VS{21}Alors tout le peuple du pays se réjouit, et la ville fut tranquille, bien qu'on eût mis à mort Athalie par l'épée.
\Chap{24}
\TextTitle{Joas règne sur Juda ; ses travaux sur le temple\FTNTT{2 R. 11:21-12:8}}
\VerseOne{}Joas était âgé de sept ans quand il devint roi, et il régna quarante ans à Jérusalem. Sa mère avait pour nom Tsibja, de Beer-Schéba\FTNT{2 R. 11:21 ; 2 R. 12:1-3.}.
\VS{2}Joas fit ce qui est droit aux yeux de Yahweh, pendant toute la vie de Jehojada, le sacrificateur.
\VS{3}Et Jehojada prit pour lui deux femmes, dont il eut des fils et des filles.
\VS{4}Après cela Joas eut la pensée de renouveler la maison de Yahweh\FTNT{2 R. 12:4-16.}.
\VS{5}Il assembla donc les sacrificateurs et les Lévites, et leur dit : Allez vers les villes de Juda, et recueillez de l'argent par tout Israël, suffisamment pour réparer la maison de votre Dieu d'année en année, et hâtez-vous pour cette affaire. Mais les Lévites ne se hâtèrent point.
\VS{6}Alors le roi appela Jehojada, leur chef, et lui dit : Pourquoi n'as-tu pas veillé à ce que les Lévites aient apporté de Juda et de Jérusalem, l'impôt sur l'assemblée d'Israël, selon Moïse, serviteur de Yahweh, pour la tente du témoignage ?
\VS{7}Car l'impie Athalie et ses fils ont ravagé la maison de Dieu ; et même ils ont employé pour les Baals toutes les choses consacrées à la maison de Yahweh.
\TextTitle{Offrandes volontaires pour la réparation du temple\FTNTT{2 R. 12:9-16}}
\VS{8}Et le roi ordonna qu'on fasse un seul coffre, et qu'on le mette à la porte de la maison de Yahweh, à l'extérieur.
\VS{9}Puis on publia dans Juda et dans Jérusalem, pour qu'on qu'on apportât à Yahweh l'impôt mis par Moïse, serviteur de Dieu, sur Israël dans le désert.
\VS{10}Tous les chefs et tout le peuple s'en réjouirent, et l'on apporta et jeta le tribut dans le coffre, jusqu'à ce qu'il fût plein.
\VS{11}Au moment venu, les Lévites apportaient le coffre aux inspecteurs du roi, car ceux-ci voyaient qu'il y avait beaucoup d'argent. Un secrétaire du roi et un commissaire du souverain sacrificateur venaient et vidaient le coffre ; puis ils le rapportaient et le remettaient à sa place. Ils faisaient ainsi jour après jour, et ils recueillaient de l'argent en abondance.
\VS{12}Le roi et Jehojada le donnaient à ceux qui étaient chargés de l'ouvrage pour le service de la maison de Yahweh, et ceux-ci engageaient des tailleurs de pierres et des charpentiers pour réparer la maison de Yahweh, et aussi des ouvriers pour le fer et l'airain, afin de réparer la maison de Yahweh.
\VS{13}Ceux qui étaient chargés de l'ouvrage travaillèrent donc ; et par leurs mains, les travaux s'exécutèrent, de sorte qu'ils rétablirent la maison de Dieu en son état, et l'affermirent.
\VS{14}Lorsqu'ils eurent achevé, ils apportèrent devant le roi et devant Jehojada le reste de l'argent ; et on fit faire des ustensiles pour la maison de Yahweh, des ustensiles pour le service et pour les holocaustes, des coupes et d'autres ustensiles d'or et d'argent. Et on offrit continuellement des holocaustes dans la maison de Yahweh, tant que vécut Jehojada.
\TextTitle{Mort de Jehojada, Joas abandonne Yahweh\FTNTT{2 R. 12:9-16}}
\VS{15}Or Jehojada devint vieux et rassasié de jours, et il mourut ; il était âgé de cent trente ans quand il mourut.
\VS{16}On l'ensevelit dans la cité de David avec les rois ; car il avait fait du bien à Israël, et à l'égard de Dieu et de sa maison.
\VS{17}Mais, après la mort de Jehojada, les chefs de Juda vinrent et se prosternèrent devant le roi ; et le roi les écouta.
\VS{18}Ils abandonnèrent la maison de Yahweh, le Dieu de leurs pères, et ils servirent les idoles d'Astarté et les faux dieux ; et la colère de Yahweh fut sur Juda et sur Jérusalem, parce qu'ils s'étaient ainsi rendus coupables.
\VS{19}Yahweh envoya parmi eux des prophètes, pour les faire retourner à lui par leurs avertissements ; mais ils ne voulurent point les écouter.
\VS{20}Alors l'Esprit de Dieu revêtit Zacharie, fils de Jehojada, le sacrificateur, et se tenant devant le peuple, il leur dit : Dieu m'a parlé ainsi : Pourquoi transgressez-vous les commandements de Yahweh ? Vous ne prospérerez point ; car vous avez abandonné Yahweh, et il vous abandonnera aussi.
\VS{21}Mais ils se liguèrent contre lui et le lapidèrent, par ordre du roi, dans le parvis de la maison de Yahweh.
\VS{22}Ainsi le roi Joas ne se souvint point de la bonté dont Jehojada, père de Zacharie, avait usé envers lui ; et il tua son fils, qui dit en mourant : Yahweh le voit, et il en demandera compte !
\TextTitle{Invasion des Syriens, conspiration et mort de Joas\FTNTT{2 R. 12:17-21 ; cp. 2 R. 13:7}}
\VS{23}A la fin de cette année-là, l'armée de Syrie monta contre Joas, et vint en Juda et à Jérusalem. Ils détruisirent, d'entre le peuple, tous les chefs du peuple, et ils envoyèrent au roi de Damas tout leur butin.
\VS{24}Et quoique l'armée venue de Syrie fût peu nombreuse, Yahweh livra entre leurs mains une armée très nombreuse, parce qu'ils avaient abandonné Yahweh, le Dieu de leurs pères. Ainsi les Syriens furent le châtiment de Joas.
\VS{25}Quand ils s'éloignèrent de lui, après l'avoir laissé dans de grandes souffrances, ses serviteurs conspirèrent contre lui, à cause du sang des fils de Jehojada, le sacrificateur ; ils le tuèrent sur son lit, et il mourut. On l'ensevelit dans la cité de David, mais on ne l'ensevelit pas dans les sépulcres des rois.
\VS{26}Ce sont ici ceux qui conspirèrent contre lui : Zabad, fils de Schimeath, femme ammonite, et Jozabad, fils de Schimrith, femme Moabite.
\VS{27}Quant à ses fils et à la grande charge qui reposa sur lui, et à la réparation de la maison de Dieu, voici, ces choses sont écrites dans les mémoires du livre des rois. Amatsia, son fils, régna à sa place.
\Chap{25}
\TextTitle{Amatsia règne sur Juda\FTNTT{2 R. 12:21 ; 2 R. 14:1-6}}
\VerseOne{}Amatsia devint roi à l'âge de vingt-cinq ans, et il régna vingt-neuf ans à Jérusalem. Sa mère avait pour nom Joaddan, de Jérusalem\FTNT{2 R. 12:21 ; 2 R. 14:1-20}.
\VS{2}Il fit ce qui est droit aux yeux de Yahweh, mais non d'un cœur entier.
\VS{3}Après qu'il fut affermi dans son règne, il fit mourir ses serviteurs qui avaient tué le roi, son père.
\VS{4}Mais il ne fit pas mourir leurs fils ; car il fit selon ce qui est écrit dans la loi, dans le livre de Moïse, où Yahweh a donné ce commandement : Les pères ne mourront point pour les fils, et les fils ne mourront point pour les pères ; mais chacun mourra pour son péché.
\TextTitle{Amatsia en guerre contre les Edomites, sa victoire\FTNTT{2 R. 14:7}}
\VS{5}Puis Amatsia rassembla ceux de Juda, et il les rangea selon les maisons paternelles, par chefs de milliers et par chefs de centaines, pour tout Juda et Benjamin ; il en fit le dénombrement depuis l'âge de vingt ans et au-dessus ; et il trouva trois cent mille hommes d'élite, propres à l'armée, maniant la lance et le bouclier.
\VS{6}Il prit aussi à sa solde, pour cent talents d'argent, cent mille vaillants hommes de guerre d'Israël.
\VS{7}Mais un homme de Dieu vint à lui, et lui dit : Ô roi ! Que l'armée d'Israël ne marche point avec toi ; car Yahweh n'est point avec Israël ni avec tous ces fils d'Ephraïm.
\VS{8}Si tu vas avec eux, quand bien même tu ferais de vaillants combats, Dieu te fera tomber devant l'ennemi ; car Dieu a la puissance d'aider et de faire tomber.
\VS{9}Amatsia dit à l'homme de Dieu : Mais que faire des cent talents que j'ai donnés à la troupe d'Israël ? L'homme de Dieu dit : Yahweh peut t'en donner beaucoup plus.
\VS{10}Ainsi Amatsia sépara les troupes qui lui étaient venues d'Ephraïm, et les fit retourner chez elles ; mais leur colère s'enflamma très ardemment contre Juda, et ces gens retournèrent chez eux dans une grande colère.
\VS{11}Alors Amatsia prit courage, conduisit son peuple et s'en alla dans la vallée du sel, où il battit dix mille hommes des fils de Séir.
\VS{12}Les fils de Juda prirent dix mille hommes vivants, et les ayant amenés sur le sommet d'une roche, ils les jetèrent du haut de la roche, de sorte qu'ils furent tous brisés.
\VS{13}Mais les gens de la troupe qu'Amatsia avait renvoyée, afin qu'ils n'aillent pas avec lui à la guerre, firent une incursion dans les villes de Juda, depuis Samarie jusqu'à Beth-Horon. Ils y tuèrent trois mille personnes et emportèrent un gros butin.
\TextTitle{Idolâtrie d'Amatsia\FTNTT{2 R. 14:7}}
\VS{14}Lorsqu'Amatsia fut de retour de la défaite des Edomites, et ayant apporté les dieux des fils de Séir, il se les établit pour dieux ; il se prosterna devant eux et leur brûla de l'encens.
\VS{15}Et la colère de Yahweh s'enflamma contre Amatsia, et il envoya vers lui un prophète qui lui dit : Pourquoi as-tu recherché les dieux d'un peuple qui n'ont point délivré leur peuple de ta main ?
\VS{16}Et comme il parlait au roi, il lui répondit : T'a-t-on établi conseiller du roi ? Cesse maintenant ! Pourquoi veux-tu qu'on te tue ? Et le prophète se retira, mais en disant : Je sais que Dieu a résolu de te détruire, parce que tu as fait cela, et que tu n'as point écouté mon conseil.
\TextTitle{Défaite d'Amatsia contre Israël\FTNTT{2 R. 14:8-14}}
\VS{17}Puis Amatsia, roi de Juda, ayant tenu conseil, envoya vers Joas, fils de Joachaz, fils de Jéhu, roi d'Israël, pour lui dire : Viens, voyons-nous en face !
\VS{18}Mais Joas, roi d'Israël, envoya dire à Amatsia, roi de Juda : L'épine du Liban envoya dire au cèdre du Liban : Donne ta fille pour femme à mon fils ! Et les bêtes sauvages qui sont au Liban passèrent et foulèrent l'épine.
\VS{19}Voici, tu dis que tu as frappé les Edomites, et ton cœur s'est élevé pour te glorifier. Maintenant, reste dans ta maison ! Pourquoi t'engagerais-tu dans un combat où tu tomberais, et Juda avec toi ?
\VS{20}Mais Amatsia ne l'écouta point ; Dieu avait résolu de le livrer aux mains de Joas parce qu'il eût recours aux dieux d'Edom.
\VS{21}Joas, roi d'Israël, monta ; et ils se virent en face, lui et Amatsia, roi de Juda, à Beth-Schémesch, qui est de Juda.
\VS{22}Juda fut battu en face d'Israël, et chacun s'enfuit dans sa tente.
\VS{23}Joas, roi d'Israël, prit Amatsia, roi de Juda, fils de Joas, fils de Joachaz, à Beth-Schémesch. Il l'emmena à Jérusalem et fit une brèche de quatre cents coudées dans la muraille de Jérusalem, depuis la porte d'Ephraïm jusqu'à la porte de l'angle.
\VS{24}Il prit l'or, l'argent, tous les vases qui se trouvaient dans la maison de Dieu sous la garde d'Obed-Edom, les trésors de la maison du roi ; il fit des otages et il retourna à Samarie.
\TextTitle{Assassinat d'Amatsia\FTNTT{2 R. 14:17-20}}
\VS{25}Amatsia, fils de Joas, roi de Juda, vécut quinze ans, après que Joas, fils de Joachaz, roi d'Israël, mourut.
\VS{26}Le reste des actions d'Amatsia, les premières et les dernières, voici cela n'est-il pas écrit dans le livre des rois de Juda et d'Israël ?
\VS{27}Or depuis le moment où Amatsia se détourna de Yahweh, on fit une conspiration contre lui à Jérusalem, et il s'enfuit à Lakis ; mais on le poursuivit à Lakis, et on le fit mourir.
\VS{28}Puis on le transporta sur des chevaux, et on l'ensevelit avec ses pères dans la ville de Juda.
\Chap{26}
\TextTitle{Ozias règne sur Juda ; il est fidèle à Yahweh\FTNTT{2 R. 14:21-15:4}}
\VerseOne{}Alors, tout le peuple de Juda prit Ozias, âgé de seize ans, et l'établit roi à la place de son père Amatsia\FTNT{2 R. 14:21 ; 2 R. 15:1-4.}.
\VS{2}Ce fut lui qui bâtit Eloth, et la ramena sous la puissance de Juda, après que le roi se fut endormi avec ses pères.
\VS{3}Ozias était âgé de seize ans quand il devint roi, et il régna cinquante-deux ans à Jérusalem. Sa mère avait pour nom Jecolia, de Jérusalem.
\VS{4}Il fit ce qui est droit aux yeux de Yahweh, comme avait fait Amatsia, son père.
\VS{5}Il s'appliqua à rechercher Dieu pendant les jours de Zacharie, qui avait une intelligence dans les visions de Dieu et pendant les jours où il rechercha Yahweh, Dieu le fit prospérer.
\VS{6}Il sortit et fit la guerre contre les Philistins. Il brisa la muraille de Gath, la muraille de Jabné, et la muraille d'Asdod ; et il bâtit des villes dans le pays d'Asdod et chez les Philistins.
\VS{7}Dieu le secourut contre les Philistins et contre les Arabes qui habitaient à Gur-Baal, et contre les Maonites.
\VS{8}Même les Ammonites faisaient des présents à Ozias, et sa renommée parvint jusqu'à l'entrée de l'Egypte ; car il était devenu très puissant.
\VS{9}Ozias bâtit des tours à Jérusalem, sur la porte de l'angle, sur la porte de la vallée, sur l'angle, et il les fortifia.
\VS{10}Il bâtit des tours dans le désert, et il creusa de nombreux puits, parce qu'il avait de nombreux troupeaux dans la plaine et dans la campagne, des laboureurs et des vignerons sur les montagnes, et au Carmel ; car il aimait l'agriculture.
\VS{11}Ozias avait une armée de gens de guerre, allant à la guerre par bandes, selon le compte de leur dénombrement fait par Jeïel le scribe, et Maaséja le commissaire, et sous la conduite de Hanania l'un des chefs du roi.
\VS{12}Le nombre total des chefs des maisons paternelles, des vaillants guerriers, était de deux mille six cents.
\VS{13}Il y avait sous leur conduite une armée de trois cent sept mille cinq cents combattants, tous gens de guerre, puissants et vaillants, capables de soutenir le roi contre l'ennemi.
\VS{14}Ozias leur procura, pour toute l'armée, des boucliers, des lances, des casques, des cuirasses, des arcs et des pierres de fronde.
\VS{15}Il fit faire à Jérusalem des machines inventées par un ingénieur, pour être placées sur les tours et sur les angles, pour lancer des flèches et de grosses pierres. Et sa renommée se répandit au loin ; car il fut extraordinairement soutenu, jusqu'à ce qu'il devienne fort puissant.
\TextTitle{Ozias pèche et est frappé de lèpre\FTNTT{2 R. 15:5-7, 32}}
\VS{16}Mais dès qu'il fut puissant, son cœur s'éleva pour le corrompre. Et il pécha contre Yahweh, son Dieu : Il entra dans le temple de Yahweh pour brûler des parfums sur l'autel des parfums\FTNT{2 R. 15:5-7.}.
\VS{17}Mais Azaria le sacrificateur, entra après lui, et avec lui quatre-vingts sacrificateurs de Yahweh, hommes vaillants,
\VS{18}qui s'opposèrent au roi Ozias, et lui dirent : Ce n'est pas à toi, Ozias, d'offrir le parfum à Yahweh, mais aux sacrificateurs, fils d'Aaron, qui sont consacrés pour cela. Sors du sanctuaire, car tu as péché ! Et cela ne sera pas à ta gloire devant Yahweh Dieu.
\VS{19}Alors Ozias, qui avait à la main un encensoir pour faire brûler le parfum, se mit en colère. Et comme il s'irritait contre les sacrificateurs, la lèpre parut sur son front, en présence des sacrificateurs, dans la maison de Yahweh, près de l'autel des parfums.
\VS{20}Azaria, le principal sacrificateur, le regarda ainsi que tous les sacrificateurs. Et voici, il avait de la lèpre sur le front. Ils le pressèrent et lui-même se hâta de sortir, parce que Yahweh l'avait frappé.
\VS{21}Le roi Ozias fut ainsi lépreux jusqu'au jour de sa mort ; il habita seul comme lépreux dans une maison écartée, car il était exclu de la maison de Yahweh. Et Jotham, son fils, avait la charge de la maison du roi, jugeant le peuple du pays.
\VS{22}Esaïe, fils d'Amots, le prophète, a écrit le reste des actions d'Ozias, les premières et les dernières.
\VS{23}Ozias s'endormit avec ses pères, et on l'ensevelit avec ses pères dans le champ de la sépulture des rois ; car on dit : Il est lépreux. Et Jotham, son fils, régna à sa place.
\Chap{27}
\TextTitle{Jotham règne sur Juda ; sa mort\FTNTT{2 R. 15:7, 32-38}}
\VerseOne{}Jotham était âgé de vingt-cinq ans quand il devint roi, et il régna seize ans à Jérusalem\FTNT{1 R. 15:7.}. Sa mère avait le nom de Jeruscha, fille de Tsadok.
\VS{2}Il fit ce qui est droit aux yeux de Yahweh, tout comme Ozias, son père, avait fait ; mais il n'entra pas dans le temple de Yahweh. Néanmoins, le peuple se corrompait encore.
\VS{3}Ce fut lui qui bâtit la porte supérieure de la maison de Yahweh, et il fit beaucoup de constructions sur les murs de la colline.
\VS{4}Il bâtit des villes sur les montagnes de Juda, des châteaux et des tours dans les forêts.
\VS{5}Il fut en guerre avec le roi des fils d'Ammon, et fut le plus fort. Cette année-là, les fils d'Ammon lui donnèrent cent talents d'argent, dix mille cors de froment, et dix mille d'orge. Les fils d'Ammon lui en donnèrent autant la seconde et la troisième année.
\VS{6}Jotham devint donc très puissant, parce qu'il avait affermi ses voies devant Yahweh, son Dieu.
\VS{7}Le reste des actions de Jotham, tous ses combats et sa conduite, voici, toutes ces choses sont écrites dans le livre des rois d'Israël et de Juda.
\VS{8}Il était âgé de vingt-cinq ans quand il devint roi, et il régna seize ans à Jérusalem.
\VS{9}Puis Jotham s'endormit avec ses pères, et on l'ensevelit dans la cité de David. Et Achaz, son fils, régna à sa place.
\Chap{28}
\TextTitle{Achaz règne sur Juda\FTNTT{2 R. 15:38-16:4}}
\VerseOne{}Achaz était âgé de vingt ans quand il devint roi, et il régna seize ans à Jérusalem\FTNT{2 R. 15:38 ; 2 R. 16:1-4.}. Il ne fit point ce qui est droit aux yeux de Yahweh, comme David, son père.
\VS{2}Il suivit la voie des rois d'Israël ; et il fit même des images de fonte pour les Baals.
\VS{3}Il brûla des parfums dans la vallée du fils de Hinnom, et il brûla ses fils au feu, suivant les abominations des nations que Yahweh avait chassées devant les enfants d'Israël.
\VS{4}Il offrait aussi des sacrifices et brûlait des parfums dans les hauts lieux, sur les collines, et sous tout arbre vert.
\TextTitle{La Syrie et Israël envahissent Juda\FTNTT{2 R. 16:5-6}}
\VS{5}C'est pourquoi Yahweh, son Dieu, le livra entre les mains du roi de Syrie. Les Syriens le battirent et lui prirent un grand nombre de prisonniers, qu'ils emmenèrent à Damas. Il fut livré aussi entre les mains du roi d'Israël, qui lui fit endurer une grande défaite\FTNT{2 R . 16:5-20.}.
\VS{6}Car Pékach, fils de Remalia, tua en un seul jour en Juda cent vingt mille hommes, tous vaillants, parce qu'ils avaient abandonné Yahweh, le Dieu de leurs pères.
\VS{7}Zicri, homme vaillant d'Ephraïm, tua Maaséja, fils du roi, et Azrikam, chef de la maison, et Elkana, le second après le roi.
\VS{8}Les fils d'Israël emmenèrent prisonniers deux cent mille de leurs frères, tant femmes que fils et filles ; ils firent aussi sur eux un gros butin. Ils emmenèrent le butin à Samarie.
\TextTitle{Les captifs de Juda libérés grâce à Obed}
\VS{9}Or il y avait un prophète de Yahweh nommé Oded. Il sortit au-devant de cette armée qui revenait à Samarie, et leur dit : Voici, Yahweh, le Dieu de vos pères, étant indigné contre Juda, les a livrés entre vos mains, et vous les avez tués avec une colère telle qu'elle est parvenue aux cieux.
\VS{10}Et maintenant, vous pensez assujettir les fils de Juda et de Jérusalem pour serviteurs et pour servantes ! Mais n'êtes-vous pas également coupables envers Yahweh, votre Dieu ?
\VS{11}Maintenant écoutez-moi, et ramenez les prisonniers que vous vous êtes faits parmi vos frères ; car la colère ardente de Yahweh est sur vous.
\VS{12}Alors quelques-uns des chefs des fils d'Ephraïm, Azaria, fils de Jochanan, Bérékia, fils de Meschillémoth, Ezéchias, fils de Schallum, et Amasa, fils de Hadlaï, s'élevèrent contre ceux qui retournaient de la guerre,
\VS{13}et leur dirent : Vous ne ferez point entrer ici ces captifs. C'est pour nous rendre coupables devant Yahweh, voulez-vous en rajouter à nos péchés et à notre culpabilité ; car nous sommes déjà grandement coupables, et une colère ardente est sur Israël.
\VS{14}Alors les soldats abandonnèrent les captifs et le butin devant les chefs et toute l'assemblée.
\VS{15}Et des hommes, désignés par leurs noms, se levèrent, prirent les captifs, utilisèrent le butin pour revêtir tous ceux d'entre eux qui étaient nus avec des vêtements et des chaussures. Ils leur donnèrent à manger et à boire, les oignirent et ils conduisirent sur des ânes tous ceux qui étaient affaiblis pour les emmener à Jéricho, la ville des palmiers, auprès de leurs frères ; puis ils s'en retournèrent à Samarie.
\TextTitle{Achaz fait appel aux Assyriens\FTNTT{2 R. 15:29 ; 16:7-18}}
\VS{16}En ce temps-là, le roi Achaz envoya demander du secours aux rois d'Assyrie.
\VS{17}Les Edomites étaient revenus, avaient battu Juda et avaient emmené des prisonniers.
\VS{18}Les Philistins s'étaient aussi jetés sur les villes de la plaine et du sud de Juda ; et ils avaient pris Beth-Schémesch, Ajalon, Guedéroth, Soco et les villes de son ressort, Thimna et les villes de son ressort, Guimzo et les villes de son ressort, et ils y demeurèrent.
\VS{19}Car Yahweh humilia Juda, à cause d'Achaz, roi d'Israël, parce qu'il avait mis le désordre en Juda, et qu'il avait commis des transgressions contre Yahweh.
\VS{20}Tilgath-Pilnéser, roi d'Assyrie, vint vers lui ; mais il l'assiégea, et ne le fortifia pas.
\VS{21}Or Achaz dépouilla la maison de Yahweh, la maison du roi et celle des chefs, pour faire des dons au roi d'Assyrie, mais sans avoir du secours.
\TextTitle{Achaz irrite Yahweh par ses péchés}
\VS{22}Dans le temps de sa détresse, il continua à pécher contre Yahweh, lui, le roi Achaz.
\VS{23}Il sacrifia aux dieux de Damas qui l'avaient battu, et il dit : Puisque les dieux des rois de Syrie leur viennent en aide, je leur sacrifierai, afin qu'ils me viennent en aide. Mais ils furent la cause de sa chute et de celle de tout Israël.
\VS{24}Or Achaz rassembla les ustensiles de la maison de Dieu, et il mit en pièces les ustensiles de la maison de Dieu. Il ferma les portes de la maison de Yahweh, et se fit des autels dans tous les coins de Jérusalem.
\VS{25}Il fit des hauts lieux dans chaque ville de Juda, pour offrir des parfums à d'autres dieux ; et il irrita Yahweh, le Dieu de ses pères.
\TextTitle{Mort d'Achaz\FTNTT{2 R. 16:19-20}}
\VS{26}Quant au reste de ses actions et de toutes ses voies, les premières et les dernières, voici, elles sont écrites dans le livre des rois de Juda et d'Israël.
\VS{27}Puis Achaz s'endormit avec ses pères, et on l'ensevelit dans la ville de Jérusalem ; car on ne le mit point dans les sépulcres des rois d'Israël. Et Ezéchias, son fils, régna à sa place.
\Chap{29}
\TextTitle{Ezéchias règne sur Juda ; le réveil du peuple\FTNTT{2 R. 18:1-7 ; cp. Es. 36-39}}
\VerseOne{}Ezéchias devint roi à l'âge de vingt-cinq ans, et il régna vingt-neuf ans à Jérusalem\FTNT{ Es. 36 ; Es. 37 ; Es. 38, Es. 39 ; 2 R. 18:1-7 ;.}. Sa mère avait pour nom Abija, fille de Zacharie.
\VS{2}Il fit ce qui est droit aux yeux de Yahweh, tout comme avait fait David, son père.
\VS{3}La première année de son règne, au premier mois, il ouvrit les portes de la maison de Yahweh, et il les répara.
\VS{4}Il fit venir les sacrificateurs et les Lévites, et les rassembla dans la place orientale.
\VS{5}Et il leur dit : Ecoutez-moi, Lévites ! Sanctifiez-vous et sanctifiez la maison de Yahweh, le Dieu de vos pères, et ôtez du sanctuaire tout ce qui est impur.
\VS{6}Car nos pères ont péché, ils ont fait ce qui est mal aux yeux de Yahweh, notre Dieu. Ils l'ont abandonné, ils ont détourné leurs faces du tabernacle de Yahweh et lui ont tourné le dos.
\VS{7}Ils ont même fermé les portes du portique et ont éteint les lampes, ils n'ont fait ni monter d'offrande, ni brûler du parfum et des holocaustes au Dieu d'Israël dans le sanctuaire.
\VS{8}C'est pourquoi la colère de Yahweh a été sur Juda et sur Jérusalem ; et il les a livrés à de grands troubles, à la ruine et à la moquerie, comme vous le voyez de vos yeux.
\VS{9}Car voici, nos pères sont tombés par l'épée, et nos fils, nos filles et nos femmes sont en captivité.
\VS{10}Maintenant donc j'ai à cœur de traiter alliance avec Yahweh, le Dieu d'Israël, pour que son ardente colère se détourne de nous.
\VS{11}Or mes fils, cessez d'être négligents ; car Yahweh vous a choisis, afin que vous vous teniez devant lui à son service, comme ses serviteurs, pour lui brûler des parfums.
\VS{12}Les Lévites se levèrent : Machath, fils d'Amasaï, Joël, fils d'Azaria, des fils des Kehathites ; et des fils des Merarites, Kis, fils d'Abdi, Azaria, fils de Jehalléleel ; et des Guerschonites, Joach, fils de Zimma, et Eden, fils de Joach ;
\VS{13}et des fils d'Elitsaphan, Schimri et Jeïel ; et des fils d'Asaph, Zacharie et Matthania ;
\VS{14}et des fils d'Héman, Jehiel et Schimeï, et des fils de Jeduthun, Schemaeja et Uzziel.
\VS{15}Ils assemblèrent leurs frères, et ils se sanctifièrent ; puis ils entrèrent selon l'ordre du roi, et d'après la parole de Yahweh, pour purifier la maison de Yahweh.
\VS{16}Ainsi les sacrificateurs entrèrent à l'intérieur de la maison de Yahweh pour la purifier. Ils firent sortir dans le parvis de la maison de Yahweh toutes les impuretés qu'ils trouvèrent dans le temple de Yahweh. Les Lévites les prirent pour les emporter dehors, au torrent de Cédron.
\VS{17}Ils commencèrent à sanctifier le temple le premier jour du premier mois. Le huitième jour du mois, ils entrèrent au portique de Yahweh, et ils sanctifièrent la maison de Yahweh pendant huit jours. Le seizième jour du premier mois, ils avaient achevé.
\VS{18}Puis ils se rendirent chez le roi Ezéchias, et dirent : Nous avons purifié toute la maison de Yahweh, l'autel des holocaustes et ses ustensiles, la table des pains de proposition et ses ustensiles\FTNT{Ex. 29.}.
\VS{19}Nous avons remis en état et sanctifié tous les ustensiles que le roi Achaz avait rendus odieux pendant son règne, par ses transgressions ; ils sont maintenant devant l'autel de Yahweh.
\TextTitle{Nouvelle consécration du temple}
\VS{20}Alors le roi Ezéchias se leva de bonne heure, rassembla les chefs de la ville, et monta à la maison de Yahweh.
\VS{21}Ils amenèrent sept taureaux, sept béliers, sept agneaux et sept boucs sans défaut, en sacrifice pour le péché, pour le royaume, pour le sanctuaire et pour Juda\FTNT{Lé 4:3-26}. Puis le roi dit aux sacrificateurs, fils d'Aaron, de les faire monter en offrande sur l'autel de Yahweh.
\VS{22}Ils égorgèrent donc les bœufs, et les sacrificateurs recueillirent le sang et en aspergèrent l'autel ; ils égorgèrent les béliers et aspergèrent le sang sur l'autel ; ils égorgèrent les agneaux et aspergèrent le sang sur l'autel.
\VS{23}Puis on fit approcher les boucs pour le sacrifice du péché, devant le roi et devant l'assemblée, et ils posèrent leurs mains sur eux\FTNT{Lé 8:14.}.
\VS{24}Alors les sacrificateurs les égorgèrent, et offrirent en expiation leur sang vers l'autel, afin de faire propitiation pour tout Israël ; car le roi avait ordonné cet holocauste et ce sacrifice d'expiation pour tout Israël.
\VS{25}Il plaça aussi les Lévites dans la maison de Yahweh, avec des cymbales, des luths et des harpes, comme l'avait ordonné David, Gad, le voyant du roi, et Nathan le prophète ; car c'était un commandement de Yahweh, par ses prophètes.
\VS{26}Les Lévites se tinrent donc là avec les instruments de David, et les sacrificateurs avec les trompettes.
\VS{27}Alors Ezéchias ordonna de faire monter en offrande l'holocauste sur l'autel ; et au moment où commença l'holocauste, le cantique de Yahweh commença aussi, avec les trompettes et les instruments de David, roi d'Israël.
\VS{28}Toute l'assemblée se prosterna en chantant le cantique, et les trompettes sonnèrent ; et cela continua jusqu'à ce que l'holocauste fût achevé.
\VS{29}Et quand on eut achevé de faire monter l'holocauste, le roi et tous ceux qui se trouvaient avec lui fléchirent les genoux et se prosternèrent.
\VS{30}Puis le roi Ezéchias et les chefs dirent aux Lévites de célébrer Yahweh par les paroles de David et d'Asaph le voyant ; et ils le célébrèrent dans des réjouissances, et s'inclinèrent pour se prosterner.
\VS{31}Alors Ezéchias prit la parole, et dit : Vous avez maintenant consacré vos mains à Yahweh. Approchez-vous, amenez des sacrifices et faites des sacrifices de reconnaissances dans la maison de Yahweh. Et l'assemblée amena des sacrifices et firent des sacrifices de reconnaissances, et tous ceux qui étaient d'un cœur volontaire offrirent des holocaustes.
\VS{32}Le nombre des holocaustes que l'assemblée offrit fut de soixante-dix taureaux, cent béliers, deux cents agneaux, le tout en holocauste à Yahweh.
\VS{33}Et les autres choses consacrées furent, six cents bœufs, et trois brebis moutons.
\VS{34}Mais ils étaient peu de sacrificateurs et ne purent pas dépouiller tous les holocaustes ; les Lévites, leurs frères, les aidèrent jusqu'à ce que cette œuvre fut achevée, et jusqu'à ce que les autres sacrificateurs se fussent sanctifiés ; car les Lévites avaient eu plus à cœur de se sanctifier que les sacrificateurs.
\VS{35}Il y eut aussi un grand nombre d'holocaustes, avec les graisses des offrande de paix et avec les libations des holocaustes. Ainsi, le service de la maison de Yahweh fut rétabli.
\VS{36}Ezéchias et tout le peuple se réjouirent de ce que Dieu avait ainsi disposé le peuple ; car les choses se firent instantanément.
\Chap{30}
\TextTitle{Rétablissement de la Pâque}
\VerseOne{}Puis Ezéchias envoya dire à tout Israël et à Juda ; et il écrivit aussi des lettres à Ephraïm et à Manassé, pour les faire venir à la maison de Yahweh à Jérusalem, pour célébrer la Pâque en l'honneur de Yahweh, le Dieu d'Israël.
\VS{2}Le roi, ses chefs et toute l'assemblée avaient tenu un conseil à Jérusalem afin de célébrer la Pâque au second mois\FTNT{No. 9:10-11.} ;
\VS{3}car on ne pouvait la célébrer au temps ordinaire, parce qu'il n'y avait pas un nombre suffisant de sacrificateurs sanctifiés, et que le peuple n'était pas rassemblé à Jérusalem.
\VS{4}Le roi vit cela d'un bon œil ainsi que toute l'assemblée.
\VS{5}Ils décidèrent de faire une publication dans tout Israël, depuis Beer-Schéba jusqu'à Dan, pour que l'on vienne à Jérusalem célébrer la Pâque à Yahweh, le Dieu d'Israël. Car elle n'était pas célébrée par la multitude depuis longtemps conformément à ce qui était écrit.
\VS{6}Les coureurs allèrent donc avec des lettres de la part du roi et de ses chefs, partout en Israël et en Juda. Selon que le roi l'avait ordonné, ils disaient : Enfants d'Israël, retournez à Yahweh, le Dieu d'Abraham, d'Isaac et d'Israël, et afin qu'il revienne vers vous, qui êtes le reste échappé de la main des rois d'Assyrie.
\VS{7}Ne soyez pas comme vos pères ni comme vos frères, qui ont péché contre Yahweh, le Dieu de leurs pères, c'est pourquoi il les a livrés à la désolation, comme vous le voyez.
\VS{8}Maintenant, ne raidissez pas votre cou comme vos pères. Tendez les mains vers Yahweh, venez à son sanctuaire consacré pour toujours, servez Yahweh, votre Dieu, et son ardente colère se détournera de vous.
\VS{9}Car si vous revenez à Yahweh, vos frères et vos fils trouveront grâce auprès de ceux qui les ont emmenés captifs, et ils reviendront en ce pays parce que Yahweh, votre Dieu, est compatissant et miséricordieux ; et il ne détournera point sa face de vous, si vous revenez à lui.
\VS{10}Les coureurs passaient ainsi de ville en ville, par le pays d'Ephraïm et de Manassé jusqu'à Zabulon ; mais on riait et on se moquait d'eux.
\VS{11}Toutefois, quelques-uns d'Aser, de Manassé et de Zabulon s'humilièrent, et vinrent à Jérusalem.
\VS{12}La main de Dieu fut aussi sur Juda, pour leur donner un même cœur, afin d'exécuter l'ordre du roi et des chefs, selon la parole de Yahweh.
\VS{13}C'est pourquoi il s'assembla un grand peuple à Jérusalem pour célébrer la fête des pains sans levain\FTNT{Ex. 12:15 ; Lé. 23:6.}, au second mois. Ce fut une très grande assemblée.
\VS{14}Ils se levèrent et ôtèrent les autels qui étaient à Jérusalem ; ils ôtèrent aussi tous ceux où l'on brûlait de l'encens, et ils les jetèrent dans le torrent de Cédron.
\VS{15}Puis on immola la Pâque, au quatorzième jour du second mois ; car les sacrificateurs et les Lévites avaient eu honte et s'étaient sanctifiés, et ils amenèrent les holocaustes dans la maison de Yahweh.
\VS{16}Ils se tinrent à leur poste, selon leur charge, d'après la loi de Moïse, homme de Dieu. Et les sacrificateurs répandaient le sang qu'ils recevaient des mains des Lévites.
\VS{17}Car il y en avait un grand nombre dans cette assemblée qui ne s'étaient pas sanctifiés ; c'est pourquoi les Lévites eurent la charge d'immoler la Pâque pour tous ceux qui n'étaient pas purs, afin de les consacrer à Yahweh.
\VS{18}Car une grande partie du peuple, à savoir la plupart de ceux d'Ephraïm, de Manassé, d'Issacar et de Zabulon, ne s'étaient pas purifiés et mangèrent la Pâque contrairement à ce qui est écrit. Mais Ezéchias pria pour eux, en disant : Que Yahweh, qui est bon, tienne la propitiation pour faite,
\VS{19}pour quiconque a disposé son cœur à rechercher Dieu Yahweh, le Dieu de leurs pères, bien qu'il ne soit pas purifié conformément au sanctuaire !
\VS{20}Yahweh exauça Ezéchias, et guérit le peuple.
\VS{21}Les enfants d'Israël qui se trouvèrent à Jérusalem célébrèrent donc la fête des pains sans levain, pendant sept jours, dans une grande réjouissance ; les Lévites et les sacrificateurs célébraient Yahweh chaque jour, avec les instruments qui retentissaient à la louange de Yahweh.
\VS{22}Ezéchias parla au cœur de tous les Lévites, qui prêtaient une grande attention et de l'intelligence au service de Yahweh. Ils mangèrent pendant la fête, sept jours durant, offrant des sacrifices d'offrande de paix, et louant Yahweh, le Dieu de leurs pères.
\TextTitle{Sept jours supplémentaires pour la Pâque}
\VS{23}Puis toute l'assemblée fut d'avis de célébrer sept autres jours. Et ils célébrèrent ces sept jours dans la joie.
\VS{24}Car Ezéchias, roi de Juda, offrit à l'assemblée mille taureaux et sept mille brebis ; et les chefs donnèrent à l'assemblée mille taureaux et dix mille brebis ; et des sacrificateurs en grand nombre s'étaient sanctifiés.
\VS{25}Toute l'assemblée de Juda, avec les sacrificateurs et les Lévites, et toute l'assemblée venue d'Israël, ainsi que les étrangers venus du pays d'Israël, et ceux qui habitaient en Juda, se réjouirent.
\VS{26}Il y eut une grande joie à Jérusalem ; car depuis le temps de Salomon, fils de David, roi d'Israël, il ne s'était pas fait une telle chose dans Jérusalem.
\VS{27}Puis les sacrificateurs et les Lévites se levèrent et bénirent le peuple, et leur voix fut entendue, leur prière parvint jusqu'aux cieux, jusqu'à la sainte demeure de Yahweh.
\Chap{31}
\TextTitle{Destruction des idoles et organisation des services du temple}
\VerseOne{}Lorsque tout cela fut achevé, tous ceux d'Israël qui s'étaient retrouvés là, allèrent dans les villes de Juda, et brisèrent les statues, abattirent les idoles d'Astarté et renversèrent les hauts lieux et les autels, dans tout Juda et Benjamin, dans Ephraïm et Manassé, jusqu'à détruire tout\FTNT{2 R. 18:4.}. Puis tous les enfants d'Israël retournèrent dans leurs villes, chacun dans sa possession.
\VS{2}Et Ezéchias rétablit les classes des sacrificateurs et des Lévites, selon leur partage, chacun suivant sa charge, tant les sacrificateurs que les Lévites, pour les holocaustes et les offrandes de paix, pour faire le service, célébrer et chanter les louanges aux portes du camp de Yahweh.
\VS{3}Le roi donna une portion de ses biens pour les holocaustes, pour les holocaustes du matin et du soir, pour les holocaustes des sabbats, des nouvelles lunes et des fêtes, comme cela est écrit dans la loi de Yahweh.
\VS{4}Il dit au peuple, aux habitants de Jérusalem, de donner la portion des sacrificateurs et des Lévites, afin de s'appliquer à la loi de Yahweh.
\VS{5}Dès que la chose fut publiée, les enfants d'Israël amenèrent en abondance les prémices du blé, du moût, de l'huile, du miel et de tous les produits des champs ; ils apportèrent les dîmes de tout, en abondance.
\VS{6}Les enfants d'Israël et de Juda, qui demeuraient dans les villes de Juda, apportèrent aussi les dîmes du gros et du menu bétail, et les dîmes des choses saintes, qui étaient consacrées à Yahweh, leur Dieu ; et ils les mirent par tas.
\VS{7}Ils commencèrent à former les tas au troisième mois, et ils les achevèrent au septième mois.
\VS{8}Alors Ezéchias et les chefs vinrent voir les tas, et ils bénirent Yahweh et son peuple d'Israël.
\VS{9}Ezéchias interrogea les sacrificateurs et les Lévites au sujet de ces tas.
\VS{10}Le souverain sacrificateur Azaria, de la maison de Tsadok, lui répondit, et parla ainsi : Depuis qu'on a commencé à apporter des offrandes à la maison de Yahweh, nous avons mangé et avons été rassasiés, et il est resté cette grande quantité ; car Yahweh a béni son peuple, et cette grande quantité est le reste.
\VS{11}Alors Ezéchias leur dit de préparer des chambres dans la maison de Yahweh ; et ils les préparèrent.
\VS{12}On y apporta fidèlement les offrandes et les dîmes, les choses consacrées. Conania, le Lévite, en eut l'intendance, et Schimeï, son frère, était son second.
\VS{13}Jehiel, Azazia, Nachath, Asaël, Jerimoth, Jozabad, Eliel, Jismakia, Machath, et Benaja, étaient commis sous l'autorité de Conania et de Schimeï, son frère, d'après l'indication du roi Ezéchias, et d'Azaria, chef de la maison de Dieu.
\VS{14}Koré, le Lévite, fils de Jimna, portier de l'orient, avait la charge des offrandes volontaires offertes à Dieu, pour distribuer l'offrande élevée à Yahweh, et les choses consacrées et saintes.
\VS{15}Il avait sous sa direction Eden, Minjamin, Josué, Schemaeja, Amaria, et Schecania, dans les villes des sacrificateurs, pour distribuer fidèlement les portions à leurs frères, grands et petits, suivant leurs divisions,
\VS{16}à ceux qui étaient enregistrés comme mâles, depuis l'âge de trois ans et au-delà ; à tous ceux qui entraient dans la maison de Yahweh, pour le service quotidien, pour servir dans leurs charges et suivant leurs divisions ;
\VS{17}aux sacrificateurs et aux Lévites enregistrés selon la maison de leurs pères, depuis ceux de vingt ans et au-delà, selon leurs charges et selon leurs divisions ;
\VS{18}à ceux de toute l'assemblée enregistrés avec leurs petits enfants, leurs femmes, leurs fils et leurs filles ; car ils se consacraient avec fidélité aux choses saintes ;
\VS{19}et pour les enfants d'Aaron, les sacrificateurs, qui étaient à la campagne et dans les faubourgs de leurs villes, dans chaque ville, il y avait des gens désignés par leur nom, pour distribuer les portions à tous les mâles des sacrificateurs, et à tous les Lévites enregistrés.
\VS{20}Ezéchias en fit ainsi dans tout Juda ; et il fit ce qui est bon, droit et véritable, devant Yahweh, son Dieu.
\VS{21}Il travailla de tout son cœur et il réussit dans tout l'ouvrage qu'il entreprit pour le service de la maison de Dieu, et pour la loi, et pour les commandements, en recherchant son Dieu.
\Chap{32}
\TextTitle{Menaces de Sanchérib, roi d'Assyrie\FTNTT{2 R. 19:17-37 ; 19:8-13 ; Es. 36:2-20}}
\VerseOne{}Après que ces choses furent bien établies, Sanchérib, roi d'Assyrie, vint et entra en Juda, et campa contre les villes fortes, dans l'intention de faire une brèche\FTNT{Es. 36:2-21 ; 2 R. 18:13-37.}.
\VS{2}Ezéchias, voyant que Sanchérib était venu, et qu'il se tournait vers Jérusalem pour lui faire la guerre,
\VS{3}tint conseil avec ses chefs et ses vaillants hommes pour boucher les sources d'eau qui étaient hors de la ville, et ils l'aidèrent.
\VS{4}Un peuple nombreux s'assembla, et ils bouchèrent toutes les sources et le torrent qui coule par le milieu de la contrée, en disant : Pourquoi les rois d'Assyrie trouveraient-ils à leur venue de l'eau en abondance ?
\VS{5}Il se fortifia et rebâtit toute la muraille où il y avait une brèche, et l'éleva jusqu'aux tours ; il bâtit une autre muraille en dehors ; il répara Millo, dans la cité de David, et il fit faire beaucoup d'armes et de boucliers.
\VS{6}Il donna des chefs de guerre au peuple, les assembla auprès de lui sur la place de la porte de la ville, et parla à leur cœur, en disant :
\VS{7}Fortifiez-vous, soyez forts ! Ne craignez point et ne soyez pas effrayés devant le roi d'Assyrie et toute la multitude qui est avec lui ; car avec nous il y a quelqu'un de plus puissant.
\VS{8}Avec lui est le bras de la chair, mais avec nous est Yahweh, notre Dieu, pour nous aider et pour combattre dans nos combats. Et le peuple s'appuya sur les paroles d'Ezéchias, roi de Juda.
\VS{9}Après cela, Sanchérib, roi d'Assyrie, pendant qu'il était devant Lakis, ayant avec lui toutes les forces de son royaume, envoya ses serviteurs à Jérusalem vers Ezéchias, roi de Juda, et vers tous ceux de Juda qui étaient à Jérusalem, pour leur dire :
\VS{10}Ainsi parle Sanchérib, roi d'Assyrie : Sur qui vous confiez-vous pour que vous restiez à Jérusalem pour y être assiégés ?
\VS{11}Ezéchias ne vous incite-t-il pas pour vous livrer à la mort, par la famine et par la soif, en vous disant : Yahweh, notre Dieu, nous délivrera de la main du roi d'Assyrie ?
\VS{12}Cet Ezéchias n'a-t-il pas ôté les hauts lieux et les autels, et n'a-t-il pas ordonné à Juda et à Jérusalem : Vous vous prosternerez devant un seul autel pour y brûler le parfum ?
\VS{13}Ne savez-vous pas ce que nous avons fait, moi et mes pères, à tous les peuples des autres pays ? Les dieux des nations de ces pays ont-ils pu de quelque manière que ce soit délivrer leur pays de ma main ?
\VS{14}Quel est celui de tous les dieux de ces nations, que mes pères ont entièrement détruites, qui ait pu délivrer son peuple de ma main, pour que votre Dieu puisse vous délivrer de ma main ?
\VS{15}Maintenant donc, qu'Ezéchias ne vous abuse point, et qu'il ne vous incite plus de cette manière, et ne le croyez pas ; car aucun dieu d'aucune nation ni d'aucun royaume n'a pu délivrer son peuple de ma main ni de la main de mes pères ; combien moins votre Dieu vous délivrerait-il de ma main ?
\VS{16}Ses serviteurs parlèrent encore contre Yahweh Dieu, et contre Ezéchias, son serviteur.
\VS{17}Il écrivit aussi une lettre pour blasphémer contre Yahweh, le Dieu d'Israël, en parlant ainsi : Comme les dieux des nations des autres pays n'ont pu délivrer leur peuple de ma main, ainsi le Dieu d'Ezéchias ne pourra délivrer son peuple de ma main\FTNT{2 R. 19:14-37.}.
\VS{18}Et ses serviteurs crièrent à haute voix en langue judaïque, au peuple de Jérusalem qui était sur la muraille, pour les effrayer et les épouvanter, afin de prendre la ville.
\VS{19}Ils parlèrent du Dieu de Jérusalem comme des dieux des peuples de la terre, qui ne sont qu'un ouvrage de mains d'homme.
\TextTitle{Prière d'Ezéchias et exaucement de Yahweh\FTNTT{2 R. 19:14-37 ; Es. 36:21-37:35}}
\VS{20}Alors le roi Ezéchias, et Esaïe, le prophète, fils d'Amots, prièrent à ce sujet et crièrent vers les cieux.
\VS{21}Et Yahweh envoya un ange, dans le camp du roi d'Assyrie, qui extermina tous les vaillants hommes, les princes et les chefs, en sorte qu'il retourna dans son pays, dans la honte. Il entra dans la maison de son dieu ; et là, ceux qui étaient sortis de ses entrailles le firent tomber par l'épée.
\VS{22}C'est ainsi que Yahweh sauva Ezéchias et les habitants de Jérusalem de la main de Sanchérib, roi d'Assyrie, et de la main de tout homme, et il les protégea de toutes parts.
\VS{23}Plusieurs apportèrent des offrandes à Yahweh, à Jérusalem, et des choses précieuses à Ezéchias, roi de Juda, qui après cela fut élevé aux yeux de toutes les nations.
\TextTitle{Maladie et guérison d'Ezéchias\FTNTT{2 R. 20:1-11}}
\VS{24}En ces jours-là, Ezéchias fut malade à en mourir, et il pria Yahweh, qui l'exauça et lui accorda un prodige\FTNT{2 R. 20:1-11}.
\VS{25}Mais Ezéchias ne fut pas reconnaissant du bienfait qu'il avait reçu ; car son cœur s'éleva, et il y eut des maux contre lui, et contre Juda et Jérusalem.
\VS{26}Mais Ezéchias s'humilia de l'élévation de son cœur, lui et les habitants de Jérusalem, et la colère de Yahweh ne vint plus sur eux durant les jours d'Ezéchias.
\TextTitle{Fin du règne d'Ezéchias, sa mort\FTNTT{2 R. 20:12-21 ; cp. Es. 39}}
\VS{27}Ezéchias eut de très grandes richesses et de la gloire, et il se fit des trésors d'argent, d'or, de pierres précieuses, d'aromates, de boucliers, et de toutes sortes d'objets précieux ;
\VS{28}des magasins pour les récoltes de blé, de moût et d'huile, des étables pour toutes sortes de bétail, avec des rangées dans les étables.
\VS{29}Il se fit aussi des villes, et il acquit des troupeaux du gros et du menu bétail en abondance ; car Dieu lui avait donné de très grandes richesses.
\VS{30}Ce fut Ezéchias, qui boucha le canal du haut des eaux de Guihon, et les conduisit directement en bas, vers l'occident de la cité de David. Ainsi Ezéchias réussit dans tout ce qu'il fit.
\VS{31}Toutefois, lorsque les princes de Babylone envoyèrent des messagers vers lui pour s'informer du prodige qui s'était produit dans le pays, Dieu l'abandonna pour le mettre à l'épreuve, afin de connaître tout ce qui était dans son cœur\FTNT{Es. 29.}.
\VS{32}Le reste des actions d'Ezéchias, ses bonnes œuvres, voici, elles sont écrites dans la vision d'Esaïe, le prophète, fils d'Amots, dans le livre des rois de Juda et d'Israël.
\VS{33}Puis Ezéchias s'endormit avec ses pères, et on l'ensevelit au plus haut des sépulcres des fils de David ; et tout Juda, et Jérusalem lui firent honneur à sa mort, et Manassé, son fils régna à sa place.
\Chap{33}
\TextTitle{Manassé, roi impie de Juda\FTNTT{2 R. 21:1-9}}
\VerseOne{}Manassé était âgé de douze ans quand il devint roi, et il régna cinquante-cinq ans à Jérusalem.
\VS{2}Il fit ce qui est mal aux yeux de Yahweh, suivant les abominations des nations que Yahweh avait chassées devant les enfants d'Israël.
\VS{3}Il rebâtit les hauts lieux qu'Ezéchias, son père, avait démolis, il redressa les autels aux Baals, il fit des idoles d'Astarté, et se prosterna devant toute l'armée des cieux et la servit.
\VS{4}Il bâtit aussi des autels dans la maison de Yahweh, de laquelle Yahweh avait parlé ainsi : Mon Nom sera dans Jérusalem à jamais.
\VS{5}Il bâtit des autels à toute l'armée des cieux, dans les deux parvis de la maison de Yahweh.
\VS{6}Il fit passer ses fils par le feu dans la vallée du fils de Hinnom ; il pratiquait la magie, les sorcelleries et la voyance ; il établit des gens qui évoquaient les esprits et des devins. Il s'adonna à faire à l'extrême ce qui est mal aux yeux de Yahweh, pour l'irriter.
\VS{7}Il posa aussi une image taillée, une idole qu'il avait faite, dans la maison de Dieu, de laquelle Dieu avait dit à David, et à Salomon, son fils : Je mettrai à perpétuité mon Nom dans cette maison et dans Jérusalem, que j'ai choisie entre toutes les tribus d'Israël ;
\VS{8}et je ne ferai plus sortir Israël de la terre que j'ai assignée à leurs pères, pourvu seulement qu'ils prennent garde à faire tout ce que je leur ai ordonné, selon toute la loi, les préceptes et les ordonnances prescrites par Moïse.
\VS{9}Manassé donc fit s'égarer Juda et les habitants de Jérusalem, jusqu'à faire pire que les nations que Yahweh avait exterminées de devant les enfants d'Israël.
\TextTitle{Yahweh avertit Manassé\FTNTT{2 R. 21:10-16}}
\VS{10}Yahweh parla à Manassé et à son peuple ; mais ils ne furent pas attentifs.
\TextTitle{Manassé emmené captif se repent\FTNTT{2 R. 21:17-18}}
\VS{11}Alors Yahweh fit venir contre eux les chefs de l'armée du roi d'Assyrie, qui mirent Manassé dans les fers ; ils le lièrent d'une double chaîne d'airain, et l'emmenèrent à Babylone.
\VS{12}Et dès qu'il fut dans l'angoisse, il supplia Yahweh, son Dieu, et il s'humilia profondément devant le Dieu de ses pères.
\VS{13}Il lui adressa ses supplications, et Dieu se laissa fléchir par sa prière, et exauça sa supplication. Il le fit retourner à Jérusalem, dans son royaume. Manassé reconnut alors que c'est Yahweh qui est Dieu.
\VS{14}Après cela, il bâtit une muraille extérieure à la cité de David, vers l'occident de Guihon, dans la vallée, jusqu'à l'entrée de la porte des poissons ; il environna la colline et l'éleva à une grande hauteur ; il établit aussi des chefs d'armée dans toutes les villes fortes de Juda.
\VS{15}Il ôta de la maison de Yahweh l'idole, et les dieux étrangers, et tous les autels qu'il avait bâtis sur la montagne de la maison de Yahweh et à Jérusalem, et les jeta hors de la ville.
\VS{16}Puis il rebâtit l'autel de Yahweh et y offrit des sacrifices d'offrande de paix et de reconnaissance ; et il ordonna à Juda de servir Yahweh, le Dieu d'Israël.
\VS{17}Toutefois, le peuple sacrifiait encore dans les hauts lieux, mais seulement à Yahweh, son Dieu.
\VS{18}Le reste des actions de Manassé, et la prière qu'il fit à son Dieu, et les paroles des voyants qui lui parlaient, au Nom de Yahweh, le Dieu d'Israël, voilà, toutes ces choses sont écrites dans les actes des rois d'Israël.
\VS{19}Sa prière, et comment Dieu se laissa fléchir par sa prière, ses péchés et ses infidélités, les lieux sur lesquels il bâtit des hauts lieux, et dressa des idoles d'Astarté et des images taillées, avant de s'être humilié, voici cela est écrit dans le livre de Hozaï.
\VS{20}Puis Manassé s'endormit avec ses pères, et on l'ensevelit dans sa maison. Et Amon, son fils, régna à sa place.
\TextTitle{Amon règne brièvement sur Juda\FTNTT{2 R. 21:18-26}}
\VS{21}Amon était âgé de vingt-deux ans quand il devint roi, et il régna deux ans à Jérusalem.
\VS{22}Il fit ce qui est mal aux yeux de Yahweh, comme avait fait Manassé, son père. Il sacrifia à toutes les images taillées que Manassé, son père, avait faites, et il les servit.
\VS{23}Mais il ne s'humilia point devant Yahweh, comme s'était humilié Manassé, son père, mais se rendit de plus en plus coupable.
\VS{24}Et ses serviteurs ayant fait une conspiration contre lui, le firent mourir dans sa maison.
\VS{25}Mais le peuple du pays frappa tous ceux qui avaient conspiré contre le roi Amon. Et le peuple du pays établit pour roi, à sa place, Josias, son fils.
\Chap{34}
\TextTitle{Josias règne sur Juda ; ses réformes\FTNTT{2 R. 22:1-2}}
\VerseOne{}Josias était âgé de huit ans quand il devint roi, et il régna trente et un ans à Jérusalem.
\VS{2}Il fit ce qui est droit aux yeux de Yahweh. Il marcha dans les voies de David, son père ; et ne s'en détourna ni à droite ni à gauche.
\VS{3}La huitième année de son règne, lorsqu'il était jeune, il commença à rechercher le Dieu de David, son père ; et à la douzième année, il commença à purifier Juda et Jérusalem des hauts lieux, des idoles d'Astarté, et des images taillées, et des images de fonte.
\VS{4}On démolit dans sa présence les autels des Baals, et il abattit les tentes solaires\FTNT{Tentes solaires : lieux d’idolâtrie} qui étaient par-dessus. Il brisa les idoles d'Astarté, les images taillées et les images de fonte ; et les ayant réduites en poudre, il la répandit sur les sépulcres de ceux qui leur avaient sacrifié.
\VS{5}Puis il brûla les os des prêtres sur leurs autels, et il purifia ainsi Juda et Jérusalem.
\VS{6}Il fit la même chose dans les villes de Manassé, d'Ephraïm et de Siméon, et jusqu'à Nephthali, dans leurs ruines et tout autour.
\VS{7}Il démolit les autels et mit en pièces les idoles d'Astarté et les images taillées, et il les réduisit en poussière ; il abattit toutes les tentes solaires dans tout le pays d'Israël. Puis il revint à Jérusalem.
\TextTitle{Restauration du temple\FTNTT{2 R. 22:3-7}}
\VS{8}La dix-huitième année de son règne, après avoir purifié le pays et le temple, il envoya Schaphan, fils d'Atsalia, et Maaséja, chefs de la ville, et Joach, fils de Joachaz, commis sur les registres, pour réparer la maison de Yahweh, son Dieu.
\VS{9}Ils vinrent vers Hilkija, le souverain sacrificateur ; et on livra l'argent qui avait été apporté dans la maison de Dieu et que les Lévites, gardes du seuil, avaient amassé des mains de Manassé, d'Ephraïm et de tout le reste d'Israël, et aussi de tout Juda et Benjamin ; puis ils s'en retournèrent à Jérusalem.
\VS{10}On le remit entre les mains de ceux qui avaient la charge de l'ouvrage, qui étaient préposés sur la maison de Yahweh. Et ceux qui avaient la charge de l'ouvrage et qui travaillaient dans la maison de Yahweh le distribuèrent pour restaurer et réparer la maison de Yahweh.
\VS{11}Ils le donnèrent aux charpentiers et aux maçons, pour acheter des pierres de taille et du bois pour les poutres et pour la charpente des maisons que les rois de Juda avaient détruites.
\VS{12}Ces hommes s'employaient fidèlement à cet ouvrage. Jachath et Abdias, Lévites d'entre les fils de Merari, étaient préposés sur eux, et Zacharie et Meschullam, d'entre les fils des Kehathites, pour les diriger. Ces Lévites avaient tous de l'intelligence pour les instruments de musique.
\VS{13}Ils surveillaient ceux qui portaient les fardeaux, et dirigeaient tous ceux qui faisaient l'ouvrage, dans quelque service que ce soit ; les scribes, les administrateurs et les portiers, d'entre les Lévites.
\TextTitle{Le livre de la loi redécouvert\FTNTT{2 R. 22:8-10}}
\VS{14}Au moment où l'on sortit l'argent qui avait été apporté dans la maison de Yahweh, Hilkija, le sacrificateur, trouva le livre de la loi de Yahweh, donné par Moïse.
\VS{15}Alors Hilkija, prenant la parole, dit à Schaphan, le secrétaire : J'ai trouvé le livre de la loi dans la maison de Yahweh. Et Hilkija donna le livre à Schaphan.
\VS{16}Schaphan apporta le livre au roi, et rapporta tout au roi, en disant : Les mains de tes serviteurs ont fait tout ce qui leur a été donné à faire.
\VS{17}Ils ont amassé l'argent qui se trouvait dans la maison de Yahweh, et l'ont livré entre les mains des administrateurs, et entre les mains de ceux qui ont la charge de l'ouvrage.
\VS{18}Schaphan, le secrétaire, raconta en disant au roi : Hilkija, le sacrificateur, m'a donné un livre ; et Schaphan le lut devant le roi.
\TextTitle{Lecture du livre de la loi\FTNTT{2 R. 22:11-13}}
\VS{19}Lorsque le roi entendit les paroles de la loi, il déchira ses vêtements.
\VS{20}Il ordonna à Hilkija, à Achikam, fils de Schaphan, à Abdon, fils de Michée, à Schaphan, le secrétaire, et à Asaja, serviteur du roi, en disant :
\VS{21}Allez, consultez Yahweh pour moi et pour ce qui reste en Israël et en Juda, touchant les paroles du livre qui a été trouvé ; car la colère de Yahweh est grande, et elle s'est déversée sur nous, parce que nos pères n'ont point gardé la parole de Yahweh, pour faire selon tout ce qui est écrit dans ce livre.
\TextTitle{Instruction de la prophétesse Hulda\FTNTT{2 R. 22:14-20}}
\VS{22}Hilkija et les gens du roi allèrent vers Hulda, la prophétesse, femme de Schallum, fils de Thokehath, fils de Hasra, garde des vêtements, laquelle demeurait à Jérusalem, dans un autre quartier, et lui en parlèrent.
\VS{23}Alors elle leur répondit : Ainsi parle Yahweh, le Dieu d'Israël : Dites à l'homme qui vous a envoyés vers moi :
\VS{24}Ainsi parle Yahweh : Voici, je vais faire venir le malheur sur ce lieu et sur ses habitants, à savoir toutes les malédictions du serment qui sont écrites dans le livre qu'on a lu devant le roi de Juda.
\VS{25}Parce qu'ils m'ont abandonné, et qu'ils ont fait brûler des parfums aux autres dieux, pour m'irriter par toutes les œuvres de leurs mains, ma colère s'est déversée sur ce lieu, et elle ne sera point éteinte.
\VS{26}Mais quant au roi de Juda, qui vous a envoyés pour consulter Yahweh, vous lui direz : Ainsi parle Yahweh, le Dieu d'Israël, au sujet des paroles que tu as entendues :
\VS{27}Parce que ton cœur a été touché, et que tu t'es humilié devant Dieu, quand tu as entendu ses paroles contre ce lieu et contre ses habitants, et que t'étant humilié devant moi, tu as déchiré tes vêtements et pleuré devant moi, je t'ai aussi entendu, dit Yahweh.
\VS{28}Voici, je vais te recueillir avec tes pères, et tu seras recueilli dans tes sépulcres en paix, et tes yeux ne verront point tout ce mal que je vais faire venir sur ce lieu et sur ses habitants. Et ils rapportèrent cette parole au roi.
\TextTitle{Renouvellement de l'alliance avec Yahweh\FTNTT{2 R. 23:1-3}}
\VS{29}Alors le roi envoya assembler tous les anciens de Juda et de Jérusalem.
\VS{30}Le roi monta à la maison de Yahweh avec tous les hommes de Juda et les habitants de Jérusalem, les sacrificateurs et les Lévites, et tout le peuple, depuis le plus grand jusqu'au plus petit ; et on lut devant eux toutes les paroles du livre de l'alliance, qui avait été trouvé dans la maison de Yahweh.
\VS{31}Et le roi se tint debout à sa place ; et traita devant Yahweh cette alliance qu'ils suivraient Yahweh, et qu'ils garderaient ses commandements, ses témoignages et ses lois, chacun de tout son cœur et de toute son âme, en pratiquant les paroles de l'alliance écrites dans ce livre.
\VS{32}Et il fit tenir debout tous ceux qui se trouvèrent à Jérusalem et en Benjamin ; et les habitants de Jérusalem firent selon l'alliance de Dieu, le Dieu de leurs pères.
\VS{33}Josias ôta de tous les pays qui appartenaient aux enfants d'Israël, toutes les abominations ; et il obligea tous ceux qui se trouvaient en Israël à servir Yahweh, leur Dieu. Pendant toute sa vie, ils ne se détournèrent point de Yahweh, le Dieu de leurs pères.
\Chap{35}
\TextTitle{Josias rétablit la Pâque\FTNTT{2 R. 23:21-27}}
\VerseOne{}Or Josias célébra la Pâque à Yahweh à Jérusalem, et on immola la Pâque le quatorzième jour du premier mois.
\VS{2}Il rétablit les sacrificateurs dans leurs charges, et les encouragea au service de la maison de Yahweh.
\VS{3}Il dit aussi aux Lévites qui enseignaient tout Israël et qui étaient consacrés à Yahweh : Mettez l'arche sainte dans la maison que Salomon, fils de David, roi d'Israël, a bâtie. Qu'elle ne soit plus une charge sur vos épaules. Maintenant, servez Yahweh, votre Dieu, et son peuple d'Israël.
\VS{4}Préparez-vous, selon les maisons de vos pères, selon vos divisions suivant l'écrit de David, roi d'Israël, et suivant l'écrit de Salomon, son fils.
\VS{5}Tenez-vous dans le sanctuaire pour vos frères, les fils du peuple, selon les classes des maisons des pères, et selon que chaque famille des Lévites est partagée.
\VS{6}Immolez la Pâque, sanctifiez-vous, et préparez-la pour vos frères, afin qu'ils puissent la faire selon la parole que Yahweh a donnée par Moïse.
\VS{7}Josias éleva une offrande pour les gens du peuple et pour tous ceux qui se trouvaient là, des troupeaux d'agneaux et de chevreaux, au nombre de trente mille, et trois mille bœufs, le tout pour la Pâque ; cela fut pris sur les biens du roi.
\VS{8}Ses chefs élevèrent une offrande de bon gré pour le peuple, aux sacrificateurs et aux Lévites. Hilkija, Zacharie et Jehiel, princes de la maison de Dieu, donnèrent aux sacrificateurs, pour la Pâque, deux mille six cents agneaux, et trois cents bœufs.
\VS{9}Conania, Schemaeja et Nethaneel, ses frères, et Haschabia, Jeïel et Jozabad, qui étaient les princes des Lévites, élevèrent une offrande de cinq mille agneaux aux Lévites pour faire la Pâque, et cinq cents bœufs.
\VS{10}Le service étant préparé, les sacrificateurs se tinrent à leurs postes, et les Lévites suivant leurs divisions, selon l'ordre du roi.
\VS{11}Puis on immola la Pâque ; et les sacrificateurs répandaient le sang reçu de leurs mains, et les Lévites les dépouillaient.
\VS{12}Ils mirent à part les holocaustes, pour les donner aux gens du peuple, suivant les divisions des maisons de leurs pères, afin de les offrir à Yahweh, selon ce qui est écrit au livre de Moïse ; ils firent de même pour les bœufs.
\VS{13}Ils firent cuire la Pâque au feu, selon l'ordonnance ; et ils firent cuire dans des chaudières, des chaudrons et des poêles, les choses consacrées ; et ils les apportèrent rapidement à tous les gens du peuple.
\VS{14}Puis ils apprêtèrent ce qui était pour eux et pour les sacrificateurs, car les sacrificateurs, fils d'Aaron, furent occupés jusqu'à la nuit à élever en offrande les holocaustes et les graisses ; c'est pourquoi les Lévites apprêtèrent ce qui était pour eux et pour les sacrificateurs, fils d'Aaron.
\VS{15}Les chantres, fils d'Asaph, étaient à leur place, selon l'ordre de David, d'Asaph, d'Héman et de Jeduthun, le voyant du roi. Les portiers étaient à chaque porte, ils n'eurent pas à se détourner de leur service, car leurs frères les Lévites apprêtaient ce qui était pour eux.
\VS{16}Ainsi, tout le service de Yahweh, en ce jour-là, fut réglé pour faire la Pâque et pour élever en offrande les holocaustes sur l'autel de Yahweh, selon l'ordre du roi Josias.
\VS{17}Les fils d'Israël qui s'y trouvèrent célébrèrent donc la Pâque en ce temps-là, et la fête des pains sans levain pendant sept jours.
\VS{18}Or on n'avait point célébré en Israël de Pâque semblable à celle-là depuis les jours de Samuel le prophète ; et aucun des rois d'Israël n'avait célébré une Pâque pareille comme le fit Josias, avec les sacrificateurs et les Lévites, et tout Juda et Israël, qui s'y étaient trouvés avec les habitants de Jérusalem.
\VS{19}Cette Pâque fut célébrée la dix-huitième année du règne de Josias.
\TextTitle{Blessure et mort de Josias\FTNTT{2 R. 23:28-30}}
\VS{20}Après tout cela, quand Josias eut réparé la maison de Yahweh, Néco, roi d'Egypte, monta pour faire la guerre à Carkemisch, sur l'Euphrate. Josias sortit à sa rencontre.
\VS{21}Mais Néco envoya vers lui des messagers pour lui dire : Qu'y a-t-il entre nous, roi de Juda ? Ce n'est pas à toi que j'en veux aujourd'hui, mais à une maison qui me fait la guerre ; et Dieu m'a dit de me hâter. Désiste-toi donc de venir contre Dieu, qui est avec moi, de peur qu'il ne te détruise.
\VS{22}Cependant Josias ne se détourna point de lui, mais se déguisa pour combattre contre lui et il n'écouta pas les paroles de Néco, qui venaient de la bouche de Dieu. Il vint donc pour combattre dans la vallée de Meguiddo.
\VS{23}Les archers tirèrent sur le roi Josias ; et le roi dit à ses serviteurs : Emportez-moi, car je suis très blessé.
\VS{24}Ses serviteurs l'ôtèrent du char, le mirent sur un second char qu'il avait, et le menèrent à Jérusalem. Il mourut et il fut enseveli dans les sépulcres de ses pères, et tous ceux de Juda et de Jérusalem menèrent le deuil de Josias.
\VS{25}Jérémie fit aussi des lamentations sur Josias ; et tous les chanteurs et toutes les chanteuses parlèrent dans leurs complaintes sur Josias jusqu'à ce jour ; et on en a fait une coutume en Israël. Voici, ces choses sont écrites dans les lamentations.
\VS{26}Le reste des actions de Josias, et ses œuvres de piété, selon ce qui est écrit dans la loi de Yahweh,
\VS{27}ses premières et ses dernières actions, sont écrites dans le livre des rois d'Israël et de Juda.
\Chap{36}
\TextTitle{Joachaz règne brièvement sur Juda\FTNTT{2 R. 23:31-33}}
\VerseOne{}Alors le peuple du pays prit Joachaz, fils de Josias, et on l'établit roi à Jérusalem, à la place de son père.
\VS{2}Joachaz était âgé de vingt-trois ans quand il devint roi, et il régna trois mois à Jérusalem.
\VS{3}Le roi d'Egypte le destitua à Jérusalem, et condamna le pays à une amende de cent talents d'argent et d'un talent d'or.
\TextTitle{Règne de Jojakim, déportation à Babylone\FTNTT{2 R. 23:34-24:4-9}}
\VS{4}Le roi d'Egypte établit pour roi sur Juda et Jérusalem Eliakim, frère de Joachaz ; et changea son nom en celui de Jojakim. Puis Néco prit Joachaz, son frère, et l'emmena en Egypte.
\VS{5}Jojakim était âgé de vingt-cinq ans quand il devint roi, et il régna onze ans à Jérusalem. Il fit ce qui est mal aux yeux de Yahweh, son Dieu.
\VS{6}Nebucadnetsar, roi de Babylone, monta contre lui et le lia de doubles chaînes d'airain pour le mener à Babylone.
\VS{7}Nebucadnetsar emporta aussi à Babylone des ustensiles de la maison de Yahweh, et il les mit dans son temple à Babylone.
\VS{8}Le reste des actions de Jojakim, et les abominations qu'il commit, et ce qui fut trouvé en lui, cela est écrit dans le livre des rois d'Israël et de Juda. Et Jojakin, son fils, régna à sa place.
\VS{9}Jojakin était âgé de huit ans quand il devint roi, et il régna trois mois et dix jours à Jérusalem. Il fit ce qui est mal aux yeux de Yahweh.
\TextTitle{Sédécias le dernier roi de Juda, autres déportations à Babylone\FTNTT{2 R.24:10-20 ; cp. 2 R. 25:1-21 ; Jé. 39:8-10}}
\VS{10}Et l'année suivante, le roi Nebucadnetsar envoya, et le fit emmener à Babylone avec les ustensiles précieux de la maison de Yahweh ; et il établit roi sur Juda et Jérusalem, Sédécias, son frère.
\VS{11}Sédécias était âgé de vingt et un ans quand il devint roi, et il régna onze ans à Jérusalem.
\VS{12}Il fit ce qui est mal aux yeux de Yahweh, son Dieu ; et il ne s'humilia point devant Jérémie le prophète, qui lui parlait de la part de Yahweh.
\VS{13}Et même il se rebella contre le roi Nebucadnetsar, qui l'avait fait prêter serment par le Nom de Dieu. Il raidit son cou, et il obstina son cœur pour ne point retourner à Yahweh, le Dieu d'Israël.
\VS{14}Pareillement, tous les chefs des sacrificateurs et le peuple furent infidèles et continuèrent de plus en plus à pécher, selon toutes les abominations des nations ; et ils souillèrent la maison que Yahweh avait sanctifiée dans Jérusalem.
\VS{15}Or Yahweh, le Dieu de leurs pères, les avait sommés par ses messagers qu'il envoya de bonne heure, car il voulait épargner son peuple et sa propre demeure.
\VS{16}Mais ils se moquèrent des messagers de Dieu, ils méprisèrent ses paroles et traitèrent ses prophètes de séducteurs, jusqu'à ce que la fureur de Yahweh montat contre son peuple au point qu'il n'y eut plus de remède.
\VS{17}C'est pourquoi il fit monter contre eux le roi des Chaldéens, qui tua par l'épée leurs jeunes gens dans la maison de leur sanctuaire ; il n'épargna ni le jeune homme, ni la vierge, ni le vieillard, ni l'homme à cheveux blancs ; il les livra tous entre ses mains.
\VS{18}Il fit apporter à Babylone tous les ustensiles de la maison de Dieu, grands et petits, les trésors de la maison de Yahweh, et les trésors du roi et ceux de ses chefs.
\VS{19}Ils brûlèrent la maison de Dieu, ils démolirent les murailles de Jérusalem, ils livrèrent au feu tous ses palais et détruisirent tout ce qu'il y avait comme objets précieux.
\VS{20}Puis le roi de Babylone transporta à Babylone le reste qui échappa à l'épée, et ils furent ses esclaves et ceux de ses fils, jusqu'à la domination du royaume de Perse,
\VS{21}afin que la parole de Yahweh, prononcée par la bouche de Jérémie, fût accomplie ; jusqu'à ce que la terre eût pris plaisir à ses sabbats et durant tous les jours qu'elle demeura dévastée ; elle se reposa pour accomplir les soixante-dix années.
\TextTitle{L'édit de Cyrus autorise les juifs à retourner dans leurs villes}
\VS{22}Mais la première année de Cyrus, roi de Perse, afin que la parole de Yahweh prononcée par Jérémie fût accomplie, Yahweh réveilla l'esprit de Cyrus, roi de Perse, qui fit publier dans tout son royaume, et même par écrit, en disant :
\VS{23}Ainsi parle Cyrus, roi de Perse : Yahweh, le Dieu des cieux, m'a donné tous les royaumes de la terre, et lui-même m'a ordonné de lui bâtir une maison à Jérusalem, qui est en Juda. Qui d'entre vous est de son peuple ? Que Yahweh, son Dieu, soit avec lui, et qu'il monte !
\PPE{}
\end{multicols}

%\addcontentsline{toc}{chapter}{Évangiles}\clearpage
%\clearpage\ShortTitle{Matthieu}\BookTitle{Matthieu}\BFont
\noindent\hrulefill
{\footnotesize
\textit{
\bigskip
{\centering{}
\\Auteur : Matthieu
\\(Gr. : Matthaios)
\\Signifie : Don de Yahweh
\\Thème : Jésus le Roi
\\Date de rédaction : Env. 50 ap. J.-C.\\}
}
%\bigskip
\textit{
\\Matthieu, également connu sous le nom de Lévi, était un juif percepteur d'impôts au service des Romains. Appelé par
Jésus-Christ à Capernaüm et choisi pour être l'un des douze disciples, il offrit un banquet en l'honneur de Jésus dans
sa maison, ce qui lui valut l'hostilité des pharisiens. Rédigé à Antioche de Syrie, son évangile était destiné à des juifs
convertis comme en témoignent ses nombreuses allusions à l'Ancienne Alliance.
%\bigskip
\\Matthieu, démontre l'hégémonie de Jésus, fils de David, fils d'Abraham, roi d'Israël. Il évoque son règne qui se manifestera un jour physiquement, lorsque le Roi jugera tous les hommes du « trône de sa gloire ». Il fait concorder les paroles et les événements de la vie de Jésus avec les prophéties de l'Ancienne Alliance.
%\bigskip
\\Son récit exalte la royauté de Jésus et expose l'Evangile du Royaume.\bigskip
}
}
\par\nobreak\noindent\hrulefill
\begin{multicols}{2}
\Chap{1}
\TextTitle{Présentation du Roi : La généalogie de Jésus-Christ}
\VerseOne{}Livre de la généalogie de Jésus-Christ, fils de David, fils d'Abraham.
\VS{2}Abraham engendra Isaac ; Isaac engendra Jacob ; Jacob engendra Juda et ses frères ;
\VS{3}Juda engendra Pérets et Zara, de Thamar ; Pérets engendra Esrom ; Esrom engendra Aram ;
\VS{4}Aram engendra Aminadab ; Aminadab engendra Naasson ; et Naasson engendra Salmon ;
\VS{5}Salmon engendra Boaz, de Rahab\FTNT{Rahab était une prostituée cananéenne qui est devenue l'ancêtre du Messie (Jos. 6).} ; Boaz engendra Obed, de Ruth\FTNT{Ruth était une Moabite, son peuple était issu de la relation incestueuse de Lot et sa fille aînée (Ge. 19:36-37). Elle est devenue l'ancêtre du Messie (Ru. 4:17).} ; Obed engendra Isaï ;
\VS{6}Isaï engendra le roi David ; le roi David engendra Salomon, de la femme d'Urie ;
\VS{7}Salomon engendra Roboam ; Roboam engendra Abia ; Abia engendra Asa ;
\VS{8}Asa engendra Josaphat ; Josaphat engendra Joram ; Joram engendra Ozias ;
\VS{9}Ozias engendra Joatham ; Joatham engendra Achaz ; Achaz engendra Ezéchias ;
\VS{10}Ezéchias engendra Manassé ; Manassé engendra Amon ; Amon engendra Josias ;
\VS{11}Josias engendra Jéchonias et ses frères, au temps de la déportation à Babylone.
\VS{12}Après la déportation à Babylone, Jéchonias engendra Salathiel ; Salathiel engendra Zorobabel ;
\VS{13}Zorobabel engendra Abiud ; Abiud engendra Eliakim ; Eliakim engendra Azor ;
\VS{14}Azor engendra Sadok ; Sadok engendra Achim ; Achim engendra Eliud ;
\VS{15}Eliud engendra Eléazar ; Eléazar engendra Matthan ; Matthan engendra Jacob ;
\VS{16}Jacob engendra Joseph, l'époux de Marie, de laquelle est né Jésus, qui est appelé Christ\FTNT{Christ : Du grec « christos », ce qui signifie « oint », est l'équivalent grec du mot hébreu « mashiyach », traduit par « messie » en français (Da. 9:25-26). C'est le titre officiel du Seigneur Jésus.}.
\VS{17}Ainsi, il y a en tout quatorze générations depuis Abraham jusqu'à David, quatorze générations depuis David jusqu'à la déportation à Babylone, et quatorze générations depuis la déportation à Babylone jusqu'au Christ.
\TextTitle{Naissance miraculeuse de Jésus-Christ\FTNTT{Lu. 1:26-38 ; 2:1-7 ; Jn. 1:1-2,14}}
\VS{18}Voici de quelle manière arriva la naissance de Jésus-Christ. Marie, sa mère, ayant été fiancée à Joseph, se trouva enceinte par l'opération du Saint-Esprit, avant qu'ils aient habité ensemble.
\VS{19}Joseph, son époux, qui était un homme juste et qui ne voulait pas la diffamer, se proposa de la répudier secrètement.
\VS{20}Mais comme il y pensait, voici, l'Ange du Seigneur lui apparut en songe et lui dit : Joseph, fils de David, ne crains point de prendre avec toi Marie, ta femme, car l'enfant qu'elle a conçu est du Saint-Esprit.
\VS{21}Elle enfantera un fils, et tu lui donneras le nom de Jésus. C'est lui qui sauvera son peuple de ses péchés.
\VS{22}Tout cela arriva afin que s'accomplisse ce que le Seigneur avait annoncé par le prophète :
\VS{23}Voici, la vierge deviendra enceinte, elle enfantera un fils ; et on lui donnera le nom d'Emmanuel\FTNT{Es. 7:14.}, ce qui signifie, Dieu est avec nous.
\VS{24}Joseph s'étant donc réveillé de son sommeil, fit ce que l'Ange du Seigneur lui avait ordonné, et il prit sa femme.
\VS{25}Mais il ne la connut point jusqu'à ce qu'elle ait enfanté son fils premier-né, auquel il donna le nom de Jésus.
\Chap{2}
\TextTitle{Les mages adorent Jésus}
\VerseOne{}Jésus étant né à Bethléhem, ville de Juda, au temps du roi Hérode, voici des mages d'orient arrivèrent à Jérusalem.
\VS{2}En disant : Où est le Roi des Juifs qui vient de naître ? Car nous avons vu son étoile en orient, et nous sommes venus l'adorer.
\VS{3}Le roi Hérode ayant entendu, fut troublé et tout Jérusalem avec lui.
\VS{4}Et ayant assemblé tous les principaux sacrificateurs et les scribes du peuple, il s'informa auprès d'eux où le Christ devait naître.
\VS{5}Et ils lui dirent : A Bethléhem, ville de Judée ; car voici ce qui a été écrit par le prophète :
\VS{6}Et toi, Bethléhem, terre de Juda, tu n'es nullement la plus petite parmi les gouverneurs de Juda, car de toi sortira le Chef qui paîtra mon peuple d'Israël\FTNT{Mi. 5:1.}.
\VS{7}Alors Hérode ayant appelé en secret les mages, s'informa soigneusement auprès d'eux depuis combien de temps brillait l'étoile.
\VS{8}Puis il les envoya à Bethléhem, en leur disant : Allez, et prenez des informations exactes sur le petit enfant ; et quand vous l'aurez trouvé, faites-le-moi savoir, afin que j'aille aussi moi-même l'adorer.
\VS{9}Après avoir entendu le roi, ils partirent. Et voici, l'étoile\FTNT{Le Seigneur Jésus-Christ s'est révélé à Jean comme l'étoile brillante du matin (Ap. 22:16).} qu'ils avaient vue en orient allait devant eux, jusqu'au moment où, arrivée au-dessus du lieu où était le petit enfant, elle s'arrêta.
\VS{10}Quand ils virent l'étoile, ils furent saisis d'une très grande joie.
\VS{11}Ils entrèrent dans la maison, virent le petit enfant avec Marie sa mère, se prosternèrent et l'adorèrent. Ils ouvrirent ensuite leurs trésors et lui offrirent des présents : De l'or, de l'encens et de la myrrhe.
\VS{12}Puis, divinement avertis en songe de ne pas retourner vers Hérode, ils regagnèrent leur pays par un autre chemin.
\TextTitle{Fuite en Egypte}
\VS{13}Lorsqu'ils furent partis, voici, l'Ange du Seigneur apparut dans un songe à Joseph et lui dit : Lève-toi et prends le petit enfant et sa mère, fuis en Egypte, et demeure là, jusqu'à ce que je te le dise ; car Hérode cherchera le petit enfant pour le faire mourir.
\VS{14}Joseph donc étant réveillé, prit de nuit le petit enfant et sa mère, et se retira en Egypte.
\VS{15}Il y resta là jusqu'à la mort d'Hérode ; afin que s'accomplisse ce que le Seigneur avait annoncé par le prophète : J'ai appelé mon Fils hors d'Egypte\FTNT{Os. 11:1}.
\TextTitle{Hérode envoie tuer des enfants innocents}
\VS{16}Alors Hérode voyant que les mages s'étaient moqués de lui, se mit dans une grande colère, et il envoya tuer tous les enfants qui étaient à Bethléhem, et dans tout son territoire ; depuis l'âge de deux ans, et au-dessous, selon la date dont il s'était exactement enquis auprès des mages.
\VS{17}Alors s'accomplit ce qui avait été annoncé par Jérémie le prophète :
\VS{18}On a entendu à Rama des cris, des lamentations, des plaintes, et des grands gémissements : Rachel pleure ses enfants et n'a pas voulu être consolée, parce qu'ils ne sont plus\FTNT{Jé. 31:15}.
\TextTitle{Joseph revient en Israël et s'installe à Nazareth\FTNTT{Lu. 2:39-52}}
\VS{19}Mais après qu'Hérode fut mort, voici, l'Ange du Seigneur apparut dans un songe à Joseph en Egypte,
\VS{20}et lui dit : Lève-toi, et prends le petit enfant et sa mère, et va dans le pays d'Israël ; car ceux qui cherchaient à ôter la vie au petit enfant sont morts.
\VS{21}Joseph donc s'étant réveillé, prit le petit enfant et sa mère, et alla dans le pays d'Israël.
\VS{22}Mais quand il eut appris qu'Archélaüs régnait en Judée, à la place d'Hérode son père, il craignit d'y aller ; et étant divinement averti dans un songe, il se retira dans le territoire de la Galilée,
\VS{23}et vint habiter dans la ville appelée Nazareth ; afin que s'accomplisse ce qui avait été dit par les prophètes : Il sera appelé Nazaréen.
\Chap{3}
\TextTitle{Ministère de Jean-Baptiste\FTNTT{Mc. 1:1-8 ; Lu. 3:1-20 ; Jn. 1:6-8,15-37}}
\VerseOne{}Or, en ce temps-là arriva Jean-Baptiste, prêchant dans le désert de la Judée.
\VS{2}Il disait : Repentez-vous, car le Royaume des cieux est proche.
\VS{3}Car c'est celui dont Esaïe le prophète a parlé, en disant : C'est ici la voix de celui qui crie dans le désert : Préparez le chemin du Seigneur, aplanissez ses sentiers.
\VS{4}Jean avait un vêtement de poils de chameau et une ceinture de cuir autour de ses reins. Et il se nourrissait de sauterelles et de miel sauvage.
\VS{5}Alors les habitants de Jérusalem, et de toute la Judée, et de tout le pays des environs du Jourdain, vinrent à lui ;
\VS{6}et confessant leurs péchés, ils se faisaient baptiser par lui dans le Jourdain.
\VS{7}Mais, voyant venir à son baptême beaucoup de pharisiens et de sadducéens, il leur dit : Race de vipères, qui vous a appris à fuir la colère à venir ?
\VS{8}Produisez donc des fruits convenables à la repentance
\VS{9}et ne prétendez pas dire en vous-mêmes : Nous avons Abraham pour père ! Car je vous dis que Dieu peut faire naître de ces pierres mêmes des enfants à Abraham.
\VS{10}Et déjà la cognée est mise à la racine des arbres ; c'est pourquoi tout arbre qui ne produit pas de bons fruits sera coupé et jeté au feu.
\VS{11}Pour moi, je vous baptise d'eau en signe de repentance ; mais celui qui vient après moi est plus puissant que moi, et je ne suis pas digne de porter ses souliers ; celui-là vous baptisera du Saint-Esprit et de feu\FTNT{Le baptême du Saint-Esprit ne doit pas être confondu avec la plénitude du Saint-Esprit. Le baptême est un acte définitif qui nous greffe au corps du Christ lors de la conversion (1 Co. 12:13). La plénitude consiste quant à elle en un constant renouvellement que nous devons impérativement rechercher (Ep. 5:18). Certains courants chrétiens charismatiques enseignent que le parler en langues est le signe distinctif du baptême du Saint-Esprit. Cette doctrine est basée sur au moins trois passages : Ac. 2:4 ; Ac. 10:44-46 et Ac. 19:1-7. Si cela était vraiment le cas, plusieurs chrétiens seraient encore dans leurs péchés et n'appartiendraient pas au Seigneur Jésus-Christ. En effet, Ro. 8:9 déclare ceci : « Si quelqu'un n'a pas l'Esprit de Christ, il ne lui appartient pas ». Or il est manifeste que bon nombre de chrétiens nés d'en haut ne parlent pas en langues, ce qui est d'ailleurs attesté par l'apôtre Paul (1 Co. 12:30). Il n'y a aucun verset dans les Ecritures qui nous ordonne de chercher le baptême du Saint-Esprit pour la bonne et simple raison que nous le recevons à la conversion.}.
\VS{12}Il a son van à la main, et il nettoiera entièrement son aire, et il assemblera son froment dans le grenier ; mais il brûlera la paille dans un feu qui ne s'éteint point.
\TextTitle{Jean baptise Jésus-Christ\FTNTT{Mc. 1:9-11 ; Lu. 3:21-22 ; Jn. 1:31-34}}
\VS{13}Alors Jésus vint de Galilée au Jourdain vers Jean pour être baptisé par lui.
\VS{14}Mais Jean l'en empêchait avec force en lui disant : J'ai besoin d'être baptisé par toi, et tu viens vers moi ?
\VS{15}Et Jésus répondit en disant : Laisse-moi faire pour le moment, car il nous est ainsi convenable d'accomplir tout ce qui est juste. Et alors il le laissa faire.
\VS{16}Dès que Jésus eut été baptisé, il sortit aussitôt hors de l'eau. Et voici, les cieux lui furent ouverts, et Jean vit l'Esprit de Dieu descendant comme une colombe et venant sur lui.
\VS{17}Et voici une voix du ciel déclara : Celui-ci est mon Fils bien-aimé en qui j'ai mis toute mon affection.
\Chap{4}
\TextTitle{La tentation\FTNTT{Ge. 3:6 ; Mc. 1:12-13 ; Lu. 4:1-13 ; 1 Jn. 2:16.}}
\VerseOne{}Alors Jésus fut emmené par l'Esprit dans le désert, pour être tenté par le diable.
\VS{2}Après avoir jeûné quarante jours et quarante nuits, finalement il eut faim.
\VS{3}Et le tentateur s'étant approché, lui dit : Si tu es le Fils de Dieu, ordonne que ces pierres deviennent des pains.
\VS{4}Mais Jésus répondit et dit : Il est écrit : L'homme ne vivra point de pain seulement, mais de toute parole qui sort de la bouche de Dieu\FTNT{De. 8:3.}.
\VS{5}Alors le diable le transporta dans la sainte ville et le mit sur le haut du temple ;
\VS{6}et il lui dit : Si tu es le Fils de Dieu, jette-toi en bas ; car il est écrit : Il ordonnera à ses anges de te porter sur leurs mains de peur que ton pied ne heurte contre une pierre\FTNT{Ps. 91:12-13.}.
\VS{7}Jésus lui dit : Il est aussi écrit : Tu ne tenteras point le Seigneur ton Dieu\FTNT{De. 6:16.}.
\VS{8}Le diable le transporta encore sur une forte haute montagne, et lui montra tous les royaumes du monde et leur gloire ;
\VS{9}et il lui dit : Je te donnerai toutes ces choses, si tu te prosternes et m'adores.
\VS{10}Mais Jésus lui dit : Retire-toi Satan ! Car il est écrit : Tu adoreras le Seigneur ton Dieu, et tu le serviras lui seul\FTNT{De. 6:13 ; De. 10:20.}.
\VS{11}Alors le diable le laissa. Et voici, des anges s'approchèrent et le servirent.
\TextTitle{Etablissement de Jésus à Capernaüm\FTNTT{Mc. 1:14-15 ; Lu. 4:14-15}}
\VS{12}Jésus, ayant appris que Jean avait été mis en prison, se retira dans la Galilée.
\VS{13}Et ayant quitté Nazareth, il alla demeurer à Capernaüm, ville maritime, sur les confins de Zabulon et de Nephthali ;
\VS{14}afin que s'accomplisse ce qui avait été annoncé par Esaïe, le prophète, en disant :
\VS{15}Le pays de Zabulon et le pays de Nephthali, de la contrée voisine de la mer, au-delà du Jourdain, et la Galilée des Gentils ;
\VS{16}Ce peuple, assis dans les ténèbres, a vu une grande lumière ; et à ceux qui étaient assis dans la région et l'ombre de la mort, la lumière elle-même s'est levée\FTNT{Es. 9:1.}.
\VS{17}Dès lors, Jésus commença à prêcher et à dire : Repentez-vous, car le Royaume des cieux est proche.
\TextTitle{Appel de Pierre, André, Jacques et Jean\FTNTT{Mc. 1:16-20 ; Lu. 5:1-11 ; Jn. 1:35-51}}
\VS{18}Comme Jésus marchait le long de la mer de Galilée, il vit deux frères, Simon, appelé Pierre, et André, son frère, qui jetaient leurs filets dans la mer ; car ils étaient pêcheurs.
\VS{19}Et il leur dit : Suivez-moi et je vous ferai pêcheurs d'hommes.
\VS{20}Et ayant aussitôt quitté leurs filets, ils le suivirent.
\VS{21}Et de là étant allé plus en avant, il vit deux autres frères, Jacques, fils de Zébédée, et Jean, son frère, dans une barque, avec Zébédée leur père, qui réparaient leurs filets, et il les appela.
\VS{22}Et ayant aussitôt quitté leur barque et leur père, ils le suivirent.
\TextTitle{Ministère de Jésus-Christ en Galilée}
\VS{23}Jésus allait par toute la Galilée, enseignant dans leurs synagogues, prêchant l'Evangile du Royaume, et guérissant toutes sortes de maladies, et toutes sortes d'infirmités parmi le peuple.
\VS{24}Et sa renommée se répandit par toute la Syrie ; et on lui présentait tous ceux qui se portaient mal, tourmentés de diverses maladies, des démoniaques, des lunatiques, des paralytiques ; et il les guérissait.
\VS{25}Une grande foule le suivit, de Galilée, de la Décapole, de Jérusalem, de Judée et au-delà du Jourdain.
\Chap{5}
\TextTitle{L'enseignement de Jésus sur la montagne\FTNTT{Lu. 6:20-49 ; Mt. 5:1-48}}
\VerseOne{}Voyant la foule, Jésus monta sur la montagne ; puis s'étant assis, ses disciples s'approchèrent de lui.
\VS{2}Puis, ayant ouvert la bouche, il les enseigna de la sorte :
\VS{3}Heureux les pauvres en esprit, car le Royaume des cieux est à eux.
\VS{4}Heureux ceux qui pleurent, car ils seront consolés.
\VS{5}Heureux les humbles, car ils hériteront la terre.
\VS{6}Heureux ceux qui ont faim et soif de la justice, car ils seront rassasiés.
\VS{7}Heureux les miséricordieux, car ils obtiendront miséricorde.
\VS{8}Heureux sont ceux qui sont purs de cœur, car ils verront Dieu.
\VS{9}Heureux ceux qui procurent la paix, car ils seront appelés enfants de Dieu.
\VS{10}Heureux ceux qui sont persécutés pour la justice, car le Royaume des cieux est à eux.
\VS{11}Heureux serez-vous lorsqu'on vous outragera, qu'on vous persécutera et qu'on dira faussement de vous toute sorte de mal à cause de moi.
\VS{12}Réjouissez-vous et soyez dans l'allégresse, parce que votre récompense sera grande dans les cieux ; car c'est ainsi qu'on a persécuté les prophètes qui ont été avant vous.
\TextTitle{Le sel de la terre et la lumière du monde\FTNTT{Mc. 4:21-23 ; Lu. 8:16-18 ; 11:33-36}}
\VS{13}Vous êtes le sel de la terre mais si le sel perd sa saveur, avec quoi le salera-t-on ? Il ne sert plus qu'à être jeté dehors, et foulé aux pieds par les hommes.
\VS{14}Vous êtes la lumière du monde. Une ville située sur une montagne ne peut être cachée,
\VS{15}et on n'allume point la lampe pour la mettre sous un boisseau, mais sur un chandelier et elle éclaire tous ceux qui sont dans la maison.
\VS{16}Ainsi, que votre lumière luise devant les hommes, afin qu'ils voient vos bonnes œuvres et qu'ils glorifient votre Père qui est dans les cieux.
\TextTitle{Le Messie et la loi}
\VS{17}Ne croyez pas que je sois venu abolir la loi ou les prophètes ; je ne suis pas venu les abolir, mais les accomplir.
\VS{18}Car, je vous le dis en vérité, tant que le ciel et la terre ne passeront point, il ne disparaîtra pas de la loi un seul iota ou un seul trait de lettre jusqu'à ce que tout soit arrivé.
\VS{19}Celui donc qui aura violé l'un de ces petits commandements, et qui aura enseigné les hommes à faire de même, sera appelé le plus petit au Royaume des cieux ; mais celui qui les observera et qui enseignera à les observer, celui-là sera appelé grand au Royaume des cieux.
\VS{20}Car je vous dis que si votre justice ne surpasse celle des scribes et des pharisiens, vous n'entrerez point dans le Royaume des cieux.
\VS{21}Vous avez entendu qu'il a été dit aux anciens : Tu ne tueras point, et celui qui tuera, sera puni par les juges.
\VS{22}Mais moi je vous dis que quiconque se met en colère sans cause contre son frère sera puni par les juges ; et celui qui dira à son frère : Raca\FTNT{Raca : Expression de mépris utilisée parmis les Juifs au temps de Jésus signifiant vide, indigne ou encore vaurien.} ! sera puni par le conseil ; et celui qui lui dira : Insensé ! sera puni par le feu de la géhenne\FTNT{La géhenne ou le lac de feu : Voir commentaire en Ap. 20:14.}.
\VS{23}Si donc tu apportes ton offrande à l'autel, et que là tu te souviennes que ton frère a quelque chose contre toi,
\VS{24}laisse là ton offrande devant l'autel et va te réconcilier d'abord avec ton frère, puis viens et offre ton offrande.
\VS{25}Accorde-toi rapidement avec ta partie adverse, tandis que tu es en chemin avec elle ; de peur que ta partie adverse ne te livre au juge, et que le juge ne te livre à l'officier de justice, et que tu ne sois mis en prison.
\VS{26}En vérité, je te dis, tu ne sortiras point de là, jusqu'à ce que tu n'aies payé le dernier quart de sou.
\TextTitle{Convoitise, adultère et divorce\FTNTT{Mt. 19:3-11 ; Mc. 10:2-12 ; 1 Co. 7:1-16}}
\VS{27}Vous avez entendu qu'il a été dit aux anciens : Tu ne commettras point d'adultère.
\VS{28}Mais moi, je vous dis que quiconque regarde une femme pour la convoiter a déjà commis dans son cœur un adultère avec elle.
\VS{29}Si ton œil droit est pour toi une occasion de chute, arrache-le et jette-le loin de toi ; car il est avantageux pour toi qu'un seul de tes membres périsse et que ton corps entier ne soit pas jeté dans la géhenne.
\VS{30}Si ta main droite est pour toi une occasion de chute, coupe-la et jette-la loin de toi ; car il est avantageux pour toi qu'un seul de tes membres périsse et que ton corps entier ne soit pas jeté dans la géhenne.
\VS{31}Il a été dit encore : Si quelqu'un répudie sa femme, qu'il lui donne une lettre de divorce.
\VS{32}Mais moi, je vous dis que celui qui répudie sa femme, si ce n'est pour cause d'adultère, l'expose à devenir adultère ; et que celui qui épouse une femme répudiée commet un adultère.
\TextTitle{Acquittement des promesses faites au Seigneur ; attitude face à son prochain}
\VS{33}Vous avez aussi appris qu'il a été dit aux anciens : Tu ne te parjureras point, mais tu rendras au Seigneur ce que tu auras promis par serment.
\VS{34}Mais moi, je vous dis de ne jurer aucunement, ni par le ciel, parce que c'est le trône de Dieu ;
\VS{35}ni par la terre, parce que c'est le marchepied de ses pieds, ni par Jérusalem, parce que c'est la ville du grand Roi.
\VS{36}Ne jure pas non plus par ta tête, car tu ne peux pas rendre blanc ou noir un seul cheveu.
\VS{37}Mais que votre parole soit : Oui, oui ; non, non ; car ce qui est de plus, vient du malin.
\VS{38}Vous avez appris qu'il a été dit : Œil pour œil et dent pour dent.
\VS{39}Mais moi, je vous dis : Ne résistez point au méchant. Si quelqu'un te frappe sur ta joue droite, présente-lui aussi l'autre.
\VS{40}Si quelqu'un veut plaider contre toi, et prendre ta tunique, laisse-lui encore ton manteau.
\VS{41}Si quelqu'un te force à faire un mille, fais-en deux avec lui.
\VS{42}Donne à celui qui te demande et ne te détourne point de celui qui veut emprunter de toi.
\TextTitle{Le standard de l'Amour\FTNTT{Lu. 6:27-36}}
\VS{43}Vous avez appris qu'il a été dit : Tu aimeras ton prochain et tu haïras ton ennemi.
\VS{44}Mais moi, je vous dis : Aimez vos ennemis, bénissez ceux qui vous maudissent, faites du bien à ceux qui vous haïssent, et priez pour ceux qui vous maltraitent et vous persécutent,
\VS{45}afin que vous soyez les fils de votre Père qui est dans les cieux ; car il fait lever son soleil sur les méchants et sur les gens de bien, et il envoie sa pluie sur les justes et sur les injustes.
\VS{46}Car si vous aimez seulement ceux qui vous aiment, quelle récompense en aurez-vous ? Les publicains aussi n'en font-ils pas tout autant ?
\VS{47}Et si vous faites accueil seulement à vos frères, que faites-vous de plus que les autres ? Les publicains aussi ne le font-ils pas de même ?
\VS{48}Soyez donc parfaits, comme votre Père qui est dans les cieux est parfait.
\Chap{6}
\TextTitle{Jésus condamne l'hypocrisie}
\VerseOne{}Gardez-vous de pratiquer votre justice devant les hommes, pour en être vus ; autrement, vous ne recevrez point la récompense de votre Père qui est dans les cieux.
\VS{2}Donc, lorsque tu fais ton aumône, ne fais point sonner la trompette devant toi, comme font les hypocrites dans les synagogues et dans les rues, afin d'être glorifiés par les hommes. Je vous le dis en vérité, ils reçoivent leur récompense.
\VS{3}Mais quand tu fais ton aumône, que ta main gauche ne sache pas ce que fait ta droite ;
\VS{4}afin que ton aumône se fasse en secret, et ton Père, qui voit ce qui se fait dans le secret, te récompensera publiquement.
\VS{5}Et quand tu pries, ne sois point comme les hypocrites ; car ils aiment à prier en se tenant debout dans les synagogues et aux coins des rues, pour être vus des hommes. Je vous le dis en vérité, ils reçoivent leur récompense.
\VS{6}Mais toi, quand tu pries, entre dans ta chambre, et ayant fermé ta porte, prie ton Père, qui est là dans ce lieu secret ; et ton Père qui te voit dans ce lieu secret, te récompensera publiquement.
\VS{7}Quand vous priez, ne multipliez pas de vaines paroles, comme font les Gentils ; car ils s'imaginent qu'à force de paroles ils seront exaucés.
\VS{8}Ne leur ressemblez donc point ; car votre Père sait de quoi vous avez besoin, avant que vous le lui demandiez.
\TextTitle{Instructions de Jésus sur la prière}
\VS{9}Voici donc comment vous devez prier : Notre Père qui es aux cieux, que ton Nom soit sanctifié.
\VS{10}Que ton règne vienne. Que ta volonté soit faite sur la terre comme au ciel.
\VS{11}Donne-nous aujourd'hui notre pain quotidien.
\VS{12}Et remets nous nos dettes\FTNT{Du grec « opheilema » : « ce qui est légalement dû, une dette ». Les Ecritures considèrent le péché comme une dette. Voir Mat. 18:21-35.}, comme nous aussi nous remettons les dettes à nos débiteurs.
\VS{13}Ne nous induis pas en tentation ; mais délivre-nous du mal. Car c'est à toi qu'appartiennent, dans tous les siècles, le règne, et la puissance et la gloire. Amen !
\VS{14}Car si vous pardonnez aux hommes leurs offenses, votre Père céleste vous pardonnera aussi les vôtres.
\VS{15}Mais si vous ne pardonnez point aux hommes leurs offenses, votre Père ne vous pardonnera point non plus vos offenses.
\TextTitle{Attitude pendant le jeûne}
\VS{16}Et quand vous jeûnerez, ne prenez pas un air triste, comme font les hypocrites ; car ils se rendent le visage tout défait, afin de montrer aux hommes qu'ils jeûnent. Je vous le dis en vérité, ils reçoivent leur récompense.
\VS{17}Mais toi, quand tu jeûnes, oint ta tête et lave ton visage,
\VS{18}afin qu'il ne paraisse pas aux hommes que tu jeûnes, mais à ton Père qui est présent dans ton lieu secret ; et ton Père qui te voit dans ton lieu secret, te récompensera publiquement.
\TextTitle{Le trésor selon Dieu}
\VS{19}Ne vous amassez point des trésors sur la terre, où les vers et la rouille détruisent, et où les voleurs percent et dérobent.
\VS{20}Mais amassez-vous des trésors dans le ciel, où les vers et la rouille ne détruisent point, et où les voleurs ne percent ni ne dérobent.
\VS{21}Car là où est ton trésor, là aussi sera ton cœur.
\VS{22}L'œil est la lampe du corps. Si donc ton œil est en bon état, tout ton corps sera éclairé.
\VS{23}Mais si ton œil est mal disposé, tout ton corps sera ténébreux. Si donc la lumière qui est en toi n'est que ténèbres, combien seront grandes les ténèbres même ?
\VS{24}Nul ne peut servir deux maîtres. Car, ou il haïra l'un et aimera l'autre ; ou il s'attachera à l'un et méprisera l'autre. Vous ne pouvez pas servir Dieu et Mammon\FTNT{Mammon : Mot d'origine araméenne signifiant « riche ». Certains le rapprochent de l'hébreu « matmon » signifiant « trésor, argent ». D'autres le rapprochent du phénicien « mommon » signifiant « bénéfice ». Dans les évangiles, il signifie « possession » (matérielle), mais il est parfois personnifié.}.
\TextTitle{Rechercher le Royaume}
\VS{25}C'est pourquoi je vous dis : Ne vous inquiétez pas pour votre vie, de ce que vous mangerez, et de ce que vous boirez ; ni pour votre corps, de quoi vous serez vêtus. La vie n'est-elle pas plus que la nourriture et le corps plus que le vêtement ?
\VS{26}Considérez les oiseaux du ciel ; car ils ne sèment, ni ne moissonnent, ni n'assemblent dans des greniers, et cependant votre Père céleste les nourrit. N'êtes vous pas beaucoup plus excellents qu'eux ?
\VS{27}Et qui est celui d'entre vous qui puisse, par ses inquiétudes, ajouter une coudée à sa taille ?
\VS{28}Et pourquoi vous inquiéter au sujet du vêtement ? Apprenez comment croissent les lis des champs : Ils ne travaillent ni ne filent ;
\VS{29}cependant je vous dis que Salomon même, dans toute sa gloire, n'a pas été vêtu comme l'un d'eux.
\VS{30}Si donc Dieu revêt ainsi l'herbe des champs, qui est aujourd'hui sur pied, et qui demain sera jetée au four, ne vous vêtira-t-il pas à plus forte raison, ô gens de petite foi ?
\VS{31}Ne vous inquiétez donc point en disant : Que mangerons-nous ? Ou, que boirons-nous ? Ou de quoi serons-nous vêtus ?
\VS{32}Vu que les païens recherchent toutes ces choses; car votre Père céleste sait que vous avez besoin de toutes ces choses.
\VS{33}Mais cherchez premièrement le Royaume de Dieu et sa justice, et toutes ces choses vous seront données par-dessus.
\VS{34}Ne vous inquiétez donc pas pour le lendemain ; car le lendemain prendra soin de lui-même. A chaque jour suffit sa peine.
\Chap{7}
\TextTitle{Le jugement hypocrite\FTNTT{Lu. 6:37-42}}
\VerseOne{}Ne jugez point afin que vous ne soyez point jugés.
\VS{2}Car de tel jugement que vous jugez, vous serez jugés ; et de telle mesure que vous mesurerez, on vous mesurera réciproquement.
\VS{3}Et pourquoi vois-tu la paille qui est dans l'œil de ton frère, et n'aperçois-tu pas la poutre qui est dans ton œil ?
\VS{4}Ou comment peux-tu dire à ton frère : Permets que j'ôte de ton œil cette paille et n'aperçois-tu pas la poutre dans ton œil ?
\VS{5}Hypocrite, ôte premièrement de ton œil la poutre, et après cela tu verras comment tu ôteras la paille de l'œil de ton frère.
\VS{6}Ne donnez point les choses saintes aux chiens et ne jetez point vos perles devant les pourceaux, de peur qu'ils ne les foulent aux pieds, ne se retournent et ne vous déchirent.
\TextTitle{Exhortation à la prière}
\VS{7}Demandez et il vous sera donné. Cherchez et vous trouverez. Frappez et l'on vous ouvrira.
\VS{8}Car quiconque demande, reçoit ; et celui qui cherche trouve ; et l'on ouvre à celui qui frappe.
\VS{9}Lequel de vous donnera une pierre à son fils, s'il lui demande du pain ?
\VS{10}Ou, s'il lui demande un poisson, lui donnera-t-il un serpent ?
\VS{11}Si donc vous, méchants comme vous l'êtes, savez donner à vos enfants de bonnes choses, à combien plus forte raison votre Père qui est dans les cieux, donnera-t-il des bonnes choses à ceux qui les lui demandent ?
\TextTitle{La règle d'or de la loi et des prophètes\FTNTT{Lu. 6:31; Ep.4:32}}
\VS{12}Tout ce que vous voulez que les hommes fassent pour vous, faites-le de même pour eux, car c'est la loi et les prophètes.
\TextTitle{Les deux chemins\FTNTT{Ps. 1}}
\VS{13}Entrez par la porte étroite, car c'est la porte large et le chemin spacieux qui mènent à la perdition, et il y en a beaucoup qui entrent par elle.
\VS{14}Mais étroite est la porte, resserré le chemin qui mènent à la vie, et il y en a peu qui les trouvent.
\TextTitle{Les faux prophètes, reconnaissables à leurs fruits\FTNTT{Lu. 6:43-45}}
\VS{15}Gardez-vous des faux prophètes, ils viennent à vous en habits de brebis, mais au-dedans ce sont des loups ravisseurs.
\VS{16}Vous les reconnaîtrez à leurs fruits. Cueille-t-on des raisins sur des épines, ou des figues sur des chardons ?
\VS{17}Ainsi tout bon arbre porte de bons fruits ; mais le mauvais arbre porte de mauvais fruits.
\VS{18}Un bon arbre ne peut porter de mauvais fruits ni le mauvais arbre porter de bons fruits.
\VS{19}Tout arbre qui ne porte pas de bons fruits est coupé et jeté au feu.
\VS{20}Vous les reconnaîtrez donc à leurs fruits.
\TextTitle{Fausse confession\FTNTT{Lu. 6:46}}
\VS{21}Ceux qui me disent : Seigneur ! Seigneur ! N'entreront pas tous dans le Royaume des cieux ; mais celui qui fait la volonté de mon Père qui est dans les cieux.
\VS{22}Plusieurs me diront en ce jour-là : Seigneur ! Seigneur ! N'avons-nous pas prophétisé en ton Nom ? N'avons-nous pas chassé les démons en ton Nom ? N'avons-nous pas fait beaucoup de miracles en ton Nom ?
\VS{23}Alors je leur dirai ouvertement : Je ne vous ai jamais connus. Retirez-vous de moi, vous qui commettez l'iniquité.
\TextTitle{Parabole des deux bâtisseurs et des deux fondements\FTNTT{Lu. 6:47-49}}
\VS{24}Quiconque entend ces paroles que je dis, et les met en pratique, je le comparerai à un homme prudent qui a bâti sa maison sur le roc.
\VS{25}La pluie est tombée, les torrents sont venus, les vents ont soufflé contre cette maison : Elle n'est point tombée parce qu'elle était fondée sur le roc\FTNT{Jésus-Christ le Rocher : voir Es. 8:13-17.}.
\VS{26}Mais quiconque entend ces paroles que je dis et ne les met point en pratique, sera semblable à un homme insensé qui a bâti sa maison sur le sable.
\VS{27}La pluie est tombée, les torrents sont venus, les vents ont soufflé contre cette maison : Elle est tombée et sa ruine a été grande.
\TextTitle{Effet de l'enseignement}
\VS{28}Or il arriva que quand Jésus eut achevé ce discours, la foule fut frappée de sa doctrine ;
\VS{29}car il les enseignait comme ayant de l'autorité et non comme les scribes.
\Chap{8}
\TextTitle{Le lépreux guérit\FTNTT{Mc. 1:40-45}}
\VerseOne{}Et quand il fut descendu de la montagne, de grandes foules le suivirent.
\VS{2}Et voici, un lépreux vint et se prosterna devant lui, en lui disant : Seigneur\FTNT{Seigneur : Du grec « kurios ». C'est la première fois que ce terme est appliqué à Jésus. Notez que c'est un lépreux qui a eu la révélation que Jésus-Christ est YHWH.}, si tu veux, tu peux me rendre pur.
\VS{3}Et Jésus étendit la main, le toucha, en disant : Je le veux, sois pur. A l'instant même il fut purifié de sa lèpre.
\VS{4}Puis Jésus lui dit : Prends garde de ne le dire à personne ; mais va te montrer au sacrificateur et offre l'offrande que Moïse a prescrite afin que cela leur serve de témoignage.
\TextTitle{Guérison du serviteur d'un centenier\FTNTT{Lu. 7:1-10}}
\VS{5}Et quand Jésus fut entré dans Capernaüm, un centenier vint à lui, le priant
\VS{6}et disant : Seigneur, mon serviteur qui est paralytique est couché à la maison et il souffre extrêmement.
\VS{7}Jésus lui dit : J'irai et je le guérirai.
\VS{8}Mais le centenier lui répondit : Seigneur, je ne suis pas digne que tu entres sous mon toit ; mais dis seulement une parole et mon serviteur sera guéri.
\VS{9}Car moi-même qui suis un homme soumis à l'autorité d'un autre, j'ai des soldats sous mes ordres, et je dis à l'un : Va ! et il va ; et à un autre : Viens ! et il vient ; et à mon serviteur : Fais cela ! et il le fait.
\VS{10}Après l'avoir entendu, Jésus fut étonné et dit à ceux qui le suivaient : Je vous le dis en vérité, même en Israël je n'ai pas trouvé une aussi grande foi.
\VS{11}Or, je vous dis que plusieurs viendront de l'orient et de l'occident, et seront à table dans le Royaume des cieux, avec Abraham, Isaac et Jacob.
\VS{12}Et les enfants du Royaume seront jetés dans les ténèbres du dehors, où il y aura des pleurs et des grincements de dents.
\VS{13}Alors Jésus dit au centenier : Va, et qu'il te soit fait selon ta foi. Et à l'heure même, son serviteur fut guéri.
\TextTitle{Guérison de la belle-mère de Pierre\FTNTT{Mc. 1:29-34 ; Lu. 4:38-41}}
\VS{14}Puis Jésus alla à la maison de Pierre, dont il vit la belle-mère couchée et ayant la fièvre.
\VS{15}Il toucha sa main, et la fièvre la quitta ; puis elle se leva, et les servit.
\VS{16}Et le soir étant venu, on lui amena plusieurs démoniaques. Et il chassa par sa parole les esprits malins, et guérit tous ceux qui étaient malades,
\VS{17}afin que s'accomplisse ce qui avait été annoncé par Esaïe le prophète, en disant : Il a pris nos faiblesses et a porté nos maladies\FTNT{Es. 53:4.}.
\TextTitle{Les disciples éprouvés dans leur consécration\FTNTT{Lu. 9:57-62}}
\VS{18}Or Jésus voyant autour de lui de grandes foules, donna l'ordre de passer à l'autre rive.
\VS{19}Et un scribe s'approchant, lui dit : Maître, je te suivrai partout où tu iras.
\VS{20}Jésus lui dit : Les renards ont des tanières, et les oiseaux du ciel ont des nids ; mais le Fils de l'homme n'a pas de place pour reposer sa tête.
\VS{21}Puis un autre de ses disciples lui dit : Seigneur, permets-moi d'aller d'abord ensevelir mon père.
\VS{22}Et Jésus lui dit : Suis-moi et laisse les morts ensevelir leurs morts.
\TextTitle{Autorité de Jésus face à la tempête\FTNTT{Mc. 4:35-41 ; Lu. 8:22-25}}
\VS{23}Il monta dans la barque et ses disciples le suivirent.
\VS{24}Et voici, il s'éleva sur la mer une si grande tempête que la barque était couverte de flots ; et Jésus dormait.
\VS{25}Et ses disciples vinrent le réveiller en lui disant : Seigneur, sauve-nous, nous périssons !
\VS{26}Et il leur dit : Pourquoi avez-vous peur, gens de peu de foi ? Alors s'étant levé, il menaça les vents et la mer, et il se fit un grand calme.
\VS{27}Et les gens qui étaient là furent étonnés, et dirent : Qui est celui-ci à qui obéissent même les vents et la mer ?
\TextTitle{Deux aveugles et un démoniaque guéris\FTNTT{Mc. 5:1-20 ; Lu. 8:26-40}}
\VS{28}Et quand il fut passé de l'autre côté, dans le pays des Gadaréniens, deux démoniaques sortant des sépulcres, vinrent le rencontrer. Ils étaient si dangereux que personne ne pouvait passer par ce chemin-là.
\VS{29}Et voici, ils s'écrièrent : Qu'y a-t-il entre nous et toi, Jésus Fils de Dieu ? Es-tu venu ici nous tourmenter avant le temps ?
\VS{30}Et il y avait loin d'eux un grand troupeau de pourceaux qui paissaient.
\VS{31}Et les démons le priaient en disant : Si tu nous chasses dehors, permets-nous d'entrer dans ce troupeau de pourceaux.
\VS{32}Et il leur dit : Allez ! Et ils sortirent et entrèrent dans le troupeau de pourceaux. Et voici, tout le troupeau de pourceaux se précipita des pentes escarpées dans la mer et ils périrent dans les eaux.
\VS{33}Ceux qui les gardaient s'enfuirent et allèrent dans la ville, ils racontèrent toutes ces choses et ce qui était arrivé aux démoniaques.
\VS{34}Et voici, toute la ville alla à la rencontre de Jésus, et l'ayant vu, ils le prièrent de se retirer de leur pays.
\Chap{9}
\TextTitle{Un paralytique guéri\FTNTT{Mc. 2:3-12 ; Lu. 5:18-26}}
\VerseOne{}Alors, étant monté dans une barque, il traversa la mer et vint dans sa ville.
\VS{2}Et voici, on lui présenta un paralytique couché sur un lit. Et Jésus voyant leur foi, dit au paralytique : Prends courage, mon enfant ! Tes péchés te sont pardonnés.
\VS{3}Et voici, quelques-uns des scribes disaient au dedans d'eux : Cet homme blasphème.
\VS{4}Mais Jésus, connaissant leurs pensées, leur dit : Pourquoi avez-vous de mauvaises pensées dans vos cœurs ?
\VS{5}Car lequel est le plus aisé de dire : Tes péchés te sont pardonnés ; ou de dire : Lève-toi et marche ?
\VS{6}Or afin que vous sachiez que le Fils de l'homme a le pouvoir sur la terre de pardonner les péchés : Lève-toi, dit-il au paralytique, prends ton lit et va dans ta maison.
\VS{7}Et il se leva et s'en alla dans sa maison.
\VS{8}Quand la foule vit cela, elle fut saisie d'étonnement, et elle glorifiait Dieu qui a donné aux hommes un tel pouvoir.
\TextTitle{L'appel de Matthieu\FTNTT{Mc. 2:14 ; Lu. 5:27-28}}
\VS{9}De là, étant allé plus loin, Jésus vit un homme nommé Matthieu, assis au bureau du péage et il lui dit : Suis-moi ; et il se leva et le suivit.
\TextTitle{L'appel des pécheurs\FTNTT{Mc. 2:15-20 ; Lu. 5:29-35}}
\VS{10}Comme Jésus était à table dans la maison de Matthieu, beaucoup de publicains et des gens de mauvaise vie, qui étaient venus là, se mirent à table avec Jésus et avec ses disciples.
\VS{11}Les pharisiens virent cela et ils dirent à ses disciples : Pourquoi votre Maître mange-t-il avec les publicains et les gens de mauvaise vie ?
\VS{12}Jésus l'ayant entendu, leur dit : Ce ne sont pas ceux qui sont en bonne santé qui ont besoin de médecin, mais les malades.
\VS{13}Mais allez et apprenez ce que veulent dire ces paroles : Je prends plaisir à la miséricorde et non aux sacrifices\FTNT{Os. 6:6.}. Car je ne suis pas venu appeler à la repentance les justes, mais les pécheurs.
\VS{14}Alors les disciples de Jean vinrent auprès de lui et lui dirent : Pourquoi nous et les pharisiens jeûnons-nous souvent, tandis que tes disciples ne jeûnent point ?
\VS{15}Et Jésus leur répondit : Les amis de l'époux peuvent-ils s'affliger pendant que l'époux est avec eux ? Mais les jours viendront où l'époux leur sera enlevé, alors ils jeûneront.
\TextTitle{Parabole du drap neuf et des outres neuves\FTNTT{Mc. 2:21-22 ; Lu. 5:36-39}}
\VS{16}Aussi personne ne met une pièce de drap neuf à un vieil habit car la pièce emporterait une partie de l'habit et la déchirure serait pire.
\VS{17}On ne met pas non plus du vin nouveau dans de vieilles outres ; autrement les outres se rompent, et le vin se répand, et les outres sont perdues ; mais on met le vin nouveau dans des outres neuves, et l'un et l'autre se conservent.
\TextTitle{Résurrection de la fille de Jaïrus et guérison de la femme à la perte de sang\FTNTT{Mc. 5:21-43 ; Lu. 8:41-56}}
\VS{18}Tandis qu'il leur disait ces choses, voici, arriva un chef qui se prosterna devant lui, en lui disant : Ma fille est morte il y a un instant, mais viens, et impose-lui ta main et elle vivra.
\VS{19}Et Jésus s'étant levé le suivit avec ses disciples.
\VS{20}Et voici, une femme atteinte d'une perte de sang depuis douze ans s'approcha par-derrière et toucha le bord de son vêtement.
\VS{21}Car elle disait en elle-même : Si je puis seulement toucher son vêtement, je serai guérie.
\VS{22}Et Jésus se retourna, et dit en la voyant : Prends courage, ma fille ! Ta foi t'a sauvée. Et cette femme fut guérie à l'heure même.
\VS{23}Lorsque Jésus fut arrivé à la maison du chef et qu'il vit les joueurs de flûte et une foule bruyante,
\VS{24}il leur dit : Retirez-vous car la jeune fille n'est pas morte, mais elle dort ; et ils se moquaient de lui.
\VS{25}Quand la foule eut été renvoyée, il entra, prit la main de la jeune fille et elle se leva.
\VS{26}Et le bruit s'en répandit dans toute la contrée.
\TextTitle{Deux aveugles et un démoniaque guéris}
\VS{27}Etant parti de là, Jésus fut suivi par deux aveugles qui criaient : Fils de David, aie pitié de nous !
\VS{28}Et quand il fut arrivé dans la maison, les aveugles s'approchèrent de lui et Jésus leur dit : Croyez-vous que je puisse faire ce que vous me demandez ? Ils lui répondirent : Oui, Seigneur !
\VS{29}Alors il toucha leurs yeux en disant : Qu'il vous soit fait selon votre foi.
\VS{30}Et leurs yeux s'ouvrirent. Alors Jésus leur dit sévèrement : Prenez garde que personne ne le sache.
\VS{31}Mais, dès qu'ils furent sortis, ils répandirent sa renommée dans tout le pays.
\VS{32}Comme ils s'en allaient, voici, on présenta à Jésus un homme muet et démoniaque.
\VS{33}Et le démon ayant été chassé, le muet parla ; et les foules étonnées disaient : Jamais pareille chose ne s'est vue en Israël.
\VS{34}Mais les pharisiens disaient : Il chasse les démons par le prince des démons.
\VS{35}Jésus allait dans toutes les villes et les villages, enseignant dans leurs synagogues, et prêchant l'Evangile du Royaume, et guérissant toutes sortes de maladies et toutes sortes d'infirmités parmi le peuple.
\TextTitle{Jésus ému de compassion pour la foule\FTNTT{Mc. 6:34}}
\VS{36}Et voyant les foules, il fut ému de compassion, parce qu'elles étaient dispersées et errantes comme des brebis qui n'ont point de pasteur.
\VS{37}Et il dit à ses disciples : La moisson est grande, mais il y a peu d'ouvriers.
\VS{38}Priez donc le Maître de la moisson d'envoyer des ouvriers dans sa moisson.
\Chap{10}
\TextTitle{Appel et mission des douze apôtres\FTNTT{Mc. 6:7-13 ; Lu. 9:1-6}}
\VerseOne{}Alors Jésus ayant appelé ses douze disciples, leur donna le pouvoir de chasser les esprits impurs et de guérir toutes sortes de maladies et toutes sortes d'infirmités.
\VS{2}Et voici les noms des douze apôtres : Le premier est Simon, nommé Pierre, et André son frère ; Jacques, fils de Zébédée, et Jean, son frère ;
\VS{3}Philippe et Barthélemy ; Thomas, et Matthieu le péager ; Jacques, fils d'Alphée, et Lebbée, surnommé Thaddée.
\VS{4}Simon le Cananite, et Judas Iscariot, celui qui le livra.
\VS{5}Tels sont les douze que Jésus envoya, et leur donna ses ordres en disant : N'allez point vers les Gentils et n'entrez point dans aucune ville des Samaritains ;
\VS{6}Mais allez plutôt vers les brebis perdues de la maison d'Israël.
\VS{7}Et quand vous serez partis, prêchez, en disant : Le Royaume des cieux est proche.
\VS{8}Guérissez les malades, rendez purs les lépreux, ressuscitez les morts, chassez les démons hors des possédés. Vous l'avez reçu gratuitement, donnez-le gratuitement\FTNT{Vous avez reçu gratuitement : Aucun chrétien, quel que soit son appel ou son don ne peut prétendre qu'il a payé pour avoir les talents qu'il a reçus du Seigneur. Dans 1 Co. 4:7 Paul nous pose une question : « Qu'as-tu que tu n'aies reçu et si tu l'as reçu pourquoi te glorifies-tu ? » Dieu interroge également Job : « De qui suis-je le débiteur ? » (Job 41:2). Vendre quelque chose qu'on a reçu gratuitement n'est rien d'autre que du vol. Donnez gratuitement : C'est la suite logique des choses, on reçoit gratuitement et on donne gratuitement. Si nous aimons Dieu, nous devons garder sa Parole et marcher comme lui a marché (Jn. 14:15 ; 1 Jn. 2:6). Il a donné ses enseignements et nourri les gens gratuitement. Dans Ap. 21:6 et 22:17, le Seigneur invite toutes les personnes qui ont soif à venir s'abreuver gratuitement. Alors pourquoi vendre la Parole qu'on a reçue gratuitement ? Le Seigneur a envoyé les douze en mission et leur a demandé d'apporter l'évangile du Royaume, de guérir les malades et de délivrer les possédés gratuitement (Ac. 8:18-24 ; Ac. 20:33-35 ; Ap. 21:6 ; Ap. 22:17).}.
\VS{9}Ne prenez ni or, ni argent, ni monnaie dans vos ceintures ;
\VS{10}ni de sac pour le voyage, ni deux tuniques, ni souliers, ni bâton ; car l'ouvrier mérite sa nourriture.
\VS{11}Et dans quelque ville ou village que vous entriez, informez-vous qui y est digne de vous loger ; et demeurez chez lui jusqu'à ce que vous partiez de là.
\VS{12}Et quand vous entrerez dans quelque maison, saluez-la.
\VS{13}Et si cette maison en est digne, que votre paix vienne sur elle ; mais si elle n'en est pas digne, que votre paix retourne à vous.
\VS{14}Mais lorsque quelqu'un ne vous recevra point et n'écoutera point vos paroles, secouez, en partant de cette maison ou de cette ville, la poussière de vos pieds.
\VS{15}Je vous dis en vérité que ceux du pays de Sodome et de Gomorrhe seront traités moins rigoureusement au jour du jugement que cette ville-là.
\TextTitle{La proclamation du Royaume avant le retour du Messie}
\VS{16}Voici, je vous envoie comme des brebis au milieu des loups ; soyez donc prudents comme des serpents et simples comme des colombes.
\VS{17}Et mettez-vous en garde contre les hommes ; car ils vous livreront aux tribunaux et vous battront de verges dans leurs synagogues.
\VS{18}Et vous serez menés devant des gouverneurs et même devant des rois, à cause de moi, pour rendre témoignage de moi devant eux et aux nations.
\VS{19}Mais, quand ils vous livreront, ne vous inquiétez pas de ce que vous aurez à dire, ni comment vous parlerez. Ce que vous aurez à dire vous sera donné à l'heure même.
\VS{20}Car ce n'est pas vous qui parlez, mais c'est l'Esprit de votre Père qui parlera en vous.
\VS{21}Le frère livrera son frère à la mort, et le père son enfant ; et les enfants s'élèveront contre leurs pères et leurs mères, et les feront mourir.
\VS{22}Et vous serez haïs de tous à cause de mon Nom ; mais celui qui persévérera jusqu'à la fin sera sauvé.
\VS{23}Quand ils vous persécuteront dans une ville, fuyez dans une autre. Je vous le dis en vérité, vous n'aurez pas achevé de parcourir toutes les villes d'Israël, que le Fils de l'homme sera venu.
\TextTitle{La consécration du disciple et sa récompense}
\VS{24}Le disciple n'est point au-dessus du maître, ni le serviteur au-dessus de son seigneur.
\VS{25}Il suffit au disciple d'être traité comme son maître, et au serviteur comme son seigneur. S'ils ont appelé le père de famille Béelzébul, à combien plus forte raison appelleront-ils ainsi ses domestiques ?
\VS{26}Ne les craignez donc point. Car il n'y a rien de caché qui ne doive être découvert, ni rien de secret qui ne doive être connu.
\VS{27}Ce que je vous dis dans les ténèbres, dites-le dans la lumière ; et ce que je vous dis à l'oreille, prêchez-le sur les toits.
\VS{28}Et ne craignez point ceux qui tuent le corps et qui ne peuvent tuer l'âme ; mais craignez plutôt celui qui peut faire périr et l'âme et le corps en les jetant dans la géhenne.
\VS{29}Ne vend-on pas deux passereaux pour un sou ? Cependant, il n'en tombe pas un à terre sans la volonté de votre Père.
\VS{30}Et même les cheveux de votre tête sont tous comptés.
\VS{31}Ne craignez donc point : Vous valez plus que beaucoup de passereaux.
\VS{32}Quiconque donc me confessera devant les hommes, je le confesserai aussi devant mon Père qui est aux cieux.
\VS{33}Mais quiconque me reniera devant les hommes, je le renierai aussi devant mon Père qui est dans les cieux.
\VS{34}Ne croyez pas que je sois venu apporter la paix sur la terre. Je ne suis pas venu apporter la paix, mais l'épée.
\VS{35}Car je suis venu mettre en division le fils contre son père, et la fille contre sa mère, et la belle-fille contre sa belle-mère.
\VS{36}Et les propres domestiques d'un homme seront ses ennemis.
\VS{37}Celui qui aime son père ou sa mère plus que moi, n'est pas digne de moi ; et celui qui aime son fils ou sa fille plus que moi, n'est pas digne de moi.
\VS{38}Et quiconque ne prend pas sa croix et ne vient pas après moi, n'est pas digne de moi.
\VS{39}Celui qui aura conservé sa vie la perdra ; mais celui qui aura perdu sa vie pour l'amour de moi la retrouvera.
\VS{40}Celui qui vous reçoit me reçoit, et celui qui me reçoit, reçoit celui qui m'a envoyé.
\VS{41}Celui qui reçoit un prophète en qualité de prophète, recevra la récompense d'un prophète ; et celui qui reçoit un juste en qualité de juste recevra la récompense d'un juste.
\VS{42}Et quiconque aura donné à boire seulement un verre d'eau froide à l'un de ces petits parce qu'il est mon disciple, je vous le dis en vérité qu'il ne perdra point sa récompense.
\Chap{11}
\TextTitle{Jean-Baptiste le plus grand des hommes\FTNTT{Lu. 7:19-35}}
\VerseOne{}Et il arriva que quand Jésus eut achevé de donner ses ordres à ses douze disciples, il partit de là pour aller enseigner et prêcher dans leurs villes.
\VS{2}Jean, ayant entendu parler dans sa prison des œuvres du Christ, envoya deux de ses disciples pour lui dire :
\VS{3}Es-tu celui qui devait venir, ou devons-nous en attendre un autre ?
\VS{4}Et Jésus leur répondit : Allez, et rapportez à Jean les choses que vous entendez et que vous voyez.
\VS{5}Les aveugles recouvrent la vue, les boiteux marchent, les lépreux sont purifiés, les sourds entendent, les morts sont ressuscités, et l'Evangile est annoncé aux pauvres\FTNT{Jésus-Christ est le Dieu véritable dont la venue était annoncée par Esaïe (Es. 35:4-6).}.
\VS{6}Mais, heureux est celui qui n'aura point été scandalisé en moi ;
celui pour qui je ne serai pas une occasion de chute !
\VS{7}Et comme ils s'en allaient, Jésus se mit à dire à la foule au sujet de Jean : Mais qu'êtes-vous allés voir dans le désert ? Un roseau agité par le vent ?
\VS{8}Mais qu'êtes-vous allés voir ? Un homme vêtu de précieux vêtements ? Voici, ceux qui portent des habits précieux sont dans les maisons des rois.
\VS{9}Mais qu'êtes-vous allés voir ? Un prophète ? Oui, vous dis-je, et plus qu'un prophète.
\VS{10}Car c'est celui dont il est écrit : Voici, j'envoie mon messager\FTNT{Mal. 3:1.} devant ta face, pour préparer ton chemin devant toi.
\VS{11}En vérité, je vous le dis, parmi ceux qui sont nés de femmes, il n'en a point paru de plus grand que Jean-Baptiste. Toutefois, le plus petit dans le Royaume des cieux, est plus grand que lui.
\VS{12}Or depuis le temps de Jean-Baptiste jusqu'à maintenant, le Royaume des cieux est forcé et ce sont les violents qui s'en emparent.
\VS{13}Car tous les prophètes et la loi ont prophétisé jusqu'à Jean.
\VS{14}Et si vous voulez recevoir mes paroles, c'est lui qui est l'Elie\FTNT{Mal. 4:5-6.} qui devait venir.
\VS{15}Que celui qui a des oreilles pour entendre, entende.
\VS{16}Mais à qui comparerai-je cette génération ? Elle est semblable aux petits-enfants qui sont assis sur les places publiques, et qui crient à leurs compagnons
\VS{17}et leur disent : Nous vous avons joué de la flûte et vous n'avez point dansé ; nous vous avons chanté des complaintes et vous ne vous êtes point lamentés.
\VS{18}Car Jean est venu ne mangeant ni ne buvant et ils disent : Il a un démon.
\VS{19}Le Fils de l'homme est venu mangeant et buvant et ils disent : C'est un mangeur et un buveur, un ami des publicains et des gens de mauvaise vie. Mais la sagesse a été justifiée par ses enfants.
\TextTitle{Jésus dénonce les indifférents}
\VS{20}Alors il se mit à faire des reproches aux villes où il avait fait beaucoup de miracles, parce qu'elles ne s'étaient point repenties.
\VS{21}Malheur à toi, Chorazin ! Malheur à toi, Bethsaïda ! Car si les miracles qui ont été faits au milieu de vous, avaient été faits dans Tyr et dans Sidon, il y a longtemps qu'elles se seraient repenties, en prenant le sac et la cendre.
\VS{22}C'est pourquoi je vous dis que Tyr et Sidon seront traitées moins rigoureusement que vous, au jour du jugement.
\VS{23}Et toi Capernaüm, qui as été élevée jusqu'au ciel, tu seras précipitée jusqu'à Hadès\FTNT{Voir commentaire Mt. 16:18} ; car si les miracles qui ont été faits au milieu de toi, avaient été faits dans Sodome, elle subsisterait encore.
\VS{24}C'est pourquoi je vous dis que ceux de Sodome seront traités moins rigoureusement que toi, au jour du jugement.
\TextTitle{La relation personnelle du disciple avec son Seigneur}
\VS{25}En ce temps-là, Jésus prenant la parole dit : Je te loue, ô mon Père ! Seigneur du ciel et de la terre, de ce que tu as caché ces choses aux sages et aux intelligents, et que tu les as révélées aux petits enfants.
\VS{26}Oui, Père, je te loue parce que telle a été ta bonne volonté.
\VS{27}Toutes choses m'ont été données par mon Père ! Et personne ne connaît le Fils si ce n'est le Père ; et personne ne connaît le Père si ce n'est le Fils, et celui à qui le Fils veut le révéler.
\VS{28}Venez à moi vous tous qui êtes fatigués et chargés, et je vous donnerai du repos.
\VS{29}Prenez mon joug sur vous et recevez mes instructions, car je suis doux et humble de cœur ; et vous trouverez le repos pour vos âmes.
\VS{30}Car mon joug est doux et mon fardeau est léger.
\Chap{12}
\TextTitle{Jésus, le Maître du sabbat\FTNTT{Mc. 2:23-28 ; Lu. 6:1-5}}
\VerseOne{}En ce temps-là, Jésus traversa des champs de blé un jour de sabbat. Et ses disciples qui avaient faim se mirent à arracher des épis et à les manger.
\VS{2}Les pharisiens voyant cela, lui dirent : Voici, tes disciples font ce qu'il n'est pas permis de faire le jour du sabbat.
\VS{3}Mais il leur dit : N'avez-vous pas lu ce que fit David quand il eut faim, lui et ceux qui étaient avec lui ?
\VS{4}Comment il entra dans la maison de Dieu, et mangea les pains de proposition, qu'il ne lui était pas permis de manger ni à lui, ni à ceux qui étaient avec lui, mais aux sacrificateurs seulement ?
\VS{5}Ou n'avez-vous pas lu dans la loi, qu'aux jours du sabbat, les sacrificateurs violent le sabbat dans le temple, sans se rendre coupables ?
\VS{6}Or, je vous le dis, qu'il y a ici quelqu'un de plus grand que le temple.
\VS{7}Si vous saviez ce que signifient ces paroles : Je veux la miséricorde, et non pas le sacrifice, vous n'auriez pas condamné ceux qui ne sont pas coupables\FTNT{1 S. 15:22 ; Os. 6:6.}.
\VS{8}Car le Fils de l'homme est Maître même du sabbat.
\TextTitle{Jésus guérit l'homme à la main sèche le jour du sabbat\FTNTT{Mc. 3:1-5 ; Lu. 6:6-11}}
\VS{9}Puis étant parti de là, il entra dans leur synagogue.
\VS{10}Et voici, il s'y trouvait un homme qui avait la main sèche. Et pour avoir sujet de l'accuser, ils l'interrogèrent en disant : Est-il permis de guérir les jours du sabbat ?
\VS{11}Et il répondit : Lequel d'entre vous s'il n'a qu'une brebis, et qu'elle vienne à tomber dans une fosse le jour du sabbat, ne la saisira-t-il pas pour l'en retirer ?
\VS{12}Combien un homme ne vaut-il pas plus qu'une brebis ! Il est donc permis de faire du bien les jours du sabbat.
\VS{13}Alors il dit à cet homme : Etends ta main. Il l'étendit et elle devint saine comme l'autre.
\TextTitle{Jésus accomplit de nombreuses guérisons}
\VS{14}Les pharisiens sortirent et ils se consultèrent sur les moyens de le faire périr.
\VS{15}Mais Jésus, l'ayant su, partit de là, et de grandes foules le suivirent. Il les guérit tous.
\VS{16}Et il leur défendit avec menaces de le faire connaître,
\VS{17}afin que s'accomplît ce qui avait été annoncé par Esaïe le prophète, en disant :
\VS{18}Voici mon serviteur que j'ai élu, mon bien-aimé, qui est l'objet de mon amour, je mettrai mon Esprit en lui et il annoncera le jugement aux nations.
\VS{19}Il ne contestera point, il ne criera point et personne n'entendra sa voix dans les rues.
\VS{20}Il ne brisera point le roseau cassé et n'éteindra point le lumignon qui fume, jusqu'à ce qu'il ait fait triompher la justice.
\VS{21}Et les nations espéreront en son nom\FTNT{Es. 42:1-4.}.
\VS{22}Alors on lui amena un homme tourmenté d'un démon, aveugle et muet, et il le guérit ; de sorte que celui qui avait été aveugle et muet, parlait et voyait.
\VS{23}Et toutes les foules en furent étonnées, et elles disaient : Celui-ci n'est-il pas le Fils de David ?
\TextTitle{Le blasphème contre le Saint-Esprit\FTNTT{Mc. 3:22-30 ; Lu. 11:15-23}}
\VS{24}Mais les pharisiens ayant entendu cela, disaient : Celui-ci ne chasse les démons que par Béelzébul, prince des démons.
\VS{25}Mais Jésus connaissant leurs pensées, leur dit : Tout royaume divisé contre lui-même sera réduit en désert ; et toute ville, ou maison, divisée contre elle-même ne subsistera point.
\VS{26}Or si Satan chasse Satan, il est divisé contre lui-même ; comment donc son royaume subsistera-t-il ?
\VS{27}Et si je chasse les démons par Béelzébul, par qui vos fils les chassent-ils ? C'est pourquoi ils seront eux-mêmes vos juges.
\VS{28}Mais si je chasse les démons par l'Esprit de Dieu, certes le Royaume de Dieu est donc venu jusqu'à vous.
\VS{29}Ou, comment quelqu'un peut-il entrer dans la maison d'un homme fort et piller ses biens, sans avoir auparavant lié cet homme fort ? Alors il pillera sa maison.
\VS{30}Celui qui n'est pas avec moi, est contre moi, et celui qui n'assemble pas avec moi disperse.
\VS{31}C'est pourquoi je vous dis, que tout péché et tout blasphème sera pardonné aux hommes ; mais le blasphème contre l'Esprit ne leur sera point pardonné.
\VS{32}Quiconque parlera contre le Fils de l'homme, il lui sera pardonné ; mais quiconque parlera contre le Saint-Esprit, il ne lui sera pardonné ni dans ce siècle, ni dans le siècle à venir\FTNT{Le blasphème contre le Saint-Esprit : Le blasphème est un outrage, une calomnie à l'encontre de Dieu. En attribuant l'œuvre de Dieu à Satan, les pharisiens ont commis l'impardonnable. Beaucoup de personnes craignent d'avoir commis ce péché par inadvertance, en ayant par exemple un doute sur l'origine d'un miracle. La Parole nous recommande de ne pas ajouter foi à tout esprit, mais d'éprouver les esprits pour savoir s'ils sont de Dieu (1 Jn. 4:1). On ne pèche donc pas lorsqu'on exerce son discernement. De plus, si l'on a commis une erreur de jugement par ignorance, le Seigneur ne nous en tiendra pas rigueur (Ac. 17:30). Le blasphème contre le Saint-Esprit est commis par des personnes qui, bien qu'ayant la connaissance et la capacité de différencier le bien du mal, font preuve de mauvaise foi. Ainsi, les pharisiens avaient constaté les bons fruits portés par Jésus, mais ils ont hypocritement qualifié de mal le bien qu'il faisait (Es. 5:20). Ceux qui blasphèment contre le Saint-Esprit sont loin d'être ignorants. Comme nous l'atteste Hé. 6:4-6, parmi ces gens, certains « ont goûté le don céleste », « ont eu part au Saint-Esprit, et ont goûté la bonne parole de Dieu, et les puissances du siècle à venir ». En choisissant sciemment de pécher, alors qu'ils ont expérimenté la grâce de Dieu, ils retournent à ce qu'ils ont vomi et outragent ainsi le Seigneur (2 Pi. 2:18-22). Leur cœur endurci à l'extrême rejette volontairement la vérité pour s'attacher au mensonge. Constatant leur refus définitif de se repentir, le Saint-Esprit finit par se retirer pour laisser la place à l'esprit d'égarement qui les maintiendra dans l'erreur (2 Th. 2:9-12). Enfin, il est à noter qu'en Ap. 14:9-11, ceux qui ont reçu la marque de la bête sont condamnés d'office. Il ne faut nullement conclure que le Seigneur leur a refusé son pardon, mais plutôt que les personnes ayant reçu cette marque ont aussi blasphémé contre le Saint-Esprit. Voir commentaire en Ap. 13:16.}.
\TextTitle{Toute parole proclamée appelle un jugement}
\VS{33}Ou dites que l'arbre est bon et son fruit est bon ; ou dites que l'arbre est mauvais et son fruit est mauvais ; car on connaît l'arbre par le fruit.
\VS{34}Race de vipères, comment pourriez-vous dire de bonnes choses, méchants comme vous l'êtes ? Car c'est de l'abondance du cœur que la bouche parle.
\VS{35}L'homme de bien\FTNT{Le mot « bien » dans ce passage vient du grec « Agathos » qui signifie : « de bonne constitution ou nature », « utile », « salutaire », « bon », « agréable », plaisant », « joyeux », « heureux », « excellent », « distingué », « droit », « honorable ».} tire de bonnes choses du bon trésor de son cœur ; et l'homme méchant tire de mauvaises choses du mauvais trésor de son cœur.
\VS{36}Je vous le dis : Les hommes rendront compte au jour du jugement, de toute parole vaine qu'ils auront proférée.
\VS{37}Car tu seras justifié par tes paroles et tu seras condamné par tes paroles.
\TextTitle{Le miracle du prophète Jonas\FTNTT{Jon. 2:1 ; Lu. 11:29-32.}}
\VS{38}Alors quelques-uns des scribes et des pharisiens lui dirent : Maître, nous voudrions bien te voir faire quelque miracle.
\VS{39}Mais il leur répondit et dit : Une génération méchante et adultère demande un miracle, mais il ne lui sera point donné d'autre miracle que celui de Jonas le prophète.
\VS{40}Car, de même que Jonas fut trois jours et trois nuits dans le ventre d'un grand poisson, de même le Fils de l'homme sera trois jours et trois nuits dans le sein de la terre.
\VS{41}Les Ninivites se lèveront au jour du jugement contre cette génération et la condamneront, parce qu'ils se repentirent à la prédication de Jonas ; et voici, il y a ici plus que Jonas.
\TextTitle{La condamnation de cette génération par la reine de Séba\FTNTT{2 Ch. 9:1-12}}
\VS{42}La reine du Midi se lèvera au jour du jugement contre cette nation et la condamnera, parce qu'elle vint des extrémités de la terre pour entendre la sagesse de Salomon ; et voici, il y a ici plus que Salomon.
\TextTitle{Le retour de l'esprit impur\FTNTT{Lu. 11:24-26}}
\VS{43}Lorsque l'esprit impur est sorti d'un homme, il va par des lieux arides, cherchant du repos, mais il n'en trouve point.
\VS{44}Alors il dit : Je retournerai dans ma maison, d'où je suis sorti ; et quand il arrive, il la trouve vide, balayée et ornée.
\VS{45}Puis il s'en va et prend avec lui sept autres esprits plus méchants que lui ; qui y étant entrés, habitent là ; et ainsi la dernière condition de cet homme est pire que la première. Il en sera de même pour cette génération perverse.
\TextTitle{La famille spirituelle\FTNTT{Mc. 3:31-35 ; Lu. 8:19-21}}
\VS{46}Et comme il parlait encore aux foules, voici, sa mère et ses frères se tenaient dehors, cherchant à lui parler.
\VS{47}Et quelqu'un lui dit : Voici, ta mère et tes frères sont là dehors, qui cherchent à te parler.
\VS{48}Mais il répondit à celui qui lui avait dit cela : Qui est ma mère et qui sont mes frères ?
\VS{49}Et étendant sa main sur ses disciples, il dit : Voici ma mère et mes frères.
\VS{50}Car, quiconque fera la volonté de mon Père qui est dans les cieux, celui-là est mon frère, et ma sœur, et ma mère.
\Chap{13}
\TextTitle{1. Parabole des quatre terrains\FTNTT{Mc. 4:1-20 ; Lu. 8:4-15}}
\VerseOne{}Ce même jour, Jésus sortit de la maison et s'assit au bord de la mer.
\VS{2}Une grande foule s'assembla auprès de lui, c'est pourquoi il monta dans une barque et il s'assit. Aussi, toute la foule se tenait sur le rivage.
\VS{3}Et il leur parla en paraboles sur beaucoup de choses et il dit : Un semeur sortit pour semer.
\VS{4}Et comme il semait, une partie de la semence tomba le long du chemin, et les oiseaux vinrent, et la mangèrent toute.
\VS{5}Et une autre partie tomba dans les endroits pierreux où elle n'avait pas beaucoup de terre : Elle leva aussitôt parce qu'elle n'entrait pas profondément dans la terre ;
\VS{6}mais, quand le soleil parut, elle fut brûlée et sécha parce qu'elle n'avait point de racines.
\VS{7}Une autre partie tomba parmi les épines ; et les épines montèrent et l'étouffèrent.
\VS{8}Une autre partie tomba dans la bonne terre : Et elle donna du fruit, un grain en donna cent, un autre, soixante, et un autre, trente.
\VS{9}Que celui qui a des oreilles pour entendre, qu'il entende.
\TextTitle{Explication aux disciples}
\VS{10}Alors les disciples s'approchèrent et lui dirent : Pourquoi leur parles-tu en paraboles ?
\VS{11}Il leur répondit et dit : Parce qu'il vous a été donné de connaître les mystères du Royaume des cieux, et que cela ne leur a pas été donné de les connaître.
\VS{12}Car on donnera à celui qui a, et il sera dans l'abondance, mais à celui qui n'a pas, on ôtera même ce qu'il a.
\VS{13}C'est pourquoi je leur parle en paraboles, parce qu'en voyant, ils ne voient point, et qu'en entendant, ils n'entendent point et ne comprennent point.
\VS{14}Et ainsi s'accomplit pour eux la prophétie d'Esaïe qui dit : Vous entendrez de vos oreilles et vous ne comprendrez point ; et vous regarderez des yeux, et vous ne verrez point.
\VS{15}Car le cœur de ce peuple est engraissé, et ils ont endurci leurs oreilles, et ils ont fermé leurs yeux de peur qu'ils ne voient de leurs yeux, qu'ils n'entendent de leurs oreilles, qu'ils ne comprennent de leur cœur, qu'ils ne se convertissent et que je ne les guérisse\FTNT{Es. 6:9-10.}.
\VS{16}Mais heureux sont vos yeux, car ils voient ; et vos oreilles, parce qu'elles entendent.
\VS{17}Je vous le dis en vérité, beaucoup de prophètes et de justes ont désiré voir les choses que vous voyez, et ils ne les ont point vues, entendre les choses que vous entendez, et ils ne les ont point entendues.
\VS{18}Vous donc, écoutez la signification de la parabole du semeur.
\VS{19}Lorsqu'un homme écoute la parole du Royaume et ne la comprend pas, le malin vient et ravit ce qui est semé dans son cœur ; cet homme est celui qui a reçu la semence le long du chemin.
\VS{20}Et celui qui a reçu la semence dans les endroits pierreux, c'est celui qui entend la parole et la reçoit aussitôt avec joie ;
\VS{21}mais il n'a point de racine en lui-même, il croit pour un temps, et dès que survient une tribulation ou une persécution à cause de la parole, il y trouve une occasion de chute.
\VS{22}Et celui qui a reçu la semence parmi les épines, c'est celui qui entend la parole de Dieu, mais en qui les soucis du siècle et la séduction des richesses étouffent la parole et la rendent infructueuse.
\VS{23}Mais celui qui a reçu la semence dans la bonne terre, c'est celui qui entend la parole et la comprend. Il porte du fruit, et un grain donne cent, un autre soixante, et un autre trente.
\TextTitle{2. Parabole du blé et de l'ivraie}
\VS{24}Il leur proposa une autre parabole et il dit\FTNT{La parabole du blé et de l'ivraie. En méditant cette parabole, nous remarquons que lorsque le blé eut poussé et donné du fruit, l'ivraie parut aussi. Il est vrai que lorsqu'il y a un réveil spirituel divin dans une assemblée ou dans un pays, l'ennemi suscite aussi un faux réveil avec des faux ouvriers et des fausses manifestations spirituelles. Voilà pourquoi l'ivraie côtoiera le blé jusqu'à la fin du monde. Le mot « ivraie » se dit « ebriacus » en latin, ce qui donne « ébriété » en français. Nous comprenons donc que l'un des rôles de l'ivraie est d'enivrer le blé (les enfants de Dieu). Dans les Ecritures, l'ivresse est synonyme de la débauche spirituelle ou physique. En grec l'ivraie se dit « zizanion » qui donne en français « zizanie ». Voir Mt. 12:25. La division est l'œuvre de l'ivraie dans les églises qui cherche à créer des sectes et des partis pris.} : Le Royaume des cieux est semblable à un homme qui a semé de la bonne semence dans son champ.
\VS{25}Mais, pendant que les hommes dormaient, son ennemi vint, sema de l'ivraie parmi le blé, puis s'en alla.
\VS{26}Lorsque l'herbe eut poussé et donné du fruit, l'ivraie parut aussi.
\VS{27}Et les serviteurs du maître de la maison vinrent à lui et lui dirent : Seigneur, n'as-tu pas semé de la bonne semence dans ton champ ? D'où vient donc qu'il y a de l'ivraie ?
\VS{28}Mais il leur répondit : C'est un ennemi qui a fait cela. Et les serviteurs lui dirent : Veux-tu donc que nous allions l'arracher ?
\VS{29}Et il leur dit : Non, de peur qu'en arrachant l'ivraie, vous ne déraciniez le blé en même temps.
\VS{30}Laissez-les croître tous deux ensemble, jusqu'à la moisson ; et au temps de la moisson, je dirai aux moissonneurs : Arrachez premièrement l'ivraie, et liez-la en gerbes pour la brûler, mais amassez le blé dans mon grenier.
\TextTitle{3. Parabole du grain de sénevé\FTNTT{Mc. 4:30-32 ; Lu. 13:18-19}}
\VS{31}Il leur proposa une autre parabole et il dit : Le Royaume des cieux est semblable au grain de sénevé qu'un homme a pris et semé dans son champ.
\VS{32}C'est la plus petite de toutes les semences ; mais, quand il a poussé, il est plus grand que les autres plantes et devient un arbre, de sorte que les oiseaux du ciel viennent habiter et font leurs nids dans ses branches.
\TextTitle{4. Parabole du levain\FTNTT{Lu. 13:20-21}}
\VS{33}Il leur dit une autre parabole : Le Royaume des cieux est semblable à du levain qu'une femme a pris et mis dans trois mesures de farine, jusqu'à ce que toute la pâte soit levée.
\VS{34}Jésus dit à la foule toutes ces choses en paraboles, et il ne lui parlait point sans paraboles,
\VS{35}afin que s'accomplisse ce qui avait été annoncé par le prophète : J'ouvrirai ma bouche en paraboles, je déclarerai les choses qui ont été cachées dès la fondation du monde\FTNT{Ps. 78:2.}.
\TextTitle{Explication de la parabole du blé et de l'ivraie}
\VS{36}Alors Jésus renvoya la foule et entra dans la maison, et ses disciples s'approchèrent de lui et lui dirent : Explique-nous la parabole de l'ivraie du champ.
\VS{37}Et il leur répondit et dit : Celui qui sème la bonne semence, c'est le Fils de l'homme ;
\VS{38}le champ, c'est le monde ; la bonne semence ce sont les fils du Royaume, et l'ivraie ce sont les fils du malin ;
\VS{39}et l'ennemi qui l'a semée, c'est le diable ; la moisson, c'est la fin du monde, et les moissonneurs sont les anges.
\VS{40}Or, comme on arrache l'ivraie et qu'on la brûle au feu, il en sera de même à la fin de ce monde.
\VS{41}Le Fils de l'homme enverra ses anges qui arracheront de son Royaume tous les scandales et ceux qui commettent l'iniquité,
\VS{42}et les jetteront dans la fournaise ardente, où il y aura des pleurs et des grincements de dents.
\VS{43}Alors les justes resplendiront comme le soleil dans le Royaume de leur Père. Que celui qui a des oreilles pour entendre, qu'il entende.
\TextTitle{5. Parabole du trésor caché}
\VS{44}Le Royaume des cieux est encore semblable à un trésor caché dans un champ. L'homme qui l'a trouvé, le cache ; puis dans sa joie, il va vendre tout ce qu'il a, et achète ce champ.
\TextTitle{6. Parabole de la perle}
\VS{45}Le Royaume des cieux est encore semblable à un marchand qui cherche de bonnes perles.
\VS{46}Il a trouvé une perle de grand prix et il est allé vendre tout ce qu'il avait, et l'a achetée.
\TextTitle{7. Parabole du filet}
\VS{47}Le Royaume des cieux est encore semblable à un filet jeté dans la mer et ramassant toutes sortes de choses.
\VS{48}Quand il est rempli, les pêcheurs le tirent en haut sur le rivage, puis s'étant assis, ils mettent ce qu'il y a de bon à part dans leurs vases et jettent dehors ce qui est mauvais.
\VS{49}Il en sera de même à la fin du monde, les anges viendront séparer les méchants d'avec les justes,
\VS{50}et les jetteront dans la fournaise ardente, où il y aura des pleurs et des grincements de dents.
\VS{51}Jésus leur dit : Avez-vous compris toutes ces choses ? Ils lui répondirent : Oui, Seigneur.
\TextTitle{8. Le maître de la maison}
\VS{52}Et il leur dit : C'est pourquoi, tout scribe instruit de ce qui regarde le Royaume des cieux, est semblable à un père de famille qui tire de son trésor des choses nouvelles et des choses anciennes.
\TextTitle{Jésus à Nazareth\FTNTT{Mc. 6:1-6}}
\VS{53}Et quand Jésus eut achevé ces paraboles, il partit de là.
\VS{54}Et s'étant rendu dans sa patrie, il enseignait dans la synagogue, de telle sorte que ceux qui l'entendirent étaient étonnés et disaient : D'où lui viennent cette sagesse et ces miracles ?
\VS{55}Celui-ci n'est-il pas le fils du charpentier ? Sa mère ne s'appelle-t-elle pas Marie ? Et ses frères ne s'appellent-ils pas Jacques, Joseph, Simon et Jude ?
\VS{56}Et ses sœurs ne sont-elles pas toutes parmi nous ? D'où lui viennent donc toutes ces choses ?
\VS{57}Et il était pour eux une occasion de chute. Mais Jésus leur dit : Un prophète n'est méprisé que dans sa patrie et dans sa maison.
\VS{58}Et il ne fit là que peu de miracles, à cause de leur incrédulité.
\Chap{14}
\TextTitle{Mort de Jean-Baptiste\FTNTT{Mc. 6:14-29; Lu. 9:7-9}}
\VerseOne{}En ce temps-là, Hérode le tétrarque entendit parler de la renommée de Jésus, et il dit à ses serviteurs : C'est Jean-Baptiste !
\VS{2}Il est ressuscité des morts, c'est pourquoi la puissance de faire des miracles agit puissamment en lui.
\VS{3}Car Hérode avait fait arrêter Jean, et l'avait fait lier et mettre en prison, à cause d'Hérodias, femme de Philippe son frère.
\VS{4}Parce que Jean lui disait : Il ne t'est pas permis de l'avoir pour femme.
\VS{5}Et il voulait le faire mourir, mais il craignait la foule, parce qu'elle regardait Jean comme un prophète.
\VS{6}Or, le jour où l'on célébra la naissance d'Hérode, la fille d'Hérodias dansa au milieu de l'assemblée et plut à Hérode.
\VS{7}C'est pourquoi il lui promit avec serment de lui donner tout ce qu'elle demanderait.
\VS{8}A l'instigation de sa mère, elle dit : Donne-moi ici, sur un plat, la tête de Jean-Baptiste.
\VS{9}Le roi fut attristé ; mais à cause de ses serments et de ceux qui étaient à table avec lui, il commanda qu'on la lui donne.
\VS{10}Et il envoya décapiter Jean dans la prison.
\VS{11}Et sa tête fut apportée sur un plat et donnée à la fille qui la présenta à sa mère.
\VS{12}Puis ses disciples vinrent, et emportèrent son corps, et l'ensevelirent. Et ils allèrent l'annoncer à Jésus.
\VS{13}Et Jésus, ayant appris ce qu'Hérode avait fait, partit de là dans une barque, pour se retirer à l'écart dans un lieu désert ; et la foule l'ayant appris, sortit des villes voisines et le suivit à pied.
\VS{14}Et Jésus étant sorti, vit une grande foule, et il fut ému de compassion pour elle, et guérit les malades.
\TextTitle{Multiplication des pains pour les cinq mille hommes\FTNTT{Mc. 6:32-44 ; Lu. 9:12-17 ; Jn. 6:1-14}}
\VS{15}Et comme il se faisait tard, ses disciples vinrent à lui et lui dirent : Ce lieu est désert et l'heure est déjà avancée. Renvoie la foule, afin qu'elle aille dans les villages, pour s'acheter des vivres.
\VS{16}Mais Jésus leur dit : Ils n'ont pas besoin de s'en aller ; donnez-leur vous-mêmes à manger.
\VS{17}Et ils lui dirent : Nous n'avons ici que cinq pains et deux poissons.
\VS{18}Et il leur dit : Apportez-les-moi ici.
\VS{19}Et après avoir ordonné à la foule de s'asseoir sur l'herbe, il prit les cinq pains et les deux poissons, et levant les yeux au ciel, il rendit grâces à Dieu. Puis ayant rompu les pains, il les donna aux disciples qui les distribuèrent à la foule.
\VS{20}Tous en mangèrent et furent rassasiés, et l'on emporta douze paniers pleins des morceaux qui restaient.
\VS{21}Ceux qui avaient mangé étaient environ cinq mille hommes, sans compter les femmes et les petits enfants.
\TextTitle{Jésus marche sur les eaux, incrédulité de Pierre\FTNTT{Mc. 6:45-56 ; Jn. 6:15-21}}
\VS{22}Aussitôt après, Jésus obligea ses disciples à monter dans la barque et à passer avant lui de l'autre côté, pendant qu'il renverrait la foule.
\VS{23}Et quand il l'eut renvoyée, il monta sur une montagne pour être à part, afin de prier ; et le soir étant venu, il était là seul.
\VS{24}La barque, déjà au milieu de la mer, était battue par les flots ; car le vent était contraire.
\VS{25}Et vers la quatrième veille de la nuit, Jésus alla vers eux, marchant sur la mer.
\VS{26}Et ses disciples le voyant marcher sur la mer, ils furent troublés et ils dirent : C'est un fantôme ! Et, dans leur frayeur, ils poussèrent des cris.
\VS{27}Jésus leur dit aussitôt : Rassurez-vous, c'est moi, n'ayez pas de peur !
\VS{28}Et Pierre lui répondit : Seigneur, si c'est toi, ordonne que j'aille vers toi sur les eaux.
\VS{29}Et il lui dit : Viens ! Pierre sortit de la barque, marcha sur les eaux pour aller vers Jésus.
\VS{30}Mais voyant que le vent était fort, il eut peur ; et comme il commençait à enfoncer, il s'écria : Seigneur ! Sauve-moi !
\VS{31}Et aussitôt Jésus étendit sa main et le prit en lui disant : Homme de peu de foi, pourquoi as-tu douté ?
\VS{32}Et quand ils furent montés dans la barque, le vent s'apaisa.
\VS{33}Alors ceux qui étaient dans la barque, vinrent adorer Jésus et dirent : Certes, tu es le Fils de Dieu.
\TextTitle{Jésus guérit des malades à Génésareth\FTNTT{Mc. 6:53-56}}
\VS{34}Après avoir traversé la mer, ils vinrent dans le pays de Génézareth.
\VS{35}Les gens de ce lieu ayant reconnu Jésus, envoyèrent des messagers dans tous les environs et on lui amena tous les malades.
\VS{36}Et ils le prièrent de leur permettre de toucher seulement le bord de son vêtement. Et tous ceux qui le touchèrent furent guéris.
\Chap{15}
\TextTitle{Jésus-Christ condamne les traditions\FTNTT{Mc. 7:1-13}}
\VerseOne{}Alors des scribes et des pharisiens vinrent de Jérusalem auprès de Jésus et lui dirent :
\VS{2}Pourquoi tes disciples transgressent-ils la tradition des anciens ? Car ils ne se lavent point les mains quand ils prennent leur repas.
\VS{3}Il leur répondit : Et vous, pourquoi transgressez-vous le commandement de Dieu par votre tradition ?
\VS{4}Car Dieu a dit : Honore ton père et ta mère. Et il a dit aussi : Celui qui maudira son père ou sa mère finira à la mort.
\VS{5}Mais vous, vous dites : Celui qui dira à son père ou à sa mère : Tout ce dont j'aurais pu t'assister est une offrande à Dieu, n'est pas coupable, quoiqu'il n'honore pas son père ou sa mère.
\VS{6}Vous annulez ainsi le commandement de Dieu par votre tradition.
\VS{7}Hypocrites, Esaïe a bien prophétisé de vous, en disant :
\VS{8}Ce peuple s'approche de moi de sa bouche et m'honore des lèvres ; mais son cœur est très éloigné de moi.
\VS{9} Mais ils m'honorent en vain, en enseignant des doctrines qui ne sont que des commandements d'hommes\FTNT{Es. 29:13.}.
\TextTitle{Verdict sur le coeur humain\FTNTT{Mc. 7:14-23}}
\VS{10}Puis ayant appelé à lui la foule, il lui dit : Ecoutez, et comprenez ceci :
\VS{11}Ce n'est pas ce qui entre dans la bouche qui souille l'homme ; mais ce qui sort de la bouche c'est ce qui souille l'homme.
\VS{12}Sur cela les disciples s'approchant, lui dirent : Sais-tu que les pharisiens ont été scandalisés quand ils ont entendus ce discours ?
\VS{13}Et il répondit et dit : Toute plante que mon Père céleste n'a pas plantée sera déracinée.
\VS{14}Laissez-les, ce sont des aveugles, conducteurs d'aveugles ; si un aveugle conduit un autre aveugle, ils tomberont tous deux dans la fosse.
\VS{15}Alors Pierre prenant la parole, lui dit : Explique-nous cette parabole.
\VS{16}Et Jésus dit : Vous aussi, êtes-vous encore sans intelligence ?
\VS{17}Ne comprenez-vous pas encore que tout ce qui entre dans la bouche va dans le ventre, puis est jeté dans les lieux secrets ?
\VS{18}Mais les choses qui sortent de la bouche partent du cœur, et ces choses-là souillent l'homme.
\VS{19}Car c'est du cœur que sortent les mauvaises pensées, les meurtres, les adultères, les fornications, les vols, les faux témoignages, les médisances.
\VS{20}Ce sont ces choses-là qui souillent l'homme ; mais de manger sans avoir les mains lavées, cela ne souille point l'homme.
\TextTitle{Jésus et la femme cananéenne\FTNTT{Mc. 7:24-30}}
\VS{21}Alors Jésus, partant de là se retira dans le territoire de Tyr et de Sidon.
\VS{22}Et voici, une femme cananéenne, qui venait de ces contrées, lui cria : Seigneur ! Fils de David, aie pitié de moi ! Ma fille est cruellement tourmentée par le démon.
\VS{23}Mais il ne lui répondit pas un mot. Et ses disciples s'approchèrent et lui dirent : Renvoie-la, car elle crie derrière nous.
\VS{24}Et il répondit : Je n'ai été envoyé qu'aux brebis perdues de la maison d'Israël.
\VS{25}Mais elle vint et l'adora, disant : Seigneur, assiste-moi !
\VS{26}Et il lui répondit en disant : Il ne convient pas de prendre le pain des enfants et de le jeter aux petits chiens.
\VS{27}Mais elle dit : Cela est vrai, Seigneur ! Cependant les petits chiens mangent des miettes qui tombent de la table de leurs maîtres.
\VS{28}Alors Jésus répondant, lui dit : Ô femme ! Ta foi est grande. Qu'il te soit fait comme tu le souhaites. Et, à l'heure même, sa fille fut guérie.
\TextTitle{Nouvelles guérisons}
\VS{29}Et Jésus quitta ces lieux et vint près de la mer de Galilée. Puis il monta sur une montagne et s'y assit là.
\VS{30}Et une grande foule vint à lui, ayant avec elle des boiteux, des aveugles, des muets, des estropiés et beaucoup d'autres malades. On les mit aux pieds de Jésus et il les guérit ;
\VS{31}de sorte que la foule était dans l'admiration de voir que les muets parlaient, que les estropiés étaient guéris, que les boiteux marchaient, que les aveugles voyaient ; et elle glorifiait le Dieu d'Israël.
\TextTitle{Seconde multiplication des pains\FTNTT{Mc. 8:1-9}}
\VS{32}Alors Jésus, ayant appelé ses disciples, dit : Je suis ému de compassion pour cette foule de gens ; car voilà trois jours qu'ils sont près de moi et ils n'ont rien à manger. Je ne veux pas les renvoyer à jeun, de peur que les forces ne leur manquent en chemin.
\VS{33}Et ses disciples lui dirent : D'où pourrions-nous tirer dans ce désert assez de pains pour rassasier une si grande multitude ?
\VS{34}Et Jésus leur dit : Combien avez-vous de pains ? Ils lui dirent : Sept, et quelque peu de petits poissons.
\VS{35}Alors il commanda aux foules de s'asseoir par terre.
\VS{36}Et ayant prit les sept pains et les poissons, et après avoir béni Dieu, il les rompit et les donna à ses disciples, qui les distribuèrent à la foule.
\VS{37}Et tous mangèrent et furent rassasiés, et l'on emporta sept corbeilles pleines des morceaux qui restaient.
\VS{38}Or, ceux qui avaient mangé étaient quatre mille hommes, sans compter les femmes et les petits enfants.
\VS{39}Et Jésus renvoya la foule, monta sur une barque, et se rendit dans le territoire de Magdala.
\Chap{16}
\TextTitle{La cécité d'une génération méchante et adultère\FTNTT{Mc. 8:10-14}}
\VerseOne{}Alors les pharisiens et les sadducéens vinrent à lui, et pour l'éprouver, lui demandèrent de leur faire voir un signe venant du ciel.
\VS{2}Mais il leur répondit : Quand le soir est venu, vous dites : Il fera beau temps, car le ciel est rouge.
\VS{3}Et le matin vous dites : Il y aura de l'orage aujourd'hui, car le ciel est d'un rouge sombre. Hypocrites, vous savez bien discerner l'aspect du ciel, et vous ne pouvez discerner les signes des temps !
\VS{4}Une génération méchante et adultère demande un miracle ; mais il ne lui sera point donné d'autre miracle que celui de Jonas le prophète. Puis il les quitta et s'en alla.
\VS{5}Et ses disciples, en passant sur l'autre bord, avaient oublié de prendre des pains.
\TextTitle{Le levain des pharisiens et des sadducéens, une doctrine corrompue\FTNTT{Mc. 8:15-21 ; Lu. 12:1-15}}
\VS{6}Et Jésus leur dit : Gardez-vous avec soin du levain des pharisiens et des sadducéens.
\VS{7}Ils résonnaient en eux-mêmes et disaient : C'est parce que nous n'avons pas pris de pains.
\VS{8}Et Jésus connaissant leurs pensées leur dit : Gens de peu de foi, pourquoi raisonnez-vous en vous-mêmes sur le fait que vous n'avez pas pris de pains ?
\VS{9}Ne comprenez-vous point encore, et ne vous rappelez-vous plus les cinq pains des cinq mille hommes et combien de paniers vous avez emportés,
\VS{10}ni des sept pains des quatre mille hommes et combien de corbeilles vous avez emportées ?
\VS{11}Comment ne comprenez-vous pas que ce n'est pas au sujet du pain que je vous ai dit, de vous garder du levain des pharisiens et des sadducéens ?
\VS{12}Alors ils comprirent que ce n'était pas du levain du pain qu'il leur avait dit de se garder, mais de la doctrine des pharisiens et des sadducéens.
\TextTitle{Pierre reconnaît Jésus comme le Messie\FTNTT{Mc. 8:27-30 ; Lu. 9:18-21 ; Jn. 6:66-71}}
\VS{13}Jésus, étant arrivé dans le territoire de Césarée de Philippe, demanda à ses disciples : Qui disent les hommes que je suis, moi le Fils de l'homme ?
\VS{14}Et ils lui répondirent : Les uns disent que tu es Jean-Baptiste ; les autres, Elie ; et les autres, Jérémie, ou l'un des prophètes.
\VS{15}Il leur dit : et vous, qui dites-vous que je suis ?
\VS{16}Simon Pierre répondit et dit : Tu es le Christ, le Fils du Dieu vivant.
\TextTitle{Jésus bâtit son Eglise}
\VS{17}Et Jésus lui répondit et dit : Tu es heureux, Simon, fils de Jonas, car ce ne sont pas la chair et le sang qui t'ont révélé cela, mais mon Père qui est dans les cieux.
\VS{18}Et moi je te dis, que tu es Pierre, et que sur ce Roc\FTNT{Le Roc : Ce passage a été mal traduit dans beaucoup de Bibles comme suit : « Et moi, je te dis que tu es Pierre, et que sur cette pierre je bâtirai mon Eglise… ». Or pour une bonne compréhension des propos de Jésus, il est important d'insister sur la distinction que le grec fait entre « Petros » (pierre, caillou), l'apôtre Pierre, et « Petra » (roc, rocher), qui n'est autre que Jésus-Christ, le rocher des siècles (Es. 17:10 ; Es. 26:4 ; 1 Co. 10:4). De là en découle un enseignement fondamental : l'Eglise n'est bâtie ni par un homme ni sur l'homme, en l'occurrence Pierre et ses supposés successeurs (papes), mais par Jésus-Christ lui-même qui en est la Pierre Angulaire et le fondement inébranlable (Ac. 4:11 ; Ep. 2:20).} je bâtirai mon Eglise ; et les portes de l'enfer\FTNT{Enfer : du grec « Hadès ». Hadès chez les Grecs ou Pluton chez les Romains, était considéré comme le dieu des profondeurs souterraines et le maître des enfers. Ce terme est parfois traduit par « séjour des morts », équivalent hébreu de « Scheol ». Les Grecs utilisaient l'euphémisme Pylartes, signifiant « aux portes solidement closes », pour parler du très craint Hadès. En effet, Juifs, Grecs et Romains avaient conscience que les portes closes de l'enfer ne laissaient personne sortir du royaume de la mort. Tous les impies, et même les croyants d'avant Jésus-Christ, étaient retenus par les portes de l'enfer. Toutefois, les croyants allaient dans une partie de l'enfer que les juifs appelaient « sein d'Abraham » (1 Sam. 28:7-19 ; Lu. 16:22-25 ; Lu. 23:43) où ils ne subissaient pas les tourments infligés aux impies. Lorsque le Seigneur est mort, il est descendu « dans les régions inférieures de la terre » pour prendre les clés du Hadès, clés du séjour des morts (Col. 2:15 ; Ap. 1:17-18) et libérer les captifs pieux. Jésus affirme que les portes de l'enfer ne prévaudront jamais contre son Eglise puisque c'est lui qui l'a bâtie. Malgré tout, Hadès, bien que vaincu par le Seigneur, essaie d'attirer l'Eglise que le Seigneur a établie dans les lieux célestes (Ep. 2:4-9 ; Col. 3:1) vers le royaume des ténèbres, au travers des fausses doctrines et du péché. Au jour du jugement dernier, Hadès et la mort, qui sont deux démons, seront jetés dans l'étang de feu et de soufre (Ap. 20:11-15).} ne prévaudront point contre elle.
\VS{19}Je te donnerai les clefs du Royaume des cieux ; et tout ce que tu lieras sur la terre, sera lié dans les cieux ; et tout ce que tu délieras sur la terre, sera délié dans les cieux\FTNT{Une mauvaise compréhension de ce verset a contribué à propager l'idée erronée selon laquelle Pierre serait le médiateur entre Dieu et les hommes, puisque c'est lui qui détiendrait les clés du Royaume des cieux. Toutefois, Es. 22:22 affirme que seul Jésus-Christ détient ces clés qui symbolisent l'autorité et la domination. Or dans le cadre de l'héritage que le Seigneur nous a laissé, cette autorité est désormais exercée en son Nom par tous les membres du corps de Christ (Mt. 18:18).}.
\VS{20}Alors il commanda expressément à ses disciples de ne dire à personne qu'il était Jésus le Christ.
\TextTitle{Jésus parle de sa mort et de sa résurrection\FTNTT{Mc. 8:31-33 ; Lu. 9:22}}
\VS{21}Dès lors Jésus commença à déclarer à ses disciples qu'il fallait qu'il aille à Jérusalem, qu'il souffre beaucoup de la part des anciens, des principaux sacrificateurs et des scribes, qu'il soit mis à mort, et qu'il ressuscite le troisième jour.
\VS{22}Mais Pierre l'ayant pris à part, se mit à le reprendre en lui disant : Seigneur, aie pitié de toi, cela ne t'arrivera point !
\VS{23}Mais lui, s'étant retourné, dit à Pierre : Arrière de moi, Satan ! Tu m'es en scandale, car tu ne comprends pas les choses qui sont de Dieu, mais celles qui sont des hommes.
\TextTitle{La consécration du disciple\FTNTT{Mc. 8:34-38 ; Lu. 9:23-26}}
\VS{24}Alors Jésus dit à ses disciples : Si quelqu'un veut venir après moi, qu'il renonce à lui-même, et qu'il se charge de sa croix, et qu'il me suive.
\VS{25}Car quiconque voudra sauver son âme, la perdra ; mais quiconque perdra son âme à cause de son amour pour moi, la trouvera.
\VS{26}Et que servirait-il à un homme de gagner tout le monde, s'il perdait son âme ? Ou, que donnerait un homme en échange de son âme ?
\VS{27}Car le Fils de l'homme doit venir dans la gloire de son Père avec ses anges ; et alors il rendra à chacun selon ses œuvres.
\VS{28}Je vous le dis en vérité, quelques-uns de ceux qui sont ici présents, ne mourront point, qu'ils n'aient vu le Fils de l'homme venir dans son règne\FTNT{Ce passage doit être lu de concert avec Mt. 24:32-34. Jésus utilise un langage prophétique pour expliquer deux réalités. La première réalité est spirituelle et concerne ses contemporains qui allaient vivre l'effusion de l'Esprit pour rétablir le Royaume de Dieu dans le cœur des gens. En effet, le Seigneur ne les a pas laissés orphelins, mais il est revenu sous la forme de l'Esprit (Jn. 14:17-18 ; Ac. 2 ; Ac. 16:7). Aussi, les apôtres ont pu proclamer ce Royaume partout où ils allaient (Ac. 20:25). La deuxième réalité est matérielle et concerne le fleurissement du figuier, c'est-à-dire Israël. L'histoire atteste le fleurissement de ce figuier tant sur le plan géographique que sur le plan numérique. Depuis le 14 mai 1948, date de la naissance officielle de l'état hébreu, Israël ne cesse de s'étendre. Cette nation est l'horloge des temps car le Messie gouvernera le monde entier depuis Jérusalem (Mi. 4 ; Za. 14).}.
\Chap{17}
\TextTitle{Transfiguration de Jésus-Christ\FTNTT{Mc. 9:1-8 ; Lu. 9:27-36}}
\VerseOne{}Six jours après, Jésus prit Pierre, Jacques et Jean son frère, et les conduisit à l'écart sur une haute montagne.
\VS{2}Et il fut transfiguré en leur présence et son visage resplendit comme le soleil ; et ses vêtements devinrent blancs comme la lumière.
\VS{3}Et voici, ils virent Moïse et Elie qui s'entretenaient avec lui.
\VS{4}Alors Pierre prenant la parole, dit à Jésus : Seigneur, il est bon que nous soyons ici. Faisons-y, si tu le veux, trois tentes, une pour toi, une pour Moïse, et une pour Elie.
\VS{5}Et comme il parlait encore, voici une nuée resplendissante les couvrit de son ombre. Et voici, une voix fit entendre de la nuée ces paroles : Celui-ci est mon Fils bien-aimé, en qui j'ai pris mon bon plaisir : Ecoutez-le !
\VS{6}Lorsque les disciples entendirent cette voix, ils tombèrent le visage contre terre et furent saisis d'une très grande frayeur.
\VS{7}Mais Jésus, s'approchant, les toucha et leur dit : Levez-vous et n'ayez pas peur.
\VS{8}Ils levèrent les yeux, et ne virent personne, excepté Jésus tout seul.
\VS{9}Et comme ils descendaient de la montagne, Jésus leur donna cet ordre, en disant : Ne parlez à personne de cette vision, jusqu'à ce que le Fils de l'homme soit ressuscité des morts.
\VS{10}Et ses disciples l'interrogèrent, en disant : Pourquoi donc les scribes disent-ils qu'il faut qu'Elie vienne premièrement ?
\VS{11}Et Jésus répondant, leur dit : Il est vrai qu'Elie viendra premièrement et rétablira toutes choses.
\VS{12}Mais je vous dis qu'Elie est déjà venu, et ils ne l'ont pas reconnu et ils lui ont fait tout ce qu'ils ont voulu. De même, le Fils de l'homme doit souffrir aussi de leur part.
\VS{13}Alors les disciples comprirent que c'était de Jean-Baptiste qu'il leur parlait.
\TextTitle{Le manque de foi des disciples\FTNTT{Mc. 9:14-29 ; Lu. 9:37-43}}
\VS{14}Et quand ils furent arrivés près de la foule, un homme s'approcha et se mit à genoux devant lui,
\VS{15}et lui dit : Seigneur ! Aie pitié de mon fils qui est lunatique et misérablement affligé ; car il tombe souvent dans le feu et souvent dans l'eau.
\VS{16}Et je l'ai présenté à tes disciples, mais ils n'ont pas pu le guérir.
\VS{17}Et Jésus répondit et dit : Ô race incrédule et perverse, jusqu'à quand serai-je avec vous ? Jusqu'à quand vous supporterai-je ? Amenez-le-moi ici.
\VS{18}Et Jésus parla sévèrement au démon, qui sortit de lui, et à l'heure même l'enfant fut guéri.
\VS{19}Alors les disciples s'approchèrent de Jésus et lui dirent en particulier : Pourquoi n'avons-nous pas pu le chasser ?
\VS{20}Et Jésus leur répondit : C'est à cause de votre incrédulité. Je vous le dis en vérité, si vous aviez de la foi, comme un grain de sénevé, vous diriez à cette montagne : Transporte-toi d'ici là, et elle se transporterait ; et rien ne vous serait impossible.
\VS{21}Mais cette sorte de démon ne sort que par la prière et par le jeûne.
\TextTitle{Jésus évoque à nouveau sa mort et sa résurrection\FTNTT{Mc. 9:30-32 ; Lu. 9:44-45}}
\VS{22}Et comme ils se trouvaient en Galilée, Jésus leur dit : Il arrivera que le Fils de l'homme sera livré entre les mains des hommes ;
\VS{23}Et qu'ils le feront mourir, mais le troisième jour il ressuscitera. Et les disciples en furent fort attristés.
\TextTitle{La pièce d'argent dans la bouche d'un poisson\FTNTT{Mc. 12:13-17}}
\VS{24}Et lorsqu'ils arrivèrent à Capernaüm, ceux qui percevaient les deux drachmes s'adressèrent à Pierre et lui dirent : Votre Maître ne paye-t-il pas les deux drachmes ?
\VS{25}Oui dit-il. Et quand il fut entré dans la maison, Jésus le prévint en lui disant : Qu'est-ce qu'il t'en semble, Simon ? Les rois de la terre, de qui perçoivent-ils des tributs ou des impôts ? Est-ce de leurs enfants ou des étrangers ?
\VS{26}Pierre dit : Des étrangers. Jésus lui répondit : Les enfants en sont donc exempts.
\VS{27}Mais afin que nous ne les scandalisions point, va-t'en à la mer et jette l'hameçon, et prends le premier poisson qui viendra ; ouvre-lui la bouche, tu trouveras un statère. Prends-le et donne-le-leur pour moi et pour toi.
\Chap{18}
\TextTitle{L'humilité, secret de la vraie grandeur\FTNTT{Mc. 9:33-37; Lu. 9:46-48}}
\VerseOne{}En cette même heure-là, les disciples s'approchèrent de Jésus, en lui disant : Qui est le plus grand dans le Royaume des cieux ?
\VS{2}Et Jésus ayant appelé un petit enfant, le mit au milieu d'eux,
\VS{3}et leur dit : Je vous le dis en vérité, que si vous ne vous convertissez pas et si vous ne devenez pas comme les petits enfants, vous n'entrerez pas dans le Royaume des cieux.
\VS{4}C'est pourquoi quiconque deviendra humble, comme ce petit enfant, celui-là est le plus grand dans le Royaume des cieux.
\VS{5}Et quiconque reçoit en mon Nom un petit enfant comme celui-ci, il me reçoit.
\VS{6}Mais, quiconque scandalise un de ces petits qui croient en moi, il vaudrait mieux pour lui qu'on mette à son cou une meule d'âne, et qu'on le jette au fond de la mer.
\TextTitle{Les scandales et les occasions de chute}
\VS{7}Malheur au monde à cause des scandales ! Car il est nécessaire qu'il arrive des scandales ; mais malheur à l'homme par qui le scandale arrive !
\VS{8}Si ta main ou ton pied est pour toi une occasion de chute, coupe-les et jette-les loin de toi ; car il vaut mieux que tu entres boiteux ou manchot dans la vie, que d'avoir deux pieds ou deux mains, et d'être jeté dans le feu éternel.
\VS{9}Et si ton œil est pour toi une occasion de chute, arrache-le et jette-le loin de toi ; car il vaut mieux que tu entres dans la vie n'ayant qu'un œil, que d'avoir deux yeux, et d'être jeté dans le feu de la géhenne.
\VS{10}Gardez-vous de mépriser un seul de ces petits ; car je vous dis que dans les cieux leurs anges voient continuellement la face de mon Père qui est aux cieux.
\VS{11}Car le Fils de l'homme est venu pour sauver ce qui était perdu.
\TextTitle{Parabole de la brebis égarée\FTNTT{Lu. 15:3-7}}
\VS{12}Que vous en semble ? Si un homme a cent brebis, et que l'une d'elles s'égare, ne laisse-t-il pas les quatre-vingt-dix-neuf autres, pour aller dans les montagnes chercher celle qui s'est égarée ?
\VS{13}Et, s'il arrive qu'il la trouve, je vous le dis en vérité, qu'il en a plus de joie, que les quatre-vingt-dix-neuf qui ne se sont pas égarées.
\VS{14}Ainsi la volonté de votre Père qui est aux cieux n'est pas qu'un seul de ces petits périsse.
\TextTitle{Discipline dans les assemblées}
\VS{15}Que si ton frère a péché contre toi, va, et reprends-le entre toi et lui seul. S'il t'écoute, tu as gagné ton frère.
\VS{16}Mais s'il ne t'écoute pas, prends encore avec toi une ou deux personnes, afin que par la bouche de deux ou trois témoins toute parole soit ferme\FTNT{De. 19:15.}.
\VS{17}S'il refuse de les écouter, dis-le à l'Eglise ; et s'il refuse aussi d'écouter l'Eglise, qu'il soit pour toi comme un païen et comme un publicain.
\VS{18}En vérité je vous dis que tout ce que vous lierez sur la terre, sera lié dans le ciel ; et tout ce que vous délierez sur la terre sera délié dans le ciel\FTNT{Voir commentaire en Mt. 16:19.}.
\VS{19}Je vous dis aussi que si deux d'entre vous s'accordent sur la terre, tout ce qu'ils demanderont leur sera donné par mon Père qui est aux cieux.
\VS{20}Car là où deux ou trois sont assemblés en mon Nom, je suis là au milieu d'eux.
\TextTitle{Ne jamais se lasser de pardonner}
\VS{21}Alors Pierre s'approchant, lui dit : Seigneur, combien de fois mon frère péchera-il contre moi et lui pardonnerai-je ? Sera-ce jusqu'à sept fois ?
\VS{22}Jésus lui répondit : Je ne te dis pas jusqu'à sept fois, mais jusqu'à soixante-dix fois sept fois.
\TextTitle{Parabole du roi et du méchant serviteur}
\VS{23}C'est pourquoi le Royaume des cieux est semblable à un roi qui voulut faire rendre compte à ses serviteurs.
\VS{24}Et quand il se mit à compter, on lui en présenta un qui lui devait dix mille talents.
\VS{25}Et parce qu'il n'avait pas de quoi payer, son maître ordonna qu'il soit vendu, lui, sa femme, ses enfants et tout ce qu'il avait, et que la dette soit payée.
\VS{26}Mais ce serviteur se jetant à ses pieds, le suppliait en disant : Seigneur, aie patience envers moi et je te rendrai le tout.
\VS{27}Alors le maître de ce serviteur, ému de compassion, le relâcha et lui remit la dette.
\VS{28}Mais ce serviteur étant sorti, rencontra un de ses compagnons de service, qui lui devait cent deniers ; et l'ayant pris, il l'étranglait, en lui disant : paye-moi ce que tu me dois.
\VS{29}Mais son compagnon de service se jetant à ses pieds, le suppliait en disant : Aie patience et je te rendrai le tout.
\VS{30}Mais l'autre ne voulut pas et il alla le jeter en prison, jusqu'à ce qu'il ait payé la dette.
\VS{31}Or ses autres compagnons de service voyant ce qui était arrivé, en furent extrêmement attristés et ils allèrent raconter à leur maître tout ce qui s'était passé.
\VS{32}Alors son maître le fit venir et lui dit : Méchant serviteur, je t'avais remis en entier ta dette, parce que tu m'en avais supplié ;
\VS{33}Ne te fallait-il pas aussi avoir pitié de ton compagnon de service, comme j'avais eu pitié de toi ?
\VS{34}Et son maître étant en colère le livra aux bourreaux, jusqu'à ce qu'il lui ait payé tout ce qu'il devait.
\VS{35}C'est ainsi que vous fera mon Père céleste, si vous ne pardonnez de tout votre cœur, chacun à son frère, ses fautes.
\Chap{19}
\TextTitle{Enseignement de Jésus sur le mariage et le divorce\FTNTT{Mt. 5:31-32 ; Mc. 10:2-12 ; Lu. 16:18 ; Ro. 7:1-3 ; 1 Co. 7:10-16}}
\VerseOne{}Et il arriva que quand Jésus eut achevé ces discours, il quitta la Galilée, et alla dans le territoire de la Judée, au-delà du Jourdain.
\VS{2}Et de grandes foules le suivirent, et là il guérit leurs malades.
\VS{3}Alors les pharisiens vinrent à lui pour l'éprouver, et ils lui dirent : Est-il permis à un homme de répudier sa femme pour quelque cause que ce soit ?
\VS{4}Et il répondit et leur dit : N'avez-vous pas lu que le Créateur, au commencement, fit l'homme et la femme ?
\VS{5}Et il dit : A cause de cela, l'homme quittera son père et sa mère, et s'attachera à sa femme, et les deux ne seront qu'une seule chair ?
\VS{6}Ainsi ils ne sont plus deux, mais une seule chair\FTNT{Ge. 2:24.}. Que l'homme donc ne sépare pas ce que Dieu a mis ensemble sous un joug\FTNT{La majorité des traducteurs traduisent ce verset par « Que l'homme donc ne sépare pas ce que Dieu a joint. ». Or le terme grecque « suzeugnumi » qu'ils ont traduit par « joint » signifie plutôt « attacher un joug à quelqu'un, mettre ensemble sous un joug ». Un joug est une pièce de bois servant à atteler une paire d'animaux. De ce fait, les animaux sont contraints d'avancer dans la même direction, côte à côte (Ge. 2:22). En De. 22:10, Dieu interdit d'atteler un âne avec un bœuf ensemble. L'adverbe « ensemble » vient de l'hébreu « Yachad » et signifie « union d'une façon unitaire ». Ce verset fait référence symboliquement aux paroles de l'apôtre Paul en 2 Co. 6:14-16 qui nous mettent en garde contre le mariage avec des infidèles. Le mariage est donc semblable à un joug qui nous contraint à marcher à l'unisson dans la même direction. Ainsi, si on se lie à un inconverti, ce dernier risque de nous entraîner sur la voie de la perdition. En Mt 11:29, Christ nous invite à nous mettre sous son joug, qui est doux et léger. Quelle belle demande en mariage !}.
\VS{7}Ils lui dirent : Pourquoi donc Moïse a-t-il commandé de donner la lettre de divorce, et de répudier sa femme\FTNT{De. 24:1.} ?
\VS{8}Il leur répondit : C'est à cause de la dureté de votre cœur que Moïse vous a permis de répudier vos femmes, mais au commencement il n'en était pas ainsi.
\VS{9}Et moi je vous dis, que quiconque répudiera sa femme, si ce n'est pour cause d'adultère\FTNT{Adultère : Du grec « porneia » c'est-à-dire relation sexuelle illicite, impudicité.}, et se mariera à une autre, commet un adultère ; et que celui qui se sera marié à celle qui est répudiée, commet un adultère.
\VS{10}Ses disciples lui dirent : Si telle est la condition de l'homme à l'égard de sa femme, il ne convient pas de se marier.
\VS{11}Mais il leur répondit : Tous ne sont pas capables de cela, mais seulement ceux à qui il est donné.
\VS{12}Car il y a des eunuques, qui sont ainsi nés dés le ventre de leur mère ; et il y a des eunuques, qui ont été faits eunuques par les hommes ; et il y a des eunuques qui se sont faits eux-mêmes eunuques pour le Royaume des cieux. Que celui qui peut comprendre ceci, le comprenne.
\TextTitle{Le Royaume des cieux pour ceux qui ressemblent aux petits enfants\FTNTT{Mc. 10:13-16 ; Lu. 18:15-17}}
\VS{13}Alors on lui présenta des petits enfants, afin qu'il leur impose les mains et qu'il prie pour eux. Mais les disciples les en reprenaient.
\VS{14}Et Jésus leur dit : Laissez venir à moi les petits enfants et ne les empêchez pas ; car le Royaume des cieux est pour ceux qui leur ressemblent.
\VS{15}Puis il leur imposa les mains et il partit de là.
\TextTitle{Le jeune homme riche\FTNTT{Mc. 10:17-31 ; Lu. 10:25-37 ; Lu. 18:18-27.}}
\VS{16}Et voici, quelqu'un s'approchant lui dit : Maître qui est bon, quel bien ferai-je pour avoir la vie éternelle ?
\VS{17}Il lui répondit : Pourquoi m'appelles-tu bon ? Dieu est le seul être qui soit bon. Que si tu veux entrer dans la vie, garde les commandements.
\VS{18}Il lui dit : Lesquels ? Et Jésus lui répondit : Tu ne tueras point. Tu ne commettras point d'adultère. Tu ne déroberas point. Tu ne diras point de faux témoignage.
\VS{19}Honore ton père et ta mère ; et tu aimeras ton prochain comme toi-même\FTNT{Ex. 20:12-16 ; Lé. 19:18.}.
\VS{20}Le jeune homme lui dit : J'ai gardé toutes ces choses dès ma jeunesse. Que me manque-t-il encore ?
\VS{21}Jésus lui dit : Si tu veux être parfait, va, vends ce que tu as, et donne-le aux pauvres, et tu auras un trésor dans le ciel ; puis viens et suis-moi.
\VS{22}Mais quand ce jeune homme eut entendu cette parole, il s'en alla tout triste, parce qu'il avait de grands biens.
\VS{23}Alors Jésus dit à ses disciples : Je vous le dis en vérité, un riche entrera difficilement dans le Royaume des cieux.
\VS{24}Je vous le dis encore : Il est plus aisé à un chameau de passer par le trou d'une aiguille\FTNT{Le trou d'une aiguille : Jésus fait référence à une porte de la ville de Jérusalem qui était trop basse pour que les chameaux puissent y passer avec leurs chargements.}, qu'il ne l'est qu'un riche entre dans le Royaume de Dieu.
\VS{25}Ses disciples ayant entendu ces choses furent très étonnés et dirent : Qui peut donc être sauvé ?
\VS{26}Et Jésus les regarda et leur dit : Quant aux hommes, cela est impossible, mais quant à Dieu toutes choses sont possibles.
\TextTitle{Récompenses actuelles et dans le Royaume à venir\FTNTT{Mc. 10:28-31 ; Lu. 18:28-30}}
\VS{27}Alors Pierre prenant la parole, lui dit : Voici, nous avons tout quitté et nous t'avons suivi ; que nous en arrivera-t-il donc ?
\VS{28}Et Jésus leur dit : Je vous le dis en vérité, quand le Fils de l'homme, au renouvellement de toutes choses, sera assis sur le trône de sa gloire, vous qui m'avez suivi, vous serez assis sur douze trônes et vous jugerez les douze tribus d'Israël.
\VS{29}Et quiconque aura quitté ou maisons, ou frères, ou sœurs, ou père, ou mère, ou femme, ou enfants, ou champs, à cause de mon Nom, il en recevra cent fois autant et héritera la vie éternelle.
\VS{30}Mais plusieurs qui sont les premiers seront les derniers et les derniers seront les premiers.
\Chap{20}
\TextTitle{Parabole des ouvriers}
\VerseOne{} Car le Royaume des cieux est semblable à un père de famille, qui sortit dès le point du jour afin de louer des ouvriers pour sa vigne.
\VS{2}Et quand il eut accordé avec les ouvriers à un denier par jour, il les envoya à sa vigne.
\VS{3}Puis étant sorti vers la troisième heure, il en vit d'autres qui étaient sur la place publique, sans rien faire.
\VS{4}Il leur dit : Allez aussi à ma vigne, et je vous donnerai ce qui sera raisonnable.
\VS{5}Et ils y allèrent. Puis il sortit de nouveau vers la sixième heure et vers la neuvième, et il fit de même.
\VS{6}Et étant sorti vers la onzième heure, il en trouva d'autres qui étaient sur la place publique sans rien faire, et il leur dit : Pourquoi vous tenez-vous ici toute la journée sans rien faire ?
\VS{7}Ils lui répondirent : Parce que personne ne nous a loués. Et il leur dit : Allez-vous aussi à ma vigne et vous recevrez ce qui sera raisonnable.
\VS{8}Et le soir étant venu, le maître de la vigne dit à son intendant : Appelle les ouvriers et paye-leur le salaire, en commençant depuis les derniers jusqu'aux premiers.
\VS{9}Alors ceux qui avaient été loués vers la onzième heure vinrent et reçurent chacun un denier.
\VS{10}Or quand les premiers furent venus, ils croyaient recevoir davantage, mais ils reçurent aussi chacun un denier.
\VS{11}Et l'ayant reçu, ils murmuraient contre le père de famille,
\VS{12}en disant : Ces derniers n'ont travaillé qu'une heure, et tu les as faits égaux à nous, qui avons supporté le poids du jour et la chaleur.
\VS{13}Et il répondit à l'un d'eux et lui dit : Mon ami, je ne te fais pas de tort, n'es-tu pas tombé d'accord avec moi pour un denier ?
\VS{14}Prends ce qui est à toi et va-t'en. Mais je veux donner à ce dernier autant qu'à toi,
\VS{15}ne m'est-il pas permis de faire ce que je veux de mes biens ? Ou vois-tu d'un mauvais œil que je sois bon ?
\VS{16}Ainsi les derniers seront les premiers et les premiers seront les derniers, car il y a beaucoup d'appelés, mais peu d'élus.
\TextTitle{Jésus annonce à nouveau sa mort et sa résurrection\FTNTT{Mt. 12:38-42 ; 16:21-28 ; 17:22-23 ; Mc. 10:32-34 ; Lu. 18:31-34.}}
\VS{17}Pendant que Jésus montait à Jérusalem, il prit à part ses douze disciples et il leur dit en chemin :
\VS{18}Voici, nous montons à Jérusalem, et le Fils de l'homme sera livré aux principaux sacrificateurs et aux scribes, et ils le condamneront à la mort.
\VS{19}Ils le livreront aux nations pour qu'elles se moquent de lui, le battent de verges et le crucifient ; et le troisième jour il ressuscitera.
\TextTitle{Réponse de Jésus à la requête de la mère de Jacques et Jean\FTNTT{Mc. 10:35-45}}
\VS{20}Alors la mère des fils de Zébédée s'approcha de lui avec ses fils et se prosterna pour lui demander quelque chose.
\VS{21}Et il lui dit : Que veux-tu ? Elle lui dit : Ordonne que mes deux fils, qui sont ici, soient assis l'un à ta droite, et l'autre à ta gauche dans ton Royaume.
\VS{22}Et Jésus répondit et dit : Vous ne savez pas ce que vous demandez. Pouvez-vous boire la coupe que je dois boire, et être baptisés du baptême dont je dois être baptisé ? Ils lui répondirent : Nous le pouvons.
\VS{23}Et il leur dit : Il est vrai que vous boirez ma coupe et que vous serez baptisés du baptême dont je serai baptisé ; mais pour ce qui est d'être assis à ma droite ou à ma gauche, cela ne dépend pas de moi, et ne sera donné qu'à ceux à qui mon Père l'a réservé.
\VS{24}Les dix autres disciples ayant entendu cela, furent indignés contre les deux frères.
\VS{25}Mais Jésus les appela et leur dit : Vous savez que les princes des nations les dominent, et que les grands les asservissent.
\VS{26}Mais il n'en sera pas ainsi entre vous. Au contraire, quiconque veut être grand entre vous, qu'il soit votre serviteur.
\VS{27}Et quiconque veut être le premier parmi vous, qu'il soit votre serviteur.
\VS{28}De même que le Fils de l'homme n'est pas venu pour être servi, mais pour servir, et afin de donner sa vie en rançon pour plusieurs.
\TextTitle{Jésus guérit deux aveugles\FTNTT{Mc. 10:46-53 ; Lu. 18:35-43}}
\VS{29}Et comme ils partaient de Jéricho, une grande foule le suivit.
\VS{30}Et voici, deux aveugles qui étaient assis au bord du chemin, entendirent que Jésus passait, et crièrent en disant : Seigneur, Fils de David ! Aie pitié de nous !
\VS{31}Et la foule les reprenait pour les faire taire ; mais ils criaient encore plus fort : Seigneur, Fils de David ! Aie pitié de nous !
\VS{32}Jésus s'arrêta les appela et leur dit : Que voulez-vous que je vous fasse ?
\VS{33}Ils lui dirent : Seigneur, que nos yeux soient ouverts.
\VS{34}Et Jésus étant ému de compassion, toucha leurs yeux, et aussitôt ils recouvrèrent la vue et ils le suivirent.
\Chap{21}
\TextTitle{Jésus-Christ se présente publiquement comme Roi\FTNTT{Za. 9:9 ; Mc. 11:1-11 ; Lu. 19:28-40 ; Jn. 12:12-19.}}
\VerseOne{}Et quand ils furent près de Jérusalem, et qu'ils furent arrivés à Bethphagé vers le Mont des Oliviers, Jésus envoya alors deux disciples,
\VS{2}en leur disant : Allez au village qui est devant vous. Vous trouverez une ânesse attachée, et son ânon avec elle. Détachez-les et amenez-les-moi.
\VS{3}Et si quelqu'un vous dit quelque chose, vous direz que le Seigneur en a besoin ; et aussitôt il les laissera aller.
\VS{4}Or, tout cela arriva afin que s'accomplisse ce qui avait été annoncé par le prophète, en disant :
\VS{5}Dites à la fille de Sion : Voici, ton Roi vient à toi, plein de douceur, et monté sur un âne, sur un ânon, le petit d'une ânesse\FTNT{Za. 9:9.}.
\VS{6}Les disciples donc s'en allèrent et firent ce que Jésus leur avait ordonné.
\VS{7}Et ils amenèrent l'ânesse et l'ânon, et mirent leurs vêtements sur eux, et le firent asseoir dessus.
\VS{8}Alors de grandes foules étendirent leurs vêtements sur le chemin, et les autres coupaient des rameaux des arbres, et les étendaient sur le chemin.
\VS{9}Et les foules qui allaient devant, et celles qui suivaient, criaient en disant : Hosanna au Fils de David ! Béni soit celui qui vient au Nom du Seigneur ! Hosanna dans les lieux très hauts !
\VS{10}Lorsqu'il entra dans Jérusalem, toute la ville fut émue et l'on disait : Qui est celui-ci ?
\VS{11}Et les foules disaient : C'est Jésus, le prophète de Nazareth en Galilée.
\TextTitle{Jésus chasse les marchands du temple\FTNTT{Mc. 11:15-18 ; Lu. 19:45-46 ; Jn. 2:13-16}}
\VS{12}Jésus entra dans le temple de Dieu. Il chassa dehors tous ceux qui vendaient et qui achetaient dans le temple ; il renversa les tables des changeurs et les sièges de ceux qui vendaient des pigeons ;
\VS{13}et il leur dit : Il est écrit : Ma maison sera appelée une maison de prière, mais vous en avez fait une caverne de voleurs\FTNT{Es. 56:7 ; Jé. 7:11.}.
\VS{14}Alors des aveugles et des boiteux s'approchèrent de lui dans le temple et il les guérit.
\VS{15}Mais les principaux sacrificateurs et les scribes furent indignés à la vue des choses merveilleuses qu'il avait faites, et des enfants qui criaient dans le temple : Hosanna au Fils de David !
\VS{16}Et ils lui dirent : Entends-tu ce qu'ils disent ? Oui, leur répondit Jésus. N'avez-vous jamais lu ces paroles : Tu as tiré des louanges de la bouche des enfants, et de ceux qui sont à la mamelle\FTNT{Ps. 8:3.} ?
\VS{17}Et, les ayant laissés, il sortit de la ville, pour aller à Béthanie, où il passa la nuit.
\TextTitle{Le figuier stérile\FTNTT{Mc. 11:12-14,20-26}}
\VS{18}Le matin, comme il retournait à la ville, il eut faim.
\VS{19}Et voyant un figuier qui était sur le chemin, il s'en approcha, mais il n'y trouva que des feuilles ; et il lui dit : Qu'aucun fruit ne naisse plus jamais de toi ! Et aussitôt le figuier sécha.
\VS{20}Les disciples qui virent cela furent étonnés et dirent : Comment ce figuier est-il devenu sec en un instant ?
\VS{21}Jésus leur répondit : Je vous le dis en vérité, si vous aviez la foi, et que vous ne doutiez point, non seulement vous ferez ce qui a été fait à ce figuier, mais quand vous diriez à cette montagne : Ôte-toi de là et jette-toi dans la mer, cela se ferait.
\VS{22}Et quoi que vous demandiez en priant Dieu si vous croyez, vous le recevrez.
\TextTitle{L'incrédulité des principaux sacrificateurs et des anciens\FTNTT{Mc. 11:27-33 ; Lu. 20:1-8}}
\VS{23}Puis, s'étant rendu dans le temple, les principaux sacrificateurs et les anciens du peuple vinrent auprès de lui, pendant qu'il enseignait, et lui dirent : Par quelle autorité fais-tu ces choses ; et qui t'a donné cette autorité ?
\VS{24}Jésus répondant leur dit : Je vous interrogerai aussi sur une chose, et si vous me répondez, je vous dirai par quelle autorité je fais ces choses.
\VS{25}Le baptême de Jean d'où venait-il ? Du ciel ou des hommes ? Mais ils raisonnèrent ainsi entre eux : Si nous disons : Du ciel, il nous dira : Pourquoi n'avez-vous pas cru en lui ?
\VS{26}Et si nous disons : Des hommes, nous craignons la foule, car tous tiennent Jean pour un prophète.
\VS{27}Alors ils répondirent à Jésus : Nous ne savons pas. Et il leur dit : Moi non plus, je ne vous dirai pas par quelle autorité je fais ces choses.
\TextTitle{Parabole des deux fils}
\VS{28}Mais que vous en semble ? Un homme avait deux fils ; et s'adressant au premier, il lui dit : Mon fils, va travailler aujourd'hui dans ma vigne.
\VS{29}Il répondit : Je ne veux pas y aller. Ensuite il se repentit et y alla.
\VS{30}S'adressant à l'autre, il lui dit la même chose. Et ce fils répondit : Je veux bien, seigneur. Et il n'alla pas.
\VS{31}Lequel des deux a fait la volonté du père ? Ils lui répondirent : Le premier. Et Jésus leur dit : Je vous le dis en vérité, les publicains et les prostituées vous devanceront dans le Royaume de Dieu.
\VS{32}Car Jean est venu à vous dans la voie de la justice et vous ne l'avez pas cru ; mais les publicains et les femmes débauchées ont cru en lui. Et vous, qui avez vu cela, vous ne vous êtes pas ensuite repentis pour croire en lui.
\TextTitle{Parabole des vignerons\FTNTT{Es. 5:1-7 ; Mc. 12:1-12 ; Lu. 20:9-18.}}
\VS{33}Ecoutez une autre parabole : Il y avait un père de famille qui planta une vigne, et l'entoura d'une haie, et y creusa un pressoir, et bâtit une tour ; puis il l'afferma à des vignerons, et quitta le pays.
\VS{34}Lorsque la saison de la récolte fut arrivé, il envoya ses serviteurs vers les vignerons pour recevoir les fruits.
\VS{35}Mais les vignerons s'étant saisis de ses serviteurs, fouettèrent l'un, tuèrent l'autre et lapidèrent le troisième.
\VS{36}Il envoya encore d'autres serviteurs en plus grand nombre que les premiers, et ils leur firent de même.
\VS{37}Enfin, il envoya vers eux son propre fils, en disant : Ils auront du respect pour mon fils.
\VS{38}Mais quand les vignerons virent le fils, ils dirent entre eux : Voici l'héritier. Venez, tuons-le et emparons-nous de son héritage.
\VS{39}Et s'étant saisis de lui, il le jetèrent hors de la vigne et le tuèrent.
\VS{40}Quand donc le maître de la vigne viendra, que fera-t-il à ces vignerons ?
\VS{41}Ils lui dirent : Il les fera périr malheureusement comme des méchants et louera sa vigne à d'autres vignerons, qui lui en rendront les fruits en leur saison.
\VS{42}Et Jésus leur dit : N'avez-vous jamais lu dans les Ecritures : La pierre qu'ont rejetée ceux qui bâtissaient, est devenue la principale de l'angle. C'est du Seigneur que cela est venu, et c'est un prodige à nos yeux\FTNT{Es. 8:13-17 ; Es. 28:16.} ?
\VS{43}C'est pourquoi je vous dis que le Royaume de Dieu vous sera enlevé et il sera donné à une nation qui en rendra les fruits.
\VS{44}Celui qui tombera sur cette pierre s'y brisera, et celui sur qui elle tombera sera écrasé.
\VS{45}Après avoir entendu ses paraboles, les principaux sacrificateurs et les pharisiens comprirent qu'il parlait d'eux.
\VS{46}Et ils cherchaient à se saisir de lui, mais ils craignaient la foule, parce qu'elle le tenait pour un prophète.
\Chap{22}
\TextTitle{Parabole des noces\FTNTT{Lu. 14:16-24}}
\VerseOne{}Alors Jésus, prenant la parole, leur parla de nouveau en paraboles et il dit :
\VS{2}Le Royaume des cieux est semblable à un roi qui fit des noces pour son fils.
\VS{3}Il envoya ses serviteurs pour appeler ceux qui avaient été conviés aux noces ; mais ils ne voulurent pas venir.
\VS{4}Il envoya encore d'autres serviteurs, disant : Dites aux conviés : Voici, j'ai préparé mon festin ; mes bœufs et mes bêtes grasses sont tués, et tout est prêt ; venez aux noces.
\VS{5}Mais, sans tenir compte de l'invitation, ils s'en allèrent l'un à son champ, et l'autre à son trafic.
\VS{6}Et les autres se saisirent de ses serviteurs les outragèrent, et les tuèrent.
\VS{7}Quand le roi l'entendit, il se mit en colère ; il envoya ses troupes, fit périr ces meurtriers et brûla leur ville.
\VS{8}Puis il dit à ses serviteurs : Les noces sont prêtes, mais les conviés n'en étaient pas dignes.
\VS{9}Allez donc dans les carrefours des chemins, et autant de gens que vous trouverez, appelez-les aux noces.
\VS{10}Alors ces serviteurs allèrent dans les chemins et rassemblèrent tous ceux qu'ils trouvèrent, méchants et bons, et la salle des noces fut remplie de conviés qui étaient à table.
\VS{11}Et le roi étant entré pour voir ceux qui étaient à table, il aperçut là un homme qui n'avait pas revêtu un habit de noces\FTNT{Ap. 19:7-8.}.
\VS{12}Et il lui dit : Mon ami, comment es-tu entré ici sans avoir un habit de noces ? Cet homme eut la bouche fermée.
\VS{13}Alors le roi dit aux serviteurs : Liez-lui les pieds et les mains, emportez-le et jetez-le dans les ténèbres de dehors, où il y aura des pleurs et des grincements de dents.
\VS{14}Car il y a beaucoup d'appelés, mais peu d'élus.
\TextTitle{Le tribut dû à César\FTNTT{Mc. 12:13-17 ; Lu. 20:19-26}}
\VS{15}Alors les pharisiens allèrent se consulter ensemble sur les moyens de le surprendre par ses propres paroles.
\VS{16}Ils envoyèrent auprès de lui leurs disciples, avec des hérodiens, qui dirent : Maître, nous savons que tu es véritable, que tu enseignes la voie de Dieu selon la vérité, sans t'inquiéter de personne ; car tu ne regardes point à l'apparence des hommes.
\VS{17}Dis-nous donc ce qu'il t'en semble : Est-il permis de payer le tribut à César, ou non ?
\VS{18}Et Jésus connaissant leur malice, dit : Hypocrites, pourquoi me tentez-vous ?
\VS{19}Montrez-moi la monnaie avec laquelle on paie le tribut ; et ils lui présentèrent un denier.
\VS{20}Il leur demanda : De qui porte-t-il l'image et l'inscription ?
\VS{21}De César, lui répondirent-ils. Alors il leur dit : Rendez donc à César ce qui est à César, et à Dieu, ce qui est à Dieu.
\VS{22}Et ayant entendu cela, ils furent étonnés, ils le quittèrent et s'en allèrent.
\TextTitle{Enseignement de Jésus sur la résurrection\FTNTT{Mc. 12:18-27 ; Lu. 20:27-38}}
\VS{23}Le même jour, les sadducéens, qui disent qu'il n'y a pas de résurrection, vinrent auprès de lui et lui posèrent cette question,
\VS{24}en disant : Maître, Moïse a dit : Si quelqu'un meurt sans enfants, son frère épousera sa femme et suscitera une postérité à son frère.
\VS{25}Or, il y avait parmi nous sept frères. Le premier se maria et mourut ; et, n'ayant pas eu d'enfants, il laissa sa femme à son frère.
\VS{26}Il en fut de même du deuxième, puis du troisième, jusqu'au septième.
\VS{27}Après eux tous, la femme mourut aussi.
\VS{28}A la résurrection, duquel des sept sera-t-elle la femme ? Car tous l'ont eue.
\VS{29}Mais Jésus répondant leur dit : Vous êtes dans l'erreur, parce que vous ne connaissez ni les Ecritures, ni la puissance de Dieu.
\VS{30}Car à la résurrection on ne prendra ni on ne donnera de femmes en mariage, mais on sera comme les anges de Dieu dans le ciel.
\VS{31}Et quant à la résurrection des morts, n'avez-vous point lu ce dont Dieu vous a parlé, disant :
\VS{32}Je suis le Dieu d'Abraham, le Dieu d'Isaac et le Dieu de Jacob\FTNT{Ge. 17:7 ; Ge. 26:24 ; Ge. 28:21.}. Or Dieu n'est pas le Dieu des morts, mais des vivants.
\VS{33}Ce que les foules ayant entendu, elles admirèrent sa doctrine.
\TextTitle{Le plus grand commandement de la loi\FTNTT{Mc. 12:28-34 ; Lu. 10:25-28}}
\VS{34}Quand les pharisiens apprirent qu'il avait fermé la bouche aux sadducéens, ils se rassemblèrent dans un même lieu,
\VS{35}et l'un d'eux, qui était docteur de la loi, l'interrogea pour l'éprouver, en disant :
\VS{36}Maître, quel est le plus grand commandement de la loi ?
\VS{37}Jésus lui dit : Tu aimeras le Seigneur ton Dieu de tout ton cœur, de toute ton âme et de toute ta pensée.
\VS{38}Celui-ci est le premier et le grand commandement.
\VS{39}Et voici le deuxième qui lui est semblable : Tu aimeras ton prochain comme toi-même.
\VS{40}De ces deux commandements dépendent toute la loi et les prophètes.
\TextTitle{Jésus interroge les pharisiens au sujet du Messie\FTNTT{Mc. 12:35-37 ; Lu. 20:39-44}}
\VS{41}Et les Pharisiens étant assemblés, Jésus les interrogea,
\VS{42}Disant : que pensez-vous du Christ ? De qui est-il Fils ? Ils lui répondirent : de David.
\VS{43}Et il leur dit : Comment donc David, parlant par l'Esprit, l'appelle-t-il son Seigneur ? Disant :
\VS{44}Le Seigneur a dit à mon Seigneur, assieds-toi à ma droite, jusqu'à ce que j'aie mis tes ennemis pour le marchepied de tes pieds\FTNT{Ps. 110:1.}.
\VS{45}Si donc David l'appelle son Seigneur, comment est-il son Fils ?
\VS{46}Et personne ne pouvait lui répondre un seul mot. Et depuis ce jour, personne n'osa plus lui poser des questions.
\Chap{23}
\TextTitle{Caractéristiques des scribes et des pharisiens\FTNTT{Mc. 12:38-40 ; Lu. 11:39-54 ; Lu. 20:45-47}}
\VerseOne{}Alors Jésus parla à la foule et à ses disciples,
\VS{2}disant : Les scribes et les pharisiens sont assis dans la chaire de Moïse.
\VS{3}Toutes les choses donc qu'ils vous diront d'observer, observez-les et faites-les, mais non point leurs œuvres : parce qu'ils disent et ne font pas.
\VS{4}Car ils lient ensemble des fardeaux pesants et insupportables et les mettent sur les épaules des hommes ; mais ils ne veulent point les remuer de leur doigt.
\VS{5}Et ils font toutes leurs œuvres pour être vus des hommes. Ainsi, ils portent de larges phylactères et de longues franges à leurs vêtements.
\VS{6}Ils aiment les premières places dans les festins, et les premiers sièges dans les synagogues.
\VS{7}Ils aiment les salutations dans les places publiques, et à être appelés par les hommes : Notre maître ! Notre maître !
\VS{8}Mais vous, ne vous faites pas appeler, Notre maître ; car Christ seul est votre Docteur ; et vous êtes tous frères.
\VS{9}Et n'appelez personne sur la terre votre père ; car un seul est votre Père, celui qui est dans les cieux.
\VS{10}Et ne soyez point appelés Docteurs : car Christ seul est votre Docteur.
\VS{11}Mais que celui qui est le plus grand entre vous, soit votre serviteur.
\VS{12}Car quiconque s'élèvera sera abaissé ; et quiconque s'abaissera, sera élevé.
\VS{13}Mais malheur à vous, scribes et pharisiens hypocrites, qui fermez le Royaume des cieux aux hommes : car vous-mêmes n'y entrez point, et vous n'y laissez pas entrer ceux qui veulent y entrer.
\VS{14}Malheur à vous, scribes et pharisiens hypocrites, car vous dévorez les maisons des veuves, même sous le prétexte de faire de longues prières, c'est pourquoi vous en recevrez une plus grande condamnation.
\VS{15}Malheur à vous, scribes et pharisiens hypocrites ! Parce que vous courez la mer et la terre pour faire un prosélyte, et quand il l'est devenu, vous le rendez fils de la géhenne, deux fois plus que vous.
\VS{16}Malheur à vous conducteurs aveugles, qui dites : Si quelqu'un jure par le temple, ce n'est rien ; mais si quelqu'un jure par l'or du temple, il est engagé.
\VS{17}Insensés et aveugles ! Car lequel est le plus grand, l'or, ou le temple qui sanctifie l'or ?
\VS{18}Si quelqu'un, dites-vous encore, jure par l'autel, ce n'est rien ; mais si quelqu'un jure par l'offrande qui est sur l'autel, il est engagé.
\VS{19}Insensés et aveugles ! Car lequel est le plus grand, l'offrande, ou l'autel qui sanctifie l'offrande ?
\VS{20}Celui donc qui jure par l'autel, jure par l'autel et par toutes les choses qui sont dessus.
\VS{21}Celui qui jure par le temple, jure par le temple et par celui qui y habite ;
\VS{22}et celui qui jure par le ciel, jure par le trône de Dieu et par celui qui y est assis.
\VS{23}Malheur à vous, scribes et pharisiens hypocrites ! Parce que vous payez la dîme\FTNT{Voir commentaire en Mal. 3:10.} de la menthe, de l'aneth et du cumin ; et vous laissez les choses les plus importantes de la loi, c'est-à-dire la justice, la miséricorde et la fidélité. Il fallait pratiquer ces choses-là, sans négliger les autres choses.
\VS{24}Conducteurs aveugles ! Vous coulez le moucheron et vous engloutissez le chameau\FTNT{Les pharisiens filtraient leur eau par crainte d'avaler un moucheron.}.
\VS{25}Malheur à vous, scribes et pharisiens hypocrites ! Parce que vous nettoyez le dehors de la coupe et du plat ; alors qu'au-dedans ils sont pleins de rapine et d'intempérance.
\VS{26}Pharisien aveugle, nettoie premièrement l'intérieur de la coupe et du plat, afin que l'extérieur aussi devienne net.
\VS{27}Malheur à vous, scribes et pharisiens hypocrites ! Parce que vous êtes semblables aux sépulcres blanchis, qui paraissent beaux au-dehors, et qui au-dedans sont pleins d'ossements de morts, et de toutes espèces d'impuretés.
\VS{28}Ainsi, au-dehors vous paraissez justes aux hommes, mais au-dedans vous êtes pleins d'hypocrisie et d'iniquité.
\VS{29}Malheur à vous, scribes et pharisiens hypocrites ! Parce que vous bâtissez les tombeaux des prophètes et vous ornez les sépulcres des justes ;
\VS{30}et vous dites : Si nous avions vécu du temps de nos pères, nous n'aurions pas participé avec eux au meurtre des prophètes.
\VS{31}Ainsi vous êtes témoins contre vous-mêmes, que vous êtes les enfants de ceux qui ont fait mourir les prophètes.
\VS{32}Et vous achevez de remplir la mesure de vos pères.
\VS{33}Serpents, race de vipères ! Comment éviterez-vous le supplice de la géhenne ?
\VS{34}Car voici, je vous envoie des prophètes, des sages et des scribes. Vous tuerez et crucifierez les uns, vous battrez de verges les autres dans vos synagogues, et vous les persécuterez de ville en ville,
\VS{35}afin que vienne sur vous tout le sang innocent qui a été répandu sur la terre, depuis le sang d'Abel le juste, jusqu'au sang de Zacharie, fils de Barachie, que vous avez tué entre le temple et l'autel.
\VS{36}Je vous le dis en vérité, que toutes ces choses viendront sur cette génération.
\TextTitle{Lamentations de Jésus sur Jérusalem\FTNTT{Jé. 22:5 ; Lu. 13:34-35 ; 19:41-44.}}
\VS{37}Jérusalem, Jérusalem, qui tues les prophètes, et qui lapides ceux qui te sont envoyés, combien de fois ai-je voulu rassembler tes enfants, comme la poule rassemble ses poussins sous ses ailes, et vous ne l'avez point voulu !
\VS{38}Voici, votre maison va devenir déserte.
\VS{39}Car je vous dis, que désormais vous ne me verrez plus, jusqu'à ce que vous disiez : Béni soit celui qui vient au Nom du Seigneur\FTNT{Ps. 118:26.}!
\Chap{24}
\TextTitle{Prophétie sur la destruction du temple de Jérusalem\FTNTT{Mc. 13:1-2 ; Lu. 21:5-6}}
\VerseOne{}Comme Jésus sortait et s'en allait du temple, ses disciples s'approchèrent de lui pour lui faire remarquer les bâtiments du temple.
\VS{2}Mais Jésus leur dit : Voyez-vous bien toutes ces choses ? Je vous le dis en vérité, il ne restera pas ici pierre sur pierre qui ne soit démolie.
\TextTitle{Le signe de l'accomplissement\FTNTT{Mc. 13:3-4 ; Lu. 21:7}}
\VS{3}Puis s'étant assis sur la Montagne des Oliviers, ses disciples vinrent à lui en particulier et lui dirent : Dis-nous quand ces choses arriveront, et quel sera le signe de ton avènement, et de la fin du monde ?
\TextTitle{Les temps de la fin\FTNTT{Da. 9:27 ; Mc. 13:5-13 ; Lu. 21:8-11}}
\VS{4}Et Jésus répondant leur dit : Prenez garde que personne ne vous séduise.
\VS{5}Car plusieurs viendront sous mon Nom, disant : Je suis le Christ. Et ils en séduiront plusieurs.
\VS{6}Vous entendrez parler de guerres et de bruits de guerres ; gardez-vous d'être troublés ; car il faut que toutes ces choses arrivent ; mais ce ne sera pas encore la fin.
\VS{7}Car une nation s'élèvera contre une autre nation, et un royaume contre un autre royaume ; et il y aura des famines, des pestes, et des tremblements de terre en divers lieux.
\VS{8}Mais toutes ces choses ne seront que le commencement des douleurs.
\VS{9}Alors ils vous livreront aux tourments, et vous tueront ; et vous serez haïs de toutes les nations, à cause de mon Nom.
\VS{10}Alors aussi plusieurs seront scandalisés, se trahiront et se haïront les uns les autres.
\VS{11}Et il s'élèvera plusieurs faux prophètes, qui en séduiront plusieurs.
\VS{12}Et parce que l'iniquité sera multipliée, la charité de plusieurs se refroidira.
\VS{13}Mais celui qui persévérera jusqu'à la fin, sera sauvé.
\VS{14}Cet Evangile du Royaume sera prêché dans toute la terre habitable, pour servir de témoignage à toutes les nations, et alors viendra la fin.
\TextTitle{L'abomination de la désolation\FTNTT{Da. 9:27 ; Da.11:32-35 ; Mc. 13:14-18 ; Lu. 21:20-23}}
\VS{15}Or quand vous verrez l'abomination qui causera la désolation, qui a été prédite par Daniel le Prophète\FTNT{Daniel fut le premier à parler de l'abomination de la désolation (Da. 9:24-27). « Et les forces se présenteront de sa part, elles profaneront le sanctuaire, la forteresse, elles feront cesser le sacrifice perpétuel, et dresseront l'abomination qui causera la désolation. » Da. 11:31. Cette prophétie s'est partiellement accomplie en 168 av. J.-C. lorsqu'Antiochus Epiphane (215 av. J.-C. - 163 av. J.-C.), roi de Syrie, défenseur zélé de la culture grecque, finança la construction du temple de Zeus à Athènes. Sa tentative d'hellénisation forcée de la Judée, soutenue par les grands prêtres Jason et Ménélas, provoqua la colère des Juifs traditionalistes. Antiochus avait interdit le culte mosaïque et consacra le temple de Jérusalem aux dieux grecs. En effet, il le pilla et y installa un autel du dieu Baal Shamen, puis il détruisit les murailles de la ville. Dans un édit de décembre 167 av. J.-C., il ordonna d'offrir des porcs en holocauste, interdit la circoncision, la lecture de la Torah, l'observance des fêtes de Yahweh et pourchassa les adversaires de l'hellénisation. En agissant de la sorte, il y avait deux choses principales que cet archétype de l'antichrist espérait changer en Israël : Les temps (le calendrier juif) et la Torah (la loi selon Da. 7:25). La deuxième partie de cette prophétie s'est accomplie en l'an 70 lors de la destruction du temple de Jérusalem par Titus (39-81 ap. J.-C.). La dernière partie de cette prophétie est en train de s'accomplir actuellement dans les assemblées où Satan distille des enseignements erronés au travers des faux prophètes. La prédication d'un évangile expurgé de son caractère christocentrique, de la nécessité de porter sa croix, axé sur les choses de ce monde, maintient les chrétiens dans une vie de péché. Ainsi, alors qu'ils sont censés être des temples vivants du Saint-Esprit (1 Co. 6:19), Satan s'est établi dans leurs cœurs. Enfin, la prophétie de Daniel trouvera son parfait accomplissement pendant le règne de la bête. L'homme impie s'introduira alors dans le temple de Jérusalem qui sera rebâti, se fera passer pour Dieu et se fera adorer à sa place (2 Th. 2:4).}, être établie dans le lieu saint, que celui qui lit ce prophète y fasse attention !
\VS{16}Alors, que ceux qui seront en Judée fuient dans les montagnes ;
\VS{17}et que celui qui sera sur le toit ne descende pas pour emporter quoi que ce soit de sa maison ;
\VS{18}que celui qui sera dans les champs ne retourne pas en arrière pour prendre ses habits.
\VS{19}Malheur aux femmes enceintes, et à celles qui allaiteront en ces jours-là.
\VS{20}Priez pour que votre fuite n'arrive pas en hiver, ni un jour de sabbat\FTNT{Sous la loi mosaïque, il était interdit aux juifs de parcourir plus de 2 000 coudées du lieu où ils se trouvaient pendant le sabbat (Ex. 16:29).}.
\TextTitle{La grande tribulation\FTNTT{Ps. 2:5 ; Jé. 30:5-8 ; Da.12:1 ; Mc. 13:19-23 ; Lu. 21:23-24}}
\VS{21}Car alors, la détresse sera si grande qu'il n'y en a point eu de semblable depuis le commencement du monde jusqu'à présent, et qu'il n'y en aura jamais.
\VS{22}Et si ces jours n'étaient abrégés, personne ne serait sauvé ; mais à cause des élus, ces jours seront abrégés.
\VS{23}Alors si quelqu'un vous dit : Voici, le Christ est ici ; ou, il est là ; ne le croyez point.
\VS{24}Car il s'élèvera de faux christs et de faux prophètes, ils feront de grands prodiges et des miracles, pour séduire même les élus, s'il était possible.
\VS{25}Voici, je vous l'ai prédit.
\VS{26}Si on vous dit : Voici, il est dans le désert, ne sortez point ; voici, il est dans les chambres, ne le croyez point.
\VS{27}Car, comme l'éclair part de l'orient et se montre jusqu'en occident, il en sera de même de l'avènement du Fils de l'homme.
\VS{28}Car là où est le cadavre, là s'assembleront les vautours.
\TextTitle{Retour du Roi sur la terre\FTNTT{Mc. 13:24-27 ; Lu. 21:25-28}}
\VS{29}Aussitôt après ces jours de détresse, le soleil s'obscurcira, la lune ne donnera plus sa lumière, et les étoiles tomberont du ciel, et les puissances des cieux seront ébranlées.
\VS{30}Alors le signe du Fils de l'homme paraîtra dans le ciel, toutes les tribus de la terre se lamenteront en se frappant la poitrine, et verront le Fils de l'homme venant sur les nuées du ciel, avec une grande puissance et une grande gloire.
\VS{31}Il enverra ses anges avec un grand son de trompette et ils rassembleront ses élus, des quatre vents, d'une extrémité des cieux à l'autre.
\TextTitle{Parabole du figuier\FTNTT{Mc. 13:28-31 ; Lu. 21:29-33}}
\VS{32}Mais apprenez la leçon tirée de la parabole du figuier. Dès que ses jeunes branches deviennent tendres et que ses feuilles poussent, vous savez que l'été est proche.
\VS{33}De même, quand vous verrez toutes ces choses, sachez que le Fils de l'homme est proche, à la porte.
\VS{34}Je vous le dis en vérité, cette génération ne passera point, jusqu'à ce que tout cela n'arrive.
\VS{35}Le ciel et la terre passeront, mais mes paroles ne passeront point.
\TextTitle{Exhortation à la vigilance\FTNTT{Mc. 13:32-37 ; Lu. 21:34-38}}
\VS{36}Pour ce qui est du jour et de l'heure, personne ne le sait, ni les anges des cieux, mais mon Père seul.
\VS{37}Mais comme il en était aux jours de Noé, il en sera de même de l'avènement du fils de l'homme.
\VS{38}Car, comme dans les jours avant le déluge, les hommes mangeaient et buvaient, se mariaient, et donnaient en mariage, jusqu'au jour où Noé entra dans l'arche ;
\VS{39}et ils ne connurent point que le déluge viendrait, jusqu'à ce qu'il vint et les emporta tous ; il en sera de même de l'avènement du Fils de l'homme.
\VS{40}Alors, de deux hommes qui seront dans un champ ; l'un sera pris, et l'autre laissé ;
\VS{41}de deux femmes qui moudront au moulin, l'une sera prise et l'autre laissée.
\VS{42}Veillez donc, car vous ne savez point à quelle heure votre Seigneur doit venir.
\VS{43}Mais sachez ceci, que si un père de famille savait à quelle veille de la nuit le voleur doit venir, il veillerait et ne laisserait pas percer sa maison.
\VS{44}C'est pourquoi, vous aussi tenez-vous prêts ; car le Fils de l'homme viendra à l'heure où vous n'y penserez pas.
\VS{45}Quel est donc le serviteur fidèle et prudent, que son maître a établi sur tous ses serviteurs, pour leur donner la nourriture au temps opportun ?
\VS{46}Heureux est ce serviteur que son maître en arrivant trouvera agir de cette manière.
\VS{47}Je vous le dis en vérité, il l'établira sur tous ses biens.
\VS{48}Mais si c'est un méchant serviteur, qui dit en lui-même : Mon maître tarde à venir ;
\VS{49}et s'il se met à battre ses compagnons de service, s'il mange et boit avec les ivrognes,
\VS{50}le maître de ce serviteur viendra le jour où il ne s'y attend pas et à l'heure qu'il ne connaît pas.
\VS{51}Et il le séparera, et le mettra au rang des hypocrites ; là il y aura des pleurs et des grincements de dents.
\Chap{25}
\TextTitle{Parabole des dix vierges}
\VerseOne{}Alors le Royaume des cieux sera semblable à dix vierges qui, ayant pris leurs lampes, allèrent à la rencontre de l'époux.
\VS{2}Or il y en avait cinq sages et cinq folles.
\VS{3}Les folles, en prenant leurs lampes, ne prirent pas d'huile avec elles ;
\VS{4}mais les sages prirent de l'huile dans leurs vases avec leurs lampes.
\VS{5}Et comme l'époux tardait à venir, elles s'assoupirent et s'endormirent toutes.
\VS{6}Or à minuit il se fit un cri disant : Voici, l'époux vient, allez à sa rencontre !
\VS{7}Alors toutes ces vierges se réveillèrent\FTNT{Réveiller : Du grec « egeiro ». Ce terme signifie également ressusciter. Les saints qui attendent le retour du Seigneur connaîtront un réveil après un temps de sommeil spirituel (Ro. 13:11).} et préparèrent leurs lampes.
\VS{8}Et les folles dirent aux sages : Donnez-nous de votre huile, car nos lampes s'éteignent.
\VS{9}Mais les sages répondirent en disant : Nous ne pouvons pas vous en donner, de peur que nous n'en ayons pas assez pour nous et pour vous ; mais allez plutôt chez ceux qui en vendent et achetez-en pour vous-mêmes.
\VS{10}Or pendant qu'elles allaient en acheter, l'époux arriva. Celles qui étaient prêtes entrèrent avec lui dans la salle des noces, puis la porte fut fermée.
\VS{11}Après cela, les autres vierges vinrent aussi, et dirent : Seigneur ! Seigneur ! Ouvre-nous !
\VS{12}Mais il leur répondit, et dit : Je vous le dis en vérité, je ne vous connais point.
\VS{13}Veillez donc ; car vous ne savez ni le jour ni l'heure en laquelle le Fils de l'homme viendra.
\TextTitle{Parabole des talents}
\VS{14}Car il en sera comme d'un homme qui, partant pour un voyage, appela ses serviteurs et leur remit ses biens.
\VS{15}Il donna à l'un cinq talents, à l'autre deux, et au troisième un ; à chacun selon sa capacité ; et aussitôt après il partit.
\VS{16}Celui qui avait reçu les cinq talents, s'en alla, et les fit valoir, et gagna cinq autres talents.
\VS{17}De même, celui qui avait reçu les deux talents, en gagna aussi deux autres.
\VS{18}Mais celui qui n'en avait reçu qu'un, alla et creusa dans la terre, et y cacha l'argent de son maître.
\VS{19}Longtemps après, le maître de ces serviteurs revint et leur fit rendre compte.
\VS{20}Alors celui qui avait reçu les cinq talents, vint et présenta cinq autres talents, en disant : Seigneur, tu m'as confié cinq talents, voici, j'en ai gagné cinq autres par-dessus.
\VS{21}Et son Seigneur lui dit : Cela est bien, bon et fidèle serviteur ; tu as été fidèle en peu de choses, je t'établirai sur beaucoup ; viens participer à la joie de ton Seigneur.
\VS{22}Ensuite, celui qui avait reçu les deux talents, vint et dit : Seigneur, tu m'as confié deux talents ; voici, j'en ai gagné deux autres par-dessus.
\VS{23}Et son Seigneur lui dit : Cela est bien, bon et fidèle serviteur, tu as été fidèle en peu de choses, je t'établirai sur beaucoup ; viens prendre part à la joie de ton Seigneur.
\VS{24}Mais celui qui n'avait reçu qu'un talent, vint et dit : Seigneur, je savais que tu es un homme dur, qui moissonnes où tu n'as point semé, et qui amasses où tu n'as point vanné,
\VS{25}c'est pourquoi craignant de perdre ton talent, je suis allé le cacher dans la terre. Voici, tu as ici ce qui t'appartient.
\VS{26}Et son Seigneur répondant, lui dit : Méchant et lâche serviteur, tu savais que je moissonnais où je n'ai point semé, et que j'amassais où je n'ai point vanné,
\VS{27}il te fallait donc remettre mon argent aux banquiers et à mon retour, je l'aurais retiré avec l'intérêt.
\VS{28}Ôtez-lui donc le talent et donnez-le à celui qui a les dix talents.
\VS{29}Car à celui qui a, il sera donné et il en aura encore plus, mais à celui qui n'a rien, cela même qu'il a, lui sera ôté.
\VS{30}Jetez donc le serviteur inutile dans les ténèbres de dehors ; où il y aura des pleurs et des grincements de dents.
\TextTitle{Séparation et jugement des brebis et des boucs\FTNTT{1 Co. 6:2}}
\VS{31}Or quand le Fils de l'homme viendra environné de sa gloire et accompagné de tous les saints anges, alors il s'assiéra sur le trône de sa gloire.
\VS{32}Et toutes les nations seront assemblées devant lui ; et il séparera les uns d'avec les autres, comme le berger sépare les brebis d'avec les boucs.
\VS{33}Et il mettra les brebis à sa droite et les boucs à sa gauche.
\VS{34}Alors le Roi dira à ceux qui seront à sa droite : Venez, vous qui êtes bénis de mon Père, possédez en héritage le Royaume qui vous a été préparé dès la fondation du monde.
\VS{35}Car j'ai eu faim et vous m'avez donné à manger ; j'ai eu soif et vous m'avez donné à boire ; j'étais étranger et vous m'avez recueilli ;
\VS{36}j'étais nu et vous m'avez vêtu ; j'étais malade et vous m'avez visité ; j'étais en prison et vous êtes venus vers moi.
\VS{37}Alors les justes lui répondront : Seigneur, quand t'avons-nous vu avoir faim et t'avons-nous donné à manger ; ou avoir soif et t'avons-nous donné à boire ?
\VS{38}Quand t'avons-nous vu étranger et t'avons-nous recueilli ; ou nu, et t'avons-nous vêtu ?
\VS{39}Ou quand t'avons-nous vu malade, ou en prison, et sommes-nous allés vers toi ?
\VS{40}Et le Roi répondant, leur dira : Je vous le dis en vérité, toutes les fois que vous avez fait ces choses à l'un de ces plus petits de mes frères, c'est à moi que vous les avez faites.
\VS{41}Alors il dira aussi à ceux qui seront à sa gauche : Maudits, retirez-vous de moi et allez dans le feu éternel, qui a été préparé pour le diable et pour ses anges.
\VS{42}Car j'ai eu faim et vous ne m'avez point donné à manger ; j'ai eu soif et vous ne m'avez point donné à boire ;
\VS{43}j'étais étranger et vous ne m'avez point recueilli ; j'ai été nu, et vous ne m'avez point vêtu ; j'ai été malade et en prison, et vous ne m'avez point visité.
\VS{44}Alors ils répondront aussi en disant : Seigneur, quand t'avons-nous vu avoir faim, ou avoir soif, ou être étranger, ou nu, ou malade, ou en prison, et ne t'avons-nous point secouru ?
\VS{45}Alors il leur répondra, en disant : Je vous le dis en vérité, toutes les fois que vous n'avez pas fait ces choses à l'un de ces plus petits, c'est à moi que vous ne les avez pas faites.
\VS{46}Et ceux-ci iront au châtiment éternel, mais les justes à la vie éternelle.
\Chap{26}
\TextTitle{Le complot\FTNTT{Mc. 14:1-2 ; Lu. 22:1-2}}
\VerseOne{}Et il arriva que quand Jésus eut achevé tous ces discours, il dit à ses disciples :
\VS{2}Vous savez que la fête de Pâque a lieu dans deux jours ; et le Fils de l'homme sera livré pour être crucifié.
\VS{3}Alors les principaux sacrificateurs, les scribes et les anciens du peuple, se réunirent dans la cour du souverain sacrificateur, appelé Caïphe ;
\VS{4}et tinrent conseil ensemble pour se saisir de Jésus par finesse, afin de le faire mourir.
\VS{5}Mais ils dirent : Que ce ne soit pas pendant la fête, de peur qu'il ne se fasse quelque tumulte parmi le peuple.
\TextTitle{Geste prophétique de Marie de Béthanie\FTNTT{Mc. 14:3-9 ; Jn. 12:1-8}}
\VS{6}Comme Jésus était à Béthanie, dans la maison de Simon le lépreux,
\VS{7}une femme s'approcha de lui tenant un vase d'albâtre, plein d'un parfum de grand prix, et pendant qu'il était à table, elle répandit le parfum sur sa tête.
\VS{8}Mais ses disciples voyant cela, en furent indignés et dirent : À quoi sert cette perte ?
\VS{9}Car ce parfum pouvait être vendu bien cher et être donné aux pauvres.
\VS{10}Mais Jésus connaissant cela, leur dit : Pourquoi faites-vous de la peine à cette femme ? Car elle a fait une bonne action à mon égard ;
\VS{11}car vous aurez toujours des pauvres avec vous ; mais vous ne m'aurez pas toujours.
\VS{12}En répandant ce parfum sur mon corps, elle l'a fait pour ma sépulture.
\VS{13}Je vous le dis en vérité, partout où cet Evangile sera prêché, dans le monde entier, on racontera aussi en mémoire de cette femme ce qu'elle a fait.
\TextTitle{La trahison de Judas\FTNTT{Mc. 14:10-11 ; Lu. 22:3-6}}
\VS{14}Alors l'un des douze, appelé Judas Iscariot, alla vers les principaux sacrificateurs,
\VS{15}et leur dit : Que voulez-vous me donner, et je vous le livrerai ? Et ils lui comptèrent trente pièces d'argent\FTNT{Za. 11:12-13.}.
\VS{16}Et dès lors, il cherchait une occasion favorable pour le livrer.
\TextTitle{La dernière Pâque\FTNTT{Mc. 14:12-21 ; Lu. 22:7-20 ; Jn. 13:1-12}}
\VS{17}Or le premier jour des pains sans levain, les disciples s'approchèrent de Jésus pour lui dire : Où veux-tu que nous te préparions le repas de la Pâque ?
\VS{18}Il répondit : Allez à la ville chez un tel et dites-lui : Le Maître dit : Mon temps est proche ; je ferai la Pâque chez toi avec mes disciples.
\VS{19}Les disciples firent comme Jésus leur avait ordonné et préparèrent la Pâque.
\VS{20}Et quand le soir fut venu, il se mit à table avec les douze.
\VS{21}Et comme ils mangeaient, il dit : Je vous le dis en vérité, l'un de vous me trahira.
\VS{22}Ils furent profondément attristés, et chacun d'eux commença à lui dire : Seigneur, est-ce moi ?
\VS{23}Mais il leur répondit : Celui qui a mis avec moi la main dans le plat pour tremper, c'est celui qui me trahira.
\VS{24}Le Fils de l'homme s'en va, selon qu'il est écrit de lui ; mais malheur à cet homme par qui le Fils de l'homme est trahi ! Mieux vaudrait pour cet homme qu'il ne soit pas né.
\VS{25}Judas qui le trahissait, prit la parole et dit : Maître, est-ce moi ? Jésus lui dit : Tu l'as dit.
\TextTitle{Le repas de la Pâque\FTNTT{Mc. 14:22-25 ; Lu. 22:17-20 ; Jn. 13:12-30 ; 1 Co. 11:23-26}}
\VS{26}Pendant qu'ils mangeaient, Jésus prit le pain, et après avoir rendu grâces à Dieu, il le rompit et le donna à ses disciples et leur dit : Prenez, mangez, ceci est mon corps.
\VS{27}Puis ayant pris la coupe, et béni Dieu, il la leur donna, en leur disant : buvez-en tous.
\VS{28}car ceci est mon sang, le sang de la Nouvelle Alliance, qui est répandu pour beaucoup, pour la rémission des péchés.
\VS{29}Or je vous dis : que depuis cette heure je ne boirai point de ce fruit de vigne, jusqu'au jour que je le boirai de nouveau avec vous, dans le Royaume de mon Père.
\TextTitle{Jésus informe Pierre de son triple reniement\FTNTT{Mc. 14:26-31 ; Lu. 22:31-34 ; Jn. 13:36-38}}
\VS{30}Quand ils eurent chanté le cantique\FTNT{Les cantiques : Du grec « humneo », chants d'hymnes pascals. Il s'agit plus précisément des psaumes 113 à 118 et du psaume 136, que les Juifs appellent le « grand Hallel ». Le Hallel consiste en six psaumes (113 à 118). Cet ensemble de textes est généralement entonné à haute voix par toute la communauté de prière lors de l'office religieux du matin, à l'issue de la « Amidah » (prière récitée debout), à l'occasion de la Pâque (le premier soir), de la Pentecôte et des Tabernacles, ainsi que pour Hanoucca et Rosh Hodesh. Voir Mc. 14:26.}, ils se rendirent à la Montagne des Oliviers.
\VS{31}Alors Jésus leur dit : Vous serez tous cette nuit scandalisés à cause de moi ; car il est écrit : Je frapperai le Berger, et les brebis du troupeau seront dispersées\FTNT{Za. 13:7.}.
\VS{32}Mais, après que je serai ressuscité, je vous précéderai en Galilée.
\VS{33}Pierre, prenant la parole, lui dit : Quand même tous seraient scandalisés à cause de toi, je ne le serai jamais.
\VS{34}Jésus lui dit : En vérité je te dis, qu'en cette même nuit, avant que le coq ait chanté, tu me renieras trois fois.
\VS{35}Pierre lui répondit : Même s'il me fallait mourir avec toi, je ne te renierai pas. Et tous les disciples dirent la même chose.
\TextTitle{Jésus dans le jardin de Gethsémané\FTNTT{Mc. 14:32-42 ; Lu. 22:39-46 ; Jn. 18:1}}
\VS{36}Alors Jésus alla avec eux dans un lieu appelé Gethsémané et il dit à ses disciples : Asseyez-vous ici, jusqu'à ce que j'aie prié dans le lieu où je vais.
\VS{37}Il prit avec lui Pierre et les deux fils de Zébédée, et il commença à être attristé et fort angoissé.
\VS{38}Alors il leur dit : Mon âme est de toutes parts saisie de tristesse jusqu'à la mort ; demeurez ici et veillez avec moi.
\TextTitle{Première prière de Jésus\FTNTT{Mc. 14:35-38 ; Lu. 22:41-42}}
\VS{39}Puis, ayant fait quelques pas en avant, il se prosterna le visage contre terre, priant et disant : Mon Père, s'il est possible, fais que cette coupe passe loin de moi ; toutefois non point comme je le veux, mais comme tu le veux.
\TextTitle{Jésus trouve les disciples endormis\FTNTT{Mc. 14:37-40 ; Lu. 22:45-46}}
\VS{40}Puis il vint vers ses disciples, qu'il trouva endormis, et il dit à Pierre : Vous n'avez pas pu veiller une heure avec moi ?
\VS{41}Veillez et priez, afin que vous ne tombiez pas en tentation : car l'esprit est prompt, mais la chair est faible.
\TextTitle{Deuxième prière\FTNTT{Mc. 14:39 ; Lu. 22:44}}
\VS{42}Il s'éloigna encore pour la seconde fois, et il pria, disant : Mon Père, s'il n'est pas possible que cette coupe s'éloigne sans que je la boive, que ta volonté soit faite.
\VS{43}Il revint ensuite et les trouva encore endormis ; car leurs yeux étaient appesantis.
\TextTitle{Troisième prière\FTNTT{Mc. 14:41}}
\VS{44}Et les ayant laissés, il s'en alla encore, et pria pour la troisième fois, disant les mêmes paroles.
\VS{45}Puis il alla vers ses disciples et leur dit : Dormez maintenant et reposez-vous ; voici, l'heure est proche, et le Fils de l'homme va être livré entre les mains des méchants.
\VS{46}Levez-vous, allons. Voici, celui qui me trahit s'approche.
\TextTitle{Jésus trahi et arrêté\FTNTT{Mc. 14:43-50 ; Lu. 22:47-53 ; Jn. 18:2-11}}
\VS{47}Comme il parlait encore, voici, Judas, l'un des douze, vint, et avec lui une grande foule, avec des épées et des bâtons, envoyée par les principaux sacrificateurs et par les anciens du peuple.
\VS{48}Celui qui le trahissait leur avait donné ce signe : Celui à qui je donnerai un baiser, c'est lui, saisissez-le.
\VS{49}Aussitôt, s'approchant de Jésus, il lui dit : Maître, je te salue ; et il le baisa.
\VS{50}Et Jésus lui dit : Mon ami, pour quel sujet es-tu ici ? Alors s'étant approchés, ils mirent les mains sur Jésus et le saisirent.
\VS{51}Et voici, l'un de ceux qui étaient avec Jésus, étendit la main et tira son épée ; il frappa le serviteur du souverain sacrificateur, et lui emporta l'oreille.
\VS{52}Alors Jésus lui dit : Remets ton épée à sa place ; car tous ceux qui prendront l'épée, périront par l'épée.
\VS{53}Crois-tu que je ne puisse pas maintenant prier mon Père, qui me donnerait à l'instant plus de douze légions d'anges ?
\VS{54}Mais comment donc s'accompliraient les Ecritures qui disent qu'il faut que cela arrive ainsi ?
\VS{55}En ce même instant Jésus dit à la foule : Vous êtes venus avec des épées et des bâtons, comme après un brigand, pour me prendre ; j'étais tous les jours assis parmi vous, enseignant dans le temple, et vous ne m'avez pas saisi.
\VS{56}Mais tout ceci est arrivé afin que les Ecritures des prophètes soient accomplies. Alors tous les disciples l'abandonnèrent et s'enfuirent.
\TextTitle{Jésus devant Caïphe et le sanhédrin\FTNTT{Mc. 14:53-65 ; Jn. 18:12-14, 19-24.}}
\VS{57}Ceux qui avaient saisi Jésus l'amenèrent chez Caïphe, le souverain sacrificateur, où les scribes et les anciens étaient assemblés.
\VS{58}Pierre le suivit de loin, jusqu'à la cour du souverain sacrificateur, y entra et s'assit avec les officiers pour voir comment cela finirait.
\VS{59}Les principaux sacrificateurs, les anciens et tout le sanhédrin cherchaient des faux témoignages contre Jésus pour le faire mourir.
\VS{60}Mais ils n'en trouvèrent point, et bien que plusieurs faux témoins se soient présentés, ils n'en trouvèrent point de propres ; mais à la fin, deux faux témoins s'approchèrent
\VS{61}et dirent : Celui-ci a dit : Je puis détruire le temple de Dieu et le rebâtir en trois jours.
\VS{62}Alors le souverain sacrificateur se leva et lui dit : Ne réponds-tu rien ? Qu'est-ce que ces hommes déposent contre toi ?
\VS{63}Jésus garda le silence. Et le souverain sacrificateur prenant la parole, lui dit : Je te somme par le Dieu vivant, de nous dire si tu es le Christ, le Fils de Dieu.
\VS{64}Jésus lui dit : Tu l'as dit. De plus, je vous dis que désormais vous verrez le Fils de l'homme assis à la droite de la puissance de Dieu et venant sur les nuées du ciel.
\VS{65}Alors le souverain sacrificateur déchira ses vêtements, en disant : Il a blasphémé ! Qu'avons-nous encore besoin de témoins ? Voici, vous avez entendu maintenant son blasphème. Que vous en semble ?
\VS{66}Ils répondirent : Il est digne de mort.
\VS{67}Alors ils lui crachèrent au visage, et lui donnèrent des coups de poing et des soufflets, et les autres le frappaient avec leurs bâtons ;
\VS{68}en disant : Christ, prophétise-nous qui est celui qui t'a frappé.
\TextTitle{Le triple reniement de Pierre\FTNTT{Mc. 14:66-72 ; Lu. 22:55-62 ; Jn. 18:15-18, 25-27.}}
\VS{69}Or Pierre était assis dehors dans la cour. Une servante s'approcha de lui et lui dit : Toi aussi, tu étais aussi avec Jésus le Galiléen.
\VS{70}Mais il le nia devant tous, en disant : Je ne sais pas ce que tu dis.
\VS{71}Et comme il était sorti dans le vestibule, une autre servante le vit et elle dit à ceux qui étaient là : Celui-ci aussi était avec Jésus de Nazareth.
\VS{72}Et il le nia encore avec serment, disant : Je ne connais pas cet homme.
\VS{73}Peu après, ceux qui se trouvaient là s'approchèrent et dirent à Pierre : Certainement tu es aussi de ces gens-là, car ton langage te fait connaître.
\VS{74}Alors il commença à faire des imprécations et à jurer, en disant : Je ne connais pas cet homme. Et aussitôt le coq chanta.
\VS{75}Et Pierre se souvint de la parole de Jésus, qui lui avait dit : Avant que le coq chante, tu me renieras trois fois. Et étant sorti dehors, il pleura amèrement.
\Chap{27}
\TextTitle{Jésus devant le gouverneur Pilate ; suicide de Judas\FTNTT{Ac. 1:16-19.}}
\VerseOne{}Puis quand le matin fut venu, tous les principaux sacrificateurs et les anciens du peuple tinrent conseil contre Jésus pour le faire mourir.
\VS{2}Après l'avoir lié, ils l'amenèrent et le livrèrent à Ponce Pilate, qui était le gouverneur.
\VS{3}Alors Judas qui l'avait trahi, voyant qu'il était condamné, se repentit et rapporta les trente pièces d'argent aux principaux sacrificateurs et aux anciens,
\VS{4}en leur disant : J'ai péché en trahissant le sang innocent ; mais ils lui dirent : Que nous importe ? Cela te regarde.
\VS{5}Et après avoir jeté les pièces d'argent dans le temple, il se retira et alla se pendre.
\VS{6}Mais les principaux sacrificateurs prirent les pièces d'argent et dirent : Il n'est pas permis de les mettre dans le trésor ; car c'est le prix du sang.
\VS{7}Et, après en avoir délibéré, ils achetèrent avec cet argent le champ d'un potier pour la sépulture des étrangers.
\VS{8}C'est pourquoi ce champ-là a été appelé jusqu'à aujourd'hui, le champ du sang.
\VS{9}Alors s'accomplit ce qui avait été annoncé par Jérémie le prophète, disant : Ils ont pris les trente pièces d'argent, le prix de celui qui a été estimé, qu'on a estimé de la part des enfants d'Israël ;
\VS{10}et ils les ont données pour acheter le champ d'un potier, selon ce que le Seigneur m'avait ordonné\FTNT{Ce verset se réfère certainement à Za. 11:12-13, avec une allusion à Jé. 18:1-4.}.
\VS{11}Jésus comparut devant le gouverneur. Le gouverneur l'interrogea : Es-tu le Roi des Juifs ? Jésus lui répondit : Tu le dis.
\VS{12}Mais il ne répondit rien aux accusations des principaux sacrificateurs et des anciens.
\VS{13}Alors Pilate lui dit : N'entends-tu pas de combien de choses ils t'accusent ?
\VS{14}Mais il ne lui donna de réponse sur aucune parole, ce qui étonna beaucoup le gouverneur.
\TextTitle{Jésus ou Barabbas ?\FTNTT{Mc. 15:6-15 ; Lu. 23:17-25 ; Jn. 18:39-40}}
\VS{15}Or le gouverneur avait coutume de relâcher un prisonnier à chaque fête, celui que demandait la foule.
\VS{16}Et il y avait alors un prisonnier fameux, nommé Barabbas.
\VS{17}Comme ils étaient assemblés, Pilate leur dit : Lequel voulez-vous que je vous relâche ? Barabbas ou Jésus qu'on appelle Christ ?
\VS{18}Car il savait bien qu'ils l'avaient livré par envie.
\VS{19}Et pendant qu'il siégeait au tribunal, sa femme envoya lui dire : Ne te mêle point de l'affaire de ce juste, car j'ai beaucoup souffert aujourd'hui en songe à cause de lui.
\VS{20}Les principaux sacrificateurs et les anciens persuadèrent la multitude du peuple de demander Barabbas et de faire périr Jésus.
\VS{21}Et le gouverneur prenant la parole leur dit : Lequel des deux voulez-vous que je vous relâche ? Ils dirent : Barabbas.
\VS{22}Pilate leur dit : Que ferai-je donc de Jésus qu'on appelle Christ ? Ils lui dirent tous : Qu'il soit crucifié !
\VS{23}Et le gouverneur leur dit : Mais quel mal a-t-il fait ? Et ils crièrent encore plus fort, en disant : Qu'il soit crucifié !
\VS{24}Alors Pilate voyant qu'il ne gagnait rien, mais que le tumulte s'augmentait, prit de l'eau et lava ses mains devant le peuple, en disant : Je suis innocent du sang de ce juste. Cela vous regarde.
\VS{25}Et tout le peuple répondit : Que son sang retombe sur nous et sur nos enfants !
\VS{26}Alors il leur relâcha Barabbas ; et après avoir fait fouetter Jésus, il le leur livra pour être crucifié.
\TextTitle{Le Roi couronné d'épines\FTNTT{Mc. 15:16-23 ; Lu. 23:26-32 ; Jn. 19:16-17}}
\VS{27}Les soldats du gouverneur amenèrent Jésus dans le prétoire et assemblèrent devant lui toute la cohorte.
\VS{28}Et après l'avoir dépouillé, ils le revêtirent d'un manteau d'écarlate.
\VS{29}Puis, ayant fait une couronne d'épines entrelacées, ils la mirent sur sa tête et ils lui mirent un roseau dans sa main droite ; puis s'agenouillant devant lui, ils se moquaient de lui, en disant : Nous te saluons, Roi des Juifs !
\VS{30}Et ils crachaient contre lui, prenaient le roseau et frappaient sur sa tête.
\VS{31}Après s'être ainsi moqués de lui, ils lui ôtèrent le manteau, et lui remirent ses vêtements, et l'amenèrent pour le crucifier.
\VS{32}Comme ils sortaient, ils rencontrèrent un homme de Cyrène, appelé Simon et ils le forcèrent à porter la croix de Jésus.
\TextTitle{La crucifixion de Jésus\FTNTT{Mc. 15:24-32 ; Lu. 23:33-43 ; Jn. 19:17-24}}
\VS{33}Et étant arrivés au lieu appelé Golgotha, c'est-à-dire le lieu du crâne,
\VS{34}ils lui donnèrent à boire du vinaigre mêlé avec du fiel\FTNT{Le vinaigre mêlé au fiel (Ps. 69:22) : Ce breuvage, appelé « posca », était un vin amer composé qui se transformait en vinaigre à cause des mauvaises conditions de conservation. Allongé avec de l'eau et parfois adoucie avec de l'œuf, cette boisson bon marché et très rafraîchissante était consommée principalement par les légionnaires et les esclaves. Connue pour ses vertus antiseptiques, les soldats de l'Antiquité avaient coutume d'y ajouter des drogues comme la myrrhe et le fiel (opium) pour atténuer les souffrances. En refusant de le boire, le Seigneur Jésus-Christ a réellement pris sur lui la plénitude du châtiment que nous méritons à cause de nos péchés.} ; mais quand il l'eut goûté, il ne voulut pas boire.
\VS{35}Et après l'avoir crucifié, ils partagèrent ses vêtements, en tirant au sort, afin que s'accomplît ce qui avait été annoncé par le prophète : Ils se sont partagés mes vêtements, et ont jeté ma tunique au sort\FTNT{Ps. 22:19.}.
\VS{36}Puis s'étant assis, ils le gardaient là.
\VS{37}Ils mirent aussi au-dessus de sa tête un écriteau, où la cause de sa condamnation était marquée en ces mots : CELUI-CI EST JESUS, LE ROI DES JUIFS.
\VS{38}Avec lui furent crucifiés deux brigands, l'un à sa droite et l'autre à sa gauche.
\VS{39}Et Ceux qui passaient par là, l'injuriaient et secouaient la tête
\VS{40}en disant : Toi qui détruis le temple et qui le rebâtis en trois jours, sauve-toi toi-même ! Si tu es le Fils de Dieu, descends de la croix !
\VS{41}Pareillement aussi, les principaux sacrificateurs avec les scribes et les anciens, se moquant, disaient :
\VS{42}Il a sauvé les autres et il ne peut pas se sauver lui-même ! S'il est le Roi d'Israël, qu'il descende maintenant de la croix et nous croirons en lui.
\VS{43}Il se confie en Dieu ; mais si Dieu l'aime, qu'il le délivre maintenant, car il a dit : Je suis le Fils de Dieu.
\VS{44}Les brigands aussi qui étaient crucifiés avec lui, lui reprochaient la même chose.
\TextTitle{Jésus accomplit la loi par sa mort\FTNTT{Mc. 15:33-41 ; Lu. 23:44-49 ; Jn. 19:30-37 ; Hé. 9:3-8 ; 10:19-20}}
\VS{45}Depuis la sixième heure jusqu'à la neuvième, il y eut des ténèbres sur toute la terre.
\VS{46}Et vers la neuvième heure, Jésus s'écria d'une voix forte : Eli, Eli, lama sabachthani ? C'est-à-dire : Mon Dieu ! Mon Dieu ! Pourquoi m'as-tu abandonné ?
\VS{47}Quelques-uns de ceux qui étaient là présents, ayant entendu cela, disaient : Il appelle Elie.
\VS{48}Et aussitôt l'un d'entre eux courut prendre une éponge, qu'il remplit de vinaigre, et l'ayant fixée au bout d'un roseau, lui donna à boire.
\VS{49}Mais les autres disaient : Laisse, voyons si Elie viendra le sauver.
\VS{50}Alors Jésus, poussa de nouveau un grand cri, et rendit l'esprit.
\VS{51}Et voici, le voile du temple se déchira en deux, depuis le haut jusqu'en bas\FTNT{C'est ici que s'achève la Première Alliance. Cette dernière était relative à la loi de Moïse, c'est-à-dire aux ordonnances liées au culte, qui reposait sur le sacerdoce lévitique et les sacrifices d'animaux, et au sanctuaire terrestre, à savoir le temple de Jérusalem (Hé. 9:1). Le Seigneur ayant offert une fois pour toutes le sacrifice parfait, les exigences de la justice divine ont été pleinement satisfaites (Hé. 9:11-12 ; 25-26). Désormais, la Première Alliance n'a plus de raison d'être et peut donc disparaître (Hé. 8:13). Non seulement la déchirure du voile séparant le lieu saint du Saint des saints atteste la fin de la Première Alliance, mais invite aussi tout homme à s'approcher de Dieu en esprit, sans intermédiaires (Lévites, sacrificateurs, pasteurs, prophètes…) ni nécessité de se rendre dans un temple (Jn. 4:23). La Nouvelle Alliance est aussi un testament puisque Jésus-Christ, notre légataire, est passé par la mort (Hé. 9:16-18). Voir aussi commentaire en Ex. 19:5.} ; et la terre trembla, et les pierres se fendirent.
\TextTitle{Le voile déchiré : Fin de la loi mosaïque ou de la Première Alliance}
\VS{52}Et les sépulcres s'ouvrirent et plusieurs corps des saints qui étaient morts ressuscitèrent.
\VS{53}Et étant sortis des sépulcres après la résurrection de Jésus, ils entrèrent dans la ville sainte et se montrèrent à plusieurs.
\VS{54}Le centenier et ceux qui étaient avec lui pour garder Jésus, ayant vu le tremblement de terre et tout ce qui venait d'arriver, furent saisis d'une grande frayeur et dirent : Certainement cet homme était le Fils de Dieu.
\VS{55}Il y avait là aussi plusieurs femmes qui regardaient de loin, et qui avaient suivi Jésus depuis la Galilée, pour le servir.
\VS{56}Entre lesquelles étaient Marie de Magdala, Marie mère de Jacques et de Joseph, et la mère des fils de Zébédée.
\TextTitle{Jésus enseveli\FTNTT{Mc. 15:42-47 ; Lu. 23:50-56 ; Jn. 19:38-42}}
\VS{57}Le soir étant venu, un homme riche d'Arimathée, appelé Joseph, qui était aussi disciple de Jésus,
\VS{58}se rendit vers Pilate et demanda le corps de Jésus. En même temps Pilate ordonna que le corps soit rendu.
\VS{59}Joseph prit le corps et l'enveloppa d'un linceul pur ;
\VS{60}et le mit dans un sépulcre neuf, qu'il s'était fait tailler dans le roc. Puis il roula une grande pierre à l'entrée du sépulcre et il s'en alla.
\VS{61}Marie de Magdala et l'autre Marie étaient là, assises vis-à-vis du sépulcre.
\TextTitle{Le sépulcre scellé et gardé}
\VS{62}Le lendemain, qui était le jour de la préparation du sabbat, les principaux sacrificateurs et les pharisiens allèrent ensemble auprès de Pilate,
\VS{63}et lui dirent : Seigneur ! Nous nous souvenons que ce séducteur disait, quand il était encore en vie : Après trois jours je ressusciterai.
\VS{64}Ordonne donc que le sépulcre soit gardé sûrement jusqu'au troisième jour ; de peur que ses disciples ne viennent de nuit, et ne dérobent son corps, et qu'ils ne disent au peuple : Il est ressuscité des morts. Cette dernière imposture serait pire que la première.
\VS{65}Pilate leur dit : Vous avez une garde ; allez et faites-le garder comme vous l'entendez.
\VS{66}Ils s'en allèrent donc, et s'assurèrent du sépulcre, au moyen d'une garde, après avoir scellé la pierre.
\Chap{28}
\TextTitle{Résurrection et apparition de Jésus-Christ\FTNTT{Mc. 16:1-14 ; Lu. 24:1-49 ; Jn. 20:1-23}}
\VerseOne{}Après le sabbat, à l'aube du premier jour de la semaine, Marie de Magdala et l'autre Marie allèrent voir le sépulcre.
\VS{2}Et voici, il eut un grand tremblement de terre ; car un ange du Seigneur descendit du ciel, vint rouler la pierre à côté de l'entrée du sépulcre et s'assit dessus.
\VS{3}Son visage était comme un éclair, et son vêtement blanc comme de la neige.
\VS{4}Les gardes furent tellement saisis de frayeur, qu'ils devinrent comme morts.
\VS{5}Mais l'ange prit la parole et dit aux femmes : Pour vous, ne craignez pas ; car je sais que vous cherchez Jésus, qui a été crucifié.
\VS{6}Il n'est point ici car il est ressuscité comme il l'avait dit. Venez et voyez le lieu où le Seigneur était couché,
\VS{7}et allez-vous-en promptement, et dites à ses disciples qu'il est ressuscité des morts. Et voici, il vous précède en Galilée ; c'est là que vous le verrez. Voici, je vous l'ai dit.
\VS{8}Alors elles sortirent promptement du sépulcre avec crainte et grande joie ; et coururent l'annoncer à ses disciples.
\VS{9}Mais comme elles allaient pour l'annoncer à ses disciples, voici, Jésus se présenta devant elles et leur dit : Je vous salue. Et elles s'approchèrent, embrassèrent ses pieds et l'adorèrent.
\VS{10}Alors Jésus leur dit : Ne craignez point. Allez et dites à mes frères d'aller en Galilée, c'est là qu'ils me verront.
\TextTitle{Les soldats soudoyés par les sacrificateurs}
\VS{11} Or quand elles furent parties, voici, quelques-uns de la garde vinrent dans la ville et ils rapportèrent aux principaux sacrificateurs toutes les choses qui étaient arrivées.
\VS{12}Sur quoi les sacrificateurs s'assemblèrent avec les anciens, et après avoir tenu conseil, donnèrent une forte somme d'argent aux soldats,
\VS{13}en leur disant : Dites : Ses disciples sont venus de nuit le dérober, pendant que nous dormions.
\VS{14}Et si le gouverneur l'apprend, nous l'apaiserons et nous vous tirerons de peine.
\VS{15}Les soldats prirent l'argent et suivirent les instructions qui leur furent données. Et ce bruit s'est répandu parmi les juifs, jusqu'à aujourd'hui.
\TextTitle{Mission des apôtres\FTNTT{Mc. 16:15-18 ; Lu. 24:46-48 ; Jn. 17:18 ; 20:21 ; Ac. 1:8 ; 1 Co. 15:6}}
\VS{16}Mais les onze disciples allèrent en Galilée, sur la montagne, où Jésus leur avait ordonné de se rendre.
\VS{17}Quand ils le virent, ils l'adorèrent, mais quelques-uns doutèrent.
\VS{18}Jésus s'étant approché, leur parla, en disant : Tout puissance m'a été donnée dans le ciel et sur la terre.
\VS{19}Allez donc et enseignez toutes les nations, les baptisant au Nom du Père, du Fils et du Saint-Esprit ;
\VS{20}les enseignant à garder toutes les choses que je vous ai commandées ; et voici, je suis avec vous toujours jusqu'à la fin du monde. Amen.
\PPE{}
\end{multicols}

%\clearpage\ShortTitle{Marc}\BookTitle{Marc}\BFont
\noindent\hrulefill
\textit{
\bigskip
{\centering{}
\\Signifie : Qui brille, luisant
\\Thème : Jésus le serviteur
\\Auteur : Marc
\\Date de rédaction : Env. 68 apr. J.-C.\\}
}
%\bigskip
\textit{
\\Originaire de Jérusalem, Marc, aussi appelé Jean, fut l’auteur de l’évangile du même nom. Cousin de Barnabas et collaborateur de Paul, ce dernier l’éconduit lors d’un voyage car Marc l’avait abandonné lors d’une précédente mission. Ce fut d’ailleurs la cause de la séparation entre Barnabas et Paul.  Par la suite, il renoua le contact avec Paul et devint un de ses fidèles compagnons de ministère. Lié à l’apôtre Pierre tel un fils, ce fut probablement sous son autorité qu’il écrivit. En effet, l’évangile de Marc expose le témoignage de Pierre sur Christ.
\bigskip
\\Adressé aux gentils, cet évangile contient peu de références à l’ancienne alliance ; on y découvre Jésus l’inlassable serviteur de Dieu et des hommes. Marc y exposa la richesse de ses bonnes œuvres, son incomparable dévouement et révéla les sentiments intimes du maître. Même si Marc présenta principalement Jésus en tant que serviteur, son récit des miracles met en exergue toute la puissance du Christ.\bigskip
}
\par\nobreak\noindent\hrulefill
\begin{multicols}{2}
\TextTitle{[Ministère de Jean-Baptiste]
\\(Mt. 3:1-12 ; Lu. 3:1-20 ; Jn. 1:6-8,15-37)}
\Chap{1}
\VerseOne{}Commencement de l'Evangile de Jésus-Christ, Fils de Dieu ;
\VS{2}selon qu'il est écrit dans les prophètes : Voici, j'envoie mon messager devant ta face, lequel préparera ta voie devant toi.
\VS{3}C’est la voix de celui qui crie dans le désert : Préparez le chemin du Seigneur, aplanissez ses sentiers{\FTNT{Es. 40:3 ; Mal. 3:1.}}.
\VS{4}Jean baptisait dans le désert, et prêchait le baptême de repentance, pour obtenir la rémission des péchés.
\VS{5}Et tout le pays de Judée, et les habitants de Jérusalem allaient vers lui, et confessant leurs péchés, ils se faisaient tous baptiser par lui dans le fleuve du Jourdain.
\VS{6}Jean était vêtu de poils de chameau, il avait une ceinture de cuir autour de ses reins, et mangeait des sauterelles et du miel sauvage.
\VS{7}Et il prêchait, en disant : Il vient après moi, celui qui est plus puissant que moi, et je ne suis pas digne de délier en me baissant la courroie de ses souliers.
\VS{8}Moi, je vous ai baptisés d'eau ; mais lui, il vous baptisera du Saint-Esprit.
\TextTitle{[Baptême de Jésus-Christ]
\\(Mt. 3:13-17 ; Lu. 3:21-22 ; Jn. 1:31-34)}
\VS{9}En ce temps-là, Jésus vint de Nazareth, ville de Galilée, et il fut baptisé par Jean dans le Jourdain.
\VS{10}Au moment où il sortait de l'eau, Jean vit les cieux s’ouvrir, et le Saint-Esprit descendre sur lui comme une colombe.
\VS{11}Et une voix fit entendre des cieux ces paroles : Tu es mon Fils bien-aimé, en qui j'ai mis toute mon affection.
\TextTitle{[La tentation]
\\(Mt. 4:1-11 ; Lu. 4:1-13)}
\VS{12}Aussitôt l'Esprit le poussa à se rendre dans un désert,
\VS{13}où il passa quarante jours, tenté par Satan. Il était avec les bêtes sauvages, et les anges le servaient.
\TextTitle{[Jésus en Galilée]
\\(Mt. 4:12-17 ; Lu. 4:14-15)}
\VS{14}Après que Jean eut été mis en prison, Jésus alla dans la Galilée, prêchant l'Evangile du Royaume de Dieu.
\VS{15}Il disait : Le temps est accompli, et le Royaume de Dieu est proche. Repentez-vous, et croyez à 1'Evangile.
\TextTitle{[Appel de Simon (Pierre), André, Jacques et Jean]
\\(Lu. 5:1-11 ; Jn. 1:35-51)}
\VS{16}Comme il marchait près de la mer de Galilée, il vit Simon et André son frère, qui jetaient leurs filets dans la mer, car ils étaient pêcheurs.
\VS{17}Jésus leur dit : Suivez-moi, et je vous ferai pêcheurs d'hommes.
\VS{18}Aussitôt ils laissèrent leurs filets et ils le suivirent.
\VS{19}Etant allé un peu plus loin, il vit Jacques fils de Zébédée, et Jean son frère, qui raccommodaient leurs filets dans la barque.
\VS{20}Aussitôt, il les appela ; et laissant leur père Zébédée dans la barque avec les ouvriers, ils le suivirent.
\TextTitle{[Jésus chasse un démon dans la synagogue]
\\(Lu. 4:31-37)}
\VS{21}Ils entrèrent dans Capernaüm. Et le jour du sabbat, Jésus entra d’abord dans la synagogue, et il enseigna.
\VS{22}Ils étaient étonnés de sa doctrine ; car il les enseignait comme ayant autorité, et non pas comme les scribes.
\VS{23}Il se trouvait dans leur synagogue un homme qui avait un esprit impur, et qui s'écria,
\VS{24}en disant : Ha ! Qu’y a-t-il entre toi et nous, Jésus de Nazareth ? Es-tu venu pour nous perdre ? Je sais qui tu es : Tu es le Saint de Dieu.
\VS{25}Mais Jésus le menaça, disant : Tais-toi, et sors de cet homme.
\VS{26}Alors l'esprit impur sortit de cet homme, en l’agitant avec violence, et en poussant un grand cri.
\VS{27}Tous furent étonnés, de sorte qu'ils se demandaient les uns aux autres, et disaient : Qu'est-ce que ceci ? Quelle est cette nouvelle doctrine ? Il commande avec autorité même aux esprits impurs, et ils lui obéissent.
\VS{28}Et sa renommée se répandit aussitôt dans tout le pays des environs de la Galilée.
\TextTitle{[Jésus guérit la belle-mère de Pierre]
\\(Mt. 8:14-15 ; Lu. 4:38-39)}
\VS{29}En sortant de la synagogue, ils se rendirent avec Jacques et Jean à la maison de Simon et d'André.
\VS{30}La belle-mère de Simon était couchée, ayant la fièvre ; et aussitôt on parla d’elle à Jésus.
\VS{31}S’étant approché, il la fit lever en la prenant par la main ; et à l'instant la fièvre la quitta ; et elle les servit.
\TextTitle{[Jésus guérit les malades et chasse des démons ; prédications en Galilée]
\\(Mt. 8:16-17 ; Lu. 4:40-44)}
\VS{32}Le soir étant venu, comme le soleil se couchait, on lui amena tous les malades, et les démoniaques.
\VS{33}Et toute la ville était assemblée devant sa porte.
\VS{34}Il guérit beaucoup de malades qui avaient différentes maladies et chassa beaucoup de démons, et il ne permettait pas aux démons de parler, parce qu’ils le connaissaient.
\VS{35}Vers le matin, pendant qu’il faisait encore très sombre, il se leva, et sortit pour aller dans un lieu désert, où il pria.
\VS{36}Simon et ceux qui étaient avec lui se mirent à sa recherche,
\VS{37}et quand ils l’eurent trouvé, ils lui dirent : Tous te cherchent.
\VS{38}Et il leur dit : Allons aux bourgades voisines, afin que j'y prêche aussi ; car c’est pour cela que je suis venu.
\VS{39}Il prêchait donc dans leurs synagogues, par toute la Galilée, et chassait les démons.
\TextTitle{[Jésus guérit un lépreux]
\\(Mt. 8:2-4 ; Lu. 5:12-14)}
\VS{40}Un lépreux vint à lui, le priant et se mettant à genoux devant lui, et lui dit : Si tu veux, tu peux me rendre pur.
\VS{41}Jésus, ému de compassion, étendit sa main et le toucha, en lui disant : Je le veux, sois pur.
\VS{42}La lèpre quitta aussitôt cet homme, et il fut purifié.
\VS{43}Jésus le renvoya sur-le-champ, avec de sévères recommandations,
\VS{44}et lui dit : Garde-toi de ne rien dire à personne ; mais va te montrer au sacrificateur, et présente pour ta purification les choses que Moïse a commandées, pour leur servir de témoignage{\FTNT{Loi sur la purification de la lèpre~: Lé. 14:1-32. Avant sa mort et sa résurrection, Jésus-Christ observait la loi de Moïse (Mt. 23:1-2).}}.
\VS{45}Mais cet homme, s’en étant allé, commença à publier ouvertement la chose et à divulguer ce qui s'était passé ; de sorte que Jésus ne pouvait plus entrer publiquement dans la ville, mais il se tenait dehors, dans des lieux déserts, et l’on venait à lui de toutes parts.
\TextTitle{[Jésus guérit un paralytique]
\\(Mt. 9:2-8 ; Lu. 5:18-26)}
\Chap{2}
\VerseOne{}Quelques jours après, Jésus revint à Capernaüm. On apprit qu'il était à la maison,
\VS{2}et aussitôt il s’assembla un si grand nombre de personnes, que l'espace même devant la porte ne pouvait plus les contenir. Il leur annonçait la parole.
\VS{3}Et quelques-uns vinrent à lui, amenant un paralytique qui était porté par quatre personnes.
\VS{4}Comme ils ne pouvaient pas s’approcher de lui à cause de la foule, ils découvrirent le toit du lieu où il était, et l'ayant percé, ils descendirent le lit dans lequel le paralytique était couché.
\VS{5}Jésus, voyant leur foi, dit au paralytique : Mon enfant, tes péchés te sont pardonnés.
\VS{6}Et quelques scribes qui étaient assis là, raisonnaient ainsi en eux-mêmes :
\VS{7}Comment cet homme parle-t-il ainsi ? Il blasphème. Qui peut pardonner les péchés, si ce n’est Dieu seul ?
\VS{8}Jésus, ayant aussitôt connu par son esprit qu'ils raisonnaient ainsi en eux-mêmes, leur dit : Pourquoi avez-vous de telles pensées dans vos cœurs ?
\VS{9}Lequel est le plus aisé de dire au paralytique : Tes péchés te sont pardonnés, ou de dire : Lève-toi, prends ton lit, et marche ?
\VS{10}Mais afin que vous sachiez que le Fils de l'homme a le pouvoir sur la terre de pardonner les péchés, il dit au paralytique :
\VS{11}Je te dis : Lève-toi, prends ton lit, et va dans ta maison.
\VS{12}Et il se leva aussitôt, et ayant pris son lit, il sortit en présence de tous ; de sorte qu'ils furent tous étonnés, et ils glorifièrent Dieu, en disant : Nous n’avons jamais rien vu de pareil.
\TextTitle{[Appel de Matthieu]
\\(Mt. 9:9 ; Lu. 5:27-28)}
\VS{13}Jésus sortit de nouveau du côté de la mer, toute la foule venait à lui, et il les enseignait.
\VS{14}En passant, il vit Lévi, fils d'Alphée, assis au bureau des péages, et il lui dit : Suis-moi. Et Lévi s'étant levé, le suivit.
\TextTitle{[Jésus appelle des pêcheurs à la repentance, non des justes]
\\(Mt. 9:10-15 ; Lu. 5:29-35)}
\VS{15}Comme Jésus était à table dans la maison de Lévi, plusieurs publicains et des gens de mauvaise vie se mirent aussi à table avec lui et avec ses disciples ; car ils étaient nombreux, et l'avaient suivi.
\VS{16}Mais les scribes et les pharisiens voyant qu'il mangeait avec les publicains et les gens de mauvaise vie, disaient à ses disciples : Pourquoi mange-t-il et boit-il avec les publicains et les gens de mauvaise vie ?
\VS{17}Jésus ayant entendu cela, leur dit : Ce ne sont pas ceux qui se portent bien qui ont besoin de médecin, mais les malades. Je ne suis pas venu appeler à la repentance les justes, mais les pécheurs.
\TextTitle{[Les pharisiens et les disciples de Jean interrogent Jésus sur le jeûne]}
\VS{18}Les disciples de Jean et ceux des pharisiens jeûnaient ; ils vinrent à Jésus et lui dirent : Pourquoi les disciples de Jean, et ceux des pharisiens, jeûnent-ils, tandis que tes disciples ne jeûnent point ?
\VS{19}Jésus leur répondit : Les amis de l'Epoux peuvent-ils jeûner pendant que l'Epoux est avec eux ? Aussi longtemps qu’ils ont avec eux l'Epoux, ils ne peuvent jeûner.
\VS{20}Mais les jours viendront où l'Epoux leur sera ôté, alors ils jeûneront en ce jour-là.
\TextTitle{[Parabole du drap neuf et des outres neuves]
\\(Mt. 9:16-17 ; Lu. 5:36-39)}
\VS{21}Personne ne coud une pièce de drap neuf à un vieil habit ; autrement, la pièce du drap neuf emporterait une partie du vieux, et la déchirure serait pire.
\VS{22}Et personne ne met du vin nouveau dans de vieilles outres ; autrement, le vin nouveau fait rompre les outres, et le vin se répand, et les outres sont perdues ; mais le vin nouveau doit être mis dans des outres neuves.
\TextTitle{[Jésus, le Maître du sabbat]
\\(Mt. 12:1-8 ; Lu. 6:1-5)}
\VS{23}Il arriva, un jour de sabbat, que Jésus traversa des champs de blé. Ses disciples en marchant se mirent à arracher des épis.
\VS{24}Les pharisiens lui dirent : Regarde, pourquoi font-ils ce qui n'est pas permis les jours de sabbat ?
\VS{25}Mais il leur dit : N'avez-vous jamais lu ce que fit David quand il fut dans la nécessité, et qu'il eut faim, lui et ceux qui étaient avec lui ?
\VS{26}Comment il entra dans la maison de Dieu, au temps du souverain sacrificateur Abiathar, et mangea les pains de proposition{\FTNT{1 S. 21:1-7.}} , qu’il n'est permis qu'aux sacrificateurs de manger ; et il en donna même à ceux qui étaient avec lui!
\VS{27}Puis il leur dit : Le sabbat a été fait pour l'homme, et non pas l'homme pour le sabbat ;
\VS{28}de sorte que le Fils de l'homme est Maître même du sabbat.
\TextTitle{[Jésus-Christ guérit un homme à la main sèche le jour du sabbat]
\\(Mt. 12:9-13 ; Lu. 6:6-11)}
\Chap{3}
\VerseOne{}Jésus entra de nouveau dans la synagogue, et il y avait là un homme qui avait une main sèche.
\VS{2}Ils l'observaient, pour voir s'il le guérirait le jour du sabbat, afin de l'accuser.
\VS{3}Et Jésus dit à l'homme qui avait la main sèche : Lève-toi, et tiens-toi là au milieu.
\VS{4}Puis il leur dit : Est-il permis de faire du bien les jours de sabbat, ou de faire du mal, de sauver une personne, ou de la tuer ? Mais ils gardèrent le silence.
\VS{5}Alors, les regardant tous avec indignation, et étant affligé de l'endurcissement de leur cœur, il dit à cet homme : Etends ta main. Il l'étendit, et sa main fut rendue saine comme l'autre.
\TextTitle{[Nombreuses guérisons de Jésus]
\\(Mt. 12:15-16 ; Lu. 6:17-19}
\VS{6}Alors les pharisiens sortirent, et aussitôt, ils se consultèrent avec les hérodiens, sur les moyens de le faire périr.
\VS{7}Mais Jésus se retira vers la mer avec ses disciples. Une grande multitude le suivit de la Galilée,
\VS{8}de Judée, de Jérusalem, de l’Idumée, d’au-delà du Jourdain, et des environs de Tyr et de Sidon, une grande multitude, ayant entendu les grandes choses qu'il faisait, vint vers lui en grand nombre.
\VS{9}Et il dit à ses disciples de tenir toujours à sa disposition une petite barque, afin de ne pas être pressé par la foule.
\VS{10}Car, comme il guérissait beaucoup de gens, tous ceux qui avaient des maladies se jetaient sur lui pour le toucher.
\VS{11}Et les esprits impurs, quand ils le voyaient, se prosternaient devant lui, et s'écriaient en disant : Tu es le Fils de Dieu.
\VS{12}Mais il leur défendait avec de grandes menaces de le faire connaître.
\TextTitle{[L'appel des douze apôtres]
\\(Mt. 10:1-4 ; Lu. 6:13-16)}
\VS{13}Puis il monta sur une montagne, appela ceux qu'il voulut, et ils vinrent auprès de lui.
\VS{14}Il en établit douze pour être avec lui,
\VS{15}et pour les envoyer prêcher, avec la puissance de guérir les maladies, et de chasser les démons.
\VS{16}Voici les douze qu’il établit : Simon qu'il nomma Pierre ;
\VS{17}Jacques fils de Zébédée, et Jean, frère de Jacques, auxquels il donna le nom de Boanergès, ce qui veut dire fils de tonnerre.
\VS{18}André ; Philippe ; Barthélemy ; Matthieu ; Thomas ; Jacques, fils d'Alphée ; Thaddée ; Simon le Cananite ;
\VS{19}et Judas Iscariot, celui qui livra Jésus.
\VS{20}Ils se rendirent à la maison, et une grande multitude s’assembla de nouveau, en sorte qu’ils ne pouvaient même pas prendre leur repas.
\VS{21}Quand les parents de Jésus apprirent cela, ils sortirent pour se saisir de lui. Car ils disaient : Il est hors de sens.
\TextTitle{[Le blasphème contre le Saint-Esprit]
\\(Mt. 12:24-32 ; Lu. 11:15-23)}
\VS{22}Et les scribes, qui étaient descendus de Jérusalem, disaient : Il est possédé par Béelzébul ; c’est par le prince des démons qu’il chasse les démons.
\VS{23}Mais Jésus les appela, et leur dit sous forme de paraboles : Comment Satan peut-il chasser Satan ?
\VS{24}Si un royaume est divisé contre lui-même, ce royaume ne peut subsister ;
\VS{25}et si une maison est divisée contre elle-même, cette maison ne peut subsister.
\VS{26}Si donc Satan s'élève contre lui-même, il est divisé, il ne peut subsister, mais il tend vers sa fin.
\VS{27}Personne ne peut entrer dans la maison d'un homme fort et piller ses biens, sans avoir auparavant lié cet homme fort ; alors il pillera sa maison.
\VS{28}Je vous le dis en vérité, que toutes sortes de péchés seront pardonnés aux enfants des hommes, et aussi toutes sortes de blasphèmes par lesquels ils auront blasphémé ;
\VS{29}mais quiconque blasphémera contre le Saint-Esprit n’obtiendra jamais de pardon : Il est coupable et subira une condamnation éternelle {\FTNT{Voir commentaire Mt. 12:32.}}.
\VS{30}Jésus parla ainsi parce qu'ils disaient : Il est possédé d'un esprit impur.
\TextTitle{[La famille spirituelle]
\\(Mt. 12:46-50 ; Lu. 8:19-21)}
\VS{31}Survinrent ses frères et sa mère qui, se tenant dehors, l'envoyèrent appeler. La multitude était assise autour de lui,
\VS{32}et on lui dit : Voici, ta mère et tes frères sont dehors et te demandent.
\VS{33}Mais il leur répondit : Qui est ma mère, et qui sont mes frères ?
\VS{34}Et, jetant les regards sur ceux qui étaient assis tout autour de lui, il dit : Voici ma mère et mes frères.
\VS{35}Car quiconque fera la volonté de Dieu, celui-là est mon frère, ma sœur, et ma mère.
\TextTitle{[Parabole du semeur et des quatre terrains]
\\(Mt. 13:1-17 ; Lu. 8:4-10)}
\Chap{4}
\VerseOne{}Jésus se mit de nouveau à enseigner près de la mer, et une grande foule s’étant assemblée auprès de lui, il monta dans une barque et s’assit dans la barque, sur la mer. Toute la foule était à terre sur le rivage de la mer.
\VS{2}Il leur enseignait beaucoup de choses en paraboles, et il leur dit dans son enseignement :
\VS{3}Ecoutez. Un semeur sortit pour semer.
\VS{4}Comme il semait, une partie de la semence tomba le long du chemin, et les oiseaux du ciel vinrent, et la mangèrent toute.
\VS{5}Une autre partie tomba dans les endroits pierreux, où elle n'avait pas beaucoup de terre ; elle leva aussitôt, parce qu'elle n'entrait pas profondément dans la terre ;
\VS{6}mais, quand le soleil parut, elle fut brûlée, et parce qu'elle n'avait pas de racine, elle se sécha.
\VS{7}Une autre partie tomba parmi les épines ; et les épines montèrent, et l'étouffèrent, et elle ne donna pas de fruit.
\VS{8}Une autre partie tomba dans la bonne terre, et donna du fruit qui montait et croissait en sorte qu'un grain en rapporta trente, un autre soixante, et un autre cent.
\VS{9}Et il leur dit : Que celui qui a des oreilles pour entendre, qu'il entende !
\VS{10}Lorsqu’il fut à l’écart, ceux qui étaient autour de lui avec les douze, l'interrogèrent touchant cette parabole.
\VS{11}Et il leur dit : Il vous est donné de connaître le mystère du Royaume de Dieu ; mais pour ceux qui sont dehors, tout se passe en paraboles,
\VS{12}afin qu'en voyant ils voient et n'aperçoivent point, et qu'en entendant ils entendent et ne comprennent point, de peur qu'ils ne se convertissent, et que leurs péchés ne leur soient pardonnés.
\TextTitle{[Explication de la parabole]
\\(Mt. 13:18-23 ; Lu. 8:11-15)}
\VS{13}Puis il leur dit : Ne comprenez-vous pas cette parabole ? Et comment donc comprendrez-vous toutes les paraboles ?
\VS{14}Le semeur c'est celui qui sème la parole.
\VS{15}Ceux qui sont le long du chemin, ce sont ceux en qui la parole est semée. Quand ils l’ont entendue, aussitôt Satan vient et enlève la parole qui a été semée dans leurs cœurs.
\VS{16}De même, ceux qui reçoivent la semence dans les endroits pierreux, ce sont ceux qui entendent la parole, ils la reçoivent aussitôt avec joie ;
\VS{17}mais ils n'ont pas de racine en eux-mêmes, ils croient pour un temps, et dès que survient une tribulation ou une persécution à cause de la parole, ils y trouvent une occasion de chute.
\VS{18}D’autres reçoivent la semence parmi les épines ; ce sont ceux qui entendent la parole,
\VS{19}mais en qui les soucis de ce monde, et la séduction des richesses, et les convoitises des autres choses étant entrées dans leurs esprits, étouffent la parole, et elle devient infructueuse.
\VS{20}Mais ceux qui ont reçu la semence dans la bonne terre, ce sont ceux qui entendent la parole, la reçoivent, et portent du fruit : L'un trente, et l'autre soixante, et l'autre cent{\FTNT{Voir commentaire Mt. 13:8.}}.
\TextTitle{[Parabole de la lampe]
\\(Mt. 5:15-16 ; Lu. 8:16-18 ; 11:33-36)}
\VS{21}Il leur dit encore : Apporte-t-on la lampe pour la mettre sous un boisseau, ou sous un lit ? N'est-ce pas pour la mettre sur un chandelier ?
\VS{22}Car il n'y a rien de secret qui ne doive être découvert, rien de caché qui ne doive être mis à jour.
\VS{23}Si quelqu'un a des oreilles pour entendre, qu'il entende.
\VS{24}Il leur dit encore : Prenez garde à ce que vous entendez. On vous mesurera avec la mesure dont vous vous serez servis, et on y ajoutera pour vous.
\VS{25}Car on donnera à celui qui a ; mais à celui qui n’a pas, on ôtera même ce qu’il a.
\TextTitle{[Parabole de la semence et de la croissance spirituelle]}
\VS{26}Il dit encore : Il en est du Royaume de Dieu comme quand un homme jette la semence en terre ;
\VS{27}qu’il dorme ou qu’il veille, nuit et jour, la semence germe et croit, sans qu'il sache comment.
\VS{28}Car la terre produit d'elle-même, premièrement l'herbe, ensuite l'épi, et puis le grain formé dans l'épi ;
\VS{29}et quand le fruit est mûr, on y met aussitôt la faucille, parce que la moisson est prête.
\TextTitle{[Parabole du grain de moutarde]
\\(Mt. 13:31-33 ; Lu. 13:18-19)}
\VS{30}Il dit encore : A quoi comparerons-nous le Royaume de Dieu, ou par quelle parabole le représenterons-nous ?
\VS{31}Il en est comme du grain de moutarde, qui, lorsqu'on le sème dans la terre, est la plus petite de toutes les semences qui sont jetées dans la terre.
\VS{32}Mais après qu'il a été semé, il monte et devient plus grand que toutes les autres plantes, et pousse de grandes branches, en sorte que les oiseaux du ciel peuvent faire leurs nids sous son ombre.
\VS{33}C’est par beaucoup de paraboles de cette sorte qu’il leur annonçait la parole de Dieu, selon qu'ils pouvaient l'entendre.
\VS{34}Et il ne leur parlait point sans paraboles ; mais en particulier, il expliquait tout à ses disciples.
\TextTitle{[Jésus apaise la tempête]
\\(Mt. 8:23-27 ; Lu. 8:22-25)}
\VS{35}Ce même jour sur le soir, Jésus leur dit : Passons sur l’autre bord.
\VS{36}Après avoir renvoyé la foule, ils l'emmenèrent avec eux, dans la barque ; et il y avait aussi d'autres petites barques avec lui.
\VS{37}Et il se leva un grand tourbillon, et les flots se jetaient dans la barque, de sorte qu'elle se remplissait déjà.
\VS{38}Et lui, il dormait à la poupe sur un oreiller. Ils le réveillèrent, et lui dirent : Maître, ne t’inquiètes-tu pas de ce que nous périssions ?
\VS{39}S’étant réveillé, il menaça le vent, et dit à la mer : Silence ! Tais-toi ! Et le vent cessa, et il eut un grand calme.
\VS{40}Puis il leur dit : Pourquoi avez-vous si peur ? Comment n'avez-vous point de foi ?
\VS{41}Et ils furent saisis d'une grande crainte, et ils se dirent les uns les autres : Quel est donc celui-ci, à qui obéissent le vent et la mer ?
\TextTitle{[Jésus-Christ délivre un possédé à Gadara]
\\(Mt. 8:28-34 ; Lu. 8:26-40)}
\Chap{5}
\VerseOne{}Ils arrivèrent sur l’autre bord de la mer, dans le pays des Gadaréniens.
\VS{2}Aussitôt que Jésus fut descendu de la barque, un homme possédé d’un esprit impur, sortit des sépulcres, et vint le rencontrer.
\VS{3}Cet homme avait sa demeure dans les sépulcres, et personne ne pouvait plus le lier, pas même avec des chaînes.
\VS{4}Car souvent, il avait eu les fers aux pieds et avait été lié de chaînes, mais il avait rompu les chaînes et brisé les fers, et personne ne pouvait le dompter.
\VS{5}Il était continuellement, nuit et jour sur les montagnes, et dans les sépulcres, criant et se meurtrissant avec des pierres.
\VS{6}Ayant vu Jésus de loin, il courut et se prosterna devant lui.
\VS{7}Et s’écria d’une voix forte : Qu'y a-t-il entre toi et moi, Jésus, Fils du Dieu Très-Haut ? Je te conjure au Nom de Dieu de ne pas me tourmenter.
\VS{8}Car Jésus lui disait : Sors de cet homme, esprit impur.
\VS{9}Alors il lui demanda : Quel est ton nom ? Légion{\FTNT{Une légion romaine contenait entre trois et six mille soldats. C’est autant de démons dont l’homme était possédé.}} est mon nom, lui répondit-il, car nous sommes plusieurs.
\VS{10}Et il le priait instamment de ne pas les envoyer hors de cette contrée.
\VS{11}Il y avait là, vers les montagnes, un grand troupeau de pourceaux qui paissaient.
\VS{12}Et tous ces démons le priaient en disant : Envoie-nous dans les pourceaux, afin que nous entrions en eux ; et aussitôt Jésus le leur permit.
\VS{13}Alors ces esprits impurs étant sortis, entrèrent dans les pourceaux, qui était environ deux mille, et le troupeau se précipita des pentes escarpées dans la mer ; et ils se noyèrent dans la mer.
\VS{14}Ceux qui paissaient les pourceaux s'enfuirent, et répandirent la nouvelle dans la ville et dans les campagnes.
\VS{15}Ceux de la ville sortirent pour voir ce qui était arrivé. Ils vinrent à Jésus et ils virent le démoniaque, celui qui avait eu la légion, assis et vêtu, et dans son bon sens ; et ils furent saisis de crainte.
\VS{16}Et ceux qui avaient vu le miracle leur racontèrent ce qui était arrivé au démoniaque et aux pourceaux.
\VS{17}Alors ils se mirent à supplier Jésus de quitter leur territoire.
\VS{18}Comme il montait dans la barque, celui qui avait été démoniaque le pria de lui permettre de rester avec lui.
\VS{19}Mais Jésus ne le lui permit pas, mais il lui dit : Va dans ta maison, vers les tiens, et raconte-leur les grandes choses que le Seigneur t'a faites, et comment il a eu pitié de toi.
\VS{20}Il s'en alla donc, et se mit à publier dans la Décapole les grandes choses que Jésus lui avait faites. Et tous furent dans l’étonnement.
\TextTitle{[La résurrection de la fille de Jaïrus et la guérison de la femme atteinte d'une perte de sang]
\\(Mt. 9:18-26 ; Lu. 8:41-56)}
\VS{21}Jésus dans la barque regagna l’autre rive, où une grande foule s’assembla près de lui. Il était près de la mer.
\VS{22}Alors vint un des chefs de la synagogue, nommé Jaïrus, qui l’ayant aperçu, se jeta à ses pieds,
\VS{23}et le pria instamment, en disant : Ma petite fille est à l'extrémité. Je te prie de venir et de lui imposer les mains, afin qu'elle soit guérie et qu'elle vive.
\VS{24}Jésus s'en alla donc avec lui. Et de grandes foules de gens le suivaient et le pressaient.
\VS{25}Or, il y avait une femme qui avait une perte de sang depuis douze ans,
\VS{26}et qui avait beaucoup souffert entre les mains de plusieurs médecins. Elle avait dépensé tout ce qu’elle possédait, sans avoir éprouvé aucun soulagement, mais était allée plutôt en empirant.
\VS{27}Ayant entendu parler de Jésus, elle vint dans la foule par derrière et toucha son vêtement.
\VS{28}Car elle disait : Si je puis seulement toucher ses vêtements, je serai guérie.
\VS{29}Au même instant, la perte de sang s'arrêta ; et elle sentit en son corps qu'elle était guérie de son fléau.
\VS{30}Et aussitôt Jésus connut en lui-même qu’une force était sortie de lui, et, se retournant vers la foule, il dit : Qui a touché mes vêtements ?
\VS{31}Et ses disciples lui dirent : Tu vois que la foule te presse, et tu dis : Qui m'a touché ?
\VS{32}Mais il regardait tout autour pour voir celle qui avait fait cela.
\VS{33}Alors la femme saisie de crainte et toute tremblante, sachant ce qui s’était passé en elle, vint et se jeta à ses pieds, et lui déclara toute la vérité.
\VS{34}Mais Jésus lui dit : Ma fille ! Ta foi t'a sauvée. Va en paix, et sois guérie de ton fléau.
\VS{35}Comme il parlait encore, il vint des gens de chez le chef de la synagogue, qui lui dirent : Ta fille est morte, pourquoi importuner davantage le Maître ?
\VS{36}Mais aussitôt que Jésus eut entendu cela, il dit au chef de la synagogue : Ne crains pas, crois seulement.
\VS{37}Et il ne permit à personne de le suivre, si ce n’est à Pierre, à Jacques, et à Jean, frère de Jacques.
\VS{38}Ils arrivèrent à la maison du chef de la synagogue, où Jésus vit le tumulte, c'est-à-dire ceux qui pleuraient et qui poussaient de grands cris.
\VS{39}Il entra, et leur dit : Pourquoi faites-vous tout ce bruit, et pourquoi pleurez-vous ? L’enfant n'est pas morte, mais elle dort.
\VS{40}Et ils se moquèrent de lui. Mais Jésus les ayant tous fait sortir, prit le père et la mère de la petite fille, et ceux qui étaient avec lui, et entra là où la petite fille était couchée.
\VS{41}Il la saisit par la main, et lui dit : Talitha koumi, ce qui signifie : Jeune fille, je te dis lève-toi.
\VS{42}Aussitôt la petite fille se leva, et se mit à marcher ; car elle était âgée de douze ans. Et ils furent dans un grand étonnement.
\VS{43}Jésus leur recommanda fort expressément que personne ne le sache ; et il dit qu'on donne à manger à la jeune fille.
\TextTitle{[Jésus à Nazareth]}
\Chap{6}
\VerseOne{}Jésus partit de là, et se rendit dans sa patrie. Ses disciples le suivirent.
\VS{2}Quand le jour du sabbat fut venu, il se mit à enseigner dans la synagogue. Et beaucoup de ceux qui l'entendaient étaient dans l'étonnement, et ils disaient : D'où lui viennent ces choses ? Et quelle est cette sagesse qui lui a été donnée, et comment de tels prodiges se font-ils par ses mains ?
\VS{3}N’est-ce pas le charpentier, le Fils de Marie, frère de Jacques, de Joses, de Jude, et de Simon ? Et ses sœurs ne sont-elles pas ici parmi nous ? Et ils étaient scandalisés à cause de lui.
\VS{4}Mais Jésus leur dit : Un prophète n'est méprisé que dans sa patrie, parmi ses parents et dans sa famille.
\VS{5}Et il ne put faire là aucun miracle, si ce n’est qu'il guérit quelques malades en leur imposant les mains.
\VS{6}Et il s'étonnait de leur incrédulité. Jésus parcourait les villages d'alentour, en enseignant.
\TextTitle{[Mission des apôtres]
\\(Mt. 10:1-42 ; Lu. 9:1-6)}
\VS{7}Alors il appela les douze, et commença à les envoyer deux à deux, en leur donnant pouvoir sur les esprits impurs.
\VS{8}Il leur prescrit de ne rien prendre pour le chemin, si ce n’est un bâton, et de ne porter ni sac, ni pain, ni monnaie dans leur ceinture ;
\VS{9}de chausser des sandales, et de ne pas porter deux tuniques.
\VS{10}Il leur disait aussi : Dans quelque maison que vous entriez, demeurez-y jusqu'à ce que vous partiez de là.
\VS{11}Et tous ceux qui ne vous recevront pas, et ne vous écouteront pas, en partant de là, secouez la poussière de vos pieds, en témoignage contre eux. Je vous le dis en vérité que ceux de Sodome et de Gomorrhe seront traités moins rigoureusement au jour du jugement que cette ville-là.
\VS{12}Ils partirent, et ils prêchèrent la repentance.
\VS{13}Ils chassèrent beaucoup de démons hors des possédés, et ils oignirent d'huile beaucoup de malades et les guérirent.
\TextTitle{[Jean-Baptiste décapité]
\\(Mt. 14:1-14 ; Lu. 9:7-9)}
\VS{14}Le roi Hérode entendit parler de Jésus, dont le nom était devenu fort célèbre, et il dit : C’est Jean-Baptiste qui est ressuscité des morts ; c'est pourquoi la puissance de faire des miracles agit puissamment en lui.
\VS{15}D’autres disaient : C'est Elie. Et les autres disaient : C'est un prophète, comme l’un des prophètes.
\VS{16}Mais Hérode en apprenant cela, disait : C'est Jean que j'ai fait décapiter, il est ressuscité des morts.
\VS{17}Car Hérode avait fait arrêter Jean, et l'avait fait lier en prison, à cause d'Hérodias, femme de Philippe son frère, parce qu'il l'avait prise en mariage.
\VS{18}Et que Jean lui disait : Il ne t'est pas permis d'avoir la femme de ton frère.
\VS{19}C'est pourquoi Hérodias était irritée contre Jean, et voulait le faire mourir, mais elle ne le pouvait pas ;
\VS{20}parce qu’Hérode craignait Jean, sachant que c'était un homme juste et saint ; il le protégeait, et, après l’avoir entendu, il faisait beaucoup selon ses avis, et l’écoutait avec plaisir.
\VS{21}Cependant, un jour propice arriva, lorsque Hérode à l’occasion du jour de sa naissance, donna un festin aux grands de sa cour, aux chefs militaires et aux principaux de la Galilée.
\VS{22}La fille d'Hérodias entra dans la salle ; elle dansa et plut à Hérode, et à ceux qui étaient à table avec lui. Le roi dit à la jeune fille : Demande-moi ce que tu voudras, et je te le donnerai.
\VS{23}Il ajouta avec serment : Tout ce que tu me demanderas, je te le donnerai, serait-ce la moitié de mon royaume.
\VS{24}Etant sortie, elle dit à sa mère : Que demanderai-je ? Et sa mère lui dit : La tête de Jean-Baptiste.
\VS{25}Et étant revenue en toute hâte vers le roi, et lui fit cette demande : Je veux que tu me donnes à l’instant sur un plat, la tête de Jean-Baptiste.
\VS{26}Le roi fut attristé, mais à cause de son serment et des convives, il ne voulut pas refuser.
\VS{27}Il envoya sur-le-champ l’un de ses gardes, avec ordre d'apporter la tête de Jean.
\VS{28}Le garde alla décapiter Jean dans la prison, et apporta sa tête sur un plat, et la donna à la jeune fille. Et la jeune fille la donna à sa mère.
\VS{29}Les disciples de Jean ayant appris cela, vinrent et emportèrent son corps, et le mirent dans un sépulcre.
\TextTitle{[Les apôtres rendent compte de leur mission à Jésus]
\\(Lu. 9:10)}
\VS{30}Les apôtres se rassemblèrent auprès de Jésus, et lui racontèrent tout ce qu'ils avaient fait et enseigné.
\VS{31}Jésus leur dit : Venez à l'écart dans un lieu désert, et reposez-vous un peu ; car il y avait beaucoup de gens qui allaient et qui venaient, de sorte qu'ils n'avaient même pas le temps de manger.
\TextTitle{[Multiplication des pains pour les cinq mille hommes]
\\(Mt. 14:12-21 ; Lu. 9:15-17 ; Jn. 6:1-14)}
\VS{32}Ils s'en allèrent donc dans une barque, à l’écart, dans un lieu désert.
\VS{33}Beaucoup de gens les virent s’en aller et les reconnurent, et de toutes les villes on accourut à pied et on les devança au lieu où ils se rendaient.
\VS{34}Quand il sortit, Jésus vit une grande foule, et fut ému de compassion pour elle, parce qu’ils étaient comme des brebis qui n'ont pas de pasteur ; et il se mit à leur enseigner plusieurs choses.
\VS{35}Comme il était déjà tard, ses disciples s'approchèrent de lui, en disant : Ce lieu est désert, et il est déjà tard,
\VS{36}renvoie-les, afin qu’ils s'en aillent dans les campagnes et dans les villages des environs pour s’acheter des pains ; car ils n'ont rien à manger.
\VS{37}Jésus leur répondit : Donnez-leur vous-mêmes à manger. Et ils lui dirent : Irions-nous acheter des pains pour deux cents deniers, et leur donnerions-nous à manger ?
\VS{38}Et il leur dit : Combien avez-vous de pains ? Allez voir. Et quand ils le surent, ils répondirent : Cinq, et deux poissons.
\VS{39}Alors il leur commanda de les faire tous asseoir par groupes sur l'herbe verte.
\VS{40}Et ils s'assirent par rangées de cent et de cinquante personnes.
\VS{41}Il prit les cinq pains et les deux poissons, et, levant les yeux vers le ciel, il bénit Dieu et rompit les pains, puis il les donna à ses disciples, afin qu'ils les distribuent à la foule. Il partagea aussi les deux poissons entre tous.
\VS{42}Tous mangèrent et furent rassasiés.
\VS{43}Et l’on emporta douze paniers pleins de morceaux de pains et de ce qui restait des poissons.
\VS{44}Ceux qui avaient mangé les pains étaient environ cinq mille hommes.
\TextTitle{[Jésus marche sur la mer]
\\(Mt. 14:22-33 ; Jn. 6:15-21)}
\VS{45}Et aussitôt après, il obligea ses disciples à monter dans la barque, et à le devancer sur l’autre bord, vers Bethsaïda, pendant que lui-même renverrait la foule.
\VS{46}Quand il l’eut renvoyée, il s'en alla sur la montagne pour prier.
\VS{47}Le soir étant venu, la barque était au milieu de la mer, et Jésus était seul à terre.
\VS{48}Il vit qu'ils avaient beaucoup de peine à ramer, parce que le vent leur était contraire. Vers la quatrième veille de la nuit, il alla vers eux marchant sur la mer, et il voulait les devancer.
\VS{49}Quand ils le virent marcher sur la mer, ils crurent que c’était un fantôme, et ils poussèrent des cris ;
\VS{50}car ils le voyaient tous, et ils furent troublés. Mais il leur parla aussitôt, et leur dit : Rassurez-vous, c'est moi. N’ayez pas peur.
\VS{51}Et il monta vers eux dans la barque, et le vent cessa. Et ils furent en eux-mêmes excessivement étonnés et remplis d’admiration.
\VS{52}Car ils n'avaient pas compris le miracle des pains, parce que leur cœur était endurci.
\TextTitle{[Jésus guérit les malades à Génésareth]
\\(Mt. 14:34-36)}
\VS{53}Après avoir traversé la mer, ils arrivèrent dans la contrée de Génésareth, où ils abordèrent.
\VS{54}Et dès qu’ils furent sortis de la barque, les gens, ayant aussitôt reconnu Jésus,
\VS{55}parcoururent tous les environs, et se mirent à lui apporter de tous côtés les malades sur de petits lits, partout où ils apprenaient qu'il était.
\VS{56}Et partout où il entrait, dans les villages, dans les villes, ou dans les campagnes, ils mettaient les malades dans les places publiques, et ils le priaient de leur permettre seulement de toucher le bord de son vêtement. Et tous ceux qui le touchaient étaient guéris.
\TextTitle{[Jésus condamne les traditions]
\\(Mt. 15:1-9)}
\Chap{7}
\VerseOne{}Alors les pharisiens, et quelques scribes qui étaient venus de Jérusalem, s'assemblèrent auprès de Jésus.
\VS{2}Ils virent quelques-uns de ses disciples mangeant du pain avec des mains impures, c'est-à-dire non lavées, et ils les blâmèrent.
\VS{3}Or, les pharisiens et tous les Juifs ne mangent pas sans s’être lavé leurs mains jusqu’au coude, conformément à la tradition des anciens.
\VS{4}Et quand ils reviennent de la place publique, ils ne mangent qu’après s’être lavés{\FTNT{Le verbe laver vient du grec «~baptizo~»~: «~Plonger, immerger, submerger, purifier en plongeant ou en submergeant, laver, rendre pur avec de l'eau, se baigner~» (Mt. 3:6-16~; Mt. 28:19~; Ac. 1:5~; Ac. 2:38~; 1 Co. 12:13, etc.) Jésus évoque ici les rites de purification chez les Juifs au premier siècle. A cette époque, le souci de purification avait conduit des groupes comme les pharisiens et les esséniens à multiplier les rites d’eau. Les découvertes de Qumran ont montré que les esséniens vivaient dans la hantise de ce qui aurait pu les rendre impurs. Ainsi, les rituels de purification avec de l’eau rythmaient la vie des juifs. A titre d’exemple, les jarres de Cana étaient utilisés à cet effet (Jn. 2:6).}}. Il y a plusieurs autres observances dont ils se sont chargés, comme le lavage des coupes, de cruches, des vases d'airain, et des lits.
\VS{5}Et les pharisiens et les scribes l'interrogèrent, en disant : Pourquoi tes disciples ne se conduisent-ils pas selon la tradition des anciens, mais prennent-ils leur repas sans se laver les mains ?
\VS{6}Jésus leur répondit : Hypocrites, Esaïe a bien prophétisé de vous, ainsi qu’il est écrit : Ce peuple m'honore des lèvres, mais leur cœur est éloigné de moi{\FTNT{Es. 29:13.}}.
\VS{7}C’est en vain qu’ils m'honorent, en enseignant des doctrines qui sont des commandements d'hommes.
\VS{8}Vous abandonnez le commandement de Dieu, et vous retenez la tradition des hommes, à savoir le lavage des cruches et des coupes, et vous faites beaucoup d'autres choses semblables.
\VS{9}Il leur dit aussi : Vous rejetez bien le commandement de Dieu, afin de garder votre tradition.
\VS{10}Car Moïse a dit : Honore ton père et ta mère ; et : celui qui maudira son père ou sa mère, sera puni de mort.
\VS{11}Mais vous, vous dites : Si quelqu'un dit à son père ou à sa mère : Tout ce dont je pourrais t’assister est corban, c’est-à-dire une offrande à Dieu, il ne sera point coupable.
\VS{12}Et vous ne lui permettez plus de rien faire pour son père ou pour sa mère,
\VS{13}anéantissant ainsi la parole de Dieu par votre tradition que vous avez établie. Et vous faites encore beaucoup d’autres choses semblables.
\TextTitle{[Le coeur humain]
\\(Mt. 15:10-20)}
\VS{14}Ensuite, ayant appelé la foule, il leur dit : Ecoutez-moi vous tous, et comprenez.
\VS{15}Il n’est hors de l’homme rien qui, entrant en lui, puisse le souiller ; mais ce qui sort de l’homme, c’est ce qui le souille.
\VS{16}Si quelqu'un a des oreilles pour entendre, qu'il entende.
\VS{17}Lorsqu’il fut entré dans la maison, loin de la foule, ses disciples l'interrogèrent sur cette parabole.
\VS{18}Et il leur dit : Vous aussi, êtes-vous sans intelligence ? Ne comprenez-vous pas que rien de ce qui du dehors entre dans l’homme ne peut le souiller ?
\VS{19}Car cela n'entre pas dans son cœur, mais dans son ventre, puis s’en va dans les lieux secrets, qui purifient le corps de tous les aliments.
\VS{20}Mais il leur dit : Ce qui sort de l'homme, c'est ce qui souille l'homme.
\VS{21}Car c’est du dedans, c'est-à-dire du cœur des hommes, que sortent les mauvaises pensées, les adultères, les fornications, les meurtres,
\VS{22}les vols, les cupidités, les méchancetés, la fraude, l'impudicité, le regard envieux, la calomnie, l’orgueil, la folie.
\VS{23}Tous ces maux sortent du dedans, et souillent l'homme.
\TextTitle{[Jésus et la femme syro-phénicienne]
\\(Mt. 15:21-28)}
\VS{24}Jésus, étant parti de là, s'en alla dans le territoire de Tyr et de Sidon. Il entra dans une maison, désirant que personne ne le sache ; mais il ne put rester caché.
\VS{25}Car une femme, dont la fille était possédée d'un esprit impur, ayant entendu parler de lui, vint et se jeta à ses pieds.
\VS{26}Cette femme était Grecque, Syro-Phénicienne d’origine. Elle le pria de chasser le démon hors de sa fille. Jésus lui dit :
\VS{27}Laisse premièrement les enfants se rassasier ; car il n'est pas raisonnable de prendre le pain des enfants, et de le jeter aux petits chiens.
\VS{28}Et elle lui répondit : Cela est vrai, Seigneur ! Cependant les petits chiens mangent sous la table les miettes que les enfants laissent tomber.
\VS{29}Alors il lui dit : A cause de cette parole va, le démon est sorti de ta fille.
\VS{30}Et quand elle rentra dans sa maison, elle trouva l’enfant couchée sur le lit, le démon étant sorti.
\TextTitle{[Jésus guérit un sourd-muet]
\\(Mt. 15:29-31)}
\VS{31}Jésus quitta le territoire de Tyr et de Sidon, et revint vers la mer de Galilée en traversant le pays de la Décapole.
\VS{32}On lui amena un sourd qui avait la parole empêchée, et on le pria de lui imposer les mains.
\VS{33}Jésus le prit à part, hors de la foule, lui mit les doigts dans les oreilles, et lui toucha la langue avec sa propre salive.
\VS{34}Puis, levant les yeux vers le ciel, il soupira, et lui dit : Ephphatha, c'est-à-dire : Ouvre-toi.
\VS{35}Aussitôt ses oreilles s'ouvrirent, et le lien de sa langue se délia, et il parla aisément.
\VS{36}Jésus leur recommanda de ne le dire à personne ; mais plus il le leur recommanda, plus ils le publièrent.
\VS{37}Et ils en étaient extrêmement étonnés, et disaient : Il fait tout à merveille ; même il fait entendre les sourds, et parler les muets.
\TextTitle{[Seconde multiplication des pains]
\\(Mt. 15:32-39)}
\Chap{8}
\VerseOne{}En ces jours-là, une grande foule s’était de nouveau réunie et n’avait rien à manger. Jésus appela ses disciples, et leur dit :
\VS{2}Je suis ému de compassion pour cette foule, car il y a déjà trois jours qu'ils sont près de moi, et ils n'ont rien à manger.
\VS{3}Si je les renvoie chez eux à jeun, ils tomberont en défaillance en chemin, car quelques-uns d'eux sont venus de loin.
\VS{4}Ses disciples lui répondirent : Comment pourrait-t-on les rassasier de pains, ici, dans un désert ?
\VS{5}Jésus leur demanda : Combien avez-vous de pains ? Sept lui répondirent-ils.
\VS{6}Alors il ordonna à la foule de s'asseoir par terre, et il prit les sept pains, et après avoir béni Dieu, il les rompit, et les donna à ses disciples pour les distribuer ; et ils les distribuèrent à la foule.
\VS{7}Ils avaient aussi quelques petits poissons ; et après avoir béni Dieu, il les fit aussi distribuer.
\VS{8}Ils mangèrent, et furent rassasiés ; et l’on remporta sept corbeilles pleines des morceaux qui restaient.
\VS{9}Ceux qui avaient mangé étaient environ quatre mille. Ensuite Jésus les renvoya.
\TextTitle{[L'enseignement corrompu des pharisiens]
\\(Mt. 16:1-12)}
\VS{10}Aussitôt après, il monta dans la barque avec ses disciples, et se rendit dans la contrée de Dalmanutha.
\VS{11}Les pharisiens survinrent, se mirent à discuter avec lui, et pour l'éprouver, lui demandèrent un signe venant du ciel.
\VS{12}Alors, Jésus soupirant profondément en son esprit, dit : Pourquoi cette génération demande-t-elle un signe ? Je vous le dis en vérité, il ne sera point donné de signe à cette génération.
\VS{13}Puis il les quitta, et remonta dans la barque, pour passer à l'autre rivage.
\VS{14}Les disciples avaient oublié de prendre des pains ; et ils n'en avaient qu'un seul avec eux dans la barque.
\VS{15}Jésus leur fit cette recommandation : Gardez-vous avec soin du levain des pharisiens et du levain d'Hérode.
\VS{16}Ils raisonnaient entre eux, disant : C'est parce que nous n'avons pas de pains.
\VS{17}Jésus, le sachant, leur dit : Pourquoi discourez-vous sur ce que vous n'avez pas de pains ? N’entendez-vous pas encore, et ne comprenez-vous pas ?
\VS{18}Avez-vous encore votre cœur endurci ? Ayant des yeux, ne voyez-vous point ? Ayant des oreilles, n'entendez-vous point ? Et n'avez-vous point de mémoire ?
\VS{19}Quand j’ai rompu les cinq pains pour les cinq mille hommes, combien de paniers pleins de morceaux avez-vous emportés ? Douze, lui répondirent-ils.
\VS{20}Et quand j’ai rompu les sept pains pour quatre mille hommes, combien de corbeilles pleines de morceaux avez-vous emportées ? Sept, répondirent-ils.
\VS{21}Et il leur dit : Comment n'avez-vous pas d'intelligence ?
\TextTitle{[Jésus guérit un aveugle]}
\VS{22}Ils se rendirent à Bethsaïda, et on lui présenta un aveugle, qu’on le pria de toucher.
\VS{23}Alors il prit la main de l'aveugle, et le conduisit hors du village ; puis il lui mit de la salive sur les yeux, lui imposa les mains, et lui demanda s'il voyait quelque chose.
\VS{24}Et cet homme ayant regardé, dit : Je vois des hommes qui marchent, et qui me paraissent comme des arbres.
\VS{25}Jésus lui mit de nouveau les mains sur les yeux, et lui dit de regarder ; et il fut rétabli, et les voyait tous distinctement.
\VS{26}Puis il le renvoya dans sa maison, en lui disant : N'entre pas dans le village, et ne le dis à personne du village.
\TextTitle{[Pierre reconnaît Jésus comme le Messie]
\\(Mt. 16:13-16 ; Lu. 9:18-21 ; Jn. 6:67-71)}
\VS{27}Jésus s’en alla, avec ses disciples, dans les villages de Césarée de Philippe, et sur le chemin il interrogea ses disciples, leur disant : Qui dit-on que je suis ?
\VS{28}Ils répondirent : Les uns disent que tu es Jean-Baptiste ; les autres, Elie ; et les autres, l'un des prophètes.
\VS{29}Alors il leur dit : Et vous, qui dites-vous que je suis ? Pierre lui répondit : Tu es le Christ.
\VS{30}Et il leur défendit très sévèrement de ne dire cela de lui à personne.
\VS{31}Alors il commença à leur enseigner qu'il fallait que le Fils de l'homme souffre beaucoup, qu'il soit rejeté par les anciens, par les principaux sacrificateurs et par les scribes, qu'il soit mis à mort, et qu'il ressuscite trois jours après.
\VS{32}Il leur tenait ces discours ouvertement. Et Pierre l’ayant pris à part, se mit à le reprendre.
\VS{33}Mais Jésus, se retournant et regardant ses disciples, réprimanda Pierre en lui disant : Va arrière de moi, Satan ! Car tu ne comprends pas les choses de Dieu, mais celles des hommes.
\TextTitle{[La consécration du disciple]
\\(Mt. 16:24-28 ; Lu. 9:23-26)}
\VS{34}Puis, ayant appelé la foule et ses disciples, il leur dit : Si quelqu’un veut venir après moi, qu'il renonce à lui-même, qu'il se charge de sa croix, et qu’il me suive.
\VS{35}Car quiconque voudra sauver son âme, la perdra ; mais quiconque perdra son âme pour l'amour de moi et de l'Evangile, celui-là la sauvera.
\VS{36}Car que sert-il à un homme de gagner tout le monde, s'il perd son âme ?
\VS{37}Que donnerait un homme en échange de son âme ?
\VS{38}Car quiconque aura honte de moi et de mes paroles au milieu de cette génération adultère et pécheresse, le Fils de l'homme aura aussi honte de lui, quand il viendra dans la gloire de son Père avec les saints anges.
\TextTitle{[La transfiguration]
\\(Mt. 17:1-8 ; Lu. 9:27-36)}
\Chap{9}
\VerseOne{}Il leur disait aussi : Je vous le dis en vérité, quelques-uns de ceux qui sont ici présents, ne mourront point qu’ils n’aient vu le Royaume de Dieu venir avec puissance{\FTNT{Voir commentaire Mt. 16:28.}}.
\VS{2}Six jours après, Jésus prit avec lui Pierre, Jacques et Jean, et les conduisit seuls à l'écart sur une haute montagne. Il fut transfiguré devant eux,
\VS{3}ses vêtements devinrent resplendissants, et blancs comme de la neige, tels qu'il n’est pas de foulon sur la terre qui puisse blanchir ainsi.
\VS{4}Et en même temps leur apparurent Moïse et Elie, qui s’entretenaient avec Jésus.
\VS{5}Alors Pierre prenant la parole, dit à Jésus : Rabbi, il est bon que nous soyons ici ; faisons donc trois tentes, une pour toi, une pour Moïse, et une pour Elie.
\VS{6}Car il ne savait pas quoi dire, ils étaient épouvantés.
\VS{7}Une nuée vint les couvrir de son ombre, et de la nuée sortit une voix : Celui-ci est mon Fils bien-aimé, écoutez-le.
\VS{8}Aussitôt les disciples regardèrent tout autour, et ils ne virent que Jésus seul avec eux.
\VS{9}Comme ils descendaient de la montagne, Jésus leur recommanda expressément de ne raconter à personne ce qu'ils avaient vu, jusqu’à ce que le Fils de l'homme soit ressuscité des morts.
\VS{10}Ils retinrent cette parole, se demandant entre eux ce que c'était que ressusciter des morts.
\VS{11}Les disciples l'interrogèrent, disant : Pourquoi les scribes disent-ils qu'il faut qu'Elie vienne premièrement ?
\VS{12}Il leur répondit : Il est vrai, Elie viendra premièrement, et rétablira toutes choses. Et pourquoi est-il écrit du Fils de l'homme qu’il doit beaucoup souffrir et être méprisé ?
\VS{13}Mais je vous dis qu’Elie est venu, et qu'ils lui ont fait tout ce qu’ils ont voulu, selon qu’il est écrit de lui.
\TextTitle{[Incapacité des disciples et la toute-puissance de Jésus-Christ]
\\(Mt. 17:14-21 ; Lu. 9:37-43)}
\VS{14}Lorsqu’il fut arrivé près des disciples, il vit une grande foule autour d’eux, et des scribes qui discutaient avec eux.
\VS{15}Dès que la foule vit Jésus, elle fut saisie d'étonnement, et accourut pour le saluer.
\VS{16}Alors il demanda aux scribes : De quoi discutez-vous avec eux ?
\VS{17}Et un homme de la foule prenant la parole, dit : Maître, je t'ai amené mon fils qui est possédé d’un esprit muet.
\VS{18}En quelque lieu qu’il le saisisse, il le jette par terre ; l’enfant écume, grince des dents, et devient tout raide. J’ai prié tes disciples de le chasser, mais ils n'ont pas pu.
\VS{19}Alors Jésus leur répondit : Ô génération incrédule ! Jusqu’à quand serai-je avec vous ? Jusqu’à quand vous supporterai-je ? Amenez-le-moi. Ils le lui amenèrent.
\VS{20}Et aussitôt que l’enfant vit Jésus, l'esprit l'agita sur-le-champ avec violence ; il tomba par terre, et se roulait en écumant.
\VS{21}Jésus demanda au père de l'enfant : Combien y a-t-il de temps que cela lui arrive ? Et il dit : Dès son enfance.
\VS{22}Et souvent l’esprit l’a jeté dans le feu et dans l'eau pour le faire périr. Mais si tu peux quelque chose, secours-nous, aie compassion de nous.
\VS{23}Alors Jésus lui dit : Si tu peux croire, tout est possible à celui qui croit.
\VS{24}Et aussitôt le père de l'enfant s'écriant avec larmes : Je crois, Seigneur ! Secours-moi dans mon incrédulité.
\VS{25}Jésus voyant accourir la foule, reprit sévèrement l’esprit impur, et lui dit : Esprit muet et sourd, je te l’ordonne, sors de cet enfant, et n'y rentre plus !
\VS{26}Et le démon sortit, en poussant des cris, et en l’agitant avec une grande violence. L’enfant devint comme mort, de sorte que plusieurs disaient qu’il était mort.
\VS{27}Mais Jésus, l'ayant pris par la main, le fit lever. Et il se tint debout.
\VS{28}Quand Jésus fut entré dans la maison, ses disciples lui demandèrent en particulier : Pourquoi n’avons-nous pas pu chasser cet esprit ?
\VS{29}Il leur répondit : Cette sorte de démons ne peut sortir que par la prière et par le jeûne.
\TextTitle{[Jésus annonce sa mort et sa résurrection]
\\(Mt. 17:22-23 ; Lu. 9:44-45)}
\VS{30}Puis étant partis de là, ils traversèrent la Galilée. Jésus ne voulait pas qu’on le sache.
\VS{31}Car il enseignait ses disciples, et il leur dit : Le Fils de l'homme va être livré entre les mains des hommes, et ils le feront mourir, mais après qu'il aura été mis à mort, il ressuscitera le troisième jour.
\VS{32}Mais ils ne comprenaient point ce discours, et ils craignaient de l'interroger.
\TextTitle{[L'humilité, secret de la vraie grandeur]
\\(Mt. 18:1-6 ; Lu. 9:46-48)}
\VS{33}Après ces choses il vint à Capernaüm, et quand il fut arrivé à la maison, il leur demanda : De quoi discutiez-vous ensemble en chemin ?
\VS{34}Mais ils gardèrent le silence, car ils avaient discuté entre eux en chemin sur celui qui serait le plus grand.
\VS{35}Alors il s’assit, appela les douze, et leur dit : Si quelqu'un veut être le premier parmi vous, il sera le dernier de tous, et le serviteur de tous.
\VS{36}Et ayant pris un petit enfant, il le mit au milieu d'eux, et après l'avoir pris entre ses bras, il leur dit :
\VS{37}Quiconque reçoit en mon Nom un de ces petits enfants, me reçoit ; et quiconque me reçoit, ce n'est pas moi qu’il reçoit, mais celui qui m'a envoyé.
\TextTitle{[Jésus condamne l'esprit sectaire]
\\(Lu. 9:49-50)}
\VS{38}Alors Jean prit la parole, et dit : Maître, nous avons vu quelqu'un qui chasse les démons en ton Nom et qui ne nous suit pas, et nous l'en avons empêché, parce qu'il ne nous suit pas.
\VS{39}Mais Jésus leur dit : Ne l'en empêchez pas ; car il n’est personne qui, faisant un miracle en mon Nom, puisse aussitôt après parler mal de moi.
\VS{40}Qui n'est pas contre nous est pour nous.
\VS{41}Et quiconque vous donnera à boire un verre d'eau en mon Nom, parce que vous êtes à Christ, je vous le dis en vérité, il ne perdra point sa récompense.
\TextTitle{[Avertissement de Jésus concernant les occasions de chute]}
\VS{42}Mais quiconque scandalisera un de ces petits qui croient en moi, il vaudrait mieux pour lui qu'on lui mette une pierre de moulin au cou, et qu'on le jette dans la mer.
\VS{43}Si ta main est pour toi une occasion de chute, coupe-la ; mieux vaut pour toi entrer manchot dans la vie, que d'avoir les deux mains, et d’aller dans la géhenne, dans le feu qui ne s'éteint point ;
\VS{44}là où leur ver ne meurt point, et le feu ne s'éteint point.
\VS{45}Si ton pied est pour toi une occasion de chute, coupe-le ; mieux vaut pour toi entrer boiteux dans la vie, que d'avoir les deux pieds, et d’être jeté dans la géhenne, dans le feu qui ne s'éteint point ;
\VS{46}là où leur ver ne meurt point, et où le feu ne s'éteint point.
\VS{47}Si ton œil est pour toi une occasion de chute, arrache-le ; mieux vaut pour toi entrer dans le Royaume de Dieu n'ayant qu'un œil, que d'avoir les deux yeux, et d’être jeté dans le feu de la géhenne,
\VS{48}où leur ver ne meurt point, et où le feu ne s'éteint point.
\VS{49}Car chacun sera salé de feu ; et toute offrande sera salée de sel.
\VS{50}Le sel est une bonne chose ; mais si le sel devient sans saveur, avec quoi lui rendra-t-on sa saveur ?
\VS{51}Ayez du sel en vous-mêmes, et soyez en paix les uns avec les autres.
\TextTitle{[Enseignement de Jésus sur le mariage et le divorce]
\\(Mt. 5:31-32 ; 19:1-9 ; Lu. 16:18 ; Ro. 7:1-3 ; 1 Co. 7:10-16)}
\Chap{10}
\VerseOne{}Jésus, étant parti de là, se rendit dans le territoire de la Judée, au-delà du Jourdain. La foule s’assembla de nouveau auprès de lui, et selon sa coutume, il se mit à l’enseigner.
\VS{2}Alors les pharisiens vinrent à lui, et, pour l'éprouver, ils lui demandèrent s’il est permis à un homme de répudier sa femme.
\VS{3}Il répondit et leur dit : Qu'est-ce que Moïse vous a prescrit ?
\VS{4}Moïse, dirent-ils, a permis d'écrire une lettre de divorce, et de répudier ainsi sa femme{\FTNT{De. 24:1.}}.
\VS{5}Et Jésus leur répondit : C’est à cause de la dureté de votre cœur que Moïse vous a donné ce commandement.
\VS{6}Mais au commencement de la création, Dieu fit l’homme et la femme.
\VS{7}C'est pourquoi l'homme quittera son père et sa mère, et s'attachera à sa femme,
\VS{8}et les deux deviendront une seule chair. Ainsi, ils ne sont plus deux, mais ils sont une seule chair.
\VS{9}Que l'homme donc ne sépare pas ce que Dieu a mis ensemble sous un joug{\FTNT{Voir commentaire Mt. 19:6.}}.
\VS{10}Lorsqu’ils furent dans la maison, ses disciples l'interrogèrent encore là-dessus.
\VS{11}Il leur dit : Celui qui répudie sa femme et qui en épouse une autre, commet un adultère à son égard.
\VS{12}Pareillement si la femme répudie son mari, et se marie à un autre, elle commet un adultère.
\TextTitle{[Jésus bénit les petits enfants]
\\(Mt. 19:13-15 ; Lu. 18:15-17)}
\VS{13}On lui amena de petits enfants afin qu'il les touche. Mais les disciples reprirent ceux qui les amenaient.
\VS{14}Jésus, voyant cela, fut indigné, et leur dit : Laissez venir à moi les petits enfants et ne les en empêchez point, car le Royaume de Dieu appartient à ceux qui leur ressemblent.
\VS{15}Je vous le dis en vérité, quiconque ne recevra pas comme un petit enfant le Royaume de Dieu, il n'y entrera point.
\VS{16}Après les avoir pris dans ses bras, il les bénit, en leur imposant les mains.
\TextTitle{[Le jeune homme riche]
\\(Mt. 19:16-30 ; Lu. 18:18-30 ; Lu. 10:25-37)}
\VS{17}Comme Jésus se mettait en chemin, un homme accourut, et se jetant à genoux devant lui : Bon Maître, lui demanda-t-il, que dois-je faire pour hériter la vie éternelle ?
\VS{18}Jésus lui répondit : Pourquoi m'appelles-tu bon ? Il n'y a de bon que Dieu seul{\FTNT{La même histoire est racontée en Lu. 18:18 qui précise que c’était un chef qui avait interrogé Jésus. La réponse du Seigneur est ironique. Jésus aurait aussi pu lui poser la question comme suit : «~Puisque tu penses que je ne suis qu’un simple homme, pourquoi m’appelles-tu bon ?~».}}.
\VS{19}Tu connais les commandements : Ne commets point d’adultère ; ne tue point ; ne dérobe point ; ne dis point de faux témoignage ; ne fais aucun tort à personne ; honore ton père et ta mère.
\VS{20}Il lui répondit : Maître, j'ai observé toutes ces choses dès ma jeunesse.
\VS{21}Jésus, l’ayant regardé, l'aima, et lui dit : Il te manque une chose : Va, et vends tout ce que tu as, et donne-le aux pauvres, et tu auras un trésor dans le ciel. Puis, viens, et suis-moi en te chargeant de ta croix.
\VS{22}Mais, affligé de cette parole, il s'en alla tout triste, parce qu'il avait de grands biens.
\TextTitle{[Tout est possible à Dieu]}
\VS{23}Alors Jésus, ayant regardé autour de lui, dit à ses disciples : Qu’il est difficile à ceux qui ont des richesses d’entrer dans le Royaume de Dieu.
\VS{24}Ses disciples furent étonnés de ces paroles ; mais Jésus reprenant la parole, leur dit : Mes enfants, qu'il est difficile à ceux qui se confient dans les richesses d'entrer dans le Royaume de Dieu !
\VS{25}Il est plus facile à un chameau de passer par le trou d'une aiguille{\FTNT{Voir commentaire Mt. 19:24.}}, qu’à un riche d’entrer dans le Royaume de Dieu.
\VS{26}Les disciples furent encore plus étonnés, et ils se dire les uns les autres : Et qui peut être sauvé ?
\VS{27}Mais Jésus les ayant regardés, leur dit : Cela est impossible aux hommes, mais non à Dieu ; car tout est possible à Dieu.
\TextTitle{[La fidélité à Jésus-Christ sera récompensée]}
\VS{28}Alors Pierre se mit à lui dire : Voici, nous avons tout quitté et nous t'avons suivi.
\VS{29}Et Jésus répondit, disant : Je vous le dis en vérité, il n’est personne qui, ayant quitté pour l'amour de moi et de l’Evangile, sa maison, ou ses frères, ou ses sœurs, ou son père, ou sa mère, ou sa femme, ou ses enfants, ou ses terres,
\VS{30}ne reçoive au centuple, présentement dans ce temps-ci, des maisons, des frères, des sœurs, des mères, des enfants, et des terres, avec des persécutions ; et dans le siècle à venir, la vie éternelle.
\VS{31}Plusieurs des premiers seront les derniers ; et plusieurs des derniers seront les premiers.
\TextTitle{[Jésus annonce sa mort et sa résurrection]
\\(Mt. 20:17-19 ; Lu. 18:31-34)}
\VS{32}Ils étaient en chemin, pour monter à Jérusalem, et Jésus allait devant eux. Les disciples étaient troublés, et le suivaient avec crainte. Et Jésus prit de nouveau à l'écart les douze, et commença à leur déclarer ce qui devait lui arriver,
\VS{33}disant : Voici, nous montons à Jérusalem, et le Fils de l'homme sera livré aux principaux sacrificateurs et aux scribes. Ils le condamneront à mort, et le livreront aux gentils
\VS{34}qui se moqueront de lui, le battront de verges, cracheront sur lui, et le feront mourir ; et il ressuscitera trois jours après.
\TextTitle{[Jésus répond à la question de Jacques et Jean]}
\VS{35}Alors Jacques et Jean, fils de Zébédée, s’approchèrent de Jésus et lui dirent : Maître, nous voudrions que tu fasses pour nous ce que nous te demanderons.
\VS{36}Il leur dit : Que voulez-vous que je fasse pour vous ?
\VS{37}Et ils lui dirent : Accorde-nous, lui dirent-ils, d’être assis l’un à ta droite et l’autre à ta gauche, quand tu seras dans ta gloire.
\VS{38}Jésus leur dit : Vous ne savez pas ce que vous demandez. Pouvez-vous boire la coupe que je dois boire, et être baptisés du baptême dont je dois être baptisé ?
\VS{39}Ils lui répondirent : Nous le pouvons. Et Jésus leur répondit : Il est vrai que vous boirez la coupe que je dois boire, et que vous serez baptisés du baptême dont je dois être baptisé ;
\VS{40}mais pour ce qui est d'être assis à ma droite et à ma gauche, ce n'est pas à moi de l’accorder ; mais cela ne sera donné qu’à ceux à qui cela est préparé.
\VS{41}Les dix autres, ayant entendu cela, commencèrent à s’indigner contre Jacques et Jean.
\VS{42}Jésus les appela et leur dit : Vous savez que ceux qu’on regarde comme les chefs des nations les dominent, et que les grands les asservissent.
\VS{43}Il n'en sera pas de même parmi vous. Mais quiconque veut être le plus grand parmi vous, qu’il soit votre serviteur,
\VS{44}et quiconque veut être le premier parmi vous, qu’il soit l’esclave de tous.
\VS{45}Car le Fils de l'homme est venu, non pour être servi, mais pour servir et donner sa vie en rançon pour plusieurs.
\TextTitle{[Jésus guérit l'aveugle Bartimée]
\\(Mt. 20:29-34 ; Lu. 18:35-43)}
\VS{46}Ils arrivèrent à Jéricho. Et lorsque Jésus en sortit, avec ses disciples et une grande foule, un aveugle, appelé Bartimée, c'est-à-dire le fils de Timée, était assis au bord du chemin et mendiait.
\VS{47}Il entendit que c'était Jésus de Nazareth, et il se mit à crier et à dire : Jésus, Fils de David, aie pitié de moi !
\VS{48}Plusieurs le reprenaient pour le faire taire ; mais il criait beaucoup plus fort : Fils de David, aie pitié de moi !
\VS{49}Jésus s’arrêta, et dit : Appelez-le. Ils appelèrent l’aveugle en lui disant : Prends courage, lève-toi, il t'appelle.
\VS{50}L’aveugle jeta son manteau, il se leva et vint vers Jésus.
\VS{51}Jésus, prenant la parole, lui dit : Que veux-tu que je te fasse ? Et l'aveugle lui dit : Maître, que je recouvre la vue.
\VS{52}Et Jésus lui dit : Va, ta foi t'a sauvé.
\VS{53}Et aussitôt il recouvra la vue, et suivit Jésus dans le chemin.
\TextTitle{[Entrée de Jésus à Jérusalem]
\\(Mt. 21:1-11 ; Lu. 19:28-40 ; Jn. 12:12-19 ; Za. 9:9)}
\Chap{11}
\VerseOne{}Lorsqu’ils approchaient de Jérusalem, et qu’ils furent près de Bethphagé et de Béthanie, vers le Mont des oliviers, Jésus envoya deux de ses disciples,
\VS{2}en leur disant : Allez au village qui est devant vous. Dès que vous y serez entrés, vous trouverez un ânon attaché, sur lequel aucun homme ne s’est encore assis. Détachez-le, et amenez-le.
\VS{3}Si quelqu'un vous dit : Pourquoi faites-vous cela ? Dites que le Seigneur en a besoin ; et à l’instant, il le laissera venir ici.
\VS{4}Ils partirent donc, et trouvèrent l'ânon qui était attaché dehors, près d’une porte, au contour du chemin, et ils le détachèrent.
\VS{5}Quelques-uns de ceux qui étaient là leur dirent : Pourquoi détachez-vous cet ânon ?
\VS{6}Ils leur répondirent comme Jésus l’avait ordonné ; et on les laissa faire.
\VS{7}Ils amenèrent donc l'ânon à Jésus, sur lequel ils jetèrent leurs vêtements, et Jésus s’assit dessus.
\VS{8}Beaucoup étendirent leurs vêtements sur le chemin, et d'autres des branches qu’ils coupèrent dans les champs.
\VS{9}Ceux qui allaient devant, et ceux qui suivaient, criaient en disant : Hosanna ! Béni soit celui qui vient au Nom du Seigneur !
\VS{10}Béni soit le règne de David notre père, le règne qui vient au Nom du Seigneur ! Hosanna dans les lieux très hauts !
\VS{11}Jésus entra ainsi à Jérusalem, dans le temple. Quand il eut tout considéré, il était déjà tard, il sortit pour aller à Béthanie avec les douze.
\TextTitle{[Le figuier sans fruit]
\\(Mt. 21:18-22)}
\VS{12}Le lendemain, après qu’ils furent sortis de Béthanie, Jésus eut faim.
\VS{13}Apercevant de loin un figuier qui avait des feuilles, il alla voir s'il y trouverait quelque chose ; et s’en étant approché, il ne trouva que des feuilles, car ce n'était pas la saison des figues.
\VS{14}Jésus prenant la parole dit au figuier : Que jamais personne ne mange de ton fruit ! Et ses disciples l'entendirent.
\TextTitle{[Jésus chasse les marchands du temple]
\\(Mt. 21:12-13 ; Lu. 19:45-46 ; Jn. 2:13-16)}
\VS{15}Ils arrivèrent donc à Jérusalem, et Jésus entra dans le temple. Il se mit à chasser dehors ceux qui vendaient, et ceux qui achetaient dans le temple, et il renversa les tables des changeurs, et les sièges de ceux qui vendaient des pigeons.
\VS{16}Il ne laissait personne porter aucun objet à travers le temple.
\VS{17}Et il les enseignait, en leur disant : N'est-il pas écrit : Ma maison sera appelée une maison de prière par toutes les nations ? Mais vous, vous en avez fait une caverne de voleurs{\FTNT{Jé. 7:11.}}.
\VS{18}Les scribes et les principaux sacrificateurs l’ayant entendu, cherchèrent les moyens de le faire périr ; car ils le craignaient, parce que toute la foule était frappée de sa doctrine.
\VS{19}Le soir étant venu, Jésus sortit de la ville.
\TextTitle{[La prière de la foi]
\\(1 Jn. 5:14-15)}
\VS{20}Le matin, en passant, les disciples virent le figuier séché jusqu’aux racines.
\VS{21}Pierre s'étant souvenu de ce qui s'était passé, dit à Jésus : Maître, voici, le figuier que tu as maudit a séché.
\VS{22}Jésus répondant, leur dit : Ayez foi en Dieu.
\VS{23}Je vous le dis en vérité, si quelqu’un dit à cette montagne : Ôte-toi de là et jette-toi dans la mer, et s’il ne doute point en son cœur, mais croit que ce qu’il a dit arrive, il le verra s’accomplir.
\VS{24}C'est pourquoi je vous dis : Tout ce que vous demanderez en priant, croyez que vous l’avez reçu, et vous le verrez s’accomplir.
\TextTitle{[Le pardon]}
\VS{25}Mais quand vous vous présenterez pour faire votre prière, si vous avez quelque chose contre quelqu'un, pardonnez-lui, afin que votre Père qui est dans les cieux vous pardonne aussi vos fautes.
\VS{26}Mais si vous ne pardonnez pas, votre Père qui est dans les cieux ne vous pardonnera point aussi vos fautes.
\TextTitle{[L'autorité de Jésus-Christ mise en doute]
\\(Mt. 21:23-27 ; Lu. 20:1-8)}
\VS{27}Ils se rendirent de nouveau à Jérusalem, et pendant que Jésus marchait dans le temple, les principaux sacrificateurs, les scribes et les anciens vinrent à lui,
\VS{28}et lui dirent : Par quelle autorité fais-tu ces choses, et qui t'a donné cette autorité pour faire les choses que tu fais ?
\VS{29}Jésus leur répondit : Je vous demanderai aussi une chose, et répondez-moi ; puis je vous dirai par quelle autorité je fais ces choses.
\VS{30}Le baptême de Jean venait-il du ciel, ou des hommes ? Répondez-moi.
\VS{31}Et ils raisonnaient entre eux, disant : Si nous disons, du ciel : Il nous dira : Pourquoi donc n’avez-vous pas cru en lui ?
\VS{32}Et si nous disons : Des hommes, nous avons à craindre le peuple, car tous croyaient que Jean était un vrai prophète.
\VS{33}Alors ils répondirent à Jésus : Nous ne savons pas. Et Jésus leur dit : Moi non plus je ne vous dirai pas par quelle autorité je fais ces choses.
\TextTitle{[Parabole des vignerons]
\\(Mt. 21:33-46 ; Lu. 20:9-18 ; Es. 5:1-7)}
\Chap{12}
\VerseOne{}Jésus se mit à leur parler en paraboles : Quelqu'un, dit-il, planta une vigne, et l'environna d'une haie, creusa un pressoir, et bâtit une tour ; puis il la loua à des vignerons, et quitta le pays.
\VS{2}Au temps de la récolte, il envoya un serviteur vers les vignerons, pour recevoir d'eux le fruit de la vigne.
\VS{3}S’étant saisis de lui, ils le battirent, et le renvoyèrent à vide.
\VS{4}Il envoya de nouveau un autre serviteur vers eux. Ils lui jetèrent des pierres, le frappèrent à la tête, et le renvoyèrent après l'avoir outragé.
\VS{5}Il en envoya de nouveau un troisième, qu’ils tuèrent ; et plusieurs autres, et ils battirent les uns, et tuèrent les autres.
\VS{6}Il avait encore un fils, son bien-aimé, il le leur envoya le dernier, disant : Ils auront du respect pour mon fils.
\VS{7}Mais ces vignerons dirent entre eux : Voici l'héritier, venez, tuons-le, et l'héritage sera à nous.
\VS{8}Ils se saisirent de lui, le tuèrent, et le jetèrent hors de la vigne.
\VS{9}Que fera donc le maître de la vigne ? Il viendra, et fera périr ces vignerons, et donnera la vigne à d'autres.
\VS{10}N'avez-vous pas lu cette parole de l’Ecriture ? La pierre qu’ont rejetée ceux qui bâtissaient est devenue la principale de l’angle{\FTNT{Jésus-Christ, la pierre angulaire : Ps. 118:22-23 ; Es. 8:13-17.}} ?
\VS{11}Cela a été fait par le Seigneur, et c'est une chose merveilleuse à nos yeux.
\VS{12}Alors ils cherchaient à se saisir de lui, mais ils craignirent la foule. Ils avaient compris que c’était pour eux que Jésus avait dit cette parabole. Et ils le quittèrent, et s’en allèrent.
\TextTitle{[Le tribut dû à César]
\\(Mt. 22:15-22 ; Lu. 20:19-26)}
\VS{13}Mais ils envoyèrent quelques-uns des pharisiens et des hérodiens auprès de Jésus afin de le surprendre par ses discours.
\VS{14}Et ils vinrent lui dire : Maître, nous savons que tu es vrai, et que tu ne t’inquiètes de personne ; car tu ne regardes pas à l'apparence des hommes, et tu enseignes la voie de Dieu selon la vérité. Est-il permis ou non de payer le tribut à César ? Devons-nous payer, ou ne pas payer ?
\VS{15}Mais Jésus, connaissant leur hypocrisie, leur dit : Pourquoi me tentez-vous ? Apportez-moi un denier, afin que je le voie.
\VS{16}Ils lui en apportèrent un. Alors il leur dit : De qui porte-t-il l’image et l’inscription ? De César, lui répondirent-ils.
\VS{17}Alors Jésus leur dit : Rendez à César ce qui est à César, et à Dieu ce qui est à Dieu. Et ils furent remplis d’admiration pour lui.
\TextTitle{[Jésus répond aux sadducéens sur la résurrection]
\\(Mt. 22:23-33 ; Lu. 20:27-38)}
\VS{18}Alors les sadducéens, qui disent qu'il n'y a point de résurrection, vinrent à lui, et l'interrogèrent, disant :
\VS{19}Maître, voici ce que Moïse nous a prescrit : Si le frère de quelqu'un meurt, et laisse sa femme sans avoir d'enfants, son frère épousera sa veuve et suscitera une postérité à son frère.
\VS{20}Or, il y avait sept frères. Le premier prit une femme et mourut sans laisser d'enfants.
\VS{21}Le deuxième prit la veuve pour femme, et mourut sans laisser de postérité. Il en fut de même du troisième,
\VS{22}et les sept l’épousèrent sans laisser de postérité. Après eux tous, la femme mourut aussi.
\VS{23}A la résurrection, quand ils seront ressuscités, duquel d’entre eux sera-t-elle la femme ? Car les sept l’ont eue pour femme.
\VS{24}Jésus leur répondit : La raison pour laquelle vous tombez dans l'erreur, c'est que vous ne connaissez ni les Ecritures ni la puissance de Dieu.
\VS{25}Car, à la résurrection des morts, les hommes ne prendront point de femmes, ni les femmes de maris, mais ils seront comme les anges dans les cieux.
\VS{26}Et quant aux morts, pour vous montrer qu'ils ressuscitent, n'avez-vous point lu dans le livre de Moïse, comment Dieu lui parla dans le buisson, en disant : Je suis le Dieu d'Abraham, et le Dieu d'Isaac, et le Dieu de Jacob ?
\VS{27}Or il n'est pas le Dieu des morts, mais le Dieu des vivants. Vous êtes donc dans une grande erreur.
\TextTitle{[Jésus répond aux pharisiens concernant le plus grand commandement de la loi]
\\(Mt. 22:34-40 ; Lu. 10:25-28)}
\VS{28}Un des scribes, qui les avait entendus discuter, voyant qu'il leur avait bien répondu, s'approcha de lui, et lui demanda : Quel est le premier de tous les commandements ?
\VS{29}Jésus lui répondit : Le premier de tous les commandements est : Ecoute Israël{\FTNT{Ecoute Israël~: Jésus se réfère ici à De. 6:4~: «~Ecoute, Israël ! Yahweh, notre Dieu Yahweh est Un~». Le Shema Israël est le noyau central de la prière que le Juif adulte doit lire matin et soir. C’est la confession de foi juive. Jacob est le premier à l’avoir enseignée à ses enfants dans Ge. 49:1-2.}}, le Seigneur notre Dieu, le Seigneur est Un{\FTNT{Jésus-Christ, notre Seigneur et notre modèle, a confirmé le Shema Israël qui déclare haut et fort que Dieu est Un et non trois en un. Le scribe, homme versé dans les Ecritures, était satisfait de la réponse de Jésus car il croyait aussi en un seul Dieu. Or le monothéisme est le fondement de la foi juive et des premiers chrétiens.}}.
\VS{30}Tu aimeras le Seigneur ton Dieu de tout ton cœur, de toute ton âme, de toute ta pensée, et de toute ta force. C'est là le premier commandement.
\VS{31}Voici le second, qui est semblable au premier : Tu aimeras ton prochain comme toi-même. Il n'y a pas d'autre commandement plus grand que ceux-là.
\VS{32}Et le scribe lui dit : Maître, tu as bien dit selon la vérité, qu'il y a un seul Dieu, et qu'il n'y en a point d'autre que lui ;
\VS{33}et que de l'aimer de tout son cœur, de toute son intelligence, de toute son âme, et de toute sa force ; et d'aimer son prochain comme soi-même, c'est plus que tous les holocaustes et les sacrifices.
\VS{34}Jésus voyant que ce scribe avait répondu prudemment, lui dit : Tu n'es pas loin du Royaume de Dieu. Et personne n'osait plus l'interroger.
\TextTitle{[Jésus dénonce les scribes]
\\(Mt. 22:41-46 ; Lu. 20:39-44)}
\VS{35}Comme Jésus enseignait dans le temple, il prit la parole et dit : Comment les scribes disent-ils que le Christ est le Fils de David ?
\VS{36}Car David lui-même a dit par le Saint-Esprit : Le Seigneur a dit à mon Seigneur : Assieds-toi à ma droite, jusqu'a ce que je fasse de tes ennemis ton marchepied{\FTNT{Ps. 110:1.}}.
\VS{37}David lui-même l'appelle son Seigneur, comment est-il son fils ? Et une grande foule l’écoutait avec plaisir.
\VS{38}Il leur disait dans son enseignement : Gardez-vous des scribes qui prennent plaisir à se promener en robes longues, et qui aiment les salutations dans les places publiques,
\VS{39}qui recherchent les premiers sièges dans les synagogues, et les premières places dans les festins ;
\VS{40}qui dévorent entièrement les maisons des veuves, et qui font pour l’apparence de longues prières. Ils seront jugés plus sévèrement.
\TextTitle{[L'offrande de la pauvre veuve]
\\(Lu. 21:1-4)}
\VS{41}Jésus, s’étant assis vis-à-vis du tronc, regardait comment la foule y mettait de l'argent. Plusieurs riches y mettaient beaucoup.
\VS{42}Et une pauvre veuve vint, elle y mit deux petites pièces, faisant le quart d’un sou.
\VS{43}Et Jésus, ayant appelé ses disciples, leur dit : Je vous le dis en vérité, cette pauvre veuve a plus mis dans le tronc que tous ceux qui y ont mis.
\VS{44}Car tous ont mis de leur superflu ; mais elle a mis de son nécessaire, tout ce qu'elle possédait, tout ce qu’elle avait pour vivre.
\TextTitle{[Les deux questions des disciples et la prophétie sur la destruction du temple de Jérusalem]
\\Mt. 24:3 ; Lu. 21:7)}
\Chap{13}
\VerseOne{}Lorsque Jésus sortit du temple, un de ses disciples lui dit : Maître, regarde quelles pierres et quelles constructions !
\VS{2}Jésus lui répondit : Vois-tu ces grands bâtiments ? Il ne restera pas pierre sur pierre qui ne soit pas démolie.
\VS{3}Il s’assit sur le Mont des oliviers, en face du temple. Et Pierre, Jacques, Jean et André, lui posèrent en particulier cette question :
\VS{4}Dis-nous quand cela arrivera-t-il, et à quel signe connaîtra-t-on que ces choses vont s'accomplir ?
\TextTitle{[Les temps de la fin]}
\VS{5}Jésus se mit à leur dire : Prenez garde que personne ne vous séduise.
\VS{6}Car plusieurs viendront en mon Nom, disant : C'est moi qui suis le Christ. Et ils séduiront beaucoup de gens.
\VS{7}Quand vous entendrez parler de guerres et des bruits de guerres, ne soyez point troublés ; parce qu'il faut que ces choses arrivent ; mais ce ne sera pas encore la fin.
\VS{8}Car une nation s'élèvera contre une autre nation, et un royaume contre un autre royaume ; et il y aura des tremblements de terre en divers lieux, et il y aura des famines et des troubles. Ces choses seront le commencement des douleurs.
\VS{9}Mais prenez garde à vous-mêmes. Car ils vous livreront aux tribunaux, et aux synagogues, vous serez battus de verges ; vous serez présentés devant les gouverneurs et devant les rois, à cause de moi, pour leur servir de témoignage.
\VS{10}Mais il faut premièrement que l'Evangile soit prêché à toutes les nations.
\VS{11}Et quand ils vous emmèneront pour vous livrer, ne vous inquiétez pas d’avance de ce que vous aurez à dire, mais dites ce qui vous sera donné à l’instant ; car ce n’est pas vous qui parlerez, mais le Saint-Esprit.
\VS{12}Le frère livrera son frère à la mort, et le père son enfant ; et les enfants se soulèveront contre leurs parents, et les feront mourir.
\VS{13}Vous serez haïs de tous à cause de mon Nom ; mais celui qui persévérera jusqu’à la fin, sera sauvé.
\TextTitle{[L'abomination de la désolation]
\\(Mt. 24:15-28 ; Ps. 2.5 ; Lu. 21:20-24 ; Ap. 7:14)}
\VS{14}Lorsque vous verrez l'abomination qui cause la désolation{\FTNT{Voir commentaire Mt. 24:15.}} qui a été prédite par Daniel, le prophète, établie là où elle ne doit pas être, que celui qui lit ce prophète fasse attention ! Alors que ceux qui seront en Judée fuient dans les montagnes.
\VS{15}Que celui qui sera sur le toit, ne descende pas dans la maison, et n’entre pas pour emporter quoi que ce soit de sa maison,
\VS{16}et que celui qui sera dans les champs, ne retourne pas en arrière pour emporter son manteau.
\VS{17}Malheur aux femmes qui seront enceintes, et à celles qui allaiteront en ces jours-là.
\VS{18}Priez Dieu que votre fuite n'arrive pas en hiver.
\VS{19}Car la détresse, en ces jours, sera telle qu’il n’y en a point eu de semblable depuis le commencement du monde que Dieu a créé jusqu’à présent, et qu’il n’y en aura jamais.
\VS{20}Et si le Seigneur n’avait abrégé ces jours, personne ne serait sauvé ; mais il les a abrégés, à cause des élus qu'il a choisis.
\VS{21}Si quelqu'un vous dit : Voici, le Christ est ici ; ou voici, il est là, ne le croyez point.
\VS{22}Car il s'élèvera des faux christs et des faux prophètes, qui feront des prodiges et des miracles, pour séduire même les élus s'il était possible.
\VS{23}Soyez sur vos gardes ; voici, je vous ai tout annoncé d’avance.
\TextTitle{[Retour du Messie sur la terre]
\\(Mt. 24:29-31 ; Lu. 21:25-28)}
\VS{24}Mais dans ces jours, après cette détresse, le soleil s’obscurcira, et la lune ne donnera plus sa clarté ;
\VS{25}les étoiles du ciel tomberont, et les puissances qui sont dans les cieux seront ébranlées.
\VS{26}Alors ils verront le Fils de l'homme venant sur les nuées, avec une grande puissance et une grande gloire.
\VS{27}Alors il enverra ses anges, et il rassemblera ses élus des quatre vents, de l’extrémité de la terre jusqu’à l’extrémité du ciel.
\TextTitle{[Parabole du figuier]
\\(Mt. 24:32-35 ; Lu. 21:29-33)}
\VS{28}Instruisez-vous par une comparaison tirée du figuier. Dès que ses branches deviennent tendres, et que les feuilles poussent, vous savez que l'été est proche.
\VS{29}Ainsi, quand vous verrez ces choses arriver, sachez que le Fils de l’homme est proche, à la porte.
\VS{30}Je vous le dis en vérité, cette génération ne passera point, que toutes ces choses ne soient arrivées.
\VS{31}Le ciel et la terre passeront, mais mes paroles ne passeront point.
\TextTitle{[Exhortation de Jésus sur la vigilance]
\\(Mt. 24:36-51 ; Lu. 21:34-38)}
\VS{32}Pour ce qui est du jour ou de l’heure, personne ne le sait, ni les anges dans le ciel, ni le Fils{\FTNT{Comment expliquer l’ignorance du Fils quant à l’heure de son retour ? En prenant la condition d’un homme, Jésus s’est dépouillé de ses prérogatives divines et a connu des limites propres au genre humain (Ph. 2:7)~: la fatigue (Jn. 4:6 ; Mc. 4:38), la faim (Mc. 11:12), l’angoisse et la peur (Mc. 14:33), la mortalité physique… Ce dépouillement incluait le renoncement à l’omniscience, d’où le fait que Jésus-Christ homme ne connaissait pas le jour et l’heure de son retour.}}, mais mon Père seul.
\VS{33}Prenez garde, veillez et priez ; car vous ne savez quand ce temps viendra.
\VS{34}Il en sera comme d’un homme qui, partant pour un voyage, laisse sa maison, remet l’autorité à ses serviteurs, marquant à chacun sa tâche, et ordonne au portier de veiller.
\VS{35}Veillez donc, car vous ne savez quand le Maître de la maison viendra, ou le soir, ou à minuit, ou à l'heure où le coq chante, ou le matin ;
\VS{36}craignez qu’il ne vous trouve endormis, à son arrivée soudaine.
\VS{37}Ce que je vous dis, je le dis à tous : Veillez.
\TextTitle{[Le complot]
\\(Mt. 26:1-5 ; Lu. 22:1-2)}
\Chap{14}
\VerseOne{}La fête de Pâque et des pains sans levain devait avoir lieu deux jours après. Les principaux sacrificateurs et les scribes cherchaient les moyens de se saisir de Jésus par ruse, et de le faire mourir.
\VS{2}Mais ils disaient : Que ce ne soit pas pendant la fête, afin qu'il n’y ait pas de tumulte parmi le peuple.
\TextTitle{[Marie de Béthanie oint Jésus pour sa sépulture]
\\(Mt. 26:6-13 ; Jn. 12:1-8)}
\VS{3}Comme Jésus était à Béthanie, dans la maison de Simon le lépreux, et pendant qu’il était à table, une femme vint à lui avec un vase d'albâtre, rempli d'un parfum de nard pur et de grand prix ; et ayant rompu le vase, elle répandit le parfum sur la tête de Jésus.
\VS{4}Quelques-uns en furent indignés en eux-mêmes, et ils disaient : A quoi sert la perte de ce parfum ?
\VS{5}On aurait pu le vendre plus de trois cents deniers, et les donner aux pauvres. Ainsi ils murmuraient contre elle.
\VS{6}Mais Jésus dit : Laissez-la. Pourquoi lui faites-vous de la peine ? Elle a fait une bonne action à mon égard.
\VS{7}Parce que vous aurez toujours des pauvres avec vous, et vous pouvez leur faire du bien quand vous voulez ; mais vous ne m'aurez pas toujours.
\VS{8}Elle a fait ce qu’elle a pu ; elle a d’avance embaumé mon corps pour la sépulture.
\VS{9}Je vous le dis en vérité, partout où cet Evangile sera prêché, dans le monde entier, on racontera aussi en mémoire de cette femme ce qu’elle a fait.
\TextTitle{[La trahison de Judas]
\\(Mt. 26:14-16 ; Lu. 22:3-6)}
\VS{10}Alors Judas Iscariot, l'un des douze, alla vers les principaux sacrificateurs pour le livrer.
\VS{11}Après l’avoir entendu, ils furent dans la joie, et promirent de lui donner de l'argent. Et Judas cherchait une occasion favorable pour le livrer.
\TextTitle{[La dernière Pâque]
\\(Mt. 26:17-25 ; Lu. 22:7-20 ; Jn. 13:1-12)}
\VS{12}Le premier jour des pains sans levain, où l’on sacrifiait l'agneau de Pâque, ses disciples lui dirent : Où veux-tu que nous allions te préparer l'agneau de Pâque afin que tu manges ?
\VS{13}Et il envoya deux de ses disciples, et leur dit : Allez dans la ville, vous rencontrerez un homme portant une cruche d'eau, suivez-le.
\VS{14}Où qu’il entre, dites au maître de la maison : Le Maître dit : Où est le lieu où je mangerai l'agneau de Pâque avec mes disciples ?
\VS{15}Et il vous montrera une grande chambre haute, meublée et toute prête : C’est là que vous nous préparerez l'agneau de Pâque.
\VS{16}Ses disciples partirent, arrivèrent dans la ville, ils trouvèrent les choses comme il l’avait dit ; et ils apprêtèrent l'agneau de Pâque.
\VS{17}Le soir étant venu, il arriva avec les douze.
\VS{18}Pendant qu’ils étaient à table, et qu'ils mangeaient, Jésus leur dit : Je vous le dis en vérité, l'un de vous, qui mange avec moi, me trahira.
\VS{19}Ils commencèrent à s'attrister, et ils lui dirent l'un après l'autre : Est-ce moi ?
\VS{20}Mais il leur répondit : C'est l'un des douze qui trempe avec moi dans le plat.
\VS{21}Certes le Fils de l'homme s'en va, selon qu'il est écrit de lui. Mais malheur à l'homme par qui le Fils de l'homme est trahi ! Mieux vaudrait pour cet homme qu’il ne soit pas né.
\TextTitle{[Le repas de la Pâque]
\\(Mt. 26:26-29 ; Lu. 22:17-20 ; Jn. 13:12-30 ; 1 Co. 11:23-26)}
\VS{22}Pendant qu’ils mangeaient, Jésus prit du pain, et après avoir béni Dieu, il le rompit et le leur donna, et leur dit : Prenez, mangez, ceci est mon corps.
\VS{23}Il prit ensuite une coupe, et après avoir rendu grâces, il la leur donna, et ils en burent tous.
\VS{24}Et il leur dit : Ceci est mon sang{\FTNT{Nouvelle Alliance : Voir Jn. 19:30.}}, le sang de la nouvelle alliance, qui est répandu pour plusieurs.
\VS{25}Je vous le dis en vérité, je ne boirai plus du fruit de la vigne jusqu'au jour où j’en boirai du nouveau dans le Royaume de Dieu.
\TextTitle{[Jésus avertit Pierre de son triple reniement]
\\(Mt. 26:30-35 ; Lu. 22:31-34 ; Jn. 13:36-38)}
\VS{26}Après avoir chanté les cantiques{\FTNT{Cantiques~: Voir Mt. 26:30.}}, ils se rendirent à la montagne des oliviers.
\VS{27}Jésus leur dit : Vous serez tous cette nuit scandalisés en moi ; car il est écrit : Je frapperai le Berger, et les brebis seront dispersées{\FTNT{Za. 13:7.}}.
\VS{28}Mais, après que je serai ressuscité, je vous précéderai en Galilée.
\VS{29}Pierre lui dit : Quand même tous seraient scandalisés, je ne le serai pourtant pas moi.
\VS{30}Et Jésus lui dit : Je te le dis en vérité, qu'aujourd'hui, cette nuit même, avant que le coq chante deux fois, tu me renieras trois fois.
\VS{31}Mais Pierre disait encore plus fortement : Quand même il me faudrait mourir avec toi, je ne te renierai pas. Et tous lui dirent la même chose.
\TextTitle{[Jésus dans le jardin de Gethsémané]
\\(Mt. 26:36-46 ; Lu. 22:39-46 ; Jn. 18:1)}
\VS{32}Ils allèrent dans un lieu appelé Gethsémané, et Jésus dit à ses disciples : Asseyez-vous ici jusqu'à ce que j’aie prié.
\VS{33}Il prit avec lui Pierre, Jacques et Jean, et il commença à être effrayé et fort angoissé.
\VS{34}Il leur dit : Mon âme est saisie de tristesse jusqu’à la mort, restez ici, et veillez.
\TextTitle{[Première prière de Jésus]
\\(Mt. 26:36 ; Lu. 22:41-42)}
\VS{35}Puis s'en allant un peu plus en avant, il se jeta contre terre, et pria que s'il était possible, cette heure s’éloigne de lui.
\VS{36}Il disait : Abba, Père, toutes choses te sont possibles, éloigne de moi cette coupe ! Toutefois, non pas ce que je veux, mais ce que tu veux.
\VS{37}Puis il vint vers les disciples qu’il trouva endormis, et il dit à Pierre : Simon, tu dors ! Tu n’as pas pu veiller une heure !
\VS{38}Veillez et priez afin que vous ne tombiez pas en tentation, l'esprit est bien disposé, mais la chair est faible.
\TextTitle{[Deuxième prière]
\\(Mt. 26:42 ; Lu. 22:44)}
\VS{39}Il s’éloigna de nouveau, et fit la même prière, disant les mêmes paroles.
\VS{40}Il revint, et les trouva encore endormis, car leurs yeux étaient appesantis. Ils ne surent que lui répondre.
\TextTitle{[Troisième prière]
\\(Mt. 26:44)}
\VS{41}Il revint encore, pour la troisième fois, et leur dit : Dormez maintenant, et reposez-vous ! C’est assez ! L’heure est venue ; voici, le Fils de l'homme est livré entre les mains des méchants.
\VS{42}Levez-vous, allons ; voici, celui qui me trahit s'approche.
\TextTitle{[Jésus trahi, abandonné et arrêté]
\\(Mt. 26:47-56 ; Lu. 22:47-53 ; Jn. 18:2-11)}
\VS{43}Et aussitôt, comme il parlait encore, Judas, l'un des douze, vint, et avec lui une grande foule ayant des épées et des bâtons, envoyée par les principaux sacrificateurs, par les scribes et par les anciens.
\VS{44}Celui qui le trahissait leur avait donné ce signe : Celui que j'embrasserai, c’est lui ; saisissez-le, et emmenez-le sûrement.
\VS{45}Dès qu’il fut arrivé, il s'approcha aussitôt de Jésus, et lui dit : Rabbi, Rabbi ! Et il l’embrassa.
\VS{46}Alors ils mirent la main sur Jésus, et le saisirent.
\VS{47}Un de ceux qui étaient là présents, tirant son épée, frappa le serviteur du souverain sacrificateur et lui emporta l'oreille.
\VS{48}Alors Jésus prit la parole, et leur dit : Vous êtes venus comme après un brigand, avec des épées et des bâtons, pour m’arrêter.
\VS{49}J’étais tous les jours parmi vous, enseignant dans le temple, et vous ne m'avez point saisi ; mais tout ceci est arrivé afin que les Ecritures soient accomplies.
\VS{50}Alors tous ses disciples l'abandonnèrent et s'enfuirent.
\VS{51}Un jeune homme le suivait, n’ayant sur le corps qu’un drap. Et quelques jeunes gens le saisirent,
\VS{52}mais il abandonna son linceul, et se sauva tout nu.
\TextTitle{[Jésus devant Caïphe et le sanhédrin]
\\(Mt. 26:57-68 ; Jn. 18:12-14,19-24)}
\VS{53}Ils emmenèrent Jésus chez le souverain sacrificateur, où s'assemblèrent tous les principaux sacrificateurs, les anciens et les scribes.
\VS{54}Pierre le suivait de loin jusque dans la cour du souverain sacrificateur ; et il était assis avec les serviteurs, et se chauffait près du feu.
\VS{55}Les principaux sacrificateurs et tout le sanhédrin cherchaient quelque témoignage contre Jésus pour le faire mourir, mais ils n'en trouvaient point.
\VS{56}Car plusieurs rendaient de faux témoignages contre lui, mais les témoignages ne s’accordaient pas.
\VS{57}Alors quelques-uns s'élevèrent, et portèrent de faux témoignages contre lui, disant :
\VS{58}Nous l’avons entendu dire : Je détruirai ce temple qui est fait de main d’homme, et en trois jours j'en rebâtirai un autre qui ne sera pas fait de main d’homme.
\VS{59}Même sur ce point-là leurs témoignages ne s’accordaient pas.
\VS{60}Alors le souverain sacrificateur se levant au milieu, interrogea Jésus, disant : Ne réponds-tu rien ? Qu’est-ce que ces gens déposent contre toi ?
\VS{61}Mais Jésus garda le silence, et ne répondit rien. Le souverain sacrificateur l'interrogea de nouveau, et lui dit : Es-tu le Christ, le Fils du Dieu béni ?
\VS{62}Jésus lui répondit : Je le suis. Et vous verrez le Fils de l'homme assis à la droite de la puissance de Dieu, et venant sur les nuées du ciel.
\VS{63}Alors le souverain sacrificateur déchira ses vêtements et dit : Qu'avons-nous encore besoin de témoins ?
\VS{64}Vous avez entendu le blasphème. Que vous en semble ? Alors tous le condamnèrent comme méritant la mort.
\VS{65}Et quelques-uns se mirent à cracher sur lui, à lui voiler le visage, et à lui donner des soufflets, en lui disant : Prophétise ! Et les serviteurs lui donnaient des coups avec leurs verges.
\TextTitle{[Triple reniement de Pierre]
\\(Mt. 26:69-75 ; Lu. 22:55-62 ; Jn. 18:15-18,25-27)}
\VS{66}Pendant que Pierre était en bas dans la cour, une des servantes du souverain sacrificateur vint.
\VS{67}Apercevant Pierre qui se chauffait, elle le regarda en face, et lui dit : Toi aussi, tu étais avec Jésus de Nazareth.
\VS{68}Mais il le nia, disant : Je ne le connais pas, et je ne sais pas ce que tu dis ; puis il sortit dehors pour aller dans le vestibule. Et le coq chanta.
\VS{69}La servante l'ayant vu de nouveau, elle se mit à dire à ceux qui étaient là présents : Celui-ci est de ces gens-là. Et il le nia de nouveau.
\VS{70}Peu après, ceux qui étaient là présents, dirent à Pierre : Certainement tu es de ces gens-là, car tu es Galiléen, et ton langage s'y rapporte.
\VS{71}Alors il commença à faire des imprécations et à jurer : Je ne connais pas cet homme dont vous parlez.
\VS{72}Et le coq chanta pour la seconde fois. Et Pierre se souvint de la parole que Jésus lui avait dite : Avant que le coq chante deux fois, tu me renieras trois fois. Et étant sorti promptement, il pleura.
\TextTitle{[Jésus livré à Pilate]
\\(Mt. 27:1-2,11-15 ; Lu. 23:1-7,13-18 ; Jn. 18:28-38 ; 19:1-15)}
\Chap{15}
\VerseOne{}Dès le matin, les principaux sacrificateurs tinrent conseil avec les anciens et les scribes, et tout le sanhédrin. Après avoir lié Jésus, ils l'emmenèrent, et le livrèrent à Pilate.
\VS{2}Pilate l'interrogea : Es-tu le Roi des Juifs ? Et Jésus répondit : Tu le dis.
\VS{3}Les principaux sacrificateurs l'accusaient de plusieurs choses, mais il ne répondit rien.
\VS{4}Pilate l'interrogea de nouveau : Ne réponds-tu rien ? Vois de combien de choses ils t’accusent.
\VS{5}Mais Jésus ne donna plus aucune réponse, ce qui étonna Pilate.
\TextTitle{[Jésus ou Barabbas ?]
\\(Mt. 27:15-26 ; Lu. 23:17-25 ; Jn. 18:39)}
\VS{6}A chaque fête, il relâchait un prisonnier, celui que demandait la foule.
\VS{7}Il y avait en prison un nommé Barabbas avec ses complices pour une sédition, dans laquelle ils avaient commis un meurtre.
\VS{8}La foule se mit à demander à Pilate, avec de grands cris, ce qu’il avait coutume de leur accorder.
\VS{9}Pilate leur répondit : Voulez-vous que je vous relâche le Roi des Juifs ?
\VS{10}Car il savait bien que les principaux sacrificateurs l'avaient livré par envie.
\VS{11}Mais les principaux sacrificateurs excitèrent la foule, afin que Pilate leur relâche plutôt Barabbas.
\VS{12}Pilate reprenant la parole, leur dit encore : Que voulez-vous donc que je fasse de celui que vous appelez Roi des Juifs ?
\VS{13}Ils crièrent de nouveau : Crucifie-le !
\VS{14}Alors Pilate leur dit : Mais quel mal a-t-il fait ? Et ils crièrent encore plus fort : Crucifie-le !
\VS{15}Pilate, voulant satisfaire la foule, leur relâcha Barabbas ; et après avoir fait battre de verges Jésus, il le livra pour être crucifié.
\TextTitle{[Jésus couronné d'épines]
\\(Mt. 27:27-31 ; Jn. 19:16-17)}
\VS{16}Alors les soldats emmenèrent Jésus dans l’intérieur de la cour, c’est-à-dire dans le prétoire, et ils assemblèrent toute la cohorte.
\VS{17}Ils le revêtirent d'une robe de pourpre, et posèrent sur sa tête une couronne d'épines qu’ils avaient tressée.
\VS{18}Puis ils commencèrent à le saluer, en lui disant : Nous te saluons, Roi des Juifs !
\VS{19}Et ils lui frappaient la tête avec un roseau, et crachaient sur lui, et fléchissant les genoux, ils se prosternaient devant lui.
\VS{20}Et après s'être ainsi moqués de lui, ils le dépouillèrent de la robe de pourpre, lui remirent ses habits, et l'emmenèrent dehors pour le crucifier.
\VS{21}Et un certain homme de Cyrène, nommé Simon, père d’Alexandre et de Rufus, passant par là en revenant des champs, fut forcé à porter la croix de Jésus.
\VS{22}Et ils conduisirent Jésus au lieu appelé Golgotha{\FTNT{Golgotha~: Le Golgotha (crâne) était une colline située à l'extérieur de Jérusalem, sur laquelle les Romains crucifiaient les condamnés.}}, c'est-à-dire, le lieu du Crâne.
\VS{23}Ils lui donnèrent à boire du vin mêlé de myrrhe, mais il ne le prit pas.
\TextTitle{[Jésus crucifié]
\\(Mt. 27:33-56 ; Lu. 23:33-49 ; Jn. 19:17-37)}
\VS{24}Ils le crucifièrent, et se partagèrent ses vêtements, en tirant au sort pour savoir ce que chacun aurait.
\VS{25}C’était la troisième heure, quand ils le crucifièrent.
\VS{26}L’écriteau indiquant la cause de sa condamnation portait ces mots : Le Roi des Juifs.
\VS{27}Ils crucifièrent aussi avec lui deux brigands, l'un à sa droite, et l'autre à sa gauche.
\VS{28}Et ainsi fut accomplie l'Ecriture, qui dit : Et il a été mis au rang des malfaiteurs{\FTNT{Es. 53:12.}}.
\VS{29}Les passants l’injuriaient, et secouaient la tête, en disant : Hé ! Toi qui détruis le temple et qui le rebâtis en trois jours,
\VS{30}sauve-toi toi-même, et descends de la croix !
\VS{31}Les principaux sacrificateurs aussi avec les scribes se moquaient entre eux, et disaient : Il a sauvé les autres, et il ne peut se sauver lui-même.
\VS{32}Que le Christ, le Roi d’Israël descende maintenant de la croix, afin que nous le voyions et que nous croyions ! Ceux qui étaient crucifiés avec lui l’insultaient aussi.
\VS{33}La sixième heure étant venue, il y eut des ténèbres sur toute la terre jusqu'à la neuvième heure.
\VS{34}Et à la neuvième heure, Jésus s’écria d’une voix forte : Eloï, Eloï, lama sabachthani ? C’est-à-dire : Mon Dieu ! Mon Dieu ! Pourquoi m'as-tu abandonné ?
\VS{35}Quelques-uns de ceux qui étaient là présents, l’ayant entendu, dirent : Voici, il appelle Elie.
\VS{36}Et l’un d’eux courut remplir une éponge de vinaigre{\FTNT{Le vinaigre~: Voir Mt. 27:34.}}, et l'ayant fixée au bout d'un roseau, il lui donna à boire, en disant : Laissez, voyons si Elie viendra le descendre de la croix.
\VS{37}Mais Jésus, ayant poussé un grand cri, expira.
\VS{38}Et le voile du temple se déchira en deux, depuis le haut jusqu'en bas{\FTNT{Hé 10:19-20.}}.
\TextTitle{[Fin de la Première Alliance]
\\(Hé. 9:16-18)}
\VS{39}Le centenier, qui était en face de Jésus, voyant qu'il avait expiré en criant de la sorte, dit : Certainement cet homme était Fils de Dieu.
\VS{40}Il y avait là aussi des femmes qui regardaient de loin. Parmi elles étaient Marie de Madgala, Marie mère de Jacques le mineur et de Joses, et Salomé,
\VS{41}qui le suivaient et le servaient lorsqu'il était en Galilée, et plusieurs autres qui étaient montées avec lui à Jérusalem.
\TextTitle{[Jésus enseveli]}
\VS{42}Le soir étant venu, comme c'était la préparation, c’est-à-dire le sabbat,
\VS{43}arriva Joseph d'Arimathée, conseiller de distinction, qui attendait aussi le Royaume de Dieu. Il osa se rendre vers Pilate pour demander le corps de Jésus.
\VS{44}Pilate s'étonna qu'il soit mort si tôt ; il fit venir le centenier, et lui demanda s'il était mort depuis longtemps.
\VS{45}S’en étant assuré par le centenier, il donna le corps à Joseph.
\VS{46}Et Joseph ayant acheté un linceul, descendit Jésus de la croix, et l'enveloppa du linceul, et le déposa dans un sépulcre taillé dans le roc. Puis il roula une pierre sur l'entrée du sépulcre.
\VS{47}Marie de Magdala, et Marie mère de Joses regardaient où on le mettait.
\TextTitle{[Jésus, ressuscité, apparaît à plusieurs disciples]
\\(Mt. 28:1-15 ; Lu. 24:1-49 ; Jn. 20:1-23)}
\Chap{16}
\VerseOne{}Lorsque le sabbat fut passé, Marie de Magdala, Marie mère de Jacques, et Salomé, achetèrent des aromates pour embaumer Jésus.
\VS{2}Le premier jour de la semaine, de grand matin, elles se rendirent au sépulcre, comme le soleil venait de se lever.
\VS{3}Elles disaient entre elles : Qui nous roulera la pierre de l'entrée du sépulcre ?
\VS{4}Et levant les yeux, elles virent que la pierre, qui était très grande, avait été roulée.
\VS{5}Elles entrèrent dans le sépulcre, virent un jeune homme assis à droite, vêtu d'une robe blanche, et elles furent épouvantées.
\VS{6}Mais il leur dit : Ne vous épouvantez pas. Vous cherchez Jésus de Nazareth qui a été crucifié. Il est ressuscité, il n'est point ici ; voici le lieu où on l'avait mis.
\VS{7}Mais allez, et dites à ses disciples, et à Pierre, qu'il vous précède en Galilée. C’est là que vous le verrez, comme il vous l'a dit.
\VS{8}Elles partirent aussitôt et s'enfuirent du sépulcre. La peur et le trouble les avaient saisies ; et elles ne dirent rien à personne, à cause de la peur.
\VS{9}Jésus étant ressuscité, le matin du premier jour de la semaine{\FTNT{Jésus a-t-il été crucifié un vendredi ? Si c’est le cas, comment a-t-il pu séjourner trois jours dans le tombeau s’il est ressuscité le dimanche matin comme l’enseigne la tradition catholique et la majorité des églises protestantes et évangéliques ? Tout d’abord il convient de signaler que selon Ge. 1, le jour commence au coucher du soleil, aux environs de dix-huit heures, et s’achève le lendemain au coucher du soleil. Chez les Romains, le jour commence à minuit et se termine le lendemain à minuit. C’est de cette manière que l’évangile de Jean compte les heures. Dans les autres évangiles, les journées commencent avec le lever du soleil. Jésus a été crucifié à «~la troisième heure~» (Mc. 15:25), ce qui correspond à neuf heures du matin. Ensuite, les évangiles nous apprennent qu’il y a eu des ténèbres sur la terre de la sixième à la neuvième heure, donc de midi à quinze heures (Mt. 27:45-46 ; Mc. 15:33-34 ; Lu. 23:44). Jésus est donc mort avant dix-huit heures. Ainsi, il est évident qu’il n’a pas pu passer toute la journée du vendredi au tombeau. Les Ecritures ne déclarent pas spécifiquement quel jour de la semaine Jésus a été crucifié. Les deux opinions dominantes sont vendredi et mercredi. D’autres font la synthèse des deux et acceptent le jeudi comme étant le jour de la crucifixion.  Jésus dit dans Mt. 12:40~: «~Car, de même que Jonas fut trois jours et trois nuits dans le ventre d’une baleine, de même le Fils de l'homme sera trois jours et trois nuits dans le sein de la terre~». Ceux qui défendent la crucifixion un vendredi disent qu’il est possible de compter de telle manière qu’on puisse effectivement considérer qu’il a été dans la tombe pendant trois jours. L’argument principal pour le vendredi se trouve dans Mc. 15:42 qui précise que Jésus a été crucifié la «~veille du sabbat~». S’il s’agit bien du sabbat hebdomadaire, c’est à dire le samedi, alors la crucifixion a bien eu lieu un vendredi. Un autre argument en faveur du vendredi se fonde sur des versets tels que Mt. 16:21 et Lu. 9:22 où Jésus enseigne qu’il ressuscitera le troisième jour, ce qui suppose qu’il ne restera pas trois jours et trois nuits entiers dans la tombe. Plusieurs traducteurs utilisent l’expression «~le troisième jour~», mais pas tous. Cependant, aucun d’eux ne conteste la manière de traduire ces versets. Dans Mc. 8:31, il est bien dit que Jésus sera ressuscité «~après~» trois jours.  Le débat sur le jeudi se construit sur celui du vendredi en concluant qu’il y a trop d’événements, selon ses défenseurs, qui se passent entre l’ensevelissement du Christ et dimanche matin, pour que tout se soit déroulé entre vendredi et dimanche matin. Il faut signaler qu’il est particulièrement problématique que le seul jour plein entre vendredi et dimanche soit le samedi. Un jour de plus ou deux résolvent ce problème. L’hypothèse du mercredi avance qu’il y avait deux sabbats cette semaine-là. Après le premier sabbat (celui qui débute le soir de la crucifixion, Mc. 15:42 ; Lu. 23:52-54), les femmes sont allées acheter les aromates. Notez bien qu’elles les ont achetées après le sabbat (Mc. 16:1). Dans cette hypothèse du mercredi, ce premier sabbat est la Pâque (cf. Lé. 16:29-31 ; Lé. 23:24-32,39) car les jours très saints sont aussi appelés sabbats. Le second sabbat de cette semaine était le sabbat hebdomadaire classique, le samedi. Notez que dans Lu. 23:56, les femmes qui avaient acheté les aromates après le premier sabbat s’en retournèrent, préparèrent les aromates puis «~se reposèrent durant le sabbat~» (Lu. 23:56). On ne peut pas imaginer qu’elles ont acheté les aromates après le sabbat et qu’elles les ont préparées avant le sabbat que s’il y a eu deux sabbats cette semaine-là. Avec l’hypothèse des deux sabbats, si le Messie a été crucifié un jeudi, alors le jour très saint (la Pâque) aurait débuté au coucher du soleil le jeudi et pris fin le vendredi au coucher du soleil – juste au début du sabbat hebdomadaire – le samedi. Acheter les aromates après le premier sabbat signifierait alors en faire l’acquisition le samedi, en violation des lois du sabbat.  L’hypothèse du mercredi est la seule qui corrobore les récits bibliques des femmes et des aromates et confirme la prophétie du Seigneur en Mt. 12:40. Le Messie a été arrêté à Gethsémané le mardi soir selon le calendrier romain et le mercredi selon le calendrier hébraïque. Le premier sabbat était un jour très saint, celui de Pâque (Mt. 26 ; Mc. 14 ; Lu. 22), un jeudi selon le calendrier hébraïque. Les femmes achetèrent les aromates le vendredi et s’en retournèrent les préparer le jour même. Elles se sont reposées le samedi, qui était le sabbat hebdomadaire, et ont enfin apporté les aromates au tombeau tôt le dimanche matin. Jésus a été enseveli au moment du coucher du soleil le mercredi, ce qui est le début du jeudi selon le calendrier juif. En usant de ce calendrier, nous avons~: - le jeudi nuit (première nuit), jeudi jour (premier jour) - le vendredi nuit (deuxième nuit), vendredi jour (deuxième jour) - le samedi nuit (troisième nuit), samedi jour (troisième jour). Nous ne savons pas exactement à quelle heure Jésus est ressuscité, mais sous savons que ce fut avant le lever du soleil du dimanche. En effet, Jn. 20:1 nous apprend que Marie de Magdala vint au tombeau «~alors qu’il faisait encore sombre~». Ainsi, Jésus serait ressuscité juste après le coucher du soleil du samedi soir, ce qui correspond au premier jour de la semaine pour les juifs.}} apparut d’abord à Marie de Madgala, de laquelle il avait chassé sept démons.
\VS{10}Elle alla l’annoncer à ceux qui avaient été avec lui, et qui étaient dans le deuil et pleuraient.
\VS{11}Mais quand ils entendirent qu'il était vivant, et qu'elle l'avait vu, ils ne la crurent point.
\VS{12}Après cela, il se montra sous une autre forme à deux d'entre eux, qui étaient en chemin pour aller à la campagne.
\VS{13}Ils revinrent l’annoncer aux autres, mais ils ne les crurent pas non plus.
\VS{14}Enfin, il se montra aux onze, qui étaient assis ensemble, et il leur reprocha leur incrédulité et leur dureté de cœur, parce ce qu'ils n'avaient pas cru ceux qui l'avaient vu ressuscité.
\TextTitle{[Nouvelle mission aux onze apôtres]
\\(Mt. 28:16-20 ; Lu. 24:46-48 ; Jn. 17:18 ; 20:21 ; Ac. 1:8)}
\VS{15}Puis il leur dit : Allez par tout le monde, et prêchez l'Evangile à toute créature.
\VS{16}Celui qui croira et qui sera baptisé, sera sauvé ; mais celui qui ne croira pas sera condamné.
\VS{17}Voici les miracles qui accompagneront ceux qui auront cru : Ils chasseront les démons en mon Nom ; ils parleront de nouvelles langues ;
\VS{18}ils saisiront les serpents avec la main, et s’ils boivent quelque breuvage mortel, il ne leur fera point de mal ; ils imposeront les mains aux malades, et les malades seront guéris.
\TextTitle{[Jésus enlevé au ciel]
\\(Lu. 24:49-53 ; Ac. 1:9-11)}
\VS{19}Le Seigneur, après leur avoir parlé de la sorte, fut enlevé au ciel, et il s'assit à la droite de Dieu.
\VS{20}Et ils s’en allèrent prêcher partout. Le Seigneur travaillait avec eux, et confirmait la parole par les miracles qui l'accompagnaient.
\PPE{}
\end{multicols}

%\clearpage\ShortTitle{Luc}\BookTitle{Luc}\BFont
\noindent\hrulefill
{\footnotesize
\textit{
\bigskip
{\centering{}
\\Auteur : Luc
\\(Gr. : Loukas)
\\Signifie : Qui donne la lumière
\\Thème : Jésus le Fils de l'homme
\\Date de rédaction : Env. 60 ap. J.-C.\\}
}
%\bigskip
\textit{
\\D'origine grecque, Luc fut l'auteur de l'évangile éponyme et du livre des « Actes des apôtres ». Celui que Paul appelait le « médecin bien-aimé », et qui fut son compagnon d'œuvre, avait entrepris des investigations visant à narrer avec exactitude la vie terrestre de Jésus-Christ dont il était devenu le disciple, probablement à la suite d'une prédication de Paul. Adressés initialement à Théophile, Luc était loin de penser que ses écrits constitueraient avec le temps une véritable richesse pour l'Eglise et pour le monde.
%\bigskip
\\L'évangile de Luc présente l'humanité parfaite de Jésus, sa compassion et sa miséricorde à l'égard des plus faibles. Rédigé avec rigueur et soin, il retrace le parcours du Fils de l'homme, de sa naissance à son adolescence, puis de sa mort à sa résurrection, et enfin son ascension. Il souligne aussi sa vie de prière et son fardeau pour le salut de l'homme. Par ailleurs, il fait ressortir la manière dont les femmes ont assisté Jésus par leurs biens durant son ministère.
%\bigskip
\\Fruit de recherches minutieuses, le récit de Luc présente certaines similitudes avec ceux de Matthieu et Marc, mais il est le seul à relater la célèbre parabole du fils prodigue, profonde représentation de l'amour du Père.\bigskip
}
}
\par\nobreak\noindent\hrulefill
\begin{multicols}{2}
\Chap{1}
\TextTitle{Introduction}
\VerseOne{}Parce que plusieurs se sont appliqués à mettre par ordre un récit des évènements qui ont été pleinement certifiés parmi nous,
\VS{2}suivant ce que nous ont transmis ceux qui ont été des témoins oculaires dès le commencement et sont devenus des ministres de la parole,
\VS{3}Il m'a aussi semblé bon, après avoir examiné exactement toutes choses depuis le commencement jusqu'à la fin, très excellent Théophile, de te les mettre en ordre par écrit,
\VS{4}afin que tu connaisses la certitude des choses dont tu as été informé.
\TextTitle{Annonce de la naissance de Jean-Baptiste}
\VS{5}Au temps d'Hérode, roi de Judée, il y avait un sacrificateur nommé Zacharie, de la classe d'Abia ; et sa femme était d'entre les filles d'Aaron, et s'appelait Elisabeth.
\VS{6}Et ils étaient tous deux justes devant Dieu, marchant dans tous les commandements, et dans {toutes } les ordonnances du Seigneur, sans reproche.
\VS{7}Et ils n'avaient point d'enfants, parce qu'Elisabeth était stérile, et qu'ils étaient fort avancés en âge.
\VS{8}Or il arriva que comme Zacharie exerçait la sacrificature devant Dieu, selon le tour de sa classe, il fut appelé par le sort,
\VS{9}selon la coutume d'exercer le sacerdoce, à entrer dans le temple du Seigneur pour offrir le parfum.
\VS{10}Toute la multitude du peuple était dehors en prière, à l'heure du parfum.
\VS{11}Et l'ange du Seigneur lui apparut, et se tint debout à droite de l'autel des parfums.
\VS{12}Zacharie fut troublé quand il le vit, et il fut saisi de crainte.
\VS{13}Mais l'ange lui dit : Zacharie, ne crains point ; car ta prière est exaucée. Et Elisabeth, ta femme, t'enfantera un fils, et tu lui donneras le nom de Jean.
\VS{14}Et il sera pour toi le sujet d'une grande joie et d'allégresse, et plusieurs se réjouiront de sa naissance.
\VS{15}Car il sera grand devant le Seigneur. Et il ne boira ni vin, ni boisson forte et il sera rempli du Saint-Esprit dès le ventre de sa mère.
\VS{16}Et il ramènera plusieurs des enfants d'Israël au Seigneur, leur Dieu.
\VS{17}Car il marchera devant lui animé de l'esprit et de la puissance d'Elie, pour ramener les cœurs des pères vers les enfants\FTNT{Mal. 4:6.}, et les rebelles à la sagesse des justes, pour préparer au Seigneur un peuple bien disposé.
\VS{18}Alors Zacharie dit à l'ange : A quoi reconnaîtrai-je cela ? Car je suis vieux, et ma femme est fort âgée.
\VS{19}L'ange répondant lui dit : Je suis Gabriel, je me tiens devant Dieu, et j'ai été envoyé pour te parler, et pour t'annoncer cette bonne nouvelle.
\VS{20}Et voici, tu seras muet, et tu ne pourras point parler jusqu'au jour où ces choses arriveront, parce que tu n'as pas cru à mes paroles qui s'accompliront en leur temps.
\VS{21}Or le peuple attendait Zacharie, et on s'étonnait de ce qu'il tardait tant dans le temple.
\VS{22}Mais quand il fut sorti, il ne pouvait pas leur parler, et ils comprirent qu'il avait eu une vision dans le temple ; car il leur faisait des signes et il resta muet.
\VS{23}Et il arriva que quand les jours de son ministère furent achevés, il retourna dans sa maison.
\VS{24}Et après ces jours-là, Elisabeth sa femme conçut, et elle se cacha l’espace de cinq mois, en disant :
\VS{25}Certes, le Seigneur en a agi avec moi ainsi aux jours qu’il m’a regardée pour ôter mon opprobre d’entre les hommes.
\TextTitle{Annonce de la naissance de Jésus-Christ}
\VS{26}Or au sixième mois, l'ange Gabriel fut envoyé par Dieu dans une ville de Galilée, appelée Nazareth,
\VS{27}vers une vierge fiancée à un homme nommé Joseph, qui était de la maison de David. Et le nom de la vierge était Marie.
\VS{28}Et l'ange étant entré dans le lieu où elle était, lui dit : Je te salue, toi à qui une grâce a été faite. Le Seigneur est avec toi ; tu es bénie parmi les femmes.
\VS{29}Troublée par cette parole, Marie se demandait ce que pouvait signifier une telle salutation.
\VS{30}L'ange lui dit : Marie, ne crains point ; car tu as trouvé grâce devant Dieu.
\VS{31}Et voici, tu concevras en ton ventre, et tu enfanteras un fils, et tu lui donneras le Nom de JESUS.
\VS{32}Il sera grand, et sera appelé le Fils du Très-Haut, et le Seigneur Dieu lui donnera le trône de David, son père.
\VS{33}Il régnera sur la maison de Jacob éternellement, et son règne n'aura pas de fin.
\TextTitle{Naissance miraculeuse de Jésus-Christ}
\VS{34}Alors Marie dit à l'ange : Comment cela se fera-t-il, puisque je ne connais point d'homme ?
\VS{35}L'ange lui répondit et dit : Le Saint-Esprit viendra sur toi, et la puissance du Très-Haut te couvrira de son ombre. C'est pourquoi, le Saint qui naîtra de toi sera appelé Fils de Dieu.
\VS{36}Voici, Elizabeth, ta cousine, a conçu elle aussi un fils en sa vieillesse, celle qui était appelée stérile est dans son sixième mois de grossesse.
\VS{37}Car rien n'est impossible à Dieu.
\VS{38}Et Marie dit : Voici la servante du Seigneur, qu'il me soit fait selon ta parole ! Et l'ange la quitta.
\TextTitle{Marie se rend chez Elisabeth}
\VS{39}Dans ce même temps, Marie se leva, et s'en alla en hâte au pays des montagnes dans une ville de Juda.
\VS{40}Elle entra dans la maison de Zacharie, et salua Elisabeth.
\VS{41}Et il arriva, comme Elisabeth entendait la salutation de Marie, que le petit enfant tressaillit dans son ventre ; et Elisabeth fut remplie de l’Esprit Saint,
\VS{42}Elle s'écria d'une voix forte et dit : Tu es bénie entre les femmes, et béni est le fruit de ton ventre.
\VS{43}Comment m'est-il accordé que la mère de mon Seigneur vienne vers moi ?
\VS{44}Car voici, dès que la voix de ta salutation est parvenue à mes oreilles, le petit enfant a tressailli de joie dans mon ventre.
\VS{45}Heureuse celle qui a cru, parce que les choses qui lui ont été dites par le Seigneur auront leur accomplissement.
\TextTitle{Cantique de Marie\FTNTT{Cp. 1 S. 2:1-10}}
\VS{46}Alors Marie dit : Mon âme magnifie le Seigneur,
\VS{47}et mon esprit se réjouit en Dieu, mon Sauveur.
\VS{48}Car il a jeté les yeux sur la bassesse de sa servante. Voici, certes désormais toutes les générations me diront bienheureuse,
\VS{49}parce que le Tout-Puissant a fait pour moi de grandes choses, et son Nom est Saint.
\VS{50}Et sa miséricorde s'étend de génération en génération en faveur de ceux qui le craignent.
\VS{51}Il a puissamment opéré par son bras. Il a dissipé les desseins que les orgueilleux formaient dans leurs cœurs.
\VS{52}Il a renversé de dessus leurs trônes les puissants, et il a élevé les petits.
\VS{53}Il a rassasié de biens les affamé, il a renvoyé les riches à vide.
\VS{54}Il a pris sous sa protection Israël, son serviteur, et il s'est souvenu de sa miséricorde,
\VS{55}comme il l'avait dit à nos pères, envers Abraham et sa postérité à jamais.
\VS{56}Marie demeura avec elle environ trois mois. Puis elle retourna dans sa maison.
\TextTitle{Naissance de Jean}
\VS{57}Le temps où Elisabeth devait accoucher arriva, et elle enfanta un fils.
\VS{58}Ses voisins et ses parents ayant appris que le Seigneur avait fait éclater sa miséricorde envers elle, s'en réjouissaient avec elle.
\VS{59}Et il arriva qu'au huitième jour, ils vinrent pour circoncire le petit enfant, et ils l'appelaient Zacharie, du nom de son père.
\VS{60}Mais sa mère prit la parole, et dit : Non, mais il sera appelé Jean.
\VS{61}Et ils lui dirent : Il n'y a personne dans ta parenté qui soit appelé de ce nom.
\VS{62}Alors ils firent signe à son père pour savoir comment il voulait qu'on l'appelle.
\VS{63}Et Zacharie ayant demandé des tablettes, écrivit : Jean est son nom. Et tous furent dans l'étonnement.
\VS{64}Au même instant, sa bouche s'ouvrit et sa langue se délia, et il parlait, bénissant Dieu.
\VS{65}Tous ses voisins furent saisis de crainte et toutes ces choses furent divulguées dans tout le pays des montagnes de Judée.
\VS{66}Tous ceux qui les apprirent les gardèrent dans leur cœur, disant : Que sera donc cet enfant ? Et la main du Seigneur était avec lui.
\TextTitle{Cantique de Zacharie}
\VS{67}Alors Zacharie, son père, fut rempli du Saint-Esprit, et il prophétisa en ces mots :
\VS{68}Béni soit le Seigneur, le Dieu d'Israël, de ce qu'il a visité et délivré son peuple
\VS{69}et de ce qu'il nous a suscité un puissant Sauveur dans la maison de David, son serviteur,
\VS{70}selon ce qu'il avait dit par la bouche de ses saints prophètes des temps anciens :
\VS{71}Un Sauveur qui nous délivre de nos ennemis et de la main de tous ceux qui nous haïssent !
\VS{72}C'est ainsi qu'il manifeste sa miséricorde envers nos pères, et se souvient de sa sainte alliance.
\VS{73}Selon le serment par lequel il avait juré à Abraham notre père,
\VS{74}de nous permettre, après que nous serions délivrés de la main de nos ennemis, de le servir sans crainte,
\VS{75}en marchant devant lui dans la sainteté et dans la justice tous les jours de notre vie.
\VS{76}Et toi, petit enfant, tu seras appelé prophète du Très-Haut ; car tu marcheras devant la face du Seigneur, pour préparer ses voies,
\VS{77}afin de donner à son peuple la connaissance du salut, par la rémission de leurs péchés,
\VS{78}grâce aux entrailles de la miséricorde de notre Dieu, en vertu de laquelle le Soleil Levant nous a visités d'en haut,
\VS{79}pour éclairer ceux qui sont assis dans les ténèbres et dans l'ombre de la mort, et pour conduire nos pas dans le chemin de la paix.
\VS{80}Or, le petit enfant croissait, et se fortifiait en esprit. Et il demeura dans les déserts jusqu'au jour où il se présenta à Israël.
\TextTitle{[Naissance de Jésus à Bethléhem]
\\(Mt. 1:18-25 ; 2:1 ; cp. Jn. 1:14}
\Chap{2}
\VerseOne{}En ces jours-là fut publié un édit par César Auguste, ordonnant un recensement de toute la terre.
\VS{2}Ce premier recensement eut lieu pendant que Quirinius était gouverneur de Syrie.
\VS{3}Ainsi, tous allaient pour s'inscrire, chacun dans sa ville.
\VS{4}Joseph aussi monta de Galilée en Judée, de la ville de Nazareth, la ville de David, appelée Bethléhem, parce qu'il était de la maison et de la famille de David ;
\VS{5}afin de se faire inscrire avec Marie, sa fiancée, qui était enceinte.
\VS{6}Pendant qu'ils étaient là, le temps où Marie devait accoucher arriva,
\VS{7}et elle enfanta son fils, premier-né et elle l'emmaillota et le coucha dans une crèche, parce qu'il n'y avait point de place pour eux dans l'hôtellerie.
\TextTitle{L'Ange du Seigneur annonce la naissance de Jésus}
\VS{8}Et il y avait dans cette même contrée des bergers qui couchaient dans les champs, et qui gardaient leur troupeau pendant les veilles de la nuit.
\VS{9}Et voici, l'Ange du Seigneur survint vers eux, et la gloire du Seigneur resplendit autour d'eux, et ils furent saisis d'une grande peur.
\VS{10}Mais l'Ange leur dit : Ne craignez point ; car, voici je vous annonce une bonne nouvelle qui sera un sujet de joie pour tout le peuple :
\VS{11}C'est qu'aujourd'hui, dans la ville de David, vous est né le Sauveur, qui est le Christ, le Seigneur.
\VS{12}Et voici à quel signe vous le reconnaîtrez : Vous trouverez le petit enfant emmailloté, et couché dans une crèche.
\VS{13}Et aussitôt il se joignit à l'Ange une multitude de l'armée céleste, louant Dieu et disant :
\VS{14}Gloire soit à Dieu dans les lieux très-hauts, que la paix soit sur la terre et la bonne volonté dans les hommes !
\TextTitle{Les bergers de Bethléhem}
\VS{15}Et il arriva qu'après que les anges s’en furent allés d’avec eux au ciel, les bergers se dirent les uns aux autres : Allons donc jusqu'à Bethléhem, et voyons cette chose qui est arrivée, ce que le Seigneur nous a fait connaître.
\VS{16}Ils y allèrent donc en hâte, et ils trouvèrent Marie et Joseph, et le petit enfant couché dans une crèche.
\VS{17}Après l'avoir vu, ils divulguèrent ce qui leur avait été dit au sujet de ce petit enfant.
\VS{18}Tous ceux qui les entendirent furent dans l'étonnement de ce que leur disaient les bergers.
\VS{19}Et Marie gardait soigneusement toutes ces choses, et les repassait dans son esprit.
\VS{20}Puis les bergers s'en retournèrent, glorifiant et louant Dieu pour tout ce qu'ils avaient entendu et vu, et qui était conforme à ce qui leur avait été annoncé.
\TextTitle{Jésus circoncis et présenté au temple de Jérusalem\FTNTT{Cp. Ex. 13:12,15}}
\VS{21}Et quand les huit jours furent accomplis pour circoncire l’enfant, on lui donna le Nom de Jésus, nom qu'avait indiqué l'Ange avant qu'il soit conçu dans le sein de sa mère.
\VS{22}Et quand les jours de la purification\FTNT{Lé : 12:2-6.} de Marie furent accomplis selon la loi de Moïse, Joseph et Marie le portèrent à Jérusalem, pour le présenter au Seigneur,
\VS{23}selon ce qui est écrit dans la loi du Seigneur : Tout mâle premier-né sera appelé Saint au Seigneur\FTNT{Ex. 13:2 ; Ex. 13:12 ; No. 3:13 ; No. 8:17.} ;
\VS{24}et pour offrir en sacrifice deux tourterelles ou deux jeunes pigeons comme cela est prescrit dans la loi du Seigneur\FTNT{Lé. 12:8.}.
\TextTitle{Adoration de Siméon et sa prophétie}
\VS{25}Et voici, il y avait à Jérusalem un homme appelé Siméon. Et cet homme était juste et pieux, il attendait la consolation d'Israël, et le Saint-Esprit était sur lui.
\VS{26}Il avait été averti divinement par le Saint-Esprit qu'il ne mourrait point avant d'avoir vu le Christ du Seigneur.
\VS{27}Il vint au temple, poussé par l'Esprit. Et comme les parents apportaient dans le temple l'enfant Jésus, pour accomplir à son égard ce qu'ordonnait la loi,
\VS{28}il le prit dans ses bras, bénit Dieu, et dit :
\VS{29}Seigneur, tu laisses maintenant ton serviteur s'en aller en paix selon ta parole.
\VS{30}Car mes yeux ont vu ton salut.
\VS{31}Lequel tu as préparé devant la face de tous les peuples.
\VS{32}La lumière pour éclairer les nations ; et pour être la gloire de ton peuple d'Israël.
\VS{33}Joseph et sa mère s'étonnaient des choses qui étaient dites de lui.
\VS{34}Siméon le bénit, et dit à Marie, sa mère : Voici, cet enfant est destiné à être une occasion de chute et de relèvement de beaucoup en Israël, et à devenir un signe qui provoquera la contradiction,
\VS{35}en sorte que les pensées de beaucoup de cœurs seront découvertes. Et pour toi, une épée te transpercera l'âme.
\TextTitle{Anne témoigne du Messie}
\VS{36}Il y avait aussi Anne, la prophétesse, fille de Phanuel de la tribu d'Aser, qui était déjà avancée en âge, et qui avait vécu avec son mari sept ans depuis sa virginité.
\VS{37}Restée veuve, et âgée d'environ quatre-vingt-quatre ans, elle ne quittait pas le temple, et elle servait Dieu nuit et jour dans le jeûne et dans les prières.
\VS{38}Etant arrivée à cette heure, elle louait aussi Dieu, et parlait de lui à tous ceux qui attendaient la délivrance de Jérusalem.
\TextTitle{Retour à Nazareth\FTNTT{Suite aux évènements de Mt. 2.}}
\VS{39}Et quand ils eurent accompli tout ce qui est ordonné par la loi du Seigneur, ils s'en retournèrent en Galilée, à Nazareth, leur ville.
\VS{40}Et le petit enfant croissait et se fortifiait en esprit. Il était rempli de sagesse, et la grâce de Dieu était sur lui.
\TextTitle{Jésus assis dans le temple de Jérusalem au milieu des docteurs}
\VS{41}Ses parents allaient tous les ans à Jérusalem, à la fête de Pâque.
\VS{42}Lorsqu'il fut âgé de douze ans, ses parents montèrent à Jérusalem selon la coutume de la fête.
\VS{43}Puis, quand les jours furent écoulés, et qu'ils s'en retournèrent, l'enfant Jésus resta à Jérusalem. Et son père et sa mère ne s'en aperçurent point.
\VS{44}Mais croyant qu'il était avec leurs compagnons de voyage, ils marchèrent une journée, puis ils le cherchèrent parmi leurs parents et parmis leur connaissance.
\VS{45}Et ne le trouvant point, ils retournèrent à Jérusalem, pour le chercher.
\VS{46}Or il arriva que trois jours après, ils le trouvèrent dans le temple, assis au milieu des docteurs, les écoutant, et les interrogeant.
\VS{47}Tous ceux qui l'entendaient s'étonnaient de sa sagesse et de ses réponses.
\VS{48}Quand ses parents le virent, ils furent saisis d'étonnement, et sa mère lui dit : Mon enfant, pourquoi nous as-tu fait ainsi ? Voici, ton père et moi te cherchions avec angoisse.
\VS{49}Et il leur dit : Pourquoi me cherchiez-vous ? Ne saviez-vous pas qu'il faut que je m'occupe des affaires de mon Père ?
\VS{50}Mais ils ne comprirent point ce qu'il leur disait.
\VS{51}Alors il descendit avec eux, et vint à Nazareth ; et il leur était soumis. Et sa mère gardait toutes ces paroles dans son cœur.
\TextTitle{Jésus grandit en sagesse, en stature et en grâce}
\VS{52}Et Jésus croissait en sagesse, en stature, et en grâce, au près de Dieu et devant les hommes.
\Chap{3}
\TextTitle{Ministère de Jean-Baptiste\FTNTT{Mt. 3:1-12 ; Mc. 1:1-8 ; Jn. 1:6-8,15-37}}
\VerseOne{}La quinzième année du règne de Tibère César, lorsque Ponce Pilate était gouverneur de la Judée, Hérode tétrarque de la Galilée, et son frère Philippe tétrarque de l'Iturée et du territoire de la Trachonite, et Lysanias tétrarque de l'Abilène,
\VS{2}et du temps des souverains sacrificateurs Anne et Caïphe, la parole de Dieu fut adressée à Jean, fils de Zacharie, dans le désert.
\VS{3}Et il alla dans tout le pays des environs du Jourdain, prêchant le baptême de repentance, pour la rémission des péchés,
\VS{4}comme il est écrit dans le livre des paroles d'Esaïe, le prophète disant : C'est la voix de celui qui crie dans le désert : Préparez le chemin du Seigneur, aplanissez ses sentiers.
\VS{5}Toute vallée sera comblée, et toute montagne et toute colline seront abaissées, et ce qui est tortueux sera redressé, et les chemins raboteux seront aplanis.
\VS{6}Et toute chair verra le salut de Dieu\FTNT{Es. 40:3-5.}.
\VS{7}Il disait donc à ceux qui venaient en foule pour être baptisés par lui : Races de vipères, qui vous a appris à fuir la colère à venir ?
\VS{8}Produisez donc des fruits dignes de la repentance, et ne vous mettez point à dire en vous-mêmes : Nous avons Abraham pour père. Car je vous dis que Dieu peut faire naître, même de ces pierres, des enfants à Abraham.
\VS{9}Or la cognée est déjà mise à la racine des arbres ; tout arbre donc qui ne produit pas de bon fruit, sera coupé, et jeté au feu.
\VS{10}Alors la foule l'interrogeait, disant : Que ferons-nous donc ?
\VS{11}Et il répondit, et leur dit : Que celui qui a deux tuniques partage avec celui qui n'en a point ; et que celui qui a de quoi manger en fasse de même.
\VS{12}Il vint aussi à lui des publicains pour être baptisés, et ils lui dirent : Maître, que ferons-nous ?
\VS{13}Et il leur dit : N'exigez rien au-delà de ce qui vous a été ordonné.
\VS{14}Des soldats l'interrogèrent aussi, disant : Et nous, que ferons-nous ? Et Il leur répondit : Ne commettez ni extorsion ni fraude envers personne, mais contentez-vous de votre solde.
\VS{15}Et comme le peuple était dans l'attente, et que tous se demandaient dans leurs cœurs si Jean n'était pas le Christ,
\VS{16}Jean prit la parole, et dit à tous : Moi, je vous baptise d'eau ; mais il vient, celui qui est plus puissant que moi, et je ne suis pas digne de délier la courroie de ses souliers. Lui, il vous baptisera du Saint-Esprit et de feu.
\VS{17}Il a son van à la main ; il nettoiera entièrement son aire, et amassera le froment dans son grenier, mais il brûlera la paille dans un feu qui ne s'éteint point.
\VS{18}Et faisant aussi plusieurs autres exhortations, il évangélisait le peuple.
\VS{19}Mais Hérode le tétrarque, étant repris par Jean au sujet d'Hérodias, femme de Philippe son frère, et à cause de toutes les choses méchantes qu'il faisait,
\VS{20}ajouta encore à toutes les autres celle de mettre Jean en prison.
\TextTitle{Baptême de Jésus-Christ\FTNTT{Mc. 1:9-11; cp. Jn. 1:31-34}}
\VS{21}Tout le peuple se faisait baptiser, Jésus aussi fut baptisé, et pendant qu'il priait, le ciel s'ouvrit,
\VS{22}et le Saint-Esprit descendit sur lui sous une forme corporelle, comme celle d'une colombe. Et une voix fit entendre du ciel ces paroles : Tu es mon Fils bien-aimé, en toi j'ai trouvé mon plaisir.
\TextTitle{Généalogie de Jésus-Christ\FTNTT{v. 31 ; Mt. 1:1-16}}
\VS{23}Jésus avait environ trente ans, lorsqu'il commença son ministère, étant comme on l'estimait, fils de Joseph, fils d'Héli,
\VS{24}fils de Matthat, fils de Lévi, fils de Melchi, fils de Jannaï, fils de Joseph,
\VS{25}fils de Mattathias, fils d'Amos, fils de Nahum, fils d'Esli, fils de Naggaï,
\VS{26}fils de Maath, fils de Mattathias, fils de Sémeï, fils de Josech, fils de Joda,
\VS{27}fils de Joanan, fils de Rhésa, fils de Zorobabel, fils de Salathiel, fils de Néri,
\VS{28}fils de Melchi, fils d'Addi, fils de Kosam, fils d'Elmadam, fils d'Er,
\VS{29}fils de Jésus, fils d'Eliézer, fils de Jorim, fils de Matthat, fils de Lévi,
\VS{30}fils de Siméon, fils de Juda, fils de Joseph, fils de Jonam, fils d'Eliakim,
\VS{31}fils de Méléa, fils de Menna, fils de Matthata, fils de Nathan, fils de David,
\VS{32}fils d'Isaï, fils d'Obed, fils de Booz, fils de Salmon, fils de Naasson,
\VS{33}fils d'Aminadab, fils d’Admin, fils d'Aram, fils d'Esrom, fils de Pérets, fils de Juda,
\VS{34}fils de Jacob, fils d'Isaac, fils d'Abraham, fils de Thara, fils de Nachor,
\VS{35}fils de Seruch, fils de Ragau, fils de Phalek, fils d'Eber, fils de Sala,
\VS{36}fils de Kaïnam, fils d’Arphaxad, fils de Sem, fils de Noé, fils de Lamech,
\VS{37}fils de Mathusala, fils d'Hénoc, fils de Jared, fils de Maléléel, fils de Kaïnan,
\VS{38}fils d'Enos, fils de Seth, fils d'Adam, fils de Dieu.
\Chap{4}
\TextTitle{Tentation de Jésus-Christ\FTNTT{Mt. 4:1 ; Mc. 1:12-13 ; cp. Ge. 3:6 ; 1 Jn. 2:16}}
\VerseOne{}Jésus, rempli du Saint-Esprit, revint du Jourdain, et il fut conduit par l'Esprit dans le désert,
\VS{2}où il fut tenté par le diable quarante jours. Et il ne mangea rien durant ces jours-là, et après qu'ils furent écoulés, il eut faim.
\VS{3}Le diable lui dit : Si tu es le Fils de Dieu, ordonne à cette pierre qu'elle devienne du pain.
\VS{4}Jésus lui répondit, en disant : Il est écrit que l'homme ne vivra pas seulement de pain, mais de toute parole de Dieu\FTNT{De. 8:3.}.
\VS{5}Alors le diable l'emmena sur une haute montagne, et lui montra en un instant tous les royaumes de la terre,
\VS{6}et le diable lui dit : Je te donnerai toute cette puissance et leur gloire ; car elle m'a été donnée, et je la donne à qui je veux.
\VS{7}Si donc tu m'adores, elle sera à toi.
\VS{8}Jésus lui répondit : Va arrière de moi, Satan ! Car il est écrit : Tu adoreras le Seigneur ton Dieu, et tu le serviras lui seul\FTNT{De. 6:13.}.
\VS{9}Le diable le conduisit encore à Jérusalem, et le plaça sur le haut du temple, et lui dit : Si tu es le Fils de Dieu, jette-toi d'ici en bas.
\VS{10}Car il est écrit : Il ordonnera à ses anges à ton sujet, afin qu'ils te gardent;
\VS{11}et ils te porteront dans leurs mains, de peur que ton pied ne heurte contre une pierre\FTNT{Ps. 91:11-12.}.
\VS{12}Mais Jésus répondant, lui dit : Il est écrit : Tu ne tenteras pas le Seigneur, ton Dieu\FTNT{De. 6:16.}.
\VS{13}Après l'avoir tenté de toutes ces manières, le diable s'éloigna de lui pour un temps.
\TextTitle{Jésus-Christ retourne en Galilée\FTNTT{Mt. 4:12-17 ; Mc. 1:14-15}}
\VS{14}Jésus retourna en Galilée dans la puissance de l'Esprit, et sa renommée se répandit dans tout le pays d'alentour.
\VS{15}Il enseignait dans leurs synagogues, et il était glorifié par tous.
\TextTitle{Jésus dans la synagogue de Nazareth le jour du sabbat\FTNTT{cp. Mt. 13:54-58 ; Mc. 6:1-6}}
\VS{16}Il se rendit à Nazareth, où il avait été élevé, et selon sa coutume, il entra dans la synagogue le jour du sabbat et il se leva pour faire la lecture,
\VS{17}et on lui donna le livre du prophète Esaïe et l'ayant déroulé, il trouva le passage où il est écrit :
\VS{18}L'Esprit du Seigneur est sur moi, parce qu'il m'a oint pour évangéliser les pauvres ; il m'a envoyé pour guérir ceux qui ont le cœur brisé,
\VS{19}pour proclamer aux captifs la délivrance, et aux aveugles le recouvrement de la vue ; pour mettre en liberté les opprimés ; pour publier une année de grâce du Seigneur\FTNT{Es. 61:1-2.}.
\VS{20}Ensuite, il roula le livre, le rendit au serviteur, et s'assit. Les yeux de tous ceux qui étaient dans la synagogue étaient fixés sur lui.
\VS{21}Alors il commença à leur dire : Aujourd'hui, cette parole de l'Ecriture que vous venez d'entendre, est accomplie.
\VS{22}Et tous lui rendaient témoignage, et s'étonnaient des paroles pleines de grâce qui sortaient de sa bouche ; et ils disaient : Celui-ci n'est-il pas le fils de Joseph ?
\VS{23}Et il leur dit : Assurément vous me direz ce proverbe : Médecin, guéris-toi toi-même. Et fais ici, dans ton pays, tout ce que nous avons appris que tu as fait à Capernaüm.
\VS{24}Mais il leur dit : En vérité je vous dis qu’aucun prophète n'est reçu dans son pays.
\VS{25}Je vous le dis en vérité : Il y avait plusieurs veuves en Israël, du temps d'Elie, lorsque le ciel fut fermé trois ans et six mois et qu'il y eut une grande famine dans tout le pays ;
\VS{26}toutefois Elie ne fut envoyé vers aucune d'elles, mais seulement vers une femme veuve à Sarepta, dans le pays de Sidon.
\VS{27}Il y avait aussi plusieurs lépreux en Israël du temps d'Elisée, le prophète, toutefois aucun d'eux ne fut purifié, si ce n'est Naaman, le Syrien.
\VS{28}Ils furent tous remplis de colère dans la synagogue lorsqu'ils entendirent ces choses.
\VS{29}Et s'étant levés, ils le chassèrent hors de la ville, et le menèrent jusqu'au bord de la montagne sur laquelle leur ville était bâtie, pour le jeter du haut en bas.
\VS{30}Mais il passa au milieu d'eux, et s'en alla.
\TextTitle{Jésus guérit un possédé\FTNTT{Mc. 1:21-28}}
\VS{31}Il descendit à Capernaüm, ville de Galilée, et il les enseignait les jours de sabbat.
\VS{32}Ils étaient frappés de sa doctrine ; car il parlait avec autorité.
\VS{33}Il y avait dans la synagogue un homme qui avait un esprit de démon impur, et qui s'écria d'une voix forte,
\VS{34}en disant : Ah ! Qu'y a-t-il entre nous et toi, Jésus de Nazareth ? Es-tu venu pour nous détruire ? Je sais qui tu es, le Saint de Dieu.
\VS{35}Jésus le menaça, en lui disant : Tais-toi, et sors de cet homme. Et le démon, l'ayant jeté avec impétuosité au milieu de l'assemblée, sortit de cet homme, sans lui faire aucun mal.
\VS{36}Et tous furent saisis de stupeur, et ils parlaient entre eux, et disaient : Quelle est cette parole ? Il commande avec autorité et puissance aux esprits impurs, et ils sortent ?
\VS{37}Et sa renommée se répandit dans tous les lieux d'alentour.
\TextTitle{Guérison de la belle-mère de Pierre et de plusieurs malades\FTNTT{Mt. 8:14-17 ; Mc. 1:29-34}}
\VS{38}Et quand Jésus se fut levé de la synagogue, et il se rendit à la maison de Simon, et la belle-mère de Simon avait une violente fièvre, et ils le prièrent en sa faveur.
\VS{39}Et s'étant penché sur elle, il menaça la fièvre, et la fièvre la quitta. A l'instant elle se leva, et les servit.
\VS{40}Et après le coucher du soleil, tous ceux qui avaient des malades atteints de diverses maladies, les lui amenèrent. Il imposa les mains à chacun d'eux, et il les guérit.
\VS{41}Les démons aussi sortirent de beaucoup de personnes, en criant et en disant : Tu es le Christ, le Fils de Dieu. Mais il les menaçait fortement, et ne leur permettait pas de dire qu'ils savaient qu'il était le Christ.
\VS{42}Dès que le jour parut, il sortit et alla dans un lieu désert et une foule de gens se mirent à sa recherche, et arrivèrent jusqu'à lui et ils voulaient le retenir, afin qu'il ne les quittât point.
\VS{43}Mais il leur dit : Il faut que j'annonce aux autres villes l'Evangile du Royaume de Dieu, car c'est pour cela que j'ai été envoyé.
\VS{44}Et il prêchait dans les synagogues de la Galilée.
\Chap{5}
\TextTitle{Appel des premiers disciples\FTNTT{Mt. 4:18-22 ; Mc. 1:16-20 ; cp. Jn. 1:35-51 ; 21:1-8}}
\VerseOne{}Or il arriva, comme la foule se jetait toute sur lui pour entendre la parole de Dieu, qu’il se tenait sur le bord du lac de Génézareth.
\VS{2}Et voyant deux barques qui étaient au bord du lac, et dont les pêcheurs étaient descendus, et lavaient leurs rets, il monta dans l’une de ces barques, qui était à Simon.
\VS{3}Il monta dans l'une de ces barques, qui était à Simon, et il le pria de s'éloigner un peu de terre. Puis il s'assit, et de la barque il enseignait la foule.
\VS{4}Et quand il eut cessé de parler, il dit à Simon : Avance en pleine eau, et jetez vos filets pour pêcher.
\VS{5}Et Simon répondant, lui dit : Maître, nous avons travaillé toute la nuit, et nous n’avons rien pris ; toutefois à ta parole je jetterai les filets.
\VS{6}Et ayant fait cela, ils prirent une si grande quantité de poissons que leur filet se rompait.
\VS{7}Et ils firent signe à leurs compagnons qui étaient dans l’autre barque, de venir les aider ; et étant venus, ils remplirent les deux barques, tellement qu’elles s’enfonçaient.
\VS{8}Et quand Simon Pierre vit cela, il se jeta aux genoux de Jésus, en lui disant : Seigneur, retire-toi de moi ; car je suis un homme pécheur.
\VS{9}Parce que la frayeur l’avait saisi, lui et tous ceux qui étaient avec lui, à cause de la prise de poissons qu’ils venaient de faire ; de même que Jacques et Jean, fils de Zébédée, qui étaient compagnons de Simon.
\VS{10}Alors Jésus dit à Simon : Ne crains point ; désormais tu seras un pêcheur d'hommes vivants.
\VS{11}Et quand ils eurent ramené les barques à terre, ils abandonnèrent tout et le suivirent.
\TextTitle{Guérison d'un lépreux\FTNTT{Mt. 8:2-4 ; Mc. 1:40-45}}
\VS{12}Et il arriva, comme il était dans une des villes, voici un homme plein de lèpre, voyant Jésus, se jeta sur sa face et le supplia, disant : Seigneur, si tu veux, tu peux me rendre pur.
\VS{13}Jésus étendit la main, et le toucha, en disant : Je le veux, sois pur. Aussitôt la lèpre le quitta.
\VS{14}Et il lui commanda de ne le dire à personne, mais va, lui dit-il, et montre-toi sacrificateur, et offre pour ta purification ce que Moïse a commandé\FTNT{Lé. 13 et 14.}, pour leur servir de témoignage.
\VS{15}Et sa renommée se répandait de plus en plus, tellement que de grandes foules s’assemblaient pour l’entendre, et pour être guéries par lui de leurs maladies.
\VS{16}Mais il se tenait retiré dans les déserts, et priait.
\TextTitle{Guérison d'un paralytique\FTNTT{Mt. 9:2-8 ; Mc. 2:3-12}}
\VS{17}Un jour Jésus enseignait. Et des pharisiens et des docteurs de la loi étaient là assis, venus de tous les villages de la Galilée, et de la Judée et de Jérusalem ; et la puissance du Seigneur se manifestait par des guérisons.
\VS{18}Et voici des hommes qui portaient sur un lit un homme qui était paralytique, et ils cherchaient le moyen de le porter dans la maison, et de le mettre devant lui.
\VS{19}Comme ils ne savaient pas par où l'introduire, à cause de la foule, ils montèrent sur le toit, et ils le descendirent par une ouverture, avec son lit, au milieu de la foule, devant Jésus.
\VS{20}Voyant leur foi, il dit au paralytique : Homme, tes péchés te sont pardonnés.
\VS{21}Alors les scribes et les pharisiens commencèrent à raisonner en eux-mêmes, disant : Qui est celui-ci qui profère des blasphèmes ? Qui est-ce qui peut pardonner les péchés, si ce n'est Dieu seul ?
\VS{22}Mais Jésus, connaissant leurs pensées, prit la parole et leur dit : Pourquoi raisonnez-vous ainsi en vous-mêmes ?
\VS{23}Lequel est le plus aisé de dire : Tes péchés te sont pardonnés ; ou de dire : Lève-toi et marche ?
\VS{24}Or afin que vous sachiez que le Fils de l'homme a le pouvoir sur la terre de pardonner les péchés, il dit au paralytique : Je te l'ordonne, lève-toi, prends ton lit, et va dans ta maison.
\VS{25}Et à l'instant, le paralytique s'étant levé devant eux, prit le lit sur lequel il était couché, et s'en alla dans sa maison, glorifiant Dieu.
\VS{26}Ils furent tous saisis d'étonnement, et ils glorifiaient Dieu ; et étant remplis de crainte, ils disaient : certainement nous avons vu aujourd'hui des choses étranges.
\TextTitle{Appel de Lévi\FTNTT{Mt. 9:9 ; Mc. 2:13-14}}
\VS{27}Après cela, Jésus sortit, et il vit un publicain nommé Lévi, assis au bureau des péages, et il lui dit : Suis-moi.
\VS{28}Et abandonnant tout, il se leva, et le suivit.
\TextTitle{Appelle des pécheurs à la repentance\FTNTT{Mt. 9:10-15 ; Mc. 2:13-14}}
\VS{29}Et Lévi lui fit un grand festin dans sa maison ; et il y avait une grande foule de publicains et d’autres gens qui étaient avec eux à table.
\VS{30}Les scribes de ce lieu-là et les pharisiens, murmuraient contre ses disciples en disant : Pourquoi mangez-vous et buvez-vous avec les publicains et les gens de mauvaise vie ?
\VS{31}Mais Jésus, prenant la parole, leur dit : ceux qui sont en santé n’ont pas besoin de médecin, mais ceux qui se portent mal.
\VS{32}Je ne suis point venu appeler à la repentance les justes, mais les pécheurs.
\VS{33}Ils lui dirent aussi : pourquoi est-ce que les disciples de Jean jeûnent souvent, et font des prières et également ceux des Pharisiens, mais les tiens mangent et boivent ?
\VS{34}Il leur répondit : Pouvez-vous faire jeûner les amis de l'Epoux pendant que l'Epoux est avec eux ?
\VS{35}Mais les jours viendront où l'Epoux leur sera enlevé alors ils jeûneront en ces jours-là.
\TextTitle{Parabole du drap neuf et des outres neuves\FTNTT{Mt. 9:16-17 ; Mc. 2:21-22}}
\VS{36}Puis il leur dit cette parabole : Personne ne met une pièce d'un habit neuf à un vieil habit ; autrement le neuf déchire le vieux, et la pièce du neuf ne s'accorde pas avec le vieux.
\VS{37}Et personne ne met du vin nouveau dans de vieilles outres ; autrement le vin nouveau fait rompre les outres, et il se répand, et les outres sont perdues.
\VS{38}Mais le vin nouveau doit être mis dans des outres neuves ; et ainsi ils se conservent l'un et l'autre.
\VS{39}Et personne, après avoir bu du vin vieux, ne veut du nouveau, car il dit : Le vieux est meilleur.
\Chap{6}
\TextTitle{Jésus, le Maître du sabbat\FTNTT{Mt. 12:1-8 ; Mc. 2:23-28}}
\VerseOne{}Or il arriva un jour de sabbat appelé second-premier, qu'il passait par des blés ; et ses disciples arrachaient des épis et les froissant dans leurs mains, ils les mangeaient.
\VS{2}Et quelques pharisiens leur dirent : Pourquoi faites-vous ce qu'il n'est pas permis de faire les jours du sabbat ?
\VS{3}Et Jésus prenant la parole, leur dit : N'avez-vous pas lu ce que fit David quand il eut faim, lui et ceux qui étaient avec lui ;
\VS{4}comment il entra dans la maison de Dieu, et prit les pains de proposition, et en mangea, et en donna aussi à ceux qui étaient avec lui, bien qu'il ne soit permis qu'aux sacrificateurs d'en manger ?\FTNT{2 S. 21:1-7.}
\VS{5}Puis il leur dit : Le Fils de l'homme est Maître même du sabbat.
\TextTitle{Guérison d'un homme à la main sèche\FTNTT{Mt. 12:9-13 ; Mc. 3:1-5}}
\VS{6}Et il arriva, un autre jour de sabbat, qu'il entra dans la synagogue, et qu'il enseignait et il s'y trouvait là un homme dont la main droite était sèche.
\VS{7}Or les scribes et les pharisiens l'observaient pour voir s'il ferait une guérison le jour du sabbat ; c'était afin d'avoir sujet de l'accuser.
\VS{8}Mais il connaissait leurs pensées et il dit à l'homme qui avait la main sèche : Lève-toi, et tiens-toi debout au milieu. Et il se leva et se tint debout.
\VS{9}Puis Jésus leur dit : Je vous demande une chose : Est-il permis de faire du bien les jours de sabbat, ou de faire du mal ? De sauver une personne, ou de la laisser mourir ?
\VS{10}Et ayant regardé tous ceux qui étaient autour de lui, il dit à l’homme : étends ta main ; ce qu’il fit, et sa main fut rendue saine comme l’autre.
\VS{11}Et ils furent remplis de fureur, et ils s’entretenaient ensemble touchant ce qu’ils pourraient faire à Jésus.
\VS{12}Or il arriva en ces jours-là, qu’il s’en alla sur une montagne pour prier, et qu’il passa toute la nuit à prier Dieu.
\TextTitle{Choix des douze apôtres\FTNTT{cp. Mt. 10:2-4 ; Mc. 3:13-19}}
\VS{13}Et quand le jour fut venu, il appela ses disciples. Et en ayant choisi douze d’entre eux, il les nomma apôtres :
\VS{14}Simon, qu'il nomma Pierre, et André son frère, Jacques et Jean, Philippe et Barthélemy ;
\VS{15}Matthieu et Thomas, Jacques fils d'Alphée, et Simon surnommé zélote\FTNT{Zélote : « Celui qui est zélé ». Les zélotes faisaient partie d'un mouvement politique Juif du premier siècle ap. J.-C., qui cherchait à inciter les gens de la province de Judée à se rebeller contre l'Empire Romain, et à le chasser du pays par les armes, pendant la Grande Révolte Juive (66-70 ap. J.-C.). Lorsque les Romains introduisirent le culte impérial, les Juifs se rebellèrent et furent réprimés. Les zélotes considéraient qu'Israël appartenait seulement à un roi Juif de la descendance de David. De plus, reconnaître l'empereur équivalait, à leurs yeux, à renier Dieu. Le mouvement zélote se réclamait intentionnellement de modèles bibliques tels que Phinées, le fils zélé d'Eléazar, fils d'Aaron (No. 25:11). Ce dernier s'était illustré par l'assassinat d'un prince de tribu d'Israël qui s'était fourvoyé dans la luxure aux yeux de tous.} ;
\VS{16}Jude, frère de Jacques, et Judas Iscariot, qui devint traître.
\TextTitle{Enseignement sur la montagne\FTNTT{Mt. 5-7}}
\VS{17}Puis descendant avec eux, il s'arrêta sur une plaine avec la foule de ses disciples et une grande multitude de peuple de toute la Judée, et de Jérusalem, et de la contrée maritime de Tyr et de Sidon, qui étaient venus pour l'entendre, et pour être guéris de leurs maladies.
\VS{18}Ceux aussi qui étaient tourmentés par des esprits impurs furent guéris.
\VS{19}Et toute la foule cherchait à le toucher, parce qu'une force sortait de lui et les guérissait tous.
\TextTitle{Enseignement de Jésus\FTNTT{Mt. 5:3-12}}
\VS{20}Alors Jésus, levant les yeux vers ses disciples, leur dit : Heureux vous qui êtes pauvres, car le Royaume de Dieu vous appartient.
\VS{21}Heureux vous qui avez faim maintenant, car vous serez rassasiés ! Heureux vous qui pleurez maintenant, car vous serez dans la joie !
\VS{22}Heureux serez-vous quand les hommes vous haïront, vous chasseront, vous outrageront, et rejetteront votre nom comme infâme, à cause du Fils de l'homme.
\VS{23}Réjouissez-vous en ce jour-là, et tressaillez d'allégresse, parce que votre récompense sera grande dans le ciel ; car leurs pères en faisaient de même aux prophètes.
\VS{24}Mais malheur à vous riches, car vous avez votre consolation.
\VS{25}Malheur à vous qui êtes rassasiés, car vous aurez faim. Malheur à vous qui riez maintenant, car vous serez dans le deuil et dans les larmes.
\VS{26}Malheur à vous quand tous les hommes diront du bien de vous ; car leurs pères en faisaient de même aux faux prophètes.
\VS{27}Mais à vous qui m’entendez, je vous dis : aimez vos ennemis ; faites du bien à ceux qui vous haïssent.
\VS{28}Bénissez ceux qui vous maudissent, et priez pour ceux qui vous maltraitent.
\VS{29}Si quelqu'un te frappe sur une joue, présente-lui aussi l'autre. Si quelqu'un prend ton manteau, ne l'empêche pas de prendre aussi ta tunique.
\VS{30}Donne à quiconque te demande, et ne réclame pas ton bien à celui qui s'en empare.
\VS{31}Ce que vous voulez que les hommes fassent pour vous, faites-le de même pour eux.
\VS{32}Mais si vous aimez seulement ceux qui vous aiment, quel gré vous en saura-t-on ? Les pécheurs aussi aiment ceux qui les aiment.
\VS{33}Et si vous faites du bien à ceux qui vous font du bien, quel gré vous en saura-t-on ? Les pécheurs aussi font de même.
\VS{34}Et si vous prêtez à ceux de qui vous espérez recevoir, quel gré vous en saura-t-on ? Les pécheurs aussi prêtent aux pécheurs, afin de recevoir la pareille.
\VS{35}C'est pourquoi aimez vos ennemis et faites-leur du bien, et prêtez sans rien espérer, et votre récompense sera grande, et vous serez les fils du Très-Haut, car il est bon envers les ingrats et les méchants.
\VS{36}Soyez donc miséricordieux comme votre Père est miséricordieux.
\VS{37}Ne jugez point, et vous ne serez point jugés ; ne condamnez point, et vous ne serez point condamnés ; absolvez, et vous serez absous.
\VS{38}Donnez, et il vous sera donné : On versera dans votre sein une bonne mesure, serrée, secouée et qui déborde ; car on vous mesurera avec la mesure dont vous vous serez servis.
\VS{39}Il leur disait aussi cette parabole : Un aveugle peut-il conduire un aveugle ? Ne tomberont-ils pas tous deux dans la fosse ?
\VS{40}Le disciple n'est pas au-dessus de son maître ; mais tout disciple accompli sera comme son maître.
\VS{41}Pourquoi regardes-tu la paille qui est dans l'œil de ton frère, et n'aperçois-tu pas la poutre qui est dans ton propre œil ?
\VS{42}Ou comment peux-tu dire à ton frère : Mon frère, laisse-moi enlever la paille qui est dans ton œil, toi qui ne vois pas la poutre qui est dans ton œil ? Hypocrite, ôte premièrement la poutre de ton œil, et après cela tu verras comment ôter la paille qui est dans l'œil de ton frère.
\VS{43}Ce n'est pas un bon arbre qui porte du mauvais fruit, ni un mauvais arbre qui porte du bon fruit.
\VS{44}Car chaque arbre se reconnaît à son fruit. On ne cueille pas des figues sur des épines, et l'on ne vendange pas des raisins sur des ronces.
\VS{45}L'homme de bien tire de bonnes choses du bon trésor de son cœur, et l'homme méchant tire de mauvaises choses du mauvais trésor de son cœur ; car c'est de l'abondance du cœur que la bouche parle.
\TextTitle{Parabole des deux bâtisseurs et des deux fondements\FTNTT{Mt. 7:24-27}}
\VS{46}Mais pourquoi m'appelez-vous Seigneur, Seigneur, et ne faites-vous pas ce que je dis ?
\VS{47}Je vous montrerai à qui est semblable celui qui vient à moi, entend mes paroles, et les met en pratique.
\VS{48}Il est semblable à un homme qui bâtissant une maison, a creusé, creusé profondément, et a mis le fondement sur le roc. Une inondation est venue, et le torrent s'est jeté contre cette maison, sans pouvoir l'ébranler, parce qu'elle était bâtie sur le roc.
\VS{49}Mais celui qui entend mes paroles, et ne les met pas en pratique, est semblable à un homme qui a bâti sa maison sur la terre, sans fondement. Le torrent s'est jeté contre elle ; aussitôt elle est tombée, et la ruine de cette maison a été grande.
\Chap{7}
\TextTitle{Guérison du serviteur d'un centenier\FTNTT{Mt. 8:5-13}}
\VerseOne{}Et quand il eut achevé tout ce discours devant le peuple qui l'écoutait, il entra dans Capernaüm.
\VS{2}Un centenier avait un serviteur, auquel il était très attaché, et qui était malade, sur le point de mourir.
\VS{3}Ayant entendu parler de Jésus, il envoya vers lui quelques anciens des Juifs, pour le prier de venir guérir son serviteur.
\VS{4}Et étant venu à Jésus, ils lui prièrent instamment, disant : Il mérite que tu lui accordes cela.
\VS{5}Car, disaient-ils, il aime notre nation, et c'est lui qui a bâti notre synagogue.
\VS{6}Jésus s'en alla donc avec eux. Il n'était guère éloigné de la maison, quand le centenier envoya ses amis au-devant de lui, pour lui dire : Seigneur, ne te fatigue point ; car je ne suis pas digne que tu entres sous mon toit.
\VS{7}C'est pourquoi aussi je ne me suis pas cru digne d'aller moi-même vers toi ; mais dis seulement une parole, et mon serviteur sera guéri.
\VS{8}Car, moi qui suis un homme soumis à des supérieurs, j'ai des soldats sous mes ordres ; et je dis à l'un : Va, et il va ; et à un autre : Viens, et il vient ; et à mon serviteur : Fais cela, et il le fait.
\VS{9}Lorsque Jésus entendit ces paroles, il admira le centenier ; et se tournant vers la foule qui le suivait, il dit : Je vous le dis, je n'ai pas trouvé, même en Israël, une si grande foi.
\VS{10}Et quand ceux qui avaient été envoyés furent de retour à la maison, ils trouvèrent le serviteur qui avait été malade, se portant bien.
\TextTitle{Le fils de la veuve de Naïn ressuscite}
\VS{11}Le jour suivant, Jésus alla dans une ville appelée Naïn ; plusieurs de ses disciples et une grande foule allaient avec lui.
\VS{12}Et comme il approchait de la porte de la ville, voici, on portait en terre un mort, fils unique de sa mère, qui était veuve ; et il y avait avec elle un grand nombre de gens de la ville.
\VS{13}Le Seigneur l'ayant vue, fut ému de compassion pour elle ; et il lui dit : Ne pleure pas !
\VS{14}Il s'approcha, et toucha le cercueil. Ceux qui le portaient s'arrêtèrent. Et il dit : Jeune homme, je te dis, lève-toi !
\VS{15}Et le mort s'assit, et se mit à parler. Et Jésus le rendit à sa mère.
\VS{16}Et ils furent tous saisis de crainte, et ils glorifiaient Dieu, disant : Certainement un grand prophète a paru parmi nous ; et Dieu a visité son peuple.
\VS{17}Cette parole sur ce miracle se répandit dans toute la Judée, et dans tout le pays d'alentour.
\VS{18}Jean fut informé de toutes ces choses par ses disciples.
\TextTitle{Jean-Baptiste, le plus grand des hommes\FTNTT{Mt. 11:1-19}}
\VS{19}Il en appela deux, et les envoya vers Jésus pour lui dire : Es-tu celui qui devait venir, ou devons-nous en attendre un autre ?
\VS{20}Et étant venus à lui, ils lui dirent : Jean-Baptiste nous a envoyés auprès de toi pour te dire : Es-tu celui qui devait venir, ou devons-nous en attendre un autre ?
\VS{21}A l'heure même, Jésus guérit plusieurs personnes de maladies,  d'infirmités, et d'esprits malins ; et il rendit la vue à plusieurs aveugles.
\VS{22}Ensuite Jésus leur répondit, et leur dit : Allez, et rapportez à Jean ce que vous avez vu et entendu : Les aveugles recouvrent la vue, les boiteux marchent, les lépreux sont purifiés, les sourds entendent, les morts ressuscitent, l'Evangile est annoncé aux pauvres.
\VS{23}Heureux celui qui n'aura point été scandalisé à cause de moi !
\VS{24}Lorsque les messagers de Jean furent partis, Jésus se mit à dire à la foule au sujet de Jean : Qu'êtes-vous allés voir au désert ? Un roseau agité par le vent ?
\VS{25}Mais qu'êtes-vous allés voir ? Un homme vêtu d'habits précieux ? Voici, ceux qui portent des habits magnifiques, et qui vivent dans les délices, sont dans les maisons des rois.
\VS{26}Mais qu'êtes-vous donc allés voir ? Un prophète ? Oui, vous dis-je, et plus qu'un prophète.
\VS{27}C'est de lui qu'il est écrit : Voici, j'envoie mon messager devant ta face, et il préparera ta voie devant toi\FTNT{Mal. 3:1.}.
\VS{28}Car je vous dis, parmi ceux qui sont nés de femmes, il n'y a aucun prophète plus grand que Jean-Baptiste. Cependant, le plus petit dans le Royaume de Dieu est plus grand que lui.
\VS{29}Et tout le peuple qui entendait cela, et les publicains, justifiaient Dieu, ayant été baptisés du baptême de Jean.
\VS{30}Mais les pharisiens, et les docteurs de la loi, qui n'avaient point été baptisés par lui, rendirent le dessein de Dieu inutile à leur égard.
\VS{31}Alors le Seigneur dit : A qui donc comparerai-je les hommes de cette génération ; et à quoi ressemblent-ils ?
\VS{32}Ils sont semblables aux enfants qui sont assis sur la place publique, et qui se parlant les uns aux autres, disent : Nous vous avons joué de la flûte, et vous n'avez pas dansé ; nous vous avons chanté des complaintes, et vous n'avez pas pleuré.
\VS{33}Car Jean-Baptiste est venu ne mangeant point de pain, et ne buvant point de vin ; et vous dites : Il a un démon.
\VS{34}Le Fils de l'homme est venu mangeant et buvant ; et vous dites : Voici un mangeur et un buveur, un ami des publicains et des pécheurs.
\VS{35}Mais la sagesse a été justifiée par tous ses enfants.
\TextTitle{Une pécheresse pardonnée par Jésus}
\VS{36}Un des pharisiens pria Jésus de manger chez lui ; et Jésus entra dans la maison de ce pharisien, et se mit à table.
\VS{37}Et voici, il y avait dans la ville une femme pécheresse, qui ayant su que Jésus était à table dans la maison du pharisien, apporta un vase d'albâtre plein de parfum,
\VS{38}et se tenant derrière à ses pieds, et pleurant, elle les mouilla de ses larmes, elle les essuya avec ses propres cheveux, et lui baisa les pieds, et les oignit de cette huile odoriférante.
\VS{39}Mais le pharisien qui l'avait invité, voyant cela, dit en lui-même : Si cet homme était prophète, certes il saurait qui et de quelle espèce est la femme qui le touche, il saurait que c'est une pécheresse.
\TextTitle{Parabole des deux débiteurs}
\VS{40}Et Jésus prenant la parole, lui dit : Simon, j'ai quelque chose à te dire. Maître, parle, répondit-il.
\VS{41}Un créancier avait deux débiteurs : L'un lui devait cinq cents deniers, et l'autre cinquante.
\VS{42}Et comme ils n'avaient pas de quoi payer, il leur remit à tous deux leur dette. Lequel l'aimera le plus ?
\VS{43}Et Simon répondant lui dit : Celui, je pense, à qui il a le plus remis. Jésus lui dit : Tu as droitement jugé.
\VS{44}Alors se tournant vers la femme, il dit à Simon : Vois-tu cette femme ? Je suis entré dans ta maison, et tu ne m'as point donné d'eau pour laver mes pieds ; mais elle, elle les a mouillés de ses larmes, elle les a essuyés avec ses propres cheveux.
\VS{45}Tu ne m'as point donné un baiser, mais elle, depuis que je suis entré, n'a cessé d'embrasser mes pieds.
\VS{46}Tu n'as pas oint ma tête d'huile ; mais elle, elle a oint mes pieds d'une huile odoriférante.
\VS{47}C'est pourquoi je te le dis, ses nombreux péchés ont été pardonnés, car elle a beaucoup aimé. Or celui à qui on pardonne peu, aime peu.
\VS{48}Puis il dit à la femme : Tes péchés sont pardonnés.
\VS{49}Ceux qui étaient avec lui à table, se mirent à dire en eux-mêmes : Qui est celui-ci qui pardonne même les péchés ?
\VS{50}Mais il dit à la femme : Ta foi t'a sauvée. Va en paix.
\Chap{8}
\TextTitle{Les femmes au service de Jésus durant son ministère}
\VerseOne{}Or il arriva après cela qu'il allait de ville en ville, et de villages en villages, prêchant et annonçant l'Evangile du Royaume de Dieu.
\VS{2}Les douze disciples étaient auprès de lui avec quelques femmes aussi qu'il avait délivrées d'esprits malins et de maladies : Marie de Magdala, de laquelle étaient sortis sept démons,
\VS{3}Et Jeanne, femme de Chuza, intendant d'Hérode, Susanne, et plusieurs autres qui l'assistaient de leurs biens.
\TextTitle{Parabole du semeur\FTNTT{Mt. 13:1-23 ; Mc. 4:1-20}}
\VS{4}Et comme une grande foule s'étant assemblée, et des gens étant venus de diverses villes auprès de lui, il leur dit cette parabole :
\VS{5}Un semeur sortit pour semer sa semence ; et en semant, une partie de la semence tomba le long du chemin ; elle fut foulée aux pieds, et les oiseaux du ciel la mangèrent toute.
\VS{6}Une autre partie tomba dans un endroit pierreux ; et quand elle fut levée, elle sécha, parce qu'elle n'avait point d'humidité.
\VS{7}Une autre partie tomba au milieu des épines ; les épines crurent avec elle, et l'étouffèrent.
\VS{8}Une autre partie tomba dans une bonne terre ; quand elle fut levée, elle donna du fruit au centuple. En disant ces choses, Jésus dit à haute voix : Que celui qui a des oreilles pour entendre, qu'il entende.
\VS{9}Et ses disciples l'interrogèrent pour savoir ce que signifiait cette parabole.
\VS{10}Il répondit : Il vous a été donné de connaître les mystères du Royaume de Dieu, mais pour les autres, cela leur est dit en paraboles, afin qu'en voyant ils ne voient point, et qu'en entendant ils ne comprennent point.
\VS{11}Voici donc ce que signifie cette parabole : La semence, c'est la parole de Dieu.
\VS{12}Ceux qui ont reçu la semence le long du chemin, ce sont ceux qui entendent la parole ; mais ensuite le diable vient et ôte la parole de leur cœur, de peur qu'ils ne croient et soient sauvés.
\VS{13}Et ceux qui ont reçu la semence dans un endroit pierreux, ce sont ceux qui, lorsqu'ils entendent la parole, la reçoivent avec joie ; mais ils n'ont point de racine ; ils croient pour un temps, mais au moment de la tentation ils se retirent.
\VS{14}Et ce qui est tombé parmi les épines, ce sont ceux qui ayant entendu la parole, s'en vont, et la laissent étouffer par les soucis, les richesses, et les plaisirs de la vie, et ils ne portent point de fruit qui vienne à maturité.
\VS{15}Mais ce qui est tombé dans une bonne terre, ce sont ceux qui ayant entendu la parole, la retiennent dans un cœur honnête et bon, et portent du fruit avec persévérance.
\TextTitle{Parabole du chandelier\FTNTT{Mt. 5:15-16 ; Mc. 4:21-23 ; Lu. 11:33-36}}
\VS{16}Personne, après avoir allumé la lampe, ne la couvre d'un vase ni ne la met sous un lit, mais il la met sur un chandelier, afin que ceux qui entrent voient la lumière.
\VS{17}Car il n'est rien de secret qui ne doive être découvert ; rien de caché qui ne doive être connu et qui ne vienne en évidence.
\VS{18}Prenez donc garde à la manière dont vous écoutez ; car on donnera à celui qui a, mais à celui qui n'a pas, on ôtera même ce qu'il croit avoir.
\TextTitle{La famille spirituelle\FTNTT{Mt. 12:46-50 ; Mc. 3:31-35}}
\VS{19}Alors sa mère et ses frères vinrent vers lui, mais ils ne pouvaient l'aborder à cause de la foule.
\VS{20}Et on vint lui dire : Ta mère et tes frères sont dehors, et ils désirent te voir.
\VS{21}Mais il répondit : Ma mère et mes frères sont ceux qui écoutent la parole de Dieu, et qui la mettent en pratique.
\TextTitle{Jésus calme la tempête\FTNTT{Mt. 8:23-27 ; Mc. 4:35-41}}
\VS{22}Or il arriva qu'un jour, Jésus monta dans une barque avec ses disciples, et il leur dit : Passons de l'autre côté du lac ; et ils partirent.
\VS{23}Pendant qu'ils naviguaient, il s'endormit. Un vent impétueux se leva sur le lac, la barque se remplissait d'eau, et ils étaient en danger.
\VS{24}Ils s'approchèrent et le réveillèrent, en disant : Maître ! Maître ! Nous périssons ! S'étant réveillé, il menaça le vent et les flots qui s'apaisèrent, et le calme revint.
\VS{25}Alors il leur dit : Où est votre foi ? Saisis de frayeur et d'étonnement, ils se dirent les uns aux autres : Quel est donc celui-ci qui commande même aux vents et à l'eau, et à qui ils obéissent ?
\TextTitle{Le démoniaque de Gérasa (Gadara) délivré\FTNTT{Mt. 8:28-34 ; Mc. 5:1-20}}
\VS{26}Puis ils abordèrent dans le pays des Géraséniens qui est vis-à-vis de la Galilée.
\VS{27}Et quand il fut descendu à terre, il vint à sa rencontre un homme de cette ville, qui depuis longtemps était possédé de plusieurs démons. Il ne portait point de vêtements, avait sa demeure, non dans une maison, mais dans les sépulcres.
\VS{28}Ayant vu Jésus, il s'écria et se prosterna devant lui, disant à haute voix : Qu'y a-t-il entre moi et toi, Jésus, Fils du Dieu Très-Haut ? Je te prie, ne me tourmente point.
\VS{29}Car Jésus commandait à l'esprit impur de sortir de cet homme, dont il s'était emparé depuis longtemps. On le gardait lié de chaînes et les fers aux pieds, mais il rompait les liens, et il était entrainé par le démon dans les déserts.
\VS{30}Jésus lui demanda : Quel est ton nom ? Légion\FTNT{Voir commentaire en Mc. 5:9.} répondit-il. Car plusieurs démons étaient entrés en lui.
\VS{31}Et ils priaient Jésus de ne pas leur ordonner d'aller dans l'abîme.
\VS{32}Or il y avait là, dans la montagne, un grand troupeau de pourceaux qui paissaient. Les démons supplièrent Jésus de leur permettre d'entrer dans ces pourceaux. Il le leur permit.
\VS{33}Les démons sortirent de cet homme et entrèrent dans les pourceaux ; et le troupeau se précipita des pentes escarpées dans le lac, et se noya.
\VS{34}Ceux qui les faisaient paître, voyant ce qui était arrivé, s'enfuirent et allèrent le raconter dans la ville et dans les campagnes.
\VS{35}Les gens sortirent pour voir ce qui était arrivé. Ils vinrent auprès de Jésus, et ils trouvèrent l'homme de qui étaient sortis les démons, assis aux pieds de Jésus, vêtu et dans son bon sens ; et ils furent saisis de frayeur.
\VS{36}Et ceux qui avaient vu ce qui s'était passé leur racontèrent comment le démoniaque avait été délivré.
\VS{37}Alors toute cette multitude venue de divers endroits voisins des Géraséniens, le prièrent de se retirer de chez eux ; car ils étaient saisis d'une grande crainte. Jésus monta donc dans la barque, et s'en retourna.
\VS{38}L'homme de qui étaient sortis les démons lui demanda la permission de rester avec lui ; mais Jésus le renvoya, en lui disant :
\VS{39}Retourne dans ta maison, et raconte tout ce que Dieu t'a fait. Il s'en alla donc, et publia par toute la ville tout ce que Jésus avait fait pour lui.
\VS{40}Quand Jésus fut de retour, la foule le reçut avec joie ; car tous l'attendaient.
\TextTitle{Les deux guérisons\FTNTT{Mt. 9:18-26 ; Mc. 5:21-43}}
\VS{41}Et voici, un homme appelé Jaïrus, qui était chef de la synagogue, vint et se jetant aux pieds de Jésus, le pria d'entrer dans sa maison
\VS{42}parce qu'il avait une fille unique, âgée d'environ douze ans, qui se mourait. Pendant que Jésus y allait, il était pressé par la foule.
\VS{43}Or, il y avait une femme atteinte d'une perte de sang depuis douze ans, et qui avait dépensé tout son bien pour les médecins, sans qu'aucun n'ait pu la guérir.
\VS{44}S'approchant de lui par derrière, elle toucha le bord de son vêtement. Au même instant la perte de sang s'arrêta.
\VS{45}Jésus dit : Qui m'a touché ? Comme tous le niaient, Pierre et ceux qui étaient avec lui, dirent : Maître, la foule qui t'entoure te presse et tu dis : Qui m'a touché ?
\VS{46}Mais Jésus dit : Quelqu'un m'a touché, car j'ai connu qu'une force est sortie de moi.
\VS{47}Alors la femme, se voyant découverte, vint toute tremblante se jeter à ses pieds, lui déclara devant tout le peuple pour quelle raison elle l'avait touché, et comment elle avait été guérie à l'instant.
\VS{48}Jésus lui dit : Ma fille, rassure-toi. Ta foi t'a guérie. Va en paix.
\VS{49}Et comme il parlait encore, quelqu'un vint de chez le chef de la synagogue, qui lui dit : Ta fille est morte, n'importune pas le Maître.
\VS{50}Mais Jésus ayant entendu cela, dit au père de la fille : Ne crains point ; crois seulement, et elle sera guérie.
\VS{51}Et quand il fut arrivé à la maison, il ne permit à personne d'entrer avec lui, si ce n'est à Pierre, à Jacques et à Jean, et au père et à la mère de la fille.
\VS{52}Or il la pleuraient tous et de douleur, ils se frappaient la poitrine ; mais il leur dit : Ne pleurez point, elle n'est pas morte, mais elle dort.
\VS{53}Et ils se moquaient de lui, sachant bien qu'elle était morte.
\VS{54}Mais les ayant tous fait sortir, il prit la main de la fille, et dit d'une voix forte : Enfant, lève-toi !
\VS{55}Et son esprit revint en elle, et à l'instant elle se leva ; et Jésus ordonna qu'on lui donne à manger.
\VS{56}Les parents de la fille furent dans l'étonnement, et il leur commanda de ne dire à personne ce qui était arrivé.
\Chap{9}
\TextTitle{Mission des douze apôtres\FTNTT{Mt. 10:1-42 ; cp. Mc. 6:7-13}}
\VerseOne{}Puis, Jésus ayant assemblé ses douze disciples, leur donna puissance et autorité sur tous les démons, avec le pouvoir de guérir les malades.
\VS{2}Il les envoya prêcher le Royaume de Dieu et guérir les malades.
\VS{3}Il leur dit : Ne prenez rien pour le voyage, ni bâton, ni sac, ni pain, ni argent ; et n'ayez pas chacun deux tuniques.
\VS{4}Dans quelque maison que vous entriez, demeurez-y jusqu'à ce que vous partiez de là.
\VS{5}Et partout où l'on ne vous recevra pas, en partant de cette ville secouez la poussière de vos pieds, en témoignage contre eux.
\VS{6}Ils partirent, et ils allèrent de village en village, évangélisant et opérant des guérisons partout.
\VS{7}Or Hérode le Tétrarque entendit parler de toutes les choses que Jésus faisait ; et il ne savait que penser. Car quelques-uns disaient que Jean était ressuscité des morts ;
\VS{8}d'autres, qu'Elie était apparu ; et d'autres, que quelqu'un des anciens prophètes était ressuscité.
\VS{9}Mais Hérode dit : J'ai fait décapiter Jean. Qui est donc celui-ci de qui j'entends dire de telles choses ? Et il cherchait à le voir.
\VS{10}Puis les apôtres étant de retour, lui racontèrent toutes les choses qu'ils avaient faites. Jésus les prit avec lui, et se retira dans un lieu désert, près de la ville appelée Bethsaïda.
\VS{11}Les foules l'ayant su, le suivirent. Jésus les accueillit, et il leur parlait du Royaume de Dieu ; il guérit aussi ceux qui avaient besoin d'être guéris.
\TextTitle{Multiplication des pains pour cinq mille hommes\FTNTT{Mt. 14:15-21 ; Mc. 6:32-44 ; Jn. 6:1-14}}
\VS{12}Comme le jour commençait à baisser, les douze disciples s'approchèrent, et lui dirent : Renvoie la foule, afin qu'elle aille dans les villages et dans les campagnes des environs, pour se loger et pour trouver à manger ; car nous sommes ici dans un pays désert.
\VS{13}Et il leur dit : Donnez-leur vous-mêmes à manger. Et ils dirent : Nous n'avons que cinq pains et deux poissons ; à moins que nous n'allions nous-mêmes acheter des vivres pour tout ce peuple.
\VS{14}Or, il y avait environ cinq mille hommes. Jésus dit aux disciples : Faites-les asseoir par rangées de cinquante chacune.
\VS{15}Ils le firent ainsi, et les firent tous asseoir.
\VS{16}Jésus prit les cinq pains et les deux poissons, et levant les yeux au ciel, il les bénit. Puis, il les rompit, et il les donna à ses disciples afin qu'ils les distribuent à la foule.
\VS{17}Tous mangèrent et furent rassasiés, et l'on remporta douze paniers pleins de morceaux de pain qui restaient.
\TextTitle{Pierre reconnaît Jésus comme le Messie\FTNTT{Mt. 16:13-16 ; Mc. 8:27-30 ; Jn. 6:66-71}}
\VS{18}Or il arriva que comme il était dans un lieu retiré pour prier, et que les disciples étaient avec lui, il les interrogea, disant : Qui disent les foules que je suis ?
\VS{19}Ils lui répondirent : Les uns disent que tu es Jean-Baptiste ; les autres, Elie ; et les autres, qu'un des anciens prophètes est ressuscité.
\VS{20}Il leur dit alors : Et vous, qui dites-vous que je suis ? Et Pierre répondit : Tu es le Christ de Dieu.
\VS{21}Jésus leur défendit sévèrement de ne le dire à personne.
\TextTitle{Jésus annonce sa mort et sa résurrection\FTNTT{Mt. 16:21-23 ; Mc. 8:31-33}}
\VS{22}Et il leur dit : Il faut que le Fils de l'homme souffre beaucoup, et qu'il soit rejeté par les anciens, par les principaux sacrificateurs et par les scribes, et qu'il soit mis à mort, et qu'il ressuscite le troisième jour.
\TextTitle{La consécration du disciple\FTNTT{Mt. 16:24-28 ; Mc. 8:34-38}}
\VS{23}Puis il dit à tous : Si quelqu'un veut venir après moi, qu'il renonce à lui-même, qu'il se charge chaque jour de sa croix, et qu'il me suive.
\VS{24}Car celui qui voudra sauver sa vie, la perdra ; mais celui qui perdra sa vie à cause de son amour pour moi, la sauvera.
\VS{25}Et que servirait-il à un homme de gagner tout le monde, s'il se détruisait ou se perdait lui-même ?
\VS{26}Car quiconque aura honte de moi et de mes paroles, le Fils de l'homme aura honte de lui quand il viendra dans sa gloire, et dans celle du Père et des saints anges.
\TextTitle{La transfiguration\FTNTT{Mt. 17:1-8 ; Mc. 9:1-8}}
\VS{27}Je vous le dis, en vérité, quelques-uns de ceux qui sont ici présents, ne mourront point qu'ils n'aient vu le Royaume de Dieu\FTNT{Voir commentaire en  Mt. 16:28.}.
\VS{28}Or il arriva environ huit jours après ces paroles, qu'il prit avec lui Pierre, et Jean, et Jacques, et qu'il monta sur une montagne pour prier.
\VS{29}Et comme il priait, l'aspect de son visage changea, et son vêtement devint blanc et resplendissant comme un éclair.
\VS{30}Et voici, deux hommes savoir Moïse et Elie, parlaient avec lui,
\VS{31}et ils apparurent environnés de gloire, et ils parlaient de sa mort\FTNT{Départ : Du grec « exodos », ce qui signifie « départ », « mort », « sortie », « hors de ». Jésus est le prophète de l'Exode dont Moïse a parlé dans De. 18:15.} qu'il allait accomplir à Jérusalem.
\VS{32}Or Pierre et ceux qui étaient avec lui étaient accablés de sommeil ; et quand ils furent réveillés, ils virent sa gloire, et les deux hommes qui étaient avec lui.
\VS{33}Et il arriva qu'au moment où ces hommes se séparaient de Jésus, Pierre dit : Maître, il est bon que nous soyons ici, dressons trois tentes, une pour toi, une pour Moïse, et une pour Elie. Il ne savait pas ce qu'il disait.
\VS{34}Et comme il parlait ainsi, une nuée vint les couvrir de son ombre ; et les disciples furent saisis de frayeur en les voyant entrer dans la nuée.
\VS{35}Et une voix vint de la nuée, disant : Celui-ci est mon Fils bien-aimé ; écoutez-le.
\VS{36}Quand la voix se fit entendre, Jésus se trouva seul. Les disciples gardèrent le silence, et ils ne rapportèrent rien à personne en ce temps-là de ce qu'ils avaient vu.
\TextTitle{Les disciples de Jésus montrent leur limite}
\VS{37}Or il arriva le jour suivant, lorsqu'ils furent descendus de la montagne, une grande foule vint à sa rencontre.
\VS{38}Et voici, du milieu de la foule un homme s'écria : Maître, je t'en prie, porte les regards sur mon fils, car c'est mon fils unique.
\VS{39}Et voici un esprit le saisit, et aussitôt le fait crier, et l'agite avec violence en le faisant écumer, et c'est à peine s'il se retire de lui après l'avoir broyé.
\VS{40}J'ai prié tes disciples de le chasser, mais ils n'ont pas pu.
\VS{41}Jésus répondit : Ô génération incrédule et perverse, jusqu'à quand serai-je avec vous, et vous supporterai-je ? Amène ici ton fils.
\VS{42}Comme il approchait, le démon l'agita violemment comme s'il voulait le déchirer ; mais Jésus menaça fortement l'esprit impur, guérit l'enfant, et le rendit à son père.
\VS{43}Et tous furent étonnés de la puissance magnifique de Dieu. Et comme ils étaient tous dans l'admiration de tout ce que Jésus faisait, il dit à ses disciples :
\TextTitle{Jésus annonce de nouveau sa mort et sa résurrection\FTNTT{Mt. 17:22-23 ; Mc. 9:30-32}}
\VS{44}Vous, écoutez bien ces discours : Le Fils de l'homme sera livré entre les mains des hommes.
\VS{45}Mais les disciples ne comprirent pas cette parole, elle était voilée pour eux, afin qu'ils n'en aient pas le sens ; et ils craignaient de l'interroger à ce sujet.
\TextTitle{L'humilité, le secret de la véritable grandeur\FTNTT{Mt. 18:1-6 ; Mc. 9:33-37}}
\VS{46}Or, une pensée leur vint à l'esprit, savoir lequel d'entre eux était le plus grand.
\VS{47}Mais Jésus voyant la pensée de leur cœur, prit un petit enfant et le mit auprès de lui.
\VS{48}Puis il leur dit : Quiconque reçoit ce petit enfant en mon Nom, me reçoit ; et quiconque me reçoit, reçoit celui qui m'a envoyé. Car celui qui est le plus petit d'entre vous tous, c'est celui-là qui est grand.
\TextTitle{Jésus condamne l'esprit sectaire de Jacques et Jean\FTNTT{Mc. 9:38-40}}
\VS{49}Et Jean prit la parole et dit : Maître, nous avons vu quelqu'un qui chassait les démons en ton Nom, et nous l'en avons empêché parce qu'il ne nous suit pas.
\VS{50}Mais Jésus lui dit : Ne l'en empêchez pas ; car celui qui n'est pas contre nous, est pour nous.
\TextTitle{Mission de Jésus : sauver les âmes}
\VS{51}Lorsque le temps où il devait être enlevé du monde approcha, Jésus prit la résolution d'aller à Jérusalem.
\VS{52}Il envoya devant lui des messagers, qui se mirent en route, et entrèrent dans un bourg des Samaritains, pour lui préparer un logement.
\VS{53}Mais les Samaritains ne le reçurent pas, parce qu'il se dirigeait sur Jérusalem.
\VS{54}Et quand Jacques et Jean, ses disciples virent cela, ils dirent : Seigneur ! Veux-tu que nous commandions que le feu descende du ciel, et les consume, comme fit Elie ?
\VS{55}Mais Jésus se tourna vers eux et les réprimanda fortement, en leur disant : Vous ne savez pas de quel esprit vous êtes animés.
\VS{56}Car le Fils de l'homme n'est pas venu pour perdre les âmes des hommes, mais pour les sauver. Ainsi, ils allèrent dans un autre bourg.
\TextTitle{Epreuves de l'engagement du disciple pour suivre Jésus\FTNTT{Mt. 8:19-22}}
\VS{57}Pendant qu'ils étaient en chemin, un homme lui dit : Seigneur, je te suivrai partout où tu iras.
\VS{58}Mais Jésus lui répondit : Les renards ont des tanières, et les oiseaux du ciel ont des nids, mais le Fils de l'homme n'a pas où reposer sa tête.
\VS{59}Puis il dit à un autre : Suis-moi. Et il répondit : Permets-moi d'aller d'abord ensevelir mon père.
\VS{60}Mais Jésus lui dit : Laisse les morts ensevelir leurs morts ; mais toi, va, et annonce le Royaume de Dieu.
\VS{61}Un autre aussi lui dit : Seigneur, je te suivrai ; mais permets-moi de prendre d'abord congé de ceux de ma maison.
\VS{62}Mais Jésus lui répondit : Quiconque met la main à la charrue, et regarde en arrière, n'est pas bien disposé pour le Royaume de Dieu.
\Chap{10}
\TextTitle{Soixante-dix disciples envoyés en mission}
\VerseOne{}Or après ces choses, le Seigneur désigna soixante-dix autres disciples, et il les envoya deux à deux devant lui, dans toutes les villes et dans tous les lieux où il devait aller.
\VS{2}Il leur dit : La moisson est grande, mais il y a peu d'ouvriers ; priez donc le seigneur de la moisson qu'il pousse des ouvriers dans sa moisson.
\VS{3}Allez, voici, je vous envoie comme des agneaux au milieu des loups.
\VS{4}Ne portez ni bourse, ni sac, ni souliers, et ne saluez personne en chemin.
\VS{5}En quelque maison que vous entriez, dites premièrement : Que la paix soit sur cette maison !
\VS{6}Et s'il y a là quelqu'un qui soit digne de paix, votre paix reposera sur lui ; sinon elle retournera à vous.
\VS{7}Et demeurez dans cette maison, mangeant et buvant de ce qui sera mis devant vous ; car l'ouvrier mérite son salaire. N'allez pas de maison en maison.
\VS{8}Dans quelque ville que vous entriez, et où l'on vous recevra, mangez ce qui sera mis devant vous,
\VS{9}guérissez les malades qui s'y trouveront, et dites-leur : Le Royaume de Dieu s'est approché de vous.
\VS{10}Mais dans quelque ville que vous entriez, et où l'on ne vous recevra pas, sortez dans ses rues et dites :
\VS{11}Nous secouons contre vous-mêmes la poussière de votre ville qui s'est attachée à nous ; toutefois sachez que le Royaume de Dieu s'est approché de vous.
\VS{12}Je vous le dis qu'en ce jour Sodome sera traitée moins rigoureusement que cette ville-là.
\TextTitle{Jésus dénonce les indifférents\FTNTT{Mt. 11:20-24}}
\VS{13}Malheur à toi Chorazin, malheur à toi Bethsaïda ! Car si les miracles qui ont été faits au milieu de vous avaient été faits dans Tyr et dans Sidon, il y a longtemps qu'elles se seraient repenties, couvertes d'un sac, et assises sur la cendre.
\VS{14}C'est pourquoi Tyr et Sidon seront traitées moins rigoureusement que vous au jour du jugement.
\VS{15}Et toi, Capernaüm, qui as été élevée jusqu'au ciel, tu seras précipitée jusque dans le Hadès\FTNT{Voir commentaire en Mt. 16:18}.
\VS{16}Celui qui vous écoute, m'écoute ; et celui qui vous rejette, me rejette. Or celui qui me rejette, rejette celui qui m'a envoyé.
\VS{17}Or les soixante-dix revinrent avec joie, disant : Seigneur, les démons mêmes nous sont soumis en ton Nom.
\VS{18}Jésus leur dit : Je voyais Satan tomber du ciel comme un éclair.
\VS{19}Voici, je vous ai donné le pouvoir de marcher sur les serpents et sur les scorpions, et sur toute la force de l'ennemi ; et rien ne pourra vous nuire.
\VS{20}Toutefois, ne vous réjouissez pas de ce que les esprits vous sont soumis, mais réjouissez-vous plutôt de ce que vos noms sont écrits dans les cieux.
\VS{21}En ce moment même, Jésus se réjouit en esprit, et dit : Je te loue, ô Père ! Seigneur du ciel et de la terre, de ce que tu as caché ces choses aux sages et aux intelligents, et que tu les as révélées aux petits enfants. Oui, Père, parce que telle a été ta bonne volonté.
\VS{22}Toutes choses m'ont été données en main par mon Père ; et personne ne connaît qui est le Fils, si ce n'est le Père ; ni qui est le Père, si ce n'est le Fils et celui à qui le Fils veut le révéler.
\VS{23}Puis, se tournant vers ses disciples, il leur dit en particulier : Heureux sont les yeux qui voient ce que vous voyez.
\VS{24}Car je vous dis que beaucoup de prophètes et de rois ont désiré voir ce que vous voyez, et ne l'ont pas vu, et entendre ce que vous entendez, et ne l'ont pas entendu.
\TextTitle{Un docteur de la loi tente d'éprouver Jésus\FTNTT{cp. Mt. 22:34-40 ; Mc. 12:28-34}}
\VS{25}Alors voici, un docteur de la loi s'étant levé pour l'éprouver lui dit : Maître, que dois-je faire pour avoir la vie éternelle ?
\VS{26}Et il lui dit : Qu'est-il écrit dans la loi ? Qu'y lis-tu ?
\VS{27}Il répondit : Tu aimeras le Seigneur ton Dieu de tout ton cœur, de toute ton âme, de toute ta force, et de toute ta pensée ; et ton prochain comme toi-même.
\VS{28}Jésus lui dit : Tu as bien répondu. Fais cela, et tu vivras.
\VS{29}Mais lui, voulant se justifier, dit à Jésus : Et qui est mon prochain ?
\TextTitle{Parabole du Samaritain}
\VS{30}Jésus reprit la parole et dit : Un homme descendait de Jérusalem à Jéricho. Il tomba entre les mains des brigands, qui le dépouillèrent, le chargèrent de plusieurs coups, et s'en allèrent, le laissant à demi mort.
\VS{31}Un sacrificateur, qui par hasard descendait par le même chemin, ayant vu cet homme, passa outre.
\VS{32}Un Lévite, qui arriva aussi dans ce lieu, l'ayant vu, passa outre.
\VS{33}Mais un Samaritain, qui voyageait, étant venu là, fut ému de compassion lorsqu'il le vit.
\VS{34}Il s'approcha, banda ses plaies, en y versant de l'huile et du vin ; puis le mit sur sa propre monture, et le conduisit à une hôtellerie, et prit soin de lui.
\VS{35}Et le lendemain, en partant il tira de sa bourse deux deniers, et les donna à l'hôte, en lui disant : Aie soin de lui ; et tout ce que tu dépenseras de plus, je te le rendrai à mon retour.
\VS{36}Lequel donc de ces trois te semble-t-il avoir été le prochain de celui qui était tombé entre les mains des brigands ?
\VS{37}Il répondit : C'est celui qui a usé de miséricorde envers lui. Jésus donc lui dit : Va, et toi aussi fais de même.
\TextTitle{Marthe et Marie}
\VS{38}Et il arriva comme ils s'en allaient, qu'il entra dans une bourgade ; et une femme nommée Marthe le reçut dans sa maison.
\VS{39}Elle avait une sœur nommée Marie, qui se tenant assise aux pieds de Jésus, écoutait sa parole.
\VS{40}Mais Marthe était distraite par divers soins domestiques ; et étant venue à Jésus, elle dit : Seigneur, ne te soucies-tu point que ma sœur me laisse servir toute seule, dis-lui donc de m'aider de son côté.
\VS{41}Jésus lui répondit : Marthe, Marthe, tu t'inquiètes et tu t'agites pour beaucoup de choses.
\VS{42}Mais une chose est nécessaire ; et Marie a choisi la bonne part, qui ne lui sera point ôtée.
\Chap{11}
\TextTitle{Enseignement de Jésus sur la prière\FTNTT{cp. Mt. 6:9-15}}
\VerseOne{}Et il arriva, comme il était en prière en un certain lieu, qu'après qu'il eut cessé de prier, un de ses disciples lui dit : Seigneur, enseigne-nous à prier, comme Jean l'a enseigné à ses disciples.
\VS{2}Il leur dit : Quand vous prierez, dites : Notre Père qui es aux cieux ! Que ton Nom soit sanctifié, que ton règne vienne ; que ta volonté soit faite sur la terre comme au ciel.
\VS{3}Donne-nous chaque jour notre pain quotidien.
\VS{4}Et pardonne-nous nos péchés ; car nous aussi, nous remettons les dettes à tous ceux qui nous doivent ; et ne nous induis point en tentation, mais délivre-nous du mal.
\TextTitle{Parabole des trois amis et de la prière importune}
\VS{5}Puis il leur dit : Si l'un de vous a un ami, et qu'il aille le trouver à minuit pour lui dire : Mon ami, prête-moi trois pains,
\VS{6}car un de mes amis est arrivé de voyage chez moi, et je n'ai rien à lui offrir,
\VS{7}et si, de l'intérieur de sa maison, cet ami lui répond : Ne m'importune pas ; car ma porte est déjà fermée, mes enfants et moi nous sommes au lit ; je ne puis me lever pour t'en donner.
\VS{8}Je vous le dis, même s'il ne se levait pas pour les lui donner parce que c'est son ami, il se lèverait à cause de son importunité, et lui donnerait tout ce dont il a besoin.
\VS{9}Ainsi je vous dis : Demandez, et il vous sera donné ; cherchez, et vous trouverez ; frappez, et l'on vous ouvrira.
\VS{10}Car quiconque demande, reçoit ; et celui qui cherche, trouve ; et l'on ouvre à celui qui frappe.
\TextTitle{Parabole du père}
\VS{11}Quel est parmi vous le père qui donnera une pierre à son fils, s'il lui demande du pain ? Ou, s'il lui demande un poisson, lui donnera-t-il un serpent au lieu d'un poisson ?
\VS{12}Ou, s'il demande un œuf, lui donnera-t-il un scorpion ?
\VS{13}Si donc vous qui êtes méchants, vous savez donner à vos enfants des choses bonnes, combien plus le Père qui est du ciel donnera-t-il l'Esprit Saint à ceux qui le lui demandent ,
\TextTitle{Jésus guérit un démoniaque}
\VS{14}Alors il chassa un démon qui était muet. Lorsque le démon fut sorti, le muet parla ; et la foule fut dans l'admiration.
\TextTitle{Le blasphème contre le Saint-Esprit\FTNTT{Mt. 12:24-32 ; Mc. 3:22-30}}
\VS{15}Mais quelques-uns d'entre eux dirent : C'est par Béelzebul\FTNT{Béelzebul : Dans 2 R. 1:2, il est fait mention de « Baal Zebub, dieu d'Eqrôn ». Littéralement, la formule signifie « maître (Baal) des mouches ». Ce mot a une autre signification que le grec de la Septante a adoptée en traduisant par Baal-myia, « Baal-mouche ». Le prince des démons.}, prince des démons, qu'il chasse les démons.
\VS{16}Mais les autres pour l'éprouver, lui demandaient un miracle venant du ciel.
\VS{17}Mais lui, connaissant leurs pensées, leur dit : Tout royaume divisé contre lui-même sera réduit en désert ; et toute maison divisée contre elle-même tombe en ruine.
\VS{18}Si donc Satan est divisé contre lui-même, comment son royaume subsistera-t-il ? Car vous dites que je chasse les démons par Béelzebul.
\VS{19}Et si moi, je chasse les démons par Béelzebul, vos fils par qui les chassent-ils ? C'est pourquoi ils seront eux-mêmes vos juges.
\VS{20}Mais si je chasse les démons par le doigt de Dieu, alors le royaume de Dieu est parvenu jusqu'à vous.
\VS{21}Lorsqu'un homme fort et bien armé garde sa bergerie\FTNT{Bergerie : Chez les Grecs, du temps d'Homère, c'était un espace découvert autour de la maison, fermé par un mur, tandis que chez les Orientaux, il s'agissait d'un espace dans la campagne, entouré d'un mur, où les troupeaux passaient la nuit. La bergerie désigne aussi la partie non couverte d'une maison. Dans la première alliance, il s'agit particulièrement du « parvis » du tabernacle et du temple à Jérusalem. Les demeures des gens de la haute société possédaient généralement deux de ces « cours » : une entre la porte et la rue, l'autre entourée par l'immeuble lui-même. C'est ce qui est mentionné en Mt. 26:69. Enfin, ce terme fait allusion à la maison elle-même, un palais.}, les biens qu'il a sont en sûreté.
\VS{22}Mais si un plus fort que lui survient et le vainque, il lui enlève toutes ses armes dans lesquelles il se confiait, et il partage ses dépouilles.
\VS{23}Celui qui n'est point avec moi est contre moi ; et celui qui n'assemble pas avec moi, il disperse.
\TextTitle{Le retour de l'esprit impur\FTNTT{Mt. 12:43-45}}
\VS{24}Quand l'esprit impur est sorti d'un homme, il va par des lieux secs, cherchant du repos. N'en trouvant point, il dit : Je retournerai dans ma maison, d'où je suis sorti,
\VS{25}et quand il arrive, il la trouve balayée et parée.
\VS{26}Alors il s'en va, et prend avec lui sept autres esprits plus méchants que lui, et ils entrent et demeurent là ; de sorte que la dernière condition de cet homme-là est pire que la première.
\VS{27}Or il arriva comme il disait ces choses, qu'une femme élevant sa voix du milieu de la foule, lui dit : Heureux est le ventre qui t'a porté, et les mamelles que tu as tétées !
\VS{28}Et il répondit : Heureux plutôt ceux qui écoutent la parole de Dieu, et qui la gardent !
\TextTitle{Le signe du prophète Jonas\FTNTT{Mt. 12:38-41}}
\VS{29}Et comme les foules s'amassaient ensemble, il se mit à dire : Cette génération est méchante ; elle demande un miracle, mais il ne lui sera donné d'autre miracle que celui de Jonas le prophète.
\VS{30}Car, de même que Jonas fut un miracle pour les Ninivites, de même le Fils de l'homme en sera un pour cette génération.
\VS{31}La reine du Midi se lèvera au jour du jugement contre les hommes de cette génération et les condamnera, parce qu'elle vint des extrémités de la terre pour entendre la sagesse de Salomon ; et voici, il y a ici plus que Salomon.
\VS{32}Les gens de Ninive se lèveront au jour du jugement contre cette génération et la condamneront, parce qu'ils se sont repentis à la prédication de Jonas ; et voici, il y a ici plus que Jonas.
\TextTitle{Parabole de la lampe\FTNTT{Mt. 5:14-16 ; Mc. 4:21-23 ; cp. Lu. 8:16-18}}
\VS{33}Or personne n'allume une lampe pour la mettre dans un lieu caché ou sous le boisseau, mais sur un chandelier, afin que ceux qui entrent voient la lumière.
\VS{34}La lumière du corps c'est l'œil. Si donc ton œil est sain, tout ton corps aussi sera éclairé ; mais s'il est mauvais, ton corps aussi sera ténébreux.
\VS{35}Prends donc garde que la lumière qui est en toi ne soit pas ténèbres.
\VS{36}Si donc ton corps est éclairé, n'ayant aucune partie dans les ténèbres, il sera entièrement éclairé, comme lorsque la lampe t'éclaire de sa lumière.
\VS{37}Comme il parlait, un pharisien le pria de dîner chez lui. Il entra, et se mit à table.
\VS{38}Mais le pharisien vit avec étonnement qu'il ne s'était pas premièrement lavé avant le dîner.
\TextTitle{Malheurs sur les pharisiens et les docteurs de la loi\FTNTT{cp. Mt. 12:38-41}}
\VS{39}Mais le Seigneur lui dit : Vous autres pharisiens, vous nettoyez le dehors de la coupe et du plat ; et à l'intérieur vous êtes pleins de rapine et de méchanceté.
\VS{40}Insensés, celui qui a fait le dehors, n'a-t-il pas fait aussi le dedans ?
\VS{41}Donnez plutôt en aumône ce qui est dedans, et voici, toutes choses seront pures pour vous.
\VS{42}Mais malheur à vous, pharisiens ! Car vous payez la dîme de la menthe, de la rue\FTNT{La rue : Il s'agit d'un arbuste ayant des propriétés médicinales. Les pharisiens poussaient leur zèle jusqu'à payer la dîme sur certaines herbes. Toutefois, en négligeant la justice et l'amour de Dieu, ils passaient à côté de l'essentiel. Toutes leurs œuvres étaient par conséquent vaines.}, et de toutes sortes d'herbes, et vous négligez la justice et l'amour de Dieu. C'est là ce qu'il fallait pratiquer, sans négliger les autres choses.
\VS{43}Malheur à vous, pharisiens, qui aimez les premières places dans les synagogues, et les salutations sur les places publiques.
\VS{44}Malheur à vous, scribes et pharisiens hypocrites, car vous êtes comme les sépulcres qui ne paraissent pas, et sur lesquels on marche sans le voir.
\VS{45}Alors un des docteurs de la loi prit la parole, et lui dit : Maître, en disant ces choses, tu nous outrages aussi.
\VS{46}Et il dit : À vous aussi, malheur, docteurs de la loi ! Car vous chargez les hommes de fardeaux difficiles à porter, et vous-mêmes vous ne touchez pas ces fardeaux d'un seul de vos doigts.
\VS{47}Malheur à vous, car vous bâtissez les sépulcres des prophètes, que vos pères ont tués.
\VS{48}Vous rendez donc témoignage aux œuvres de vos pères et vous y prenez plaisir ; car eux, ils les ont tués, et vous, vous bâtissez leurs sépulcres.
\VS{49}C'est pourquoi aussi la sagesse de Dieu a dit : Je leur enverrai des prophètes et des apôtres, et ils tueront les uns, et persécuteront les autres,
\VS{50}afin que le sang de tous les prophètes qui a été répandu dès la fondation du monde, soit redemandé à cette nation.
\VS{51}Depuis le sang d'Abel, jusqu'au sang de Zacharie, qui fut tué entre l'autel et le temple. Oui, je vous le dis qu'il sera redemandé à cette nation.
\VS{52}Malheur à vous, docteurs de la loi ! Parce que vous avez enlevé la clef de la science. Vous n'êtes pas entrés vous-mêmes, et vous avez empêché ceux qui entraient.
\VS{53}Et comme il leur disait ces choses, les scribes et les pharisiens commencèrent à le presser violemment, et à le faire parler sur beaucoup de choses,
\VS{54}lui dressant des pièges, et cherchant à tirer quelque chose de sa bouche, afin de l'accuser.
\TextTitle{[Enseignements divers de Jésus]
\\(cp. Mt. 16:6-12 ; Mc. 8:14-21}
\Chap{12}
\VerseOne{}Cependant les gens s'étaient rassemblés par milliers, au point de s'écraser les uns les autres. Jésus se mit à dire à ses disciples : Avant tout, gardez-vous surtout du levain des pharisiens qui est l'hypocrisie.
\VS{2}Car il n'y a rien de caché, qui ne doive être révélé, ni de secret, qui ne doive être connu.
\VS{3}C'est pourquoi tout ce que vous aurez dit dans les ténèbres, sera entendu dans la lumière ; et ce que vous aurez dit à l'oreille dans les chambres, sera prêché sur les toits.
\VS{4}Je vous dis à vous qui êtes mes amis : Ne craignez pas ceux qui tuent le corps, et qui après cela ne peuvent rien faire de plus.
\VS{5}Je vous montrerai qui vous devez craindre. Craignez celui qui, après avoir tué, a le pouvoir de jeter dans la géhenne ; oui, vous dis-je, craignez celui-là.
\VS{6}Ne vend-on pas cinq petits passereaux pour deux sous ? Cependant, aucun d'eux n'est oublié devant Dieu.
\VS{7}Et même les cheveux de votre tête sont tous comptés. Ne craignez donc point ; vous valez plus que beaucoup de passereaux.
\VS{8}Or, je vous dis, quiconque me confessera devant les hommes, le Fils de l'homme le confessera aussi devant les anges de Dieu.
\VS{9}Mais quiconque me reniera devant les hommes, il sera renié devant les anges de Dieu.
\VS{10}Et quiconque parlera contre le Fils de l'homme, il lui sera pardonné ; mais celui qui aura blasphémé contre le Saint-Esprit\FTNT{Voir commentaire en Mt. 12:32.}, il ne lui sera point pardonné.
\VS{11}Quand ils vous mèneront devant les synagogues, les magistrats et les autorités, ne vous inquiétez pas de la manière dont vous vous défendrez ni de ce que vous aurez à dire.
\VS{12}Car le Saint-Esprit vous enseignera à l'heure même ce qu'il faudra dire.
\TextTitle{Parabole du riche insensé}
\VS{13}Et quelqu'un de la foule lui dit : Maître, dis à mon frère qu'il partage avec moi notre héritage.
\VS{14}Mais il lui répondit : Ô homme ! Qui m'a établi sur vous pour être votre juge, et pour faire vos partages ?
\VS{15}Puis il leur dit : Gardez-vous avec soin de toute avarice ; car quoique les biens de quelqu'un abondent, il n'a pas la vie par ses biens.
\VS{16}Et il leur dit cette parabole : Les champs d'un homme riche avaient beaucoup rapporté.
\VS{17}Et il raisonnait en lui-même, disant : Que ferai-je, car je n'ai pas assez de place pour recueillir mes fruits ?
\VS{18}Puis il dit : Voici ce que je ferai : J'abattrai mes greniers et j'en bâtirai de plus grands, et j'y amasserai toute ma récolte et tous mes biens.
\VS{19}Puis je dirai à mon âme : Mon âme, tu as beaucoup de biens assemblés pour beaucoup d'années, repose-toi, mange, bois, et réjouis-toi.
\VS{20}Mais Dieu lui dit : Insensé ! Cette même nuit ton âme te sera redemandée ; et ces choses que tu as préparées, à qui seront-elles ?
\VS{21}Il en est ainsi de celui qui amasse des biens pour lui-même, et qui n'est pas riche en Dieu.
\TextTitle{Exhortation à se confier en Dieu}
\VS{22}Jésus dit à ses disciples : C'est pourquoi je vous dis : Ne vous inquiétez pas pour votre vie, de ce que vous mangerez, ni pour votre corps, de quoi vous serez vêtus.
\VS{23}La vie est plus que la nourriture, et le corps est plus que le vêtement.
\VS{24}Considérez les corbeaux, ils ne sèment, ni ne moissonnent, et ils n'ont point de cellier, ni de grenier, et cependant Dieu les nourrit. Combien ne valez-vous pas plus que les oiseaux ?
\VS{25}Qui de vous qui par ses inquiétudes peut ajouter une coudée à la durée de sa vie ?
\VS{26}Si donc vous ne pouvez pas même la moindre chose, pourquoi êtes-vous inquiets du reste ?
\VS{27}Considérez comment croissent les lis, ils ne travaillent, ni ne filent, et cependant je vous dis que Salomon même, dans toute sa gloire, n'a pas été vêtu comme l'un d'eux.
\VS{28}Si Dieu revêt ainsi l'herbe qui est aujourd'hui au champ, et qui demain sera jetée au four, à combien plus forte raison vous vêtira-t-il, ô gens de petite foi ?
\VS{29}Ne dites donc point : Que mangerons-nous, ou que boirons-nous ? Et ne soyez pas inquiets,
\VS{30}car toutes ces choses, ce sont les païens du monde qui les recherchent. Votre Père sait que vous en avez besoin.
\VS{31}Mais cherchez plutôt le Royaume de Dieu, et toutes ces choses vous seront données par-dessus.
\VS{32}Ne crains point petit troupeau, car il a plu à votre Père de vous donner le Royaume.
\VS{33}Vendez ce que vous avez, et donnez-le en aumône. Faites-vous des bourses qui ne s'usent point, un trésor dans les cieux qui ne défaille jamais, et où le voleur n'approche point, et où la teigne ne gâte rien.
\VS{34}Car là où est votre trésor, là sera aussi votre cœur.
\TextTitle{Importance de veiller en attendant le Maître\FTNTT{Mt. 24:36-25:30}}
\VS{35}Que vos reins soient ceints, et vos lampes allumées.
\VS{36}Et soyez semblables aux serviteurs qui attendent que leur maître revienne des noces, afin de lui ouvrir dès qu'il frappera.
\VS{37}Heureux ces serviteurs que le maître à son arrivée, trouvera veillant ! En vérité je vous le dis, il se ceindra, les fera mettre à table, et s'approchera pour les servir.
\VS{38}Qu'il arrive à la seconde veille ou à la troisième veille, heureux ces serviteurs, s'il les trouve veillant !
\VS{39}Or sachez ceci, si le père de famille savait à quelle heure le voleur doit venir, il veillerait, et ne laisserait pas percer sa maison.
\VS{40}Vous donc aussi tenez-vous prêts, car le Fils de l'homme viendra à l'heure où vous n'y penserez pas.
\TextTitle{Parabole des deux serviteurs}
\VS{41}Pierre lui dit : Seigneur, dis-tu cette parabole pour nous, ou aussi pour tous ?
\VS{42}Et le Seigneur dit : Quel est donc l'économe fidèle et prudent, que le maître établira sur les domestiques de sa maison pour leur donner la nourriture au temps convenable ?
\VS{43}Heureux ce serviteur, que son maître, à son arrivée, trouvera faisant ainsi.
\VS{44}Je vous le dis en vérité, il l'établira sur tous ses biens.
\VS{45}Mais si ce serviteur dit en son cœur : Mon maître tarde longtemps à venir, s'il se met à battre les serviteurs et les servantes, à manger, à boire et à s'enivrer,
\VS{46}le maître de cet esclave-là viendra en un jour qu'il n'attend pas, et à une heure qu'il ne sait pas, et il le coupera en deux\FTNT{Le mot grec « dichotomeo » signifie « couper en deux parts », « de couper quelqu'un en deux », « châtiant en coupant », « fléau sévère ». Certains peuples, dont les Hébreux, employaient cette méthode cruelle comme châtiment corporel.}, et lui donnera sa part avec les infidèles.
\VS{47}Or le serviteur qui a connu la volonté de son maître, et qui ne s'est pas tenu prêt, et n'a point fait selon sa volonté, sera battu de plusieurs coups.
\VS{48} Mais celui qui ne l'a point connue, et qui a fait des choses dignes de châtiment, sera battu de peu de coups. Et il sera beaucoup redemandé à quiconque il aura été beaucoup donné; et on exigera plus de celui à qui on aura beaucoup confié.
\TextTitle{Jésus objet de divisions}
\VS{49}Je suis venu jeter un feu sur la terre, et qu'ai-je à désirer, s'il est déjà allumé ?
\VS{50}Il est un baptême dont je dois être baptisé, et combien suis-je pressé jusqu'à ce qu'il soit accompli.
\VS{51}Pensez-vous que je sois venu apporter la paix sur la terre ? Non, vous dis-je ; mais plutôt la division.
\VS{52}Car désormais cinq dans une maison, seront divisés, trois contre deux, et deux contre trois.
\VS{53}Le père sera divisé contre le fils, et le fils contre le père ; la mère contre la fille, et la fille contre la mère ; la belle-mère contre sa belle-fille, et la belle-fille contre sa belle-mère.
\VS{54}Puis il dit encore aux foules : Quand vous voyez un nuage se lever à l'occident, vous dites aussitôt : La pluie vient, et cela arrive ainsi.
\VS{55}Et quand vous voyez souffler le vent du midi, vous dites qu'il fera chaud ; et cela arrive.
\VS{56}Hypocrites, vous savez bien discerner l'aspect du ciel et de la terre ; et comment ne discernez-vous point cette saison ?
\VS{57}Et pourquoi aussi ne reconnaissez-vous pas de vous-mêmes ce qui est juste ?
\VS{58}Or quand tu vas avec ton adversaire devant le magistrat, tâche en chemin de t'en délivrer, de peur qu'il ne te traîne devant le juge, et que le juge ne te livre à l'officier de justice et que celui-ci ne te mette en prison.
\VS{59}Je te le dis, tu ne sortiras pas de là que tu n'aies payé jusqu'au dernier pite\FTNT{Petite pièce de monnaie en laiton. Voir en annexe le « tableau des monnaies au temps de Jésus-Christ ».}.
\Chap{13}
\TextTitle{Exhortation à la repentance}
\VerseOne{}En ce même temps, quelques-uns qui se trouvaient là présents racontèrent à Jésus ce qui était arrivé à des Galiléens, dont Pilate avait mêlé le sang avec celui de leurs sacrifices.
\VS{2}Et Jésus répondant leur dit : Croyez-vous que ces Galiléens étaient de plus grands pécheurs que tous les autres Galiléens, parce qu'ils ont souffert de la sorte ?
\VS{3}Non, vous dis-je ; mais si vous ne vous repentez pas, vous périrez tous de la même manière.
\VS{4}Ou bien, ces dix-huit personnes sur qui est tombée la tour de Siloé et qu'elle a tuées, croyez-vous qu'elles étaient plus coupables que tous les habitants de Jérusalem ?
\VS{5}Non, vous dis-je ; mais si vous ne vous repentez pas, vous périrez tous de la même manière.
\TextTitle{Parabole du figuier stérile et le jugement différé d'Israël\FTNTT{cp. Mt. 21:18-21}}
\VS{6}Il disait aussi cette parabole : Un homme avait un figuier planté dans sa vigne. Il vint pour y chercher du fruit, mais il n'en trouva point.
\VS{7}Et il dit au vigneron : Voilà trois ans que je viens chercher du fruit à ce figuier, et je n'en trouve point. Coupe-le ; pourquoi occupe-t-il inutilement la terre ?
\VS{8}Et le vigneron lui répondit : Seigneur, laisse-le encore pour cette année, je creuserai tout autour, et j'y mettrai du fumier.
\VS{9}Peut-être portera-t-il du fruit ; sinon, tu le couperas après cela.
\TextTitle{Guérison de la femme courbée le jour du sabbat}
\VS{10}Or comme il enseignait dans une de leurs synagogues un jour de sabbat,
\VS{11}voici, il y avait là une femme qui était possédée d'un démon qui la rendait infirme depuis dix-huit ans, et elle était courbée, et ne pouvait nullement se redresser.
\VS{12}Et quand Jésus la vit, il l'appela, et lui dit : Femme, tu es délivrée de ton infirmité.
\VS{13}Et il lui imposa les mains ; et à l'instant elle se redressa, et glorifia Dieu.
\VS{14}Mais le chef de la synagogue, indigné de ce que Jésus avait opéré cette guérison un jour du sabbat, prenant la parole dit à l'assemblée : Il y a six jours pour travailler ; venez donc vous faire guérir ces jours-là, et non pas le jour du sabbat.
\VS{15}Hypocrites ! lui répondit le Seigneur, chacun de vous ne détache-t-il pas son bœuf ou son âne de la crèche le jour du sabbat, et ne les mène-t-il pas boire ?
\VS{16}Et ne fallait-il pas délier de ce lien le jour du sabbat cette femme qui est fille d'Abraham, et que Satan tenait liée depuis dix-huit ans ?
\VS{17}Comme il disait ces choses, tous ses adversaires étaient confus ; mais toutes les foules se réjouissaient de toutes les choses glorieuses qu'il opérait.
\TextTitle{Parabole du grain de moutarde et du levain\FTNTT{voir Mt. 13:31,33}}
\VS{18}Il disait aussi : A quoi est semblable le Royaume de Dieu, et à quoi le comparerai-je ?
\VS{19}Il est semblable au grain de semence de moutarde qu'un homme a pris et jeté dans son jardin ; il pousse, devient un grand arbre, et les oiseaux du ciel font leurs nids dans ses branches.
\VS{20}Il dit encore : A quoi comparerai-je le Royaume de Dieu ?
\VS{21}Il est semblable au levain qu'une femme a pris et mis dans trois mesures de farine, pour faire lever toute la pâte.
\TextTitle{Enseignements de Jésus sur le chemin de Jérusalem}
\VS{22}Puis il s'en allait par les villes et les villages, enseignant, et faisant route vers Jérusalem.
\VS{23}Quelqu'un lui dit : Seigneur, n'y a-t-il que peu de gens qui soient sauvés ? Il leur répondit :
\VS{24}Efforcez-vous d'entrer par la porte étroite. Car je vous le dis que beaucoup chercheront à entrer, et ne le pourront pas.
\VS{25}Quand le père de famille se sera levé, et aura fermé la porte, et que vous, étant dehors, vous vous mettrez à frapper à la porte, en disant : Seigneur ! Seigneur ! Ouvre-nous ! Il vous répondra : Je ne sais pas d'où vous êtes.
\VS{26}Alors vous vous mettrez à dire : Nous avons mangé et bu en ta présence, et tu as enseigné dans nos rues.
\VS{27}Mais il dira : Je vous le dis, je ne sais pas d'où vous êtes. Retirez-vous de moi, vous tous qui faites le métier d'iniquité.
\VS{28}C'est là qu'il y aura des pleurs et des grincements de dents, quand vous verrez Abraham, Isaac, et Jacob, et tous les prophètes dans le Royaume de Dieu, et que vous serez jetés dehors.
\VS{29}Il en viendra aussi d'orient et d'occident, du nord et du sud, et ils se mettront à table dans le Royaume de Dieu.
\VS{30}Et voici, ceux qui sont les derniers seront les premiers, et ceux qui sont les premiers seront les derniers.
\VS{31}En ce même jour, quelques pharisiens vinrent à lui et lui dirent : Retire-toi et va-t'en d'ici, car Hérode veut te tuer.
\VS{32}Il leur répondit : Allez, et dites à ce renard : Voici, je chasse les démons et j'achève de faire des guérisons aujourd'hui et demain, et le troisième jour je prends fin.
\VS{33}C'est pourquoi il me faut marcher aujourd'hui et demain, et le jour suivant ; car il ne convient pas qu'un prophète meure hors de Jérusalem.
\TextTitle{Lamentations de Jésus sur Jérusalem\FTNTT{Mt. 23:37-39 ; Lu. 19:41-44 ; cp. Jé. 22:5}}
\VS{34}Jérusalem, Jérusalem, qui tues les prophètes et qui lapides ceux qui te sont envoyés ; combien de fois ai-je voulu rassembler tes enfants, comme la poule rassemble ses poussins sous ses ailes, et vous ne l'avez pas voulu !
\VS{30}Voici, votre maison va être déserte ; et je vous le dis en vérité, que vous ne me verrez plus, jusqu'à ce que vous disiez : Béni soit celui qui vient au nom du Seigneur.
\Chap{14}
\TextTitle{Jésus guérit un hydropique le jour du sabbat\FTNTT{cp. Mt. 12:9-13}}
\VerseOne{}Jésus entra un jour de sabbat dans la maison d'un des chefs des pharisiens pour prendre un repas, les pharisiens l'observaient.
\VS{2}Et voici, un homme hydropique était là devant lui.
\VS{3}Jésus prit la parole, et dit aux docteurs de la loi et aux pharisiens : Est-il permis, ou non, de faire une guérison le jour du sabbat ?
\VS{4}Ils gardèrent le silence. Alors Jésus prit le malade, le guérit, et le renvoya.
\VS{5}Puis s'adressant à lui, il leur dit : Lequel de vous, si son fils ou son bœuf tombe dans un puits, ne l'en retirera pas aussitôt, le jour du sabbat ?
\VS{6}Et ils ne pouvaient répliquer à ces choses.
\TextTitle{Parabole de l'invité}
\VS{7}Il proposa aussi aux conviés une parabole, en voyant qu'ils choisissaient les premières places ; et il leur dit :
\VS{8}Quand tu seras convié par quelqu'un à des noces, ne te mets pas à la première place à table, de peur qu'il ne se trouve parmi les conviés une personne plus honorable que toi,
\VS{9}et que celui qui vous a conviés l'un et l'autre ne vienne te dire : Cède ta place à cette personne-là. Tu aurais alors honte d'aller occuper la dernière place.
\VS{10}Mais lorsque tu seras convié, va te mettre à la dernière place, afin que quand celui qui t'a convié viendra, il te dise : Mon ami, monte plus haut. Alors cela te fera honneur devant tous ceux qui seront à table avec toi.
\VS{11}Car quiconque s'élève, sera abaissé ; et quiconque s'abaisse, sera élevé.
\VS{12}Il dit aussi à celui qui l'avait convié : Lorsque tu fais un dîner ou un souper, n'invite pas tes amis, ni tes frères, ni tes parents, ni tes riches voisins ; de peur qu'ils ne te convient à leur tour, et qu'on ne te rende la pareille.
\VS{13}Mais, lorsque tu donneras un festin, convie les pauvres, les impotents, les boiteux et les aveugles.
\VS{14}Et tu seras heureux de ce qu'ils n'ont pas de quoi te rendre la pareille ; car elle te sera rendue à la résurrection des justes.
\TextTitle{Parabole du grand festin\FTNTT{Mt. 22:1-14}}
\VS{15}Un de ceux qui étaient à table, ayant entendu ces paroles, lui dit : Heureux celui qui mangera du pain dans le Royaume de Dieu.
\VS{16}Et Jésus lui répondit : Un homme fit un grand festin, et il convia beaucoup de gens.
\VS{17}Et à l'heure du souper, il envoya son serviteur pour dire aux conviés : Venez, car tout est déjà prêt.
\VS{18}Mais ils commencèrent tous unanimement à s'excuser. Le premier lui dit : J'ai acheté un champ, et il me faut nécessairement partir pour aller le voir ; je te prie, excuse-moi.
\VS{19}Un autre dit : J'ai acheté cinq paires de bœufs, et je vais les essayer ; je te prie, excuse-moi.
\VS{20}Et un autre dit : J'ai épousé une femme, c'est pourquoi je ne puis aller.
\VS{21}Le serviteur, de retour, rapporta ces choses à son maître. Alors le père de famille irrité, dit à son serviteur : Va promptement dans les places et dans les rues de la ville, et amène ici les pauvres, les impotents, les boiteux et les aveugles.
\VS{22}Puis le serviteur dit : Maître, ce que tu as commandé a été fait, et il y a encore de la place.
\VS{23}Et le maître dit au serviteur : Va dans les chemins et le long des haies, et ceux que tu trouveras, contrains-les d'entrer, afin que ma maison soit remplie.
\VS{24}Car je vous dis, qu'aucun de ces hommes qui avaient été conviés ne goûtera de mon souper.
\TextTitle{Test de la consécration du disciple}
\VS{25}Or de grandes foules faisaient route avec Jésus. Il se retourna et leur dit :
\VS{26}Si quelqu'un vient à moi, et ne hait pas son père et sa mère, sa femme et ses enfants, ses frères et ses sœurs, et même sa propre vie, il ne peut être mon disciple.
\VS{27}Et quiconque ne porte pas sa croix, et ne me suit pas, ne peut être mon disciple.
\TextTitle{Parabole de la tour}
\VS{28}Car lequel de vous, s'il veut bâtir une tour, ne s'assied pas premièrement pour calculer la dépense et voir s'il a de quoi l'achever ?
\VS{29}De peur qu'après avoir posé les fondements, il ne puisse pas l'achever, et que tous ceux qui le verront ne commencent à se moquer de lui,
\VS{30}en disant : Cet homme a commencé à bâtir, et il n'a pas pu achever.
\TextTitle{Parabole du roi qui se prépare à la guerre}
\VS{31}Ou, quel roi, s'il va faire la guerre à un autre roi, ne s'assied pas premièrement pour examiner s'il peut, avec dix mille hommes, aller à la rencontre de celui qui vient contre lui avec vingt mille ?
\VS{32}Autrement, pendant que cet autre roi est encore loin, il lui envoie une ambassade pour demander la paix.
\VS{33}Ainsi donc, quiconque d'entre vous ne renonce pas à tout ce qu'il possède ne peut être mon disciple.
\TextTitle{Parabole du sel}
\VS{34}Le sel est bon ; mais si le sel perd sa saveur, avec quoi l'assaisonnera-t-on ?
\VS{35}Il n'est bon ni pour la terre, ni pour le fumier ; mais on le jette dehors. Que celui qui a des oreilles pour entendre, qu'il entende !
\Chap{15}
\TextTitle{Trois paraboles sur la repentance}
\VerseOne{}Or tous les publicains et les pécheurs s'approchaient de Jésus pour l'entendre.
\VS{2}Mais les pharisiens et les scribes murmuraient, disant : Cet homme reçoit les pécheurs, et mange avec eux.
\TextTitle{Parabole de la brebis perdue\FTNTT{Mt. 18:12-14}}
\VS{3}Mais il leur proposa cette parabole, disant :
\VS{4}Lequel d'entre vous, s'il a cent brebis, et qu'il en perd une, ne laisse pas les quatre-vingt-dix-neuf dans le désert, pour aller à la recherche de celle qui est perdue, jusqu'à ce qu'il la trouve ?
\VS{5}Et l'ayant retrouvée, il la met avec joie sur ses épaules,
\VS{6}et, de retour à la maison, il appelle ses amis et ses voisins, et il leur dit : Réjouissez-vous avec moi ; car j'ai trouvé ma brebis qui était perdue.
\VS{7}De même, je vous le dis il y aura plus de joie dans le ciel pour un seul pécheur qui se repent, que pour les quatre-vingt-dix-neuf justes qui n'ont pas besoin de repentance.
\TextTitle{Parabole de la drachme perdue}
\VS{8}Ou quelle femme, si elle a dix drachmes, et qu'elle en perde une, n'allume pas une lampe, ne balaie la maison, et ne cherche avec soin, jusqu'à ce qu'elle la trouve ?
\VS{9}Lorsqu'elle l'a trouvée, elle appelle ses amies et ses voisines, en leur disant : Réjouissez-vous avec moi ; car j'ai trouvé la drachme que j'avais perdue.
\VS{10}Ainsi je vous le dis, il y a de la joie devant les anges de Dieu pour un seul pécheur qui vient à se repentir.
\TextTitle{Parabole du fils perdu}
\VS{11}Il leur dit aussi : Un homme avait deux fils ;
\VS{12}et le plus jeune dit à son père : Mon père, donne-moi la part de bien qui m'appartient ; et il leur partagea ses biens.
\VS{13}Et peu de jours après, le plus jeune fils, ayant tout ramassé, partit pour un pays éloigné, où il dissipa son bien en vivant dans la débauche.
\VS{14}Et après qu'il eut tout dépensé, une grande famine survint dans ce pays-là, et il commença à se trouver dans la disette.
\VS{15}Alors il alla se mettre au service d'un des habitants du pays, qui l'envoya dans ses possessions pour paître les pourceaux.
\VS{16}Il aurait bien voulu se rassasier des carouges que les pourceaux mangeaient ; mais personne ne lui en donnait.
\VS{17}Or étant revenu à lui-même, il dit : Combien d'ouvriers chez mon père ont du pain en abondance, et moi je meurs de faim !
\VS{18}Je me lèverai, j'irai vers mon père, et je lui dirai : Mon père, j'ai péché contre le ciel et devant toi ;
\VS{19}et je ne suis plus digne d'être appelé ton fils ; traite-moi comme l'un de tes ouvriers.
\VS{20}Il se leva donc, et alla vers son père. Et comme il était encore loin, son père le vit et fut ému de compassion, il courut se jeter à son cou et le baisa.
\VS{21}Mais le fils lui dit : Mon père, j'ai péché contre le ciel et devant toi ; et je ne suis plus digne d'être appelé ton fils.
\VS{22}Et le père dit à ses serviteurs : Apportez la plus belle robe et revêtez-le, mettez-lui un anneau au doigt, et des souliers aux pieds.
\VS{23}Amenez-moi le veau gras, et tuez-le. Mangeons et réjouissons-nous.
\VS{24}Car mon fils que voici, était mort, mais il est ressuscité ; il était perdu, mais il est retrouvé. Et ils commencèrent à se réjouir.
\VS{25}Or son fils aîné était dans les champs. Lorsqu'il revint et approcha de la maison, il entendit la musique et les danses.
\VS{26}Il appela un des serviteurs, et il lui demanda ce que c'était.
\VS{27}Ce serviteur lui dit : Ton frère est de retour, et ton père a tué le veau gras, parce qu'il l'a recouvré sain et sauf.
\VS{28}Mais il se mit en colère, et ne voulut pas entrer. Son père sortit et le pria d'entrer.
\VS{29}Mais il répondit, et dit à son père : voici, il y a tant d'années que je te sers, et jamais je n'ai transgressé ton commandement, et cependant tu ne m'as jamais donné un chevreau pour que je me réjouisse avec mes amis.
\VS{30}Mais quand ton fils est arrivé, celui qui a mangé ton bien avec des prostituées, c'est pour lui que tu as tué le veau gras.
\VS{31}Et le père lui dit : Mon enfant, tu es toujours avec moi, et tous mes biens sont à toi.
\VS{32}Or il fallait bien s'égayer et se réjouir, parce que ton frère que voici était mort et qu'il est ressuscité, parce qu'il était perdu et qu'il est retrouvé.
\Chap{16}
\TextTitle{Parabole de l'économe infidèle}
\VerseOne{}Il disait aussi à ses disciples : Il y avait un homme riche qui avait un économe, qui fut accusé devant lui comme dissipant ses biens.
\VS{2}Il l'appela et lui dit : Qu'est-ce que j'entends dire de toi ? Rends compte de ton administration ; car tu n'auras plus le pouvoir d'administrer mes biens.
\VS{3}Alors l'économe dit en lui-même : Que ferai-je, puisque mon maître m'ôte l'administration ? Travailler à la terre ? Je ne le puis. Mendier ? J'en ai honte.
\VS{4}Je sais ce que je ferai, afin que les gens me reçoivent dans leurs maisons quand mon administration me sera ôtée.
\VS{5}Alors il appela chacun des débiteurs de son maître, et il dit au premier : Combien dois-tu à mon maître ?
\VS{6}Il dit : Cent mesures d'huile. Et il lui dit : Prends ton billet, et assied-toi vite, et écris cinquante.
\VS{7}Puis il dit à un autre : Et toi, combien dois-tu ? Il dit : Cent mesures de froment. Et il lui dit : Prends ton billet, et écris quatre-vingts.
\VS{8}Et le maître loua l'économe infidèle de ce qu'il avait agi prudemment. Ainsi les enfants de ce siècle sont plus prudents dans leur génération, que les enfants de lumière.
\VS{9}Et moi aussi je vous dis : Faites-vous des amis avec les richesses injustes ; afin que quand vous viendrez à manquer, ils vous reçoivent dans les tabernacles éternels.
\VS{10}Celui qui est fidèle en très peu de chose, est fidèle aussi dans les grandes choses ; et celui qui est injuste en très peu de chose, est injuste aussi dans les grandes choses.
\VS{11}Si donc vous n'avez pas été fidèles dans les richesses injustes, qui vous confiera les véritables richesses ?
\VS{12}Et si en ce qui est à autrui vous n’avez pas été fidèles, qui vous donnera ce qui est vôtre ?
\VS{13}Nul serviteur ne peut servir deux maîtres. Car, ou il haïra l'un, et aimera l'autre ; ou il s'attachera à l'un, et méprisera l'autre. Vous ne pouvez pas servir Dieu et Mamon\FTNT{Voir commentaire en Mt. 6:24.}.
\TextTitle{L'avarice condamnée par Jésus}
\VS{14}Or les pharisiens aussi, qui étaient avares, entendaient toutes ces choses, et ils se moquaient de lui.
\VS{15}Et il leur dit : Vous, vous cherchez à paraître justes devant les hommes ; mais Dieu connaît vos cœurs ; c'est pourquoi ce qui est élevé parmi les hommes est une abomination devant Dieu.
\VS{16}La loi et les prophètes ont duré jusqu'à Jean ; depuis lors, le Royaume de Dieu est prêché, et chacun y fait violence.
\VS{17}Or il est plus aisé que le ciel et la terre passent, qu'il ne l'est qu'un trait de la lettre de la loi vienne à tomber.
\TextTitle{Enseignement de Jésus sur le divorce\FTNTT{Mt. 5:31-32 ; 19:1-9 ; Mc. 10:2-12}}
\VS{18}Quiconque répudie sa femme, et se marie à une autre, commet un adultère, et quiconque prend celle qui a été répudiée par son mari, commet un adultère.
\TextTitle{Histoire de l'homme riche et de Lazare}
\VS{19}Il y avait un homme riche, qui était vêtu de pourpre et de fin lin, et qui tous les jours se réjouissait d'une vie somptueuse.
\VS{20}Il y avait un pauvre, nommé Lazare, couché à la porte du riche, tout couvert d'ulcères,
\VS{21}et qui désirait se rassasier des miettes qui tombaient de la table du riche ; et même les chiens venaient encore lécher ses ulcères.
\VS{22}Et il arriva que le pauvre mourut, et il fut porté par les anges dans le sein d'Abraham\FTNT{Contrairement aux idées reçues, le sein d'Abraham ne se trouvait pas au ciel. En effet, le Seigneur a dit que personne n'était monté au ciel si ce n'est lui-même (Jn. 3:13). Avant l'ère de la grâce, tous les morts allaient dans le séjour des morts où ils étaient retenus prisonniers par le dieu Hadès (voir commentaire sur l'enfer en Mt. 16:18). Toutefois, ce lieu était séparé en deux parties distinctes, l'une réservée aux impies, où ils y subissaient des tourments, et l'autre réservée aux personnes pieuses qui se tenaient en repos, sans souffrir. En effet, lorsque Saül fit appel à une voyante pour faire remonter Samuel du séjour des morts afin de le consulter, Samuel lui annonça qu'il le rejoindrait dès le lendemain à l'endroit où il se trouvait (1 S. 28:19). De plus, dans le récit de la mort du pauvre Lazare et du riche, deux points importants sont à noter. D'une part, bien qu'étant séparés l'un de l'autre, ils pouvaient se voir et communiquer ensemble (Lu. 16:23-26). D'autre part, il est évident que le riche souffrait tandis que le pauvre Lazare était consolé (Lu. 16:25). Lorsque le Seigneur est mort, il est descendu dans « les régions inférieures de la terre » pour délivrer les captifs pieux qui avaient vécu avant Jésus-Christ (Ep. 4:8-9 ; 1 S. 2:6). Par la même occasion, il confirma la condamnation des impies (1 Pi. 3:19). Maintenant que Jésus-Christ est mort et ressuscité, tous ceux qui meurent dans le Seigneur vont au ciel (2 Co. 5:1-3. ; Ph. 1:22-23).}. Le riche mourut aussi, et il fut enseveli.
\VS{23}Etant en enfer\FTNT{Le mot traduit par « enfer » vient du grec « Hades ». Voir commentaire Mt. 16:18}, il leva ses yeux ; et, tandis qu'il était dans les tourments, il vit de loin Abraham et Lazare dans son sein.
\VS{24}Il s'écria : Père Abraham aie pitié de moi, et envoie Lazare, pour qu'il trempe le bout de son doigt dans l'eau et me rafraichisse la langue ; car je suis grièvement tourmenté dans cette flamme.
\VS{25}Abraham répondit : Mon enfant, souviens-toi que tu as reçu tes biens pendant ta vie, et que Lazare a eu ses maux pendant la sienne ; maintenant il est ici consolé, et toi, tu es grièvement tourmenté.
\VS{26}D'ailleurs, il y a entre nous et vous un grand abîme ; en sorte que ceux qui veulent passer d'ici vers vous ne le peuvent, et que ceux qui veulent passer de là ne traversent pas non plus vers nous.
\VS{27}Et il dit : Je te prie donc, père, de l'envoyer dans la maison de mon père ; car j'ai cinq frères.
\VS{28}Afin qu'il leur rende témoignage de l'état où je suis ; de peur qu'eux aussi ne viennent dans ce lieu de tourment.
\VS{29}Abraham lui répondit : Ils ont Moïse et les prophètes ; qu'ils les écoutent.
\VS{30}Mais il dit : Non, père Abraham, mais si quelqu'un des morts va vers eux, ils se repentiront.
\VS{31}Et Abraham lui dit : S'ils n'écoutent pas Moïse et les prophètes, ils ne seront pas non plus persuadés quand quelqu'un des morts ressusciterait.
\Chap{17}
\TextTitle{Instructions de Jésus au sujet des scandales, du pardon et de la foi\FTNTT{Mt. 5:31-32 ; 19:1-9 ; Mc. 10:2-12}}
\VerseOne{}Or il dit à ses disciples : Il est impossible qu'il n'arrive pas des scandales ; mais malheur à celui par qui ils arrivent.
\VS{2}Il vaudrait mieux pour lui qu'on lui mette une pierre de moulin autour de son cou, et qu'on le jette dans la mer, que de scandaliser un seul de ces petits.
\VS{3}Prenez garde à vous-mêmes. Si donc ton frère a péché contre toi, reprends-le ; et s'il se repent, pardonne-lui.
\VS{4}Et s'il a péché contre toi sept fois dans un jour et que sept fois il revienne à toi, disant : Je me repens, tu lui pardonneras.
\VS{5}Alors les apôtres dirent au Seigneur : Augmente-nous la foi.
\VS{6}Et le Seigneur dit : Si vous aviez de la foi aussi gros qu'un grain de semence de moutarde, vous diriez à ce sycomore : Déracine-toi, et plante-toi dans la mer ; et il vous obéirait.
\TextTitle{Les serviteurs inutiles}
\VS{7}Mais qui de vous, ayant un serviteur qui laboure ou paît les troupeaux, lui dira, quand il revient des champs : Approche-toi vite, et mets-toi à table.
\VS{8}Ne lui dira-t-il pas plutôt : Prépare-moi à souper, ceins-toi, et sers-moi jusqu'à ce que j'aie mangé et bu ; et après cela tu mangeras et tu boiras ?
\VS{9}Doit-il de la reconnaissance à ce serviteur parce qu'il a fait ce qui lui était ordonné ? Je ne le pense pas.
\VS{10}Vous de même, quand vous aurez fait tout ce qui vous a été ordonné, dites : Nous sommes des serviteurs inutiles ;  ce que nous étions obligés de faire, nous l'avons fait.
\TextTitle{Guérison de dix lépreux}
\VS{11}Et il arriva qu’en allant à Jérusalem, il passait par le milieu de la Samarie, et de la Galilée.
\VS{12}Et comme il entrait dans un village, dix hommes lépreux vinrent à sa rencontre. Se tenant à distance, ils élevèrent la voix, et dirent :
\VS{13}Jésus, Maître, aie pitié de nous !
\VS{14}Et quand il les eut vus, il leur dit : Allez, montrez-vous aux sacrificateurs\FTNT{Lé. 13.}. Et, pendant qu'ils y allaient, ils furent purifiés.
\VS{15}L'un d'eux se voyant guéri, revint sur ses pas, glorifiant Dieu à haute voix.
\VS{16}Et il se jeta en terre sur sa face aux pieds de Jésus, lui rendant grâces. Or c’était un Samaritain.
\VS{17}Alors Jésus prenant la parole, dit : Les dix n'ont-ils pas été rendus purs ? Et les neuf autres, où sont-ils ?
\VS{18}Il n'y a eu que cet étranger qui soit revenu pour rendre gloire à Dieu.
\VS{19}Alors il lui dit : Lève-toi. Va, ta foi t'a sauvé.
\TextTitle{Les pharisiens demandent à voir le Royaume de Dieu\FTNTT{cp. Lu. 19:11-27}}
\VS{20}Or les pharisiens demandèrent à Jésus quand viendrait le Royaume de Dieu. Il leur répondit, et leur dit : Le Royaume de Dieu ne vient pas de manière à attirer l'attention.
\VS{21}Et on ne dira point : Il est ici ; ou : Il est là. Car voici, le Royaume de Dieu est au milieu de vous.
\TextTitle{Jésus annonce sa seconde venue\FTNTT{voir De. 30:3}}
\VS{22}Il dit aussi à ses disciples : Des jours viendront où vous désirerez voir un des jours du Fils de l'homme, mais vous ne le verrez point. On vous dira :
\VS{23}Il est ici, ou : Il est là. N'allez pas, et ne les suivez point.
\VS{24}Car, comme l'éclair brille et resplendit d'une extrémité du ciel à l'autre, ainsi sera le Fils de l'homme en son jour.
\VS{25}Mais il faut premièrement qu'il souffre beaucoup, et qu'il soit rejeté par cette génération.
\VS{26}Ce qui arriva aux jours de Noé, arrivera de même aux jours du Fils de l'homme.
\VS{27}On mangeait et on buvait ; on prenait et on donnait des femmes en mariage jusqu'au jour où Noé entra dans l'arche ; le déluge vint, et les fit tous périr.
\VS{28}C’est encore ce qui arriva aux jours de Lot : On mangeait, on buvait, on achetait, on vendait, on plantait et on bâtissait.
\VS{29}Mais le jour où Lot sortit de Sodome, une pluie de feu et de soufre tomba du ciel, et les fit tous périr.
\VS{30}Il en sera de même au jour où le Fils de l'homme paraîtra.
\VS{31}En ce jour-là, que celui qui sera sur le toit, et qui aura ses effets dans la maison, ne descende point pour les prendre ; et que celui qui sera dans les champs, ne retourne pas non plus à ce qui est resté en arrière.
\VS{32}Souvenez-vous de la femme de Lot.
\VS{33}Quiconque cherchera à sauver sa vie, la perdra ; et quiconque la perdra, la retrouvera.
\VS{34}Je vous dis, qu’en cette nuit-là deux seront dans un même lit : l’un sera pris, et l’autre laissé ;
\VS{35}deux femmes moudront ensemble, l’une sera prise et l’autre laissée ;
\VS{36}deux seront aux champs, l’un sera pris et l’autre laissé.
\VS{37}Les disciples lui dirent : Où, Seigneur ? Et il leur dit, Là où est le corps, là aussi s’assembleront les aigles.
\Chap{18}
\TextTitle{Parabole du juge inique}
\VerseOne{}Et il leur proposa une parabole, pour montrer qu'il faut toujours prier, et ne point se relâcher,
\VS{2}disant : Il y avait dans une ville un juge qui ne craignait point Dieu et qui ne respectait personne.
\VS{3}Et dans la même ville, il y avait une veuve, qui venait souvent lui dire : Fais-moi justice de ma partie adverse.
\VS{4}Pendant longtemps il refusa. Mais après cela il dit en lui-même : Quoique je ne craigne point Dieu, et que je ne respecte personne,
\VS{5}néanmoins, parce que cette veuve me donne de la peine, je lui ferai justice, de peur qu'elle ne vienne sans cesse me casser la tête.
\VS{6}Et le Seigneur dit : Ecoutez ce que dit le juge inique.
\VS{7}Et Dieu ne ferait-il point justice à ses élus, qui crient à lui jour et nuit, quoiqu’il use de patience avant d’intervenir pour eux ?
\VS{8}Je vous le dis que bientôt il les vengera. Mais quand le Fils de l'homme viendra, pensez-vous qu'il trouvera la foi sur la terre ?
\TextTitle{Parabole du pharisien et du publicain}
\VS{9}Il dit aussi cette parabole au sujet de certaines personnes se persuadant qu'elles étaient justes, et ne faisant aucun cas des autres :
\VS{10}Deux hommes montèrent au temple pour prier, l'un était pharisien, et l'autre, publicain.
\VS{11}Le pharisien, se tenant debout, priait en lui-même en ces termes : Ô Dieu ! Je te rends grâces de ce que je ne suis pas comme le reste des hommes, qui sont ravisseurs, injustes, adultères, ni même comme ce publicain.
\VS{12}Je jeûne deux fois la semaine, et je donne la dîme de tout ce que je possède.
\VS{13}Mais le publicain se tenant loin, n'osait même pas lever les yeux vers le ciel, mais il se frappait la poitrine, en disant : Ô Dieu ! Sois apaisé envers moi qui suis pécheur !
\VS{14}Je vous dis que celui-ci descendit dans sa maison justifié, plutôt que l'autre ; car quiconque s'élève, sera abaissé, et quiconque s'abaisse, sera élevé.
\TextTitle{Le Royaume des cieux, pour ceux qui ressemblent aux petits enfants\FTNTT{Mt. 19:13-15 ; Mc. 10:13-16}}
\VS{15}Et quelques-uns lui présentèrent aussi de petits enfants, afin qu'il les touchât, mais les disciples voyant cela, reprenaient ceux qui les présentaient.
\VS{16}Mais Jésus les appela, et dit : Laissez venir à moi les petits enfants, et ne les en empêchez pas ; car le Royaume de Dieu est pour ceux qui leur ressemblent.
\VS{17}Je vous le dis en vérité, quiconque ne recevra point comme un enfant le Royaume de Dieu, n’y entrera point.
\TextTitle{Jésus dénonce l'attachement aux richesses\FTNTT{Mt. 19:16-30 ; Mc. 10:17-31 ; cp. Lu. 10:25-37}}
\VS{18}Un chef interrogea Jésus et dit : Bon Maître, que dois-je faire pour hériter la vie éternelle ?
\VS{19}Jésus lui dit : Pourquoi m'appelles-tu bon ? Il n'y a de bon que Dieu seul\FTNT{Voir commentaire Mc. 10:18.}.
\VS{20}Tu connais les commandements : Tu ne commettras point d'adultère. Tu ne tueras point. Tu ne déroberas point. Tu ne diras point de faux témoignage. Honore ton père et ta mère.
\VS{21}Et il lui dit : J'ai observé toutes ces choses dès ma jeunesse.
\VS{22}Et quand Jésus eut entendu cela, lui dit : Il te manque encore une chose : Vends tout ce que tu as, et distribue-le aux pauvres, et tu auras un trésor dans les cieux. Puis viens, et suis-moi.
\VS{23}Lorsqu'il entendit ces choses, il devint tout triste, car il était extrêmement riche.
\VS{24}Jésus voyant qu'il était devenu tout triste, dit : Qu'il est difficile à ceux qui ont des richesses d'entrer dans le Royaume de Dieu !
\VS{25}Car il est plus facile à un chameau de passer par le trou d'une aiguille, qu'à un riche d'entrer dans le Royaume de Dieu\FTNT{Voir commentaire Mt. 19:24.}.
\VS{26}Ceux qui entendirent cela, dirent : Et qui peut donc être sauvé ?
\VS{27}Jésus leur répondit : Ce qui est impossible aux hommes est possible à Dieu.
\TextTitle{Récompense pour un vrai disciple de Jésus}
\VS{28}Pierre dit : Voici, nous avons tout quitté, et nous t'avons suivi.
\VS{29}Et il leur dit : Je vous le dis en vérité, il n'est personne qui, ayant quitté pour l'amour du Royaume de Dieu, sa maison, ou ses parents, ou ses frères, ou sa femme, ou ses enfants,
\VS{30}ne reçoive beaucoup plus dans ce siècle-ci, et dans le siècle à venir la vie éternelle.
\TextTitle{Jésus annonce à nouveau sa mort et sa résurrection\FTNTT{Mt. 20:17-19 ; Mc. 10:32-34}}
\VS{31}Jésus prit à part les douze, et il leur dit : Voici, nous montons à Jérusalem, et tout ce qui est écrit par les prophètes au sujet du Fils de l'homme, s'accomplira.
\VS{32}Car il sera livré aux Gentils ; on se moquera de lui, on l'outragera, et on lui crachera au visage,
\VS{33}et après l'avoir battu de verges, on le fera mourir ; mais il ressuscitera le troisième jour.
\VS{34}Mais ils ne comprirent rien à cela, et ce discours était si obscur pour eux qu'ils ne comprirent point ce qu'il leur disait.
\TextTitle{Bartimée voit !\FTNTT{cp. Mt. 20:29-34 ; Mc. 10:46-53}}
\VS{35}Or comme il approchait de Jéricho, un aveugle était assis au bord du chemin, et mendiait.
\VS{36}Et entendant la foule qui passait, il demanda ce que c'était.
\VS{37}Et on lui dit : C'est Jésus de Nazareth qui passe.
\VS{38}Alors il cria, disant : Jésus, Fils de David, aie pitié de moi !
\VS{39}Ceux qui marchaient devant le reprenaient, pour le faire taire ; mais il criait beaucoup plus fort : Fils de David, aie pitié de moi !
\VS{40}Et Jésus s'étant arrêté ordonna qu'on le lui amène ; et, quand il se fut approché,
\VS{41}il lui demanda : Que veux-tu que je te fasse ? Il répondit : Seigneur, que je recouvre la vue.
\VS{42}Jésus lui dit : Recouvre la vue ; ta foi t'a sauvé.
\VS{43}Et à l'instant il recouvra la vue et suivit Jésus, glorifiant Dieu. Et tout le peuple voyant cela, loua Dieu.
\Chap{19}
\TextTitle{Conversion de Zachée}
\VerseOne{}Jésus, étant entré dans Jéricho, traversait la ville.
\VS{2}Et voici, un homme riche, appelé Zachée, chef des publicains, cherchait à voir qui était Jésus,
\VS{3}mais il ne le pouvait pas à cause de la foule, car il était de petite taille.
\VS{4}C'est pourquoi il accourut devant, et monta sur un sycomore pour le voir ; car il devait passer par là.
\VS{5}Et quand Jésus fut arrivé à cet endroit-là, il leva les yeux, le vit, et lui dit : Zachée, hâte-toi de descendre ; car il faut que je demeure aujourd'hui dans ta maison.
\VS{6}Zachée se hâta de descendre, et le reçut avec joie.
\VS{7}Et tous voyant cela murmuraient, et disaient : Il est entré chez un homme pécheur pour y loger.
\VS{8}Et Zachée, se présentant devant le Seigneur, lui dit : Voici, Seigneur, je donne la moitié de mes biens aux pauvres ; et si j'ai fait tort de quelque chose à quelqu'un, je lui rends le quadruple\FTNT{Lé. 5:20-24.}.
\VS{9}Et Jésus lui dit : Aujourd'hui le salut est entré dans cette maison ; parce que celui-ci aussi est fils d'Abraham.
\VS{10}Car le Fils de l'homme est venu chercher et sauver ce qui était perdu.
\TextTitle{Parabole des dix mines\FTNTT{Lu. 17:21}}
\VS{11}Et comme ils entendaient ces choses, Jésus poursuivit son discours, et proposa une parabole, parce qu'il était près de Jérusalem, et qu'ils pensaient que le royaume de Dieu allait immédiatement paraître.
\VS{12}Il dit donc : Un homme noble s'en alla dans un pays éloigné, pour prendre possession d'un Royaume, et revenir ensuite.
\VS{13}Il appela dix de ses serviteurs, il leur donna dix mines et leur dit : Faites-les valoir jusqu'à ce que je revienne.
\VS{14}Or ses concitoyens le haïssaient, c'est pourquoi ils envoyèrent après lui une ambassade, pour dire : Nous ne voulons pas que cet homme règne sur nous.
\VS{15} Il arriva donc après qu'il fut de retour, et après avoir pris possession du Royaume, qu'il fit appeler auprès de lui les serviteurs auxquels il avait confié son argent, afin de connaître comment chacun l'avait fait valoir.
\VS{16}Alors le premier vint, et dit : Seigneur, ta mine a produit dix autres mines.
\VS{17}Il lui dit : C'est bien, bon serviteur ; parce que tu as été fidèle en peu de choses, reçois le gouvernement de dix villes.
\VS{18}Et le second vint, et dit : Seigneur, ta mine a produit cinq autres mines.
\VS{19}Il dit aussi à celui-ci : Toi aussi, sois établi sur cinq villes.
\VS{20}Un autre vint, et dit : Seigneur, voici ta mine que j'ai gardée enveloppée dans un linge ;
\VS{21}car j'avais peur de toi, parce que tu es un homme sévère ; tu prends ce que tu n'as point déposé, et tu moissonnes ce que tu n'as pas semé.
\VS{22}Il lui dit : Méchant serviteur, je te jugerai sur tes propres paroles : Tu savais que je suis un homme sévère, prenant ce que je n'ai point déposé, et moissonnant ce que je n'ai point semé.
\VS{23}Pourquoi donc n'as-tu pas mis mon argent dans une banque, afin qu'à mon retour je le retire avec un intérêt ?
\VS{24}Alors il dit à ceux qui étaient présents : Ôtez-lui la mine, et donnez-la à celui qui a les dix.
\VS{25}Ils lui dirent : Seigneur, il a dix mines.
\VS{26}Ainsi je vous le dis, on donnera à celui qui a, mais à celui qui n'a pas, on ôtera ce qu'il a.
\VS{27}Au reste, amenez ici mes ennemis qui n'ont pas voulu que je règne sur eux, et tuez-les devant moi.
\TextTitle{Jésus fait son entrée à Jérusalem\FTNTT{Za. 9:9 ; Mt. 21:1-11 ; Mc. 11:1-11 ; Jn. 12:12-19}}
\VS{28}Après avoir ainsi parlé, Jésus marcha devant la foule, pour monter à Jérusalem.
\VS{29}Lorsqu'il approcha de Bethphagé et de Béthanie, vers la montagne appelée Montagne des Oliviers, Jésus envoya deux de ses disciples,
\VS{30}en leur disant : Allez au village qui est en face ; quand vous y serez entrés, vous trouverez un ânon attaché, sur lequel aucun homme n'est monté ; détachez-le, et amenez-le-moi.
\VS{31}Si quelqu'un vous demande pourquoi le détachez-vous, vous lui répondrez : Le Seigneur en a besoin.
\VS{32}Et ceux qui étaient envoyés s'en allèrent, et trouvèrent l'ânon comme il le leur avait dit.
\VS{33}Comme ils le détachaient, ses maîtres leur dirent : Pourquoi détachez-vous cet ânon ?
\VS{34}Ils répondirent : Le Seigneur en a besoin.
\VS{35}Ils emmenèrent à Jésus l'ânon, sur lequel ils jetèrent leurs vêtements, et firent monter Jésus dessus.
\VS{36}Quand il fut en marche, les gens étendirent leurs vêtements sur le chemin.
\VS{37}Et lorsque déjà il approchait de Jérusalem, vers la descente de la Montagne des Oliviers, toute la multitude des disciples saisie de joie, se mit à louer Dieu à haute voix, pour tous les miracles qu'ils avaient vus.
\VS{38}Ils disaient : Béni soit le Roi qui vient au Nom du Seigneur\FTNT{Ps. 118:26.} ! Paix dans le ciel, et gloire dans les lieux très hauts.
\VS{39}Quelques pharisiens, du milieu de la foule, lui dirent : Maître, reprends tes disciples.
\VS{40}Et Jésus répondit : Je vous le dis, s'ils se taisent, les pierres crieront.
\TextTitle{Nouvelles lamentations de Jésus sur Jérusalem\FTNTT{cp. Mt. 23:37-39 ; Lu. 13:34-35}}
\VS{41}Comme il approchait de la ville, Jésus, en la voyant, pleura sur elle, et dit :
\VS{42}Ô ! Si toi aussi, au moins en ce jour qui t'est donné, tu connaissais les choses qui appartiennent à ta paix ! Mais maintenant elles sont cachées à tes yeux.
\VS{43}Il viendra sur toi des jours où tes ennemis t'environneront de tranchées, t'enfermeront, et te serreront de tous côtés ;
\VS{44}ils te raseront, toi et tes enfants qui sont au milieu de toi, et ils ne laisseront pas en toi pierre sur pierre, parce que tu n'as pas connu le temps de ta visitation.
\TextTitle{Jésus chasse les marchands du temple}
\VS{45}Il entra dans le temple, et il se mit à chasser dehors ceux qui vendaient et qui achetaient.
\VS{46}Leur disant : Il est écrit : Ma maison sera appelée la maison de prière ; mais vous, vous en avez fait une caverne de voleurs\FTNT{Es. 56:7 ; Jé. 7:11.}.
\VS{47}Il enseignait tous les jours dans le temple. Et les principaux sacrificateurs et les scribes cherchaient à le faire mourir.
\VS{48}Mais ils ne savaient comment s'y prendre ; car tout le peuple s'attachait à ses paroles.
\Chap{20}
\TextTitle{L'autorité de Jésus et celle de Jean-Baptiste\FTNTT{Mt. 21:23-27 ; Mc. 11:27-33}}
\VerseOne{}Et il arriva un de ces jours-là, comme Jésus enseignait le peuple dans le temple, et qu'il évangélisait, les principaux sacrificateurs, les scribes et les anciens survinrent,
\VS{2}et lui parlèrent en disant : Dis-nous par quelle autorité fais-tu ces choses, ou qui est celui qui t'a donné cette autorité ?
\VS{3}Jésus leur répondit : Je vous adresserai aussi une question, et répondez-moi.
\VS{4}Le baptême de Jean venait-il du ciel ou des hommes ?
\VS{5}Ils raisonnaient entre eux, disant : Si nous répondons : Du ciel ; il dira : Pourquoi n'avez-vous pas cru en lui ?
\VS{6}Et si nous répondons : Des hommes, tout le peuple nous lapidera ; car il est persuadé que Jean était un prophète ;
\VS{7}Alors, ils répondirent qu'ils ne savaient d'où il était.
\VS{8}Et Jésus leur dit : Moi non plus, je ne vous dirai pas par quelle autorité je fais ces choses.
\TextTitle{Parabole des vignerons\FTNTT{Es. 5:1-7 ; Mt. 21:33-46 ; Mc. 12:1-12}}
\VS{9}Alors il se mit à dire au peuple cette parabole : Un homme planta une vigne, et la loua à des vignerons, et fut longtemps absent.
\VS{10}Et à la saison de la récolte, il envoya un serviteur vers les vignerons, afin qu'ils lui donnent du fruit de la vigne. Les vignerons le battirent, et le renvoyèrent à vide.
\VS{11}Il leur envoya encore un autre serviteur ; mais ils le battirent aussi, et après l'avoir traité indignement, ils le renvoyèrent à vide.
\VS{12}Il en envoya encore un troisième, mais ils le blessèrent aussi, et le jetèrent dehors.
\VS{13}Alors le maître de la vigne dit : Que ferai-je ? J'enverrai mon fils bien-aimé ; peut-être que quand ils le verront, ils le respecteront.
\VS{14}Mais quand les vignerons le virent, ils raisonnèrent entre eux, et dirent : Voici l'héritier ; venez, tuons-le, afin que l'héritage soit à nous.
\VS{15}Et ils le jetèrent hors de la vigne, et le tuèrent. Que leur fera donc le maître de la vigne ?
\VS{16}Il viendra, et fera périr ces vignerons-là, et il donnera la vigne à d'autres. Lorsqu'ils entendirent cela, ils dirent : A Dieu ne plaise !
\VS{17}Alors il les regarda, et dit : Que signifie donc ce qui est écrit : La pierre qu'on rejetée ceux qui bâtissaient est devenue la principale de l'angle\FTNT{Ps. 118:22.} ?
\VS{18}Quiconque tombera sur cette pierre, sera brisé ; et elle écrasera celui sur qui elle tombera.
\TextTitle{Le tribut à César\FTNTT{Mt. 22:15-22 ; Mc. 12:13-17}}
\VS{19}Les principaux sacrificateurs et les scribes cherchèrent à mettre la main sur lui à l'heure même, mais ils craignirent le peuple. Ils avaient compris que c'était pour eux que Jésus avait dit cette parabole.
\VS{20}Ils se mirent à observer Jésus ; et ils envoyèrent des agents secrets, qui feignaient d'être justes, pour lui tendre des pièges et saisir de lui quelque parole afin de le livrer au magistrat et à l'autorité du gouverneur.
\VS{21}Ils l'interrogèrent, en disant : Maître, nous savons que tu parles et enseignes conformément à la justice, et que tu ne regardes pas à l'apparence des personnes, mais que tu enseignes la voie de Dieu selon la vérité.
\VS{22}Nous est-il permis de payer le tribut à César, ou non ?
\VS{23}Jésus, apercevant leur ruse, leur dit : Pourquoi me tentez-vous ?
\VS{24}Montrez-moi un denier. De qui a-t-il l'image et l'inscription ? Ils lui répondirent : De César.
\VS{25}Alors il leur dit : Rendez donc à César ce qui est à César ; et à Dieu ce qui est à Dieu.
\VS{26}Ainsi ils ne purent le surprendre dans ses paroles devant le peuple ; mais, étonnés de sa réponse, ils gardèrent le silence.
\TextTitle{Les preuves de la résurrection\FTNTT{Mt. 22:23-33 ; Mc.12:18-27}}
\VS{27}Alors quelques-uns des sadducéens, qui nient formellement la résurrection, s'approchèrent et l'interrogèrent,
\VS{28}disant : Maître, voici ce que Moïse nous a prescrit : Si le frère de quelqu'un meurt, ayant une femme et pas d'enfants, son frère épousera la femme, et suscitera une postérité à son frère.
\VS{29}Or, il y avait sept frères. Le premier se maria, et mourut sans enfants.
\VS{30}Le deuxième épousa la femme et mourut sans enfants.
\VS{31}Puis le troisième l'épousa aussi, et tous les sept de même ; et ils moururent sans laisser d'enfants.
\VS{32}Enfin, la femme mourut aussi.
\VS{33}Duquel d'entre eux donc sera-t-elle la femme à la résurrection ? Car les sept l'ont eue pour femme.
\VS{34}Jésus leur répondit : Les enfants de ce siècle prennent des femmes et des maris ;
\VS{35}mais ceux qui seront trouvés dignes d'avoir part au siècle à venir et à la résurrection des morts, ne prendront ni femmes ni maris.
\VS{36}Car ils ne pourront plus mourir, parce qu'ils seront semblables aux anges, et qu'ils seront fils de Dieu, étant fils de la résurrection.
\VS{37}Que les morts ressuscitent, c'est ce que Moïse a fait connaître quand, à propos du buisson, il appelle le Seigneur le Dieu d'Abraham, le Dieu d'Isaac, et le Dieu de Jacob.
\VS{38}Or, Dieu n'est pas le Dieu des morts, mais des vivants ; car tous vivent en lui.
\TextTitle{Jésus dénonce l'attitude des scribes\FTNTT{cp. Mt. 22:41-23:36 ; Mc. 12:35-40}}
\VS{39}Quelques-uns des scribes prenant la parole, dirent : Maître, tu as bien parlé.
\VS{40}Et ils n'osaient plus lui poser aucune question.
\VS{41}Jésus leur dit : Comment dit-on que le Christ est Fils de David ?
\VS{42}Car David lui-même dit au livre des psaumes : Le Seigneur a dit à mon Seigneur : Assieds-toi à ma droite,
\VS{43}jusqu'à ce que j'aie mis tes ennemis pour le marchepied de tes pieds\FTNT{Ps. 110:1.}.
\VS{44}David donc l'appelle son Seigneur, comment est-il son Fils ?
\VS{45}Comme tout le peuple l'écoutait, il dit à ses disciples :
\VS{46}Gardez-vous des scribes, qui aiment à se promener en robes longues, et qui aiment les salutations sur les places publiques ; qui recherchent les premiers sièges dans les synagogues, et les premières places dans les festins ;
\VS{47}qui dévorent entièrement les maisons des veuves, et qui font pour l'apparence de longues prières. Ils seront jugés plus sévèrement.
\Chap{21}
\TextTitle{Offrande de la pauvre veuve\FTNTT{Mc. 12:41-44}}
\VerseOne{}Comme Jésus regardait, il vit des riches qui mettaient leurs offrandes dans le tronc.
\VS{2}Il vit aussi une pauvre veuve qui y mettait deux petites pièces de monnaie.
\VS{3}Et il dit : Je vous le dis en vérité, cette pauvre veuve a mis plus que tous les autres.
\VS{4}Car tous ceux-ci ont mis aux offrandes de Dieu, de leur superflu ; mais elle a mis de son nécessaire, tout ce qu'elle avait pour vivre.
\TextTitle{Enseignement sur le Mont des Oliviers\FTNTT{Mt. 24-25 ; Mc. 13}}
\VS{5}Comme quelques-uns disaient que le temple était orné de belles pierres et d'offrandes, il dit :
\VS{6}Vous contemplez ces choses ! Les jours viendront où, il ne restera pas pierre sur pierre qui ne soit démolie.
\TextTitle{Les disciples posent deux questions à Jésus\FTNTT{Mt. 24:3 ; Mc.13:3-4}}
\VS{7}Ils lui demandèrent : Maître, quand donc cela arrivera-t-il, et à quel signe connaîtra-t-on que ces choses vont arriver ?
\TextTitle{Les temps de la fin\FTNTT{Mt. 24:4-14 ; Mc. 13:5-13}}
\VS{8}Jésus répondit : Prenez garde que vous ne soyez point séduits. Car plusieurs viendront en mon Nom, disant : C'est moi qui suis le Christ et le temps approche. Ne les suivez pas.
\VS{9}Quand vous entendrez parler des guerres et des soulèvements, ne soyez pas effrayés ; car il faut que ces choses arrivent premièrement. Mais ce ne sera pas encore la fin.
\VS{10}Alors il leur dit : Une nation s'élèvera contre une autre nation, et un royaume contre un autre royaume.
\VS{11}Il y aura de grands tremblements de terre en divers lieux, des famines et des pestes ; il y aura des choses terribles, et de grands signes dans le ciel.
\TextTitle{Souffrance des croyants}
\VS{12}Mais, avant toutes ces choses, ils mettront la main sur vous, et l'on vous persécutera ; on vous livrera aux synagogues, on vous jettera en prison, on vous mènera devant des rois et devant des gouverneurs, à cause de mon Nom.
\VS{13}Cela vous arrivera pour que vous serviez de témoignage.
\VS{14}Mettez-vous donc dans vos cœurs de ne pas préméditer votre défense.
\VS{15}Car je vous donnerai une bouche et une sagesse à laquelle vos adversaires ne pourront résister ou contredire.
\VS{16}Vous serez livrés même par vos parents, par vos frères, par vos proches et par vos amis, et ils feront mourir plusieurs d'entre vous.
\VS{17}Vous serez haïs de tous à cause de mon Nom.
\VS{18}Mais il ne se perdra pas un cheveu de votre tête.
\VS{19}Vous sauverez vos âmes par votre persévérance.
\TextTitle{La destruction de Jérusalem prophétisée}
\VS{20}Lorsque vous verrez Jérusalem environnée par les armées, sachez alors que sa désolation est proche.
\VS{21}Alors, que ceux qui seront en Judée, fuient dans les montagnes ; et que ceux qui seront au milieu de Jérusalem, en sortent, et que ceux qui seront dans les champs, n'entrent pas dans la ville.
\VS{22}Car ce seront des jours de vengeance, afin que toutes les choses qui sont écrites soient accomplies.
\VS{23}Malheur aux femmes qui seront enceintes, et à celles qui allaiteront en ces jours-là ; car il y aura une grande calamité sur le pays, et une grande colère contre ce peuple.
\VS{24}Ils tomberont sous le tranchant de l'épée, ils seront emmenés captifs\FTNT{Les Juifs se révoltèrent plusieurs fois contre le joug des Romains installés en Palestine depuis l'an 65 av. J.-C. En 70, Titus s'empara de Jérusalem après une guerre de plusieurs années et un siège meurtrier de sept mois. Cette même année, le temple fut détruit. A la suite d'une dernière révolte, la ville fut prise de nouveau sous Hadrien. En l'an 135, les Juifs furent en grande partie exterminés, et les survivants furent à jamais chassés de Jérusalem. Ces événements marquèrent symboliquement les débuts de la dispersion des Juifs à travers le monde.} parmi toutes les nations ; et Jérusalem sera foulée par les nations, jusqu'à ce que les temps des nations soient accomplis.
\TextTitle{Retour du Messie sur la terre\FTNTT{Mt. 24:29-31 ; Mc. 13:24-27}}
\VS{25}Il y aura des signes dans le soleil, dans la lune, et dans les étoiles. Et sur terre, il y aura de la détresse chez les nations qui ne sauront que faire, au bruit de la mer et des flots,
\VS{26}les hommes seront comme rendant l'âme de frayeur, dans l'attente des choses qui surviendront dans le monde ; car les puissances des cieux seront ébranlées.
\VS{27}Alors on verra le Fils de l'homme venant sur une nuée avec puissance et grande gloire.
\VS{28}Quand ces choses commenceront à arriver, regardez en haut et levez vos têtes, parce que votre délivrance approche.
\TextTitle{Parabole du figuier\FTNTT{Mt. 24:29-31 ; Mc. 13:24-27}}
\VS{29}Et il leur proposa cette comparaison : Voyez le figuier, et tous les autres arbres.
\VS{30}Dès qu'ils ont poussé, vous savez de vous-mêmes, en regardant, que déjà l'été est proche.
\VS{31}Vous aussi de même, quand vous verrez arriver ces choses, sachez que le Royaume de Dieu est proche.
\VS{32}En vérité je vous le dis, que cette génération ne passera point, que toutes ces choses ne soient arrivées.
\VS{33}Le ciel et la terre passeront, mais mes paroles ne passeront point.
\TextTitle{Exhortation à veiller\FTNTT{Mt. 24:36-51 ; Mc. 13:32-37}}
\VS{34}Prenez donc garde à vous-mêmes, de peur que vos cœurs ne soient appesantis par la gourmandise et l'ivrognerie, et par les soucis de cette vie ; et que ce jour-là ne vous surprenne subitement.
\VS{35}Car il viendra comme un filet sur tous ceux qui habitent sur la surface de toute la terre.
\VS{36}Veillez donc, et priez en tout temps, afin que vous soyez trouvés dignes d'échapper à toutes ces choses qui arriveront, et de paraître devant le Fils de l'homme.
\VS{37}Pendant le jour, Jésus enseignait dans le temple, et il allait passer la nuit à la montagne appelée Montagne des Oliviers.
\VS{38}Et dès le point du jour, tout le peuple venait vers lui au temple pour l'entendre.
\Chap{22}
\TextTitle{Trahison de Judas\FTNTT{Mt. 26:14-16 ; Mc. 14:1-2,10-11}}
\VerseOne{}La fête des pains sans levain, qu'on appelle Pâque, approchait.
\VS{2}Les principaux sacrificateurs et les scribes cherchaient les moyens de faire mourir Jésus ; car ils craignaient le peuple.
\VS{3}Or, Satan entra dans Judas, surnommé Iscariot, qui était du nombre des douze.
\VS{4}Et Judas alla, et parla avec les principaux sacrificateurs et les chefs de gardes, sur la manière de le leur livrer.
\VS{5}Ils furent dans la joie, et convinrent de lui donner de l'argent.
\VS{6}Après s'être engagé, il cherchait une occasion favorable pour leur livrer Jésus à l'insu de la foule.
\TextTitle{La dernière Pâque\FTNTT{Mt. 26:17-25 ; Mc. 14:12-21 ; Jn. 13:1-12}}
\VS{7}Le jour des pains sans levain, où l'on devait immoler la Pâque, arriva.
\VS{8}Et Jésus envoya Pierre et Jean, en leur disant : Allez, et apprêtez-nous l'agneau de Pâque, afin que nous le mangions.
\VS{9}Et ils lui dirent : Où veux-tu que nous l'apprêtions ?
\VS{10}Il leur dit : Voici, quand vous serez entrés dans la ville vous rencontrerez un homme portant une cruche d'eau, suivez-le dans la maison où il entrera.
\VS{11}Et dites au maître de la maison : Le Maître te dit : Où est le lieu où je mangerai l'agneau de Pâque avec mes disciples ?
\VS{12}Et il vous montrera une grande chambre haute, meublée ; c'est là que vous apprêterez l'agneau de Pâque.
\VS{13}Ils partirent, et trouvèrent les choses comme il leur avait dit ; et ils apprêtèrent l'agneau de Pâque.
\VS{14}Et quand l'heure fut venue, il se mit à table, et les douze apôtres avec lui.
\VS{15}Il leur dit : J'ai désiré vivement manger cet agneau de Pâque avec vous avant de souffrir.
\VS{16}Car, je vous dis, que je ne le mangerai plus jusqu'à ce qu'il soit accompli dans le Royaume de Dieu.
\VS{17}Et, ayant pris la coupe, il rendit grâces, et il dit : Prenez cette coupe, et distribuez-la entre vous.
\VS{18}Car, je vous dis, que je ne boirai plus du fruit de la vigne, jusqu'à ce que le Royaume de Dieu soit venu.
\TextTitle{Institution du repas de la Pâque\FTNTT{Mt. 26:26-29 ; Mc. 14:22-25 ; cp. Jn. 13:12-30 ; 1 Co. 11:23-26}}
\VS{19}Ensuite il prit du pain, et après avoir rendu grâces, il le rompit et le leur donna, en disant : Ceci est mon corps, qui est donné pour vous ; faites ceci en mémoire de moi.
\VS{20}Il prit de même la coupe, après le souper, et la leur donna, en disant : Cette coupe est la Nouvelle Alliance en mon sang, qui est répandu pour vous.
\TextTitle{Jésus annonce qu'il sera livré\FTNTT{Mt. 26:21-25 ; Mc. 14:18-21 ; Jn. 13:18-30}}
\VS{21}Cependant voici, la main de celui qui me trahit est avec moi à table.
\VS{22}Le Fils de l'homme s'en va ; selon ce qui est déterminé. Mais malheur à cet homme par qui il est trahi.
\VS{23}Et ils commencèrent à se demander les uns aux autres, qui était celui d'entre eux qui ferait cela.
\TextTitle{Leçon d'humilité\FTNTT{Mt. 20:20-28 ; Mc. 9.33-37 ; 10:35-45 ; Jn. 13:1-17}}
\VS{24}Il s'éleva une contestation parmi les apôtres, pour savoir lequel d'entre eux devait être estimé le plus grand.
\VS{25}Jésus leur dit : Les rois des nations les maîtrisent ; et ceux qui les dominent sont appelés bienfaiteurs.
\VS{26}Mais il n'en sera pas ainsi de vous : Au contraire, que le plus grand parmi vous soit comme le plus petit ; et celui qui gouverne, comme celui qui sert.
\VS{27}Car lequel est le plus grand, celui qui est à table, ou celui qui sert ? N'est-ce pas celui qui est à table ? Or je suis au milieu de vous comme celui qui sert.
\TextTitle{Le Royaume, une récompense}
\VS{28}Vous, vous êtes ceux qui avez persévéré avec moi dans mes épreuves ;
\VS{29}c'est pourquoi je vous confie le Royaume comme mon Père me l'a confié,
\VS{30}afin que vous mangiez et buviez à ma table dans mon Royaume, et que vous soyez assis sur des trônes, pour juger les douze tribus d'Israël.
\TextTitle{Jésus prophétise le triple reniement de Pierre\FTNTT{Mt. 26:30-35 ; Mc. 14:26-31 ; Jn. 13:36-38}}
\VS{31}Le Seigneur dit aussi : Simon, Simon, voici, Satan vous a réclamés pour vous cribler comme le froment ;
\VS{32}mais j'ai prié pour toi afin que ta foi ne défaille point ; et toi donc, quand tu seras un jour converti, affermis tes frères.
\VS{33}Pierre lui dit : Seigneur, je suis prêt à aller avec toi en prison et à la mort.
\VS{34}Mais Jésus lui dit : Pierre, je te dis que le coq ne chantera pas aujourd'hui, que tu n'aies nié trois fois de me connaître.
\TextTitle{Recommandation aux disciples\FTNTT{cp. Jn. 14-16 ; contraste Mt. 10:9-13}}
\VS{35}Puis il leur dit : Quand je vous ai envoyés sans bourse, sans sac, et sans souliers, avez-vous manqué de quelque chose ? Ils répondirent : De rien.
\VS{36}Et il leur dit : Maintenant au contraire, que celui qui a une bourse la prenne, et de même celui qui a un sac ; et que celui qui n'a point d'épée vende son vêtement, et achète une épée.
\VS{37}Car je vous le dis, il faut que cette parole qui est écrite s'accomplisse en moi : Il a été mis au nombre des malfaiteurs\FTNT{Es. 53:12.}. Parce qu'en effet, ce qui me concerne est sur le point d'arriver.
\VS{38}Ils dirent : Seigneur, voici ici deux épées. Et il leur dit : Cela suffit.
\TextTitle{Gethsémané\FTNTT{Mt. 26:36-46 ; Mc. 14:32-42 ; Jn. 18:1 ; cp. Hé. 5:7-8}}
\VS{39}Après être sorti, il alla, selon sa coutume, au Mont des Oliviers ; et ses disciples le suivirent.
\VS{40}Lorsqu'ils arrivèrent dans ce lieu, il leur dit : Priez afin que vous ne tombiez pas en tentation.
\VS{41}Puis s'étant éloigné d'eux à la distance d'environ un jet de pierre, et s'étant mis à genoux, il pria,
\VS{42}disant : Père, si tu voulais éloigner cette coupe loin de moi ; toutefois que ma volonté ne soit point faite, mais la tienne.
\VS{43}Et un ange lui apparut du ciel, pour le fortifier.
\VS{44}Etant en agonie, il priait plus instamment, et sa sueur devint comme des grumeaux de sang qui tombaient à terre.
\VS{45}Après avoir prié, il revint vers ses disciples, qu'il trouva endormis de tristesse ;
\VS{46}et il leur dit : Pourquoi dormez-vous ? Levez-vous, et priez, afin que vous ne tombiez pas en tentation.
\TextTitle{Trahison de Judas\FTNTT{Mt. 26:47-54 ; Mc. 14:43-47 ; Jn. 18:2-11}}
\VS{47}Et comme il parlait encore, voici une foule arriva ; et celui qui s'appelait Judas, l'un des douze, marchait devant elle. Il s'approcha de Jésus pour l'embrasser.
\VS{48}Et Jésus lui dit : Judas, c'est par un baiser que tu trahis le Fils de l'homme ?
\VS{49}Alors ceux qui étaient autour de lui, voyant ce qui allait arriver, lui dirent : Seigneur, frapperons-nous de l'épée ?
\VS{50}Et l'un d'eux frappa le serviteur du souverain sacrificateur, et lui emporta l'oreille droite.
\VS{51}Mais Jésus prenant la parole dit : Laissez-les faire jusqu'ici. Et, ayant touché son oreille, il le guérit.
\VS{52}Puis Jésus dit aux principaux sacrificateurs, aux chefs des gardes du temple, et aux anciens qui étaient venus contre lui : Etes-vous venus comme après un brigand avec des épées et des bâtons ?
\VS{53}J'étais tous les jours avec vous dans le temple, et vous n'avez pas mis la main sur moi. Mais c'est ici votre heure, et la puissance des ténèbres.
\TextTitle{Triple reniement de Pierre\FTNTT{Mt. 26:55-58,69-75 ; Mc. 14:48-54,66-72 ; Jn. 18:15-18,25-27}}
\VS{54}Après avoir saisi Jésus, ils l'emmenèrent, et le conduisirent dans la maison du souverain sacrificateur. Pierre suivait de loin.
\VS{55}Ils allumèrent du feu au milieu de la cour, et ils s'assirent ensemble. Pierre s'assit aussi parmi eux.
\VS{56}Une servante le voyant assis auprès du feu, fixa sur lui les regards, et dit : Celui-ci aussi était avec lui.
\VS{57}Mais il le nia, disant : Femme, je ne le connais point.
\VS{58}Peu après, un autre le voyant, dit : Tu es aussi de ces gens-là, mais Pierre dit : Ô homme ! Je n'en suis point.
\VS{59}Environ une heure plus tard, un autre affirmait et disait : Certainement celui-ci aussi était avec lui car il est Galiléen.
\VS{60}Pierre dit : Ô homme ! Je ne sais pas ce que tu dis. Au même instant, comme il parlait encore, le coq chanta.
\VS{61}Et le Seigneur, s'étant retourné, regarda Pierre. Et Pierre se souvint de la parole que le Seigneur lui avait dite : Avant que le coq chante, tu me renieras trois fois.
\VS{62}Alors Pierre étant sorti dehors, pleura amèrement.
\TextTitle{Jésus est outragé\FTNTT{Mt. 26:67-68 ; Mc. 14:65 ; Jn. 18:22-23}}
\VS{63}Les hommes qui tenaient Jésus se moquaient de lui, et le frappaient.
\VS{64}Ils lui bandèrent les yeux, ils lui donnaient des coups sur le visage, et l'interrogeaient, disant : Devine qui est celui qui t'a frappé ?
\VS{65}Et ils proféraient contre lui beaucoup d'autres injures.
\TextTitle{Jésus déclare qu'il est fils de Dieu\FTNTT{Mt. 26:59-68 ; 27:1 ; Mc. 14:55-65 ; 15:1 ; Jn. 18:19-24}}
\VS{66}Quand le jour fut venu, les anciens du peuple, les principaux sacrificateurs, et les scribes, s'assemblèrent, et firent amener Jésus dans le sanhédrin.
\VS{67}Ils dirent : Si tu es le Christ, dis-le-nous. Et il leur répondit : Si je vous le dis, vous ne le croirez point ;
\VS{68}et si je vous interroge, vous ne me répondrez pas, et vous ne me laisserez pas aller.
\VS{69}Désormais le Fils de l'homme sera assis à la droite de la puissance de Dieu.
\VS{70}Alors ils dirent tous : Tu es donc le Fils de Dieu ? Et il leur répondit : Vous le dites vous-mêmes, je le suis.
\VS{71}Alors ils dirent : Qu'avons-nous besoin encore de témoignage ? Nous l'avons entendu nous-mêmes de sa bouche.
\Chap{23}
\TextTitle{Jésus devant Pilate\FTNTT{Mt. 27:2,11-14 ; Mc. 15:1-5 ; Jn. 18:28-38}}
\VerseOne{}Puis ils se levèrent tous, et ils conduisirent Jésus devant Pilate.
\VS{2}Et ils se mirent à l'accuser, disant : Nous avons trouvé cet homme excitant notre nation à la révolte, et empêchant de payer le tribut à César, et se disant lui-même Christ, Roi.
\VS{3}Pilate l'interrogea, disant : Es-tu le Roi des Juifs ? Et Jésus lui répondit : Tu le dis.
\VS{4}Alors Pilate dit aux principaux sacrificateurs et à la foule : Je ne trouve aucun crime en cet homme.
\VS{5}Mais ils insistèrent, et dirent : Il soulève le peuple, enseignant par toute la Judée, depuis la Galilée où il a commencé, jusqu'ici.
\TextTitle{Jésus envoyé devant Hérode par Pilate}
\VS{6}Quand Pilate entendit parler de la Galilée, il demanda si cet homme était Galiléen,
\VS{7}et, ayant appris qu'il était de la juridiction d'Hérode, il le renvoya à Hérode, qui se trouvait aussi à Jérusalem.
\VS{8}Lorsque Hérode vit Jésus, il en eut une grande joie ; car depuis longtemps il désirait le voir, à cause de ce qu'il avait entendu dire de lui, et il espérait qu'il le verrait faire quelque miracle.
\VS{9}Il lui adressa beaucoup de questions ; mais Jésus ne lui répondit rien.
\VS{10}Les principaux sacrificateurs et les scribes étaient là, et l'accusaient avec violence.
\VS{11}Mais Hérode, avec ses gardes, le traita avec mépris ; et, après s'être moqué de lui et l'avoir revêtu d'un vêtement éclatant, il le renvoya à Pilate.
\VS{12}Ce même jour, Pilate et Hérode devinrent amis ; car auparavant ils étaient ennemis.
\TextTitle{Hérode renvoie Jésus à Pilate\FTNTT{Mt. 27:15-26 ; Mc. 15:6-15 ; Jn. 18:39-19:15}}
\VS{13}Pilate, ayant assemblé les principaux sacrificateurs, les magistrats, et le peuple, leur dit :
\VS{14}Vous m'avez présenté cet homme comme soulevant le peuple. Et voici, je l'ai interrogé devant vous, et je ne l'ai trouvé coupable d'aucun des crimes dont vous l'accusez.
\VS{15}Hérode non plus ; car il nous l'a renvoyé, et voici, cet homme n'a rien fait qui soit digne de mort.
\VS{16}Je le relâcherai donc, après l'avoir châtié.
\VS{17}A chaque fête, il était obligé de leur relâcher un prisonnier.
\VS{18}Toutes les foules s'écrièrent ensemble, disant : Ôte celui-ci, et relâche-nous Barabbas.
\VS{19}Cet homme avait été mis en prison pour une sédition qui avait eu lieu dans la ville, et pour un meurtre.
\VS{20}Pilate leur parla de nouveau, ayant envie de relâcher Jésus.
\VS{21}Et ils crièrent : Crucifie, crucifie-le !
\VS{22}Pilate leur dit pour la troisième fois : Mais quel mal a fait cet homme ? Je ne trouve rien en lui qui soit digne de mort. Après l'avoir fait battre de verges, je le relâcherai.
\VS{23}Mais ils insistèrent à grands cris, demandant qu'il soit crucifié ; et leurs cris et ceux des principaux sacrificateurs l'emportèrent.
\VS{24}Alors Pilate prononça que ce qu'ils demandaient, serait fait.
\VS{25}Il leur relâcha celui qui avait été mis en prison pour sédition et pour meurtre, et qu'ils demandaient ; et il abandonna Jésus à leur volonté.
\TextTitle{Sur le chemin de Golgotha\FTNTT{Mt. 27:31-32 ; Mc. 15:20-21 ; Jn. 19:16-17}}
\VS{26}Comme ils l'emmenaient, ils prirent un certain Simon, de Cyrène, qui revenait des champs, et le chargèrent de la croix pour qu'il la porte derrière Jésus.
\VS{27}Il était suivi d'une grande multitude des gens du peuple et de femmes, qui se frappaient la poitrine, et se lamentaient sur lui.
\VS{28}Mais Jésus se tourna vers elles, leur dit : Filles de Jérusalem, ne pleurez point sur moi, mais pleurez sur vous-mêmes, et sur vos enfants.
\VS{29}Car voici, des jours viendront où l'on dira : Heureuses les stériles, les entrailles qui n'ont point enfanté, et les mamelles qui n'ont point allaité !
\VS{30}Alors ils se mettront à dire aux montagnes : Tombez sur nous ; et aux collines : Couvrez-nous !
\VS{31}Car s'ils font ces choses au bois vert, que sera-t-il fait au bois sec ?
\VS{32}On conduisait en même temps deux malfaiteurs, qui devaient être mis à mort avec Jésus.
\TextTitle{Crucifixion de Jésus\FTNTT{Mt. 27:33-43 ; Mc. 15:24-32 ; Jn. 19:17-37}}
\VS{33}Lorsqu'ils furent arrivés au lieu qui est appelé Calvaire (le Crâne), ils le crucifièrent là, et les malfaiteurs aussi, l'un à la droite, et l'autre à la gauche.
\VS{34}Jésus dit : Père, pardonne-leur, car ils ne savent pas ce qu'ils font. Ils se partagèrent ensuite ses vêtements, en tirant au sort.
\VS{35}Le peuple se tenait là, et regardait. Les magistrats se moquaient de Jésus disant : Il a sauvé les autres, qu'il se sauve lui-même, s'il est le Christ, l'élu de Dieu.
\VS{36}Les soldats aussi se moquaient de lui ; s'approchant et lui présentant du vinaigre,
\VS{37}ils disaient : Si tu es le Roi des Juifs, sauve-toi toi-même !
\VS{38}Or il y avait au-dessus de lui un écriteau en lettres Grecques, Romaines et Hébraïques, en ces mots : Celui-ci est le roi des juifs.
\TextTitle{Repentance du malfaiteur crucifié\FTNTT{cp. Mt. 27:44 ; Mc. 15:32}}
\VS{39}L'un des malfaiteurs qui étaient crucifiés, l'outrageait, disant : Si tu es le Christ, sauve-toi toi-même, et sauve-nous !
\VS{40}Mais l'autre le reprenait, et disait : Ne crains-tu pas Dieu, car tu es condamné au même supplice ?
\VS{41}Pour nous, c'est juste, car nous recevons ce qu'ont mérité nos crimes ; mais celui-ci n'a fait aucun mal.
\VS{42}Et il dit à Jésus : Seigneur ! Souviens-toi de moi quand tu viendras dans ton règne.
\VS{43}Jésus lui dit : Je te le dis en vérité, aujourd'hui tu seras avec moi dans le paradis.
\TextTitle{Jésus remet son esprit\FTNTT{Mt. 27:45-56 ; Mc. 15:33-41 ; Jn. 19:30-37}}
\VS{44}Il était déjà environ la sixième heure, et il eut des ténèbres sur toute la terre jusqu'à la neuvième heure.
\VS{45}Le soleil s'obscurcit, et le voile du temple se déchira par le milieu.
\VS{46}Et Jésus criant à haute voix, dit : Père, je remets mon esprit entre tes mains ! Et, en disant cela, il expira\FTNT{C'est la fin de la Première Alliance. Voir commentaire Jn. 19:30.}.
\TextTitle{Fin de la loi mosaïque ou de la Première Alliance}
\VS{47}Le centenier, voyant ce qui était arrivé, glorifia Dieu, et dit : Certes, cet homme était juste.
\VS{48}Et tous ceux qui assistaient en foule à ce spectacle, après avoir vu ce qui était arrivé, s'en retournèrent, se frappant la poitrine.
\VS{49}Tous ceux qui connaissaient Jésus, et les femmes qui l'avaient suivi de Galilée, se tenaient dans l'éloignement et regardaient ces choses.
\TextTitle{Sépulture de Jésus\FTNTT{Mt. 27:57-61 ; Mc. 15:42-47 ; Jn. 19:38-42}}
\VS{50}Il y avait un conseiller, nommé Joseph, homme bon et juste,
\VS{51}qui n'avait point participé au conseil et aux actes des autres ; il était d'Arimathée, ville des Juifs, et il attendait le Royaume de Dieu.
\VS{52}Cet homme se rendit vers Pilate et lui demanda le corps de Jésus.
\VS{53}Il le descendit de la croix, l'enveloppa d'un linceul, et le déposa dans un sépulcre taillé dans le roc, où personne n'avait encore été mis.
\VS{54}C'était le jour de la préparation, et le sabbat allait commencer.
\VS{55}Les femmes qui étaient venues de Galilée avec Jésus, accompagnèrent Joseph, virent le sépulcre, et la manière dont le corps de Jésus y fut déposé.
\VS{56}Et s'en étant retournées, elles préparèrent des aromates et des parfums ; et le jour du sabbat elles se reposèrent selon la loi.
\Chap{24}
\TextTitle{Résurrection du Messie\FTNTT{Mt. 28:1-15 ; Mc. 16:1-11 ; Jn. 20:1-18}}
\VerseOne{}Le premier jour de la semaine, elles se rendirent au sépulcre de grand matin, apportant les aromates qu'elles avaient préparés.
\VS{2}Elles trouvèrent la pierre roulée à côté du sépulcre.
\VS{3}Et, étant entrées, elles ne trouvèrent point le corps du Seigneur Jésus.
\VS{4}Comme elles ne savaient que penser de cela, voici, deux hommes leur apparurent en habits resplendissants.
\VS{5}Saisies de frayeur, elles baissèrent le visage contre terre, mais ils leur dirent : Pourquoi cherchez-vous parmi les morts celui qui est vivant ?
\VS{6}Il n'est point ici, mais il est ressuscité. Souvenez-vous comment il vous a parlé quand il était encore en Galilée,
\VS{7}et qu'il disait : Il faut que le Fils de l'homme soit livré entre les mains des pécheurs, et qu'il soit crucifié, et qu'il ressuscite le troisième jour.
\VS{8}Et elles se souvinrent de ses paroles.
\VS{9}A leur retour du sépulcre, elles annoncèrent toutes ces choses aux onze disciples, et à tous les autres.
\VS{10}Or c'étaient Marie de Magdala, Jeanne, Marie, mère de Jacques, et les autres qui étaient avec elles, qui dirent ces choses aux apôtres.
\VS{11}Mais les paroles de ces femmes leur semblèrent comme des paroles futiles, et ils ne les crurent point.
\VS{12}Mais Pierre s'étant levé, courut au sépulcre et s'étant courbé pour regarder, il ne vit que les linges là tout seuls, puis il s'en alla chez lui, dans l'étonnement de ce qui était arrivé.
\TextTitle{Jésus et les deux disciples sur le chemin d'Emmaüs\FTNTT{Mc. 16:12-13}}
\VS{13}Or voici, deux d'entre eux étaient ce jour-là en chemin, pour aller à un village nommée Emmaüs, éloigné de Jérusalem de soixante stades.
\VS{14}Et ils s'entretenaient ensemble de toutes ces choses qui étaient arrivées.
\VS{15}Et il arriva que, comme ils s'entretenaient et discutaient entre eux, Jésus lui-même s'approcha et se mit à marcher avec eux.
\VS{16}Mais leurs yeux étaient retenus de sorte qu'ils ne le reconnaissaient pas.
\VS{17}Et il leur dit : Quels sont ces discours que vous tenez ensemble en marchant ? Et pourquoi êtes-vous tout tristes ?
\VS{18}Et l'un d'eux, nommé Cléopas, lui répondit, et lui dit : Es-tu le seul étranger dans Jérusalem qui ne sache point les choses qui s'y sont passées ces jours-ci ?
\VS{19}Et il leur dit : Quelles ? Ils répondirent : Celles concernant Jésus de Nazareth, qui était un prophète puissant en œuvres et en paroles devant Dieu, et devant tout le peuple.
\VS{20}Et comment les principaux sacrificateurs et nos magistrats l'ont livré pour être condamné à mort, et l'ont crucifié.
\VS{21}Or nous espérions que ce serait lui qui délivrerait Israël ; mais avec tout cela, c'est aujourd'hui le troisième jour que ces choses sont arrivées.
\VS{22}Toutefois quelques femmes d'entre nous nous ont fort étonnés, car elles ont été de grand matin au sépulcre
\VS{23}et n'ayant point trouvé son corps, elles sont venues dire que même elles avaient vu une apparition d'anges, qui disaient qu'il est vivant.
\VS{24}Et quelques-uns des nôtres sont allés au sépulcre, et ont trouvé les choses comme les femmes l'avaient dit ; mais lui, ils ne l'ont point vu.
\VS{25}Alors Jésus leur dit : Ô gens sans intelligence, et dont le cœur est lent à croire tout ce que les prophètes ont annoncé !
\VS{26}Ne fallait-il pas que le Christ souffrît ces choses, et qu'il entra dans sa gloire ?
\VS{27}Puis commençant par Moïse, et continuant par tous les prophètes, il leur expliquait dans toutes les Ecritures ce qui le concernait.
\VS{28}Et comme ils furent près du village où ils allaient, il faisait comme s'il voulait aller plus loin.
\VS{29}Mais ils le forcèrent, en lui disant : Reste avec nous, car le soir approche et le jour commence à baisser. Et il entra donc pour rester avec eux.
\VS{30}Et il arriva comme il était à table avec eux, il prit le pain, et il le bénit ; et l'ayant rompu, il leur distribua.
\VS{31}Alors leurs yeux s'ouvrirent, et ils le reconnurent ; mais il disparut de devant eux.
\VS{32}Et ils se dirent l'un à l'autre : Notre cœur ne brûlait-il pas au-dedans de nous lorsqu'il nous parlait en chemin, et qu'il nous ouvrait les Ecritures ?
\TextTitle{Nouvelles apparitions du réssuscité\FTNTT{Mc. 16:14 ; Jn. 20:19-25 ; cp. Jn. 20:26-21:25}}
\VS{33}Et se levant à l'heure même, ils retournèrent à Jérusalem, et ils trouvèrent assemblés les onze et ceux qui étaient avec eux,
\VS{34}qui disaient : Le Seigneur est véritablement ressuscité, et il est apparu à Simon.
\VS{35}A leur tour, ils racontèrent ce qui leur était arrivé en chemin, et comment il avait été reconnu d'eux en rompant le pain.
\VS{36}Comme ils tenaient ces discours, Jésus se présenta lui-même au milieu d'eux, et leur dit : Que la paix soit avec vous !
\VS{37}Mais eux tout terrifiés et effrayés croyaient voir un esprit.
\VS{38}Et il leur dit : Pourquoi êtes-vous troublés, et pourquoi monte-t-il des pensées dans vos coeurs ?
\VS{39}Voyez mes mains et mes pieds, c'est bien moi. Touchez-moi, et voyez : Car un esprit n'a ni chair ni os, comme vous voyez que j'ai.
\VS{40}Et en disant cela, il leur montra ses mains et ses pieds.
\VS{41}Mais comme de joie, ils ne croyaient point encore, et qu'ils s'étonnaient, il leur dit : Avez-vous ici quelque chose à manger ?
\VS{42}Et ils lui présentèrent un morceau de poisson rôti, et un rayon de miel.
\VS{43}Et l'ayant pris, il mangea devant eux.
\TextTitle{La nouvelle mission des onze\FTNTT{Mt. 28:18-20 ; Mc. 16:15-18 ; Jn. 17:18 ; 20:21 ; Ac. 1:8}}
\VS{44}Puis il leur dit : Ce sont ici les paroles que je vous disais lorsque j'étais encore avec vous, qu'il fallait que s'accomplisse tout ce qui est écrit de moi dans la loi de Moïse, dans les prophètes, et dans les psaumes.
\VS{45}Alors il leur ouvrit l'esprit\FTNT{Pour comprendre les Ecritures, nous avons besoin de l'aide de l'Esprit de Dieu. La vraie connaissance ne vient pas des hommes, mais de Dieu (Da. 9:22).} afin qu'ils comprennent les Ecritures.
\VS{46}Et il leur dit : Il est ainsi écrit, et ainsi il fallait que le Christ souffre, et qu'il ressuscite des morts le troisième jour,
\VS{47}et que la repentance et le pardon des péchés seraient prêchés en son Nom à toutes les nations, à commencer par Jérusalem\FTNT{Es. 53.}.
\VS{48}Et vous êtes témoins de ces choses. 
\VS{49}Et voici, j'enverrai sur vous la promesse de mon Père, mais vous donc restez dans la ville de Jérusalem, jusqu'à ce que vous soyez revêtus de la puissance d'en haut.
\TextTitle{Jésus enlevé au ciel\FTNTT{Mc. 16:19-20 ; Ac. 1:9-11}}
\VS{50}Après quoi il les conduisit dehors jusqu'en Béthanie, et levant ses mains en haut, il les bénit.
\VS{51}Et pendant qu'il les bénissait, il se sépara d'eux, et fut élevé au ciel.
\VS{52}Pour eux, après l'avoir adoré, ils retournèrent à Jérusalem avec une grande joie.
\VS{53}Et ils étaient toujours dans le temple, louant et bénissant Dieu. Amen !
\PPE{}
\end{multicols}

%\clearpage\ShortTitle{Jean}\BookTitle{Jean}\BFont
\noindent\hrulefill
\textit{
\bigskip
{\centering{}
\\Signifie : Dieu pardonne, don de Dieu
\\Thème : Christ, Dieu
\\Auteur : Jean
\\Date de rédaction : Env. 85-90 apr. J.-C.\\}
}
%\bigskip
\textit{
\\Auteur d’un des quatre évangiles, des trois épîtres éponymes et de l’Apocalypse, Jean, fils de Zébédée, fut l’un des douze. Témoin oculaire du ministère terrestre de Jésus-Christ, il attesta par l’essence de ses écrits le caractère divin de ce dernier.
\bigskip
\\Fidèle au livre d’Exode où Yahweh se révéla comme étant « Je suis », Jean reprit les propos de Jésus et le présenta comme la Parole incarnée, le Pain de vie, la Lumière du monde, la Porte des brebis, le Bon berger, la Résurrection, la Vie… Proche du maître, Jean fut à même de relater les évènements marquants de sa vie comme la gloire de la Transfiguration, l’angoisse de la passion exprimée à Gethsémané, ou encore les déclarations solennelles précédées de l’expression «  En vérité, en vérité »… Il mit également en évidence la controverse suscitée par le Christ et l’opposition dont il fit l’objet de la part de certains pharisiens qui souhaitaient sa mort.
\bigskip
\\L’évangile de Jean exprime la nécessité de la nouvelle naissance et dévoile les attributs du Fils de Dieu, le Messie tant attendu.\bigskip
}
\par\nobreak\noindent\hrulefill
\begin{multicols}{2}
\TextTitle{[La divinité de Jésus-Christ]
\\(Jn. 10:30 ; Hé. 1:5-13)}
\Chap{1}
\VerseOne{}Au commencement était la Parole, et la Parole était avec Dieu, et la Parole était Dieu.
\VS{2}Elle était au commencement avec Dieu.
\TextTitle{[L'oeuvre de Jésus avant son incarnation]}
\VS{3}Toutes choses ont été faites par elle, et rien de ce qui a été fait, n'a été fait sans elle.
\VS{4}En elle était la vie, et la vie était la Lumière des hommes\FTNT{Jésus-Christ notre Lumière : Es. 60:19-20.}.
\VS{5}Et la Lumière luit dans les ténèbres, mais les ténèbres ne l'ont point reçue.
\TextTitle{[Ministère de Jean-Baptiste]}
\VS{6}Il y eut un homme appelé Jean, qui fut envoyé de Dieu.
\VS{7}Il vint pour servir de témoin, pour rendre témoignage à la Lumière, afin que tous croient par lui.
\VS{8}Il n'était pas la Lumière, mais il était envoyé pour rendre témoignage à la Lumière.
\TextTitle{[Jésus-Christ, la véritable lumière]
\\(Jn. 3:17-21 ; 8:12 ; 9:5 ; 12:46)}
\VS{9}Cette Lumière était la véritable Lumière, qui en venant dans le monde éclaire tout homme.
\VS{10}Elle était dans le monde, et le monde a été fait par elle ; mais le monde ne l'a point connue.
\VS{11}Elle est venue chez les siens ; et les siens ne l'ont point reçue.
\VS{12}Mais à tous ceux qui l'ont reçue, à ceux qui croient en son Nom, elle leur a donné le pouvoir de devenir enfants de Dieu.
\VS{13}Lesquels sont nés, non du sang, ni de la volonté de la chair, ni de la volonté de l'homme ; mais ils sont nés de Dieu.
\TextTitle{[La Parole faite chair]
\\(Jn. 14:9 ; Mt. 1:18-23 ; Lu. 1:30-35 ; 2:11 ; 1Tim. 3:16)}
\VS{14}Et la Parole a été faite chair, elle a habité parmi nous, pleine de grâce et de vérité, et nous avons contemplé sa gloire, une gloire, comme la gloire du Fils unique du Père.
\TextTitle{[Premier témoignage de Jean-Baptiste]
\\(Mt. 3:1-12 ; Mc. 1:1-11 ; Lu. 3:1-22)}
\VS{15}Jean a donc rendu témoignage de lui, et s’est écrié, disant : C'est celui dont j’ai dit : Celui qui vient après moi m’a précédé, car il était avant moi.
\VS{16}Et nous avons tous reçu de sa plénitude, et grâce pour grâce.
\VS{17}Car la loi\FTNT{La loi a été promulguée par Moïse.} a été donnée par Moïse, la grâce et la vérité sont venues par Jésus-Christ.
\VS{18}Personne n’a jamais vu Dieu, le Fils unique qui est dans le sein du Père, est celui qui nous l'a révélé.
\VS{19}Et c'est ici le témoignage de Jean, lorsque les Juifs envoyèrent de Jérusalem des sacrificateurs et des lévites pour l'interroger, et lui dire : Toi qui es-tu ?
\VS{20}Il confessa, et ne le nia point, il déclara, en disant : Ce n'est pas moi qui suis le Christ.
\VS{21}Et ils lui demandèrent : Quoi donc ? Es-tu Elie ? Et il dit : Je ne le suis point\FTNT{En Mt. 11:14 Jésus confirme pourtant que Jean-Baptiste est bien l’Elie qui devait venir. Comment expliquer qu’il nia l’être lorsqu’il fut interrogé par les pharisiens ? La seule explication plausible c’est qu’il l’ignorait. Toutefois, il avait conscience qu’il était ~la voix~ prophétisée par Esaïe. Remarquez que lorsqu’il fut emprisonné, il avait envoyé quelques-uns de ses disciples pour demander à Jésus s’il était bien le Messie (Mt. 11:13 ; Lu. 7:19-20) alors qu’il fut le premier à rendre témoignage du Seigneur. Ces éléments ne sont pas contradictoires, ils ne font que révéler les failles liées à la nature humaine de Jean.}. Es-tu le Prophète ? Et il répondit : Non.
\VS{22}Ils lui dirent donc : Qui es-tu, afin que nous donnions une réponse à ceux qui nous ont envoyés. Que dis-tu de toi-même ?
\VS{23}Il dit : Je suis la voix de celui qui crie dans le désert : Aplanissez le chemin du Seigneur, comme a dit Esaïe le prophète\FTNT{Es. 40:3.}.
\VS{24}Or ceux qui avaient été envoyés vers lui étaient des pharisiens.
\VS{25}Ils l'interrogèrent encore, et lui dirent : Pourquoi donc baptises-tu si tu n'es point le Christ, ni Elie, ni le Prophète ?
\VS{26}Jean leur répondit : Pour moi, je baptise d'eau ; mais il y a quelqu’un au milieu de vous que vous ne connaissez point.
\VS{27}C'est celui qui vient après moi, il m’a précédé, et je ne suis pas digne de délier la courroie de ses souliers.
\VS{28}Ces choses se passèrent à Béthanie, au-delà du Jourdain, où Jean baptisait.
\VS{29}Le lendemain Jean vit Jésus venir à lui, et il dit : Voici l'Agneau de Dieu, qui ôte le péché du monde.
\VS{30}C'est celui dont j’ai dit : Après moi vient un homme qui m’a précédé ; car il était avant moi.
\VS{31}Et pour moi, je ne le connaissais pas ; mais c'est afin qu'il soit manifesté à Israël que je suis venu baptiser d'eau.
\VS{32}Jean rendit aussi témoignage, en disant : J'ai vu l'Esprit descendre du ciel comme une colombe, et s'arrêter sur lui.
\VS{33}Et pour moi, je ne le connaissais point ; mais celui qui m'a envoyé baptiser d'eau, m'avait dit : Celui sur qui tu verras l'Esprit descendre et s’arrêter, c'est celui qui baptise du Saint-Esprit.
\VS{34}Et je l'ai vu, et j'ai rendu témoignage, que c'est lui qui est le Fils de Dieu.
\TextTitle{[Premiers disciples de Jésus-Christ]
\\(Mt. 4:18-22 ; Mc. 1:16-20 ; Lu. 5:1-11)}
\VS{35}Le lendemain Jean était encore là, avec deux de ses disciples ;
\VS{36}et regardant Jésus qui marchait, il dit : Voici l'Agneau de Dieu.
\VS{37}Les deux disciples l'entendirent prononcer ces paroles, et ils suivirent Jésus.
\VS{38}Et Jésus se retournant, et voyant qu'ils le suivaient, il leur dit : Que cherchez-vous ? Ils lui répondirent : Rabbi, c'est-à-dire Maître, où demeures-tu ?
\VS{39}Il leur dit : Venez, et voyez. Ils y allèrent, et ils virent où il demeurait ; et ils demeurèrent avec lui ce jour-là ; car il était environ dix heures.
\VS{40}André, frère de Simon Pierre, était l'un des deux qui avaient entendu les paroles de Jean et qui avaient suivi Jésus.
\VS{41}Ce fut lui qui rencontra le premier Simon son frère, et il lui dit : Nous avons trouvé le Messie, c'est-à-dire le Christ.
\VS{42}Et il le conduisit vers Jésus, et Jésus l’ayant regardé, dit : Tu es Simon, fils de Jonas, tu seras appelé Céphas ; c'est-à-dire, Pierre.
\VS{43}Le lendemain Jésus voulut se rendre en Galilée, et il trouva Philippe. Et il lui dit : Suis-moi.
\VS{44}Philippe était de Bethsaïda, la ville d'André et de Pierre.
\VS{45}Philippe rencontra Nathanaël, et lui dit : Nous avons trouvé celui de qui Moïse a écrit dans la loi, et dont les prophètes ont parlé, Jésus, qui est de Nazareth, fils de Joseph.
\VS{46}Et Nathanaël lui dit : Peut-il venir quelque chose de bon de Nazareth ? Philippe lui dit : Viens, et vois.
\VS{47}Jésus aperçut Nathanaël venir vers lui, et il dit de lui : Voici vraiment un Israëlite dans lequel il n'y a point de fraude.
\VS{48}Nathanaël lui dit : D'où me connais-tu ? Jésus répondit et lui dit : Avant que Philippe t’appelle, quand tu étais sous le figuier, je t’ai vu.
\VS{49}Nathanaël répondit et lui dit : Maître, tu es le Fils de Dieu. Tu es le Roi d'Israël.
\VS{50}Jésus lui répondit et dit : Parce que je t'ai dit que je t’ai vu sous le figuier, tu crois. Tu verras des choses plus grandes encore.
\VS{51}Il lui dit aussi : En vérité, en vérité je vous dis : Désormais vous verrez le ciel ouvert, et les anges de Dieu monter et descendre sur le Fils de l'homme.
\TextTitle{[Premier miracle, à Cana]}
\Chap{2}
\VerseOne{}Trois jours après, il y eut des noces à Cana en Galilée, et la mère de Jésus était là.
\VS{2}Et Jésus fut aussi convié aux noces avec ses disciples.
\VS{3}Et le vin ayant manqué, la mère de Jésus lui dit : Ils n'ont plus de vin.
\VS{4}Jésus lui répondit : Qu'y a-t-il entre moi et toi, femme ? Mon heure n'est point encore venue.
\VS{5}Sa mère dit aux serviteurs : Faites tout ce qu'il vous dira.
\VS{6}Or il y avait là six vases de pierre, destinés aux purifications des Juifs, et contenant chacun deux ou trois mesures.
\VS{7}Et Jésus leur dit : Remplissez d'eau ces vases. Et ils les remplirent jusqu’au bord.
\VS{8}Puis il leur dit : Puisez maintenant, et apportez-en au maître d'hôtel. Et ils lui en apportèrent.
\VS{9}Quand le maître d'hôtel eut goûté l'eau changée en vin, ne sachant d'où venait ce vin, tandis que les serviteurs qui avaient puisé l'eau le savaient bien, il s'adressa à l'époux
\VS{10}et lui dit : Tout homme sert d’abord le bon vin, et ensuite le moins bon, après qu’on s’est enivré ; mais toi, tu as gardé le bon vin jusqu'à maintenant.
\VS{11}Jésus fit ce premier miracle à Cana en Galilée, et il manifesta sa gloire, et ses disciples crurent en lui.
\VS{12}Après cela, il descendit à Capernaüm avec sa mère, et ses frères, et ses disciples ; mais ils y demeurèrent peu de jours.
\TextTitle{[La première Pâque]
\\(Jn. 6:4 ; 11:55)}
\VS{13}La Pâque des Juifs était proche ; c'est pourquoi Jésus monta à Jérusalem.
\VS{14}Et il trouva dans le temple des vendeurs de bœufs, de brebis, et de pigeons ; et les changeurs qui y étaient assis.
\VS{15}Et ayant fait un fouet avec des petites cordes, il les chassa tous du temple, avec les brebis, et les bœufs ; et il dispersa la monnaie des changeurs, et renversa les tables.
\VS{16}Et il dit aux vendeurs des pigeons : Otez ces choses d'ici, et ne faites pas de la maison de mon Père une maison de marché.
\VS{17}Alors ses disciples se souvinrent qu'il était écrit : Le zèle de ta maison me dévore\FTNT{Ps. 69:10.}.
\VS{18}Mais les Juifs prenant la parole, lui dirent : Quel signe nous montres-tu pour agir de la sorte ?
\VS{19}Jésus répondit et leur dit : Détruisez ce temple, et en trois jours je le relèverai.
\VS{20}Et les Juifs dirent : Il a fallu quarante-six ans pour bâtir ce temple, et toi, tu le relèveras en trois jours !
\VS{21}Mais il parlait du temple de son corps.
\VS{22}C'est pourquoi lorsqu'il fut ressuscité des morts, ses disciples se souvinrent qu'il leur avait dit cela, et ils crurent à l'Ecriture et à la parole que Jésus avait dite.
\VS{23}Et comme il était à Jérusalem le jour de la fête de Pâque, plusieurs crurent en son Nom, voyant les miracles qu'il faisait.
\VS{24}Mais Jésus ne se fiait point à eux, parce qu'il les connaissait tous ;
\VS{25}et parce qu'il n'avait pas besoin qu’on lui rende témoignage d'aucun homme ; car il savait lui-même ce qui était dans l'homme.
\TextTitle{[Jésus et Nicodème : la naissance d'en haut]}
\Chap{3}
\VerseOne{}Mais il y eut un homme d'entre les pharisiens, nommé Nicodème, qui était un des chefs des Juifs,
\VS{2}qui vint de nuit auprès de Jésus, et lui dit : Rabbi, nous savons que tu es un Docteur venu de Dieu, car personne ne peut faire les miracles que tu fais, si Dieu n'est avec lui.
\VS{3}Jésus lui répondit et dit : En vérité, en vérité je te le dis : Si quelqu'un ne naît d’en haut\FTNT{Naître d’en haut : Dans la plupart des Bibles modernes, on trouve l’expression naître « de nouveau », or cette traduction n’est pas correcte puisque le texte grec utilise l'expression naître « d’en haut ». L’adverbe « d’en haut » vient du mot grec « ahothen » qui signifie : depuis le haut, depuis un endroit plus élevé, ce qui vient des cieux ou de Dieu, depuis le début, l'origine. Ce mot se retrouve dans Mt. 27:51 ; Mc. 15:38 ; Lu. 1:3 ; Jn. 3:31 ; Jn. 19:11 ; Jn. 19:23 ; Ja. 1:17 ; Ja. 3:15 ; Ja. 3:17. ~Anothen~ vient de ~ano~ : choses d’en haut. En Ga. 4:26 ~ano~ peut se référer au lieu ou au temps. Le lieu : La Jérusalem qui est au-dessus, dans les cieux. Le temps : La Jérusalem éternelle qui a précédé la terrestre. Le mot ~ano~ a été traduit par ~en haut~ dans Jn. 8:23 ; Jn. 11:41 ; Ac. 2:19 ; Ga. 4:26 ; Col. 3:1-2 ; et par ~céleste~ dans Ph. 3:14. Jésus nous enseigne donc que la nouvelle naissance est en réalité la naissance d’en haut, une naissance qui a eu lieu dans la Nouvelle Jérusalem.}, il ne peut voir le Royaume de Dieu.
\VS{4}Nicodème lui dit : Comment un homme peut-il naître quand il est vieux ? Peut-il rentrer dans le sein de sa mère et naître une seconde fois ?
\VS{5}Jésus répondit : En vérité, en vérité je te dis : Si quelqu’un ne naît d'eau et d'Esprit, il ne peut entrer dans le Royaume de Dieu.
\VS{6}Ce qui est né de la chair est chair ; et ce qui est né de l'Esprit est esprit.
\VS{7}Ne t'étonne pas de ce que je t'ai dit : Il faut que vous naissiez d’en haut.
\VS{8}Le vent souffle où il veut, et tu en entends le bruit ; mais tu ne sais pas d'où il vient ni où il va : Il en est ainsi de tout homme qui est né de l'Esprit.
\VS{9}Nicodème lui dit : Comment cela peut-il se faire ?
\VS{10}Jésus répondit et lui dit : Tu es le docteur d'Israël, et tu ne connais point ces choses !
\VS{11}En vérité, en vérité je te le dis, nous disons ce que nous savons, et nous rendons témoignage de ce que nous avons vu ; et vous ne recevez pas notre témoignage.
\VS{12}Si vous ne croyez pas quand je vous ai parlé des choses terrestres, comment croirez-vous quand je vous parlerai des choses célestes ?
\VS{13}Personne n'est monté au ciel, si ce n’est celui qui est descendu du ciel, le Fils de l'homme qui est dans le ciel.
\VS{14}Et comme Moïse éleva le serpent\FTNT{Le serpent d’airain : No. 21:9}dans le désert, il faut de même que le Fils de l'homme soit élevé,
\VS{15}afin que quiconque croit en lui ne périsse point, mais qu'il ait la vie éternelle.
\VS{16}Car Dieu a tant aimé le monde, qu'il a donné son Fils unique, afin que quiconque croit en lui ne périsse point, mais qu'il ait la vie éternelle.
\VS{17}Car Dieu n'a point envoyé son Fils dans le monde pour condamner le monde, mais afin que le monde soit sauvé par lui.
\VS{18}Celui qui croit en lui ne sera point jugé ; mais celui qui ne croit point est déjà jugé ; parce qu'il n'a point cru au Nom du Fils unique de Dieu.
\VS{19}Et ce jugement c’est que la lumière est venue dans le monde et que les hommes ont préféré les ténèbres à la lumière, parce que leurs œuvres étaient mauvaises.
\VS{20}Car quiconque fait le mal, hait la lumière, et ne vient point à la lumière, de peur que ses œuvres ne soient condamnées.
\VS{21}Mais celui qui agit selon la vérité, vient à la lumière, afin que ses œuvres soient manifestées, parce qu'elles sont faites selon Dieu.
\TextTitle{[Nouveau témoignage de Jean-Baptiste]}
\VS{22}Après ces choses, Jésus s’en alla avec ses disciples dans la terre de Judée ; et là, il demeurait avec eux et il baptisait.
\VS{23}Jean aussi baptisait à Enon, près de Salim, parce qu'il y avait là beaucoup d'eau, et on y venait pour être baptisé.
\VS{24}Car Jean n'avait pas encore été mis en prison.
\VS{25}Or, il y eut une dispute entre les disciples de Jean et les Juifs touchant la purification.
\VS{26}Ils vinrent trouver Jean, et lui dirent : Maître, celui qui était avec toi au-delà du Jourdain, et à qui tu as rendu témoignage, voilà, il baptise, et tous vont à lui.
\VS{27}Jean répondit et dit : Un homme ne peut recevoir que ce qui lui a été donné du ciel.
\VS{28}Vous-mêmes m'êtes témoins que j'ai dit : Ce n'est pas moi qui suis le Christ, mais j’ai été envoyé devant lui.
\VS{29}Celui à qui appartient l'Epouse c’est l'Epoux ; mais l'ami de l'Epoux qui se tient là et qui l’entend, éprouve une grande joie à cause de la voix de l’Epoux ; c'est pourquoi cette joie qui est la mienne est parfaite.
\VS{30}Il faut qu'il croisse, et que je diminue.
\TextTitle{[Conclusion apportée par Jean]}
\VS{31}Celui qui vient d'en haut est au-dessus de tous ; celui qui est venu de la terre est de la terre, et il parle comme venant de la terre. Celui qui est venu du ciel est au-dessus de tous.
\VS{32}Et ce qu'il a vu et entendu, il le témoigne ; mais personne ne reçoit son témoignage.
\VS{33}Celui qui a reçu son témoignage a certifié que Dieu est véritable.
\VS{34}Car celui que Dieu a envoyé annonce les paroles de Dieu ; car Dieu ne lui donne point l'Esprit avec mesure.
\VS{35}Le Père aime le Fils, et il a remis toutes choses entre ses mains.
\VS{36}Celui qui croit au Fils a la vie éternelle, mais celui qui désobéit au Fils ne verra point la vie, mais la colère de Dieu demeure sur lui.
\TextTitle{[Jésus se rend en Galilée]}
\Chap{4}
\VerseOne{}Le Seigneur sut que les pharisiens avaient appris qu'il faisait et baptisait plus de disciples que Jean.
\VS{2}Toutefois Jésus ne baptisait point lui-même, mais c'étaient ses disciples.
\VS{3}Il quitta la Judée, et retourna encore en Galilée.
\TextTitle{[Jésus et la femme samaritaine]}
\VS{4}Comme il fallait qu'il passe par la Samarie,
\VS{5}il arriva dans une ville de Samarie nommée Sychar, près du champ que Jacob avait donné à Joseph son fils\FTNT{Ge. 48:22.}.
\VS{6}Or il y avait là le puits de Jacob ; et Jésus, fatigué du voyage, se tenait là, assis au bord du puits. C'était environ la sixième heure\FTNT{Sixième heure ou midi.}.
\VS{7}Une femme Samaritaine vint puiser de l'eau, Jésus lui dit : Donne-moi à boire.
\VS{8}Car ses disciples étaient allés à la ville pour acheter des vivres.
\VS{9}La femme Samaritaine lui dit : Comment toi qui es Juif, me demandes-tu à boire, à moi qui suis une femme Samaritaine ? Les Juifs, en effet, n’ont pas de relations avec les Samaritains.
\VS{10}Jésus lui répondit et dit : Si tu connaissais le don de Dieu, et qui est celui qui te dit : Donne-moi à boire, tu lui aurais toi-même demandé à boire, et il t’aurait donné de l'eau vive.
\VS{11}La femme lui dit : Seigneur, tu n'as rien pour puiser, et le puits est profond ; d'où aurais-tu donc cette eau vive ?
\VS{12}Es-tu plus grand que Jacob, notre père, qui nous a donné ce puits, et qui en a bu lui-même, ainsi que ses enfants et son bétail ?
\VS{13}Jésus répondit et lui dit : Quiconque boit de cette eau aura encore soif ;
\VS{14}mais celui qui boira de l'eau que je lui donnerai, n'aura jamais soif ; mais l'eau que je lui donnerai deviendra en lui une source d'eau qui jaillira jusque dans la vie éternelle.
\VS{15}La femme lui dit : Seigneur, donne-moi de cette eau, afin que je n'aie plus soif, et que je ne vienne plus ici puiser de l'eau.
\VS{16}Jésus lui dit : Va, appelle ton mari, et viens ici.
\VS{17}La femme lui répondit et dit : Je n'ai point de mari. Jésus lui dit : Tu as bien dit : Je n'ai point de mari.
\VS{18}Car tu as eu cinq maris, et celui que tu as maintenant n'est point ton mari ; en cela tu as dit la vérité.
\VS{19}La femme lui dit : Seigneur, je vois que tu es un prophète.
\VS{20}Nos pères ont adoré sur cette montagne\FTNT{Cette montagne, dont parle la Samaritaine, c'est le Mont Garizim (ou montagne de Sichem) sur lequel les samaritains construisirent leur temple et établirent leur culte, au temps de Néhémie.}, et vous, vous dites que le lieu où il faut adorer est à Jérusalem.
\VS{21}Jésus lui dit : Femme, crois-moi, l'heure vient où ce ne sera ni sur cette montagne ni à Jérusalem que vous adorerez le Père.
\VS{22}Vous adorez ce que vous ne connaissez pas ; nous, nous adorons ce que nous connaissons ; car le salut vient des Juifs.
\VS{23}Mais l'heure vient, et elle est déjà venue, où les vrais adorateurs adoreront le Père en esprit et en vérité ; car ce sont là les adorateurs que le Père demande.
\VS{24}Dieu est Esprit, et il faut que ceux qui l'adorent, l'adorent en esprit et en vérité.
\VS{25}La femme lui répondit : Je sais que le Messie, c'est-à-dire le Christ, doit venir ; quand il sera venu, il nous annoncera toutes choses.
\VS{26}Jésus lui dit : Je le suis, moi qui te parle.
\VS{27}Là-dessus arrivèrent ses disciples, et ils s'étonnèrent de ce qu'il parlait avec une femme. Toutefois aucun ne dit : Que demandes-tu ? Ou : Pourquoi parles-tu avec elle ?
\VS{28}La femme, ayant laissé sa cruche, s'en alla dans la ville, et elle dit aux habitants :
\VS{29}Venez voir un homme qui m'a dit tout ce que j'ai fait, ne serait-ce point le Christ ?
\VS{30}Ils sortirent donc de la ville, et vinrent vers lui.
\VS{31}Cependant les disciples le pressaient, disant : Maître, mange.
\VS{32}Mais il leur dit : J'ai à manger une nourriture que vous ne connaissez point.
\VS{33}Sur quoi les disciples se demandaient entre eux : Quelqu'un lui aurait-il apporté à manger ?
\VS{34}Jésus leur dit : Ma nourriture est de faire la volonté de celui qui m'a envoyé, et d’accomplir son œuvre.
\VS{35}Ne dites-vous pas qu'il y a encore quatre mois jusqu’à la moisson ? Voici, je vous dis, levez vos yeux, et regardez les champs qui déjà blanchissent pour la moisson.
\VS{36}Celui qui moissonne reçoit un salaire, et amasse des fruits pour la vie éternelle ; afin que celui qui sème et celui qui moissonne se réjouissent ensemble.
\VS{37}Car en ceci ce qu’on dit d’ordinaire est vrai : L’un sème et l'autre moissonne.
\VS{38}Je vous ai envoyés moissonner où vous n'avez point travaillé ; d'autres ont travaillé, et vous êtes entrés dans leur travail.
\TextTitle{[Jésus et les samaritains]}
\VS{39}Plusieurs Samaritains de cette ville crurent en lui, à cause de la parole de la femme qui avait rendu ce témoignage : Il m'a dit tout ce que j'ai fait.
\VS{40}Quand donc les Samaritains vinrent le trouver, ils le prièrent de demeurer avec eux ; et il demeura là deux jours.
\VS{41}Et beaucoup plus de gens crurent à cause de sa parole ;
\VS{42}et ils disaient à la femme : Ce n'est plus à cause de ta parole que nous croyons ; car nous l'avons entendu nous-mêmes, et nous savons qu’il est véritablement le Christ, le Sauveur du monde.
\VS{43}Après ces deux jours, Jésus partit de là, et s'en alla en Galilée.
\VS{44}Car il avait rendu témoignage qu'un prophète n'est pas honoré dans son pays.
\VS{45}Lorsqu’il arriva en Galilée, les Galiléens le reçurent, ayant vu toutes les choses qu'il avait faites à Jérusalem le jour de la Fête, car eux aussi étaient allés à la Fête.
\TextTitle{[Jésus guérit le fils d'un officier]}
\VS{46}Jésus retourna encore à Cana de Galilée, où il avait changé l'eau en vin. Or il y avait à Capernaüm un officier du roi, dont le fils était malade.
\VS{47}Ayant appris que Jésus était venu de Judée en Galilée, il alla vers lui, et le pria de descendre pour guérir son fils qui était près de mourir.
\VS{48}Mais Jésus lui dit : Si vous ne voyez pas des prodiges et des miracles, vous ne croyez point.
\VS{49}L’officier du roi lui dit : Seigneur, descends avant que mon fils meure.
\VS{50}Jésus lui dit : Va, ton fils vit. Cet homme crut à la parole que Jésus lui avait dite, et il s'en alla.
\VS{51}Et comme il descendait déjà, ses serviteurs vinrent au-devant de lui, et lui apportèrent des nouvelles, disant : Ton fils vit.
\VS{52}Et il leur demanda à quelle heure il s'était trouvé mieux ; et ils lui dirent : Hier, à la septième heure, la fièvre l’a quitté.
\VS{53}Le père reconnut que c'était à cette même heure-là que Jésus lui avait dit : Ton fils vit. Et il crut, avec toute sa maison.
\VS{54}Jésus fit encore ce second miracle quand il fut venu de Judée en Galilée.
\TextTitle{[Nouvelle fête des juifs et guérison d'un paralytique à la piscine de Béthesda]}
\Chap{5}
\VerseOne{}Après ces choses, il y eut une fête des Juifs, et Jésus monta à Jérusalem.
\VS{2}Or à Jérusalem, près de la porte des brebis, il y avait une piscine appelée en hébreu Béthesda, et qui avait cinq portiques.
\VS{3}Sous ces portiques étaient couchés un grand nombre de malades, des aveugles, des boiteux, des paralytiques, attendant le mouvement de l'eau.
\VS{4}Car un ange descendait de temps en temps dans la piscine, et agitait l'eau ; et alors le premier qui y descendait après que l'eau avait été agitée, était guéri, quelle que fût sa maladie.
\VS{5}Or il y avait là un homme malade depuis trente-huit ans.
\VS{6}Jésus, le voyant couché par terre, et sachant qu'il était déjà malade depuis longtemps, lui dit : Veux-tu être guéri ?
\VS{7}Le malade lui répondit : Seigneur, je n'ai personne pour me jeter dans la piscine quand l'eau est agitée, et pendant que j'y vais, un autre descend avant moi.
\VS{8}Jésus lui dit : Lève-toi, prends ton lit, et marche.
\VS{9}Et aussitôt cet homme fut guéri, il prit son lit, et marcha. Or c'était un jour de sabbat.
\VS{10}Les Juifs dirent donc à celui qui avait été guéri : C'est un jour de sabbat, il ne t'est pas permis de prendre ton lit.
\VS{11}Il leur répondit : Celui qui m'a guéri m'a dit : Prends ton lit et marche.
\VS{12}Alors ils lui demandèrent : Qui est celui qui t'a dit : Prends ton lit et marche ?
\VS{13}Mais celui qui avait été guéri ne savait pas qui c'était, car Jésus s'était éclipsé du milieu de la foule qui était en ce lieu-là.
\VS{14}Depuis, Jésus le trouva dans le temple, et lui dit : Voici, tu as été guéri ; ne pèche plus désormais, de peur qu’il ne t'arrive quelque chose de pire.
\VS{15}Cet homme s'en alla, et rapporta aux Juifs que c'était Jésus qui l'avait guéri.
\VS{16}C'est pourquoi les Juifs poursuivaient Jésus et cherchaient à le faire mourir, parce qu'il avait fait ces choses le jour du sabbat.
\TextTitle{[Jésus déclare son égalité avec le Père]}
\VS{17}Mais Jésus leur répondit : Mon Père agit jusqu'à présent ; moi aussi, j’agis.
\VS{18}A cause de cela, les Juifs cherchaient encore plus à le faire mourir, parce que non seulement il avait violé le sabbat, mais aussi parce qu'il disait que Dieu était son propre Père, se faisant égal à Dieu.
\VS{19}Mais Jésus répondit et leur dit : En vérité, en vérité je vous le dis, le Fils ne peut rien faire de lui-même, il ne fait que ce qu'il voit faire au Père ; et tout ce que le Père fait, le Fils le fait pareillement.
\VS{20}Car le Père aime le Fils, et lui montre toutes les choses qu'il fait ; et il lui montrera de plus grandes œuvres que celles-ci, afin que vous soyez dans l'admiration.
\VS{21}Car comme le Père ressuscite les morts et donne la vie, de même aussi le Fils donne la vie à ceux qu'il veut.
\VS{22}Car le Père ne juge personne ; mais il a donné tout jugement au Fils,
\VS{23}afin que tous honorent le Fils, comme ils honorent le Père ; celui qui n'honore point le Fils, n'honore point le Père qui l'a envoyé.
\VS{24}En vérité, en vérité je vous le dis, celui qui entend ma parole, et croit à celui qui m'a envoyé, a la vie éternelle et ne vient pas en jugement, mais il est passé de la mort à la vie.
\TextTitle{[Les deux résurrections]}
\VS{25}En vérité, en vérité je vous le dis, l'heure vient, et elle est déjà venue, où les morts entendront la voix du Fils de Dieu, et ceux qui l'auront entendue vivront.
\VS{26}Car comme le Père a la vie en lui-même, ainsi il a donné au Fils d'avoir la vie en lui-même.
\VS{27}Et il lui a donné le pouvoir de juger parce qu'il est le Fils de l'homme.
\VS{28}Ne soyez point étonnés de cela ; car l'heure vient où tous ceux qui sont dans les sépulcres entendront sa voix, et en sortiront.
\VS{29}Ceux qui auront fait le bien, ressusciteront pour la vie, mais ceux qui auront fait le mal, ressusciteront pour le jugement.
\TextTitle{[Témoignages confirmant celui de Jésus]}
\VS{30}Je ne puis rien faire de moi-même : Je juge conformément à ce que j'entends, et mon jugement est juste ; car je ne cherche point ma volonté, mais la volonté du Père qui m'a envoyé.
\VS{31}Si je rends témoignage de moi-même, mon témoignage n'est pas digne de foi.
\VS{32}C'est un autre qui rend témoignage de moi, et je sais que le témoignage qu'il rend de moi est digne de foi.
\TextTitle{[a. Le témoignage de Jean-Baptiste]}
\VS{33}Vous avez envoyé une délégation vers Jean, et il a rendu témoignage à la vérité.
\VS{34}Or je ne cherche point le témoignage des hommes ; mais je dis ces choses afin que vous soyez sauvés.
\VS{35}Jean était une lampe ardente et brillante ; et vous avez voulu vous réjouir pour un peu de temps à sa lumière.
\TextTitle{[b. Le témoignage des oeuvres de Jésus]}
\VS{36}Mais moi, j'ai un témoignage plus grand que celui de Jean ; car les œuvres que mon Père m'a donné d’accomplir, ces œuvres mêmes que je fais, témoignent de moi que c’est mon Père qui m'a envoyé.
\TextTitle{[c. Le témoignage du Père]
\\(Mt. 3:17)}
\VS{37}Et le Père qui m'a envoyé, a lui-même rendu témoignage de moi. Vous n’avez jamais entendu sa voix, vous n’avez jamais vu sa face.
\VS{38}Et sa parole ne demeure point en vous, puisque vous ne croyez pas à celui qu'il a envoyé.
\TextTitle{[d. Le témoignage de l'Ecriture]
\\(Lu. 24:27,44)}
\VS{39}Vous sondez les Ecritures, car vous pensez avoir en elles la vie éternelle, et ce sont elles qui rendent témoignage de moi.
\VS{40}Et vous ne voulez pas venir à moi, pour avoir la vie.
\VS{41}Je ne tire pas ma gloire des hommes.
\VS{42}Mais je sais que vous n'avez point l'amour de Dieu en vous.
\VS{43}Je suis venu au Nom de mon Père, et vous ne me recevez pas, si un autre vient en son propre nom, vous le recevrez.
\VS{44}Comment pouvez-vous croire, puisque vous recevez la gloire les uns des autres, et ne cherchez point la gloire qui vient de Dieu seul ?
\VS{45}Ne croyez point que je vous accuserai devant mon Père ; Moïse sur qui vous vous fondez, est celui qui vous accusera.
\VS{46}Car si vous croyiez Moïse, vous me croiriez aussi ; parce qu’il a écrit à mon sujet.
\VS{47}Mais si vous ne croyez pas à ses écrits, comment croirez-vous à mes paroles ?
\TextTitle{[Une autre Pâque et la multiplication des pains pour les cinq mille hommes]
\\(Mt. 14:15-21 ; Mc. 6:32-44 ; Lu. 9:12-17)}
\Chap{6}
\VerseOne{}Après ces choses, Jésus s'en alla au-delà de la mer de Galilée, qui est la mer de Tibériade.
\VS{2}Une grande foule le suivait, parce qu’elle voyait les miracles qu'il opérait sur les malades.
\VS{3}Jésus monta sur une montagne, et il s'assit là avec ses disciples.
\VS{4}Or, la Pâque, la fête des Juifs, était proche.
\VS{5}Et Jésus ayant levé ses yeux, et voyant qu’une grande foule venait à lui, dit à Philippe : Où achèterons-nous des pains, afin que ces gens aient à manger ?
\VS{6}Il disait cela pour l'éprouver, car il savait bien ce qu'il allait faire.
\VS{7}Philippe lui répondit : Les pains qu’on aurait pour deux cents deniers ne suffiraient pas pour que chacun en reçoive un peu.
\VS{8}Un de ses disciples, André, frère de Simon Pierre, lui dit :
\VS{9}Il y a ici un petit garçon qui a cinq pains d'orge et deux poissons ; mais qu'est-ce que cela pour tant de gens ?
\VS{10}Alors Jésus dit : Faites asseoir les gens. Il y avait beaucoup d'herbe dans ce lieu. Ils s'assirent au nombre d'environ cinq mille.
\VS{11}Et Jésus prit les pains ; et après avoir rendu grâces, il les distribua aux disciples, et les disciples à ceux qui étaient assis, et de même des poissons, autant qu'ils en voulaient.
\VS{12}Et après qu'ils furent rassasiés, il dit à ses disciples : Ramassez les morceaux qui restent, afin que rien ne soit perdu.
\VS{13}Ils les ramassèrent donc, et ils remplirent douze paniers avec les morceaux qui restèrent des cinq pains d'orge, après que tous eurent mangé.
\VS{14}Ces gens, ayant vu le miracle que Jésus avait fait, disaient : Celui-ci est véritablement le Prophète qui devait venir dans le monde.
\TextTitle{[Jésus marche sur les eaux]
\\(Mt. 14:22-33 ; Mc. 6:45-52)}
\VS{15}Mais Jésus, sachant qu'ils allaient venir l'enlever pour le faire Roi, se retira encore, lui seul, sur la montagne.
\VS{16}Et quand le soir fut venu, ses disciples descendirent à la mer.
\VS{17}Etant montés dans la barque, ils traversaient la mer pour se rendre à Capernaüm. Il faisait déjà nuit, et Jésus ne les avait pas encore rejoints.
\VS{18}Il soufflait un grand vent, et la mer était agitée.
\VS{19}Après avoir ramé environ vingt-cinq ou trente stades, ils virent Jésus marchant sur la mer, et s'approchant de la barque. Et ils eurent peur.
\VS{20}Mais il leur dit : C’est moi, ne craignez point.
\VS{21}Ils le reçurent donc avec plaisir dans la barque, et aussitôt la barque aborda au lieu où ils allaient.
\TextTitle{[Jésus, le pain de vie]}
\VS{22}Le lendemain, la foule qui était restée de l'autre côté de la mer, vit qu’il ne se trouvait là qu’une seule barque, et que Jésus n’était pas monté avec ses disciples dans la barque, mais qu’ils étaient partis seuls.
\VS{23}Cependant, d’autres barques étaient arrivées de Tibériade près du lieu où ils avaient mangé le pain, après que le Seigneur eut rendu grâces.
\VS{24}Quand la foule vit que ni Jésus ni ses disciples n’étaient là, les gens montèrent eux-mêmes dans ces barques, et allèrent à Capernaüm chercher Jésus.
\VS{25}Et l'ayant trouvé au-delà de la mer, ils lui dirent : Rabbi, quand es-tu arrivé ici ?
\VS{26}Jésus leur répondit et leur dit : En vérité, en vérité je vous le dis : Vous me cherchez, non parce que vous avez vu des miracles, mais parce que vous avez mangé des pains et que vous avez été rassasiés.
\VS{27}Travaillez, non pour la nourriture qui périt, mais pour celle qui est permanente jusqu’à la vie éternelle, et que le Fils de l'homme vous donnera ; car c’est lui que le Père, que Dieu, a marqué de son sceau.
\VS{28}Ils lui dirent donc : Que devons-nous faire pour accomplir les œuvres de Dieu ?
\VS{29}Jésus répondit et leur dit : C’est ici l’œuvre de Dieu, que vous croyiez en celui qu'il a envoyé.
\TextTitle{[Jésus envoyé du ciel]}
\VS{30}Alors ils lui dirent : Quel miracle fais-tu donc, afin que nous le voyions, et que nous croyions en toi ? Quelle œuvre fais-tu ?
\VS{31}Nos pères ont mangé la manne dans le désert ; selon ce qui est écrit : Il leur a donné à manger le pain du ciel\FTNT{} (1).
\VS{32}Mais Jésus leur dit : En vérité, en vérité je vous le dis : Moïse ne vous a pas donné le pain du ciel ; mais mon Père vous donne le vrai pain du ciel.
\VS{33}Car le pain de Dieu c'est celui qui est descendu du ciel et qui donne la vie au monde.
\VS{34}Ils lui dirent donc : Seigneur, donne-nous toujours ce pain-là.
\VS{35}Et Jésus leur dit : Je suis le pain de vie. Celui qui vient à moi, n'aura jamais faim ; et celui qui croit en moi, n'aura jamais soif.
\VS{36}Mais, je vous ai dit que vous m'avez vu, et cependant vous ne croyez point.
\VS{37}Tous ceux que mon Père me donne viendront à moi ; et je ne mettrai point dehors celui qui viendra à moi.
\VS{38}Car je suis descendu du ciel, non point pour faire ma volonté, mais la volonté de celui qui m'a envoyé.
\VS{39}Or, la volonté du Père qui m'a envoyé, c’est que je ne perde aucun de tous ceux qu'il m'a donnés, mais que je les ressuscite au dernier jour.
\VS{40}La volonté de celui qui m'a envoyé, c’est que quiconque contemple le Fils, et croit en lui, ait la vie éternelle ; et je le ressusciterai au dernier jour.
\VS{41}Les Juifs murmuraient contre lui de ce qu'il avait dit : Je suis le pain qui est descendu du ciel.
\VS{42}Et ils disaient : N’est-ce pas là Jésus, le fils de Joseph, celui dont nous connaissons le père et la mère ? Comment donc dit-il : Je suis descendu du ciel ?
\VS{43}Jésus leur répondit et leur dit : Ne murmurez pas entre vous.
\VS{44}Nul ne peut venir à moi, si le Père qui m'a envoyé ne l’attire ; et je le ressusciterai au dernier jour.
\VS{45}Il est écrit dans les prophètes : Ils seront tous enseignés de Dieu. Ainsi, quiconque a entendu le Père et a été instruit de ses intentions, vient à moi.
\VS{46}C’est que nul n’a vu le Père, sinon celui qui vient de Dieu, celui-là a vu le Père.
\VS{47}En vérité, en vérité je vous le dis : Celui qui croit en moi a la vie éternelle.
\VS{48}Je suis le pain de vie.
\VS{49}Vos pères ont mangé la manne dans le désert, et ils sont morts.
\VS{50}C'est ici le pain qui est descendu du ciel, afin que celui qui en mange, ne meure point.
\VS{51}Je suis le pain vivant qui est descendu du ciel. Si quelqu'un mange de ce pain, il vivra éternellement ; et le pain que je donnerai, c'est ma chair, que je donnerai pour la vie du monde.
\VS{52}Les Juifs donc discutaient entre eux, et disaient : Comment peut-il nous donner sa chair à manger ?
\VS{53}Et Jésus leur dit : En vérité, en vérité je vous le dis : Si vous ne mangez pas la chair du Fils de l'homme, et ne buvez pas son sang, vous n'aurez point la vie en vous-mêmes.
\VS{54}Celui qui mange ma chair, et qui boit mon sang, a la vie éternelle ; et je le ressusciterai au dernier jour.
\VS{55}Car ma chair est une véritable nourriture, et mon sang est un véritable breuvage.
\VS{56}Celui qui mange ma chair, et qui boit mon sang, demeure en moi, et moi en lui.
\VS{57}Comme le Père qui est vivant m'a envoyé, et que je suis vivant par le Père ; ainsi celui qui me mangera, vivra aussi par moi.
\VS{58}C'est ici le pain qui est descendu du ciel. Il n’en est pas comme de vos pères qui ont mangé la manne, et qui sont morts ; celui qui mangera ce pain, vivra éternellement.
\VS{59}Il dit ces choses dans la synagogue, enseignant à Capernaüm.
\TextTitle{[Epreuve de la consécration des disciples]
\\(Mt. 8:19-22 ; 10:36 ; Lu. 9:23-26)}
\VS{60}Plusieurs de ses disciples l'ayant entendu, dirent : Cette parole est dure, qui peut l’écouter ?
\VS{61}Mais Jésus sachant en lui-même que ses disciples murmuraient à ce sujet, leur dit : Cela vous scandalise-t-il ?
\VS{62}Que sera-ce donc si vous voyez le Fils de l'homme monter où il était auparavant ?
\VS{63}C'est l'Esprit qui vivifie ; la chair ne sert à rien. Les paroles que je vous ai dites, sont Esprit et vie.
\VS{64}Mais il en est parmi vous qui ne croient point. En effet, Jésus savait dès le commencement qui étaient ceux qui ne croiraient point, et qui était celui qui le trahirait.
\VS{65}Il leur dit donc : C’est pour cela que je vous ai dit, que nul ne peut venir à moi, si cela ne lui a pas été donné par mon Père.
\TextTitle{[Pierre reconnaît Jésus comme le Christ]
\\(Mt. 16:13-16 ; Mc. 8:27-30 ; Lu. 9:18-21}
\VS{66}Dès ce moment, plusieurs de ses disciples l'abandonnèrent, et ils ne marchèrent plus avec lui.
\VS{67}Et Jésus dit aux douze : Et vous, ne voulez-vous pas aussi vous en aller ?
\VS{68}Mais Simon Pierre lui répondit : Seigneur ! Auprès de qui irions-nous ? Tu as les paroles de la vie éternelle.
\VS{69}Et nous avons cru, et nous avons connu que tu es le Christ, le Fils du Dieu vivant.
\VS{70}Jésus leur répondit : Ne vous ai-je pas choisis, vous les douze ? Et toutefois l'un de vous est un démon.
\VS{71}Il parlait de Judas Iscariot, fils de Simon ; car c'était lui qui devait le trahir, quoiqu'il fût l'un des douze.
\TextTitle{[Jésus engagé par ses frères incrédules à se rendre à Jérusalem]}
\Chap{7}
\VerseOne{}Après ces choses, Jésus parcourait la Galilée, car il ne voulait pas parcourir la Judée, parce que les Juifs cherchaient à le faire mourir.
\VS{2}Or la fête des Juifs, appelée la fête des tabernacles, était proche.
\VS{3}Et ses frères lui dirent : Pars d'ici, et va en Judée, afin que tes disciples aussi contemplent les œuvres que tu fais.
\VS{4}Personne n’agit en secret, lorsqu'il cherche à être connu ; si tu fais ces choses, montre-toi toi-même au monde.
\VS{5}Car ses frères non plus ne croyaient pas en lui.
\VS{6}Et Jésus leur dit : Mon temps n'est pas encore venu, mais votre temps est toujours prêt.
\VS{7}Le monde ne peut pas vous haïr, mais il me hait parce que je rends témoignage contre lui que ses œuvres sont mauvaises.
\VS{8}Montez, vous, à cette fête ; pour moi, je n’y monte pas encore, parce que mon temps n'est pas encore accompli.
\VS{9}Après leur avoir dit ces choses, il resta en Galilée.
\TextTitle{[Jésus à la fête des tabernacles]}
\VS{10}Lorsque ses frères furent montés, alors il y monta aussi lui-même, non publiquement, mais comme en secret.
\VS{11}Les Juifs le cherchaient pendant la fête, et ils disaient : Où est-il ?
\VS{12}Et il y avait un grand murmure à son sujet parmi la foule. Les uns disaient : C’est un homme de bien ; et les autres disaient : Non, il séduit le peuple.
\VS{13}Toutefois personne ne parlait franchement de lui, à cause de la crainte qu'on avait des Juifs.
\VS{14}Vers le milieu de la fête, Jésus monta au temple. Et il enseignait.
\VS{15}Les Juifs s’étonnaient, disant : Comment connaît-il les Ecritures, lui qui n’a point étudié ?
\VS{16}Jésus leur répondit et dit : Ma doctrine n'est pas de moi, mais de celui qui m'a envoyé.
\VS{17}Si quelqu'un veut faire sa volonté, il connaîtra si ma doctrine est de Dieu, ou si je parle de moi-même.
\VS{18}Celui qui parle de son propre chef cherche sa propre gloire ; mais celui qui cherche la gloire de celui qui l'a envoyé, est véritable, et il n'y a point d'injustice en lui.
\VS{19}Moïse ne vous a-t-il pas donné la loi ? Cependant, nul de vous n'observe la loi. Pourquoi cherchez-vous à me faire mourir ?
\VS{20}La foule répondit : Tu as un démon ; qui est-ce qui cherche à te faire mourir ?
\VS{21}Jésus répondit et leur dit : J’ai fait une œuvre, et vous en êtes tous étonnés.
\VS{22}Moïse vous a donné la circoncision, non qu’elle vienne de Moïse, mais des pères, vous circoncisez bien un homme le jour du sabbat.
\VS{23}Si un homme reçoit la circoncision le jour du sabbat, afin que la loi de Moïse ne soit pas violée, pourquoi êtes-vous irrités contre moi de ce que j'ai guéri un homme tout entier le jour du sabbat ?
\VS{24}Ne jugez pas selon les apparences, mais jugez selon la justice.
\VS{25}Alors quelques-uns de ceux de Jérusalem disaient : N'est-ce pas celui qu'ils cherchent à faire mourir ?
\VS{26}Et cependant voici, il parle librement, et ils ne lui disent rien ! Est-ce que vraiment les chefs auraient reconnu qu’il est véritablement le Christ ?
\VS{27}Cependant celui-ci, nous savons d'où il est ; mais quand le Christ viendra, personne ne saura d'où il est.
\VS{28}Jésus, enseignant dans le temple, s’écria : Vous me connaissez, et vous savez d'où je suis ! Je ne suis pas venu de moi-même, mais celui qui m'a envoyé est véritable, et vous ne le connaissez pas.
\VS{29}Mais moi, je le connais ; car je viens de lui, et c'est lui qui m'a envoyé.
\VS{30}Ils cherchaient donc à se saisir de lui, mais personne ne mit la main sur lui, parce que son heure n'était pas encore venue.
\VS{31}Cependant, plusieurs parmi la foule crurent en lui, et ils disaient : Quand le Christ sera venu, fera-t-il plus de miracles que celui-ci n'a fait ?
\VS{32}Les pharisiens entendirent la foule murmurant ces choses de lui. Alors les principaux sacrificateurs et les pharisiens envoyèrent des huissiers pour le prendre.
\VS{33}Et Jésus leur dit : Je suis encore pour un peu de temps avec vous, puis je m'en vais vers celui qui m'a envoyé.
\VS{34}Vous me chercherez, mais vous ne me trouverez pas, et vous ne pouvez pas venir où je serai.
\VS{35}Les Juifs dirent donc entre eux : Où ira-t-il, pour que nous ne le trouvions pas ? Ira-t-il parmi ceux qui sont dispersés chez les Grecs, et enseignera-t-il les Grecs ?
\VS{36}Quel est ce discours qu'il a tenu : Vous me chercherez, mais vous ne me trouverez pas, vous ne pouvez pas venir où je serai ?
\TextTitle{[La grande prophétie sur le secret de la puissance du Saint-Esprit]
\\(Ac. 2:2-4 ; Jn. 4:14)}
\VS{37}Le dernier jour, le grand jour de la fête, Jésus, se tenant debout, s’écria : Si quelqu'un a soif, qu'il vienne à moi, et qu'il boive.
\VS{38}Celui qui croit en moi, des fleuves d'eau vive couleront de son sein, comme dit l'Ecriture.
\VS{39}Il dit cela de l'Esprit que devaient recevoir ceux qui croiraient en lui ; car le Saint-Esprit n'était pas encore donné, parce que Jésus n'était pas encore glorifié.
\TextTitle{[Diversité d'opinions au sujet de Jésus]}
\VS{40}Plusieurs de la foule ayant entendu ce discours, disaient : Celui-ci est véritablement le Prophète.
\VS{41}Les autres disaient : Celui-ci est le Christ. Et les autres disaient : Est-ce bien de la Galilée que doit venir le Christ ?
\VS{42}L'Ecriture ne dit-elle pas que le Christ doit venir de la postérité de David, et du village de Bethléem, où était David ?
\VS{43}Il y eut donc division parmi la foule à cause de lui.
\VS{44}Et quelques-uns d'entre eux voulaient le saisir, mais personne ne mit la main sur lui.
\VS{45}Ainsi les huissiers retournèrent vers les principaux sacrificateurs et les pharisiens, qui leur dirent : Pourquoi ne l'avez-vous pas amené ?
\VS{46}Les huissiers répondirent : Jamais homme n’a parlé comme cet homme.
\VS{47}Mais les pharisiens leur répondirent : Est-ce que vous aussi, vous avez été séduits ?
\VS{48}Y a-t-il quelqu’un des chefs ou des pharisiens qui ait cru en lui ?
\VS{49}Mais cette foule, qui ne connaît pas la loi, ce sont des maudits.
\VS{50}Nicodème, qui était venu vers Jésus de nuit, et qui était l'un d'entre eux, leur dit :
\VS{51}Notre loi condamne-t-elle un homme avant qu’on l’entende et qu’on ne sache ce qu’il a fait ?
\VS{52}Ils lui répondirent : Es-tu aussi Galiléen ? Examine, et tu verras qu'aucun prophète n’est sorti de la Galilée.
\VS{53}Et chacun s'en alla dans sa maison.
\TextTitle{[Les scribes et les pharisiens accusent une femme surprise en flagrant délit d'adultère]}
\Chap{8}
\VerseOne{}Jésus se rendit à la montagne des oliviers.
\VS{2}Et, dès le matin, il alla de nouveau dans le temple, et tout le peuple vint à lui ; et s'étant assis, il les enseignait.
\VS{3}Alors les scribes et les pharisiens lui amenèrent une femme surprise en adultère ;
\VS{4}et l'ayant placée au milieu du peuple, ils dirent à Jésus : Maître, cette femme a été surprise en flagrant délit d’adultère.
\VS{5}Moïse nous a ordonné dans la loi de lapider celles qui sont dans son cas ; toi donc qu'en dis-tu ?
\VS{6}Or ils disaient cela pour l'éprouver, afin de pouvoir l'accuser. Mais Jésus s'étant penché en bas, écrivait avec son doigt sur la terre.
\VS{7}Et comme ils continuaient à l'interroger, s'étant relevé, il leur dit : Que celui de vous qui est sans péché, jette le premier la pierre contre elle.
\VS{8}Et s'étant encore baissé, il écrivait sur la terre.
\VS{9}Quand ils entendirent cela, accusés par leur conscience, ils se retirèrent un à un, depuis les plus âgés jusqu’aux derniers ; et Jésus resta seul avec la femme qui était là au milieu.
\VS{10}Alors Jésus s'étant relevé, et ne voyant plus que la femme, il lui dit : Femme, où sont ceux qui t'accusaient ? Personne ne t’a-t-il condamnée ?
\VS{11}Elle dit : Non, Seigneur. Et Jésus lui dit : Je ne te condamne pas non plus ; va, et ne pèche plus.
\TextTitle{[Point crucial du conflit entre Jésus et les pharisiens : l'origine de Christ, Lumière du monde]
\\(Jn. 1:9)}
\VS{12}Et Jésus leur parla encore, en disant : Je suis la Lumière du monde ; celui qui me suit ne marchera pas dans les ténèbres, mais il aura la lumière de la vie.
\VS{13}Alors les pharisiens lui dirent : Tu rends témoignage de toi-même, ton témoignage n'est pas digne de foi.
\VS{14}Jésus répondit et leur dit : Quoique je rende témoignage de moi-même, mon témoignage est digne de foi ; car je sais d'où je suis venu et où je vais ; mais vous ne savez pas d'où je viens ni où je vais.
\VS{15}Vous jugez selon la chair, mais moi, je ne juge personne.
\VS{16}Et si je juge, mon jugement est digne de foi ; car je ne suis pas seul, mais le Père qui m'a envoyé est avec moi.
\VS{17}Il est même écrit dans votre loi que le témoignage de deux hommes est digne de foi\FTNT{De. 19:15.}.
\VS{18}Je rends témoignage de moi-même, et le Père qui m'a envoyé rend aussi témoignage de moi.
\VS{19}Alors ils lui dirent : Où est ton Père ? Jésus répondit : Vous ne connaissez ni moi ni mon Père. Si vous me connaissiez, vous connaîtriez aussi mon Père.
\VS{20}Jésus dit ces paroles au lieu où était le trésor, enseignant dans le temple ; mais personne ne le saisit, parce que son heure n'était pas encore venue.
\VS{21}Et Jésus leur dit encore : Je m'en vais, et vous me chercherez, et vous mourrez dans vos péchés ; vous ne pouvez pas venir où je vais.
\VS{22}Les Juifs disaient donc : Se tuera-t-il lui-même, puisqu’il dit : Vous ne pouvez pas venir où je vais ?
\VS{23}Alors il leur dit : Vous êtes d'en bas, mais moi, je suis d'en haut ; vous êtes de ce monde, mais moi, je ne suis pas de ce monde.
\VS{24}C'est pourquoi je vous ai dit que vous mourrez dans vos péchés ; car si vous ne croyez pas que je suis l'envoyé de Dieu, vous mourrez dans vos péchés.
\VS{25}Alors ils lui dirent : Toi, qui es-tu ? Et Jésus leur dit : Ce que je vous dis dès le commencement.
\VS{26}J'ai beaucoup de choses à dire de vous et à juger en vous, mais celui qui m'a envoyé est véritable, et les choses que j'ai entendues de lui, je les dis au monde.
\VS{27}Ils ne comprirent point qu'il leur parlait du Père.
\VS{28}Jésus leur dit donc : Quand vous aurez élevé le Fils de l'homme, vous connaîtrez alors que je suis l'envoyé de Dieu, et que je ne fais rien de moi-même, mais que je dis ces choses selon ce que mon Père m'a enseigné.
\VS{29}Celui qui m'a envoyé est avec moi ; le Père ne m'a pas laissé seul, parce que je fais toujours les choses qui lui plaisent.
\VS{30}Comme il disait ces choses, plusieurs crurent en lui.
\VS{31}Et Jésus disait aux Juifs qui avaient cru en lui : Si vous demeurez dans ma parole, vous serez vraiment mes disciples.
\VS{32}Vous connaîtrez la vérité, et la vérité vous rendra libres.
\VS{33}Ils lui répondirent : Nous sommes la postérité d'Abraham, et nous ne fûmes jamais esclaves de personne ; comment donc dis-tu : Vous deviendrez libres ?
\VS{34}Jésus leur répondit : En vérité, en vérité je vous le dis : Quiconque se livre au péché, est esclave du péché.
\VS{35}Or l'esclave ne demeure pas toujours dans la maison ; le fils y demeure toujours.
\VS{36}Si donc le Fils vous affranchit, vous serez véritablement libres.
\VS{37}Je sais que vous êtes la postérité d'Abraham, pourtant vous cherchez à me faire mourir, parce que ma parole n'est pas reçue dans vos cœurs.
\VS{38}Je vous dis ce que j'ai vu chez mon Père ; et vous aussi vous faites les choses que vous avez vues chez votre père.
\VS{39}Ils répondirent et lui dirent : Notre père c'est Abraham. Jésus leur dit : Si vous étiez enfants d'Abraham, vous feriez les œuvres d'Abraham.
\VS{40}Mais maintenant vous cherchez à me faire mourir, moi, un homme qui vous ai dit la vérité que j'ai entendue de Dieu. Cela, Abraham ne l’a point fait.
\VS{41}Vous faites les œuvres de votre père. Et ils lui dirent : Nous ne sommes pas des enfants illégitimes ; nous avons un seul père, Dieu.
\VS{42}Mais Jésus leur dit : Si Dieu était votre Père, certes vous m'aimeriez, car c’est de Dieu que je suis sorti et que je viens ; je ne suis pas venu de moi-même, mais c'est lui qui m'a envoyé.
\VS{43}Pourquoi ne comprenez-vous pas mon langage ? C’est parce que vous ne pouvez pas écouter ma parole.
\VS{44}Vous avez pour père le diable, et vous voulez accomplir les désirs de votre père. Il a été meurtrier dès le commencement, et il n'a pas persévéré dans la vérité, car la vérité n'est pas en lui. Toutes les fois qu'il profère le mensonge, il parle de son propre fond ; car il est menteur et le père du mensonge.
\VS{45}Et moi, parce que je dis la vérité, vous ne me croyez pas.
\VS{46}Qui de vous me convaincra de péché ? Si je dis la vérité, pourquoi ne me croyez-vous pas ?
\VS{47}Celui qui est de Dieu écoute les paroles de Dieu ; vous n’écoutez pas, parce que vous n'êtes pas de Dieu.
\VS{48}Alors les Juifs répondirent : N’avons-nous pas raison de dire que tu es un Samaritain, et que tu as un démon ?
\VS{49}Jésus répondit : Je n'ai point un démon, mais j'honore mon Père, et vous m’outragez.
\VS{50}Je ne cherche point ma gloire ; il y en a un qui la cherche, et qui juge.
\VS{51}En vérité, en vérité je vous le dis : Si quelqu'un garde ma parole, il ne verra jamais la mort.
\VS{52}Les Juifs lui dirent donc : Maintenant nous savons que tu as un démon. Abraham est mort, et les prophètes aussi, et tu dis : Si quelqu'un garde ma parole, il ne verra jamais la mort.
\VS{53}Es-tu plus grand que notre père Abraham qui est mort ? Les prophètes aussi sont morts. Qui prétends-tu être ?
\VS{54}Jésus répondit : Si je me glorifie moi-même, ma gloire n'est rien ; mon Père est celui qui me glorifie, celui que vous dites être votre Dieu.
\VS{55}Toutefois vous ne l'avez point connu, mais moi je le connais ; et si je disais que je ne le connais point, je serais un menteur, semblable à vous ; mais je le connais, et je garde sa parole.
\VS{56}Abraham votre père a tressailli de joie de ce qu’il verrait mon jour ; et il l'a vu, et il s’est réjoui.
\VS{57}Les Juifs lui dirent : Tu n'as pas encore cinquante ans, et tu as vu Abraham !
\VS{58}Jésus leur dit : En vérité, en vérité je vous le dis : Avant qu'Abraham fût, Je suis\FTNT{Je suis : L'évangile de Jean rapporte plusieurs déclarations incroyables que Jésus a faites à son sujet : Je suis le pain de vie (6:35), Je suis la Lumière du monde (8:12), Je suis le bon berger (10:11), Je suis la porte (10:7), Je suis la résurrection (11:25), Je suis le chemin, la vérité et la vie (14:6), Je suis la vraie vigne (15:1). Toutefois, dans ce verset, en déclarant être ~Je suis~, il s’identifie clairement au Nom que YHWH avait révélé à Moïse dans Ex. 3:14. C'est précisément pour cette raison que les juifs ont voulu le lapider.}.
\VS{59}Alors ils prirent des pierres pour les jeter contre lui, mais Jésus se cacha et sortit du temple, passant au milieu d'eux ; et ainsi il s'en alla.
\TextTitle{[Jésus guérit un aveugle-né]}
\Chap{9}
\VerseOne{}Comme Jésus passait, il vit un homme aveugle de naissance.
\VS{2}Ses disciples lui posèrent cette question : Rabbi, qui a péché ? Cet homme ou ses parents pour qu’il soit né aveugle ?
\VS{3}Jésus répondit : Ce n’est pas que lui ou ses parents aient péché ; mais c'est afin que les œuvres de Dieu soient manifestées en lui.
\VS{4}Il faut que je fasse, tandis qu’il est jour, les œuvres de celui qui m'a envoyé. La nuit vient, où personne ne peut travailler.
\VS{5}Pendant que je suis dans le monde, je suis la Lumière du monde.
\VS{6}Ayant dit ces paroles, il cracha à terre et fit de la boue avec sa salive, et mit de cette boue sur les yeux de l'aveugle.
\VS{7}Et il lui dit : Va, et lave-toi au réservoir de Siloé (nom qui veut dire envoyé). Il y alla donc, se lava, et s’en retourna voyant clair.
\VS{8}Ses voisins et ceux qui auparavant l’avaient connu comme mendiant disaient : N'est-ce pas celui qui était assis et qui mendiait ?
\VS{9}Les uns disaient : C’est lui. Et les autres disaient : Il lui ressemble. Mais lui-même disait : C'est moi.
\VS{10}Ils lui dirent donc : Comment tes yeux ont-ils été ouverts ?
\VS{11}Il répondit et dit : Cet homme, qu'on appelle Jésus, a fait de la boue et il l'a mise sur mes yeux, et m'a dit : Va au réservoir de Siloé et lave-toi. J’y suis allé, je me suis lavé, et j’ai recouvert la vue.
\VS{12}Alors ils lui dirent : Où est cet homme ? Il répondit : Je ne sais pas.
\VS{13}Ils amenèrent vers les pharisiens celui qui auparavant avait été aveugle.
\VS{14}Or c'était en un jour de sabbat que Jésus avait fait de la boue et lui avait ouvert les yeux.
\VS{15}C'est pourquoi les pharisiens l'interrogèrent encore, comment il avait pu voir ; et il leur dit : Il a mis de la boue sur mes yeux, et je me suis lavé, et je vois.
\VS{16}Sur quoi quelques-uns des pharisiens dirent : Cet homme n'est pas un envoyé de Dieu ; car il n’observe pas le sabbat ; mais d'autres disaient : Comment un homme pécheur peut-il faire de tels prodiges ? Et il y avait de la division entre eux.
\VS{17}Ils dirent encore à l'aveugle : Toi, que dis-tu de lui, sur ce qu'il t'a ouvert les yeux ? Il répondit : C’est un Prophète.
\VS{18}Mais les Juifs ne crurent point que cet homme avait été aveugle, et qu'il avait pu voir, jusqu'à ce qu'ils aient fait venir ses parents.
\VS{19}Et ils les interrogèrent, disant : Est-ce là votre fils, que vous dites être né aveugle ? Comment donc voit-il maintenant ?
\VS{20}Ses parents leur répondirent : Nous savons que c'est notre fils et qu'il est né aveugle.
\VS{21}Mais comment il voit maintenant, ou qui lui a ouvert les yeux, nous ne le savons pas ; il a de l'âge, interrogez-le, il parlera de ce qui le regarde.
\VS{22}Ses parents dirent ces choses parce qu'ils craignaient les Juifs ; car les Juifs avaient déjà convenu que si quelqu'un reconnaissait Jésus pour le Christ, il serait exclu de la synagogue.
\VS{23}C’est pourquoi ses parents dirent : Il a de l'âge, interrogez-le lui-même.
\VS{24}Ils appelèrent donc pour la seconde fois l'homme qui avait été aveugle et ils lui dirent : Donne gloire à Dieu ; nous savons que cet homme est un pécheur.
\VS{25}Il répondit : Je ne sais pas si c’est un pécheur ; je sais une chose, c’est que j’étais aveugle et que maintenant je vois.
\VS{26}Ils lui dirent donc encore : Que t'a-t-il fait ? Comment a-t-il ouvert tes yeux ?
\VS{27}Il leur répondit : Je vous l'ai déjà dit, et vous ne l'avez point écouté, pourquoi voulez-vous l’entendre encore ? Voulez-vous aussi devenir ses disciples ?
\VS{28}Alors ils l'injurièrent et lui dirent : C’est toi son disciple ; nous, nous sommes disciples de Moïse.
\VS{29}Nous savons que Dieu a parlé à Moïse ; mais celui-ci, nous ne savons pas d'où il est.
\VS{30}Cet homme répondit : Certes, c'est une chose étrange que vous ne sachiez point d'où il est ; et toutefois il a ouvert mes yeux.
\VS{31}Nous savons que Dieu n'exauce point les méchants, mais si quelqu'un est pieux envers Dieu, et fait sa volonté, il l'exauce.
\VS{32}Jamais on n’a entendu dire que quelqu’un ait ouvert les yeux d’un aveugle-né.
\VS{33}Si cet homme n'était pas un envoyé de Dieu, il ne pourrait rien faire de semblable.
\VS{34}Ils répondirent : Tu es entièrement né dans le péché, et tu nous enseignes ! Et ils le chassèrent dehors.
\TextTitle{[Jésus affirme sa divinité]}
\VS{35}Jésus apprit qu'ils l'avaient chassé dehors ; et l'ayant rencontré, il lui dit : Crois-tu au Fils de Dieu ?
\VS{36}Cet homme lui répondit : Qui est-il Seigneur, afin que je croie en lui ?
\VS{37}Jésus lui dit : Tu l'as vu, et c'est celui qui te parle.
\VS{38}Alors il dit : Je crois, Seigneur ; et il l'adora\FTNT{Au travers de la lecture de la Bible, on constate que les anges refusent l’adoration (Ap. 19:9-10) de même que les apôtres (Ac. 10:25-26 ; Ac. 14:5-18). Seul Dieu accepte l’adoration puisqu’il en est le seul digne. Jésus n’a jamais refusé l’adoration des hommes, car il est Dieu.}.
\VS{39}Et Jésus dit : Je suis venu dans ce monde pour exercer le jugement, afin que ceux qui ne voient point voient ; et que ceux qui voient deviennent aveugles.
\VS{40}Quelques pharisiens qui étaient avec lui, ayant entendu ces paroles, dirent : Et nous, sommes-nous aussi aveugles ?
\VS{41}Jésus leur répondit : Si vous étiez aveugles, vous n'auriez point de péché ; mais maintenant vous dites : Nous voyons. C’est à cause de cela que votre péché demeure.
\TextTitle{[Jésus, le Bon Berger]
\\(Ps. 23 ; Hé. 13:20 ; 1Pi. 5:4)}
\Chap{10}
\VerseOne{}En vérité, en vérité je vous le dis : Celui qui n'entre point par la porte dans la bergerie des brebis, mais y monte par ailleurs, est un voleur et un brigand.
\VS{2}Mais celui qui entre par la porte est le berger des brebis.
\VS{3}Le portier lui ouvre, et les brebis entendent sa voix, et il appelle les brebis qui lui appartiennent par leur nom, et il les conduit dehors.
\VS{4}Lorsqu’il a fait sortir toutes ses brebis dehors, il marche devant elles, et les brebis le suivent, parce qu'elles connaissent sa voix.
\VS{5}Mais elles ne suivront point un étranger, au contraire, elles fuiront loin de lui ; parce qu'elles ne connaissent point la voix des étrangers.
\VS{6}Jésus leur dit cette parabole, mais ils ne comprirent pas ce qu'il leur disait.
\VS{7}Jésus leur dit encore : En vérité, en vérité je vous le dis : Je suis la Porte par où entrent les brebis\FTNT{La porte des brebis était située près du temple et avait été bâtie du temps de Néhémie (Né. 3:1). Les animaux que l’on sacrifiait à Dieu franchissaient probablement cette porte.}.
\VS{8}Tout ceux qui sont venus avant moi sont des brigands et des voleurs ; mais les brebis ne les ont point écoutés.
\VS{9}Je suis la Porte : Si quelqu'un entre par moi, il sera sauvé ; il entrera et il sortira, et il trouvera des pâturages.
\VS{10}Le voleur ne vient que pour dérober, tuer et détruire ; moi, je suis venu afin que mes brebis aient la vie, et qu'elles l'aient même en abondance.
\VS{11}Je suis le bon berger : Le bon berger donne sa vie pour ses brebis.
\VS{12}Mais le mercenaire, qui n’est pas le berger, à qui n'appartiennent pas les brebis, voit venir le loup, abandonne les brebis, et s'enfuit ; et le loup ravit et disperse les brebis.
\VS{13}Ainsi le mercenaire s'enfuit, parce qu'il est mercenaire, et qu'il ne se soucie pas des brebis. Je suis le bon berger.
\VS{14}Je connais mes brebis, et mes brebis me connaissent.
\VS{15}Comme le Père me connaît, et comme je connais le Père ; et je donne ma vie pour mes brebis.
\VS{16}J'ai encore d'autres brebis qui ne sont pas de cette bergerie ; celles-là, il faut aussi que je les amène ; elles entendront ma voix, et il y aura un seul troupeau, et un seul berger.
\VS{17}Le Père m'aime, parce que je donne ma vie, afin de la reprendre.
\VS{18}Personne ne me l'ôte, mais je la donne de moi-même. J'ai le pouvoir de la donner, et j’ai le pouvoir de la reprendre ; j'ai reçu cet ordre de mon Père.
\VS{19}Il y eut de nouveau division parmi les Juifs à cause de ces discours.
\VS{20}Car plusieurs disaient : Il a un démon, il est fou ! Pourquoi l'écoutez-vous ?
\VS{21}Et les autres disaient : Ce ne sont pas les paroles d'un démoniaque ; un démon peut-il ouvrir les yeux des aveugles ?
\TextTitle{[Jésus réaffirme sa divinité]
\\(Jn. 5:26-27 ; 14:9 ; 20:28-29)}
\VS{22}On célébrait la fête de la dédicace\FTNT{Le terme ~dédicace~ est la traduction du mot hébreu ~Hanoukka~ qui sert à désigner la consécration ou l'inauguration de l'autel servant à offrir des sacrifices à Dieu (No. 7:10 ; 2 Ch. 7:9). La Bible l’utilise aussi pour parler de l'inauguration des murailles de Jérusalem après leur reconstruction au temps de Néhémie (Né. 12:27). La fête d’Hanoukka a été instituée par Judas Maccabé en 164 av. J.-C. en mémoire de la purification du temple qui avait été profané par Antiochus Epiphane. Elle débute le 25 du mois de chisleu (mi décembre) de chaque année et dure huit jours.} à Jérusalem. Et c'était l’hiver.
\VS{23}Et Jésus se promenait dans le temple, au portique de Salomon.
\VS{24}Et les Juifs l’entourèrent et lui dirent : Jusqu’à quand tiendras-tu notre âme en suspens ? Si tu es le Christ, dis-le-nous franchement.
\VS{25}Jésus leur répondit : Je vous l'ai dit, et vous ne le croyez point. Les œuvres que je fais au Nom de mon Père rendent témoignage de moi.
\VS{26}Mais vous ne croyez point, parce que vous n'êtes point de mes brebis, comme je vous l'ai dit.
\VS{27}Mes brebis entendent ma voix ; je les connais, et elles me suivent.
\VS{28}Et moi, je leur donne la vie éternelle, et elles ne périront jamais ; et personne ne les ravira de ma main.
\VS{29}Mon Père, qui me les a données, est plus grand que tous ; et personne ne peut les ravir des mains de mon Père.
\VS{30}Moi et le Père nous sommes un.
\VS{31}Alors les Juifs prirent de nouveau des pierres pour le lapider.
\VS{32}Jésus leur dit : Je vous ai fait voir plusieurs bonnes œuvres de la part de mon Père : Pour laquelle me lapidez-vous ?
\VS{33}Les Juifs répondirent : Ce n’est pas pour une bonne œuvre que nous te lapidons, mais pour un blasphème, parce que toi qui es un homme, tu te fais Dieu.
\VS{34}Jésus leur répondit : N’est-il pas écrit dans votre loi : J'ai dit : Vous êtes des dieux\FTNT{Ps. 82:6 : Le sens du mot ~dieu~ peut désigner des personnes ayant un certain pouvoir. D'ailleurs, le mot hébreu utilisé dans Ps. 82:6 est ~Elohim~, or ce mot signifie aussi ~juge~. De plus, dans le contexte du psaume, ~vous êtes des dieux~ ne s'applique pas à tous, mais seulement à une certaine catégorie de personnes qui exerçaient un pouvoir en Israël : rois, scribes, souverains sacrificateurs… Rappelons-nous aussi que Dieu a fait de Moïse un dieu pour Aaron (Ex. 7:1-2), mais cela n’a pas fait de lui le Dieu Créateur pour autant. En Jn. 17:3, Jésus atteste qu'il n’y a qu’un seul vrai Dieu. Satan veut nous faire croire que nous sommes des dieux et nous amener ainsi à pécher par l’orgueil (Ge. 3:5). Toutefois, comme le souligne si bien l’apôtre Paul, même s’il existe des créatures qu’on appelle dieux ou déesses, il ne reste pas moins vrai qu’il n’y a qu’un seul Dieu (1 Co. 8:5-7).} ?
\VS{35}Si elle a appelé dieux ceux à qui la parole de Dieu est adressée, et cependant l'Ecriture ne peut être anéantie,
\VS{36}celui que le Père a sanctifié et envoyé dans le monde, vous lui dites : Tu blasphèmes ! Et cela parce que j’ai dit : Je suis le Fils de Dieu ?
\VS{37}Si je ne faisais pas les œuvres de mon Père, ne me croyez pas.
\VS{38}Mais si je les fais, même si vous ne me croyez pas, croyez à ces œuvres, afin que vous sachiez que le Père est en moi et que je suis dans le Père.
\VS{39}Là-dessus, ils cherchaient encore à le saisir ; mais il s’échappa de leurs mains.
\TextTitle{[Jésus se retire de Jérusalem]}
\VS{40}Il s'en alla de nouveau au-delà du Jourdain, à l'endroit où Jean avait baptisé au commencement, et il demeura là.
\VS{41}Beaucoup de gens vinrent à lui, et ils disaient : Jean n’a fait aucun miracle ; mais tout ce que Jean a dit de cet homme, était vrai.
\VS{42}Et dans ce lieu-là, plusieurs crurent en lui.
\TextTitle{[Jésus ressuscite Lazare de Béthanie]}
\Chap{11}
\VerseOne{}Il y avait un homme malade, Lazare, de Béthanie, village de Marie et de Marthe, sa sœur.
\VS{2}C’était cette Marie qui oignit de parfum le Seigneur, et qui essuya ses pieds avec ses cheveux ; et c’était son frère Lazare qui était malade.
\VS{3}Ses sœurs envoyèrent donc dire à Jésus : Seigneur, voici, celui que tu aimes est malade.
\VS{4}Après avoir entendu cela, Jésus dit : Cette maladie n'est point à la mort, mais elle est pour la gloire de Dieu, afin que le Fils de Dieu soit glorifié par elle.
\VS{5}Or Jésus aimait Marthe, sa sœur, et Lazare.
\VS{6}Après qu'il eut appris que Lazare était malade, il resta deux jours encore dans le lieu où il était,
\VS{7}et il dit à ses disciples : Retournons en Judée.
\VS{8}Les disciples lui dirent : Rabbi, les Juifs tout récemment cherchaient à te lapider, et tu retournes en Judée !
\VS{9}Jésus répondit : N'y a-t-il pas douze heures au jour ? Si quelqu'un marche pendant le jour, il ne bronche point ; car il voit la lumière de ce monde.
\VS{10}Mais si quelqu'un marche pendant la nuit, il bronche ; car il n'y a point de lumière avec lui.
\VS{11}Après ces paroles, il leur dit : Notre ami Lazare dort, mais je vais le réveiller.
\VS{12}Ses disciples lui dirent : Seigneur, s'il dort, il sera guéri.
\VS{13}Jésus avait parlé de sa mort, mais ils pensaient qu'il parlait de l’assoupissement.
\VS{14}Alors Jésus leur dit ouvertement : Lazare est mort.
\VS{15}Et je me réjouis, à cause de vous, de ce que je n’étais pas là, afin que vous croyiez. Mais allons vers lui.
\VS{16}Alors Thomas, appelé Didyme, dit aux autres disciples : Allons-y aussi, afin que nous mourions avec lui.
\VS{17}Jésus, étant arrivé, trouva que Lazare était déjà depuis quatre jours dans le sépulcre.
\VS{18}Et comme Béthanie était près de Jérusalem à quinze stades environ,
\VS{19}beaucoup de Juifs étaient venus vers Marthe et Marie pour les consoler au sujet de leur frère.
\VS{20}Lorsque Marthe apprit que Jésus arrivait, elle alla au-devant de lui ; mais Marie se tenait assise à la maison.
\VS{21}Marthe dit à Jésus : Seigneur, si tu avais été ici mon frère ne serait pas mort.
\VS{22}Mais maintenant je sais que tout ce que tu demanderas à Dieu, Dieu te le donnera.
\VS{23}Jésus lui dit : Ton frère ressuscitera.
\VS{24}Marthe lui dit : Je sais qu'il ressuscitera à la résurrection, au dernier jour.
\VS{25}Jésus lui dit : Je suis la résurrection et la vie : Celui qui croit en moi vivra même s’il meurt.
\VS{26}Et quiconque vit et croit en moi ne mourra jamais ; crois-tu cela ?
\VS{27}Elle lui dit : Oui, Seigneur, je crois que tu es le Christ, le Fils de Dieu, qui devait venir dans le monde.
\VS{28}Ayant ainsi parlé, elle alla appeler secrètement Marie sa sœur, en lui disant : Le Maître est ici, et il t'appelle.
\VS{29}Aussitôt que Marie eut entendu, elle se leva rapidement, et alla vers lui.
\VS{30}Or Jésus n'était pas encore entré dans le village, mais il était au lieu où Marthe l'avait rencontré.
\VS{31}Alors les Juifs qui étaient avec Marie à la maison, et qui la consolaient, ayant vu qu'elle s'était levée si promptement, et qu'elle était sortie, la suivirent en disant : Elle va au sépulcre pour y pleurer.
\VS{32}Lorsque Marie fut arrivée où était Jésus, et qu’elle le vit, elle se jeta à ses pieds, en lui disant : Seigneur, si tu avais été ici, mon frère ne serait pas mort.
\VS{33}Jésus, la voyant pleurer, elle et les Juifs qui étaient venus avec elle, frémit en son esprit et fut tout ému.
\VS{34}Et il dit : Où l'avez-vous mis ? Ils lui répondirent : Seigneur, viens et vois.
\VS{35}Jésus pleura.
\VS{36}Sur quoi les Juifs dirent : Voyez comme il l'aimait.
\VS{37}Et quelques-uns d'entre eux disaient : Lui qui a ouvert les yeux de l'aveugle, ne pouvait-il pas faire aussi que cet homme ne meure point ?
\VS{38}Alors Jésus frémissant de nouveau en lui-même, se rendit au sépulcre. C'était une grotte, et il y avait une pierre placée devant.
\VS{39}Jésus dit : Ôtez la pierre. Mais Marthe, la sœur du mort, lui dit : Seigneur, il sent déjà, car il est là depuis quatre jours.
\VS{40}Jésus lui dit : Ne t'ai-je pas dit que si tu crois tu verras la gloire de Dieu ?
\VS{41}Ils ôtèrent donc la pierre de dessus le lieu où le mort était couché. Et Jésus levant ses yeux au ciel, dit : Père, je te rends grâces de ce que tu m'as exaucé.
\VS{42}Pour moi, je savais que tu m'exauces toujours ; mais j’ai parlé à cause de la foule qui m’entoure, afin qu’ils croient que c’est toi qui m'as envoyé.
\VS{43}Ayant dit ces choses, il cria à haute voix : Lazare sors dehors !
\VS{44}Alors le mort sortit, ayant les mains et les pieds liés de bandes ; et son visage était enveloppé d'un linge. Jésus leur dit : Déliez-le, et laissez-le aller.
\TextTitle{[Nombreuses conversions]
\\(Jn. 12:10-11)}
\TextTitle{[Conspiration des Pharisiens]}
\VS{45}Plusieurs des Juifs qui étaient venus vers Marie, et qui avaient vu ce que Jésus avait fait, crurent en lui.
\VS{46}Mais quelques-uns d'entre eux allèrent trouver les pharisiens et leur dirent les choses que Jésus avait faites.
\VS{47}Alors les principaux sacrificateurs et les pharisiens assemblèrent le sanhédrin, et ils dirent : Que ferons-nous ? Car cet homme fait beaucoup de miracles.
\VS{48}Si nous le laissons faire, tout le monde croira en lui, et les Romains viendront et ils détruiront et ce lieu et notre nation.
\VS{49}Alors l'un d'eux appelé Caïphe, qui était le souverain sacrificateur cette année-là, leur dit : Vous n’y comprenez rien.
\VS{50}Et vous ne réfléchissez pas qu'il est de notre intérêt qu'un homme meure pour le peuple, et que toute la nation ne périsse point.
\VS{51}Or il ne dit pas cela de lui-même, mais étant souverain sacrificateur de cette année-là, il prophétisa que Jésus devait mourir pour la nation.
\VS{52}Et non pas seulement pour la nation, mais aussi pour rassembler en un seul corps les enfants de Dieu dispersés.
\VS{53}Depuis ce jour, ils se concertèrent ensemble pour le faire mourir.
\VS{54}C'est pourquoi Jésus ne se montrait plus ouvertement parmi les Juifs, mais il se retira dans la contrée voisine du désert, dans une ville appelée Ephraïm, et il demeura là avec ses disciples.
\VS{55}La Pâque des Juifs était proche. Et beaucoup de gens du pays montèrent à Jérusalem avant Pâque, afin de se purifier.
\VS{56}Et ils cherchaient Jésus, et se disaient les uns les autres dans le temple : Que vous en semble ? Ne viendra-t-il pas à la Fête ?
\VS{57}Or, les principaux sacrificateurs et les pharisiens avaient donné l’ordre que si quelqu'un savait où il était, il le déclare, afin qu’on se saisisse de lui.
\TextTitle{[Marie de Béthanie oint les pieds de Jésus]
\\(Mt. 26:6-13 ; Mc. 14:3-9)}
\Chap{12}
\VerseOne{}Six jours avant la Pâque, Jésus arriva à Béthanie, où était Lazare qui avait été mort, et qu'il avait ressuscité des morts.
\VS{2}Là, on lui fit un souper ; Marthe servait, et Lazare était un de ceux qui étaient à table avec lui.
\VS{3}Alors Marie ayant pris une livre de nard pur de grand prix, oignit les pieds de Jésus, et les essuya avec ses cheveux ; et la maison fut remplie de l'odeur du parfum.
\VS{4}Alors Judas Iscariot, fils de Simon, l'un de ses disciples, celui qui devait le trahir, dit :
\VS{5}Pourquoi ce parfum n'a-t-il pas été vendu trois cents deniers, pour donner cet argent aux pauvres ?
\VS{6}Il dit cela, non parce qu’il se mettait en peine des pauvres, mais parce qu'il était voleur, et que tenant la bourse, il prenait ce qu’on y mettait.
\VS{7}Mais Jésus lui dit : Laisse-la faire ; elle l'a gardé pour le jour de ma sépulture.
\VS{8}Car vous aurez toujours des pauvres avec vous ; mais vous ne m'aurez pas toujours.
\VS{9}Une grande multitude des Juifs apprirent que Jésus était à Béthanie, et ils y vinrent, non seulement à cause de lui, mais aussi pour voir Lazare qu'il avait ressuscité des morts.
\VS{10}Sur quoi les principaux sacrificateurs résolurent de faire mourir aussi Lazare.
\VS{11}Car plusieurs des Juifs se retiraient d'avec eux à cause de lui, et croyaient en Jésus.
\TextTitle{[Entrée triomphante de Jésus à Jérusalem]
\\(Mt. 21:1-11 ; Mc. 11:1-11 ; Lu. 19:28-40 ; Za. 9:9 ; Ap. 19:11-16}
\VS{12}Le lendemain, une grande quantité de foules qui étaient venues à la fête, ayant entendu dire que Jésus se rendait à Jérusalem,
\VS{13}prit des branches de palmes, et sortit au-devant de lui en criant : Hosanna ! Béni soit le Roi d'Israël qui vient au Nom du Seigneur !
\VS{14}Jésus trouva un ânon, s'assit dessus, selon ce qui est écrit : 15 Ne crains point, fille de Sion ; voici, ton Roi vient, assis sur le petit d'une ânesse\FTNT{Za. 9:9.}.
\VS{16}Ses disciples ne comprirent pas d'abord ces choses ; mais quand Jésus eut été glorifié, ils se souvinrent alors qu’elles étaient écrites de lui, et qu’elles avaient été accomplies à son égard.
\VS{17}Tous ceux qui avaient été avec Jésus, quand il appela Lazare du sépulcre et le ressuscita des morts, lui rendaient témoignage ;
\VS{18}et la foule alla au-devant de lui, parce qu’elle avait appris qu'il avait fait ce miracle.
\VS{19}Sur quoi les pharisiens dirent entre eux : Vous ne voyez pas que vous ne gagnez rien ? Voici, le monde va après lui.
\TextTitle{[Quelques Grecs cherchent à voir Jésus]}
\VS{20}Quelques Grecs du nombre de ceux qui étaient montés pour adorer Dieu pendant la fête,
\VS{21}s’adressèrent à Philippe, qui était de Bethsaïda de Galilée, et lui dirent avec instances : Seigneur ! Nous voudrions voir Jésus.
\VS{22}Philippe alla le dire à André, et André et Philippe le dirent à Jésus.
\TextTitle{[Jésus annonce sa crucifixion]}
\VS{23}Jésus leur répondit, disant : L'heure est venue où le Fils de l'homme doit être glorifié.
\VS{24}En vérité, en vérité je vous le dis : Si le grain de blé qui est tombé en la terre ne meurt, il reste seul ; mais s'il meurt, il porte beaucoup de fruits.
\VS{25}Celui qui aime sa vie la perdra ; et celui qui hait sa vie dans ce monde, la conservera pour la vie éternelle.
\VS{26}Si quelqu'un me sert, qu'il me suive ; et là où je serai, là aussi sera celui qui me sert ; et si quelqu'un me sert, mon Père l'honorera.
\VS{27}Maintenant mon âme est troublée. Et que dirai-je ? Ô Père, délivre-moi de cette heure ? Mais c'est pour cela que je suis venu jusqu’à cette heure.
\VS{28}Père glorifie ton Nom ! Alors une voix vint du ciel et dit : Je l'ai glorifié, et je le glorifierai encore.
\VS{29}Et la foule qui était là, et qui avait entendu cette voix, disait que c'était un coup de tonnerre ; les autres disaient : Un ange lui a parlé.
\VS{30}Jésus prit la parole et dit : Ce n’est pas à cause de moi que cette voix s’est fait entendre ; c’est à cause de vous.
\VS{31}Maintenant est venu le jugement de ce monde ; maintenant le prince de ce monde sera jeté dehors.
\VS{32}Et moi, quand je serai élevé de la terre, j’attirerai tous les hommes à moi.
\VS{33}En parlant ainsi, il indiquait de quelle mort il devait mourir.
\VS{34}La foule lui répondit : Nous avons appris par la loi que le Christ demeure éternellement, comment donc dis-tu qu'il faut que le Fils de l'homme soit élevé ? Qui est ce Fils de l'homme ?
\VS{35}Alors Jésus leur dit : La Lumière est encore avec vous pour un peu de temps : Marchez pendant que vous avez la Lumière, de peur que les ténèbres ne vous surprennent ; car celui qui marche dans les ténèbres ne sait pas où il va.
\VS{36}Pendant que vous avez la Lumière, croyez en la Lumière, afin que vous soyez enfants de lumière. Jésus dit ces choses, puis il s'en alla, et se cacha de devant eux.
\VS{37}Malgré tant de miracles qu’il avait faits en leur présence, ils ne croyaient point en lui,
\VS{38}afin que s’accomplisse cette parole qui a été dite par Esaïe le prophète : Seigneur, qui a cru à notre parole, et à qui a été révélé le bras du Seigneur\FTNT{Es. 53:1.} ?
\VS{39}C'est pourquoi ils ne pouvaient pas croire, parce qu'Esaïe a dit encore :
\VS{40}Il a aveuglé leurs yeux, et il a endurci leur cœur, de peur qu'ils ne voient de leurs yeux, qu'ils ne comprennent du cœur, qu'ils ne se convertissent, et que je ne les guérisse\FTNT{Es. 6:9-10.}.
\VS{41}Esaïe dit ces choses quand il vit sa gloire, et qu'il parla de lui.
\VS{42}Cependant, même parmi les chefs, plusieurs crurent en lui ; mais ils ne le confessaient pas à cause des pharisiens, de peur d'être exclus de la synagogue.
\VS{43}Car ils aimèrent la gloire des hommes, plus que la gloire de Dieu.
\VS{44}Or Jésus s'écria et dit : Celui qui croit en moi, ne croit pas seulement en moi, mais en celui qui m'a envoyé.
\VS{45}Et celui qui me voit, voit celui qui m'a envoyé.
\VS{46}Je suis venu dans le monde pour en être la Lumière, afin que quiconque croit en moi ne demeure point dans les ténèbres.
\VS{47}Si quelqu'un entend mes paroles, et ne les garde point, ce n’est pas moi qui le juge ; car je ne suis point venu pour juger le monde, mais pour sauver le monde.
\VS{48}Celui qui me rejette et qui ne reçoit pas mes paroles, a son juge : La parole que j'ai annoncée sera celle qui le jugera au dernier jour.
\VS{49}Car je n'ai point parlé de moi-même, mais le Père qui m'a envoyé, m'a prescrit ce que je dois dire et annoncer.
\VS{50}Et je sais que son commandement est la vie éternelle ; les choses donc que je dis, je les dis comme mon Père me les a dites.
\TextTitle{[L'entretien de Jn. 13-14 eut lieu dans la chambre haute ; Mc. 14:14-16]}
\Chap{13}
\VerseOne{}Avant la fête de Pâque, Jésus sachant que son heure était venue de passer de ce monde au Père, et ayant aimé les siens, qui étaient dans le monde, il les aima jusqu'à la fin.
\TextTitle{[La dernière Pâque ; Jésus lave les pieds de ses disciples]
\\(Mt. 26:20-24 ; Mc. 14:17 ; Lu. 22:14,21-23)}
\VS{2}Pendant le souper, alors que le diable avait déjà mis dans le cœur de Judas Iscariot, fils de Simon, de le trahir,
\VS{3}Jésus sachant que le Père avait remis toutes choses entre ses mains, qu'il était venu de Dieu, et qu’il s'en allait à Dieu,
\VS{4}se leva de table, ôta ses vêtements, et prit un linge, dont il se ceignit.
\VS{5}Puis il mit de l'eau dans un bassin, et se mit à laver les pieds de ses disciples, et à les essuyer avec le linge dont il se ceignit.
\VS{6}Alors il vint à Simon Pierre, mais Pierre lui dit : Toi, Seigneur, tu me laves les pieds ?
\VS{7}Jésus répondit et lui dit : Tu ne comprends pas maintenant ce que je fais, mais tu le sauras dans la suite.
\VS{8}Pierre lui dit : Tu ne me laveras jamais les pieds ! Jésus lui répondit : Si je ne te lave pas, tu n'auras point de part avec moi.
\VS{9}Simon Pierre lui dit : Seigneur, non seulement mes pieds, mais aussi les mains et la tête.
\VS{10}Jésus lui dit : Celui qui est baigné n’a besoin que de se laver les pieds pour être entièrement pur ; vous êtes purs, mais non pas tous.
\VS{11}Car il savait qui était celui qui le trahirait ; c'est pourquoi il dit : Vous n'êtes pas tous purs.
\VS{12}Après qu'il leur eut lavé les pieds, il reprit ses vêtements, et s'étant remis à table, il leur dit : Comprenez-vous ce que je vous ai fait ?
\VS{13}Vous m'appelez Maître et Seigneur ; et vous dites bien, car je le suis.
\VS{14}Si donc moi, qui suis le Seigneur et le Maître, j'ai lavé vos pieds, vous devez aussi vous laver les pieds les uns des autres.
\VS{15}Car je vous ai donné un exemple, afin que vous fassiez comme je vous ai fait.
\VS{16}En vérité, en vérité je vous le dis : Le serviteur n'est pas plus grand que son maître ni l’apôtre plus grand que celui qui l'a envoyé.
\VS{17}Si vous savez ces choses, vous êtes heureux, pourvu que vous les pratiquiez.
\VS{18}Je ne parle pas de vous tous, je connais ceux que j'ai choisis. Mais il faut que l’Ecriture s’accomplisse : Celui qui mange le pain avec moi, a levé son talon contre moi\FTNT{Ps. 41:10.}.
\VS{19}Je vous dis ceci dès maintenant, avant que la chose arrive, afin que lorsqu’elle arrivera, vous croyiez que c'est moi que le Père a envoyé.
\VS{20}En vérité, en vérité je vous le dis : Celui qui reçoit celui que j’aurai envoyé, me reçoit ; et celui qui me reçoit, reçoit celui qui m'a envoyé.
\TextTitle{[Jésus annonce la trahison de Judas]
\\(Mt. 26:21-25 ; Mc. 14:18-21 ; Lu. 22:21-23)}
\VS{21}Ayant ainsi parlé, Jésus fut ému dans son esprit, et il déclara : En vérité, en vérité je vous le dis, l'un de vous me trahira.
\VS{22}Alors les disciples se regardaient les uns les autres, ne sachant de qui il parlait.
\VS{23}Un des disciples, celui que Jésus aimait, était à table couché sur le sein de Jésus.
\VS{24}Simon Pierre lui fit signe de demander qui était celui dont Jésus parlait.
\VS{25}Et ce disciple, s’étant penché sur la poitrine de Jésus, lui dit : Seigneur, qui est-ce ?
\VS{26}Jésus répondit : C'est celui à qui je donnerai le morceau trempé ; et ayant trempé le morceau, il le donna à Judas Iscariot, fils de Simon.
\VS{27}Après que Judas eut pris le morceau, Satan entra en lui. Jésus lui dit : Ce que tu fais, fais-le promptement.
\VS{28}Mais aucun de ceux qui étaient à table ne comprit pourquoi il lui avait dit cela.
\VS{29}Car quelques-uns pensaient que, comme Judas avait la bourse, Jésus voulait lui dire : Achète ce qui nous est nécessaire pour la Fête ; ou qu'il lui commandait de donner quelque chose aux pauvres.
\VS{30}Judas, ayant pris le morceau, sortit aussitôt. Il faisait nuit.
\VS{31}Lorsque Judas fut sorti, Jésus dit : Maintenant le Fils de l'homme est glorifié ; et Dieu est glorifié en lui.
\VS{32}Si Dieu est glorifié en lui, Dieu aussi le glorifiera en lui-même, et il le glorifiera bientôt.
\VS{33}Mes petits enfants, je suis encore pour un peu de temps avec vous ; vous me chercherez, mais comme j'ai dit aux Juifs : Vous ne pouvez pas venir où je vais, je vous le dis aussi maintenant.
\VS{34}Je vous donne un nouveau commandement : Aimez-vous les uns les autres. Comme je vous ai aimés, vous aussi, aimez-vous les uns les autres.
\VS{35}A ceci tous connaîtront que vous êtes mes disciples, si vous avez de l'amour les uns pour les autres.
\TextTitle{[Jésus annonce le reniement de Pierre]
\\(Mt. 26:30-35 ; Mc. 14:26-31 ; Lu. 22:31-34)}
\VS{36}Simon Pierre lui dit : Seigneur ! Où vas-tu ? Jésus lui répondit : Là où je vais, tu ne peux pas me suivre maintenant, mais tu me suivras plus tard.
\VS{37}Pierre lui dit : Seigneur ! Pourquoi ne puis-je pas te suivre maintenant ? J’exposerai ma vie pour toi.
\VS{38}Jésus lui répondit : Tu exposeras ta vie pour moi ? En vérité, en vérité je te le dis, le coq ne chantera pas, que tu ne m'aies renié trois fois.
\TextTitle{[Jésus réconforte les apôtres : Il reviendra vers eux]}
\Chap{14}
\VerseOne{}Que votre cœur ne se trouble point ; vous croyez en Dieu, croyez aussi en moi.
\VS{2}Il y a plusieurs demeures dans la maison de mon Père. Si cela n’était pas, je vous l’aurais dit ; je vais vous préparer une place.
\VS{3}Et quand je m'en serai allé, et que je vous aurai préparé une place, je reviendrai, et je vous prendrai avec moi ; afin que là où je suis, vous y soyez aussi.
\VS{4}Et vous savez où je vais, et vous en savez le chemin.
\VS{5}Thomas lui dit : Seigneur ! Nous ne savons point où tu vas, comment donc pouvons-nous en savoir le chemin ?
\VS{6}Jésus lui dit : Je suis le chemin, la vérité, et la vie ; nul ne vient au Père que par moi.
\TextTitle{[Le Père et le Fils sont un]}
\VS{7}Si vous me connaissiez, vous connaîtriez aussi mon Père ; mais dès maintenant vous le connaissez, et vous l'avez vu.
\VS{8}Philippe lui dit : Seigneur ! Montre-nous le Père, et cela nous suffit.
\VS{9}Jésus lui répondit : Je suis depuis si longtemps avec vous, et tu ne m'as pas connu ? Philippe ! Celui qui m'a vu a vu mon Père. Et comment dis-tu : Montre-nous le Père ?
\VS{10}Ne crois-tu pas que je suis dans le Père, et que le Père est en moi ? Les paroles que je vous dis, je ne les dis pas de moi-même ; mais le Père qui demeure en moi est celui qui fait les œuvres.
\VS{11}Croyez-moi, je suis dans le Père, et le Père est en moi, sinon croyez-moi à cause de ces œuvres.
\VS{12}En vérité, en vérité je vous le dis : Celui qui croit en moi fera les œuvres que je fais, et il en fera même de plus grandes que celles-ci, parce que je m'en vais vers mon Père.
\TextTitle{[Nouveau privilège par la prière]}
\VS{13}Et tout ce que vous demanderez en mon Nom, je le ferai ; afin que le Père soit glorifié dans le Fils.
\VS{14}Si vous demandez en mon Nom quelque chose, je le ferai.
\TextTitle{[Promesse quant à l'habitation de l'Esprit dans le coeur du croyant]}
\VS{15}Si vous m'aimez, gardez mes commandements.
\VS{16}Et moi, je prierai le Père, et il vous donnera un autre Consolateur, pour demeurer avec vous éternellement,
\VS{17}l'Esprit de vérité que le monde ne peut recevoir, parce qu'il ne le voit point, et qu'il ne le connaît point ; mais vous le connaissez, car il demeure avec vous, et il sera en vous.
\VS{18}Je ne vous laisserai pas orphelins, je viendrai vers vous.
\VS{19}Encore un peu de temps, et le monde ne me verra plus ; mais vous me verrez, parce que je vis, et vous aussi vous vivrez.
\VS{20}En ce jour-là vous connaîtrez que je suis en mon Père, que vous êtes en moi, et moi en vous.
\VS{21}Celui qui a mes commandements et qui les garde, c'est celui qui m'aime ; et celui qui m'aime sera aimé de mon Père ; je l'aimerai, et je me ferai connaître à lui.
\VS{22}Jude, non pas Iscariot, lui dit : Seigneur ! D’où vient que tu te feras connaître à nous, et non au monde ?
\VS{23}Jésus répondit et lui dit : Si quelqu'un m'aime, il gardera ma parole, et mon Père l'aimera, et nous viendrons à lui, et nous ferons notre demeure chez lui.
\VS{24}Celui qui ne m'aime point ne garde point mes paroles. Et la parole que vous entendez n'est point ma parole, mais c'est celle du Père qui m'a envoyé.
\VS{25}Je vous ai dit ces choses pendant que je demeure avec vous.
\VS{26}Mais le Consolateur, le Saint-Esprit, que le Père enverra en mon Nom, vous enseignera toutes choses, et il vous rappellera tout ce que je vous ai dit.
\TextTitle{[Christ donne sa paix]}
\VS{27}Je vous laisse la paix, je vous donne ma paix ; je ne vous la donne pas comme le monde la donne ; que votre cœur ne se trouble point, et ne s’alarme point.
\VS{28}Vous avez entendu que je vous ai dit : Je m'en vais, et je reviens à vous ; si vous m'aimiez, vous seriez certes joyeux de ce que j'ai dit : Je m'en vais au Père, car le Père est plus grand que moi.
\VS{29}Et maintenant je vous l'ai dit avant que cela soit arrivé, afin que quand il sera arrivé, vous croyiez.
\VS{30}Je ne parlerai plus guère avec vous ; car le prince de ce monde vient ; mais il n'a rien en moi.
\VS{31}Mais afin que le monde sache que j'aime le Père, et que je fais ce que le Père m'a commandé : Levez-vous, partons d'ici.
\TextTitle{[Le cep et les sarments]}
\Chap{15}
\VerseOne{}Je suis le vrai Cep\FTNT{Jésus est l’arbre de vie qui produit de bons fruits en nous, à condition que nous nous tenions loin de l’arbre de la connaissance du bien et du mal. Jésus, le vrai Cep, est la source de vie. La viabilité du sarment dépend de son attachement au Cep. Jésus a été pendu au bois (Ac 5:30), s’est chargé de nos malédictions (Ga. 3:13) et a été retranché à notre place.}, et mon Père est le Vigneron.
\VS{2}Il retranche tout sarment qui est en moi et qui ne porte pas de fruits ; et tout sarment qui porte du fruit, il l'émonde afin qu'il porte encore plus de fruits.
\VS{3}Vous êtes déjà purs à cause de la parole que je vous ai enseignée.
\VS{4}Demeurez en moi, et je demeurerai en vous ; comme le sarment ne peut de lui-même porter du fruit s'il ne demeure pas attaché au cep ; ainsi vous ne le pouvez pas non plus si vous ne demeurez pas en moi.
\VS{5}Je suis le Cep, et vous en êtes les sarments ; celui qui demeure en moi, et en qui je demeure porte beaucoup de fruits ; car hors de moi, vous ne pouvez rien produire.
\VS{6}Si quelqu'un ne demeure point en moi, il est jeté dehors comme le sarment, et il se sèche ; puis on l'amasse, on le met au feu, et il brûle.
\VS{7}Si vous demeurez en moi, et que mes paroles demeurent en vous, demandez tout ce que vous voudrez, et cela vous sera fait.
\VS{8}Si vous portez beaucoup de fruits, mon Père sera glorifié et vous serez alors mes disciples.
\VS{9}Comme le Père m'a aimé, ainsi je vous ai aimés, demeurez dans mon amour.
\VS{10}Si vous gardez mes commandements, vous demeurerez dans mon amour ; comme j'ai gardé les commandements de mon Père, et je demeure dans son amour.
\VS{11}Je vous ai dit ces choses afin que ma joie demeure en vous, et que votre joie soit parfaite.
\VS{12}C'est ici mon commandement : Aimez-vous les uns les autres comme je vous ai aimés.
\VS{13}Il n’y a pas de plus grand amour que de donner sa vie pour ses amis.
\VS{14}Vous serez mes amis, si vous faites tout ce que je vous commande.
\TextTitle{[Nouvelle intimité entre le Seigneur et les siens]}
\VS{15}Je ne vous appelle plus serviteurs, car le serviteur ne sait pas ce que fait son maître; mais je vous ai appelés mes amis, parce que je vous ai fait connaître tout ce que j'ai appris de mon Père.
\VS{16}Ce n'est pas vous qui m'avez choisi ; mais moi, je vous ai choisis, et je vous ai établis afin que vous alliez partout et que vous produisiez du fruit, et que votre fruit demeure ; afin que tout ce que vous demanderez au Père en mon Nom, il vous le donne.
\VS{17}Ce que je vous commande, c’est de vous aimer les uns les autres.
\TextTitle{[L'attitude du monde à l'égard des croyants en Christ]}
\VS{18}Si le monde vous hait, sachez qu’il m’a haï avant vous.
\VS{19}Si vous étiez du monde, le monde aimerait ce qui est à lui ; mais parce que vous n'êtes pas du monde, et que je vous ai choisis du milieu du monde, à cause de cela le monde vous hait.
\VS{20}Souvenez-vous de la parole que je vous ai dite : Le serviteur n'est pas plus grand que son maître ; s'ils m'ont persécuté, ils vous persécuteront aussi ; s'ils ont gardé ma parole, ils garderont aussi la vôtre.
\VS{21}Mais ils vous feront toutes ces choses à cause de mon Nom, parce qu'ils ne connaissent point celui qui m'a envoyé.
\VS{22}Si je n’étais pas venu, et que je ne leur avais point parlé, ils n'auraient point de péché, mais maintenant ils n'ont point d'excuse de leur péché.
\VS{23}Celui qui me hait, hait aussi mon Père.
\VS{24}Si je n’avais pas fait parmi eux les œuvres qu'aucun autre n'a faites, ils n'auraient point de péché ; mais maintenant ils les ont vues, et ils ont haï et moi et mon Père.
\VS{25}Mais cela est arrivé afin que s’accomplisse la parole qui est écrite dans leur loi : Ils m'ont haï sans cause\FTNT{Ps. 35:19 ; Ps. 69:5.}.
\VS{26}Mais quand le Consolateur sera venu, que je vous enverrai de la part de mon Père, l'Esprit de vérité qui procède de mon Père, il rendra témoignage de moi.
\VS{27}Et vous aussi, vous rendrez témoignage, car vous êtes dès le commencement avec moi.
\TextTitle{[Jésus avertit les siens de la persécution]
\\(Mt. 24:9-10 ; Lu. 21:16-19)}
\Chap{16}
\VerseOne{}Je vous ai dit ces choses, afin que vous ne soyez pas scandalisés.
\VS{2}Ils vous chasseront des synagogues ; et même l’heure vient où quiconque vous fera mourir, croira rendre un culte à Dieu.
\VS{3}Et ils vous feront ces choses, parce qu'ils n’ont connu ni le Père ni moi.
\VS{4}Je vous ai dit ces choses, afin que, lorsque l'heure sera venue, vous vous souveniez que je vous les ai dites ; je ne vous en ai pas parlé dès le commencement, parce que j'étais avec vous.
\VS{5}Mais maintenant je m'en vais vers celui qui m'a envoyé, et aucun de vous ne me demande : Où vas-tu ?
\VS{6}Mais parce que je vous ai dit ces choses, la tristesse a rempli votre cœur.
\TextTitle{[La triple activité de l'Esprit agit en faveur du monde]}
\VS{7}Toutefois je vous dis la vérité, il vous est avantageux que je m'en aille, car si je ne m'en vais pas, le Consolateur ne viendra pas vers vous ; mais si je m'en vais, je vous l'enverrai.
\VS{8}Et quand il sera venu, il convaincra le monde de péché, de justice, et de jugement :
\VS{9}du péché, parce qu'ils ne croient point en moi,
\VS{10}de justice, parce que je m'en vais à mon Père, et que vous ne me verrez plus ;
\VS{11}de jugement, parce que le prince de ce monde est déjà jugé.
\TextTitle{[Après son ascension, Christ continuera de révéler la verité par l'Esprit]}
\VS{12}J'ai encore beaucoup de choses à vous dire, mais vous ne pouvez pas les porter maintenant.
\VS{13}Mais quand le Consolateur sera venu, l'Esprit de vérité, il vous conduira dans toute la vérité ; car il ne parlera pas de lui-même, mais il dira tout ce qu'il aura entendu, et il vous annoncera les choses à venir.
\VS{14}Il me glorifiera, car il prendra ce qui est à moi, et vous l'annoncera.
\VS{15}Tout ce que mon Père a, est à moi ; c'est pourquoi j'ai dit qu'il prendra ce qui est à moi et qu’il vous l'annoncera.
\TextTitle{[Jésus parle de sa mort, de sa grandeur]}
\VS{16}Encore un peu de temps, et vous ne me verrez plus ; et après un peu de temps, vous me verrez, car je m'en vais à mon Père.
\VS{17}Quelques-uns de ses disciples dirent entre eux : Qu'est-ce qu'il nous dit : Encore un peu de temps, et vous ne me verrez plus ; et un peu de temps après, vous me verrez, car je m'en vais à mon Père ?
\VS{18}Ils disaient donc : Que signifient ces mots : Encore un peu de temps ? Nous ne comprenons pas ce qu'il dit.
\VS{19}Jésus sachant qu'ils voulaient l’interroger, leur dit : Vous vous demandez entre vous sur ce que j'ai dit : Encore un peu de temps, et vous ne me verrez plus, et un peu de temps après, vous me verrez.
\VS{20}En vérité, en vérité je vous le dis : Vous pleurerez et vous vous lamenterez, et le monde se réjouira ; vous serez, dis-je, attristés ; mais votre tristesse sera changée en joie.
\VS{21}La femme, lorsqu’elle enfante, éprouve de la tristesse, parce que son heure est venue ; mais, lorsqu’elle a donné le jour à l’enfant, elle ne se souvient plus de la souffrance, à cause de ce qu’un homme est né dans le monde.
\VS{22}Vous donc aussi, vous êtes maintenant dans la tristesse ; mais je vous reverrai encore, et votre cœur se réjouira, et personne ne vous ôtera votre joie.
\VS{23}En ce jour-là, vous ne m'interrogerez plus sur rien. En vérité, en vérité je vous le dis : Tout ce que vous demanderez au Père en mon Nom, il vous le donnera.
\VS{24}Jusqu'à présent vous n'avez rien demandé en mon Nom ; demandez, et vous recevrez, afin que votre joie soit parfaite.
\VS{25}Je vous ai dit ces choses en paraboles. Mais l'heure vient où je ne vous parlerai plus en paraboles ; mais je vous parlerai ouvertement de mon Père.
\VS{26}En ce jour-là, vous demanderez des grâces en mon Nom, et je ne vous dis pas que je prierai le Père pour vous ;
\VS{27}car le Père lui-même vous aime, parce que vous m'avez aimé, et que vous avez cru que je suis sorti de Dieu.
\VS{28}Je suis sorti du Père, et je suis venu dans le monde ; maintenant je quitte le monde, et je m'en vais au Père.
\VS{29}Ses disciples lui dirent : Voici, maintenant tu parles ouvertement, et tu n'uses plus de paraboles.
\VS{30}Maintenant nous savons que tu sais toutes choses\FTNT{Jésus est omniscient. Il s’est lui-même présenté à l’apôtre Jean comme celui qui est, qui était et qui sera (Ap. 1:7-8).}, et que tu n'as pas besoin que quelqu’un t'interroge ; à cause de cela nous croyons que tu es sorti de Dieu.
\VS{31}Jésus leur répondit : Croyez-vous maintenant ?
\VS{32}Voici, l'heure vient, et elle est déjà venue, où vous serez dispersés chacun de son côté, et vous me laisserez seul ; mais je ne suis pas seul, car le Père est avec moi.
\VS{33}Je vous ai dit ces choses afin que vous ayez la paix en moi. Vous aurez des tribulations dans le monde, mais prenez courage, j'ai vaincu le monde.
\TextTitle{[La prière d'intercession de Christ, le souverain sacrificateur]}
\Chap{17}
\VerseOne{}Après avoir ainsi parlé, Jésus leva ses yeux au ciel, et dit : Père, l'heure est venue, glorifie ton Fils, afin que ton Fils te glorifie ;
\VS{2}selon que tu lui as donné pouvoir sur tous les hommes ; afin qu'il donne la vie éternelle à tous ceux que tu lui as donnés.
\VS{3}Or, la vie éternelle, ce qu’ils te connaissent, toi, le seul vrai Dieu, et celui que tu as envoyé, Jésus-Christ.
\VS{4}Je t'ai glorifié sur la terre, j'ai achevé l’œuvre que tu m'avais donnée à faire.
\VS{5}Et maintenant glorifie-moi, toi Père, auprès de toi, de la gloire que j’avais auprès de toi avant que le monde soit.
\VS{6}J'ai fait connaître ton Nom aux hommes que tu m'as donnés du milieu du monde ; ils étaient à toi, et tu me les as donnés ; et ils ont gardé ta parole.
\VS{7}Maintenant ils ont connu que tout ce que tu m'as donné vient de toi.
\VS{8}Car je leur ai donné les paroles que tu m'as données, et ils les ont reçues, et ils ont vraiment connu que je suis sorti de toi, et ils ont cru que tu m'as envoyé.
\VS{9}C’est pour eux que je prie ; je ne prie pas pour le monde, mais pour ceux que tu m'as donnés, parce qu'ils sont à toi.
\VS{10}Et tout ce qui est à moi est à toi, et ce qui est à toi est à moi ; et je suis glorifié en eux.
\VS{11}Et maintenant je ne suis plus dans le monde, et ils sont dans le monde ; et moi je vais à toi. Père saint, garde en ton Nom ceux que tu m'as donnés, afin qu'ils soient un comme nous sommes un.
\VS{12}Quand j'étais avec eux dans le monde, je les gardais en ton Nom ; j'ai gardé ceux que tu m'as donnés, et aucun d'eux ne s’est perdu, sinon le fils de perdition, afin que l'Ecriture soit accomplie.
\VS{13}Et maintenant je vais à toi, et je dis ces choses étant encore dans le monde, afin qu'ils aient ma joie parfaite en eux-mêmes.
\VS{14}Je leur ai donné ta parole, et le monde les a haïs, parce qu'ils ne sont pas du monde, comme moi je ne suis pas du monde.
\VS{15}Je ne te prie pas de les ôter du monde, mais de les préserver du mal.
\VS{16}Ils ne sont pas du monde, comme moi je ne suis pas du monde.
\VS{17}Sanctifie-les par ta vérité ; ta parole est la vérité.
\VS{18}Comme tu m'as envoyé dans le monde, ainsi je les ai envoyés dans le monde.
\VS{19}Et je me sanctifie moi-même pour eux, afin qu'eux aussi soient sanctifiés par la vérité.
\VS{20}Je ne prie pas seulement pour eux, mais aussi pour ceux qui croiront en moi par leur parole.
\VS{21}Afin que tous soient un, ainsi que toi, Père, tu es en moi, et moi en toi ; afin qu'eux aussi soient un en nous ; et que le monde croie que c'est toi qui m'as envoyé.
\VS{22}Je leur ai donné la gloire que tu m'as donnée, afin qu'ils soient un comme nous sommes un.
\VS{23}Je suis en eux, et toi en moi, afin qu'ils soient parfaitement un, et que le monde connaisse que c'est toi qui m'as envoyé, et que tu les aimes, comme tu m'as aimé.
\VS{24}Père, mon désir est que ceux que tu m'as donnés soient avec moi là où je suis, afin qu'ils contemplent la gloire que tu m'as donnée ; parce que tu m'as aimé avant la fondation du monde.
\VS{25}Père juste, le monde ne t'a point connu ; mais moi je t'ai connu, et ceux-ci ont connu que c'est toi qui m'as envoyé.
\VS{26}Et je leur ai fait connaître ton Nom, et je le leur ferai connaître, afin que l'amour dont tu m'as aimé soit en eux, et que je sois en eux.
\TextTitle{[Jésus à Gethsémané]
\\(Mt. 26:36-46 ; Mc. 14:32-42 ; Lu. 22:39-46)}
\Chap{18}
\VerseOne{}Après que Jésus eut dit ces choses, il s'en alla avec ses disciples au-delà du torrent de Cédron, où il y avait un jardin dans lequel il entra avec ses disciples.
\TextTitle{[Jésus trahi et arrêté]
\\Mt. 26:47-56 ; Mc. 14:43-50 ; Lu. 22:47-54)}
\VS{2}Or Judas, qui le trahissait, connaissait aussi ce lieu-là, car Jésus s'y était souvent assemblé avec ses disciples.
\VS{3}Judas donc, ayant pris la cohorte, et des huissiers qu’envoyèrent les principaux sacrificateurs et les pharisiens, s'en vint là avec des lanternes, des flambeaux, et des armes.
\VS{4}Jésus, sachant tout ce qui devait lui arriver, s'avança et leur dit : Qui cherchez-vous ?
\VS{5}Ils lui répondirent : Jésus de Nazareth. Jésus leur dit : Moi, Je suis\FTNT{~Moi, Je suis~ (~ego eimi~), ce qui fait écho au nom sous lequel Dieu s’était révélé à Moïse en Ex. 3:14.}. Et Judas qui le trahissait était aussi avec eux.
\VS{6}Or après que Jésus leur eut dit : Moi Je suis, ils reculèrent, et tombèrent par terre.
\VS{7}Il leur demanda une seconde fois : Qui cherchez-vous ? Et ils répondirent : Jésus de Nazareth.
\VS{8}Jésus répondit : Je vous ai dit que moi, Je suis ; si donc vous me cherchez, laissez aller ceux-ci.
\VS{9}Il dit cela afin que s’accomplisse la parole qu'il avait dite : Je n'ai perdu aucun de ceux que tu m'as donnés.
\TextTitle{[Malchus frappé par Pierre]}
\VS{10}Simon Pierre, qui avait une épée, la tira, frappa le serviteur du souverain sacrificateur, et lui coupa l'oreille droite. Ce serviteur s’appelait Malchus.
\VS{11}Mais Jésus dit à Pierre : Remets ton épée au fourreau : Ne boirai-je pas la coupe que le Père m'a donnée ?
\TextTitle{[Jésus conduit auprès du souverain sacrificateur]
\\(v. 27 ; Mt. 26:57-68 ; Mc. 14:53-65 ; Lu. 22:63-71)}
\VS{12}La cohorte, le tribun, et les huissiers des Juifs se saisirent alors de Jésus et le lièrent.
\VS{13}Et ils l'emmenèrent premièrement chez Anne, car il était le beau-père de Caïphe, qui était le souverain sacrificateur de cette année-là.
\VS{14}Et Caïphe était celui qui avait donné ce conseil aux Juifs, qu'il était avantageux qu'un seul homme meure pour le peuple.
\TextTitle{[Le triple reniement de Pierre]
\\(v. 25-27 ; Mt. 26:69-75 ; Mc. 14:66-72 ; Lu. 22:54-62)}
\VS{15}Simon Pierre, avec un autre disciple, suivait Jésus ; et ce disciple était connu du souverain sacrificateur, et il entra avec Jésus dans la cour du souverain sacrificateur.
\VS{16}Mais Pierre était dehors à la porte, et l'autre disciple, qui était connu du souverain sacrificateur, sortit dehors et parla à la portière, et il fit entrer Pierre.
\VS{17}Et la servante qui était la portière dit à Pierre : N'es-tu pas aussi des disciples de cet homme ? Il dit : Je n'en suis point.
\VS{18}Les serviteurs et les huissiers qui étaient là avaient allumé un feu, parce qu'il faisait froid, et ils se chauffaient ; Pierre aussi était avec eux, et se chauffait.
\VS{19}Et le souverain sacrificateur interrogea Jésus sur ses disciples et sur sa doctrine.
\VS{20}Jésus lui répondit : J’ai ouvertement parlé au monde ; j'ai toujours enseigné dans la synagogue et dans le temple, où les Juifs s'assemblent toujours, et je n'ai rien dit en secret.
\VS{21}Pourquoi m'interroges-tu ? Interroge ceux qui ont entendu ce que je leur ai dit ; voici, ils savent ce que j'ai dit.
\VS{22}Quand il eut dit ces choses, un des huissiers qui se tenait là, donna un coup de sa verge à Jésus en lui disant : Est-ce ainsi que tu réponds au souverain sacrificateur ?
\VS{23}Jésus lui répondit : Si j'ai mal parlé, explique-moi ce que j’ai dit de mal ; et si j'ai bien parlé, pourquoi me frappes-tu ?
\VS{24}Anne l’envoya lié à Caïphe, le souverain sacrificateur.
\VS{25}Simon Pierre était là, et se chauffait. On lui dit : N’es-tu pas aussi de ses disciples ? Il le nia et dit : Je n'en suis point.
\VS{26}Un des serviteurs du souverain sacrificateur, parent de celui à qui Pierre avait coupé l'oreille, dit : Ne t'ai-je pas vu dans le jardin avec lui ?
\VS{27}Mais Pierre le nia de nouveau, et aussitôt le coq chanta.
\TextTitle{[Jésus devant Pilate]
\\(Mt. 27:2,11-14 ; Mc. 15:1-5 ; Lu. 23:1-7,13-15)}
\VS{28}Ils conduisirent Jésus de chez Caïphe au Prétoire\FTNT{Le Prétoire était à l'origine le nom du quartier général de la légion romaine. Il s’agissait plus particulièrement de la tente du général en chef d'une armée.} ; c'était le matin. Mais ils n'entrèrent point eux-mêmes dans le Prétoire, afin de ne pas se souiller, et de pouvoir manger l'agneau de Pâque.
\VS{29}C'est pourquoi Pilate\FTNT{Ponce Pilate était le préfet procurateur de la province romaine de Judée au Ier siècle (de 26 à 36).} sortit vers eux, et leur dit : Quelle accusation portez-vous contre cet homme ?
\VS{30}Ils lui répondirent : Si ce n'était pas un malfaiteur, nous ne te l’aurions pas livré.
\VS{31}Alors Pilate leur dit : Prenez-le vous-mêmes, et jugez-le selon votre loi. Mais les Juifs lui dirent : Il ne nous est pas permis de mettre quelqu’un à mort.
\VS{32}C’était afin que s’accomplisse la parole que Jésus avait dite, lorsqu’il indiquait de quelle mort il devait mourir.
\VS{33}Pilate entra de nouveau dans le Prétoire, et ayant appelé Jésus, il lui dit : Es-tu le Roi des Juifs ?
\VS{34}Jésus lui répondit : Est-ce de toi-même que tu dis cela, ou d’autres te l’ont dit de moi ?
\VS{35}Pilate répondit : Suis-je Juif ? Ta nation et les principaux sacrificateurs t'ont livré à moi ; qu'as-tu fait ?
\VS{36}Jésus répondit : Mon Royaume n'est pas de ce monde ; si mon Royaume était de ce monde, mes serviteurs auraient combattu pour moi afin que je ne sois pas livré aux Juifs ; mais maintenant mon règne n'est point d'ici-bas.
\VS{37}Alors Pilate lui dit : Es-tu donc Roi ? Jésus répondit : Tu le dis, que je suis Roi ; je suis né pour cela, et c'est pour cela que je suis venu dans le monde, pour rendre témoignage à la vérité. Quiconque est de la vérité entend ma voix.
\VS{38}Pilate lui dit : Qu'est-ce que la vérité ? Et quand il eut dit cela, il sortit de nouveau vers les Juifs, et il leur dit : Je ne trouve aucun crime en lui.
\TextTitle{[Barabbas libéré et Jésus condamné]
\\(Mt. 27:15-21 ; Mc. 15:6-11 ; Lu. 23:18-19)}
\VS{39}Or, comme c’est parmi vous une coutume que je vous relâche un prisonnier à la fête de Pâque ; voulez-vous donc que je vous relâche le Roi des Juifs ?
\VS{40}Et tous s'écrièrent, disant : Non pas celui-ci, mais Barrabas ; or Barrabas était un brigand.
\TextTitle{[Jésus couronné d'épines]
\\(Mt. 27:-30 ; Mc. 15:16-18)}
\Chap{19}
\VerseOne{}Alors Pilate prit Jésus, et le fit battre de verges.
\VS{2}Les soldats tressèrent une couronne d'épines qu'ils posèrent sur sa tête, et le vêtirent d'un vêtement de pourpre.
\VS{3}Puis ils lui disaient : Roi des Juifs, nous te saluons ; et ils lui donnaient des coups avec leurs verges.
\TextTitle{[Pilate fait un ultime effort pour relâcher Jésus]
\\(Mt. 27:22-26 ; Mc. 15:12-15 ; Lu. 23:20-25)}
\VS{4}Pilate sortit de nouveau dehors, et leur dit : Voici, je vous l'amène dehors, afin que vous sachiez que je ne trouve aucun crime en lui.
\VS{5}Jésus donc sortit portant la couronne d'épines et le manteau de pourpre ; et Pilate leur dit : Voici l'homme.
\VS{6}Mais quand les principaux sacrificateurs et leurs huissiers le virent, ils s'écrièrent, en disant : Crucifie ! Crucifie ! Pilate leur dit : Prenez-le vous-mêmes et crucifiez-le, car je ne trouve point de crime en lui.
\VS{7}Les Juifs lui répondirent : Nous avons une loi, et selon notre loi il doit mourir, car il s'est fait Fils de Dieu.
\VS{8}Quand Pilate entendit cette parole, sa frayeur augmenta.
\VS{9}Et il rentra dans le Prétoire et dit à Jésus : D'où es-tu ? Mais Jésus ne lui donna point de réponse.
\VS{10}Et Pilate lui dit : Est-ce à moi que tu ne parles pas ? Ne sais-tu pas que j'ai le pouvoir de te crucifier, et que j’ai le pouvoir de te délivrer ?
\VS{11}Jésus lui répondit : Tu n'aurais aucun pouvoir sur moi s'il ne t‘avait été donné d'en haut ; c'est pourquoi celui qui m'a livré à toi, commet un plus grand péché.
\VS{12}Dès ce moment, Pilate cherchait à le délivrer ; mais les Juifs criaient en disant : Si tu le délivres, tu n'es pas ami de César ; car quiconque se fait Roi est contre César.
\VS{13}Pilate, ayant entendu ces paroles, amena Jésus dehors, et il siégea au tribunal, au lieu appelé le Pavé, et en hébreu Gabbatha.
\VS{14}C’était la préparation de la Pâque, et environ la sixième heure ; et Pilate dit aux Juifs : Voici votre Roi.
\VS{15}Mais ils criaient : Ôte, ôte, crucifie-le ! Pilate leur dit : Crucifierai-je votre Roi ? Les principaux sacrificateurs répondirent : Nous n'avons pas d'autre roi que César.
\TextTitle{[Jésus crucifié]
\\(Mt. 27:31-50 ; Mc. 15:19-37 ; Lu. 23:26-46)}
\VS{16}Alors il le leur livra pour être crucifié. Ils prirent donc Jésus et l'emmenèrent.
\VS{17}Jésus, portant sa croix, arriva au lieu appelé le Crâne, et en hébreu Golgotha,
\VS{18}où ils le crucifièrent, et deux autres avec lui, un de chaque côté, et Jésus au milieu.
\VS{19}Pilate fit un écriteau, qu'il mit sur la croix, où étaient écrits ces mots : Jésus de Nazareth, le Roi des juifs.
\VS{20}Beaucoup des Juifs lurent cet écriteau, parce que le lieu où Jésus était crucifié, était près de la ville ; et cet écriteau était en hébreu, en grec et en latin.
\VS{21}C'est pourquoi les principaux sacrificateurs des Juifs dirent à Pilate : N'écris pas le Roi des Juifs, mais que celui-ci a dit : Je suis le Roi des Juifs.
\VS{22}Pilate répondit : Ce que j'ai écrit, je l'ai écrit.
\VS{23}Les soldats, après avoir crucifié Jésus, prirent ses vêtements, et ils en firent quatre parts, une part pour chaque soldat. Ils prirent aussi sa tunique, qui était sans couture, d’un seul tissu depuis le haut jusqu'en bas.
\VS{24}Ils se dirent entre eux : Ne la déchirons pas, mais tirons au sort, pour savoir à qui elle sera. Et cela arriva ainsi, afin que s’accomplisse cette parole de l’Ecriture : Ils ont partagé entre eux mes vêtements, et ils ont tiré au sort ma tunique\FTNT{Ps. 22:19.} ; ainsi firent les soldats.
\VS{25}Près de la croix de Jésus se tenaient sa mère, et la sœur de sa mère, Marie femme de Cléopas, et Marie de Magdala.
\VS{26}Jésus voyant sa mère, et auprès d'elle le disciple qu'il aimait, il dit à sa mère : Femme, voilà ton Fils.
\VS{27}Puis il dit au disciple : Voilà ta mère ; et dès ce moment, ce disciple la prit chez lui.
\VS{28}Après cela, Jésus sachant que toutes choses étaient déjà accomplies, il dit, afin que l'Ecriture soit accomplie : J'ai soif.
\VS{29}Et il y avait là un vase plein de vinaigre. Les soldats en remplirent une éponge et la mirent au bout d'une branche d'hysope, et la lui présentèrent à la bouche.
\VS{30}Quand Jésus eut pris le vinaigre, il dit : Tout est accompli\FTNT{La fin de la période de la première alliance n’a pas eu lieu à la naissance du Seigneur. En effet, Ga. 4:4 nous dit que Jésus est né sous la loi de Moïse et le récit des quatre évangiles atteste que depuis sa naissance jusqu’à sa mort, Jésus a scrupuleusement respecté et accompli toute la loi. En effet, il a lui-même dit : ~Ne croyez pas que je sois venu abolir la loi ou les prophètes ; je ne suis pas venu les abolir, mais les accomplir.~ (Mt. 5:17). Ainsi, durant son ministère terrestre, le Seigneur demandait à ce qu’on applique la loi (Mt. 8:4 ; Mt. 23:23 ; Lu. 17:11-14) tout en préparant ses disciples à la nouvelle alliance. L’évangile de Matthieu nous relate un événement capital qui a eu lieu juste après la mort du Seigneur : ~Alors Jésus, poussa de nouveau un grand cri, et rendit l'esprit. Et voici, le voile du temple se déchira en deux, depuis le haut jusqu'en bas ; et la terre trembla, et les pierres se fendirent.~ (Mt. 27:50-51). Il convient de rappeler que le temple était divisé en trois parties : le Parvis, le lieu Saint et le Saint des saints. Le Parvis était accessible à tout le monde, y compris aux non-Juifs. Le lieu Saint n’était accessible qu’aux lévites. La troisième partie, le Saint des saints, n’était accessible qu’au souverain sacrificateur. Le lieu Saint était séparé du Saint des saints par un voile qui symbolisait le mur d’inimitié (Es. 59:2 ; Ro. 3:23) qui sépare l’homme pécheur de la présence de Dieu, représentée dans le temple par l’arche de l’alliance. Ce voile n’avait rien d’un tissu léger et vaporeux, mais il ressemblait davantage à un épais tapis, opaque et surtout très résistant, et donc très difficile à déchirer. Le souverain sacrificateur rentrait seulement une fois par an dans le Saint des saints pour y offrir le sacrifice d’expiation pour le peuple ainsi que pour lui-même (Lé. 16 ; Hé. 9:7). Toutefois, la nécessité de répéter ce sacrifice chaque année prouvait que les exigences de la justice divine n’étaient pas pleinement satisfaites (Hé. 10:3-4). L’auteur de l’épître aux Hébreux nous apprend que le voile symbolisait également le corps physique de Christ (Hé. 10:19-20). Ainsi, lorsque le Seigneur a succombé à ses meurtrissures, le fameux voile s’est déchiré du haut jusqu’au bas. Or tant que le voile subsistait, l’accès à la présence de Dieu était fermé (Hé. 9:8). La déchirure atteste donc qu’en Christ, nous pouvons désormais nous approcher avec assurance du trône de Dieu, sans autre médiateur que le Seigneur lui-même (1 Ti. 2:5). ~Or là où les péchés sont pardonnés, il n'y a plus d’offrande pour le péché. Ainsi donc, mes frères, nous avons la liberté d'entrer dans le Saint des saints au moyen du sang de Jésus, qui est le chemin nouveau et vivant qu'il nous a frayé au travers du voile, c’est-à-dire de sa chair. Et ayant un Souverain Sacrificateur établi sur la maison de Dieu, approchons-nous de lui avec un cœur sincère et une foi inébranlable, ayant les cœurs purifiés d’une mauvaise conscience, et le corps lavé d’une eau pure. Retenons sans fléchir la profession de notre espérance, car celui qui nous a fait la promesse est fidèle.~ Hé. 10:18-23. Jésus-Christ est notre Pâque (1 Co. 5:5-8), il est le sacrifice parfait qui a expié nos péchés une fois pour toutes (Hé. 10:10). Par conséquent, il est celui à qui nous devons nous adresser pour recevoir pardon, miséricorde et compassion. ~Tout est accompli~. En s’écriant de la sorte, Jésus-Christ a proclamé la fin de l'ancienne alliance. En effet, la loi a été promulguée par Moïse, mais la grâce et la vérité sont venues par Jésus-Christ (Jn. 1:17). Toutefois, la nouvelle alliance n’a réellement débuté qu’à la Pentecôte avec l’effusion du Saint-Esprit. Voir commentaire en Actes 2.} ; et ayant baissé la tête, il rendit l'esprit.
\TextTitle{[Fin de la première Alliance]
\\(Mt.27:50-51 ; Mc. 15:37-38 ; Lu. 23:45-46)}
\VS{31}De peur que les corps ne restent sur la croix pendant le sabbat, car c’était la préparation, et ce jour de sabbat était un grand jour, les Juifs demandèrent à Pilate qu’on rompe les jambes aux crucifiés, et qu’on les enlève.
\VS{32}Les soldats vinrent donc, et ils rompirent les jambes au premier, et de même à l'autre qui était crucifié avec lui.
\VS{33}Puis étant venus à Jésus, et voyant qu'il était déjà mort, ils ne lui rompirent point les jambes ;
\VS{34}mais un des soldats lui perça le côté avec une lance, et aussitôt il sortit du sang et de l'eau.
\VS{35}Celui qui l'a vu en a rendu témoignage, et son témoignage est digne de foi ; et il sait qu'il dit vrai, afin que vous le croyiez.
\VS{36}Ces choses sont arrivées, afin que l’Ecriture soit accomplie : Aucun de ses os ne sera brisé\FTNT{Ps. 34:21 ; Ex. 12:46 ; No 9:12.}.
\VS{37}Et encore une autre Ecriture, qui dit : Ils verront celui qu'ils ont percé\FTNT{Za. 12:10.}.
\TextTitle{[Jésus enseveli]
\\(Mt. 27:57-66 ; Mc. 15:42-47 ; Lu. 23:50-56)}
\VS{38}Après ces choses, Joseph d'Arimathée, qui était disciple de Jésus, mais en secret parce qu'il craignait les Juifs, demanda à Pilate la permission d’enlever le corps de Jésus ; et Pilate le lui ayant permis, il vint et prit le corps de Jésus.
\VS{39}Nicodème, qui auparavant était allé de nuit vers Jésus, vint aussi, apportant un mélange de myrrhe et d'aloès d'environ cent livres.
\VS{40}Et ils prirent le corps de Jésus, et l'enveloppèrent de linges avec des aromates, comme les Juifs ont coutume d'ensevelir.
\VS{41}Or il y avait un jardin dans le lieu où Jésus fut crucifié, et dans le jardin un sépulcre neuf, où personne n'avait encore été mis.
\VS{42}Ce fut là qu’ils déposèrent Jésus, à cause de la préparation des Juifs, parce que le sépulcre était proche.
\TextTitle{[Déroulement des événements du jour de la résurection]
\\(Mt. 28:1-15 ; Mc. 16:1-14 ; Lu. 24:1-32)}
\Chap{20}
\VerseOne{}Le premier jour de la semaine, Marie de Magdala se rendit dès le matin au sépulcre, comme il faisait encore obscur ; et elle vit que la pierre était ôtée du sépulcre.
\VS{2}Elle courut vers Simon Pierre et vers l'autre disciple que Jésus aimait, et elle leur dit : Ils ont enlevé le Seigneur du sépulcre, et nous ne savons pas où ils l’ont mis.
\VS{3}Alors Pierre partit avec l'autre disciple, et ils s'en allèrent au sépulcre.
\VS{4}Ils couraient tous deux ensemble, mais l'autre disciple courait plus vite que Pierre, et il arriva le premier au sépulcre.
\VS{5}Et s'étant baissé, il vit les linges à terre ; mais il n'y entra point.
\VS{6}Alors Simon Pierre qui le suivait, arriva, et entra dans le sépulcre, et vit les linges à terre,
\VS{7}et le linge qu’on avait mis sur la tête de Jésus, non pas avec les bandes, mais plié dans un lieu à part.
\VS{8}Alors l'autre disciple, qui était arrivé le premier au sépulcre, entra aussi, il vit, et il crut.
\VS{9}Car ils ne comprenaient pas encore que, selon l'Ecriture, Jésus devait ressusciter des morts.
\VS{10}Et les disciples s'en retournèrent chez eux.
\TextTitle{[Jésus apparait aux disciples, Thomas étant absent]
\\(Mc. 16:14 ; Lu. 24:33-49)}
\VS{11}Mais Marie se tenait près du sépulcre dehors, et pleurait. Comme elle pleurait, elle se baissa dans le sépulcre,
\VS{12}et elle vit deux anges vêtus de blanc, assis à la place où avait été couché le corps de Jésus, l'un à la tête et l'autre aux pieds.
\VS{13}Ils lui dirent : Femme, pourquoi pleures-tu ? Elle leur dit : Parce qu'on a enlevé mon Seigneur, et je ne sais point où on l'a mis.
\VS{14}En disant cela, elle se retourna, et elle vit Jésus qui était là, mais elle ne savait pas que c’était Jésus.
\VS{15}Jésus lui dit : Femme, pourquoi pleures-tu ? Qui cherches-tu ? Elle, pensant que c’était le jardinier, lui dit : Seigneur, si c’est toi qui l'as emporté, dis-moi où tu l'as mis, et je le prendrai.
\VS{16}Jésus lui dit : Marie ! Et elle se retourna et lui dit : Rabbouni ! C’est-à-dire, mon Maître !
\VS{17}Jésus lui dit : Ne me touche pas ; car je ne suis point encore monté vers mon Père. Mais va trouver mes frères, et dis-leur que je monte vers mon Père et votre Père, vers mon Dieu et votre Dieu.
\VS{18}Marie de Magdala alla annoncer aux disciples qu'elle avait vu le Seigneur, et qu'il lui avait dit ces choses.
\VS{19}Le soir de ce jour, qui était le premier de la semaine, les portes du lieu où les disciples étaient assemblés, à cause de la crainte qu'ils avaient des Juifs, étaient fermées. Jésus vint, se présenta au milieu d'eux, et il leur dit : Que la paix soit avec vous !
\VS{20}Et quand il leur eut dit cela, il leur montra ses mains et son côté. Les disciples furent dans la joie en voyant le Seigneur.
\VS{21}Jésus leur dit de nouveau : Que la paix soit avec vous ! Comme mon Père m'a envoyé, ainsi je vous envoie.
\VS{22}Après ces paroles, il souffla sur eux, et leur dit : Recevez le Saint-Esprit.
\VS{23}Ceux à qui vous pardonnerez les péchés, ils leur seront pardonnés ; et ceux à qui vous les retiendrez, ils leur seront retenus.
\TextTitle{[Jésus apparait aux disciples, Thomas étant présent]}
\VS{24}Thomas, appelé Didyme, l'un des douze, n'était pas avec eux quand Jésus vint.
\VS{25}Les autres disciples lui dirent : Nous avons vu le Seigneur. Mais il leur dit : Si je ne vois pas les marques des clous dans ses mains, et si je ne mets pas mon doigt où étaient les clous, et si je ne mets pas ma main dans son côté, je ne le croirai point.
\VS{26}Huit jours après, les disciples étaient de nouveau dans la maison, et Thomas se trouvait avec eux. Jésus vint, les portes étant fermées, se présenta au milieu d'eux, et il leur dit : Que la paix soit avec vous !
\VS{27}Puis il dit à Thomas : Mets ton doigt ici, et regarde mes mains, avance aussi ta main, et mets-la dans mon côté ; et ne sois point incrédule, mais crois.
\VS{28}Et Thomas répondit et lui dit : Mon Seigneur, et mon Dieu !
\VS{29}Jésus lui dit : Parce que tu m'as vu, Thomas, tu as cru. Heureux sont ceux qui n'ont pas vu et qui ont cru.
\TextTitle{[But de l'Evangile selon Jean]}
\VS{30}Jésus fit encore, en présence de ses disciples, beaucoup d’autres miracles qui ne sont pas écrits dans ce livre.
\VS{31}Mais ces choses sont écrites afin que vous croyiez que Jésus est le Christ, le Fils de Dieu, et qu'en croyant vous ayez la vie par son Nom.
\TextTitle{[Jésus apparait à sept apôtres au bord de la mer de Galilée]}
\Chap{21}
\VerseOne{}Après cela, Jésus se montra de nouveau à ses disciples, près de la mer de Tibériade. Et voici de quelle manière il se montra.
\VS{2}Simon Pierre, Thomas, appelé Didyme, Nathanaël, de Cana en Galilée, les fils de Zébédée, et deux autres disciples de Jésus étaient ensemble.
\TextTitle{[Christ et notre service :
\\a. Le service de la volonté propre, sous des directives humaines]}
\VS{3}Simon Pierre leur dit : Je vais pêcher. Ils lui dirent : Nous allons aussi avec toi. Ils partirent et montèrent dans une barque ; mais ils ne prirent rien cette nuit-là.
\VS{4}Le matin étant venu, Jésus se trouva sur le rivage ; mais les disciples ne savaient pas que c’était Jésus.
\TextTitle{[b. Inutilité du service de la volonté propre]}
\VS{5}Jésus leur dit : Mes enfants, avez-vous quelque petit poisson à manger ? Ils lui répondirent : Non.
\TextTitle{[c. Résultat du service sous les directives de Christ]}
\VS{6}Et il leur dit : Jetez le filet du côté droit de la barque, et vous trouverez. Ils le jetèrent donc, et ils ne pouvaient plus le retirer à cause de la grande quantité de poissons.
\VS{7}Alors le disciple que Jésus aimait dit à Pierre : C’est le Seigneur. Et quand Simon Pierre eut entendu que c'était le Seigneur, il mit sa tunique et sa ceinture parce qu'il était nu, et il se jeta dans la mer.
\VS{8}Les autres disciples vinrent dans la barque, car ils n'étaient pas loin de terre, mais seulement à environ deux cents coudées, traînant le filet de poissons.
\VS{9}Lorsqu’ils furent descendus à terre, ils virent de la braise, et du poisson dessus, et du pain.
\VS{10}Jésus leur dit : Apportez des poissons que vous venez maintenant de prendre.
\VS{11}Simon Pierre monta et tira le filet à terre, plein de cent cinquante-trois grands poissons ; et quoiqu'il y en eût tant, le filet ne se rompit point.
\TextTitle{[d) Les ressources de Christ pour ses serviteurs]
\\(Lu. 22:35 ; Ph. 4:19)}
\VS{12}Jésus leur dit : Venez et mangez. Et aucun de ses disciples n'osait lui demander : Qui es-tu ? Sachant que c'était le Seigneur.
\VS{13}Jésus donc vint, et prit du pain, et leur en donna ; il fit de même du poisson aussi.
\VS{14}C’était déjà la troisième fois que Jésus se montrait à ses disciples depuis qu’il était ressuscité des morts.
\TextTitle{[e) La charité, seul motif valable pour le vrai service]
\\(1 Co. 13 ; 2 Co. 5:14 ; Ap. 2:4-5)}
\VS{15}Après qu'ils eurent mangé, Jésus dit à Simon Pierre : Simon fils de Jonas, m'aimes-tu plus que ne m’aiment ceux-ci ? Il lui répondit : Oui, Seigneur ! Tu sais que je t'aime. Il lui dit : Pais mes agneaux.
\VS{16}Il lui dit encore : Simon fils de Jonas, m'aimes-tu ? Il lui répondit : Oui, Seigneur ! Tu sais que je t'aime. Il lui dit : Pais mes brebis.
\VS{17}Il lui dit pour la troisième fois : Simon fils de Jonas, m'aimes-tu ? Pierre fut attristé de ce qu'il lui avait dit pour la troisième fois : M'aimes-tu ? Et il lui répondit : Seigneur, tu sais toutes choses, tu sais que je t'aime. Jésus lui dit : Pais mes brebis.
\TextTitle{[f) Le Maître révèle à Pierre qu'il Lui appartient de fixer le temps et la forme de sa mort]}
\VS{18}En vérité, en vérité je te le dis : Quand tu étais plus jeune, tu te ceignais toi-même, et tu allais où tu voulais ; mais quand tu seras vieux, tu étendras tes mains, et un autre te ceindra, et te mènera où tu ne voudras pas.
\VS{19}Il dit cela pour indiquer par quelle mort Pierre glorifierait Dieu. Et ayant ainsi parlé, il lui dit : Suis-moi.
\TextTitle{[g) Tous ses serviteurs ne mourront pas]
\\(1 Co. 15:51-52 ; 1 Th. 4:14-18)}
\VS{20}Pierre se retournant, vit venir après eux le disciple que Jésus aimait, celui qui pendant le souper s'était penché sur la poitrine de Jésus et avait dit : Seigneur, qui est celui qui te trahit ?
\VS{21}Quand donc Pierre le vit, il dit à Jésus : Seigneur, et celui-ci, que lui arrivera-t-il ?
\VS{22}Jésus lui dit : Si je veux qu'il demeure jusqu'à ce que je vienne, que t'importe ? Toi, suis-moi.
\VS{23}Là-dessus, le bruit courut parmi les frères que ce disciple ne mourrait point. Cependant Jésus ne lui avait pas dit : Il ne mourra point ; mais : Si je veux qu'il demeure jusqu'à ce que je vienne, que t'importe ?
\VS{24}C'est ce disciple qui rend témoignage de ces choses, et qui les a écrites. Et nous savons que son témoignage est digne de foi.
\VS{25}Jésus a fait encore beaucoup d’autres choses. Si on les écrivait en détail, je ne pense pas que le monde même pourrait contenir les livres qu’on écrirait. AMEN !
\PPE{}
\end{multicols}

%\addcontentsline{toc}{chapter}{Testament de Jésus}\clearpage
%\clearpage\ShortTitle{Ac.}\BookTitle{Actes}\BFont
\noindent\hrulefill
{\footnotesize
\textit{
\bigskip
{\centering{}
\\Auteur~: Luc
\\Thème~: Les missions du 1er siècle
\\Date de rédaction~: Env. 60 ap. J.-C.\\}
}
\textit{
\\D'origine grecque, Luc fut l'auteur du livre communément appelé «~actes des apôtres~» et de l'évangile éponyme tous deux adressés à Théophile. Ce livre retrace la genèse de l'Eglise, de l'ascension de Jésus à la Pentecôte, de la prédication vivante et fructueuse de Pierre à la conversion de Paul, jusqu'au voyage de celui-ci à Rome en tant que prisonnier. On y découvre des apôtres déterminés, des ouvriers de Christ qui acceptèrent de subir l'humiliation et la persécution par amour de la vérité. Sont également présentés des hommes et des femmes qui - touchés par la simplicité de l'Evangile du Royaume - se convertirent puis se firent baptiser.
\\Bien plus qu'un recueil relatant de banales manifestations, ce livre est avant tout celui des actes du Saint-Esprit. Il témoigne de la résurrection et de la puissance de Jésus-Christ manifestée au travers de son Corps. Il retrace l'origine et le développement du premier réveil après Jésus-Christ, qui fut un véritable bouleversement au sein d'un empire en proie à l'impiété et à l'idolâtrie.\bigskip
}
}
\par\nobreak\noindent\hrulefill
\begin{multicols}{2}
\Chap{1}
\TextTitle{Introduction~: le Messie ressuscité parle des choses qui concernent le Royaume de Dieu pendant quarante jours}
\VerseOne{}Nous avons rempli le premier traité, ô Théophile~! De toutes les choses que Jésus a faites et enseignées, 
\VS{2}jusqu'au jour où il fut élevé au ciel, après avoir donné par le Saint-Esprit, ses ordres aux apôtres qu'il avait élus.
\VS{3}A qui aussi, après avoir souffert, il se présenta lui-même vivant, avec plusieurs preuves assurées, étant vu par eux pendant quarante jours, et leur parlant des choses qui concernent le Royaume de Dieu.
\VS{4}Et les ayant assemblés, il leur ordonna de ne pas partir de Jérusalem, mais d'attendre la promesse du Père, ce que vous avez entendu de moi~;
\VS{5}car Jean a baptisé d'eau, mais vous serez baptisés du Saint-Esprit dans peu de jours.
\VS{6}Eux donc étant assemblés, l'interrogèrent, disant~: Seigneur, est-ce en ce temps-ci que tu rétabliras le royaume d'Israël~?
\VS{7}Mais il leur dit~: Ce n'est pas à vous de connaître les temps et les moments que le Père a fixés de sa propre autorité.
\TextTitle{La puissance du Saint-Esprit pour évangéliser les nations\FTNTT{\vref{Mt. 28:18-20}~; \vref{Mc. 16:15-18}~; \vref{Lu. 24:47-48}~; \vref{Jn. 20:21-22}.}}
\VS{8}Mais vous recevrez la puissance du Saint-Esprit qui viendra sur vous, et vous serez mes témoins, tant à Jérusalem que dans toute la Judée, et la Samarie, et jusqu'aux extrémités de la terre.
\VS{9}Et ayant dit ces choses, il fut élevé, comme ils le regardaient, une nuée le prit et l'emporta de devant leurs yeux.
\TextTitle{Promesse du retour de Jésus}
\VS{10}Et comme ils avaient les yeux fixés vers le ciel, à mesure qu'il s'en allait, voici, deux hommes en vêtements blancs se présentèrent devant eux,
\VS{11}et leur dirent~: Hommes Galiléens, pourquoi vous arrêtez-vous à regarder au ciel~? Ce Jésus qui a été élevé du milieu de vous au ciel, en descendra de la même manière que vous l'avez contemplé montant au ciel\FTNT{Jésus-Christ est monté au ciel depuis la Montagne des Oliviers et lors de son retour, ses pieds se poseront sur cette montagne. Voir \vref{Za. 14}.}.
\TextTitle{Attente du Saint-Esprit promis}
\VS{12}Alors ils s'en retournèrent à Jérusalem de la montagne appelée la Montagne des Oliviers, qui est près de Jérusalem, le chemin d'un sabbat\FTNT{Chemin de sabbat~: C'est la distance qu'il est permis à un juif de parcourir le jour de sabbat (\vref{Ex. 16:29}). Elle correspond à deux mille coudées ou 1100 m.}.
\VS{13}Et quand ils furent entrés dans la ville, ils montèrent dans une chambre haute où demeuraient Pierre et Jacques, Jean et André, Philippe et Thomas, Barthélemy et Matthieu, Jacques, fils d'Alphée, et Simon le zélote, et Jude, frère de Jacques.
\VS{14}Tous ceux-ci, d'un commun accord, persévéraient dans la prière et dans la supplication avec les femmes, avec Marie, mère de Jésus, et avec ses frères.
\TextTitle{Matthias désigné apôtre pour remplacer Judas}
\VS{15}Et en ces jours-là, Pierre se leva au milieu des disciples, qui étaient là assemblés au nombre d'environ cent vingt personnes, et il leur dit~:
\VS{16}Hommes frères, il fallait que s'accomplisse ce qui a été écrit, ce que le Saint-Esprit a annoncé d'avance par la bouche de David, au sujet de Judas, qui a été le guide de ceux qui ont saisi Jésus.
\VS{17}Car il était compté parmi nous, et il avait reçu en partage ce même service.
\VS{18}Mais après avoir acquis un champ avec le salaire du crime qui lui avait été donné, il est tombé, s'est rompu par le milieu, et toutes ses entrailles ont été répandues.
\VS{19}Et ceci a été connu de tous les habitants de Jérusalem, de sorte que ce champ a été appelé dans leur propre langue Hakeldama, c'est-à-dire le champ du sang.
\VS{20}Car il est écrit dans le livre des Psaumes~: Que sa demeure soit déserte, que personne ne l'habite\FTNT{\vref{Ps. 69:26}.}, et qu'un autre prenne sa charge\FTNT{Charge~: Du grec «~episkope~», il s'agit de la fonction d'un ancien. \vref{Ps. 109:8}.}.
\VS{21}Il faut donc que d'entre ces hommes qui se sont assemblés avec nous pendant tout le temps que le Seigneur Jésus a vécu entre nous,
\VS{22}en commençant depuis le baptême de Jean jusqu'au jour où il a été enlevé du milieu de nous, qu'il y en ait un qui soit témoin avec nous de sa résurrection.
\VS{23}Et ils en présentèrent deux~: Joseph, appelé Barsabbas, surnommé Justus, et Matthias.
\VS{24}Et en priant, ils dirent~: Toi, Seigneur, qui connais les cœurs de tous, désigne lequel de ces deux tu as choisi,
\VS{25}afin qu'il prenne part à ce service et à cet apostolat que Judas a abandonné pour aller en son lieu.
\VS{26}Puis ils les tirèrent au sort, et le sort tomba sur Matthias, qui, d'une commune voix, fut mis au rang des onze apôtres.
\Chap{2}
\TextTitle{Effusion de l'Esprit à la Pentecôte~; naissance de l'Eglise\FTNT{\vref{Joë. 2:32}.}}
\VerseOne{}Et comme le jour de la Pentecôte s'accomplissait, ils étaient tous ensemble dans un même lieu.
\VS{2}Et il se fit tout à coup un bruit du ciel, comme est le bruit d'un vent qui souffle avec véhémence, et il remplit toute la maison où ils étaient assis.
\VS{3}Et il leur apparut des langues divisées, comme de feu, qui se posèrent sur chacun d'eux.
\VS{4}Et ils furent tous remplis du Saint-Esprit, et commencèrent à parler des langues étrangères selon que l'Esprit leur donnait de parler.
\VS{5}Or il y avait à Jérusalem des Juifs qui y séjournaient, hommes pieux, de toute nation qui est sous le ciel.
\VS{6}Et ce bruit s'étant répandu, une multitude vint ensemble, et fut confondue de ce que chacun les entendait parler dans sa propre langue. 
\VS{7}Ils en étaient donc tout surpris, et s'en étonnaient, disant l'un à l'autre~: Voici, tous ceux-ci qui parlent ne sont-ils pas Galiléens~?
\VS{8}Comment donc chacun de nous les entendons-nous parler la propre langue du pays où nous sommes nés~? 
\VS{9}Parthes, Mèdes, Elamites, et ceux qui habitent la Mésopotamie, la Judée, la Cappadoce, le Pont, l'Asie,
\VS{10}la Phrygie, la Pamphylie, l'Egypte, le territoire de la Libye qui est près de Cyrène, et ceux qui sont venus de Rome~? Juifs et Prosélytes,
\VS{11} Crétois et Arabes, comment les entendons-nous parler chacun dans notre langue des merveilles de Dieu~?
\VS{12}Ils étaient donc tout étonnés, et ils ne savaient que penser, disant l'un à l'autre~: Que veut dire ceci~?
\VS{13}Mais les autres se moquaient, et disaient~: C'est qu'ils sont pleins de vin doux.
\TextTitle{Prédication de Pierre}
\VS{14}Alors Pierre, se présentant avec les onze, éleva sa voix, et leur dit~: Hommes Juifs, et vous tous qui habitez à Jérusalem, apprenez ceci, et faites attention à mes paroles~!
\VS{15}Ces gens ne sont pas ivres, comme vous le pensez, car c'est la troisième heure\FTNT{Neuf heures du matin.} du jour.
\VS{16}Mais c'est ici ce qui a été dit par le prophète Joël~:
\VS{17}Et il arrivera dans les derniers jours, dit Dieu, que je répandrai de mon Esprit sur toute chair~; vos fils et vos filles prophétiseront, vos jeunes gens auront des visions, et vos vieillards songeront des songes.
\VS{18}Et dans ces jours-là je répandrai de mon Esprit sur mes serviteurs et sur mes servantes, et ils prophétiseront.
\VS{19}Et je ferai des choses merveilleuses en haut dans le ciel, et des prodiges en bas sur la terre, du sang, du feu, et une vapeur de fumée.
\VS{20}Le soleil se changera en ténèbres, et la lune en sang, avant que ce grand et notable jour du Seigneur vienne.
\VS{21}Mais il arrivera que quiconque invoquera le Nom du Seigneur sera sauvé\FTNT{\vref{Joë. 2:28-32}.}.
\TextTitle{Proclamation de la résurrection du Messie}
\VS{22}Hommes Israélites, écoutez ces paroles~! Jésus de Nazareth, homme approuvé de Dieu parmi vous par les miracles et les prodiges et les signes que Dieu a faits par lui au milieu de vous, comme vous-mêmes vous le savez,
\VS{23}ayant été livré selon le dessein arrêté et selon la prescience de Dieu, vous l'avez pris et mis à la croix, vous l'avez fait mourir par les mains des impies.
\VS{24}Mais Dieu l'a ressuscité, ayant brisé les liens de la mort, parce qu'il n'était pas possible qu'il soit retenu par elle.
\VS{25}Car David dit de lui~: Je contemplais constamment le Seigneur devant moi, parce qu'il est à ma droite, afin que je ne sois point ébranlé\FTNT{\vref{Ps. 16:8-11}.}.
\VS{26}C'est pourquoi mon cœur est dans la joie, et ma langue dans l'allégresse~; et de plus, ma chair reposera avec espérance.
\VS{27}Car tu ne laisseras point mon âme en enfer\FTNT{Voir commentaire en \vref{Mt. 16:18}.} et tu ne permettras point que ton Saint voie la corruption.
\VS{28}Tu m'as fait connaître le chemin de la vie, tu me rempliras de joie dans ta présence\FTNT{\vref{Ps. 16:11}.}.
\VS{29}Hommes frères, qu'il me soit permis de vous dire librement, au sujet du patriarche David, qu'il est mort, qu'il a été enseveli, et que son sépulcre existe encore parmi nous jusqu'à ce jour.
\VS{30}Mais comme il était prophète, et qu'il savait que Dieu lui avait promis avec serment, que du fruit de ses reins il ferait naître selon la chair le Christ, pour le faire asseoir sur son trône~;
\VS{31}c'est la résurrection du Christ qu'il a prévue et annoncée, en disant qu'il ne serait pas abandonné en enfer et que sa chair ne verrait pas la corruption.
\VS{32}Dieu a ressuscité ce Jésus~; nous en sommes tous témoins.
\VS{33}Après donc qu'il a été élevé au ciel par la puissance de Dieu, et qu'il a reçu de son Père la promesse du Saint-Esprit, il a répandu ce que maintenant vous voyez et ce que vous entendez.
\VS{34}Car David n'est pas monté au ciel~; mais lui-même dit~: Le Seigneur a dit à mon Seigneur~: Assieds-toi à ma droite,
\VS{35}jusqu'à ce que j'aie mis tes ennemis pour le marchepied de tes pieds\FTNT{\vref{Ps. 110:1}.}. 
\VS{36}Que toute la maison d'Israël sache donc avec certitude que Dieu a fait Seigneur et Christ, ce Jésus, dis-je, que vous avez crucifié.
\TextTitle{Exhortation à la repentance}
\VS{37}Après avoir entendu ces choses, ils eurent le cœur touché de componction\FTNT{Componction~: Tristesse produite par les effets du repentir, le regret d'avoir offensé Dieu.}, et ils dirent à Pierre et aux autres apôtres~: Hommes frères, que ferons-nous~?
\VS{38}Et Pierre leur dit~: Repentez-vous, et que chacun de vous soit baptisé au Nom de Jésus-Christ, pour obtenir le pardon de vos péchés, et vous recevrez le don du Saint-Esprit.
\VS{39}Car à vous et à vos enfants est faite la promesse, et à tous ceux qui sont loin, autant que le Seigneur, notre Dieu en appellera à lui.
\VS{40}Et par plusieurs autres paroles, il les conjurait et les exhortait, en disant~: Sauvez-vous de cette génération perverse.
\TextTitle{Conversion et baptême de trois mille personnes~; les débuts de l'Eglise}
\VS{41}Ceux donc qui reçurent de bon cœur sa parole, furent baptisés~; et en ce jour-là furent ajoutées à l'Eglise environ trois mille âmes.
\VS{42}Et ils persévéraient tous dans la doctrine des apôtres, dans la communion fraternelle, dans la fraction du pain, et dans les prières.
\VS{43}Et tout le monde avait de la crainte, et beaucoup de miracles et de prodiges se faisaient par les apôtres.
\VS{44}Tous ceux qui croyaient étaient ensemble dans le même lieu, et ils avaient tout en commun~;
\VS{45}et ils vendaient leurs possessions et leurs biens, et les distribuaient à tous, selon les besoins de chacun.
\VS{46}Et tous les jours, ils persévéraient tous d'un commun accord dans le temple~; et rompant le pain de maison en maison, ils prenaient leur repas avec joie et simplicité de cœur~;
\VS{47}louant Dieu et se rendant agréables à tout le peuple. Et le Seigneur ajoutait tous les jours à l'Eglise des gens pour être sauvés.
\Chap{3}
\TextTitle{Guérison d'un homme boiteux de naissance}
\VerseOne{}Et comme Pierre et Jean montaient ensemble au temple à l'heure de la prière~; c'était la neuvième heure.
\VS{2}Et il y avait un homme boiteux de naissance, qu'on portait, et qu'on mettait tous les jours à la porte du temple, appelée la Belle, pour demander l'aumône à ceux qui entraient dans le temple. 
\VS{3}Cet homme voyant Pierre et Jean qui allaient entrer au temple, les pria de lui donner l'aumône.
\VS{4}Alors Pierre, de même que Jean, fixa les yeux sur lui, et lui dit~: Regarde-nous.
\VS{5}Et il les regardait attentivement, s'attendant de recevoir quelque chose d'eux.
\VS{6}Mais Pierre lui dit~: Je n'ai ni argent, ni or~; mais ce que j'ai, je te le donne~: Au Nom de Jésus-Christ de Nazareth, lève-toi et marche.
\VS{7}Et l'ayant pris par la main droite, il le fit lever~; et aussitôt les plantes et les chevilles de ses pieds devinrent fermes.
\VS{8}Et faisant un saut, il se tint debout, et marcha~; et il entra avec eux au temple, marchant, sautant, et louant Dieu.
\VS{9}Et tout le peuple le vit marchant et louant Dieu.
\VS{10}Et reconnaissant que c'était celui-là même qui était assis à la Belle, porte du temple, pour avoir l'aumône, ils furent remplis d'admiration et d'étonnement de ce qui lui était arrivé.
\VS{11}Et comme le boiteux, qui avait été guéri, tenait par la main Pierre et Jean, tout le peuple étonné accourut vers eux, au portique qu'on appelle de Salomon.
\TextTitle{Christ, le Messie annoncé par les prophètes}
\VS{12}Mais Pierre voyant cela, dit au peuple~: Hommes Israélites, pourquoi vous étonnez-vous de ceci~? Ou pourquoi avez-vous les regards fixés sur nous, comme si par notre puissance ou par notre piété, nous avions fait marcher cet homme~?
\VS{13}Le Dieu d'Abraham, d'Isaac, et de Jacob, le Dieu de nos pères, a glorifié son Fils Jésus, que vous avez livré et renié devant Pilate, quoiqu'il jugeât qu'il devait être relâché.
\VS{14}Mais vous avez renié le Saint et le Juste, et vous avez demandé qu'on vous relâche un meurtrier.
\VS{15}Vous avez fait mourir le Prince de la vie, que Dieu a ressuscité des morts~; nous en sommes témoins.
\VS{16}C'est par la foi en son Nom, que son Nom a raffermi les pieds de cet homme que vous voyez et connaissez. La foi, dis-je, que nous avons en lui, a donné à cet homme cette entière guérison de tous ses membres, en présence de vous tous.
\VS{17}Et maintenant, mes frères, je sais que vous avez agi par ignorance, de même que vos chefs.
\VS{18}Mais Dieu a ainsi accompli les choses qu'il avait prédites par la bouche de tous ses prophètes, que le Christ devait souffrir\FTNT{\vref{Es. 53}.}.
\VS{19}Repentez-vous donc, et convertissez-vous, afin que vos péchés soient effacés~;
\VS{20}afin que des temps de rafraîchissement viennent par la présence du Seigneur, et qu'il envoie celui qui vous a été auparavant annoncé, Jésus-Christ,
\VS{21}lequel il faut que le ciel reçoive, jusqu'au temps du rétablissement de toutes les choses que Dieu a prononcées par la bouche de tous ses saints prophètes, dès le commencement du monde.
\VS{22}Car Moïse lui-même a dit à nos pères~: Le Seigneur votre Dieu, vous suscitera d'entre vos frères un Prophète comme moi~; vous l'écouterez dans tout ce qu'il vous dira,
\VS{23}et il arrivera que toute personne qui n'aura pas écouté ce Prophète, sera exterminé du milieu du peuple\FTNT{\vref{De. 18:15-19}.}.
\VS{24}Et même tous les Prophètes depuis Samuel, et ceux qui l'ont suivi, tout autant qu'il y en a eu qui ont parlé, ont aussi prédit ces jours.
\VS{25}Vous êtes les enfants des prophètes et de l'Alliance que Dieu a traitée avec nos pères, en disant à Abraham~: Toutes les familles de la terre seront bénies en ta postérité\FTNT{\vref{Ge. 12:2}.}.
\VS{26}C'est à vous premièrement que Dieu, ayant suscité son Fils Jésus, l'a envoyé pour vous bénir, en détournant chacun de vous de vos iniquités.
\Chap{4}
\TextTitle{Première persécution de l'Eglise~: Pierre et Jean jetés en prison}
\VerseOne{}Mais comme ils parlaient au peuple, survinrent les prêtres, le commandant du temple et les sadducéens,
\VS{2}étant offensés de ce qu'ils enseignaient le peuple, et qu'ils annonçaient la résurrection des morts au Nom de Jésus.
\VS{3}Et les ayant fait arrêter, ils les mirent en prison jusqu'au lendemain, parce qu'il était déjà tard.
\VS{4}Et plusieurs de ceux qui avaient entendu la parole crurent~; et le nombre des personnes fut d'environ cinq mille.
\TextTitle{Pierre et Jean convoqués au sanhédrin}
\VS{5}Or il arriva que le lendemain, les chefs, les anciens et les scribes s'assemblèrent à Jérusalem~;
\VS{6}avec Anne, le grand-prêtre, Caïphe, Jean, Alexandre, et tous ceux qui étaient de la race des principaux prêtres.
\VS{7}Et ayant fait comparaître devant eux Pierre et Jean, ils leur demandèrent~: Par quelle puissance, ou au nom de qui avez-vous fait cette guérison~?
\VS{8}Alors Pierre étant rempli du Saint-Esprit, leur dit~: Chefs du peuple, et vous anciens d'Israël~:
\VS{9}Puisque nous sommes jugés aujourd'hui sur un bienfait accordé à un homme impotent, afin que nous disions comment il a été guéri,
\VS{10}sachez, vous tous et tout le peuple d'Israël, que c'est au Nom de Jésus-Christ de Nazareth, que vous avez crucifié, et que Dieu a ressuscité des morts~; c'est en son Nom, que cet homme qui parait ici devant vous, a été guéri.
\VS{11}C'est cette pierre rejetée, par vous qui bâtissez, qui est devenue la pierre principale de l'angle\FTNT{\vref{Ps. 118:22}.}.
\VS{12}Il n'y a de salut en aucun autre~: Car il n'y a sous le ciel aucun autre Nom qui ait été donné aux hommes par lequel nous devions être sauvés.
\TextTitle{Le sanhédrin interdit aux apôtres de prêcher au Nom de Jésus}
\VS{13}Eux, voyant la hardiesse de Pierre et de Jean, et sachant aussi qu'ils étaient des hommes sans instruction et du commun peuple~; s'en étonnaient, et ils reconnaissaient bien qu'ils avaient été avec Jésus.
\VS{14}Et voyant que l'homme qui avait été guéri, était présent avec eux, ils ne pouvaient contredire en rien.
\VS{15}Alors ils leur ordonnèrent de sortir hors du sanhédrin, et ils délibérèrent entre eux, disant~: Que ferons-nous à ces gens~?
\VS{16}Car il est manifeste pour tous les habitants de Jérusalem, qu'un miracle a été fait par eux, et cela est si évident que nous ne pouvons le nier.
\VS{17}Mais afin qu'il ne soit plus divulgué parmi le peuple, défendons-leur avec menaces expresses, qu'ils n'aient plus à parler à qui que ce soit en ce Nom.
\VS{18}Et les ayant donc appelés, ils leur ordonnèrent de ne plus parler ni d'enseigner en aucune manière au Nom de Jésus. 
\VS{19}Mais Pierre et Jean leur répondirent~: Jugez s'il est juste devant Dieu de vous obéir plutôt qu'à Dieu.
\VS{20}Car nous ne pouvons pas ne pas parler de ce que nous avons vu et entendu.
\VS{21}Alors ils les relâchèrent avec menaces, ne trouvant point comment ils pourraient les punir, à cause du peuple, parce que tous glorifiaient Dieu de ce qui avait été fait.
\VS{22}Car l'homme en qui cette miraculeuse guérison avait été faite, avait plus de quarante ans.
\TextTitle{L'Eglise demande l'assistance de Dieu}
\VS{23}Après avoir été relâchés, ils allèrent vers les leurs, et leur racontèrent tout ce que les principaux prêtres et les anciens leur avaient dit.
\VS{24}Eux l'ayant entendu, élevèrent tous ensemble la voix à Dieu, et dirent~: Seigneur, tu es le Dieu qui as fait le ciel et la terre, la mer, et toutes les choses qui y sont~;
\VS{25}et qui as dit par la bouche de David ton serviteur~: Pourquoi ce tumulte parmi les nations et ces vaines pensées parmi les peuples~?
\VS{26}Les rois de la terre se sont soulevés en personne, et les princes se sont ligués ensemble contre le Seigneur, et contre son Christ\FTNT{\vref{Ps. 2:1-2}.}.
\VS{27}En effet, contre ton Saint Fils Jésus, que tu as oint, se sont assemblés Hérode et Ponce Pilate, avec les Gentils, et le peuple d'Israël,
\VS{28}pour faire toutes les choses que ta main et ton conseil avaient auparavant déterminé qui seraient faites. 
\VS{29}Maintenant donc, Seigneur, regarde à leurs menaces, et donne à tes serviteurs d'annoncer ta parole avec toute hardiesse~;
\VS{30}en étendant ta main afin qu'il se fasse des guérisons, des prodiges, et des merveilles, par le Nom de ton Saint Fils Jésus.
\VS{31}Et quand ils eurent prié, le lieu où ils étaient assemblés trembla~; et ils furent tous remplis du Saint-Esprit, et ils annonçaient la parole de Dieu avec hardiesse.
\TextTitle{La multitude unie comme un seul corps\FTNTT{\vref{Ac. 2:42-47}.}}
\VS{32}Or la multitude de ceux qui croyaient n'était qu'un cœur et qu'une âme et nul ne disait d'aucune des choses qu'il possédait, qu'elle fût à lui, mais toutes choses étaient communes entre eux.
\VS{33}Aussi les apôtres rendaient témoignage avec une grande force à la résurrection du Seigneur Jésus~; et une grande grâce était sur eux tous.
\VS{34}Car il n'y avait parmi eux aucun indigent~; parce que tous ceux qui possédaient des champs ou des maisons, les vendaient, et ils apportaient le prix des choses vendues,
\VS{35}et le mettaient aux pieds des apôtres~; et il était distribué à chacun selon qu'il en avait besoin.
\VS{36}Or Joseph, surnommé par les apôtres Barnabas, c'est-à-dire, fils de consolation, Lévite, originaire de Chypre,
\VS{37}ayant une possession, la vendit, et en apporta le prix, et le mit aux pieds des apôtres.
\Chap{5}
\TextTitle{Mensonge d'Ananias et Saphira~: Leur mort}
\VerseOne{}Mais un homme appelé Ananias, et Saphira sa femme, vendit une possession,
\VS{2}et retint une partie du prix, sa femme le sachant~; puis il apporta le reste, et le déposa aux pieds des apôtres.
\VS{3}Mais Pierre lui dit~: Ananias comment Satan s'est-il emparé de ton cœur jusqu'à t'inciter à mentir au Saint-Esprit, et à soustraire une partie du prix de la possession~?
\VS{4}Si tu l'avais gardée, ne te restait-elle pas~? Et après qu'elle ait été vendue, le prix n'était-il pas à ta disposition~? Comment as-tu pu mettre en ton cœur un pareil dessein~? Tu n'as pas menti aux hommes mais à Dieu.
\VS{5}Et Ananias, entendant ces paroles, tomba et rendit l'âme~; ce qui causa une grande crainte à tous ceux qui en entendirent parler.
\VS{6}Et quelques jeunes hommes se levant le prirent, et l'emportèrent dehors, et l'ensevelirent.
\VS{7}Et il arriva environ trois heures après que sa femme entra, sans savoir ce qui était arrivé.
\VS{8}Et Pierre prenant la parole, lui dit~: Dis-moi, avez-vous autant vendu le champ~? Et elle dit~: Oui, autant.
\VS{9}Alors Pierre lui dit~: Pourquoi avez-vous fait un complot entre vous pour tenter l'Esprit du Seigneur~? Voici, à la porte, les pieds de ceux qui ont enterré ton mari, et ils t'emporteront.
\VS{10}Et au même instant, elle tomba à ses pieds et rendit l'esprit. Et quand les jeunes hommes furent entrés, ils la trouvèrent morte, et ils l'emportèrent dehors, et l'ensevelirent auprès de son mari.
\VS{11}Et cela donna une grande crainte à toute l'Eglise, et à tous ceux qui entendaient ces choses.
\TextTitle{Miracles à Jérusalem}
\VS{12}Beaucoup de prodiges et de miracles se faisaient parmi le peuple par les mains des apôtres~; et ils étaient tous d'un commun accord au portique de Salomon.
\VS{13}Cependant aucun des autres n'osait se joindre à eux, mais le peuple les louait hautement.
\VS{14}Et le nombre de ceux qui croyaient au Seigneur, tant d'hommes que de femmes, se multipliait de plus en plus.
\VS{15}Et on apportait les malades dans les rues, et on les mettait sur de petits lits et sur des couchettes, afin que quand Pierre viendrait, au moins son ombre passe sur quelqu'un d'eux.
\VS{16}La multitude accourait aussi des villes voisines à Jérusalem, amenant des malades, et ceux qui étaient tourmentés des esprits impurs~; et tous étaient guéris.
\TextTitle{Deuxième persécution de l'Eglise~: Les apôtres en prison puis devant le sanhédrin}
\VS{17}Alors le grand-prêtre se leva, lui et tous ceux qui étaient avec lui, à savoir la secte des sadducéens, et ils furent remplis de jalousie~;
\VS{18}et mettant la main sur les apôtres, ils les jetèrent dans la prison publique.
\VS{19}Mais l'Ange du Seigneur ouvrit pendant la nuit les portes de la prison, les fit sortir, et leur dit~:
\VS{20}Allez, et présentez-vous dans le temple, annoncez au peuple toutes les paroles de cette vie.
\VS{21}Ayant entendu cela, ils entrèrent dès le matin dans le temple, et se mirent à enseigner. Mais le grand-prêtre et ceux qui étaient avec lui étant arrivés, ils convoquèrent le sanhédrin et tous les anciens des fils d'Israël, et ils envoyèrent chercher les apôtres à la prison.
\VS{22}Mais, les huissiers à leur arrivée, ne les trouvèrent point dans la prison. Ils retournèrent, et firent leur rapport,
\VS{23}en disant~: Nous avons trouvé la prison fermée avec toute sûreté, et les gardes aussi qui étaient devant les portes~; mais après l'avoir ouverte, nous n'avons trouvé personne dedans.
\VS{24}Lorsque le grand-prêtre, le commandant du temple, et les principaux prêtres, eurent entendu ces paroles, ils ne savaient que penser au sujet des apôtres, ne sachant ce qui arriverait de tout cela.
\VS{25}Mais quelqu'un vint leur dire~: Voici, les hommes que vous avez mis en prison sont dans le temple, et ils enseignent le peuple.
\VS{26}Alors le commandant du temple partit avec les huissiers, et il les conduisit sans violence, car ils avaient peur d'être lapidés par le peuple.
\VS{27}Après qu'ils les eurent amenés, ils les présentèrent au sanhédrin. Et le grand-prêtre les interrogea, disant~: 
\VS{28}Ne vous avons-nous pas défendu expressément d'enseigner en ce Nom-là~? Et cependant voici, vous avez rempli Jérusalem de votre doctrine, et vous voulez faire retomber sur nous le sang de cet homme.
\VS{29}Alors Pierre et les autres apôtres répondant, dirent~: Il faut plutôt obéir à Dieu qu'aux hommes.
\VS{30}Le Dieu de nos pères a ressuscité Jésus, que vous avez fait mourir en le pendant au bois.
\VS{31}Dieu l'a élevé par sa puissance pour être Prince et Sauveur, afin de donner à Israël la repentance et la rémission des péchés.
\VS{32}Nous sommes témoins de ce que nous disons, de même que le Saint-Esprit que Dieu a donné à ceux qui lui obéissent, en est aussi témoin.
\TextTitle{Parole de sagesse de Gamaliel}
\VS{33}Mais eux, ayant entendu ces choses, grinçaient les dents, et consultaient pour les faire mourir.
\VS{34}Mais un pharisien nommé Gamaliel, docteur de la loi, honoré de tout le peuple, se leva dans le sanhédrin, et ordonna de faire sortir un instant les apôtres.
\VS{35}Puis il leur dit~: Hommes Israélites, prenez garde à ce que vous allez faire à l'égard de ces gens.
\VS{36}Car il n'y a pas longtemps que Theudas s'éleva, se disant être quelque chose, et auquel se joignit un nombre d'environ quatre cents hommes~; mais il fut tué, et tous ceux qui s'étaient joints à lui ont été dissipés et réduits à rien.
\VS{37}Après lui parut Judas le Galiléen au temps du recensement, et il attira à lui un grand peuple~; il périt aussi, et tous ceux qui s'étaient joints à lui ont été dispersés.
\VS{38}Maintenant donc je vous dis~: Ne continuez plus vos poursuites contre ces hommes, et laissez-les. Car si cette entreprise ou cette œuvre vient des hommes, elle sera détruite~;
\VS{39}mais si elle vient de Dieu, vous ne pourrez pas la détruire. Et prenez garde qu'il ne se trouve que vous combattiez contre Dieu.
\VS{40}Et ils furent de son avis. Et ayant appelé les apôtres, ils les firent battre de verges, ils leur défendirent de parler au Nom de Jésus, et ils les relâchèrent.
\TextTitle{Frappés, les apôtres continuent de prêcher le Nom de Jésus}
\VS{41}Et les apôtres se retirèrent de devant le sanhédrin, joyeux d'avoir été jugés dignes de subir des outrages pour le Nom de Jésus.
\VS{42}Et tous les jours, ils ne cessaient d'enseigner, et d'annoncer l'Evangile de Jésus-Christ dans le temple, et de maison en maison.
\Chap{6}
\TextTitle{Sept hommes choisis pour le service}
\VerseOne{}En ces jours-là, comme les disciples se multipliaient, il s'éleva un murmure des Hellénistes\FTNT{Les Hellénistes étaient des juifs issus de la diaspora ayant adopté la culture et la langue grecque.} contre les Hébreux, parce que leurs veuves étaient méprisées dans le service ordinaire.
\VS{2}C'est pourquoi les douze, ayant convoqué la multitude des disciples, leur dirent~: Il n'est pas raisonnable que nous laissions la parole de Dieu pour servir aux tables.
\VS{3}Regardez donc, mes frères, pour choisir sept hommes d'entre vous, de qui on ait bon témoignage, pleins du Saint-Esprit et de sagesse, auxquels nous confierons ce devoir.
\VS{4}Et nous, nous continuerons à vaquer à la prière et au service de la parole.
\VS{5}Et ce discours plut à toute l'assemblée qui était là présente~; et ils élurent Etienne, homme plein de foi et du Saint-Esprit, Philippe, Prochore, Nicanor, Timon, Parménas, et Nicolas, prosélyte d'Antioche.
\VS{6}Ils les présentèrent aux apôtres~; qui, après avoir prié, leur imposèrent les mains.
\VS{7}Et la parole de Dieu croissait, et le nombre des disciples se multipliait beaucoup dans Jérusalem~; un grand nombre aussi de prêtres obéissait à la foi.
\VS{8}Or Etienne, plein de foi et de puissance, faisait de grands miracles et de grands prodiges parmi le peuple.
\TextTitle{Troisième persécution de l'Eglise~; Etienne convoqué au sanhédrin}
\VS{9}Quelques-uns de la synagogue appelée la synagogue des affranchis\FTNT{Affranchis~: Du grec «~libertinos~», c'est-à-dire «~libertins~»~: Hommes libres. Fraction de la communauté Juive qui avait sa propre synagogue à Jérusalem. Probablement des Juifs qui avaient été faits prisonniers par Pompée et d'autres généraux romains, qui avaient été déportés à Rome, puis libérés.}, de celle des Cyrénéens et de celle des Alexandrins, avec ceux de Cilicie et d'Asie, se levèrent pour disputer contre Etienne.
\VS{10}Mais ils ne pouvaient pas résister à la sagesse et à l'Esprit par lequel il parlait.
\VS{11}Alors ils soudoyèrent des hommes qui dirent~: Nous l'avons entendu proférer des paroles blasphématoires contre Moïse et contre Dieu.
\VS{12}Et ils soulevèrent le peuple, les anciens, et les scribes, et se jetant sur lui, ils l'enlevèrent et l'amenèrent au sanhédrin.
\VS{13}Et ils présentèrent de faux témoins qui dirent~: Cet homme ne cesse de proférer des paroles blasphématoires contre ce saint lieu et contre la loi.
\VS{14}Car nous l'avons entendu dire que Jésus, ce Nazaréen, détruira ce lieu-ci, et changera les coutumes que Moïse nous a données.
\VS{15}Tous ceux qui siégeaient au sanhédrin avaient les yeux fixés sur lui, son visage leur parut comme celui d'un ange.
\Chap{7}
\TextTitle{Discours d'Etienne devant le sanhédrin}
\VerseOne{}Alors le grand-prêtre lui dit~: Ces choses sont-elles ainsi~?
\VS{2}Etienne répondit~: Hommes frères et pères, écoutez-moi~! Le Dieu de gloire apparut à notre père Abraham, lorsqu'il était en Mésopotamie, avant qu'il s'établisse à Charran, et lui dit~:
\VS{3}Sors de ton pays et de ta famille, et va dans le pays que je te montrerai.
\VS{4}Il sortit donc du pays des Chaldéens, et alla demeurer à Charran. De là, après la mort de son père, Dieu le fit passer dans ce pays que vous habitez maintenant.
\VS{5}Et il ne lui donna aucun héritage dans ce pays, non pas même d'un pied de terre, quoiqu'il lui ait promis de le lui donner en possession, et à sa postérité après lui, dans un temps où il n'avait point encore d'enfant.
\VS{6}Dieu lui parla ainsi~: Ta postérité séjournera dans une terre étrangère pendant quatre cents ans~; et on la réduira à la servitude et on la maltraitera.
\VS{7}Mais je jugerai la nation à laquelle ils auront été asservis, dit Dieu~; et après cela ils sortiront, et me serviront en ce lieu-ci\FTNT{\vref{Ge. 15:13-14}.}.
\VS{8}Puis il donna à Abraham l'alliance de la circoncision~; et après cela Abraham engendra Isaac qu'il circoncit le huitième jour. Isaac engendra Jacob, et Jacob les douze patriarches.
\VS{9}Les patriarches, jaloux de Joseph, le vendirent pour être emmené en Egypte.
\VS{10}Mais Dieu était avec lui, et le délivra de toutes ses afflictions~; et l'ayant rempli de sagesse il le rendit agréable à Pharaon, roi d'Egypte, qui l'établit gouverneur sur l'Egypte, et sur toute sa maison.
\VS{11}Or il survint dans tout le pays d'Egypte, et dans celui de Canaan, une famine et une grande détresse, en sorte que nos pères ne pouvaient trouver des vivres.
\VS{12}Mais Jacob apprit qu'il y avait du blé en Egypte, il y envoya une première fois nos pères.
\VS{13}Et la seconde fois, Joseph fut reconnu par ses frères, et la famille de Joseph fut déclarée à Pharaon.
\VS{14}Alors Joseph envoya chercher Jacob, son père, et toute sa famille, composée de soixante-quinze personnes.
\VS{15}Jacob descendit en Egypte, et il y mourut, lui et nos pères~;
\VS{16}qui furent transportés à Sichem, et mis dans le sépulcre qu'Abraham avait acheté à prix d'argent des fils d'Hamor, fils de Sichem.
\VS{17}Mais comme le temps de la promesse, pour laquelle Dieu avait juré à Abraham, s'approchait, le peuple s'augmenta et se multiplia en Egypte~;
\VS{18}jusqu'à ce que parut en Egypte un autre roi, qui n'avait pas connu Joseph.
\VS{19}Ce roi, usant d'artifice contre notre race, maltraita nos pères jusqu'à leur faire exposer leurs enfants à l'abandon, afin d'en faire périr la race.
\VS{20}En ce temps-là naquit Moïse, qui fut divinement beau. Et il fut nourri trois mois dans la maison de son père.
\VS{21}Mais ayant été exposé à l'abandon, la fille de Pharaon le recueillit et l'éleva comme son fils.
\VS{22}Moïse fut instruit dans toute la sagesse des Egyptiens~; et il était puissant en paroles et en œuvres.
\VS{23}Mais quand il fut parvenu à l'âge de quarante ans, il forma le dessein d'aller visiter ses frères, les enfants d'Israël.
\VS{24}Et voyant l'un d'eux à qui l'on faisait tort, il le défendit, et vengea celui qui était outragé en tuant l'Egyptien.
\VS{25}Il croyait que ses frères comprendraient par là que Dieu les délivrerait par son moyen~; mais ils ne le comprirent point.
\VS{26}Et le jour suivant, il parut au milieu d'eux comme ils se querellaient, et il tâcha de les mettre d'accord en leur disant~: Hommes, vous êtes frères, pourquoi vous faites-vous tort l'un à l'autre~?
\VS{27}Mais celui qui maltraitait son prochain le repoussa, en disant~: Qui t'a établi prince et juge sur nous~?
\VS{28}Veux-tu me tuer, comme tu as tué hier l'Egyptien~?
\VS{29}Alors Moïse s'enfuit sur un tel discours, et fut étranger dans le pays de Madian, où il eut deux fils.
\VS{30}Et quarante ans étant accomplis, l'Ange du Seigneur lui apparut au désert de la montagne de Sinaï, dans la flamme d'un buisson en feu.
\VS{31}Et quand Moïse le vit, il fut étonné de la vision, et comme il approchait pour considérer ce que c'était, la voix du Seigneur lui fut adressée, disant~:
\VS{32} Je suis le Dieu de tes pères, le Dieu d'Abraham, le Dieu d'Isaac, et le Dieu de Jacob. Et Moïse tout tremblant n'osait pas regarder.
\VS{33}Le Seigneur lui dit~: Ôte tes souliers de tes pieds, car le lieu sur lequel tu te tiens est une terre sainte.
\VS{34}J'ai vu, j'ai vu l'affliction de mon peuple qui est en Egypte, et j'ai entendu leur gémissement, et je suis descendu pour les délivrer. Maintenant donc, va, je t'enverrai en Egypte.
\VS{35}Ce Moïse, qu'ils avaient rejeté en disant~: Qui t'a établi prince et juge~? C'est lui que Dieu envoya comme prince et comme libérateur par le moyen de l'Ange qui lui était apparu dans le buisson.
\VS{36}C'est celui qui les tira dehors, en opérant des miracles et des prodiges au pays d'Egypte, au sein de la Mer Rouge, et au désert pendant quarante ans.
\VS{37}C'est ce Moïse qui a dit aux enfants d'Israël~: Le Seigneur votre Dieu vous suscitera d'entre vos frères un Prophète comme moi~; écoutez-le\FTNT{\vref{De. 18:15}.}~!
\VS{38}C'est lui, qui, lors de l'assemblée au désert, étant avec l'Ange qui lui parlait sur la montagne de Sinaï et avec nos pères, reçut les paroles de vie pour nous les donner.
\VS{39}Nos pères ne voulurent pas lui obéir, mais ils le rejetèrent, et ils tournèrent leur cœur vers l'Egypte,
\VS{40}en disant à Aaron~: Fais-nous des dieux qui marchent devant nous~; car nous ne savons point ce qui est arrivé à ce Moïse qui nous a amenés hors du pays d'Egypte.
\VS{41}Ils firent donc en ces jours-là un veau, et ils offrirent des sacrifices à l'idole, et se réjouirent de l'œuvre de leurs mains.
\VS{42}C'est pourquoi aussi Dieu se détourna d'eux, et les livra au culte de l'armée du ciel, ainsi qu'il est écrit dans le livre des prophètes~: Maison d'Israël, m'avez-vous offert des sacrifices et des victimes pendant quarante ans au désert~?
\VS{43}Mais vous avez porté la tente de Moloc\FTNT{\vref{Lé. 18:21}.}, et l'étoile de votre dieu Remphan~; qui sont des figures que vous avez faites pour les adorer. C'est pourquoi je vous transporterai au-delà de Babylone.
\VS{44}Nos pères avaient au désert le tabernacle du témoignage, comme l'avait ordonné celui qui avait dit à Moïse de le faire selon le modèle qu'il avait vu.
\VS{45}Et nos pères avaient reçu ce tabernacle, ils le portèrent sous la conduite de Josué dans le pays qui était possédé par les nations que Dieu chassa de devant eux, et il y resta jusqu'aux jours de David.
\VS{46}David trouva grâce devant Dieu, et demanda de pouvoir dresser une tente pour le Dieu de Jacob.
\VS{47}Et ce fut Salomon qui lui bâtit une maison.
\VS{48}Mais le Très-Haut n'habite pas dans des temples faits de main d'homme, selon ces paroles du prophète~:
\VS{49}Le ciel est mon trône, et la terre est le marchepied de mes pieds~: Quelle maison me bâtirez-vous, dit le Seigneur, ou quel pourrait être le lieu de mon repos~?
\VS{50}Ma main n'a-t-elle pas fait toutes ces choses\FTNT{\vref{Es. 66:1}.}~?
\VS{51}Hommes au cou raide, et incirconcis de cœur et d'oreilles, vous vous obstinez toujours contre le Saint-Esprit~; vous faites comme vos pères ont fait.
\VS{52}Lequel des prophètes vos pères n'ont-ils pas persécuté~? Ils ont même tué ceux qui annonçaient d'avance l'avènement du Juste, dont vous avez été les traîtres et les meurtriers,
\VS{53}vous qui avez reçu la loi par une ordonnance des anges, et qui ne l'avez point gardée.
\TextTitle{Etienne~: Premier martyr}
\VS{54}En entendant ces choses, leur cœur s'enflamma de colère, et ils grinçaient des dents contre lui.
\VS{55}Mais Etienne, rempli du Saint-Esprit, et fixant les yeux vers le ciel, vit la gloire de Dieu, et Jésus qui était à la droite de Dieu.
\VS{56}Et il dit~: Voici, je vois les cieux ouverts, et le Fils de l'homme étant à la droite de Dieu.
\VS{57}Alors ils s'écrièrent à haute voix, et bouchèrent leurs oreilles, et tous d'un commun accord se jetèrent sur lui.
\VS{58}Et l'ayant tiré hors de la ville, ils le lapidèrent~; et les témoins déposèrent leurs vêtements aux pieds d'un jeune homme nommé Saul.
\VS{59}Et ils lapidaient Etienne qui priait et disait~: Seigneur Jésus, reçois mon esprit\FTNT{Dans \vref{Ec. 12:9}, il est dit qu'à la mort, l'esprit retourne à Dieu qui l'a donné. Jésus est donc Dieu puisqu'il a reçu l'esprit d'Etienne.}~!
\VS{60}Et s'étant mis à genoux, il cria à haute voix~: Seigneur, ne leur impute point ce péché~! Et quand il eut dit cela, il s'endormit.
\Chap{8}
\TextTitle{Quatrième persécution de l'Eglise~: Saul opprime les saints}
\VerseOne{}Or Saul consentait à la mort d'Etienne, et en ce temps-là, il y eut une grande persécution contre l'Eglise de Jérusalem. Et tous, excepté les apôtres, se dispersèrent dans les contrées de la Judée et de la Samarie.
\VS{2}Et quelques hommes pieux emportèrent Etienne pour l'ensevelir, et le pleurèrent à grand bruit.
\VS{3}Mais Saul ravageait l'église, entrant dans toutes les maisons, et traînant par force hommes et femmes, il les mettait en prison.
\TextTitle{Le déploiement des chrétiens\FTNTT{\vref{Ac. 11:19-21}}}
\VS{4}Ceux qui avaient été dispersés allaient de lieu en lieu, annonçant la parole de Dieu.
\TextTitle{Philippe en Samarie~; Simon le magicien}
\VS{5}Philippe, étant descendu dans la ville de Samarie, leur prêcha Christ.
\VS{6}Et les foules tout entières étaient attentives à ce que Philippe disait, l'écoutant, lorsqu'elles virent les miracles qu'il faisait,
\VS{7}car les esprits impurs sortaient, en criant à haute voix, hors de plusieurs qui en étaient possédés, et beaucoup de paralytiques et de boiteux furent guéris.
\VS{8}Ce qui causa une grande joie dans cette ville-là.
\VS{9}Or il y avait auparavant dans la ville un homme nommé Simon qui exerçait l'art d'enchanteur, et ensorcelait le peuple de Samarie, se disant être quelque grand personnage.
\VS{10}Tous, depuis le plus petit jusqu'au plus grand étaient attachés à lui, et disaient~: Celui-ci est la grande puissance de Dieu.
\VS{11}Et ils étaient attachés à lui, parce que depuis longtemps il les avait éblouis par sa magie.
\VS{12}Mais quand ils eurent cru ce que Philippe leur annonçait, touchant l'Evangile du Royaume de Dieu, et le Nom de Jésus-Christ, tant les hommes que les femmes furent baptisés.
\VS{13}Et Simon crut aussi lui-même, et après avoir été baptisé, il ne quittait plus Philippe~; et voyant les prodiges et les grands miracles qui se faisaient, il était comme ravi hors de lui même.
\VS{14}Or quand les apôtres, qui étaient à Jérusalem eurent entendu que la Samarie avait reçu la parole de Dieu, ils leur envoyèrent Pierre et Jean~;
\VS{15}qui y étant descendus prièrent pour eux, afin qu'ils reçoivent le Saint-Esprit,
\VS{16}car il n'était pas encore descendu sur aucun d'eux, mais seulement ils étaient baptisés au Nom du Seigneur Jésus.
\VS{17}Puis ils leur imposèrent les mains, et ils reçurent le Saint-Esprit.
\VS{18}Lorsque Simon vit que le Saint-Esprit était donné par l'imposition des mains des apôtres, il leur présenta de l'argent,
\VS{19}en leur disant~: Donnez-moi aussi ce pouvoir, afin que tous ceux à qui j'imposerai les mains reçoivent le Saint-Esprit.
\VS{20}Mais Pierre lui dit~: Que ton argent périsse avec toi, puisque tu as estimé que le don de Dieu s'acquérait avec de l'argent.
\VS{21}Tu n'as point de part ni d'héritage en cette affaire~; car ton cœur n'est point droit devant Dieu.
\VS{22}Repens-toi donc de cette méchanceté, et prie Dieu, afin que, s'il est possible, la pensée de ton cœur te soit pardonnée.
\VS{23}Car je vois que tu es dans un fiel très amer et dans un lien d'iniquité.
\VS{24}Alors Simon répondit, et dit~: Vous, priez le Seigneur pour moi, afin qu'il ne m'arrive rien de ce que vous avez dit. 
\VS{25}Eux donc après avoir prêché et annoncé la parole du Seigneur, retournèrent à Jérusalem et annoncèrent l'Evangile dans plusieurs villages des Samaritains.
\TextTitle{Conversion et baptême de l'eunuque éthiopien}
\VS{26}Puis l'Ange du Seigneur parla à Philippe, en disant~: Lève-toi et va vers le Midi, sur le chemin qui descend de Jérusalem à Gaza, celui qui est désert.
\VS{27}Il se leva donc, et s'en alla. Et voici, un homme éthiopien, un eunuque, qui était un des principaux seigneurs de la cour de Candace, reine des Ethiopiens, et surintendant de toutes ses richesses, venu à Jérusalem pour adorer,
\VS{28}s'en retournait, assis dans son char, et lisait le prophète Esaïe.
\VS{29}L'Esprit dit à Philippe~: Avance, et approche-toi de ce char.
\VS{30}Philippe accourut et entendit l'Ethiopien qui lisait le prophète Esaïe~; et il lui dit~: Comprends-tu ce que tu lis~?
\VS{31}Et il lui dit~: Comment pourrais-je le comprendre, si quelqu'un ne me guide pas~? Et il pria Philippe de monter et s'asseoir avec lui.
\VS{32}Le passage de l'Ecriture qu'il lisait était celui-ci~: Il a été mené comme une brebis à la boucherie, et comme un agneau muet devant celui qui le tond~; en sorte qu'il n'a point ouvert sa bouche.
\VS{33}Dans son humiliation, son jugement a été levé~; mais qui racontera sa durée~? Car sa vie est retranchée de la terre\FTNT{\vref{Es. 53:7-8}.}.
\VS{34}Et l'eunuque prenant la parole, dit à Philippe~: Je te prie, de qui est-ce que le prophète dit cela~? Est-ce de lui-même, ou de quelque autre~?
\VS{35}Alors Philippe, ouvrant sa bouche, et commençant par cette Ecriture, lui annonça l'Evangile de Jésus.
\VS{36}Comme ils continuaient leur chemin, ils arrivèrent à un endroit où il y avait de l'eau. Et l'eunuque dit~: Voici de l'eau, qu'est-ce qui empêche que je ne sois baptisé~?
\VS{37}Philippe dit~: Si tu crois de tout ton cœur, cela t'est permis~; et l'eunuque répondit~: Je crois que Jésus-Christ est le Fils de Dieu.
\VS{38}Il fit arrêter le char~; Philippe et l'eunuque descendirent tous deux dans l'eau, et Philippe le baptisa.
\VS{39}Quand ils furent sortis de l'eau, l'Esprit du Seigneur enleva Philippe, et l'eunuque ne le vit plus. Tandis que tout joyeux il continua son chemin,
\VS{40}Philippe se trouva dans Azot, d'où il alla jusqu'à Césarée, en évangélisant toutes les villes par lesquelles il passait.
\Chap{9}
\TextTitle{Jésus se révèle à Saul\FTNTT{\vref{Ac. 22:1-16}~; \vref{26:9-18}}}
\VerseOne{}Or Saul, respirant encore la menace et le carnage contre les disciples du Seigneur, s'adressa au grand-prêtre,
\VS{2}et lui demanda des lettres de sa part pour les porter aux synagogues de Damas, afin que, s'il trouvait quelques-uns de cette secte, hommes ou femmes, il les amène liés à Jérusalem.
\VS{3}Or il arriva qu'en marchant, il approcha de Damas et tout à coup une lumière resplendit du ciel comme un éclair autour de lui.
\VS{4}Il tomba par terre et il entendit une voix qui lui disait~: Saul, Saul, pourquoi me persécutes-tu~?
\VS{5}Et il répondit~: Qui es-tu, Seigneur~? Et le Seigneur lui dit~: Je suis Jésus, que tu persécutes. Il te serait dur de regimber contre les aiguillons.
\VS{6}Alors, tout tremblant et tout effrayé, il dit~: Seigneur, que veux-tu que je fasse~? Et le Seigneur lui dit~: Lève-toi, et entre dans la ville, et on te dira ce que tu dois faire.
\VS{7}Les hommes qui l'accompagnaient s'arrêtèrent tout épouvantés, entendant bien la voix, mais ne voyant personne.
\VS{8}Et Saul se leva de terre, et ouvrant ses yeux, il ne voyait personne~; c'est pourquoi ils le conduisirent par la main, et le menèrent à Damas,
\VS{9}où il fut trois jours sans voir, sans manger ni boire.
\VS{10}Or il y avait à Damas un disciple, nommé Ananias, à qui le Seigneur dit en vision~: Ananias~! Et il répondit~: Me voici Seigneur~!
\VS{11}Et le Seigneur lui dit~: Lève-toi, va dans la rue appelée la droite, et cherche dans la maison de Judas un homme appelé Saul, de Tarse.
\VS{12}Car il prie. Or Saul avait vu en vision un homme appelé Ananias, entrant et lui imposant les mains, afin qu'il recouvre la vue. Et Ananias répondit~:
\VS{13} Seigneur, j'ai entendu parler plusieurs fois de cet homme-là~; et combien de maux il a faits à tes saints dans Jérusalem.
\VS{14}Il a même ici le pouvoir de la part des principaux prêtres, de lier tous ceux qui invoquent ton Nom.
\VS{15}Mais le Seigneur lui dit~: Va~; car il m'est un vase\FTNTT{Le mot «~vase~» vient du grec «~skeuos~». «~Vase~» était une métaphore grecque commune pour «~le corps~» car les Grecs pensaient que l'âme vivait temporairement dans les corps. \vref{2 Co. 4:7}~; \vref{Ro. 9:21-23}~; \vref{2 Ti. 2:20-21}.} que j'ai choisi, pour porter mon Nom devant les Gentils, et les rois, et les enfants d'Israël.
\VS{16}Car je lui montrerai combien il aura à souffrir pour mon Nom.
\TextTitle{Saul rempli du Saint-Esprit}
\VS{17}Ananias sortit~; et lorsqu'il fut arrivé dans la maison, il imposa les mains à Saul, et lui dit~: Saul mon frère, le Seigneur Jésus, qui t'est apparu sur le chemin par lequel tu venais, m'a envoyé, afin que tu recouvres la vue, et que tu sois rempli du Saint-Esprit.
\TextTitle{Saul est baptisé et évangélise Damas}
\VS{18}Et aussitôt il tomba de ses yeux comme des écailles~; et à l'instant il recouvra la vue. Puis il se leva, et fut baptisé.
\VS{19}Et ayant mangé, il reprit ses forces. Et Saul fut quelques jours avec les disciples qui étaient à Damas.
\VS{20}Et aussitôt il prêcha dans les synagogues que Jésus était le Fils de Dieu.
\VS{21}Et tous ceux qui l'entendaient, étaient comme ravis hors d'eux-mêmes, et ils disaient~: N'est-ce pas celui-là qui a détruit à Jérusalem ceux qui invoquaient ce Nom, et qui est venu ici exprès pour les amener liés aux principaux prêtres~?
\VS{22}Mais Saul se fortifiait de plus en plus, et confondait les Juifs qui habitaient à Damas, prouvant que Jésus était le Christ.
\TextTitle{Complot contre Saul}
\VS{23}Longtemps après, les Juifs conspirèrent ensemble pour le faire mourir~;
\VS{24}et leur complot parvint à la connaissance de Saul. Or ils gardaient les portes jour et nuit, afin de le faire mourir.
\VS{25}Mais pendant une nuit, les disciples le prirent, et le descendirent par la muraille dans une corbeille.
\TextTitle{Saul rencontre Barnaba et les apôtres à Jérusalem}
\VS{26}Lorsqu'il se rendit à Jérusalem, Saul tâcha de se joindre aux disciples~; mais tous le craignaient, ne croyant pas qu'il fût un disciple.
\VS{27}Alors Barnabas, l'ayant pris avec lui, le conduisit vers les apôtres, et leur raconta comment sur le chemin, Saul avait vu le Seigneur, qui lui avait parlé, et comment à Damas il parlait librement au Nom de Jésus.
\VS{28}Et il allait et venait avec eux dans Jérusalem, il parlait franchement au Nom du Seigneur, se montrant publiquement.
\VS{29}Et parlant sans déguisement au Nom du Seigneur Jésus, il disputait contre les Hellénistes, mais ils tentaient de le faire mourir.
\TextTitle{Retour à Tarse}
\VS{30}Les frères, l'ayant découvert, l'emmenèrent à Césarée, et le firent partir à Tarse.
\VS{31}Les églises étaient en paix dans toute la Judée, la Galilée, et la Samarie, étant édifiées et marchant dans la crainte du Seigneur~; et elles s'accroissaient par le rafraîchissement du Saint-Esprit.
\TextTitle{Guérison d'Enée, le paralytique}
\VS{32}Or il arriva que comme Pierre les visitait tous, il descendit aussi vers les saints qui demeuraient à Lydde.
\VS{33}Il y vint aussi un homme appelé Enée, qui était couché dans un petit lit depuis huit ans, car il était paralytique.
\VS{34}Et Pierre lui dit~: Enée, Jésus-Christ te guérit~! Lève-toi et arrange ton lit. Et aussitôt il se leva.
\VS{35}Tous ceux qui habitaient à Lydde et à Saron le virent, et ils se convertirent au Seigneur.
\TextTitle{Résurrection de Tabitha}
\VS{36}Il y avait à Joppé une femme disciple, appelée Tabitha, qui signifie en grec Dorcas~; elle faisait beaucoup de bonnes œuvres et d'aumônes.
\VS{37}Elle tomba malade en ce temps-là, et mourut. Après l'avoir lavée, on la déposa dans une chambre haute.
\VS{38}Comme Lydde était près de Joppé, les disciples ayant appris que Pierre était à Lydde, ils envoyèrent vers lui deux hommes, pour le prier de venir chez eux sans tarder.
\VS{39}Pierre se leva, et partit avec ces hommes. Lorsqu'il fut arrivé, on le conduisit dans la chambre haute. Toutes les veuves l'entourèrent en pleurant, et lui montrèrent les tuniques et les vêtements que faisait Dorcas quand elle était avec elles.
\VS{40}Pierre fit sortir tout le monde, se mit à genoux, et pria~; puis se tournant vers le corps, il dit~: Tabitha, lève-toi~! Et elle ouvrit ses yeux, et voyant Pierre, elle s'assit.
\VS{41}Il lui donna la main, et la fit lever. Puis ayant appelé les saints et les veuves, il la leur présenta vivante.
\VS{42}Cela fut connu dans tout Joppé~; et plusieurs crurent au Seigneur.
\VS{43}Et il arriva qu'il demeura plusieurs jours à Joppé, chez un corroyeur nommé Simon.
\Chap{10}
\TextTitle{Un ange de Dieu apparait à Corneille}
\VerseOne{}Il y avait à Césarée un homme nommé Corneille, centenier d'une cohorte de la légion appelée Italienne.
\VS{2}Cet homme était pieux et craignait Dieu avec toute sa famille. Il faisait aussi beaucoup d'aumônes au peuple, et priait Dieu continuellement.
\VS{3}Vers la neuvième heure du jour, il vit clairement dans une vision un ange de Dieu qui entra chez lui, et qui lui dit~: Corneille~!
\VS{4}Corneille ayant les yeux fixés sur lui, et tout effrayé, lui dit~: Qu'y a-t-il Seigneur~? Et il lui dit~: Tes prières et tes aumônes sont montées devant Dieu, et il s'en est souvenu.
\VS{5}Maintenant donc envoie des gens à Joppé, et fais venir Simon, surnommé Pierre.
\VS{6}Il est logé chez un certain Simon, corroyeur, qui a sa maison près de la mer~; c'est lui qui te dira ce qu'il faut que tu fasses.
\VS{7}Dès que l'ange qui lui parlait fut parti, Corneille appela deux de ses serviteurs, et un soldat craignant Dieu, d'entre ceux qui se tenaient près de lui.
\VS{8}Et après leur avoir tout raconté, il les envoya à Joppé.
\TextTitle{Vision de Pierre~: Une nappe descend du ciel}
\VS{9}Le lendemain, comme ils marchaient et qu'ils approchaient de la ville, Pierre monta sur le toit, vers la sixième heure, pour prier.
\VS{10}Et il arriva qu'ayant faim, il voulut prendre son repas. Pendant qu'on lui préparait à manger, il tomba en extase.
\VS{11}Il vit le ciel ouvert, et un vase descendant sur lui semblable à une grande nappe, attachée par les quatre coins, qui descendait vers la terre,
\VS{12}où se trouvaient tous les quadrupèdes, les bêtes sauvages, les reptiles et les oiseaux du ciel.
\VS{13}Et une voix lui dit~: Pierre, lève-toi, tue, et mange.
\VS{14}Mais Pierre répondit~: Non, Seigneur, car je n'ai jamais rien mangé de souillé ni d'impur.
\VS{15}Et la voix lui dit encore pour la seconde fois~: Les choses que Dieu a purifiées, ne les tiens point pour souillées.
\VS{16}Et cela arriva jusqu'à trois fois, et puis le vase fut retiré au ciel.
\VS{17}Comme Pierre ne savait pas en lui-même que penser du sens de la vision qu'il avait eue, voici, les hommes envoyés par Corneille s'étant mis en quête de la maison de Simon, se présentèrent à la porte,
\VS{18}et demandèrent à haute voix si c'était là que logeait Simon, surnommé Pierre.
\VS{19}Et comme Pierre pensait à la vision, l'Esprit lui dit~: Voici trois hommes qui te demandent.
\VS{20}Lève-toi donc et descends, et pars avec eux sans hésiter, car c'est moi qui les ai envoyés.
\VS{21}Pierre donc, descendit vers les gens qui lui avaient été envoyés par Corneille et leur dit~: Voici, je suis celui que vous cherchez~; pour quel sujet êtes-vous venus~?
\VS{22}Et ils dirent~: Corneille, centenier, homme juste et craignant Dieu, et à qui toute la nation des Juifs rend un bon témoignage, a été averti de Dieu par un saint ange de te faire venir dans sa maison et d'entendre tes paroles.
\TextTitle{Pierre chez Corneille}
\VS{23}Alors Pierre les fit entrer, et les logea. Le lendemain il s'en alla avec eux, et quelques-uns des frères de Joppé l'accompagnèrent.
\VS{24}Ils arrivèrent à Césarée le jour suivant. Corneille les attendait, et avait invité ses parents et ses amis.
\VS{25}Lorsque Pierre entra, Corneille qui était allé au-devant de lui, se jeta à ses pieds, et se prosterna.
\VS{26}Mais Pierre le releva en lui disant~: Lève-toi, moi aussi je suis un homme.
\VS{27}Et s'entretenant avec lui, il entra et trouva plusieurs personnes réunies.
\VS{28}Et il leur dit~: Vous savez qu'il n'est pas permis à un homme Juif de se lier avec un étranger, ou d'aller chez lui, mais Dieu m'a montré que je ne devais estimer aucun homme être impur ou souillé.
\VS{29}C'est pourquoi, ayant été appelé, je suis venu sans difficulté. Je vous demande donc pour quel sujet vous m'avez fait venir.
\VS{30}Corneille lui dit~: Il y a quatre jours, à cette heure-ci, j'étais en jeûne et en prière dans ma maison, et tout à coup, un homme, vêtu d'un habit resplendissant, se présenta devant moi et me dit~:
\VS{31}Corneille, ta prière est exaucée, et Dieu s'est souvenu de tes aumônes.
\VS{32}Envoie donc quelqu'un à Joppé, et fais venir Simon, surnommé Pierre, qui est logé dans la maison de Simon, le corroyeur, près de la mer. Quand il sera venu, il te parlera.
\VS{33}Aussitôt j'ai envoyé quelqu'un vers toi, et tu as bien fait de venir. Maintenant donc nous sommes tous présents devant Dieu pour entendre tout ce que Dieu t'a ordonné de nous dire.
\TextTitle{Pierre évangélise les Gentils\FTNTT{\vref{Ac. 2:14-41}}}
\VS{34}Alors Pierre prenant la parole, dit~: En vérité, je reconnais que Dieu n'a point égard à l'apparence des personnes,
\VS{35}mais qu'en toute nation celui qui le craint et qui pratique la justice, lui est agréable.
\VS{36}C'est ce qu'il a fait entendre aux enfants d'Israël, en leur annonçant la paix par Jésus-Christ, qui est le Seigneur de tous.
\VS{37}Vous savez ce qui est arrivé dans toute la Judée, après avoir commencé en Galilée, à la suite du baptême que Jean a prêché~;
\VS{38}vous savez comment Dieu a oint du Saint-Esprit et de force Jésus de Nazareth, qui allait de lieu en lieu, faisant du bien et guérissant tous ceux qui étaient sous l'empire du diable, car Dieu était avec lui.
\VS{39}Nous sommes témoins de toutes les choses qu'il a faites, dans le pays des Juifs et à Jérusalem. Cependant ils l'ont fait mourir en le pendant au bois.
\VS{40}Dieu l'a ressuscité le troisième jour, et il a permis qu'il apparaisse,
\VS{41}non à tout le peuple, mais aux témoins choisis d'avance par Dieu, à nous, qui avons mangé et bu avec lui après qu'il fut ressuscité des morts.
\VS{42}Et il nous a ordonné de prêcher au peuple, et d'attester que c'est lui qui a été établi par Dieu, juge des vivants et des morts.
\VS{43}Tous les prophètes rendent de lui le témoignage que quiconque croit en lui, reçoit la rémission de ses péchés par son Nom.
\TextTitle{Le Saint-Esprit descend sur les Gentils}
\VS{44}Comme Pierre prononçait encore ce discours, le Saint-Esprit descendit sur tous ceux qui écoutaient la parole.
\VS{45}Tous les fidèles circoncis qui étaient venus avec Pierre, furent étonnés de ce que le don du Saint-Esprit était aussi répandu sur les Gentils.
\VS{46}Car ils les entendaient parler diverses langues et glorifier Dieu.
\VS{47}Alors Pierre prenant la parole, dit~: Quelqu'un pourrait-il empêcher qu'on baptise dans l'eau ceux qui ont reçu le Saint-Esprit aussi bien que nous~?
\VS{48}Et il ordonna qu'ils soient baptisés au Nom du Seigneur. Après cela, ils le prièrent de rester quelques jours auprès d'eux.
\Chap{11}
\TextTitle{Dieu accorde la repentance aux Gentils}
\VerseOne{}Or les apôtres et les frères qui étaient en Judée apprirent que les Gentils aussi avaient reçu la parole de Dieu.
\VS{2}Et quand Pierre fut monté à Jérusalem, ceux de la circoncision disputaient contre lui,
\VS{3}disant~: Tu es entré chez des hommes incirconcis, et tu as mangé avec eux.
\VS{4}Alors Pierre commençant, leur exposa le tout par ordre, disant~:
\VS{5}J'étais dans la ville de Joppé, et pendant que je priais, je tombai en extase et j'eus une vision. Un vase semblable à une grande nappe, attachée par les quatre coins, descendit du ciel, et vint jusqu'à moi.
\VS{6}Les regards fixés sur cette nappe, j'examinai, et je vis les quadrupèdes, les bêtes sauvages, les reptiles, et les oiseaux du ciel.
\VS{7}Et j'entendis une voix qui me disait~: Pierre, lève-toi, tue, et mange.
\VS{8}Et je répondis~: Non Seigneur, car jamais rien de souillé ni d'impur n'est entré dans ma bouche.
\VS{9}La voix me parla du ciel une seconde fois~: Ce que Dieu a déclaré pur, ne le regarde pas comme souillé.
\VS{10}Cela arriva jusqu'à trois fois, puis toutes ces choses furent retirées dans le ciel.
\VS{11}Et voici, aussitôt trois hommes qui avaient été envoyés de Césarée vers moi, se présentèrent à la maison où j'étais.
\VS{12}L'Esprit me dit de partir avec eux sans hésiter. Les six frères que voici m'accompagnèrent, et nous entrâmes dans la maison de Corneille.
\VS{13}Cet homme nous raconta comment il avait vu dans sa maison un ange qui s'était présenté à lui, et lui avait dit~: Envoie des gens à Joppé, et fais venir Simon, surnommé Pierre,
\VS{14}qui te dira des choses par lesquelles tu seras sauvé, toi et toute ta maison.
\VS{15}Lorsque je me fus mis à parler, le Saint-Esprit descendit sur eux, comme il était descendu sur nous au commencement.
\VS{16}Et je me souvins de cette parole du Seigneur, et comment il avait dit~: Jean a baptisé d'eau, mais vous, vous serez baptisés du Saint-Esprit.
\VS{17}Or puisque Dieu leur a accordé le même don qu'à nous qui avons cru au Seigneur Jésus-Christ, pouvais-je, moi, m'opposer à Dieu~?
\VS{18}Après avoir entendu ces choses, ils s'apaisèrent, et ils glorifièrent Dieu en disant~: Dieu a donc accordé la repentance aussi aux Gentils, afin qu'ils aient la vie.
\TextTitle{Les disciples appelés «~chrétiens~» pour la première fois à Antioche}
\VS{19}Ceux qui avaient été dispersés par la persécution survenue à cause d'Etienne, allèrent jusqu'en Phénicie, dans l'île de Chypre, et à Antioche\FTNT{Antioche~: Capitale de la Syrie située sur le fleuve Oronte, fondée en 300 av. J.-C., et ainsi nommée en l'honneur de son fondateur Antiochus. De nombreux Juifs grecs y vivaient et c'est là que les disciples de Christ furent appelés pour la première fois, chrétiens.}, n'annonçant la parole à personne, seulement aux Juifs.
\VS{20}Mais il y eut parmi eux quelques hommes de Chypre et de Cyrène qui, étant venus à Antioche, parlèrent aussi aux Grecs, et leur annoncèrent l'Evangile du Seigneur Jésus.
\VS{21}La main du Seigneur était avec eux, et un grand nombre de personnes crurent et se convertirent au Seigneur.
\VS{22}Le bruit en parvint aux oreilles de l'Eglise qui était à Jérusalem, et ils envoyèrent Barnabas jusqu'à Antioche.
\VS{23}Lorsqu'il fut arrivé, et qu'il eut vu la grâce de Dieu, il s'en réjouit, et il les exhortait tous à demeurer attachés au Seigneur de tout leur cœur.
\VS{24}Car c'était un homme de bien, plein du Saint-Esprit et de foi. Et un grand nombre de personnes se joignirent au Seigneur.
\VS{25}Barnabas s'en alla à Tarse pour chercher Saul~;
\VS{26}et l'ayant trouvé, il l'amena à Antioche. Pendant toute une année, ils se réunirent aux assemblées de l'Eglise, et ils enseignèrent beaucoup de personnes. Ce fut à Antioche que, pour la première fois, les disciples furent appelés chrétiens.
\TextTitle{Prophétie d'Agabus}
\VS{27}En ce temps-là, quelques prophètes descendirent de Jérusalem à Antioche.
\VS{28}L'un d'eux, nommé Agabus, se leva et déclara par l'Esprit qu'une grande famine devait arriver sur toute la terre. Elle arriva, en effet, sous Claude César.
\VS{29}Les disciples résolurent d'envoyer, chacun selon ses moyens, quelque secours pour subvenir aux besoins des frères qui habitaient la Judée.
\VS{30}Ils le firent parvenir aux anciens par les mains de Barnabas et de Saul.
\Chap{12}
\TextTitle{Cinquième persécution de l'Eglise~: Meurtre de Jacques et arrestation de Pierre}
\VerseOne{}En ce même temps, le roi Hérode se mit à maltraiter quelques membres de l'Eglise~;
\VS{2}et il fit mourir par l'épée Jacques, frère de Jean.
\VS{3}Voyant que cela était agréable aux Juifs, il fit aussi arrêter Pierre. C'était pendant les jours des pains sans levain.
\VS{4}Après l'avoir saisi et jeté en prison, il le mit sous la garde de quatre bandes de quatre soldats chacune, avec l'intention de le faire comparaître devant le peuple après la fête de Pâque.
\TextTitle{L'Ange du Seigneur délivre Pierre de la prison}
\VS{5}Pierre était donc gardé dans la prison~; mais l'Eglise faisait sans cesse des prières à Dieu pour lui.
\VS{6}La nuit qui précéda le jour où Hérode devait l'envoyer au supplice, Pierre dormait entre deux soldats, lié de deux chaînes~; et les gardes qui étaient devant la porte gardaient la prison.
\VS{7}Et voici, l'Ange du Seigneur survint, et une lumière resplendit dans la prison. L'Ange réveilla Pierre en le frappant au côté, et en disant~: Lève-toi promptement~! Et les chaînes tombèrent de ses mains.
\VS{8}Et l'Ange lui dit~: Mets ta ceinture et tes sandales. Et il fit ainsi. L'Ange lui dit encore~: Enveloppe-toi de ton manteau et suis-moi.
\VS{9}Pierre sortit et le suivit, ne sachant pas que ce qui se faisait par l'Ange était réel, car il croyait qu'il avait une vision.
\VS{10}Lorsqu'ils eurent passé la première et la seconde garde, ils arrivèrent à la porte de fer qui mène à la ville, et qui s'ouvrit d'elle-même devant eux~; et ils sortirent et s'avancèrent dans une rue. Et subitement, l'Ange quitta Pierre.
\VS{11}Revenu à lui-même, Pierre dit~: Je vois à présent d'une manière certaine que le Seigneur a envoyé son Ange, et qu'il m'a délivré de la main d'Hérode, et de toute l'attente du peuple Juif.
\VS{12}Après avoir réfléchi, il alla à la maison de Marie, mère de Jean, surnommé Marc, où plusieurs personnes étaient assemblées et priaient.
\VS{13}Il frappa à la porte du vestibule, une servante, appelée Rhode, vint pour écouter.
\VS{14}Elle reconnut la voix de Pierre, et dans sa joie elle n'ouvrit pas la porte du vestibule, mais elle courut dans la maison et annonça que Pierre était devant la porte.
\VS{15}Ils lui dirent~: Tu es folle. Mais elle affirma que ce qu'elle disait était vrai.
\VS{16}Et ils dirent~: C'est son ange. Cependant Pierre continuait à frapper. Et quand ils eurent ouvert, ils le virent, et furent étonnés de le voir.
\VS{17}Mais leur ayant fait signe de la main de se taire, il leur raconta comment le Seigneur l'avait fait sortir de la prison, et il leur dit~: Annoncez ces choses à Jacques et aux frères. Puis sortant de là il s'en alla dans un autre lieu.
\VS{18}Quand il fit jour, les soldats furent dans une grande agitation, pour savoir ce que Pierre était devenu.
\VS{19}Et Hérode l'ayant cherché, et ne le trouvant point, après en avoir fait le procès aux gardes, il commanda qu'ils fussent menés au supplice.
\TextTitle{Mort d'Hérode}
\VS{20}Hérode avait le dessein de faire la guerre aux Tyriens et aux Sidoniens~; mais ils vinrent le trouver d'un commun accord~; et ayant gagné Blaste, son Chambellan, ils demandèrent la paix, parce que leur pays tirait sa subsistance de celui du roi.
\VS{21}A un jour marqué, Hérode, revêtu de ses habits royaux, s'assit sur son trône et les harangua publiquement.
\VS{22}Le peuple s'écria~: Voix d'un dieu et non point d'un homme~!
\VS{23}Et à l'instant l'Ange du Seigneur le frappa, parce qu'il n'avait pas donné gloire à Dieu. Et il expira, rongé des vers.
\VS{24}Cependant la parole de Dieu se répandait de plus en plus, et le nombre des disciples augmentait.
\VS{25}Barnabas et Saul, après s'être acquittés de leur service, s'en retournèrent de Jérusalem, ayant aussi pris avec eux Jean, surnommé Marc.
\Chap{13}
\TextTitle{Saul et Barnabas mis à part par le Saint-Esprit}
\VerseOne{}Or il y avait dans l'église qui était à Antioche des prophètes et des docteurs, Barnabas, Siméon, appelé Niger, Lucius, le Cyrénien, Manahen, qui avait été élevé avec Hérode, le tétrarque, et Saul.
\VS{2}Et tandis qu'ils servaient\FTNT{Certains traducteurs ont rajouté la phrase «~dans leur ministère~» alors que les textes originaux ne la mentionne pas.} le Seigneur et jeûnaient, le Saint-Esprit dit~: Séparez-moi maintenant Barnabas et Saul pour l'œuvre à laquelle je les ai appelés.
\VS{3}Alors après avoir jeûné et prié, ils leur imposèrent les mains, et les laissèrent partir\FTNT{Voir annexe «~Les voyages missionnaires de Paul~».}.
\TextTitle{Saul, Barnabas et Jean sur l'île de Chypre}
\VS{4}Barnabas et Saul, envoyés par le Saint-Esprit, descendirent à Séleucie, et de là ils s'embarquèrent pour l'île de Chypre.
\VS{5}Et lorsqu'ils furent à Salamine, ils annoncèrent la parole de Dieu dans les synagogues des Juifs~; ils avaient Jean avec eux pour les aider.
\TextTitle{Bar-Jésus aveuglé et conversion du proconsul Sergius Paulus}
\VS{6}Ayant ensuite traversé l'île jusqu'à Paphos, ils trouvèrent là un certain magicien, faux prophète Juif, nommé Bar-Jésus,
\VS{7}qui était avec le proconsul Sergius Paulus, homme intelligent qui fit appeler Barnabas et Saul, désirant entendre la parole de Dieu.
\VS{8}Mais Elymas, le magicien, car c'est ce que signifie ce nom, leur résistait, cherchant à détourner de la foi le proconsul.
\VS{9}Alors Saul, appelé aussi Paul, rempli du Saint-Esprit, fixa les yeux sur lui et dit~:
\VS{10}Ô homme plein de toute fraude et de toute ruse, fils du diable, ennemi de toute justice, ne cesseras-tu point de renverser les voies droites du Seigneur~?
\VS{11}C'est pourquoi, voici la main du Seigneur est sur toi, tu seras aveugle, et pour un temps tu ne verras pas le soleil. Aussitôt l'obscurité et les ténèbres tombèrent sur lui, et il cherchait, en tâtonnant, des personnes pour le guider.
\VS{12}Alors le proconsul voyant ce qui était arrivé, crut, étant rempli d'admiration pour la doctrine du Seigneur.
\VS{13}Et quand Paul et ceux qui étaient avec lui furent partis de Paphos, ils vinrent à Perge, ville de Pamphylie. Jean se sépara d'eux et retourna à Jérusalem.
\TextTitle{Paul à Antioche de Pisidie~: Le salut par la foi en Jésus}
\VS{14}De Perge, ils poursuivirent leur route, et arrivèrent à Antioche, ville de Pisidie\FTNT{Antioche de Pisidie~: Ville de Pisidie (en Turquie), à la frontière de Phrygie, fondée par Seleucus Nicanor. Elle devint une colonie romaine et fut aussi appelée Césarée.}, et étant entrés dans la synagogue le jour du sabbat, ils s'assirent.
\VS{15}Après la lecture de la loi et des prophètes, les chefs de la synagogue leur envoyèrent dire~: Hommes frères, si vous avez quelque parole d'exhortation pour le peuple, dites-la.
\VS{16}Alors Paul s'étant levé, et ayant fait signe de la main qu'on fasse silence, dit~: Hommes Israélites, et vous qui craignez Dieu, écoutez.
\VS{17}Le Dieu de ce peuple d'Israël a choisi nos pères. Il a distingué glorieusement ce peuple pendant son séjour au pays d'Egypte, et il l'en fit sortir par son bras élevé.
\VS{18}Il les supporta\FTNT{Le verbe supporter vient du grec «~tropophoreo~» qui signifie supporter les manières, endurer le caractère de quelqu'un.} au désert environ quarante ans.
\VS{19}Et ayant détruit sept nations au pays de Canaan, il leur distribua le pays par le sort.
\VS{20}Après cela, durant quatre cent cinquante ans, il leur donna des juges, jusqu'à Samuel le prophète.
\VS{21}Puis ils demandèrent un roi, et Dieu leur donna Saül fils de Kis, homme de la tribu de Benjamin~; et ainsi se passèrent quarante ans.
\VS{22}Et Dieu l'ayant rejeté, il leur suscita pour roi David, auquel il a rendu ce témoignage~: J'ai trouvé David, fils d'Isaï, homme selon mon cœur, qui exécutera toute ma volonté.
\VS{23}C'est de la postérité de David que Dieu, selon sa promesse, a suscité Jésus pour être le Sauveur d'Israël.
\VS{24}Avant la venue de Jésus, Jean avait prêché le baptême de repentance à tout le peuple d'Israël.
\VS{25}Et comme Jean achevait sa course, il disait~: Qui pensez-vous que je sois~? Je ne suis point le Christ~; mais voici, il en vient un après moi, dont je ne suis pas digne de délier le soulier de ses pieds.
\VS{26}Hommes frères, fils de la race d'Abraham, et vous qui craignez Dieu, c'est à vous que la parole de ce salut a été envoyée.
\VS{27}Car les habitants de Jérusalem et leurs chefs ont méconnu Jésus, et en le condamnant, ils ont accompli les paroles des prophètes qui se lisent chaque sabbat.
\VS{28}Quoiqu'ils n'aient rien trouvé en lui qui soit digne de mort, ils demandèrent à Pilate de le faire mourir.
\VS{29}Et après qu'ils eurent accompli toutes les choses qui avaient été écrites de lui, ils le descendirent du bois, et le déposèrent dans un sépulcre.
\VS{30}Mais Dieu l'a ressuscité des morts.
\VS{31}Il est apparu pendant plusieurs jours à ceux qui étaient montés avec lui de Galilée à Jérusalem, et qui sont ses témoins devant le peuple.
\VS{32}Et nous, nous vous annonçons cette bonne nouvelle que la promesse faite à nos pères,
\VS{33}Dieu l'a accomplie pour nous, leurs enfants, en ressuscitant Jésus, selon qu'il est écrit dans le deuxième psaume~: Tu es mon Fils, je t'ai aujourd'hui engendré\FTNT{\vref{Ps. 2:7}.}.
\VS{34}Et pour montrer qu'il l'a ressuscité des morts, pour ne plus devoir retourner au sépulcre, il a dit ainsi~: Je vous donnerai les grâces saintes promises à David, ces grâces qui sont assurées.
\VS{35}C'est pourquoi il a dit aussi dans un autre endroit~: Tu ne permettras point que ton Saint voie la corruption\FTNT{\vref{Ps. 16:10}.}.
\VS{36}Or David, après avoir servi en son temps au dessein de Dieu, est mort, a été réuni à ses pères, et a vu la corruption.
\VS{37}Mais celui que Dieu a ressuscité n'a pas vu la corruption.
\VS{38}Sachez donc, hommes frères, que c'est par lui que la rémission des péchés vous est annoncée,
\VS{39}et que quiconque croit est justifié par lui, de tout ce dont vous n'avez pas pu être justifiés par la loi de Moïse.
\VS{40}Prenez donc garde qu'il ne vous arrive ce qui est dit dans les prophètes~:
\VS{41}Voyez, vous mépriseurs, soyez étonnés et disparaissez~: Car je vais faire une œuvre en votre temps, une œuvre que vous ne croiriez pas si quelqu'un vous la racontait.
\VS{42}Lorsqu'ils sortirent de la synagogue des Juifs, les Gentils les prièrent de parler le sabbat suivant sur les mêmes choses.
\VS{43}Et quand l'assemblée fut séparée, beaucoup de Juifs et de prosélytes craignant Dieu, suivirent Paul et Barnabas qui les exhortèrent à persévérer dans la grâce de Dieu.
\TextTitle{Les juifs d'Antioche rejettent la Parole~; l'Évangile annoncé aux Gentils\FTNTT{\vref{Ac. 18:6}~; \vref{28:25-28}.}}
\VS{44}Le sabbat suivant, presque toute la ville s'assembla pour entendre la parole de Dieu.
\VS{45}Mais les Juifs voyant toute cette foule, furent remplis de jalousie, et ils s'opposaient à ce que Paul disait, en le contredisant et en blasphémant.
\VS{46}Alors Paul et Barnabas leur dirent avec assurance~: C'est à vous premièrement qu'il fallait annoncer la parole de Dieu, mais puisque vous la rejetez, et que vous vous jugez vous-mêmes indignes de la vie éternelle, voici, nous nous tournons vers les Gentils.
\VS{47}Car ainsi nous l'a ordonné le Seigneur~: Je t'ai établi pour être la lumière des Gentils, pour porter le salut jusqu'aux extrémités de la terre.
\VS{48}Les Gentils en entendant cela, se réjouissaient et ils glorifiaient la parole du Seigneur~; et tous ceux qui étaient destinés à la vie éternelle crurent.
\VS{49}Ainsi la parole du Seigneur se répandait dans tout le pays.
\VS{50}Mais les Juifs excitèrent quelques femmes dévotes et distinguées, et les principaux de la ville, et ils provoquèrent une persécution contre Paul et Barnabas, et les chassèrent de leur territoire.
\VS{51}Paul et Barnabas secouèrent contre eux la poussière de leurs pieds et allèrent à Icone,
\VS{52}tandis que les disciples étaient remplis de joie et du Saint-Esprit.
\Chap{14}
\TextTitle{Paul et Barnabas à Icone}
\VerseOne{}A Icone, Paul et Barnabas entrèrent ensemble dans la synagogue des Juifs, et ils parlèrent d'une telle manière qu'une grande multitude de Juifs et de Grecs crurent.
\VS{2}Mais ceux des Juifs qui furent rebelles, émurent et irritèrent les esprits des Gentils contre les frères.
\VS{3}Ils restèrent cependant assez longtemps à Icone, parlant avec assurance du Seigneur, qui rendait témoignage à la parole de sa grâce, en faisant par leurs mains des prodiges et des miracles.
\VS{4}La population de la ville fut partagée en deux, et les uns étaient du côté des Juifs, et les autres du côté des apôtres.
\TextTitle{Paul prêche à Derbe et à Lystre~; guérison d'un boiteux de naissance}
\VS{5}Et comme il se faisait une émeute des Gentils et des Juifs, avec leurs principaux chefs, pour outrager et lapider les apôtres,
\VS{6}Paul et Barnabas en ayant eu connaissance, se réfugièrent dans les villes de Lycaonie, à Lystre, à Derbe, et dans les contrées d'alentour.
\VS{7}Et ils y annoncèrent l'Evangile.
\VS{8}A Lystre, se tenait assis un homme impotent des pieds, boiteux dès sa naissance, et qui n'avait jamais marché.
\VS{9}Cet homme écoutait parler Paul. Et Paul fixant ses yeux sur lui, et voyant qu'il avait la foi pour être guéri,
\VS{10}lui dit à haute voix~: Lève-toi droit sur tes pieds. Et il se leva en sautant, et marcha.
\VS{11}Et les gens qui étaient là assemblés, ayant vu ce que Paul avait fait, élevèrent leur voix, disant en langue lycaonienne~: Les dieux sous une forme humaine, sont descendus vers nous.
\VS{12}Et ils appelaient Barnabas Jupiter, et Paul Mercure, parce que c'était lui qui portait la parole.
\VS{13}Le prêtre de Jupiter, qui était à l'entrée de leur ville, ayant amené des taureaux et des couronnes jusqu'à l'entrée de la porte voulait, de même que la foule, offrir un sacrifice.
\VS{14}Mais les apôtres Barnabas et Paul ayant appris cela, déchirèrent leurs vêtements et se précipitèrent au milieu de la foule,
\VS{15}et disant~: Ô hommes, pourquoi faites-vous cela~? Nous aussi, nous sommes des hommes, sujets aux mêmes passions que vous, et vous apportant l'Evangile, nous vous exhortons à renoncer à ces choses vaines, pour vous convertir au Dieu vivant, qui a fait le ciel et la terre, la mer, et tout ce qui s'y trouve.
\VS{16}Ce Dieu, dans les siècles passés, a laissé toutes les nations marcher dans leurs voies,
\VS{17}quoiqu'il n'ait cessé de rendre témoignage de ce qu'il est, en faisant du bien, en nous dispensant du ciel les pluies et les saisons fertiles, en nous donnant la nourriture avec abondance, et en remplissant nos cœurs de joie.
\VS{18}A peine purent-ils, par ces paroles, empêcher la foule de leur offrir un sacrifice.
\TextTitle{Paul lapidé à Lystre}
\VS{19}Alors survinrent quelques Juifs d'Antioche et d'Icone qui gagnèrent la foule, et qui après avoir lapidé Paul, le traînèrent hors de la ville, croyant qu'il était mort.
\VS{20}Mais les disciples s'étant assemblés autour de lui, il se leva et entra dans la ville~; et le lendemain il s'en alla avec Barnabas à Derbe.
\TextTitle{Vote et établissement des anciens dans les églises}
\VS{21}Quand ils eurent évangélisé cette ville, et fait un certain nombre de disciples, ils retournèrent à Lystre, à Icone, et à Antioche~;
\VS{22}fortifiant l'esprit des disciples, et les exhortant à persévérer dans la foi, disant que c'est par beaucoup de tribulations qu'il nous faut entrer dans le Royaume de Dieu.
\VS{23}Après le vote à main levée des assemblées, ils établirent des anciens dans chaque église, et après avoir prié et jeûné, ils les recommandèrent au Seigneur, en qui ils avaient cru.
\VS{24}Traversant ensuite la Pisidie, ils allèrent en Pamphylie,
\VS{25}annoncèrent la parole à Perge, et descendirent à Attalie.
\TextTitle{Retour à Antioche}
\VS{26}De là, ils s'embarquèrent pour Antioche, d'où ils avaient été recommandés à la grâce de Dieu, pour l'œuvre qu'ils venaient d'accomplir.
\VS{27}Et quand ils furent arrivés, ils convoquèrent l'église, et ils racontèrent toutes les choses que Dieu avait faites par eux, et comment il avait ouvert aux Gentils la porte de la foi.
\VS{28}Et ils demeurèrent assez longtemps avec les disciples.
\Chap{15}
\TextTitle{Des hommes venus de Judée veulent imposer la circoncision}
\VerseOne{}Quelques hommes qui étaient descendus de Judée, enseignaient les frères en disant~: Si vous n'êtes pas circoncis selon le rite de Moïse, vous ne pouvez pas être sauvés.
\VS{2}Paul et Barnabas eurent avec eux un débat et une vive discussion~; et les frères décidèrent que Paul et Barnabas, avec quelques-uns des leurs, monteraient à Jérusalem vers les apôtres et les anciens, pour traiter cette question.
\VS{3}Après avoir été accompagnés par l'assemblée, ils traversèrent la Phénicie et la Samarie, racontant la conversion des Gentils~; et ils causèrent une grande joie à tous les frères.
\VS{4}Arrivés à Jérusalem, ils furent reçus par l'église, les apôtres et les anciens, et ils racontèrent toutes les choses que Dieu avait faites par leur moyen.
\VS{5}Mais quelques-uns, de la secte des pharisiens qui avaient cru, se levèrent, en disant qu'il fallait circoncire les Gentils et leur ordonner de garder la loi de Moïse.
\TextTitle{Opposition de Pierre~: Les Gentils n'ont pas à être sous le joug de la loi}
\VS{6}Alors les apôtres et les anciens se réunirent pour examiner cette affaire.
\VS{7}Et après une grande discussion, Pierre se leva et leur dit~: Hommes frères, vous savez que depuis longtemps Dieu m'a choisi parmi nous, afin que par ma bouche, les Gentils entendent la parole de l'Evangile, et qu'ils croient.
\VS{8}Et Dieu, qui connaît les cœurs, leur a rendu témoignage en leur donnant le Saint-Esprit, de même qu'à nous.
\VS{9}Il n'a fait aucune différence entre nous et eux, ayant purifié leurs cœurs par la foi.
\VS{10}Maintenant donc pourquoi tentez-vous Dieu en voulant imposer aux disciples un joug que ni nos pères ni nous n'avons pu porter~?
\VS{11}Mais nous croyons que nous serons sauvés par la grâce du Seigneur Jésus-Christ, comme eux aussi.
\VS{12}Alors toute l'assemblée garda le silence, et l'on écouta Barnabas et Paul qui racontèrent tous les miracles et les prodiges que Dieu avait faits par leur moyen au milieu des Gentils.
\TextTitle{Discours de Jacques~: Les prophètes ont annoncé le salut pour les Gentils} 
\VS{13}Lorsqu'ils eurent cessé de parler, Jacques prit la parole et dit~: Hommes frères, écoutez-moi~!
\VS{14}Simon a raconté comment Dieu a premièrement jeté les regards sur les nations pour choisir du milieu d'elles un peuple consacré à son Nom. 
\VS{15}Et avec cela s'accordent les paroles des prophètes, selon qu'il est écrit~:
\VS{16}Après cela, je reviendrai, et je rebâtirai le tabernacle de David qui est tombé, je réparerai ses ruines et je le relèverai\FTNT{\vref{Am. 9:11}.}.
\VS{17}Afin que le reste des hommes recherche le Seigneur, et aussi toutes les nations sur lesquelles mon Nom est invoqué, dit le Seigneur, qui fait toutes ces choses.
\VS{18}Toutes les œuvres de Dieu lui sont connues de toute éternité.
\TextTitle{Les chrétiens issus des nations ne sont pas soumis à la loi mosaïque}
\VS{19}C'est pourquoi je suis d'avis qu'on ne crée pas des difficultés à ceux des Gentils qui se convertissent à Dieu~;
\VS{20}mais qu'on leur écrive de s'abstenir des souillures des idoles et de la fornication, des animaux étouffés et du sang.
\VS{21}Car depuis bien des générations, Moïse a, dans chaque ville, des gens qui le prêchent, puisqu'on le lit tous les jours de sabbat dans les synagogues.
\VS{22}Alors il parut bon aux apôtres et aux anciens avec toute l'Eglise, de choisir parmi eux et d'envoyer à Antioche avec Paul et Barnabas, Jude, appelé Barsabas, et Silas, hommes considérés entre les frères.
\VS{23}Ils écrivirent par eux en ces termes~: Les apôtres, les anciens, et les frères, aux frères d'entre les Gentils qui sont à Antioche, en Syrie, et en Cilicie, salut~!
\VS{24}Ayant appris que quelques hommes partis de chez nous, et auxquels nous n'avons donné aucun ordre, vous ont troublés par leurs discours et ont ébranlé vos âmes, en vous disant qu'il faut être circoncis et garder la loi,
\VS{25}nous avons été d'avis, étant assemblés tous d'un commun accord, d'envoyer vers vous, avec nos très chers Barnabas et Paul, des hommes que nous avons choisis. 
\VS{26}Ce sont des hommes qui ont abandonné leurs vies pour le Nom de notre Seigneur Jésus-Christ.
\VS{27}Nous avons donc envoyé Jude et Silas, qui vous feront entendre les mêmes choses de vive voix.
\VS{28}Car il a paru bon au Saint-Esprit et à nous, de ne vous imposer d'autre charge que ce qui est nécessaire,
\VS{29}savoir, de vous abstenir des viandes sacrifiées aux idoles, du sang, des animaux étouffés, et de la fornication~; choses contre lesquelles vous vous trouverez bien de vous tenir en garde. Adieu~!
\TextTitle{Mission de Jude et Silas à Antioche}
\VS{30}Après avoir donc pris congé de l'église, ils allèrent à Antioche, et ayant assemblé l'église, ils remirent la lettre.
\VS{31}Après l'avoir lue, les frères d'Antioche furent réjouis de la consolation qu'elle leur apportait.
\VS{32}Jude et Silas, qui étaient eux-mêmes prophètes, exhortèrent les frères par plusieurs discours, et les fortifièrent.
\VS{33}Au bout de quelque temps, ils furent renvoyés en paix par les frères vers les apôtres.
\VS{34}Toutefois Silas trouva bon de rester.
\VS{35}Et Paul et Barnabas demeurèrent aussi à Antioche, enseignant et annonçant, avec plusieurs autres, la parole du Seigneur.
\TextTitle{Paul et Barnabas se séparent}
\VS{36}Quelques jours après, Paul dit à Barnabas~: Retournons visiter nos frères dans toutes les villes où nous avons annoncé la parole du Seigneur, pour voir quel est leur état\FTNT{Voir annexe «~Les voyages missionaires de Paul~».}.
\VS{37}Barnabas voulait emmener avec eux Jean, surnommé Marc.
\VS{38}Mais Paul jugea plus convenable de ne pas prendre avec eux celui qui les avait quittés depuis la Pamphylie, et qui ne les avait point accompagnés dans leur œuvre.
\VS{39}Il y eut donc entre eux une contestation, en sorte qu'ils se séparèrent l'un de l'autre. Barnabas, prenant Marc avec lui, s'embarqua pour l'île de Chypre.
\VS{40}Mais Paul, ayant choisi Silas, pour l'accompagner, partit après avoir été recommandé à la grâce de Dieu par les frères.
\VS{41}Il traversa la Syrie et la Cilicie, fortifiant les églises.
\Chap{16}
\TextTitle{Circoncis, Timothée rejoint Paul dans la mission}
\VerseOne{}Il se rendit à Derbe et à Lystre, et voici, il y avait là un disciple, nommé Timothée, fils d'une femme juive fidèle et d'un père grec.
\VS{2}Les frères de Lystre et d'Icone rendaient de lui un bon témoignage.
\VS{3}C'est pourquoi Paul voulut l'emmener avec lui~; et l'ayant pris, il le circoncit, à cause des Juifs qui étaient dans ces lieux-là, car ils savaient tous que son père était grec.
\VS{4}En passant par les villes, ils recommandaient aux frères d'observer les ordonnances établies par les apôtres et les anciens de Jérusalem.
\VS{5}Ainsi les églises étaient affermies dans la foi et augmentaient en nombre chaque jour.
\TextTitle{Vision de Paul}
\VS{6}Ayant traversé la Phrygie et le pays de Galatie, le Saint-Esprit leur défendit d'annoncer la parole dans l'Asie.
\VS{7}Arrivés près de la Mysie, ils se disposaient à entrer en Bithynie~; mais l'Esprit de Jésus\FTNT{Notons que le Saint-Esprit est appelé Esprit de Jésus. Ainsi, de la même manière qu'on ne peut dissocier un homme de son esprit pour en faire deux entités distinctes, on ne peut dissocier Jésus de son Esprit. Dieu est un.} ne le leur permit pas.
\VS{8}Ils traversèrent ensuite la Mysie, et descendirent à Troas.
\VS{9}Pendant la nuit, Paul eut une vision d'un homme macédonien qui se présenta devant lui, et le pria, disant~: Passe en Macédoine et secours-nous~!
\VS{10}Après cette vision de Paul, nous cherchâmes aussitôt à nous rendre en Macédoine, concluant que le Seigneur nous appelait à les évangéliser.
\TextTitle{Paul à Philippes}
\VS{11}Ainsi étant partis de Troas, nous fîmes voile directement vers la Samothrace, et le lendemain à Néapolis.
\VS{12}De là nous allâmes à Philippes, qui est la première ville d'un district de Macédoine, et une colonie romaine. Nous séjournâmes quelque temps dans la ville.
\VS{13}Et le jour du sabbat nous sortîmes de la ville, et allâmes au lieu où on avait accoutumé de faire la prière, près du fleuve, et nous étant là assis nous parlâmes aux femmes qui y étaient assemblées.
\TextTitle{Conversion de Lydie}
\VS{14}L'une d'elles, appelée Lydie, marchande de pourpre, de la ville de Thyatire, était une femme craignant Dieu, et elle nous écoutait. Le Seigneur lui ouvrit le cœur, afin qu'elle soit attentive à ce que disait Paul.
\VS{15}Lorsqu'elle eut été baptisée, avec sa famille, elle nous fit cette demande~: Si vous me jugez fidèle au Seigneur, entrez dans ma maison, et demeurez-y. Et elle nous pressa par ses instances.
\TextTitle{Paul et Silas battus de verges et mis en prison}
\VS{16}Or il arriva que comme nous allions à la prière, une servante qui avait un esprit de python, et qui, en devinant, apportait un grand profit à ses maîtres, nous rencontra,
\VS{17}et elle se mit à nous suivre, Paul et nous, en criant et disant~: Ces hommes sont les serviteurs du Dieu Très-Haut, et ils vous annoncent la voie du salut~!
\VS{18}Elle fit cela pendant plusieurs jours. Mais Paul, fatigué, se retourna et dit à l'esprit~: Je t'ordonne au Nom de Jésus-Christ de sortir de cette fille. Et il sortit au même instant.
\VS{19}Mais les maîtres de la servante voyant disparaître l'espoir de leur gain, se saisirent de Paul et de Silas, et les traînèrent sur la place publique devant les magistrats.
\VS{20}Ils les présentèrent aux préteurs, en disant~: Ces hommes, qui sont Juifs, troublent notre ville.
\VS{21}Car ils annoncent des coutumes qu'il ne nous est pas permis de recevoir ni de suivre, à nous qui sommes Romains.
\VS{22}La foule se souleva aussi contre eux, et les préteurs, ayant fait déchirer leurs vêtements, ordonnèrent qu'ils soient battus de verges.
\VS{23}Après qu'on les eut chargés de coups de fouet, ils les mirent en prison, en recommandant au geôlier de les garder sûrement.
\VS{24}Le geôlier ayant reçu cet ordre, les mit au fond de la prison, et leur serra les pieds dans des ceps.
\TextTitle{Libération miraculeuse de Paul et Silas}
\VS{25}Vers minuit, Paul et Silas priaient et chantaient les louanges de Dieu, et les prisonniers les entendaient.
\VS{26}Tout à coup, il se fit un grand tremblement de terre, en sorte que les fondements de la prison furent ébranlés~; au même instant, toutes les portes s'ouvrirent et les liens de tous furent rompus.
\VS{27}Le geôlier se réveilla, et voyant les portes de la prison ouvertes, il tira son épée et allait se tuer, croyant que les prisonniers s'étaient enfuis.
\VS{28}Mais Paul cria d'une voix forte~: Ne te fais point de mal, nous sommes tous ici.
\TextTitle{Conversion et baptême du geôlier et de sa famille}
\VS{29}Alors le geôlier, ayant demandé de la lumière, entra précipitamment dans le cachot, et se jeta tout tremblant aux pieds de Paul et de Silas.
\VS{30}Il les fit sortir, et dit~: Seigneur, que faut-il que je fasse pour être sauvé~?
\VS{31}Paul et Silas répondirent~: Crois au Seigneur Jésus-Christ et tu seras sauvé, toi et ta famille.
\VS{32}Et ils lui annoncèrent la parole du Seigneur, et à tous ceux qui étaient dans sa maison.
\VS{33}Après cela, les prenant en cette même heure de la nuit, il lava leurs plaies, et aussitôt après il fut baptisé, avec tous ceux de sa maison.
\VS{34}Les ayant amenés dans sa maison, il leur servit à manger, et il se réjouit avec toute sa famille de ce qu'il avait cru en Dieu.
\TextTitle{Paul et Silas relâchés}
\VS{35}Quand il fit jour, les préteurs envoyèrent des huissiers pour dire au geôlier~: Relâche ces hommes.
\VS{36}Et le geôlier rapporta ces paroles à Paul, disant~: Les préteurs ont envoyé dire qu'on vous relâche~; maintenant donc sortez, et allez en paix.
\VS{37}Mais Paul dit aux huissiers~: Après nous avoir battus de verges publiquement et sans jugement, nous qui sommes Romains, ils nous ont jetés en prison, et maintenant ils nous font sortir secrètement~! Il n'en sera pas ainsi. Qu'ils viennent eux-mêmes nous mettre en liberté.
\VS{38}Les licteurs rapportèrent ces paroles aux préteurs qui furent effrayés en apprenant qu'ils étaient Romains.
\VS{39}Ils vinrent vers eux et leur firent des excuses, et ils les mirent en liberté en les priant de quitter la ville.
\VS{40}Quand ils furent sortis de la prison, ils entrèrent chez Lydie, et après avoir vu et consolé les frères, ils partirent.
\Chap{17}
\TextTitle{Paul et Silas à Thessalonique}
\VerseOne{}Paul et Silas passèrent par Amphipolis et par Apollonie, et ils arrivèrent à Thessalonique, où les Juifs avaient une synagogue.
\VS{2}Paul y entra, selon sa coutume. Pendant trois sabbats, il discuta avec eux d'après les Ecritures~;
\VS{3}expliquant et établissant que le Christ devait souffrir et ressusciter des morts. Et ce Jésus, que je vous annonce, disait-il, c'est lui qui est le Christ.
\VS{4}Quelques-uns d'entre eux crurent, et se joignirent à Paul et à Silas, ainsi qu'une grande multitude de Grecs craignant Dieu, et beaucoup de femmes de qualité.
\TextTitle{Emeute à Thessalonique}
\VS{5}Mais les Juifs rebelles et jaloux, prirent avec eux quelques hommes méchants et fainéants de la populace, provoquèrent des attroupements, et répandirent l'agitation dans la ville. Ils se rendirent à la maison de Jason, et ils cherchèrent Paul et Silas, pour les amener vers le peuple.
\VS{6}Ne les ayant pas trouvés, ils traînèrent Jason et quelques frères devant les magistrats de la ville, en criant~: Ces gens, qui ont bouleversé le monde, sont aussi venus ici, et Jason les a reçus chez lui.
\VS{7}Ils sont tous rebelles aux édits de César, disant qu'il y a un autre Roi, qu'ils nomment Jésus.
\VS{8}Ils soulevèrent donc le peuple et les magistrats de la ville, qui, entendant ces choses,
\VS{9}ne laissèrent aller Jason et les autres qu'après avoir obtenu d'eux une caution. 
\TextTitle{Paul et Silas fuient à Bérée}
\VS{10}Aussitôt les frères firent partir de nuit Paul et Silas pour Bérée. Lorsqu'ils furent arrivés, ils entrèrent dans la synagogue des Juifs.
\VS{11}Ces Juifs avaient des sentiments plus nobles que ceux de Thessalonique~; ils reçurent la parole avec beaucoup de promptitude, et ils examinaient tous les jours les Ecritures, pour voir si ce qu'on leur disait était exact.
\VS{12}Plusieurs d'entre eux crurent, ainsi que des femmes grecques de distinction, et des hommes en assez grand nombre.
\VS{13}Mais quand les Juifs de Thessalonique surent que Paul annonçait aussi à Bérée la parole de Dieu, ils vinrent y agiter la foule.
\VS{14}Alors les frères firent aussitôt partir Paul du côté de la mer~; Silas et Timothée restèrent à Bérée.
\VS{15}Ceux qui avaient pris la charge de mettre Paul en sûreté, le conduisirent jusqu'à Athènes. Puis ils s'en retournèrent, après avoir reçu l'ordre de Paul de dire à Silas et à Timothée de le rejoindre au plus tôt.
\TextTitle{Paul à Athènes}
\VS{16}Comme Paul les attendait à Athènes, il sentit au-dedans son esprit s'irriter à la vue de cette ville entièrement adonnée à l'idolâtrie.
\VS{17}Il s'entretenait donc dans la synagogue avec les Juifs et les hommes craignant Dieu, et tous les jours sur la place publique avec ceux qui s'y rencontraient.
\VS{18}Quelques philosophes épicuriens\FTNT{L'épicurisme a été fondé par Epicure (341 av. J.-C. - 270 av. J.-C.). Cette philosophie est axée sur la recherche du bonheur par l'évitement de la souffrance et des inquiétudes (ataraxie).} et stoïciens\FTNT{Les stoïciens étaient disciples de Zénon (336-264 av. J.-C.). Leur philosophie se fondait sur la conception d'un homme se suffisant à lui-même, sur une discipline rigoureuse, et sur la solidarité du genre humain.} se mirent à parler avec lui. Et les uns disaient~: Que veut dire ce discoureur~? Les autres disaient~: Il semble qu'il annonce des divinités étrangères. Parce qu'il leur annonçait Jésus et la résurrection.
\VS{19}Alors ils le prirent et le menèrent à l'Aréopage\FTNT{A l'origine, l'aréopage désignait le tribunal d'Athènes qui siégeait sur la colline d'Arès. Le sens figuré est le suivant~: Assemblée de juges, de savants, d'hommes de lettres très compétents.}, et lui dirent~: Pourrions-nous savoir quelle est cette nouvelle doctrine que tu enseignes~?
\VS{20}Car tu nous remplis les oreilles de certaines choses étranges~; nous voudrions donc savoir ce que veulent dire ces choses.
\VS{21}Or tous les Athéniens et les étrangers qui demeuraient à Athènes, ne passaient leur temps qu'à dire ou à écouter des nouvelles.
\TextTitle{Prédication de Paul à l'Aréopage}
\VS{22}Paul, debout au milieu de l'Aréopage, leur dit~: Hommes Athéniens, je vous trouve à tous égards extrêmement religieux.
\VS{23}Car en passant et en regardant vos divinités, j'ai même trouvé un autel sur lequel était écrit~: Au Dieu inconnu~! Celui que vous révérez sans le connaître, c'est celui que je vous annonce.
\VS{24}Le Dieu qui a fait le monde et tout ce qui s'y trouve, étant le Seigneur du ciel et de la terre, n'habite point dans des temples faits de main d'homme.
\VS{25}Il n'est point servi par les mains des hommes, comme s'il avait besoin de quoi que ce soit, lui qui donne à tous la vie, la respiration, et toutes choses.
\VS{26}Il a fait que tous les hommes, sortis d'un seul sang, habitent sur toute l'étendue de la terre, ayant déterminé la durée des temps et les bornes de leur habitation.
\VS{27}Il a voulu qu'ils cherchent le Seigneur, et qu'ils s'efforcent de le trouver en tâtonnant, quoiqu'il ne soit pas loin de chacun de nous,
\VS{28}car c'est par lui que nous avons la vie, le mouvement et l'être. C'est ce qu'ont dit quelques-uns même de vos poètes~: De lui nous sommes la race.
\VS{29}Ainsi donc, étant de la race de Dieu, nous ne devons pas croire que la divinité soit semblable à de l'or, ou à de l'argent, ou à de la pierre taillée par l'art et l'industrie des hommes.
\VS{30}Mais Dieu, sans tenir compte des temps d'ignorance, annonce maintenant à tous les hommes en tous lieux qu'ils se repentent,
\VS{31}parce qu'il a arrêté un jour où il jugera le monde selon la justice, par l'homme qu'il a établi pour cela, ce dont il a donné à tous une preuve certaine, en le ressuscitant des morts.
\VS{32}Lorsqu'ils entendirent parler de la résurrection des morts, les uns se moquèrent, et les autres dirent~: Nous t'entendrons là-dessus une autre fois.
\VS{33}Ainsi Paul se retira du milieu d'eux.
\VS{34}Quelques-uns néanmoins se joignirent à lui et crurent~: Denys, juge de l'Aéropage, une femme nommée Damaris, et d'autres avec eux.
\Chap{18}
\TextTitle{Paul enseigne à Corinthe pendant un an et demi}
\VerseOne{}Après cela, Paul partit d'Athènes, et se rendit à Corinthe.
\VS{2}Il y trouva un Juif, nommé Aquilas, originaire du Pont, récemment arrivé d'Italie, avec Priscille, sa femme, parce que Claude avait ordonné à tous les Juifs de sortir de Rome. Il s'approcha d'eux,
\VS{3}et comme il était du même métier qu'eux, il demeura chez eux et y travailla. Et leur métier était de faire des tentes.
\VS{4}Paul discourait dans la synagogue chaque sabbat, et il persuadait des Juifs et des Grecs.
\VS{5}Quand Silas et Timothée furent arrivés de Macédoine, Paul étant poussé par l'Esprit, rendait témoignage aux Juifs que Jésus était le Christ.
\VS{6}Mais comme ils s'opposaient à lui et qu'ils blasphémaient, il secoua ses vêtements, et leur dit~: Que votre sang retombe sur votre tête~! J'en suis pur~! Dès maintenant, j'irai vers les Gentils.
\VS{7}Et sortant de là, il entra dans la maison d'un homme appelé Justus, homme craignant Dieu, et dont la maison était contiguë à la synagogue.
\VS{8}Cependant Crispus, le chef de la synagogue, crut au Seigneur avec toute sa famille. Et plusieurs Corinthiens qui avaient entendu Paul, crurent aussi, et ils furent baptisés.
\VS{9}Le Seigneur dit à Paul dans une vision pendant la nuit~: Ne crains point, mais parle et ne te tais point,
\VS{10}parce que je suis avec toi, et personne ne mettra la main sur toi pour te faire du mal. Parle, car j'ai un peuple nombreux dans cette ville.
\VS{11}Il y demeura un an et six mois, enseignant parmi eux la parole de Dieu.
\TextTitle{Soulèvement des Juifs contre Paul}
\VS{12}Pendant que Gallion était proconsul de l'Achaïe, les Juifs se soulevèrent d'un commun accord contre Paul, et le menèrent devant le tribunal,
\VS{13}en disant~: Cet homme incite les gens à servir Dieu d'une manière contraire à la loi.
\VS{14}Et comme Paul voulait ouvrir la bouche pour parler, Gallion dit aux Juifs~: Ô Juifs~! S'il s'agissait de quelque injustice, ou de quelque crime, je vous écouterais patiemment, autant qu'il serait raisonnable.
\VS{15}Mais il s'agit de discussions sur une parole, sur des noms, et sur votre loi, vous y mettrez de l'ordre vous-mêmes, car je ne veux pas être juge de ces choses.
\VS{16}Et il les renvoya du tribunal.
\VS{17}Alors tous les Grecs se saisirent de Sosthène, le chef de la synagogue, le battirent devant le tribunal, sans que Gallion s'en mît en peine.
\TextTitle{Paul fait un vœu\FTNTT{\vref{Ga. 3:23-28}~; \vref{2 Co. 3:7-14}~; \vref{Ro. 6:14}}}
\VS{18}Paul resta encore assez longtemps à Corinthe. Ensuite il prit congé des frères et s'embarqua pour la Syrie, avec Priscille et Aquilas, après s'être fait raser la tête à Cenchrées, car il avait fait un vœu.
\VS{19}Ils arrivèrent à Ephèse, et Paul y laissa ses compagnons. Etant entré dans la synagogue, il s'entretint avec les Juifs,
\VS{20}qui le prièrent de rester encore plus longtemps avec eux.
\VS{21}Mais il n'y consentit point, et il prit congé d'eux en leur disant~: Il faut absolument que je célèbre la fête prochaine à Jérusalem. Je reviendrai vers vous, s'il plaît à Dieu. Ainsi il partit d'Ephèse.
\VS{22}Etant débarqué à Césarée, il monta à Jérusalem, et après avoir salué l'église, il descendit à Antioche.
\VS{23}Et ayant séjourné là quelque temps, il s'en alla, et traversa tout de suite la contrée de Galatie et de Phrygie, fortifiant tous les disciples\FTNT{Voir annexe «~Les voyages missionnaires de Paul~»}.
\TextTitle{Apollos annonce l'Evangile à Ephèse et à Corinthe}
\VS{24}En ce temps-là, un Juif, nommé Apollos, originaire d'Alexandrie, homme éloquent et puissant dans les Ecritures, vint à Ephèse.
\VS{25}Il était en quelque sorte instruit dans la voie du Seigneur, et fervent d'esprit~; il expliquait et enseignait avec exactitude ce qui concerne Jésus, bien qu'il ne connaisse que le baptême de Jean.
\VS{26}Il commança donc à parler avec hardiesse dans la synagogue~; et quand Aquilas et Priscille l'eurent entendu, ils le prirent avec eux, et lui exposèrent plus exactement la voie de Dieu.
\VS{27}Et comme il voulut passer en Achaïe, les frères, qui l'y encouragèrent écrivirent aux disciples de bien le recevoir. Quand il fut arrivé, il aida beaucoup ceux qui avaient cru par la grâce.
\VS{28}Car il réfutait publiquement les Juifs avec une grande véhémence, démontrant par les Ecritures que Jésus était le Christ.
\Chap{19}
\TextTitle{Paul enseigne à Ephèse\FTNTT{v. \vref{9-10}~; \vref{Ac. 20:31}}}
\VerseOne{}Pendant qu'Apollos était à Corinthe, Paul, après avoir parcouru toutes les hautes provinces de l'Asie, arriva à Ephèse. Ayant rencontré quelques disciples, il leur dit~:
\VS{2}Avez-vous reçu le Saint-Esprit quand vous avez cru~? Ils lui répondirent~: Nous n'avons même pas entendu dire qu'il y ait un Saint-Esprit.
\VS{3}Et il leur dit~: De quel baptême donc avez-vous été baptisés~? Ils répondirent~: Du baptême de Jean.
\VS{4}Alors Paul dit~: Il est vrai que Jean a baptisé du baptême de repentance, disant au peuple de croire en celui qui venait après lui, c'est-à-dire en Jésus-Christ.
\VS{5}Après avoir entendu ces choses, ils furent baptisés au Nom du Seigneur Jésus.
\VS{6}Lorsque Paul leur eut imposé les mains, le Saint-Esprit descendit sur eux, et ils parlaient diverses langues et prophétisaient.
\VS{7}Ils étaient en tout environ douze hommes.
\VS{8}Ensuite, Paul entra dans la synagogue où il parla librement. Pendant trois mois, il discourut sur les choses qui concernent le Royaume de Dieu avec persuasion.
\VS{9}Mais comme quelques-uns restaient endurcis et rebelles, décriant devant la multitude la voie du Seigneur, il se retira d'eux, sépara les disciples, et enseigna tous les jours dans l'école d'un nommé Tyrannus.
\VS{10}Cela dura deux ans, de sorte que tous ceux qui habitaient l'Asie, Juifs et Grecs, entendirent la parole du Seigneur Jésus.
\TextTitle{Réveil et prodiges à Ephèse}
\VS{11}Et Dieu faisait des prodiges extraordinaires par les mains de Paul,
\VS{12}au point qu'on appliquait sur les malades des mouchoirs ou des linges qui avaient touché son corps, et ils étaient guéris de leurs maladies, et les esprits malins sortaient.
\VS{13}Alors quelques exorcistes Juifs ambulants essayèrent d'invoquer le Nom du Seigneur Jésus sur ceux qui étaient possédés d'esprits malins, en disant~: Nous vous conjurons par ce Jésus que Paul prêche~!
\VS{14}Ceux qui faisaient cela étaient sept fils de Scéva, un homme Juif, l'un des principaux prêtres.
\VS{15}Mais l'esprit malin leur répondit~: Je connais Jésus, et je sais qui est Paul~; mais vous, qui êtes-vous~?
\VS{16}Et l'homme dans lequel était l'esprit malin se jeta sur eux, se rendit maître de deux d'entre eux, et les maltraita de telle sorte qu'ils s'enfuirent de cette maison nus et blessés.
\VS{17}Cela fut connu de tous les Juifs et de tous les Grecs qui demeuraient à Ephèse~; et ils furent tous saisis de crainte, et le Nom du Seigneur Jésus était glorifié.
\VS{18}Plusieurs de ceux qui avaient cru venaient, confessant et déclarant ce qu'ils avaient fait.
\VS{19}Et un grand nombre de ceux qui s'étaient adonnés à des pratiques magiques, apportèrent leurs livres et les brûlèrent devant tous. On en estima la valeur à cinquante mille pièces d'argent.
\VS{20}Ainsi la parole du Seigneur se répandait sensiblement, et produisait de grands effets.
\VS{21}Après que ces choses se furent passées, Paul se proposa par un mouvement de l'Esprit\FTNT{Paul fut conduit par le Saint-Esprit (\vref{Jn. 3:8}).} d'aller à Jérusalem, en traversant la Macédoine et l'Achaïe. Quand j'y serai allé, se disait-il, il faut aussi que je voie Rome.
\VS{22}Il envoya en Macédoine deux de ceux qui l'assistaient, Timothée et Eraste, et il resta lui-même quelque temps en Asie.
\TextTitle{Emeute suscitée par Démétrius}
\VS{23}Mais en ce temps-là il arriva un grand trouble, à cause de la doctrine.
\VS{24}Car un certain homme, nommé Démétrius, orfèvre, fabriquait de petits temples d'argent de Diane, et apportait beaucoup de profit aux ouvriers du métier.
\VS{25}Il les rassembla, avec ceux du même métier, et dit~: Ô hommes, vous savez que tout notre gain vient de cet ouvrage,
\VS{26}et vous voyez et entendez que, non seulement à Ephèse, mais dans presque toute l'Asie, ce Paul par ses persuasions a détourné beaucoup de monde, en disant que les dieux faits de main d'homme ne sont pas des dieux.
\VS{27}Et il n'y a pas seulement à craindre pour nous que notre métier ne soit décrié, mais même que le temple de la grande Diane ne tombe dans le mépris, et que sa majesté, que toute l'Asie et que le monde entier révère, ne soit anéantie.
\VS{28}Ayant entendu ces choses, ils furent tous remplis de colère, et s'écrièrent, disant~: Grande est la Diane des Ephésiens~!
\VS{29}Et toute la ville fut remplie de confusion~; et ils se jetèrent en foule dans le théâtre, et enlevèrent Gaïus et Aristarque Macédoniens, compagnons de voyage de Paul.
\VS{30}Et comme Paul voulait entrer vers le peuple, les disciples ne le lui permirent point.
\VS{31}Quelques-uns même des Asiarques, qui étaient ses amis, envoyèrent quelqu'un vers lui pour le prier de ne pas se présenter au théâtre.
\VS{32}Les uns criaient d'une manière, les autres d'une autre, car l'assemblée était confuse, et la plupart ne savaient pas pourquoi ils s'étaient assemblés.
\VS{33}Alors Alexandre fut contraint de sortir hors de la foule, les Juifs le poussant en avant~; et Alexandre, faisant signe de la main, voulait présenter quelque excuse au peuple.
\VS{34}Mais quand ils reconnurent qu'il était Juif, tous d'une seule voix crièrent pendant deux heures~: Grande est la Diane des Ephésiens~!
\VS{35}Cependant, le secrétaire de la ville, ayant apaisé la foule, dit~: Hommes éphésiens, quel est celui des hommes qui ignore que la ville d'Ephèse est la gardienne de la grande déesse Diane et de son image tombée de Jupiter\FTNT{Tombée de Jupiter~: C'est-à-dire du ciel.}~?
\VS{36}Cela étant donc incontestable, vous devez vous apaiser et ne rien faire avec précipitation.
\VS{37}Car ces gens que vous avez amenés ne sont ni sacrilèges ni blasphémateurs de votre déesse.
\VS{38}Mais si Démétrius et ses ouvriers ont à se plaindre de quelqu'un, il y a des jours d'audience et des proconsuls~; qu'ils s'appellent en justice les uns les autres~!
\VS{39}Et si vous avez quelque autre chose à réclamer, on pourra en décider dans une assemblée légale.
\VS{40}Car nous risquons d'être accusés de sédition pour ce qui s'est passé aujourd'hui, n'ayant aucune raison pour justifier ce rassemblement. Après ces paroles, il congédia l'assemblée.
\Chap{20}
\TextTitle{Paul annonce l'Evangile en Macédoine et en Grèce}
\VerseOne{}Lorsque le tumulte eut cessé, Paul fit venir les disciples, et après les avoir embrassés, il partit pour aller en Macédoine.
\VS{2}Il parcourut cette contrée, en adressant aux disciples de nombreuses exhortations.
\VS{3}Puis il se rendit en Grèce où il séjourna trois mois. Il était sur le point de s'embarquer pour la Syrie, quand les Juifs lui dressèrent des embûches. Alors il se décida à reprendre la route de la Macédoine.
\VS{4}Il avait pour l'accompagner jusqu'en Asie~: Sopater de Bérée, Aristarque et Second de Thessalonique, Gaïus de Derbe, Timothée, ainsi que Tychique et Trophime, originaires d'Asie.
\VS{5}Ceux-ci prirent les devants et nous attendirent à Troas.
\TextTitle{Paul ressuscite un jeune homme à Troas}
\VS{6}Et nous, ayant levé l'ancre à Philippes, après les jours des pains sans levain, nous arrivâmes au bout de cinq jours auprès d'eux à Troas, et nous y séjournâmes sept jours.
\VS{7}Le premier jour de la semaine, les disciples étant assemblés pour rompre le pain, Paul, qui devait partir le lendemain, leur fit un discours qu'il étendit jusqu'à minuit.
\VS{8}Or il y avait beaucoup de lampes dans la chambre haute où ils étaient assemblés.
\VS{9}Et un jeune homme nommé Eutychus, qui était assis sur une fenêtre, s'endormit profondément pendant le long discours de Paul~; entraîné par le sommeil, il tomba du troisième étage en bas, et quand on voulut le relever, il était mort.
\VS{10}Mais Paul, étant descendu, se pencha sur lui, le prit dans ses bras, et dit~: Ne vous troublez pas, car son âme est en lui.
\VS{11}Quand il fut remonté, il rompit le pain et mangea, et il parla longtemps encore jusqu'au jour. Après quoi il partit.
\VS{12}Ils ramenèrent le jeune homme vivant, et ce fut le sujet d'une grande consolation.
\TextTitle{Passage à Milet}
\VS{13}Pour nous, étant montés sur un navire, nous fîmes voile vers Assos, où nous avions convenu de reprendre Paul, parce qu'il devait faire la route à pied.
\VS{14}Lorsqu'il nous eut rejoints à Assos, nous le prîmes avec nous, et nous allâmes à Mytilène.
\VS{15}Puis étant partis de là, le jour suivant nous abordâmes vis-à-vis de Chios. Le lendemain, nous arrivâmes vers Samos, et nous nous arrêtâmes à Trogyle~; le jour d'après, nous vînmes à Milet.
\VS{16}Car Paul avait résolu de passer devant Ephèse sans s'y arrêter, afin de ne pas perdre de temps en Asie~; parce qu'il se hâtait pour être, si cela lui était possible, à Jérusalem le jour de la Pentecôte.
\TextTitle{Paul exhorte et prend congé des anciens d'Ephèse}
\VS{17}Cependant de Milet, il envoya chercher à Ephèse les anciens de l'Eglise.
\VS{18}Lorsqu'ils furent arrivés vers lui, il leur dit~: Vous savez de quelle manière je me suis toujours conduit avec vous dès le premier jour où je suis entré en Asie~;
\VS{19}servant le Seigneur en toute humilité, avec beaucoup de larmes, et au milieu des épreuves que me suscitaient les embûches des Juifs.
\VS{20}Vous savez que je n'ai rien caché de ce qui vous était utile, et que je n'ai pas craint de vous prêcher et de vous enseigner publiquement et dans les maisons,
\VS{21}prêchant tant aux Juifs qu'aux Grecs la repentance envers Dieu, et la foi en Jésus-Christ, notre Seigneur.
\VS{22}Et maintenant voici, étant lié par l'Esprit, je vais à Jérusalem, ignorant ce qui m'y arrivera~;
\VS{23}seulement, de ville en ville le Saint-Esprit m'avertit que des liens et des tribulations m'attendent.
\VS{24}Mais je ne fais pour moi-même aucun cas de ma vie, comme si elle m'était précieuse, pourvu que j'achève ma course avec joie, et le service que j'ai reçu du Seigneur Jésus, pour rendre témoignage à l'Evangile de la grâce de Dieu.
\VS{25}Et maintenant voici, je sais que vous ne verrez plus mon visage, vous tous au milieu desquels j'ai passé en prêchant le Royaume de Dieu.
\VS{26}C'est pourquoi je vous prends aujourd'hui à témoin que je suis net du sang de tous.
\VS{27}Car je vous ai annoncé tout le conseil de Dieu, sans en rien cacher.
\VS{28}Prenez donc garde à vous-mêmes, et à tout le troupeau sur lequel le Saint-Esprit vous a établis évêques\FTNT{Evêque, «~episcopos~» en grec~: surveillant, gardien. Ce terme désigne la fonction des anciens. Dans la Nouvelle Alliance, les évêques (ou anciens) sont des personnes dont la mission est de veiller au bon fonctionnement des assemblées locales. Jésus-Christ, notre Dieu, est l'Evêque par excellence (\vref{1 Pi. 2:25}).}, pour paître l'Eglise de Dieu, qu'il a acquise par son propre sang.
\VS{29}Car je sais qu'après mon départ, il s'introduira parmi vous des loups très dangereux, qui n'épargneront pas le troupeau,
\VS{30}et qu'il se lèvera du milieu de vous des hommes qui enseigneront des doctrines corrompues dans le but d'attirer les disciples après eux.
\VS{31}C'est pourquoi veillez, vous souvenant que durant l'espace de trois ans, je n'ai cessé nuit et jour d'avertir chacun de vous avec larmes.
\VS{32}Et maintenant, mes frères, je vous recommande à Dieu, et à la parole de sa grâce, à celui qui est puissant pour achever de vous édifier, et pour vous donner l'héritage avec tous les saints.
\VS{33}Je n'ai désiré ni l'argent, ni l'or, ni les vêtements de personne.
\VS{34}Et vous savez vous-mêmes que ces mains ont pourvu à mes besoins et à ceux des personnes qui étaient avec moi.
\VS{35}Je vous ai montré de toutes manières que c'est en travaillant ainsi qu'il faut soutenir les faibles, et se rappeler les paroles du Seigneur Jésus, qui a dit lui-même~: Il y a plus de bénédiction à donner qu'à recevoir\FTNT{\vref{Lu. 14:12}.}.
\VS{36}Après avoir ainsi parlé, il se mit à genoux et il pria avec eux tous.
\VS{37}Alors tous fondirent en larmes, et se jetant au cou de Paul,
\VS{38}ils l'embrassèrent, étant principalement affligés de ce qu'il avait dit qu'ils ne verraient plus son visage. Et ils l'accompagnèrent jusqu'au navire.
\Chap{21}
\TextTitle{L'équipe missionnaire à Tyr~; avertissement de l'Esprit}
\VerseOne{}Nous nous embarquâmes, après nous être séparés d'eux, et nous allâmes directement à Cos, et le jour suivant à Rhodes, et de là à Patara.
\VS{2}Et ayant trouvé un navire qui faisait la traversée vers la Phénicie, nous montâmes et partîmes.
\VS{3}Puis ayant découvert l'île de Chypre, nous la laissâmes à gauche, nous fîmes route vers la Syrie, nous arrivâmes à Tyr, car le navire devait y décharger sa cargaison.
\VS{4}Nous trouvâmes les disciples et nous restâmes là sept jours. Les disciples, poussés par l'Esprit, disaient à Paul de ne pas monter à Jérusalem.
\VS{5}Mais ces jours étant passés, nous partîmes et nous nous acheminâmes pour partir de Tyr, et tous nous accompagnèrent avec leurs femmes et leurs enfants, jusqu'à l'extérieur de la ville. Nous nous mîmes à genoux sur le rivage et nous fîmes la prière.
\VS{6}Et après nous être embrassés les uns les autres, nous montâmes sur le navire, et les autres retournèrent chez eux.
\TextTitle{Escales à Ptolémaïs puis à Césarée~; prophétie d'Agabus}
\VS{7}Et ainsi achevant notre navigation, nous allâmes de Tyr à Ptolémaïs~; et après avoir salué les frères, nous passâmes un jour avec eux.
\VS{8}Nous partîmes le lendemain, et nous arrivâmes à Césarée. Etant entrés dans la maison de Philippe, l'évangéliste, qui était l'un des sept, nous restâmes chez lui.
\VS{9}Il avait quatre filles vierges qui prophétisaient.
\VS{10}Comme nous étions là depuis plusieurs jours, un prophète, nommé Agabus, arriva de Judée
\VS{11}et vint nous trouver. Il prit la ceinture de Paul, se lia les mains et les pieds, et il dit~: Voici ce que déclare le Saint-Esprit~: L'homme à qui appartient cette ceinture, les Juifs le lieront de la même manière à Jérusalem, et le livreront entre les mains des Gentils.
\VS{12}Quand nous entendîmes ces choses, nous et ceux de l'endroit, nous priâmes Paul de ne pas monter à Jérusalem.
\VS{13}Mais Paul répondit~: Que faites-vous en pleurant et en affligeant mon cœur~? Je suis prêt, non seulement à être lié, mais aussi à mourir à Jérusalem pour le Nom du Seigneur Jésus.
\VS{14}Comme il ne se laissait pas persuader, nous n'insistâmes pas, et nous dîmes~: Que la volonté du Seigneur soit faite~!
\TextTitle{QUATRIEME VOYAGE~: DE JERUSALEM A ROME}
\TextTitle{Arrivée à Jérusalem~; accueil des anciens}
\VS{15}Quelques jours après, nous fîmes nos préparatifs et nous montâmes à Jérusalem \FTNT{Voir annexe «~Les voyages missionnaires de Paul~»}.
\VS{16}Quelques disciples de Césarée vinrent avec nous, amenant avec eux un homme appelé Mnason, de l'île de Chypre, disciple de longue date, chez qui nous devions loger.
\VS{17}Lorsque nous arrivâmes à Jérusalem, les frères nous reçurent avec joie.
\VS{18}Et le jour suivant, Paul se rendit avec nous chez Jacques, et tous les anciens s'y réunirent.
\VS{19}Après les avoir embrassés, il raconta en détail les choses que Dieu avait faites au milieu des Gentils par son service.
\VS{20}Quand ils l'eurent entendu, ils glorifièrent le Seigneur. Puis ils dirent à Paul~: Tu vois frère, combien de milliers de Juifs ont cru~; mais ils sont tous zélés pour la loi.
\VS{21}Or ils ont appris que tu enseignes à tous les Juifs qui sont parmi les Gentils, à renoncer à Moïse, en leur disant qu'ils ne doivent pas circoncire leurs enfants et de ne pas vivre selon les ordonnances de la loi.
\VS{22}Que faut-il donc faire~? Il faut absolument rassembler la multitude des fidèles, car ils apprendront que tu es venu.
\VS{23}C'est pourquoi fais ce que nous allons te dire~: Nous avons quatre hommes qui ont fait un vœu,
\VS{24}prends-les avec toi, purifie-toi avec eux, et pourvois à leurs besoins afin qu'ils se rasent la tête. Et ainsi tous sauront que ce qu'ils ont entendu sur ton compte est faux, mais que toi aussi tu te conduis en observateur de la loi.
\VS{25}A l'égard des Gentils qui ont cru, nous avons décidé et nous leur avons écrit qu'ils doivent s'abstenir des viandes sacrifiées aux idoles, du sang, des animaux étouffés, et de la débauche.
\VS{26}Alors Paul prit ces hommes, se purifia, et entra le lendemain dans le temple avec eux, pour annoncer quel jour leur purification devait s'achever, et quand l'offrande devait être présentée pour chacun d'eux.
\TextTitle{Paul chassé du temple et brutalisé par les Juifs}
\VS{27}A la fin des sept jours, les Juifs d'Asie ayant vu Paul dans le temple, soulevèrent tout le peuple, et mirent la main sur lui,
\VS{28}en criant~: Hommes israélites, au secours~! Voici l'homme qui prêche partout et à tout le monde contre le peuple, contre la loi, et contre ce lieu. Il a même introduit des Grecs dans le temple, et a profané ce saint lieu.
\VS{29}Car ils avaient vu auparavant Trophime d'Ephèse avec lui dans la ville, et ils croyaient que Paul l'avait fait entrer dans le temple.
\VS{30}Toute la ville fut émue, et le peuple accourut de toutes parts. Ils se saisirent de Paul et le traînèrent hors du temple, dont les portes furent aussitôt fermées.
\TextTitle{Intervention des soldats et des centeniers}
\VS{31}Comme ils cherchaient à le tuer, le bruit vint au tribun de la cohorte que tout Jérusalem était en trouble.
\VS{32}A l'instant, il prit des soldats et des centeniers, et courut vers eux. Voyant le tribun et les soldats, ils cessèrent de frapper Paul.
\VS{33}Alors le tribun s'approcha, se saisit de Paul, et le fit lier de deux chaînes. Puis il demanda qui il était et ce qu'il avait fait.
\VS{34}Les uns criaient d'une manière, et les autres d'une autre, dans la foule. Ne pouvant donc rien apprendre de certain à cause du tumulte, il ordonna de mener Paul dans la forteresse.
\VS{35}Lorsque Paul fut sur les degrés, il dut être porté par les soldats, à cause de la violence de la foule~;
\VS{36}car la multitude du peuple le suivait, en criant~: Fais-le mourir~!
\VS{37}Comme on allait faire entrer Paul dans la forteresse, il dit au tribun~: M'est-il permis de te dire quelque chose~? Et le tribun répondit~: Tu sais parler le grec~?
\VS{38}Tu n'es donc pas cet Egyptien qui a excité une sédition dernièrement, et qui a emmené dans le désert quatre mille brigands~?
\VS{39}Paul lui dit~: Je suis Juif de Tarse, citoyen de la ville renommée de la Cilicie. Permets-moi, je te prie, de parler au peuple.
\VS{40}Et quand il le lui permit, Paul se tenant sur les degrés fit signe de la main au peuple, et s'étant fait un grand silence, il leur parla en langue hébraïque, disant~:
\Chap{22}
\TextTitle{Paul raconte son témoignage de conversion\FTNTT{\vref{Ac. 9:1-18}~; \vref{26:9-18}.}}
\VerseOne{}Hommes frères et pères, écoutez ce que j'ai maintenant à vous dire pour ma défense~!
\VS{2}Lorsqu'ils entendirent qu'il leur parlait en langue hébraïque, ils redoublèrent de silence. Et Paul leur dit~:
\VS{3}Je suis Juif, né à Tarse en Cilicie~; mais j'ai été élevé dans cette ville-ci aux pieds de Gamaliel et instruit dans la connaissance exacte de la loi de nos pères, étant plein de zèle pour la loi de Dieu, comme vous l'êtes tous aujourd'hui.
\VS{4}J'ai persécuté à mort cette doctrine, liant et mettant en prison hommes et femmes.
\VS{5}Le grand-prêtre lui-même et toute l'assemblée des anciens m'en sont témoins. J'ai même reçu d'eux des lettres pour les frères de Damas, où je me rendis afin d'amener liés à Jérusalem ceux qui se trouvaient là et de les faire punir.
\VS{6}Or il arriva comme j'étais en chemin, que j'approchais de Damas, tout à coup, vers midi, une grande lumière venant du ciel resplendit comme un éclair autour de moi.
\VS{7}Je tombai par terre, et j'entendis une voix qui me dit~: Saul, Saul, pourquoi me persécutes-tu~?
\VS{8}Je répondis~: Qui es-tu Seigneur~? Et il me dit~: Je suis Jésus de Nazareth, que tu persécutes.
\VS{9}Ceux qui étaient avec moi furent tout effrayés, ils virent bien la lumière, mais ils ne comprirent pas la voix de celui qui me parlait. Alors je dis~: Que ferai-je Seigneur~?
\VS{10}Et le Seigneur me dit~: Lève-toi, va à Damas, et là on te dira tout ce que tu dois faire.
\VS{11}Comme je ne voyais rien, à cause de l'éclat de cette lumière, ceux qui étaient avec moi me prirent par la main, et j'arrivai à Damas.
\VS{12}Or un nommé Ananias, homme pieux selon la loi, et de qui tous les Juifs demeurant à Damas rendaient un bon témoignage, vint me trouver
\VS{13}et me dit~: Saul, mon frère, recouvre la vue. Au même instant, je recouvrai la vue et je le regardai.
\VS{14}Et il me dit~: Le Dieu de nos pères t'a destiné à connaître sa volonté, à voir le Juste, et à entendre les paroles de sa bouche.
\VS{15}Car tu lui serviras de témoin auprès de tous les hommes, des choses que tu as vues et entendues.
\VS{16}Et maintenant, pourquoi tardes-tu~? Lève-toi, et sois baptisé et purifié de tes péchés, en invoquant le Nom du Seigneur.
\TextTitle{Le Seigneur appelle Paul à quitter Jérusalem et l'envoie dans les nations}
\VS{17}Or il arriva qu'après que je sois retourné à Jérusalem, comme je priais dans le temple, je fus ravi en extase,
\VS{18}et je vis le Seigneur qui me disait~: Hâte-toi, et sors promptement de Jérusalem, parce qu'ils ne recevront pas le témoignage que tu leur rendras de moi.
\VS{19}Et je dis~: Seigneur, ils savent eux-mêmes que je faisais mettre en prison et battre de verges dans les synagogues ceux qui croyaient en toi~;
\VS{20}et que lorsque le sang d'Etienne, ton martyr, fut répandu, j'étais moi-même présent, je consentais à sa mort, et je gardais les vêtements de ceux qui le faisaient mourir.
\VS{21}Alors il me dit~: Va, car je t'enverrai au loin vers les Gentils.
\TextTitle{Les Juifs demandent la mort de Paul}
\VS{22}Et ils l'écoutèrent jusqu'à cette parole~; mais alors ils élevèrent leur voix, en disant~: Ote de la terre un tel homme~! Car il n'est pas concevable qu'il vive.
\VS{23}Et comme ils criaient à haute voix, secouaient leurs vêtements, et jetaient de la poussière en l'air,
\VS{24}le tribun commanda de faire entrer Paul dans la forteresse, et de lui donner la question par le fouet, afin de savoir pour quel sujet ils criaient ainsi contre lui.
\TextTitle{Paul revendique ses droits de citoyen romain}
\VS{25}Comme on l'attachait pour le frapper, Paul dit au centenier qui était près de lui~: Vous est-il permis de fouetter un homme romain, et qui n'est même pas condamné~?
\VS{26}A ces mots, le centenier alla vers le tribun pour l'avertir, disant~: Prends garde à ce que tu feras, car cet homme est Romain.
\VS{27}Et le tribun, étant venu, dit à Paul~: Dis-moi, es-tu Romain~? Et il répondit~: Oui, je le suis.
\VS{28}Le tribun lui dit~: J'ai acquis ce droit de citoyen pour une grande somme d'argent. Et moi, dit Paul, je l'ai par ma naissance.
\VS{29}Aussitôt, ceux qui devaient lui donner la question se retirèrent, et le tribun, voyant que Paul était Romain, fut dans la crainte parce qu'il l'avait fait lier.
\TextTitle{Paul devant le sanhédrin}
\VS{30}Le lendemain, voulant savoir avec certitude de quoi les Juifs l'accusaient, le tribun lui fit ôter ses liens, et donna l'ordre aux principaux prêtres et à tout le sanhédrin de se réunir~; puis, il fit descendre Paul, et il le présenta devant eux.
\Chap{23}
\VerseOne{}Paul regardant fixement le sanhédrin, dit~: Hommes frères~! Je me suis conduit en toute bonne conscience devant Dieu jusqu'à ce jour.
\VS{2}Le grand-prêtre Ananias ordonna à ceux qui étaient près de lui de le frapper sur la bouche.
\VS{3}Alors Paul lui dit~: Dieu te frappera, muraille blanchie~! Tu es assis pour me juger selon la loi, et tu violes la loi en ordonnant qu'on me frappe~!
\VS{4}Ceux qui étaient présents lui dirent~: Tu insultes le grand-prêtre de Dieu~?
\VS{5}Et Paul dit~: Je ne savais pas, mes frères, que c'était le grand-prêtre~; car il est écrit~: Tu ne parleras pas mal du chef de ton peuple.
\TextTitle{Dissensions entre pharisiens et sadducéens}
\VS{6}Paul, sachant qu'une partie de l'assemblée était composée de sadducéens et l'autre de pharisiens, s'écria dans le sanhédrin~: Hommes frères~! Je suis pharisien, fils de pharisien, c'est à cause de l'espérance et de la résurrection des morts que je suis mis en jugement.
\VS{7}Quand il eut dit cela, il s'éleva un débat entre les pharisiens et les sadducéens~; et l'assemblée se divisa.
\VS{8}Car les sadducéens disent qu'il n'y a point de résurrection, ni d'ange, ni d'esprit, mais les pharisiens soutiennent les deux choses.
\VS{9}Il y eut une grande clameur. Alors les scribes du parti des pharisiens se levèrent et contestèrent, disant~: Nous ne trouvons aucun mal en cet homme~; peut-être un esprit ou un ange lui a parlé, ne combattons point contre Dieu.
\VS{10}Et comme il se faisait une grande division, le tribun craignant que Paul ne soit mis en pièces par eux, ordonna que les soldats descendent, et qu'ils l'enlèvent du milieu d'eux, et l'amènent dans la forteresse.
\TextTitle{Le Seigneur fortifie Paul}
\VS{11}La nuit suivante, le Seigneur apparut à Paul et lui dit~: Prends courage~; car, de même que tu as rendu témoignage de moi dans Jérusalem, il faut aussi que tu rendes témoignage à Rome.
\TextTitle{Complot des Juifs pour tuer Paul}
\VS{12}Quand le jour fut venu, les Juifs formèrent un complot, et firent des imprécations contre eux-mêmes, en disant qu'ils ne mangeraient pas ni ne boiraient jusqu'à ce qu'ils aient tué Paul.
\VS{13}Ceux qui formèrent ce complot étaient plus de quarante,
\VS{14}et ils s'adressèrent aux principaux prêtres et aux anciens, et leur dirent~: Nous nous sommes engagés, avec des imprécations contre nous-mêmes, à ne rien manger jusqu'à ce que nous ayons tué Paul.
\VS{15}Vous donc, maintenant, adressez-vous, avec le sanhédrin, au tribun pour le faire descendre demain au milieu de vous, comme si vous vouliez examiner sa cause plus exactement~; et nous, avant qu'il approche, nous sommes tous prêts à le tuer.
\VS{16}Le fils de la sœur de Paul, ayant eu connaissance de ce complot, alla dans la forteresse et le rapporta à Paul.
\VS{17}Paul appela l'un des centeniers et lui dit~: Mène ce jeune homme au tribun, car il a quelque chose à lui rapporter.
\VS{18}Il le prit donc et le mena au tribun, et il lui dit~: Le prisonnier Paul m'a appelé et m'a prié de t'amener ce jeune homme qui a quelque chose à te dire.
\VS{19}Et le tribun le prenant par la main, se retira à part, et lui demanda~: Qu'est-ce que tu as à me rapporter~?
\VS{20}Et il lui dit~: Les Juifs ont conspiré de te prier que demain tu envoies Paul au sanhédrin, comme s'ils voulaient s'enquérir de lui plus exactement de quelque chose.
\VS{21}Mais n'y consens point, car plus de quarante hommes d'entre eux sont en embûches contre lui, qui ont fait un vœu avec exécration de serment, de ne manger ni boire jusqu'à ce qu'ils l'aient tué~; et ils sont maintenant tous prêts, attendant ce que tu leur permettras.
\VS{22}Le tribun donc renvoya le jeune homme, en lui recommandant de ne parler à personne de ce rapport qu'il lui avait fait.
\TextTitle{Paul transféré à Césarée}
\VS{23}Ensuite, il appela deux des centeniers, et il leur dit~: Tenez prêts, dès la troisième heure de la nuit, deux cents soldats, soixante-dix cavaliers, et deux cents archers, pour aller jusqu'à Césarée.
\VS{24}Et ayez soin qu'il y ait des montures prêtes, afin qu'ayant fait monter Paul, ils le mènent sûrement au gouverneur Félix. \FTNT{Marcus Antonuis Félix était procurateur de la province romaine de la Judée de 52 à 60 ap. J.-C.}.
\VS{25}Et il lui écrivit une lettre en ces termes~:
\VS{26}Claude Lysias au très excellent gouverneur Félix, salut~!
\VS{27}Les Juifs s'étaient saisis de cet homme et allaient le tuer, lorsque je survins avec des soldats et le leur enlevai, ayant appris qu'il était Romain.
\VS{28}Voulant connaître le motif pour lequel ils l'accusaient, je l'amenai devant leur sanhédrin.
\VS{29}J'ai trouvé qu'il était accusé au sujet de questions relatives à leur loi, mais qu'il n'avait commis aucun crime qui mérite la mort ou la prison.
\VS{30}Ayant été averti des embûches que les Juifs avaient dressées contre lui, je te l'ai aussitôt envoyé, en ordonnant à ses accusateurs de te dire eux-mêmes ce qu'ils ont contre lui. Adieu~!
\TextTitle{Paul arrive à Césarée}
\VS{31}Les soldats prirent Paul, selon l'ordre qu'ils avaient reçu, et le conduisirent pendant la nuit jusqu'à Antipatris.
\VS{32}Le lendemain, laissant les cavaliers poursuivre la route avec Paul, ils retournèrent à la forteresse.
\VS{33}Arrivés à Césarée, les cavaliers remirent la lettre au gouverneur, et lui présentèrent aussi Paul.
\VS{34}Le gouverneur, après avoir lu la lettre, demanda à Paul de quelle province il était. Ayant appris qu'il était de Cilicie~:
\VS{35}Je t'entendrai, lui dit-il, plus amplement quand tes accusateurs seront venus. Et il ordonna qu'il soit gardé dans le Prétoire d'Hérode.
\Chap{24}
\TextTitle{Paul devant le gouverneur Félix~; accusation des Juifs}
\VerseOne{}Or cinq jours après, Ananias le grand-prêtre descendit avec les anciens, et un certain orateur, nommé Tertulle, qui comparurent devant le gouverneur contre Paul. 
\VS{2}Et Paul étant appelé, Tertulle commença à l'accuser, en disant~:
\VS{3}Très excellent Félix, nous reconnaissons en toutes choses partout et avec une entière reconnaissance, que nous avons obtenu une grande tranquillité par ton moyen, et par les bons règlements que tu as faits pour ce peuple, selon ta prudence.
\VS{4}Mais afin de ne pas te retenir plus longtemps, je te prie de nous entendre, selon ton équité, dans ce que nous allons te dire en peu de paroles.
\VS{5}Nous avons trouvé cet homme, qui est une peste, qui sème des divisions parmi tous les Juifs du monde entier, et qui est le chef de la secte des Nazaréens.
\VS{6}Il a même tenté de profaner le temple~; et nous l'avons saisi, et avons voulu le juger selon notre loi.
\VS{7}Mais le tribun Lysias étant survenu, il nous l'a arraché de nos mains avec une grande violence,
\VS{8}en ordonnant à ses accusateurs de venir vers toi. Tu pourras toi-même, en l'interrogeant, apprendre de lui tout ce dont nous l'accusons.
\VS{9}Les Juifs consentirent à cela, en disant que les choses étaient ainsi.
\TextTitle{Paul défend sa cause devant Félix}
\VS{10}Et après que le gouverneur eut fait signe à Paul de parler, il répondit~: Sachant qu'il y a déjà plusieurs années que tu es le juge de cette nation je réponds pour moi avec plus de courage:
\VS{11}Puisque tu peux comprendre qu'il n'y a pas plus de douze jours que je suis monté à Jérusalem pour adorer Dieu.
\VS{12}Mais ils ne m'ont point trouvé dans le temple disputant avec personne, ni faisant un amas de peuple, soit dans les synagogues, soit dans la ville. 
\VS{13}Et ils ne sauraient soutenir les choses dont ils m'accusent présentement.
\VS{14}Or je te confesse bien ce point, que selon la voie qu'ils appellent secte, je sers ainsi le Dieu de mes pères, croyant toutes les choses qui sont écrites dans la loi et dans les prophètes,
\VS{15}et ayant en Dieu cette espérance, comme ils l'ont eux-mêmes, qu'il y aura une résurrection des justes et des injustes.
\VS{16}C'est pourquoi aussi je travaille pour avoir toujours une conscience pure devant Dieu, et devant les hommes.
\VS{17}Or après plusieurs années, je suis venu pour faire des aumônes et des offrandes dans ma nation.
\VS{18}Et comme je m'occupais de ces choses, quelques Juifs d'Asie m'ont trouvé purifié dans le temple, sans attroupement ni tumulte.
\VS{19}Ils auraient dû eux-mêmes comparaître devant toi et m'accuser, s'ils avaient eu quelque chose contre moi.
\VS{20}Ou bien, que ceux-ci eux-mêmes disent, s'ils ont trouvé en moi quelque injustice, quand j'ai été présenté au sanhédrin~;
\VS{21}à moins que ce ne soit uniquement cette parole que j'ai fait entendre au milieu d'eux~; c'est à cause de la résurrection des morts que je suis aujourd'hui mis en jugement devant vous.
\VS{22}Félix, qui était parfaitement au courant de ce qui concerne cette secte, les ajourna, en disant~: Quand le tribun Lysias sera venu, j'examinerai votre affaire.
\VS{23}Et il donna l'ordre au centenier de garder Paul, en lui laissant une certaine liberté, et n'empêchant aucun des siens de le servir, ou de venir vers lui.
\TextTitle{Paul prêche Christ au gouverneur et à sa femme}
\VS{24}Quelques jours après, Félix vint avec Drusille, sa femme, qui était Juive, et il envoya chercher Paul. Il l'entendit sur la foi en Christ.
\VS{25}Et comme il parlait de la justice, de la tempérance, et du jugement à venir, Félix tout effrayé répondit~: Pour le moment retire-toi~; et quand j'aurai la commodité, je te rappellerai.
\VS{26}Il espérait en même temps que Paul lui donnerait de l'argent afin de le délivrer, c'est pourquoi il l'envoyait chercher souvent, et s'entretenait avec lui.
\TextTitle{Paul emprisonné deux ans à Césarée}
\VS{27}Deux ans s'écoulèrent ainsi, et Félix eut pour successeur Porcius Festus\FTNT{Porcius Festus était procurateur de Judée d'environ 60 à 62, succédant à Antonius Félix.}, qui voulant faire plaisir aux Juifs, laissa Paul en prison.
\Chap{25}
\TextTitle{Paul devant le gouverneur Festus}
\VerseOne{}Festus, étant arrivé dans la province, monta trois jours après de Césarée à Jérusalem.
\VS{2}Le grand-prêtre, et les principaux d'entre les Juifs portèrent plainte contre Paul devant lui. Ils firent des instances auprès de Festus, et dans des vues hostiles,
\VS{3}lui demandèrent une faveur contre lui~: Qu'il le fasse venir à Jérusalem. Ils avaient dressé des embûches pour le tuer en chemin.
\VS{4}Mais Festus leur répondit que Paul était bien gardé à Césarée, et que lui-même devait partir sous peu.
\VS{5}Et il ajouta~: Que les principaux d'entre vous descendent avec moi, et s'il y a quelque chose de coupable contre cet homme, qu'ils l'accusent.
\VS{6}Festus ne passa que dix jours parmi eux, puis il descendit à Césarée. Le lendemain, siégeant au tribunal, il ordonna que Paul soit amené.
\VS{7}Quand il fut amené, les Juifs qui étaient descendus de Jérusalem l'entourèrent et portèrent contre lui de nombreuses et graves accusations, qu'ils ne pouvaient pas prouver.
\VS{8}Tandis que Paul parlait pour sa défense~: Je n'ai rien fait de coupable, ni contre la loi des Juifs, ni contre le temple, ni contre César.
\VS{9}Mais Festus voulant faire plaisir aux Juifs, répondit à Paul, et dit~: Veux-tu monter à Jérusalem et y être jugé sur ces choses devant moi~?
\TextTitle{Paul en appelle à César}
\VS{10}Paul dit~: Je comparais devant le tribunal de César, où il faut que je sois jugé. Je n'ai fait aucun tort aux Juifs, comme tu le sais très bien.
\VS{11}Si j'ai commis quelque injustice, ou un crime digne de mort, je ne refuse pas de mourir~; mais si les choses dont ils m'accusent sont fausses, personne n'a le droit de me livrer à eux. J'en appelle à César.
\VS{12}Alors Festus ayant conféré avec le conseil, lui répondit~: En as-tu appelé à César~? Tu iras à César.
\TextTitle{Le roi Agrippa informé du cas de Paul}
\VS{13}Quelques jours après, le roi Agrippa\FTNT{Agrippa II (27-28 ap. J.-C. – 93-101 ap. J.-C.) était le fils d'Agrippa I(10 av. J.-C. – 44 ap. J.-C.), qui était lui-même le petit-fils d'Hérode le Grand (73 av. J.-C. – 4 av. J.-C.).} et Bérénice\FTNT{Bérénice (née vers 28 ap. J.-C.) était la fille d'Agrippa I et donc la sœur d'Agrippa II. Pendant tout le règne de son frère, elle fut présentée comme reine à ses cotés, raison pour laquelle on soupçonna une liaison incestueuse entre eux.} arrivèrent à Césarée pour saluer Festus.
\VS{14}Comme ils passèrent là plusieurs jours, Festus fit mention au roi de l'affaire de Paul, en disant~: Félix a laissé prisonnier un homme
\VS{15}contre lequel, lorsque j'étais à Jérusalem, les principaux prêtres et les anciens des Juifs ont porté plainte, en demandant sa condamnation.
\VS{16}Mais je leur ai répondu que ce n'est pas la coutume des Romains de livrer quelqu'un à la mort, avant que l'inculpé ait été mis en présence de ses accusateurs, et qu'il ait eu la liberté de se défendre sur le crime dont on l'accuse.
\VS{17}Ils sont donc venus ici, et sans différer, je siégeai le lendemain, et je donnai l'ordre qu'on amène cet homme.
\VS{18}Ses accusateurs s'étant présentés, ne lui imputèrent aucun des crimes dont je pensais qu'ils l'accuseraient.
\VS{19}Mais ils avaient avec lui des discussions relatives à leurs superstitions, et à un certain Jésus qui est mort, que Paul affirmait être vivant.
\VS{20}Ne sachant quel parti prendre dans ce débat, je demandai à cet homme s'il voulait aller à Jérusalem et y être jugé sur ces choses.
\VS{21}Mais Paul en ayant appelé, pour que sa cause soit réservée à la connaissance de l'empereur, j'ai ordonné qu'on le garde jusqu'à ce que je l'envoie à César.
\VS{22}Alors Agrippa dit à Festus~: Je voudrais bien aussi entendre cet homme. Demain, dit-il, tu l'entendras.
\TextTitle{Paul est amené dans la salle d'audience}
\VS{23}Le lendemain donc, Agrippa et Bérénice étant venus en grande pompe, et étant entrés dans la salle d'audience avec les tribuns et les principaux de la ville, Paul fut amené sur l'ordre de Festus.
\VS{24}Et Festus dit~: Roi Agrippa, et vous tous qui êtes ici avec nous, vous voyez cet homme au sujet duquel toute la multitude des Juifs s'est adressée à moi, soit à Jérusalem soit ici, en s'écriant qu'il ne devait plus vivre.
\VS{25}Pour moi, ayant trouvé qu'il n'avait rien fait qui mérite la mort, et lui-même en ayant appelé à Auguste, j'ai résolu de le faire partir.
\VS{26}Comme je n'ai rien de certain à écrire à l'empereur sur son compte, je vous l'ai présenté, et principalement à toi, roi Agrippa, afin qu'après en avoir fait l'examen, j'aie de quoi écrire.
\VS{27}Car il me semble qu'il n'est pas raisonnable d'envoyer un prisonnier sans marquer les faits dont on l'accuse.
\Chap{26}
\TextTitle{Discours de Paul devant Agrippa\FTNTT{\vref{Ac. 9:1-18}~; \vref{22:1-16}}}
\VerseOne{}Agrippa dit à Paul~: Il t'est permis de parler pour toi-même. Alors Paul ayant étendu la main, parla ainsi pour sa défense~:
\VS{2}Roi Agrippa~! Je m'estime béni de ce que je dois me défendre aujourd'hui devant toi, de toutes les choses dont les Juifs m'accusent~;
\VS{3}car tu connais parfaitement leurs coutumes et leurs discussions. Je te prie donc de m'écouter avec patience.
\VS{4}Ma vie, dès les premiers temps de ma jeunesse, est connue de tous les Juifs, puisqu'elle s'est passée à Jérusalem, au milieu de ma nation.
\VS{5}Car ils savent depuis longtemps, s'ils veulent en rendre témoignage, que j'ai vécu en pharisien, selon la secte la plus rigide de notre religion.
\VS{6}Et maintenant, je suis mis en jugement pour l'espérance de la promesse que Dieu a faite à nos pères,
\VS{7}et à laquelle nos douze tribus, qui servent Dieu continuellement nuit et jour, espèrent parvenir~; et c'est pour cette espérance, ô roi Agrippa, que je suis accusé par les Juifs.
\VS{8}Quoi~? Jugez-vous incroyable que Dieu ressuscite les morts~?
\VS{9}Pour moi, j'avais cru devoir agir vigoureusement contre le Nom de Jésus de Nazareth.
\VS{10}C'est ce que j'ai fait à Jérusalem. J'ai mis en prison plusieurs des saints, après en avoir reçu le pouvoir des principaux prêtres, et quand on les faisait mourir, je joignais mon suffrage à celui des autres.
\VS{11}Je les ai souvent châtiés dans toutes les synagogues, et les forçais à blasphémer. Dans mes excès de fureur contre eux, je les persécutais même jusque dans les villes étrangères.
\VS{12}Comme j'allais aussi à Damas dans ce dessein, avec l'autorisation et la permission des principaux prêtres,
\VS{13}en plein midi, ô roi, je vis en chemin resplendir autour de moi et de mes compagnons, une lumière venant du ciel et dont l'éclat surpassait celui du soleil.
\VS{14}Nous tombâmes tous par terre, et j'entendis une voix qui me parlait en langue hébraïque~: Saul, Saul, pourquoi me persécutes-tu~? Il te serait dur de regimber contre les aiguillons.
\VS{15}Je répondis~: Qui es-tu Seigneur~? Et il répondit~: Je suis Jésus que tu persécutes.
\VS{16}Mais lève-toi, et tiens-toi sur tes pieds~; car je te suis apparu pour t'établir serviteur et témoin des choses que tu as vues et de celles pour lesquelles je t'apparaîtrai.
\VS{17}Je t'ai arraché du milieu du peuple et des Gentils, vers qui je t'envoie maintenant,
\VS{18}pour ouvrir leurs yeux afin qu'ils passent des ténèbres à la lumière, et de la puissance de Satan à Dieu~; afin que par la foi qu'ils auront en moi, ils reçoivent la rémission de leurs péchés et qu'ils aient part à l'héritage des saints.
\VS{19}Ainsi, ô roi Agrippa, je n'ai pas été désobéissant à la vision céleste.
\VS{20}A ceux de Damas d'abord, puis à Jérusalem, dans toute la Judée, et chez les Gentils, j'ai prêché la repentance et la conversion à Dieu, avec la pratique d'œuvres dignes de la repentance.
\VS{21}C'est pour cela que les Juifs se sont saisis de moi dans le temple, et ont tâché de me tuer.
\VS{22}Mais ayant été secouru par l'aide de Dieu, je suis vivant jusqu'à ce jour, rendant témoignage aux petits et aux grands, sans m'écarter en rien de ce que les prophètes et Moïse ont prédit devoir arriver,
\VS{23}à savoir que le Christ souffrirait, et que ressuscité le premier d'entre les morts, il annoncerait la lumière au peuple et aux nations.
\TextTitle{Paul exhorte Agrippa}
\VS{24}Comme il parlait ainsi pour sa défense, Festus dit à haute voix~: Tu es fou Paul~! Ton grand savoir dans les lettres te fait déraisonner.
\VS{25}Et Paul dit~: Je ne suis point fou, très excellent Festus, mais je dis des paroles de vérité et de bon sens.
\VS{26}Car le roi est bien informé de ces choses~; et je lui en parle librement, parce que je suis persuadé qu'il n'en ignore aucune, puisque ce n'est pas en cachette qu'elles se sont passées.
\VS{27}Ô Roi Agrippa~! Crois-tu aux prophètes~? Je sais que tu y crois.
\VS{28}Et Agrippa répondit à Paul~: Tu vas bientôt me persuader de devenir chrétien~!
\VS{29}Et Paul lui dit~: Je souhaiterais devant Dieu que non seulement toi, mais aussi tous ceux qui m'écoutent aujourd'hui, vous deveniez tels que je suis à l'exception de ces liens~!
\VS{30}Paul ayant dit ces choses, le roi se leva, avec le gouverneur et Bérénice, et ceux qui étaient assis avec eux.
\VS{31}Et s'étant retirés à part, ils se disaient les uns les autres~: Cet homme n'a rien fait qui mérite la mort ou la prison.
\VS{32}Et Agrippa dit à Festus~: Cet homme aurait pu être relâché s'il n'avait pas appelé à César.
\Chap{27}
\TextTitle{Toujours prisonnier, Paul embarque pour Rome}
\VerseOne{}Lorsqu'il fut décidé que nous embarquerions pour l'Italie, on remit Paul avec quelques autres prisonniers à un nommé Julius, centenier d'une cohorte de la légion appelée Auguste.
\VS{2}Nous montâmes sur un navire d'Adramytte, nous partîmes prenant notre route vers les côtes de l'Asie, ayant avec nous Aristarque, un Macédonien de la ville de Thessalonique.
\VS{3}Le jour suivant, nous arrivâmes à Sidon~; et Julius, qui traitait Paul avec bienveillance, lui permit d'aller vers ses amis afin de recevoir leurs soins.
\VS{4}Puis étant partis de là, nous longeâmes l'île de Chypre, parce que les vents étaient contraires.
\VS{5}Après avoir traversé la mer de Cilicie et de Pamphylie, nous arrivâmes à Myra, ville de Lycie.
\VS{6}Et là, le centenier trouva un navire d'Alexandrie qui allait en Italie, dans lequel il nous fit monter.
\VS{7}Et comme nous naviguions lentement pendant plusieurs jours, et que nous étions arrivés avec peine vis-à-vis de Cnide, parce que le vent ne nous permettait pas d'avancer, nous naviguâmes en dessous de la Crète, vers Salmone.
\VS{8}Nous la côtoyâmes avec peine, nous arrivâmes à un lieu qui est appelé Beaux-Ports, près duquel était la ville de Lasée.
\VS{9}Il s'était écoulé beaucoup de temps, et la navigation devenait dangereuse, car le temps du jeûne était déjà passé\FTNT{Ce jeûne correspondait au jour de l'expiation célébré le dixième jour du septième mois. \vref{Lé. 23:27}.}.
\VS{10}C'est pourquoi Paul les avertit, en disant~: Ô hommes, je vois que la navigation ne se fera pas sans péril et sans dommage, non seulement pour la cargaison et pour le navire, mais aussi pour nos propres vies.
\VS{11}Mais le centenier écouta plus le pilote et le maître du navire, plutôt que les paroles de Paul.
\VS{12}Et comme le port n'était pas bon pour y passer l'hiver, la plupart furent d'avis de partir de là, pour tâcher de gagner Phénix, qui est un port de Crète, qui regarde le vent d'Afrique et le couchant septentrional, afin d'y passer l'hiver.
\VS{13}Un vent du midi commença à souffler doucement, et se croyant maîtres de leur dessein, ils levèrent l'ancre et côtoyèrent de près l'île de Crète.
\TextTitle{Une tempête de plusieurs jours}
\VS{14}Mais bientôt un vent impétueux, du nord-est, qu'on appelle Euraquilon\FTNT{Euraquilon~: Vagues et vent d'Est}, se leva du côté de l'île.
\VS{15}Le navire fut emporté par la violence de la tempête, et ne pouvant résister, nous nous laissâmes aller au gré du vent.
\VS{16}Nous passâmes au-dessous d'une petite île nommée Clauda, et nous eûmes de la peine à nous rendre maîtres de la chaloupe~;
\VS{17}après l'avoir hissée, les matelots se servirent des moyens de secours pour ceindre le navire, et dans la crainte de tomber sur la Syrte\FTNT{Syrte~: Il s'agit de la Grande Syrte et de la Petite Syrte~: deux bancs de sables mouvants très redoutés.}, ils abaissèrent les voiles. C'est ainsi qu'on se laissa emporter par le vent.
\VS{18}Comme nous étions violemment battus par la tempête, le jour suivant, ils jetèrent la cargaison à la mer~;
\VS{19}et le troisième jour, nous jetâmes de nos propres mains les agrès du navire.
\VS{20}Le soleil et les étoiles ne parurent pas pendant plusieurs jours, et la tempête nous agitait si violemment que nous perdîmes enfin toute espérance de nous sauver.
\TextTitle{Paul rassure les membres du navire}
\VS{21}On n'avait pas mangé depuis longtemps. Paul se tenant alors debout au milieu d'eux, leur dit~: Ô hommes, il fallait m'écouter et ne pas partir de Crète, afin d'éviter cette tempête et ce dommage.
\VS{22}Maintenant je vous exhorte à prendre courage~; car aucun de vous ne perdra la vie, et il n'y aura de perte que celle du navire.
\VS{23}Car un ange du Dieu à qui j'appartiens et que je sers m'est apparu cette nuit,
\VS{24}et m'a dit~: Paul, ne crains point~; il faut que tu comparaisses devant César~; et voici, Dieu t'a donné tous ceux qui naviguent avec toi.
\VS{25}C'est pourquoi, ô hommes, prenez courage, car j'ai cette confiance en Dieu que la chose arrivera comme elle m'a été dite.
\VS{26}Mais nous devons échouer sur une île.
\VS{27}La quatorzième nuit, vers minuit, tandis que nous étions ballotés sur l'Adriatique, les matelots soupçonnèrent qu'on approchait de quelque terre.
\VS{28}Ayant jeté la sonde, ils trouvèrent vingt brasses~; puis étant passés un peu plus loin, et ayant encore jeté la sonde, ils trouvèrent quinze brasses.
\VS{29}Mais craignant de heurter contre des écueils, ils jetèrent quatre ancres de la poupe, et attendirent le jour avec impatience.
\VS{30}Mais comme les matelots cherchaient à s'échapper du navire, et mettaient la chaloupe à la mer, sous prétexte de jeter les ancres de la proue,
\VS{31}Paul dit au centenier et aux soldats~: Si ces hommes ne restent pas dans le navire, vous ne pouvez pas être sauvés.
\VS{32}Alors les soldats coupèrent les cordes de la chaloupe, et la laissèrent tomber.
\VS{33}Avant que le jour paraisse, Paul les exhorta tous à prendre de la nourriture, en leur disant~: C'est aujourd'hui le quatorzième jour que vous êtes en attente et que vous persistez à vous abstenir de manger.
\VS{34}Je vous exhorte donc à prendre quelque nourriture, vu que cela est nécessaire pour votre conservation, et aucun de vos cheveux ne se perdra.
\VS{35}Ayant ainsi parlé, il prit du pain, et rendit grâces à Dieu en présence de tous~; il le rompit et se mit à manger.
\VS{36}Et tous, reprenant courage, mangèrent aussi.
\VS{37}Nous étions dans le navire deux cent soixante-seize personnes.
\VS{38}Quand ils eurent mangé jusqu'à être rassasiés, ils allégèrent le navire en jetant le blé dans la mer.
\TextTitle{Naufrage du navire}
\VS{39}Lorsque le jour fut venu, ils ne reconnurent point la terre~; mais ayant aperçu un golfe avec un rivage, ils résolurent d'y faire échouer le navire, s'ils le pouvaient.
\VS{40}Ayant donc retiré les ancres, ils abandonnèrent le navire à la mer, lâchant en même temps les attaches des gouvernails~; et ayant tendu la voile de l'artimon, ils tâchaient de se diriger vers le rivage.
\VS{41}Mais ils rencontrèrent une langue de terre, où ils firent échouer le navire~; et la proue, s'étant engagée, resta immobile, tandis que la poupe se brisait par la violence des vagues.
\VS{42}Les soldats furent d'avis de tuer les prisonniers, de peur que quelqu'un d'eux ne s'échappe à la nage.
\VS{43}Mais le centenier, voulant sauver Paul, les empêcha d'exécuter ce conseil. Il ordonna à ceux qui savaient nager de se jeter les premiers dans l'eau pour gagner la terre,
\VS{44}et aux autres de se mettre sur des planches ou sur des débris du navire. Et ainsi tous parvinrent à terre sains et saufs.
\Chap{28}
\TextTitle{Paul mordu par une vipère sur l'île de Malte}
\VerseOne{}Une fois hors de danger, ils reconnurent alors que l'île s'appelait Malte.
\VS{2}Les barbares nous traitèrent avec beaucoup d'humanité~; ils nous recueillirent tous auprès d'un grand feu, qu'ils avaient allumé parce que la pluie tombait et qu'il faisait très froid.
\VS{3}Paul ayant ramassé un tas de broussailles et l'ayant mis au feu, une vipère en sortit à cause de la chaleur et s'attacha à sa main.
\VS{4}Et quand les barbares virent cette bête suspendue à sa main, ils se dirent les uns les autres~: Certainement cet homme est un meurtrier~; puisque après être échappé de la mer, la justice ne permet pas qu'il vive. 
\VS{5}Mais Paul ayant secoué la bête dans le feu, ne ressentit aucun mal.
\VS{6}Les barbares s'attendaient à le voir enfler ou tomber subitement mort~; mais après avoir longtemps attendu, voyant qu'il ne lui arrivait aucun mal, ils changèrent de langage et dirent que c'était un dieu.
\TextTitle{Guérison du père de Publius}
\VS{7}Or en cet endroit-là étaient des terres qui appartenaient au principal de l'île, nommé Publius, qui nous reçut et nous logea pendant trois jours avec beaucoup de bonté.
\VS{8}Et il arriva que le père de Publius était au lit, malade de la fièvre et de la dysenterie~; Paul s'étant rendu vers lui, pria, lui imposa les mains, et le guérit.
\VS{9}Là-dessus, vinrent tous les autres malades de l'île, et ils furent guéris.
\VS{10}Ils nous rendirent de grands honneurs, et à notre départ, on nous fournit ce qui nous était nécessaire.
\TextTitle{Paul arrive à Rome}
\VS{11}Trois mois après, nous partîmes sur un navire d'Alexandrie qui avait passé l'hiver dans l'île, et qui avait pour enseigne Castor et Pollux.
\VS{12}Ayant abordé à Syracuse, nous y restâmes trois jours.
\VS{13}De là, en suivant la côte, nous arrivâmes à Reggio~; et un jour après, le vent du Midi s'étant levé, nous fîmes en deux jours le trajet jusqu'à Pouzzoles,
\VS{14}où nous trouvâmes des frères qui nous prièrent de passer sept jours avec eux. Et ensuite, nous arrivâmes à Rome.
\VS{15}Et les frères qui y étaient, ayant appris de nos nouvelles, vinrent à notre rencontre jusqu'au Forum d'Appius et aux Trois-Tavernes~; en les voyant, Paul rendit grâces à Dieu et prit courage.
\TextTitle{Paul annonce Christ aux Juifs de Rome}
\VS{16}Lorsque nous fûmes arrivés à Rome, le centenier mit les prisonniers entre les mains du préfet du prétoire~; mais quant à Paul, il lui permit de demeurer dans un domicile particulier avec un soldat qui le gardait.
\VS{17}Or il arriva que trois jours après que Paul convoqua les principaux des Juifs~; et quand ils furent réunis, il leur dit~: Hommes frères~! Sans avoir rien fait contre le peuple ni contre les coutumes des pères, j'ai été mis en prison à Jérusalem, et livré entre les mains des Romains,
\VS{18}qui après m'avoir examiné, voulaient me relâcher parce qu'il n'y avait en moi aucun crime qui mérite la mort.
\VS{19}Mais les Juifs s'y opposèrent, j'ai été contraint d'en appeler à César~; n'ayant du reste aucun dessein d'accuser ma nation.
\VS{20}C'est pour ce sujet que je vous ai appelés, afin de vous voir et vous parler~; car c'est pour l'espérance d'Israël que je porte cette chaîne.
\VS{21}Mais ils lui répondirent~: Nous n'avons reçu de Judée aucune lettre à ton sujet, et il n'est venu aucun frère qui ait rapporté ou dit quelque mal de toi.
\VS{22}Cependant nous entendrons volontiers de toi quel est ton sentiment~; car quant à cette secte, il nous est connu qu'on la contredit partout.
\VS{23}Et après lui avoir assigné un jour, plusieurs vinrent auprès de lui dans son logis~; et il leur expliquait par plusieurs témoignages le Royaume de Dieu, et depuis le matin jusqu'au soir, il cherchait à les persuader de ce qui concerne Jésus, tant par la loi de Moïse que par les prophètes.
\VS{24}Et les uns furent persuadés par les choses qu'il disait~; et les autres n'y crurent point.
\TextTitle{Incrédulité des Juifs~: Paul se tourne vers les Gentils\FTNTT{\vref{Ap. 13:14}~; \vref{18:6}.}}
\VS{25}C'est pourquoi n'étant pas d'accord entre eux, ils se retirèrent après que Paul leur eut dit ces paroles~: Le Saint-Esprit a bien parlé à nos pères par le prophète Esaïe en disant~:
\VS{26}Va vers ce peuple et dis-lui~: Vous entendrez de vos oreilles, et vous ne comprendrez point~; vous regarderez de vos yeux, et vous ne verrez point.
\VS{27}Car le cœur de ce peuple est devenu insensible~; ils ont endurci leurs oreilles, et ils ont fermé leurs yeux~; de peur qu'ils ne voient des yeux, qu'ils n'entendent des oreilles, qu'ils ne comprennent de leur cœur, qu'ils ne se convertissent, et que je ne les guérisse\FTNT{\vref{Es. 6:10}.}.
\VS{28}Sachez donc que ce salut de Dieu est envoyé aux Gentils, et ils l'écouteront.
\VS{29}Lorsqu'il eut dit cela, les Juifs s'en allèrent, discutant vivement entre eux.
\VS{30}Paul demeura deux ans entiers dans une maison qu'il avait louée. Il recevait tous ceux qui venaient le voir,
\VS{31}prêchant le Royaume de Dieu, et enseignant les choses qui concernent le Seigneur Jésus-Christ en toute liberté dans les paroles et sans aucun empêchement.
\PPE{}
\end{multicols}

%\clearpage\ShortTitle{Jacques}\BookTitle{Jacques}\BFont
\noindent\hrulefill
{\footnotesize
\textit{
\bigskip
{\centering{}
\\Auteur : Jacques
\\Signification : Qui supplante
\\Thème : La vie chrétienne sous son aspect pratique
\\Date de rédaction : Env. 45-50 ap. J.-C.\\}
}
%\bigskip
\textit{
\\Jacques, frère de Jésus-Christ homme, et ancien au sein de la première église chrétienne située à Jérusalem, écrit aux chrétiens d'origine juive, dispersés dans l'empire romain. Il les console suite à l'adversité qu'ils rencontraient et les exhorte à tenir ferme, leur expliquant que la foi authentique doit être accompagnée d'œuvres. Il les met en garde contre la convoitise, source de toutes les tentations, et les prévient également quant à l'amour du monde et la confiance que certains peuvent mettre dans l'argent. Pour terminer, il leur recommande d'être patients dans l'épreuve et de prier sans cesse jusqu'au retour du Seigneur.\bigskip
}
}
\par\nobreak\noindent\hrulefill
\begin{multicols}{2}
\Chap{1}
\TextTitle{Introduction}
\VerseOne{}Jacques, serviteur de Dieu, et du Seigneur Jésus-Christ, aux douze tribus qui sont dispersées, salut !
\TextTitle{La nécessité de l'épreuve de la foi}
\VS{2}Mes frères, regardez comme un sujet d'une parfaite joie quand vous êtes exposés à diverses épreuves,
\VS{3}sachant que l'épreuve de votre foi produit la patience.
\VS{4}Mais il faut que la patience accomplisse parfaitement son œuvre, afin que vous soyez parfaits et accomplis, en sorte qu'il ne vous manque rien.
\VS{5}Et si quelqu'un de vous manque de sagesse, qu'il la demande à Dieu, qui la donne à tous libéralement, et sans reproche, et elle lui sera donnée.
\VS{6}Mais qu'il la demande avec foi, ne doutant nullement ; car celui qui doute est semblable au flot de la mer, agité et poussé çà et là par le vent.
\VS{7}Qu'un tel homme ne s'attende pas à recevoir quelque chose du Seigneur.
\VS{8}L'homme double de coeur est inconstant dans toutes ses voies.
\VS{9}Que le frère de basse condition se glorifie dans son élévation.
\VS{10}Que le riche, au contraire, se glorifie dans sa basse condition ; car il passera comme la fleur de l'herbe.
\VS{11}En effet, le soleil s'est levé avec sa chaleur ardente, et l'herbe a séché, et sa fleur est tombée, et son éclat a péri, ainsi le riche se flétrira dans ses entreprises.
\TextTitle{Dieu ne tente personne ; la justice de Dieu}
\VS{12}Heureux l'homme qui endure la tentation\FTNT{Le terme grec « peirasmos » utilisé dans ce verset veut aussi dire « épreuve ».} ; car après avoir été éprouvé, il recevra la couronne de vie, que le Seigneur a promise à ceux qui l'aiment.
\VS{13}Quand quelqu'un est tenté, qu'il ne dise pas : Je suis tenté par Dieu. Car Dieu ne peut être tenté par le mal, et aussi ne tente-t-il personne.
\VS{14}Mais chacun est tenté quand il est attiré et amorcé par sa propre convoitise.
\VS{15}Puis quand la convoitise a conçu, elle enfante le péché ; et le péché, étant consommé, produit la mort.
\VS{16}Mes frères bien-aimés, ne vous y trompez pas :
\VS{17}Tout ce qui nous est donné d'excellent et tout don parfait viennent d'en haut et descendent du Père des lumières, en qui il n'y a ni changement ni ombre de variation.
\VS{18}Il nous a engendrés de sa propre volonté, par la parole de la vérité, afin que nous soyons comme les prémices de ses créatures.
\VS{19}Ainsi, mes frères bien-aimés, que tout homme soit prompt à écouter, lent à parler et lent à la colère ;
\VS{20}car la colère de l'homme n'accomplit pas la justice de Dieu.
\TextTitle{Importance de la mise en pratique de la parole}
\VS{21}C'est pourquoi, rejetant toute souillure et tout résidu\FTNT{Le mot « résidu » vient du grec « perisseia », ce mot signifie « abondance », « surabondamment », « tout excès », « reste ». Les Grecs utilisaient ce terme pour décrire l'excès de cire dans leurs oreilles. Il est question de la méchanceté qui reste dans un Chrétien et qui provient de son état antérieur à sa conversion.} de méchanceté, recevez avec douceur la parole qui a été plantée en vous et qui peut sauver vos âmes.
\VS{22}Et mettez en pratique la parole, et ne l'écoutez pas seulement, en vous trompant vous-mêmes par de vains discours.
\VS{23}Car, si quelqu'un écoute la parole et ne la met pas en pratique, il est semblable à un homme qui regarde dans un miroir son visage naturel,
\VS{24}et qui, après s'être regardé, s'en va, et oublie aussitôt comment il était.
\VS{25}Mais celui qui aura plongé les regards dans la loi parfaite, la loi de la liberté, et qui aura persévéré, n'étant point un auditeur oublieux, mais pratiquant les œuvres qui lui sont prescrites, celui-là sera heureux dans son oeuvre.
\TextTitle{La religion pure et sans tache}
\VS{26}Si quelqu'un parmi vous croit être religieux alors qu'il ne tient pas sa langue en bride, mais séduit son cœur, la religion d'un tel homme est vaine.
\VS{27}La religion pure et sans tache envers notre Dieu et notre Père, c'est de visiter les orphelins et les veuves dans leurs afflictions, et de se conserver pur des souillures de ce monde.
\Chap{2}
\TextTitle{L'amour pour son prochain en pratique}
\VerseOne{}Mes frères, n'ayez point la foi en notre Seigneur Jésus-Christ glorieux, en ayant égard à l'apparence des personnes.
\VS{2}En effet, s'il entre dans votre assemblée un homme qui porte un anneau d'or et un habit magnifique, et qu'il y entre aussi un pauvre misérablement vêtu ;
\VS{3}et que vous ayez égard à celui qui porte l'habit magnifique et lui disiez : Toi, assieds-toi ici honorablement ! Et que vous disiez au pauvre : Toi, tiens-toi là debout ! Ou, assieds-toi ici sur mon marchepied !
\VS{4}n'avez-vous pas fait de différence en vous-mêmes, et n'êtes-vous pas des juges qui avez des pensées injustes ?
\VS{5}Ecoutez, mes frères bien-aimés : Dieu n'a-t-il pas choisi les pauvres de ce monde, qui sont riches en la foi, et héritiers du Royaume qu'il a promis à ceux qui l'aiment ?
\VS{6}Mais vous avez déshonoré le pauvre ! Et cependant les riches ne vous oppriment-ils pas, et ne vous traînent-ils pas devant les tribunaux ?
\VS{7}N'est-ce pas eux qui blasphèment le beau Nom qui a été invoqué sur vous ?
\VS{8}Si, en effet, vous accomplissez la loi royale, qui est selon l'Ecriture : Tu aimeras ton prochain comme toi-même\FTNT{Lé. 19:18.}, vous faites bien.
\VS{9}Mais si vous avez égard à l'apparence des personnes, vous commettez un péché, et vous êtes convaincus par la loi comme des transgresseurs.
\VS{10}Car quiconque observe toute la loi, mais pèche contre un seul commandement, devient coupable de tous.
\VS{11}En effet, celui qui a dit : Tu ne commettras point d'adultère, a dit aussi : Tu ne tueras point. Or, si donc tu ne commets point d'adultère\FTNT{Ex. 20:13-14.}, mais que tu tues, tu deviens transgresseur de la loi.
\VS{12}Ainsi parlez et ainsi agissez comme devant être jugés par la loi de la liberté,
\VS{13}car il y aura un jugement sans miséricorde sur celui qui n'aura point usé de miséricorde\FTNT{Mt. 7:2.} ; mais la miséricorde triomphe du jugement.
\TextTitle{Les œuvres de la foi}
\VS{14}Mes frères, que servira-t-il à quelqu'un de dire qu'il a la foi, s'il n'a pas les œuvres ? Cette foi peut-elle le sauver ?
\VS{15}Et si un frère ou une sœur sont nus et manquent de ce qui leur est nécessaire chaque jour pour vivre,
\VS{16}et que l'un d'entre vous leur dise : Allez en paix, chauffez-vous, et rassasiez-vous ! Et que vous ne leur donniez pas les choses nécessaires pour le corps, que leur servira cela ?
\VS{17}De même aussi la foi, si elle n'a pas les œuvres, elle est morte en elle-même.
\VS{18}Mais quelqu'un dira : Tu as la foi ; et moi, j'ai les œuvres. Montre-moi donc ta foi sans les œuvres, et moi, je te montrerai ma foi par mes œuvres.
\VS{19}Tu crois qu'il n'y a qu'un Dieu, tu fais bien ; les démons le croient aussi, et ils tremblent.
\VS{20}Mais, ô homme vain, veux-tu savoir que la foi qui est sans les œuvres est morte ?
\TextTitle{La foi d'Abraham et de Rahab manifestée dans leurs œuvres\FTNT{Ro. 4:1-25}}
\VS{21}Abraham, notre père, ne fut-il pas justifié par les œuvres, quand il offrit son fils Isaac sur l'autel ?
\VS{22}Ne vois-tu donc pas que sa foi agissait avec ses œuvres, et que ce fut par ses œuvres que sa foi fut rendue parfaite ?
\VS{23}Ainsi s'accomplit ce que dit l'Ecriture : Abraham crut en Dieu, et cela lui fut imputé à justice\FTNT{Ge. 15:6.}; et il fut appelé ami de Dieu.
\VS{24}Vous voyez donc que l'homme est justifié par les œuvres, et non par la foi seulement.
\VS{25}Pareillement, Rahab, la prostituée, ne fut-elle pas également justifiée par les œuvres, lorsqu'elle reçut les messagers, et qu'elle les fit partir par un autre chemin\FTNT{Jos. 2:1-21.} ?
\VS{26}Car, comme le corps sans esprit est mort, de même la foi sans les œuvres est morte.
\Chap{3}
\TextTitle{Les enseignants jugés plus sévèrement}
\VerseOne{}Ne soyez pas nombreux, mes frères, à devenir des enseignants\FTNT{Du grec « didaskalos » : « maître », « professeur », « docteur chargé d'instruire, d'enseigner la parole ». Mt. 8:19 ; Mt. 22:16 ; 1 Co. 12:28.}, sachant que nous en recevrons un plus grand jugement.
\TextTitle{Enseignements sur la langue}
\VS{2}Car nous péchons tous en plusieurs choses. Si quelqu'un ne pèche pas en paroles, c'est un homme parfait, et il peut même tenir en bride tout le corps.
\VS{3}Voici, nous mettons le mors dans la bouche des chevaux, afin qu'ils nous obéissent, et nous menons çà et là tout le corps.
\VS{4}Voici, aussi les navires, quoiqu'ils soient si grands et qu'ils soient agités par la tempête, ils sont dirigés partout çà et là par un petit gouvernail, selon qu'il plaît à celui qui les gouverne.
\VS{5}Il en est ainsi de la langue, c'est un petit membre, et cependant elle peut se vanter de grandes choses. Voici, un petit feu, combien de bois allume-t-il?
\VS{6}La langue aussi est un feu ; c'est le monde de l'iniquité. La langue est placée parmi nos membres, souillant tout le corps, et enflammant tout le cours de la vie, étant elle-même enflammée par le feu de la géhenne.
\VS{7}Car toutes les espèces d'animaux sauvages, d'oiseaux, de reptiles, et d'animaux marins, se domptent et ont été domptés par la nature humaine ;
\VS{8}mais nul homme ne peut dompter la langue ; c'est un mal qu'on ne peut réprimer ; elle est pleine d'un venin mortel.
\VS{9}Par elle nous bénissons Dieu notre Père, et par elle nous maudissons les hommes faits à la ressemblance de Dieu.
\VS{10}De la même bouche sortent la bénédiction et la malédiction. Il ne faut pas qu'il en soit ainsi, mes frères.
\VS{11}Une fontaine fait-elle jaillir par la même ouverture l'eau douce et l'eau amère ?
\VS{12}Mes frères, un figuier peut-il produire des olives, ou une vigne des figues ? De même, aucune fontaine ne peut produire de l'eau salée et de l'eau douce.
\TextTitle{La sagesse humaine et la sagesse d'en haut}
\VS{13}Y a-t-il parmi vous quelque homme sage et intelligent ? Qu'il fasse voir ses oeuvres par une bonne conduite avec douceur et sagesse.
\VS{14}Mais si vous avez une envie amère et un esprit de querelle dans vos cœurs, ne vous glorifiez pas, et ne mentez pas en déshonorant la vérité de l'Evangile.
\VS{15}Car ce n'est pas là la sagesse qui descend d'en haut ; mais c'est une sagesse terrestre, animale\FTNT{Animale ou charnelle.} et diabolique.
\VS{16}Car là où il y a de l'envie et un esprit de querelle, là est le désordre, et toute sorte de mal.
\VS{17}Mais la sagesse d'en haut est premièrement pure, ensuite pacifique, modérée, conciliante, pleine de miséricorde et de bons fruits, sans partialité, et sans hypocrisie.
\VS{18}Or le fruit de la justice est semé dans la paix pour ceux qui s'adonnent à la paix.
\Chap{4}
\TextTitle{Condamnation des mauvais désirs}
\VerseOne{}D'où viennent parmi vous les disputes et les querelles ? N'est-ce pas de vos voluptés qui combattent dans vos membres ?
\VS{2}Vous convoitez, et vous n'obtenez pas ce que vous désirez ; vous avez une envie mortelle, vous êtes jaloux, et vous ne pouvez obtenir ce que vous enviez ; vous vous querellez, vous vous disputez, et vous n'avez pas ce que vous désirez, parce que vous ne demandez pas.
\VS{3}Vous demandez, et vous ne recevez point, parce que vous demandez mal, dans le but de satisfaire vos voluptés.
\VS{4}Hommes et femmes adultères ! Ne savez-vous pas que l'amitié du monde est inimitié contre Dieu ? Celui donc qui veut être ami du monde, se rend ennemi de Dieu.
\VS{5}Pensez-vous que l'Ecriture parle en vain ? L'Esprit qui habite en nous, vous inspire-t-il l'envie ?
\TextTitle{S'humilier devant Yahweh, le juste Juge}
\VS{6}Il vous accorde, au contraire, une plus grande grâce ; c'est pourquoi l'Ecriture dit : Dieu résiste aux orgueilleux, mais il fait grâce aux humbles\FTNT{Pr. 3:34.}.
\VS{7}Soumettez-vous donc à Dieu ; résistez au diable, et il s'enfuira de vous.
\VS{8}Approchez-vous de Dieu, et il s'approchera de vous. Pécheurs, nettoyez vos mains ; et vous qui êtes doubles de cœur, purifiez vos cœurs.
\VS{9}Sentez vos misères ; et soyez dans le deuil et dans les larmes ; que votre rire se change en pleurs, et votre joie en tristesse.
\VS{10}Humiliez-vous dans la présence du Seigneur, et il vous élèvera.
\VS{11}Mes frères, ne médisez point les uns des autres. Celui qui médit de son frère, et qui condamne son frère, médit de la loi et juge la loi. Or, si tu juges la loi, tu n'es pas observateur de la loi, mais le juge.
\VS{12}Il n'y a qu'un seul Législateur, qui peut sauver et qui peut perdre ; mais toi, qui es-tu, qui juges les autres ?
\TextTitle{Abandonner ses désirs au profit de la volonté de Dieu}
\VS{13}A vous, maintenant, qui dites : Aujourd'hui ou demain nous irons dans telle ou telle ville, et nous y passerons une année, et nous trafiquerons et nous gagnerons !
\VS{14}Qui toutefois ne savez pas ce qui arrivera le lendemain ! Car qu'est-ce que votre vie ? Ce n'est certes qu'une vapeur qui parait pour un peu de temps, et qui ensuite s'évanouit.
\VS{15}Au lieu de dire : Si le Seigneur le veut, et si nous vivons, nous ferons aussi ceci ou cela.
\VS{16}Mais maintenant vous vous glorifiez dans vos pensées orgueilleuses. Une telle gloire est mauvaise.
\VS{17}Il y a donc du péché en celui qui sait faire le bien, et qui ne le fait pas.
\Chap{5}
\TextTitle{Avertissement aux riches}
\VerseOne{}A vous maintenant, riches ! Pleurez et gémissez à cause des malheurs qui vont tomber sur vous.
\VS{2}Vos richesses sont pourries et vos vêtements sont rongés par les vers.
\VS{3}Votre or et votre argent sont rouillés ; et leur rouille s'élèvera en témoignage contre vous et dévorera vos chairs comme un feu. Vous avez amassé des trésors pour les derniers jours.
\VS{4}Voici, le salaire des ouvriers qui ont moissonné vos champs, et dont vous les avez frustrés, crie ; et les cris des moissonneurs sont parvenus aux oreilles du Seigneur des armées.
\VS{5}Vous avez vécu dans les délices sur la terre, vous vous êtes livrés aux voluptés, et vous avez rassasié vos cœurs comme en un  jour de sacrifices.
\VS{6}Vous avez condamné et mis à mort le juste qui ne vous a pas résisté.
\TextTitle{Se préparer à l'avènement du Seigneur}
\VS{7}Mais vous, mes frères, attendez patiemment jusqu'à l'avènement du Seigneur. Voici, le laboureur attend le précieux fruit de la terre, prenant patience à son égard, jusqu'à ce qu'il ait reçu les pluies de la première et de la dernière saison.
\VS{8}Vous aussi, attendez patiemment, et affermissez vos cœurs, car l'avènement du Seigneur est proche.
\VS{9}Mes frères, ne vous plaignez pas les uns des autres, afin que vous ne soyez pas condamnés. Voici, le Juge se tient à la porte.
\VS{10}Mes frères, prenez pour exemple de patience dans les afflictions les prophètes qui ont parlé au Nom du Seigneur.
\VS{11}Voici, nous tenons pour bienheureux ceux qui ont enduré l'épreuve avec patience. Vous avez appris quelle a été la patience de Job, et vous avez vu la fin du Seigneur, car le Seigneur est plein de compassion et de miséricorde.
\VS{12}Avant toutes choses, mes frères, ne jurez ni par le ciel, ni par la terre, ni par aucun autre serment. Mais que votre oui soit oui, et que votre non soit non, afin que vous ne tombiez pas sous le jugement\FTNT{Mt. 5:37 ; Mt. 12:36.}.
\VS{13}Quelqu'un parmi vous est-il dans la souffrance ? Qu'il prie. Quelqu'un est-il dans la joie ? Qu'il chante.
\VS{14}Quelqu'un parmi vous est-il malade ? Qu'il appelle les anciens de l'église, et qu'ils prient pour lui en l'oignant d'huile au Nom du Seigneur.
\VS{15}Et la prière faite avec foi sauvera le malade, et le Seigneur le relèvera ; et s'il a commis des péchés, ils lui seront pardonnés.
\VS{16}Confessez donc vos péchés les uns les autres, et priez les uns pour les autres afin que vous soyez guéris. Car la prière du juste faite avec ferveur est de grande efficacité.
\VS{17}Elie était un homme sujet aux mêmes infirmités que nous, et cependant il pria avec instance pour qu'il ne pleuve point, et il ne tomba point de pluie sur la terre pendant trois ans et six mois\FTNT{1 R. 17:1.}.
\VS{18}Puis il pria de nouveau, et le ciel donna de la pluie, et la terre produisit son fruit.
\TextTitle{Conclusion}
\VS{19}Mes frères, si quelqu'un parmi vous s'est égaré loin de la vérité, et qu'un autre l'y ramène,
\VS{20}qu'il sache que celui qui ramènera un pécheur de son égarement, sauvera une âme de la mort et couvrira une multitude de péchés.
\PPE{}
\end{multicols}

%\clearpage\ShortTitle{Ga.}\BookTitle{Galates}\BFont
\noindent\hrulefill
{\footnotesize
\textit{
\bigskip
{\centering{}
\\Auteur~: Paul
\\Thème~: Le salut par la grâce
\\Date de rédaction~: Env. 50 ap. J.-C.\\}
}
\textit{
\\Province antique de l'Asie Mineure, la Galatie se situait en Anatolie. Elle devait son nom aux Galates, Celtes provenant des Balkans.
\\La lettre de Paul aux Galates est la seule épître dont le début ne contient pas de témoignage d'affection. Paul commence par justifier l'origine de son appel, en employant un ton sec et sévère. Les Galates, qu'il avait lui-même évangélisés lors de son premier voyage, s'étaient promptement détournés de l'Evangile qu'ils avaient reçu. Ils ne l'avaient pas totalement abandonné, mais y avaient ajouté ce qui ne leur avait point été prescrit. Troublés par les enseignements des judaïsants - des juifs ayant cru en Jésus-Christ, mais persistant toujours dans la pratique de la loi - les Galates avaient repris à leur compte leurs traditions, annihilant ainsi l'œuvre de la croix. Par cette lettre, Paul les exhorte d'une part à revenir à l'Evangile véritable et d'autre part à marcher par l'Esprit afin d'en porter le fruit.\bigskip
}
}
\par\nobreak\noindent\hrulefill
\begin{multicols}{2}
\Chap{1}
\TextTitle{Introduction}
\VerseOne{}Paul, apôtre, non de la part des hommes, ni de la part d'aucun homme, mais de la part de Jésus-Christ, et de la part de Dieu le Père, qui l'a ressuscité des morts,
\VS{2}et tous les frères qui sont avec moi, aux églises de Galatie\FTNT{La Galatie, ou Gallo-Grèce, était une province de l'Asie Mineure (région de la Turquie actuelle). Au nord, elle était délimitée par la Bithynie et la
Paphlagonie, à l'est par le Pont et la Cappadoce, au sud par la Cappadoce, la Lycaonie et la Phrygie, et à l'ouest par la Phrygie et la Bithynie. Son nom vient des Gaulois qui s'étaient installés dans la région en 279 av. J.-C. Conquise par les Romains en 189 av. J.-C., elle devint une province de l'Empire en 25 av. J.-C.} :
\VS{3}Que la grâce et la paix vous soient données de la part de Dieu le Père, et de la part de notre Seigneur Jésus-Christ,
\VS{4}qui s'est donné lui-même pour nos péchés, afin de nous arracher du présent siècle mauvais, selon la volonté de Dieu notre Père.
\VS{5}A lui soit la gloire aux siècles des siècles. Amen~!
\TextTitle{Les Galates se détournent de l'Evangile véritable}
\VS{6}Je m'étonne que vous abandonniez si promptement celui qui vous avait appelés à la grâce de Christ, pour passer à un autre évangile. 
\VS{7}Non qu'il y ait un autre évangile, mais il y a des gens qui vous troublent, et qui veulent renverser l'Evangile de Christ.
\VS{8}Mais quand nous-mêmes, ou quand un ange venu du ciel vous évangéliserait, outre\FTNT{Voir 1 R. 13:11-34.} ce que nous vous avons évangélisé, qu'il soit anathème~!
\VS{9}Comme nous l'avons déjà dit, je le dis encore maintenant~: Si quelqu'un vous évangélise outre ce que vous avez reçu, qu'il soit anathème~!
\TextTitle{Paul reçoit la révélation de l'Evangile}
\VS{10}Car est-ce les hommes que je prêche ou Dieu~? Ou est-ce que je cherche à plaire aux hommes~? Certes si je plaisais encore aux hommes, je ne serais pas le serviteur de Christ.
\VS{11}Je vous le déclare donc, mes frères, que l'Evangile que j'ai annoncé n'est pas selon l'homme,
\VS{12}parce que je ne l'ai ni reçu ni appris d'aucun homme, mais par la révélation de Jésus-Christ.
\VS{13}Car vous avez appris quelle a été autrefois ma conduite dans le judaïsme, et comment je persécutais à outrance l'Eglise de Dieu et la ravageais,
\VS{14}et comment j'étais plus avancé dans le judaïsme que beaucoup de ceux de mon âge et de ma nation, étant le plus ardent zélateur des traditions de mes pères.
\VS{15}Mais quand il a plu à Dieu, qui m'avait choisi dès le ventre de ma mère, et qui m'a appelé par sa grâce,
\VS{16}de révéler en moi son Fils, afin que je le prêche parmi les Gentils, aussitôt, je ne consultai ni la chair ni le sang,
\VS{17}et je ne montai point à Jérusalem vers ceux qui furent apôtres avant moi, mais je partis pour l'Arabie, puis je revins encore à Damas.
\VS{18}Ensuite, trois ans après, je montai à Jérusalem pour visiter Pierre, et je demeurai chez lui quinze jours.
\VS{19}Et je ne vis aucun des autres apôtres, sinon Jacques, le frère du Seigneur.
\VS{20}Or, dans les choses que je vous écris, voici, devant Dieu je vous dis que je ne mens point.
\VS{21}J'allai ensuite dans les pays de Syrie et de Cilicie.
\VS{22}Or j'étais inconnu de visage aux églises de Judée qui sont en Christ,
\VS{23}mais elles avaient seulement entendu dire~: Celui qui autrefois nous persécutait, annonce maintenant la foi qu'il détruisait autrefois.
\VS{24}Et elles glorifiaient Dieu à cause de moi.
\Chap{2}
\TextTitle{Paul et Barnabas se rendent à Jérusalem\FTNTT{Ac. 15.}}
\VerseOne{}Quatorze ans après, je montai de nouveau à Jérusalem\FTNT{La grande assemblée de Jérusalem. Voir Ac. 15.}, avec Barnabas, et je pris aussi avec moi Tite.
\VS{2}Et ce fut d'après une révélation que j'y montai. J'exposai l'Evangile que je prêche parmi les Gentils à ceux de Jérusalem, en particulier à ceux qui sont les plus considérés, afin de ne pas courir ou avoir couru en vain.
\VS{3}Et même on n'obligea pas Tite, qui était avec moi, de se faire circoncire quoiqu'il fût Grec.
\VS{4}Et cela à cause des faux frères qui s'étaient furtivement introduits et glissés dans l'église pour épier la liberté que nous avons en Jésus-Christ, afin de nous ramener dans la servitude.
\VS{5}Nous ne leur cédâmes pas un instant et nous résistâmes à leurs exigences, afin que la vérité de l'Evangile soit maintenue parmi vous.
\VS{6}Et je ne suis différent en rien de ceux qui sont les plus estimés, quels qu'ils aient été autrefois, Dieu n'ayant point d'égard à l'apparence extérieure de l'homme car ceux qui sont en estime ne m'ont rien communiqué de plus.
\VS{7}Au contraire, quand ils virent que la prédication de l'Evangile pour les incirconcis m'avait été confiée, comme à Pierre pour les circoncis,
\VS{8}car celui qui a opéré avec efficacité par Pierre dans la charge d'apôtre pour les circoncis, a aussi opéré avec efficacité par moi envers les Gentils.
\VS{9}Jacques, dis-je, Céphas, et Jean, qui sont estimés comme des colonnes, ayant reconnu la grâce que j'avais reçue, me donnèrent, à moi et à Barnabas, la main d'association, afin que nous allions, nous vers les Gentils, et qu'ils aillent eux vers les circoncis.
\VS{10}Ils nous recommandèrent seulement de nous souvenir des pauvres, ce que j'ai eu bien soin de faire.
\TextTitle{Paul reprend Pierre à Antioche}
\VS{11}Mais lorsque Pierre vint à Antioche, je lui résistai en face parce qu'il méritait d'être repris.
\VS{12}Car avant l'arrivée de quelques personnes envoyées par Jacques, il mangeait avec les Gentils, mais quand elles furent venues, il s'esquiva et se sépara des Gentils, craignant les circoncis.
\VS{13}Les autres Juifs aussi usèrent de dissimulation comme lui, de sorte que Barnabas même se laissait entraîner par leur hypocrisie.
\VS{14}Mais quand je vis qu'ils ne marchaient pas droit selon la vérité de l'Evangile, je dis à Pierre devant tous~: Si toi qui es Juif, tu vis comme les Gentils, et non pas comme les Juifs, pourquoi contrains-tu les Gentils à judaïser~?
\TextTitle{Le chrétien est mort à la loi mosaïque}
\VS{15}Nous qui sommes Juifs de naissance, et non point pécheurs d'entre les Gentils,
\VS{16}sachant que l'homme n'est pas justifié par les œuvres de la loi, mais seulement par la foi en Jésus-Christ\FTNT{La justification. Voir Ro. 5:1.}, nous, dis-je, nous avons cru en Jésus-Christ, afin que nous soyons justifiés par la foi en Christ, et non point par les œuvres de la loi~; parce que personne ne sera justifié par les œuvres de la loi.
\VS{17}Or si en cherchant à être justifiés par Christ, nous sommes aussi trouvés pécheurs, Christ est-il pourtant serviteur du péché~? A Dieu ne plaise~!
\VS{18}Car si je rebâtis les choses que j'ai renversées, je montre que je suis moi-même un transgresseur.
\VS{19}Car c'est par la loi que je suis mort à la loi, afin de vivre pour Dieu.
\TextTitle{La vie chrétienne doit refléter la vie de Jésus-Christ\FTNTT{Ga. 5:15-23.}}
\VS{20}Je suis crucifié avec Christ~; et si je vis, ce n'est plus moi qui vis, c'est Christ qui vit en moi~; si je vis maintenant dans la chair, je vis dans la foi au Fils de Dieu, qui m'a aimé et qui s'est livré lui-même pour moi.
\VS{21}Je n'anéantis point la grâce de Dieu, car si la justice vient de la loi, Christ est donc mort inutilement.
\Chap{3}
\TextTitle{L'Esprit s'acquiert par la foi}
\VerseOne{}Ô Galates insensés~! Qui vous a ensorcelés pour faire que vous n'obéissiez point à la vérité, vous, aux yeux de qui Jésus-Christ a été auparavant dépeint crucifié, au milieu de vous~?
\VS{2}Je voudrais seulement entendre ceci de vous~: Avez-vous reçu l'Esprit par les œuvres de la loi, ou par la prédication de la foi~?
\VS{3}Etes-vous si insensés, qu'après avoir commencé par l'Esprit, voulez-vous maintenant finir par la chair~?
\VS{4}Avez-vous tant souffert en vain~? Si toutefois c'est en vain.
\VS{5}Celui donc qui vous donne l'Esprit, et qui produit en vous les dons miraculeux, le fait-il par les œuvres de la loi ou par la prédication de la foi~?
\TextTitle{L'alliance avec Abraham, une promesse fondée sur la foi\FTNTT{Ro. 4.}}
\VS{6}Comme Abraham crut à Dieu, et cela lui fut imputé à justice,
\VS{7}sachez donc que ce sont ceux qui ont la foi qui sont fils d'Abraham.
\VS{8}Aussi, l'Ecriture prévoyant que Dieu justifierait les Gentils par la foi, a auparavant évangélisé à Abraham, en lui disant~: Toutes les nations seront bénies en toi\FTNT{Ge. 12:3.}.
\VS{9}C'est pourquoi ceux qui ont la foi sont bénis avec Abraham, le croyant.
\TextTitle{L'attachement aux œuvres de la loi produit la malédiction}
\VS{10}Car tous ceux qui s'attachent aux œuvres de la loi sont sous la malédiction~; car il est écrit~: Maudit est quiconque ne persévère pas dans toutes les choses qui sont écrites dans le livre de la loi et ne les met pas en pratique\FTNT{De. 27:26.}.
\VS{11}Et que nul ne soit justifié devant Dieu par la loi, cela est évident, puisqu'il est dit~: Le juste vivra de la foi\FTNT{Ha. 2:4.}.
\VS{12}Or la loi ne procède pas de la foi, mais elle dit~: L'homme qui mettra ces choses en pratique vivra par elles\FTNT{Lé. 18:5.}.
\TextTitle{Le Messie a racheté les chrétiens de la malédiction de la loi}
\VS{13}Christ nous a rachetés de la malédiction de la loi quand il a été fait malédiction pour nous~; car il est écrit~: Maudit est quiconque est pendu au bois\FTNT{De. 21:23.},
\VS{14}afin que la bénédiction d'Abraham ait son accomplissement pour les Gentils en Jésus-Christ, et que nous recevions par la foi l'Esprit qui avait été promis.
\VS{15}Mes frères, je parle à la manière des hommes, un testament en bonne forme, bien que fait par un homme, n'est annulé par personne, et personne n'y ajoute.
\VS{16}Or les promesses ont été faites à Abraham et à sa postérité. Il n'est pas dit~: Et aux postérités, comme s'il avait parlé de plusieurs, mais comme parlant d'une seule, et à sa postérité, c'est-à-dire Christ.
\VS{17}Voici ce que j'entends~: Une alliance, que Dieu a confirmée antérieurement, ne peut pas être annulée, et ainsi la promesse rendue vaine, par la loi survenue quatre cent trente ans plus tard.
\VS{18}Car si l'héritage venait de la loi, il ne viendrait plus de la promesse. Or c'est par la promesse que Dieu a fait à Abraham ce don de sa grâce.
\TextTitle{La loi~: Pédagogue révélant le péché et conduisant à Christ}
\VS{19}A quoi donc sert la loi~? Elle a été donnée ensuite à cause des transgressions, jusqu'à ce que vienne la postérité à qui la promesse avait été faite~; et elle a été promulguée par des anges, au moyen d'un médiateur.
\VS{20}Or le médiateur n'est pas médiateur d'un seul, mais Dieu est un seul.
\VS{21}La loi a-t-elle donc été ajoutée contre les promesses de Dieu~? Nullement ! Car s'il avait été donné une loi qui puisse procurer la vie, la justice viendrait réellement de la loi.
\VS{22}Mais l'Ecriture a renfermé tous les hommes sous le péché, afin que ce qui avait été promis soit donné par la foi en Jésus-Christ à ceux qui croient.
\VS{23}Or avant que la foi vienne, nous étions renfermés sous la garde de la loi, en vue de la foi qui devait être révélée.
\VS{24}Ainsi la loi a donc été notre pédagogue\FTNT{Le mot «~pédagogue~» du grec «~paidagogos~»~: «~celui qui dirige un garçon~». Un pédagogue était un tuteur, un gardien et un guide de garçons. Parmi les Grecs et les Romains, le mot était appliqué aux esclaves dignes de confiance qui étaient chargés de veiller à la vie et à la moralité des garçons appartenant aux classes supérieures. Les garçons ne pouvaient faire le moindre pas hors de la maison sans ces tuteurs tant qu'ils n'avaient pas atteint leur majorité.} pour nous amener à Christ, afin que nous soyons justifiés par la foi.
\VS{25}Mais la foi étant venue, nous ne sommes plus sous ce pédagogue.
\TextTitle{Ceux qui croient au Messie sont justifiés}
\VS{26}Parce que vous êtes tous fils de Dieu par la foi en Jésus-Christ,
\VS{27}car vous tous qui avez été baptisés en Christ, vous avez revêtu Christ.
\VS{28}Il n'y a plus ni Juif ni Grec, il n'y a plus ni esclave ni libre, il n'y a plus ni homme ni femme~; car vous êtes tous un en Jésus-Christ\FTNT{Ro. 10:12~; Col. 3:11.}.
\VS{29}Et si vous êtes de Christ, vous êtes donc la postérité d'Abraham, et héritiers selon la promesse.
\Chap{4}
\VerseOne{}Or aussi longtemps que l'héritier est enfant\FTNT{«~Enfant~», du grec «~nepios~», signifie aussi «~ignorant~».}, je dis qu'il ne diffère en rien d'un esclave, quoiqu'il soit le maître de tout.
\VS{2}Mais il est sous des tuteurs et des administrateurs jusqu'au temps déterminé par le Père.
\VS{3}Nous aussi, lorsque nous étions enfants, nous étions sous l'esclavage des rudiments du monde.
\VS{4}Mais lorsque les temps ont été accomplis, Dieu a envoyé son Fils, né d'une femme, né sous la loi,
\VS{5}afin qu'il rachète ceux qui étaient sous la loi, afin que nous recevions l'adoption.
\VS{6}Et parce que vous êtes fils, Dieu a envoyé l'Esprit de son Fils dans vos cœurs, lequel crie~: Abba ! C'est-à-dire Père.
\VS{7}Maintenant donc tu n'es plus esclave, mais fils~; or si tu es fils, tu es aussi héritier de Dieu par Christ.
\TextTitle{Le légalisme et la religiosité privent de la grâce}
\VS{8}Autrefois, ne connaissant pas Dieu, vous serviez des dieux qui ne le sont pas de leur nature.
\VS{9}Et maintenant que vous avez connu Dieu, ou plutôt que vous avez été connus de Dieu, comment retournez-vous encore à ces faibles et misérables éléments, auxquels vous voulez encore vous asservir comme auparavant~?
\VS{10}Vous observez les jours, les mois, les temps et les années.
\VS{11}Je crains d'avoir travaillé inutilement pour vous.
\VS{12}Soyez comme moi~; car je suis aussi comme vous~; je vous en prie mes frères.
\VS{13}Vous ne m'avez fait aucun tort. Et vous savez que ce fut à cause d'une infirmité de la chair\FTNT{Les Ecritures ne donnent pas de précisions au sujet de l'infirmité de la chair dont souffrait Paul. On suppose toutefois qu'il avait un handicap au niveau de ses yeux. Quatre arguments viennent renforcer cette hypothèse. Tout d'abord, l'allusion de Paul aux Galates qui étaient prêts à «~s'arracher les yeux~» pour les lui donner (Ga. 4:15) et le fait qu'il ait lui-même écrit cette épître avec de «~grandes lettres~» (Ga. 6:11). Ensuite, lors de sa comparution devant le sanhédrin à Jérusalem, Paul n'a pas reconnu le grand-prêtre pourtant facilement identifiable par sa tenue vestimentaire (Ac. 23:5). Enfin, l'apôtre avait l'habitude de dicter ses lettres, ce qui constitue un argument majeur. L'épître aux Galates était une exception parce qu'il n'avait sans doute pas de secrétaire à disposition.} que je vous ai pour la première fois évangélisés.
\VS{14}Et vous ne m'avez point méprisé ni rejeté à cause de ces épreuves que j'ai dans ma chair~; mais vous m'avez reçu comme un ange de Dieu, et comme Jésus-Christ.
\VS{15}Où donc est l'expression de votre bonheur~? Car je vous atteste que, si cela avait été possible, vous vous seriez arrachés les yeux pour me les donner.
\VS{16}Suis-je donc devenu votre ennemi en vous disant la vérité~?
\VS{17}Ils ont du zèle pour vous, mais non loyalement. Au contraire, ils veulent vous détacher de nous afin que vous soyez zélés pour eux.
\VS{18}Il est bon d'être zélé pour le bien en tout temps, et non pas seulement quand je suis présent parmi vous.
\TextTitle{La loi et la grâce ne peuvent cohabiter~: Agar et Sara représentent deux alliances}
\VS{19}Mes petits enfants, pour qui j'éprouve de nouveau les douleurs de l'enfantement, jusqu'à ce que Christ soit formé en vous,
\VS{20}je voudrais être maintenant avec vous, et changer de langage, car je suis dans une grande inquiétude à votre sujet.
\VS{21}Dites-moi, vous qui voulez être sous la loi, ne comprenez-vous point la loi~?
\VS{22}Car il est écrit qu'Abraham eut deux fils, un de l'esclave, et un de la femme libre.
\VS{23}Mais celui de l'esclave naquit selon la chair~; et celui de la femme libre naquit en vertu de la promesse.
\VS{24}Ces faits ont une valeur allégorique, car ces deux femmes sont deux alliances~: L'une du Mont Sinaï, qui n'enfante que des esclaves, et c'est Agar.
\VS{25}Car le nom d'Agar veut dire Sinaï, qui est une montagne en Arabie correspondant à la Jérusalem actuelle qui est dans la servitude avec ses enfants.
\VS{26}Mais la Jérusalem d'en haut est la femme libre, et c'est notre mère à nous tous.
\VS{27}Car il est écrit~: Réjouis-toi, stérile, toi qui n'enfantes point~! Eclate et pousse des cris, toi qui n'as pas éprouvé les douleurs de l'enfantement~! Car les enfants de la délaissée seront plus nombreux que les enfants de celle qui était mariée\FTNT{Es. 54:1.}.
\VS{28}Or pour nous, mes frères, nous sommes enfants de la promesse comme Isaac.
\VS{29}Et de même qu'alors, celui qui était né selon la chair persécutait celui qui était né selon l'Esprit, il en est de même maintenant.
\VS{30}Mais que dit l'Ecriture~? Chasse l'esclave et son fils, car le fils de l'esclave n'héritera pas avec le fils de la femme libre\FTNT{Ge. 21:10.}.
\VS{31}C'est pourquoi, mes frères, nous ne sommes pas enfants de l'esclave, mais de la femme libre.
\Chap{5}
\TextTitle{Le Messie nous a libérés de la servitude}
\VerseOne{}Demeurez donc fermes dans la liberté pour laquelle Christ nous a affranchis, et ne vous mettez plus sous le joug de la servitude.
\VS{2}Moi, Paul, je vous dis que si vous vous faites circoncire, Christ ne vous servira à rien.
\VS{3}Et j'affirme encore une fois à tout homme qui se fait circoncire qu'il est tenu de pratiquer la loi tout entière.
\VS{4}Vous êtes séparés de Christ, vous tous qui cherchez la justification dans la loi~; vous êtes déchus de la grâce.
\VS{5}Mais pour nous, nous attendons par l'Esprit l'espérance d'être justifiés par la foi.
\VS{6}Car en Jésus-Christ ni la circoncision ni le prépuce\FTNT{Voir le commentaire en 1 Co. 7:18.} n'ont de valeur, mais seulement la foi qui opère par la charité.
\VS{7}Vous couriez bien~: Qui vous a arrêtés pour vous empêcher d'obéir à la vérité~?
\VS{8}Cette influence ne vient pas de celui qui vous appelle.
\VS{9}Un peu de levain fait lever toute la pâte\FTNT{1 Co. 5:6.}.
\VS{10}J'ai cette confiance en vous dans le Seigneur que vous n'aurez pas d'autre sentiment~; mais celui qui vous trouble, quel qu'il soit, en portera la condamnation.
\VS{11}Quant à moi, mes frères, si je prêche encore la circoncision, pourquoi suis-je encore persécuté~? Le scandale de la croix est donc aboli.
\VS{12}Plaise à Dieu que ceux qui vous troublent soient retranchés~!
\VS{13}Car, mes frères, vous avez été appelés à la liberté, seulement ne faites pas de cette liberté une occasion de vivre selon la chair, mais servez-vous les uns les autres avec charité.
\VS{14}Car toute la loi est accomplie dans cette seule parole~: Tu aimeras ton prochain comme toi-même\FTNT{Lé. 19:18~; Mt. 22:39.}.
\VS{15}Mais si vous vous mordez et vous dévorez les uns les autres, prenez garde que vous ne soyez détruits les uns par les autres.
\VS{16}Je vous dis donc~: Marchez selon l'Esprit, et vous n'accomplirez point les désirs de la chair.
\TextTitle{La chair et ses œuvres s'opposent à l'Esprit de Dieu\FTNTT{Ro. 8:2.}}
\VS{17}Car la chair a des désirs contraires à ceux de l'Esprit, et l'Esprit en a des contraires à ceux de la chair~; et ils sont opposés entre eux afin que vous ne fassiez point ce que vous voudriez.
\VS{18}Or si vous êtes conduits par l'Esprit, vous n'êtes point sous la loi.
\VS{19}Car les œuvres de la chair sont évidentes~: Ce sont l'adultère, la fornication, l'impureté, l'impudicité,
\VS{20}l'idolâtrie, la sorcellerie\FTNT{La sorcellerie~: du grec «~pharmakeia~»~: «~usage~» ou «~administration~» de drogues, «~empoisonnement~», «~sorcellerie~», «~arts magiques~», souvent trouvés en liaison avec l'idolâtrie et nourrie par celle-ci.}, les inimitiés, les querelles, les jalousies, les animosités, les disputes, les divisions, les sectes,
\VS{21}les envies, les meurtres, l'ivrognerie, les excès de table, et les choses semblables à celles-là, au sujet desquelles je vous prédis, comme je vous l'ai déjà dit, que ceux qui commettent de telles choses n'hériteront point le Royaume de Dieu.
\TextTitle{Le fruit de l'Esprit\FTNTT{Jn. 15:1-5~; Ga. 2:20.}}
\VS{22}Mais le fruit de l'Esprit c'est la charité\FTNT{Il est question ici de l'amour «~agape~»~: l'amour fraternel, la charité désintéressée.}, la joie, la paix, la patience, la bonté, la bienveillance, la foi, la douceur, la tempérance.
\VS{23}La loi n'est pas contre ces choses.
\VS{24}Ceux qui sont à Christ ont crucifié la chair avec ses passions et ses désirs.
\VS{25}Si nous vivons par l'Esprit, marchons aussi par l'Esprit.
\VS{26}Ne cherchons pas une vaine gloire, en nous provoquant les uns les autres et en nous portant envie les uns aux autres.
\Chap{6}
\TextTitle{La mise en pratique de la vie nouvelle en Jésus-Christ}
\VerseOne{}Mes frères, lorsqu'un homme est surpris en quelque faute, vous qui êtes spirituels, redressez-le avec un esprit de douceur. Prends garde à toi-même, de peur que tu ne sois aussi tenté.
\VS{2}Portez les fardeaux les uns des autres, et vous accomplirez ainsi la loi de Christ.
\VS{3}Car si quelqu'un pense être quelque chose, quoiqu'il ne soit rien, il s'abuse lui-même.
\VS{4}Que chacun examine ses propres œuvres, et alors il aura de quoi se glorifier pour lui-même seulement, et non par rapport aux autres.
\VS{5}Car chacun portera son propre fardeau.
\VS{6}Que celui à qui l'on enseigne la parole fasse part en tous biens à celui qui l'enseigne\FTNT{Le mot «~bien~» vient du grec «~agathos~» qui donne en français~: «~de bonne constitution ou nature~», «~utile~», «~salutaire~», «~bon~», «~agréable~», «~plaisant~», «~joyeux~», «~heureux~», «~excellent~», «~distingué~», «~droit~», «~honorable~» et n'a rien à voir avec les biens matériels (Voir Ga. 6:10). Il ne doit en aucun cas servir de prétexte à ceux qui enseignent la Parole de Dieu pour exiger l'argent et les biens matériels des chrétiens. Ces derniers doivent donner sans contrainte, s'ils le veulent et comme ils le veulent (2 Co. 9:7). Le salaire de l'ouvrier du Seigneur c'est avant tout le gîte et le couvert (Mt. 10:10~; Lu. 10:8~; 1 Ti. 6:8). Ainsi, malgré le droit qu'il avait de moissonner les biens matériels pour avoir semé des biens spirituels (1 Co. 9:11-12), Paul «~n'a désiré ni l'or ni l'argent~» mais a travaillé de ses propres mains afin de pourvoir à ses besoins et de n'être à la charge de personne (Ac. 20:33-35~; 1 Th. 2:9~; 2 Th. 3:8~; 2 Co. 12:14 ).}.
\VS{7}Ne vous séduisez pas, on ne se moque pas de Dieu. Ce qu'un homme aura semé, il le moissonnera aussi.
\VS{8}C'est pourquoi celui qui sème pour sa chair moissonnera de la chair la corruption~; mais celui qui sème pour l'Esprit moissonnera de l'Esprit la vie éternelle.
\VS{9}Ne nous lassons pas de faire le bien~; car nous moissonnerons au temps convenable, si nous ne nous relâchons pas.
\VS{10}C'est pourquoi, pendant que nous en avons le temps, faisons du bien envers tous, mais principalement envers ceux qui sont de la famille de la foi.
\VS{11}Vous voyez avec quelles grandes lettres je vous ai écrit de ma propre main.
\VS{12}Tous ceux qui veulent se rendre agréables selon la chair vous contraignent à vous faire circoncire, uniquement afin de ne pas être persécutés pour la croix de Christ.
\VS{13}Car les circoncis eux-mêmes n'observent pas la loi~; mais ils veulent que vous soyez circoncis pour se glorifier dans votre chair.
\VS{14}Pour ce qui me concerne, loin de moi la pensée de me glorifier d'autre chose que de la croix de notre Seigneur Jésus-Christ, par qui le monde est crucifié pour moi, comme je le suis pour le monde~!
\VS{15}Car ce n'est rien que d'être circoncis ou incirconcis~; ce qui est quelque chose c'est d'être une nouvelle créature.
\VS{16}Que la paix et la miséricorde soient sur tous ceux qui suivront cette règle, et sur l'Israël de Dieu !
\TextTitle{Conclusion}
\VS{17}Au reste, que personne ne me fasse de la peine, car je porte sur mon corps les marques du Seigneur Jésus.
\VS{18}Mes frères, que la grâce de notre Seigneur Jésus-Christ soit avec votre esprit~! Amen~!
\PPE{}
\end{multicols}

%\clearpage\ShortTitle{1 Th.}\BookTitle{1 Thessaloniciens}\BFont
\noindent\hrulefill
{\footnotesize
\textit{
\bigskip
{\centering{}
\\Auteur~: Paul
\\Thème~: Le retour de Christ
\\Date de rédaction~: Env. 51 ap. J.-C.\\}
}
\textit{
\\Autrefois appelée Therme ou Therma, qui signifie «~source chaude~», Thessalonique reçut son nouveau nom de Cassandre, en l'honneur de sa femme Thessalonike, qui était aussi la sœur d'Alexandre le Grand (356 av. J.-C. - 323 av. J.-C.), à qui il succéda. Cette ville est située au nord de la Grèce actuelle, sur la côte de la mer Egée. Du temps de Paul, ce pays était divisé en deux parties. Dans la région du nord, la Macédoine, se trouvaient les villes de Philippes, Thessalonique et Bérée. Quant à la région du sud, l'Achaïe, elle comportait les villes d'Athènes et de Corinthe. Aujourd'hui, la ville s'appelle Salonique.
\\En ce temps-là, Thessalonique comptait environ 200 000 habitants (Grecs, Romains et Juifs) et jouissait d'une importante fréquentation puisqu'elle figurait parmi les trois ports principaux de la Méditerranée et se situait sur l'une des plus grandes routes commerciales de l'époque~: La Voie Egnatienne reliant Rome à Byzance.
\\Sur le plan religieux, les habitants étaient polythéistes et pratiquaient une variété de cultes, dont le culte impérial. Durant trois semaines, Paul enseigna dans une synagogue à Thessalonique et réussit à constituer un groupe de croyants composé de Juifs, de Gentils, de pauvres et de plusieurs femmes de la haute société. Toutefois, une violente persécution l'obligea à quitter promptement la ville, laissant la communauté nouvellement formée vulnérable et fragile.
\\La première épître adressée par Paul aux Thessaloniciens leur parvint quelques mois après le passage de l'équipe apostolique et après la visite de Timothée. Cette lettre avait pour but d'affermir les Thessaloniciens dans les vérités fondamentales qui leur avaient été enseignées, de les exhorter à vivre une vie de sainteté pour être agréables à Dieu, de les éclairer quant au devenir des défunts et de les assurer du retour certain du Seigneur.\bigskip
}
}
\par\nobreak\noindent\hrulefill
\begin{multicols}{2}
\Chap{1}
\TextTitle{Introduction}
\VerseOne{}Paul, et Silvain, et Timothée, à l'église des Thessaloniciens qui est en Dieu le Père, et en Jésus-Christ, notre Seigneur~: Que la grâce et la paix vous soient données de la part de Dieu notre Père, et du Seigneur Jésus-Christ~!
\VS{2}Nous rendons toujours grâces à Dieu pour vous tous, faisant mention de vous dans nos prières,
\VS{3}en nous rappelant sans cesse l'œuvre de votre foi, le travail de votre charité, et l'immuabilité de votre espérance en notre Seigneur Jésus-Christ devant notre Dieu et Père,
\VS{4}sachant, mes frères bien-aimés de Dieu, votre élection.
\TextTitle{Proclamation de l'Evangile avec puissance et avec l'Esprit Saint}
\VS{5}Car notre Evangile ne vous a pas été prêché en paroles seulement, mais aussi en puissance, avec l'Esprit Saint, et avec une pleine persuasion~; car vous n'ignorez pas que nous nous sommes montrés ainsi parmi vous, à cause de vous.
\VS{6}Aussi avez-vous été nos imitateurs et ceux du Seigneur, ayant reçu avec la joie du Saint-Esprit, la parole au milieu de grandes afflictions,
\VS{7}de sorte que vous avez été des modèles à tous les fidèles de la Macédoine\FTNT{La Macédoine était le pays natal d'Alexandre le Grand. Elle fut conquise par les Romains et devint une province romaine, dont la capitale était Thessalonique.} et de l'Achaïe\FTNT{L'Achaïe était une province romaine placée sous l'autorité d'un proconsul résidant dans la capitale qui était Corinthe (2 Co. 1:1).}.
\VS{8}Car la parole du Seigneur a retenti de chez vous, non seulement dans la Macédoine et dans l'Achaïe mais aussi en tous lieux, et votre foi envers Dieu est si répandue, que nous n'avons pas besoin d'en parler.
\VS{9}Car eux-mêmes racontent de nous quel accès nous avons eu auprès de vous, et comment vous vous êtes convertis à Dieu en vous séparant des idoles, pour servir le Dieu vivant et vrai,
\VS{10}et pour attendre des cieux son Fils Jésus, qu'il a ressuscité des morts, et qui nous délivre de la colère à venir\FTNT{La colère à venir. Voir les sept coupes de la colère de Dieu (Ap. 15:5-8~; 16:1-21).}.
\Chap{2}
\TextTitle{Annoncer l'Evangile en recherchant l'approbation de Dieu et non celle des hommes}
\VerseOne{}Car, mes frères, vous savez vous-mêmes que notre entrée au milieu de vous n'a point été vaine. 
\VS{2}Après avoir souffert et reçu des outrages à Philippes\FTNT{Philippes était une ville de Macédoine située en Thrace, près de la côte nord de la mer Egée. Voir Ac. 16:12-40 et l'épître de Paul aux Philippiens.}, comme vous le savez, nous avons pris de l'assurance en notre Dieu, pour vous annoncer l'Evangile de Dieu au milieu de beaucoup de combats.
\VS{3}Car il n'y a eu dans notre prédication ni séduction, ni motif impur, ni fraude.
\VS{4}Mais comme Dieu nous a considérés dignes de nous confier la prédication de l'Evangile, ainsi nous parlons non comme pour plaire aux hommes, mais à Dieu qui éprouve nos cœurs.
\VS{5}Car, en effet, nous n'avons jamais été surpris avec des paroles flatteuses, comme vous le savez~; jamais nous n'avons eu pour prétexte la cupidité, Dieu en est témoin.
\VS{6}Et nous n'avons point cherché la gloire qui vient des hommes, ni de vous, ni des autres~; nous aurions pu nous imposer comme apôtres de Christ,
\VS{7}mais nous avons été doux au milieu de vous, de même qu'une nourrice chérit ses enfants.
\VS{8}Nous aurions voulu dans notre affection envers vous, non seulement vous donner l'Evangile de Dieu, mais encore notre propre vie, tant vous nous étiez devenus chers.
\VS{9}Car, mes frères, vous vous souvenez de notre peine et de notre travail~; vu que nous vous avons prêché l'Evangile de Dieu, en travaillant nuit et jour, pour n'être point à charge à aucun de vous.
\VS{10}Vous êtes témoins et Dieu aussi, combien notre conduite envers vous qui croyez a été sainte, juste, et irréprochable.
\VS{11}Et vous savez que nous avons exhorté chacun de vous, comme un père exhorte ses enfants,
\VS{12}en vous exhortant, vous encourageant et vous conjurant de vous conduire d'une manière digne de Dieu, qui vous appelle à son Royaume et à sa gloire.
\VS{13}C'est pourquoi nous rendons sans cesse grâces à Dieu, de ce que, quand vous avez reçu de nous la parole de la prédication de Dieu, vous l'avez reçue non comme une parole des hommes, mais ainsi qu'elle est véritablement, comme la parole de Dieu, laquelle aussi agit avec efficacité en vous qui croyez.
\VS{14}En effet, mes frères, vous êtes devenus les imitateurs des églises de Dieu qui sont en Jésus-Christ dans la Judée, parce que vous aussi, vous avez souffert de la part de ceux de votre propre nation les mêmes choses qu'elles ont souffertes de la part des Juifs,
\VS{15}qui ont même mis à mort le Seigneur Jésus, et leurs propres prophètes, qui nous ont persécutés, qui ne plaisent point à Dieu, et qui sont ennemis de tous les hommes,
\VS{16}nous empêchant de parler aux Gentils afin qu'ils soient sauvés, comblant ainsi toujours plus la mesure de leurs péchés. Mais la colère de Dieu est venue sur eux jusqu'au plus haut degré.
\VS{17}Pour nous, mes frères, après avoir été quelque temps séparés de vous de corps et non de cœur, nous avons eu d'autant plus d'ardeur et d'empressement de vous revoir.
\VS{18}Nous avons donc voulu, une et même deux fois, aller chez vous, au moins, moi Paul~; mais Satan nous en a empêchés.
\VS{19}Car quelle est notre espérance, ou notre joie, ou notre couronne de gloire~? N'est-ce pas vous qui l'êtes, devant notre Seigneur Jésus-Christ lors de son avènement~?
\VS{20}Certes, vous êtes notre gloire et notre joie.
\Chap{3}
\TextTitle{La persévérance des Thessaloniciens dans l'affliction}
\VerseOne{}C'est pourquoi ne pouvant plus soutenir la privation de vos nouvelles, nous avons trouvé bon de demeurer seuls à Athènes.
\VS{2}Et nous avons envoyé Timothée, notre frère, serviteur de Dieu, et notre compagnon d'œuvre dans l'Evangile de Christ, pour vous affermir et vous exhorter au sujet de votre foi,
\VS{3}afin que nul ne soit troublé dans ces afflictions, puisque vous savez vous-mêmes que nous sommes destinés à cela.
\VS{4}Et lorsque nous étions avec vous, nous vous annoncions d'avance que nous aurions à souffrir des afflictions, comme cela est aussi arrivé, et vous le savez.
\VS{5}C'est pourquoi, dis-je, ne pouvant plus soutenir cette inquiétude, j'ai envoyé Timothée pour connaître l'état de votre foi, de peur que le tentateur ne vous ait tentés en quelque sorte, et que nous n'ayons travaillé en vain.
\VS{6}Mais Timothée étant revenu depuis peu de chez vous, nous a apporté d'agréables nouvelles de votre foi et de votre charité, et nous a dit que vous conservez toujours un bon souvenir de nous, désirant nous voir, comme nous désirons aussi vous voir.
\VS{7}C'est pourquoi, mes frères, nous avons été consolés par votre foi, dans toutes nos afflictions et dans toutes nos détresses.
\VS{8}Car maintenant nous vivons puisque vous demeurez fermes dans le Seigneur.
\VS{9}Et quelles actions de grâces ne pouvons-nous pas rendre à Dieu à votre sujet, pour toute la joie que nous éprouvons devant notre Dieu, à cause de vous.
\VS{10}Nuit et jour, nous le prions avec une extrême ardeur de nous permettre de vous voir, et de compléter\FTNT{Compléter~: du grec «~katartizo~» qui signifie «~redresser~», «~ajuster~», «~compléter~», «~raccommoder~» (ce qui a été abîmé), «~réparer~». Ce verbe est également utilisé dans Mt. 4:21 lorsque Jacques et Jean réparaient leurs filets. Le terme «~katartismos~» traduit par «~perfectionnement~» dans Ep. 4:11 vient de ce verbe. Ainsi, l'un des rôles de ces services est le perfectionnement des saints et non leur destruction.} ce qui manque à votre foi.
\VS{11}Que Dieu lui-même, notre Père, et notre Seigneur Jésus-Christ, aplanisse\FTNT{Le verbe «~aplanir~» vient du grec «~kateuthuno~». On constate que ce verbe est conjugué au singulier, y compris dans le texte original grec, ce qui atteste l'unité entre le Père et le Fils (voir 2 Th. 2:16-17).} notre chemin pour que nous allions vers vous.
\VS{12}Et que le Seigneur vous fasse croître et abonder de plus en plus en charité les uns envers les autres, et envers tous, comme nous abondons aussi en charité envers vous~;
\VS{13}qu'il affermisse vos cœurs pour qu'ils soient irréprochables dans la sainteté, devant Dieu qui est notre Père, lors de l'avènement de notre Seigneur Jésus-Christ, accompagné de tous ses saints.
\Chap{4}
\TextTitle{Appel à la sanctification et à l'amour fraternel}
\VerseOne{}Au reste, mes frères, nous vous prions donc, et nous vous conjurons par le Seigneur Jésus, que comme vous avez appris de nous de quelle manière on doit se conduire, et plaire à Dieu, vous y fassiez tous les jours de nouveaux progrès.
\VS{2}Car vous savez quels préceptes nous vous avons donnés de la part du Seigneur Jésus.
\VS{3}Parce que c'est ici la volonté de Dieu~; savoir votre sanctification\FTNT{La sanctification personnelle (1 Pi. 1:15-18~; Hé. 12:14~; Ap. 22:11). Chaque chrétien doit fournir un effort, en se servant quotidiennement de la Parole de Dieu et de la prière, pour se maintenir dans la sanctification. Cela implique la séparation d'avec le mal et des mauvaises compagnies (2 Co. 6:14-18). Elle se développe au prix de nombreuses souffrances et de multiples sacrifices (Ro. 12:1-3).}, et que vous vous absteniez de la fornication,
\VS{4}c'est que chacun de vous sache posséder son corps dans la sanctification et dans l'honneur,
\VS{5}et sans se laisser aller aux désirs de la convoitise, comme les Gentils qui ne connaissent point Dieu.
\VS{6}Que personne n'use de fraude envers son frère et de cupidité dans les affaires, parce que le Seigneur tire vengeance de toutes ces choses, comme nous vous l'avons dit et attesté.
\VS{7}Car Dieu ne nous a pas appelés à l'impureté, mais à la sanctification.
\VS{8}C'est pourquoi celui qui rejette ceci ne rejette pas un homme, mais Dieu qui a aussi donné son Saint-Esprit.
\VS{9}Quant à la charité fraternelle\FTNT{Le mot grec employé ici est «~philadelphia~». Ce terme désigne l'amour fraternel, l'amour que les chrétiens se portent entre eux.}, vous n'avez pas besoin que je vous en écrive~; car vous-mêmes vous êtes enseignés de Dieu à vous aimer les uns les autres,
\VS{10}et c'est aussi ce que vous faites à l'égard de tous les frères qui sont dans toute la Macédoine. Mais, mes frères, nous vous prions de vous perfectionner tous les jours davantage,
\VS{11}et de tâcher de vivre paisiblement~; de faire vos propres affaires, et de travailler de vos propres mains, ainsi que nous vous l'avons ordonné.
\VS{12}En sorte que vous vous conduisiez honnêtement envers ceux du dehors, et que vous n'ayez besoin de rien.
\TextTitle{L'enlèvement de l'Eglise}
\VS{13}Or, mes frères, je ne veux pas que vous soyez dans l'ignorance au sujet de ceux qui dorment, afin que vous ne soyez point attristés comme les autres qui n'ont point d'espérance. 
\VS{14}Car si nous croyons que Jésus est mort, et qu'il est ressuscité~; de même aussi ceux qui dorment en Jésus, Dieu les ramènera avec lui.
\VS{15}Car nous vous disons ceci par la parole du Seigneur, que nous qui vivrons et resterons pour l'avènement du Seigneur, ne précéderons point ceux qui dorment.
\VS{16}Car le Seigneur lui-même, avec un cri de commandement\FTNT{L'expression «~cri de commandement~» vient du grec «~keleuma~», ce mot signifie un ordre, et en particulier un cri stimulant, comme celui que reçoit un animal pressé par un homme, tels les chevaux par les conducteurs de chariots, les chiens de chasse par les chasseurs, etc.~; ou par lequel un ordre est donné par le capitaine d'un navire, aux soldats par un chef, un appel de trompette. La sagesse de Dieu crie (Pr. 8). Esaïe devait crier à plein gosier (Es. 58:1). Le cri du Seigneur ne sera entendu que par l'Eglise véritable qui est son épouse (Mt. 25:6).}, et une voix d'archange, et avec la trompette de Dieu, descendra du ciel, et les morts en Christ ressusciteront premièrement.
\VS{17}Puis nous qui vivrons et qui resterons, serons enlevés ensemble avec eux dans les nuées, à la rencontre du Seigneur, dans les airs et ainsi nous serons toujours avec le Seigneur. 
\VS{18}C'est pourquoi consolez-vous les uns les autres par ces paroles.
\Chap{5}
\TextTitle{Veiller en attendant le jour du Seigneur~; encouragements divers\FTNTT{Joë. 1:15.}}
\VerseOne{}Pour ce qui est des temps et des moments, mes frères, vous n'avez pas besoin qu'on vous en écrive,
\VS{2}puisque vous savez vous-mêmes très bien que le jour du Seigneur viendra comme un voleur dans la nuit\FTNT{Mt. 25:6~; 2 Pi. 3:10~; Ap. 3:3~; 16:15.}.
\VS{3}Quand ils diront~: Nous sommes en paix et en sûreté. Alors une destruction soudaine les surprendra, comme les douleurs de l'enfantement surprennent la femme enceinte, et ils n'échapperont point.
\VS{4}Mais quant à vous, mes frères, vous n'êtes pas dans les ténèbres pour que ce jour-là vous surprenne comme un voleur.
\VS{5}Vous êtes tous des enfants de la lumière, et des enfants du jour. Nous ne sommes point de la nuit ni des ténèbres.
\VS{6}Ne dormons donc point comme les autres, mais veillons et soyons sobres.
\VS{7}Car ceux qui dorment, dorment la nuit, et ceux qui s'enivrent, s'enivrent la nuit.
\VS{8}Mais nous qui sommes enfants du jour, soyons sobres, ayant revêtu la cuirasse de la foi et de la charité, et ayant pour casque l'espérance du salut\FTNT{Ro. 13:12~; Ep. 6:14~; 6:17.}.
\VS{9}Car Dieu ne nous a pas destinés à la colère\FTNT{La colère à venir. Voir 1 Th. 1:9-10.}, mais à l'acquisition du salut par notre Seigneur Jésus-Christ,
\VS{10}qui est mort pour nous, afin que soit que nous veillons, soit que nous dormions, nous vivions avec lui.
\VS{11}C'est pourquoi exhortez-vous réciproquement, et édifiez-vous tous, les uns les autres, comme aussi vous le faites.
\VS{12}Nous vous prions, mes frères, d'avoir de la considération pour ceux qui travaillent parmi vous, qui dirigent dans le Seigneur, et qui vous exhortent.
\VS{13}Ayez pour eux beaucoup d'affection\FTNT{Littéralement «~agape~»~: amour, charité, affection.} à cause de l'œuvre qu'ils font. Soyez en paix entre vous.
\VS{14}Nous vous en prions aussi, mes frères, avertissez ceux qui vivent dans le désordre\FTNT{Mt. 18:15~; Ga. 6:1.}, consolez ceux qui ont l'esprit abattu, supportez les faibles, et soyez patients envers tous.
\VS{15}Prenez garde que personne ne rende à autrui le mal pour le mal\FTNT{Mt. 5:44~; Ro. 12:21.}~; mais recherchez toujours ce qui est bon, soit entre vous, soit envers tous les hommes.
\VS{16}Soyez toujours joyeux.
\VS{17}Priez sans cesse.
\VS{18}Rendez grâces pour toutes choses, car c'est la volonté de Dieu par Jésus-Christ.
\VS{19}N'éteignez point l'Esprit.
\VS{20}Ne méprisez point les prophéties.
\VS{21}Eprouvez toutes choses~; retenez ce qui est bon.
\VS{22}Abstenez-vous de toute apparence de mal.
\VS{23}Que le Dieu de paix veuille vous sanctifier entièrement, et faire que votre être entier, l'esprit, l'âme et le corps soient conservés sans reproche lors de la venue de notre Seigneur Jésus-Christ\FTNT{L'avènement du Seigneur. Voir Mt. 24:1-3.}.
\VS{24}Celui qui vous appelle est fidèle, c'est pourquoi il fera ces choses en vous.
\TextTitle{Salutations}
\VS{25}Mes frères, priez pour nous.
\VS{26}Saluez tous les frères par un saint baiser.
\VS{27}Je vous en conjure par le Seigneur que cette épître soit lue à tous les saints frères.
\VS{28}Que la grâce de notre Seigneur Jésus-Christ soit avec vous~! Amen~!
\PPE{}
\end{multicols}

%\clearpage\ShortTitle{2 Th.}\BookTitle{2 Thessaloniciens}\BFont
\noindent\hrulefill
{\footnotesize
\textit{
\bigskip
{\centering{}
\\Auteur~: Paul
\\Thème~: Le jour de Christ
\\Date de rédaction~: Env. 51 ap. J.-C.\\}
}
\textit{
\\Autrefois appelée Therme ou Therma, qui signifie «~source chaude~», Thessalonique reçut son nouveau nom de Cassandre, en l'honneur de sa femme Thessalonike, qui était aussi la sœur d'Alexandre le Grand (356 av. J.-C. - 323 av. J.-C.), à qui il succéda.
\\Cette ville est située au nord de la Grèce actuelle, sur la côte de la mer Egée. Du temps de Paul, ce pays était divisé en deux parties. Dans la région du nord, la Macédoine, se trouvaient les villes de Philippes, Thessalonique et Bérée. Quant à la région du sud, l'Achaïe, comportait les villes d'Athènes et de Corinthe. Aujourd'hui, la ville s'appelle Salonique. La seconde épître de Paul aux Thessaloniciens fut rédigée peu de temps après la première. Elle fut motivée par des troubles survenus dans la communauté à la suite d'une annonce basée sur une lettre faussement attribuée à Paul prétendant que le «~jour du Seigneur~» était arrivé. Dans cette seconde épître, l'apôtre exhorte les chrétiens de Thessalonique à tenir ferme dans leur foi malgré la persécution, leur expliquant que le «~jour de Christ~» devait être précédé par l'apostasie et la venue de l'homme impie. Il conclut sa lettre en demandant aux chrétiens de s'éloigner de ceux qui vivent dans le désordre.\bigskip
}
}
\par\nobreak\noindent\hrulefill
\begin{multicols}{2}
\Chap{1}
\TextTitle{Introduction}
\VerseOne{}Paul, Silvain, et Timothée à l'église des Thessaloniciens\FTNT{Thessalonique. Voir Ac. 17:1-9.} qui est en Dieu notre Père, et en notre Seigneur Jésus-Christ~:
\VS{2}Que la grâce et la paix vous soient données de la part de Dieu notre Père, et de la part du Seigneur Jésus-Christ~!
\TextTitle{La persévérance dans l'affliction~; Dieu, le juste Juge}
\VS{3}Mes frères, nous devons toujours rendre grâces à Dieu à cause de vous, comme il est bien raisonnable, parce que votre foi augmente beaucoup, et que votre charité mutuelle fait des progrès.
\VS{4}De sorte que nous-mêmes nous nous glorifions de vous dans les églises de Dieu, à cause de votre persévérance et de votre foi au milieu de toutes vos persécutions, et des afflictions que vous avez à supporter,
\VS{5}qui sont une manifeste démonstration du juste jugement de Dieu, afin que vous soyez jugés dignes du Royaume de Dieu, pour lequel aussi vous souffrez.
\TextTitle{La fin de ceux qui ne connaissent pas Dieu et qui n'obéissent pas à l'Evangile}
\VS{6}Car il est juste devant Dieu qu'il rende l'affliction à ceux qui vous affligent,
\VS{7}et qu'il vous donne du repos à vous qui êtes affligés, de même qu'à nous, lorsque le Seigneur Jésus se révélera\FTNT{Révélation, du grec «~apokalupsis~», signifie «~mettre à nu, révélation d'une vérité~». Le fait de rendre visible ce qui était caché.} du ciel avec les anges de sa puissance,
\VS{8}avec des flammes de feu, pour exercer la vengeance contre ceux qui ne connaissent pas Dieu, contre ceux qui n'obéissent pas à l'Evangile de notre Seigneur Jésus-Christ.
\VS{9}Ils auront pour châtiment une ruine éternelle, loin de la face du Seigneur, et de la gloire de sa force,
\VS{10}quand il viendra pour être glorifié en ce jour-là dans ses saints, et pour être admiré dans tous ceux qui croient, parce le témoignage que avons rendu auprès de vous à été cru.
\VS{11}C'est pourquoi nous prions toujours pour vous, afin que notre Dieu vous juge dignes de la vocation, et qu'il accomplisse puissamment en vous tout le bon plaisir de sa bonté, et l'œuvre de la foi,
\VS{12}afin que le Nom de notre Seigneur Jésus-Christ soit glorifié en vous, et vous en lui, selon la grâce de notre Dieu et Seigneur Jésus-Christ.
\Chap{2}
\TextTitle{Le jour du Seigneur et l'apparition de l'homme impie}
\VerseOne{}Or pour ce qui concerne l'avènement\FTNT{L'avènement du Seigneur Jésus-Christ. Voir Mt. 24:1-3.} de notre Seigneur Jésus-Christ et notre réunion en lui, mes frères, nous vous prions
\VS{2}de ne pas vous laisser subitement ébranler dans votre entendement, ni troubler par une inspiration, ni par une parole, ou par quelque lettre qu'on dirait venir de nous, comme si le jour de Christ était déjà là.
\VS{3}Que personne donc ne vous séduise d'aucune manière~; car il faut que l'apostasie soit arrivée auparavant et que l'homme de péché, le fils de la perdition\FTNT{Il est question ici de l'homme impie, de l'antichrist, qui est la bête qui monte de la mer décrite par Jean (Ap. 13:11-18). Voir aussi Da. 11:36-38.}, soit révélé,
\VS{4}lequel s'oppose et s'élève contre tout ce qui est appelé Dieu, ou qu'on adore, jusqu'à être assis comme Dieu dans le temple de Dieu\FTNT{Selon les chapitres 40 à 42 d'Ezéchiel, le culte lévitique sera restauré à la fin des temps, ce qui suppose nécessairement la reconstruction du temple de Jérusalem. Cette prophétie est actuellement (2014-2015) en train de s'accomplir puisque des juifs religieux militent activement pour la réalisation de ce projet. L'organisation la plus connue œuvrant en ce sens est l'Institut du temple (fondé en 1987), qui a déjà restauré un grand nombre d'objets servant au culte. Toutefois, il ne faut pas sous-estimer la ruse de Satan, car au-delà du temple physique, il cherche prioritairement à s'asseoir dans les temples spirituels que sont les chrétiens (1 Co. 6:19). Pour parvenir à ses fins, Satan a envoyé plusieurs de ses émissaires pour prêcher un autre évangile et un autre christ. C'est ainsi que de nombreuses assemblées, séduites et captivées par de faux docteurs, n'ont plus Jésus-Christ comme Seigneur, mais Satan en personne. L'apostasie étant installée premièrement dans les cœurs, l'antichrist n'aura donc aucun mal à se faire passer pour le Christ et à s'asseoir dans le temple physique, où il usurpera l'adoration qui revient au Dieu véritable.} se proclamant lui-même être Dieu.
\VS{5}Ne vous souvenez-vous pas que je vous disais ces choses, lorsque j'étais encore chez vous~?
\VS{6}Et maintenant vous savez ce qui le retient, afin qu'il soit révélé en son temps.
\VS{7}Car le mystère de l'iniquité\FTNT{Le mystère de l'iniquité. Paul nous enseigne que ce mystère était déjà à l'œuvre au sein des églises primitives. Le prophète Zacharie, au chapitre 5 de son livre, l'avait personnifié en relatant une vision dans laquelle il avait vu «~deux femmes avec des ailes de cigogne~» emportant l'épha de l'iniquité des enfants d'Israël. Sur cet épha était assise une femme personnifiant l'iniquité, c'est-à-dire la femme de l'homme impie, la Babylone religieuse. Ces deux femmes aux ailes de cigogne allaient lui bâtir une maison au pays de Schinéar (Babylone selon Ge. 10:6-14).} opère déjà, seulement celui qui le retient en ce moment le fera jusqu’à ce qu’il soit hors du chemin.
\VS{8}Et alors sera révélé le méchant\FTNT{Es. 11:4.}, que le Seigneur détruira par le souffle de sa bouche et qu'il anéantira par l'éclat de son avènement.
\VS{9}L'avènement\FTNT{Il y aura un autre avènement, celui de l'homme impie.} de cet impie, se fera par la puissance de Satan, avec toutes sortes de miracles, de signes, et de prodiges mensongers,
\VS{10}et avec toutes les séductions de l'iniquité, pour ceux qui périssent parce qu'ils n'ont pas reçu l'amour de la vérité pour être sauvés.
\VS{11}C'est pourquoi Dieu leur envoie une puissance d'égarement\FTNT{L'esprit d'égarement. Voir Ro. 1:26,28~; 1 R. 22.}, pour qu'ils croient au mensonge,
\VS{12}afin que tous ceux qui n'ont pas cru à la vérité, mais qui ont pris plaisir à l'iniquité soient condamnés.
\TextTitle{Encouragements}
\VS{13}Mais nous, mes frères bien-aimés du Seigneur, nous devons toujours rendre grâces à Dieu pour vous, de ce que Dieu vous a élus dès le commencement pour le salut par la sanctification de l'Esprit, et par la foi en la vérité.
\VS{14}C'est à quoi il vous a appelés par notre Evangile, afin que vous possédiez la gloire qui nous a été acquise par notre Seigneur Jésus-Christ.
\VS{15}C'est pourquoi, mes frères, demeurez fermes, et retenez les enseignements que vous avez appris, soit par notre parole, soit par notre lettre.
\VS{16}Et que notre Seigneur Jésus-Christ lui-même, et notre Dieu et Père, qui nous a aimés, et qui nous a donné une consolation éternelle, et une bonne espérance par sa grâce,
\VS{17}console vos cœurs, et vous affermisse en toute bonne parole, et en toute bonne œuvre.
\Chap{3}
\VerseOne{}Au reste, mes frères, priez pour nous, afin que la parole du Seigneur poursuive sa course, et qu'elle soit glorifiée comme elle l'est parmi vous,
\VS{2}et que nous soyons délivrés des hommes méchants et pervers, car tous n'ont pas la foi.
\VS{3}Le Seigneur est fidèle, il vous affermira et vous gardera du mal.
\VS{4}Nous avons à votre égard cette confiance dans le Seigneur, que vous faites et que vous ferez les choses que nous recommandons.
\VS{5}Que le Seigneur veuille diriger vos cœurs vers l'amour de Dieu et vers l'attente de Christ~!
\TextTitle{Se séparer des mauvaises compagnies~; être un modèle~; subvenir à ses besoins}
\VS{6}Nous vous recommandons aussi, mes frères, au Nom de notre Seigneur Jésus-Christ, de vous éloigner\FTNT{La séparation d'avec la mauvaise compagnie. Voir 1 Co. 5:9-13, 15:33~; 2 Co. 6:14-18~; Ro. 16:17-18~; Tit. 3:10-11~; 2 Jn. 2-11.} de tout homme qui se dit frère, et qui vit d'une manière déréglée, et non selon les enseignements qu'il a reçus de nous.
\VS{7}Car vous savez vous-mêmes comment il faut nous imiter, puisque nous n'avons pas marché dans le désordre parmi vous,
\VS{8}et nous n'avons mangé gratuitement le pain de personne. Mais dans le labeur et dans la peine, nous avons travaillé nuit et jour, pour n'être à la charge\FTNT{Les véritables ouvriers de Dieu ne s'attendent pas aux hommes pour avoir leur salaire. Ils mettent leur confiance en Dieu qui est leur rémunérateur. Voir Ac. 20:33-35.} d'aucun de vous.
\VS{9}Ce n'est pas que nous n'en ayons pas le droit, mais c'est pour donner en nous-mêmes un modèle à imiter.
\VS{10}Car lorsque nous étions avec vous, nous vous déclarions expressément que si quelqu'un ne veut pas travailler, qu'il ne mange pas non plus.
\VS{11}Car nous apprenons qu'il y en a quelques-uns parmi vous qui marchent dans le désordre, qui ne travaillent pas, mais qui s'occupent de futilités.
\VS{12}C'est pourquoi nous recommandons donc à ces gens-là et nous les exhortons par notre Seigneur Jésus-Christ, à manger leur propre pain en travaillant paisiblement.
\VS{13}Mais pour vous, mes frères, ne vous lassez pas de faire le bien.
\VS{14}Et si quelqu'un n'obéit pas à ce que nous vous disons par cette épître, faites-le connaître, et n'ayez pas de relation avec lui, afin qu'il éprouve de la honte.
\VS{15}Toutefois, ne le regardez pas comme un ennemi, mais avertissez-le comme un frère.
\TextTitle{Conclusion}
\VS{16}Que le Seigneur de paix vous donne toujours la paix en tout temps~! Que le Seigneur soit avec vous tous~!
\VS{17}La salutation est de ma propre main, de moi Paul, c'est là ma signature dans toutes mes épîtres, c'est ainsi que j'écris.
\VS{18}Que la grâce de notre Seigneur Jésus-Christ soit avec vous tous~! Amen~!
\PPE{}
\end{multicols}

%\clearpage\ShortTitle{1 Corinthiens}\BookTitle{1 Corinthiens}\BFont
\noindent\hrulefill
{\footnotesize
\textit{
\bigskip
{\centering{}
\\Auteur : Paul
\\Thème : Le comportement du chrétien
\\Date de rédaction : Env. 56 ap. J.-C.\\}
}
%\bigskip
\textit{
\\Dans l’antiquité, Corinthe, capitale de l’Achaïe, était la ville la plus prospère et la plus puissante de Grèce. Située sur
un isthme séparant la mer Egée de la mer Ionienne, Corinthe était au carrefour de l’Asie et de l’Italie et constituait un  véritable centre commercial où les produits orientaux et occidentaux se croisaient.
%\bigskip
\\L’apôtre Paul arriva à Corinthe en 51, sous le règne de l’empereur romain Claude (10 av. J.-C. – 54 apr. J.-C.), et y demeura 18 mois. Il trouva une ville riche en pleine expansion, une population parlant diverses langues et rendant des cultes à une multitude de divinités. Rédigée au terme des trois ans passés à Ephèse, la première épître de Paul aux Corinthiens répond à une lettre dans laquelle ceux-ci s’interrogeaient sur le mariage et sur les aliments consacrés aux idoles. Ce fut aussi l’occasion pour lui de procéder à la correction de cette jeune église dont l’état charnel constituait un frein à l’avancée spirituelle. Les Corinthiens avaient en effet confondu le culte raisonnable et les pratiques liées aux cultes à mystères.\bigskip
}
}
\par\nobreak\noindent\hrulefill
\begin{multicols}{2}
\Chap{1}
\TextTitle{La grâce de Christ manifeste dans la vie des saints\FTNTT{Ro. 5:1-2 ; Ep. 1:3-14}}
\VerseOne{}Paul, appelé à être apôtre de Jésus-Christ, par la volonté de Dieu, et le frère Sosthène,
\VS{2}à l'église de Dieu qui est à Corinthe, aux sanctifiés en Jésus-Christ, appelés à être saints, et à tous ceux qui en quelque lieu que ce soit invoquent le Nom de notre Seigneur Jésus-Christ, leur Seigneur et le nôtre.
\VS{3}Que la grâce et la paix vous soient données de la part de Dieu notre Père et du Seigneur Jésus-Christ.
\VS{4}Je rends toujours grâces à mon Dieu à votre sujet, pour la grâce de Dieu qui vous a été donnée en Jésus-Christ.
\VS{5}Car en lui vous avez été enrichis de toutes les richesses qui concernent la parole et la connaissance,
\VS{6}selon que le témoignage de Jésus-Christ a été confirmé en vous,
\VS{7}de sorte qu'il ne vous manque aucun don, pendant que vous attendez la manifestation de notre Seigneur Jésus-Christ.
\VS{8}Qui vous affermira aussi jusqu’à la fin pour que vous soyez irrépréhensibles au jour de notre Seigneur Jésus-Christ.
\VS{9}Dieu qui vous a appelés à la communion de son Fils Jésus-Christ notre Seigneur est fidèle.
\TextTitle{Les rivalités, causes de divisions}
\VS{10}Je vous prie, mes frères, par le Nom de notre Seigneur Jésus-Christ, à tenir tous un même langage, et à ne point avoir de divisions parmi vous, mais à être parfaitement unis dans une même pensée et dans un même jugement.
\VS{11}Car mes frères, j’ai été informé par ceux de la maison de Chloé qu'il y a des dissensions parmi vous.
\VS{12}Je veux dire que chacun de vous parle ainsi : Moi je suis de Paul ! Et moi d'Apollos ! Et moi de Céphas ! Et moi de Christ !
\VS{13}Christ est-il divisé ? Paul a-t-il été crucifié pour vous ? Ou avez-vous été baptisés au nom de Paul ?
\VS{14}Je rends grâces à Dieu de ce que je n'ai baptisé aucun de vous, sinon Crispus et Gaïus,
\VS{15}afin que personne ne dise que j'ai baptisé en mon nom.
\VS{16}J'ai bien aussi baptisé la famille de Stéphanas ; du reste, je ne sais pas si j'ai baptisé quelque autre.
\VS{17}Car Christ ne m'a pas envoyé pour baptiser, mais pour évangéliser, non pas avec des discours de la sagesse humaine, afin que la croix de Christ ne soit pas anéantie.
\TextTitle{La sagesse de Dieu à la croix, dépasse l'entendement humain}
\VS{18}Car la prédication de la croix est une folie pour ceux qui périssent, mais pour nous qui sommes sauvés, elle est la puissance de Dieu.
\VS{19}Car il est écrit : Je détruirai la sagesse des sages et j'anéantirai l'intelligence des hommes intelligents\FTNT{Es. 29:14.}.
\VS{20}Où est le sage ? Où est le scribe ? Où est le disputeur de ce siècle ? Dieu n'a-t-il pas convaincu de folie la sagesse de ce monde ?
\VS{21}Puisque le monde, avec sa sagesse, n’a pas connu Dieu, dans la sagesse de Dieu, il a plu à Dieu de sauver les croyants par la folie de la prédication.
\VS{22}Les Juifs demandent des miracles et les Grecs cherchent la sagesse,
\VS{23}mais pour nous, nous prêchons Christ crucifié, scandale pour les Juifs, et folie pour les Grecs,
\VS{24}à ceux qui sont appelés, tant Juifs que Grecs, nous leur prêchons Christ, la puissance de Dieu et la sagesse de Dieu.
\VS{25}Parce que la folie de Dieu est plus sage que les hommes, et la faiblesse de Dieu est plus forte que les hommes.
\TextTitle{Dieu se sert des choses viles pour confondre le monde et sa sagesse}
\VS{26}Considérez, mes frères, que parmi vous qui avez été appelés, il n’y a pas beaucoup de sages selon la chair, ni beaucoup de puissants, ni beaucoup de nobles.
\VS{27}Mais Dieu a choisi les choses folles de ce monde pour confondre les sages ; et Dieu a choisi les choses faibles de ce monde pour confondre les fortes ;
\VS{28}et Dieu a choisi les choses viles de ce monde et les méprisées, même celles qui ne sont point, pour réduire à néant celles qui sont,
\VS{29}afin que nulle chair ne se glorifie devant lui.
\VS{30}Or c'est par lui que vous êtes en Jésus-Christ, lequel, de par Dieu, a été fait pour nous sagesse, justice, sanctification et rédemption ;
\VS{31}afin que comme il est écrit, celui qui se glorifie se glorifie dans le Seigneur\FTNT{Jé. 9:24.}.
\Chap{2}
\TextTitle{La foi en Dieu ne se base pas sur la sagesse humaine}
\VerseOne{}Pour moi donc, mes frères, lorsque je suis allé chez vous, ce n’est pas avec des discours pompeux, remplis de la sagesse humaine, que je suis allé vous annoncer le témoignage de Dieu.
\VS{2}Car je n’ai pas eu la pensée de savoir parmi vous autre chose que Jésus-Christ et Jésus-Christ crucifié.
\VS{3}Et j'ai même été parmi vous dans la faiblesse, dans la crainte, et dans un grand tremblement.
\VS{4}Et ma parole et ma prédication ne reposaient pas sur les discours persuasifs de la sagesse humaine, mais sur une démonstration d'Esprit et de puissance ;
\VS{5}afin que votre foi ne soit pas fondée sur la sagesse des hommes, mais sur la puissance de Dieu.
\VS{6}Cependant, nous prêchons une sagesse parmi les parfaits, une sagesse, dis-je, qui n'est pas de ce monde, ni des chefs de ce siècle, qui vont être anéantis.
\VS{7}Mais nous prêchons la sagesse de Dieu, qui est un mystère, c'est-à-dire cachée, que Dieu avant les siècles, avait prédestinée pour notre gloire,
\VS{8}sagesse qu’aucun des chefs de ce siècle n'a connue, car s'ils l’avaient connue, ils n’auraient pas crucifié le Seigneur de gloire.
\TextTitle{C'est l'Esprit de Dieu qui revèle les profondeurs de Dieu}
\VS{9}Mais comme il est écrit : Ce sont des choses que l’œil n'a point vues, que l'oreille n'a point entendues, et qui ne sont point montées au cœur de l'homme, des choses que Dieu a préparées pour ceux qui l’aiment\FTNT{Es. 64:4.}.
\VS{10}Mais Dieu nous les a révélées par son Esprit. Car l'Esprit sonde toutes choses, même les choses profondes de Dieu.
\VS{11}Qui donc, parmi les hommes, connaît les choses de l'homme, sinon l’esprit de l'homme qui est en lui ? De même aussi, personne ne connaît les choses de Dieu, si ce n’est l'Esprit de Dieu.
\VS{12}Or nous, nous n’avons pas reçu l'esprit de ce monde, mais l'Esprit qui vient de Dieu, afin que nous connaissions les choses qui nous ont été données de Dieu.
\TextTitle{La sagesse humaine n'accepte pas les choses de l'Esprit}
\VS{13}Et nous en parlons, non avec des discours que la sagesse humaine enseigne, mais avec celle qu'enseigne le Saint-Esprit, communiquant des choses spirituelles à ceux qui sont spirituels.
\VS{14}Mais l'homme animal\FTNT{L’homme animal (ou naturel) est un homme incrédule. C’est un homme  non-régénéré, ayant le principe de la vie animale, c’est-à-dire ce que les hommes ont en commun avec les brutes. Sa nature sensuelle est sujette aux appétits et aux passions (Jud. 1:19).} ne comprend pas les choses de l'Esprit de Dieu, car elles sont une folie pour lui ; et il ne peut même pas les entendre, parce c’est spirituellement qu’on en juge.
\VS{15}Mais l'homme spirituel\FTNT{L’homme spirituel est un homme dont l’esprit est régénéré et qui marche par l’Esprit. Il a la pensée de Christ et porte les fruits de l’Esprit.} juge de tout et il n'est jugé par personne.
\VS{16}Car qui a connu la pensée du Seigneur pour pouvoir l’instruire\FTNT{Es. 40:13.} ? Mais nous, nous avons la pensée de Christ.
\Chap{3}
\TextTitle{Les œuvres de la chair nuisent à la croissance chrétienne}
\VerseOne{}Pour moi, mes frères, je n'ai pas pu vous parler comme à des hommes spirituels, mais comme à des hommes charnels\FTNT{L’homme charnel est gouverné par la nature humaine et non par l'Esprit de Dieu (Ga. 5:16-21). L’homme charnel est un enfant en Christ, littéralement «~ignorant~» (Ga. 4:1). Il est comparé à un esclave.}, c'est-à-dire comme à des enfants en Christ.
\VS{2}Je vous ai donné du lait à boire, et non pas de la viande, parce que vous ne pouviez pas la supporter ; et même maintenant vous ne le pouvez pas encore, parce que vous êtes encore charnels.
\VS{3}Car puisqu'il y a parmi vous de la jalousie, des disputes, et des divisions, n'êtes-vous pas charnels, et ne vous conduisez-vous pas à la manière des hommes ?
\VS{4}Car quand l'un dit : Moi je suis de Paul ; et l'autre : Moi je suis d'Apollos, n'êtes-vous pas charnels ?
\TextTitle{Dieu est le maître de tout}
\VS{5}Qu’est-ce donc Paul, et qui est Apollos ? Des ministres, par le moyen desquels vous avez cru, selon que le Seigneur l’a donné à chacun.
\VS{6}J'ai planté, Apollos a arrosé, mais c'est Dieu qui a donné l'accroissement,
\VS{7}en sorte que ce n’est pas celui qui plante qui est quelque chose, ni celui qui arrose, mais Dieu qui donne l'accroissement.
\VS{8}Celui qui plante et celui qui arrose sont égaux, et chacun recevra sa récompense selon son propre travail.
\VS{9}Car nous sommes ouvriers avec Dieu. Vous êtes le champ de Dieu et l'édifice de Dieu.
\VS{10}Selon la grâce de Dieu qui m'a été donnée, j'ai posé le fondement comme un sage architecte, et un autre édifie dessus. Mais que chacun prenne garde comment il édifie dessus.
\TextTitle{Le seul fondement : Jésus-Christ}
\VS{11}Car personne ne peut poser un autre fondement que celui qui a été posé, à savoir Jésus-Christ.
\TextTitle{Deux types de construction}
\VS{12}Si quelqu'un édifie sur ce fondement avec de l'or, de l'argent, des pierres précieuses, du bois, du foin, du chaume, l’œuvre de chacun sera manifestée ;
\VS{13}car le jour la fera connaître, parce qu'elle sera manifestée par le feu ; et le feu éprouvera ce qu’est l’œuvre de chacun.
\VS{14}Si l’œuvre édifiée par quelqu’un sur le fondement subsiste, il recevra la récompense.
\VS{15}Si l’œuvre de quelqu'un est consumée, il perdra sa récompense ; mais pour lui, il sera sauvé, toutefois comme au travers du feu.
\VS{16}Ne savez-vous pas que vous êtes le temple\FTNT{Le temple de Dieu. Beaucoup veulent construire des bâtiments qu’ils appellent «~temples ou maisons de Dieu~» alors que chaque chrétien est le temple de Dieu. Voir Es. 66:1 ; Ac. 17:24 ; 1 Co. 6:19.} de Dieu et que l’Esprit de Dieu habite en vous ?
\VS{17}Si quelqu'un détruit le temple de Dieu, Dieu le détruira ; car le temple de Dieu est saint, et vous êtes ce temple.
\VS{18}Que personne ne s'abuse lui-même : Si quelqu'un d'entre vous croit être sage selon ce monde, qu'il devienne fou, afin de devenir sage.
\VS{19}Parce que la sagesse de ce monde est une folie devant Dieu ; car il est écrit : Il surprend les sages dans leur ruse\FTNT{Job 5:13.}.
\VS{20}Et encore : Le Seigneur connaît les pensées des sages, il sait qu’elles sont vaines\FTNT{Ps. 94:11.}.
\VS{21}Que personne donc ne mette sa gloire dans les hommes, car toutes choses sont à vous,
\VS{22}soit Paul, soit Apollos, soit Céphas, soit le monde, soit la vie, soit la mort, soit les choses présentes, soit les choses à venir, toutes choses sont à vous,
\VS{23}et vous à Christ, et Christ à Dieu.
\Chap{4}
\TextTitle{Le Seigneur est le seul véritable juge}
\VerseOne{}Que chacun nous regarde comme des serviteurs de Christ et des dispensateurs des mystères de Dieu.
\VS{2}Du reste, il est exigé des dispensateurs que chacun soit trouvé fidèle.
\VS{3}Pour moi, il m’importe fort peu d'être jugé par vous, ou par un jugement d'homme. Je ne me juge pas non plus moi-même, car je ne me sens coupable de rien,
\VS{4}mais ce n’est pas pour cela que je suis justifié. Celui qui me juge, c'est le Seigneur.
\VS{5}C'est pourquoi ne jugez de rien avant le temps, jusqu'à ce que le Seigneur vienne, alors il mettra en lumière les choses cachées dans les ténèbres et manifestera les desseins des cœurs. Alors chacun recevra de Dieu la louange qui lui sera due.
\VS{6}Or mes frères, j’ai fait de ces choses une application à ma personne et à celle d’Apollos, à cause de vous ; afin que vous appreniez de nous à ne point aller au-delà de ce qui est écrit, et que nul de vous ne conçoive de l’orgueil en faveur de l’un contre l’autre.
\VS{7}Car qui est-ce qui met de la différence entre toi et un autre ? Qu’as-tu que tu n’aies reçu ? Et si tu l'as reçu, pourquoi te glorifies-tu comme si tu ne l'avais pas reçu\FTNT{Les diverses grâces que Dieu accorde à ses enfants doivent les amener à l’humilité.} ?
\VS{8}Vous êtes déjà rassasiés, vous êtes déjà enrichis, vous êtes devenus rois sans nous. Plaise à Dieu que vous régniez en effet, afin que nous aussi nous régnions avec vous !
\TextTitle{L'humilité et la patience}
\VS{9}Car je pense que Dieu nous a exposés publiquement, nous qui sommes les derniers des apôtres, comme des gens condamnés à la mort, puisque nous avons été en spectacle au monde, aux anges et aux hommes.
\VS{10}Nous sommes fous pour l'amour de Christ, mais vous êtes sages en Christ ; nous sommes faibles, et vous êtes forts ; vous êtes dans l'estime, et nous sommes dans le mépris.
\VS{11}Jusqu'à cette heure, nous souffrons la faim, la soif, la nudité ; on nous frappe au visage, et nous sommes errants çà et là ;
\VS{12}nous nous fatiguons à travailler de nos propres mains ; on dit du mal de nous, et nous bénissons ; nous sommes persécutés, et nous le supportons.
\VS{13}Nous sommes calomniés, et nous prions ; nous sommes devenus comme les balayures du monde, comme le rebut de tous, jusqu'à maintenant.
\VS{14}Je n'écris pas ces choses pour vous faire honte, mais je vous avertis comme mes chers enfants.
\VS{15}Car même si vous aviez dix mille maîtres en Christ, vous n'avez pourtant pas plusieurs pères, car c'est moi qui vous ai engendrés en Jésus-Christ par l'Evangile.
\VS{16}Je vous prie donc d'être mes imitateurs.
\VS{17}C'est pour cela que je vous ai envoyé Timothée, qui est mon fils bien-aimé, et qui est fidèle dans le Seigneur, afin qu'il vous rappelle quelles sont mes voies en Christ et comment j'enseigne partout dans toutes les églises.
\TextTitle{L'autorité de Paul}
\VS{18}Quelques-uns se sont enflés d’orgueil comme si je ne devais pas aller chez vous.
\VS{19}Mais j'irai bientôt chez vous, si le Seigneur le veut ; et je connaîtrai non les paroles, mais la puissance de ceux qui se sont glorifiés.
\VS{20}Car le Royaume de Dieu ne consiste pas en paroles, mais en puissance.
\VS{21}Que voulez-vous ? Que j’aille chez vous avec la verge, ou avec charité et dans un esprit de douceur ?
\Chap{5}
\TextTitle{L'inceste à Corinthe}
\VerseOne{}On entend dire de toutes parts qu'il y a parmi vous de l’impudicité, et une impudicité telle qu’elle ne se rencontre même pas chez les gentils ; c'est au point où l’un de vous a la femme de son père\FTNT{L’inceste est interdit par la loi (Lé. 18:6-8).}.
\TextTitle{Oter le mal dans l'Eglise}
\VS{2}Et vous êtes enflés d'orgueil ! Et vous n'avez pas été plutôt dans le deuil, afin que celui qui a commis cette action soit retranché du milieu de vous.
\VS{3}Pour moi, étant absent de corps, mais présent en esprit, j'ai déjà jugé comme si j'étais présent, celui qui a commis une telle action.
\VS{4}Vous et mon esprit étant assemblés au nom de notre Seigneur Jésus-Christ, j'ai ordonné, avec la puissance de notre Seigneur Jésus-Christ,
\VS{5}qu'un tel homme soit livré à Satan\FTNT{Cette déclaration de Paul peut paraître choquante pour certains, mais elle nous rappelle l'histoire de Job, qui fut mis à l'épreuve par Yahweh qui l'avait livré à Satan (Job. 1:12). Paul espérait ainsi amener cet homme à la repentance en l’excluant de l’assemblée.} pour la destruction de la chair, afin que l'esprit soit sauvé au jour du Seigneur Jésus.
\VS{6}Votre vanité est mal fondée. Ne savez-vous pas qu'un peu de levain\FTNT{Le levain fait gonfler ou enfler. Il symbolise la cause principale de nombreux péchés : l'orgueil. Dans la Bible, le levain représente aussi des péchés spirituellement destructeurs comme la malice, la méchanceté, l'hypocrisie et les faux enseignements (Lu. 12:1, Mt. 16:11-12).} fait lever toute la pâte ?
\VS{7}Otez donc le vieux levain, afin que vous soyez une nouvelle pâte, puisque vous êtes sans levain ; car Christ, notre Pâque\FTNT{Ex. 12.}, a été sacrifié pour nous.
\VS{8}C'est pourquoi célébrons donc la fête, non avec du vieux levain, non avec un levain de méchanceté et de malice, mais avec les pains sans levain de la sincérité et de la vérité.
\TextTitle{Le disciple du Seigneur ne doit pas fréquenter les faux frères}
\VS{9}Je vous ai écrit dans ma lettre de ne pas vous mêlez\FTNT{Mêler vient du grec «~sunanamignumi~» qui signifie : «~mêler ensemble, se tenir en compagnie avec, être intime avec quelqu'un. Avoir des relations, être en communication~» (Ps. 1:1 ; Ro. 16:17-18 ; 1 Co. 15:33 ; Tit. 3:10).} avec les fornicateurs,
\VS{10}non pas d’une manière absolue avec les fornicateurs de ce monde, ou avec les cupides, ou les ravisseurs, ou les idolâtres ; autrement, il vous faudrait sortir du monde.
\VS{11}Maintenant, ce que je vous ai écrit, c’est de ne pas avoir de relations avec quelqu’un qui, se nommant frère, est fornicateur, ou cupide, ou idolâtre, ou médisant, ou ivrogne, ou ravisseur, de ne même pas manger avec un tel homme.
\VS{12}Car qu'ai-je à juger ceux qui sont dehors ? N’est-ce pas ceux du dedans que vous avez à juger ?
\VS{13}Mais Dieu juge ceux qui sont du dehors. Otez donc le méchant du milieu de vous.
\Chap{6}
\TextTitle{Procès entre chrétiens ou face aux non croyants}
\VerseOne{}Quand quelqu'un d'entre vous a une affaire contre un autre, ose-t-il bien aller en jugement devant les injustes, et il ne va pas devant les saints ?
\VS{2}Ne savez-vous pas que les saints jugeront le monde\FTNT{L’Eglise jugera les nations. Les douze apôtres jugeront Israël (Mt. 19:28 ; Lu. 22:30).} ? Or si le monde doit être jugé par vous, êtes-vous indignes de rendre les moindres jugements ?
\VS{3}Ne savez-vous pas que nous jugerons les anges\FTNT{Le mot ange vient du grec «~aggelos~» et veut dire «~messager, envoyé, ange~». Ce terme s’applique donc aussi bien aux hommes qu’aux créatures spirituelles.} ? Et à plus forte raison les choses de cette vie ?
\VS{4}Si donc vous avez des procès pour les affaires de cette vie, prenez pour juge ceux qui sont des moins estimés dans l'Eglise !
\VS{5}Je le dis à votre honte. Ainsi il n’y a parmi vous pas un seul homme sage qui puisse prononcer un jugement entre frères.
\VS{6}Mais un frère a des procès contre son frère, et cela devant les infidèles.
\VS{7}C'est déjà un grand défaut chez vous que vous ayez des procès entre vous. Pourquoi ne souffrez-vous pas plutôt quelque injustice ? Pourquoi ne vous laissez-vous pas plutôt dépouiller ?
\VS{8}Mais c’est vous qui commettez l’injustice et qui dépouillez, et c’est envers des frères que vous agissez de la sorte !
\TextTitle{Le chrétien est sanctifié, lavé et justifié}
\VS{9}Ne savez-vous pas que les injustes n'hériteront point le Royaume de Dieu ? Ne vous y trompez pas : Ni les fornicateurs, ni les idolâtres, ni les adultères,
\VS{10}ni les efféminés, ni les homosexuels, ni les voleurs, ni les avares, ni les ivrognes, ni les médisants, ni les ravisseurs, n'hériteront le Royaume de Dieu.
\VS{11}Et c’est là ce que vous étiez ; mais vous avez été lavés, mais vous avez été sanctifiés, mais vous avez été justifiés au nom du Seigneur Jésus, et par l'Esprit de notre Dieu.
\VS{12}Tout m’est permis, mais tout n’est pas utile ; tout m’est permis, mais je ne me rendrai esclave d’aucune chose.
\TextTitle{Le chrétien appartient au Seigneur}
\VS{13}Les aliments sont pour le ventre, et le ventre pour les aliments ; et Dieu détruira l'un comme les autres. Or le corps n'est point pour la fornication, mais pour le Seigneur, et le Seigneur pour le corps.
\VS{14}Et Dieu qui a ressuscité le Seigneur, nous ressuscitera aussi par sa puissance.
\VS{15}Ne savez-vous pas que vos corps sont les membres de Christ ? Prendrai-je donc les membres de Christ pour en faire les membres d'une prostituée ? Loin de là !
\VS{16}Ne savez-vous pas que celui qui s'unit à la prostituée devient un même corps avec elle ? Car il est dit : Les deux deviendront une même chair\FTNT{Ge. 2:24.}.
\VS{17}Mais celui qui s’unit au Seigneur est avec lui un seul esprit.
\VS{18}Fuyez la fornication. Quelque autre péché qu’un homme commette, ce péché est hors du corps ; mais le fornicateur pèche contre son propre corps.
\TextTitle{Le chrétien est le temple Saint-Esprit}
\VS{19}Ne savez-vous pas que votre corps est le temple du Saint-Esprit qui est en vous, et que vous avez reçu de Dieu, et que vous ne vous appartenez point à vous-mêmes ?
\VS{20}Car vous avez été achetés à un prix ; glorifiez donc Dieu dans votre corps et dans votre esprit, qui appartiennent à Dieu.
\Chap{7}
\TextTitle{La sainteté dans le mariage}
\VerseOne{}Pour ce qui concerne les choses au sujet desquelles vous m'avez écrit : Je vous dis qu'il est bon à l'homme de ne pas se marier.
\VS{2}Toutefois, pour éviter la fornication, que chacun ait sa femme, et que chaque femme ait son mari.
\VS{3}Que le mari rende à sa femme la bienveillance qui lui est due ; et que la femme de même la rende à son mari.
\VS{4}Car la femme n'a pas de pouvoir sur son propre corps, mais c’est son mari. De même, le mari n'a pas de pouvoir sur son propre corps, mais c’est sa femme.
\VS{5}Ne vous privez point l'un de l'autre, si ce n'est par un consentement mutuel, pour un temps, afin que vous vaquiez au jeûne et à la prière, mais après cela retournez ensemble, de peur que Satan ne vous tente par votre manque de contrôle.
\VS{6}Or je dis ceci par conseil, et non par commandement.
\VS{7}Car je voudrais que tous les hommes soient comme moi ; mais chacun a reçu de Dieu un don particulier, l'un d’une manière, l’autre d’une autre.
\VS{8}A ceux qui ne sont pas mariés, et aux veuves, je dis qu'il leur est bon de demeurer comme moi.
\VS{9}Mais s'ils manquent de maîtrise, qu'ils se marient ; car il vaut mieux se marier que de brûler.
\TextTitle{Recommandations à ceux qui sont mariés}
\VS{10}Et quant à ceux qui sont mariés, je leur ordonne, non pas moi, mais le Seigneur, que la femme ne se sépare point de son mari.
\VS{11}Et si elle s'en sépare, qu'elle demeure sans être mariée, ou qu'elle se réconcilie avec son mari ; que le mari aussi ne quitte point sa femme.
\VS{12}Mais aux autres je leur dis, et non pas le Seigneur : Si un frère a une femme incrédule et qu'elle consente d'habiter avec lui, qu'il ne la quitte point.
\VS{13}Et si une femme a un mari incrédule et qu'il consente d'habiter avec elle, qu'elle ne le quitte point.
\VS{14}Car le mari incrédule est sanctifié par la femme, et la femme incrédule est sanctifiée par le mari ; autrement vos enfants seraient impurs, or maintenant ils sont saints.
\VS{15}Que si l'incrédule se sépare, qu'il se sépare ; le frère ou la sœur ne sont point liés dans ce cas-là, car Dieu nous a appelés à la paix.
\VS{16}Car sais-tu, femme, si tu sauveras ton mari ? Ou que sais-tu, mari, si tu sauveras ta femme ?
\TextTitle{La circoncision et l'incirconcision ne sont rien, Dieu est tout}
\VS{17}Toutefois, que chacun marche selon le don qu'il a reçu de Dieu, chacun selon l’appel qu’il a reçu du Seigneur. C’est ainsi que je l’ordonne dans toutes les églises.
\VS{18}Quelqu'un a-t-il été appelé étant circoncis ? Qu’il ne redevienne pas incirconcis\FTNT{Vient du grec «~Epispaomai~» qui a pour définition : Ne pas devenir incirconcis. Aux jours  d'Antiochus IV, dit aussi Antioche Epiphane (voir commentaire en Da.8:9), certains Juifs, voulant échapper aux persécutions, cachaient le signe de leur nationalité, la circoncision, en se faisant reproduire artificiellement le prépuce par une opération chirurgicale qui étendait la peau restante.}. Quelqu'un a-t-il été appelé incirconcis ? Qu’il ne se fasse pas circoncire.
\VS{19}La circoncision n'est rien, et l’incirconcision aussi n'est rien, mais l'observation des commandements de Dieu est tout.
\VS{20}Que chacun demeure dans la condition où il était quand il a été appelé.
\VS{21}As-tu été appelé étant esclave ? Ne t'en inquiète pas ; mais si tu peux être mis en liberté, profites-en plutôt.
\VS{22}Car l’esclave qui a été appelé par notre Seigneur est un affranchi du Seigneur ; de même, celui qui est appelé étant libre, est un esclave de Christ.
\VS{23}Vous avez été rachetés à un prix, ne devenez pas les esclaves des hommes.
\VS{24}Mes frères, que chacun demeure devant Dieu dans l'état où il était quand il a été appelé.
\TextTitle{Conseils de Paul aux célibataires}
\VS{25}Pour ce qui concerne les vierges, je n'ai point de commandement du Seigneur, mais je donne un avis comme ayant obtenu miséricorde du Seigneur pour être fidèle.
\VS{26}Voici donc ce que j'estime bon, à cause des afflictions présentes : Il est avantageux à chacun de demeurer comme il est.
\VS{27}Es-tu lié à une femme ? Ne cherche pas à rompre ce lien. N’es-tu pas lié à une femme ? Ne cherche point de femme.
\VS{28}Si tu te maries, tu ne pèches point ; et si la vierge se marie, elle ne pèche point aussi ; mais ceux qui sont mariés auront des afflictions dans la chair ; or je voudrais vous les épargner.
\VS{29}Mais je vous dis ceci, mes frères : Le temps est court, que désormais ceux qui ont une femme soient comme n’en ayant pas ;
\VS{30}ceux qui pleurent comme ne pleurant pas, ceux qui se réjouissent comme ne se réjouissant pas, ceux qui achètent comme ne possédant pas,
\VS{31}et ceux qui usent de ce monde comme n'en usant pas, car la figure de ce monde passe.
\VS{32}Or je voudrais que vous soyez sans inquiétude. Celui qui n'est pas marié s’occupe des choses du Seigneur, cherchant à plaire au Seigneur.
\VS{33}Mais celui qui est marié s’occupe des choses de ce monde, cherchant à plaire à sa femme, et ainsi il est divisé.
\VS{34}Il y a de même une différence entre la femme mariée et la vierge : Celle qui n’est pas mariée s’occupe des choses du Seigneur, afin d’être sainte de corps et d'esprit ; mais celle qui est mariée s’occupe des choses du monde pour plaire à son mari.
\VS{35}Je dis cela dans votre intérêt, ce n’est pas pour vous tendre un piège, mais pour vous porter à ce qui est bienséant et propre à vous unir au Seigneur sans aucune distraction.
\VS{36}Mais si quelqu'un croit qu’il n’est pas honorable que sa fille dépasse la fleur de l’âge sans être mariée, et qu’il faille la marier, qu'il fasse ce qu'il veut, il ne pèche point ; qu'elle soit mariée.
\VS{37}Mais celui qui a pris une ferme résolution, sans contrainte, et avec l’exercice de sa propre volonté en son cœur, de garder sa fille vierge, celui-là fait bien.
\VS{38}Celui donc qui la marie fait bien, mais celui qui ne la marie pas fait mieux.
\VS{39}La femme est liée par la loi pendant tout le temps que son mari est en vie\FTNT{Dieu est contre le divorce. Pour le Seigneur, le mariage doit être un engagement à vie (Mal. 2:16 ; Ro. 7:1-3).}, mais si son mari meurt, elle est libre de se marier à qui elle veut ; seulement, que ce soit dans le Seigneur.
\VS{40}Elle est néanmoins plus heureuse si elle demeure ainsi, selon mon avis ; or j'estime que j'ai aussi l'Esprit de Dieu.
\Chap{8}
\TextTitle{Viandes sacrifiées aux idoles et les limites de la liberté chrétienne}
\VerseOne{}Pour ce qui concerne les choses qui sont sacrifiées aux idoles\FTNT{A Corinthe, on offrait rituellement des viandes sacrifiées aux idoles. A ces occasions, certaines parties des animaux sacrifiés étaient déposées sur l’autel de l’idole, d’autres étaient données aux prêtres et aux adorateurs, qui les mangeaient lors d’un repas ou d’un festin, soit dans le temple, soit dans une maison particulière. Certains morceaux de la chair offerte aux idoles étaient ensuite apportés au marché pour être vendus (Da. 1).}, nous savons que nous avons tous de la connaissance. La connaissance enfle, mais la charité édifie.
\VS{2}Et si quelqu'un croit savoir quelque chose, il n'a encore rien connu comme il faut connaître.
\VS{3}Mais si quelqu'un aime Dieu, il est connu de lui.
\VS{4}Pour ce qui est donc de manger des choses sacrifiées aux idoles, nous savons que l'idole n'est rien dans le monde et qu'il n'y a aucun autre Dieu qu’un seul\FTNT{Paul affirme avec force que le Dieu Créateur n’est pas mélangé avec d’autres divinités. Voir Dt. 6:4.}.
\VS{5}Car s’il est des êtres qui sont appelés dieux, soit dans le ciel, soit sur la terre, comme il existe réellement plusieurs dieux, et plusieurs seigneurs,
\VS{6}nous n’avons pourtant qu'un seul Dieu, qui est le Père, de qui viennent toutes choses, et pour qui nous sommes ; et un seul Seigneur : Jésus-Christ, par qui sont toutes choses, et par qui nous sommes.
\VS{7}Mais tous n’ont pas cette connaissance. Car quelques-uns, d’après la manière dont ils envisagent encore l'idole, mangent de ces choses comme étant sacrifiées aux idoles, et leur conscience qui est faible en est souillée.
\VS{8}Ce n’est pas une viande qui nous rend agréables à Dieu ; car si nous en mangeons, nous n'avons rien de plus ; si nous n’en mangeons pas, nous n’avons rien de moins.
\VS{9}Mais prenez garde que cette liberté que vous avez ne soit en quelque sorte un scandale pour les faibles.
\VS{10}Car si quelqu'un te voit, toi qui as de la connaissance, être à table dans le temple des idoles, sa conscience, à lui qui est faible, ne le portera-t-elle pas à manger des choses sacrifiées aux idoles ?
\VS{11}Et ainsi ton frère, qui est faible, et pour lequel Christ est mort, périra par ta connaissance.
\VS{12}Or quand vous péchez ainsi contre vos frères, et que vous blessez leur conscience qui est faible, vous péchez contre Christ.
\VS{13}C'est pourquoi, si la viande scandalise mon frère, je ne mangerai jamais de chair pour ne point scandaliser mon frère.
\Chap{9}
\TextTitle{Paul défend son apostolat\FTNTT{Ga. 1:11 ; 2:21}}
\VerseOne{}Ne suis-je pas apôtre ? Ne suis-je pas libre ? N’ai-je pas vu notre Seigneur Jésus-Christ ? N’êtes-vous pas mon ouvrage dans le Seigneur ?
\VS{2}Si je ne suis pas apôtre pour les autres, je le suis au moins pour vous, car vous êtes le sceau de mon apostolat dans le Seigneur.
\VS{3}C'est là ma défense contre ceux qui me condamnent.
\VS{4}N'avons-nous pas le droit de manger et de boire ?
\VS{5}N'avons-nous pas le droit de mener avec nous une sœur qui soit notre femme, comme font les autres apôtres, et les frères du Seigneur, et Céphas ?
\VS{6}N'y a-t-il que Barnabas et moi qui n'ayons pas le droit de ne pas travailler ?
\TextTitle{Dieu prend soin de ses serviteurs}
\VS{7}Qui est-ce qui va à la guerre à ses propres frais ? Qui est-ce qui plante une vigne et n’en mange pas le fruit ? Qui est-ce qui fait paître un troupeau et ne se nourrit pas du lait du troupeau ?
\VS{8}Ces choses que je dis n’existent-elles que dans la coutume des hommes ? La loi ne dit-elle pas aussi la même chose ?
\VS{9}Car il est écrit dans la Loi de Moïse : Tu ne muselleras pas le bœuf qui foule le grain\FTNT{De. 25:4.}. Dieu se met-il en peine des bœufs ?
\VS{10}Ou parle-t-il uniquement à cause de nous ? Oui, c’est à cause de nous qu’il a été écrit que celui qui laboure doit labourer avec espérance, et celui qui foule le blé, le foule avec l’espérance d’y avoir part.
\VS{11}Si nous avons semé parmi vous des biens spirituels, est-ce une grosse affaire si nous moissonnons vos biens temporels ?
\VS{12}Si d'autres usent de ce droit à votre égard, pourquoi n'en userions-nous pas plutôt qu'eux ? Cependant nous n'avons point usé de ce droit, mais au contraire, nous supportons toutes sortes d'incommodités, afin de ne pas créer d’obstacle à l'Evangile de Christ.
\VS{13}Ne savez-vous pas que ceux qui font le service sacré mangent des choses sacrées ; et que ceux qui servent à l'autel participent à l'autel\FTNT{No. 18:8-31.} ?
\VS{14}Le Seigneur a ordonné que ceux qui annoncent l'Evangile vivent de l'Evangile.
\VS{15}Pour moi, je n’ai usé d’aucun de ces droits, et ce n’est pas afin de les réclamer en ma faveur que j’écris ainsi ; car j’aimerais mieux mourir que de me laisser enlever cette gloire.
\VS{16}Car si j'évangélise, ce n’est pas pour moi un sujet de gloire, c’est parce que la nécessité m'en est imposée ; et malheur à moi si je n'évangélise pas !
\VS{17}Si je le fais de bon cœur, j’en aurai la récompense ; mais si je le fais malgré moi, c’est une charge qui m’est confiée.
\VS{18}Quelle récompense en ai-je donc ? C’est qu'en prêchant l'Evangile, je prêche l'Evangile de Christ sans qu'il en coûte rien\FTNT{Paul annonçait l’Evangile gratuitement. Donnez gratuitement : C’est la suite logique des choses, on reçoit gratuitement et on donne gratuitement. Si nous sommes comme Christ (car là est le sens du mot disciple), nous devons agir comme lui. Il a donné ses enseignements et nourrit les gens gratuitement. Dans Ap. 21:6 et 22:17, le Seigneur invite toutes les personnes qui ont soif à venir s’abreuver gratuitement. Alors pourquoi vendre la parole c’est-à-dire l’eau qu’on a reçue gratuitement ? Nous devons donner gratuitement.}, afin que je n'abuse pas de mon autorité dans l'Evangile.
\TextTitle{L'attitude d'un vrai serviteur de Dieu}
\VS{19}Car bien que je sois libre à l'égard de tous, je me suis pourtant rendu le serviteur de tous, afin de gagner plus de personnes.
\VS{20}Avec les Juifs, j’ai été comme Juif, afin de gagner les Juifs ; avec ceux qui sont sous la loi, comme si j'étais sous la loi, afin de gagner ceux qui sont sous la loi ;
\VS{21}avec ceux qui sont sans loi, comme si j'étais sans loi (quoique je ne sois point sans la Loi de Dieu, étant sous la Loi de Christ), afin de gagner ceux qui sont sans loi.
\VS{22}J’ai été faible avec les faibles, afin de gagner les faibles ; je me suis fait tout à tous, afin d’en sauver au moins quelques-uns.
\VS{23}Je fais cela à cause de l'Evangile, afin que j'en sois fait aussi participant avec les autres.
\VS{24}Ne savez-vous pas que ceux qui courent dans le stade, courent tous, mais qu’un seul remporte le prix ? Courez de manière à le remporter.
\VS{25}Tout homme qui combat, vit entièrement de régime ; et ces gens-là le font pour obtenir une couronne corruptible\FTNT{Couronne corruptible : Aux Jeux panhelléniques, il n’y avait qu’un seul vainqueur qui remportait pour prix une couronne de feuillage. Sur chacun des sites, les couronnes étaient fabriquées avec des feuillages différents : – A Olympie, c’était une couronne d’olivier sauvage – A Delphes, une couronne de laurier – A l’Isthme (Corinthe), une couronne de pin – A Némée, une couronne de céleri. En plus de sa couronne, l’athlète victorieux recevait un ruban de laine rouge. Des amphores remplies d’huile d’olive étaient également remises au vainqueur. A cette époque, l’huile d’olive était extrêmement précieuse et valait beaucoup d’argent. D’autres prix, comme des trépieds en bronze (grands vases munis de trois pieds), des boucliers en bronze ou des coupes en argent, pouvaient aussi faire partie des lots. La modeste couronne de feuillage était cependant la plus haute récompense attribuée alors dans le monde grec, car elle garantissait l’honneur et le respect de tous à celui qui la recevait.} ; mais nous, faisons-le pour une couronne incorruptible.
\VS{26}Moi donc je cours, non pas comme à l’aventure ; je combats, mais non pas comme battant l'air.
\VS{27}Mais je traite durement mon corps et je le tiens assujetti, de peur d’être moi-même désapprouvé après avoir prêché aux autres.
\Chap{10}
\TextTitle{Paul donne l'exemple d'Israël dans le désert}
\VerseOne{}Mes frères, je ne veux pas que vous ignoriez que nos pères ont tous été sous la nuée, et qu'ils ont tous passé au travers de la mer,
\VS{2}et qu'ils ont tous été baptisés en Moïse dans la nuée et dans la mer ;
\VS{3}et qu'ils ont tous mangé la même viande spirituelle ;
\VS{4}et qu'ils ont tous bu le même breuvage spirituel : Car ils buvaient de l'eau du rocher spirituel qui les suivait, et ce rocher\FTNT{Jésus-Christ, le Rocher des âges. Voir Es. 8:13-17.} était Christ.
\VS{5}Mais la plupart d’entre eux ne furent point agréables à Dieu puisqu’ils périrent dans le désert.
\VS{6}Or ces choses ont été des exemples pour nous, afin que nous ne convoitions point des choses mauvaises, comme eux-mêmes les ont convoitées.
\VS{7}Ne devenez point idolâtres, comme quelques-uns d’entre eux, selon qu'il est écrit : Le peuple s’assit pour manger et pour boire, puis ils se levèrent pour jouer\FTNT{Ex. 32:6.}.
\VS{8}Ne nous livrons pas à la fornication, comme quelques-uns d’entre eux s’y livrèrent, de sorte qu’il en tomba vingt-trois mille en un jour\FTNT{No. 25:9.}.
\VS{9}Ne tentons\FTNT{Tenter : Du grec «~ekpeirazo~» : mettre à l’épreuve ; éprouver le caractère de Dieu et son pouvoir.} point Christ, comme le tentèrent\FTNT{Tenter : Du grec «~peirazo~» : essayer si une chose peut être faite ; éprouver malicieusement, astucieusement, pour prouver ses sentiments et ses jugements ; essayer ou éprouver la foi, la vertu, le caractère par la séduction du péché ; solliciter à pécher ; infliger des maux dans le but d’éprouver. Ce terme est aussi utilisé lorsque les hommes veulent tenter Dieu en montrant leur méfiance, par une conduite impie ou méchante, pour éprouver la justice et la patience de Dieu, et le défier, pour le pousser à donner une preuve de ses perfections.} quelques-uns d’entre eux qui périrent par les serpents\FTNT{No. 21:6-9.}.
\VS{10}Ne murmurez point, comme quelques-uns d’entre eux qui périrent par le destructeur\FTNT{No. 14:2-29 ; No. 26:63-65.}.
\TextTitle{L'Eglise doit s'instruire par l'expérience d'Israël}
\VS{11}Or toutes ces choses leur sont arrivées pour servir d’exemples, et elles ont été écrites pour notre instruction, comme étant ceux auxquels les derniers temps sont parvenus.
\VS{12}Que celui donc qui pense demeurer debout prenne garde qu'il ne tombe.
\VS{13}Aucune tentation ne vous a éprouvés, qui n’ait été une tentation humaine, et Dieu qui est fidèle ne permettra pas que vous soyez tentés au-delà de vos forces, mais avec la tentation il préparera aussi le moyen d’en sortir, afin que vous puissiez la supporter.
\VS{14}C'est pourquoi, mes bien-aimés, fuyez l'idolâtrie.
\VS{15}Je vous parle comme à des personnes intelligentes, jugez vous-mêmes de ce que je dis.
\TextTitle{Distinction entre le repas et l'idolâtrie}
\VS{16}La coupe de bénédiction, que nous bénissons, n'est-elle pas la communion du sang de Christ ? Et le pain que nous rompons, n'est-il pas la communion au corps de Christ ?
\VS{17}Parce qu'il n'y a qu'un seul pain, nous qui sommes plusieurs sommes un seul corps ; car nous sommes tous participants du même pain.
\VS{18}Voyez l'Israël selon la chair, ceux qui mangent les sacrifices ne sont-ils pas en communion avec l'autel ?
\VS{19}Que dis-je donc ? Que l'idole soit quelque chose ? Ou que ce qui est sacrifié à l'idole soit quelque chose ? Nullement.
\VS{20}Mais je dis que les choses que les Gentils sacrifient, ils les sacrifient aux démons, et non à Dieu ; or je ne veux pas que vous soyez en communion avec des démons.
\VS{21}Vous ne pouvez pas boire la coupe du Seigneur et la coupe des démons ; vous ne pouvez pas participer à la table du Seigneur et à la table des démons\FTNT{L’apôtre Paul nous parle de deux sortes de tables : la table de Jézabel (ou des démons) et la table du Seigneur. La table du Seigneur à été révélée à Moïse (Ex. 25:23-30 ; Lé. 24:5-9). Il y avait dessus 12 pains destinés à la consommation des sacrificateurs. Ces pains étaient renouvelés chaque sabbat et représentaient Christ, le Pain de Dieu, qui est l’aliment du croyant-sacrificateur (Jn. 6:33-58). La table de Jézabel nous est présentée dans 1 R. 18:19 : «~Fais maintenant rassembler tout Israël auprès de moi, à la montagne du Carmel, et aussi les quatre cent cinquante prophètes de Baal et les quatre cents prophètes d’Astarté qui mangent à la table de Jézabel~». Jézabel avait à sa table 850 faux prophètes qui partageaient son repas. Voir Ap. 17. Satan est maître en matière de déguisement et d’imitation (2 Co. 11:13-15). Il a donc imité la table du Seigneur et propose aux hommes les mets du roi et le vin de la débauche (Da. 1). Il invite ceux qui cherchent Dieu à sa table afin de les détourner de la vision du ciel. Voir Mt. 6:24 ; Lu. 16:13.}.
\VS{22}Voulons-nous provoquer la jalousie du Seigneur ? Sommes-nous plus forts que lui ?
\TextTitle{La loi de l'amour s'applique dans le manger et le boire\FTNTT{Ro. 14:1-23}}
\VS{23}Toutes choses me sont permises, mais toutes ne sont pas utiles ; toutes choses me sont permises, mais toutes n'édifient pas.
\VS{24}Que personne ne cherche son propre intérêt, mais que chacun cherche celui d’autrui.
\VS{25}Mangez de tout ce qui se vend au marché, sans vous enquérir de rien par motif de conscience\FTNT{1 Ti. 4:3-5.}.
\VS{26}Car la terre avec tout ce qu'elle contient est au Seigneur.
\VS{27}Si un incrédule vous invite et que vous vouliez aller, mangez de tout ce qui sera mis devant vous, sans vous enquérir par motif de conscience.
\VS{28}Mais si quelqu'un vous dit : Ceci a été sacrifié aux idoles, n'en mangez pas, à cause de celui qui vous a avertis, et à cause de la conscience ; car la terre avec tout ce qu'elle contient est au Seigneur.
\VS{29}Je parle ici, non de votre conscience, mais de celle de l'autre. Pourquoi ma liberté serait-elle condamnée par la conscience d'un autre ?
\VS{30}Et si par la grâce j'en suis participant, pourquoi suis-je blâmé pour une chose dont je rends grâces ?
\VS{31}Soit donc que vous mangiez, soit que vous buviez, ou que vous fassiez quelque autre chose, faites tout à la gloire de Dieu.
\VS{32}Soyez tels que vous ne donniez aucun scandale ni aux Juifs, ni aux Grecs, ni à l'Eglise de Dieu,
\VS{33}de la même manière que moi aussi, je m’efforce en toutes choses de complaire à tous, cherchant, non pas mon avantage, mais celui du plus grand nombre, afin qu’ils soient sauvés.
\Chap{11}
\VerseOne{}Soyez mes imitateurs comme je le suis moi-même de Christ.
\TextTitle{Homme et femme devant Dieu}
\VS{2}Or mes frères, je vous loue de ce que vous vous souvenez de tout ce qui me concerne, et de ce que vous retenez mes instructions comme je vous les ai données.
\VS{3}Mais je veux que vous sachiez que Christ est le chef\FTNT{Le mot «~chef~» vient du grec «~kephal~» qui signifie tête. Jésus-Christ est la seule tête et l’unique chef de l’Eglise (Ep. 1:22-23 ; Col. 1:18). Toute personne qui se proclame la tête de l’église devient naturellement antéchrist.} de tout homme, que l’homme est le chef de la femme, et que Dieu est le chef de Christ.
\VS{4}Tout homme qui prie ou qui prophétise, ayant quelque chose sur la tête, déshonore son chef.
\VS{5}Toute femme au contraire qui prie, ou qui prophétise sans avoir la tête couverte, déshonore son chef, c'est comme si elle était rasée.
\VS{6}Car si une femme n'est pas couverte, qu’on lui coupe aussi les cheveux. Or, s'il est honteux pour une femme d'avoir les cheveux coupés, ou d'être rasée, qu'elle se voile.
\VS{7}Car pour ce qui est de l'homme, il ne doit point couvrir sa tête, vu qu'il est l'image et la gloire de Dieu ; mais la femme est la gloire de l'homme.
\VS{8}Parce que l'homme n'a point été tiré de la femme, mais la femme a été tirée de l'homme.
\VS{9}Et aussi l'homme n'a pas été créé pour la femme, mais la femme pour l'homme.
\VS{10}C'est pourquoi la femme à cause des anges doit avoir sur la tête une marque de l’autorité de son mari dont elle dépend.
\VS{11}Toutefois, dans le Seigneur, l'homme n'est point sans la femme ni la femme sans l'homme.
\VS{12}Car comme la femme est par l'homme, de même l'homme est par la femme, et tout cela procède de Dieu.
\VS{13}Jugez-en vous-mêmes : Est-il convenable que la femme prie Dieu sans être couverte ?
\VS{14}La nature elle-même ne vous enseigne-t-elle pas que c’est une honte pour l'homme d’avoir de longs cheveux,
\VS{15}mais que c’est une gloire pour la femme de porter des longs cheveux, parce que la chevelure lui a été donnée pour lui servir de voile ?
\VS{16}Si quelqu'un aime à contester, nous n'avons pas une telle coutume, ni les églises de Dieu.
\TextTitle{Le repas du Seigneur et les abus dénoncés par Paul}
\VS{17}Or en ce que je vais vous dire, je ne vous loue point : C’est que vous vous assemblez, non pour devenir meilleurs, mais pour empirer.
\VS{18}Car premièrement, lorsque vous vous réunissez en assemblée, j'apprends qu'il y a des divisions parmi vous et j'en crois une partie,
\VS{19}car il faut qu'il y ait même des hérésies parmi vous, afin que ceux qui sont dignes d’être approuvés soient reconnus parmi vous.
\VS{20}Quand donc vous vous assemblez ainsi tous ensemble, ce n'est pas pour manger le repas du Seigneur ;
\VS{21}car, quand on se met à table, chacun commence par prendre son souper particulier, et l'un a faim tandis que l'autre est ivre.
\VS{22}N'avez-vous donc pas de maisons pour manger et pour boire ? Ou méprisez-vous l'Eglise de Dieu et faites-vous honte à ceux qui n'ont rien ? Que vous dirai-je ? Vous louerai-je ? Je ne vous loue point en cela.
\TextTitle{Le repas du Seigneur}
\VS{23}Car j'ai reçu du Seigneur ce qu'aussi je vous ai donné ; c’est que le Seigneur Jésus, la nuit où il fut trahi, prit du pain,
\VS{24}et après avoir rendu grâces, le rompit et dit : Prenez, mangez : Ceci est mon corps qui est rompu pour vous ; faites ceci en mémoire de moi.
\VS{25}De même aussi après le souper, il prit la coupe, en disant : Cette coupe est la nouvelle alliance en mon sang ; faites ceci toutes les fois que vous en boirez, en mémoire de moi\FTNT{Mt. 26:26-28 ; Mc. 14:22-24 ; Lu. 22:19-20.}.
\VS{26}Car toutes les fois que vous mangerez de ce pain, et que vous boirez de cette coupe, vous annoncerez la mort du Seigneur, jusqu’à ce qu'il vienne.
\VS{27}C'est pourquoi quiconque mangera de ce pain ou boira de la coupe du Seigneur indignement, sera coupable envers le corps et le sang du Seigneur.
\VS{28}Que chacun donc s'éprouve soi-même, et ainsi qu'il mange de ce pain, et qu'il boive de cette coupe.
\VS{29} Car celui qui en mange et qui en boit indignement, mange et boit sa condamnation, ne distinguant point le corps du Seigneur.
\VS{30}C’est pour cela qu’il y a parmi vous beaucoup d’infirmes et de malades, et que plusieurs dorment.
\VS{31}Car si nous nous jugions nous-mêmes, nous ne serions point jugés.
\VS{32}Mais quand nous sommes jugés, nous sommes enseignés par le Seigneur, afin que nous ne soyons point condamnés avec le monde.
\VS{33}C'est pourquoi, mes frères, quand vous vous assemblez pour manger, attendez-vous les uns les autres.
\VS{34}Et si quelqu'un a faim, qu'il mange dans sa maison, afin que vous ne vous assembliez pas pour votre condamnation.  Touchant les autres points, je les réglerai quand je serai arrivé.
\Chap{12}
\TextTitle{L'Esprit révèle Christ}
\VerseOne{}Pour ce qui concerne les dons spirituels, je ne veux point, mes frères, que vous soyez ignorants.
\VS{2}Vous savez que lorsque vous étiez des gentils, vous vous laissiez entraîner vers les idoles muettes, selon que vous étiez conduits.
\VS{3}C'est pourquoi je vous fais savoir que personne, s’il parle par l'Esprit de Dieu, ne dit : Jésus est anathème ! Et personne ne peut dire : Jésus est le Seigneur ! Si ce n’est par le Saint-Esprit.
\TextTitle{La diversité des dons de l'Esprit\FTNTT{Ep. 4:7-16}}
\VS{4} Or il y a diversité de dons, mais il n'y a qu'un même Esprit.
\VS{5}Il y a aussi diversité de ministères, mais il n'y a qu'un même Seigneur.
\VS{6}Il y a aussi diversité d'opérations, mais il n'y a qu'un même Dieu qui opère toutes choses en tous.
\VS{7}Or à chacun est donnée la manifestation de l'Esprit pour l'utilité commune.
\VS{8}Car à l'un est donnée par l'Esprit, la parole de sagesse ; et à l'autre par le même Esprit, la parole de connaissance ;
\VS{9}et à un autre, la foi par ce même Esprit ; à un autre, les dons de guérison par ce même Esprit ;
\VS{10}et à un autre, les opérations des miracles ; à un autre, la prophétie ; à un autre, le don de discerner les esprits ; à un autre, la diversité de langues ; et à un autre, le don d'interpréter les langues.
\VS{11}Un seul et même Esprit opère toutes ces choses, distribuant à chacun ses dons en particulier comme il lui plaît.
\TextTitle{Chaque membre à son utilité dans le corps de Christ}
\VS{12}Car comme le corps est un, et cependant a plusieurs membres, et comme tous les membres du corps, malgré leur nombre, ne forment qu’un seul corps, il en est de même de Christ.
\VS{13}Nous avons tous, en effet, été baptisés d'un même Esprit\FTNT{Le baptême du Saint-Esprit : Les signes du baptême du Saint-Esprit (la conversion) sont les fruits de l’Esprit et sont abordés en Ga. 5:22. A aucun endroit, les écritures stipulent que le parler en langues, qui est un don gratuit (Mt. 7:16-20), est en soi le signe du baptême du Saint-Esprit. Ainsi, il nous est dit que chaque croyant en Christ a le Saint-Esprit (1 Co. 12:13 ; Ro. 8:9 ; Ep. 1:13-14) mais que tous les croyants ne parlent pas forcément en langues (1 Co. 12:29-31).}, pour être un même corps, soit Juifs, soit Grecs, soit esclaves, soit libres, nous avons tous, dis-je, été abreuvés d'un seul Esprit.
\VS{14}Ainsi, le corps n’est pas un seul membre, mais il est formé de plusieurs membres.
\VS{15}Si le pied dit : Parce que je ne suis pas la main, je ne suis point du corps ; ne serait-il pas pourtant du corps ?
\VS{16}Et si l'oreille dit : Parce que je ne suis pas l’œil, je ne suis point du corps ; ne serait-elle pas pourtant du corps ?
\VS{17}Si tout le corps est l’œil, où serait l'ouïe ? Si tout est l'ouïe, où serait l'odorat ?
\VS{18}Mais maintenant Dieu a placé chaque membre dans le corps comme il a voulu.
\VS{19}Et si tous étaient un seul membre, où serait le corps ?
\VS{20}Maintenant donc, il y a plusieurs membres et un seul corps.
\VS{21}L’œil ne peut pas dire à la main : Je n'ai pas besoin de toi ; ni la tête dire aux pieds : Je n'ai pas besoin de vous.
\VS{22}Et qui plus est, les membres du corps qui semblent être les plus faibles sont beaucoup plus nécessaires ;
\VS{23}et ceux que nous estimons être les moins honorables au corps, nous les entourons d’un plus grand honneur. Ainsi, nos membres les moins décents reçoivent le plus d’honneur,
\VS{24}Car les parties qui sont belles en nous, n’en ont pas besoin. Mais Dieu a disposé le corps de manière à donner plus d’honneur à ce qui en manquait,
\VS{25}afin qu'il n'y ait pas de division dans le corps, mais que les membres aient un soin mutuel les uns des autres.
\VS{26}Et si l'un des membres souffre quelque chose, tous les membres souffrent avec lui ; si l'un des membres est honoré, tous les membres ensemble se réjouissent avec lui.
\VS{27}Vous êtes le corps de Christ, et vous êtes chacun l’un de ses membres.
\VS{28}Et Dieu a établi dans l'Eglise premièrement des apôtres, deuxièmement des prophètes, troisièmement des docteurs, ensuite ceux qui ont le don des miracles, puis ceux qui ont les dons de guérir, de secourir, de gouverner, de parler diverses langues.
\VS{29}Tous sont-ils apôtres ? Tous sont-ils prophètes ? Tous sont-ils docteurs ? Tous ont-ils le don des miracles ?
\VS{30}Tous ont-ils les dons de guérisons ? Tous parlent-ils diverses langues ? Tous interprètent-ils ?
\VS{31}Désirez avec ardeur des dons plus excellents, et je vais vous montrer la voie la plus excellente.
\Chap{13}
\TextTitle{L'amour est la base de tout}
\VerseOne{}Quand je parlerais toutes les langues des hommes\FTNT{Les langues des hommes. Les 120 Galiléens ont été rendus capables de s’exprimer dans diverses langues afin de pouvoir annoncer la vérité aux personnes en voyage à Jérusalem dans leurs propres langues. Voir Es. 28:11-12 ; Ac. 2:1-13.}, et même des anges\FTNT{La langue des anges ou langue inconnue est incompréhensible à notre intelligence, elle est un des moyens par lequel nous disons des mystères à Dieu. Voir Ro. 8:25-26 ; 1 Co. 14:2 et 28. Il faut une interprétation si l’on veut parler cette langue dans l’assemblée à cause des non croyants qui nous visitent (1 Co. 14:23). Voir Mc. 16:17.}, si je n'ai pas la charité\FTNT{Il est question ici de l’amour «~agape~» : l’amour divin et désintéressé, l’amour fraternel.}, je suis un airain qui résonne ou une cymbale qui retentit.
\VS{2}Et quand j'aurais le don de prophétie et que je connaîtrais tous les mystères et la science de toutes choses ; et quand j'aurais même toute la foi qu'on puisse avoir, jusqu’à transporter les montagnes, si je n'ai pas la charité, je ne suis rien.
\VS{3}Et quand je distribuerais tous mes biens pour la nourriture des pauvres, quand je livrerais mon corps pour être brûlé, si je n'ai pas la charité, cela ne me sert à rien.
\VS{4}La charité est patiente, la charité est douce, la charité n'est point envieuse, la charité n'use point d'insolence, elle ne s’enfle point d’orgueil,
\VS{5}elle ne fait rien de malhonnête, elle ne cherche point son intérêt, elle ne s’irrite point, elle n’impute pas le mal,
\VS{6}elle ne se réjouit point de l'injustice, mais elle se réjouit de la vérité.
\VS{7}Elle couvre\FTNT{Dans ce passage, le grec utilisé, «~stego~», signifie «~toit, couverture, protéger ou garder en recouvrant, préserver~» (Pr. 10:12 ; Pr. 17:9). La charité ne rappelle pas sans cesses les erreurs des uns et des autres, mais sait préserver en gardant secret les fautes est expiées. Par contre, en aucun cas elle ne permet la compromission du péché en ne dénonçant pas les oeuvres des ténèbres (Mt. 18:15-18 ; Ja. 5:19-20).} tout, elle croit tout, elle espère tout, elle supporte tout.
\VS{8}La charité ne périt jamais. Les prophéties seront abolies et les langues cesseront, la connaissance sera abolie.
\VS{9}Car nous connaissons en partie et nous prophétisons en partie.
\VS{10}Mais quand la perfection sera venue, alors ce qui est en partie sera aboli.
\VS{11}Quand j'étais enfant, je parlais comme un enfant, je jugeais comme un enfant, je pensais comme un enfant ; mais quand je suis devenu homme, j'ai aboli ce qui était de l'enfance.
\VS{12}Car aujourd’hui nous voyons au moyen d’un miroir, de manière obscure, mais alors nous verrons face à face. Aujourd’hui je connais en partie, mais alors je connaîtrai comme j'ai été connu.
\VS{13}Maintenant ces trois choses demeurent : La foi, l'espérance et la charité ; mais la plus excellente de ces trois vertus c'est la charité.
\Chap{14}
\TextTitle{Importance du don de prophétie}
\VerseOne{}Recherchez la charité. Désirez avec ardeur les dons spirituels, mais surtout celui de prophétiser.
\VS{2}Parce que celui qui parle une langue inconnue ne parle point aux hommes, mais à Dieu, car personne ne le comprend, et c’est en esprit qu’il dit des mystères.
\VS{3}Mais celui qui prophétise, édifie, exhorte et console les hommes qui l'entendent.
\VS{4}Celui qui parle une langue inconnue s'édifie lui-même, mais celui qui prophétise édifie l'Eglise.
\VS{5}Je désire que vous parliez tous diverses langues, mais encore plus que vous prophétisiez. Celui qui prophétise est plus grand que celui qui parle diverses langues, à moins que ce dernier n’interprète, afin que l'Eglise en reçoive de l'édification.
\VS{6}Maintenant donc, mes frères, si je viens à vous et que je parle des langues inconnues, que vous servira cela si je ne vous parle pas par révélation, ou par science, ou par prophétie, ou par doctrine ?
\VS{7}De même, si les choses inanimées qui rendent un son, comme une flûte ou une harpe, ne rendent pas des sons distincts, comment reconnaîtra-t-on ce qui est joué sur la flûte ou sur la harpe ?
\VS{8}Et si la trompette rend un son confus, qui se préparera à la bataille ?
\VS{9}De même vous, si vous ne prononcez dans votre langue une parole distincte, comment saura-t-on ce que vous dites ? Car vous parlerez en l'air.
\VS{10}Et il y a, selon qu'il se rencontre, tant de divers sons dans le monde, et cependant aucun de ces sons n'est muet ;
\VS{11}mais si je ne sais point ce qu'on veut signifier par la parole, je serai un barbare pour celui qui parle, et celui qui parle sera un barbare pour moi.
\VS{12}Ainsi, puisque vous désirez avec ardeur les dons spirituels, que ce soit pour l’édification de l'Eglise que vous cherchiez à en posséder abondamment.
\VS{13}C'est pourquoi que celui qui parle une langue inconnue prie pour avoir le don d’interpréter.
\VS{14}Car si je prie dans une langue inconnue mon esprit est en prière, mais l'intelligence que j'en ai, est sans fruit.
\VS{15}Que faire donc ? Je prierai par l’esprit, mais je prierai aussi d'une manière à être entendu ; je chanterai par l’esprit, mais je chanterai aussi d'une manière à être entendu.
\VS{16}Autrement, si tu rends grâces par l’esprit, comment celui qui est du simple peuple dira-t-il Amen ! à ton action de grâces\FTNT{L’expression «~actions de grâces~» vient du grec «~eucharisteo~» ce qui signifie être reconnaissant, rendre grâces, remercier. Contrairement à ce que l’on enseigne dans beaucoup d’églises, il n’est pas question ici de faire une offrande d’argent mais de se montrer reconnaissant envers le Seigneur. Voir aussi commentaires en Lé. 3 et Lé. 7.}, puisqu'il ne sait pas ce que tu dis ?
\VS{17}Il est vrai que tu rends grâces, mais l’autre n’est pas édifié.
\VS{18}Je rends grâces à mon Dieu de ce que je parle plus de langues que vous tous.
\VS{19}Mais j'aime mieux prononcer dans l'Eglise cinq paroles d'une manière à être entendu, afin d’instruire aussi les autres, que dix mille paroles dans une langue inconnue.
\VS{20}Mes frères, ne soyez point des enfants sous le rapport du jugement, mais soyez des enfants à l’égard de la malice ; et à l'égard du jugement, soyez des hommes faits.
\VS{21}Il est écrit dans la loi : je parlerai à ce peuple par des gens d'une autre langue, et par des lèvres étrangères, et ils ne m’écouteront pas même ainsi, dit le Seigneur\FTNT{Es. 28:11.}.
\VS{22}C’est pourquoi les langues sont un signe, non pour les croyants, mais pour les non-croyants ; la prophétie, au contraire, est un signe, non pour les non-croyants, mais pour les croyants.
\TextTitle{L'exercice des dons spirituels dans les églises locales}
\VS{23}Si donc, l’Eglise entière s'assemble en un corps, et que tous parlent des langues étrangères et qu'il entre des gens du commun peuple ou des non-croyants, ne diront-ils pas que vous êtes hors de sens ?
\VS{24}Mais si tous prophétisent, et qu'il entre un non-croyant ou quelqu'un du commun peuple, il est convaincu par tous et il est jugé de tous,
\VS{25}ainsi les secrets de son cœur sont manifestés, de telle sorte qu'il tombera sur sa face, il adorera Dieu et publiera que Dieu est véritablement parmi vous.
\VS{26}Que faire donc mes frères ? Lorsque vous vous assemblez, les uns ou les autres parmi vous ont-ils un cantique, une instruction, une langue étrangère, une révélation, une interprétation, que tout se fasse pour l'édification.
\VS{27}Et si quelqu'un parle une langue inconnue, que cela se fasse par deux, ou tout au plus par trois, chacun à son tour, et que quelqu’un interprète ;
\VS{28}s'il n'y a point d'interprète, que cet homme se taise dans l'Eglise, et qu'il parle à lui-même et à Dieu.
\VS{29}Et que deux ou trois prophètes parlent, et que les autres en jugent ;
\VS{30}et si quelque chose est révélé à un autre qui est assis, que le premier se taise.
\VS{31}Car vous pouvez tous prophétiser l'un après l'autre, afin que tous soient instruits et que tous soient consolés.
\VS{32}Et les esprits des prophètes sont soumis aux prophètes.
\VS{33}Car Dieu n'est point un Dieu de confusion, mais de paix, comme on le voit dans toutes les églises des saints.
\VS{34}Que les femmes qui sont parmi vous se taisent dans les églises ; car il ne leur est point permis d’y parler, mais elles doivent être soumises, comme le dit aussi la loi.
\VS{35}Et si elles veulent s’instruire sur quelque chose, qu'elles interrogent leurs maris à la maison ; car il est honteux à une femme de parler dans l'église.
\VS{36}Est-ce de chez vous que la parole de Dieu est sortie ? Ou est-elle parvenue seulement à vous ?
\VS{37}Si quelqu'un croit être prophète, ou spirituel, qu'il reconnaisse que les choses que je vous écris sont des commandements du Seigneur.
\VS{38}Et si quelqu'un l’ignore, qu'il l’ignore.
\VS{39}C'est pourquoi, mes frères, désirez avec ardeur de prophétiser, et n'empêchez point de parler diverses langues.
\VS{40}Que toutes choses se fassent avec bienséance, et avec ordre.
\Chap{15}
\TextTitle{L'Evangile basé sur la résurrection de Christ}
\VerseOne{}Or, mes frères, je vous rappelle l'Evangile que je vous ai annoncé, que vous avez reçu, et auquel vous vous tenez ferme,
\VS{2}et par lequel vous êtes sauvés, si vous le retenez tel je vous l'ai annoncé ; à moins que vous n'ayez cru en vain. 
\VS{3}Car avant toutes choses, je vous ai donné ce que j'avais aussi reçu, à savoir que Christ est mort pour nos péchés, selon les Ecritures,
\VS{4}et qu'il a été enseveli, et qu'il est ressuscité\FTNT{La Résurrection du Messie. La résurrection de Jésus est un espoir pour tous les êtres humains. Elle est un principe fondamental de la foi chrétienne. Contrairement à toutes les autres religions, la foi chrétienne est la seule qui apporte l’espérance face à la mort. Toutes les autres religions ont été fondées par des hommes, leurs prophètes ou fondateurs sont morts et aucun n’est revenu à la vie. En tant que disciples de Jésus, nous sommes réconfortés par le fait que notre Dieu s'est fait homme, afin de mourir pour nos péchés, et est ressuscité le troisième jour. L’Enfer ne pouvait pas le retenir, et il tient les clés de la mort et de l’Enfer (Ap. 1:18). Voir Jn. 11:25-26. Jésus-Christ est la Résurrection.} le troisième jour, selon les Ecritures ;
\VS{5}et qu'il a été vu de Céphas, et ensuite des douze.
\VS{6}Depuis, il a été vu de plus de cinq cents frères à la fois, dont plusieurs sont encore vivants, et quelques-uns sont morts.
\VS{7}Depuis, il est apparu à Jacques, puis à tous les apôtres.
\VS{8}Après eux tous, il a été vu aussi de moi, comme d'un avorton.
\VS{9}Car je suis le moindre des apôtres, je ne suis pas digne d'être appelé apôtre, parce que j'ai persécuté l'Eglise de Dieu.
\VS{10}Mais par la grâce de Dieu, je suis ce que je suis ; et sa grâce envers moi n'a pas été vaine, mais j'ai travaillé plus qu'eux tous, toutefois non pas moi, mais la grâce de Dieu qui est avec moi.
\VS{11}Soit donc moi, soit eux, nous prêchons ainsi et vous l'avez cru ainsi.
\TextTitle{Importance de la résurrection de Christ}
\VS{12}Or si on prêche que Christ est ressuscité des morts, comment disent quelques-uns d'entre vous qu'il n'y a point de résurrection des morts ?
\VS{13}Car s'il n'y a point de résurrection des morts, Christ aussi n'est point ressuscité.
\VS{14}Et si Christ n'est pas ressuscité, notre prédication est donc vaine, et votre foi aussi est vaine.
\VS{15}Et même nous sommes de faux témoins de la part de Dieu, car nous avons rendu témoignage à l’égard de Dieu qu'il a ressuscité Christ, tandis qu’il ne l’aurait pas ressuscité, si les morts ne ressuscitent point.
\VS{16}Car si les morts ne ressuscitent point, Christ non plus n'est point ressuscité.
\VS{17}Et si Christ n'est pas ressuscité, votre foi est vaine, et vous êtes encore dans vos péchés,
\VS{18}et par conséquent aussi ceux qui dorment en Christ sont perdus.
\VS{19}Si nous n'avons d'espérance en Christ que pour cette vie seulement, nous sommes les plus misérables de tous les hommes.
\TextTitle{Détails sur les résurrections}
\VS{20}Mais maintenant Christ est ressuscité des morts, il est les prémices de ceux qui dorment.
\VS{21}Car puisque la mort est venue par un seul homme, c’est aussi par un homme qu’est venue la résurrection des morts.
\VS{22}Car comme tous meurent en Adam, de même aussi tous seront vivifiés en Christ.
\VS{23}Mais chacun en son rang, Christ comme prémices, puis ceux qui sont à Christ seront vivifiés lors de son avènement.
\VS{24}Ensuite viendra la fin, quand il aura remis le Royaume à Dieu le Père, après avoir aboli tout empire, toute puissance, et toute force.
\VS{25}Car il faut qu'il règne jusqu'à ce qu'il ait mis tous ses ennemis sous ses pieds\FTNT{Ps. 110:1.}.
\VS{26}L'ennemi qui sera détruit le dernier c'est la mort.
\VS{27}Car Dieu a tout mis sous ses pieds. Mais lorsqu’il dit que tout lui a été soumis, il est évident que celui qui lui a soumis toutes choses est excepté.
\VS{28}Et lorsque toutes choses lui auront été soumises, alors le Fils lui-même sera soumis à celui qui lui a soumis toutes choses, afin que Dieu soit tout en tous.
\VS{29}Autrement que feraient ceux qui se font baptiser pour les morts ? Si les morts ne ressuscitent absolument pas, pourquoi se font-ils baptiser pour les morts ?
\VS{30}Et nous, pourquoi sommes-nous en danger à toute heure ?
\VS{31}Tous les jours je suis exposé à la mort, je l’atteste, par la gloire de notre Seigneur Jésus-Christ.
\VS{32}Si j'ai combattu contre les bêtes à Ephèse dans des vues humaines, quel profit m’en revient-il ? Si les morts ne ressuscitent pas, mangeons et buvons, car demain nous mourrons.
\VS{33}Ne soyez point séduits : Les mauvaises compagnies corrompent les bonnes mœurs.
\VS{34}Réveillez-vous pour vivre justement, et ne péchez point ; car quelques-uns ne connaissent pas Dieu, je le dis à votre honte.
\TextTitle{Corps de résurrection}
\VS{35}Mais quelqu'un dira : Comment les morts ressuscitent-ils, et avec quel corps viennent-ils ?
\VS{36}Insensé ! Ce que tu sèmes ne reprend point vie s'il ne meurt pas\FTNT{Jn. 12:24.}.
\VS{37}Et ce que tu sèmes, tu ne sèmes point le corps qui naîtra, c’est un simple grain, de blé peut-être, ou d’une autre semence.
\VS{38}Mais Dieu lui donne le corps comme il veut, et à chacune des semences son propre corps.
\VS{39}Toute chair n'est pas de la même chair, mais autre est la chair des hommes, autre la chair des bêtes, autre celle des poissons, autre celle des oiseaux.
\VS{40}Il y a aussi des corps célestes, et des corps terrestres ; mais autre est l’éclat des corps célestes, et autre celui des corps terrestres.
\VS{41}Autre est l’éclat du soleil, autre l’éclat de la lune, autre l’éclat des étoiles ; même une étoile diffère d'une autre étoile en éclat.
\VS{42}Il en sera aussi de même à la résurrection des morts : Le corps est semé corruptible, il ressuscitera incorruptible.
\VS{43}Il est semé en déshonneur, il ressuscite glorieux ; il est semé en faiblesse, il ressuscite plein de force.
\VS{44}Il est semé corps animal, il ressuscitera corps spirituel. S’il y a un corps animal, il y a aussi un corps spirituel.
\VS{45}Comme aussi il est écrit : Le premier homme, Adam, devint une âme vivante\FTNT{Ge. 2:7.}. Le dernier Adam est devenu un Esprit vivifiant\FTNT{Jn. 5 : 21 ; Ro. 8 : 11. Jésus-Christ est le dernier Adam. Voir Ph. 2:7 ; 1 T. 3:16.}.
\VS{46}Or ce qui est spirituel n'est pas le premier, mais ce qui est animal ; et puis vient ce qui est spirituel.
\VS{47}Le premier homme, étant de la terre, est tiré de la poussière, mais le second homme, à savoir le Seigneur, est du ciel.
\VS{48}Tel qu'est celui qui est tiré de la poussière, tels aussi sont ceux qui sont tirés de la poussière ; et tel qu'est le céleste, tels aussi sont les célestes.
\VS{49}Et comme nous avons porté l'image de celui qui est tiré de la poussière, nous porterons aussi l'image du céleste.
\VS{50}Voici donc ce que je dis, mes frères, c'est que la chair et le sang ne peuvent hériter le Royaume de Dieu, et que la corruption n'hérite pas l'incorruptibilité.
\TextTitle{Mystère de la résurrection\FTNTT{1 Th. 4:14-17}}
\VS{51}Voici, je vous dis un mystère : Nous ne mourrons pas tous, mais tous nous serons changés,
\VS{52}en un instant, en un clin d’œil, à la dernière trompette\FTNT{Le mot dernier dans ce passage est «~eschatos~»  qui signifie «~dernier en temps ou en lieu, dernier dans des séries de lieux, dernier dans une succession dans le temps~». Paul associe le mystère de la résurrection à la dernière trompette. Or, dans le livre d'Apocalypse il n'y a que sept trompettes (Ap. 8 ; 9 et 11:15-19), et c'est à la dernière, c'est-à-dire la septième que le mystère de Dieu s'accomplit.}. La trompette sonnera, et les morts ressusciteront incorruptibles, et nous, nous serons changés.
\VS{53}Car il faut que ce corps corruptible revête l'incorruptibilité, et que ce corps mortel revête l'immortalité.
\TextTitle{La mort engloutie}
\VS{54}Lorsque ce corps corruptible aura revêtu l'incorruptibilité, et que ce corps mortel aura revêtu l'immortalité, alors cette parole de l'Ecriture sera accomplie : La mort a été engloutie dans la victoire\FTNT{Es. 25:8.}.
\VS{55}Ô mort, où est ta victoire ? Ô mort, où est ton aiguillon\FTNT{Os. 13:14.} ?
\VS{56}L'aiguillon de la mort c'est le péché ; et la puissance du péché c'est la loi.
\VS{57}Mais grâces soient rendues à Dieu qui nous a donné la victoire par notre Seigneur Jésus-Christ !
\VS{58}C'est pourquoi, mes frères bien-aimés, soyez fermes, inébranlables, vous appliquant toujours avec un nouveau zèle à l’œuvre du Seigneur, sachant que votre travail ne sera pas vain dans le Seigneur.
\Chap{16}
\TextTitle{Instructions et salutations de Paul}
\VerseOne{}Pour ce qui concerne la collecte en faveur des saints, faites comme je l’ai ordonné aux églises de Galatie.
\VS{2}C’est que chaque premier jour de la semaine, chacun de vous mette à part chez lui ce qu’il pourra assembler, selon la prospérité que Dieu lui accordera, afin qu’on n’attende pas mon arrivée pour recueillir les dons.
\VS{3}Puis quand je serai arrivé, j'enverrai les personnes que vous aurez approuvées avec des lettres pour porter votre libéralité à Jérusalem.
\VS{4}Et s’il convient que j’y aille moi-même, ils viendront aussi avec moi.
\VS{5}J'irai donc chez vous quand j’aurai traversé la Macédoine, car je traverserai par la Macédoine.
\VS{6}Et peut-être que je séjournerai parmi vous, ou même que j'y passerai l'hiver, afin que vous me conduisiez partout où j’irai.
\VS{7}Car je ne veux pas cette fois vous voir en passant, mais j’espère demeurer quelque temps auprès de vous, si le Seigneur le permet.
\VS{8}Toutefois, je resterai à Ephèse jusqu'à la Pentecôte.
\VS{9}Car une grande porte et un accès efficace m'y est ouverte, et les adversaires sont nombreux.
\VS{10}Si Timothée arrive, faites en sorte qu'il soit en sûreté parmi vous, car il travaille à l’œuvre du Seigneur comme moi-même.
\VS{11}Que personne donc ne le méprise. Accompagnez-le en paix, afin qu'il vienne vers moi, car je l'attends avec les frères.
\VS{12}Quant à Apollos, notre frère, je l'ai beaucoup exhorté à se rendre chez vous avec les frères, mais ce n’était décidément pas sa volonté de le faire maintenant ; il partira quand il en aura l’occasion.
\VS{13}Veillez, soyez fermes dans la foi, agissez courageusement, fortifiez-vous.
\VS{14}Que tout ce que vous faites se fasse avec charité.
\VS{15}Or, mes frères, vous connaissez la famille de Stéphanas, et vous savez qu'elle est les prémices de l'Achaïe, et qu’elle s’est entièrement appliquée au service des saints.
\VS{16}Je vous prie de vous soumettre à de tels hommes, et à tous de ceux qui s’emploient à l’œuvre du Seigneur, et qui travaillent avec nous.
\VS{17}Je me réjouis de l’arrivée de Stéphanas, de Fortunatus, et d’Achaïcus, parce qu'ils ont suppléé à votre absence.
\VS{18}Car ils ont tranquillisé mon esprit et le vôtre. Ayez donc de la considération pour de telles personnes.
\VS{19}Les églises d'Asie vous saluent. Aquilas et Priscille, avec l'église qui est dans leur maison, vous saluent affectueusement dans le Seigneur.
\VS{20}Tous les frères vous saluent. Saluez-vous les uns les autres par un saint baiser.
\VS{21}Je vous salue, moi Paul, de ma propre main.
\VS{22}Si quelqu'un n'aime pas le Seigneur Jésus-Christ, qu'il soit anathème ! Maranatha\FTNT{Maranatha signifie littéralement «~Le Seigneur vient !~}.
\VS{23}Que la grâce de notre Seigneur Jésus-Christ soit avec vous !
\VS{24}Mon amour est avec vous tous en Jésus-Christ. Amen.
\PPE{}
\end{multicols}

%\clearpage\ShortTitle{2 Corinthiens}\BookTitle{2 Corinthiens}\BFont
\noindent\hrulefill
{\footnotesize
\textit{
\bigskip
{\centering{}
\\Auteur : Paul avec Tite et Luc
\\Thème : L'autorité de Paul
\\Date de rédaction : Env. 57 ap. J.-C.\\}
}
%\bigskip
\textit{
\\Dans l’antiquité, Corinthe, capitale de l’Achaïe, était la ville la plus prospère et la plus puissante de Grèce. Située sur
un isthme séparant la mer Egée de la mer Ionienne, Corinthe était au carrefour de l’Asie et de l’Italie et constituait un
véritable centre commercial où les produits orientaux et occidentaux se croisaient.
%\bigskip
\\Rédigée quelques mois après la première, la seconde lettre de Paul aux Corinthiens fait état d’une vague de méfiance à l’égard de Paul et exprime les souffrances qui furent les siennes et qui somme toute authentifient son apostolat.\bigskip
}
}
\par\nobreak\noindent\hrulefill
\begin{multicols}{2}
\Chap{1}
\TextTitle{Introduction}
\VerseOne{}Paul, apôtre de Jésus-Christ par la volonté de Dieu, et le frère Timothée, à l'église de Dieu qui est à Corinthe, et à tous les saints qui sont dans toute l'Achaïe.
\VS{2}Que la grâce et la paix vous soient données de la part de Dieu notre Père et du Seigneur Jésus-Christ.
\TextTitle{Consolation de Paul dans ses afflictions}
\VS{3}Béni soit Dieu, le Père de notre Seigneur Jésus-Christ, le Père des miséricordes et le Dieu de toute consolation,
\VS{4}qui nous console dans toutes nos afflictions, afin que par la consolation dont nous sommes l’objet de la part de Dieu, nous puissions consoler ceux qui se trouvent dans l’affliction.
\VS{5}Car de même que les souffrances de Christ abondent en nous, de même notre consolation abonde par Christ.
\VS{6}Et si nous sommes affligés, c'est pour votre consolation et pour votre salut ; si nous sommes consolés, c'est pour votre consolation et pour votre salut qui se réalise par la patience à supporter les mêmes souffrances que nous endurons aussi.
\VS{7}Et l'espérance que nous avons de vous est ferme, sachant que comme vous êtes participants des souffrances, de même aussi vous le serez de la consolation.
\VS{8}Car mes frères, nous ne voulons pas que vous ignoriez l’affliction qui nous est survenue en Asie, que nous avons été excessivement accablés, au-delà de nos forces, de telle sorte que nous avions perdu l'espérance de conserver notre vie.
\VS{9}Et nous regardions comme certain notre arrêt de mort, afin de ne pas placer notre confiance en nous-mêmes, mais en Dieu qui ressuscite les morts.
\VS{10}C’est lui qui nous a délivrés et qui nous délivrera d'une si grande mort, et en qui nous espérons qu'il nous délivrera aussi à l'avenir.
\VS{11}Etant aussi aidés par la prière que vous faites pour nous, afin que la grâce obtenue pour nous par plusieurs soit pour plusieurs une occasion de rendre grâces à notre sujet.
\TextTitle{La sincérité de Paul dans son ministère}
\VS{12}Car ce qui fait notre gloire c’est le témoignage de notre conscience, que nous nous sommes conduits dans le monde, et surtout à votre égard, avec simplicité et sincérité de Dieu, non point avec une sagesse charnelle, mais avec la grâce de Dieu.
\VS{13}Nous ne vous écrivons pas autre chose que ce que vous lisez, et vous-mêmes le reconnaissez. Et j'espère que vous les reconnaîtrez aussi jusqu'à la fin,
\VS{14}de même que vous avez reconnu en partie que nous sommes votre gloire, comme vous serez aussi la nôtre au jour du Seigneur Jésus.
\TextTitle{Sa manière d'agir}
\VS{15}C’est dans une telle confiance que je voulais premièrement aller vers vous, afin que vous ayez une seconde grâce ;
\VS{16}et passer de chez vous en Macédoine, puis de Macédoine revenir vers vous, et être accompagné par vous en Judée.
\VS{17}Or quand je me proposais cela, ai-je usé de légèreté ? ou les choses  que je propose, sont-elles proposées selon la chair, de sorte qu'il y ait eu en moi le oui et le non ?
\VS{18}Mais Dieu est fidèle, la parole que nous vous avons adressée n’a pas été oui et non.
\VS{19}Car le Fils de Dieu, Jésus-Christ, qui a été prêché par nous au milieu de vous, par moi, par Silvain, et par Timothée, n'a pas été oui et non, mais il a été oui en lui.
\VS{20}Car autant qu’il y a de promesses de Dieu, elles sont oui en lui, et amen en lui, afin que Dieu soit glorifié par nous.
\VS{21}Or celui qui nous affermit avec vous en Christ, et qui nous a oints, c'est Dieu,
\VS{22}lequel nous a aussi marqués d’un sceau, et a mis dans nos cœurs les arrhes\FTNT{Du grec «~arrhabon~»~: arrhes~; monnaie donnée en gage d’un futur paiement, en attendant que le solde soit payé.} de l'Esprit.
\VS{23}Or j'appelle Dieu à témoin sur mon âme, que c’est pour vous épargner que je ne suis plus allé à Corinthe.
\VS{24}Non que nous dominions sur votre foi, mais nous contribuons à votre joie, puisque vous demeurez fermes dans la foi.
\Chap{2}
\TextTitle{Les fruits de la repentance}
\VerseOne{}Je résolus en moi-même de ne pas retourner chez vous avec tristesse.
\VS{2}Car si je vous attriste, qui peut me réjouir, sinon celui que j'aurai moi-même affligé ?
\VS{3}Je vous ai écrit ceci, pour ne pas éprouver à mon arrivée de la tristesse de la part de ceux de qui je devais recevoir de la joie, ayant en vous tous cette confiance que ma joie est la vôtre à tous.
\VS{4}Je vous ai écrit dans une grande affliction et angoisse de cœur, avec beaucoup de larmes, non pas afin que vous soyez attristés, mais afin que vous connaissiez la charité\FTNT{Littéralement «~agape~»~: amour fraternel.} toute particulière que j'ai pour vous.
\VS{5}Si quelqu'un a été la cause de cette tristesse, ce n'est pas moi seul qu'il a attristé, afin que je ne le surcharge point, mais en quelque sorte c'est vous tous.
\VS{6}C'est assez pour cet homme, de la correction qui lui a été faite par plusieurs,
\VS{7}en sorte que vous devez bien plutôt lui pardonner et le consoler, de peur qu’il ne soit accablé par une trop grande tristesse.
\VS{8}C'est pourquoi je vous prie de confirmer envers lui votre charité.
\VS{9}C’est aussi pour cela que je vous ai écrit, afin de vous éprouver, et de connaître si vous êtes obéissants en toutes choses.
\VS{10}Or à celui à qui vous pardonnez quelque chose, je pardonne aussi ; si j’ai pardonné à celui à qui j'ai pardonné, c’est à cause de vous, en présence de Christ,
\VS{11}afin que Satan n'ait pas le dessus sur nous, car nous n'ignorons pas ses machinations.
\VS{12}Au reste, lorsque je fus arrivé à Troas pour l'Evangile de Christ, quoique la porte m'y fût ouverte par le Seigneur, je n’eus point de repos en mon esprit, parce que je ne trouvai pas Tite, mon frère ;
\VS{13}mais ayant pris congé d'eux, je partis pour la Macédoine.
\VS{14}Grâces soient rendues à Dieu, qui nous fait toujours triompher en Christ, et qui manifeste par nous l'odeur de sa connaissance en tout lieu.
\VS{15}Nous sommes, en effet, pour Dieu le parfum de Christ, parmi ceux qui sont sauvés, et parmi ceux qui périssent :
\VS{16}Aux uns, une odeur mortelle, pour la mort ; aux autres, une odeur vivifiante, pour la vie. Mais qui est suffisant pour ces choses ?
\VS{17}Car nous ne falsifions pas la parole de Dieu, comme font plusieurs, mais nous parlons de Christ avec sincérité, comme de la part de Dieu, et devant Dieu.
\Chap{3}
\TextTitle{Les corinthiens : lettre de Christ écrite avec l'Esprit du Dieu vivant}
\VerseOne{}Commençons-nous de nouveau à nous recommander nous-mêmes ? Ou avons-nous besoin, comme quelques-uns, de lettres de recommandation auprès de vous, ou de lettres de recommandation de votre part ?
\VS{2}Vous êtes vous-mêmes notre lettre, écrite dans nos cœurs, connue et lue de tous les hommes.
\VS{3}Car il est manifeste que vous êtes la lettre de Christ, écrite par notre ministère, non avec de l'encre, mais avec l'Esprit du Dieu vivant, non sur des tables de pierre, mais sur les tables de chair, qui sont vos cœurs.
\VS{4}Or nous avons une telle confiance en Dieu par Christ.
\VS{5}Non que nous soyons capables de nous-mêmes de penser quelque chose, comme de nous-mêmes, mais notre capacité vient de Dieu.
\TextTitle{Paul ministre de la nouvelle alliance}
\VS{6}Lequel nous a rendus capables d'être ministres de la nouvelle alliance\FTNT{Dans la plupart des versions, le mot grec «~diatheke~» a été traduit par «~testament~» alors que ce mot signifie aussi «~alliance~». On le retrouve notamment dans les passages suivants~: Mt. 26:28~; Mc. 14:24~; Lu. 1:72~; 22:20~; Ac. 3:25~; 7:8~; Ro. 9:4~; 11:27~; 1 Co. 11:25~; Ga. 3:15,17~; 4:24~; Ep. 2:12~; Hé. 7:22~; 8:6,8-9~; 9:4,15-17,20~; 10:16,29~; 12:24~; 13:20~; Ap. 11:19. Le fait d’avoir regroupé les écrits de Genèse à Malachie sous l’appellatif «~Ancien Testament~» a induit beaucoup de chrétiens en erreur. L’Ancienne Alliance correspond uniquement à la loi cérémonielle de Moïse qui a été accomplie par Christ à la croix (Jn. 19:30). Ainsi, avant la mort du Seigneur, on ne peut pas parler de testament puisqu’il faut qu’il y ait au préalable la mort du testateur. Or il est évident que les animaux sacrifiés sous la loi ne nous ont rien légué (Hé. 9:1-16).}, non de la lettre, mais de l'Esprit ; car la lettre tue, mais l'Esprit vivifie.
\VS{7}Or si le ministère de la mort, écrit sur des lettres, et gravé avec des pierres, a été glorieux au point que les enfants d'Israël ne pouvaient regarder fixement le visage de Moïse, à cause de la gloire de son visage, bien que cette gloire devait disparaître,
\VS{8}comment le ministère de l'Esprit ne sera-t-il pas plus glorieux ?
\VS{9}Car si le ministère de la condamnation a été glorieux, le ministère de la justice le surpasse de beaucoup en gloire.
\VS{10}Et même ce premier ministère qui a été si glorieux, ne l'a pas été en comparaison du second qui le surpasse de beaucoup en gloire.
\VS{11}Car si ce qui devait disparaître a été glorieux, ce qui est permanent est beaucoup plus glorieux.
\VS{12}Ayant donc une telle espérance, nous usons d'une grande liberté,
\VS{13}et pas comme Moïse qui mettait un voile sur son visage, afin que les enfants d'Israël ne fixassent pas les yeux sur la fin de ce qui devait disparaître.
\VS{14}Mais ils sont devenus durs d’entendement. Car jusqu'à aujourd'hui, ce même voile qui n’est ôté que par Christ, demeure quand ils font la lecture de l'ancienne alliance.
\VS{15}Jusqu'à ce jour, quand on lit Moïse, un voile est jeté sur leur cœur.
\VS{16}Mais lorsque les cœurs se convertissent au Seigneur, le voile est ôté.
\VS{17}Or le Seigneur c'est l'Esprit ; et là où est l'Esprit du Seigneur, là est la liberté.
\VS{18}Nous tous qui contemplons comme dans un miroir la gloire du Seigneur à visage découvert, nous sommes transformés en la même image, de gloire en gloire, comme par l'Esprit du Seigneur.
\Chap{4}
\TextTitle{La vérité pratique de son ministère}
\VerseOne{}C'est pourquoi, ayant ce ministère selon la miséricorde que nous avons reçue, nous ne nous relâchons point.
\VS{2}Mais nous avons entièrement rejeté les choses honteuses que l'on cache, ne marchant point avec ruse, et ne falsifiant point la parole de Dieu, mais nous rendant approuvés à toute conscience des hommes devant Dieu, par la manifestation de la vérité.
\VS{3}Que si notre Evangile est encore voilé, il ne l'est que pour ceux qui périssent ;
\VS{4}pour les incrédules dont le dieu de ce siècle a aveuglé l’esprit, afin qu’ils ne soient pas éclairés par la lumière de l'Evangile de la gloire de Christ, lequel est l'image de Dieu.
\VS{5}Car nous ne nous prêchons pas nous-mêmes mais nous prêchons Jésus-Christ le Seigneur, et nous déclarons que nous sommes vos serviteurs pour l'amour de Jésus.
\VS{6}Car Dieu qui a dit que la lumière resplendisse des ténèbres\FTNT{Ge. 1:3.}, est celui qui a resplendi dans nos coeurs, pour manifester la connaissance de la gloire de Dieu en la présence de Jésus-Christ.
\VS{7}Mais nous avons ce trésor dans des vases de terre, afin que l'excellence de cette puissance soit de Dieu, et non pas de nous.
\TextTitle{Les souffrances de Paul}
\VS{8}Etant affligés à tous égards, mais non réduits entièrement à l’extrémité ; étant en perplexité, mais non sans secours ;
\VS{9}étant persécutés, mais non abandonnés ; étant abattus, mais non perdus ;
\VS{10}portant toujours partout dans notre corps la mort du Seigneur Jésus, afin que la vie de Jésus soit aussi manifestée dans notre corps.
\VS{11}Car nous qui vivons, nous sommes sans cesse livrés à la mort pour l'amour de Jésus, afin que la vie de Jésus soit aussi manifestée dans notre chair mortelle.
\VS{12}De sorte que la mort agit en nous, et la vie agit en vous.
\VS{13}Or ayant un même esprit de foi, selon qu'il est écrit : J'ai cru, c'est pourquoi j'ai parlé\FTNT{Ps. 116:10.} ! Nous croyons aussi , et c'est aussi pourquoi nous parlons,
\VS{14}sachant que celui qui a ressuscité le Seigneur Jésus nous ressuscitera aussi par Jésus, et nous fera comparaître en sa présence avec vous.
\VS{15}Car toutes ces choses sont pour vous, afin que cette grâce surabonde, à la gloire de Dieu, les actions de grâces d’un très grand nombre.
\VS{16}C'est pourquoi nous ne nous relâchons pas. Mais quoique notre homme extérieur se détruit, toutefois l'intérieur est renouvelé de jour en jour.
\VS{17}Car nos légères afflictions du moment, produisent pour nous, au-delà de toute mesure, un poids éternel d'une gloire souverainement excellente,
\VS{18}quand nous ne regardons point aux choses visibles, mais aux invisibles ; car les choses visibles ne sont que pour un temps, mais les invisibles sont éternelles.
\Chap{5}
\TextTitle{Ses ambitions}
\VerseOne{} Car nous savons que si notre habitation terrestre, qui n'est qu'une tente, est détruite, nous avons un édifice de Dieu qui n’a pas été fait de main d’homme, une maison éternelle dans les cieux.
\VS{2}Car c'est aussi pour cela que nous gémissons, désirant avec ardeur d'être revêtus de notre domicile qui est du ciel,
\VS{3}si toutefois nous sommes trouvés vêtus, et non pas nus.
\VS{4}Car nous qui sommes dans cette tente, nous gémissons, accablés, parce que nous désirons, non pas d'être dépouillés, mais d'être revêtus, afin que ce qui est mortel soit englouti par la vie.
\VS{5}Et celui qui nous a formés pour cela c'est Dieu qui nous a donné les arrhes de l'Esprit.
\VS{6}Nous avons donc toujours confiance ; et nous savons que logeant dans ce corps, nous demeurons loin du Seigneur,
\VS{7}car nous marchons par la foi, et non par la vue.
\VS{8}Nous avons, dis-je, de la confiance, et nous aimons mieux être absents de ce corps, et être avec le Seigneur.
\VS{9}C'est pourquoi aussi nous nous efforçons de lui être agréables, et présents, et absents.
\VS{10}Car il nous faut tous comparaître devant le tribunal\FTNT{Le Tribunal de Christ n’a pas vocation à déterminer le salut des enfants de Dieu. Les chrétiens y seront jugés en fonction des œuvres produites sur la terre. En effet, chacun devra rendre compte de ce qu’il aura fait et de la gestion des dons et ministères reçus. Voir Ro. 14:10~; 1 Co. 4:4~; 2 Co. 3:10-14~; 2 Tim. 4:8.} de Christ, afin que chacun reçoive en son corps selon ce qu'il aura fait, soit bien, soit mal.
\TextTitle{Ses motifs d'action}
\VS{11}Connaissant donc combien le Seigneur doit être craint, nous persuadons les hommes; et nous sommes connus de Dieu, et j’espère que dans vos consciences vous nous connaissez aussi.
\VS{12}Car nous ne nous recommandons pas de nouveau à vous, mais nous vous donnons l'occasion de vous glorifier à notre sujet, afin que vous ayez de quoi répondre à ceux qui se glorifient de l'apparence, et non pas de ce qui est dans le cœur.
\VS{13}Car soit que nous soyons hors de sens, c’est pour Dieu ; soit que nous soyons de bon sens c’est pour vous.
\VS{14}Car la charité de Christ nous lie, parce que nous jugeons que, si un est mort pour tous, tous aussi sont morts ;
\VS{15}et qu'il est mort pour tous, afin que ceux qui vivent, ne vivent plus pour eux-mêmes, mais pour celui qui est mort et ressuscité pour eux.
\VS{16}C'est pourquoi dès maintenant nous ne connaissons personne selon la chair ; et si nous avons connu Christ selon la chair, maintenant nous ne le connaissons plus ainsi.
\VS{17}Si donc quelqu'un est en Christ, il est une nouvelle créature ; les choses anciennes sont passées ; voici, toutes choses sont faites nouvelles.
\VS{18}Et tout cela vient de Dieu, qui nous a réconciliés avec lui par Jésus-Christ, et qui nous a donné le ministère de la réconciliation.
\VS{19}Car Dieu était en Christ, réconciliant le monde avec lui-même, en ne leur imputant point leurs péchés, et il a mis en nous la parole de la réconciliation.
\VS{20}Nous sommes donc ambassadeurs pour Christ, c'est comme si Dieu vous exhortait par notre ministère ; nous vous supplions donc pour l’amour de Christ : Réconciliez-vous avec Dieu !
\VS{21}Car il a fait celui qui n'a point connu de péché, être péché pour nous, afin que nous soyons en lui justice de Dieu.
\Chap{6}
\TextTitle{Son humilité}
\VerseOne{}Puisque nous travaillons avec le Seigneur, nous vous prions de ne pas recevoir la grâce de Dieu en vain.
\VS{2}Car il dit : Je t'ai exaucé au temps favorable et t'ai secouru au jour du salut\FTNT{Es. 49:8.} ; voici maintenant le temps favorable, voici maintenant le jour du salut.
\VS{3}Ne donnant aucun scandale en quoi que ce soit, afin que notre ministère ne soit point blâmé.
\VS{4}Mais nous rendant recommandables, en toutes choses, comme ministres de Dieu, en grande patience, en afflictions, en nécessités, en détresses,
\VS{5}sous les coups, dans les prisons, dans les troubles, dans les travaux, dans les veilles, dans les jeûnes,
\VS{6}par la pureté, par la connaissance, par la persévérance, par la douceur, par le Saint-Esprit, par une charité sincère,
\VS{7}par la parole de vérité, par la puissance de Dieu, par les armes de justice que l'on porte à la main droite et à la main gauche ;
\VS{8}au milieu de la gloire et de l'ignominie, au milieu de la mauvaise et de la bonne réputation ; étant regardés comme des séducteurs, quoique véridiques,
\VS{9}comme inconnus, quoique bien connus, comme mourants, et voici nous vivons, comme châtiés, quoique non mis à mort,
\VS{10}comme attristés, et nous sommes toujours joyeux, comme pauvres, et nous en enrichissons plusieurs, comme n'ayant rien, et nous possédons toutes choses.
\TextTitle{Appel à la séparation et à la purification}
\VS{11}Ô Corinthiens ! Notre bouche s’est ouverte pour vous, notre cœur s'est élargi.
\VS{12}Vous n’y êtes point à l'étroit, mais c’est votre cœur qui s’est rétréci pour nous.
\VS{13}Rendez-nous la pareille (je vous parle comme à mes enfants) élargissez aussi votre cœur !
\VS{14}Ne portez pas un même joug avec les infidèles ; car quelle communion y a-t-il entre justice et l'iniquité ? Ou qu’y a-t-il de commun entre la lumière et les ténèbres ?
\VS{15}Et quel accord y a-t-il entre Christ et Bélial\FTNT{Bélial~: De l’hébreu «~beliya’al~»~: méchants, pervers, pervertis, vil, destruction, dangereusement. L’un des noms de Satan qui signifie «~indignité, méchanceté, impiété~».} ? Ou quelle part a le fidèle avec l'infidèle ?
\VS{16}Et quel rapport y a-t-il entre le temple de Dieu et les idoles ? Car vous êtes le temple du Dieu vivant, selon ce que Dieu a dit : J'habiterai au milieu d'eux et j'y marcherai ; je serai leur Dieu, et ils seront mon peuple\FTNT{Lé. 26:12~; Ez. 37:26.}.
\VS{17}C'est pourquoi sortez du milieu d'eux, et séparez-vous, dit le Seigneur ; ne touchez pas à ce qui est impur, et je vous accueillerai\FTNT{Es. 52:11~; Ap. 18:4.}.
\VS{18}Je serai pour vous un Père, et vous serez pour moi des fils et des filles, dit le Seigneur Tout-Puissant\FTNT{Jn. 1:12~; Ap. 21:7.}.
\Chap{7}
\VerseOne{}Or donc mes bien-aimés, puisque nous avons de telles promesses, nettoyons-nous de toute souillure de la chair et de l'esprit, perfectionnant la sanctification dans la crainte de Dieu.
\TextTitle{Paul ouvre son coeur aux Corinthiens}
\VS{2}Recevez-nous, nous n'avons fait tort à personne, nous n'avons corrompu personne, nous n'avons pillé personne.
\VS{3}Je ne dis pas ceci pour vous condamner, car je vous ai déjà dit que vous êtes dans nos cœurs à la vie et à la mort.
\VS{4}J'ai une grande liberté envers vous, j'ai grand sujet de me glorifier de vous ; je suis rempli de consolation, je suis comblé de joie au milieu de toutes nos afflictions.
\VS{5}Car depuis notre arrivée en Macédoine, notre chair n’eut aucun repos, mais nous avons été affligés de toute manière, ayant eu des combats au dehors, et des craintes au dedans.
\VS{6}Mais Dieu qui console les abattus nous a consolés par l’arrivée de Tite,
\VS{7}et non seulement par son arrivée, mais aussi par la consolation qu'il a reçue de vous ; car il nous a raconté votre grand désir, vos larmes, votre affection ardente pour moi, en sorte que je m'en suis extrêmement réjoui.
\VS{8}Quoique je vous aie attristés par ma lettre, je ne m'en repens pas. Et si je m'en suis repenti, car je vois que cette lettre vous a affligés, bien que momentanément,
\VS{9}je me réjouis à présent, non de ce que vous avez été affligés, mais de ce que votre tristesse vous a portés à la repentance ; car vous avez été attristés selon Dieu, de sorte que vous n'avez reçu aucun dommage de notre part.
\VS{10}En effet, la tristesse selon Dieu produit une repentance à salut dont on ne se repent jamais, mais que la tristesse du monde produit la mort.
\VS{11}En effet, cette tristesse qui est selon Dieu, quel empressement n’a-t-elle pas produit en vous ! quelle justification, quelle indignation, quelle crainte, quel grand désir, quel zèle, quelle vengeance! Vous  vous êtes montrés de toutes manières purs dans cette affaire.
\VS{12}Quoi que je vous aie écrit, ce n'était ni à cause de celui qui a commis la faute, ni à cause de celui envers qui elle a été commise, mais pour faire voir parmi vous l'empressement que j'ai de vous devant Dieu.
\VS{13}C'est pourquoi nous avons été consolés de ce que vous avez fait pour notre consolation. Mais nous nous sommes encore plus réjouis de la joie qu'a eu Tite, en ce que son esprit a été tranquillisé par vous tous.
\VS{14}Et si en quelque chose je me suis glorifié de vous devant lui, je n’en ai point eu de confusion ; mais, comme nous avons toujours parlé selon la vérité, ce dont nous nous sommes glorifiés auprès de Tite s’est trouvé être aussi la vérité.
\VS{15}C'est pourquoi quand il se souvient de l'obéissance de vous tous, et comment vous l'avez reçu avec crainte et tremblement, son affection pour vous en est beaucoup plus grande.
\VS{16}Je me réjouis d'avoir confiance en vous en toutes choses.
\Chap{8}
\TextTitle{Exemple des Macédoniens concernant la collecte en faveur des pauvres de Jérusalem}
\VerseOne{}Au reste, mes frères, nous voulons vous faire connaître la grâce que Dieu a faite aux églises de la Macédoine.
\VS{2}Au travers de leur grande épreuve d'affliction, leur joie a été augmentée, et leur profonde pauvreté s'est répandue en richesses par leur prompte libéralité.
\VS{3}Car je suis témoin qu'ils ont donné volontairement selon leurs moyens, et même au-delà de leurs moyens,
\VS{4}Nous pressant avec de grandes prières de recevoir la grâce et de prendre part à cette contribution en faveur des Saints.
\VS{5}Et ils n'ont pas fait seulement comme nous l'espérions, mais ils se sont donnés premièrement eux-mêmes au Seigneur, et puis à nous, par la volonté de Dieu.
\VS{6}Nous avons exhorté Tite, comme il avait auparavant commencé, d'achever aussi cette grâce envers vous.
\TextTitle{Exemple du Messie}
\VS{7}C'est pourquoi, comme vous excellez en toutes choses, en foi, en parole, en connaissance, en toute diligence, et dans la charité que vous avez pour nous, faites en sorte d’exceller aussi dans cette œuvre de charité.
\VS{8}Je ne dis pas cela pour vous donner un ordre, mais pour éprouver par l’empressement des autres la sincérité de votre charité.
\VS{9}Car vous connaissez la grâce de notre Seigneur Jésus-Christ qui, étant riche, s'est fait pauvre pour vous, afin que par sa pauvreté vous soyez enrichis.
\VS{10}C’est un avis que je donne là-dessus, parce qu'il vous est convenable, à vous qui non seulement avez commencé à agir, mais en ayant même eu la volonté dès l'année passée.
\VS{11}Achevez donc maintenant d'agir, afin que comme vous avez été prompts à en avoir la volonté ; vous l'accomplissiez aussi selon vos moyens.
\VS{12}Car si la promptitude de la volonté existe,  on est agréable selon ce qu'on a, et non point selon ce qu'on n'a pas.
\VS{13}Je ne veux pas vous exposer à la détresse pour soulager les autres, mais suivre une règle d’égalité. Dans la circonstance présente, votre superflu pourvoira à leurs besoins,
\VS{14}afin que leur superflu pourvoie pareillement aux vôtres, en sorte qu’il y ait égalité,
\VS{15}selon ce qui est écrit : Celui qui avait beaucoup n'a rien eu de superflu, et celui qui avait peu n'en a pas eu moins.\FTNT{Ex. 16:18.}.
\TextTitle{Exemple des églises}
\VS{16}Grâces soient rendues à Dieu qui a mis dans le cœur de Tite le même empressement pour vous ;
\VS{17}lequel a bien reçu mon exhortation, et c'est avec un nouveau zèle et de son plein gré qu'il part pour aller chez vous.
\VS{18}Et nous avons aussi envoyé avec lui le frère dont la louange dans l'Evangile est répandue par toutes les églises ;
\VS{19}de plus, il a été choisi par élection des églises pour être notre compagnon de voyage et pour cette grâce\FTNT{ également traduit par «~les aumônes~»} qui est administrée par nous à la gloire du Seigneur même, et afin de répondre à l’ardeur de votre zèle. %et pour servir à la promptitude de votre zèle.
\VS{20}Evitant ainsi que personne ne nous blâme dans cette abondante collecte, qui est administrée par nous;
\VS{21}ayant soin de faire ce qui est bon, non seulement devant le Seigneur, mais aussi devant les hommes.
\VS{22}Nous avons envoyé aussi avec eux notre autre frère, dont nous avons souvent éprouvé le zèle à plusieurs occasions, qui est maintenant encore plus zélé, à cause de la grande confiance qu'il a en vous.
\VS{23}Ainsi donc, quant à Tite, il est mon associé et mon compagnon d’œuvre auprès de vous ; et quant à nos frères, ils sont les envoyés des églises, la gloire de Christ.
\VS{24}Montrez donc envers eux et devant les églises une preuve de votre charité, et du sujet que nous avons de nous glorifier de vous.
\Chap{9}
\TextTitle{Encouragement par rapport aux dons}
\VerseOne{}Il est superflu que je vous écrive touchant la collecte destinée aux saints.
\VS{2}Car je vois la promptitude de votre zèle, dont je me glorifie de vous devant ceux de Macédoine, leur disant que l’Achaïe est prête dès l'année dernière ; et votre zèle en a excité plusieurs.
\VS{3}J’ai envoyé ces frères, afin que ce en quoi je me suis glorifié de vous, ne soit pas vain en cette occasion, et que vous soyez prêts, comme j'ai dit.
\VS{4}De peur que ceux de Macédoine venant avec moi, et ne vous trouvant pas prêts, nous, pour ne pas dire vous, n'ayons de honte de l'assurance dont nous nous sommes glorifiés.
\VS{5}C'est pourquoi j'ai estimé nécessaire de prier les frères à se rendre premièrement vers vous, et d'achever de préparer votre bienfait déjà promis, afin qu'il soit prêt, comme un bienfait, et non comme de l'avarice.
\VS{6}Au reste, je vous avertis que celui qui sème peu moissonnera peu, et celui qui sème abondamment moissonnera abondamment.
\VS{7}Mais que chacun contribue selon qu'il se l'est proposé en son cœur, non à regret, ou par contrainte; car Dieu aime celui qui donne avec joie.
\VS{8}Et Dieu est Tout-Puissant pour vous combler de toutes sortes de grâces, afin qu'ayant toujours tout ce qui suffit en toute chose, vous soyez abondants en toute bonne œuvre,
\VS{9}selon ce qui est écrit : Il a fait des largesses, il a donné aux pauvres ; sa justice demeure éternellement\FTNT{Ps. 112:9.}.
\VS{10}Que celui qui fournit de la semence au semeur veuille aussi vous donner du pain à manger, et multiplier votre semence, et augmenter les revenus de votre justice ;
\VS{11}afin que vous soyez pleinement enrichis pour exercer une parfaite libéralité, laquelle fait que nous en rendons grâces à Dieu.
\VS{12}Car le service de cette assistance est non seulement suffisant pour subvenir aux nécessités des Saints, mais il abonde aussi de telle sorte, que plusieurs ont de quoi en rendre grâces à Dieu.
\VS{13}Glorifiant Dieu pour l’épreuve qu’ils font de cette assistance, en ce que vous vous soumettez à l’Évangile de Christ ; et de votre prompte et libérale communication envers eux, et envers tous.
\VS{14}ils prient Dieu pour vous, et ils vous aiment\FTNT{Aimer~: Du grec «~epipotheo~»~: désirer, chérir.} très affectueusement à cause de la grâce excellente que Dieu vous a accordée.
\VS{15}Grâces soient rendues à Dieu pour son don inexprimable.
\Chap{10}
\TextTitle{Paul défend son autorité apostolique}
\VerseOne{}Au reste, je vous prie, moi Paul, par la douceur et la bonté de Christ - moi qui parais méprisable lorsque je suis en votre présence, et plein de hardiesse quand je suis éloigné,
\VS{2}je vous prie, dis-je, que lorsque je serai présent, il ne faille point que j'use de hardiesse, laquelle je me propose d’user contre quelques-uns qui nous regardent comme marchant selon la chair.
\VS{3}Mais en marchant dans la chair, nous ne combattons pas selon la chair.
\VS{4}Car les armes de notre guerre ne sont pas charnelles, mais elles sont puissantes par la vertu de Dieu, pour la destruction des forteresses.
\VS{5}Détruisant les raisonnements et toute hauteur qui s'élèvent contre la connaissance de Dieu, et amenant toute pensée captive à l'obéissance de Christ.
\VS{6}Et étant prêts à tirer vengeance de toute désobéissance, lorsque votre obéissance sera complète.
\VS{7}Considérez-vous les choses selon l'apparence ? Si quelqu'un se persuade qu’il est de Christ, qu'il se dise bien en lui-même que, comme il est de Christ, nous aussi nous sommes de Christ.
\VS{8}Car si même je veux me glorifier davantage de l’autorité que le Seigneur nous a donnée pour votre édification et non votre destruction, je ne saurais en avoir honte,
\VS{9}afin que je ne paraisse pas vouloir vous effrayer par mes lettres.
\VS{10}Car mes lettres, disent-ils, sont graves et fortes, mais la présence de corps est faible, et la parole est méprisable.
\VS{11}Que celui qui est tel, considère que tels nous sommes en paroles dans nos lettres, étant absents, tels aussi nous sommes dans nos actes, étant présents.
\VS{12}Car nous n'osons pas nous joindre ni nous comparer à quelques-uns de ceux qui se recommandent eux-mêmes. Mais en se mesurant à leur propre mesure et en se comparant à eux-mêmes, ils manquent d’intelligence.
\VS{13}Mais pour nous, nous ne voulons pas nous glorifier outre mesure, mais seulement dans la limite du champ d’action que Dieu nous assigné en nous amenant jusqu’à vous.
\VS{14}Car nous ne nous étendons pas nous même au delà des limites prescrites, comme si nous n'étions pas parvenus jusqu'à vous ; vu que nous sommes parvenus même jusqu’à vous par la prédication de l'Evangile de Christ.
\VS{15}Nous ne nous glorifions pas des travaux d’autrui qui sont hors de nos limites. Mais nous avons l’espérance, si votre foi augmente, de devenir encore plus grands parmi vous, selon les limites qui nous sont assignées,
\VS{16}jusqu'à évangéliser dans les lieux qui sont au-delà de chez vous, sans nous glorifier de ce qui a déjà été fait dans le domaine des autres.
\VS{17}Que celui qui se glorifie se glorifie dans le Seigneur.
\VS{18}Car ce n'est pas celui qui se recommande lui-même qui est approuvé, c'est celui que le Seigneur recommande\FTNT{Le Seigneur recommande ses serviteurs, il témoigne d’eux auprès des autres (Ac. 10:1-48). Un véritable serviteur de Dieu laisse au Seigneur le soin de témoigner de lui auprès des autres alors que les faux ouvriers se recommandent eux-mêmes (2 Co. 3:1).}.
\Chap{11}
\VerseOne{}Oh ! Si vous pouviez supportez de ma part un peu de folie ! Mais vous me supportez !
\VS{2}Car je suis jaloux de vous d'une jalousie de Dieu, parce que je vous ai fiancés à un seul Epoux, pour vous présenter à Christ comme une vierge pure.
\TextTitle{Les faux docteurs}
\VS{3}Mais je crains que comme le serpent séduisit Eve\FTNT{Ge. 3:1-6.} par sa ruse, vos pensées aussi ne se corrompent en se détournant de la simplicité à l’égard de Christ.
\VS{4}Car si quelqu'un vient vous prêcher un autre Jésus que nous n'avons pas prêché, ou si vous recevez un autre esprit que celui que vous avez reçu, ou un autre évangile que celui que vous avez embrassé, vous le supportez fort bien.
\VS{5}Or j'estime que je n'ai été en rien moindre que les plus excellents apôtres.
\VS{6}Si je suis un ignorant sous le rapport du langage, je ne le suis pourtant point sous celui de la connaissance, et nous l’avons montré parmi vous à tous égards et en toutes choses.
\VS{7}Ai-je commis une faute, en m’abaissant moi-même afin que vous soyez élevés, quand je vous ai annoncé gratuitement l’Evangile de Dieu ?
\VS{8}J'ai dépouillé les autres églises prenant de quoi m'entretenir pour vous . Et lorsque j’étais chez vous et que je me suis trouvé dans le besoin, je n’ai été à la charge de personne,
\VS{9}car les frères venus de la Macédoine ont pourvu à ce qui me manquait. Et en toutes choses, je me suis gardé d’être à votre charge, et je m'en garderai encore.
\VS{10}Par la vérité de Christ qui est en moi, j’atteste que ce sujet de gloire ne me sera point ravi dans les contrées de l'Achaïe.
\VS{11}Pourquoi ? Est-ce parce que je ne vous aime point ? Dieu le sait !
\VS{12}Mais ce que je fais, je le ferai encore, pour ôter ce prétexte à ceux qui cherchent un prétexte, afin qu’ils soient trouvés tels que nous dans les choses dont ils se glorifient.
\VS{13}Car ces hommes-là sont de faux apôtres, des ouvriers trompeurs qui se déguisent en apôtres de Christ.
\VS{14}Et cela n'est pas étonnant puisque Satan lui-même se déguise en ange de lumière\FTNT{Satan est maître en matière de déguisement et d’imitation.}.
\VS{15}Ce n'est donc pas un grand sujet d'étonnement si ses ministres aussi se déguisent en ministres de justice ; mais leur fin sera conforme à leurs œuvres.
\TextTitle{Sujets de gloire de Paul\FTNTT{2 Co. 11:16-12:18}}
\VS{16}Je le dis encore, afin que personne ne me regarde comme un insensé ; sinon, supportez-moi comme un insensé, afin que je me glorifie aussi un peu.
\VS{17}Ce que je dis, avec l’assurance d’avoir sujet de me glorifier, je ne le dis pas selon le Seigneur, mais comme par folie.
\VS{18}Puisqu’il en est plusieurs qui se glorifient selon la chair, je me glorifierai aussi.
\VS{19}Car vous supportez bien volontiers les insensés, vous qui êtes sages.
\VS{20}Si quelqu'un vous asservit, si quelqu'un vous dévore, si quelqu'un prend votre bien, si quelqu'un est arrogant, si quelqu'un vous frappe au visage, vous le supportez.
\VS{21}Je le dis avec honte, nous avons montré de la faiblesse. Mais si en quelque chose quelqu'un ose se glorifier, je parle en insensé, j'ai la même hardiesse !
\VS{22}Sont-ils Hébreux ? Moi aussi. Sont-ils Israélites ? Moi aussi. Sont-ils de la postérité d'Abraham ? Moi aussi.
\VS{23}Sont-ils ministres de Christ ? - je parle comme un insensé - je le suis plus qu'eux ; par les travaux, bien plus ; par les blessures, bien plus ; par les emprisonnements, bien plus. Plusieurs fois en danger de mort,
\VS{24}cinq fois j’ai reçu des Juifs quarante coups moins un,
\VS{25}j'ai été battu de verges trois fois, j'ai été lapidé une fois, j'ai fait naufrage trois fois, j'ai passé un jour et une nuit dans l’abîme.
\VS{26}Fréquemment en voyage, j’ai été en péril sur les fleuves, en péril de la part des brigands, en péril de la part de ceux de ma nation, en péril de la part des gentils, en péril dans les villes, en péril dans les déserts, en péril sur la mer, en péril parmi de faux frères.
\VS{27}J’ai été dans le travail et dans la peine, exposé à de nombreuses veilles, à la faim et à la soif, à des jeûnes multipliés, au froid et à la nudité.
\VS{28}Outre les choses de dehors, ce qui me tient assiégé tous les jours, c'est le soucis que j'ai de toutes les églises.
\VS{29}Qui est affaibli, que je ne sois faible ? Qui est scandalisé, que je n'en sois aussi brûlé ?
\VS{30}S'il faut se glorifier, je me glorifierai des choses qui sont de mon infirmité.
\VS{31}Le Dieu et Père de notre Seigneur Jésus-Christ, lui qui est béni éternellement, sait que je ne mens point.
\VS{32}A Damas, le gouverneur du roi Arétas avait fait garder la ville des Damascéniens pour me prendre,
\VS{33}mais on me descendit par une fenêtre, dans une corbeille, le long de la muraille, et ainsi j'échappai de ses mains.
\Chap{12}
\VerseOne{}Certes, il ne me convient pas de me glorifier, car j’en viendrai jusqu’aux visions et aux révélations du Seigneur.
\VS{2}Je connais un homme en Christ, qui fut ravi jusqu’au troisième ciel, il y a quatorze ans passés, (si ce fut dans son corps, je ne sais pas ; si ce fut hors du corps, je ne sais pas ; Dieu le sait).
\VS{3}Et je sais que cet homme (si ce fut dans son corps, ou si ce fut hors du corps, je ne sais pas ; Dieu le sait),
\VS{4}fut ravi dans le paradis, et qu’il entendit des paroles inexprimables qu'il n'est pas permis à l'homme de révéler.
\TextTitle{Paul et son écharde}
\VS{5}Je me glorifierai d'un tel homme, mais je ne me glorifierai point de moi-même, sinon de mes infirmités.
\VS{6}Si je voulais me glorifier, je ne serais pas un insensé, car je dirais la vérité ; mais je m'en abstiens, afin que personne ne m'estime au-dessus de ce qu'il me voit être, ou de ce qu'il entend dire de moi.
\VS{7}Mais pour que je ne sois pas enflé d’orgueil, à cause de l'excellence de ces révélations, il m'a été mis une écharde\FTNT{La nature exacte de l'écharde de Paul ne nous est pas détaillée. Elle lui avait été infligée par un «~ange de Satan~», par la volonté de Dieu. Nous constatons une chose qui est commune à tous les enfants de Dieu~: Paul avait un adversaire constamment aux aguets pour essayer de le décourager, le détruire ou l’intimider en s'opposant par tous les moyens à la mission que le Seigneur lui avait confiée. Cette écharde était aussi un moyen utilisé par Dieu pour garder Paul dans l’humilité.} dans la chair, un ange de Satan pour me souffleter et m’empêcher de m’enorgueillir.
\VS{8}Trois fois j'ai prié le Seigneur de faire que cet ange de Satan se retire de moi.
\VS{9}Mais le Seigneur m'a dit : Ma grâce te suffit, car ma puissance s’accomplit dans la faiblesse. Je me glorifierai donc bien volontiers de mes faiblesses, afin que la puissance de Christ habite en moi.
\VS{10}C’est pourquoi je me plais dans les faiblesses, dans les outrages, dans les calamités, dans les persécutions, et dans les angoisses pour Christ ; car quand je suis faible, c'est alors que je suis fort.
\TextTitle{Avertissements}
\VS{11}J'ai été insensé en me glorifiant, mais vous m'y avez contraint ; c’est par vous que je devais être recommandé, car je n'ai été inférieur en rien aux apôtres par excellence, quoique je ne sois rien.
\VS{12}Certainement les preuves de mon apostolat ont éclaté au milieu de vous par une patience à toute épreuve, par des signes, des prodiges et des miracles.
\VS{13}Car en quoi avez-vous été inférieurs aux autres églises, sinon en ce que je n’ai point été à votre charge ? Pardonnez-moi ce tort.
\VS{14}Voici pour la troisième fois que je suis prêt à aller vers vous, et je ne serai point à votre charge ; car ce ne sont pas vos biens que je cherche, c’est vous-mêmes. Ce n’est pas, en effet, aux enfants d’amasser pour les parents, mais aux parents pour les enfants\FTNT{Un bon père amasse dans le but de préparer l’avenir de ses enfants et non l’inverse.}.
\VS{15}Pour moi, je dépenserai très volontiers pour vous tout ce que j’ai, et je me donnerai encore moi-même pour vos âmes. En vous aimant davantage, serais-je moins aimé de vous ?
\VS{16}Soit ! Dira-t-on, que je ne vous ai point été à charge, c'est qu'étant un homme intelligent, je vous ai pris par ruse !
\VS{17}Ai-je donc tiré profit de vous par quelqu’un de ceux que je vous ai envoyés ?
\VS{18}J'ai engagé Tite à aller chez vous, et avec lui j’ai envoyé le frère. Tite a-t-il tiré profit de vous ? Et n'avons-nous pas lui et moi marché dans le même esprit ? N'avons-nous pas marché sur les mêmes traces ?
\VS{19}Pensez-vous encore que nous voulions nous justifier auprès de vous ? Nous parlons devant Dieu en Christ, et tout cela, mes très chers frères, pour votre édification.
\VS{20}Car je crains de ne pas vous trouver, à mon arrivée, tels que je voudrais, et d’être moi-même trouvé par vous tel que vous ne voudriez pas. Je crains de trouver des querelles, de la jalousie, des animosités, des rivalités, des médisances, des calomnies, de l’orgueil, des troubles.
\VS{21}Je crains qu’à mon arrivée, mon Dieu ne m’humilie de nouveau à votre sujet, et que je n’aie à pleurer sur plusieurs de ceux qui ont péché précédemment, et qui ne se sont pas repentis de l’impureté, de la débauche et des dérèglements dont ils se sont rendus coupables.
\Chap{13}
\TextTitle{S'examiner}
\VerseOne{}Je vais chez vous pour la troisième fois. Toute affaire se réglera sur la déclaration de deux ou de trois témoins\FTNT{De. 19:15.}.
\VS{2}Lorsque j’étais présent pour la deuxième fois, j’ai déjà dit, et aujourd’hui que je suis absent je dis encore d’avance à ceux qui ont péché précédemment et à tous les autres, que si je retourne chez vous, je n'épargnerai personne,
\VS{3}puisque vous cherchez la preuve que Christ parle par moi, lui qui n'est point faible envers vous, mais qui est puissant parmi vous.
\VS{4}Car il a été crucifié à cause de sa faiblesse, mais il vit par la puissance de Dieu ; et nous de même, nous sommes aussi faibles comme lui, mais nous vivrons avec lui par la puissance que Dieu a déployée envers vous.
\VS{5}Examinez-vous vous-mêmes pour savoir si vous êtes dans la foi ; éprouvez-vous vous-mêmes. Ne reconnaissez-vous pas que Jésus-Christ est en vous ? A moins peut-être que vous ne soyez désapprouvés.
\VS{6}Mais j'espère que vous reconnaîtrez que nous, nous ne sommes pas désapprouvés.
\VS{7}Et je prie Dieu que vous ne fassiez rien de mal, non pour paraître nous-mêmes approuvés, mais afin que vous pratiquiez ce qui est bien et que nous, nous soyons comme désapprouvés.
\VS{8}Car nous n’avons pas de pouvoir contre la vérité, nous n’en avons que pour la vérité.
\VS{9}Nous nous réjouissons lorsque nous sommes faibles, tandis que vous êtes forts ; et ce que nous demandons à Dieu, c’est votre perfectionnement.
\VS{10}C'est pourquoi j'écris ces choses étant absent, afin que présent, je n’aie pas à user de rigueur, selon l’autorité que le Seigneur m'a donnée pour l'édification et non point pour la destruction.
\TextTitle{Conclusion}
\VS{11}Au reste, mes frères, réjouissez-vous, perfectionnez-vous, consolez-vous, ayez un même sentiment, vivez en paix ; et le Dieu de charité et de paix sera avec vous.
\VS{12}Saluez-vous les uns les autres par un saint baiser. Tous les saints vous saluent.
\VS{13}Que la grâce du Seigneur Jésus-Christ, la charité de Dieu, et la communion du Saint-Esprit soient avec vous tous. Amen !
\PPE{}
\end{multicols}

%\clearpage\ShortTitle{Romains}\BookTitle{Romains}\BFont
\noindent\hrulefill
{\footnotesize
\textit{
\bigskip
{\centering{}
\\Thème : L'Evangile de Dieu
\\Auteur : Paul
\\Date de rédaction : Env. 56\\}
}
%\bigskip
\textit{
\\Rome est une ville située dans la région du Latium, au centre de l’Italie, à la confluence de l’Aniene et du Tibre. Centre de l’Empire romain,  elle domina  l’Europe, l’Afrique du Nord et le Moyen-Orient du 1er siècle avant J.-C. au 5ème siècle après J.-C.
%\bigskip
\\La lettre était destinée à l’Eglise de Rome, fondée sans doute par des chrétiens convertis au  travers du ministère de Paul et d’autres apôtres itinérants. Cette Eglise comptait quelques juifs mais surtout des membres d’origine païenne. Cette épître fut rédigée au cours du 3ème voyage missionnaire de Paul,  pendant les trois mois que l’apôtre passa à Corinthe. En attendant de leur rendre visite physiquement, Paul avait le désir de communiquer aux chrétiens de Rome les grandes lignes du principe de la grâce, dont il avait eu la révélation. Il y aborda plusieurs doctrines majeures comme le salut par la foi et la grâce ainsi que des enseignements pratiques sur l’amour, le devoir du chrétien et la sainteté.\bigskip
}
}
\par\nobreak\noindent\hrulefill
\begin{multicols}{2}
\Chap{1}
\TextTitle{Introduction : L’Evangile de Christ, puissance de Dieu pour le salut de tous}
\VerseOne{}Paul, serviteur de Jésus-Christ, appelé à être apôtre, mis à part pour annoncer l'Evangile de Dieu,
\VS{2}qu’il avait auparavant promis par ses prophètes dans les saintes Ecritures,
\VS{3}et qui concerne son Fils, qui est né de la postérité de David, selon la chair,
\VS{4}et qui a été pleinement déclaré Fils de Dieu avec puissance, selon l'Esprit de sainteté, par sa résurrection d'entre les morts, c'est-à-dire, Jésus-Christ notre Seigneur,
\VS{5}par qui nous avons reçu la grâce et l’apostolat, pour amener en son Nom tous les gentils à l’obéissance de la foi,
\VS{6}parmi lesquels aussi vous êtes, vous qui êtes appelés par Jésus-Christ.
\VS{7}A vous tous qui êtes à Rome, bien-aimés de Dieu, appelés à être saints\FTNT{Le terme «~saint~» est tiré du grec «~hagios~» qui signifie «~consacré à Dieu~», «~saint~», «~sacré~», «~pieux~». Ce mot est souvent utilisé au pluriel dans le testament de Jésus (Ac. 9:13 ;  Ac. 26:10). Il n’y a aucun rapport entre la compréhension catholique romaine du terme «~saint~» et l’enseignement biblique. Dans l’enseignement catholique romain, une personne ne devient pas sainte tant qu’elle n’a pas été béatifiée ou canonisée par le pape ou l’éminent évêque. Dans la Bible, tous ceux qui reçoivent Jésus-Christ par la foi sont appelés saints car ils sont mis à part pour Dieu. Dans le culte de l’église catholique romaine, les saints sont vénérés, priés, et parfois adorés. Dans la Bible, les saints sont appelés à adorer et à prier Dieu seul par Jésus-Christ homme, le seul Médiateur entre Dieu et les hommes (1 Ti. 2:5). Dans le Tanakh, les mots «~sanctifié~», «~saint~» et leurs dérivés viennent du mot hébreu «~Quodesch~» dont le sens général est : «~Mis à part pour Dieu~». Dans la Bible, ces mots sont appliqués à des objets et à des personnes. Le terme «~sanctification~» appliqué à des objets sous-entend l'idée qu'ils sont réservés uniquement pour le service de Dieu ; ils sont sanctifiés, mis à part pour Dieu. Dans le Testament de Jésus, il est appliqué aux personnes et comprend plusieurs sens : - Les croyants, par leur position, sont éternellement mis à part pour Dieu par la rédemption (Hé. 10:10-14). Ils sont donc considérés comme «~saints~», «~sanctifiés~» dès leur conversion (Ph. 1:1 ; Hé. 3:1). - Les croyants sont amenés à la sanctification par l'action du Saint-Esprit au moyen des Ecritures (Jn. 15:3 ; Jn. 17:17 ; 2 Co. 3:18 ; Ep. 5:25-26 ; 1Th. 5:23-24). - Les croyants attendent la venue du Seigneur pour la réalisation complète de leur sanctification (1 Co. 15:29-50 ; Ph. 3:20-21 ; Ep. 5:27 ; 1 Jn. 3:2 ; Ap. 22:12).} ; que la grâce et la paix vous soient données de la part de Dieu notre Père et du Seigneur Jésus-Christ.
\VS{8}Je rends premièrement grâces à mon Dieu par Jésus-Christ, au sujet de vous tous, de ce que votre foi est renommée dans le monde entier.
\VS{9}Car Dieu, que je sers en mon esprit dans l'Evangile de son Fils, m'est témoin que je fais sans cesse mention de vous,
\VS{10}demandant continuellement dans mes prières que je puisse enfin trouver, par la volonté de Dieu, quelque moyen favorable pour aller vers vous.
\VS{11}Car je désire extrêmement vous voir, pour vous communiquer quelque don spirituel, afin que vous soyez affermis ;
\VS{12}et aussi, afin qu'étant parmi vous, nous nous consolions ensemble par la foi qui nous est commune.
\VS{13}Or mes frères, je ne veux pas que vous ignoriez que j’ai souvent formé le dessein d'aller vers vous, afin de recueillir quelque fruit parmi vous, comme parmi les autres nations ; mais j'en ai été empêché jusqu'à présent.
\VS{14}Je me dois aux Grecs et aux barbares, aux sages et aux ignorants.
\VS{15}Ainsi, autant qu’il dépend de moi, je suis prêt à vous annoncer aussi l'Evangile à vous qui êtes à Rome.
\VS{16}Car je n'ai point honte de l'Evangile de Christ, vu qu'il est la puissance de Dieu pour le salut de tous ceux qui croient : Du Juif premièrement, puis du Grec,
\VS{17}parce qu’en lui est révélée la justice de Dieu pleinement de foi en foi, selon qu'il est écrit : Le juste vivra par la foi\FTNT{Ha. 2:4.}.
\TextTitle{Jugement sur ceux qui retiennent la vérité captive}
\VS{18}Car la colère de Dieu se révèle pleinement du ciel contre toute impiété et injustice des hommes qui retiennent injustement la vérité captive.
\VS{19}Car ce qu’on peut connaître de Dieu est manifesté parmi eux ; car Dieu le leur a fait connaître.
\VS{20}En effet, les perfections invisibles de Dieu, à savoir sa puissance éternelle et sa divinité, se voient comme à l’œil nu, depuis la création du monde, quand on les considère dans ses ouvrages, de sorte qu'ils sont inexcusables.
\VS{21}Parce qu'ayant connu Dieu, ils ne l'ont point glorifié comme Dieu, et ils ne lui ont point rendu grâces, mais ils se sont égarés dans leurs pensées, et leur cœur sans intelligence a été plongé dans les ténèbres.
\VS{22}Se vantant d’être sages, ils sont devenus fous.
\VS{23}Et ils ont changé la gloire du Dieu incorruptible en images\FTNT{Ex. 20:4-5 ; Mt. 22:20 ; Mc. 12:16 ; Lu. 20:24. Ap.13:14-15 ; Ap. 14:9-11 ; Ap. 15:2 ; Ap. 16:2 ; 19:20 ;  Ap. 20:4.} représentant l'homme corruptible, des oiseaux, des quadrupèdes, et des reptiles.
\TextTitle{Les conséquences de l'endurcissement des hommes}
\VS{24}C'est pourquoi aussi Dieu les a livrés aux convoitises de leurs cœurs et à l’impureté\FTNT{Dieu les a livrés à l’esprit d’égarement (1 R. 22 ; 2 Th. 2:10-13).}, ainsi ils déshonorent eux-mêmes leurs propres corps ;
\VS{25}eux qui ont changé la vérité de Dieu en mensonge, et qui ont adoré et servi la créature, au lieu du Créateur, qui est béni éternellement. Amen !
\VS{26}C'est pourquoi Dieu les a livrés à des passions infâmes, car leurs femmes ont changé l'usage naturel en celui qui est contre la nature.
\VS{27}Et de même les hommes, abandonnant l'usage naturel de la femme, se sont enflammés dans leurs désirs les uns envers les autres, commettant homme avec homme des choses infâmes, et recevant en eux-mêmes le salaire que méritait leur égarement.
\VS{28}Car comme ils ne se sont pas souciés de connaître Dieu, aussi Dieu les a livrés à leur sens réprouvé, pour commettre des choses indignes.
\VS{29}Etant remplis de toute espèce d’injustice, d'impureté, de méchanceté, d'avarice, de malignité, pleins d'envie, de meurtre, de querelles, de fraude, de mauvaises mœurs,
\VS{30}rapporteurs, médisants, haïssant Dieu, outrageux, orgueilleux, vains, ingénieux au mal, rebelles à leurs parents,
\VS{31}dépourvus d’intelligence, de loyauté, d’affection naturelle, de miséricorde.
\VS{32}Et bien qu'ils connaissent le jugement de Dieu, déclarant dignes de mort ceux qui commettent de telles choses, non seulement ils les font, mais encore ils approuvent ceux qui les font.
\Chap{2}
\TextTitle{Condamnation du moralisme}
\VerseOne{}C'est pourquoi, ô homme, qui que tu sois, toi qui juges les autres, tu es donc inexcusable ; car en jugeant les autres, tu te condamnes toi-même, puisque toi qui juges, tu commets les mêmes choses.
\VS{2}Or nous savons que le jugement de Dieu est selon la vérité pour ceux qui commettent de telles choses.
\VS{3}Et penses-tu, ô homme, qui juges ceux qui commettent de telles choses, et qui les commets, que tu échapperas au jugement de Dieu ?
\VS{4}Ou méprises-tu les richesses de sa douceur, et de sa patience, et de sa bonté ; ne reconnaissant pas que la bonté de Dieu te convie à la repentance ?
\VS{5}Mais par ta dureté, et par ton cœur qui est sans repentance, tu t'amasses la colère pour le jour de la colère, et de la manifestation du juste jugement de Dieu,
\VS{6}qui rendra à chacun selon ses œuvres ;
\VS{7}à savoir la vie éternelle à ceux qui, en persévérant dans les bonnes œuvres, cherchent la gloire, l'honneur et l'immortalité.
\VS{8}Mais il y aura de l'indignation et de la colère contre ceux qui ont un esprit de dispute, et qui se rebellent contre la vérité, et obéissent à l'injustice.
\VS{9}Il y aura tribulation et angoisse sur toute âme d'homme qui fait le mal, pour le Juif premièrement, puis pour le Grec.
\VS{10}Mais gloire, honneur, et paix pour quiconque fait le bien ; pour le Juif premièrement, puis pour le Grec.
\VS{11}Car Dieu n'a point d'égard à l'apparence des personnes.
\VS{12}Tous ceux qui auront péché sans la loi, périront aussi sans la loi ; et tous ceux qui auront péché ayant la loi, seront jugés par la loi.
\VS{13}Car ce ne sont pas, en effet, ceux qui écoutent la loi qui sont justes devant Dieu, mais ce sont ceux qui la mettent en pratique qui seront justifiés.
\VS{14}Or quand les gentils, qui n'ont point la loi, font naturellement ce que prescrit la loi, n'ayant point la loi, ils sont une loi pour eux-mêmes.
\VS{15}Et ils montrent par-là que l’œuvre de la loi est écrite dans leurs cœurs, puisque leur conscience leur rend témoignage, et que leurs pensées les accusent ou les défendent.
\VS{16}Tous, dis-je, donc seront jugés le jour où Dieu jugera les secrets des hommes par Jésus-Christ, selon mon Evangile.
\TextTitle{Les Juifs, connaissant la loi, sont condamnés par leur transgression de la loi}
\VS{17}Voici, tu portes le nom de Juif, tu te reposes entièrement sur la loi, et tu te glorifies de Dieu ;
\VS{18}tu connais sa volonté, et tu sais discerner ce qui est contraire, étant instruit par la loi ; 
\VS{19}et tu te crois être le conducteur des aveugles, la lumière de ceux qui sont dans les ténèbres,
\VS{20}le docteur des insensés, le maître des ignorants, ayant le modèle de la science et de la vérité dans la loi.
\VS{21}Toi donc qui enseignes les autres, tu ne t’enseignes pas toi-même ! Toi qui prêches de ne pas dérober, tu dérobes !
\VS{22}Toi qui dis de ne pas commettre d’adultère, tu commets l’adultère ! Toi qui as en abomination les idoles, tu commets des sacrilèges !
\VS{23}Toi qui te glorifies de la loi, tu déshonores Dieu par la transgression de la loi.
\VS{24}Car le nom de Dieu est blasphémé parmi les gentils à cause de vous comme cela est écrit.
\VS{25}Il est vrai que la circoncision est profitable, si tu gardes la loi ; mais si tu es transgresseur de la loi, ta circoncision devient incirconcision.
\VS{26}Si donc l’incirconcis observe les ordonnances de la loi, son incirconcision ne sera-t-elle pas tenue pour circoncision ?
\VS{27}L’incirconcis de nature, qui accomplit la loi, ne te condamnera-t-il pas, toi qui la transgresses, tout en ayant la lettre de la loi et la circoncision ?
\VS{28}Le Juif, ce n’est pas celui qui en a les apparences\FTNT{Le formalisme (2 Ti. 3:5). L’apparence de la piété correspond aux vêtements des brebis : «~Gardez-vous des faux prophètes, ils viennent à vous en habits de brebis, mais au-dedans ce sont des loups ravisseurs.~» (Mt 7:15). «~Puis je vis une autre bête qui montait de la terre, et qui avait deux cornes semblables à celles de l'Agneau ; mais elle parlait comme le dragon.~» Ap. 13:11. Il a l’apparence d’un agneau, mais sa voix est celle du dragon, c’est-à-dire Satan.} ; et la circoncision, ce n’est pas celle qui est visible dans la chair.
\VS{29}Mais le Juif, c’est celui qui l’est intérieurement ; et la circoncision, c’est celle du cœur, selon l’Esprit et non selon la lettre. La louange de ce Juif ne vient pas des hommes, mais de Dieu.
\Chap{3}
\TextTitle{L'avantage du Juif peut devenir une condamnation}
\VerseOne{}Quel est donc l'avantage du Juif, ou quelle est l’utilité de la circoncision ?
\VS{2}Cet avantage est grand de toute manière, et tout d’abord en ce que les oracles de Dieu leur ont été confiés.
\VS{3}Eh quoi ! Si quelques-uns n'ont pas cru, leur incrédulité anéantira-t-elle la fidélité de Dieu ?
\VS{4}Nullement ! Que Dieu au contraire soit reconnu pour vrai, et tout homme pour menteur ; selon ce qui est écrit : Afin que tu sois trouvé juste dans tes paroles, et que tu triomphes lorsqu’on te juge\FTNT{Ps. 51:6.}.
\VS{5}Mais si notre injustice établit la justice de Dieu, que dirons-nous ? Dieu est-il injuste quand il déchaine sa colère ? (Je parle à la manière des hommes.)
\VS{6}Nullement ! Autrement, comment Dieu jugera-t-il le monde ?
\VS{7}Et si par mon mensonge la vérité de Dieu est plus abondante pour sa gloire, pourquoi suis-je encore condamné comme pécheur ?
\VS{8}Et pourquoi ne ferions-nous pas le mal, afin qu'il en arrive du bien, comme quelques-uns, qui nous calomnient, prétendent que nous le disons ? La condamnation de ces gens est juste.
\TextTitle{Juifs et Grecs coupables devant Dieu}
\VS{9}Quoi donc ! Sommes-nous plus excellents ? Nullement. Car nous avons déjà prouvé que tous, tant Juifs que Grecs, sont assujettis au péché.
\VS{10}Selon qu'il est écrit : Il n'y a point de juste, pas même un seul\FTNT{Ps. 14:3.}.
\VS{11}Il n'y a personne qui ait de l'intelligence, il n'y a personne qui recherche Dieu.
\VS{12}Ils se sont tous égarés, ils se sont tous corrompus : Il n'y en a aucun qui fasse le bien, pas même un seul.
\VS{13}Leur gosier est un sépulcre ouvert ; ils se servent de leur langue pour tromper ; il y a du venin d'aspic sous leurs lèvres.
\VS{14}Leur bouche est pleine de malédictions et d'amertume.
\VS{15}Leurs pieds sont légers pour répandre le sang.
\VS{16}La destruction et la misère sont sur leurs voies.
\VS{17}Et ils n'ont point connu la voie de la paix.
\VS{18}La crainte de Dieu n'est pas devant leurs yeux\FTNT{Ps. 14.}.
\VS{19}Or nous savons que tout ce que la loi dit, elle le dit à ceux qui sont sous la loi, afin que toute bouche soit fermée, et que tout le monde soit reconnu coupable devant Dieu.
\VS{20}C'est pourquoi personne ne sera justifié devant lui par les œuvres de la loi, puisque c’est par la loi que vient la connaissance du péché.
\TextTitle{La justification par la foi}
\VS{21}Mais maintenant, sans la loi, la justice de Dieu est manifestée, à laquelle rendent témoignage la loi et les prophètes.
\VS{22}La justice, dis-je, de Dieu par la foi en Jésus-Christ, envers tous et sur tous ceux qui croient. Car il n'y a point de distinction.
\VS{23}Car tous ont péché\FTNT{Le mot péché vient du terme grec «~hamartano~» : «~manquer la marque, manquer le chemin de la droiture et de l’honneur, s’éloigner de la loi de Dieu~». Le péché est la violation délibérée de la loi divine et l’absence de la droiture.} et sont entièrement privés de la gloire de Dieu.
\VS{24}Et ils sont gratuitement justifiés par sa grâce, par la rédemption\FTNT{La rédemption est la délivrance par le paiement d’un prix. Trois termes grecs sont utilisés pour parler de la rédemption : - Agorazo : acheter un objet au marché (agora signifiant marché). Les pécheurs sont considérés comme des esclaves vendus au marché (Ro. 7:14). - Exagorazo : acheter et amener un objet hors du marché (Ga. 3:13 ; Ga. 4:5). L’esclave acheté et amené hors du marché est définitivement délivré. - Lutroo : détacher, rendre libre (Lu. 24:21 ; Tit. 2:14 ; 1 P. 1:18.) Jésus-Christ nous a délivrés du péché, de la puissance de Satan et de la loi mosaïque. (Col. 1:12-14 ; Col. 2:14-17 ; 1 Jn. 3:5).} qui est en Jésus-Christ.
\VS{25}C’est lui que Dieu a destiné à être, par son sang, la victime propitiatoire\FTNT{Le terme propitiation vient du grec «~hilastérion~» qui signifie «~ce qui est expié, ce qui rend propice ou le don qui assure la propitiation~». C’est aussi le lieu où s’accomplit la propitiation (Hé. 9:5), c’est-à-dire le couvercle de l’arche. Lors du grand jour des expiations (Yom Kippour en hébreu), l’aspersion du sang était faite sur le propitiatoire (Lé. 16:14). Le Seigneur Jésus-Christ est notre victime expiatoire (1 Jn. 2:2 ; 1 Jn. 4:10).} pour ceux qui croiraient, afin de montrer sa justice, parce qu’il avait laissé impunis les péchés commis auparavant, au temps de sa patience.
\VS{26}Il montre, dis-je, sa justice dans le temps présent, de manière à être trouvé juste tout en justifiant celui qui a la foi en Jésus.
\VS{27}Où est donc le sujet de se glorifier ? Il est exclu. Par quelle loi ? Est-ce par la loi des œuvres ? Non, mais par la loi de la foi.
\VS{28}Nous concluons donc que l'homme est justifié par la foi, sans les œuvres de la loi.
\TextTitle{Circoncis et incirconcis, justifiés par la foi}
\VS{29}Dieu est-il seulement le Dieu des Juifs ? Ne l'est-il pas aussi des gentils ? Certes, il l'est aussi des gentils,
\VS{30}puisqu’il y a un seul Dieu qui justifiera par la foi les circoncis, et aussi les incirconcis par la foi.
\VS{31}Anéantissons-nous donc la loi par la foi ? Nullement ! Mais au contraire, nous affermissons la loi.
\Chap{4}
\TextTitle{Abraham et David justifiés par la foi\FTNTT{cp. v. 18-25}}
\VerseOne{}Que dirons-nous donc, qu'Abraham, notre père, a obtenu selon la chair ?
\VS{2}Certes, si Abraham a été justifié par les œuvres, il a de quoi se glorifier, mais non pas envers Dieu.
\VS{3}Car que dit l'Ecriture ? Qu’Abraham a cru en Dieu, et que cela lui a été imputé à justice\FTNT{Ge. 15:6.}.
\VS{4}Or à celui qui fait les œuvres, le salaire ne lui est pas imputé comme une grâce, mais comme une chose due.
\VS{5}Mais à celui qui ne fait pas les œuvres, mais qui croit en celui qui justifie le méchant, sa foi lui est imputée à justice.
\VS{6}De même, David exprime le bonheur de l'homme à qui Dieu impute la justice sans les œuvres, en disant :
\VS{7}Heureux sont ceux à qui les iniquités sont pardonnées, et dont les péchés sont couverts.
\VS{8}Heureux l'homme à qui le Seigneur n’impute pas son péché\FTNT{Ps. 32:1-2.}.
\TextTitle{Abraham obtient la justification par la foi avant sa circoncision}
\VS{9}Cette déclaration de bénédiction, est-elle seulement pour les circoncis, ou également pour les incirconcis ? Car nous disons que la foi a été imputée à Abraham à justice.
\VS{10}Comment donc lui a-t-elle été imputée ? Etait-ce après, ou avant sa circoncision ? Il n’était pas encore circoncis, il était incirconcis.
\VS{11}Et il reçut le signe de la circoncision comme sceau de la justice, qu’il avait obtenue par la foi, quand il était incirconcis, afin d’être le père de tous les incirconcis qui croient, pour que la justice leur soit aussi imputée ;
\VS{12}et le père des circoncis, qui ne sont pas seulement circoncis, mais encore qui marchent sur les traces de la foi de notre père Abraham, quand il était incirconcis.
\TextTitle{La justification s'accomplit sans la loi}
\VS{13}En effet, ce n’est pas par loi que la promesse d'être héritier du monde a été faite à Abraham, ou à sa postérité, mais par la justice de la foi.
\VS{14}Car, si les héritiers le sont par la loi, la promesse est annulée, et la foi est vaine
\VS{15}car la loi produit la colère ; car là où il n'y a point de loi, il n'y a point non plus de transgression.
\VS{16}C'est pourquoi les héritiers le sont par la foi, pour que ce soit par la grâce, et afin que la promesse soit assurée à toute la postérité ; non seulement à celle qui est de la loi, mais aussi à celle qui est de la foi d'Abraham, qui est le père de nous tous,
\VS{17}selon qu'il est écrit : Je t'ai établi père de plusieurs nations\FTNT{Ge 17:4-5.}. Il est notre père devant celui auquel il a cru, Dieu qui donne la vie aux morts, et qui appelle les choses qui ne sont point, comme si elles étaient.
\VS{18}Et Abraham ayant espéré contre toute espérance, crut qu'il deviendrait le père de plusieurs nations, selon ce qui lui avait été dit : Ainsi sera ta postérité.
\VS{19}Et sans faiblir dans la foi, il ne considéra point que son corps était déjà usé ; puisqu’il avait environ cent ans, et que Sara n’était plus en âge d'avoir des enfants.
\VS{20}Et il ne douta point de la promesse de Dieu par incrédulité, mais il fut fortifié par la foi, donnant gloire à Dieu,
\VS{21}étant pleinement persuadé que celui qui lui avait fait la promesse était aussi puissant pour l'accomplir.
\VS{22}C'est pourquoi cela lui fut imputé à justice.
\VS{23}Mais ce n’est pas à cause de lui seul qu’il est écrit que cela lui fut imputé à justice ;
\VS{24}c’est encore à cause de nous, à qui cela sera imputé, à nous, dis-je, qui croyons en celui qui a ressuscité des morts, Jésus notre Seigneur,
\VS{25}qui a été livré pour nos offenses, et est ressuscité pour notre justification.
\Chap{5}
\TextTitle{La justification permet la réconciliation avec Dieu}
\VerseOne{}Etant donc justifiés\FTNT{La justification est l’œuvre de Dieu par laquelle la justice de Jésus est comptée en faveur du pécheur, de sorte que le pécheur est déclaré juste par Dieu (Ro. 4 : 3 ; Ro. 5:1-9 ; Ga. 2:16 ; Ga. 3:11). Cette justice n’est pas obtenue par les efforts de la personne sauvée. La justification est une action instantanée qui a pour résultat la vie éternelle. Elle repose totalement et exclusivement sur le sacrifice de Jésus à la croix (1 Pi. 2:24). Elle ne peut être reçue que par la foi en Jésus-Christ (Ep. 2:8-9). La justification est un acte d’imputation divine et non une reconnaissance personnelle de l’homme. Elle provient de la grâce (Ro. 3:24 ; Tit. 3:7).} par la foi, nous avons la paix avec Dieu, par notre Seigneur Jésus-Christ.
\VS{2}Par lequel aussi nous avons été amenés par la foi à cette grâce, dans laquelle nous tenons ferme ; et nous nous glorifions dans l'espérance de la gloire de Dieu.
\VS{3}Bien plus, nous nous glorifions même dans les afflictions ; sachant que l'affliction produit la persévérance ;
\VS{4}et la persévérance l'épreuve ; et l'épreuve l'espérance.
\VS{5}Or l'espérance ne trompe point, parce que l'amour de Dieu est répandu dans nos cœurs par le Saint-Esprit qui nous a été donné.
\VS{6}Car lorsque nous étions encore sans force, Christ est mort en son temps pour nous qui étions des impies.
\VS{7}A peine mourrait-on pour un juste ; quelqu’un peut-être mourrait pour un homme de bien.
\VS{8}Mais Dieu prouve son amour envers nous, en ce que lorsque nous étions encore pécheurs, Christ est mort pour nous.
\VS{9}Etant donc maintenant justifiés par son sang, à plus forte raison serons-nous sauvés par lui de la colère.
\VS{10}Car si, lorsque nous étions ennemis, nous avons été réconciliés avec Dieu par la mort de son Fils, à plus forte raison, étant réconciliés, serons-nous sauvés par sa vie.
\VS{11}Et non seulement cela, mais encore nous nous glorifions même en Dieu par notre Seigneur Jésus-Christ, par qui nous avons maintenant obtenu la réconciliation.
\TextTitle{Parallèle entre l'œuvre de Jésus-Christ et celle d'Adam}
\VS{12}C'est pourquoi comme par un seul homme le péché est entré dans le monde, et par le péché la mort, et qu’ainsi la mort s’est étendue sur tous les hommes, parce que tous ont péché…
\VS{13}Car jusqu'à la loi le péché était dans le monde ; or le péché n'est point imputé quand il n'y a point de loi.
\VS{14}Mais la mort a régné depuis Adam jusqu'à Moïse, même sur ceux qui n'avaient pas péché par une transgression semblable à celle d’Adam, lequel est la figure de celui qui devait venir.
\VS{15}Mais il n'en est pas du don gratuit comme de l'offense ; car si par l'offense d'un seul il en est beaucoup qui sont morts, à plus forte raison la grâce de Dieu, et le don de la grâce, venant d'un seul homme, à savoir de Jésus-Christ, ont-ils été abondamment répandus sur plusieurs.
\VS{16}Et il n'en est pas du don comme de ce qui est arrivé par un seul qui a péché ; car c’est après une seule offense que le jugement est devenu condamnation, mais le don gratuit devient justification après plusieurs offenses.
\VS{17}Si par l'offense d'un seul la mort a régné par lui seul, à plus forte raison ceux qui reçoivent l'abondance de la grâce, et du don de la justice, régneront-ils dans la vie par Jésus-Christ lui seul.
\VS{18}Ainsi donc, comme par une seule offense la condamnation est venue sur tous les hommes, de même par un acte de justice la justification qui donne la vie s’étend à tous les hommes.
\VS{19}Car, comme par la désobéissance d'un seul homme plusieurs ont été rendus pécheurs, de même par l'obéissance d'un seul plusieurs seront rendus justes.
\VS{20}Or la loi est intervenue afin que l'offense abonde, mais là où le péché a abondé, la grâce a surabondé,
\VS{21}afin que, comme le péché a régné par la mort, ainsi la grâce règne par la justice pour donner la vie éternelle, par Jésus-Christ notre Seigneur.
\Chap{6}
\TextTitle{Délivré de la puissance du péché lié au cœur de l’homme }
\VerseOne{}Que dirons-nous donc ? Demeurerions-nous dans le péché, afin que la grâce abonde ?
\VS{2}A Dieu ne plaise ! Car nous qui sommes morts au péché, comment vivrions-nous encore dans le péché ?
\VS{3}Ignoriez-vous que nous tous qui avons été baptisés en Jésus-Christ, c’est en sa mort que nous avons été baptisés ?
\VS{4}Nous avons donc été ensevelis avec lui par le baptême en sa mort ; afin que comme Christ est ressuscité des morts par la gloire du Père, de même nous aussi nous marchions en nouveauté de vie.
\VS{5}Car, si nous sommes devenus une même plante avec lui par la conformité à sa mort, nous le serons aussi par la conformité à sa résurrection.
\VS{6}Sachant que notre vieil homme a été crucifié avec lui, afin que le corps du péché soit détruit, pour que nous ne soyons plus esclaves du péché.
\VS{7}Car celui qui est mort est libre du péché.
\VS{8}Or si nous sommes morts avec Christ, nous croyons que nous vivrons aussi avec lui,
\VS{9}sachant que Christ ressuscité des morts ne meurt plus, et que la mort n'a plus de pouvoir sur lui.
\VS{10}Car il est mort, et c’est pour le péché qu’il est mort une fois pour toutes ; il est revenu à la vie, et c’est pour Dieu qu’il est vivant.
\TextTitle{Mort au péché pour une vie nouvelle en Dieu}
\VS{11}Ainsi vous-mêmes, considérez-vous comme morts au péché, et comme vivants pour Dieu en Jésus-Christ notre Seigneur.
\VS{12}Que le péché ne règne donc point dans votre corps mortel, et n’obéissez pas à ses convoitises.
\VS{13}Et ne livrez pas vos membres au péché comme des instruments d'iniquité ; mais donnez-vous vous-mêmes à Dieu comme de morts étant devenus vivants, et offrez vos membres à Dieu pour être des instruments de justice.
\VS{14}Car le péché n'aura pas de domination sur vous, parce que vous n'êtes point sous la loi, mais sous la grâce.
\VS{15}Quoi donc ? Pécherions-nous parce que nous ne sommes point sous la loi, mais sous la grâce ? A Dieu ne plaise !
\VS{16}Ne savez-vous pas qu’en vous livrant à quelqu’un comme esclaves pour lui obéir, vous êtes esclaves de celui à qui vous obéissez, soit du péché qui conduit à la mort, soit de l'obéissance qui conduit à la justice ?
\VS{17}Mais grâces à Dieu de ce qu'ayant été les esclaves du péché, vous avez obéi de cœur à la forme expresse de la doctrine dans laquelle vous avez été élevés.
\VS{18}Ayant donc été affranchis du péché, vous avez été asservis à la justice.
\VS{19}Je parle à la façon des hommes, à cause de l'infirmité de votre chair. Comme donc vous avez appliqué vos membres pour servir à la souillure et à l’iniquité, ainsi appliquez vos membres pour servir à la justice en sainteté.
\VS{20}Car lorsque vous étiez esclaves du péché, vous étiez libres à l'égard de la justice.
\VS{21}Quel fruit portiez-vous alors ? Des fruits dont vous avez honte maintenant. Car la fin de ces choses c’est la mort.
\VS{22}Mais maintenant que vous êtes affranchis du péché, et asservis à Dieu, vous avez pour fruit la sanctification, et pour fin la vie éternelle.
\VS{23}Car le salaire du péché, c'est la mort ; mais le don gratuit de Dieu, c'est la vie éternelle par Jésus-Christ notre Seigneur.
\Chap{7}
\TextTitle{Le chrétien lié à Christ comme à un époux}
\VerseOne{}Ignorez-vous, frères, car je parle à des gens qui connaissent la loi, que la loi exerce son pouvoir sur l’homme aussi longtemps qu’il vit ?
\VS{2}Car la femme qui est sous la puissance d'un mari, est liée à son mari par la loi tandis qu'il est en vie ; mais si son mari meurt, elle est délivrée de la loi du mari.
\VS{3}Si donc, du vivant de son mari, elle épouse un autre homme, elle sera appelée adultère ; mais si son mari meurt, elle est délivrée de la loi, de sorte qu'elle ne sera point adultère si elle épouse un autre homme.
\VS{4}Ainsi donc, vous aussi, mes frères, vous avez été, par le corps de Christ, mis à mort en ce qui concerne la loi, pour que vous apparteniez à un autre, à savoir, à celui qui est ressuscité des morts, afin que nous portions des fruits pour Dieu.
\VS{5}Car lorsque nous étions dans la chair, les passions des péchés excitées par la loi, agissaient dans nos membres de manière à produire des fruits pour la mort.
\VS{6}Mais maintenant nous sommes délivrés de la loi, étant morts à cette loi sous laquelle nous étions retenus ; afin que nous servions Dieu dans un esprit nouveau, et non selon la lettre qui a vieilli.
\TextTitle{La loi a révélé le péché mais la délivrance vient par Jésus-Christ}
\VS{7}Que dirons-nous donc ? La loi est-elle péché ? Nullement ! Au contraire, je n'ai connu le péché que par la loi ; car je n’aurais pas connu la convoitise, si la loi n’avait pas dit : Tu ne convoiteras point\FTNT{Ex. 20:17.}.
\VS{8}Et le péché, saisissant l’occasion, produisit en moi, par le commandement, toutes sortes de convoitises ; parce que sans la loi le péché est mort.
\VS{9}Pour moi, étant autrefois sans loi, je vivais. Mais quand le commandement vint, le péché reprit vie, et moi je mourus.
\VS{10}Ainsi, le commandement qui conduit à la vie se trouva pour moi conduire à la mort.
\VS{11}Car le péché, saisissant l’occasion, me séduisit par le commandement, et par lui me fit mourir.
\VS{12}La loi donc est sainte, et le commandement est saint, juste, et bon.
\VS{13}Ce qui est bon a-t-il donc été pour moi une cause de mort ? Nullement ! Mais c’est le péché, afin qu'il se manifeste comme péché, en me donnant la mort par ce qui est bon, et que par le commandement, il devienne condamnable au plus haut point.
\VS{14}Car nous savons, en effet, que la loi est spirituelle ; mais moi, je suis charnel, vendu au péché.
\TextTitle{[la connaissance du bien incapable de délivrer l'homme du péché}
\VS{15}Car je n'approuve pas ce que je fais, puisque je ne fais point ce que je veux, mais je fais ce que je hais.
\VS{16}Or si je fais ce que je ne veux pas, je reconnais par cela même que la loi est bonne.
\VS{17}Et maintenant donc ce n'est plus moi qui fais cela, mais c'est le péché qui habite en moi.
\VS{18}Ce qui est bon, je le sais, n’habite pas en moi, c’est-à-dire dans ma chair. J’ai la volonté, mais non le pouvoir de faire le bien.
\VS{19}Car je ne fais pas le bien que je veux, mais je fais le mal que je ne veux point.
\VS{20}Or si je fais ce que je ne veux point, ce n'est plus moi qui le fais, mais c'est le péché qui habite en moi.
\VS{21}Je trouve donc cette loi au-dedans de moi : Quand je veux faire le bien, le mal est attaché à moi.
\VS{22}Car je prends bien plaisir à la loi de Dieu quant à l'homme intérieur,
\VS{23}mais je vois dans mes membres une autre loi, qui combat contre la loi de mon entendement\FTNT{Entendement : Du grec «~nous~», c’est-à-dire l’esprit, l’intelligence, le bon sens, la raison.}, et qui me rend prisonnier à la loi du péché qui est dans mes membres.
\VS{24}Ah, misérable que je suis ! Qui me délivrera du corps de cette mort ?
\TextTitle{Seul l'Esprit de Christ libère de la loi du péché}
\VS{25}Je rends grâces à Dieu par Jésus-Christ notre Seigneur !… Ainsi donc, moi-même, je suis par l’entendement esclave de la loi de Dieu, et je suis par la chair esclave de la loi du péché.
\Chap{8}
\VerseOne{}Il n'y a donc maintenant aucune condamnation pour ceux qui sont en Jésus-Christ, qui marchent, non selon la chair, mais selon l'Esprit.
\VS{2}Parce que la loi de l'Esprit de vie qui est en Jésus-Christ m'a affranchi de la loi du péché et de la mort.
\VS{3}Car chose impossible à la loi, parce que la chair la rendait impuissante, Dieu a condamné le péché dans la chair, en envoyant, à cause du péché, son propre Fils dans une chair semblable à celle du péché.
\VS{4}Afin que la justice de la loi soit accomplie en nous, qui ne marchons point selon la chair, mais selon l'Esprit.
\TextTitle{L'affection de l'Esprit opposée à celle de la chair\FTNTT{cp. Ga. 5:15-18}}
\VS{5}Car ceux, en effet, qui vivent selon la chair, s’affectionnent aux choses de la chair, tandis que ceux qui vivent selon l'Esprit, s’affectionnent aux choses de l'Esprit.
\VS{6}Or l'affection de la chair c’est la mort, tandis que l'affection de l'Esprit c’est la vie et la paix.
\VS{7}Car l'affection de la chair est inimitié contre Dieu, parce qu’elle ne se soumet pas à la loi de Dieu, et qu’elle ne le peut même pas.
\VS{8}C'est pourquoi ceux qui vivent selon la chair ne sauraient plaire à Dieu.
\VS{9}Pour vous, vous ne vivez pas selon la chair, mais selon l'Esprit, si du moins l'Esprit de Dieu habite en vous. Si quelqu'un n'a pas l'Esprit de Christ\FTNT{Notez que le Saint-Esprit est aussi appelé l’Esprit de Jésus (Ac. 16:7).}, il ne lui appartient pas.
\VS{10}Et si Christ est en vous, le corps est bien mort à cause du péché, mais l'Esprit est vie à cause de la justice.
\VS{11}Et si l'Esprit de celui qui a ressuscité Jésus d’entre les morts habite en vous, celui qui a ressuscité Christ d’entre les morts rendra aussi la vie à vos corps mortels par son Esprit qui habite en vous.
\VS{12}Ainsi donc, mes frères, nous ne sommes point redevables à la chair, pour vivre selon la chair.
\VS{13}Car si vous vivez selon la chair, vous mourrez ; mais si par l'Esprit vous faites mourir les actions du corps, vous vivrez.
\TextTitle{L'Esprit d'adoption\FTNTT{Ga. 4:7}}
\VS{14}Car tous ceux qui sont conduits par l'Esprit de Dieu sont enfants de Dieu.
\VS{15}Et vous n'avez point reçu un esprit de servitude pour être encore dans la crainte ; mais vous avez reçu l'Esprit d'adoption, par lequel nous crions Abba, c'est-à-dire Père.
\VS{16}L’Esprit lui-même rend témoignage à notre esprit que nous sommes enfants de Dieu.
\VS{17}Et si nous sommes enfants, nous sommes aussi héritiers : Héritiers, dis-je, de Dieu, et cohéritiers de Christ ; si toutefois nous souffrons avec lui, afin d’être glorifiés avec lui.
\TextTitle{La gloire à venir\FTNTT{cp. Ge. 3:18-19}}
\VS{18}Car tout bien compté, j'estime que les souffrances du temps présent ne sauraient être comparables à la gloire à venir qui doit être révélée pour nous.
\VS{19}Aussi, la création attend-elle avec un ardent désir la révélation des fils de Dieu.
\VS{20}Car la création a été soumise à la vanité, non de son gré, mais à cause de celui qui l’y a soumise ;
\VS{21}avec l’espérance qu’elle aussi sera affranchie de la servitude de la corruption, pour avoir part à la liberté de la gloire des enfants de Dieu.
\VS{22}Or, nous savons que jusqu’à ce jour, toute la création soupire et souffre les douleurs de l’enfantement.
\VS{23}Et non seulement elle, mais nous aussi, qui avons les prémices de l'Esprit ; nous-mêmes, dis-je, soupirons en nous-mêmes, en attendant l'adoption, c'est-à-dire la rédemption de notre corps\FTNT{1 Co. 15:35-43 ; 1 Co. 15:51-54.}.
\VS{24}Car c’est en espérance que nous sommes sauvés. Or l’espérance qu’on voit n’est plus espérance : Ce qu’on voit, peut-on l’espérer encore ?
\VS{25}Mais si nous espérons ce que nous ne voyons pas, c'est que nous l'attendons avec patience.
\TextTitle{L'Esprit intercède pour les saints\FTNTT{Hé. 7:25}}
\VS{26}De même aussi l’Esprit nous aide dans notre faiblesse, car nous ne savons pas ce qu’il nous convient de demander dans nos prières. Mais l’Esprit lui-même intercède par des soupirs inexprimables.
\VS{27}Et celui qui sonde les cœurs connaît quelle est la pensée de l'Esprit, car il intercède en faveur des saints, selon Dieu.
\TextTitle{Le plan de Dieu s'accomplit par l'Evangile}
\VS{28}Or nous savons aussi que toutes choses concourent au bien de ceux qui aiment Dieu, c'est-à-dire de ceux qui sont appelés selon son dessein.
\VS{29}Car ceux qu'il a connus d’avance, il les a aussi prédestinés à être semblables à l'image de son Fils, afin qu'il soit le premier-né de beaucoup de frères.
\VS{30}Et ceux qu'il a prédestinés, il les a aussi appelés ; et ceux qu'il a appelés, il les a aussi justifiés ; et ceux qu'il a justifiés, il les a aussi glorifiés.
\VS{31}Que dirons-nous donc à l’égard de ces choses ? Si Dieu est pour nous, qui sera contre nous ?
\VS{32}Lui qui n'a point épargné son propre Fils, mais qui l'a livré pour nous tous, comment ne nous donnera-t-il point aussi toutes choses avec lui ?
\VS{33}Qui accusera les élus de Dieu ? Dieu est celui qui justifie.
\VS{34}Qui les condamnera ? Christ est mort ; et bien plus, il est ressuscité, il est à la droite de Dieu, et il intercède pour nous.
\TextTitle{L'amour de Christ résiste contre tout}
\VS{35}Qui nous séparera de l'amour de Christ ? Sera-ce l'oppression, ou l'angoisse, ou la persécution, ou la famine, ou la nudité, ou le péril, ou l'épée ?
\VS{36}Ainsi qu'il est écrit : C’est à cause de toi que nous sommes livrés à la mort tous les jours, et qu’on nous regarde comme des brebis destinées à la boucherie\FTNT{Ps. 44:23.}.
\VS{37}Mais dans toutes ces choses nous sommes plus que vainqueurs par celui qui nous a aimés.
\VS{38}Car j’ai l’assurance que ni la mort, ni la vie, ni les anges, ni les principautés, ni les puissances, ni les choses présentes, ni les choses à venir,
\VS{39}ni la hauteur, ni la profondeur, ni aucune autre créature, ne pourra nous séparer de l'amour de Dieu manifesté en Jésus-Christ notre Seigneur.
\Chap{9}
\TextTitle{Le chagrin de Paul pour Israël son peuple}
\VerseOne{}Je dis la vérité en Christ, je ne mens point, ma conscience m’en rend témoignage par le Saint-Esprit :
\VS{2}J’éprouve une grande tristesse et un chagrin continuel dans mon cœur.
\VS{3}Car moi-même je souhaiterais être anathème et séparé de Christ pour mes frères, mes parents selon la chair,
\TextTitle{Les enfants de la chair et ceux de la promesse}
\VS{4}qui sont Israélites, à qui appartiennent l'adoption, la gloire, les alliances, l'ordonnance de la loi, le culte,
\VS{5}les promesses, les patriarches, et de qui est issu selon la chair Christ, qui est Dieu au-dessus de toutes choses, béni éternellement, Amen !
\VS{6}Toutefois il ne peut pas se faire que la parole de Dieu soit anéantie. Car tous ceux qui descendent d’Israël ne sont pas Israël.
\VS{7}Et bien qu’ils soient de la postérité d'Abraham, ils ne sont pas tous ses enfants, car il est dit : C'est en Isaac que tu auras une postérité appelée de ton nom ;
\VS{8}c'est-à-dire que ce ne sont pas ceux qui sont enfants de la chair qui sont enfants de Dieu, mais que ce sont les enfants de la promesse qui sont regardés comme la postérité.
\VS{9}Car voici la parole de la promesse : Je viendrai à cette même époque, et Sara aura un fils\FTNT{Ge. 18:10.}.
\VS{10}Et de plus, il en fut ainsi de Rébecca, qui conçut du seul Isaac notre père ;
\VS{11}car les enfants n’étaient pas encore nés et ils n’avaient fait ni bien ni mal, afin que le dessein arrêté selon l'élection de Dieu subsiste, sans dépendre des œuvres, mais par la volonté de celui qui appelle,
\VS{12}il lui fut dit : L’aîné sera assujetti au plus petit\FTNT{Ge. 25:23.}, selon qu’il est écrit :
\VS{13}J'ai aimé Jacob, et j'ai haï Esaü\FTNT{Mal. 1:2-3.}.
\TextTitle{La volonté souveraine de Dieu}
\VS{14}Que dirons-nous donc : Y a-t-il de l’injustice en Dieu ? A Dieu ne plaise !
\VS{15}Car il dit à Moïse : J'aurai compassion de celui de qui j’aurai compassion et je ferai miséricorde à celui à qui je ferai miséricorde\FTNT{Ex. 33:19.}.
\VS{16}Ainsi donc, cela ne vient pas de celui qui veut, ni de celui qui court, mais de Dieu qui fait miséricorde.
\VS{17}Car l'Ecriture dit à Pharaon : Je t'ai suscité dans le but de démontrer en toi ma puissance, et afin que mon Nom soit publié par toute la terre\FTNT{Ex. 9:16.}.
\VS{18}Ainsi, il fait miséricorde à qui il veut, et il endurcit qui il veut.
\VS{19}Tu me diras : Pourquoi se plaint-il encore ? Car qui est celui qui peut résister à sa volonté ?
\VS{20}Mais plutôt, ô homme, qui es-tu, toi qui contestes contre Dieu ? Le vase d’argile dira-t-il à celui qui l'a formé : Pourquoi m'as-tu ainsi fait ?
\VS{21}Le potier n'a-t-il pas le pouvoir de faire avec la même masse de terre un vase d’honneur et un vase d’un usage vil ?
\VS{22}Et que dire, si Dieu, en voulant montrer sa colère, et faire connaître sa puissance, a supporté avec une grande patience les vases de colère, préparés pour la perdition ?
\VS{23}Et s’il a voulu faire connaître les richesses de sa gloire envers les vases de miséricorde, qu'il a préparés d’avance pour la gloire ?
\VS{24}Ainsi il nous a appelés, non seulement d'entre les juifs, mais aussi d'entre les gentils,
\TextTitle{Les prophéties concernant l'aveuglement d'Israël et la grâce sur les gentils}
\VS{25}selon ce qu'il dit dans Osée : J'appellerai mon peuple celui qui n'était point mon peuple ; et la bien-aimée, celle qui n'était point la bien-aimée ;
\VS{26}et il arrivera qu'au lieu où il leur a été dit : Vous n’êtes pas mon peuple, là ils seront appelés les fils du Dieu vivant\FTNT{Os. 2:1.}.
\VS{27}Aussi Esaïe s'écrie au sujet d'Israël : Quand le nombre des enfants d'Israël serait comme le sable de la mer, un petit reste seulement sera sauvé.
\VS{28}Car le Seigneur exécutera pleinement et promptement sa parole sur la terre ce qu’il a résolu\FTNT{Es. 10:22-23.}.
\VS{29}Et comme Esaïe avait dit auparavant : Si le Seigneur des armées ne nous avait laissé une postérité, nous serions devenus comme Sodome, et nous aurions été semblables à Gomorrhe\FTNT{Es. 1:9.}.
\VS{30}Que dirons-nous donc ? Que les gentils, qui ne cherchaient pas la justice, ont obtenu la justice, la justice qui vient de la foi,
\VS{31}tandis qu’Israël qui cherchait la loi de la justice, n'est pas parvenu à cette loi.
\VS{32}Pourquoi ? Parce qu’Israël l’a cherchée non par la foi, mais comme provenant des œuvres de la loi. Ils se sont heurtés contre la pierre d'achoppement,
\VS{33}selon qu’il est écrit : Voici, je mets en Sion la pierre d'achoppement ; et un rocher de scandale, et quiconque croit en lui ne sera point confus\FTNT{Es. 28:16.}.
\Chap{10}
\TextTitle{La foi, seule condition du salut}
\VerseOne{}Mes frères, le souhait de mon cœur, et la prière que je fais à Dieu pour les Israélites, c'est qu'ils soient sauvés.
\VS{2}Car je leur rends témoignage qu'ils ont du zèle pour Dieu, mais sans connaissance.
\VS{3}Parce que ne connaissant point la justice de Dieu, et cherchant à établir leur propre justice, ils ne se sont point soumis à la justice de Dieu.
\VS{4}Car Christ est la fin de la loi\FTNT{Il est question de la loi cérémonielle relative au culte mosaïque. Avant sa mort, Jésus qui était né sous la loi (Ga. 4:4), demandait aux gens de l’appliquer. Ainsi, il demanda au lépreux qu’il avait guéri de présenter une offrande pour sa purification au temple (Mt. 8:1-4) et à ses disciples d'observer l'enseignement des scribes (Mt. 23:1-2). En effet, il fallait que les lois cérémonielles soient respectées jusqu’à sa résurrection. Une fois que Jésus eut dit «~tout est accompli~» (Jn. 19:30), toutes ces lois n’avaient plus aucune raison d’être (Col. 2:14-17 ; Hé. 7:11-22 ; Hé. 10:1-2).} pour la justification de tous ceux qui croient.
\VS{5}En effet, Moïse décrit ainsi la justice qui vient de la loi : L'homme qui fera ces choses vivra par elles\FTNT{Lé. 18:5.}.
\VS{6}Mais voici comment s'exprime la justice qui vient de la foi : Ne dis pas en ton cœur : Qui montera au ciel ? C’est en faire descendre Christ.
\VS{7}Ou : Qui descendra dans l'abîme ? C’est faire remonter Christ d’entre les morts.
\VS{8}Mais que dit-elle ? La parole est près de toi, dans ta bouche, et dans ton cœur. Or voilà la parole de foi que nous prêchons.
\VS{9}C'est pourquoi, si tu confesses de ta bouche le Seigneur Jésus, et si tu crois dans ton cœur que Dieu l'a ressuscité des morts, tu seras sauvé.
\VS{10}Car c’est en croyant du cœur qu’on parvient à la justice, et c’est en confessant de la bouche qu’on parvient au salut, selon ce que dit l’Ecriture :
\VS{11}Quiconque croit en lui ne sera point confus\FTNT{Es. 49:23.}.
\VS{12}Parce qu'il n'y a point de différence, en effet, entre le Juif et le Grec, puisqu’ils ont un même Seigneur, qui est riche pour tous ceux qui l'invoquent.
\VS{13}Car quiconque invoquera le nom du Seigneur sera sauvé\FTNT{Joë. 2:32.}.
\TextTitle{La proclamation de l'Evangile dans les nations}
\VS{14}Mais comment invoqueront-ils celui en qui ils n'ont point cru ? Et comment croiront-ils en celui dont ils n'ont point entendu parler ? Et comment en entendront-ils parler s'il n'y a personne qui leur prêche ?
\VS{15}Et comment y aura-t-il des prédicateurs, s’ils ne sont pas envoyés ? Selon qu'il est écrit : Qu’ils sont beaux les pieds de ceux qui annoncent la paix, de ceux qui annoncent de bonnes nouvelles\FTNT{Es. 52:7.} !
\VS{16}Mais tous n'ont pas obéi à l'Evangile ; car Esaïe dit : Seigneur, qui a cru à notre prédication\FTNT{Es. 53:1.} ?
\VS{17}Ainsi la foi vient de ce qu’on entend, et ce qu’on entend vient de la parole de Christ.
\VS{18}Mais je dis : Ne l'ont-ils point entendue ? Au contraire, leur voix est allée par toute la terre, et leur parole jusqu’aux extrémités du monde.
\VS{19}Mais je dis : Israël ne l'a-t-il point su ? Moïse le premier dit : J’exciterai votre jalousie par ce qui n'est point une nation, je provoquerai votre colère par une nation sans intelligence\FTNT{De. 32:21.}.
\VS{20}Et Esaïe pousse la hardiesse jusqu’à dire : J'ai été trouvé par ceux qui ne me cherchaient point, et je me suis clairement manifesté à ceux qui ne me demandaient pas\FTNT{Es. 65:1.}.
\VS{21}Mais au sujet d’Israël, il dit : J'ai tout le jour tendu mes mains vers un peuple rebelle et contredisant\FTNT{Es. 65:2.}.
\Chap{11}
\TextTitle{Un reste d'Israël participe à la grâce}
\VerseOne{}Je dis donc : Dieu a-t-il rejeté son peuple ? A Dieu ne plaise ! Car je suis aussi Israélite, de la postérité d'Abraham, de la tribu de Benjamin.
\VS{2}Dieu n'a point rejeté son peuple, qu’il a connu d’avance. Et ne savez-vous pas ce que l'Ecriture dit d'Elie, comment il a fait requête à Dieu contre Israël, disant :
\VS{3}Seigneur, ils ont tué tes prophètes, et ils ont démoli tes autels, et je suis resté moi seul ; et ils cherchent à m'ôter la vie\FTNT{1 R. 19:10.}.
\VS{4}Mais quelle réponse Dieu lui donna-t-il ? Je me suis réservé sept mille hommes, qui n'ont point fléchi le genou devant Baal\FTNT{1 R.19:18.}.
\VS{5}De même aussi dans le temps présent, il y a un reste selon l'élection de la grâce.
\VS{6}Or si c'est par la grâce, ce n'est plus par les œuvres ; autrement la grâce n'est plus la grâce. Mais si c'est par les œuvres, ce n'est plus par une grâce ; autrement l’œuvre n'est plus une œuvre.
\TextTitle{La nation d'Israël est temporairement mise à l'écart mais non rejetée}
\VS{7}Quoi donc ? Ce qu'Israël cherche, il ne l'a point obtenu ; mais les élus l’ont obtenu, tandis que les autres ont été endurcis,
\VS{8}selon qu'il est écrit : Dieu leur a donné un esprit d’assoupissement, des yeux pour ne point voir, et des oreilles pour ne point entendre\FTNT{Es. 29:10.}, jusqu’à ce jour. Et David dit :
\VS{9}Que leur table soit pour eux un filet, un piège, une occasion de chute, et cela pour leur récompense.
\VS{10}Que leurs yeux soient obscurcis pour ne point voir\FTNT{Ps. 69:23-24.} ; et tiens continuellement leur dos courbé !
\VS{11}Mais je dis : Est-ce pour tomber qu’ils ont bronché ? Nullement ! Mais par leur chute, le salut est accordé aux Gentils, afin qu’ils soient excités à la jalousie.
\VS{12}Or si leur chute est la richesse du monde, et leur amoindrissement la richesse des Gentils, combien plus en sera-t-il quand ils se convertiront tous ?
\TextTitle{Avertissement aux gentils}
\VS{13}Car je vous parle à vous, Gentils, en tant qu’apôtre des Gentils, je glorifie mon ministère,
\VS{14}afin, s’il est possible, d’exciter la jalousie de ceux de ma race et d’en sauver quelques-uns.
\VS{15}Car si leur mise à l’écart a été la réconciliation du monde, quelle sera leur réintégration, sinon le passage de la mort à la vie ?
\VS{16}Or si les prémices sont saintes, la masse l'est aussi ; et si la racine est sainte, les branches le sont aussi.
\VS{17}Mais si quelques-unes des branches ont été retranchées, et si toi qui étais un olivier sauvage, tu as été greffé à leur place et rendu participant de la racine et de la graisse de l'olivier,
\VS{18}ne te glorifie pas contre ces branches ; car si tu te glorifies, ce n'est pas toi qui portes la racine, mais c'est la racine qui te porte.
\VS{19}Mais tu diras : Les branches ont été retranchées, afin que moi je sois greffé.
\VS{20}Cela est vrai, elles ont été retranchées à cause de leur incrédulité, et tu es debout par la foi ; ne t'élève donc point par orgueil, mais crains.
\VS{21}Car si Dieu n'a point épargné les branches naturelles, prends garde qu'il ne t'épargne pas non plus.
\VS{22}Considère donc la bonté et la sévérité de Dieu ; la sévérité envers ceux qui sont tombés ; et la bonté envers toi, si tu persévères dans cette bonté : Car autrement tu seras aussi retranché.
\VS{23}Eux de même, s'ils ne persistent pas dans leur incrédulité, ils seront greffés ; car Dieu est puissant pour les greffer de nouveau.
\VS{24}Car si toi tu as été coupé de l'olivier sauvage selon sa nature, et greffé contrairement à ta nature sur l'olivier franc, à plus forte raison eux seront-ils greffés selon leur nature sur leur propre olivier.
\VS{25}Car mes frères, je ne veux pas que vous ignoriez ce mystère, afin que vous ne vous regardiez point comme sages : Une partie d’Israël est tombée dans l’endurcissement, jusqu’à ce que la totalité des gentils soit entrée.
\TextTitle{Yahweh prédit le salut futur d'Israël\FTNTT{Es. 66:8}}
\VS{26}Et ainsi tout Israël sera sauvé, selon qu’il est écrit : Le Libérateur viendra de Sion, et il détournera de Jacob les infidélités ;
\VS{27}et c'est là l'alliance que je ferai avec eux, lorsque j'ôterai leurs péchés\FTNT{Es. 59:20-21.}.
\VS{28}Ils sont certes ennemis par rapport à l'Evangile, à cause de vous ; mais en ce qui concerne l’élection, ils sont aimés à cause de leurs pères.
\VS{29}Car Dieu ne se repent pas de ses dons et de sa vocation.
\VS{30}De même que vous avez autrefois désobéi à Dieu et que par leur désobéissance vous avez maintenant obtenu miséricorde,
\VS{31}de même ils ont maintenant désobéi, afin que par la miséricorde qui vous a été faite, ils obtiennent aussi miséricorde.
\VS{32}Car Dieu les a tous renfermés sous la rébellion afin de faire miséricorde à tous.
\TextTitle{Les voies incompréhensibles de Dieu}
\VS{33}Ô profondeur de la richesse, de la sagesse et de la connaissance de Dieu ! Que ses jugements sont insondables et ses voies incompréhensibles !
\VS{34}Car qui a connu la pensée du Seigneur ? Ou qui a été son conseiller ?
\VS{35}Qui lui a donné le premier, pour qu’il ait à recevoir en retour ?
\VS{36}Car c’est de lui, par lui, et pour lui que sont toutes choses. A lui soit la gloire éternellement. Amen !
\Chap{12}
\TextTitle{Le culte raisonnable}
\VerseOne{}Je vous exhorte donc, mes frères, par les compassions de Dieu, à offrir vos corps comme un sacrifice vivant, saint, agréable à Dieu, ce qui est votre culte raisonnable.
\VS{2}Et ne vous conformez pas au siècle présent, mais soyez transformés\FTNT{Le verbe «~transformer~» est la traduction du terme grec «~metamorphoo~» qui a donné en français «~transfigurer~». C’est le même terme qui a été utilisé en Mt. 17:2 pour parler de la transfiguration du Seigneur. Si Paul recommandait cela à des personnes déjà converties, c’est parce que Dieu les appelait à aller plus loin. La transformation d’une chenille en papillon est un très bel exemple pour illustrer le changement radical qui doit s’opérer en nous. Pour atteindre ce stade, cet insecte passe par plusieurs étapes. La transformation nous permet de croître spirituellement. En effet, tout enfant de Dieu est appelé à devenir mature, à passer du stade de petit enfant à celui de jeune homme, et de celui de jeune homme à celui de père (1 Jn. 2:12-14).} par le renouvellement de votre entendement, afin que vous discerniez quelle est la volonté de Dieu, ce qui est bon, agréable et parfait.
\TextTitle{Exhortation à l'humilité et au service selon les dons de l'Esprit}
\VS{3}Par la grâce qui m’a été donnée, je dis à chacun de vous que nul ne présume d'être plus sage qu'il ne faut, mais d’avoir des sentiments modestes, selon la mesure de foi que Dieu a départie à chacun.
\VS{4}Car comme nous avons plusieurs membres dans un seul corps, et que tous les membres n'ont pas la même fonction,
\VS{5}ainsi, nous qui sommes plusieurs, nous formons un seul corps en Christ, et nous sommes tous membres les uns des autres.
\VS{6}Puisque nous avons des dons différents, selon la grâce qui nous est donnée, que celui qui a le don de prophétie l’exerce en analogie de la foi ;
\VS{7}que celui qui est appelé au ministère, s’attache à son ministère ; que celui qui enseigne s’attache à son enseignement,
\VS{8}et celui qui exhorte, à l’exhortation ; que celui qui donne, le fasse avec simplicité ; que celui qui préside, le fasse avec zèle ; que celui qui exerce la miséricorde, le fasse avec joie.
\TextTitle{Les relations mutuelles entre chrétien}
\VS{9}Que la charité soit sincère. Ayez en horreur le mal, attachez-vous fortement au bien.
\VS{10}Par charité fraternelle, soyez pleins d’affection les uns pour les autres ; par honneur, usez de prévenances réciproques.
\VS{11}Ne soyez point paresseux à vous employer pour autrui. Soyez fervents d'esprit. Servez le Seigneur.
\VS{12}Soyez joyeux dans l'espérance. Soyez patients dans la tribulation. Persévérez dans la prière.
\VS{13}Pourvoyez aux besoins des saints. Exercez l'hospitalité.
\VS{14}Bénissez ceux qui vous persécutent ; bénissez-les, et ne les maudissez point.
\VS{15}Réjouissez-vous avec ceux qui se réjouissent. Pleurez avec ceux qui pleurent.
\VS{16}Ayez les mêmes sentiments les uns envers les autres. N’aspirez pas à ce qui est élevé, mais laissez-vous attirer par ce qui est humble. Ne soyez point sages à votre propre jugement.
\TextTitle{Les relations du chrétiens avec ceux du dehors}
\VS{17}Ne rendez à personne le mal pour le mal. Recherchez les choses honnêtes devant tous les hommes.
\VS{18}S’il est possible, autant que cela dépend de vous, soyez en paix avec tous les hommes.
\VS{19}Ne vous vengez point vous-mêmes, mes bien-aimés, mais laissez agir la colère de Dieu, car il est écrit : A moi appartient la vengeance, à moi la rétribution, dit le Seigneur\FTNT{De. 32:35.}.
\VS{20}Si donc ton ennemi a faim, donne-lui à manger ; s'il a soif, donne-lui à boire, car en faisant cela, tu amasseras des charbons ardents sur sa tête.
\VS{21}Ne te laisse pas vaincre par le mal, mais surmonte le mal par le bien.
\Chap{13}
\TextTitle{Le chrétien et les autorités}
\VerseOne{}Que toute personne soit soumise aux autorités supérieures, car il n'y a point d’autorité qui ne vienne pas de Dieu, et les autorités qui existent ont été instituées de Dieu.
\VS{2}C'est pourquoi celui qui s’oppose à l’autorité résiste à l’ordre de Dieu ; et ceux qui y résistent attireront la condamnation sur eux-mêmes.
\VS{3}Car ce n’est pas pour une bonne action, c’est pour une mauvaise que les magistrats sont à craindre. Veux-tu ne pas craindre l’autorité ? Fais le bien, et tu auras sa louange.
\VS{4}Car le magistrat est un serviteur de Dieu pour ton bien. Mais si tu fais le mal, crains, car ce n’est pas en vain qu’il porte l'épée, étant serviteur de Dieu, ordonné pour faire justice en punissant celui qui fait le mal.
\VS{5}C'est pourquoi il faut être soumis, non seulement à cause de la punition, mais aussi à cause de la conscience.
\VS{6}Car c'est aussi pour cela que vous payez les impôts, parce que les magistrats sont les ministres de Dieu, s'employant à rendre la justice.
\VS{7}Rendez donc à tous ce qui leur est dû : L’impôt à qui vous devez l’impôt, le tribut à qui vous devez le tribut, le péage à qui vous devez le péage, la crainte à qui vous devez la crainte, l’honneur à qui vous devez l'honneur.
\TextTitle{L'amour de son prochain : accomplissement de la loi\FTNTT{cp. Lu. 10:29-37}}
\VS{8}Ne devez rien à personne, si ce n’est de vous aimer les uns les autres ; car celui qui aime les autres a accompli la loi.
\VS{9}En effet, les commandements : Tu ne commettras point d’adultère, tu ne tueras point, tu ne déroberas point, tu ne convoiteras point, et ceux qu’il peut encore y avoir, se résument dans cette parole : Tu aimeras ton prochain comme toi-même\FTNT{Ex. 20:12-17 ; Mt. 22:39.}.
\VS{10}La charité ne fait point de mal au prochain ; la charité est donc l'accomplissement de la loi.
\VS{11}Cela importe d’autant plus que vous savez en quelle saison nous sommes ; parce qu'il est déjà l’heure de nous réveiller du sommeil ; car maintenant le salut est plus près de nous que lorsque nous avons cru.
\VS{12}La nuit est avancée\FTNT{Mt. 25:1-13.} et le jour approche. Rejetons donc les œuvres des ténèbres, et soyons revêtus des armes de lumière.
\VS{13}Marchons honnêtement, comme en plein jour, loin des orgies\FTNT{Orgies : Du grec «~komos~». Ce terme désigne la procession nocturne et rituelle, qui avait lieu après un souper, de gens à moitié ivres, à l'esprit folâtre, qui défilaient à travers les rues avec torches et musique en l'honneur de Bacchus ou quelque autre divinité, et chantaient et jouaient devant les maisons de leurs amis, hommes ou femmes. Ce mot est aussi utilisé pour les fêtes et beuveries de nuit qui se terminaient en orgies.} et de l’ivrognerie, de la luxure et de la débauche, des querelles et des jalousies.
\VS{14}Mais revêtez-vous du Seigneur Jésus-Christ, et n'ayez point soin de la chair pour en satisfaire les convoitises.
\TextTitle{[L'attitude du chrétien face aux opinions différentes]
\\(cp. 1 Co. 8:1-10:33}
\Chap{14}
\VerseOne{}Or quant à celui qui est faible dans la foi, recevez-le, et n'ayez point avec lui des discussions sur les opinions.
\VS{2}L'un croit qu'on peut manger de tout, et l'autre, qui est faible, mange des légumes.
\VS{3}Que celui qui mange de tout ne méprise pas celui qui n'en mange point ; et que celui qui n'en mange point, ne juge point celui qui en mange, car Dieu l'a accueilli.
\VS{4}Qui es-tu, toi qui juges le serviteur d'autrui ? S’il se tient ferme ou s'il tombe, c’est à son maître de le juger ; mais il sera affermi, car Dieu est Puissant pour l'affermir.
\VS{5}Tel fait une distinction entre les jours, tel autre les estime tous égaux. Que chacun ait en son esprit une pleine conviction.
\VS{6}Celui qui distingue entre les jours agit ainsi pour le Seigneur. Celui qui mange, c’est pour le Seigneur qu’il mange, car il rend grâces à Dieu ; celui qui ne mange pas, c’est pour le Seigneur qu’il ne mange pas, et il rend grâces à Dieu.
\VS{7}Car nul de nous ne vit pour lui-même, et nul ne meurt pour lui-même.
\VS{8}Car si nous vivons, nous vivons pour le Seigneur ; et si nous mourons, nous mourons pour le Seigneur. Soit donc que nous vivions, soit que nous mourions, nous sommes au Seigneur.
\VS{9}Car c'est pour cela que Christ est mort, qu'il est ressuscité, et qu'il a repris la vie, afin de dominer sur les morts et sur les vivants.
\VS{10}Mais toi, pourquoi juges-tu ton frère ? Ou toi, pourquoi méprises-tu ton frère ? Puisque nous comparaîtrons tous devant le tribunal de Christ.
\VS{11}Car il est écrit : Je suis vivant, dit le Seigneur, tout genou fléchira devant moi, et toute langue donnera gloire à Dieu\FTNT{Es. 45:23 ; Ph. 2:10-11.}.
\VS{12} Ainsi, chacun de nous rendra compte à Dieu pour lui-même.
\TextTitle{Se garder d'être une occasion de chute}
\VS{13}Ne nous jugeons donc plus les uns les autres ; mais pensez plutôt à ne rien faire qui soit pour votre frère une pierre d’achoppement ou une occasion de chute.
\VS{14}Je sais, et je suis persuadé par le Seigneur Jésus, que rien n'est souillé en soi, et qu’une chose n’est souillée que par celui qui la croit souillée.
\VS{15}Mais si ton frère est attristé au sujet d’un aliment, tu ne marches plus selon la charité ; ne détruis point, par ton aliment, celui pour qui Christ est mort.
\VS{16}Que votre privilège ne soit pas un sujet de calomnie.
\VS{17}Car le Royaume de Dieu ne consiste ni dans le manger ni dans le boire, mais dans la justice, la paix et la joie par le Saint-Esprit.
\VS{18}Celui qui sert Christ de cette manière est agréable à Dieu et approuvé des hommes.
\VS{19}Recherchons donc ce qui contribue à la paix et à l’édification mutuelle.
\VS{20}Ne détruis pas l’œuvre de Dieu pour un aliment. Il est vrai que toutes choses sont pures, mais il est mal à l’homme, quand il mange, de devenir une pierre d’achoppement.
\VS{21}Il est bien de ne pas manger de viande, de ne pas boire de vin, et de s’abstenir de ce qui peut être pour ton frère une occasion de chute, de scandale ou de faiblesse.
\VS{22}As-tu la foi ? Garde-la devant Dieu. Heureux est celui qui ne se condamne pas lui-même dans ce qu'il approuve.
\VS{23}Mais celui qui a des doutes au sujet de ce qu’il mange est condamné, parce qu’il n’agit pas avec foi. Tout ce que l’on ne fait pas avec foi est un péché.
\Chap{15}
\VerseOne{}Nous devons, nous qui sommes forts, supporter les infirmités des faibles, et ne pas nous complaire en nous-mêmes.
\VS{2}Que chacun de nous plaise au prochain pour ce qui est bien, en vue de l’édification.
\VS{3}Car même Jésus-Christ n'a pas cherché ce qui lui plaisait, mais, selon qu’il est écrit : Les outrages de ceux qui t’insultent sont tombés sur moi\FTNT{Ps. 69:10.}.
\TextTitle{Les Juifs et gentils rachetés par un même salut}
\VS{4}Or tout ce qui a été écrit autrefois, a été écrit pour notre instruction, afin que par la patience et la consolation que donnent les Ecritures, nous possédions l’espérance.
\VS{5}Que le Dieu de patience et de consolation vous donne d’avoir les mêmes sentiments les uns envers les autres, selon Jésus-Christ,
\VS{6}afin que tous d'un même cœur et d'une même bouche, vous glorifiiez Dieu, qui est le Père de notre Seigneur Jésus-Christ.
\VS{7}C'est pourquoi, accueillez-vous les uns les autres, comme Christ nous a accueillis, pour la gloire de Dieu.
\VS{8}Je dis donc que Jésus-Christ a été Ministre des circoncis, pour prouver la vérité de Dieu, afin de confirmer les promesses faites aux pères,
\VS{9}afin que les gentils glorifient Dieu pour sa miséricorde, selon ce qui est écrit : C’est pourquoi je te louerai parmi les nations, et je chanterai à la gloire de ton Nom\FTNT{Ps. 18:50.}. Et il est dit encore :
\VS{10}Nations, réjouissez-vous avec son peuple\FTNT{De. 32:43.} !
\VS{11}Et encore : Louez le Seigneur, vous toutes les nations, et célébrez-le, vous tous les peuples\FTNT{Ps. 117:1.}. Esaïe dit aussi :
\VS{12}Il sortira d’Isaï un rejeton, qui se lèvera pour régner sur les nations ; les nations espéreront en lui\FTNT{Es. 11:1 ; Es. 11:10.}.
\VS{13}Que le Dieu de l’espérance vous remplisse de toute joie et de toute paix, dans la foi, afin que vous abondiez en espérance par la puissance du Saint-Esprit.
\TextTitle{Paul envisage d'aller à Jérusalem, à Rome et en Espagne}
\VS{14}Pour moi, mes frères, je suis persuadé que vous êtes pleins de bonté, remplis de toute connaissance, et capables de vous exhorter les uns les autres.
\VS{15}Cependant, mes frères, je vous ai écrit en quelque sorte plus librement, comme pour réveiller vos souvenirs, à cause de la grâce que Dieu m’a faite,
\VS{16}d’être ministre de Jésus Christ parmi les gentils ; je m’acquitte du divin service de l'Evangile de Dieu, afin que les gentils lui soient une offrande agréable, étant sanctifiée par le Saint-Esprit.
\VS{17}J'ai donc sujet de me glorifier en Jésus-Christ pour ce qui regarde les choses de Dieu.
\VS{18}Car je n’oserais parler de quoi que ce soit que Christ n’ait opéré par moi, pour amener les gentils à son obéissance, par la parole et par les œuvres,
\VS{19}par la puissance des prodiges et des miracles, par la puissance de l'Esprit de Dieu. Ainsi, depuis Jérusalem et les pays voisins jusqu’en Illyrie, j’ai abondamment répandu l’Evangile de Christ.
\VS{20}M'attachant ainsi avec affection à annoncer l’Evangile là où Christ n’avait point encore été prêché, afin que je ne bâtisse pas sur le fondement qu'un autre a déjà posé. 
\VS{21}Mais selon qu'il est écrit : Ceux à qui il n'a point été annoncé le verront ; et ceux qui n'en avaient point entendu parler l’entendront\FTNT{Es. 52:15.}.
\VS{22}Et c’est ce qui m'a souvent empêché d’aller vous voir.
\VS{23}Mais maintenant, n’ayant plus rien qui me retienne dans ces contrées, et ayant depuis plusieurs années le désir d’aller vers vous,
\VS{24}j’espère vous voir en passant, quand je me rendrai en Espagne, et y être accompagné par vous, après que j’aurai satisfait en partie mon désir de me trouver chez vous.
\VS{25}Maintenant je vais à Jérusalem pour assister les saints.
\VS{26}Car il a semblé bon à ceux de Macédoine et d’Achaïe de s’imposer une contribution pour les pauvres parmi les saints de Jérusalem.
\VS{27}Ils l’ont bien voulu, et ils le leur devaient, car si les gentils ont eu part à leurs avantages spirituels, ils doivent aussi les assister dans les choses temporelles.
\VS{28}Dès que j'aurai achevé cette affaire, et que je leur aurai remis ce fruit, j'irai en Espagne en passant par vos quartiers.
\VS{29}Et je sais qu’en allant vers vous, j’irai avec une pleine bénédiction de l'Evangile de Christ.
\VS{30}Je vous exhorte, mes frères, par notre Seigneur Jésus-Christ, et par la charité de l'Esprit, à combattre avec moi en adressant des prières à Dieu en ma faveur,
\VS{31}afin que je sois délivré des incrédules de Judée, et que mon ministère\FTNT{Ministère : Du grec «~diakonia~», terme qui désigne le service ou le ministère de ceux qui répondent aux besoins des autres. Ce vocable fait aussi allusion à l’office des diacres.} à Jérusalem soit agréable aux saints,
\VS{32}en sorte que, par la volonté de Dieu, j’arrive chez vous avec joie, et que je me repose avec vous.
\VS{33}Que le Dieu de paix soit avec vous tous. Amen !
\Chap{16}
\TextTitle{Salutations personnelles de Paul}
\VerseOne{}Je vous recommande notre sœur Phœbé, qui est diaconesse de l'église de Cenchrées,
\VS{2}afin que vous la receviez selon le Seigneur, comme il faut recevoir les saints, et que vous l'assistiez dans tout ce dont elle aura besoin ; car elle a exercé l'hospitalité à l'égard de plusieurs, et même à mon égard.
\VS{3}Saluez Priscille et Aquilas, mes compagnons d’œuvre en Jésus-Christ,
\VS{4}qui ont exposé leur cou pour ma vie ; ce n’est pas moi seul qui leur rends grâces, mais aussi toutes les églises des gentils.
\VS{5}Saluez aussi l'église qui est dans leur maison. Saluez Epaïnète, mon bien-aimé, qui a été pour Christ les prémices d'Achaïe.
\VS{6}Saluez Marie, qui a beaucoup travaillé pour nous.
\VS{7}Saluez Andronicus et Junias, mes parents, qui ont été prisonniers avec moi, et qui sont distingués parmi les apôtres, et qui ont même été en Christ avant moi.
\VS{8}Saluez Amplias, mon bien-aimé dans le Seigneur.
\VS{9}Saluez Urbain, notre compagnon d’œuvre en Christ, et Stachys, mon bien-aimé.
\VS{10}Saluez Apellès, qui est éprouvé en Christ. Saluez ceux de chez Aristobule.
\VS{11}Saluez Hérodion, mon parent. Saluez ceux de chez Narcisse qui sont dans le Seigneur.
\VS{12}Saluez Tryphène et Tryphose, qui travaillent pour le Seigneur. Saluez Perside, la bien-aimée qui a beaucoup travaillé pour le Seigneur.
\VS{13}Saluez Rufus, l’élu du Seigneur, et sa mère, qui est aussi la mienne.
\VS{14}Saluez Asyncrite, Phlégon, Hermas, Patrobas, Hermès, et les frères qui sont avec eux.
\VS{15}Saluez Philologue et Julie, Nérée et sa sœur, et Olympe, et tous les saints qui sont avec eux.
\VS{16}Saluez-vous les uns les autres par un saint baiser. Les églises de Christ vous saluent.
\TextTitle{Se garder de ceux qui causent des divisions et des scandales}
\VS{17}Je vous exhorte, mes frères, à prendre garde à ceux qui causent des divisions et des scandales contre la doctrine que vous avez apprise. Eloignez-vous d'eux.
\VS{18}Car de tels hommes ne servent point notre Seigneur Jésus-Christ, mais leur propre ventre, et par des paroles douces et flatteuses, ils séduisent les cœurs des simples.
\VS{19}Pour vous, votre obéissance est connue de tous ; je me réjouis donc à votre sujet, et je désire que vous soyez sages à l’égard du bien, et purs à l’égard du mal.
\VS{20}Le Dieu de paix brisera bientôt Satan sous vos pieds. Que la grâce de notre Seigneur Jésus-Christ soit avec vous. Amen !
\VS{21}Timothée, mon compagnon d’œuvre, vous salue, ainsi que Lucius, et Jason et Sosipater, mes parents.
\VS{22}Je vous salue dans le Seigneur, moi Tertius, qui ai écrit cette lettre.
\VS{23}Gaïus, mon hôte, et celui de toute l'église, vous salue. Eraste, l’économe de la ville, vous salue, et Quartus, notre frère.
\TextTitle{Bénédiction}
\VS{24}Que la grâce de notre Seigneur Jésus-Christ soit avec vous tous. Amen !
\VS{25}or à celui qui est puissant pour vous affermir selon mon Evangile, et selon la prédication de Jésus-Christ, conformément à la révélation du mystère qui a été caché dans les temps passés.
\VS{26}mais manifesté maintenant par les écrits des prophètes, d’après l’ordre du Dieu éternel, et porté à la connaissance de toutes les nations, afin qu’elles obéissent à la foi.
\VS{27}A Dieu, seul sage, soit la gloire éternellement, par Jésus-Christ. Amen !
\PPE{}
\end{multicols}

%\clearpage\ShortTitle{Ep.}\BookTitle{Ephésiens}\BFont
\noindent\hrulefill
{\footnotesize
\textit{
\bigskip
{\centering{}
\\Auteur~: Paul
\\Thème~: L'Eglise, corps de Christ
\\Date de rédaction~: Env. 60 ap. J.-C.\\}
}
\textit{
\\Ephèse figurait parmi les principales villes de l'Empire romain sous le règne de l'empereur Claude Ier (10 av. J.-C. – 54 ap. J.-C.). Bien que Pergame était considérée comme la capitale de l'Asie Mineure, en raison de sa position géographique et grâce à ses affluents, Ephèse possédait le plus grand port de la région, ce qui lui a valu le contrôle du trafic commercial. Richissime et prospère, elle était renommée pour son faste et sa liberté de parole, et constituait donc un endroit privilégié pour les philosophes. C'était une ville où l'activité culturelle tenait une grande place (Jeux olympiques, théâtres, cirques, etc.) et où chacun pouvait y pratiquer la religion de son choix (croyances gréco-romaines, égyptiennes, judaïque etc.).
\\Ephèse, dont le nom signifie «~désirable~», était la gardienne de l'Artémision, temple dédié à la déesse grecque Artémis, la Diane des Ephésiens.
\\L'église d'Ephèse vit le jour lors du second voyage missionnaire de Paul (50-52). Quand il repartit, il laissa à Aquilas et Priscille la charge de la toute jeune assemblée. Paul s'installa à Ephèse lors de son troisième voyage (53-57) et y demeura presque trois ans. Il discourut pendant trois mois dans la synagogue sur le Royaume de Dieu, mais se retrouva confronté à l'endurcissement de certains. C'est alors qu'il se retira pour enseigner dans l'école d'un certain Tyrannus durant deux ans, de sorte que tous ceux qui habitaient l'Asie, Juifs et Grecs, entendirent parler de Jésus-Christ.
\\Plusieurs de ceux qui avaient cru confessèrent leurs péchés et un certain nombre de ceux qui avaient pratiqué la magie allèrent même jusqu'à brûler leurs livres publiquement. C'est ainsi que l'église d'Ephèse croissait en puissance et en force. La prédication de Paul vint troubler le marché fructueux des fabricants d'idoles au point que Démétrius (orfèvre tirant un grand profit de cette industrie) entraina une émeute contre lui. Paul était cependant soutenu par des amis influents~: Les asiarques.
\\Rédigée en prison, cette épître a pour vocation d'enseigner les chrétiens d'Ephèse sur la manière dont il convient de vivre les uns avec les autres au sein de l'Eglise, corps du Christ.\bigskip
}
}
\par\nobreak\noindent\hrulefill
\begin{multicols}{2}
\Chap{1}
\TextTitle{Introduction}
\VerseOne{}Paul, apôtre de Jésus-Christ par la volonté de Dieu, aux saints et fidèles en Jésus-Christ qui sont à Ephèse~:
\VS{2}Que la grâce et la paix vous soient données par Dieu notre Père, et par le Seigneur Jésus-Christ~!
\TextTitle{La position des élus dans le Royaume de Dieu}
\VS{3}Béni soit Dieu, qui est le Père de notre Seigneur Jésus-Christ, qui nous a bénis de toutes bénédictions spirituelles dans les lieux célestes en Christ~!
\VS{4}Selon qu'il nous a élus en lui avant la fondation du monde, afin que nous soyons saints et irrépréhensibles devant lui dans la charité, 
\VS{5}nous ayant prédestinés pour nous adopter pour lui par Jésus-Christ, selon le bon plaisir de sa volonté,
\VS{6}à la louange de la gloire de sa grâce, par laquelle il nous a rendus agréables en son bien-aimé.
\VS{7}En lui nous avons la rédemption par son sang, à savoir la rémission des offenses, selon les richesses de sa grâce,
\VS{8}qu'il a fait abonder sur nous en toute sagesse et intelligence,
\VS{9}nous ayant donné à connaître le mystère de sa volonté, qu'il avait premièrement arrêté en lui-même,
\VS{10}afin que dans l'accomplissement des temps qu'il avait réglés, il réunit tout en Christ, tant ce qui est dans les cieux, que ce qui est sur la terre, en lui-même. 
\VS{11}En qui nous sommes aussi devenus héritiers, ayant été prédestinés, suivant la résolution de celui qui accomplit toutes choses avec efficacité selon le conseil de sa volonté,
\VS{12}afin que nous soyons à la louange de sa gloire, nous qui avons les premiers espéré en Christ.
\VS{13}En qui vous êtes aussi, ayant entendu la parole de la vérité, qui est l'Evangile de votre salut, et auquel ayant cru, vous avez été scellés du Saint-Esprit qui avait été promis,
\VS{14}lequel est le gage de notre héritage jusqu'à la rédemption de ceux qu'il s'est acquis à la louange de sa gloire.
\VS{15}C'est pourquoi, ayant aussi entendu parler de la foi que vous avez en notre Seigneur Jésus, et de la charité que vous avez envers tous les saints,
\VS{16}je ne cesse de rendre grâces pour vous dans mes prières,
\VS{17}afin que le Dieu de notre Seigneur Jésus-Christ, le Père de gloire, vous donne l'Esprit de sagesse et de révélation, dans ce qui regarde sa connaissance.
\VS{18}Qu'il illumine les yeux de votre esprit, afin que vous sachiez quelle est l'espérance de sa vocation, et quelles sont les richesses de la gloire de son héritage qu'il réserve aux saints,
\VS{19}et quelle est l'excellente grandeur de sa puissance envers nous qui croyons selon l'efficacité de la puissance de sa force, 
\VS{20}qu'il a déployée avec efficacité en Christ, quand il l'a ressuscité des morts et qu'il l'a fait asseoir à sa droite dans les lieux célestes,
\VS{21}au-dessus de toute principauté, de toute puissance, de toute dignité et de toute domination, et au-dessus de tout nom qui se nomme, non seulement dans le siècle présent, mais aussi dans celui qui est à venir.
\TextTitle{Le Messie est le Chef suprême de l'Eglise}
\VS{22}Et il a assujetti toutes choses sous ses pieds, et l'a établi sur toutes choses pour être le Chef de l'Eglise,
\VS{23}qui est son corps, et la plénitude de celui qui remplit tout en tous.
\Chap{2}
\TextTitle{Le salut par la grâce}
\VerseOne{}Et vous étiez morts par vos offenses et par vos péchés,
\VS{2}dans lesquels vous marchiez autrefois, suivant le train de ce monde, selon le prince de la puissance de l'air, qui est l'esprit qui agit maintenant avec efficacité dans les fils rebelles à Dieu,
\VS{3}parmi lesquels nous vivions tous autrefois, selon les convoitises de notre chair, accomplissant les désirs de la chair et de nos pensées. Et nous étions par nature des enfants de colère comme les autres.
\VS{4}Mais Dieu, qui est riche en miséricorde, à cause de sa grande charité dont il nous a aimés,
\VS{5}lorsque nous étions morts dans nos offenses, il nous a vivifiés ensemble avec Christ~; c'est par grâce que vous êtes sauvés.
\VS{6}Et il nous a ressuscités ensemble, et nous a fait asseoir ensemble dans les lieux célestes en Jésus-Christ,
\VS{7}afin qu'il montre dans les siècles à venir les immenses richesses de sa grâce par sa bonté envers nous, en Jésus-Christ.
\VS{8}Car vous êtes sauvés par la grâce, par la foi~; et cela ne vient pas de vous, c'est le don de Dieu~;
\VS{9}non pas par les œuvres, afin que personne ne se glorifie.
\VS{10}Car nous sommes son ouvrage, ayant été créés en Jésus-Christ pour les bonnes œuvres que Dieu a préparées d'avance, afin que nous marchions en elles.
\VS{11}C'est pourquoi, souvenez-vous que vous qui étiez autrefois Gentils dans la chair, et qui étiez appelés incirconcis par ceux qu'on appelle circoncis, et qui le sont dans la chair par la main des hommes, 
\VS{12}vous étiez en ce temps-là sans Christ, privés du droit de cité en Israël, étant étrangers des alliances de la promesse, n'ayant pas d'espérance, et étant sans Dieu dans le monde.
\VS{13}Mais maintenant, par Jésus-Christ, vous qui étiez autrefois éloignés, vous avez été rapprochés par le sang de Christ.
\TextTitle{Juifs et Gentils forment un seul corps}
\VS{14}Car il est notre paix, lui qui des deux n'en a fait qu'un en détruisant le mur de séparation,
\VS{15}ayant aboli dans sa chair l'inimitié, à savoir la loi des commandements qui consiste en ordonnances, afin de créer les deux en lui-même pour être un homme nouveau, en faisant la paix~;
\VS{16}et de réconcilier les uns et les autres avec Dieu pour former un seul corps par sa croix, ayant détruit par elle l'inimitié.
\VS{17}Et il est venu prêcher la paix à vous qui étiez loin, et à ceux qui étaient près,
\VS{18}car nous avons par lui les uns et les autres accès auprès du Père dans un même Esprit.
\TextTitle{L'Eglise véritable}
\VS{19}C'est pourquoi vous n'êtes plus des étrangers ni des gens de dehors, mais concitoyens des saints et gens de la maison de Dieu~;
\VS{20}étant édifiés sur le fondement\FTNT{Le fondement a été posé une fois pour toutes par les apôtres et les prophètes. Et ce fondement est notre Seigneur Jésus-Christ (1 Co. 3:11).} des apôtres et des prophètes, et Jésus-Christ lui-même étant la pierre angulaire~;
\VS{21}en qui tout l'édifice, bien ajusté ensemble, s'élève pour être un temple saint dans le Seigneur,
\VS{22}en qui vous êtes édifiés ensemble, pour être une habitation de Dieu en Esprit.
\Chap{3}
\TextTitle{Le mystère caché de tout temps\FTNTT{Col. 1:24-27.}}
\VerseOne{}C'est pour cela que moi, Paul, je suis prisonnier de Jésus-Christ pour vous Gentils.
\VS{2}Si toutefois vous avez entendu quelle est la gestion de la grâce de Dieu qui m'a été donnée pour vous,
\VS{3}comment par révélation ce mystère m'a été manifesté, ainsi que je l'ai écrit ci-dessus en peu de mots~;
\VS{4}d'où vous pouvez voir en le lisant, quelle est l'intelligence que j'ai du mystère de Christ,
\VS{5}lequel n'a pas été manifesté aux fils des hommes dans les autres générations, comme il a été révélé maintenant par l'Esprit à ses saints apôtres et à ses prophètes,
\VS{6}à savoir que les Gentils sont cohéritiers et d'un même corps, et qu'ils participent ensemble à sa promesse en Christ par l'Evangile,
\VS{7}dont j'ai été fait serviteur, selon le don de la grâce de Dieu qui m'a été donnée selon l'efficacité de sa puissance.
\VS{8}Cette grâce, dis-je, m'a été donnée à moi, qui suis le moindre de tous les saints, pour annoncer parmi les Gentils les richesses incompréhensibles de Christ,
\VS{9}et pour mettre en évidence devant tous quelle est la communion qui nous a été accordée du mystère qui était caché de tout temps en Dieu, lequel a créé toutes choses par Jésus-Christ,
\VS{10}afin que les principautés et les puissances dans les lieux célestes connaissent aujourd'hui par l'Eglise la sagesse infiniment variée de Dieu,
\VS{11}suivant le dessein arrêté dès les siècles, qu'il a établi en Jésus-Christ, notre Seigneur,
\VS{12}par lequel nous avons hardiesse et accès avec confiance, par la foi que nous avons en lui.
\VS{13}C'est pourquoi, je vous prie de ne pas vous relâcher à cause de mes afflictions que je souffre pour l'amour de vous, ce qui est votre gloire.
\VS{14}A cause de cela, je fléchis mes genoux devant le Père de notre Seigneur Jésus-Christ,
\VS{15}duquel toute parenté est nommée dans les cieux et sur la terre,
\VS{16}afin que selon les richesses de sa gloire, il vous donne d'être puissamment fortifiés par son Esprit dans l'homme intérieur,
\VS{17}en sorte que Christ habite dans vos cœurs par la foi~; afin qu'étant enracinés et fondés dans la charité,
\VS{18}vous puissiez comprendre avec tous les saints quelle est la largeur et la longueur, la profondeur et la hauteur,
\VS{19}et connaître la charité de Christ qui surpasse toute connaissance, afin que vous soyez remplis de toute la plénitude de Dieu.
\VS{20}Or à celui qui par la puissance qui agit en nous avec efficacité, peut faire infiniment au-delà de tout ce que nous demandons et pensons,
\VS{21}à lui soit la gloire dans l'Eglise, en Jésus-Christ, dans toutes les générations, aux siècles des siècles~! Amen~!
\Chap{4}
\TextTitle{L'unité}
\VerseOne{}Je vous prie donc, moi, le prisonnier dans le Seigneur, à marcher d'une manière digne de la vocation à laquelle vous êtes appelés,
\VS{2}avec toute humilité et douceur, avec patience, vous supportant les uns les autres dans la charité,
\VS{3}vous efforçant de garder l'unité de l'Esprit par le lien de la paix.
\VS{4}Il y a un seul corps, un seul Esprit, comme aussi vous êtes appelés à une seule espérance par votre vocation~;
\VS{5}il y a un seul Seigneur, une seule foi, un seul baptême,
\VS{6}un seul Dieu et Père de tous, qui est au-dessus de tous, parmi tous, et en vous tous.
\TextTitle{Les dons de Christ pour le perfectionnement et l'édification de son Corps\FTNTT{1 Co. 12:4-11.}}
\VS{7}Mais la grâce est donnée à chacun de nous selon la mesure du don de Christ.
\VS{8}C'est pourquoi il est dit~: Etant monté en haut, il a emmené captive une grande multitude de captifs, et il a donné des dons aux hommes\FTNT{Ps. 68:19.}.
\VS{9}Or que signifie~: Il est monté, sinon qu'il est premièrement descendu dans les parties les plus basses de la terre~?
\VS{10}Celui qui est descendu, c'est le même qui est monté au-dessus de tous les cieux, afin de remplir toutes choses.
\VS{11}Lui-même donc a donné les uns pour être apôtres, les autres pour être prophètes, les autres pour être évangélistes, les autres pour être pasteurs et docteurs,
\VS{12}pour travailler au perfectionnement\FTNT{Le mot «~perfectionnement~» vient du grec «~katartismos~», qui tire son origine du terme «~katartizo~»~: «~redresser, ajuster, compléter, raccommoder (ce qui a été abîmé), réparer~». Ainsi, les divers services ont vocation, d'une part, à réparer les dégâts causés par le péché dans les âmes, et d'autre part, à préparer les disciples à rentrer à leur tour dans leur propre service.} des saints, pour l'œuvre du service\FTNT{Le mot «~service~» vient du grec «~diakonos~», il signifie «~service, ministère de ceux qui répondent aux besoins des autres~». Jésus-Christ lui-même a pris la forme d'un serviteur pour nous servir et non pour être servi (Mt. 20:28).}, pour l'édification du corps de Christ,
\VS{13}jusqu'à ce que nous soyons tous parvenus à l'unité de la foi et de la connaissance du Fils de Dieu, à l'état d'homme parfait, à la mesure de la parfaite stature de Christ,
\VS{14}afin que nous ne soyons plus des enfants flottants et emportés çà et là à tous vents de doctrine, par la tromperie des hommes et par leur ruse à séduire artificieusement.
\VS{15}Mais afin que, suivant la vérité avec la charité, nous croissions en toutes choses en celui qui est le Chef, c'est-à-dire Christ,
\VS{16}dont tout le corps bien ajusté et lié ensemble par toutes les jointures de son assistance, tire son accroissement selon la force qu'il distribue à chaque membre, afin qu'il soit édifié dans la charité.
\TextTitle{Se dépouiller du vieil homme}
\VS{17}Je vous dis donc, et je vous conjure de la part du Seigneur, de ne plus vous conduire comme le reste des Gentils qui suivent la vanité de leurs pensées.
\VS{18}Ils ont l'intelligence obscurcie par les ténèbres et sont étrangers à la vie de Dieu, à cause de l'ignorance qui est en eux, par l'endurcissement de leur cœur.
\VS{19}Ils ont perdu tout sentiment, et se sont abandonnés à la dissolution pour commettre toute sorte d'impureté avec cupidité.
\VS{20}Mais vous n'avez pas ainsi appris Christ,
\VS{21}si toutefois vous l'avez entendu, et si vous avez été enseignés par lui~, selon que la vérité est en Jésus~; 
\VS{22}à savoir que vous dépouilliez le vieil homme, pour ce qui est de votre conduite précédente, qui se corrompt par les convoitises qui séduisent~; 
\VS{23}et que vous soyez renouvelés dans l'esprit de votre entendement,
\VS{24}et que vous soyez revêtus du nouvel homme, créé selon Dieu dans une justice et une sainteté véritables.
\VS{25}C'est pourquoi, ayant dépouillé le mensonge, parlez en vérité chacun avec son prochain~; car nous sommes membres les uns des autres.
\VS{26}Si vous vous mettez en colère, ne péchez pas, que le soleil ne se couche pas sur votre colère.
\VS{27}Ne donnez pas lieu au diable de vous perdre.
\VS{28}Que celui qui dérobait ne dérobe plus~; mais plutôt qu'il travaille en faisant de ses mains ce qui est bon, pour avoir de quoi donner à celui qui est dans le besoin.
\VS{29}Qu'aucun discours malhonnête ne sorte de votre bouche, mais seulement celui qui est propre à édifier, afin qu'il soit agréable à ceux qui l'écoutent.
\VS{30}Et n'attristez pas le Saint-Esprit de Dieu, par lequel vous avez été scellés pour le jour de la rédemption.
\VS{31}Que toute amertume, toute colère, toute irritation, toute clameur, toute médisance, et toute malice soient bannies du milieu de vous.
\VS{32}Mais soyez doux les uns envers les autres, pleins de compassion, et vous pardonnant les uns aux autres, ainsi que Dieu vous a pardonné par Christ.
\Chap{5}
\VerseOne{}Soyez donc les imitateurs de Dieu, comme ses enfants bien-aimés~;
\VS{2}et marchez dans la charité, ainsi que Christ nous a aimés et s'est livré lui-même pour nous comme une offrande et un sacrifice de bonne odeur à Dieu.
\VS{3}Que la fornication, ni aucune impureté, ni la cupidité, ne soient pas même nommées parmi vous, ainsi qu'il est convenable à des saints.
\VS{4}Qu'on n'entende ni parole grossière, ni propos insensés, ni plaisanterie, choses qui sont contraires à la bienséance, mais plutôt des actions de grâces.
\VS{5}Car sachez-le bien qu'aucun fornicateur, ni impur, ni cupide, qui est un idolâtre, n'a d'héritage dans le Royaume de Christ et de Dieu.
\VS{6}Que personne ne vous séduise par de vains discours~; car à cause de ces choses la colère de Dieu vient sur les fils de la rébellion.
\VS{7}Ne soyez donc pas leurs associés.
\VS{8}Car vous étiez autrefois ténèbres, mais maintenant vous êtes lumière dans le Seigneur. Conduisez-vous donc comme des enfants de la lumière~!
\VS{9}car le fruit de l'Esprit consiste en toute bonté, justice et vérité,
\VS{10}éprouvant ce qui est agréable au Seigneur~;
\VS{11}et ne participez pas aux œuvres infructueuses des ténèbres, mais au contraire condamnez-les~!
\VS{12}Car il est honteux de dire les choses qu'ils font en secret~;
\VS{13}mais toutes choses, étant mises en évidence par la lumière, sont rendues manifestes, car la lumière est celle qui manifeste tout.
\VS{14}C'est pourquoi il est dit~: Réveille-toi, toi qui dors, et relève-toi d'entre les morts, et Christ t'éclairera\FTNT{Es. 60:1.}.
\VS{15}Prenez donc garde de vous conduire soigneusement, non pas comme étant dépourvus de sagesse, mais comme étant sages,
\VS{16}rachetant le temps, car les jours sont mauvais.
\VS{17}C'est pourquoi ne soyez pas sans intelligence, mais comprenez bien quelle est la volonté du Seigneur.
\VS{18}Et ne vous enivrez pas du vin dans lequel il y a de la dissolution, mais soyez remplis de l'Esprit.
\VS{19}Entretenez-vous par des psaumes, des hymnes et des cantiques spirituels, chantant et psalmodiant de votre cœur au Seigneur~;
\VS{20}rendez toujours grâces pour toutes choses à Dieu notre Père, au Nom de notre Seigneur Jésus-Christ~;
\VS{21}soumettez-vous les uns aux autres dans la crainte de Christ.
\TextTitle{Le mariage selon Dieu}
\VS{22}Femmes, soyez soumises à vos maris comme au Seigneur~;
\VS{23}car le mari est le chef de la femme, comme Christ est le Chef de l'Eglise, qui est son corps, et dont il est le Sauveur.
\VS{24}Or de même que l'Eglise est soumise à Christ, les femmes aussi doivent l'être à leurs maris en toutes choses.
\VS{25}Et vous maris, aimez vos femmes, comme Christ a aimé l'Eglise, et s'est livré lui-même pour elle,
\VS{26}afin de la sanctifier en la purifiant et en la lavant par l'eau de la parole~;
\VS{27}afin de faire paraître devant lui cette Eglise glorieuse, sans tache, ni ride, ni rien de semblable, mais sainte et irréprochable.
\VS{28}C'est ainsi que les maris doivent aimer leurs femmes comme leurs propres corps. Celui qui aime sa femme s'aime lui-même,
\VS{29}car personne n'a jamais eu en haine sa propre chair, mais il la nourrit et l'entretient, comme le Seigneur entretient l'Eglise,
\VS{30}car nous sommes membres de son corps étant de sa chair et de ses os.
\VS{31}C'est pourquoi l'homme quittera son père et sa mère et s'attachera à sa femme, et les deux deviendront une seule chair.
\VS{32}Ce mystère est grand, or je parle de Christ et de l'Eglise.
\VS{33}Que chacun de vous donc aime sa femme comme lui-même, et que la femme respecte son mari.
\Chap{6}
\TextTitle{La famille selon Dieu}
\VerseOne{}Enfants, obéissez à vos pères et à vos mères, dans ce qui est selon le Seigneur, car cela est juste.
\VS{2}Honore ton père et ta mère, c'est le premier commandement avec une promesse,
\VS{3}afin que tout aille bien pour toi et que tu vives longtemps sur la terre.
\VS{4}Et vous, pères, n'irritez pas vos enfants, mais élevez-les\FTNT{Le verbe «~élever~» vient du grec «~ektrepho~» qui signifie nourrir jusqu'à maturité.} en les instruisant et les avertissant selon le Seigneur.
\TextTitle{Les rapports entre les maîtres et les serviteurs selon Dieu}
\VS{5}Serviteurs, obéissez à vos maîtres selon la chair avec crainte et tremblement, dans la simplicité de votre cœur, comme à Christ,
\VS{6}ne les servant pas seulement sous leurs yeux, comme cherchant à plaire aux hommes, mais comme serviteurs de Christ, faisant de bon cœur la volonté de Dieu,
\VS{7}servant avec bienveillance, comme servant le Seigneur et non pas les hommes~;
\VS{8}sachant que chacun, soit esclave, soit libre, recevra du Seigneur le bien qu'il aura fait.
\VS{9}Et vous maîtres, faites envers eux la même chose et renoncez aux menaces, sachant que leur Seigneur et le vôtre est dans les cieux, et qu'il n'y a pas en lui acception de personnes.
\TextTitle{Le combat spirituel}
\VS{10}Au reste, mes frères, fortifiez-vous dans le Seigneur, et dans la puissance de sa force.
\VS{11}Revêtez-vous de toutes les armes de Dieu, afin de pouvoir résister aux embûches du diable.
\VS{12}Car nous n'avons pas à lutter\FTNT{Le mot grec utilisé ici est «~pale~», il était employé pour parler de la lutte entre deux combattants où chacun essaye de renverser l’autre~; la victoire étant acquise par le maintien de l’adversaire au sol, et en lui mettant la main sur la nuque.} contre la chair et le sang, mais contre les principautés, contre les puissances, contre les seigneurs du monde des ténèbres de ce siècle, contre les méchancetés spirituelles qui sont dans les lieux célestes.
\VS{13}C'est pourquoi prenez toutes les armes de Dieu\FTNT{Voir en annexe «~Les armes du chrétien~».}, afin de pouvoir résister dans le mauvais jour, et tenir ferme après avoir tout surmonté.
\VS{14}Soyez donc fermes, ayant à vos reins la vérité pour ceinture, ayant revêtu la cuirasse de la justice~;
\VS{15}et ayant vos pieds chaussés, prêts pour l'Evangile de paix~;
\VS{16}par-dessus tout, prenez le bouclier de la foi, avec lequel vous pourrez éteindre tous les dards enflammés du malin~;
\VS{17}prenez aussi le casque du salut, et l'épée de l'Esprit, qui est la parole de Dieu~;
\VS{18}priant en votre esprit par toutes sortes de prières et de supplications en tout temps, veillant à cela avec une entière persévérance, et priant pour tous les saints,
\VS{19}et pour moi aussi, afin qu'il me soit donné de parler en toute liberté et avec hardiesse, pour faire connaître le mystère de l'Evangile,
\VS{20}pour lequel je suis ambassadeur quoique chargé de chaînes, afin, dis-je, que je parle librement, ainsi qu'il faut que je parle.
\TextTitle{Salutations}
\VS{21}Or afin que vous aussi vous sachiez ce qui me concerne et ce que je fais, Tychique, notre frère bien-aimé et fidèle serviteur du Seigneur, vous fera tout savoir.
\VS{22}Je l'envoie exprès vers vous, afin que vous connaissiez notre situation, et pour qu'il console vos cœurs.
\VS{23}Que la paix soit avec les frères, et la charité avec la foi, de la part de Dieu le Père, et du Seigneur Jésus-Christ~!
\VS{24}Que la grâce soit avec tous ceux qui aiment notre Seigneur Jésus-Christ dans l'incorruptibilité~! Amen~!
\PPE{}
\end{multicols}

%\clearpage\ShortTitle{Ph.}\BookTitle{Philippiens}\BFont
\noindent\hrulefill
{\footnotesize
\textit{
\bigskip
{\centering{}
\\Auteur~: Paul
\\Thème~: Expérience chrétienne
\\Date de rédaction~: Env. 60 ap. J.-C.\\}
}
\textit{
\\Fondée par Philippe II (382 av. J.-C. – 336 av. J.-C.) en 356 av. J.-C., Philippes est une ville grecque de Macédoine orientale. Située sur une voie romaine qui traversait les Balkans (Via Egnatia), elle est restée de taille modeste en dépit de son fort taux de fréquentation.
\\La première mention de l'assemblée de Philippes se trouve dans Actes 16, lors de la rencontre de Paul avec des femmes réunies à l'extérieur de la ville pour la prière. Au travers des paroles de Paul, le Seigneur toucha particulièrement Lydie qui, après avoir été baptisée avec sa famille, reçut Paul et ses compagnons dans sa maison.
\\C'est à Rome, sous le règne de Néron (37-68), que Paul, alors captif, rédigea cette lettre. Cet écrit de l'apôtre accusait réception d'un don monétaire que l'église de Philippes lui avait fait parvenir par le biais d'Epaphrodite. Paul y exprimait sa joie en dépit des souffrances et invitait les Philippiens à faire de même. Loin des erreurs doctrinales reprochées à d'autres, ces chrétiens recevaient ainsi l'expression de l'affection de Paul et ses encouragements à persévérer dans la foi en Christ en toutes circonstances.\bigskip
}
}
\par\nobreak\noindent\hrulefill
\begin{multicols}{2}
\Chap{1}
\TextTitle{Introduction}
\VerseOne{}Paul et Timothée, serviteurs de Jésus-Christ, à tous les saints en Jésus-Christ qui sont à Philippes, avec les évêques et les diacres~:
\VS{2}Que la grâce et la paix vous soient données de la part de Dieu, notre Père et du Seigneur Jésus-Christ~!
\VS{3}Je rends grâces à mon Dieu toutes les fois que je fais mention de vous,
\VS{4}en priant toujours pour vous tous avec joie dans toutes mes prières,
\VS{5}à cause de votre attachement à l'Evangile, depuis le premier jour jusqu'à maintenant.
\VS{6}Etant persuadé de cela même, que celui qui a commencé cette bonne œuvre en vous, l'achèvera jusqu'au jour de Jésus-Christ.
\VS{7}Comme il est juste que je pense ainsi de vous tous, parce que je retiens dans mon cœur, que vous avez tous été participants de la grâce avec moi dans mes liens, et dans la défense et la confirmation de l'Evangile.
\VS{8}Car Dieu m'est témoin que je vous aime tous tendrement, conformément à la charité de Jésus-Christ.
\VS{9}Et je lui demande cette grâce~: Que votre charité abonde encore de plus en plus avec connaissance et toute intelligence,
\VS{10}pour le discernement des choses contraires, afin que vous soyez purs et irréprochables pour le jour de Christ,
\VS{11}étant remplis de fruits de justice, qui sont par Jésus-Christ, à la gloire et à la louange de Dieu.
\TextTitle{Les chrétiens encouragés par la souffrance de Paul}
\VS{12}Or, mes frères, je veux bien que vous sachiez que les choses qui me sont arrivées, sont arrivées pour un plus grand avancement de l'Evangile.
\VS{13}De sorte que mes liens en Christ ont été rendus célèbres dans tout le Prétoire, et partout ailleurs. 
\VS{14}Et que plusieurs de nos frères en notre Seigneur, étant rassurés par mes liens, osent annoncer la parole plus hardiment, et sans crainte. 
\VS{15}Il est vrai que quelques-uns prêchent Christ par envie et par un esprit de dispute~; et que les autres le font, au contraire, par une bonne volonté. 
\VS{16}Les uns, dis-je, annoncent Christ par un esprit de dispute, et non pas purement, croyant ajouter de l'affliction à mes liens.
\VS{17}Mais les autres le font par charité, sachant que je suis établi pour la défense de l'Evangile\FTNT{Les versets 16 et 17 sont inversés dans les versions Segond, Darby et TOB notamment. Ces bibles sont basées sur les textes minoritaires, moins précis. Les versions Martin, Ostervald et King James, basées sur le texte majoritaire (byzantin), utilisent bien cet ordre des versets que l'on retrouvent ainsi dans les écrits grecs.}.
\VS{18}Quoi donc~? Toutefois, de toute manière, que ce soit par ostentation, ou par amour de la vérité, Christ n'est pas moins annoncé. Je m'en réjouis, et je m'en réjouirai encore.
\VS{19}Car je sais que cela tournera à mon salut par vos prières et par le secours de l'Esprit de Jésus-Christ,
\VS{20}selon ma ferme attente et mon espérance, je ne serai confus en rien, mais qu'en toute assurance, Christ sera maintenant, comme il l'a toujours été glorifié dans mon corps, soit par ma vie, soit par ma mort.
\VS{21}Car Christ est ma vie, et la mort m'est un gain.
\VS{22}Mais s'il est utile pour mon œuvre de vivre dans la chair, ce que je dois choisir, je n'en sais rien.
\VS{23}Car je suis pressé des deux côtés~: Mon désir tendant bien à déloger, et à être avec Christ, ce qui me serait beaucoup meilleur. 
\VS{24}Mais il est plus nécessaire pour vous que je demeure dans la chair.
\VS{25}Et je suis persuadé, je sais que je demeurerai et que je resterai avec vous tous, pour votre avancement et pour votre joie dans la foi.
\VS{26}Afin que vous ayez en moi un sujet de vous glorifier de plus en plus en Jésus-Christ, par mon retour au milieu de vous. 
\VS{27}Seulement, conduisez-vous dignement comme il est séant selon l'Evangile de Christ~; afin que, soit que je vienne, et que je vous voie~; soit que je sois absent, j'entende quant à votre état, que vous persistez dans un même esprit, combattant ensemble d'un même courage par la foi de l'Evangile, et n'étant en rien épouvantés par les adversaires.
\VS{28}Ce qui est pour eux une preuve de perdition, mais pour vous de salut~; et cela de la part de Dieu.
\TextTitle{Souffrir pour Christ: Une grâce}
\VS{29}Parce qu'il vous a été gratuitement donné dans ce qui a du rapport à Christ, non seulement de croire en lui, mais aussi de souffrir pour lui,
\VS{30}en soutenant le même combat que vous m'avez vu soutenir, et que vous apprenez maintenant que je soutiens encore.
\Chap{2}
\TextTitle{Exhortation à l'unité}
\VerseOne{}Si donc il y a quelque consolation en Christ, s'il y a quelque soulagement dans la charité, s'il y a quelque communion d'esprit, s'il y a quelques cordiales affections et quelques compassions,
\VS{2}rendez ma joie parfaite, ayant un même sentiment, un même amour, une même âme, et consentant tous à une même chose.
\VS{3}Ne faites rien par esprit de parti\FTNT{Parti~: «~eritheia~» en grec. Avant la Nouvelle Alliance, ce mot ne se trouve que dans les écrits d'Aristote (philosophe grec, disciple de Platon, né en 384 et mort en 322 av. J.-C) où il dénote une «~recherche personnelle, la poursuite d'une fonction politique par des moyens injustes~». Ce mot signifie aussi «~faire une campagne électorale~» ou «~intriguer pour une fonction dans un esprit partisan, querelleur~». On retrouve le mot grec dans Ja. 3:14,16~; Ga. 5:20~; Ro. 2:8~; Ph. 1:16}, ou par vaine gloire~; mais que l'humilité de cœur vous fasse regarder les autres comme étant au-dessus de vous-mêmes.
\VS{4}Ne regardez point chacun, à votre intérêt particulier, mais que chacun ait égard aussi à ce qui concerne les autres.
\TextTitle{L'humilité de Christ}
\VS{5}Qu'il y ait donc en vous un même sentiment qui a été en Jésus-Christ. 
\VS{6}Lequel étant en forme de Dieu, n'a point regardé son égalité avec Dieu comme une usurpation.
\VS{7}Cependant il s'est vidé lui-même, ayant pris la forme de serviteur, fait à la ressemblance des hommes.
\VS{8}Et, étant trouvé en apparence comme un homme, il s'est abaissé lui-même, en se rendant obéissant jusqu'à la mort, même jusqu'à la mort de la croix. 
\VS{9}C'est pourquoi aussi Dieu l'a souverainement élevé, et lui a donné le Nom qui est au-dessus de tout nom~;
\VS{10}afin qu'au Nom de Jésus, tout genou fléchisse, tant de ceux qui sont dans les cieux, que de ceux qui sont sur la terre, et sous la terre,
\VS{11}et que toute langue confesse que Jésus-Christ est le Seigneur, à la gloire de Dieu le Père.
\VS{12}C'est pourquoi, mes bien-aimés, comme vous avez toujours obéi, mettez en œuvre votre propre salut avec crainte et tremblement, non seulement comme en ma présence, mais beaucoup plus maintenant que je suis absent.
\VS{13}Car c'est Dieu qui produit en vous avec efficacité le vouloir et le faire, selon son bon plaisir.
\VS{14}Faites toutes choses sans murmures et sans disputes,
\VS{15}afin que vous soyez sans reproche, et purs, des enfants de Dieu, irrépréhensibles au milieu de la génération corrompue et perverse, parmi lesquels vous brillez comme des flambeaux dans le monde, qui portent au devant d'eux la parole de la vie. 
\VS{16}Pour me glorifier au jour de Christ de n'avoir point couru en vain, ni travaillé en vain. 
\VS{17}Et même si je sers de libation sur le sacrifice et sur le service de votre foi, je m'en réjouis, et je m'en réjouis avec vous tous.
\VS{18}Vous aussi pareillement, réjouissez-vous, et réjouissez-vous avec moi.
\TextTitle{Paul témoigne de Timothée et d'Epaphrodite}
\VS{19}Or j'espère avec la grâce du Seigneur Jésus, vous envoyer bientôt Timothée afin que j'aie aussi plus de courage quand j'aurai connu votre état. 
\VS{20}Car je n'ai personne d'un pareil courage, et qui soit vraiment soigneux de ce qui vous concerne,
\VS{21}parce que tous cherchent leur intérêt particulier, et non les intérêts de Jésus-Christ. 
\VS{22}Mais vous savez l'épreuve que j'ai faite de lui, puisqu'il a servi avec moi en l'Evangile, comme l'enfant sert son père.
\VS{23}J'espère donc vous l'envoyer dès que j'aurai pourvu à mes affaires.
\VS{24}Et j'ai cette confiance en notre Seigneur que moi-même aussi j'irai bientôt.
\VS{25}Mais j'ai cru nécessaire de vous envoyer Epaphrodite, mon frère, mon compagnon d'œuvre et mon compagnon d'armes, par qui vous m'aviez envoyé de quoi pourvoir à mes besoins.
\VS{26}Car aussi il désirait ardemment vous voir tous, et il était fort affligé de ce que vous aviez appris qu'il avait été malade.
\VS{27}En effet, il a été malade et tout près de la mort~; mais Dieu a eu pitié de lui, et non seulement de lui, mais aussi de moi, afin que je n'aie pas tristesse sur tristesse.
\VS{28}Je l'ai donc envoyé à cause de cela avec plus de soin, afin qu'en le revoyant vous ayez de la joie, et que j'aie moins de tristesse.
\VS{29}Recevez-le donc en notre Seigneur, avec toute sorte de joie~; et ayez de l'estime pour ceux qui sont tels que lui. 
\VS{30}Car il a été proche de la mort pour l'œuvre de Christ, n'ayant eu aucun égard à sa propre vie, afin de suppléer au défaut de votre service envers moi.
\Chap{3}
\TextTitle{Le légalisme et la justice de la loi mosaïque}
\VerseOne{}Au reste, mes frères, réjouissez-vous dans le Seigneur. Je ne me lasse point de vous écrire les mêmes choses, mais pour vous c'est une sécurité.
\VS{2}Prenez garde aux chiens~; prenez garde aux mauvais ouvriers~; prenez garde aux faux circoncis.
\VS{3}Car c'est nous qui sommes les circoncis, qui rendons à Dieu notre culte en Esprit, et qui nous glorifions en Jésus-Christ, et qui n'avons point de confiance en la chair.
\VS{4}Moi aussi, cependant, j'aurais sujet de mettre ma confiance en la chair. Si quelqu'un estime qu'il a de quoi se confier en la chair, je le puis bien davantage~:
\VS{5}Moi, circoncis le huitième jour, de la race d'Israël, de la tribu de Benjamin, Hébreu né d'Hébreux, pharisien en ce qui concerne la loi~;
\VS{6}quant au zèle, persécutant l'Eglise~; et quant à la justice à l'égard de la loi, étant sans reproche.
\TextTitle{Le Messie, objet de notre foi} 
\VS{7}Mais ces choses qui étaient pour moi un gain, je les ai regardées comme une perte à cause de l'amour de Christ.
\VS{8}Et certes, je regarde toutes les autres choses comme m'étant nuisibles en comparaison de l'excellence de la connaissance de Jésus-Christ, mon Seigneur, pour l'amour duquel je me suis privé de toutes ces choses, et je les estime comme du fumier, afin de gagner Christ,
\VS{9}et que je sois trouvé en lui, ayant non pas ma justice qui est de la loi, mais celle qui est par la foi en Christ, c'est-à-dire, la justice qui est de Dieu par la foi.
\VS{10}Ainsi, je connaîtrai Jésus-Christ et la puissance de sa résurrection, et la communion de ses souffrances, en devenant conforme à lui dans sa mort, pour parvenir,
\VS{11}si je puis, à la résurrection d'entre les morts.
\VS{12}Non que j'aie déjà atteint le but, ou que je sois déjà rendu parfait, mais je poursuis ce but pour tâcher d'y parvenir~; c'est pourquoi aussi j'ai été pris par Jésus-Christ.
\VS{13}Mes frères, pour moi, je ne me persuade pas d'avoir atteint le but~;
\VS{14}mais je fais une chose~: Oubliant les choses qui sont en arrière, et me portant vers celles qui sont en avant, je cours vers le but, pour remporter le prix de la vocation céleste de Dieu en Jésus-Christ.
\TextTitle{Paul exhorte les croyants à l'unité}
\VS{15}C'est pourquoi, nous tous qui sommes parfaits, ayons ce même sentiment~; et si vous êtes en quelque point d'un autre avis, Dieu vous le révélera aussi.
\VS{16}Cependant, marchons suivant une même règle pour les choses auxquelles nous sommes parvenus, et ayons un même sentiment.
\VS{17}Soyez tous ensemble mes imitateurs, mes frères, et portez les regards sur ceux qui marchent selon le modèle que vous avez en nous.
\VS{18}Car il y en a plusieurs qui marchent d'une telle manière, que je vous ai souvent dit, et maintenant je vous le dis encore en pleurant, qu'ils sont ennemis de la croix de Christ. 
\VS{19}Eux dont la fin est la perdition, qui ont pour dieu leur ventre, et dont la gloire est dans leur confusion, n'ayant d'affection que pour les choses de la terre.
\TextTitle{Le Messie: Notre espérance}
\VS{20}Mais pour nous, notre cité est dans les cieux, d'où nous aussi nous attendons le Sauveur, le Seigneur Jésus-Christ,
\VS{21}qui transformera notre corps vil, afin qu'il soit rendu conforme à son corps glorieux, selon cette efficacité\FTNT{Ep. 3:7} par laquelle il peut même s'assujettir toutes choses. 
\Chap{4}
\TextTitle{Avoir le même sentiment}
\VerseOne{}C'est pourquoi, mes très chers frères bien-aimés, vous qui êtes ma joie et ma couronne, demeurez ainsi fermes dans le Seigneur, mes bien-aimés.
\VS{2}J'exhorte Evodie, et j'exhorte aussi Syntyche, à être d'un même sentiment dans le Seigneur.
\VS{3}Et toi aussi, mon vrai compagnon\FTNT{Le mot compagnon est la traduction du grec «~suzugos~» qui signifie littéralement: Ensemble sous le joug, compagnon de peine, de joug. Cette expression renvoie à 2 Co. 6:14.}, oui je te prie de les aider, elles qui ont combattu avec moi pour l'Evangile, avec Clément, et mes autres compagnons d'œuvre, dont les noms sont écrits dans le livre de vie.
\VS{4}Réjouissez-vous toujours dans le Seigneur~; je vous le répète, réjouissez-vous~!
\VS{5}Que votre douceur soit connue de tous les hommes. Le Seigneur est proche.
\VS{6}Ne vous inquiétez de rien, mais en toutes choses présentez vos demandes à Dieu par des prières et des supplications, avec des actions de grâces.
\VS{7}Et la paix de Dieu, qui surpasse toute intelligence, gardera vos cœurs et vos sentiments en Jésus-Christ.
\TextTitle{L'objet de nos pensées}
\VS{8} Au reste, mes frères, que toutes les choses qui sont véritables, toutes les choses qui sont vénérables, toutes les choses qui sont justes, toutes les choses qui sont pures, toutes les choses qui sont aimables, toutes les choses qui sont de bonne renommée, toutes celles où il y a quelque vertu et quelque louange~; pensez à ces choses.
\VS{9}Car vous les avez aussi apprises, reçues, entendues et vues en moi. Faites ces choses, et le Dieu de paix sera avec vous. 
\TextTitle{Dieu soutient ses serviteurs}
\VS{10}Or je me suis fort réjoui en notre Seigneur, de ce qu'à la fin vous avez fait revivre le soin que vous aviez pour moi~; à quoi aussi vous pensiez, mais vous n'en aviez pas l'occasion. 
\VS{11}Je ne dis pas ceci à cause de mes besoins, car, moi, j'ai appris à être content en moi-même dans les circonstances où je me trouve.
\VS{12}Je sais être abaissé, je sais aussi être dans l'abondance~; partout et en toutes choses je suis instruit tant à être rassasié, qu'à avoir faim~; tant à être dans l'abondance, que dans la disette.
\VS{13}Je puis toutes choses en Christ qui me fortifie.
\VS{14}Néanmoins, vous avez bien fait de prendre part à mon affliction.
\VS{15}Vous savez aussi, vous Philippiens, qu'au commencement de la prédication de l'Evangile, quand je partis de Macédoine, aucune église ne me communiqua rien en matière de donner et de recevoir, excepté vous seuls. 
\VS{16}Et même lorsque j'étais à Thessalonique, vous m'avez envoyé une fois, et même deux fois, ce dont j'avais besoin. 
\VS{17}Ce n'est pas que je recherche des présents, mais je cherche le fruit qui abonde pour votre compte.
\VS{18}J'ai tout reçu, et je suis dans l'abondance, et j'ai été comblé de biens en recevant d'Epaphrodite ce qui vient de vous, comme un parfum de bonne odeur, comme un sacrifice que Dieu accepte et qui lui est agréable.
\VS{19}Aussi mon Dieu pourvoira à tout ce dont vous aurez besoin selon ses richesses, avec gloire en Jésus-Christ. 
\TextTitle{Salutations}
\VS{20}Or à notre Dieu et Père soit la gloire aux siècles des siècles~! Amen~!
\VS{21}Saluez tous les saints en Jésus-Christ. Les frères qui sont avec moi vous saluent.
\VS{22}Tous les saints vous saluent, et principalement ceux qui sont de la maison de César.
\VS{23}Que la grâce de notre Seigneur Jésus-Christ soit avec vous tous~! Amen~!
\PPE{}
\end{multicols}

%\clearpage\ShortTitle{Col.}\BookTitle{Colossiens}\BFont
\noindent\hrulefill
{\footnotesize
\textit{
\bigskip
{\centering{}
\\Auteur~: Paul
\\Thème~: La prééminence de Christ
\\Date de rédaction~: Env. 60 ap. J.-C.\\}
}
\textit{
\\Située en Asie Mineure, Colosses était une ville de Phrygie qui se trouvait à environ deux cents kilomètres d'Ephèse.
\\Rédigée lors de la première captivité romaine de Paul, la lettre aux Colossiens a pour but de rétablir la suprématie de Christ. En effet, cette église - dont Epaphras, le probable fondateur, s'était converti à Ephèse au cours des trois années que Paul y passa - était sous l'influence d'enseignements séducteurs basés sur le gnosticisme. Cette philosophie à la fois attrayante et très dangereuse prônait entre autres le salut par la connaissance et le dualisme.\bigskip
}
}
\par\nobreak\noindent\hrulefill
\begin{multicols}{2}
\Chap{1}
\TextTitle{Introduction}
\VerseOne{}Paul, apôtre de Jésus-Christ, par la volonté de Dieu, et le frère Timothée~:
\VS{2}Aux saints et frères, fidèles en Christ, qui sont à Colosses, que la grâce et la paix vous soient données de la part de Dieu notre Père et de la part du Seigneur Jésus-Christ~!
\VS{3}Nous rendons grâces à Dieu, qui est le Père de notre Seigneur Jésus-Christ, et nous prions toujours pour vous,
\VS{4}ayant entendu parler de votre foi en Jésus-Christ, et de votre charité envers tous les saints,
\VS{5}à cause de l'espérance des biens qui vous sont réservés dans les cieux, et dont vous avez eu précédemment connaissance par la parole de la vérité, c'est-à-dire par l'Evangile,
\VS{6}qui est parvenu jusqu'à vous, comme il l'est aussi dans le monde entier. Et il porte des fruits, comme aussi parmi vous, depuis le jour où vous avez entendu et connu la grâce de Dieu dans la vérité,
\VS{7}ainsi que vous en avez aussi été instruits par Epaphras, notre cher compagnon de service, qui est pour vous un fidèle serviteur de Christ,
\VS{8}et qui nous a fait connaître votre charité par le Saint-Esprit.
\TextTitle{Prière de Paul pour les Colossiens}
\VS{9}C'est pourquoi depuis le jour où nous l'avons appris, nous ne cessons point de prier pour vous, et de demander à Dieu que vous soyez remplis de la connaissance de sa volonté, en toute sagesse et intelligence spirituelle,
\VS{10}afin que vous vous conduisiez d'une manière digne du Seigneur, pour lui plaire en toutes choses, portant des fruits en toutes sortes de bonnes œuvres, et croissant dans la connaissance de Dieu,
\VS{11}étant fortifiés en toute force, selon la puissance de sa gloire, pour toute patience, et constance, avec joie.
\TextTitle{Le salut de Dieu}
\VS{12}Rendant grâces au Père, qui nous a rendus capables d'avoir part à l'héritage des saints dans la lumière,
\VS{13}qui nous a délivrés de la puissance des ténèbres, et nous a transportés dans le Royaume du Fils de son amour,
\VS{14}en qui nous avons la rédemption par son sang, à savoir la rémission des péchés.
\VS{15}Lequel est l'image de Dieu invisible, le premier-né\FTNT{Dans les Ecritures, l'expression «~premier-né~» est appliquée au Seigneur pour exprimer trois réalités. Tout d'abord, on parle de Jésus en tant que premier-né de Marie, c'est-à-dire son fils aîné (Lu. 2:6-7). Ensuite, on trouve cette expression au sens figuré, pour marquer une distinction (par exemple concernant Israël~; Ex. 4:2) ou désigner la particularité et la suprématie d'une personne. Ainsi, bien que David était le dernier-né de son père Isaï (Ps. 89:28), Dieu en fit un premier-né, «~le plus élevé des rois de la terre~» (Ps. 89:28). Il en va de même pour Jésus-Christ. Il n'est pas le premier-né de la création dans le sens de rang de naissance ou de création, autrement Paul aurait employé le terme grec «~prôtoktisis~» qui signifie «~premier-créé~», au lieu de «~prôtotokos~», c'est-à-dire «~premier-né~». Il faut donc voir dans cette expression un titre de supériorité et d'hiérarchie, pour marquer sa prééminence. En effet, la Parole de Dieu déclare clairement que le Seigneur Jésus-Christ est l'Alpha et le Commencement de toutes choses (Ap. 1:8~; Ap. 21:6~; Ap. 22:13), le Créateur suprême (Ge. 1:1~; Ge. 2:7~; Es. 45:11-18~; Ps. 104:30~; Job. 33:4~; Jn. 1:3~; 1 Co. 8:6~; Col. 1:12-16~; Ap. 22:3~; Ap. 14:6). D'ailleurs il l'a lui-même affirmé sans ambigüité~: «~Avant qu'Abraham fût, Je suis~» (Jn. 8:58). Enfin, Jésus-Christ est aussi appelé le premier-né d'entre les morts (Col. 1:18). Cela ne signifie pas qu'il a été le premier à ressusciter, car il y a eu plusieurs résurrections avant la sienne, mais il fut le premier à ressusciter avec un corps glorieux. Sa résurrection est donc le gage de la promesse de la résurrection de tous ceux qui ont foi en lui (Jn. 3:16).} de toute la création.
\VS{16}Car par lui ont été créées toutes les choses qui sont dans les cieux et sur la terre, les visibles et les invisibles, soit les trônes, ou les dominations, ou les principautés, ou les puissances, toutes choses ont été créées par lui, et pour lui.
\VS{17}Et il est avant toutes choses, et toutes choses subsistent par lui.
\VS{18}Et c'est lui qui est le Chef du corps de l'Eglise, et qui est le commencement et le premier-né d'entre les morts, afin qu'il tienne le premier rang en toutes choses,
\VS{19}car le bon plaisir du Père a été que toute plénitude habitât en lui.
\VS{20}Et de réconcilier par lui toutes choses avec lui même, ayant fait la paix par le sang de sa croix, à savoir, tant les choses qui sont dans les cieux que celles qui sont sur la terre.
\VS{21}Et vous, qui étiez autrefois étrangers, et qui étiez ses ennemis dans votre entendement, et dans les mauvaises œuvres, il vous a maintenant réconciliés 
\VS{22}par le corps de sa chair, par sa mort, pour vous présenter saints, et sans tache, et irrépréhensibles devant lui.
\VS{23}Si toutefois vous demeurez dans la foi, étant fondés et fermes, et n'étant point transportés hors de l'espérance de l'Evangile que vous avez entendu, lequel est prêché à toute créature qui est sous le ciel, dont moi Paul, j'ai été fait le serviteur.
\VS{24}Je me réjouis donc maintenant dans mes souffrances pour vous~; et j'accomplis le reste des afflictions de Christ dans ma chair, pour son corps, qui est l'Eglise.
\VS{25}C'est d'elle que j'ai été fait le serviteur, selon la gestion que Dieu m'a donnée auprès de vous, afin que j'exécute pleinement la parole de Dieu,
\VS{26}à savoir le mystère qui avait été caché dans tous les siècles et dans tous les âges, mais qui est maintenant manifesté à ses saints~;
\VS{27}auxquels Dieu a voulu donner à connaître quelles sont les richesses de la gloire de ce mystère parmi les Gentils, c'est à savoir Christ, qui a été prêché parmi vous, et qui est l'espérance de la gloire~; 
\VS{28}lequel nous annonçons, en exhortant tout homme, et en enseignant tout homme en toute sagesse, afin que nous présentions tout homme parfait en Jésus-Christ.
\TextTitle{Le combat de Paul}
\VS{29}A quoi aussi je travaille, en combattant selon son efficacité\FTNTT{le terme «~efficacité~» vient du grec «~energeia~» qui signifie «~action~», «~fonctionnement~», «~compétence~», ou encore «~force à l'œuvre dans~». Ce mot est utilisé seulement pour parler du pouvoir surhumain que ce soit celui de Dieu ou celui du diable. (Ep. 1:19~; Ph. 3:21~; 2 Ti. 2:9).}, qui agit puissamment en moi.
\Chap{2}
\VerseOne{}Or je veux, en effet, que vous sachiez combien est grand le combat que j'ai pour vous, et pour ceux qui sont à Laodicée, et pour tous ceux qui n'ont pas vu mon visage dans la chair,
\VS{2}afin que leurs cœurs soient consolés, étant unis ensemble dans la charité, et enrichis d'une pleine intelligence, pour la connaissance du mystère de notre Dieu et Père, et de Christ,
\VS{3}en qui sont cachés tous les trésors de la sagesse et de la connaissance.
\TextTitle{Mise en garde contre les discours séduisants et la philosophie\FTNTT{1 Co. 2:4~; Ro. 16:17-18~; 2 Pi. 2:3.}}
\VS{4}Or je dis ceci afin que personne ne vous trompe par des discours séduisants.
\VS{5}Car, quoique je sois absent de corps, toutefois je suis avec vous en esprit, me réjouissant, et voyant votre ordre et la fermeté de votre foi, que vous avez en Christ.
\VS{6}Ainsi, comme vous avez reçu le Seigneur Jésus-Christ, marchez en lui,
\VS{7}étant enracinés et édifiés en lui, et fortifiés en la foi, selon que vous avez été enseignés, abondant en elle avec action de grâces.
\VS{8}Prenez garde que personne ne fasse de vous sa proie par la philosophie, et par de vaines tromperies conformes à la tradition des hommes et aux rudiments du monde, et non point à la doctrine de Christ.
\TextTitle{La divinité du Christ}
\VS{9}Car en lui habite corporellement toute la plénitude de la divinité\FTNT{En Jésus-Christ habite toute la plénitude de la divinité. Il est le Dieu Tout-Puissant.}.
\VS{10}Et vous êtes rendus accomplis en lui, qui est le Chef de toute principauté et puissance.
\TextTitle{L'oeuvre de la croix}
\VS{11}En qui aussi vous êtes circoncis d'une circoncision faite sans main, qui consiste à dépouiller le corps des péchés de la chair, ce qui est la circoncision de Christ.
\VS{12}Etant ensevelis avec lui par le baptême, en qui aussi vous êtes ensemble ressuscités par la foi de l'efficacité de Dieu, qui l'a ressuscité des morts.
\VS{13}Et lorsque vous étiez morts dans vos offenses, et dans l'incirconcision de votre chair, il vous a vivifiés ensemble avec lui, vous ayant gratuitement pardonné toutes vos offenses.
\VS{14}Il a effacé l'acte qui était contre nous, qui consistait en des ordonnances, et qui nous était contraire, et il l'a entièrement aboli en le clouant à la croix.
\VS{15}Il a dépouillé les principautés et les puissances, et les a exposées publiquement en spectacle, en triomphant d'elles par la croix.
\TextTitle{Mise en garde contre les commandements et les doctrines des hommes}
\VS{16}Que personne donc ne vous juge au sujet du manger ou du boire, ou au sujet d'un jour de fête, ou d'un jour de nouvelle lune, ou de sabbat,
\VS{17}qui sont l'ombre des choses qui devaient venir, mais le corps est en Christ.
\VS{18}Que personne ne vous enlève à son gré le prix de la course, sous l'apparence d'humilité d'esprit et par un culte des anges, s'ingérant dans les choses qu'il n'a pas vues, étant témérairement enflé par ses pensées charnelles,
\VS{19}sans s'attacher au Chef, dont tout le corps étant joint et ajusté ensemble par des jointures et des liens, s'accroît d'un accroissement de Dieu.
\VS{20}Si donc vous êtes morts avec Christ quant aux rudiments du monde, pourquoi vous impose-t-on ces ordonnances, comme si vous viviez dans le monde~?
\VS{21}A savoir: Ne prends pas~! Ne goûte pas~! Ne touche pas~!
\VS{22}Lesquelles sont toutes périssables par l'usage, et établies suivant les commandements et les doctrines des hommes~; 
\VS{23}et qui ont pourtant quelque apparence de sagesse en dévotion volontaire, et en humilité d'esprit, et en ce qu'elles n'épargnent pas le corps, et n'ont aucun égard à la satisfaction de la chair.
\Chap{3}
\TextTitle{Rechercher les choses d'en haut}
\VerseOne{}Si donc vous êtes ressuscités avec Christ, cherchez les choses qui sont en haut, où Christ est assis à la droite de Dieu.
\VS{2}Pensez aux choses d'en haut, et non à celles qui sont sur la terre.
\VS{3}Car vous êtes morts, et votre vie est cachée avec Christ en Dieu.
\VS{4}Quand Christ, qui est votre vie, apparaîtra, vous paraîtrez aussi alors avec lui dans la gloire.
\TextTitle{La mort à soi en pratique}
\VS{5}Mortifiez donc vos membres qui sont sur la terre~: La fornication, l'impureté, les passions, les mauvais désirs, et la cupidité, qui est une idolâtrie.
\VS{6}C'est à cause de ces choses que la colère de Dieu vient sur les fils de la rébellion,
\VS{7}parmi lesquels vous marchiez autrefois, quand vous viviez dans ces choses.
\VS{8}Mais maintenant, vous aussi, rejetez toutes ces choses~: La colère, l'animosité, la médisance, et les paroles déshonnêtes qui pourraient sortir de votre bouche.
\VS{9}Ne mentez point les uns aux autres, vous étant dépouillés du vieil homme et de ses œuvres,
\VS{10}et ayant revêtu le nouvel homme, qui se renouvelle dans la connaissance, selon l'image de celui qui l'a créé.
\VS{11}En qui il n'y a ni Grec ni Juif, ni circoncis ni incirconcis, ni barbare ni Scythe, ni esclave ni libre~; mais Christ y est tout et en tous.
\VS{12}Ainsi donc, comme des élus de Dieu, saints et bien-aimés, revêtez-vous des entrailles de miséricorde, de bonté, d'humilité, de douceur, de patience.
\VS{13}Vous supportant les uns les autres, et vous pardonnant les uns aux autres~; et si l'un a querelle contre l'autre, comme Christ vous a pardonné, vous aussi faites-en de même.
\VS{14}Mais par-dessus toutes ces choses, revêtez-vous de la charité, qui est le lien de la perfection.
\VS{15}Et que la paix de Dieu, à laquelle aussi vous êtes appelés pour être un seul corps, tienne le principal lieu dans vos cœurs. Et soyez reconnaissants.
\VS{16}Que la parole de Christ habite en vous abondamment en toute sagesse~; vous enseignant et vous exhortant l'un l'autre par des psaumes, et des hymnes et des cantiques spirituels, avec grâce, chantant de votre cœur au Seigneur.
\VS{17}Et quoi que vous fassiez, en parole ou en œuvre, faites tout au Nom du Seigneur Jésus, rendant grâces par lui à notre Dieu et Père.
\TextTitle{La famille selon Dieu}
\VS{18}Femmes, soyez soumises à vos maris, comme il convient dans le Seigneur\FTNT{Ep. 5:22.}.
\VS{19}Maris, aimez vos femmes, et ne vous aigrissez pas contre elles\FTNT{Ep. 5:25.}.
\VS{20}Enfants, obéissez à vos pères et à vos mères en toutes choses, car cela est agréable au Seigneur\FTNT{Ep. 6:1-2.}.
\VS{21}Pères, n'irritez pas vos enfants\FTNT{Ep. 6:4.}, afin qu'ils ne se découragent pas.
\TextTitle{Les rapports entre serviteurs et maîtres selon Dieu}
\VS{22}Serviteurs, obéissez en toutes choses à ceux qui sont vos maîtres selon la chair, ne servant point seulement sous leurs yeux, comme voulant complaire aux hommes, mais en simplicité de cœur, craignant Dieu\FTNT{Ep. 6:5-6.}.
\VS{23}Et quoi que vous fassiez, faites tout de bon cœur, comme le faisant pour le Seigneur, et non pas pour les hommes,
\VS{24}sachant que vous recevrez du Seigneur l'héritage pour récompense. Car vous servez Christ, le Seigneur.
\VS{25}Mais celui qui agit injustement recevra ce qu'il aura fait injustement, car en Dieu il n'y a point d'égard à l'apparence des personnes.
\Chap{4}
\VerseOne{}Maîtres, accordez à vos serviteurs ce qui est juste et équitable, sachant que vous avez, vous aussi, un Maître dans les cieux.
\TextTitle{La persévérance dans la prière}
\VS{2}Persévérez dans la prière, veillant dans cet exercice avec des actions de grâces.
\VS{3}Priez aussi tous ensemble pour nous, afin que Dieu nous ouvre la porte de la parole, pour annoncer le mystère de Christ pour lequel aussi je suis prisonnier, 
\VS{4}afin que je le fasse connaître comme je dois en parler.
\VS{5}Conduisez-vous sagement envers ceux du dehors, et rachetez le temps.
\VS{6}Que votre parole soit toujours assaisonnée de sel, avec grâce, afin que vous sachiez comment vous avez à répondre à chacun.
\TextTitle{Salutations}
\VS{7}Tychique, notre frère bien-aimé, et fidèle serviteur, et mon compagnon de service en notre Seigneur, vous fera savoir tout mon état.
\VS{8}Je l'envoie vers vous expressément, afin qu'il connaisse quel est votre état, et qu'il console vos cœurs~;
\VS{9}avec Onésime, notre fidèle et bien-aimé frère, qui est des vôtres. Ils vous feront connaitre toutes les choses d'ici.
\VS{10}Aristarque, qui est prisonnier avec moi, vous salue aussi, et Marc qui est le cousin de Barnabas, au sujet duquel vous avez reçu un ordre, s'il vient à vous, recevez-le.
\VS{11}Et Jésus, appelé Justus, vous salue aussi. Ils sont du nombre des circoncis, et les seuls qui travaillent avec moi pour le Royaume de Dieu, et qui ont été pour moi une consolation.
\VS{12}Epaphras, qui est des vôtres, et serviteur de Jésus-Christ, vous salue~; il ne cesse de combattre pour vous dans ses prières afin que vous demeuriez parfaits et accomplis en toute la volonté de Dieu.
\VS{13}Car je lui rends témoignage qu'il a un grand zèle pour vous, et pour ceux de Laodicée, et pour ceux d'Hiérapolis.
\VS{14}Luc, le médecin bien-aimé, vous salue, et Démas aussi.
\VS{15}Saluez les frères qui sont à Laodicée, et Nymphas, avec l'église qui est dans sa maison.
\VS{16}Et quand cette lettre aura été lue entre vous, faites en sorte qu'elle soit aussi lue dans l'église des Laodicéens, et que vous lisiez aussi celle qui viendra de Laodicée.
\VS{17}Et dites à Archippe~: Prends garde au service que tu as reçu dans le Seigneur afin de bien le remplir.
\VS{18}Je vous salue, moi Paul, de ma propre main. Souvenez-vous de mes liens. Que la grâce soit avec vous~! Amen~!
\PPE{}
\end{multicols}

%\clearpage\ShortTitle{Philémon}\BookTitle{Philémon}\BFont
\noindent\hrulefill
{\footnotesize
\textit{
\bigskip
{\centering{}
\\Auteur : Paul
\\(Gr. : Philemon)
\\Signification : Attentionné, qui embrasse
\\Thème : Un exemple d'amour
\\Date de rédaction : Env. 60 ap. J.-C.\\}
}
%\bigskip
\textit{
\\Paul écrivit cette lettre en prison, lors de sa deuxième captivité à Rome vers l'été 62, en même temps que l'épître aux Colossiens. Il s'adresse à Philémon, chrétien fortuné de Colosses ainsi qu'à sa femme Apphia, son fils Archippe et à l'église qui se réunissait dans leur maison. Paul demande à Philémon de pardonner à Onésime, son esclave, de s'être échappé d'auprès de lui. Il assure à Philémon que désormais une nouvelle relation le lierait à Onésime qui avait accepté Jésus-Christ dans sa vie. Il va même jusqu'à proposer de payer personnellement ce qu'Onésime lui devait tout en exprimant l'espoir que Philémon ferait plus que ce qu'il lui demande. Ainsi, Paul plaide pour Onésime comme Christ le fit en notre faveur.\bigskip
}
}
\par\nobreak\noindent\hrulefill
\begin{multicols}{2}
\Chap{1}
\TextTitle{Introduction}
\VerseOne{}Paul, prisonnier de Jésus-Christ, et le frère Timothée, à Philémon notre bien-aimé et compagnon d'œuvre ;
\VS{2}à Apphia, notre bien-aimée, à Archippe, notre compagnon de combat, et à l'église qui est dans ta maison.
\VS{3}Que la grâce et la paix vous soient données de la part de Dieu notre Père, et de la part du Seigneur Jésus-Christ.
\VS{4}Je rends grâces à mon Dieu, faisant toujours mention de toi dans mes prières ;
\VS{5}apprenant la foi que tu as au Seigneur Jésus, et ta charité envers tous les saints.
\VS{6}Afin que la communication de ta foi devienne efficace, en se faisant connaître par tout le bien qui est en vous, par Jésus-Christ.
\VS{7}Car, mon frère, nous avons une grande joie et une grande consolation de ta charité, en ce que tu as réjoui les entrailles des saints.
\TextTitle{Paul plaide en faveur d'Onésime}
\VS{8}C'est pourquoi, bien que j'aie une grande liberté en Christ de t'ordonner ce qui est convenable,
\VS{9}cependant je te prie plutôt par la charité, bien que je suis ce que je suis, à savoir Paul, un vieillard, et même maintenant prisonnier de Jésus-Christ ;
\VS{10}je te prie donc pour mon fils Onésime, que j'ai engendré dans mes liens ;
\VS{11}qui t'a été autrefois inutile, mais qui maintenant est bien utile à toi et à moi, et que je te renvoie.
\VS{12}Reçois-le donc comme mes propres entrailles.
\VS{13}Je voulais le retenir auprès de moi, afin qu'il me serve à ta place, dans les liens de l'Evangile.
\VS{14}Mais je n'ai rien voulu faire sans ton avis, afin que ce ne soit point comme par contrainte, mais volontairement, que tu me laisses un bien qui est à toi.
\VS{15}Car peut-être n'a-t-il été séparé de toi que pour un temps, afin que tu le recouvres\FTNT{Synonymes : retrouver, reconquérir, regagner.} pour toujours ;
\VS{16}non plus comme un esclave, mais comme étant au-dessus d'un esclave, à savoir comme un frère bien-aimé, principalement de moi ; et combien plus de toi, soit selon la chair, soit selon le Seigneur ?
\VS{17}Si donc tu me tiens pour ton compagnon, reçois-le comme moi-même.
\VS{18}Que s'il t'a fait quelque tort, ou s'il te doit quelque chose, mets-le sur mon compte.
\VS{19}Moi Paul, j'ai écrit ceci de ma propre main, je te le payerai ; pour ne pas te dire que tu te dois toi-même à moi.
\VS{20}Oui, mon frère, que je reçoive ce plaisir de toi en notre Seigneur ; réjouis mes entrailles en notre Seigneur.
\VS{21}Je t'ai écrit m'assurant de ton obéissance, et sachant que tu feras même plus que ce que je te dis.
\TextTitle{Conclusion}
\VS{22}Mais aussi, en même temps, prépare-moi un logement ; car j'espère que je vous serai rendu par vos prières.
\VS{23}Epaphras, qui est prisonnier avec moi en Jésus-Christ, te salue ;
\VS{24}Marc aussi, Aristarque, Démas, et Luc, mes compagnons d'œuvre.
\VS{25}Que la grâce de notre Seigneur Jésus-Christ soit avec votre esprit, Amen !
\PPE{}
\end{multicols}

%\clearpage\ShortTitle{1 Ti.}\BookTitle{1 Timothée}\BFont
\noindent\hrulefill
{\footnotesize
\textit{
\bigskip
{\centering{}
\\Auteur~: Paul
\\(Gr.~: Timotheos)
\\Signification~: Qui adore ou honore Dieu
\\Thème~: Comment se conduire dans l'église
\\Date de rédaction~: Env. 64 ap. J.-C.\\}
}
\textit{
\\Cette lettre s'adresse à Timothée dont le père était grec et la mère juive. Le jeune homme se convertit à Christ avec sa mère et sa grand-mère dès le premier voyage missionnaire de Paul au cours duquel il passa à Lystre.
\\Cette épître fut rédigée après la première captivité de Paul à Rome. Alors que les Eglises connaissaient une certaine expansion, Paul s'adresse à Timothée, jeune et fidèle compagnon d'œuvre qu'il a lui-même formé, sur des questions d'ordre disciplinaire et sur la pureté de la foi. Dans cette épître, dite pastorale, Paul donne des instructions précises à Timothée pour enseigner, exhorter, diriger le culte public et choisir ses collaborateurs.\bigskip
}
}
\par\nobreak\noindent\hrulefill
\begin{multicols}{2}
\Chap{1}
\TextTitle{Introduction}
\VerseOne{}Paul, apôtre de Jésus-Christ par l'ordre de Dieu, notre Sauveur, et du Seigneur Jésus-Christ, notre espérance,
\VS{2}à Timothée mon véritable fils dans la foi~: Que la grâce, la miséricorde et la paix te soient données de la part de Dieu notre Père, et de Jésus-Christ, notre Seigneur.
\TextTitle{Mise en garde contre les erreurs doctrinales~; le but de la loi} 
\VS{3}Suivant la prière que je te fis de demeurer à Ephèse, lorsque j'allais en Macédoine, je te prie encore d'ordonner à certaines personnes de ne pas enseigner une autre doctrine,
\VS{4}et de ne pas s'adonner aux fables et aux généalogies sans fin, qui produisent des disputes plutôt que l'édification en Dieu qui consiste dans la foi.
\VS{5}Or le but du commandement c'est la charité qui procède d'un cœur pur, d'une bonne conscience, et d'une foi sincère.
\VS{6}Quelques-uns, s'étant détournés de ces choses, se sont écartés dans de vains discours,
\VS{7}voulant être docteurs de la loi~; mais ils ne comprennent ni ce qu'ils disent ni ce qu'ils affirment.
\VS{8}Or nous savons que la loi est bonne pour celui qui en fait un usage légitime,
\VS{9}sachant ceci, que ce n'est pas pour le juste que la loi a été établie, mais pour les méchants et les rebelles, pour les impies et les pécheurs, pour les irréligieux et les profanes, pour les parricides, les meurtriers,
\VS{10}pour les fornicateurs, pour les homosexuels, pour les voleurs d'hommes, pour les menteurs, pour les parjures, et contre telle autre chose qui est contraire à la saine doctrine,
\VS{11}selon l'Evangile de la gloire du Dieu béni, Evangile qui m'a été confié.
\TextTitle{Témoignage de Paul}
\VS{12}Je rends grâces à celui qui m'a fortifié, c'est-à-dire à Jésus-Christ, notre Seigneur, de ce qu'il m'a estimé fidèle en m'établissant dans le service,
\VS{13}moi qui auparavant étais un blasphémateur, un persécuteur, et un homme violent~; mais j'ai obtenu miséricorde parce que j'agissais par ignorance, étant dans l'incrédulité.
\VS{14}Or la grâce de notre Seigneur a surabondé en moi, avec la foi et l'amour qui est en Jésus-Christ.
\VS{15}Cette parole est certaine et entièrement digne d'être reçue, que Jésus-Christ est venu dans le monde pour sauver les pécheurs, dont je suis le premier.
\VS{16}Mais j'ai obtenu miséricorde, afin que Jésus-Christ fasse voir en moi le premier, toute sa clémence, pour que je serve d'exemple à ceux qui croiraient en lui pour la vie éternelle.
\VS{17}Or au Roi des siècles, immortel, invisible, à Dieu seul sage, soient honneur et gloire aux siècles des siècles~! Amen~!
\TextTitle{Recommandations à Timothée}
\VS{18}Mon fils Timothée, je te recommande ce commandement que conformément aux prophéties qui auparavant ont été faites sur toi, tu t'acquittes, selon elles, du devoir de combattre dans cette bonne guerre,
\VS{19}en gardant la foi et une bonne conscience, laquelle quelques-uns ayant rejetée, ont fait naufrage quant à la foi.
\VS{20}De ce nombre sont Hyménée et Alexandre, que j'ai livrés à Satan, afin qu'ils apprennent par ce châtiment à ne plus blasphémer.
\Chap{2}
\TextTitle{Instructions sur la prière}
\VerseOne{}J'exhorte donc, avant toutes choses, à faire des requêtes, des prières, des supplications, et des actions de grâces pour tous les hommes,
\VS{2}pour les rois et pour tous ceux qui sont constitués en dignité, afin que nous menions une vie paisible et tranquille, en toute piété et honnêteté.
\VS{3}Car cela est bon et agréable devant Dieu, notre Sauveur,
\VS{4}qui veut que tous les hommes soient sauvés et qu'ils viennent à la connaissance de la vérité.
\VS{5}Car il y a un seul Dieu, et aussi un seul Médiateur entre Dieu et les hommes, à savoir Jésus-Christ homme,
\VS{6}qui s'est donné lui-même en rançon pour tous. C'est le témoignage qui a été rendu en son propre temps.
\VS{7}C'est dans cette vue que j'ai été établi prédicateur, apôtre (je dis la vérité en Christ, je ne mens point) et docteur des Gentils dans la foi et dans la vérité.
\VS{8}Je veux donc que les hommes prient en tout lieu, levant leurs mains pures, sans colère, et sans dispute.
\TextTitle{Le tenue de la femme}
\VS{9}Et de même, que les femmes, vêtues d'une manière décente, avec pudeur et modestie, ne se parent ni de tresses, ni d'or, ni de perles, ni d'habits somptueux,
\VS{10}mais qu'elles se parent de bonnes œuvres, comme il convient à des femmes qui font profession de servir Dieu.
\TextTitle{Le comportement de la femme envers son mari}
\VS{11}Que la femme apprenne dans le silence, en toute soumission.
\VS{12}Car je ne permets pas à la femme d'enseigner ni d'user d'autorité\FTNT{Le mot «~autorité~» vient du grec «~authenteo~» et signifie «~celui qui tue de ses propres mains un autre ou lui-même~; celui qui agit de sa propre autorité, autocrate~; un maître absolu~; gouverneur, exercer une domination~».} sur le mari~; mais elle doit demeurer dans le silence\FTNT{Le mot grec traduit par «~silence~» est «~hesuchia~» qui signifie «~en silence~; paisiblement~». La racine de ce terme est «~hesuchios~»~: tranquille, paisible.}.
\VS{13}Car Adam a été formé le premier, Eve ensuite.
\VS{14}Et ce n'est pas Adam qui a été séduit, mais la femme, ayant été séduite, a été la cause de la transgression.
\VS{15}Elle sera néanmoins sauvée en mettant des enfants au monde\FTNT{Il est évident que le salut ne dépend pas du fait d'enfanter puisque nous sommes sauvés par grâce et non par les œuvres. Ce verset fait référence à Eve, la mère de tous les vivants. Par elle le péché et la mort sont entrés dans le monde (Ro. 5:12) mais c'est aussi par sa postérité, à savoir Christ (Ge. 3:15), qu'elle, ainsi que tout le genre humain (hommes et femmes), sera sauvé.}, pourvu qu'elle persévère dans la foi, dans la charité, et dans la sanctification, avec modestie.
\Chap{3}
\TextTitle{Les évêques et les diacres doivent manifester le caractère de Christ}
\VerseOne{}Cette parole est certaine, si quelqu'un désire la charge d'évêque\FTNT{Evêque, du grec «~episkope~», signifie «~investigation, inspection, visite d'inspection~». C'est un acte par lequel Dieu visite les hommes, observe leurs voies, leurs caractères, pour leur accorder en partage joie ou tristesse. Ce terme signifie également surveillance, charge, contrôle, fonction, la fonction d'un ancien. Voir Ac. 1:20.}, il désire une œuvre excellente.
\VS{2}Mais il faut que l'évêque soit irrépréhensible, mari\FTNT{Paul ne dit pas que les évêques ne peuvent pas être célibataires. Il y a en effet une différence entre mari et marié. L'apôtre met l'accent sur la monogamie. Un homme célibataire peut en effet être évêque s'il remplit les caractéristiques décrites dans ce passage.} d'une seule femme, vigilant, modéré, honorable, hospitalier, propre à enseigner.
\VS{3}Il faut qu'il ne soit ni adonné au vin, ni violent, ni porté au gain déshonnête, mais modéré, éloigné des querelles, exempt d'avarice.
\VS{4}Il faut qu'il dirige honnêtement sa propre maison, et qu'il tienne ses enfants dans la soumission et dans une parfaite honnêteté~;
\VS{5}car si quelqu'un ne sait pas diriger sa propre maison, comment pourra-t-il gouverner l'église de Dieu~?
\VS{6}Il ne faut pas qu'il soit un nouveau converti, de peur qu'enflé d'orgueil, il ne tombe sous le jugement du diable.
\VS{7}Il faut aussi qu'il reçoive un bon témoignage de ceux du dehors, afin de ne pas tomber dans l'opprobre et dans les pièges du diable.
\VS{8}Que les diacres aussi soient honnêtes, éloignés de la duplicité, des excès du vin, d'un gain sordide,
\VS{9}conservant le mystère de la foi dans une conscience pure.
\VS{10}Que ceux-ci aussi soient premièrement éprouvés, et qu'ensuite ils servent, après avoir été trouvés sans reproche.
\VS{11}Leurs femmes, de même, doivent être honnêtes, non médisantes, sobres, fidèles en toutes choses.
\VS{12}Les diacres doivent être maris d'une seule femme, dirigeant honnêtement leurs enfants, et leurs propres maisons.
\VS{13}Car ceux qui auront bien servi s'acquièrent un rang honorable, et une grande liberté dans la foi qui est en Jésus-Christ.
\VS{14}Je t'écris ces choses espérant que j'irai bientôt vers toi~;
\VS{15}mais si je tarde, je t'écris ces choses afin que tu saches comment il faut se conduire dans la maison de Dieu, qui est l'Eglise du Dieu vivant, la colonne et l'appui de la vérité.
\VS{16}Et sans contredit, le mystère de la piété\FTNT{Le mystère de la piété. Il s'agit de la connaissance de Dieu manifestée en chair dans la personne de Jésus-Christ, 100\% homme et 100\% Dieu. C'est l'incarnation du Dieu Tout-Puissant dans le seul but de sauver les hommes et de produire dans leurs cœurs la véritable piété.} est grand~: Dieu a été manifesté en chair, justifié par l'Esprit, vu des anges, prêché aux Gentils, cru dans le monde, et élevé dans la gloire.
\Chap{4}
\TextTitle{L'apostasie et la séduction: Signes des derniers temps}
\VerseOne{}Mais l'Esprit dit expressément que dans les derniers temps, quelques-uns se détourneront de la foi pour s'attacher à des esprits séducteurs et à des doctrines de démons\FTNT{Il est indéniable que nous vivons les dernières minutes avant le retour glorieux de Jésus-Christ. Toutes les conditions sont pratiquement réunies pour que le Seigneur revienne, c'est pourquoi chaque enfant de Dieu doit se préparer à la rencontre avec l'Epoux. Les prophètes, notamment Paul, ont annoncé que la fin des temps serait caractérisée par la séduction et l'abandon de la foi de beaucoup de chrétiens.},
\VS{2}par l'hypocrisie de faux docteurs, ayant leur propre conscience marquée au fer rouge\FTNT{L'expression «~marqué au fer~» ou «~marque de la flétrissure~» se dit «~kauteriazo~» en grec et veut dire «~ceux dont l'âme est stigmatisée par les marques du péché~». Dans un sens médical, ce mot signifie «~cautériser~». Ce passage fait allusion à la marque de la bête qui sera imprimée dans la conscience des hommes~; voilà pourquoi Dieu nous demande de garder sa parole dans nos cœurs (Ps. 119:11). Les Juifs devaient avoir sur leurs mains et sur leurs fronts la marque de Dieu qui est sa parole (De. 6:6-8). La main se dit «~yad~» en hébreu, ce qui signifie «~pouvoir~», «~force~» ou encore «~autorité~»~; elle symbolise donc l'action. Le front se dit «~towphaphah~» en hébreu, ce qui signifie «~marque~»~; il s'agit de la pensée.}~;
\VS{3}défendant de se marier et commandant de s'abstenir des viandes que Dieu a créées afin que les fidèles, et ceux qui ont connu la vérité, en usent avec actions de grâces.
\VS{4}Car tout ce que Dieu a créé est bon, et rien ne doit être rejeté, pourvu qu'on le prenne avec actions de grâces,
\VS{5}parce que tout est sanctifié par la parole de Dieu et par la prière.
\TextTitle{S'exercer à la piété}
\VS{6}En exposant ces choses aux frères, tu seras un bon serviteur de Jésus-Christ, nourri des paroles de la foi et de la bonne doctrine que tu as exactement suivie.
\VS{7}Mais rejette les fables profanes, et semblables aux récits de vieilles femmes.
\VS{8}Exerce-toi à la piété~; car l'exercice corporel est utile à peu de chose, tandis que la piété est utile à toutes choses, ayant les promesses de la vie présente et de celle qui est à venir.
\VS{9}C'est là une parole certaine et digne d'être entièrement reçue.
\VS{10}Car c'est aussi à cause de cela que nous endurons des travaux et des opprobres, parce que nous espérons dans le Dieu vivant, qui est le Sauveur de tous les hommes, mais principalement des fidèles.
\VS{11}Déclare ces choses et enseigne-les.
\VS{12}Que personne ne méprise ta jeunesse~; mais sois le modèle pour les fidèles en paroles, en conduite, en charité, en esprit, en foi, en pureté.
\VS{13}Applique-toi à la lecture, à l'exhortation et à l'enseignement, jusqu'à ce que je vienne.
\VS{14}Ne néglige pas le don qui est en toi, et qui t'a été donné par prophétie, par l'imposition des mains de l'assemblée des anciens.
\VS{15}Pratique ces choses et donne-toi tout entier à elles, afin que tes progrès soient évidents pour tous.
\VS{16}Veille sur toi-même et sur la doctrine~; persévère dans ces choses, car en agissant ainsi, tu te sauveras toi-même et tu sauveras ceux qui t'écoutent.
\Chap{5}
\TextTitle{Recommandations concernant les veuves}
\VerseOne{}Ne reprends pas rudement le vieillard, mais exhorte-le comme un père~; les jeunes gens comme des frères,
\VS{2}les femmes âgées comme des mères, celles qui sont jeunes comme des sœurs, en toute pureté.
\VS{3}Honore les veuves qui sont véritablement veuves.
\VS{4}Mais si une veuve a des enfants, ou des petits enfants, qu'ils apprennent avant tout à exercer la piété envers leur propre famille, et à rendre à leurs parents ce qu'ils ont reçu d'eux~; car cela est bon et agréable à Dieu.
\VS{5}Or celle qui est véritablement veuve, et qui est laissée seule, espère en Dieu, et persévère nuit et jour dans les supplications et les prières.
\VS{6}Mais celle qui vit dans les plaisirs est morte quoique vivante.
\VS{7}Avertis-les donc de ces choses, afin qu'elles soient irrépréhensibles.
\VS{8}Que si quelqu'un n'a pas soin des siens, et principalement de ceux de sa famille, il a renié la foi, et il est pire qu'un infidèle.
\VS{9}Qu'une veuve, pour être enregistrée sur le rôle\FTNT{Inscription sur le rôle~: Expression qui s'apparente à l'enrôlement des soldats. Il est question des veuves ayant une place importante dans l'église, du fait qu'elles exercent une certaine responsabilité sur le reste des femmes, et ayant en charge les veuves et les orphelins pris en compte pour la dépense publique.}, n'ait pas moins de soixante ans, qu'elle ait été la femme d'un seul mari,
\VS{10}ayant le témoignage d'avoir fait de bonnes œuvres, comme d'avoir bien élevé ses propres enfants, d'avoir exercé l'hospitalité envers les étrangers, d'avoir lavé les pieds des saints, d'avoir secouru les affligés, et de s'être ainsi constamment appliquée à toutes sortes de bonnes œuvres.
\VS{11}Mais refuse les veuves qui sont plus jeunes~; car quand elles sont devenues lascives\FTNT{Ce mot vient du grec «~katastreniao~»~: «~ressentir les pulsions du désir sexuel~».} contre Christ, elles veulent se marier,
\VS{12}ayant leur condamnation, en ce qu'elles ont violé leur première foi.
\VS{13}Et avec cela aussi, étant oisives, elles apprennent à aller de maison en maison~; et non seulement elles sont oisives, mais encore causeuses, et curieuses, et parlant de choses qui ne sont pas bienséantes.
\VS{14}Je veux donc que les jeunes veuves se marient, qu'elles aient des enfants, qu'elles gouvernent leur ménage, et qu'elles ne donnent à l'adversaire aucune occasion de médire.
\VS{15}Car quelques-unes se sont déjà détournées pour suivre Satan.
\VS{16}Si quelque fidèle, homme ou quelque femme, a des veuves, qu'ils les assistent, et que l'église n'en soit point chargée, afin qu'elle puisse assister celles qui sont véritablement veuves.
\TextTitle{Recommandations concernant les anciens}
\VS{17}Que les anciens qui dirigent\FTNT{Du grec «~proistemi~»~: «~disposer~» ou «~placer devant~», «~diriger~», «~présider~» (1 Th. 5:12~; Ro. 12:8~; \vref{1 Ti. 3:4-5,12}.} convenablement soient jugés dignes d'un double honneur, spécialement ceux qui travaillent à la prédication et à l'enseignement.
\VS{18}Car l'Ecriture dit~: Tu n'emmuselleras point le bœuf quand il foule le grain\FTNT{De. 25:4.}. Et l'ouvrier mérite son salaire\FTNT{Lu. 10:7.}.
\VS{19}Ne reçois point d'accusation contre un ancien, si ce n'est sur la déposition de deux ou de trois témoins\FTNT{De. 19:15~; Mt. 18:16~; 2 Co. 13:1}.
\VS{20}Reprends publiquement ceux qui pèchent, afin que les autres aussi en aient de la crainte.
\VS{21}Je te conjure devant Dieu, et devant le Seigneur Jésus-Christ, et devant les anges élus, d'observer ces choses sans préférer l'un à l'autre, et de ne rien faire avec partialité.
\VS{22}N'impose les mains à personne avec précipitation, et ne participe pas aux péchés d'autrui~; toi-même, conserve-toi pur.
\VS{23}Ne bois plus uniquement de l'eau~; mais use d'un peu de vin, à cause de ton estomac et de tes fréquentes maladies.
\VS{24}Les péchés de certains hommes sont manifestes, même avant tout jugement, alors que chez d'autres, ils ne se découvrent qu'après.
\VS{25}De même, les bonnes œuvres sont manifestes, et celles qui ne les sont pas ne peuvent pas rester cachées\FTNT{Mt. 10:26~; Mc. 4:22~; Lu. 8:17~; Lu. 12:2.}.
\Chap{6}
\TextTitle{L'attitude du serviteur envers son maître}
\VerseOne{}Que tous les esclaves qui sont sous le joug sachent qu'ils doivent à leurs maîtres toute sorte d'honneur, afin qu'on ne blasphème pas le Nom de Dieu et sa doctrine.
\VS{2}Et que ceux qui ont des fidèles pour maîtres ne les méprisent point sous prétexte qu'ils sont leurs frères, mais qu'ils les servent d'autant mieux que ce sont des fidèles et des bien-aimés de Dieu, étant participants de la grâce. Enseigne ces choses et recommande-les.
\VS{3}Si quelqu'un enseigne des fausses doctrines, et ne se soumet pas aux saines paroles de notre Seigneur Jésus-Christ, et à la doctrine qui est selon la piété,
\VS{4}il est enflé d'orgueil, il ne sait rien~; mais il a la maladie des questions et des disputes de mots, d'où naissent l'envie, les querelles, les médisances et les mauvais soupçons,
\VS{5}les vaines disputes d'hommes corrompus d'entendement et privés de la vérité, qui estiment que la piété est un moyen de gagner. Sépare-toi de ces sortes de gens.
\TextTitle{L'amour de l'argent~: La racine de tous les maux}
\VS{6}Or la piété avec le contentement d'esprit est un grand gain.
\VS{7}Car nous n'avons rien apporté dans le monde, et aussi il est évident que nous n'en pouvons rien emporter.
\VS{8}Si nous avons la nourriture et le vêtement, cela nous suffira.
\VS{9}Mais ceux qui veulent devenir riches tombent dans la tentation\FTNT{La tentation se rapporte à l'envie de toujours posséder, de s'enrichir et de gagner plus d'argent. Cela finit par faire tomber les gens dans l'orgueil, le mensonge, la duplicité, dans la fornication, etc.}, dans le piège\FTNT{Le mot «~piège~» vient du grec «~pagis~» qui donne en français «~trappe~», «~filet~». «~Car il surprendra comme un filet tous ceux qui habitent sur la surface de toute la terre.~» Lu. 21:35. Ce mot suggère l'inattendu, l'improviste, la surprise, car les oiseaux et autres animaux pris dans le filet sont attrapés par surprise. Les conséquences de la cupidité sont nombreuses, notamment le mensonge et l'adultère. En effet, une personne cupide finit en général par tromper son conjoint.}, et dans beaucoup de désirs insensés et pernicieux\FTNT{Les désirs insensés et pernicieux sont multiples~: l'envie de toujours posséder plus que les autres, la convoitise, les rivalités, la concurrence, la folie des grandeurs. Ces choses sortent les gens de la vision du Seigneur (Mc. 4:19).} qui plongent les hommes dans la ruine et la perdition\FTNT{La ruine et la perdition. Une personne cupide se perd en s'éloignant du Seigneur (2 Pi. 2). Selon Salomon, l'argent ne rassasie personne. «~Celui qui aime l'argent n'est point rassasié par l'argent, et celui qui aime un grand train, n'en est pas nourri~; cela aussi est une vanité.~» (Ec. 5:9). Selon les Ecritures, le système bancaire mondial s'écroulera dans les prochaines années (Ap. 18).}.
\VS{10}Car l'amour de l'argent est la racine de tous les maux\FTNT{L'amour de l'argent est la racine de tous les maux. Ceux qui espèrent en une sécurité divine doivent renoncer à la sécurité matérielle et financière que la chair désire.}~; et quelques-uns en étant possédés, se sont détournés de la foi et se sont jetés eux-mêmes dans bien des tourments.
\VS{11}Mais toi, homme de Dieu~! Fuis ces choses, et recherche la justice, la piété, la foi, la charité, la patience, la douceur.
\VS{12}Combats le bon combat de la foi, saisis la vie éternelle, à laquelle aussi tu as été appelé, et pour laquelle tu as fait une belle confession en présence de plusieurs témoins.
\VS{13}Je t'ordonne, devant Dieu qui donne la vie à toutes choses, et devant Jésus-Christ qui a fait cette belle confession devant Ponce Pilate,
\VS{14}de garder ce commandement, en te conservant sans tache et irrépréhensible, jusqu'à l'apparition de notre Seigneur Jésus-Christ,
\VS{15}qui sera manifesté en son temps, qui est le Béni et seul Prince, le Roi des rois, et le Seigneur des seigneurs,
\VS{16}qui seul possède l'immortalité, et qui habite une lumière inaccessible, que nul homme n'a vu ni ne peut voir, à qui appartiennent l'honneur et la puissance éternelle. Amen~!
\VS{17}Ordonne à ceux qui sont riches dans ce monde, qu'ils ne soient pas hautains, et qu'ils ne mettent pas leur confiance dans l'incertitude des richesses, mais dans le Dieu vivant, qui nous donne toutes choses abondamment pour en jouir.
\VS{18}Qu'ils fassent du bien, qu'ils soient riches en bonnes œuvres, qu'ils soient prompts à donner, avec libéralité,
\VS{19}s'amassant ainsi pour l'avenir un trésor placé sur un fondement solide, afin qu'ils obtiennent la vie éternelle.
\TextTitle{Conclusion}
\VS{20}Timothée, garde le dépôt, en fuyant les discours vains et profanes, et les contradictions d'une science faussement ainsi nommée,
\VS{21}dont font profession quelques-uns qui se sont détournés de la foi. Que la grâce soit avec toi~! Amen~!
\PPE{}
\end{multicols}

%\clearpage\ShortTitle{Tit.}\BookTitle{Tite}\BFont
\noindent\hrulefill
{\footnotesize
\textit{
\bigskip
{\centering{}
\\Auteur : Paul
\\(Gr. : Titos)
\\Signifie : Nourrice, honorable
\\Thème : L'ordre dans les églises
\\Date de rédaction : Env. 65 ap. J.-C.\\}
}
%\bigskip
\textit{
\\Cette épître pastorale fut écrite après la libération de Paul de sa première captivité romaine, peut-être dans la ville de Philippes. Tite, disciple d'origine païenne et collaborateur de Paul, se trouvait alors en Crète où Paul l'avait laissé afin qu'il organise les églises. Dans cette lettre, l'apôtre traite des conditions requises pour assumer la charge d'ancien en mettant l'accent sur la saine doctrine. Mentionnant également les obligations morales des jeunes, des personnes âgées, ainsi que des serviteurs, il exhorte Tite à veiller et à s'éloigner des apostats.\bigskip
}
}
\par\nobreak\noindent\hrulefill
\begin{multicols}{2}
\Chap{1}
\TextTitle{Introduction}
\VerseOne{}Paul, serviteur de Dieu, et apôtre de Jésus-Christ, selon la foi des élus de Dieu et la connaissance de la vérité qui est selon la piété,
\VS{2}dans l'espérance de la vie éternelle, que Dieu, qui ne peut mentir, avait promise avant les temps éternels,
\VS{3}mais qu'il a manifestée en son propre temps par sa parole, dans la prédication qui m'a été confiée, par le commandement de Dieu notre Sauveur,
\VS{4}à Tite mon vrai fils, selon la foi qui nous est commune: Que la grâce, la miséricorde, et la paix te soient données de la part de Dieu notre Père, et de la part du Seigneur Jésus-Christ, notre Sauveur !
\TextTitle{Les caractéristiques d'un ancien}
\VS{5}La raison pour laquelle je t'ai laissé en Crète, c'est afin que tu achèves de mettre en bon ordre les choses qui restent à régler, et que tu établisses des anciens de ville en ville, suivant ce que je t'ai ordonné,
\VS{6}s'il s'y trouve un homme qui soit irrépréhensible, mari d'une seule femme, ayant des enfants fidèles, qui ne soient ni accusés de dissolution, ni rebelles.
\VS{7}Car il faut que l'évêque soit irrépréhensible, comme étant économe dans la maison de Dieu ; qu'il ne soit ni arrogant, ni coléreux, ni adonné au vin, ni violent, non convoiteux d'un gain déshonnête ;
\VS{8}mais hospitalier, aimant les gens de bien, sage, juste, saint, tempérant,
\VS{9}attaché à la parole de la vérité comme elle lui a été enseignée, afin qu'il soit capable tant d'exhorter par la saine doctrine, que de réfuter les contredisants\FTNT{Lu. 2:34 ; Jn. 19:12 ; Ac. 13:45 ; 28:19 ; 28:22 ; Ro. 10:21.}.
\VS{10}Car il y en a plusieurs qui ne veulent pas se soumettre, vains discoureurs, et séducteurs d'esprits, principalement ceux qui sont de la circoncision,
\VS{11}auxquels il faut fermer la bouche, et qui renversent les maisons tout entières enseignant pour un gain déshonnête des choses qu'on ne doit point enseigner.
\VS{12}Quelqu'un d'entre eux, qui était leur propre prophète, a dit : Les Crétois sont toujours menteurs, de mauvaises bêtes, des ventres paresseux.
\VS{13}Ce témoignage est véritable. C'est pourquoi reprends-les vivement, afin qu'ils soient sains dans la foi,
\VS{14}et qu'ils ne s'attachent point aux fables judaïques et aux commandements d'hommes qui se détournent de la vérité.
\VS{15}Toutes choses sont bien pures pour ceux qui sont purs, mais rien n'est pur pour les impurs et les infidèles ; mais leur entendement et leur conscience sont souillés.
\VS{16}Ils font profession de connaître Dieu, mais ils le renient par leurs œuvres, car ils sont abominables, et rebelles, et réprouvés pour toute bonne œuvre.
\Chap{2}
\TextTitle{Recommandations de Paul à Tite}
\VerseOne{}Mais toi, annonce les choses qui conviennent à la saine doctrine.
\VS{2}Que les vieillards soient sobres, honnêtes, prudents, sains dans la foi, dans la charité, et dans la patience.
\VS{3}De même, que les femmes âgées règlent leur extérieur d'une manière convenable à la sainteté ; qu'elles ne soient ni médisantes, ni sujettes à beaucoup de vin, mais qu'elles enseignent de bonnes choses,
\VS{4}afin qu'elles instruisent les jeunes femmes à être modestes, à aimer leurs maris, à aimer leurs enfants,
\VS{5}à être modérées, pures, occupées aux soins domestiques, bonnes, soumises à leurs maris, afin que la Parole de Dieu ne soit point blasphémée.
\VS{6}Exhorte aussi les jeunes hommes à être modérés,
\VS{7}te montrant toi-même un modèle de bonnes œuvres en toutes choses, en une doctrine exempte de toute altération, en pureté, en intégrité,
\VS{8}en paroles saines, que l'on ne puisse point condamner, afin que celui qui vous est contraire, soit rendu confus, n'ayant aucun mal à dire de vous.
\VS{9}Que les serviteurs soient soumis à leurs maîtres, leur complaisant en toutes choses, n'étant point contredisants,
\VS{10}ne dérobant rien de ce qui appartient à leurs maîtres, mais faisant toujours paraître une grande fidélité, afin de rendre honorable en toutes choses la doctrine de Dieu, notre Sauveur.
\VS{11}Car la grâce de Dieu, salutaire à tous les hommes, a été manifestée.
\VS{12}Et elle nous enseigne à renoncer à l'impiété et aux passions mondaines, et à vivre dans le présent siècle, selon la sagesse, la justice et la piété,
\VS{13}en attendant la bienheureuse espérance, et l'apparition de la gloire du grand Dieu et notre Sauveur Jésus-Christ,
\VS{14}qui s'est donné lui-même pour nous, afin de nous racheter de toute iniquité, et de nous purifier, pour lui être un peuple qui lui appartienne en propre, et qui soit zélé pour les bonnes œuvres.
\VS{15}Enseigne ces choses, exhorte, et reprends avec une pleine autorité. Et que personne ne te méprise.
\Chap{3}
\TextTitle{Conseils pratiques de Paul}
\VerseOne{}Rappelle-leur d'être soumis aux magistrats et aux autorités, d'obéir aux gouverneurs, d'être prêts à faire toutes sortes de bonnes actions,
\VS{2}de ne médire de personne, de n'être point querelleurs, mais doux, et montrant une parfaite douceur envers tous les hommes.
\VS{3}Car nous aussi, nous étions autrefois insensés, désobéissants, égarés, asservis à toute espèce de convoitises et de voluptés, vivant dans la méchanceté et dans l'envie, dignes d'être haïs, et nous haïssant les uns les autres.
\VS{4}Mais, quand la bonté de Dieu notre Sauveur et son amour envers les hommes ont été manifestés, il nous a sauvés,
\VS{5}non par des œuvres de justice que nous aurions faites, mais selon la miséricorde, par le bain de la régénération et le renouvellement du Saint-Esprit,
\VS{6}qu'il a répandu abondamment sur nous par Jésus-Christ notre Sauveur,
\VS{7}afin qu'ayant été justifiés par sa grâce, nous soyons les héritiers de la vie éternelle selon notre espérance.
\VS{8}Cette parole est certaine, et je veux que tu affirmes ces choses, afin que ceux qui ont cru en Dieu aient soin principalement de s'appliquer à pratiquer les bonnes œuvres. Voilà les choses qui sont bonnes et utiles aux hommes.
\VS{9}Mais évite les discussions folles, les généalogies, les querelles et les disputes de la loi ; car elles sont inutiles et vaines.
\VS{10}Rejette l'homme hérétique, après le premier et le second avertissement,
\VS{11}sachant qu'un tel homme est perverti, et qu'il pèche en se condamnant lui-même.
\TextTitle{Salutations}
\VS{12}Quand je t'enverrai Artémas ou Tychique, hâte-toi de venir vers moi à Nicopolis ; car j'ai résolu d'y passer l'hiver.
\VS{13}Accompagne soigneusement Zénas, docteur de la loi, et Apollos, afin que rien ne leur manque.
\VS{14}Que les nôtres aussi apprennent à être les premiers à s'appliquer aux bonnes œuvres, pour les usages nécessaires, afin qu'ils ne soient point sans fruits.
\VS{15}Tous ceux qui sont avec moi te saluent. Salue ceux qui nous aiment dans la foi. Grâce soit avec vous tous ! Amen !
\PPE{}
\end{multicols}

%\clearpage\ShortTitle{1 Pierre}\BookTitle{1 Pierre}\BFont
\noindent\hrulefill
{\footnotesize
\textit{
\bigskip
{\centering{}
\\Auteur : Pierre
\\(Gr. : Petro)
\\Signification : Roc, pierre
\\Thème : La victoire sur la souffrance
\\Date de rédaction : Env. 65 ap. J.-C.\\}
}
%\bigskip
\textit{
\\Cette lettre semble avoir été écrite à Rome même si Pierre y parlait de « Babylone ». En ces temps de persécutions, les chrétiens devaient être prudents quant à la manière dont ils parlaient du pouvoir en place, c'est pourquoi ils utilisaient souvent des codes. C'est donc durant une période difficile que fut rédigée cette épître qui s'adressait à des églises d'Asie Mineure dont la plupart furent fondées par Paul. Au travers de ces quelques lignes, Pierre exhorte les frères et sœurs à tenir ferme dans la foi malgré les souffrances liées aux épreuves, et les encourage à espérer en Jésus-Christ, leur salut. Il finit cette épître en donnant des conseils quant à l'attitude à avoir au sein de l'église.\bigskip
}
}
\par\nobreak\noindent\hrulefill
\begin{multicols}{2}
\Chap{1}
\TextTitle{Introduction}
\VerseOne{}Pierre, apôtre de Jésus-Christ, à ceux qui sont étrangers et dispersés dans le Pont\FTNT{Le Pont : province formant presque la totalité de l'Asie Mineure.}, la Galatie, la Cappadoce, l'Asie et la Bithynie,
\VS{2}élus selon la prescience de Dieu le Père, par la sanctification de l'Esprit, afin d'obéir à Jésus-Christ, et qu'ils participent à l'aspersion de son sang : Que la grâce et la paix vous soient multipliées !
\TextTitle{Les souffrances du chrétien et sa conduite à la lumière d'un salut parfait}
\VS{3}Béni soit Dieu, et le Père de notre Seigneur Jésus-Christ, qui, par sa grande miséricorde, nous a régénérés, pour une espérance vivante, par la résurrection de Jésus-Christ d'entre les morts,
\VS{4}pour un héritage incorruptible, et qui ne peut ni se souiller, ni se flétrir, qui est conservé dans les cieux pour nous,
\VS{5}qui sommes gardés par la puissance de Dieu, par la foi, afin que nous obtenions le salut, qui est prêt à être révélé dans les derniers temps !
\VS{6}En quoi vous vous réjouissez, quoique vous soyez maintenant affligés pour un peu de temps par diverses épreuves, vu que cela est convenable,
\VS{7}afin que l'épreuve de votre foi, beaucoup plus précieuse que l'or périssable, et qui toutefois est éprouvé par le feu, ait pour résultat la louange, l'honneur et la gloire, lorsque Jésus-Christ sera révélé.
\VS{8}Lequel vous aimez quoique vous ne l'ayez point vu, en qui vous croyez, quoique maintenant vous ne le voyiez pas, et vous vous réjouissez d'une joie ineffable et glorieuse,
\VS{9}remportant la fin de votre foi, savoir le salut de vos âmes.
\VS{10}C'est au sujet de ce salut que les prophètes, qui ont prophétisé touchant la grâce qui vous était destinée, ont fait leurs recherches et leurs investigations.
\VS{11}Ils voulaient sonder l'époque et les circonstances marquées par l'Esprit prophétique de Christ qui était en eux, et qui rendait à l'avance témoignage, leur faisant connaître les souffrances de Christ et la gloire dont elles seraient suivies.
\VS{12}Mais il leur fut révélé que ce n'était pas pour eux-mêmes, mais pour nous, qu'ils administraient ces choses, que vous ont annoncées maintenant ceux qui vous ont prêché l'Evangile par le Saint-Esprit envoyé du ciel, et dans lesquelles les anges désirent plonger leurs regards.
\VS{13}C'est pourquoi, ceignez les reins de votre entendement, soyez sobres, et ayez une entière espérance dans la grâce qui vous est présentée, jusqu'à ce que Jésus-Christ soit révélé\FTNT{Révélé, voir commentaire en 2 Th. 1 :7.}.
\VS{14}Comme des enfants obéissants, ne vous conformez pas à vos convoitises d'autrefois, pendant votre ignorance. 
\VS{15}Mais, comme celui qui vous a appelés est saint, vous aussi de même soyez saints dans toute votre conduite,
\VS{16}selon ce qu'il est écrit : Soyez saints, car je suis saint\FTNT{Lé. 11:44.}.
\VS{17}Et si vous invoquez comme votre Père celui qui juge selon l'œuvre de chacun, sans favoritisme, conduisez-vous avec crainte pendant le temps de votre séjour sur la terre,
\VS{18}sachant que vous avez été rachetés de votre vaine conduite, qui vous avait été enseignée par vos pères, non point par des choses corruptibles, comme par argent, ou par or,
\VS{19}mais par le sang précieux de Christ, comme d'un agneau sans défaut et sans tache,
\VS{20}prédestiné avant la fondation du monde, et manifesté dans les derniers temps, pour vous.
\VS{21}Par lui, vous croyez en Dieu, qui l'a ressuscité des morts et lui a donné la gloire, afin que votre foi et votre espérance reposent sur Dieu.
\VS{22}Ayant donc purifié vos âmes en obéissant à la vérité par le Saint-Esprit, afin que vous ayez un amour fraternel et sans hypocrisie, aimez-vous ardemment les uns les autres d'un cœur pur,
\VS{23}puisque vous avez été régénérés, non par une semence corruptible, mais par une semence incorruptible, par la parole de Dieu qui vit et demeure éternellement.
\VS{24}Car toute chair est comme l'herbe, et toute la gloire de l'homme comme la fleur de l'herbe. L'herbe sèche, et sa fleur tombe ;
\VS{25}mais la parole du Seigneur demeure éternellement\FTNT{Es. 40:6-8.}. Et cette parole est celle qui vous a été annoncée par l'Evangile.
\Chap{2}
\VerseOne{}Ayant donc renoncé à toute sorte de malice, et de toute fraude, et de dissimulation, et d'envie et de toutes médisances,
\VS{2}désirez ardemment, comme des enfants nouveau-nés, le lait spirituel et pur, afin que vous croissiez par lui,
\VS{3}si toutefois vous avez goûté combien le Seigneur est bon.
\VS{4}Et vous approchant de lui, pierre vivante, rejetée par les hommes, mais choisie et précieuse devant Dieu ;
\VS{5}et vous aussi, comme des pierres vivantes, vous êtes édifiés pour être une maison spirituelle, et une sainte sacrificature, afin d'offrir des sacrifices spirituels, agréables à Dieu par Jésus-Christ. 
\VS{6}C'est pourquoi aussi, il est dit dans l'Ecriture : Voici, je mets en Sion la principale pierre\FTNT{Jésus-Christ est la Pierre rejetée par les bâtisseurs. Voir Es. 28:16 ; Ps. 118:22.} de l'angle, choisie et précieuse ; et celui qui croit en elle ne sera point confus.
\VS{7}Elle est donc précieuse pour vous qui croyez. Mais, par rapport aux rebelles, il est dit : La pierre que ceux qui bâtissaient ont rejetée, est devenue la principale de l'angle, 
\VS{8}et une pierre d'achoppement, et un rocher de scandale ; ils se heurtent contre la parole, et sont rebelles et c'est à cela qu'ils sont destinés.
\TextTitle{La position du croyant}
\VS{9}Mais vous, vous êtes la race élue, vous êtes la sacrificature royale, la nation sainte, le peuple acquis, afin que vous annonciez les vertus de celui qui vous a appelés des ténèbres à sa merveilleuse lumière,
\VS{10}vous qui autrefois n'étiez pas son peuple, mais qui maintenant êtes le peuple de Dieu, vous qui n'aviez point obtenu miséricorde, mais qui maintenant avez obtenu miséricorde.
\VS{11}Mes bien-aimés, je vous exhorte, comme étrangers et voyageurs, à vous abstenir des convoitises charnelles qui font la guerre à l'âme.
\VS{12}Ayant une conduite honnête avec les Gentils, afin que, là même où ils vous calomnient comme si vous étiez des malfaiteurs, ils remarquent vos bonnes œuvres, et glorifient Dieu, au jour où il les visitera.
\VS{13}Soyez donc soumis à tout établissement humain, pour l'amour de Dieu : Soit au roi, comme à celui qui est au-dessus des autres,
\VS{14}soit aux gouverneurs, comme à ceux qui sont envoyés de sa part pour punir les méchants et pour honorer les gens de bien.
\VS{15}Car c'est là la volonté de Dieu, qu'en faisant bien vous fermiez la bouche à l'ignorance des hommes insensés.
\VS{16}Comme libres, et non pas comme ayant la liberté pour servir de voile à la méchanceté, mais agissant comme des serviteurs de Dieu.
\VS{17}Honorez tout le monde ; aimez tous vos frères ; craignez Dieu ; honorez le roi.
\VS{18}Serviteurs, soyez soumis en toute crainte à vos maîtres, non seulement à ceux qui sont bons et équitables, mais aussi à ceux qui sont méchants.
\VS{19}Car c'est une chose agréable à Dieu si quelqu'un à cause de la conscience qu'il a envers Dieu, endure des afflictions en souffrant injustement. 
\VS{20}Autrement, quelle gloire en aurez-vous, si lorsque vous péchez et qu’on vous frappe, vous le supportez  patiemment ? Mais si quand vous faites le bien et que vous souffrez, vous le supportez patiemment, voilà où Dieu prend plaisir. 
\TextTitle{Les souffrances de Christ, le Substitut des hommes}
\VS{21}Car vous êtes appelés à cela, vu même que Christ a souffert pour nous, nous laissant un modèle, afin que vous suiviez ses traces, 
\VS{22}lui qui n'a point commis de péché, et dans la bouche duquel il ne s'est point trouvé de fraude ;
\VS{23}qui, lorsqu'on lui disait des outrages, n'en rendait point, et quand on lui faisait du mal, n'usait point de menaces, mais il se remettait à celui qui juge justement ; 
\VS{24}lui qui a porté lui-même nos péchés en son corps sur le bois, afin qu'étant morts aux péchés nous vivions pour la justice ; lui par la meurtrissure\FTNT{Es. 53:5.} duquel même vous avez été guéris.
\VS{25}Car vous étiez comme des brebis errantes, mais maintenant vous êtes convertis au Pasteur et à l'Evêque de vos âmes. 
\Chap{3}
\TextTitle{La conduite chrétienne à la maison et à l'église}
\VerseOne{}Femmes, soyez de même soumises à vos maris, afin que, si quelques-uns n'obéissent point à la parole, ils soient gagnés sans paroles par la conduite de leurs femmes,
\VS{2}lorsqu'ils verront la pureté de votre conduite, accompagnée de crainte.
\VS{3}Et que votre ornement ne soit point celui de dehors, qui consiste dans la frisure des cheveux, et dans une parure d'or, et dans la magnificence des habits,
\VS{4}mais que votre parure consiste dans l’homme caché dans le cœur, c’est-à-dire dans l’incorruptibilité d’un esprit doux et paisible, qui est d’un grand prix devant Dieu.
\VS{5}Car c'est ainsi que se paraient aussi autrefois les saintes femmes qui espéraient en Dieu, étant soumises à leurs maris,
\VS{6}comme Sara, qui obéissait à Abraham et l'appelait son seigneur. C'est d'elle que vous êtes devenues les filles, en faisant ce qui est bien et sans vous laisser troubler par aucune crainte.
\VS{7}Et vous, maris, de même comportez-vous selon la sagesse avec vos femmes, comme un vase\FTNT{Paul utilise une métaphore connu des grecs pour parler du corps : le vase.} plus fragile, c'est-à-dire, féminin ; leur portant honneur comme étant aussi ensemble héritiers de la grâce de la vie, afin que vos prières ne soient pas interrompues. 
\VS{8}Enfin, soyez tous d'un même sentiment, remplis de compassion les uns envers les autres, d'amour fraternel, miséricordieux et doux.
\VS{9}Ne rendez point mal pour mal, ou injure pour injure\FTNT{Mt. 5:44.} ; mais, au contraire, bénissez ; sachant que c'est à cela que vous êtes appelés, afin d'hériter la bénédiction.
\VS{10}Car celui qui veut aimer sa vie et voir des jours heureux, qu'il préserve sa langue du mal, et ses lèvres de prononcer aucune fraude,
\VS{11}qu'il se détourne du mal, et fasse le bien, qu'il recherche la paix, et qu'il tâche de se la procurer ;
\VS{12}car les yeux du Seigneur sont sur les justes, et ses oreilles sont attentives à leurs prières, mais la face du Seigneur est contre ceux qui se conduisent mal.
\TextTitle{La conduite chrétienne aux yeux du monde}
\VS{13}Et qui vous maltraitera, si vous êtes les imitateurs de celui qui est bon ?
\VS{14}Que si toutefois vous souffrez quelque chose pour la justice, vous êtes bienheureux. Mais ne craignez point les maux dont ils veulent vous faire peur, et n'en soyez point troublés ;
\VS{15}mais sanctifiez le Seigneur dans vos coeurs, et soyez toujours prêts à répondre, avec douceur et avec respect, à chacun qui vous demande raison de l'espérance qui est en vous, 
\VS{16}et ayant une bonne conscience, afin que, ceux qui blâment votre bonne conduite en Christ, soient confus de ce qu'ils médisent de vous, comme si vous étiez des malfaiteurs.
\VS{17}Car il vaut mieux, si telle est la volonté de Dieu, que vous souffriez en faisant le bien qu’en faisant le mal.
\TextTitle{Les souffrances de Christ}
\VS{18}Car aussi Christ a souffert une fois pour les péchés, lui juste pour les injustes, afin de nous amener à Dieu, étant mort en la chair, mais vivifié par l'Esprit,
\VS{19}par lequel aussi étant allé, il a  prêché aux esprits qui sont en prison\FTNT{La possibilité du salut après la mort n’a aucun fondement biblique (Hé. 9 :27). Dans ce passage, il est fait mention des pécheurs qui ont vécu du temps de Noé et auxquels le Seigneur Jésus a confirmé la condamnation lorsqu’il est descendu dans l’Hadès ( l'enfer; Ep. 4 :9). Voir aussi commentaire en Mt 16 :18.},
\VS{20}et qui avaient été autrefois incrédules, quand la patience de Dieu les attendait, durant les jours de Noé, tandis que l'arche se préparait dans laquelle un petit nombre, à savoir huit personnes furent sauvées par l'eau.
\VS{21}A quoi aussi maintenant répond la figure qui nous sauve, c'est-à-dire, le baptême ; non point celui par lequel les ordures de la chair sont nettoyées, mais la promesse faite à Dieu d'une conscience pure, par la résurrection de Jésus-Christ,
\VS{22}qui est à la droite de Dieu, étant allé au Ciel, et auquel sont assujettis les anges, et les dominations et les puissances.
\Chap{4}
\TextTitle{Souffrir dans la chair}
\VerseOne{}Puisque Christ a souffert pour nous dans la chair, vous aussi armez-vous de la même pensée. Car celui qui a souffert dans la chair a cessé de pécher,
\VS{2}afin de vivre, non plus selon les convoitises des hommes, mais selon la volonté de Dieu, pendant le temps qui lui reste à vivre dans la chair.
\VS{3}Car il nous suffit d'avoir accompli la volonté des Gentils, pendant le temps de notre vie passée, quand nous nous abandonnions aux impudicités, aux convoitises, à l'ivrognerie, aux excès dans le manger et dans le boire, et aux idolâtries abominables.
\VS{4}Ce que ces Gentils trouvent fort étrange, ils vous calomnient de ce que vous ne courez pas avec eux dans un même débordement de dissolution. 
\VS{5}Mais ils rendront compte à celui qui est prêt à juger les vivants et les morts.
\VS{6}Car c'est aussi pour cela que les morts ont été évangélisés, afin qu'ils soient jugés selon les hommes dans la chair, et qu'ils vivent selon Dieu dans l'esprit. 
\TextTitle{La conduite chrétienne dans le temps présent}
\VS{7}Or la fin de toutes choses est proche : Soyez donc sobres et vigilants pour prier. 
\VS{8}Mais surtout, ayez les uns pour les autres une ardente charité, car la charité couvre une multitude de péchés.
\VS{9}Soyez hospitaliers les uns envers les autres, sans murmures. 
\VS{10}Que chacun selon le don qu'il a reçu, l'emploie pour le service des autres, comme de bons gestionnaires des diverses grâces de Dieu. 
\VS{11}Si quelqu'un parle, qu'il parle comme annonçant les paroles de Dieu ; si quelqu'un administre, qu'il administre comme par la puissance que Dieu lui en a fournie, afin qu'en toutes choses Dieu soit glorifié par Jésus-Christ, auquel appartient la gloire et la force, aux siècles des siècles. Amen ! 
\VS{12}Mes bien-aimés, ne trouvez point étrange quand vous êtes comme dans une fournaise pour votre épreuve, comme s'il vous arrivait quelque chose d'extraordinaire. 
\VS{13}Mais réjouissez-vous, de ce que vous participez aux souffrances de Christ, afin qu'aussi à la révélation de sa gloire, vous vous réjouissiez avec allégresse.
\VS{14}Si on vous dit des injures pour le Nom de Christ, vous êtes heureux, car l'Esprit de gloire et de Dieu repose sur vous, lequel est blasphémé par ceux qui vous noircissent mais pour vous, vous le glorifiez.
\VS{15}Que nul de vous, ne souffre comme meurtrier, ou voleur, ou malfaiteur, ou curieux des affaires d'autrui. 
\VS{16}Mais si quelqu'un souffre comme chrétien, qu'il n'en ait point de honte, mais qu'il glorifie Dieu en cela.
\VS{17}Car il est temps que le jugement commence par la maison de Dieu\FTNT{Le jugement commence par la maison de Dieu. Ez. 9:1-11.}. Or s'il commence premièrement par nous, quelle sera la fin de ceux qui n'obéissent pas à l'Evangile de Dieu ?
\VS{18}Et si le juste est difficilement sauvé, où comparaîtra le méchant et le pécheur ? 
\VS{19}Que ceux-là donc aussi, qui souffrent par la volonté de Dieu, puisqu'ils font ce qui est bon lui recommandent leurs âmes, comme au fidèle Créateur. 
\Chap{5}
\TextTitle{Servir sans rien attendre en retour}
\VerseOne{}Je prie les anciens qui sont parmi vous, moi qui suis ancien avec eux, et témoin des souffrances de Christ, et participant de la gloire qui doit être révélée et je leur dis : 
\VS{2}Paissez le troupeau de Dieu qui vous est commis, en prenant garde sur lui, non point par contrainte, mais volontairement ; non point pour un gain déshonnête, mais par un principe d'affection. 
\VS{3}Et non pas comme ayant domination sur les héritages du Seigneur, mais de telle manière que vous soyez les modèles du troupeau. 
\VS{4}Et quand le souverain Pasteur\FTNT{Jésus est notre Souverain Pasteur. Voir Ps. 23 ; Jn. 10.} apparaîtra, vous obtiendrez la couronne incorruptible de la gloire.
\VS{5}De même, vous jeunes gens, soyez soumis aux anciens. Et ayant tous de la soumission les uns pour les autres, soyez parés par-dedans d'humilité; parce que Dieu résiste aux orgueilleux, mais il fait grâce aux humbles. 
\VS{6}Humiliez-vous donc sous la puissante main de Dieu, afin qu'il vous élève quand le temps sera venu ;
\VS{7}remettez-lui tout ce qui peut vous inquiéter, car il prend soin de vous.
\VS{8}Soyez sobres et veillez : Car le diable, votre adversaire, tourne autour de vous comme un lion rugissant, cherchant qui il pourra dévorer. 
\VS{9}Résistez-lui donc en demeurant fermes dans la foi, sachant que les mêmes souffrances s'accomplissent dans la compagnie de vos frères qui sont dans le monde. 
\TextTitle{Salutations}
\VS{10}Or que le Dieu de toute grâce, qui nous a appelés à sa gloire éternelle en Jésus-Christ, après que vous aurez souffert un peu de temps, vous rende parfaits, vous affermisse, vous fortifie et vous établisse. 
\VS{11}A lui soient la gloire et la force, aux siècles des siècles ! Amen !
\VS{12}Je vous ai écrit brièvement par Silvain, notre frère, que je crois vous être fidèle, vous déclarant et vous protestant que la grâce de Dieu dans laquelle vous êtes est la véritable. 
\VS{13}L'Eglise qui est à Babylone, élue avec vous, et Marc, mon fils, vous saluent. 
\VS{14}Saluez-vous les uns les autres par un baiser de charité. Que la paix soit avec vous tous qui êtes en Jésus-Christ ! Amen !
\PPE{}
\end{multicols}

%\clearpage\ShortTitle{2 Pierre}\BookTitle{2 Pierre}\BFont
\begin{multicols}{2}
\TextTitle{[Introduction]}
\Chap{1}
\VerseOne{}Simon Pierre, serviteur et apôtre de Jésus-Christ, à vous qui avez reçu en partage une foi du même prix que la nôtre, par la justice de notre Dieu et Sauveur Jésus-Christ (1).
\VS{2}Que la grâce et la paix vous soient multipliées, par la connaissance de Dieu, et de notre Seigneur Jésus.
\TextTitle{[Les grandes vertus chrétiennes]}
\VS{3}Puisque sa divine puissance nous a donné tout ce qui appartient à la vie et à la piété, par la connaissance de celui qui nous a appelés par sa gloire et par sa vertu,
\VS{4}par lesquelles nous sont données les grandes et précieuses promesses, afin que par elles vous soyez faits participants de la nature divine, en fuyant la corruption qui règne dans le monde par la convoitise.
\VS{5}A cause de cela même, faites tous vos efforts pour ajouter la vertu à votre foi ; à la vertu, la connaissance,
\VS{6}à la connaissance, la tempérance, à la tempérance, la patience, à la patience, la piété,
\VS{7}à la piété, l'amour fraternel, et à l'amour fraternel, la charité.
\VS{8}Car si ces choses sont en vous, et y abondent, elles ne vous laisseront point oisifs ni stériles pour la connaissance de notre Seigneur Jésus-Christ.
\VS{9}Mais celui en qui ces choses ne se trouvent point est aveugle, et ne voit point de loin, ayant oublié la purification de ses anciens péchés.
\VS{10}C'est pourquoi, mes frères, efforcez-vous plutôt à affermir votre vocation, et votre élection ; car en faisant cela vous ne broncherez jamais.
\VS{11}Car par ce moyen, l'entrée au Royaume éternel de notre Seigneur et Sauveur Jésus-Christ vous sera abondamment accordée.
\TextTitle{[Sollicitude de l'Apôtre pour ses lecteurs ; autorité de son témoignage et de la parole des prophètes]}
\VS{12}C'est pourquoi je ne négligerai pas de vous rappeler sans cesse ces choses, quoique vous ayez de la connaissance, et que vous soyez fondés dans la vérité présente.
\VS{13}Car je crois qu'il est juste que je vous réveille par des avertissements, pendant que je suis dans cette tente (2),
\VS{14}sachant que dans peu de temps je dois la quitter, comme notre Seigneur Jésus-Christ lui-même me l'a déclaré.
\TextTitle{[Souvenir de la transfiguration]}
\VS{15}Mais j'aurai soin qu’après mon départ vous puissiez toujours vous souvenir de ces choses.
\VS{16}Car ce n’est pas en suivant des fables composées avec artifice, que nous vous avons fait connaître la puissance et l’avènement (3) de notre Seigneur Jésus-Christ, mais comme ayant vu sa majesté de nos propres yeux.
\VS{17}Car il reçut de Dieu le Père, honneur et gloire, lorsque cette voix lui fut adressée du milieu de la gloire magnifique : Celui-ci est mon Fils bien-aimé, en qui j'ai mis toute mon affection (4).
\VS{18}Et nous entendîmes cette voix envoyée du ciel, lorsque nous étions avec lui sur la sainte montagne.
\TextTitle{[Témoignage à la véracité des Ecritures prophétiques]}
\VS{19}Nous avons aussi la parole des prophètes qui est très ferme, à laquelle vous faites bien d'être attentifs, comme à une lampe qui brille dans un lieu obscur, jusqu'à ce que le jour vienne à paraître et que l'étoile du matin (5) se lève dans vos cœurs.
\VS{20}Sachant premièrement ceci, qu'aucune prophétie de l'Ecriture ne procède d’une interprétation particulière.
\VS{21}Car la prophétie n'a jamais été autrefois apportée par la volonté humaine, mais les saints hommes de Dieu étant poussés par le Saint-Esprit, ont parlé.
\TextTitle{[Avertissement contre les faux docteurs]}
\Chap{2}
\VerseOne{}Mais comme il y a eu de faux prophètes parmi le peuple, il y aura aussi parmi vous de faux docteurs, qui introduiront secrètement des sectes pernicieuses, et qui reniant le Seigneur qui les a rachetés, attireront sur eux-mêmes une ruine soudaine.
\VS{2}Et plusieurs suivront leurs sectes de perdition, et à cause d'eux, la voie de la vérité sera blasphémée.
\VS{3}Par cupidité, ils trafiqueront (1) de vous au moyen de paroles déguisées, mais la condamnation qui leur est destinée depuis longtemps ne tarde point, et leur perdition ne sommeille point.
\VS{4}Car si Dieu n'a pas épargné les anges qui ont péché, s’il les a précipités dans l'abîme (2), les a liés avec des chaînes d'obscurité, les a livrés pour y être gardés jusqu’au jugement ;
\VS{5}et s'il n'a point épargné l’ancien monde, mais a gardé Noé (3), lui huitième, qui était le prédicateur de la justice ; et a fait venir le déluge sur le monde des impies ;
\VS{6}et s'il a condamné à la destruction totale les villes de Sodome et de Gomorrhe, les réduisant en cendres, et les mettant pour être un exemple à ceux qui vivraient dans l'impiété ;
\VS{7}et s'il a délivré le juste Lot (4), qui cruellement affligé de la conduite de ces hommes sans frein, eut beaucoup à souffrir de ces abominables par leur infâme conduite ;
\VS{8}car cet homme juste, qui habitait au milieu d’eux, affligeait chaque jour son âme juste, à cause de ce qu’il voyait et entendait dire de leurs méchantes actions.
\VS{9}Le Seigneur sait ainsi délivrer de l’épreuve les hommes pieux, et réserver les injustes pour être punis au jour du jugement ;
\VS{10}principalement ceux qui vont après la chair, dans la passion de l'impureté, et qui méprisent l’autorité. Gens audacieux et arrogants, ils ne craignent point d’injurier les gloires ;
\VS{11}alors que les anges qui sont supérieurs en force et en puissance, ne prononcent point contre elles de jugement blasphématoire devant le Seigneur ;
\VS{12}Mais eux, semblables à des bêtes brutes, qui s’abandonnent à leurs penchants naturels, et qui sont nées pour être prises et détruites, ils parlent d’une manière blasphématoire de ce qu’ils ignorent, et ils périront par leur propre corruption.
\VS{13}Et ils recevront la récompense de leur iniquité. Ils aiment à être tous les jours dans les délices. Ce sont des taches et des souillures, et ils font leurs délices de leurs tromperies dans les repas qu'ils font avec vous.
\VS{14}Ils ont les yeux pleins d'adultère, ils ne cessent jamais de pécher, ils attirent les âmes mal affermies ; ils ont le cœur exercé à la cupidité, ce sont des enfants de malédiction,
\TextTitle{[Caractéristiques des faux docteurs
\\a. ils ressemblent à Balaam]}
\VS{15}qui ayant laissé le droit chemin, se sont égarés, et ont suivi la voie de Balaam (5), fils de Bosor, qui aima le salaire de l'iniquité ; mais il fut repris pour sa transgression.
\VS{16}Car une ânesse muette parlant d'une voix humaine, arrêta la folie du prophète.
\TextTitle{[b. ils sont dépourvus dIntroduction]}
\VS{17}Ce sont des fontaines sans eau, des nuées agitées par le tourbillon, et des gens à qui l'obscurité des ténèbres est réservée éternellement.
\TextTitle{[c. leurs discours sont savants et prétentieux]
\\cp. 1 Co. 2:1-5)}
\VS{18}Car en prononçant des discours fort enflés de vanité, ils amorcent par les convoitises de la chair, et par leurs impudicités, ceux qui s'étaient véritablement retirés de ceux qui vivent dans l’égarement ;
\TextTitle{[d. ils corrompent la liberté chrétienne]}
\VS{19}ils leur promettent la liberté, quand ils sont eux-mêmes esclaves de la corruption ; car chacun est esclave de ce qui a triomphé de lui.
\VS{20}En effet, si après s’être retirés des souillures du monde, par la connaissance du Seigneur et Sauveur Jésus-Christ, ils s’y engagent de nouveau et sont vaincus, leur dernière condition est pire que la première (6).
\VS{21}Car mieux valait pour eux n'avoir pas connu la voie de la justice, que de l'avoir connue et se détourner du saint commandement qui leur avait été donné.
\TextTitle{[e. ils retournent à leurs premiers péchés]}
\VS{22}Mais ce qu'on dit par un proverbe véritable leur est arrivé : Le chien est retourné à ce qu'il avait vomi ; et la truie lavée est retournée se vautrer dans le bourbier.
\TextTitle{[Le but de l'Epitre]}
\Chap{3}
\VerseOne{}Mes bien-aimés, c'est ici la seconde lettre que je vous écris, afin de réveiller, dans l'une et dans l'autre, par mes avertissements, les sentiments purs que vous avez.
\VS{2}Et afin que vous vous souveniez des paroles qui ont été dites auparavant par les saints prophètes, et du commandement que vous avez reçu de nous, qui sommes apôtres du Seigneur et Sauveur.
\VS{3}Sur toutes choses, sachez qu'aux derniers jours (1) il viendra des moqueurs, se conduisant selon leurs propres convoitises,
\TextTitle{[La seconde venue de Christ et le jour du Seigneur
\\a. l'incrédulité sera générale quant au retour de Christ]}
\VS{4}et disant : Où est la promesse de son avènement ? Car depuis que les pères sont morts, toutes choses demeurent comme elles ont été dès le commencement de la création.
\VS{5}Car ils ignorent volontairement ceci, c’est que les cieux furent autrefois créés par la parole de Dieu, et que la terre est sortie de l'eau, et qu'elle subsiste parmi l'eau ;
\VS{6}et que par ces choses-là, le monde d'alors périt, étant submergé par les eaux du déluge (2).
\VS{7}Mais les cieux et la terre d’à présent sont gardés par la même parole, étant réservés pour le feu au jour du jugement, et de la destruction des hommes impies.
\VS{8}Mais vous mes bien-aimés, n'ignorez pas ceci, qu'un jour est devant le Seigneur comme mille ans, et mille ans comme un jour (3).
\VS{9}Le Seigneur ne retarde point l'exécution de sa promesse, comme quelques-uns croient qu'il y ait du retard, mais il est patient envers nous, ne voulant qu'aucun ne périsse, mais que tous se repentent.
\TextTitle{[b) la purification des cieux et de la terre]}
\VS{10}Or le jour (4) du Seigneur viendra comme un voleur dans la nuit, et en ce jour-là, les cieux passeront avec le bruit d’une effroyable tempête, et les éléments seront dissous par l'ardeur du feu ; et la terre avec toutes les œuvres qu’elle renferme sera brûlée entièrement.
\VS{11}Puisque toutes ces choses doivent se dissoudre, quelles ne doivent pas être la sainteté de votre conduite et votre piété.
\VS{12}En attendant, et en hâtant par vos désirs la venue du jour de Dieu, par lequel les cieux étant enflammés seront dissous, et les éléments se fondront par l'ardeur du feu.
\VS{13}Mais nous attendons, selon sa promesse, de nouveaux cieux et une nouvelle terre (5), où la justice habitera.
\VS{14}C'est pourquoi, mes bien-aimés, en attendant ces choses, appliquez-vous à être trouvés par lui sans tache et sans reproche dans la paix.
\VS{15}Et croyez que la longue patience de notre Seigneur est la preuve qu'il veut votre salut ; comme Paul, notre frère bien-aimé, vous l’a aussi écrit selon la sagesse qui lui a été donnée ;
\VS{16}comme il le fait aussi dans toutes ses lettres, où il parle de ces points, dans lesquels il y a des choses difficiles à comprendre, dont les personnes ignorantes et mal affermies tordent le sens (6), comme celui des autres Ecritures, pour leur propre perdition.
\TextTitle{[Conclusion]}
\VS{17}Vous donc mes bien-aimés, puisque vous êtes déjà avertis, prenez garde qu'étant emportés avec les autres par la séduction des abominables, vous ne veniez à déchoir de votre fermeté.
\VS{18}Mais croissez dans la grâce et dans la connaissance de notre Seigneur et Sauveur Jésus-Christ. A lui soit la gloire maintenant, et jusqu'au jour d'éternité ! Amen !
\PPE{}
\end{multicols}

%\clearpage\ShortTitle{2 Timothée}\BookTitle{2 Timothée}\BFont
\noindent\hrulefill
{\footnotesize
\textit{
\bigskip
{\centering{}
\\Signifie : Qui adore ou honore Dieu
\\Thème : Le maintien de la vérité
\\Auteur : Paul
\\Date de rédaction : Env. 67\\}
}
%\bigskip
\textit{
\\Cette lettre s’adresse à Timothée dont le père était grec et la mère juive. Le jeune homme se convertit à Christ avec sa mère et sa grand-mère dès le premier voyage missionnaire de Paul au cours duquel il passa à Lystre.
%\bigskip
\\Paul écrit cette épître pastorale en prison à Rome, après avoir été arrêté dans une province orientale à Ephèse ou Troas.  Ses conditions de détentions étant plus rudes que la première fois, Paul restait dubitatif quant à sa mise en liberté. Il demanda donc à Timothée, son fils dans la foi et fidèle compagnon d’œuvre, de le rejoindre à Rome afin semble-t-il de recevoir ses dernières volontés. Après avoir exposé à Timothée les qualités et les devoirs d’un bon serviteur de l’évangile, il l’encouragea à lutter contre les faux docteurs et l’apostasie en prêchant la Parole en toutes circonstances.\bigskip
}
}
\par\nobreak\noindent\hrulefill
\begin{multicols}{2}
\TextTitle{[Introduction]}
\Chap{1}
\VerseOne{}Paul, apôtre de Jésus-Christ, par la volonté de Dieu, selon la promesse de la vie qui est en Jésus-Christ.
\VS{2}A Timothée, mon fils bien-aimé, que la grâce, la miséricorde et la paix te soient données de la part de Dieu le Père, et de la part de Jésus-Christ notre Seigneur.
\TextTitle{[Paul encourage Timothée]}
\VS{3}Je rends grâces à Dieu, que mes ancêtres ont servi et que je sers avec une conscience pure, faisant sans cesse mention de toi dans mes prières nuit et jour,
\VS{4}me souvenant de tes larmes, je désire fort te voir afin que je sois rempli de joie.
\VS{5}Et me souvenant de la foi sincère qui est en toi, et qui a premièrement habité en Loïs, ta grand-mère, et en Eunice, ta mère, et qui, je suis persuadé qu'elle habite aussi en toi.
\VS{6}C'est pourquoi je t'exhorte de ranimer le don de Dieu qui est en toi par l'imposition de mes mains.
\VS{7}Car Dieu ne nous a pas donné un esprit de timidité, mais de force, de charité\FTNT{Il est question ici de l’amour «~agape~», c’est-à-dire divin.} et de sagesse.
\VS{8}N’aie donc point honte du témoignage à rendre à notre Seigneur ni de moi, qui suis son prisonnier ; mais souffre avec moi les afflictions de l'Evangile, selon la puissance de Dieu,
\VS{9}qui nous a sauvés et qui nous a appelés par une sainte vocation, non selon nos œuvres, mais selon son propre dessein, et selon la grâce qui nous a été donnée en Jésus-Christ avant les temps éternels,
\VS{10}et qui maintenant a été manifestée par l'apparition de notre Sauveur Jésus-Christ, qui a détruit la mort et qui a mis en lumière la vie et l'immortalité par l'Evangile,
\VS{11}pour lequel j'ai été établi prédicateur, apôtre et docteur des Gentils.
\VS{12}C'est pourquoi aussi je souffre ces choses, mais je n'en ai point de honte ; car je connais celui en qui j'ai cru, et je suis persuadé qu'il est Puissant pour garder mon dépôt\FTNT{Dépôt : Il est question ici de la connaissance correcte et de la pure doctrine de l’Evangile qui doit être fermement et fidèlement gardée, et qui doit être consciencieusement délivrée aux autres.} jusqu'à ce jour-là.
\VS{13}Retiens dans la foi et dans la charité qui est en Jésus-Christ le modèle des saines paroles que tu as apprises de moi.
\VS{14}Garde le bon dépôt par le Saint-Esprit qui habite en nous.
\VS{15}Tu sais que tous ceux qui sont en Asie se sont éloignés de moi ; entre lesquels sont Phygelle et Hermogène.
\VS{16}Que le Seigneur accorde sa miséricorde à la maison d'Onésiphore, car souvent il m'a consolé, et il n'a point eu honte de mes chaînes.
\VS{17}Au contraire, quand il a été à Rome, il m'a cherché avec beaucoup d’empressement, et il m'a trouvé.
\VS{18}Que le Seigneur lui fasse trouver miséricorde envers le Seigneur en ce jour-là ; et tu sais mieux que personne combien il m'a rendu de services à Ephèse.
\TextTitle{[La conduite d'un disciple de Christ dans les jours d'apostasie]}
\Chap{2}
\VerseOne{}Toi donc, mon fils, sois fortifié dans la grâce qui est en Jésus-Christ.
\VS{2}Et les choses que tu as entendues de moi devant plusieurs témoins, confie-les à des personnes fidèles qui soient capables de les enseigner aussi à d'autres.
\VS{3}Toi donc, souffre avec moi comme un bon soldat de Jésus-Christ.
\VS{4}Il n’est pas de soldat qui s'embarrasse des affaires de cette vie s’il veut plaire à celui qui l'a enrôlé pour la guerre.
\VS{5}De même, l’athlète qui combat n'est point couronné s'il n'a pas combattu selon les règles.
\VS{6}Il faut aussi que le laboureur travaille premièrement, et ensuite il recueille les fruits.
\VS{7}Considère ce que je dis, car le Seigneur te donne de l’intelligence en toutes choses.
\VS{8}Souviens-toi que Jésus-Christ, qui est de la semence de David, est ressuscité des morts, selon mon Evangile,
\VS{9}pour lequel je souffre beaucoup de maux, jusqu'à être mis dans les chaînes comme un malfaiteur, mais cependant, la parole de Dieu n'est point liée.
\VS{10}C'est pourquoi je souffre tout pour l'amour des élus, afin qu'eux aussi obtiennent le salut qui est en Jésus-Christ, avec la gloire éternelle.
\VS{11}Cette parole est certaine, que si nous mourons avec lui, nous vivrons aussi avec lui.
\VS{12}Si nous souffrons avec lui, nous régnerons aussi avec lui. Si nous le renions, il nous reniera aussi\FTNT{Lu. 9:26.}.
\VS{13}Si nous sommes infidèles, il demeure fidèle, car il ne peut pas se renier lui-même.
\VS{14}Remets ces choses en mémoire, protestant devant Dieu qu'on ait pas de disputes de mots, qui est une chose dont il ne revient aucun profit, mais elle est la ruine des auditeurs.
\VS{15}Efforce-toi de te rendre approuvé\FTNT{Approuvé vient du grec ~dokimos~. Du temps de l’apôtre Paul, les systèmes bancaires actuels n’existaient pas, toute la monnaie était en métal. Pour obtenir les pièces de monnaie, le métal était fondu et versé dans des moules et après le démoulage, il était nécessaire d’enlever les bavures. Or de nombreuses personnes les grattaient pour récupérer le surplus de métal et même davantage, ce qui faussait le poids de la monnaie. Face à ce problème, de nombreuses lois furent promulguées à Athènes pour éradiquer la pratique du rognage des pièces en circulation. Il existait toutefois quelques changeurs intègres qui ne mettaient en circulation que des pièces au bon poids. On appelait ces personnes des ~dokimos~, ce qui signifie ~éprouvés~ ou ~approuvés~.} devant Dieu, comme un ouvrier sans reproche, enseignant purement la parole de la vérité.
\VS{16}Mais évite les discours vains et profanes ; car ceux qui les tiennent avanceront toujours plus dans l'impiété,
\VS{17}et leur parole rongera comme une gangrène. Et parmi ceux-là sont Hyménée et Philète,
\VS{18}qui se sont écartés\FTNT{Ecarter, dévier, s'écarter de, manquer le but. A l’époque des apôtres, il y avait plusieurs faux frères qui semaient la zizanie au milieu des enfants de Dieu. Parmi eux étaient Alexandre le forgeron (1 Ti. 1:18-20), Hyménée (1 Ti. 1:18-20), Philète (2 Ti. 2:16-18), les judaïsants (Ac. 15 ; Ga. 2), Diotrèphe (3 Jn.). Les faux frères sont des séducteurs.} de la vérité, en disant que la résurrection est déjà arrivée, et qui renversent la foi de quelques-uns.
\VS{19}Toutefois, le fondement de Dieu demeure ferme, ayant ce sceau : Le Seigneur connaît ceux qui lui appartiennent\FTNT{Le Seigneur connaît ses brebis. Voir No. 16:5 ; Jn. 10:14.} ; et : Quiconque invoque le nom du Seigneur, qu'il s’éloigne de l'iniquité.
\VS{20}Or dans une grande maison, il n'y a pas seulement des vases d'or et d'argent, mais il y en a aussi de bois et de terre. Les uns sont des vases d’honneur et les autres sont d’un usage vil.
\VS{21}Si quelqu'un donc se purifie de ces choses, il sera un vase d’honneur, sanctifié et utile au Seigneur, et préparé pour toute bonne œuvre.
\VS{22}Fuis aussi les désirs de la jeunesse, et recherche la justice, la foi, la charité, et la paix avec ceux qui invoquent le Seigneur d'un cœur pur.
\VS{23}Et rejette les questions\FTNT{Questions folles. Il est question ici de disputes, débats, discussions ou questions oiseuses.} folles, et qui sont sans instruction, sachant qu'elles ne font que produire des querelles.
\VS{24}Or, il ne faut pas que le serviteur du Seigneur soit querelleur, il doit au contraire avoir de la douceur envers tout le monde, propre à enseigner, supportant patiemment les mauvais,
\VS{25}enseignant avec douceur ceux qui ont un sentiment contraire, dans l'espérance qu'un jour Dieu leur donnera la repentance pour reconnaître la vérité,
\VS{26}et afin qu'ils se réveillent pour sortir des pièges du diable, par lesquels ils ont été pris pour faire sa volonté.
\TextTitle{[L'Ecriture, l'arme du chrétien face à l'apostasie]}
\Chap{3}
\VerseOne{}Or sache ceci, que dans les derniers jours\FTNT{Les derniers jours. Voir Ge. 49:1-2.} il surviendra des temps difficiles.
\VS{2}Car les hommes seront idolâtres d’eux-mêmes, amis de l’argent, fanfarons, orgueilleux, blasphémateurs, rebelles à leurs parents, ingrats, irréligieux,
\VS{3}sans affection naturelle, sans fidélité, calomniateurs, intempérants, cruels, haïssant les gens de bien,
\VS{4}traîtres, emportés, enflés d'orgueil, amis des voluptés plûtot qu'amis de Dieu\FTNT{(Le mot grec «~philotheos~» (amour de Dieu) du préfixe «~philos~» qui signifie «~amis, être lié d'amitié avec quelqu'un~» (Matt. 11:19 ; Lu. 7:6 ;Jn. 15:13-15 etc...) et de «~theos~» qui signifie «~Dieu~».}
\VS{5}ayant l'apparence\FTNT{L’apparence de la piété. Le mot ~apparnec~ vient du grec ~morphosis~ et du latin ~forma~ qui donnent ~forme~ en français. Il est question du formalisme, de l’attachement excessif aux règles, aux rites, aux coutumes et aux traditions. Dans l’église de Laodicée, l’accent est plutôt mis sur les règles à observer et les apparences que sur la vie spirituelle et intérieure. Les manifestations extérieures du formalisme sont : les lieux «sacrés» pour adorer (temples, cathédrales, pèlerinages, etc.) ; l’observation des jours sacrés (dimanche et sabbat) ; les rituels censés permettre au croyant d’expérimenter Dieu et de rentrer dans une vie bénie (circoncision, ordination, bénédiction nuptiale, paiement de la dîme, présentation des enfants à Dieu par le pasteur...) ; une manière spéciale de s’habiller (toge, soutane, collet clérical, kippa, voile, costume/cravate, un régime alimentaire spécial, etc.). Voir Mt. 6:1-8.} de la piété, mais en ayant renié la force. Eloigne-toi donc de telles gens.
\VS{6}Il en est parmi eux qui se glissent dans les maisons et qui tiennent captives les femmes chargées de péchés et agitées de diverses convoitises,
\VS{7}qui apprennent toujours, mais qui ne peuvent jamais parvenir à la pleine connaissance de la vérité.
\VS{8}Et comme Jannès et Jambrès ont résisté à Moïse, ceux-ci de même résistent à la vérité, étant des gens qui ont l'esprit corrompu, et qui sont réprouvés quant à la foi.
\VS{9}Mais ils ne feront pas de plus grands progrès, car leur folie sera manifestée à tous, comme le fut celle de ceux-là.
\VS{10}Mais pour toi, tu as pleinement compris ma doctrine, ma conduite, mon intention, ma foi, ma douceur, ma charité, ma persévérance.
\VS{11}Et tu sais les persécutions et les afflictions qui me sont arrivées à Antioche, à Iconie, et à Lystre. Quelles persécutions n’ai-je pas supportées ? Et comment le Seigneur m'a délivré de toutes.
\VS{12}Or tous ceux aussi qui veulent vivre pieusement en Jésus-Christ seront persécutés.
\VS{13}Mais les hommes méchants et imposteurs iront en empirant, séduisant les autres, et étant séduits.
\VS{14}Mais toi, demeure ferme dans les choses que tu as apprises et qui t'ont été confiées, sachant de qui tu les as apprises,
\VS{15}vu même que dès ton enfance tu as la connaissance des saintes lettres, qui peuvent te rendre sage pour le salut par la foi en Jésus-Christ.
\VS{16}Toute l'Ecriture est inspirée de Dieu, et utile pour enseigner, pour convaincre, pour corriger, et pour instruire selon la justice,
\VS{17}afin que l'homme de Dieu soit accompli et parfaitement instruit pour toute bonne œuvre.
\TextTitle{[Paul encourage solennellement Timothée à prêcher la parole]}
\Chap{4}
\VerseOne{}Je te somme devant Dieu, et devant le Seigneur Jésus-Christ, qui doit juger les vivants et les morts, lors de son apparition et de son règne.
\VS{2}Prêche la parole, insiste en toute occasion, favorable ou non. Reprends, censure, exhorte avec toute douceur d'esprit, et avec doctrine.
\VS{3}Car il viendra un temps où les hommes ne supporteront pas la saine doctrine, mais aimant qu'on leur chatouille les oreilles par des discours agréables, ils chercheront des docteurs qui répondent à leurs désirs\FTNT{Beaucoup refusent la saine doctrine et acceptent un évangile basé sur les biens matériels.}.
\VS{4}Et ils détourneront leurs oreilles de la vérité, et se tourneront vers les fables.
\VS{5}Mais toi, veille en toutes choses, souffre les afflictions, fais l'œuvre d'un évangéliste, rends ton ministère pleinement approuvé.
\VS{6}Car pour moi, je m'en vais maintenant servir de libation, et le temps de mon départ est proche.
\VS{7}J'ai combattu le bon combat, j'ai achevé la course, j'ai gardé la foi.
\VS{8}Au reste, la couronne de justice m'est réservée, et le Seigneur, juste Juge, me la rendra en ce jour-là, et non seulement à moi, mais aussi à tous ceux qui auront aimé son apparition.
\VS{9}Hâte-toi de venir bientôt vers moi.
\VS{10}Car Démas m'a abandonné, ayant aimé le présent siècle, et il s'en est allé à Thessalonique ; Crescens est allé en Galatie ; et Tite en Dalmatie.
\VS{11}Luc est seul avec moi ; prends Marc, et amène-le avec toi, car il m'est fort utile pour le ministère.
\VS{12}J'ai aussi envoyé Tychique à Ephèse.
\VS{13}Quand tu viendras, apporte avec toi le manteau que j'ai laissé à Troas, chez Carpus, et les livres aussi ; mais principalement mes parchemins.
\VS{14}Alexandre le forgeron m'a fait beaucoup de mal. Le Seigneur lui rendra selon ses œuvres.
\VS{15}Garde-toi donc de lui, car il s'est fortement opposé à nos paroles.
\VS{16}Personne ne m'a assisté dans ma première défense, mais tous m'ont abandonné ; toutefois que cela ne leur soit point imputé !
\VS{17}Mais le Seigneur m'a assisté et fortifié, afin que ma prédication soit pleinement approuvée, et que tous les Gentils l’entendent ; et j'ai été délivré de la gueule du lion.
\VS{18}Le Seigneur aussi me délivrera de toute mauvaise œuvre, et me sauvera dans son Royaume céleste. A lui soit la gloire aux siècles des siècles. Amen !
\TextTitle{[Conclusion]}
\VS{19}Salue Priscille et Aquilas, et la famille d'Onésiphore.
\VS{20}Eraste est resté à Corinthe, et j'ai laissé Trophime malade à Milet.
\VS{21}Hâte-toi de venir avant l'hiver. Eubulus et Pudens, et Linus, et Claudia, et tous les frères te saluent.
\VS{22}Que le Seigneur Jésus-Christ soit avec ton esprit. Que la grâce soit avec vous. Amen !
\PPE{}
\end{multicols}

%\clearpage\ShortTitle{Jude}\BookTitle{Jude}\BFont
\begin{multicols}{2}
\TextTitle{[Introduction]}
\Chap{1}
\VerseOne{}Jude serviteur de Jésus-Christ, et frère de Jacques, à ceux qui ont été appelés par l'Evangile, que Dieu a sanctifiés et gardés pour Jésus-Christ :
\VS{2}Que la miséricorde, la paix et l'amour vous soient multipliés.
\TextTitle{[Mise en garde contre l'apostasie]}
\VS{3}Mes bien-aimés, comme je désirais vous écrire avec empressement au sujet de notre salut commun, j’ai jugé nécessaire de le faire pour vous exhorter à combattre pour la foi qui a été transmise aux saints une fois pour toutes.
\VS{4}Car il s’est glissé parmi vous, certains hommes dont la condamnation est écrite depuis longtemps, des impies qui changent la grâce de notre Dieu en dissolution, et qui renient le seul Dominateur Jésus-Christ, notre Dieu et Seigneur.
\TextTitle{[Exemples historiques d'incrédulité et de révolte]}
\VS{5}Je veux vous rappeler une chose que vous savez déjà : C'est que le Seigneur après avoir délivré le peuple du pays d'Egypte, fit ensuite périr les incrédules,
\VS{6}qu’il a réservés pour le jugement du grand jour, enchainés éternellement par les ténèbres, les anges qui n'ont pas gardé leur origine, mais qui ont abandonné leur propre demeure ;
\VS{7}que Sodome et Gomorrhe, et les villes voisines qui s'étaient abandonnées comme eux à l'impureté et à des vices contre nature, sont données en exemples, subissant la peine d’un feu éternel.
\TextTitle{[Description des faux docteurs]}
\VS{8}Malgré cela, ces hommes aussi, plongés dans leurs rêveries, souillent leur chair, méprisent l’autorité, et blasphèment contre les dignités.
\VS{9}Or, l'archange Michel, lorsqu’il contestait avec le diable et lui disputait le corps de Moïse, n'osa pas prononcer contre lui un jugement blasphématoire, mais il dit seulement : Que le Seigneur te réprime !
\VS{10}Eux, au contraire, ils blasphèment contre tout ce qu'ils ignorent, et ils se corrompent dans tout ce qu'ils savent naturellement, comme font les bêtes brutes.
\VS{11}Malheur à eux ! Car ils ont suivi la voie de Caïn, et ils se sont jetés dans l’égarement de Balaam, pour l’amour du gain, ils se sont perdus par la rébellion de Koré (1).
\VS{12}Ce sont des écueils dans vos agapes, lorsqu’ils prennent leurs repas avec vous sans aucune retenue, et se repaissant eux-mêmes ; ce sont des nuées sans eau, emportées par des vents çà et là ; des arbres d’automne dont le fruit se pourrit, et sans fruits, deux fois morts, et déracinés ;
\VS{13}des vagues impétueuses de la mer, jetant l'écume de leurs impuretés ; des étoiles errantes, à qui l'obscurité des ténèbres est réservée éternellement.
\VS{14}C’est aussi pour eux qu’Hénoc, le septième homme après Adam, a prophétisé en disant :
\VS{15}Voici, le Seigneur est venu avec ses saintes myriades, pour exercer un jugement contre tous les hommes, et pour convaincre tous les impies parmi eux de tous les actes d'impiété qu’ils ont commis et de toutes les paroles blasphématoires qu’ont proférées contre lui des pécheurs impies.
\VS{16}Ce sont des gens qui murmurent, qui se plaignent toujours, qui marchent selon leurs convoitises, qui ont à la bouche des discours hautains, qui admirent les personnes pour le profit qui leur en revient.
\VS{17}Mais vous, mes bien-aimés, souvenez-vous des choses qui ont été prédites par les apôtres de notre Seigneur Jésus-Christ.
\VS{18}Ils vous disaient que dans les derniers temps il y aurait des moqueurs, qui marcheraient selon leurs convoitises impies.
\VS{19}Ce sont ceux qui provoquent des divisions, des gens sensuels, n'ayant pas l'Esprit.
\TextTitle{[Exhortation aux chrétiens]}
\VS{20}Mais vous, mes bien-aimés, vous édifiant vous-mêmes sur votre très sainte foi, et priant par le Saint-Esprit,
\VS{21}maintenez-vous les uns les autres dans l'amour de Dieu, en attendant la miséricorde de notre Seigneur Jésus-Christ, pour obtenir la vie éternelle.
\VS{22}Et ayez pitié des uns en usant de discernement ;
\VS{23}sauvez-en d’autres avec crainte, en les arrachant hors du feu, haïssant jusqu’à la tunique souillée par la chair.
\TextTitle{[Conclusion]}
\VS{24}Or, à celui qui est puissant pour vous préserver de toute chute et vous faire paraître devant sa gloire irréprochables et dans l’allégresse,
\VS{25}à Dieu, seul sage, notre Sauveur, par Jésus-Christ notre Seigneur, soient gloire et magnificence, force et puissance, dès maintenant et dans tous les siècles, Amen !
\PPE{}
\end{multicols}

%\clearpage\ShortTitle{Hé.}\BookTitle{Hébreux}\BFont
\noindent\hrulefill
{\footnotesize
\textit{
\bigskip
{\centering{}
\\Auteur~: Inconnu
\\Thème~: La prêtrise du Messie
\\Date de rédaction~: Env. 68 ap. J.-C.\\}
}
\textit{
\\Cette épître fut rédigée avant la destruction de Jérusalem, car le temple y subsistait encore. Elle s'adressait à des juifs convertis connaissant bien l'auteur. Parmi eux, certains étaient tentés de retourner au judaïsme à cause des persécutions. L'auteur désire affermir ces chrétiens en leur montrant que l'objectif de la loi avait été réalisé par Christ qui est supérieur aux anges, aux prophètes et à Moïse. Il leur montre combien son œuvre rédemptrice est parfaite et les invite à suivre le Seigneur avec une foi indéfectible en persévérant dans l'amour fraternel.\bigskip
}
}
\par\nobreak\noindent\hrulefill
\begin{multicols}{2}
\Chap{1}
\TextTitle{Dieu parle par le Fils}
\VerseOne{}Dieu ayant anciennement parlé à nos pères par les prophètes, à plusieurs reprises et de plusieurs manières,
\VS{2}nous a parlé dans ces derniers jours\FTNT{Les derniers jours ont commencé avec la naissance de l'Eglise. Voir Joë. 2:28~; Ac. 2:14-17.} par son Fils, qu'il a établi héritier de toutes choses, et par lequel il a aussi créé l'univers~;
\VS{3}et qui étant la splendeur de sa gloire, et l'empreinte de sa substance, et soutenant toutes choses par sa parole puissante, ayant fait par lui-même la purification de nos péchés, s'est assis à la droite de la Majesté divine dans les lieux très hauts.
\TextTitle{Le Fils, supérieur aux anges}
\VS{4}Etant devenu d'autant supérieur aux anges, il a hérité d'un nom plus excellent que le leur.
\VS{5}Car auquel des anges a-t-il jamais dit~: Tu es mon Fils, je t'ai engendré aujourd'hui\FTNT{Ps. 2:7.}~? Et encore~: Je serai pour lui un Père, et il sera pour moi un Fils\FTNT{2 S. 7:14.}~?
\VS{6}Et quand il introduit de nouveau dans le monde son Fils premier-né\FTNT{Voir commentaire en Col. 1:15.}, il est dit~: Et que tous les anges de Dieu l'adorent\FTNT{Ps. 97:7.}~!
\VS{7}Car quant aux anges, il est dit~: Il fait de ses anges des vents, et de ses serviteurs des flammes de feu\FTNT{Ps. 104:4.}.
\VS{8}Mais à l'égard du Fils, il dit~: Ô Dieu, ton trône demeure aux siècles des siècles~; et le sceptre de ton Royaume est un sceptre d'équité~;
\VS{9}tu as aimé la justice, et tu as haï l'iniquité~; c'est pourquoi, ô Dieu, ton Dieu t'a oint d'une huile de joie par-dessus tous tes semblables\FTNT{Ps. 45:7-8.}~!
\VS{10}Et dans un autre endroit~: Toi, Seigneur, tu as fondé la terre dès le commencement, et les cieux sont les ouvrages de tes mains~;
\VS{11}ils périront, mais tu es permanent~; et ils vieilliront tous comme un vêtement,
\VS{12}et tu les rouleras comme un manteau et ils seront changés~; mais toi, tu restes le même, et tes années ne finiront point\FTNT{Es. 50:9~; Es. 51:6~; Ps. 102:27-28.}.
\VS{13}Et auquel des anges a-t-il jamais dit~: Assieds-toi à ma droite, jusqu'à ce que j'aie mis tes ennemis pour le marchepied de tes pieds\FTNT{Ps. 110:1.}~?
\VS{14}Ne sont-ils pas tous des esprits administrateurs, envoyés pour servir en faveur de ceux qui doivent recevoir l'héritage du salut~?
\Chap{2}
\TextTitle{Ne pas négliger le salut}
\VerseOne{}C'est pourquoi il nous faut prendre garde de plus près aux choses que nous avons entendues, de peur que nous les laissions s'échapper.
\VS{2}Car, si la parole prononcée par les anges a été ferme, et si toute transgression et toute désobéissance a reçu une juste rétribution,
\VS{3}comment échapperons-nous, si nous négligeons un si grand salut, qui, ayant été premièrement annoncé par le Seigneur, nous a été confirmé par ceux qui l'avaient entendu~?
\VS{4}Dieu confirmant aussi leur témoignage par des prodiges, et des miracles, et par plusieurs autres différents effets de sa puissance, et par les dons du Saint-Esprit, selon sa volonté.
\TextTitle{Toutes choses doivent être soumises à Christ}
\VS{5}Car, ce n'est pas aux anges qu'il a soumis le monde à venir dont nous parlons.
\VS{6}Et quelqu'un a rendu ce témoignage en quelque autre endroit, disant~: Qu'est-ce que l'homme, pour que tu te souviennes de lui, ou le fils de l'homme, pour que tu le visites~?
\VS{7}Tu l'as fait un peu moindre que les anges, tu l'as couronné de gloire et d'honneur, et l'as établi sur les œuvres de tes mains.
\VS{8}Tu as assujetti toutes choses sous ses pieds\FTNT{Ps. 8:5-7.}. En effet, en lui assujettissant toutes choses, il n'a rien laissé qui ne lui soit assujetti. Mais, nous ne voyons pourtant pas encore que toutes choses lui soient assujetties.
\TextTitle{Jésus abaissé un peu de temps pour sauver l'homme}
\VS{9}Mais celui qui a été fait un peu moindre que les anges, Jésus, nous le voyons couronné de gloire et d'honneur par la passion de sa mort, afin que par la grâce de Dieu, il souffrît la mort pour tous.
\VS{10}Car il était convenable, que celui pour qui sont toutes choses et par qui sont toutes choses, puisqu'il a amené plusieurs enfants à la gloire, consacre le Prince de leur salut par les afflictions.
\VS{11}Car, et celui qui sanctifie et ceux qui sont sanctifiés descendent tous d'un même père. C'est pourquoi il n'a pas honte de les appeler ses frères,
\VS{12}disant~: J'annoncerai ton Nom à mes frères, et je te louerai au milieu de l'assemblée\FTNT{Ps. 22:23.}.
\VS{13}Et encore~: Je me confierai en lui. Et encore~: Me voici, moi et les enfants que Dieu m'a donnés\FTNT{Es. 8:17-18.}.
\VS{14}Ainsi donc, puisque les enfants participent à la chair et au sang, lui aussi de même a participé aux mêmes choses, afin que, par la mort, il rende impuissant celui qui avait le pouvoir de la mort, c'est-à-dire le diable,
\VS{15}et qu'il délivre tous ceux qui, par crainte de la mort, étaient assujettis toute leur vie à la servitude.
\VS{16}Car, certes, il n'a nullement secouru les anges, mais il a secouru la postérité d'Abraham.
\VS{17}C'est pourquoi il a fallu qu'il soit semblable en toutes choses à ses frères, afin qu'il soit un Grand-Prêtre miséricordieux et fidèle dans les choses qui doivent être faites envers Dieu, pour faire la propitiation pour les péchés du peuple~;
\VS{18}car, parce qu'il a souffert lui-même, étant tenté, il est puissant pour secourir ceux qui sont tentés.
\Chap{3}
\TextTitle{Christ, supérieur à Moïse}
\VerseOne{}C'est pourquoi, mes frères saints, qui avez part à la vocation céleste, considérez attentivement Jésus-Christ, l'Apôtre et le Grand-Prêtre de notre profession,
\VS{2}qui a été fidèle à celui qui l'a établi, comme le fut Moïse dans toute sa maison.
\VS{3}Car Jésus-Christ a été jugé digne d'une gloire d'autant supérieure à celle de Moïse, que celui qui a construit une maison, a plus d'honneur que la maison même.
\VS{4}Car chaque maison est construite par quelqu'un, mais celui qui a construit toutes choses, c'est Dieu.
\VS{5}Et quant à Moïse, il a été fidèle dans toute sa maison, comme serviteur, pour témoigner des choses qui devaient être dites~;
\VS{6}mais Christ l'est comme Fils sur sa maison~; et nous sommes sa maison\FTNT{L'Eglise véritable est la maison de Dieu. Voir Es. 66:1~; 1 Co. 3:16~; 1 Co. 6:19~; Ep. 2:21-22. Les bâtiments ne sont pas la maison de Dieu. Le premier bâtiment d'église avait été édifié par des fidèles sous le règne d'Alexandre Sévère en 222-235. L'Eglise véritable est composée de pierres vivantes qui ont pour fondement le Roc (Jésus), parce qu'elle est bâtie par Jésus-Christ lui-même et qu'elle est sa propriété~; les démons ne peuvent pas la détruire. L'Eglise véritable ne peut donc être confondue avec un bâtiment ou une maison physique.}, pourvu que nous retenions fermement jusqu'à la fin l'assurance et la gloire de l'espérance.
\TextTitle{Résultat de l'incrédulité de la génération qui sortit d'Egypte}
\VS{7}C'est pourquoi, comme dit le Saint-Esprit~: Aujourd'hui, si vous entendez sa voix,
\VS{8}n'endurcissez point vos cœurs, comme il arriva dans le lieu de la rébellion, au jour de la tentation dans le désert,
\VS{9}où vos pères me tentèrent et m'éprouvèrent, et ils virent mes œuvres pendant quarante ans\FTNT{Ps. 95:8-11.}.
\VS{10}C'est pourquoi je fus irrité contre cette génération, et je dis~: Leur cœur s'égare toujours. Et ils n'ont pas connu mes voies.
\VS{11}Aussi, je jurai dans ma colère~: Ils n'entreront pas dans mon repos~!
\VS{12}Mes frères, prenez garde que quelqu'un de vous n'ait un cœur mauvais et incrédule, au point de se révolter contre le Dieu vivant,
\VS{13}mais exhortez-vous les uns les autres chaque jour, aussi longtemps qu'on peut dire~: Aujoud'hui~! De peur que quelqu'un d'entre vous ne s'endurcisse par la séduction du péché.
\VS{14}Car nous sommes devenus participants de Christ, pourvu que nous gardions ferme jusqu'à la fin notre première assurance,
\VS{15}pendant qu'il est dit~: Aujourd'hui, si vous entendez sa voix, n'endurcissez pas vos cœurs, comme il arriva dans le lieu de la rébellion.
\VS{16}Car, quelques-uns l'ayant entendue, le provoquèrent à la colère~; mais ce ne furent pas tous ceux qui étaient sortis d'Egypte par Moïse. 
\VS{17}Et contre qui Dieu fut-il irrité pendant quarante ans~? Ne fut-ce pas contre ceux qui péchèrent, et dont les cadavres tombèrent dans le désert~?
\VS{18}Et à qui jura-t-il qu'ils n'entreraient point dans son repos, sinon à ceux qui furent rebelles~?
\VS{19}Aussi, nous voyons qu'ils ne purent y entrer à cause de leur incrédulité.
\Chap{4}
\TextTitle{Le repos}
\VerseOne{}Craignons donc, que quelqu'un d'entre vous, venant à négliger la promesse d'entrer dans son repos, ne s'en trouve privé.
\VS{2}Car il nous a été évangélisé, aussi bien qu'à eux~; mais la parole qu'ils entendirent ne leur servit de rien, parce qu'elle n'était pas mêlée avec la foi dans ceux qui l'entendirent.
\VS{3}Pour nous qui avons cru, nous entrons dans le repos, suivant ce qui a été dit~: C'est pourquoi je jurai dans ma colère, ils n'entreront pas dans mon repos\FTNT{Hé. 3:11.}~! Il dit cela, quoique ses œuvres aient été achevées depuis la fondation du monde.
\VS{4}Car il a parlé quelque part ainsi du septième jour~: Et Dieu se reposa de toutes ses œuvres le septième jour\FTNT{Ge. 2:2.}.
\VS{5}Et encore dans ce passage~: Ils n'entreront pas dans mon repos~!
\VS{6}Puisqu'il reste donc à quelques-uns d'y entrer, et que ceux à qui d'abord il a été évangélisé n'y sont pas entrés à cause de leur désobéissance,
\VS{7}Dieu détermine de nouveau un certain jour, qu'il appelle aujourd'hui, en disant par David si longtemps après, selon ce qui a été dit~: Aujourd'hui, si vous entendez sa voix, n'endurcissez point vos cœurs\FTNT{Ps. 95:8-11.}.
\VS{8}Car, si Josué les avait introduits dans le repos, jamais après cela il n'aurait parlé d'un autre jour.
\TextTitle{Entrer dans le repos de Dieu}
\VS{9}Il reste donc encore un repos réservé au peuple de Dieu.
\VS{10}Car celui qui est entré dans son repos, se repose aussi de ses œuvres, comme Dieu s'est reposé des siennes.
\VS{11}Efforçons-nous donc d'entrer dans ce repos-là, de peur que quelqu'un ne tombe en imitant une semblable désobéissance.
\VS{12}Car la Parole de Dieu est vivante et efficace, et plus pénétrante qu'une épée quelconque à deux tranchants, et atteignant jusqu'à la division de l'âme et de l'esprit, et des jointures et des mœlles~; et elle juge les pensées et les intentions du cœur.
\VS{13}Et il n'y a aucune créature qui soit cachée devant lui, mais toutes choses sont nues et entièrement découvertes aux yeux de celui devant lequel nous devons rendre compte.
\VS{14}Ainsi, puisque nous avons un Souverain Grand-Prêtre, Jésus, le Fils de Dieu, qui a traversé les cieux, tenons ferme notre profession.
\VS{15}Car nous n'avons pas un Grand-Prêtre qui ne puisse avoir compassion de nos infirmités~; mais, nous avons celui qui a été tenté comme nous en toutes choses, mais sans pécher.
\VS{16}Approchons donc avec assurance du trône de la grâce, afin d'obtenir miséricorde et de trouver grâce, pour être secourus dans le temps convenable.
\Chap{5}
\TextTitle{Le service du grand-prêtre}
\VerseOne{}Or tout grand-prêtre pris d'entre les hommes est établi pour les hommes dans les choses qui concernent Dieu, afin qu'il offre des dons et des sacrifices pour les péchés.
\VS{2}Etant capable d'avoir de l'indulgence pour les ignorants et les égarés, puisqu'il est aussi lui-même enveloppé d'infirmité.
\VS{3}Et à cause de cette infirmité, il doit offrir pour les péchés, non seulement pour le peuple, mais aussi pour lui-même.
\VS{4}Et nul ne s'attribue cet honneur, si ce n'est celui qui est appelé de Dieu, comme Aaron.
\TextTitle{Christ, Grand-Prêtre selon l'ordre de Melchisédek}
\VS{5}De même, aussi Christ ne s'est point glorifié lui-même d'être fait Grand-Prêtre, mais celui qui lui a dit~: C'est toi qui es mon Fils, je t'ai engendré aujourd'hui\FTNT{Ps. 2:7.}~!
\VS{6}Comme il dit encore ailleurs~: Tu es prêtre éternellement, selon l'ordre de Melchisédek\FTNT{Ps. 110:4.}.
\VS{7}C'est lui qui, pendant les jours de sa chair, a offert avec de grands cris et avec larmes des prières et des supplications à celui qui pouvait le sauver de la mort, et il a été exaucé à cause de sa piété.
\VS{8}Quoiqu'il soit le Fils de Dieu, il a pourtant appris l'obéissance par les choses qu'il a souffertes.
\VS{9}Après avoir été consacré, il est devenu l'auteur du salut éternel pour tous ceux qui lui obéissent,
\VS{10}étant appelé de Dieu à être Grand-Prêtre selon l'ordre de Melchisédek~;
\VS{11}de qui nous avons beaucoup de choses à dire, mais elles sont difficiles à expliquer, parce que vous êtes devenus lents à comprendre.
\TextTitle{Du lait à la nourriture solide\FTNTT{jusqu'à Hé. 6:12}}
\VS{12}En effet, tandis que vous devriez être maîtres depuis longtemps, vous avez encore besoin qu'on vous enseigne quels sont les premiers rudiments des oracles de Dieu, et vous êtes devenus tels, que vous avez encore besoin de lait et non d'une nourriture solide.
\VS{13}Or quiconque use de lait, ne sait point ce que c'est que la parole de la justice, parce qu'il est un enfant\FTNT{Le mot enfant dans ce passage vient du grec «~nepios~» qui signifie «~ignorant~».}.
\VS{14}Mais la viande solide est pour ceux qui sont déjà hommes faits, {c'est-à-dire}, pour ceux qui, pour y être habitués, ont les sens exercés à discerner le bien et le mal.
\Chap{6}
\TextTitle{Tendre à la perfection}
\VerseOne{}C'est pourquoi, laissant la parole qui n'enseigne que les premiers principes de Christ, tendons à la perfection, ne posant pas de nouveau le fondement de la repentance des œuvres mortes, et de la foi en Dieu,
\VS{2}de la doctrine des baptêmes, et de l'imposition des mains, et de la résurrection des morts, et du jugement éternel.
\VS{3}Et c'est ce que nous ferons, si Dieu le permet.
\VS{4}Or il est impossible que ceux qui ont été une fois illuminés, et qui ont goûté le don céleste, et qui ont été fait participants au Saint-Esprit,
\VS{5}qui ont goûté la bonne parole de Dieu, et les puissances du siècle à venir,
\VS{6}s'ils retombent, soient changés de nouveau par la repentance, vu que, quant à eux, ils crucifient de nouveau le Fils de Dieu, et l'exposent à l'opprobre.
\VS{7}Car la terre qui est abreuvée par la pluie qui tombe souvent sur elle, et qui produit des herbes propres à ceux par qui elle est labourée, reçoit la bénédiction de Dieu~;
\VS{8}mais, celle qui produit des épines et des chardons, est rejetée et proche de malédiction, et sa fin est d'être brûlée.
\VS{9}Mais nous sommes persuadés, quoique nous parlions ainsi, en ce qui vous concerne, mes bien-aimés, des choses meilleures et qui tiennent au salut.
\VS{10}Car Dieu n'est pas injuste, pour oublier votre œuvre, et le travail de la charité que vous avez témoigné pour son Nom, en ce que vous avez secouru les saints, et que vous les secourez encore.
\VS{11}Or nous souhaitons que chacun de vous montre jusqu'à la fin le même empressement pour la pleine certitude de l'espérance,
\VS{12}afin que vous ne vous relâchiez point, mais que vous imitiez ceux qui, par la foi et par la patience, héritent ce qui leur a été promis.
\TextTitle{Christ entré au-delà du voile}
\VS{13}Car, lorsque Dieu fit la promesse à Abraham, ne pouvant jurer par un plus grand, il jura par lui-même,
\VS{14}en disant~: Certainement, je te bénirai abondement et je te multiplierai merveilleusement\FTNT{Ge. 22:16-17.}.
\VS{15}Et ainsi, Abraham ayant attendu patiemment, obtint ce qui lui avait été promis.
\VS{16}Or les hommes jurent par celui qui est plus grand qu'eux, et le serment qu'ils font pour confirmer leur parole met fin à tous leurs différends.
\VS{17}C'est pourquoi Dieu, voulant faire mieux connaître aux héritiers de la promesse la fermeté immuable de sa résolution, il y a fait intervenir le serment,
\VS{18}afin que, par deux choses immuables, dans lesquelles il est impossible que Dieu mente, nous ayons une ferme consolation, nous qui avons notre refuge à obtenir l'espérance qui nous est proposée.
\VS{19}Laquelle nous tenons comme une ancre sûre et ferme de l'âme, et qui pénètre jusqu'au-delà du voile,
\VS{20}où Jésus est entré comme notre précurseur, ayant été fait Grand-Prêtre éternellement, selon l'ordre de Melchisédek\FTNT{Voir Ge. 14.}.
\Chap{7}
\TextTitle{Melchisédek, type de Christ\FTNTT{Ge. 14}}
\VerseOne{}En effet, ce Melchisédek était Roi de Salem et Prêtre du Dieu Très-Haut\FTNT{Ge. 14:18.}. Il alla au-devant d'Abraham lorsqu'il revenait de la défaite des rois, et il le bénit,
\VS{2}et auquel Abraham donna pour sa part la dîme de tout\FTNT{Ge. 14:20. Pour en savoir plus sur la dîme, voir les commentaires en De. 14:22, No. 18:21 et Mal. 3:10.}. Son nom signifie premièrement Roi de justice, et puis il a été Roi de Salem, c'est-à-dire, Roi de paix.
\VS{3}Il est sans père, sans mère, sans généalogie, n'ayant ni commencement de jours ni fin de vie, mais il est rendu semblable au Fils de Dieu. Il demeure Prêtre continuellement.
\TextTitle{La prêtrise de Melchisédek, supérieure à celle d'Aaron}
\VS{4}Considérez donc combien est grand celui à qui même Abraham, le patriarche, donna la dîme du butin.
\VS{5}Car, quant à ceux d'entre les fils de Lévi qui reçoivent la prêtrise, ils ont bien une ordonnance de dîmer le peuple selon la loi, c'est-à-dire, de dîmer leurs frères, bien qu'ils soient sortis des reins d'Abraham.
\VS{6}Mais celui qui n'était pas de la même famille qu'eux reçut d'Abraham la dîme, et bénit celui qui avait les promesses.
\VS{7}Or sans contredit, celui qui est le moindre est béni par celui qui est le plus grand.
\VS{8}Et ici, ce sont les hommes mortels qui prennent les dîmes~; mais là, c'est celui de qui il est rendu témoignage qu'il est vivant.
\VS{9}Et pour ainsi dire, Lévi même qui prend des dîmes, les a payées en Abraham~;
\VS{10}car il était encore dans les reins de son père, quand Melchisédek alla au-devant de lui.
\TextTitle{La prêtrise selon l'ordre d'Aaron n'a rien amené à la perfection}
\VS{11}Si donc la perfection s'était trouvée dans la prêtrise lévitique, (car c'est sous elle que le peuple a reçu la loi) quel besoin était-il après cela qu'un autre prêtre se lève selon l'ordre de Melchisédek, et qui ne soit point nommé selon l'ordre d'Aaron~?
\VS{12}Or la prêtrise étant changée, il est nécessaire qu'il y ait aussi un changement de loi.
\VS{13}Car, celui à l'égard duquel ces choses sont dites, appartient à une autre tribu, de laquelle nul n'a assisté à l'autel~;
\VS{14}car il est évident que notre Seigneur est descendu de la tribu de Juda\FTNT{Mt. 1:2.}, à l'égard de laquelle Moïse n'a rien dit de la prêtrise.
\VS{15}Et cela est encore plus incontestable, en ce qu'un autre prêtre, à la ressemblance de Melchisédek, est suscité~;
\VS{16}qui n'a point été fait prêtre selon la loi du commandement charnel, mais selon la puissance de la vie impérissable.
\VS{17}Car Dieu lui rend ce témoignage~: Tu es prêtre éternellement, selon l'ordre de Melchisédek.
\VS{18}Or il se fait une abolition du commandement qui a précédé, à cause de sa faiblesse, et parce qu'il ne pouvait point profiter.
\VS{19}Car la loi n'a rien amené à la perfection, mais ce qui a amené à la perfection, c'est ce qui a été introduit par-dessus, à savoir une meilleure espérance, par laquelle nous approchons de Dieu.
\VS{20}D'autant plus, même que cela n'a pas été sans serment,
\VS{21}car les Lévites sont devenus prêtres sans serment, mais celui-ci l'est devenu avec serment par celui qui lui a dit~: Le Seigneur l'a juré, et il ne s'en repentira pas\FTNT{Voir Ps. 110:4}~: Tu es prêtre éternellement, selon l'ordre de Melchisédek.
\VS{22}C'est donc d'une alliance d'autant plus excellente que Jésus a été fait le garant.
\TextTitle{Les prêtres sont mortels, seul Christ est éternel}
\VS{23}Et quant aux prêtres, il y en a eu plusieurs qui se sont succédés parce que la mort les empêchait d'être perpétuels.
\VS{24}Mais lui, parce qu'il demeure éternellement, possède une prêtrise qui n'est pas transmissible.
\VS{25}C'est pourquoi aussi il peut sauver parfaitement ceux qui s'approchent de Dieu par lui, étant toujours vivant pour intercéder\FTNT{Le Seigneur Jésus-Christ est le modèle parfait en ce qui concerne la prière d'intercession. Il se tient devant le Père pour nous. En tant qu'homme (1 Ti. 2:5) et Grand-Prêtre, il se tient entre le Père et l'homme pécheur, comme le faisaient les prêtres sous la loi mosaïque. Voir Lu. 22:31-32~; Ro. 8:34~; 1 Jn. 2:1-2.} pour eux.
\VS{26}Or il nous était convenable d'avoir un tel Grand-Prêtre, saint, innocent, sans tache, séparé des pécheurs, et élevé au-dessus des cieux,
\VS{27}qui n'avait pas besoin, comme les grands-prêtres, d'offrir tous les jours des sacrifices, premièrement pour ses péchés, et ensuite pour ceux du peuple, vu qu'il a fait cela une fois, s'étant offert lui-même.
\VS{28}Car, la loi établit grands-prêtres des hommes faibles~; mais la parole du serment qui a été fait après la loi, établit le Fils, qui est parfait pour toujours.
\Chap{8}
\TextTitle{L'ancienne prêtrise~: L'ombre des choses célestes}
\VerseOne{}La chose principale de notre discours, c'est que nous avons un tel Grand-Prêtre, qui est assis à la droite du trône de la majesté de Dieu dans les cieux,
\VS{2}serviteur du sanctuaire, et du véritable tabernacle, que le Seigneur a dressé et non pas les hommes.
\VS{3}Car tout grand-prêtre est établi pour offrir des offrandes et des sacrifices~; c'est pourquoi il est nécessaire que celui-ci ait aussi quelque chose à offrir.
\VS{4}Vu même que s'il était sur la terre, il ne serait pas prêtre, pendant qu'il y aurait encore des prêtres qui offrent les offrandes selon la loi~;
\VS{5}lesquels font le service dans le lieu qui n'est que l'image et l'ombre des choses célestes, selon que Dieu le dit à Moïse, quand il devait achever le tabernacle~: Or prends garde, lui dit-il, de faire toutes choses selon le modèle qui t'a été montré sur la montagne\FTNT{Ex. 25:40.}.
\TextTitle{Christ, le Médiateur d'une alliance plus excellente}
\VS{6}Mais maintenant, notre Grand-Prêtre a obtenu un service d'autant supérieur qu'il est le Médiateur d'une alliance plus excellente, qui a été établie sur de meilleures promesses.
\TextTitle{Les prophètes ont annoncé la Première Alliance}
\VS{7}En effet, si la Première Alliance avait été irréprochable, il n'y aurait pas eu lieu d'en chercher une seconde.
\VS{8}Car en censurant les Juifs, Dieu leur dit~: Voici, les jours viendront, dit le Seigneur, où je traiterai avec la maison d'Israël et avec la maison de Juda une Alliance Nouvelle,
\VS{9}non selon l'alliance que je traitai avec leurs pères, le jour où je les saisis par la main pour les tirer du pays d'Egypte~; car ils n'ont pas persévéré dans mon alliance, c'est pourquoi je les ai méprisés, dit le Seigneur.
\VS{10}Mais voici l'alliance que je traiterai, après ces jours-là, avec la maison d'Israël, dit le Seigneur~: Je mettrai mes lois dans leur esprit, et je les écrirai dans leur cœur, je serai leur Dieu, et ils seront mon peuple.
\VS{11}Personne n'enseignera plus son prochain, ni personne son frère, en disant~: Connais le Seigneur~! Parce que tous me connaîtront, depuis le plus petit jusqu'au plus grand d'entre eux~;
\VS{12}car je serai miséricordieux par rapport à leurs injustices, et je ne me souviendrai plus de leurs péchés, ni de leurs iniquités\FTNT{Jé. 31:31-34.}.
\VS{13}En disant une Nouvelle Alliance, il a déclaré vieille la première~; or, ce qui devient vieux et ancien, est près d'être aboli.
\Chap{9}
\TextTitle{Les ordonnances et le sanctuaire de la Première Alliance~: Des symboles}
\VerseOne{}En vérité, la Première Alliance avait aussi des ordonnances touchant le service divin, et un sanctuaire terrestre.\FTNT{Ex. 25:1-9.}.
\VS{2}Car il fut construit un premier tabernacle, appelé le lieu saint, dans lequel étaient le chandelier, et la table, et les pains de proposition\FTNT{Ex. 25:30.}.
\VS{3}Et après le second voile\FTNT{Ex. 26:31-35.} était le tabernacle, qui était appelé le Saint des saints,
\VS{4}ayant un encensoir d'or\FTNT{Encensoir ou autel d'or pour les parfums~: Lé. 16:12.}, et l'arche de l'alliance\FTNT{Ex. 25:10.}, entièrement couverte d'or tout autour, dans laquelle était le vase d'or\FTNT{Ex. 16:33.} où était la manne, et la verge d'Aaron\FTNT{No. 17:1-10.} qui avait fleuri, et les tables de l'alliance\FTNT{Les tables de l'alliance ou tables du témoignage~: Ex. 34:29~; De. 10:2-5.}.
\VS{5}Et au-dessus de l'arche étaient les chérubins de la gloire, couvrant de leur ombre le propitiatoire\FTNT{Propitiatoire ou couvercle de l'arche de l'alliance~: Lé. 9:7~; Lé. 16:15-17.}. Ce n'est pas le moment de parler en détail là-dessus.
\VS{6}Or ces choses étant ainsi disposées, les prêtres qui font le service entrent en tout temps dans le premier tabernacle\FTNT{No. 28:3.}~;
\VS{7}mais seul le grand-prêtre entre dans le second une fois par an, non sans y porter du sang, qu'il offre pour lui-même et pour les péchés du peuple\FTNT{Lé. 16:34.}.
\VS{8}Le Saint-Esprit faisant connaitre par là que le chemin du Saint des saints n'était pas encore manifesté, tandis que le premier tabernacle était encore debout,
\VS{9}lequel était une figure destinée pour le temps présent, durant lequel étaient offerts des offrandes et des sacrifices qui ne pouvaient point sanctifier la conscience de celui qui faisait le service,
\VS{10}ordonnés seulement en aliments, et en breuvages, en diverses ablutions, et en des cérémonies charnelles, jusqu'au temps de la réforme.
\TextTitle{La réalité du sacrifice s'accomplit en Christ}
\VS{11}Mais Christ est venu comme Grand-Prêtre des biens à venir~; il a traversé un tabernacle plus excellent et plus parfait, qui n'est pas un tabernacle construit de main d'homme, c'est-à-dire, qui n'est pas de cette création~;
\VS{12}et il est entré une fois pour toutes dans le Saint des saints, non avec le sang des veaux ou des boucs, mais avec son propre sang, après avoir obtenu une rédemption éternelle.
\VS{13}Car si le sang des taureaux et des boucs, et la cendre de la génisse\FTNT{No. 19:1-12.}, répandue sur ceux qui sont souillés, sanctifient et procurent la pureté de la chair,
\VS{14}combien plus le sang de Christ, qui, par l'Esprit éternel, s'est offert lui-même à Dieu sans nulle tache, purifiera-t-il votre conscience des œuvres mortes, pour servir le Dieu vivant~?
\VS{15}C'est pourquoi il est le Médiateur de la Nouvelle Alliance, afin que, la mort étant intervenue pour la rançon des transgressions commises sous la Première Alliance, ceux qui ont été appelés reçoivent l'héritage éternel qui leur a été promis.
\TextTitle{Les clauses du testament du Messie}
\VS{16}Car là où il y a un testament, il est nécessaire que la mort du testateur intervienne,
\VS{17}parce que c'est par la mort du testateur qu'un testament est rendu ferme, puisqu'il n'a aucune force tant que le testateur est en vie.
\VS{18}C'est pourquoi la Première Alliance elle-même n'a point été confirmée sans le sang.
\VS{19}Car Moïse, après avoir prononcé devant tout le peuple tous les commandements de la loi, prit le sang des veaux et des boucs, avec de l'eau, et de la laine écarlate, et de l'hysope~; et il en fit l'aspersion sur le livre et sur tout le peuple, en disant~:
\VS{20}Ceci est le sang de l'Alliance que Dieu vous a ordonné d'observer\FTNT{Ex. 24:3-8.}.
\VS{21}Puis il fit aussi aspersion avec du sang sur le tabernacle et sur tous les ustensiles du service\FTNT{Ex. 29:12~; Ex. 29:36.}.
\VS{22}Et presque toutes choses, selon la loi, sont purifiées par le sang, et sans effusion de sang il n'y a pas de rémission des péchés.
\TextTitle{Un sacrifice plus excellent\FTNTT{Lé. 16:33}}
\VS{23}Il a donc fallu que les choses qui représentaient celles qui sont aux cieux, soient purifiées par de telles choses, mais que les célestes le soient par des sacrifices plus excellents que ceux-là.
\VS{24}Car Christ n'est pas entré dans un sanctuaire fait de main d'homme, et qui n'était que la figure du véritable, mais il est entré dans le ciel même, afin de comparaître maintenant pour nous devant la face de Dieu.
\VS{25}Et ce n'est pas pour s'offrir lui-même plusieurs fois qu'il y est entré, ainsi que le grand-prêtre entre dans le Saint des saints, chaque année, avec un autre sang~;
\VS{26}autrement, il aurait fallu qu'il ait souffert plusieurs fois depuis la création du monde~; mais maintenant, à la fin des siècles, il a paru une seule fois pour l'abolition du péché par son sacrifice.
\VS{27}Et comme il est réservé aux hommes de mourir une seule fois\FTNT{Ce passage réfute la doctrine de la réincarnation.}, et après cela suit le jugement,
\VS{28}de même aussi Christ, qui s'est offert une seule fois pour ôter les péchés de plusieurs, apparaîtra sans péché une seconde fois à ceux qui l'attendent pour le salut.
\Chap{10}
\TextTitle{Le sacrifice unique de Christ est supérieur à tous les sacrifices}
\VerseOne{}Car la loi qui possède l'ombre des biens à venir, et non l'image exacte des choses, ne peut jamais, par les mêmes sacrifices que l'on offre continuellement chaque année, sanctifier ceux qui s'y attachent.
\VS{2}Autrement, n'auraient-ils pas cessé d'être offerts~? Parce que les adorateurs, une fois expurgés, n'auraient plus eu conscience des péchés.
\VS{3}Or le souvenir des péchés est réitéré dans ces sacrifices chaque année~;
\VS{4}car il est impossible que le sang des taureaux et des boucs ôte les péchés.
\VS{5}C'est pourquoi Jésus-Christ, en entrant dans le monde, a dit~: Tu n'as pas voulu de sacrifice, ni d'offrande, mais tu m'as formé un corps~;
\VS{6}tu n'as pas pris plaisir aux holocaustes, ni aux sacrifices pour le péché\FTNT{Ps. 40:7-9.}.
\VS{7}Alors j'ai dit~: Me voici, je viens, il est écrit de moi au commencement du livre~: Que je fasse, ô Dieu, ta volonté~!
\VS{8}Après avoir dit d'abord~: Tu n'as pas voulu de sacrifice, ni d'offrande, ni d'holocauste, ni d'offrande pour le péché et tu n'y as point pris plaisir, lesquelles choses sont pourtant offertes selon la loi, alors il dit~: Me voici, je viens afin de faire, ô Dieu, ta volonté~!
\VS{9}Il abolit ainsi le premier afin d'établir le second.
\VS{10}Or c'est par cette volonté que nous sommes sanctifiés, à savoir par l'offrande du corps de Jésus-Christ qui a été faite une fois pour toutes.
\VS{11}De plus, tout prêtre fait chaque jour le service et offre souvent les mêmes sacrifices, qui ne peuvent jamais ôter les péchés,
\VS{12}mais lui, après avoir offert un seul sacrifice pour les péchés, s'est assis pour toujours à la droite de Dieu,
\VS{13}attendant désormais que ses ennemis soient mis pour le marchepied de ses pieds.
\VS{14}Car, par une seule offrande, il a rendu parfaits pour toujours ceux qui sont sanctifiés.
\VS{15}Et c'est aussi ce que le Saint-Esprit nous témoigne~; car, après avoir dit premièrement~:
\VS{16}Voici l'alliance que je ferai avec eux, après ces jours-là, dit le Seigneur\FTNT{Voir Jé. 31:31-34.}~: C'est que je mettrai mes lois dans leur cœur, et je les écrirai dans leur esprit~;
\VS{17}et je ne me souviendrai plus de leurs péchés, ni de leurs iniquités.
\VS{18}Or, là où les péchés sont pardonnés, il n'y a plus d'offrande pour le péché.
\TextTitle{Exhortation à s'approcher de Dieu avec foi}
\VS{19}Ainsi donc, mes frères, nous avons la liberté d'entrer dans le Saint des saints au moyen du sang de Jésus,
\VS{20}qui est le chemin\FTNT{Jésus est le chemin qui conduit au Saint des saints, à la vie (Voir Jn. 14:6), et ce chemin n'était pas encore manifesté avant sa naissance. Hé. 9:8.} nouveau et vivant qu'il a inauguré pour nous à travers le voile, c'est-à-dire sa propre chair,
\VS{21}et ayant un Grand-Prêtre établi sur la maison de Dieu,
\VS{22}approchons-nous de lui avec un cœur sincère, et une foi inébranlable, ayant les cœurs purifiés d'une mauvaise conscience, et le corps lavé d'une eau pure.
\VS{23}Retenons fermement la profession de notre espérance, car celui qui nous a fait la promesse est fidèle.
\VS{24}Veillons les uns sur les autres pour nous exciter à la charité et aux bonnes œuvres.
\VS{25}N'abandonnons pas notre assemblée\FTNT{Assemblée~: Du grec «~episunagoge~» qui veut dire «~être assemblé en un lieu, assemblée religieuse des chrétiens~». Il est question de ne pas abandonner la communion fraternelle et non une église locale. En effet, il est du devoir du chrétien de se séparer des faux frères de peur d'être entraîné dans leur égarement (Mt. 18:15-17~; 1 Co. 5:11~; 1 Co. 15:33).}, comme c'est la coutume de quelques-uns~; mais exhortons-nous les uns les autres, et cela d'autant plus que vous voyez approcher le jour.
\TextTitle{Ne pas mépriser le sacrifice de Christ}
\VS{26}Car, si nous péchons volontairement après avoir reçu la connaissance de la vérité, il ne reste plus de sacrifice pour les péchés,
\VS{27}mais une attente terrible du jugement et l'ardeur d'un feu qui doit dévorer les adversaires.
\VS{28}Si quelqu'un avait méprisé la loi de Moïse, il mourait sans miséricorde, sur la déposition de deux ou de trois témoins\FTNT{De. 17:6.}~;
\VS{29}de combien pires tourments pensez-vous donc que sera jugé digne celui qui aura foulé aux pieds le Fils de Dieu, et qui aura tenu pour une chose profane le sang de l'Alliance, par lequel il avait été sanctifié, et qui aura outragé l'Esprit de grâce~?
\VS{30}Car nous connaissons celui qui a dit~: C'est à moi que la vengeance appartient, et je le rendrai~! Dit le Seigneur. Et encore~: Le Seigneur jugera son peuple.\FTNT{De. 32:35-36.}
\VS{31}C'est une chose terrible que de tomber entre les mains du Dieu vivant.
\VS{32}Or rappelez-vous des premiers jours, où, après avoir été éclairés, vous avez soutenu un grand combat de souffrances,
\VS{33}ayant été, d'une part, exposés à la vue de tout le monde par des opprobres et des afflictions, et de l'autre, ayant participé aux maux de ceux qui ont souffert de semblables indignités.
\VS{34}Car vous avez aussi été participants de l'affliction de mes liens, et vous avez reçu avec joie l'enlèvement de vos biens, sachant en vous-mêmes que vous avez dans les cieux des biens meilleurs et permanents.
\VS{35}N'abandonnez donc pas cette fermeté que vous avez fait paraître, et qui sera bien récompensée.
\VS{36}Parce que vous avez besoin de patience, afin qu'après avoir fait la volonté de Dieu, vous receviez l'effet de sa promesse.
\TextTitle{La marche par la foi~: Exemples d'hommes et de femmes de foi}
\VS{37}Car, encore un peu de temps, et celui qui doit venir, viendra, et il ne tardera point.
\VS{38}Or le juste vivra de la foi~; mais si quelqu'un se retire, mon âme ne prend point de plaisir en lui\FTNT{Ha. 2:4.}.
\VS{39}Mais pour nous, nous ne sommes pas de ceux qui se retirent~; ce serait notre perdition~; mais nous persévérons dans la foi, pour le salut de l'âme.
\Chap{11}
\VerseOne{}Or la foi rend présentes les choses qu'on espère, et elle est une démonstration de celles qu'on ne voit point.
\VS{2}Car c'est par elle que les anciens ont obtenu un bon témoignage.
\VS{3}Par la foi, nous comprenons que l'univers a été fait par la parole de Dieu, de sorte que les choses qui se voient, n'ont pas été faites des choses visibles.
\VS{4}Par la foi, Abel\FTNT{Ge. 4:3-5.} offrit à Dieu un sacrifice plus excellent que Caïn~; et par elle il obtînt le témoignage d'être juste, parce que Dieu rendait témoignage de ses offrandes~; et c'est par elle qu'il parle encore, quoique mort.
\VS{5}Par la foi, Hénoc\FTNT{Ge. 5:22-24.} fut enlevé pour ne pas voir la mort, et il ne parut plus parce que Dieu l'avait enlevé~; car, avant qu'il soit enlevé, il avait obtenu le témoignage d'avoir été agréable à Dieu.
\VS{6}Or il est impossible de lui être agréable sans la foi~; car il faut que celui qui vient à Dieu, croie que Dieu est, et qu'il est le rémunérateur de ceux qui le cherchent.
\VS{7}Par la foi, Noé\FTNT{Ge. 6:14-22.}, ayant été divinement averti des choses qui ne se voyaient point encore, craignit, et bâtit l'arche pour la conservation de sa famille~; et par cette arche, il condamna le monde, et devint héritier de la justice qui est selon la foi.
\VS{8}Par la foi, Abraham\FTNT{Ge 12:1-4.}, étant appelé, obéit, pour aller sur la terre, qu'il devait recevoir en héritage, et il partit sans savoir où il allait.
\VS{9}Par la foi, il demeura comme étranger sur la terre, qui lui avait été promise, comme si elle ne lui avait point appartenu, demeurant sous des tentes avec Isaac et Jacob, qui étaient héritiers avec lui de la même promesse.
\VS{10}Car il attendait la cité qui a des fondements, celle dont Dieu est l'architecte et le constructeur.
\VS{11}Par la foi, aussi Sara\FTNT{Ge. 21:1-2.} reçut la force de concevoir un enfant, et elle enfanta hors d'âge, parce qu'elle fut persuadée que celui qui le lui avait promis, était fidèle.
\VS{12}C'est pourquoi d'un seul homme, et qui était déjà affaibli, il est né une multitude aussi nombreuse que les étoiles du ciel, et que le sable du bord de la mer, qui ne peut se compter\FTNT{Ge. 22:17.}.
\VS{13}Tous ceux-ci sont morts dans la foi, sans avoir reçu les choses dont ils avaient eu les promesses, mais ils les ont vues de loin, crues, et saluées, et ils ont fait profession qu'ils étaient étrangers et voyageurs sur la terre\FTNT{1 Pi. 2:11.}.
\VS{14}Car ceux qui tiennent ces discours montrent clairement qu'ils cherchent encore leur patrie.
\VS{15}Et certes, s'ils avaient eu en vue celle d'où ils étaient sortis, ils auraient eu le temps d'y retourner.
\VS{16}Mais maintenant, ils en désirent une meilleure, c'est-à-dire une céleste. C'est pourquoi Dieu n'a pas honte d'être appelé leur Dieu, parce qu'il leur a préparé une cité\FTNT{Jn. 14:2~; Ap. 21:2.}.
\VS{17}Par la foi, Abraham étant éprouvé, offrit Isaac~; celui qui avait reçu les promesses offrit même son fils unique\FTNT{Ge. 22:1.},
\VS{18}à l'égard duquel il lui avait été dit~: Les descendants d'Isaac seront ta véritable postérité\FTNT{Ge. 21:12.}.
\VS{19}Ayant estimé que Dieu pouvait même le ressusciter d'entre les morts~; c'est pourquoi aussi il le recouvra par une espèce de résurrection.
\VS{20}Par la foi, Isaac bénit Jacob et Esaü, en vue des choses à venir\FTNT{Ge. 27:26-40.}.
\VS{21}Par la foi, Jacob, mourant, bénit chacun des fils de Joseph\FTNT{Ge. 48:1-22.}, et adora Dieu, appuyé sur l'extrémité de son bâton\FTNT{Ge. 47:31.}.
\VS{22}Par la foi, Joseph mourant fit mention de la sortie des enfants d'Israël, et il donna des ordres au sujet de ses os\FTNT{Ge. 50:24-25.}.
\VS{23}Par la foi, Moïse\FTNT{Ex. 2:1-3.}, à sa naissance, fut caché pendant trois mois par son père et sa mère, parce qu'ils virent que l'enfant était beau, et ils ne craignirent pas l'ordre du roi.
\VS{24}Par la foi, Moïse, devenu grand, refusa d'être nommé fils de la fille de Pharaon,
\VS{25}choisissant plutôt d'être affligé avec le peuple de Dieu, que de jouir pour un peu de temps des délices du péché.
\VS{26}Et ayant estimé que l'opprobre de Christ était un plus grand trésor que les richesses de l'Egypte, parce qu'il avait égard à la rémunération.
\VS{27}Par la foi, il quitta l'Egypte, sans craindre la fureur du roi~; car il demeura ferme, comme voyant celui qui est invisible.
\VS{28}Par la foi, il fit la Pâque et l'aspersion du sang, afin que le destructeur qui tuait les premiers-nés, ne touche pas aux premiers-nés des Israélites\FTNT{Ex. 12:1-51.}.
\VS{29}Par la foi, ils traversèrent la Mer Rouge, comme un lieu sec, ce que les Egyptiens essayèrent de tenter, ils furent engloutis dans les eaux\FTNT{Ex. 14:13-31.}.
\VS{30}Par la foi, les murs de Jéricho tombèrent, après qu'on en eut fait le tour pendant sept jours\FTNT{Jos. 6:1-20.}.
\VS{31}Par la foi, Rahab, la prostituée, ne périt pas avec les incrédules, parce qu'elle avait reçu les espions et les avait renvoyés en paix\FTNT{Jos. 2:1-21~; Jos. 6:23.}.
\VS{32}Et que dirai-je encore~? Car le temps me manquerait si je voulais parler de Gédéon\FTNT{Jg. 6:11.}, et de Barak\FTNT{Jg. 4:6.}, et de Samson\FTNT{Jg. 13:24.}, et de Jephté\FTNT{Jg. 11:1.}, et de David\FTNT{1 S. 16-17.}, et de Samuel\FTNT{1 S. et 2 S.}, et des prophètes,
\VS{33}qui par la foi combattirent des royaumes, exercèrent la justice, obtinrent des promesses, fermèrent la gueule des lions,
\VS{34}éteignirent la force du feu, échappèrent au tranchant des épées, des malades devinrent vigoureux, se montrèrent fort dans la bataille, et mirent en fuite des armées étrangères.
\VS{35}Des femmes recouvrèrent leurs morts par le moyen de la résurrection~; et d'autres furent livrés aux tourments et n'acceptèrent point d'être délivrés, afin d'obtenir une meilleure résurrection.
\VS{36}Et d'autres subirent les moqueries et le fouet, les chaînes et la prison~;
\VS{37}ils furent lapidés, sciés, subirent de rudes épreuves, ils furent mis à mort par le tranchant de l'épée, ils errèrent çà et là, vêtus de peaux de brebis et de chèvres, réduits à la misère, affligés, tourmentés,
\VS{38}eux dont le monde n'était pas digne, errant dans les déserts et dans les montagnes, et dans les cavernes et dans les trous de la terre.
\VS{39} Et quoiqu'ils aient tous été recommandables par leur foi, ils n'ont pourtant point reçu l'effet de la promesse,
\VS{40}Dieu ayant pourvu quelque chose de meilleur pour nous, en sorte qu'ils ne parviennent pas à la perfection sans nous.
\Chap{12}
\TextTitle{Fixer les regards sur Jésus}
\VerseOne{}Nous donc aussi, puisque nous sommes environnés d'une si grande nuée de témoins\FTNT{Témoin~: du grec «~martus~», terme qui dans un sens légal et historique signifie «~celui qui est spectateur d'une chose~». Dans un sens éthique, il est question de «~ceux qui ont prouvé la force et l'authenticité de leur foi en Christ en supportant une mort violente~». «~Martus~» a donné le mot «~martyr~» en français.}, rejetons tout fardeau, et le péché qui nous enveloppe si aisément, et poursuivons constamment la course qui nous est proposée,
\VS{2}portant les yeux sur Jésus, le chef et le consommateur de la foi qui en échange de la joie qui lui était réservée, il a souffert la croix, ayant méprisé la honte, et s'est assis à la droite du trône de Dieu.
\VS{3}C'est pourquoi, considérez soigneusement celui qui a supporté contre sa personne une telle opposition de la part des pécheurs, afin que vous ne succombiez point, en perdant courage.
\VS{4}Vous n'avez pas encore résisté jusqu'à répandre votre sang en combattant contre le péché.
\TextTitle{La correction du Père}
\VS{5}Et cependant vous avez oublié l'exhortation qui vous est adressée comme à ses fils, disant~: Mon fils, ne méprise pas le châtiment du Seigneur, et ne perds point courage lorsqu'il te reprend~;
\VS{6}car le Seigneur châtie celui qu'il aime, et il frappe de la verge tous ceux qu'il reconnaît pour ses fils\FTNT{Pr. 3:11-12.}.
\VS{7}Si vous endurez le châtiment, Dieu se présente à vous comme à ses fils~; car qui est le fils que le père ne châtie point~?
\VS{8}Mais si vous êtes sans châtiment auquel tous participent, vous êtes donc des enfants illégitimes, et non pas des fils.
\VS{9}Et puisque nos pères selon la chair nous ont châtiés, et que malgré cela nous les avons respectés, ne serons-nous pas beaucoup plus soumis au Père des esprits, pour avoir la vie~?
\VS{10}Car par rapport à ceux-là, ils nous châtiaient pour un peu de temps, suivant leur volonté, mais celui-ci nous châtie pour notre profit, afin que nous soyons participants de sa sainteté.
\VS{11}Or tout châtiment ne semble pas sur l'heure être un sujet de joie, mais de tristesse~; mais ensuite il produit un fruit paisible de justice à ceux qui sont exercés par ce moyen.
\VS{12}Fortifiez donc vos mains languissantes et vos genoux affaiblis~;
\VS{13}et suivez avec vos pieds des chemins droits, afin que ce qui est boiteux ne dévie pas, mais plutôt se consolide.
\VS{14}Recherchez la paix avec tous, et la sanctification, sans laquelle nul ne verra le Seigneur.
\TextTitle{Que nul ne se prive de la grâce de Dieu !}
\VS{15}Veillez à ce que personne ne se prive de la grâce de Dieu~; à ce qu'aucune racine d'amertume, poussant des rejetons, ne vous trouble, et que plusieurs n'en soient souillés par elles~;
\VS{16}que nul de vous ne soit fornicateur, ou profane comme Esaü, qui pour un aliment vendit son droit d'aînesse\FTNT{Ge. 25:33}.
\VS{17}Car vous savez que plus tard, désirant hériter la bénédiction, il fut rejeté, car il ne trouva point de lieu à la repentance, quoiqu'il l'ait demandée avec larmes.
\TextTitle{L'Eglise véritable s'est approchée de Sion}
\VS{18}Vous ne vous êtes pas approchés d'une montagne qu'on pouvait toucher avec la main\FTNT{Ex. 19:12.}, ni du feu brûlant, ni de la nuée épaisse, ni des ténèbres, ni de la tempête,
\VS{19}ni du retentissement de la trompette, ni du son des paroles, au sujet duquel ceux qui l'entendirent prièrent que la parole ne leur soit plus adressée\FTNT{Ex. 20:18-26.},
\VS{20}car ils ne pouvaient pas supporter ce qui était ordonné, que si même une bête touche la montagne, elle sera lapidée ou percée d'un dard\FTNT{Ex. 19:13.}.
\VS{21}Et ce spectacle était si terrible que Moïse dit~: Je suis épouvanté et tout tremblant~!
\VS{22}Mais vous vous êtes approchés de la montagne de Sion, de la Cité du Dieu vivant, la Jérusalem céleste, d'une multitude innombrable d'anges,
\VS{23}et de l'assemblée et de l'Eglise des premiers-nés qui sont inscrits dans les cieux, du Dieu qui est le juge de tous, et des esprits des justes qui ont été rendus parfaits,
\VS{24}de Jésus, qui est le Médiateur de la Nouvelle Alliance, et du sang de l'aspersion, qui prononce des meilleurs choses que celui d'Abel.
\TextTitle{Exhortation à la crainte de Dieu}
\VS{25}Prenez garde de ne pas mépriser celui qui vous parle~; car si ceux qui méprisèrent celui qui leur parlait sur la terre, n'ont pas échappé, nous serons punis beaucoup plus, si nous nous détournons de celui qui parle des cieux,
\VS{26}lui, dont la voix ébranla alors la terre, mais à l'égard du temps présent, il a fait cette promesse, disant~: J'ébranlerai encore une fois non seulement la terre, mais aussi le ciel\FTNT{Ag. 2:6.}.
\VS{27}Or ces mots~: Une fois encore, marquent le changement des choses ébranlées, comme étant faites pour un temps, afin que celles qui sont inébranlables demeurent.
\VS{28}C'est pourquoi, saisissant le Royaume qui ne peut point être ébranlé, retenons la grâce par laquelle nous servions Dieu, en sorte que nous lui soyons agréables avec respect et avec crainte,
\VS{29}car notre Dieu est aussi un feu dévorant\FTNT{De. 4:24.}.
\Chap{13}
\TextTitle{Exhortations~; invariabilité de Christ}
\VerseOne{}Que la charité fraternelle demeure dans vos cœurs.
\VS{2}N'oubliez pas l'hospitalité~; car, par elle, quelques-uns ont logé des anges sans le savoir.
\VS{3}Souvenez-vous des prisonniers, comme si vous étiez emprisonnés avec eux~; et de ceux qui sont maltraités, comme étant aussi vous-mêmes du même corps.
\VS{4}Le mariage est honorable entre tous, et le lit sans souillure~; mais Dieu jugera les fornicateurs et les adultères.
\VS{5}Que votre conduite soit sans avarice, étant contents de ce que vous avez présentement~; car lui-même a dit~: Je ne te délaisserai point, et je ne t'abandonnerai point\FTNT{De. 31:6.}.
\VS{6}De sorte que nous pouvons dire avec assurance~: Le Seigneur est mon aide, et je ne craindrai point ce que l'homme pourrait me faire\FTNT{Ps. 118:6.}.
\VS{7}Souvenez-vous de vos conducteurs qui vous ont annoncé la parole de Dieu~; considérez quelle a été la fin de leur vie, et imitez leur foi.
\VS{8}Jésus-Christ est le même hier, aujourd'hui, et il l'est aussi éternellement.
\VS{9}Ne soyez point emportés çà et là par des doctrines diverses et étrangères~; car il est bon que le cœur soit affermi par la grâce, et non point par les aliments, lesquelles n'ont en rien profité à ceux qui s'y sont attachés.
\TextTitle{Porter ses regards sur la cité céleste}
\VS{10}Nous avons un autel dont ceux qui servent dans le tabernacle n'ont pas le droit de manger.
\VS{11}Car les corps des animaux, dont le sang est porté dans le sanctuaire par le grand-prêtre pour le péché, sont brûlés hors du camp.
\VS{12}C'est pourquoi aussi Jésus, afin de sanctifier le peuple par son propre sang, a souffert hors de la porte\FTNT{Ex. 29:14. Jésus a souffert hors de Jérusalem (Jn. 19:17-18).}.
\VS{13}Sortons donc vers lui, hors du camp\FTNT{Le mot «~camp~» dans ce passage vient du grec «~parambole~», terme faisant référence au judaïsme antique dans lequel s'étaient embourbés les chrétiens d'origine hébraïque. Aujourd'hui, il représente plutôt le christianisme paganisé, essentiellement basé sur la loi de Moïse et constituant une prison qui empêche certains enfants de Dieu de vivre pleinement leur liberté en Christ.}, en portant son opprobre.
\VS{14}Car nous n'avons point ici-bas de cité permanente, mais nous recherchons celle qui est à venir.
\TextTitle{Le sacrifice de louange et du serviteur de Dieu}
\VS{15}Offrons donc par lui sans cesse à Dieu un sacrifice de louange, c'est-à-dire, le fruit des lèvres, en confessant son Nom.
\VS{16}Or n'oubliez pas la bienfaisance et de faire part de vos biens, car Dieu prend plaisir à de tels sacrifices.
\TextTitle{L'obéissance aux conducteurs}
\VS{17}Obéissez\FTNT{Le terme «~obéissez~», en grec «~peitho~», veut dire «~se laisser persuader par des mots~». Il signifie aussi «~donner avec persuasion l'envie à quelqu'un de faire quelque chose en le rassurant~». Par conséquent, les conducteurs doivent comprendre que la soumission et l'obéissance des chrétiens n'a rien à voir avec la dictature et l'autoritarisme. Ils doivent les rassurer et les convaincre - car tout ce qui n'est pas fait avec foi est péché (Ro. 14:23) - et ne pas tyranniser leurs frères en les obligeant à leur obéir (Mt. 20:25~; 1 Pi. 5:2-3).} à vos conducteurs, et soyez-leur soumis, car ils veillent pour vos âmes, comme devant en rendre compte~; afin que ce qu'ils en font, ils le fassent avec joie, et non en gémissant, car cela ne vous serait pas profitable.
\VS{18} Priez pour nous, car nous nous assurons que nous avons une bonne conscience, désirant nous conduire honnêtement parmi tous.
\VS{19}C'est avec instance que je vous demande de le faire, afin que je vous sois rendu plus tôt.
\TextTitle{Bénédictions et salutations}
\VS{20}Que le Dieu de paix, qui a ramené d'entre les morts le grand Pasteur des brebis, par le sang de l'Alliance éternelle, notre Seigneur Jésus-Christ,
\VS{21}vous rende capables de toute bonne œuvre pour faire sa volonté~; qu'il fasse en vous ce qui lui est agréable par Jésus-Christ~; auquel soit la gloire aux siècles des siècles~! Amen~!
\VS{22}Aussi, mes frères, je vous prie de supporter la parole d'exhortation, car je vous ai écrit en peu de mots.
\VS{23}Sachez que notre frère Timothée a été relâché~; s'il vient bientôt, je vous verrai avec lui.
\VS{24}Saluez tous vos conducteurs, et tous les saints. Ceux d'Italie vous saluent.
\VS{25}Que la grâce soit avec vous tous~! Amen~!
\PPE{}
\end{multicols}

%\clearpage\ShortTitle{1 Jean}\BookTitle{1 Jean}\BFont
\noindent\hrulefill
{\footnotesize
\textit{
\bigskip
{\centering{}
\\Auteur : Jean
\\Signification : Yahweh a fait grâce
\\Thème : La communion fraternelle, la connaissance et l'amour
\\Date de rédaction : Env. 85 ap. J.-C.\\}
}
%\bigskip
\textit{
\\Cette épître, écrite par Jean à Ephèse, était destinée aux églises de la province d’Asie qu’il connaissait bien. Il souhaite rendre leur joie parfaite en fortifiant leur foi en Christ et en leur donnant l’assurance de la vie éternelle ;  tout en les mettant en garde contre les faux docteurs.\bigskip
}
}
\par\nobreak\noindent\hrulefill
\begin{multicols}{2}
\Chap{1}
\TextTitle{La Parole incarnée}
\VerseOne{}Ce qui était dès le commencement, ce que nous avons entendu, ce que nous avons vu de nos propres yeux, ce que nous avons contemplé, et que nos propres mains ont touché concernant la Parole de vie,
\VS{2}car la vie a été manifestée, et nous l'avons vue et nous lui rendons témoignage, et nous vous annonçons la vie éternelle, qui était avec le Père, et qui nous a été manifestée.
\TextTitle{Communion avec le Père et le Fils}
\VS{3}Ce que nous avons vu dis-je, et ce que nous avons entendu, nous vous l'annonçons, afin que vous soyez en communion avec nous, et que notre communion soit avec le Père et avec son Fils Jésus-Christ.
\VS{4}Et nous vous écrivons ces choses, afin que votre joie soit parfaite.
\TextTitle{De la communion avec Dieu, qui est Lumière, et de la confession des péchés}
\TextTitle{[Conditions de la communion avec Dieu
\\a. Position de l'enfant de Dieu dans la Lumière]}
\VS{5}Or c'est ici la déclaration que nous avons entendue de lui et que nous vous annonçons, à savoir que Dieu est Lumière et qu'il n'y a point en lui de ténèbres.
\VS{6}Si nous disons que nous sommes en communion avec lui, et que nous marchions dans les ténèbres, nous mentons, et nous n'agissons pas selon la vérité.
\VS{7}Mais si nous marchons dans la Lumière, comme Dieu est dans la Lumière, nous sommes en communion les uns avec les autres, et le sang de son Fils Jésus-Christ nous purifie de tout péché.
\TextTitle{b. Reconnaissance de la présence du péché en nous}
\VS{8}Si nous disons que nous n'avons point de péché, nous nous séduisons nous-mêmes, et la vérité n'est point en nous.
\TextTitle{c. la confession des péchés, le pardon et la purification}
\VS{9}Si nous confessons nos péchés, il est fidèle et juste pour nous les pardonner, et pour nous purifier de toute iniquité.
\VS{10}Si nous disons que nous n'avons point de péché, nous le faisons menteur, et sa Parole n'est point en nous.
\TextTitle{Celui qui connaît Jésus-Christ garde ses commandements}
\Chap{2}
\TextTitle{d. Christ, notre avocat pour nos péchés}
\VerseOne{}Mes petits-enfants, je vous écris ces choses afin que vous ne péchiez point. Et si quelqu'un a péché, nous avons un avocat\FTNT{Jésus, notre Avocat. Le mot grec «~parakletos~», traduit ici par «~avocat~», se trouve également en Jean 14 et 16, où il est traduit par «~Consolateur~» et s’applique au Saint-Esprit. Le Seigneur exerce la fonction d'avocat actuellement pour nous dans le ciel. Voir Ro. 8:33 ; Hé. 7:25.} auprès du Père, Jésus-Christ, le Juste.
\VS{2}Car c'est lui qui est la victime de propitiation pour nos péchés, et non seulement pour les nôtres, mais aussi pour ceux de tout le monde.
\TextTitle{e. reconnaissance de la sainteté de Dieu}
\VS{3}Et nous savons que nous l'avons connu, si nous gardons ses commandements.
\VS{4}Celui qui dit : Je l'ai connu, et qui ne garde point ses commandements, est un menteur, et il n'y a point de vérité en lui.
\VS{5}Mais celui qui garde sa Parole, l'amour de Dieu est véritablement parfait en lui : et c'est par cela que nous savons que nous sommes en lui.
\VS{6}Celui qui dit qu'il demeure en lui doit aussi vivre comme Jésus-Christ lui-même a vécu.
\VS{7}Mes frères, je ne vous écris point un commandement nouveau, mais un commandement ancien, que vous avez eu dès le commencement ; et ce commandement ancien c'est la Parole que vous avez entendue dès le commencement.
\VS{8}Cependant, le commandement que je vous écris est un commandement nouveau, c’est une chose véritable en lui et en vous, parce que les ténèbres sont passées, et que la véritable Lumière paraît déjà.
\VS{9}Celui qui dit qu'il est dans la Lumière, et qui hait son frère, est dans les ténèbres jusqu'à présent.
\VS{10}Celui qui aime son frère demeure dans la Lumière, et il n'y a rien en lui qui puisse le faire tomber.
\VS{11}Mais celui qui hait son frère est dans les ténèbres, et il marche dans les ténèbres, et il ne sait pas où il va car les ténèbres ont aveuglé ses yeux.
\TextTitle{Exhortations à la famille spirituelle}
\VS{12}Mes petits enfants, je vous écris parce que vos péchés vous sont pardonnés à cause de son Nom.
\VS{13}Pères, je vous écris parce que vous avez connu celui qui est dès le commencement. Jeunes gens, je vous écris parce que vous avez vaincu l’esprit du malin.
\VS{14}Jeunes enfants, je vous écris parce que vous avez connu le Père. Pères, je vous ai écrit parce que vous avez connu celui qui est dès le commencement. Jeunes gens, je vous ai écrit parce que vous êtes forts et que la Parole de Dieu demeure en vous, et que vous avez vaincu l’esprit du malin.
\TextTitle{Les enfants de Dieu ne doivent pas aimer le monde}
\VS{15}N'aimez point le monde ni les choses qui sont dans le monde ; si quelqu'un aime le monde, l'amour du Père n'est point en lui.
\VS{16}Car tout ce qui est dans le monde, c'est-à-dire la convoitise de la chair, la convoitise des yeux et l'orgueil de la vie, ne vient point du Père, mais vient du monde.
\VS{17}Et le monde passe, avec sa convoitise ; mais celui qui fait la volonté de Dieu demeure éternellement.
\TextTitle{Les enfants de Dieu mis en garde contre les apostats}
\VS{18}Petits enfants, c'est ici la dernière\FTNT{Dernière, du grec «~eschatos~», signifie «~dernier dans une succession dans le temps~». Voir Ge. 49:1-2.} heure ; et comme vous avez entendu que l'Antéchrist viendra, il y a maintenant plusieurs antéchrists ; et par là nous connaissons que c'est la dernière heure.
\VS{19}Ils sont sortis du milieu de nous, mais ils n'étaient pas des nôtres ; car s'ils avaient été des nôtres, ils seraient demeurés avec nous, mais c'est afin qu'il soit manifeste que tous ne sont point des nôtres.
\VS{20}Mais vous avez été oints par le Saint-Esprit, et vous connaissez toutes choses.
\VS{21}Je ne vous ai pas écrit comme si vous ne connaissiez point la vérité, mais parce que vous la connaissez, et qu'aucun mensonge ne vient de la vérité.
\VS{22}Qui est le menteur, sinon celui qui nie que Jésus est le Christ ? Celui-là est l'Antéchrist qui nie le Père et le Fils.
\VS{23}Quiconque nie le Fils, n'a point non plus le Père ; quiconque confesse le Fils, a aussi le Père.
\VS{24}Que ce que vous avez entendu dès le commencement demeure en vous, car si ce que vous avez entendu dès le commencement demeure en vous, vous demeurerez aussi dans le Fils et dans le Père.
\VS{25}Et c'est ici la promesse qu'il nous a faite, à savoir la vie éternelle.
\VS{26}Je vous ai écrit ces choses au sujet de ceux qui vous séduisent.
\VS{27}Mais l'onction que vous avez reçue de lui demeure en vous, et vous n'avez pas besoin qu'on vous enseigne ; mais comme la même onction vous enseigne toutes choses, qu'elle est véritable et n'est pas un mensonge, demeurez en lui selon les enseignements qu’elle vous a donnés.
\TextTitle{Exhortations à deumeurer en Christ}
\VS{28}Maintenant donc, mes petits enfants, demeurez en lui ; afin que quand il apparaîtra, nous ayons de l'assurance, et que nous ne soyons point confus devant lui lors de son avènement\FTNT{Avènement, du grec «~parousia~», veut dire «~l’arrivée~» ou «~la présence~». Lors de cette seconde venue, le Messie prendra son Epouse pour les noces, ensuite il posera ses pieds sur le Mont des Oliviers, détruira les armées de l'Antichrist, puis commencera son règne de mille ans. Voir Za. 14.}.
\VS{29}Si vous savez qu'il est juste, sachez que quiconque fait ce qui est juste est né de lui.
\Chap{3}
\VerseOne{}Voyez quelle charité le Père nous a témoignée, pour que nous soyons appelés enfants de Dieu ! Mais le monde ne nous connaît point, parce qu'il ne l'a point connu.
\VS{2}Mes bien-aimés, nous sommes maintenant enfants de Dieu, et ce que nous serons n'est pas encore manifesté ; or nous savons que lorsque le Fils de Dieu apparaîtra, nous serons semblables à lui, car nous le verrons tel qu'il est.
\VS{3}Et quiconque a cette espérance en lui se purifie, comme lui aussi est pur.
\TextTitle{Caractéristiques des enfants de Dieu et des enfants du diable}
\VS{4}Quiconque pèche, transgresse la loi, car le péché est la transgression de la loi.
\VS{5}Or vous savez qu'il est apparu pour ôter nos péchés ; et il n'y a point de péché en lui.
\VS{6}Quiconque demeure en lui ne pèche point ; quiconque pèche, ne l'a pas vu, et ne l'a pas connu.
\VS{7}Mes petits-enfants, que personne ne vous séduise. Celui qui fait ce qui est juste est une personne juste, comme Jésus-Christ est juste.
\VS{8}Celui qui vit dans le péché est du diable, car le diable pèche dès le commencement. Or le Fils de Dieu est apparu afin de détruire les œuvres du diable.
\VS{9}Quiconque est né de Dieu ne vit pas dans le péché, car la semence de Dieu demeure en lui ; et il ne peut pécher, parce qu'il est né de Dieu.
\VS{10}Et c'est par là que nous connaissons les enfants de Dieu et les enfants du diable. Quiconque ne fait pas ce qui est juste et qui n'aime pas son frère n'est point de Dieu.
\VS{11}Car ce qui vous a été annoncé et ce que vous avez entendu dès le commencement c’est que nous nous aimions les uns les autres.
\VS{12}Et que nous ne soyons pas comme Caïn\FTNT{La doctrine de la semence du serpent est présentée par certains comme une explication du sens caché de la chute de l’homme dans le jardin en Eden et du péché originel. Selon cette doctrine, l’acte sexuel serait le fruit de l’arbre de la connaissance du bien et du mal. Cependant, cette doctrine n’est pas biblique, Eve n’a jamais eu de relations sexuelles avec le serpent. Dans Jean 8:44, lorsque Jésus dit aux pharisiens «~vous avez pour père le diable…~» suppose-t-il que le diable engendre des enfants physiquement ? Bien sûr que non !}, qui était de l'esprit malin et qui tua son frère. Et pourquoi le tua-t-il ? C’est parce que ses œuvres étaient mauvaises, et que celles de son frère étaient justes.
\VS{13}Mes frères, ne vous étonnez point si le monde vous hait.
\VS{14}Nous savons que nous sommes passés de la mort à la vie parce que nous aimons nos frères. Celui qui n'aime pas son frère demeure dans la mort.
\VS{15}Quiconque hait son frère est un meurtrier, et vous savez qu'aucun meurtrier ne possède la vie éternelle.
\VS{16}Nous avons connu la charité en ce qu'il a donné sa vie pour nous ; nous aussi, nous devons donner nos vies pour nos frères\FTNT{Jn. 15:13.}.
\VS{17}Si quelqu’un possède les biens du monde, et que voyant son frère dans la nécessité, il lui ferme ses entrailles, comment la charité de Dieu demeure-t-elle en lui ?
\VS{18}Mes petits-enfants, n'aimons pas en paroles et avec la langue, mais par des œuvres et en vérité.
\VS{19}Car c'est par là que nous connaissons que nous sommes de la vérité ; et nous rassurerons ainsi nos cœurs devant lui.
\VS{20}Si notre cœur nous condamne, certes Dieu est plus grand que notre cœur, et il connaît toutes choses.
\VS{21}Mes bien-aimés, si notre cœur ne nous condamne point, nous avons de l’assurance devant Dieu.
\VS{22}Et quoi que nous demandions, nous le recevons de lui, parce que nous gardons ses commandements, et que nous faisons les choses qui lui sont agréables.
\VS{23}Et c'est ici son commandement, que nous croyions au Nom de son Fils Jésus-Christ, et que nous nous aimions les uns les autres, selon le commandement qu’il nous a donné.
\VS{24}Celui qui garde ses commandements demeure en Jésus-Christ, et Jésus-Christ demeure en lui ; et par là nous connaissons qu'il demeure en nous, par l'Esprit qu'il nous a donné.
\Chap{4}
\TextTitle{Il faut éprouver les esprits}
\VerseOne{}Mes bien-aimés, ne croyez pas à tout esprit, mais éprouvez les esprits pour savoir s'ils sont de Dieu, car plusieurs faux prophètes sont venus dans le monde.
\TextTitle{[Caractéristiques des faux prophètes
\\a. leur confession sur Jésus-Christ]}
\VS{2}Reconnaissez à cette marque l'Esprit de Dieu : Tout esprit qui confesse que Jésus-Christ est venu en chair est de Dieu.
\VS{3}Et tout esprit qui ne confesse point que Jésus-Christ est venu en chair n'est point de Dieu ; c’est l'esprit de l'Antéchrist, dont vous avez appris la venue, et qui maintenant est déjà dans le monde.
\VS{4}Mes petits-enfants, vous êtes de Dieu, et vous les avez vaincus, parce que celui qui est en vous est plus grand que celui qui est dans le monde.
\TextTitle{b. leur appartenance au monde}
\VS{5}Eux, ils sont du monde, c'est pourquoi ils parlent comme étant du monde, et le monde les écoute.
\VS{6}Nous sommes de Dieu ; celui qui connaît Dieu nous écoute ; mais celui qui n'est pas de Dieu ne nous écoute point ; c’est par là que nous connaissons l'esprit de vérité et l'esprit de l’erreur.
\TextTitle{La charité de Dieu}
\VS{7}Mes bien-aimés, aimons-nous les uns les autres, car la charité est de Dieu ; et quiconque aime son prochain est né de Dieu et connaît Dieu.
\VS{8}Celui qui n'aime point son prochain n'a pas connu Dieu, car Dieu est Charité\FTNT{« Agapé » en grec.}.
\VS{9}La charité de Dieu a été manifestée envers nous en ce que Dieu a envoyé son Fils unique dans le monde, afin que nous vivions par lui.
\VS{10}Et cette charité consiste, non point en ce que nous avons aimé Dieu, mais en ce qu'il nous a aimés, et qu'il a envoyé son Fils pour être la propitiation\FTNT{Du grec «~hilasmos~» qui signifie «~apaisement~». Les écritures nous parlent aussi du «~propitiatoire~», c'est-à-dire « le siège de la misericorde » ou « Lieu de l'expiation ». Le propitiatoire était une plaque en or du sommet de l'Arche de l'Alliance. Le souverain sacrificateur l'aspergeait sept fois, le jour de l'expiation afin de reconcilier symboliquement Yahweh et son peuple. Voir Ex. 25:17-22} pour nos péchés.
\VS{11}Mes bien-aimés, si Dieu nous a ainsi aimés, nous devons aussi nous aimer les uns les autres.
\VS{12}Personne n'a jamais vu Dieu ; si nous nous aimons les uns les autres, Dieu demeure en nous et sa charité est parfaite en nous.
\VS{13}A ceci nous connaissons que nous demeurons en lui, et lui en nous, c'est qu'il nous a donné de son Esprit.
\VS{14}Et nous l'avons vu, et nous témoignons que le Père a envoyé le Fils pour être le sauveur du monde.
\VS{15}Quiconque confessera que Jésus est le Fils de Dieu, Dieu demeure en lui, et lui en Dieu.
\VS{16}Et nous, nous avons connu et cru en la charité que Dieu a pour nous. Dieu est charité ; et celui qui demeure dans la charité, demeure en Dieu, et Dieu en lui.
\VS{17}Tel il est, tels aussi nous sommes dans ce monde : C’est en cela que la charité est parfaite en nous, afin que nous ayons de l’assurance au jour du jugement.
\VS{18}Il n'y a point de crainte dans la charité, mais la parfaite charité bannit la crainte, car la crainte suppose un châtiment ; or celui qui craint n'est pas accompli dans la charité.
\VS{19}Nous l'aimons, parce qu'il nous a aimés le premier.
\VS{20}Si quelqu'un dit : J'aime Dieu, et qu’il haïsse son frère, c’est un menteur ; car comment celui qui n'aime point son frère, qu'il voit, peut-il aimer Dieu, qu’il ne voit pas ?
\VS{21}Et nous avons ce commandement de sa part, que celui qui aime Dieu, aime aussi son frère.
\Chap{5}
\TextTitle{La foi, principe qui triomphe des conflits avec le monde}
\VerseOne{}Quiconque croit que Jésus est le Christ, est né de Dieu, et quiconque aime celui qui l'a engendré, aime aussi celui qui est né de lui.
\VS{2}Nous connaissons à ceci que nous aimons les enfants de Dieu, lorsque nous aimons Dieu et que nous gardons ses commandements.
\VS{3}Car c'est en ceci que consiste notre amour pour Dieu : Que nous gardions ses commandements. Et ses commandements ne sont point pénibles.
\VS{4}Parce que tout ce qui est né de Dieu est victorieux du monde ; et ce qui nous fait remporter la victoire sur le monde, c'est notre foi.
\VS{5}Qui est celui qui a remporté la victoire sur le monde, sinon celui qui croit que Jésus est le Fils de Dieu ?
\VS{6}C'est ce Jésus, le Christ, qui est venu avec l’eau et le sang, et pas seulement avec l'eau, mais avec l'eau et le sang ; et c'est l'Esprit qui rend témoignage, or l'Esprit est la vérité.
\VS{7}Car il y en a trois dans le ciel qui rendent témoignage, le Père, la Parole, et le Saint-Esprit ; et ces trois-là ne sont qu'un\FTNT{Dieu est UN. Voir De. 6:4.}.
\VS{8}Il y en a aussi trois qui rendent témoignage sur la terre, à savoir l'Esprit, l'eau, et le sang, et ces trois-là se rapportent à un.
\TextTitle{Une assurance bénie}
\VS{9}Si nous recevons le témoignage des hommes, le témoignage de Dieu est plus grand, car le témoignage de Dieu consiste en ce qu’il a rendu témoignage à son Fils.
\VS{10}Celui qui croit au Fils de Dieu a le témoignage de Dieu en lui-même ; mais celui qui ne croit pas Dieu, le fait menteur, car il ne croit pas au témoignage que Dieu a rendu de son Fils.
\VS{11}Et c'est ici le témoignage, à savoir que Dieu nous a donné la vie éternelle, et cette vie est dans son Fils.
\VS{12}Celui qui a le Fils a la vie, celui qui n'a pas le Fils de Dieu n'a pas la vie.
\VS{13}Je vous ai écrit ces choses, à vous qui croyez au Nom du Fils de Dieu, afin que vous sachiez que vous avez la vie éternelle, et afin que vous croyiez au Nom du Fils de Dieu.
\VS{14}Et c'est ici l’assurance que nous avons en Dieu, que si nous demandons quelque chose selon sa volonté, il nous exauce.
\VS{15}Et si nous savons qu'il nous exauce, quelque chose que nous demandions, nous savons que nous possédons la chose que nous lui avons demandée.
\VS{16}Si quelqu'un voit son frère commettre un péché qui ne mène point à la mort\FTNT{Le péché qui mène à la mort c’est le blasphème contre le Saint-Esprit. Voir commentaire en Mt. 12:32.}, qu’il prie pour lui, et Dieu donnera la vie à ce frère. Il la donnera à ceux qui commettent un péché qui ne mène point à la mort. Il y a un péché qui mène à la mort ; je ne te dis point de prier pour ce péché-là.
\VS{17}Toute iniquité est un péché, mais il y a quelque péché qui ne mène pas à la mort.
\VS{18}Nous savons que quiconque est né de Dieu ne pèche point ; mais celui qui est engendré de Dieu se garde lui-même, et le malin ne le touche point.
\VS{19}Nous savons que nous sommes nés de Dieu, mais le monde entier est plongé dans le mal.
\TextTitle{Conclusion}
\VS{20}Or nous savons que le Fils de Dieu est venu, et il nous a donné l'intelligence pour connaître le Véritable ; et nous sommes dans le Véritable, en son Fils Jésus-Christ. Il est le vrai\FTNT{Dans Jean 17:3, le Père est présenté comme le Vrai Dieu ; le terme grec traduit par «~vrai~» dans Jean est aussi appliqué à Jésus dans ce passage. Jésus est donc le Vrai Dieu.} Dieu, et la vie éternelle.
\VS{21}Mes petits enfants, gardez-vous des idoles. Amen.
\PPE{}
\end{multicols}

%\clearpage\ShortTitle{2 Jean}\BookTitle{2 Jean}\BFont
\noindent\hrulefill
{\footnotesize
\textit{
\bigskip
{\centering{}
\\Auteur : Jean
\\(Gr. : Ioannes)
\\Signification : Yahweh a fait grâce
\\Thème : Amour et vérité
\\Date de rédaction : Env. 85 ap. J.-C.\\}
}
%\bigskip
\textit{
\\Il semblerait que cette épître était adressée à une église se réunissant chez une personne du nom de Kyria. Jean les invite à demeurer dans la communion avec Dieu et les met en garde contre les hérésies et la fréquentation des faux docteurs.\bigskip
}
}
\par\nobreak\noindent\hrulefill
\begin{multicols}{2}
\Chap{1}
\TextTitle{Introduction}
\VerseOne{}L'ancien à Kyria l’élue, et à ses enfants, que j'aime dans la vérité, et ce n’est pas moi seul qui les aime, mais aussi tous ceux qui ont connu la vérité.
\VS{2}A cause de la vérité qui demeure en nous, et qui sera avec nous éternellement,
\VS{3}que la grâce, la miséricorde, et la paix de la part de Dieu le Père, et de la part du Seigneur Jésus-Christ, le Fils du Père, soient avec vous dans la vérité et dans la charité.
\TextTitle{La marche dans la vérité et dans la charité}
\VS{4}Je me suis fort réjoui d'avoir trouvé quelques-uns de tes enfants qui marchent dans la vérité selon le commandement que nous avons reçu du Père.
\VS{5}Et maintenant, ô Kyria ! Je te prie, non comme t'écrivant un nouveau commandement, mais celui que nous avons eu dès le commencement, que nous ayons de la charité les uns pour les autres.
\VS{6}Et c'est ici la charité, que nous marchions selon ses commandements. Et c'est là son commandement, comme vous l'avez entendu dès le commencement, afin que vous l'observiez.
\TextTitle{Le signe du séducteur et de l'antéchrist}
\VS{7}Car plusieurs séducteurs sont venus dans le monde, qui ne confessent point que Jésus-Christ est venu en chair ; un tel homme est un séducteur et un Antéchrist.
\VS{8}Prenez garde à vous-mêmes, afin que vous ne perdiez point le fruit du travail que vous avez fait, mais que vous en receviez une pleine récompense.
\VS{9}Quiconque transgresse la doctrine de Jésus-Christ et ne lui demeure point fidèle n'a point Dieu ; celui qui demeure dans la doctrine de Christ a le Père et le Fils.
\VS{10}Si quelqu'un vient à vous, et qu'il n'apporte point cette doctrine, ne le recevez point dans votre maison, et ne le saluez pas ;
\VS{11}car celui qui le salue participe à ses mauvaises œuvres.
\TextTitle{Conclusion}
\VS{12}Quoique j’aie plusieurs choses à vous écrire, je n’ai pas voulu les écrire avec du papier et de l'encre, mais j'espère aller vers vous, et vous parler bouche à bouche, afin que notre joie soit parfaite.
\VS{13}Les enfants de ta sœur élue te saluent. Amen !
\PPE{}
\end{multicols}

%\clearpage\ShortTitle{3 Jean}\BookTitle{3 Jean}\BFont
\noindent\hrulefill
{\footnotesize
\textit{
\bigskip
{\centering{}
\\Signifie : Yahweh a fait grâce
\\Thème : Sincérité, hospitalité et caractère chrétien
\\Auteur : Jean
\\Date de rédaction : Env. 85\\}
}
%\bigskip
\textit{
\\Cette épître fut destinée à Gaïus, l’un des responsables d’une église d’Asie Mineure dont Jean loua la piété et la générosité. Il l’avertit de l’orgueil et des agissements de Diotrèphe qui étaient contraires à la Parole mais souligna le bon témoignage de Démétrius.\bigskip
}
}
\par\nobreak\noindent\hrulefill
\begin{multicols}{2}
\TextTitle{[Introduction]}
\Chap{1}
\VerseOne{}L'ancien à Gaïus, le bien-aimé, que j'aime dans la vérité.
\VS{2}Bien-aimé, je souhaite que tu prospères{\FTNT{La prospérité dont il est question dans ce passage n’a rien à voir avec l’Evangile de prospérité qui met l’accent sur la richesse matérielle. Le mot grec «~euodoo~» signifie «~concevoir~» un voyage prospère et diligent~», «~mener par une voie directe et facile~», «~prospérer~», «~être heureux~».}} en toutes choses, et que tu sois en bonne santé, comme ton âme est en prospérité.
\VS{3}Car j’ai été fort réjoui quand les frères sont venus et ont rendu témoignage de ta sincérité, et comment tu marches dans la vérité.
\VS{4}Je n'ai pas de plus grande joie que d’apprendre que mes enfants marchent dans la vérité.
\TextTitle{[L'hospitalité]}
\VS{5}Bien-aimé, tu agis fidèlement dans tout ce que tu fais envers les frères et envers les étrangers,
\VS{6}qui en présence de l'église ont rendu témoignage de ta charité. Et tu feras bien de les accompagner dignement, comme il est séant selon Dieu.
\VS{7}Car ils sont partis pour son Nom, ne prenant rien des gentils.
\VS{8}Nous devons donc recevoir de tels hommes, afin d’être ouvriers avec eux pour la vérité.
\TextTitle{[Les mauvais actes de Diotrèphe et son caractère dominateur]}
\VS{9}J'ai écrit à l'Eglise, mais Diotrèphe, qui aime être le premier parmi eux, ne nous reçoit point.
\VS{10}C'est pourquoi, si je viens, je rappellerai les actions qu'il commet, en tenant contre nous de mauvais discours ; et n'étant pas content de cela, non seulement il ne reçoit pas les frères, mais il empêche même ceux qui veulent les recevoir et les chasse de l'église.
\VS{11}Bien-aimé, n'imite point le mal, mais le bien. Celui qui fait le bien est de Dieu ; mais celui qui fait le mal n'a point vu Dieu.
\TextTitle{[Témoignage de Démétrius]}
\VS{12}Tous rendent témoignage à Démétrius, et la vérité même le lui rend, et nous aussi nous lui rendons témoignage, et vous savez que notre témoignage est véritable.
\TextTitle{[Conclusion]}
\VS{13}J'avais plusieurs choses à écrire, mais je ne veux pas t'écrire avec l'encre et avec la plume.
\VS{14}Mais j'espère te voir bientôt, et nous parlerons de bouche à bouche.
\VS{15}Que la paix soit avec toi ! Les amis te saluent. Salue les amis, chacun par son nom.
\PPE{}
\end{multicols}

%\clearpage\ShortTitle{Ap.}\BookTitle{Apocalypse}\BFont
\noindent\hrulefill
{\footnotesize
\textit{
\bigskip
{\centering{}
\\Auteur~: Jean
\\Thème~: L'aboutissement de toutes choses
\\(Gr.~: Apokalupsis)
\\Signification~: Mettre à nu, révélation d'une vérité, action de révéler
\\Date de rédaction~: Env. 95 ap. J.-C.\\}
}
\textit{
\\Le terme apocalypse, du grec «~apokalupsis~», évoque «~l'action de révéler ce qui était caché ou inconnu~». Ce mot a pour racine «~apokalupto~» qui signifie aussi «~découvrir, dévoiler ce qui est voilé ou recouvert~».
\\C'est à Patmos, île grecque de la mer Egée - où il s'exila en raison de la persécution de l'empereur Domitien (51 - 96 ap. J.C.) - que Jean reçut une révélation de Jésus-Christ ainsi qu'un message s'adressant aux «~sept églises~» qui constituaient certainement les villes de l'Asie Mineure où se trouvaient les principales concentrations de chrétiens. Si Ephèse figure dans les écrits de la Nouvelle Alliance et que Thyatire et Laodicée y sont brièvement mentionnées, les quatre autres églises - qu'on ne retrouve nulle part ailleurs dans les Ecritures - étaient sans doute le fruit du travail missionnaire de Paul. Les sept lettres s'adressent à l'ange de chacune de ces assemblées locales, autrement dit aux messagers de celles-ci (probablement un ancien ou un responsable).
\\Ce livre, qui arrive en conclusion des Ecritures, annonce les événements qui doivent précéder la fin de l'histoire de l'humanité.\bigskip
}
} 
\par\nobreak\noindent\hrulefill
\begin{multicols}{2}
\Chap{1}
\TextTitle{Introduction}
\VerseOne{}La révélation\FTNT{«~Apokalupsis~» en grec. Voir l'introduction du livre.} de Jésus-Christ, que Dieu lui a donnée pour montrer à ses serviteurs les choses qui doivent arriver bientôt, et qui les a fait connaître en les envoyant par son ange à Jean, son serviteur,
\VS{2}qui a annoncé la parole de Dieu, et le témoignage de Jésus-Christ, et toutes les choses qu'il a vues.
\VS{3}Béni est celui qui lit et ceux qui écoutent les paroles de cette prophétie, et qui gardent les choses qui y sont écrites~! Car le temps est proche.
\TextTitle{Jésus-Christ}
\VS{4}Jean aux sept églises qui sont en Asie~: Que la grâce et la paix vous soient données de la part de celui QUI EST, QUI ETAIT, et QUI VIENT\FTNT{Les prophètes ont prophétisé la venue de Yahweh en personne~: Es. 35:4~; Es. 40:10-11~; Es. 60:1-2~; Za. 14:1-21~; Jn. 14:1-3. Jésus-Christ est bien Yahweh qui vient.}, et de la part des sept Esprits qui sont devant son trône,
\VS{5}et de la part de Jésus-Christ, qui est le témoin fidèle, le premier-né d'entre les morts\FTNT{Voir commentaire Col. 1:15.}, et le Prince des rois de la terre.
\VS{6}A lui, dis-je, qui nous a aimés, et qui nous a lavés de nos péchés dans son sang, et qui a fait de nous des rois et des prêtres pour Dieu, son Père, à lui soient la gloire et la force aux siècles des siècles. Amen~!
\TextTitle{La venue du Christ}
\VS{7}Voici, il vient avec les nuées, et tout œil le verra, et même ceux qui l'ont percé~; et toutes les tribus de la terre se lamenteront devant lui. Oui, amen~!
\VS{8}Je suis l'Alpha et l'Oméga, le commencement et la fin, dit le Seigneur, QUI EST, QUI ETAIT, et QUI VIENT, le Tout-Puissant.
\TextTitle{La vision doit être écrite}
\VS{9}Moi Jean, qui suis aussi votre frère et qui participe à la tribulation, au règne, et à la patience de Jésus-Christ, j'étais sur l'île appelée Patmos à cause de la parole de Dieu, et du témoignage de Jésus-Christ.
\VS{10}Je fus ravi en esprit au jour du Seigneur, et j'entendis derrière moi une voix forte, comme le son d'une trompette,
\VS{11}qui disait~: Je suis l'Alpha et l'Oméga, le premier et le dernier. Ecris dans un livre ce que tu vois, et envoie-le aux sept églises qui sont en Asie, à savoir à Ephèse, à Smyrne, à Pergame, à Thyatire, à Sardes, à Philadelphie, et à Laodicée.
\VS{12}Alors je me retournai pour voir celui dont la voix m'avait parlé, et après m'être retourné, je vis sept chandeliers d'or,
\VS{13}et au milieu des sept chandeliers d'or, quelqu'un qui ressemblait à un fils d'homme, vêtu d'une longue robe, et ayant une ceinture d'or sur la poitrine.
\VS{14}Sa tête et ses cheveux étaient blancs comme de la laine blanche, et comme de la neige, et ses yeux étaient comme une flamme de feu.
\VS{15}Ses pieds étaient semblables à de l'airain ardent, comme s'ils avaient été embrasés dans une fournaise~; et sa voix était comme le bruit des grandes eaux.
\VS{16}Et il avait dans sa main droite sept étoiles, et de sa bouche sortait une épée aiguë à deux tranchants, et son visage était semblable au soleil lorsqu'il brille dans sa force.
\VS{17}Quand je le vis, je tombai à ses pieds comme mort, et il mit sa main droite sur moi, en me disant~: Ne crains pas~!
\VS{18}Je suis le premier et le dernier, et je vis~; j'étais mort, et voici, je suis vivant aux siècles des siècles. Amen~! Et je tiens les clefs de Hadès\FTNT{Voir commentaire Mt. 16:18.} et de la mort.
\VS{19}Ecris les choses que tu as vues, celles qui sont présentement, et celles qui doivent arriver ensuite.
\VS{20}Le mystère des sept étoiles que tu as vues dans ma main droite, et les sept chandeliers d'or. Les sept étoiles sont les anges des sept églises~; et les sept chandeliers que tu as vus sont les sept églises.
\Chap{2}
\TextTitle{Ephèse~: L'église qui a perdu le premier amour}
\VerseOne{}Ecris à l'ange\FTNT{Ange, du grec «~aggelos~»~: Envoyé, messager, un ange. Un messager de Dieu. Ce terme sert à désigner aussi bien les créatures spirituelles que les êtres humains.} de l'église d'Ephèse~: Voici ce que dit celui qui tient les sept étoiles dans sa main droite, et qui marche au milieu des sept chandeliers d'or~:
\VS{2}Je connais tes œuvres, et ton travail, et ta patience, et je sais que tu ne peux pas supporter les méchants, et que tu as éprouvé ceux qui se disent être apôtres et qui ne le sont pas, et que tu les as trouvés menteurs~;
\VS{3}et que tu as souffert, et que tu as eu de la patience, et que tu as travaillé pour mon Nom, et que tu ne t'es pas lassé.
\VS{4}Mais j'ai quelque chose contre toi, c'est que tu as abandonné ta première charité\FTNT{Il est question ici de l'amour «~agape~»~: L'amour fraternel, l'amour divin.}.
\VS{5}C'est pourquoi souviens-toi donc d'où tu es tombé, repens-toi, et fais tes premières œuvres. Autrement, je viendrai à toi à toute vitesse, et j'ôterai ton chandelier de sa place si tu ne te repens pas.
\VS{6}Mais pourtant tu as ceci de bon, c'est que tu hais les œuvres des Nicolaïtes\FTNT{Nicolaïtes~: Tiré du nom Nicolas, qui signifie littéralement «~victorieux du peuple~». Il s'agit d'une secte dont les membres furent peut-être des disciples d'un certain Nicolas, l'un des diacres de l'église d'Antioche qui aurait dévié (Ac. 6:5). Ces derniers suivaient la doctrine de Balaam, enseignant aux chrétiens qu'à cause du principe de liberté, ils pouvaient manger des viandes sacrifiées aux idoles et commettre des actes immoraux comme les Gentils.}, œuvres que je hais moi aussi.
\VS{7}Que celui qui a des oreilles entende ce que l'Esprit dit aux églises~! A celui qui vaincra, je lui donnerai à manger de l'arbre de vie, qui est au milieu du paradis de Dieu.
\TextTitle{Smyrne~: L'église sous la persécution}
\VS{8}Ecris aussi à l'ange de l'église de Smyrne~: Voici ce que dit celui qui est le premier et le dernier, qui a été mort, et qui est revenu à la vie~:
\VS{9}Je connais tes œuvres, ton affliction et ta pauvreté, quoique tu sois riche, et le blasphème de ceux qui se disent être Juifs et qui ne le sont pas, mais qui sont la synagogue de Satan.
\VS{10}Ne crains rien des choses que tu as à souffrir. Voici, il arrivera que le diable mettra quelques-uns d'entre vous en prison, afin que vous soyez éprouvés~; et vous aurez une affliction de dix jours. Sois fidèle jusqu'à la mort, et je te donnerai la couronne de vie.
\VS{11}Que celui qui a des oreilles, entende ce que l'Esprit dit aux églises~! Celui qui vaincra n'aura pas à souffrir la seconde mort.
\TextTitle{Pergame~: L'église établie dans le monde}
\VS{12}Ecris aussi à l'ange de l'église de Pergame~: Voici ce que dit celui qui a l'épée aiguë à deux tranchants\FTNT{Hé. 4:12.}~: 
\VS{13}Je connais tes œuvres, et le lieu où tu habites, à savoir là où est le trône de Satan. Et que cependant tu retiens mon Nom, et tu n'as pas renié ma foi, même aux jours d'Antipas, mon fidèle martyr\FTNT{Du grec «~martus~» qui signifie «~témoin~».}, qui a été mis à mort chez vous, là où Satan habite.
\VS{14}Mais j'ai quelque chose contre toi, c'est que tu as là des gens attachés à la doctrine de Balaam, qui enseignait à Balak à mettre un scandale devant les enfants d'Israël, afin qu'ils mangent des viandes sacrifiées aux idoles, et qu'ils se livrent à la fornication\FTNT{No. 25:1-2~; No. 31:16.}.
\VS{15}De même, toi aussi tu as des gens attachés à la doctrine des Nicolaïtes~; ce que je hais~!
\VS{16}Repens-toi donc, autrement je viendrai à toi à toute vitesse, et je les combattrai avec l'épée de ma bouche.
\VS{17}Que celui qui a des oreilles, entende ce que l'Esprit dit aux églises~! A celui qui vaincra, je lui donnerai à manger de la manne qui est cachée, et je lui donnerai un caillou blanc, et sur ce caillou sera écrit un nouveau nom, que nul ne connaît, sinon celui qui le reçoit.
\TextTitle{Thyatire~: L'église en temps d'idolâtrie}
\VS{18}Ecris aussi à l'ange de l'église de Thyatire~: Voici ce que dit le Fils de Dieu, qui a ses yeux comme une flamme de feu, et dont les pieds sont semblables à de l'airain ardent.
\VS{19}Je connais tes œuvres, ta charité, ton service, ta foi, ta patience, et que tes dernières œuvres surpassent les premières.
\VS{20}Mais j'ai quelque peu de chose contre toi, c'est que tu laisses cette femme Jézabel\FTNT{1 R. 16:31~; 1 R. 21:25~; 2 R. 9:7~; 2 R. 9:22.}, qui se dit prophétesse, enseigner et séduire mes serviteurs pour les porter à la fornication, et leur faire manger des choses sacrifiées aux idoles.
\VS{21}Et je lui ai donné du temps, afin qu'elle se repente de sa prostitution, mais elle ne s'est pas repentie.
\VS{22}Voici, je vais la jeter sur un lit, et mettre dans une grande affliction ceux qui commettent l'adultère avec elle, s'ils ne se repentent pas de leurs œuvres.
\VS{23}Et je ferai mourir de mort ses enfants~; et toutes les églises connaîtront que je suis celui qui sonde les reins et les cœurs, et je rendrai à chacun de vous selon ses œuvres.
\VS{24}Mais je vous dis à vous et aux autres qui sont à Thyatire, à tous ceux qui n'ont pas cette doctrine, et qui n'ont pas connu les profondeurs de Satan, comme ils disent, je vous dis~: Je ne mettrai pas sur vous d'autre charge.
\VS{25}Mais retenez ce que vous avez, jusqu'à ce que je vienne.
\VS{26}Car à celui qui aura vaincu, et qui aura gardé mes œuvres jusqu'à la fin, je lui donnerai autorité sur les nations.
\VS{27}Et il les gouvernera avec un sceptre de fer, et elles seront brisées comme les vases d'un potier, ainsi que j'en ai moi-même reçu le pouvoir de mon Père.
\VS{28}Et je lui donnerai l'étoile du matin.
\VS{29}Que celui qui a des oreilles entende ce que l'Esprit dit aux églises~!
\Chap{3}
\TextTitle{Sardes~: L'église morte}
\VerseOne{}Ecris aussi à l'ange de l'église de Sardes~: Voici ce que dit celui qui a les sept Esprits de Dieu, et les sept étoiles~: Je connais tes œuvres. Tu as la réputation d'être vivant, mais tu es mort.
\VS{2}Sois vigilant, et affermis le reste qui va mourir~; car je n'ai pas trouvé tes œuvres parfaites devant Dieu.
\VS{3}Souviens-toi donc des choses que tu as reçues et entendues, garde-les, et repens-toi. Si tu ne veilles pas, je viendrai contre toi comme un voleur, et tu ne sauras pas à quelle heure je viendrai contre toi\FTNT{Mt. 24:43~; Lu. 12:39~; 1 Th. 5:2~; 2 Pi. 3:10.}.
\VS{4}Toutefois, tu as quelque peu de personnes à Sardes qui n'ont pas souillé leurs vêtements, et qui marcheront avec moi en vêtements blancs, car ils en sont dignes.
\VS{5}Celui qui vaincra sera vêtu de vêtements blancs, et je n'effacerai pas son nom du Livre de vie, mais je confesserai son nom devant mon Père, et devant ses anges.
\VS{6}Que celui qui a des oreilles, entende ce que l'Esprit dit aux églises~!
\TextTitle{Philadelphie~: L'église réveillée et fidèle}
\VS{7}Ecris aussi à l'ange de l'église de Philadelphie~: Voici ce que dit le Saint et le Véritable, qui a la clef de David, qui ouvre et nul ne ferme, qui ferme et nul n'ouvre.
\VS{8}Je connais tes œuvres. Voici, j'ai ouvert une porte devant toi, et personne ne peut la fermer~; parce que tu as peu de puissance, que tu as gardé ma parole, et que tu n'as pas renié mon Nom.
\VS{9}Voici, je ferai venir ceux de la synagogue de Satan qui se disent Juifs, et ne le sont pas, mais qui mentent~; voici, dis-je, je les ferai venir et se prosterner à tes pieds, et ils connaîtront que je t'aime.
\VS{10}Parce que tu as gardé la parole de ma persévérance, je te garderai aussi de l'heure de la tentation qui doit arriver dans le monde entier, pour éprouver les habitants de la terre.
\VS{11}Voici, je viens à toute vitesse. Tiens ferme ce que tu as, afin que personne ne t'enlève ta couronne.
\VS{12}Celui qui vaincra, je ferai de lui une colonne dans le temple de mon Dieu, et il n'en sortira plus~; et j'écrirai sur lui le Nom de mon Dieu, et le nom de la cité de mon Dieu, qui est la nouvelle Jérusalem qui descend du ciel d'auprès de mon Dieu, et mon nouveau Nom.
\VS{13}Que celui qui a des oreilles entende ce que l'Esprit dit aux églises~!
\TextTitle{Laodicée~: L'église apostate}
\VS{14}Ecris aussi à l'ange de l'église de Laodicée~: Voici ce que dit l'Amen, le témoin fidèle et véritable, le commencement de la création de Dieu~:
\VS{15}Je connais tes œuvres. Je sais que tu n'es ni froid ni bouillant~; puisses-tu être ou froid ou bouillant~!
\VS{16}Parce que tu es tiède, et que tu n'es ni froid ni bouillant, je te vomirai de ma bouche.
\VS{17}Car tu dis~: Je suis riche, je suis dans l'abondance, et je n'ai besoin de rien~; mais tu ne sais pas que tu es malheureux, misérable, pauvre, aveugle et nu.
\VS{18}Je te conseille d'acheter de moi de l'or éprouvé par le feu, afin que tu deviennes riche; et des vêtements blancs, afin que tu sois vêtu et que la honte de ta nudité ne paraisse pas~; et d'oindre tes yeux de collyre, afin que tu voies.
\VS{19}Moi, je reprends et je châtie\FTNT{De. 8:5~; 2 S. 7:14~; Pr. 13:24~; Hé. 12:7.} tous ceux que j'aime. Aie donc du zèle et repens-toi.
\TextTitle{Le Messie se retrouve hors des églises apostates}
\VS{20}Voici, je me tiens à la porte, et je frappe. Si quelqu'un entend ma voix et m'ouvre la porte, j'entrerai chez lui, et je souperai avec lui, et lui avec moi.
\VS{21}Celui qui vaincra, je le ferai asseoir avec moi sur mon trône, ainsi que j'ai vaincu et me suis assis avec mon Père sur son trône.
\VS{22}Que celui qui a des oreilles entende ce que l'Esprit dit aux églises~!
\Chap{4}
\TextTitle{Vision avant l'ouverture des sceaux}
\VerseOne{}Après ces choses, je regardai, et voici une porte était ouverte dans le ciel. Et la première voix que j'avais entendue, comme le son d'une trompette, et qui parlait avec moi, me dit~: Monte ici, et je te montrerai les choses qui doivent arriver à l'avenir.
\VS{2}Aussitôt, je fus ravi en esprit. Et voici, un trône était dressé dans le ciel, et sur ce trône, quelqu'un était assis.
\VS{3}Et celui qui y était assis était semblable à une pierre de jaspe et de sardoine~; et le trône était environné d'un arc-en-ciel semblable à de l'émeraude.
\TextTitle{Les trônes des vingt-quatre anciens}
\VS{4}Et il y avait autour du trône vingt-quatre trônes et je vis sur ces trônes vingt-quatre anciens assis, vêtus de vêtements blancs, et ayant sur leurs têtes des couronnes d'or.
\VS{5}Et du trône sortaient des éclairs, des tonnerres, et des voix~; et il y avait devant le trône sept lampes de feu ardentes, qui sont les sept Esprits de Dieu.
\TextTitle{Le Messie est digne de recevoir la louange et la gloire}
\VS{6}Et devant le trône, il y avait une mer de verre semblable à du cristal~; et au milieu du trône et autour du trône quatre animaux, pleins d'yeux devant et derrière.
\VS{7}Et le premier animal était semblable à un lion~; le second animal était semblable à un veau~; le troisième animal avait la face comme un homme~; et le quatrième animal était semblable à un aigle qui vole.
\VS{8}Et les quatre animaux avaient chacun six ailes, et tout autour et au-dedans ils étaient pleins d'yeux~; et ils ne cessent pas de dire jour et nuit~: Saint~! Saint~! Saint est le Seigneur Dieu Tout-puissant, QUI ETAIT, QUI EST, et QUI VIENT.
\VS{9}Et quand ces animaux rendaient gloire et honneur et des actions de grâces à celui qui était assis sur le trône, à celui qui est vivant aux siècles des siècles,
\VS{10}les vingt-quatre anciens se prosternaient devant celui qui était assis sur le trône, et adoraient celui qui est vivant aux siècles des siècles, et ils jetaient leurs couronnes devant le trône, en disant~:
\VS{11}Seigneur, tu es digne de recevoir gloire, honneur et puissance~; car tu as créé toutes choses, et c'est par ta volonté qu'elles existent et qu'elles ont été créées.
\Chap{5}
\TextTitle{Le Messie est le seul digne d'ouvrir le livre}
\VerseOne{}Puis je vis dans la main droite de celui qui était assis sur le trône, un livre écrit en dedans et en dehors, scellé de sept sceaux.
\VS{2}Et je vis aussi un ange remarquable par sa force, qui proclamait d'une voix forte~: Qui est digne d'ouvrir le livre, et d'en rompre les sceaux~?
\VS{3}Et il n'y avait personne, ni dans le ciel, ni sur la terre, ni sous la terre qui pouvait ouvrir le livre, ni le regarder.
\VS{4}Et je pleurais beaucoup parce que personne n'était trouvé digne d'ouvrir le livre, ni de le lire, ni de le regarder.
\VS{5}Et l'un des anciens me dit~: Ne pleure pas, voici le Lion qui vient de la tribu de Juda, de la racine de David, a vaincu pour ouvrir le livre et pour en rompre les sept sceaux.
\VS{6}Et je regardai, et voici il y avait au milieu du trône et des quatre animaux, et au milieu des anciens, un Agneau qui se tenait là comme immolé, ayant sept cornes, et sept yeux, qui sont les sept Esprits de Dieu envoyés par toute la terre.
\VS{7}Et il vint et prit le livre de la main droite de celui qui était assis sur le trône.
\TextTitle{L'Agneau est adoré\FTNTT{Ph. 2:9-11.}}
\VS{8}Et quand il eut pris le livre, les quatre animaux et les vingt-quatre anciens se prosternèrent devant l'Agneau, ayant chacun des harpes et des coupes d'or pleines de parfums, qui sont les prières des saints.
\VS{9}Et ils chantaient un cantique nouveau, en disant~: Tu es digne de prendre le livre, et d'en ouvrir les sceaux~; car tu as été mis à mort, et tu nous as rachetés pour Dieu par ton sang, de toute tribu, de toute langue, de tout peuple, et de toute nation~;
\VS{10}et tu as fait de nous des rois et des prêtres pour notre Dieu~; et nous régnerons sur la terre.
\VS{11}Puis je regardai, et j'entendis la voix de plusieurs anges autour du trône, et des anciens; et leur nombre était de plusieurs millions.
\VS{12}Et ils disaient à haute voix~: L'Agneau qui a été mis à mort est digne de recevoir puissance, richesses, sagesse, force, honneur, gloire et louange.
\VS{13}J'entendis aussi toutes les créatures qui sont dans le ciel, sur la terre, et sous la terre, et dans la mer, et toutes les choses qui y sont, disant~: A celui qui est assis sur le trône et à l'Agneau, soient louange, honneur, gloire, et force, aux siècles des siècles~!
\VS{14}Et les quatre animaux disaient~: Amen~! Et les vingt-quatre anciens se prosternèrent et adorèrent celui qui est vivant aux siècles des siècles.
\Chap{6}
\TextTitle{Premier sceau~: Le cavalier qui part pour vaincre}
\VerseOne{}Et quand l'Agneau eut ouvert l'un des sceaux, je regardai, et j'entendis l'un des quatre animaux qui disait comme avec une voix de tonnerre~: Viens, et vois.
\VS{2}Je regardai, et je vis un cheval blanc~; celui qui était monté dessus avait un arc, et il lui fut donné une couronne~; et il est sortit en vainqueur pour vaincre\FTNT{Contrairement aux apparences, ce cavalier couronné d'un diadème et qui monte un cheval blanc n'est pas Jésus-Christ, mais l'Antichrist qui singe le retour glorieux du Seigneur~: Da. 7:21~; Mt. 24:4-5~; 2 Th. 2:9-12~; Ap. 13:7. Le vrai Christ revenant triomphalement avec son Eglise est décrit en Ap. 19:11-16.}.
\TextTitle{Deuxième sceau~: La guerre}
\VS{3}Et quand il eut ouvert le second sceau, j'entendis le second animal qui disait~: Viens, et vois.
\VS{4}Et il sortit un autre cheval qui était roux~; il fut donné à celui qui était monté dessus de pouvoir ôter la paix de la terre, afin que les hommes se tuent les uns les autres~; et il lui fut donné une grande épée.
\TextTitle{Troisième sceau~: La famine}
\VS{5}Et quand il eut ouvert le troisième sceau, j'entendis le troisième animal qui disait~: Viens, et vois. Je regardai, et je vis un cheval noir, et celui qui était monté dessus avait une balance dans sa main.
\VS{6}Et j'entendis au milieu des quatre animaux une voix qui disait~: Une mesure de blé pour un denier, et les trois mesures d'orge pour un denier~; mais ne fais pas de mal au vin et à l'huile.
\TextTitle{Quatrième sceau~: La mort}
\VS{7}Et quand il eut ouvert le quatrième sceau, j'entendis la voix du quatrième animal qui disait~: Viens, et vois.
\VS{8}Je regardai, et je vis un cheval verdâtre~; et celui qui était monté dessus se nommait la Mort, et le Hadès l'accompagnait. Il leur fut donné le pouvoir sur le quart de la terre pour tuer par l'épée, par la famine, par la mortalité, et par les bêtes sauvages de la terre.
\TextTitle{Cinquième sceau~: Les martyrs}
\VS{9}Et quand il eut ouvert le cinquième sceau, je vis sous l'autel les âmes de ceux qui avaient été tués pour la parole de Dieu, et pour le témoignage qu'ils avaient gardé.
\VS{10}Et elles criaient à haute voix, disant~: Jusqu'à quand, Seigneur qui es saint et véritable, ne jugeras-tu pas et ne vengeras-tu pas notre sang de ceux qui habitent sur la terre~?
\VS{11}Et il leur fut donné à chacun des robes blanches, et il leur fut dit de se tenir en repos encore un peu de temps, jusqu'à ce que le nombre de leurs compagnons de service, et de leurs frères qui doivent être mis à mort comme eux, soit complet.
\TextTitle{Sixième sceau~: L'anarchie}
\VS{12}Et je regardai quand il eut ouvert le sixième sceau, et voici, il se fit un grand tremblement de terre, et le soleil devint noir comme un sac de crin, et la lune entière devint comme du sang.
\VS{13}Et les étoiles du ciel tombèrent sur la terre\FTNT{Mt. 24:29~; Mc. 13:25.}, comme lorsque le figuier est agité par un grand vent et laisse tomber ses figues encore vertes.
\VS{14}Et le ciel se retira comme un livre qu'on roule~; et toutes les montagnes et les îles furent remuées de leurs places.
\VS{15}Et les rois de la terre, et les princes, et les riches, et les capitaines, et les puissants, et tout esclave, et tout homme libre se cachèrent dans les cavernes et entre les rochers des montagnes.
\VS{16}Et ils disaient aux montagnes et aux rochers~: Tombez sur nous\FTNT{Lu. 23:30.}, et cachez-nous devant la face de celui qui est assis sur le trône, et devant la colère de l'Agneau~;
\VS{17}car le grand jour de sa colère est venu, et qui peut subsister~?
\Chap{7}
\TextTitle{Les 144 000 marqués du sceau de Dieu}
\VerseOne{}Après cela, je vis quatre anges qui se tenaient aux quatre coins de la terre, et qui retenaient les quatre vents de la terre, afin qu'ils ne soufflent pas sur la terre, ni sur la mer, ni sur aucun arbre.
\VS{2}Puis je vis un autre ange qui montait du côté de l'orient, tenant le sceau du Dieu vivant, et il cria d'une voix forte aux quatre anges à qui il avait été donné de faire du mal à la terre et à la mer,
\VS{3}et leur dit~: Ne faites pas de mal à la terre, ni à la mer, ni aux arbres, jusqu'à ce que nous ayons marqué du sceau les serviteurs de notre Dieu sur leurs fronts.
\VS{4}Et j'entendis que le nombre de ceux qui avaient été marqués du sceau était de cent quarante-quatre mille, de toutes les tribus des enfants d'Israël.
\VS{5}de la tribu de Juda, douze mille marqués du sceau~; de la tribu de Ruben, douze mille marqués du sceau~; de la tribu de Gad, douze mille marqués du sceau~;
\VS{6}de la tribu d'Aser, douze mille marqués du sceau~; de la tribu de Nephthali, douze mille marqués du sceau~; de la tribu de Manassé, douze mille marqués du sceau~;
\VS{7}de la tribu de Siméon, douze mille marqués du sceau~; de la tribu de Lévi, douze mille marqués du sceau~; de la tribu d'Issacar, douze mille marqués du sceau~;
\VS{8}de la tribu de Zabulon, douze mille marqués du sceau~; de la tribu de Joseph, douze mille marqués du sceau~; de la tribu de Benjamin, douze mille marqués du sceau.
\TextTitle{Multitude de sauvés pendant la grande tribulation}
\VS{9}Après cela, je regardai, et voici une grande multitude de gens, que personne ne pouvait compter, de toute nation, de toute tribu, de tout peuple et de toute langue, se tenaient devant le trône, et devant l'Agneau, vêtus de longues robes blanches, et ils avaient des palmes dans leurs mains.
\VS{10}Et ils criaient d'une voix forte, en disant~: Le salut est à notre Dieu, qui est assis sur le trône, et à l'Agneau.
\VS{11}Et tous les anges se tenaient autour du trône, et des anciens, et des quatre animaux, et ils se prosternèrent devant le trône sur leurs faces et adorèrent Dieu,
\VS{12}en disant~: Amen~! La louange, la gloire, la sagesse, les actions de grâces, l'honneur, la puissance et la force soient à notre Dieu, aux siècles des siècles. Amen~!
\VS{13}Et l'un des anciens prit la parole et me dit~: Ceux qui sont revêtus de longues robes blanches, qui sont-ils et d'où sont-ils venus~?
\VS{14}Et je lui dis~: Seigneur, tu le sais. Et il me dit~: Ce sont ceux qui sont venus de la grande tribulation\FTNT{Les saints ont toujours été persécutés. Cela a débuté dès la Genèse avec Caïn qui tua son frère Abel (Ge. 4:5-10). La grande tribulation correspond néanmoins à une période de persécutions particulièrement cruelles qui seront orchestrées par l'homme impie (à la tête de plusieurs nations) principalement contre les Juifs (Jé. 30:7~; Da. 9:24~; Lu. 21:20-24) et sans doute contre les personnes converties à Christ issues des nations (Ap. 7:9-17~; Ap. 12:17). Le Seigneur Jésus a prédit la grande tribulation à ses disciples (Mt. 24:15-29~; Mc. 13:14-19) en précisant qu'en ce temps là on verrait «~l'abomination de la désolation~» établie en lieu saint prophétisée par Daniel (Da. 11:31). La grande tribulation durera trois ans et demi, c'est ce que Daniel appelle «~un temps, des temps et la moitié d'un temps~» (Da. 7:25~; Ap. 11:3.) L'ère de paix factice instaurée par l'impie cédera alors soudainement la place à un temps d'angoisse sans précédent (1 Th. 5:3).}, et qui ont lavé et blanchi leurs longues robes dans le sang de l'Agneau.
\VS{15}C'est pourquoi ils sont devant le trône de Dieu, et ils le servent jour et nuit dans son temple~; et celui qui est assis sur le trône habitera avec eux.
\VS{16}Ils n'auront plus faim ni soif, et le soleil ne les frappera plus, ni aucune chaleur.
\VS{17}Car l'Agneau qui est au milieu du trône les paîtra, et les conduira aux sources des eaux de la vie, et Dieu essuiera toutes les larmes de leurs yeux.
\Chap{8}
\TextTitle{Septième sceau~: Annonce des sept trompettes\FTNTT{Ap. 4:1.}}
\VerseOne{}Et quand il eut ouvert le septième sceau, il y eut un silence dans le ciel d'environ une demi-heure.
\VS{2}Et je vis les sept anges qui se tiennent devant Dieu, et sept trompettes leur furent données.
\VS{3}Et un autre ange vint et se tint devant l'autel, ayant un encensoir d'or, et plusieurs parfums lui furent donnés pour les offrir, avec les prières de tous les saints, sur l'autel d'or qui est devant le trône.
\VS{4}Et la fumée des parfums monta avec les prières des saints de la main de l'ange devant Dieu.
\VS{5}Puis l'ange prit l'encensoir, et l'ayant rempli du feu de l'autel, il le jeta sur la terre~; et il y eut des tonnerres, des voix, des éclairs, et un tremblement de terre.
\VS{6}Alors les sept anges qui avaient les sept trompettes se préparèrent à en sonner.
\TextTitle{Première trompette~: Grêle et feu mêlés de sang}
\VS{7}Et le premier ange sonna de la trompette. Et il y eut de la grêle et du feu mêlés de sang, qui furent jetés sur la terre~; et le tiers des arbres fut brûlé, et toute herbe verte aussi fut brûlée.
\TextTitle{Deuxième trompette~: La montagne embrasée}
\VS{8}Et le second ange sonna de la trompette, et je vis comme une grande montagne embrasée de feu, qui fut jetée dans la mer~; et le tiers de la mer devint du sang,
\VS{9}et le tiers des créatures vivantes qui étaient dans la mer mourut, et le tiers des navires périt.
\TextTitle{Troisième trompette~: Absinthe, l'étoile tombée du ciel}
\VS{10}Et le troisième ange sonna de la trompette, et il tomba du ciel une grande étoile ardente comme un flambeau, et elle tomba sur le tiers des fleuves et sur les sources des eaux.
\VS{11}Le nom de l'étoile est Absinthe~; et le tiers des eaux fut changé en absinthe, et beaucoup d'hommes moururent par les eaux, parce qu'elles étaient devenues amères.
\TextTitle{Quatrième trompette~: Des signes dans le ciel}
\VS{12}Puis le quatrième ange sonna de la trompette, et le tiers du soleil fut frappé, ainsi que le tiers de la lune, et le tiers des étoiles, afin que le tiers en soit obscurci~; le jour fut privé d'un tiers de sa clarté, et la nuit de même.
\VS{13}Je regardai, et j'entendis un ange qui volait au milieu du ciel et qui disait à haute voix~: Malheur~! Malheur~! Malheur aux habitants de la terre à cause des autres sons de trompettes que les trois autres anges vont faire retentir.
\Chap{9}
\TextTitle{Cinquième trompette~: Ouverture du puits de l'abîme}
\VerseOne{}Le cinquième ange sonna de la trompette, et je vis une étoile qui tomba du ciel sur la terre, et la clef du puits de l'abîme fut donnée à cet ange.
\VS{2}Et il ouvrit le puits de l'abîme, et une fumée monta du puits comme la fumée d'une grande fournaise~; et le soleil et l'air furent obscurcis par la fumée du puits.
\VS{3}Des sauterelles sortirent de la fumée du puits et se répandirent sur la terre, et il leur fut donné un pouvoir comme le pouvoir qu'ont les scorpions de la terre.
\VS{4}Et il leur fut dit de ne pas faire de mal à l'herbe de la terre, ni à aucune verdure, ni à aucun arbre, mais seulement aux hommes qui n'avaient pas la marque de Dieu sur leurs fronts.
\VS{5}Et il leur fut donné, non de les tuer, mais de les tourmenter pendant cinq mois~; et le tourment qu'elles causaient était comme le tourment que cause le scorpion quand il pique un homme.
\VS{6}Et en ces jours-là, les hommes chercheront la mort, mais ils ne la trouveront pas~; et ils désireront mourir, mais la mort fuira loin d'eux.
\VS{7}Ces sauterelles ressemblaient à des chevaux préparés pour la guerre, et sur leurs têtes il y avait comme des couronnes semblables à de l'or, et leurs faces étaient comme des faces d'hommes.
\VS{8}Elles avaient les cheveux comme des cheveux de femmes~; et leurs dents étaient comme des dents de lions.
\VS{9}Elles avaient des cuirasses comme des cuirasses de fer~; et le bruit de leurs ailes était comme le bruit des chars à plusieurs chevaux qui courent à la guerre.
\VS{10}Elles avaient des queues armées d'aiguillons, comme les scorpions, et c'est dans leurs queues qu'était le pouvoir de faire du mal aux hommes pendant cinq mois.
\VS{11}Elles avaient sur elles comme roi l'ange de l'abîme, dont le nom en hébreu est Abaddon, mais en grec son nom est Apollyon\FTNT{Abaddon ou Apollyon~: Le nom de ce démon signifie «~Le destructeur~».}.
\VS{12}Le premier malheur est passé, et voici venir encore deux malheurs après celui-ci.
\TextTitle{Sixième trompette~: Les quatre anges de l'Euphrate déliés\FTNTT{Ap. 16:12.}}
\VS{13}Alors le sixième ange sonna de sa trompette, et j'entendis une voix sortant des quatre cornes de l'autel d'or qui est devant Dieu,
\VS{14}et disant au sixième ange qui avait la trompette~: Délie les quatre anges qui sont liés sur le grand fleuve, l'Euphrate.
\VS{15}On délia donc les quatre anges qui étaient prêts pour l'heure, le jour, le mois et l'année, afin de tuer le tiers des hommes.
\VS{16}Le nombre des cavaliers de l'armée était de deux cents millions, car j'en entendis le nombre.
\VS{17}Et je vis aussi dans la vision les chevaux et ceux qui étaient montés dessus, ayant des cuirasses de feu, d'hyacinthe et de soufre~; et les têtes des chevaux étaient comme des têtes de lions~; et de leurs bouches sortaient du feu, de la fumée et du soufre.
\VS{18}Le tiers des hommes fut tué par ces trois fléaux, par le feu, et par la fumée et par le soufre qui sortaient de leur bouche.
\VS{19}Car le pouvoir des chevaux était dans leurs bouches et dans leurs queues~; et leurs queues étaient semblables à des serpents ayant des têtes, et c'est avec elles qu'ils faisaient du mal.
\VS{20}Mais les autres hommes qui ne furent pas tués par ces fléaux, ne se repentirent pas des œuvres de leurs mains, ils ne cessèrent pas d'adorer les démons, les idoles d'or, d'argent, de cuivre, de pierre, et de bois, qui ne peuvent ni voir, ni entendre, ni marcher.
\VS{21}Et ils ne se repentirent pas aussi de leurs meurtres, ni de leurs enchantements, ni de leur impudicité, ni de leurs vols.
\Chap{10}
\TextTitle{Un ange puissant descend du ciel}
\VerseOne{}Je vis un autre ange puissant qui descendait du ciel, environné d'une nuée, au-dessus de sa tête était l'arc-en-ciel, son visage était comme le soleil, et ses pieds comme des colonnes de feu.
\VS{2}Et il avait dans sa main un petit livre ouvert, et il posa son pied droit sur la mer, et le pied gauche sur la terre~;
\VS{3}et il cria d'une voix forte, comme lorsqu'un lion rugit. Et quand il eut crié, les sept tonnerres firent entendre leurs voix.
\VS{4}Et après que les sept tonnerres eurent fait entendre leurs voix, j'allais écrire, mais j'entendis une voix du ciel qui me disait~: Scelle les choses que les sept tonnerres ont fait entendre, et ne les écris pas.
\VS{5}Et l'ange que j'avais vu se tenant sur la mer et sur la terre, leva sa main vers le ciel,
\VS{6}et jura par celui qui est vivant aux siècles des siècles, qui a créé le ciel avec les choses qui y sont, et la terre avec les choses qui y sont, et la mer avec les choses qui y sont, qu'il n'y aurait plus de temps~;
\VS{7}mais qu'aux jours de la voix du septième ange, quand il commencera à sonner de la trompette, le mystère de Dieu sera accompli, comme il l'a déclaré à ses serviteurs les prophètes.
\TextTitle{Nouvelle mission de Jean}
\VS{8}Et la voix que j'avais entendue du ciel me parla encore et me dit~: Va, et prends le petit livre ouvert qui est dans la main de l'ange qui se tient sur la mer et sur la terre.
\VS{9}Et j'allai vers l'ange, en lui disant~: Donne-moi le petit livre~; et il me dit~: Prends-le et mange-le~; il remplira tes entrailles d'amertume, mais il sera doux dans ta bouche comme du miel\FTNT{Ez. 3:1-3.}.
\VS{10}Je pris donc le petit livre de la main de l'ange, et je le mangeai~; il fut doux dans ma bouche comme du miel, mais quand je l'eus mangé, mes entrailles furent remplies d'amertume.
\VS{11}Alors il me dit~: Il faut que tu prophétises de nouveau sur beaucoup de peuples, et sur plusieurs nations, sur plusieurs langues et plusieurs rois.
\Chap{11}
\TextTitle{Le temps des nations}
\VerseOne{}On me donna un roseau semblable à une verge, et l'ange se présenta et me dit~: Lève-toi et mesure le temple de Dieu et l'autel, et ceux qui y adorent.
\VS{2}Mais laisse de côté le parvis extérieur du temple, et ne le mesure pas~; car il est donné aux Gentils, et ils fouleront aux pieds la ville sainte pendant quarante-deux mois\FTNT{C'est le temps que durera la grande tribulation, soit trois ans et demi. Daniel parle d'une semaine, un jour comptant pour une année (Da. 9:27). La grande tribulation débutera à la moitié de cette semaine, ce qui correspond bien à quarante-deux mois (Ap. 13:5) et à mille deux cent soixante jours (Ap. 11:3~; Ap. 12:6).}.
\TextTitle{Les deux témoins ressuscitent}
\VS{3}Mais je donnerai à mes deux témoins de prophétiser pendant mille deux cent soixante jours, revêtus de sacs.
\VS{4}Ce sont les deux oliviers\FTNT{Za. 4:14.} et les deux chandeliers qui se tiennent devant le Dieu de la terre. 
\VS{5}Et si quelqu'un veut leur faire du mal, du feu sort de leurs bouches et dévore leurs ennemis~; car si quelqu'un veut leur faire du mal, il faut qu'il soit tué de cette manière.
\VS{6}Ils ont le pouvoir de fermer le ciel, afin qu'il ne pleuve pas pendant les jours de leur prophétie~; ils ont aussi le pouvoir de changer les eaux en sang, et de frapper la terre de toutes sortes de plaies, toutes les fois qu'ils le voudront.
\VS{7}Et quand ils auront achevé de rendre leur témoignage, la bête qui monte de l'abîme\FTNT{L'homme impie, l'Antichrist, ou encore le fils de la perdition dont il est question dans 2 Th. 2:3~; 2 Th. 2:8-9.} leur fera la guerre, les vaincra, et les tuera.
\VS{8}Et leurs cadavres seront étendus sur les places de la grande ville, qui est appelée spirituellement Sodome et Egypte, où aussi notre Seigneur a été crucifié.
\VS{9}Et ceux des tribus, des peuples, des langues, et des nations verront leurs cadavres pendant trois jours et demi, et ils ne permettront pas que leurs cadavres soient mis dans des sépulcres.
\VS{10}Et les habitants de la terre se réjouiront, ils seront dans l'allégresse, ils s'enverront des présents les uns aux autres, parce que ces deux prophètes ont tourmenté les habitants de la terre.
\VS{11}Mais après ces trois jours et demi, l'Esprit de vie venant de Dieu entra en eux, et ils se tinrent sur leurs pieds, et une grande crainte saisit ceux qui les virent.
\VS{12}Après cela, ils entendirent une forte voix du ciel, leur disant~: Montez ici~! Et ils montèrent au ciel sur une nuée, et leurs ennemis les virent.
\VS{13}Et à cette même heure-là, il eut un grand tremblement de terre, et la dixième partie de la ville tomba, et sept mille hommes furent tués par ce tremblement de terre~; et les autres furent épouvantés et donnèrent gloire au Dieu du ciel.
\VS{14}Le second malheur est passé. Voici, le troisième malheur vient bientôt.
\TextTitle{Septième trompette~: Le règne du Messie annoncé, cantique des vingt-quatre vieillards\FTNTT{Ap. 8:2.}}
\VS{15}Le septième ange sonna de la trompette, et il se fit entendre au ciel de grandes voix qui disaient~: Les royaumes du monde sont soumis à notre Seigneur et à son Christ, et il régnera aux siècles des siècles.
\VS{16}Alors les vingt-quatre anciens qui étaient assis devant Dieu sur leurs trônes, se prosternèrent sur leurs faces et adorèrent Dieu,
\VS{17}en disant~: Nous te rendons grâces, Seigneur Dieu Tout-Puissant, QUI ES, QUI ETAIS, et QUI VIENS, de ce que tu as fait éclater ta grande puissance, et de ce que tu as agi en Roi.
\VS{18}Les nations se sont irritées, mais ta colère est venue, et le temps est venu de juger les morts, et de donner la récompense à tes serviteurs les prophètes et aux saints, et à ceux qui craignent ton Nom, petits et grands, et de détruire ceux qui corrompent la terre.
\VS{19}Et le temple de Dieu fut ouvert dans le ciel, et l'arche de son alliance apparut dans son temple. Et il y eut des éclairs, des voix, des tonnerres, un tremblement de terre, et une grosse grêle.
\Chap{12}
\TextTitle{Vision de la femme et du dragon}
\VerseOne{}Et un grand signe parut dans le ciel~: Une femme revêtue du soleil, la lune sous ses pieds, et sur sa tête une couronne de douze étoiles\FTNT{En Ge. 37:9-10, Joseph raconte à ses parents et à ses frères un songe particulier où il voyait le soleil, la lune et onze étoiles se prosterner devant lui. Jacob comprit que les onze étoiles représentaient ses enfants, la lune sa femme Rachel, qui était la mère de Joseph, et que le soleil c'était lui-même. Il est donc question ici d'Israël, qui a toujours été identifié à une femme (Ez. 16) de qui est issu le Messie selon la chair (Ro. 9:5).}.
\VS{2}Elle était enceinte, et elle criait, étant en travail d'enfant, souffrant les grandes douleurs de l'enfantement.
\VS{3}Il parut aussi un autre signe dans le ciel, et voici un grand dragon rouge feu ayant sept têtes et dix cornes, et sur ses têtes sept diadèmes.
\VS{4}Sa queue entraînait le tiers des étoiles du ciel et les jeta sur la terre\FTNT{Da. 8:10.}. Puis le dragon s'arrêta devant la femme qui devait accoucher, afin de dévorer son enfant\FTNT{Cet enfant est évidemment Jésus-Christ (Mt. 2:16.)}, dès qu'elle l'aurait mis au monde.
\TextTitle{La naissance du Messie}
\VS{5}Et elle accoucha d'un fils, qui doit gouverner toutes les nations avec un sceptre de fer\FTNT{Ps. 2:8-9.}. Et son enfant fut enlevé vers Dieu et vers son trône\FTNT{Lu. 24:51~; Ac. 1:9-11.}.
\VS{6}Et la femme s'enfuit dans un désert, où elle avait un lieu préparé par Dieu, afin d'y être nourrie pendant mille deux cent soixante jours.
\TextTitle{Guerre entre l'archange Michel et le dragon}
\VS{7}Et il y eut une guerre dans le ciel. Michel et ses anges combattirent contre le dragon. Et le dragon et ses anges combattirent contre Michel,
\VS{8}mais ils ne furent pas les plus forts, et leur place ne fut plus trouvée dans le ciel.
\VS{9}Et il fut précipité le grand dragon, le serpent ancien, appelé le diable et Satan, celui qui séduit toute la terre, il fut précipité sur la terre, et ses anges furent précipités avec lui\FTNT{Es. 14:12-15~; Ez. 28~; Lu. 10:18.}.
\VS{10}Et j'entendis une voix forte dans le ciel qui disait~: Maintenant le salut est arrivé, ainsi que la force, le règne de notre Dieu, et la puissance de son Christ~; car l'accusateur de nos frères, qui les accusait devant notre Dieu jour et nuit, a été précipité.
\VS{11}Et ils l'ont vaincu à cause du sang de l'Agneau, et à cause de la parole de leur témoignage, et ils n'ont pas aimé leurs vies, mais les ont exposées à la mort.
\VS{12}C'est pourquoi réjouissez-vous cieux, et vous qui y habitez. Mais malheur à vous habitants de la terre et de la mer~! Car le diable est descendu vers vous animé d'une grande fureur, sachant qu'il a peu de temps.
\TextTitle{Le dragon persécute la femme, sa postérité et les témoins du Messie}
\VS{13}Quand le dragon vit qu'il avait été précipité sur la terre, il persécuta la femme qui avait enfanté le fils.
\VS{14}Mais deux ailes d'un grand aigle furent données à la femme, afin qu'elle s'envole de devant le serpent au désert, où elle est nourrie un temps, des temps, et la moitié d'un temps.
\VS{15}Et de sa gueule, le serpent lança de l'eau comme un fleuve derrière la femme, afin de l'entraîner par le fleuve.
\VS{16}Mais la terre secourut la femme, elle ouvrit sa bouche, et elle engloutit le fleuve que le dragon avait lancé de sa gueule.
\VS{17}Alors le dragon fut irrité contre la femme, et s'en alla faire la guerre contre les autres qui sont de la semence de la femme, qui gardent les commandements de Dieu, et qui ont le témoignage de Jésus-Christ.
\VS{18}Et je me tins sur le sable qui borde la mer.
\Chap{13}
\TextTitle{La bête qui monte de la mer, l'antichrist}
\VerseOne{}Et je vis monter de la mer une bête\FTNT{Cette bête représente deux entités. Tout d'abord l'homme impie, l'Antichrist, et ensuite un système politique. Les dix cornes sur sa tête symbolisent les dix nations les plus puissantes de la terre avec lesquelles il imposera sa dictature mondiale (Da. 7:16-25). L'alliage des quatre métaux dans la statue de Nébucadnetsar en Da. 2 et la vision des quatre animaux en Da. 7, annoncent l'instauration d'un quatrième empire ou encore le système politique à la tête duquel sera la bête.} qui avait sept têtes et dix cornes, et sur ses cornes dix diadèmes, et sur ses têtes des noms de blasphème\FTNT{Voir annexe «~La bête d'apocalypse~».}.
\VS{2}Et la bête que je vis était semblable à un léopard, ses pieds étaient comme ceux d'un ours~; sa gueule était comme la gueule d'un lion\FTNT{Da. 7:7.}. Et le dragon lui donna sa puissance, son trône, et une grande autorité.
\VS{3}Et je vis l'une de ses têtes comme blessée à mort, mais sa blessure mortelle fut guérie. Remplie d'admiration, la terre entière suivit la bête.
\VS{4}Et ils adorèrent le dragon, parce qu'il avait donné l'autorité à la bête, et ils adorèrent aussi la bête, en disant~: Qui est semblable à la bête, et qui peut combattre contre elle~?
\VS{5}Et il lui fut donné une bouche qui proférait des discours pleins d'orgueil, et des blasphèmes~; et il lui fut aussi donné le pouvoir d'agir pendant quarante-deux mois.
\VS{6}Elle ouvrit sa bouche pour blasphémer contre Dieu, pour blasphémer son Nom et son tabernacle, et ceux qui habitent dans le ciel.
\VS{7}Et il lui fut donné de faire la guerre aux saints et de les vaincre. Il lui fut aussi donné autorité sur toute tribu, toute langue et toute nation.
\VS{8}Et tous les habitants de la terre l'adoreront, ceux dont les noms n'ont pas été écrits dans le livre de vie de l'Agneau immolé dès la fondation du monde.
\VS{9}Si quelqu'un a des oreilles qu'il entende.
\VS{10}Si quelqu'un est destiné à la captivité, il ira en captivité~; si quelqu'un tue avec l'épée, il faut qu'il soit lui-même tué avec l'épée. C'est ici la persévérance et la foi des saints.
\TextTitle{La bête qui monte de la terre, le faux prophète}
\VS{11}Puis je vis une autre bête qui montait de la terre\FTNT{Cette bête est identifiée au faux-prophète car son rôle consiste à amener les habitants de la terre à adorer la première bête, tout comme les vrais prophètes invitent les gens à l'adoration du Dieu véritable (Mt. 7:15).}, et qui avait deux cornes semblables à celles de l'Agneau~; mais elle parlait comme le dragon.
\VS{12}Et elle exerçait toute l'autorité de la première bête en sa présence, et elle obligeait la terre et ses habitants à adorer la première bête, dont la blessure mortelle avait été guérie\FTNT{Cette bête a existé par le passé sous la forme de l'empire romain qui s'est écroulé le 4 septembre 476. Ce régime a marqué l'histoire par son caractère universel et brutal. Le fait que cette bête blessée à mort reprenne vie, annonce l'instauration d'un empire universel qui aura les caractéristiques combinées de l'empire babylonien, médo-perse, gréco-macédonien et romain, ceux-ci correspondant aux quatre animaux de la vision de Da. 7:1-8~: Le lion, l'ours, le léopard et le quatrième animal.}.
\VS{13}Elle opérait de grands prodiges, même jusqu'à faire descendre le feu du ciel sur la terre devant les hommes.
\VS{14}Et elle séduisait les habitants de la terre, à cause des prodiges qu'il lui était donné d'opérer en présence de la bête, disant aux habitants de la terre de faire une image\FTNT{Dieu interdit la vénération des images (Ex. 20:4-5) La particularité de l'image de la bête est qu'elle possède un esprit (démon).} de la bête qui avait reçu le coup mortel de l'épée, et qui était bien vivante.
\VS{15}Et il lui fut donné de mettre un esprit à l'image de la bête, afin que même l'image de la bête parle, et qu'elle fasse que tous ceux qui n'adoreraient pas l'image de la bête soient mis à mort.
\VS{16}Elle fit que tous, petits et grands, riches et pauvres, libres et esclaves, reçoivent une marque sur leur main droite, ou sur leur front\FTNT{Il s'agit d'une marque qui est avant tout spirituelle. Car de la même façon que nous sommes scellés et marqués par l'Esprit de Dieu qui produit en nous la sainteté (Ga. 5:22~; Ro. 6:20-22~; Ep. 1:13~; Ep. 4:30) Satan marque les siens par le péché (1 Ti. 4:1-2~; 2 Ti. 3:1-5).}~;
\VS{17}et que personne ne puisse acheter ni vendre, sans avoir la marque ou le nom de la bête, ou le nombre de son nom.
\VS{18}Ici est la sagesse~: Que celui qui a de l'intelligence compte le nombre de la bête, car c'est un nombre d'homme, et son nombre est six cent soixante-six.
\Chap{14}
\TextTitle{L'Agneau et les 144 000}
\VerseOne{}Puis je regardai, et voici, l'Agneau se tenait sur la montagne de Sion, et il y avait avec lui cent quarante-quatre mille personnes qui avaient le Nom de son Père écrit sur leurs fronts.
\VS{2}Et j'entendis une voix du ciel comme le bruit des grandes eaux, et comme le bruit d'un grand tonnerre~; et j'entendis une voix de joueurs de harpe jouant de leurs harpes.
\VS{3}Et ils chantaient comme un cantique nouveau devant le trône, et devant les quatre animaux, et devant les anciens. Et personne ne pouvait apprendre le cantique, si ce n'est les cent quarante-quatre mille qui avaient été rachetés de la terre.
\VS{4}Ce sont ceux qui ne se sont pas souillés avec les femmes, car ils sont vierges~; ce sont ceux qui suivent l'Agneau partout où il va. Ils ont été rachetés d'entre les hommes pour être des prémices pour Dieu et pour l'Agneau.
\VS{5}Et dans leur bouche il ne s'est pas trouvé de fraude, car ils sont sans tache devant le trône de Dieu\FTNT{Ps. 32:2.}.
\TextTitle{l'Evangile éternel et la chute de Babylone}
\VS{6}Puis je vis un autre ange qui volait au milieu du ciel, il avait l'Evangile éternel pour évangéliser les habitants de la terre, de toute nation, de toute tribu, de toute langue et de tout peuple.
\VS{7}Il disait d'une voix forte~: Craignez Dieu, et donnez-lui gloire, car l'heure de son jugement est venue~; et adorez celui qui a fait le ciel et la terre, la mer et les sources des eaux.
\VS{8}Et un autre ange le suivit, disant~: Elle est tombée, elle est tombée Babylone, la grande ville, parce qu'elle a abreuvé toutes les nations du vin de la fureur de son impudicité~!
\TextTitle{Le jugement des adorateurs de la bête}
\VS{9}Et un troisième ange les suivit, disant d'une voix forte~: Si quelqu'un adore la bête et son image, et reçoit la marque sur son front ou sur sa main,
\VS{10}il boira, lui aussi, du vin de la colère de Dieu, du vin pur versé dans la coupe de sa colère, et il sera tourmenté dans le feu et le soufre devant les saints anges et devant l'Agneau.
\VS{11}Et la fumée de leur tourment montera aux siècles des siècles, et ils n'auront de repos ni jour ni nuit, ceux qui adorent la bête et son image, et quiconque reçoit la marque de son nom.
\VS{12}Ici est la persévérance des saints~; ici sont ceux qui gardent les commandements de Dieu, et la foi de Jésus.
\TextTitle{Bénédiction de ceux qui meurent en Christ}
\VS{13}Alors j'entendis une voix du ciel qui me disait~: Ecris~: Bénis sont dès à présent les morts qui meurent dans le Seigneur~! Oui, c'est vrai~! dit l'Esprit, afin qu'ils se reposent de leurs travaux, car leurs œuvres les suivent.
\TextTitle{Prophétie sur Harmaguédon}
\VS{14}Et je regardai, et voici, il y avait une nuée blanche, et sur la nuée était assis quelqu'un qui ressemblait à un homme\FTNT{Ez. 1:26~; Da. 7:13~; Mt. 24:30~; Mt. 26:64~; Ap. 1:13.}, ayant sur sa tête une couronne d'or, et dans sa main une faucille tranchante.
\VS{15}Et un autre ange sortit du temple, criant à haute voix à celui qui était assis sur la nuée~: Jette ta faucille, et moissonne~; car c'est ton heure de moissonner, parce que la moisson de la terre est mûre\FTNT{Jé. 51:33~; Mt. 13:30-39.}.
\VS{16}Alors celui qui était assis sur la nuée jeta sa faucille sur la terre, et la terre fut moissonnée.
\VS{17}Et un autre ange sortit du temple qui est dans le ciel, ayant lui aussi une faucille tranchante.
\VS{18}Et un autre ange, qui avait autorité sur le feu, sortit de l'autel, et s'adressant d'une voix forte à celui qui avait la faucille tranchante, dit~: Jette ta faucille tranchante, et vendange les grappes de la vigne de la terre, car ses raisins sont mûrs.
\VS{19}Et l'ange jeta sa faucille tranchante sur la terre et vendangea la vigne de la terre, et il jeta la vendange dans la grande cuve de la colère de Dieu.
\VS{20}Et la cuve fut foulée hors de la ville~; et du sang sortit de la cuve, jusqu'aux mors des chevaux, sur une étendue de mille six cents stades\FTNT{Es. 63:1-6.}.
\Chap{15}
\TextTitle{Une scène glorieuse au ciel}
\VerseOne{}Puis je vis dans le ciel un autre signe, grand et admirable~: Sept anges qui tenaient les sept derniers fléaux, car c'est par eux que s'accomplit la colère de Dieu.
\VS{2}Et je vis aussi comme une mer de verre mêlée de feu, et ceux qui avaient vaincu la bête et son image, et sa marque, et le nombre de son nom, étaient debout sur la mer qui était comme de verre, et ayant les harpes de Dieu.
\VS{3}Ils chantaient le cantique de Moïse, serviteur de Dieu, et le cantique de l'Agneau, en disant~: Tes œuvres sont grandes et merveilleuses, ô Seigneur Dieu Tout-Puissant~! Tes voies sont justes et véritables, ô Roi des saints~!
\VS{4}Seigneur, qui ne te craindrait, et qui ne glorifierait ton Nom~? Car toi seul tu es Saint, c'est pourquoi toutes les nations viendront et se prosterneront devant toi~; car tes jugements sont pleinement manifestés.
\VS{5}Et après ces choses, je regardai, et voici le temple du tabernacle du témoignage fut ouvert dans le ciel.
\VS{6}Et les sept anges qui avaient les sept fléaux sortirent du temple, revêtus d'un lin pur et blanc, et ayant des ceintures d'or autour de leurs poitrines.
\VS{7}Et l'un des quatre animaux donna aux sept anges sept coupes d'or, pleines de la colère du Dieu qui vit aux siècles des siècles.
\VS{8}Et le temple fut rempli de la fumée à cause de la gloire de Dieu et de sa puissance~; et personne ne pouvait entrer dans le temple jusqu'à ce que les sept fléaux des sept anges soient accomplis.
\Chap{16}
\TextTitle{Première coupe~: Les ulcères}
\VerseOne{}Et j'entendis du temple une voix éclatante qui disait aux sept anges~: Allez, et versez sur la terre les coupes de la colère de Dieu.
\VS{2}Et le premier ange s'en alla, et versa sa coupe sur la terre. Et un ulcère malin et dangereux frappa les hommes qui avaient la marque de la bête, et ceux qui adoraient son image.
\TextTitle{Deuxième coupe~: La mer changée en sang}
\VS{3}Et le second ange versa sa coupe sur la mer, et elle devint comme le sang d'un corps mort, et tout être qui vivait dans la mer mourut.
\TextTitle{Troisième coupe~: Les sources changées en sang}
\VS{4}Et le troisième ange versa sa coupe sur les fleuves et sur les sources des eaux, et elles devinrent du sang.
\VS{5}Et j'entendis l'ange des eaux qui disait~: Seigneur, QUI ES, QUI ETAIS, et QUI VIENS, tu es juste, parce que tu as exercé ce jugement.
\VS{6}Parce qu'ils ont répandu le sang des saints et des prophètes, tu leur as aussi donné du sang à boire, car ils le méritent.
\VS{7}Et j'entendis un autre de l'autel, qui disait~: Certainement, Seigneur Dieu Tout-Puissant, tes jugements sont véritables et justes.
\TextTitle{Quatrième coupe~: Une chaleur extrême}
\VS{8}Ensuite, le quatrième ange versa sa coupe sur le soleil, et le pouvoir lui fut donné de brûler les hommes par le feu,
\VS{9}de sorte que les hommes furent brûlés par de grandes chaleurs, et ils blasphémèrent le Nom de Dieu qui a puissance sur ces fléaux~; et ils ne se repentirent pas pour lui donner gloire.
\TextTitle{Cinquième coupe~: Les ténèbres sur le trône de la bête}
\VS{10}Après cela, le cinquième ange versa sa coupe sur le trône de la bête. Et son royaume fut couvert de ténèbres, et les hommes se mordaient la langue à cause de la douleur qu'ils ressentaient.
\VS{11}Et ils blasphémèrent le Dieu du ciel à cause de leurs douleurs et de leurs ulcères~; et ils ne se repentirent pas de leurs œuvres.
\TextTitle{Sixième coupe~: L'Euphrate asséché}
\VS{12}Puis le sixième ange versa sa coupe sur le grand fleuve, l'Euphrate. Et son eau tarit, afin de préparer la voie des rois venant du côté où le soleil se lève.
\VS{13}Et je vis sortir de la gueule du dragon, et de la gueule de la bête, et de la bouche du faux prophète, trois esprits impurs semblables à des grenouilles.
\VS{14}Car ce sont des esprits de démons, qui font des prodiges, et qui vont vers les rois de la terre et du monde entier, afin de les assembler pour le combat de ce grand jour du Dieu Tout-Puissant.
\VS{15}Voici, je viens comme un voleur. Béni est celui qui veille et qui garde ses vêtements, afin de ne pas marcher nu, et qu'on ne voie pas sa honte~!
\VS{16}Et ils les assemblèrent dans le lieu qui est appelé en hébreu Harmaguédon\FTNT{Le terme «~Harmaguédon~», mentionné uniquement dans ce passage, vient du mot hébreu «~Har-Magidown~», ce qui signifie «~Montagne de Megiddo~». Bien qu'il n'existe pas de montagne portant spécifiquement ce nom, l'emplacement probable de cet endroit est la plaine de Meggido se trouvant à proximité de Jérusalem. Par le passé, elle fut le théâtre de la victoire de Barak sur les Cananéens (Jg. 4:15) et de celle de Gédéon sur les Madianites (Jg. 7). C'est aussi à cet endroit que Saül et ses fils (1 Sa. 31~:8) ainsi que le roi Josias (2 R. 23:29-30~; 2 Ch. 35:22) trouvèrent la mort. Pour toutes ces raisons, elle devint au fil du temps le symbole de l'affrontement entre Dieu et la puissance des ténèbres. Selon les prophéties bibliques, la plaine de Meggido et la vallée de Jizréel constitueront le site de l'ultime guerre mondiale, celle opposant l'Antichrist et ses alliés (dirigeants des nations) contre Israël. Le Seigneur interviendra alors ouvertement dans les affaires humaines pour déverser la coupe de sa colère (Ap. 16:1) et anéantir l'homme impie et toute son armée (Ez. 38-39~; Joë. 3~; Mi. 4:11~; So. 1~; Za. 14~; Mt. 24:29-30~; Ap. 20:1-3~; Ap. 20:7-10).}.
\TextTitle{Septième coupe~: Une grosse grêle tombe du ciel}
\VS{17}Puis le septième ange versa sa coupe dans l'air~; et il sortit du temple du ciel une voix forte qui venait du trône, disant~: C'en est fait.
\VS{18}Et il y eut des éclairs, et des voix, et des tonnerres, et il se fit un grand tremblement de terre, dis-je, tel qu'il n'y en avait jamais eu depuis que les hommes sont sur la terre.
\VS{19}La grande ville fut divisée en trois parties, et les villes des nations tombèrent, et Dieu se souvint de Babylone la grande, pour lui donner la coupe du vin de son ardente colère.
\VS{20}Toutes les îles s'enfuirent et les montagnes ne furent plus retrouvées.
\VS{21}Une grosse grêle, dont les grêlons pesaient un talent\FTNT{Un talent d'argent pesait 45 kg, un talent d'or pesait 90 kg.}, tomba du ciel sur les hommes~; et les hommes blasphémèrent Dieu, à cause du fléau de la grêle, car le fléau qu'elle causa fut très grand.
\Chap{17}
\TextTitle{La prostituée}
\VerseOne{}Puis l'un des sept anges qui tenaient les sept coupes vint, et il m'adressa la parole, en disant~: Viens, je te montrerai le jugement de la grande prostituée, qui est assise sur les grandes eaux.
\VS{2}Avec elle, les rois de la terre ont commis la fornication, et les habitants de la terre ont été enivrés du vin de sa prostitution.
\VS{3}Il me transporta en esprit dans un désert~; et je vis une femme assise sur une bête écarlate, pleine de noms de blasphème, ayant sept têtes et dix cornes.
\VS{4}Et la femme était vêtue de pourpre et d'écarlate, et parée d'or, de pierres précieuses, et de perles~; et elle tenait à la main une coupe d'or, pleine des abominations de l'impureté de sa prostitution.
\VS{5}Et il y avait sur son front un nom écrit, un mystère~: Babylone la grande, la mère des impudicités et des abominations de la terre\FTNT{Symboliquement, Babylone la grande incarne l'Eglise apostate. Elle est soutenue par la bête qu'elle chevauche, c'est-à-dire l'homme impie. Ces deux entités forment un système impie où la politique et la religion se mélangent (Da. 2:43).}.
\VS{6}Et je vis cette femme ivre du sang des saints, et du sang des martyrs de Jésus. Et quand je la vis, je fus saisi d'un grand étonnement.
\TextTitle{Alliance entre la prostituée et la bête}
\VS{7}Et l'ange me dit~: Pourquoi t'étonnes-tu~? Je te dirai le mystère de la femme et de la bête qui la porte, qui a les sept têtes et les dix cornes.
\VS{8}La bête que tu as vue, était, et elle n'est plus. Elle doit monter de l'abîme, et aller à la perdition. Et les habitants de la terre, ceux dont les noms ne sont pas écrits dans le Livre de vie dès la fondation du monde, s'étonneront en voyant la bête parce qu'elle était, et qu'elle n'est plus, et qui toutefois est.
\VS{9}C'est ici qu'il faut un esprit intelligent et qui ait de la sagesse. Les sept têtes sont sept montagnes sur lesquelles la femme est assise.
\VS{10}Ce sont aussi sept rois, les cinq sont tombés~; l'un est, et l'autre n'est pas encore venu~; et quand il sera venu, il faut qu'il demeure pour un peu de temps.
\VS{11}Et la bête qui était, et qui n'est plus, est elle-même un huitième roi, et elle est du nombre des sept, mais elle tend à sa ruine.
\VS{12}Et les dix cornes que tu as vues sont dix rois, qui n'ont pas encore commencé à régner, mais ils recevront autorité comme rois en même temps avec la bête, pour une heure.
\VS{13}Ils ont un même dessein, et ils donneront leur puissance et leur autorité à la bête.
\TextTitle{Victoire de l'Agneau sur la prostituée}
\VS{14}Ils combattront contre l'Agneau et l'Agneau les vaincra, parce qu'il est le Seigneur des seigneurs, et le Roi des rois~; et les appelés, les élus et les fidèles qui sont avec lui, les vaincront aussi.
\VS{15}Puis il me dit~: Les eaux que tu as vues, et sur lesquelles la prostituée est assise, sont des peuples, des nations et des langues.
\VS{16}Les dix cornes que tu as vues sur la bête haïront la prostituée, la rendront désolée et nue, la dépouilleront, et mangeront sa chair, et la brûleront au feu.
\VS{17}Car Dieu a mis dans leurs cœurs de faire ce qu'il lui plaît, et de former un même dessein, et de donner leur royaume à la bête, jusqu'à ce que les paroles de Dieu soient accomplies.
\VS{18}Et la femme que tu as vue, c'est la grande ville, qui règne sur les rois de la terre.
\Chap{18}
\TextTitle{Babylone détruite}
\VerseOne{}Après ces choses, je vis descendre du ciel un autre ange, qui avait une grande autorité, et la terre fut illuminée de sa gloire.
\VS{2}Il cria avec force à haute voix, et il dit~: Elle est tombée, elle est tombée Babylone la grande, et elle est devenue la demeure de démons, et la retraite de tout esprit impur, et le repaire de tout oiseau impur et exécrable.
\VS{3}Car toutes les nations ont bu du vin de sa prostitution effrénée, et les rois de la terre ont commis la fornication avec elle, et les marchands de la terre se sont enrichis par l'excès de son luxe.
\VS{4}Puis j'entendis une autre voix du ciel, qui disait~: Sortez de Babylone, mon peuple, afin que vous ne participiez pas à ses péchés, et que vous n'ayez pas de part à ses fléaux.
\VS{5}Car ses péchés sont montés jusqu'au ciel, et Dieu s'est souvenu de ses iniquités.
\VS{6}Rendez-lui selon ce qu'elle vous a fait, et payez-lui au double selon ses œuvres~; et dans la même coupe où elle vous a versé à boire versez-lui au double.
\VS{7}Autant elle s'est glorifiée et plongée dans le luxe, autant donnez-lui de tourment et de deuil~; car elle dit en son cœur~: Je siège en reine, je ne suis pas veuve, et je ne verrai pas de deuil.
\VS{8}C'est pourquoi ses plaies, qui sont la mort, le deuil, et la famine, viendront en un même jour, et elle sera entièrement brûlée au feu~; car le Seigneur Dieu qui la jugera est puissant.
\TextTitle{Conséquence de la chute de Babylone~: Gémissements des habitants de la terre}
\VS{9}Et les rois de la terre, qui ont commis la fornication avec elle, et qui ont vécu dans le luxe, la pleureront, et mèneront deuil sur elle en se frappant la poitrine, quand ils verront la fumée de son embrasement~;
\VS{10}et ils se tiendront éloignés dans la crainte de son tourment, et diront~: Malheur~! Malheur~! Babylone la grande, cette ville si puissante, comment ta condamnation est-elle venue en une seule heure~?
\VS{11}Les marchands de la terre aussi pleureront, et seront dans le deuil à cause d'elle, parce que personne n'achète plus de leurs marchandises,
\VS{12}qui sont des marchandises d'or, d'argent, de pierres précieuses, de perles, de fin lin, de pourpre, de soie, d'écarlate, de toute sorte de bois odoriférant, de toute espèce de bois de senteur, d'ivoire, et de toute espèce de vaisseaux de bois très précieux, d'airain, de fer, et de marbre,
\VS{13}du cinnamome, des parfums, des essences, de l'encens, du vin, de l'huile, de la fine fleur de farine, du blé, des bœufs, des brebis, des chevaux, des chars, des esclaves, et des âmes d'hommes.
\VS{14}Car les fruits du désir de ton âme se sont éloignés de toi, et toutes les choses délicates et excellentes sont perdues pour toi, et dorénavant tu ne les trouveras plus.
\VS{15}Les marchands, dis-je, de ces choses, qui se sont enrichis par elle, se tiendront éloignés, dans la crainte de son tourment~; ils pleureront et seront dans le deuil,
\VS{16}et diront~: Malheur~! Malheur~! La grande ville qui était vêtue de fin lin, de pourpre, d'écarlate, qui était parée d'or, ornée de pierres précieuses, et de perles, comment en une seule heure tant de richesses ont été détruites~?
\VS{17}Et tous les pilotes aussi, tous ceux qui naviguent vers ce lieu, tous les marins, et tous ceux qui exploitent la mer, se tiendront éloignés,
\VS{18}et, en voyant la fumée de son embrasement, ils s'écrieront, en disant~: Quelle ville était semblable à cette grande ville~?
\VS{19}Ils jetteront de la poussière sur leurs têtes, pleurant et menant deuil, ils crieront, en disant~: Malheur~! Malheur~! La grande ville, où se sont enrichis par son opulence tous ceux qui ont des navires sur la mer, comment a-t-elle été réduite en désert en une seule heure~?
\TextTitle{Réjouissance des anges suite à la chute de Babylone}
\VS{20}Ô ciel~! Réjouis-toi à cause d'elle~; et vous aussi saints apôtres et prophètes, réjouissez-vous~! Car Dieu l'a punie à cause de vous.
\VS{21}Alors un ange d'une grande force prit une pierre semblable à une grande meule, et la jeta dans la mer, en disant~: Ainsi sera précipitée avec impétuosité Babylone, cette grande ville, et elle ne sera plus retrouvée\FTNT{Jé. 51:63-64.}.
\VS{22}Et l'on entendra plus chez toi les sons des joueurs de harpe, des musiciens, des joueurs de flûte, et de ceux qui sonnent de la trompette~; et on ne trouvera plus chez toi aucun artisan d'un métier quelconque, on n'entendra plus chez toi le bruit de la meule,
\VS{23}et la lumière de la lampe ne brillera plus chez toi, et la voix de l'époux et de l'épouse ne sera plus entendue chez toi~; car tes marchands étaient des princes de la terre, et parce que par tes enchantements toutes les nations ont été séduites,
\VS{24}et l'on a trouvé chez elle le sang des prophètes et des saints, et de tous ceux qui ont été mis à mort sur la terre.
\Chap{19}
\TextTitle{Allégresse dans les cieux suite au jugement de la grande prostituée\FTNTT{Ap. 17:16-17~; 18:8.}}
\VerseOne{}Après cela, j'entendis dans le ciel une voix forte d'une foule nombreuse, disant~: Alléluia~! Le salut, la gloire, l'honneur et la puissance appartiennent au Seigneur, notre Dieu,
\VS{2}car ses jugements sont véritables et justes~; car il a jugé la grande prostituée qui a corrompu la terre par son impudicité, et parce qu'il a vengé le sang de ses serviteurs versé de la main de la prostituée.
\VS{3}Et ils dirent encore~: Alléluia~! Et sa fumée monte aux siècles des siècles.
\VS{4}Et les vingt-quatre anciens et les quatre animaux se prosternèrent sur leurs faces, et adorèrent Dieu, qui était assis sur le trône, en disant~: Amen~! Alléluia~!
\VS{5}Et il sortit du trône une voix qui disait~: Louez notre Dieu, vous tous ses serviteurs, et vous qui le craignez, tant les petits que les grands\FTNT{Ps. 134.}.
\VS{6}J'entendis ensuite comme la voix d'une grande assemblée, et comme le bruit de grandes eaux, et comme l'éclat de grands tonnerres, disant~: Alléluia~! Car le Seigneur notre Dieu Tout-Puissant a pris possession de son Royaume.
\TextTitle{Festin des noces de l'Agneau}
\VS{7}Réjouissons-nous et tressaillons de joie, et donnons-lui gloire, car les noces de l'Agneau sont venues, et son Epouse s'est préparée.
\VS{8}Et il lui a été donné de se revêtir d'un fin lin pur et éclatant. Car le fin lin désigne la justice des saints.
\VS{9}Alors il me dit~: Ecris~: Bénis sont ceux qui sont appelés au festin des noces de l'Agneau\FTNT{Mt. 22:1-13~; Lu. 14:15-24.}~! Il me dit aussi~: Ces paroles de Dieu sont véritables.
\VS{10}Alors je tombai à ses pieds pour l'adorer, mais il me dit~: Garde-toi de le faire~! Je suis ton compagnon de service, et celui de tes frères qui ont le témoignage de Jésus. Adore Dieu~! Car le témoignage de Jésus est l'Esprit de la prophétie.
\TextTitle{Seconde venue du Messie dans la gloire\FTNTT{Mt. 24:16-30.}}
\VS{11}Puis je vis le ciel ouvert, et voici parut un cheval blanc. Et celui qui était monté dessus s'appelle FIDELE et VERITABLE, et il juge et combat avec justice.
\VS{12}Et ses yeux étaient comme une flamme de feu~; il y avait sur sa tête plusieurs diadèmes, et il avait un nom écrit que personne ne connaît, si ce n'est lui-même.
\VS{13}Il était revêtu d'un vêtement teint de sang, et son Nom s'appelle LA PAROLE DE DIEU.
\VS{14}Les armées qui sont dans le ciel le suivaient sur des chevaux blancs, revêtues de fin lin blanc et pur.
\VS{15}De sa bouche sortait une épée tranchante\FTNT{Es. 11:4~; 2 Th. 2:8~; Hé. 4:12.}, pour frapper les nations~; il les gouvernera avec un sceptre de fer\FTNT{Ps. 2:8-9.}, et il foulera la cuve du vin de l'indignation et de la colère du Dieu Tout-Puissant.
\VS{16}Et sur son vêtement et sur sa cuisse étaient écrits ces mots~: LE ROI DES ROIS ET LE SEIGNEUR DES SEIGNEURS.
\TextTitle{Bataille d'Harmaguédon\FTNTT{Ap. 16:16.}}
\VS{17}Puis je vis un ange qui se tenait dans le soleil. Il cria d'une voix forte, et dit à tous les oiseaux qui volaient au milieu du ciel~: Venez et rassemblez-vous pour le grand festin de Dieu,
\VS{18}afin de manger la chair des rois, la chair des chefs militaires, la chair des puissants, la chair des chevaux et de ceux qui les montent, et la chair de toute sorte de personnes libres, esclaves, petits et grands.
\VS{19}Alors je vis la bête et les rois de la terre, et leurs armées rassemblées pour faire la guerre\FTNT{Guerre d' Harmaguédon~: Voir commentaire Ap. 16:16.} contre celui qui était monté sur le cheval et contre son armée.
\TextTitle{Condamnation de la bête et du faux prophète}
\VS{20}Et la bête fut prise, et avec elle le faux prophète qui avait fait devant elle les prodiges par lesquels il avait séduit ceux qui avaient pris la marque de la bête, et adoré son image. Et ils furent tous deux jetés vivants dans l'étang ardent de feu et de soufre.
\TextTitle{Condamnation des rois et des armées}
\VS{21}Et le reste fut tué par l'épée qui sortait de la bouche de celui qui était monté sur le cheval, et tous les oiseaux furent rassasiés de leur chair.
\Chap{20}
\TextTitle{Satan lié pour mille ans et règne du Messie}
\VerseOne{}Après cela, je vis descendre du ciel un ange, qui avait la clef de l'abîme et une grande chaîne dans sa main.
\VS{2}Il saisit le dragon, le serpent ancien, qui est le diable et Satan, et le lia pour mille ans.
\VS{3}Il le jeta dans l'abîme, et il l'enferma et mit le sceau sur lui, afin qu'il ne séduise plus les nations, jusqu'à ce que les mille ans soient accomplis. Après quoi, il faut qu'il soit délié pour un peu de temps.
\TextTitle{Dernière phase de la première résurrection}
\VS{4}Je vis des trônes, sur lesquels des gens s'assirent, à qui l'autorité de juger fut donnée\FTNT{1 Co. 6:2.}. Et je vis les âmes de ceux qui avaient été décapités pour le témoignage de Jésus, et pour la parole de Dieu, et de ceux qui n'avaient pas adoré la bête ni son image, et qui n'avaient pas pris sa marque sur leurs fronts, ou sur leurs mains. Et ils vécurent\FTNT{Jn. 14:19.} et régnèrent avec Christ mille ans.
\VS{5}Les autres morts ne revinrent pas à la vie jusqu'à ce que les mille ans soient accomplis. C'est la première résurrection.
\VS{6}Bénis et saints sont ceux qui ont part à la première résurrection~! La seconde mort n'a pas de puissance sur eux, mais ils seront prêtres de Dieu, et de Christ, et ils régneront avec lui mille ans.
\TextTitle{Satan délié~; sa chute finale}
\VS{7}Et quand les mille ans seront accomplis, Satan sera délié de sa prison.
\VS{8}Et il sortira pour séduire les nations qui sont aux quatre coins de la terre, Gog et Magog, afin de les rassembler pour la guerre, et leur nombre est comme le sable de la mer.
\VS{9}Ils montèrent et se répandirent à la surface de la terre, et ils environnèrent le camp des saints, et la ville bien-aimée. Mais Dieu fit descendre un feu du ciel qui les dévora.
\TextTitle{Satan jeté dans l'étang de feu}
\VS{10}Et le diable qui les séduisait fut jeté dans l'étang de feu et de soufre, où sont la bête et le faux prophète. Et ils seront tourmentés jour et nuit, aux siècles des siècles.
\TextTitle{Résurrection des impies et jugement dernier~; l'Hadès (ou enfer) et la mort jetés dans l'étang de feu}
\VS{11}Puis je vis un grand trône blanc, et celui qui était assis dessus. La terre et le ciel s'enfuirent devant sa face, et il ne fut plus trouvé de place pour eux.
\VS{12}Et je vis les morts, les grands et les petits, qui se tenaient devant Dieu. Des livres furent ouverts. Et un autre livre fut ouvert, celui qui est le Livre de vie. Et les morts furent jugés selon les choses qui étaient écrites dans les livres, c'est-à-dire selon leurs œuvres.
\VS{13}Et la mer rendit les morts qui étaient en elle, et la mort et l'enfer\FTNT{le mot «~enfer~» vient de l'hébreu «~Hadès~». Voir commentaire dans Mt. 16:18.} rendirent les morts qui étaient en eux~; et ils furent jugés chacun selon ses œuvres.
\VS{14}Et la mort et l'enfer furent jetés dans l'étang de feu\FTNT{L'étang de feu est aussi appelé «~seconde mort~», c'est la destination finale de tous les impies, des démons et de Satan. On l'appelle «~la seconde mort~» parce qu'elle a été précédée de la mort physique. Cette mort n'est pas un anéantissement, mais une condition de souffrances éternelles. C'est la séparation définitive d'avec Dieu. A l'issue du jugement dernier, le séjour des morts (le dieu Hadès ou l'enfer) sera jeté dans le lac de feu (voir commentaire en Mt. 16:18). La Bible utilise également le mot «~géhenne~» pour décrire l'endroit où les impies passeront l'éternité. Ce terme vient de l'hébreu «~ge-hinnom~», autrement dit vallée de Ben Hinnom (littéralement «~le lieu du feu~») qui se trouve en Israël, en contrebas du mont Sion sur lequel est bâtie la ville de Jérusalem (Mt. 5:22~; Mt. 5:29-30~; Mt. 10:28~; Mt. 18:9~; Mt. 23:15~; Mt. 23:33~; Mc. 9:47~; Lu. 12:5~; Ja. 3:6). Autrefois, on y brûlait des enfants en l'honneur de Moloc, une divinité ammonite (2 R. 23:10~; Jé. 32:35), puis des immondices. Ce lieu est devenu avec le temps le symbole du péché et de l'affliction et c'est ainsi qu'il finit par désigner le lieu du châtiment éternel.}. C'est la seconde mort.
\VS{15}Et quiconque ne fut pas trouvé écrit dans le Livre de vie fut jeté dans l'étang de feu.
\Chap{21}
\TextTitle{Nouveaux cieux et une nouvelle terre~; la nouvelle Jérusalem}
\VerseOne{}Puis je vis un nouveau ciel et une nouvelle terre~; car le premier ciel et la première terre avaient disparu, et la mer n'était plus.
\VS{2}Et moi, Jean, je vis la ville sainte, la nouvelle Jérusalem, qui descendait du ciel, d'auprès de Dieu, parée comme une épouse qui s'est ornée pour son mari.
\VS{3}Et j'entendis du trône une voix forte qui disait~: Voici le tabernacle de Dieu avec les hommes~! Il habitera avec eux, et ils seront son peuple, et Dieu lui-même sera leur Dieu, et il sera avec eux.
\VS{4}Et Dieu essuiera toute larme de leurs yeux, et la mort ne sera plus~; et il n'y aura plus ni deuil, ni cri, ni douleur, car les premières choses sont passées.
\VS{5}Et celui qui était assis sur le trône dit~: Voici, je fais toutes choses nouvelles. Puis il me dit~: Ecris, car ces paroles sont véritables et certaines.
\VS{6}Il me dit aussi~: Tout est accompli. Je suis l'Alpha et l'Oméga, le commencement et la fin. A celui qui a soif, je lui donnerai de la source d'eau vive gratuitement\FTNT{Es. 55:1-2~; Mt. 10:8~; Ap. 22:17. Voir commentaire Mt. 10:8.}.
\VS{7}Celui qui vaincra héritera toutes choses~; je serai son Dieu, et il sera mon fils.
\VS{8}Mais pour les timides, les incrédules, les abominables, les meurtriers, les fornicateurs, les sorciers, les idolâtres et tous les menteurs, leur part sera dans l'étang ardent de feu et de soufre, qui est la seconde mort.
\TextTitle{L'Epouse de l'Agneau et la nouvelle Jérusalem}
\VS{9}Puis l'un des sept anges qui tenaient les sept coupes pleines des sept derniers fléaux s'approcha de moi et me parla, en disant~: Viens, et je te montrerai l'Epouse, la femme de l'Agneau.
\VS{10}Et il me transporta en esprit sur une grande et haute montagne, et il me montra la grande ville, la sainte Jérusalem, qui descendait du ciel d'auprès de Dieu,
\VS{11}ayant la gloire de Dieu. Son éclat était semblable à une pierre très précieuse, comme à une pierre de jaspe transparente comme du cristal.
\VS{12}Et elle avait une grande et haute muraille, avec douze portes, et aux portes douze anges, et des noms écrits sur elles, qui sont les noms des douze tribus des fils d'Israël\FTNT{Ez. 48:31-34.}.
\VS{13}A l'orient, trois portes, au nord, trois portes, du côté du sud, trois portes et du côté de l'occident, trois portes.
\VS{14}Et la muraille de la ville avait douze fondements, et les noms des douze apôtres de l'Agneau étaient écrits dessus\FTNT{Lu. 22:29-30~; Ep. 2:20.}.
\VS{15}Et celui qui parlait avec moi avait un roseau d'or pour mesurer la ville, ses portes et sa muraille.
\VS{16}Et la ville était bâtie en carré, et sa longueur était aussi grande que sa largeur. Il mesura donc la ville avec le roseau d'or, jusqu'à douze mille stades~; la longueur, la largeur et la hauteur étaient égales.
\VS{17}Puis il mesura la muraille qui fut de cent quarante-quatre coudées, de la mesure du personnage, c'est-à-dire de l'ange.
\VS{18}Et le bâtiment de la muraille était de jaspe, mais la ville était d'or pur, semblable à du verre fort transparent.
\VS{19}Et les fondements de la muraille de la ville étaient ornés de toutes sortes de pierres précieuses\FTNT{Es. 54:11-12.}~: Le premier fondement était de jaspe, le second de saphir, le troisième de calcédoine, le quatrième d'émeraude,
\VS{20}le cinquième de sardonyx, le sixième de sardoine, le septième de chrysolithe, le huitième de béryl, le neuvième de topaze, le dixième de chrysoprase, le onzième d'hyacinthe, le douzième d'améthyste.
\VS{21}Et les douze portes étaient douze perles~; chacune des portes était d'une seule perle. Et la place de la ville était d'or pur, comme du verre transparent.
\VS{22}Et je ne vis pas de temple dans la ville, parce que le Seigneur Dieu Tout-Puissant et l'Agneau en sont le Temple.
\VS{23}Et la ville n'a pas besoin du soleil ni de la lune pour l'éclairer, car la gloire de Dieu l'éclaire, et l'Agneau est son flambeau\FTNT{Es. 60:19.}.
\VS{24}Et les nations qui auront été sauvées marcheront à la faveur de sa lumière, et les rois de la terre y apporteront ce qu'ils ont de plus magnifique et de plus précieux.
\VS{25}Et ses portes ne se fermeront pas le jour, car il n'y aura pas de nuit\FTNT{Es. 60:11.}.
\VS{26}Et on y apportera la gloire et l'honneur des nations.
\VS{27}Il n'entrera chez elle rien de souillé, ni personne qui s'abandonne à l'abomination et au mensonge~; mais seulement ceux qui sont écrits dans le Livre de vie de l'Agneau.
\Chap{22}
\TextTitle{Règne éternel des saints avec l'Agneau}
\VerseOne{}Puis il me montra un fleuve d'eau de la vie\FTNT{Ce fleuve représente le Saint-Esprit~: Ez. 47:1-12~; Ps. 46:5~; Da. 7:9-10~; Jn. 7:38-39.}, transparent comme du cristal, qui sortait du trône de Dieu et de l'Agneau.
\VS{2}Et au milieu de la place de la ville, et des deux côtés du fleuve, était l'arbre de vie, portant douze fruits, et rendant son fruit chaque mois et les feuilles de l'arbre servaient à la guérison des nations\FTNT{Ge. 2:9~; Ge. 3:22~; Ez. 47:12.}.
\VS{3}Et il n'y aura plus d'anathème. Le trône de Dieu et de l'Agneau sera dans la ville, et ses serviteurs le serviront,
\VS{4}et ils verront sa face, et son Nom sera sur leurs fronts.
\VS{5}Et il n'y aura plus de nuit~; et ils n'auront besoin ni de lumière, ni de lampe, ni du soleil, parce que le Seigneur Dieu les éclairera, et ils régneront aux siècles des siècles.
\TextTitle{Certitude des prophéties de ce livre}
\VS{6}Puis il me dit~: Ces paroles sont certaines et véritables~; et le Seigneur, le Dieu des saints prophètes, a envoyé son ange pour manifester à ses serviteurs les choses qui doivent arriver bientôt.
\VS{7}Voici, je viens à toute vitesse\FTNT{Dans la plupart des traductions, ce passage a été traduit par «~Je viens bientôt~». Or le texte grec utilise le mot «~tachu~» qui signifie «~rapidement, à toute vitesse (sans tarder)~». Beaucoup doutent de cette promesse du Seigneur en faisant la même réflexion évoquée par Pierre~: «~Où est la promesse de son avènement~? Car depuis que les pères sont morts, toutes choses demeurent comme elles ont été dès le commencement de la création.~» (2 Pi. 3:4). Or le Seigneur ne tarde pas dans l'accomplissement de sa promesse, car il a fixé de sa propre autorité une date pour son retour, que lui seul connaît (Za. 14:7~; Mt. 24:36~; Mc. 13:32~; Ac. 1:6-7). Il sera donc fidèle à son calendrier et ne tardera pas (2 Pi. 3:9.~; Hé. 10:37).}. Béni est celui qui garde les paroles de la prophétie de ce livre~!
\VS{8}C'est moi, Jean, qui ai entendu et vu ces choses. Et après les avoir entendues et vues, je tombai à terre aux pieds de l'ange qui me les montrait pour l'adorer.
\VS{9}Mais il me dit~: Garde-toi de le faire~! Car je suis ton compagnon de service\FTNT{Hé. 1:14.} et celui de tes frères les prophètes, et de ceux qui gardent les paroles de ce livre. Adore Dieu~!
\VS{10}Il me dit aussi~: Ne scelle pas les paroles de la prophétie de ce livre. Car le temps est proche.
\VS{11}Que celui qui est injuste soit encore injuste, et que celui qui est souillé se souille encore~; et que celui qui est juste pratique encore la justice~; et que celui qui est saint se sanctifie encore~!
\VS{12}Voici, je viens à toute vitesse, et ma rétribution est avec moi\FTNT{Jésus affirme de nouveau ici sa divinité et confirme les prophéties d'Es. 35:4~; Es. 40:10~; Es. 62:11, où il est dit que Yahweh lui-même viendra avec ses rétributions.} pour rendre à chacun selon son œuvre.
\VS{13}Je suis l'Alpha et l'Oméga, le premier et le dernier, le commencement et la fin.
\VS{14}Bénis sont ceux qui lavent leurs robes afin d'avoir droit à l'arbre de vie, et d'entrer par les portes dans la ville.
\VS{15}Mais seront laissés dehors les chiens, les empoisonneurs, les fornicateurs, les meurtriers, les idolâtres et quiconque aime et pratique le mensonge.
\VS{16}Moi, Jésus, j'ai envoyé mon ange\FTNT{Cette déclaration de Jésus fait écho au verset 6 où il est dit que le Seigneur, le Dieu des esprits des prophètes, a envoyé son ange. Jésus confirme donc qu'il est Seigneur et Dieu.} pour vous confirmer ces choses dans les églises. Je suis le rejeton et la postérité de David, l'étoile brillante du matin.
\VS{17}Et l'Esprit et l'Epouse disent~: Viens~! Et que celui qui entend dise~: Viens~! Et que celui qui a soif vienne~; que celui qui veut, prenne gratuitement de l'eau de la vie.
\TextTitle{Nul ne doit y ajouter ou y retrancher}
\VS{18}Je le déclare à quiconque entend les paroles de la prophétie de ce livre~: Si quelqu'un y ajoute quelque chose, Dieu le frappera des fléaux décrits dans ce livre,
\VS{19}et si quelqu'un retranche quelque chose des paroles du livre de cette prophétie, Dieu retranchera la part qu'il a dans le livre de vie, dans la ville sainte et dans les choses qui sont écrites dans ce livre.
\VS{20}Celui qui rend témoignage de ces choses, dit~: Certainement, je viens à toute vitesse. Amen~! Oui, Seigneur Jésus, viens~!
\VS{21}Que la grâce de notre Seigneur Jésus-Christ soit avec vous tous~! Amen~!
\PPE{}
\end{multicols}

% inclusion des annexes
%\addcontentsline{toc}{chapter}{Aide}\clearpage
%\addcontentsline{toc}{section}{Dictionnaire}\clearpage
%% annexe dictionnaire
%\makeatletter
%    % mise en forme dictionnaire
%    \def\@oddhead{{\small{Dictionnaire\hfil\thepage\hfil\rightmark---\leftmark}}}
%    \def\@evenhead{{\small{\rightmark---\leftmark\hfil\thepage\hfil Dictionnaire}}}\clearpage
%    \makeatother
%        % inclusion dictionnaire
%        \small{\parindent=0mm{\begin{center}\Large\bfseries{Dictionnaire}\end{center}}\par\begin{multicols}{2}

\DicoEntry{AARON}\textit{, de l'hébreu «~Aharown~»~: «~haut placé~» ou «~éclairé~»}\newline
Issu de la tribu de Lévi, frère aîné de Moïse dont il fut le porte-parole. Premier grand prêtre* en Israël. Voir \vref{Ex. 4:14}~; \vref{Ex. 6:16-20} et \vref{Ex. 28}.

\DicoEntry{ABSALOM}\textit{, de l'hébreu «~'Abiyshalowm~»~: «~père de la paix~»}\newline
Fils du roi David et de Maaca, né à Hébron. Il tua Amnon son demi-frère aîné, car ce dernier avait déshonoré sa sœur Tamar. Quelques années plus tard, il conspira contre son père et se fit proclamer roi à Hébron. Il fut finalement tué par Joab, chef de l'armée de David. Voir \vref{2 S. 3:3}~; \vref{2 S. 13}~; \vref{2 S. 15-19}.

\DicoEntry{ABDIAS}\textit{, de l'hébreu «~Obadyah~»~: «~adorateur~» ou «~serviteur de Yahweh~»}\newline
Prophète de Yahweh dont le livre éponyme figure dans le Tanakh.

\DicoEntry{ABEL}\textit{, de l'hébreu «~Hebel~»~: «~souffle, vapeur~»}\newline
Deuxième fils d'Adam et Eve et première victime d'homicide de l'histoire, il fut assassiné par son frère Caïn et déclaré juste par Yahweh. Voir \vref{Ge. 4:2,8} et \vref{Mt. 23:35}.

\DicoEntry{ABIRAM}\textit{, de l'hébreu «~'Abyiram~»~: «~mon père est exalté~»}\newline
Issu de la tribu de Ruben, fils d'Eliab et frère de Dathan, il conspira avec Koré contre Moïse et Aaron. Voir \vref{No. 16:1-35}.

\DicoEntry{ABLUTION}\textit{, de l'hébreu «~rachats~»~: «~laver, baigner, nettoyer~»}\newline
Lavage de purification prescrit par la loi mosaïque et effectué avec de l'eau. Voir \vref{Ex. 29:4} et \vref{Hé. 9:10}.

\DicoEntry{ABOMINATION}\textit{, de l'hébreu «~tow'ebah~»~: «~une chose dégoûtante, abominable~» et du grec «~bdelugma~»~: «~chose folle, détestable~»}\newline
Pratique violant la loi de Yahweh et manifestant l'infidélité à Dieu comme l'idolâtrie* sous toutes ses formes, la magie ou l'homosexualité*. Voir \vref{Lé. 18:6-29}~; \vref{De. 29:17-18} et \vref{Ap. 21:27}.

\DicoEntry{ABRAM}\textit{, de l'hébreu «~Abryram~»~: «~père élevé~»}\newline
Voir ABRAHAM.

\DicoEntry{ABRAHAM}\textit{, de l'hébreu «~'Abraham~»~: «~père d'une multitude~»}\newline
Hébreu, fils de Térach, et originaire d'Ur en Chaldée. Dieu lui demanda de quitter sa terre et sa famille pour Canaan, lui promettant que sa postérité hériterait de cette terre. De sa servante Agar, lui naquit un premier fils, Ismaël, ancêtre du peuple arabe. De sa femme Sara, lui naquit Isaac qui hérita des promesses. Il mourut à cent soixante-quinze ans. Voir \vref{Ge. 12:1-7}~; \vref{Ge. 17:4-13}~; \vref{Ge. 16}~; \vref{Ge. 21:1-8} et \vref{Ge. 25:7}.

\DicoEntry{ACACIA}\textit{, de l'hébreu «~shittah~»~: «~acacia, bois d'acacia~»}\newline
Arbre épineux poussant en abondance dans la péninsule du Sinaï et dans la vallée du Jourdain, il est aussi appelé bois de Sittim. Il fut l'un des matériaux utilisés pour la fabrication des objets du culte lévitique, dont l'arche*. Voir \vref{Ex. 25:10,13,23,28}.

\DicoEntry{ACHAB}\textit{, de l'hébreu «~Ach'ab~»~: «~un frère du père~»}\newline
Fils d'Omri, il fut roi d'Israël pendant vingt-deux ans. Marié à Jézabel*, fille du roi des Sidoniens, Achab et sa femme commirent de grandes abominations* et s'opposèrent au prophète Elie*. Voir \vref{1 R. 16:29-31}~; \vref{1 R. 18:1-40} et \vref{1 R. 22:29-40}.

\DicoEntry{ADAM}\textit{, de l'hébreu «~'Adam~»~: «~être humain~» ou «~de la terre~»}\newline
Premier homme, il vécut la première partie de sa vie dans le jardin d'Eden* avec sa femme Eve. Après avoir désobéi à Dieu en goûtant le fruit de l'arbre de la connaissance du bien et du mal, ils furent chassés du jardin. Adam fut le père de Caïn, Abel et Seth. Il mourut à neuf cent trente ans. Voir \vref{Ge. 2:7-8}~; \vref{Ge. 3}~; \vref{Ge. 4:1-2}, \vref{25-26} et \vref{Ge. 5:5}.

\DicoEntry{ADONIJA}\textit{, de l'hébreu «~'Adoniyah~»~: «~Yahweh est Seigneur~»}\newline
Quatrième fils de David et de Haggith. Peu avant la mort de son père, il s'autoproclama roi tentant en vain de prendre la place qui devait revenir à Salomon. Ce dernier lui laissa la vie sauve, mais le fit tuer plus tard alors qu'il semblait encore convoiter le trône d'Israël. Voir \vref{2 S. 3:4} et \vref{1 R. 1,2}.

\DicoEntry{ADOPTION}\textit{, du grec «~huiothesia~»~: «~adoption, adoption comme fils~»}\newline
Manifestation de l'amour éternel de Dieu, l'adoption permet à tout homme de devenir par la foi enfant de Dieu. Ce privilège, autrefois réservé au peuple d'Israël, fut étendu à toutes les nations par le sacrifice de Jésus. Cette adoption est manifestée par l'Esprit de Dieu qui témoigne à l'esprit du chrétien son appartenance à Dieu~; elle inclut les avantages du fils, dont l'héritage. Voir \vref{Jn. 1:12}~; \vref{Ga. 4:7}~; \vref{Ro. 8:15-17}~; \vref{Ro. 9:4}~; \vref{Ep. 1:5,11} et \vref{1 Jn. 3:1}.

\DicoEntry{AGAR}\textit{, de l'hébreu «~Hagar~»~: «~fuite~»}\newline
Servante égyptienne de Sara que cette dernière donna à Abraham comme concubine. Elle enfanta Ismaël, fils premier-né d'Abraham. Après la naissance d'Isaac, Abraham la chassa avec son fils. Voir \vref{Ge. 16} et \vref{Ge. 21:1-18}.

\DicoEntry{AGABUS}\textit{, de l'hébreu «~Chagab~» et du grec «~Agabos~»~: «~sauterelle~»}\newline
Prophète de Yahweh suscité au temps de l'Eglise primitive. Il prophétisa une famine qui se réalisa sous le règne de l'empereur Claude. Il annonça aussi l'arrestation de Paul à Jérusalem. Voir \vref{Ac. 11:27-28} et \vref{Ac. 21:10-33}.

\DicoEntry{AGGÉE}\textit{, de l'hébreu «~Chaggay~»~: «~en fête~» ou «~né un jour de fête~»}\newline
Prophète de Yahweh d'après la captivité, dont le livre éponyme figure dans le Tanakh.

\DicoEntry{AGNEAU}\textit{, de l'hébreu «~kebes~»~: «~agneau, brebis, jeune bélier~»}\newline
Animal sacrifié et consommé lors de la Pâque des juifs. Il préfigurait Christ, l'Agneau de Dieu qui ôte le péché du monde. Voir \vref{Ex. 12:1-28} et \vref{Jn. 1:29}.

\DicoEntry{AÏ}\textit{, de l'hébreu «~'Ay~»~: «~tas de ruines~»}\newline
Ville située au sud-est de Béthel, à proximité de laquelle Abraham dressa sa tente à deux reprises. Il s'agit également de la deuxième ville que Dieu livra entre les mains de Josué après la prise de Jéricho. Voir \vref{Ge. 12:8}~; \vref{Ge. 13:3} et \vref{Jos. 8}.

\DicoEntry{ALLÉLUIA}\textit{, de l'hébreu «~allelouia~»~: «~Louez Yahweh~»}\newline
Retrouvé à maintes reprises dans les Psaumes sous la forme «~Louez Yahweh~», cette exclamation encourage à célébrer Dieu et à se réjouir en lui. Voir \vref{Ap. 19:1-6}.

\DicoEntry{ALLIANCE}\textit{, de l'hébreu «~beriyth~»~: «~pacte, alliance, engagement~»}\newline
Dieu a conclu plusieurs alliances avec les hommes (ex~: Noé, Abraham, David). On distingue communément deux alliances majeures dans les Ecritures~: l'Ancienne Alliance - conclue avec Israël au travers de Moïse - et la Nouvelle Alliance inaugurée par Jésus-Christ. Voir \vref{Ge. 9:8-17}~; \vref{Ge. 17}~; \vref{Ex. 19-34}~; \vref{2 S. 7:12-16} et \vref{Hé. 9-13}.

\DicoEntry{ALPHA ET OMEGA}\textit{}\newline
Première et dernière lettre de l'alphabet grec, la combinaison de ces deux lettres mentionnées ensemble se rapporte à l'idée que Dieu est le premier et le dernier. Jésus fut présenté plusieurs fois comme étant «~l'alpha et l'oméga~» soulignant ainsi son caractère éternel. Voir \vref{Ap. 1:8}~; \vref{Ap. 21:6} et \vref{Ap. 22:13}.

\DicoEntry{ÂME}\textit{, de l'hébreu «~nephesh~»~: «~âme, une personne, la vie, être vivant~», «~ce qui respire~», «~ce qui a une vie par le sang~» et du grec «~psuche~»~: «~le souffle, la vie, l'âme~»}\newline
L'âme correspond au sang~; elle est le siège des émotions, de la volonté humaine et de l'intelligence. Avec l'esprit et le corps, l'âme constitue l'être humain. Voir \vref{Ge. 19:20}~; \vref{Ge. 44:30}~; \vref{Lé. 17:11}~; \vref{Mt. 10:28}~; \vref{Ac. 20:10} et \vref{1 Th. 5:23}.

\DicoEntry{AMEN}\textit{, de l'hébreu «~'amen~»~: «~assuré, établi~» ou «~ainsi soit-il~!~»}\newline
Se rapportant exclusivement à ce qui est sûr, avéré et certain, ce terme est souvent utilisé comme interjection. Christ est appelé «~l'Amen~», faisant référence à la vérité qu'il incarne. Voir \vref{Jé. 28:6}~; \vref{1 Ch. 16:36}~; \vref{2 Co. 1:20} et \vref{Ap. 3:14}.

\DicoEntry{AMOUR}\textit{}\newline
Il existe plusieurs traductions et définitions du mot «~amour~» en hébreu et en grec, elles varient selon le contexte.
\\- Les termes hébreux désignant l'amour~:
\\1. «~'Ahab~»~: «~amours~»
\\Amours, amis. Voir \vref{Os. 8:9} et \vref{Pr. 5:19}.
\\2. «~'Ahabah~»~: «~amour humain, amour de Dieu pour son peuple~»
\\Amour, affection, aimer. Voir \vref{De. 7:8}~; \vref{1 S. 20:17} et \vref{Pr. 10:12}.
\\3. «~Checed~»~: «~bonté, miséricorde, fidélité~»
\\Grâce, miséricorde, compassion, affection. Voir \vref{Ge. 40:14}~; \vref{Ex. 34:7} et Nb. \vref{14:19}.
\\4. «~Yediyd~»~: «~bien-aimé~»
\\Bien-aimé, amour. Voir \vref{De. 33:12} et \vref{Es. 5:1}.
\\- Les termes grecs désignant l'amour~:
\\1. «~Agape~»~: «~amour, charité, affection, bienveillance~»
\\Amour de Dieu, amour désintéressé que doit manifester l'homme né d'en haut. Voir \vref{Jn. 15:13}~; \vref{Jn. 17:26}~; \vref{1 Co. 8:1}~; \vref{1 Co. 13:3}~; \vref{Ro. 5:5} et \vref{1 Jn. 4:8}.
\\2. «~Eros~»~: «~l'amour qui prend~»
\\Amour dans la dimension sexuelle.
\\3. «~Phileo~»~: «~aimer, montrer des signes d'amour~»
\\Amour filial. Voir \vref{Jn. 21:17}~; \vref{1 Co. 16:22}.
\\4. «~Philadelphia~»~: «~amour fraternel~»
\\Amour des frères et sœurs d'une même famille, amour des chrétiens les uns pour les autres. Voir \vref{1 Th. 4:9}~; \vref{Ro. 12:10} et \vref{Hé. 13:1}.
\\5. «~Storge~»~: «~amour filial~»
\\Amour familial, affection naturelle. Voir \vref{Ro. 1:31}.

\DicoEntry{AMOS}\textit{, de l'hébreu «~'Amowc~»~: «~fardeau, porteur de fardeaux~»}\newline
Originaire de Tekoa en Juda, prophète de Yahweh dont le livre éponyme figure dans le Tanakh.

\DicoEntry{AMMONITES}\textit{, de l'hébreu «~'Ammown~»~: «~appartenant à la nation~»}\newline
Peuple issu de Ben-Ammi, né de l'inceste entre Lot et sa fille cadette. Ils furent ennemis d'Israël. Voir \vref{Ge. 19:30-38} et \vref{Ez. 25:1-7}.

\DicoEntry{ANAKIM}\textit{, de l'hébreu «~'Anaqiy~»~: «~au long cou~»}\newline
Descendants d'Anak, race de géants habitant Canaan avant sa conquête par le peuple d'Israël. Ils furent vaincus par Josué et Caleb qui hérita d'une partie de leur territoire. Voir \vref{No. 13:28-33}~; \vref{De. 9:1-3}~; \vref{Jos. 11:21-22} et \vref{Jos. 14:6-15}.

\DicoEntry{ANANIAS}\textit{, de l'hébreu «~Chananyah~»~: «~Dieu a été miséricordieux~»}\newline
1. Chrétien ayant vendu un champ avec sa femme Saphira et ayant fait croire qu'ils avaient donné la totalité du prix rapporté pour l'Eglise alors qu'ils en avaient secrètement gardé une partie. Ce mensonge les conduisit tous deux à la mort. Voir \vref{Ac. 5:1-10}.
\\2. Homme pieux vivant à Damas que le Seigneur envoya imposer les mains à Saul qui venait de se convertir afin qu'il recouvre la vue. C'est également lui qui le baptisa. Voir \vref{Ac. 9:10-18} et \vref{Ac. 22:12-16}.

\DicoEntry{ANATHÈME}\textit{, du grec «~anathema~»~: «~tout ce qui est livré au malheur~»}\newline
Terme désignant une personne ou une chose maudite, vouée à la destruction. Voir \vref{Ga. 1:8} et \vref{1 Co. 12:3}.

\DicoEntry{ANCIENS}\textit{, de l'hébreu «~zaqen~»~: «~vieux, aîné, de ceux qui ont de l'autorité~» et du grec «~presbuteros~»~: «~ayant de l'âge~»}\newline
Chez les juifs, il s'agissait des chefs de famille ou de clan qui représentaient le peuple dans les affaires religieuses et civiles. Voir \vref{Ex. 3:16}~; \vref{Lé. 4:15} et \vref{De. 31:28}. Sous la Nouvelle Alliance, les églises de la Galatie avaient élu des anciens («~presbuteros~») pour prendre soin des frères et sœurs. Il s'agit d'un terme relatif aux personnes ayant de l'âge et non à la fonction d'évêque*. Voir \vref{Ac. 14:23}~; \vref{1 Ti. 5:17}~; \vref{Tit. 1:5-9} et \vref{1 Pi. 5:1-5}.

\DicoEntry{ANDRÉ}\textit{, du grec «~Andreas~»~: «~virilité~»}\newline
Frère de Simon Pierre, originaire de Bethsaïda en Galilée, et pêcheur de métier. Il devint l'un des douze apôtres de Jésus-Christ. Voir \vref{Mt. 10:2}~; \vref{Mc. 1:16-17} et \vref{Jn. 1:40}.

\DicoEntry{ANGE}\textit{, de l'hébreu «~mal'ak~» et du grec «~aggelos~»~: «~messager, envoyé~»}\newline
Etre spirituel au service de Dieu pouvant prendre une forme humaine. Les anges sont au service de Yahweh pour des missions spécifiques au ciel ou sur la terre. Ils peuvent avoir une fonction de messager, protecteur ou combattant. Voir \vref{Da. 10:10-13}~; \vref{Lu. 1:26-38} et \vref{Ap. 12:7}.

\DicoEntry{ANNE}\textit{, de l'hébreu «~Channah~»~: «~grâce, faveur~»}\newline
1. Une des deux femmes d'Elkana. Stérile, elle pria Yahweh de lui accorder un fils qu'elle lui consacrerait. Elle enfanta ainsi Samuel qui entra au service de Yahweh dès son plus jeune âge. Voir \vref{1 S. 1,2}.
\\2. Fille de Phanuel de la tribu d'Aser, prophétesse. Veuve, elle servait le Seigneur nuit et jour dans le temple. Elle rencontra Jésus nourrisson, lorsqu'il fut amené au temple pour y être présenté à Dieu. Voir \vref{Lu. 2:36-38}.
\\3. Grand prêtre, beau-père de Caïphe*. Il participa à la conspiration qui mena Jésus à la croix. Voir \vref{Lu. 3:2} et \vref{Jn. 18:13}.

\DicoEntry{ANTICHRIST}\textit{, du grec «~antichristos~»~: «~l'adversaire du Messie~»}\newline
Aussi appelé «~homme impie~» et «~fils de la perdition~», personnage dont l'apparition se fera avant le retour glorieux du Seigneur. Il dominera le monde avant d'être vaincu par Christ. Voir \vref{2 Th. 2:1-4}~; \vref{2 Jn. 1:7} et \vref{Ap. 19:19-21}.

\DicoEntry{ANTIOCHE}\textit{, du grec «~Antiocheia~»~: «~rapide comme un char~»}\newline
Capitale de la Syrie, elle fut fondée en 300 av. J.-C. par Séleucus Nicator (358-281 av. J.-C.) qui la baptisa du nom de son père Antiochus. Cette ville accueillit des chrétiens en exil~; l'évangile y fut ainsi annoncé aux juifs puis aux Grecs et un grand nombre de personnes se convertirent. Barnabas et Paul y demeurèrent une année durant laquelle ils enseignèrent la parole. C'est à Antioche que les disciples furent appelés chrétiens pour la première fois. Voir \vref{Ac. 11:19-26}.

\DicoEntry{APIS}\textit{}\newline
Divinité égyptienne symbolisant la force et la fertilité. Il est représenté sous la forme d'un veau d'or ou d'un homme à tête de taureau dont les cornes entourent un disque solaire. Les Hébreux se corrompirent plusieurs fois en le vénérant. Voir \vref{Ex. 32:1-6} et \vref{1 R. 12:28-30}.

\DicoEntry{APOCALYPSE}\textit{, du grec «~apokalupsis~»~: «~mettre à nu, révélation d'une vérité, action de révéler~»}\newline
Dernier livre de la Bible écrit par Jean, ce récit comporte une révélation de la gloire de Jésus-Christ et raconte les derniers événements de l'histoire de l'humanité jusqu'à l'avènement de la Nouvelle Jérusalem.

\DicoEntry{APOLLOS}\textit{, du grec «~Apollos~»~: «~donné par Apollon~»}\newline
Juif érudit d'Alexandrie ayant une très bonne connaissance des Ecritures et enseignant avec exactitude au sujet de Jésus. Sa rencontre avec Aquilas et Priscille lui permit d'aller plus en profondeur dans la Parole et d'annoncer avec plus de force, notamment aux Juifs, que Jésus est le Messie en se basant sur les écrits du Tanakh. Il réalisa plusieurs voyages missionnaires notamment à Corinthe. Voir \vref{Ac. 18:24-28}, \vref{1 Co. 3:5-6} et \vref{1 Co. 16:12}.

\DicoEntry{APOSTASIE}\textit{, du grec «~apostasia~»~: «~action de s'éloigner de, désertion, défection~»}\newline
Abandon de la foi en Jésus-Christ et de la saine doctrine* se manifestant sous deux formes principales. Certaines personnes abandonnent ouvertement la foi, la communion avec Dieu et l'assemblée des saints. D'autres continuent de fréquenter les assemblées chrétiennes, mais ont laissé la saine doctrine pour s'attacher à des doctrines séductrices. Voir \vref{Mt. 24:11-12}~; \vref{2 Th. 2:3}~; \vref{1 Ti. 4:1-3}~; \vref{2 Pi. 2:1-3}~; \vref{2 Ti. 3:1-8}~; \vref{Jud. 1:17-19} et \vref{1 Jn. 4:1}.

\DicoEntry{APÔTRE}\textit{, du grec «~apostolos~»~: «~envoyé en avant, messager, ambassadeur~»}\newline
Lors de son service terrestre, Jésus choisit douze apôtres qu'il forma pour continuer l'œuvre après lui. Plusieurs autres apôtres furent suscités au temps de l'Eglise primitive, notamment Paul et Jacques, frère du Seigneur, qui avec Jean et Pierre furent les principaux instruments utilisés pour poser les fondements de la doctrine de l'Eglise. Le service apostolique existe encore aujourd'hui, mais la mission des apôtres actuels n'est pas d'écrire des épîtres, car la fondation a déjà été posée. Leur travail aujourd'hui consiste davantage à enseigner et veiller à ce que le fondement demeure. Voir \vref{Mc. 3:14}~; \vref{Ac. 15}~; \vref{Ga. 2:19}~; \vref{Ro. 1:1}~; \vref{Ep. 2:20} et \vref{Ep. 4:11}.

\DicoEntry{AQUILAS}\textit{, du latin «~Aquilas~»~: «~un aigle~» et PRISCILLE, du latin «~Priscilla~»~: «~petite vieille~»}\newline
Couple de Juifs ayant accepté l'évangile. Après avoir été chassés de Rome, ils s'installèrent à Corinthe où ils hébergèrent Paul à son arrivée et devinrent par la suite compagnons d'œuvre de ce dernier. Ils participèrent à plusieurs voyages missionnaires, notamment à Ephèse où ils enseignèrent Apollos*. Voir \vref{Ac. 18:1-3,18,24-26} et \vref{Ro. 16:3-5}.

\DicoEntry{ARBRE}\textit{, de l'hébreu «~'ets~»~: «~arbre, bois~»}\newline
Organisme vivant porteur de semence produisant des feuilles et des fruits selon les espèces. Lors de la création, Dieu créa différents arbres dont les fruits furent donnés pour nourrir l'homme et également deux arbres spécifiques placés au milieu du jardin d'Eden.

\DicoEntry{ARBRE DE LA CONNAISSANCE DU BIEN ET DU MAL}\textit{}\newline
Arbre dont le fruit contenait la connaissance du bien et du mal. Dieu interdit la consommation de ce dernier à l'homme sous peine de mort, mais Adam et Eve transgressèrent le commandement. C'est ainsi que le péché et la mort régnèrent sur l'humanité. Voir \vref{Ge. 2:17}~; \vref{Ge. 3:1-6} et \vref{Ro. 5:12}.

\DicoEntry{ARBRE DE VIE}\textit{}\newline
Arbre dont la consommation donne la vie éternelle. Après la chute d'Adam et Eve, Dieu les chassa du jardin pour les empêcher d'y accéder. L'arbre de vie se trouve dans la ville sainte, la Nouvelle Jérusalem~; ses feuilles servent à la guérison des nations. Voir \vref{Ge. 3:22-24} et \vref{Ap. 22:2,14,19}.

\DicoEntry{ARC-EN-CIEL}\textit{, de l'hébreu «~qesheth~»~: «~arc~»}\newline
Signe de l'alliance* que Dieu conclut avec Noé et les générations qui le suivraient suite au déluge*. Cette alliance stipulait que Yahweh ne détruirait plus les hommes par les eaux. Voir \vref{Ge. 9:12-17}.

\DicoEntry{ARCHANGES}\textit{, du grec «~archaggelos~»~: «~chef des anges~»}\newline
Catégorie d'anges* ayant un rang et une dignité plus élevés que les autres. Voir \vref{1 Th. 4:16} et \vref{Jud. 1:9}.

\DicoEntry{ARCHE DE NOÉ}\textit{, de l'hébreu «~tebah~»~: «~arche, vaisseau, coffre~»}\newline
Embarcation construite par Noé pour le sauver lui, sa famille ainsi que les animaux, du déluge* qui allait s'abattre sur la terre. Voir \vref{Ge. 6:8-16}~; \vref{Mt. 24:37-39} et \vref{Lu. 17:26-27}.

\DicoEntry{ARCHE DU TEMOIGNAGE ou DE L'ALLIANCE}\textit{, de l'hébreu «~'arown~»~: «~arche, coffre, cercueil~»}\newline
Coffre rectangulaire en bois d'acacia* recouvert d'or pur, contenant les tables de l'alliance, la verge d'Aaron et une urne contenant un échantillon de la manne. Construite selon le modèle que Moïse avait reçu au mont Sinaï, elle était couverte par le propitiatoire*. L'arche fut placée dans le Saint des saints du tabernacle*, puis du temple*. Voir \vref{Ex. 25:10-22}~; \vref{1 R. 8:6}~; \vref{2 R. 25:8-9} et \vref{Hé. 9:4}.

\DicoEntry{ARTAXERXÈS}\textit{, (règne~: 465 av. J.-C.- 424 av. J.-C.), du persan «~Artachshashta~»~: «~celui qui fait régner la loi sacrée~»}\newline
Fils d'Assuérus*, roi de Perse. Il autorisa Esdras à retourner à Jérusalem avec des prêtres et des Lévites pour effectuer le sacerdoce dans le temple et faire respecter la loi de Yahweh. Voir \vref{Esd. 7:11-28}.

\DicoEntry{ASAPH}\textit{, de l'hébreu «~'Acaph~»~: «~celui qui rassemble, collecteur~»}\newline
Lévite et chef des chantres sous David, il participa au transfert de l'arche* à Jérusalem et écrivit certains psaumes. Voir \vref{1 Ch. 15:16-19} et \vref{1 Ch. 16:4-7}.

\DicoEntry{ASER}\textit{, de l'hébreu «~'Asher~»~: «~heureux~»}\newline
Fils de Jacob et de Zilpa, servante de Léa, il est le père de la tribu d'Aser. Voir \vref{Ge. 30:13}.

\DicoEntry{ASHERAH}\textit{, de l'hébreu «~'Asherah~»~: «~pieu sacré~».}\newline
Voir commentaire en \vref{Ex. 34:13}.

\DicoEntry{ASSUÉRUS ou XERXÈS Ier}\textit{, (485 av. J.-C. – 465 av. J.-C.), du persan «~'Achashverowsh~»~: «~je serai silencieux et pauvre~».}\newline
Père d'Artaxerxès, roi de Perse et époux d'Esther*. Voir le livre d'Esther.

\DicoEntry{ASTARTE}\\textit{, de l'hébreu «~Ashtoreth~»~: «~étoile~».}\newline
Voir commentaire en \vref{Jg. 2:13}.

\DicoEntry{AUTEL}\textit{, de l'hébreu «~mizbeach~»~: «~autel~»}\newline
Table généralement façonnée avec des monticules de pierres ou en terre et élevée spécialement pour offrir des holocaustes* et des sacrifices en l'honneur de Dieu. Voir \vref{Ge. 12:7}~; \vref{Ge. 35:7}~; \vref{Ex. 20:24-26} et \vref{Ex. 30:1-8}.

\DicoEntry{BAAL}\textit{, de l'hébreu «~Ba'al~»~: «~maître, possesseur, seigneur~»}\newline
Dieu primaire des Phéniciens et des Cananéens auquel les Israélites s'attachèrent à plusieurs reprises pour l'adorer. Voir \vref{No. 25:3}~; \vref{Jg. 2:11} et \vref{1 R. 18:21}. Voir aussi commentaire en \vref{Jg. 2:11}.

\DicoEntry{BABEL ou BABYLONE}\textit{, de l'hébreu «~Babel~»~: «~confusion (par le mélange)~»}\newline
Ville de Mésopotamie* située sur l'Euphrate, capitale de la Babylonie. Les hommes y entreprirent la construction de la tour de Babel. Cependant, Yahweh confondit leur langage et les dispersa sur toute la terre. Voir \vref{Ge. 10:8-10} et \vref{Ge. 11:1-9}.

\DicoEntry{BALAAM}\textit{, de l'hébreu «~Bil'am~»~: «~sans peuple~», «~dévorant~»}\newline
Prophète de Yahweh ayant vécu pendant la marche d'Israël dans le désert, il fut séduit par Balak, roi de Moab, qui lui proposa de maudire Israël contre de généreux présents. Son témoignage a été utilisé plusieurs fois pour avertir les enfants de Dieu des scandales* dont ils pourraient être la cause en suivant la voie de la cupidité. Voir \vref{No. 22-24}~; \vref{No. 31:8}~; \vref{Jud. 1:11} et \vref{Ap. 2:14}.

\DicoEntry{BALAK}\textit{, de l'hébreu «~Balaq~»~: «~gaspilleur, dévastateur~»}\newline
Roi de Moab, il essaya de convaincre Balaam* de maudire Israël qu'il redoutait. Voir \vref{No. 22-24}.

\DicoEntry{BANNIÈRE}\textit{, de l'hébreu «~nec~»~: «~quelque chose de levé, étendard, signal, enseigne~»}\newline
Drapeau, étendard élevé en signe d'appartenance à ce qu'il représente. Moïse bâtit un autel du nom de Yahweh-Nissi~: «~Yahweh ma bannière~». Voir \vref{Ex. 17:15}~; \vref{Es. 11:10,12} et \vref{Ps. 60:4}.

\DicoEntry{BAPTÊME}\textit{, du grec «~baptizo~»~: «~plonger, immerger, purifier en plongeant~»}\newline
On distingue trois types de baptêmes dans les Ecritures (\vref{Mt. 3:11})~:
\\1. le baptême d'eau~: acte suivant la conversion par lequel une personne est immergée dans l'eau - symbolisant la mort et la résurrection en Jésus-Christ. Il s'agit selon Pierre de l'«~engagement d'une bonne conscience envers Dieu~». Voir \vref{Ac. 2:38}~; \vref{Ac. 16:30-33}~; \vref{Col. 2:12-13} et \vref{1 Pi. 3:21}.
\\2. le baptême du Saint-Esprit~: lors de la naissance d'en haut, gage que le Seigneur donne au nouveau converti par l'envoi du Saint-Esprit. Voir \vref{Jn. 3:5-6}~; \vref{Tit. 3:4-7} et \vref{Ep. 1:13}.
\\3. le baptême de feu~: symbole des souffrances que Christ a endurées à la croix et par lesquelles tous les chrétiens sont appelés à passer pour être purifiés. Voir \vref{Mc. 10:35-39}~; \vref{Lu. 12:50}~; \vref{1 Pi. 1:6-9} et \vref{1 Pi. 4:12-13}.

\DicoEntry{BARAK}\textit{, de l'hébreu «~Baraq~»~: «~éclairs, foudre~»}\newline
Fils d'Abinoam, issu de la tribu de Nephtali, il vécut en Israël au temps des juges. Encouragé et accompagné par Débora, il battit l'armée de Jabin, roi de Canaan. Voir \vref{Jg. 4}.

\DicoEntry{BARTHÉLÉMY}\textit{, du grec «~Bartholomaios~»~: «~fils de Tolmaï~»}\newline
Un des douze apôtres de Jésus. Voir \vref{Mt. 10:3}.

\DicoEntry{BARTIMÉE}\textit{, du grec «~Bartimaios~»~: «~fils de Timée~»}\newline
Fils de Timée, mendiant aveugle que Jésus guérit suite à ses cris de supplications sur la route de Jéricho. Voir \vref{Mc. 10:46-52}.

\DicoEntry{BATH-SCHEBA}\textit{, de l'hébreu «~Bath-Sheba`~»~: «~fille d'un serment~»}\newline
Fille d'Eliam, femme d'Urie* le Héthien que David fit mourir après l'avoir mise enceinte. Elle devint la femme de David et fut la mère de Salomon. Voir \vref{2 S. 11:3-5,26-27} et \vref{2 S. 12:24-25}.

\DicoEntry{BEELZÉBUL}\textit{, de l'hébreu «~Ba'al-Zebuwb~»~: «~seigneur des mouches~»}\newline
Divinité adorée par les Philistins et considérée comme le prince des démons. Voir \vref{2 R. 1:2-6} et \vref{Mc. 3:22-26}.

\DicoEntry{BÉLIAL}\textit{, de l'hébreu «~Beliya'al~»~: «~indignité~»}\newline
Symbolisant l'infidélité, la méchanceté et la perversité, il s'agit d'un autre nom de Satan. Voir \vref{De. 15:9}~; \vref{1 S. 1:16}~; \vref{2 Co. 6:15}.

\DicoEntry{BÉNÉDICTION}\textit{, de l'hébreu «~barak~», «~berakah~», et du grec «~eulogia~»~: «~louange~»}\newline
Parole au travers de laquelle le Seigneur annonce sa grâce sur la vie d'une personne ou d'un peuple~; les bontés liées à la bénédiction sont cependant conditionnées par l'obéissance du bénéficiaire. Sous l'Ancienne Alliance, les pères avaient coutume de bénir leurs enfants~; la bénédiction se manifestait souvent par la prospérité matérielle, la fécondité et la santé. La bénédiction est la marque du chrétien qui voit avec un œil spirituel la faveur de Dieu dans sa vie et qui bénit Dieu dans toutes les circonstances. Voir \vref{Ge. 49:1-28}~; \vref{De. 28:1-14}~; \vref{Ps. 103:1-2} et \vref{Ep. 1:3}.

\DicoEntry{BENJAMIN}\textit{, de l'hébreu «~Binyamiyn~»~: «~fils de ma main droite~»}\newline
Dernier fils de Jacob et Rachel~; sa mère mourut en lui donnant naissance. Il est l'ancêtre de la tribu de Benjamin. Voir \vref{Ge. 35:16-18} et \vref{Ge. 49:27}.

\DicoEntry{BÊTE}\textit{, de l'araméen «~cheyva´~» et du grec «~therion~»~: «~bête, animal~»}\newline
Dans les récits à caractère apocalyptique, les bêtes sont des animaux symbolisant les puissances politiques. Voir \vref{Da. 7} et \vref{Ap. 13,17}.

\DicoEntry{BÉTHANIE}\textit{, du grec «~Bethania~»~: «~maison des dattes non mûres~», «~maison de l'affligé~»}\newline
Village proche de Jérusalem, près de la Montagne des Oliviers, où vivaient Simon le lépreux, Marthe*, Marie* et Lazare* que Jésus ressuscita des morts. Voir \vref{Mc. 11:1}~; \vref{Mc. 14:3} et \vref{Jn. 11:1}.

\DicoEntry{BÉTHEL}\textit{, de l'hébreu «~Beyth-'El~»~: «~maison de Dieu~»}\newline
Ville cananéenne située à l'occident de Aï. Autrefois appelé Luz - mais renommé par Jacob quand il y eut la visitation de Yahweh - Béthel devient la possession de la tribu d'Ephraïm lors de la conquête de Canaan conduite par Josué. Elle était connue pour être un lieu d'adoration où on y rendait un culte à Yahweh. Malheureusement suite au schisme d'Israël - et notamment sous le règne de Jéroboam, roi de Juda - elle devient un lieu d'abomination. C'est Josias son successeur qui, désirant marcher avec Yahweh, y ôta les faux dieux, rétablissant ainsi le culte en l'honneur du Dieu d'Israël. (\vref{Ge. 28:10-22}~; \vref{Ge. 31:13}~; \vref{1 R. 12:26-32}~; \vref{2 R. 23:1-15})

\DicoEntry{BETHLEHEM}\textit{, de l'hébreu «~Beyth Lechem~»~: «~maison du pain~»}\newline
Ville de Juda, lieu de naissance de David et de Jésus-Christ. Voir \vref{1 S. 16}~; \vref{Mt. 2:16} et \vref{Lu. 2:4-7}.

\DicoEntry{BIBLE}\textit{, du grec «~biblia~»~: «~livres~»}\newline
Aussi appelée «~Parole de Dieu~», recueil de livres inspirés de Dieu et utiles pour enseigner, convaincre, corriger et instruire dans la justice. Voir \vref{2 Ti. 3:16}.

\DicoEntry{BLASPHÈME}\textit{, de l'hébreu «~na'ats~»~: «~repousser, mépriser, rejeter~» et du grec «~blasphemia~»~: «~discours impie et injurieux envers Dieu~»}\newline
Parole outrageante ou insultante envers Dieu. Voir \vref{2 S. 12:14} et \vref{Ap. 16:9}.

\DicoEntry{BLASPHÈME CONTRE LE SAINT-ESPRIT}\textit{}\newline
Voir commentaire \vref{Mt. 12:22-32}.

\DicoEntry{BOAZ}\textit{, de l'hébreu «~Bo`az~»~: «~en lui est la force~»}\newline
Fils de Salmon et arrière-grand-père du roi David, il épousa Ruth la Moabite. Voir \vref{Ru. 4:13} et \vref{Mt. 1:1-6}.

\DicoEntry{BREBIS}\textit{}\newline
Femelle du bélier, c'est l'animal pour qui le berger donne sa vie. Elle est le symbole du véritable disciple qui n'obéit qu'à la voix de son Maître et qui se laisse conduire et choyer par Jésus, le bon berger. Voir \vref{Jn. 10:1-16}.

\DicoEntry{CAIN}\textit{, de l'hébreu «~Qayin~»~: «~possession~», «~artisan, forgeron~»}\newline
Fils aîné d'Adam et Eve, il fut l'auteur du premier homicide en tuant son frère Abel. Il engendra Lémec, premier polygame de l'histoire. Voir \vref{Ge. 4:1-8,16-19}.

\DicoEntry{CAÏPHE}\textit{, du grec «~Kaiaphas~»~: «~avenant, pierre~»}\newline
Grand prêtre nommé par Valerius Gratus, gouverneur de Judée de 15 à 26 ap. J.-C. Caïphe exerça sa fonction de 18 à 36. N'ayant pas reconnu en Christ le Messie, il déclara néanmoins qu'il était avantageux qu'un seul homme meure pour le peuple et participa à la condamnation à mort de Jésus. Voir \vref{Mt. 26:3,57-66}~; \vref{Jn. 11:47-53} et \vref{Jn. 18:12-14}.

\DicoEntry{CALEB}\textit{, de l'hébreu «~Kaleb~»~: «~chien~»}\newline
Fils de Jephunné, issu de la tribu de Juda, il fut l'un des espions envoyés pour explorer le pays de Canaan. Avec Josué, il fut le seul, parmi la génération sortie d'Egypte, à entrer dans la terre promise. Voir \vref{No. 13:1-6} et \vref{No. 14:22-30}.

\DicoEntry{CALENDRIER HEBRAÏQUE}\textit{}\newline
Nisan (ou Abib) = Mars~; Iyyar (ou Ziv) = Avril~; Sivan = Mai~; Thammuz = Juin~; Ab = Juillet~; Elul = Août~; Tisri (ou Ethanim) = Septembre~; Marchesvan (ou Bul) = Octobre~; Chislev (ou Kisleu) = Novembre~; Tébeth = Décembre~; Schebat = Janvier~; Adar = Février

\DicoEntry{CAMP}\textit{, de l'hébreu «~machaneh~»~: «~campement, camp~»}\newline
Lieu de stationnement temporaire d'un groupement civil ou militaire. Voir \vref{Ge. 32:2} et \vref{Ex. 14:19}.

\DicoEntry{CANAAN}\textit{, de l'hébreu «~Kena'an~»~: «~terre basse~», «~marchand~»}\newline
Fils de Cham. Ses descendants occupèrent la région éponyme qui correspond plus ou moins aujourd'hui aux territoires réunissant la Palestine, l'État d'Israël, l'ouest de la Jordanie, le sud du Liban et l'ouest de la Syrie. Ce territoire correspondait également à la terre promise par Dieu aux Israélites dont ils prirent possession sous la conduite de Josué. Voir \vref{Ge. 9:18}~; \vref{Jos. 6-21} et \vref{Ac. 13:19}.

\DicoEntry{CÉSAR, Jules}\textit{, (100 av. J.-C - 44 av. J.-C.) du latin «~kaisar~»~: «~séparé~», «~chef~»}\newline
Général romain. Son nom devint par la suite celui de certains empereurs romains. Dans les Ecritures, César symbolise également les autorités séculières. Voir \vref{Mt. 22:21}.

\DicoEntry{CESARÉE de Philippes}\textit{, du grec «~Kaisereia~»~: «~appartenant à César~»}\newline
Située près des sources du Jourdain, territoire qui doit son nom à l'empereur Tibère. C'est dans cette contrée que Pierre reconnut en Jésus le Messie, le Fils du Dieu vivant. Voir \vref{Mt. 16:13-17}.

\DicoEntry{CHAIR}\textit{, du grec «~sarx~»~: «~la chair, le corps, la nature sensuelle de l'homme, la nature animale~»}\newline
Selon le contexte, désigne le corps humain, l'être humain ou la nature humaine conduite par le péché*. Voir \vref{Lu. 3:6}~; \vref{Lu. 24:39}~; \vref{Jn. 17:2}~; \vref{Ga. 5:16-21}~; \vref{Ro. 8:5-9} et \vref{Ep. 2:3}.

\DicoEntry{CHALDÉE}\textit{, de l'hébreu «~Kasdiy~»~: «~briseurs de mottes~», «~comme des démons~»}\newline
Région située au sud de la Mésopotamie dont Abraham est originaire. Voir \vref{Ge. 11:28}.

\DicoEntry{CHAM}\textit{, de l'hébreu «~Cham~»~: «~chaud, bouillant~»}\newline
Fils de Noé et père de Canaan qui fut maudit par Noé. Voir \vref{Ge. 9:18-27}.

\DicoEntry{CHARAN}\textit{, de l'hébreu «~Charan~»~: «~montagnard~», «~route, caravane~»}\newline
Région proche d'Ur en Chaldée où Abraham séjourna jusqu'à la mort de son père Térach. Voir \vref{Ge. 11:31} et \vref{Ge. 12:4}.

\DicoEntry{CHEMIN DE SABBAT}\textit{}\newline
Selon la loi de Moïse, distance maximum que les juifs peuvent parcourir de leur demeure le jour du sabbat* (cf. tableau des mesures et.distances). Voir \vref{Ac. 1:12}.

\DicoEntry{CHÉRUBINS}\textit{, de l'hébreu «~keruwb~»~: «~être angélique, chérubin~»}\newline
Catégorie d'anges portant et.ou gardant la gloire de Dieu. Yahweh en avait placé à l'entrée du jardin d'Eden pour empêcher l'homme d'y accéder. Deux chérubins sur lesquels Dieu siégeait étaient représentés sur le propitiatoire*. Avant sa chute, Satan était un chérubin protecteur. Voir \vref{Ge. 3:24}~; \vref{Ex. 25:17-20}~; \vref{Es. 37:16} et \vref{Ez. 28:14}.

\DicoEntry{CHRÉTIEN}\textit{, du grec «~christanos~»~: «~de Christ~», «~petit christ~», «~comme Christ~»}\newline
Comme son étymologie le suggère, le chrétien appartient à Christ, dont il a la nature et à qui il ressemble. Il est donc un disciple* de Jésus-Christ qui suit son enseignement et le met en pratique. Ce terme fut employé pour la première fois à Antioche. Voir \vref{Ac. 11:26}.

\DicoEntry{CHRIST}\textit{, du grec «~christos~» et de l'hébreu «~mashiyach~»~: «~oint~»}\newline
Souvent accolé au nom de Jésus*, ce terme suggère que ce dernier est l'oint* de Dieu, le Messie tant attendu. Jésus annonça l'émergence de faux christs (= faux ouvriers de Christ) à la fin des temps. Voir \vref{Ro. 1:1}~; \vref{Mt. 16:15-16}~; \vref{Mt. 24:24} et \vref{Mc. 13:22-23} et \vref{Hé. 1:9}.

\DicoEntry{CIRCONCISION}\textit{, de l'hébreu «~muwlah~»~: «~circoncision~: couper autour~»}\newline
Section et ablation du prépuce. En signe d'alliance, Dieu ordonna à Abraham de circoncire tous les mâles de sa maison~; les enfants d'Israël ont perpétré cette pratique. Sous la Nouvelle Alliance, la circoncision requise est celle du cœur. Voir \vref{Ge. 17:9-14}~; \vref{Lu. 1:59}~; \vref{1 Co. 7:19} et \vref{Ro. 2:25-29}.

\DicoEntry{CLAUDE}\textit{, (10 av. J.-C. – 54 ap. J.-C.), du grec «~Klaudios~»~: «~boiteux~»}\newline
Fils de Nero Claudius Drusus (38 av. J.-C. – 9 av. J.-C.). Empereur romain qui régna de 41 à 54 ap. J.-C.~; il chassa les Juifs de Rome, parmi lesquels Aquilas et Priscille. Voir \vref{Ac. 18:2}.

\DicoEntry{CLERGÉ}\textit{, du grec «~klêrikos~»~: «~homme d'église~»}\newline
Au sein de l'Eglise catholique, corps séparé des fidèles ayant une fonction gouvernante~; ses membres sont appelés les clercs ou les ecclésiastiques. Ils accèdent à leur position par le sacrement de l'ordre (ou ordination*) qui comporte trois classes~: les diacres, les prêtres et les évêques.

\DicoEntry{CLÉRICALISME}\textit{, dérivé de clérical~: «~dévoué aux intérêts du clergé~»}\newline
Tendance en vertu de laquelle le clergé sort du domaine religieux pour se mêler des affaires publiques et politiques afin d'y exercer une influence et faire prédominer ses idées.

\DicoEntry{CŒUR}\textit{, de l'hébreu «~lebab~»~: «~homme intérieur, volonté, cœur, partie interne, pensée~»}\newline
Organe permettant la circulation du sang, les Ecritures définissent le cœur comme un grand abîme. Siège des émotions et des pensées intimes, il peut être une bonne ou une mauvaise source. Voir \vref{Ge. 20:6}~; \vref{Lé. 19:17}~; \vref{De. 4:29}~; \vref{1 S. 12:24} et \vref{Mc. 7:21}.

\DicoEntry{COLOSSES}\textit{, du grec «~Kolossai~»~: «~monstruosités~»}\newline
Située en Asie Mineure, ville de Phrygie se trouvant à environ deux cents kilomètres d'Ephèse. Il s'y trouvait une église à qui Paul écrivit une lettre qui figure dans le canon biblique.

\DicoEntry{COMMUNION}\textit{, du grec «~koinonia~»~: «~ce qui est commun à plusieurs personnes, association, union~»}\newline
Le disciple* de Christ est appelé à vivre deux types de communion. Il doit tout d'abord être en communion intime avec Dieu puis avec d'autres membres du corps de Christ pour vivre la communion fraternelle. Voir \vref{Ps. 133}~; \vref{Ac. 2:42}~; \vref{2 Co. 13:11-13} et \vref{1 Jn. 1:3}.

\DicoEntry{CONCILE}\textit{, du latin «~concilium~»~: «~assemblée~»}\newline
Assemblée d'évêques de l'Eglise catholique (également connue sous l'appellation «~pères de l'Eglise catholique~») réunis dans le but de définir les règles de la foi chrétienne. Cette pratique va à l'encontre du message de Christ puisqu'il a strictement condamné la modification du message qu'il a lui-même prêché et confié aux apôtres*. Voir \vref{Mt. 5:18} et \vref{Ga. 1:8-9}.

\DicoEntry{CONFESSION}\textit{, du grec «~exomologeo~»~: «~confesser, professer, reconnaître ouvertement~»}\newline
On peut confesser des péchés pour exposer les ténèbres ou le nom du Seigneur pour le louer et annoncer la vérité. Voir \vref{Mc. 1:5}~; \vref{Ac. 19:18} et \vref{Ph. 2:11}.

\DicoEntry{CONVERSION}\textit{, du grec «~epistrepho~»~: «~action de se retourner, de se tourner vers~»}\newline
Fruit d'une sincère repentance, la conversion est la décision de se tourner vers Christ et de se détourner des œuvres des ténèbres. Voir \vref{Ac. 26:20}~; \vref{Ga. 4:9}~; \vref{2 Co. 3:16} et \vref{1 Pi. 2:25}.

\DicoEntry{CONVOITISE}\textit{, du grec «~epithumia~»~: «~désir, convoitise, luxure~»}\newline
Précédant l'acte du péché, désir amorcé par les sens humains et lié à la soif de posséder ce qui est défendu et ce que le monde offre. Voir \vref{Ja. 1:14-15} et \vref{1 Jn. 2:15-17}.

\DicoEntry{CORINTHE}\textit{, du grec «~Korinthos~»~: «~rassasié~»}\newline
Dans l'Antiquité, Corinthe, capitale de l'Achaïe, était la ville la plus prospère et la plus puissante de Grèce. Située sur un isthme séparant la mer Egée de la mer Ionienne, Corinthe était au carrefour de l'Asie et de l'Italie et constituait un véritable centre commercial où les produits orientaux et occidentaux se croisaient. Paul demeura au moins un an et six mois à Corinthe, durée pendant laquelle il enseigna la parole de Dieu. Il écrivit par la suite deux lettres aux saints de cette ville qu'on retrouve dans le canon biblique.

\DicoEntry{CORNEILLE}\textit{, du grec «~Kornelios~»~: «~d'une corne~»}\newline
Centenier romain juste et craignant Dieu. Il vivait à Césarée où Simon Pierre fut envoyé pour lui annoncer la Parole. Au travers de l'expérience de Corneille, Dieu confirma que le salut était pour toutes les nations. Voir \vref{Ac. 10}.

\DicoEntry{COURONNE}\textit{, du grec «~stephanos~»~: «~couronne, une marque de rang royal, récompense de la justice, ornement~»}\newline
Jésus-Christ reçut une couronne d'épines lors de la crucifixion pour rappeler ironiquement son titre de «~roi des Juifs~». Après la résurrection, les chrétiens recevront une couronne en récompense de leur intégrité. Devant le trône de Dieu, les vingt-quatre vieillards jettent leurs couronnes pour rendre gloire à Dieu. Voir \vref{Mt. 27:29}~; \vref{Ja. 1:12}~; \vref{1 Co. 9:25}~; \vref{1 Pi. 5:4}~; \vref{2 Ti. 4:8}~; \vref{Ap. 2:10} et \vref{Ap. 4:4,10}.

\DicoEntry{CROIX}\textit{, du grec «~stauros~»~: «~pieu, croix~»}\newline
Châtiment romain consistant à clouer les mains et les pieds des condamnés sur des poteaux en bois en forme de croix. Symbole du sacrifice de Jésus pour le pardon des péchés, la croix est aussi l'image de la vie de souffrance et de consécration totale à laquelle est appelé tout disciple du Seigneur. Voir \vref{Es. 53}~; \vref{Mt. 16:24} et \vref{Lu. 9:23}.

\DicoEntry{CUPIDITÉ}\textit{, du grec «~pleonexia~»~: «~désir avide d'avoir plus, avarice~»}\newline
Forme d'idolâtrie*, péché consistant à désirer de manière excessive les biens de ce monde (argent, richesses, etc.) et menant à la perdition. Voir \vref{Ep. 5:3}~; \vref{Col. 3:5} et \vref{2 Pi. 2:14}.

\DicoEntry{CYRÈNE}\textit{, du grec «~Kurene~»~: «~suprématie de la bride~», «~qui gouverne, froid~»}\newline
Ville prospère située dans la région fertile d'Afrique du Nord (actuelle Libye) où vivait une importante communauté juive et de laquelle était originaire Simon à qui l'on demanda de porter la croix de Jésus. Voir \vref{Mc. 15:20-22} et \vref{Ac. 2:10}.

\DicoEntry{CYRUS II LE GRAND}\textit{(règne~: 559-530 av. J.-C.), du persan «~Kowresh~»~: «~possède la puissance, puissance suprême~»}\newline
Fils de Cambyse, il régna sur l'Empire perse. Réveillé par Yahweh, il publia un édit en faveur du retour des Juifs à Jérusalem pour la reconstruction du temple. Voir \vref{Esd. 1:1-2} et \vref{2 Ch. 36:22-23}.

\DicoEntry{DAGON}\textit{, «~Dagown~»~: «~un poisson~»}\newline
Divinité païenne adorée par les Philistins, il était représenté par un personnage avec des mains et une face humaine et le corps d'un poisson. Voir \vref{1 S. 5:1-5}.

\DicoEntry{DAN}\textit{, de l'hébreu «~dan~»~: «~un juge~»}\newline
Fils de Jacob et de Bilha, servante de Rachel, il est le père de la tribu des Danites. Voir \vref{Ge. 30:1-6} et \vref{Ge. 49:16-18}.

\DicoEntry{DANIEL}\textit{, de l'hébreu «~Daniye'l~»~: «~Dieu est mon juge~»}\newline
Issu d'une famille princière de Juda, il fut déporté pendant sa jeunesse de Jérusalem à Babylone où il reçut le nom de Beltshatsar. Son histoire est racontée dans le livre éponyme.

\DicoEntry{DARIQUE}\textit{, de l'hébreu «~darkemown~»~: «~darique, drachme, unité de mesure~»}\newline
Utilisée après le retour de l'exil babylonien, monnaie d'or mise en place par le roi Darius et circulant dans l'Empire perse. Voir \vref{Esd. 8:26-27} et \vref{Né. 7:71-72}.

\DicoEntry{DARIUS Ier}\textit{, (règne~: 522 av. J.-C. – 486 av. J.-C.), de l'hébreu «~Dar`yavesh~»~: «~seigneur~» (origine~: perse)}\newline
Fils d'Assuérus, d'origine mède, roi des Chaldéens. Il encouragea la reconstruction du temple de Jérusalem après la découverte des instructions laissées par Cyrus sur un rouleau retrouvé dans la province de Médie. Voir \vref{Esd. 6}.

\DicoEntry{DATHAN}\textit{, de l'hébreu «~Dathan~»~: «~appartenant à une fontaine~»}\newline
Issu de la tribu de Ruben, fils d'Eliab et frère d'Abiram, il participa avec Koré à la révolte contre Moïse et Aaron. Voir \vref{No. 16:1-35}.

\DicoEntry{DAVID}\textit{, de l'hébreu «~David~»~: «~bien aimé~»}\newline
Issu de la tribu de Juda et dernier fils d'Isaï, il entra dès son plus jeune âge au service du roi Saül avant de devenir roi d'Israël. Homme selon le cœur de Dieu, il connut de grands succès sur les champs de bataille et fut l'auteur de nombreux psaumes. Il régna quarante-quatre ans sur Israël puis son fils Salomon* lui succéda. Voir \vref{1 S. 13:14}~; \vref{1 S. 16:14-23}~; \vref{1 S. 17}~; \vref{1 R. 2:10-11} et \vref{Ac. 13:22}.

\DicoEntry{DÉBORA}\textit{, de l'hébreu «~Debowrah~»~: «~abeille~»}\newline
Femme de Lapiddoth, elle exerça les fonctions de prophétesse et juge en Israël. Elle fut utilisée par Dieu pour prophétiser la victoire d'Israël sur Canaan par Barak qu'elle accompagna sur le champ de bataille. Voir \vref{Jg. 4-5}.

\DicoEntry{DÉLUGE}\textit{, de l'hébreu «~mabbuwl~»~: «~inondation, déluge~»}\newline
Pluie torrentielle s'étant abattue sur la terre pendant quarante jours et quarante nuits au temps de Noé. Le déluge symbolisait le jugement de Dieu sur une génération dont la méchanceté avait atteint un niveau sans précédent. Tous les habitants et les animaux de la terre furent emportés par les eaux du déluge hormis Noé, sa famille et les animaux qui étaient avec eux dans l'arche*. Voir \vref{Ge. 6-8}.

\DicoEntry{DEMAS}\textit{, du grec «~Demas~»~: «~gouverneur du peuple~»}\newline
Compagnon d'œuvre de Paul qui le délaissa «~par amour pour le siècle présent~». Voir \vref{Col. 4:14} et \vref{2 Ti. 4:10}.

\DicoEntry{DEMETRIUS}\textit{, du grec «~Demetrios~»~: «~qui appartient à Déméter (déesse grecque de l'agriculture)~»}\newline
Orfèvre qui fabriquait des statues de la déesse Diane à Ephèse. Voyant son commerce mis en danger par les prédications de Paul, il déclencha une émeute contre ce dernier. Voir \vref{Ac. 19:23-41}.

\DicoEntry{DÉMONS}\textit{, du grec «~daimonion~»~: «~divinité inférieure, mauvais esprit, ministres du diable~»}\newline
Egalement appelés «~esprits impurs~», anges* déchus ayant pris part à la révolte et à la chute de Satan*. Ils peuvent posséder le corps d'une personne, mais sont soumis à la puissance de Jésus, au nom duquel les chrétiens peuvent les chasser. Voir \vref{Mt. 10:8}~; \vref{Mc. 7:26}~; \vref{Mc. 16:17}~; \vref{Lu. 4:33}~; \vref{Lu. 10:17}~; \vref{Jud. 1:6} et \vref{Ap. 12:4}.

\DicoEntry{DIABLE}\textit{}\newline
Voir SATAN.

\DicoEntry{DIACRE}\textit{, du grec «~diakonos~»~: «~domestique, subordonné, messager~»}\newline
Les premiers diacres étaient des hommes remplis de l'Esprit Saint et de sagesse~; ils furent nommés pour faire un travail complémentaire aux ministres de la Parole au sein de l'église de Jérusalem. Etienne* était l'un d'eux. Il existait aussi des femmes diaconesses comme Phœbe, de l'église de Cenchrées. Voir \vref{Ac. 6:1-8}~; \vref{Ro. 16:1-2} et \vref{1 Ti. 3:8-13}.

\DicoEntry{DIANE}\textit{, du grec «~Artemis~»~: «~de la lumière~»}\newline
Aussi appelée «~Artemis d'Ephèse~», divinité révérée dans toute l'Asie. Il existait un temple en son honneur à Ephèse. Voir \vref{Ac. 19:24-37}.

\DicoEntry{DIEU}\textit{}\newline
Dieu des dieux et Seigneur des seigneurs, il est le Créateur de l'univers, du ciel, de la terre et de tout ce qui s'y trouve. Architecte d'excellence, il forma l'homme à son image et lui manifesta un amour inconditionnel par son incarnation en Jésus-Christ*. Dieu se présenta à Moïse sous le nom YHWH* (=Je suis celui qui suis) montrant son caractère éternel. Il s'est révélé à différentes personnes sous divers noms et aspects, en fonction des situations traversées montrant qu'il est celui qui remplit tout en tous et qu'il est et a tout ce dont l'homme a besoin. Ainsi, on le découvre dans les Ecritures comme étant grand, unique et indivisible, omniprésent, omniscient, souverain, incorruptible, sage, patient, saint, parfait, merveilleux, tout-puissant, fidèle, juste et bon. Bien évidemment, Dieu ne peut en aucun cas être défini dans tout ce qu'il est, dans la mesure où sa nature même échappe à toute possibilité de frontière ou de limite. Toutefois, les saints auront l'éternité pour découvrir ce Père incomparable. Voir \vref{Ge. 1,2}~; \vref{Ge. 17:1}~; \vref{Ex. 3:14}~; \vref{De. 6:14}~; \vref{De. 10:17}~; \vref{Es. 6:3}~; \vref{Mal. 3:6}~; \vref{Ps. 11:7}~; \vref{Ps. 139:7-10}~; \vref{La. 3:22-23}~; \vref{Lu. 1:49}~; \vref{Ja. 1:17}~; \vref{1 Th. 4:17}~; \vref{1 Co. 1:9}~; \vref{Ro. 1:23}~; \vref{Ro. 2:4}~; \vref{Ro. 11:33-36}~; \vref{2 Ti. 4:8}~; \vref{Hé. 4:13} et \vref{1 Jn. 4:8}.

\DicoEntry{DÎME}\textit{, de l'hébreu «~ma'aser~»~: «~dîme, dixième partie~»}\newline
Abraham donna à Melchisédek la dîme du butin d'une bataille remportée (\vref{Ge. 14:17-20} et \vref{Hé. 7:1-2}). Yahweh instaura, au travers de Moïse, la dîme comme une loi à respecter par les enfants d'Israël. Il en existait quatre sortes~:
\\1. la dîme que les Lévites prélevaient sur le peuple (\vref{No. 18:21-24})
\\2. la dîme de la dîme, que les prêtres prélevaient sur les Lévites (\vref{No. 18:25-31}~; \vref{Né. 10:38})
\\3. la dîme consommée par les Juifs eux-mêmes lors des fêtes de Yahweh (\vref{De. 14:22-26})
\\4. la dîme pour l'étranger, la veuve, l'orphelin et le Lévite, donnée tous les trois ans (\vref{De. 14:28-29}).
\\Cette loi concernait exclusivement Israël et non l'Eglise – Jésus-Christ ayant accompli la loi (\vref{Mt. 5:17}). Sous la grâce, les chrétiens sont invités à faire des offrandes * librement et sans contrainte.

\DicoEntry{DINA}\textit{, de l'hébreu «~Diynah~»~: «~jugement, justice~»}\newline
Fille de Jacob et Léa. Elle fut enlevée et déshonorée par Sichem, fils de Hamor, prince du pays de Canaan. Sichem et tous les hommes de la ville furent ensuite tués par les frères de la jeune fille, Siméon et Lévi. Voir \vref{Ge. 34}.

\DicoEntry{DIOTRÈPHE}\textit{, du grec «~Diotrephes~»~: «~nourri par Zeus~»}\newline
Chrétien dont Jean dénonça l'arrogance et les mauvais agissements. Voir \vref{3 Jn. 1:9-11}.

\DicoEntry{DISCIPLE}\textit{, du grec «~mathetes~»~: «~un étudiant, un élève, un disciple~»}\newline
Personne qui écoute les enseignements de son maître et les met en pratique en vue de devenir comme lui. Jésus en choisit douze qu'il forma pendant son service. Le disciple de Christ doit manifester le caractère de son maître, lui être pleinement consacré et être prêt à souffrir en son nom. Voir \vref{Mt. 10}~; \vref{Lu. 6:12-16}~; \vref{Lu. 14:26-33}.

\DicoEntry{DIVORCE}\textit{, du grec «~apostasion~»~: «~divorce, répudiation, lettre de divorce~»}\newline
Brisement des liens du mariage*. Il fut autorisé sous la loi de Moïse à cause de la dureté des cœurs, mais Christ rappela l'indissolubilité du mariage au commencement. Voir \vref{De. 24:1-3} et \vref{Mt. 19:3-8}.

\DicoEntry{DOCTEUR}\textit{, du grec «~didaskalos~»~: «~professeur~», «~maître~».}\newline
Sous la loi de Moïse, les docteurs de la loi étaient chargés d'expliquer la Torah. Certains d'entre eux s'opposèrent à Jésus. Sous la Nouvelle Alliance, le docteur est un des cinq services liés à la Parole évoqués en \vref{Ep. 4:11}. Il enseigne la Parole de Dieu qui guérit les blessures de l'âme. Selon \vref{Ja. 3:1}, nous ne sommes pas tous appelés à être des docteurs. Voir \vref{Lu. 2:46}~; \vref{Lu. 5:17}~; \vref{1 Co. 12:28} et \vref{Ep. 4:11}.

\DicoEntry{DONS SPIRITUELS}\textit{, du grec «~charisma~»~: «~faveur que reçoit quelqu'un sans aucun mérite de sa part~», «~dons provenant du pouvoir de la grâce divine~»}\newline
Capacités distribuées par le Saint-Esprit aux chrétiens en vue de la formation et de l'édification des saints. Voir \vref{1 Co. 12:1-11}~; \vref{1 Co. 14:12}~; \vref{Ro. 12:6} et \vref{1 Pi. 4:10}.

\DicoEntry{EDEN}\textit{, de l'hébreu «~'Eden~»~: «~plaisir, délices~»}\newline
Appelé aussi jardin de Dieu, premier lieu de résidence d'Adam et Eve. Yahweh y avait fait pousser des arbres de toutes espèces~; il y avait également placé au milieu l'arbre de vie* ainsi que l'arbre de la connaissance du bien et du mal* dont la consommation des fruits conduirait à la mort. L'homme fut établi en tant que gardien et cultivateur de ce jardin. Cependant, il pêcha avec la femme et ils furent chassés de ce lieu des délices. Voir \vref{Ge. 2}~; \vref{Ge. 3:23-24} et \vref{Ez. 28:13}.

\DicoEntry{ÉGLISE}\textit{, du grec «~ekklesia~»~: «~appel hors de~»}\newline
Peuple mis à part dont Christ est le chef. L'Eglise est la sainte habitation de Dieu en esprit, le corps de Christ, l'épouse de l'Agneau. On distingue l'Eglise universelle - qui regroupe tous les saints du monde entier - de l'église locale - qui est composée de tous les chrétiens d'une ville. Voir \vref{Ac. 2:47}~; \vref{1 Th. 1:1}~; \vref{1 Co. 1:2}~; \vref{1 Co. 3:16}~; \vref{1 Co. 12:27}~; \vref{Ep. 2:20-22}~; \vref{Ep. 5:22-32}~; \vref{Ph. 1:1} et \vref{1 Ti. 2:4}.

\DicoEntry{ÉLÉAZAR}\textit{, de l'hébreu «~'El'azar~»~: «~Dieu a secouru~»}\newline
Fils d'Aaron, il était chef des chefs des Lévites avant de devenir le second grand prêtre d'Israël. Voir \vref{No. 3:32} et \vref{No. 20:25-28}.

\DicoEntry{ÉLECTION, ELU}\textit{, de l'hébreu «~bachiyr~» et du grec «~eklektos~»~: «~choisi, élu de Dieu~»}\newline
Dans le Tanakh, Israël fut présenté comme le peuple élu de Yahweh, appelé à être un exemple pour toutes les nations de la terre. Cette élection n'est pas synonyme de préférence, car la volonté de Dieu est de sauver tous les hommes. Au travers de l'œuvre de la croix, Dieu a effectivement montré que son choix se porte vers l'humanité tout entière en payant le prix des péchés de tous. Dans son omniscience, il sait toutefois d'avance qui croira en lui ou pas. Selon la parole, même après la conversion, le chrétien doit travailler son élection, c'est-à-dire se sanctifier et obéir aux commandements de Yahweh pour entrer dans son royaume. Voir \vref{Es. 45:4}~; \vref{Es. 49:6}~; \vref{Mt. 22:14}~; \vref{Ep. 1:4-6} et \vref{2 Pi. 1:10-11}.

\DicoEntry{ÉLIE}\textit{, de l'hébreu «~Eliyah~»~: «~Yahweh est mon Dieu~»}\newline
Prophète d'origine tschibite que Dieu suscita en Israël au temps du roi Achab*. Il ne connut point la mort, mais fut enlevé par le Seigneur. Son histoire, ses combats et ses exploits sont racontés dans les livres des Rois.

\DicoEntry{ÉLISÉE}\textit{, de l'hébreu «~'Eliysha'~»~: «~Dieu est sauveur~»}\newline
Prophète du royaume d'Israël, il succéda à Elie après avoir reçu la double portion de l'esprit qui était sur ce dernier. Après sa mort, ses os rendirent la vie à un défunt. Voir \vref{1 R. 19:16-21}~; \vref{2 R. 2:9-11} et \vref{2 R. 13:20-21}.

\DicoEntry{ENFER}\textit{}\newline
Voir SEJOUR DES MORTS.

\DicoEntry{ENLÈVEMENT}\textit{, du grec «~metathesis~»~: «~transfert d'un lieu à un autre, changement~»}\newline
Ravissement d'hommes au ciel sans que ces derniers ne connaissent la mort*. Dans le Tanakh, se trouvent deux cas d'enlèvement~: Hénoc (\vref{Ge. 5:24}~; \vref{Hé. 11:5}) et Elie (\vref{2 R. 2:11}). L'Eglise sera de même enlevée par le Seigneur au son de la dernière trompette. Voir \vref{1 Th. 4:17} et \vref{1 Co. 15:51-57}.

\DicoEntry{ÉPHÈSE}\textit{, du grec «~Ephesos~»~: «~permis~»}\newline
Une des principales villes de l'empire romain sous le règne de l'empereur Claude 1er (10 av. J.-C. – 54 ap. J.-C.) Ephèse possédait le plus grand port de l'Asie Mineure, ce qui lui attribuait le contrôle du trafic commercial. Richissime et prospère, elle était renommée pour son faste, sa liberté de parole et constituait donc un endroit privilégié pour les philosophes. L'église d'Ephèse naquit du ministère de Paul, qui y enseigna pendant au moins deux ans lors de son troisième voyage missionnaire. Cette église - figurant parmi les sept du livre d'Apocalypse - fit preuve de discernement et pratiquait de bonnes œuvres, mais le Seigneur avait néanmoins un reproche à lui adresser. Elle représente l'église apostate. Voir Epître aux Ephésiens et \vref{Ap. 2:1-7}.

\DicoEntry{ÉPHOD}\textit{, de l'hébreu «~ephowd~»~: «~couverture~»}\newline
Vêtement que les prêtres portaient par-dessus leur tunique lorsqu'ils étaient en service. L'éphod du grand prêtre était de broderie~; le pectoral était posé sur son devant. Voir \vref{Ex. 28} et \vref{Lé. 8:7}.

\DicoEntry{ÉPHRAIM}\textit{, de l'hébreu «~'Ephrayim~»: «~double fertilité~»}\newline
Second fils de Joseph né en Egypte, il fut adopté par Jacob avant sa mort et devint ainsi l'ancêtre d'une des douze tribus d'Israël. Voir \vref{Ge. 41:52}~; \vref{Ge. 48:5} et \vref{Jos. 14:4}.

\DicoEntry{ÉPICURIENS}\textit{, du grec «~epikoureios~»~: «~celui qui aide, le défenseur~»}\newline
Fondé à Athènes en 306 av. J.-C., groupe de philosophes se réclamant de la doctrine d'Epicure (341 av. J.-C. à 270 av. J.-C.). Ce dernier fonda une des plus importantes écoles philosophiques de l'Antiquité. Il développa une théorie athée selon laquelle l'homme est encouragé à rechercher les plaisirs matériels et sensuels. Rejetant la pensée d'une vie après la mort, les épicuriens renient l'existence d'un créateur qui se préoccuperait des hommes. Des adeptes de cette philosophie se confrontèrent à la doctrine de Christ annoncée par Paul et cherchèrent à l'entendre. Voir \vref{Ac. 17:18-20}.

\DicoEntry{ÉSAÏE}\textit{, de l'hébreu «~Yesha'yah~»~: «~Yahweh a sauvé~»}\newline
Fils d'Amotz, un prophète de Yahweh contemporain des rois Ozias, Jotham, Achaz et Ezéchias, il annonça la venue du Messie. L'ensemble de ses prophéties est contenu dans le livre portant son nom.

\DicoEntry{ÉSAÜ}\textit{, de l'hébreu «~'Esav~»~: «~velu, poilu, chevelu~»}\newline
Fils d'Isaac et Rebecca et frère jumeau de Jacob, qui lui soutira son droit d'aînesse et sa bénédiction. Il prit pour femmes Judith et Basmath, toutes deux originaires de Canaan. Egalement connu sous le nom d'Edom, il devint l'ancêtre des Edomites. Voir \vref{Ge. 25:25-34}~; \vref{Ge. 27} et \vref{Ge. 36}.

\DicoEntry{ESDRAS}\textit{, de l'hébreu «~`Ezra'~»~: «~secours~»}\newline
Fils de Sereja et descendant du grand prêtre Aaron, Esdras était scribe et prêtre. Il enseigna le peuple de Dieu dans la loi et mit en place des réformes après la reconstruction du temple. Son histoire se trouve dans le livre éponyme.

\DicoEntry{ESPRIT}\textit{, de l'hébreu «~ruwach~»~: «~vent, souffle, esprit~» et du grec «~pneuma~»~: «~vérité, inspiration, souffle, vent~»}\newline
L'esprit humain est aussi appelé homme intérieur, il constitue la partie spirituelle de l'homme lui permettant d'agir, de prendre des décisions et d'être en contact avec Dieu ou tout autre esprit. Principe vital, il amène l'âme à la vie. Avec l'âme* et le corps, l'esprit constitue l'être humain. Voir \vref{Ge. 6:3}~; \vref{Ex. 31:3}~; \vref{Job 27:3}~; \vref{Job 32:8}~; \vref{Mt. 12:28} et \vref{1 Th. 5:23}.

\DicoEntry{ESPRIT IMPUR}\textit{}\newline
Voir DEMONS.

\DicoEntry{ESTHER}\textit{, dérivation du perse «~'Ecter~»~: «~étoile~»}\newline
Cousine de Mardochée, Juif d'origine benjaminite, reine de Perse, épouse du roi Assuérus. Son nom juif était Hadassa~: «~myrte~». Son histoire, qui se déroula à Suse, est racontée dans le livre portant son nom.

\DicoEntry{ÉTANG DE FEU}\textit{}\newline
Lieu de douleur et de damnation éternelle créé initialement pour le diable et ses anges. Y seront jetés la bête et le faux prophète, le diable, la mort et le séjour des morts* ainsi que tous ceux dont le nom ne sera pas trouvé dans le livre de vie. Voir \vref{Mt. 25:41}~; \vref{Ap. 19:20} et \vref{Ap. 20:7-15}.

\DicoEntry{ÉTIENNE}\textit{, du grec «~stephanos~»~: «~couronne~»}\newline
Diacre* de l'église de Jérusalem rempli de sagesse et d'Esprit Saint. Premier martyr chrétien, sa mort marqua le début d'une grande persécution contre l'Eglise. Voir \vref{Ac. 6:1-6}~; \vref{Ac. 7} et \vref{Ac. 8:1-3}.

\DicoEntry{EUNUQUE}\textit{, du grec «~cariyc~»~: «~eunuque, chambellan castré~»}\newline
Homme dans l'incapacité de procréer ou émasculé. Dans l'Antiquité, les rois se choisissaient des eunuques pour les servir. En les castrant, ils s'assuraient de la fidélité et l'intégrité de ces derniers. En outre, Jésus distingua trois types d'eunuques. Voir \vref{2 R. 20:18}~; \vref{Da. 1:7}~; \vref{1 Ch. 28:1} et \vref{Mt. 19:12}).

\DicoEntry{ÉVANGELISTE}\textit{}\newline
Un des cinq ministères d'\vref{Ep. 4:11} dont la mission est de prêcher la repentance et la conversion à Jésus-Christ. Comme les ministères évoqués en \vref{Ep. 4:11}, il travaille également à la perfection des saints. Philippe exerça ce ministère, Timothée fut de même encouragé à faire l'œuvre d'un évangéliste. Tous les chrétiens doivent également évangéliser. Voir \vref{Ac. 21:8}~; \vref{Ep. 4:11} et \vref{2 Ti. 4:5}.

\DicoEntry{ÉVANGILE}\textit{}\newline
Enseignement donné par Jésus-Christ, la prédication de la croix (la mort et la résurrection de Jésus-Christ) et du Royaume de Dieu qui s'est approché des hommes (voir ROYAUME DE DIEU). Ce message annonce le salut, la guérison du cœur, la joie en Jésus-Christ, la justice, la paix, la grâce et la vie éternelle accordée à l'homme repentant, mais aussi le jugement à venir. Les apôtres propagèrent l'Evangile~; de même, tous les chrétiens sont appelés à le faire. Voir \vref{Es. 61}~; \vref{Mt. 10:7}~; \vref{Mt. 28:19-20}~; \vref{1 Co. 15:1-4}~; \vref{Ro. 1:16} et \vref{2 Ti. 4:1}.

\DicoEntry{EVE}\textit{, de l'hébreu «~Chavvah~»~: «~vie~»}\newline
Première femme et épouse d'Adam, elle fut formée à partir de la côte de son mari dans le but d'être l'aide de ce dernier. Séduite par Satan déguisé en serpent, elle mangea le fruit de la connaissance du bien et du mal et fut avec Adam chassée du jardin. Elle donna naissance à Caïn, Abel et Seth. Voir \vref{Ge. 2:18-24}~; \vref{Ge. 3:1-13} et \vref{Ge. 4:1-2,25}.

\DicoEntry{ÉVÊQUE}\textit{, du grec «~episcopos~»~: «~investigation, inspection, visite d'inspection~», «~acte par lequel Dieu visite les hommes, observe leurs voies, leurs caractères, pour leur accorder en partage joie ou tristesse~», «~surveillance, contrôle, fonction d'un ancien~», «~la charge d'une église chrétienne~».}\newline
Il est question ici d'une fonction consistant à visiter les assemblées, les inspecter afin de s'assurer du bon ordre. Voir \vref{Lu. 19:44}~; \vref{Ac. 1:20}~; \vref{1 Ti. 3:1}~; \vref{1 Pi. 2:12}.

\DicoEntry{EXPIATION}\textit{, de l'hébreu «~kaphar~»~: «~couvrir, purger, faire une expiation~»}\newline
Action de couvrir les fautes et les souillures de l'homme afin qu'il soit réconcilié avec Dieu. Sous l'Ancienne Alliance, le grand prêtre faisait tous les ans un sacrifice d'expiation en entrant dans le Saint des saints pour ses péchés et les péchés du peuple. Par son sacrifice, Christ est devenu la victime expiatoire pour les péchés de tous les hommes en les prenant sur lui à la croix~; il est l'Agneau de Dieu qui ôte les péchés du monde. Voir \vref{Lé. 16}~; \vref{Jn. 1:29}~; \vref{1 Jn. 2:2} et \vref{1 Jn. 4:10}.

\DicoEntry{ÉZÉCHIAS}\textit{, de l'hébreu «~Yechizqiyah~»~: «~Yahweh est ma force~»}\newline
Fils d'Osée, roi de Juda sur qui il régna vingt-neuf ans. Figurant parmi les rois les plus intègres, son règne fut caractérisé par la droiture et la fidélité à Yahweh. Voir \vref{2 R. 18-19}.

\DicoEntry{ÉZÉCHIEL}\textit{, de l'hébreu «~Yechezqe'l~»~: «~Dieu fortifie~»}\newline
Fils de Buri, prêtre et prophète de Yahweh ayant été déporté à Babylone. Il reçut de nombreuses visions - sur son temps et les temps de la fin - racontées dans le livre qui porte son nom.

\DicoEntry{FÉLIX}\textit{, du grec «~Phestos~»~: «~joyeux, en fête~»}\newline
Gouverneur de Judée de 52 à 60 ap. J.-C., il emprisonna Paul à la suite des plaintes des Juifs. S'entretenant avec lui de temps en temps et lui octroyant certaines libertés, Felix garda Paul en prison deux ans pour plaire aux Juifs. Voir \vref{Ac. 24}.

\DicoEntry{FESTUS}\textit{, du grec «~Phestos~»~: «~en fête, joyeux~»}\newline
Gouverneur de Judée qui succéda à Félix* de 60 à 62 ap. J.-C. Il poursuivit l'instruction du procès de Paul que les Juifs accusaient. Il permit à Paul de s'exprimer devant le roi Agrippa* et l'envoya à Rome afin qu'il comparaisse devant César*. Voir \vref{Ac. 24:27} et \vref{Ac. 25,26}.

\DicoEntry{FÊTES DE YAHWEH}\textit{}\newline
Selon la loi juive, sept fêtes étaient célébrées en l'honneur de Yahweh~: la Pâque de Yahweh, la fête des pains sans levain~; la fête des prémices~; la Pentecôte~; la fête des trompettes~; le jour des expiations et la fête des tabernacles. Voir \vref{Lé. 23:6-43}.

\DicoEntry{FIGUIER}\textit{}\newline
Arbre fruitier sous lequel il était coutume d'étudier la Torah en Israël. Ses fruits excellents et doux servaient en médecine. Le figuier est retrouvé dans de nombreuses histoires et paraboles des Ecritures. Il symbolise la douceur et l'humilité. Voir \vref{Jg. 9:11}~; \vref{2 R. 20:1-7}~; \vref{Lu. 13:6-9} et \vref{Jn. 1:43-51}.

\DicoEntry{FILS DE DIEU}\textit{}\newline
Expression désignant selon le contexte~:
\\1. les anges. Voir \vref{Ge. 6:2-4}~; \vref{Job 38:7} et \vref{Da. 3:25}.
\\2. Adam. Voir \vref{Lu. 3:38}.
\\3. les chrétiens. Voir \vref{Ga. 3:26} et \vref{Ro. 8:14}.
\\4. Jésus-Christ, le Fils unique de Dieu, en qui habite la plénitude de la divinité. Voir \vref{Mc. 15:39}~; \vref{Lu. 22:70}~; \vref{Jn. 1:14,34,49}~; \vref{Ro. 1:4}~; \vref{Col. 2:9} et \vref{1 Jn. 4:9,15}.

\DicoEntry{FILS DE L'HOMME}\textit{}\newline
Expression désignant un être humain, elle fut attribuée au prophète Ezéchiel près de cent fois. A de nombreuses reprises, Jésus-Christ se nomma lui-même «~Fils de l'homme~» afin de souligner sa nature humaine. Voir \vref{Ez. 2:1}~; \vref{Ez. 3:10}~; \vref{Ez. 4:1}~; \vref{Mc. 14:62}~; \vref{Jn. 5:27}~; \vref{Ro. 8:3} et \vref{Ph. 2:5-7}.

\DicoEntry{FIN DES TEMPS}\textit{, du grec «~eschatos~»~: «~extrême, dernier, fin~» et «~chronos~»~: «~temps, date, siècles~»}\newline
Appelée aussi derniers jours, période précédant la fin du monde*. Elle a commencé à l'effusion du Saint-Esprit selon la prophétie de Joël. La fin des temps est caractérisée d'un côté par des manifestations extraordinaires de l'Esprit de Dieu et l'annonce de l'Evangile à tous les peuples~; de l'autre par la séduction, l'apostasie* et le péché dans des dimensions jamais atteintes auparavant. Voir \vref{Joë. 2:28-29}~; \vref{Mt. 24:3-14}~; \vref{Ac. 2:16-18}~; \vref{1 Ti. 4:1} et \vref{2 Ti. 3:1-5}.

\DicoEntry{FIN DU MONDE}\textit{, du grec «~eschatos~»~: «~extrême, dernier, fin~» et «~aion~»~: «~monde, univers, période de temps~»}\newline
Cet événement correspond à la fin de notre ère. Après le jugement dernier, les impies iront dans l'étang de feu*, tandis que la Nouvelle Jérusalem accueillera les saints~; la terre sera détruite. Voir \vref{Mt. 13:36-43}~; \vref{2 Pi. 3:10-13}~; \vref{Ap. 20:11-15} et \vref{Ap. 21}.

\DicoEntry{FOI}\textit{, du grec «~pistis~»~: «~conviction de la vérité~»}\newline
Confiance en la véracité de Dieu, ses paroles et l'accomplissement de ses promesses. Bien qu'il n'existe qu'une seule foi, elle est présentée sous trois formes principales sous la Nouvelle Alliance~:
\\1. en tant que fruit de l'esprit*, c'est la foi qui sauve (\vref{Ga. 5:22} et \vref{Ro. 10:9})
\\2. en tant que don de l'Esprit,* c'est la foi accordée pour accomplir une tâche particulière (\vref{1 Co. 12:9})
\\3. en tant que Parole, c'est la foi liée à la saine doctrine, la vérité (\vref{Ro. 10:17} et \vref{2 Ti. 4:7})
\\Condition essentielle pour être agréable à Dieu~; la foi est éprouvée tout au long de la vie du croyant. Voir \vref{Lu. 7:50}~; \vref{Hé. 11} et \vref{1 Pi. 1:7}.

\DicoEntry{FORNICATION ou IMPUDICITÉ}\textit{, du grec «~pœrneia~»~: «~relation sexuelle illicite~»}\newline
Tous les rapports sexuels condamnés par la Parole, voir \vref{Lé. 18}~; \vref{1 Co. 6:13,16-18} et \vref{1 Co. 7:2}.

\DicoEntry{FRUIT DE L'ESPRIT}\textit{}\newline
Résultat de l'action de l'Esprit Saint dans l'homme intérieur dans le but de communiquer le caractère de Yahweh au chrétien né d'en haut. Voir \vref{Ga. 5:22}.

\DicoEntry{GABRIEL}\textit{, de l'hébreu «~Gabriy'el~»~: «~héros de Dieu~» ou «~homme de Dieu~»}\newline
Archange* que Dieu envoya pour délivrer des messages, notamment à Daniel, Zacharie et Marie. Voir \vref{Da. 9:21-27}~; \vref{Lu. 1:11-20} et \vref{Lu. 1:26-38}.

\DicoEntry{GAD}\textit{, de l'hébreu «~Gad~»~: «~bonheur~», «~heureux~», «~troupe~»}\newline
Fils de Jacob et Zilpa, servante de Léa, il devint l'ancêtre de la tribu de Gad. Voir \vref{Ge. 30:11} et \vref{Ge. 49:16}.

\DicoEntry{GALATIE}\textit{, du grec «~Galatia~»~: «~territoire des Gaulois, Gaule~»}\newline
Province antique de l'Asie Mineure, la Galatie se situait en Anatolie, dans l'actuelle Turquie autour d'Ankara. Elle devait son nom aux Galates, Celtes provenant des Balkans. Lors de son premier voyage missionnaire, Paul avait traversé cette région où plusieurs assemblées émergèrent. Il y revint plus tard pour fortifier les disciples et leur écrivit une lettre suite au trouble apporté par les judaïsants. Voir \vref{Ac. 16:6}~; \vref{Ac. 18:23} et \vref{Ga. 1-5}.

\DicoEntry{GALILÉE}\textit{, de l'hébreu «~Galiyl~»~: «~cercle, région, district~»}\newline
Région située au nord de la Palestine dans laquelle se trouve la localité de Nazareth où Jésus grandit. Il y commença son ministère, c'est aussi là qu'il se montra vivant à ses disciples après sa résurrection. Les disciples de Jésus étaient originaires de Galilée. Voir \vref{Mt. 2:19-23}~; \vref{Mc. 16:7}~; \vref{Jn. 2}~; \vref{Ac. 1:11} et \vref{Ac. 2:7}.

\DicoEntry{GARIZIM}\textit{, de l'hébreu «~Geriziym~»~: «~lieux arides~»}\newline
Montagne située au sud de Sichem, en face du mont Ebal, de laquelle les enfants d'Israël devaient prononcer la bénédiction* une fois entrés en Canaan. Voir \vref{De. 11:29}~; \vref{Jg. 9:7} et \vref{Jos. 8:33}.

\DicoEntry{GÉDÉON}\textit{, de l'hébreu «~Gid'own~»~: «~coupant, abattant~»}\newline
Issu de la tribu de Manassé et fils de Joas. Il fut mandaté pour délivrer Israël de la main des Madianites et fut juge en Israël pendant quarante ans. Voir \vref{Jg. 6-8}.

\DicoEntry{GÉHENNE}\textit{, du grec «~geena~»~: «~vallée de Hinnom~»}\newline
Initialement, vallée située au sud de Jérusalem où des enfants étaient jetés dans le feu en sacrifice à Moloc. Le terme «~géhenne~» représente la destruction future des méchants et se rapporte à l'étang de feu*. Voir \vref{2 R. 23:10} et \vref{Mt. 10:28}.

\DicoEntry{GENTILS}\textit{, du grec «~ethnos~»~: «~nations~», «~peuples~»}\newline
Dans les Ecritures, ce terme se rapportait initialement à tous ceux n'appartenant pas au peuple juif. Paul fut mandaté pour évangéliser les Gentils. A partir du IIIème siècle, le terme «~païen~» fut introduit dans le jargon chrétien pour désigner le «~non-chrétien~». Voir \vref{Mt. 18:17} et \vref{Ac. 26:17}.

\DicoEntry{GERME}\textit{, de l'hébreu «~tsemach~»~: «~pousse, croissance, branche~»}\newline
Terme désignant le Messie dans certains écrits prophétiques. Voir \vref{Es. 4:2}~; \vref{Jé. 23:5} et \vref{Za. 3:8}.

\DicoEntry{GLOIRE}\textit{, de l'hébreu «~kabhod~»~: «~poids~» ou «~kabowd~»~: «~gloire, honneur, richesse~»}\newline
La gloire se rapporte à ce qui a du poids, ce qui est lourd et écrasant – il est en effet difficile pour l'homme de supporter la splendeur et la magnificence de Yahweh. Image de sa sainteté, elle s'est manifestée dans un feu dévorant sur le mont Sinaï et fut révélée à Moïse au travers de la bonté et du nom de Dieu. Cette gloire sanctifie et génère de grands miracles~; elle est racontée par les cieux et toute la création. La gloire de Yahweh sera le luminaire de la Nouvelle Jérusalem. Elle invite à la crainte, la révérence, l'humilité, la louange~; lui seul mérite la gloire. Voir \vref{Ex. 16:10}~; \vref{Ex. 24:17}~; \vref{Ex. 29:43}~; \vref{Ex. 33:18-23}~; \vref{Es. 42:8}~; \vref{Es. 48:11}~; \vref{Ez. 44:4}~; \vref{Ps. 19:1}~; \vref{Pr. 15:33}~; \vref{2 Ch. 5:14}~; \vref{1 Th. 2:12} et \vref{Ap. 21:23}.

\DicoEntry{GOG}\textit{, de l'hébreu «~Gowg~»~: «~montagne~»}\newline
Très certainement le chef du pays de Magog. Voir \vref{Ez. 38} et \vref{Ap. 20:8}.

\DicoEntry{GOLGOTHA}\textit{, de l'araméen «~gulgoleth~»~: «~tête, crâne~»}\newline
Lieu de la crucifixion de Jésus-Christ, situé non loin de Jérusalem. Voir \vref{Jn. 19:17-20}.

\DicoEntry{GOMORRHE}\textit{, de l'hébreu «~Amorah~»~: «~submersion~»}\newline
Ville située dans la plaine du Jourdain. Après avoir atteint un haut degré de perversion et de débauche, elle fut détruite par Yahweh avec sa ville voisine, Sodome. Voir \vref{Ge. 13:10}~; \vref{Ge. 18:20-21} et \vref{Ge. 19:24}.

\DicoEntry{GRÂCE}\textit{, du grec «~charis~»~: «~bonne volonté~», «~bonté~», «~faveur~»}\newline
Don immérité de Dieu, elle est la source du salut* de tous les hommes et invite à la crainte de Dieu. La grâce est venue par Jésus-Christ et fut révélée au travers de l'œuvre parfaite de la croix*. Voir \vref{Jn. 1:17}~; \vref{Ro. 3:23-24}~; \vref{Tit. 2:11-12}.

\DicoEntry{GRAND PRÊTRE}\textit{}\newline
Voir PREMIER PRÊTRE.

\DicoEntry{GRANDE TRIBULATION}\textit{}\newline
Voir commentaire \vref{Ap. 7:14}.

\DicoEntry{GUILGAL}\textit{, de l'hébreu «~Gilgal~»~: «~action de rouler~»}\newline
Territoire situé à l'ouest du Jourdain et à l'est de Jéricho~; il fut le lieu de campement des Israélites après avoir passé le Jourdain à sec. Voir \vref{Jos. 4-5}.

\DicoEntry{HABAKUK}\textit{, de l'hébreu «~Chabaqquwq~»~: «~embrasser~», «~amour~»}\newline
Prophète de Yahweh qui exerça son ministère dans le royaume de Juda. L'ensemble de ses prophéties se trouve dans le livre éponyme.

\DicoEntry{HARMAGUEDON}\textit{, de l'hébreu «~Armageddon~»~: «~montagne de Méguiddo~»}\newline
Lieu situé au nord d'Israël dans la tribu de Zabulon. A la fin des temps, les rois et puissants de la terre s'y rassembleront pour combattre Yahweh et son armée. Voir \vref{2 R. 23:29} et \vref{Ap. 16:13-16}.

\DicoEntry{HÉBREU}\textit{, de l'hébreu «~`Ibriy~»~: «~qui provient de l'autre côté, qui traverse~»}\newline
Terme désignant les descendants d'Héber, fils de Schélach, de la postérité de Sem, dont est issu Abraham. Voir \vref{Ge. 11:10-32}~; \vref{Ge. 14:13-14} et \vref{Ex. 1:15-22}.

\DicoEntry{HELLÉNISTE}\textit{, du grec «~hellenistes~»~: «~celui qui adopte les manières et coutumes des Grecs~»}\newline
Israélites nés hors de la terre promise ayant adopté le mode de vie grec et parlant la langue grecque. Voir \vref{Ac. 6:1}.

\DicoEntry{HÉNOC}\textit{, de l'hébreu «~Chanowk~»~: «~consacré, dédié~»}\newline
Fils de Jéred et père de Metuschéla. Homme pieux ayant vécu trois cent soixante-cinq ans avant d'être enlevé au ciel sans connaître la mort. Voir \vref{Ge. 5:21-24} et \vref{Hé. 11:5}.

\DicoEntry{HÉRODE LE GRAND}\textit{, (73 av. J.-C.à 4 av. J.-C), du grec «~Herodes~»: «~héroïque~»}\newline
Roi de Judée, il fut l'instigateur du massacre des enfants de la région de Bethléhem au moment de la naissance de Jésus. Il mourut quand Jésus était encore enfant. Voir \vref{Mt. 2}.

\DicoEntry{HÉRODE ANTIPAS}\textit{, (ou le Tétrarque) (\ 21 av. J.-C. à 39 ap. J.-C.)}\newline
Fils d'Hérode le Grand*, il exerça la fonction de tétrarque* de Galilée et fut contemporain à Jésus-Christ pendant presque toute la vie de ce dernier. Hérode épousa sa belle-sœur Hérodias* et fit décapiter Jean-Baptiste. Il fut qualifié de «~renard~» par Jésus et s'accorda avec son ennemi Pilate lors de la crucifixion du Seigneur. Voir \vref{Mc. 6:14-28}~; \vref{Lu. 3:1}~; \vref{Lu. 13:31-32} et \vref{Lu. 23:8-12}.

\DicoEntry{HÉRODE AGRIPPA Ier}\textit{, (\ 10 av. J.-C. à 44 ap. J.-C)}\newline
Roi et tétrarque de Judée et petit fils du roi Hérode le Grand, il accéda au pouvoir à la genèse de l'Eglise primitive. Pour plaire aux Juifs, il fit mourir Jacques, fils de Zébédée, et emprisonna Pierre. Il mourut brusquement après avoir reçu du peuple la gloire qui devait revenir à Dieu. Voir \vref{Ac. 12}.

\DicoEntry{HÉRODE AGRIPPA II}\textit{, (\ 27 ap. J.-C. à 93 ap. J.-C.)}\newline
Fils d'Agrippa Ier, il est appelé «~roi Agrippa~» dans les Ecritures. Il fut inspecteur du temple de Jérusalem et avait le pouvoir de choisir les grands prêtres. Il rencontra Paul à Césarée lors d'une visite au gouverneur Festus*. Voir \vref{Ac. 25-26}.

\DicoEntry{HÉRODIAS}\textit{, du grec «~Herodias~»~: «~héroïque~»}\newline
Femme de Philippe I puis de son frère, Hérode le tétrarque. Elle commanda la décapitation de Jean-Baptiste. Voir \vref{Mc. 6:17-28}.

\DicoEntry{HOLOCAUSTE}\textit{, de l'hébreu «~'olah~»~: «~offrande entièrement consumée~»}\newline
Prescrit par la loi de Moïse, sacrifice consumé par le feu d'une agréable odeur à Yahweh. Il préfigurait le sacrifice à la croix de Jésus-Christ, l'Agneau de Dieu. Voir \vref{Lé. 1:1-17}~; \vref{Hé. 9:11-22} et \vref{Hé. 10:1-19}.

\DicoEntry{HOMOSEXUALITÉ}\textit{}\newline
Pratique abominable et fermement réprouvée par Dieu consistant en l'union de deux personnes du même sexe. Voir \vref{Lé. 18}~; \vref{1 Co. 6:9-10} et \vref{Ro. 1:24-32}.

\DicoEntry{HOSANNA}\textit{, de l'hébreu «~yasha'~»~: «~sauve~» et «~na´~»~: «~je te prie, maintenant~» et du grec «~hosanna~»~: «~sauve maintenant~!~»}\newline
Cri par lequel Jésus fut accueilli par la foule quand il entra à Jérusalem. Voir \vref{Mt. 21:9,15}~; \vref{Mc. 11:9-10} et \vref{Jn. 12:13}.

\DicoEntry{HULDA}\textit{, de l'hébreu «~chuldah~»~: «~belette, taupe~»}\newline
Femme de Schallum, prophétesse habitant à Jérusalem du temps de Josias, roi de Juda. Le roi chercha à consulter Yahweh au travers d'elle quand il découvrit le livre de la loi et les malheurs qui devaient suivre la désobéissance d'Israël. Voir \vref{2 R. 22:14-20} et \vref{2 Ch. 34:21-33}.

\DicoEntry{HYSOPE}\textit{, du grec «~hussopos~»~: «~hysope, branche d'hysope~»}\newline
Plante aromatique utilisée pour faire l'aspersion du sang ou d'eau sous l'Ancienne Alliance. C'est à l'aide d'une branche d'hysope qu'on présenta à Jésus une éponge trempée de vinaigre lors de sa crucifixion. Voir \vref{Ex. 12:22}~; \vref{Lé. 14:1-7}~; \vref{No. 19:18-19}~; \vref{Jn. 19:29} et \vref{Hé. 9:19}.

\DicoEntry{IDOLE, IDOLÂTRIE}\textit{, de l'hébreu «~gilluwl~»~: «~image~» et du grec «~eidolon~»~: «~image pour adorer~»}\newline
Une idole peut être l'image d'un faux dieu, l'image faussée de Yawheh ou encore une personne, un objet, une activité à qui l'on donne le rang de Dieu. L'idolâtrie - culte rendu à ces idoles – est fermement réprouvée dans la Parole. Voir \vref{Ex. 20:3-5}~; \vref{Ex. 32}~; \vref{1 R. 15:11-13}~; \vref{1 Co. 6:9}~; \vref{Ep. 5:5} et \vref{Col. 3:5}.

\DicoEntry{IMPOSITION DES MAINS}\textit{}\newline
Avant leur mort, les patriarches imposaient les mains à leurs enfants pour les bénir (\vref{Ge. 48:14}). Moïse imposa également les mains à Josué qui devait lui succéder (\vref{De. 34:9}). Sous la Nouvelle Alliance, on peut imposer les mains à quelqu'un en vue de lui transmettre la guérison divine, l'autorité liée à une fonction particulière, les dons spirituels et même le Saint-Esprit dans certains cas. Ce geste ne doit cependant pas être fait dans la précipitation. Voir \vref{Lu. 4:40}~; \vref{Mc. 16:18}~; \vref{Ac. 6:6}~; \vref{Ac. 8:17}~; \vref{1 Ti. 4:14} et \vref{1 Ti. 5:22}.

\DicoEntry{INCORRUPTIBILITÉ}\textit{, du grec «~aphtharsia~»~: «~perpétuité, pureté, sincérité~»}\newline
Terme désignant ce qui ne peut ni se corrompre, ni se flétrir, ni se détruire. A l'enlèvement de l'Eglise, les morts en Christ ressusciteront incorruptibles et les chrétiens revêtiront de même des corps incorruptibles. Voir \vref{Mt. 24:35} et \vref{1 Co. 15:40-57}.

\DicoEntry{INCREDULITÉ}\textit{, du grec «~apistia~»~: «~infidélité, sans foi, faiblesse dans la foi~»}\newline
Rejet, doute par rapport à la véracité de Dieu et de sa parole. Thomas fit preuve d'incrédulité quant à la résurrection de Christ avant de le voir vivant. Les incrédules ne peuvent pas hériter le Royaume de Dieu. Voir \vref{Jn. 1:1-14}~; \vref{Jn. 14:6}~; \vref{Jn. 20:24-29} et \vref{Ap. 21:8}.

\DicoEntry{INIQUITÉ}\textit{, du grec «~adikia~»~: «~injustice, tortuosité d'un cœur, violation volontaire de la loi~»}\newline
Tout ce qui constitue une violation de la loi* et la justice de Dieu. Voir \vref{Ro. 6:13}~; \vref{2 Pi. 2:13} et \vref{1 Jn. 5:17}.

\DicoEntry{MYSTEÈRE DE L'INIQUITÉ}\textit{}\newline
Voir commentaire \vref{2 Th. 2:7}.

\DicoEntry{INTERCESSION}\textit{, de l'hébreu «~palal~»~: «~intervenir, s'interposer, prier, agir en médiateur~»}\newline
Sous l'Ancienne Alliance, le grand prêtre avait la mission d'intercéder pour les péchés du peuple en offrant des sacrifices. A présent, Jésus-Christ le grand prêtre à perpétuité et l'avocat intercède pour ses enfants après s'être offert en sacrifice pour les péchés de l'humanité. Les hommes peuvent aussi faire des prières d'intercession comme Abraham pour Lot, Moïse pour Marie et l'Eglise pour tous les hommes. Voir \vref{Ge. 18:16-33}~; \vref{Lé. 16}~; \vref{No. 12:10-15}~; \vref{1 Ti. 2:1}~; \vref{Hé. 9:11-15} et \vref{1 Jn. 2:1}.

\DicoEntry{ISAAC}\textit{, de l'hébreu «~Yitschaq~»~: «~il rit~»}\newline
Fils de la promesse qui naquit à Abraham et Sara dans leur vieillesse. Il fut épargné quand Yahweh demanda à Abraham de lui offrir son fils en sacrifice. Isaac épousa Rébecca avec qui il eut deux fils~: Esaü et Jacob. Voir \vref{Ge. 17:17-21}~; \vref{Ge. 22:1-13} et \vref{Ge. 25:19-26}.

\DicoEntry{ISAÏ}\textit{, de l'hébreu «~Yishay~»~: «~je possède~»}\newline
Bethléhémite, petit-fils de Boaz et de Ruth, fils d'Obed et père de David. Voir \vref{Ru. 4:13-22}.

\DicoEntry{ISMAËL}\textit{, de l'hébreu «~Yishma'e'l~»~: «~Dieu entend~»}\newline
Fils d'Abraham et d'Agar, servante de Sara. Béni par Yahweh même après avoir été chassé avec sa mère par Sara, il devint le père des douze tribus ismaélites. Voir \vref{Ge. 16} et \vref{Ge. 25:12-16}.

\DicoEntry{ISSACAR}\textit{, de l'hébreu «~Yissaskar~»~: «~il donnera un salaire~»}\newline
Fils de Jacob et Léa, il devint l'ancêtre de la tribu d'Isaacar. Voir \vref{Ge. 30:18} et \vref{Ge. 49:14}.

\DicoEntry{ISRAËL}\textit{, de l'hébreu~: «~Yisra'el~»~: «~Dieu prévaut~»}\newline
Nom que Dieu donna à Jacob* après avoir lutté avec lui. Il s'agit également du nom désignant le peuple issu des douze fils de Jacob et le territoire que Dieu leur donna en héritage dont Jérusalem était la capitale. Après le schisme*, Israël se rapportait au royaume du nord composé de dix tribus. Voir \vref{Ge. 32:28}~; \vref{De. 33:5} et \vref{1 R. 12:1-24}.

\DicoEntry{IVRAIE}\textit{, du grec «~zizanion~»~: «~ivraie, ressemblant au blé, mais avec des grains noirs~»}\newline
Comme le blé, l'ivraie est une plante de la famille des graminées, mais c'est une mauvaise semence qui étouffe le blé. Elle représente les enfants du diable qui s'introduisent discrètement parmi les enfants de Dieu et qui en seront séparés uniquement à la fin du monde* pour aller vers la damnation éternelle. Voir \vref{Mt. 13:24-30,36-42}.

\DicoEntry{JACOB}\textit{, de l'hébreu «~Ya`aqob~»~: «~celui qui prend par le talon~» ou «~qui supplante~»}\newline
Fils d'Isaac et de Rebecca et frère jumeau d'Esaü. Il usa de stratagèmes pour ravir le droit d'aînesse ainsi que la bénédiction qui devaient revenir à son frère Esaü. Après avoir fui ce dernier, il se réfugia chez son oncle Laban dont il épousa les deux filles~: Léa et Rachel. De retour en Canaan après plusieurs années, Yahweh le rencontra en chemin et changea son nom en Israël. Jacob eut douze fils qui formèrent par la suite la nation d'Israël. Voir \vref{Ge. 25:21-34}~; \vref{Ge. 27-28}~; \vref{Ge. 29:1-30} et \vref{Ge. 49:1-28}.

\DicoEntry{JACQUES}\textit{, de l'hébreu~: «~Iakob~»~: «~qui supplante~» (variante de Jacob)}\newline
1. Fils de Zébédée et frère de Jean. Un des douze apôtres. Le roi Hérode Agrippa Ier* le fit mourir par l'épée. Voir \vref{Mt. 4:21-22}~; \vref{Lu. 6:12-16}~; \vref{Mc. 9:2-8}~; \vref{Mc. 14:32-33} et \vref{Ac. 12:1-2}.
\\2. Fils d'Alphée, un des douze apôtres~; il était aussi appelé Jacques le mineur. Voir \vref{Mt. 10:1-4}~; \vref{Mc. 15:40} et \vref{Lu. 6:12-16}.
\\3. Frère du Seigneur et apôtre, auteur de l'épître de Jacques. Voir \vref{Ac. 15:13-21}~; \vref{Ga. 1:19} et \vref{Mc. 6:3}.
\\4. Père de Jude, l'apôtre. Voir \vref{Lu. 6:16} et \vref{Ac. 1:13}.

\DicoEntry{JAPHET}\textit{, de l'hébreu «~Yepheth~»~: «~ouvert~», «~qui s'étend~»}\newline
Dernier des trois fils de Noé. Voir \vref{Ge. 10:1}.

\DicoEntry{JEAN}\textit{, de l'hébreu «~Yowchanan~» et du grec «~Ioannes~»~: «~Yahweh a fait grace~»}\newline
1. Fils de Zébédée, frère de Jacques et disciple aimé du Seigneur. Jean fut l'auteur de l'évangile éponyme, des trois épîtres qui portent son nom et de l'Apocalypse. Voir \vref{Mt. 10:2} et \vref{Jn. 13:23}.
\\2. Fils de Zacharie et Elisabeth, cousin de Jésus. Plus connu sous le nom de Jean-Baptiste, il fut envoyé pour préparer le chemin du Seigneur. Il fut décapité par Hérode Antipas*. Voir \vref{Lu. 1}~; \vref{Mal. 3:1-6}~; \vref{Mt. 1:12}~; \vref{Lu. 7:28} et \vref{Mt. 14:1-12}.

\DicoEntry{JELEK}\textit{, de l'hébreu «~yekeq~»~: «~jeune sauterelle~»}\newline
Désignant les sauterelles, il est souvent employé dans les Ecritures pour symboliser un grand nombre ou le dévoreur que Dieu envoie. Voir \vref{Joë. 1:4} et \vref{Na. 3:15-16}.

\DicoEntry{JÉRÉMIE}\textit{, de l'hébreu «~Yirmeyah~»~: «~celui que Yahweh a désigné~»}\newline
Fils de Hilkija, issu d'une famille de prêtres. Prophète de Yahweh, Jérémie fut appelé dès son plus jeune âge et exerça un ministère prophétique avant et pendant les premières années de déportation. Appelé à être eunuque*, il ne se maria jamais et n'eut point d'enfant. Il fut l'auteur des livres Jérémie et Lamentations de Jérémie.

\DicoEntry{JÉRICHO}\textit{, de l'hébreu «~Yeriychow~»~: «~ville de la lune~» ou «~ville des palmiers~»}\newline
Ville située à l'est de la tribu de Benjamin, près des rives du Jourdain. A la sortie du désert, les espions hébreux y furent cachés par Rahab la prostituée~; Jéricho fut ensuite détruite et livrée miraculeusement entre les mains d'Israël. C'est à Jéricho que Jésus guérit l'aveugle Bartimée et fut reçu par Zachée. Voir \vref{Jos. 2,6}~; \vref{Mc. 10:46-53} et \vref{Lu. 19:1-10}.

\DicoEntry{JÉROBOAM}\textit{, de l'hébreu «~Yarob'am~»~: «~le peuple devient nombreux~»}\newline
Fils de Nebath et serviteur de Salomon, il devint plus tard son ennemi. Après le schisme, il fut le premier roi du royaume du nord sur lequel il régna vingt-deux ans. Il fut une occasion de chute pour le peuple qu'il plongea dans l'idolâtrie*. Voir \vref{1 R. 11:26-40}~; \vref{1 R. 12-13}.

\DicoEntry{JÉRUSALEM}\textit{, de l'hébreu «~Yeruwshalem~»~: «~fondement de la paix~»}\newline
Ville située en Palestine, au nord de la Judée. Lors de la conquête de Canaan, la ville fut sous le contrôle des Jébusiens. Aux environs du Xe siècle av. J.-C., David reprit la ville alors devenue forteresse jébusienne. Il en fit la capitale politique et religieuse du royaume en y faisant établir l'arche de l'alliance. Salomon y construisit le temple* sur le mont Morija. En 586 av. J.-C., bien après le schisme*, les Babyloniens la détruisirent. Elle fut rebâtie par Néhémie après le retour de la captivité babylonienne. Jésus-Christ se lamenta sur la ville à cause de son incrédulité* et y annonça sa future destruction. Jérusalem fut en effet détruite par le général romain Titus en 70 ap. J.-C puis de nouveau rebâtie. Lors de son retour glorieux, le Seigneur Jésus posera ses pieds sur le Mont des Oliviers* qui est situé à Jérusalem. Le livre d'Apocalypse annonce après la fin du monde l'apparition de la Nouvelle Jérusalem, cité céleste. Voir \vref{2 S. 5:6-9}~; \vref{2 S. 6}~; \vref{Za. 14:1-4}~; \vref{2 Ch. 3:1}~; \vref{Lu. 19:41-44}~; et \vref{Ap. 21:2}.

\DicoEntry{JÉSUS}\textit{, de l'hébreu «~Yehowshuwa~»~: «~Yahweh est salut~»}\newline
Fils de l'homme* et fils de Dieu*, Jésus est le Dieu vivant manifesté en chair. Il fut conçu dans le ventre de Marie par la puissance du Saint-Esprit alors que cette dernière n'avait point connu d'homme. Selon les recherches de l'historien Flavius Josèphe, sa date de naissance se situerait autour de l'an 6 av. J.-C. - l'an zéro n'étant qu'une indication approximative. Fils adoptif de Joseph le charpentier et cousin de Jean Baptiste, il vécut la plus grande partie de sa vie en Galilée, dans la ville de Nazareth. Vers l'âge de 30 ans, il se fit baptiser dans le Jourdain et commença par la suite son ministère public. Grâce à sa vie exemplaire sans péché, il put se présenter comme une offrande agréable à Dieu répondant aux exigences de la justice divine pour sauver le monde. Remplissant toutes les prophéties relatives au Messie*, il fut trahi par un de ses disciples, Judas Iscariot*. Arrêté, maltraité puis crucifié, il mourut portant le poids des péchés de l'humanité, mais il ressuscita le troisième jour. Le salut réside dans la foi en son nom. Vivant de toute éternité, Jésus est le Dieu véritable et la vie éternelle. Sa justice ne tardera pas à se manifester~: il revient à toute vitesse. Voir \vref{Es. 53}~; \vref{Mt. 1:18-25}~; \vref{Mt. 2:23}~; \vref{Mc. 2:28}~; \vref{Lu. 1:36}~; \vref{Lu. 6:16}~; \vref{Lu. 24:46}~; \vref{Jn. 1:34}~; \vref{Ac. 4:12}~; \vref{2 Co. 5:21}~; \vref{1 Ti. 3:16}~; \vref{1 Pi. 2:21-25}~; \vref{1 Jn. 5:20} et \vref{Ap. 22:20}.

\DicoEntry{JETHRO}\textit{, de l'hébreu «~Yithrow~»~: «~son abondance, excellence~»}\newline
Prêtre de Madian chez qui Moise se réfugia après avoir fui l'Egypte. Il donna sa fille Séphora* pour femme à Moïse*. Voir \vref{Ex. 2:15-21}.

\DicoEntry{JEÛNE}\textit{, de l'hébreu «~tsuwn~»~: «~s'abstenir de nourriture~» et du grec «~nesteia~»~: «~le jeûne, un exercice volontaire et religieux~»}\newline
Privation totale ou partielle de nourriture dans le but d'humilier sa chair et d'adresser à Dieu des prières spécifiques. Le jeûne doit être exempt de toute hypocrisie et accompagné d'actes de justice pour être agréé de Dieu. Voir \vref{Es. 58}~; \vref{Est. 4:16}~; \vref{Da. 10}~; \vref{Mt. 6:16-18}~; \vref{Lu. 2:37}.

\DicoEntry{JÉZABEL}\textit{, de l'hébreu «~'Iyzebel~»~: «~Baal est l'époux~» ou «~l'impudique~»}\newline
Fille d'Ethbaal, roi de Sidon, et femme d'Achab roi d'Israël, elle extermina les prophètes de Yahweh et accueillait huit cent cinquante faux prophètes à sa table. Elle conduisit le peuple d'Israël dans l'idolâtrie au temps d'Elie*. Jézabel est associée à l'esprit du même nom qui prolifère de faux enseignements et entraîne le peuple de Dieu dans l'impudicité. Voir \vref{1 R. 16:31}, \vref{1 R. 18:4,19} et \vref{Ap. 2:20}.

\DicoEntry{JOB}\textit{, de l'hébreu «~'Iyowb~»~: «~haï, ennemi~» ou «~Je m'exclamerai~»}\newline
Originaire du pays d'Uts, homme prospère dont Yahweh témoigna l'intégrité et la droiture. Il subit en très peu de temps une succession de malheurs que Dieu permit pour se révéler à lui. Son histoire est racontée dans le livre portant son nom.

\DicoEntry{JOËL}\textit{, de l'hébreu «~Yow'el~»~: «~Yahweh est Dieu~»}\newline
Fils de Pethuel, il exerça la fonction de prophète dans le royaume de Juda. Il annonça la venue du Saint-Esprit sur toute chair à la fin des temps. Le contenu de son message se trouve dans le livre éponyme.

\DicoEntry{JONAS}\textit{, de l'hébreu «~Yonah~»~: «~colombe~»}\newline
Prophète de Yahweh envoyé à Ninive pour leur annoncer la destruction de leur ville. Son refus d'obéir à Dieu le conduisit dans le ventre d'un grand poisson. Son histoire est racontée dans le livre portant son nom.

\DicoEntry{JONATHAN}\textit{, de l'hébreu «~Yehownathan~»~: «~Yahweh a donné~»}\newline
Fils du roi Saül, homme de guerre reconnu. Lié à David par une très forte amitié, il protégea plusieurs fois ce dernier des relents meurtriers de son père. Il mourut à la bataille de Guilboa avec son père et ses frères. Voir \vref{1 S. 14:1-15}, \vref{1 S. 18:1-4}~; \vref{1 S. 19:1-8}~; \vref{1 S. 20} et \vref{1 S. 31:1-2}.

\DicoEntry{JOSAPHAT}\textit{, de l'hébreu «~Yehowshaphat~»~: «~Yahweh a jugé~»}\newline
Fils d'Asa et d'Azuba, il fut roi de Juda pendant vingt-cinq ans. Il eut un règne prospère et fit ce qui est droit aux yeux de Yahweh. Voir \vref{1 R. 15:24}~; \vref{1 R. 22:41-46} et \vref{2 Ch. 17}.

\DicoEntry{JOSEPH}\textit{, de l'hébreu «~Yowceph~»~: «~que Yahweh ajoute~» ou «~il enlève~»}\newline
1. Fils de Jacob et Rachel. Vendu comme esclave par ses frères, il devint, après plusieurs années de prison, gouverneur d'Egypte. Ses fils, Ephraïm et Manasée, furent adoptés par son père Jacob et furent les pères de deux des douze tribus d'Israël. Voir \vref{Ge. 30:22-24}~; \vref{Ge. 37,39,40,45,46}~; \vref{Ge. 48:5} et \vref{Jos. 14:4}.
\\2. Fils d'Héli, charpentier originaire de la tribu de Juda. Epoux de Marie, la mère de Jésus. Voir \vref{Mt. 1:18-25} et \vref{Mt. 13:55}.

\DicoEntry{JOSIAS}\textit{, de l'hébreu «~Yo'shiyah~»~: «~Yahweh guérit~»}\newline
Fils d'Amon, il devint roi de Juda à huit ans et y régna durant trente et un ans. Grand réformateur, à l'origine d'un grand réveil spirituel, il répara le temple, purifia le royaume des idoles et conclut une alliance de fidélité envers Yahweh. Voir \vref{2 R. 22-23}.

\DicoEntry{JOSUÉ}\textit{, de l'hébreu «~Yehowshuwa`~»~: «~Yahweh est salut~»}\newline
Fils de Nun de la tribu d'Ephraïm, choisi par Dieu pour succéder à Moïse. Accompagné de la puissante main de Dieu, il conduisit Israël à entrer en possession de Canaan. Son histoire se trouve dans le livre portant son nom.

\DicoEntry{JOURDAIN}\textit{, de l'hébreu «~Yarden~»~: «~celui qui descend~»}\newline
Très certainement le fleuve le plus connu des Ecritures, il est situé aux limites est de l'actuel territoire d'Israël. Josué et le peuple d'Israël passèrent le fleuve à sec. De même, Elie, puis Elisée, partagèrent les eaux du fleuve en deux. Après s'y être baigné sept fois sur les conseils d'Elisée, Naaman fut guéri de la lèpre. Jésus se fit baptiser par Jean dans le Jourdain. Voir \vref{Jos. 3}~; \vref{2 R. 2:8,12-14}~; \vref{2 R. 5:10-14} et \vref{Mt. 3:13-17}.

\DicoEntry{JOUR DU SEIGNEUR}\textit{}\newline
Jour où Yahweh manifestera sa justice et frappera les nations à cause de leurs péchés. Ce jour arrivera comme un voleur et surprendra beaucoup. Voir \vref{Es. 13:6-16}~; \vref{So. 1}~; \vref{2 Pi. 3:10}.

\DicoEntry{JUDA}\textit{, de l'hébreu «~Yehuwdah~»~: «~qu'il (Dieu) soit loué~»}\newline
Fils de Jacob et Léa, il est le père de la tribu du même nom installée au sud de Canaan. Sa descendance reçut la prédominance et la royauté~; David et Jésus-Christ étaient issus de cette tribu. Après le schisme*, Juda désigna aussi le nom du royaume du sud composé des tribus de Juda et Benjamin. Voir \vref{Ge. 29:35}~; \vref{Ge. 49:8-12}~; \vref{Jos. 15:1-12}~; \vref{1 R. 12:16-24}~; \vref{Mt. 1:1-16}.

\DicoEntry{JUDAS ISCARIOT}\textit{, de l'hébreu «~Yehuwdah~»~: «~qu'il (Dieu) soit loué~»}\newline
Fils de Simon Iscariot, il fut un des douze disciples de Jésus-Christ et était chargé de la trésorerie. Il trahit le Seigneur, ce qu'il regretta amèrement et le poussa à se suicider. Voir \vref{Mt. 26:14-16}~; \vref{Mt. 27:3-5}~; \vref{Lu. 6:16} et \vref{Jn. 12:4-6}.

\DicoEntry{JUDE}\textit{, de l'hébreu «~Yehuwdah~»~: «~qu'il (Dieu) soit loué~»}\newline
1. Fils de Jacques, un des douze apôtres, connu également sous le nom de Thadée. Voir \vref{Mc. 3:18} et \vref{Lu. 6:16}.
\\2. Prophète également appelé Barsabas, compagnon d'œuvre de Silas. Voir \vref{Ac. 15:22,32}.
\\3. Frère du Seigneur, auteur d'une épître qui porte son nom. Voir \vref{Mt. 13:55}~; \vref{Mc. 6:3} et \vref{Jud. 1:1}.

\DicoEntry{JUDÉE}\textit{, de l'hébreu «~Yehuwdah~»~: «~qu'il (Dieu) soit loué~»}\newline
Région située au sud de la Palestine où se trouvent notamment Jérusalem et Bethléhem. Elle correspondrait approximativement au territoire de l'ancien royaume de Juda. Ce terme n'est pas utilisé dans le Tanakh. Voir \vref{Mt. 2:1}~; \vref{Mc. 1:5} et \vref{Ga. 1:22}.

\DicoEntry{JUGE, JUGEMENT}\textit{, de l'hébreu «~shaphat~»~: «~juger, gouverner, défendre, punir~», «~agir comme un législateur, juge ou gouverneur~», «~exécuter un jugement~»}\newline
Dans toute la Parole, Yahweh est présenté comme le juge droit et incorruptible. Après la sortie d'Egypte, des juges ont été suscités par Dieu au milieu d'Israël pour délivrer le peuple de ses ennemis et le ramener vers lui (voir livre des Juges). Le Seigneur a toujours envoyé des prophètes pour annoncer ses jugements et ses décisions ainsi que des juges pour faire respecter sa loi. Sous la Nouvelle Alliance, l'homme spirituel est appelé à juger (discerner selon la Parole), mais condamner et décider du sort final d'une personne demeure la prérogative de Dieu. Yahweh est en effet le juste juge qui siège et tranche non seulement au tribunal de Christ, mais également au jugement dernier. Voir \vref{Ge. 18:25}~; \vref{Jé. 11:20}~; \vref{2 Co. 5:10} et \vref{Ap. 20:11-15}.

\DicoEntry{JUPITER}\textit{, du grec «~Zeus~»~: «~un père des secours~»}\newline
Divinité romaine assimilée à Zeus chez les Grecs. Lors d'une guérison miraculeuse à Lystres, la foule pensa voir en Paul la réincarnation de Mercure et en Barnabas celle de Jupiter. Pour cela, on voulut les adorer, ce qu'ils refusèrent avec véhémence. Voir \vref{Ac. 14:8-15}.

\DicoEntry{JUSTE, JUSTICE}\textit{, de l'hébreu «~tsedeq~»~: «~droiture, exactitude, conforme~» ou encore «~tsadiq~»~: «~juste, exact, innocent~» et du grec «~dikaiosune~»~: «~la condition acceptable par Dieu~» ou «~intégrité, vertu, pureté de vie, droiture~»}\newline
En qualité de juste juge, Yahweh a toujours recherché cette qualité chez l'homme, mais il ne l'a pas trouvé déclarant que nul n'est juste. Au travers de l'œuvre de la croix et avec l'aide du Saint-Esprit, le chrétien peut à présent marcher dans la justice de Dieu. Il est appelé à la rechercher plus que tout et à devenir esclave de la justice. Voir \vref{Mt. 5-7}~; \vref{Lu. 1:75}~; \vref{Ro. 3:10}~; \vref{Ro. 6:18} et \vref{2 Ti. 4:8}.

\DicoEntry{JUSTIFICATION}\textit{, du grec «~dikaiosis~»~: «~état du juste~»}\newline
Au travers de l'œuvre de la croix, Jésus-Christ est devenu la justification de tous ceux qui croient en lui, les rendant acceptables et libres de toute culpabilité. Voir \vref{Ro. 3:23-28}~; \vref{Ro. 4:25} et \vref{Ro. 5:18}.

\DicoEntry{KORÉ}\textit{, de l'hébreu «~Qorach~»~: «~chauve~»}\newline
Fils de Jitsehar, originaire de la tribu de Lévi, il se révolta avec Dathan* et Abiram* contre Moïse* et Aaron*. Suite à sa rébellion, il périt avec les gens de sa maison. Voir \vref{No. 16:1-35}.

\DicoEntry{LAÏC}\textit{, du grec «~laos~»~: «~peuple~»}\newline
Notion propre à l'Eglise catholique romaine. Opposé au clergé*, les laïcs sont les autres membres de l'église, ceux qui n'ont pas de fonction dirigeante, mais qui sont tout de même appelés à honorer Dieu dans leur vie et faire connaître leur foi au milieu du monde.

\DicoEntry{LANGUES}\textit{, de l'hébreu «~lashown~»~: «~langue, langage~» et du grec «~glossa~»~: «~la langue~» ou «~le langage d'un peuple particulier~»}\newline
Selon les Ecritures, les langues sont nées à Babylone* lorsque les hommes se sont rebellés contre la volonté de Yahweh et que ce dernier a confondu leur langage dans le but de les disperser. Dans la Parole sont cités différents types de langues, chacune liée à un don ou une manifestation particulière de l'Esprit de Dieu. Lors de l'effusion du Saint-Esprit à la Pentecôte, les disciples reçurent la capacité de parler des merveilles de Dieu dans des langues étrangères. Il s'agit du don spirituel* appelé la diversité des langues et concerne uniquement les langues usuelles. Il existe également des langues angéliques ou dites inconnues que le croyant peut utiliser pour s'adresser à Dieu. Les langues étrangères tout comme les langues des anges peuvent donner lieu à une interprétation, c'est ce qu'on appelle le don d'interpréter les langues. Voir \vref{Ge. 11}~; \vref{Ac. 2:1-11}~; \vref{1 Co. 12:10}~; \vref{1 Co. 13:1} et \vref{1 Co. 14:1-14,26-27}.

\DicoEntry{LAODICÉE}\textit{, du grec «~Laodikeia~»~: «~justice du peuple~»}\newline
Capitale de la Phrygie, l'une des provinces de l'Asie Mineure, réputée dans le domaine du commerce, notamment dans l'industrie textile. Ses vêtements et sa tapisserie principalement de couleur noire, firent sa renommée. Elle possédait une grande école de médecine qui fabriquait des remèdes réputés pour les yeux, notamment le fameux collyre. L'église de Laodicée est la dernière à qui fut adressée une lettre dans l'Apocalypse. Caractérisée par la tiédeur, l'affection aux choses terrestres et l'aveuglement spirituel, le Seigneur l'appela à la repentance*. Elle est l'image de l'église matérialiste. Voir \vref{Ap. 3:14-22}.

\DicoEntry{LAZARE}\textit{, du grec «~Lazaros~»~: «~Yahweh a secouru~»}\newline
1. Homme pauvre qui fut recueilli dans le sein d'Abraham après sa mort. Voir \vref{Lu. 16:19-21}.
\\2. Frère de Marthe et de Marie de Béthanie, et ami de Jésus-Christ qui le ressuscita des morts. Voir \vref{Jn. 11}.

\DicoEntry{LÉA}\textit{, de l'hébreu «~Le'ah~»~: «~lasse~»}\newline
Fille aînée de Laban et première femme de Jacob. Elle enfanta six fils, pères de six des douze tribus d'Israël (Ruben, Siméon, Lévi, Juda, Issacar et Zabulon) ainsi qu'une fille nommée Dina. Voir \vref{Ge. 29:16-23}~; \vref{Ge. 30:21} et \vref{Ge. 35:23}.

\DicoEntry{LÉMEC}\textit{, de l'hébreu «~Lemek~»~: «~puissant~»}\newline
Fils de Metuschaël et descendant de Caïn, il fut le premier polygame de l'histoire en prenant deux femmes~: Ada et Tsilla. Voir \vref{Ge. 4:16-24}.

\DicoEntry{LÈPRE}\textit{, de l'hébreu «~tsara'~»~: «~être morbide de peau~»}\newline
Commune en Egypte et en orient, maladie de la peau dont le virus peut se développer dans tout le corps. Contagieuse, elle peut même souiller les vêtements et les habitations. Sous la loi mosaïque, les personnes atteintes de cette maladie étaient considérées comme impures et devaient se tenir à l'écart. Durant son ministère, Jésus guérit plusieurs lépreux. Voir \vref{Lé. 13-14} et \vref{Lu. 17:11-14}.

\DicoEntry{LEVAIN}\textit{, de l'hébreu «~chametz~»~: «~ce qui est levé~»}\newline
Symbole du mal et de la corruption, le levain était interdit dans la quasi-totalité des offrandes. Jésus a assimilé le levain des pharisiens à l'hypocrisie, à la doctrine erronée. Les chrétiens sont appelés à faire disparaître le vieux levain et à devenir le levain du monde en y faisant progresser l'évangile du royaume. Voir \vref{Lé. 2:11}~; \vref{Mt. 16:6-12}~; \vref{Mt. 13:33} et \vref{1 Co. 5:6-7}.

\DicoEntry{LÉVI, LÉVITES}\textit{, de l'hébreu «~Leviy~»~: «~attachement~»}\newline
Fils de Jacob et Léa, Lévi participa avec son frère Siméon au massacre des hommes de la ville de Sichem après le viol de leur sœur Dina. Consacrés au service de Yahweh, ses descendants, les Lévites, n'eurent point d'héritage en Canaan, mais habitèrent différentes villes qui leur furent spécifiquement attribuées en Israël. Voir \vref{Ge. 29:34}~; \vref{Ge. 34}~; \vref{No. 18:20-24} et \vref{Jos. 13:14}.

\DicoEntry{LOI}\textit{, de l'hébreu «~towrah~»~: «~loi, direction, commandement~», «~loi mosaïque~»}\newline
L'ensemble des préceptes et ordonnances relatifs à l'alliance conclue entre Yahweh et le peuple hébreu, par l'intermédiaire de Moïse, est contenu dans les cinq premiers livres de la Bible appelée aussi «~le Pentateuque~». Selon la tradition juive, il existerait 613 commandements relatifs à la moralité, la vie en société et le culte rendu à Yahweh. L'homme en étant incapable, Jésus-Christ a accompli les exigences de la loi. Il est donc possible aux hommes d'obtenir le salut par la foi et non plus par les œuvres. La loi est maintenant gravée dans les cœurs des enfants de Dieu à qui le Saint-Esprit rappelle les paroles de Jésus. Voir \vref{Ex. 18:20}~; \vref{Ex. 24:12}~; \vref{Jn. 14:26} et \vref{Ro. 3:19-31}.

\DicoEntry{LOI DU PÉCHÉ}\textit{, du grec «~nomos~»~: «~toute chose établie, une coutume, un commandement~»}\newline
Loi spirituelle inscrite dans la chair qui pousse l'homme charnel à se révolter contre Dieu en commettant le péché. Voir \vref{Ro. 7:13-25}.

\DicoEntry{LOT}\textit{, de l'hébreu~: «~Lowt~»~: «~voile, couverture~»}\newline
Fils de Haran et neveu d'Abraham, Lot quitta Ur avec ce dernier avant de s'en séparer. Grâce à l'intercession d'Abraham, il fut sauvé de la destruction de Sodome avec ses deux filles. Ces dernières enivrèrent leur père et eurent des relations incestueuses avec lui de qui naquirent Moab, père des Moabites, et Amon, père des Ammonites. Voir \vref{Ge. 11:31}~; \vref{Ge. 13:1-13}~; et \vref{Ge. 19}.

\DicoEntry{LUC}\textit{, du grec «~Loukas~»~: «~qui donne la lumière~»}\newline
Médecin de métier, il fut un des compagnons d'œuvre de Paul et l'auteur de l'évangile qui porte son nom et du livre Actes des Apôtres. Voir \vref{Col. 4:14} et \vref{Phm. 1:24}.

\DicoEntry{MACÉDOINE}\textit{, du grec «~Makedonia~»~: «~terre étendue~»}\newline
Province romaine située au nord de la Grèce. Paul y effectua quelques voyages missionnaires et y implanta plusieurs assemblées. Voir \vref{Ac. 16:9-12}~; \vref{Ac. 20:1-3}~; \vref{1 Co. 8:1} et \vref{2 Co. 11:9} et \vref{Ro. 15:23}.

\DicoEntry{MADIAN}\textit{, de l'hébreu «~Midyan~»~: «~lutte, dispute~»}\newline
Un des fils issu de l'union d'Abraham* et Ketura, il devint l'ancêtre des Madianites, peuple qui habita à l'est de Canaan et au nord du désert d'Arabie. Voir \vref{Ge. 25:1-2}~; \vref{No. 31:1-12} et \vref{Jg. 6:2}.

\DicoEntry{MAGOG}\textit{, de l'hébreu «~Magowg~»~: «~territoire de montagne, qui domine~»}\newline
Fils de Japhet. Associé à Gog, il correspond aussi à la nation d'où vient le roi Gog qui fera la guerre à Dieu et à son peuple juste avant le jugement dernier. Voir \vref{Ge. 10:2,9} et \vref{Ap. 20:8}.

\DicoEntry{MAIN}\textit{, de l'hébreu «~yad~»~: «~main, force, pouvoir~»}\newline
Partie du corps permettant de toucher, saisir ou posséder, elle représente aussi l'action, la provision, la protection ou le joug. Tout au long des Ecritures, la main de Yahweh révèle sa puissance et sa bienveillance. Voir \vref{Es. 40:2}~; \vref{Jé. 18:6}~; \vref{Ps. 71:4}~; \vref{Pr. 10:4}~; \vref{Mc. 14:58}~; \vref{Lu. 11:20} et \vref{Ac. 11:21}.

\DicoEntry{MALACHIE}\textit{, de l'hébreu «~Mal`akiy~»~: «~mon messager~»}\newline
Dernier prophète du Tanakh, il condamna les péchés et l'hypocrisie des enfants d'Israël et annonça la venue de Jean-Baptiste. L'ensemble de ses prophéties est contenu dans le livre portant son nom.

\DicoEntry{MALÉDICTION}\textit{, de l'hébreu «~arar, meerah, qelalah~» et du grec «~ara, katara~»}\newline
Parole attirant le malheur sur un bien, une personne ou un peuple. Dieu a le pouvoir de maudire et aussi d'écarter toute malédiction. La malédiction de Dieu, contraire de la bénédiction*, fait suite à la désobéissance. A la nouvelle naissance, toutes les chaînes de malédiction qui liaient le chrétien sont brisées. Le chrétien ne doit pas maudire, mais bénir en tout temps, même ses ennemis. Voir \vref{De. 28:15-68}~; \vref{Mt. 5:44}~; \vref{2 Co. 5:17} et \vref{Ro. 8:1}.

\DicoEntry{MALFAITEUR REPENTANT}\textit{}\newline
Un des hommes coupables qui fut crucifié à côté de Jésus. Son humilité, sa sincérité et sa repentance lui permirent d'accéder au salut, Jésus-Christ lui ayant garanti l'accès au paradis. Voir \vref{Lu. 23:33-43}.

\DicoEntry{MAMON}\textit{, du grec «~Mammonas~»~: «~richesses~»}\newline
Dieu de l'argent. Jésus utilisa ce terme pour personnifier la richesse que beaucoup idolâtrent et qui est par conséquent en concurrence avec Yahweh dans le cœur de certains. Voir \vref{Mt. 6:24}.

\DicoEntry{MANASSÉ}\textit{, de l'hébreu «~Menashsheh~»~: «~oublieux~»}\newline
1. Fils aîné de Joseph* et d'Asnath, adopté par Jacob avant sa mort, ancêtre de la tribu de Manassé. Voir \vref{Ge. 41:51}~; \vref{Ge. 48:5} et \vref{Jos. 14:4}.
\\2. Fils d'Ezéchias et de Hephtsiba, il fut l'un des pires rois du royaume de Juda qui régna 55 ans. Malgré le réveil impulsé par son père, il se détourna entièrement de Yahweh et servit des dieux étrangers. Voir \vref{2 R. 21:1-18}.

\DicoEntry{MANNE}\textit{, de l'hébreu «~man~»~: «~qu'est-ce que cela~?~»}\newline
Nourriture céleste - à l'aspect de la graine de coriandre et au goût de gâteau de miel - que Dieu donna quotidiennement aux Israélites durant toute leur marche dans le désert. \vref{Ex. 16:15,31-35}.

\DicoEntry{MARANATHA}\textit{, de l'araméen «~maran atha~»~: «~le Seigneur vient~» ou «~Seigneur, viens~»}\newline
Expression prononcée par Paul quand il s'adressa aux Corinthiens et qui doit également être le cri du cœur de tout enfant de Dieu. Voir \vref{1 Co. 16:22} et \vref{Ap. 22:17,20}.

\DicoEntry{MARC}\textit{, du grec «~Markos~»~: «~une défense~» ou «~grand marteau~»}\newline
Appelé aussi Jean, cousin de Barnabas, il fut la cause de la séparation de Paul et Barnabas. Il partit avec ce dernier à Chypre et devint par la suite un fidèle compagnon d'œuvre de Paul. Il écrivit l'évangile portant son nom. Voir \vref{Ac. 12:12}~; \vref{Ac. 15:36-39}~; \vref{Col. 4:10} et \vref{Phm. 24}.

\DicoEntry{MARDOCHÉE}\textit{, de l'hébreu «~Mordekay~»~: «~petit homme~»}\newline
Fils de Jaïr de la tribu de Benjamin*, il adopta Esther*, fille de son oncle. Il sauva la vie du roi Assuérus* en déjouant les plans de Bigthan et Théresch et préserva le peuple juif des desseins meurtriers d'Haman. Il devint puissant dans la maison du roi et instaura la fête du Purim. Voir le livre d'Esther.

\DicoEntry{MARIAGE}\textit{, de l'hébreu «~chathan~»~: «~devenir un gendre, s'allier~»}\newline
Bénédiction de Dieu, le mariage est une alliance en principe indissoluble entre un homme et une femme dans le but d'accomplir le plan de Dieu. Il doit être célébré dans le respect des autorités du pays dans lequel le couple se trouve et honoré de tous, particulièrement des parents dont la bénédiction est essentielle. Voir \vref{Ge. 2:22-24}~; \vref{Ge. 24:60}~; \vref{Pr. 18:22}~; \vref{1 Co. 7} et \vref{Hé. 13:4}. Voir commentaire en \vref{Mt. 19:6}.

\DicoEntry{MARIE}\textit{, de l'hébreu «~Miryam~»~: «~rébellion, obstination~»}\newline
1. Sœur de Moïse et d'Aaron, prophètesse. Elle se rebella contre Moïse et fut frappée par la lèpre, mais en guérit grâce à l'intercession de Moïse. Voir \vref{Ex. 15:20} et \vref{No. 12}.
\\2. Mère de Jésus~: Elle conçut, par la vertu du Saint-Esprit, Jésus homme. Elle devint une de ses disciples et se trouvait parmi ceux qui persévéraient dans la prière dans la chambre haute lors de l'effusion du Saint-Esprit promis. Voir \vref{Es. 7:14}~; \vref{Mt. 1:18-25}~; \vref{Mc. 15:40-41}~; \vref{Lu. 1:26-38} et \vref{Ac. 1:13-14}.
\\3. Marie de Magdala~: Elle fut délivrée de sept démons par Jésus qu'elle suivit pendant son ministère terrestre, et ce jusqu'à la croix. Elle fut mandatée par le Seigneur pour annoncer sa résurrection aux apôtres. Voir \vref{Mt. 27:55-56}~; \vref{Mc. 16:1-11}~; \vref{Lu. 8:2} et \vref{Jn. 20:1-18}.
\\4. Marie de Béthanie~: Sœur de Marthe* et de Lazare*, que Jésus ressuscita des morts. Contrairement à sa sœur, elle choisit la bonne part en restant aux pieds du Maître. Elle toucha le cœur de ce dernier en l'oignant d'un parfum de grand prix. Voir \vref{Lu. 10:38-42}~; \vref{Jn. 11:1-44} et \vref{Jn. 12:1-7}.

\DicoEntry{MARTHE}\textit{, du grec «~Martha~»~: «~maîtresse, dame~»}\newline
Sœur de Lazare* - dont elle fut témoin de la résurrection - et de Marie* de Béthanie, elle reçut Christ dans sa maison, mais ce dernier lui reprocha son activisme au détriment de l'écoute de sa Parole. Voir \vref{Lu. 10:38-42} et \vref{Jn. 11:1-44}.

\DicoEntry{MATTHIAS}\textit{, de l'hébreu «~Mattithyah~»~: «~don de Yahweh~»}\newline
Disciple de Jésus et témoin oculaire de son ministère, il fut désigné pour devenir l'un des douze apôtres en remplacement de Judas Iscariot* qui avait trahi le Seigneur pour ensuite se suicider. Voir \vref{Ac. 1:15-26}.

\DicoEntry{MATTHIEU}\textit{, du grec «~Matthaios~»~: «~don de Yahweh~»}\newline
Collecteur d'impôts, il fut l'un des douze apôtres de Jésus et l'auteur de l'évangile qui porte son nom. Voir \vref{Mt. 9:9} et \vref{Mt. 10:3}.

\DicoEntry{MEÉDIATEUR}\textit{, du grec «~mesites~»~: «~celui qui intervient entre deux parties~», «~intermédiaire de communication~»}\newline
Moïse a exercé cette fonction auprès du peuple d'Israël qui avait expressément demandé que Dieu ne leur parle pas directement. Christ, garant d'une Nouvelle Alliance, est à présent l'unique intermédiaire et médiateur entre Dieu et les hommes. Voir \vref{Ex. 20:19}~; \vref{1 Ti. 2:5}~; \vref{Hé. 8:6} et \vref{Hé. 9:15}.

\DicoEntry{MELCHISÉDEK}\textit{, de l'hébreu «~Malkiy-Tsedeq~»~: «~roi de justice~»}\newline
Roi de Salem et prêtre du Dieu Très-Haut, il était une apparition de Jésus-Christ avant son apparition. Il bénit Abraham après sa victoire contre Kedorlaomer. Jésus-Christ est grand prêtre selon l'ordre de Melchisédek. Voir \vref{Ge. 14:14-20}~; \vref{Hé. 5:5-10} et \vref{Hé. 6:20}.

\DicoEntry{MENSONGE}\textit{, de l'hébreu «~sheqer~»~: «~mensonge, déception, fausseté, tromperie, fraude~» et du grec «~pseudos~»~: «~fausseté consciente et intentionnelle~»}\newline
Modification de la vérité. Satan est appelé père du mensonge et les menteurs auront droit à la même sentence que lui. Voir \vref{Ex. 20:16}~; \vref{Jn. 8:44} et \vref{Ap. 21:8}.

\DicoEntry{MÉSOPOTAMIE}\textit{, de l'hébreu «~'Aram Naharayim~»~: «~pays entre deux fleuves~»}\newline
Située entre le Tigre et l'Euphrate, région correspondant à l'actuel Irak. Avant son appel, Abraham vivait à Ur en Chaldée qui se trouvait au sud de la Mésopotamie. Voir \vref{Ge. 11:31}.

\DicoEntry{MESSIE}\textit{, de l'hébreu «~mashiyach~»~: «~oint, celui qui est l'oint~»}\newline
Voir CHRIST.

\DicoEntry{MICHÉE}\textit{, de l'hébreu «~Miykayehuw~»~: «~qui est comme Dieu~?~»}\newline
Originaire de Moréscheth, Michée exerça la fonction de prophète dans le royaume du sud au temps d'Ezéchias, roi de Juda. L'ensemble de ses prophéties se trouve dans le livre éponyme.

\DicoEntry{MICHEL ou MICHAËL}\textit{, de l'hébreu «~Miyka'el~»~: «~qui est semblable à Dieu~?~»}\newline
Archange* de Dieu, il est un des principaux chefs des anges*. Souvent présent dans les grandes batailles, il lutta notamment contre le roi de Perse et contre le diable. Voir \vref{Da. 10:13-21}~; \vref{Jud. 1:9} et \vref{Ap. 12:7}.

\DicoEntry{MILLE}\textit{, du grec «~million~»~: «~distance de mille pas~»}\newline
Unité de mesure romaine correspondant à 1480m environ. Voir \vref{Mt. 5:41}.

\DicoEntry{MILLÉNIUM}\textit{}\newline
Période de paix de mille ans durant laquelle le Seigneur régnera sur la terre. Voir \vref{Es. 11,12} et \vref{Ap. 20:2-7}.

\DicoEntry{MINISTÈRE}\textit{, du grec «~diakonia~»~: «~service~», dérivé du mot grec «~diakonos~»~: «~domestique~»}\newline
voir SERVICE. 

\DicoEntry{MISÉRICORDE}\textit{, de l'hébreu «~checed~»~: «~bonté, miséricorde, fidélité~» et du grec «~eleos~»~: «~bonne volonté envers le misérable associée à un désir de l'aider~»}\newline
Comme en témoigne le plan du salut* qu'il a déployé, Dieu est riche en miséricorde. Le disciple de Christ doit comme son maître se revêtir d'entrailles de miséricorde afin de représenter le Royaume de Dieu. \vref{Ge. 24:7}~; \vref{No. 24:18}~; \vref{Mt. 9:13}~; \vref{Lu. 1:78}~; \vref{Ro. 11:31} et \vref{2 Jn. 1:3}.

\DicoEntry{MOAB}\textit{, de l'hébreu «~Mow'ab~»~: «~issu d'un père~»}\newline
Fils de Lot*, né de sa relation incestueuse avec sa fille aînée, il donna naissance au peuple des moabites. Ils s'établirent au sud-est de la mer morte et s'opposèrent plusieurs fois aux enfants d'Israël. Voir \vref{Ge. 19:37}~; \vref{Jg. 3:12}~; \vref{2 S. 8:2}~; \vref{Ez. 25:8-11}.

\DicoEntry{MODALISME}\textit{}\newline
Doctrine enseignée à Rome au début du troisième siècle par Sabellius selon laquelle le Père, le Fils et le Saint-Esprit sont différents aspects au travers desquels Dieu se révèle et non trois personnes distinctes. Réfutant ainsi la doctrine de la trinité* largement acceptée par les catholiques, Sabellius fut condamné par le pape Callixte à cause de son enseignement pourtant biblique. Voir \vref{1 Th. 3:11}~; \vref{2 Th. 2:16-17} et \vref{1 Jn. 5:20}.

\DicoEntry{MOÏSE}\textit{, de l'hébreu «~Mosheh~»~: «~tiré de~»}\newline
Issu de la tribu de Lévi, il fut miraculeusement sauvé du massacre des enfants de sa génération pendant la servitude d'Israël en Egypte. Il vécut les quarante premières années de sa vie dans la maison de Pharaon puis les quarante suivantes dans le désert auprès de Madian*. A l'issue de cette deuxième période, Yahweh se révéla à lui et le mandata pour délivrer le peuple d'Israël de la captivité égyptienne afin de le faire entrer dans la terre promise. Après l'avoir fait sortir au milieu des miracles et des prodiges, Moïse conduisit le peuple dans le désert pendant quarante années au cours desquelles il leur communiqua l'intégralité de la loi*. Il mourut à la porte de la terre promise à l'âge de cent vingt ans. On lui attribue l'écriture des cinq premiers livres du Tanakh. Voir \vref{Ex. 1-2}~; \vref{Ex. 12:40-41}~; \vref{Ex. 14:21-31}~; \vref{Ex. 24:12}~; \vref{De. 8:2}~; \vref{De. 34:5-7}~; \vref{Ac. 7:20-43} et \vref{Hé. 11:23-29}.

\DicoEntry{MOISSON}\textit{, de l'hébreu «~qatsiyr~»~: «~moisson, travail de la moisson, récolte~»}\newline
Sous la loi, la fête des prémices avait lieu lors de la moisson. Jésus utilise ce terme pour parler du champ missionnaire, les personnes à qui l'évangile doit être annoncé. Dans le cadre de la fin du monde, la moisson se rapporte au jugement de Dieu qui va apporter la séparation entre ses fils et les fils du diable. Voir \vref{Lé. 23:10-14}~; \vref{Mt. 9:37-38}~; \vref{Mt. 13:33-43}.

\DicoEntry{MOLOC}\textit{, de l'hébreu «~Molek~»~: «~roi, conseiller~»}\newline
Divinité vénérée par les Ammonites à qui il était coutume de sacrifier des enfants brûlés vifs. Les Israélites se prostituèrent plusieurs fois à Moloc. Voir \vref{1 R. 11:5-7} et \vref{2 R. 23:10}.

\DicoEntry{MONT DES OLIVIERS}\textit{}\newline
Colline située à l'est de Jérusalem près de la vallée du Cédron. C'est du Mont des Oliviers que Jésus fut enlevé au ciel après avoir donné ses dernières recommandations aux apôtres~; c'est à ce même endroit qu'il posera les pieds lors de son glorieux retour. Voir \vref{Za. 14:1-4} et \vref{Ac. 1:4-12}.

\DicoEntry{MORT}\textit{, de l'hébreu «~muwth~»~: «~mourir, tuer, être exécuté~» et du grec «~thanatos~»~: «~mort du corps~»}\newline
La Bible distingue deux morts. La première entra dans le monde suite à la désobéissance de l'homme et correspond à la séparation d'avec Dieu et à la mort physique. La deuxième mort concerne uniquement ceux dont le nom n'est pas écrit dans le livre de vie et correspond à la souffrance éternelle dans l'étang de feu. Voir \vref{Ge. 3}~; \vref{Ro. 5:12}~; \vref{Ro. 6:23} et \vref{Ap. 20:11-15}.

\DicoEntry{MYRRHE}\textit{, de l'hébreu «~more~»~: «~myrrhe~»}\newline
Résine provenant de certains arbres d'Asie et d'Afrique, réputée pour son arôme de grand prix. Elle était utilisée sous forme d'huile pour l'onction sainte et pouvait atténuer les douleurs quand elle était mélangée au vin. Les mages offrirent de la myrrhe à Jésus lors de sa naissance. Voir \vref{Ex. 30:22-30}~; \vref{Mt. 2:11}~; \vref{Mt. 27:34} et \vref{Mc. 15:23}.

\DicoEntry{NAHUM}\textit{, de l'hébreu «~Nachuwm~»~: «~consolation, qui a compassion~»}\newline
Prophète de Yahweh né à Elkosch, il annonça la destruction de Ninive. L'ensemble de ses prophéties se trouve dans le livre portant son nom.

\DicoEntry{NAISSANCE D'EN HAUT}\textit{, du grec «~anothen~»~: «~depuis le haut, depuis un endroit plus élevé~»}\newline
Naissance d'eau et d'esprit symbolisant respectivement la Parole qui purifie et le Saint-Esprit* qui est le gage de l'appartenance à Dieu. La naissance d'en haut est l'œuvre du Saint-Esprit qui délivre une personne du royaume des ténèbres et la transporte dans le Royaume de Dieu. L'homme charnel devient alors spirituel, le cœur de pierre est ôté pour accueillir un cœur de chair, le citoyen terrestre se transforme en citoyen céleste et le vieil homme laisse place à une nouvelle créature. Voir \vref{Ez. 36:25-27}~; \vref{Jn. 3:1-8}~; \vref{Ja. 1:18}~; \vref{1 Co. 12:13} et \vref{2 Co. 5:17}~; \vref{Ep. 2:6}~; \vref{Ep. 5:26}~; \vref{1 Jn. 3:9}.

\DicoEntry{NATHAN}\textit{, de l'hébreu «~Nathan~»~: «~il (Yahweh) a donné~»}\newline
Prophète de Yahweh au temps du roi David. Il prophétisa le règne éternel de la postérité de David et la construction du temple par son fils. Il reprit David lorsque ce dernier fit assassiner Urie* pour prendre sa femme. Voir \vref{2 S. 7,12}.

\DicoEntry{NAZARÉEN, NAZIRÉEN}\textit{, de l'hébreu «~naziyr~»~: «~consacré ou voué~»}\newline
Terme pouvant désigner soit un habitant de la ville de Nazareth, soit une personne qui s'est consacrée à Yahweh dans le cadre d'un vœu de naziréat. Voir \vref{No. 6}.

\DicoEntry{NAZARETH}\textit{, du grec «~Nazareth~»~: «~verdoyant, germe, rejeton~»}\newline
Ville située dans la région de Galilée où Jésus passa la majeure partie de sa vie. Voir \vref{Mt. 2:22-23}.

\DicoEntry{NEBUCADNETSAR}\textit{, (règne~: 605 av. J.-C. – 562 av. J.-C.), «~Nebuwkadne'tstsar~»~: «~que Nebo protège la couronne, les frontières~» (origine inconnue)}\newline
Roi de Babylone, il mit fin au royaume de Juda en emmenant le peuple en captivité~; il détruisit le temple de Jérusalem. Il reçut l'interprétation de plusieurs songes au travers de Daniel* et reconnut le règne dominant et éternel de Yahweh. Voir \vref{2 R. 25}, \vref{Da. 1:1} et \vref{Da. 2,4}.

\DicoEntry{NÉHÉMIE}\textit{, de l'hébreu «~Nechemyah~»~: «~Yahweh a consolé~»}\newline
Fils d'Hacalia, il fut échanson du roi Artaxerxès* à Suse, pendant la captivité de Juda. Il entreprit la réparation des murailles de Jérusalem et initia une réforme en son temps. Il devint ensuite gouverneur de Juda. Son histoire est racontée dans le livre éponyme.

\DicoEntry{NÉPHILIM}\textit{, de l'hébreu «~nephiyl~»~: «~géant~», racine~: «~naphal~»~: «~tomber, chuter~»}\newline
Etres de grande taille nés de l'union des fils de Dieu et des filles des hommes avant le déluge. On en retrouve aussi en Canaan lorsque les douze espions hébreux étaient allés observer la terre promise. Voir \vref{Ge. 6:4} et \vref{No. 13:32-33}.

\DicoEntry{NEPHTHALI}\textit{, de l'hébreu «~Naphtaliy~»~: «~lutte, mon combat~»}\newline
Fils de Jacob* et de Bilha, servante de Rachel*, il est l'ancêtre de la tribu de Nephthali. Voir \vref{Ge. 30:8} et \vref{Ge. 49:21}.

\DicoEntry{NICODÈME}\textit{, du grec «~Nikodemos~»~: «~victorieux du peuple~»}\newline
Docteur de la loi, il s'approcha de Jésus de nuit, qui l'enseigna sur la naissance d'en haut. Après la crucifixion, il aida Joseph d'Arimathée pour embaumer le corps du Seigneur et pour le mettre dans un sépulcre. Voir \vref{Jn. 3:1-21} et \vref{Jn. 19:38-42}.

\DicoEntry{NICOLAÏTES}\textit{, du grec «~Nikolaites~»~: «~destruction du peuple~»}\newline
Secte suivant la doctrine de Nicolas, liée à la doctrine de Balaam, qui poussait à la consommation de viandes sacrifiées aux idoles et à l'impudicité. Voir \vref{Ap. 2:6,14-15}.

\DicoEntry{NIL}\textit{, de l'hébreu «~Shiychowr~»~: «~sombre, noir, boueux~»}\newline
Principal fleuve d'Egypte situé à l'est du pays. Voir \vref{Es. 23:3}~; \vref{Jé. 2:18}.

\DicoEntry{NIMROD}\textit{, de l'hébreu «~Nimrowd~»~: «~rebelle~»}\newline
Fils de Cush et descendant de Noé. Chasseur, il fut le premier homme puissant sur la terre et régna sur plusieurs villes dont Babel*. Voir \vref{Ge. 10:8-11} et \vref{Ge. 11:1-9}.

\DicoEntry{NINIVE}\textit{, de l'hébreu «~Niyneveh~»~: «~habitation de Ninus~»}\newline
Grande ville située sur la rive est du Tigre. Ses habitants se repentirent de leurs mauvaises voies suite à la prédication de Jonas, mais ils retombèrent dans le péché quelques années plus tard. Ninive fut finalement détruite sous le jugement de Dieu. Voir livres de Jonas et de Nahum.

\DicoEntry{NOCES}\textit{, du grec «~gamos~»~: «~fête du mariage~»}\newline
Festivités célébrant le mariage. Dans la tradition juive, les noces duraient sept jours même si une longue période pouvait parfois s'écouler entre la conclusion du mariage (accord des familles) et la consommation du mariage (nuit de noces). Ainsi, la fiancée devait se tenir prête pour les noces à tout moment. De même, l'Eglise se prépare à être enlevée par Jésus à tout moment pour les noces de l'Agneau qui seront célébrées au ciel pendant sept ans. Voir \vref{Ge. 29:27}~; \vref{Jg. 14:12}~; \vref{1 Th. 4:16-17} et \vref{Ap. 19:7}.

\DicoEntry{NOÉ}\textit{, de l'hébreu «~Noach~»~: «~repos, tranquillité~»}\newline
Fils de Lamech, il fut le père de trois fils~: Sem, Cham et Japhet. Qualifié d'homme juste et intègre en son temps, il trouva grâce devant Yahweh qui lui ordonna de construire une arche* pour le sauver lui, sa famille et une partie des animaux de la terre du déluge qui arrivait. Son obéissance sauva la race humaine. Il vécut 950 ans. Voir \vref{Ge. 6-9}.

\DicoEntry{NOUVELLE NAISSANCE}\textit{}\newline
Voir NAISSANCE D'EN HAUT.

\DicoEntry{OFFRANDE}\textit{, de l'hébreu «~minchah~»~: «~don, tribut, présent, oblation, sacrifice~»}\newline
Sous la loi, le peuple d'Israël avait reçu des prescriptions relatives aux offrandes agréables à Yahweh~; elles consistaient essentiellement en bétail et produits naturels et étaient offertes dans le cadre de cérémonies spécifiques. Des offrandes en argent pouvaient aussi être données, notamment pour soutenir l'entretien du temple. Sous la Nouvelle Alliance, les offrandes monétaires doivent être libres et volontaires~; l'offrande la plus importante aux yeux de Dieu reste la vie consacrée de ses enfants. Voir \vref{Lé. 1-7}~; \vref{Mc. 12:41-42}~; \vref{2 Co. 8:10-12}~; \vref{2 Co. 9:7}~; \vref{Ro. 12:1} et \vref{Ro. 15:15-16}.

\DicoEntry{OLIVIER}\textit{}\newline
Arbre fruitier donnant des olives avec lesquelles on produit de l'huile. Sous la loi de Moïse, elle était notamment utilisée pour alimenter les lampes qui devaient brûler continuellement dans le temple et pour oindre les personnes désignées par Dieu pour une tâche spécifique. L'olivier symbolise en outre le témoignage et la paix. Voir \vref{Ex. 27:20-21}~; \vref{Ex. 30:22-25}~; \vref{Jg. 9:8-9}~; \vref{1 S. 16:3} et \vref{1 R. 19:16}.

\DicoEntry{OMEGA}\textit{}\newline
Dernière lettre de l'alphabet grec désignant aussi la fin d'une chose (voir ALPHA et OMEGA).

\DicoEntry{ONCTION}\textit{, de l'hébreu «~mishchah~»~: «~portion consacrée, huile d'onction, oindre~» et du grec «~chrisma~»~: «~toute chose qui sert à enduire~» de la racine «~chrio~»~: «~oindre, imprégner les chrétiens des dons du Saint-Esprit~»}\newline
Sous l'Ancienne Alliance, l'onction était souvent accordée par l'action de verser de l'huile sur la tête de la personne ou de l'objet à consacrer. On oignait ainsi les prêtres, les rois et les prophètes selon leur mandat. Sous la Nouvelle Alliance, l'onction demeure en celui qui a reçu en lui le Seigneur Jésus. Toutefois, l'onction d'huile peut être pratiquée dans le cadre de la prière pour les malades. Voir \vref{Ex. 30:22-31}~; \vref{1 S. 16:3} et \vref{1 R. 19:16}~; \vref{Ac. 1:8}~; \vref{Ja. 5:14} et \vref{1 Jn. 2:20-27}.

\DicoEntry{ORDINATION}\textit{, du latin «~ordinatio~»~: «~action de disposer, de mettre en œuvre~»}\newline
Rite initiatique mis en place par l'Eglise catholique qu'on ne retrouve pas dans les Ecritures. Elle confère, par l'imposition des mains accompagnée d'une prière, la capacité d'exercer une fonction dirigeante au sein de l'église locale.

\DicoEntry{OTHNIEL}\textit{, de l'hébreu «~`Othniy'el~»~: «~Dieu est puissant~»}\newline
Fils de Kenaz et frère cadet de Caleb, il fut le premier juge en Israël, fonction qu'il exerça pendant 40 ans. Il délivra les enfants d'Israël du joug du roi de Mésopotamie, Cuschan-Rischeathaïm. Voir \vref{Jg. 3:8-11}.

\DicoEntry{OSÉE}\textit{, de l'hébreu «~Howshea`~»~: «~salut, sauve~»}\newline
Fils de Beéri, prophète qui, sous les ordres de Yahweh, épousa une prostituée pour illustrer l'infidélité des enfants d'Israël envers leur Dieu. L'histoire d'Osée et l'ensemble de ses prophéties se trouvent dans le livre portant son nom.

\DicoEntry{PAÏEN}\textit{, du latin «~paganus~» qui signifie «~paysan~», qui provient lui-même du mot «~pagus~» qui signifie «~campagne~».}\newline
Personne qui pratiquait une des religions polythéistes de l'Antiquité.

\DicoEntry{PAIX}\textit{, de l'hébreu «~shalowm~»~: «~état complet, perfection, bien-être, paix~» et du grec «~eirene~»~: «~état de tranquillité, paix entre les individus, harmonie, sécurité~»}\newline
Sous l'Ancienne Alliance, la paix était matérialisée par la prospérité, l'absence de guerre et de toutes sortes de malheurs. Sous la Nouvelle Alliance, la paix est un fruit de l'Esprit*, une promesse acquise en Jésus qui est lui-même le Prince de Paix. Différente de celle que le monde offre, la paix de Christ permet de rester confiant en toutes circonstances. Voir \vref{Lé. 26:6}~; \vref{Es. 26:12}~; \vref{Jn. 14:27}~; \vref{Jn. 16:33} et \vref{Ga. 5:22}.

\DicoEntry{PALMIER}\textit{, de l'hébreu «~tamar~»~: «~palmier, dattier~»}\newline
Arbre à tronc peu ou pas ramifié, on le retrouve essentiellement dans le désert. L'image du palmier fut utilisée en décoration dans le temple. Ses branches étaient utilisées pendant la fête des tentes. Symbole de la justice et de la victoire, on le retrouve lors de l'entrée royale de Jésus à Jérusalem et devant le trône de Dieu. Voir \vref{Lé. 23:40}~; \vref{1 R. 6:29}~; \vref{Jn. 12:12-13} et \vref{Ap. 7:9}.

\DicoEntry{PÂQUE}\textit{, de l'hébreu «~pecach~»~: «~passer outre, épargner~», «~sacrifice de la Pâque~» ou «~fête de la Pâque~»}\newline
Première fête du calendrier hébraïque, elle fut instituée par ordonnance perpétuelle dès la sortie d'Egypte. Cette fête commémore le salut de Yahweh accordé par le sacrifice de l'agneau~; elle préfigurait Christ, l'Agneau de Dieu qui est «~notre Pâque~». Voir \vref{Ex. 12}~; \vref{Lé. 23:5}~; \vref{Jn. 1:29} et \vref{1 Co. 5:7-8}.

\DicoEntry{PARADIS}\textit{, du grec «~paradeisos~»~: «~jardin~»}\newline
Lieu de repos et de félicité, le paradis fut ouvert par Jésus lors de sa résurrection. Il y emmena les justes décédés qui étaient jusque-là captifs dans le séjour des morts*. Les chrétiens rejoignent ce lieu céleste à leur décès, en attendant la résurrection*. A la croix, Christ garantit l'accès à ce lieu au malfaiteur repentant. Paul fut ravi à cet endroit où il entendit des paroles merveilleuses. Voir \vref{Lu. 23:43}~; \vref{2 Co. 12:2-4}~; \vref{Ep. 4:8-10} et \vref{Hé. 10:19-20}.

\DicoEntry{PARDON}\textit{, de l'hébreu «~nas a´~»~: «~action de lever, supporter, prendre~» et du grec «~aphesis~»~: «~libérer de l'esclavage~» ou «~oubli des péchés, rémission des peines~»}\newline
Sous l'Ancienne Alliance, le pardon était conditionné par les sacrifices d'animaux, mais Jésus-Christ a accompli cette prérogative en devenant la victime expiatoire pour nos péchés. En lui, l'homme repentant est pardonné de ses fautes et trouve également la force de pardonner à ceux qui l'offensent. Voir \vref{Lé. 4-6}~; \vref{Mt. 6:12,14-15}~; \vref{Jn. 1:29}~; \vref{Ac. 10:43} et \vref{1 Jn. 1:9}.

\DicoEntry{PARVIS}\textit{, de l'hébreu «~chatser~»~: «~cour, enclos, colonie, ville, village~» (voir illustration du temple)}\newline
Première des trois parties du tabernacle et du temple~; il s'agissait d'une cour dans laquelle se trouvait l'autel d'airain où se faisaient des sacrifices et la cuve d'airain contenant de l'eau pour la purification. Voir \vref{Ex. 27:9-19}.

\DicoEntry{PASTEUR}\textit{, de l'hébreu «~ra`ah~»~: «~berger~»}\newline
Un des cinq services d'\vref{Ep. 4:11} travaillant en collège, établi pour veiller pour le troupeau, le nourrir de la Parole et encourager les chrétiens à exercer pleinement et librement leur ministère. Toutefois, Jésus-Christ demeure le pasteur par excellence, le bon berger qui donne sa vie pour ses brebis et le gardien des âmes qui ne sommeille ni ne dort. Voir \vref{Ep. 4:11}~; \vref{Jn. 10:11-16}~; \vref{Ps. 23} et \vref{1 Pi. 2:25}.

\DicoEntry{PATMOS}\textit{, du grec «~Patmos~»~: «~mortel, fascinant~»}\newline
Petite île grecque de la mer Egée sur laquelle Jean fut exilé à la fin de sa vie. Il y reçut la révélation de l'Apocalypse. Voir \vref{Ap. 1:9}.

\DicoEntry{PAUL}\textit{, du grec «~Paulos~»~: «~petit~»}\newline
Issu de la tribu de Benjamin et né dans la ville de Tarse, son nom était initialement Saul*. Pharisien, son zèle excessif le poussa à persécuter violemment les chrétiens à la naissance de l'Eglise. Il rencontra Christ sur la route de Damas et devint par la suite l'apôtre des Gentils annonçant l'Evangile de villes en villes et de pays en pays au cours de nombreux voyages. Même en prison, il continua l'œuvre de Dieu en écrivant plusieurs lettres riches en enseignements que l'on peut retrouver dans le canon biblique. Voir \vref{Ac. 9-28} et les épîtres de Paul.

\DicoEntry{PÉAGER ou PUBLICAIN}\textit{, du grec «~telones~»~: «~un loueur, un collecteur de taxes~»}\newline
Les péagers d'origine juive étaient dépréciés de leurs compatriotes et assimilés à des pécheurs, car on les considérait comme des collaborateurs au service des romains. De plus, certains profitaient de leur fonction pour s'enrichir. Voir \vref{Mt. 9:10}~; \vref{Mt. 21:31}~; \vref{Lu. 3:12-13} et \vref{Lu. 19:2-8}.

\DicoEntry{PÉCHÉ}\textit{, de l'hébreu «~chatta'ah~»~: «~ce qui manque le but~» et du grec «~hamartano~»~: «~erreur, faux état d'esprit~»}\newline
Le péché entra dans le monde par la transgression d'Adam et Eve et tous les hommes en furent infectés. Origine de la séparation entre Dieu et les hommes, le péché conduit à la mort*. Voir \vref{Ge. 3}~; \vref{1 Co. 15:3}~; \vref{Ro. 5:12}~; \vref{Ro. 6:23}~; \vref{Ro. 8:1-4} et \vref{1 Pi. 2:21-24}.

\DicoEntry{PENTECÔTE}\textit{, du grec «~pentekoste~»~: «~le cinquantième jour~»}\newline
Fête annuelle juive célébrant la moisson des blés. La venue du Saint-Esprit* promis par Jésus eut lieu pendant la célébration de la Pentecôte. Voir \vref{Lé. 23:15-22}~; \vref{Jn. 16:7-11}~; \vref{Ac. 1:5} et \vref{Ac. 2:1-21}.

\DicoEntry{PHARAON}\textit{, de l'hébreu «~Par`oh~»~: «~grand palais~»}\newline
Titre donné aux rois égyptiens durant l'Antiquité. Voir \vref{Ge. 37:36} et \vref{Ge. 41}.

\DicoEntry{PHARISIEN}\textit{, du grec «~Pharisaios~»~: «~séparé~»}\newline
Secte juive dont les membres manifestaient un attachement excessif aux coutumes et traditions religieuses. Certains d'entre eux combattirent Jésus qui dénonça ouvertement leur fausse piété et leur dévouement hypocrite envers Dieu. Désirant la mort du Seigneur, ils participèrent à la conspiration qui précéda sa crucifixion. Voir \vref{Mt. 23:23-39}~; \vref{Mc. 7:1-13} et \vref{Jn. 18:2-3}.

\DicoEntry{PHILADELPHIE}\textit{, du grec «~Philadelpheia~»~: «~amour fraternel~»}\newline
Ville de Lydie en Asie Mineure. Irriguée par le fleuve Hermus, Philadelphie était une contrée très fertile, propice à l'agriculture et surtout à la culture de la vigne. Elle fut construite par le roi de Pergame, et plusieurs fois sujette à des tremblements de terre. Une des sept lettres d'Apocalypse s'adressait à l'église de Philadelphie. Cette dernière - contrairement aux autres qui cumulèrent des reproches – fut très encouragée par le Seigneur. Bien que située à 45 km de Sardes à laquelle elle était rattachée, l'Eglise de Philadelphie resta ferme en retenant la Parole de Dieu et ne se laissa pas influencer par les séductions du péché. Elle incarne ainsi l'Eglise que Jésus revient chercher, l'Eglise réveillée.

\DicoEntry{PHILÉMON}\textit{, du grec «~Philemon~»~: «~attentionné, qui embrasse~»}\newline
Disciple de Colosses qui recevait une église dans sa maison. Il avait un esclave nommé Onésime au sujet duquel Paul lui écrivit une lettre. Voir épître de Paul à Philémon.

\DicoEntry{PHILIPPE}\textit{, du grec «~Philippos~»~: «~aimant les chevaux~»}\newline
1. Homme de Bethsaïda, il fut l'un des douze apôtres* choisis par Jésus. Voir \vref{Mt. 10:3}~; \vref{Mc. 3:18} et \vref{Lu. 6:14}.
\\2. Un des sept diacres élus au sein de l'église de Jérusalem. Evangéliste*, il prêcha le Christ dans la ville de Samarie, à l'eunuque éthiopien qu'il baptisa et dans différentes villes. Voir \vref{Ac. 6:5}~; \vref{Ac. 8:4-8,26-40} et \vref{Ac. 21:8}.

\DicoEntry{PHILIPPES}\textit{, du grec «~Philippoi~»~: «~appartenant à Philippe~»}\newline
Fondée par Philippe II (382 av. J.-C. – 336 av. J.-C.) en 356 av. J.-C., ville grecque de Macédoine orientale. Située sur une voie romaine qui traversait les Balkans, elle est restée de taille modeste en dépit de son fort taux de fréquentation. Une église y naquit après la rencontre de Paul avec des femmes qui priaient à l'extérieur de la ville. L'apôtre leur écrivit une lettre qui figure dans le canon biblique. Voir \vref{Ac. 16:9} et l'épître aux Philippiens.

\DicoEntry{PHILISTINS}\textit{, de l'hébreu~: «~Pelesheth~»~: «~immigrants~»}\newline
Peuple qui habitait à l'extrême ouest de Canaan, le long de la mer Méditerranée. Ils furent plusieurs fois en conflit avec les Israélites~; Goliath était philistin. Voir \vref{Jg. 13-16} et \vref{1 S. 17}.

\DicoEntry{PHILOSOPHIE}\textit{, du grec «~philosophia~»~: «~amour de la sagesse~»}\newline
Discipline existant depuis l'Antiquité et ayant plusieurs courants de pensée en son sein comme les épicuriens* et les stoïciens*. Elle pousse ses adeptes à rechercher la sagesse par l'intelligence humaine. Paul invita les chrétiens à se garder de ces doctrines. Voir \vref{Ac. 17:16-20} et \vref{Col. 2:8}.

\DicoEntry{PHINÉES}\textit{, de l'hébreu «~Piynechac~»~: «~bouche de cuivre~»}\newline
Fils d'Eléazar, petit-fils d'Aaron et prêtre. Il se démarqua par son zèle pour Dieu pour arrêter un fléau sur Israël. A cette occasion, Yahweh fit alliance perpétuelle avec Phinées et sa descendance. Voir \vref{No. 25}.

\DicoEntry{PIERRE}\textit{, de l'hébreu «~Cephas~» et du grec «~Petros~»~: «~un roc ou une pierre~»}\newline
Fils de Jonas et frère d'André*, son nom était initialement Simon*. Pêcheur de métier originaire de la ville de Bethsaïda, il fut choisi comme apôtre pour les circoncis. Il écrivit deux épîtres portant son nom. Il aurait été crucifié à Rome. Voir \vref{Mt. 10:2}~; \vref{Jn. 1:42-44}~; \vref{Ga. 2:7-8}~; 1 Pi. et 2 Pi.

\DicoEntry{PILATE}\textit{, du grec «~Pilatos~»~: «~armé d'une lance~»}\newline
Gouverneur romain de la Judée en fonction pendant le ministère de Jésus. Il s'accorda avec son ennemi Hérode lorsqu'il fallut crucifier le Seigneur. N'ayant pas trouvé de crime en Jésus, il permit finalement sa crucifixion et fit mettre l'inscription suivante sur sa croix~: Jésus de Nazareth, roi des Juifs. Voir \vref{Lu. 3:1}~; \vref{Lu. 23:11} et \vref{Jn. 19:1-19}.

\DicoEntry{PRÉDESTINATION}\textit{, du grec «~proginosko~»~: «~avoir la connaissance avant~»}\newline
Révélant l'omniscience de Dieu qui connaît toutes choses à l'avance, la prédestination concerne l'œuvre de la croix prévue de toute éternité – l'Agneau ayant été immolé avant la fondation du monde. La prédestination est non pas la décision de Dieu d'envoyer certaines personnes en enfer, mais plutôt la capacité de Yahweh à connaître à l'avance ceux qui allaient devenir ses enfants d'adoption, transformés à l'image du Fils, en acceptant sa parole. Voir \vref{Jn. 1:12}~; \vref{Ro. 8:29-30}~; \vref{Ep. 1:5} et \vref{1 Pi. 1:19-20}.

\DicoEntry{PREMIER PRÊTRE}\textit{, de l'hébreu «~rosh~»~: «~tête, dessus, sommet, partie supérieure, chef, principal, premier, total, somme, hauteur, front, le devant, commencement~» et de «~kohen~»~: «~prêtre, intendant principal, ministre d'état~»}\newline
Sous la loi, le premier prêtre descendait d'Aaron. Il servait le Seigneur dans le sanctuaire* et enseignait la loi*. Tel un médiateur* entre Yahwhe et le peuple, il portait constamment le jugement de ce dernier pour qui il consultait Dieu au moyen de l'urim et du thummim. Il devait, une fois par an, entrer dans le Saint des saints et offrir des sacrifices d'animaux pour ses propres péchés et pour ceux du peuple. Par la suite, Jésus-Christ est devenu premier prêtre à perpétuité en s'offrant comme victime expiatoire et en présentant son sang une fois pour toutes dans le Saint des saints du temple céleste. Voir \vref{Ex. 28:30}~; \vref{Ex. 29:9}~; \vref{Esd. 2:63}~; \vref{No. 35:25}~; \vref{Hé. 4:14-16}~; \vref{Hé. 7:25-28} et \vref{Hé. 9:6-12,24-28}.

\DicoEntry{PRÉTOIRE}\textit{, du grec «~praitorion~»~: «~quartier général dans un camp romain, la tente du commandant en chef~»}\newline
Dans les évangiles, lieu de résidence des gouverneurs dans lequel se trouvaient notamment un tribunal et une prison. Voir \vref{Mt. 27:27}~; \vref{Jn. 18:28-29} et \vref{Ac. 23:35}.

\DicoEntry{PRIÈRE}\textit{, de l'hébreu «~palal~»~: «~intervenir, s'interposer, prier~» ou «~'athar~»~: «~prier, supplier, implorer~» et du grec «~proseuche~»~: «~prière adressée à Dieu~» ou «~parakaleo~»~: «~appeler à, convoquer, supplier, exhorter~»}\newline
Acte par lequel on s'approche de Dieu et on instaure un dialogue avec lui, en ayant foi dans sa présence et son action. Invité à prier constamment, le chrétien peut le faire pour se repentir, intercéder en faveur d'une situation particulière, demander quelque chose à Dieu, le remercier, le louer ou tout simplement lui exprimer son amour. La prière garde les enfants de Dieu dans la paix. Dieu connaissant toutes les pensées de l'homme, le plus important dans la prière reste l'écoute de la voix de Yahweh. Voir \vref{Ge. 20:17}~; \vref{1 S. 2:1}~; \vref{Job 22:27}~; \vref{Mt. 14:36}~; \vref{Ac. 16:9}~; \vref{1 Th. 5:17}~; \vref{Ph. 4:6-7} et \vref{1 Pi. 4:7}.

\DicoEntry{PROPHÈTE}\textit{, de l'hébreu «~nabiy'~»~: «~l'homme qui parle, celui qui est appelé, qui a reçu une inspiration~» et du grec «~prophetes~»~: «~celui qui interprète des oracles~», «~quelqu'un qui déclare ce qu'il a reçu par inspiration~»}\newline
Sous l'Ancienne Alliance, Dieu suscita de nombreux prophètes oints de l'Esprit afin qu'ils annoncent des messages particuliers et conduisent le peuple à l'obéissance et à la crainte de Yahweh. Sous la Nouvelle Alliance, il existe au moins trois types de prophètes. Le premier concerne ceux et celles qui prophétisent au sein des assemblées locales (\vref{Ac. 21:8-9}~; \vref{1 Co. 14:29-32}), ils exhortent, édifient et consolent le peuple (\vref{1 Co. 14:1-3}). Le deuxième concerne les personnes qui ont reçu la charge d'enseigner, poser les fondements, implanter des assemblées selon \vref{Ep. 4:11}. Parmi ces prophètes, on compte Barnabas, Siméon, Lucius de Cyrène, Manahen, Saul (\vref{Ac. 13:1-5}), Jude et Silas (\vref{Ac. 15:32-33}). Le troisième concerne tous les chrétiens qui sont des potentiels prophètes puisqu'ils ont l'Esprit de Christ en eux (\vref{1 Co. 14:23-25}~; \vref{1 Co. 14:31}). Dieu peut se servir n'importe quel chrétien pour prophétiser, c'est-à-dire communiquer une parole inspirée. Voir \vref{Ep. 2:20} et \vref{Ep. 4:11}.

\DicoEntry{PROPHÉTIE}\textit{, du grec «~propheteia~»~: «~discours émanant de l'inspiration divine et déclarant les desseins de Dieu~»}\newline
Depuis l'effusion du Saint-Esprit, tous les chrétiens nés d'en haut peuvent prophétiser sans pour autant avoir le ministère de prophète. La prophétie est en effet un don spirituel* auquel il faut aspirer et qui est attribué par le Saint-Esprit selon la volonté de Dieu. Voir \vref{Ac. 2:16-18}~; \vref{1 Co. 12:4-10} et \vref{1 Co. 14:1}.

\DicoEntry{PROPITIATOIRE}\textit{, de l'hébreu «~kapporeth~»~: «~siège de miséricorde, lieu d'expiation~»}\newline
Couvercle de l'arche* composé d'or pur, il était surmonté de deux chérubins* d'or se faisant face au milieu desquels Yahweh siégeait et se manifestait pour donner des instructions à Israël. Une fois par an, le grand prêtre entrait dans le Saint des saints et aspergeait le propitiatoire du sang des animaux sacrifiés pour la purification des péchés d'Israël. Voir \vref{Ex. 25:17-22} et \vref{Lé. 16}.

\DicoEntry{PROSÉLYTE}\textit{, du grec «~proselutos~»~: «~un nouveau venu, un étranger~»}\newline
Personne issue d'une nation païenne s'étant agrégée au peuple d'Israël par le rite de la circoncision* et la pratique de la loi mosaïque. Voir \vref{Mt. 23:15}~; \vref{Ac. 2:10}~; \vref{Ac. 6:5} et \vref{Ac. 13:43}.

\DicoEntry{PYTHON}\textit{, du grec «~Puthon~»~: «~un serpent ou un dragon}\newline
Esprit de divination auquel Paul fut confronté en Macédoine. Voir \vref{Ac. 16:16-18}.

\DicoEntry{RABBI}\textit{, de l'hébreu «~rab~»~: «~capitaine, chef~» et du grec «~rhabbi~»~: «~maître~», «~un grand monsieur, honorable~» ou «~un enseignant~»}\newline
Les disciples appelaient Jésus «~Rabbi~». Cependant, il a exhorté la foule et les conducteurs religieux à ne pas attribuer une telle marque de distinction aux hommes rappelant que seul Yahweh est maître. Voir \vref{Mt. 23:8}~; \vref{Mc. 11:21}~; \vref{Jn. 9:2}.

\DicoEntry{RACHEL}\textit{, de l'hébreu «~Rachel~»~: «~agnelle, brebis~»}\newline
Fille de Laban, deuxième femme de Jacob pour laquelle il travailla quatorze ans. Longtemps stérile, Yahweh lui donna finalement deux garçons~: Joseph et Benjamin. Elle mourut à l'accouchement du deuxième. Voir \vref{Ge. 29:10-31}~; \vref{Ge. 30:22-24} et \vref{Ge. 35:16-19}.

\DicoEntry{RAHAB}\textit{, de l'hébreu «~Rachab~»~: «~large, spacieux, tumultueux~»}\newline
Prostituée habitant Jéricho, elle cacha les deux espions juifs chez elle. Grâce à son acte, Josué* lui laissa la vie sauve ainsi qu'à sa famille lorsqu'il détruisit la ville et tous ceux qui s'y trouvaient. Rahab habita ensuite au milieu d'Israël~; elle figure non seulement parmi les héros de la foi, mais aussi dans la généalogie de Jésus-Christ. Voir \vref{Jos. 2:1}~; \vref{Jos. 6:17-25}~; \vref{Mt. 1:5-16} et \vref{Hé. 11:31}.

\DicoEntry{REBECCA}\textit{, de l'hébreu «~Ribqah~»~: «~ensorcelante, qui prend au piège~»}\newline
Fille de Bethuel et sœur de Laban, elle fut l'épouse d'Isaac*. Yahweh mit fin à sa stérilité et elle donna naissance à des jumeaux, Esaü et Jacob, qui devinrent deux grandes nations. Voir \vref{Ge. 24} et \vref{Ge. 25:21-26}.

\DicoEntry{RÉCONCILIATION}\textit{, du grec «~katallage~»~: «~échange, change, ajustement d'une différence~»}\newline
Jésus-Christ mourut à la croix pour réconcilier l'homme avec Dieu, c'est-à-dire le faire passer de l'état de séparation (causée par le péché) à l'état d'intimité avec Dieu. L'Eglise a le ministère de réconciliation et doit en ce sens présenter à l'homme pécheur la voie de la réconciliation avec Dieu au travers de la prédication de l'Evangile*. Voir \vref{Ro. 5:11}~; \vref{Hé. 10:18-20} et \vref{2 Co. 5:18-20}.

\DicoEntry{RÉDEMPTION}\textit{, de l'hébreu~: «~peduwth~»~: «~rachat~» et du grec: «~apolutrosis~»~: «~libération effectuée suite au paiement d'une rançon~»}\newline
Jésus-Christ a payé le prix nécessaire au rachat des péchés de tous les hommes par son sacrifice à la croix, leur permettant d'échapper à la mort éternelle au moyen de la foi*. Voir \vref{Ro. 3:23-24}~; \vref{Col. 1:14}~; \vref{Ep. 1:7} et \vref{Hé. 9:12}.

\DicoEntry{RÉFORME}\textit{, de l'hébreu «~yatab~»~: «~agir bien~» et du grec «~diorthosis~»~: «~remettre droit~»}\newline
La plupart des prophètes du Tanakh sont des réformateurs dans la mesure où ils prônent un retour à Dieu~; le roi Josias a institué une profonde réforme pendant son règne en déployant des efforts pour revenir à l'obéissance de la Parole. Jésus-Christ est le plus grand réformateur en ce qu'il marchait à contre-courant et vint restaurer l'homme à sa condition originelle, celle d'avant la chute. Ainsi, l'homme qui reçoit Jésus entre dans un processus où il est continuellement réformé par le Saint-Esprit au travers de la Parole. Voir \vref{2 R. 22}~; \vref{Jé. 7:5}~; \vref{Jé. 26:13}~; \vref{Os. 6:1}~; \vref{Mt. 19:8} et \vref{Jn. 16:7-15}.

\DicoEntry{REPENTANCE}\textit{, du grec «~metanoia~»~: «~changement de mentalité, d'intention~», «~tristesse qu'on éprouve de ses péchés~»}\newline
Un des points majeurs de la prédication de Jean-Baptiste* puis des apôtres*. La repentance est essentielle pour obtenir la rémission des péchés et doit être accompagnée de fruits. La repentance ne concerne pas uniquement le nouveau converti, mais tout disciple de Christ qui, jusqu'à la fin de sa vie, est dans un processus de perfectionnement. Voir \vref{Mc. 1:4}~; \vref{Lu. 3:8}~; \vref{2 Co. 7:9-10}~; \vref{Ro. 2:4}~; \vref{Ac. 2:38}~; \vref{Ac. 13:24}~; \vref{Ac. 17:30} et \vref{Ac. 26:20}.

\DicoEntry{RÉSURRECTION}\textit{, du grec «~anastasis~»~: «~se lever, ressusciter de la mort~».}\newline
Christ fut le premier à expérimenter la résurrection d'entre les morts. Au son de la dernière trompette*, les chrétiens décédés ressusciteront de même avec des corps incorruptibles pour les noces* de l'Agneau. Voir \vref{Mt. 28:6}~; \vref{1 Pi. 1:3}~; \vref{Ap. 1:5}~; \vref{1 Co. 15:52} et \vref{1 Th. 4:16}.

\DicoEntry{RÉVEIL}\textit{, du grec «~egeiro~»~: «~réveiller du sommeil, revenir à la vie, se lever~»}\newline
Prise de conscience personnelle ou collective sur sa condition de péché et.ou l'imminence du jugement de Dieu. Il en résulte la repentance*, la véritable conversion*, la crainte de Dieu, la préparation à la rencontre de Yahweh. Une personne réveillée a les yeux focalisés sur Christ et peut accomplir la volonté du Seigneur. Voir Jon.~; \vref{Ep. 5:14} et \vref{Ro. 13:11-14}.

\DicoEntry{ROBOAM}\textit{, de l'hébreu «~Rhoboam~»~: «~qui affranchit le peuple~»}\newline
Fils et successeur du roi Salomon. C'est sous son règne que se produit le schisme* entre les royaumes du nord et celui du sud. Il régna sur Juda dix-sept années pendant lesquelles il fut en guerre avec le royaume du nord et fit ce qui est mal aux yeux de Yahweh. Voir \vref{1 R. 12:1-24} et \vref{1 R. 14:21-31}.

\DicoEntry{ROMAIN}\textit{, du grec «~rhome~»~: «~force~»}\newline
Pendant la vie de Jésus et pendant l'époque de l'Eglise primitive, Israël était sous la domination de l'Empire romain qui l'oppressait et lui soutirait des impôts. Paul, né à Tarse - ville romaine - put bénéficier des privilèges liés à la nationalité romaine quand il fut livré aux tribunaux. Ce dernier écrivit une lettre aux chrétiens romains - figurant dans le canon biblique - avant de les rencontrer physiquement. Voir \vref{Mt. 22:17}~; \vref{Jn. 11:48}~; \vref{Ac. 16:35-39}~; \vref{Ac. 22:25-29}~; \vref{Ac. 23:27} et \vref{Ac. 25:16}.

\DicoEntry{ROME}\textit{, du grec «~rhome~»~: «~force~»}\newline
Capitale de l'Empire romain située en Italie, Rome jouissait d'une grande notoriété à l'époque de l'Eglise primitive. Bien que l'empereur Claude* ait ordonné aux Juifs de quitter la ville, Paul manifesta le désir de s'y rendre pour y annoncer l'Evangile. Il y arriva après bien des difficultés quelques années plus tard en tant que prisonnier. Voir \vref{Ac. 18:1-2}~; \vref{Ac. 19:21}~; \vref{Ac. 23:11} et \vref{Ac. 28:14-31}.

\DicoEntry{ROYAUME DE DIEU}\textit{, du grec «~basileia~»~: «~pouvoir royal, royauté, domination, autorité~»}\newline
Lors de son service terrestre, Jésus a annoncé que le Royaume de Dieu était proche. Il parlait de son autorité sur toutes choses et de son règne. Ne consistant pas dans les choses terrestres, ce royaume se manifeste par la puissance de Dieu, la justice, la paix et la joie par le Saint-Esprit. Voir \vref{Lu. 9:1-2}~; \vref{Lu. 11:17-20}~; \vref{Lu. 17:20-21} et \vref{Ro. 14:17}.

\DicoEntry{RUBEN}\textit{, de l'hébreu «~Re'uwben~»~: «~voici un fils~»}\newline
Premier fils de Jacob et Léa, il devint le père de la tribu des Rubénites qui s'installa à l'est de la terre promise. Il perdit son droit d'aînesse après avoir eu des rapports intimes avec Bilha, concubine de son père. Voir \vref{Ge. 29:32}~; \vref{Ge. 35:22} et \vref{Ge. 49:3-4}.

\DicoEntry{RUTH}\textit{, de l'hébreu «~Ruwth~»~: «~amitié, une amie~»}\newline
Originaire de Moab, belle-fille de Naomi avec qui elle s'installa à Bethléhem. Elle y épousa Boaz avec qui elle eut un fils, Obed, grand-père du roi David*. Son histoire est racontée dans le livre portant son nom.

\DicoEntry{SABBAT}\textit{, de l'hébreu «~shabbath~»~: «~repos, cessation d'activité~»}\newline
Septième et dernier jour de la semaine consacré à Yahweh pendant lequel aucune activité ne devait être pratiquée selon la loi. Le sabbat figure dans les dix commandements, son infraction devait être punie de mort. Suscitant de vives critiques de la part des religieux, Jésus a plusieurs fois enfreint le sabbat dont il s'est déclaré le maître. Sous la Nouvelle Alliance, le sabbat se trouve en Jésus-Christ, le chrétien n'est donc pas tenu de le respecter comme ce fut le cas sous la loi de Moïse. \vref{Ex. 20:8-11}~; \vref{Ex. 31:14-15}~; \vref{De. 5:12-15}~; \vref{Mt. 11:28-30}~; \vref{Mc. 2:23-28} et \vref{Mc. 3:1-6}.

\DicoEntry{SACERDOTALISME}\textit{}\newline
Doctrine d'origine catholique reconnaissant le prêtre ou le pasteur comme l'intermédiaire entre Dieu et les hommes. Voir \vref{1 Ti. 2:5}.

\DicoEntry{SACRIFICATURE, SACERDOCE}\textit{, de l'hébreu «~kahan~»~: «~service~»}\newline
Sous la loi mosaïque, il était exercé par les Lévites descendants d'Aaron dans le tabernacle puis le temple et consistait notamment à accomplir les différents rituels relatifs aux sacrifices d'animaux et aux offrandes de toutes sortes. Depuis le sacrifice de Jésus à la croix, le sacerdoce concerne tous les enfants de Dieu qui sont non seulement les prêtres, mais aussi les sacrifices auxquels le Seigneur prend plaisir. Voir Lé.~; \vref{Ro. 12:1}~; \vref{1 Pi. 2:9} et \vref{Ap. 1:6}.

\DicoEntry{SADDUCÉENS}\textit{, du grec «~saddoukaios~»~: «~les justes~»}\newline
Parti religieux juif attaché au Pentateuque de manière stricte, ils ne croyaient ni en la résurrection des morts ni aux anges. Ils s'opposèrent au service de Jésus qui les reprit sévèrement et échappa à leurs pièges. Ils combattirent de même les apôtres qu'ils jetèrent en prison. Voir \vref{Mt. 16:6-12}~; \vref{Mt. 22:23-33}~; \vref{Ac. 5:17-19} et \vref{Ac. 23:1-10}.

\DicoEntry{SAINT}\textit{, de l'hébreu «~qodesh~»~: «~consacré, mis à part~» et du grec «~hagios~»~: «~chose très sainte, consacré, un saint~»}\newline
Dieu appela Israël à la sainteté, c'est-à-dire à ne pas se mélanger avec les autres peuples de peur d'être contaminés par leurs pratiques méchantes et idolâtres. Yahweh est le Saint d'Israël. Sous la Nouvelle Alliance, les chrétiens sont appelés saints, car le Saint-Esprit qui est en eux leur communique sa nature, les purifie et leur enseigne la haine du péché. Voir \vref{De. 7:6}~; \vref{Es. 49:7}~; \vref{1 Co. 6:11,19}~; \vref{1 Th. 4:1-8} et \vref{Hé. 12:14}.

\DicoEntry{SAINT-ESPRIT}\textit{, (voir étymologie des mots «~saint~» et «~esprit~»)}\newline
Le Saint-Esprit est l'esprit de Dieu, l'Esprit de Jésus~; il est Dieu. Lors de son service terrestre, le Seigneur déclara qu'un consolateur viendrait habiter dans les corps des croyants. Cette parole s'accomplit lors de la Pentecôte. Le Saint-Esprit a pour mission de convaincre le monde en ce qui concerne le péché, la justice et le jugement. A la naissance d'en haut, il régénère l'esprit du chrétien sur qui il dépose son sceau, gage de l'adoption. Il enseigne et guide le chrétien tout au long de sa marche avec Dieu. Il transforme son caractère et distribue les dons spirituels pour l'édification de l'Eglise. Voir \vref{1 S. 10:10}~; \vref{2 Ch. 15:1}~; \vref{Jn. 14:16-17,26}~; \vref{Jn. 16:7-15}~; \vref{Ac. 2}~; \vref{1 Co. 6:11}~; \vref{Ro. 8:9}~; \vref{1 Co. 3:16}~; \vref{1 Co. 12:4-13}~; \vref{Ep. 1:13} et \vref{Ga. 5:16,22}.

\DicoEntry{SALOMON}\textit{, de l'hébreu «~Shelomoh~»~: «~paix, pacifique~»}\newline
Fils de David, il succéda à son père et fut roi d'Israël pendant quarante ans. Il construisit le premier temple de Yahweh sur le mont Morija à Jérusalem puis un palais royal. Outre ses importantes richesses, c'est la grande sagesse que Dieu lui donna qui fit sa renommée parmi tous les peuples. Il eut sept cents femmes et trois cent concubines - dont un grand nombre de femmes étrangères - ce qui détourna son cœur de son Dieu. On lui attribue la rédaction des livres Cantique des cantiques et Ecclésiaste~; il a aussi écrit certains psaumes et plusieurs proverbes. Voir \vref{1 R. 4:29-34}~; \vref{1 R. 5-7}~; \vref{1 R. 9:15-28}~; \vref{1 R. 11:1-10,42}~; \vref{Ps. 72,127} et \vref{Pr. 25-29}.

\DicoEntry{SALUT}\textit{, de l'hébreu «~yesha'~» et du grec «~soteria~»~: «~délivrance, sûreté, sécurité~»}\newline
Libération des chaînes du péché, de la condamnation et de tout type d'asservissement spirituel, le salut est un don gratuit de Dieu qui s'obtient par la grâce, au moyen de la foi. C'est la manifestation de l'amour éternel de Dieu qui - ne voulant pas que l'homme périsse dans le feu de la géhenne - a payé le prix pour lui offrir la vie éternelle. Le salut réside dans le seul nom de Jésus-Christ. Voir \vref{Jn. 3:16}~; \vref{Ac. 4:12}~; \vref{Ro. 8:1}~; \vref{1 Th. 5:9}~; \vref{Tit. 3:4-6} et \vref{Ep. 2:4-8}.

\DicoEntry{SAMARIE}\textit{, de l'hébreu «~Shomerown~» et du grec «~Samareia~»~: «~montagne de guet~»}\newline
Située dans l'actuelle Cisjordanie, ville fondée par Omri, roi d'Israël, et qui devint la capitale du royaume du nord. La ville fut prise par Salmanasar, roi d'Assyrie, sous le règne d'Osée, roi d'Israël. Au temps de Jésus, la Samarie n'était qu'une simple circonscription romaine dont la population était issue du métissage entre Israélites et des colons assyriens. Suite aux persécutions subies par l'Eglise primitive à Jérusalem, des chrétiens s'y réfugièrent et l'Evangile s'y propagea. Voir \vref{1 R. 16:23-24}~; \vref{2 R. 3:1}~; \vref{2 R. 18:9}~; \vref{Os. 7} et \vref{Ac. 8:1-17}.

\DicoEntry{SAMARITAINS}\textit{, du grec «~samareites~»~: «~un habitant de Samarie~»}\newline
Après l'assujettissement de la Samarie par Salmanasar, roi d'Assyrie, des peuples étrangers s'y établirent et s'assemblèrent avec les Israélites. Au IVe siècle av J.-C., les samaritains construisirent un temple sur le mont Garizim, qui devint le centre religieux de Samarie, entraînant une séparation avec le reste des Juifs qui adoraient à Jérusalem. Les samaritains étaient considérés comme des étrangers et non comme de véritables juifs du fait de la mixité de leur religion. Jésus-Christ ouvrit la voie de la réconciliation avec ce peuple en racontant la parabole du bon samaritain et en annonçant la bonne nouvelle à la femme samaritaine. Voir \vref{2 R. 17:3,24-29} et \vref{2 R. 18:9}~; \vref{Jn. 4:4-26} et \vref{Lu. 10:30-37}.

\DicoEntry{SAMSON}\textit{, de l'hébreu «~Shimshown~»~: «~petit soleil~»}\newline
Fils de Manoach, de la tribu de Dan, il fut juge en Israël pendant vingt-ans. Consacré à Dieu dès le sein maternel et doté d'une force extraordinaire, il réalisa des prouesses qui suscitèrent la crainte de ses ennemis. Choisi pour être le libérateur d'Israël, il fut incompris par les siens qui ne le soutinrent pas. Il mourut suite à la trahison de Delila, une femme d'origine philistine. Voir \vref{Jg. 13-16}.

\DicoEntry{SAMUEL}\textit{, de l'hébreu «~Shemuw'el~»~: «~entendu ou exaucé de Dieu~»}\newline
Fils d'Elkana, de la tribu d'Ephraïm, et d'Anne, il fut consacré au service de Yahweh dès son plus jeune âge. Il exerça les fonctions de juge, prêtre et prophète sur Israël. Il oignit les deux premiers rois d'Israël~: Saül et David. Son histoire est racontée dans les deux livres du Tanakh portant son nom.

\DicoEntry{SANCTIFICATION}\textit{, de l'hébreu «~Qadash~» et du grec «~hagiasmos~»~: «~consécration, purification, sainteté~» ou «~l'effet de la purification~»}\newline
Fruit de l'action conjointe de la Parole et l'Esprit de Dieu dans la vie du croyant, la sanctification doit être recherchée par le chrétien tout au long de sa vie. Sans elle, nul ne verra Dieu. \vref{Jn. 17:17}~; \vref{1 Th. 4:3-8}~; \vref{Hé. 12:14} et \vref{Ap. 22:11}.

\DicoEntry{SANCTUAIRE}\textit{, de l'hébreu «~miqdash~»~: «~lieu sacré, lieu saint, sanctuaire de Yahweh~»}\newline
Le sanctuaire terrestre, dont Moïse avait reçu le modèle, était une représentation de celui qui se trouve au ciel et où Jésus alla présenter son sang. Voir \vref{Ex. 25:8-9} et \vref{Hé. 9:1-24}.

\DicoEntry{SANG}\textit{, de l'hébreu «~dam~» et du grec «~haima~»~: «~sang~»}\newline
Déterminant le lien de famille et la lignée, le sang est, selon les Ecritures, l'âme*, la vie. Ainsi, l'effusion de sang fut nécessaire pour le pardon des péchés et le sang de Christ, qui ôte définitivement le péché, donne la vie. Voir \vref{Lé. 17:11}~; \vref{Ac. 17:26}~; \vref{Ro. 5:9}~; \vref{Hé. 9:22-28} et \vref{Ap. 5:9}.

\DicoEntry{SANHÉDRIN}\textit{, du grec «~sunedrion~»~: «~conseil, tribunal~»}\newline
Désignait d'une part, les petits tribunaux se tenant dans chaque ville pour régler les affaires locales et d'autre part, le grand conseil de Jérusalem où étaient traitées les affaires plus importantes. Ce dernier était composé de soixante et onze membres sélectionnés parmi l'élite religieuse et les anciens d'Israël~; le grand prêtre en était le président. Sous la domination romaine, ce tribunal fonctionnait de manière quasi autonome~; la sentence de la peine de mort devait néanmoins être validée par le gouverneur romain. Jésus fut jugé coupable de blasphème par le sanhédrin qui le condamna à mort. Voir \vref{No. 11:16-17}~; \vref{Mt. 5:22}~; \vref{Mt. 26:59-66}~; \vref{Jn. 11:47}~; \vref{Ac. 5:21-41} et \vref{Ac. 6:12-15}.

\DicoEntry{SARA}\textit{, de l'hébreu «~Sarah~»~: «~princesse, femme noble~»}\newline
Femme d'Abraham, elle enfanta Isaac à l'âge de 90 ans selon la promesse de Yahweh. Sara figure parmi les héros de la foi~; elle mourut à cent vingt-sept ans. Voir \vref{Ge. 12:5}~; 17~; \vref{Ge. 15-16}~; \vref{Ge. 21:1-7}~; \vref{Ge. 23:1} et \vref{Hé. 11:11}.

\DicoEntry{SARDES}\textit{, du grec «~sardeis~»~: «~les rouges~», «~prince de joie~»}\newline
Capitale antique de la Lydie, Sardes se situait sur la rivière Pactole, à environ 50 km au sud de Thyatire et 75 km à l'est de Smyrne. Réputée riche et puissante en raison de ses ressources en or, ses épithètes étaient sournois car sa forteresse reposait sur un sol boueux. En effet, au VIème siècle av. J.-C., Cyrus Le Grand - vainqueur de Crésus alors roi de Lydie - s'empara de Sardes par une attaque nocturne. Par la suite, la ville subit plusieurs invasions puis un tremblement de terre en 17 ap. J.-C. L'église de Sardes fut probablement fondée par Paul au cours d'un voyage à Ephèse. Au moment où ils reçurent le message de l'ange de l'Apocalypse, il semblerait que certains chrétiens de Sardes étaient retournés au culte licencieux de Cybèle, déesse-mère et gardienne des savoirs. Ceux qui s'étaient gardés purs devaient ainsi revivifier les autres membres. Cette église symbolise l'église morte. Voir \vref{Ap. 3:1-6}.

\DicoEntry{SATAN}\textit{, de l'hébreu «~Satan~»~: «~adversaire, ennemi~»}\newline
Autrefois chérubin protecteur, il a péché en voulant s'approprier la gloire qui ne revient qu'à Dieu. Dans sa rébellion, il entraîna un tiers des anges qui furent précipités avec lui sur la terre. Connu également sous les noms «~Prince de ce monde~», «~Prince des ténèbres~», «~Belzébul~», «~le malin~», «~l'accusateur~» ou «~le diable~», il est l'adversaire des enfants de Dieu à qui il fait la guerre. Il a cependant été vaincu à la croix par Jésus-Christ, au nom duquel les chrétiens peuvent le chasser. Satan sera enchaîné pendant le millenium puis libéré pour un peu de temps. Il sera finalement jeté dans l'étang de feu pour l'éternité. Voir \vref{Ez. 28:14-19}~; \vref{Es. 14:12-17}~; \vref{Ap. 12:4}~; \vref{Lu. 10:18-19}~; \vref{Ja. 4:7}~; \vref{Jn. 16:11}~; \vref{Ap. 12:4} et \vref{Ap. 20:1-15}.

\DicoEntry{SAÜL}\textit{, de l'hébreu «~Sha'uwl~»~: «~désiré, demandé (à Dieu)~»}\newline
1. Fils de Kis, israélite de la tribu de Benjamin, il fut choisi par Dieu pour être le premier roi d'Israël sur qui il régna pendant quarante ans. Il désobéit à la loi de Yahweh et tenta plusieurs fois d'assassiner David, choisi par Dieu pour lui succéder sur le trône. Saül mourut avec ses trois fils pendant la bataille de Guilboa. Voir \vref{1 S. 10}~; \vref{1 S. 13:1-14}~; \vref{1 S. 15:10-11}~; \vref{1 S. 18:8-16}~; \vref{1 S. 19:8-17} et \vref{1 S. 31}.
\\2. Nom initial de Paul*.

\DicoEntry{SCANDALE}\textit{, de l'hébreu «~mikshowl~»~: «~trébucher~» et du grec «~skandalon~»~: «~obstacle, piège~»}\newline
Pierre qu'on rencontre et qui peut faire glisser sur le chemin ou encore situation ou comportement qui provoque un trouble emmenant quelqu'un à fauter. Le scandale n'est pas forcément une mauvaise action en soi~; Christ lui-même fut un scandale pour les Juifs. Toutefois, il reste souvent lié aux œuvres de la chair et peut être provoqué par un manque de discernement. Le chrétien doit veiller par rapport aux scandales. Voir \vref{Ps. 106:36}~; \vref{Mt. 13:41}~; \vref{Mt. 18:7}~; \vref{1 Co. 1:23} et \vref{1 Pi. 2:7-8}.

\DicoEntry{SCEAU}\textit{, du grec «~sphragizo~»~: «~mettre un sceau dessus, poser une marque par l'impression d'un sceau~»}\newline
Sous l'Ancienne Alliance, la circoncision était une marque de l'alliance établie entre Yahweh et son peuple. A la naissance d'en haut, le chrétien est scellé du Saint-Esprit, témoignant de son appartenance à Christ. Voir \vref{Ge. 17:10-11}~; \vref{Ep. 1:13}~; \vref{Ap. 7:3} et \vref{Ap. 9:4}.

\DicoEntry{SCHEOL}\textit{}\newline
Voir SÉJOUR DES MORTS.

\DicoEntry{SCHISME D'ISRAËL}\textit{}\newline
Le schisme est la séparation d'Israël en deux royaumes suite à la dérive de Salomon. En 931 av. J.-C., Roboam succéda à son père Salomon sur le trône royal et n'accepta pas d'alléger le joug que son père avait mis sur eux, cela entraîna la séparation du royaume en deux. On retrouva d'une part, le royaume d'Israël dirigé par Jéroboam - appelé aussi royaume du nord -, composé des dix tribus du nord et d'autre part, le royaume de Juda gouverné par le roi Roboam composé des deux tribus du sud (Benjamin et Juda). Voir \vref{1 R. 12:1-24}.

\DicoEntry{SCRIBE}\textit{, de l'hébreu «~caphar~»~: «~secrétaire, scribe~», «~homme instruit, qui a le savoir~»}\newline
Les scribes occupaient une position importante auprès du peuple juif, ayant non seulement une mission d'enseignement de la loi, mais également une fonction au sein de la justice juive en prenant part au sanhédrin*. Voir \vref{Esd. 7:6-10} et \vref{Mt. 16:21}.

\DicoEntry{SECTE}\textit{, du grec «~hairesis~»~: «~action de prendre, capturer~»}\newline
Groupement de personnes adhérant à une doctrine particulière et vivant marginalement, comme les sadducéens ou les pharisiens. Les premiers disciples furent qualifiés de «~secte des Nazaréens~». Pierre met en garde contre les faux prophètes qui introduisent des sectes pernicieuses pour ravir la foi des chrétiens afin de les entraîner dans la dissolution. Voir \vref{Ac. 5:17}~; \vref{Ac. 26:5}~; \vref{Ac. 24:5} et \vref{2 Pi. 2:1}.

\DicoEntry{SÉDÉCIAS}\textit{, de l'hébreu «~Tsidqiyah~»~: «~Yahweh est justice~»}\newline
Fils d'Hamoutal et oncle de Jojakin, il fut le dernier roi de Juda sur qui il régna onze ans. Son nom initial, Matthania, fut changé en Sédécias par Nebucadnetsar, roi de Babylone. Il fit ce qui est mal aux yeux de Yahweh et connut un destin tragique~: ses fils furent égorgés devant lui, Nebucadnestar lui creva ensuite les yeux, Jérusalem et le temple* furent détruits et il fut emmené captif avec le peuple à Babylone. Voir \vref{2 R. 24:17-19}~; \vref{2 R. 25:1-21}~; \vref{Jé. 21}~; \vref{Jé. 22:1-9}~; \vref{Jé. 37,38,39:6-7}.

\DicoEntry{SÉJOUR DES MORTS}\textit{de l'hébreu «~she'owl~»~: «~monde souterrain, tombe, enfer, fosse~» et du grec «~hades~»~: «~dieu des profondeurs de la terre~»}\newline
Lieu de captivité où allaient les âmes de tous les défunts avant le sacrifice de Christ. Il était scindé en deux parties séparées par un grand abîme. D'un côté se trouvait un lieu de tourments et de souffrances extrêmes accueillant tous les méchants qui ont vécu dans le péché durant leur vie terrestre et qui n'y ont pas renoncé. D'un autre côté, il y avait le sein d'Abraham où reposaient et séjournaient les âmes des justes qui avaient foi en Yahweh. Après la résurrection de Jésus, ces derniers ont été arrachés du séjour des morts par le Seigneur qui les a emmenés au paradis*. Le ciel, en tant que destination des personnes décédées, fut en effet ouvert par Christ après sa résurrection. Par conséquent, le sein d'Abraham n'a jamais accueilli de chrétiens. Le séjour des morts est à présent composé uniquement d'impies~; à la fin du monde, il sera jeté avec tous ses habitants dans l'étang de feu*. Voir \vref{Lu. 16:19-31}~; \vref{1 S. 28:6-20}~; \vref{Mt. 11:23}~; \vref{Ac. 2:27}~; \vref{Jn. 3:13}~; \vref{Ep. 4:8} et \vref{Ap. 20:14}.
\\Note~: L'histoire de Lazare et de l'homme riche racontée dans Luc \vref{16:19-31} n'est pas une parabole. A la différence de tous les récits à caractère parabolique contés dans les Ecritures, cette histoire mentionne un nom.

\DicoEntry{SEM}\textit{, de l'hébreu «~Shem~»~: «~nom, renommée~»}\newline
Fils aîné de Noé et ancêtre d'Abraham. Voir \vref{Ge. 10:1} et \vref{Ge. 11:10-27}.

\DicoEntry{SÉNEVÉ}\textit{, du grec «~sinapi~»~: «~graine de moutarde~»}\newline
Plante des régions orientales ayant la forme d'une petite semence pouvant grandir de manière exponentielle et atteignant jusqu'à trois mètres. Elle symbolise spirituellement la puissance de la foi* capable de déplacer les montagnes. Voir \vref{Mt. 13:32-33} et \vref{Mt. 17:20}.

\DicoEntry{SEPHORA}\textit{, de l'hébreu «~Tsipporah~»~: «~petit oiseau, moineau~»}\newline
Fille de Jéthro, femme de Moïse et mère d'Eliézer et de Guerschom. Elle partit dans le pays d'Egypte avec Moïse quand il répondit à l'appel de Yahweh pour aller libérer Israël. \vref{Ex. 2:15-21} et \vref{Ex. 4:18-20}.

\DicoEntry{SÉRAPHINS}\textit{, de l'hébreu «~saraph~»~: «~être majestueux avec six ailes au service de Dieu~»}\newline
Catégorie d'anges* proclamant la sainteté de Dieu. Voir \vref{Es. 6:1-7}.

\DicoEntry{SERPENT}\textit{, de l'hébreu «~nachash~»~: «~serpent, reptile~»}\newline
C'est sous la forme du serpent que Satan vint séduire Eve dans le jardin d'Eden. Le serpent fut maudit d'entre tous les animaux pour son action. Le serpent ancien ou rusé désigne le diable*~; il s'oppose au serpent d'airain, Jésus, qui a donné à ses enfants le pouvoir de marcher sur les serpents. Voir \vref{Ge. 3:1-14}~; \vref{No. 21:4-9}~; \vref{2 Co. 11:3}~; \vref{Jn. 3:14-15}~; \vref{Lu. 10:19} et \vref{Ap. 12:9,14-15}.

\DicoEntry{SERVICE}\textit{, du grec «~diakonia~»~: «~service~», dérivé du mot grec «~diakonos~»~: «~domestique~»}\newline
Tâche que le chrétien exerce au service de Dieu et des hommes selon l'onction* et le mandat que Dieu lui donne. Le serviteur de Dieu est donc un serviteur inutile, un simple instrument utilisé pour la gloire de Yahweh. Voir \vref{Lu. 17:10}~; \vref{1 Co. 12}~; \vref{2 Co. 3:5}~; \vref{Ro. 12} et \vref{1 Pi. 4:10-11}.

\DicoEntry{SERVITEUR}\textit{, du grec «~diakonos~»~: «~domestique, subordonné, messager~» ou «~doulos~»~: «~esclave~»}\newline
Christ a renoncé à sa gloire et a pris la forme d'un simple serviteur. De même, le chrétien n'est pas uniquement serviteur de Dieu, il doit comme le maître servir son prochain. Voir \vref{Mc. 10:45}~; \vref{Ph. 1:1}~; \vref{Ph. 2:5-8} et \vref{2 Co. 6:4}.

\DicoEntry{SETH}\textit{, de l'hébreu «~Sheth~»~: «~compensation, mis à la place~»}\newline
Troisième fils d'Adam et Eve~; il naquit après le meurtre de son frère Abel que Caïn avait tué. Seth fut l'ancêtre de Noé et de Jésus-Christ. Voir \vref{Ge. 4:25}~; \vref{Ge. 5:6-29} et \vref{Lu. 3:38}.

\DicoEntry{SHOFAR}\textit{, de l'hébreu «~showphar~»~: «~corne, corne de bélier~».}\newline
Instrument de musique à vent fait à partir de la corne de bélier. Voir TROMPETTE.

\DicoEntry{SIDON}\textit{, de l'hébreu «~Tsiydown~»~: «~abondance de poisson, pêche~»}\newline
Ville de l'antique Phénicie (actuel Liban) située non loin de Tyr~; on y vénérait les Baals et les Astartés. La reine Jézabel était originaire de Sidon. Voir \vref{Jg. 10:6} et \vref{1 R. 16:31}.

\DicoEntry{SILAS}\textit{, du grec «~Silas~»~: «~de la forêt, demandé~»}\newline
Prophète, compagnon d'œuvre de Paul avec qui il effectua plusieurs voyages missionnaires. Voir \vref{Ac. 15-18}.

\DicoEntry{SILO}\textit{, de l'hébreu «~Shiyloh~»~: «~lieu de repos~»}\newline
Ville située au nord-est de la tribu d'Ephraïm où les enfants d'Israël se répartirent les territoires avant la conquête de Canaan. Avant d'être placée à Jérusalem, l'arche de l'alliance se trouvait à Silo. Voir \vref{Jos. 18:10}~; \vref{Jos. 19:51}~; \vref{1 S. 3:19-21} et \vref{1 S. 4:3}.

\DicoEntry{SILOÉ}\textit{, de l'hébreu «~Shiloach~»~: «~envoyé~»}\newline
Source d'eau se trouvant au sud-est de Jérusalem. Voir \vref{Né. 3:15} et \vref{Jn. 9:6-7}.

\DicoEntry{SIMÉON}\textit{, de l'hébreu «~Shim`own~»~: «~qui écoute, qui a été entendu~»}\newline
1. Fils de Jacob et Léa. Avec Lévi, son frère, il vengea le déshonneur de sa sœur Dina, en tuant Sichem, prince de Canaan, son père Hamor, et tous leurs hommes. Il fut gardé comme otage en Egypte, lorsque Joseph voulut éprouver la sincérité de ses frères. Il devint le père de la tribu des Siméonites qui s'installèrent au sud de Canaan. Voir \vref{Ge. 29:33}~; \vref{Ge. 34}~; \vref{Ge. 42:21-38} et \vref{Jos. 19:1-9}.
\\2. Homme de foi à qui le Saint-Esprit avait promis qu'il ne mourrait pas sans avoir vu le Messie. Il rencontra Jésus lorsqu'il était enfant, à Jérusalem. Voir \vref{Lu. 2:25-35}.

\DicoEntry{SIMON}\textit{, de l'hébreu «~Shiymown~»~: «~désert~» ou «~qui entend~»}\newline
1. Simon Pierre, le nom originel de Pierre* était Simon. Voir \vref{Jn. 1:40-42}.
\\2. Simon le zélote, il faisait partie du groupuscule des zélotes avant de devenir apôtre de Christ. Voir \vref{Lu. 6:13-16}.
\\3. Simon de Cyrène, il fut contraint d'aider Jésus à porter la croix jusqu'à Golgotha. Voir \vref{Mt. 27:32}.
\\4. Simon le magicien, originaire de la ville de Samarie, il fut baptisé par Philippe et crut pouvoir acheter à prix d'argent la puissance du Saint-Esprit. Voir \vref{Ac. 8:9-24}.

\DicoEntry{SION}\textit{, de l'hébreu «~Tsiyown~»~: «~lieu desséché~»}\newline
Autre nom pour parler de Jérusalem. Sous la Nouvelle Alliance, la montagne de Sion est l'image de la Jérusalem céleste. Voir \vref{De. 4:48}~; \vref{1 R. 8:1}~; \vref{Es. 2:3}~; \vref{2 S. 5:6-7} et \vref{Hé. 12:22}.

\DicoEntry{SISERA}\textit{, de l'hébreu «~Ciycera'~»~: «~déploiement, champ de bataille~»}\newline
Chef de l'armée du roi cananéen Jabin, son armée fut vaincue par Barak et Sisera fut tué par Jaël, femme de Héber, le Kénien. Voir \vref{Jg. 4}.

\DicoEntry{SMYRNE}\textit{, du grec «~Smurna~»~: «~myrrhe~»}\newline
Cité de la côte occidentale de l'Asie Mineure, Smyrne (aujourd'hui Izmir) était située au nord d'Ephèse et réputée pour sa splendeur et ses richesses. Ses forteresses et ses tours de l'acropole évoquaient une couronne. Très unie à Rome, des cultes en l'honneur du dieu Zeus, de la déesse Cybèle, ou encore de l'empereur Tibère et sa mère Julie y étaient célébrés. Proche d'Ephèse, l'église de Smyrne fut probablement le fruit du travail apostolique de Paul. En proie à ces doctrines impies, l'église de Smyrne était fortement persécutée aussi bien par les Romains que par «~les faux Juifs~» membres «~d'une synagogue de Satan~». Sa persévérance face aux afflictions lui permit de recevoir un bon témoignage du Seigneur. Elle incarne l'église persécutée. Voir \vref{Ap. 2:8-11}.

\DicoEntry{SODOME}\textit{, de l'hébreu «~Cedom~»~: «~qui brûle~»}\newline
Ville cananéenne située dans la plaine du Jourdain à proximité de laquelle Lot s'installa après s'être séparé d'Abraham. Ses habitants étaient de grands pêcheurs devant Yahweh à un tel point qu'il détruisit la ville - avec Gomorrhe* - en faisant tomber du ciel une pluie de feu et de soufre. Lot et ses deux filles furent épargnés grâce à l'intercession* d'Abraham. Voir \vref{Ge. 13:10-13} et \vref{Ge. 19:1-29}.

\DicoEntry{SOPHONIE}\textit{, de l'hébreu «~Tsephanyah~»~: «~Yahweh a caché, protégé~»}\newline
Fils de Cuschi, descendant du roi Ezéchias, prophète de Yahweh ayant vécu au temps du roi Josias. L'ensemble de ses prophéties se trouve dans le livre portant son nom.

\DicoEntry{STOÏCIENS}\textit{, du grec «~stoikos~»~: «~appartenant au portique~»}\newline
Adeptes de la doctrine de Zénon de Kition (336 av. J.-C. – 264 av. J.-C.) qui fonda le stoïcisme à Chypre en 301 av. J.-C. Le stoïcisme était l'une des principales doctrines philosophiques de la Grèce antique avec l'épicurisme. Elle reposait sur la morale et la maîtrise de ses sentiments par une vie en conformité avec la nature. A Athènes, quelques stoïciens, accompagnés d'épicuriens, se confrontèrent à Paul, le menant à l'aréopage afin de l'interroger. Voir \vref{Ac. 17:18-20}.

\DicoEntry{SYNAGOGUE}\textit{, du grec «~sunagogue~»~: «~assemblée, lieu de réunion~»}\newline
Assemblée de Juifs réunis pour prier et écouter la lecture des Ecritures. Jésus y enseigna régulièrement pendant son service. Les apôtres annoncèrent également l'Evangile dans des synagogues. Voir \vref{Mt. 4:23}~; \vref{Mt. 9:35}~; \vref{Mc. 6:2} et \vref{Ac. 14:1}.

\DicoEntry{TABERNACLE}\textit{, de l'hébreu «~mishkan~»~: «~sanctuaire, demeure, lieu d'habitation~»}\newline
Appelée aussi tente d'assignation, habitation mobile de Yahweh construite selon le modèle que Dieu donna à Moïse dans le désert. Les Lévites en assuraient le service avec tous les ustensiles qui lui étaient dédiés. Une nuée s'élevait au-dessus du tabernacle pour signifier aux Israélites qu'ils devaient lever le camp* et poursuivre leur marche. Voir \vref{Ex. 25:8-9}~; \vref{Ex. 39:32}~; \vref{No. 1:50-51}~; \vref{Ex. 40:36-38} et \vref{1 Ch. 6:48}.

\DicoEntry{TANAKH}\textit{}\newline
Voir Introduction.

\DicoEntry{TEMPLE}\textit{, de l'hébreu «~heykal~»~: «~palais, temple, sanctuaire~» (voir illustration)}\newline
David projeta de construire un temple pour Yahweh~; son fils Salomon fut mandaté pour l'ériger en remplacement du tabernacle. Il fut détruit une première fois par les Babyloniens au VIème siècle av. J.-C. Reconstruit lors du retour d'exil des Juifs, il fut de nouveau détruit en 70 par les Romains~; il n'en reste qu'un mur aujourd'hui appelé «~mur des lamentations~». Sous la Nouvelle Alliance, Yahweh a choisi pour temple l'Eglise*, le corps de chaque chrétien en qui il vient résider à la naissance d'en haut. Voir \vref{2 S. 7}~; \vref{1 R. 6}~; \vref{2 R. 25:8-9}~; \vref{Esd. 6:15}~; \vref{Ep. 2:21-22} et \vref{1 Co. 6:19}.

\DicoEntry{TÉNÈBRES}\textit{, de l'hébreu «~chosnek~»~: «~obscurité, ténèbres, nuit, lieu caché~»}\newline
Dès la Genèse, la lumière est séparée des ténèbres qui peuvent symboliser le péché, l'ignorance et l'absence de la vie de Dieu. Véritable prison rendant les hommes captifs, les ténèbres éternelles du séjour des morts* seront pour les anges déchus, le diable et tous les méchants. Voir \vref{Ge. 1:2-5}~; \vref{2 S. 22:29}~; \vref{Ps. 107:10}~; \vref{Job 17:13}~; \vref{Ro. 13:12}~; \vref{1 Th. 5:5}~; \vref{2 Pi. 2:4}~; \vref{1 Jn. 2:11} et \vref{Jud. 1:6-13}.

\DicoEntry{TÉRÉBINTHE}\textit{, de l'hébreu «~'elah~»~: «~térébinthe ou chêne~»}\newline
Grand arbre robuste dont l'ombrage est agréable, il est répandu en Israël. Jacob enterra les dieux étrangers de sa maison sous un térébinthe. L'ange de Yahweh apparut sous un térébinthe à Gédéon. Des cultes idolâtres étaient célébrés à l'ombre de ces arbres. C'est à la vallée des térébinthes, située au sud-ouest de Jérusalem, que David tua Goliath. Voir \vref{Ge. 35:4}~; \vref{Jg. 6:11,19}~; \vref{2 S. 18:9}~; \vref{1 Ch. 10:12}~; \vref{Os. 4:13}~; \vref{Es. 57:5} et \vref{1 S. 17:1-50}.

\DicoEntry{TÉTRARQUE}\textit{, du grec «~tetrarches~»~: «~tétrarque~»}\newline
Titre donné au gouverneur d'un territoire sous domination romaine. Hérode Antipas* était le tétrarque de Galilée. Voir \vref{Mt. 14:1} et \vref{Lu. 3:1}.

\DicoEntry{THADÉE}\textit{}\newline
Voir JUDE.

\DicoEntry{THÉRAPHIM}\textit{, de l'hébreu «~teraphiym~»~: «~idolâtries, idoles~»}\newline
Amulette utilisée dans les cultes idolâtres. Rachel déroba les théraphim de son père Laban avant de quitter sa maison. Voir \vref{Ge. 31:19,34-35}.

\DicoEntry{THESSALONIQUE}\textit{, du grec «~Thessalonike~»~: «~victoire de ce qui est faux~»}\newline
Ville située au Nord de la Grèce actuelle, sur la côte de la mer Egée, elle jouissait d'une importante fréquentation puisqu'elle figurait parmi les trois ports principaux de la Méditerranée et se situait sur l'une des plus grandes routes commerciales de l'époque~: la Voie Egnatienne reliant Rome à Byzance. Sur le plan religieux, les habitants étaient polythéistes et pratiquaient une variété de cultes, dont le culte impérial. Durant trois semaines, Paul enseigna dans une synagogue* à Thessalonique~; de là, il réussit à constituer un groupe de croyants. Toutefois, une violente persécution l'obligea à quitter promptement la ville, laissant la communauté nouvellement formée vulnérable et fragile. Il écrivit deux lettres aux saints de Thessalonique qui figurent dans le canon biblique.

\DicoEntry{THOMAS}\textit{, de l'hébreu «~Ta'own~»~: «~jumeau~»}\newline
Surnommé Didyme, il était l'un des douze apôtres*. Dans un premier temps incrédule quant à la résurrection de Jésus, il confessa la Seigneurie de ce dernier lorsqu'il le vit ressuscité. Voir \vref{Lu. 6:12}~; \vref{Jn. 11:16} et \vref{Jn. 20:24-29}.

\DicoEntry{TIMOTHÉE}\textit{, du grec «~Timotheos~»~: «~qui adore, ou honore Dieu~»}\newline
Fils d'une femme juive croyante et d'un père grec. Lié à Paul comme un fils à son père, il devint l'un de ses plus fidèles collaborateurs et l'accompagna à plusieurs reprises dans ses voyages missionnaires. Malgré sa jeunesse, il lui fut confié des tâches liées à la direction des églises, notamment à Ephèse. Timothée reçut de Paul deux lettres regorgeant de conseils et d'instructions pour être un bon serviteur de l'Evangile*. Voir \vref{Ac. 16:1-3}~; \vref{Ac. 18:5}~; \vref{1 Co. 16:10} et les deux épîtres de Paul à Timothée.

\DicoEntry{TITE}\textit{, du grec «~Titos~»~: «~nourrice, honorable~»}\newline
D'origine grecque, Tite fut un fidèle compagnon d'œuvre de l'apôtre Paul. Il l'accompagna à Jérusalem, œuvra à Corinthe et en Dalmatie et s'occupa plus particulièrement de l'église de Crète. Il reçut une lettre de Paul qui figure dans le canon biblique. Voir \vref{2 Co. 8:6,23}~; \vref{Ga. 2:1}~; \vref{2 Ti. 4:10} et l'épître de Paul à Tite.

\DicoEntry{TRIBULATION}\textit{, du grec «~thlipsis~»~: «~une pression, une oppression~»}\newline
Persécution, tourment provoqué par l'annonce de l'Evangile. Inévitables pour entrer dans le royaume de Dieu*, les tribulations ont pour but de rendre le chrétien patient, joyeux et persévérant en toutes circonstances. Voir \vref{Mc. 4:17}~; \vref{Jn. 16:33}~; \vref{Ac. 14:22}~; \vref{2 Co. 6:4}~; \vref{2 Co. 8:2}~; \vref{Ph. 1:29}~; \vref{1 Th. 3:3} et \vref{2 Th. 1:4}.

\DicoEntry{TRIBUNAL}\textit{, de l'hébreu «~qahal~»~: «~assembler, convoquer~» et du grec «~bema~»~: «~tribune~»}\newline
Lieu où les hommes sont jugés afin de recevoir une sentence en fonction des actes qu'ils ont posés. Chaque être humain comparaîtra devant le tribunal de Christ afin de rendre compte pour lui-même. Voir \vref{Ro. 14:10-12} et \vref{2 Co. 5:10}.

\DicoEntry{TRINITÉ}\textit{}\newline
Doctrine selon laquelle le Dieu unique se manifesterait en trois personnes distinctes~: Père, Fils et Saint-Esprit. Inspirée des triades païennes (babylonienne, égyptienne…), cette fausse doctrine d'origine catholique apparut au IIème siècle et fut fixée aux Conciles de Nicée en 325 et de Constantinople I en 381. Elle fut largement reprise par les protestants et la plupart des mouvements chrétiens alors que ni le mot ni le concept de trinité n'apparaissent dans les Ecritures. Voir \vref{De. 6:4}~; \vref{Es. 9:5}~; \vref{Jn. 4:23-24}~; \vref{Col. 2:8-10}~; \vref{2 Th. 1:12} et \vref{1 Jn. 5:20}.

\DicoEntry{TROMPETTE}\textit{, plusieurs mots hébreux ont été traduits par trompette, les plus utilisés sont~: «~chatsotserah~»~: «~trompette, clairon~»~; «~yobel~»~: «~bélier, corne de bélier~», «~retentissant~», «~jubilé~», et «~showphar~»~: «~corne de bélier~». Plusieurs mots grecs ont aussi été utilisés, notamment «~salpigx~»~: «~une trompette~» et «~salpizo~»~: «~sonner de la trompette~».}\newline
Sous l'Ancienne Alliance, on l'utilisait pour donner un signal, publier une sainte convocation, fêter des moments de joie, signifier une victoire, chanter des cantiques en l'honneur de Yahweh, avertir et rassembler le peuple. Le son de la trompette est aussi l'image des voix prophétiques qui crient et appellent le peuple à revenir totalement à Dieu. Selon les Ecritures, lorsque la dernière trompette retentira, l'Eglise sera enlevée pour les noces. Dans le livre d'Apocalypse, la voix du Seigneur est comparée au son d'une trompette. Voir \vref{Ex. 19:13}~; \vref{Lé. 23:24}~; \vref{No. 10:9-10}~; \vref{1 Ch. 16:42}~; \vref{Ez. 33:3}~; \vref{Mt. 24:31}~; \vref{1 Co. 15:52}~; \vref{1 Th. 4:16} et \vref{Ap. 1:10}.

\DicoEntry{TYR}\textit{, de l'hébreu «~Tsor~»~: «~un rocher~»}\newline
Ville de l'antique Phénicie (actuel Liban). Hiram, roi de Tyr, donna- en échange de vivres - du bois de cèdre et du bois de cyprès à Salomon pour la construction du temple. Le roi de Tyr est une image de Satan* dans une prophétie d'Ezéchiel. Voir \vref{1 R. 5:1-12} et \vref{Ez. 28}.

\DicoEntry{UR}\textit{, de l'hébreu «~'Uwr~»~: «~flamme, éclat, feu~»}\newline
Ville de Chaldée située au sud de la Babylonie et d'où Abraham était originaire. Voir \vref{Ge. 11:27-31}.

\DicoEntry{URIE}\textit{, de l'hébreu «~Uwriyah~»~: «~Yahweh est ma lumière~»}\newline
Héthien, mari de Bath-Schéba. Il mourut sur le champ de bataille suite à une conspiration de David qui avait connu sa femme et l'avait mise enceinte. Voir \vref{2 S. 11}.

\DicoEntry{VIE ÉTERNELLE}\textit{, de l'hébreu «~aionios~»~: «~sans commencement ni fin~»}\newline
La vie éternelle est un don gratuit de Dieu, un héritage, une promesse qui commence dès la conversion au travers de la connaissance de Dieu. La vie éternelle est Christ lui-même. Voir \vref{Jn. 3:16,36}~; \vref{Ro. 2:7}~; \vref{Ro. 6:23}~; \vref{Tit. 3:7}~; \vref{1 Jn. 2:25} et \vref{1 Jn. 5:20}.

\DicoEntry{VIGNE}\textit{}\newline
Arbre cultivé pour son fruit, la vigne est assimilée à la joie à cause du vin produit par le raisin et consommé dans le cadre de festivités. Le peuple d'Israël était la première vigne de Yahweh, mais elle ne porta pas de fruits. Le royaume de Dieu est aussi associé à la vigne~: Dieu est le vigneron, Jésus-Christ est le cep et tous les enfants de Dieu sont les sarments. Tout sarment qui ne porte pas de fruits est jeté au feu, c'est-à-dire en enfer. Voir \vref{Es. 5:1-7}~; \vref{Mt. 21:33-43} et \vref{Jn. 15:1-8}.

\DicoEntry{VOILE}\textit{, de l'hébreu «~porokhet~»~: «~rideau, voile~» et du grec «~peribolaion~»~: «~une couverture, une enveloppe~»}\newline
1. Etoffe de fin lin retors qui servait de séparation entre le lieu saint et le Saint des saints. Lorsque Jésus-Christ fut crucifié, ce voile se déchira en deux, de haut en bas, ouvrant ainsi l'accès au Saint des saints. Cet événement symbolisait que, par Jésus, tout homme pouvait accéder librement à la présence du Père. Voir \vref{Ex. 26:31-33}~; \vref{Lé. 16:11-19}~; \vref{Mt. 27:50-51}~; \vref{Hé. 9:7-8} et \vref{Hé. 10:19-20}.
\\2. Pan de tissu utilisé pour se couvrir la tête dans certaines cultures. Paul expliqua que les longs cheveux étaient une gloire pour la femme et qu'ils faisaient office de voile naturel. Voir \vref{Ge. 24:65} et \vref{1 Co. 11:15}.
\\3. Au sens figuré, le voile symbolise l'intelligence obscurcie, le cœur non converti et le manque de révélation de la parole qui sont des barrières à la compréhension de la loi. Voir \vref{2 Co. 3:14-16}.

\DicoEntry{YHWH}\textit{}\newline
Aussi appelé tétragramme (mot de quatre lettres), nom avec lequel Dieu se révéla à Moïse lorsque ce dernier le rencontra pour la première fois à Horeb. Ce nom, prononcé Yahweh, signifie «~Je suis celui qui suis~» et souligne le caractère éternel de Dieu. Voir \vref{Ex. 3:1-14}.

\DicoEntry{ZABULON}\textit{, de l'hébreu «~Zebuwluwn~»~: «~habitation~»}\newline
Fils de Jacob et Léa, il devint l'ancêtre de la tribu de Zabulon. Voir \vref{Ge. 30:19-20} et \vref{No. 2:7}.

\DicoEntry{ZACHARIE}\textit{, de l'hébreu «~Zekaryah~»~: «~Yahweh se souvient~»}\newline
1. Fils de Jéroboam, roi d'Israël sur qui il régna uniquement six mois. Il fit ce qui est mal devant Yahweh et fut tué suite à une conspiration contre lui. Voir \vref{2 R. 15:8-11}.
\\2. Prophète et prêtre, fils de Bérékia et petit-fils d'Iddo. Avec le prophète Aggée, il assista Zorobabel, gouverneur de Juda, et Josué, grand prêtre, dans la restauration du temple de Yahweh au retour de la captivité des Juifs. L'ensemble de ses prophéties se trouve dans le livre portant son nom. Voir \vref{Esd. 5:1-2} et \vref{Esd. 6:14-5}.
\\3. Prêtre et père de Jean-Baptiste qu'il eut avec sa femme Elisabeth à un âge avancé. Voir \vref{Lu. 1:5}.

\DicoEntry{ZÉLOTE}\textit{, du grec «~zelotes~»~: «~celui qui est zélé~»}\newline
Patriotes juifs fervents défenseurs de la loi et des traditions ayant pour objectif de résister à l'invasion romaine. Simon, l'un des douze apôtres, en faisait partie. Voir \vref{Lu. 6:15} et \vref{Ac. 1:13}.

\DicoEntry{ZOROBABEL}\textit{, de l'hébreu «~Zerubbabel~»~: «~rejeton de Babylone~»}\newline
Fils de Schealthiel, gouverneur de Juda, il participa à la restauration du temple* de Yahweh après le retour de la captivité du peuple juif. Il figure dans la généalogie de Jésus. Voir \vref{Esd. 3:2}~; \vref{Esd. 5:2}~; \vref{Ag. 1:14}~; \vref{Mt. 1:13} et \vref{Lu. 3:27}.

\end{multicols}
\clearpage}\markboth{}{}
%    \makeatletter
%    % réinitialiser mise en forme
%    \def\@oddhead{\hfil\thepage\hfil}
%    \def\@evenhead{\hfil}
%\makeatother
%% tableaux
%\addcontentsline{toc}{section}{Histoire de la Bible}\clearpage
%\begin{center}Histoire de la Bible 1\end{center}\clearpage
%\begin{center}Histoire de la Bible 2\end{center}\clearpage
%\addcontentsline{toc}{section}{Dénominations}\clearpage
%\begin{center}Dénominations 1\end{center}\clearpage
%\begin{center}Dénominations 2\end{center}\clearpage
%\addcontentsline{toc}{section}{Doctrines}\clearpage
%\begin{center}Doctrines 1\end{center}\clearpage
%\begin{center}Doctrines 2\end{center}\clearpage
%\begin{center}Doctrines 3\end{center}\clearpage
%\begin{center}Doctrines 4\end{center}\clearpage
%\begin{center}Doctrines 5\end{center}\clearpage
%\begin{center}Doctrines 6\end{center}\clearpage
%\begin{center}Doctrines 7\end{center}\clearpage
%\begin{center}Doctrines 8\end{center}\clearpage
%\begin{center}Doctrines 9\end{center}\clearpage
%\begin{center}Doctrines 10\end{center}\clearpage
%\begin{center}Doctrines 11\end{center}\clearpage
%\addcontentsline{toc}{section}{Monnaies}\clearpage
%\begin{center}Monnaies\end{center}\clearpage
%\addcontentsline{toc}{section}{Longueurs / Liquides}\clearpage
%\begin{center}Longueurs / Liquides\end{center}\clearpage
%\addcontentsline{toc}{section}{Poids}\clearpage
%\begin{center}Poids\end{center}\clearpage
%\addcontentsline{toc}{section}{Fêtes de Yahweh}\clearpage
%\begin{center}Fêtes de Yahweh\end{center}\clearpage
%\addcontentsline{toc}{section}{Alphabet hébreu}\clearpage
%\begin{center}Alphabet hébreu\end{center}\clearpage
\end{document}

\clearpage\ShortTitle{Lamentations de Jérémie}\BookTitle{Lamentations de Jérémie}\BFont
\noindent\hrulefill
{\footnotesize
\textit{
\bigskip
{\centering{}
\\Auteur : Jérémie
\\(Heb. : Eikha)
\\Signification : Où ?
\\Thème : Affliction pour Jérusalem
\\Date de rédaction : 6\up{ème} siècle av. J.-C\\}
}
%\bigskip
\textit{
\\Recueil de pièces poétiques, les lamentations de Jérémie furent composées selon un procédé visant à accentuer le caractère funèbre, de façon à ce qu'elles soient récitées avec gémissements. Ses complaintes exposent la profonde désolation du prophète face au fardeau du peuple qu'il portait dans ses entrailles tout comme la douleur et la tristesse de Yahweh face à Israël.
%\bigskip
\\Très différentes des prophéties retrouvées dans le livre de Jérémie, les Lamentations reflètent l'affliction convenant à la
gravité du châtiment subi : famine, pillage et ruine du temple, déportation, cessation du culte, diverses calamités… Jérémie rappelle ainsi les conséquences de l'endurcissement du cœur face aux appels à la repentance ; il présente aussi les bontés éternelles de Yahweh.\bigskip
}
}
\par\nobreak\noindent\hrulefill
\begin{multicols}{2}
\Chap{1}
\TextTitle{Pleurs et désolation de Jérusalem}
\VerseOne{}[Aleph.] Comment est-il arrivé que la ville si peuplée se trouve si solitaire ? Que celle qui était grande entre les nations est devenue comme une veuve ? Que celle qui était noble dame entre les provinces a été rendue tributaire ?
\VS{2}[Beth.] Elle ne cesse de pleurer pendant la nuit, et ses larmes sont sur ses joues ; il n'y a pas un de tous ses amis qui la console ; ses intimes amis ont agi perfidement contre elle, ils sont devenus ses ennemis.
\VS{3}[Guimel.] Juda a été emmenée captive tant elle est affligée, et tant est grande sa servitude ; elle demeure maintenant entre les nations, et ne trouve point de repos ; tous ses persécuteurs l'ont attrapée dans sa détresse\FTNT{Jé. 52:26.}.
\VS{4}[Daleth.] Les chemins de Sion mènent deuil de ce qu'il n'y a plus personne qui vienne aux fêtes solennelles ; toutes ses portes sont désolées, ses sacrificateurs sanglotent, ses vierges sont accablées de tristesse ; elle est remplie d'amertume. 
\VS{5}[He.] Ses adversaires sont établis pour chefs, ses ennemis prospèrent ; car Yahweh l'a humiliée à cause de la multitude de ses transgressions ; ses petits enfants ont marché captifs devant l'adversaire\FTNT{Jé. 30:14.}.
\VS{6}[Vav.] Et tout l'honneur de la fille de Sion s'est retiré d'elle ; ses chefs sont devenus semblables à des cerfs qui ne trouvent pas de pâture, et qui fuient sans force devant celui qui les poursuit.
\VS{7}[Zayin.] Jérusalem dans les jours de son affliction et de son pauvre état s'est souvenue de toutes ses choses précieuses qu'elle avait depuis si longtemps, lorsque son peuple est tombé par la main de l'ennemi, sans aucun secours ; les ennemis l'ont vue, et se sont moqués de ses sabbats.
\VS{8}[Heth.] Jérusalem a grièvement péché ; c'est pourquoi elle est devenue un objet de dégoût ; tous ceux qui l'honoraient l'ont méprisée parce qu'ils ont vu son ignominie ; elle en a aussi sangloté, et s'est retournée en arrière.
\VS{9}[Teth.] Sa souillure était dans les pans de sa robe, et elle ne s'est pas souvenue de sa dernière fin ; elle a été extraordinairement abaissée, et elle n'a pas de consolateur. Vois ma misère, ô Yahweh ! Car l'ennemi s'est élevé avec orgueil !
\VS{10}[Yod.] L'ennemi a étendu sa main sur toutes ses choses désirables ; car elle a vu entrer dans son sanctuaire les nations au sujet desquelles tu avais donné cet ordre : Elles n'entreront point dans ton assemblée\FTNT{De. 23:3.}.
\VS{11}[Kaf.] Tout son peuple gémit, cherchant du pain\FTNT{Jé. 52:6.} ; ils ont donné leurs choses désirables pour des aliments, afin ranimer leur vie. Vois, ô Yahweh ! Regarde combien je suis méprisée.
\VS{12}[Lamed.] Cela ne vous touche-t-il point ? Vous tous passants, contemplez, et voyez s'il est une douleur comme ma douleur, celle dont j'ai été frappée ! Moi que Yahweh a accablée de douleur au jour de l'ardeur de sa colère.
\VS{13}[Mem.] Il a envoyé d'en haut, dans mes os, un feu qui les domine ; il a tendu un filet sous mes pieds, et m'a fait revenir en arrière ; il m'a mise dans la désolation, dans une langueur de tous les jours.
\VS{14}[Nun.] Le joug de mes iniquités est lié par sa main ; elles sont entrelacées, et appliquées sur mon cou ; il a renversé ma force ; le Seigneur m'a livrée entre les mains de ceux contre qui je ne pourrai pas me lever.
\VS{15}[Samech.] Le Seigneur a abattu tous les hommes forts que j'avais au milieu de moi ; il a appelé contre moi, au temps fixé, une armée pour détruire mes jeunes hommes ; le Seigneur a foulé au pressoir la vierge, fille de Juda.
\VS{16}[Ayin.] À cause de ces choses, je pleure, mes yeux fondent en larmes ; car le consolateur qui restaurait ma vie est loin de moi. Mes fils sont dans la désolation parce que l'ennemi a été plus fort.
\VS{17}[Pe.] Sion a étendu les mains, et personne ne l'a consolée ; Yahweh a ordonné aux ennemis de Jacob de l'entourer de toutes parts. Jérusalem a été comme une impureté au milieu d'eux.
\VS{18}[Tsade.] Yahweh est juste car j'ai été rebelle à ses ordres. Ecoutez, vous tous, peuples, et voyez ma douleur ! Mes vierges et mes jeunes hommes sont allés en captivité.
\VS{19}[Qof.] J'ai appelé mes amis, mais ils m'ont trompé. Mes sacrificateurs et mes anciens sont morts dans la ville : Ils cherchaient de la nourriture afin de restaurer leur vie.
\VS{20}[Resh.] Regarde Yahweh ! car je suis dans la détresse ; mes entrailles bouillonnent, mon coeur palpite au dedans de moi, parce que je n'ai fait qu'être rebelle ; au dehors l’épée m’a privée d’enfants ; au dedans il y a comme la mort. 
\VS{21}[Shin.] On m'a entendu sangloter et je n'ai personne qui me console ; tous mes ennemis ont appris mon malheur, et s’en sont réjouis, parce que tu l’as fait ; tu amèneras le jour que tu as assigné, et ils seront dans mon état.
\VS{22}[Tav.]Que toute leur méchanceté vienne devant toi, et traite-leur comme tu m'as traitée à cause de tous mes péchés ; car mes sanglots sont en grand nombre et mon coeur est languissant. 
\Chap{2}
\TextTitle{Le jour de la colère de Yahweh}
\VerseOne{}[Aleph.] Comment est-il arrivé que le Seigneur a couvert de sa colère la fille de Sion tout à l'entour, comme d'une nuée, et qu'il a précipité du ciel sur la terre la beauté d'Israël, et ne s'est pas souvenu du marchepied de ses pieds\FTNT{Ez. 43:7.} au jour de sa colère ?
\VS{2}[Beth.] Le Seigneur a englouti sans épargner toutes les habitations de Jacob ; il a dans sa fureur renversé les forteresses de la fille de Juda, il les a jetées par terre ; il a profané le royaume et ses chefs.
\VS{3}[Guimel.] Il a retranché toute la force d'Israël par l'ardeur de sa colère ; il a retiré sa droite en arrière devant l'ennemi ; il s'est allumé dans Jacob comme un feu flamboyant qui le consume de toutes parts.
\VS{4}[Daleth.] Il a tendu son arc comme un ennemi ; sa droite s'est dressée comme celle d'un adversaire ; il a tué tout ce qui était agréable à l'œil dans la tente de la fille de Sion ; il a répandu sa fureur comme un feu.
\VS{5}[He.] Le Seigneur a été comme un ennemi ; il a englouti Israël, il a englouti tous ses palais, il a détruit toutes ses forteresses ; il a multiplié chez la fille de Juda le deuil et les afflictions.
\VS{6}[Vav.] Il a mis en pièces avec violence sa tente comme un jardin ; il a détruit le lieu de son assemblée ; Yahweh a fait oublier dans Sion la fête solennelle et le sabbat, et dans sa violente colère, il a rejeté le roi et le sacrificateur.
\VS{7}[Zayin.] Le Seigneur a rejeté au loin son autel, il a dédaigné son sanctuaire ; il a livré entre les mains de l'ennemi les murailles de ses palais ; ils ont poussé des cris dans la maison de Yahweh, comme aux jours des fêtes solennelles.
\VS{8}[Heth.] Yahweh avait projeté de détruire les murailles de la fille de Sion ; il a étendu le cordeau, il n'a pas fait revenir sa main sans les avoir engloutis ; il a plongé dans le deuil remparts et murailles, ils ont été ruinés tous ensemble.
\VS{9}[Teth.] Ses portes sont enfoncées dans la terre ; il en a détruit et brisé les barres. Son roi et ses chefs sont parmi les nations ; la loi n'est plus. Même les prophètes ne reçoivent plus aucune vision de Yahweh\FTNT{Ez. 7:26.}.
\VS{10}[Yod.] Les anciens de la fille de Sion sont assis à terre, ils sont muets ; ils ont couvert leur tête de poussière, ils se sont ceints de sacs ; les vierges de Jérusalem baissent leurs têtes vers la terre.
\VS{11}[Kaf.] Mes yeux se consument à force de larmes, mes entrailles bouillonnent, ma bile se répand sur la terre. À cause des ruines de la fille de mon peuple, des enfants et des nourrissons qui tombent en défaillance dans les rues de la ville.
\VS{12}[Lamed.] Ils disaient à leurs mères : Où y a-t-il du blé et du vin ? Et ils tombaient comme morts dans les rues de la ville, comme un homme blessé à mort, ils rendaient l'âme sur le sein de leurs mères.
\VS{13}[Mem] Qui dois-je prendre à témoin ? À qui te comparer, fille de Jérusalem ? Qui pourrait t'égaler, et quelle consolation te donner, vierge, fille de Sion ? Car ta ruine est grande comme une mer : Qui pourrait te guérir\FTNT{Es. 51:19-20.} ?
\VS{14}[Nun.] Tes prophètes ont eu pour toi des visions vaines et insensées ; ils n'ont pas découvert ton iniquité, afin de détourner ta captivité ; ils t'ont prophétisé des oracles mensongers et trompeurs\FTNT{Jé. 2:8 ; Jé. 5:31 ; Jé. 14:14.}.
\VS{15}[Samech.] Tous les passants applaudissent sur toi, ils sifflent, ils secouent leur tête contre la fille de Jérusalem : Est-ce ici la ville de laquelle on disait : La parfaite en beauté, la joie de toute la terre\FTNT{Na. 3:19.} ?
\VS{16}[Pe.] Tous tes ennemis ouvrent la bouche contre toi, ils sifflent, ils grincent des dents, ils disent : Nous l'avons engloutie ! C'est ici le jour que nous attendions, nous l'avons atteint, nous le voyons !
\VS{17}[Ayin.] Yahweh a fait ce qu'il avait projeté, il a accompli sa parole qu'il avait ordonnée depuis longtemps, il a détruit sans épargner, il a fait de toi la joie de l'ennemi, il a donné de la force à tes adversaires.
\VS{18}[Tsade.] Leur cœur crie au Seigneur… Muraille de la fille de Sion, fais couler des larmes jour et nuit, comme un torrent\FTNT{Jé. 14:17.} ! Ne te donne pas de repos ; et que la prunelle de tes yeux ne se repose pas !
\VS{19}[Qof.] Lève-toi, pousse des cris dès le commencement des veilles de la nuit ! Répands ton cœur comme de l'eau en présence du Seigneur ! Lève tes mains vers lui pour l'âme de tes enfants qui meurent de faim aux coins de toutes les rues !
\VS{20}[Resh.] Vois, ô Yahweh ! Regarde qui tu as traité avec sévérité ! Les femmes n'ont-elles pas mangé leur fruit : leurs petits enfants objets de leur tendresse ? Le sacrificateur et le prophète n'ont-ils pas été tués dans le sanctuaire du Seigneur\FTNT{Lé. 26:29 ; De. 28:53 ; Jé. 19:9.} ?
\VS{21}[Shin.] Les jeunes gens et les vieillards sont couchés par terre dans les rues ; mes vierges et mes jeunes hommes sont tombés par l'épée ; tu as tué au jour de ta colère, tu as massacré sans épargner.
\VS{22}[Tav.] Tu as convié comme pour un jour solennel mes frayeurs de toutes parts. Au jour de la colère de Yahweh, il n'y a eu ni réchappé ni survivant. Ceux que j'avais langés et élevés, mon ennemi les a consumés.
\Chap{3}
\TextTitle{Jérémie partage l'affliction des siens}
\VerseOne{}[Aleph.] Je suis l'homme qui a vu l'affliction par la verge de sa fureur\FTNT{Jé. 15:15-18.}.
\VS{2}Il m'a conduit, mené dans les ténèbres, et non dans la lumière.
\VS{3}Certes c'est contre moi qu'il a tout le jour tourné et retourné sa main.
\VS{4}[Beth.] Il a fait vieillir ma chair et ma peau, il a brisé mes os\FTNT{Es. 38:13.}.
\VS{5}Il a bâti autour de moi, il m'a environné de venin et de peine.
\VS{6}Il me fait habiter dans les lieux ténèbreux, comme ceux qui sont morts depuis longtemps.
\VS{7}[Guimel.] Il a fait une cloison autour de moi, afin que je ne sorte point ; il a appesanti mes chaînes.
\VS{8}Même quand je crie et que j'élève ma voix, il rejette ma prière.
\VS{9}Il a fait un mur de pierres de taille pour fermer mes chemins, il a renversé mes sentiers.
\VS{10}[Daleth.] Il a été pour moi un ours en embuscade, un lion qui se tient dans un lieu caché\FTNT{Os. 13:8.}.
\VS{11}Il a détourné mes chemins, il m'a mis en pièces, il m'a mis dans la désolation.
\VS{12}Il a tendu son arc, et il m'a placé comme une cible pour sa flèche.
\VS{13}[He.] Il a fait entrer dans mes reins les flèches de son carquois.
\VS{14}Je suis la risée pour tout mon peuple, et leur chanson\FTNT{Ps. 69:13 ; Job. 30:9.} tout le jour.
\VS{15}Il m'a rassasié d'amertume, il m'a enivré d'absinthe.
\VS{16}[Vav.] Il a brisé mes dents avec du gravier, il m'a couvert de cendres.
\VS{17}Tellement que la paix s'est éloignée de mon âme, j'ai oublié ce que c'est que d'être à son aise.
\VS{18}Et j'ai dit : Ma force est perdue, et mon espérance aussi que j'avais en Yahweh.
\VS{19}[Zayin.] Souviens-toi de mon affliction, et de mon pauvre état qui n'est qu'absinthe et que fiel ;
\VS{20}Mon âme s'en souvient sans cesse, et elle est abattue au-dedans de moi.
\VS{21}Mais je rappellerai ceci en mon coeur, et c'est pourquoi j'aurai de l'espérance :
\VS{22}[Heth.] C'est une grâce de Yahweh que nous n'avons point été consumés parce que ses compassions ne sont pas épuisées\FTNT{Ps. 103:10.} ;
\VS{23}elles se renouvellent chaque matin. C'est une chose grande que ta fidélité !
\VS{24}Yahweh est ma portion, dit mon âme ; c'est pourquoi j'aurai espérance en lui\FTNT{Ps. 16:5.}.
\VS{25}[Teth.] Yahweh est bon pour ceux qui s'attendent à lui, pour l'âme qui le cherche.
\VS{26}Il est bon d'espérer et d'attendre en silence la délivrance de Yahweh.
\VS{27}Il est bon pour l'homme de porter le joug dans sa jeunesse.
\VS{28}[Yod.] Il sera assis solitaire et silencieux parce qu'on le lui impose.
\VS{29}Il mettra sa bouche dans la poussière, peut-être y aura-t-il quelque espérance ?
\VS{30}Il présentera la joue à celui qui le frappe, il se rassasiera d'opprobres.
\VS{31}[Kaf.] Car le Seigneur ne rejette pas à toujours\FTNT{Es. 57:16 ; Ps. 77:8.}.
\VS{32}Mais s'il afflige quelqu'un, il a aussi compassion selon la grandeur de sa miséricorde.
\VS{33}Car ce n'est pas sa volonté d'affliger et d'humilier les fils des hommes.
\VS{34}[Lamed.] Lorsqu'on foule aux pieds tous les prisonniers de la terre,
\VS{35}lorsqu'on pervertit la justice humaine en la présence du Très-Haut,
\VS{36}lorsqu'on fait tort à quelqu'un dans son procès, le Seigneur ne le voit-il pas ?
\VS{37}[Mem.] Qui est-ce qui dit qu'une chose est arrivée sans que le Seigneur l'ait commandé ?
\VS{38}Les maux et les biens\FTNT{Es. 45:7 ; Am. 3:6 ; Job. 1:21.} ne procèdent-ils pas de la bouche du Très-Haut ?
\VS{39}Pourquoi un homme vivant se plaindrait-il, un homme, à cause de la peine de ses péchés ?
\TextTitle{Le peuple appelé à s'examiner pour revenir à Yahweh}
\VS{40}[Nun.] Recherchons nos voies, sondons-les, et retournons à Yahweh\FTNT{Ps. 119:59 ; 2 Co. 13:5.} ;
\VS{41}élevons nos cœurs et nos mains vers Dieu qui est au ciel :
\VS{42}Nous avons péché, nous avons été rebelles ! Tu n'as pas pardonné !
\VS{43}[Samech.] Tu nous as couverts de ta colère, et tu nous as poursuivis ; tu as tué sans épargner ;
\VS{44}tu t'es couvert d'une nuée pour que les prières ne te parviennent pas.
\VS{45}Tu nous as fait être la raclure et le rebut au milieu des peuples.
\VS{46}[Pe.] Tous nos ennemis ouvrent leur bouche contre nous.
\VS{47}La frayeur et la fosse, le dégât et la calamité nous sont arrivés\FTNT{Es. 24:18 ; Jé. 48:44.}.
\VS{48}De mes yeux coulent des torrents d'eau à cause de la ruine de la fille de mon peuple.
\VS{49}[Ayin.] Mon œil fond en larmes, sans repos, sans relâche,
\VS{50}jusqu'à ce que Yahweh regarde et voie des cieux\FTNT{Ps. 80:15 ; Ps. 102:20.} ;
\VS{51}mon œil fait souffrir mon âme à cause de toutes les filles de ma ville.
\TextTitle{Yahweh, le soutien de Jérémie dans la détresse}
\VS{52}[Tsade.] Ceux qui sont mes ennemis sans cause m'ont poursuivi à outrance, comme après un oiseau.
\VS{53}Ils ont voulu anéantir ma vie dans une fosse, et ils ont jeté une pierre sur moi.
\VS{54}Les eaux ont coulé par-dessus ma tête ; je disais : Je suis retranché !
\VS{55}[Qof.] J'ai invoqué ton nom, ô Yahweh, du fond de la fosse\FTNT{Jé. 38:6.}.
\VS{56}Tu as entendu ma voix : Ne ferme pas tes oreilles à mes soupirs, à mes cris !
\VS{57}Au jour où je t'ai invoqué, tu t'es approché, et tu as dit : Ne crains rien !
\VS{58}[Resh.] Ô Seigneur, tu as plaidé la cause de mon âme, tu as racheté ma vie.
\VS{59}Tu as vu, ô Yahweh ! le tort qu'on me fait, fais-moi justice !
\VS{60}Tu as vu toutes les vengeances dont ils ont usé, et toutes leurs machinations contre moi.
\VS{61}[Shin.] Yahweh, tu as entendu leurs outrages, toutes leurs machinations contre moi,
\VS{62}les discours de ceux qui se lèvent contre moi, et leur dessein qu'ils ont contre moi tout au long du jour.
\VS{63}Considère quand ils sont assis et quand ils se lèvent, car je suis leur chanson.
\VS{64}[Tav.] Rends-leur la pareille, ô Yahweh, selon l'œuvre de leurs mains ;
\VS{65}livre-les à l'endurcissement de leur cœur, à ta malédiction.
\VS{66}Poursuis-les dans ta colère, et extermine-les de dessous les cieux, ô Yahweh !
\Chap{4}
\TextTitle{Crimes et apostasie du peuple}
\VerseOne{}[Aleph.] Comment l'or est-il devenu obscur, et le fin or s'est-il altéré ? Comment les pierres du sanctuaire sont-elles répandues aux coins de toutes les rues ?
\VS{2}[Beth.] Comment les chers fils de Sion, qui étaient estimés à l'égal de l'or pur, sont-ils reputés comme des vases de terre, ouvrage des mains du potier !
\VS{3}[Guimel.] Il y a même des monstres marins qui présentent leurs mamelles et allaitent leurs petits ; mais la fille de mon peuple est devenue cruelle comme les autruches du désert.
\VS{4}[Daleth.] La langue de celui qui têtait s'est attachée à son palais dans sa soif ; les enfants demandent du pain, et personne ne leur en donne\FTNT{Jé. 52:6.}.
\VS{5}[He.] Ceux qui mangeaient des mets délicats sont en désolation dans les rues ; ceux qui étaient nourris sur l'étoffe écarlate embrassent le fumier.
\VS{6}[Vav.] L'iniquité de la fille de mon peuple est plus grande que le péché de Sodome, renversée en un instant, sans que personne n'ait tourné la main sur elle.
\VS{7}[Zayin.] Ses naziréens étaient plus purs que la neige, plus blancs que le lait ; leur teint était plus vermeil que les pierres précieuses ; ils étaient polis comme un saphir.
\VS{8}[Heth.] Leur apparence est plus sombre que le noir ; on ne les reconnaît pas dans les rues ; ils ont la peau collée sur les os ; elle est devenue sèche comme du bois\FTNT{Job. 30:30.}.
\VS{9}[Teth.]Ceux qui ont été mis à mort par l'épée, ont été plus heureux que ceux qui sont morts par la famine, qui eux sont consumés peu à peu, transpercés par le défaut du fruit des champs.
\VS{10}[Yod.] Les mains des femmes, naturellement tendres, font cuire leurs enfants ; ils leur servent de nourriture dans la ruine de la fille de mon peuple\FTNT{De. 28:57 ; 2 R. 6:29.}.
\VS{11}[Kaf.] Yahweh a accompli sa fureur, il a répandu l'ardeur de sa colère ; il a allumé dans Sion un feu qui en dévore les fondements.
\VS{12}[Lamed.] Les rois de la terre, et tous les habitants de la terre habitable n'auraient jamais cru que l'adversaire et l'ennemi entrerait dans les portes de Jérusalem.
\VS{13}[Mem.] Cela est arrivé à cause des péchés de ses prophètes, et des iniquités de ses sacrificateurs, qui répandaient le sang des justes au milieu d'elle\FTNT{Jé. 5:29-31. Le péché des conducteurs donne accès à l'ennemi pour les détruire ainsi que les biens qui leur ont été confiés (Lu. 11:21-22).}.
\VS{14}[Nun.] Ils erraient comme des aveugles dans les rues, souillés de sang, au point qu'on ne pouvait pas toucher leurs vêtements.
\VS{15}[Samech.] On leur criait : retirez-vous, souillés, retirez-vous, retirez-vous, ne nous touchez point. Quand ils se sont enfuis, ils ont erré ça et là ; on a dit parmi les nations : Ils n'auront plus leur demeure !
\VS{16}[Pe.] La face de Yahweh les a dispersés, il ne veut plus les regarder ; ils n'ont pas eu de respect pour les sacrificateurs, et n'ont pas été miséricordieux envers les vieillards.
\VS{17}[Ayin.] Pour nous, nos yeux se consumaient après un vain secours ; nous regardions du haut de nos lieux élevés vers une nation qui ne pouvait pas délivrer\FTNT{Jé. 18:15.}.
\VS{18}[Tsade.] Ils ont épié nos pas afin de nous empêcher d'aller sur nos places ; notre fin s'approchait, nos jours étaient accomplis… Notre fin est arrivée !
\VS{19}[Qof.] Nos persécuteurs étaient plus légers que les aigles des cieux ; ils nous ont poursuivis sur les montagnes, ils ont mis des embûches contre nous dans le désert.
\VS{20}[Resh.] Le souffle de nos narines, l'oint de Yahweh\FTNT{L'oint en question est le roi Josias (2 R. 21:24 ; 22 ; 23).}, a été pris dans leurs fosses, celui de qui nous disions : Nous vivrons sous son ombre parmi les nations.
\VS{21}[Shin.] Réjouis-toi, sois dans l'allégresse, fille d'Edom, habitante du pays d'Uts ! La coupe passera aussi vers toi ; tu en seras enivrée, et tu seras mise à nu\FTNT{Jé. 25:15-18 ; Ps. 137:7.}.
\VS{22}[Tav.] Fille de Sion, ton iniquité est expiée ; il ne t'enverra plus en exil. Fille d'Edom, il châtiera ton iniquité, il découvrira tes péchés.
\Chap{5}
\TextTitle{Supplications de Jérémie à Yahweh}
\VerseOne{}Souviens-toi, ô Yahweh, de ce qui nous est arrivé ! Regarde et vois notre opprobre !
\VS{2}Notre héritage a été renversé par des étrangers, nos maisons par des inconnus.
\VS{3}Nous sommes devenus comme des orphelins qui sont sans pères, et nos mères sont comme des veuves.
\VS{4}Nous buvons notre eau à prix d'argent, et notre bois nous est vendu.
\VS{5}Ceux qui nous poursuivent sont sur notre cou ; nous sommes épuisés, nous n'avons pas de repos.
\VS{6}Nous avons étendu la main vers l'Egypte, et vers l'Assyrie pour nous rassasier de pain.
\VS{7}Nos pères ont péché, ils ne sont plus, et c'est nous qui portons la peine de leurs iniquités\FTNT{Chaque homme naît pécheur et hérite de la nature pécheresse d'Adam (Ge. 3:20 ; Ac. 17:26 ; Ro. 5:12-21). De ce fait, les péchés commis par les parents ont des conséquences sur les enfants (Ex. 20:4-5 ). Jésus-Christ nous a délivrés du péché d'Adam et de celui de nos ancêtres à la croix (Col. 1:12). Lors de notre naissance d'en haut, les péchés de notre passé et de nos origines sont expiés (2 Co. 5:17 ; Ep. 1:7 ; Ep. 2:1-15 ; Col. 1:12-14 ; Col. 2:13-15 ; 1 Pi. 1:18-19 ; 1 Jn. 1:7 ; 1 Jn. 1:9). Les péchés des ancêtres et leurs conséquences touchent les personnes qui vivent dans les péchés de leurs ancêtres, c'est-à-dire ceux qui haïssent Dieu et ses commandements (De. 24:16 ; Jé. 31:29-34 ; Ez. 18:17-20).}.
\VS{8}Les esclaves dominent sur nous, et personne ne nous délivre de leurs mains.
\VS{9}Nous amenons notre pain au péril de notre vie, à cause de l'épée du désert.
\VS{10}Notre peau est brûlante comme un four, à cause l'ardeur de la faim.
\VS{11}Ils ont déshonoré les femmes dans Sion, les vierges dans les villes de Juda.
\VS{12}Des chefs ont été pendus par leurs mains ; et ils n'ont pas honoré la personne des vieillards.
\VS{13}Ils ont pris les jeunes gens pour moudre, et les enfants sont tombés sous le bois.
\VS{14}Les vieillards ont cessé de se trouver aux portes, et les jeunes gens de chanter.
\VS{15}La joie a disparu de notre cœur, et notre danse est changée en deuil.
\VS{16}La couronne de notre tête est tombée ! Malheur à nous, parce que nous avons péché !
\VS{17}C'est pourquoi notre coeur est languissant. À cause de ces choses, nos yeux sont obscurcis.
\VS{18}À cause de la montagne de Sion qui est désolée ; les renards s'y promènent.
\VS{19}Toi, ô Yahweh, tu demeures éternellement, et ton trône subsiste de génération en génération.
\VS{20}Pourquoi nous oublierais-tu à jamais ? pourquoi nous délaisserais-tu si longtemps ?
\VS{21}Convertis-nous à toi, ô Yahweh ! et nous serons convertis ; renouvelle nos jours comme ils étaient autrefois\FTNT{Jé. 30:20 ; Jé. 31:18 ; Ps. 80:3.}.
\VS{22}Ou bien, nous aurais-tu entièrement rejetés ? Serais-tu extrêmement courroucé contre nous ?
\PPE{}
\end{multicols}

\clearpage%& -output-directory="../../pdf/books/"
% type document & taille police
\documentclass[11pt]{book}
% package format document
\usepackage[paperwidth=6.5in, paperheight=9.05in, top=0in, bottom=0in, left=0in, right=0in]{geometry}
% formatage marges, etc.
\setlength{\voffset}{-0.7in} % offset haut
%\setlength{\hoffset}{-0.3in} % offset gauche
\setlength{\topmargin}{0in} % marge en tête
\setlength{\headsep}{0.2in} % marge header/body
\setlength{\oddsidemargin}{-0.5in} % marge texte gauche
\setlength{\evensidemargin}{-0.5in} % marge texte droite
\setlength{\textheight}{8in} % hauteur du texte
\setlength{\textwidth}{5.5in} % largeur du texte
\setlength{\columnseprule}{0.4pt} % épaisseur séparateur colonne
\setlength{\parskip}{0pt} % espace entre paragraphes
% package pour afficher les cadres
%\usepackage{showframe}
% package langue
\usepackage[francais]{babel}
% package polices système
\usepackage{fontspec}
% définition police
\setmainfont[Ligatures=TeX,Scale=0.95]{Liberation Serif}
\setsansfont{Liberation Sans}
\setmonofont{Liberation Mono}
% package parskip pour espaces entre paragraphes
\usepackage{parskip}
% package multicolonne
\usepackage{multicol}
% package liens cliquables
\usepackage[xetex]{hyperref}
% package inclusion copyright (dépandant de hyperref)
\usepackage{hyperxmp}
% copyright
\hypersetup{
    pdfauthor = {ANJC Productions},
    pdftitle = {Bible de Jésus-Christ},
    pdfkeywords = {BJC, Bible, Jesus},
    pdfcopyright = {ANJC Productions. Distribution et Diffusion Libre - Pas d'Utilisation Commerciale - Pas de Dénaturation de l'Œuvre - International},
    pdflicenseurl = {http://www.bible-de-jesus.org/}
}
% ???
\setcounter{collectmore}{-1}
% style
\pagestyle{myheadings}
% ???
\sloppy\hyphenpenalty=2000
% titres de livres
\newcommand{\ShortTitle}[1]{\def\webbook{#1}\par\goodbreak\bigskip\setcounter{footnote}{0}}
\newcommand{\BookTitle}[1]{\par\goodbreak\bigskip{\parindent=0mm\begin{center}{\small\bfseries{\LARGE #1\nopagebreak}}\end{center}}\addcontentsline{toc}{subsection}{#1}\nopagebreak\par\nobreak}
% chapitres
\newcommand{\Chap}[1]{\def\webchap{#1:}\def\webvs{0}\def\vchap{#1}\ssubsection{\centerline{\textbf{{CHAPITRE\ #1}}}}}
% versets
\newcommand{\VerseOne}{\def\webvs{1}{\up{\footnotesize 1}}\markboth{\webbook\ \webchap 1}{\webbook\ \webchap 1}}
\newcommand{\VS}[1]{\def\webvs{#1}{\up{\footnotesize #1}}\markboth{\webbook\ \webchap #1}{\webbook\ \webchap #1}}
\newcommand{\vref}[1]{\NoAutoSpaceBeforeFDP{#1}}
% commentaires
%\interfootnotelinepenalty=10000 % longueur max commentaires
\renewcommand{\thefootnote}{\alph{footnote}} % repères alphabetiques
\renewcommand{\footnoterule}{\hrule width \textwidth} % longueur ligne
\newcommand{\FTNT}[1]{\ifnum\value{footnote}>25\setcounter{footnote}{0}\fi\footnote{[\webchap\webvs]\ #1}}
\newcounter{webvst}
\newcommand{\FTNTT}[1]{
    \ifnum\value{footnote}>25\setcounter{footnote}{0}\fi
    \setcounter{webvst}{\webvs}\addtocounter{webvst}{1}
    \footnote{[\webchap\thewebvst]\ #1}
}
% titres de paragraphes
\newcommand{\ssubsection}[1]{\subsection*{\centering\footnotesize\normalfont #1}}
\newcommand{\ssubsubsection}[1]{\subsubsection*{\centering\footnotesize\normalfont #1}}
\newcommand{\TextTitle}[1]{\ssubsubsection{[\textit{#1}]}}
% dictionnaire
\newcommand{\DicoEntry}[1]{\smallskip\parindent=0mm{\textbf{#1}}\markboth{#1}{#1}}
% commandes diverses
\newcommand{\BFont}{\normalfont\small}
\newcommand{\PP}{\par\parindent=0mm}
\newcommand{\PPE}{\par\parindent=4mm}
% debut document
\begin{document}
% en-tête vide
\makeatletter
    \def\@evenhead{}
    \def\@oddhead{}
\makeatother
% inclusion intro
%\begin{center}{\LARGE Introduction}\end{center}
\begin{small}
\subsection*{Pourquoi cette Bible révisée~?}

En novembre 2013, alors que j'étais en prière, je demandais au Seigneur ce qu'il attendait de moi. Ce dernier m'a répondu à travers plusieurs songes dans lesquels il me disait de réviser la Bible. Je dois dire que j'ai eu du mal à croire que Dieu puisse me demander une telle chose. De plus, je me sentais incapable d'assumer un si grand projet, aussi je lui ai demandé à plusieurs reprises de me confirmer que c'était bien sa volonté, chose qu'il a faite. J'ai ensuite parlé de ce que j'avais reçu à des frères et sœurs qui travaillent avec moi et ces derniers m'ont confirmé que cette vision venait bien du Seigneur. Une dynamique s'est créée aussitôt et bien qu'aucun d'entre nous ne se sentît à la hauteur de la tâche qui nous était confiée, nous nous sommes rapidement organisés pour concrétiser cette vision, comptant sur le Seigneur pour qu'il nous donne les capacités et la sagesse dont nous avions besoin.\bigskip

Deux constats majeurs nous ont amenés à la conclusion qu'une révision de la Bible était plus que nécessaire. Tout d'abord, la plupart des bibles modernes les plus diffusées sont basées sur le texte minoritaire comportant une quantité importante de fautes de traduction, d'omissions et de rajouts qui altèrent la compréhension du message et induisent par conséquent le lecteur en erreur. Or il est du devoir de tout chrétien de mettre en pratique la Parole, notamment en veillant sur son authenticité.\bigskip

«~\emph{Car, je vous le dis en vérité, tant que le ciel et la terre ne passeront point, il ne disparaîtra pas de la loi un seul iota ou un seul trait de lettre jusqu'à ce que tout soit arrivé. Celui donc qui aura violé l'un de ces petits commandements, et qui aura enseigné les hommes à faire de même, sera appelé le plus petit au Royaume des cieux~; mais celui qui les observera, et qui enseignera à les observer, celui-là sera appelé grand au Royaume des cieux.}~» Matthieu 5:18-19.\bigskip

«~\emph{Je le déclare à quiconque entend les paroles de la prophétie de ce livre~: Si quelqu'un y ajoute quelque chose, Dieu le frappera des fléaux décrits dans ce livre. Et si quelqu'un retranche quelque chose des paroles du livre de cette prophétie, Dieu retranchera sa part de l'arbre de vie, et de la ville sainte, décrits dans ce livre.}~» Apocalypse 22:18-19.\bigskip

Nous ne devons pas oublier que la Bible a été initialement écrite en trois langues, à savoir l'hébreu, le grec et quelques versets en araméen. En réalisant cette révision, notre but est de restituer le sens des mots d'origine et d'expurger toute l'influence de l'ennemi. Ce travail a permis de mettre en lumière une évidence~: la personne de Jésus-Christ occupe une place centrale de Genèse à Apocalypse, ce qui ne fait que confirmer et attester sa divinité.\bigskip

«~\emph{Puis il leur dit~: C'est là ce que je disais lorsque j'étais encore avec vous, qu'il fallait que s'accomplisse tout ce qui est écrit de moi dans la loi de Moïse, dans les prophètes, et dans les psaumes.}~» Luc 24:44.\bigskip

Ensuite, nous déplorons le fait que la majorité des bibles en circulation soient vendues alors que Jésus-Christ a dit «~\emph{Vous l'avez reçu gratuitement, donnez-le gratuitement}~» (Mt. 10:8). Il est donc impensable que celui qui a chassé du temple vendeurs et changeurs puisse approuver un seul instant le commerce qui est fait avec sa Parole (Jn. 2:14-16).\bigskip

«~\emph{Vous tous qui avez soif, venez aux eaux, et vous qui n'avez pas d'argent, venez, achetez et mangez~; venez, dis-je, achetez du vin et du lait sans argent et sans rien payer~!}~» Esaïe 55:1.\bigskip

«~\emph{Il me dit aussi~: Tout est accompli. Je suis l'Alpha et l'Oméga, le commencement et la fin. A celui qui a soif, je lui donnerai de la source d'eau vive, gratuitement.}~» Apocalypse 21:6.\bigskip

«~\emph{Et l'Esprit et l'épouse disent~: Viens. Et que celui qui entend dise~: Viens. Et que celui qui a soif vienne~; que celui qui veut prenne gratuitement de l'eau de la vie.}~» Apocalypse 22:17.\bigskip

Les apôtres ont scrupuleusement respecté l'ordre du Seigneur en Matthieu 10:8. Pierre a dénoncé avec la plus grande sévérité Simon, le magicien qui avait eu la folie de croire que le don de Dieu pouvait être monnayé. Et durant tout son service, Paul a enseigné l'Evangile gratuitement.\bigskip

«~\emph{Puis ils leur imposèrent les mains, et ils reçurent le Saint-Esprit. Lorsque Simon vit que le Saint-Esprit était donné par l’imposition des mains des apôtres, il leur présenta de l’argent, en leur disant : Donnez-moi aussi ce pouvoir, afin que tous ceux à qui j’imposerai les mains reçoivent le Saint-Esprit. Mais Pierre lui dit~: Que ton argent périsse avec toi, puisque tu as estimé que le don de Dieu s’acquérait avec de l’argent. Tu n’as point de part ni d’héritage en cette affaire ; car ton coeur n’est point droit devant Dieu. Repens-toi donc de cette méchanceté, et prie Dieu, afin que, s’il est possible, la pensée de ton coeur te soit pardonnée. Car je vois que tu es dans un fiel très amer et dans un lien d’iniquité.}~» Actes 8:17-23.\bigskip

«~\emph{Je n'ai désiré ni l'argent, ni l'or, ni les vêtements de personne.}~» Actes 20:33.\bigskip

«~\emph{Quelle récompense en ai-je donc~? C’est qu’en prêchant l’Evangile, je prêche l’Evangile de Christ sans qu’il en coûte rien, afin que je n’abuse pas de mon pouvoir dans l’Evangile.}~» 1 Corinthiens 9:18.\bigskip

Nous pensons qu'il est juste et honnête que la Bible porte le nom de son véritable auteur et qu'elle soit gratuitement diffusée selon sa volonté et l'ordre clair qu'il a donné. Cette Bible s'appelle donc La Bible de Jésus-Christ et est gratuitement mise à la disposition de ceux qui souhaitent se la procurer.\bigskip

\subsection*{Comment a été réalisée cette révision~?}

Pour réaliser cette révision, nous nous sommes appuyés sur le texte majoritaire (originaux et traductions). Ainsi, tout en essayant de conserver un vocabulaire qui soit à la portée de tous, certains mots et expressions ont été changés pour restituer pleinement leur signification initiale. A titre d'exemple, vous constaterez régulièrement que certains mots sont répétés deux fois de suite. Cela n'est pas une erreur mais la restitution littérale de certaines expressions qui insistent sur une vérité (voir commentaire en Gn. 2:15-17). En effet, Dieu parle une fois et une seconde fois pour avertir les hommes (Job. 33:14). «~Et quant à ce que le songe a été réitéré à Pharaon pour la seconde fois, c'est que la chose est arrêtée de la part de Dieu, et que Dieu se hâtera de l'exécuter~» (Genèse 41:32).\bigskip

Les Ecrits ont été classés dans l'ordre de la tradition juive pour le Tanakh et dans l'ordre chronologique de leur rédaction pour les épîtres afin de permettre au lecteur de mieux comprendre le contexte et le déroulement de la prophétie biblique. L'appellation «~Ancien Testament~» a été remplacée par l'acronyme hébreu Tanakh (voir sommaire). Quant à ce qu'on appelle communément le «~Nouveau Testament~», il sera désormais question du Testament de Jésus. En effet, l'Ancienne Alliance n'étant pas un testament, on ne peut donc pas parler de «~Nouveau Testament~» mais plutôt d'une Nouvelle Alliance (voir commentaires en Ex. 19:5~; Mt. 27:51~; Jn. 19:30).\bigskip

Je remercie tout d'abord le Seigneur pour son aide précieuse qu'il m'a apportée pour la révision de cette Bible, ainsi qu'à celles et ceux qui m’ont assisté dans ce travail.\newline

\begin{flushright}
Shora KUETU
\end{flushright}
\end{small}

%% formatage sommaire
%\makeatletter
%\renewcommand\tableofcontents{
%    \begin{center}{\LARGE Sommaire}\end{center}
%    \setlength{\columnseprule}{0pt} % désactivation séparateur colonne temp
%    \begin{multicols}{2}\medskip\footnotesize{\@starttoc{toc}}\end{multicols}
%    \setlength{\columnseprule}{0.4pt} % réactivation séparateur colonne
%}
%\makeatother
%% inclusion table des matières
%\clearpage\tableofcontents\clearpage
%% en-tête pages
%\makeatletter
\def\@evenhead{{\NoAutoSpaceBeforeFDP{\small{\rightmark\hfil\thepage\hfil\leftmark}}}}
\def\@oddhead{{\NoAutoSpaceBeforeFDP{\small{\rightmark\hfil\thepage\hfil\leftmark}}}}
\makeatother
% inclusion des livres
\addcontentsline{toc}{chapter}{Tanakh}\pagenumbering{arabic}\clearpage
%\addcontentsline{toc}{section}{Torah (Loi)}\clearpage
%\clearpage\ShortTitle{Genèse}\BookTitle{Genèse}\BFont
\noindent\hrulefill
{\footnotesize
\textit{
\bigskip
{\centering{}
\\Auteur : Probablement Moïse
\\(Heb. : Bereshit)
\\Signification : Au commencement
\\Thème : Le Messie d'Israël
\\Date de rédaction : Env. 1450-1410 av. J.-C.\\}
}
%\bigskip
\textit{
\\Premier livre du Tanakh, la Genèse est le livre des commencements.
Elle relate l’histoire des origines de l’humanité, la création des cieux, de la terre et de tout ce qui s’y trouve par Yahweh, le Dieu créateur.
%\bigskip
\\Il y est décrit le péché de l’homme et sa séparation d’avec Dieu, ainsi que la décadence de l’univers qui en résulta. En réponse à la méchanceté du cœur de l’homme, Yahweh  exerça sa justice en détruisant la terre par le déluge.
Dans sa prescience, Yahweh avait cependant résolu de se réconcilier avec l’homme. Il se révéla donc comme sauveur en accordant sa grâce à Noé et à sa famille. Après cet événement, les hommes se tournèrent une fois de plus vers le mal en tentant Dieu par la construction de la tour de Babel, œuvre à l’origine de la dispersion des nations.
%\bigskip
\\Ce livre présente aussi l’élection d’Abraham, originaire d’Ur en Chaldée - actuelle Mésopotamie - qui reçut la promesse divine de devenir une grande nation, en qui toutes les familles de la terre seraient bénies. Le récit se poursuit par l’histoire de ses descendants Isaac, Jacob et ses douze fils,  qui formèrent par la suite la nation d’Israël.\bigskip
}
}
\par\nobreak\noindent\hrulefill
\begin{multicols}{2}
\Chap{1}
\VerseOne{}Au commencement, Dieu créa les cieux et la terre.
\TextTitle{La terre devient informe et vide}
\VS{2}Et la terre devint informe et vide\FTNT{Les termes «~informe~» et «~vide~» viennent des mots hébreux «~tohuw~» et «~bohuw~»  qui désignent la confusion, le chaos, la vanité.}, les ténèbres étaient à la surface de l'abîme ; et l'Esprit de Dieu se mouvait au-dessus des eaux.
\TextTitle{Jour «~un~» : Apparition de la lumière}
\VS{3}Dieu dit : Que la lumière apparaisse\FTNT{«~Que la lumière apparaisse !~» (Es. 9:1 ; Mt. 4:16 ; Jn. 1:1-5). Cette lumière n’est autre que Yahweh lui-même qui va s’incarner en la personne de Jésus-Christ pour chasser les ténèbres (2 S. 22:9-12 ; Es. 60:1 ; 60:19-20 ; Jn. 1:1-2 ; 8:12-14 ; 2 Co. 4:6).} ; et la lumière apparut.
\VS{4}Et Dieu vit que la lumière était bonne ; et Dieu sépara la lumière des ténèbres.
\VS{5}Dieu appela la lumière jour, et il appela les ténèbres nuit\FTNT{La lumière et les ténèbres, ainsi que leurs champs lexicaux respectifs, personnifient  souvent Jésus et Satan.  Ainsi, Jésus est la Lumière du monde (Jn. 9:5), l’Etoile brillante du matin (Ap. 22:16), le Soleil levant ou le Soleil de la justice (Mal. 4:2 ; Ps. 19:6 ; Lu. 1:78). Il est associé au jour (Jn. 9:4) d’où les expressions  «~Jour du Seigneur~» (1 Th. 5:2) ou «~Jour de Yahweh~» (Joë. 1:15). A l’inverse, la Bible associe Satan aux ténèbres (Es. 8:23 ; Ps. 143:3 ; Ep. 6:12 ; Col. 1:13) et à la nuit (Jn. 9:4 ; Ro. 13:12).}. Ainsi fut le soir, ainsi fut le matin\FTNT{Contrairement au calendrier grégorien où le jour commence à minuit, selon Dieu et le calendrier hébraïque, le jour commence le soir à 18 heures pour se terminer le lendemain à la même heure. Voir commentaire en Mc. 16:9.} ; ce fut le  jour un\FTNT{L’hébreu utilise le terme «~ehad~» qui signifie «~un~», au sens de l’indivisible, pour qualifier le premier jour. Ce jour nous parle de Yahweh tel qu’il s’est présenté à son peuple sur le mont Sinaï en De. 6:4:«~Shema Yisrael Yahweh elohénou Yahweh ehad~» («~écoute Israël, Yahweh [est] notre Dieu, Yahweh [est] UN~»). Un n’est pas divisible sinon on obtient un zéro ce qui équivaut au néant. Dieu est tout sauf le néant, il remplit tout (Ep. 1:23), il est partout (Ps. 139:7-13), les cieux des cieux ne peuvent le contenir (1 R. 8:27).}.
\TextTitle{Second jour : Une étendue entre les eaux}
\VS{6}Puis Dieu dit : Qu'il y ait une étendue entre les eaux, et qu'elle sépare les eaux d'avec les eaux.
\VS{7}Dieu donc fit l'étendue, et il sépara les eaux qui sont au-dessous de l'étendue d'avec celles qui sont au-dessus de l'étendue, et il fut ainsi.
\VS{8}Et Dieu appela l'étendue cieux. Ainsi fut le soir, ainsi fut le matin ; ce fut le second jour.
\TextTitle{Troisième jour : Les mers, la terre et la végétation}
\VS{9}Puis Dieu dit : Que les eaux qui sont au-dessous des cieux soient rassemblées en un lieu, et que le sec paraisse ; et il fut ainsi.
\VS{10}Et Dieu appela le sec terre ; et il appela l'amas des eaux mers ; et Dieu vit que cela était bon.
\VS{11}Puis Dieu dit : Que la terre produise de la verdure, de l'herbe portant de la semence, et des arbres fruitiers portant du fruit selon leur espèce, qui aient leur semence en eux-mêmes sur la terre ; et il fut ainsi.
\VS{12}La terre donc produisit de la verdure, de l'herbe portant de la semence selon son espèce ; et des arbres portant du fruit qui avaient leur semence en eux-mêmes selon leur espèce ; et Dieu vit que cela était bon.
\VS{13}Ainsi fut le soir, ainsi fut le matin ; ce fut le troisième jour.
\TextTitle{Quatrième jour : Les luminaires du ciel}
\VS{14}Puis Dieu dit : Qu'il y ait des luminaires dans l'étendue du ciel pour séparer la nuit d'avec le jour, et qui servent de signes pour les saisons, pour les jours, et pour les années ;
\VS{15}et qu’ils servent de luminaires dans l'étendue du ciel afin d'éclairer la terre ; et il fut ainsi.
\VS{16}Dieu donc fit deux grands luminaires, le plus grand luminaire pour présider au jour, et le plus petit luminaire pour présider à la nuit ; il fit aussi les étoiles.
\VS{17}Dieu les plaça dans l'étendue du ciel pour éclairer la terre,
\VS{18}pour présider au jour et à la nuit, et pour séparer la lumière d’avec les ténèbres ; et Dieu vit que cela était bon.
\VS{19}Ainsi fut le soir, ainsi fut le matin ; ce fut le quatrième jour.
\TextTitle{Cinquième jour : Les animaux vivant dans les eaux et les airs\FTNTT{ Ge. 2:19}}
\VS{20}Puis Dieu dit : Que les eaux produisent en toute abondance des reptiles vivants ; et qu'il y ait des oiseaux qui volent sur la terre vers l'étendue du ciel.
\VS{21}Dieu créa les grands poissons et tous les animaux vivants qui se meuvent et que les eaux produisirent en toute abondance selon leur espèce ; il créa aussi tout oiseau ayant des ailes selon son espèce ; et Dieu vit que cela était bon.
\VS{22}Dieu les bénit en disant : Soyez féconds, multipliez, et remplissez les eaux des mers ; et que les oiseaux multiplient sur la terre.
\VS{23}Ainsi fut le soir, ainsi fut le matin ; ce fut le cinquième jour.
\TextTitle{Sixième jour : Les animaux terrestres}
\VS{24}Puis Dieu dit : Que la terre produise des animaux selon leur espèce, le bétail, les reptiles, et les bêtes de la terre selon leur espèce ; et il fut ainsi.
\VS{25}Dieu donc fit les animaux de la terre selon leur espèce, et le bétail selon son espèce, et les reptiles de la terre selon leur espèce ; et Dieu vit que cela était bon.
\TextTitle{Mission confiée à l’homme ; son autorité sur la création}
\VS{26}Puis Dieu dit : Faisons l'homme à notre image, selon notre ressemblance\FTNT{L’image de Dieu n’est autre que Jésus-Christ lui-même (Col. 1:15). Adam, qui signifie terrien, a été créé à  l’image du dernier Adam (1 Co. 15:40-49) qui est venu comme Fils, afin de nous montrer le modèle de fils et de filles que Dieu souhaite (Ro. 8:29). Nous avons ici une autre image de l’incarnation de Dieu en la personne de Jésus-Christ. Ainsi, avant que l’homme ne pèche, le projet de la rédemption était déjà là (1 Pi. 1:19-21).}, et qu'il domine sur les poissons de la mer, sur les oiseaux du ciel, sur le bétail, sur toute la terre, et sur tout reptile qui rampe sur la terre.
\VS{27}Dieu créa l'homme à son image, il le créa à l'image de Dieu, il les créa mâle et femelle.
\TextTitle{Autorité de l'homme sur la création}
\VS{28}Dieu les bénit et leur dit : Soyez féconds, multipliez, remplissez la terre, et assujettissez-la ; et dominez sur les poissons de la mer, sur les oiseaux du ciel, et sur toute bête qui se meut sur la terre.
\VS{29}Et Dieu dit : Voici, je vous donne toute herbe portant de la semence qui est sur toute la terre, et tout arbre ayant en lui du fruit d'arbre et portant de la semence, ce sera votre nourriture.
\VS{30}Et à tout animal de la terre, à tout oiseau du ciel, et à tout ce qui se meut sur la terre, ayant en soi un souffle de vie, je donne toute herbe verte pour nourriture. Et cela fut ainsi.
\VS{31}Dieu vit tout ce qu'il avait fait, et voici cela était très bon. Ainsi fut le soir, ainsi fut le matin ; ce fut le sixième jour.
\Chap{2}
\TextTitle{Septième jour : Le sabbat}
\VerseOne{}Les cieux donc et la terre furent achevés, avec toute leur armée.
\VS{2}Dieu acheva au septième jour son œuvre qu'il avait faite, et il se reposa au septième jour de toute son œuvre qu'il avait faite.
\VS{3}Dieu bénit le septième jour, et le sanctifia, parce qu'en ce jour-là il s'était reposé de toute son œuvre qu'il avait créée en la faisant.
\VS{4}Telles sont les origines des cieux et de la terre, lorsqu'ils furent créés.
\VS{5}Lorsque Yahweh Dieu fit la terre et les cieux, aucun arbuste des champs n’était encore sur la terre, et aucune herbe des champs ne germait encore ; car Yahweh Dieu n'avait pas fait pleuvoir sur la terre, et il n'y avait point d'homme pour cultiver la terre.
\TextTitle{Yahweh forme l’homme et le place en Eden\FTNTT{Job 10:8-9 ; Ps. 119:73}}
\VS{6}Et il monta une vapeur de la terre qui arrosa toute la surface de la terre.
\VS{7}Yahweh Dieu forma l'homme de la poussière de la terre, et il souffla dans ses narines un souffle de vie ; et l'homme devint une âme vivante.
\VS{8}Aussi Yahweh Dieu planta un jardin en Eden, du côté de l’orient, et il y mit l'homme qu'il avait formé.
\TextTitle{Description du jardin en Eden\FTNTT{Ge. 1:28-3:6}}
\VS{9}Yahweh Dieu fit germer de la terre des arbres de toute espèce, agréables à voir et bons à manger, et l'arbre de la vie au milieu du jardin, et l'arbre de la connaissance du bien et du mal.
\VS{10}Un fleuve sortait d'Eden pour arroser le jardin ; et de là il se divisait en quatre bras.
\VS{11}Le nom du premier est Pischon ; c'est le fleuve qui coule en entourant tout le pays de Havila où se trouve l'or.
\VS{12}L'or de ce pays est bon ; c'est là aussi que se trouvent le bdellium et la pierre d'onyx.
\VS{13}Le nom du second fleuve est Guihon ; c'est celui qui coule en entourant tout le pays de Cusch.
\VS{14}Le nom du troisième fleuve est Hiddékel, qui coule vers l'Assyrie ; et le quatrième fleuve est l'Euphrate.
\TextTitle{Commandement donné par Yahweh à l’homme\FTNTT{Ge. 1:28}}
\VS{15}Yahweh Dieu prit donc l'homme et le mit dans le jardin d'Eden pour le cultiver et pour le garder.
\VS{16}Puis Yahweh Dieu donna cet ordre à l'homme, en disant : Tu mangeras, tu mangeras\FTNT{En hebreu, le mot «akal» signifie «~manger~», «~se nourrir~», «~goûter~», «~jouir~», «~dévorer~», «~consumer~», et il a été utilisé deux fois de suite dans ce passage.} de tout arbre du jardin.
\VS{17}Mais quant à l'arbre de la connaissance du bien et du mal, tu n'en mangeras point, car le jour où tu en mangeras, tu mourras, tu mourras\FTNT{Dans la plupart des versions ce passage est mal traduit par «~tu mourras certainement~», alors que le terme mort en hébreu «~muwth~» est utilisé deux fois dans ce passage, et les écritures nous parlent de la mort physique et de la seconde mort, qui est le lac de feu (Ap. 2:11 ; Ap. 20:6,14). La mort physique précède la seconde physique.}.
\TextTitle{Yahweh forme une femme pour l'homme\FTNTT{Ge. 1:27}}
\VS{18}Yahweh Dieu dit : Il n'est pas bon que l'homme soit seul ; je lui ferai une aide semblable à lui.
\VS{19}Car Yahweh Dieu forma de la terre tous les animaux des champs et tous les oiseaux du ciel, puis il les fit venir vers Adam pour voir comment il les nommerait, et afin que le nom qu'Adam donnerait à tout animal fût, son nom.
\VS{20}Et Adam donna des noms à tout le bétail, et aux oiseaux du ciel, et à tous les animaux des champs ; mais pour Adam, il ne trouva point d'aide semblable à lui.
\VS{21}Et Yahweh Dieu fit tomber un profond sommeil sur Adam, qui s'endormit ; et Dieu prit une de ses côtes, et referma la chair à la place de cette côte.
\VS{22}Yahweh Dieu forma une femme de la côte qu'il avait prise d'Adam, et il l’amena vers Adam\FTNT{1 Co. 11:8.}.
\TextTitle{Union d’Adam et Eve}
\VS{23}Alors Adam dit : Voici cette fois celle qui est os de mes os et chair de ma chair ; on l’appellera femme, parce qu'elle a été prise de l'homme.
\VS{24}C'est pourquoi l'homme quittera son père et sa mère et s’attachera à sa femme, et ils deviendront une seule chair\FTNT{Ep. 5:30-31 ; Mt. 19:5 ; Mc. 10:7 ; 1 Co. 6:16.}.
\VS{25}Adam et sa femme étaient tous deux nus, et ils n’en avaient pas honte.
\Chap{3}
\TextTitle{Séduction du serpent et chute de l’homme}
\VerseOne{}Or le serpent\FTNT{Satan ou le serpent ancien (Ap. 12:9 ; Ap. 20:2).} était le plus prudent\FTNT{La prudence, la ruse, la subtilité du serpent, sont marquées dans l'Ecriture comme des qualités qui le distinguent des autres animaux (Mt. 10:16).} de tous les animaux des champs que Yahweh Dieu avait faits ; et il dit à la femme : Quoi ! Dieu a dit : Vous ne mangerez pas de tous les arbres du jardin ?
\VS{2}La femme répondit au serpent : Nous mangeons du fruit des arbres du jardin ;
\VS{3}mais quant au fruit de l'arbre qui est au milieu du jardin, Dieu a dit : Vous n'en mangerez point, et vous ne le toucherez point, de peur que vous ne mouriez.
\VS{4}Alors le serpent dit à la femme : Vous ne mourrez nullement ;
\VS{5}mais Dieu sait que le jour où vous en mangerez, vos yeux seront ouverts, et vous serez comme des dieux, connaissant le bien et le mal.
\VS{6}La femme donc voyant que le fruit de l'arbre était bon à manger et agréable à la vue, et que cet arbre était désirable pour donner de la science ; elle prit de son fruit, et en mangea, et elle en donna aussi à son mari qui était auprès d’elle, et il en mangea.
\TextTitle{La connaissance du bien et du mal}
\VS{7}Les yeux de tous les deux s’ouvrirent, ils connurent qu'ils étaient nus, et ils cousirent ensemble des feuilles de figuier, et s'en firent des ceintures.
\VS{8}Alors ils entendirent au vent du jour la voix de Yahweh Dieu qui se promenait par le jardin ; et Adam et sa femme se cachèrent loin de la face de Yahweh Dieu, au milieu des arbres du jardin.
\VS{9}Mais Yahweh Dieu appela Adam et lui dit : Où es-tu ?
\VS{10}Il répondit : J'ai entendu ta voix dans le jardin, et j'ai eu peur parce que je suis nu, et je me suis caché.
\VS{11}Et Dieu dit : Qui t'a appris que tu es nu ? Est-ce que tu as mangé du fruit de l'arbre dont je t'avais défendu de manger ?
\VS{12}Adam répondit : La femme que tu m'as donnée pour être avec moi m'a donné du fruit de l'arbre, et j'en ai mangé.
\VS{13}Et Yahweh Dieu dit à la femme : Pourquoi as-tu fait cela ? Et la femme répondit : Le serpent m'a séduite, et j'en ai mangé.
\TextTitle{La création soumise à la vanité\FTNTT{Ro. 8:20-22}}
\VS{14}Alors Yahweh Dieu dit au serpent : Parce que tu as fait cela, tu seras maudit entre tout le bétail et entre tous les animaux des champs ; tu marcheras sur ton ventre, et tu mangeras la poussière\FTNT{La poussière dont il est question n’est autre que l’homme pécheur (Ge. 3:19). Satan ne peut rien contre les véritables enfants de Dieu (Mt. 16:18 ; Lu. 10:19).} tous les jours de ta vie.
\VS{15}Je mettrai inimitié entre toi et la femme\FTNT{La femme représente en premier lieu Eve, la mère de tous les hommes. Ici, elle représente aussi Israël, l’épouse de Yahweh selon Ge. 37:5-11 et Ap. 12:1.}, et entre ta postérité\FTNT{La postérité du serpent regroupe l’homme impie (2 Th. 2:3-4 ; 1 Jn. 2:18-22), et tous ceux qui n’ont pas reçu Jésus-Christ comme Seigneur et Sauveur. En effet, seuls ceux qui ont reçu Jésus dans leur vie sont appelés enfants de Dieu (Jn. 1:12 ; 1 Jn. 3:8-10 ; 1 Jn. 5:19).} et sa postérité\FTNT{La postérité de la femme regroupe Jésus-Christ homme (Es. 7:14 ; Lu. 2:4-7), et l’Église, le Corps de Christ (Col. 1:24).} ; celle-ci te brisera la tête, et tu lui blesseras le talon.
\VS{16}Et il dit à la femme : J'augmenterai beaucoup la souffrance de tes grossesses ; tu enfanteras dans la douleur tes enfants ; tes désirs se porteront vers ton mari, et il dominera sur toi.
\VS{17}Puis il dit à Adam : Parce que tu as obéi à la parole de ta femme, et que tu as mangé le fruit de l'arbre au sujet duquel je t'avais donné cet ordre, en disant : Tu n'en mangeras point, la terre sera maudite à cause de toi ; tu en mangeras les fruits dans la peine, tous les jours de ta vie.
\VS{18}Et elle te produira des épines et des chardons ; et tu mangeras l'herbe des champs.
\VS{19}C’est à la sueur de ton visage que tu mangeras du pain, jusqu'à ce que tu retournes dans la terre, d’où tu as été pris ; car tu es poussière, et tu retourneras dans la poussière.
\TextTitle{L’homme et la femme revêtu de tuniques de peaux}
\VS{20}Et Adam appela sa femme Eve, parce qu'elle a été la mère de tous les vivants.
\VS{21}Yahweh Dieu fit à Adam et à sa femme des tuniques de peaux, et il les en revêtit.
\VS{22}Yahweh Dieu dit : Voici, l'homme est devenu comme l'un de nous, connaissant le bien et le mal. Mais maintenant il faut prendre garde, qu’il n’avance sa main, et aussi qu’il ne prenne de l'arbre de vie, et qu’il n’en mange, et ne vive éternellement.
\TextTitle{L’homme chassé du jardin}
\VS{23}Et Yahweh Dieu le chassa du jardin d’Eden pour qu’il cultive la terre d’où il avait été pris.
\VS{24}C’est ainsi qu’il chassa l'homme, et il mit à l’orient du jardin d’Eden des chérubins qui tournent ça et là une épée flamboyante pour garder le chemin de l'arbre de vie.
\Chap{4}
\TextTitle{La jalousie de Caïn contre son frère Abel}
\VerseOne{}Adam connut Eve sa femme ; elle conçut, et enfanta Caïn ; et elle dit : J'ai acquis un homme de par Yahweh.
\VS{2}Elle enfanta encore Abel, son frère ; et Abel fut berger, et Caïn laboureur.
\VS{3}Or, au bout de quelque temps, Caïn offrit à Yahweh une offrande des fruits de la terre\FTNT{Caïn était du diable, il est l’archétype du religieux qui pense pouvoir être sauvé par les œuvres (Lu. 11:51 ; 1 Jn. 3:12). Son offrande fut rejetée car il avait apporté devant Dieu le fruit de la terre qui avait été maudite (Ge. 3:17).  Cela revenait à offrir à Dieu le péché, la malédiction.} ;
\VS{4}et Abel, de son côté, offrit des premiers-nés de son troupeau, et de leur graisse\FTNT{Abel était juste et pieux, aussi il sut instinctivement apporter une offrande agréable à Dieu (Mt. 23:35 ; Lu. 11:51 ; Hé. 11:4). En l’occurrence, son offrande préfigurait le sacrifice du Seigneur.}. Yahweh eut égard à Abel, et à son offrande.
\VS{5}Mais il n'eut point d'égard à Caïn ni à son offrande ; et Caïn fut fort irrité, et son visage fut abattu.
\TextTitle{Yahweh avertit Caïn}
\VS{6}Et Yahweh dit à Caïn : Pourquoi es-tu irrité, et pourquoi ton visage est-il abattu ?
\VS{7}Si tu agis bien, tu relèveras ton visage, et si tu agis mal, le péché est couché à la porte, et ses désirs se portent vers toi ;  mais toi, domine sur lui.
\TextTitle{Caïn tue son frère Abel\FTNTT{Ge. 4:23}}
\VS{8}Et Caïn parla avec Abel son frère, et comme ils étaient dans les champs, Caïn se jeta sur Abel, son frère, et le tua.
\VS{9}Yahweh dit à Caïn : Où est Abel ton frère ? Et il lui répondit : Je ne sais, suis-je le gardien de mon frère, moi ?
\VS{10}Et Dieu dit : Qu'as-tu fait ? La voix du sang de ton frère crie de la terre à moi.
\VS{11}Maintenant donc tu seras maudit de la terre, qui a ouvert sa bouche pour recevoir de ta main le sang de ton frère.
\VS{12}Quand tu cultiveras la terre, elle ne te donnera plus son fruit, et tu seras vagabond et fugitif sur la terre.
\VS{13}Caïn dit à Yahweh : Mon châtiment est trop grand pour être supporté.
\VS{14}Voici, tu me chasses aujourd'hui de cette terre ; je serai caché loin de ta face, je serai vagabond et fugitif sur la terre, et quiconque me trouvera me tuera.
\VS{15}Yahweh lui dit : Si quelqu’un tuait Caïn, Caïn serait vengé sept fois. Ainsi Yahweh mit une marque sur Caïn afin que quiconque le trouverait ne le tue point.
\TextTitle{Caïn bâtit une cité loin de Yahweh}
\VS{16}Alors, Caïn s’éloigna de la face de Yahweh, et habita dans la terre de Nod, à l'orient d’Eden.
\VS{17}Puis Caïn connut sa femme ; elle conçut et enfanta Hénoc. Il bâtit une ville, et il donna à cette ville le nom de son fils Hénoc.
\VS{18}Hénoc engendra Irad, Irad engendra Mehujaël, Mehujaël engendra Metuschaël, et Métuschaël engendra Lémec.
\VS{19}Lémec prit deux femmes ; le nom de l'une était Ada, et le nom de l'autre Tsilla.
\VS{20}Ada enfanta Jabal : Il fut le père de ceux qui habitent dans les tentes et près des troupeaux.
\VS{21}Le nom de son frère était Jubal : Il fut le père de tous ceux qui jouent de la harpe et du chalumeau.
\VS{22}Tsilla aussi enfanta Tubal-Caïn, qui forgeait toutes sortes d'instruments d'airain et de fer. La soeur de Tubal-Caïn était Naama.
\VS{23}Lémec dit à Ada et à Tsilla ses femmes : Ecoutez ma voix femmes de Lémec, écoutez ma parole ! J’ai tué un homme pour ma blessure et un jeune homme pour ma meurtrissure.
\VS{24}Car si Caïn est vengé sept fois, Lémec le sera soixante-dix-sept fois.
\TextTitle{Naissance de Seth}
\VS{25}Adam connut encore sa femme ; elle enfanta un fils, et il l’appela du nom de Seth, car, dit-il, Dieu m'a donné un autre fils à la place d'Abel, que Caïn a tué.
\VS{26}Il naquit aussi un fils à Seth, et il l'appela du nom d’Enosch. C’est alors que l’on commença à proclamer le nom de Yahweh.
\Chap{5}
\TextTitle{La postérité d'Adam soumise à la mort\FTNTT{Ro. 5:12}}
\VerseOne{}Voici le livre de la postérité d'Adam, depuis le jour où Dieu créa l'homme, il le fit à la ressemblance de Dieu.
\VS{2}Il les créa mâle et femelle, et les bénit, et il leur donna le nom d'homme, le jour où ils furent créés.
\VS{3}Adam vécut cent trente ans, et engendra un fils à sa ressemblance, selon son image\FTNT{Désormais les hommes naissent à la ressemblance d’Adam, c’est-à-dire pécheurs (Ro. 3:23 ; Ro. 5:14-17).}, et il lui donna le nom de Seth.
\VS{4}Les jours d'Adam, après qu'il eut engendré Seth, furent de huit cents ans, et il engendra des fils et des filles.
\VS{5}Tous les jours qu'Adam vécut furent de neuf cent trente ans ; puis il mourut.
\TextTitle{De Seth aux fils de Noé\FTNTT{Ro. 5:12}}
\VS{6}Seth aussi vécut cent cinq ans, et engendra Enosch.
\VS{7}Seth, après qu'il eut engendré Enosch, vécut huit cent sept ans ; et il engendra des fils et des filles.
\VS{8}Tous les jours que Seth vécut furent de neuf cent douze ans ; puis il mourut.
\VS{9}Enosch, ayant vécu quatre-vingt-dix ans, engendra Kénan.
\VS{10}Enosch, après qu'il eut engendré Kénan, vécut huit cent quinze ans, et il engendra des fils et des filles.
\VS{11}Tous les jours qu'Enosch vécut furent de neuf cent cinq ans ; puis il mourut.
\VS{12}Kénan, ayant vécu soixante-dix ans, engendra Mahalaleel.
\VS{13}Kénan, après qu'il eut engendré Mahalaleel, vécut huit cent quarante ans ; et il engendra des fils et des filles.
\VS{14}Tous les jours que Kénan vécut furent de neuf cent dix ans ; puis il mourut.
\VS{15}Mahalaleel vécut soixante-cinq ans ; et il engendra Jéred.
\VS{16}Et Mahalaleel, après qu'il eut engendré Jéred, vécut huit cent trente ans, et il engendra des fils et des filles.
\VS{17}Tous les jours donc que Mahalaleel vécut furent de huit cent quatre-vingt-quinze ans ; puis il mourut.
\VS{18}Jéred, ayant vécu cent soixante-deux ans, engendra Hénoc.
\VS{19}Jéred, après avoir engendré Hénoc, vécut huit cents ans, et il engendra des fils et des filles.
\VS{20}Tous les jours que Jéred vécut furent de neuf cent soixante-deux ans ; puis il mourut.
\VS{21}Hénoc vécut soixante-cinq ans, et engendra Metuschélah.
\VS{22}Hénoc, après qu'il eut engendré Metuschélah, marcha avec Dieu trois cents ans ; et il engendra des fils et des filles.
\VS{23}Tous les jours qu'Hénoc vécut furent de trois cent soixante-cinq ans.
\VS{24}Hénoc marcha avec Dieu ; mais il ne parut plus parce que Dieu le prit.
\VS{25}Metuschélah, ayant vécu cent quatre-vingt-sept ans, engendra Lémec.
\VS{26}Metuschélah, après qu'il eut engendré Lémec, vécut sept cent quatre-vingt-deux ans ; et il engendra des fils et des filles.
\VS{27}Tous les jours que Metuschélah vécut furent de neuf cent soixante-neuf ans ; puis il mourut.
\VS{28}Lémec aussi vécut cent quatre-vingt-deux ans, et il engendra un fils.
\VS{29}Il l'appela Noé, en disant : Celui-ci nous consolera de notre oeuvre, et du travail pénible de nos mains, sur la terre que Yahweh a maudite.
\VS{30}Lémec, après qu'il eut engendré Noé, vécut cinq cent quatre-vingt-quinze ans ; et il engendra des fils et des filles.
\VS{31}Tous les jours que Lémec vécut furent de sept cent soixante-dix-sept ans ; puis il mourut.
\VS{32}Noé, âgé de cinq cents ans, engendra Sem, Cham, et Japhet.
\Chap{6}
\TextTitle{Le mal dans le cœur de l'homme\FTNTT{Ro. 5:12}}
\VerseOne{}Lorsque les hommes eurent commencé à se multiplier sur la face de la terre, et qu'ils eurent engendré des filles,
\VS{2}les fils de Dieu\FTNT{Ici, les fils de Dieu sont des anges qui ont quitté leur demeure (Jud. 1:5-7).} virent que les filles des hommes étaient belles, et ils en prirent pour femmes parmi toutes celles qu'ils choisirent.
\TextTitle{Yahweh ne conteste plus avec les hommes}
\VS{3}Yahweh dit : Mon Esprit ne contestera point à toujours avec les hommes\FTNT{C’est le Saint-Esprit qui nous convainc de péché, de jugement et de justice (Jn. 16:8). Lorsqu’il constate que le cœur d’une personne est définitivement endurci au point de refuser la repentance, il renonce à la convaincre de péché et il se retire. La génération antédiluvienne avait définitivement rejeté Dieu en choisissant de faire du mal son idole (Ge. 6:5). Elle était allée si loin dans l’abomination au point de s’accoupler avec des anges déchus (Ge. 6:4), ce qui laisse supposer un culte volontaire aux démons. Lorsque le Saint-Esprit est retiré d’une personne, il est remplacé par l’esprit d’égarement qui enferme le pécheur dans l’erreur et l’entraîne ainsi à sa condamnation éternelle (Mt. 12:31 ; 2 Th. 2:11 )}, car les hommes ne sont que chair, et leurs jours seront de cent vingt ans.
\TextTitle{Le monde avant le déluge\FTNTT{Lu. 17:27}}
\VS{4}Les géants étaient sur la terre en ce temps-là. Il en fut de même après que les fils de Dieu furent venus vers les filles des hommes, et qu’elles leur eurent donné des enfants. Ce sont ces hommes vaillants qui furent des gens de renom dans l’antiquité.
\TextTitle{Yahweh prépare un jugement}
\VS{5}Yahweh vit que la méchanceté des hommes était très grande sur la terre, et que toute l'imagination des pensées de leur cœur n'était que mal en tout temps.
\VS{6}Yahweh se repentit d'avoir fait l'homme sur la terre, et il fut affligé en son cœur.
\VS{7}Et Yahweh dit : J'exterminerai de la face de la terre les hommes que j'ai créés, depuis les hommes jusqu'au bétail, jusqu'aux reptiles, et même jusqu'aux oiseaux du ciel ; car je me repens de les avoir faits.
\TextTitle{La grâce de Yahweh sur Noé : Construction de l'arche}
\VS{8}Mais Noé trouva grâce aux yeux de Yahweh.
\VS{9}Voici la postérité de Noé. Noé était un homme juste et intègre en son temps ; Noé marchait avec Dieu.
\VS{10}Noé engendra trois fils : Sem, Cham, et Japhet.
\VS{11}Et la terre était corrompue devant Dieu, et remplie de violence.
\VS{12}Dieu donc regarda la terre, et voici elle était corrompue ; car toute chair avait corrompu sa voie sur la terre.
\VS{13}Et Dieu dit à Noé : La fin de toute chair est venue devant moi ; car ils ont rempli la terre de violence, et voici, je les détruirai avec la terre.
\VS{14}Fais-toi une arche\FTNT{L’arche  est un type de Christ et du salut en lui et par lui.  On peut voir plusieurs aspects de Jésus-Christ et de la rédemption dans la structure de l’arche :
- L’arche a été une révélation de Jésus-Christ donnée à Noé.  C’est en Jésus-Christ que nous avons le salut et la protection (Col. 1:12-13 ; Col. 3:3). 
-L’arche était faite de bois de gopher, probablement du cèdre.  Ce bois est un bois qui ne pourrit pas en condition normale. Ce bois préfigurait l’incorruptibilité de Jésus-Christ homme (Es. 53:9 ;  Hé. 4:15 ; 1 Pi. 2:22). 
-Au verset 14, on lit que Dieu demande à Noé d’enduire l’arche en dedans et en dehors avec de la  poix, c’est-à-dire du bitume.  Le mot «~poix~» vient de l’hébreu «~kaphar~», qui signifie «~expiation~».  Ce mot est traduit près de 70 fois dans le Tanakh par expiation.  Il est également traduit par «~réconciliation~», «~pardon~», «~miséricordieux~» et «~apaiser~».  L’allusion à l’expiation des péchés faite par Jésus-Christ est claire. Par son sacrifice, nous sommes rendus parfaits à jamais (Hé. 10:14-15).} de bois de gopher ; tu feras cette arche en cellules, et tu l’enduiras de poix en dedans et en dehors.
\VS{15}Et voici comment tu la feras : La longueur de l'arche sera de trois cents coudées ; sa largeur de cinquante coudées, et sa hauteur de trente coudées.
\VS{16}Tu feras une fenêtre à l'arche, et feras son comble d'une coudée de hauteur, et tu mettras la porte de l'arche à son côté, et tu la feras avec un bas, un second, et un troisième étage.
\VS{17}Et voici, je ferai venir un déluge d'eau sur la terre, pour détruire toute chair dans laquelle il y a souffle de vie sous les cieux ; et tout ce qui est sur la terre expirera.
\VS{18}Mais j'établirai mon alliance avec toi ; et tu entreras dans l'arche toi et tes fils, et ta femme, et les femmes de tes fils avec toi.
\VS{19}Et de tout ce qui a vie d'entre toute chair, tu en feras entrer deux de chaque espèce dans l'arche, pour les conserver en vie avec toi, à savoir le mâle et la femelle.
\VS{20}Des oiseaux, selon leur espèce, des bêtes à quatre pattes, selon leur espèce, et de tous les reptiles, selon leur espèce. Ils y entreront tous par paires avec toi, afin que tu les conserves en vie.
\VS{21}Prends aussi avec toi de tous les aliments que l’on mange, et rassemble-les auprès de toi, afin qu'ils servent pour ta nourriture et pour celle des animaux.
\VS{22}Et Noé fit selon tout ce que Dieu lui avait ordonné ; il le fit ainsi.
\Chap{7}
\TextTitle{Le jugement par le déluge}
\VerseOne{}Yahweh dit à Noé : Entre dans l’arche, toi et toute ta maison ; car je t'ai vu juste devant moi parmi cette génération. 
\VS{2} Tu prendras de toutes les bêtes pures sept de chaque espèce, le mâle et sa femelle ; mais des bêtes qui ne sont point pures, un couple, le mâle et la femelle.
\VS{3}Tu prendras aussi des oiseaux du ciel sept de chaque espèce, le mâle et sa femelle ; afin d'en conserver la race sur toute la terre.
\VS{4}Car dans sept jours, je ferai pleuvoir sur la terre pendant quarante jours et quarante nuits ; et j'exterminerai de la surface de la terre tous les êtres qui subsistent que j'ai faits.
\VS{5}Noé fit selon tout ce que Yahweh lui avait ordonné.
\VS{6}Noé était âgé de six cents ans quand le déluge des eaux vint sur la terre.
\VS{7}Noé donc entra dans l’arche avec ses fils, sa femme, et les femmes de ses fils, pour échapper aux eaux du déluge.
\VS{8}Des bêtes pures, des bêtes qui ne sont point pures, des oiseaux, et tout ce qui se meut sur la terre.
\VS{9}Elles entrèrent deux à deux vers Noé dans l'arche, le mâle et la femelle, comme Dieu l’avait ordonné à Noé.
\VS{10}Sept jours après, les eaux du déluge furent sur la terre.
\VS{11}En l'an six cent de la vie de Noé, au second mois, le dix-septième jour du mois, en ce jour-là toutes les sources du grand abîme furent rompues, et les écluses des cieux furent ouvertes.
\VS{12}La pluie tomba sur la terre pendant quarante jours et quarante nuits.
\VS{13}Ce même jour entrèrent dans l’arche Noé, Sem, Cham, et Japhet, fils de Noé, avec la femme de Noé, et les trois femmes de ses fils avec eux.
\VS{14}Eux, et tous les animaux selon leur espèce, et tout le bétail selon son espèce, et tous les reptiles qui se meuvent sur la terre selon leur espèce, et tous les oiseaux selon leur espèce ; et tout petit oiseau ayant des ailes, de quelque sorte que ce soit.
\VS{15}Ils entrèrent dans l'arche auprès de Noé, deux à deux, de toute chair ayant souffle de vie.
\VS{16}Il en entra mâle et femelle de toute chair comme Dieu l’avait ordonné à Noé, puis Yahweh ferma l'arche sur lui.
\VS{17}Le déluge fut pendant quarante jours sur la terre ; et les eaux crurent et élevèrent l'arche, et elle fut élevée au-dessus de la terre.
\VS{18}Les eaux  grossirent et s'accrurent beaucoup sur la terre, et l'arche flottait au-dessus des eaux.
\VS{19}Les eaux grossirent de plus en plus sur la terre, et toutes les hautes montagnes qui sont sous le ciel entier en furent couvertes.
\VS{20}Les eaux s’élevèrent de quinze coudées au-dessus des montagnes  qui furent couvertes.
\VS{21}Toute chair qui se mouvait sur la terre périt, tant les oiseaux que le bétail et les animaux, tous les reptiles qui rampaient sur la terre, et tous les hommes.
\VS{22}Tout ce qui avait respiration, souffle de vie dans ses narines, et qui était sur la terre sèche mourut.
\VS{23}Tous les êtres qui étaient sur la face de la terre furent donc exterminés, depuis les hommes jusqu’au bétail, aux reptiles et aux oiseaux du ciel ; ils furent exterminés de la face de la terre ; il ne resta seulement que Noé, et ce qui était avec lui dans l'arche.
\VS{24}Les eaux furent grosses sur la terre pendant cent cinquante jours.
\Chap{8}
\TextTitle{Fin du déluge}
\VerseOne{}Dieu se souvint de Noé, de tous les animaux et de tout le bétail qui étaient avec lui dans l'arche ; et Dieu fit passer un vent sur la terre, et les eaux s’apaisèrent.
\VS{2}Les sources de l'abîme et les écluses des cieux furent fermées et la pluie ne tomba plus du ciel.
\VS{3}Au bout de cent cinquante jours, les eaux se retirèrent sans interruption de dessus la terre, et diminuèrent.
\VS{4}Le dix-septième jour du septième mois, l'arche s'arrêta sur les montagnes d'Ararat.
\VS{5}Les eaux allèrent en diminuant de plus en plus jusqu'au dixième mois ; et au premier jour du dixième mois, les sommets des montagnes apparurent.
\VS{6}Au bout de quarante jours, Noé ouvrit la fenêtre qu’il avait faite à l'arche.
\VS{7}Il lâcha le corbeau, qui sortit, allant et revenant, jusqu'à ce que les eaux aient séché sur la terre.
\VS{8}Il lâcha aussi une colombe pour voir si les eaux avaient diminué à la surface de la terre.
\VS{9}Mais la colombe ne trouvant aucun lieu pour poser la plante de son pied, retourna à lui dans l'arche, car les eaux étaient sur toute la terre ; et Noé avançant sa main la reprit et la fit entrer dans l'arche.
\VS{10}Il attendit encore sept autres jours, il lâcha de nouveau la colombe hors de l'arche.
\VS{11}Sur le soir, la colombe revint à lui ; et voici, elle avait dans son bec une feuille d'olivier qu'elle avait arrachée ; et Noé connut que les eaux avaient diminué sur la terre.
\VS{12}Il attendit encore sept autres jours, puis il lâcha la colombe qui ne retourna plus à lui.
\VS{13}L’an six cent un de l'âge de Noé, le premier jour du premier mois, les eaux avaient diminué sur la terre. Noé ôta la couverture de l'arche, regarda, et voici, la surface de la terre avait séché.
\VS{14}Le vingt-septième jour du second mois la terre fut sèche.
\TextTitle{Noé sort de l'arche: Le règne des hommes\FTNTT{Ge. 8:11-15:32}}
\VS{15}Puis Dieu parla à Noé, en disant :
\VS{16}Sors de l'arche, toi et ta femme, tes fils, et les femmes de tes fils avec toi.
\VS{17}Fais sortir avec toi tous les animaux qui sont avec toi, de toute chair, tant les oiseaux que le bétail, et tous les reptiles qui rampent sur la terre ; qu'ils se répandent sur la terre, et qu'ils soient féconds et multiplient sur la terre.
\VS{18}Noé donc sortit, et avec lui ses fils, sa femme, et les femmes de ses fils.
\VS{19}Tous les animaux, tous les reptiles, tous les oiseaux, tout ce qui se meut sur la terre, selon leurs espèces, sortirent de l'arche.
\VS{20}Noé bâtit un autel à Yahweh, il prit de toutes les bêtes pures, et de tout oiseau pur, et il offrit des holocaustes sur l'autel.
\VS{21}Yahweh respira une odeur agréable, et dit en son cœur : Je ne maudirai plus la terre à cause des hommes, quoique les dispositions du coeur des hommes soient mauvaises dès leur jeunesse ; et je ne frapperai plus tout ce qui est vivant, comme je l’ai fait.
\VS{22}Tant que la terre subsistera, les semailles et les moissons, le froid et la chaleur, l'été et l'hiver, le jour et la nuit ne cesseront point.
\Chap{9}
\TextTitle{Yahweh établit une alliance avec Noé\FTNTT{Ge. 9:16}}
\VerseOne{}Dieu bénit Noé et ses fils, et leur dit : Soyez féconds, multipliez, et remplissez la terre.
\VS{2}Vous serez un sujet de crainte et d’effroi pour tout animal de la terre, pour tout oiseau du ciel, pour tout ce qui se meut sur la terre, et pour tous les poissons de la mer : Ils sont livrés entre vos mains.
\VS{3}Tout ce qui se meut et qui a vie sera votre nourriture ; je vous donne tout cela comme l'herbe verte.
\VS{4}Seulement, vous ne mangerez point de chair avec son âme, c'est-à-dire, son sang.
\VS{5}Sachez-le aussi, je redemanderai votre sang, le sang de vos âmes, je le redemanderai à tout animal ; et je redemanderai l’âme de l’homme de la main de l’homme, de la main de son frère.
\VS{6}Celui qui aura versé le sang de l'homme, par l'homme son sang sera versé ; car Dieu a fait l'homme à son image.
\VS{7}Vous donc, soyez féconds et multipliez, répandez-vous sur la terre et multipliez sur elle.
\VS{8}Dieu parla aussi à Noé et à ses fils qui étaient avec lui, en disant :
\VS{9}Et quant à moi, voici, j'établis mon alliance avec vous, et avec votre postérité après vous ;
\VS{10}avec tous les êtres vivants qui sont avec vous, tant les oiseaux que le bétail, et tous les animaux de la terre qui sont avec vous, tous ceux qui sont sortis de l'arche jusqu'à tous les animaux de la terre.
\VS{11}J'établis donc mon alliance avec vous ; aucune chair ne sera plus exterminée par les eaux du déluge, et il n'y aura plus de déluge pour détruire la terre.
\VS{12}Puis Dieu dit : C'est ici le signe de l'alliance que j’établis entre moi et vous, et tous les êtres vivants qui sont avec vous, pour les générations à toujours :
\VS{13}J’ai placé mon arc dans la nuée, et il servira de signe de l'alliance entre moi et la terre.
\VS{14}Quand j’aurai rassemblé des nuages au-dessus de la terre, l’arc paraîtra dans la nuée ;
\VS{15}et je me souviendrai de mon alliance entre moi et vous, et tous les êtres vivants de toute chair, et les eaux ne deviendront plus un déluge pour détruire toute chair.
\VS{16}L'arc donc sera dans la nuée, et je le regarderai, et je me souviendrai de l'alliance perpétuelle entre Dieu et tous les êtres vivants  de toute chair qui est sur la terre.
\VS{17}Dieu donc dit à Noé : C'est là le signe de l'alliance que j'ai établie entre moi et toute chair qui est sur la terre.
\VS{18}Les fils de Noé qui sortirent de l'arche étaient Sem, Cham, et Japhet. Cham fut père de Canaan.
\VS{19}Ce sont là les trois fils de Noé, et c’est leur postérité qui peupla toute la terre.
\TextTitle{Le péché de Noé}
\VS{20}Or, Noé commença à cultiver la terre, et planta de la vigne.
\VS{21}Il but du vin, s'enivra, et se découvrit au milieu de sa tente.
\VS{22}Cham, père de Canaan, vit la nudité de son père\FTNT{Lé. 18:6-19 ; Lé. 20:11-21.}, et il le rapporta dehors à ses deux frères.
\VS{23}Alors Sem et Japhet prirent un manteau qu'ils mirent sur leurs deux épaules, et marchant à reculons, ils couvrirent la nudité de leur père ; et leurs visages étaient tournés en arrière, de sorte qu'ils ne virent point la nudité de leur père.
\VS{24}Et quand Noé se réveilla de son vin, il apprit ce que lui avait fait son fils cadet.
\TextTitle{Noé prononce une malédiction contre Canaan}
\VS{25}C'est pourquoi il dit : Maudit soit Canaan\FTNT{Une idée erronée selon laquelle les noirs auraient été maudits par Dieu au travers de la malédiction de Canaan s’est répandue pendant des siècles. On a ainsi légitimé la domination des peuples africains par les puissances occidentales blanches, et par la même occasion l’esclavage. Il faut préciser que les descendants de Cham furent Cush (Ethiopie), Mitsraïm (Egypte), Puth (les Celtes) et Canaan (Palestine, pays que Dieu a donné aux descendants de Sem, selon Ge. 15). Cham est le fils cadet, c’est-à-dire le coupable aux yeux de Noé. Mais c’est à Canaan (Palestine), le fils de Cham, donc petit-fils de Noé, que s’adresse la malédiction. Selon la Bible, les peuples africains sont des descendants de Cham, mais par son fils Cush et non par Canaan. La prétendue malédiction des noirs n’a donc aucun fondement.} ; il sera serviteur des serviteurs de ses frères.
\VS{26}Il dit aussi : Béni soit Yahweh, Dieu de Sem ; et que Canaan soit leur serviteur.
\VS{27}Que Dieu étende en douceur Japhet, et qu’il habite dans les tentes de Sem ; et que Canaan soit leur serviteur.
\VS{28}Noé vécut après le déluge trois cent cinquante ans.
\VS{29}Tout le temps donc que Noé vécut fut de neuf cent cinquante ans ; puis il mourut.
\Chap{10}
\TextTitle{La postérité de Noé}
\VerseOne{}Voici la postérité des enfants de Noé, Sem, Cham et Japhet ; il leur naquit des fils après le déluge.
\VS{2}Les fils de Japhet furent : Gomer, Magog, Madaï, Javan, Tubal, Mésech, et Tiras.
\VS{3}Les fils de Gomer : Aschkenaz, Riphat, et Togarma.
\VS{4}Les fils de Javan : Elischa, Tarsis, Kittim, et Dodanim.
\VS{5}C’est par eux qu’ont été peuplées les îles des nations selon leurs terres, chacun selon sa langue, selon leurs familles, entre leurs nations.
\VS{6}Les fils de Cham furent : Cusch, Mitsraïm, Puth, et Canaan.
\VS{7}Les fils de Cusch : Saba, Havila, Sabta, Raema, et Sabteca. Les fils de Raema : Séba et Dedan.
\VS{8}Cusch engendra aussi Nimrod\FTNT{Nimrod ou Nemrod, dont le nom signifie «~rebelle~», fut le premier roi de l’histoire biblique. Fils de Cusch (Ethiopie), lui-même premier-né de Cham, fils de Noé (Ge. 10:8-10), il fut à la tête du premier empire après le déluge. Il se distingua en qualité de puissant chasseur  «~devant Yahweh~» ou «~contre Yahweh~». Le contexte du chapitre 10 laisse entendre que Nimrod était un puissant chasseur qui provoquait Dieu. Fondateur de Ninive, il est surtout connu pour avoir été à l’origine du projet de la tour de Babel.}, c’est lui qui commença à être puissant sur la terre.
\VS{9}Il fut un puissant chasseur devant Yahweh, c'est pourquoi l'on a dit : Comme Nimrod, le puissant chasseur devant Yahweh.
\VS{10}Il régna d’abord sur Babel\FTNT{Le nom Babel signifie confusion par le mélange.}, Erec, Accad, et Calné au pays de Schinear.
\VS{11}De ce pays-là sortit Assur, et il bâtit Ninive et les rues de la ville, Rehoboth-Hir et Calach,
\VS{12}et Résen, entre Ninive et Calach, qui est une grande ville.
\VS{13}Mitsraïm engendra les Ludim, les Anamim, les Lehabim, les Naphtuhim,
\VS{14}les Patrusim, les Casluhim, d’où sont sortis les Philistins, et les Caphtorim.
\VS{15}Canaan engendra Sidon, son premier-né, et Heth ;
\VS{16}et les Jébusiens, les Amoréens, les Guirgasiens,
\VS{17}les Héviens, les Arkiens, les Siniens,
\VS{18}les Arvadiens, les Tsemariens, les Hamathiens. Ensuite, les familles des Cananéens se sont dispersées.
\VS{19}Les limites des Cananéens furent depuis Sidon, quand on vient vers Guérar, jusqu'à Gaza, en allant vers Sodome et Gomorrhe, Adma, et Tseboïm, jusqu'à Léscha.
\VS{20}Ce sont là les fils de Cham selon leurs familles et leurs langues, selon leurs pays, et selon leurs nations.
\VS{21}Il naquit aussi des fils à Sem, père de tous les fils d'Héber, et frère aîné de Japhet.
\VS{22}Les fils de Sem furent : Elam, Assur, Arpacschad, Lud et Aram.
\VS{23}Les fils d'Aram : Uts, Hul, Guéter et Masch.
\VS{24}Arpacschad engendra Schélach ; et Schélach engendra Héber.
\VS{25}Il naquit à Héber deux fils : Le nom de l'un était Péleg, parce que de son temps la terre fut partagée ; et le nom de son frère était Jokthan.
\VS{26}Jokthan engendra Almodad, Schéleph, Hatsarmaveth, Jérach,
\VS{27}Hadoram, Uzal, Dikla,
\VS{28}Obal, Abimaël, Séba,
\VS{29}Ophir, Havila, et Jobab. Tous ceux-là sont les enfants de Jokthan.
\VS{30}Ils habitèrent depuis Méscha, du côté de Sephar jusqu’à la montagne de l’orient.
\VS{31}Ce sont là les fils de Sem, selon leurs familles, selon leurs langues, selon leurs pays, et selon leurs nations.
\VS{32}Telles sont les familles des fils de Noé, selon leurs lignées, selon leurs nations. Et c’est d’eux que sont sorties les nations qui se sont répandues sur la terre après le déluge.
\Chap{11}
\TextTitle{Un projet humain : La tour de Babel}
\VerseOne{}Alors toute la terre avait un même langage et les mêmes paroles.
\VS{2}Mais il arriva qu'étant partis d'orient, ils trouvèrent une vallée au pays de Schinear où ils habitèrent.
\VS{3}Et ils se dirent l'un à l'autre : Allons ! Faisons des briques, et cuisons-les très bien au feu. Et la brique leur servit de pierre, et le bitume leur servit d’argile.
\VS{4}Puis ils dirent : Allons ! Bâtissons-nous une ville, et une tour dont le sommet soit jusqu’aux cieux ; et faisons-nous un nom, de peur que nous ne soyons dispersés sur toute la terre.
\VS{5}Alors Yahweh descendit pour voir la ville et la tour que les fils des hommes bâtissaient.
\VS{6}Et Yahweh dit : Voici, ce n'est qu'un seul et même peuple, ils ont un même langage, et ils commencent à travailler ; et maintenant rien ne les empêchera d'exécuter ce qu'ils ont projeté.
\TextTitle{Yahweh confond le langage humain}
\VS{7}Allons ! Descendons, et là confondons leur langage afin qu'ils n'entendent point le langage les uns des autres.
\VS{8}Ainsi Yahweh les dispersa de là par toute la terre, et ils cessèrent de bâtir la ville.
\VS{9}C'est pourquoi on l’appela du nom de Babel, car c’est là que Yahweh confondit le langage de toute la terre, et c’est de là que Yahweh les dispersa sur toute la terre.
\TextTitle{La postérité de Sem, ancêtre d’Abram}
\VS{10}Voici la postérité de Sem : Sem, âgé de cent ans, engendra Arpacschad, deux ans après le déluge.
\VS{11}Sem, après qu'il eut engendré Arpacschad, vécut cinq cents ans, et engendra des fils et des filles.
\VS{12}Arpacschad vécut trente-cinq ans, et engendra Schélach.
\VS{13}Arpacschad, après qu'il eut engendré Schélach, vécut quatre cent trois ans, et engendra des fils et des filles.
\VS{14}Schélach, ayant vécu trente ans, engendra Héber.
\VS{15}Schélach, après qu'il eut engendré Héber, vécut quatre cent trois ans, et engendra des fils et des filles.
\VS{16}Héber, ayant vécu trente-quatre ans, engendra Péleg.
\VS{17}Héber, après qu'il eut engendré Péleg, vécut quatre cent trente ans, et engendra des fils et des filles.
\VS{18}Péleg, ayant vécu trente ans, engendra Rehu.
\VS{19}Péleg, après qu'il eut engendré Rehu, vécut deux cent neuf ans, et engendra des fils et des filles.
\VS{20}Rehu, ayant vécu trente-deux ans, engendra Serug.
\VS{21}Rehu, après qu'il eut engendré Serug, vécut deux cent sept ans, et engendra des fils et des filles.
\VS{22}Serug, ayant vécu trente ans, engendra Nachor.
\VS{23}Serug, après qu'il eut engendré Nachor, vécut deux cents ans, et engendra des fils et des filles.
\VS{24}Nachor, ayant vécu vingt-neuf ans, engendra Térach.
\VS{25}Nachor, après qu'il eut engendré Térach, vécut cent dix-neuf ans, et engendra des fils et des filles.
\VS{26}Térach, ayant vécu soixante-dix ans, engendra Abram, Nachor, et Haran.
\VS{27}Voici la postérité de Térach : Térach engendra Abram, Nachor, et Haran ; et Haran engendra Lot.
\VS{28}Et Haran mourut en présence de son père, au pays de sa naissance, à Ur en Chaldée.
\VS{29}Abram et Nachor prirent chacun une femme. Le nom de la femme d'Abram était Saraï ; et le nom de la femme de Nachor était Milca, fille de Haran, père de Milca et de Jisca.
\VS{30}Saraï était stérile, et n'avait point d'enfants.
\TextTitle{Séjour à Charan}
\VS{31}Térach prit son fils Abram, et Lot fils de son fils, qui était fils de Haran, et Saraï, sa belle-fille, femme d'Abram, son fils, et ils sortirent ensemble d'Ur en Chaldée pour aller au pays de Canaan, et ils vinrent jusqu'à Charan, et ils y habitèrent.
\VS{32}Les jours de Térach furent de deux cent cinq ans ; puis il mourut à Charan.
\Chap{12}
\TextTitle{Appel d'Abram : La promesse de Yahweh\FTNTT{Ge. 12:2 ; 13:14-18 ; 15:1-21 ; 17:4-8 ; 22:15-24 ; 26:1-5 ; 28:10-15}}
\VerseOne{}Yahweh dit à Abram : Va pour toi, hors de ta terre, de ta patrie, et de la maison de ton père, vers la terre que je te montrerai\FTNT{Ac. 7:3 ; Hé. 11:8.}.
\VS{2}Je te ferai devenir une grande nation, et je te bénirai, je rendrai ton nom grand, et tu seras béni.
\VS{3}Je bénirai ceux qui te béniront, et je maudirai ceux qui te maudiront ; et toutes les familles de la terre seront bénies en toi\FTNT{Ac. 3:25 ; Ga. 3:8.}.
\TextTitle{Abram sur la terre de Canaan}
\VS{4}Abram donc partit, comme Yahweh le lui avait dit, et Lot alla avec lui. Abram était âgé de soixante-quinze ans quand il sortit de Charan.
\VS{5}Abram prit aussi Saraï, sa femme, et Lot, fils de son frère, avec tous les biens qu'ils avaient acquis, et les personnes qu'ils avaient eues à Charan ; et ils partirent pour aller dans le pays de Canaan, et ils arrivèrent au pays de Canaan\FTNT{Ac. 7:4.}.
\VS{6}Abram parcourut le pays jusqu'au lieu nommé Sichem, et jusqu'aux chênes de Moré ; et les Cananéens étaient alors dans le pays.
\VS{7}Yahweh apparut à Abram, et lui dit : Je donnerai ce pays à ta postérité. Et Abram bâtit là un autel à Yahweh qui lui était apparu.
\VS{8}Il se transporta de là vers la montagne, à l'orient de Béthel, et il dressa ses tentes, ayant Béthel à l'occident, et Aï à l'orient ; et il bâtit là un autel à Yahweh, et invoqua le nom de Yahweh.
\VS{9}Puis Abram partit de là, marchant et s'avançant vers le midi.
\TextTitle{Abram en Egypte}
\VS{10}Mais la famine étant survenue dans le pays, Abram descendit en Egypte pour s'y retirer, car la famine était grande dans le pays.
\VS{11}Comme il était près d'entrer en Egypte, il dit à Saraï, sa femme : Voici, je sais que tu es une fort belle femme ;
\VS{12}c'est pourquoi, quand les Egyptiens te verront, ils diront : C'est la femme de cet homme, et ils me tueront, mais ils te laisseront vivre.
\VS{13}Dis donc, je te prie, que tu es ma sœur, afin que je sois bien traité à cause de toi, et que par ton moyen, ma vie soit préservée.
\VS{14}Il arriva donc qu'aussitôt qu'Abram fut arrivé en Egypte, les Egyptiens virent que cette femme était fort belle.
\VS{15}Les principaux de la cour de Pharaon la virent aussi et la vantèrent à Pharaon, et elle fut enlevée pour être menée dans la maison de Pharaon.
\VS{16}Il traita bien Abram à cause d'elle, de sorte qu'il en eut des brebis, des bœufs, des ânes, des serviteurs, des servantes, des ânesses, et des chameaux.
\VS{17}Mais Yahweh frappa de grandes plaies Pharaon et sa maison, à cause de Saraï, femme d'Abram.
\VS{18}Alors Pharaon appela Abram, et lui dit : Qu'est-ce que tu m'as fait ? Pourquoi ne m'as-tu pas déclaré que c'était ta femme ?
\VS{19}Pourquoi as-tu dit : C'est ma sœur ? Car je l'avais prise pour ma femme ; mais maintenant, voici ta femme, prends-la, et va-t'en.
\VS{20}Et Pharaon ayant donné ordre à ses gens, ils le renvoyèrent, lui, sa femme, et tout ce qui était à lui.
\Chap{13}
\TextTitle{Retour d'Abram à Canaan}
\VerseOne{}Abram donc monta d'Egypte vers le midi, lui, sa femme, et tout ce qui lui appartenait, et Lot avec lui.
\VS{2}Et Abram était très riche en bétail, en argent, et en or.
\VS{3}Et il s'en retourna en suivant la route qu'il avait suivie du midi à Béthel, jusqu'au lieu où il avait dressé ses tentes au commencement, entre Béthel et Aï,
\VS{4}au même lieu où était l'autel qu'il y avait bâti au commencement, et Abram invoqua là le nom de Yahweh.
\TextTitle{Abram se sépare de Lot\FTNTT{Ge. 13:12}}
\VS{5}Lot aussi, qui marchait avec Abram, avait des brebis, des boeufs, et des tentes.
\VS{6}Et le pays ne pouvait les porter pour demeurer ensemble ; car leurs biens étaient si grand qu'ils ne pouvaient demeurer ensemble.
\VS{7}Il y eut querelle entre les bergers du bétail d'Abram et les bergers du bétail de Lot ; or en ce temps-là, les Cananéens et les Phérésiens habitaient dans le pays.
\VS{8}Et Abram dit à Lot : Je te prie qu'il n'y ait point de dispute entre moi et toi, ni entre mes bergers et les tiens, car nous sommes frères.
\VS{9}Tout le pays n'est-il pas devant toi ? Sépare-toi je te prie d'avec moi. Si tu vas à gauche, j’irai à droite ; et si tu vas à droite, j’irai à gauche.
\TextTitle{Lot s'établit à Sodome\FTNTT{Ge. 13:10}}
\VS{10}Lot, levant les yeux, vit que toute la plaine du Jourdain était entièrement arrosée. Avant que Yahweh ait détruit Sodome et Gomorrhe, c’était, jusqu'à Tsoar, comme le jardin de Yahweh, et comme le pays d'Egypte.
\VS{11}Lot choisit pour lui toute la plaine du Jourdain, et alla du côté de l’orient ; ainsi ils se séparèrent l'un de l'autre.
\VS{12}Abram habita dans le pays de Canaan, et Lot habita dans les villes de la plaine, et dressa ses tentes jusqu'à Sodome.
\VS{13}Les habitants de Sodome étaient méchants et de grands pécheurs contre Yahweh.
\TextTitle{Yahweh confirme son alliance avec Abram}
\VS{14}Yahweh dit à Abram, après que Lot se fut séparé de lui : Lève maintenant tes yeux, et regarde du lieu où tu es vers le nord, le midi, l'orient, et l'occident.
\VS{15}Car je te donnerai, à toi et à ta postérité pour toujours, tout le pays que tu vois.
\VS{16}Je rendrai ta postérité comme la poussière de la terre ; en sorte que si quelqu'un peut compter la poussière de la terre, il comptera aussi ta postérité\FTNT{Ro. 4:18 ; Hé. 11:12.}.
\VS{17}Lève-toi donc et promène-toi dans le pays, dans sa longueur et dans sa largeur, car je te le donnerai.
\VS{18}Abram ayant transporté ses tentes, alla habiter dans les plaines de Mamré, qui sont près d’Hébron et là, il bâtit un autel à Yahweh.
\Chap{14}
\TextTitle{Abram va au secours de Lot}
\VerseOne{}Dans le temps d'Amraphel, roi de Schinear, d'Arjoc, roi d'Ellasar, de Kedorlaomer, roi d'Elam, et de Tideal, roi de Gojim,
\VS{2}il arriva qu’ils firent la guerre contre Béra, roi de Sodome, et contre Birscha, roi de Gomorrhe, et contre Schineab, roi d'Adma, et contre Schémeéber, roi de Tseboïm, et contre le roi de Béla, qui est Tsoar.
\VS{3}Tous ceux-ci se joignirent dans la vallée de Siddim, qui est la mer salée.
\VS{4}Ils avaient été asservis douze années à Kedorlaomer, et la treizième année, ils s'étaient révoltés.
\VS{5}A la quatorzième année, Kedorlaomer et les rois qui étaient avec lui vinrent et ils battirent les Rephaïm à Aschteroth-Karnaïm, les Zuzim à Ham, et les Emin à la plaine de Schavé-Kirjathaïm,
\VS{6}et les Horiens dans leur montagne de Séir, jusqu'au chêne de Paran, qui est près du désert.
\VS{7}Puis ils s’en retournèrent et vinrent à En-Mischpath, qui est Kadès ; et ils frappèrent tout le pays des Amalécites et des Amoréens qui habitaient dans Hatsatson-Thamar.
\VS{8}Alors le roi de Sodome, le roi de Gomorrhe, le roi d'Adma, le roi de Tseboïm, et le roi de Béla qui est Tsoar, sortirent et rangèrent leurs troupes contre eux dans la vallée de Siddim.
\VS{9}C'est-à-dire contre Kedorlaomer, roi d'Elam, et contre Tideal, roi de Gojim, et contre Amraphel, roi de Schinear, et contre Arjoc, roi d'Ellasar : Quatre rois contre cinq.
\VS{10}La vallée de Siddim était pleine de puits de bitume ; les rois de Sodome et de Gomorrhe s'enfuirent et y tombèrent, et le reste s'enfuit dans la montagne.
\VS{11}Ils prirent donc toutes les richesses de Sodome et de Gomorrhe, et tous leurs vivres ; puis ils se retirèrent.
\VS{12}Ils prirent aussi Lot, fils du frère d'Abram, qui habitait dans Sodome, et tous ses biens ; puis ils s'en allèrent.
\VS{13}Un fuyard vint avertir Abram, l’Hébreu, qui demeurait dans les plaines de Mamré, l’Amoréen, frère d'Eschcol, et frère d’Aner, qui avaient fait alliance avec Abram.
\VS{14}Dès qu’Abram eut appris que son frère avait été emmené prisonnier, il arma trois cent dix-huit de ses plus braves serviteurs, nés dans sa maison, et il poursuivit ces rois jusqu'à Dan.
\VS{15}Il divisa sa troupe, il se jeta sur eux de nuit, lui et ses serviteurs ; il les battit et les poursuivit jusqu'à Choba, qui est à la gauche de Damas.
\VS{16}Il ramena tous les biens qu'ils avaient pris ; il ramena aussi Lot, son frère, ses biens, les femmes et le peuple.
\TextTitle{Melchisédek, sacrificateur d'El Elyon (Dieu Très-Haut)}
\VS{17}Le roi de Sodome sortit à la rencontre d’Abram qui revenait vainqueur de Kedorlaomer, et des rois qui étaient avec lui, dans la vallée de la plaine, qui est la vallée royale.
\VS{18}Melchisédek\FTNT{Melchisédek est un type de Christ (Ps. 110:4 ; Hé. 5:5-6 ; Hé. 6:20 ; Hé. 7:1-2). Ce personnage nous montre l’aspect de Christ en tant que roi de Salem, ce qui signifie «~paix~», et Souverain Sacrificateur possédant un sacerdoce non transmissible (Hé. 7:24).}, roi de Salem, fit apporter du pain et du vin, or il était Sacrificateur du Dieu Très-Haut.
\VS{19}Il bénit Abram en disant : Béni soit Abram par le Dieu Très-Haut, Maître du ciel et de la terre.
\VS{20}Béni soit le Dieu Très-Haut qui a livré tes ennemis entre tes mains. Et Abram lui donna la dîme\FTNT{Voir commentaire sur la dîme en No. 18:21 et Mal. 3:10.} de tout.
\VS{21}Le roi de Sodome dit à Abram : Donne-moi les personnes, et prends pour toi les richesses.
\VS{22}Abram répondit au roi de Sodome : Je lève ma main vers Yahweh, le Dieu Très-Haut, Maître du ciel et de la terre :
\VS{23}Je ne prendrai rien de tout ce qui est à toi, pas même un fil, ni un cordon de soulier, afin que tu ne dises point : J'ai enrichi Abram.
\VS{24}Seulement, ce que les jeunes gens ont mangé, et la part des hommes qui sont venus avec moi, Aner, Eschcol, et Mamré, qui prendront leur part.
\Chap{15}
\TextTitle{Yahweh promet un enfant à Abram}
\VerseOne{}Après ces choses, la parole de Yahweh fut adressée à Abram dans une vision, en disant : Abram, ne crains point, je suis ton bouclier, et ta récompense sera très grande.
\VS{2}Abram répondit : Seigneur Yahweh, que me donneras-tu ? Je m'en vais sans laisser d'enfants après moi, et l’héritier de ma maison c'est Eliézer de Damas.
\VS{3}Abram dit aussi : Voici, tu ne m'as point donné d'enfants ; et voilà, le serviteur né dans ma maison sera mon héritier.
\VS{4}Alors la parole de Yahweh lui fut adressée ainsi : Ce n’est pas lui qui sera ton héritier, mais c’est celui qui sortira de tes entrailles qui sera ton héritier.
\VS{5}Puis l'ayant fait sortir dehors, il lui dit : Lève maintenant les yeux au ciel et compte les étoiles si tu peux les compter. Et il lui dit : Ainsi sera ta postérité.
\VS{6}Abram crut à Yahweh qui lui imputa cela à justice\FTNT{Ga. 3:6 ; Ja. 2:23 ; Ro. 4:3.}.
\TextTitle{Yahweh annonce l'esclavage de la postérité d'Abram}
\VS{7}Et il lui dit : Je suis Yahweh qui t'ai fait sortir d'Ur en Chaldée, afin de te donner ce pays-ci pour le posséder.
\VS{8}Abram répondit : Seigneur Yahweh, à quoi connaîtrai-je que je le posséderai ?
\VS{9}Et Yahweh lui répondit : Prends une génisse de trois ans,  une chèvre de trois ans, un bélier de trois ans, une tourterelle, et un pigeon.
\VS{10}Abram prit tous ces animaux, les coupa par le milieu, et mit chaque morceau l’un vis-à-vis de l’autre, mais il ne partagea point les oiseaux.
\VS{11}Les oiseaux de proie descendirent sur les cadavres, mais Abram les chassa.
\VS{12}Au coucher du soleil, un profond sommeil tomba sur Abram, et voici, une frayeur d'une grande obscurité tomba sur lui.
\VS{13}Et Yahweh dit à Abram : Sache comme une chose certaine que tes descendants habiteront quatre cents ans comme étrangers dans un pays qui ne leur appartiendra point, et qu’ils seront asservis aux habitants du pays qui les opprimera\FTNT{Ac. 7:6 ; Ga. 3:17.}.
\VS{14}Mais je jugerai la nation à laquelle ils seront asservis, et après cela ils sortiront avec de grands biens\FTNT{Ex. 3:22.}.
\VS{15}Et toi tu iras vers tes pères en paix, et tu seras enterré après une heureuse vieillesse.
\VS{16}A la quatrième génération, ils reviendront ici ; car l'iniquité des Amoréens n'est pas encore à son comble.
\VS{17}Quand le soleil fut couché, il y eut une obscurité profonde, et voici, ce fut une fournaise fumante, et des flammes passèrent entre les animaux qui avaient été partagés.
\VS{18}En ce jour-là, Yahweh traita alliance avec Abram, en disant : Je donne ce pays à ta postérité, depuis le fleuve d'Egypte jusqu'au grand fleuve, le fleuve d'Euphrate ;
\VS{19}le pays des Kéniens, des Keniziens, des Kadmoniens,
\VS{20}des Héthiens, des Phéréziens, des Rephaïm,
\VS{21}des Amoréens, des Cananéens, des Guirgasiens, et des Jébusiens.
\Chap{16}
\TextTitle{Saraï pousse Abram dans les bras de sa servante}
\VerseOne{}Saraï, femme d'Abram, ne lui avait enfanté aucun enfant, mais elle avait une servante égyptienne nommée Agar.
\VS{2}Et Saraï dit à Abram : Voici, Yahweh m'a rendue stérile ; viens je te prie vers ma servante, peut-être aurai-je des enfants par elle. Et Abram écouta la voix de Saraï.
\VS{3}Alors Saraï, femme d'Abram, prit Agar, sa servante égyptienne, et la donna pour femme à Abram, son mari, après qu’Abram eut habité dix ans dans le pays de Canaan.
\VS{4}Il alla donc vers Agar, et elle conçut. Quand Agar se vit enceinte, elle regarda sa maîtresse avec mépris.
\VS{5}Et Saraï dit à Abram : L'outrage qui m'est fait retombe sur toi. J’ai mis ma servante dans ton sein, mais quand elle a vu qu'elle avait conçu, elle m'a regardée avec mépris. Que Yahweh soit juge entre moi et toi !
\VS{6}Alors Abram répondit à Saraï : Voici, ta servante est entre tes mains, traite-la comme il te plaira. Saraï donc la maltraita, et Agar s'enfuit de devant elle.
\VS{7}Mais l'Ange de Yahweh la trouva auprès d'une fontaine d'eau dans le désert, près de la fontaine qui est sur le chemin de Schur.
\VS{8}Il lui dit : Agar, servante de Saraï, d'où viens-tu ? Et où vas-tu ? Et elle répondit : Je m'enfuis de devant Saraï, ma maîtresse.
\VS{9}L'Ange de Yahweh lui dit : Retourne vers ta maîtresse et humilie-toi sous sa main.
\VS{10}L'Ange de Yahweh lui dit : Je multiplierai beaucoup ta postérité, elle sera si nombreuse qu'on ne pourra la compter.
\VS{11}L'Ange de Yahweh lui dit aussi : Voici, tu as conçu, et tu enfanteras un fils que tu appelleras Ismaël, car Yahweh a entendu ton affliction.
\VS{12}Et ce sera un homme farouche comme un âne sauvage ; sa main sera contre tous, et la main de tous contre lui ; et il habitera en face de tous ses frères.
\VS{13}Alors elle appela Atta-El-roï (tu es le Dieu qui me voit) le nom de Yahweh qui lui avait parlé ; car elle dit : N'ai-je pas même, ici, vu celui qui me voyait ?
\VS{14}C'est pourquoi on a appelé ce puits le puits du vivant qui me voit ; lequel est entre Kadès et Bared.
\TextTitle{Naissance d'Ismaël}
\VS{15}Agar donc enfanta un fils à Abram ; et Abram donna le nom d’Ismaël  au fils qu'Agar lui avait enfanté\FTNT{Ga. 4:22.}.
\VS{16}Abram était âgé de quatre-vingt-six ans quand Agar enfanta Ismaël à Abram.
\Chap{17}
\TextTitle{El Schaddaï (Dieu Tout-Puissant) confirme sa promesse}
\VerseOne{}Lorsqu’Abram fut âgé de quatre-vingt-dix-neuf ans, Yahweh lui apparut et lui dit : Je suis le Dieu Tout-Puissant\FTNT{Dieu se révèle ici à Abraham comme le Dieu Tout-Puissant. Or Christ s’est présenté à l’apôtre Jean comme le Dieu Tout Puissant (Ap. 1:8).  Plus loin en Ap. 5:6, le Seigneur apparaît au milieu du trône céleste sous la forme d’un Agneau ayant sept cornes qui représentent sa toute-puissance. Jésus est bien le Dieu Tout-Puissant qui s’était révélé à Abraham (Da. 8:20-22).}. Marche devant ma face, et sois intègre.
\VS{2}J’établirai mon alliance entre moi et toi, et je te multiplierai très abondamment.
\VS{3}Alors Abram tomba sur sa face, et Dieu lui parla et lui dit :
\TextTitle{Abram devient Abraham}
\VS{4}Quant à moi, voici, mon alliance est avec toi, et tu deviendras père d'une multitude de nations\FTNT{Ro. 4:17.}.
\VS{5}On ne t’appellera plus Abram\FTNT{Né. 9:7.}, mais ton nom sera Abraham ; car je t'ai établi père d'une multitude de nations.
\TextTitle{Promesse d’une alliance éternelle}
\VS{6}Je te rendrai fécond  à l’extrême, et je te ferai devenir des nations ; même des rois sortiront de toi\FTNT{Mt. 1:6.}.
\VS{7}J'établirai donc mon alliance entre moi et toi, et entre ta postérité après toi, selon leurs générations, ce sera une alliance éternelle en vertu de laquelle je serai ton Dieu et celui de ta postérité après toi.
\VS{8}Je te donnerai, et à ta postérité après toi, le pays où tu demeures comme étranger, à savoir tout le pays de Canaan, en possession perpétuelle, et je serai leur Dieu.
\TextTitle{La circoncision, signe de l'alliance}
\VS{9}Dieu dit encore à Abraham : Tu garderas donc mon alliance, toi et ta postérité après toi, selon leurs générations.
\VS{10}C’est ici mon alliance entre moi et vous, et entre ta postérité après toi, que vous garderez : Tout mâle parmi vous sera circoncis.
\VS{11}Vous circoncirez la chair de votre prépuce ; et cela sera le signe de l'alliance entre moi et vous\FTNT{Ac. 7:8 ; Ro. 4:11.}.
\VS{12}Tout enfant mâle de huit jours sera circoncis parmi vous dans vos générations, tant celui qui est né dans la maison que l'esclave acquis à prix d’argent de tout étranger qui n'est point de ta race\FTNT{Lu. 2:21 ; Lé. 12:3.}.
\VS{13}On ne manquera donc point de circoncire celui qui est né dans ta maison, et celui qui est acquis à prix d’argent, et mon alliance sera dans votre chair pour être une alliance perpétuelle.
\VS{14}Et le mâle incirconcis qui n’aura pas été circoncis dans sa  chair sera retranché du milieu de son peuple parce qu'il aura violé mon alliance.
\TextTitle{Saraï devient Sara ; promesse de la naissance d’Isaac}
\VS{15}Dieu dit aussi à Abraham : Quant à Saraï, ta femme, tu n'appelleras plus son nom Saraï, mais son nom sera Sara.
\VS{16}Je la bénirai, et même je te donnerai un fils d'elle. Je la bénirai et elle deviendra des nations ; des rois, chefs de peuples sortiront d'elle.
\VS{17}Alors Abraham se prosterna la face contre terre, et sourit en disant en son cœur : Naîtrait-il un fils à un homme âgé de cent ans ? Et Sara, âgée de quatre-vingt-dix ans, aurait-elle un enfant ?
\VS{18}Et Abraham dit à Dieu : Je te prie, qu'Ismaël vive devant toi.
\VS{19}Et Dieu dit : Certainement Sara, ta femme, t'enfantera un fils, et tu appelleras son nom Isaac ; et j'établirai mon alliance avec lui pour être une alliance perpétuelle pour sa postérité après lui.
\TextTitle{Une nation sortira d'Ismaël}
\VS{20}Je t'ai aussi exaucé touchant Ismaël : Voici, je le bénirai, et je le ferai croître et multiplier très abondamment. Il engendrera douze princes, et je le ferai devenir une grande nation.
\VS{21}Mais j'établirai mon alliance avec Isaac, que Sara t'enfantera l'année qui vient, en cette même saison.
\VS{22}Et Dieu ayant achevé de parler, s’éleva au-dessus d'Abraham.
\VS{23}Et Abraham prit son fils Ismaël, avec tous ceux qui étaient nés dans sa maison, et tous ceux qu'il avait acquis à prix d’argent, tous les mâles qui étaient des gens de sa maison, et il circoncit la chair de leur prépuce en ce même jour-là, comme Dieu le lui avait dit.
\VS{24}Abraham était âgé de quatre-vingt-dix-neuf ans quand il circoncit la chair de son prépuce ;
\VS{25}et Ismaël, son fils, était âgé de treize ans lorsqu'il fut circoncis.
\VS{26}En ce même jour, Abraham fut circoncis, et son fils Ismaël aussi.
\VS{27}Et tous les gens de sa maison, tant ceux qui étaient nés dans sa maison que ceux qui avaient été acquis à prix d’argent des étrangers, furent circoncis avec lui.
\Chap{18}
\TextTitle{Abraham, ami de Yahweh\FTNTT{Jn 3:29 ; 15:13-15}}
\VerseOne{}Puis Yahweh lui apparut dans les plaines de Mamré, comme il était assis à la porte de sa tente, pendant la chaleur du jour.
\VS{2}Levant ses yeux, il regarda : Et voici, trois hommes parurent devant lui. Quand il les vit, il courut au-devant d'eux depuis la porte de sa tente, et se prosterna à terre\FTNT{Hé. 13:2.} ;
\VS{3}Et il dit : Mon Seigneur, je te prie, si j'ai trouvé grâce devant tes yeux, ne passe point outre, je te prie, et arrête-toi chez ton serviteur.
\VS{4}Qu'on prenne, je vous prie, un peu d'eau, et lavez vos pieds, et reposez-vous sous un arbre.
\VS{5}J’apporterai un morceau de pain pour fortifier votre cœur, après quoi vous passerez outre ; car c'est pour cela que vous êtes venus vers votre serviteur. Et ils dirent : Fais ce que tu as dit.
\VS{6}Abraham donc s'en alla en hâte dans la tente vers Sara, et lui dit : Hâte-toi, prends trois mesures de fleur de farine, pétris-les, et fais des gâteaux.
\VS{7}Puis Abraham courut au troupeau et prit un veau tendre et bon, et le donna à un serviteur qui se hâta de l'apprêter.
\VS{8}Ensuite, il prit du beurre et du lait, et le veau qu'on avait apprêté, et le mit devant eux ; et il se tint auprès d'eux sous l'arbre, et ils mangèrent.
\VS{9}Et ils lui dirent : Où est Sara ta femme ? Et il répondit : La voilà dans la tente.
\VS{10}Et l'un d'entre eux dit : Je ne manquerai pas de revenir vers toi en ce même temps où nous sommes, et voici, Sara, ta femme, aura un fils. Et Sara écoutait à la porte de la tente qui était derrière lui\FTNT{Ro. 9:9.}.
\VS{11}Or Abraham et Sara étaient vieux, fort avancés en âge ; et Sara n'avait plus ce que les femmes sont accoutumées d'avoir\FTNT{Ro. 4:19 ; Hé. 11:11.}.
\VS{12}Et Sara rit en elle-même et dit : Etant vieille, et mon Seigneur étant fort âgé, aurai-je encore des désirs ?
\VS{13}Et Yahweh dit à Abraham : Pourquoi Sara a-t-elle ri en disant : Serait-il vrai que j'aurais un enfant, étant vieille comme je suis ?
\VS{14}Y a-t-il quelque chose qui soit difficile à Yahweh ? Je reviendrai vers toi à cette époque, en ce même temps où nous sommes et Sara aura un fils\FTNT{Mt. 19:26 ; Lu. 1:37.}.
\VS{15}Et Sara le nia en disant : Je n'ai point ri ; car elle avait peur. Mais il dit : Cela n'est pas, car tu as ri.
\VS{16}Et ces hommes se levèrent de là, et regardèrent vers Sodome ; et Abraham alla  avec eux pour les accompagner.
\VS{17}Et Yahweh dit : Cacherai-je à Abraham ce que je vais faire ?
\VS{18}Abraham deviendra certainement une nation grande et puissante, et toutes les nations de la terre seront bénies en lui\FTNT{Ac. 3:25 ; Ga. 3:8.}.
\VS{19}Car je le connais, et je sais qu'il ordonnera à ses enfants, et à sa maison après lui, de garder la voie de Yahweh, pour faire ce qui est juste et droit ; afin que Yahweh fasse venir sur Abraham tout ce qu'il lui a dit.
\VS{20}Et Yahweh dit : Le cri contre Sodome et Gomorrhe s’est accru, et leur péché s’est fort aggravé.
\VS{21}Je descendrai maintenant, et je verrai s'ils ont fait entièrement selon le cri qui est venu jusqu'à moi ; et si cela n'est pas, je le saurai.
\VS{22}Ces hommes donc partant de là allèrent vers Sodome ; mais Abraham se tint encore devant Yahweh.
\TextTitle{Intercession d'Abraham}
\VS{23}Et Abraham s'approcha et dit : Feras-tu périr le juste avec le méchant ?
\VS{24}Peut-être y a-t-il cinquante justes dans la ville, les feras-tu périr aussi ? Ne pardonneras-tu point à la ville à cause des cinquante justes qui sont au milieu d’elle ?
\VS{25}Non, il n'arrivera pas que tu fasses une telle chose, que tu fasses mourir le juste avec le méchant, et que le juste soit traité comme le méchant ! Non, tu ne le feras point. Celui qui juge toute la terre ne fera-t-il point justice\FTNT{Ro. 3:5-6.} ?
\VS{26}Et Yahweh dit : Si je trouve dans Sodome cinquante justes au milieu de la ville, je pardonnerai à toute la ville à cause d'eux.
\VS{27}Et Abraham répondit en disant : Voici, j'ai pris maintenant la hardiesse de parler au Seigneur, moi qui ne suis que poussière et cendres.
\VS{28}Peut-être en manquera-t-il cinq des cinquante justes ; détruiras-tu toute la ville pour ces cinq-là ? Et Yahweh lui répondit : Je ne la détruirai point si j'y trouve quarante-cinq justes.
\VS{29}Abraham continua de lui parler en disant : Peut-être s'y trouvera-t-il quarante ? Et il dit : Je ne la détruirai point pour l'amour des quarante.
\VS{30}Abraham dit : Je prie le Seigneur de ne pas s'irriter si je parle encore. Peut-être s'en trouvera-t-il trente ? Et il dit : Je ne la détruirai point si j'y trouve trente.
\VS{31}Abraham dit : Voici, maintenant j'ai pris la hardiesse de parler au Seigneur : Peut-être s'en trouvera-t-il vingt ? Et il dit : Je ne la détruirai point pour l'amour des vingt.
\VS{32}Abraham dit : Je prie le Seigneur de ne pas s'irriter, je parlerai encore une seule fois : Peut-être s'y trouvera-t-il dix. Et Yahweh dit : Je ne la détruirai point pour l'amour des dix.
\VS{33}Yahweh s'en alla quand il eut achevé de parler avec Abraham. Et Abraham retourna dans sa demeure.
\Chap{19}
\TextTitle{Des anges chez Lot\FTNTT{Ge. 13:10, 12 ; 19:33}}
\VerseOne{}Sur le soir, les deux anges arrivèrent à Sodome, et Lot était assis à la porte de Sodome. Quand Lot les vit, il se leva pour aller au-devant d'eux, et se prosterna la face contre terre.
\VS{2}Et il leur dit : Voici, je vous prie, mes seigneurs, entrez maintenant dans la maison de votre serviteur, et passez-y la nuit ; lavez-vous les pieds ; puis vous vous lèverez dès le matin et continuerez votre chemin ; et ils dirent : Non, mais nous passerons la nuit dans la rue.
\VS{3}Mais il les pressa tellement qu'ils se retirèrent chez lui ; et quand ils furent entrés dans sa maison, il leur fit un festin, et fit cuire des pains sans levain, et ils mangèrent.
\VS{4}Ils n’étaient pas encore couchés que les hommes de la ville, les hommes de Sodome, environnèrent la maison, depuis les plus jeunes jusqu'aux vieillards, tout le peuple était ensemble.
\VS{5}Ils appelèrent Lot et ils lui dirent : Où sont les hommes qui sont venus cette nuit chez toi ? Fais-les sortir afin que nous les connaissions.
\VS{6}Mais Lot sortit de sa maison pour leur parler à la porte, et ayant fermé la porte après lui,
\VS{7}il leur dit : Je vous prie, mes frères, ne leur faites point de mal.
\VS{8}Voici, j'ai deux filles qui n'ont point encore connu d'homme ; je vous les amènerai et vous les traiterez comme il vous plaira. Seulement, ne faites pas de mal à ces hommes, car ils sont venus à l'ombre de mon toit.
\VS{9}Ils lui dirent : Retire-toi de là. Ils dirent aussi : Cet homme seul est venu pour habiter ici comme étranger, et il veut nous gouverner ? Maintenant nous te ferons pis qu'à eux. Et faisant violence à Lot,  ils s'approchèrent pour briser la porte\FTNT{2 Pi. 2:7-8.}.
\VS{10}Mais les hommes étendirent leurs mains, firent rentrer Lot vers eux dans la maison, et fermèrent la porte.
\VS{11}Et ils frappèrent d’aveuglement les hommes qui étaient à la porte de la maison, depuis le plus petit jusqu'au plus grand, de sorte qu'ils se lassèrent à chercher la porte.
\VS{12}Alors ces hommes dirent à Lot : Qui as-tu encore ici qui t'appartienne ? Gendres, fils et filles, et tout ce qui t'appartient dans la ville, fais-les sortir de ce lieu.
\VS{13}Car nous allons détruire ce lieu parce que le cri contre ses habitants est grand devant Yahweh. Yahweh nous a envoyés pour le détruire.
\VS{14}Lot sortit donc et parla à ses gendres, qui devaient prendre ses filles, et leur dit : Levez-vous, sortez de ce lieu, car Yahweh va détruire la ville. Mais aux yeux de ses gendres, il parut plaisanter.
\TextTitle{Jugement sur Sodome}
\VS{15}Dès l’aube du jour, les anges pressèrent Lot en disant : Lève-toi, prends ta femme et tes deux filles qui se trouvent ici, de peur que tu ne périsses dans le châtiment de la ville.
\VS{16}Et comme il tardait, ces hommes le prirent par la main, et ils prirent aussi par la main sa femme et ses deux filles, parce que Yahweh voulait l'épargner ; et ils l'emmenèrent et le mirent hors de la ville.
\VS{17}Après les avoir fait sortir, l'un d’eux dit : Sauve ta vie, ne regarde point derrière toi, et ne t'arrête en aucun endroit de la plaine ; sauve-toi sur la montagne, de peur que tu ne périsses.
\VS{18}Lot leur répondit : Non, Seigneur, je te prie.
\VS{19}Voici, ton serviteur a maintenant trouvé grâce devant toi, et tu as montré la grandeur de ta bonté à mon égard en préservant ma vie, mais je ne pourrai pas me sauver vers la montagne avant que le mal ne m'atteigne, et je mourrai.
\VS{20}Voici, je te prie, cette ville-là est proche ; je puis m'y enfuir, et elle est petite. Je te prie, que je m'y sauve ; n'est-elle pas petite ? Et mon âme vivra.
\VS{21}Et il lui dit : Voici, je t'ai exaucé encore en cela, de ne point détruire la ville dont tu as parlé.
\VS{22}Hâte-toi, sauve-toi là, car je ne pourrai rien faire jusqu'à ce que tu y sois entré ; c'est pourquoi cette ville fut appelée Tsoar.
\VS{23}Comme le soleil se levait sur la terre, Lot entra dans Tsoar.
\VS{24}Alors Yahweh fit pleuvoir du ciel, sur Sodome et sur Gomorrhe, du soufre et du feu, de la part de Yahweh\FTNT{De. 29:23 ; Lu. 17:29 ; Jud. 1:7.} ;
\VS{25}et il détruisit ces villes-là, et toute la plaine, et tous les habitants des villes, et les herbes de la terre.
\VS{26}Mais la femme de Lot regarda en arrière, et elle devint une statue de sel\FTNT{Lu. 17:31-33.}.
\VS{27}Abraham se leva de bon matin et vint au lieu où il s'était tenu devant Yahweh ;
\VS{28}et regardant vers Sodome et Gomorrhe, et vers toute la terre de cette plaine-là, il vit monter de la terre une fumée comme la fumée d'une fournaise.
\VS{29}Lorsque Dieu détruisit les villes de la plaine, il se souvint d'Abraham, et laissa Lot s’en aller  du milieu du désastre par lequel il détruisit les villes où Lot avait établi sa demeure.
\TextTitle{Une abomination commise dans la famille de Lot\FTNTT{Ge. 13:10,12 ; 19:1 ; Lu. 22:31-62}}
\VS{30}Lot quitta Tsoar et habita sur la montagne avec ses deux filles, car il craignait de demeurer dans Tsoar, et il se retira dans une caverne avec ses deux filles.
\VS{31}L'aînée dit à la plus jeune : Notre père est vieux, et il n'y a personne sur la terre pour venir vers nous, selon la coutume de tous les pays.
\VS{32}Viens, donnons du vin à notre père, et couchons avec lui  afin que nous conservions la race de notre père.
\VS{33}Elles donnèrent donc du vin à boire à leur père cette nuit-là ; et l'aînée vint, et coucha avec son père, mais il ne s'aperçut point ni quand elle se coucha ni quand elle se leva.
\VS{34}Le lendemain, l'aînée dit à la plus jeune : Voici, j'ai couché la nuit dernière avec mon père, donnons-lui encore du vin à boire cette nuit, puis va et couche avec lui, et nous conserverons la race de notre père.
\VS{35}Elles firent boire du vin à leur père encore cette nuit-là ; et la plus jeune se leva et coucha avec lui ; mais il ne s'aperçut point ni quand elle se coucha ni quand elle se leva.
\VS{36}Ainsi, les deux filles de Lot conçurent de leur père.
\VS{37}L’aînée enfanta un fils qu’elle appela du nom de Moab ; c'est le père des Moabites jusqu'à ce jour.
\VS{38}La plus jeune aussi enfanta un fils qu’elle appela du nom de Ben-Ammi ; c'est le père des Ammonites jusqu'à ce jour.
\Chap{20}
\TextTitle{Faute d'Abraham à Guérar\FTNTT{Ge. 26:6-32}}
\VerseOne{}Abraham s'en alla de là pour le pays du midi ; il demeura entre Kadès et Schur, et il habita comme étranger à Guérar.
\VS{2}Abraham disait de Sara sa femme : C'est ma sœur. Et Abimélec, roi de Guérar, envoya des gens prendre Sara.
\VS{3}Mais Dieu apparut la nuit dans un songe à Abimélec, et lui dit : Voici, tu vas mourir, à cause de la femme que tu as prise, car elle a un mari.
\VS{4}Abimélec, qui ne s'était point approché d'elle, répondit : Seigneur, feras-tu donc mourir une nation juste ?
\VS{5}Ne m'a-t-il pas dit : C'est ma sœur? Et elle-même aussi n'a-t-elle pas dit : C'est mon frère ? J'ai fait ceci dans l'intégrité de mon cœur et dans la pureté de mes mains.
\VS{6}Dieu lui dit en songe : Je sais que tu l'as fait dans l'intégrité de ton cœur, aussi ai-je empêché que tu ne pèches contre moi ; c'est pourquoi je n'ai pas permis que tu la touches.
\VS{7}Maintenant donc rends la femme de cet homme, car il est prophète ; et il priera pour toi et tu vivras. Mais si tu ne la rends pas, sache que tu mourras toi et tout ce qui est à toi.
\VS{8}Abimélec se leva de bon matin, appela tous ses serviteurs, et rapporta à leurs oreilles  toutes ces choses, et ils furent saisis de crainte.
\VS{9}Puis Abimélec appela Abraham et lui dit : Que nous as-tu fait ? Et en quoi t'ai-je offensé que tu aies fait venir sur moi et sur mon royaume un grand péché ? Tu m'as fait des choses qui ne doivent point se faire.
\VS{10}Abimélec dit aussi à Abraham : Qu'as-tu vu qui t'aie obligé de faire cela ?
\VS{11}Abraham répondit : C'est parce que je disais : Assurément, il n'y a point de crainte de Dieu dans ce pays, et ils me tueront à cause de ma femme.
\VS{12}De plus, il est vrai qu’elle est ma soeur, fille de mon père ; mais elle n'est pas fille de ma mère ; et elle m'a été donnée pour femme.
\VS{13}Lorsque Dieu me fit errer loin de la maison de mon père, je dis à Sara : Voici la grâce que tu me feras, dis de moi dans tous les lieux où nous irons : C'est mon frère.
\VS{14}Alors Abimélec prit des brebis, des bœufs, des serviteurs et des servantes, et les donna à Abraham, et lui rendit Sara, sa femme.
\VS{15}Abimélec lui dit : Voici, mon pays est à ta disposition, demeure où il te plaira.
\VS{16}Et il dit à Sara : Voici, je donne à ton frère mille pièces d'argent ; cela te sera un voile sur les yeux  pour tous ceux qui sont avec toi, et envers tous les autres ; et ainsi elle fut reprise.
\VS{17}Abraham pria Dieu, et Dieu guérit Abimélec, sa femme, et ses servantes ; et elles eurent des enfants.
\VS{18}Car Yahweh avait frappé de stérilité en  fermant toute matrice de la maison d'Abimélec, à cause de Sara, femme d'Abraham.
\Chap{21}
\TextTitle{Naissance d'Isaac}
\VerseOne{}Et Yahweh visita Sara, comme il avait dit ; et il agit selon ses paroles.
\VS{2}Sara donc conçut, et enfanta un fils à Abraham dans sa vieillesse, au temps précis que Dieu lui avait dit.
\VS{3}Abraham donna le nom d’Isaac au fils qui lui était né, que Sara lui avait enfanté.
\VS{4}Abraham circoncit son fils Isaac âgé de huit jours, comme Dieu le lui avait ordonné.
\VS{5}Abraham était âgé de cent ans quand Isaac, son fils, lui naquit.
\VS{6}Et Sara dit : Dieu m'a donné de quoi rire ; tous ceux qui l'apprendront riront avec moi.
\VS{7}Elle dit aussi : Qui aurait dit à Abraham que Sara allaiterait des enfants ? Car je lui ai enfanté un fils dans sa vieillesse.
\VS{8}L'enfant grandit et fut sevré ; et Abraham fit un grand festin le jour où Isaac fut sevré.
\TextTitle{Abraham chasse Agar avec Ismaël\FTNTT{Ga. 4:21-31}}
\VS{9}Sara vit rire le fils qu’Agar, l’Egyptienne, avait enfanté à Abraham ;
\VS{10}et elle dit à Abraham : Chasse cette servante et son fils, car le fils de cette servante n'héritera point avec mon fils, avec Isaac\FTNT{Ga. 4:30.}.
\VS{11}Cette parole déplut fort à Abraham à cause de son fils.
\VS{12}Mais Dieu dit à Abraham : N'aie point de chagrin au sujet de l'enfant ni de ta servante ;  écoute la parole de Sara dans toutes les choses qu’elle te dira, car en Isaac te sera donnée une postérité.
\VS{13}Je ferai aussi devenir le fils de la servante une nation, parce qu'il est ta semence.
\VS{14}Puis Abraham se leva de bon matin et prit du pain et une outre d'eau, et il les donna à Agar en les mettant sur son épaule. Il lui donna aussi l'enfant et la renvoya. Elle se mit en chemin et fut errante au désert de Beer-Schéba.
\VS{15}Quand l'eau de l’outre fut épuisée, elle jeta l'enfant sous un arbrisseau,
\VS{16}et elle alla s’asseoir vis-à-vis, à une portée d’arc, car elle dit : Que je ne voie pas mourir mon enfant. Elle s’assit donc vis-à-vis de lui, éleva la voix et pleura.
\VS{17}Dieu entendit la voix de l'enfant, et l'Ange de Dieu appela des cieux Agar et lui dit : Qu'as-tu Agar ? Ne crains point, car Dieu a entendu la voix de l'enfant du lieu où il est.
\VS{18}Lève-toi, lève l'enfant, et prends-le par la main, car je le ferai devenir une grande nation.
\VS{19}Et Dieu lui ouvrit les yeux et elle vit un puits d'eau ; elle alla remplir d'eau l’outre, et donna à boire à l'enfant.
\VS{20}Dieu fut avec l'enfant, qui devint grand, et demeura dans le désert ; et il fut tireur d'arc.
\VS{21}Il habita dans le désert de Paran ; et sa mère lui prit une femme du pays d'Egypte.
\TextTitle{Abraham à Beer-Schéba}
\VS{22}Et il arriva en ce temps-là qu'Abimélec, et Picol, chef de son armée, parla à Abraham en disant : Dieu est avec toi dans toutes les choses que tu fais.
\VS{23}Maintenant donc jure-moi ici par le nom de Dieu que tu ne me mentiras point, ni à mes enfants ni aux enfants de mes enfants, et que selon la faveur que je t'ai faite, tu agiras envers moi et envers le pays où tu séjournes comme étranger.
\VS{24}Abraham répondit : Je te le jurerai.
\VS{25}Mais Abraham fit des reproches à Abimélec au sujet d'un puits d'eau, dont les serviteurs d'Abimélec s'étaient emparés de force.
\VS{26}Abimélec répondit : J’ignore qui a fait cela, et aussi tu ne m'en as point informé, et moi, je ne l’apprends qu’aujourd’hui.
\VS{27}Alors Abraham prit des brebis et des bœufs, et les donna à Abimélec, et ils firent alliance ensemble.
\VS{28}Abraham mit à part sept jeunes brebis de son troupeau.
\VS{29}Et Abimélec dit à Abraham : Que veulent dire ces sept jeunes brebis que tu as mises à part ?
\VS{30}Il répondit : C'est que tu prendras ces sept jeunes brebis de ma main pour me servir de témoignage que j'ai creusé ce puits.
\VS{31}C'est pourquoi on appela ce lieu-là Beer-Schéba, car tous deux y jurèrent.
\VS{32}Ils traitèrent donc alliance à Beer-Schéba, puis Abimélec se leva avec Picol, chef de son armée, et ils retournèrent au pays des Philistins.
\VS{33}Abraham planta des tamaris à Beer-Schéba ; et là il invoqua le nom de Yahweh, le Dieu de l’éternité.
\VS{34}Abraham séjourna beaucoup de jours comme étranger dans le pays des Philistins.
\Chap{22}
\TextTitle{Abraham présente Isaac en sacrifice\FTNTT{Hé. 11:17-19}}
\VerseOne{}Or, il arriva après ces choses, que Dieu éprouva Abraham et lui dit : Abraham ! Et il répondit : Me voici.
\VS{2}Et Dieu lui dit : Prends maintenant ton fils, ton unique, celui que tu aimes, Isaac, et va-t'en au pays de Morija, et là offre-le en holocauste sur l'une des montagnes que je te dirai.
\VS{3}Abraham donc s'étant levé de bon matin, sella son âne, et prit deux de ses serviteurs avec lui, et Isaac son fils ; et ayant fendu le bois pour l'holocauste, il se mit en chemin et s'en alla au lieu que Dieu lui avait dit.
\VS{4}Le troisième jour, Abraham levant ses yeux, vit le lieu de loin.
\VS{5}Et Abraham dit à ses serviteurs : Restez ici avec l'âne ; moi et l'enfant nous irons jusque-là pour adorer, après quoi nous reviendrons auprès de vous.
\VS{6}Abraham prit le bois de l'holocauste et le mit sur Isaac, son fils, et prit le feu dans sa main, et un couteau ; et ils s'en allèrent tous deux ensemble.
\VS{7}Alors Isaac parla à Abraham, son père, et dit : Mon père ! Abraham répondit : Me voici mon fils. Et il dit : Voici le feu et le bois, mais où est l’agneau pour l'holocauste\FTNT{Isaac est un autre type de Christ qui s’offre en sacrifice pour l’expiation de nos péchés. La réponse à sa question au v. 7:«~Voici le feu et le bois, mais où est l’agneau pour l’holocauste ?~», a été apportée bien des siècles plus tard par Jean-Baptiste:«~Voici l’agneau de Dieu, qui ôte le péché du monde~». (Jn. 1:29).} ?
\VS{8}Abraham répondit : Mon fils, Dieu se pourvoira lui-même de l’agneau pour l'holocauste. Et ils marchèrent tous deux ensemble.
\VS{9}Et étant arrivés au lieu que Dieu lui avait dit, Abraham bâtit là un autel, et rangea le bois, et ensuite il lia Isaac, son fils, et le mit sur l'autel, par-dessus le bois\FTNT{Ja. 2:21.}.
\VS{10}Puis Abraham étendit sa main et prit le couteau pour égorger son fils.
\VS{11}Mais l'Ange de Yahweh l’appela des cieux et dit : Abraham, Abraham ! Il répondit : Me voici.
\VS{12}L’Ange lui dit : Ne porte pas ta main sur l'enfant, et ne lui fais rien ; car maintenant je sais que tu crains Dieu, puisque tu ne m’as point refusé ton fils, ton unique.
\VS{13}Abraham leva les yeux et regarda ; et voici,  il vit derrière lui un bélier qui était retenu à un buisson par ses cornes ; et Abraham alla prendre le bélier et l'offrit en holocauste à la place de son fils.
\VS{14}Abraham donna à ce lieu le nom de Yahweh-Jiré (Yahweh pourvoira) ; c'est pourquoi on dit aujourd'hui : Dans la montagne de Yahweh il y sera pourvu.
\VS{15}L'Ange de Yahweh appela des cieux Abraham pour la seconde fois,
\VS{16}et dit : Je le jure par moi-même\FTNT{Hé. 6:13-15.}, parole de Yahweh ! Parce que tu as fait cela, et que tu n'as point refusé ton fils, ton unique,
\VS{17}certainement je te bénirai, et je multiplierai très abondamment ta postérité, comme les étoiles du ciel et comme le sable qui est sur le bord de la mer ; et ta postérité possédera la porte de ses ennemis.
\VS{18}Toutes les nations de la terre seront bénies en ta postérité, parce que tu as obéi à ma voix.
\VS{19}Ainsi Abraham retourna vers ses serviteurs, et ils se levèrent et s'en allèrent ensemble à Beer-Schéba ; car Abraham demeurait à Beer-Schéba.
\VS{20}Après ces choses, quelqu'un apporta des nouvelles à Abraham, en disant : Voici, Milca a aussi enfanté des fils à Nachor, ton frère.
\VS{21}Uts, son premier-né, et Buz, son frère, Kemuel, père d'Aram,
\VS{22}Késed, Hazo, Pildasch, Jidlaph et Bethuel.
\VS{23}Bethuel a engendré Rebecca. Milca enfanta ces huit fils à Nachor, frère d'Abraham.
\VS{24}Sa concubine, nommée Réuma, enfanta aussi Thébach, Gaham, Tahasch, et Maaca.
\Chap{23}
\TextTitle{Mort de Sara}
\VerseOne{}Or, Sara vécut cent vingt-sept ans ; ce sont là les années de la vie de Sara.
\VS{2}Sara mourut à Kirjath-Arba, qui est Hébron, dans le pays de Canaan ; et Abraham vint pour mener deuil sur Sara et pour la pleurer.
\VS{3}Et Abraham se leva de devant son mort, il parla aux fils de Heth, en disant :
\VS{4}Je suis étranger et habitant parmi vous ; donnez-moi une possession de sépulcre parmi vous, afin que j'enterre mon mort et que je l'ôte de devant moi\FTNT{Ac. 7:5.}.
\VS{5}Les fils de Heth répondirent à Abraham et lui dirent :
\VS{6}Mon seigneur, écoute-nous ! Tu es un prince de Dieu parmi nous, enterre ton mort dans le plus distingué de nos sépulcres ; nul de nous ne te refusera son sépulcre afin que tu y enterres ton mort.
\VS{7}Alors Abraham se leva et se prosterna devant le peuple du pays, devant les Héthiens.
\VS{8}Et il leur parla et dit : S'il vous plaît que j'enterre mon mort et que je l'ôte de devant moi ; écoutez-moi, et intercédez pour moi envers Ephron, fils de Tsochar,
\VS{9}afin qu'il me cède sa caverne de Macpéla, qui est à l’extrémité de son champ ; qu'il me la cède contre sa valeur en argent, afin qu’elle me serve de possession sépulcrale au milieu de vous.
\VS{10}Ephron était assis parmi les fils de Heth. Et Ephron, l’Héthien, répondit à Abraham, en présence des fils de Heth qui l'écoutaient, devant tous ceux qui entraient par la porte de sa ville, et dit :
\VS{11}Non, mon seigneur, écoute-moi ! Je te donne le champ, je te donne aussi la caverne qui y est, je te la donne en présence des enfants de mon peuple ; enterres-y ton mort.
\VS{12}Abraham se prosterna devant le peuple du pays.
\VS{13}Et il parla ainsi à Ephron, en présence de tout le peuple du pays qui écoutait et dit : S'il te plaît, je te prie, écoute-moi ! Je donnerai l'argent du champ ; reçois-le de moi, et j'y enterrerai mon mort.
\VS{14}Et Ephron répondit à Abraham, en disant :
\VS{15}Mon seigneur, écoute-moi ! La terre vaut quatre cents sicles d'argent, qu’est-ce que cela entre moi et toi ? Enterre donc ton mort.
\VS{16}Abraham ayant entendu Ephron, lui paya l'argent dont il avait parlé, en présence des fils de Heth, à savoir quatre cents sicles d'argent ayant cours chez les marchands\FTNT{Ac. 7:16.}.
\VS{17}Le champ d'Ephron, qui était à Macpéla, vis-à-vis de Mamré, le champ et la caverne qui y est, et tous les arbres qui sont dans le champ et dans toutes ses limites alentour,
\VS{18}tout fut acquis comme propriété d’Abraham, en présence des fils de Heth, et de tous ceux qui entraient par la porte de la ville.
\VS{19}Après cela, Abraham enterra Sara, sa femme, dans la caverne du champ de Macpéla, vis-à-vis de Mamré, qui est Hébron, dans le pays de Canaan.
\VS{20}Le champ et la caverne qui y est demeurèrent à Abraham comme possession sépulcrale, acquise des fils de Heth.
\Chap{24}
\TextTitle{Abraham recherche une épouse pour Isaac}
\VerseOne{}Or, Abraham devint vieux et fort avancé en âge ; et Yahweh avait béni Abraham en toute chose.
\VS{2}Abraham dit à son serviteur, le plus ancien des serviteurs de sa maison, l’intendant de tout ce qui lui appartenait : Mets, je te prie, ta main sous ma cuisse ;
\VS{3}et je te ferai jurer par Yahweh, le Dieu du ciel et le Dieu de la terre, que tu ne prendras point de femme pour mon fils parmi les filles des Cananéens, au milieu desquels j'habite.
\VS{4}Mais tu iras dans mon pays et vers mes parents, et tu y prendras une femme pour mon fils Isaac.
\VS{5}Le serviteur lui répondit : Peut-être que la femme ne voudra-t-elle pas me suivre dans ce pays ; me faudra-t-il nécessairement ramener ton fils dans le pays d'où tu es sorti ?
\VS{6}Abraham lui dit : Garde-toi bien d'y ramener mon fils !
\VS{7}Yahweh, le Dieu du ciel, qui m'a fait sortir de la maison de mon père et de ma patrie, qui m'a parlé et qui m’a juré en disant : Je donnerai ce pays à ta postérité, enverra lui-même son ange devant toi ; et c’est là que tu prendras une femme pour mon fils.
\VS{8}Si la femme ne veut pas te suivre, tu seras quitte de ce serment que je te fais faire. Quoi qu'il en soit, tu n’y ramèneras point mon fils.
\VS{9}Le serviteur mit la main sous la cuisse d'Abraham, son Seigneur, et lui jura d’observer ces choses.
\VS{10}Alors le serviteur prit dix chameaux parmi les chameaux de son maître, et s'en alla, ayant à sa disposition tous les biens. Il partit donc et s'en alla en Mésopotamie, à la ville de Nachor.
\VS{11}Il fit reposer les chameaux sur leurs genoux hors de la ville, près d'un puits d'eau, sur le soir, au temps où sortent celles qui vont puiser de l'eau.
\VS{12}Et il dit : Ô Yahweh, Dieu de mon seigneur Abraham, fais que j'aie une heureuse rencontre aujourd'hui, et sois favorable à mon seigneur Abraham.
\VS{13}Voici, je me tiens près de la source d'eau, et les filles des gens de la ville vont sortir pour puiser de l'eau.
\VS{14}Fais donc que la jeune fille à laquelle je dirai : Penche ta cruche, je te prie, afin que je boive, et qui me répondra : Bois, et je donnerai aussi à boire à tes chameaux, soit celle que tu as destinée à ton serviteur Isaac, et par là je connaîtrai que tu es favorable à mon seigneur.
\VS{15}Il n’avait pas encore fini de parler que sortit sa cruche sur l’épaule, Rebecca, fille de Bethuel, fils de Milca, femme de Nachor, frère d'Abraham.
\VS{16}Et la jeune fille était très belle de figure ; elle était vierge, et aucun homme ne l'avait connue. Elle descendit donc à la source, et comme elle remontait après avoir rempli sa cruche,
\VS{17}le serviteur courut au-devant d'elle et lui dit : Laisse-moi boire, je te prie, un peu d’eau de ta cruche.
\VS{18}Elle répondit : Mon seigneur, bois. Elle s’empressa d’abaisser sa cruche sur sa main,  et elle lui donna à boire.
\VS{19}Quand elle eut achevé de lui donner à boire, elle dit : Je puiserai aussi pour tes chameaux jusqu'à ce qu'ils aient achevé de boire.
\VS{20}Et elle s’empressa de vider sa cruche dans l’abreuvoir ; elle courut encore au puits pour puiser de l'eau, et elle puisa pour tous ses chameaux.
\VS{21}L’homme la regardait avec étonnement et sans rien dire, pour voir si Yahweh faisait réussir son voyage ou non.
\VS{22}Quand les chameaux eurent fini de boire, l’homme prit un anneau d'or, du poids d'un demi-sicle, et deux bracelets, pour les mettre sur les mains de cette fille, pesant dix sicles d'or.
\VS{23}Et il lui dit : De qui es-tu fille ? Je te prie, fais-le-moi savoir. Y a-t-il dans la maison de ton père de la place pour nous loger ?
\VS{24}Elle lui répondit : Je suis fille de Bethuel, fils de Milca et de Nachor.
\VS{25}Elle lui dit encore : Il y a chez nous de la paille et du fourrage en abondance, et de la place pour loger.
\VS{26}Alors l’homme s'inclina et adora Yahweh,
\VS{27}et dit : Béni soit Yahweh, le Dieu de mon seigneur Abraham, qui n'a point cessé d'exercer sa bonté et sa fidélité envers mon Seigneur ! Lorsque j'étais en chemin, Yahweh m'a conduit dans la maison des frères de mon seigneur.
\VS{28}La jeune fille courut et rapporta toutes ces choses à la maison de sa mère.
\VS{29}Rebecca avait un frère nommé Laban, qui courut dehors vers l’homme près de la source.
\VS{30}Il avait vu l’anneau et les bracelets aux mains de sa sœur, et il avait entendu les paroles de Rebecca sa sœur, disant : Ainsi m’a parlé l’homme. Il vint donc à cet homme qui se tenait auprès des chameaux, près de la source,
\VS{31}et il lui dit : Entre, béni de Yahweh ! Pourquoi te tiens-tu dehors ? J'ai préparé la maison et une place pour tes chameaux.
\VS{32}L'homme donc entra dans la maison. Laban fit décharger les chameaux, et il donna de la paille et du fourrage  aux chameaux ; et il apporta de l'eau pour laver les pieds de l’homme et les pieds de ceux qui étaient avec lui.
\VS{33}Et il lui présenta à manger. Mais il dit : Je ne mangerai point avant d’avoir dit  ce que j'ai à dire. Parle ! dit Laban.
\VS{34}Alors il dit : Je suis serviteur d'Abraham.
\VS{35}Yahweh a comblé de bénédictions mon seigneur qui est devenu puissant. Il lui a donné des brebis, des bœufs, de l'argent, de l'or, des serviteurs, des servantes, des chameaux, et des ânes.
\VS{36}Sara, la femme de mon seigneur, a enfanté dans sa vieillesse un fils à mon seigneur ; et il lui a donné tout ce qu'il possède.
\VS{37}Mon seigneur m'a fait jurer en disant : Tu ne prendras point de femme pour mon fils parmi les filles des Cananéens dans le pays desquels j’habite ;
\VS{38}mais tu iras dans la maison de mon père et de ma famille prendre une femme pour mon fils.
\VS{39}J’ai dit à mon seigneur : Peut-être que la femme ne voudra-t-elle pas me suivre.
\VS{40}Et il m’a répondu : Yahweh, devant la face de qui j'ai marché, enverra son ange avec toi, et fera réussir ton voyage ; et tu prendras pour mon fils une femme de ma famille et de la maison de mon père.
\VS{41}Quand tu auras été vers ma famille, tu seras alors dégagé de la punition du serment que je te fais faire ; et si on ne te la donne pas, tu seras dégagé de la punition du serment que je te fais faire.
\VS{42}Je suis arrivé aujourd'hui à la source et j'ai dit : Ô Yahweh ! Dieu de mon seigneur Abraham, si tu daignes faire réussir le voyage que j'ai entrepris,
\VS{43}voici, je me tiendrai près de la source d'eau, et la jeune fille qui sortira pour puiser à qui je dirai : Laisse-moi boire, je te prie, un peu d’eau de ta cruche ; et qui me répondra :
\VS{44}Bois toi-même, et je puiserai aussi pour tes chameaux, que cette jeune fille soit la femme que Yahweh a destinée au fils de mon seigneur.
\VS{45}Avant que j’ai fini de parler en mon cœur, voici, Rebecca est sortie, ayant sa cruche sur son épaule ; elle est descendue à la source et a puisé de l'eau ; et je lui ai dit : Donne-moi, à boire, je te prie.
\VS{46}Elle s’est empressée d’abaisser sa cruche de dessus son épaule et m'a dit : Bois, et même je donnerai à boire à tes chameaux. J'ai donc bu, et elle a aussi donné à boire aux chameaux.
\VS{47}Puis je l'ai interrogée en disant : De qui es-tu fille ? Elle a répondu : Je suis fille de Bethuel, fils de Nachor et de Milca. Alors je lui ai mis un anneau à son nez et les bracelets à ses mains.
\VS{48}Puis je me suis incliné, j’ai adoré Yahweh, et j'ai béni Yahweh, le Dieu de mon seigneur Abraham, qui m'a conduit fidèlement, afin que je prenne la fille du frère de mon seigneur pour son fils.
\VS{49}Maintenant donc, si vous voulez user de bonté et de fidélité envers mon seigneur, déclarez-le-moi ; sinon, déclarez-le-moi aussi ; et je me tournerai à droite ou à gauche.
\VS{50}Laban et Bethuel répondirent et dirent : Cette affaire vient de Yahweh, nous ne pouvons te parler ni en bien ni en mal.
\VS{51}Voici Rebecca est devant toi, prends-la et va, et qu'elle soit la femme du fils de ton seigneur, comme Yahweh l’a dit.
\VS{52}Lorsque le serviteur d'Abraham eut entendu leurs paroles, il se prosterna à terre devant Yahweh.
\VS{53}Et le serviteur sortit des objets d'argent et d'or, et des vêtements, et les donna à Rebecca. Il donna aussi de riches présents à son frère et à sa mère.
\VS{54}Puis ils mangèrent et burent, lui et les gens qui étaient avec lui, et ils passèrent la nuit.  Le matin,  quand ils furent levés, le serviteur dit : Laissez-moi retourner vers mon seigneur.
\VS{55}Le frère et la mère lui dirent : Que la jeune fille reste avec nous quelques jours encore, une dizaine de jours ;  après quoi, elle s'en ira.
\VS{56}Il leur répondit : Ne me retardez pas puisque Yahweh a fait réussir mon voyage ; laissez-moi partir  afin que je m'en aille vers mon seigneur.
\VS{57}Alors ils dirent : Appelons la jeune fille et demandons-lui son avis.
\VS{58}Ils appelèrent donc Rebecca et lui dirent : Veux-tu aller avec cet homme ? Et elle répondit : J'irai.
\VS{59}Ainsi ils laissèrent partir Rebecca, leur sœur, et sa nourrice, avec le serviteur d'Abraham et ses gens.
\VS{60}Ils bénirent Rebecca et lui dirent : Tu es notre sœur, puisses-tu devenir des milliers de myriades, et que ta postérité possède la porte de ses ennemis !
\VS{61}Alors Rebecca se leva avec ses servantes, et elles montèrent sur les chameaux et suivirent l’homme. Et le serviteur prit Rebecca et s'en alla.
\VS{62}Or Isaac revenait du puits de Lachaï-roï, et il habitait dans le pays du midi.
\VS{63}Un soir qu’Isaac était sorti dans les champs pour prier, il leva les yeux et regarda, et voici, des chameaux arrivaient.
\VS{64}Rebecca leva aussi les yeux, vit Isaac, et descendit de son chameau ;
\VS{65}car elle avait dit au serviteur : Qui est cet homme qui marche dans les champs à notre rencontre ? Et le serviteur avait répondu : C'est mon seigneur ; et elle prit son voile et se couvrit.
\VS{66}Le serviteur raconta à Isaac toutes les choses qu'il avait faites.
\VS{67}Alors Isaac conduisit Rebecca dans la tente de Sara, sa mère ; il prit Rebecca pour sa femme\FTNT{Pr. 18:22 ; Pr. 31:10-31.} et l'aima. Ainsi Isaac fut consolé après la mort de sa mère.
\Chap{25}
\TextTitle{Ketura, femme d'Abraham}
\VerseOne{}Or, Abraham prit une autre femme nommée Ketura.
\VS{2}Elle lui enfanta Zimram, Jokschan, Medan, Madian, Jischbak, et Schuach.
\VS{3}Jokschan engendra Séba et Dedan. Les fils de Dedan furent Aschurim, Letuschim et Leummim.
\VS{4}Les fils de Madian furent Epha, Epher, Hénoc, Abida, Eldaa. Ce sont là tous les fils de Ketura.
\TextTitle{Isaac hérite d'Abraham\FTNTT{Hé. 1:2}}
\VS{5}Abraham donna tout ce qui lui appartenait à Isaac.
\VS{6}Mais il fit des dons aux fils de ses concubines, et tandis qu’il vivait encore, il les envoya loin de son fils Isaac, du côté de l'orient, dans le pays d’orient.
\TextTitle{Mort d'Abraham}
\VS{7}Voici les jours des années de la vie d’Abraham : Il vécut cent soixante-quinze ans.
\VS{8}Abraham expira et mourut après une heureuse vieillesse, fort âgé et rassasié de jours, et il fut recueilli auprès de son peuple.
\VS{9}Isaac et Ismaël, ses fils, l'enterrèrent dans la caverne de Macpéla, dans le champ d'Ephron, fils de Tschoar, le Héthien, qui est vis-à-vis de Mamré.
\VS{10}C’est le champ qu'Abraham avait acheté des fils de Heth. Là furent enterrés Abraham et Sara, sa femme.
\VS{11}Après la mort d'Abraham, Dieu bénit Isaac son fils.  Isaac habitait près du puits de Lachaï-roï.
\TextTitle{Postérité d'Ismaël}
\VS{12}Voici la postérité d'Ismaël, fils d'Abraham, qu'Agar l’Egyptienne, servante de Sara, avait enfanté à Abraham.
\VS{13}Voici les noms des fils d'Ismaël, par leurs noms, selon leurs générations. Le premier-né d'Ismaël fut Nebajoth, puis Kédar, Adbeel, Mibsam,
\VS{14}Mischma, Duma, Massa,
\VS{15}Hadad, Théma, Jethur, Naphisch, et Kedma.
\VS{16}Ce sont là les fils d'Ismaël, et ce sont là leurs noms, selon leurs parcs, et selon leurs enclos ; douze princes de leurs peuples.
\VS{17}Et voici les années de la vie d'Ismaël : Cent trente-sept ans. Il expira et mourut, et il fut recueilli auprès de son peuple.
\VS{18}Ses descendants habitèrent depuis Havila jusqu'à Schur, qui est vis-à-vis de l'Egypte, en allant vers l'Assyrie. Et le pays qui était échu à Ismaël était à la vue de tous ses frères.
\TextTitle{Postérité d'Isaac}
\VS{19}Voici la postérité d'Isaac, fils d'Abraham.
\VS{20}Abraham engendra Isaac. Isaac était âgé de quarante ans quand il épousa Rebecca, fille de Bethuel, le Syrien, de Paddan-Aram, sœur de Laban, le Syrien.
\VS{21}Isaac pria instamment Yahweh au sujet de sa femme parce qu'elle était stérile ; et Yahweh exauça ses prières ; et Rebecca, sa femme, conçut.
\VS{22}Mais les enfants se heurtaient dans son ventre, et elle dit : S'il en est ainsi, pourquoi suis-je enceinte ? Et elle alla consulter Yahweh.
\VS{23}Et Yahweh lui dit : Deux nations sont dans ton ventre, et deux peuples se sépareront au sortir de tes entrailles ; un de ces peuples sera plus fort que l'autre, et le plus grand sera asservi au plus petit\FTNT{Ro. 9:12.}.
\TextTitle{Naissance des jumeaux : Esaü et Jacob}
\VS{24}Les jours où elle devait accoucher s’accomplirent ; et voici, il y avait deux jumeaux dans son ventre.
\VS{25}Celui qui sortit le premier était roux et tout velu, comme un manteau de poil ; et on lui donna le nom d’Esaü.
\VS{26}Ensuite sortit son frère, tenant de sa main le talon d'Esaü ; c'est pourquoi il fut appelé Jacob\FTNT{Jacob:«~celui qui prend par le talon~» ou «~qui supplante~».}. Isaac était âgé de soixante ans quand ils naquirent.
\TextTitle{Esaü méprise son droit d'aînesse}
\VS{27}Depuis, les enfants devinrent grands. Esaü devint un habile chasseur, et un homme des champs ; mais Jacob fut un homme intègre, se tenant dans les tentes.
\VS{28}Isaac aimait Esaü ; car le gibier était sa nourriture. Mais Rebecca aimait Jacob.
\VS{29}Comme Jacob faisait cuire du potage, Esaü arriva des champs, et il était fatigué.
\VS{30}Et Esaü dit à Jacob : Donne-moi, je te prie, à manger de ce roux, de ce roux-là\FTNT{Probablement un plat de lentilles.} ; car je suis fatigué. C'est pourquoi on appela son nom, Edom\FTNT{Edom:«~rouge, de couleur rousse~».}.
\VS{31}Mais Jacob lui dit : Vends-moi aujourd'hui ton droit d'aînesse.
\VS{32}Et Esaü répondit : Voici, je m'en vais mourir ; et de quoi me servira le droit d'aînesse ?
\VS{33}Et Jacob dit : Jure-moi aujourd'hui ; et il lui jura ; ainsi il vendit son droit d'aînesse à Jacob\FTNT{Hé. 12:16.}.
\VS{34}Et Jacob donna à Esaü du pain et du potage de lentilles ; et il mangea et but ; puis il se leva et s'en alla ; ainsi Esaü méprisa son droit d'aînesse.
\Chap{26}
\TextTitle{Yahweh confirme son alliance à Isaac}
\VerseOne{}Or, il y eut une famine dans le pays, outre la première famine qui eut lieu du temps d'Abraham ; et Isaac s'en alla vers Abimélec, roi des Philistins, à Guérar.
\VS{2}Yahweh lui apparut et lui dit : Ne descends pas en Egypte ; demeure dans le pays que je te dirai.
\VS{3}Demeure dans ce pays-ci, et je serai avec toi, et je te bénirai ; car je donnerai toutes ces contrées à toi et à ta postérité, et j’accomplirai le serment que j'ai fait à ton père Abraham.
\VS{4}Je multiplierai ta postérité comme les étoiles du ciel ; et je donnerai ces contrées à ta postérité ; et toutes les nations de la terre seront bénies en ta postérité,
\VS{5}parce qu'Abraham a obéi à ma voix, et qu'il a gardé mon ordonnance, mes commandements, mes statuts et mes lois.
\TextTitle{Faute d'Isaac à Guérar\FTNTT{Ge. 20}}
\VS{6}Isaac donc demeura à Guérar.
\VS{7}Et quand les gens du lieu posaient des questions sur sa femme, il disait : C'est ma sœur ; car il craignait de dire : C'est ma femme ; de peur, disait-il, que les habitants du lieu ne me tuent à cause de Rebecca, car elle est belle de figure.
\VS{8}Comme son séjour se prolongeait, il arriva qu'Abimélec, roi des Philistins, regardant par la fenêtre, vit Isaac qui plaisantait avec Rebecca, sa femme\FTNT{Ge. 20.}.
\VS{9}Alors Abimélec appela Isaac et lui dit : Voici, c'est véritablement ta femme. Comment as-tu pu dire : C'est ma soeur ? Et Isaac lui répondit : C'est parce que j'ai dit : Il ne faut pas que je meure à cause d'elle.
\VS{10}Et Abimélec dit : Que nous as-tu fait ? Il s'en est peu fallu que quelqu'un du peuple n'ait couché avec ta femme, et tu nous aurais rendus coupables.
\VS{11}Abimélec donc fit une ordonnance à tout le peuple en disant : Celui qui touchera cet homme, ou à sa femme, sera certainement puni de mort.
\VS{12}Isaac sema dans cette terre-là et il recueillit cette année-là le centuple ; car Yahweh le bénit.
\VS{13}Cet homme devint riche, et il alla s’enrichissant de plus en plus, jusqu'à ce qu'il devint fort riche.
\VS{14}Il avait des troupeaux de menu bétail et des troupeaux de gros bétail, et un grand nombre de serviteurs ; et les Philistins lui portèrent envie ; 
\VS{15}Et tous les puits que les serviteurs de son père avaient creusés, du temps de son père Abraham, les Philistins les bouchèrent et les remplirent de terre.
\VS{16}Abimélec aussi dit à Isaac : Va-t’en de chez nous, car tu es devenu beaucoup plus puissant que nous.
\TextTitle{Les puits d'Isaac}
\VS{17}Isaac donc partit de là, et campa dans la vallée de Guérar, où il s’établit.
\VS{18}Isaac creusa de nouveau les puits d'eau qu'on avait creusés du temps d'Abraham, son père, et que les Philistins avaient bouchés après la mort d'Abraham, et il leur donna les mêmes noms que son père leur avait donnés.
\VS{19}Les serviteurs d'Isaac creusèrent dans cette vallée et y trouvèrent un puits d'eau vive.
\VS{20}Mais les bergers de Guérar eurent une querelle avec les bergers d'Isaac, disant : L'eau est à nous. Et il appela le nom du puits Esek parce qu'ils avaient contesté avec lui.
\VS{21}Ensuite, ils creusèrent un autre puits, pour lequel ils contestèrent aussi ; et il appela son nom Sitna.
\VS{22}Alors il se transporta de là et creusa un autre puits pour lequel ils ne contestèrent point, et il le nomma Rehoboth, en disant : C'est parce que Yahweh nous a maintenant mis au large, et nous fructifierons dans le pays.
\VS{23}Et de là il remonta à Beer-Schéba.
\VS{24}Yahweh lui apparut cette nuit-là et lui dit : Je suis le Dieu d'Abraham, ton père ; ne crains point, car je suis avec toi, je te bénirai et je multiplierai ta postérité à cause d'Abraham, mon serviteur.
\VS{25}Alors il bâtit là un autel, et invoqua le nom de Yahweh, et il y dressa ses tentes. Et les serviteurs d'Isaac y creusèrent un puits.
\VS{26}Abimélec vint à lui de Guérar avec Ahuzath, son ami, et Picol, chef de son armée.
\VS{27}Mais Isaac leur dit : Pourquoi venez-vous vers moi, puisque vous me haïssez et que vous m'avez renvoyé de chez vous ?
\VS{28}Ils répondirent : Nous avons vu clairement que Yahweh est avec toi ; et nous avons dit : Qu'il y ait maintenant un serment solennel entre nous, c'est-à-dire entre nous et toi ; et traitons alliance avec toi.
\VS{29}Jure que tu ne nous feras aucun mal, de même que nous ne t'avons point maltraité, que nous t'avons fait seulement du bien, et que nous t’avons laissé partir en paix. Toi qui es maintenant béni de Yahweh.
\VS{30}Alors il leur fit un festin, et ils mangèrent et burent.
\VS{31}Ils se levèrent de bon matin, et jurèrent l'un à l'autre. Puis Isaac les renvoya, et ils s'en allèrent en paix.
\VS{32}Ce même jour, les serviteurs d'Isaac vinrent lui parler du puits qu'ils avaient creusé, et lui dirent : Nous avons trouvé de l'eau.
\VS{33}Et il l'appela Schiba. C'est pourquoi le nom de la ville a été Beer-Schéba jusqu'à aujourd'hui.
\VS{34}Esaü, âgé de quarante ans, prit pour femmes Judith, fille de Beéri, le Héthien, et Basmath, fille d'Elon, le Héthien.
\VS{35}Elles furent un sujet d’amertume pour l’esprit d’Isaac et de Rebecca.
\Chap{27}
\TextTitle{Jacob prend la bénédiction d'Isaac à la place d'Esaü}
\VerseOne{}Et il arriva que quand Isaac fut devenu vieux, et que ses yeux furent si affaiblis qu'il ne pouvait plus voir, il appela Esaü, son fils aîné, et lui dit : Mon fils ! Et il lui répondit : Me voici.
\VS{2}Isaac lui dit : Voici, maintenant je suis devenu vieux, et je ne connais pas le jour de ma mort.
\VS{3}Maintenant donc, je te prie, prends tes armes, ton carquois et ton arc, va dans les champs, et chasse-moi du gibier.
\VS{4}Apprête-moi un mets comme j’aime, et apporte-le-moi, afin que je mange, et que mon âme te bénisse avant que je meure.
\VS{5}Or Rebecca écoutait pendant qu'Isaac parlait à Esaü, son fils. Esaü donc s'en alla dans les champs pour chasser du gibier et pour le rapporter.
\VS{6}Et Rebecca parla à Jacob, son fils, et lui dit : Voici, j'ai entendu parler ton père à Esaü, ton frère, disant :
\VS{7}Apporte-moi du gibier, et fais-moi un mets, afin que je le mange et je te bénirai devant Yahweh avant de mourir.
\VS{8}Maintenant donc, mon fils, obéis à ma parole, et fais ce que je vais te commander.
\VS{9}Va maintenant à la bergerie, et prends-moi là deux bons chevreaux parmi les chèvres, et j'en ferai un mets pour ton père comme il aime.
\VS{10}Et tu le porteras à ton père, afin qu'il le mange et qu'il te bénisse avant sa mort.
\VS{11}Jacob répondit à Rebecca sa mère : Voici, Esaü, mon frère, est un homme velu, et je suis un homme sans poil.
\VS{12}Peut-être que mon père me touchera-t-il, et il me regardera comme un homme qui a voulu le tromper, et j'attirerai sur moi sa malédiction et non pas sa bénédiction.
\VS{13}Sa mère lui dit : Mon fils, que la malédiction que tu crains retombe sur moi ! Obéis seulement à ma parole, et va me prendre ce que je t'ai dit.
\TextTitle{Déception d'Esaü\FTNTT{Hé. 12:16-17}}
\VS{14}Jacob alla les prendre et les apporta à sa mère ; et sa mère fit un mets comme son père aimait.
\VS{15}Puis Rebecca prit les plus précieux habits d'Esaü, son fils aîné, qu'elle avait dans la maison, et elle les fit mettre à Jacob, son fils cadet.
\VS{16}Elle couvrit ses mains et son cou, qui étaient sans poil, des peaux des chevreaux.
\VS{17}Puis elle mit entre les mains de son fils Jacob le mets et le pain qu'elle avait apprêtés.
\VS{18}Il vint vers son père, et lui dit : Mon père ! Il répondit : Me voici ; qui es-tu, mon fils ?
\VS{19}Jacob répondit à son père : Je suis Esaü, ton fils aîné ; j'ai fait ce que tu m’as dit. Lève-toi, je te prie, assieds-toi et mange de mon gibier, afin que ton âme me bénisse.
\VS{20}Isaac dit à son fils : Eh quoi ! Tu en as déjà trouvé, mon fils ! Et il dit : Yahweh ton Dieu l'a fait venir devant moi.
\VS{21}Isaac dit à Jacob : Approche-toi, je te prie, mon fils, et que je te touche, afin que je sache si tu es mon fils Esaü ou non.
\VS{22}Jacob donc s'approcha de son père Isaac, qui le toucha et dit : Cette voix est la voix de Jacob, mais ces mains sont les mains d'Esaü.
\VS{23}Et il ne le reconnut pas, car ses mains étaient velues comme les mains de son frère Esaü ; et il le bénit.
\VS{24}Il dit : C’est toi, mon fils Esaü ? Il répondit : Je le suis.
\VS{25}Isaac lui dit : Apporte-moi donc la viande, et que je mange du gibier de mon fils, afin que mon âme te bénisse. Jacob l'apporta, et Isaac mangea ; il lui apporta aussi du vin, et il but.
\VS{26}Puis Isaac, son père, lui dit : Approche-toi, je te prie, et embrasse-moi mon fils.
\VS{27}Jacob s'approcha et l’embrassa. Isaac sentit l'odeur de ses habits, et le bénit en disant : Voici l'odeur de mon fils, comme l'odeur d'un champ que Yahweh a béni.
\VS{28}Que Dieu te donne de la rosée du ciel, et de la graisse de la terre, du blé et du vin en abondance\FTNT{Hé. 11:20.} !
\VS{29}Que des peuples te servent, et que des nations se prosternent devant toi ! Sois le maître de tes frères, et que les fils de ta mère se prosternent devant toi ! Maudit soit quiconque te maudira, et béni soit quiconque te bénira.
\VS{30}Isaac avait fini de bénir Jacob, et Jacob avait à peine quitté son père Isaac, qu’Esaü, son frère, revint de la chasse.
\VS{31}Il apprêta aussi un mets, l’apporta à son père, et lui dit : Que mon père se lève et mange du gibier de son fils, afin que ton âme me bénisse.
\VS{32}Isaac, son père, lui dit : Qui es-tu ? Et il dit : Je suis ton fils, ton fils aîné, Esaü.
\VS{33}Isaac fut saisi d'une grande, d’une violente émotion, et dit : Qui est donc celui qui a chassé du gibier et me l’a apporté ? J'ai mangé de tout avant que tu ne viennes, et je l'ai béni. Aussi sera-t-il béni !
\VS{34}Dès qu'Esaü entendit les paroles de son père, il poussa de forts cris, pleins d’amertume, et il dit à son père : Bénis-moi aussi, bénis-moi, mon père !
\VS{35}Mais il dit : Ton frère est venu avec tromperie, et il a enlevé ta bénédiction.
\VS{36}Esaü dit : N'est-ce pas avec raison qu'on a appelé son nom Jacob ? Car il m'a déjà supplanté deux fois ; il m'a enlevé mon droit d'aînesse, et voici, maintenant il a enlevé ma bénédiction. Puis il dit : Ne m'as-tu point réservé de bénédiction ?
\VS{37}Isaac répondit à Esaü en disant : Voici, je l'ai établi ton maître, et lui ai donné tous ses frères pour serviteurs, et je l'ai pourvu de blé et de vin ; et que ferai-je maintenant pour toi, mon fils ?
\VS{38}Esaü dit à son père : N'as-tu qu'une bénédiction, mon père ? Bénis-moi aussi, bénis-moi, mon père ! Et Esaü éleva la voix et pleura\FTNT{Hé. 12:17.}.
\VS{39}Isaac, son père, répondit, et dit : Voici, ta demeure sera privée de la graisse de la terre, et de la rosée du ciel, d'en haut.
\VS{40}Tu vivras par ton épée, et tu seras asservi à ton frère ; mais il arrivera qu'étant devenu maître, tu briseras son joug de dessus ton cou.
\TextTitle{Fuite de Jacob chez Laban}
\VS{41}Esaü conçut de la haine contre Jacob, à cause de la bénédiction dont son père l'avait béni ; et Esaü dit en son cœur : Les jours du deuil de mon père approchent, et je tuerai Jacob, mon frère.
\VS{42}On rapporta à Rebecca les paroles d'Esaü, son fils aîné ; et elle fit alors appeler Jacob, son fils cadet, et lui dit : Voici, Esaü, ton frère, se console dans l'espérance qu'il a de te tuer.
\VS{43}Maintenant donc, mon fils, obéis à ma parole ! Lève-toi, et enfuis-toi à Charan, vers Laban, mon frère.
\VS{44}Et reste avec lui quelque temps, jusqu'à ce que la fureur de ton frère soit passée ;
\VS{45}jusqu’à ce que la colère de ton frère se détourne de toi, et qu'il oublie ce que tu lui as fait. Pourquoi serais-je privée de vous deux en un même jour ?
\VS{46}Rebecca dit à Isaac : Je suis dégoûtée de la vie, à cause de filles de Heth. Si Jacob prend une femme, comme celles-ci, parmi les filles de Heth, parmi les filles du pays, à quoi me sert la vie ?
\Chap{28}
\TextTitle{A Béthel, Yahweh confirme son alliance à Jacob}
\VerseOne{}Isaac donc appela Jacob, et le bénit, et lui donna cet ordre : Tu ne prendras point de femme parmi les filles de Canaan.
\VS{2}Lève-toi, va à Paddan-Aram, à la maison de Bethuel, père de ta mère, et prends-toi une femme de là, parmi les filles de Laban, frère de ta mère.
\VS{3}Que le Dieu Tout-Puissant te bénisse, te rende fécond et te multiplie, afin que tu deviennes une assemblée de peuples.
\VS{4}Qu’il te donne la bénédiction d'Abraham, à toi et à ta postérité avec toi, afin que tu obtiennes en héritage le pays où tu as été étranger, que Dieu a donné à Abraham.
\VS{5}Isaac donc fit partir Jacob, qui s'en alla à Paddan-Aram, vers Laban, fils de Bethuel, le Syrien, frère de Rebecca, mère de Jacob et d'Esaü.
\VS{6}Esaü vit qu'Isaac avait béni Jacob, et qu'il l'avait envoyé à Paddan-Aram afin qu'il prenne une femme de ce pays-là pour lui, et qu'il lui avait donné cet ordre, quand il le bénissait, disant : Ne prends point de femme parmi les filles de Canaan ;
\VS{7}il vit que Jacob avait obéi à son père et à sa mère, et qu’il était parti à Paddan-Aram.
\VS{8}Esaü comprit ainsi que les filles de Canaan déplaisaient à Isaac, son père.
\VS{9}Et Esaü s’en alla vers Ismaël. Il prit pour femme, outre ses autres femmes, Mahalath, fille d'Ismaël, fils d'Abraham, sœur de Nebajoth.
\VS{10}Jacob partit de Beer-Schéba et s'en alla à Charan.
\VS{11}Il arriva dans un lieu où il passa la nuit, parce que le soleil était couché. Il y prit donc une pierre\FTNT{1 Pi. 2:4. Voir  commentaire en Es. 8:13-15.}, et en fit son chevet, et il se coucha dans ce lieu-là.
\VS{12}Il eut un songe ; et voici, une échelle dressée sur la terre, dont le sommet touchait le ciel. Et voici, les anges de Dieu montaient et descendaient par cette échelle\FTNT{Jn. 1:51.}.
\VS{13}Et voici, Yahweh se tenait sur l'échelle, et il lui dit : Je suis Yahweh, le Dieu d'Abraham, ton père, et le Dieu d'Isaac ; je te donnerai à toi et à ta postérité, la terre sur laquelle tu es couché.
\VS{14}Ta postérité sera comme la poussière de la terre, et tu t'étendras à l'occident et à l'orient, au nord et au midi, et toutes les familles de la terre seront bénies en toi et en ta postérité.
\VS{15}Voici, je suis avec toi ; et je te garderai partout où tu iras ; et je te ramènerai dans ce pays ; car je ne t'abandonnerai point que je n'aie exécuté ce que je t'ai dit.
\VS{16}Et quand Jacob fut réveillé de son sommeil, il dit : Certainement, Yahweh est en ce lieu-ci, et moi, je ne le savais pas !
\VS{17}Il eut peur et dit : Que ce lieu-ci est effrayant ! C'est ici la maison de Dieu, et c'est ici la porte des cieux !
\VS{18}Et Jacob se leva de bon matin, prit la pierre dont il avait fait son chevet, il la dressa pour monument, et versa de l'huile sur son sommet.
\VS{19}Il donna à ce lieu le nom de Béthel ; mais auparavant la ville s'appelait Luz.
\VS{20}Jacob fit un vœu en disant : Si Dieu est avec moi, et s'il me garde pendant le voyage que je fais, s'il me donne du pain à manger, et des habits pour me vêtir,
\VS{21}et si je retourne en paix à la maison de mon père, certainement Yahweh sera mon Dieu.
\VS{22}Cette pierre que j'ai dressée pour monument sera la maison de Dieu ; et de tout ce que tu m'auras donné, je t'en donnerai entièrement la dîme\FTNT{Voir commentaire sur la dîme en No. 18:21 et Mal. 3:10.}.
\Chap{29}
\TextTitle{Jacob épouse Léa et Rachel chez Laban}
\VerseOne{}Jacob donc se mit en chemin, et s'en alla au pays des fils de l’orient.
\VS{2}Il regarda. Et voici, il y avait un puits dans un champ ; et voici il y avait à côté trois troupeaux de brebis couchées près du puits, car c’était à ce puits qu’on abreuvait les troupeaux.  Et il y avait une grosse pierre sur l'ouverture du puits.
\VS{3}Tous les troupeaux se rassemblaient là ; on roulait la pierre de dessus l'ouverture du puits, et on abreuvait les troupeaux ; et ensuite on remettait la pierre à sa place, sur l'ouverture du puits.
\VS{4}Jacob leur dit : Mes frères, d'où êtes-vous ? Ils répondirent : Nous sommes de Charan.
\VS{5}Il leur dit : Connaissez-vous Laban, fils de Nachor ? Ils répondirent : Nous le connaissons.
\VS{6}Il leur dit : Se porte-t-il bien ? Ils lui répondirent : Il se porte bien ; et voici Rachel, sa fille, qui vient avec le troupeau.
\VS{7}Il dit : Voici, il est encore grand jour, et il n'est pas temps de rassembler les troupeaux ; abreuvez les brebis, puis allez et faites-les paître.
\VS{8}Ils répondirent : Nous ne le pouvons pas, jusqu'à ce que tous les troupeaux soient rassemblés et qu'on ait ôté la pierre de dessus l'ouverture du puits, afin d'abreuver les troupeaux.
\VS{9}Comme il parlait encore avec eux, Rachel arriva avec le troupeau de son père ; car elle était bergère.
\VS{10}Lorsque Jacob vit Rachel, fille de Laban, frère de sa mère, et le troupeau de Laban, frère de sa mère, il s'approcha et roula la pierre de dessus l'ouverture du puits, et abreuva le troupeau de Laban, frère de sa mère.
\VS{11}Et Jacob embrassa Rachel, et il éleva sa voix et pleura.
\VS{12}Jacob apprit à Rachel qu'il était frère de son père, et qu'il était fils de Rebecca ; et elle courut le rapporter à son père.
\VS{13}Dès que Laban eut entendu parler de Jacob, fils de sa soeur, il courut au-devant de lui, il le prit dans ses bras et l’embrassa, et il le fit venir dans sa maison ; et Jacob raconta à Laban tout ce qui lui était arrivé.
\VS{14}Et Laban lui dit : Certainement, tu es mon os et ma chair. Jacob demeura un mois entier chez Laban.
\VS{15}Puis Laban dit à Jacob : Me serviras-tu pour rien parce que tu es mon frère ? Dis-moi quel sera ton salaire ?
\VS{16}Or Laban avait deux filles : L'aînée s'appelait Léa, et la cadette Rachel.
\VS{17}Léa avait les yeux délicats, mais Rachel était belle de taille et belle de figure.
\VS{18}Jacob aimait Rachel, et il dit : Je te servirai sept ans pour Rachel, ta cadette.
\VS{19}Et Laban répondit : Il vaut mieux que je te la donne que de la donner à un autre homme ; demeure avec moi.
\VS{20}Ainsi Jacob servit sept années pour Rachel ; et elles furent à ses yeux comme quelques jours, parce qu'il l'aimait.
\VS{21}Et Jacob dit à Laban : Donne-moi ma femme, car mon temps est accompli, et j’irai vers elle.
\VS{22}Laban réunit tous les gens du lieu et fit un festin.
\VS{23}Mais quand le soir fut venu, il prit Léa, sa fille, et l'amena vers Jacob qui s’approcha d’elle.
\VS{24}Et Laban donna Zilpa, sa servante, à Léa, sa fille, pour servante.
\VS{25}Le lendemain matin, voilà que c'était Léa. Alors Jacob dit à Laban : Qu'est-ce que tu m'as fait ? N'ai-je pas servi chez toi pour Rachel ? Et pourquoi m'as-tu trompé ?
\VS{26}Laban répondit : On ne fait pas ainsi dans ce lieu de donner la plus jeune avant l'aînée.
\VS{27}Achève la semaine avec celle-ci, et nous te donnerons aussi l'autre, pour le service que tu feras encore chez moi sept autres années.
\VS{28}Jacob donc fit ainsi, et il acheva la semaine avec Léa ; et Laban lui donna aussi pour femme Rachel, sa fille.
\VS{29}Et Laban donna Bilha, sa servante, à Rachel, sa fille, pour servante.
\VS{30}Jacob alla aussi vers Rachel, et il aima Rachel plus que Léa ; et il servit encore chez Laban sept autres années.
\VS{31}Yahweh vit que Léa était haïe, et il ouvrit sa matrice, tandis que Rachel était stérile.
\TextTitle{Les enfants de Jacob}
\VS{32}Léa conçut et enfanta un fils à qui elle donna le nom de Ruben, car elle dit : C'est parce que Yahweh a vu mon affliction, et maintenant mon mari m'aimera.
\VS{33}Elle conçut encore et enfanta un fils, et elle dit : Parce que Yahweh a entendu que j'étais haïe, il m'a aussi donné celui-ci. Et elle lui donna le nom de Siméon.
\VS{34}Elle conçut encore et enfanta un fils, et elle dit : Maintenant mon mari s'attachera à moi, car je lui ai enfanté trois fils. C'est pourquoi on lui donna le nom de Lévi.
\VS{35}Elle conçut encore et enfanta un fils, et elle dit : Cette fois je louerai Yahweh. C'est pourquoi elle lui donna le nom de Juda. Et elle cessa d'avoir des enfants.
\Chap{30}
\TextTitle{Les enfants de Jacob (suite)}
\VerseOne{}Alors Rachel, voyant qu’elle ne donnait point d'enfants à Jacob, fut jalouse de Léa, sa sœur, et elle dit à Jacob : Donne-moi des enfants, autrement je meurs !
\VS{2}La colère de Jacob s’enflamma contre Rachel, et il dit : Suis-je à la place de Dieu pour t’empêcher d'avoir des enfants ?
\VS{3}Elle dit : Voici ma servante Bilha ; va vers elle ; qu’elle enfante sur mes genoux, et que j’aie des fils par elle.
\VS{4}Et elle lui donna pour femme Bilha, sa servante, et Jacob alla vers elle.
\VS{5}Bilha conçut et enfanta un fils à Jacob.
\VS{6}Rachel dit : Dieu a jugé en ma faveur, et il a aussi exaucé ma voix, et m'a donné un fils ; c'est pourquoi elle l’appela du nom de Dan.
\VS{7}Bilha, servante de Rachel, conçut encore et enfanta un second fils à Jacob.
\VS{8}Rachel dit : J'ai fortement lutté contre ma sœur, aussi j'ai eu la victoire ; c'est pourquoi elle l’appela du nom de Nephthali.
\VS{9}Alors Léa, voyant qu'elle avait cessé de faire des enfants, prit Zilpa, sa servante, et la donna pour femme à Jacob.
\VS{10}Zilpa, servante de Léa, enfanta un fils à Jacob.
\VS{11}Léa dit : Le bonheur est arrivé, c'est pourquoi elle l’appela du nom de Gad.
\VS{12}Zilpa, servante de Léa, enfanta un second fils à Jacob.
\VS{13}Léa dit : C'est pour me rendre heureuse, car les filles me diront bienheureuse ; c'est pourquoi elle l’appela du nom d’Aser.
\VS{14}Ruben sortit au temps de la moisson des blés, trouva des mandragores\FTNT{La mandragore, appelée pomme d'amour, était utilisée comme excitant du désir sexuel ainsi que pour favoriser la procréation. On attribuait à cette plante aux propriétés hallucinogènes des vertus magiques.} aux champs, et les apporta à Léa, sa mère ; et Rachel dit à Léa : Donne-moi, je te prie, des mandragores de ton fils.
\VS{15}Elle lui répondit : Est-ce peu que tu aies pris mon mari, pour que tu prennes aussi les mandragores de mon fils ? Et Rachel dit : Qu'il couche donc cette nuit avec toi pour les mandragores de ton fils.
\VS{16}Le soir, comme Jacob revenait des champs, Léa sortit au-devant de lui et lui dit : Tu viendras vers moi, car je t'ai acheté pour les mandragores de mon fils ; et il coucha avec elle cette nuit-là.
\VS{17}Dieu exauça Léa, et elle conçut et enfanta à Jacob un cinquième fils.
\VS{18}Léa dit : Dieu m'a récompensée, parce que j'ai donné ma servante à mon mari ; c'est pourquoi elle l’appela du nom d’Issacar.
\VS{19}Léa conçut encore et enfanta un sixième fils à Jacob.
\VS{20}Léa dit : Dieu m'a donné un beau don ; maintenant mon mari habitera avec moi, car je lui ai enfanté six fils ; c'est pourquoi elle l’appela du nom de Zabulon.
\VS{21}Puis elle enfanta une fille et la nomma Dina.
\VS{22}Dieu se souvint de Rachel, il l’exauça et il ouvrit sa matrice.
\VS{23}Alors elle conçut et enfanta un fils, et elle dit : Dieu a ôté mon opprobre.
\VS{24}Et elle lui donna le nom de Joseph, en disant : Que Yahweh m'ajoute un autre fils !
\TextTitle{Jacob devient de plus en plus riche}
\VS{25}Lorsque Rachel eut enfanté Joseph, Jacob dit à Laban : Laisse-moi partir, pour que je m’en aille chez moi, dans mon pays.
\VS{26}Donne-moi mes femmes et mes enfants, pour lesquels je t'ai servi, et je m'en irai ; car tu sais de quelle manière je t'ai servi.
\VS{27}Laban lui répondit : Ecoute, je te prie, si j'ai trouvé grâce à tes yeux ; j’ai deviné que Yahweh m'a béni à cause de toi.
\VS{28}Il lui dit aussi : Fixe-moi le salaire que tu veux, et je te le donnerai.
\VS{29}Jacob lui répondit : Tu sais comment je t'ai servi et ce qu'est devenu ton bétail avec moi.
\VS{30}Car le peu que tu avais avant que je vienne s’est beaucoup accru, et Yahweh t'a béni depuis que j’ai mis mes pieds chez toi. Et maintenant, quand ferai-je aussi quelque chose pour ma maison ?
\VS{31}Laban lui dit : Que te donnerai-je ? Et Jacob répondit : Tu ne me donneras rien ; mais je ferai paître encore tes troupeaux, et je les garderai, si tu consens à ce que je vais te dire.
\VS{32}Je parcourrai aujourd'hui tes troupeaux, mets à part parmi toutes les brebis tachetées et marquetées, et tous les agneaux noirs, et les chèvres marquetées et tachetées. Ce sera mon salaire.
\VS{33}Ma justice me rendra témoignage à l’avenir devant toi ; quand tu viendras reconnaître mon salaire, en ta présence ; et tout ce qui ne sera pas marqueté ou tacheté parmi les chèvres, et noirs parmi les agneaux, sera considéré comme un vol s'il est trouvé chez moi.
\VS{34}Laban dit : Voici, qu'il te soit fait comme tu l'as dit.
\VS{35}Ce même jour, il sépara les boucs rayés et marquetés, et toutes les chèvres tachetées et marquetées, toutes celles où il y avait du blanc, et tous les agneaux noirs. Il les remit entre les mains de ses fils.
\VS{36}Puis il mit l'espace de trois journées de chemin entre lui et Jacob ; et Jacob fit paître le reste des troupeaux de Laban.
\VS{37}Mais Jacob prit des branches vertes de peuplier, d’amandier et de platane ; il y pela des bandes blanches,  mettant à nu le blanc qui était sur les branches.
\VS{38}Puis il plaça les branches qu’il avait pelées dans les auges, dans les abreuvoirs, sous les yeux des brebis qui venaient boire, et elles entraient en chaleur quand elles venaient boire.
\VS{39}Les brebis entraient en chaleur près des branches, et elles faisaient des brebis rayées, tachetées et marquetées.
\VS{40}Jacob séparait les agneaux, et il mettait ensemble ce qui était rayé et tout ce qui était noir dans les troupeaux de Laban. Il se fit ainsi des troupeaux à part, qu’il ne réunit point aux troupeaux de Laban.
\VS{41}Toutes les fois que les brebis vigoureuses entraient en chaleur, Jacob mettait les branches dans les auges sous les yeux des brebis, afin qu'elles entrent en chaleur près des  branches.
\VS{42}Mais pour les brebis chétives, il ne les mettait point ; de sorte que les chétives appartenaient à Laban et les vigoureuses à Jacob.
\VS{43}Ainsi cet homme devint de plus en plus riche ; il eut du menu bétail en abondance, des servantes et des serviteurs, des chameaux et des ânes.
\Chap{31}
\TextTitle{Yahweh demande à Jacob de rentrer dans la terre de se pères}
\VerseOne{}Or Jacob entendit les discours des fils de Laban qui disaient : Jacob a pris tout ce qui appartenait à notre père, et c’est de ce qui était à notre père qu’il s’est acquis toute cette richesse.
\VS{2}Jacob regarda le visage de Laban, et voici, il n'était plus à son égard comme auparavant.
\VS{3}Alors Yahweh dit à Jacob : Retourne au pays de tes pères et vers ta parenté, et je serai avec toi.
\VS{4}Jacob fit appeler Rachel et Léa qui étaient aux champs vers son troupeau,
\VS{5}et leur dit : Je vois au visage de votre père qu'il n'est plus envers moi comme il était auparavant ; toutefois le Dieu de mon père a été avec moi.
\VS{6}Vous savez que j'ai servi votre père de tout mon pouvoir.
\VS{7}Mais votre père s'est moqué de moi et a changé dix fois mon salaire ; mais Dieu ne lui a pas permis de me faire du mal.
\VS{8}Quand il disait : Les tachetées seront ton salaire, alors toutes les brebis faisaient des agneaux tachetés ; et quand il disait : Les marquetées seront ton salaire, alors toutes les brebis faisaient des agneaux marquetés.
\VS{9}Ainsi Dieu a ôté à votre père son bétail et me l'a donné.
\VS{10}Au temps où les brebis entraient en chaleur, je levai mes yeux et vis en songe que les boucs qui couvraient les brebis étaient rayés, tachetés, et marquetés.
\VS{11}Et l'Ange de Dieu\FTNT{Gn. 7:7} me dit en songe : Jacob ! Et je répondis : Me voici.
\VS{12}Il dit : Lève maintenant tes yeux et regarde : Tous les boucs qui couvrent les brebis sont rayés, tachetés et marquetés, car j'ai vu tout ce que te fait Laban.
\VS{13}Je suis le Dieu de Béthel, où tu oignis la pierre que tu dressas pour monument, où tu me fis un vœu.  Maintenant lève-toi, sors de ce pays, et retourne au pays de ta naissance.
\TextTitle{Jacob fuit de chez Laban avec sa famille}
\VS{14}Alors Rachel et Léa lui répondirent et dirent : Avons-nous encore quelque portion et quelque héritage dans la maison de notre père ?
\VS{15}Ne nous a-t-il pas traitées comme des étrangères ? Car il nous a vendues, et même il a entièrement mangé notre argent.
\VS{16}Car toutes les richesses que Dieu a ôtées à notre père nous appartenaient ainsi qu’à nos enfants. Maintenant donc fais tout ce que Dieu t'a dit.
\VS{17}Ainsi Jacob se leva, et fit monter ses enfants et ses femmes sur des chameaux.
\VS{18}Il emmena tout son bétail et tous les biens qu'il avait acquis, et tout ce qu'il possédait et qu'il avait acquis à Paddan-Aram, pour aller vers Isaac, son père, au pays de Canaan.
\VS{19}Or comme Laban était allé tondre ses brebis, Rachel déroba les théraphim de son père\FTNT{Les théraphim étaient des idoles utilisées dans un sanctuaire de maison ou dans un lieu de culte. Voir Jg. 18:14 ; 2 R. 23:24.}.
\VS{20}Et Jacob trompa Laban, le Syrien, en ne l’avertissant pas de son dessein, parce qu'il s'enfuyait.
\VS{21}Il s'enfuit avec tout ce qui lui appartenait ; il se leva, passa le fleuve, et se dirigea vers la montagne de Galaad.
\VS{22}Le troisième jour, on rapporta à Laban que Jacob s’était enfui.
\VS{23}Alors il prit avec lui ses frères, et il le poursuivit sept journées de marche, et l'atteignit à la montagne de Galaad.
\VS{24}Mais Dieu apparut à Laban, le Syrien, en songe la nuit, et lui dit : Garde-toi de parler à Jacob ni en bien ni en mal.
\VS{25}Laban donc atteignit Jacob. Jacob avait dressé ses tentes sur la montagne ; et Laban dressa aussi les siennes avec ses frères sur la montagne de Galaad.
\VS{26}Et Laban dit à Jacob : Qu'as-tu fait ? Tu m’as trompé, tu as emmené mes filles comme des prisonnières de guerre.
\VS{27}Pourquoi as-tu pris la fuite secrètement, m’as-tu trompé et ne m’as-tu pas averti ? Car je t'aurais laissé partir avec joie et avec des chansons, au son des tambours et des violons.
\VS{28}Tu ne m'as pas laissé embrasser mes fils et mes filles ! C’est en insensé que tu as agi.
\VS{29}J'ai en main le pouvoir de vous faire du mal, mais le Dieu de votre père m'a parlé la nuit passée et m'a dit : Garde-toi de ne parler à Jacob ni en bien ni en mal.
\VS{30}Maintenant que tu es parti, parce que tu  languissais après la maison de ton père, pourquoi as-tu dérobé mes dieux ?
\VS{31}Jacob répondit et dit à Laban : Je me suis enfui parce que je craignais ; car je me disais qu'il fallait prendre garde que tu ne me ravisses tes filles.
\VS{32}Mais celui chez qui tu trouveras tes dieux ne vivra point. En présence de nos frères, examine  s'il y a chez moi quelque chose qui t'appartienne, et prends-le ; car Jacob ignorait que Rachel les avait dérobés.
\VS{33}Alors Laban entra dans la tente de Jacob, et dans celle de Léa, et dans la tente des deux servantes, et il ne les trouva point ; et étant sorti de la tente de Léa, il entra dans la tente de Rachel.
\VS{34}Mais Rachel avait prit les théraphim et les avait mis dans le bât d'un chameau, et s’était assise dessus ; et Laban fouilla toute la tente et ne les trouva point.
\VS{35}Elle dit à son père : Que mon seigneur ne se fâche point de ce que je ne puis me lever devant lui, car j'ai ce que les femmes ont coutume d'avoir ; et il fouilla, mais il ne trouva point les théraphim.
\VS{36}Jacob se mit en colère et querella Laban. Il reprit la parole et lui dit : Quel est mon crime ? Quel est mon péché, pour que tu me poursuives avec tant d’ardeur ?
\VS{37}Car tu as fouillé tous mes effets, qu'as-tu trouvé des effets de ta maison ? Mets-les ici devant mes frères et les tiens, et qu'ils soient juges entre nous deux.
\VS{38}Voilà vingt ans que j’ai passés chez toi ; tes brebis et tes chèvres n'ont point avorté, je n'ai point mangé les moutons de tes troupeaux.
\VS{39}Je ne t'ai point rapporté de bêtes déchirées par les bêtes sauvages, j'en ai moi-même subi la perte ; et tu redemandais de ma main ce qui avait été dérobé de jour et ce qui avait été dérobé de nuit.
\VS{40}Le jour la chaleur me consumait, et la nuit le froid ; et le sommeil fuyait de mes yeux.
\VS{41}Voilà vingt ans que j’ai passés dans ta maison, quatorze ans pour tes deux filles, et six ans pour tes troupeaux, et tu m'as changé dix fois mon salaire.
\VS{42}Si je n’avais pas eu pour moi le Dieu de mon père, le Dieu d'Abraham, et celui que craint Isaac, certes tu m’aurais maintenant renvoyé à vide. Mais Dieu a regardé mon affliction et le travail de mes mains, et il t'a repris la nuit passée.
\VS{43}Laban répondit à Jacob et dit : Ces filles sont mes filles, et ces enfants sont mes enfants, et ces troupeaux sont mes troupeaux, et tout ce que tu vois est à moi ; et que ferais-je aujourd'hui à mes filles et aux enfants qu'elles ont enfantés ?
\VS{44}Maintenant donc, viens, faisons ensemble une alliance, et qu’elle serve de témoignage entre moi et toi.
\VS{45}Jacob prit une pierre et il la dressa pour monument.
\VS{46}Jacob dit à ses frères : Ramassez des pierres. Et ils prirent des pierres et ils en firent un monceau, et ils mangèrent là sur ce monceau.
\VS{47}Laban l'appela Jegar-Sahadutha, et Jacob l'appela Galed.
\VS{48}Et Laban dit : Ce monceau sera aujourd'hui témoin entre moi et toi ; c'est pourquoi il fut nommé Galed (poste d’observation).
\VS{49}Il fut aussi appelé Mitspa ; parce que Laban dit : Que Yahweh veille sur moi et sur toi, quand nous nous serons l'un et l'autre perdus de vue.
\VS{50}Si tu maltraites mes filles et si tu prends une autre femme que mes filles, ce n’est pas un homme qui sera témoin entre nous, prends-y garde ; c'est Dieu qui est témoin entre moi et toi.
\VS{51}Laban dit encore à Jacob : Regarde ce monceau, et considère le monument que j'ai dressé entre moi et toi.
\VS{52}Que ce monceau soit témoin et que ce monument soit témoin que je n’irai pas vers toi au-delà de ce monceau, et que tu ne viendras pas vers moi au-delà de ce monceau et de ce monument pour me faire du mal.
\VS{53}Que le Dieu d'Abraham et le Dieu de Nachor, le Dieu de leur père, juge entre nous ; mais Jacob jura par celui que craignait Isaac, son père.
\VS{54}Jacob offrit un sacrifice sur la montagne et invita ses frères pour manger du pain ; ils mangèrent donc du pain et passèrent la nuit sur la montagne.
\VS{55}Laban se leva de bon matin, embrassa ses fils et ses filles, et les bénit. Ensuite il s'en alla. Ainsi Laban retourna chez lui.
\Chap{32}
\TextTitle{Jacob devient Israël}
\VerseOne{}Et Jacob continua son chemin, et des anges de Dieu le rencontrèrent.
\VS{2}En les voyant, Jacob dit : C'est ici le camp de Dieu ! Et il donna à ce lieu le nom de Mahanaïm.
\VS{3}Jacob envoya devant lui des messagers vers Esaü, son frère, au pays de Séir, dans le territoire d'Edom.
\VS{4}Il leur donna cet ordre : Vous parlerez de cette manière à mon seigneur Esaü : Ainsi a dit ton serviteur Jacob : J'ai séjourné comme étranger chez Laban, et j’y ai habité jusqu'à présent ;
\VS{5}j’ai des bœufs, des ânes, des brebis, des serviteurs, et des servantes ; et j'envoie l’annoncer à mon seigneur, afin de trouver grâce à  tes yeux.
\VS{6}Et les messagers revinrent auprès de Jacob et lui dirent : Nous sommes allés vers ton frère Esaü, et il marche aussi à ta rencontre avec quatre cents hommes.
\VS{7}Alors Jacob fut très effrayé et rempli d’angoisse ; et il partagea le peuple qui était avec lui, et les brebis, et les boeufs, et les chameaux, en deux camps ; et  il dit :
\VS{8}Si Esaü attaque l'un des camps et le frappe, le camp qui restera pourra s’échapper.
\VS{9}Jacob dit aussi : Ô Dieu de mon père Abraham, Dieu de mon père Isaac, ô Yahweh qui m'as dit : Retourne dans ton pays, et vers ta parenté, et je te ferai du bien.
\VS{10}Je suis trop petit pour toutes les faveurs et pour toute la fidélité dont tu as usé envers ton serviteur ; car j'ai passé ce Jourdain avec mon bâton, et maintenant je forme deux camps.
\VS{11}Je te prie, délivre-moi de la main de mon frère Esaü ; car je crains qu'il ne vienne, et qu'il ne me frappe, et qu'il ne tue la mère avec les enfants.
\VS{12}Et toi, tu as dit : Certes, je te ferai du bien, et je rendrai ta postérité comme le sable de la mer, si abondant qu’on ne saurait le compter.
\VS{13}C’est dans ce lieu-là que Jacob passa la nuit.  Il prit de ce qu’il avait sous la main pour faire un présent à Esaü, son frère :
\VS{14}à savoir deux cents chèvres, vingt boucs, deux cents brebis et vingt béliers.
\VS{15}Trente femelles de chameaux qui allaitaient, et leurs petits ; quarante jeunes vaches, dix jeunes taureaux, vingt ânesses et dix ânes.
\VS{16}Il les mit entre les mains de ses serviteurs, chaque troupeau à part, et leur dit : Passez devant moi, et faites qu'il y ait un intervalle entre chaque troupeau.
\VS{17}Il donna cet ordre au premier, disant : Quand Esaü, mon frère, te rencontrera et te demandera, disant : A qui es-tu ? Et où vas-tu ? Et à qui sont ces choses qui sont devant toi ?
\VS{18}Alors tu diras : Je suis à ton serviteur Jacob ; c'est un présent qu'il envoie à mon seigneur Esaü ; et voici, il vient lui-même derrière nous.
\VS{19}Il donna le même ordre au deuxième, au troisième, et à tous ceux qui suivaient les troupeaux, disant : C’est ainsi que vous parlerez à mon seigneur Esaü, quand vous le rencontrerez.
\VS{20}Vous lui direz : Voici, ton serviteur Jacob vient aussi derrière nous. Car il se disait : J'apaiserai sa colère par ce présent qui va devant moi, et après cela, je verrai sa face ; peut-être qu'il me regardera favorablement.
\VS{21}Le présent passa devant lui ; mais il resta cette nuit-là dans le camp.
\VS{22}Il se leva cette nuit, et prit ses deux femmes, ses deux servantes, et ses onze enfants, et passa le gué de Jabbok.
\VS{23}Il les prit donc, et leur fit passer le torrent ; il fit aussi passer tout ce qu'il avait.
\VS{24}Jacob demeura seul. Alors un homme lutta avec lui jusqu'au lever de l’aurore.
\VS{25}Et quand cet homme vit qu'il ne pouvait pas le vaincre, il frappa à l'emboîture de la hanche de Jacob ; ainsi l'emboîture de l'os de la hanche de Jacob se démit pendant qu’il luttait avec lui.
\VS{26}Et cet homme lui dit : Laisse-moi, car l'aube du jour est levée. Mais il dit : Je ne te laisserai point que tu ne m'aies béni.
\VS{27}Cet homme lui dit : Quel est ton nom ? Il répondit : Jacob.
\VS{28}Alors il dit : Ton nom ne sera plus Jacob, mais tu seras appelé Israël ; car tu as été le vainqueur en luttant avec Dieu et avec les hommes, et tu as été le plus fort.
\VS{29}Jacob l’interrogea en disant : Je te prie, déclare-moi ton nom. Et il répondit : Pourquoi demandes-tu mon nom ? Et il le bénit là\FTNT{Jg. 13:18.}.
\VS{30}Jacob appela ce lieu du nom de Peniel ; car, dit-il, j’ai vu Dieu face à face, et mon âme a été délivrée.
\VS{31}Le soleil se levait lorsqu’il passa Peniel. Jacob boitait de la hanche.
\VS{32}C'est pourquoi, jusqu'à ce jour, les enfants d'Israël ne mangent point le tendon qui est à l’emboîture de la hanche ; parce que Dieu frappa Jacob à l'emboîture de la hanche, au tendon.
\Chap{33}
\TextTitle{Jacob demande pardon à son frère Esaü}
\VerseOne{}Et Jacob leva ses yeux et regarda ; et voici, Esaü arrivait avec quatre cents hommes. Et Jacob répartit les enfants entre Léa, Rachel, et les deux servantes.
\VS{2}Il plaça en tête les servantes avec leurs enfants ; Léa et ses enfants ensuite ; et Rachel et Joseph au dernier rang.
\VS{3}Quant à lui,  il passa devant eux et se prosterna à terre sept fois, jusqu'à ce qu'il soit près de son frère.
\VS{4}Esaü courut à sa rencontre ; il le prit dans ses bras, se jeta sur son cou, et l’embrassa. Et ils pleurèrent.
\VS{5}Esaü leva ses yeux, vit les femmes et les enfants, et dit : Qui sont ceux-là ? Sont-ils à toi ? Jacob lui répondit : Ce sont les enfants que Dieu, par sa grâce, a donnés à ton serviteur.
\VS{6}Les servantes s'approchèrent, elles et leurs enfants, et se prosternèrent.
\VS{7}Puis Léa aussi s'approcha avec ses enfants, et ils se prosternèrent, et ensuite Joseph et Rachel s'approchèrent et se prosternèrent aussi.
\VS{8}Esaü dit : Que veux-tu faire avec tout ce camp que j'ai rencontré ? Et Jacob répondit : C'est pour trouver grâce aux yeux de mon seigneur.
\VS{9}Esaü dit : Je suis dans l’abondance, mon frère ; garde ce qui est à toi.
\VS{10}Et Jacob répondit : Non, je te prie, si j'ai maintenant trouvé grâce à tes yeux, reçois ce présent de ma main ; parce que j'ai vu ta face comme si j'avais vu la face de Dieu, et parce que tu m’as accueilli favorablement.
\VS{11}Accepte, je te prie, mon présent qui t'a été offert ; car Dieu m’a comblé de grâce, et je ne manque de rien. Il le pressa tant qu'il le prit.
\VS{12}Esaü dit : Partons et marchons, et je marcherai devant toi.
\VS{13}Mais Jacob lui dit : Mon seigneur sait que ces enfants sont jeunes et que j’ai des brebis et des vaches qui allaitent ; si l’on forçait leur marche un seul jour, tout le troupeau mourra.
\VS{14}Je te prie que mon seigneur passe devant son serviteur, et je m’avancerai tout doucement, au pas de ce bétail qui est devant moi, et au pas de ces enfants, jusqu'à ce que j'arrive chez mon seigneur à Séir.
\VS{15}Esaü dit : Je te prie, je vais au moins laisser avec toi une partie de ce peuple qui est avec moi ; et il répondit : Pourquoi cela ? Je te prie que je trouve grâce aux yeux de mon seigneur.
\VS{16}Ainsi Esaü  retourna ce jour-là par son chemin à Séir.
\TextTitle{Jacob dresse un autel à El-Elohé-Israël (Dieu Fort, Dieu d'Israël)}
\VS{17}Jacob partit pour Succoth. Il bâtit une maison pour lui, et il fit des cabanes pour son bétail. C’est pourquoi il appela ce lieu du nom de Succoth.
\VS{18}A son retour de  Paddan-Aram, Jacob arriva sain et sauf à la ville de Sichem, dans le pays de Canaan, et il campa devant la ville.
\VS{19}Il acheta une portion du champ où il avait dressé sa tente de la main des fils de Hamor, père de Sichem, pour cent pièces d'argent.
\VS{20}Et là, il dressa un autel qu'il appela El-Elohé-Israël (le Dieu Fort, le Dieu d'Israël).
\Chap{34}
\TextTitle{Déshonneur de Dina et vengeance de ses frères}
\VerseOne{}Or Dina, la fille que Léa avait enfantée à Jacob, sortit pour voir les filles du pays.
\VS{2}Elle fut aperçue de Sichem, fils de Hamor, le Hévien, prince du pays. Il l'enleva et coucha avec elle, et la déshonora.
\VS{3}Son cœur fut attaché à Dina, fille de Jacob ; il aima la jeune fille et sut parler au cœur de la jeune fille.
\VS{4}Et Sichem parla à Hamor, son père, en disant : Prends-moi cette fille pour femme.
\VS{5}Jacob apprit qu'il avait déshonoré Dina, sa fille. Or ses fils étaient avec son bétail aux champs ; Jacob garda le silence jusqu’à leur retour.
\VS{6}Hamor, père de Sichem, sortit vers Jacob pour lui  parler.
\VS{7}Et les fils de Jacob revinrent des champs dès qu’ils apprirent ce qui était arrivé ; ces hommes furent dans  une grande douleur, et furent fort irrités de l'infamie que Sichem avait commise contre Israël, en couchant avec la fille de Jacob, ce qui ne devait point se faire.
\VS{8}Hamor leur parla en disant : L’âme de Sichem, mon fils, s’est attachée à votre fille ; donnez-la-lui je vous prie pour femme.
\VS{9}Alliez-vous avec nous, vous nous donnerez vos filles, et vous prendrez pour vous les nôtres.
\VS{10}Vous habiterez avec nous, et le pays sera à votre disposition ; restez pour y trafiquer et y acquérir des possessions.
\VS{11}Sichem dit aussi au père et aux frères de la fille : Que je trouve grâce à vos yeux, et je donnerai tout ce que vous me direz.
\VS{12}Exigez de moi une forte dot, et beaucoup de présents que vous voudrez, et je les donnerai comme vous me direz ; et donnez-moi la jeune fille pour femme.
\VS{13}Alors les fils de Jacob répondirent avec ruse à Sichem et à Hamor, son père ; ils parlèrent ainsi parce que Sichem avait déshonoré Dina, leur sœur.
\VS{14}Ils leur dirent : C’est une chose que nous ne pouvons pas faire, que de donner notre sœur à un homme incirconcis, car ce serait un opprobre pour nous.
\VS{15}Mais nous ne consentirons à ce que vous demandez que si vous deveniez semblables à nous en circoncisant tous les mâles qui sont parmi vous.
\VS{16}Alors nous vous donnerons nos filles, et nous prendrons vos filles pour nous, et nous habiterons avec vous, et nous ne serons qu'un seul peuple.
\VS{17}Mais si vous ne voulez pas nous écouter et vous circoncire, nous prendrons notre fille et nous nous en irons.
\VS{18}Leurs discours plurent à Hamor et à Sichem, fils d'Hamor.
\VS{19}Le jeune homme ne tarda point à faire ce qu'on lui avait proposé, car la fille de Jacob lui plaisait beaucoup ; et il était le plus considéré de tous ceux de la maison de son père.
\VS{20}Hamor et Sichem, son fils, se rendirent à la porte de leur ville et parlèrent aux gens de leur ville en leur disant :
\VS{21}Ces hommes sont paisibles à notre égard ; qu'ils habitent dans le pays et qu'ils y trafiquent ; car voici, le pays est assez vaste pour eux. Nous prendrons pour femmes leurs filles, et nous leur donnerons nos filles.
\VS{22}Mais ces hommes ne consentiront à habiter avec nous, pour former un seul peuple, que si tout mâle qui est parmi nous est circoncis, comme ils sont eux-mêmes circoncis.
\VS{23}Leur bétail, et leurs biens, et toutes leurs bêtes, ne seront-ils pas à nous ? Accordons-leur seulement cela, et qu'ils demeurent avec nous.
\VS{24}Tous ceux qui sortaient par la porte de leur ville obéirent à Hamor et à Sichem, son fils ; et tout mâle d'entre tous ceux qui sortaient par la porte de leur ville fut circoncis.
\VS{25}Le troisième jour, pendant qu’ils étaient souffrants, deux des fils de Jacob, Siméon et Lévi, frères de Dina, prirent leurs épées, entrèrent hardiment dans la ville et tuèrent tous les mâles.
\VS{26}Ils passèrent aussi au tranchant de l'épée Hamor et Sichem, son fils ; ils enlevèrent Dina de la maison de Sichem, et sortirent.
\VS{27}Les fils de Jacob se jetèrent sur les morts et pillèrent la ville, parce qu'on avait déshonoré leur sœur.
\VS{28}Ils prirent leurs troupeaux, leurs bœufs, leurs ânes, et ce qui était dans la ville et dans les champs ;
\VS{29}et toutes leurs richesses, leurs petits enfants, et ils emmenèrent prisonnières leurs femmes ; et ils les pillèrent avec tout ce qui était dans les maisons.
\VS{30}Alors Jacob dit à Siméon et Lévi : Vous m'avez troublé en me rendant odieux aux habitants du pays, aux Cananéens et aux Phérésiens, et je n'ai qu’un petit nombre d’hommes ; ils s'assembleront contre moi, et me frapperont, et me détruiront, moi et ma maison.
\VS{31}Ils répondirent : Doit-on traiter notre sœur comme une prostituée ?
\Chap{35}
\TextTitle{Jacob revient à Béthel pour adorer Yahweh}
\VerseOne{}Or Dieu dit à Jacob : Lève-toi, monte à Béthel, et demeures-y ; là, tu y dresseras un autel au Dieu qui t'apparut lorsque tu fuyais Esaü, ton frère.
\VS{2}Jacob dit à sa famille, et à tous ceux qui étaient avec lui : Otez les dieux des étrangers qui sont au milieu de vous, purifiez-vous, et changez de vêtements\FTNT{Jos. 24:23.}.
\VS{3}Levons-nous et montons à Béthel ; là je dresserai un autel au Dieu qui m'a exaucé dans le jour de ma détresse, et qui a été avec moi dans le chemin où j'ai marché.
\VS{4}Alors ils donnèrent à Jacob tous les dieux des étrangers qui étaient entre leurs mains, et les anneaux qui étaient à leurs oreilles, et il les cacha sous un térébinthe qui est près de Sichem.
\VS{5}Puis ils partirent. Et Dieu frappa de terreur les villes qui les entouraient, et l’on ne poursuivit point les fils de Jacob.
\VS{6}Ainsi Jacob, et tout le peuple qui était avec lui, arrivèrent à Luz, qui est Béthel, dans le pays de Canaan.
\VS{7}Il bâtit là un autel, et il appela ce lieu El-Béthel (le Dieu Puissant de Béthel) ; car c’est là que Dieu s’était révélé à lui lorsqu’il fuyait son frère.
\VS{8}Débora,  nourrice de Rebecca, mourut ; et elle fut ensevelie au-dessous de Béthel sous un chêne, auquel on donna le nom d’Allon-Bacuth (chêne des pleurs).
\VS{9}Dieu apparut encore à Jacob, après son retour de Paddan-Aram, et il le bénit\FTNT{Os. 12:5.}.
\VS{10}Dieu lui dit : Ton nom est Jacob, mais tu ne seras plus appelé Jacob, car ton nom sera Israël. Et il lui donna le nom d’Israël.
\VS{11}Dieu lui dit aussi : Je suis le Dieu Fort, Tout-Puissant. Sois fécond et multiplie : Une nation et une multitude de nations naîtront de toi, et des rois sortiront de tes reins.
\VS{12}Je te donnerai le pays que j'ai donné à Abraham et à Isaac, et je le donnerai à ta postérité après toi.
\VS{13}Dieu s’éleva au-dessus de lui dans le lieu où il lui avait parlé.
\VS{14}Et Jacob dressa un monument dans le lieu où Dieu lui avait parlé, à savoir un monument de pierre, et il fit dessus une aspersion et y versa de l'huile.
\VS{15}Jacob donna le nom de Béthel au lieu où Dieu lui avait parlé.
\VS{16}Puis ils partirent de Béthel, et il y avait encore une certaine distance jusqu’à Ephrata\FTNT{Ephrata : «~lieu de la fécondité~».} lorsque Rachel accoucha. Elle eut un accouchement difficile ;
\VS{17}et comme elle avait beaucoup de peine à accoucher, la sage-femme lui dit : Ne crains point, car tu as encore un fils.
\VS{18}Et comme elle rendait l'âme, car elle était mourante, elle lui donna le nom de Ben-Oni\FTNT{Ben-Oni : «~fils de ma douleur~».}, mais son père l’appela Benjamin\FTNT{Benjamin : «~fils de ma main droite~», «~fils de félicité~».}.
\VS{19}C'est ainsi que mourut Rachel, et elle fut ensevelie sur le chemin d'Ephrata, qui est Bethléhem.
\VS{20}Jacob dressa un monument sur son sépulcre. C'est le monument du sépulcre de Rachel qui subsiste encore aujourd'hui.
\VS{21}Puis Israël partit et dressa ses tentes au-delà de Migdal-Eder.
\VS{22}Pendant qu’Israël habitait dans ce pays, Ruben alla coucher avec Bilha, concubine de son père.  Et Israël l'apprit. Or Jacob avait douze fils.
\VS{23}Les fils de Léa étaient Ruben, premier-né de Jacob, Siméon, Lévi, Juda, Issacar, et Zabulon.
\VS{24}Les fils de Rachel : Joseph et Benjamin.
\VS{25}Les fils de Bilha, servante de Rachel : Dan et Nephthali.
\VS{26}Les fils de Zilpa, servante de Léa : Gad et Aser. Ce sont là les enfants de Jacob qui lui naquirent à Paddan-Aram.
\TextTitle{Jacob voit vers son père Isaac avant sa mort}
\VS{27}Jacob arriva auprès d’Isaac, son père, à la plaine de Mamré, à Kirjath-Arba, qui est Hébron, où Abraham et Isaac avaient séjourné comme étrangers.
\VS{28}Les jours d’Isaac furent de cent quatre-vingts ans.
\VS{29}Isaac expira et mourut, et fut recueilli auprès de son peuple, âgé et rassasié de jours ; et Esaü et Jacob ses fils l'ensevelirent.
\Chap{36}
\TextTitle{Postérité d'Esaü (Edom)}
\VerseOne{}Et voici la postérité d'Esaü, qui est Edom.
\VS{2}Esaü prit ses femmes parmi les filles de Canaan, à savoir Ada, fille d'Elon, le Héthien, Oholibama, fille d’Ana, petite-fille de Tsibeon, le Hévien.
\VS{3}Il prit aussi Basmath, fille d'Ismaël, sœur de Nebajoth.
\VS{4}Ada enfanta à Esaü Eliphaz ; et Basmath enfanta Réuel.
\VS{5}Et Oholibama enfanta Jéusch, Jaelam et Koré. Ce sont là les enfants d'Esaü qui lui naquirent dans le  pays de Canaan.
\VS{6}Esaü prit ses femmes, ses fils et ses filles, et toutes les personnes de sa maison, tous ses troupeaux, ses bêtes, et tout le bien qu'il avait acquis dans le pays de Canaan, et il s'en alla dans un autre pays, loin de Jacob, son frère.
\VS{7}Car leurs richesses étaient si grandes qu'ils n'auraient pas pu demeurer ensemble ; et le pays où ils séjournaient comme étrangers ne pouvait plus les contenir à cause de leurs troupeaux.
\VS{8}Ainsi Esaü habita dans la montagne de Séir ; Esaü est Edom.
\VS{9}Voici la postérité d'Esaü, père d'Edom, dans la montagne de Séir.
\VS{10}Voici les noms des fils d'Esaü : Eliphaz fils d’Ada, femme d'Esaü ; Réuel, fils de Basmath, femme d'Esaü.
\VS{11}Les fils d'Eliphaz furent : Théman, Omar, Tsepho, Gaetham et Kenaz.
\VS{12}Et Timna était la concubine d'Eliphaz, fils d'Esaü, et elle enfanta à Eliphaz Amalek. Ce sont là les fils d’Ada, femme d'Esaü.
\VS{13}Voici les fils de Réuel : Nahath, Zérach, Schamma et Mizza. Ce sont là les fils de Basmath, femme d'Esaü.
\VS{14}Voici les fils d'Oholibama, fille d’Ada, petite fille de Tsibeon, femme d'Esaü ; elle enfanta à Esaü Jéusch, Jaelam et Koré.
\VS{15}Voici les chefs des fils d'Esaü. Voici les fils d'Eliphaz, premier-né d'Esaü, le chef Théman, le chef Omar, le chef Tsepho, le chef Kenaz,
\VS{16}le chef Koré, le chef Gaetham, le chef Amalek. Ce sont là les chefs d'Eliphaz dans le  pays d'Edom. Ce sont les fils d’Ada.
\VS{17}Voici les fils de Réuel, fils d'Esaü : le chef Nahath, le chef Zérach, le chef Schamma, et le chef Mizza. Ce sont là les chefs sortis de Réuel, dans le pays d'Edom.  Ce sont là les fils de Basmath, femme d'Esaü.
\VS{18}Voici les fils d'Oholibama, femme d'Esaü : Le chef Jéusch, le chef Jaelam, le chef Koré. Ce sont là les chefs sortis d'Oholibama, fille d’Ana, femme d'Esaü.
\VS{19}Ce sont là les fils d'Esaü, qui est Edom, et ce sont là leurs chefs.
\VS{20}Voici les fils de Séir, le Horien, qui avaient habité dans le pays : Lothan, Schobal, Tsibeon, Ana,
\VS{21}Dischon, Etser, et Dischan. Ce sont là les chefs des Horiens, fils de Séir, dans le pays d'Edom.
\VS{22}Les fils de Lothan furent Hori et Héman.  Et Thimna était sœur de Lothan.
\VS{23}Voici les fils de Schobal : Alvan, Manahath, Ebal, Schepho et Onam.
\VS{24}Voici les fils de Tsibeon : Ajja et Ana. C’est cet Ana qui trouva les sources chaudes dans le désert, quand il faisait paître les ânes de Tsibeon, son père.
\VS{25}Voici les fils d’Ana : Dischon, et Oholibama, fille d’Ana.
\VS{26}Voici les fils de Dischon : Hemdan, Eschban, Jithran et Keran.
\VS{27}Voici les fils d'Etser : Bilhan, Zaavan et Akan.
\VS{28}Voci les fils de Dischan : Huts et Aran.
\VS{29}Voici les chefs des Horiens : Le chef Lothan, le chef Schobal, le chef Tsibeon, le chef Ana.
\VS{30}Le chef Dischon, le chef Etser, le chef Dischan. Ce sont là les chefs des Horiens, les chefs qu’ils établirent dans le pays de Séir.
\VS{31}Voici les rois qui ont régné dans le pays d'Edom, avant qu’un roi règne sur les enfants d'Israël.
\VS{32}Béla, fils de Béor, régna sur Edom, et le nom de sa ville était Dinhaba.
\VS{33}Béla mourut, et Jobab, fils de Zérach de Botsra, régna à sa place.
\VS{34}Jobab mourut, et Huscham, du pays des Thémanites, régna à sa place.
\VS{35}Huscham mourut, et Hadad, fils de Bédad, régna à sa place. C’est lui qui frappa Madian dans le territoire de Moab ; et le nom de sa ville était Avith.
\VS{36}Hadad mourut, et Samla, de Masréka, régna à sa place.
\VS{37}Samla mourut, et Saül de Réhoboth sur le fleuve, régna à sa place.
\VS{38}Saül mourut, et Baal-Hanan, fils d’Acbor, régna à sa place.
\VS{39}Baal-Hanan, fils de Hacbor mourut, et Hadar régna à sa place. Le nom de sa ville était Pau ; et le nom de sa femme Mehéthabeel, fille de Mathred, petite-fille de Mézahab.
\VS{40}Voici les noms des chefs d'Esaü selon leurs familles, selon leurs territoires, et d’après leurs noms : Le chef Thimna, le chef Alva, le chef Jétheth,
\VS{41}le chef Oholibama, le chef Ela, le chef Pinon,
\VS{42}Le chef Kenaz, le chef Théman, le chef Mibtsar,
\VS{43}le chef Magdiel, et le chef Iram. Ce sont là les chefs d'Edom, selon leurs habitations dans le pays qu’ils possédaient. C'est Esaü le père d'Edom.
\Chap{37}
\TextTitle{Jacob aime Joseph plus que ses autres fils}
\VerseOne{}Or Jacob demeura dans le pays de Canaan, pays où avait séjourné son père comme étranger.
\VS{2}Voici la postérité de Jacob. Joseph, âgé de dix-sept ans, faisait paître le troupeau avec ses frères ; et il était jeune garçon auprès des fils de Bilha et des fils de Zilpa, femmes de son père. Et Joseph rapportait à leur père leurs mauvais propos.
\VS{3}Or Israël aimait Joseph plus que tous ses autres fils, parce qu'il l'avait eu dans sa vieillesse, et il lui fit une tunique de plusieurs couleurs.
\VS{4}Ses frères voyant que leur père l'aimait plus qu'eux tous, le haïssaient et ne pouvaient lui parler paisiblement.
\VS{5}Joseph eut un songe et il raconta à ses frères ; et ils le haïrent encore davantage.
\VS{6}Il leur dit donc : Ecoutez, je vous prie, le songe que j'ai eu.
\VS{7}Voici, nous étions à lier des gerbes au milieu d'un champ ; et voici, ma gerbe se leva et se tint droite ; et voici, vos gerbes l’entourèrent et se prosternèrent devant elle.
\TextTitle{Joseph haï par ses frères}
\VS{8}Alors ses frères lui dirent : Régnerais-tu sur nous ? Et dominerais-tu sur nous ? Et ils le haïrent encore plus pour ses songes et pour ses paroles.
\VS{9}Il eut encore un autre songe, et il le raconta à ses frères, en disant : Voici, j'ai eu encore un songe ; et voici, le soleil, la lune et onze étoiles se prosternaient devant moi\FTNT{Ap. 12:1.}.
\VS{10}Il le raconta à son père et à ses frères. Son père le réprimanda et lui dit : Que veut dire ce songe que tu as eu ? Faut-il que nous venions moi, ta mère, et tes frères, nous prosterner à terre devant toi ?
\VS{11}Ses frères eurent de l'envie contre lui, mais son père garda ses discours\FTNT{Ac. 7:9.}.
\VS{12}Les frères de Joseph s'en allèrent paître les troupeaux de leur père à Sichem.
\VS{13}Israël dit à Joseph : Tes frères ne font-ils pas paître le troupeau à Sichem ? Viens, que je t'envoie vers eux ; et il lui répondit : Me voici.
\VS{14}Israël lui dit : Va maintenant, vois si tes frères se portent bien, et si le troupeau est en bon état, et rapporte-le-moi. Ainsi il l'envoya de la vallée d’Hébron, et il alla jusqu'à Sichem.
\VS{15}Un homme le rencontra, comme il errait dans les champs ; et cet homme le questionna et lui dit : Que cherches-tu ?
\VS{16}Joseph répondit : Je cherche mes frères ; je te prie, dis-moi où ils font paître leur troupeau.
\VS{17}Et l'homme dit : Ils sont partis d'ici, et je les ai entendus dire : Allons à Dothan. Joseph alla après ses frères et les trouva à Dothan.
\VS{18}Ils le virent de loin ; et avant qu'il soit près d’eux, ils complotèrent contre lui pour le tuer.
\VS{19}Ils se dirent l'un à l'autre : Voici ce maître songeur qui arrive.
\TextTitle{Joseph dans la citerne}
\VS{20}Venez maintenant, tuons-le, et jetons-le dans l’une de ces citernes ; et nous dirons qu'une bête féroce l'a dévoré, et nous verrons ce que deviendront ses songes.
\VS{21}Mais Ruben entendit cela et le délivra de leurs mains en disant : Ne lui ôtons point la vie.
\VS{22}Ruben leur dit encore : Ne répandez point le sang ; jetez-le dans cette citerne qui est au désert, mais ne mettez point la main sur lui. C'était pour le délivrer de leurs mains et le renvoyer à son père.
\VS{23}Lorsque Joseph fut arrivé auprès de ses frères, ils le dépouillèrent de sa tunique, de cette tunique de plusieurs couleurs qui était sur lui.
\VS{24}Ils le prirent et le jetèrent dans la citerne.  Cette citerne était vide, il n'y avait point d'eau.
\VS{25}Ensuite, ils s'assirent pour manger du pain ; et levant les yeux, ils virent une caravane d'Ismaélites qui passait et qui venait de Galaad ; et leurs chameaux étaient chargés d’aromates, du baume et de la myrrhe, qu’ils transportaient en Egypte.
\VS{26}Et Juda dit à ses frères : Que gagnerons-nous à tuer notre frère et à cacher son sang ?
\VS{27}Venez, vendons-le à ces Ismaélites, et ne mettons point notre main sur lui, car il est notre frère, notre chair ; et ses frères lui obéirent.
\TextTitle{Joseph vendu à des marchands et emmené en Egypte}
\VS{28}Et comme les marchands Madianites passaient, ils tirèrent et firent remonter Joseph de la citerne, et le vendirent pour vingt pièces d'argent aux Ismaélites, qui emmenèrent Joseph en Egypte\FTNT{Ps. 105:17.}.
\VS{29}Puis Ruben revint à la citerne, et voici, Joseph n'était plus dans la citerne. Alors il déchira ses vêtements.
\VS{30}Il retourna vers ses frères et leur dit : L'enfant n’y est plus ! Et moi ! Moi ! Où irai-je ?
\VS{31}Ils prirent la tunique de Joseph et tuèrent un bouc d'entre les chèvres, ils plongèrent la tunique dans le sang.
\VS{32}Puis ils envoyèrent et firent porter à leur père la tunique de plusieurs couleurs, en lui disant : Voici ce que nous avons trouvé ! Reconnais maintenant si c'est la tunique de ton fils ou non.
\VS{33}Jacob la reconnut, et dit : C'est la tunique de mon fils ! Une bête féroce l'a dévoré ! Certainement Joseph a été déchiré !
\VS{34}Et Jacob déchira ses vêtements, il mit un sac sur ses reins, et il porta le deuil de son fils durant plusieurs jours.
\VS{35}Tous ses fils et toutes ses filles vinrent pour le consoler, mais il rejeta toute consolation. Il disait : C’est en pleurant que je descendrai vers mon fils dans le scheol ! C'est ainsi que son père le pleurait.
\VS{36}Les Madianites le vendirent en Egypte à Potiphar, eunuque de Pharaon, chef des gardes.
\Chap{38}
\TextTitle{Péché de Juda}
\VerseOne{}Il arriva qu’en ce temps-là, Juda s’éloigna de ses frères et se retira vers un homme d’Adullam, nommé Hira.
\VS{2}Là, Juda vit la fille d'un Cananéen, nommé Schua, il la prit pour femme et alla vers elle.
\VS{3}Elle conçut et enfanta un fils qu’elle appela Er.
\VS{4}Elle conçut encore et enfanta un fils qu’elle appela Onan.
\VS{5}Elle enfanta de nouveau un fils qu’elle appela Schéla. Juda était à Czib quand elle l’enfanta.
\VS{6}Juda prit une femme pour Er, son premier-né, une femme nommée Tamar.
\VS{7}Mais Er le premier-né de Juda était méchant devant Yahweh, et Yahweh le fit mourir\FTNT{No. 26:19.}.
\VS{8}Alors Juda dit à Onan : Va vers la femme de ton frère, et prends-la pour femme, comme tu es son beau-frère, et suscite des enfants à ton frère\FTNT{Lé. 25:25  ; Lé. 25:48. Voir commentaire en Ru. 2:20.}.
\VS{9}Mais Onan, sachant que les enfants ne seraient pas à lui, se souillait à terre lorsqu’il allait vers la femme de son frère, afin de ne pas donner de postérité à son frère.
\VS{10}Ce qu'il faisait déplut à Yahweh, c'est pourquoi il le fit aussi mourir.
\VS{11}Et Juda dit à Tamar, sa belle-fille : Demeure veuve dans la maison de ton père, jusqu'à ce que Schéla, mon fils, soit grand ; car il dit : Il faut prendre garde qu'il ne meure comme ses frères. Ainsi Tamar s'en alla et demeura dans la maison de son père.
\VS{12}Et après plusieurs jours, la fille de Schua, femme de Juda, mourut ; lorsque Juda fut consolé, il monta vers ceux qui tondaient ses brebis à Thimna, avec Hira, l’Adullamite, son ami intime.
\VS{13}On en informa Tamar et on lui dit : Voici, ton beau-père monte à Thimna pour tondre ses brebis.
\VS{14}Alors elle ôta ses habits de veuve, se couvrit d'un voile, et s'enveloppa, et elle s’assit à l’entrée d’Enaïm, sur le chemin de Thimna ; car elle voyait que Schéla était devenu grand et qu’elle ne lui était point donnée pour femme.
\VS{15}Et quand Juda la vit, il s'imagina que c'était une prostituée, car elle avait couvert son visage.
\VS{16}Il l’aborda sur le chemin et lui dit : Permets, je te prie, que je vienne vers toi ; car il ne savait pas que c’était sa belle-fille. Et elle répondit : Que me donneras-tu pour venir vers moi ?
\VS{17}Il répondit : Je t'enverrai un chevreau d'entre les chèvres du troupeau. Elle répondit : Me donneras-tu un gage jusqu'à ce que tu l'envoies ?
\VS{18}Il répondit : Quel gage te donnerai-je ? Et elle répondit : Ton cachet, ton cordon, et ton bâton que tu as à la main. Et il les lui donna. Il alla vers elle, et elle devint enceinte de lui.
\VS{19}Puis elle se leva et s'en alla ; elle ôta son voile et remit ses habits de veuve.
\VS{20}Juda envoya un chevreau d'entre ses chèvres par son ami intime l’Adullamite, pour qu'il retire le gage de la main de la femme, mais il ne la trouva point.
\VS{21}Il interrogea les hommes du lieu où elle avait été, en disant : Où est cette prostituée qui était à Enaïm, sur le chemin ? Ils répondirent : Il n'y a point eu ici de prostituée.
\VS{22}Il retourna auprès de Juda et lui dit : Je ne l'ai point trouvée ; et même les gens du lieu m'ont dit : Il n'y a point eu ici de prostituée.
\VS{23}Juda dit : Qu'elle garde le gage, il ne faut pas nous faire mépriser. Voici, j'ai envoyé ce chevreau, mais tu ne l'as point trouvée.
\VS{24}Environ trois mois après, on fit un rapport à Juda, en disant : Tamar, ta belle-fille, a commis un adultère, et voici elle est même enceinte. Et Juda dit : Faites-la sortir, et qu'elle soit brûlée.
\VS{25}Comme on la faisait sortir, elle envoya dire à son beau-père : Je suis enceinte de l'homme à qui ces choses appartiennent. Elle dit aussi : Reconnais, je te prie, à qui est ce cachet, ce cordon, et ce bâton.
\VS{26}Alors Juda les reconnut et il dit : Elle est plus juste que moi, parce que je ne l'ai point donnée à Schéla, mon fils ; et il ne la connut plus.
\VS{27}Quand elle fut au moment d'accoucher, voici, des jumeaux étaient dans son ventre.
\VS{28}Et pendant qu’elle accouchait, il y en eut un qui présenta la main ; la sage-femme la prit et y attacha un fil cramoisi, en disant : Celui-ci sort le premier.
\VS{29}Mais il retira la main, et son frère sortit. Alors la sage-femme dit : Quelle brèche tu as faite ! Et elle lui donna le nom de Pérets.
\VS{30}Ensuite sortit son frère, qui avait à la main le fil cramoisi ; et on lui donna le nom de Zérach.
\Chap{39}
\TextTitle{Joseph fidèle à Yahweh devant la tentation}
\VerseOne{}Or, quand on fit descendre Joseph en Egypte, Potiphar, eunuque de Pharaon, chef des gardes, Egyptien, l'acheta de la main des Ismaélites qui l'y avaient amené.
\VS{2}Yahweh était avec Joseph ; et il prospéra, et demeura dans la maison de son maître,  l’Egyptien.
\VS{3}Son maître vit que Yahweh était avec lui, et que Yahweh faisait prospérer entre ses mains tout ce qu'il faisait.
\VS{4}C'est pourquoi Joseph trouva grâce aux yeux de son maître, qui l’employa à son service. Et son maître l'établit sur sa maison, et lui remit entre les mains tout ce qui lui appartenait.
\VS{5}Dès que Potiphar l’eut établi sur sa maison et sur tout ce qu’il possédait, Yahweh bénit la maison de l’Egyptien, à cause de Joseph ; et la bénédiction de Yahweh fut sur tout ce qui lui appartenait, soit à la maison, soit aux champs.
\VS{6}Il abandonna aux mains de Joseph tout ce qui lui appartenait, et il n’avait avec lui d’autre soin que celui de prendre sa nourriture. Or Joseph était beau de  taille et beau de figure.
\VS{7}Après ces choses, il arriva que la femme de son maître porta les yeux sur Joseph, et elle lui dit : Couche avec moi\FTNT{Pr. 7:9-13.} !
\VS{8}Mais il le refusa, et dit à la femme de son maître : Voici, mon maître ne prend avec moi connaissance de rien dans la maison, et il a remis entre mes mains tout ce qui lui appartient.
\VS{9}Il n'y a personne dans cette maison qui soit plus grand que moi, et il ne m'a rien interdit excepté toi, parce que tu es sa femme ; et comment ferais-je un si grand mal et pécherais-je contre Dieu ?
\VS{10}Quoiqu’elle parlât tous les jours à Joseph, il refusa de coucher auprès d’elle, d’être avec elle.
\VS{11}Un jour qu'il était entré dans la maison pour faire son ouvrage, et qu'il n'y avait là aucun des gens dans la maison,
\VS{12}elle le saisit par son vêtement et lui dit : Couche avec moi ! Mais il laissa son vêtement entre ses mains, s'enfuit, et sortit dehors\FTNT{1 Co. 6:18.}.
\TextTitle{Fausse accusation contre Joseph}
\VS{13}Et lorsqu'elle vit qu'il lui avait laissé son vêtement entre les mains, et qu'il s'était enfui dehors,
\VS{14}elle appela les gens de sa maison, et leur parla en disant : Voyez, on nous a amené un Hébreu pour se moquer de nous.  Cet homme est venu vers moi pour coucher avec moi ; mais j'ai crié à haute voix.
\VS{15}Et dès qu’il a entendu que j’élevais la voix et que je  criais, il a laissé son vêtement à côté de moi, et s’est enfui dehors.
\VS{16}Et elle garda le vêtement de Joseph jusqu'à ce que son maître rentre à la maison.
\VS{17}Alors elle lui parla en ces mêmes termes et dit : Le serviteur Hébreu que tu nous as amené est venu vers moi pour se moquer de moi.
\VS{18}Mais comme j'ai élevé ma voix et que j'ai crié, il a laissé son vêtement à côté de moi et s'est enfui.
\VS{19}Et dès que le maître de Joseph eut entendu les paroles de sa femme qui lui disait : Ton serviteur m'a fait ce que je t'ai dit, sa colère s'enflamma.
\VS{20}Et le maître de Joseph le prit et le mit dans une étroite prison ; dans l'endroit où les prisonniers du roi étaient enfermés, et il fut là en prison.
\VS{21}Mais Yahweh fut avec Joseph ; il étendit sa bonté sur lui et lui fit trouver grâce auprès du chef de la prison.
\VS{22}Et le chef de la prison mit entre les mains de Joseph tous les prisonniers qui étaient dans la prison, et tout ce qu'il y avait à faire, il le faisait.
\VS{23}Le chef de la prison ne prenait aucune connaissance de ce que Joseph avait en main, parce que Yahweh était avec lui. Et Yahweh faisait prospérer tout ce qu'il faisait.
\Chap{40}
\TextTitle{Joseph demeure en prison}
\VerseOne{}Après ces choses, il arriva que l'échanson et le panetier du roi d'Egypte offensèrent leur maître, le roi d'Egypte.
\VS{2}Pharaon fut fort irrité contre ces deux eunuques, contre le chef des échansons, et contre le chef des panetiers.
\VS{3}Et il les fit mettre dans la maison du chef des gardes, dans la prison étroite, dans le même lieu où Joseph était enfermé.
\VS{4}Le chef des gardes les mit entre les mains de Joseph qui les servait ; et ils furent quelques jours en prison.
\VS{5}Pendant une même nuit, l’échanson et le panetier du roi d’Egypte, qui étaient enfermés dans la prison, eurent tous les deux un songe, chacun le sien, pouvant recevoir une explication distincte.
\VS{6}Joseph, étant venu le matin vers eux, les regarda ; et voici, ils étaient fort tristes.
\VS{7}Et il interrogea ces deux eunuques de Pharaon, qui étaient avec lui dans la prison de son maître, et leur dit : Pourquoi avez-vous mauvais visage aujourd'hui ?
\VS{8}Ils lui répondirent : Nous avons eu des songes, et il n'y a personne qui les interprète. Et Joseph leur dit : Les interprétations n’appartiennent-elles pas à Dieu ? Je vous prie, racontez-moi vos songes\FTNT{1 Co. 12:8-10 ; Job. 33:15.}.
\VS{9}Le chef des échansons raconta son songe à Joseph et lui dit : Dans mon songe, voici, il y avait un cep devant moi.
\VS{10}Ce cep avait trois sarments. Quand il eut poussé, sa fleur se développa et ses grappes donnèrent des raisins mûrs.
\VS{11}La coupe de Pharaon était dans ma main. Je pris les raisins, je les pressai dans la coupe de Pharaon, et je mis la coupe dans la main de Pharaon.
\VS{12}Joseph lui dit : Voici son interprétation : Les trois sarments sont trois jours.
\VS{13}Dans trois jours Pharaon élèvera ta tête et te rétablira dans ta charge, et tu mettras la coupe dans sa main, comme tu le faisais auparavant, lorsque tu étais son échanson.
\VS{14}Mais souviens-toi de moi quand tu  seras heureux, et use de bonté envers moi je te prie ; fais mention de moi à Pharaon, afin  qu’il me fasse sortir de cette maison.
\VS{15}Car certainement j'ai été enlevé du pays des Hébreux ; ici non plus je n’ai rien fait  pour  être mis en prison.
\VS{16}Le chef des panetiers, voyant que Joseph avait interprété favorablement ce songe, lui dit : Voici, il y avait aussi dans mon songe trois corbeilles de pain blanc sur ma tête.
\VS{17}Dans la corbeille la plus élevée, il y avait pour Pharaon des mets de toute espèce, cuits au four ; et les oiseaux les mangeaient dans la corbeille au-dessus de ma tête.
\VS{18}Joseph répondit et dit : Voici son interprétation : Les trois corbeilles sont trois jours.
\VS{19}Dans trois jours Pharaon enlèvera ta tête de dessus toi et te fera pendre à un bois, et les oiseaux mangeront ta chair sur toi.
\VS{20}Le troisième jour, jour de la naissance de Pharaon, il fit un festin à tous ses serviteurs ; et il éleva la tête du chef des échansons et la tête du chef des panetiers, au milieu de ses serviteurs.
\VS{21}Il rétablit le chef des échansons dans sa charge d’échanson, pour qu’il mette la coupe dans la main de Pharaon.
\VS{22}Mais il fit pendre le chef des panetiers, selon l’explication que Joseph leur avait donnée.
\VS{23}Cependant, le chef des échansons ne pensa plus à Joseph. Il l’oublia.
\Chap{41}
\TextTitle{Les songes de Pharaon}
\VerseOne{}Mais il arriva qu’au bout de deux ans entiers, Pharaon eut un songe. Et il lui semblait qu'il était près du fleuve.
\VS{2}Et voici, sept jeunes vaches belles à voir, grasses de chair, montèrent hors du fleuve et se mirent à paître dans les  prairies.
\VS{3}Et voici sept autres jeunes vaches, laides à voir, et maigres de chair, montèrent hors du fleuve derrière les autres et se tinrent auprès des autres jeunes vaches sur le bord du fleuve.
\VS{4}Les jeunes vaches laides à voir, et maigres, mangèrent les sept jeunes vaches belles à voir, et grasses. Alors Pharaon s'éveilla.
\VS{5}Il se rendormit et il eut un second songe. Voici, sept épis gras et beaux montèrent sur une même tige.
\VS{6}Et sept épis maigres et brûlés par le vent d’orient poussèrent après eux.
\VS{7}Les épis maigres engloutirent les sept épis gras et pleins. Et Pharaon s'éveilla ; et voilà le songe.
\VS{8}Le matin, Pharaon eut l’esprit troublé, et il envoya appeler tous les magiciens et tous les sages d'Egypte, et leur raconta ses songes.  Mais personne ne put les interpréter à Pharaon.
\VS{9}Alors le chef des échansons parla à Pharaon en disant : Je rappellerai aujourd'hui le souvenir de mes fautes.
\VS{10}Lorsque Pharaon fut irrité contre ses serviteurs, et nous fit mettre, le chef des panetiers et moi, en prison, dans la maison du chef des gardes,
\VS{11}nous eûmes l’un et l’autre un songe dans une même nuit ; et chacun de nous reçut une interprétation en rapport avec le songe qu’il avait eu.
\VS{12}Il y avait là avec nous un garçon Hébreu, esclave du chef des gardes. Nous lui racontâmes nos songes, et il nous les expliqua.
\VS{13}Les choses sont arrivées comme il nous les avait interprétées ; car le roi me rétablit dans ma charge et fit pendre le chef des panetiers.
\TextTitle{Joseph sort de prison et est établi sur l'Egypte par Pharaon}
\VS{14}Alors Pharaon envoya appeler Joseph.  On le fit sortir en hâte de la prison ; on le rasa, et on lui fit changer de vêtements ; puis il se rendit vers Pharaon.
\VS{15}Pharaon dit à Joseph : J'ai eu un songe, et personne ne peut  l'expliquer ; or j'ai appris que tu sais expliquer les songes.
\VS{16}Joseph répondit à Pharaon en disant : Ce n’est pas moi !  C’est Dieu qui donnera une réponse concernant la paix de Pharaon.
\VS{17}Pharaon dit alors à Joseph : Dans mon songe, voici, je me tenais sur le bord du fleuve.
\VS{18}Et voici, sept vaches grasses de chair et belles d’apparence montèrent hors du fleuve et se mirent à paître dans la prairie.
\VS{19}Sept autres vaches montèrent derrière elles, maigres, fort laides d’apparence, et décharnées ; je n’en ai point vu d’aussi laides dans tout le pays d’Egypte.
\VS{20}Les vaches décharnées et laides mangèrent les sept premières vaches qui étaient grasses ;
\VS{21}elles les engloutirent dans leur ventre, sans qu’on s’aperçoive qu’elles y étaient entrées ; et leur apparence était laide comme auparavant. Et je m’éveillai.
\VS{22}Je vis encore en songe sept épis pleins et beaux, qui montèrent sur une même tige.
\VS{23}Et sept épis vides, maigres, brûlés par le vent d’orient, poussèrent après eux.
\VS{24}Les épis maigres engloutirent les sept beaux épis. Je l’ai dit aux magiciens, mais personne ne m’a donné l’explication. 25 Joseph dit à Pharaon : Ce qu’a rêvé Pharaon est une seule chose ; Dieu a fait connaître à Pharaon ce qu’il va faire.
\VS{26}Les sept vaches belles sont sept années ; et les sept épis beaux sont sept années ; c’est un seul songe.
\VS{27}Les sept vaches décharnées et laides, qui montaient derrière les premières, sont sept années ; et les sept épis vides, brûlés par le vent d’orient, seront sept années de famine.
\VS{28}Ainsi, comme je viens de le dire à Pharaon, Dieu a fait connaître à Pharaon ce qu’il va faire.
\VS{29}Voici, il y aura sept années de grande abondance dans tout le pays d’Egypte.
\VS{30}Sept années de famine viendront après elles ; et l’on oubliera toute cette abondance au pays d’Egypte, et la famine consumera le pays.
\VS{31}Cette famine qui suivra sera si forte qu’on ne s’apercevra plus de l’abondance dans le pays.
\VS{32}Si Pharaon a vu le songe se répéter une seconde fois, c’est que la chose est arrêtée de la part de Dieu, et que Dieu se hâtera de l’exécuter.
\VS{33}Maintenant que Pharaon choisisse un homme intelligent et sage, et qu'il l'établisse sur le pays d'Egypte.
\VS{34}Que Pharaon établisse et institue des commissaires sur le pays, et qu'ils prennent la cinquième partie du revenu du pays d'Egypte durant les sept années d'abondance.
\VS{35}Qu’ils rassemblent tous les produits de ces bonnes années  qui viennent ; qu’ils fassent sous l’autorité de Pharaon des amas de blé, des approvisionnements dans les villes, et qu’ils en aient la garde.
\VS{36}Ces provisions seront en réserve pour le pays durant les sept années de famine qui seront dans le  pays d'Egypte, afin que le pays ne soit pas consumé par la famine.
\VS{37}Ces paroles plurent à Pharaon et à tous ses serviteurs\FTNT{Ac. 7:10.}.
\VS{38}Et Pharaon dit à ses serviteurs : Trouverions-nous un homme semblable à celui-ci, qui a l'Esprit de Dieu ?
\VS{39}Et Pharaon dit à Joseph : Puisque Dieu t'a fait connaître toutes ces choses, il n'y a personne qui soit aussi intelligent et aussi sage que toi.
\VS{40}C’est toi qui seras sur ma maison, et tout mon peuple obéira à tes ordres ; je serai seulement plus grand que toi par le trône.
\VS{41}Pharaon dit encore à Joseph : Regarde, je t’établis sur tout le pays d'Egypte.
\VS{42}Alors Pharaon ôta son anneau de sa main et le mit à la main de Joseph ; il le fit revêtir d'habits de fin lin et lui mit un collier d'or au cou.
\VS{43}Il le fit monter sur le char qui suivait le sien, et on criait devant lui : A genoux ! Et il l'établit sur tout le pays d'Egypte.
\VS{44}Et Pharaon dit à Joseph : Je suis Pharaon ! Et sans toi nul ne lèvera la main ni le pied dans tout le pays d'Egypte.
\TextTitle{Joseph épouse une égyptienne}
\VS{45}Pharaon appela Joseph du nom de Tsaphnath-Paenéach ; et il lui donna pour femme Asnath, fille de Poti-Phéra, prêtre d'On. Et Joseph alla visiter le pays d'Egypte.
\VS{46}Joseph était âgé de trente ans lorsqu’il se présenta devant Pharaon, roi d'Egypte ; et il quitta Pharaon et parcourut tout le pays d'Egypte.
\VS{47}Et la terre rapporta très abondamment pendant les sept années de fertilité.
\VS{48}Joseph rassembla tous les produits de ces sept années dans le pays d’Egypte ; il fit des approvisionnements dans les villes, mettant dans l’intérieur de chaque ville les productions des champs d’alentour.
\VS{49}Ainsi Joseph amassa une grande quantité de blé, comme le sable de la mer ; tellement qu'on cessa de le compter, parce qu’il n’y avait plus de nombre.
\VS{50}Avant les années de famine, il naquit à Joseph deux fils, que lui enfanta Asnath, fille de Poti-Phéra, prêtre  d'On.
\VS{51}Joseph donna au premier-né le nom de Manassé, parce que, dit-il, Dieu m'a fait oublier toute ma peine et toute la maison de mon père.
\VS{52}Et il donna au second le nom d’Ephraïm, parce que, dit-il, Dieu m'a fait fructifier dans le  pays de mon affliction.
\VS{53}Alors finirent les sept années de l'abondance qui avaient été dans le pays d'Egypte.
\VS{54}Et les sept années de la famine commencèrent à venir comme Joseph l'avait prédit. Et la famine fut dans tous les pays ; mais il y avait du pain dans tout le pays d'Egypte.
\VS{55}Ensuite tout le pays d'Egypte fut affamé, et le peuple cria à Pharaon pour avoir du pain. Et Pharaon répondit à tous les Egyptiens : Allez vers Joseph, et faites ce qu'il vous dira.
\VS{56}La famine régnait dans tout le pays. Joseph ouvrit tous les lieux d’approvisionnements et vendit du blé aux Egyptiens. La famine augmentait dans le pays d’Egypte.
\VS{57}On venait de tous les pays jusqu’en Egypte, pour acheter du blé  auprès de Joseph ; car la famine était fort grande sur toute la terre.
\Chap{42}
\TextTitle{Les frères de Joseph viennent acheter des vivres en Egypte}
\VerseOne{}Et Jacob, voyant qu'il y avait du blé à vendre en Egypte, dit à ses fils : Pourquoi vous regardez-vous les uns les autres ?
\VS{2}Il leur dit aussi : Voici, j'ai appris qu'il y a du blé à vendre en Egypte, descendez-y pour nous en acheter là, afin que nous vivions, et que nous ne mourions point.
\VS{3}Alors les frères de Joseph descendirent pour acheter du blé en Egypte.
\VS{4}Mais Jacob n'envoya point Benjamin, frère de Joseph, avec ses frères ; car il disait : Il faut prendre garde qu’un malheur ne lui arrive.
\VS{5}Ainsi les fils d'Israël allèrent en Egypte pour acheter du blé avec ceux qui y allaient, car la famine était dans le pays de Canaan.
\TextTitle{Joseph met ses frères à l'épreuve}
\VS{6}Joseph commandait dans le pays, et c’était lui qui vendait le blé à tous les peuples de la terre. Les frères de Joseph vinrent et se prosternèrent devant lui la face contre terre.
\VS{7}Joseph vit ses frères et les reconnut ; mais il feignit d’être un étranger pour eux, et il leur parla rudement, en leur disant : D'où venez-vous ? Et ils répondirent : Du pays de Canaan, pour acheter des vivres.
\VS{8}Joseph reconnut ses frères, mais eux ne le connurent point.
\VS{9}Alors Joseph se souvint des songes qu'il avait eus à leur sujet et leur dit : Vous êtes des espions, vous êtes venus pour observer les lieux faibles du pays.
\VS{10}Et ils lui répondirent : Non, mon seigneur, mais tes serviteurs sont venus pour acheter des vivres.
\VS{11}Nous sommes tous enfants d'un même homme, nous sommes des gens de bien ; tes serviteurs ne sont pas des espions.
\VS{12}Et il leur dit : Nullement ; vous êtes venus pour observer les lieux faibles du pays.
\VS{13}Et ils répondirent : Nous, tes serviteurs, étions douze frères, fils d'un même homme, dans le pays de Canaan. Et voici, le plus jeune est aujourd'hui avec notre père, et l'un n'est plus.
\VS{14}Joseph leur dit : C'est ce que je vous disais, vous êtes des espions.
\VS{15}Voici comment vous serez éprouvés : Par la vie de Pharaon ! Vous ne sortirez pas d'ici que votre jeune frère ne soit venu ici.
\VS{16}Envoyez l’un de vous et qu’il amène votre frère ; et  vous, restez prisonniers. Vos paroles seront éprouvées et je saurai si vous avez dit la vérité. Autrement, par la vie de Pharaon ! Vous êtes des espions.
\VS{17}Et il les mit tous ensemble en prison pendant trois jours.
\VS{18}Le troisième jour, Joseph leur dit : Faites ceci, et vous vivrez. Je crains Dieu !
\VS{19}Si vous êtes sincères, que l'un de vos frères reste enfermé dans votre prison ; et vous, partez et emportez du blé pour nourrir vos familles.
\VS{20}Puis amenez-moi votre jeune frère afin que vos paroles soient éprouvées, et vous ne mourrez point ; et ils firent ainsi.
\TextTitle{Siméon gardé en Egypte en attendant que Benjamin soit présenté à Joseph}
\VS{21}Et ils se dirent alors l'un à l'autre : Nous sommes certainement coupables à l'égard de notre frère ; car nous avons vu l'angoisse de son âme quand il nous demandait grâce, et nous ne l'avons point écouté ; c'est pour cela que cette détresse nous est arrivée.
\VS{22}Ruben leur répondit en disant : Ne vous disais-je pas : Ne commettez point ce péché contre l'enfant ? Et vous ne m’avez point écouté ; et voici que son sang vous est redemandé.
\VS{23}Ils ne savaient pas que Joseph les comprenait, parce qu'il se servait d’un interprète pour leur parler.
\VS{24}Il s’éloigna d’eux pour pleurer. Et il revint, leur parla ; puis il prit parmi eux Siméon, et le fit enchaîner sous leurs yeux.
\VS{25}Et Joseph ordonna qu'on remplisse leurs sacs de blé, et qu'on remette l'argent de chacun d’eux dans son sac, et qu'on leur donne de la provision pour la route ; et cela fut fait ainsi.
\VS{26}Ils chargèrent donc leur blé sur leurs ânes, et s'en allèrent.
\VS{27}L’un d'eux ouvrit son sac pour donner du fourrage à son âne dans l'hôtellerie ; et il vit son argent qui était à l’entrée de son sac.
\VS{28}Il dit à ses frères : Mon argent m'a été rendu ; et le voici dans mon sac. Alors leur cœur fut en défaillance ; et ils furent saisis de peur, et se dirent l'un à l'autre : Qu'est-ce que Dieu nous a fait ?
\VS{29}Et étant arrivés dans le pays de Canaan, vers Jacob leur père, ils lui racontèrent toutes les choses qui leur étaient arrivées, en disant :
\VS{30}L'homme qui est le seigneur du pays, nous a parlé rudement et nous a pris pour des espions du pays.
\VS{31}Mais nous lui avons répondu : Nous sommes sincères, nous ne sommes point des espions.
\VS{32}Nous étions douze frères, fils de notre père ; l'un n'est plus, et le plus jeune est aujourd'hui avec notre père dans le pays de Canaan.
\VS{33}Et cet homme, qui est le seigneur  du pays, nous a dit : A ceci je connaîtrai que vous êtes sincères : Laissez-moi l'un de vos frères, et prenez de quoi nourrir vos familles et partez.
\VS{34}Puis amenez-moi votre jeune frère, et je saurai que vous n'êtes point des espions, que vous êtes sincères ; je vous rendrai votre frère, et vous pourrez librement trafiquer dans le  pays.
\VS{35}Lorsqu’ils vidèrent leurs sacs, voici, le paquet d’argent de chacun était dans son sac. Ils virent, eux et leur père, leurs paquets d’argent, et ils eurent peur.
\VS{36}Jacob leur père leur dit : Vous me privez de mes enfants ! Joseph n'est plus, et Siméon n'est plus, et vous prendriez Benjamin ! C’est sur moi que tout cela retombe.
\VS{37}Ruben parla à son père et lui dit : Fais mourir deux de mes fils si je ne te ramène pas Benjamin ! Remets-le entre mes mains et je te le ramènerai.
\VS{38}Jacob répondit : Mon fils ne descendra point avec vous, car son frère est mort, et il reste seul ; s’il lui arrivait un malheur dans le voyage que vous allez faire, vous feriez descendre mes cheveux blancs avec douleur dans le scheol.
\Chap{43}
\TextTitle{Jacob renvoie ses fils en Egypte\FTNTT{Ge. 37:26-28}}
\VerseOne{}Or la famine devint fort grande dans le pays.
\VS{2}Et quand  ils eurent achevé de manger le blé qu'ils avaient apporté d'Egypte, leur père leur dit : Retournez, achetez-nous un peu de vivres.
\VS{3}Juda lui répondit et lui dit : Cet homme nous a expressément déclaré, disant : Vous ne verrez point ma face, à moins que votre frère ne soit avec vous.
\VS{4}Si donc tu envoies notre frère avec nous, nous descendrons en Egypte et nous t'achèterons des vivres.
\VS{5}Mais si tu ne l'envoies pas, nous n'y descendrons point ; car cet homme nous a dit : Vous ne verrez point ma face, à moins que votre frère ne soit avec vous.
\VS{6}Et Israël dit : Pourquoi avez-vous mal agi à mon égard, en disant à cet homme que vous aviez encore un frère ?
\VS{7}Ils répondirent : Cet homme nous a interrogés sur nous et sur notre famille, en disant : Votre père vit-il encore ? N'avez-vous point de frère ? Et nous lui avons déclaré selon ce qu'il nous avait demandé ; pouvions-nous savoir qu'il dirait : Faites descendre votre frère ?
\VS{8}Juda dit à Israël, son père : Laisse venir l'enfant avec moi, afin que nous nous levions et que nous partions ; et nous vivrons et nous ne mourons point, nous, toi et nos enfants.
\VS{9}Je réponds de lui, tu le redemanderas de ma main. Si je ne te le ramène pas auprès de toi et si je ne le remets pas devant ta face, je serai coupable toute ma vie envers toi.
\VS{10}Car si nous n’avions pas tardé, certainement nous serions déjà de retour deux fois.
\VS{11}Alors Israël leur père leur dit : Si cela est ainsi, faites ceci, prenez dans vos bagages les meilleures productions du pays, pour en  porter un présent à cet homme, un peu de baume, et un peu de miel, des épices, de la myrrhe, des dattes, et des amandes.
\VS{12}Prenez avec vous de l'argent au double dans vos mains, et rapportez l’argent qu’on avait mis à l’entrée de vos sacs ; peut-être était-ce une erreur.
\VS{13}Prenez votre frère, et levez-vous, retournez vers cet homme.
\VS{14}Que le Dieu Tout-Puissant vous fasse trouver grâce devant cet homme, afin qu'il relâche votre autre frère et Benjamin ; et s'il faut que je sois privé de ces deux fils, que j'en sois privé.
\VS{15}Alors ils prirent le présent, et ayant pris de l'argent au double dans leurs mains, et Benjamin, ils se levèrent et descendirent en Egypte ; puis ils se présentèrent devant Joseph.
\VS{16}Dès que Joseph vit Benjamin avec eux, il dit à l’intendant de sa maison : Fais entrer ces gens dans la maison, tue et apprête quelques bêtes, car ils mangeront à midi avec moi.
\VS{17}Cet homme fit ce que Joseph lui avait dit ; et il conduit ces gens dans la maison de Joseph.
\VS{18}Ils eurent peur lorsqu’ils furent conduits dans la maison de Joseph, et ils dirent : Nous sommes emmenés à cause de l'argent remis l’autre fois dans nos sacs ; c’est pour se jeter sur nous, se précipiter sur nous ; c’est pour nous prendre comme esclaves et s’emparer de nos ânes.
\VS{19}Ils s’approchèrent de l’intendant de la maison de Joseph, et lui adressèrent la parole, à l’entrée de la maison.
\VS{20}Ils dirent : Pardon ! Mon seigneur, nous sommes déjà descendus une fois pour acheter des vivres.
\VS{21}Puis, quand nous arrivâmes, au lieu où nous devions passer la nuit, nous avons ouvert nos sacs ; et voici, l’argent de chacun était à l’entrée de son sac, notre argent selon son poids ;  nous le rapportons avec nous.
\VS{22}Nous avons aussi apporté d'autre argent dans nos mains pour acheter des vivres ; et nous ne savons point qui a remis notre argent dans nos sacs.
\VS{23}L’intendant leur dit : Tout va bien pour vous, ne craignez point. C’est votre Dieu, le Dieu de votre père vous a donné un trésor dans vos sacs ; votre argent est parvenu jusqu'à moi ; et il leur amena Siméon.
\VS{24}Cet homme les fit entrer dans la maison de Joseph, et leur donna de l'eau, et ils lavèrent leurs pieds ; il donna aussi à manger à leurs ânes.
\VS{25}Ils préparèrent leur présent en attendant que Joseph revienne à midi ; car ils avaient appris qu'ils mangeraient du pain chez lui.
\VS{26}Quand  Joseph fut arrivé à la maison, ils lui offrirent le présent qu'ils avaient dans leurs mains, et se prosternèrent à terre devant lui dans la maison.
\VS{27}Il leur demanda comment ils se portaient et leur dit : Votre vieux père, dont vous m'avez parlé, se porte-t-il bien ? Vit-il encore ?
\VS{28}Ils répondirent : Ton serviteur, notre père, se porte bien, il vit encore. Et ils s’inclinèrent et se prosternèrent.
\VS{29}Joseph leva les yeux, il vit Benjamin, son frère, fils de sa mère, et il dit : Est-ce là votre jeune frère dont vous m'avez parlé ? Et il ajouta : Mon fils, Dieu te fasse grâce !
\VS{30}Et Joseph se retira promptement, car ses entrailles étaient émues à la vue de son frère, et il cherchait un lieu pour pleurer ; il entra dans sa chambre et il y pleura.
\VS{31}Après s’être lavé le visage, il sortit de là, et faisant des efforts pour se contenir, il dit : Servez le pain.
\VS{32}On servit Joseph à part, et ses frères à part, et les Egyptiens qui mangeaient avec lui furent aussi servis à part, car les Egyptiens ne pouvaient manger du pain avec les Hébreux,  parce que c’est à leurs yeux une abomination.
\VS{33}Les frères de Joseph s’assirent en sa présence, le premier-né selon son droit d’aînesse, et le plus jeune selon son âge ; et ils se regardaient les uns les autres avec étonnement.
\VS{34}Joseph leur fit porter des mets qui étaient devant lui, et Benjamin en eut cinq fois plus que les autres. Ils burent et s’enivrèrent  avec lui.
\Chap{44}
\TextTitle{Juda se rend esclave de Joseph à la place de Benjamin\FTNTT{Ge. 43:9}}
\VerseOne{}Et Joseph donna un ordre à son intendant, en disant : Remplis de vivres les sacs de ces gens, autant qu'ils en pourront porter, et remets l'argent de chacun à l’entrée de son sac.
\VS{2}Tu mettras aussi ma coupe, la coupe d'argent, à l’entrée du sac du plus petit avec l'argent de son blé ; et il fit comme Joseph lui avait dit.
\VS{3}Le matin, dès qu'il fit jour, on renvoya ces hommes avec leurs ânes.
\VS{4}Ils étaient sortis de la ville, ils n’en étaient guère éloignés, lorsque Joseph dit à son intendant : Va, poursuis ces hommes, et quand tu les auras atteints, tu leur diras : Pourquoi avez-vous rendu le mal pour le bien ?
\VS{5}N'est-ce pas la coupe dont se sert mon seigneur pour boire et pour deviner ? Vous avez mal fait d’agir ainsi.
\VS{6}L’intendant les atteignit, et leur dit ces paroles.
\VS{7}Ils lui répondirent : Pourquoi mon seigneur parle-t-il ainsi ? Loin de tes serviteurs la pensée de faire pareille chose !
\VS{8}Voici, nous t'avons rapporté du pays de Canaan l'argent que nous avions trouvé à l’entrée de nos sacs, et comment aurions-nous dérobé de l'argent ou de l'or de la maison de ton maître ?
\VS{9}Que celui de tes serviteurs sur qui se trouvera la coupe meure ; et nous serons aussi esclaves de mon seigneur !
\VS{10}Il leur dit : Qu'il soit fait maintenant selon vos paroles ! Qu’il en soit ainsi ! Que celui sur qui se trouvera la coupe soit mon esclave, et vous, vous serez innocents.
\VS{11}Et ils se hâtèrent de déposer chacun son sac à terre ; et chacun ouvrit son sac.
\VS{12}L’intendant les fouilla, en commençant par le plus âgé, et finissant par le plus jeune ; et la coupe fut trouvée dans le sac de Benjamin.
\VS{13}Alors ils déchirèrent leurs vêtements, et chacun rechargea son âne, et ils retournèrent à la ville.
\VS{14}Juda et ses frères arrivèrent à la maison de Joseph, qui était encore là, et ils se jetèrent à terre devant lui.
\VS{15}Joseph leur dit : Quelle action avez-vous faite ? Ne savez-vous pas qu'un homme tel que moi ne manque pas de deviner ?
\VS{16}Juda lui répondit : Que dirons-nous à mon seigneur ? Comment parlerons-nous ? Et comment nous justifierons-nous ? Dieu a trouvé l'iniquité de tes serviteurs ; voici, nous sommes esclaves de mon seigneur, nous, et celui entre les mains de qui la coupe a été trouvée.
\VS{17}Mais il dit : Loin de moi la pensée d’agir ainsi ! L’homme dans la main duquel la coupe a été trouvée sera mon esclave ; mais vous, remontez en paix vers votre père.
\VS{18}Alors Juda s'approcha de lui en disant : Pardon mon seigneur ! Je te prie, que ton serviteur dise un mot, je te prie aux oreilles de mon seigneur, et que ta colère ne s'enflamme point contre ton serviteur, car tu es comme Pharaon.
\VS{19}Mon seigneur interrogea ses serviteurs en disant : Avez-vous un père ou un frère ?
\VS{20}Nous avons répondu à mon seigneur : Nous avons notre père qui est âgé, et un enfant de sa vieillesse, et qui est le plus jeune d'entre nous ; son frère est mort, et celui-ci est resté le seul enfant  de sa mère ; et son père l'aime.
\VS{21}Tu as dis à tes serviteurs : Faites-le descendre vers moi, et que je le voie de mes yeux.
\VS{22}Nous avons répondu  à mon seigneur : Cet enfant ne peut quitter son père, car s'il le quitte, son père mourra.
\VS{23}Alors tu dis à tes serviteurs : Si votre petit frère ne descend avec vous, vous ne verrez plus ma face.
\VS{24}Lorsque nous sommes remontés auprès de ton serviteur, mon père, nous lui avons rapporté les paroles de mon seigneur.
\VS{25}Notre père nous a dit : Retournez, et achetez-nous un peu de vivres.
\VS{26}Nous lui avons répondu : Nous ne pouvons pas descendre ; mais si notre petit frère est avec nous, nous descendrons, car nous ne pouvons pas voir la face de cet homme, à moins que notre jeune frère ne soit avec nous.
\VS{27}Ton serviteur, mon père, nous répondit : Vous savez que ma femme m'a enfanté deux fils.
\VS{28}L’un étant sorti de chez moi, je pense qu’il a été sans doute déchiré, car je ne l’ai pas revu jusqu’à présent.
\VS{29}Si vous me prenez encore celui-ci, et qu’il lui arrive un malheur, vous ferez descendre mes cheveux blancs avec douleur dans le scheol.
\VS{30}Maintenant, si je retourne auprès de ton serviteur, mon père, sans avoir avec nous l’enfant à l’âme duquel son âme est attachée,
\VS{31}il mourra, en voyant que l’enfant n’y est pas ; et tes serviteurs feront descendre avec douleur dans le scheol les cheveux blancs de ton serviteur, notre père.
\VS{32}De plus, ton serviteur a répondu pour l'enfant, en le prenant à mon père, en disant : Si je ne te le ramène pas, je serai pour toujours coupable envers mon père.
\VS{33}Permets donc, je te prie, à ton serviteur de rester à la place de l’enfant, comme esclave de mon seigneur ; et que l’enfant remonte avec ses frères.
\VS{34}Car comment pourrai-je remonter vers mon père, si l'enfant n'est pas avec moi ? Que je ne voie point l'affliction qu'en aurait mon père !
\Chap{45}
\TextTitle{Joseph révèle son identité à ses frères}
\VerseOne{}Alors Joseph, ne pouvant plus se contenir devant tous ceux qui étaient là présents, cria : Faites sortir tout le monde ! Et il ne resta personne quand il se fit connaître à ses frères.
\VS{2}Et en pleurant, il éleva sa voix, et les Egyptiens l'entendirent, et la maison de Pharaon l'entendit aussi.
\VS{3}Et Joseph dit à ses frères : Je suis Joseph ! Mon père vit-il encore ? Mais ses frères ne pouvaient lui répondre, car ils étaient tout troublés en sa présence.
\VS{4}Joseph dit encore à ses frères : Je vous prie, approchez-vous de moi ; et ils s'approchèrent, et il leur dit : Je suis Joseph, votre frère, que vous avez vendu pour être mené en Egypte\FTNT{Ac. 7:13.}.
\VS{5}Mais maintenant ne soyez pas en peine, et n'ayez point de regret de ce que vous m'avez vendu pour être mené ici, car Dieu m'a envoyé devant vous pour la conservation de votre vie.
\VS{6}Car voici, il y a déjà deux ans que la famine est sur la terre, et il y aura encore cinq ans pendant lesquels il n'y aura ni labour ni moisson.
\VS{7}Mais Dieu m'a envoyé devant vous, pour vous faire subsister sur la terre, et vous faire vivre par une grande délivrance.
\VS{8}Maintenant donc ce n'est pas vous qui m'avez envoyé ici, mais c'est Dieu ; il m'a établi père de Pharaon, et seigneur sur toute sa maison, et gouverneur de tout le pays d'Egypte.
\VS{9}Hâtez-vous d'aller vers mon père, et dites-lui : Ainsi a dit ton fils, Joseph : Dieu m'a établi seigneur sur toute l'Egypte, descends vers moi, ne t'arrête point.
\VS{10}Et tu habiteras dans la contrée de Gosen, et tu seras près de moi, toi, tes fils, et les fils de tes fils, tes brebis, et tes bœufs, et tout ce qui est à toi.
\VS{11}Là, je te nourrirai, car il y aura encore cinq années de famine ; et ainsi tu ne périras point, toi et ta maison, et tout ce qui est à toi.
\VS{12}Et voici, vous voyez de vos yeux, et Benjamin mon frère voit aussi de ses yeux, que c'est moi qui vous parle de ma propre bouche.
\VS{13}Rapportez donc à mon père quelle est ma gloire en Egypte, et tout ce que vous avez vu ; hâtez-vous, et faites descendre ici mon père.
\VS{14}Alors il se jeta sur le cou de Benjamin, son frère, et pleura. Benjamin pleura aussi sur son cou.
\VS{15}Puis il embrassa tous ses frères et pleura sur eux ; après cela ses frères parlèrent avec lui.
\TextTitle{Jacob pardonne ses frères et fait venir son père Jacob\FTNTT{Ge. 43:9}}
\VS{16}Et le bruit se répandit dans la maison de Pharaon que les frères de Joseph étaient venus, ce qui plut fort à Pharaon et à ses serviteurs.
\VS{17}Alors Pharaon dit à Joseph : Dis à tes frères : Faites ceci : Chargez vos bêtes, et allez, retournez dans le pays de Canaan ;
\VS{18}et prenez votre père et vos familles, et revenez vers moi, et je vous donnerai le meilleur du pays d'Egypte ; et vous mangerez la graisse de la terre.
\VS{19}Tu as ordre de leur dire: Faites ceci : Prenez dans le pays d’Egypte des chars pour vos enfants et pour vos femmes; amenez votre père, et venez.
\VS{20}Ne regrettez point ce que vous laisserez, car ce qu’il y a de meilleur dans tout le pays d’Egypte sera pour vous.
\VS{21}Et les fils d'Israël firent ainsi. Et Joseph leur donna des chars selon l'ordre de Pharaon ; il leur donna aussi de la provision pour la route.
\VS{22}Il leur donna à chacun des vêtements de rechange ; et il donna à Benjamin trois cents pièces d'argent et cinq vêtements de rechange.
\VS{23}Il envoya aussi à son père dix ânes chargés des plus excellentes choses qu'il y avait en Egypte, et dix ânesses portant du blé, du pain, et des vivres à son père pour la route.
\VS{24}Il renvoya donc ses frères, et ils partirent ; et il leur dit : Ne vous querellez point en chemin.
\VS{25}Ainsi ils remontèrent d'Egypte, et vinrent dans le  pays de Canaan auprès de Jacob, leur père.
\VS{26}Et ils lui rapportèrent et lui dirent : Joseph vit encore, et même c’est lui qui gouverne tout le pays d'Egypte ; mais le cœur de Jacob resta froid, parce qu’il ne les croyait pas.
\VS{27}Et ils lui dirent toutes les paroles que Joseph leur avait dites ; puis il vit les chars que Joseph avait envoyés pour le porter ; et l'esprit de Jacob, leur père, se ranima.
\VS{28}Alors Israël dit : C'est assez ! Joseph, mon fils, vit encore ! J'irai, et je le verrai avant que je meure.
\Chap{46}
\TextTitle{Jacob en Egypte}
\VerseOne{}Israël donc partit avec tout ce qui lui appartenait, et vint à Beer-Schéba, et il offrit des sacrifices au Dieu de son père Isaac.
\VS{2}Et Dieu parla à Israël dans une vision pendant la nuit et lui dit : Jacob, Jacob ! Et il répondit : Me voici.
\VS{3}Et Dieu lui dit : Je suis le Dieu, le Dieu de ton père. Ne crains point de descendre en Egypte, car là je te ferai devenir une grande nation.
\VS{4}Je descendrai avec toi en Egypte, et je t'en ferai aussi très certainement remonter ; et Joseph te fermera les yeux avec sa main.
\VS{5}Ainsi Jacob partit de Beer-Schéba, et les fils d'Israël mirent Jacob, leur père, et leurs petits enfants, et leurs femmes, sur les chars que Pharaon avait envoyés pour le porter.
\VS{6}Ils emmenèrent aussi leur bétail et leur bien qu'ils avaient acquis dans le pays de Canaan ; et Jacob et toute sa famille avec lui vinrent en Egypte.
\VS{7}Il amena avec lui en Egypte ses fils, et les fils de ses fils, ses filles, et les filles de ses fils, et toute sa famille.
\TextTitle{Les fils de Jacob en Egypte}
\VS{8}Voici les noms des fils d'Israël qui vinrent en Egypte : Jacob et ses fils. Le premier-né de Jacob fut Ruben.
\VS{9}Et les fils de Ruben : Hénoc, Pallu, Hetsron, et Carmi.
\VS{10}Et les fils de Siméon : Jemuel, Jamin, Ohad, Jakin, Tsochar, et Saül, fils d'une Cananéenne.
\VS{11}Et les fils de Lévi : Guerschon, Kehath, et Merari.
\VS{12}Et les fils de Juda : Er, Onan, Schéla, Pérets et Zérach ; mais Er et Onan moururent au pays de Canaan. Les fils de Pérets furent Hetsron et Hamul.
\VS{13}Et les fils d'Issacar : Thola, Puva, Job et Schimron.
\VS{14}Et les fils de Zabulon : Séred, Elon et Jahleel.
\VS{15}Ce sont là les fils de Léa, qu'elle enfanta à Jacob à Paddan-Aram, avec Dina, sa fille. Ses fils et ses filles formaient en tout trente-trois personnes.
\VS{16}Et les fils de Gad : Tsiphjon, Haggi, Schuni, Etsbon, Eri, Arodi et Areéli.
\VS{17}Et les fils d'Aser : Jimna, Jischva, Jischvi, Beria et Sérach, leur sœur. Les fils de Beria : Héber et Malkiel.
\VS{18}Ce sont là les fils de Zilpa que Laban donna à Léa, sa fille ; et elle les enfanta à Jacob. En tout seize personnes.
\VS{19}Les fils de Rachel, femme de Jacob, furent Joseph et Benjamin.
\VS{20}Et il naquit à Joseph dans le  pays d'Egypte, Manassé et Ephraïm, qu'Asnath, fille de Poti-Phéra, prêtre d'On, lui enfanta.
\VS{21}Et les fils de Benjamin étaient Béla, Béker, Aschbel, Guéra, Naaman, Ehi, Rosch, Muppim, Huppim et Ard.
\VS{22}Ce sont là les fils de Rachel, qu'elle enfanta à Jacob. En tout quatorze personnes.
\VS{23}Et les fils de Dan : Huschim.
\VS{24}Et les enfants de Nephthali : Jahtseel, Guni, Jetser, et Schillem.
\VS{25}Ce sont là les fils de Bilha, que Laban donna à Rachel, sa fille, et elle les enfanta à Jacob. En tout sept personnes.
\VS{26}Toutes les personnes appartenant à Jacob qui vinrent en Egypte, et qui étaient issues de lui, sans les femmes des fils de Jacob, furent en tout soixante-dix.
\VS{27}Et les fils de Joseph qui lui étaient nés en Egypte furent deux personnes. Toutes les personnes de la maison de Jacob qui vinrent en Egypte furent soixante-dix.
\VS{28}Jacob envoya Juda devant lui vers Joseph, pour l’informer qu’il se rendait en Gosen. Ils vinrent donc dans la contrée de Gosen.
\VS{29}Et Joseph fit atteler son char, et y monta pour aller à la rencontre d'Israël, son père, en Gosen. Dès qu’il le vit, il se jeta à son cou, et pleura longtemps sur son cou.
\VS{30}Et Israël dit à Joseph : Que je meure à présent, puisque j'ai vu ton visage, et que tu vis encore.
\VS{31}Puis Joseph dit à ses frères et à la famille de son père : Je monterai pour informer Pharaon, et je lui dirai : Mes frères et la famille de mon père, qui étaient au pays de Canaan, sont arrivés auprès de moi.
\VS{32}Et ces hommes sont bergers, ils se sont toujours occupés du bétail, et ils ont amené leurs brebis et leurs bœufs, et tout ce qui était à eux.
\VS{33}Et quand Pharaon vous fera appeler et vous dira : Quel est votre métier ?
\VS{34}Vous direz : Tes serviteurs se sont toujours occupés de bétail dès leur jeunesse jusqu'à maintenant, nous, et nos pères. De cette manière, vous habiterez dans le pays de Gosen, car les Egyptiens ont en abomination les bergers.
\Chap{47}
\TextTitle{La famille de Jacob honorée en Egypte}
\VerseOne{}Joseph alla avertir Pharaon, et lui dit : Mon père  et mes frères sont arrivés du pays de Canaan avec leurs troupeaux et leurs bœufs, et tout ce qui est à eux ; et voici, ils sont dans le pays de Gosen.
\VS{2}Et il prit une partie de ses frères, à savoir cinq, et il les présenta à Pharaon.
\VS{3}Et Pharaon dit aux frères de Joseph : Quel est votre métier ? Ils répondirent à Pharaon : Tes serviteurs sont bergers, comme l'ont été nos pères.
\VS{4}Ils dirent aussi à Pharaon : Nous sommes venus séjourner comme étrangers dans ce pays, parce qu'il n'y a plus de pâturages pour les troupeaux de tes serviteurs, et il y a une grande famine au pays de Canaan ; maintenant nous te prions que tes serviteurs demeurent dans le pays de Gosen.
\VS{5}Et Pharaon parla à Joseph et lui dit : Ton père et tes frères sont arrivés auprès de toi.
\VS{6}Le pays d'Egypte est à ta disposition ; fais habiter ton père et tes frères dans le meilleur endroit du pays ; qu'ils demeurent dans la terre de Gosen ; et si tu connais parmi eux des hommes habiles tu les établiras chefs de tous mes troupeaux.
\VS{7}Alors Joseph amena Jacob, son père, et le présenta à Pharaon ; et Jacob bénit Pharaon.
\VS{8}Et Pharaon dit à Jacob : Quel est le nombre de jours de tes années ?
\VS{9}Jacob répondit à Pharaon : Les jours des années de mes pèlerinages sont de cent trente ans ; les jours des années de ma vie ont été courts et mauvais et n'ont point atteint les jours des années de la vie de mes pères, du temps de leurs pèlerinages.
\VS{10}Jacob donc bénit Pharaon, et sortit de devant lui.
\VS{11}Et Joseph assigna une demeure à son père et à ses frères, et leur donna une possession au pays d'Egypte, au meilleur endroit du pays, dans le pays d'Egypte, comme Pharaon l'avait ordonné.
\VS{12}Et Joseph fournit du pain à son père et à ses frères, et à toute la maison de son père, selon le nombre de leurs familles.
\VS{13}Or il n'y avait point de pain sur toute la terre, car la famine était très grande ; et le pays d'Egypte et le pays de Canaan étaient épuisés par la famine.
\VS{14}Et Joseph amassa tout l'argent qui se trouva dans le pays d'Egypte, et dans le pays de Canaan, contre le blé qu'on achetait ; et il apporta l'argent à la maison de Pharaon.
\VS{15}Quand l'argent du pays d'Egypte et du pays de Canaan fut épuisé, tous les Egyptiens vinrent à Joseph en disant : Donne-nous du pain ; et pourquoi mourrions-nous en ta présence, parce que l'argent manque ?
\VS{16}Joseph répondit : Donnez votre bétail, et je vous en donnerai pour votre bétail, puisque l'argent manque.
\VS{17}Alors ils amenèrent à Joseph leur bétail, et Joseph leur donna du pain pour des chevaux, pour des troupeaux de brebis, pour des troupeaux de boeufs, et pour des ânes ; ainsi il leur fournit du pain en échange de leurs troupeaux cette année-là.
\VS{18}Lorsque cette année fut écoulée, ils revinrent à Joseph l'année suivante et lui dirent : Nous ne cacherons point à mon seigneur que l'argent est épuisé et les troupeaux de bétail ont été amenés à mon seigneur, il ne nous reste plus rien devant mon seigneur que nos corps et nos terres.
\VS{19}Pourquoi mourrions-nous sous tes yeux ? Achète-nous avec nos terres, pour du pain ; et nous serons esclaves de Pharaon, et nos terres seront à lui ; donne-nous aussi de quoi semer, afin que nous vivions et ne mourions point, et que nos terres ne soient point désolées.
\VS{20}Ainsi, Joseph acheta toutes les terres de l’Egypte pour Pharaon ; car les Egyptiens vendirent chacun son champ, parce que la famine les pressait. Et le pays devint la propriété de Pharaon.
\VS{21}Et il fit passer le peuple dans les villes, d’un bout à l’autre des frontières de l’Egypte.
\VS{22}Seulement, il n’acheta point les terres des prêtres, parce qu’il y avait une loi de Pharaon en faveur des prêtres, qui vivaient du revenu que leur assurait Pharaon, c’est pourquoi ils ne vendirent point leurs terres.
\VS{23}Et Joseph dit au peuple : Voici, je vous ai achetés aujourd'hui, vous et vos terres pour Pharaon, voilà de la semence pour ensemencer la terre.
\VS{24}Et quand le temps de la récolte viendra, vous donnerez la cinquième partie à Pharaon, et les quatre autres seront à vous, pour ensemencer les champs, et pour votre nourriture, et pour celle de ceux qui sont dans vos maisons, et pour la nourriture de vos petits enfants.
\VS{25}Et ils dirent : Tu nous sauves la vie ! Que nous trouvions grâce aux yeux de mon seigneur, et nous serons esclaves de Pharaon.
\VS{26}Et Joseph fit de cela une loi qui a subsisté jusqu’à ce jour, et d’après laquelle un cinquième du revenu des terres de l’Egypte appartient à Pharaon ; il n’y a que les terres des prêtres qui ne soient point à Pharaon.
\TextTitle{Jacob demande à être enterré à Canaan}
\VS{27}Israël habita dans le pays d’Egypte, dans le pays de Gosen. Ils eurent des possessions, ils furent féconds et multiplièrent beaucoup.
\VS{28}Jacob vécut dix-sept ans dans le pays d’Egypte ; et les jours des années de la vie de Jacob furent de cent quarante-sept ans.
\VS{29}Et quand le jour de la mort d'Israël approcha, il appela Joseph, son fils, et lui dit : Je te prie, si j'ai trouvé grâce à tes yeux, mets présentement ta main sous ma cuisse, et jure-moi que tu useras envers moi de bonté et de fidélité : Je te prie, ne m'enterre point en Egypte !
\VS{30}Quand  je serai couché avec mes pères, tu me transporteras hors de l'Egypte, et m'enterreras dans leur sépulcre. Et il répondit : Je le ferai selon ta parole.
\VS{31}Et Jacob lui dit : Jure-le-moi ; et il le lui jura. Et Israël se prosterna sur le chevet du lit.
\Chap{48}
\TextTitle{Bénédiction de Jacob sur les fils de Joseph}
\VerseOne{}Or il arriva après ces choses que l'on vint dire à Joseph : Voici, ton père est malade. Et il prit avec lui ses deux fils, Manassé et Ephraïm.
\VS{2}On avertit Jacob et on lui dit : Voici Joseph, ton fils, qui vient vers toi. Alors Israël rassembla ses forces et s’assit sur son lit.
\VS{3}Puis Jacob dit à Joseph : Le Dieu Tout-Puissant m’est apparu  à Luz, au pays de Canaan, et m’a béni.
\VS{4}Et il m’a dit : Voici, je te ferai croître et multiplier, et je te ferai devenir une assemblée de peuples, et je donnerai ce pays en possession perpétuelle à ta postérité après toi.
\VS{5}Et maintenant tes deux fils, qui te sont nés au pays d'Egypte, avant mon arrivée vers toi, seront à moi : Ephraïm et Manassé seront à moi comme Ruben et Siméon.
\VS{6}Mais les enfants que tu auras engendrés après eux, seront à toi, et ils seront appelés selon le nom de leurs frères dans leur héritage.
\VS{7}A mon retour de Paddan, Rachel mourut en route auprès de moi, dans le pays de Canaan, à quelque distance d’Ephrata ; et c’est là que je l’ai enterrée, sur le chemin d’Ephrata, qui est Bethléhem.
\VS{8}Puis Israël vit les fils de Joseph, et il dit : Qui sont ceux-ci ?
\VS{9}Et Joseph répondit à son père : Ce sont mes fils que Dieu m'a donnés ici ; et il dit : Amène-les-moi, je te prie, afin que je les bénisse.
\VS{10}Or les yeux d'Israël étaient appesantis par la vieillesse, et il ne pouvait plus voir ; et il les fit approcher de lui, les embrassa et les prit dans ses bras.
\VS{11}Et Israël dit à Joseph : Je ne pensais pas revoir ton visage ; et voici, Dieu m'a fait voir et toi et ta postérité.
\VS{12}Et Joseph les retira des genoux de son père, et se prosterna le visage contre terre.
\VS{13}Puis Joseph les prit tous deux, Ephraïm de sa main droite à la gauche d’Israël, et Manassé de sa main gauche à la droite d’Israël, et il les fit approcher de lui.
\VS{14}Israël étendit sa main droite et la posa sur la tête d’Ephraïm qui était le plus jeune, et il posa sa main gauche sur la tête de Manassé ; ce fut avec intention qu’il posa ses mains ainsi, car Manassé était le premier-né.
\VS{15}Il bénit Joseph et dit : Que le Dieu en présence duquel ont marché mes pères, Abraham et Isaac, que le Dieu qui m’a conduit depuis que j’existe jusqu’à ce jour\FTNT{Hé. 11:21.},
\VS{16}que l’Ange qui m’a délivré de tout mal, bénisse ces enfants ! Qu’ils soient appelés de mon nom et du nom de mes pères, Abraham et Isaac, et qu’ils multiplient en abondance comme les poissons au milieu du pays.
\VS{17}Joseph vit avec déplaisir que son père posait sa main droite sur la tête d’Ephraïm ; il saisit la main de son père, pour la détourner de dessus la tête d’Ephraïm, et la diriger sur celle de Manassé.
\VS{18}Et Joseph dit à son père : Ce n'est pas ainsi mon père ! Car celui-ci est l'aîné ; mets ta main droite sur sa tête.
\VS{19}Mais son père le refusa en disant : Je le sais, mon fils, je le sais. Celui-ci deviendra aussi un peuple, et même il sera grand ; mais toutefois son frère, qui est plus jeune, sera plus grand que lui, et sa postérité sera une multitude de nations.
\VS{20}Il les bénit ce jour-là et dit : C’est par toi qu’Israël bénira en disant : Que Dieu te traite comme Ephraïm et comme Manassé ! Et il mit Ephraïm avant Manassé.
\VS{21}Puis Israël dit à Joseph : Voici, je  vais mourir, mais Dieu sera avec vous, et vous fera retourner au pays de vos pères.
\VS{22}Et je te donne une portion  de plus qu'à tes frères, celle que j'ai prise avec mon épée et mon arc sur les Amoréens.
\Chap{49}
\TextTitle{Prophétie de Jacob qui bénit ses fils}
\VerseOne{}Puis Jacob appela ses fils et leur dit : Assemblez-vous, et je vous annoncerai ce qui vous arrivera dans les derniers jours\FTNT{L’expression «~dans les derniers jours~» vient de l’hébreu «~achariyth~» qui veut dire «~dernier~». Son équivalent grec est «~eschatos~» : «~dernier~», «~extrémité~» etc. Jacob est le premier homme à avoir utilisé  cette expression. Cette promesse de Jacob devait arriver à Israël dans les derniers jours, selon leurs tribus. Ainsi, les promesses du droit d'aînesse de Ge. 49 étaient pour l'âge messianique, lequel est associé aux derniers jours, et a commencé à la Fête de la Pentecôte (Ac. 2:14-21). 
Ces jours impliquent :
- L’effusion de l’Esprit, le réveil de l’Eglise de Christ (Mt. 25:1-13 ; Ac. 2)
- Le réveil des faux prophètes ou l’apostasie (2 Pi. 3:3 ; 1 Jn. 2)
- La dégradation de la moralité (2 Ti. 3)
- L’enrichissement des hommes de ce monde (Ja. 5:3 ; Ap. 3:14-22)
- Le fait que Dieu nous parle par le Fils (Hé. 1:2)
- La future résurrection des saints lors du retour du Messie (Jn. 6:39-54 ; 1 Th. 4:12-17).Le temps des nations (fin des temps) s’achèvera lors du retour visible de Jésus-Christ pour établir son règne sur toute la terre. Le temps des nations a commencé lorsque, à la suite de l’infidélité d’Israël, la gloire de Dieu a quitté le temple et la ville de Jérusalem (Ez. 11), la puissance fut confiée aux nations en la personne de Nebucadnetsar qui s’empara de Jérusalem (2 R. 24 et 25 ; 2 Ch. 36:6-21 ; Da. 1 ; Jé 39). Ces temps dureront jusqu’à la destruction finale du dernier empire des nations représenté par la Bête romaine ressuscitée (Ap. 13:3). Cette destruction n’aura lieu que lorsque Jésus-Christ, la pierre détachée sans le secours d’aucune main, deviendra une grande montagne qui remplira toute la terre (Da. 2:34 ; Mi. 4). Jérusalem ne sera délivrée du joug des nations qu’à ce moment-là. Les temps des nations ne seront accomplis que lorsque le trône de Dieu sera de nouveau établi à Jérusalem.}.
\VS{2}Rassemblez-vous, et écoutez, fils de Jacob ; écoutez Israël\FTNT{«~Ecoutez Israël~» : Le «~shema~» Israël est le texte principal de la liturgie juive. Composé de trois extraits de la Torah, on le récite matin et soir accompagné de bénédictions. Voir De. 6:4-9.}, votre père.
\VS{3}Ruben, tu es mon premier-né, ma force et le commencement de ma vigueur, qui excelle en dignité et qui excelle aussi en force ;
\VS{4}impétueux comme les eaux ; tu n'auras pas la prééminence, car tu es monté sur la couche de ton père, et tu as souillé mon lit en y montant.
\VS{5}Siméon et Lévi, sont frères, leurs glaives sont des instruments de violence dans leurs demeures.
\VS{6}Que mon âme n'entre point dans leur conseil secret, que ma gloire ne soit point jointe à leur compagnie, car ils ont tué les gens dans leur colère, et ont enlevé les bœufs pour leur plaisir.
\VS{7}Maudite soit leur colère, car elle a été violente ; et leur fureur, car elle a été cruelle ; je les diviserai dans Jacob, et les disperserai dans Israël.
\VS{8}Juda, quant à toi, tes frères te loueront ; ta main sera sur la nuque de tes ennemis ; les fils de ton père se prosterneront devant toi.
\VS{9}Juda est un jeune lion. Mon fils, tu reviens du carnage, mon fils ! Il ploie les genoux, il se couche comme un lion, comme une lionne : Qui le fera lever ?
\VS{10}Le sceptre ne s’éloignera point de Juda, ni le bâton de législateur d'entre ses pieds, jusqu'à ce que le Schilo vienne, et que  les peuples lui obéissent.
\VS{11}Il attache à la vigne son ânon, et au cep excellent le petit de son ânesse ; il lavera son vêtement dans le vin, et son vêtement dans le sang des raisins.
\VS{12}Il a les yeux rouges de vin, et les dents blanches de lait.
\VS{13}Zabulon habitera sur la côte des mers, il sera un port des navires ; et ses côtés s'étendront vers Sidon.
\VS{14}Issacar est un âne robuste, couché entre les barres des étables.
\VS{15}Il voit que le lieu où il repose est agréable et que la contrée est magnifique ; et il courbe son épaule sous le fardeau, il s’assujettit à un tribut.
\VS{16}Dan jugera son peuple, comme l’une des tribus d'Israël.
\VS{17}Dan sera un serpent sur le chemin, une vipère sur le sentier, mordant les talons du cheval, pour que le cavalier tombe à la renverse.
\VS{18}Ô Yahweh ! J’espère en ton salut\FTNT{Le mot secours se dit «~yeshuw`ah~» en hébreu et veut littéralement dire  «~salut~», «~délivrance~». Avant de mourir, Jacob a  donc  placé son espérance en Jésus-Christ qui est la résurrection et la vie (Jn. 11:25). Voir commentaire en Es. 26:1.} !
\VS{19}Quant à Gad, des troupes viendront l’attaquer, mais il ravagera leur arrière-garde.
\VS{20}Le pain excellent viendra d'Aser, et il fournira les mets délicats des rois.
\VS{21}Nephtali est une biche en liberté ; il profère des belles paroles.
\VS{22}Joseph est un fils fertile, un rameau fertile près d'une fontaine ; ses branches se sont étendues sur la muraille.
\VS{23}Des archers l’ont provoqué, ils ont lancé des traits ; les archers l’ont poursuivi de leur haine.
\VS{24}Mais son arc est demeuré ferme, et ses mains ont été fortifiées par les mains du Puissant de Jacob : Il est ainsi devenu le pasteur, le rocher d’Israël.
\VS{25}C’est l’œuvre du Dieu de ton père qui t’aidera ; c’est l’œuvre du Tout-Puissant qui te bénira des bénédictions des cieux en haut, des bénédictions des eaux en bas, des bénédictions des mamelles et du sein maternel.
\VS{26}Les bénédictions de ton père ont surpassé les bénédictions de ceux qui m'ont engendré, jusqu'à la cime des antiques collines ; elles seront sur la tête de Joseph, et sur le sommet de la tête du Nazaréen d'entre ses frères.
\VS{27}Benjamin est un loup qui déchirera ; le matin il dévorera la proie, et sur le soir il partagera le butin.
\VS{28}Ce sont là tous ceux qui forment les douze tribus d'Israël.  Et c’est là ce que leur père leur dit en les bénissant. Il bénit chacun d'eux selon la bénédiction qui lui était propre.
\VS{29}Il leur donna aussi cet ordre : Je vais être recueilli auprès de mon peuple, enterrez-moi avec mes pères dans la caverne qui est au champ d'Ephron, le Héthien,
\VS{30}dans la caverne du champ de Macpéla, vis-à-vis de Mamré, dans le pays de Canaan. C’est le champ qu’Abraham a acheté d’Ephron, le Héthien, comme propriété sépulcrale.
\VS{31}C'est là qu'on a enterré Abraham avec Sara, sa femme ; c'est là qu'on a enterré Isaac et Rebecca, sa femme ; et c'est là que j'ai enterré Léa.
\VS{32}Le champ a été acquis des fils de Heth avec la caverne qui s’y trouve.
\VS{33}Lorsque Jacob eut achevé de donner ses ordres à ses fils, il retira ses pieds dans le lit, il expira, et fut recueilli auprès de son peuple.
\Chap{50}
\TextTitle{Mort de Jacob}
\VerseOne{}Alors Joseph se jeta sur le visage de son père, pleura sur lui et l’embrassa.
\VS{2}Et Joseph ordonna à ceux de ses serviteurs qui étaient médecins d'embaumer son père ; et les médecins embaumèrent Israël.
\VS{3}Et on employa quarante jours à l'embaumer, car c'était la coutume d'embaumer les corps pendant quarante jours ; et les Egyptiens le pleurèrent soixante-dix jours.
\VS{4}Quand les jours du deuil furent passés, Joseph s’adressa aux gens de la maison de Pharaon, et leur dit : Si j’ai trouvé grâce à vos yeux, rapportez, je vous prie, à Pharaon ce que je vous dis.
\VS{5}Mon père m’a fait jurer en disant : Voici, je vais mourir ! Tu m’enterreras dans le sépulcre que je me suis acheté au pays de Canaan. Je voudrais donc y monter, pour enterrer mon père ; et je reviendrai.
\VS{6}Et Pharaon répondit : Monte, et enterre ton père comme il t'a fait jurer.
\VS{7}Alors Joseph monta pour enterrer son père, et les serviteurs de Pharaon, les anciens de la maison de Pharaon, et tous les anciens du pays d'Egypte montèrent avec lui.
\VS{8}Et toute la maison de Joseph, et ses frères, et la maison de son père y montèrent aussi, laissant seulement leurs familles, et leurs troupeaux, et leurs bœufs dans le pays de Gosen.
\VS{9}Il y avait encore avec Joseph des chars et des cavaliers, en sorte que le cortège était très nombreux.
\VS{10}Arrivés à l’aire d’Athad, qui est au-delà du Jourdain, ils firent entendre de grandes et profondes lamentations ; et Joseph fit en l’honneur de son père un deuil de sept jours.
\VS{11}Et les Cananéens, habitants du pays, voyant ce deuil dans l'aire d'Athad, dirent : Ce deuil est grand pour les Egyptiens ; c'est pourquoi cette aire, qui est au-delà du Jourdain, fut nommée Abel-Mitsraïm\FTNT{Abel-Mitsraïm : «~Pré du deuil de l’Egypte~».}.
\VS{12}Les fils de Jacob firent à l'égard de son corps ce qu'il leur avait ordonné.
\VS{13}Ils le transportèrent au pays de Canaan, et l’enterrèrent dans la caverne du champ de Macpéla, qu’Abraham avait achetée d’Ephron, le Héthien, comme propriété sépulcrale, et qui est vis-à-vis de Mamré.
\VS{14}Et après que Joseph eut enseveli son père, il retourna en Egypte avec ses frères et tous ceux qui étaient montés avec lui pour ensevelir son père.
\VS{15}Et les frères de Joseph, voyant que leur père était mort, se dirent entre eux : Peut-être que Joseph nous aura en haine, et ne manquera pas de nous rendre tout le mal que nous lui avons fait.
\VS{16}Et ils firent dire à Joseph : Ton père a donné cet ordre avant de mourir en disant :
\VS{17}Vous parlerez ainsi à Joseph : Je te prie, pardonne maintenant l'iniquité de tes frères, et leur péché, car ils t'ont fait du mal. Maintenant, je te supplie, pardonne le crime des serviteurs du Dieu de ton père. Et Joseph pleura quand on lui parla.
\VS{18}Ses frères vinrent eux-mêmes se prosterner devant lui, et ils dirent : Nous sommes tes serviteurs.
\VS{19}Et Joseph leur dit : Ne craignez point, car suis-je à la place de Dieu ?
\VS{20}Vous aviez médité de me faire du mal : Dieu l’a changé en bien, pour accomplir ce qui arrive aujourd’hui, pour sauver la vie à un peuple nombreux.
\VS{21}Soyez donc sans crainte ; je vous entretiendrai, vous et vos familles ; et il les consola en parlant à leur cœur.
\VS{22}Joseph demeura donc en Egypte, lui et la maison de son père, et vécut cent dix ans.
\VS{23}Et Joseph vit les fils d'Ephraïm jusqu'à la troisième génération. Makir aussi, fils de Manassé, eut des fils qui furent élevés sur les genoux de Joseph.
\VS{24}Et Joseph dit à ses frères : Je vais mourir ! Mais Dieu ne manquera pas de vous visiter, et il vous fera remonter de ce pays au pays qu’il a juré  de donner à Abraham, Isaac et à Jacob.
\VS{25}Et Joseph fit jurer les enfants d'Israël et leur dit : Dieu ne manquera pas de vous visiter, et alors vous transporterez mes os d'ici\FTNT{Hé. 11:22 ; Ex. 13:19.}.
\VS{26}Puis Joseph mourut, âgé de cent dix ans. On l'embauma, et on le mit dans un cercueil en Egypte.
\PPE{}
\end{multicols}

%\clearpage\ShortTitle{Ex.}\BookTitle{Exode}\BFont
\noindent\hrulefill
{\footnotesize
\textit{
\bigskip
{\centering{}
\\Auteur~: Probablement Moïse
\\(Heb.~: Shemot)
\\Signification~: Noms
\\Thème~: La délivrance
\\Date de rédaction~: Env. 1450-1410 av. J.-C.\\}
}
%\bigskip
\textit{
\\Les fils de Jacob s'étaient retrouvés en Egypte pour survivre à une famine qui avait frappé la terre entière pendant plusieurs années. Grâce à leur frère Joseph, alors gouverneur d'Egypte, ils bénéficièrent d'un bon traitement. Mais la mort de ce dernier et la montée au pouvoir d'un nouveau Pharaon (probablement Ramsès II) inaugurèrent une période de quatre siècles de souffrances pour le peuple élu.
%\bigskip
\\En effet, les Hébreux avaient été réduits en esclavage. En réponse aux cris de douleur de son peuple, Dieu suscita Moïse, dont le nom signifie «~tiré de~». Ce descendant de Lévi fut élevé dans le palais de Pharaon, mais dut s'enfuir parce qu'il avait tué un Egyptien. Après quarante ans passés dans le pays de Madian, le Dieu qui s'appelle «~Je suis~» se révéla à Moïse sur la montagne d'Horeb et lui confia la mission d'aller délivrer son peuple du joug égyptien.
%\bigskip
\\Ce livre retrace la sortie d'Egypte et le début de la traversée du désert, jalonnée de prodiges exceptionnels.\bigskip
}
}
\par\nobreak\noindent\hrulefill
\begin{multicols}{2}
\Chap{1}
\TextTitle{Après la mort de Joseph}
\VerseOne{}Et ce sont ici les noms des fils d'Israël qui entrèrent en Egypte avec Jacob. Ils y entrèrent chacun avec sa famille~: 
\VS{2}Ruben, Siméon, Lévi, et Juda,
\VS{3}Issacar, Zabulon, et Benjamin,
\VS{4}Dan, et Nephthali, Gad, et Aser.
\VS{5}Toutes les personnes issues des reins de Jacob étaient soixante-dix âmes. Joseph était alors en Egypte.
\VS{6}Joseph mourut ainsi que tous ses frères et toute cette génération-là.
\VS{7}Les enfants d'Israël fructifièrent et s'accrurent abondamment, et se multiplièrent et devinrent extrêmement puissants, de sorte que le pays en fut rempli\FTNT{De. 26:5~; Ac. 7:17.}.
\TextTitle{Israël esclave en Egypte}
\VS{8}Depuis, il s'éleva un nouveau roi sur l'Egypte, qui n'avait point connu Joseph.
\VS{9}Et il dit à son peuple~: Voici, le peuple des enfants d'Israël est plus grand et plus puissant que nous.
\VS{10}Agissons donc prudemment avec lui, de peur qu'il ne se multiplie, et que s'il survenait une guerre, il ne se joigne à nos ennemis, ne fasse la guerre contre nous, et qu'il ne s'en aille du pays.
\VS{11}Ils établirent donc sur le peuple des commissaires d'impôts, pour l'affliger en le surchargeant~; car le peuple bâtit des villes à greniers pour Pharaon~; à savoir Pithom et Ramsès.
\VS{12}Mais plus ils l'affligeaient et plus il multipliait et croissait en toute abondance~; c'est pourquoi ils haïssaient les enfants d'Israël\FTNT{Ps. 105:24.}.
\VS{13}Et les Egyptiens assujettirent les enfants d'Israël à une rude servitude\FTNT{Ge. 15:13.}.
\VS{14}Tellement qu'ils leur rendirent la vie amère par un rude travail, en les employant à faire du mortier, des briques, et toute sorte d'ouvrage qui se fait aux champs~; c'était avec cruauté qu'ils leur imposaient toutes ces charges.
\VS{15}Le roi d'Egypte parla aussi aux sages-femmes des Hébreux, nommées l'une Schiphra et l'autre Pua.
\VS{16}Il leur dit~: Quand vous accoucherez les femmes des Hébreux, et que vous les verrez sur les sièges, si c'est un fils, mettez-le à mort~; mais si c'est une fille, qu'elle vive.
\VS{17}Mais les sages-femmes craignirent Dieu et ne firent pas ce que le roi d'Egypte leur avait dit~; car elles laissèrent vivre les fils.
\VS{18}Alors le roi d'Egypte appela les sages-femmes et leur dit~: Pourquoi avez-vous fait cela, et avez-vous laissé vivre les fils~?
\VS{19}Les sages-femmes répondirent à Pharaon~: Parce que les femmes des Hébreux ne sont point comme les femmes Egyptiennes~; car elles sont vigoureuses, elles ont accouché avant que la sage-femme ne soit arrivée chez elles.
\VS{20}Dieu fit du bien aux sages-femmes~; et le peuple multiplia et devint très puissant.
\VS{21}Parce que les sages-femmes craignirent Dieu, il leur édifia des maisons.
\VS{22}Alors Pharaon donna cet ordre à tout son peuple~: Jetez dans le fleuve tous les fils qui naîtront, mais laissez vivre toutes les filles.
\Chap{2}
\TextTitle{Naissance de Moïse\FTNTT{Hé. 11:23-27.}}
\VerseOne{}Un homme de la maison de Lévi s'en alla et prit une fille de Lévi\FTNT{No. 26:59.}.
\VS{2}Cette femme conçut et enfanta un fils. Voyant qu'il était beau, elle le cacha pendant trois mois\FTNT{Hé. 11:23}.
\VS{3}Mais ne pouvant le tenir caché plus longtemps, elle prit une arche de jonc, et l'enduisit de bitume et de poix, mit l'enfant dedans, et le posa parmi des roseaux sur le bord du fleuve.
\VS{4}Et la sœur de cet enfant se tenait loin pour savoir ce qu'il en arriverait.
\VS{5}La fille de Pharaon descendit à la rivière pour se baigner, et ses servantes se promenaient sur le bord de la rivière, et ayant vu le coffret au milieu des roseaux, elle envoya une de ses servantes pour le prendre.
\VS{6}Et l'ayant ouvert, elle vit l'enfant et voici l'enfant pleurait. Elle en fut touchée de compassion et dit~: C'est un des enfants de ces Hébreux~!
\VS{7}Alors la sœur de l'enfant dit à la fille de Pharaon~: Irai-je appeler une femme d'entre les Hébreux, qui allaite~? Et elle t'allaitera cet enfant.
\VS{8}La fille de Pharaon lui répondit~: Va~! Et la jeune fille s'en alla et appela la mère de l'enfant.
\VS{9}Et La fille de Pharaon lui dit~: Emporte cet enfant, et allaite-le moi, je te donnerai ton salaire~; et la femme prit l'enfant et l'allaita.
\VS{10}Quand l'enfant fut devenu grand, elle l'amena à la fille de Pharaon~; il fut pour elle comme un fils. Elle lui donna le nom de Moïse parce que, dit-elle, je l'ai tiré des eaux.
\TextTitle{Moïse prend à cœur le sort d'Israël~; fuite à Madian}
\VS{11}Or il arriva, en ce temps-là, que Moïse, étant devenu grand, sortit vers ses frères et vit leurs travaux~; il vit aussi un Egyptien qui frappait un Hébreu d'entre ses frères\FTNT{Hé. 11:24-25.}.
\VS{12}Et ayant regardé çà et là, et voyant qu'il n'y avait personne, il tua l'Egyptien et le cacha dans le sable.
\VS{13}Il sortit encore le second jour~; et voici, deux hommes Hébreux se querellaient. Il dit à celui qui avait tort~: Pourquoi frappes-tu ton prochain~?
\VS{14}Lequel répondit~: Qui t'a établi prince et juge sur nous~? Veux-tu me tuer comme tu as tué l'Egyptien~? Et Moïse craignit, et dit~: Certainement le fait est connu.
\VS{15}Or Pharaon ayant appris ce fait-là, chercha à faire mourir Moïse~; mais Moïse s'enfuit de devant Pharaon, s'arrêta au pays de Madian et s'assit près d'un puits.
\VS{16}Or le prêtre de Madian avait sept filles qui vinrent puiser de l'eau, et elles remplirent les auges pour abreuver le troupeau de leur père.
\VS{17}Mais des bergers survinrent et les chassèrent~; et Moïse se leva et les secourut, et abreuva leur troupeau.
\VS{18}Et quand elles furent revenues chez Réuel, leur père, il leur dit~: Comment êtes-vous revenues si tôt aujourd'hui~?
\VS{19}Elles répondirent~: Un homme Egyptien nous a délivrées de la main des bergers~; et même il nous a puisé abondamment de l'eau et a abreuvé le troupeau.
\VS{20}Il dit à ses filles~: Où est-il~? Pourquoi avez-vous ainsi laissé cet homme~? Appelez-le, et qu'il mange du pain.
\VS{21}Et Moïse s'accorda de demeurer avec cet homme-là, qui donna Séphora, sa fille, à Moïse.
\VS{22}Et elle enfanta un fils, et il le nomma Guerschom~: Car, dit-il, je séjourne dans un pays étranger.
\TextTitle{Yahweh entend les cris de son peuple}
\VS{23}Or il arriva longtemps après que le roi d'Egypte mourut, et les enfants d'Israël soupirèrent à cause de la servitude, et ils crièrent~; et leur cri monta jusqu'à Dieu, à cause de la servitude\FTNT{No. 20:15-16.}.
\VS{24}Dieu entendit leurs gémissements, et Dieu se souvint de l'alliance qu'il avait traitée avec Abraham, Isaac et Jacob.
\VS{25}Ainsi Dieu regarda les enfants d'Israël et il fit attention à leur état.
\Chap{3}
\TextTitle{Yahweh se révèle à Moïse dans le buisson ardent}
\VerseOne{}Or Moïse fut berger du troupeau de Jéthro, son beau-père, prêtre de Madian~; il mena le troupeau derrière le désert, et vint à la montagne de Dieu à Horeb.
\VS{2}Et l'Ange de Yahweh lui apparut dans une flamme de feu, du milieu d'un buisson. Il regarda, et voici, le buisson était tout en feu, et le buisson ne se consumait point.
\VS{3}Alors Moïse dit~: Je me détournerai maintenant, et je regarderai cette grande vision, pourquoi le buisson ne se consume point.
\VS{4}Et Yahweh vit que Moïse s'était détourné pour regarder~; et Dieu l'appela du milieu du buisson, en disant~: Moïse~! Moïse~! Et il répondit~: Me voici~!
\VS{5}Et Dieu dit~: N'approche point d'ici~; déchausse tes souliers de tes pieds, car le lieu où tu es arrêté est une terre sainte.
\VS{6}Il dit aussi~: Je suis le Dieu de ton père, le Dieu d'Abraham, le Dieu d'Isaac et le Dieu de Jacob\FTNT{Mt. 22:32~; Mc. 12:26~; Lu. 20:37~; Ac. 7:32.}~; Moïse cacha son visage, parce qu'il craignait de regarder vers Dieu.
\VS{7} Et Yahweh dit~: J'ai très bien vu l'affliction de mon peuple qui est en Egypte et j'ai entendu le cri qu'ils ont jeté à cause de leurs oppresseurs, car je connais leurs douleurs.
\VS{8}C'est pourquoi je suis descendu pour le délivrer de la main des Egyptiens et pour le faire remonter de ce pays-là, dans un pays bon et vaste, dans un pays découlant de lait et de miel~; au lieu où sont les Cananéens, les Héthiens, les Amoréens, les Phéréziens, les Héviens et les Jébusiens.
\VS{9}Et maintenant, voici le cri des enfants d'Israël est parvenu à moi, et j'ai vu aussi l'oppression dont les Egyptiens les oppriment.
\VS{10}Maintenant donc viens, et je t'enverrai vers Pharaon~; et tu retireras mon peuple, les enfants d'Israël, hors d'Egypte\FTNT{Os. 12:14~; Mi. 6:4.}.
\VS{11}Et Moïse répondit à Dieu~: Qui suis-je, moi, pour aller vers Pharaon, et pour retirer de l'Egypte les enfants d'Israël~?
\VS{12}Et Dieu lui dit~: Va, car je serai avec toi. Et tu auras ce signe que c'est moi qui t'envoie~: C'est que quand tu auras retiré mon peuple d'Egypte, vous servirez Dieu près de cette montagne.
\TextTitle{Yahweh révèle son Nom à Moïse}
\VS{13}Et Moïse dit à Dieu~: Voici, quand je serai venu vers les enfants d'Israël, et que je leur aurai dit~: Le Dieu de vos pères m'a envoyé vers vous, s'ils me disent alors~: Quel est son Nom~? Que leur dirai-je~?
\VS{14} Et Dieu dit à Moïse~: JE SUIS CELUI QUI SUIS. Il dit aussi~: Tu diras ainsi aux enfants d'Israël~: Celui qui s'appelle JE SUIS\FTNT{Je suis («~Ehyeh~» en hébreu), c'est de là que vient le Nom de Yahweh. Or le Nom de Jésus signifie «~Yahweh est Salut~». Dieu révèle son Nom à Moïse~: «~Je suis celui qui suis~». Or Jésus-Christ s'est ouvertement attribué ce Nom en Jn. 8:58. N'ayant compris ni le plan de Dieu ni qui était celui qui les visitait, les religieux Juifs ont voulu le lapider car ils estimaient qu'il blasphémait. Car en déclarant être «~Je suis~», Jésus-Christ proclamait ouvertement sa divinité (Ro. 9:5), chose que les Juifs ne pouvaient concevoir. Dans l'évangile de Jean, Jésus déclare clairement qu'il est le «~JE SUIS~» d'Ex. 3:14. «~Je suis le pain de vie~» (Jn. 6:35), «~Je suis la lumière du monde~» (Jn. 8:12), «~Je suis le bon berger~» (Jn. 10:11), «~Je suis la porte~» (Jn. 10:7), «~Je suis la résurrection~» (Jn. 11:25), «~Je suis le chemin, la vérité et la vie~» (Jn. 14:6), «~Je suis la vraie vigne~» (Jn. 15:1).}, m'a envoyé vers vous.
\VS{15}Dieu dit encore à Moïse~: Tu diras ainsi aux enfants d'Israël~: Yahweh, le Dieu de vos pères, le Dieu d'Abraham, le Dieu d'Isaac et le Dieu de Jacob m'a envoyé vers vous. C'est ici mon Nom éternellement, et c'est ici le souvenir que vous aurez de moi de génération en génération.
\VS{16}Va, et rassemble les anciens d'Israël, et dis leur~: Yahweh, le Dieu de vos pères, le Dieu d'Abraham, d'Isaac et de Jacob, m'est apparu, en disant~: Certainement je vous ai visités, et j'ai vu ce qu'on vous fait en Egypte.
\VS{17}Et j'ai dit~: Je vous ferai remonter de l'Egypte où vous êtes affligés, dans le pays des Cananéens, des Héthiens, des Amoréens, des Phéréziens, des Héviens et des Jébusiens, qui est un pays découlant de lait et de miel.
\VS{18}Et ils obéiront à ta parole~; et tu iras, toi et les anciens d'Israël, vers le roi d'Egypte, et vous lui direz~: Yahweh, le Dieu des Hébreux, est venu nous rencontrer. Et maintenant donc, laisse-nous aller, nous te prions, à trois jours de marche dans le désert, afin que nous puissions sacrifier à Yahweh, notre Dieu.
\VS{19}Or je sais que le roi d'Egypte ne vous permettra point de vous en aller, si ce n'est par une main forte.
\VS{20}Mais j'étendrai ma main et je frapperai l'Egypte par toutes les merveilles que je ferai au milieu d'elle~; et après cela, il vous laissera aller.
\VS{21}Je ferai que ce peuple trouve grâce envers les Egyptiens, et il arrivera que, quand vous partirez, vous ne vous en irez point à vide.
\VS{22}Mais chacune demandera à sa voisine, et à l'hôtesse de sa maison, des vases d'argent, des vases d'or, et des vêtements, que vous mettrez sur vos fils et sur vos filles~: Ainsi vous dépouillerez les Egyptiens.
\Chap{4}
\TextTitle{Moïse résiste en évoquant l'incrédulité du peuple}
\VerseOne{}Et Moïse répondit, et dit~: Mais voici, ils ne me croiront pas et n'obéiront pas à ma parole~; car ils diront~: Yahweh ne t'est point apparu.
\VS{2}Et Yahweh lui dit~: Qu'est-ce que tu as dans ta main~? Il répondit~: Une verge.
\VS{3}Et Dieu lui dit~: Jette-la par terre~; il la jeta par terre et elle devint un serpent. Et Moïse s'enfuyait de devant lui.
\VS{4}Et Yahweh dit à Moïse~: Etends ta main et saisis sa queue~; et il étendit sa main et l'empoigna~; et il redevint une verge dans sa main.
\VS{5}C'est là ce que tu feras, afin qu'ils croient que Yahweh, le Dieu de leurs pères, le Dieu d'Abraham, le Dieu d'Isaac et le Dieu de Jacob, t'est apparu.
\VS{6}Yahweh lui dit encore~: Mets maintenant ta main dans ton sein, et il mit sa main dans son sein~; puis il la tira~; et voici, sa main était blanche de lèpre comme la neige.
\VS{7}Et Dieu lui dit~: Remets ta main dans ton sein~; et il remit sa main dans son sein~; puis il la retira hors de son sein~; et voici, elle était redevenue comme son autre chair.
\VS{8}Mais s'il arrive qu'ils ne te croient point, et qu'ils n'obéissent point à la voix du premier signe, ils croiront à la voix du second signe.
\VS{9}S'il arrive qu'ils ne croient point à ces deux signes et qu'ils n'obéissent point à ta parole, tu prendras de l'eau du fleuve et tu la répandras sur la terre, et les eaux que tu auras prises du fleuve deviendront du sang sur la terre.
\TextTitle{Moïse résiste en évoquant son incapacité à parler}
\VS{10}Et Moïse répondit à Yahweh~: Hélas~! Seigneur~! Je ne suis point un homme qui ait, ni d'hier ni d'avant-hier, la parole aisée, ni même depuis que tu parles à ton serviteur~; car j'ai la bouche et la langue empêchées.
\VS{11}Et Yahweh lui dit~: Qui a fait la bouche de l'homme~? Ou qui a fait le muet, ou le sourd, ou le voyant, ou l'aveugle~? N'est-ce pas moi Yahweh\FTNT{Ps. 94:9.}~?
\VS{12}Va donc maintenant, je serai avec ta bouche et je t'enseignerai ce que tu auras à dire\FTNT{Lu. 12:12~; Mt. 10:19~; Mc. 13:11.}.
\VS{13}Et Moïse répondit~: Hélas~! Seigneur~! Envoie, je te prie, celui que tu dois envoyer.
\VS{14}Et la colère de Yahweh s'enflamma contre Moïse, et il lui dit~: Aaron, le Lévite, n'est-il pas ton frère~? Je sais qu'il parlera très bien, et même le voilà qui sort à ta rencontre, et quand il te verra, il se réjouira dans son cœur.
\VS{15}Tu lui parleras donc et tu mettras ces paroles dans sa bouche~; je serai avec ta bouche et avec la sienne, et je vous enseignerai ce que vous aurez à faire.
\VS{16}Et il parlera pour toi au peuple, et ainsi il te sera pour bouche, et tu lui seras pour Dieu.
\VS{17}Tu prendras aussi dans ta main cette verge, avec laquelle tu feras ces signes-là.
\TextTitle{Moïse accepte sa mission et part en Egypte}
\VS{18}Ainsi Moïse s'en alla, et retourna vers Jéthro, son beau-père, et lui dit~: Je te prie, que je m'en aille, et que je retourne vers mes frères qui sont en Egypte, pour voir s'ils vivent encore. Et Jéthro lui dit~: Va en paix~!
\VS{19}Or Yahweh dit à Moïse au pays de Madian~: Va, et retourne en Egypte~; car tous ceux qui cherchaient ta vie sont morts.
\VS{20}Moïse prit sa femme et ses fils, les mit sur un âne, et retourna au pays d'Egypte. Moïse prit aussi la verge de Dieu dans sa main.
\VS{21}Et Yahweh avait dit à Moïse~: Quand tu t'en iras pour retourner en Egypte, tu prendras garde à tous les miracles que j'ai mis dans ta main~; et tu les feras devant Pharaon~; mais j'endurcirai son cœur et il ne laissera point aller le peuple.
\VS{22}Tu diras donc à Pharaon, ainsi parle Yahweh~: Israël est mon fils, mon premier-né\FTNT{Os. 11:1.}.
\VS{23}Et je t'ai dit~: Laisse aller mon fils, afin qu'il me serve. Mais tu as refusé de le laisser aller~: Voici, je m'en vais tuer ton fils, ton premier-né.
\VS{24}Or il arriva que, comme Moïse était en chemin dans l'hotellerie, Yahweh le rencontra et chercha à le faire mourir.
\VS{25}Et Séphora prit un couteau tranchant, coupa le prépuce de son fils et le jeta à ses pieds, et dit~: Certes, tu es pour moi un époux de sang~!
\VS{26}Alors Yahweh se retira de lui~; et Séphora dit~: Epoux de sang~; à cause de la circoncision.
\TextTitle{Yahweh envoie Aaron vers Moïse}
\VS{27}Et Yahweh dit à Aaron~: Va dans le désert, au-devant de Moïse. Il y alla donc, et le rencontra sur la montagne de Dieu et l'embrassa.
\VS{28}Et Moïse raconta à Aaron toutes les paroles de Yahweh qui l'avait envoyé, et tous les signes qu'il lui avait ordonné de faire.
\VS{29}Moïse donc poursuivit son chemin avec Aaron~; et ils assemblèrent tous les anciens des enfants d'Israël.
\VS{30}Et Aaron rapporta toutes les paroles que Yahweh avait dites à Moïse, et il exécuta les signes aux yeux du peuple.
\VS{31}Et le peuple crut. Ils apprirent que Yahweh avait visité les enfants d'Israël, qu'il avait vu leur affliction~; et ils s'inclinèrent et se prosternèrent.
\Chap{5}
\TextTitle{Pharaon s'oppose à Moïse\FTNTT{Ex. 5-14.}}
\VerseOne{}Après cela, Moïse et Aaron se rendirent ensuite auprès de Pharaon et lui dirent~: Ainsi parle Yahweh, le Dieu d'Israël~: Laisse aller mon peuple, afin qu'il me célèbre une fête solennelle dans le désert.
\VS{2}Mais Pharaon dit~: Qui est Yahweh pour que j'obéisse à sa voix et que je laisse aller Israël~? Je ne connais point Yahweh et je ne laisserai point aller Israël.
\VS{3}Et ils dirent~: Le Dieu des Hébreux est venu au-devant de nous. Permets-nous de faire trois journées de marche dans le désert, et que nous sacrifions à Yahweh, notre Dieu~; de peur qu'il ne se jette sur nous par la peste ou par l'épée.
\VS{4}Et le roi d'Egypte leur dit~: Moïse et Aaron, pourquoi détournez-vous le peuple de son ouvrage~? Allez maintenant à vos charges.
\VS{5}Pharaon dit aussi~: Voici, le peuple de ce pays est maintenant en grand nombre, et vous lui feriez cesser leur travail~!
\VS{6}Et ce jour-là, Pharaon donna cet ordre aux oppresseurs établis sur le peuple et à ses commissaires, en disant~:
\VS{7}Vous ne donnerez plus de paille à ce peuple pour faire des briques comme auparavant, mais qu'ils aillent s'amasser de la paille.
\VS{8}Néanmoins, vous leur imposerez la quantité de briques qu'ils faisaient auparavant, sans en rien diminuer~; car ils sont paresseux, et c'est pour cela qu'ils crient, en disant~: Allons et sacrifions à notre Dieu~!
\VS{9}Que la servitude soit aggravée sur ces gens-là, et qu'ils s'occupent, et ne s'amusent plus à des paroles de mensonge.
\VS{10}Alors les oppresseurs du peuple et ses commissaires sortirent et dirent au peuple~: Ainsi parle Pharaon~: Je ne vous donnerai plus de paille.
\VS{11}Allez vous-mêmes et prenez de la paille où vous en trouverez~; mais il ne sera rien diminué de votre travail.
\VS{12}Alors le peuple se répandit par tout le pays d'Egypte, pour ramasser du chaume au lieu de paille.
\VS{13}Et les oppresseurs les pressaient en disant~: Achevez vos ouvrages, chaque jour sa tâche, comme lorsque la paille vous était fournie.
\VS{14}Même les commissaires des enfants d'Israël, que les oppresseurs de Pharaon avaient établis sur eux, furent battus, et on leur dit~: Pourquoi n'avez-vous point achevé votre tâche en faisant des briques hier et aujourd'hui, comme auparavant~?
\VS{15}Alors les commissaires des enfants d'Israël vinrent crier à Pharaon, en disant~: Pourquoi fais-tu ainsi à tes serviteurs~?
\VS{16}On ne donne point de paille à tes serviteurs, et toutefois on nous dit~: Faites des briques. Et voici, tes serviteurs sont battus, et ton peuple est traité comme coupable.
\VS{17}Et il répondit~: Vous êtes des paresseux, des paresseux~! C'est pourquoi vous dites~: Allons, sacrifions à Yahweh~!
\VS{18}Maintenant donc allez, travaillez~; car on ne vous donnera point de paille, et vous rendrez la même quantité de briques.
\VS{19}Les commissaires des enfants d'Israël virent qu'ils souffraient, puisqu'on disait~: Vous ne diminuerez rien de vos briques sur la tâche de chaque jour.
\VS{20}Et en sortant de chez Pharaon, ils rencontrèrent Moïse et Aaron, qui se trouvèrent au-devant d'eux~;
\VS{21}et ils leur dirent~: Que Yahweh vous regarde, et en juge, vu que vous nous avez mis en mauvaise odeur devant Pharaon et devant ses serviteurs, leur mettant l'épée à la main pour nous tuer.
\VS{22}Alors Moïse retourna vers Yahweh, et dit~: Seigneur~! Pourquoi as-tu fait maltraiter ce peuple~? Pourquoi m'as-tu envoyé~?
\VS{23}Car depuis que je suis allé vers Pharaon pour parler en ton Nom, il a maltraité ce peuple, et tu n'as point délivré ton peuple.
\Chap{6}
\TextTitle{Yahweh fortifie Moïse et rappelle son alliance avec Israël}
\VerseOne{}Et Yahweh dit à Moïse~: Tu verras maintenant ce que je ferai à Pharaon~; car il les laissera aller y étant contraint par une main puissante, étant, dis-je contraint par ma main puissante, il les chassera de son pays.
\VS{2}Dieu parla encore à Moïse et lui dit~: Je suis Yahweh.
\VS{3}Je suis apparu à Abraham, à Isaac et à Jacob, comme le Dieu Tout-Puissant, mais je n'ai point été connu d'eux par mon Nom YAHWEH.
\VS{4}J'ai aussi fait cette alliance avec eux, que je leur donnerai le pays de Canaan, le pays de leurs pèlerinages, dans lequel ils ont demeuré comme étrangers.
\VS{5}Et j'ai entendu les sanglots des enfants d'Israël, que les Egyptiens tiennent esclaves, et je me suis souvenu de mon alliance~;
\VS{6}c'est pourquoi dis aux enfants d'Israël~: Je suis Yahweh, et je vous retirerai de dessous les charges des Egyptiens, et je vous délivrerai de leur servitude, je vous rachèterai à bras étendu, et par de grands jugements.
\VS{7}Et je vous prendrai pour être mon peuple, je vous serai Dieu~; et vous connaîtrez que je suis Yahweh, votre Dieu, qui vous retire de dessous les charges des Egyptiens.
\VS{8}Et je vous ferai entrer dans le pays au sujet duquel j'ai levé ma main que je le donnerai à Abraham, à Isaac et à Jacob, et je vous le donnerai en héritage~; je suis Yahweh.
\VS{9}Moïse donc parla de cette manière aux enfants d'Israël. Mais ils n'écoutèrent point Moïse, à cause de l'angoisse de leur esprit, et à cause de leur dure servitude.
\VS{10}Et Yahweh parla à Moïse, en disant~:
\VS{11}Va, et dis à Pharaon, roi d'Egypte, qu'il laisse sortir les enfants d'Israël de son pays.
\VS{12}Alors Moïse parla devant Yahweh, en disant~: Voici, les enfants d'Israël ne m'ont point écouté, et comment Pharaon m'écoutera-t-il, moi, qui suis incirconcis des lèvres~?
\VS{13} Mais Yahweh parla à Moïse et à Aaron, et leur ordonna d'aller trouver les enfants d'Israël, et Pharaon, roi d'Egypte, pour retirer les fils d'Israël du pays d'Egypte.
\TextTitle{Les chefs d'Israël}
\VS{14}Voici les chefs des pères~: Les fils de Ruben, premier-né d'Israël~: Hénoc et Pallu, Hetsron et Carmi~; ce sont là les familles de Ruben\FTNT{Ge. 46:9~; No. 26:5~; 1 Ch. 5:3.}.
\VS{15}Les fils de Siméon~: Jemuel, Jamin, Ohad, Jakin et Tsochar, et Saül, fils d'une Cananéenne~; ce sont là les familles de Siméon.
\VS{16}Voici les noms des fils de Lévi selon leur naissance~: Guerschon, Kehath et Merari. Les années de la vie de Lévi furent de cent trente-sept ans.
\VS{17}Les fils de Guerschon~: Libni et Schimeï, selon leurs familles.
\VS{18}Les fils de Kehath~: Amram, Jitsehar, Hébron et Uziel. Et les années de la vie de Kehath furent de cent trente-trois ans.
\VS{19}Les fils de Merari~: Machli et Muschi~; ce sont là les familles de Lévi selon leurs générations.
\VS{20}Or Amram prit Jokébed, sa tante, pour femme, qui lui enfanta Aaron et Moïse~; les années de la vie d'Amram furent de cent trente-sept ans.
\VS{21}Et les fils de Jitsehar~: Koré, Népheg et Zicri.
\VS{22}Et les fils d'Uziel~: Mischaël, Eltsaphan, et Sithri.
\VS{23}Aaron prit pour femme Elischéba, fille d'Amminadab, sœur de Nachschon, qui lui enfanta Nadab, Abihu, Eléazar et Ithamar.
\VS{24}Et les fils de Koré~: Assir, Elkana, et Abiasaph. Ce sont là les familles des Korites.
\VS{25}Eléazar, fils d'Aaron, prit pour femme une des filles de Puthiel, qui lui enfanta Phinées. Ce sont là les chefs des pères des Lévites selon leurs familles.
\VS{26}Or c'est là cet Aaron et ce Moïse à qui Yahweh dit~: Retirez les enfants d'Israël du pays d'Egypte selon leurs armées.
\VS{27}Ce sont eux qui parlèrent à Pharaon, roi d'Egypte, pour retirer d'Egypte les enfants d'Israël. C'est ce Moïse et c'est cet Aaron.
\VS{28}Le jour où Yahweh parla à Moïse dans le pays d'Egypte,
\VS{29}Yahweh parla à Moïse et dit~: Je suis Yahweh~; dis à Pharaon, roi d'Egypte, toutes les paroles que je t'ai dites.
\VS{30}Et Moïse dit en présence de Yahweh~: Voici, je suis incirconcis des lèvres, comment Pharaon m'écoutera-t-il~?
\Chap{7}
\TextTitle{L'appel de Moïse confirmé}
\VerseOne{}Et Yahweh dit à Moïse~: Voici, je t'ai établi pour être Dieu à Pharaon, et Aaron, ton frère, sera ton prophète.
\VS{2}Tu diras tout ce que je t'ordonnerai, et Aaron, ton frère, parlera à Pharaon pour qu'il laisse aller les enfants d'Israël hors de son pays.
\VS{3}J'endurcirai le cœur de Pharaon, et je multiplierai mes signes et mes miracles dans le pays d'Egypte.
\VS{4}Pharaon ne vous écoutera point~; je mettrai ma main sur l'Egypte, et je sortirai mes armées, mon peuple, les enfants d'Israël, du pays d'Egypte, par de grands jugements.
\VS{5}Les Egyptiens connaîtront\FTNT{Les nations reconnaîtront que Jésus-Christ est le Dieu d'Israël lorsqu'il reviendra en Sion pour délivrer et restaurer son peuple (Za. 14).} que je suis Yahweh quand j'aurai étendu ma main sur l'Egypte, et que j'aurai retiré du milieu d'eux les enfants d'Israël.
\VS{6}Et Moïse et Aaron firent comme Yahweh leur avait ordonné~; ils firent ainsi.
\VS{7}Or Moïse était âgé de quatre-vingts ans, et Aaron de quatre-vingt-trois ans quand ils parlèrent à Pharaon.
\TextTitle{La verge d'Aaron devient un serpent}
\VS{8}Yahweh parla à Moïse et à Aaron, en disant~:
\VS{9}Quand Pharaon vous parlera, en disant~: Faites un miracle~; tu diras alors à Aaron~: Prends ta verge, jette-la devant Pharaon et elle deviendra un serpent.
\VS{10}Moïse donc et Aaron allèrent auprès de Pharaon, et firent comme Yahweh avait ordonné~; Aaron jeta sa verge devant Pharaon et devant ses serviteurs, et elle devint un serpent.
\VS{11}Mais Pharaon fit venir aussi les sages et les enchanteurs~; et les magiciens d'Egypte, et eux aussi firent autant par leurs enchantements.
\VS{12}Ils jetèrent donc chacun leurs verges et elles devinrent des serpents~; mais la verge d'Aaron engloutit leurs verges.
\VS{13}Le cœur de Pharaon s'endurcit et il ne les écouta point~; selon ce que Yahweh avait dit.
\TextTitle{Les eaux du fleuve changées en sang}
\VS{14}Yahweh dit à Moïse~: Le cœur de Pharaon est endurci, il a refusé de laisser aller le peuple.
\VS{15}Va-t'en dès le matin vers Pharaon~; voici, il sortira pour aller près de l'eau~; tu te présenteras donc devant lui sur le bord du fleuve, et tu prendras dans ta main la verge qui a été changée en serpent.
\VS{16}Et tu lui diras~: Yahweh, le Dieu des Hébreux, m'avait envoyé vers toi pour te dire~: Laisse aller mon peuple, afin qu'il me serve au désert~; mais voici, tu ne m'as point écouté jusqu'ici.
\VS{17}Ainsi parle Yahweh~: A ceci tu sauras que je suis Yahweh~; je m'en vais frapper de la verge qui est dans ma main les eaux du fleuve, et elles seront changées en sang.
\VS{18}Et le poisson qui est dans le fleuve mourra, le fleuve deviendra puant, et les Egyptiens éprouveront du dégoût à boire des eaux du fleuve.
\VS{19}Yahweh parla aussi à Moïse~: Dis à Aaron~: Prends ta verge et étends ta main sur les eaux des Egyptiens, sur leurs rivières, sur leurs ruisseaux, et sur leurs marais, et sur tous les amas de leurs eaux, et elles deviendront du sang~; il y aura du sang par tout le pays d'Egypte, dans les vases de bois et de pierre.
\VS{20}Moïse donc et Aaron firent ce que Yahweh avait ordonné. Aaron, ayant levé la verge, en frappa les eaux du fleuve, sous les yeux de Pharaon et de ses serviteurs~; et toutes les eaux du fleuve furent changées en sang.
\VS{21}Et le poisson qui était dans le fleuve mourut, et le fleuve en devint puant, tellement que les Egyptiens ne pouvaient point boire les eaux du fleuve~; il y eut du sang dans tout le pays d'Egypte.
\VS{22}Et les magiciens d'Egypte en firent de même par leurs enchantements. Et le cœur de Pharaon s'endurcit tellement, qu'il ne les écouta point, selon ce que Yahweh avait dit.
\VS{23}Et Pharaon leur ayant tourné le dos, alla dans sa maison, et ne prit même pas à cœur ces choses qu'il avait vues.
\VS{24}Or tous les Egyptiens creusèrent autour du fleuve pour trouver de l'eau à boire, parce qu'ils ne pouvaient pas boire de l'eau du fleuve.
\VS{25}Il se passa sept jours depuis que Yahweh eut frappé le fleuve.
\TextTitle{Invasion de grenouilles}
\VS{26}Après cela, Yahweh dit à Moïse~: Va vers Pharaon et dis-lui~: Ainsi parle Yahweh~: Laisse aller mon peuple, afin qu'il me serve.
\VS{27}Si tu refuses de le laisser aller, voici, je m'en vais frapper de grenouilles toutes tes contrées~;
\VS{28}et le fleuve fourmillera de grenouilles, qui monteront et entreront dans ta maison, et dans la chambre où tu couches, et sur ton lit, et dans les maisons de tes serviteurs, et parmi tout ton peuple, dans tes fours et dans tes maies.
\VS{29}Ainsi, les grenouilles monteront sur toi, sur ton peuple et sur tous tes serviteurs.
\Chap{8}
\VerseOne{}Yahweh donc dit à Moïse~: Dis à Aaron~: Etends ta main avec ta verge sur les fleuves, sur les rivières, et sur les marais, et fais monter les grenouilles sur le pays d'Egypte.
\VS{2}Et Aaron étendit sa main sur les eaux de l'Egypte, et les grenouilles montèrent et couvrirent le pays d'Egypte.
\VS{3}Mais les magiciens firent de même par leurs enchantements et firent monter des grenouilles sur le pays d'Egypte.
\VS{4}Alors Pharaon appela Moïse et Aaron, et leur dit~: Fléchissez Yahweh par vos prières, afin qu'il retire les grenouilles de dessus moi et de dessus mon peuple~; et je laisserai aller le peuple, afin qu'ils sacrifient à Yahweh.
\VS{5}Et Moïse dit à Pharaon~: Glorifie-toi sur moi~! Pour quel temps fléchirai-je par mes prières Yahweh pour toi, et pour tes serviteurs et pour ton peuple, afin que Yahweh retire les grenouilles loin de toi et de tes maisons~? Il en demeurera seulement dans le fleuve.
\VS{6}Alors il répondit~: Pour demain. Et Moïse dit~: Il sera fait selon ta parole, afin que tu saches qu'il n'y a nul Dieu tel que Yahweh, notre Dieu.
\VS{7}Les grenouilles donc se retireront de toi, de tes maisons, de tes serviteurs et de ton peuple~; il en demeurera seulement dans le fleuve.
\VS{8}Alors Moïse et Aaron sortirent de chez Pharaon~; Moïse cria à Yahweh au sujet des grenouilles qu'il avait fait venir sur Pharaon.
\VS{9}Et Yahweh fit selon la parole de Moïse. Ainsi les grenouilles moururent dans les maisons, dans les villages et dans les champs.
\VS{10}On les amassa par monceaux, et la terre en fut infectée.
\VS{11}Mais Pharaon, voyant qu'il y avait du relâche, endurcit son cœur et ne les écouta point, selon ce que Yahweh avait dit.
\TextTitle{Invasion de poux}
\VS{12}Et Yahweh dit à Moïse~: Dis à Aaron~: Etends ta verge et frappe la poussière de la terre, et elle deviendra des poux dans tout le pays d'Egypte.
\VS{13}Et ils firent ainsi~; et Aaron étendit sa main avec sa verge, et frappa la poussière de la terre~; et elle fut changée en poux, sur les hommes et sur les bêtes~; toute la poussière du pays fut changée en poux dans tout le pays d'Egypte.
\VS{14}Et les magiciens voulurent faire de même par leurs enchantements, pour produire des poux, mais ils ne purent pas. Les poux furent donc tant sur les hommes que sur les bêtes.
\VS{15}Alors les magiciens dirent à Pharaon~: C'est ici le doigt de Dieu\FTNT{Lu. 11:20.}~! Toutefois, le cœur de Pharaon s'endurcit et il ne les écouta point, selon ce que Yahweh avait dit.
\TextTitle{Invasion de mouches}
\VS{16}Puis Yahweh dit à Moïse~: Lève-toi de bon matin, et présente-toi devant Pharaon~; voici, il sortira près de l'eau, et tu lui diras~: Ainsi parle Yahweh~: Laisse aller mon peuple, afin qu'il me serve.
\VS{17}Car si tu ne laisses pas aller mon peuple, voici, je m'en vais envoyer contre toi, contre tes serviteurs, contre ton peuple et contre tes maisons, un mélange d'insectes~; et les maisons des Egyptiens seront remplies de ce mélange, et la terre aussi sur laquelle ils seront \FTNT{Ps. 105:31~; Ps. 78:43.}.
\VS{18}Mais je distinguerai ce jour-là le pays de Gosen, où se tient mon peuple, tellement qu'il n'y aura nul mélange d'insectes~; afin que tu saches que je suis Yahweh au milieu de la terre.
\VS{19}Et je ferai la différence entre ton peuple et mon peuple~; demain, ce signe-là se fera.
\VS{20}Et Yahweh le fit ainsi~; et un grand mélange d'insectes entra dans la maison de Pharaon et dans chaque maison de ses serviteurs, et dans tout le pays d'Egypte, de sorte que la terre fut gâtée par ce mélange.
\TextTitle{Pharaon tente de compromettre Moïse}
\VS{21} Et Pharaon appela Moïse et Aaron, et leur dit~: Allez, sacrifiez à votre Dieu dans ce pays.
\VS{22}Mais Moïse dit~: Il n'est pas convenable de faire ainsi~; car nous sacrifierions à Yahweh, notre Dieu, l'abomination des Egyptiens. Voici, si nous sacrifions l'abomination des Egyptiens devant leurs yeux, ne nous lapideraient-ils pas~?
\VS{23}Nous irons le chemin de trois jours au désert, et nous sacrifierons à Yahweh, notre Dieu, comme il nous dira.
\VS{24}Alors Pharaon dit~: Je vous laisserai aller pour sacrifier dans le désert à Yahweh, votre Dieu~; toutefois, vous ne vous éloignerez pas en y allant. Fléchissez Yahweh pour moi, par vos prières.
\VS{25}Moïse dit~: Voici, je sors de chez toi et je supplierai Yahweh, afin que le mélange d'insectes se retire demain de Pharaon, de ses serviteurs, et de son peuple. Mais que Pharaon ne continue point à se moquer en ne laissant point aller le peuple pour sacrifier à Yahweh.
\VS{26}Alors Moïse sortit de chez Pharaon et fléchit Yahweh par la prière.
\VS{27}Et Yahweh fit selon la parole de Moïse~; et le mélange d'insectes se retira de Pharaon, de ses serviteurs et de son peuple~; il n'en resta pas un seul insecte.
\VS{28}Mais Pharaon endurcit son cœur cette fois encore et ne laissa point aller le peuple.
\Chap{9}
\TextTitle{La mort des troupeaux}
\VerseOne{}Alors Yahweh dit à Moïse~: Va vers Pharaon et dis-lui~: Ainsi parle Yahweh, le Dieu des Hébreux~: Laisse aller mon peuple, afin qu'il me serve.
\VS{2}Car si tu refuses de les laisser aller et si tu le retiens encore,
\VS{3}voici, la main de Yahweh sera sur ton bétail qui est dans les champs, tant sur les chevaux que sur les ânes, sur les chameaux, sur les bœufs, et sur les brebis, et il y aura une très grande mortalité.
\VS{4}Et Yahweh distinguera le bétail des Israélites du bétail des Egyptiens, afin que rien de ce qui est aux enfants d'Israël ne meure.
\VS{5}Et Yahweh fixa un temps, en disant~: Demain, Yahweh fera ceci dans le pays.
\VS{6}Yahweh donc fit cela dès le lendemain~; et tout le bétail des Egyptiens mourut~; mais du bétail des enfants d'Israël, il ne mourut pas une seule bête.
\VS{7}Et Pharaon envoya examiner, et voici, il n'y avait pas une seule bête morte du bétail des enfants d'Israël. Toutefois, le cœur de Pharaon s'endurcit, et il ne laissa point aller le peuple.
\TextTitle{Des ulcères sur les Egyptiens et les bêtes}
\VS{8}Alors Yahweh dit à Moïse et à Aaron~: Remplissez vos mains de cendre de fournaise~; et que Moïse les répande vers les cieux en la présence de Pharaon.
\VS{9}Et elles deviendront de la poussière sur tout le pays d'Egypte, et il s'en fera des ulcères bourgeonnant en pustules tant sur les hommes que sur les bêtes, dans tout le pays d'Egypte.
\VS{10}Ils prirent donc de la cendre de fournaise et se tinrent devant Pharaon~; Moïse la répandit vers les cieux et il se forma des ulcères bourgeonnant en pustules tant sur les hommes que sur les bêtes.
\VS{11} Et les magiciens ne purent se tenir devant Moïse, à cause des ulcères~; car les magiciens avaient des ulcères, comme tous les Egyptiens.
\VS{12} Et Yahweh endurcit le cœur de Pharaon, et il ne les écouta point selon ce que Yahweh avait dit à Moïse.
\TextTitle{L'Egypte frappée par la grêle et le feu}
\VS{13}Puis Yahweh parla à Moïse~: Lève-toi de bon matin, et présente-toi devant Pharaon, et dis-lui~: Ainsi parle Yahweh, le Dieu des Hébreux~: Laisse aller mon peuple, afin qu'il me serve.
\VS{14}Car cette fois, je vais faire venir toutes mes plaies contre ton cœur, sur tes serviteurs et sur ton peuple, afin que tu saches qu'il n'y a nul Dieu semblable à moi sur toute la terre.
\VS{15}Car maintenant si j'avais étendu ma main, je t'aurais frappé de la peste, toi et ton peuple, et tu serais effacé de la terre.
\VS{16}Mais certainement, je t'ai fait subsister pour te faire voir ma puissance, afin que mon Nom soit célébré sur toute la terre\FTNT{Ro. 9:17.}.
\VS{17}T'élèves-tu encore contre mon peuple, pour ne point le laisser aller~?
\VS{18}Voici, je m'en vais faire pleuvoir demain à cette même heure, une grêle tellement forte qu'il n'y en a point eu de semblable en Egypte, depuis le jour où elle fut fondée jusqu'à maintenant.
\VS{19}Maintenant envoie rassembler ton bétail et tout ce que tu as à la campagne~; car la grêle tombera sur tous les hommes, sur le bétail qui se trouvera à la campagne, et qu'on n'aura pas renfermé, et ils mourront.
\VS{20}Celui d'entre les serviteurs de Pharaon, qui craignit la parole de Yahweh, fit promptement retirer dans les maisons ses serviteurs et ses bêtes.
\VS{21}Mais celui qui n'appliqua point son cœur à la parole de Yahweh, laissa ses serviteurs et ses bêtes à la campagne.
\VS{22}Et Yahweh dit à Moïse~: Etends ta main vers les cieux, et il y aura de la grêle sur tout le pays d'Egypte, sur les hommes et sur les bêtes, et sur toutes les herbes des champs au pays d'Egypte.
\VS{23}Moïse donc étendit sa verge vers les cieux, et Yahweh envoya des tonnerres et de la grêle, et le feu se promenait sur la terre. Yahweh fit pleuvoir de la grêle sur le pays d'Egypte.
\VS{24}Il y eut donc de la grêle et du feu entremêlé avec la grêle, laquelle était si grosse qu'il n'y en avait point eu de semblable sur toute la terre d'Egypte, depuis qu'elle a été habitée.
\VS{25}La grêle frappa dans tout le pays d'Egypte tout ce qui était aux champs, depuis les hommes jusqu'aux bêtes. La grêle frappa aussi toutes les herbes des champs et brisa tous les arbres des champs.
\VS{26}Il n'y eut que la contrée de Gosen, dans laquelle étaient les enfants d'Israël, où il n'y eut point de grêle.
\TextTitle{Pharaon continue d'endurcir son cœur}
\VS{27}Alors Pharaon envoya appeler Moïse et Aaron, et leur dit~: J'ai péché cette fois~; Yahweh est juste, mais moi et mon peuple sommes méchants.
\VS{28}Fléchissez par des prières Yahweh~: Que ce soit assez, et que Dieu ne fasse plus tonner ni grêler, car je vous laisserai aller, et on ne vous arrêtera plus.
\VS{29}Alors Moïse dit~: Aussitôt que je sortirai de la ville, j'étendrai mes mains vers Yahweh et les tonnerres cesseront. Il n'y aura plus de grêle, afin que tu saches que la terre est à Yahweh\FTNT{Ps. 24:1.}.
\VS{30}Mais quant à toi et tes serviteurs, je sais que vous ne craindrez pas encore Yahweh Dieu.
\VS{31}Or le lin et l'orge avaient été frappés, car l'orge était en épis et c'était la floraison du lin.
\VS{32} Mais le blé et l'épeautre ne furent point frappés, parce qu'ils sont tardifs.
\VS{33}Moïse donc sortit de chez Pharaon pour aller hors de la ville. Il étendit ses mains vers Yahweh, et les tonnerres cessèrent, et la grêle et la pluie ne tombèrent plus sur la terre.
\VS{34}Pharaon, voyant que la pluie, la grêle, et les tonnerres avaient cessé, continua encore à pécher, et il endurcit son cœur, lui et ses serviteurs.
\VS{35}Le cœur donc de Pharaon s'endurcit et il ne laissa point aller les enfants d'Israël, selon ce que Yahweh avait dit par l'intermédiaire de Moïse.
\Chap{10}
\TextTitle{Invasion de sauterelles}
\VerseOne{} Et Yahweh dit à Moïse~: Va vers Pharaon, car j'ai endurci son cœur et le cœur de ses serviteurs, afin que je mette au-dedans de lui les signes que je m'en vais faire~;
\VS{2}et afin que tu racontes à ton fils et au fils de ton fils, les signes que j'accomplirai sur les Egyptiens et les prodiges que je ferai au milieu d'eux, et que vous sachiez que je suis Yahweh.
\VS{3}Moïse donc et Aaron vinrent vers Pharaon, et lui dirent~: Ainsi parle Yahweh, le Dieu des Hébreux~: Jusqu'à quand refuseras-tu de t'humilier devant moi~? Laisse aller mon peuple, afin qu'il me serve.
\VS{4}Car si tu refuses de laisser aller mon peuple, voici, je ferai venir demain des sauterelles dans ton territoire.
\VS{5}Elles couvriront la face de la terre, et l'on ne pourra plus voir la terre~; elles dévoreront le reste de ce qui a échappé, ce que la grêle vous a laissé~; et elles dévoreront tous les arbres qui poussent dans vos champs.
\VS{6}Et elles rempliront tes maisons, et les maisons de tous tes serviteurs, et les maisons de tous les Egyptiens~; ce que tes pères n'ont point vu ni les pères de tes pères, depuis qu'ils existent sur la terre jusqu'à ce jour. Puis, ayant tourné le dos à Pharaon, il sortit d'auprès de lui.
\VS{7}Et les serviteurs de Pharaon lui dirent~: Jusqu'à quand celui-ci nous sera-t-il un piège~? Laisse aller ces gens et qu'ils servent Yahweh, leur Dieu. Attendras-tu de savoir avant cela que l'Egypte est perdue~?
\VS{8}Alors on fit revenir Moïse et Aaron vers Pharaon, il leur dit~: Allez, servez Yahweh, votre Dieu. Qui sont tous ceux qui iront~?
\VS{9} Et Moïse répondit~: Nous irons avec nos jeunes gens et nos vieillards, avec nos fils et nos filles~; nous irons avec nos brebis et nos bœufs~; car nous avons à célébrer une fête solennelle à Yahweh.
\VS{10}Alors il leur dit~: Que Yahweh soit avec vous, comme je laisserai aller vos petits enfants~! Prenez garde, car le mal est devant vous.
\VS{11}Il n'en sera pas ainsi que vous l'avez demandé~; mais vous, hommes, allez maitenant et servez Yahweh~; car c'est ce que vous demandiez. Et on les chassa de la présence de Pharaon.
\VS{12}Alors Yahweh dit à Moïse~: Etends ta main sur le pays d'Egypte, pour faire venir les sauterelles, afin qu'elles montent sur le pays d'Egypte, qu'elles dévorent toute l'herbe de la terre, tout ce que la grêle a laissé.
\VS{13}Moïse étendit donc sa verge sur le pays d'Egypte~; et Yahweh amena sur le pays, tout ce jour-là et toute la nuit, un vent d'orient~; le matin vint, et le vent d'orient enleva les sauterelles.
\VS{14}Et il fit monter les sauterelles sur tout le pays d'Egypte, et les mit dans toutes les contrées d'Egypte~; elles étaient fort grosses et il y n'en avait point eu avant elles de semblables, et il n'y en aura point de semblables après elles.
\VS{15}Et elles couvrirent la face de tout le pays, tellement que le pays en fut obscurci~; elles dévorèrent toute l'herbe de la terre, tout le fruit des arbres que la grêle avait laissé~; il ne resta aucune verdure aux arbres ni aux herbes des champs, dans tout le pays d'Egypte.
\VS{16}Aussitôt Pharaon se hâta d'appeler Moïse et Aaron, et dit~: J'ai péché contre Yahweh, votre Dieu, et contre vous.
\VS{17}Mais pardonne, je te prie, mon péché, pour cette fois seulement~; et suppliez Yahweh, votre Dieu, par vos prières, afin qu'il retire de moi cette mort-ci seulement.
\VS{18}Il sortit donc de chez Pharaon, et fléchit Yahweh par ses prières.
\VS{19}Et Yahweh fit lever un vent d'occident très fort qui enleva les sauterelles et les précipita dans la Mer Rouge. Il ne resta pas une seule sauterelle dans tout le territoire de l'Egypte.
\VS{20}Mais Yahweh endurcit le cœur de Pharaon et il ne laissa point aller les enfants d'Israël.
\TextTitle{Les ténèbres sur les Egyptiens}
\VS{21}Puis Yahweh dit à Moïse~: Etends ta main vers les cieux, qu'il y ait sur le pays d'Egypte des ténèbres si épaisses, qu'on puisse les toucher à la main.
\VS{22}Moïse étendit donc sa main vers les cieux, et il y eut d'épaisses ténèbres dans tout le pays d'Egypte, pendant trois jours\FTNT{Ps. 105:28.}.
\VS{23}On ne se voyait pas l'un l'autre, et nul ne se leva de sa place pendant trois jours. Mais pour tous les enfants d'Israël, il y eut de la lumière dans le lieu de leurs demeures.
\TextTitle{Pharaon tente encore de compromettre Moïse}
\VS{24}Alors Pharaon appela Moïse et dit~: Allez, servez Yahweh~; que vos brebis et vos bœufs seuls demeurent~; vos petits enfants iront aussi avec vous.
\VS{25}Moïse répondit~: Tu mettras toi-même entre nos mains de quoi faire des sacrifices et des holocaustes, que nous ferons à Yahweh, notre Dieu.
\VS{26}Et même, nos troupeaux viendront aussi avec nous, il n'en restera pas un sabot. Car nous en prendrons pour servir Yahweh, notre Dieu~; car nous ne savons pas ce que nous choisirons pour offrir à Yahweh, jusqu'à ce que nous soyons arrivés en ce lieu là.
\VS{27}Mais Yahweh endurcit le cœur de Pharaon et il ne voulut point les laisser aller.
\VS{28}Et Pharaon lui dit~:Va-t-en~!Arrière de moi~! Garde-toi de revoir ma face, car le jour où tu verras ma face, tu mourras.
\VS{29}Alors Moïse répondit~: Tu as bien dit, je ne reverrai plus ta face\FTNT{Hé. 11:27.}.
\Chap{11}
\TextTitle{Pharaon méprise l'avertissement sur la mort des premiers-nés}
\VerseOne{}Or Yahweh dit à Moïse~: Je ferai venir encore une plaie sur Pharaon, et sur l'Egypte, et après cela il vous laissera aller d'ici~; il vous laissera entièrement aller, et vous chassera tout à fait d'ici.
\VS{2}Parle maintenant aux oreilles du peuple, et dis leur~: Que chacun demande à son voisin, et chacune à sa voisine, des vases d'argent et des vases d'or.
\VS{3}Or Yahweh fit trouver grâce au peuple devant les Egyptiens~; et même Moïse passait pour un grand homme dans le pays d'Egypte, tant parmi les serviteurs de Pharaon que parmi le peuple.
\VS{4}Et Moïse dit~: Ainsi parle Yahweh~: Vers le milieu de la nuit, je passerai au travers de l'Egypte~;
\VS{5}et tout premier-né mourra dans le pays d'Egypte, depuis le premier-né de Pharaon, qui devait être assis sur son trône, jusqu'au premier-né de la servante qui est derrière la meule, et jusqu'à tous les premiers-nés des bêtes.
\VS{6}Et il y aura un grand cri dans tout le pays d'Egypte, tel qu'il n'y en a jamais eu et qu'il n'y en aura jamais de semblable.
\VS{7}Mais contre tous les enfants d'Israël, un chien même ne remuera point sa langue, depuis l'homme jusqu'aux bêtes~; afin que vous sachiez que Dieu fera la différence entre les Egyptiens et les Israélites.
\VS{8}Et tous tes serviteurs viendront vers moi, et se prosterneront devant moi, en disant~: Sors, toi, et tout le peuple qui est avec toi. Après cela, je sortirai. Ainsi, Moïse sortit de chez Pharaon dans une ardente colère.
\VS{9}Yahweh donc dit à Moïse~: Pharaon ne vous écoutera point, afin que mes miracles soient multipliés dans le pays d'Egypte.
\VS{10}Et Moïse et Aaron firent tous ces miracles-là devant Pharaon. Et Yahweh endurcit le cœur de Pharaon, tellement qu'il ne laissa point aller les enfants d'Israël hors de son pays.
\Chap{12}
\TextTitle{La première Pâque}
\VerseOne{}Or Yahweh dit à Moïse et à Aaron dans le pays d'Egypte~:
\VS{2}Ce mois-ci sera pour vous le premier des mois, il sera pour vous le premier des mois de l'année.
\VS{3}Parlez à toute l'assemblée d'Israël, en disant~: Jusqu'au dixième jour de ce mois, que chacun prenne un petit d'entre les brebis ou d'entre les chèvres, selon les familles des pères~; un petit, dis-je, d'entre les brebis ou d'entre les chèvres, par famille.
\VS{4}Mais si la famille est moindre qu'il ne faut pour manger un petit d'entre les brebis ou d'entre les chèvres, qu'elle le prenne avec son voisin qui est près de sa maison, selon le nombre de personnes~; vous compterez combien il en faudra pour manger d'entre les brebis ou d'entre les chèvres, ayant égard à ce que chacun de vous peut manger.
\VS{5}Or le petit d'entre les brebis ou d'entre les chèvres sera sans défaut, et sera un mâle ayant un an\FTNT{La Pâque juive était célébrée le 14ème jour du premier mois de l'année juive soit, le 14 du mois de Nissan (Ex. 12:2~; No. 9:1-5). L'agneau pascal était une préfiguration de Jésus-Christ~: L'agneau de Dieu qui ôte le péché du monde (Jn. 1:29). Ses caractéristiques sont les suivantes~: \\- L'agneau devait nécessairement être un mâle sans défaut (Ex. 12:5). Jésus est l'enfant mâle mis au monde par une vierge, il n'a pas été affecté par le sang corrompu d'Adam, il est donc sans défaut (Es. 7:14~; Mt. 1:20-21). Pour être certains de la perfection de l'animal, les hébreux devaient l'examiner pendant quatre jours avant de l'immoler (Ex. 12:3-6). Il est à noter que la loi juive exigeait que deux ou trois témoins soient présents pour constater un crime ou un péché (De. 17:6~; De. 19:15), ces quatre jours font donc office de quatre témoins pour attester de la pureté de l'animal. De même, les quatre auteurs de l'évangile attestent la sainteté du Seigneur. De plus, avant sa mise à mort, le Seigneur a été examiné par deux législations~: juive (le sanhédrin) et romaine (Ponce Pilate). Ces deux législations attestèrent, malgré elles, son innocence (Mt. 25:60~; Mt. 27:24~; Mc. 14:55-56~; Mc. 15:14~; Lu. 23:4~; Jn. 18:31~; Jn. 19:6) et confirmèrent qu'il était sans défaut et donc digne d'être offert en sacrifice.\\- Yahweh avait prescrit aux hébreux d'immoler l'agneau entre les deux soirs (Ex. 12:6), c'est-à-dire avant le crépuscule, entre la neuvième et la onzième heure. Jésus fut arrêté la nuit de Pâque (Mc. 14:12-41). Sa crucifixion eut lieu le lendemain, à la troisième heure (Mc. 15:25), et sa mort survint à la neuvième heure (Mt. 27:45). L'agneau devait être rôti au feu puis consommé avec du pain sans levain et des herbes amères (Ex. 12:8). Le feu symbolise le jugement que le Seigneur a pris sur lui à cause de nos péchés (Es. 53:5~; Ro. 4:25~; 1 Pi. 1:18-20). Le pain sans levain est une autre image de Jésus, le pain de vie (Jn. 6:35) sans aucun péché (1 Co. 5:8). Les herbes amères préfigurent, quant à elles, l'affliction et la souffrance du Seigneur (Hé. 2:10).}~; vous le prendrez d'entre les brebis ou d'entre les chèvres~;
\VS{6}et vous le garderez jusqu'au quatorzième jour de ce mois~; et toute la congrégation de l'assemblée d'Israël l'égorgera entre les deux soirs.
\VS{7} Et ils prendront de son sang, et le mettront sur les deux poteaux et sur le linteau de la porte des maisons où ils le mangeront.
\VS{8} Et ils en mangeront la chair rôtie au feu cette nuit-là~; et ils la mangeront avec des pains sans levain, et avec des herbes amères.
\VS{9}N'en mangez rien à demi cuit, ni qui ait été bouilli dans l'eau~; mais qu'il soit rôti au feu, sa tête, ses jambes et ses entrailles.
\VS{10} Et ne laissez aucun reste jusqu'au matin, mais s'il en reste quelque chose le matin, vous le brûlerez au feu.
\VS{11}Et vous le mangerez ainsi~: Vos reins seront ceints, vous aurez vos souliers à vos pieds, et votre bâton à la main, et vous le mangerez à la hâte. C'est la Pâque de Yahweh.
\TextTitle{Le sang qui sauve~; l'instauration de la fête de la Pâque}
\VS{12}Car je passerai cette nuit-là par le pays d'Egypte, et je frapperai tout premier-né au pays d'Egypte, depuis les hommes jusqu'aux bêtes~; et j'exercerai des jugements sur tous les dieux de l'Egypte. Je suis Yahweh.
\VS{13}Et le sang sera pour vous un signe sur les maisons où vous serez~; car je verrai le sang et je passerai par-dessus vous, et il n'y aura point de plaie à destruction quand je frapperai le pays d'Egypte.
\VS{14}Et ce jour là, vous conserverez le souvenir de ce jour, et vous le célébrerez comme une fête solennelle à Yahweh~; vous le célébrerez comme une fête solennelle par une ordonnance perpétuelle de génération en génération.
\VS{15}Vous mangerez pendant sept jours des pains sans levain, et dès le premier jour, vous ôterez le levain de vos maisons~; car quiconque mangera du pain levé, depuis le premier jour jusqu'au septième, cette personne-là sera retranchée d'Israël.
\VS{16}Au premier jour il y aura une sainte convocation, et il y aura de même au septième jour une sainte convocation~; il ne se fera aucune œuvre dans ces jours-là~; seulement, on vous apprêtera à manger ce qu'il faudra pour chaque personne.
\VS{17}Vous prendrez donc garde aux pains sans levain, parce qu'en ce même jour, j'aurai retiré vos armées du pays d'Egypte~; vous observerez donc ce jour-là de génération en génération par une ordonance perpétuelle.
\VS{18}Au premier mois, le quatorzième jour du mois, au soir, vous mangerez des pains sans levain jusqu'au vingt et unième jour du mois, au soir.
\VS{19}Il ne se trouvera point de levain dans vos maisons pendant sept jours, car quiconque mangera du pain levé, cette personne-là sera retranchée de l'assemblée d'Israël, tant celui qui habite comme étranger que celui qui est né au pays.
\VS{20}Vous ne mangerez point de pain levé~; mais vous mangerez dans tous les lieux où vous demeurerez des pains sans levain.
\VS{21}Moïse donc appela tous les anciens d'Israël et leur dit~: Choisissez et prenez un petit d'entre les brebis ou d'entre les chèvres selon vos familles, et égorgez la Pâque.
\VS{22}Puis vous prendrez un bouquet d'hysope et le tremperez dans le sang qui sera dans un bassin, et vous arroserez du sang qui sera dans le bassin, le linteau et les deux poteaux~; et nul de vous ne sortira de la porte de sa maison jusqu'au matin.
\VS{23}Car Yahweh passera pour frapper l'Egypte et il verra le sang sur le linteau et sur les deux poteaux, et Yahweh passera par-dessus la porte, et ne permettra point que le destructeur entre dans vos maisons pour frapper.
\VS{24}Vous garderez ceci comme une ordonnance perpétuelle pour toi et pour tes fils.
\VS{25}Quand donc vous serez entrés dans le pays que Yahweh vous donnera, selon qu'il en a parlé, vous observerez ce service.
\VS{26}Et quand vos fils vous diront~: Que signifie pour vous ce service~?
\VS{27}Alors vous répondrez~: C'est le sacrifice de la Pâque à Yahweh, qui passa en Egypte par-dessus les maisons des enfants d'Israël, quand il frappa l'Egypte, et qu'il préserva nos maisons. Alors le peuple s'inclina et se prosterna.
\VS{28}Ainsi les enfants d'Israël s'en allèrent et firent comme Yahweh l'ordonna à Moïse et à Aaron, ils le firent ainsi.
\TextTitle{Les premiers-nés d'Egypte frappés}
\VS{29}Et il arriva qu'à minuit Yahweh frappa tous les premiers-nés du pays d'Egypte, depuis le premier-né de Pharaon, qui devait être assis sur son trône, jusqu'aux premiers-nés des captifs qui étaient dans la prison, et tous les premiers-nés des bêtes.
\VS{30}Et Pharaon se leva de nuit, lui et ses serviteurs, et tous les Egyptiens~; et il y eut un grand cri en Egypte, parce qu'il n'y avait point de maison où il n'y ait eu un mort\FTNT{Hé. 11:28~; No. 8:17~; Ps. 78:51~; Ps. 105:36.}.
\TextTitle{Israël sort d'Egypte}
\VS{31}Il appela donc Moïse et Aaron de nuit, et leur dit~: Levez-vous, sortez du milieu de mon peuple, tant vous que les enfants d'Israël, allez et servez Yahweh, comme vous en avez parlé.
\VS{32}Prenez aussi votre menu et gros bétail, comme vous en avez parlé, et allez-vous-en et bénissez-moi.
\VS{33}Et les Egyptiens pressaient le peuple et se hâtaient de les faire sortir du pays, car ils disaient~: Nous sommes tous morts.
\VS{34}Le peuple donc prit sa pâte avant qu'elle fût levée, ayant leurs mains liées avec leurs vêtements, sur leurs épaules.
\VS{35}Or les enfants d'Israël firent selon la parole de Moïse, et demandèrent aux Egyptiens des vases d'argent et d'or, et des vêtements.
\VS{36}Et Yahweh fit trouver grâce au peuple auprès des Egyptiens, qui les leur prêtèrent~; de sorte qu'ils dépouillèrent les Egyptiens.
\VS{37}Ainsi, les fils d'Israël étant partis de Ramsès, vinrent à Succoth, environ six cent mille hommes de pied, sans les enfants.
\VS{38}Il s'en alla aussi avec eux un grand nombre de toutes sortes de gens~; et du menu et du gros bétail, en fort grands troupeaux.
\VS{39} Or parce qu'ils avaient été chassés d'Egypte, et qu'ils n'avaient pas pu tarder plus longtemps, et que même ils n'avaient fait aucune provision, ils cuisirent par gâteaux sans levain, la pâte qu'ils avaient emportée d'Egypte~; car ils ne l'avaient point fait lever. 
\VS{40}Or le séjour des enfants d'Israël en Egypte fut de quatre cent trente ans\FTNT{Ge. 15:13~; Ac. 7:6~; Ga. 3:17.}.
\VS{41}Il arriva donc au bout de quatre cent trente ans, il arriva dis-je, en ce propre jour-là, que toutes les armées de Yahweh sortirent du pays d'Egypte.
\VS{42}C'est la nuit qui doit être soigneusement observée en l'honneur de Yahweh, parce qu'alors il les retira du pays d'Egypte~; cette nuit-là est à observer en l'honneur de Yahweh, par tous les enfants d'Israël de génération en génération\FTNT{De. 16:1-6.}.
\VS{43}Yahweh dit aussi à Moïse et à Aaron~: C'est ici l'ordonnance de la Pâque~: Aucun étranger n'en mangera~;
\VS{44}mais tout esclave qu'on aura acheté par argent sera circoncis, et alors il en mangera.
\VS{45}L'étranger et le mercenaire n'en mangeront point.
\VS{46}On la mangera dans une même maison, et vous n'emporterez point de sa chair hors de la maison, et vous n'en casserez point les os.
\VS{47}Toute l'assemblée d'Israël la fera.
\VS{48}Et si quelque étranger qui habite chez toi veut faire la Pâque à Yahweh, que tout mâle qui lui appartient soit circoncis~; et alors il s'approchera pour la faire, et il sera comme celui qui est né dans le pays~; mais aucun incirconcis n'en mangera.
\VS{49}Il y aura une même loi pour celui qui est né dans le pays et pour l'étranger qui habite parmi vous.
\VS{50}Tous les enfants d'Israël firent ce que Yahweh avait ordonné à Moïse et à Aaron~; ils le firent ainsi.
\VS{51}Il arriva donc en ce même jour que Yahweh retira les enfants d'Israël du pays d'Egypte, selon leurs armées.
\Chap{13}
\TextTitle{Consécration des premiers-nés à Yahweh}
\VerseOne{}Et Yahweh parla à Moïse, et dit~:
\VS{2}Sanctifie-moi tout premier-né, tout premier-né issu du sein maternel parmi les fils d'Israël, tant des hommes que des bêtes, car il est à moi\FTNT{Lé. 27:26-27~; No. 3:13~; No. 8:17~; Lu. 2:22-23.}.
\VS{3}Moïse donc dit au peuple~: Souvenez-vous de ce jour où vous êtes sortis d'Egypte, de la maison de servitude~; car Yahweh vous en a retirés par sa main puissante~; on ne mangera donc point de pain levé.
\VS{4}Vous sortez aujourd'hui dans le mois où les épis mûrissent.
\VS{5}Quand donc Yahweh t'aura introduit dans le pays des Cananéens, des Héthiens, des Amoréens, des Héviens et des Jébusiens, qu'il a juré à tes pères de te donner, et qui est un pays découlant de lait et de miel, alors tu feras ce service durant ce mois-ci.
\VS{6}Pendant sept jours tu mangeras des pains sans levain, et au septième jour il y aura une fête solennelle à Yahweh.
\VS{7}On mangera durant sept jours des pains sans levain~; il ne sera point vu chez toi de pain levé et même il ne sera point vu de levain dans toutes tes contrées.
\VS{8}Et ce jour-là, tu feras entendre ces choses à tes enfants, en disant~: C'est à cause de ce que Yahweh m'a fait en me retirant d'Egypte.
\VS{9}Et ceci te sera pour signe sur ta main, et comme un rappel entre tes yeux, afin que la loi de Yahweh soit dans ta bouche, car Yahweh t'aura retiré d'Egypte par sa main puissante\FTNT{De. 6:8~; De. 11:18.}.
\VS{10}Tu observeras cette ordonnance au jour fixé d'année en année.
\VS{11}Aussi, quand Yahweh t'aura introduit dans le pays des Cananéens, selon qu'il a juré à toi et à tes pères, et qu'il te l'aura donné,
\VS{12}tu consacreras à Yahweh tout premier-né issu du sein de sa mère, même tout premier-né des animaux que tu auras~; les mâles appartiendront à Yahweh.
\VS{13}Et tu rachèteras avec un petit d'entre les brebis ou d'entre les chèvres, tout premier-né de l'ânesse, et si tu ne le rachètes point, tu lui briseras la nuque. Tu rachèteras aussi tout premier-né des hommes parmi tes fils.
\VS{14}Et quand ton fils t'interrogera à l'avenir, en disant~: Que veut dire ceci~? Alors tu lui diras~: Yahweh nous a retirés par main forte hors d'Egypte, de la maison de servitude.
\VS{15}Car il arriva que, quand Pharaon s'obstinait à ne point nous laisser aller, Yahweh tua tous les premiers-nés au pays d'Egypte, depuis les premiers-nés des hommes jusqu'aux premiers-nés des bêtes. Voilà pourquoi je sacrifie à Yahweh tout premier-né mâle issu du sein de sa mère, et je rachète tout premier-né de mes fils.
\VS{16}Ceci te sera donc pour signe sur ta main, et pour fronteaux entre tes yeux, que Yahweh nous a retirés d'Egypte par sa main puissante.
\TextTitle{Début du voyage, Yahweh dirige son peuple}
\VS{17}Or lorsque Pharaon laissa aller le peuple, Dieu ne les conduisit point par le chemin du pays des Philistins, bien qu'il fût le plus court~; car Dieu dit~: C'est afin qu'il n'arrive que le peuple se repente quand il verra la guerre, et qu'il ne retourne en Egypte.
\VS{18}Mais Dieu fit tourner le peuple par le chemin du désert, vers la Mer Rouge. Ainsi, les enfants d'Israël montèrent en armes hors du pays d'Egypte.
\VS{19}Et Moïse avait pris avec lui les ossements de Joseph, parce que Joseph avait expressément fait jurer les enfants d'Israël, en leur disant~: Dieu vous visitera très certainement, et vous transporterez donc avec vous mes ossements d'ici\FTNT{Ge. 50:25~; Jos. 24:32.}.
\VS{20}Et ils partirent de Succoth, et campèrent à Etham, qui est à l'extrémité du désert.
\VS{21}Et Yahweh allait devant eux, de jour dans une colonne de nuée pour les conduire par le chemin~; et de nuit dans une colonne de feu pour les éclairer, afin qu'ils marchent jour et nuit\FTNT{No. 9:13-23~; No. 10:43~; De. 1:33~; Né. 9:12-19~; 1 Co. 10:1.}.
\VS{22}Et il ne retira point la colonne de nuée le jour, ni la colonne de feu la nuit de devant le peuple.
\Chap{14}
\TextTitle{Pharaon et son armée à la poursuite d'Israël}
\VerseOne{}Et Yahweh parla à Moïse et dit~:
\VS{2}Parle aux enfants d'Israël et dis-leur: Qu'ils se détournent, et qu'ils campent devant Pi-Hahiroth, entre Migdol et la mer, vis-à-vis de Baal-Tsephon. Vous camperez vis-à-vis de ce lieu-là près de la mer\FTNT{No. 33:7.}.
\VS{3}Pharaon dira des enfants d'Israël~: Ils sont confus dans le pays, le désert les a enfermés.
\VS{4}Et j'endurcirai le cœur de Pharaon, et il vous poursuivra. Ainsi je serai glorifié en Pharaon et en toute son armée et les Egyptiens sauront que je suis Yahweh~; et ils firent ainsi.
\VS{5}Or on avait rapporté au roi d'Egypte que le peuple s'enfuyait, et le cœur de Pharaon et de ses serviteurs fut changé à l'égard du peuple, et ils dirent~: Qu'est-ce que nous avons fait en laissant aller Israël, de sorte qu'il ne nous servira plus~?
\VS{6}Alors il fit atteler son char et il prit son peuple avec lui.
\VS{7}Il prit donc six cents chars d'élite et tous les chars de l'Egypte~; et il y avait des capitaines sur tout cela.
\VS{8}Et Yahweh endurcit le cœur de Pharaon, roi d'Egypte, qui poursuivit les enfants d'Israël. Or les fils d'Israël étaient sortis à main levée\FTNT{Lé. 26:13~; No. 33:3.}.
\VS{9}Les Egyptiens donc les poursuivirent~; et tous les chevaux des chars de Pharaon, ses cavaliers et son armée les atteignirent comme ils étaient campés près de la mer, vers Pi-Hahiroth vis-à-vis de Baal-Tsephon.
\VS{10}Et Pharaon approchait. Les enfants d'Israël levèrent leurs yeux, et voici, les Egyptiens marchaient après eux. Et les fils d'Israël eurent une grande frayeur et crièrent à Yahweh.
\VS{11}Ils dirent aussi à Moïse~: Est-ce qu'il n'y avait pas des sépulcres en Egypte pour que tu nous aies emmenés pour mourir au désert~? Que nous as-tu fait en nous faisant sortir d'Egypte~?
\VS{12}N'est-ce pas ce que nous te disions en Egypte, en disant~: Retire-toi de nous et que nous servions les Egyptiens~? Car nous aimons mieux les servir que de mourir au désert.
\TextTitle{Délivrance miraculeuse par Yahweh}
\VS{13}Et Moïse dit au peuple~: Ne craignez point, arrêtez-vous et voyez la délivrance que Yahweh vous donnera aujourd'hui~; car les Egyptiens que vous voyez aujourd'hui, vous ne les verrez plus.
\VS{14}Yahweh combattra pour vous et vous demeurerez tranquilles.
\VS{15}Or Yahweh avait dit à Moïse~: Que cries-tu à moi~? Parle aux enfants d'Israël, qu'ils marchent.
\VS{16}Et toi, élève ta verge, étends ta main sur la mer, et fends-la~; et que les enfants d'Israël entrent au milieu de la mer à sec.
\VS{17} Et quant à moi, voici, je m'en vais endurcir le cœur des Egyptiens, afin qu'ils entrent après eux~; et je serai glorifié en Pharaon, et en toute son armée, en ses chars et en ses cavaliers.
\VS{18}Et les Egyptiens sauront que je suis Yahweh, quand j'aurai été glorifié en Pharaon, avec ses chars et ses cavaliers.
\VS{19}Et l'Ange de Dieu qui allait devant le camp d'Israël partit, et s'en alla derrière eux~; et la colonne de nuée partit de devant eux et se tint derrière eux.
\VS{20}Et elle vint entre le camp des Egyptiens et le camp d'Israël. Elle était aux uns une nuée et une obscurité~; et pour les autres, elle les éclairait la nuit. L'un des camps n'approcha point de l'autre durant toute la nuit.
\VS{21}Or Moïse avait étendu sa main sur la mer, et Yahweh fit reculer la mer toute la nuit par un vent d'orient qui souffla avec puissance~; il mit la mer à sec, et les eaux se fendirent\FTNT{Jos. 4:23~; Ps. 66:6~; Ps. 106:9~; Hé. 11:29.}.
\VS{22}Et les enfants d'Israël entrèrent au milieu de la mer à sec, et les eaux leur servaient de mur à droite et à gauche.
\VS{23}Et les Egyptiens les poursuivirent~; et ils entrèrent après eux au milieu de la mer, à savoir tous les chevaux de Pharaon, ses chars et ses cavaliers.
\VS{24}Mais il arriva que sur la veille du matin, Yahweh étant dans la colonne de feu et dans la nuée, regarda le camp des Egyptiens et le mit en déroute.
\VS{25}Il ôta les roues de leurs chars et alourdit leur marche. Alors les Egyptiens dirent~: Fuyons de devant les Israëlites, car Yahweh combat pour eux contre les Egyptiens.
\VS{26}Et Yahweh dit à Moïse~: Etends ta main sur la mer, et les eaux retourneront sur les Egyptiens, sur leurs chars et sur leurs cavaliers.
\VS{27}Moïse donc étendit sa main sur la mer, et la mer reprit son impétuosité vers le matin. Et les Egyptiens s'enfuyant rencontrèrent la mer qui s'était rejointe~; et ainsi Yahweh jeta les Egyptiens au milieu de la mer.
\VS{28}Car les eaux retournèrent et couvrirent les chars et les cavaliers de toute l'armée de Pharaon, qui étaient entrés après les Israélites dans la mer, et il n'en resta pas un seul.
\VS{29}Mais les enfants d'Israël marchèrent au milieu de la mer à sec, et les eaux leur servaient de mur à droite et à gauche.
\VS{30}Ainsi, Yahweh délivra, en ce jour-là, Israël de la main des Egyptiens~; et Israël vit sur le bord de la mer les Egyptiens morts.
\VS{31}Israël vit donc la grande puissance que Yahweh avait déployée contre les Egyptiens~; et le peuple craignit Yahweh, ils crurent en Yahweh, et en Moïse, son serviteur.
\Chap{15}
\TextTitle{Cantique de délivrance}
\VerseOne{}Alors Moïse et les enfants d'Israël chantèrent ce cantique à Yahweh, et dirent~: Je chanterai à Yahweh, car il est hautement élevé~; il a jeté dans la mer le cheval et celui qui le montait.
\VS{2}Yahweh est ma force et ma louange, et il a été mon Sauveur, mon Dieu. Je lui dresserai un tabernacle, c'est le Dieu de mon père, je l'exalterai.
\VS{3}Yahweh est un vaillant guerrier, son Nom est Yahweh.
\VS{4}Il a jeté dans la mer les chars de Pharaon et son armée~; l'élite de ses capitaines a été submergée dans la Mer Rouge.
\VS{5}Les gouffres les ont couverts, ils sont descendus au fond des eaux comme une pierre\FTNT{Né. 9:11.}.
\VS{6}Ta droite, ô Yahweh, s'est montrée magnifique en force~! Ta droite, ô Yahweh, a brisé l'ennemi\FTNT{Ps. 118:15-16~; Ps. 77:16.}~!
\VS{7}Tu as ruiné par la grandeur de ta majesté ceux qui s'élevaient contre toi~; tu as lâché ta colère et elle les a consumés comme du chaume.
\VS{8}Par le souffle de tes narines, les eaux ont été amoncelées~; les eaux courantes se sont arrêtés comme un monceau~; les gouffres ont été gelés au milieu de la mer.
\VS{9}L'ennemi disait~: Je poursuivrai, j'atteindrai, je partagerai le butin~; mon âme sera assouvie d'eux, je tirerai mon épée, ma main les détruira.
\VS{10}Tu as soufflé de ton vent, la mer les a couverts~; ils ont été enfoncés comme du plomb au plus profond des eaux.
\VS{11}Qui est comme toi parmi les dieux, ô Yahweh~! Qui est comme toi, magnifique en sainteté, digne d'être révéré et célébré, faisant des choses merveilleuses~?
\VS{12}Tu as étendu ta droite, la terre les a engloutis.
\VS{13}Tu as conduit par ta miséricorde ce peuple que tu as racheté~; tu l'as conduit par ta force à la demeure de ta sainteté.
\VS{14}Les peuples l'ont entendu, et ils en ont tremblé~; la douleur a saisi les habitants du pays des Philistins.
\VS{15}Alors les princes d'Edom seront troublés, et le tremblement saisira les puissants de Moab, tous les habitants de Canaan se fondront.
\VS{16}La frayeur et l'épouvante tomberont sur eux~; ils seront rendus muets comme une pierre par la grandeur de ton bras, jusqu'à ce que ton peuple soit passé, ô Yahweh~! Jusqu'à ce que ce peuple que tu as acquis soit passé\FTNT{De. 2:25~; De. 11:25~; Jos. 2:9.}.
\VS{17}Tu les introduiras et les planteras sur la montagne de ton héritage, au lieu que tu as préparé pour ta demeure, ô Yahweh~! Au lieu saint, ô Seigneur, que tes mains ont établi~!
\VS{18}Yahweh régnera à jamais et à perpétuité.
\VS{19}Car les chevaux de Pharaon, ses chars et ses cavaliers sont entrés dans la mer, et Yahweh a fait retourner sur eux les eaux de la mer~; mais les enfants d'Israël ont marché à sec au milieu de la mer.
\VS{20}Et Marie, la prophétesse, sœur d'Aaron, prit un tambour dans sa main, et toutes les femmes sortirent après elle, avec des tambours et des flûtes.
\VS{21}Et Marie leur répondait~: Chantez à Yahweh, car il est hautement élevé~; il a jeté dans la mer le cheval et celui qui le montait.
\TextTitle{Yahweh pourvoit pour son peuple}
\VS{22}Après cela, Moïse fit partir les Israélites de la Mer Rouge, et ils partirent vers le désert de Schur~; et ayant marché trois jours dans le désert, ils ne trouvèrent point d'eau.
\VS{23}De là, ils vinrent à Mara, mais ils ne purent boire les eaux de Mara, parce qu'elles étaient amères~; c'est pourquoi ce lieu fut appelé Mara.
\VS{24}Et le peuple murmura contre Moïse en disant~: Que boirons-nous~?
\VS{25}Et Moïse cria à Yahweh, et Yahweh lui montra\FTNT{«~Montra~» de l'hébreu «~yarah~» qui veut également dire «~enseigner~», «~signaler~», «~lancer~», «~instruire~», «~informer~», «~montrer~», «~jeter~» etc.} un certain bois qu'il jeta dans les eaux~; et les eaux devinrent douces. Il lui proposa là une ordonnance et une loi, et il l'éprouva là,
\VS{26}et lui dit~: Si tu écoutes attentivement la voix de Yahweh, ton Dieu, si tu fais ce qui est droit devant lui, si tu prêtes l'oreille à ses commandements, si tu gardes toutes ses ordonnances, je ne ferai venir sur toi aucune des infirmités que j'ai fait venir sur l'Egypte, car je suis Yahweh qui te guérit\FTNT{De. 7:12-15.}.
\VS{27}Puis ils vinrent à Elim, où il y avait douze fontaines d'eau, et soixante-dix palmiers. Et ils campèrent là, près des eaux.
\Chap{16}
\TextTitle{Yahweh envoie la manne}
\VerseOne{}Et toute l'assemblée des enfants d'Israël étant partie d'Elim, vint au désert de Sin, qui est entre Elim et Sinaï, le quinzième jour du second mois après qu'ils furent sortis du pays d'Egypte.
\VS{2}Et toute l'assemblée des enfants d'Israël murmura dans ce désert contre Moïse et Aaron.
\VS{3}Et les enfants d'Israël leur dirent~: Ah~! Pourquoi ne sommes-nous point morts par la main de Yahweh dans le pays d'Egypte, quand nous étions assis près des pots de viande, et que nous mangions du pain à satiété~? Car vous nous avez amenés dans ce désert pour faire mourir de faim toute cette assemblée\FTNT{1 Co. 10:10~; No. 11:4.}.
\VS{4}Et Yahweh dit à Moïse~: Voici, je vais vous faire pleuvoir des cieux du pain, et le peuple sortira et en recueillera chaque jour la provision d'un jour, afin que je l'éprouve, pour voir s'il observera ma loi ou non.
\VS{5}Mais qu'ils apprêtent au sixième jour ce qu'ils auront apporté, et qu'il y ait le double de ce qu'ils recueilleront chaque jour.
\VS{6}Moïse donc et Aaron dirent à tous les enfants d'Israël~: Ce soir vous saurez que Yahweh vous a tirés du pays d'Egypte.
\VS{7}Et au matin vous verrez la gloire de Yahweh, parce qu'il a entendu vos murmures, qui sont contre Yahweh~; car que sommes-nous pour que vous murmuriez contre nous~?
\VS{8}Moïse dit donc~: Ce sera quand Yahweh vous aura donné ce soir de la chair à manger, et qu'au matin, il vous aura rassasiés de pain, parce qu'il a entendu vos murmures, par lesquels vous avez murmuré contre lui. Car que sommes-nous~? Vos murmures ne sont pas contre nous, mais contre Yahweh.
\VS{9}Et Moïse dit à Aaron~: Dis à toute l'assemblée des enfants d'Israël~: Approchez-vous de la présence de Yahweh, car il a entendu vos murmures.
\VS{10}Or il arriva qu'aussitôt qu'Aaron eut parlé à toute l'assemblée des enfants d'Israël, ils regardèrent vers le désert, et voici, la gloire de Yahweh se montra dans la nuée.
\VS{11} Et Yahweh parla à Moïse, en disant~:
\VS{12}J'ai entendu les murmures des enfants d'Israël. Parle-leur et dis-leur~: Entre les deux soirs, vous mangerez de la chair, et au matin vous serez rassasiés de pain~; et vous saurez que je suis Yahweh, votre Dieu.
\VS{13}Sur le soir donc, il monta des cailles qui couvrirent le camp, et au matin il y eut une couche de rosée autour du camp.
\VS{14}Et cette couche de rosée étant évanouie, voici, sur la surface du désert, quelque chose de menu et de rond, comme du grain sur la terre.
\VS{15}Ce que les enfants d'Israël ayant vu, ils se dirent l'un à l'autre~: Qu'est-ce~? Car ils ne savaient ce que c'était. Et Moïse leur dit~: C'est le pain que Yahweh vous donne à manger\FTNT{Ps. 105:40.}.
\TextTitle{Récolte de la manne}
\VS{16}Or ce que Yahweh a ordonné, c'est que chacun en recueille autant qu'il lui en faut pour sa nourriture, un homer par tête, selon le nombre de vos personnes~; chacun en prendra pour ceux qui sont dans sa tente.
\VS{17}Les enfants d'Israël firent donc ainsi~; et les uns en recueillirent plus, les autres moins.
\VS{18}Et ils le mesuraient par homer~; et celui qui en avait recueilli beaucoup n'en avait pas plus qu'il ne lui en fallait~; ni celui qui en avait recueilli peu, n'en avait pas moins~; mais chacun en recueillait selon ce qu'il en pouvait manger.
\VS{19}Et Moïse leur avait dit~: Que personne n'en laisse rien de reste jusqu'au matin.
\VS{20}Mais il y en eut qui n'obéirent point à Moïse, car quelques-uns en réservèrent jusqu'au matin~; et il s'y engendra des vers, et cela puait. Et Moïse se mit en grande colère contre eux.
\VS{21}Ainsi, chacun en recueillait tous les matins autant qu'il lui en fallait pour se nourrir, et lorsque la chaleur du soleil était venue, elle se fondait.
\VS{22}Mais le sixième jour, ils recueillirent du pain en double, deux homers pour chacun~; et les principaux de l'assemblée vinrent pour le rapporter à Moïse.
\TextTitle{Le sabbat\FTNTT{Né. 9:13-14~; Mt. 12:1.}}
\VS{23} Et il leur dit~: C'est ce que Yahweh a dit~: Demain est le repos, le sabbat consacré à Yahweh~; faites cuire ce que vous avez à cuire, et faites bouillir ce que vous avez à bouillir, et serrez tout ce qui sera de surplus, pour le garder jusqu'au matin.
\VS{24}Ils le serrèrent donc jusqu'au matin, comme Moïse l'avait ordonné, et il ne pua point, et il n'y eut point de vers dedans.
\VS{25}Alors Moïse dit~: Mangez-le aujourd'hui, car c'est aujourd'hui le repos de Yahweh~; aujourd'hui vous n'en trouverez point dans les champs.
\VS{26}Durant six jours vous le recueillerez, mais le septième est le sabbat, il n'y en aura point ce jour-là.
\VS{27}Et au septième jour, quelques-uns du peuple sortirent pour en recueillir, mais ils n'en trouvèrent point.
\VS{28}Et Yahweh dit à Moïse~: Jusqu'à quand refuserez-vous de garder mes commandements et mes lois~?
\VS{29}Considérez que Yahweh vous a ordonné le sabbat, c'est pourquoi il vous donne au sixième jour du pain pour deux jours~; que chacun demeure au lieu où il sera, et qu'aucun ne sorte du lieu où il est le septième jour.
\VS{30}Le peuple donc se reposa le septième jour.
\VS{31}Et la maison d'Israël nomma ce pain manne\FTNT{Le mot «~manne~» vient de l'hébreu «~man~» et veut dire «~Qu'est-ce que cela~?~». La manne est une image de Jésus, le Pain de vie descendu du ciel (Jn. 6:32-52). La consommation quotidienne du Pain de vie, qui est aussi la Parole de Dieu, apporte la vie éternelle.}. Elle était comme de la semence de coriandre blanche, et ayant le goût d'un gâteau au miel.
\VS{32}Et Moïse dit~: Voici ce que Yahweh a ordonné~: Qu'on en remplisse un homer pour le garder pour vos générations, afin qu'on voie le pain que je vous ai fait manger au désert, après vous avoir retirés du pays d'Egypte.
\VS{33}Moïse dit à Aaron~: Prends un vase, et mets-y un plein d'homer de manne, et pose-le devant Yahweh, afin qu'il soit conservé pour vos générations.
\VS{34}Et Aaron le posa devant le témoignage pour y être gardé, selon que le Seigneur l'avait ordonné à Moïse.
\VS{35}Et les enfants d'Israël mangèrent la manne durant quarante ans, jusqu'à leur arrivée dans un pays habité~; ils mangèrent, dis-je, la manne, jusqu'à leur arrivée aux frontières du pays de Canaan.
\VS{36}Or un homer est la dixième partie d'un épha.
\Chap{17}
\TextTitle{Miracle de l'eau qui sort du rocher}
\VerseOne{}Et toute l'assemblée des enfants d'Israël partit du désert de Sin, selon l'ordre de marche que Yahweh leur avait ordonné, et ils campèrent à Rephidim, où il n'y avait point d'eau à boire pour le peuple.
\VS{2}Et le peuple se souleva contre Moïse et ils lui dirent~: Donnez-nous de l'eau à boire. Et Moïse leur dit~: Pourquoi vous soulevez-vous contre moi~? Pourquoi tentez-vous Yahweh\FTNT{No. 20:2-5.}~?
\VS{3}Le peuple donc eut soif en ce lieu-là, par faute d'eau~; et ainsi le peuple murmura contre Moïse, en disant~: Pourquoi nous as-tu fait monter hors d'Egypte, pour nous faire mourir de soif, nous, nos enfants, et nos troupeaux~?
\VS{4}Et Moïse cria à Yahweh, en disant~: Que ferai-je à ce peuple~? Encore un peu, et ils me lapideront.
\VS{5}Et Yahweh répondit à Moïse: Passe devant le peuple, et prends avec toi des anciens d'Israël, prends aussi dans ta main la verge avec laquelle tu as frappé le fleuve, et viens~!
\VS{6}Voici, je vais me tenir là devant toi sur le rocher d'Horeb~; et tu frapperas le rocher, et il en sortira des eaux, et le peuple en boira. Moïse donc fit ainsi aux yeux des anciens d'Israël\FTNT{De. 9:8~; Ps. 78:15~; 1 Co. 10:4.}.
\VS{7}Et il nomma le lieu Massa et Meriba, à cause de la querelle des enfants d'Israël, et parce qu'ils avaient tenté Yahweh, en disant~: Yahweh est-il au milieu de nous ou non~?
\TextTitle{Bataille et victoire contre Amalek}
\VS{8}Alors Amalek vint et livra bataille contre Israël à Rephidim\FTNT{De. 25:17-18.}.
\VS{9}Et Moïse dit à Josué~: Choisis-nous des hommes, et sors pour combattre contre Amalek~; et je me tiendrai demain sur le sommet de la colline, et la verge de Dieu sera dans ma main.
\VS{10}Et Josué fit comme Moïse lui avait ordonné en combattant contre Amalek. Mais Moïse, Aaron et Hur montèrent au sommet de la colline.
\VS{11}Et il arrivait que lorsque Moïse élevait sa main, Israël était alors le plus fort, mais quand il reposait sa main, alors Amalek était le plus fort.
\VS{12}Et les mains de Moïse étant devenues pesantes, ils prirent une pierre et la mirent sous lui, et il s'assit dessus~; Aaron et Hur soutenaient ses mains, l'un d'un côté, et l'autre de l'autre côté~; et ainsi ses mains furent fermes jusqu'au soleil couchant.
\VS{13}Josué donc défit Amalek et son peuple au tranchant de l'épée.
\VS{14}Et Yahweh dit à Moïse~: Ecris ceci pour mémoire dans un livre, et fais entendre à Josué que j'effacerai entièrement la mémoire d'Amalek de dessous les cieux.
\VS{15}Et Moïse bâtit un autel et le nomma Yahweh, ma bannière.
\VS{16}Il dit aussi~: Parce que la main a été levée contre le trône de Yahweh, Yahweh aura toujours la guerre contre Amalek.
\Chap{18}
\TextTitle{Jéthro conseille Moïse}
\VerseOne{}Or Jéthro, prêtre de Madian, beau-père de Moïse, apprit toutes les choses que Yahweh avait faites à Moïse, et à Israël, son peuple, à savoir comment Yahweh avait retiré Israël de l'Egypte.
\VS{2}Jéthro, beau-père de Moïse, prit Séphora la femme de Moïse, après que Moïse l'eut renvoyée,
\VS{3}et les deux fils de cette femme, dont l'un s'appelait Guerschom, car il avait dit~: J'habite un pays étranger~;
\VS{4}et l'autre Eliézer, car il avait dit~: Le Dieu de mon père m'a secouru et m'a délivré de l'épée de Pharaon.
\VS{5}Jéthro donc, beau-père de Moïse, vint vers Moïse avec ses fils et sa femme au désert, où il était campé, à la montagne de Dieu.
\VS{6}Il fit dire à Moïse~: Jéthro, ton beau-père, vient vers toi, et ta femme et ses deux fils avec elle.
\VS{7}Et Moïse sortit au-devant de son beau-père, et s'étant prosterné, il l'embrassa~; et ils s'enquirent l'un de l'autre, de leur santé, puis ils entrèrent dans la tente.
\VS{8}Et Moïse raconta à son beau-père toutes les choses que Yahweh avait faites à Pharaon et aux Egyptiens en faveur d'Israël, et toute la fatigue qu'ils avaient soufferte en chemin, et comment Yahweh les avait délivrés.
\VS{9}Et Jéthro se réjouit de tout le bien que Yahweh avait fait à Israël, parce qu'il les avait délivrés de la main des Egyptiens.
\VS{10}Puis Jéthro dit~: Béni soit Yahweh qui vous a délivrés de la main des Egyptiens et de la main de Pharaon, qui a, dis-je, délivré le peuple de la main des Egyptiens~!
\VS{11}Je sais maintenant que le Seigneur est plus grand que tous les dieux, car la chose même en laquelle ils se sont enorgueillis, il a eu le dessus sur eux.
\VS{12}Jéthro, beau-père de Moïse, apporta aussi un holocauste et des sacrifices pour les offrir à Dieu. Et Aaron et tous les anciens d'Israël vinrent pour manger du pain avec le beau-père de Moïse dans la présence de Dieu.
\VS{13}Et il arriva, le lendemain, comme Moïse siégeait pour juger le peuple, et que le peuple se tenait devant Moïse depuis le matin jusqu'au soir,
\VS{14}que le beau-père de Moïse vit tout ce qu'il faisait au peuple, et il lui dit~: Qu'est-ce que tu fais à l'égard de ce peuple~? Pourquoi es-tu assis seul, et tout le peuple se tient devant toi depuis le matin jusqu'au soir~?
\VS{15} Et Moïse répondit à son beau-père~: C'est que le peuple vient à moi pour s'enquérir de Dieu.
\VS{16}Quand ils ont quelque affaire, ils viennent à moi, et je juge entre l'un et l'autre, et je leur fais entendre les ordonnances de Dieu et ses lois.
\VS{17}Mais le beau-père de Moïse lui dit~: Ce que tu fais n'est pas bien.
\VS{18}Certainement, tu succomberas, toi et ce peuple qui est avec toi~; car cela est trop pesant pour toi, tu ne saurais faire cela toi seul.
\VS{19}Ecoute donc mon conseil~; je te conseillerai et Dieu sera avec toi: Sois pour ce peuple auprès de Dieu, et rapporte les causes à Dieu.
\VS{20}Et instruis-les des ordonnances et des lois~; et fais-leur connaître la voie par laquelle ils auront à marcher et ce qu'ils auront à faire.
\VS{21}Et choisis-toi d'entre tout le peuple des hommes vertueux, craignant Dieu~; des hommes véritables, haïssant le gain déshonnête, et établis-les chefs de milliers, chefs de centaines, chefs de cinquantaines et chef des dizaines.
\VS{22}Et qu'ils jugent le peuple en tout temps, mais qu'ils te rapportent toutes les grandes affaires, et qu'ils jugent toutes les petites causes~; ainsi ils te soulageront et porteront une partie de la charge avec toi.
\VS{23}Si tu fais cela, et que Dieu te l'ordonne, tu pourras subsister, et tout le peuple parviendra en paix à destination.
\VS{24}Moïse donc obéit à la parole de son beau-père, et fit tout ce qu'il lui avait dit.
\VS{25}Ainsi, Moïse choisit de tout Israël des hommes vertueux, et les établit chefs sur le peuple, chefs de milliers, chefs de centaines, chefs de cinquantaines, et chefs de dizaines,
\VS{26}lesquels devaient juger le peuple en tout temps, mais ils devaient rapporter à Moïse les choses difficiles, et juger de toutes les petites affaires.
\VS{27}Puis Moïse laissa partir son beau-père, qui s'en alla dans son pays.
\Chap{19}
\TextTitle{DEBUT DE LA PÉRIODE DE LA LOI MOSAÏQUE OU DE LA PREMIÈRE ALLIANCE}
\VerseOne{}Au premier jour du troisième mois, après que les enfants d'Israël furent sortis du pays d'Egypte, en ce même jour-là, ils vinrent au désert de Sinaï.
\VS{2}Etant donc partis de Rephidim, ils vinrent au désert de Sinaï et campèrent au désert. Et Israël campa vis-à-vis de la montagne.
\VS{3}Et Moïse monta vers Dieu, car Yahweh l'avait appelé de la montagne pour lui dire~: Tu parleras ainsi à la maison de Jacob, et tu annonceras ceci aux enfants d'Israël~:
\VS{4}Vous avez vu ce que j'ai fait aux Egyptiens~; comment je vous ai portés comme sur des ailes d'aigle et vous ai amenés à moi.
\VS{5}Maintenant donc, si vous obéissez exactement à ma voix, et si vous gardez mon alliance, vous serez aussi d'entre tous les peuples mon plus précieux joyau, car toute la terre m'appartient\FTNT{C'est ici que débute la période de la Loi ou Première Alliance. Le fait d'avoir réuni les textes de Genèse à Malachie sous l'appellation «~Ancien Testament~» a induit beaucoup de personnes en erreur quant à leur compréhension du plan de Dieu pour nos vies. Tout d'abord, l'emploi du mot «~testament~» est inapproprié puisqu'on ne peut parler de testament sans qu'il y ait au préalable la mort du testateur (Hé. 9:16-17). Certes, des animaux étaient tués sous la Loi pour couvrir les péchés. Toutefois, ces sacrifices étaient imparfaits et par conséquent prévus pour ne durer qu'un temps, en attendant le sacrifice parfait de Jésus-Christ (Hé. 10:1-14). De plus, il est évident que les animaux sacrifiés ne nous ont rien légué.\\ Ensuite, il est à noter que tous les textes classés dans ce que l'on appelle à tort «~Ancien Testament~» ne se rapportent pas exclusivement et nécessairement à la Loi. Ainsi, des prophètes, en commençant par Moïse en personne, ayant vécu sous la Loi, ont prophétisé et écrit sur d'autres sujets que la Loi, notamment sur la grâce et la fin des temps. N'oublions pas non plus que Jésus-Christ est né et a vécu sous la Loi (Ga. 4:4). En tant que juif, il l'a scrupuleusement respectée de telle sorte qu'elle fut totalement accomplie en Lui (Mt. 5:17-18~; Jn. 19:30). En conséquence, la fin de la Loi mosaïque eut lieu après la mort du Seigneur, précisément au moment où le Seigneur a dit «~Tout est accompli~», et lorsque le voile du temple s'est déchiré de haut en bas (Mt. 27:50-51~; Jn. 19:30). La Nouvelle Alliance, ou le Testament de Jésus débuta avec l'effusion de l'Esprit (Ac. 2). De la mort du Seigneur à la Pentecôte, une période de transition de cinquante jours s'est écoulée. Jésus-Christ s'est présenté pendant ce temps dans le sanctuaire céleste pour présenter son sang dans le Saint des saints. Une fois son sacrifice examiné et accepté, le Saint-Esprit qui avait été retiré de l'homme (Ge. 6:3) put de nouveau revenir habiter le cœurs des croyants.\\Mais qu'est-ce que la loi exactement~? Beaucoup de chrétiens sont dans la confusion à ce sujet. En réalité, il n'y avait pas qu'une loi mais trois sortes de lois~: Les lois morales et les lois cérémonielles qui préexistaient depuis l'éternité~; et les lois civiles qui ont débuté avec Moïse car elles ne concernaient que son peuple.\\- Les lois civiles régissaient le fonctionnement de la vie en communauté des Hébreux. Elles étaient exclusivement réservées au peuple d'Israël dans le camp puis dans le pays de Canaan (Ex. 21:1-2~; De. 23).\\- Les lois morales font référence à la nature de Dieu~: Son amour, sa justice, sa sainteté, etc. Les dix commandements, à l'exception du sabbat tel que prescrit par Moïse (Ex. 16:28-29~; Lé. 15:32), font partie des lois morales (Ex. 20:1-17). Les dix paroles ne constituent qu'une base, un résumé. Ainsi, d'autres règles morales sont énoncées tout au long des Ecritures notamment sur la sexualité (Lé. 18:1-22), l'interdiction des sacrifices humains et de l'occultisme (De. 18:10-13), le respect d'autrui et l'entraide (Lé. 19:10-18~; Lé. 19:29-36). Comme il est impossible de consigner dans un livre tous les péchés moraux, le Seigneur a inscrit les lois morales dans le cœur de l'homme afin qu'il sache instinctivement faire la différence entre le bien et mal (Ro. 2:14-15). Jésus les a résumées en ces quelques mots~: «~Tu aimeras le Seigneur ton Dieu de tout ton cœur, et de toute ton âme, et de toute ta pensée. Celui-ci est le premier et le plus grand commandement. Et le second semblable à celui-là, est~: Tu aimeras ton prochain comme toi-même.~» (Mt. 22:37-39). Ces lois sont encore en vigueur aujourd'hui et le resteront pour toujours.\\- Les lois cérémonielles étaient relatives au culte et au sanctuaire terrestre, c'est-à-dire le tabernacle puis le temple de Jérusalem (Hé. 9:1-10). Elles regroupent toutes les ordonnances concernant les sacrifices, les ablutions, les sabbats, les fêtes de Yahweh, la dîme des Lévites et des prêtres (voir commentaire en No. 18:21 et Mal. 3:10). Les livres du Lévitique et des Nombres exposent en détail toutes les ordonnances reçues par Moïse d'après le modèle céleste que Yahweh lui avait montré sur le Mont Sinaï (Ex. 26:30). Les lois cérémonielles préexistaient donc depuis l'éternité.\\Les lois cérémonielles représentent la Première Alliance qui avait pour fondement la loi morale. Or cette alliance a vieilli puis disparu car elle n'était que l'ombre des choses à venir (Hé. 8:13). En effet, elle était basée sur quatre points principaux~: Le temple, le culte centralisé, le sacrifice et les prêtres. En Christ, nous n'avons plus besoin d'un temple physique puisque nous sommes devenus les temples vivants de Yahweh (1 Co. 6:19~; Ep. 2:22). Nous pouvons désormais adorer le Seigneur en Esprit et en vérité, à tout moment et en tout lieu (Jn. 4:23). Le sacerdoce lévitique ayant été aboli, chaque enfant de Dieu est devenu un prêtre (Ap. 5:10) qui offre en sacrifice sa propre vie consacrée au Seigneur (Ro. 12:1).\\Les lois cérémonielles ont donc trouvé leur parfait accomplissement en Jésus-Christ~: Tous les sacrifices sanglants le préfiguraient, toutes les solennités ont été réalisées en Lui (voir note en Lé. 23). Christ est donc la fin de la Loi, non pas morale, mais cérémonielle (Ro. 10:4).\\Un lien étroit existe entre les lois morales et les lois cérémonielles. La loi morale est comme un diagnostic qui révèle une pathologie incurable comme le sida~: Le péché (Ro. 5:13-20~; Ro. 7:7-14). En la découvrant, l'homme se sent condamné car il réalise qu'il ne peut pas répondre aux exigences de la justice divine. La loi cérémonielle (le sang des animaux - Hé. 9:1-13~; Hé. 10:11) a donné aux hommes une sorte de trithérapie pour les soulager provisoirement de leurs péchés mais sans pour autant les ôter (guérir, délivrer, nettoyer, laver) définitivement. Seul le sang de la Nouvelle Alliance, c'est-à-dire le sang de Jésus-Christ, a pu nous délivrer une fois pour toutes (Jn. 1:29~; Hé. 9:11-26~; Hé. 10:1-23~; Ap. 1:6).}.
\VS{6}Et vous me serez un royaume de prêtres, et une nation sainte~; ce sont là les discours que tu tiendras aux enfants d'Israël.
\VS{7}Puis Moïse vint et appela les anciens du peuple, et proposa devant eux toutes ces choses-là que Yahweh lui avait ordonné.
\VS{8}Et tout le peuple répondit d'un commun accord, en disant~: Nous ferons tout ce que Yahweh a dit. Et Moïse rapporta à Yahweh toutes les paroles du peuple.
\TextTitle{Moïse doit sanctifier le peuple pour qu'il rencontre Yahweh}
\VS{9}Et Yahweh dit à Moïse~: Voici, je viendrai à toi dans une nuée épaisse, afin que le peuple entende quand je parlerai avec toi, et qu'il te croie aussi toujours~; car Moïse avait rapporté à Yahweh les paroles du peuple.
\VS{10}Yahweh dit aussi à Moïse~: Va-t'en vers le peuple, et sanctifie-les aujourd'hui et demain, et qu'ils lavent leurs vêtements.
\VS{11}Et qu'ils soient tous prêts pour le troisième jour, car au troisième jour, Yahweh descendra sur la montagne de Sinaï, à la vue de tout le peuple.
\VS{12}Or tu mettras des bornes pour le peuple tout autour, et tu diras~: Gardez-vous de monter sur la montagne et de toucher aucune de ses extrémités. Quiconque touchera la montagne sera puni de mort.
\VS{13}Aucune main ne la touchera, et certainement il sera lapidé, ou percé de flèches~; soit bête, soit homme, il ne vivra point. Quand la trompette sonnera longuement, ils monteront vers la montagne.
\VS{14}Et Moïse descendit de la montagne vers le peuple, et sanctifia le peuple, et ils lavèrent leurs vêtements.
\VS{15}Et il dit au peuple~: Soyez tous prêts pour le troisième jour, et ne vous approchez point de vos femmes.
\VS{16}Et le troisième jour au matin, il y eut des tonnerres, et des éclairs, et une grosse nuée sur la montagne, avec un très fort son de shofar, et tout le peuple dans le camp fut effrayé.
\VS{17}Alors Moïse fit sortir le peuple du camp pour aller au-devant de Dieu~; et ils s'arrêtèrent au pied de la montagne.
\VS{18}Or le mont Sinaï était tout couvert de fumée, parce que Yahweh y était descendu en feu~; et sa fumée montait comme la fumée d'une fournaise, et toute la montagne tremblait fort.
\VS{19}Et comme le son du shofar se renforçait de plus en plus, Moïse parla, et Dieu lui répondit par une voix.
\VS{20}Yahweh donc étant descendu sur la montagne de Sinaï, au sommet de la montagne, Yahweh appela Moïse au sommet de la montagne~; et Moïse y monta.
\VS{21}Et Yahweh dit à Moïse~: Descends. Somme le peuple qu'il ne rompe point les barrières pour monter vers Yahweh afin de regarder~; de peur qu'un grand nombre d'entre eux ne périsse.
\VS{22}Et même, que les prêtres qui s'approchent de Yahweh se sanctifient aussi, de peur qu'il n'arrive que Yahweh se jette sur eux.
\VS{23} Et Moïse dit à Yahweh~: Le peuple ne pourra pas monter sur la montagne de Sinaï, parce que tu nous as sommés en me disant~: Mets des bornes sur la montagne, et sanctifie-la.
\VS{24}Et Yahweh lui dit~: Va, descends~; puis tu monteras, toi, et Aaron avec toi~; mais que les prêtres et le peuple ne rompent point les bornes pour monter vers Yahweh, de peur qu'il n'arrive qu'il se jette sur eux.
\VS{25}Moïse descendit donc vers le peuple, et lui dit ces choses.
\Chap{20}
\TextTitle{Les dix paroles}
\VerseOne{}Alors Dieu prononça toutes ces paroles, disant~:
\VS{2}Je suis Yahweh, ton Dieu, qui t'ai retiré du pays d'Egypte, de la maison de servitude.
\VS{3}Tu n'auras point d'autres dieux devant ma face.
\VS{4}Tu ne te feras point d'image taillée, ni aucune ressemblance des choses qui sont là-haut aux cieux, ni ici-bas sur la terre, ni dans les eaux sous la terre\FTNT{Lé. 26:1.}.
\VS{5}Tu ne te prosterneras point devant elles, et ne les serviras point~; car je suis Yahweh, ton Dieu~; le Dieu qui est jaloux, punissant l'iniquité des pères sur les fils, jusqu'à la troisième et à la quatrième génération de ceux qui me haïssent~;
\VS{6}et faisant miséricorde en mille générations à ceux qui m'aiment et qui gardent mes commandements.
\VS{7}Tu ne prendras point le Nom de Yahweh, ton Dieu, en vain~; car Yahweh ne tiendra point pour innocent celui qui aura pris son Nom en vain\FTNT{Lé. 19:12~; Mt. 5:33.}.
\VS{8}Souviens-toi du jour du repos pour le sanctifier.
\VS{9}Tu travailleras six jours, et tu feras toute ton œuvre.
\VS{10}Mais le septième jour est le repos de Yahweh ton Dieu. Tu ne feras aucune œuvre en ce jour-là, ni toi, ni ton fils, ni ta fille, ni ton serviteur, ni ta servante, ni ton bétail, ni ton étranger qui est dans tes portes.
\VS{11}Car Yahweh a fait en six jours les cieux, la terre, la mer, et tout ce qui est en eux, et s'est reposé le septième jour~; c'est pourquoi Yahweh a béni le jour du repos et l'a sanctifié\FTNT{Ge. 2:3~; Ex. 31:14~; Ez. 20:12.}.
\VS{12}Honore ton père et ta mère, afin que tes jours soient prolongés sur la terre que Yahweh, ton Dieu, te donne\FTNT{Lé. 19:3~; De. 5:16~; Mt. 15:4~; Ep. 6:2.}.
\VS{13}Tu ne commettras pas de meurtre\FTNT{Mt. 5:21.}.
\VS{14}Tu ne commettras pas d'adultère\FTNT{Lé. 20:10~; De. 5:18~; Pr. 6:32~; Mt. 5:32~; Ro. 7:3.}.
\VS{15}Tu ne déroberas pas.
\VS{16}Tu ne diras pas de faux témoignage contre ton prochain.
\VS{17}Tu ne convoiteras pas la maison de ton prochain~; tu ne convoiteras pas la femme de ton prochain, ni son serviteur, ni sa servante, ni son bœuf, ni son âne, ni aucune chose qui soit à ton prochain.
\TextTitle{Le peuple tout tremblant devant Yahweh}
\VS{18}Or tout le peuple apercevait les tonnerres, les éclairs, le son du shofar, et la montagne fumante. Et le peuple voyant cela tremblait et se tenait loin.
\VS{19}Et ils dirent à Moïse~: Parle, toi, avec nous, et nous écouterons~; mais que Dieu ne parle point avec nous, de peur que nous ne mourrions\FTNT{De. 5:23-24~; Hé. 12:18-19.}.
\VS{20}Et Moïse dit au peuple~: Ne craignez point car Dieu est venu pour vous éprouver, et afin que sa crainte soit devant vous, et que vous ne péchiez point.
\VS{21}Le peuple donc se tint loin, mais Moïse s'approcha de l'obscurité dans laquelle Dieu était.
\VS{22}Et Yahweh dit à Moïse~: Tu diras ainsi aux enfants d'Israël~: Vous avez vu que je vous ai parlé des cieux.
\VS{23}Vous ne vous ferez point avec moi de dieux d'argent ni de dieux d'or.
\VS{24}Tu me feras un autel de terre, sur lequel tu sacrifieras tes holocaustes, et tes offrandes de paix\FTNT{Voir commentaire en Lé. 3:1.}, ton menu et ton gros bétail. En quelque lieu que ce soit où je mettrai la mémoire de mon Nom, je viendrai là à toi, et je te bénirai.
\VS{25}Si tu me fais un autel de pierres, ne les taille point~; car si tu fais passer le fer dessus, tu le souillerais.
\VS{26}Et tu ne monteras point à mon autel par des marches, de peur que ta nudité ne soit découverte en y montant.
\Chap{21}
\TextTitle{Lois sur les maîtres et leurs esclaves}
\VerseOne{}Ce sont ici les lois que tu leur proposeras.
\VS{2}Si tu achètes un esclave Hébreu, il te servira six ans, et au septième il sortira pour être libre, sans rien payer\FTNT{Lé. 25:39-43~; De. 15:12~; Jé. 34:14.}.
\VS{3}S'il est venu avec son corps seulement, il sortira avec son corps~; s'il avait une femme, sa femme sortira aussi avec lui.
\VS{4}Si son maître lui a donné une femme qui lui ait enfanté des fils ou des filles, sa femme et les enfants qu'il en aura seront à son maître, mais il sortira avec son corps.
\VS{5}Si l'esclave dit positivement: J'aime mon maître, ma femme, et mes fils, je ne sortirai point pour être libre.
\VS{6}Alors son maître le fera venir devant les juges, et le fera approcher de la porte ou du poteau, et son maître lui percera l'oreille avec un poinçon~; et il le servira pour toujours.
\VS{7}Si quelqu'un vend sa fille pour être esclave, elle ne sortira point comme les esclaves sortent.
\VS{8}Si elle déplaît à son maître, qui ne l'aura point fiancée, il la fera acheter~; mais il n'aura pas le pouvoir de la vendre à un peuple étranger, après lui avoir été infidèle.
\VS{9}Mais s'il l'a fiancée à son fils, il fera pour elle selon le droit des filles.
\VS{10}S'il en prend une pour lui, il ne retranchera rien de sa nourriture, de ses vêtements et du droit conjugal.
\VS{11}S'il ne fait pas pour elle ces trois choses-là, elle sortira sans payer aucun argent.
\TextTitle{Lois sur les dommages corporels}
\VS{12}Si quelqu'un frappe un homme et qu'il en meure,on le fera mourir de mort \FTNT{Lé. 24:17~; No. 35:11-16~; De. 19:2-11~; Jos. 20:2.}.
\VS{13}S'il ne lui a point dressé d'embûches, mais que Dieu l'ait fait tomber entre ses mains, je t'établirai un lieu où il s'enfuira.
\VS{14}Mais si quelqu'un s'élève de propos délibéré contre son prochain, pour le tuer par ruse, tu le tireras de mon autel, afin qu'il meure.
\VS{15}Celui qui aura frappé son père ou sa mère sera puni de mort\FTNT{Lé. 20:9~; De. 27:16~; Mt. 15:4.}.
\VS{16}Si quelqu'un dérobe un homme et le vend, ou s'il est trouvé entre ses mains, on le fera mourir de mort.
\VS{17}Celui qui aura maudit son père ou sa mère sera puni de mort.
\VS{18}Si quelques uns ont une querelle, et que l'un ait frappé l'autre d'une pierre ou du poing, sans causer sa mort, mais qu'il soit obligé de se mettre au lit,
\VS{19}s'il se lève et marche dehors en s'appuyant sur son bâton, celui qui l'aura frappé sera absous~; toutefois, il le dédommagera de ce qu'il a chômé et le fera guérir entièrement.
\VS{20}Si quelqu'un a frappé du bâton son serviteur ou sa servante, et qu'il soit mort sous sa main, on ne manquera point de le venger.
\VS{21}Mais s'il survit un jour ou deux, il ne sera point vengé, car c'est son argent.
\VS{22}Si des hommes se querellent, et que l'un d'eux frappe une femme enceinte, et qu'elle en accouche, s'il n'y a pas cas de mort, il sera condamné à l'amende que le mari de la femme lui imposera, et il la donnera selon que les juges en ordonneront.
\VS{23}Mais s'il y a cas de mort, tu donneras vie pour vie,
\VS{24}œil pour œil, dent pour dent, main pour main, pied pour pied\FTNT{Lé. 24:20~; De. 19:21~; Mt. 5:38.},
\VS{25}brûlure pour brûlure, plaie pour plaie, meurtrissure pour meurtrissure.
\VS{26}Si quelqu'un frappe l'œil de son serviteur, ou l'œil de sa servante, et lui gâte l'œil, il le laissera aller libre pour son œil~;
\VS{27}et s'il fait tomber une dent à son serviteur, ou à sa servante, il le laissera aller libre pour sa dent.
\VS{28}Si un bœuf heurte de sa corne un homme ou une femme, et que la personne en meure, le bœuf sera lapidé sans nulle exception, et on ne mangera point de sa chair, mais le maître du bœuf sera absous.
\VS{29}Si le bœuf était auparavant sujet à frapper de sa corne, et que son maître en ait été averti avec protestation, et qu'il ne l'ait point surveillé, s'il tue un homme ou une femme, le bœuf sera lapidé, et on fera aussi mourir son maître.
\VS{30}Si on lui impose un prix pour se racheter, il donnera la rançon de sa vie, selon tout ce qui lui sera imposé.
\VS{31}Si le bœuf heurte de sa corne un fils ou une fille, il lui sera fait selon cette même loi.
\VS{32}Si le bœuf heurte de sa corne un esclave, soit homme, soit femme, celui à qui est le bœuf donnera trente sicles d'argent au maître de l'esclave, et le bœuf sera lapidé.
\VS{33}Si quelqu'un découvre une fosse, ou si quelqu'un creuse une fosse, et ne la couvre point, et qu'il y tombe un bœuf ou un âne,
\VS{34}le maître de la fosse donnera satisfaction, et rendra l'argent au maître du bœuf, mais la bête morte lui appartiendra.
\VS{35}Et si le bœuf de quelqu'un blesse le bœuf de son prochain, et qu'il en meure, ils vendront le bœuf vivant, et en partageront l'argent par moitié, ils partageront aussi par moitié le bœuf mort.
\VS{36}Mais s'il est connu que le bœuf avait auparavant l'habitude de heurter avec sa corne, et que le maître ne l'ait point gardé, il restituera bœuf pour bœuf~; mais le bœuf mort sera pour lui.
\Chap{22}
\TextTitle{Lois sur les torts causés à autrui}
\VerseOne{}Si quelqu'un dérobe un bœuf, ou un chevreau, ou un agneau, et qu'il le tue, ou le vende, il restituera cinq bœufs pour le bœuf, et quatre agneaux ou chevreaux pour l'agneau ou pour le chevreau.
\VS{2}Si le voleur est trouvé dérobant avec effraction, et est frappé de sorte qu'il en meure, celui qui l'aura frappé ne sera point coupable de meurtre.
\VS{3}Mais si le soleil est levé sur lui, il sera coupable de meurtre. Il fera donc une entière restitution~; et s'il n'a pas de quoi, il sera vendu pour son vol.
\VS{4}Si ce qui a été dérobé est trouvé vivant entre ses mains, soit bœuf, soit âne, soit brebis ou chèvre, il rendra le double.
\VS{5}Si quelqu'un fait brouter dans un champ ou dans une vigne, en lâchant son bétail qui aille paître dans le champ d'autrui, il rendra le meilleur de son champ et le meilleur de sa vigne.
\VS{6}Si un feu éclate et rencontre des épines, et que le blé qui est en tas, ou sur pied, ou le champ, soit consumé, celui qui aura allumé le feu rendra entièrement ce qui en aura été brûlé.
\VS{7}Si quelqu'un donne à son prochain de l'argent ou des vases à garder, et qu'on le dérobe de sa maison, et si l'on trouve le voleur, il rendra le double\FTNT{Lé. 5:20-26.}.
\VS{8}Mais si on ne trouve point le voleur, on fera venir le maître de la maison devant les juges pour jurer s'il n'a point mis sa main sur le bien de son prochain.
\VS{9}Dans toute affaire d'infidélité concernant un bœuf, un âne, une brebis, une chèvre, un vêtement ou tout objet perdu, dont quelqu'un dira qu'il lui appartient, la cause des deux parties viendra devant les juges~; et celui que les juges auront condamné, rendra le double à son prochain.
\VS{10}Si quelqu'un donne à garder à son prochain un âne, un bœuf, quelque menue ou grosse bête, et qu'elle meure, ou qu'elle se soit cassée quelque membre, ou qu'on l'ait emmenée sans que personne l'ait vue,
\VS{11}le serment de Yahweh interviendra entre les deux parties\FTNT{Hé. 6:16.}, pour savoir s'il n'a point mis sa main sur le bien de son prochain, et le maître de la bête se contentera du serment, et l'autre ne la rendra point.
\VS{12}Mais s'il est vrai qu'elle lui a été dérobée, il la rendra à son maître.
\VS{13}S'il est vrai qu'elle ait été déchirée par les bêtes sauvages, il la produira en témoignage, et il ne rendra point ce qui a été déchiré.
\VS{14}Si quelqu'un a emprunté de son prochain quelque bête, et qu'elle se casse quelque membre, ou qu'elle meure, son maître n'étant point présent, il ne manquera pas de la rendre.
\VS{15}Mais si son maître est avec lui, il ne la rendra point~; si elle a été louée, on payera seulement son louage.
\TextTitle{Lois diverses}
\VS{16}Si un homme séduit une vierge non fiancée, et couche avec elle, il faut qu'il la dote, et qu'il la prenne pour femme\FTNT{De. 22:28.}.
\VS{17}Mais si le père de la fille refuse absolument de la lui donner, il lui comptera autant d'argent qu'on en donne pour la dot des vierges.
\VS{18}Tu ne laisseras point vivre la sorcière\FTNT{De. 18:10-11~; Lé. 20:27.}.
\VS{19}Celui qui couche avec une bête sera puni de mort\FTNT{Lé. 18:23~; Lé. 20:15~; De. 27:21.}.
\VS{20}Celui qui sacrifie à d'autres dieux qu'à Yahweh seul sera dévoué à la façon de l'interdit\FTNT{Lé. 17:7~; De. 13:6-16~; De. 17:2-5.}.
\VS{21}Tu ne fouleras ni n'opprimeras point l'étranger~; car vous avez été étrangers au pays d'Egypte\FTNT{Lé. 19:34.}.
\VS{22}Vous n'affligerez point la veuve ni l'orphelin\FTNT{De. 24:17-18~; Za. 7:10.}.
\VS{23}Si vous les affligez en quoi que ce soit, et qu'ils crient à moi, certainement j'entendrai leur cri.
\VS{24}Et ma colère s'embrasera, et je vous ferai mourir par l'épée~; et vos femmes seront veuves, et vos fils orphelins.
\VS{25}Si tu prêtes de l'argent à mon peuple, au pauvre qui est avec toi, tu ne te comporteras point avec lui en créancier, vous ne lui exigerez point d'intérêt.
\VS{26}Si tu prends en gage le vêtement de ton prochain, tu le lui rendras avant que le soleil soit couché\FTNT{De. 24:10-13.}.
\VS{27}Car c'est sa seule couverture, c'est son vêtement pour couvrir sa peau~; où coucherait-il~? S'il arrive donc qu'il crie à moi, je l'entendrai~; car je suis miséricordieux.
\VS{28}Tu ne maudiras point les juges, et tu ne maudiras point le prince de ton peuple\FTNT{Lé. 24:15-16.}.
\VS{29}Tu ne différeras point de m'offrir de ton abondance et de tes liqueurs~; tu me donneras le premier-né de tes fils\FTNT{Ex. 13:12-15~; De. 26:2-11.}.
\VS{30}Tu feras la même chose de ta vache, de ta brebis, et de ta chèvre. Il sera sept jours avec sa mère, et le huitième jour tu me le donneras.
\VS{31}Vous me serez saints, et vous ne mangerez point de la chair déchirée dans les champs, mais vous la jetterez aux chiens.
\Chap{23}
\TextTitle{Lois diverses (suite)}
\VerseOne{}Tu ne léveras point de faux bruit, et tu ne te joindras point au méchant pour être un faux témoin, afin que violence soit faite\FTNT{Ex. 20:16~; De. 19:16-21.}.
\VS{2}Tu ne suivras point la multitude pour faire le mal~; et tu ne témoigneras point dans un procès en sorte que tu te détournes après un grand nombre pour pervertir le droit.
\VS{3}Tu n'honoreras point le pauvre dans son procès\FTNT{De. 1:17.}.
\VS{4}Si tu rencontres le bœuf de ton ennemi, ou son âne égaré, tu ne manqueras point de le lui ramener.
\VS{5}Si tu vois l'âne de celui qui te hait, abattu sous sa charge, tu t'arrêteras pour le secourir, et tu ne manqueras pas de l'aider.
\VS{6}Tu ne pervertiras point le droit de l'indigent, qui est au milieu de toi, dans son procès.
\VS{7}Tu t'éloigneras de toute parole fausse, et tu ne feras point mourir l'innocent et le juste~; car je ne justifierai point le méchant.
\VS{8}Tu ne prendras point de présent~; car le présent aveugle les plus éclairés, et pervertit les paroles des justes.
\VS{9}Tu n'opprimeras point l'étranger~; car vous savez ce que c'est que d'être étrangers, parce que vous avez été étrangers au pays d'Egypte.
\TextTitle{Le sabbat, le repos de la terre}
\VS{10}Pendant six ans tu ensemenceras ta terre, et en recueilleras le revenu.
\VS{11}Mais la septième année, tu lui donneras du relâche, et la laisseras reposer, afin que les pauvres de ton peuple en mangent, et que les bêtes des champs mangent ce qui restera. Tu en feras de même de ta vigne et de tes oliviers.
\VS{12}Tu travailleras six jours, mais tu te reposeras au septième jour, afin que ton bœuf et ton âne se reposent, et que le fils de ta servante et l'étranger reprennent courage.
\VS{13}Vous prendrez garde à toutes les choses que je vous ai ordonnées. Vous ne ferez point mention du nom des dieux étrangers, on ne l'entendra point de ta bouche\FTNT{Jos. 23:7~; Ps. 16:4.}.
\TextTitle{Les fêtes solennelles}
\VS{14}Trois fois l'an, tu me célébreras une fête solennelle\FTNT{Lé. 23:4-44.}.
\VS{15}Tu garderas la fête solennelle des pains sans levain\FTNT{Ex. 29:2.}~; tu mangeras des pains sans levain pendant sept jours, comme je t'ai ordonné, en la saison et au mois où les épis mûrissent~; car c'est en ce mois-là que tu es sorti d'Egypte~; et nul ne se présentera devant ma face à vide.
\VS{16}Et la fête solennelle de la moisson des premiers fruits de ton travail, de ce que tu auras semé au champ~; et la fête de la récolte, après la fin de l'année, quand tu auras recueilli du champ les fruits de ton travail\FTNT{Ex. 34:22.}.
\VS{17}Trois fois l'an, tous les mâles d'entre vous se présenteront devant le Seigneur Yahweh.
\VS{18}Tu ne sacrifieras point le sang de mon sacrifice avec du pain levé~; et la graisse de ma fête solennelle ne passera point la nuit jusqu'au matin\FTNT{Ex. 34:25-26.}.
\VS{19}Tu apporteras dans la maison de Yahweh, ton Dieu, les prémices des premiers fruits de ta terre. Tu ne feras point cuire le chevreau dans le lait de sa mère.
\TextTitle{Mises en garde et promesses de Yahweh}
\VS{20}Voici, j'envoie un Ange devant toi, afin qu'il te garde dans le chemin, et qu'il t'introduise dans le lieu que je t'ai préparé.
\VS{21}Garde-toi de provoquer sa colère, et écoute sa voix, et ne l'irrite point, car il ne pardonnera point votre péché~; car mon Nom est en lui.
\VS{22}Mais si tu écoutes attentivement sa voix, et si tu fais tout ce que je te dirai, je serai l'ennemi de tes ennemis, et j'affligerai ceux qui t'affligeront.
\VS{23}Car mon Ange marchera devant toi, et t'introduira au pays des Amoréens, des Héthiens, des Phéréziens, des Cananéens, des Héviens, et des Jébusiens, et je les exterminerai.
\VS{24}Tu ne te prosterneras point devant leurs dieux, et tu ne les serviras point, et tu ne feras point selon leurs œuvres, mais tu les détruiras entièrement, et tu briseras entièrement leurs statues\FTNT{Ex. 20:5~; Ex. 34:13~; No. 33:52.}.
\VS{25}Vous servirez Yahweh, votre Dieu. Et il bénira ton pain et tes eaux~; et j'ôterai les maladies du milieu de toi\FTNT{Ex. 15:26~; De. 6:13~; De. 7:15-16~; Mt. 4:10.}.
\VS{26}Il n'y aura point dans ton pays de femme qui avorte, ou qui soit stérile~; j'accomplirai le nombre de tes jours.
\VS{27}J'enverrai la terreur de mon Nom devant toi, et j'effrayerai tout peuple vers lequel tu arriveras, et je ferai que tous tes ennemis tourneront le dos devant toi\FTNT{De. 7:23.}.
\VS{28}Et j'enverrai des frelons devant toi, qui chasseront les Héviens, les Cananéens, et les Héthiens, de devant ta face\FTNT{De. 7:20~; Jos. 24:12.}.
\VS{29}Je ne les chasserai point loin de devant ta face en une année, de peur que le pays ne devienne un désert, et que les bêtes des champs ne se multiplient contre toi.
\VS{30}Mais je les chasserai peu à peu loin de devant toi, jusqu'à ce que tu te sois accru, et que tu possèdes le pays.
\VS{31}Et je mettrai des bornes depuis la Mer Rouge jusqu'à la mer des Philistins, et depuis le désert jusqu'au fleuve~; car je livrerai entre tes mains les habitants du pays et je les chasserai de devant toi.
\VS{32}Tu ne traiteras point d'alliance avec eux ni avec leurs dieux.
\VS{33}Ils n'habiteront point dans ton pays, de peur qu'ils ne te fassent pécher contre moi~; car tu servirais leurs dieux, et ce serait un piège pour toi.
\Chap{24}
\TextTitle{La loi lue au peuple~; le sang de l'Alliance}
\VerseOne{}Puis il dit à Moïse~: Monte vers Yahweh, toi et Aaron, Nadab et Abihu, et soixante-dix des anciens d'Israël, et vous vous prosternerez de loin.
\VS{2}Et Moïse s'approchera seul de Yahweh, mais eux ne s'en approcheront point, et le peuple ne montera point avec lui.
\VS{3}Alors Moïse vint, et récita au peuple toutes les paroles de Yahweh, et toutes ses lois, et tout le peuple répondit d'une voix, et dit~: Nous ferons toutes les choses que Yahweh a dites.
\VS{4}Or Moïse écrivit toutes les paroles de Yahweh, et s'étant levé de bon matin, il bâtit un autel au bas de la montagne, et dressa pour monument douze pierres pour les douze tribus d'Israël.
\VS{5}Et il envoya des jeunes hommes, des enfants d'Israël, qui offrirent des holocaustes et qui sacrifièrent des veaux à Yahweh en sacrifice d'offrande de paix.
\VS{6}Et Moïse prit la moitié du sang, et le mit dans des bassins, et répandit l'autre moitié sur l'autel.
\VS{7}Ensuite, il prit le livre de l'Alliance et le lut, et le peuple qui l'écoutait dit~: Nous ferons tout ce que Yahweh a dit, et nous obéirons.
\VS{8}Moïse donc prit le sang, et le répandit sur le peuple, en disant~: Voici le sang de l'Alliance que Yahweh a traitée avec vous, selon toutes ces paroles\FTNT{Mt. 26:28~; Mc. 14:24~; Lu. 22:20~; 1 Co. 11:25~; Hé. 9:20.}.
\TextTitle{Yahweh fait monter Moïse sur la montagne}
\VS{9}Puis Moïse, Aaron, Nadab, Abihu, et les soixante-dix anciens d'Israël montèrent.
\VS{10}Et ils virent le Dieu d'Israël, et sous ses pieds comme un ouvrage de saphir transparent, comme le ciel dans toute sa pureté.
\VS{11}Et il ne mit point sa main sur ceux qui avaient été choisis d'entre les enfants d'Israël~; ainsi, ils virent Dieu, et ils mangèrent et burent.
\VS{12}Et Yahweh dit à Moïse~: Monte vers moi sur la montagne, et demeure là~; et je te donnerai des tables de pierre, la loi et les commandements que j'ai écrits pour les enseigner.
\VS{13}Alors Moïse se leva avec Josué qui le servait~; et Moïse monta sur la montagne de Dieu.
\VS{14}Et il dit aux anciens d'Israël~: Demeurez ici en nous attendant jusqu'à ce que nous retournions vers vous. Et voici, Aaron et Hur seront avec vous~; quiconque aura quelque affaire, qu'il s'adresse à eux.
\VS{15}Moïse donc monta sur la montagne, et une nuée couvrit la montagne\FTNT{Ex. 19:9-16.}.
\VS{16}Et la gloire de Yahweh demeura sur la montagne de Sinaï, et la nuée la couvrit pendant six jours. Et au septième jour, il appela Moïse du milieu de la nuée.
\VS{17}Et ce qu'on voyait de la gloire de Yahweh au sommet de la montagne, était comme un feu dévorant aux yeux des enfants d'Israël\FTNT{De. 4:24~; De. 9:3~; Hé. 12:29.}.
\VS{18}Et Moïse entra dans la nuée et monta sur la montagne. Moïse fut sur la montagne quarante jours et quarante nuits.
\Chap{25}
\TextTitle{Des offrandes volontaires pour les matériaux du tabernacle}
\VerseOne{}Et Yahweh parla à Moïse, en disant~:
\VS{2}Parle aux enfants d'Israël, et qu'on prenne une offrande pour moi. Vous prendrez mon offrande de tout homme dont le cœur me l'offrira volontairement.
\VS{3}Et c'est ici l'offrande que vous prendrez d'eux~: De l'or, de l'argent, de l'airain,
\VS{4}de la pourpre, de l'écarlate, du cramoisi\FTNT{La couleur cramoisi s'obtient grâce à la femelle cochenille aptère qui contient dans son corps et dans ses œufs un pigment rouge à base d'acide carminique qui permet à l'insecte et à ses larves de se protéger des prédateurs. Au moment de la ponte, cette dernière fixe fermement son corps au tronc d'un arbre puis libère ses œufs qui demeurent ainsi protégés en dessous d'elle jusqu'à leur éclosion. Ensuite, l'insecte meurt en libérant cette substance rouge qui se propage sur tout son corps et sur le bois hôte. C'est ce fluide que l'homme récupère pour en faire un colorant à la couleur caractéristique. Une subtile analogie peut être faire entre la cochenille et le Seigneur qui a versé son sang à la croix pour nous donner la vie. «~Et moi, je suis un ver, et non un homme, l'opprobre des hommes et le méprisé du peuple~» (Ps. 22~:7).}, du fin lin, du poil de chèvre,
\VS{5}des peaux de béliers teintes en rouge, des peaux de taissons\FTNT{Le mot hébreu employé ici est «~tachash~», il désigne le matériau servant à fabriquer la couverture extérieure de la tente d'assignation. Si tout le monde s'accorde pour dire qu'il s'agissait d'une fourrure ou d'une peau d'animal, un doute subsiste sur la race exacte de l'animal. On hésite entre le marsouin, le dauphin, le blaireau (taisson) ou peut-être le mouton. Dans de nombreuses bibles, on a pris le parti de traduire par «~peaux de dauphins~». Cette hypothèse est cependant très peu probable. D'une part parce que le dauphin n'a pas de fourrure~; d'autre part parce que sa peau n'est absolument pas adaptée à la vie terrestre. Elle est donc impossible à conserver et à transformer, en particulier dans le contexte d'un climat propre au désert. Certains pensent qu'il s'agit tout simplement de peaux de béliers. Dans ce cas, comment expliquer qu'on n'ait pas employé le terme «~'ayil~» comme cela est mentionné pour les peaux teintes en rouge~? Il reste donc les peaux de taissons, c'est-à-dire de blaireaux, dont la fourrure est utilisée depuis des siècles. On peut objecter qu'il est impossible que le Seigneur puisse accepter la peau d'un animal impur pour construire son sanctuaire. Si tel était le cas, la peau du dauphin, qui est également un animal impur, n'aurait pas non-plus été autorisée. Toutefois, d'un point de vue prophétique, le symbole est important~: La présence de cet animal impur préfigurait Christ qui a pris une chair semblable à celle du péché (Ro. 8:3) mais aussi le levain que l'on peut trouver dans la pate nouvelle (1 Co. 5:6-7), l'ivraie qui se glisse parmi le blé (Mt. 13:25-43).}, du bois d'acacia,
\VS{6}de l'huile pour le luminaire, des aromates pour l'huile d'onction et pour le parfum odoriférant,
\VS{7}des pierres d'onyx, et d'autres pierres pour la garniture de l'éphod et pour le pectoral.
\VS{8}Et ils me feront un sanctuaire, et j'habiterai au milieu d'eux\FTNT{Ex. 29:45-46.}.
\VS{9}Ils le feront conformément à tout ce que je vais te montrer, selon le modèle du tabernacle et selon le modèle de tous ses ustensiles~; vous le ferez donc ainsi.
\TextTitle{L'arche de l'alliance}
\VS{10}Et ils feront une arche de bois d'acacia~; et sa longueur sera de deux coudées et demie, et sa largeur d'une coudée et demie, et sa hauteur d'une coudée et demie.
\VS{11}Et tu la couvriras d'or pur, tu l'en couvriras en dehors et en dedans~; et tu feras sur elle un couronnement d'or tout autour\FTNT{Ex. 37:1-9.}.
\VS{12}Et tu fondras pour elle quatre anneaux d'or, que tu mettras à ses quatre coins, deux anneaux à l'un de ses côtés, et deux autres de l'autre côté.
\VS{13}Tu feras aussi des barres de bois d'acacia, et tu les couvriras d'or.
\VS{14}Puis tu feras entrer les barres dans les anneaux aux côtés de l'arche, pour porter l'arche avec elles.
\VS{15}Les barres seront dans les anneaux de l'arche, et on ne les en tirera point.
\VS{16}Et tu mettras dans l'arche le témoignage que je te donnerai\FTNT{Hé. 9:4.}.
\VS{17}Tu feras aussi un propitiatoire d'or pur, dont la longueur sera de deux coudées et demie, et la largeur d'une coudée et demie.
\VS{18}Et tu feras deux chérubins d'or~; tu les feras d'ouvrage étendu au marteau, tirés des deux extrémités du propitiatoire.
\VS{19}Fais donc un chérubin tiré des extrémités et un chérubin tiré de l'autre extrémité~; vous ferez les chérubins tirés du propitiatoire à ses deux extrémités.
\VS{20}Et les chérubins étendront les ailes en haut, couvrant de leurs ailes le propitiatoire, et leurs faces seront vis-à-vis l'une de l'autre~; et le regard des chérubins sera vers le propitiatoire\FTNT{1 R. 8:6-7~; Hé. 9:5.}.
\VS{21}Et tu poseras le propitiatoire au-dessus de l'arche, et tu mettras dans l'arche le témoignage que je te donnerai.
\VS{22}Et je me rencontrerai là avec toi, et je te dirai de dessus le propitiatoire, d'entre les deux chérubins qui seront sur l'arche du témoignage, toutes les choses que je t'ordonnerai pour les enfants d'Israël\FTNT{Ex. 29:42-43~; No. 7:89.}.
\TextTitle{La table des pains de proposition}
\VS{23}Tu feras aussi une table de bois d'acacia. Sa longueur sera de deux coudées, et sa largeur d'une coudée, et sa hauteur d'une coudée et demie.
\VS{24}Tu la couvriras d'or pur, et tu lui feras un couronnement d'or tout autour.
\VS{25}Tu lui feras aussi à l'entour une clôture d'une largeur de main, et tout autour de sa clôture tu feras un couronnement d'or.
\VS{26}Tu lui feras aussi quatre anneaux d'or que tu mettras aux quatre coins qui seront à ses quatre pieds.
\VS{27}Les anneaux seront à l'endroit de la clôture, afin d'y mettre les barres pour porter la table.
\VS{28}Tu feras les barres de bois d'acacia, et tu les couvriras d'or, et on portera la table avec elles.
\VS{29}Tu feras aussi ses plats, ses tasses, ses gobelets, et ses bassins, avec lesquels on fera les aspersions~; tu les feras d'or pur\FTNT{Ex. 37:10-16.}.
\VS{30}Et tu mettras sur cette table le pain de proposition continuellement devant moi\FTNT{Lé. 24:5-9.}.
\TextTitle{Le chandelier d'or pur}
\VS{31}Tu feras aussi un chandelier d'or pur\FTNT{Le chandelier avait une double symbolique. D'une part, il préfigurait Jésus-Christ, notre Lumière (Jn. 1:4-5~; Jn. 8:12). Les sept lampes évoquaient l'omniscience de l'Esprit de Jésus-Christ (Za. 3:9~; Jn. 16:29-30~; Ap. 1:4~; Ap. 3:1~; Ap. 4:5~; Ap. 5:6). Il est à noter que ce chandelier comportait des calices en forme de fleurs, de pommes et d'amandes (Ex. 25:33) qui symbolisaient les fruits de l'Esprit que nous devons nécessairement porter (Ga. 5:22). D'autre part, il est une image de l'Eglise (Ap. 1:20). Voir commentaire Ex. 37:17-24~; Es. 8:13-17.}. Le chandelier sera étendu au marteau~; son pied, sa tige et ses branches, ses plats, ses pommeaux et ses fleurs seront tirés de lui.
\VS{32}Six branches sortiront de ses côtés~: Trois branches d'un côté du chandelier, et trois autres de l'autre côté du chandelier.
\VS{33}Il y aura sur l'une des branches trois petits plats en forme d'amande, un pommeau et une fleur~; sur l'autre branche trois petits plats en forme d'amande, un pommeau et une fleur~; il en sera de même des six branches sortant du chandelier.
\VS{34}ll y aura aussi au chandelier quatre petits plats en forme d'amande, ses pommeaux et ses fleurs.
\VS{35}Un pommeau sous deux branches tirées du chandelier, un pommeau sous deux autres branches tirées de lui, et un pommeau sous deux autres branches tirées de lui~; il en sera de même des six branches sortant du chandelier.
\VS{36}Leurs pommeaux et leurs branches seront tirés de lui, et tout le chandelier sera un seul ouvrage étendu au marteau, et d'or pur.
\VS{37}Tu feras aussi ses sept lampes, et on les allumera afin qu'elles éclairent vis-à-vis du chandelier.
\VS{38}Et ses mouchettes et ses encensoirs destinés à recevoir ce qui tombe des lampes seront d'or pur.
\VS{39}On le fera avec tous ses ustensiles d'un talent d'or pur.
\VS{40}Regarde donc, et fais selon le modèle qui t'est montré sur la montagne.
\Chap{26}
\TextTitle{Les tapis de fin lin}
\VerseOne{}Tu feras aussi le tabernacle de dix tapis de fin lin retors, de pourpre, d'écarlate, et de cramoisi~; et tu les feras semés de chérubins d'un ouvrage exquis\FTNT{Ex. 36:8-38.}.
\VS{2}La longueur d'un tapis sera de vingt-huit coudées, et la largeur du même tapis de quatre coudées~; tous les tapis auront une même mesure.
\VS{3}Cinq de ces tapis seront joints l'un à l'autre, et les cinq autres seront aussi joints l'un à l'autre.
\VS{4}Fais aussi des lacets de pourpre sur le bord d'un tapis, au bord du premier assemblage~; et tu feras la même chose au bord du dernier tapis dans l'autre assemblage.
\VS{5}Tu feras donc cinquante lacets au premier tapis, et tu feras cinquante lacets au bord du tapis qui est dans le second assemblage. Les lacets seront vis-à-vis l'un de l'autre.
\VS{6}Tu feras aussi cinquante crochets d'or, et tu attacheras les tapis l'un à l'autre avec les crochets~; ainsi le tabernacle ne fera qu'un.
\TextTitle{Les tapis de poils de chèvre}
\VS{7}Tu feras aussi des tapis de poils de chèvres pour servir de tente sur le tabernacle~; tu feras onze de ces tapis.
\VS{8}La longueur d'un tapis sera de trente coudées, et la largeur du même tapis sera de quatre coudées~; les onze tapis auront une même mesure.
\VS{9}Puis tu joindras séparément cinq de ces tapis, et les six tapis à part~; mais tu redoubleras le sixième tapis sur le devant du tabernacle.
\VS{10}Tu feras aussi cinquante lacets sur le bord de l'un des tapis, à savoir au dernier qui est assemblé, et cinquante lacets au bord du tapis du second assemblage.
\VS{11}Tu feras aussi cinquante crochets d'airain, et tu feras entrer les crochets dans les lacets~; et tu assembleras ainsi la tente qui fera un tout.
\VS{12}Mais ce qu'il y aura en surplus dans les tapis de la tente, à savoir la moitié du tapis de reste, retombera sur le derrière du tabernacle.
\VS{13}La coudée d'une part, et la coudée d'autre part, qui seront de reste sur la longueur des tapis de la tente, retomberont sur les deux côtés du tabernacle, pour le couvrir.
\TextTitle{Les couvertures de peaux de béliers}
\VS{14}Tu feras aussi pour ce tabernacle une couverture de peaux de béliers teintes en rouge, et une couverture de peaux de taissons par-dessus\FTNT{Ex. 35:7~; Ex. 35:23~; Ex. 36:19~; Ex. 39:34.}.
\TextTitle{Les planches et leurs bases}
\VS{15}Et tu feras pour le tabernacle, des planches de bois d'acacia, qu'on fera tenir debout\FTNT{Ex. 36:20-34.}.
\VS{16}La longueur d'une planche sera de dix coudées, et la largeur d'une même planche d'une coudée et demie.
\VS{17}Il y aura à chaque planche deux tenons joints l'un à l'autre~; et tu feras de même pour toutes les planches du tabernacle.
\VS{18}Tu feras donc les planches du tabernacle, à savoir vingt planches qui regardent vers le midi.
\VS{19}Et au-dessous des vingt planches, tu feras quarante bases d'argent~; deux bases sous une planche pour ses deux tenons, et deux bases sous l'autre planche pour ses deux tenons.
\VS{20}Et vingt planches de l'autre côté du tabernacle, du coté nord.
\VS{21}Et leurs quarante bases seront d'argent, deux bases sous une planche, et deux bases sous l'autre planche.
\VS{22}Et pour le fond du tabernacle, vers l'occident, tu feras six planches.
\VS{23}Tu feras aussi deux planches pour les angles du tabernacle, aux deux cotés du fond.
\VS{24}Et elles seront égales par le bas, et elles seront jointes et unies par le haut avec un anneau~; il en sera de même des deux planches qui seront aux deux angles.
\VS{25}Il y aura donc huit planches, et seize bases d'argent~; deux bases sous une planche et deux bases sous une autre planche.
\VS{26}Après cela, tu feras cinq barres de bois d'acacia, pour les planches d'un des côtés du tabernacle.
\VS{27}Pareillement, tu feras cinq barres pour les planches de l'autre côté du tabernacle~; et cinq barres pour les planches du côté du tabernacle, pour le fond, vers le côté de l'occident.
\VS{28}Et la barre du milieu sera au milieu des planches d'une extrémité à l'autre.
\VS{29}Tu couvriras aussi d'or les planches, et tu feras d'or leurs anneaux pour mettre les barres, et tu couvriras d'or les barres.
\VS{30}Tu dresseras le tabernacle selon le modèle qui t'est montré sur la montagne.
\TextTitle{Les voiles intérieurs et extérieurs}
\VS{31}Et tu feras un voile\FTNT{Le voile intérieur symbolisait la chair de Jésus-Christ qui a été brisée à cause de nos péchés (Es. 53:5~; Hé. 10:20). Ex. 36:35-38~; Mt. 27:51~; Hé. 9:3.} de pourpre, d'écarlate, de cramoisi, et de fin lin retors~; on le fera d'ouvrage exquis, avec des chérubins.
\VS{32}Et tu le mettras sur quatre piliers de bois d'acacia couverts d'or, ayant leurs crochets d'or. Et ils seront sur quatre bases d'argent.
\VS{33}Puis tu mettras le voile sous les crochets, et tu feras entrer là-dedans, c'est-à-dire au-dedans du voile, l'arche du témoignage~; et ce voile vous fera la séparation entre le lieu saint et le Saint des saints.
\VS{34}Et tu poseras le propitiatoire sur l'arche du témoignage, dans le Saint des saints.
\VS{35}Et tu mettras la table au dehors de ce voile, et le chandelier vis-à-vis de la table, au côté du tabernacle, vers le sud~; et tu placeras la table côté nord.
\VS{36}Et à l'entrée du tabernacle, tu feras un rideau de pourpre, d'écarlate, de cramoisi et de fin lin retors, d'ouvrage de broderie.
\VS{37}Tu feras aussi pour ce rideau cinq piliers de bois d'acacia, que tu couvriras d'or, et leurs crochets seront d'or~; et tu fondras pour eux cinq bases d'airain.
\Chap{27}
\TextTitle{L'autel d'airain}
\VerseOne{}Tu feras aussi un autel de bois d'acacia, ayant cinq coudées de long, et cinq coudées de large~; l'autel sera carré, et sa hauteur sera de trois coudées.
\VS{2}Tu feras ses cornes à ses quatre coins~; ses cornes seront tirées de lui, et tu le couvriras d'airain\FTNT{C'est sur l'autel d'airain que les animaux étaient sacrifiés. Il préfigurait la croix et le jugement que Jésus-Christ a pris sur lui à notre place (Es. 53:5~; 2 Co. 13:4~; Ph. 2:8).}.
\VS{3}Tu feras ses chaudrons pour recevoir ses cendres, et ses racloirs, ses bassins, ses fourchettes, et ses encensoirs~; tu feras tous ses ustensiles d'airain.
\VS{4}Tu lui feras une grille d'airain en forme de treillis, et tu feras au treillis quatre anneaux d'airain à ses quatre coins.
\VS{5}Et tu le mettras au-dessous de l'enceinte de l'autel en bas, et le treillis s'étendra jusqu'au milieu de l'autel.
\VS{6}Tu feras aussi des barres pour l'autel, des barres de bois d'acacia, et tu les couvriras d'airain.
\VS{7}Et on fera passer ses barres dans les anneaux~; les barres seront aux deux côtés de l'autel pour le porter.
\VS{8}Tu le feras creux avec des planches~; ils le feront ainsi qu'il t'a été montré sur la montagne\FTNT{Ex. 38:1-7.}.
\TextTitle{Le parvis}
\VS{9}Tu feras aussi le parvis du tabernacle, du côté qui regarde vers le sud~; il y aura pour former le parvis, des courtines de fin lin retors~; la longueur de l'un des côtés sera de cent coudées.
\VS{10}Il y aura vingt piliers avec leurs vingt bases d'airain, mais les crochets des piliers et leurs filets seront d'argent.
\VS{11}Ainsi du côté nord, il y aura également des courtines sur une longueur de cent coudées, avec vingt piliers avec leurs vingt bases d'airain~; mais les crochets des piliers avec leurs filets seront d'argent.
\VS{12}La largeur du parvis du côté de l'occident sera de cinquante coudées de courtines, qui auront dix piliers, avec leurs dix bases.
\VS{13}Et la largeur du parvis du côté de l'orient, directement vers le levant, sera de cinquante coudées.
\VS{14}A l'un des côtés, il y aura quinze coudées de courtines, avec leurs trois piliers et leurs trois bases.
\VS{15}Et de l'autre côté, quinze coudées de courtines, avec leurs trois piliers et leurs trois bases.
\TextTitle{La porte du parvis}
\VS{16}Il y aura aussi pour la porte du parvis un rideau de vingt coudées, fait de pourpre, d'écarlate, de cramoisi, et de fin lin retors, ouvrage de broderie, avec quatre piliers et quatre bases.
\VS{17}Tous les piliers du parvis seront ceints d'un filet d'argent, et leurs crochets seront d'argent, mais leurs bases seront d'airain.
\VS{18}La longueur du parvis sera de cent coudées, et la largeur de cinquante, de chaque côté~; et la hauteur de cinq coudées. Il sera de fin lin retors, et les bases des piliers seront d'airain.
\VS{19}Que tous les ustensiles du tabernacle, pour tout son service, et tous ses pieux, avec les pieux du parvis, soient d'airain\FTNT{Ex. 38:9-20.}.
\TextTitle{L'huile d'olive vierge pour les lampes}
\VS{20}Tu ordonneras aux fils d'Israël qu'ils t'apportent de l'huile d'olive vierge pour le luminaire, afin de faire luire les lampes continuellement\FTNT{Ex. 35:8-28~; Lé. 24:1-4.}.
\VS{21}Aaron avec ses fils les prépareront dans la présence de Yahweh, depuis le soir jusqu'au matin, dans la tente d'assignation, hors du voile qui est devant le témoignage~; ce sera une ordonnance perpétuelle pour les enfants d'Israël.
\Chap{28}
\TextTitle{La prêtrise}
\VerseOne{}Et toi, fais approcher de toi Aaron, ton frère, et ses fils avec lui, d'entre les enfants d'Israël, pour m'exercer la prêtrise, à savoir Aaron, Nadab et Abihu, Eléazar et Ithamar, fils d'Aaron.
\VS{2}Et tu feras à Aaron, ton frère, de saints vêtements pour gloire et pour ornement.
\TextTitle{Les vêtements sacrés des prêtres}
\VS{3}Et tu parleras à tous les hommes d'esprit, à chacun de ceux que j'ai remplis de l'esprit de science, afin qu'ils fassent des vêtements à Aaron pour le sanctifier, afin qu'il m'exerce la prêtrise.
\VS{4}Et ce sont ici les vêtements qu'ils feront~: Le pectoral, l'éphod, la robe, la tunique brodée, la tiare, et la ceinture. Ils feront donc les saints vêtements à Aaron, ton frère, et à ses fils, pour m'exercer la prêtrise.
\VS{5}Et ils prendront de l'or, de la pourpre, de l'écarlate, du cramoisi, et du fin lin.
\TextTitle{L'éphod}
\VS{6}Et ils feront l'éphod d'or, de pourpre, d'écarlate et de cramoisi, et de fin lin retors~; d'un ouvrage exquis.
\VS{7}Il aura deux épaulettes qui se joindront par les deux bouts~; et c'est ainsi qu'il sera joint.
\VS{8}La ceinture exquise dont il sera ceint, et qui sera par-dessus, sera de même ouvrage, et tirée de lui, étant d'or, de pourpre, d'écarlate, de cramoisi, et de fin lin retors.
\VS{9}Et tu prendras deux pierres d'onyx, et tu graveras sur elles les noms des enfants d'Israël~:
\VS{10}Six de leurs noms sur une pierre et les six noms des autres sur l'autre pierre, selon leur naissance.
\VS{11}Tu graveras sur les deux pierres les noms des enfants d'Israël, comme on grave les pierres et les cachets, tu les entoureras de montures d'or.
\VS{12}Et tu mettras les deux pierres sur les épaulettes de l'éphod, afin qu'elles soient des pierres de souvenir pour les enfants d'Israël~; car Aaron portera leurs noms sur ses deux épaules devant Yahweh, pour souvenir.
\VS{13}Tu feras aussi des montures d'or,
\VS{14}et deux chaînettes d'or pur que tu tresseras en forme de cordons, et tu fixeras aux montures les chaînettes ainsi tressées.
\TextTitle{Le pectoral}
\VS{15}Tu feras aussi le pectoral du jugement d'un ouvrage exquis, comme l'ouvrage de l'éphod, d'or, de pourpre, d'écarlate, de cramoisi, et de fin lin retors.
\VS{16}Il sera carré et double~; et sa longueur sera d'un empan, et sa largeur d'un empan.
\VS{17}Et tu le rempliras de garniture de pierres, à quatre rangées de pierres précieuses. A la première rangée, on mettra une sardoine, une topaze, et une émeraude.
\VS{18}Et à la seconde rangée, une escarboucle, un saphir, et un jaspe.
\VS{19}Et à la troisième rangée, une opale, une agate, et une améthyste.
\VS{20}Et à la quatrième rangée, un chrysolithe, un onyx et un béryl, qui seront enchâssés dans de l'or, selon leur garniture.
\VS{21}Et ces pierres-là seront selon les noms des enfants d'Israël, douze selon leurs noms, chacune d'elles gravées comme des cachets, selon le nom qu'elle doit porter, et elles seront pour les douze tribus.
\VS{22}Tu feras donc pour le pectoral des chaînettes d'or pur, tressées en forme de cordon.
\VS{23}Et tu feras sur le pectoral deux anneaux d'or, et tu mettras les deux anneaux aux deux bouts du pectoral.
\VS{24}Et tu mettras les deux chaînettes d'or, faites en cordon, dans les deux anneaux à l'extrémité du pectoral.
\VS{25}Et tu mettras les deux autres bouts des deux chaînettes en cordon sur les deux montures, et tu les mettras sur les épaulettes de l'éphod, sur le devant de l'éphod.
\VS{26}Tu feras aussi deux autres anneaux d'or, que tu mettras aux deux autres bouts du pectoral, sur le bord qui sera du côté de l'éphod à l'intérieur.
\VS{27}Et tu feras deux autres anneaux d'or, que tu mettras aux deux épaulettes de l'éphod par le bas, sur le devant, à l'endroit où il se joint, au-dessus de la ceinture exquise de l'éphod.
\VS{28}Et ils joindront le pectoral élevé par ses anneaux, aux anneaux de l'éphod, avec un cordon de pourpre, afin qu'il tienne au-dessus de la ceinture exquise de l'éphod, et que le pectoral ne puisse pas se séparer de l'éphod.
\VS{29}Ainsi, Aaron portera sur son cœur les noms des enfants d'Israël gravés sur le pectoral du jugement, quand il entrera dans le lieu saint, pour souvenir devant Yahweh continuellement.
\TextTitle{L'urim et le thummim}
\VS{30}Et tu mettras sur le pectoral de jugement l'urim et le thummim\FTNT{L'urim («~lumières~») et le thummim («~perfections~») étaient deux pierres du pectoral que l'on utilisait ensemble pour déterminer la décision de Dieu sur certaines questions.}, qui seront sur le cœur d'Aaron, quand il viendra devant Yahweh~; et Aaron portera le jugement des enfants d'Israël sur son cœur devant Yahweh continuellement.
\TextTitle{La robe de l'éphod}
\VS{31}Tu feras aussi la robe de l'éphod entièrement de pourpre.
\VS{32}Il y aura, au milieu, une ouverture pour la tête, et cette ouverture aura tout autour un bord tissé, comme l'ouverture d'une cotte de mailles, afin que la robe ne se déchire pas.
\VS{33}Tu feras à ses bords des grenades de pourpre, d'écarlate, et de cramoisi tout autour, et des clochettes d'or entre elles tout autour.
\VS{34}Une clochette d'or, puis une grenade, une clochette d'or, puis une grenade, aux bords de la robe tout autour.
\VS{35}Et Aaron en sera revêtu quand il fera le service, et on en entendra le son lorsqu'il entrera dans le lieu saint devant Yahweh, et quand il en sortira, afin qu'il ne meure pas.
\TextTitle{La lame d'or gravée~: La sainteté à Yahweh}
\VS{36}Et tu feras une lame d'or pur, sur laquelle tu graveras ces mots, comme on grave un cachet~: La sainteté à Yahweh.
\VS{37}Tu l'attacheras avec un cordon de pourpre sur la tiare, sur le devant de la tiare.
\VS{38}Et elle sera sur le front d'Aaron~; et Aaron portera l'iniquité commise par les enfants d'Israël, en faisant leurs saintes offrandes, elle sera continuellement sur son front devant Yahweh, pour qu'il leur soit favorable.
\TextTitle{Les vêtements de service d'Aaron et ses fils}
\VS{39}Tu feras aussi une tunique de fin lin qui s'appliquera sur le corps, et tu feras aussi la tiare de fin lin~; mais tu feras la ceinture d'ouvrage de broderie\FTNT{Ex. 39:1-32.}.
\VS{40}Tu feras aussi aux fils d'Aaron des tuniques, des ceintures, et des bonnets, pour leur gloire et leur ornement.
\VS{41}Et tu en revêtiras Aaron, ton frère, et ses fils avec lui~; tu les oindras, tu les consacreras et tu les sanctifieras~; puis ils exerceront la prêtrise pour moi\FTNT{Lé. 8:12~; Lé. 16:32~; No. 3:3.}.
\VS{42}Et tu leur feras des caleçons de lin, pour couvrir leur nudité, qui tiendront depuis les reins jusqu'au bas des cuisses.
\VS{43}Et Aaron et ses fils seront ainsi habillés quand ils entreront dans la tente d'assignation, ou quand ils approcheront de l'autel pour faire le service dans le lieu saint~; et ils ne porteront point la peine d'aucune iniquité, et ne mourront point. Ce sera une ordonnance perpétuelle pour lui et pour sa postérité après lui.
\Chap{29}
\TextTitle{Les prêtres consacrés au service de Yahweh}
\VerseOne{}Or c'est ici ce que tu leur feras, quand tu les sanctifieras pour exercer la prêtrise pour moi~: Prends un veau du troupeau, et deux béliers sans tare\FTNT{Lé. 8:2~; Lé. 9:2~; Hé. 7:26-28.}~;
\VS{2}et des pains sans levain, et des gâteaux sans levain pétris à l'huile, et des beignets sans levain, oints d'huile~; et tu les feras de fine farine de froment\FTNT{Lé. 6:13.}.
\VS{3}Tu les mettras dans une corbeille, et tu les présenteras dans la corbeille~; tu présenteras aussi le veau et les deux moutons.
\VS{4}Puis, tu feras approcher Aaron et ses fils à l'entrée de la tente d'assignation, et tu les laveras avec de l'eau\FTNT{Ex. 40:12.}.
\VS{5}Ensuite, tu prendras les vêtements, et tu feras vêtir à Aaron la tunique et la robe de l'éphod, l'éphod et pectoral, et tu le ceindras par-dessus avec la ceinture exquise de l'éphod.
\VS{6}Puis, tu mettras sur sa tête la tiare et la couronne de sainteté sur la tiare.
\VS{7}Et tu prendras l'huile d'onction et la répandras sur sa tête~; et tu l'oindras ainsi.
\VS{8}Puis, tu feras approcher ses fils, et tu leur feras vêtir les tuniques.
\VS{9}Et tu les ceindras des ceintures, Aaron, dis-je, et ses fils\FTNT{Es. 11:5~; Ep. 6:14.}, et tu leur attacheras des bonnets, et ils posséderont la prêtrise par ordonnance perpétuelle. Et tu consacreras ainsi Aaron et ses fils.
\VS{10}Et tu feras approcher le veau devant la tente d'assignation, et Aaron et ses fils poseront leurs mains sur la tête du veau.
\VS{11}Et tu égorgeras le veau devant Yahweh, à l'entrée de la tente d'assignation.
\VS{12}Puis, tu prendras du sang du veau, et le mettras avec ton doigt sur les cornes de l'autel, et tu répandras tout le reste du sang au pied de l'autel.
\VS{13}Tu prendras aussi toute la graisse qui couvre les entrailles, et le lobe du foie, les deux rognons, la graisse qui les entoure, et tu les feras fumer sur l'autel.
\VS{14}Mais tu brûleras au feu la chair du veau, sa peau, et ses excréments, hors du camp. C'est un sacrifice pour le péché\FTNT{Lé. 1:3-13~; Hé. 9:11~; Hé. 13:11.}.
\VS{15}Puis, tu prendras l'un des béliers, et Aaron et ses fils poseront leurs mains sur la tête du bélier.
\VS{16}Puis, tu égorgeras le bélier, et prenant son sang, tu le répandras sur l'autel tout autour.
\VS{17}Après, tu couperas le bélier par pièces, et ayant lavé ses entrailles et ses jambes, tu les mettras sur ses pièces et sur sa tête.
\VS{18} Et tu feras fumer tout le bélier sur l'autel~; c'est un holocauste à Yahweh, c'est un sacrifice consumé par le feu d'une agréable odeur à Yahweh.
\VS{19}Puis, tu prendras l'autre bélier, et Aaron et ses fils mettront leurs mains sur sa tête.
\VS{20}Et tu égorgeras le bélier, et prenant de son sang, tu le mettras sur le lobe de l'oreille droite d'Aaron, et sur le lobe de l'oreille droite de ses fils, sur le pouce de leur main droite, sur le gros orteil de leur pied droit, et tu répandras le reste du sang sur l'autel tout autour.
\VS{21}Et tu prendras du sang qui sera sur l'autel, de l'huile d'onction, et tu en feras l'aspersion sur Aaron, sur ses vêtements, sur ses fils, et sur les vêtements de ses fils avec lui. Ainsi, lui, ses vêtements, ses fils, et les vêtements de ses fils, seront sanctifiés avec lui.
\VS{22}Tu prendras aussi la graisse du bélier, la queue, et la graisse qui couvre les entrailles, le grand lobe du foie, les deux rognons, la graisse qui est dessus, et l'épaule droite~; car c'est le bélier de consécration.
\VS{23}Tu prendras aussi un pain, un gâteau à l'huile, et un beignet dans la corbeille où seront ces choses sans levain, laquelle sera devant Yahweh.
\VS{24}Et tu mettras toutes ces choses sur les mains d'Aaron et sur les mains de ses fils, et tu les agiteras de côté et d'autre devant Yahweh\FTNT{No. 6:19.}.
\VS{25}Puis, les recevant de leurs mains, tu les feras fumer sur l'autel, sur l'holocauste, pour être une odeur agréable devant Yahweh~; c'est un sacrifice consumé par le feu à Yahweh.
\TextTitle{La part des prêtres}
\VS{26}Tu prendras aussi la poitrine du bélier des consécrations, qui est pour Aaron, et tu l'agiteras de côté et d'autre en offrande agitée devant Yahweh. Ce sera ta part.
\VS{27}Tu sanctifieras donc la poitrine de l'offrande agitée, et l'épaule de l'offrande élevée, tant ce qui aura été agité que ce qui aura été élevé du bélier de consécration, de ce qui est pour Aaron et de ce qui est pour ses fils. \FTNT{Lé. 10:14~; No. 18:18.}.
\VS{28}Et ceci sera une ordonnance perpétuelle pour Aaron et pour ses fils, de ce qui sera offert par les enfants d'Israël~; car c'est une offrande élevée. Quand il y aura une offrande élevée de celles qui sont faites par les enfants d'Israël, de leurs offrandes de paix, leur offrande élevée sera à Yahweh.
\VS{29}Et les saints vêtements qui seront pour Aaron, seront pour ses fils après lui, afin qu'ils soient oints et consacrés dans ces vêtements.
\VS{30}Le prêtre qui succédera à sa place d'entre ses fils, et qui viendra à la tente d'assignation, pour faire le service dans le lieu saint, en sera revêtu durant sept jours.
\VS{31}Or tu prendras le bélier des consécrations, et tu feras bouillir sa chair dans un lieu saint~;
\VS{32}et Aaron et ses fils mangeront à l'entrée de la tente d'assignation la chair du bélier et le pain qui sera dans la corbeille.
\VS{33}Ils mangeront donc ces choses, par lesquelles la propitiation aura été faite, pour les consacrer et les sanctifier~; mais l'étranger n'en mangera point, parce qu'elles sont saintes.
\VS{34}S'il y a des restes de la chair des consécrations et du pain jusqu'au matin, tu brûleras ces restes-là au feu~; on n'en mangera point, parce que c'est une chose sainte.
\VS{35}Tu feras donc ainsi à Aaron et à ses fils, selon toutes les choses que je t'ai ordonnées~; tu les consacreras durant sept jours\FTNT{Lé. 8:31-35.}.
\VS{36}Et tu offriras comme sacrifice pour l'expiation tous les jours un veau pour faire l'expiation, et tu purifieras l'autel par cette propitiation, et tu l'oindras pour le sanctifier\FTNT{Ez. 43:19-20.}.
\VS{37}Pendant sept jours, tu feras propitiation pour l'autel, et tu le sanctifieras~; l'autel sera une chose très sainte~; tout ce qui touchera l'autel sera saint\FTNT{No. 28:3.}.
\TextTitle{L'holocauste perpétuel}
\VS{38}Or c'est ici ce que tu feras sur l'autel~: Tu offriras chaque jour continuellement deux agneaux d'un an.
\VS{39}Tu sacrifieras l'un des agneaux au matin, et l'autre agneau entre les deux soirs~;
\VS{40}avec un dixième de fine farine pétrie dans la quatrième partie d'un hin d'huile vierge, et avec une libation de vin de la quatrième partie d'un hin pour chaque agneau,
\VS{41}et tu sacrifieras l'autre agneau entre les deux soirs, avec un gâteau comme au matin, et tu lui feras la même libation, en bonne odeur~; c'est un sacrifice consumé par le feu à Yahweh.
\VS{42}Ce sera l'holocauste perpétuel qui sera offert en vos générations, à l'entrée de la tente d'assignation, devant Yahweh, où je me trouverai avec vous pour te parler.
\VS{43}Je me trouverai là pour les enfants d'Israël, et la tente sera sanctifiée par ma gloire.
\VS{44}Je sanctifierai donc la tente d'assignation et l'autel. Je sanctifierai aussi Aaron et ses fils, afin qu'ils exercent la prêtrise pour moi.
\VS{45} Et j'habiterai au milieu des enfants d'Israël, et je serai leur Dieu.
\VS{46}Et ils sauront que je suis Yahweh, leur Dieu, qui les ai tirés du pays d'Egypte, pour habiter au milieu d'eux. Je suis Yahweh leur Dieu.
\Chap{30}
\TextTitle{L'autel des parfums}
\VerseOne{}Tu feras aussi un autel pour les parfums, et tu le feras de bois d'acacia.
\VS{2}Sa longueur sera d'une coudée, et sa largeur d'une coudée~; il sera carré~; mais sa hauteur sera de deux coudées, et ses cornes seront tirées de lui.
\VS{3}Tu le couvriras d'or pur, tant le dessus, que ses côtés tout autour, et ses cornes. Et tu lui feras un couronnement d'or tout autour.
\VS{4}Tu lui feras aussi deux anneaux d'or au-dessous de son couronnement, à ses deux côtés, lesquels tu mettras aux deux coins, pour y faire passer les barres qui serviront à le porter.
\VS{5}Tu feras les barres de bois d'acacia, et tu les couvriras d'or.
\VS{6}Et tu les mettras devant le voile, qui est au-devant de l'arche du témoignage, à l'endroit du propitiatoire qui est sur le témoignage, où je me trouverai avec toi.
\VS{7}Et Aaron fera sur cet autel un parfum de choses aromatiques~; il y fera un parfum chaque matin, quand il préparera les lampes.
\VS{8}Et quand Aaron allumera les lampes entre les deux soirs, il y fera aussi le parfum, à savoir le parfum perpétuel devant Yahweh dans vos générations\FTNT{Ex. 37:25-29~; 2 Ch. 13:11.}.
\VS{9}Vous n'offrirez point sur cet autel aucun parfum étranger, ni d'holocauste, ni d'offrande, et vous n'y répandrez aucune libation.
\VS{10}Mais Aaron fera une fois l'an la propitiation sur les cornes de cet autel~; il fera, dis-je, la propitiation une fois l'an sur cet autel dans vos générations, avec le sang de l'offrande pour l'expiation faite pour les propitiations. C'est une chose très sainte à Yahweh.
\TextTitle{L'offrande du rachat\FTNTT{Ex. 15:1-21~; Ps. 107:1-2.}}
\VS{11}Yahweh parla aussi à Moïse, et lui dit~:
\VS{12}Quand tu feras le dénombrement des fils d'Israël, selon leur nombre, ils donneront chacun à Yahweh le rachat de sa personne, quand tu en feras le dénombrement, et il n'y aura point de plaie sur eux quand tu en feras le dénombrement\FTNT{No. 1:2.}.
\VS{13}Tous ceux qui passeront par le dénombrement donneront un demi-sicle, selon le sicle du sanctuaire, qui est de vingt guéras~; le demi-sicle donc sera l'offrande que l'on donnera à Yahweh\FTNT{Lé. 27:25~; No. 3:47~; Ez. 45:12.}.
\VS{14}Tous ceux qui passeront par le dénombrement, depuis l'âge de vingt ans et au-dessus, donneront cette offrande à Yahweh.
\VS{15}Le riche n'augmentera rien, et le pauvre ne diminuera rien du demi-sicle, quand ils donneront à Yahweh l'offrande pour faire le rachat de vos personnes.
\VS{16}Tu prendras donc des enfants d'Israël l'argent des expiations, et tu l'appliqueras à l'œuvre de la tente d'assignation. Ce sera pour les fils d'Israël, un souvenir devant Yahweh pour faire le rachat de vos personnes.
\TextTitle{Purification par l'eau de la cuve d'airain\FTNTT{Jn. 13:3-10~; Hé. 10:22~; 1 Jn. 1:9.}}
\VS{17}Yahweh parla encore à Moïse, en disant~:
\VS{18}Fais aussi une cuve d'airain, avec sa base d'airain, pour laver. Et tu la mettras entre la tente d'assignation et l'autel, et tu mettras de l'eau dedans~;
\VS{19}Aaron et ses fils y laveront leurs mains et leurs pieds.
\VS{20}Quand ils entreront dans la tente d'assignation ils se laveront avec de l'eau, afin qu'ils ne meurent point, et quand ils approcheront de l'autel pour faire le service, afin de faire fumer l'offrande consumée par le feu à Yahweh.
\VS{21}Ils laveront donc leurs pieds et leurs mains, afin qu'ils ne meurent point~; ce leur sera une ordonnance perpétuelle, tant pour Aaron que pour ses fils, en leur génération.
\TextTitle{L'huile pour l'onction sainte\FTNTT{Jn. 4:23~; Ep. 2:18, 5:18-19.}}
\VS{22}Yahweh parla aussi à Moïse, en disant~:
\VS{23}Prends des choses aromatiques les plus exquises~; de la myrrhe franche le poids de cinq cent sicles, la moitié de cinnamome odoriférant, c'est-à-dire, le poids de deux cent cinquante sicles, et du roseau aromatique deux cent cinquante sicles.
\VS{24}De la casse le poids de cinq cent sicles, selon le sicle du sanctuaire, et un hin d'huile d'olive.
\VS{25}Et tu en feras de l'huile pour l'onction sainte, un onguent composé selon l'art du parfumeur, ce sera l'huile de l'onction sainte.
\VS{26}Puis tu en oindras la tente d'assignation, et l'arche du témoignage.
\VS{27}La table et tous ses ustensiles, le chandelier et ses ustensiles, et l'autel du parfum,
\VS{28} et l'autel des holocaustes et tous ses ustensiles, la cuve et sa base.
\VS{29}Ainsi, tu les sanctifieras, et ils seront une chose très-sainte~; tout ce qui les touchera sera saint.
\VS{30}Tu oindras aussi Aaron et ses fils, et les sanctifieras pour m'exercer la prêtrise.
\VS{31}Tu parleras aussi aux enfants d'Israël, en disant~: Ce me sera une huile d'onction sainte dans toutes vos générations.
\VS{32}On n'en oindra point la chair d'aucun homme, et vous n'en ferez point d'autre de même composition~; elle est sainte, elle vous sera sainte.
\VS{33}Quiconque composera un onguent semblable, et qui en mettra sur un autre, sera retranché de ses peuples.
\TextTitle{L'encens pur parfumé}
\VS{34}Yahweh dit aussi à Moïse~: Prends des aromates, à savoir de la gomme, de l'ongle odorant, du galbanum, le tout préparé, et de l'encens pur, le tout à poids égal.
\VS{35}Et tu en feras un parfum aromatique selon l'art du parfumeur, et tu y mettras du sel~; vous le ferez pur, et ce sera pour vous une chose sainte.
\VS{36}Et quand tu l'auras pilé bien menu, tu en mettras dans la tente d'assignation devant le témoignage, où je me trouverai avec toi. Ce sera pour vous une chose très sainte.
\VS{37}Quant au parfum que tu feras, vous ne ferez point pour vous de semblable composition~; ce sera une chose sainte pour Yahweh.
\VS{38}Quiconque en fera un semblable pour le sentir sera retranché de ses peuples.
\Chap{31}
\TextTitle{Yahweh suscite des artisans}
\VerseOne{}Yahweh parla aussi à Moïse, en disant~:
\VS{2}Regarde, j'ai appelé par son nom Betsaleel, fils d'Uri, fils de Hur, de la tribu de Juda.
\VS{3}Et je l'ai rempli de l'Esprit de Dieu, de sagesse, d'intelligence et de science pour toutes sortes d'ouvrages,
\VS{4}afin d'inventer des dessins pour travailler  l'or, l'argent et l'airain~;
\VS{5}dans la sculpture des pierres précieuses, pour les mettre en œuvre, et dans la menuiserie pour travailler dans toutes sortes d'ouvrages.
\VS{6}Et voici, je lui ai donné pour compagnon Oholiab, fils d'Ahisamac, de la tribu de Dan~; et au cœur de tout homme sage, j'ai mis de l'intelligence, afin qu'ils fassent toutes les choses que je t'ai ordonnées,
\VS{7}à savoir la tente d'assignation, l'arche du témoignage, et le propitiatoire qui doit être au-dessus, et tous les ustensiles du tabernacle~;
\VS{8}et la table avec tous ses ustensiles~; et le chandelier pur avec tous ses ustensiles~; et l'autel du parfum~;
\VS{9}et l'autel de l'holocauste avec tous ses ustensiles, la cuve et sa base~;
\VS{10}et les vêtements du service~; les saints vêtements pour le prêtre Aaron, et les vêtements de ses fils pour exercer la prêtrise~;
\VS{11}et l'huile d'onction, et le parfum des choses aromatiques pour le sanctuaire, et ils feront toutes les choses que je t'ai ordonnées.
\TextTitle{Le sabbat comme signe entre Yahweh et Israël}
\VS{12}Yahweh parla encore à Moïse, en disant~:
\VS{13}Toi aussi parle aux enfants d'Israël, en disant~: Certes, vous garderez mes sabbats, car c'est un signe entre moi et vous, et parmi vos générations, afin que vous sachiez que je suis Yahweh qui vous sanctifie.
\VS{14}Gardez donc le sabbat, car il doit vous être saint. Quiconque le violera sera puni de mort~; quiconque,dis-je, fera une œuvre en ce jour-là sera retranché du milieu de son peuple.
\VS{15}On travaillera six jours, mais le septième jour est le sabbat du repos, consacré à Yahweh~; quiconque fera une œuvre le jour du repos sera puni de mort.
\VS{16}Ainsi, les enfants d'Israël garderont le sabbat pour célébrer le jour du repos, en leur génération, par une alliance perpétuelle.
\VS{17}C'est un signe entre moi et les enfants d'Israël à perpétuité~; car Yahweh a fait en six jours les cieux et la terre, et il a cessé au septième, et s'est reposé\FTNT{Ge. 2:2~; Ez. 20:12.}.
\VS{18}Et Dieu donna à Moïse, après qu'il eut achevé de parler avec lui sur la montagne de Sinaï, les deux tables du témoignage~; tables de pierre, écrites du doigt de Dieu\FTNT{De. 9:10.}.
\Chap{32}
\TextTitle{Le culte du veau d'or}
\VerseOne{}Mais le peuple, voyant que Moïse tardait tant à descendre de la montagne, s'assembla autour d'Aaron, et lui dit~: Lève-toi, fais-nous des dieux qui marchent devant nous, car quant à ce Moïse, cet homme qui nous a fait monter du pays d'Egypte, nous ne savons ce qui lui est arrivé\FTNT{Ac. 7:40.}.
\VS{2}Et Aaron leur répondit~: Mettez en pièces les anneaux d'or qui sont aux oreilles de vos femmes, de vos fils, et de vos filles, et apportez-les-moi\FTNT{Ex. 35:22.}.
\VS{3} Et aussitôt, tout le peuple mit en pièces les anneaux d'or qui étaient à leurs oreilles, et ils les apportèrent à Aaron, 
\VS{4}qui les ayant reçu de leurs mains, forma l'or avec un burin, et en fit un veau\FTNT{Selon toute vraisemblance, les Israélites s'étaient inspirés d'une idole égyptienne, le taureau sacré Apis, pour faire le veau d'or. Dieu de la puissance sexuelle, de la fertilité et de la force, il était souvent représenté sous la forme d'un homme avec une tête de taureau, puis avec un disque solaire entre les cornes à partir du Nouvel Empire.} en métal fondu. Et ils dirent~: Ce sont ici tes dieux, ô Israël, qui t'ont fait monter du pays d'Egypte.
\VS{5}Ce qu'Aaron ayant vu, il bâtit un autel devant le veau, et cria en disant~: Demain, il y aura une fête solennelle à Yahweh.
\VS{6}Ainsi, ils se levèrent le lendemain, dès le matin, et ils offrirent des holocaustes, et présentèrent des offrandes de paix. Et le peuple s'assit pour manger et pour boire, puis ils se levèrent pour jouer\FTNT{1 Co. 10:7.}.
\TextTitle{Yahweh condamne l'idolâtrie d'Israël}
\VS{7}Alors Yahweh dit à Moïse~: Va, descends, car ton peuple que tu as fait monter du pays d'Egypte s'est corrompu\FTNT{De. 32:5.}.
\VS{8}Ils se sont promptement détournés de la voie que je leur avais ordonnée, et ils se sont fait un veau en métal fondu, et se sont prosternés devant lui, ils lui ont offert des sacrifices, puis ont dit~: Ce sont ici tes dieux, ô Israël, qui t'ont fait monter du pays d'Egypte\FTNT{1 R. 12:28.}.
\VS{9}Yahweh dit encore à Moïse~: J'ai regardé ce peuple, et voici, c'est un peuple au cou raide.
\VS{10}Maintenant laisse-moi, et ma colère s'embrasera contre eux, et je les consumerai~; mais je te ferai devenir une grande nation.
\TextTitle{Moïse implore Yahweh pour le peuple}
\VS{11}Alors Moïse supplia Yahweh, son Dieu, et dit~: Ô Yahweh, pourquoi ta colère s'embraserait-elle contre ton peuple, que tu as retiré du pays d'Egypte par une grande puissance et par une main forte\FTNT{Ps. 106:23.}~?
\VS{12}Pourquoi les Egyptiens diraient~: Il les a retirés dans de mauvaises vues, pour les tuer sur les montagnes, et pour les consumer de dessus la terre~? Reviens de l'ardeur de ta colère, et repens-toi de ce mal que tu veux faire à ton peuple\FTNT{No. 14:11-15~; De. 9:28.}.
\VS{13}Souviens-toi d'Abraham, d'Isaac et d'Israël, tes serviteurs, auxquels tu as juré par toi-même en leur disant~: Je multiplierai votre postérité comme les étoiles des cieux, et je donnerai à votre postérité tout ce pays, dont j'ai parlé, et ils l'hériteront à jamais\FTNT{De. 34:4.}.
\VS{14}Et Yahweh se repentit du mal qu'il avait dit qu'il ferait à son peuple.
\TextTitle{Jugement sur le peuple}
\VS{15}Moïse regarda, et descendit de la montagne, ayant dans sa main les deux tables du témoignage, et les tables étaient écrites des deux côtés, écrites de l'un et de l'autre côté.
\VS{16}Et les tables étaient l'ouvrage de Dieu, et l'écriture était l'écriture de Dieu, gravée sur les tables.
\VS{17}Et Josué, entendant la voix du peuple qui faisait un grand bruit, dit à Moïse~: Il y a un bruit de bataille au camp.
\VS{18}Et Moïse lui répondit~: Ce n'est pas une voix ni un cri de gens qui soient les plus forts, ni une voix ni un cri de gens qui soient les plus faibles~; mais j'entends une voix de gens qui chantent.
\VS{19}Et il arriva que lorsque Moïse fut approché du camp, il vit le veau et les danses, et la colère de Moïse s'embrasa~; et il jeta de ses mains les tables, et les rompit au pied de la montagne.
\VS{20}Il prit ensuite le veau qu'ils avaient fait, et le brûla au feu, et le moulut jusqu'à ce qu'il fut en poudre, puis il répandit cette poudre dans de l'eau, et il en fit boire aux enfants d'Israël\FTNT{De. 9:17-21.}.
\VS{21}Et Moïse dit à Aaron~: Que t'a fait ce peuple pour que tu aies fait venir sur lui un si grand péché~?
\VS{22}Et Aaron lui répondit~: Que la colère de mon seigneur ne s'embrase point, tu sais que ce peuple est porté au mal.
\VS{23}Ils m'ont dit~: Fais-nous un dieu qui marche devant nous, car ce Moïse, cet homme qui nous a fait monter du pays d'Egypte, nous ne savons ce qui lui est arrivé.
\VS{24}Alors je leur ai dit~: Que celui qui a de l'or, le mette en pièces~! Et ils me l'ont donné~; et je l'ai jeté au feu, et ce veau en est sorti.
\VS{25}Or Moïse vit que le peuple était dénudé, car Aaron l'avait dénudé pour être en opprobre parmi leurs ennemis.
\VS{26}Et Moïse se tenant à la porte du camp, dit~: Qui est pour Yahweh~? Qu'il vienne vers moi~! Et tous les fils de Lévi s'assemblèrent vers lui.
\VS{27}Et il leur dit~: Ainsi parle Yahweh, le Dieu d'Israël~: Que chacun mette son épée à son côté, passez et repassez de porte en porte par le camp, et que chacun de vous tue son frère, son ami, et son voisin.
\VS{28}Et les fils de Lévi firent selon la parole de Moïse~; et ce jour-là il tomba parmi le peuple environ trois mille hommes.
\VS{29}Car Moïse avait dit~: Consacrez aujourd'hui vos mains à Yahweh, chacun même contre son fils, et contre son frère, afin que vous attiriez aujourd'hui sur vous la bénédiction.
\TextTitle{Moïse intercède pour Israël}
\VS{30}Et le lendemain, Moïse dit au peuple~: Vous avez commis un grand péché~; mais je monterai vers Yahweh, et peut-être je ferais propitiation pour votre péché.
\VS{31}Moïse donc retourna vers Yahweh et dit~: Hélas~! Je te prie, ce peuple a commis un grand péché, en se faisant des dieux d'or.
\VS{32}Maintenant pardonne leur péché~! Sinon, efface-moi maintenant de ton livre que tu as écrit.
\VS{33}Et Yahweh répondit à Moïse~: C'est celui qui aura péché contre moi que j'effacerai de mon livre\FTNT{Ap. 3:5~; Ap. 20:15~; Ap. 21:27.}.
\VS{34}Va maintenant, conduis le peuple au lieu duquel je t'ai parlé. Voici, mon Ange ira devant toi~; et le jour où je ferai punition, je punirai sur eux leur péché.
\VS{35} Ainsi, Yahweh frappa le peuple, parce qu'ils avaient été les auteurs du veau qu'Aaron avait fait.
\Chap{33}
\TextTitle{Yahweh ne veut plus marcher avec Israël}
\VerseOne{}Yahweh donc dit à Moïse~: Va, monte d'ici, toi et le peuple que tu as fait monter du pays d'Egypte, au pays que j'ai juré de donner à Abraham, à Isaac, et à Jacob, en disant~: Je le donnerai à ta postérité.
\VS{2}Et j'enverrai un Ange devant toi, et je chasserai les Cananéens, les Amoréens, les Héthiens, les Phéréziens, les Héviens et les Jébusiens,
\VS{3}pour vous conduire au pays découlant de lait et de miel, mais je ne monterai point au milieu de toi, parce que tu es un peuple au cou raide, de peur que je ne te consume en chemin.
\VS{4}Et le peuple entendit ces tristes nouvelles, et en mena le deuil, et aucun d'eux ne mit ses ornements sur soi.
\VS{5}Car Yahweh avait dit à Moïse~: Dis aux enfants d'Israël~: Vous êtes un peuple au cou raide~; je monterai en un moment au milieu de toi, et je te consumerai. Maintenant donc ôte tes ornements de dessus toi, et je saurai ce que je te ferai.
\VS{6}Ainsi, les enfants d'Israël se dépouillèrent de leurs ornements vers la montagne d'Horeb.
\TextTitle{Moïse dresse la tente d'assignation hors du camp}
\VS{7}Et Moïse prit une tente, et la tendit pour soi hors du camp, l'éloignant du camp~; et il l'appela la tente d'assignation~; et tous ceux qui cherchaient Yahweh sortaient vers la tente d'assignation qui était hors du camp.
\VS{8}Et il arrivait qu'aussitôt que Moïse sortait vers la tente, tout le peuple se levait, et chacun se tenait à l'entrée de sa tente, et regardait Moïse par-derrière, jusqu'à ce qu'il soit entré dans la tente.
\VS{9}Et sitôt que Moïse était entré dans la tente, la colonne de nuée descendait et s'arrêtait à la porte de la tente, et Yahweh parlait avec Moïse.
\VS{10}Et tout le peuple voyant la colonne de nuée s'arrêtant à la porte de la tente se levait, et chacun se prosternait à la porte de sa tente.
\VS{11}Et Yahweh parlait à Moïse face à face, comme un homme parle avec son intime ami. Puis Moïse retournait au camp, mais son serviteur Josué, fils de Nun, jeune homme, ne bougeait point de la tente\FTNT{No. 12:8~; De. 34:10~; Jn. 15:14-15.}.
\TextTitle{Moïse demande que Yahweh marche avec Israël}
\VS{12}Moïse donc dit à Yahweh~: Regarde, tu m'as dit~: Fais monter ce peuple, et tu ne m'as point fait connaître celui que tu dois envoyer avec moi~; tu as même dit~: Je te connais par ton nom, et aussi, tu as trouvé grâce devant mes yeux.
\VS{13}Or maintenant, je te prie, si j'ai trouvé grâce devant tes yeux, fais-moi connaître ton chemin, et je te connaîtrai, afin que je trouve grâce devant tes yeux~; considère aussi que cette nation est ton peuple\FTNT{Ps. 25:4.}.
\VS{14}Et Yahweh dit~: Ma face ira, et je te donnerai du repos.
\VS{15}Et Moïse lui dit~: Si ta face ne vient, ne nous fais point monter d'ici.
\VS{16}Car en quoi connaîtra-t-on que nous avons trouvé grace devant tes yeux, moi et ton peuple~? Ne sera-ce pas quand tu marcheras avec nous~? Et alors, moi et ton peuple serons en admiration plus que tous les peuples qui sont sur la terre~?
\VS{17}Et Yahweh dit à Moïse~: Je ferai aussi ce que tu dis~; car tu as trouvé grâce devant mes yeux, et je te connais par ton nom.
\TextTitle{Moïse veut voir la gloire de Yahweh}
\VS{18}Moïse dit aussi~: Je te prie, fais-moi voir ta gloire~!
\VS{19}Et Dieu dit~: Je ferai passer toute ma bonté devant ta face, et je crierai le Nom de Yahweh devant toi~; et je ferai grâce à qui je ferai grâce, et j'aurai compassion de celui de qui j'aurai compassion\FTNT{Ro. 9:15.}.
\VS{20}Puis il dit~: Tu ne pourras pas voir ma face, car nul homme ne peut me voir et vivre\FTNT{Jn. 1:18~; Jn. 14:8-11.}.
\VS{21}Yahweh dit aussi~: Voici, il y a un lieu près de moi, et tu t'arrêteras sur le rocher\FTNT{Le rocher préfigurait Jésus-Christ, le Roc sur lequel nous devons bâtir nos vies et le fondement de l'Eglise (Ps. 18:32~; Mt. 7:24-25~; Mt. 16:18~; 1 Co. 3:11). Voir commentaire Es. 8:13-17.}.
\VS{22}Et quand ma gloire passera, je te mettrai dans un creux du rocher, et te couvrirai de ma main, jusqu'à ce que je sois passé.
\VS{23}Puis je retirerai ma main, et tu me verras par-derrière, mais ma face ne se verra point.
\Chap{34}
\TextTitle{De nouvelles tables~; la gloire de Yahweh\FTNTT{Ex. 33:18-23.}}
\VerseOne{}Et Yahweh dit à Moïse~: Aplanis-toi deux tables de pierre comme les premières, et j'écrirai sur elles les paroles qui étaient sur les premières tables que tu as rompues\FTNT{De. 10:1.}.
\VS{2}Et sois prêt au matin, et monte au matin sur la montagne de Sinaï, et présente-toi là devant moi sur le haut de la montagne.
\VS{3}Mais que personne ne monte avec toi, et même que personne ne paraisse sur toute la montagne~; et que ni menu ni gros bétail ne paisse sur cette montagne\FTNT{Ex. 19:12-13.}.
\VS{4}Moïse donc aplanit deux tables de pierre comme les premières, et se leva de bon matin, et monta sur la montagne de Sinaï, comme Yahweh le lui avait ordonné, et il prit dans sa main les deux tables de pierre.
\VS{5}Et Yahweh descendit dans la nuée, et s'arrêta là avec lui, et cria le Nom de Yahweh.
\VS{6}Comme donc Yahweh passait par devant lui, il cria~: Yahweh, Yahweh~! Le Dieu compatissant, miséricordieux, lent à la colère, abondant en bonté et en fidélité\FTNT{No. 14:18~; 2 Ch. 30:9~; Né. 9:17~; Ps. 103:8.}.
\VS{7}Qui conserve sa bonté jusqu'à mille générations, ôtant l'iniquité, le crime, et le péché, qui ne tient point le coupable pour innocent, et qui punit l'iniquité des pères sur les fils, et sur les fils des fils, jusqu'à la troisième et à la quatrième génération\FTNT{Ex. 20:6~; De. 5:10~; Jé. 32:18.}.
\VS{8}Et Moïse se hatant, baissa la tête contre terre et se prosterna. 
\VS{9}Et il dit~: Ô Seigneur~! Je te prie, si j'ai trouvé grâce à tes yeux, que le Seigneur marche maintenant au milieu de nous, car c'est un peuple au cou raide. Pardonne donc nos iniquités et notre péché, et prends-nous pour ta possession.
\TextTitle{Yahweh renouvelle ses promesses\FTNTT{Ex. 33:18-23.}}
\VS{10}Et il répondit~: Voici, je traite alliance devant tout ton peuple, je ferai des merveilles qui n'ont point été faites sur toute la terre ni dans aucune nation. Et tout le peuple au milieu duquel tu es, verra l'œuvre de Yahweh, car ce que je m'en vais faire avec toi sera une chose redoutable.
\VS{11}Garde soigneusement ce que je t'ordonne aujourd'hui. Voici, je m'en vais chasser devant toi les Amoréens, les Cananéens, les Héthiens, les Phéréziens, les Héviens, et les Jébusiens.
\VS{12}Garde-toi de traiter alliance avec les habitants du pays où tu dois entrer, de peur que peut-être ils ne soient un piège pour toi\FTNT{De. 7:2~; Jos. 23:12-13~; 2 Co. 6:14.}.
\VS{13}Mais vous démolirez leurs autels, vous briserez leurs statues, et vous couperez leurs emblèmes d'Asherah\FTNT{Cité au moins quarante fois dans le Tanakh, le terme hébreu «~Asherah~» fait référence à un «~arbre sacré, un pieu près d'un autel~» ou encore «~une idole~», terme par lequel il est majoritairement traduit. Il s'agit également de l'objet en bois utilisé dans le culte de la parèdre de Baal. Manassé, roi de Juda, introduisit l'emblème d'Asherah dans le temple (2 R. 21:1-7) en dépit de l'interdiction formelle de Yahweh (De. 16:21). Il n'en fut enlevé que lors des réformes de Josias et d'Ezéchias (2 R. 18:3-4~; 2 R. 23:6-14). Pourtant, Yahweh a toujours exigé la destruction de celle qu'il a nommé «~l'abomination des Sidoniens~», de peur que son peuple y trouve une occasion de chute (Jg. 2:13~; Jg. 10:6~; 1 S. 31:10~; 1 R. 11:5-33~; 2 R. 23:13). Bien qu'étant majoritairement citée dans les Ecritures en tant qu'objet de culte, Asherah est également associée à la divinité Astarté, connue pour être la Diane des Ephésiens (Ac. 19:23-40), la reine du ciel (Jé. 7:18~; Jé. 44:15-30), l'Isis des Egyptiens et l'épouse de Baal (voir commentaire en Jg. 2:13).}.
\VS{14}Car tu ne te prosterneras point devant un autre dieu, parce que Yahweh se nomme le Dieu jaloux~; c'est le Dieu qui est jaloux.
\VS{15}Afin qu'il n'arrive que tu traites alliance avec les habitants du pays, et que quand ils viendront à se prostituer après leurs dieux et à sacrifier à leurs dieux, quelqu'un ne t'invite et que tu ne manges de leurs sacrifices~;
\VS{16}et que tu ne prennes de leurs filles pour tes fils, lesquelles se prostituant après leurs dieux, n'entraînent tes fils à se prostituer après leurs dieux.
\VS{17}Tu ne te feras aucun dieu de métal fondu.
\TextTitle{Les fêtes et le sabbat\FTNTT{Lé. 23:4-44.}}
\VS{18}Tu garderas la fête solennelle des pains sans levain~; tu mangeras les pains sans levain pendant sept jours, comme je te l'ai ordonné, dans la saison où les épis mûrissent~; car c'est dans le mois des épis que tu es sorti du pays d'Egypte.
\VS{19}Tout premier-né sera à moi~; même le premier mâle qui naîtra de toutes les bêtes, tant du gros que du menu bétail.
\VS{20}Mais tu rachèteras avec un agneau ou un chevreau le premier-né d'un âne. Si tu ne le rachètes pas, tu lui couperas le cou. Tu rachèteras tout premier-né de tes fils~; et nul ne se présentera devant ma face à vide.
\VS{21}Tu travailleras six jours, mais au septième tu te reposeras~; tu te reposeras au temps du labourage et de la moisson.
\VS{22}Tu feras la fête solennelle des semaines au temps des premiers fruits de la moisson du froment~; et la fête solennelle de la récolte à la fin de l'année.
\VS{23}Trois fois l'an, tout mâle d'entre vous comparaîtra devant le Seigneur Yahweh, le Dieu d'Israël.
\VS{24}Car je déposséderai les nations de devant toi, et j'étendrai tes limites, et nul ne convoitera ton pays lorsque tu monteras pour comparaître trois fois l'an devant Yahweh, ton Dieu.
\VS{25}Tu n'offriras point le sang de mon sacrifice avec du pain levé~; on ne gardera rien du sacrifice de la fête solennelle de la Pâque jusqu'au matin.
\VS{26}Tu apporteras les prémices des premiers fruits de la terre dans la maison de Yahweh, ton Dieu. Tu ne feras point cuire le chevreau dans le lait de sa mère.
\VS{27}Yahweh dit aussi à Moïse~: Ecris ces paroles, car suivant la teneur de ces paroles, j'ai traité alliance avec toi et avec Israël.
\VS{28}Et Moïse demeura là avec Yahweh quarante jours et quarante nuits, sans manger de pain et sans boire d'eau~; et Yahweh écrivit sur les tables les paroles de l'alliance, c'est-à-dire les dix paroles.
\TextTitle{La gloire de Yahweh sur le visage de Moïse}
\VS{29}Or il arriva que lorsque Moïse descendait de la montagne de Sinaï, tenant dans sa main les deux tables du témoignage, lorsque, dis-je, il descendait de la montagne, il ne s'aperçut point que la peau de son visage était devenue rayonnante pendant qu'il parlait avec Dieu.
\VS{30}Mais Aaron et tous les enfants d'Israël ayant vu Moïse, et s'étant aperçus que la peau de son visage était rayonnante, ils craignirent de s'approcher de lui.
\VS{31}Mais Moïse les appela, et Aaron et tous les principaux de l'assemblée retournèrent vers lui~; et Moïse parla avec eux.
\VS{32}Après quoi, tous les enfants d'Israël s'approchèrent, et il leur ordonna toutes les choses que Yahweh lui avait dites sur la montagne de Sinaï.
\VS{33}Ainsi, Moïse acheva de leur parler~; or il avait mis un voile sur son visage.
\VS{34}Et quand Moïse entrait vers Yahweh pour parler avec lui, il ôtait le voile jusqu'à ce qu'il sorte~; et étant sorti, il disait aux enfants d'Israël ce qui lui avait été ordonné.
\VS{35}Or les enfants d'Israël avaient vu que le visage de Moïse, la peau, dis-je, de son visage rayonnait. C'est pourquoi Moïse remettait le voile sur son visage, jusqu'à ce qu'il entre pour parler avec Yahweh.
\Chap{35}
\TextTitle{Rappels sur le sabbat}
\VerseOne{}Moïse donc assembla toute la congrégation des enfants d'Israël, et leur dit~: Ce sont ici les choses que Yahweh a ordonnées de faire.
\VS{2}On travaillera six jours, mais le septième jour il y aura sainteté pour vous, car c'est le sabbat du repos consacré à Yahweh~; quiconque travaillera en ce jour-là sera puni de mort.
\VS{3}Vous n'allumerez point de feu dans aucune de vos demeures le jour du repos.
\TextTitle{Les offrandes pour le tabernacle\FTNTT{Ex. 25:1-8.}}
\VS{4}Puis Moïse parla à toute l'assemblée des enfants d'Israël, et leur dit~: C'est ici ce que Yahweh vous a ordonné, en disant~:
\VS{5}Prenez des choses qui sont chez vous une offrande pour Yahweh. Quiconque sera de bonne volonté, apportera cette offrande pour Yahweh, à savoir de l'or, de l'argent, de l'airain\FTNT{Ex. 25:2~; 2 Co. 8:12.},
\VS{6}de la pourpre, de l'écarlate, du cramoisi, du fin lin, du poil de chèvre,
\VS{7}des peaux de béliers teintes en rouge, des peaux de taissons, du bois d'acacia,
\VS{8}de l'huile pour le chandelier, des aromates pour l'huile d'onction et pour le parfum odoriférant,
\VS{9}des pierres d'onyx, et des pierres pour la garniture de l'éphod et pour le pectoral.
\VS{10}Et tous les hommes d'esprit d'entre vous viendront et feront tout ce que Yahweh a ordonné, 
\VS{11}à savoir le tabernacle, sa tente, et sa couverture, ses agrafes, ses planches, ses barres, ses colonnes, et ses bases~;
\VS{12}l'arche et ses barres, le propitiatoire et le voile qui sert de rideau~;
\VS{13}la table et ses barres, et tous ses ustensiles, et le pain de proposition~;
\VS{14}et le chandelier du luminaire, ses ustensiles, ses lampes et l'huile du luminaire~;
\VS{15}l'autel du parfum et ses barres~; l'huile d'onction, le parfum odoriférant, le rideau de la porte pour l'entrée du tabernacle~;
\VS{16}l'autel de l'holocauste, sa grille d'airain, ses barres et tous ses ustensiles~; la cuve avec sa base~;
\VS{17}les courtines du parvis, ses colonnes, ses bases, et le rideau de la porte du parvis~;
\VS{18}les pieux du tabernacle, les pieux du parvis et leur cordage~;
\VS{19}les vêtements du service pour faire le service dans le sanctuaire, les saints vêtements d'Aaron, le prêtre, et les vêtements de ses fils pour exercer la prêtrise.
\VS{20}Alors toute l'assemblée des enfants d'Israël sortit de la présence de Moïse.
\VS{21}Et quiconque fut ému en son cœur, quiconque, dis-je, se sentit porté à la libéralité, apporta l'offrande de Yahweh pour l'ouvrage de la tente d'assignation et pour tout son service et pour les saints vêtements.
\VS{22}Et les hommes vinrent avec les femmes~; quiconque fut de cœur volontaire, apporta des boucles, des bagues, des anneaux, des bracelets, et des joyaux d'or~; et quiconque offrit quelque offrande d'or à Yahweh.
\VS{23}Tout homme aussi chez qui se trouvait de la pourpre, de l'écarlate, du cramoisi, du fin lin, du poil de chèvre, des peaux de béliers teintes en rouge et des peaux de taissons, les apportèrent.
\VS{24}Tout homme qui avait de quoi faire une offrande d'argent et d'airain, l'apporta pour l'offrande de Yahweh~; tout homme aussi chez qui fut trouvé du bois d'acacia pour tout l'ouvrage du service, l'apporta.
\VS{25}Toute femme adroite fila de sa main et apporta ce qu'elle avait filé~: De la pourpre, de l'écarlate, du cramoisi, et du fin lin\FTNT{Pr. 31:19.}.
\VS{26}Toutes les femmes aussi dont le cœur les y porta en sagesse, filèrent du poil de chèvre.
\VS{27}Les principaux aussi de l'assemblée apportèrent des pierres d'onyx, et d'autres pierres pour la garniture de l'éphod et du pectoral~;
\VS{28}et des aromates, et de l'huile tant pour le chandelier que pour l'huile d'onction, et pour le parfum odoriférant.
\VS{29}Tout homme donc et toute femme que le cœur incita à la libéralité pour apporter de quoi faire l'ouvrage que Yahweh avait ordonné par le moyen de Moïse, tous les enfants, dis-je, d'Israël apportèrent volontairement des présents à Yahweh.
\TextTitle{Betsaleel et Oholiab oints pour l'œuvre du tabernacle}
\VS{30}Alors Moïse dit aux fils d'Israël~: Voyez, Yahweh a appelé par son nom Betsaleel, fils d'Uri, fils de Hur, de la tribu de Juda.
\VS{31}Et il l'a rempli de l'Esprit de Dieu, de sagesse, d'intelligence, de science, pour toutes sortes d'ouvrages.
\VS{32}Même afin d'inventer des dessins, pour travailler l'or, l'argent et l'airain~;
\VS{33}dans la sculpture des pierres précieuses pour les mettre en œuvre, et dans la menuiserie pour travailler en tout ouvrage exquis.
\VS{34}Et il lui a mis aussi au cœur, tant à lui qu'à Oholiab, fils d'Ahisamac, de la tribu de Dan, de l'enseigner.
\VS{35}Et il les a remplis de sagesse pour faire toutes sortes d'ouvrages d'ouvrier, même d'ouvrier en ouvrage exquis, et en broderie, en pourpre, en écarlate, en cramoisi, et en fin lin, et d'ouvrage de tisserand, faisant toutes sortes d'ouvrages et inventant toutes sortes de dessins\FTNT{Es. 28:26.}.
\Chap{36}
\TextTitle{Construction du tabernacle d'après le modèle donné par Yahweh\FTNTT{Ex. 36-39.}}
\VerseOne{}Et Betsaleel et Oholiab, et tous les hommes au coeur sage auxquels Yahweh avait donné de la sagesse et de l'intelligence pour savoir faire tout l'ouvrage du service du sanctuaire, firent selon toutes les choses que Yahweh avait ordonnées.
\VS{2}Moïse donc appela Betsaleel et Oholiab, et tous les hommes d'esprit, dans le cœur desquels Yahweh avait mis de la sagesse, et tous ceux qui furent émus en leur cœur de se présenter pour faire cet ouvrage.
\VS{3}Lesquels emportèrent de devant Moïse toute l'offrande que les enfants d'Israël avaient apportée pour faire l'ouvrage du service du sanctuaire. Or on apportait encore chaque matin quelques offrandes volontaires.
\VS{4}C'est pourquoi, tous les hommes sages qui faisaient tout l'ouvrage du sanctuaire, vinrent chacun de l'ouvrage qu'ils faisaient,
\VS{5}et parlèrent à Moïse, en disant~: Le peuple ne cesse d'apporter plus qu'il ne faut pour le service et pour l'ouvrage que Yahweh a ordonné de faire.
\VS{6}Alors, par l'ordre de Moïse, on fit crier dans le camp que ni homme ni femme ne fasse plus d'ouvrage pour l'offrande du sanctuaire~; et ainsi, on empêcha le peuple d'offrir.
\VS{7}Car ils avaient du travail suffisant pour tout l'ouvrage à faire, et il y en avait même de reste.
\TextTitle{Les tapis de fin lin}
\VS{8}Tous les hommes donc au coeur sage d'entre ceux qui faisaient l'ouvrage, firent le tabernacle, à savoir dix tapis de fin lin retors, de pourpre, d'écarlate, et de cramoisi~; et ils les firent semés de chérubins d'un ouvrage exquis.
\VS{9}La longueur d'un tapis était de vingt-huit coudées, et la largeur du même tapis de quatre coudées~; tous les tapis avaient une même mesure\FTNT{Ex. 26:1-6.}.
\VS{10}Et ils joignirent cinq tapis l'un à l'autre, et cinq autres tapis l'un à l'autre.
\VS{11}Et ils firent des lacets de pourpre sur le bord d'un tapis, à savoir au bord de celui qui était attaché~; ils en firent ainsi au bord du dernier tapis dans l'assemblage de l'autre.
\VS{12}Ils firent cinquante lacets à un tapis, et cinquante lacets au bord du tapis qui était dans l'assemblage de l'autre~; les lacets étant vis-à-vis l'un de l'autre.
\VS{13}Puis on fit cinquante agrafes d'or, et on attacha les tapis l'un à l'autre avec les agrafes~; ainsi fut fait le tabernacle.
\TextTitle{Les tapis de poils de chèvres}
\VS{14}Puis on fit des tapis de poils de chèvres, pour servir de tente au-dessus du tabernacle~; on fit onze de ces tapis.
\VS{15}La longueur d'un tapis était de trente coudées, et la largeur du même tapis de quatre coudées~; et les onze tapis étaient d'une même mesure.
\VS{16}Et on assembla cinq de ces tapis à part, et six tapis à part.
\VS{17}On fit aussi cinquante lacets sur le bord de l'un des tapis, à savoir au dernier qui était attaché, et cinquante lacets sur le bord de l'autre tapis, qui était attaché.
\VS{18}On fit aussi cinquante agrafes d'airain pour assembler la tente, afin qu'il n'y en eut qu'une.
\TextTitle{Les couvertures de peaux de béliers et de taissons}
\VS{19}Puis, on fit pour la tente une couverture de peaux de béliers teintes en rouge, et une couverture de peaux de taissons par-dessus.
\TextTitle{Les planches et leurs bases}
\VS{20}Et on fit pour le tabernacle des planches de bois d'acacia, qu'on fit tenir debout.
\VS{21}La longueur d'une planche était de dix coudées, et la largeur de la même planche d'une coudée et demie.
\VS{22}Il y avait deux tenons à chaque planche en façon d'échelons l'un après l'autre~; on fit la même chose pour toutes les planches du tabernacle.
\VS{23}On fit donc les planches pour le tabernacle~; à savoir vingt planches au côté qui regardaient directement vers le sud.
\VS{24}Et au-dessous des vingt planches, on fit quarante bases d'argent, deux bases sous une planche, pour ses deux tenons, et deux bases sous l'autre planche, pour ses deux tenons.
\VS{25}On fit aussi vingt planches pour l'autre côté du tabernacle, du côté nord,
\VS{26}et leurs quarante bases d'argent~: Deux bases sous une planche, et deux bases sous l'autre planche.
\VS{27}Et pour le fond du tabernacle, vers l'occident, on fit six planches.
\VS{28}Et on fit deux planches pour les angles du tabernacle aux deux cotés du fond~;
\VS{29}qui étaient égales par le bas, et qui étaient jointes et unies par le haut avec un anneau~; on fit la même chose aux deux planches qui étaient aux deux angles.
\VS{30}Il y avait donc huit planches et seize bases d'argent, à savoir deux bases sous chaque planche.
\VS{31}Puis on fit cinq barres de bois d'acacia, pour les planches de l'un des côtés du tabernacle~;
\VS{32}et cinq barres pour les planches de l'autre côté du tabernacle~; et cinq barres pour les planches du tabernacle pour le fond, vers le côté de l'occident.
\VS{33}Et on fit que la barre du milieu passait par le milieu des planches d'une extrémité à l'autre.
\VS{34}Et on couvrit d'or les planches, et on fit leurs anneaux d'or pour y faire passer les barres, et on couvrit d'or les barres.
\TextTitle{Le voile et le rideau extérieur}
\VS{35}On fit aussi le voile de pourpre, d'écarlate, de cramoisi, et de fin lin retors~; on le fit d'ouvrage exquis, avec des chérubins.
\VS{36}Et on lui fit quatre piliers de bois d'acacia, qu'on couvrit d'or, ayant leurs crochets d'or~; et on fondit pour eux quatre bases d'argent.
\VS{37}On fit aussi à l'entrée de la tente un rideau de pourpre, d'écarlate, de cramoisi, et de fin lin retors~; d'ouvrage de broderie~;
\VS{38}et ses cinq piliers avec leurs crochets~; et on couvrit d'or leurs chapiteaux et leurs filets~; mais leurs cinq bases étaient d'airain.
\Chap{37}
\TextTitle{L'arche de l'alliance}
\VerseOne{}Puis Betsaleel fit l'arche de bois d'acacia. Sa longueur était de deux coudées et demie, et sa largeur d'une coudée et demie, et sa hauteur d'une coudée et demie\FTNT{Ex. 23:10-31.}.
\VS{2}Et il la couvrit par dedans et par dehors de pur or, et lui fit un couronnement d'or tout autour.
\VS{3}Et il lui fondit pour elle quatre anneaux d'or pour les mettre sur ses quatre coins, à savoir deux anneaux à l'un de ses côtés, et deux autres à l'autre côté.
\VS{4}Et il fit aussi des barres de bois d'acacia, et les couvrit d'or.
\VS{5}Et il fit entrer les barres dans les anneaux aux côtés de l'arche, pour porter l'arche.
\TextTitle{Le propitiatoire}
\VS{6}Il fit aussi le propitiatoire d'or pur~; sa longueur était de deux coudées et demie, et sa largeur d'une coudée et demie.
\VS{7}Et il fit deux chérubins d'or~; il les fit d'ouvrage étendu au marteau, tirés des deux extrémités du propitiatoire~;
\VS{8}à savoir un chérubin tiré de l'une des extrémités et un chérubin tiré de l'autre extrémité~; il fit, dis-je, les chérubins tirés du propitiatoire~; à savoir de ses deux extrémités. 
\VS{9}Et les chérubins étendaient leurs ailes en haut, couvrant de leurs ailes le propitiatoire~; et leurs faces étaient vis-à-vis l'une de l'autre, et les chérubins regardaient vers le propitiatoire.
\TextTitle{La table des pains de proposition}
\VS{10}Il fit aussi la table de bois d'acacia~; sa longueur était de deux coudées, et sa largeur d'une coudée, et sa hauteur d'une coudée et demie.
\VS{11}Et il la couvrit d'or pur, et lui fit un couronnement d'or tout autour.
\VS{12}Il lui fit aussi à l'entour un rebord d'une largeur d'une paume, et à l'entour de sa bordure un couronnement d'or.
\VS{13}Et il lui fondit quatre anneaux d'or, et il mit les anneaux aux quatre coins, qui étaient à ses quatre pieds.
\VS{14}Les anneaux étaient à coté du rebord, pour y mettre les barres afin de porter la table avec elles.
\VS{15}Et il fit les barres de bois d'acacia, et les couvrit d'or pour porter la table.
\VS{16}Il fit aussi d'or pur des ustensiles pour poser sur la table, ses plats, ses tasses, ses bassins et ses gobelets avec lesquels on devait faire les aspersions.
\TextTitle{Le chandelier}
\VS{17}Il fit aussi le chandelier d'or pur~; il le fit d'ouvrage façonné au marteau~; sa tige, ses branches, ses plats, ses pommeaux et ses fleurs étaient tirés de lui.
\VS{18}Et six branches sortaient de ses côtés, trois branches d'un côté du chandelier, et trois de l'autre côté du chandelier.
\VS{19}Il y avait sur l'une des branches trois plats en forme d'amande, un pommeau et une fleur~; et sur l'autre branche trois plats en forme d'amande, un pommeau et une fleur~; il fit la même chose aux six branches qui sortaient du chandelier.
\VS{20}Et il y avait sur le chandelier quatre plats en forme d'amande, ses pommeaux et ses fleurs.
\VS{21}Et un pommeau sous deux branches tirées du chandelier, et un pommeau sous deux autres branches, tirées de lui, et un pommeau sous deux autres branches, tirées de lui, à savoir des six branches qui procédaient du chandelier.
\VS{22}Leurs pommeaux et leurs branches étaient tirés de lui, et tout le chandelier était un ouvrage d'une seule pièce étendue au marteau et d'or pur.
\VS{23}Il fit aussi ses sept lampes, ses mouchettes, et ses encensoirs d'or pur.
\VS{24}Et il le fit avec toute sa garniture d'un talent d'or pur.
\TextTitle{L'autel des parfums}
\VS{25}Il fit aussi de bois d'acacia l'autel des parfums. Sa longueur était d'une coudée, et sa largeur d'une coudée. Il était carré, et sa hauteur était de deux coudées, et ses cornes procédaient de lui\FTNT{Ex. 30:1-10.}.
\VS{26} Et il couvrit d'or pur le dessus de l'autel, ses côtés tout à l'entour, et ses cornes~; et il lui fit tout à l'entour un couronnement d'or.
\VS{27}Il fit aussi au-dessous de son couronnement deux anneaux d'or à ses deux côtés, lesquels il mit aux deux coins, pour y faire passer les barres afin de le porter avec elles.
\VS{28}Et il fit les barres de bois d'acacia, et les couvrit d'or.
\TextTitle{L'huile d'onction et le parfum}
\VS{29}Il composa aussi l'huile pour l'onction, qui était une chose sainte, et le parfum pur odoriférant, d'ouvrage de parfumeur.
\Chap{38}
\TextTitle{L'autel des holocaustes}
\VerseOne{}Il fit aussi de bois d'acacia l'autel des holocaustes. Et sa longueur était de cinq coudées, et sa largeur de cinq coudées. Il était carré, et sa hauteur était de trois coudées\FTNT{Ex. 27:1-8.}.
\VS{2}Et il fit ses cornes à ses quatre coins. Ses cornes sortaient de lui, et il le couvrit d'airain.
\VS{3}Il fit aussi tous les ustensiles de l'autel~: Les chaudrons, les racloirs, les bassins, les fourchettes et les encensoirs~; il fit tous ses ustensiles d'airain.
\VS{4}Et il fit pour l'autel une grille d'airain en forme de treillis, au-dessous de l'enceinte de l'autel, depuis le bas jusqu'au milieu.
\VS{5}Et il fondit quatre anneaux aux quatre coins de la grille d'airain pour mettre les barres.
\VS{6} Et il fit les barres de bois d'acacia, et les couvrit d'airain.
\VS{7}Et il fit passer les barres dans les anneaux, au cotés de l'autel, pour le porter avec elles. Il le fit creux, avec des planches.
\TextTitle{La cuve d'airain}
\VS{8}Il fit aussi la cuve d'airain et sa base d'airain avec les miroirs des femmes qui s'assemblaient à l'entrée de la tente d'assignation\FTNT{Ex. 30:14-18.}.
\TextTitle{Le parvis}
\VS{9}Il fit aussi un parvis, pour le côté qui regarde vers le sud, et des courtines de fin lin retors, de cent coudées, pour le parvis.
\VS{10}Il fit d'airain leurs vingt colonnes avec leurs vingt bases, mais les crochets des colonnes et leurs filets étaient d'argent.
\VS{11}Et pour le côté nord, il fit des courtines de cent coudées, leurs vingt colonnes et leurs vingt bases étaient d'airain, mais les crochets des colonnes et leurs filets étaient d'argent.
\VS{12}Pour le côté de l'occident, des courtines de cinquante coudées, leurs dix colonnes, et leurs dix bases. Les crochets des colonnes et leurs filets étaient d'argent.
\VS{13}Pour le côté de l'orient droit vers le levant, des courtines de cinquante coudées.
\VS{14}Il fit pour l'un des côtés quinze coudées de courtines, et leurs trois colonnes avec leurs trois bases.
\VS{15}Et pour l'autre côté, quinze coudées de courtines, afin qu'il y en ait autant de part et d'autre de la porte du parvis, et leurs trois colonnes avec leurs trois bases.
\VS{16}Il fit donc toutes les courtines du parvis qui étaient tout autour de fin lin retors.
\VS{17}Il fit aussi d'airain les bases des colonnes, mais il fit d'argent les crochets des colonnes et les filets, et leurs chapiteaux furent couverts d'argent~; et toutes les colonnes du parvis furent ceintes tout autour d'un filet d'argent.
\TextTitle{La porte du parvis}
\VS{18} Et il fit le rideau de la porte du parvis de pourpre, d'écarlate, et de cramoisi et de fin lin retors, d'ouvrage de broderie, de la longueur de vingt coudées, et de la hauteur qui était comme la largeur de cinq coudées, à la correspondance des courtines du parvis~;
\VS{19}et ses quatre colonnes avec leurs bases d'airain, et leurs crochets d'argent, la couverture aussi de leur chapiteaux et leurs filets d'argent~; 
\VS{20} et tous les pieux du tabernacle et du parvis tout autour d'airain.
\TextTitle{Les comptes du tabernacle}
\VS{21}C'est ici le compte des choses qui furent employées au tabernacle, à savoir à la tente d'assignation, selon que le compte en fut fait par l'ordre de Moïse, à quoi furent employés les Lévites, sous la conduite d'Ithamar, fils du prêtre Aaron.
\VS{22}Et Betsaleel, fils d'Uri, fils de Hur, de la tribu de Juda, fit toutes les choses que Yahweh avait ordonnées à Moïse~;
\VS{23}et avec lui Oholiab, fils d'Ahisamac, de la tribu de Dan, les ouvriers, et ceux qui travaillaient en ouvrage exquis, et les brodeurs en pourpre, en écarlate, en cramoisi, et en fin lin.
\VS{24}Tout l'or qui fut employé pour l'ouvrage, à savoir pour tout l'ouvrage du sanctuaire, qui était l'or des offrandes, fut de vingt-neuf talents et sept cent trente sicles, selon le sicle du sanctuaire.
\VS{25} Et l'argent de ceux de l'assemblée qui furent dénombrés fut de cent talents et mille sept cent soixante-quinze sicles, selon le sicle du sanctuaire.
\VS{26}Un demi-sicle par tête, la moitié d'un sicle selon le sicle du sanctuaire. Tous ceux qui passèrent par le dénombrement depuis l'âge de vingt ans et au-dessus, furent six cent trois mille cinq cent cinquante.
\VS{27}Il y eut donc cent talents d'argent pour fondre les bases du sanctuaire, et les bases du voile, à savoir cent bases de cent talents, un talent pour chaque base.
\VS{28}Mais des mille sept cent soixante-quinze sicles, il fit les crochets pour les colonnes, et il couvrit leurs chapiteaux et en fit des filets tout autour.
\VS{29}L'airain des offrandes fut de soixante-dix talents et deux mille quatre cents sicles~;
\VS{30}dont on fit les bases de la porte de la tente d'assignation, et l'autel d'airain avec sa grille d'airain, et tous les ustensiles de l'autel~;
\VS{31}et les bases tout autour du parvis, les bases de la porte du parvis, et tous les pieux du tabernacle, et tous les pieux du parvis tout autour.
\Chap{39}
\TextTitle{Les vêtements sacrés d'Aaron}
\VerseOne{}Ils firent aussi de pourpre, d'écarlate, et de cramoisi les vêtements du service, pour faire le service du sanctuaire. Et ils firent les saints vêtements sacrés pour Aaron, comme Yahweh l'avait ordonné à Moïse\FTNT{Ex. 28.}.
\VS{2}On fit donc l'éphod d'or, de pourpre, d'écarlate, de cramoisi, et de fin lin retors.
\VS{3}Or on étendit des lames d'or, et on les coupa par filets pour les brocher parmi la pourpre, l'écarlate, le cramoisi, et le fin lin d'ouvrage exquis.
\VS{4}On fit à l'éphod des épaulettes\FTNT{Voir annexe «~Les habits du grand-prêtre~».} qui s'attachaient, en sorte qu'il était joint par ses deux extrémités.
\VS{5}Et la ceinture exquise de laquelle il était ceint, était tirée de lui, et de même ouvrage, d'or, de pourpre, d'écarlate, de cramoisi, et de fin lin retors, comme Yahweh l'avait ordonné à Moïse.
\VS{6}On enchassa aussi les pierres d'onyx dans leurs montures d'or, ayant les noms des enfants d'Israël gravés comme on grave les cachets.
\VS{7}Et on les mit sur les épaulettes de l'éphod, afin qu'elles soient des pierres de souvenir pour les enfants d'Israël, comme Yahweh l'avait ordonné à Moïse.
\VS{8}On fit aussi le pectoral\FTNT{Voir annexe «~Les habits du grand-prêtre~».} d'ouvrage exquis, comme l'ouvrage de l'éphod, d'or, de pourpre, d'écarlate, de cramoisi, et de fin lin retors.
\VS{9}On fit le pectoral carré et double~; sa longueur était d'une paume, et sa largeur d'une paume de part et d'autre.
\VS{10}Et on le garnit de quatre rangs de pierres~: A la première rangée on mit une sardoine, une topaze et une émeraude.
\VS{11}A la seconde rangée une escarboucle, un saphir, et un jaspe.
\VS{12}A la troisième rangée, une opale, une agate, et une améthyste.
\VS{13}A la quatrième rangée, un chrysolithe, un onyx, et un béryl\FTNT{Ap. 21:18-19.}, enchâssés dans leur monture d'or.
\VS{14}Ainsi, il y avait autant de pierres qu'il y avait de noms des enfants d'Israël, douze selon leurs noms, chacune d'elles gravées comme des cachets, selon le nom qu'elle devait porter, et elles étaient pour les douze tribus.
\VS{15}Et on fit sur le pectoral des chaînettes à bouts en façon de cordon, d'or pur.
\VS{16}On fit aussi deux montures d'or et deux anneaux d'or, on mit les deux anneaux aux deux extrémités du pectoral.
\VS{17}Et on mit les deux chaînettes d'or faites à cordon dans les deux anneaux à l'extrémité du pectoral.
\VS{18}Et on mit les deux autres bouts des deux chaînettes faites à cordon aux deux montures, sur les épaulettes de l'éphod, sur le devant de l'éphod.
\VS{19}On fit aussi deux autres anneaux d'or, et on les mit aux deux autres extrémités du pectoral sur son bord, qui était du côté de l'éphod à l'intérieur.
\VS{20}On fit aussi deux autres anneaux d'or, et on les mit aux deux épaulettes de l'éphod par le bas, répondant sur le devant de l'éphod, à l'endroit où il se joignait au-dessus de la ceinture exquise de l'éphod.
\VS{21}Et on joignit le pectoral élevé par ses anneaux aux anneaux de l'éphod, avec un cordon de pourpre, afin qu'il tienne au-dessus de la ceinture exquise de l'éphod, et que le pectoral ne bouge de dessus l'éphod, comme Yahweh l'avait ordonné à Moïse.
\VS{22}On fit aussi la robe de l'éphod d'ouvrage tissé et entièrement de pourpre.
\VS{23} Et l'ouverture pour passer la tête était au milieu de la robe, comme l'ouverture d'une cotte de mailles~; et il y avait un ourlet à l'ouverture de la robe tout autour, afin qu'elle ne se déchire pas.
\VS{24}Aux bordures de la robe, on fit des grenades de pourpre, d'écarlate et de cramoisi, à fil retors.
\VS{25}On fit aussi des clochettes d'or pur~; et on mit les clochettes entre les grenades aux bordures de la robe tout autour, parmi les grenades~;
\VS{26}à savoir une clochette puis une grenade, une clochette puis une grenade, sur la bordure de la robe tout autour, pour faire le service, comme Yahweh l'avait ordonné à Moïse.
\VS{27}On fit aussi à Aaron et à ses fils des tuniques de fin lin d'ouvrage tissé.
\VS{28}Et la tiare de fin lin, et les ornements des calottes de fin lin, et les caleçons de lin, de fin lin retors.
\VS{29}Et la ceinture de fin lin retors, de pourpre, d'écarlate et de cramoisi, d'ouvrage de broderie~; comme Yahweh l'avait ordonné à Moïse~;
\VS{30}et la lame du saint diadème d'or pur, sur laquelle on écrivit comme on grave un cachet~: La sainteté à Yahweh.
\VS{31}Et on mit sur elle un cordon de pourpre, pour l'appliquer à la tiare par-dessus, comme Yahweh l'avait ordonné à Moïse.
\TextTitle{Le matériel pour excercer la prêtrise est prêt}
\VS{32}Ainsi fut achevé tout l'ouvrage du tabernacle, de la tente d'assignation. Les enfants d'Israël firent selon toutes les choses que Yahweh avait ordonnées à Moïse~; ils les firent ainsi.
\VS{33}Et ils apportèrent à Moïse le tabernacle, la tente, et tous ses ustensiles, ses crochets, ses planches, ses barres, ses colonnes, et ses bases~;
\VS{34}la couverture de peaux de béliers teintes en rouge, la couverture de peaux de taissons, et le voile qui sert de rideau devant le Saint des saints~;
\VS{35}l'arche du témoignage et ses barres, et le propitiatoire~;
\VS{36}la table avec tous ses ustensiles, et les pains de proposition\FTNT{Ex. 31:8-10.}~;
\VS{37}et le chandelier d'or pur avec toutes ses lampes arrangées, et tous ses ustensiles, et l'huile du chandelier~;
\VS{38}et l'autel d'or, l'huile d'onction, le parfum odoriférant, et le rideau de l'entrée de la tente~;
\VS{39}l'autel d'airain, avec sa grille d'airain, ses barres et tous ses ustensiles~; la cuve et sa base~;
\VS{40}et les courtines du parvis, ses colonnes, ses bases, le rideau pour la porte du parvis, son cordage, ses pieux, et tous les ustensiles pour le service du tabernacle, pour la tente d'assignation~;
\VS{41}les vêtements du service pour faire le service du sanctuaire, les saints vêtements pour le prêtre Aaron, et les vêtements de ses fils pour exercer la prêtrise.
\VS{42}Les enfants d'Israël donc firent tout l'ouvrage, comme Yahweh l'avait ordonné à Moïse.
\VS{43}Et Moïse vit tout l'ouvrage, et voici, on l'avait fait ainsi que Yahweh l'avait ordonné, on l'avait, dis-je, fait ainsi. Et Moïse les bénit.
\Chap{40}
\TextTitle{Moïse dresse le tabernacle}
\VerseOne{}Et Yahweh parla à Moïse, en disant~:
\VS{2}Au premier jour du premier mois, tu dresseras le tabernacle de la tente d'assignation.
\VS{3}Et tu y mettras l'arche du témoignage, au-devant de laquelle tu tendras le voile.
\VS{4}Puis tu apporteras la table et y arrangeras ce qui doit y être arrangé. Tu apporteras aussi le chandelier et allumeras ses lampes.
\VS{5}Tu mettras aussi l'autel d'or pour le parfum au-devant de l'arche du témoignage, et tu mettras le rideau de l'entrée au tabernacle.
\VS{6}Tu mettras aussi l'autel de l'holocauste vis-à-vis de l'entrée du tabernacle de la tente d'assignation.
\VS{7}Tu mettras aussi la cuve entre la tente d'assignation et l'autel, et y mettras de l'eau.
\VS{8}Tu mettras aussi le parvis tout autour, et tu mettras le rideau à la porte du parvis.
\VS{9}Tu prendras aussi l'huile de l'onction, et tu en oindras le tabernacle, et tout ce qui y est, et tu le sanctifieras avec tous ses ustensiles~; et il sera saint.
\VS{10}Tu oindras aussi l'autel de l'holocauste, et tous ses ustensiles, et tu sanctifieras l'autel, et l'autel sera très saint.
\VS{11}Tu oindras aussi la cuve et sa base, et la sanctifieras.
\VS{12}Tu feras aussi approcher Aaron et ses fils à l'entrée de la tente d'assignation, et les laveras avec de l'eau.
\VS{13}Et tu feras vêtir à Aaron les saints vêtements, et tu l'oindras et le sanctifieras~; et il exercera la prêtrise pour moi.
\VS{14}Tu feras aussi approcher ses fils que tu revêtiras de tuniques.
\VS{15} Et tu les oindras comme tu auras oint leur père~; et ils m'exerceront la prêtrise, et leur onction leur sera pour exercer la prêtrise à toujours parmi leur génération.
\VS{16} Ce que Moïse fit selon toutes les choses que Yahweh lui avait ordonnées~; il le fit ainsi.
\VS{17}Car au premier jour du premier mois de la seconde année, le tabernacle fut dressé.
\VS{18}Moïse donc dressa le tabernacle, mit ses bases, posa ses planches, mit ses barres et dressa ses colonnes.
\VS{19}Et il étendit la tente sur le tabernacle, et mit la couverture de la tente au-dessus du tabernacle par le haut, comme Yahweh l'avait ordonné à Moïse.
\VS{20}Puis il prit et posa le témoignage dans l'arche et mit les barres à l'arche~; il mit aussi le propitiatoire au-dessus de l'arche.
\VS{21}Et il apporta l'arche dans le tabernacle, et posa le voile qui sert de rideau, et le mit au-devant de l'arche du témoignage, comme Yahweh l'avait ordonné à Moïse.
\VS{22}Il mit aussi la table dans la tente d'assignation, au côté du tabernacle vers le nord, en dehors du voile.
\VS{23}Et il arrangea sur elle les rangées de pains devant Yahweh, comme Yahweh l'avait ordonné à Moïse.
\VS{24}Il mit aussi le chandelier dans la tente d'assignation, vis-à-vis de la table, du côté du tabernacle, vers le sud.
\VS{25}Et il alluma les lampes devant Yahweh, comme Yahweh l'avait ordonné à Moïse.
\VS{26}Il posa aussi l'autel d'or dans la tente d'assignation, devant le voile.
\VS{27}Et il fit fumer sur lui le parfum odoriférant, comme Yahweh l'avait ordonné à Moïse.
\VS{28}Il mit aussi le rideau de l'entrée du tabernacle.
\VS{29}Et il mit l'autel de l'holocauste à l'entrée du tabernacle de la tente d'assignation~; et offrit sur lui l'holocauste et l'offrande, comme Yahweh l'avait ordonné à Moïse.
\VS{30}Et il plaça la cuve entre la tente d'assignation et l'autel, et y mit de l'eau pour se laver.
\VS{31}Et Moïse et Aaron avec ses fils en lavèrent leurs mains et leurs pieds.
\VS{32}Et quand ils entraient dans la tente d'assignation, et qu'ils approchaient de l'autel, ils se lavaient, selon que Yahweh l'avait ordonné à Moïse.
\VS{33}Il dressa aussi le parvis tout autour du tabernacle et de l'autel, et tendit le rideau de la porte du parvis. Ainsi Moïse acheva l'ouvrage.
\TextTitle{La gloire de Yahweh sur le tabernacle}
\VS{34}Et la nuée couvrit la tente d'assignation, et la gloire de Yahweh remplit le tabernacle\FTNT{No. 9:15~; 1 R. 8:10.},
\VS{35}tellement que Moïse ne put entrer dans la tente d'assignation, car la nuée se tenait dessus et la gloire de Yahweh remplissait le tabernacle.
\VS{36} Or quand la nuée se levait de dessus le tabernacle, les enfants d'Israël partaient dans toutes leurs marches.
\VS{37}Mais si la nuée ne se levait point, ils ne partaient point, jusqu'au jour où elle se levait.
\VS{38}Car la nuée de Yahweh était le jour sur le tabernacle, et le feu y était la nuit, devant les yeux de toute la maison d'Israël, dans toutes leurs marches.
\PPE
\end{multicols}

%\clearpage\ShortTitle{Lé.}\BookTitle{Lévitique}\BFont
\noindent\hrulefill
{\footnotesize
\textit{
\bigskip
{\centering{}
\\Auteur : Probablement Moïse
\\(Heb. : Vayiqra)
\\Signification : Et Il (Yahweh) appela
\\Thème : La sainteté
\\Date de rédaction : Env. 1450-1410 av. J.-C.\\}
}
%\bigskip
\textit{
\\Après avoir construit et dressé le tabernacle selon le modèle que Yahweh avait donné à Moïse, les fils d'Israël reçurent le détail des prescriptions relatives aux offrandes, aux sacrifices et aux fêtes en l'honneur de Yahweh. 
%\bigskip
\\Ce livre, dont le nom tire son origine de Lévi, explique la manière dont Aaron et ses fils devaient exercer la sacrificature et amener le peuple à s'approcher de Dieu dans le respect de ses ordonnances.
%\bigskip
\\Les lois que Moïse avait recueillies présentent la voie du pardon, laquelle est impossible sans effusion de sang. Bien que les mêmes sacrifices furent réitérés tous les ans, ces préceptes mettaient en évidence l'impuissance de l'homme à atteindre la justice de Dieu par ses propres moyens.\bigskip
}
}
\par\nobreak\noindent\hrulefill
\begin{multicols}{2}
\Chap{1}
\TextTitle{L'holocauste\FTNTT{voir Lé. 6:1-6.}}
\VerseOne{}Et Yahweh appela Moïse, et lui parla de la tente d'assignation, en disant :
\VS{2}Parle aux enfants d'Israël, et dis-leur : Quand quelqu'un d'entre vous offrira à Yahweh une offrande d'une bête à quatre pattes, il fera son offrande de gros ou de menu bétail.
\VS{3}Si son offrande pour un holocauste est de gros bétail, il offrira un mâle sans défaut\FTNT{L'holocauste était le sacrifice pour l'expiation par excellence. Contrairement aux autres sacrifices, l'holocauste était entièrement consumé sur l'autel. Il symbolisait d'une part le sacrifice parfait de Christ et d'autre part notre vie, volontairement offerte à Dieu (Ro. 12:1). Les animaux aptes à être offerts en holocauste devaient être des mâles sans défaut :
\\- Le veau (Lé. 1:5), image de Christ, l'humble serviteur, soumis et obéissant (Mt. 20:28 ; Ph. 2:5-8).
\\- L'agneau ou le chevreau, image de Christ qui livre sa vie à la croix sans résistance ni contestation, et qui prend sur lui nos péchés (Es. 53:7 ; Mt. 26:63 ; Ac. 8:32). 
\\- Les tourterelles ou les jeunes pigeons, image de la simplicité de Christ (Mt. 10:16).
\\Toutes les étapes de la réalisation de ce sacrifice enseignent le disciple sur la mort à soi-même et le dépouillement des œuvres de la chair (Ga. 5:19-21).
\\Le sang de l'animal égorgé devait être répandu sur l'autel (Lé. 1:5), image de la croix. L'âme (contenue dans le sang selon Lé. 17:14), liée à la chair et ses désirs, doit être crucifiée (Ga. 2:20 ; Ga. 5:24). L'objet de la mise à mort était certainement un couteau tranchant comme une épée, image de la Parole de Dieu (Hé. 4:12). La mise en pratique de la Parole nous amène nécessairement à nous séparer du monde et à renoncer à soi-même.} ; il l'offrira de son bon gré à l'entrée de la tente d'assignation ; devant Yahweh\FTNT{Ex. 29:10-11.}.
\VS{4}Et il posera sa main sur la tête de l'holocauste, et il sera agréé pour lui, afin de faire la propitiation pour lui.
\VS{5}Puis, on égorgera le jeune taureau devant Yahweh ; et les fils d'Aaron, les prêtres, en offriront le sang et ils répandront le sang sur l'autel tout autour, qui est à l'entrée de la tente d'assignation.
\VS{6}Et on égorgera l'holocauste et le coupera en morceaux.
\VS{7}Les fils du prêtre Aaron mettront le feu sur l'autel, et disposeront le bois sur le feu.
\VS{8}Et les fils d'Aaron, les prêtres, poseront les morceaux, la tête et la graisse sur le bois qui sera au feu sur l'autel.
\VS{9}Mais il lavera avec de l'eau les entrailles et les jambes ; et le prêtre brûlera toutes ces choses sur l'autel. C'est un holocauste, un sacrifice consumé par le feu, d'une bonne odeur à Yahweh.
\VS{10}Si son offrande est un holocauste de menu bétail, d'entre les agneaux ou d'entre les chèvres, il offrira un mâle sans défaut.
\VS{11}Et on l'égorgera à côté de l'autel, vers le nord, devant Yahweh ; et les prêtres, fils d'Aaron, en répandront le sang sur l'autel tout autour.
\VS{12}Puis on le coupera en morceaux, avec sa tête et sa graisse ; et le prêtre les posera sur le bois qui sera au feu sur l'autel.
\VS{13}Mais il lavera avec de l'eau les entrailles et les jambes. Puis le prêtre offrira toutes ces choses, et les brûlera sur l'autel. C'est un holocauste, un sacrifice consumé par le feu, d'une agréable odeur à Yahweh\FTNT{Ez. 40:38.}.
\VS{14}Si son offrande à Yahweh est un holocauste d'oiseaux, il offrira son offrande de tourterelles, ou de jeunes pigeons.
\VS{15}Le prêtre l'apportera sur l'autel, lui ouvrira la tête avec l'ongle, la brûlera sur l'autel, et il en exprimera le sang contre un côté de l'autel.
\VS{16}Il ôtera son jabot avec ses plumes, et le jettera près de l'autel, vers l'orient, dans le lieu où seront les cendres.
\VS{17}Il le déchirera avec ses ailes, sans le séparer ; et le prêtre le brûlera sur l'autel, sur le bois qui sera au feu. C'est un holocauste, un sacrifice consumé par le feu, d'une agréable odeur à Yahweh.
\Chap{2}
\TextTitle{L'offrande de gâteau\FTNTT{Lé. 6:7-16.}}
\VerseOne{}Lorsque quelqu'un offrira l'offrande de gâteau\FTNT{L'offrande de farine ou de gâteau correspond aux perfections de la vie du Seigneur Jésus-Christ en tant qu'homme. Ce sacrifice ne comporte ni victime ni sang, mais seulement de la farine, de l'huile, de l'encens et du sel. Jésus, le grain de blé (Jn. 12:24), a été complètement broyé, pétri et oint d'huile, éprouvé par toutes sortes de douleurs. Sa vie sainte était pour le Père un parfum de bonne odeur. Son amour pour les âmes, sa dépendance totale au Père, sa persévérance, sa douceur, sa sagesse et sa bonté, n'ont pas varié malgré toutes les souffrances par lesquelles il est passé. Voilà quelques-uns des fruits admirables qui correspondent à l'offrande de gâteau saupoudrée d'encens. Le levain, image du péché (1 Co. 5:6-8), n'y entrait pas, ni le miel, symbole des affections humaines (Pr. 5:3). Quant au sel, il préserve de la corruption des aliments, il est comparé à la saveur des disciples de Christ (Mt. 5:13).} à Yahweh, son offrande sera de fine farine ; il versera de l'huile dessus, et mettra de l'encens.
\VS{2}Il l'apportera aux fils d'Aaron, les prêtres, et le prêtre prendra une pleine poignée de cette fine farine, et d'huile, avec tout l'encens, et il brûlera son souvenir\FTNT{En hébreu « azkarah », offrande de souvenir, la portion de nourriture offerte et qui est consumée.} sur l'autel. C'est une offrande d'une bonne odeur à Yahweh.
\VS{3}Ce qui restera du gâteau sera pour Aaron et ses fils ; c'est une chose très sainte parmi les offrandes consumées par le feu à Yahweh.
\VS{4}Et quand tu offriras une offrande de gâteaux cuits au four, ce sera de fine farine, des gâteaux sans levain, pétris avec de l'huile, et des galettes sans levain, ointes d'huile.
\VS{5}Si ton offrande est un gâteau cuit sur la plaque, elle sera de fine farine pétrie à l'huile, sans levain.
\VS{6}Tu la rompras en morceaux, et tu verseras de l'huile sur elle ; c'est une offrande de gâteau.
\VS{7}Si ton offrande est un gâteau cuit sur le gril, elle sera faite de fine farine avec de l'huile.
\VS{8}Puis tu apporteras à Yahweh l'offrande de gâteaux qui sera faite de ces choses, et on la présentera au prêtre, qui l'apportera sur l'autel.
\VS{9}Le prêtre lèvera de l'offrande de gâteaux, son souvenir, et le brûlera sur l'autel. C'est une offrande consumée par le feu de bonne odeur à Yahweh.
\VS{10}Ce qui restera de l'offrande de gâteau sera pour Aaron et ses fils ; c'est une chose très sainte parmi les offrandes consumées par le feu devant Yahweh.
\VS{11}Aucune offrande de gâteau que vous offrirez à Yahweh ne sera faite avec du levain ; car vous ne brûlerez point de levain ni de miel, parmi l'offrande consumée par le feu devant Yahweh.
\VS{12}Vous pourrez bien les offrir à Yahweh dans l'offrande des prémices, mais ils ne seront point mis sur l'autel comme offrande d'une bonne odeur.
\VS{13}Tu mettras du sel\FTNT{Voir No. 18:19 ; 2 Ch. 13:5. Le sel est un agent purificateur (2 R. 2:19-22). Le sel préserve de la corruption et conserve les aliments. Les chrétiens sont le sel de la terre (Mt. 5:13). Nos paroles doivent être assaisonnées de sel (Col. 4:6).} sur toutes tes offrandes de gâteaux, et tu ne laisseras point ton offrande de gâteau manquer de sel, signe de l'alliance de ton Dieu ; mais sur toutes tes offrandes, tu offriras du sel.
\VS{14}Si tu offres à Yahweh une offrande de gâteau des premiers fruits, tu offriras, pour l'offrande de gâteau des premiers fruits, des épis qui commencent à mûrir, rôtis au feu, les grains de quelques épis bien grenés, broyés entre les mains.
\VS{15}Puis tu mettras de l'huile sur le gâteau, et tu mettras aussi de l'encens dessus : C'est une offrande de gâteaux.
\VS{16}Et le prêtre brûlera son souvenir, pris de ses grains broyés, et de son huile avec tout l'encens. C'est une offrande consumée par le feu à Yahweh.
\Chap{3}
\TextTitle{Le sacrifice d'offrande de paix\FTNTT{Lé. 7:11-21.}}
\VerseOne{}Si son offrande est un sacrifice d'offrande de paix\FTNT{La plupart des traducteurs ont traduit par « sacrifice d'actions de grâces », or l'étymologie hébraïque du mot grâce est « shelem », ce qui signifie d'abord « paix ». Ce terme peut aussi vouloir dire « remerciement » ou « reconnaissance ». La racine de « shelem » est « shalam » : « être dans une alliance de paix », « être en paix ».Il est donc question ici d'une offrande de paix qui préfigure l'ensemble de l'œuvre de la croix accomplie par le Messie, et grâce à laquelle nous sommes réconciliés avec le Père (Col. 1:20 ; Ep. 2:14-17). Cette offrande préfigure aussi la Pâque incarnée par le Messie (1 Co. 5:7) ainsi que le repas du Seigneur. En effet, sur cette offrande, Dieu prenait pour lui la graisse et la queue entière (Lé. 3:3 ; Lé. 3:9-17), le prêtre prenait la poitrine et l'épaule droite (Lé. 7:31-34), et celui qui offrait l'animal pouvait consommer le reste avec d'autres personnes pures (Lé. 7:20). Ainsi, comme pour le repas du Seigneur, tous ceux qui étaient saints pouvaient participer au repas (1 Co. 11:27-34).}, et qu'il offre du gros bétail, soit mâle, soit femelle, il l'offrira sans défaut devant Yahweh.
\VS{2}Il posera sa main sur la tête de son offrande, et l'égorgera à l'entrée de la tente d'assignation, et les fils d'Aaron, les prêtres, répandront le sang sur l'autel tout autour.
\VS{3}Puis on offrira de cette offrande de paix, un sacrifice consumé par le feu à Yahweh, à savoir la graisse qui couvre les entrailles et toute la graisse qui est sur les entrailles ;
\VS{4}les deux rognons avec la graisse qui est dessus et qui est sur les flancs ; et on ôtera le grand lobe qui est sur le foie pour le mettre avec les rognons.
\VS{5}Les fils d'Aaron brûleront tout cela sur l'autel, sur l'holocauste, qui sera sur le bois mis au feu. C'est une offrande consumée par le feu d'agréable odeur à Yahweh\FTNT{Ex. 29:13-25.}.
\VS{6}Si son offrande pour le sacrifice d'offrande de paix à Yahweh est de menu bétail, soit mâle, soit femelle, il l'offrira sans défaut.
\VS{7}S'il offre un agneau pour son offrande, il l'offrira devant Yahweh.
\VS{8}Il posera sa main sur la tête de son offrande, et l'égorgera devant la tente d'assignation, et les fils d'Aaron répandront son sang sur l'autel tout autour.
\VS{9}De ce sacrifice d'offrande de paix, il offrira en offrande consumée par le feu à Yahweh, sa graisse et sa queue entière, séparée jusqu'à l'échine, avec la graisse qui couvre les entrailles et toute la graisse qui est sur les entrailles,
\VS{10}les deux rognons avec la graisse qui est dessus, sur les flancs, et il ôtera le grand lobe qui est sur le foie, jusqu'aux rognons.
\VS{11}Le prêtre brûlera tout cela sur l'autel. C'est un aliment d'offrande consumée par le feu à Yahweh\FTNT{No. 28:2.}.
\VS{12}Si son offrande est une chèvre, il l'offrira devant Yahweh.
\VS{13}Il posera sa main sur sa tête, et l'égorgera devant la tente d'assignation ; et les fils d'Aaron répandront son sang sur l'autel tout autour.
\VS{14}Puis il offrira son offrande en sacrifice consumé par le feu à Yahweh, la graisse qui couvre les entrailles et toute la graisse qui est sur les entrailles,
\VS{15}les deux rognons, et la graisse qui est dessus, sur les flancs, et il ôtera le grand lobe qui est sur le foie, jusqu'aux rognons.
\VS{16}Puis le prêtre brûlera toutes ces choses sur l'autel. C'est un aliment d'offrande consumée par le feu de bonne odeur. Toute graisse appartient à Yahweh.
\VS{17}C'est une loi perpétuelle pour vos descendants, dans toutes vos demeures : Vous ne mangerez ni graisse ni sang\FTNT{Ge. 9:4 ; 1 S. 14:33.}.
\Chap{4}
\TextTitle{Le sacrifice pour l'expiation\FTNTT{Lé. 6:17-23.}}
\VerseOne{}Yahweh parla encore à Moïse en disant :
\VS{2}Parle aux enfants d'Israël, et dis-leur : Quand une personne aura péché involontairement\FTNT{Avant la promulgation de la loi, certains hommes péchaient par ignorance (Ro. 5:13). Néanmoins, ces péchés étaient tout de même punis et nécessitaient un sacrifice (Lé. 4:13-14. No. 15:22-36 ; Job. 1). Sous la grâce, l'excuse du péché par ignorance ne peut être invoquée puisque nous sommes scellés du Saint-Esprit qui nous enseigne toutes choses (1 Jn. 2:20 et 27).} contre l'un des commandements de Yahweh, en commettant des choses qui ne doivent point se faire, et qu'il aura fait une de ces choses ;
\VS{3} si c'est le prêtre oint qui ait commis un péché, semblable à quelque faute du peuple, il offrira à Yahweh pour son péché qu'il aura fait, un jeune taureau sans défaut, pris du troupeau en sacrifice pour l'expiation.
\VS{4}Il amènera le taureau à l'entrée de la tente d'assignation, devant Yahweh, il posera sa main sur la tête du taureau, et l'égorgera devant Yahweh.
\VS{5}Et le prêtre oint prendra du sang du taureau, et l'apportera dans la tente d'assignation.
\VS{6}Le prêtre trempera son doigt dans le sang, et fera sept fois l'aspersion du sang devant Yahweh, en face du voile du lieu saint\FTNT{No. 19:4.}.
\VS{7}Le prêtre mettra aussi devant Yahweh du sang sur les cornes de l'autel des parfums odoriférants, qui est dans la tente d'assignation ; et il répandra tout le reste du sang du taureau au pied de l'autel de l'holocauste, qui est à l'entrée de la tente d'assignation.
\VS{8}Il enlèvera toute la graisse du taureau du sacrifice pour l'expiation, à savoir, la graisse qui couvre les entrailles, et toute la graisse qui est sur les entrailles,
\VS{9}et les deux rognons avec la graisse qui les entoure, qui couvre les flancs, et il ôtera le grand lobe qui est sur le foie, pour le mettre sur les rognons.
\VS{10}Comme on les enlève du taureau du sacrifice d'offrande de paix\FTNT{Voir commentaire en Lé. 3:1.}, et le prêtre brûlera toutes ces choses-là sur l'autel de l'holocauste.
\VS{11}Mais quant à la peau du taureau et toute sa chair, avec sa tête, ses jambes, ses entrailles, et ses excréments,
\VS{12}et même tout le taureau, il l'emportera hors du camp, dans un lieu pur, où l'on répand les cendres, et il le brûlera au feu sur du bois : Il sera brûlé au lieu où l'on répand les cendres.
\VS{13}Et si toute l'assemblée d'Israël a péché involontairement, et que la chose soit restée cachée aux yeux de l'assemblée, et qu'ils aient violé l'un des commandements de Yahweh, en commettant des choses qui ne doivent pas se faire, et s'en soit rendu coupable,
\VS{14}et que le péché qu'ils ont fait vienne en évidence, l'assemblée offrira en sacrifice pour l'expiation un jeune taureau pris du troupeau, et on l'amènera devant la tente d'assignation.
\VS{15}Les anciens de l'assemblée poseront leurs mains sur la tête du taureau devant Yahweh, et on égorgera le taureau devant Yahweh.
\VS{16}Et le prêtre oint, apportera du sang du taureau dans la tente d'assignation ;
\VS{17}ensuite le prêtre trempera son doigt dans le sang, et en fera aspersion devant Yahweh en face du voile, par sept fois.
\VS{18}Et il mettra du sang sur les cornes de l'autel, qui est devant Yahweh dans la tente d'assignation ; et il répandra tout le reste du sang au pied de l'autel de l'holocauste, qui est à l'entrée de la tente d'assignation.
\VS{19}Il enlèvera toute sa graisse et la brûlera sur l'autel.
\VS{20}Et il fera de ce taureau comme il l'a fait du taureau pour le sacrifice d'expiation. Le prêtre fera ainsi ; il fera propitiation pour eux, et il leur sera pardonné.
\VS{21}Puis il emportera le taureau hors du camp, et le brûlera comme il a brûlé le premier taureau. Car c'est le sacrifice pour l'expiation de l'assemblée.
\VS{22}Que si un chef a péché involontairement, en violant l'un des commandements de Yahweh son Dieu, ce qui ne doit point se faire, et s'en soit rendu coupable,
\VS{23}et qu'on vienne à connaître le péché qu'il a commis, il amènera pour sacrifice un jeune bouc, mâle, sans défaut ;
\VS{24}et il posera sa main sur la tête du bouc, et l'égorgera au lieu où l'on égorge l'holocauste devant Yahweh. C'est un sacrifice pour expiation.
\VS{25}Puis le prêtre prendra avec son doigt du sang de l'offrande pour l'expiation, et le mettra sur les cornes de l'autel de l'holocauste, et il répandra le reste de son sang au pied de l'autel de l'holocauste.
\VS{26}Et il brûlera toute sa graisse sur l'autel, comme la graisse du sacrifice d'offrande de paix. Ainsi le prêtre fera propitiation pour lui de son péché, et il lui sera pardonné.
\VS{27}Que si quelqu'un du peuple du pays a péché involontairement, en violant l'un des commandements de Yahweh, et en commettant des choses qui ne doivent point se faire, et s'en soit rendu coupable,
\VS{28}et qu'on vienne à connaître le péché qu'il a commis, il amènera pour offrande une jeune chèvre, femelle, sans défaut, pour le péché qu'il a commis.
\VS{29}Et il posera sa main sur la tête de l'offrande pour le péché, et égorgera l'offrande pour l'expiation au lieu où l'on égorge l'holocauste.
\VS{30}Puis le prêtre prendra du sang de la chèvre avec son doigt, et le mettra sur les cornes de l'autel de l'holocauste, et il répandra tout le reste de son sang au pied de l'autel.
\VS{31}Et il ôtera toute sa graisse, comme on ôte la graisse de dessus le sacrifice d'offrande de paix, et le prêtre la brûlera sur l'autel, en bonne odeur à Yahweh. Il fera propitiation pour lui, et il lui sera pardonné.
\VS{32}Que s'il amène un agneau comme offrande, pour le sacrifice d'expiation, il amènera une femelle sans défaut.
\VS{33}Et il posera sa main sur la tête de l'offrande d'expiation, et on l'égorgera en sacrifice pour l'expiation au lieu où l'on égorge l'holocauste.
\VS{34}Puis le prêtre prendra avec son doigt du sang de l'offrande pour l'expiation, et le mettra sur les cornes de l'autel de l'holocauste, et il répandra tout le reste de son sang au pied de l'autel.
\VS{35}Et il ôtera toute sa graisse, comme on ôte la graisse de l'agneau du sacrifice d'offrande de paix, et le prêtre la brûlera sur l'autel, par-dessus les sacrifices de Yahweh consumés par le feu, et il fera propitiation pour lui, pour son péché qu'il aura commis, et il lui sera pardonné.
\Chap{5}
\TextTitle{Le sacrifice de culpabilité\FTNTT{Lé. 7:1-7.}}
\VerseOne{}Et quand quelqu'un, étant témoin, après avoir entendu la parole du serment, aura péché en ne déclarant pas ce qu'il a vu ou ce qu'il sait, il portera son iniquité\FTNT{Pr. 29:24.}.
\VS{2}Et quand quelqu'un, à son insu, aura touché une chose souillée, soit le cadavre d'un animal impur, soit le cadavre d'une bête sauvage impure, soit le cadavre d'un reptile impur, il sera souillé et coupable\FTNT{Ag. 2:14 ; 2 Co. 6:17.}.
\VS{3}Ou quand il aura touché à l'impureté d'un homme, quelle que soit son impureté par laquelle il se rend impur, et que cela lui soit resté caché, quand il le sait, alors il est coupable.
\VS{4}Ou quand quelqu'un, parlant légèrement de ses lèvres, a juré de faire du mal ou du bien, selon tout ce que l'homme profère légèrement en jurant, et que cela lui soit resté caché, quand il le sait, alors il est coupable dans l'un de ces points-là.
\VS{5}Quand donc quelqu'un sera coupable sur l'un de ces points là, il confessera ce en quoi il aura péché.
\VS{6}Et il amènera son sacrifice de culpabilité à Yahweh pour le péché qu'il a commis, à savoir, une femelle du menu bétail, soit une brebis, soit une chèvre, pour l'offrande d'expiation. Et le prêtre fera pour lui propitiation de son péché.
\VS{7}Et s'il n'a pas le moyen de trouver une brebis ou une chèvre, il apportera en offrande pour le péché à Yahweh, pour sa culpabilité, deux tourterelles ou deux jeunes pigeons, l'un comme sacrifice pour l'expiation, l'autre pour l'holocauste\FTNT{Lu. 2:24.}.
\VS{8}Il les apportera au prêtre, qui offrira premièrement celui qui est pour l'offrande d'expiation. Il leur ouvrira la tête avec l'ongle, près du cou, sans la séparer ;
\VS{9}puis il fera l'aspersion du sang du sacrifice d'expiation sur un côté de l'autel, et ce qui restera du sang sera exprimé au pied de l'autel : C'est un sacrifice pour l'expiation.
\VS{10}Et il fera de l'autre un holocauste, selon l'ordonnance. Et le prêtre fera pour lui la propitiation pour son péché qu'il aura commis, et il lui sera pardonné.
\VS{11}Si celui qui aura péché n'a pas le moyen de trouver deux tourterelles ou deux jeunes pigeons, il apportera pour son offrande un dixième d'épha de fine farine en offrande pour le sacrifice d'expiation ; il ne mettra ni huile ni encens, car c'est un sacrifice d'expiation.
\VS{12}Il l'apportera au prêtre, et le prêtre qui en prendra une pleine poignée pour souvenir\FTNT{Lé. 2:2.}, la brûlera sur l'autel, comme offrande consumée par le feu à Yahweh : C'est un sacrifice d'expiation.
\VS{13}Ainsi le prêtre fera propitiation pour lui, pour le péché qu'il a commis dans l'une de ces choses, et il lui sera pardonné. Le reste sera pour le prêtre, comme étant une offrande de gâteau.
\VS{14}Yahweh parla aussi à Moïse, en disant :
\VS{15}Quand quelqu'un aura commis une transgression et péchera involontairement, en retenant des choses consacrées à Yahweh, il amènera en sacrifice de culpabilité à Yahweh, à savoir un bélier sans défaut, pris du troupeau, avec l'estimation que tu feras de la chose sainte, la faisant en sicles d'argent, selon le sicle du sanctuaire, à cause de son péché.
\VS{16}Il restituera donc ce en quoi il aura péché en retenant de la chose sainte et il y ajoutera un cinquième par dessus, et le donnera au prêtre ; et le prêtre fera propitiation pour lui, par le bélier du sacrifice de culpabilité, et il lui sera pardonné.
\VS{17}Lorsque quelqu'un aura péché, en violant, sans le savoir, l'un des commandements de Yahweh, des choses qu'on ne doit point faire, il sera coupable et portera son iniquité.
\VS{18} Il amènera donc en sacrifice de culpabilité au prêtre un bélier sans tâche, pris du troupeau, avec l'estimation que tu feras du péché involontaire ; et le prêtre fera propitiation pour lui du péché involontaire qu'il a commis et dont il ne se sera point aperçu ; et ainsi il lui sera pardonné.
\VS{19}C'est un sacrifice de culpabilité. Il s'est rendu coupable contre Yahweh.
\TextTitle{La restitution au jour du sacrifice de culpabilité\FTNTT{Lé. 7:1-7.}}
\VS{20}Yahweh parla aussi à Moïse, en disant :
\VS{21}Quand quelqu'un aura péché et aura commis une transgression contre Yahweh, en mentant à son prochain pour un dépôt, pour une chose qu'on aura mise entre ses mains, un vol, ou qu'il ait extorqué son prochain,
\VS{22}ou s'il a trouvé quelque chose perdue, et qu'il mente à ce sujet, ou s'il jure faussement concernant l'une des choses qu'un homme fait en péchant ;
\VS{23}quand il péchera et se rendra coupable, il rendra la chose qu'il a volée ou extorquée, ou le dépôt qui lui a été donné en garde, ou la chose perdue qu'il a trouvée,
\VS{24}ou tout ce dont il aura juré faussement. Il le restituera totalement, et il y ajoutera un cinquième ; il le donnera à celui à qui il appartenait, le jour de son sacrifice de culpabilité.
\VS{25}Et il amènera pour Yahweh, au prêtre le sacrifice de culpabilité, à savoir un bélier sans défaut, pris du troupeau, avec l'estimation que tu feras de la culpabilité.
\VS{26}Et le prêtre fera propitiation pour lui devant Yahweh, et il lui sera pardonné, quelle que soit la faute dont il se sera rendu coupable. 
\Chap{6}
\TextTitle{Loi de l'holocauste\FTNTT{Lé. 1:1-17.}}
\VerseOne{}Yahweh parla aussi à Moïse, en disant :
\VS{2}Ordonne à Aaron et à ses fils, et dis-leur : C'est ici la loi de l'holocauste. L'holocauste demeurera sur le foyer de l'autel toute la nuit jusqu'au matin, et le feu brûlera sur l'autel.
\VS{3}Et le prêtre revêtira sa tunique de lin, mettra ses caleçons de lin sur son corps, et il enlèvera la cendre de l'holocauste que le feu aura consumé sur l'autel, puis il la mettra près de l'autel.
\VS{4}Alors il ôtera ses vêtements et portera d'autres vêtements pour transporter les cendres hors du camp, dans un lieu pur.
\VS{5}Et quant au feu qui brûle sur l'autel, il continuera de brûler, on ne l'éteindra point ; le prêtre y brûlera du bois tous les matins, il préparera l'holocauste sur le bois, et y brûlera les graisses des offrandes de paix.
\VS{6}Le feu brûlera continuellement sur l'autel, on ne le laissera point s'éteindre.
\TextTitle{Loi de l'offrande de gâteau\FTNTT{Lé. 2:1-16.}}
\VS{7}Et c'est ici la loi de l'offrande de gâteau. Les fils d'Aaron l'offriront devant Yahweh sur l'autel\FTNT{No. 15:4.}.
\VS{8}Et on lèvera une poignée de la fine farine du gâteau et de son huile, avec tout l'encens qui est sur le gâteau, et on le brûlera sur l'autel, en bonne odeur, en mémorial à Yahweh.
\VS{9}Aaron et ses fils mangeront ce qui en restera ; ils le mangeront sans levain dans un lieu saint, ils le mangeront dans le parvis de la tente d'assignation\FTNT{Ex. 29:26-37.}.
\VS{10}On ne le cuira point avec du levain. Je leur ai donné cela pour leur portion d'entre mes offrandes consumées par le feu. C'est une chose très sainte, comme le sacrifice d'expiation et le sacrifice de culpabilité.
\VS{11}Tout mâle d'entre les fils d'Aaron en mangera. C'est une ordonnance perpétuelle pour vos descendants concernant les offrandes consumées par le feu à Yahweh : Quiconque les touchera sera sanctifié.
\VS{12}Yahweh parla aussi à Moïse, en disant :
\VS{13}C'est ici l'offrande d'Aaron et de ses fils, qu'ils offriront à Yahweh le jour où il sera oint : Un dixième d'épha de fine farine, comme offrande de gâteau perpétuelle, une moitié le matin et une moitié le soir.
\VS{14}Elle sera apprêtée sur une plaque avec de l'huile, tu l'apporteras mélangée, et tu offriras les morceaux cuits du gâteau en bonne odeur à Yahweh.
\VS{15}Et le prêtre, d'entre ses fils, qui sera oint à sa place, fera cela. C'est une ordonnance perpétuelle devant Yahweh : On le brûlera tout entier.
\VS{16}Tout le gâteau du prêtre sera entièrement consumé ; on n'en mangera pas.
\TextTitle{Loi de l'offrande pour le péché\FTNTT{Lé. 4:1-35.}}
\VS{17}Yahweh parla aussi à Moïse, en disant :
\VS{18}Parle à Aaron et à ses fils, et dis-leur : C'est ici la loi du sacrifice d'expiation. L'offrande pour l'expiation sera égorgée devant Yahweh, dans le même lieu où l'on égorge l'holocauste : C'est une chose très sainte.
\VS{19}Le prêtre qui offrira l'offrande pour l'expiation la mangera ; elle se mangera dans un lieu saint, dans le parvis de la tente d'assignation\FTNT{No. 18:10.}.
\VS{20}Quiconque touchera sa chair sera saint. Et s'il en jaillit du sang sur le vêtement, ce sur quoi il aura jailli sera lavé dans un lieu saint.
\VS{21}Et le vase de terre dans lequel on l'aura fait cuire sera brisé ; mais si on l'a fait cuire dans un vase d'airain, il sera nettoyé et lavé dans l'eau.
\VS{22}Tout mâle d'entre les prêtres en mangera ; car c'est une chose très sainte.
\VS{23}Aucune offrande pour le sacrifice d'expiation, dont on portera le sang dans la tente d'assignation pour faire la propitiation dans le sanctuaire, ne sera mangée, mais elle sera brûlée au feu\FTNT{Hé. 13:11.}.
\Chap{7}
\TextTitle{Loi du sacrifice de culpabilité\FTNTT{Lé. 5:1-26.}}
\VerseOne{}Or c'est ici la loi du sacrifice de culpabilité : C'est une chose très sainte.
\VS{2}Au même lieu où l'on égorgera l'holocauste, on égorgera le sacrifice de culpabilité. On en répandra le sang sur l'autel tout autour.
\VS{3}Puis on en offrira toute la graisse, avec la queue, et toute la graisse qui couvre les entrailles,
\VS{4}les deux rognons, la graisse qui est dessus sur les flancs, et le grand lobe qui est sur le foie, qu'on ôtera jusqu'aux rognons.
\VS{5}Le prêtre brûlera toutes ces choses sur l'autel comme offrande consumée par le feu à Yahweh : C'est un sacrifice pour la culpabilité.
\VS{6}Tout mâle d'entre les prêtres en mangera ; il sera mangé dans un lieu saint ; car c'est une chose très sainte.
\VS{7}Le sacrifice pour l'expiation sera semblable au sacrifice de culpabilité, il y aura une même loi pour les deux ; et la victime appartiendra au prêtre qui aura fait propitiation par elle.
\VS{8}Et le prêtre qui offrira l'holocauste de quelqu'un aura la peau de l'holocauste qu'il aura offert.
\VS{9}Et toute offrande de gâteau cuit au four, apprêtée sur le gril ou sur la plaque, appartiendra au prêtre qui l'offre.
\VS{10}Et toute offrande pétrie à l'huile, ou sèche, sera pour tous les fils d'Aaron, pour l'un comme pour l'autre.
\TextTitle{Loi du sacrifice d'offrande de paix\FTNTT{Lé. 3:1-17.}} 
\VS{11}Et c'est ici, la loi du sacrifice d'offrande de paix\FTNT{Voir commentaire en Lé. 3:1.} qu'on offrira à Yahweh.
\VS{12}Si quelqu'un l'offre pour un sacrifice de reconnaissance, il offrira avec le sacrifice de reconnaissance, des gâteaux sans levain pétris à l'huile, des galettes sans levain ointes d'huile, et des gâteaux de fine farine mêlés et pétris à l'huile.
\VS{13}En plus des gâteaux, il offrira pour son offrande du pain levé avec le sacrifice de reconnaissance de ses offrandes de paix.
\VS{14}Il présentera une part de chaque offrande, qu'il offrira comme offrande élevée à Yahweh ; elle sera pour le prêtre qui a répandu le sang du sacrifice d'offrande de paix.
\VS{15}Mais la chair du sacrifice de reconnaissance de ses offrandes de paix sera mangée le jour où elle sera offerte ; on n'en laissera rien jusqu'au matin.
\VS{16}Que si le sacrifice de son offrande est un vœu ou une offrande volontaire, son sacrifice sera mangé le jour où il l'aura offert ; ce qui en restera sera mangé le lendemain.
\VS{17}Mais ce qui restera de la chair du sacrifice sera brûlé au feu le troisième jour.
\VS{18}Que si on mange de la chair du sacrifice d'offrande de paix le troisième jour, celui qui l'aura offert ne sera point agréé, il ne lui sera point imputé, ce sera une chose infâme, et la personne qui en mangera portera son iniquité\FTNT{Ez. 4:14.}.
\VS{19}Et la chair de ce sacrifice qui a touché quelque chose d'impure ne sera point mangée, elle sera brûlée au feu. Mais quiconque sera pur, mangera de cette chair.
\VS{20}Car une personne qui mangera de la chair du sacrifice d'offrande de paix, laquelle appartient à Yahweh, et qui aura sur elle son impureté, cette personne-là sera retranchée de son peuple.
\VS{21}Si une personne touche quelque chose d'impure, soit une impureté d'homme, soit une bête impure, ou quelque autre chose impure, et qu'il mange de la chair du sacrifice d'offrande de paix qui appartient à Yahweh, cette personne-là sera retranchée d'entre son peuple.
\VS{22}Yahweh parla à Moïse, en disant :
\VS{23}Parle aux enfants d'Israël, et dis-leur : Vous ne mangerez aucune graisse de bœuf, ni d'agneau, ni de chèvre.
\VS{24}On pourra se servir pour un usage quelconque de la graisse d'une bête morte ou de la graisse d'une bête déchirée ; mais vous n'en mangerez point.
\VS{25}Car quiconque mangera de la graisse d'une bête que l'on offre comme offrande consumée par le feu à Yahweh, la personne qui en mangera, sera retranché de son peuple.
\VS{26}Vous ne mangerez point de sang, ni d'oiseaux, ni d'autres bêtes, dans aucune de vos demeures.
\VS{27}Toute personne, qui aura mangé de quelque sang que ce soit, sera retranchée de son peuple.
\VS{28}Yahweh parla à Moïse, en disant :
\VS{29}Parle aux enfants d'Israël, et dis-leur : Celui qui offrira son sacrifice d'offrande de paix à Yahweh, apportera son offrande à Yahweh, prise sur son sacrifice d'offrande de paix.
\VS{30}Il apportera de ses mains les offrandes consumées par le feu devant Yahweh. Il apportera la graisse avec la poitrine, la poitrine pour l'agiter d'un côté et de l'autre devant Yahweh.
\VS{31}Puis le prêtre brûlera la graisse sur l'autel, mais la poitrine sera pour Aaron et ses fils.
\VS{32}Vous donnerez aussi au prêtre pour offrande élevée, l'épaule droite de vos sacrifices d'offrande de paix\FTNT{No. 18:18.}.
\VS{33}Celui des fils d'Aaron qui offrira le sang et la graisse de l'offrande de paix, aura pour sa part l'épaule droite.
\VS{34}Car je prends sur les enfants d'Israël, la poitrine qu'on agite d'un côté et de l'autre, et l'épaule qu'on présente par élévation, de tous les sacrifices d'offrande de paix, et je les donne à Aaron le prêtre et à ses fils, par une ordonnance perpétuelle, de la part des fils d'Israël.
\VS{35}C'est là, le droit de l'onction d'Aaron et de l'onction de ses fils sur ces offrandes consumées par le feu devant Yahweh, depuis le jour où on les aura présentés pour exercer la sacrificature à Yahweh.
\VS{36}Et c'est ce que Yahweh ordonne aux enfants d'Israël de leur donner, depuis le jour où on les aura oints ; par une loi perpétuelle parmi leurs descendants\FTNT{Ex. 40:15.}.
\VS{37}Telle est donc la loi de l'holocauste, du gâteau, du sacrifice pour l'expiation, du sacrifice pour la culpabilité, de la consécration et du sacrifice d'offrande de paix.
\VS{38}Yahweh l'ordonna à Moïse sur la montagne de Sinaï, le jour où il ordonna aux enfants d'Israël d'offrir leurs offrandes à Yahweh dans le désert de Sinaï.
\Chap{8}
\TextTitle{Consécration d'Aaron et ses fils}
\VerseOne{}Yahweh parla aussi à Moïse en disant :
\VS{2}Prends Aaron et ses fils avec lui, les vêtements, l'huile d'onction, un jeune taureau pour le sacrifice d'expiation, deux béliers et une corbeille de pains sans levain\FTNT{Ex. 29:1-2 ; Ex. 30:25.} ;
\VS{3}et convoque toute l'assemblée à l'entrée de la tente d'assignation.
\VS{4}Et Moïse fit comme Yahweh lui avait ordonné ; et l'assemblée se rassembla à l'entrée de la tente d'assignation.
\VS{5}Moïse dit à l'assemblée : Voici ce que Yahweh a ordonné de faire.
\TextTitle{La purification avec l'eau} 
\VS{6}Et Moïse fit approcher Aaron et ses fils, et les lava avec de l'eau.
\TextTitle{Les vêtements d'Aaron} 
\VS{7}Et il mit sur Aaron la tunique, il le ceignit de la ceinture, le revêtit de la robe, mit sur lui l'éphod, et le ceignit avec la ceinture de l'éphod dont il le lia.
\VS{8}Puis il mit sur lui le pectoral, après avoir mis au pectoral l'urim et le thummim.
\VS{9}Il lui mit aussi la tiare sur la tête, et il mit sur le devant de la tiare la lame d'or, la couronne de sainteté, comme Yahweh l'avait ordonné à Moïse\FTNT{Ex. 28.}.
\TextTitle{L'onction d'huile} 
\VS{10}Puis Moïse prit l'huile d'onction, et oignit le tabernacle et toutes les choses qui y étaient, et les sanctifia.
\VS{11}Et il en fit l'aspersion sur l'autel par sept fois, et il oignit l'autel, tous ses ustensiles, et la cuve avec sa base, pour les sanctifier.
\VS{12}Il versa aussi de l'huile d'onction sur la tête d'Aaron, et l'oignit pour le sanctifier\FTNT{Ps. 133:2.}.
\TextTitle{Les vêtements des fils d'Aaron}
\VS{13}Puis Moïse fit approcher les fils d'Aaron, les revêtit des tuniques, les ceignit des ceintures et leur attacha des turbans, comme Yahweh l'avait ordonné à Moïse.
\TextTitle{Les offrandes et les sacrifices}
\VS{14}Alors il fit approcher le jeune taureau pour le sacrifice d'expiation, et Aaron et ses fils posèrent leurs mains sur la tête du taureau pour le sacrifice d'expiation.
\VS{15}Et Moïse l'égorgea, prit de son sang, et en mit avec son doigt sur les cornes de l'autel tout autour, et purifia l'autel ; et il répandit le reste du sang au pied de l'autel, ainsi il le sanctifia pour faire la propitiation sur lui.
\VS{16}Puis il prit toute la graisse qui était sur les entrailles, le grand lobe du foie, les deux rognons avec leur graisse, et Moïse les brûla sur l'autel.
\VS{17}Mais il brûla au feu, hors du camp, le jeune taureau avec sa peau, sa chair, et ses excréments, comme Yahweh l'avait ordonné à Moïse.
\VS{18}Il fit aussi approcher le bélier de l'holocauste, et Aaron et ses fils posèrent leurs mains sur la tête du bélier.
\VS{19}Et Moïse l'égorgea et répandit le sang sur l'autel tout autour.
\VS{20}Puis il coupa le bélier en morceaux, et Moïse brûla la tête, les morceaux, et la graisse.
\VS{21}Et il lava dans l'eau les entrailles et les jambes, et brûla tout le bélier sur l'autel : Ce fut un holocauste d'une agréable odeur, c'était une offrande consumée par le feu à Yahweh, comme Yahweh l'avait ordonné à Moïse.
\VS{22}Il fit aussi approcher l'autre bélier, le bélier de consécration, et Aaron et ses fils posèrent les mains sur la tête du bélier.
\VS{23}Et Moïse l'égorgea, prit de son sang, et le mit sur le lobe de l'oreille droite d'Aaron, et sur le pouce de sa main droite et sur le gros orteil de son pied droit.
\VS{24}Il fit aussi approcher les fils d'Aaron, et mit du même sang sur le lobe de leur oreille droite, et sur le pouce de leur main droite, et sur le gros orteil de leur pied droit, et Moïse répandit le reste du sang sur l'autel tout autour.
\VS{25}Après, il prit la graisse, la queue, toute la graisse qui est sur les entrailles, et le grand lobe du foie, et les deux rognons avec leur graisse, et l'épaule droite.
\VS{26}Il prit aussi de la corbeille des pains sans levain, qui étaient devant Yahweh, un gâteau sans levain, et un gâteau de pain fait à l'huile et une galette, et il les mit sur les graisses, et sur l'épaule droite.
\VS{27}Puis il mit toutes ces choses sur les paumes des mains d'Aaron et sur les paumes des mains de ses fils, et les agita d'un côté et de l'autre devant Yahweh.
\VS{28}Puis Moïse les prit de leurs mains et les brûla sur l'autel, sur l'holocauste : Ce fut l'offrande de consécration de bonne odeur, c'est une offrande consumée par le feu devant Yahweh.
\VS{29}Moïse prit aussi la poitrine du bélier de consécration, et l'agita d'un côté et de l'autre devant Yahweh : Ce fut la part de Moïse, comme Yahweh l'avait ordonné à Moïse.
\TextTitle{L'aspersion d'huile et de sang}
\VS{30}Moïse prit de l'huile d'onction et du sang qui était sur l'autel, et il en fit l'aspersion sur Aaron et sur ses vêtements, sur ses fils et sur les vêtements de ses fils ; ainsi il sanctifia Aaron et ses vêtements, les fils d'Aaron et les vêtements de ses fils.
\TextTitle{La nourriture de consécration\FTNTT{Ex. 29:26 ; Lé. 7:31-34 ; 8:29.}}
\VS{31}Après cela, Moïse dit à Aaron et à ses fils : Faites cuire la chair à l'entrée de la tente d'assignation, et vous la mangerez là, avec le pain qui est dans la corbeille de consécration, comme je l'ai ordonné, en disant : Aaron et ses fils la mangeront.
\VS{32}Mais vous brûlerez au feu ce qui restera de la chair et du pain.
\TextTitle{Les prêtres mis à part}
\VS{33}Et vous ne sortirez point pendant sept jours, de l'entrée de la tente d'assignation, jusqu'à ce que vos jours de consécration soient accomplis ; car on emploiera sept jours à vous consacrer.
\VS{34}Yahweh a ordonné de faire en ces autres jours comme on a fait en celui-ci, pour faire la propitiation en votre faveur.
\VS{35}Vous resterez donc pendant sept jours à l'entrée de la tente d'assignation, jour et nuit, et vous observerez ce que Yahweh vous a ordonné d'observer, afin que vous ne mouriez pas ; car il m'a été ainsi ordonné.
\VS{36}Ainsi Aaron et ses fils firent toutes les choses que Yahweh avait ordonnées par Moïse.
\Chap{9}
\TextTitle{Aaron et ses fils commencent leur service dans le tabernacle}
\VerseOne{}Et il arriva au huitième jour, que Moïse appela Aaron et ses fils, et les anciens d'Israël.
\VS{2}Et il dit à Aaron : Prends un jeune taureau du troupeau pour l'offrande d'expiation, et un bélier pour l'holocauste, tous deux sans défaut, et offre-les devant Yahweh.
\VS{3}Et tu parleras aux enfants d'Israël, en disant : Prenez un bouc pour l'offrande d'expiation, un jeune taureau et un agneau, tous deux d'un an et sans défaut, pour l'holocauste ;
\VS{4}un bœuf et un bélier pour l'offrande de paix\FTNT{Voir commentaire en Lé. 3:1.}, pour les sacrifier devant Yahweh ; et un gâteau pétri à l'huile. Car aujourd'hui Yahweh vous apparaîtra.
\VS{5}Ils prirent donc les choses que Moïse avait ordonné et les amenèrent devant la tente d'assignation, et toute l'assemblée s'approcha, et se tint devant Yahweh.
\VS{6}Et Moïse dit : Faites ce que Yahweh vous a ordonné, et la gloire de Yahweh vous apparaîtra.
\VS{7}Moïse dit à Aaron : Approche-toi de l'autel, fais ton sacrifice pour l'expiation et ton holocauste, et fais propitiation pour toi et pour le peuple ; présente l'offrande pour le peuple, et fais propitiation pour eux, comme Yahweh l'a ordonné\FTNT{Hé. 7:26-27.}.
\VS{8}Alors Aaron s'approcha de l'autel et égorgea le veau de son sacrifice d'expiation.
\VS{9}Et les fils d'Aaron lui présentèrent le sang, et il trempa son doigt dans le sang et le mit sur les cornes de l'autel ; puis il répandit le reste du sang au pied de l'autel.
\VS{10}Mais il brûla sur l'autel la graisse, et les rognons, et le grand lobe du foie de l'offrande pour le péché, comme Yahweh l'avait ordonné à Moïse.
\VS{11}Et il brûla au feu la chair et la peau hors du camp.
\VS{12}Il égorgea aussi l'holocauste. Les fils d'Aaron lui présentèrent le sang, lequel il répandit sur l'autel tout autour.
\VS{13}Puis ils lui présentèrent l'holocauste coupé en morceaux, avec la tête, et il les brûla sur l'autel.
\VS{14}Et il lava les entrailles et les jambes, qu'il brûla sur l'holocauste, sur l'autel.
\VS{15}Il offrit l'offrande du peuple. Il prit le bouc pour le sacrifice d'expiation du peuple, il l'égorgea et l'offrit pour le péché, comme la première offrande.
\VS{16}Il l'offrit en holocauste, faisant selon l'ordonnance.
\VS{17}Ensuite, il offrit l'offrande du gâteau, et il en remplit la paume de sa main, et la brûla sur l'autel, outre l'holocauste du matin.
\VS{18}Il égorgea aussi le bœuf et le bélier pour le sacrifice d'offrande de paix, qui était pour le peuple. Les fils d'Aaron lui présentèrent le sang, lequel il répandit sur l'autel tout autour.
\VS{19}Ils présentèrent la graisse du bœuf et du bélier, la queue, ce qui couvre les entrailles, les rognons, et le grand lobe du foie ;
\VS{20}ils mirent les graisses sur les poitrines, et il brûla les graisses sur l'autel.
\VS{21}Et Aaron agita d'un côté et de l'autre devant Yahweh les poitrines et l'épaule droite, comme Yahweh l'avait ordonné à Moïse.
\VS{22}Aaron éleva aussi ses mains vers le peuple, et le bénit. Puis il descendit, après avoir offert le sacrifice pour l'expiation, l'holocauste et l'offrande de paix.
\VS{23}Moïse donc et Aaron entrèrent dans la tente d'assignation, puis ils sortirent et ils bénirent le peuple. Et la gloire de Yahweh apparut à tout le peuple.
\VS{24}Car le feu sortit de devant Yahweh, et consuma sur l'autel l'holocauste et les graisses. Tout le peuple le vit et ils poussèrent des cris de joie et tombèrent sur leur face\FTNT{1 R. 18:38 ; 2 Ch. 7:1.}.
\Chap{10}
\TextTitle{Un feu étranger présenté à Yahweh}
\VerseOne{}Or les fils d'Aaron, Nadab et Abihu, prirent chacun leur encensoir, mirent du feu, et ils posèrent dessus du parfum ; ils offrirent devant Yahweh un feu étranger\FTNT{Ce passage nous avertit du danger auquel s'exposent ceux qui apportent un feu étranger dans le temple. Les feux étrangers sont les fausses doctrines, le péché, les conceptions cartésiennes, pernicieuses, mercantiles, destinés à remplacer la Parole de Dieu et à conduire le chrétien dans les ténèbres.}, ce qu'il ne leur avait point été ordonné.
\VS{2}Et le feu sortit de devant Yahweh, et les dévora ; ils moururent devant Yahweh\FTNT{No. 3:4.}.
\VS{3}Moïse dit à Aaron : C'est ce dont Yahweh avait parlé, en disant : Je serai sanctifié par ceux qui s'approchent de moi, et je serai glorifié en présence de tout le peuple. Et Aaron se tut.
\VS{4}Et Moïse appela Mischaël et Eltsaphan, les fils d'Uziel, oncle d'Aaron, et leur dit : Approchez-vous, emportez vos frères de devant le sanctuaire, hors du camp.
\VS{5}Alors ils s'approchèrent et les emportèrent avec leurs tuniques hors du camp, comme Moïse l'avait dit.
\TextTitle{Instructions données par Moïse} 
\VS{6}Puis Moïse dit à Aaron, à Eléazar et à Ithamar, ses fils : Ne découvrez point vos têtes, et ne déchirez point vos vêtements, de peur que vous ne mouriez, et que Yahweh ne se mette en colère contre toute l'assemblée. Mais que vos frères, toute la maison d'Israël, pleurent à cause de l'embrasement que Yahweh a allumé\FTNT{Ez. 24:17.}.
\VS{7}Et ne sortez point de l'entrée de la tente d'assignation, de peur que vous ne mouriez, car l'huile de l'onction de Yahweh est sur vous. Et ils firent selon la parole de Moïse.
\VS{8}Et Yahweh parla à Aaron, en disant :
\VS{9}Vous ne boirez point de vin, ni de boisson forte, ni toi ni tes fils avec toi, quand vous entrerez dans la tente d'assignation, de peur que vous ne mouriez ; c'est une ordonnance perpétuelle pour vos descendants\FTNT{No. 6:3 ; Jg. 13:7.},
\VS{10}afin que vous puissiez discerner entre ce qui est saint et ce qui est profane, entre ce qui est impur et ce qui est pur,
\VS{11}afin que vous enseigniez aux enfants d'Israël toutes les ordonnances que Yahweh leur a prononcées par Moïse.
\VS{12}Puis Moïse parla à Aaron, à Eléazar et à Ithamar, ses fils qui lui restaient : Prenez l'offrande de gâteau, leur dit-il, ce qui reste des offrandes de Yahweh consumées par le feu, et mangez-la avec des pains sans levain auprès de l'autel, car c'est une chose très sainte.
\VS{13}Vous la mangerez dans un lieu saint, parce que c'est la portion qui est assignée à toi et à tes fils sur les offrandes consumées par le feu à Yahweh ; car il m'a été ainsi ordonné.
\VS{14}Vous mangerez aussi la poitrine offerte par agitation et l'épaule présentée par élévation dans un lieu pur, toi, tes fils et tes filles avec toi ; car ces choses-là t'ont été données, dans les sacrifices d'offrande de paix\FTNT{Voir commentaire en Lé. 3:1.} des enfants d'Israël, comme ton droit et le droit de tes fils.
\VS{15}Ils apporteront l'épaule présentée par élévation et la poitrine offerte par agitation, avec les offrandes consumées par le feu, qui sont les graisses, pour les agiter en offrande çà et là devant Yahweh : Cela t'appartiendra, et à tes fils avec toi, par une ordonnance perpétuelle, comme Yahweh l'a ordonné.
\VS{16}Or Moïse cherchait soigneusement le bouc de l'offrande pour l'expiation mais voici, il avait été brûlé. Et Moïse se mit en grande colère contre Eléazar et Ithamar, les fils d'Aaron qui lui restaient, et leur dit :
\VS{17}Pourquoi n'avez-vous point mangé l'offrande pour l'expiation dans un lieu saint ? Car c'est une chose très sainte ; vu qu'elle vous a été donnée pour porter l'iniquité de l'assemblée, afin de faire propitiation pour eux devant Yahweh.
\VS{18}Voici, son sang n'a point été porté dans l'intérieur du sanctuaire ; ne manquez donc plus à la manger dans le lieu saint, comme je l'avais ordonné.
\VS{19}Alors Aaron répondit à Moise : Voici, ils ont offert aujourd'hui leur offrande pour l'expiation et leur holocauste devant Yahweh, et ces choses-ci me sont arrivées. Si j'avais mangé aujourd'hui l'offrande pour le péché, cela aurait-il plu à Yahweh ?
\VS{20}Et Moïse l'entendit, et cela fut bon à ses yeux.
\Chap{11}
\TextTitle{Lois de purification : les bêtes pures et impures}
\VerseOne{}Et Yahweh parla à Moïse et à Aaron, et leur dit :
\VS{2}Parlez aux enfants d'Israël, et dites-leur : Ce sont ici les bêtes dont vous mangerez d'entre toutes les bêtes qui sont sur la terre\FTNT{De. 14:4 ; Ac. 10:11-14.}.
\VS{3}Vous mangerez d'entre les bêtes de tous ceux qui ont le sabot fendu, qui ont le pied fourchu, et qui ruminent.
\VS{4}Mais vous ne mangerez point de celles qui ruminent uniquement, ou qui ont uniquement le sabot fendu : Comme le chameau, car il rumine mais il n'a point le sabot fendu : Il vous sera impur.
\VS{5}Et le lapin, car il rumine mais il n'a point le sabot fendu : Il vous sera impur.
\VS{6}Le lièvre, car il rumine mais il n'a point le sabot fendu : Il vous sera impur.
\VS{7}Le porc, car il a bien le sabot fendu et le pied fourchu, mais il ne rumine pas : Il vous sera impur.
\VS{8}Vous ne mangerez point de leur chair, même vous ne toucherez point leur cadavre : Ils vous seront impurs.
\VS{9}Vous mangerez de ceci d'entre tout ce qui est dans les eaux. Vous mangerez de tout ce qui a des nageoires et des écailles dans les eaux, soit dans la mer, soit dans les fleuves.
\VS{10}Mais vous ne mangerez rien de ce qui n'a point de nageoires et d'écailles, soit dans la mer, soit dans les fleuves, tant des reptiles des eaux, que de toute chose vivante qui est dans les eaux, cela vous sera en abomination.
\VS{11}Elles vous seront donc en abomination, vous ne mangerez point de leur chair, et vous tiendrez pour une chose abominable leur cadavre.
\VS{12}Tout ce donc qui vit dans les eaux et qui n'a point de nageoires et d'écailles, vous sera en abomination.
\VS{13}Et d'entre les oiseaux vous tiendrez ceux-ci pour abominables, on n'en mangera point, ils vous seront en abomination : L'aigle, l'orfraie, l'aigle de mer ;
\VS{14}le vautour, et le milan, selon leur espèce ;
\VS{15}tout corbeau, selon son espèce ;
\VS{16}l'autruche, le hibou, la mouette, et l'épervier selon leur espèce ;
\VS{17}le chat-huant, le plongeon, la chouette ;
\VS{18}le cygne, le cormoran, le pélican ;
\VS{19}la cigogne, le héron selon leur espèce, la huppe et la chauve-souris,
\VS{20}et tout reptile volant qui marche sur quatre pattes vous sera en abomination.
\VS{21}Mais, vous pourrez manger de toute chose rampante qui vole et qui va sur quatre pattes qui ont des jambes au-dessus de leurs pieds, pour sauter avec celles-ci sur la terre.
\VS{22}Ce sont donc ici ceux dont vous mangerez : La sauterelle selon son espèce, le solam\FTNT{« Solam », « hargol » et « hagab » sont diverses espèces de sauterelles.} selon son espèce, le hargol, selon son espèce et le hagab, selon son espèce.
\VS{23}Mais tout autre reptile volant qui a quatre pattes vous sera en abomination.
\VS{24}Vous serez donc impurs par ces bêtes ; quiconque touchera leur cadavre sera impur jusqu'au soir,
\VS{25}et quiconque aussi portera leur cadavre lavera ses vêtements et sera impur jusqu'au soir.
\VS{26}Toute bête qui a le sabot fendu, et qui n'a point le pied fourchu et ne rumine point, vous sera impur : Quiconque les touchera sera impur.
\VS{27}Tout ce qui marche sur ses pattes, entre tous les animaux qui marchent à quatre pieds, vous sera impur : Quiconque touchera leur cadavre sera impur jusqu'au soir,
\VS{28}et celui qui portera leur cadavre lavera ses vêtements et sera impur jusqu'au soir. Ils vous seront impurs.
\VS{29}Ceci aussi vous sera impur entre les reptiles, qui rampent sur la terre : La taupe, la souris et la tortue, selon leur espèce ;
\VS{30}le hérisson, la grenouille, le lézard, la limace et le caméléon.
\VS{31}Ces choses vous seront impures entre les reptiles : Quiconque les touchera mortes sera impur jusqu'au soir.
\VS{32}Aussi, tout ce sur quoi il en tombera quelque chose quand elles seront mortes sera impur, soit ustensile de bois, soit vêtement, soit peau, ou sac, quelque objet que ce soit dont on se sert pour faire quelque chose ; il sera mis dans l'eau, et sera impur jusqu'au soir ; puis il sera pur.
\VS{33}Mais s'il en tombe quelque chose dans quelque vase de terre que ce soit, tout ce qui est dedans sera impur, et vous casserez le vase.
\VS{34}Et tout aliment qu'on mange, sur lequel il y aura eu de cette eau, sera impur ; tout breuvage qu'on boit dans quelque vase que ce soit, en sera impur.
\VS{35}Et s'il tombe quelque chose de leur cadavre sur quoi que ce soit, cela sera impur ; le four et le foyer seront détruits : Ils seront impurs, et ils vous seront impurs.
\VS{36}Toutefois, la source, le puits ou tel autre amas d'eaux resteront purs ; mais celui donc qui touchera leur cadavre sera impur.
\VS{37}Et s'il est tombé de leur cadavre sur quelque semence qui se sème, elle restera pure.
\VS{38}Mais si on avait mis de l'eau sur la semence, et que quelque chose de leur cadavre tombe sur elle, elle vous sera impure.
\VS{39}Et si une des bêtes qui vous servent pour nourriture meurt, celui qui en touchera le cadavre sera impur jusqu'au soir ;
\VS{40}celui qui mangera de son cadavre lavera ses vêtements et sera impur jusqu'au soir, et celui aussi qui portera le cadavre de cette bête, lavera ses vêtements et sera impur jusqu'au soir.
\VS{41}Tout reptile donc qui rampe sur la terre vous sera en abomination ; et on n'en mangera point\FTNT{cp. Ge. 3:14.}.
\VS{42}Vous ne mangerez point de tout ce qui rampe sur la poitrine, ni de tout ce qui marche sur les quatre pieds, ni de tout ce qui a plusieurs pieds entre tous les reptiles qui se traînent sur la terre ; car ils seront en abomination.
\VS{43}Ne rendez point vos personnes abominables par aucun reptile qui se traîne ; ne vous rendez point impurs par eux, ne vous souillez point par eux.
\VS{44}Car je suis Yahweh, votre Dieu ; vous vous sanctifierez donc et vous serez saints, car je suis saint\FTNT{1 Pi. 1:16.} ! Ainsi, vous ne rendrez point vos personnes impures par aucun reptile qui se traîne sur la terre.
\VS{45}Car je suis Yahweh, qui vous ai fait monter du pays d'Egypte, afin que je sois votre Dieu, et que vous soyez saints ; car je suis saint !
\VS{46}Telle est la loi touchant les animaux, les oiseaux, tout être vivant, qui se meut dans les eaux, et toute être vivant, qui se rampe sur la terre,
\VS{47}afin de discerner entre la chose impure et la chose pure, entre les animaux qu'on peut manger et les animaux dont on ne doit point manger.
\Chap{12}
\TextTitle{Lois de purification : Le flux de sang\FTNTT{Ps. 51:7.}}
\VerseOne{}Yahweh parla aussi à Moïse, en disant :
\VS{2}Parle aux enfants d'Israël, et dis-leur : Si la femme après avoir conçu, enfante un mâle, elle sera impure pendant sept jours ; elle sera impure comme au temps de son indisposition menstruelle.
\VS{3}Et au huitième jour, on circoncira la chair du prépuce de l'enfant\FTNT{Les parents de Jésus ont observé cette loi (Lu. 2:21-24). Jn. 7:22.}.
\VS{4}Et elle demeurera trente-trois jours à se purifier de son sang ; elle ne touchera aucune chose sainte, et ne viendra point au sanctuaire, jusqu'à ce que les jours de sa purification soient accomplis.
\VS{5}Si elle enfante une fille, elle sera impure deux semaines, comme au temps de son indisposition menstruelle, et elle restera soixante-six jours à se purifier de son sang.
\VS{6}Après que le temps de sa purification sera accompli, soit pour un fils ou pour une fille, elle présentera au prêtre un agneau d'un an en holocauste, et un jeune pigeon ou une tourterelle en sacrifice d'expiation, à l'entrée de la tente d'assignation\FTNT{No. 6:10.}.
\VS{7}Et le prêtre offrira ces choses devant Yahweh, et fera propitiation pour elle ; et elle sera purifiée du flux de son sang. Telle est la loi pour celle qui enfante un fils ou une fille.
\VS{8}Et que, si elle n'a pas le moyen de trouver un agneau, alors elle prendra deux tourterelles ou deux jeunes pigeons, l'un pour l'holocauste, et l'autre pour le sacrifice d'expiation. Le prêtre fera propitiation pour elle, et elle sera pure.
\Chap{13}
\TextTitle{Lois de purification : La lèpre}
\VerseOne{}Yahweh parla aussi à Moïse et à Aaron, en disant :
\VS{2}L'homme qui aura sur la peau de son corps une tumeur, une dartre, ou une tache blanche, et que cela paraîtra sur la peau de son corps comme une plaie de lèpre, on l'amènera à Aaron, le prêtre, ou à l'un de ses fils prêtres.
\VS{3}Et le prêtre regardera la plaie qui est sur la peau du corps. Si le poil de la plaie est devenu blanc, et si la plaie, à la voir, est plus profonde que la peau du corps, c'est une plaie de lèpre : Le prêtre donc le regardera et le jugera impur.
\VS{4}Mais si la tache est blanche sur la peau du corps, et qu'à la voir, elle n'est point plus profonde que la peau, et si son poil n'est pas devenu blanc, le prêtre fera enfermer pendant sept jours celui qui a la plaie.
\VS{5}Et le prêtre la regardera le septième jour. Si à ses yeux la plaie s'est arrêtée, et qu'elle ne s'est point étendue sur la peau, le prêtre le fera renfermer pendant sept autres jours.
\VS{6}Et le prêtre la regardera une seconde fois le septième jour suivant. Si la plaie est devenue pâle, et qu'elle ne s'est point étendue sur la peau, le prêtre le jugera pur : C'est de la dartre ; il lavera ses vêtements, et sera pur.
\VS{7}Mais si la dartre s'est étendue sur la peau, après avoir été vu par le prêtre pour être jugé pur, il se fera examiner pour la seconde fois par le prêtre.
\VS{8}Le prêtre le regardera encore. S'il aperçoit que la dartre s'est étendue sur la peau, le prêtre le jugera impur : C'est de la lèpre.
\VS{9}Quand il y aura une plaie de lèpre sur un homme, on l'amènera au prêtre.
\VS{10}Le prêtre le regardera. Et s'il aperçoit qu'il y a une tumeur blanche sur la peau, que le poil est devenu blanc, et qu'il y a une trace de chair vive dans la tumeur,
\VS{11}c'est une lèpre invétérée dans la peau du corps : Le prêtre le jugera impur ; il ne le fera point enfermer, car il est jugé impur.
\VS{12}Si la lèpre fait une éruption sur la peau, et qu'elle couvre toute la peau de celui qui a la plaie, depuis la tête de cet homme jusqu'à ses pieds, partout où pourra voir le prêtre, le prêtre le regardera,
\VS{13}et si le prêtre voit que la lèpre couvre tout le corps de cet homme, alors il jugera pur celui qui a la plaie : La plaie est devenue toute blanche, il est pur.
\VS{14}Mais le jour où l'on apercevra de la chair vive, il sera impur ;
\VS{15}alors le prêtre regardera la chair vive, et le jugera impur : La chair vive est impure, c'est de la lèpre.
\VS{16}Si la chair vive se change et devient blanche, alors il viendra vers le prêtre ;
\VS{17}et le prêtre le regardera, et s'il aperçoit que la plaie est devenue blanche, le prêtre jugera pur celui qui a la plaie : Il est pur.
\VS{18}Si le corps a eu sur la peau un ulcère qui soit guéri,
\VS{19}et qu'à l'endroit où était l'ulcère il y ait une tumeur blanche, ou une tache blanche rougeâtre, il sera regardé par le prêtre.
\VS{20}Le prêtre donc la regardera. Et s'il aperçoit, qu'à la voir, elle paraît plus enfoncée que la peau, et que son poil est devenu blanc, alors le prêtre le jugera impur : C'est une plaie de lèpre qui a fait éruption dans l'ulcère.
\VS{21}Si le prêtre la regardant, voit que le poil n'est point blanc, et qu'elle n'est point plus enfoncée que la peau, mais qu'elle est devenue pâle, le prêtre le fera enfermer pendant sept jours.
\VS{22}Si elle s'est étendue sur la peau en quelque sorte que ce soit, le prêtre le jugera impur : C'est une plaie.
\VS{23}Mais si la tache est restée à la même place et ne s'est pas étendue, c'est une cicatrice d'ulcère : Ainsi le prêtre le jugera pur.
\VS{24}Si le corps a sur la peau une brûlure par le feu, et que la chair vive de la partie brûlée soit une tache blanche rougeâtre ou blanc seulement, le prêtre la regardera,
\VS{25}et si le poil est devenu blanc dans la tache, et qu'à la voir, elle est plus profonde que la peau, c'est de la lèpre, elle a fait éruption dans la brûlure ; le prêtre donc le jugera impur : C'est une plaie de lèpre.
\VS{26}Mais si le prêtre la regardant aperçoit qu'il n'y a point de poil blanc dans la tache, et qu'elle n'est point plus basse que la peau, qu'elle est devenue pâle, le prêtre le fera enfermer pendant sept jours.
\VS{27}Puis le prêtre la regardera le septième jour. Si la tache s'est étendue sur la peau, le prêtre le jugera impur : C'est une plaie de lèpre.
\VS{28}Si la tache est restée à la même place, ne s'est pas étendue, et est devenue pâle, c'est la tumeur de la brûlure ; et le prêtre le jugera pur ; c'est la cicatrice de la brûlure.
\VS{29}Si l'homme ou la femme a une plaie à la tête, ou l'homme à la barbe,
\VS{30}le prêtre regardera la plaie, et si à la voir, elle est plus profonde que la peau, et qu'il y ait en elle du poil jaunâtre et fin, le prêtre le jugera impur : C'est de la teigne, c'est une lèpre de la tête ou de la barbe.
\VS{31}Si le prêtre regardant la plaie de la teigne, voit qu'elle n'est point plus profonde que la peau, et n'a en elle aucun poil noir, le prêtre fera enfermer pendant sept jours celui qui a la plaie de la teigne.
\VS{32}Et le septième jour le prêtre regardera la plaie. Si la teigne ne s'est point étendue, qu'elle n'a aucun poil jaunâtre, et, qu'à voir la teigne, elle n'est pas plus profonde que la peau,
\VS{33}celui qui a la plaie de la teigne se rasera, mais il ne se rasera point à l'endroit de la teigne, et le prêtre fera enfermer pendant sept autres jours celui qui a la teigne.
\VS{34}Puis le prêtre regardera la teigne au septième jour. Si la teigne ne s'est point étendue sur la peau et, qu'à la voir, elle n'est point plus profonde que la peau, le prêtre le jugera pur, et cet homme lavera ses vêtements, et il sera pur.
\VS{35}Mais si la teigne s'est étendue sur la peau, après sa purification, le prêtre la regardera,
\VS{36}et si la teigne s'est étendue sur la peau, le prêtre ne cherchera point de poil jaunâtre : Il est impur.
\VS{37}Mais si la teigne s'est arrêtée, et qu'il y ait poussé du poil noir, la teigne est guérie : Il est pur, et le prêtre le jugera pur.
\VS{38}Si l'homme ou la femme ont sur la peau de leur corps des taches, des taches qui sont blanches,
\VS{39}le prêtre les regardera. Si sur la peau de leur corps il y a des taches d'un blanc pâle, c'est une tache blanche qui a fait éruption sur la peau : Il est donc pur.
\VS{40}Si l'homme a la tête dépouillée de cheveux, c'est un chauve : Il est pur.
\VS{41}Et si sa tête est dépouillée de cheveux du côté de son visage, c'est un front chauve : Il est pur.
\VS{42}Et si dans la partie chauve de devant ou de derrière, il y a une plaie d'un blanc rougeâtre, c'est une lèpre qui a fait éruption dans sa partie chauve de derrière ou de devant.
\VS{43}Et le prêtre le regardera. S'il aperçoit que la tumeur de la plaie est d'un blanc rougeâtre dans sa partie chauve de derrière ou de devant, semblable à la lèpre de la peau du corps,
\VS{44}l'homme est lépreux, il est impur : Le prêtre ne manquera pas de le juger impur ; sa plaie est à la tête.
\VS{45}Or le lépreux en qui sera la plaie aura ses vêtements déchirés, et sa tête nue ; et il se couvrira sur la lèvre de dessus et il criera : Impur ! Impur !
\VS{46}Pendant tout le temps qu'il aura cette plaie, il sera jugé impur : Il est impur. Il demeurera seul ; sa demeure sera hors du camp\FTNT{2 R. 7:3 ; La. 4:15 ; Lu. 17:12-13.}.
\VS{47}Et si le vêtement est infecté de la plaie de la lèpre, soit sur un vêtement de laine, soit sur un vêtement de lin,
\VS{48}à la chaîne ou à la trame du lin, ou de laine, sur la peau ou sur quelque ouvrage de peau,
\VS{49}et si cette plaie est verdâtre ou rougeâtre sur le vêtement ou sur la peau, à la chaîne ou à la trame, ou sur un objet quelconque de peau, ce sera une plaie de lèpre, et elle sera montrée au prêtre.
\VS{50}Et le prêtre regardera la plaie, et fera enfermer pendant sept jours celui qui a la plaie.
\VS{51}Et au septième jour, il regardera la plaie. Si la plaie s'est étendue sur le vêtement, à la chaîne ou à la trame, sur la peau ou sur quelque ouvrage de peau, la plaie est une lèpre invétérée : La chose est impure.
\VS{52}Il brûlera le vêtement, la chaîne ou la trame de laine ou de lin, et toutes les choses de peau, qui auront cette plaie, car c'est une lèpre rongeuse : Cela sera brûlé au feu.
\VS{53}Mais si le prêtre regarde, et que la plaie ne s'est point étendue sur le vêtement, sur la chaîne ou sur la trame, ou sur quelque objet de peau,
\VS{54}le prêtre ordonnera qu'on lave la chose où est la plaie, et il le fera enfermer pendant sept autres jours.
\VS{55}Si le prêtre, après qu'on aura fait laver la plaie, la regarde, et s'il aperçoit que la plaie n'a point changé sa couleur, et qu'elle ne s'est point étendue, c'est une chose impure : Tu la brûleras au feu ; c'est une partie de l'endroit ou de l'envers qui a été rongée.
\VS{56}Si le prêtre regarde, et aperçoit que la plaie est devenue pâle, après qu'on l'ait fait laver, il la déchirera du vêtement ou de la peau, de la chaîne ou de la trame.
\VS{57}Si elle paraît encore sur le vêtement, à la chaîne ou à la trame, ou sur quelque chose de peau, c'est une lèpre qui a fait éruption : Vous brûlerez au feu la chose où est la plaie.
\VS{58}Mais si tu as lavé le vêtement, la chaîne ou la trame, ou quelque chose de peau, et que la plaie s'en est allée, il sera lavé une seconde fois, puis il sera pur.
\VS{59}Telle est la loi sur la plaie de la lèpre sur un vêtement de laine ou de lin, la chaîne ou la trame, ou quelque chose de peau, pour la juger pure ou impure.
\Chap{14}
\TextTitle{Loi du lépreux pour le jour de sa purification}
\VerseOne{}Yahweh parla aussi à Moïse, en disant :
\VS{2}C'est ici la loi du lépreux pour le jour de sa purification. Il sera amené au prêtre\FTNT{Mt. 8:2-4 ; Mc. 1:42-44 ; Lu. 5:12-14.}.
\VS{3}Le prêtre sortira hors du camp et le regardera. Si la plaie de la lèpre du lépreux est guérie,
\VS{4}le prêtre ordonnera qu'on prenne pour celui qui doit être purifié, deux oiseaux vivants et purs, avec du bois de cèdre, du cramoisi et de l'hysope\FTNT{Ex. 12:22.}.
\VS{5}Et le prêtre ordonnera qu'on égorge l'un des oiseaux sur un vase de terre, sur de l'eau vive.
\VS{6}Puis il prendra l'oiseau vivant, le bois de cèdre, le cramoisi et l'hysope ; et il trempera toutes ces choses avec l'oiseau vivant, dans le sang de l'autre oiseau qui aura été égorgé sur de l'eau vive.
\VS{7}Il en fera sept fois l'aspersion sur celui qui doit être purifié de la lèpre. Il le déclarera pur, et il laissera aller par les champs, l'oiseau vivant.
\VS{8}Et celui qui doit être purifié lavera ses vêtements, rasera tout son poil, et se lavera dans l'eau ; et il sera pur. Ensuite il entrera dans le camp, mais il demeurera sept jours hors de sa tente.
\VS{9}Au septième jour, il rasera tout son poil, sa tête, sa barbe, les sourcils de ses yeux, tout son poil ; il rasera tout son poil ; puis il lavera ses vêtements et son corps, et il sera pur.
\VS{10}Et au huitième jour, il prendra deux agneaux sans défaut, une brebis d'un an sans défaut, et trois dixièmes de fine farine en offrande de gâteau, pétrie à l'huile, et un log d'huile.
\VS{11}Le prêtre qui fait la purification présentera celui qui doit être purifié et ces choses-là devant Yahweh, à l'entrée de la tente d'assignation.
\VS{12}Puis le prêtre prendra l'un des agneaux et l'offrira en sacrifice pour la culpabilité avec un log d'huile ; il agitera ces choses devant Yahweh, en offrande agitée.
\VS{13}Et il égorgera l'agneau au lieu où l'on égorge l'offrande pour l'expiation et l'holocauste, dans le lieu saint ; car le sacrifice pour la culpabilité appartient au prêtre, comme le sacrifice pour l'expiation ; c'est une chose très sainte.
\VS{14}Le prêtre prendra du sang de l'offrande pour la culpabilité ; il le mettra sur le lobe de l'oreille droite de celui qui doit être purifié, sur le pouce de sa main droite et sur le gros orteil de son pied droit.
\VS{15}Puis le prêtre prendra du log d'huile et en versera dans la paume de sa main gauche.
\VS{16}Et le prêtre trempera le doigt de sa main droite dans l'huile qui est dans sa paume gauche, et fera l'aspersion de l'huile avec son doigt sept fois devant Yahweh.
\VS{17}Et du reste de l'huile qui sera dans sa paume, le prêtre en mettra sur le lobe de l'oreille droite de celui qui doit être purifié, sur le pouce de sa main droite et sur le gros orteil de son pied droit, sur le sang pris de l'offrande pour la culpabilité.
\VS{18}Mais ce qui restera de l'huile sur la paume du prêtre, il le mettra sur la tête de celui qui doit être purifié ; et ainsi le prêtre fera propitiation pour lui devant Yahweh.
\VS{19}Ensuite le prêtre offrira le sacrifice pour l'expiation et fera propitiation pour celui qui doit être purifié de sa souillure. Puis il égorgera l'holocauste.
\VS{20}Le prêtre offrira l'holocauste et le gâteau sur l'autel, et fera propitiation pour celui qui doit être purifié, et il sera pur.
\VS{21}Mais s'il est pauvre et s'il n'a pas le moyen de fournir ces choses, il prendra un agneau en offrande agitée pour la culpabilité, afin de faire propitiation pour lui. Et un dixième de fine farine pétrie à l'huile pour le gâteau, avec un log d'huile.
\VS{22}Et deux tourterelles ou deux jeunes pigeons, selon ce qu'il pourra fournir, dont l'un sera pour le péché et l'autre pour l'holocauste.
\VS{23}Et le huitième jour de sa purification, il les apportera au prêtre, à l'entrée de la tente d'assignation, devant Yahweh.
\VS{24}Et le prêtre recevra l'agneau du sacrifice pour la culpabilité et le log d'huile, et les agitera devant Yahweh en offrande agitée.
\VS{25}Et il égorgera l'agneau du sacrifice pour la culpabilité. Puis le prêtre prendra du sang de l'offrande pour la culpabilité, il le mettra sur le lobe de l'oreille droite de celui qui doit être purifié, sur le pouce de sa main droite et sur le gros orteil de son pied droit.
\VS{26}Puis le prêtre versera de l'huile dans la paume de sa main gauche.
\VS{27}Et avec le doigt de sa main droite, il fera l'aspersion de l'huile qui est dans sa main gauche sept fois devant Yahweh.
\VS{28}Il mettra de cette huile qui est dans sa paume, sur le lobe de l'oreille droite de celui qui doit être purifié et sur le pouce de sa main droite et sur le gros orteil de son pied droit, sur le lieu du sang pris de l'offrande pour la culpabilité.
\VS{29}Après il mettra le reste de l'huile qui est dans sa paume sur la tête de celui qui doit être purifié, afin de faire propitiation pour lui devant Yahweh.
\VS{30}Puis il sacrifiera l'une des tourterelles ou l'un des jeunes pigeons, selon ce qu'il aura pu fournir.
\VS{31}De ce donc qu'il aura pu fournir, l'un sera pour le sacrifice d'expiation et l'autre pour l'holocauste, avec le gâteau ; ainsi le prêtre fera propitiation devant Yahweh pour celui qui doit être purifié.
\VS{32}Telle est la loi de celui qui a une plaie de lèpre, et dont les ressources sont insuffisantes à sa purification.
\TextTitle{Lois de purification d'une maison lépreuse}
\VS{33}Puis Yahweh parla à Moïse et à Aaron, en disant :
\VS{34}Quand vous serez entrés dans le pays de Canaan, que je vous donne en possession, si j'envoie une plaie de lèpre sur une maison du pays que vous posséderez,
\VS{35}celui à qui la maison appartiendra viendra et le fera savoir au prêtre, en disant : Il me semble que j'aperçois comme une plaie dans ma maison.
\VS{36}Alors le prêtre ordonnera qu'on vide la maison avant qu'il y entre pour regarder la plaie, afin que rien de ce qui est dans la maison ne soit impur, puis le prêtre entrera pour voir la maison.
\VS{37}Et il regardera la plaie. Si la plaie qui est sur les murs de la maison a des creux verdâtres ou rougeâtres, qui soient, à les voir, plus enfoncés que le mur ;
\VS{38}le prêtre sortira de la maison, à l'entrée, et fera fermer la maison pendant sept jours.
\VS{39}Au septième jour, le prêtre retournera et la regardera. Si la plaie s'est étendue sur les murs de la maison,
\VS{40}alors il ordonnera de retirer les pierres sur lesquelles est la plaie, et de les jeter hors de la ville, dans un lieu impur.
\VS{41}Il fera aussi racler l'enduit de la maison à l'intérieur, tout autour ; et l'enduit qu'on aura raclé, on le jettera hors de la ville, dans un lieu impur.
\VS{42}Puis on prendra d'autres pierres, et on les mettra à la place des premières pierres ; et on prendra d'autres mortiers pour recrépir la maison.
\VS{43}Mais si la plaie revient et fait éruption dans la maison, après avoir retiré les pierres, après avoir raclé et recrépi la maison,
\VS{44}le prêtre y entrera et la regardera. Si la plaie s'est étendue dans la maison, c'est une lèpre invétérée dans la maison : Elle est impure.
\VS{45}On démolira la maison, ses pierres, son bois, et tout le mortier de la maison ; et on les transportera hors de la ville, dans un lieu impur.
\VS{46}Si quelqu'un est entré dans la maison pendant tout le temps que le prêtre l'avait faite fermer, il sera impur jusqu'au soir.
\VS{47}Celui qui dormira dans cette maison lavera ses vêtements. Celui aussi qui mangera dans cette maison lavera ses vêtements.
\VS{48}Mais quand le prêtre y sera entré, et qu'il aura aperçu que la plaie ne s'est point étendue dans cette maison, après l'avoir recrépie, il jugera la maison pure, car sa plaie est guérie.
\VS{49}Alors il prendra pour purifier la maison deux oiseaux, du bois de cèdre, du cramoisi et de l'hysope.
\VS{50}Il égorgera l'un des oiseaux sur un vase de terre, sur de l'eau vive.
\VS{51}Il prendra le bois de cèdre, l'hysope, le cramoisi et l'oiseau vivant ; il trempera le tout dans le sang de l'oiseau qu'on aura égorgé et dans l'eau vive, puis il fera sept fois l'aspersion sur la maison.
\VS{52}Il purifiera la maison avec le sang de l'oiseau, avec l'eau vive, avec l'oiseau vivant, le bois de cèdre, l'hysope et le cramoisi.
\VS{53}Puis il laissera aller hors de la ville par les champs l'oiseau vivant. C'est ainsi qu'il fera propitiation pour la maison, et elle sera pure.
\VS{54}Telle est la loi pour toute plaie de lèpre et de teigne,
\VS{55}de lèpre de vêtement et de maison,
\VS{56}de tumeur, de dartre, et de tache ;
\VS{57}pour enseigner quand une chose est impure et quand elle est pure. Telle est la loi sur la lèpre.
\Chap{15}
\TextTitle{Lois de purification : Gonorrhée et flux menstruel\FTNTT{Jn. 13:3-10 ; Ep. 5:25-27 ; 1 Jn. 1:9.}}
\VerseOne{}Yahweh parla aussi à Moïse et à Aaron, en disant :
\VS{2}Parlez aux enfants d'Israël et dites-leur : Tout homme qui a une gonorrhée\FTNT{Gonorrhée : infection des organes génito-urinaires.} sera impur à cause de son flux.
\VS{3}Et telle sera l'impureté de son flux : Quand sa chair laissera aller son flux, ou que sa chair retiendra son flux, c'est son impureté.
\VS{4}Tout lit sur lequel se couchera celui qui est atteint d'un flux sera impur ; et toute chose sur laquelle il se sera assis sera impure.
\VS{5}L'homme aussi qui touchera son lit lavera ses vêtements et se lavera avec de l'eau ; et il sera impur jusqu'au soir.
\VS{6}Et celui qui s'assiéra sur quelque chose sur laquelle celui qui a ce flux s'est assis, lavera ses vêtements et se lavera dans l'eau, et il sera impur jusqu'au soir.
\VS{7}Et celui qui touchera la chair de celui qui a ce flux lavera ses vêtements et se lavera dans l'eau, et il sera impur jusqu'au soir.
\VS{8}Si celui qui a ce flux crache sur celui qui est pur, celui qui était pur lavera ses vêtements et se lavera dans l'eau, et il sera impur jusqu'au soir.
\VS{9}Toute monture que celui qui a ce flux aura montée sera impure.
\VS{10}Quiconque touchera quelque chose qui aura été sous lui sera impur jusqu'au soir ; et quiconque portera une telle chose lavera ses vêtements, et se lavera dans l'eau ; il sera impur jusqu'au soir.
\VS{11}Quiconque aura été touché par celui qui a ce flux, sans qu'il ait lavé ses mains dans l'eau, lavera ses vêtements et il se lavera dans l'eau, et il sera impur jusqu'au soir.
\VS{12}Et le vase de terre que celui qui a ce flux aura touché sera cassé, mais tout vase de bois sera lavé dans l'eau.
\VS{13}Or quand celui qui a ce flux sera purifié de son flux, il comptera sept jours pour sa purification ; il lavera ses vêtements et sa chair avec de l'eau vive, et ainsi il sera pur.
\VS{14}Au huitième jour, il prendra pour lui deux tourterelles ou deux jeunes pigeons, et il viendra devant Yahweh à l'entrée de la tente d'assignation, et les donnera au prêtre.
\VS{15}Et le prêtre les sacrifiera, l'un en sacrifice pour l'expiation et l'autre en holocauste ; ainsi le prêtre fera propitiation pour lui devant Yahweh à cause de son flux.
\VS{16}L'homme aussi duquel sera sortie de la semence lavera dans l'eau tout son corps, et il sera impur jusqu'au soir.
\VS{17}Et tout vêtement et toute peau sur lequel il y aura de la semence seront lavés dans l'eau, et seront impurs jusqu'au soir.
\VS{18}Même la femme qui couchera avec un tel homme se lavera dans l'eau avec son mari, et ils seront impurs jusqu'au soir.
\VS{19}Et quand la femme aura un flux, un flux de sang en sa chair, elle sera séparée sept jours. Quiconque la touchera sera impur jusqu'au soir\FTNT{Mt. 9:18-22 ; Mc. 5:21-34 ; Lu. 8:41-48.}.
\VS{20}Toute chose sur laquelle elle aura couché durant sa séparation sera impure, toute chose aussi sur laquelle elle aura été assise sera impure.
\VS{21}Quiconque aussi touchera le lit de cette femme lavera ses vêtements et se lavera dans l'eau, et il sera impur jusqu'au soir.
\VS{22}Et quiconque touchera quelque chose sur laquelle elle se sera assise lavera ses vêtements et se lavera dans l'eau, et il sera impur jusqu'au soir.
\VS{23}Même si la chose que quelqu'un aura touchée était sur le lit ou sur quelque chose sur laquelle elle était assise, quand quelqu'un aura touché cette chose-là, il sera impur jusqu'au soir.
\VS{24}Et si un homme a couché avec elle et que son impureté soit sur lui, il sera impur sept jours, et toute couche sur laquelle il dormira sera impure.
\VS{25}La femme qui aura un flux de sang pendant plusieurs jours, hors de l'époque de ses menstruations, ou dont le flux durera plus longtemps que l'époque de ses menstruations, sera impure tout le temps du flux de son impureté, comme au temps de sa séparation.
\VS{26}Toute couche sur laquelle elle couchera tous les jours de son flux lui sera comme la couche de sa séparation, et toute chose sur laquelle elle s'assiéra sera impure comme pour l'impureté de sa séparation.
\VS{27}Et quiconque aura touché ces choses-là sera impur ; il lavera ses vêtements et se lavera dans l'eau, et il sera impur jusqu'au soir.
\VS{28}Mais si elle est purifiée de son flux, elle comptera sept jours, et après elle sera pure.
\VS{29}Au huitième jour, elle prendra deux tourterelles ou deux jeunes pigeons, et les apportera au prêtre à l'entrée de la tente d'assignation.
\VS{30}Et le prêtre en sacrifiera l'un en sacrifice pour l'expiation et l'autre en holocauste ; ainsi le prêtre fera propitiation pour elle devant Yahweh, à cause du flux de son impureté.
\VS{31}Ainsi, vous séparerez les enfants d'Israël de leurs impuretés, et ils ne mourront point à cause de leurs impuretés, en rendant impur mon tabernacle, qui est au milieu d'eux.
\VS{32}Telle est la loi pour celui qui a une gonorrhée ou de celui duquel sort la semence qui le rend impur.
\VS{33}Telle est aussi la loi pour celle qui a son indisposition menstruelle ou de toute personne qui découle et qui a son flux, soit mâle, soit femelle, et de celui qui couche avec celle qui est impure.
\Chap{16}
\TextTitle{Expiation pour le prêtre, sa maison et le peuple.\FTNTT{Hé. 9:1-14.}}
\VerseOne{}Or Yahweh parla à Moïse après la mort des deux fils d'Aaron, qui moururent lorsqu'ils s'étaient approchés de la présence de Yahweh.
\VS{2}Yahweh donc dit à Moïse : Parle à Aaron, ton frère, et dis-lui qu'il n'entre point en tout temps dans le sanctuaire, au-dedans du voile, devant le propitiatoire qui est sur l'arche, afin qu'il ne meure point ; car j'apparaîtrai dans une nuée sur le propitiatoire.
\VS{3}Aaron entrera dans le sanctuaire de cette manière, après avoir offert un jeune taureau du troupeau pour le péché, et un bélier pour l'holocauste.
\VS{4}Il se revêtira de la sainte tunique de lin, et portera les caleçons de lin sur son corps ; il se ceindra de la ceinture de lin\FTNT{La ceinture de vérité (Ep. 6:14).}, et se couvrira la tête de la tiare\FTNT{La tiare, le casque du salut (Ep. 6:17).} de lin, qui sont les saints vêtements, et il s'en vêtira après avoir lavé son corps avec de l'eau\FTNT{Le lavement préfigure ici la régénération (Tit. 3:5).}.
\VS{5}Et il prendra de l'assemblée des enfants d'Israël deux jeunes boucs en offrande pour le péché et un bélier pour l'holocauste.
\VS{6}Puis Aaron offrira son veau en sacrifice pour l'expiation, et fera propitiation tant pour lui que pour sa maison.
\TextTitle{Les deux boucs expiatoires\FTNTT{2 Co. 5:21.}}
\VS{7}Et il prendra les deux boucs, et les présentera devant Yahweh, à l'entrée de la tente d'assignation.
\VS{8}Puis Aaron jettera le sort sur les deux boucs, un sort pour Yahweh et un sort pour le bouc qui doit être Azazel.
\VS{9}Et Aaron offrira le bouc sur lequel le sort sera échu pour Yahweh, et l'offrira en sacrifice pour l'expiation.
\VS{10}Mais le bouc sur lequel le sort sera tombé pour être Azazel, sera présenté vivant devant Yahweh pour faire propitiation par lui, et on l'enverra dans le désert pour être Azazel.
\VS{11}Aaron donc, présentera le veau en sacrifice pour l'expiation, et fera propitiation pour lui et pour sa maison. Et il égorgera, dis-je, son veau qui est le sacrifice pour l'expiation.
\VS{12}Puis il prendra un encensoir plein de charbons ardents, de dessus l'autel devant Yahweh, et deux poignées de parfum odoriférant en poudre ; et il les apportera au-dedans du voile ;
\VS{13}et il mettra le parfum sur le feu devant Yahweh, afin que la nuée du parfum couvre le propitiatoire qui est sur le témoignage, ainsi il ne mourra point.
\VS{14}Il prendra aussi du sang du veau, et il en fera l'aspersion avec son doigt au-devant du propitiatoire vers l'orient ; il fera l'aspersion de ce sang-là sept fois avec son doigt devant le propitiatoire.
\VS{15}Il égorgera aussi le bouc du peuple, qui est l'offrande pour l'expiation, et il apportera son sang au-dedans du voile. Il fera de son sang comme il a fait du sang du veau, en faisant l'aspersion sur le propitiatoire et sur le devant du propitiatoire.
\VS{16}Et il fera propitiation pour le sanctuaire, le purifiant des impuretés des enfants d'Israël, et de leurs transgressions, selon tous leurs péchés. Il fera la même chose pour la tente d'assignation, qui demeure avec eux au milieu de leurs impuretés.
\VS{17}Et personne ne sera dans la tente d'assignation quand le prêtre y entrera pour faire propitiation dans le sanctuaire, jusqu'à ce qu'il en sorte, lorsqu'il fera propitiation pour lui et pour sa maison, et pour toute l'assemblée d'Israël.
\VS{18}Puis il sortira vers l'autel qui est devant Yahweh, et fera propitiation pour lui ; il prendra du sang du veau et du sang du bouc, il le mettra sur les cornes de l'autel tout autour.
\VS{19}Et il fera par sept fois l'aspersion du sang avec son doigt sur l'autel, et le purifiera et le sanctifiera des impuretés des enfants d'Israël.
\VS{20}Et quand il achèvera de faire propitiation pour le sanctuaire, pour la tente d'assignation et pour l'autel, alors il offrira le bouc vivant.
\VS{21}Et Aaron posera ses deux mains sur la tête du bouc vivant, et il confessera sur lui toutes les iniquités des enfants d'Israël et toutes leurs transgressions, selon tous leurs péchés ; et il les mettra sur la tête du bouc, et l'enverra au désert par un homme prêt pour cela.
\VS{22}Et le bouc portera sur lui toutes leurs iniquités dans une terre inhabitable, puis cet homme laissera aller le bouc par le désert.
\VS{23}Et Aaron reviendra dans la tente d'assignation ; il quittera les vêtements de lin dont il s'était vêtu quand il était entré dans le sanctuaire, et les posera là.
\VS{24}Il lavera aussi son corps avec de l'eau dans le lieu saint, et se revêtira de ses vêtements. Puis il sortira, il offrira son holocauste et l'holocauste du peuple, et fera propitiation pour lui et pour le peuple.
\VS{25}Il brûlera aussi sur l'autel la graisse de l'offrande pour le péché.
\VS{26}Et celui qui aura conduit le bouc pour être Azazel lavera ses vêtements et son corps avec de l'eau ; après cela, il rentrera dans le camp.
\VS{27}Mais on tirera hors du camp le veau et le bouc qui auront été offerts en sacrifice pour l'expiation, et desquels le sang aura été porté dans le sanctuaire pour y faire propitiation, et on brûlera au feu leurs peaux, leur chair et leurs excréments\FTNT{Hé. 13:11.}.
\VS{28}Et celui qui les aura brûlés lavera ses vêtements et son corps avec de l'eau ; après cela, il rentrera dans le camp.
\VS{29}Et ceci sera pour vous une ordonnance perpétuelle : Le dixième jour du septième mois, vous affligerez vos âmes, et vous ne ferez aucune œuvre, tant celui qui est du pays que l'étranger qui fait son séjour parmi vous\FTNT{La fête des expiations (ou yom kippour) avait lieu une fois par an, le dixième jour du septième mois (Ex. 30:10 ; Lé. 16:29). A cette occasion, le grand prêtre jetait le sort sur deux boucs : un sort pour Yahweh et un sort pour Azazel (Lé. 16:8-10). Le bouc pour Yahweh était sacrifié, il préfigurait la mort expiatoire de Christ. Le bouc émissaire, pour Azazel, n'avait lui-même rien fait de mal, mais il était choisi par Dieu pour porter le péché du peuple afin qu'il soit dégagé de toute accusation. Ce que l'on faisait de ce bouc, préfigurait l'œuvre de Jésus-Christ. Il symbolisait le Seigneur qui s'est chargé de nos péchés pour les emporter loin de nous (Es. 53 ; Ps. 103:12 ; Hé. 10:17 ; Hé. 13:12-14). Christ est mort et ressuscité hors du camp et c'est là qu'il nous appelle à le rejoindre : hors du monde et des systèmes religieux (Hé. 13:10-14).}.
\VS{30}Car en ce jour-là le prêtre fera propitiation pour vous, afin de vous purifier : Ainsi vous serez purifiés de tous vos péchés devant Yahweh.
\VS{31}Ce sera pour vous donc un sabbat, un jour de repos, et vous affligerez vos âmes. C'est une ordonnance perpétuelle.
\VS{32}Et le prêtre qu'on aura oint, et qu'on aura consacré pour exercer la sacrificature à la place de son père, fera propitiation, s'étant revêtu des vêtements de lin, qui sont les saints vêtements.
\VS{33}Et il fera propitiation pour le saint sanctuaire et il fera propitiation pour la tente d'assignation et pour l'autel, et pour les prêtres et pour tout le peuple de l'assemblée.
\VS{34}Ceci donc sera pour vous une ordonnance perpétuelle, afin de faire propitiation pour les enfants d'Israël de tous leurs péchés une fois par an. On fit comme Yahweh l'avait ordonné à Moïse.
\Chap{17}
\TextTitle{Les sacrifices apportés à l'entrée de la tente d'assignation}
\VerseOne{}Yahweh parla aussi à Moïse, en disant :
\VS{2}Parle à Aaron et à ses fils, et à tous les enfants d'Israël, et dis-leur : C'est ici ce que Yahweh a ordonné, en disant :
\VS{3}Quiconque de la maison d'Israël aura égorgé un bœuf, un agneau ou une chèvre dans le camp, ou qui l'aura égorgé hors du camp\FTNT{De. 12:6.},
\VS{4}et ne l'aura point amené à l'entrée de la tente d'assignation, pour en faire une offrande à Yahweh, devant le tabernacle de Yahweh, le sang sera imputé à cet homme-là ; il a répandu du sang, c'est pourquoi cet homme-là sera retranché du milieu de son peuple.
\VS{5}C'est afin que les enfants d'Israël amènent leurs sacrifices, qu'ils sacrifient dans les champs, qu'ils les amènent à Yahweh, à l'entrée de la tente d'assignation, vers le prêtre, et qu'ils les sacrifient en sacrifices d'offrande de paix\FTNT{Voir commentaire en Lé. 3:1.} à Yahweh ;
\VS{6}et que le prêtre en répande le sang sur l'autel de Yahweh, à l'entrée de la tente d'assignation, et en brûle la graisse en bonne odeur à Yahweh.
\VS{7}Et qu'ils n'offrent plus leurs sacrifices aux démons, avec lesquels ils se sont prostitués. Ceci leur sera une ordonnance perpétuelle pour eux et leurs descendants\FTNT{De. 32:17 ; Ps. 106:37.}.
\VS{8}Tu leur diras donc : Si un homme de la maison d'Israël, ou des étrangers qui font leur séjour parmi eux, aura offert un holocauste ou un sacrifice,
\VS{9}et qui ne l'aura point amené à l'entrée de la tente d'assignation, pour le sacrifier à Yahweh, cet homme-là sera retranché d'entre ses peuples.
\TextTitle{Importance du sang}
\VS{10}Quiconque de la maison d'Israël ou des étrangers qui font leur séjour parmi eux, aura mangé de quelque sang que ce soit, je mettrai ma face contre cette personne qui aura mangé du sang, et je la retrancherai du milieu de son peuple\FTNT{Ge. 9:4 ; De. 12:16-23 ; 1 S. 14:33.}.
\VS{11}Car l'âme de la chair est dans le sang. C'est pourquoi je vous ai ordonné qu'il soit mis sur l'autel, afin de faire propitiation pour vos âmes, car c'est le sang qui fera propitiation pour l'âme.
\VS{12}C'est pourquoi j'ai dit aux enfants d'Israël : Que personne d'entre vous ne mange du sang, que même l'étranger qui fait son séjour parmi vous ne mange point de sang.
\VS{13}Et quiconque des enfants d'Israël, et des étrangers qui font leur séjour parmi eux, aura pris à la chasse une bête sauvage ou un oiseau que l'on mange, il répandra leur sang et le couvrira de poussière.
\VS{14}Car l'âme de toute chair est dans son sang, c'est son âme. C'est pourquoi j'ai dit aux enfants d'Israël : Vous ne mangerez point le sang d'aucune chair ; car l'âme de toute chair est son sang : Quiconque en mangera sera retranché.
\VS{15}Et toute personne qui aura mangé de la chair de quelque bête morte d'elle-même ou déchirée par les bêtes sauvages, tant celui qui est né dans le pays que l'étranger, lavera ses vêtements et se lavera avec de l'eau, et il sera impur jusqu'au soir ; puis il sera pur.
\VS{16}S'il ne lave pas ses vêtements et son corps, il portera son iniquité.
\Chap{18}
\TextTitle{Condamnation des incestes}
\VerseOne{}Yahweh parla encore à Moïse, en disant :
\VS{2}Parle aux enfants d'Israël et dis-leur : Je suis Yahweh, votre Dieu.
\VS{3}Vous ne ferez point ce qui se fait dans le pays d'Egypte où vous avez habité, ni ce qui se fait dans le pays de Canaan, auquel je vous amène : Vous ne vivrez point selon leurs statuts\FTNT{Jé. 10:2.}.
\VS{4}Mais vous ferez selon mes statuts, et vous garderez mes ordonnances pour marcher en elles. Je suis Yahweh, votre Dieu.
\VS{5}Vous garderez donc mes statuts et mes ordonnances, l'homme qui les pratiquera vivra par elles. Je suis Yahweh\FTNT{Ez. 20:11-13 ; Ga. 3:12 ; Ro. 10:5.}.
\VS{6}Que nul ne s'approche de celle qui est sa proche parente pour découvrir sa nudité. Je suis Yahweh.
\VS{7}Tu ne découvriras point la nudité de ton père, ni la nudité de ta mère. C'est ta mère ; tu ne découvriras point sa nudité.
\VS{8}Tu ne découvriras point la nudité de la femme de ton père. C'est la nudité de ton père\FTNT{De. 22:30 ; 1 Co. 5:1.}.
\VS{9}Tu ne découvriras point la nudité de ta sœur, fille de ton père ou fille de ta mère, née dans la maison ou hors de la maison. Tu ne découvriras point leur nudité.
\VS{10}Quant à la nudité de la fille de ton fils ou de la fille de ta fille, tu ne découvriras point leur nudité. Car elles sont ta nudité.
\VS{11}Tu ne découvriras point la nudité de la fille de la femme de ton père, née de ton père. C'est ta sœur.
\VS{12}Tu ne découvriras point la nudité de la sœur de ton père. Elle est la proche parente de ton père.
\VS{13}Tu ne découvriras point la nudité de la sœur de ta mère ; car elle est la proche parente de ta mère.
\VS{14}Tu ne découvriras point la nudité du frère de ton père. Et tu ne t'approcheras point de sa femme. Elle est ta tante.
\VS{15}Tu ne découvriras point la nudité de ta belle-fille. Elle est la femme de ton fils ; tu ne découvriras point sa nudité.
\VS{16}Tu ne découvriras point la nudité de la femme de ton frère. C'est la nudité de ton frère.
\VS{17}Tu ne découvriras point la nudité d'une femme et de sa fille. Et tu ne prendras point la fille de son fils, ni la fille de sa fille pour découvrir leur nudité. Elles sont tes proches parentes : C'est un crime.
\VS{18}Tu ne prendras point aussi une femme avec sa sœur pour exciter une rivalité en découvrant sa nudité à côté d'elle pendant sa vie.
\TextTitle{Condamnation des abominations}
\VS{19}Tu ne t'approcheras point d'une femme durant son impureté menstruelle, pour découvrir sa nudité.
\VS{20}Tu ne coucheras point avec la femme de ton prochain pour te souiller avec elle\FTNT{Ex. 20:17 ; De. 5:21 ; Mt. 5:28.}.
\VS{21}Tu ne donneras point tes enfants pour les faire passer par le feu devant Moloc\FTNT{Moloc est le nom du dieu auquel les Ammonites, peuple issu de la relation incestueuse de Loth et sa fille, sacrifiaient leurs premiers-nés en les jetant dans un brasier. De. 18:9-10 ; 1 R. 11:5-7 ; 2 R. 23:10 ; Jé. 32:35.}, et tu ne profaneras point le nom de ton Dieu. Je suis Yahweh.
\VS{22}Tu ne coucheras pas aussi avec un homme, comme on couche avec une femme. C'est une abomination\FTNT{1 Co. 6:9-10 ; Ge. 13:13 ; Ro. 1:26-27.}.
\VS{23}Tu ne coucheras point aussi avec une bête pour te souiller avec elle ; et la femme ne se prostituera point à une bête ; c'est une confusion\FTNT{1 Co. 6:9-10 ; Ro. 1:26-27.}.
\VS{24}Ne vous rendez point impurs par aucune de ces choses, car les nations que je vais chasser de devant vous se sont rendues impures par toutes ces choses.
\VS{25}Le pays a été rendu impur ; et je punirai sur lui son iniquité, et le pays vomira ses habitants.
\VS{26}Mais quant à vous, vous garderez mes ordonnances et mes jugements, et vous ne ferez aucune de ces abominations, tant celui qui est né dans le pays que l'étranger qui fait son séjour parmi vous.
\VS{27}Car les gens de ce pays-là qui ont été avant vous, ont fait toutes ces abominations, et le pays en a été rendu impur.
\VS{28}Prenez garde que le pays ne vous vomisse, si vous le rendez impur, comme il aura vomi les nations qui y étaient avant vous.
\VS{29}Car tous ceux qui feront l'une de toutes ces abominations, seront retranchés du milieu de leur peuple.
\VS{30}Vous garderez donc ce que j'ai ordonné de garder, et vous ne pratiquerez aucune de ces coutumes abominables qui ont été pratiquées avant vous, et vous ne vous rendrez point impurs par elles. Je suis Yahweh, votre Dieu.
\Chap{19}
\TextTitle{Mise en garde contre l'idolâtrie}
\VerseOne{}Yahweh parla aussi à Moïse, en disant :
\VS{2}Parle à toute l'assemblée des enfants d'Israël, et dis-leur : Soyez saints, car je suis saint, moi, Yahweh, votre Dieu.
\VS{3}Chacun de vous craindra sa mère et son père, et vous garderez mes sabbats. Je suis Yahweh, votre Dieu\FTNT{Ex. 20:12 ; De. 5:16 ; Mt. 15:4.}.
\VS{4}Vous ne vous tournerez point vers les idoles, et vous ne vous ferez aucun dieu de fonte. Je suis Yahweh, votre Dieu\FTNT{Ex. 20:3-5.}.
\TextTitle{Recommandation pour les sacrifices}
\VS{5}Si vous offrez un sacrifice d'offrande de paix\FTNT{Voir commentaire en Lé. 3:1.} à Yahweh, vous le sacrifierez de votre bon gré.
\VS{6}II se mangera le jour où vous l'aurez sacrifié, et le lendemain, mais ce qui restera jusqu'au troisième jour sera brûlé au feu.
\VS{7}Si on en mange au troisième jour, ce sera une abomination : Il ne sera point agréé.
\VS{8}Quiconque aussi en mangera portera son iniquité ; car il aura profané la chose sainte de Yahweh : Cette personne-là sera retranchée d'entre ses peuples.
\TextTitle{La justice de Yahweh, l'amour pour son prochain}
\VS{9}Quand vous ferez la moisson de votre pays, tu n'achèveras point de moissonner le bout de ton champ, et tu ne glaneras point ce qui restera à cueillir de ta moisson.
\VS{10}Tu ne grappilleras point ta vigne, ni ne recueilleras point les grains tombés de ta vigne, mais tu les laisseras au pauvre et à l'étranger\FTNT{De. 24:19.}. Je suis Yahweh, votre Dieu.
\VS{11}Vous ne déroberez point, et vous ne vous tromperez point les uns les autres ; et aucun de vous ne mentira à son prochain\FTNT{Ex. 20:15 ; Ep. 4:25 ; Col. 3:9.}.
\VS{12}Vous ne jurerez point par mon Nom en mentant, car tu profanerais le Nom de ton Dieu\FTNT{Ex. 20:7 ; De. 5:11.}. Je suis Yahweh.
\VS{13}Tu n'opprimeras point ton prochain, et tu ne le pilleras point\FTNT{De. 24:14-15 ; Ja. 5:4.}. Le salaire de ton mercenaire ne demeurera point chez toi jusqu'au lendemain.
\VS{14}Tu ne maudiras point le sourd, et tu ne mettras point d'achoppement devant l'aveugle, mais tu craindras ton Dieu. Je suis Yahweh.
\VS{15}Vous ne ferez point d'iniquité dans vos jugements : Tu n'auras point d'égard à la personne du pauvre, et tu n'honoreras point la personne du grand, mais tu jugeras ton prochain selon la justice.
\VS{16}Tu ne répandras point de calomnies parmi ton peuple. Tu ne t'élèveras point contre le sang de ton prochain. Je suis Yahweh.
\VS{17}Tu ne haïras point ton frère dans ton cœur ; tu reprendras soigneusement ton prochain\FTNT{Ge. 4:8 ; Mt. 18:15 ; 1 Jn. 2:9-11.}, et tu ne te chargeras point d'un péché à cause de lui.
\VS{18}Tu n'useras point de vengeance, et tu ne la garderas point aux enfants de ton peuple ; mais tu aimeras ton prochain comme toi-même\FTNT{Mt. 7:12 ; Mc. 12:28-34.}. Je suis Yahweh.
\VS{19}Vous garderez mes ordonnances. Tu n'accoupleras point tes bêtes de deux espèces différentes ; tu ne sèmeras point ton champ de diverses sortes de grains ; et tu ne mettras point sur toi de vêtements de diverses espèces, comme de la laine et du lin.
\VS{20}Si un homme couche et a commerce avec une femme, si c'est une esclave, fiancée à un homme, qui n'a pas été rachetée, et que la liberté ne lui a pas été donnée, ils auront le fouet, mais on ne les fera point mourir, parce qu'elle n'a pas été affranchie.
\VS{21}L'homme amènera son sacrifice pour la culpabilité à Yahweh à l'entrée de la tente d'assignation, à savoir un bélier pour la culpabilité.
\VS{22}Et le prêtre fera propitiation pour lui devant Yahweh par le bélier du sacrifice pour la culpabilité, à cause de son péché qu'il aura commis, et son péché qu'il aura commis lui sera pardonné.
\TextTitle{Ordonnances diverses}
\VS{23}Et quand vous serez entrés dans le pays, et que vous y aurez planté quelque arbre fruitier, vous considérerez son fruit comme incirconcis ; il vous sera incirconcis pendant trois ans, on n'en mangera point.
\VS{24}Mais à la quatrième année, tout son fruit sera une chose sainte à la louange de Yahweh.
\VS{25}Et à la cinquième année, vous mangerez son fruit, afin qu'il vous multiplie son produit. Je suis Yahweh, votre Dieu.
\VS{26}Vous ne mangerez rien avec le sang. Vous n'userez point de divinations, et vous ne pronostiquerez point le temps\FTNT{De. 12:23.}.
\VS{27}Vous ne couperez point en rond les coins de votre chevelure, et vous ne raserez point les coins de votre barbe.
\VS{28}Vous ne ferez point d'incisions dans votre chair pour un mort, et vous n'imprimerez point de caractères sur vous. Je suis Yahweh.
\VS{29}Tu ne profaneras point ta fille en la prostituant ; afin que le pays ne se prostitue point et ne se remplisse point de crimes.
\VS{30}Vous garderez mes sabbats et vous aurez en révérence mon sanctuaire. Je suis Yahweh.
\VS{31}Ne vous tournez point vers ceux qui évoquent les morts, ni vers les devins\FTNT{Ac. 16:16.} ; ne cherchez point à vous rendre impurs avec eux. Je suis Yahweh, votre Dieu.
\VS{32}Lève-toi devant les cheveux blancs, et tu honoreras la personne du vieillard. Tu craindras ton Dieu. Je suis Yahweh.
\VS{33}Si quelque étranger séjourne dans votre pays, vous ne lui ferez point de tort.
\VS{34}L'étranger qui séjourne parmi vous, vous sera comme celui qui est né parmi vous, et vous l'aimerez comme vous-mêmes, car vous avez été étrangers dans le pays d'Egypte. Je suis Yahweh, votre Dieu.
\VS{35}Vous ne ferez point d'iniquité dans les jugements, ni dans les mesures de dimension, ni dans les poids, ni dans les mesures de capacité.
\VS{36}Vous aurez les balances justes, les pierres à peser justes, l'épha juste et le hin juste. Je suis Yahweh, votre Dieu, qui vous ai fait sortir du pays d'Egypte.
\VS{37}Gardez donc toutes mes ordonnances et mes jugements, et pratiquez-les. Je suis Yahweh.
\Chap{20}
\TextTitle{Abominations diverses et leurs châtiments}
\VerseOne{}Yahweh parla aussi à Moïse, en disant :
\VS{2}Tu diras aux enfants d'Israël : Quiconque des enfants d'Israël ou des étrangers qui demeurent en Israël, qui donnera de sa postérité à Moloc, sera puni de mort : Le peuple du pays le lapidera.
\VS{3}Et je mettrai ma face contre un tel homme, et je le retrancherai du milieu de son peuple, parce qu'il a donné de sa postérité à Moloc, pour rendre impur mon sanctuaire et profaner le Nom de ma sainteté.
\VS{4}Si le peuple du pays ferme les yeux en quelque manière que ce soit sur cet homme-là, qui donne de sa postérité à Moloc, et s'il ne le fait pas mourir,
\VS{5}je mettrai ma face contre cet homme-là, contre sa famille, et je le retrancherai du milieu de mon peuple, avec tous ceux qui se prostituent comme lui, en se prostituant après Moloc.
\VS{6}Quant à la personne qui se tournera vers ceux qui évoquent les morts, vers les devins, en se prostituant après eux, je mettrai ma face contre cette personne-là, et je la retrancherai du milieu de son peuple.
\VS{7}Sanctifiez-vous donc, et soyez saints, car je suis Yahweh, votre Dieu.
\VS{8}Gardez aussi mes lois et pratiquez-les. Je suis Yahweh, qui vous sanctifie.
\VS{9}Un homme qui maudit son père ou sa mère sera puni de mort ; il a maudit son père ou sa mère : Son sang retombera sur lui.
\VS{10}Quant à l'homme qui commet un adultère avec la femme d'un autre, parce qu'il a commis un adultère avec la femme de son prochain, l'homme et la femme adultères seront mis à mort.
\VS{11}L'homme qui couche avec la femme de son père, découvre la nudité de son père, les deux seront mis à mort, leur sang est sur eux.
\VS{12}Quant un homme couche avec sa belle-fille, ils seront mis à mort, tous deux ; ils ont fait une confusion : Leur sang est sur eux.
\VS{13}Quant un homme couche avec un homme comme on couche avec une femme, ils ont tous deux fait une chose abominable ; ils seront mis à mort : Leur sang est sur eux.
\VS{14}Et si un homme prend pour femmes la fille et la mère, c'est un crime : Il sera brûlé au feu avec elles, afin que ce crime n'existe pas au milieu de vous.
\VS{15}Si un homme couche avec une bête, il sera puni de mort ; et vous tuerez aussi la bête.
\VS{16}Et si une femme s'approche d'une bête, tu tueras cette femme et la bête ; ils seront mis à mort : Leur sang sera sur eux.
\VS{17}Si un homme prend sa sœur, fille de son père ou fille de sa mère, et voit sa nudité, et qu'elle voit la nudité de cet homme, c'est une chose infâme ; ils seront donc retranchés sous les yeux des fils de leur peuple : Il a découvert la nudité de sa sœur, il portera son iniquité.
\VS{18}Si un homme couche avec une femme qui a son indisposition menstruelle, et qu'il découvre la nudité de cette femme, en découvrant son flux, et qu'elle découvre le flux de son sang, ils seront tous deux retranchés du milieu de leur peuple.
\VS{19}Tu ne découvriras point la nudité de la sœur de ta mère, ni de la sœur de ton père, car c'est découvrir sa proche parente, ils porteront tous deux leur iniquité.
\VS{20}Si un homme couche avec sa tante, il a découvert la nudité de son oncle ; ils porteront leur péché, et ils mourront privés d'enfants.
\VS{21}Si un homme prend la femme de son frère, c'est une impureté ; il a découvert la nudité de son frère, ils seront privés d'enfants.
\VS{22}Vous garderez toutes mes ordonnances et mes jugements et vous les pratiquerez, afin que le pays où je vous fais entrer pour y habiter ne vous vomisse point.
\VS{23}Vous ne suivrez point les statuts des nations que je vais chasser devant vous ; car elles ont fait toutes ces choses-là, et je les ai eues en abomination.
\VS{24}Et je vous ai dit : Vous posséderez leur pays, je vous le donnerai en possession : C'est un pays où coulent le lait et le miel. Je suis Yahweh, votre Dieu, qui vous ai séparés des autres peuples.
\VS{25}C'est pourquoi séparez les bêtes pures de celles qui sont impures, les oiseaux purs de ceux qui sont impurs, et ne rendez point abominables vos personnes en mangeant des bêtes et des oiseaux impurs, ni rien qui rampe sur la terre, rien de ce que je vous ai défendu comme une chose impure.
\VS{26}Vous me serez donc saints, car je suis saint, moi, Yahweh ; je vous ai séparés des autres peuples afin que vous soyez à moi.
\VS{27}Si un homme ou une femme évoquent les morts ou se livrent à la divination, on les mettra à mort ; on les lapidera : Leur sang sera sur eux.
\Chap{21}
\TextTitle{Recommandations aux prêtres}
\VerseOne{}Yahweh dit aussi à Moïse : Parle aux prêtres, fils d'Aaron, et dis-leur : Aucun d'eux ne se rendra impur parmi son peuple pour un mort,
\VS{2}excepté pour son proche parent, pour sa mère, pour son père, pour son fils, pour sa fille, et pour son frère,
\VS{3}et aussi pour sa sœur vierge, qui lui est proche, et qui n'aura point eu de mari, il se rendra impur pour elle.
\VS{4}Chef parmi son peuple, il ne se rendra point impur en se profanant.
\VS{5}Ils ne se feront point de place chauve sur la tête, ils ne raseront point les coins de leur barbe, ni ne feront point d'incisions dans leur chair.
\VS{6}Ils seront consacrés à leur Dieu, et ils ne profaneront point le Nom de leur Dieu ; car ils offrent à Yahweh les sacrifices consumés par le feu, qui sont la nourriture de leur Dieu : C'est pourquoi ils seront très saints.
\VS{7}Ils ne prendront point une femme prostituée ou déshonorée ; ils ne prendront point une femme répudiée par son mari, car ils sont saints pour leur Dieu.
\VS{8}Tu regarderas chacun d'eux comme saint, parce qu'ils offrent la nourriture de ton Dieu ; ils seront saints, car je suis saint, moi, Yahweh, qui vous sanctifie.
\VS{9}Si la fille du prêtre se profane en se prostituant, elle déshonore son père : Qu'elle soit brûlée au feu.
\VS{10}Le grand prêtre d'entre ses frères, sur la tête duquel l'huile d'onction a été répandue, et qui se sera consacré pour vêtir les saints vêtements, ne découvrira point sa tête et ne déchirera point ses vêtements.
\VS{11}Il n'ira vers aucune personne morte, il ne se rendra point impur pour son père ni pour sa mère.
\VS{12}Il ne sortira point du sanctuaire, et ne profanera point le sanctuaire de son Dieu ; car l'huile d'onction de son Dieu est une couronne sur lui. Je suis Yahweh.
\VS{13}Il prendra pour femme une vierge.
\VS{14}Il ne prendra point une veuve, ni une répudiée, ni une femme déshonorée ou prostituée ; mais il prendra pour femme une vierge parmi son peuple.
\VS{15}Il ne profanera point sa postérité parmi son peuple ; car je suis Yahweh qui le sanctifie.
\VS{16}Yahweh parla aussi à Moïse, en disant :
\VS{17}Parle à Aaron, et dis-lui : Si quelqu'un de ta postérité, parmi tes descendants, qui a quelque défaut corporel, il ne s'approchera point pour offrir la nourriture de son Dieu.
\VS{18}Car tout homme en qui il y aura un défaut n'en approchera point ; l'homme aveugle, boiteux, ayant le nez camus ou qui aura un membre allongé ;
\VS{19}ou l'homme qui aura une fracture aux pieds ou aux mains ;
\VS{20}ou qui sera bossu ou grêle, qui aura une tache à l'œil, qui aura une gale sèche, une dartre, ou qui aura les testicules écrasés.
\VS{21}Nul homme de la postérité d'Aaron, le prêtre, en qui il y aura un défaut corporel, ne s'approchera pour offrir les offrandes consumées par le feu à Yahweh ; il y a un défaut en lui, il ne s'approchera donc point pour offrir la nourriture de son Dieu.
\VS{22}Il pourra manger la nourriture de son Dieu, des choses très saintes et des choses saintes.
\VS{23}Mais il n'entrera point vers le voile, ni ne s'approchera point de l'autel, car il a un défaut corporel, et il ne profanera point mes sanctuaires, car je suis Yahweh, qui les sanctifie.
\VS{24}Moïse parla ainsi à Aaron et à ses fils, et à tous les enfants d'Israël.
\Chap{22}
\TextTitle{Consécration d'Aaron et de ses fils}
\VerseOne{}Puis Yahweh parla à Moïse, en disant :
\VS{2}Parle à Aaron et à ses fils, afin qu'ils s'abstiennent des choses saintes des enfants d'Israël, et qu'ils ne profanent point le Nom de ma sainteté dans les choses qu'ils me consacrent. Je suis Yahweh.
\VS{3}Dis-leur donc : Tout homme parmi votre génération et de vos descendants qui, étant impur, s'approchera des choses saintes que les enfants d'Israël auront sanctifiées à Yahweh, cette personne-là sera retranchée de devant moi. Je suis Yahweh.
\VS{4}Tout homme de la postérité d'Aaron, qui aura la lèpre ou une gonorrhée, ne mangera point des choses saintes jusqu'à ce qu'il soit pur. Il en sera de même pour celui qui touchera quelqu'un s'étant rendu impur en touchant un mort, ou celui qui aura une perte séminale,
\VS{5}et celui qui touchera un reptile et qui en aura été impur, ou un homme atteint d'une impureté quelconque, il en sera rendu impur.
\VS{6}La personne qui touchera ces choses sera rendu impur jusqu'au soir ; il ne mangera point des choses saintes s'il n'a point lavé son corps dans l'eau ;
\VS{7}Ensuite il sera pur après le coucher du soleil, et il mangera des choses saintes, car c'est sa nourriture.
\VS{8}Il ne mangera de la chair d'aucune bête morte d'elle-même ou déchirée par les bêtes sauvages, pour se rendre impur par elle. Je suis Yahweh.
\VS{9}Ils garderont ce que j'ai ordonné de garder, et ils ne commettront point de péché au sujet de la nourriture sainte, afin qu'ils ne meurent point, pour l'avoir profanée. Je suis Yahweh, qui les sanctifie.
\VS{10}Aucun étranger ne mangera des choses saintes ; l'étranger logé chez le prêtre et le mercenaire ne mangeront point des choses saintes.
\VS{11}Mais si le prêtre achète une personne avec son argent, elle en mangera, de même pour celui qui sera né dans sa maison ; ils mangeront de sa nourriture.
\VS{12}Si la fille du prêtre est mariée à un homme étranger, elle ne mangera point des choses saintes présentées en offrande par élévation.
\VS{13}Mais si la fille du prêtre est veuve ou répudiée, et si elle n'a point d'enfants, et est retournée dans la maison de son père, comme dans sa jeunesse, elle mangera de la nourriture de son père. Mais aucun étranger n'en mangera.
\VS{14}Si quelqu'un, pèche involontairement en mangeant d'une chose sainte, il y ajoutera un cinquième et le donnera au prêtre avec la chose sainte.
\VS{15}Et ils ne profaneront point les choses sanctifiées des enfants d'Israël, qu'ils auront offertes à Yahweh.
\VS{16}Mais on leur fera porter la peine du péché, parce qu'ils auront mangé de leurs choses saintes : Car je suis Yahweh, qui les sanctifie.
\TextTitle{Des animaux sans défaut pour les sacrifices\FTNTT{Hé. 9:14.}}
\VS{17}Yahweh parla encore à Moïse, en disant :
\VS{18}Parle à Aaron, à ses fils, et à tous les enfants d'Israël, et dis-leur : Quiconque de la maison d'Israël ou des étrangers qui sont en Israël, offrira son offrande, selon tous ses vœux, ou toutes ses offrandes volontaires, qu'on offre en holocauste à Yahweh,
\VS{19}il offrira de son bon gré, un mâle sans défaut, parmi les bœufs, les agneaux ou les chèvres.
\VS{20}Vous n'offrirez aucune chose qui ait un défaut, car elle ne serait point agréée pour vous.
\VS{21}Si un homme offre à Yahweh un sacrifice d'offrande de paix\FTNT{Voir commentaire en Lé. 3:1.} en s'acquittant d'un vœu, ou en faisant une offrande volontaire, soit de gros ou de menu bétail, elle sera sans défaut pour être agréée ; il ne doit y avoir aucun défaut.
\VS{22}Vous n'offrirez point à Yahweh ce qui sera aveugle, estropié, ou mutilé, qui ait un ulcère, une gale sèche ou une dartre ; et vous n'en ferez point sur l'autel un sacrifice consumé par le feu pour Yahweh.
\VS{23}Tu pourras bien faire une offrande volontaire d'un bœuf, ou d'une brebis, ou d'une chèvre ayant quelques membres allongés, ou quelque défaut dans ses membres, mais ils ne seront point agréés pour le vœu.
\VS{24}Vous n'offrirez point à Yahweh, et ne sacrifierez point dans votre pays un animal qui ait les testicules froissés, cassés, arrachés ou taillés.
\VS{25}Vous ne prendrez point de la main de l'étranger aucune de toutes ces choses pour les offrir comme nourriture à votre Dieu ; car la corruption qui est en eux est un défaut en elles : Elles ne seront point agréées pour vous.
\VS{26}Yahweh parla encore à Moïse, en disant :
\VS{27}Quand un veau, un agneau ou une chèvre seront nés, et qu'ils auront été sept jours sous leur mère, depuis le huitième jour et les suivants, ils seront agréables pour l'offrande du sacrifice consumé par le feu à Yahweh.
\VS{28}Vous n'égorgerez point aussi en un même jour la vache, ou la brebis, ou la chèvre, avec son petit.
\VS{29}Quand vous offrirez un sacrifice de remerciement à Yahweh, vous le sacrifierez de votre bon gré.
\VS{30}Il sera mangé le jour même ; vous n'en laisserez rien jusqu'au matin. Je suis Yahweh.
\VS{31}Gardez mes commandements et pratiquez-les. Je suis Yahweh.
\VS{32}Ne profanez point le nom de ma sainteté, car je serai sanctifié entre les enfants d'Israël. Je suis Yahweh, qui vous sanctifie,
\VS{33}et qui vous ai fait sortir du pays d'Egypte, pour être votre Dieu. Je suis Yahweh.
\Chap{23}
\TextTitle{Les fêtes de Yahweh}
\VerseOne{}Yahweh parla aussi à Moïse en disant :
\VS{2}Parle aux enfants d'Israël et dis-leur : Les fêtes\FTNT{Les fêtes de Yahweh étaient des jours solennels, c'est-à-dire des temps fixés pour s'approcher de Dieu et présenter des sacrifices (Voir le tableau en annexe « Les 7 fêtes de Yahweh » et également le dictionnaire).} solennelles de Yahweh, que vous publierez, seront de saintes convocations. Ce sont ici mes fêtes solennelles.
\VS{3}On travaillera six jours ; mais au septième jour, qui est le sabbat, le jour du repos, il y aura une sainte convocation. Vous ne ferez aucune œuvre, car c'est le sabbat à Yahweh, dans toutes vos demeures.
\TextTitle{La Pâque}
\VS{4}Ce sont ici les fêtes solennelles de Yahweh, qui seront de saintes convocations, que vous publierez en leur saison.
\VS{5}Au premier mois, le quatorzième jour du mois, entre les deux soirs, sera la Pâque\FTNT{La pâque était une fête qui commémorait la sortie d'Egypte (Ex. 12:1-14). Elle préfigurait la rédemption en Jésus-Christ, notre Pâque (1 Co. 5:7). Elle était fixée au 14ème jour du mois de Nisan, le premier mois.} à Yahweh.
\TextTitle{La fête des pains sans levain\FTNTT{Ex. 12:18 ; 13:6-8 ; 1 Co. 11:23-26.}}
\VS{6}Et le quinzième jour de ce même mois, sera la fête solennelle des pains sans levain\FTNT{La fête des pains sans levain commençait le 15ème jour du même mois (Nisan) et durait sept jours. Elle annonçait Christ, notre Pain descendu du ciel (Jn. 6:32-35). Seul le Seigneur Jésus a été sans levain, c'est-à-dire sans aucun péché. Le croyant est sauvé à la Pâque de Christ et doit vivre une vie sans péché (la fête des pains sans levain).} à Yahweh ; vous mangerez des pains sans levain pendant sept jours.
\VS{7}Le premier jour, vous aurez une sainte convocation : Vous ne ferez aucune œuvre servile.
\VS{8}Mais vous offrirez à Yahweh pendant sept jours des offrandes consumées par le feu. Et au septième jour, il y aura une sainte convocation : Vous ne ferez aucune œuvre servile.
\TextTitle{La fête des prémices\FTNTT{1 Co. 15:23.}}
\VS{9}Yahweh parla aussi à Moïse, en disant :
\VS{10}Parle aux enfants d'Israël et dis-leur : Quand vous serez entrés dans le pays que je vous donne, et que vous en aurez fait la moisson, vous apporterez alors au prêtre une gerbe des premiers fruits\FTNT{La fête des prémices annonce d'abord la résurrection du Seigneur Jésus-Christ, ensuite celle de tous ceux qui lui appartiennent (1 Th. 4:13-18 ; 1 Co. 15:23). Elle commençait le premier jour de la semaine suivant le sabbat de la Pâque, au mois de Nisan.} de votre moisson.
\VS{11}Et il agitera cette gerbe-là devant Yahweh, afin qu'elle soit agréée pour vous : Le prêtre l'agitera le lendemain du sabbat.
\VS{12}Et le jour où vous agiterez cette gerbe, vous sacrifierez un agneau sans défaut et d'un an, en holocauste à Yahweh ;
\VS{13}et le gâteau de cet holocauste sera de deux dixièmes de fine farine, pétrie à l'huile, pour offrande consumée par le feu, en bonne odeur à Yahweh ; et sa libation de vin sera le quart d'un hin.
\VS{14}Vous ne mangerez ni pain, ni grain rôti, ni grain en épi, jusqu'à ce jour-là, même jusqu'à ce que vous ayez apporté l'offrande à votre Dieu. C'est une loi perpétuelle pour vos descendants, dans toutes vos demeures.
\TextTitle{La Pentecôte ou la fête des semaines}
\VS{15}Vous compterez aussi dès le lendemain du sabbat, à savoir dès le jour où vous aurez apporté la gerbe qu'on doit agiter, sept semaines entières.
\VS{16}Vous compterez donc cinquante jours\FTNT{La fête des semaines ou fête de la moisson est désignée également comme la Pentecôte. Elle avait lieu au mois de Sivan et préfigurait l'effusion du Saint-Esprit et l'inauguration de la Nouvelle Alliance (Ac. 2:1-4). Le levain autorisé lors de cette fête évoquait par avance la présence de l'ivraie, symbole du péché et des fils du malin, parmi le blé, c'est-à-dire les enfants de Dieu (Mt. 13:24-41). Cinquante jours séparent la Pâque de la Pentecôte. Cet intervalle correspond exactement à la période séparant la résurrection du Seigneur Jésus-Christ de la naissance de l'Eglise (Ac. 2:1-4).} jusqu'au lendemain du septième sabbat ; et vous offrirez à Yahweh un gâteau nouveau.
\VS{17}Vous apporterez de vos demeures deux pains pour en faire une offrande agitée, ils seront de deux dixièmes, et de fine farine, pétris avec du levain : Ce sont les premiers fruits à Yahweh.
\VS{18}Vous offrirez aussi avec ce pain-là sept agneaux sans défaut et d'un an, un jeune taureau pris du troupeau et deux béliers, qui seront un holocauste à Yahweh, avec leurs gâteaux et leurs libations, des sacrifices consumés par le feu, en bonne odeur à Yahweh.
\VS{19}Vous sacrifierez aussi un jeune bouc en sacrifice pour l'expiation, et deux agneaux d'un an pour le sacrifice d'offrande de paix\FTNT{Voir commentaire en Lé. 3:1.}.
\VS{20}Et le prêtre les agitera avec le pain des premiers fruits, et avec les deux agneaux, en offrande agitée devant Yahweh : Ils seront saints à Yahweh, pour le prêtre.
\VS{21}Vous publierez donc, en ce même jour-là, une sainte convocation : Vous ne ferez aucune œuvre servile. C'est une ordonnance perpétuelle dans toutes vos demeures, pour vos descendants.
\VS{22}Et quand vous ferez la moisson de votre pays, tu n'achèveras point de moissonner le bout de ton champ, et tu ne glaneras point les épis qui resteront de ta moisson. Mais tu les laisseras pour le pauvre et pour l'étranger. Je suis Yahweh, votre Dieu.
\TextTitle{La fête des trompettes}
\VS{23}Yahweh parla aussi à Moïse, en disant :
\VS{24}Parle aux enfants d'Israël et dis-leur : Au septième mois, le premier jour du mois, il y aura un jour de repos pour vous, un mémorial de jubilation\FTNT{La fête des trompettes préfigure le rassemblement futur du peuple d'Israël après sa longue dispersion et l'enlèvement de l'Eglise. Cette fête était fixée au premier jour du septième mois (Tishri).}, et une sainte convocation.
\VS{25}Vous ne ferez aucune œuvre servile, et vous offrirez à Yahweh des offrandes consumées par le feu.
\TextTitle{Le jour des expiations\FTNTT{Hé. 9:1-16.}}
\VS{26}Yahweh parla aussi à Moïse, en disant :
\VS{27}Pareillement en ce même mois, qui est le septième, le dixième jour sera le jour des expiations\FTNT{Le jour des expiations ou du grand pardon (Voir Lé. 16) était célébré le dixième jour du septième mois (Tishri). Le Seigneur Jésus-Christ a fait l'expiation de nos péchés afin de nous amener à Dieu. Le propitiatoire au lieu d'être le trône du jugement, devenait ainsi le lieu de rencontre de Dieu avec le croyant (Ex. 25:22). Christ est la propitiation pour nos péchés (1 Jn. 2:2), mais il est aussi lui-même le propitiatoire (Ro. 3:25). Le péché ôté, les fautes confessées, le pardon acquis, l'holocauste offert, le chemin est ouvert pour la joie de la fête des tabernacles.} : Vous aurez une sainte convocation, vous humilierez vos âmes, et vous offrirez à Yahweh des sacrifices consumés par le feu.
\VS{28}En ce jour-là, vous ne ferez aucune œuvre, car c'est le jour des expiations, afin de faire propitiation pour vous devant Yahweh, votre Dieu.
\VS{29}Toute personne qui ne s'humiliera point en ce jour-là sera retranchée d'entre son peuple.
\VS{30}Et toute personne qui aura fait quelque œuvre en ce jour-là, je ferai périr cette personne-là du milieu de son peuple.
\VS{31}Vous ne ferez donc aucune œuvre. C'est une ordonnance perpétuelle pour vos descendants dans toutes vos demeures.
\VS{32}Ce sera pour vous un sabbat, un jour de repos, et vous humilierez vos âmes. Le neuvième jour du mois, au soir, depuis le soir jusqu'à l'autre soir, vous célébrerez votre sabbat.
\TextTitle{La fête des tabernacles\FTNTT{Esd. 3:4.}}
\VS{33}Yahweh parla aussi à Moïse, en disant :
\VS{34}Parle aux enfants d'Israël, et dis-leur : Au quinzième jour de ce septième mois sera la fête solennelle des tabernacles\FTNT{La fête des tabernacles ou des récoltes, était la fête du souvenir et de la joie. Célébrée au mois de Tishri, elle était aussi celle du repos, dans l'accomplissement des promesses. Elle préfigure le Royaume millénaire (Za. 14).} pendant sept jours, à Yahweh.
\VS{35}Au premier jour, il y aura une sainte convocation : Vous ne ferez aucune œuvre servile.
\VS{36}Pendant sept jours, vous offrirez à Yahweh des offrandes consumées par le feu. Et au huitième jour, vous aurez une sainte convocation, et vous offrirez à Yahweh des offrandes consumées par le feu ; ce sera une assemblée solennelle : Vous ne ferez aucune œuvre servile.
\VS{37}Ce sont là les fêtes solennelles de Yahweh, que vous publierez pour être des convocations saintes, afin d'offrir à Yahweh des offrandes consumées par le feu ; à savoir un holocauste, un gâteau, un sacrifice et une libation, chacune de ces choses en son jour ;
\VS{38}outre les sabbats de Yahweh, et outre vos dons, outre tous vos vœux, outre toutes les offrandes volontaires que vous présenterez à Yahweh.
\VS{39}Et aussi au quinzième jour du septième mois, quand vous aurez recueilli le produit du pays, vous célébrerez la fête solennelle de Yahweh pendant sept jours : Le premier jour sera un jour de repos, le huitième aussi sera un jour de repos.
\VS{40}Et au premier jour, vous prendrez du fruit d'un bel arbre, des branches de palmier, des rameaux d'arbres touffus et des saules de rivière ; et vous vous réjouirez pendant sept jours, devant Yahweh, votre Dieu.
\VS{41}Et vous célébrerez à Yahweh cette fête solennelle pendant sept jours dans l'année. C'est une loi perpétuelle pour vos descendants. Vous la célébrerez le septième mois.
\VS{42}Vous demeurerez sept jours sous des tentes ; tous ceux qui seront nés entre les Israélites demeureront sous des tentes,
\VS{43}afin que votre postérité sache que j'ai fait habiter les enfants d'Israël sous des tentes, quand je les ai fait sortir du pays d'Egypte. Je suis Yahweh, votre Dieu.
\VS{44}Moïse déclara ainsi aux enfants d'Israël les fêtes solennelles de Yahweh.
\Chap{24}
\TextTitle{L'huile du chandelier\FTNTT{Ex. 25:6.}}
\VerseOne{}Yahweh parla aussi à Moïse, en disant :
\VS{2}Ordonne aux enfants d'Israël de t'apporter de l'huile pure d'olives pressées pour le chandelier, afin de faire brûler les lampes continuellement.
\VS{3}Aaron les arrangera devant Yahweh continuellement, depuis le soir jusqu'au matin, en dehors du voile du témoignage dans la tente d'assignation. C'est une ordonnance perpétuelle pour vos descendants.
\VS{4}Il arrangera, dis-je, continuellement les lampes sur le chandelier pur, devant Yahweh.
\TextTitle{Les pains de proposition\FTNTT{Ex. 25:23-30.}}
\VS{5}Tu prendras aussi de la fine farine\FTNT{La fine farine est une farine de blé très pure, la première qui passe à travers les tamis de bluterie.}, et tu en feras cuire douze gâteaux\FTNT{Les pains de proposition étaient au nombre de douze et ne pouvaient être consommés que par les prêtres (Lé. 24:9). Ils préfiguraient Christ, le véritable pain de vie descendu du ciel (Jn. 6:48-51). Sous la Nouvelle Alliance, chaque enfant de Dieu est également un prêtre (Ap. 1:6), et est invité par conséquent à manger ce pain. Le nombre douze nous parle du fondement sur lequel nous devons êtres bâtis, à savoir Jésus-Christ lui-même et l'enseignement des apôtres et des prophètes (1 Co. 3:11 ; Ep. 2:20).}, chaque gâteau sera de deux dixièmes.
\VS{6}Et tu les exposeras devant Yahweh en deux rangées sur la table d'or pur, six à chaque rangée.
\VS{7}Et tu mettras de l'encens pur sur chaque rangée, qui sera comme un souvenir\FTNT{Voir commentaire en Lé. 2:2.} pour le pain, c'est une offrande consumée par le feu à Yahweh.
\VS{8}On les arrangera chaque jour de sabbat continuellement devant Yahweh, de la part des enfants d'Israël : C'est une alliance perpétuelle.
\VS{9}Et ils appartiendront à Aaron et à ses fils, qui les mangeront dans un lieu saint ; car ce sera pour eux une chose très sainte d'entre les offrandes de Yahweh consumées par le feu. C'est une ordonnance perpétuelle.
\TextTitle{Le blasphème contre le Nom de Yahweh\FTNTT{Jn. 8:59 ; 10:31.}}
\VS{10}Or le fils d'une femme israélite, qui était aussi fils d'un homme égyptien, sortit parmi les fils d'Israël, et ce fils de la femme israélite se querella dans le camp avec un homme israélite.
\VS{11}Et le fils de la femme israélite blasphéma et maudit le Nom de Yahweh. On l'amena à Moïse. Or sa mère s'appelait Schelomith, fille de Dibri, de la tribu de Dan.
\VS{12}Et on le mit en prison, jusqu'à ce que Moïse ait déclaré ce qu'il devrait faire selon la parole de Yahweh.
\VS{13}Et Yahweh parla à Moïse, en disant :
\VS{14}Tire hors du camp celui qui a maudit ; et que tous ceux qui l'ont entendu mettent les mains sur sa tête, et que toute l'assemblée le lapide.
\VS{15}Tu parleras aux enfants d'Israël, et tu leur diras : Quiconque aura maudit son Dieu, portera la peine de son péché.
\VS{16}Et celui qui aura blasphémé le Nom de Yahweh sera puni de mort : Toute l'assemblée ne manquera pas de le lapider, on fera mourir tant l'étranger que celui qui est né au pays, lequel aura blasphémé le Nom de Yahweh.
\TextTitle{La violence punie}
\VS{17}On punira aussi de mort celui qui aura frappé à mort quelque personne que ce soit.
\VS{18}Celui qui aura frappé une bête à mort, la remplacera : Vie pour vie.
\VS{19}Et quand quelque homme aura fait une blessure à son prochain, on lui fera comme il a fait :
\VS{20}Fracture pour fracture, œil pour œil, dent pour dent, selon le mal qu'il aura fait à un homme, il lui sera fait de même.
\VS{21}Celui qui frappera une bête à mort, la remplacera ; mais on fera mourir celui qui aura frappé un homme à mort.
\VS{22}Vous rendrez un même jugement. Vous traiterez l'étranger comme celui qui est né au pays ; car je suis Yahweh, votre Dieu.
\VS{23}Moïse parla aux enfants d'Israël, qui firent sortir hors du camp celui qui avait maudit, et le lapidèrent. Ainsi les fils d'Israël firent comme Yahweh l'avait ordonné à Moïse.
\Chap{25}
\TextTitle{L'année sabbatique}
\VerseOne{}Yahweh parla aussi à Moïse sur la montagne de Sinaï, en disant :
\VS{2}Parle aux enfants d'Israël, et dis-leur : Quand vous serez entrés dans le pays que je vous donne, la terre se reposera : Ce sera un sabbat à Yahweh.
\VS{3}Pendant six ans tu sèmeras ton champ, et pendant six ans tu tailleras ta vigne ; et tu en recueilleras le produit.
\VS{4}Mais la septième année il y aura un sabbat, un temps de repos pour la terre, ce sera un sabbat à Yahweh : Tu ne sèmeras point ton champ, et tu ne tailleras point ta vigne.
\VS{5}Tu ne moissonneras point ce qui proviendra des grains tombés dans ta moisson, et tu ne vendangeras point les raisins de ta vigne non taillée : Ce sera une année de repos total pour la terre.
\VS{6}Mais ce qui proviendra de la terre l'année du sabbat vous servira de nourriture, à toi, à ton serviteur et à ta servante, à ton mercenaire et à l'étranger qui demeurent avec toi,
\VS{7}à ton bétail et aux animaux qui sont dans ton pays ; tout son produit servira de nourriture.
\TextTitle{L'année du jubilé}
\VS{8}Tu compteras aussi sept sabbats d'années, à savoir sept fois sept ans, et les jours de sept sabbats feront quarante-neuf ans.
\VS{9}Puis tu feras sonner le shofar de jubilation le dixième jour du septième mois ; le jour, dis-je, des expiations, vous ferez sonner le shofar dans tout votre pays.
\VS{10}Et vous sanctifierez la cinquantième année, et publierez la liberté dans le pays à tous ses habitants : Ce sera pour vous l'année du jubilé ; et vous retournerez chacun dans sa possession, et chacun dans sa famille.
\VS{11}Cette cinquantième année vous sera l'année du jubilé : Vous ne sèmerez point et vous ne moissonnerez point ce que la terre rapportera d'elle-même, et vous ne vendangerez point les fruits de la vigne non taillée.
\VS{12}Car c'est l'année du jubilé, elle vous sera sainte. Vous mangerez ce que les champs rapporteront cette année-là.
\VS{13}En cette année du jubilé chacun de vous retournera dans sa possession.
\VS{14}Et si tu fais une vente à ton prochain, ou si tu achètes quelque chose de ton prochain, que nul de vous ne trompe son frère.
\VS{15}Mais tu achèteras de ton prochain selon le nombre des années après le jubilé. Pareillement on te fera les ventes selon le nombre des années de rapport.
\VS{16}Selon qu'il y aura plus d'années, tu augmenteras le prix de ce que tu achètes ; et selon qu'il y aura moins d'années, tu le diminueras ; car on te vend le nombre des récoltes.
\VS{17}Que nul de vous ne trompe son prochain, mais craignez votre Dieu ; car je suis Yahweh, votre Dieu.
\VS{18}Pratiquez mes ordonnances, gardez mes jugements et observez-les, et vous habiterez en sécurité dans le pays.
\VS{19}Et le pays vous donnera ses fruits, vous en mangerez, vous en serez rassasiés, et vous y habiterez en sécurité.
\VS{20}Et si vous dites : Que mangerons-nous la septième année si nous ne semons point, et si nous ne recueillons point notre récolte ?
\VS{21}J'ordonnerai à ma bénédiction de se répandre sur vous dans la sixième année, et la terre rapportera pour trois ans.
\VS{22}Puis vous sèmerez la huitième année, et vous mangerez de l'ancienne récolte jusqu'à la neuvième année ; jusqu'à ce que sa récolte soit venue, vous mangerez de l'ancienne.
\VS{23}La terre ne sera point vendue à perpétuité ; car le pays est à moi, et vous êtes étrangers et forains\FTNT{Forain : Quelqu'un d'extérieur, d'étranger à un lieu.} chez moi.
\VS{24}C'est pourquoi dans tout le pays dont vous aurez la possession, vous donnerez le droit de rachat\FTNT{Pour voir un exemple de ce droit de rachat, voir Ru. 4:1-13.} pour la terre.
\TextTitle{Le droit de rachat}
\VS{25}Si ton frère est devenu pauvre et vend quelque chose de ce qu'il possède, celui qui a le droit de rachat, à savoir son plus proche parent, viendra et rachètera la chose vendue par son frère.
\VS{26}Si cet homme n'a personne qui ait le droit de rachat, et qu'il ait trouvé de lui-même suffisamment de quoi faire le rachat de ce qu'il a vendu,
\VS{27}il comptera les années du temps qu'il a fait la vente, et il restituera le surplus à l'homme auquel il l'avait faite, et ainsi il retournera dans sa possession.
\VS{28}Mais s'il n'a pas trouvé suffisamment de quoi lui rendre, la chose qu'il aura vendue sera dans les mains de celui qui l'aura acheté, jusqu'à l'année du jubilé ; puis l'acheteur en sortira au jubilé, et le vendeur retournera dans sa possession.
\VS{29}Et si quelqu'un a vendu une maison d'habitation dans quelque ville entourée de murs, il aura le droit de rachat jusqu'à la fin de l'année de sa vente ; son droit de rachat sera d'une année.
\VS{30}Mais si elle n'est point rachetée dans l'année accomplie, la maison qui est dans la ville entourée de murs, demeurera à l'acheteur absolument et à ses descendants ; il n'en sortira point au jubilé.
\VS{31}Mais les maisons des villages, qui ne sont point entourés de murs, seront comptées comme des fonds de terre ; le vendeur aura droit de rachat, et l'acheteur sortira au jubilé.
\VS{32}Et quant aux villes des Lévites, les Lévites auront un droit de rachat perpétuel des maisons des villes de leur possession.
\VS{33}Et celui qui achètera une maison des Lévites, sortira au jubilé de la maison vendue, qui est dans la ville de sa possession ; car les maisons des villes des Lévites sont leur possession parmi les enfants d'Israël.
\VS{34}Mais les champs situés autour des villes des Lévites ne seront point vendus ; car c'est leur possession perpétuelle.
\TextTitle{Les traitements du frère pauvre}
\VS{35}Quand ton frère sera devenu pauvre, et qu'il tendra vers toi ses mains tremblantes, tu le soutiendras, tu soutiendras aussi l'étranger, et le forain, afin qu'il vive avec toi.
\VS{36}Tu ne prendras point de lui d'usure ni d'intérêt, mais tu craindras ton Dieu, et ton frère vivra avec toi.
\VS{37}Tu ne lui prêteras point ton argent à intérêt ni ne lui prêteras de tes vivres pour en tirer du profit.
\VS{38}Je suis Yahweh, votre Dieu qui vous ai fait sortir du pays d'Egypte, pour vous donner le pays de Canaan, afin d'être votre Dieu.
\VS{39}Pareillement, quand ton frère sera devenu pauvre auprès de toi, et qu'il se sera vendu à toi, tu ne te serviras point de lui comme on se sert des esclaves.
\VS{40}Mais il sera chez toi comme serait le mercenaire et l'étranger, et il te servira jusqu'à l'année du jubilé.
\VS{41}Alors il sortira de chez toi avec ses fils, il s'en retournera dans sa famille, et rentrera dans la possession de ses pères.
\VS{42}Car ils sont mes serviteurs, parce que je les ai retirés du pays d'Egypte ; c'est pourquoi ils ne seront point vendus comme on vend les esclaves.
\VS{43}Tu ne domineras point sur lui avec dureté, et tu craindras ton Dieu.
\VS{44}C'est des nations qui vous entourent que tu prendras ton esclave et ta servante qui t'appartiendront ; c'est d'elles que vous achèterez l'esclave et la servante.
\VS{45}Vous pourrez aussi en acheter des fils des étrangers qui demeureront chez toi, et même de leurs familles qui seront parmi vous, qui auront engendré dans votre pays, et vous les posséderez.
\VS{46}Vous les aurez comme un héritage pour les laisser à vos enfants après vous, afin qu'ils en héritent la possession, et vous vous servirez d'eux à perpétuité. Mais quant à vos frères, les fils d'Israël, nul ne dominera avec dureté sur son frère.
\VS{47}Et lorsque l'étranger ou le forain qui est avec toi se sera enrichi, et que ton frère qui est avec lui sera devenu si pauvre qu'il se soit vendu à l'étranger, ou au forain qui est avec toi, ou à quelqu'un de la postérité de la famille de l'étranger,
\VS{48}après s'être vendu, il y aura droit de rachat pour lui : Un de ses frères le rachètera.
\VS{49}Son oncle, ou le fils de son oncle, ou quelque autre proche parent de son sang d'entre ceux de sa famille, le rachètera ; ou lui-même, s'il en trouve le moyen, se rachètera.
\VS{50}Et il comptera avec son acheteur depuis l'année qu'il s'est vendu à lui, jusqu'à l'année du jubilé ; de sorte que l'argent du prix pour lequel il s'est vendu, se comptera à raison du nombre des années, le temps qu'il aura servi lui sera compté comme les journées d'un mercenaire.
\VS{51}S'il y a encore plusieurs années, il restituera le prix de son achat à raison de ces années, selon le prix pour lequel il a été acheté ;
\VS{52}et s'il reste peu d'années jusqu'à l'année du jubilé, il comptera avec lui, et restituera le prix de son achat à raison des années qu'il a servi.
\VS{53}Il aura été avec lui comme un mercenaire qui se loue d'année en année, et cet étranger ne dominera point sur lui avec dureté en ta présence.
\VS{54}S'il n'est pas racheté par quelqu'un de ces moyens, il sortira l'année du jubilé, lui et ses fils avec lui.
\VS{55}Car c'est de moi que les enfants d'Israël sont esclaves ; ce sont mes esclaves que j'ai fait sortir du pays d'Egypte. Je suis Yahweh, votre Dieu.
\Chap{26}
\TextTitle{Mise en garde contre le péché}
\VerseOne{}Vous ne vous ferez point d'idoles, vous ne vous dresserez point d'image taillée, ni de statue, et vous ne mettrez point de pierre sculptée dans votre pays, pour vous prosterner devant elles ; car je suis Yahweh, votre Dieu.
\VS{2}Vous garderez mes sabbats et vous révérerez mon sanctuaire. Je suis Yahweh.
\TextTitle{La bénédiction conditionnelle à l'obéissance à Yahweh}
\VS{3}Si vous marchez dans mes ordonnances et si vous gardez mes commandements et les pratiquez,
\VS{4}je vous donnerai les pluies en leur temps, la terre donnera ses produits, et les arbres des champs donneront leurs fruits.
\VS{5}Le foulage des grains atteindra la vendange chez vous, et la vendange atteindra les semailles ; vous mangerez votre pain à satiété et vous habiterez en sécurité dans votre pays.
\VS{6}Je donnerai la paix au pays, vous dormirez sans que personne ne vous trouble ; je ferai disparaître les bêtes méchantes du pays, et l'épée ne passera point par votre pays.
\VS{7}Vous poursuivrez vos ennemis, et ils tomberont par l'épée devant vous.
\VS{8}Cinq d'entre vous en poursuivront cent, et cent en poursuivront dix mille, et vos ennemis tomberont par l'épée devant vous.
\VS{9}Je me tournerai vers vous, je vous ferai fructifier et multiplier, et j'établirai mon alliance avec vous.
\VS{10}Vous mangerez de vieilles provisions, et vous sortirez le vieux pour y loger le nouveau.
\VS{11}Même, je mettrai mon tabernacle au milieu de vous, et mon âme ne vous aura point en horreur.
\VS{12}Mais je marcherai au milieu de vous, je serai votre Dieu, et vous serez mon peuple.
\VS{13}Je suis Yahweh, votre Dieu, qui vous ai fait sortir du pays d'Egypte, afin que vous ne soyez point leurs esclaves ; j'ai brisé les liens de votre joug, et je vous ai fait marcher la tête levée.
\TextTitle{Les châtiments en cas de désobéissance à Yahweh}
\VS{14}Mais si vous ne m'écoutez point et que vous ne pratiquez pas tous ces commandements,
\VS{15}et si vous rejetez mes ordonnances, et que votre âme a en horreur mes jugements, afin de ne point pratiquer tous mes commandements, et que vous rompiez mon alliance,
\TextTitle{La domination par les ennemis}
\VS{16}aussi je vous ferai ceci : Je répandrai sur vous la frayeur, la langueur et l'ardeur, qui vous consumerons les yeux et feront languir votre âme ; et vous sèmerez en vain votre semence car vos ennemis la mangeront.
\VS{17}Je tournerai ma face contre vous, vous serez battus devant vos ennemis ; ceux qui vous haïssent domineront sur vous ; et vous fuirez sans que personne ne vous poursuive.
\TextTitle{Le manque de fertilité de la terre}
\VS{18}Si après ces choses vous ne m'écoutez point, je vous châtierai sept fois plus à cause de vos péchés.
\VS{19}Je briserai l'orgueil de votre force et je ferai que votre ciel soit pour vous comme du fer, et votre terre comme de l'airain.
\VS{20}Votre force se consumera en vain, votre terre ne donnera point ses produits, et les arbres de la terre ne donneront point leurs fruits.
\TextTitle{Les attaques des bêtes des champs}
\VS{21}Si vous marchez en opposition avec moi et que vous ne voulez point m'écouter, je vous frapperai sept fois plus, selon vos péchés.
\VS{22}J'enverrai contre vous les bêtes des champs, qui vous priveront de vos enfants, qui détruiront votre bétail, et vous réduiront à un petit nombre, et vos chemins seront déserts.
\TextTitle{La peste}
\VS{23}Si après ces choses, vous ne recevez pas ma correction, et que vous marchiez en opposition avec moi,
\VS{24}je marcherai aussi en opposition avec vous, et je vous frapperai sept fois plus, selon vos péchés.
\VS{25}Et je ferai venir sur vous l'épée qui fera la vengeance de mon alliance ; et quand vous vous rassemblerez dans vos villes, j'enverrai la peste au milieu de vous, et vous serez livrés entre les mains de l'ennemi.
\TextTitle{Le manque de nourriture}
\VS{26}Lorsque je vous briserai le bâton du pain, dix femmes cuiront votre pain dans un seul four, et vous rendront votre pain au poids ; vous en mangerez, et vous n'en serez point rassasiés.
\VS{27}Si avec cela vous ne m'écoutez point, et que vous marchiez en opposition avec moi,
\VS{28}je marcherai aussi en opposition avec vous, avec fureur, et je vous châtierai aussi sept fois plus, selon vos péchés ;
\VS{29}vous mangerez la chair de vos fils, et vous mangerez aussi la chair de vos filles\FTNT{La. 4:10.}.
\VS{30}Je détruirai vos hauts lieux, j'abattrai vos statues consacrées au soleil, je mettrai vos cadavres sur les cadavres de vos idoles, et mon âme vous aura en horreur.
\VS{31}Je réduirai vos villes en désert, je dévasterai vos sanctuaires, et je ne respirerai plus l'agréable odeur de vos parfums.
\TextTitle{La dispersion dans les nations\FTNTT{De. 28:58-67.}}
\VS{32}Je dévasterai le pays, et vos ennemis qui l'habiteront, en seront étonnés.
\VS{33}Je vous disperserai parmi les nations, et je tirerai l'épée après vous, et votre pays sera dévasté, et vos villes désertes.
\VS{34}Alors la terre prendra plaisir à ses sabbats\FTNT{2 Ch. 36:21.}, tout le temps qu'elle sera dévastée, et lorsque vous serez dans le pays de vos ennemis, la terre se reposera et prendra plaisir à ses sabbats.
\VS{35}Tout le temps qu'elle sera dévastée, elle se reposera parce qu'elle ne s'était point reposée dans vos sabbats, lorsque vous y habitiez.
\VS{36}Et quant à ceux d'entre vous qui survivront dans le pays de leurs ennemis, je rendrai leur cœur lâche, de sorte que le bruit d'une feuille agitée les poursuivra, ils fuiront comme on fuit devant l'épée, et ils tomberont sans que personne ne les poursuive.
\VS{37}Et ils trébucheront les uns sur les autres comme devant l'épée, sans que personne ne les poursuive ; et vous ne tiendrez point devant vos ennemis ;
\VS{38}vous périrez parmi les nations, et le pays de vos ennemis vous consumera.
\VS{39}Et ceux d'entre vous qui survivront, se fondront à cause de leurs iniquités, dans les pays de vos ennemis ; ils se fondront aussi à cause des iniquités de leurs pères.
\TextTitle{Repentance et restauration de l'alliance d'Abraham, d'Isaac et de Jacob}
\VS{40}Alors, ils confesseront leurs iniquités et les iniquités de leurs pères, selon les transgressions qu'ils auront commises contre moi ; et aussi parce qu'ils auront marché en opposition avec moi.
\VS{41}Moi aussi, je marcherai en opposition avec eux, je les amènerai dans le pays de leurs ennemis. Et alors, leur cœur incirconcis s'humiliera, et ils recevront la peine de leur iniquité.
\VS{42}Et alors je me souviendrai de mon alliance avec Jacob, et de mon alliance avec Isaac, et je me souviendrai aussi de mon alliance avec Abraham, et je me souviendrai de la terre.
\VS{43}Quand donc la terre sera abandonnée par eux, et prendra plaisir à ses sabbats, ayant été désolée à cause d'eux ; et qu'ils recevront la peine de leur iniquité, parce qu'ils ont rejeté mes ordonnances, et que leur âme a eu mes ordonnances en horreur.
\VS{44}Je m'en souviendrai, dis-je, lorsqu'ils seront dans le pays de leurs ennemis, je ne les rejetterai point, et je ne les aurai point en horreur pour les consumer entièrement jusqu'à rompre mon alliance avec eux ; car je suis Yahweh, leur Dieu.
\VS{45}Je me souviendrai en leur faveur de la Première Alliance, par laquelle je les ai fait sortir du pays d'Egypte, aux yeux des nations, pour être leur Dieu. Je suis Yahweh.
\VS{46}Ce sont là les statuts, les ordonnances, et les lois que Yahweh établit entre lui et les enfants d'Israël sur la montagne de Sinaï, par Moïse.
\Chap{27}
\TextTitle{Lois des personnes et des biens voués à Yahweh}
\VerseOne{}Yahweh parla aussi à Moïse, en disant :
\VS{2}Parle aux enfants d'Israël, et dis-leur : quand quelqu'un aura fait un vœu important, les personnes vouées à Yahweh seront mises à ton estimation.
\VS{3}Et l'estimation que tu feras d'un homme, depuis l'âge de vingt ans jusqu'à l'âge de soixante ans, sera du prix de cinquante sicles d'argent, selon le sicle du sanctuaire.
\VS{4}Mais si c'est une femme, alors ton estimation sera de trente sicles.
\VS{5}Si c'est un homme de cinq ans jusqu'à vingt ans, alors ton estimation sera de vingt sicles ; et quant à la femme, de dix sicles.
\VS{6}Et si c'est un homme d'un mois jusqu'à cinq ans, ton estimation sera de cinq sicles d'argent ; et l'estimation d'une femme sera de trois sicles d'argent.
\VS{7}Et lorsque c'est un homme de soixante ans et au-dessus, ton estimation sera de quinze sicles ; et si c'est une femme, de dix sicles.
\VS{8}Et si celui qui a fait le vœu est plus pauvre que ton estimation, on le présentera devant le prêtre, qui en fera l'estimation, et le prêtre fera l'estimation selon les ressources de celui qui a fait le vœu.
\VS{9}Si c'est d'une des bêtes que l'on présente en offrande à Yahweh, tout ce qu'on donnera à Yahweh de la sorte sera saint.
\VS{10}On ne la changera point, et on n'en mettra point une autre à la place, d'une bonne pour une mauvaise, ou une mauvaise pour une bonne ; si l'on remplace une bête par une autre bête, elles seront l'une et l'autre chose sainte.
\VS{11}Si c'est d'une bête impure, qu'on ne peut présenter en offrande à Yahweh, on présentera la bête devant le prêtre,
\VS{12}qui en fera l'évaluation selon qu'elle sera bonne ou mauvaise, et il en sera fait ainsi, selon l'estimation du prêtre.
\VS{13}Mais si on veut la racheter, on ajoutera un cinquième à ton estimation.
\VS{14}Et quand quelqu'un sanctifiera sa maison pour être sainte à Yahweh, le prêtre l'estimera selon qu'elle sera bonne ou mauvaise, et on se tiendra à l'estimation que le prêtre en aura faite.
\VS{15}Mais si celui qui l'a sanctifiée veut racheter sa maison, il ajoutera par-dessus un cinquième de l'argent de ton estimation, et elle lui appartiendra.
\VS{16}Et si l'homme sanctifie à Yahweh une partie du champ de sa possession, ton estimation sera selon ce qu'on y sème, le homer de semence d'orge à cinquante sicles d'argent.
\VS{17}S'il a sanctifié son champ dès l'année du jubilé, on s'en tiendra à ton estimation ;
\VS{18}mais s'il sanctifie son champ après le jubilé, le prêtre estimera l'argent selon le nombre des années qui restent jusqu'à l'année du jubilé, et il sera fait une réduction sur ton estimation.
\VS{19}Et si celui qui a sanctifié le champ veut le racheter en quelque sorte que ce soit, il ajoutera par-dessus un cinquième de l'argent de ton estimation, et il lui restera.
\VS{20}Mais s'il ne rachète point le champ, et que le champ se vende à un autre homme, il ne se rachètera plus.
\VS{21}Et ce champ-là ayant passé le jubilé sera consacré à Yahweh, comme un champ d'interdit, la possession en sera au prêtre.
\VS{22}Et s'il sanctifie à Yahweh un champ qu'il ait acheté, qui ne soit point des champs de sa possession,
\VS{23}le prêtre lui comptera la somme de ton estimation jusqu'à l'année du jubilé, et il donnera en ce jour-là ton estimation, afin que ce soit une chose consacrée à Yahweh.
\VS{24}Mais l'année du jubilé, le champ retournera à celui de qui il avait été acheté, et auquel était la possession de la terre.
\VS{25}Et toute estimation que tu auras faite, sera selon le sicle du sanctuaire : Le sicle est de vingt guéras.
\TextTitle{Consécration des premiers-nés du bétail}
\VS{26}Toutefois, nul ne pourra consacrer le premier-né d'entre les bêtes, car il appartient à Yahweh par droit de primogéniture, soit de bœuf, soit d'agneau, il est à Yahweh.
\VS{27}Mais s'il s'agit d'une bête impure, il le rachètera selon ton estimation, et il ajoutera à ton estimation un cinquième ; et s'il n'est point racheté, il sera vendu selon ton estimation.
\TextTitle{Consécration des choses et personnes dévouées par interdit à Yahweh}
\VS{28}Or toute chose dévouée que quelqu'un dévouera à la façon de l'interdit à Yahweh, de tout ce qui est sien, soit homme, ou bête, ou champ de sa possession, ne se revendra ni ne se rachètera ; toute chose dévouée sera entièrement consacrée à Yahweh.
\VS{29}Nul interdit dévoué par interdit d'entre les hommes ne pourra être racheté, mais on le fera mettre à mort.
\TextTitle{Consécration de la dîme de la terre et du bétail}
\VS{30}Toute dîme de la terre, tant du grain de la terre que du fruit des arbres, est à Yahweh ; c'est une chose consacrée à Yahweh.
\VS{31}Mais si quelqu'un veut racheter en quelque sorte que ce soit quelque chose de sa dîme, il y ajoutera un cinquième par-dessus.
\VS{32}Mais toute dîme de bœufs, de brebis et de chèvres, à savoir tout ce qui passe sous la verge, le dixième en sera consacrée à Yahweh.
\VS{33}On ne choisira point le bon ou le mauvais, et l'on ne fera point d'échange ; si on l'échange, la bête changée et l'autre seront consacrées, et ne seront point rachetées.
\VS{34}Ce sont là les commandements que Yahweh donna à Moïse sur la montagne de Sinaï, pour les enfants d'Israël.
\PPE{}
\end{multicols}

%\clearpage\ShortTitle{No.}\BookTitle{Nombres}\BFont
\noindent\hrulefill
{\footnotesize
\textit{
\bigskip
{\centering{}
\\Auteur~: Probablement Moïse
\\(Heb.~: Bamidbar)
\\Signification~: Dans le désert
\\Thème~: Pérégrination dans le désert
\\Date de rédaction~: Env. 1450-1410 av. J.-C.\\}
}
\textit{
\\Ce livre commence par le recensement des fils d'Israël et relate trente-huit des quarante années qu'ils passèrent dans le désert du Sinaï. Il couvre une période qui s'étend de la deuxième année après la sortie d'Egypte à la veille de l'entrée en Canaan, terre que Dieu avait promis de donner à la descendance d'Abraham. Ce pays où coulaient le lait et le miel s'étendait de Sidon jusqu'à Lesha, en passant par Gaza et Sodome. En plus des Cananéens, il accueillait en son sein des enfants d'Anak, les Amalécites, les Hétiens, les Jébusiens et les Amoréens.
\\Ces écrits retracent les premières victoires d'Israël et regroupent diverses lois et instructions sur le partage de la terre promise. Ils témoignent également de la révolte et de l'incrédulité de la génération sortie d'Egypte dont la quasi-totalité périt dans le désert.\bigskip
}
}
\par\nobreak\noindent\hrulefill
\begin{multicols}{2}
\Chap{1}
\TextTitle{Dénombrement des hommes de guerre}
\VerseOne{}Or Yahweh parla à Moïse dans le désert de Sinaï, dans la tente d'assignation, le premier jour du second mois, la seconde année, après qu'ils furent sortis du pays d'Egypte, en disant~:
\VS{2}Faites le dénombrement de toute l'assemblée des fils d'Israël, selon leurs familles, selon les maisons de leurs pères, en comptant nom par nom, savoir tous les mâles\FTNT{Ex. 30:12~; Ex. 38:26.}, chacun par tête~;
\VS{3}depuis l'âge de vingt ans et au-dessus, tous ceux d'Israël qui peuvent aller à la guerre, vous les compterez selon leurs armées, toi et Aaron.
\VS{4}Il y aura avec vous un homme par tribu, celui qui est le chef de la maison de ses pères.
\VS{5}Voici les noms des hommes qui vous assisteront. Pour la tribu de Ruben~: Elitsur, fils de Schedéur~;
\VS{6}pour celle de Siméon~: Schelumiel, fils de Tsurischaddaï~;
\VS{7}pour celle de Juda~: Nachschon, fils d'Amminadab~;
\VS{8}pour celle d'Issacar~: Nethaneel, fils de Tsuar~;
\VS{9}pour celle de Zabulon~: Eliab, fils de Hélon~;
\VS{10}pour les fils de Joseph, pour la tribu d'Ephraïm~: Elischama, fils d'Ammihud~; pour celle de Manassé~: Gamliel, fils de Pedahtsur~;
\VS{11}pour la tribu de Benjamin~: Abidan, fils de Guideoni~;
\VS{12}pour celle de Dan~: Ahiézer, fils d'Ammischaddaï~;
\VS{13}pour celle d'Aser~: Paguiel, fils d'Ocran~;
\VS{14}pour celle de Gad~: Eliasaph, fils de Déuel~;
\VS{15}pour celle de Nephthali~: Ahira, fils d'Enan.
\VS{16}C'étaient là ceux qu'on appelait pour tenir l'assemblée~; ils étaient les princes des tribus de leurs pères, chefs des milliers d'Israël.
\VS{17}Alors Moïse et Aaron prirent ces hommes qui avaient été désignés par leurs noms,
\VS{18}et ils convoquèrent toute l'assemblée, le premier jour du second mois. On les enregistra selon leurs familles et selon la maison de leurs pères, en comptant les noms depuis l'âge de vingt ans et au-dessus, chacun par tête.
\VS{19}Comme Yahweh l'avait commandé à Moïse, il les dénombra au désert de Sinaï.
\VS{20}Les fils donc de Ruben, premier-né d'Israël, selon leurs générations, leurs familles, et les maisons de leurs pères, dont on fit le dénombrement par leur nom, et par tête, savoir tous les mâles de l'âge de vingt ans, et au dessus, tous ceux qui pouvaient aller à la guerre.
\VS{21}Ceux, dis-je, de la tribu de Ruben, qui furent dénombrés, furent quarante-six mille cinq cents.
\VS{22}Des enfants de Siméon, selon leurs générations, leurs familles, et les maisons de leurs pères, ceux qui furent dénombrés par leur nom et par tête, savoir tous les mâles de l'âge de vingt ans, et au dessus, tous ceux qui pouvaient aller à la guerre~;
\VS{23}ceux, dis-je, de la tribu de Siméon, qui furent dénombrés, furent cinquante-neuf mille trois cents.
\VS{24}Des fils de Gad, selon leurs générations, leurs familles, et les maisons de leurs pères, dénombrés chacun par leur nom, depuis l'âge de vingt ans, et au dessus, tous ceux qui pouvaient aller à la guerre~;
\VS{25}ceux, dis-je, de la tribu de Gad, qui furent dénombrés, furent quarante-cinq mille six cent cinquante.
\VS{26}Des enfants de Juda, selon leurs générations, leurs familles, et les maisons de leurs pères, dénombrés chacun par leur nom, depuis l'âge de vingt ans, et au dessus, tous ceux qui pouvaient aller à la guerre~;
\VS{27}ceux, dis-je, de la tribu de Juda, qui furent dénombrés, furent soixante-quatorze mille six cents.
\VS{28}Des fils d'Issacar, selon leurs générations, leurs familles, et les maisons de leurs pères, dénombrés chacun par leur nom, depuis l'âge de vingt ans, et au dessus, tous ceux qui pouvaient aller à la guerre~;
\VS{29}ceux, dis-je, de la tribu d'Issacar, qui furent dénombrés, furent cinquante-quatre mille quatre cents.
\VS{30}Des enfants de Zabulon, selon leurs générations, leurs familles, et les maisons de leurs pères, dénombrés chacun par leur nom, depuis l'âge de vingt ans, et au dessus, tous ceux qui pouvaient aller à la guerre~;
\VS{31}ceux, dis-je, de la tribu de Zabulon, qui furent dénombrés, furent cinquante-sept mille quatre cents.
\VS{32}Quant aux fils de Joseph~; les fils d'Ephraïm, selon leurs générations, leurs familles, et les maisons de leurs pères, dénombrés chacun par leur nom, depuis l'âge de vingt ans, et au dessus, tous ceux qui pouvaient aller à la guerre~;
\VS{33}ceux, dis-je, de la tribu d'Ephraïm, qui furent dénombrés, furent quarante mille cinq cents.
\VS{34}Des fils de Manassé, selon leurs générations, leurs familles, et les maisons de leurs pères, dénombrés chacun par leur nom, depuis l'âge de vingt ans, et au dessus, tous ceux qui pouvaient aller à la guerre~;
\VS{35}ceux, dis-je, de la tribu de Manassé, qui furent dénombrés, furent trente-deux mille deux cents.
\VS{36}Des fils de Benjamin, selon leurs générations, leurs familles, et les maisons de leurs pères, dénombrés chacun par leur nom, depuis l'âge de vingt ans, et au dessus, tous ceux qui pouvaient aller à la guerre~;
\VS{37}ceux, dis-je, de la tribu de Benjamin, qui furent dénombrés, furent trente-cinq mille quatre cents.
\VS{38}Des fils de Dan, selon leurs générations, leurs familles, et les maisons de leurs pères, dénombrés chacun par leur nom, depuis l'âge de vingt ans, et au dessus, tous ceux qui pouvaient aller à la guerre~;
\VS{39}ceux, dis-je, de la tribu de Dan qui furent dénombrés, furent soixante-deux mille sept cents.
\VS{40}Des fils d'Aser, selon leurs générations, leurs familles, et les maisons de leurs pères, dénombrés chacun par leur nom, depuis l'âge de vingt ans, et au dessus, tous ceux qui pouvaient aller à la guerre~;
\VS{41}ceux, dis-je, de la tribu d'Aser, qui furent dénombrés, furent quarante et un mille cinq cents.
\VS{42}Des fils de Nephthali, selon leurs générations, leurs familles, et les maisons de leurs pères, dénombrés chacun par leur nom, depuis l'âge de vingt ans, et au dessus, tous ceux qui pouvaient aller à la guerre~;
\VS{43}ceux, dis-je, de la tribu de Nephthali, qui furent dénombrés, furent cinquante-trois mille quatre cents.
\VS{44}Ce sont là ceux dont Moïse et Aaron firent le dénombrement, les douze princes d'entre les enfants d'Israël y étant, un pour chaque maison de leurs pères.
\VS{45}Ainsi tous ceux des enfants d'Israël, dont on fit le dénombrement, selon les maisons de leurs pères, depuis l'âge de vingt ans, et au dessus, tous ceux d'entre les Israélites, qui pouvaient aller à la guerre~;
\VS{46}tous ceux, dis-je, dont on fit le dénombrement, furent six cent trois mille cinq cent cinquante.
\VS{47}Mais les Lévites ne furent point dénombrés avec eux, selon la tribu de leurs pères.
\VS{48}Car Yahweh avait parlé à Moïse, en disant~:
\VS{49}Tu ne feras aucun dénombrement de la tribu de Lévi, et tu n'en lèveras point la somme avec les autres enfants d'Israël.
\VS{50}Mais tu donneras aux Lévites la charge du tabernacle du témoignage, et de tous ses ustensiles, et de tout ce qui lui appartient~; ils porteront le tabernacle, et tous ses ustensiles~; ils y serviront, et camperont autour du tabernacle.
\VS{51}Et quand le tabernacle partira, les Lévites le démonteront, et quand le tabernacle campera, les Lévites le dresseront. Que si quelque étranger en approche, on le fera mourir\FTNT{Ez. 44:8-9.}.
\VS{52}Or les enfants d'Israël camperont chacun dans son camp, et chacun sous sa bannière, selon leurs armées.
\VS{53}Mais les Lévites camperont autour du tabernacle du témoignage, afin qu'il n'y ait point d'indignation sur l'assemblée des enfants d'Israël, et ils prendront en leur charge le tabernacle du Témoignage.
\VS{54}Et les enfants d'Israël firent selon toutes les choses que Yahweh avait commandées à Moïse~; ils le firent ainsi.
\Chap{2}
\TextTitle{Disposition du camp d'Israël par tribu}
\VerseOne{}Et Yahweh parla à Moïse et à Aaron, en disant~:
\VS{2}Les enfants d'Israël camperont chacun sous sa bannière, avec les enseignes des maisons de leurs pères, tout autour de la tente d'assignation, vis-à-vis de lui.
\VS{3}Ceux de la bannière du camp de Juda camperont droit vers l'est, selon ses armées~; et Nachschon, fils d'Amminadab, sera le chef des fils de Juda~;
\VS{4}et son armée, et ses dénombrés, soixante-quatorze mille six cents.
\VS{5}Près de lui campera la tribu d'Issacar, et Nethanaël, fils de Tsuar, sera le chef des enfants d'Issacar~;
\VS{6}et son armée, et ses dénombrés, cinquante-quatre mille quatre cents.
\VS{7}Puis la tribu de Zabulon, et Eliab, fils de Hélon, sera le chef des enfants de Zabulon~;
\VS{8}et son armée, et ses dénombrés, cinquante-sept mille quatre cents.
\VS{9}Tous les dénombrés du camp de Juda, cent quatre-vingt-six mille quatre cents, selon leurs armées, partiront les premiers.
\VS{10}La bannière du camp de Ruben, selon ses armées, sera vers le sud, et Elitsur, fils de Schedéur, sera le chef des enfants de Ruben~;
\VS{11}et son armée, et ses dénombrés, quarante-six mille cinq cents.
\VS{12}Près de lui campera la tribu de Siméon, et Schelumiel, fils de Tsurischaddaï, sera le chef des enfants de Siméon~;
\VS{13}et son armée, et ses dénombrés, cinquante-neuf mille trois cents.
\VS{14}Puis la tribu de Gad, et Eliasaph, fils de Déuel, sera le chef des enfants de Gad~;
\VS{15}et son armée, et ses dénombrés, quarante-cinq mille six cent cinquante.
\VS{16}Tous les dénombrés du camp de Ruben, cent cinquante et un mille quatre cent cinquante, selon leurs armées, partiront les seconds.
\VS{17}Ensuite la tente d'assignation partira avec le camp des Lévites, au milieu des camps qui partiront comme ils auront campés, chacune en sa place, selon leurs bannières.
\VS{18}La bannière du camp d'Ephraïm, selon ses armées, sera vers l'occident~; et Elischama, fils de Ammihud, sera le chef des enfants d'Ephraïm~;
\VS{19}et son armée, et ses dénombrés, quarante mille cinq cents.
\VS{20}Près de lui campera la tribu de Manassé, et Gamliel, fils de Pedahtsur, sera le chef des fils de Manassé~;
\VS{21}et son armée, et ses dénombrés, trente-deux mille deux cents.
\VS{22}Puis la tribu de Benjamin, et Abidan, fils de Guideoni, sera le chef des fils de Benjamin~;
\VS{23}et son armée, et ses dénombrés, trente-cinq mille et quatre cents.
\VS{24}Tous les dénombrés pour le camp d'Ephraïm, cent huit mille et cent, selon leurs armées, partiront les troisièmes.
\VS{25}La bannière du camp de Dan, selon ses armées, sera vers le nord, et Ahiézer, fils de Ammischaddaaï, sera le chef des fils de Dan~;
\VS{26}et son armée, et ses dénombrés, soixante-deux mille sept cents.
\VS{27}Près de lui campera la tribu d'Aser, et Paguiel, fils de Ocran, sera le chef des fils d'Aser~;
\VS{28}et son armée, et ses dénombrés, quarante et un mille cinq cents.
\VS{29}Puis la tribu de Nephthali, et Ahira, fils d'Enan, sera le chef des fils de Nephthali~;
\VS{30}et son armée, et ses dénombrés, cinquante-trois mille quatre cents.
\VS{31}Tous les dénombrés du camp de Dan, cent cinquante-sept mille six cents, partiront les derniers des bannières.
\VS{32}Ce sont là ceux des enfants d'Israël dont on fit le dénombrement selon les maisons de leurs pères. Tous les dénombrés des camps selon leurs armées furent six cent trois mille cinq cent cinquante.
\VS{33}Mais les Lévites ne furent point dénombrés avec les autres enfants d'Israël, comme Yahweh l'avait commandé à Moïse.
\VS{34}Et les enfants d'Israël firent selon toutes les choses que Yahweh avait commandées à Moïse, et campèrent ainsi selon leurs bannières, et partirent ainsi, chacun selon leurs familles, et selon la maison de leurs pères.
\Chap{3}
\TextTitle{Organisation des prêtres et des Lévites}
\VerseOne{}Or ce sont ici les générations d'Aaron et de Moïse, au temps que Yahweh parla à Moïse sur la montagne de Sinaï.
\VS{2}Et ce sont ici les noms des fils d'Aaron~; Nadab, qui était l'aîné, Abihu, Eléazar, et Ithamar.
\VS{3}Ce sont là les noms des fils d'Aaron, les prêtres, qui furent oints et consacrés pour exercer la prêtrise\FTNT{Ex. 40:15~; Lé. 8:30.}.
\VS{4}Mais Nadab et Abihu moururent en la présence de Yahweh, quand ils offrirent un feu étranger devant Yahweh au désert de Sinaï, et ils n'eurent point d'enfants~; mais Eléazar et Ithamar exercèrent la prêtrise en la présence d'Aaron leur père\FTNT{Lé. 10:1-2~; 1 Ch. 24:2.}.
\VS{5}Yahweh parla à Moïse, en disant~:
\VS{6}Fais approcher la tribu de Lévi, et fais qu'elle se tienne devant Aaron, le prêtre, afin qu'ils le servent.
\VS{7}Et qu'ils aient la charge de ce qu'il leur ordonnera de garder, et de ce que toute l'assemblée leur ordonnera de garder, devant la tente d'assignation, en faisant le service du tabernacle.
\VS{8}Et qu'ils gardent tous les ustensiles de la tente d'assignation, et ce qui leur sera donné en charge par les enfants d'Israël, pour faire le service du tabernacle.
\VS{9}Ainsi tu donneras les Lévites à Aaron et à ses fils~; ils lui sont complètement donnés d'entre les enfants d'Israël.
\VS{10}Tu établiras donc Aaron et ses fils, et ils exerceront leur prêtrise. Que si quelque étranger en approche, on le fera mourir.
\VS{11}Et Yahweh parla à Moïse, en disant~:
\VS{12}Voici, j'ai pris les Lévites d'entre les enfants d'Israël, à la place de tout premier-né qui ouvre la matrice parmi les enfants d'Israël~; c'est pourquoi les Lévites seront à moi.
\VS{13}Car tout premier-né m'appartient, depuis le jour où je frappai tout premier-né au pays d'Egypte~; je me suis sanctifié tout premier-né en Israël, depuis les hommes jusqu'aux bêtes~; ils seront à moi, je suis Yahweh\FTNT{Ex. 13:2~; Ex. 22:29~; Ex. 34:19~; Lé. 27:26.}.
\TextTitle{Les familles des Lévites}
\VS{14}Yahweh parla aussi à Moïse au désert de Sinaï, en disant~:
\VS{15}Dénombre les enfants de Lévi, par les maisons de leurs pères, et par leurs familles, en comptant tout mâle depuis l'âge d'un mois, et au dessus.
\VS{16}Et Moïse les dénombra, selon le commandement de Yahweh, ainsi qu'il lui avait été ordonné.
\VS{17}Or ce sont ici les fils de Lévi selon leurs noms~: Guerschon, Kehath, et Merari.
\VS{18}Et ce sont ici les noms des fils de Guerschon, selon leurs familles, Libni, et Schimeï.
\VS{19}Et les fils de Kehath selon leurs familles, Amram, Jitsehar, Hébron et Uziel~;
\VS{20}et les fils de Merari, selon leurs familles, Machli et Muschi~; ce sont là les familles de Lévi, selon les maisons de leurs pères.
\VS{21}De Guerschon est sortie la famille de Libni, et la famille de Schimeï~; ce sont les familles des Guerschonites.
\VS{22}Ceux dont on fit le dénombrement, en comptant de tous les mâles depuis l'âge d'un mois et au dessus, furent au nombre de sept mille cinq cents.
\VS{23}Les familles des Guerschonites camperont derrière le tabernacle à l'occident.
\VS{24}Et Eliasaph, fils de Laël, sera le chef de la maison des pères des Guerschonites.
\TextTitle{Les fonctions des Lévites}
\VS{25}Et les fils de Guerschon auront en charge à la tente d'assignation, la tente, le tabernacle, sa couverture, le rideau de l'entrée de la tente d'assignation.
\VS{26}Et les courtines du parvis avec le rideau de l'entrée du parvis, qui servent pour tabernacle et pour l'autel, tout autour, et son cordage, pour tout son service.
\VS{27}Et de Kehath est sortie la famille des Amramites, la famille des Jitseharites, la famille des Hébronites, et la famille des Uziélites~; ce furent là les familles des Kehathites,
\VS{28}dont tous les mâles depuis l'âge d'un mois, et au dessus, furent au nombre de huit mille six cents, ayant la charge du sanctuaire.
\VS{29}Les familles des fils de Kehath camperont du côté du tabernacle vers le sud.
\VS{30}Et Elitsaphan, fils d'Uziel, sera le chef de la maison des pères des familles des Kehathites.
\VS{31}Et ils auront en charge l'arche, la table, le chandelier, les autels, et les ustensiles du sanctuaire avec lesquels on fait le service, et le rideau, avec tout ce qui y sert.
\VS{32}Et le chef des chefs des Lévites sera Eléazar, fils d'Aaron, le prêtre~; qui aura la surveillance sur ceux qui auront la charge du sanctuaire.
\VS{33}Et de Merari est sortie la famille des Machlites, et la famille des Muschi~; ce furent là les familles de Merari~;
\VS{34}ceux dont on fit le dénombrement, après le compte qui fut fait de tous les mâles, depuis l'âge d'un mois et au dessus, furent six mille deux cents.
\VS{35}Et Esuriel, fils d'Abihaïl, sera le chef de la maison des pères des familles des Merarites~; ils camperont du côté du tabernacle vers le nord.
\VS{36}Et on donnera aux enfants de Merari la surveillance des planches du tabernacle, de ses barres, de ses piliers, de ses bases, et de tous ses ustensiles, avec tout ce qui y sera~;
\VS{37}et des piliers du parvis tout autour, avec leurs bases, leurs pieux, et leurs cordes.
\VS{38}Et Moïse, et Aaron et ses fils campaient devant le tabernacle, à l'orient, devant la tente d'assignation, vers l'orient~; ils avaient la garde et le soin du sanctuaire, remis à la garde des enfants d'Israël~; et si quelque étranger en approche, on le fera mourir.
\VS{39}Tous ceux des Lévites dont on fit le dénombrement, lesquels Moïse et Aaron comptèrent par leurs familles, suivant le commandement de Yahweh, tous les mâles de l'âge d'un mois et au dessus, furent de vingt-deux mille.
\TextTitle{Le rachat des premiers-nés}
\VS{40}Yahweh dit à Moïse~: Fais le dénombrement de tous les premiers-nés mâles des enfants d'Israël, depuis l'âge d'un mois, et au dessus, et relève le nombre de leurs noms.
\VS{41}Et tu prendras pour moi, je suis Yahweh, les Lévites, à la place de tous les premiers-nés qui sont entre les enfants d'Israël~; tu prendras aussi les bêtes des Lévites, à la place de tous les premiers-nés des bêtes des enfants d'Israël.
\VS{42}Moïse fit le dénombrement, comme Yahweh lui avait commandé, de tous les premiers-nés qui étaient parmi les enfants d'Israël.
\VS{43}Et tous les premiers-nés des mâles, selon le nombre des noms, depuis l'âge d'un mois et au dessus, selon leur dénombrement, furent vingt-deux mille deux cent soixante-treize.
\VS{44}Et Yahweh parla à Moïse, en disant~:
\VS{45}Prends les Lévites à la place de tous les premiers-nés qui sont parmi les enfants d'Israël, et les bêtes des Lévites, à la place de leurs bêtes~; et les Lévites seront à moi~; je suis Yahweh.
\VS{46}Et quant à ceux qu'il faut racheter, les deux cent soixante-treize parmi les premiers-nés des fils d'Israël, qui sont de plus que les Lévites,
\VS{47}tu prendras cinq sicles par tête, tu les prendras selon le sicle du sanctuaire~; le sicle est de vingt guéras\FTNT{Ex. 30:13~; Lé. 27:6~; Lé. 27:25~; Ez. 45:12.}.
\VS{48}Et tu donneras à Aaron et à ses fils l'argent de ceux qui auront été rachetés, dépassant le nombre des Lévites.
\VS{49}Moïse donc prit l'argent du rachat de ceux qui étaient de plus, outre ceux qui avaient été rachetés par l'échange des Lévites.
\VS{50}Et il reçut l'argent des premiers-nés des enfants d'Israël, qui fut mille trois cent soixante-cinq sicles, selon le sicle du sanctuaire.
\VS{51}Et Moïse donna l'argent des rachetés à Aaron, et à ses fils, selon le commandement de Yahweh, ainsi que Yahweh le lui avait commandé.
\Chap{4}
\TextTitle{Les fonctions des fils de Kehath}
\VerseOne{}Et Yahweh parla à Moïse et à Aaron, en disant~:
\VS{2}Faites le dénombrement des fils de Kehath d'entre les enfants de Lévi par leurs familles, et par les maisons de leurs pères,
\VS{3}depuis l'âge de trente ans et au dessus, jusqu'à l'âge de cinquante ans, tous ceux qui entrent en rang, pour s'employer à la tente d'assignation.
\VS{4}C'est ici le service des fils de Kehath à la tente d'assignation, c'est-à-dire, le Saint des saints.
\VS{5}Quand le camp partira, Aaron et ses fils viendront démonter le voile\FTNT{Le voile intérieur est l'image du corps humain de Christ (Mt. 26:26). Ce voile fut déchiré de haut en bas lorsque le Seigneur est mort sur la croix (Mt. 27:50-51). Désormais, le croyant peut pénétrer dans la présence du Père (Hé. 10:19-20).} qui sert de rideau, et en couvriront l'arche du témoignage~;
\VS{6}puis ils mettront au dessus une couverture de peaux de taissons, ils étendront par dessus un drap de pourpre, et ils y mettront ses barres.
\VS{7}Et ils étendront un drap de pourpre sur la table des pains de proposition, et mettront sur elle les plats, les tasses, les bassins, et les calices de libations. Le pain continuel sera sur elle.
\VS{8}Ils étendront au dessus un drap teint de cramoisi, ils le couvriront d'une couverture de peaux de taissons, et ils y mettront ses barres.
\VS{9}Et ils prendront un drap de pourpre, en couvriront le chandelier du luminaire avec ses lampes, ses mouchettes, ses vases à cendre, et tous ses vases à huile, dont on fait usage pour son service\FTNT{Ex. 25:30-38.}~;
\VS{10}ils le mettront avec tous ses ustensiles, dans une couverture de peaux de taissons, et le mettront sur une perche.
\VS{11}Ils étendront sur l'autel d'or un drap de pourpre, ils le couvriront d'une couverture de peaux de taissons, et ils y mettront ses barres.
\VS{12}Ils prendront aussi tous les ustensiles du service dont on se sert dans le lieu saint, ils les mettront dans un drap de pourpre, et ils les couvriront d'une couverture de peaux de taissons, et les mettront sur des perches.
\VS{13}Ils ôteront les cendres de l'autel, et étendront dessus un drap de pourpre.
\VS{14}Et ils mettront dessus les ustensiles dont on se sert pour l'autel, les brasiers, les fourchettes, les pelles, les bassins, et tous les ustensiles de l'autel~; ils étendront dessus une couverture de peaux de taissons, et ils y mettront ses barres.
\VS{15}Le camp partira après qu'Aaron et ses fils auront achevé de couvrir le lieu saint et tous ses ustensiles, et après cela les fils de Kehath viendront pour le porter, et ils ne toucheront point les choses saintes, de peur qu'ils ne meurent~; c'est là ce que les fils de Kehath porteront de la tente d'assignation.
\TextTitle{Les fonctions d'Eléazar}
\VS{16}Et Eléazar fils d'Aaron, le prêtre, aura la surveillance de l'huile du luminaire, du parfum odoriférant, de l'offrande continuelle, et de l'huile de l'onction~; la charge de tout le tabernacle, et de toutes les choses qui sont dans le lieu saint, et de ses ustensiles\FTNT{Ex. 30:23-35.}.
\VS{17}Yahweh parla à Moïse et à Aaron, en disant~:
\VS{18}Ne retranchez pas la tribu des familles des Kehathites d'entre les Lévites.
\VS{19}Mais faites ceci pour eux, afin qu'ils vivent et ne meurent point~; c'est que quand ils approcheront du Saint des saints, Aaron et ses fils viendront, qui les placeront chacun à son service, et à sa charge.
\VS{20}Et ils n'entreront point pour regarder quand on enveloppera les choses saintes, afin qu'ils ne meurent point.
\TextTitle{Les fonctions des fils de Guerschon}
\VS{21}Yahweh parla à Moïse, en disant~:
\VS{22}Fais aussi le dénombrement des fils de Guerschon selon les maisons de leurs pères, et selon leurs familles~;
\VS{23}depuis l'âge de trente ans, et au dessus, jusqu'à l'âge de cinquante ans, dénombrant tous ceux qui entrent pour tenir leur rang, afin de s'employer à servir à la tente d'assignation.
\VS{24}C'est ici le service des familles des Guerschonites, ce à quoi, ils doivent servir et en ce qu'ils doivent porter.
\VS{25}Ils porteront donc les tapis du tabernacle, et la tente d'assignation, sa couverture, la couverture de peaux de taissons qui est sur lui par dessus, et le rideau de l'entrée de la tente d'assignation~;
\VS{26}les courtines du parvis, et le rideau de l'entrée de la porte du parvis, qui servent pour le tabernacle et pour l'autel tout autour, leurs cordages, et tous les ustensiles de leur service, et tout ce qui est fait pour eux~; c'est ce en quoi ils serviront.
\VS{27}Tout le service des fils de Guerschonites en tout ce qu'ils doivent porter, et en tout ce à quoi ils doivent servir, sera réglé par les ordres d'Aaron et de ses fils, et vous les chargerez d'observer tout ce qu'ils doivent porter.
\VS{28}C'est là le service des familles des fils des Guerschonites dans la tente d'assignation~; et leur charge sera sous la conduite d'Ithamar, fils d'Aaron, le prêtre.
\TextTitle{Les fonctions des fils de Merari}
\VS{29}Tu dénombreras aussi les fils de Mérari selon leurs familles et selon les maisons de leurs pères.
\VS{30}Tu les dénombreras depuis l'âge de trente ans et au dessus, jusqu'à l'âge de cinquante ans, tous ceux qui entrent en rang pour s'employer au service dans la tente d'assignation.
\VS{31}Or c'est ici la charge de ce qu'ils auront à porter, selon tout le service qu'ils auront à faire à la tente d'assignation, savoir les planches du tabernacle, ses barres, et ses piliers, avec ses bases\FTNT{Ex. 26:15.},
\VS{32}et les piliers du parvis tout autour, et leurs bases, leurs pieux, leurs cordages, tous leurs ustensiles, et tout ce dont on se sert en ces choses-là, et vous leur compterez, en les désignant par nom, tous les ustensiles, qu'ils auront en charge de porter, pièce par pièce.
\VS{33}C'est là le service des familles des fils de Merari, pour tout leur service à la tente d'assignation, sous la conduite d'Ithamar, fils d'Aaron, le prêtre.
\VS{34}Moïse, Aaron et les princes de l'assemblée dénombrèrent les fils des Kéhathites, selon leurs familles, et selon les maisons de leurs pères.
\VS{35}Depuis l'âge de trente ans, et au dessus, jusqu'à l'âge de cinquante ans, tous ceux qui entraient en rang pour servir à la tente d'assignation.
\VS{36}Et ceux dont on fit le dénombrement selon leurs familles, étaient deux mille sept cent cinquante.
\VS{37}Ce sont là les dénombrés des familles des Kéhathites, tous servant à la tente d'assignation, que Moïse et Aaron dénombrèrent selon le commandement que Yahweh avait fait par le moyen de Moïse.
\VS{38}Or quant aux dénombrés des fils de Guerschon selon leurs familles, et selon les maisons de leurs pères,
\VS{39}depuis l'âge de trente ans, et au dessus, jusqu'à l'âge de cinquante ans, tous ceux qui entraient en rang pour servir à la tente d'assignation,
\VS{40}ceux, dis-je, qui en furent dénombrés selon leurs familles, et selon les maisons de leurs pères, étaient deux mille six cent trente.
\VS{41}Ce sont là, les dénombrés des familles des fils de Guerschon, tous servant dans la tente d'assignation, que Moïse et Aaron dénombrèrent selon le commandement de Yahweh.
\VS{42}Et quant aux dénombrés des familles des fils de Merari, selon leurs familles, et selon les maisons de leurs pères,
\VS{43}depuis l'âge de trente ans, et au dessus, jusqu'à l'âge de cinquante ans, tous ceux qui entraient en rang, pour servir à la tente d'assignation~;
\VS{44}ceux, dis-je, qui en furent dénombrés selon leurs familles, étaient trois mille deux cents.
\VS{45}Ce sont là, les dénombrés des familles des fils de Merari, que Moïse et Aaron dénombrèrent selon le commandement que Yahweh avait fait par le moyen de Moïse.
\VS{46}Ainsi tous ces dénombrés, que Moïse, Aaron et les princes d'Israël dénombrèrent d'entre les Lévites, selon leurs familles, et selon les maisons de leurs pères~;
\VS{47}depuis l'âge de trente ans, et au dessus, jusqu'à l'âge de cinquante ans, tous ceux qui entraient en service pour s'employer en ce à quoi il fallait servir, et à ce qu'il fallait porter de la tente d'assignation.
\VS{48}Tous ceux, dis-je, qui en furent dénombrés, étaient huit mille cinq cent quatre-vingts.
\VS{49}On les dénombra selon le commandement que Yahweh en avait fait par le moyen de Moïse, chacun selon ce en quoi il avait à servir, et ce qu'il avait à porter, et la charge de chacun fut telle que Yahweh l'avait commandé à Moïse.
\Chap{5}
\TextTitle{Mise en garde contre toute souillure~; lois diverses}
\VerseOne{}Et Yahweh parla à Moïse, en disant~:
\VS{2}Ordonne aux enfants d'Israël qu'ils mettent hors du camp tout lépreux, tout homme ayant une gonorrhée, et tout homme souillé pour un mort\FTNT{Lé. 13~; Lé. 15.}.
\VS{3}Vous les mettrez dehors, tant l'homme que la femme, vous les mettrez, dis-je, hors du camp, afin qu'ils ne souillent point le camp au milieu duquel j'habite.
\VS{4}Et les enfants d'Israël firent ainsi, et les envoyèrent hors du camp, comme Yahweh l'avait dit à Moïse~; les enfants d'Israël firent ainsi.
\VS{5}Et Yahweh parla à Moïse, en disant~:
\VS{6}Parle aux enfants d'Israël~; quand un homme ou une femme aura commis un des péchés que l'homme commet en faisant un crime contre Yahweh, et qu'une telle personne en sera trouvée coupable~;
\VS{7}alors ils confesseront leur péché, qu'ils auront commis~; et le coupable restituera la somme totale de ce en quoi il aura été trouvé coupable, et il y ajoutera un cinquième par-dessus, et le donnera à celui contre qui il aura commis le délit.
\VS{8}Que si cet homme n'a personne à qui appartienne le droit de restituer pour retirer ce en quoi aura été commis le délit, cette chose-là sera restituée à Yahweh, et elle appartiendra au prêtre, outre le bélier expiatoire avec lequel on fera propitiation pour lui.
\VS{9}De même, toute offrande élevée d'entre toutes les choses sanctifiées des enfants d'Israël, qu'ils présenteront\FTNT{Ez. 44:30.} au prêtre, lui appartiendra.
\VS{10} Les choses donc que quelqu'un aura sanctifiées appartiendront au prêtre~; ce que chacun lui aura donné, lui appartiendra\FTNT{Lé. 10:12-13.}.
\VS{11}Yahweh parla à Moïse, en disant~:
\VS{12}Parle aux enfants d'Israël, et dis leur~: Si la femme de quelqu'un se détourne et lui devienne infidèle~;
\VS{13}et que quelqu'un aura couché avec elle, et l'aura connue, sans que son mari en ait rien su, mais qu'elle se soit cachée, et qu'elle se soit souillée, et qu'il n'y ait point de témoin contre elle, et qu'elle n'ait point été surprise~;
\VS{14}et que l'esprit de jalousie saisisse son mari, tellement qu'il soit jaloux de sa femme, parce qu'elle s'est souillée~; ou que l'esprit de jalousie le saisisse tellement, qu'il soit jaloux de sa femme, encore qu'elle ne se soit point souillée~;
\VS{15}cet homme-là fera venir sa femme devant le prêtre, et il apportera l'offrande de cette femme pour elle, savoir la dixième partie d'un epha de farine d'orge~; mais il ne répandra point d'huile dessus~; et il n'y mettra point d'encens~; car c'est un gâteau de jalousie, un gâteau de souvenir, pour remettre en mémoire l'iniquité\FTNT{Lé. 5:11.}.
\VS{16}Le prêtre la fera approcher et la fera tenir debout devant Yahweh.
\VS{17}Puis le prêtre prendra de l'eau sainte dans un vase de terre, et il prendra de la poussière qui sera sur le sol du tabernacle, et la mettra dans l'eau.
\VS{18}Ensuite le prêtre fera tenir debout la femme devant Yahweh, il découvrira la tête de cette femme, et lui posera sur les paumes des mains le gâteau de souvenir, le gâteau de jalousie~; le prêtre tiendra dans sa main les eaux amères, qui apportent la malédiction.
\VS{19}Et le prêtre fera jurer la femme et lui dira~: Si aucun homme n'a couché avec toi, et si étant sous la puissance de ton mari tu ne t'es point détournée et souillée, sois exempte du mal de ces eaux amères qui apportent la malédiction.
\VS{20}Mais si, étant sous la puissance de ton mari, tu t'es détournée et souillée, et si un autre homme que ton mari a couché avec toi,
\VS{21}alors le prêtre fera jurer la femme avec un serment d'imprécation et lui dira~: Que Yahweh te livre à la malédiction et à l'exécration au milieu de ton peuple, en faisant flétrir ta cuisse et enfler ton ventre,
\VS{22}et que ces eaux qui apportent la malédiction, entrent dans tes entrailles pour te faire enfler le ventre et flétrir ta cuisse~! Alors la femme répondra~: Amen~! Amen~!
\VS{23}Ensuite le prêtre écrira dans un livre ces imprécations, et les effacera avec les eaux amères.
\VS{24}Et il fera boire à la femme les eaux amères qui apportent la malédiction, et les eaux qui apportent la malédiction entreront en elle pour être amères.
\VS{25}Le prêtre donc prendra des mains de la femme le gâteau de jalousie, et l'agitera de côté et d'autre devant Yahweh, et l'offrira sur l'autel~;
\VS{26}le prêtre prendra une poignée de cette offrande comme souvenir\FTNT{Voir commentaire en Lé. 2:2.}, et il la brûlera sur l'autel. C'est après cela qu'il fera boire les eaux à la femme.
\VS{27}Et après qu'il lui aura fait boire les eaux, s'il est vrai qu'elle se soit souillée et qu'elle a été infidèle à son mari, les eaux qui apportent la malédiction entreront en elle et lui seront amères, et son ventre enflera, sa cuisse se flétrira, et cette femme sera assujettie à l'exécration du serment au milieu de son peuple.
\VS{28}Mais si la femme ne s'est point souillée, mais qu'elle soit pure, elle sera reconnue innocente et aura des enfants.
\VS{29}Telle est la loi sur la jalousie, quand la femme qui est sous la puissance de son mari se détourne et se souille,
\VS{30}ou quand un mari saisi d'un esprit de jalousie a des soupçons sur sa femme~: Le prêtre la fera tenir debout devant Yahweh et fera à l'égard de cette femme tout ce qui est ordonné par cette loi.
\VS{31}Le mari sera exempt de faute, mais cette femme portera son iniquité.
\Chap{6}
\TextTitle{Le vœu de naziréat}
\VerseOne{}Yahweh parla à Moïse, en disant~:
\VS{2}Parle aux enfants d'Israël, et dis-leur~: Lorsqu'un homme ou une femme se consacrera en faisant un vœu de naziréat pour se consacrer à Yahweh,
\VS{3}il s'abstiendra de vin et de boisson forte, il ne boira ni vinaigre fait de vin, ni vinaigre fait avec une boisson forte~; il ne boira d'aucune liqueur de raisins, et il ne mangera point de raisins, frais ou secs.
\VS{4}Durant tous les jours de son naziréat il ne mangera d'aucun fruit de la vigne, depuis les pépins jusqu'à la peau du raisin\FTNT{Jg. 13:7~; Lu. 1:15.}.
\VS{5}Le rasoir ne passera point sur sa tête durant tous les jours de son naziréat. Il sera saint jusqu'à ce que les jours pour lesquels il s'est consacré à Yahweh soient accomplis, et il laissera croître les cheveux de sa tête\FTNT{Jg. 13:5~; 1 S. 1:11.}.
\VS{6}Durant tous les jours pour lesquels il s'est consacré à Yahweh il ne s'approchera d'aucune personne morte\FTNT{Lé. 21:1-4.}~;
\VS{7}il ne se souillera point à la mort de son père, ni de sa mère, ni de son frère, ni de sa sœur, car il porte sur sa tête la consécration de son Dieu.
\VS{8}Durant tous les jours de son naziréat, il sera consacré à Yahweh.
\VS{9}Que si quelqu'un vient à mourir subitement près de lui, la tête de son naziréat sera souillée, et il rasera sa tête au jour de sa purification, il la rasera le septième jour.
\VS{10}Le huitième jour, il apportera au prêtre deux tourterelles ou deux pigeonneaux, à l'entrée de la tente d'assignation\FTNT{Lé. 1~; Lé. 12:6.}.
\VS{11}Et le prêtre en sacrifiera l'un pour le sacrifice d'expiation et l'autre en holocauste, et il fera propitiation pour lui de ce qu'il a péché à l'occasion du mort. Il sanctifiera donc ainsi sa tête en ce jour-là.
\VS{12}Et il séparera à Yahweh les jours de son naziréat, offrant un agneau d'un an pour le délit, et les premiers jours seront comptés pour rien, car son naziréat a été souillé.
\VS{13}Or c'est ici la loi du naziréen. Lorsque les jours de son naziréat seront accomplis, on le fera venir à la porte de la tente d'assignation.
\VS{14}Il présentera son offrande à Yahweh~: Un agneau d'un an et sans défaut pour l'holocauste, une brebis d'un an et sans défaut pour le sacrifice d'expiation, et un bélier sans défaut pour le sacrifice d'offrande de paix\FTNT{Voir commentaire en Lé. 3:1.}.
\VS{15}Une corbeille de pains sans levain, de gâteaux de fine farine, pétrie à l'huile, et de galettes sans levain, oints d'huile, avec leur gâteau, et leurs libations~;
\VS{16}lesquels, le prêtre offrira devant Yahweh~; il sacrifiera aussi son offrande pour le péché, et son holocauste.
\VS{17}Et il offrira le bélier en sacrifice d'offrande de paix à Yahweh, avec la corbeille des pains sans levain~; le prêtre offrira aussi son gâteau, et sa libation.
\VS{18}Et le naziréen rasera la tête de son naziréat à l'entrée de la tente d'assignation, et prendra les cheveux de la tête de son naziréat, et les mettra sur le feu qui est sous le sacrifice d'offrande de paix.
\VS{19}Et le prêtre prendra l'épaule cuite du bélier, et un gâteau sans levain de la corbeille, et une galette sans levain, et les mettra sur les paumes des mains du naziréen, après qu'il se sera fait raser son naziréat.
\VS{20}Et le prêtre les agitera de côté et d'autre devant Yahweh~: C'est une chose sainte qui appartient au prêtre, avec la poitrine agitée et l'épaule offerte par élévation. Et après cela le naziréen boira du vin\FTNT{Lé. 7:32-34~; Ex. 29:24-27.}.
\VS{21}Telle est la loi du naziréen qui aura voué à Yahweh son offrande pour son naziréat, outre ce qu'il aura encore moyen d'offrir~; il fera selon son vœu qu'il aura voué, suivant la loi de son naziréat.
\TextTitle{Aaron et ses fils bénissent Israël}
\VS{22}Yahweh parla à Moïse, en disant~:
\VS{23}Parle à Aaron et à ses fils, et dis-leur~: Vous bénirez ainsi les enfants d'Israël, en leur disant~:
\VS{24}Yahweh te bénisse, et te garde~!
\VS{25}Yahweh fasse luire sa face sur toi, et te fasse grâce\FTNT{Ps. 67:2~; Ps. 119:135.}~!
\VS{26}Yahweh tourne sa face vers toi, et te donne la paix~!
\VS{27}Ils mettront donc mon Nom sur les enfants d'Israël, et je les bénirai.
\Chap{7}
\TextTitle{Les offrandes des princes}
\VerseOne{}Or il arriva le jour que Moïse eut achevé de dresser le tabernacle, et qu'il l'eut oint et sanctifié avec tous ses ustensiles, de même que l'autel avec tous ses ustensiles, il arriva, dis-je, après qu'il les eut oints et sanctifiés~;
\VS{2}que les princes d'Israël, et les chefs des maisons de leurs pères, qui sont les princes des tribus, et qui avaient assisté à faire les dénombrements, firent leur offrande.
\VS{3}Et ils amenèrent leur offrande devant Yahweh~: Six chars couverts et douze bœufs~; chaque char pour deux des princes, et chaque bœuf pour chacun d'eux~; ils les offrirent devant le tabernacle.
\VS{4}Alors Yahweh parla à Moïse, en disant~:
\VS{5}Prends d'eux ces choses, et elles seront employées pour le service de la tente d'assignation~; et tu les donneras aux Lévites, à chacun selon ses fonctions.
\VS{6}Moïse prit donc les chars et les bœufs, et il les remit aux Lévites.
\VS{7}Il donna aux fils de Guerschon deux chars et quatre bœufs, selon leurs fonctions.
\VS{8}Mais il donna aux fils de Merari quatre chars et huit bœufs, selon leurs fonctions, sous la conduite d'Ithamar, fils d'Aaron, le prêtre.
\VS{9}Or il n'en donna point aux fils de Kehath, parce que le service du sanctuaire était de leur charge~; ils portaient ces choses saintes sur les épaules.
\VS{10}Et les princes présentèrent leur offrande pour la dédicace de l'autel, le jour où on l'oignit~; les princes, dis-je, présentèrent leur offrande devant l'autel.
\VS{11}Et Yahweh dit à Moïse~: Un des princes offrira un jour, et un autre l'autre jour, son offrande pour la dédicace de l'autel.
\VS{12}Le premier jour donc, Nachschon, fils d'Amminadab, présenta son offrande pour la tribu de Juda.
\VS{13}Il offrit un plat d'argent du poids de cent trente sicles, un bassin d'argent de soixante-dix sicles, selon le sicle du sanctuaire, tous deux pleins de fine farine pétrie à l'huile, pour l'offrande~;
\VS{14}une coupe d'or de dix sicles pleine de parfum~;
\VS{15}un jeune taureau, un bélier, un agneau d'un an, pour l'holocauste~;
\VS{16}un jeune bouc pour le sacrifice d'expiation~;
\VS{17}et pour le sacrifice d'offrande de paix, deux bœufs, cinq béliers, cinq boucs, et cinq agneaux d'un an. Telle fut l'offrande de Nachschon, fils d'Amminadab.
\VS{18}Le second jour, Nethaneel, fils de Tsuar, chef de la tribu d'Issacar, présenta son offrande.
\VS{19}Et il offrit pour son offrande un plat d'argent du poids de cent trente sicles, un bassin d'argent de soixante-dix sicles, selon le sicle du sanctuaire, tous deux pleins de fine farine pétrie à l'huile, pour l'offrande~;
\VS{20}une coupe d'or de dix sicles pleine de parfum~;
\VS{21}un jeune taureau, un bélier, un agneau d'un an, pour l'holocauste~;
\VS{22}un jeune bouc pour le sacrifice d'expiation~;
\VS{23}et pour le sacrifice d'offrande de paix, deux bœufs, cinq béliers, cinq boucs, et cinq agneaux d'un an. Telle fut l'offrande de Nethaneel, fils de Tsuar.
\VS{24}Le troisième jour, Eliab, fils de Hélon, chef des fils de Zabulon, présenta son offrande.
\VS{25}Il offrit un plat d'argent du poids de cent trente sicles, un bassin d'argent de soixante-dix sicles, selon le sicle du sanctuaire, tous deux pleins de fine farine pétrie à l'huile, pour l'offrande~;
\VS{26}une coupe d'or de dix sicles pleine de parfum~;
\VS{27}un jeune taureau, un bélier, un agneau d'un an, pour l'holocauste~;
\VS{28}un jeune bouc pour le sacrifice d'expiation~;
\VS{29}et pour le sacrifice d'offrande de paix, deux bœufs, cinq béliers, cinq boucs, et cinq agneaux d'un an. Telle fut l'offrande d'Eliab, fils de Hélon.
\VS{30}Le quatrième jour, Elitsur, fils de Schedéur, prince des fils de Ruben, présenta son offrande.
\VS{31}Il offrit un plat d'argent du poids de cent trente sicles, un bassin d'argent de soixante-dix sicles, selon le sicle du sanctuaire, tous deux pleins de fine farine pétrie à l'huile, pour l'offrande~;
\VS{32}une coupe d'or de dix sicles pleine de parfum~;
\VS{33}un jeune taureau, un bélier, un agneau d'un an, pour l'holocauste~;
\VS{34}un jeune bouc pour le sacrifice d'expiation~;
\VS{35}et pour le sacrifice d'offrande de paix, deux bœufs, cinq béliers, cinq boucs, et cinq agneaux d'un an. Telle fut l'offrande d'Elitsur, fils de Schedéur.
\VS{36}Le cinquième jour, Schelumiel, fils de Tsurischaddaï, prince des fils de Siméon, présenta son offrande.
\VS{37}Il offrit un plat d'argent du poids de cent trente sicles, un bassin d'argent de soixante-dix sicles, selon le sicle du sanctuaire, tous deux pleins de fine farine pétrie à l'huile pour l'offrande~;
\VS{38}une coupe d'or de dix sicles pleine de parfum~;
\VS{39}un jeune taureau, un bélier, un agneau d'un an, pour l'holocauste~;
\VS{40}un jeune bouc pour le sacrifice d'expiation~;
\VS{41}et pour le sacrifice d'offrande de paix, deux bœufs, cinq béliers, cinq boucs, et cinq agneaux d'un an. Telle fut l'offrande de Schelumiel, fils de Tsurischaddaï.
\VS{42}Le sixième jour, Eliasaph, fils de Déuel, prince des fils de Gad, présenta son offrande.
\VS{43}Il offrit un plat d'argent du poids de cent trente sicles, un bassin d'argent de soixante-dix sicles, selon le sicle du sanctuaire, tous deux pleins de fine farine pétrie à l'huile pour l'offrande~;
\VS{44}une coupe d'or de dix sicles pleine de parfum~;
\VS{45}un jeune taureau, un bélier, un agneau d'un an, pour l'holocauste~;
\VS{46}un jeune bouc pour le sacrifice d'expiation~;
\VS{47}et pour le sacrifice d'offrande de paix, deux bœufs, cinq béliers, cinq boucs, et cinq agneaux d'un an. Telle fut l'offrande d'Eliasaph, fils de Déuel.
\VS{48}Le septième jour, Elischama, fils d'Ammihud, prince des fils d'Ephraïm, présenta son offrande.
\VS{49}Il offrit un plat d'argent, du poids de cent trente sicles, un bassin d'argent de soixante-dix sicles, selon le sicle du sanctuaire, tous deux pleins de fine farine pétrie à l'huile pour l'offrande~;
\VS{50}une coupe d'or de dix sicles pleine de parfum~;
\VS{51}un jeune taureau, un bélier, un agneau d'un an, pour l'holocauste~;
\VS{52}un jeune bouc pour le sacrifice d'expiation~;
\VS{53}et pour le sacrifice d'offrande de paix, deux bœufs, cinq béliers, cinq boucs, et cinq agneaux d'un an. Telle fut l'offrande d'Elischama, fils d'Ammihud.
\VS{54}Le huitième jour, Gamliel, fils de Pedahtsur, prince des fils de Manassé, présenta son offrande.
\VS{55}Il offrit un plat d'argent, du poids de cent trente sicles, un bassin d'argent de soixante-dix sicles, selon le sicle du sanctuaire, tous deux pleins de fine farine pétrie à l'huile pour l'offrande~;
\VS{56}une coupe d'or de dix sicles pleine de parfum~;
\VS{57}un jeune taureau, un bélier, un agneau d'un an, pour l'holocauste~;
\VS{58}un jeune bouc pour le sacrifice d'expiation~;
\VS{59}et pour le sacrifice d'offrande de paix, deux bœufs, cinq béliers, cinq boucs, et cinq agneaux d'un an. Telle fut l'offrande de Gamliel, fils de Pedahtsur.
\VS{60}Le neuvième jour, Abidan, fils de Guideoni, prince des fils de Benjamin, présenta son offrande.
\VS{61}Il offrit un plat d'argent, du poids de cent trente sicles, un bassin d'argent de soixante-dix sicles, selon le sicle du sanctuaire, tous deux pleins de fine farine pétrie à l'huile pour l'offrande~;
\VS{62}une coupe d'or de dix sicles pleine de parfum~;
\VS{63}un jeune taureau, un bélier, un agneau d'un an, pour l'holocauste~;
\VS{64}un jeune bouc pour le sacrifice d'expiation~;
\VS{65}et pour le sacrifice d'offrande de paix, deux bœufs, cinq béliers, cinq boucs, et cinq agneaux d'un an. Telle fut l'offrande d'Abidan, fils de Guideoni.
\VS{66}Le dixième jour, Ahiézer, fils d'Ammischaddaï, prince des fils de Dan, présenta son offrande.
\VS{67}Il offrit un plat d'argent du poids de cent trente sicles, un bassin d'argent de soixante-dix sicles, selon le sicle du sanctuaire, tous deux pleins de fine farine pétrie à l'huile pour l'offrande~;
\VS{68}une coupe d'or de dix sicles pleine de parfum~;
\VS{69}Un jeune taureau, un bélier, un agneau d'un an, pour l'holocauste~;
\VS{70}un jeune bouc pour le sacrifice d'expiation~;
\VS{71}et pour le sacrifice d'offrande de paix, deux bœufs, cinq béliers, cinq boucs, et cinq agneaux d'un an. Telle fut l'offrande d'Ahiézer, fils d'Ammischaddaï.
\VS{72}Le onzième jour, Paguiel, fils d'Ocran, prince des fils d'Aser, présenta son offrande.
\VS{73}Il offrit un plat d'argent, du poids de cent trente sicles, un bassin d'argent de soixante-dix sicles, selon le sicle du sanctuaire, tous deux pleins de fine farine pétrie à l'huile pour l'offrande~;
\VS{74}une coupe d'or de dix sicles pleine de parfum~;
\VS{75}un jeune taureau, un bélier, un agneau d'un an, pour l'holocauste~;
\VS{76}un jeune bouc pour le sacrifice d'expiation~;
\VS{77}et pour le sacrifice d'offrande de paix, deux bœufs, cinq béliers, cinq boucs, et cinq agneaux d'un an. Telle fut l'offrande de Paguiel, fils d'Ocran.
\VS{78}Le douzième jour, Ahira, fils d'Enan, prince des fils de Nephthali, présenta son offrande.
\VS{79}Il offrit un plat d'argent du poids de cent trente sicles, un bassin d'argent de soixante-dix sicles, selon le sicle du sanctuaire, tous deux pleins de fine farine pétrie à l'huile pour l'offrande~;
\VS{80}une coupe d'or de dix sicles pleine de parfum~;
\VS{81}un jeune taureau, un bélier, un agneau d'un an, pour l'holocauste~;
\VS{82}un jeune bouc pour le sacrifice d'expiation~;
\VS{83}et pour le sacrifice d'offrande de paix, deux bœufs, cinq béliers, cinq boucs, et cinq agneaux d'un an. Telle fut l'offrande d'Ahira, fils d'Enan.
\TextTitle{Les dons des princes}
\VS{84}Telle fut la dédicace de l'autel, qui fut faite par les princes d'Israël, lorsqu'il fut oint. Douze plats d'argent, douze bassins d'argent, douze tasses d'or~;
\VS{85}chaque plat d'argent était de cent trente sicles, et chaque bassin de soixante-dix~; tout l'argent de ces ustensiles montait à deux mille quatre cents sicles, selon le sicle du sanctuaire~;
\VS{86}douze coupes d'or pleines de parfum, chacune de dix sicles, selon le sicle du sanctuaire~; tout l'or des tasses montait à cent-vingt sicles.
\VS{87}Tous les animaux pour l'holocauste étaient douze veaux, douze béliers, et douze agneaux d'un an, avec leurs offrandes, et douze jeunes boucs pour le sacrifice d'expiation.
\VS{88}Tous les animaux du sacrifice d'offrande de paix étaient vingt-quatre veaux, avec soixante béliers, soixante boucs, et soixante agneaux d'un an. Telle fut donc la dédicace de l'autel, après qu'on l'eut oint.
\VS{89}Et quand Moïse entrait dans la tente d'assignation pour parler avec Yahweh, il entendait une voix qui lui parlait du haut du propitiatoire placé sur l'arche du témoignage, entre les deux chérubins. Et il lui parlait\FTNT{Ex. 25:22.}.
\Chap{8}
\TextTitle{Les lampes sur le chandelier}
\VerseOne{}Yahweh parla à Moïse, en disant~:
\VS{2}Parle à Aaron, et tu lui diras~: Quand tu allumeras les lampes, les sept lampes éclaireront sur le devant du chandelier\FTNT{Ex. 25:37.}.
\VS{3}Et Aaron fit ainsi~; il plaça les lampes pour éclairer sur le devant du chandelier, comme Yahweh l'avait commandé à Moïse.
\VS{4}Or le chandelier était fait de telle manière, qu'il était d'or battu au marteau, d'ouvrage fait au marteau, sa tige aussi, et ses fleurs. On fit ainsi le chandelier selon le modèle que Yahweh en avait fait voir à Moïse\FTNT{Ex. 25:31-40.}.
\TextTitle{Purification des Lévites}
\VS{5}Puis Yahweh parla à Moïse, en disant~:
\VS{6}Prends les Lévites du milieu des enfants d'Israël, et purifie-les.
\VS{7}Tu leur feras ainsi pour les purifier. Tu feras aspersion sur eux de l'eau de purification~; ils feront passer le rasoir sur toute leur chair, ils laveront leurs vêtements, et ils se purifieront.
\VS{8}Puis ils prendront un jeune taureau avec son offrande de gâteau de fine farine pétrie à l'huile~; et tu prendras un autre jeune taureau pour le sacrifice d'expiation.
\VS{9}Alors tu feras approcher les Lévites devant la tente d'assignation, et tu convoqueras toute l'assemblée des enfants d'Israël.
\VS{10}Tu feras, dis-je, approcher les Lévites devant Yahweh, et les enfants d'Israël poseront leurs mains sur les Lévites.
\VS{11}Et Aaron fera tourner de côté et d'autre les Lévites devant Yahweh, comme offrande de la part des enfants d'Israël, et ils seront employés au service de Yahweh.
\VS{12}Et les Lévites poseront leurs mains sur la tête des veaux~; puis tu offriras l'un en sacrifice pour l'expiation, et l'autre en holocauste à Yahweh, afin de faire propitiation pour les Lévites.
\VS{13}Après tu feras tenir les Lévites devant Aaron et devant ses fils, et tu les présenteras en offrande à Yahweh.
\VS{14}Ainsi tu sépareras les Lévites du milieu des enfants d'Israël, et les Lévites m'appartiendront.
\VS{15}Après cela, les Lévites viendront pour servir dans la tente d'assignation quand tu les auras purifiés et présentés en offrande.
\VS{16}Car ils me sont entièrement donnés du milieu des enfants d'Israël~; je les ai pris pour moi à la place des premiers-nés~; de tous les premiers-nés des fils d'Israël.
\VS{17}Car tout premier-né des enfants d'Israël est à moi, tant des hommes que des animaux~; je me les suis consacrés le jour où j'ai frappé tous les premiers-nés dans le pays d'Egypte.
\VS{18}Or j'ai pris les Lévites au lieu de tous les premiers-nés d'entre les enfants d'Israël.
\VS{19}Et j'ai entièrement donné, d'entre les enfants d'Israël, les Lévites à Aaron et à ses fils, pour faire le service des enfants d'Israël dans la tente d'assignation, et pour faire propitiation pour les enfants d'Israël~; afin qu'il n'y ait point de plaie sur les enfants d'Israël, comme il y aurait si les enfants d'Israël s'approchaient du sanctuaire.
\VS{20}Moïse, Aaron et toute l'assemblée des enfants d'Israël firent à l'égard des Lévites tout ce que Yahweh avait ordonné à Moïse touchant les Lévites~; ainsi firent les enfants d'Israël.
\VS{21}Les Lévites donc se purifièrent, et lavèrent leurs vêtements, et Aaron les fit tourner de côté et d'autre comme une offrande devant Yahweh, et il fit propitiation pour eux afin de les purifier.
\VS{22}Cela étant fait, les Lévites vinrent faire leur service dans la tente d'assignation devant Aaron, et devant ses fils, selon ce que Yahweh avait commandé à Moïse touchant les Lévites~; ainsi fut-il fait à leur égard.
\VS{23}Puis Yahweh parla à Moïse, en disant~:
\VS{24}Voici ce qui concerne les Lévites. Depuis l'âge de vingt-cinq ans et au-dessus, tout Lévite entrera en fonction dans la tente d'assignation~;
\VS{25}dès l'âge de cinquante ans, il sortira du service et ne servira plus.
\VS{26}Cependant il servira ses frères dans la tente d'assignation, pour garder ce qui leur a été commis, mais il ne fera plus de service. Tu agiras ainsi à l'égard des Lévites pour ce qui concerne leurs fonctions.
\Chap{9}
\TextTitle{La Pâque}
\VerseOne{}Yahweh avait aussi parlé à Moïse dans le désert de Sinaï, le premier mois de la seconde année, après qu'ils furent sortis du pays d'Egypte, en disant~:
\VS{2}Que les enfants d'Israël célèbrent la Pâque\FTNT{Ex. 12~; 1 Co. 5:7.} au temps fixé.
\VS{3}Vous la ferez en sa saison, le quatorzième jour de ce mois entre les deux soirs, selon toutes ses ordonnances et selon tout ce qu'il faut y faire.
\VS{4}Moïse donc, parla aux enfants d'Israël afin qu'ils célèbrent la Pâque.
\VS{5}Et ils firent la Pâque le quatorzième jour du premier mois, entre les deux soirs, dans le désert de Sinaï~; selon tout ce que Yahweh avait commandé à Moïse, les enfants d'Israël le firent ainsi.
\VS{6}Or il y eut quelques-uns qui étaient impurs à cause d'un mort et qui ne purent célébrer la Pâque ce jour-là. Ils se présentèrent ce même jour devant Moïse et devant Aaron,
\VS{7}et ces hommes leur dirent~: Nous sommes impurs à cause d'un mort, pourquoi serions-nous privés de présenter l'offrande à Yahweh dans sa saison au milieu des enfants d'Israël~?
\VS{8}Et Moïse leur dit~: Arrêtez-vous, et j'entendrai ce que Yahweh commandera sur votre sujet.
\VS{9}Alors Yahweh parla à Moïse, en disant~:
\VS{10}Parle aux enfants d'Israël, et dis-leur~: Si quelqu'un d'entre vous, ou de votre postérité, est impur à cause d'un mort, ou est en voyage dans un lieu éloigné, il célébrera cependant la Pâque en l'honneur de Yahweh.
\VS{11}Ils la feront le quatorzième jour du second mois, entre les deux soirs~; et ils la mangeront avec du pain sans levain et des herbes amères\FTNT{Ex. 12:10~; Ex. 23:18~; Ex. 34:25~; De. 16:4~; Jn. 19:33-36.}.
\VS{12}Ils n'en laisseront rien jusqu'au matin, et n'en briseront point les os. Ils la feront selon toutes les ordonnances de la Pâque.
\VS{13}Mais si celui qui est pur et qui n'est pas en voyage s'abstient de célébrer la Pâque, il sera retranché d'entre ses peuples parce qu'il n'a pas présenté l'offrande de Yahweh en sa saison.
\VS{14}Et si un étranger en séjour chez vous célèbre la Pâque de Yahweh, il la fera selon l'ordonnance de la Pâque. II y aura une même ordonnance entre vous, pour l'étranger comme pour celui qui est né au pays\FTNT{Ex. 12:49.}.
\TextTitle{La nuée conduit Israël}
\VS{15}Or le jour où le tabernacle fut dressé, la nuée couvrit le tabernacle de la tente d'assignation~; et le soir jusqu'au matin, elle parut sur le tabernacle avec l'apparence d'un feu\FTNT{Ex. 13:21-22~; Ex. 40:34-38~; De. 1:33.}.
\VS{16}Il en fut ainsi continuellement~; la nuée le couvrait, mais elle paraissait la nuit comme du feu.
\VS{17}Et selon que la nuée se levait de dessus le tabernacle, les enfants d'Israël partaient~; et au lieu où la nuée s'arrêtait, les enfants d'Israël y campaient.
\VS{18}Les enfants d'Israël marchaient sur le commandement de Yahweh, et ils campaient sur le commandement de Yahweh~; ils campaient aussi longtemps que la nuée se tenait sur le tabernacle.
\VS{19}Et quand la nuée restait plusieurs jours sur le tabernacle, les enfants d'Israël observaient l'ordre de Yahweh, et ne partaient point.
\VS{20}Et pour peu de jours que la nuée fût sur le tabernacle, ils campaient sur le commandement de Yahweh, et ils partaient sur le commandement de Yahweh.
\VS{21}Et quand la nuée y était depuis le soir jusqu'au matin, et que la nuée se levait au matin, ils partaient~; fût-ce de jour ou de nuit, quand la nuée se levait, ils partaient.
\VS{22}Si la nuée s'arrêtait sur le tabernacle deux jours, ou un mois, ou plus longtemps, les enfants d'Israël restaient campés, et ne partaient point~; mais quand elle se levait, ils partaient.
\VS{23}Ils campaient donc au commandement de Yahweh, et ils partaient au commandement de Yahweh~; et ils prenaient garde à Yahweh, suivant le commandement de Yahweh, qu'il leur faisait savoir par Moïse.
\Chap{10}
\TextTitle{Les trompettes d'argent}
\VerseOne{}Puis Yahweh parla à Moïse, en disant~:
\VS{2}Fais-toi deux trompettes d'argent, battues au marteau. Elles te serviront pour convoquer l'assemblée, et pour le départ des camps.
\VS{3}Quand on en sonnera, toute l'assemblée s'assemblera auprès de toi à l'entrée de la tente d'assignation.
\VS{4}Et quand on sonnera d'une seule, les princes, qui sont les chefs des milliers d'Israël, s'assembleront vers toi.
\VS{5}Mais quand vous sonnerez avec un retentissement bruyant, ceux qui campent à l'orient partiront.
\VS{6}Et quand vous sonnerez la seconde fois avec un retentissement bruyant, ceux qui campent au midi partiront, on sonnera avec un retentissement bruyant, pour leur départ.
\VS{7}Lorsque vous convoquerez l'assemblée, vous ne sonnerez pas avec un retentissement bruyant.
\VS{8}Or les fils d'Aaron, les prêtres, sonneront des trompettes. Ce sera une loi perpétuelle pour vous et pour vos descendants.
\VS{9}Et lorsque, dans votre pays, vous irez à la guerre contre l'ennemi qui vous combattra, vous sonnerez des trompettes avec un retentissement bruyant, et Yahweh votre Dieu, se souviendra de vous, et vous serez délivrés de vos ennemis.
\VS{10}Aussi dans vos jours de joie, dans vos fêtes solennelles, et au commencement de vos mois, vous sonnerez des trompettes en offrant vos holocaustes et vos sacrifices d'offrande de paix, et elles vous serviront de souvenir devant votre Dieu. Je suis Yahweh, votre Dieu.
\TextTitle{La nuée se lève, reprise de la marche dans le désert}
\VS{11}Or il arriva le vingtième jour du second mois de la seconde année, que la nuée se leva de dessus le tabernacle du témoignage.
\VS{12}Et les enfants d'Israël partirent du désert de Sinaï, selon l'ordre fixé pour leur marche. La nuée se posa dans le désert de Paran.
\VS{13}Ils partirent donc pour la première fois, suivant le commandement de Yahweh, déclaré par Moïse.
\VS{14}Et la bannière du camp des fils de Juda partit la première, selon leurs armées. Nachschon, fils d'Amminadab, commandait l'armée de Juda~;
\VS{15}et Nethaneel, fils de Tsuar, commandait l'armée de la tribu des fils d'Issacar~;
\VS{16}et Eliab, fils de Hélon, commandait l'armée de la tribu des fils de Zabulon.
\VS{17}Et le tabernacle fut démonté~; et les fils de Guerschon, et les fils de Merari, qui portaient le tabernacle, partirent.
\VS{18}Puis la bannière du camp de Ruben partit, selon leurs armées. Et Elitsur, fils de Schedéur, commandait l'armée de Ruben~;
\VS{19}et Schelumiel, fils de Tsurischaddaï, commandait l'armée de la tribu des fils de Siméon~;
\VS{20}et Eliasaph, fils de Déuel, commandait l'armée des fils de Gad.
\VS{21}Alors les Kehathites, qui portaient le sanctuaire, partirent~; cependant on dressait le tabernacle, en attendant leur arrivée.
\VS{22}Puis la bannière du camp des fils d'Ephraïm partit, selon leurs armées. Elischama, fils d'Ammihud, commandait l'armée d'Ephraïm~;
\VS{23}et Gamliel, fils de Pedahtsur, commandait l'armée de la tribu des fils de Manassé~;
\VS{24}et Abidan, fils de Guideoni, commandait l'armée de la tribu des fils de Benjamin.
\VS{25}Enfin la bannière des camps des fils de Dan, qui faisait l'arrière-garde, partit, selon leurs armées~; et Ahiézer, fils d'Ammischaddaï, commandait l'armée de Dan.
\VS{26}Et Paguiel, fils d'Ocran, commandait l'armée de la tribu des fils d'Aser~;
\VS{27}et Ahira, fils d'Enan, commandait l'armée de la tribu des fils de Nephthali.
\VS{28}Tel fut l'ordre d'après lequel les enfants d'Israël se mirent en marche selon leurs armées, c'est ainsi qu'ils partirent.
\VS{29}Or Moïse dit à Hobab, fils de Réuel, le Madianite, beau-père de Moïse~: Nous allons au lieu dont Yahweh a dit~: Je vous le donnerai. Viens avec nous, et nous te ferons du bien, car Yahweh a promis de faire du bien à Israël.
\VS{30}Et Hobab lui répondit~: Je n'irai point, mais je m'en irai dans mon pays, et vers ma parenté.
\VS{31}Et Moïse lui dit~: Je te prie, ne nous quitte pas~; car tu nous serviras de guide, parce que tu connais les lieux où nous aurons à camper dans le désert.
\VS{32}Et il arrivera que, quand tu seras venu avec nous, et que le bien que Yahweh doit nous faire sera arrivé, nous te ferons aussi du bien.
\VS{33}Ainsi ils partirent de la montagne de Yahweh et ils marchèrent trois jours~; et l'arche de l'alliance de Yahweh alla devant eux, et fit une marche de trois jours pour leur chercher un lieu de repos.
\VS{34}Et la nuée de Yahweh était sur eux le jour, quand ils partaient du camp.
\VS{35}Or il arrivait qu'au départ de l'arche, Moïse disait~: Lève-toi, ô Yahweh, et tes ennemis seront dispersés, et ceux qui te haïssent s'enfuiront de devant toi\FTNT{Ps. 68:2.}~!
\VS{36}Et quand on la posait, il disait~: Reviens Yahweh, aux dix mille milliers d'Israël~!
\Chap{11}
\TextTitle{Jugement contre les murmures du peuple}
\VerseOne{}Après, il arriva que le peuple murmura et cela déplut aux oreilles de Yahweh. Lorsque Yahweh l'entendit, sa colère s'enflamma, et le feu de Yahweh s'alluma parmi eux et en consuma l'extrémité du camp.
\VS{2}Alors le peuple cria à Moïse. Moïse pria Yahweh, et le feu s'éteignit.
\VS{3}Et on nomma ce lieu-là Tabeéra, parce que le feu de Yahweh s'était allumé parmi eux.
\TextTitle{Le peuple regrette l'Egypte}
\VS{4}Et le peuple nombreux qui se trouvaient au milieu d'Israël fut épris de convoitise~; et même, les enfants d'Israël se mirent à pleurer disant~: Qui nous donnera de la viande à manger\FTNT{Ex. 16:3~; Ps. 106:14~; 1 Co. 10:6.}~?
\VS{5}Nous nous souvenons des poissons que nous mangions en Egypte, et qui ne nous coûtaient rien, des concombres, des melons, des poireaux, des oignons, et de l'ail.
\VS{6}Et maintenant nos âmes sont asséchées~; nos yeux ne voient que de la manne\FTNT{Ps. 78:24.}.
\VS{7}Or la manne était comme la graine de coriandre, et avait l'apparence du bdellium\FTNT{Ex. 16:14-31~; Jn. 6:31-58.}.
\VS{8}Le peuple se dispersait et la ramassait, il la moulait aux meules, ou la pilait dans un mortier, il la cuisait au pot et en faisait des gâteaux. Elle avait le goût d'une liqueur d'huile fraîche.
\VS{9}Et quand la rosée descendait la nuit sur le camp, la manne y descendait aussi.
\TextTitle{Moïse dans l'affliction}
\VS{10}Moïse donc entendit le peuple qui pleurait, chacun dans sa famille et à l'entrée de sa tente. La colère de Yahweh s'enflamma fortement et Moïse en fut attristé.
\VS{11}Et Moïse dit à Yahweh~: Pourquoi affliges-tu ton serviteur et pourquoi n'ai-je pas trouvé grâce à tes yeux, que tu aies mis sur moi la charge de tout ce peuple~?
\VS{12}Est-ce moi qui ai conçu tout ce peuple ou l'ai-je engendré pour que tu me dises~: Porte-le dans ton sein comme le nourricier porte un enfant qui tète, porte-le jusqu'au pays que tu as juré à ses pères~?
\VS{13}D'où aurais-je de la viande pour en donner à tout ce peuple~? Car il pleure auprès de moi, en disant~: Donne-nous de la viande à manger~!
\VS{14}Je ne puis, à moi seul, porter tout ce peuple, car il est trop pesant pour moi\FTNT{De. 1:9-12.}.
\VS{15}Si tu agis ainsi à mon égard, tue-moi, je te prie donc, si j'ai trouvé grâce à tes yeux, et que je ne voie pas mon malheur.
\TextTitle{Yahweh établit soixante-dix anciens autour de Moïse\FTNTT{Ex. 18:19.}}
\VS{16}Alors Yahweh dit à Moïse~: Assemble-moi soixante-dix hommes des anciens d'Israël, que tu connais être les anciens du peuple et ses officiers, et amène-les à la tente d'assignation, et qu'ils s'y présentent avec toi.
\VS{17}Puis je descendrai, et je parlerai là avec toi, je mettrai de l'Esprit qui est sur toi sur eux~; afin qu'ils portent avec toi la charge du peuple, et que tu ne la portes pas toi seul.
\VS{18}Et tu diras au peuple~: Sanctifiez-vous pour demain, et vous mangerez de la viande~; puisque vous avez pleuré aux oreilles de Yahweh, en disant~: Qui nous fera manger de la viande~? Car nous étions bien en Egypte. Ainsi Yahweh vous donnera de la viande, et vous en mangerez.
\VS{19}Vous n'en mangerez pas un jour, ni deux jours, ni cinq jours, ni dix jours, ni vingt jours,
\VS{20}mais jusqu'à un mois entier, jusqu'à ce qu'elle vous sorte par les narines, et que vous en ayez du dégoût, parce que vous avez rejeté Yahweh qui est au milieu de vous~; vous avez pleuré devant lui, en disant~: Pourquoi sommes-nous sortis d'Egypte~?
\VS{21}Moïse dit~: Six cent mille hommes de pied forment ce peuple au milieu duquel je suis, et tu as dit~: Je leur donnerai de la viande afin qu'ils en mangent un mois entier~!
\VS{22}Leur tuera-t-on des brebis ou des bœufs, en sorte qu'il y en ait assez pour eux~? Ou leur assemblera-t-on tous les poissons de la mer, en sorte qu'ils en aient assez~?
\VS{23}Yahweh répondit à Moïse: La main de Yahweh serait-elle trop courte~? Tu verras maintenant si ce que je t'ai dit arrivera ou non\FTNT{Es. 50:2~; Es. 59:1-2.}.
\VS{24}Moïse donc sortit et rapporta au peuple les paroles de Yahweh. Il assembla soixante-dix hommes des anciens du peuple, et les plaça autour de la tente.
\VS{25}Yahweh descendit dans la nuée et parla à Moïse~; il prit de l'Esprit qui était sur lui et le mit sur les soixante-dix hommes anciens. Et dès que l'Esprit reposa sur eux, ils prophétisèrent~; mais ils ne continuèrent pas.
\TextTitle{Prophétie d'Eldad et de Médad}
\VS{26}Or il y eut deux hommes restés au camp, l'un s'appelait Eldad, et l'autre Médad, sur lesquels l'Esprit reposa. Ils étaient de ceux qui avaient été inscrits, mais ils n'étaient pas allés à la tente, et ils prophétisaient dans le camp.
\VS{27}Alors un garçon courut le rapporter à Moïse, en disant~: Eldad et Médad prophétisent dans le camp.
\VS{28}Et Josué, fils de Nun, qui servait Moïse, l'un de ses jeunes gens, répondit, en disant~: Mon seigneur Moïse, empêche-les.
\VS{29}Et Moïse lui répondit~: Es-tu jaloux pour moi~? Plût à Dieu que tout le peuple de Yahweh fût prophète, et que Yahweh mît son Esprit sur eux~!
\VS{30}Puis Moïse se retira au camp, lui et les anciens d'Israël.
\TextTitle{Les cailles et le jugement de Yahweh}
\VS{31}Alors Yahweh fit lever un vent de la mer qui amena des cailles et les répandit sur le camp environ le chemin d'une journée, de çà et de là, tout autour du camp~; et il y en avait presque la hauteur de deux coudées sur la terre\FTNT{Ex. 16:13-15~; Ps. 78:26-29~; Ps. 105:40.}.
\VS{32}Et le peuple se leva tout ce jour-là, et toute la nuit, et tout le jour suivant, et amassa des cailles~; celui qui en avait amassé le moins en avait dix homers~; et ils les étendirent soigneusement pour eux tout autour du camp.
\VS{33}Mais la chair était encore entre leurs dents, avant qu'elle fût mâchée, la colère de Yahweh s'embrasa contre le peuple, et il frappa le peuple d'une très grande plaie\FTNT{Ps. 78:30-31.}.
\VS{34}Et on nomma ce lieu-là Kibroth-Hattaava~; car on ensevelit là le peuple qui avait convoité.
\VS{35}Et de Kibroth-Hattaava le peuple s'en alla pour Hatséroth, et il s'arrêta à Hatséroth.
\Chap{12}
\TextTitle{Marie et Aaron murmurent contre Moïse}
\VerseOne{}Alors Marie et Aaron parlèrent contre Moïse au sujet de la femme éthiopienne\FTNT{Voir commentaire en Ge. 2:13.} qu'il avait prise, car il avait pris une femme éthiopienne.
\VS{2}Et ils dirent~: Est-ce seulement par Moïse que Yahweh parle~? N'est-ce pas aussi par nous qu'il parle~? Et Yahweh entendit cela. 
\VS{3}Or cet homme Moïse était un homme fort doux, plus que tous les hommes qui étaient sur la terre.
\VS{4}Et soudain Yahweh dit à Moïse, à Aaron, et à Marie~: Venez vous trois à la tente d'assignation~; et ils y allèrent eux trois.
\VS{5}Alors Yahweh descendit dans la colonne de nuée et se tint à l'entrée de la tente. Puis il appela Aaron et Marie, qui s'avancèrent tous les deux.
\VS{6}Et il dit~: Ecoutez maintenant mes paroles~! Lorsqu'il y aura parmi vous un prophète, moi qui suis Yahweh je me ferai bien connaître à lui en vision, et je lui parlerai en songe.
\VS{7}Il n'en est pas ainsi de mon serviteur Moïse, qui est fidèle dans toute ma maison\FTNT{Hé. 3:2.}.
\VS{8}Je parle avec lui bouche à bouche, et il me voit en effet, et non point en obscurité, ni dans aucune représentation de Yahweh. Pourquoi donc n'avez-vous pas craint de parler contre mon serviteur, contre Moïse~?
\FTNT{Ex. 33:11~; De. 34:10.} 
\VS{9}Ainsi la colère de Yahweh s'embrasa contre eux. Et il s'en alla.
\VS{10}Car la nuée se retira de dessus la tente. Et voici, Marie était frappée d'une lèpre blanche comme la neige~; et Aaron se tourna vers Marie et la vit lépreuse.
\VS{11}Alors Aaron dit à Moïse~: Hélas, de grâce, mon seigneur~! Je te prie ne mets point sur nous ce péché, car nous avons fait follement, et nous avons péché.
\VS{12}Je te prie qu'elle ne soit pas comme un enfant mort-né, dont la moitié de la chair est déjà consumée quand il sort du ventre de sa mère~!
\VS{13}Alors Moïse cria à Yahweh, en disant~: Ô Dieu, je te prie, guéris-la, je t'en prie.
\VS{14}Et Yahweh répondit à Moïse~: Si son père lui avait craché au visage, ne serait-elle pas dans l'ignominie pendant sept jours~? Qu'elle soit enfermée sept jours en dehors du camp, après quoi, elle y sera reçue\FTNT{Lé. 13:46.}.
\VS{15}Ainsi Marie fut enfermée hors du camp sept jours~; et le peuple ne partit pas de là jusqu'à ce que Marie fût rentrée.
\VS{16}Après cela le peuple partit de Hatséroth, et il campa dans le désert de Paran.
\Chap{13}
\TextTitle{Douze espions envoyés pour explorer Canaan}
\VerseOne{}Et Yahweh parla à Moïse, en disant~:
\VS{2}Envoie des hommes pour explorer le pays de Canaan, que je donne aux enfants d'Israël. Tu enverras un homme de chaque tribu de leurs pères, tous seront des principaux d'entre eux.
\VS{3}Moïse donc les envoya du désert de Paran, d'après l'ordre de Yahweh~; et tous ces hommes étaient chefs des enfants d'Israël.
\VS{4}Et ce sont ici leurs noms~: De la tribu de Ruben~: Schammua, fils de Zaccur~;
\VS{5}de la tribu de Siméon~: Schaphath, fils de Hori~;
\VS{6}de la tribu de Juda~: Caleb, fils de Jephunné~;
\VS{7}de la tribu d'Issacar~: Jigual, fils de Joseph~;
\VS{8}de la tribu d'Ephraïm~: Hosée, fils de Nun~;
\VS{9}de la tribu de Benjamin~: Palthi, fils de Raphu~;
\VS{10}de la tribu de Zabulon~: Gaddiel, fils de Sodi~;
\VS{11}de l'autre tribu de Joseph~: la tribu de Manassé, Gaddi, fils de Susi~;
\VS{12}de la tribu de Dan~: Ammiel, fils de Guemalli~;
\VS{13}de la tribu d'Aser~: Sethur, fils de Micaël~;
\VS{14}de la tribu de Nephthali~: Nachbi, fils de Vophsi~;
\VS{15}de la tribu de Gad~: Guéuel, fils de Maki.
\VS{16}Ce sont là les noms des hommes que Moïse envoya pour explorer le pays. Moïse donna à Hosée, fils de Nun, le nom de Josué\FTNT{Moïse changea le nom d'Hosée en y ajoutant le Nom de Yahweh. Hosée signifie «~sauveur~» et Josué (ou Jésus) «~Yahweh est salut~». Josué préfigurait Jésus-Christ qui nous a délivrés et transportés dans le Royaume des cieux (Col. 1:12-14). Moïse avait compris prophétiquement que seul Jésus peut nous faire rentrer dans notre héritage.}.
\VS{17}Moïse les envoya pour explorer le pays de Canaan, et il leur dit~: Montez de ce côté par le sud~; et vous monterez sur la montagne.
\VS{18}Et vous verrez quel est ce pays-là, et quel est le peuple qui l'habite, s'il est fort ou faible~; s'il est en petit ou en grand nombre.
\VS{19}Et quel est le pays où il habite, s'il est bon ou mauvais~; et quelles sont les villes dans lesquelles il habite, si c'est dans des camps, ou dans des villes fortifiées.
\VS{20}Et quelle est la terre, si elle est grasse ou maigre, s'il y a des arbres, ou non. Ayez bon courage, et prenez du fruit du pays. Or c'était alors le temps des premiers raisins.
\VS{21}Etant donc partis, ils examinèrent le pays, depuis le désert de Tsin jusqu'à Rehob, à l'entrée de Hamath.
\VS{22}Ils montèrent par le sud, et ils allèrent jusqu'à Hébron, où étaient Ahiman, Schéschaï, et Talmaï, enfants d'Anak. Hébron avait été bâtie sept ans avant Tsoan en Egypte.
\VS{23}Et ils vinrent jusqu'au torrent d'Eschcol, et coupèrent de là un sarment de vigne, avec une grappe de raisins~; ils étaient deux à le porter avec une perche. Ils apportèrent aussi des grenades et des figues.
\VS{24}Et on donna à ce lieu le nom de vallée d'Eschcol~; à cause de la grappe que les fils d'Israël y coupèrent.
\VS{25}Et au bout de quarante jours, ils furent de retour du pays qu'ils étaient allés explorer.
\TextTitle{Comptes rendus des envoyés}
\VS{26}Et à leur arrivée, ils se rendirent auprès de Moïse et d'Aaron, et de toute l'assemblée des enfants d'Israël, dans le désert de Padan à Kadès. Ils leur firent le rapport, ainsi qu'à toute l'assemblée, ils leur montrèrent les fruits du pays.
\VS{27}Ils firent donc leur rapport à Moïse, et lui dirent~: Nous avons été dans le pays où tu nous as envoyés. Véritablement, c'est un pays où coulent le lait et le miel, et en voici les fruits.
\VS{28}Seulement, le peuple qui habite ce pays est puissant, les villes sont fortifiées, très grandes~; nous y avons vu des enfants d'Anak\FTNT{De. 1:24-28.}.
\VS{29}Les Amalécites habitent la contrée du midi~; les Héthiens, les Jébusiens et les Amoréens habitent la montagne~; les Cananéens habitent le long de la mer, et vers le rivage du Jourdain.
\VS{30}Caleb fit taire le peuple devant Moïse, et il dit~: Montons, possédons ce pays, car nous y serons vainqueurs~!
\VS{31}Mais les hommes qui y étaient montés avec lui dirent~: Nous ne pouvons pas monter contre ce peuple-là, car il est plus fort que nous.
\VS{32}Et ils décrièrent devant les enfants d'Israël le pays qu'ils avaient exploré, en disant~: Le pays que nous avons parcouru pour l'explorer est un pays qui dévore ses habitants et tous ceux que nous y avons vus sont des gens de grande taille.
\VS{33}Et nous y avons vu aussi des géants, des enfants d'Anak, de la race des géants et nous étions à nos yeux et à leurs yeux comme des sauterelles.
\Chap{14}
\TextTitle{Rébellion et incrédulité d'Israël\FTNTT{1 Co. 10:1-5~; Hé. 3:7-19}}
\VerseOne{}Alors toute l'assemblée éleva la voix et se mit à pousser des cris, et le peuple pleura cette nuit-là.
\VS{2}Et tous les enfants d'Israël murmurèrent contre Moïse et Aaron, et toute l'assemblée leur dit~: Oh~! Si nous étions morts dans le pays d'Egypte~! Ou si nous étions morts dans ce désert\FTNT{De. 1:26-27.}~!
\VS{3}Et pourquoi Yahweh nous fait-il aller dans ce pays, où nous tomberons par l'épée, où nos femmes et nos petits enfants deviendront une proie~? Ne vaut-il pas mieux retourner en Egypte~?
\VS{4}Et ils se dirent l'un à l'autre~: Etablissons-nous un chef, et retournons en Egypte.
\VS{5}Alors Moïse et Aaron tombèrent sur leurs visages devant toute l'assemblée des enfants d'Israël.
\VS{6}Et Josué, fils de Nun, et Caleb, fils de Jephunné, qui étaient parmi ceux qui avaient exploré le pays, déchirèrent leurs vêtements,
\VS{7}et parlèrent à toute l'assemblée des enfants d'Israël, en disant~: Le pays que nous avons exploré est un très bon pays.
\VS{8}Si nous sommes agréables à Yahweh, il nous fera entrer dans ce pays, et il nous le donnera. C'est un pays où coulent le lait et le miel.
\VS{9}Seulement, ne soyez point rebelles contre Yahweh, et ne craignez point le peuple de ce pays-là, car ils seront notre pain, leur protection s'est retirée de dessus eux. Yahweh est avec nous, ne les craignez point\FTNT{De. 20:3-4.}~!
\VS{10}Alors toute l'assemblée parlait de les lapider~; mais la gloire de Yahweh apparut à tous les enfants d'Israël, devant la tente d'assignation.
\TextTitle{Moïse intercède pour le pardon d'Israël}
\VS{11}Et Yahweh dit à Moïse~: Jusqu'à quand ce peuple-ci m'irritera-t-il par mépris et jusqu'à quand ne croira-t-il point en moi, malgré tous les signes que j'ai faits au milieu de lui~?
\VS{12}Je le frapperai par la peste et je le détruirai, mais je ferai de toi une nation plus grande et plus puissante que lui.
\VS{13}Et Moïse dit à Yahweh~: Mais les Egyptiens l'entendront, car tu as fait monter par ta puissance ce peuple-ci du milieu d'eux\FTNT{Ex. 32:10-12.},
\VS{14}et ils diront avec les habitants de ce pays qui auront entendu que tu étais, ô Yahweh, au milieu de ce peuple, et que tu apparaissais, ô Yahweh à vue d'œil, que ta nuée s'arrêtait sur eux, et que tu marchais devant eux le jour dans la colonne de nuée, et la nuit dans la colonne de feu~;
\VS{15}si tu fais mourir ce peuple comme un seul homme, les nations qui ont entendu parler de toi diront~:
\VS{16}Yahweh n'avait pas le pouvoir de faire entrer ce peuple dans le pays qu'il avait juré de leur donner, il l'a égorgé dans le désert.
\VS{17}Maintenant, je te prie, que la puissance du Seigneur se montre dans sa grandeur, comme tu l'as déclaré en disant~:
\VS{18}Yahweh est lent à la colère et riche en bonté, il ôte l'iniquité et pardonne la rébellion, mais il ne tient point le coupable pour innocent, et il punit l'iniquité des pères sur les fils, jusqu'à la troisième et à la quatrième génération\FTNT{Ex. 20:5~; Ex. 34:6~; Ex. 34:7~; Ps. 86:15~; Ps. 103:8~; Ps. 145:8~; Jon. 4:2~; De. 5:9.}.
\VS{19}Pardonne, je te prie, l'iniquité de ce peuple, selon la grandeur de ta miséricorde, comme tu as pardonné à ce peuple depuis l'Egypte jusqu'ici.
\TextTitle{Réponse de Yahweh à Moïse}
\VS{20}Et Yahweh dit~: Je pardonne selon ta parole.
\VS{21}Mais certainement je suis vivant, et la gloire de Yahweh remplira toute la terre.
\VS{22}Car tous ceux qui ont vu ma gloire, et les prodiges que j'ai faits en Egypte et dans le désert, qui m'ont déjà tenté par dix fois, et qui n'ont point écouté ma voix,
\VS{23}tous ceux-là ne verront point le pays que j'ai juré à leurs pères de leur donner, tous ceux, dis-je, qui m'ont irrité par mépris, ne le verront pas\FTNT{De. 1:35-38.}.
\VS{24}Mais parce que mon serviteur Caleb a été animé d'un autre esprit, et qu'il a persévéré à me suivre, je le ferai entrer dans le pays où il a été, et ses descendants le posséderont en héritage.
\VS{25}Or les Amalécites et les Cananéens habitent la vallée. Demain, tournez-vous et partez pour le désert, dans la direction de la Mer Rouge.
\VS{26}Yahweh parla à Moïse et à Aaron, en disant~:
\VS{27}Jusqu'à quand laisserai-je cette méchante assemblée murmurer contre moi~? J'ai entendu les murmures des enfants d'Israël, qui murmuraient contre moi\FTNT{Ps. 106:25.}.
\VS{28}Dis-leur~: Je suis vivant, dit Yahweh, je vous ferai ainsi que vous avez parlé à mes oreilles.
\VS{29}Vos cadavres tomberont dans ce désert, et tous ceux d'entre vous qui ont été dénombrés, selon tout le compte que vous en avez fait, depuis l'âge de vingt ans, et au dessus, vous tous qui avez murmuré contre moi~;
\VS{30}vous n'entrerez pas dans le pays que j'avais juré de vous faire habiter, excepté Caleb, fils de Jephunné, et Josué, fils de Nun.
\VS{31}Et quant à vos petits enfants, dont vous avez dit~: Ils deviendront une proie~! Je les y ferai entrer, et ils connaîtront le pays que vous avez méprisé.
\VS{32}Mais quant à vous, vos cadavres tomberont dans ce désert~;
\VS{33}mais vos enfants paîtront dans ce désert quarante ans et ils porteront la peine de vos prostitutions, jusqu'à ce que vos cadavres soient tous consumés dans le désert.
\VS{34}Selon le nombre des jours que vous avez mis à reconnaître le pays, qui ont été quarante jours, un jour pour une année, vous porterez la peine de vos iniquités quarante ans, et vous connaîtrez ma rupture de promesse.
\VS{35}Je suis Yahweh, j'ai parlé~! C'est ainsi que je traiterai cette méchante assemblée, qui s'est assemblée contre moi~; ils seront consumés dans ce désert, et ils y mourront.
\VS{36}Les hommes donc que Moïse avait envoyés pour épier le pays, et qui étant de retour avaient fait murmurer contre lui toute l'assemblée, en diffamant le pays~;
\VS{37}ces hommes-là, qui avaient décrié le pays, moururent frappés d'une plaie devant Yahweh.
\VS{38}Mais Josué, fils de Nun, et Caleb, fils de Jephunné, restèrent seuls vivants parmi ceux qui étaient allés pour explorer le pays.
\TextTitle{Israël battu par les Amalécites et les Cananéens}
\VS{39}Or Moïse dit ces choses à tous les enfants d'Israël, et le peuple fut dans un grand deuil.
\VS{40}Puis ils se levèrent de bon matin et montèrent au sommet de la montagne, en disant~: Nous voici, et nous monterons au lieu dont Yahweh a parlé car nous avons péché.
\VS{41}Mais Moïse leur dit~: Pourquoi transgressez-vous le commandement de Yahweh~? Cela ne réussira point.
\VS{42}Ne montez pas~; car Yahweh n'est pas au milieu de vous~; afin que vous ne soyez pas battus devant vos ennemis\FTNT{De. 1:41-42.}.
\VS{43}Car les Amalécites et les Cananéens sont là devant vous, et vous tomberez par l'épée~; parce que vous vous êtes détournés de Yahweh, Yahweh ne sera point avec vous.
\VS{44}Toutefois ils s'obstinèrent à monter au sommet de la montagne~; mais l'arche de l'alliance de Yahweh et Moïse ne sortirent point du milieu du camp.
\VS{45}Alors les Amalécites et les Cananéens qui habitaient sur cette montagne descendirent, les battirent, et les taillèrent en pièces jusqu'à Horma.
\Chap{15}
\TextTitle{Consignes pour le pays de Canaan}
\VerseOne{}Puis Yahweh parla à Moïse, en disant~:
\VS{2}Parle aux enfants d'Israël, et dis-leur~: Quand vous serez entrés au pays que je vous donne, où vous devez demeurer,
\VS{3}et que vous voudrez faire un sacrifice consumé par le feu à Yahweh, un holocauste, ou un sacrifice en accompagnement d'un vœu, ou en offrande volontaire, ou bien dans vos fêtes, pour produire avec votre gros ou votre menu bétail une agréable odeur à Yahweh\FTNT{Ex. 29:18~; Lé. 22:21.},
\VS{4}celui qui offrira son offrande à Yahweh présentera en offrande un dixième de fleur de farine, pétrie dans un quart de hin d'huile\FTNT{Lé. 2:1-2.},
\VS{5}et un quart de hin de vin pour la libation que tu feras sur l'holocauste, ou sur un autre sacrifice pour chaque agneau.
\VS{6}Si c'est pour un bélier, tu feras en offrande deux dixièmes de fleur de farine, pétrie dans un tiers de hin d'huile,
\VS{7}et un tiers de hin de vin pour la libation, comme offrande d'une bonne odeur à Yahweh.
\VS{8}Et si tu sacrifies un veau, soit comme holocauste, soit comme sacrifice en accompagnement d'un vœu, ou comme sacrifice d'offrande de paix à Yahweh,
\VS{9}on présentera en offrande avec le veau trois dixièmes de fleur de farine, pétrie dans un demi-hin d'huile.
\VS{10}Et tu offriras la moitié d'un hin de vin pour la libation, en offrande consumée par le feu d'une bonne odeur à Yahweh.
\VS{11}On fera de même pour chaque bœuf, chaque bélier, et chaque petit des brebis ou des chèvres.
\VS{12}Selon le nombre que vous en sacrifierez, vous ferez ainsi à chacun, d'après leur nombre.
\VS{13}Tous ceux qui sont nés au pays feront ces choses de cette manière, en offrant un sacrifice consumé par le feu, d'une bonne odeur à Yahweh.
\TextTitle{Loi sur l'étranger vivant au milieu d'Israël}
\VS{14}Si un étranger séjournant chez vous, ou se trouvant au milieu de vous en vos générations, offre un sacrifice consumé par le feu d'une bonne odeur à Yahweh, il l'offrira de la même manière que vous.
\VS{15}Ô assemblée~! Il y aura une même ordonnance pour vous et pour l'étranger qui fait son séjour parmi vous, il y aura une même ordonnance perpétuelle en vos âges~; il en sera de l'étranger comme de vous en la présence de Yahweh.
\VS{16}Il y aura une même loi et une seule ordonnance pour vous et pour l'étranger qui séjourne au milieu de vous.
\TextTitle{Lois diverses}
\VS{17}Yahweh parla à Moïse, en disant~:
\VS{18}Parle aux enfants d'Israël, et dis-leur~: Quand vous serez arrivés dans le pays où je vous ferai entrer,
\VS{19}et que vous mangerez du pain de ce pays, vous en offrirez à Yahweh une offrande élevée.
\VS{20}Vous offrirez en offrande élevée un gâteau, les prémices de votre pâte~; vous l'offrirez comme ce qu'on prélève de l'aire.
\VS{21}Vous donnerez pour Yahweh une offrande des prémices de votre pâte, dans les temps à venir.
\VS{22}Et lorsque vous aurez péché involontairement\FTNT{Voir commentaire en Lé. 4:2.}, et que vous n'aurez pas fait tous ces commandements que Yahweh a fait connaître à Moïse,
\VS{23}tout ce que Yahweh vous a commandé par Moïse, depuis le jour où Yahweh a commencé de donner ses commandements, et dans la suite dans vos générations,
\VS{24}s'il arrive que la chose ait été faite involontairement, sans que l'assemblée s'en soit aperçue, toute l'assemblée sacrifiera un jeune taureau en holocauste d'une bonne odeur à Yahweh, avec l'offrande et la libation, d'après les règles établies~; elle offrira encore un jeune bouc en sacrifice pour l'expiation.
\VS{25}Ainsi le prêtre fera propitiation pour toute l'assemblée des enfants d'Israël, et il leur sera pardonné parce que c'est une chose arrivée involontairement, et ils ont apporté leur offrande, un sacrifice consumé par le feu à Yahweh et l'offrande pour l'expiation devant Yahweh, à cause de leur péché involontaire.
\VS{26}Alors il sera pardonné à toute l'assemblée des enfants d'Israël, et à l'étranger qui séjourne au milieu d'eux, car c'est involontairement que tout le peuple a péché.
\VS{27}Si c'est une seule personne qui a péché involontairement, elle offrira une chèvre d'un an en offrande pour le péché\FTNT{Lé. 4:27-28.}.
\VS{28}Et le prêtre fera propitiation pour la personne qui aura péché involontairement, de ce qu'elle aura péché involontairement devant Yahweh, et faisant propitiation pour elle, il lui sera pardonné.
\VS{29}Il y aura une même loi pour celui qui aura fait quelque chose involontairement, tant pour celui qui est né au pays des enfants d'Israël, que pour l'étranger qui fait son séjour parmi eux.
\VS{30}Mais quant à celui qui aura péché par fierté, tant celui qui est né au pays, que l'étranger, il a outragé Yahweh, cette personne-là sera retranchée du milieu de son peuple.
\VS{31}Parce qu'il a méprisé la parole de Yahweh, et qu'il a enfreint son commandement. Cette personne donc sera certainement retranchée~; son iniquité est sur elle.
\TextTitle{Un homme lapidé selon la loi\FTNTT{Ro. 3:19~; 7:7-11~; 2 Co. 3:7-9~; Ga. 3:10.}}
\VS{32}Or comme les enfants d'Israël étaient dans le désert, on trouva un homme qui ramassait du bois le jour du sabbat.
\VS{33}Et ceux qui l'avaient trouvé ramassant du bois, l'amenèrent à Moïse, à Aaron, et à toute l'assemblée.
\VS{34}Et on le mit sous garde, car ce qu'on devait lui faire n'avait pas été déclaré.
\VS{35}Alors Yahweh dit à Moïse~: On punira de mort cet homme, et toute l'assemblée le lapidera hors du camp.
\VS{36}Toute l'assemblée donc le mena hors du camp et le lapida, et il mourut, comme Yahweh l'avait ordonné à Moïse.
\VS{37}Et Yahweh parla à Moïse, en disant~:
\VS{38}Parle aux enfants d'Israël, et dis-leur~: Qu'ils se fassent de génération en génération des franges aux bords de leurs vêtements, et qu'ils mettent sur les franges au bords de leurs vêtements un cordon de couleur pourpre\FTNT{De. 22:12~; Mt. 23:5.}.
\VS{39}Quand vous aurez cette frange, vous la regarderez et vous vous souviendrez de tous les commandements de Yahweh, pour les mettre en pratique, et vous ne suivrez pas les désirs de vos cœurs et de vos yeux, pour vous laisser entraîner à la prostitution.
\VS{40}Afin que vous vous souveniez de tous mes commandements, et que vous les fassiez, et que vous soyez saints à votre Dieu.
\VS{41}Je suis Yahweh, votre Dieu, qui vous ai retiré du pays d'Egypte, pour être votre Dieu. Je suis Yahweh, votre Dieu.
\Chap{16}
\TextTitle{La révolte de Koré\FTNTT{Jud. 11.}}
\VerseOne{}Or Koré\FTNT{Koré, Dathan et Abiram, s'étaient révoltés contre Aaron et Moïse, car ils voulaient s'attribuer l'honneur d'offrir à Dieu des sacrifices. Ils voulaient exercer la prêtrise (sacerdoce) alors que Yahweh ne les avait pas établis pour le service du culte. Vouloir servir Dieu sans avoir reçu un appel divin est dangereux.}, fils de Jitsehar, fils de Kehath, fils de Lévi, se révolta avec Dathan et Abiram, fils d'Eliab, et On, fils de Péleth, tous trois fils de Ruben.
\VS{2}Et ils s'élevèrent contre Moïse, avec deux cent cinquante hommes des fils d'Israël, qui étaient des principaux de l'assemblée, de ceux que l'on convoquait pour tenir le conseil, et qui étaient des gens de renom.
\VS{3}Et ils s'assemblèrent contre Moïse et contre Aaron, et leur dirent~: C'en est assez~! Puisque tous ceux de l'assemblée sont saints, et que Yahweh est au milieu d'eux, pourquoi vous élevez-vous au-dessus de l'assemblée de Yahweh~?
\VS{4}Quand Moïse eut entendu cela, il se jeta sur son visage.
\VS{5}Et il parla à Koré et à tous ceux qui étaient assemblés avec lui, et leur dit~: Demain au matin, Yahweh fera connaître celui qui lui appartient, et celui qui est saint, et il le fera approcher de lui~; il fera, dis-je, approcher de lui celui qu'il aura choisi.
\VS{6}Faites ceci, prenez des encensoirs, Koré et toute son assemblée.
\VS{7}Et demain, mettez-y du feu, et mettez-y du parfum devant Yahweh~; et celui que Yahweh choisira, c'est celui-là qui sera saint. C'en est assez, fils de Lévi~!
\VS{8}Moïse dit aussi à Koré~: Ecoutez maintenant, fils de Lévi~:
\VS{9}Est-ce trop peu de chose pour vous, que le Dieu d'Israël vous ait séparés de l'assemblée d'Israël, pour vous faire approcher de lui, afin de faire le service du tabernacle de Yahweh, et pour vous tenir devant l'assemblée, afin de la servir~?
\VS{10}Et qu'il t'ait fait approcher de lui, toi et tous tes frères, les fils de Lévi, et vous recherchez encore la prêtrise~!
\VS{11}C'est pourquoi toi et toute ton assemblée, vous vous êtes rassemblés contre Yahweh~! Car qui est Aaron pour que vous murmuriez contre lui~?
\VS{12}Et Moïse envoya appeler Dathan et Abiram, fils d'Eliab, qui répondirent~: Nous n'y monterons point.
\VS{13}Est-ce peu de chose que tu nous aies fait monter hors d'un pays où coulent le lait et le miel, pour nous faire mourir dans le désert, que tu veuilles aussi dominer sur nous~?
\VS{14}Certes, tu ne nous as pas fait venir dans un pays où coulent le lait et le miel~! Et tu ne nous as pas donné un héritage de champs ni de vignes~! Veux-tu crever les yeux de ces gens~? Nous ne monterons pas.
\VS{15}Alors Moïse fut très irrité, et il dit à Yahweh~: N'aie point égard à leur offrande. Je n'ai point pris d'eux un seul âne, et je n'ai fait de mal à aucun d'eux.
\VS{16}Puis Moïse dit à Koré~: Toi et tous ceux qui sont assemblés avec toi, trouvez-vous demain devant Yahweh, toi et eux avec Aaron.
\VS{17}Et prenez chacun vos encensoirs, et mettez-y du parfum~; et que chacun présente devant Yahweh son encensoir~: Il y aura deux cent cinquante encensoirs~; toi et Aaron aussi, chacun avec son encensoir.
\VS{18}Ils prirent donc chacun son encensoir, et y mirent du feu, et ensuite y posèrent du parfum, et ils se tinrent à l'entrée de la tente d'assignation, avec Moïse et Aaron.
\VS{19}Et Koré fit assembler contre eux toute l'assemblée à l'entrée de la tente d'assignation~; et la gloire de Yahweh apparut à toute l'assemblée.
\VS{20}Puis Yahweh parla à Moïse et à Aaron, en disant~:
\VS{21}Séparez-vous du milieu de cette assemblée, et je les consumerai en un seul instant\FTNT{Ex. 32:10.}.
\VS{22}Mais ils tombèrent sur leur visage et dirent~: Ô Dieu~! Dieu des esprits de toute chair~! Un seul homme a péché, et tu te mettrais en colère contre toute l'assemblée\FTNT{Hé. 12:9.}~?
\VS{23}Et Yahweh parla à Moïse, en disant~:
\VS{24}Parle à l'assemblée, et dis lui~: Retirez-vous d'auprès de la demeure de Koré, de Dathan, et d'Abiram.
\VS{25}Moïse donc se leva, et alla vers Dathan et Abiram~; et les anciens d'Israël le suivirent.
\VS{26}Et il parla à l'assemblée, en disant~: Eloignez-vous, je vous prie, d'auprès des tentes de ces méchants hommes, et ne touchez à rien qui leur appartienne, de peur que vous ne périssiez punis pour tous leurs péchés.
\VS{27}Ils se retirèrent donc d'auprès des demeures de Koré, de Dathan et d'Abiram. Et Dathan et Abiram sortirent et se tinrent debout à l'entrée de leurs tentes, avec leurs femmes, leurs fils, et leurs petits-enfants.
\VS{28}Et Moïse dit~: Vous connaîtrez à ceci que Yahweh m'a envoyé pour faire toutes ces choses, et que je n'agis pas de moi-même.
\VS{29}Si ces gens meurent comme tous les hommes meurent, et s'ils subissent le sort commun à tous les hommes, Yahweh ne m'a point envoyé~;
\VS{30}mais si Yahweh fait une chose nouvelle, et si la terre ouvre sa bouche pour les engloutir avec tout ce qui leur appartient, et qu'ils descendent vivants dans le scheol, vous saurez alors que ces hommes-là ont irrité par mépris Yahweh.
\VS{31}Et il arriva qu'aussitôt qu'il eut achevé de dire toutes ces paroles, la terre qui était sous eux se fendit.
\VS{32}Et la terre ouvrit sa bouche et les engloutit, avec leurs tentes et tous les hommes qui étaient à Koré, et tous leurs biens\FTNT{De. 11:6~; Ps. 106:17.}.
\VS{33}Ils descendirent donc vivants dans le scheol, eux et tout ceux qui leur appartenait~; la terre les recouvrit, et ils disparurent au milieu de l'assemblée.
\VS{34}Et tout Israël qui était autour d'eux s'enfuit à leurs cris~; car ils disaient~: Prenons garde que la terre ne nous engloutisse~!
\VS{35}Un feu sortit de part Yahweh et consuma les deux cent cinquante hommes qui offraient le parfum.
\VS{36}Puis Yahweh parla à Moïse, en disant~:
\VS{37}Dis à Eléazar, fils d'Aaron, le prêtre, qu'il ramasse les encensoirs du milieu de l'embrasement, et d'en répandre au loin le feu, car ils sont sanctifiés.
\VS{38}Avec les encensoirs de ceux qui ont péché contre leurs âmes, que l'on fasse des lames étendues dont on couvrira l'autel. Puisqu'ils ont été offerts devant Yahweh et qu'ils sont sanctifiés, ils serviront de signe aux enfants d'Israël.
\VS{39}Ainsi Eléazar, le prêtre, prit les encensoirs d'airain, que ces hommes qui furent brûlés avaient présentés, et on en fit des lames pour couvrir l'autel.
\VS{40}C'est un souvenir pour les enfants d'Israël, afin qu'aucun étranger qui n'est pas de la race d'Aaron, ne s'approche pour offrir du parfum devant Yahweh, et ne soit comme Koré, et comme ceux qui ont été assemblés avec lui~; selon ce que Yahweh avait déclaré par Moïse.
\TextTitle{Le peuple frappé à cause des murmures}
\VS{41}Or dès le lendemain, toute l'assemblée des enfants d'Israël murmura contre Moïse et contre Aaron, en disant~: Vous avez fait mourir le peuple de Yahweh.
\VS{42}Et il arriva comme l'assemblée s'amassait contre Moïse et contre Aaron, et comme ils tournaient les regards vers la tente d'assignation, voici la nuée la couvrit, et la gloire de Yahweh apparut.
\VS{43}Moïse donc et Aaron vinrent donc devant la tente d'assignation.
\VS{44}Et Yahweh parla à Moïse, en disant~:
\VS{45}Retirez-vous du milieu de cette assemblée, et je les consumerai en un instant. Alors ils se prosternèrent le visage contre terre~;
\VS{46}puis Moïse dit à Aaron~: Prends l'encensoir, et mets-y du feu de dessus l'autel, mets-y aussi du parfum, et va promptement à l'assemblée, et fais propitiation pour eux~; car une grande colère est sortie de devant Yahweh, la plaie a commencé.
\VS{47}Et Aaron prit l'encensoir, comme Moïse lui avait dit, et il courut au milieu de l'assemblée, et voici la plaie avait déjà commencé sur le peuple. Alors il mit du parfum et fit propitiation pour le peuple.
\VS{48}Et comme il se tenait entre les morts et les vivants, la plaie fut arrêtée.
\VS{49}Et il y en eut quatorze mille sept cents qui moururent de cette plaie, outre ceux qui étaient morts à cause de Koré.
\VS{50}Et Aaron retourna auprès de Moïse, à l'entrée de la tente d'assignation, et la plaie s'arrêta.
\Chap{17}
\TextTitle{Yahweh confirme l'appel d'Aaron, sa verge fleurit}
\VerseOne{}Après cela Yahweh parla à Moïse, en disant~:
\VS{2}Parle aux enfants d'Israël, et prends une verge de chacun d'eux selon la maison de leur père, de tous ceux qui sont les princes, selon la maison de leurs pères, douze verges, puis tu écriras le nom de chacun sur sa verge,
\VS{3}mais tu écriras le nom d'Aaron sur la verge de Lévi\FTNT{La verge d'Aaron est une image du Messie ressuscité. Elle avait produit la vie tandis que celles des autres princes n'avaient produit aucun fruit. Cette histoire nous parle également de la confirmation de l'appel d'Aaron face aux critiques dont il était l'objet. On reconnaît l'arbre par ses fruits (Mt. 7:16-20~; Lu. 7:17-22).}~; car il y aura une verge pour chaque chef des maisons de leurs pères.
\VS{4}Et tu les déposeras dans la tente d'assignation, devant le témoignage, où je me rencontre avec vous.
\VS{5}Et la verge de l'homme que j'aurai choisi fleurira~; et je ferai cesser de devant moi les murmures des enfants d'Israël, par lesquels ils murmurent contre vous.
\VS{6}Quand Moïse parla aux enfants d'Israël, tous leurs princes lui donnèrent une verge, chaque prince une verge, selon les maisons de leurs pères, soit douze verges~; or la verge d'Aaron était au milieu des leurs.
\VS{7}Et Moïse mit les verges devant Yahweh, dans la tente du témoignage.
\VS{8}Et le lendemain, lorsque Moïse entra dans la tente du témoignage, voici, la verge d'Aaron, avait fleuri, pour la maison de Lévi, et elle avait poussé des boutons, produit des fleurs et mûri des amandes.
\VS{9}Alors Moïse ôta de devant Yahweh toutes les verges et les porta à tous les fils d'Israël, afin qu'ils les voient et qu'ils prennent chacun leurs verges.
\VS{10}Et Yahweh dit à Moïse~: Reporte la verge d'Aaron devant le témoignage, pour être conservée comme un signe pour les fils de rébellion, afin que tu fasses cesser de devant moi leurs murmures et qu'ils ne meurent point\FTNT{Hé. 9:3-5.}.
\VS{11}Et Moïse fit ainsi~; il se conforma à l'ordre que Yahweh lui avait donné.
\VS{12}Les enfants d'Israël parlèrent à Moïse, en disant~: Voici, nous expirons, nous périssons, nous périssons tous~!
\VS{13}Quiconque s'approche du tabernacle de Yahweh, meurt. Serons-nous tous entièrement expirés~?
\Chap{18}
\TextTitle{Droits et devoirs des prêtres et des Lévites}
\VerseOne{}Alors Yahweh dit à Aaron~: Toi et tes fils, et la maison de ton père avec toi, vous porterez l'iniquité du sanctuaire~; et toi, et tes fils avec toi, vous porterez l'iniquité de votre prêtrise.
\VS{2}Fais aussi approcher de toi tes frères, la tribu de Lévi, qui est la tribu de ton père, afin qu'ils te soient attachés et qu'ils te servent, mais toi et tes fils avec toi, vous servirez devant la tente du témoignage.
\VS{3}Ils garderont ce que tu leur ordonneras de garder, et ce qu'il faut garder de toute la tente, mais ils n'approcheront point des ustensiles du sanctuaire, ni de l'autel de peur qu'ils ne meurent, et que vous ne mouriez avec eux.
\VS{4}Ils te seront donc attachés, et ils garderont tout ce qu'il faut garder dans la tente d'assignation, selon tout le service du tabernacle et aucun étranger n'approchera de vous.
\VS{5}Mais vous prendrez garde à ce qu'il faut faire dans le sanctuaire, et à ce qu'il faut faire à l'autel, afin qu'il n'y ait plus d'indignation sur les enfants d'Israël.
\VS{6}Car quant à moi voici, j'ai pris vos frères, les Lévites, du milieu des enfants d'Israël, qui sont donnés en pur don pour Yahweh, afin qu'ils soient employés au service de la tente d'assignation.
\VS{7}Mais toi et tes fils avec toi, vous observerez la fonction de votre prêtrise en tout ce qui concerne l'autel et ce qui est au dedans du voile, et vous y ferez le service. J'établis votre prêtrise en office de pur don~; c'est pourquoi si un étranger en approche, on le fera mourir.
\VS{8}Yahweh dit encore à Aaron~: Voici, je t'ai donné la garde de mes offrandes élevées sur toutes les choses consacrées par les enfants d'Israël~; je te les ai données, et à tes enfants, par ordonnance perpétuelle, à cause de l'onction.
\VS{9}Ceci t'appartiendra d'entre les choses très saintes qui ne sont pas brûlées, savoir toutes leurs offrandes, soit de tous leurs gâteaux, soit de tous leurs sacrifices pour l'expiation, et tous leurs sacrifices pour la culpabilité qu'ils m'apporteront~; ce sont des choses très saintes pour toi et pour tes enfants.
\VS{10}Vous les mangerez dans un lieu très saint~; tout mâle en mangera~; vous les regarderez comme saintes\FTNT{Lé. 6:17-22~; Lé. 7:6~; Lé. 10:13.}.
\VS{11}Voici encore ce qui t'appartiendra~: Tous les dons que les enfants d'Israël présenteront par élévation et en les agitant de côté et d'autre, je te les donne à toi, à tes fils, et à tes filles avec toi, par une loi perpétuelle~; quiconque sera pur dans ta maison en mangera\FTNT{Lé. 7:34~; Lé. 10:14.}.
\VS{12}Je te donne aussi leurs prémices qu'ils offriront à Yahweh~: Tout ce qu'il y aura de meilleur en huile, et tout le meilleur du moût et du blé.
\VS{13}Les premiers fruits de toutes les choses que leur terre produira, et qu'ils apporteront à Yahweh t'appartiendront~; quiconque sera pur dans ta maison, en mangera.
\VS{14}Tout ce qui sera dévoué en Israël t'appartiendra\FTNT{Lé. 27:28~; Ez. 44:29.}.
\VS{15}Tout premier-né de toute chair, qu'ils offriront à Yahweh, tant des hommes que des animaux t'appartiendront. Mais, tu feras racheter le premier-né de l'homme, et tu feras racheter le premier-né d'un animal impur.
\VS{16}Et ceux qui doivent être rachetés, depuis l'âge d'un mois, tu les rachèteras selon ton estimation que tu en feras, au prix de cinq sicles d'argent, selon le sicle du sanctuaire, qui est de vingt guéras.
\VS{17}Mais tu ne feras point racheter le premier-né du bœuf, ni le premier-né de la brebis, ni le premier-né de la chèvre~: Ce sont des choses saintes. Tu répandras leur sang sur l'autel, et tu brûleras leur graisse~: Ce sera un sacrifice consumé par le feu d'une bonne odeur à Yahweh.
\VS{18}Mais leur chair t'appartiendra, comme la poitrine qu'on agite de côté et d'autre, et comme l'épaule droite.
\VS{19}Je t'ai donné, à toi et à tes fils, et à tes filles avec toi, par une loi perpétuelle, toutes les offrandes présentées par élévation des choses sanctifiées, que les enfants d'Israël offriront à Yahweh. C'est une alliance de sel\FTNT{Le sel est un aliment pratiquement impérissable et incorruptible. Dans l'Antiquité, il symbolisait l'incorruptibilité (Lé. 2:13).} et à perpétuité devant Yahweh, pour toi et pour ta postérité avec toi.
\VS{20}Puis Yahweh dit à Aaron~: Tu ne posséderas rien dans leur pays, et il n'y aura point de part pour toi au milieu d'eux~; c'est moi qui suis ta part et ta possession, au milieu des enfants d'Israël\FTNT{De. 10:9~; De. 18:2~; Ez. 44:28.}.
\TextTitle{Lois sur les dîmes (De. 14:22-29)}
\VS{21}Et je donne comme possession aux fils de Lévi, toutes les dîmes\FTNT{Il y avait plusieurs sortes de dîmes dans la loi Mosaïque~:
\\- La 1ère dîme~: Le peuple devait payer une dîme générale au bénéfice des Lévites (No. 18:21).
\\Toutes les tribus d'Israël, à l'exception des Lévites, eurent une possession géographique qu'ils reçurent comme héritage après leur entrée en Canaan. Mais les Lévites devaient accomplir une tâche particulière au sein de la nation. Ils devaient s'occuper du service dans la tente d'assignation. En compensation de ce service, ils devaient percevoir un impôt de 10\% des revenus de tous les Israélites.
\\- La 2ème dîme~: Les Lévites devaient payer la «~dîme de la dîme~», au bénéfice des prêtres (No. 18:25-31).
\\Tous les prêtres étaient des Lévites, mais tous les Lévites n'étaient pas des prêtres. Les prêtres descendaient d'Aaron et ils exerçaient des responsabilités particulières dans la tente d'assignation, puis dans le temple. Cette seconde dîme permettait aux prêtres d'être nourris et assurait donc le bon fonctionnement du service du temple.
\\- La 3ème dîme~: Tous les Israélites devaient conserver une dîme de toute leur production en prévision de leurs pèlerinages annuels à Jérusalem (De. 14:22-26).
\\Trois fois par an, tout le peuple devait s'assembler à Jérusalem, l'endroit choisi par le Seigneur, à l'occasion des principales fêtes. Dieu avait prévu que chacun puisse disposer de ressources suffisantes pour leur permettre de se réjouir pleinement à ces occasions. C'est pour cela qu'ils devaient mettre de côté 10\% de leurs productions agricoles annuelles. Il est intéressant de noter que la dîme n'était jamais payée en argent, mais toujours en nature.
\\- La 4ème dîme~: Il fallait payer une dîme spéciale à l'intention des pauvres, des orphelins et des veuves (De. 14:28-29). 
\\Certains affirment que la dîme existait bien avant la loi. Mais ils ignorent que la Bible parle de plusieurs sortes de lois.
\\- Les lois cérémonielles (Hé. 9:1)
\\Ces lois étaient relatives au culte et concernaient le tabernacle puis le temple, les sacrifices, les ablutions (Lé. 16~; Hé. 9:1-10). Les dîmes (la dîme des prêtres) devaient être amenées dans le temple (Mal. 3:10), elles faisaient donc partie des lois cérémonielles. Or les Lévites et les prêtres de la Première Alliance n'existent plus sous la Nouvelle Alliance car les enfants de Dieu sont un royaume de rois et de prêtres (Ap. 1:6~; Ap. 5:10).
\\- Les lois morales (Ex. 20:1-17). Dieu est saint et il veut un peuple saint qui marche dans sa crainte, dans la sainteté et dans l'obéissance. Lé. 18 nous parle des lois morales~; elles n'ont pas été abolies, elles existent toujours. Elles sont inscrites dans la conscience de l'homme, elles sont gravées dans notre cœur (Hé. 8:10).
\\- Les lois sociales (Ex. 21:1-24). Ce sont des lois civiles régissant la vie sociale d'Israël, comme nous pouvons le lire dans Ex. 21 par exemple. Ces lois n'ont rien à voir avec les croyants de la Nouvelle Alliance. Les lois morales témoignent de la nature de Dieu, ce sont des lois éternelles qui existaient bien avant Abraham. Les lois cérémonielles ont commencé dès la fondation du monde (Ap. 13:8) car l'Agneau de Dieu était immolé avant la fondation du monde (1 Pi. 1:19-20). Seules les lois sociales ont débuté avec Moïse car elles concernaient exclusivement les Israélites. Ces trois sortes de lois ont été institutionnalisées par Moïse, mais les deux premières (morales et cérémonielles) existaient avant ce dernier. Les quatre sortes de dîmes faisaient bel et bien partie des lois sociales et cérémonielles. Or ces lois ne sont plus d'actualité sous la Nouvelle Alliance. En conclusion, nous pouvons dire que Jésus nous a rachetés en accomplissant les lois cérémonielles afin que nous pratiquions les lois morales (Ep. 2:10). Voir également commentaire en Mal 3~: 10.} d'Israël, pour le service auquel ils sont employés, le service de la tente d'assignation.
\VS{22}Et les enfants d'Israël n'approcheront plus de la tente d'assignation, afin qu'ils ne se chargent d'un péché et qu'ils ne meurent point.
\VS{23}Mais les Lévites s'emploieront au service de la tente d'assignation, et ils resteront chargés de leurs iniquités. Cette loi sera perpétuelle parmi vos descendants, et ils ne posséderont point d'héritage parmi les enfants d'Israël.
\VS{24}Car je donne comme possession aux Lévites les dîmes que les enfants d'Israël présenteront à Yahweh en offrande élevée~; c'est pourquoi je dis d'eux qu'ils n'auront point d'héritage parmi les fils d'Israël.
\VS{25}Puis Yahweh parla à Moïse, en disant~:
\VS{26}Tu parleras aussi aux Lévites, et tu leur diras~: Quand vous recevrez des enfants d'Israël les dîmes que je vous donne de leur part comme possession, vous en offrirez l'offrande élevée à Yahweh, la dîme de la dîme~;
\VS{27}et votre offrande élevée vous sera comptée comme le blé qu'on prélève de l'aire, et comme l'abondance qu'on prélève de la cuve.
\VS{28}C'est ainsi que vous prélèverez une offrande pour Yahweh de toutes les dîmes que vous recevrez des enfants d'Israël, et vous donnerez au prêtre Aaron l'offrande que vous en aurez prélevée pour Yahweh.
\VS{29}Sur tous les dons qui vous seront faits, vous prélèverez toute l'offrande élevée pour Yahweh~; sur tout ce qu'il y aura de meilleur, vous prélèverez la portion consacrée.
\VS{30}Et tu leur diras~: Quand vous aurez offert en offrande élevée le meilleur de la dîme, pris de la dîme même, il sera imputé aux Lévites comme le revenu de l'aire, et comme le revenu de la cuve.
\VS{31}Et vous la mangerez en tout lieu, vous et votre maison~; car c'est votre salaire pour le service auquel vous êtes employés dans la tente d'assignation.
\VS{32}Vous ne serez point coupables de péché au sujet de la dîme, quand vous en aurez offert en offrande élevée sur ce qu'il y aura de meilleur et vous ne souillerez point les choses saintes des enfants d'Israël et vous ne mourrez point.
\Chap{19}
\TextTitle{La jeune vache rousse~; l'eau de purification}
\VerseOne{}Yahweh parla à Moïse et à Aaron, en disant~:
\VS{2}Voici ce qui est ordonné par la loi que Yahweh a commandé, en disant~: Parle aux enfants d'Israël, et dis-leur qu'ils t'amènent une jeune vache rousse, entière, sans défaut, et qui n'ait point porté le joug.
\VS{3}Puis vous la donnerez à Eléazar, le prêtre, qui la mènera hors du camp, et on l'égorgera en sa présence\FTNT{Lé. 4:12~; Hé. 13:11-12.}.
\VS{4}Ensuite, Eléazar, le prêtre, prendra de son sang avec son doigt, et fera sept fois l'aspersion du sang vers le devant de la tente d'assignation.
\VS{5}Et on brûlera la jeune vache en sa présence~; on brûlera sa peau, sa chair, son sang et ses excréments\FTNT{Ex. 29:14.}.
\VS{6}Le prêtre prendra du bois de cèdre, de l'hysope, et du cramoisi, et les jettera dans le feu où sera brûlée la jeune vache.
\VS{7}Puis le prêtre lavera ses vêtements et son corps avec de l'eau~; après cela, il rentrera au camp, et le prêtre sera impur jusqu'au soir.
\VS{8}Celui qui l'aura brûlé, lavera ses vêtements dans l'eau, il lavera aussi dans l'eau son corps~; et il sera impur jusqu'au soir.
\VS{9}Et un homme pur ramassera les cendres de la jeune vache, et les mettra hors du camp, dans un lieu pur~; elles seront gardées pour l'assemblée des enfants d'Israël~; afin d'en faire l'eau de purification. C'est une purification pour le péché.
\VS{10}Celui qui aura ramassé les cendres de la jeune vache, lavera ses vêtements, et sera impur jusqu'au soir~; ce sera une loi perpétuelle pour les enfants d'Israël, et pour l'étranger en séjour au milieu d'eux.
\VS{11}Celui qui touchera un mort, un corps humain quel qu'il soit, sera impur pendant sept jours\FTNT{Ag. 2:13.}.
\VS{12}Il se purifiera avec cette eau le troisième jour et le septième jour, et il sera pur~; mais s'il ne se purifie pas le troisième jour, il ne sera pas pur le septième jour.
\VS{13}Alors celui qui touchera un mort, le corps d'un homme qui sera mort et qui ne se purifiera pas, souille le tabernacle de Yahweh~; celui-là sera retranché d'Israël. Il est impur, car l'eau de purification n'a pas été répandue sur lui, son impureté demeure encore sur lui.
\VS{14}Voici la loi. Lorsqu'un homme mourra dans une tente, quiconque entrera dans la tente, et quiconque se trouvera dans la tente sera impur pendant sept jours.
\VS{15}Aussi tout vase découvert, sur lequel il n'y aura point de couvercle attaché, sera impur.
\VS{16}Et quiconque touchera, dans les champs, un homme qui aura été tué par l'épée, ou un mort, ou des ossements humains, ou un sépulcre, sera impur durant sept jours.
\VS{17}Et on prendra, pour celui qui est impur, de la poudre de la jeune vache brûlée pour faire la purification, et on la mettra dans un vase, avec de l'eau vive par-dessus.
\VS{18}Puis un homme pur prendra de l'hysope, et la trempera dans l'eau~; il en fera aspersion sur la tente, et sur tous les ustensiles, et sur toutes les personnes qui auront été là, et sur celui qui a touché des ossements ou un homme tué, ou un mort, ou un sépulcre.
\VS{19}Celui qui est pur fera l'aspersion sur celui qui est impur, le troisième jour et le septième jour, et il le purifiera le septième jour~; puis il lavera ses vêtements, et se lavera dans l'eau, et il sera pur le soir.
\VS{20}Mais l'homme qui sera impur, et qui ne se purifiera point, sera retranché du milieu de l'assemblée, parce qu'il a souillé le sanctuaire de Yahweh~; comme l'eau de purification n'a pas été répandue sur lui, il est impur.
\VS{21}Et ce sera pour eux une loi perpétuelle, et celui qui fera l'aspersion de l'eau de purification lavera ses vêtements~; et quiconque touchera l'eau de purification sera impur jusqu'au soir.
\VS{22}Et tout ce que l'homme impur touchera sera souillé, et la personne qui le touchera sera impure jusqu'au soir.
\Chap{20}
\TextTitle{Mort de Marie}
\VerseOne{}Or toute l'assemblée des enfants d'Israël arriva dans le désert de Tsin au premier mois, et le peuple s'arrêta à Kadès. Marie mourut là, et y fut ensevelie.
\TextTitle{Murmures du peuple à cause du manque d'eau\FTNTT{De. 32:51~; cp. Ex. 17:1-7.}}
\VS{2}Et il n'y avait point d'eau pour l'assemblée~; et ils se soulevèrent contre Moïse et contre Aaron.
\VS{3}Et le peuple contesta contre Moïse et ils lui dirent~: Pourquoi ne sommes-nous pas morts quand nos frères moururent devant Yahweh~?
\VS{4}Et pourquoi avez-vous fait venir l'assemblée de Yahweh dans ce désert, pour que nous y mourions, nous et notre bétail\FTNT{Ex. 17:3.}~?
\VS{5}Et pourquoi nous avez-vous fait monter hors d'Egypte, pour nous amener dans ce méchant lieu qui n'est pas un lieu où l'on puisse semer, ni un lieu pour des figuiers, ni pour des vignes, ni pour des grenadiers, et sans eau pour boire~?
\VS{6}Alors Moïse et Aaron se retirèrent de devant l'assemblée à l'entrée de la tente d'assignation et ils tombèrent sur leurs faces~; et la gloire de Yahweh apparut.
\TextTitle{Incrédulité de Moïse et d'Aaron à Meriba}
\VS{7}Yahweh parla à Moïse, en disant~:
\VS{8}Prends la verge, et convoque l'assemblée, toi et Aaron, ton frère. Vous parlerez en leur présence au rocher\FTNT{Christ, le rocher des âges ( Es. 8:13-17~; 1 Co. 10:1-4).}, et il donnera son eau~; ainsi tu leur feras sortir de l'eau du rocher, et tu donneras à boire à l'assemblée et à leur bétail.
\VS{9}Moïse prit la verge qui était devant Yahweh, comme il lui avait ordonné.
\VS{10}Moïse et Aaron convoquèrent l'assemblée devant le rocher. Et il leur dit~: Ecoutez donc, rebelles~! Est-ce de ce rocher que nous vous ferons sortir de l'eau~?
\VS{11}Puis Moïse leva sa main, et frappa deux fois le rocher avec sa verge et il en sortit des eaux en abondance. L'assemblée but, et leur bétail aussi.
\VS{12}Alors Yahweh dit à Moïse et à Aaron~: Parce que vous n'avez pas cru en moi, pour me sanctifier aux yeux des enfants d'Israël, ainsi vous ne ferez point entrer cette assemblée dans le pays que je lui donne.
\VS{13}Ce sont là, les eaux de Meriba, où les enfants d'Israël contestèrent avec Yahweh, qui fut sanctifié en eux.
\TextTitle{La méchanceté d'Edom\FTNTT{Ge. 25:30~; Ab. 10.}}
\VS{14}Puis Moïse envoya des ambassadeurs de Kadès au roi d'Edom, pour lui dire~: Ainsi parle ton frère Israël~: Tu sais toutes les souffrances que nous avons eu.
\VS{15}Comment nos pères descendirent en Egypte, où nous avons demeuré longtemps~; et comment les Egyptiens nous ont maltraités, nous et nos pères.
\VS{16}Et nous avons crié à Yahweh, et il a entendu nos cris. Il a envoyé l'Ange et nous a retirés d'Egypte. Et voici, nous sommes à Kadès, ville qui est à l'extrémité de ton territoire\FTNT{Ex. 2:23~; Ex. 23:20~; Ac. 7:30-38.}.
\VS{17}Je te prie, laisse-nous passer par ton pays~; nous ne traverserons ni les champs ni les vignes, et nous ne boirons l'eau d'aucun puits~; nous marcherons par le chemin royal~; nous ne nous détournerons ni à droite ni à gauche, jusqu'à ce que nous ayons passé ton territoire.
\VS{18}Et Edom lui dit~: Tu ne passeras point par mon pays, de peur que je ne sorte en armes à ta rencontre.
\VS{19}Les enfants d'Israël lui répondirent~: Nous monterons par le grand chemin, et si nous buvons de tes eaux, moi et mes bêtes, je t'en payerai le prix~; je veux seulement passer à pied.
\VS{20}Mais il lui répondit~: Tu ne passeras pas~! Et sur cela, Edom sortit à sa rencontre avec une grande multitude, et à main armée.
\VS{21}Ainsi Edom ne voulut point permettre à Israël de passer par ses frontières~; c'est pourquoi Israël se détourna de lui.
\VS{22}Et toute l'assemblée des enfants d'Israël partit de Kadès et arriva à la montagne de Hor.
\TextTitle{Mort d'Aaron}
\VS{23}Et Yahweh parla à Moïse et à Aaron à la montagne de Hor, près des frontières du pays d'Edom, en disant~:
\VS{24}Aaron sera recueilli auprès de son peuple, car il n'entrera pas dans le pays que je donne aux enfants d'Israël, parce que vous avez été rebelles à mon commandement aux eaux de la dispute\FTNT{«~Meriba~»}.
\VS{25}Prends donc Aaron et Eléazar, son fils, et fais-les monter sur la montagne de Hor.
\VS{26}Puis fais dépouiller Aaron de ses vêtements, et fais-les revêtir à Eléazar, son fils. C'est là qu'Aaron sera recueilli et qu'il mourra.
\VS{27}Moïse fit ce que Yahweh avait ordonné~; et ils montèrent sur la montagne de Hor, aux yeux de toute l'assemblée.
\VS{28}Et Moïse dépouilla Aaron de ses vêtements et en fit revêtir Eléazar, son fils. Aaron mourut là, au sommet de la montagne. Moïse et Eléazar descendirent de la montagne\FTNT{De. 10:6.}.
\VS{29}Toute l'assemblée, toute la maison d'Israël, voyant qu'Aaron était mort, le pleurèrent trente jours.
\Chap{21}
\TextTitle{Les Cananéens livrés à Israël}
\VerseOne{}Quand le roi d'Arad, Cananéen, qui habitait le midi, eut appris qu'Israël venait par le chemin d'Atharim, il combattit Israël et emmena des prisonniers.
\VS{2}Alors Israël fit un vœu à Yahweh, en disant~: Si tu livres ce peuple entre mes mains, je dévouerai ses villes par le moyen de l'interdit.
\VS{3}Et Yahweh exauça la voix d'Israël et livra entre ses mains les Cananéens. On les dévoua par interdit, avec leurs villes~; et on donna à ce lieu le nom de Horma.
\TextTitle{Le serpent d'airain\FTNTT{Jn. 3:14-15~; 2 Co. 5:20.}}
\VS{4}Puis ils partirent de la montagne de Hor, par le chemin de la Mer Rouge, pour faire le tour du pays d'Edom. Le cœur du peuple s'impatienta en route,
\VS{5}et parla contre Dieu, et contre Moïse, en disant~: Pourquoi nous as-tu fait monter hors d'Egypte, pour mourir dans ce désert~? Car il n'y a point de pain ni d'eau, et notre âme est dégoûtée de cette nourriture misérable.
\VS{6}Et Yahweh envoya contre le peuple des serpents brûlants qui mordaient le peuple~; tellement qu'il en mourut un grand nombre en Israël\FTNT{1 Co. 10:9.}.
\VS{7}Alors le peuple vint vers Moïse, et dit~: Nous avons péché, car nous avons parlé contre Yahweh et contre toi. Invoque Yahweh afin qu'il éloigne de nous les serpents et Moïse pria pour le peuple.
\VS{8}Et Yahweh dit à Moïse~: Fais-toi un serpent brûlant, et mets-le sur une perche~; quiconque aura été mordu et le regardera conservera la vie.
\VS{9}Moïse fit un serpent d'airain\FTNT{Voir Jn. 3:14-16. Ceux qui regardent à Jésus-Christ, et non aux hommes, obtiennent la délivrance. L'airain nous parle du jugement (Job 20:24), le serpent de la malédiction (Ge. 3:14), et la perche parle de la croix (1 Co. 1:18). Jésus a pris nos malédictions sur la croix de Golgotha (Ga. 3:13).}, et le mit sur une perche~; quiconque avait été mordu par un serpent et regardait le serpent d'airain conservait la vie.
\VS{10}Les enfants d'Israël partirent et campèrent à Oboth.
\VS{11}Et ils partirent d'Oboth et ils campèrent en Ijjé-Abarim, dans le désert qui est vis-à-vis de Moab, vers le soleil levant.
\VS{12}Puis ils partirent de là et campèrent vers le torrent de Zéred.
\VS{13}Et ils partirent de là et campèrent de l'autre côté de l'Arnon, qui est dans le désert, en sortant du territoire des Amoréens~; car l'Arnon est la frontière de Moab, entre les Moabites et les Amoréens\FTNT{Jg. 11:18.}.
\VS{14}C'est pourquoi il est dit dans le livre des batailles de Yahweh~: Vaheb en Supha, et les torrents de l'Arnon,
\VS{15}et le cours des torrents qui s'étend du côté d'Ar et touche à la frontière de Moab.
\VS{16}De là ils allèrent à Beer. C'est là le puit où Yahweh dit à Moïse~: Rassemble le peuple, et je leur donnerai de l'eau.
\VS{17}Alors Israël chanta ce cantique~: Monte, puits~! Chantez-lui en vous répondant les uns aux autres.
\VS{18}Puits que des princes ont creusés. Que les grands du peuple ont creusé, avec le législateur, avec leurs bâtons~! Du désert ils vinrent à Matthana~;
\VS{19}de Matthana à Nahaliel~; et de Nahaliel à Bamoth~;
\VS{20}de Bamoth à la vallée qui est dans le territoire de Moab, au sommet de Pisga, et qui regarde vers Jeshimon.
\TextTitle{Israël bat le roi des Amoréens et le roi de Basan}
\VS{21}Puis Israël envoya des messagers à Sihon, roi des Amoréens, pour lui dire~:
\VS{22}Laisse-moi passer par ton pays~; nous ne nous détournerons ni dans les champs, ni dans les vignes, et nous ne boirons l'eau d'aucun des puits~; mais nous marcherons par la route royale, jusqu'à ce que nous ayons passé ton territoire.
\VS{23}Mais Sihon ne permit pas à Israël de passer sur son territoire~; il rassembla tout son peuple et sortit à la rencontre d'Israël, dans le désert~; il vint à Jahats, et combattit Israël\FTNT{De. 2:26-30~; Jg. 11:29-30.}.
\VS{24}Israël le fit passer au fil de l'épée et conquit son pays, depuis l'Arnon jusqu'à Jabbok, et jusqu'à la frontière des fils d'Ammon~; car la frontière des fils d'Ammon était forte\FTNT{De. 2:30~; De. 29:7~; Ps. 135:11-12.}.
\VS{25}Et Israël prit toutes les villes qui étaient là, et habitat dans toutes les villes des Amoréens, à Hesbon, et dans toutes les villes de son ressort.
\VS{26}Or Hesbon était la ville de Sihon, roi des Amoréens, qui avait le premier fait la guerre au roi de Moab, et pris sur lui tout son pays jusqu'à l'Arnon.
\VS{27}C'est pourquoi les poètes disent~: Venez à Hesbon~! Que la ville de Sihon soit rebâtie et fortifiée~!
\VS{28}Car le feu est sorti de Hesbon, et la flamme de la cité de Sihon~; elle a consumé Ar-Moab, les habitants des hauteurs de l'Arnon.
\VS{29}Malheur à toi, Moab~! Peuple de Kemosch, tu es perdu~! Il a livré ses fils qui se sauvaient et ses filles en captivité à Sihon, roi des Amoréens\FTNT{Jé. 48:46.}.
\VS{30}Nous les avons défaits à coups de flèches~: De Hesbon à Dibon tout est détruit~; nous les avons mis en déroute jusqu'à Nophach, jusqu'à Médeba.
\VS{31}Israël s'établit dans le pays des Amoréens.
\VS{32}Puis Moïse envoya des gens pour reconnaître Jaezer, ils prirent les villes de son ressort, et chassèrent les Amoréens qui y étaient.
\VS{33}Ensuite, ils se tournèrent et montèrent par le chemin de Basan. Og, roi de Basan, sortit à leur rencontre, avec tout son peuple pour les combattre à Edréï.
\VS{34}Et Yahweh dit à Moïse~: Ne le crains point, car je le livre entre tes mains, lui et tout son peuple, et son pays~; tu le traiteras comme tu as traité Sihon, roi des Amoréens, qui habitait à Hesbon\FTNT{De. 3:1-2.}.
\VS{35}Ils le battirent donc, lui et ses fils, et tout son peuple, sans en laisser échapper un seul, et ils s'emparèrent de son pays.
\Chap{22}
\TextTitle{Balak cherche à maudire Israël~; Balaam\FTNTT{2 Pi. 2:15~; Jud. 11~; Ap. 2:14} séduit par les honneurs}
\VerseOne{}Puis les enfants d'Israël partirent, et ils campèrent dans les plaines de Moab, au-delà du Jourdain, vis-à-vis de Jéricho.
\VS{2}Balak, fils de Tsippor, vit tout ce qu'Israël avait fait aux Amoréens.
\VS{3}Et Moab eut une grande frayeur du peuple, parce qu'il était en grand nombre, il fut saisi de terreur en face des enfants d'Israël.
\VS{4}Et Moab dit aux anciens de Madian~: Maintenant cette multitude va brouter tout ce qui nous entoure, comme le bœuf broute l'herbe des champs. Balak, fils de Tsippor, était alors roi de Moab.
\VS{5}Il envoya des messagers auprès de Balaam, fils de Beor, à Pethor, située sur le fleuve, dans le pays des fils de son peuple, afin de l'appeler et de lui dire~: Voici, un peuple est sorti d'Egypte, il couvre la surface de la terre, et il habite vis-à-vis de moi.
\VS{6}Viens donc maintenant, je te prie, maudis-moi ce peuple, car il est plus puissant que moi~; peut-être que je serai le plus fort, et que nous le battrons, et que je le chasserai du pays~; car je sais que celui que tu bénis est béni, et que celui que tu maudis est maudit.
\VS{7}Les anciens de Moab s'en allèrent avec les anciens de Madian, ayant dans leurs mains de quoi payer le devin. Ils arrivèrent auprès de Balaam, et lui rapportèrent les paroles de Balak.
\VS{8}Il leur répondit~: Demeurez ici cette nuit, et je vous répondrai d'après ce que Yahweh me dira. Et les chefs des Moabites restèrent chez Balaam.
\VS{9}Et Dieu vint à Balaam et dit~: Qui sont ces hommes que tu as chez toi~?
\VS{10}Et Balaam répondit à Dieu~: Balak, fils de Tsippor, roi de Moab, les a envoyés pour me dire~:
\VS{11}Voici, un peuple qui est sorti d'Egypte, et qui couvre la face de la terre~; viens donc, maudis-le-moi~; peut-être qu'ainsi je pourrai le combattre, et je le chasserai.
\VS{12}Et Dieu dit à Balaam~: Tu n'iras point avec eux, et tu ne maudiras point ce peuple, car il est béni.
\VS{13}Et Balaam se leva le matin, et il dit aux chefs qui avaient été envoyés par Balak~: Retournez dans votre pays, car Yahweh refuse de me laisser venir avec vous.
\VS{14}Ainsi les chefs des Moabites se levèrent et retournèrent auprès de Balak, et dirent~: Balaam a refusé de venir avec nous.
\VS{15}Et Balak envoya encore des chefs en plus grand nombre, et plus considérés que les premiers.
\VS{16}Ils arrivèrent auprès de Balaam, et lui dirent~: Ainsi parle Balak, fils de Tsippor~: Que l'on ne t'empêche donc pas de venir vers moi~;
\VS{17}car je te rendrai beaucoup d'honneur, et je ferai tout ce que tu me diras~; je te prie donc viens, maudis-moi ce peuple.
\VS{18}Et Balaam répondit et dit aux serviteurs de Balak~: Quand Balak me donnerait sa maison pleine d'or et d'argent, je ne pourrais point transgresser l'ordre de Yahweh, mon Dieu~; je ne pourrais faire aucune chose, ni petite ni grande.
\VS{19}Toutefois, je vous prie, demeurez maintenant ici encore cette nuit, et je saurai ce que Yahweh aura de plus à me dire.
\VS{20}Dieu vint, la nuit à Balaam, et lui dit~: Puisque ces hommes sont venus t'appeler, lève-toi, et va avec eux~; mais quoi qu'il en soit, tu feras ce que je te dirai.
\VS{21}Ainsi Balaam se leva le matin, et sella son ânesse, et partit avec les chefs de Moab.
\VS{22}Mais la colère de Dieu s'enflamma parce qu'il était parti~; et l'Ange de Yahweh se plaça sur le chemin pour lui résister. Balaam était monté sur son ânesse, et ses deux serviteurs étaient avec lui.
\VS{23}L'ânesse vit l'Ange de Yahweh qui se tenait sur le chemin, son épée nue dans la main~; elle se détourna du chemin et alla dans les champs. Balaam frappa l'ânesse pour la ramener dans le chemin\FTNT{2 Pi. 2:16~; Jud. 1:11.}.
\VS{24}L'Ange de Yahweh se plaça dans un sentier entre les vignes~; il y avait un mur de chaque côté.
\VS{25}L'ânesse vit l'Ange de Yahweh~; elle se serra contre le mur, et elle serra le pied de Balaam contre le mur. Balaam la frappa de nouveau.
\VS{26}Et l'Ange de Yahweh passa plus loin et s'arrêta dans un lieu étroit où il n'y avait point d'espace pour se détourner à droite ou à gauche.
\VS{27}Et l'ânesse vit l'Ange de Yahweh, et elle s'abattit sous Balaam. Balaam se mit en grande colère, et il frappa l'ânesse avec son bâton.
\VS{28}Alors Yahweh fit parler l'ânesse, et elle dit à Balaam~: Que t'ai-je fait, pour que tu m'aies déjà frappée trois fois~?
\VS{29}Et Balaam répondit à l'ânesse~: C'est parce que tu t'es moquée de moi~; si j'avais une épée dans la main, je te tuerai sur le champ !
\VS{30}Et l'ânesse dit à Balaam: Ne suis-je pas ton ânesse, sur laquelle tu montes depuis que je suis à toi, jusqu'à aujourd'hui~? Ai-je l'habitude de te faire ainsi~? Et il répondit~: Non.
\VS{31}Alors Yahweh ouvrit les yeux de Balaam, et il vit l'Ange de Yahweh qui se tenait sur le chemin, et qui avait dans sa main son épée nue~; et il s'inclina et se prosterna sur son visage.
\VS{32}Et l'Ange de Yahweh lui dit~: Pourquoi as-tu frappé ton ânesse déjà trois fois~? Voici je suis sorti pour m'opposer à toi~; car ta voie est devant moi une voie de perdition.
\VS{33}Mais l'ânesse m'a vu et elle s'est détournée de devant moi déjà trois fois~; autrement, si elle ne s'était détournée de moi, je t'aurais même déjà tué, et je lui aurais laissé la vie.
\VS{34}Alors Balaam dit à l'Ange de Yahweh~: J'ai péché, car je ne savais point que tu t'étais placé au-devant de moi sur le chemin~; et maintenant, si cela te déplaît, je m'en retournerai.
\VS{35}L'Ange de Yahweh dit à Balaam~: Va avec ces hommes~; mais tu ne feras que répéter les paroles que je te dirai. Et Balaam alla avec les chefs envoyés par Balak.
\VS{36}Et quand Balak apprit que Balaam arrivait, il sortit à sa rencontre jusqu'à la ville de Moab, qui est sur la limite de l'Arnon, à l'extrême frontière.
\VS{37}Et Balak dit à Balaam~: N'ai-je pas auparavant envoyé vers toi pour t'appeler~? Pourquoi n'es-tu pas venu vers moi~? Ne puis-je donc pas te traiter avec honneur~?
\VS{38}Et Balaam répondit à Balak~: Je suis venu vers toi~; mais pourrais-je maintenant dire quelque chose~? Je ne dirai que les paroles que Dieu m'aura mis dans la bouche.
\VS{39}Et Balaam alla avec Balak, et ils arrivèrent dans la cité de Kirjath-Hutsoth.
\VS{40}Et Balak sacrifia des bœufs et des brebis, et il en envoya à Balaam et aux chefs qui étaient venus avec lui.
\VS{41}Quand le matin fut venu, il prit Balaam et le fit monter à Bamoth-Baal, et de là il vit une partie du peuple.
\Chap{23}
\TextTitle{Balaam ne maudit pas mais bénit Israël des hauts lieux de Baal}
\VerseOne{}Et Balaam dit à Balak~: Bâtis-moi ici sept autels, et prépare-moi ici sept veaux et sept béliers.
\VS{2}Et Balak fit ce que Balaam avait dit~; et Balak offrit avec Balaam un veau et un bélier sur chaque autel.
\VS{3}Balaam dit à Balak~: Tiens-toi près de ton holocauste, et je m'éloignerai~; peut-être que Yahweh viendra à ma rencontre, et je te rapporterai tout ce qu'il me révélera. Ainsi il se retira à l'écart.
\VS{4}Et Dieu vint au-devant de Balaam, et Balaam lui dit~: J'ai dressé sept autels, et j'ai sacrifié un veau et un bélier sur chaque autel.
\VS{5}Et Yahweh mit des paroles dans la bouche de Balaam et lui dit~: Retourne vers Balak, et tu parleras ainsi.
\VS{6}Il s'en retourna donc vers lui~; et voici, Balak se tenait près de son holocauste, tant lui que tous les chefs de Moab.
\VS{7}Alors Balaam prononça son discours sentencieux et dit~: Balak, roi de Moab, m'a fait descendre d'Aram\FTNT{De l'hébreu «~Aram~» traduit par «~Aram~» ou «~Syrie~» (1 R. 11:25).}, des montagnes d'orient, en me disant~: Viens, maudis-moi Jacob~! Viens, dis-je, déteste Israël~!
\VS{8}Mais comment le maudirai-je~? Dieu ne l'a point maudit. Et comment le détesterai-je~? Yahweh ne l'a point détesté.
\VS{9}Car je le regarderai du sommet des rochers, et je le contemplerai du haut des collines~: Voici, ce peuple habitera à part, et il ne sera pas compté parmi les nations\FTNT{De. 33:28.}.
\VS{10}Qui comptera la poussière de Jacob, et dira le nombre du quart d'Israël~? Que je meure de la mort des justes, et que ma fin soit semblable à la leur~!
\VS{11}Alors Balak dit à Balaam~: Que m'as-tu fait~? Je t'ai pris pour maudire mes ennemis, et voici, tu les a bénis, tu les a bénis\FTNT{Le verbe bénir vient de l'hébreu «~Barak~», il est utilisé deux fois de suite dans ce passage. Voir commentaire en Ge. 2:16-17.}.~!
\VS{12}Et il répondit, et dit~: Ne prendrais-je pas garde de dire les paroles que Yahweh aura mis dans ma bouche~?
\TextTitle{Balaam bénit Israël au sommet de Pisga}
\VS{13}Alors Balak lui dit~: Viens, je te prie, avec moi dans un autre lieu, d'où tu pourras le voir, car tu en voyais seulement une extrémité, et tu ne le voyais pas tout entier~; maudis-le moi de là.
\VS{14}Puis l'ayant conduit au territoire de Tsophim, sur le sommet de Pisga~; il bâtit sept autels, et offrit un taureau et un bélier sur chaque autel.
\VS{15}Alors Balaam dit à Balak~: Tiens-toi ici près de ton holocauste, et je m'en irai à la rencontre de Dieu, comme j'ai déjà fait.
\VS{16}Yahweh donc vint au-devant de Balaam, il mit des paroles dans sa bouche et lui dit~: Retourne vers Balak, et tu parleras ainsi.
\VS{17}Il retourna vers Barak~; et voici, il se tenait près de son holocauste, et les chefs de Moab avec lui. Et Balak lui dit~: Qu'est-ce que Yahweh a dit~?
\VS{18}Alors il prononça son discours sentencieux et dit~: Lève-toi, Balak, écoute~! Fils de Tsippor, prête-moi l'oreille~!
\VS{19}Dieu n'est point un homme pour mentir ni fils d'un homme pour se repentir. Ce qu'il a dit, ne le fera-t-il pas~? Ce qu'il a déclaré, ne l'exécutera-t-il pas\FTNT{Ja. 1:17.}~?
\VS{20}Voici, j'ai reçu la parole pour bénir~: Puisqu'il a béni, je ne le révoquerai point.
\VS{21}Il n'a point aperçu d'iniquité en Jacob, il ne voit point de perversité en Israël~; Yahweh, son Dieu, est avec lui, et il y a en lui un chant de triomphe royal\FTNT{Jé. 50:20~; Ro. 4:7.}.
\VS{22}Dieu les a tirés d'Egypte, il est pour eux comme la vigueur du buffle.
\VS{23} Car il n'y a pas d'enchantement contre Jacob, ni la divination contre Israël. Au temps marqué, il sera dit à Jacob et à Israël~: Qu'est-ce que Dieu a fait~?
\VS{24}Voici, ce peuple se lèvera comme un vieux lion, et se dressera comme un lion qui est dans sa force~; il ne se couchera pas jusqu'à ce qu'il ait dévoré la proie, et bu le sang des blessés à mort.
\VS{25}Balak dit à Balaam~: Et bien~! Ne le maudis pas, mais du moins ne le bénis pas.
\VS{26}Et Balaam répondit à Balak~: Ne t'ai-je pas dit que tout ce que Yahweh dira, je le ferai~? 
\TextTitle{Balaam bénit Israël de Peor}
\VS{27}Balak dit encore à Balaam~: Viens maintenant, je te conduirai dans un autre lieu~; peut-être que Dieu trouvera bon que tu me le maudisses de là.
\VS{28}Balak conduisit donc Balaam sur le sommet de Peor, qui regarde du côté de Jeshimon.
\VS{29}Et Balaam lui dit~: Bâtis-moi ici sept autels, et apprête-moi ici sept veaux et sept béliers.
\VS{30}Et Balak fit donc comme Balaam lui avait dit~; puis il offrit un taureau et un bélier sur chaque autel.
\Chap{24}
\VerseOne{}Or Balaam, voyant que Yahweh voulait bénir Israël, n'alla plus comme les autres fois chercher des enchantements~; mais il tourna son visage du côté du désert.
\VS{2}Et Balaam leva les yeux, il vit Israël qui se tenait rangé selon ses tribus. Alors l'Esprit de Dieu fut sur lui.
\VS{3}Et il prononça à haute voix son discours sentencieux et dit~: Balaam, fils de Beor, dit, et l'homme qui a l'œil ouvert dit:
\VS{4}Celui qui entend les paroles de Dieu, qui voit la vision du Tout-Puissant, qui tombe à terre, et qui a les yeux ouverts dit~:
\VS{5}Que tes tentes sont belles, ô Jacob~! Et tes tabernacles, ô Israël~!
\VS{6}Ils sont étendus comme des torrents, comme des jardins près d'un fleuve, comme des arbres d'aloès que Yahweh a plantés, comme des cèdres auprès des eaux.
\VS{7}L'eau coule de ses seaux, et sa semence est parmi d'abondantes eaux. Et son roi s'élève au-dessus d'Agag, et son royaume sera haut élevé.
\VS{8}Dieu, qui l'a tiré d'Egypte, est pour lui comme la vigueur du buffle~; il consumera les nations qui sont ses ennemies~; il brisera leurs os, et les percera de ses flèches.
\VS{9}Il s'est courbé, il s'est couché comme un lion qui est dans sa force, et comme un vieux lion~; qui le réveillera~? Quiconque te bénit, sera béni, et quiconque te maudit, sera maudit.
\VS{10}Alors Balak se mit très en colère contre Balaam, il frappa des mains et Balak parla ainsi à Balaam~: Je t'ai appelé pour maudire mes ennemis, et voici, tu les as bénis, tu les a bénis\FTNT{voir le commentaire en Ge. 2:16-17.} trois fois déjà.
\VS{11}Et maintenant, fuis dans ton pays~! J'avais dit que je t'honorerais, je t'honorerais\FTNT{Le mot hébreu est utilisé deux fois dans ce passage. Voir le commentaire en Ge. 2:17.}, mais Yahweh t'empêche d'être honoré.
\VS{12}Et Balaam répondit à Balak~: N'ai-je pas dit à tes messagers que tu m'as envoyés~:
\VS{13}Quand Balak me donnerait sa maison pleine d'argent et d'or, je ne pourrais transgresser l'ordre de Yahweh pour faire de moi-même du bien ou du mal~; mais ce que Yahweh dira, je le dirai.
\VS{14}Maintenant donc je m'en vais vers mon peuple. Viens, je te donnerai un conseil, et je te dirai ce que ce peuple fera à ton peuple, dans les derniers jours.
\TextTitle{Prophétie sur le Roi qui sort de Jacob, le Messie}
\VS{15}Alors il prononça son discours sentencieux et dit~: Balaam, fils de Beor dit, et l'homme qui a l'œil ouvert dit~:
\VS{16}Celui qui entend les paroles de Dieu, qui connaît la science du Très-Haut, qui voit la vision du Tout-Puissant, qui tombe à terre, et qui a les yeux ouverts.
\VS{17}Je le vois, mais non pas maintenant~; je le regarde, mais non pas de près~; une Etoile est sortie de Jacob\FTNT{L'Etoile en question est Jésus-Christ, qui se révéla à Jean comme l'Etoile brillante du matin (Ap. 22:16).}, et un Sceptre s'est élevé d'Israël. Il transpercera les cotés de Moab, il détruira tous les enfants de Seth.
\VS{18}Edom sera sa possession, Séir sera possédé par ses ennemis, et Israël se portera vaillamment.
\VS{19}Et il y en aura un de Jacob qui dominera, il fera périr le reste de la ville.
\VS{20}Il vit aussi Amalek, il prononça son discours sentencieux et dit~: Amalek est le premier des nations, mais à la fin il sera détruit.
\VS{21}Il vit aussi les Kéniens\FTNT{Il y a plusieurs sens à ce mot~:
\\Caïn = «~possession~», «~artisan, forgeron~», fils d'Adam.
\\Kéniens = «~forgerons~» tribu du beau-père de Moïse qui vivait dans la région du sud de la Palestine.}. Il prononça à haute voix son discours sentencieux et dit~: Ta demeure est dans un lieu solide, et tu as mis ton nid dans le rocher~;
\VS{22}toutefois, le Kénien sera consumé, jusqu'à ce que l'Assyrien t'emmène en captivité.
\VS{23}Il continua à prononcer à haute voix son discours sentencieux, et il dit~: Malheur à celui qui vivra quand Dieu fera ces choses.
\VS{24}Et des navires viendront de Kittim, et ils humilieront l'Assyrien et l'Hébreu~; et lui aussi sera détruit.
\VS{25}Puis Balaam se leva, et s'en alla pour retourner chez lui. Balak aussi s'en alla son chemin.
\Chap{25}
\TextTitle{Prostitution d'Israël à Baal-Peor\FTNTT{No. 31:16~; Ja. 4:4~; Ap. 2:14.}}
\VerseOne{}Alors Israël demeurait à Sittim~; et le peuple commença à commettre la fornication avec les filles de Moab.
\VS{2}Car elles convièrent le peuple aux sacrifices de leurs dieux~; et le peuple mangea et se prosterna devant leurs dieux.
\VS{3}Et Israël s'accoupla à Baal-Peor, c'est pourquoi la colère de Yahweh s'enflamma contre Israël\FTNT{Ps. 106:28~; Os. 9:10.}.
\VS{4}Et Yahweh dit à Moïse~: Prends tous les chefs du peuple, et fais-les pendre devant Yahweh en face du soleil, afin que la colère de Yahweh se détourne d'Israël\FTNT{De. 4:3~; Jos. 22:17.}.
\VS{5}Moïse donc dit aux juges d'Israël~: Que chacun de vous fasse mourir les hommes qui sont à sa charge, et qui se sont joints à Baal-Peor.
\VS{6}Et voici, un homme des enfants d'Israël vint, et amena à ses frères une Madianite, devant Moïse et devant toute l'assemblée des fils d'Israël, tandis qu'ils pleuraient à l'entrée de la tente d'assignation.
\VS{7}Ce que Phinées, fils d'Eléazar, fils d'Aaron le prêtre, ayant vu, se leva du milieu de l'assemblée et prit une lance dans sa main.
\VS{8}Et il entra dans la tente de l'homme Israélite et les transperça tous deux, l'homme Israélite puis la femme, par le ventre. Et la plaie s'arrêta parmi les enfants d'Israël\FTNT{Ps. 106:30.}.
\VS{9}Or il y en eut vingt-quatre mille qui moururent de cette plaie.
\VS{10}Et Yahweh parla à Moïse, en disant~:
\VS{11}Phinées, fils d'Eléazar, fils d'Aaron, le prêtre, a détourné ma colère de dessus les enfants d'Israël, parce qu'il a été animé de mon zèle au milieu d'eux~; et je n'ai point, dans mon ardeur, consumé les fils d'Israël.
\VS{12}C'est pourquoi, dis-lui~: Voici, je lui donne mon alliance de paix.
\VS{13}Et l'alliance de prêtrise perpétuelle sera tant pour lui que pour sa postérité après lui, parce qu'il a été animé de zèle pour son Dieu, et qu'il a fait propitiation pour les enfants d'Israël.
\VS{14}Et le nom de l'homme Israélite tué, lequel fut tué avec la Madianite, était Zimri, fils de Salu, chef d'une maison de père des Siméonites.
\VS{15}Et le nom de la femme Madianite qui fut tuée était Cozbi, fille de Tsur, chef du peuple, et d'une maison de père en Madian.
\VS{16}Yahweh parla à Moïse, en disant~:
\VS{17}Mettez en détresse les Madianites, tuez-les~;
\VS{18}car ils vous ont serrés les premiers par leurs ruses, par lequelles ils vous surpris dans l'affaire de Peor, et dans l'affaire de Cozbi, fille d'un chef d'entre les Madianites, leur sœur, qui a été tuée le jour de la plaie, causée par l'affaire de Peor.
\Chap{26}
\TextTitle{Nouveau dénombrement des hommes de guerre}
\VerseOne{} Or il arriva qu'après cette plaie-là, que Yahweh parla à Moïse, et à Eléazar, fils d'Aaron, le prêtre, en disant~:
\VS{2}Faites le dénombrement de toute l'assemblée des enfants d'Israël, depuis l'âge de vingt ans et au-dessus, selon les maisons de leurs pères, à savoir de tous ceux d'Israël qui peuvent aller à la guerre.
\VS{3}Moïse donc et Eléazar, le prêtre, leur parlèrent donc dans les plaines de Moab, près du Jourdain de Jéricho, en disant~:
\VS{4}Qu'on fasse le dénombrement depuis l'âge de vingt ans et au-dessus, comme Yahweh l'avait ordonné à Moïse et aux enfants d'Israël, quand ils furent sortis du pays d'Egypte.
\VS{5}Ruben, premier-né d'Israël. Fils de Ruben~: Hénoc, de qui descend la famille des Hénokites~; Pallu, de qui descend la famille des Palluites~;
\VS{6}Hetsron, de qui descend la famille des Hetsronites~; Carmi, de qui descend la famille des Carmites.
\VS{7}Ce sont là les familles des Rubénites~: Ceux qui furent dénombrés étaient quarante-trois mille sept cent trente.
\VS{8}Et les fils de Pallu~: Eliab.
\VS{9}Fils d'Eliab~: Nemuel, Dathan et Abiram. Ce Dathan et cet Abiram, qui étaient de ceux qu'on appelait pour tenir l'assemblée, et qui se révoltèrent contre Moïse et contre Aaron dans l'assemblée de Koré, lors de leur révolte contre Yahweh.
\VS{10}Et lorsque la terre ouvrit sa bouche et les engloutit, ainsi que Koré, ceux qui s'étaient assemblés avec lui moururent. Et le feu dévora les deux cent cinquante hommes qui servirent d'avertissement.
\VS{11}Mais les fils de Koré ne moururent pas.
\VS{12}Les fils de Siméon selon leurs familles~: De Nemuel descend la famille des Némuélites~; de Jamin, la famille des Jaminites~; de Jakin, la famille des Jakinites~;
\VS{13}de Zérach, la famille des Zérachites~; de Saül, la famille des Saülites.
\VS{14}Ce sont là les familles des Siméonites, qui furent vingt-deux mille deux cents.
\VS{15}Fils de Gad selon leurs familles. De Tsephon, descend la famille des Tsephonites~; de Haggi, la famille des Haggites~; de Schuni, la famille des Schunites~;
\VS{16}d'Ozni, la famille des Oznites~; d'Eri, la famille des Erites~;
\VS{17}d'Arod, la famille des Arodites~; d'Areéli, la famille des Areélites.
\VS{18}Ce sont là les familles des fils de Gad, d'après leur dénombrement~: Quarante mille cinq cents.
\VS{19}Fils de Juda, Er, et Onan~; mais Er et Onan moururent au pays de Canaan\FTNT{Ge. 38:7-10~; Ge. 46:12.}.
\VS{20}Voici les fils de Juda selon leurs familles~: De Schéla descend la famille des Schélanites~; de Pérets, la famille des Péretsites~; de Zérach, la famille des Zérachites.
\VS{21}Les fils de Pérets furent~: Hetsron, de qui descend la famille des Hetsronites~; Hamul, de qui descend la famille des Hamulites.
\VS{22}Ce sont là les familles de Juda, selon leur dénombrement: Soixante-seize mille cinq cents.
\VS{23}Fils d'Issacar, selon leurs familles~: De Thola descend la famille des Tholaïtes~; de Puva, la famille des Puvites~;
\VS{24}de Jaschub, la famille des Jaschubites~; de Schimron, la famille des Schimronites.
\VS{25}Ce sont là les familles d'Issacar, d'après leur dénombrement~: Soixante-quatre mille trois cents.
\VS{26}Fils de Zabulon, selon leurs familles~: De Séred, descend la famille des Sardites~; d'Elon, la famille des Elonites~; de Jahleel, la famille des Jahleélites.
\VS{27}Ce sont là les familles des Zabulonites, d'après leur dénombrement~: Soixante mille cinq cents.
\VS{28}Fils de Joseph, selon leurs familles~: Manassé et Ephraïm.
\VS{29}Fils de Manassé. De Makir descend la famille des Makirites. Makir engendra Galaad. De Galaad descend la famille des Galaadites.
\VS{30}Voici les fils de Galaad~: Jézer, de qui descend la famille des Jézerites~; Hélek, la famille des Hélekites.
\VS{31}Asriel, la famille des Asriélites~; Sichem, la famille des Sichémites~;
\VS{32}Schemida, la famille des Schemidaïtes~; Hépher, la famille des Héphrites.
\VS{33}Tselophchad, fils de Hépher, n'eut point de fils, mais des filles. Voici les noms des filles de Tselophchad~: Machla, Noa, Hogla, Milca, et Thirtsa.
\VS{34}Ce sont là les familles de Manassé, d'après leur dénombrement~: Cinquante-deux mille sept cents.
\VS{35}Voici les fils d'Ephraïm, selon leurs familles~: De Schutélach descend la famille des Schutalchites~; de Béker, la famille des Bakrites~; de Thachan, la famille des Thachanites.
\VS{36}Voici les fils de Schutélach~: D'Eran est descendue la famille des Eranites.
\VS{37}Ce sont là les familles des fils d'Ephraïm, d'après leur dénombrement~: Trente-deux mille cinq cents. Ce sont là les fils de Joseph, selon leurs familles.
\VS{38}Fils de Benjamin, selon leurs familles~: De Béla descend la famille des Balites~; d'Aschbel, la famille des Aschbélites~; d'Achiram, la famille des Achiramites~;
\VS{39}De Schupham, la famille des Schuphamites~; de Hupham, la famille des Huphamites.
\VS{40}Les fils de Béla furent Ard et Naaman. D'Ard descend la famille des Ardites~; et de Naaman la famille des Naamanites.
\VS{41}Ce sont là les fils de Benjamin, d'après leurs familles~; et leur dénombrement~: Quarante-cinq mille six cents.
\VS{42}Voici les fils de Dan, selon leurs familles~: De Schucham descend la famille des Schuchamites. Ce sont là les familles de Dan, selon leurs familles.
\VS{43}Toutes les familles des Schuchamites, selon leur dénombrement~: Soixante-quatre mille quatre cents.
\VS{44}Fils d'Aser, selon leurs familles~: De Jimna descend la famille des Jimnites~; de Jischvi, la famille des Jischvites~; de Beria la famille des Beriites.
\VS{45}Des fils de Beria descendent~: De Héber, la famille des Hébrites~; de Malkiel, la famille des Malkiélites.
\VS{46}Et le nom de la fille d'Aser était Sérach.
\VS{47}Ce sont là les familles des fils d'Aser, d'après leur dénombrement~: Cinquante-trois mille quatre cents.
\VS{48}Fils de Nephthali, selon leurs familles~: De Jahtseel descend la famille des Jahtseélites~; de Guni, la famille des Gunites~;
\VS{49}de Jetser la famille des Jitsrites~; de Schillem, la famille des Schillémites.
\VS{50}Ce sont là les familles de Nephthali, selon leurs familles, et leur dénombrement~: Quarante-cinq mille quatre cents.
\VS{51}Voici les dénombrés des fils d'Israël, qui furent six cent un mille sept cent trente.
\VS{52}Yahweh parla à Moïse, en disant~:
\VS{53}Le pays sera partagé entre ceux-ci en héritage, selon le nombre des noms.
\VS{54}A ceux qui sont en plus grand nombre, tu donneras plus d'héritage, et à ceux qui sont en plus petit nombre tu donneras moins d'héritage~; on donnera à chacun son héritage selon le nombre de ses dénombrés.
\VS{55}Toutefois, que le pays soit partagé par le sort~; et qu'ils prennent leur héritage selon les noms des tribus de leurs pères\FTNT{Jos. 11:23~; Jos. 14:2~; Jos. 18:6-8.}.
\VS{56}L'héritage de chacun sera selon que le sort le montrera, et on aura égard au plus grand et au plus petit nombre.
\VS{57}Et ce sont ici les dénombrés de Lévi selon leurs familles~; de Guerschon, la famille des Guerschonites~; de Kehath, la famille des Kehathites~; de Merari, la famille des Merarites.
\VS{58}Ce sont ici les familles de Lévi~; la famille des Libnites, la famille des Hébronites, la famille des Machlites, la famille des Muschites, la famille des Korites. Kehath engendra Amram.
\VS{59}Et le nom de la femme d'Amram était Jokébed, fille de Lévi, qui naquit à Lévi en Egypte~; et elle enfanta à Amram: Aaron, Moïse, et Marie, leur sœur.
\VS{60}Et il naquit à Aaron~: Nadab et Abihu, Eléazar et Ithamar.
\VS{61}Nadab et Abihu moururent lorsqu'ils apportèrent du feu étranger devant Yahweh\FTNT{Lé. 10:1-2~; 1 Ch. 24:2.}.
\VS{62}Et tous les dénombrés des Lévites furent vingt-trois mille, tous mâles, depuis l'âge d'un mois, et au dessus, qui ne furent point dénombrés avec les autres enfants d'Israël, car on ne leur donna point d'héritage entre les enfants d'Israël.
\VS{63}Ce sont là ceux qui furent dénombrés par Moïse et Eléazar, le prêtre, qui firent le dénombrement des fils d'Israël dans les plaines de Moab, près du Jourdain de Jéricho.
\VS{64}Entre lesquels il ne s'en trouva aucun de ceux qui avaient été dénombrés par Moïse et Aaron le prêtre, quand ils firent le dénombrement des enfants d'Israël au désert de Sinaï.
\VS{65}Car Yahweh avait dit d'eux~: ils mourront certainement dans le désert, et qu'ainsi il n'en restera pas un, excepté Caleb, fils de Jephunné, et Josué, fils de Nun\FTNT{1 Co. 10:5.}.
\Chap{27}
\TextTitle{Loi sur les héritages\FTNTT{No. 36.}}
\VerseOne{}Or les filles de Tselophchad, fils de Hépher, fils de Galaad, fils de Makir, fils de Manassé, d'entre les familles de Manassé, fils de Joseph, s'approchèrent~; et ce sont ici les noms de ses filles~: Machla, Noa, Hogla, Milca, et Thirtsa.
\VS{2}Elles se présentèrent devant Moïse, devant Eléazar, le prêtre, et devant les princes et toute l'assemblée, à l'entrée de la tente d'assignation. Elles dirent~:
\VS{3}Notre père est mort dans le désert~; il n'était toutefois pas dans la troupe de ceux qui s'assemblèrent contre Yahweh, dans l'assemblée de Koré, mais il est mort dans son péché, et il n'avait point de fils.
\VS{4}Pourquoi le nom de notre père serait-il retranché de sa famille, parce qu'il n'a point eu de fils~? Donne-nous une possession parmi les frères de notre père.
\VS{5}Moïse rapporta leur cause devant Yahweh.
\VS{6}Et Yahweh parla à Moïse, en disant~:
\VS{7}Les filles de Tselophchad ont parlé droitement. Tu ne manqueras pas de leur donner un héritage à posséder parmi les frères de leur père, et tu leur feras passer l'héritage de leur père.
\VS{8}Tu parleras aussi aux enfants d'Israël, et tu leur diras~: Lorsqu'un homme mourra sans avoir de fils, vous ferez passer son héritage à sa fille.
\VS{9}S'il n'a pas de fille, vous donnerez son héritage à ses frères.
\VS{10}S'il n'a pas de frères, vous donnerez son héritage aux frères de son père.
\VS{11}Et si son père n'a pas de frère, vous donnerez son héritage à son parent le plus proche de sa famille, et il le possédera. Et ce sera pour les enfants d'Israël une ordonnance de droit, comme Yahweh l'a ordonné à Moïse.
\TextTitle{Moïse voit de loin le pays promis aux fils d'Israël}
\VS{12}Yahweh dit aussi à Moïse~: Monte sur cette montagne d'Abarim, et regarde le pays que je donne aux enfants d'Israël\FTNT{De. 32:48-49.}.
\VS{13}Tu le regarderas donc~; et puis tu seras toi aussi recueilli auprès de ton peuple, comme Aaron ton frère y a été recueilli~;
\VS{14}parce que vous avez été rebelles à mon ordre dans le désert de Tsin, lors de la contestation de l'assemblée, vous ne m'avez point sanctifié au sujet des eaux devant eux~; ce sont les eaux de Meriba, à Kadès, dans le désert de Tsin.
\TextTitle{Yahweh désigne Josué comme successeur de Moïse}
\VS{15}Moïse parla à Yahweh, en disant~:
\VS{16}Que Yahweh, le Dieu des esprits de toute chair, établisse sur l'assemblée un homme\FTNT{Hé. 12:9.},
\VS{17}qui sorte devant eux et qui entre devant eux, et qui les fasse sortir et qui les fasse entrer, afin que l'assemblée de Yahweh ne soit pas comme des brebis qui n'ont point de berger\FTNT{1 R. 22:17~; Mt. 9:36~; Mc. 6:34.}.
\VS{18}Alors Yahweh dit à Moïse~: Prends Josué, fils de Nun, un homme en qui est l'Esprit, et tu poseras ta main sur lui\FTNT{De. 34:9.}.
\VS{19}Tu le présenteras devant Eléazar, le prêtre, et devant toute l'assemblée~; et tu lui donneras des instructions sous leurs yeux.
\VS{20}Et tu lui feras part de ton autorité, afin que toute l'assemblée des enfants d'Israël l'écoute.
\VS{21}Et il se présentera devant Eléazar, le prêtre, qui consultera pour lui les jugements de l'urim\FTNT{Lé. 8:8.} devant Yahweh~; et à sa parole ils sortiront, et à sa parole ils entreront, lui, les enfants d'Israël, avec lui, et toute l'assemblée.
\VS{22}Moïse donc fit comme Yahweh lui avait ordonné. Il prit Josué et le présenta devant Eléazar, le prêtre, et devant toute l'assemblée.
\VS{23}Puis il posa ses mains sur lui, et lui donna des instructions, comme Yahweh l'avait dit par Moïse.
\Chap{28}
\TextTitle{Consignes relatives au temps des sacrifices}
\VerseOne{}Yahweh parla à Moïse, en disant~:
\VS{2}Donne cet ordre aux enfants d'Israël, et dis-leur~: Vous aurez soin de m'offrir en leur temps, mon offrande, ma nourriture, pour mes sacrifices consumés par le feu, qui me sont d'une bonne odeur\FTNT{Lé. 3:11~; Lé. 21:6.}.
\VS{3}Tu leur diras~: Voici le sacrifice consumé par le feu que vous offrirez à Yahweh~: Deux agneaux d'un an sans défaut, chaque jour, en holocauste perpétuel\FTNT{Ex. 29:38.}.
\VS{4}Tu sacrifieras l'un des agneaux le matin, et l'autre agneau entre les deux soirs,
\VS{5}et la dixième partie d'épha de fine farine pour le gâteau pétrie avec le quart d'un hin d'huile vierge\FTNT{Lé. 2:1~; Ex. 29:40~; Ex. 16:36.}.
\VS{6}C'est l'holocauste perpétuel, qui a été offert à la montagne de Sinaï, c'est un sacrifice consumé par le feu, d'une bonne odeur à Yahweh.
\VS{7}Et sa libation sera d'un quart de hin pour chaque agneau~: Et tu verseras dans le lieu saint la libation de boisson forte à Yahweh.
\VS{8}Et tu sacrifieras l'autre agneau entre les deux soirs, tu feras le même gâteau qu'au matin, et la même libation, en sacrifice consumé par le feu d'une bonne odeur à Yahweh.
\VS{9}Mais le jour du sabbat vous offrirez deux agneaux d'un an sans défaut, et deux dixièmes de fine farine pétrie à l'huile pour le gâteau, avec sa libation.
\VS{10}C'est l'holocauste du sabbat, pour chaque sabbat, outre l'holocauste perpétuel avec sa libation.
\VS{11}Et au commencement de vos mois, vous offrirez en holocauste à Yahweh deux jeunes taureaux, un bélier, et sept agneaux d'un an sans défaut~;
\VS{12}et trois dixièmes de fine farine pétrie à l'huile, pour le gâteau de chaque taureau, et deux dixièmes de fine farine pétrie à l'huile pour le gâteau du bélier~;
\VS{13}et un dixième de fine farine pétrie à l'huile, comme gâteau pour chaque agneau, en holocauste, d'une bonne odeur, et en sacrifice consumé par le feu à Yahweh.
\VS{14}Et leurs libations seront d'un demi-hin de vin pour chaque veau, d'un tiers de hin pour un bélier, et d'un quart de hin pour chaque agneau, c'est l'holocauste du commencement de chaque mois, selon tous les mois de l'année.
\VS{15}On sacrifiera aussi à Yahweh un jeune bouc en sacrifice d'expiation, outre l'holocauste perpétuel, et sa libation.
\VS{16}Au quatorzième jour du premier mois, ce sera la Pâque à Yahweh.
\VS{17}Et au quinzième jour du même mois sera un jour de fête. On mangera pendant sept jours des pains sans levain\FTNT{Ex. 12~; Lé. 23:5-6.}.
\VS{18}Au premier jour, il y aura une sainte convocation~: Vous ne ferez aucune œuvre servile.
\VS{19}Et vous offrirez un sacrifice consumé par le feu en holocauste à Yahweh~: Deux jeunes taureaux, un bélier, et sept agneaux d'un an, sans défaut.
\VS{20}Leur gâteau sera de fine farine pétrie à l'huile, vous en offrirez trois dixièmes pour chaque jeune taureau, et deux dixièmes pour un bélier~;
\VS{21}tu en offriras aussi un dixième pour chacun des sept agneaux,
\VS{22}et un bouc en sacrifice pour l'expiation, afin de faire propitiation pour vous.
\VS{23}Vous offrirez ces choses là, outre l'holocauste du matin, qui est l'holocauste perpétuel.
\VS{24}Vous offrirez ces choses-là chaque jour, pendant sept jours, comme l'aliment d'un sacrifice consumé par le feu, d'une bonne odeur à Yahweh. On offrira cela outre l'holocauste perpétuel, et sa libation.
\VS{25}Et au septième jour, vous aurez une sainte convocation~: Vous ne ferez aucune œuvre servile.
\VS{26}Et au jour des prémices, quand vous offrirez à Yahweh une offrande nouvelle de gâteau à votre fête des semaines, vous aurez une sainte convocation~: Vous ne ferez aucune œuvre servile.
\VS{27}Et vous offrirez en holocauste d'une bonne odeur à Yahweh, deux jeunes taureaux, un bélier, et sept agneaux d'un an.
\VS{28}Et leur gâteau sera de fine farine pétrie à l'huile, de trois dixièmes pour chaque jeune taureau, et de deux dixièmes pour le bélier,
\VS{29}et d'un dixième pour chacun des sept agneaux~;
\VS{30}et un jeune bouc, afin de faire propitiation pour vous.
\VS{31}Vous les offrirez, outre l'holocauste perpétuel et son offrande, lesquels seront sans défaut, avec leurs libations.
\Chap{29}
\TextTitle{Consignes relatives au temps des sacrifices - suite}
\VerseOne{}Et le premier jour du septième mois, vous aurez une sainte convocation~: Vous ne ferez aucune œuvre servile. Ce jour sera publié parmi vous au son des trompettes\FTNT{Lé. 23:24-25.}.
\VS{2}Et vous offrirez en holocauste de bonne odeur à Yahweh, un jeune taureau, un bélier, et sept agneaux d'un an, sans défaut.
\VS{3}Et leur gâteau sera de fine farine pétrie à l'huile, de trois dixièmes pour le jeune taureau, de deux dixièmes pour le bélier,
\VS{4}et un dixième pour chacun des sept agneaux.
\VS{5}Et un jeune bouc en sacrifice pour l'expiation, afin de faire propitiation pour vous,
\VS{6}outre l'holocauste du commencement du mois et son gâteau, et l'holocauste perpétuel et son gâteau, et leurs libations selon leur ordonnance. Ce sont des sacrifices consumés par le feu en bonne odeur à Yahweh.
\VS{7}Et au dixième jour de ce septième mois, vous aurez une sainte convocation, et vous affligerez vos âmes~: Vous ne ferez aucune œuvre\FTNT{Lé. 16:29-31~; Lé. 23:27.}.
\VS{8}Et vous offrirez en holocauste, de bonne odeur à Yahweh, un jeune taureau, un bélier, et sept agneaux d'un an, qui seront sans défaut.
\VS{9}Et leur gâteau sera de fine farine pétrie à l'huile, de trois dixièmes pour le taureau, et de deux dixièmes pour le bélier,
\VS{10}et d'un dixième pour chacun des sept agneaux.
\VS{11}Un jeune bouc aussi en sacrifice d'expiation, outre le sacrifice des expiations, l'holocauste perpétuel et son gâteau, avec leurs libations.
\VS{12}Et au quinzième jour du septième mois, vous aurez une sainte convocation~: Vous ne ferez aucune œuvre servile. Vous célébrerez une fête à Yahweh, pendant sept jours\FTNT{Lé. 23:34-43.}.
\VS{13}Et vous offrirez en holocauste un sacrifice consumé par le feu, d'une agréable odeur à Yahweh, treize jeunes taureaux, deux béliers, et quatorze agneaux d'un an, sans défaut.
\VS{14}Et leur gâteau sera de fine farine pétrie à l'huile, de trois dixièmes pour chacun des treize jeunes taureaux, de deux dixièmes pour chacun des deux béliers,
\VS{15}et d'un dixième pour chacun des quatorze agneaux.
\VS{16}Et un jeune bouc en sacrifice d'expiation, outre l'holocauste perpétuel, son gâteau, et sa libation.
\VS{17}Et au second jour, vous offrirez douze jeunes taureaux, deux béliers, et quatorze agneaux d'un an, sans défaut,
\VS{18}avec les gâteaux et les libations pour les jeunes taureaux, pour les béliers, et pour les agneaux, selon leur nombre, d'après les ordonnances.
\VS{19}Vous offrirez un jeune bouc en sacrifice d'expiation, outre l'holocauste perpétuel, et son offrande, avec leurs libations.
\VS{20}Et au troisième jour, vous offrirez onze taureaux, deux béliers, et quatorze agneaux d'un an, sans défaut~;
\VS{21}et les gâteaux et les libations pour les jeunes taureaux, les béliers et les agneaux, selon leur nombre, selon leur ordonnance.
\VS{22}Et un bouc en sacrifice d'expiation, outre l'holocauste continuel, son gâteau et sa libation.
\VS{23}Et au quatrième jour, vous offrirez dix jeunes taureaux, deux béliers, et quatorze agneaux d'un an, sans défaut,
\VS{24}les gâteaux et les libations pour les taureaux, les béliers, et les agneaux, selon leur nombre et leur ordonnance.
\VS{25}Et un jeune bouc en sacrifice d'expiation, outre l'holocauste perpétuel, son offrande, et sa libation.
\VS{26}Et au cinquième jour, vous offrirez neuf jeunes taureaux, deux béliers, et quatorze agneaux d'un an, sans défaut,
\VS{27}avec les gâteaux et les libations pour les taureaux, les béliers, et les agneaux, selon leur nombre et leur ordonnance.
\VS{28}Et un bouc en sacrifice d'expiation, outre l'holocauste continuel, son gâteau, et sa libation.
\VS{29}Et le sixième jour, vous offrirez huit jeunes taureaux, deux béliers et quatorze agneaux d'un an, sans défaut,
\VS{30}et les gâteaux, les libations pour les taureaux, les béliers, et les agneaux selon leur nombre leur ordonnance.
\VS{31}Et un bouc en sacrifice d'expiation, outre l'holocauste continuel, son offrande, et sa libation.
\VS{32}Et au septième jour, vous offrirez sept jeunes taureaux, deux béliers, et quatorze agneaux d'un an, sans défaut,
\VS{33}avec les gâteaux et les libations pour les jeunes taureaux, les béliers, et les agneaux, selon leur nombre et leur ordonnance.
\VS{34}Et un bouc en sacrifice d'expiation, outre l'holocauste continuel, son gâteau, et sa libation.
\VS{35}Et au huitième jour, vous aurez une assemblée solennelle~: Vous ne ferez aucune œuvre servile.
\VS{36}Et vous offrirez en holocauste un sacrifice consumé par le feu, d'une agréable odeur à Yahweh~: Un jeune taureau, un bélier, et sept agneaux d'un an, sans défaut,
\VS{37}avec les gâteaux et les libations pour le jeune taureau, le bélier, et les agneaux, selon leur nombre et leur ordonnance.
\VS{38}Et un bouc en sacrifice d'expiation, outre l'holocauste perpétuel, son offrande, et sa libation.
\VS{39}Vous offrirez ces choses à Yahweh dans vos fêtes solennelles, outre vos vœux, et vos offrandes volontaires, selon vos holocaustes, vos gâteaux, vos libations, et vos sacrifices d'offrande de paix.
\Chap{30}
\TextTitle{Les vœux}
\VerseOne{}Et Moïse parla aux enfants d'Israël selon toutes les choses que Yahweh lui avait ordonné.
\VS{2}Moïse parla aussi aux chefs des tribus des enfants d'Israël, en disant~: Voici ce que Yahweh ordonne.
\VS{3}Quand un homme fera un vœu à Yahweh, ou aura juré par serment, pour lier son âme par un vœu, il ne violera pas sa parole~; il fera selon toutes les choses qui sont sorties de sa bouche\FTNT{De. 23:21.}.
\VS{4}Mais quand une femme fera un vœu à Yahweh, et qu'elle se liera par un serment, dans sa jeunesse, étant encore dans la maison de son père,
\VS{5}et que son père aura entendu son vœu et le serment par lequel elle a lié son âme, si son père ne lui dit rien, tous ses vœux seront valables, et tout serment par lequel elle aura lié son âme sera valable~;
\VS{6}mais si son père la désapprouve le jour où il l'a entendue, aucun de ses vœux ou de ses serments par lesquels elle a lié son âme ne sera valable, et Yahweh lui pardonnera~; parce que son père l'a désapprouvée.
\VS{7}Et si elle a un mari, et qu'elle s'est engagée par quelque vœu ou par une parole échappée de ses lèvres par laquelle elle aura lié son âme,
\VS{8}et que son mari l'aura entendue, et que le jour même où il l'a entendue, il ne lui a rien dit, ses vœux alors seront valables, et ses serments par lesquels elle aura lié son âme seront valables~;
\VS{9}mais si son mari la désapprouve le jour où il l'a entendue, alors il annulera le vœu par lequel elle s'est engagée et la parole échappée de ses lèvres, par laquelle elle avait lié son âme~; et Yahweh lui pardonnera.
\VS{10}Mais le vœu de la veuve ou de la répudiée, tout ce par quoi elle aura lié son âme, sera valable pour elle.
\VS{11}Que si étant encore dans la maison de son mari elle a fait un vœu, ou si elle a lié son âme par serment,
\VS{12}et que son mari l'ait entendue, et ne lui en ait rien dit, et ne l'ait pas désapprouvée, alors tous ses vœux seront valables, et tout serment par lequel elle a lié son âme sera valable.
\VS{13}Mais si son mari les a entièrement annulés le jour où il les a entendus, alors rien de ce qui est sorti de ses lèvres, soit ses vœux, soit le serment par lequel elle a lié son âme ne seront valables~; parce que son mari les a annulés, et Yahweh lui pardonnera.
\VS{14}Son mari ratifiera ou son mari annulera tout vœu et toute obligation faite par serment, pour affliger l'âme.
\VS{15}Mais si son mari ne lui en a absolument rien dit, d'un jour à l'autre, il aura ratifié tous ses vœux ou toutes ses obligations dont elle était tenue~; il les aura, dis-je, ratifiés, parce qu'il ne lui en a rien dit le jour où il les a entendus.
\VS{16}Mais s'il les a expressément annulés après les avoir entendus, alors il portera l'iniquité de sa femme.
\VS{17}Telles sont les ordonnances que Yahweh ordonna à Moïse, entre un mari et sa femme~; entre un père et sa fille, étant encore dans la maison de son père, dans sa jeunesse.
\Chap{31}
\TextTitle{Jugements sur Madian\FTNTT{No. 25:6-18.}}
\VerseOne{}Yahweh parla à Moïse, en disant~:
\VS{2}Fais la vengeance des enfants d'Israël sur les Madianites, puis tu seras recueilli auprès de ton peuple.
\VS{3}Moïse donc parla au peuple, en disant~: Que quelques-uns d'entre vous s'équipent pour aller à la guerre, et qu'ils aillent contre Madian, pour exécuter la vengeance de Yahweh sur Madian.
\VS{4}Vous enverrez à la guerre mille hommes de chaque tribu, de toutes les tribus d'Israël.
\VS{5}On donna d'entre les milliers d'Israël mille hommes de chaque tribu, qui furent douze mille hommes équipés pour la guerre.
\VS{6}Moïse les envoya à la guerre, savoir mille de chaque tribu, et avec eux Phinées, fils d'Eléazar, le prêtre, qui portait les instruments sacrés et les trompettes retentissantes.
\VS{7}Ils s'avancèrent donc contre Madian, comme Yahweh l'avait ordonné à Moïse, et ils en tuèrent tous les mâles.
\VS{8}Ils tuèrent aussi les rois de Madian, outre les autres qui y furent tués, Evi, Rékem, Tsur, Hur, et Réba, cinq rois de Madian~; ils firent aussi passer au fil de l'épée Balaam, fils de Beor\FTNT{Jos. 13:21-22.}.
\VS{9}Et les fils d'Israël emmenèrent prisonniers les femmes de Madian, avec leurs petits enfants, et pillèrent tout leur gros et menu bétail, et tous leurs biens.
\VS{10}Ils brûlèrent par le feu toutes leurs villes, leurs demeures, et tous leurs châteaux.
\VS{11}Ils prirent tout le butin et tout le pillage, tant des hommes que du bétail\FTNT{De. 20:14.}~;
\VS{12}puis ils amenèrent les captifs, le pillage, et le butin, à Moïse, à Eléazar le prêtre, et à l'assemblée des enfants d'Israël, au camp, dans les plaines de Moab, qui sont près du Jourdain, vis-à-vis de Jéricho.
\VS{13}Moïse, Eléazar, le prêtre, et tous les princes de l'assemblée sortirent au-devant d'eux, hors du camp.
\VS{14}Et Moïse se mit en grande colère contre les officiers de l'armée, les chefs des milliers, et les chefs des centaines, qui revenaient de cet exploit de guerre.
\VS{15}Et Moïse leur dit~: N'avez-vous pas gardé en vie toutes les femmes~?
\VS{16}Voici ce sont elles qui, à la parole de Balaam, ont donné l'occasion aux fils d'Israël de pécher contre Yahweh dans l'affaire de Peor~; ce qui attira la plaie sur l'assemblée de Yahweh\FTNT{2 Pi. 2:15~; Ap. 2:14.}.
\VS{17}Or maintenant, tuez tous les mâles d'entre les petits enfants, et tuez toute femme qui a connu un homme en couchant avec lui\FTNT{Jg. 21:11.}~;
\VS{18}mais vous garderez en vie toutes les jeunes filles qui n'ont point connu la couche d'un homme.
\VS{19}Au reste, demeurez sept jours hors du camp~; quiconque aura tué quelqu'un, et quiconque aura touché quelqu'un qui aura été tué, se purifiera le troisième et le septième jour, tant vous que vos prisonniers.
\VS{20}Vous purifierez aussi tous vos vêtements, et tout ce qui sera fait de peau, et tout ouvrage de poil de chèvre, et toute vaisselle de bois.
\VS{21}Eléazar, le prêtre, dit aux hommes de guerre qui étaient allés au combat~: Voici l'ordonnance et la loi que Yahweh a ordonné à Moïse.
\VS{22}En général l'or, l'argent, l'airain, le fer, l'étain, le plomb~;
\VS{23}tout ce qui peut passer par le feu, vous le ferez passer par le feu pour le rendre pur. Seulement on purifiera avec l'eau de purification toutes les choses qui ne peuvent aller au feu, vous les ferez passer dans l'eau.
\VS{24}Vous laverez aussi vos vêtements le septième jour, ensuite vous serez purs~; puis vous entrerez au camp.
\TextTitle{Partage du butin}
\VS{25}Et Yahweh parla à Moïse, en disant~:
\VS{26}Fais le compte du butin et de tout ce qu'on a emmené, tant des personnes que des bêtes, toi et Eléazar, le prêtre, et les chefs des pères de l'assemblée.
\VS{27}Et partage par moitié le butin entre les combattants qui sont allés à la guerre et toute l'assemblée\FTNT{1 S. 30:24.}.
\VS{28}Tu prélèveras aussi pour Yahweh un tribut sur les hommes de guerre qui sont allés à la bataille, savoir un sur cinq cents, tant des personnes, que des bœufs, des ânes et des brebis.
\VS{29}On le prendra sur leur moitié, et tu le donneras à Eléazar, le prêtre, en offrande présentée par élévation à Yahweh.
\VS{30}Et sur la moitié qui appartient aux enfants d'Israël, tu prendras un sur cinquante, tant des personnes que des bœufs, des ânes, des brebis et de tous les autres animaux, et tu le donneras aux Lévites qui ont la charge de garder le tabernacle de Yahweh.
\VS{31}Moïse et Eléazar, le prêtre, firent comme Yahweh l'avait ordonné à Moïse.
\VS{32}Or le butin qui était resté du pillage du peuple qui était allé à la guerre, était de six cent soixante-quinze mille brebis~;
\VS{33}de soixante-douze mille bœufs~;
\VS{34}de soixante et un mille ânes,
\VS{35}quant aux femmes qui n'avaient point connu la couche d' un homme, elles étaient en tout trente-deux mille âmes.
\VS{36}Et la moitié du butin, à savoir la part de ceux qui étaient allés à la guerre, montait à trois cent trente-sept mille cinq cents brebis~;
\VS{37}dont le tribut pour Yahweh, quant aux brebis, était de six cent soixante-quinze.
\VS{38}Trente-six mille bœufs~; dont le tribut pour Yahweh, quant aux bœufs, était de soixante-douze bœufs,
\VS{39}trente mille cinq cents ânes~; dont le tribut pour Yahweh, quant aux ânes, était de soixante et un ânes~;
\VS{40}et de seize mille personnes, dont le tribut pour Yahweh était de trente-deux personnes.
\VS{41}Et Moïse donna à Eléazar, le prêtre, le tribut de l'offrande présentée par élévation à Yahweh, comme Yahweh le lui avait ordonné.
\VS{42}Et de l'autre moitié qui appartenait aux enfants d'Israël, que Moïse avait tiré des hommes qui étaient allés à la guerre~;
\VS{43}or de cette moitié qui fut pour l'assemblée, et qui montait à trois cent trente-sept mille cinq cents brebis,
\VS{44}trente-six mille bœufs,
\VS{45}trente mille cinq cents ânes,
\VS{46}et à seize mille personnes~;
\VS{47}de cette moitié, dis-je, qui appartenait aux enfants d'Israël, Moïse prit un sur cinquante, tant des personnes que des bêtes, et les donna aux Lévites qui avaient la charge de garder le tabernacle de Yahweh, comme Yahweh le lui avait ordonné.
\VS{48}Les commandants des milliers de l'armée, tant les chefs des milliers que les chefs des centaines, s'approchèrent de Moïse,
\VS{49}et lui dirent~: Tes serviteurs ont fait le compte des hommes de guerre qui étaient sous nos ordres, il ne manque pas un homme d'entre nous.
\VS{50}C'est pourquoi, nous offrons l'offrande de Yahweh, chacun les objets que nous avons trouvés~: Des joyaux d'or, des chaînes de cheville, des bracelets, des anneaux, des pendants d'oreilles et des colliers, afin de faire propitiation pour nos personnes devant Yahweh.
\VS{51}Moïse et Eléazar, le prêtre, reçurent d'eux cet or, tous ces objets travaillés.
\VS{52}Et tout l'or de l'offrande présentée par élévation à Yahweh, de la part des chefs de milliers et des chefs de centaines, montait à seize mille sept cent cinquante sicles.
\VS{53}Or les hommes de guerre gardèrent chacun pour soi ce qu'ils avaient pillé.
\VS{54}Moïse donc et Eléazar, le prêtre, prirent l'or des chefs des milliers et des chefs de centaines, et l'apportèrent à la tente d'assignation, comme souvenir pour les enfants d'Israël, devant Yahweh.
\Chap{32}
\TextTitle{Ruben et Gad en Galaad}
\VerseOne{}Les fils de Ruben et les fils de Gad avaient beaucoup de bétail, en très grande quantité, et ils virent que le pays de Jaezer et le pays de Galaad étaient un lieu propre pour du bétail.
\VS{2}Ainsi les fils de Gad et les fils de Ruben vinrent, et parlèrent à Moïse et à Eléazar, le prêtre, et aux princes de l'assemblée, en disant~:
\VS{3}Atharoth, Dibon, Jaezer, Nimra, Hesbon, et Elealé, Sebam, Nebo, et Beon,
\VS{4}ce pays-là que Yahweh a frappé devant l'assemblée d'Israël, est un pays propre pour le bétail, et tes serviteurs ont des troupeaux.
\VS{5}Ils dirent donc: Si nous avons trouvé grâce à tes yeux, que ce pays soit donné en possession à tes serviteurs~; et ne nous fais point passer le Jourdain.
\VS{6}Mais Moïse répondit aux fils de Gad, et aux fils de Ruben~: Vos frères iront-ils à la guerre, et vous, demeurerez-vous ici~?
\VS{7}Pourquoi voulez-vous décourager les enfants d'Israël de passer dans le pays que Yahweh leur a donné~?
\VS{8}C'est ainsi que firent vos pères quand je les envoyai de Kadès-Barnéa pour examiner le pays.
\VS{9}Car ils montèrent jusqu'à la vallée d'Eschcol, virent le pays, puis découragèrent les enfants d'Israël, afin qu'ils n'entrent point dans le pays que Yahweh leur avait donné.
\VS{10}C'est pourquoi la colère de Yahweh s'enflamma ce jour-là, et il jura en disant~:
\VS{11}Les hommes qui sont montés hors d'Egypte, depuis l'âge de vingt ans et au-dessus, ne verront point le pays que j'ai juré de donner à Abraham, Isaac, et à Jacob~; car ils n'ont point persévéré à me suivre\FTNT{De. 1:35.},
\VS{12}excepté Caleb, fils de Jephunné, le Kénizien, et Josué, fils de Nun, car ils ont persévéré à suivre Yahweh.
\VS{13}Ainsi la colère de Yahweh s'enflamma contre Israël et il les fit errer dans le désert pendant quarante ans, jusqu'à ce que toute la génération qui avait fait le mal aux yeux de Yahweh, ait été consumée.
\VS{14}Et voici, vous vous êtes levés à la place de vos pères, comme une race d'hommes pécheurs, pour augmenter encore l'ardeur de la colère de Yahweh contre Israël.
\VS{15}Si vous vous détournez de lui, il continuera encore à vous laisser au désert, et vous ferez détruire tout ce peuple.
\VS{16}Mais ils s'approchèrent de lui et lui dirent~: Nous bâtirons ici des cloisons pour nos troupeaux, et des villes pour nos petits enfants~;
\VS{17}et nous nous équiperons pour marcher promptement devant les enfants d'Israël, jusqu'à ce que nous les ayons introduits en leur lieu~; mais nos petits enfants demeureront dans les villes fortes, à cause des habitants du pays.
\VS{18}Nous ne retournerons point dans nos maisons avant que chacun des enfants d'Israël n'ait pris possession de son héritage~;
\VS{19}et nous ne posséderons rien en héritage avec eux au-delà du Jourdain, ni plus avant~; parce que nous aurons notre héritage de ce côté-ci du Jourdain, à l'orient.
\VS{20}Et Moïse leur dit~: Si vous faites cela, si vous vous équipez devant Yahweh pour aller à la guerre,
\VS{21}si chacun de vous étant équipé passe le Jourdain devant Yahweh, jusqu'à ce qu'il ait chassé ses ennemis loin de devant lui,
\VS{22}et que le pays soit assujetti devant Yahweh, et qu'ensuite vous vous en retournez, alors vous serez innocents envers Yahweh, et envers Israël~; et ce pays-ci vous appartiendra pour le posséder devant Yahweh.
\VS{23}Mais si vous ne faites point cela, vous péchez contre Yahweh~; et sachez que votre péché vous atteindra.
\VS{24}Bâtissez donc des villes pour vos petits enfants, et des cloisons pour vos troupeaux, et faites ce que vous avez dit.
\VS{25}Alors les fils de Gad et les fils de Ruben parlèrent à Moïse, en disant~: Tes serviteurs feront ce que mon seigneur a ordonné.
\VS{26}Nos petits enfants, nos femmes, nos troupeaux, et tout notre bétail demeureront ici dans les villes de Galaad~;
\VS{27}et tes serviteurs passeront chacun armés pour aller à la guerre devant Yahweh, prêts à combattre, comme mon seigneur a parlé.
\VS{28}Alors Moïse donna des ordres à leur sujet à Eléazar, le prêtre, à Josué, fils de Nun, et aux chefs des pères des tribus des fils d'Israël.
\VS{29}Il leur dit~: Si les fils de Gad et les fils de Ruben passent avec vous le Jourdain tous armés, prêts à combattre devant Yahweh, et que le pays vous soit assujetti, vous leur donnerez le pays de Galaad en possession.
\VS{30}Mais s'ils ne marchent point en armes avec vous, qu'ils s'établissent au milieu de vous dans le pays de Canaan.
\VS{31}Les fils de Gad et les fils de Ruben répondirent, en disant~: Nous ferons ce que Yahweh a dit à tes serviteurs.
\VS{32}Nous passerons en armes devant Yahweh au pays de Canaan, afin que nous possédions pour notre héritage ce qui est de ce côté-ci du Jourdain.
\VS{33}Ainsi Moïse donna aux fils de Gad et aux fils de Ruben, et à la demi-tribu de Manassé, fils de Joseph, le royaume de Sihon, roi des Amoréens~; et le royaume de Og, roi de Basan, le pays avec ses villes, selon les bornes des villes du pays tout autour.
\VS{34}Alors les fils de Gad rebâtirent Dibon, Atharoth, Aroër,
\VS{35}Athroth-Schophan, Jaezer, Jogbeha,
\VS{36}Beth-Nimra et Beth-Haran, villes fortifiées. Ils firent aussi des cloisons pour les troupeaux.
\VS{37}Et les fils de Ruben rebâtirent Hesbon, Elealé, Kirjathaim,
\VS{38}Nébo, Baal-Meon, et Sibma, dont ils changèrent les noms, et ils donnèrent des noms aux villes qu'ils rebâtirent.
\VS{39}Or les fils de Makir, fils de Manassé, allèrent en Galaad, le prirent et dépossédèrent les Amoréens qui y étaient.
\VS{40}Moïse donc donna Galaad à Makir, fils de Manassé, qui y habita\FTNT{De. 3:15.}.
\VS{41}Jaïr, fils de Manassé, se mit en marche, prit leurs villages, et les appela villages de Jaïr\FTNT{De. 3:14~; 1 Ch. 2:22.}.
\VS{42}Et Nobach se mit en marche, prit Kenath avec les villes de son ressort, et l'appela Nobach d'après son nom.
\Chap{33}
\TextTitle{Les stations de l'Egypte jusqu'au Jourdain}
\VerseOne{}Ce sont ici les étapes des enfants d'Israël, qui sortirent du pays d'Egypte, selon leurs armées, sous la main de Moïse et d'Aaron.
\VS{2}Moïse écrivit leurs départs, et leurs étapes, d'après l'ordre de Yahweh~! Et voici leurs étapes selon leurs départs.
\VS{3}Les enfants d'Israël donc partirent de Ramsès le quinzième jour du premier mois, dès le lendemain de la Pâque, et ils sortirent à main levée, à la vue de tous les Egyptiens\FTNT{Ex. 14:8.}.
\VS{4}Et les Egyptiens ensevelissaient ceux que Yahweh avait frappés parmi eux, à savoir tous les premiers-nés~; même Yahweh exerçait aussi ses jugements contre leurs dieux\FTNT{Ex. 12:12~; Ex. 18:11.}.
\VS{5}Et les enfants d'Israël partirent de Ramsès, et campèrent à Succoth\FTNT{Ex. 12:37.}.
\VS{6}Et ils partirent de Succoth et campèrent à Etham, qui est au bout du désert\FTNT{Ex. 13:20.}.
\VS{7}Et ils partirent d'Etham et se détournèrent vers Pi-Hahiroth, qui est vis-à-vis de Baal-Tsephon, et campèrent devant Migdol\FTNT{Ex. 14:2.}.
\VS{8}Et ils partirent de devant Pi-Hahiroth et passèrent au travers de la mer vers le désert, et firent trois journées de marche par le désert d'Etham et campèrent à Mara.
\VS{9}Puis ils partirent de Mara et vinrent à Elim où il y avait douze fontaines d'eaux et soixante-dix palmiers, et ils y campèrent\FTNT{Ex. 15:27.}.
\VS{10}Et ils partirent d'Elim et campèrent près de la Mer Rouge.
\VS{11}Puis ils partirent de la Mer Rouge et campèrent au désert de Sin\FTNT{Ex. 16:1.}.
\VS{12}Ils partirent du désert de Sin et campèrent à Dophka.
\VS{13}Puis ils partirent de Dophka et campèrent à Alusch.
\VS{14}Et ils partirent d'Alusch et campèrent à Rephidim où il n'y avait point d'eau à boire pour le peuple\FTNT{Ex. 17:1.}.
\VS{15}Puis ils partirent de Rephidim et campèrent dans le désert de Sinaï\FTNT{Ex. 17:1.}.
\VS{16}Ils partirent du désert de Sinaï et campèrent à Kibroth-Hattaava.
\VS{17}Et ils partirent de Kibroth-Hattaava et campèrent à Hatséroth.
\VS{18}Puis ils partirent de Hatséroth et campèrent à Rithma.
\VS{19}Et ils partirent de Rithma et campèrent à Rimmon-Pérets.
\VS{20}Ils partirent de Rimmon-Pérets et campèrent à Libna.
\VS{21}Et ils partirent de Libna et campèrent à Rissa.
\VS{22}Puis ils partirent de Rissa et campèrent vers Kehélatha.
\VS{23}Et ils partirent de Kehélatha et campèrent à la montagne de Schapher.
\VS{24}Ils partirent de la montagne de Schapher et campèrent à Harada.
\VS{25}Et ils partirent de Harada et campèrent à Makhéloth.
\VS{26}Puis ils partirent de Makhéloth et campèrent à Tahath.
\VS{27}Ils partirent de Tahath et campèrent à Tarach.
\VS{28}Et ils partirent de Tarach et campèrent à Mithka.
\VS{29}Puis ils partirent de Mithka et campèrent à Haschmona.
\VS{30}Ils partirent de Haschmona et campèrent à Moséroth.
\VS{31}Et ils partirent de Moséroth et campèrent à Bené-Jaakan.
\VS{32}Ils partirent de Bené-Jaakan et campèrent à Hor-Guidgad.
\VS{33}Puis ils partirent de Hor-Guidgad et campèrent vers Jothbatha.
\VS{34}Ils partirent de Jothbatha et campèrent à Abrona.
\VS{35}Et ils partirent d'Abrona et campèrent à Etsjon-Guéber.
\VS{36}Ils partirent d'Etsjon-Guéber et campèrent dans le désert de Tsin, qui est Kadès.
\VS{37}Puis ils partirent de Kadès et campèrent à la montagne de Hor, qui est au bout du pays d'Edom.
\VS{38}Et Aaron le prêtre, monta sur la montagne de Hor, suivant l'ordre de Yahweh, et mourut là, la quarantième année après que les enfants d'Israël furent sortis du pays d'Egypte, le premier jour du cinquième mois.
\VS{39}Et Aaron était âgé de cent vingt-trois ans quand il mourut sur la montagne de Hor.
\VS{40}Alors le Cananéen, roi d'Arad, qui habitait vers le midi au pays de Canaan, apprit que les enfants d'Israël venaient.
\VS{41}Et ils partirent de la montagne de Hor et campèrent à Tsalmona.
\VS{42}Puis ils partirent de Tsalmona et campèrent à Punon.
\VS{43}Et Ils partirent de Punon et campèrent à Oboth.
\VS{44}Ils partirent d'Oboth et campèrent à Ijjé-Abarim, sur les frontières de Moab.
\VS{45}Puis ils partirent d'Ijjé-Abarim et campèrent à Dibon-Gad.
\VS{46}Et ils partirent de Dibon-Gad, et campèrent à Almon-Diblathaïm.
\VS{47}Ils partirent d'Almon-Diblathaïm et campèrent aux montagnes de Abarim devant Nébo.
\VS{48}Et ils partirent des montagnes d'Abarim et campèrent aux plaines de Moab, près du Jourdain de Jéricho.
\VS{49}Pui ils campèrent près du Jourdain, depuis Beth-Jeschimoth jusqu'à Abel-Sittim, dans les plaines de Moab.
\TextTitle{Consignes pour les possessions attribuées à Israël}
\VS{50}Et Yahweh parla à Moïse dans les plaines de Moab, près du Jourdain de Jéricho, en disant~:
\VS{51}Parle aux enfants d'Israël, et dis-leur~: Puisque vous allez passer le Jourdain pour entrer au pays de Canaan,
\VS{52}vous chasserez de devant vous tous les habitants du pays, vous détruirez toutes leurs peintures, et vous ruinerez toutes leurs images de fonte, et vous démolirez tous leurs hauts lieux\FTNT{De. 7:5~; De. 12:2.}.
\VS{53}Et vous prendrez possession du pays, et vous y habiterez. Car je vous ai donné le pays pour le posséder.
\VS{54}Or vous recevrez le pays en héritage par le sort, selon vos familles. A ceux qui sont en plus grand nombre, vous donnerez plus d'héritage, et à ceux qui sont en plus petit nombre, vous donnerez moins d'héritage. Chacun aura selon ce qui lui sera échu par le sort, et vous hériterez selon les tribus de vos pères.
\VS{55}Mais si vous ne chassez pas de devant vous les habitants du pays, il arrivera que ceux d'entre eux que vous aurez laissés comme reste, seront comme des épines à vos yeux, et comme des pointes à vos côtés, et ils vous serreront de près dans le pays auquel vous habiterez\FTNT{Jos. 23:13.}.
\VS{56}Et il arrivera que je vous ferai tout comme j'ai eu dessein de leur faire.
\Chap{34}
\TextTitle{Consignes sur les limites de chaque tribu}
\VerseOne{}Yahweh parla aussi à Moïse, en disant~:
\VS{2}Donne l'ordre aux enfants d'Israël, et dis-leur~: Parce que vous allez entrer au pays de Canaan, ce pays deviendra votre héritage, le pays de Canaan selon ses limites.
\VS{3}Votre frontière du côté du sud sera depuis le désert de Tsin, le long d'Edom, et votre frontière du côté du sud commencera au bout de la mer salée, vers l'orient~;
\VS{4}et cette frontière tournera du sud vers la montée d'Akrabbim, et passera jusqu'à Tsin~; et elle aboutira du côté du sud de Kadès-Barnéa~; et sortira aussi par Hatsar-Addar, et passera jusqu'à Atsmon.
\VS{5}Et cette frontière tournera depuis Atsmon jusqu'au torrent d'Egypte~; et elle aboutira à la mer.
\VS{6}Quant à la frontière d'occident, vous aurez la grande mer et ses limites~; ce sera votre frontière occidentale.
\VS{7}Et ce sera ici votre frontière au nord~; depuis la grande mer, vous marquerez pour vos limites la montagne de Hor~;
\VS{8}et depuis la montagne de Hor, vous marquerez pour vos limites l'entrée de Hamath, et cette frontière aboutira vers Tsedad~;
\VS{9}cette frontière passera jusqu'à Ziphron, et elle aboutira à Hatsar-Enan~; telle sera votre frontière au nord.
\VS{10}Puis vous marquerez pour vos limites vers l'orient de Hatsar-Enan à Schepham.
\VS{11}Et cette frontière descendra de Schepham à Ribla, du côté de l'orient d'Aïn~; et cette frontière descendra et s'étendra le long de la Mer de Kinnéreth vers l'orient.
\VS{12}Cette frontière descendra au Jourdain pour aboutir à la Mer Salée~; tel sera le pays que vous aurez avec ses limites tout autour.
\VS{13}Et Moïse donna l'ordre aux enfants d'Israël, en disant~: C'est là le pays que vous hériterez par le sort, et que Yahweh a ordonné de donner à neuf tribus, et à la demi-tribu.
\VS{14}Car la tribu des fils de Ruben selon les familles de leurs pères, et la tribu des fils de Gad, selon les familles de leurs pères, ont pris leur héritage~; et la demi-tribu de Manassé a pris aussi son héritage.
\VS{15}Deux tribus, dis-je, et la demi-tribu ont pris leur héritage de l'autre côté du Jourdain, vis-à-vis de Jéricho, du côté du levant.
\VS{16}Et Yahweh parla à Moïse, en disant~:
\VS{17}Ce sont ici les noms des hommes qui vous partageront le pays~: Eléazar le prêtre, et Josué fils de Nun.
\VS{18}Vous prendrez aussi un prince de chaque tribu pour faire le partage du pays.
\VS{19}Et voici les noms de ces hommes. Pour la tribu de Juda~: Caleb, fils de Jephunné~;
\VS{20}pour la tribu des fils de Siméon~: Samuel, fils d'Ammihud~;
\VS{21}pour la tribu de Benjamin~: Elidad, fils de Kislon~;
\VS{22}pour la tribu des fils de Dan~: Celui qui en est le chef, Buki, fils de Jogli~;
\VS{23}pour les fils de Joseph, pour la tribu des fils de Manassé~: Celui qui en est le chef, Hanniel, fils d'Ephod~;
\VS{24}et pour la tribu des fils d'Ephraïm~: Celui qui en est le chef, Kemuel, fils de Schiphtan~;
\VS{25}pour la tribu des fils de Zabulon~: Celui qui en est le chef, Elitsaphan, fils de Parnac~;
\VS{26}pour la tribu des fils d'Issacar~: Celui qui en est le chef, Paltiel, fils d'Azzan~;
\VS{27}pour la tribu des fils d'Aser~: Celui qui en est le chef, Ahihud, fils de Schelomi~;
\VS{28}pour la tribu des fils de Nephthali~: Celui qui en est le chef, Pedahel, fils d'Ammihud.
\VS{29}Ce sont là, ceux à qui Yahweh donna l'ordre de partager l'héritage aux enfants d'Israël dans le pays de Canaan.
\Chap{35}
\TextTitle{Quarante-huit villes pour les Lévites dont six villes de refuge}
\VerseOne{}Yahweh parla à Moïse dans les plaines de Moab, près du Jourdain, vis-à-vis de Jéricho, en disant~:
\VS{2}Donne l'ordre aux enfants d'Israël qu'ils donnent aux Lévites, sur l'héritage qu'ils posséderont, des villes pour y habiter. Vous leur donnerez aussi les faubourgs qui sont autour de ces villes\FTNT{Jos. 21:2.}.
\VS{3}Ils auront donc les villes pour y habiter~; et les faubourgs de ces villes seront pour leurs bétails, pour leurs biens, et pour tous leurs animaux.
\VS{4}Les faubourgs des villes que vous donnerez aux Lévites, seront de mille coudées tout autour depuis la muraille de la ville en dehors.
\VS{5}Et vous mesurerez depuis le dehors de la ville du côté de l'orient, deux mille coudées~; et du côté du sud, deux mille coudées~; et du côté de l'occident, deux mille coudées~; et du côté du nord, deux mille coudées~; et la ville sera au milieu~; tels seront les faubourgs de leurs villes.
\VS{6}Et des villes que vous donnerez aux Lévites, il y aura six villes de refuge que vous donnerez pour que le meurtrier s'y enfuie, et outre celles-là, vous leur donnerez quarante-deux villes.
\VS{7}Toutes les villes que vous donnerez aux Lévites seront quarante-huit villes, elles et leurs faubourgs.
\VS{8}Et quant aux villes que vous leur donnerez sur la possession des enfants d'Israël, de ceux qui en auront plus vous en prendrez plus, et de ceux qui en auront moins vous en prendrez moins~; chacun donnera de ses villes aux Lévites, en proportion de l'héritage qu'il possédera.
\VS{9}Puis Yahweh parla à Moïse, en disant~:
\VS{10}Parle aux enfants d'Israël, et dis-leur~: Quand vous aurez passé le Jourdain, pour entrer au pays de Canaan~;
\VS{11}établissez-vous des villes qui vous soient des villes de refuge, afin que le meurtrier qui aura frappé à mort quelqu'un involontairement, s'y enfuie\FTNT{Jos. 20:2-3~; Ex. 21:13.}.
\VS{12}Et ces villes seront pour vous des villes de refuge contre le vengeur, afin que le meurtrier ne meure pas, jusqu'à ce qu'il ait comparu en jugement devant l'assemblée.
\VS{13}De ces villes que vous donnerez, il y en aura six de refuge pour vous.
\VS{14}Vous donnerez trois de ces villes au-delà du Jourdain, et les trois autres dans le pays de Canaan, qui seront des villes de refuge\FTNT{De. 19:2~; De. 4:41-42.}.
\VS{15}Ces six villes serviront de refuge aux enfants d'Israël, à l'étranger et à celui qui séjourne au milieu de vous, afin que quiconque aura frappé à mort quelqu'un involontairement, s'y enfuie.
\VS{16}Mais si un homme en frappe un autre avec un instrument de fer, et qu'il en meure, il est meurtrier~; on punira de mort le meurtrier.
\VS{17}Et s'il le frappe avec une pierre qu'il tenait à la main, dont on puisse mourir, et qu'il en meure, c'est un meurtrier~; le meurtrier sera puni de mort.
\VS{18}De même s'il le frappe d'un instrument de bois qu'il tenait à la main, dont on puisse mourir, et qu'il en meure, il est un meurtrier~; on punira de mort le meurtrier.
\VS{19}Et le vengeur du sang fera mourir le meurtrier quand il le rencontrera, il pourra le faire mourir.
\VS{20}Et s'il le pousse par haine, ou s'il jette quelque chose sur lui avec préméditation, et qu'il en meure~;
\VS{21}ou si par inimitié il le frappe de sa main, et qu'il en meure, on punira de mort celui qui l'a frappé, car il est meurtrier~; le vengeur du sang pourra le faire mourir quand il le rencontrera\FTNT{De. 19:11-12.}.
\VS{22}Mais s'il le pousse subitement, sans inimitié, ou s'il jette quelque chose sur lui, sans préméditation,
\VS{23}ou s'il fait tomber sur lui quelque pierre sans l'avoir vu, et qu'il en meure, n'étant pas son ennemi et ne lui cherchant pas du mal,
\VS{24}alors l'assemblée jugera entre celui qui a frappé et le vengeur du sang, selon ces ordonnances~;
\VS{25}l'assemblée délivrera le meurtrier de la main du vengeur de sang, et le fera retourner dans la ville de refuge où il s'était enfui. Il y demeurera jusqu'à la mort du grand-prêtre, qui aura été oint de la sainte huile.
\VS{26}Mais si le meurtrier sort de quelque manière que ce soit hors des bornes de la ville de son refuge, où il s'est enfui,
\VS{27}et si le vengeur du sang le rencontre hors des bornes de la ville de son refuge, et qu'il tue le meurtrier, il ne sera point coupable de meurtre.
\VS{28}Car il doit demeurer dans la ville de son refuge jusqu'à la mort du grand-prêtre~; et après la mort du grand-prêtre, le meurtrier pourra retourner dans sa possession.
\VS{29}Et ces choses-ci seront des ordonnances de jugement pour vous et pour vos générations, dans toutes vos demeures.
\VS{30}Celui qui fera mourir le meurtrier, le fera mourir sur la parole de deux témoins~; mais un seul témoin ne sera point reçu en témoignage contre quelqu'un, pour le faire mourir\FTNT{De. 17:6~; De. 19:15.}.
\VS{31}Et vous ne prendrez point de rançon pour la vie du meurtrier, qui est coupable et digne de mort~; mais il doit être puni de mort.
\VS{32}Vous ne prendrez point de rançon pour le laisser s'enfuir de sa ville de refuge, pour qu'il retourne habiter dans le pays, jusqu'à la mort du prêtre.
\VS{33}Et vous ne souillerez point le pays où vous serez, car le sang souille le pays~; et il ne se fera point de propitiation pour le pays, du sang qui y sera répandu que par le sang de celui qui l'aura répandu.
\VS{34}Vous ne souillerez donc point le pays où vous allez demeurer, et au milieu duquel j'habiterai~; car je suis Yahweh qui habite au milieu des enfants d'Israël.
\Chap{36}
\TextTitle{Loi sur les héritages\FTNTT{No. 27:1-11.}}
\VerseOne{}Or les chefs des pères de la famille des fils de Galaad, fils de Makir, fils de Manassé, d'entre les familles des fils de Joseph, s'approchèrent et parlèrent devant Moïse, et devant les princes, les chefs des pères des enfants d'Israël,
\VS{2}et ils dirent~: Yahweh a donné l'ordre à mon seigneur de donner aux enfants d'Israël le pays en héritage par le sort~; et mon seigneur a reçu l'ordre de Yahweh de donner l'héritage de Tselophchad, notre frère, à ses filles.
\VS{3}Si elles se marient à l'un des fils des autres tribus d'Israël, leur héritage sera retranché de l'héritage de nos pères et sera ajouté à l'héritage de la tribu de laquelle elles seront~; ainsi sera diminué l'héritage qui nous est échu par le sort.
\VS{4}Même quand viendra le jubilé pour les enfants d'Israël, on ajoutera leur héritage à l'héritage de la tribu à laquelle elles appartiendront, ainsi leur héritage sera retranché de l'héritage de la tribu de nos pères\FTNT{Lé. 25:10-13.}.
\VS{5}Et Moïse ordonna aux enfants d'Israël, suivant l'ordre de la bouche de Yahweh, en disant~: Ce que la tribu des fils de Joseph dit est juste.
\VS{6}C'est ici ce que Yahweh ordonne au sujet des filles de Tselophchad~: Elles se marieront à qui bon leur semblera, toutefois elles se marieront dans l'une des familles de la tribu de leurs pères.
\VS{7}Ainsi l'héritage ne sera point transporté entre les enfants d'Israël de tribu en tribu~; car chacun des enfants d'Israël se tiendra à l'héritage de la tribu de ses pères.
\VS{8}Et toute fille, qui possédera un héritage d'entre les tribus des enfants d'Israël, se mariera à quelqu'un de la famille de la tribu de son père, afin que chacun des enfants d'Israël possède l'héritage de ses pères.
\VS{9}L'héritage donc ne sera point transporté d'une tribu à une autre, mais chacune des tribus des enfants d'Israël se tiendra à son héritage.
\VS{10}Les filles de Tselophchad firent comme Yahweh avait donné à Moïse.
\VS{11}Machla, Thirtsa, Hogla, Milca, et Noa, filles de Tselophchad, se marièrent aux fils de leurs oncles.
\VS{12}Ainsi elles se marièrent à ceux qui étaient des familles des fils de Manassé, fils de Joseph~; et leur héritage demeura dans la tribu de la famille de leur père.
\VS{13}Ce sont là les ordonnances et les jugements que Yahweh ordonna par Moïse aux enfants d'Israël, dans les plaines de Moab, près du Jourdain, vis-à-vis de Jéricho.
\PPE{}
\end{multicols}

%\clearpage\ShortTitle{De.}\BookTitle{Deutéronome}\BFont
\noindent\hrulefill
{\footnotesize
\textit{
\bigskip
{\centering{}
\\Auteur~: Probablement Moïse
\\(Heb.~: Devarim)
\\Signification~: Paroles
\\Thème~: Rappel de la loi
\\Date de rédaction~: 1450-1410 av. J.-C.\\}
}
\textit{
\\Ce livre est un rappel de la loi de Yahweh. Après quarante années d'errance dans le désert, Moïse s'adresse à la nouvelle génération par des discours et des exhortations, depuis les plaines de Moab. Au travers de son serviteur, Dieu rappelle ainsi la loi donnée sur le mont Sinaï, les expériences vécues par la génération passée et par conséquent, l'importance de la soumission à Dieu. De leur obéissance dépendraient les bénédictions ou les malédictions contenues dans ce livre.\bigskip
}
}
\par\nobreak\noindent\hrulefill
\begin{multicols}{2}
\Chap{1}
\TextTitle{Rappel de l'infidélité d'Israël\FTNTT{No. 14.}}
\VerseOne{}Ce sont ici les paroles que Moïse déclara à tout Israël de l'autre côté du Jourdain, dans le désert, dans la plaine, qui est vis-à-vis de Suph, entre Paran, Tophel, Laban, Hatséroth, et Di-Zahab.
\VS{2}Il y a onze journées depuis Horeb, par le chemin de la montagne de Séir, jusqu'à Kadès-Barnéa.
\VS{3}Or il arriva dans la quarantième année, au onzième mois, le premier jour du mois, que Moïse parla aux enfants d'Israël selon tout ce que Yahweh lui avait ordonné de leur dire,
\VS{4}après qu'il eut battu Sihon, roi des Amoréens, qui habitait à Hesbon, et Og, roi de Basan, qui demeurait à Aschtaroth et à Edréi\FTNT{No. 21:23-24.}.
\VS{5}Moïse donc commença à expliquer cette loi, de l'autre côté du Jourdain, dans le pays de Moab, en disant~:
\VS{6}Yahweh, notre Dieu, nous a parlé à Horeb, en disant~: Vous avez assez demeuré dans cette montagne.
\VS{7}Tournez-vous, partez, et allez à la montagne des Amoréens et dans tous les lieux voisins, dans la plaine, dans la montagne, dans la vallée, vers le sud, sur le rivage de la mer, au pays des Cananéens et au Liban, jusqu'au grand fleuve, le fleuve d'Euphrate.
\VS{8}Regardez, j'ai mis devant vous le pays~; entrez et prenez possession du pays que Yahweh a juré de donner à vos pères, Abraham, Isaac et Jacob, et à leur postérité après eux.
\VS{9}Et je vous parlai en ce temps-là, et je vous dis~: Je ne puis pas, à moi seul, vous porter.
\VS{10}Yahweh, votre Dieu, vous a multipliés, et vous voici aujourd'hui comme les étoiles du ciel par le nombre.
\VS{11}Que Yahweh, le Dieu de vos pères, vous fasse croître mille fois au delà de ce que vous êtes et vous bénisse, comme il vous l'a dit~!
\VS{12}Comment porterais-je moi seul vos chagrins, vos charges, et vos procès~?
\VS{13}Prenez dans vos tribus des hommes sages, intelligents et connus, et je les établirai chefs sur vous.
\VS{14}Et vous me répondîtes, et dîtes~: Il est bon de faire ce que tu as dit.
\VS{15}Alors je pris les chefs de vos tribus, des hommes sages et connus, et je les établis chefs sur vous, chefs de milliers, chefs de centaines, chefs de cinquantaines, chefs de dizaines et officiers selon vos tribus. 
\VS{16}Puis j'ordonnai en ce temps-là à vos juges, en disant~: Ecoutez les différends qui seront entre vos frères, et jugez droitement entre l'homme et son frère, et entre l'étranger qui est avec lui\FTNT{Lé. 19:15~; De. 16:19~; Pr. 24:23.}.
\VS{17}Vous n'aurez point d'égard à l'apparence de la personne en jugement~; vous entendrez autant le petit que le grand~; vous ne craindrez personne, car le jugement est à Dieu~; et vous ferez venir devant moi la cause qui sera trop difficile pour vous, et je l'entendrai. 
\VS{18}Et en ce temps-là, je vous ordonnai toutes les choses que vous deviez faire.
\VS{19}Puis nous partîmes d'Horeb, et nous marchâmes dans tout ce grand et affreux désert que vous avez vu~; par le chemin de la montagne des Amoréens, ainsi que Yahweh, notre Dieu, nous l'avait ordonné, et nous vînmes jusqu'à Kadès-Barnéa.
\VS{20}Alors je vous dis~: Vous êtes arrivés jusqu'à la montagne des Amoréens, que Yahweh, notre Dieu, nous donne.
\VS{21}Regarde, Yahweh, ton Dieu, met le pays devant toi~; monte et prends-en possession, comme Yahweh, le Dieu de tes pères, te l'a dit~; ne crains point et ne t'effraie point.
\VS{22}Et vous vous approchâtes tous de moi, et dîtes~: Envoyons devant nous des hommes pour explorer le pays, et qui nous rapportent des nouvelles du chemin par lequel nous devrons monter, et des villes où nous devrons aller\FTNT{No. 13:2.}.
\VS{23}Et ce discours sembla bon à mes yeux~; et je pris douze hommes parmi vous, un homme par tribu.
\VS{24}Et ils se mirent en chemin et montèrent dans la montagne, et vinrent jusqu'au torrent d'Eschcol et explorèrent le pays.
\VS{25}Et ils prirent dans leurs mains des fruits du pays, et ils nous les apportèrent~; ils nous donnèrent des nouvelles, et nous dirent~: Le pays que Yahweh, notre Dieu, nous donne est bon. 
\VS{26}Mais vous refusâtes d'y monter, et vous fûtes rebelles à l'ordre de Yahweh, votre Dieu.
\VS{27}Et vous murmurâtes dans vos tentes, en disant~: C'est parce que Yahweh nous hait qu'il nous a fait sortir du pays d'Egypte, afin de nous livrer entre les mains des Amoréens pour nous exterminer.
\VS{28}Où monterions-nous~? Nos frères nous ont fait fondre le cœur, en disant~: Le peuple est plus grand que nous, et de plus haute taille~; les villes sont grandes et closes jusqu'au ciel~; et même nous avons vu là les fils des Anakim.
\VS{29}Mais je vous dis~: Ne tremblez point et ne les craignez point.
\VS{30}Yahweh, votre Dieu, qui marche devant vous, lui-même combattra pour vous, selon tout ce que vous avez vu qu'il a fait pour vous en Egypte~;
\VS{31}et au désert, où tu as vu de quelle manière Yahweh, ton Dieu, t'a porté comme un homme porterait son fils, sur tout le chemin où vous avez marché, jusqu'à ce que vous soyez arrivés dans ce lieu-ci.
\VS{32}Mais malgré cela, vous ne crûtes point encore en Yahweh, votre Dieu,
\VS{33}qui marchait devant vous sur le chemin afin de vous chercher un lieu pour camper, marchant de nuit dans la colonne de feu pour vous éclairer dans le chemin par lequel vous deviez marcher et de jour dans la nuée.
\VS{34}Et Yahweh entendit la voix de vos paroles et se mit en grande colère et jura, disant~:
\VS{35}Aucun des hommes de cette méchante génération ne verra ce bon pays que j'ai juré de donner à vos pères,
\VS{36}à l'exception de Caleb, fils de Jephunné~; lui le verra, et je donnerai à lui et à ses fils le pays sur lequel il a marché, parce qu'il a persévéré à suivre Yahweh\FTNT{No. 14:22-24.}.
\VS{37}Même Yahweh s'est mis en colère contre moi à cause de vous, disant~: Et toi aussi tu n'y entreras pas.
\VS{38}Josué, fils de Nun, qui te sert, y entrera~; fortifie-le, car c'est lui qui mettra les enfants d'Israël en possession de ce pays\FTNT{De. 34:4.}.
\VS{39}Et vos petits-enfants, dont vous avez dit qu'ils seront en proie, vos enfants, dis-je, qui aujourd'hui ne savent pas ce que c'est le bien ou le mal, eux y entreront, et je leur donnerai ce pays et ils le posséderont.
\VS{40}Mais vous, retournez vous-en en arrière, et allez dans le désert par le chemin de la Mer Rouge.
\VS{41}Et vous répondîtes et me dîtes~: Nous avons péché contre Yahweh, nous monterons et nous combattrons, comme Yahweh, notre Dieu, nous l'a ordonné. Et vous ceignîtes chacun vos armes de guerre, et vous entreprîtes hardiment de monter à la montagne.
\VS{42}Et Yahweh me dit~: Dis-leur~: Ne montez point et ne combattez point, car je ne suis point au milieu de vous~; afin que vous ne soyez point battus par vos ennemis.
\VS{43}Je vous parlai, mais vous ne m'écoutâtes point et vous vous rebellâtes contre l'ordre de Yahweh, et vous fûtes orgueilleux et vous montâtes sur la montagne.
\VS{44}Et les Amoréens, qui demeuraient sur cette montagne, sortirent contre vous et vous poursuivirent comme font les abeilles~; et ils vous battirent depuis Séir jusqu'à Horma.
\VS{45}Et étant retournés vous pleurâtes devant Yahweh~; mais Yahweh n'écouta point votre voix, et ne vous prêta point l'oreille.
\VS{46}Ainsi, vous demeurâtes à Kadès plusieurs jours, autant de temps que vous y aviez demeuré.
\Chap{2}
\TextTitle{Périple du peuple dans le désert}
\VerseOne{}Alors nous retournâmes en arrière, et nous partîmes pour le désert, par le chemin de la Mer Rouge, comme Yahweh me l'avait dit, et nous tournâmes autour de la montagne de Séir plusieurs jours.
\VS{2}Et Yahweh me parla, en disant~:
\VS{3}Vous avez assez tourné autour de cette montagne. Tournez-vous vers le nord.
\VS{4}Ordonne au peuple, en disant~: Vous allez passer la frontière de vos frères, les fils d'Esaü, qui demeurent en Séir. Ils auront peur de vous~; mais soyez bien sur vos gardes.
\VS{5}N'ayez pas de démêlé avec eux~; car je ne vous donnerai rien dans leur pays, pas même de quoi poser la plante du pied~: J'ai donné à Esaü la montagne de Séir en héritage.
\VS{6}Vous achèterez d'eux la nourriture à prix d'argent et vous en mangerez, et vous achèterez d'eux l'eau à prix d'argent et vous en boirez.
\VS{7}Car Yahweh, ton Dieu, t'a béni dans tout le travail de tes mains, il a connu ta marche dans ce grand désert. Yahweh, ton Dieu, a été avec toi pendant ces quarante années, et tu n'as manqué de rien.
\VS{8}Nous passâmes à distance de nos frères, les fils d'Esaü, qui demeuraient en Séir, à distance du chemin de la plaine, d'Elath et d'Etsjon-Guéber, et nous nous tournâmes, et nous passâmes par le chemin du désert de Moab.
\VS{9}Yahweh me dit~: N'assiège point Moab, et ne t'engage pas dans un combat avec lui~; car je ne te donnerai rien en héritage dans son pays~: J'ai donné Ar en héritage aux fils de Lot\FTNT{Ge. 19:36-38.}.
\VS{10}Les Emim y habitaient auparavant~; c'était un peuple grand, nombreux et de haute taille comme les Anakim.
\VS{11}Ils étaient considérés comme des Rephaïm, de même que les Anakim~; mais les Moabites les appelaient Emim.
\VS{12}Séir était habité autrefois par les Horiens~; mais les fils d'Esaü les en dépossédèrent, les détruisirent devant eux, et y habitèrent à leur place, comme l'a fait Israël dans le pays de son héritage que Yahweh lui a donné.
\VS{13}Mais maintenant, levez-vous, et passez le torrent de Zéred. Et nous passâmes le torrent de Zéred.
\VS{14}Or le temps que nous avons marché de Kadès-Barnéa, jusqu'à ce que nous ayons passé le torrent de Zéred, fut de trente-huit ans, jusqu'à ce que toute la génération des hommes de guerre eût été consumée du milieu du camp, comme Yahweh le leur avait juré.
\VS{15}La main de Yahweh fut aussi sur eux pour les détruire du milieu du camp, jusqu'à ce qu'ils eussent été consumés.
\VS{16}Or il est arrivé qu'après que tous les hommes de guerre eurent été consumés par la mort du milieu du peuple,
\VS{17}Yahweh me parla, et dit~:
\VS{18}Tu vas passer aujourd'hui la frontière de Moab, à savoir Har.
\VS{19}Tu t'approcheras en face des fils d'Ammon, mais ne les assiège point, et ne t'engage point dans un combat avec eux~; car je ne te donnerai rien en possession dans le pays des fils d'Ammon~: Je l'ai donné en héritage aux fils de Lot.
\VS{20} Ce pays était aussi considéré comme un pays de Rephaïm~; car les Rephaïm y habitaient auparavant, et les Ammonites les appelaient Zamzummim~;
\VS{21}c'était un peuple grand, nombreux, et de haute taille, comme les Anakim, Yahweh les détruisit devant eux, et ils les dépossédèrent, et habitèrent à leur place.
\VS{22}Comme il fit pour les fils d'Esaü qui demeurent en Séir, quand il détruisit les Horiens devant eux~; ils les dépossédèrent et habitèrent à leur place jusqu'à ce jour.
\VS{23}Or quant aux Avviens, qui demeuraient en Hatserim jusqu'à Gaza, ils furent détruits par les Caphtorim, sortis de Caphtor, qui demeurèrent à leur place.
\TextTitle{Yawheh livre Sihon, roi de Hesbon, entre les mains d'Israël}
\VS{24}Levez-vous, partez et passez le torrent de l'Arnon. Regarde, j'ai livré entre tes mains Sihon, roi de Hesbon, l'Amoréen, et son pays. Commence à en prendre possession, et fais-lui la guerre~!
\VS{25}Aujourd'hui, je vais commencer à mettre la frayeur et la crainte de toi sur les peuples qui sont sous les cieux~; et ayant entendu parler de toi, ils trembleront et seront dans l'angoisse à cause de ta présence.
\VS{26}J'envoyai, du désert de Kedémoth, des messagers à Sihon, roi de Hesbon, avec des paroles de paix, disant\FTNT{No. 21:21.}~:
\VS{27}Permets que je passe par ton pays~; et j'irai par le grand chemin, sans me détourner ni à droite ni à gauche.
\VS{28}Tu me vendras de la nourriture à prix d'argent, afin que je mange, et tu me donneras de l'eau à prix d'argent, afin que je boive~; seulement que j'y passe de mes pieds.
\VS{29}C'est ce qu'ont fait les fils d'Esaü qui demeurent en Séir, et les Moabites qui demeurent à Ar, jusqu'à ce que je passe le Jourdain pour entrer au pays que Yahweh, notre Dieu, nous donne.
\VS{30}Mais Sihon, roi de Hesbon, ne voulut point nous laisser passer par son pays~; car Yahweh, ton Dieu, avait endurci son esprit, et raidit son cœur afin de le livrer entre tes mains, comme tu le vois aujourd'hui.
\VS{31}Yahweh me dit~: Regarde, j'ai commencé à te livrer Sihon et son pays~; commence à posséder son pays, pour le tenir en héritage.
\VS{32}Sihon donc, sortit nous rencontrer avec tout son peuple pour nous combattre à Jahats.
\VS{33}Mais Yahweh, notre Dieu, nous le livra en face, et nous le battîmes, lui, ses fils, et tout son peuple.
\VS{34}Et en ce temps-là, nous prîmes toutes ses villes, et nous détruisîmes par le moyen de l'interdit les villes, les hommes, les femmes, et les petits enfants, sans laisser de survivants.
\VS{35}Seulement nous pillâmes les bêtes pour nous, et le butin des villes que nous avions prises.
\VS{36}Depuis Aroër, qui est sur le bord du torrent de l'Arnon, et la ville qui est dans la vallée, jusqu'à Galaad, il n'y eut pas une ville qui fût trop haute pour nous~: Yahweh, notre Dieu, nous livra tout.
\VS{37}Seulement tu n'approchas point du pays des fils d'Ammon, de tous les bords du torrent de Jabbok, des villes de la montagne, ni d'aucun lieu que Yahweh, notre Dieu, t'avait ordonné de ne point attaquer.
\Chap{3}
\TextTitle{Yawheh livre Og, roi de Basan, entre les mains d'Israël}
\VerseOne{}Alors nous nous tournâmes, et nous montâmes par le chemin de Basan. Et Og, roi de Basan, sortit nous rencontrer, avec tout son peuple, pour nous combattre à Edréi.
\VS{2}Et Yahweh me dit~: Ne le crains point~; car je le livre entre tes mains, lui, tout son peuple, et son pays~; et tu lui feras comme tu as fait à Sihon, roi des Amoréens, qui demeurait à Hesbon.
\VS{3}Ainsi Yahweh, notre Dieu, livra aussi entre nos mains Og, roi de Basan, avec tout son peuple~; nous le battîmes sans laisser de survivants.
\VS{4}En ce même temps, nous prîmes aussi toutes ses villes, et il n'y eut point de ville que nous ne lui prîmes pas~: Soixante villes, toute la contrée d'Argob, le royaume d'Og en Basan.
\VS{5}Toutes ces villes-là étaient fortifiées, avec de hautes murailles, des portes et des barres. Il y avait aussi des villes sans murailles en fort grand nombre.
\VS{6}Et nous les détruisîmes par le moyen de l'interdit, comme nous l'avions fait à Sihon, roi de Hesbon~; nous dévouâmes par le moyen de l'interdit toutes les villes, les hommes, les femmes, et les petits enfants.
\VS{7}Mais nous pillâmes pour nous toutes les bêtes et le butin des villes.
\VS{8}Nous prîmes donc, en ce temps-là, le pays de la main des deux rois des Amoréens, qui étaient de l'autre côté du Jourdain, depuis le torrent de l'Arnon jusqu'à la montagne de l'Hermon~;
\VS{9}or les Sidoniens donnent à l'Hermon le nom de Sirion, mais les Amoréens le nomment Senir~;
\VS{10}toutes les villes de la plaine, tout Galaad, et tout Basan jusqu'à Salca et Edréi, les villes du royaume d'Og en Basan.
\VS{11}Og, roi de Basan, avait survécu seul du reste des Rephaïm. Voici, son lit, un lit de fer, n'est-il pas dans Rabbath, ville des fils d'Ammon~? Sa longueur est de neuf coudées, et sa largeur de quatre coudées, en coudées d'homme.
\TextTitle{Premières terres attribuées à Ruben, Gad et à la demi-tribu de Manassé}
\VS{12}En ce temps-là donc, nous prîmes possession de ce pays. Je donnai aux Rubénites et aux Gadites le territoire à partir d'Aroër, sur le torrent de l'Arnon, et la moitié de la montagne de Galaad, avec ses villes\FTNT{Jos. 13:23-32.}.
\VS{13}Je donnai à la demi-tribu de Manassé le reste de Galaad et tout le royaume d'Og, en Basan~: Toute la contrée d'Argob avec tout le Basan, c'est ce qu'on appelait le pays des Réphaïm.
\VS{14}Jaïr, fils de Manassé, prit toute la contrée d'Argob jusqu'à la frontière des Gueschuriens et des Maacathiens, et il donna son nom à Basan, appelé villages de Jaïr jusqu'à aujourd'hui.
\VS{15}Je donnai aussi Galaad à Makir.
\VS{16}Mais aux Rubénites et aux Gadites, je donnai de Galaad jusqu'au torrent de l'Arnon, dont le milieu du torrent sert de frontière, et jusqu'au torrent de Jabbok, frontière des fils d'Ammon~;
\VS{17}la plaine, et le Jourdain, de la frontière de Kinnéreth jusqu'à la mer de la plaine, la Mer Salée, aux pieds de Pisga vers l'orient.
\VS{18}Or en ce temps-là, je vous ordonnai, en disant~: Yahweh, votre Dieu, vous donne ce pays pour le posséder. Vous tous, qui êtes vaillants, vous passerez armés devant vos frères, les fils d'Israël.
\VS{19}Seulement vos femmes, vos petits-enfants, et vos troupeaux, car je sais que vous avez beaucoup de troupeaux, resteront dans les villes que je vous ai données,
\VS{20}jusqu'à ce que Yahweh ait accordé du repos à vos frères comme à vous, et qu'eux aussi possèdent le pays que Yahweh, votre Dieu, leur donne de l'autre côté du Jourdain. Puis vous retournerez chacun dans l'héritage que je vous ai donné.
\VS{21}En ce temps-là, j'ordonnai à Josué, en disant~: Tes yeux ont vu tout ce que Yahweh, votre Dieu, a fait à ces deux rois~: Yahweh en fera de même à tous les royaumes vers lesquels tu vas passer.
\VS{22}Ne les craignez point~; car Yahweh, votre Dieu, combattra lui-même pour vous.
\TextTitle{Moïse n'entrera pas dans la terre promise}
\VS{23}En ce même temps, j'implorai la grâce de Yahweh, en disant~:
\VS{24}Seigneur Yahweh, tu as commencé à montrer à ton serviteur ta grandeur et ta main puissante~; car quel est le dieu dans le ciel et sur la terre qui puisse faire selon tes œuvres et selon ta puissance~?
\VS{25}Que je passe, je te prie, et que je voie ce bon pays de l'autre côté du Jourdain, ces bonnes montagnes et le Liban.
\VS{26}Mais Yahweh s'irrita contre moi, à cause de vous, et ne m'écouta point. Yahweh me dit~: C'est assez, ne me parle plus de cette affaire.
\VS{27}Monte au sommet du Pisga, et lève tes yeux à l'occident, au nord, au sud, et à l'orient, et regarde de tes yeux~; car tu ne passeras point ce Jourdain.
\VS{28}Donnes-en la charge à Josué, fortifie-le et affermis-le~; car c'est lui qui passera devant ce peuple et qui le mettra en possession du pays que tu verras.
\VS{29}Ainsi nous demeurâmes dans la vallée, vis-à-vis de Beth-Peor.
\Chap{4}
\TextTitle{Encouragement à garder la loi de Yahweh}
\VerseOne{}Et maintenant Israël, écoute les lois et les ordonnances que je vous enseigne, pour les pratiquer afin que vous viviez, que vous entriez et possédiez le pays que Yahweh, le Dieu de vos pères, vous donne.
\VS{2}Vous n'ajouterez\FTNT{De. 12:32~; Pr. 30:6~; Ap. 22:18-19.} rien à la parole que je vous ordonne, et vous n'en retrancherez rien~; afin de garder les commandements de Yahweh, votre Dieu, que je vous ordonne.
\VS{3}Vos yeux ont vu ce que Yahweh a fait à cause de Baal-Peor~: Yahweh, ton Dieu, a détruit du milieu de toi tout homme qui était allé après Baal-Peor\FTNT{No. 25:4-9.}.
\VS{4}Mais vous, qui vous êtes attachés à Yahweh, votre Dieu, vous êtes tous vivants aujourd'hui.
\VS{5}Regardez, je vous ai enseigné des lois et des ordonnances, comme Yahweh, mon Dieu, me l'a ordonné, afin que vous les pratiquiez au milieu du pays où vous allez pour le posséder.
\VS{6}Vous les garderez et vous les pratiquerez, car c'est là votre sagesse et votre intelligence aux yeux de tous les peuples, qui entendront ces lois, et qui diront~: Cette grande nation est un peuple sage et intelligent~!
\TextTitle{Israël, privilégié parmi tous les peuples}
\VS{7}Car quelle est la grande nation qui ait ses dieux proches d'elle, comme nous avons Yahweh, notre Dieu, toutes les fois que nous l'invoquons~?
\VS{8}Et quelle est la grande nation qui ait des lois et des ordonnances justes, comme toute cette loi que je mets aujourd'hui devant vous~?
\VS{9}Seulement, prends garde à toi et garde soigneusement ton âme, afin que tous les jours de ta vie tu n'oublies point les choses que tes yeux ont vues, et qu'elles ne sortent de ton cœur\FTNT{Pr. 4:23.}~; enseigne-les à tes fils, et aux fils de tes fils.
\VS{10}Rappelle-toi du jour où tu te tins face à Yahweh, ton Dieu, à Horeb, après que Yahweh me dit~: Convoque le peuple~! Je veux leur faire entendre mes paroles, pour qu'ils apprennent à me craindre tout le temps qu'ils seront vivants sur la terre~; et pour qu'ils les enseignent à leurs fils.
\VS{11}Et que vous vous approchâtes, et vous vous tîntes au pied de la montagne. Or la montagne était embrasée de feu jusqu'au milieu du ciel. Il y avait des ténèbres, une nuée, et une obscurité.
\VS{12}Et Yahweh vous parla du milieu du feu~; vous entendîtes le son de ses paroles, mais vous ne vîtes aucune image, vous entendîtes seulement la voix\FTNT{Ex. 19:17-19.}.
\VS{13}Et il déclara son alliance, qu'il vous ordonna d'observer, les dix paroles, qu'il écrivit sur deux tables de pierre.
\VS{14}Yahweh m'ordonna aussi, en ce temps-là, de vous enseigner les lois et les ordonnances, afin que vous les pratiquiez dans le pays que vous allez posséder.
\VS{15}Prenez bien garde à vos âmes, puisque vous n'avez vu aucune image le jour où Yahweh, votre Dieu, vous parla du milieu du feu à Horeb,
\VS{16}de peur que vous ne vous corrompiez et que vous ne vous fassiez une image taillée, une représentation d'idole ayant la forme d'un mâle ou d'une femelle,
\VS{17}ou la forme d'un animal qui soit sur la terre, ou la forme d'un oiseau ailé qui vole dans les cieux,
\VS{18}ou la forme d'un animal qui rampe sur la terre, ou la forme d'un poisson qui soit dans les eaux au-dessous de la terre.
\VS{19}De peur aussi qu'élevant tes yeux vers les cieux, et voyant le soleil, la lune, et les étoiles, toute l'armée des cieux, tu ne sois poussé à te prosterner devant elles, et que tu ne les serves~: C'est ce que Yahweh, ton Dieu, a donné en partage à tous les peuples, sous tous les cieux.
\VS{20}Mais vous, Yahweh vous a pris, et vous a fait sortir d'Egypte, du fourneau de fer, afin que vous fussiez un peuple de son héritage, comme vous l'êtes aujourd'hui.
\TextTitle{Conséquences de la désobéissance et de l'idolâtrie}
\VS{21}Or Yahweh s'irrita contre moi, à cause de vos paroles, et il jura que je ne passerais point le Jourdain, et que je n'entrerais point dans ce bon pays que Yahweh, ton Dieu, te donne en héritage.
\VS{22}Je mourrai dans ce pays-ci, je ne passerai point le Jourdain~; mais vous le passerez, et vous posséderez ce bon pays.
\VS{23}Gardez-vous d'oublier l'alliance de Yahweh, votre Dieu, qu'il a traitée avec vous, et que vous ne vous fassiez d'image taillée, de représentation quelconque, que Yahweh, votre Dieu, vous a défendues.
\VS{24}Car Yahweh, ton Dieu, est un feu dévorant\FTNT{Hé. 12:29.}, un Dieu jaloux.
\VS{25}Quand tu auras engendré des fils, et des fils de tes fils, et que vous serez depuis longtemps dans le pays, si vous vous corrompez, et que vous faites des images taillées, ou des représentations de quelque chose que ce soit, si vous faites ce qui est mal aux yeux de Yahweh, votre Dieu, afin de l'irriter
\VS{26}j'appelle aujourd'hui à témoin les cieux et la terre contre vous, certainement vous périrez promptement dans ce pays que vous allez posséder au-delà du Jourdain, vous n'y prolongerez point vos jours, car vous serez entièrement détruits.
\VS{27}Yahweh vous dispersera parmi les peuples, et vous ne resterez qu'un petit nombre parmi les nations, chez lesquelles Yahweh vous emmènera.
\VS{28}Et là, vous servirez des dieux, œuvres de main d'homme, du bois et de la pierre, qui ne peuvent voir, ni entendre, ni manger, ni sentir\FTNT{Es. 44:9~; Es. 46:7~; Ps. 115:4-7}.
\TextTitle{Yahweh, puissant, miséricordieux et fidèle à son alliance}
\VS{29}Mais de là, tu chercheras Yahweh, ton Dieu, et tu le trouveras, si tu le cherches de tout ton cœur et de toute ton âme.
\VS{30}Quand tu seras dans la détresse, et que toutes ces choses te seront arrivées, alors, dans les derniers jours, tu retourneras à Yahweh, ton Dieu, et tu obéiras à sa voix~;
\VS{31}parce que Yahweh, ton Dieu, est le Dieu puissant et miséricordieux, il ne t'abandonnera point et ne te détruira point, il n'oubliera point l'alliance de tes pères qu'il leur a jurée.
\VS{32}Interroge les premiers temps, qui ont été avant toi, depuis le jour que Dieu créa l'homme sur la terre, et d'une extrémité des cieux à l'autre, s'il a jamais été rien fait de semblable à cette grande chose, et s'il a été jamais rien entendu de semblable.
\VS{33}Est-ce qu'un peuple a entendu la voix de Dieu parlant du milieu du feu, comme tu l'as entendue, et qui soit demeuré en vie~?
\VS{34}Où Dieu a-t-il essayé de venir prendre pour lui une nation du milieu d'une nation, par des épreuves, des signes, des miracles, et des batailles, à main forte, et à bras étendu, et par des choses grandes et terribles, comme tout ce que Yahweh, notre Dieu, a fait pour vous en Egypte, sous vos yeux~?
\VS{35}Cela t'a été montré afin que tu reconnaisses que Yahweh est Dieu et qu'il n'y en a point d'autre.
\VS{36}Il t'a fait entendre sa voix des cieux pour t'instruire~; et il t'a montré son grand feu sur la terre, et tu as entendu ses paroles du milieu du feu.
\VS{37}Et parce qu'il a aimé tes pères, il a choisi leur postérité après eux et il t'a retiré d'Egypte en sa présence, par sa grande puissance~;
\VS{38}pour chasser de devant toi des nations plus grandes et plus puissantes que toi, pour te faire entrer dans leur pays, et pour te le donner en héritage, comme tu le vois aujourd'hui.
\VS{39}Sache donc aujourd'hui, et rappelle dans ton cœur que Yahweh est Dieu, en haut dans les cieux et sur la terre, et qu'il n'y en a point d'autre.
\VS{40}Garde donc ses lois et ses commandements que je t'ordonne aujourd'hui, afin que tu sois heureux, toi et tes fils après toi, et que tu prolonges tes jours sur la terre que Yahweh, ton Dieu, te donne\FTNT{Ex 20.}.
\TextTitle{Trois villes de refuge à l'est du Jourdain}
\VS{41}Alors Moïse sépara trois villes de l'autre côté du Jourdain vers le soleil levant,
\VS{42}afin que le meurtrier qui aurait tué son prochain involontairement, sans l'avoir haï auparavant, s'y enfuie~; et qu'en s'enfuyant dans l'une de ces villes-là, il eût sa vie sauve.
\VS{43}C'étaient~: Betser dans le désert, dans la plaine du pays, chez les Rubénites~; Ramoth en Galaad, chez les Gadites~; Golan en Basan, chez les Manassites.
\VS{44}C'est ici la loi que Moïse plaça face aux enfants d'Israël.
\VS{45}Voici les témoignages, les lois, et les ordonnances que Moïse déclara aux enfants d'Israël, après qu'ils furent sortis d'Egypte.
\VS{46}C'était de l'autre côté du Jourdain, dans la vallée, vis-à-vis de Beth-Peor, au pays de Sihon, roi des Amoréens, qui demeurait à Hesbon, et qui fut battu par Moïse et les enfants d'Israël après être sortis d'Egypte.
\VS{47}Et ils s'emparèrent de son pays avec le pays d'Og, roi de Basan, deux rois des Amoréens qui étaient de l'autre côté du Jourdain, vers le soleil levant.
\VS{48}Depuis Aroër, sur le bord du torrent de l'Arnon, jusqu'à la montagne de Sion, qui est l'Hermon,
\VS{49}et toute la plaine de l'autre côté du Jourdain vers l'orient, jusqu'à la mer de la plaine, au pied du Pisga.
\Chap{5}
\TextTitle{L'alliance établie à Horeb rappelée à la nouvelle génération}
\VerseOne{}Moïse appela tout Israël, et leur dit~: Ecoute, Israël, les lois et les ordonnances que je prononce aujourd'hui à vos oreilles, apprenez-les, et veillez à les mettre en pratique.
\VS{2}Yahweh, notre Dieu, a traité avec nous une alliance en Horeb\FTNT{Ex. 19:5.}.
\VS{3}Dieu n'a point traité cette alliance avec nos pères, mais avec nous, qui sommes ici aujourd'hui tous vivants.
\VS{4}Yahweh vous parla face à face sur la montagne du milieu du feu.
\VS{5}Je me tenais en ce temps-là entre Yahweh et vous, pour vous rapporter la parole de Yahweh~; parce que vous aviez peur face à ce feu, et vous ne montâtes point sur la montagne. Il dit\FTNT{Les dix paroles (Ex. 20).}~:
\VS{6}Je suis Yahweh, ton Dieu, qui t'ai fait sortir du pays d'Egypte, de la maison de servitude.
\VS{7}Tu n'auras point d'autres dieux devant ma face.
\VS{8}Tu ne te feras point d'image taillée, ni de représentation des choses qui sont en haut dans les cieux, ni sur la terre, ni dans les eaux sous la terre.
\VS{9}Tu ne te prosterneras point devant elles, et tu ne les serviras point~; car je suis Yahweh, ton Dieu, un Dieu jaloux, qui punis l'iniquité des pères sur les enfants jusqu'à la troisième et à la quatrième génération de ceux qui me haïssent,
\VS{10}et qui fais miséricorde jusqu'à mille générations à ceux qui m'aiment et qui gardent mes commandements.
\VS{11}Tu ne prendras point le Nom de Yahweh, ton Dieu, en vain~; car Yahweh ne tiendra pas pour innocent celui qui prendra son Nom en vain.
\VS{12}Garde le jour du sabbat pour le sanctifier, comme Yahweh, ton Dieu, te l'a ordonné.
\VS{13}Tu travailleras six jours, et tu feras toute ton œuvre,
\VS{14}mais le septième jour est le sabbat de Yahweh, ton Dieu~: Tu ne feras aucune œuvre, ni ton fils, ni ta fille, ni ton serviteur, ni ta servante, ni ton bœuf, ni ton âne, ni aucune de tes bêtes, ni l'étranger qui est dans tes portes, afin que ton serviteur et ta servante se reposent comme toi.
\VS{15}Et tu te souviendras que tu as été esclave au pays d'Egypte, et que Yahweh, ton Dieu, t'en a fait sortir à main forte et à bras étendu~: C'est pourquoi Yahweh, ton Dieu, t'a ordonné d'observer le jour du sabbat.
\VS{16}Honore ton père et ta mère, comme Yahweh, ton Dieu, te l'a ordonné, afin que tes jours se prolongent et que tu sois heureux sur la terre que Yahweh, ton Dieu, te donne.
\VS{17}Tu ne tueras point.
\VS{18}Tu ne commettras point d'adultère.
\VS{19}Tu ne déroberas point.
\VS{20}Tu ne diras point de faux témoignage contre ton prochain.
\VS{21}Tu ne convoiteras point la femme de ton prochain~; tu ne désireras point la maison de ton prochain, ni son champ, ni son serviteur, ni sa servante, ni son bœuf, ni son âne, ni aucune chose qui soit à ton prochain.
\VS{22}Yahweh déclara ces paroles à toute votre assemblée sur la montagne, du milieu du feu, des nuées et de l'obscurité, à voix forte sans rien ajouter. Il les écrivit sur deux tables de pierre qu'il me donna.
\TextTitle{Moïse, intermédiaire entre Yahweh et le peuple}
\VS{23}Or il arriva qu'aussitôt que vous eûtes entendu la voix du milieu de l'obscurité, parce que la montagne était embrasée par le feu, vos chefs de tribus et vos anciens s'approchèrent de moi,
\VS{24}et vous dîtes~: Voici, Yahweh, notre Dieu, nous a fait voir sa gloire et sa grandeur, et nous avons entendu sa voix du milieu du feu~; aujourd'hui, nous avons vu que Dieu a parlé avec l'homme, et qu'il est resté en vie.
\VS{25}Et maintenant pourquoi mourrions-nous~? Car ce grand feu là nous dévorera~; si nous entendons encore la voix de Yahweh, notre Dieu, nous mourrons.
\VS{26}Car qui, de toute chair, a entendu comme nous la voix du Dieu vivant parlant du milieu du feu, et qui soit resté en vie~?
\VS{27}Approche-toi et écoute tout ce que Yahweh, notre Dieu, dira~; puis tu nous diras tout ce que Yahweh, notre Dieu, t'aura dit~; nous l'entendrons, et nous le ferons.
\VS{28}Yahweh entendit la voix de vos paroles pendant que vous me parliez. Et Yahweh me dit~: J'ai entendu les paroles que ce peuple t'ont adressées~: Tout ce qu'ils ont dit est bien.
\VS{29}Ô~! S'ils avaient toujours ce même cœur pour me craindre et pour garder tous mes commandements, afin qu'ils fussent heureux, eux et leurs enfants, pour toujours~!
\VS{30}Va, dis-leur~: Retournez dans vos tentes.
\VS{31}Mais toi, reste ici avec moi, et je te dirai tous les commandements, les lois, et les ordonnances que tu leur enseigneras, afin qu'ils les pratiquent dans le pays que je leur donne en possession.
\VS{32}Vous prendrez donc garde de faire ce que Yahweh, votre Dieu, vous a ordonné~; vous ne vous en détournerez ni à droite ni à gauche.
\VS{33}Vous marcherez dans toute la voie que Yahweh, votre Dieu, vous a ordonnée, afin que vous viviez et que vous soyez heureux, et que vous prolongiez vos jours sur la terre que vous posséderez.
\Chap{6}
\TextTitle{Obéissance à la loi, source de bénédictions}
\VerseOne{}Voici les commandements, les lois et les ordonnances que Yahweh, votre Dieu, m'a ordonné de vous enseigner, afin que vous les pratiquiez dans le pays dans lequel vous allez passer pour le posséder~;
\VS{2}afin que tu craignes Yahweh, ton Dieu, en gardant durant tous les jours de ta vie, toi, ton fils, et le fils de ton fils, toutes ses lois et ses commandements que je t'ordonne, pour que tes jours soient prolongés.
\VS{3}Tu les écouteras donc, ô Israël, et tu auras soin de les mettre en pratique, afin que tu sois heureux, et que vous vous multipliiez sur la terre où coulent le lait et le miel, comme Yahweh, le Dieu de tes pères, l'a dit\FTNT{Ex. 3:8.}.
\VS{4}Ecoute Israël~! Yahweh, notre Dieu, Yahweh est Un\FTNT{Jacob fut le premier à faire cette prière qui affirme l'unicité de Dieu. Dieu est UN (en hébreu «~Echad~» ou «~Ehad~»). Loin de l'infirmer ou de la contredire, Jésus a confirmé cette prière et l'enseignement capital qu'elle contient (Mc. 12:29). Dieu n'est pas trois personnes en une, mais UN. Cette parole annonce un monothéisme absolu. Elle s'oppose catégoriquement au polythéisme des Cananéens qui adoraient de multiples dieux, les étoiles, la lune, le soleil, les arbres, les rois etc. Aussi les Hébreux avaient reçu l'ordre de la part de Yahweh de détruire toutes les idoles qu'ils trouveraient en la terre promise (De. 16:21). Voir également commentaire en Ge. 1:5.}.
\VS{5}Tu aimeras donc Yahweh, ton Dieu, de tout ton cœur, de toute ton âme, et de toute ta force\FTNT{Mt. 22:37~; Mc. 12:30.}.
\TextTitle{La loi de Yahweh doit être enseignée aux enfants}
\VS{6}Et ces paroles, que je t'ordonne aujourd'hui, seront dans ton cœur.
\VS{7}Tu les enseigneras soigneusement à tes enfants, et tu en parleras quand tu te tiendras dans ta maison, quand tu iras en voyage, quand tu te coucheras et quand tu te lèveras.
\VS{8}Et tu les lieras comme un signe sur tes mains, et elles seront comme des fronteaux entre tes yeux.
\VS{9}Tu les écriras aussi sur les poteaux de ta maison et sur tes portes.
\VS{10}Yahweh, ton Dieu, te fera entrer dans le pays qu'il a juré à tes pères, Abraham, Isaac, et Jacob, de te donner. Tu posséderas de grandes et bonnes villes que tu n'as point bâties,
\VS{11}des maisons pleines de toutes sortes de biens que tu n'as point remplies, des puits creusés que tu n'as point creusés, des vignes et des oliviers que tu n'as point plantés, tu mangeras, et tu te rassasieras.
\VS{12}Prends garde à toi, de peur que tu n'oublies Yahweh, qui t'a fait sortir du pays d'Egypte, de la maison de servitude.
\VS{13}Tu craindras Yahweh, ton Dieu, tu le serviras et tu jureras par son Nom.
\VS{14}Vous n'irez point après d'autres dieux, d'entre les dieux des peuples qui sont autour de vous~;
\VS{15}car Yahweh, ton Dieu, est un Dieu jaloux au milieu de toi~; de peur que la colère de Yahweh, ton Dieu, ne s'enflamme contre toi, et qu'il ne t'extermine de dessus la terre.
\VS{16}Vous ne tenterez point Yahweh, votre Dieu, comme vous l'avez tenté à Massa.
\VS{17}Vous garderez soigneusement les commandements de Yahweh, votre Dieu, ses ordonnances et ses lois qu'il vous a ordonnées.
\VS{18}Tu feras ce qui est droit et bon aux yeux de Yahweh, afin que tu sois heureux, que tu entres et que tu possèdes le bon pays que Yahweh a juré à tes pères,
\VS{19}après qu'il aura chassé tous tes ennemis de devant toi, comme Yahweh l'a dit.
\VS{20}Quand ton enfant t'interrogera à l'avenir, en disant~: Que veulent dire ces préceptes, ces lois, et ces ordonnances que Yahweh, notre Dieu, vous a ordonnés~?
\VS{21}Tu diras à ton enfant~: Nous étions esclaves de Pharaon en Egypte, et Yahweh nous a fait sortir de l'Egypte par sa main puissante.
\VS{22}Yahweh a fait sous nos yeux des signes et des miracles, grands et désastreux contre l'Egypte, contre Pharaon et contre toute sa maison~;
\VS{23}et il nous a fait sortir de là pour nous conduire dans le pays qu'il avait juré à nos pères de nous donner.
\VS{24}Yahweh nous a ordonné de pratiquer toutes ces lois, et de craindre Yahweh, notre Dieu, afin que nous soyons toujours heureux, et qu'il préserve notre vie, comme aujourd'hui.
\VS{25}Et ceci sera notre justice, que nous prenions garde de pratiquer tous ces commandements devant Yahweh, notre Dieu, comme il nous l'a ordonné.
\Chap{7}
\TextTitle{Yahweh interdit les alliances avec les peuples païens}
\VerseOne{}Quand Yahweh, ton Dieu, t'aura fait entrer dans le pays où tu vas entrer pour le posséder, et qu'il aura chassé de devant toi beaucoup de nations~: Les Héthiens, les Guirgasiens, les Amoréens, les Cananéens, les Phéréziens, les Héviens, et les Jébusiens, sept nations plus grandes et plus puissantes que toi~;
\VS{2}et que Yahweh, ton Dieu, te les aura livrées en face et que tu les auras battues, tu les dévoueras complètement à la façon de l'interdit, tu ne traiteras point d'alliance avec elles, et tu ne leur feras point de grâce.
\VS{3}Tu ne t'allieras point par mariage avec elles, tu ne donneras point tes filles à leurs fils, et tu ne prendras point leurs filles pour tes fils\FTNT{Jos. 23:12-13.}~;
\VS{4}car elles détourneraient de moi tes fils, et ils serviraient d'autres dieux, et la colère de Yahweh s'enflammerait contre vous~: Il te détruirait promptement.
\VS{5}Mais vous les traiterez de cette manière~: Vous renverserez leurs autels, vous briserez leurs statues, vous abattrez leurs Asherah\FTNT{Le mot idole vient de l'hébreu «~Asherah~». Il est cité au moins quarante fois dans le Tanakh. Il fait référence à un objet en bois utilisé dans le culte d'Astarté, l'épouse de Baal. Voir De. 19:21.}, et vous brûlerez au feu leurs images taillées.
\VS{6}Car tu es un peuple saint pour Yahweh, ton Dieu. Yahweh, ton Dieu, t'a choisi pour que tu sois pour lui un peuple précieux entre tous les peuples qui sont sur la face de la terre.
\VS{7}Ce n'est pas parce que vous êtes plus nombreux que tous les peuples que Yahweh vous a aimé et qu'il vous a choisis~; car vous êtes le plus petit de tous les peuples.
\VS{8}Mais c'est parce que Yahweh vous aime, et qu'il garde le serment qu'il a juré à vos pères, Yahweh vous a fait sortir par sa main puissante, et vous a rachetés de la maison de servitude, de la main de Pharaon, roi d'Egypte.
\VS{9}Sache que c'est Yahweh, ton Dieu, qui est Dieu. Ce Dieu fidèle garde son alliance et sa miséricorde jusqu'à mille générations envers ceux qui l'aiment et qui gardent ses commandements,
\VS{10}et qui rend la pareille en face à ceux qui le haïssent, et les fait périr~; il ne diffère point envers celui qui le hait, il lui rend la pareille en face.
\VS{11}Garde les commandements, les lois, et les ordonnances que je t'ordonne aujourd'hui, et mets-les en pratique.
\TextTitle{L'obéissance à Yahweh, source de bénédictions et de victoires}
\VS{12}Et il arrivera que si vous écoutez ces ordonnances, si vous les gardez et les mettez en pratique, Yahweh, ton Dieu, gardera l'alliance et la bonté qu'il a jurées à tes pères.
\VS{13}Et il t'aimera, te bénira, et te multipliera~; il bénira le fruit de tes entrailles, et le fruit de ta terre, ton blé, ton vin, et ton huile, les portées de ton gros et de ton menu bétail, sur la terre qu'il a juré de donner à tes pères.
\VS{14}Tu seras béni plus que tous les peuples~; il n'y aura chez toi et parmi tes bêtes, ni mâle ni femelle stérile\FTNT{Ex. 23:26.}.
\VS{15}Yahweh détournera de toi toute maladie~; il ne t'enverra aucun de ces mauvais maux d'Egypte qui te sont connus, mais il les fera venir sur tous ceux qui te haïssent.
\VS{16}Tu détruiras donc tous les peuples que Yahweh, ton Dieu, va te livrer, ton œil n'aura point de pitié, et tu ne serviras point leurs dieux, car cela te serait un piège.
\VS{17}Si tu dis dans ton cœur~: Ces nations sont plus nombreuses que moi, comment pourrai-je les déposséder~?
\VS{18}Ne les crains point. Rappelle-toi bien ce que Yahweh, ton Dieu, a fait à Pharaon, et à tous les Egyptiens,
\VS{19}de ces grandes épreuves que tes yeux ont vues, les signes et les miracles, la main forte et le bras étendu par lesquels Yahweh, ton Dieu, t'a fait sortir~; ainsi fera Yahweh, ton Dieu, à tous ces peuples que tu crains.
\VS{20}Yahweh, ton Dieu, enverra contre eux les frelons, jusqu'à ce que périssent ceux qui resteront, et ceux qui se seront cachés de devant toi.
\VS{21}Ne t'effraie point devant eux, car Yahweh, ton Dieu, le Dieu grand et terrible est au milieu de toi.
\VS{22}Or Yahweh, ton Dieu, chassera peu à peu ces nations de devant toi~; tu ne pourras pas les exterminer promptement, de peur que les bêtes des champs ne se multiplient contre toi.
\VS{23}Mais Yahweh, ton Dieu, les livrera devant toi~; et il les troublera par de grandes confusions, jusqu'à ce qu'elles soient détruites.
\VS{24}Et il livrera leurs rois entre tes mains, et tu feras disparaître leurs noms de dessous les cieux~; aucun homme ne tiendra face à toi, jusqu'à ce que tu les aies détruits.
\VS{25}Tu brûleras au feu les images taillées de leurs dieux. Tu ne convoiteras point et tu ne prendras point pour toi l'argent et l'or qui seront sur elles, de peur que tu en sois pris au piège~; car c'est une abomination pour Yahweh, ton Dieu.
\VS{26}Ainsi tu n'introduiras point de choses abominables dans ta maison, afin que tu ne sois pas, comme cette chose, dévoué par interdit~; tu la détesteras fortement, et tu l'auras en abomination, car c'est une chose dévouée par interdit.
\Chap{8}
\TextTitle{Le désert, lieu de formation, d'humiliation et d’épreuve}
\VerseOne{}Vous observerez et vous mettrez en pratique tous les commandements que je vous ordonne aujourd'hui, afin que vous viviez, que vous multipliiez, et que vous entriez en possession du pays que Yahweh a juré de donner à vos pères.
\VS{2}Et souviens-toi de tout le chemin par lequel Yahweh, ton Dieu, t'a fait marcher pendant ces quarante ans dans ce désert, afin de t'humilier et de t'éprouver, pour connaître ce qui était dans ton cœur, et si tu garderais ses commandements ou non.
\VS{3}Il t'a donc humilié, il t'a laissé avoir faim, mais il t'a nourri de la manne, que tu ne connaissais pas et que tes pères n'avaient pas connue, afin de te faire connaître que l'homme ne vivra pas de pain seulement, mais que l'homme vivra de tout ce qui sort de la bouche de Yahweh\FTNT{Mt. 4:4~; Lu. 4:4.}.
\VS{4}Ton vêtement ne s'est point usé sur toi, et ton pied ne s'est point enflé durant ces quarante années\FTNT{Né. 9:21.}.
\VS{5}Reconnais dans ton cœur que Yahweh, ton Dieu, te châtie comme un homme châtie son enfant\FTNT{Hé. 12:5-12.}.
\TextTitle{Se garder d'oublier Yahweh}
\VS{6}Et garde les commandements de Yahweh, ton Dieu, pour marcher dans ses voies, et pour le craindre.
\VS{7}Car Yahweh, ton Dieu, va te faire entrer dans un bon pays, un pays de torrents d'eaux, de fontaines et d'abîmes, qui jaillissent des vallées et des montagnes~;
\VS{8}un pays de blé, d'orge, de vignes, de figuiers, et de grenadiers~; un pays d'oliviers donnant de l'huile et du miel~;
\VS{9}un pays où tu ne mangeras point le pain avec disette, où tu ne manqueras de rien~; un pays dont les pierres sont du fer, et des montagnes desquelles tu tailleras l'airain.
\VS{10}Tu mangeras et tu te rassasieras, tu béniras Yahweh, ton Dieu, pour le bon pays qu'il t'a donné.
\VS{11}Prends garde à toi de peur que tu n'oublies Yahweh, ton Dieu, en ne gardant point ses commandements, ses ordonnances, et ses lois que je t'ordonne aujourd'hui~;
\VS{12}de peur que quand tu mangeras et que tu seras rassasié~; que tu bâtiras et habiteras de belles maisons~;
\VS{13}que ton gros et menu bétail se multipliera~; que ton argent et ton or augmentera, et que tout ce qui est à toi se multipliera,
\VS{14}que ton cœur ne s'élève point et que tu n'oublies point Yahweh, ton Dieu, qui t'a fait sortir du pays d'Egypte, de la maison de servitude,
\VS{15}qui t'a fait marcher dans ce grand et affreux désert de serpents brûlants et de scorpions, dans des lieux arides et sans eau, et qui a fait jaillir pour toi de l'eau du rocher le plus dur,
\VS{16}qui t'as fait manger dans ce désert la manne que tes pères n'avaient point connue, afin de t'humilier et de t'éprouver, pour te faire ensuite du bien,
\VS{17}et que tu ne dises dans ton cœur~: Ma force et la puissance de ma main m'ont acquis ces richesses.
\VS{18}Mais tu te souviendra de Yahweh, ton Dieu, car c'est lui qui te donne de la force pour acquérir ces richesses, afin de confirmer son alliance, qu'il a jurée à tes pères, comme tu le vois aujourd'hui.
\VS{19}Mais si tu oublies Yahweh, ton Dieu, et que tu vas après d'autres dieux, si tu les sers, et que tu te prosternes devant eux, je vous avertis aujourd'hui que vous périrez certainement.
\VS{20}Vous périrez comme les nations que Yahweh fait périr devant vous, parce que vous n'aurez pas obéi à la voix de Yahweh, votre Dieu.
\Chap{9}
\TextTitle{Yahweh, fidèle à son alliance malgré la rébellion du peuple}
\VerseOne{}Ecoute, Israël~! Tu vas passer aujourd'hui le Jourdain, pour aller posséder des nations plus grandes et plus puissantes que toi, des villes grandes et fortifiées jusqu'au ciel,
\VS{2}un peuple grand et de haute taille, les fils d'Anak, que tu connais, et dont tu as entendu dire~: Qui tiendra face aux fils d'Anak~?
\VS{3}Sache donc aujourd'hui que Yahweh, ton Dieu, passera devant toi, comme un feu dévorant, c'est lui qui les détruira, qui les humiliera devant toi~; tu les chasseras, et tu les feras périr promptement, comme Yahweh te l'a dit.
\VS{4}Ne parle pas en ton cœur, quand Yahweh, ton Dieu, les chassera de devant toi, en disant~: C'est à cause de ma justice que Yahweh me fait entrer en possession de ce pays. Car c'est à cause de la méchanceté de ces nations-là que Yahweh les chasse devant toi.
\VS{5}Ce n'est point pour ta justice ni pour la droiture de ton cœur que tu entres en possession de leur pays, mais c'est pour la méchanceté de ces nations-là que Yahweh, ton Dieu, les chasse de devant toi, et pour confirmer la parole que Yahweh a jurée à tes pères, Abraham, Isaac, et Jacob.
\VS{6}Sache donc que ce n'est point pour ta justice que Yahweh, ton Dieu, te donne ce bon pays pour que tu le possèdes~; car tu es un peuple au cou raide.
\VS{7}Souviens-toi, n'oublie pas que tu as excité la colère de Yahweh, ton Dieu, dans le désert. Depuis le jour où tu es sorti du pays d'Egypte jusqu'à ce que vous arriviez dans ce lieu, vous avez été rebelles contre Yahweh.
\VS{8}Même à Horeb, vous avez excité la colère de Yahweh~; et Yahweh s'irrita contre vous, pour vous détruire.
\VS{9}Quand je montai sur la montagne, pour prendre les tables de pierre, les tables de l'alliance que Yahweh a traitée avec vous, je demeurai sur la montagne quarante jours et quarante nuits, sans manger de pain et sans boire d'eau~;
\VS{10}et Yahweh me donna les deux tables de pierre écrites du doigt de Dieu, et contenant toutes les paroles que Yahweh avait déclarées sur la montagne, du milieu du feu, le jour de l'assemblée.
\VS{11}Et il arriva qu'au bout de quarante jours et quarante nuits, Yahweh me donna les deux tables de pierre, qui sont les tables de l'alliance.
\VS{12}Puis Yahweh me dit~: Lève-toi, descends promptement d'ici~; car ton peuple, que tu as fait sortir d'Egypte, s'est corrompu. Ils se sont détournés promptement de la voie que je leur avais ordonnée, ils se sont fait une image en métal fondu.
\VS{13}Yahweh me parla, en disant~: Je vois que ce peuple est un peuple au cou raide.
\VS{14}Laisse-moi les détruire et effacer leur nom de dessous les cieux~; et je te ferai devenir une nation plus puissante et plus grande que celle-ci.
\VS{15}Je retournai et je descendis de la montagne~; or la montagne était toute en feu, et j'avais les deux tables de l'alliance dans mes deux mains.
\VS{16}Puis je regardai, et voici, vous aviez péché contre Yahweh, votre Dieu, vous vous étiez fait un veau en métal fondu, vous vous étiez détournés promptement de la voie que vous avait ordonnée Yahweh.
\VS{17}Alors je saisis les deux tables, je les jetai de mes deux mains, et je les brisai devant vos yeux.
\TextTitle{Moïse, intercède pour Israël devant Yahweh}
\VS{18}Puis je me prosternai devant Yahweh, comme auparavant, quarante jours et quarante nuits, sans manger de pain et sans boire d'eau, à cause de tout votre péché, que vous aviez commis en faisant ce qui est mal aux yeux de Yahweh, afin de l'irriter.
\VS{19}Car je craignais face à la colère et à la fureur dont Yahweh était enflammé contre vous, pour vous détruire. Et Yahweh m'exauça encore cette fois.
\VS{20}Yahweh était très irrité contre Aaron, voulant le faire périr, mais j'intercédai pour Aaron en ce temps-là.
\VS{21}Puis je pris le veau\FTNT{Le veau d'or (Ex. 32).} que vous aviez fait, votre péché, et je le brûlai au feu, je le brisai en le broyant, jusqu'à ce qu'il soit réduit en poudre, et je jetai cette poudre dans le torrent qui descend de la montagne.
\VS{22}Vous avez fort irrité la colère de Yahweh à Tabeéra, à Massa, et à Kibroth-Hattaava.
\VS{23}Et quand Yahweh vous envoya à Kadès-Barnéa, en disant~: Montez, et prenez possession du pays que je vous donne~! Vous fûtes rebelles à la parole de Yahweh, votre Dieu, vous n'eûtes point confiance, et vous n'obéîtes point à sa voix.
\VS{24}Vous avez été rebelles à Yahweh depuis le jour où je vous ai connu.
\VS{25}Je me prosternai donc devant Yahweh, je me prosternai quarante jours et quarante nuits, parce que Yahweh avait dit qu'il vous détruirait.
\VS{26}Et je priai Yahweh, et je dis~: Ô Seigneur, Yahweh, ne détruis point ton peuple, ton héritage que tu as racheté par ta grandeur, et que tu as fait sortir d'Egypte par ta main puissante.
\VS{27}Souviens-toi de tes serviteurs Abraham, Isaac, et Jacob. Ne regarde point à l'obstination de ce peuple, ni à sa méchanceté, ni à son péché,
\VS{28}de peur que le pays d'où tu nous as fait sortir ne dise~: Parce que Yahweh n'était pas capable de les conduire dans le pays qu'il leur avait promis, et parce qu'il les haïssait, il les a fait sortir pour les faire mourir dans le désert.
\VS{29}Cependant ils sont ton peuple et ton héritage, que tu as fait sortir par ta grande puissance et par ton bras étendu.
\Chap{10}
\TextTitle{Rappel du remplacement des tables de la loi}
\VerseOne{}En ce temps-là Yahweh me dit~: Taille deux tables de pierre comme les premières, et monte vers moi sur la montagne~; tu feras une arche de bois\FTNT{Ex. 25:10~; 34:1-4.}.
\VS{2}Et j'écrirai sur ces tables les paroles qui étaient sur les premières tables que tu as brisées, et tu les mettras dans l'arche.
\VS{3}Ainsi je fis une arche de bois d'acacia, je taillai deux tables de pierre comme les premières, et je montai sur la montagne, les deux tables dans ma main\FTNT{Ex. 34:4.}.
\VS{4}Et Yahweh écrivit sur ces tables ce qui avait été écrit sur les premières, les dix paroles qu'il avait dites sur la montagne, du milieu du feu, le jour de l'assemblée~; puis Yahweh me les donna.
\VS{5}Je me retournai et je descendis de la montagne~; je mis les tables dans l'arche que j'avais faite, et elles y sont demeurées, comme Yahweh me l'avait ordonné.
\VS{6}Or les enfants d'Israël partirent de Beéroth-Bené-Jaakan pour Moséra. Là mourut Aaron, et il fut enseveli~; Eléazar, son fils, exerça la prêtrise à sa place.
\VS{7}De là ils partirent pour Gudgoda, et de Gudgoda pour Jothbatha, qui est un pays de torrents d'eau.
\VS{8}Or en ce temps-là, Yahweh sépara la tribu de Lévi afin de porter l'arche de l'alliance de Yahweh, de se tenir devant Yahweh, de le servir, et de bénir en son Nom, jusqu'à ce jour.
\VS{9}C'est pourquoi Lévi n'a ni portion ni d'héritage avec ses frères~: Yahweh est son héritage, comme Yahweh, ton Dieu, lui a dit.
\VS{10}Je restai sur la montagne, comme la première fois, quarante jours et quarante nuits. Yahweh m'exauça encore cette fois~; Yahweh ne voulut point te détruire.
\VS{11}Mais Yahweh me dit~: Lève-toi, va, marche devant ce peuple. Qu'ils aillent prendre possession du pays que j'ai juré à leurs pères de leur donner.
\TextTitle{Une alliance basée sur l'amour de Yahweh}
\VS{12}Maintenant donc, ô Israël, que demande de toi Yahweh, ton Dieu, sinon que tu craignes Yahweh, ton Dieu, afin de marcher dans toutes ses voies, d'aimer et de servir Yahweh, ton Dieu, de tout ton cœur, et de toute ton âme~;
\VS{13}de garder les commandements de Yahweh et ses lois que je t'ordonne aujourd'hui, afin que tu sois heureux~?
\VS{14}Voici, les cieux, et les cieux des cieux appartiennent à Yahweh, ton Dieu, la terre et tout ce qu'elle renferme.
\VS{15}Et Yahweh s'est attaché à tes pères, pour les aimer~; et après eux, il vous a choisis, vous, leur postérité, entre tous les peuples, comme vous le voyez aujourd'hui.
\VS{16}Circoncisez donc le prépuce de votre cœur, et vous ne raidirez plus votre cou.
\VS{17}Car Yahweh, votre Dieu, est le Dieu des dieux, le Seigneur des seigneurs\FTNT{Yahweh, le Dieu des dieux et le Seigneur des seigneurs n'est autre que Jésus-Christ, notre Seigneur qui s'est révélé à Jean comme le Seigneur des seigneurs et le Roi des rois (Ap. 19:16).}, le Fort, le Grand, le Puissant et le Redoutable, qui n'a point d'égard à l'apparence des personnes, et qui ne prend point de présents~;
\VS{18}qui fait justice à l'orphelin et à la veuve, qui aime l'étranger et lui donne le pain et le vêtement.
\VS{19}Vous aimerez donc l'étranger~; car vous avez été étrangers dans le pays d'Egypte.
\VS{20}Tu craindras Yahweh, ton Dieu, tu le serviras, tu t'attacheras à lui, et tu jureras par son Nom.
\VS{21}Il est ta louange, il est ton Dieu, qui a fait pour toi des choses grandes et redoutables que tes yeux ont vues.
\VS{22}Tes pères descendirent en Egypte au nombre de soixante-dix âmes~; et maintenant Yahweh, ton Dieu, t'a fait devenir comme les étoiles des cieux, tant tu es en grand nombre.
\Chap{11}
\TextTitle{Exhortation à la reconnaissance et à l'obéissance}
\VerseOne{}Tu aimeras donc Yahweh, ton Dieu, et tu garderas toujours ses lois, ses ordonnances, et ses commandements.
\VS{2}Et reconnaissez aujourd'hui, ce que n'ont point connu ni vu vos fils, le châtiment de Yahweh, votre Dieu, sa grandeur, sa main puissante, et son bras étendu,
\VS{3}ses signes, et les œuvres qu'il a accomplies au milieu de l'Egypte contre Pharaon, roi d'Egypte, et contre tout son pays~;
\VS{4}et ce que Yahweh a fait à l'armée d'Egypte, à ses chevaux et à ses chars, quand il a fait déborder sur eux les eaux de la Mer Rouge, car Yahweh les a détruits jusqu'à ce jour\FTNT{Ex. 14:28.}~;
\VS{5}ce qu'il a fait dans le désert, jusqu'à votre arrivée en ce lieu-ci~;
\VS{6}ce qu'il a fait à Dathan et à Abiram, fils d'Eliab, fils de Ruben, comment la terre ouvrit sa bouche et les engloutit, avec leurs maisons et leurs tentes, et tous les êtres qui les suivaient, au milieu de tout Israël\FTNT{No. 16:1-33.}.
\VS{7}Car ce sont vos yeux qui ont vu toutes les grandes œuvres que Yahweh a faites.
\TextTitle{Les bienfaits de la terre promise sont pour un peuple fidèle}
\VS{8}Vous garderez donc tous les commandements que je vous ordonne aujourd'hui, afin que vous ayez la force d'entrer et de vous emparer du pays où vous allez passer pour en prendre possession,
\VS{9}et afin que vous prolongiez vos jours sur la terre que Yahweh a juré à vos pères de leur donner, ainsi qu'à leur postérité, pays où coulent le lait et le miel.
\VS{10}Car le pays où tu vas entrer afin de le posséder n'est pas comme le pays d'Egypte, d'où vous êtes sortis, où tu semais ta semence, et l'arrosais avec ton pied, comme un jardin potager.
\VS{11}Mais le pays où vous allez passer pour le posséder est un pays de montagnes et de vallées, qui boit les eaux de la pluie du ciel~;
\VS{12}c'est un pays dont Yahweh, ton Dieu, prend soin, et sur lequel Yahweh, ton Dieu, a continuellement ses yeux, du commencement de l'année jusqu'à la fin de l'année.
\VS{13}Il arrivera donc que, si vous obéissez attentivement à mes commandements que je vous ordonne aujourd'hui, si vous aimez Yahweh, votre Dieu, et que vous le servez de tout votre cœur et de toute votre âme,
\VS{14}alors je donnerai à votre pays la pluie en son temps, la pluie de la première et de l'arrière-saison, et tu recueilleras ton blé, ton vin, et ton huile.
\VS{15}Je mettrai aussi dans ton champ de l'herbe pour ton bétail, tu mangeras et tu seras rassasié.
\VS{16}Prenez garde à vous, de peur que votre cœur ne soit trompé, et que vous ne vous détourniez, et ne serviez d'autres dieux, et ne vous prosterniez devant eux.
\VS{17}Et que la colère de Yahweh s'enflamme contre vous, et qu'il ne ferme les cieux, tellement qu'il n'y aurait point de pluie, la terre ne donnerait plus son produit, et vous péririez promptement dans ce bon pays que Yahweh vous donne.
\VS{18}Mettez donc dans votre cœur et dans votre âme ces paroles. Liez-les comme un signe sur vos mains, et qu'elles soient comme des fronteaux entre vos yeux.
\VS{19}Et enseignez-les à vos enfants, en leur en parlant, quand tu seras dans ta maison, quand tu partiras en voyage, quand tu te coucheras et quand tu te lèveras.
\VS{20}Tu les écriras aussi sur les poteaux de ta maison, et sur tes portes.
\VS{21}Afin que vos jours et les jours de vos fils, sur la terre que Yahweh a juré à vos pères de leur donner, soient aussi nombreux que les jours des cieux sur la terre.
\VS{22}Car si vous gardez et si vous pratiquez tous ces commandements que je vous ordonne de faire, aimant Yahweh, votre Dieu, marchant dans toutes ses voies, et vous attachant à lui,
\VS{23}alors Yahweh chassera devant vous toutes ces nations et vous prendrez possession de nations plus grandes et plus puissantes que vous.
\VS{24}Tout lieu que foulera la plante de votre pied sera à vous\FTNT{Jos. 1:3~; 14:9.}~: Votre territoire s'étendra du désert au Liban, et du fleuve, le fleuve de l'Euphrate, jusqu'à la Mer Occidentale.
\VS{25}Aucun homme ne tiendra face à vous. Yahweh, votre Dieu, mettra, comme il vous l'a dit, la frayeur et la crainte de vous sur tout le pays où vous marcherez.
\TextTitle{La malédiction et la bénédiction}
\VS{26}Regardez, je mets aujourd'hui devant vous la bénédiction et la malédiction~:
\VS{27}La bénédiction, si vous obéissez aux commandements de Yahweh, votre Dieu, que je vous ordonne aujourd'hui~;
\VS{28}la malédiction, si vous n'obéissez point aux commandements de Yahweh, votre Dieu, et si vous vous détournez du chemin que je vous ordonne aujourd'hui, pour aller après d'autres dieux que vous ne connaissez point.
\VS{29}Et quand Yahweh, ton Dieu, t'aura fait entrer dans le pays dont tu vas prendre possession, tu prononceras alors les bénédictions, étant sur la montagne de Garizim, et les malédictions, étant sur la montagne d'Ebal.
\VS{30}Ces montagnes ne sont-elles pas de l'autre côté du Jourdain, derrière le chemin du soleil couchant, au pays des Cananéens qui demeurent dans la plaine, vis-à-vis de Guilgal, près des chênes de Moré~?
\VS{31}Car vous allez passer le Jourdain, pour entrer et prendre possession du pays que Yahweh, votre Dieu, vous donne~; vous le posséderez, et vous y habiterez.
\VS{32}Vous garderez et pratiquerez toutes les lois et les ordonnances que je mets aujourd'hui devant vous.
\Chap{12}
\TextTitle{Lois sur les sacrifices offerts au lieu où résidera le Nom de Yahweh}
\VerseOne{}Ce sont ici les lois et les ordonnances que vous garderez et pratiquerez dans le pays que Yahweh, le Dieu de vos pères, vous a donné à posséder, tout le temps que vous vivrez sur cette terre.
\VS{2}Vous détruirez, vous détruirez tous les lieux où les nations que vous allez déposséder servent leurs dieux, sur les hautes montagnes et sur les collines, et sous tout arbre verdoyant.
\VS{3}Vous démolirez aussi leurs autels, vous briserez leurs statues, vous brûlerez au feu leurs asheras, vous mettrez en pièces les images taillées de leurs dieux, et vous ferez périr leur nom de ce lieu-là.
\VS{4}Vous ne ferez pas ainsi à Yahweh, votre Dieu.
\VS{5}Mais vous le chercherez dans sa demeure, et vous irez au lieu que Yahweh, votre Dieu, aura choisi d'entre toutes vos tribus, pour y mettre son Nom.
\VS{6}Et vous y apporterez vos holocaustes, vos sacrifices, vos dîmes, vos offrandes élevées, vos vœux, vos offrandes volontaires de vos mains, et les premiers-nés de votre gros et de votre menu bétail\FTNT{Lé. 17:3-4.}.
\VS{7}Et là, vous mangerez devant Yahweh, votre Dieu, et vous vous réjouirez, vous et vos familles, de toutes les choses auxquelles vous aurez mis la main, et dans lesquelles Yahweh, votre Dieu, vous aura bénis.
\VS{8}Vous ne ferez pas comme nous faisons ici aujourd'hui, où chacun fait ce qui lui semble juste à ses yeux,
\VS{9}car vous n'êtes point encore entrés dans le lieu de repos, et dans l'héritage que Yahweh, votre Dieu, vous donne.
\VS{10}Vous passerez le Jourdain, et vous habiterez dans le pays que Yahweh, votre Dieu, vous donne en héritage~; il vous donnera du repos de tous vos ennemis qui vous entourent, et vous y habiterez en sécurité.
\VS{11}Et il y aura un lieu que Yahweh, votre Dieu, choisira pour y faire habiter son Nom. Vous y apporterez tout ce que je vous ordonne, vos holocaustes, vos sacrifices, vos dîmes, vos offrandes élevées de vos mains, et toutes offrandes de choix pour les vœux que vous aurez voués à Yahweh.
\VS{12}Et là, vous vous réjouirez devant Yahweh, votre Dieu, vous, vos fils et vos filles, vos serviteurs et vos servantes, et le Lévite qui sera dans vos portes~; car il n'a ni part ni héritage avec vous.
\VS{13}Garde-toi d'offrir tes holocaustes dans tous les lieux que tu verras~;
\VS{14}mais tu offriras tes holocaustes dans le lieu que Yahweh choisira dans l'une de tes tribus, et tu y feras tout ce que je t'ordonne.
\VS{15}Toutefois, selon le désir de ton âme, tu pourras tuer et manger de la viande dans toutes tes portes, selon la bénédiction que t'accordera Yahweh, ton Dieu~; celui qui sera impur et celui qui sera pur en mangeront, comme on mange de la gazelle et du cerf.
\VS{16}Seulement, vous ne mangerez point de sang. Tu le répandras sur la terre, comme de l'eau.
\VS{17}Tu ne pourras pas manger dans tes portes la dîme de ton blé, de ton vin, de ton huile, ni les premiers-nés de ton gros et menu bétail, ni aucune de tes offrandes en accomplissement d'un vœu, ni tes offrandes volontaires, ni les offrandes élevées de tes mains.
\VS{18}Mais tu les mangeras devant Yahweh, ton Dieu, au lieu que Yahweh, ton Dieu, choisira~; toi, ton fils, ta fille, ton serviteur et ta servante, et le Lévite qui sera dans tes portes~; et tu te réjouiras devant Yahweh, ton Dieu, de tout ce à quoi tu auras mis la main.
\VS{19}Garde-toi, tout le temps que tu vivras sur la terre, d'abandonner le Lévite.
\VS{20}Quand Yahweh, ton Dieu, aura élargi tes frontières, comme il te l'a promis, et que tu diras~: Je mangerai de la chair, parce que ton âme désirera manger de la chair, tu en mangeras selon tous les désirs de ton âme.
\VS{21}Si le lieu que Yahweh, ton Dieu, aura choisi pour y mettre son Nom, est loin de toi, alors tu tueras de ton gros et menu bétail, comme je te l'ai ordonné, et tu en mangeras dans tes portes selon tous les désirs de ton âme.
\VS{22}Tu en mangeras comme on mange de la gazelle et du cerf~; celui qui sera impur et celui qui sera pur en mangeront également.
\VS{23}Seulement, garde-toi de manger le sang, car le sang c'est l'âme~; et tu ne mangeras point l'âme avec la chair\FTNT{Lé. 7:26.}.
\VS{24}Tu n'en mangeras point~: Tu le répandras sur la terre comme de l'eau.
\VS{25}Tu n'en mangeras point, afin que tu sois heureux, toi et tes enfants après toi, parce que tu auras fait ce qui est droit aux yeux de Yahweh.
\VS{26}Mais tu prendras les choses que tu auras consacrées, qui seront à toi, et ce que tu auras voué, tu les prendras et tu viendras au lieu que Yahweh aura choisi.
\VS{27}Et tu offriras tes holocaustes, la chair et le sang, sur l'autel de Yahweh, ton Dieu~; mais le sang de tes autres sacrifices sera versé sur l'autel de Yahweh, ton Dieu, et tu en mangeras la chair.
\VS{28}Garde et écoute toutes ces paroles que je t'ordonne, afin que tu sois heureux, toi et tes enfants après toi, à jamais, en faisant ce qui est bon et droit aux yeux de Yahweh, ton Dieu.
\TextTitle{Mise en garde contre la séduction et les dieux étrangers}
\VS{29}Quand Yahweh, ton Dieu, aura exterminé de devant toi les nations que tu vas prendre en possession, que tu les auras possédées, et que tu habiteras dans leur pays,
\VS{30} prends garde à toi, de peur que tu ne sois pris au piège après elles, quand elles auront été détruites de devant toi~; et que tu ne recherches leurs dieux, en disant~: Comme ces nations-là servaient leurs dieux, je le ferai aussi tout de même.
\VS{31}Tu ne feras point ainsi à Yahweh, ton Dieu~; car elles ont fait à leurs dieux tout ce qui est en abomination et qui est odieux à Yahweh, et même ils brûlaient au feu leurs fils et leurs filles à leurs dieux.
\VS{32}Vous prendrez garde de faire tout ce que je vous commande. Vous n'y ajouterez rien, et vous n'en retrancherez rien.
\Chap{13}
\TextTitle{Eprouver les faux prophètes, ôter le méchant du milieu de l'assemblée}
\VerseOne{}S'il s'élève au milieu de toi un prophète ou un songeur de songes, qui te donne un signe ou miracle,
\VS{2}et que ce signe ou ce miracle dont il t'a parlé, arrive, et qu'il te dise~: Allons après d'autres dieux que tu ne connais point, et servons-les~!
\VS{3}Tu n'écouteras point les paroles de ce prophète ni de ce songeur de songes, car Yahweh, votre Dieu, vous met à l'épreuve pour savoir si vous aimez Yahweh, votre Dieu, de tout votre cœur et de toute votre âme.
\VS{4}Vous marcherez après Yahweh, votre Dieu, vous le craindrez~; vous garderez ses commandements, vous obéirez à sa voix, vous le servirez, et vous vous attacherez à lui.
\VS{5}Mais on fera mourir ce prophète-là ou ce songeur de songes, parce qu'il a parlé de révolte\FTNT{Le mot «~révolte~» utilisé ici, traduit le terme hébreu «~carah~» et signifie «~apostasie~». L'apostasie est une déviation progressive. C'est tout d'abord l'abandon d'une vérité reçue. Paul, l'apôtre enseigne que deux événements doivent avoir lieu avant le retour du Seigneur sur la terre~: L'apostasie et la révélation de l'homme du péché, le fils de perdition, c'est-à-dire l'Antichrist (2 Th. 2:1-3~; 2 Ti. 4:1).} contre Yahweh, votre Dieu, qui vous a fait sortir du pays d'Egypte et vous a délivrés de la maison de servitude, pour vous conduire loin de la voie que Yahweh, votre Dieu, vous a ordonné de marcher. Tu ôteras le méchant du milieu de toi.
\VS{6}Quand ton frère, fils de ta mère, ou ton fils, ou ta fille, ou ta femme bien-aimée, ou ton intime ami, qui est comme ton âme, t'incitera, en te disant en secret~: Allons, et servons d'autres dieux, que tu n'as point connus, ni tes pères,
\VS{7}d'entre les dieux des peuples qui sont autour de vous, près ou loin de toi, d'une extrémité de la terre jusqu'à l'autre,
\VS{8}tu ne t'accorderas pas avec lui, et tu ne l'écouteras point. Ton œil ne le regardera pas avec pitié, tu ne l'épargneras point, et tu ne le cacheras point.
\VS{9}Mais tu le feras mourir, tu le feras mourir\FTNT{Répétition du mot «~mourir~», voir commentaire en Ge. 2:17}~; ta main sera la première sur lui pour le mettre à mort, et ensuite la main de tout le peuple.
\VS{10}Tu le lapideras avec des pierres, et il mourra, parce qu'il a cherché à t'éloigner loin de Yahweh, ton Dieu, qui t'a fait sortir du pays d'Egypte, de la maison de servitude.
\VS{11}Afin que tout Israël entende et craigne, et que l'on ne fasse plus une action aussi méchante au milieu de toi.
\TextTitle{Jugement des villes idolâtres}
\VS{12}Si tu entends dire dans l'une des villes que Yahweh, ton Dieu, t'a données pour y habiter~:
\VS{13}Des hommes, fils de Bélial, sont sortis du milieu de toi, et ont chassé les habitants de leur ville, en disant~: Allons et servons d'autres dieux, des dieux que tu ne connais point~!
\VS{14}Tu chercheras, tu examineras, tu t'enquerras bien. Et si c'est la vérité, si la chose est établie, si cette abomination a été faite au milieu de toi,
\VS{15}tu frapperas, tu frapperas\FTNT{Répétition du mot «~frapperas~». Dans les écrits hébraïque, la répétition de mots est utilisée afin d'accentuer une action, pour appuyer un fait précis et le renforcer.} du tranchant de l'épée les habitants de cette ville, tu la dévoueras par interdit, et tu passeras le bétail au fil de l'épée.
\VS{16}Tu assembleras tout son butin au milieu de la place, et tu brûleras entièrement au feu cette ville et tout son butin, devant Yahweh, ton Dieu~: Elle sera pour toujours un monceau de ruines, sans être jamais rebâtie.
\VS{17}Rien de ce qui sera dévoué ne s'attachera à ta main, afin que Yahweh revienne de l'ardeur de sa colère, qu'il te fasse miséricorde et grâce, et qu'il te multiplie, comme il a juré à tes pères,
\VS{18}si tu obéis à la voix de Yahweh, ton Dieu, en gardant tous ses commandements que je t'ordonne aujourd'hui, et en faisant ce qui est droit aux yeux de Yahweh, ton Dieu.
\Chap{14}
\TextTitle{Israël, peuple mis en part}
\VerseOne{}Vous êtes les enfants de Yahweh, votre Dieu. Vous ne vous ferez aucune incision, et vous ne vous ferez point de place chauve entre les yeux pour aucun mort.
\VS{2}Car tu es un peuple saint pour Yahweh, ton Dieu~; et Yahweh t'a choisi pour que tu lui sois un peuple qui lui appartienne entre tous les peuples qui sont sur la face de la terre.
\TextTitle{Lois sur l'alimentation}
\VS{3}Tu ne mangeras d'aucune chose abominable.
\VS{4}Ce sont ici les bêtes que vous mangerez~: Le bœuf, la brebis et la chèvre~;
\VS{5}le cerf, la gazelle et le daim~; le bouquetin, le chevreuil, la chèvre sauvage, et le mouflon.
\VS{6}Vous mangerez donc toute bête qui a le sabot divisé, le pied fendu, et qui rumine.
\VS{7}Mais vous ne mangerez point de ceux qui ruminent seulement, ou qui ont le sabot divisé et le pied fendu seulement, comme le chameau, le lièvre et le lapin, car ils ruminent bien, mais ils n'ont pas le sabot qui est fendu~: Ils vous seront impurs.
\VS{8}Le porc aussi, car il a le sabot fendu, mais il ne rumine point~: Il vous sera impur. Vous ne mangerez point de leur chair, et vous ne toucherez point à leur cadavre.
\VS{9}Voici ce que vous mangerez de tout ce qui est dans les eaux~: Vous mangerez de tout ce qui a des nageoires et des écailles.
\VS{10}Mais vous ne mangerez point de ce qui n'a ni nageoires ni écailles~: Cela vous sera impur.
\VS{11}Vous mangerez tout oiseau pur.
\VS{12}Mais voici ceux dont vous ne mangerez point~: L'aigle, l'orfraie, l'aigle de mer~;
\VS{13}le vautour, le milan, et l'autour, selon leur espèce~;
\VS{14}le corbeau, selon son espèce~;
\VS{15}l'autruche, le hibou, la mouette, l'épervier, selon son espèce~;
\VS{16}le chat-huant, la chouette et le cygne~;
\VS{17}le cormoran, le pélican, le plongeon~;
\VS{18}la cigogne, le héron, selon leur espèce, la huppe et la chauve-souris.
\VS{19}Et tout reptile qui vole sera impur pour vous~; on n'en mangera point.
\VS{20}Mais vous mangerez de tout ce qui vole et qui est pur.
\VS{21}Vous ne mangerez aucun cadavre~; tu le donneras à l'étranger qui sera dans tes portes, et il le mangera, ou tu le vendras à un étranger~; car tu es un peuple saint pour Yahweh, ton Dieu. Tu ne feras point cuire le chevreau dans le lait de sa mère.
\TextTitle{Lois sur les dîmes\FTNT{No. 18:21-32.}}
\VS{22}Tu prendras la dîme, tu prendras la dîme\FTNT{Il est question ici de la dîme que les Hébreux consommaient chaque année.} de tout le produit de ta semence, de ce qui sortira de ton champ, chaque année.
\VS{23}Et tu mangeras devant Yahweh, ton Dieu, au lieu qu'il aura choisi pour y faire habiter son Nom, la dîme de ton blé, de ton vin et de ton huile, et les premiers-nés de ton gros et menu bétail, afin que tu apprennes à toujours craindre Yahweh, ton Dieu.
\VS{24}Mais quand le chemin sera trop long pour que tu puisses les transporter, parce que le lieu que Yahweh, ton Dieu, aura choisi pour y mettre son Nom, sera trop loin de toi, lorsque Yahweh, ton Dieu, t'aura béni,
\VS{25}alors tu l'échangeras contre de l'argent, tu serreras l'argent dans ta main, et tu iras au lieu que Yahweh, ton Dieu, aura choisi.
\VS{26}Et tu donneras l'argent contre tout ce que ton âme désirera, des bœufs, des brebis, du vin et des liqueurs fortes, tout ce que ton âme demandera, tu le mangeras devant Yahweh, ton Dieu, et tu te réjouiras, toi et ta famille.
\VS{27}Tu n'abandonneras point le Lévite qui sera dans tes portes, parce qu'il n'a ni portion ni héritage avec toi\FTNT{Ce verset fait référence à la première dîme qui devait être donnée aux Lévites (Voir commentaire en No. 18:21 et Mal. 3:1).}.
\VS{28}Au bout de trois ans, tu feras sortir toutes les dîmes de tes produits de cette année-là, et tu les déposeras dans tes portes.
\VS{29}Alors le Lévite, qui n'a ni portion ni héritage avec toi, l'étranger, l'orphelin, et la veuve qui seront dans tes portes, viendront, mangeront et se rassasieront, afin que Yahweh, ton Dieu, te bénisse dans toute l'œuvre que tu feras de tes mains.
\Chap{15}
\TextTitle{Lois sur l'année de relâche~: la justice et la bonté de Yahweh}
\VerseOne{}Tous les sept ans, tu célébreras l'année de relâche\FTNT{Ex. 21:2, Jé. 34:14.}.
\VS{2}Et c'est ici la manière de célébrer l'année de relâche. Que tout homme ayant droit d'exiger quelque chose que ce soit, qu'il puisse exiger de son prochain, donnera relâche, et ne l'exigera point de son prochain ni de son frère, quand on aura proclamé le relâche, en l'honneur de Yahweh.
\VS{3}Tu l'exigeras de l'étranger~; mais ta main relâchera tout ce qui t'appartiendra chez ton frère,
\VS{4}afin qu'il n'y ait point d'indigent chez toi, car Yahweh te bénira, te bénira\FTNT{Voir commentaire en Ge. 2:16} abondamment dans le pays que Yahweh, ton Dieu, te donnera à posséder pour héritage~;
\VS{5}pourvu que tu obéisses, que tu obéisses\FTNT{Voir commentaire en Ge. 2:16} bien à la voix de Yahweh, ton Dieu, en prenant garde de pratiquer tous ces commandements que je t'ordonne aujourd'hui.
\VS{6}Parce que Yahweh, ton Dieu, te bénira comme il te l'a promis, tu prêteras sur gage à beaucoup de nations, et tu n'emprunteras point sur gage~; tu domineras sur beaucoup de nations, et elles ne domineront point sur toi.
\VS{7}Quand un de tes frères sera indigent au milieu de toi, dans l'une de tes portes, dans le pays que Yahweh, ton Dieu, te donne, tu n'endurciras point ton cœur, et tu ne fermeras point ta main à ton frère indigent.
\VS{8}Mais tu lui ouvriras, tu lui ouvriras\FTNT{Voir commentaire en Ge. 2:16} ta main, et tu lui prêteras, lui prêteras\FTNT{Voir commentaire en Ge. 2:16} sur gage autant qu'il en aura besoin pour son indigence, dans laquelle il se trouvera.
\VS{9}Prends garde à toi, de peur que tu n'aies dans ton cœur quelque chose de Bélial, et que tu ne dises~: La septième année, l'année du relâche approche~! Et que ton œil soit méchant envers ton frère indigent, afin de ne rien lui donner et qu'il ne crie à Yahweh contre toi, et qu'il n'y ait du péché en toi.
\VS{10}Tu lui donneras, lui donneras\FTNT{Voir commentaire en Ge. 2:16} et que ton cœur ne lui donne point à regret~; car à cause de cela, Yahweh, ton Dieu, te bénira dans toutes tes œuvres, et dans tout ce à quoi tu mettras tes mains.
\VS{11}Car il y aura toujours des indigents dans le pays~; c'est pourquoi je t'ordonne, et je te dis~: Tu ouvriras, tu ouvriras\FTNT{Voir commentaire en Ge. 2:16} ta main à ton frère, à l'affligé, et à l'indigent dans ton pays.
\TextTitle{Loi sur les esclaves}
\VS{12}Quand l'un de tes frères Hébreux, homme ou femme, te sera vendu, il te servira six ans~; mais la septième année, tu le renverras libre de chez toi.
\VS{13}Et quand tu le renverras libre de chez toi, tu ne le renverras point à vide.
\VS{14}Tu chargeras, chargeras\FTNT{Voir commentaire en Ge. 2:16} de quelque chose de ton menu bétail, de ton aire, de ton pressoir, et tu lui donneras de ce que Yahweh, ton Dieu, t'aura béni.
\VS{15}Et tu te souviendras que tu as été esclave au pays d'Egypte, et que Yahweh, ton Dieu, t'en a racheté~; et c'est pour cela que je t'ordonne ceci aujourd'hui.
\VS{16}Mais s'il arrive qu'il te dise~: Je ne sortirai point de chez toi~; parce qu'il t'aime, toi et ta maison, et qu'il se trouve bien chez toi,
\VS{17}alors tu prendras un poinçon\FTNT{Ex. 21:6} et tu lui perceras l'oreille contre la porte, et il sera ton serviteur pour toujours. Tu en feras de même à ta servante.
\VS{18}Ce ne sera point, à tes yeux, dur de le renvoyer libre de chez toi, car il t'a servi six ans, ce qui est le double salaire d'un mercenaire~; et Yahweh, ton Dieu, te bénira en tout ce que tu feras.
\TextTitle{Loi sur les premiers-nés des animaux}
\VS{19}Tu consacreras à Yahweh, ton Dieu, tout premier-né mâle qui naîtra parmi ton gros et ton menu bétail. Tu ne travailleras point avec le premier-né de ton bœuf, et tu ne tondras point le premier-né de tes brebis\FTNT{Ex. 13:2.}.
\VS{20}Tu le mangeras, toi et ta famille, chaque année devant Yahweh, ton Dieu, dans le lieu que Yahweh aura choisi.
\VS{21}Mais s'il a quelque défaut, boiteux ou aveugle, ou qu'il ait quelque autre mauvais défaut, tu ne le sacrifieras point à Yahweh, ton Dieu.
\VS{22}Mais tu le mangeras dans tes portes~; celui qui sera impur et celui qui sera pur en mangeront également, comme on mange de la gazelle et du cerf.
\VS{23}Seulement, tu n'en mangeras point le sang~; mais tu le répandras sur la terre comme de l'eau.
\Chap{16}
\TextTitle{La Pâque et la fête des pains sans levain}
\VerseOne{}Observe le mois des épis, et fais la Pâque à Yahweh, ton Dieu~; car c'est au mois des épis que Yahweh, ton Dieu, t'a fait sortir, de nuit, d'Egypte\FTNT{Ex. 12:2-29.}.
\VS{2}Et tu sacrifieras la Pâque à Yahweh, ton Dieu, du gros et du menu bétail, au lieu que Yahweh choisira pour y faire habiter son Nom.
\VS{3}Tu ne mangeras point de pain levé, mais tu mangeras sept jours des pains sans levain, du pain d'affliction, parce que tu es sorti précipitamment du pays d'Egypte, afin que tous les jours de ta vie tu te souviennes du jour où tu es sorti du pays d'Egypte.
\VS{4}Et il ne se verra point de levain chez toi, sur tout le territoire de ton pays pendant sept jours\FTNT{1 Co. 5:7.}~; et aucune chair que tu sacrifieras le soir du premier jour ne restera jusqu'au matin.
\VS{5}Tu ne pourras point sacrifier la Pâque dans l'une de tes portes que Yahweh, ton Dieu, te donne~;
\VS{6}mais c'est au lieu que Yahweh, ton Dieu, choisira pour y faire habiter son Nom, que tu sacrifieras la Pâque, le soir, au coucher du soleil, moment où tu es sorti d'Egypte.
\VS{7}Tu la cuiras et tu la mangeras dans le lieu que Yahweh, ton Dieu, aura choisi. Et le matin, tu t'en retourneras et tu t'en iras dans tes tentes.
\VS{8}Pendant six jours, tu mangeras des pains sans levain~; et le septième jour, il y aura une assemblée solennelle à Yahweh, ton Dieu~: Tu ne feras aucune œuvre.
\TextTitle{La fête des semaines}
\VS{9}Tu te compteras sept semaines~; tu commenceras à compter ces sept semaines dès que la faucille sera mise dans les blés.
\VS{10}Puis tu feras la fête des semaines à Yahweh, ton Dieu, en présentant l'offrande volontaire de ta main, que tu donneras, selon que Yahweh, ton Dieu, t'aura béni.
\VS{11}Et tu te réjouiras devant Yahweh, ton Dieu, toi, ton fils et ta fille, ton serviteur et ta servante, le Lévite qui sera dans tes portes, l'étranger, l'orphelin et la veuve qui seront au milieu de toi, dans le lieu que Yahweh, ton Dieu, aura choisi pour y faire habiter son Nom.
\VS{12}Et tu te souviendras que tu as été esclave en Egypte, et tu garderas et pratiqueras ces lois.
\TextTitle{La fête des tabernacles}
\VS{13}Tu feras la fête des tabernacles pendant sept jours, après que tu auras recueilli le produit de ton aire et de ton pressoir.
\VS{14}Et tu te réjouiras à cette fête, toi, ton fils et ta fille, ton serviteur et ta servante, le Lévite, l'étranger, l'orphelin, et la veuve qui seront dans tes portes.
\VS{15}Tu célébreras la fête pendant sept jours à Yahweh, ton Dieu, dans le lieu que Yahweh aura choisi~; car Yahweh, ton Dieu, te bénira dans toute ta récolte, et dans tout le travail de tes mains, et tu vivras dans la joie.
\TextTitle{Offrandes à Yahweh selon ses moyens}
\VS{16}Trois fois l'an, tout mâle d'entre vous se présentera devant Yahweh, ton Dieu, dans le lieu qu'il aura choisi, à la fête des pains sans levain, à la fête des semaines, et à la fête des tabernacles. On ne se présentera point devant Yahweh à vide.
\VS{17}Mais chacun donnera à proportion de ce qu'il aura, selon la bénédiction de Yahweh,ton Dieu, qu'il t'aura donnée. 
\TextTitle{Des juges établis pour faire respecter la justice de Yahweh}
\VS{18}Tu t'établiras des juges et des officiers dans toutes les villes que Yahweh, ton Dieu, te donne, selon tes tribus~; et ils jugeront le peuple d'un juste jugement.
\VS{19}Tu ne te détourneras point de la justice, tu ne prêteras point attention à l'apparence des personnes, et tu ne recevras point de présents, car les présents aveuglent les yeux des sages et corrompent les paroles des justes.
\VS{20}Tu suivras fermement la justice, afin que tu vives et que tu possèdes le pays que Yahweh, ton Dieu, te donne.
\TextTitle{Prescriptions sur les cultes}
\VS{21}Tu ne planteras point d'arbre d'Asherah\FTNT{Bois ou arbre d'Asherah~: Il est question d'un objet en bois, pieu sacré ou arbre utilisé dans le culte d'Astarté, l'épouse de Baal (Ex. 34:13~; De. 7:5~; De. 12:3~; Jg. 3:7~; Jg. 6:25-30~; 1 R. 14:15-23).}, près de l'autel que tu feras à Yahweh, ton Dieu.
\VS{22}Tu ne dresseras point non plus de statue~; Yahweh, ton Dieu, hait ces choses.
\Chap{17}
\VerseOne{}Tu ne sacrifieras à Yahweh, ton Dieu, ni bœuf, ni agneau qui ait quelque défaut ou quelque chose de mauvais~; car c'est une abomination à Yahweh, ton Dieu.
\TextTitle{Punition de l'idolâtrie}
\VS{2}S'il se trouve au milieu de toi dans l'une des villes que Yahweh, ton Dieu, te donne, un homme ou une femme faisant ce qui est mal aux yeux de Yahweh, ton Dieu, en transgressant son alliance,
\VS{3}et allant servir d'autres dieux et se prosterner devant eux, devant le soleil, devant la lune, ou devant toute l'armée des cieux, ce que je n'ai pas ordonné~;
\VS{4}et que cela t'aura été rapporté, et que tu l'auras entendu, alors tu feras des recherches avec soin. Si la chose est vraie, que le fait est établi, et que cette abomination a été commise en Israël,
\VS{5}alors tu feras sortir vers tes portes cet homme ou cette femme, qui aura fait cette mauvaise action, cet homme, dis-je, ou cette femme, et tu les lapideras avec des pierres, et ils mourront.
\VS{6}On fera mourir sur la parole de deux témoins ou de trois témoins\FTNT{Mt. 18:15-17.}, celui qui doit être mis à mort~; il ne sera pas mis à mort sur la parole d'un seul témoin.
\VS{7}La main des témoins sera la première sur lui pour le faire mourir, et ensuite la main de tout le peuple. Et ainsi tu ôteras le mal du milieu de toi.
\TextTitle{Soumission aux autorités}
\VS{8}Quand une affaire te paraîtra trop difficile à juger entre meurtre et meurtre, entre cause et cause, entre plaie et plaie, qui sont des affaires de procès dans tes portes, alors tu te lèveras et tu monteras au lieu que Yahweh, ton Dieu, aura choisi.
\VS{9}Et tu iras vers les prêtres, les Lévites, et vers le juge qu'il y aura en ce temps-là, tu les consulteras, et ils te feront connaître et te déclareront la sentence du jugement.
\VS{10}Tu feras conformément à la sentence qu'ils t'auront déclarée de leur bouche dans le lieu que Yahweh aura choisi, et tu prendras garde de faire tout ce qu'ils t'enseigneront.
\VS{11}Tu feras conformément à la loi qu'ils t'auront enseignée de leur bouche et selon la sentence qu'ils t'auront prononcée~; tu ne te détourneras ni à droite ni à gauche de ce qu'ils t'auront déclaré.
\VS{12}Mais l'homme qui agira par orgueil et n'obéira pas au prêtre qui se tient là pour servir Yahweh, ton Dieu, ou au juge, cet homme mourra. Tu ôteras le mal d'Israël,
\VS{13}et tout le peuple l'entendra et craindra, et n'agira plus par orgueil.
\TextTitle{Instructions sur la royauté}
\VS{14}Quand tu seras entré dans le pays que Yahweh, ton Dieu, te donne, que tu le posséderas, que tu y demeureras, et que tu diras~: J'établirai un roi sur moi, comme toutes les nations qui sont autour de moi,
\VS{15}tu ne manqueras pas de t'établir pour roi celui que Yahweh, ton Dieu, aura choisi, tu établiras un roi du milieu de tes frères, tu ne pourras point désigner un homme étranger qui ne soit pas ton frère\FTNT{Dans sa prescience, Yahweh savait que le peuple se détournerait de ses voies et réclamerait un roi, à l'identique des nations alentour. (1 S. 8). Or depuis leur sortie d'Egypte, seul Yahweh était leur Dieu et leur Roi.}.
\VS{16}Seulement, il n'aura pas de nombreux chevaux, et il ne ramènera point le peuple en Egypte pour augmenter le nombre de chevaux~; car Yahweh vous a dit~: Vous ne retournerez plus par ce chemin.
\VS{17}Il n'aura point un grand nombre de femmes, afin que son cœur ne se détourne point~; et qu'il n'accumule point beaucoup d'argent et d'or.
\VS{18}Et dès qu'il sera assis sur le trône de son royaume, il écrira pour lui, dans un livre, une copie de cette loi, qu'il prendra des prêtres, les Lévites.
\VS{19}Il l'aura auprès de lui et la lira tous les jours de sa vie, afin qu'il apprenne à craindre Yahweh, son Dieu, à prendre garde à toutes les paroles de cette loi, et à ces ordonnances, afin de les pratiquer~;
\VS{20}afin que son cœur ne s'élève point au-dessus de ses frères, et qu'il ne se détourne point de ce commandement ni à droite ni à gauche~; afin qu'il prolonge ses jours dans son royaume, lui et ses fils, au milieu d'Israël.
\Chap{18}
\TextTitle{Héritage des Lévites et des prêtres}
\VerseOne{}Les prêtres, les Lévites, et même toute la tribu de Lévi, n'auront ni part ni héritage avec Israël~; ils mangeront les sacrifices consumés par le feu de Yahweh, et de son héritage.
\VS{2}Ils n'auront point d'héritage parmi leurs frères~: Yahweh sera leur héritage, comme il leur a dit.
\VS{3}Or c'est ici le droit que les prêtres prendront du peuple, sur ceux qui offriront un sacrifice, un bœuf ou un agneau~: On donnera au prêtre l'épaule, les mâchoires et l'estomac.
\VS{4}Tu lui donneras les prémices de ton blé, de ton vin et de ton huile, et les prémices de la toison de tes brebis.
\VS{5}Car Yahweh, ton Dieu, l'a choisi d'entre toutes les tribus, afin qu'il se tienne devant lui, et qu'il fasse le service au Nom de Yahweh, lui et ses fils, à toujours.
\VS{6}Or quand le Lévite viendra de l'une de tes portes, de tout lieu où il habite en Israël, et qu'il viendra selon tout le désir de son âme, au lieu que Yahweh aura choisi,
\VS{7}et qu'il fera le service au Nom de Yahweh, son Dieu, comme tous ses frères Lévites qui se tiennent là devant Yahweh,
\VS{8}il mangera une portion égale à la leur, outre ce qu'il aura vendu de son patrimoine.
\TextTitle{Les abominations des nations interdites en Israël}
\VS{9}Quand tu seras entré dans le pays que Yahweh, ton Dieu, te donne, tu n'apprendras point à faire les abominations de ces nations-là.
\VS{10}Qu'on ne trouve au milieu de toi personne qui fasse passer par le feu son fils ou sa fille, personne qui pratique la divination, l'astrologie, l'augure, la sorcellerie,
\VS{11}ni d'enchanteur qui use d'enchantements, personne qui consulte les médiums ou disent la bonne aventure, personne qui interroge les morts\FTNT{Yahweh interdit tout contact avec le monde des esprits et des démons. Le croyant qui accepte l'Evangile comprendra sans peine et simplement en obéissant à la Parole que ce domaine est interdit. Voir Ex. 22:18~; Lé. 19:26~; Lé. 19:31~; Lé. 20:6~; Lé. 20:27~; Es. 8:19~; 2 Ch. 33:6~; Ac. 19:13-20.}.
\VS{12}Car quiconque fait ces choses est en abomination à Yahweh~; et à cause de ces abominations, Yahweh, ton Dieu, va chasser ces nations-là devant toi.
\VS{13}Tu seras intègre avec Yahweh, ton Dieu.
\VS{14}Car ces nations, que tu vas déposséder, écoutent les pronostiqueurs et les devins~; mais à toi, Yahweh, ton Dieu, ne le permet point.
\TextTitle{Annonce sur la venue du Messie}
\VS{15}Yahweh, ton Dieu, te suscitera du milieu de toi, d'entre tes frères, un prophète comme moi\FTNT{Moïse a annoncé la venue d'un prophète comme lui, c'est-à-dire un prophète de la délivrance et de l'exode. Ce prophète n'est autre que Jésus-Christ qui nous délivre de l'emprise de Satan et nous sort du monde pour nous amener dans la Nouvelle Jérusalem (Jn. 14:2~; Col. 1:13). Notons qu'au moment de la transfiguration, Elie et Moïse parlaient avec Jésus de son départ («~exodus~» en grec~; Lu. 9:31).}: Vous l'écouterez.
\VS{16}Selon tout ce que tu as demandé à Yahweh, ton Dieu, à Horeb, le jour de l'assemblée, quand tu disais~: Que je n'entende plus la voix de Yahweh, mon Dieu, et que je ne voie plus ce grand feu, de peur de mourir.
\VS{17}Alors Yahweh me dit~: Ce qu'ils ont dit est bien.
\VS{18}Je leur susciterai un prophète comme toi du milieu de leurs frères, je mettrai mes paroles dans sa bouche, et il leur dira tout ce que je lui ordonnerai.
\VS{19}Et il arrivera que si un homme n'écoute pas mes paroles qu'il dira en mon Nom, je lui en demanderai compte.
\TextTitle{Comment éprouver les prophètes~?}
\VS{20}Mais le prophète qui agira de manière orgueilleuse pour dire en mon Nom une parole que je ne lui aurai point ordonnée de dire, ou qui parlera au nom des autres dieux, ce prophète-là mourra.
\VS{21}Et si tu dis dans ton cœur~: Comment connaîtrons-nous la parole que Yahweh n'aura point dite~?
\VS{22}Quand le prophète parlera au Nom de Yahweh, et que ce qu'il aura dit n'arrivera pas, ce sera une parole que Yahweh ne lui aura point dite. C'est par orgueil que le prophète l'a dite~: N'aie point peur de lui.
\Chap{19}
\TextTitle{Les villes de refuge\FTNTT{No. 35:1-34.}}
\VerseOne{}Quand Yahweh, ton Dieu, aura exterminé les nations dont Yahweh, ton Dieu, te donne le pays, et que tu les auras dépossédées et que tu demeureras dans leurs villes, et dans leurs maisons,
\VS{2}alors tu sépareras trois villes au milieu du pays que Yahweh, ton Dieu, te donne à posséder.
\VS{3}Tu établiras des chemins, et tu diviseras en trois le territoire de ton pays, que Yahweh, ton Dieu, te donnera en héritage. Ce sera afin que tout meurtrier s'y enfuie.
\VS{4}Or voici comment on procédera envers le meurtrier qui s'enfuira pour sauver sa vie. Celui qui aura frappé son prochain involontairement, et sans l'avoir haï dans le passé~;
\VS{5}ainsi, si quelqu'un va couper du bois dans la forêt avec une autre personne, la hache à la main pour couper du bois, si le fer glisse du manche, trouve son compagnon, et s'il en meurt~; il s'enfuira alors dans une de ces villes, afin qu'il vive.
\VS{6}De peur que celui qui venge le sang ne poursuive le meurtrier, parce que son cœur est échauffé, et qu'il ne le rattrape, si le chemin est trop long, et ne le frappe à mort, alors qu'il ne mérite pas la mort, parce qu'il ne le haïssait pas auparavant\FTNT{No. 35:1-34.}.
\VS{7}C'est pourquoi je t'ordonne, en disant~: Sépare-toi trois villes.
\VS{8}Lorsque Yahweh, ton Dieu, aura élargi tes frontières, comme il l'a juré à tes pères, et qu'il t'aura donné tout le pays qu'il a promis à tes pères de te donner,
\VS{9}parce que tu auras gardé et mis en pratique tous ces commandements que je t'ordonne aujourd'hui, en aimant Yahweh, ton Dieu, et en marchant toujours dans ses voies, alors tu ajouteras encore trois villes à ces trois-là,
\VS{10}afin que le sang innocent ne soit versé au milieu du pays que Yahweh, ton Dieu, te donne en héritage, et que tu ne sois pas coupable de meurtre.
\VS{11}Mais si un homme hait son prochain, lui dresse un piège, se lève contre lui et frappe cette personne, de sorte qu'il meure, et qu'il s'enfuit dans l'une de ces villes,
\VS{12}alors les anciens de sa ville l'enverront saisir, et le livreront entre les mains du vengeur de sang, afin qu'il meure.
\VS{13}Ton œil ne l'épargnera point, mais tu feras disparaître d'Israël le sang innocent, et tu seras heureux.
\VS{14}Tu ne déplaceras point les bornes de ton prochain, fixées par tes ancêtres, dans l'héritage que tu posséderas, dans le pays que Yahweh, ton Dieu, te donne à posséder.
\TextTitle{Résoudre des différends}
\VS{15}Un seul témoin ne sera point valable contre un homme pour constater un crime ou un péché, quel que soit le péché~; mais sur la parole de deux témoins ou de trois témoins la chose sera valable.
\VS{16}Quand un faux témoin s'élèvera contre un homme pour témoigner contre lui d'un crime,
\VS{17}ces deux hommes en contestation comparaîtront devant Yahweh, en présence des prêtres et des juges qui seront là en ce temps-là.
\VS{18}Et les juges feront des recherches avec soin. Si le témoin est un faux témoin, s'il a donné un faux témoignage contre son frère,
\VS{19}tu lui feras comme il avait pensé faire à son frère. Tu ôteras ainsi le mal du milieu de toi.
\VS{20}Et les autres entendront et craindront, et ne feront plus une chose aussi méchante au milieu de toi.
\VS{21}Ton œil ne l'épargnera point~: Vie pour vie, œil pour œil, dent pour dent, main pour main, pied pour pied.
\Chap{20}
\TextTitle{Instructions diverses pour la guerre}
\VerseOne{}Quand tu iras à la guerre contre tes ennemis, et que tu verras des chevaux et des chars, et un peuple plus grand que toi, tu ne les craindras point, car Yahweh, ton Dieu, qui t'a fait monter du pays d'Egypte, est avec toi.
\VS{2}Et quand vous vous approcherez du combat, le prêtre s'avancera et parlera au peuple.
\VS{3}Et leur dira~: Ecoute Israël~: Vous vous approchez aujourd'hui pour combattre vos ennemis. Que votre cœur ne faiblisse pas~; ne craignez point, ne soyez point effrayés et ne soyez point terrifiés face à eux.
\VS{4}Car Yahweh, votre Dieu, marche avec vous, pour combattre vos ennemis, pour vous sauver.
\VS{5}Les officiers parleront au peuple, en disant~: Qui est l'homme qui a bâti une maison neuve et ne l'a pas inaugurée~? Qu'il s'en aille et retourne dans sa maison, de peur qu'il ne meure dans la bataille et qu'un autre homme ne l'inaugure.
\VS{6}Qui est celui qui a planté une vigne et n'en a point encore cueilli le fruit~? Qu'il s'en aille et retourne dans sa maison, de peur qu'il ne meure dans la bataille et qu'un autre homme n'en cueille le fruit.
\VS{7}Qui est celui qui a fiancé une femme et ne l'a point prise en mariage~? Qu'il s'en aille et retourne dans sa maison, de peur qu'il ne meure dans la bataille et qu'un autre homme ne la prenne en mariage.
\VS{8}Et les officiers continueront à parler au peuple, et diront~: Si un homme a peur et est timide, qu'il s'en aille et retourne dans sa maison, de peur que le cœur de ses frères ne devienne craintif comme le sien.
\VS{9}Quand les officiers auront fini de parler au peuple, ils désigneront les chefs des armées à la tête du peuple.
\VS{10}Quand tu t'approcheras d'une ville pour lui faire la guerre, tu l'inviteras à la paix.
\VS{11}Et si elle te donne une réponse de paix et s'ouvre à toi, tout le peuple qui s'y trouvera te sera tributaire et te servira.
\VS{12}Si elle ne fait pas la paix avec toi et qu'elle te fait la guerre, alors tu l'assiègeras.
\VS{13}Et quand Yahweh, ton Dieu, l'aura livrée entre tes mains, tu frapperas tous les mâles au fil de l'épée.
\VS{14}Mais les femmes, les enfants, le bétail, tout ce qui sera dans la ville, et tout son butin, tu le prendras pour toi et tu mangeras le butin de tes ennemis, que Yahweh, ton Dieu, t'aura donné.
\VS{15}Tu feras ainsi à toutes les villes qui sont très éloignées de toi, et qui ne sont point des villes de ces nations.
\VS{16}Mais dans les villes de ces peuples que Yahweh, ton Dieu, te donne en héritage, tu ne laisseras vivre personne qui respire.
\VS{17}Car tu ne manqueras point de les dévouer par interdit~: Héthiens, Amoréens, Cananéens, Phéréziens, Héviens, et Jébusiens, comme Yahweh, ton Dieu, te l'a ordonné.
\VS{18}Afin qu'ils ne vous enseignent point à faire toutes les abominations qu'ils font pour leurs dieux, et que vous ne péchiez point contre Yahweh, votre Dieu.
\VS{19}Quand tu assiégeras une ville durant plusieurs jours, en lui faisant la guerre pour la saisir, tu ne détruiras point les arbres à coups de hache, tu t'en nourriras et tu ne les couperas point, car l'arbre des champs est-il un homme pour être assiégé par toi~?
\VS{20}Mais seulement tu détruiras et tu couperas les arbres que tu sauras ne point être des arbres fruitiers, et tu construiras des retranchements contre la ville qui te fait la guerre, jusqu'à ce qu'elle tombe.
\Chap{21}
\TextTitle{Lois sur le meurtre anonyme}
\VerseOne{}S'il se trouve sur la terre que Yahweh, ton Dieu, te donne à posséder, un homme tué, étendu dans un champ, sans que l'on sache qui l'a frappé,
\VS{2}tes anciens et tes juges sortiront, et ils mesureront de l'homme tué jusqu'aux villes qui sont autour.
\VS{3}Puis les anciens de la ville la plus proche de l'homme tué prendront une génisse du troupeau qui n'a pas travaillé et qui n'a point tiré au joug.
\VS{4}Et les anciens de cette ville feront descendre cette génisse vers un torrent intarissable, où on ne travaille ni ne sème~; et là, ils briseront la nuque à la génisse dans le torrent.
\VS{5}Et les prêtres, fils de Lévi, s'approcheront~; car Yahweh, ton Dieu, les a choisis pour qu'ils le servent, et qu'ils bénissent au Nom de Yahweh~; et leur bouche doit décider de toute contestation et toute blessure.
\VS{6}Et tous les anciens de cette ville, qui seront les plus proches de l'homme qui aura été tué, laveront leurs mains sur la génisse à laquelle on aura brisé la nuque dans le torrent.
\VS{7}Et prenant la parole, ils diront~: Nos mains n'ont point répandu ce sang et nos yeux ne l'ont point vu.
\VS{8}Ô Yahweh~! Sois propice à ton peuple d'Israël que tu as racheté~; ne lui impute point le sang innocent qui a été répandu au milieu de ton peuple d'Israël~; et le meurtre sera expié pour eux.
\VS{9}Et tu ôteras le sang innocent du milieu de toi, en faisant ce qui est droit aux yeux de Yahweh.
\TextTitle{Lois sur le mariage et l'héritage}
\VS{10}Quand tu iras en guerre contre tes ennemis, que Yahweh, ton Dieu, les aura livrés entre tes mains, et que tu en auras emmené des captifs,
\VS{11}si tu vois parmi les captifs une femme belle de figure, et que tu désires la prendre pour femme,
\VS{12}alors tu la conduiras à l'intérieur de ta maison, et elle rasera sa tête et fera ses ongles,
\VS{13}elle ôtera les vêtements de sa captivité, elle demeurera dans ta maison, et pleurera son père et sa mère durant un mois. Puis tu iras vers elle, tu l'épouseras, et elle sera ta femme.
\VS{14}Si il arrive qu'elle ne te plaise plus, tu la renverras où elle voudra, mais tu ne la vendras certainement pas pour de l'argent ni la traiteras en esclave, parce que tu l'auras humiliée.
\VS{15}Quand un homme, qui a deux femmes, aime l'une et hait l'autre, si celle qu'il aime et celle qu'il hait enfantent des fils, et que le fils aîné est de celle qui est haïe,
\VS{16}alors, le jour où il laissera en héritage ce qu'il aura, il ne pourra pas reconnaître comme premier-né le fils de celle qu'il aime, à la place du fils de celle qui est haïe, et qui est le premier-né.
\VS{17}Mais il reconnaîtra pour premier-né le fils de celle qui est haïe, et il lui donnera la double portion de tout ce qui s'y trouvera être à lui~; car il est le commencement de sa vigueur, le droit d'aînesse lui appartient.
\TextTitle{Le fils indocile sous la loi\FTNTT{cp. Lu. 15:11-23.}}
\VS{18}Si un homme a un fils indocile et rebelle, n'obéissant point à la voix de son père, ni à la voix de sa mère, et qui, bien qu'ils l'aient châtié, ne les écoute point,
\VS{19}alors le père et la mère le prendront et le mèneront aux anciens de sa ville, et à la porte du lieu de sa demeure.
\VS{20}Et ils diront aux anciens de sa ville~: Voici notre fils qui est indocile et rebelle, qui n'obéit point à notre voix, et qui se livre à l'excès et à l'ivrognerie.
\VS{21}Et tous les gens de la ville le lapideront avec des pierres, et il mourra. Tu ôteras le mal du milieu de toi, afin que tout Israël entende et craigne.
\VS{22}Si un homme a commis un péché digne de mort, et qu'on le fait mourir, et que tu l'aies pendu à un bois,
\VS{23}son cadavre ne passera point la nuit sur le bois~; mais tu ne manqueras point de l'ensevelir le même jour, car celui qui est pendu est malédiction de Dieu\FTNT{Ga. 3:13.}, et tu ne souilleras point la terre que Yahweh, ton Dieu, te donne en héritage.
\Chap{22}
\TextTitle{Lois sur la vie en société}
\VerseOne{}Si tu vois le bœuf ou la brebis de ton frère s'égarer, tu ne t'en cacheras point, tu ne manqueras point de les ramener à ton frère.
\VS{2}Si ton frère ne demeure point près de toi, et que tu ne le connais point, tu les recueilleras dans ta maison et il sera chez toi jusqu'à ce que ton frère les cherche~; et alors tu les lui rendras.
\VS{3}Tu feras de même pour son âne, tu feras de même pour son vêtement, et tu feras de même pour tout ce que ton frère aura perdu et que tu trouveras~; tu ne devras point t'en détourner.
\VS{4}Si tu vois l'âne de ton frère ou son bœuf tombé dans le chemin, tu ne t'en détourneras point, et tu ne manqueras point de le relever.
\VS{5}La femme ne portera point l'habit d'un homme ni l'homme ne se vêtira point d'un habit de femme~; car celui qui fait ces choses est en abomination à Yahweh, ton Dieu\FTNT{Dans ce passage, Yahweh condamne le travestisme. Cette pratique était répandue chez les Cananéens. Le travestisme consiste à adopter le comportement, les habitudes sociales et la tenue vestimentaire du sexe opposé dans le but de lui ressembler.}.
\VS{6}Si tu rencontres sur le chemin, sur un arbre ou sur la terre, un nid d'oiseaux, ayant des petits ou des œufs, et la mère couchée sur les petits ou les œufs, tu ne prendras point la mère et les petits,
\VS{7}mais tu ne manqueras point de laisser aller la mère et tu ne prendras que les petits, afin que tu sois heureux et que tu prolonges tes jours.
\VS{8}Si tu bâtis une maison neuve, tu feras un parapet tout autour de ton toit, afin que tu ne mettes point de sang sur ta maison, si quelqu'un tombait de là.
\TextTitle{Lois sur les mélanges}
\VS{9}Tu ne sèmeras point dans ta vigne diverses sortes de grains~; de peur que le tout, à savoir les grains, que tu auras semés, et le rapport de ta vigne, ne soit souillé. 
\VS{10}Tu ne laboureras point avec un âne et un bœuf ensemble.
\VS{11}Tu ne te vêtiras point d'un tissu mélangé de laine et de lin ensemble.
\VS{12}Tu te feras des franges aux quatre pans du vêtement dont tu te couvriras.
\TextTitle{Lois sur la virginité, l'adultère et la fidélité}
\VS{13}Si un homme a pris une femme et est allé vers elle, et qu'il la haïsse,
\VS{14}et qu'il lui impute des choses qui donnent l'occasion de parler d'elle et de la diffamer, en disant~: J'ai pris cette femme, et quand je me suis approché d'elle, je ne l'ai point trouvé vierge,
\VS{15}alors le père et la mère de la jeune femme prendront et produiront les signes de la virginité de la jeune femme devant les anciens de la ville, à la porte.
\VS{16}Et le père de la jeune femme dira aux anciens~: J'ai donné ma fille à cet homme pour femme, et il l'a haïe~;
\VS{17}et voici, il lui impute des choses qui lui donnent l'occasion de parler d'elle, disant~: Je n'ai point trouvé ta fille vierge. Cependant, voici les signes de la virginité de ma fille. Et ils étendront le drap devant les anciens de la ville.
\VS{18}Alors les anciens de la ville prendront le mari, et le châtieront~;
\VS{19}et parce qu'il aura répandu une mauvaise réputation sur une vierge d'Israël, ils le condamneront à une amende de cent sicles d'argent, qu'ils donneront au père de la jeune femme. Elle sera sa femme, et il ne pourra pas la répudier, tant qu'il vivra.
\VS{20}Mais si la chose est vraie, si la jeune femme ne s'est point trouvée vierge,
\VS{21}alors ils feront sortir la jeune femme à l'entrée de la maison de son père~; les gens de sa ville la lapideront de pierres et elle mourra, car elle a commis une infamie en Israël, en se prostituant dans la maison de son père. Tu ôteras le mal du milieu de toi.
\VS{22}Si l'on trouve un homme couché avec une femme mariée, ils mourront tous les deux, l'homme qui a couché avec la femme, et la femme aussi. Tu ôteras ainsi le mal d'Israël.
\VS{23}Si une jeune fille vierge est fiancée à un homme, et qu'un homme la rencontre dans la ville, et couche avec elle,
\VS{24}vous les conduirez tous deux à la porte de la ville, vous les lapiderez de pierres, et ils mourront~; la jeune fille, parce qu'elle n'a point crié étant dans la ville, et l'homme parce qu'il a humilié la femme de son prochain. Tu ôteras le mal du milieu de toi.
\VS{25}Si l'homme rencontre dans les champs la jeune fille fiancée, et que l'homme lui fait violence et couche avec elle, alors l'homme qui aura couché avec elle mourra lui seul.
\VS{26}Mais tu ne feras rien à la jeune fille~; la jeune fille n'a point commis de péché digne de mort, car c'est comme si un homme s'élevait contre son prochain et lui ôtait la vie.
\VS{27}Parce que l'ayant trouvée dans les champs, la jeune fille fiancée a pu crier, sans que personne ne l'ait délivrée.
\VS{28}Si un homme rencontre une jeune fille vierge non fiancée, lui fait violence et couche avec elle, et qu'ils soient découverts,
\VS{29}l'homme qui aura couché avec elle donnera au père de la jeune fille cinquante sicles d'argent~; et il la prendra pour femme, parce qu'il l'a humiliée, et il ne pourra point la répudier, tant qu'il vivra.
\VS{30}Un homme ne prendra point la femme de son père ni ne découvrira le pan de la robe de son père.
\Chap{23}
\TextTitle{Lois sur l'accès à l'assemblée de Yahweh}
\VerseOne{}Celui dont les testicules ont été écrasés ou l'urètre coupé n'entrera point dans l'assemblée de Yahweh.
\VS{2}Le bâtard\FTNT{Le mot bâtard, «~mamzer~» en hébreu, désigne l'enfant illégitime, celui issu de l'inceste, celui né d'une population mélangée ou d'un père Juif et d'une mère païenne, et inversement.} n'entrera point dans l'assemblée de Yahweh~; même sa dixième génération n'entrera point dans l'assemblée de Yahweh.
\VS{3}L'Ammonite et le Moabite n'entreront point dans l'assemblée de Yahweh, même leur dixième génération, à jamais,
\VS{4}parce qu'ils ne sont point venus à votre rencontre avec du pain et de l'eau, sur le chemin, lorsque vous sortiez d'Egypte, et parce qu'ils ont engagé à prix d'argent contre vous Balaam, fils de Beor, de Pethor en Mésopotamie, pour qu'il vous maudisse.
\VS{5}Mais Yahweh, ton Dieu, n'a point voulu écouter Balaam~; et Yahweh, ton Dieu, a changé la malédiction en bénédiction, parce que Yahweh, ton Dieu, t'aime.
\VS{6}Tu ne chercheras jamais, tant que tu vivras, leur paix ni leur bien.
\VS{7}Tu n'auras point en abomination l'Edomite, car il est ton frère~; tu n'auras point en abomination l'Egyptien, car tu as été étranger dans son pays~:
\VS{8}Les enfants qui leur naîtront à la troisième génération entreront dans l'assemblée de Yahweh.
\TextTitle{La sainteté et la justice dans le camp de Yahweh}
\VS{9}Quand le camp sortira contre tes ennemis, garde-toi de toute chose mauvaise.
\VS{10}S'il y a parmi vous un homme qui ne soit point pur, par suite d'un accident nocturne, il sortira hors du camp, et n'entrera point dans le camp.
\VS{11}Et sur le soir, il se lavera dans l'eau, et dès que le soleil sera couché, il rentrera dans le camp.
\VS{12}Tu auras un endroit hors du camp, et tu sortiras là dehors.
\VS{13}Tu auras un pieu parmi tes bagages, et quand tu voudras aller dehors, tu creuseras, puis tu recouvriras tes excréments.
\VS{14}Car Yahweh, ton Dieu, marche au milieu de ton camp pour te délivrer et pour livrer tes ennemis devant toi~; que tout ton camp soit saint, afin qu'il ne voie chez toi aucune chose honteuse, et qu'il ne se détourne point de toi.
\VS{15}Tu ne livreras point à son maître l'esclave qui se sera sauvé chez toi d'auprès de son maître.
\VS{16}Il demeurera avec toi, au milieu de toi, dans le lieu qu'il choisira, dans l'une de tes villes, là où bon lui semblera~: Tu ne l'opprimeras point.
\VS{17}Il n'y aura, parmi les filles d'Israël, aucune prostituée, et il n'y aura, parmi les fils d'Israël, aucun qui se prostitue.
\VS{18}Tu n'apporteras point dans la maison de Yahweh, ton Dieu, le salaire d'une prostituée, ni le prix d'un chien, pour quelque vœu que ce soit~; car tous les deux sont en abomination devant Yahweh, ton Dieu.
\VS{19}Tu n'exigeras aucun intérêt à ton frère, ni intérêt pour de l'argent, ni intérêt pour des vivres, ni intérêt pour quelque chose que ce soit que l'on prête avec intérêt.
\VS{20}Tu prêteras avec intérêt à l'étranger, mais tu ne prêteras point avec intérêt à ton frère, afin que Yahweh, ton Dieu, te bénisse dans tout ce que ta main entreprendra dans le pays où tu vas entrer en possession.
\TextTitle{Vœux faits à Yahweh}
\VS{21}Si tu fais un vœu à Yahweh, ton Dieu, tu ne tarderas point à l'accomplir, car Yahweh, ton Dieu, ne manquerait point de te le redemander, ainsi il y aurait du péché en toi.
\VS{22}Mais si tu t'abstiens de faire un vœu, il n'y aura pas de péché en toi.
\VS{23}Mais tu prendras garde de faire ce qui sortira de tes lèvres, l'offrande volontaire que tu auras vouée à Yahweh, ton Dieu, et que ta bouche aura prononcée.
\TextTitle{Lois diverses}
\VS{24}Si tu entres dans la vigne de ton prochain, tu pourras manger des raisins selon ton appétit, jusqu'à en être rassasié~; mais tu n'en mettras point dans ton vase.
\VS{25}Si tu entres dans les blés de ton prochain, tu pourras arracher des épis avec ta main~; mais tu n'agiteras point la faucille sur les blés de ton prochain.
\Chap{24}
\TextTitle{Loi sur le divorce}
\VerseOne{}Quand un homme aura pris et épousé une femme, s'il arrive qu'elle ne trouve pas grâce à ses yeux, parce qu'il aura trouvé en elle quelque chose de honteux, il lui écrira une lettre de divorce, et après la lui avoir remise en main, il la renverra de sa maison.
\VS{2}Elle sortira de sa maison, s'en ira, et elle pourra devenir la femme d'un autre homme.
\VS{3}Si ce dernier homme la hait, écrit une lettre de divorce, la lui donne dans sa main, et la renvoie de sa maison, ou que ce dernier homme qui l'a prise pour femme, meure,
\VS{4}alors son premier mari qui l'avait renvoyée ne pourra pas la reprendre pour femme après avoir été souillée, car c'est une abomination devant Yahweh, ainsi tu ne feras point pécher le pays que Yahweh, ton Dieu, te donne en héritage.
\TextTitle{Lois diverses sur l'organisation de la société}
\VS{5}Quand un homme aura nouvellement épousé une femme, il n'ira point à la guerre, et on ne lui imposera aucune charge~; il en sera libre pour sa maison pendant un an, et il réjouira la femme qu'il a prise.
\VS{6}On ne prendra point pour gage les deux meules, pas même la meule de dessus~; parce qu'on prendrait pour gage la vie.
\VS{7}Si l'on trouve un homme qui ait dérobé l'un de ses frères, l'un des enfants d'Israël, qui en ait fait son esclave ou qui l'ait vendu, ce voleur mourra. Tu ôteras le mal du milieu de toi.
\VS{8}Prends garde à la plaie de la lèpre, afin de bien observer et de faire tout ce que les prêtres, les Lévites, vous enseigneront~; vous prendrez garde de faire selon ce que je leur ai ordonné.
\VS{9}Souviens-toi de ce que Yahweh, ton Dieu, fit à Marie, en chemin, après votre sortie d'Egypte.
\TextTitle{Lois en faveur des nécessiteux}
\VS{10}Lorsque tu feras à ton prochain un prêt quelconque, tu n'entreras point dans sa maison pour prendre son gage~;
\VS{11}mais tu te tiendras dehors, et l'homme à qui tu feras le prêt t'apportera le gage dehors.
\VS{12}Si cet homme est pauvre, tu ne te coucheras point ayant encore son gage~;
\VS{13}tu ne manqueras point de lui rendre le gage dès que le soleil sera couché, afin qu'il se couche dans son vêtement et qu'il te bénisse~; et cela te sera imputé à justice devant Yahweh, ton Dieu.
\VS{14}Tu n'opprimeras point le mercenaire, le pauvre et l'indigent, d'entre tes frères, ou d'entre les étrangers qui demeurent dans ton pays, dans tes portes.
\VS{15}Tu lui donneras son salaire le jour même avant que le soleil se couche~; car il est pauvre, et son désir s'y porte. Afin qu'il ne crie point contre toi à Yahweh, et que tu ne pèches point.
\VS{16}On ne fera point mourir les pères pour les fils, et on ne fera point mourir les fils pour les pères~; mais on fera mourir chacun pour son péché.
\VS{17}Tu ne feras pas d'injustice à l'étranger ni à l'orphelin, et tu ne prendras point en gage le vêtement de la veuve.
\VS{18}Et tu te souviendras que tu as été esclave en Egypte, et que Yahweh, ton Dieu, t'a racheté de là~; c'est pourquoi je t'ordonne de faire ces choses.
\VS{19}Quand tu moissonneras dans ton champ, et que tu auras oublié une gerbe dans ton champ, tu ne retourneras point la prendre~: Elle sera pour l'étranger, pour l'orphelin et pour la veuve, afin que Yahweh, ton Dieu, te bénisse dans toute l'œuvre de tes mains.
\VS{20}Quand tu secoueras tes oliviers, tu n'y retourneras point pour cueillir ce qui reste aux branches~: Ce sera pour l'étranger, pour l'orphelin et pour la veuve.
\VS{21}Quand tu vendangeras ta vigne, tu ne grappilleras point après~: Ce sera pour l'étranger, pour l'orphelin et pour la veuve.
\VS{22}Et tu te souviendras que tu as été esclave dans le pays d'Egypte~; c'est pourquoi je t'ordonne de faire ces choses.
\Chap{25}
\TextTitle{Le juste justifié et le méchant condamné}
\VerseOne{}Quand il y aura un différend entre des hommes et qu'ils viendront en jugement afin qu'on les juge, on justifiera le juste, et on condamnera le méchant.
\VS{2}Si le méchant mérite d'être battu, le juge le fera jeter par terre et frapper en sa présence par un certain nombre de coups, selon l'exigence de son crime.
\VS{3}Il le fera battre de quarante coups, pas plus, de peur que si l'on continuait à le frapper avec plus de coups, ton frère ne soit méprisé à tes yeux.
\VS{4}Tu n'emmuselleras point ton bœuf lorsqu'il foulera le grain.
\TextTitle{Loi sur la continuité de la postérité}
\VS{5}Quand des frères demeureront ensemble, et que l'un d'entre eux mourra sans fils, alors la femme du défunt ne se mariera point dehors avec un homme qui est étranger, mais son beau-frère viendra vers elle, la prendra pour femme, et l'épousera comme son beau-frère.
\VS{6}Et le premier-né qu'elle enfantera succédera au frère mort et portera son nom, afin que son nom ne soit point effacé d'Israël.
\VS{7}Et s'il ne plaît pas à cet homme-là de prendre sa belle-sœur, alors sa belle-sœur montera à la porte vers les anciens\FTNT{Ru. 4:1-10}, et dira~: Mon beau-frère refuse de relever le nom de son frère en Israël, et ne veut point m'épouser par droit de beau-frère.
\VS{8}Alors les anciens de la ville l'appelleront et lui parleront. S'il demeure ferme, et qu'il dit~: Il ne me plaît point de la prendre,
\VS{9}alors sa belle-sœur s'approchera de lui à la vue des anciens, lui ôtera son soulier du pied, et lui crachera au visage. Et prenant la parole, elle dira~: C'est ainsi qu'on fera à l'homme qui ne bâtit point la maison de son frère.
\VS{10}Et son nom sera appelé en Israël la maison du déchaussé.
\TextTitle{L'abomination sévèrement et justement punie}
\VS{11}Quand des hommes se querelleront ensemble, l'un contre l'autre, si la femme de l'un s'approche pour délivrer son mari de la main de celui qui le frappe, et qu'étendant sa main elle saisisse ses parties intimes,
\VS{12}tu lui couperas la main, et ton œil ne l'épargnera point.
\VS{13}Tu n'auras point dans ton sac deux poids différents, un grand et un petit.
\VS{14}Il n'y aura point dans ta maison deux épha différents, un grand et un petit\FTNT{Lé. 19:35-37.}.
\VS{15}Mais tu auras un poids exact et juste, tu auras un épha exact et juste, afin que tes jours se prolongent sur la terre que Yahweh, ton Dieu, te donne.
\VS{16}Car celui qui fait ces choses, celui qui commet une injustice, est en abomination à Yahweh, ton Dieu.
\TextTitle{Yahweh confirme le sort d'Amalek}
\VS{17}Souviens-toi ce que te fit Amalek en chemin, quand vous sortiez d'Egypte\FTNT{Ex. 17:8.},
\VS{18}comment il est venu te rencontrer sur le chemin, et, sans aucune crainte de Dieu, attaqua par derrière ceux qui étaient fatigués, quand toi-même tu étais épuisé.
\VS{19}Quand Yahweh, ton Dieu, t'aura accordé du repos de tous tes ennemis qui t'entourent, dans le pays que Yahweh, ton Dieu, te donne en héritage afin que tu le possèdes, alors tu effaceras la mémoire d'Amalek de dessous les cieux~: Ne l'oublie point.
\Chap{26}
\TextTitle{La loi des prémices\FTNTT{cp. Ex. 23:16-19.}}
\VerseOne{}Quand tu seras entré dans le pays que Yahweh, ton Dieu, te donne en héritage, et quand tu le posséderas et y habiteras,
\VS{2}alors tu prendras des prémices de tous les fruits que tu retireras de la terre dans le pays que Yahweh, ton Dieu, te donne~; tu les mettras dans une corbeille, et tu iras au lieu que Yahweh, ton Dieu, choisira pour y faire habiter son Nom\FTNT{Ex. 23:16-19.}.
\VS{3}Et tu viendras vers le prêtre qui sera en ce temps-là, et tu lui diras~: Je déclare aujourd'hui à Yahweh, ton Dieu, que je suis entré dans le pays que Yahweh a juré à nos pères de nous donner.
\VS{4}Et le prêtre prendra la corbeille de ta main, et la posera devant l'autel de Yahweh, ton Dieu.
\VS{5}Puis tu prendras la parole, et tu diras devant Yahweh, ton Dieu~: Mon père était un Araméen qui périssait, il descendit en Egypte avec un petit nombre de gens, il y séjourna et il y devint une nation grande, puissante, et nombreuse.
\VS{6}Puis les Egyptiens nous maltraitèrent, nous humilièrent, et nous imposèrent une dure servitude.
\VS{7}Nous criâmes à Yahweh, le Dieu de nos pères. Yahweh entendit notre voix, et il vit notre souffrance, notre travail, et notre oppression.
\VS{8}Et Yahweh nous fit sortir d'Egypte, à main forte et à bras étendu, avec une grande frayeur, avec des signes et des miracles.
\VS{9}Et il nous a conduits dans ce lieu, et nous a donné ce pays où coulent le lait et le miel.
\VS{10}Maintenant donc voici, j'apporte les prémices des fruits de la terre que tu m'as donnée, ô Yahweh~! Tu les poseras devant Yahweh, ton Dieu, et tu te prosterneras devant Yahweh, ton Dieu.
\VS{11}Et tu te réjouiras de tout le bien que Yahweh, ton Dieu, t'aura donné, et à ta maison, toi et le Lévite, et l'étranger qui sera au milieu de toi.
\VS{12}Quand tu auras achevé de lever toute la dîme de ta récolte, la troisième année, l'année de la dîme, tu la donneras au Lévite, à l'étranger, à l'orphelin, et à la veuve~; ils en mangeront dans tes portes, et ils en seront rassasiés.
\VS{13}Tu diras en la présence de Yahweh, ton Dieu~: J'ai fait disparaître de ma maison ce qui est consacré, et je l'ai donné au Lévite, à l'étranger, à l'orphelin, et à la veuve, selon tous tes commandements que tu m'as ordonnés~; je n'ai transgressé ni oublié aucun de tes commandements.
\VS{14}Je n'en ai point mangé dans mon affliction, et je n'en ai rien fait disparaître pour un usage impur, et je n'en ai point donné pour un mort~; j'ai obéi à la voix de Yahweh, mon Dieu~; j'ai fait selon tout ce que tu m'avais ordonné.
\VS{15}Regarde de ta sainte demeure, des cieux, et bénis ton peuple d'Israël et la terre que tu nous as donnée, comme tu l'avais juré à nos pères, pays où coulent le lait et le miel.
\VS{16}Aujourd'hui, Yahweh, ton Dieu, t'ordonne de mettre en pratique ces lois et ces ordonnances~; prends garde de les faire de tout ton cœur et de toute ton âme.
\VS{17}Tu as fait promettre aujourd'hui à Yahweh qu'il sera ton Dieu, pour que tu marches dans ses voies, que tu observes ses lois, ses commandements et ses ordonnances, et que tu obéisses à sa voix.
\VS{18}Et aujourd'hui, Yahweh t'a fait promettre que tu seras un peuple précieux, comme il te l'a dit, et que tu observeras tous ses commandements,
\VS{19}pour qu'il te donne sur toutes les nations qu'il a créées la supériorité en louange, en renom, et en beauté, et pour que tu sois un peuple saint à Yahweh, ton Dieu, comme il te l'a dit.
\Chap{27}
\TextTitle{La loi gravée sur des pierres au mont Ebal}
\VerseOne{}Or Moïse et les anciens d'Israël ordonnèrent au peuple, en disant~: Gardez tous les commandements que je vous ordonne aujourd'hui.
\VS{2}Le jour où vous aurez traversé le Jourdain, pour entrer dans le pays que Yahweh, ton Dieu, te donne, tu dresseras de grandes pierres, et tu les enduiras de chaux.
\VS{3}Puis tu écriras sur elles toutes les paroles de cette loi, quand tu auras traversé le Jourdain, pour entrer dans le pays que Yahweh, ton Dieu, te donne, pays où coulent le lait et le miel, comme te l'a dit Yahweh, le Dieu de tes pères.
\VS{4}Quand donc vous aurez traversé le Jourdain, vous dresserez ces pierres-là sur le mont Ebal, selon ce que je vous ordonne aujourd'hui, et tu les enduiras de chaux.
\VS{5}Tu bâtiras aussi là un autel à Yahweh, ton Dieu~; un autel, dis-je, de pierres, sur lesquelles tu ne lèveras point le fer.
\VS{6}Tu bâtiras l'autel de Yahweh, ton Dieu, de pierres entières. Tu y offriras des holocaustes à Yahweh, ton Dieu~;
\VS{7}tu y offriras aussi des offrandes de paix\FTNT{Voir commentaire en Lé. 3:1.}, et tu mangeras là et te réjouiras devant Yahweh, ton Dieu.
\VS{8}Et tu écriras sur ces pierres toutes les paroles de cette loi, en les gravant bien distinctement.
\TextTitle{Les malédictions prononcées sur le mont Ebal}
\VS{9}Et Moïse et les prêtres, les Lévites, parlèrent à tout Israël, en disant~: Ecoute et garde le silence, Israël~! Aujourd'hui, tu es devenu le peuple de Yahweh, ton Dieu.
\VS{10}Tu obéiras à la voix de Yahweh, ton Dieu, et tu feras ses commandements et ses lois que je t'ordonne aujourd'hui.
\VS{11}Moïse ordonna au peuple ce jour-là, disant~:
\VS{12}Quand vous aurez traversé le Jourdain, Siméon, Lévi, Juda, Issacar, Joseph, et Benjamin, se tiendront sur le mont Garizim, pour bénir le peuple~;
\VS{13}et Ruben, Gad, Aser, Zabulon, Dan et Nephthali, se tiendront sur le mont Ebal, pour maudire.
\VS{14}Et les Lévites prendront la parole, et diront à haute voix à tous les hommes d'Israël~:
\VS{15}Maudit soit l'homme qui fait une image taillée ou une image en métal fondu, car c'est une abomination à Yahweh, œuvre des mains d'un artisan, et qui la met dans un lieu secret~! Et tout le peuple répondra, et dira~: Amen~!
\VS{16}Maudit soit celui qui méprise son père et sa mère~! Et tout le peuple dira~: Amen~!
\VS{17}Maudit soit celui qui déplace les bornes de son prochain~! Et tout le peuple dira~: Amen~!
\VS{18}Maudit soit celui qui égare un aveugle dans le chemin~! Et tout le peuple dira~: Amen~!
\VS{19}Maudit soit celui qui fait injustice à l'étranger, à l'orphelin, et à la veuve~! Et tout le peuple dira~: Amen~!
\VS{20}Maudit soit celui qui couche avec la femme de son père, car il découvre le pan de la robe de son père~! Et tout le peuple dira~: Amen~!
\VS{21}Maudit soit celui qui couche avec une bête~! Et tout le peuple dira~: Amen~!
\VS{22}Maudit soit celui qui couche avec sa sœur, fille de son père, ou fille de sa mère~! Et tout le peuple dira~: Amen~!
\VS{23}Maudit soit celui qui couche avec sa belle-mère~! Et tout le peuple dira~: Amen~!
\VS{24}Maudit soit celui qui frappe son prochain en secret~! Et tout le peuple dira~: Amen~!
\VS{25}Maudit soit celui qui reçoit un présent pour mettre à mort un homme, en versant le sang innocent~! Et tout le peuple dira~: Amen~!
\VS{26}Maudit soit celui qui n'accomplit point les paroles de cette loi et ne les met pas en pratique~! Et tout le peuple dira~: Amen~!
\Chap{28}
\TextTitle{Les bénédictions accompagnent l'obéissance}
\VerseOne{}Or il arrivera que si tu écoutes attentivement la voix de Yahweh, ton Dieu, et que tu prennes garde de pratiquer tous ses commandements que je t'ordonne aujourd'hui, Yahweh, ton Dieu, te donnera la supériorité sur toutes les nations de la terre.
\VS{2}Voici toutes les bénédictions qui viendront sur toi, et qui t'atteindront, quand tu obéiras à la voix de Yahweh, ton Dieu~:
\VS{3}Tu seras béni dans la ville, et tu seras aussi béni aux champs.
\VS{4}Le fruit de tes entrailles, le fruit de ta terre, le fruit de tes troupeaux, les portées de ton gros et de ton menu bétail seront bénis.
\VS{5}Ta corbeille et ta huche seront bénies.
\VS{6}Tu seras béni en entrant, et tu seras béni en sortant.
\VS{7}Yahweh fera que tes ennemis qui s'élèveront contre toi seront battus devant toi, ils sortiront contre toi par un chemin, et ils s'enfuiront devant toi par sept chemins.
\VS{8}Yahweh ordonnera à la bénédiction d'être avec toi dans tes greniers et dans tout ce à quoi tu mettras ta main~; il te bénira dans le pays que Yahweh, ton Dieu, te donne.
\VS{9}Yahweh t'établira pour lui être un peuple saint, comme il te l'a juré, quand tu garderas les commandements de Yahweh, ton Dieu, et que tu marcheras dans ses voies.
\VS{10}Et tous les peuples de la terre verront que tu es appelé du Nom de Yahweh, et ils te craindront.
\VS{11}Yahweh te fera abonder de biens dans le fruit de tes entrailles, le fruit de tes troupeaux, et le fruit de ton sol, sur la terre que Yahweh a juré à tes pères de te donner.
\VS{12}Yahweh t'ouvrira son bon trésor, les cieux, pour donner à ton pays la pluie en sa saison et pour bénir tout le travail de tes mains~; tu prêteras à beaucoup de nations, et tu n'emprunteras point.
\VS{13}Yahweh te mettra à la tête et non à la queue, tu seras toujours en haut et jamais en bas, lorsque tu obéiras aux commandements de Yahweh, ton Dieu, que je t'ordonne aujourd'hui, afin que tu prennes garde de les faire,
\VS{14}et que tu ne te détournes ni à droite ni à gauche de toutes les paroles que je t'ordonne aujourd'hui, pour aller après d'autres dieux et pour les servir.
\TextTitle{Les malédictions accompagnent la désobéissance}
\VS{15}Mais si tu n'obéis point à la voix de Yahweh, ton Dieu, pour prendre garde de pratiquer tous ses commandements et ses lois que je t'ordonne aujourd'hui, voici toutes les malédictions qui viendront sur toi, et qui t'atteindront~:
\VS{16}Tu seras maudit dans la ville, et tu seras maudit dans les champs.
\VS{17}Ta corbeille et ta huche seront maudites.
\VS{18}Le fruit de tes entrailles, le fruit de ta terre, les portées de ton gros et de ton menu bétail seront maudits.
\VS{19}Tu seras maudit à ton entrée, et tu seras maudit à ta sortie.
\VS{20}Yahweh enverra sur toi la malédiction, la confusion, et la ruine dans tout ce à quoi tu mettras ta main et que tu feras, jusqu'à ce que tu sois détruit, et que tu périsses promptement, à cause de la méchanceté de tes pratiques, par lesquelles tu m'auras abandonné.
\VS{21}Yahweh fera que la peste s'attachera à toi, jusqu'à ce qu'elle te consume sur la terre où tu vas entrer pour en prendre possession.
\VS{22}Yahweh te frappera de tuberculose, de fièvre, d'inflammation, de chaleur brûlante, de l'épée, de sécheresse et de rouille, qui te poursuivront jusqu'à ce que tu périsses.
\VS{23}Les cieux sur ta tête seront d'airain, et la terre sous toi sera de fer.
\VS{24}Yahweh te donnera pour pluie à ton pays de la poussière et de la poudre, qui descendra des cieux sur toi jusqu'à ce que tu sois détruit.
\VS{25}Yahweh fera que tu seras battu devant tes ennemis~; tu sortiras par un chemin contre eux, et tu t'enfuiras devant eux par sept chemins~; et tu seras tremblant face à tous les royaumes de la terre.
\VS{26}Ton cadavre sera la viande de tous les oiseaux des cieux et des bêtes de la terre~; et il n'y aura personne qui les effraye.
\VS{27}Yahweh te frappera de l'ulcère d'Egypte, d'hémorroïdes, de gale, et de teigne, dont tu ne pourras guérir.
\VS{28}Yahweh te frappera de folie, d'aveuglement, et d'égarement d'esprit~;
\VS{29}et tu tâtonneras en plein midi comme tâtonne un aveugle dans l'obscurité, tu ne prospéreras pas dans tes voies, et tu seras opprimé et dépouillé tous les jours, et il n'y aura personne pour venir te sauver.
\VS{30}Tu fianceras une femme, mais un autre homme couchera avec elle et la violera~; tu bâtiras une maison, mais tu ne l'habiteras point~; tu planteras une vigne, mais tu n'en jouiras point.
\VS{31}Ton bœuf sera tué sous tes yeux, et tu n'en mangeras point~; ton âne sera enlevé devant toi, et on ne te le rendra point~; tes brebis seront livrées à tes ennemis, et il n'y aura personne pour te sauver.
\VS{32}Tes fils et tes filles seront livrés à un autre peuple, tes yeux le verront, et languiront tout le jour après eux, et tu n'auras aucun pouvoir en ta main.
\VS{33}Un peuple que tu n'auras point connu mangera le fruit de ta terre et tout ton travail, et tu seras opprimé et écrasé tous les jours.
\VS{34}Tu deviendras fou à cause de ce que tu verras de tes yeux.
\VS{35}Yahweh te frappera d'un ulcère malin sur les genoux et sur les cuisses dont tu ne pourras guérir, il t'en frappera depuis la plante du pied jusqu'au sommet de ta tête.
\VS{36}Yahweh te fera marcher, toi et ton roi que tu auras établi sur toi, vers une nation que tu n'auras point connue, ni toi ni tes pères. Et là, tu serviras d'autres dieux, du bois et de la pierre.
\VS{37}Et tu seras un sujet d'étonnement, de proverbes, de railleries, parmi tous les peuples vers lesquels Yahweh t'aura emmené.
\VS{38}Tu jetteras beaucoup de semence dans ton champ, et tu recueilleras peu, car les sauterelles la consumeront.
\VS{39}Tu planteras des vignes et tu les cultiveras~; mais tu n'en boiras point le vin et tu n'en recueilleras rien, car les vers la mangeront.
\VS{40}Tu auras des oliviers sur tout le territoire~; mais tu ne t'oindras point d'huile, car tes olives perdront leurs fruits.
\VS{41}Tu engendreras des fils et des filles~; mais ils ne seront pas à toi, car ils iront en captivité.
\VS{42}Les insectes posséderont tous tes arbres et le fruit de ta terre.
\VS{43}L'étranger qui sera au milieu de toi montera toujours plus au-dessus de toi, et toi, tu descendras toujours plus bas.
\VS{44}Il te prêtera, et tu ne lui prêteras point~; il sera à la tête, et tu seras à la queue.
\VS{45}Toutes ces malédictions viendront sur toi, elles te poursuivront et t'atteindront jusqu'à ce que tu sois détruit, parce que tu n'auras pas obéi à la voix de Yahweh, ton Dieu, pour garder ses commandements et ses lois qu'il t'a ordonnés.
\VS{46}Et ces choses seront à jamais pour toi et ta postérité comme des signes et des prodiges.
\VS{47}Et parce que tu n'auras pas servi Yahweh, ton Dieu, avec joie, et de bon cœur, malgré l'abondance de toutes choses,
\VS{48}tu serviras, dans la faim, dans la soif, dans la nudité, et dans la disette de toutes choses, ton ennemi que Yahweh enverra contre toi. Il mettra un joug de fer sur ton cou, jusqu'à ce qu'il t'ait détruit.
\TextTitle{Prophétie sur l'invasion babylonienne et la dispersion d'Israël}
\VS{49}Yahweh fera lever de loin, des extrémités de la terre, une nation qui volera comme l'aigle, une nation dont tu ne comprendras pas la langue,
\VS{50}une nation au visage féroce, et qui ne soutiendra point le vieillard et n'aura point pitié pour l'enfant\FTNT{Cette prophétie s'est accomplie en 587 av. J.-C. Voir 2 R. 24-25.}.
\VS{51}Elle mangera le fruit de tes troupeaux et les fruits de ta terre, jusqu'à ce que tu sois détruit~; elle n'épargnera ni blé, ni vin, ni huile, ni portée de ton gros et de ton menu bétail, jusqu'à ce qu'elle t'ait fait périr.
\VS{52}Et elle t'assiégera dans toutes tes portes, jusqu'à ce que tombent ces hautes et fortes murailles dans lesquelles tu auras mis ta confiance dans tout ton pays~; elle t'assiégera, dis-je, dans toutes tes portes, dans tout le pays que Yahweh, ton Dieu, te donne.
\VS{53}Tu mangeras le fruit de tes entrailles, la chair de tes fils et de tes filles que Yahweh, ton Dieu, t'aura donnés, durant le siège et la détresse dont ton ennemi te serrera.
\VS{54}L'homme le plus tendre et le plus délicat d'entre vous regardera d'un œil malin son frère, sa femme bien-aimée, et le reste de ses fils qu'il a épargnés~;
\VS{55}pour ne donner à aucun d'eux de la chair de ses fils, qu'il mangera, parce qu'il ne lui restera rien du tout, à cause du siège et de la détresse dont ton ennemi te serrera dans toutes tes portes.
\VS{56}La femme la plus tendre et la plus délicate d'entre vous, qui n'a point osé mettre la plante de son pied sur la terre, par délicatesse et par mollesse, regardera d'un œil malin son mari bien-aimé, son fils, et sa fille~;
\VS{57}et le placenta qui sortira d'entre ses jambes, et les fils qu'elle enfantera~; car manquant de tout, elle les mangera secrètement, à cause du siège et de la détresse, dont ton ennemi te serrera dans toutes les villes.
\VS{58}Si tu ne prends pas garde d'observer toutes les paroles de cette loi, qui sont écrites dans ce livre, en craignant le Nom glorieux et redoutable de Yahweh, ton Dieu,
\VS{59}alors Yahweh rendra difficile tes plaies et les plaies de ta postérité, par des plaies grandes et persistantes, des maladies malignes et persistantes. 
\VS{60}Et il fera retourner sur toi toutes les maladies d'Egypte, devant lesquelles tu avais peur~; et elles s'attacheront à toi.
\VS{61}Même Yahweh fera venir sur toi toutes maladies et toutes plaies, qui ne sont point écrites dans le livre de cette loi, jusqu'à ce que tu sois détruit.
\VS{62}Et vous resterez en petit nombre, après avoir été aussi nombreux que les étoiles des cieux, parce que tu n'auras point obéi à la voix de Yahweh, ton Dieu.
\VS{63}Et il arrivera que comme Yahweh s'est réjoui sur vous, en vous faisant du bien et en vous multipliant, de même Yahweh se réjouira sur vous en vous faisant périr et en vous détruisant~; et vous serez arrachés de la terre dans laquelle vous allez entrer en possession.
\VS{64}Et Yahweh te dispersera parmi tous les peuples, d'un bout de la terre jusqu'à l'autre~; et là, tu serviras d'autres dieux que ni toi ni tes pères n'avez connus, le bois et la pierre.
\VS{65}Tu n'auras aucun repos parmi ces nations, même la plante de ton pied n'aura aucun repos. Car Yahweh te donnera un cœur tremblant, des yeux languissants, et une âme souffrante.
\VS{66}Et ta vie sera en suspens devant toi, tu trembleras la nuit et le jour, et tu ne seras point sûr de ta vie.
\VS{67}Tu diras le matin~: Qui me fera voir le soir~? Et le soir tu diras~: Qui me fera voir le matin~? A cause de l'effroi dont ton cœur sera effrayé, et à cause des choses que tu verras de tes yeux.
\VS{68}Et Yahweh te fera retourner en Egypte sur des navires, pour faire le chemin dont je t'ai dit~: Tu ne le verras plus~; et là, vous vous vendrez à vos ennemis, comme esclaves et servantes~; et il n'y aura personne pour vous acheter.
\Chap{29}
\TextTitle{Yahweh rappelle sa fidélité à Israël}
\VerseOne{}Voici les paroles de l'alliance que Yahweh ordonna à Moïse de traiter avec les enfants d'Israël au pays de Moab, outre l'alliance qu'il avait traitée avec eux à Horeb.
\VS{2}Moïse donc appela tout Israël, et leur dit~: Vous avez vu tout ce que Yahweh a fait sous vos yeux, dans le pays d'Egypte, à Pharaon, à tous ses serviteurs, et à tout son pays,
\VS{3}les grandes épreuves que tes yeux ont vues, ces signes et ces grands miracles.
\VS{4}Mais, jusqu'à ce jour, Yahweh ne vous a point donné un cœur pour connaître, ni des yeux pour voir, ni des oreilles pour entendre.
\VS{5}Je t'ai conduit pendant quarante ans par le désert~; tes vêtements ne se sont point usés, et ton soulier ne s'est point usé à ton pied.
\VS{6}Vous n'avez point mangé de pain, ni bu de vin ni de liqueur forte, afin que vous connaissiez que je suis Yahweh, votre Dieu.
\VS{7}Et vous êtes parvenus dans ce lieu~; Sihon, roi de Hesbon, et Og, roi de Basan, sont sortis à notre rencontre, pour nous combattre, et nous les avons battus.
\VS{8}Et nous avons pris leur pays, et nous l'avons donné en héritage aux Rubénites, aux Gadites, et à la demi-tribu des Manassites.
\TextTitle{Béni celui qui reste fidèle à l'alliance}
\VS{9}Vous garderez donc les paroles de cette alliance, et vous les pratiquerez, afin de réussir dans tout ce que vous ferez.
\VS{10}Vous vous tiendrez aujourd'hui devant Yahweh, votre Dieu, vos chefs de tribus, vos anciens, vos officiers, tous les hommes d'Israël,
\VS{11}vos enfants, vos femmes, et l'étranger qui est au milieu de ton camp, depuis celui qui coupe ton bois jusqu'à celui qui puise ton eau~;
\VS{12}afin que tu entres dans l'alliance de Yahweh, ton Dieu, dans ce serment, que Yahweh, ton Dieu, traite aujourd'hui avec toi,
\VS{13}afin qu'il t'établisse aujourd'hui pour son peuple et qu'il soit ton Dieu, comme il te l'a dit, et comme il l'a juré à tes pères, Abraham, Isaac et Jacob.
\VS{14}Et ce n'est pas seulement avec vous que je traite cette alliance, ce serment.
\VS{15}Mais c'est avec ceux qui sont ici, avec nous aujourd'hui devant Yahweh, notre Dieu, et avec ceux qui ne sont point ici, avec nous aujourd'hui.
\TextTitle{Mise en garde contre celui qui abandonne l'alliance}
\VS{16}Car vous savez comment nous avons habité dans le pays d'Egypte, et comment nous sommes passés au milieu des nations, que vous avez traversées.
\VS{17}Et vous avez vu leurs abominations et leurs idoles, le bois et la pierre, l'argent et l'or qui sont parmi eux.
\VS{18}Qu'il n'y ait parmi vous ni homme, ni femme, ni famille, ni tribu qui détourne son cœur aujourd'hui de Yahweh, notre Dieu, pour aller servir les dieux de ces nations. Qu'il n'y ait parmi vous de racine qui produise du poison et de l'absinthe.
\VS{19}Et qu'il n'arrive que quelqu'un en entendant les paroles de cette malédiction, ne se bénisse dans son cœur, en disant~: J'aurai la paix, même si je marche dans les penchants de mon cœur, et que j'ajoute l'ivresse à la soif.
\VS{20}Yahweh ne voudra point lui pardonner. Mais la colère de Yahweh et la jalousie s'enflammeront contre cet homme, et toutes les malédictions écrites dans ce livre reposeront sur lui, et Yahweh effacera son nom de dessous les cieux.
\VS{21}Et Yahweh le séparera de toutes les tribus d'Israël, pour son malheur, selon toutes les malédictions de l'alliance écrite dans ce livre de la loi.
\VS{22}Et la génération à venir, vos fils qui se lèveront après vous, et l'étranger qui viendra d'un pays lointain, quand ils verront les plaies et les maladies, dont Yahweh aura frappé ce pays~;
\VS{23}et que toute la terre de ce pays-là ne sera que soufre, que sel, et qu'embrasement, qu'elle ne sera point semée, et qu'elle ne fera rien germer, et que nulle herbe n'en sortira, ainsi qu'en la subversion de Sodome, et de Gomorrhe, et d'Adma, et de Tseboïm, que Yahweh détruisit dans sa colère et dans sa fureur.
\VS{24}Mais toutes les nations diront~: Pourquoi Yahweh a-t-il traité ainsi ce pays~? D'où vient l'ardeur de cette grande colère~?
\VS{25}Et on répondra~: C'est parce qu'ils ont abandonné l'alliance de Yahweh, le Dieu de leurs pères, qu'il a traitée avec eux quand il les fit sortir du pays d'Egypte~;
\VS{26}car ils sont allés servir d'autres dieux et se sont prosternés devant eux~; des dieux qu'ils ne connaissaient point et qu'il ne leur avait point donnés en partage.
\VS{27}A cause de cela, la colère de Yahweh s'est enflammée contre ce pays, et il a fait venir sur lui toutes les malédictions écrites dans ce livre.
\VS{28}Et Yahweh les a arrachés de leur terre avec colère, avec fureur, avec une grande indignation, et il les a chassés sur un autre pays, comme on le voit aujourd'hui.
\VS{29}Les choses cachées sont à Yahweh, notre Dieu~; les choses révélées sont à nous et à nos fils, à jamais, afin que nous pratiquions toutes les paroles de cette loi.
\Chap{30}
\TextTitle{Yahweh bénira et restaurera le peuple repentant}
\VerseOne{}Or il arrivera que lorsque toutes ces choses seront venues sur toi, la bénédiction et la malédiction, que je mets devant toi, si tu les rappelles dans ton cœur, parmi toutes les nations vers lesquelles Yahweh, ton Dieu, t'aura chassé~;
\VS{2}si tu reviens à Yahweh, ton Dieu, et si tu obéis à sa voix de tout ton cœur, de toute ton âme, toi et tes fils, selon tout ce que je t'ordonne aujourd'hui,
\VS{3}Yahweh, ton Dieu, ramènera tes captifs et aura compassion de toi~; il te rassemblera encore du milieu de tous les peuples parmi lesquels Yahweh, ton Dieu, t'aura dispersé.
\VS{4}Quand tu seras dispersé à l'extrémité des cieux, Yahweh, ton Dieu, te rassemblera de là, et de là, il te prendra.
\VS{5}Yahweh, ton Dieu, dis-je, te ramènera dans le pays que tes pères possédaient, et tu le posséderas~; il te fera du bien, et te rendra plus nombreux que tes pères.
\VS{6}Yahweh, ton Dieu, circoncira ton cœur, et le cœur de ta postérité, pour que tu aimes Yahweh, ton Dieu, de tout ton cœur, et de toute ton âme, afin que tu vives\FTNT{Ro. 2:29.}.
\VS{7}Et Yahweh, ton Dieu, mettra toutes ces malédictions sur tes ennemis, et sur ceux qui te haïront et te persécuteront.
\VS{8}Ainsi tu retourneras à Yahweh, tu obéiras à sa voix, et tu feras tous ses commandements que je t'ordonne aujourd'hui.
\TextTitle{Faire connaître la loi aux futures générations}
\VS{9}Et Yahweh, ton Dieu, te fera abonder en bien dans toute l'œuvre de ta main, dans le fruit de tes entrailles, dans le fruit de tes troupeaux et dans le fruit de ta terre~; car Yahweh se réjouira de nouveau de ton bonheur, comme il s'est réjoui de celui de tes pères,
\VS{10}quand tu obéiras à la voix de Yahweh, ton Dieu, en gardant ses commandements et ses ordonnances écrites dans ce livre de la loi, quand tu reviendras à Yahweh, ton Dieu, de tout ton cœur et de toute ton âme.
\TextTitle{Le peuple devant un choix}
\VS{11}Car ce commandement que je t'ordonne aujourd'hui n'est pas trop difficile pour toi et hors de ta portée.
\VS{12}Il n'est pas aux cieux, pour dire~: Qui montera pour nous aux cieux, nous l'apportera et nous le fera entendre, pour que nous le fassions~?
\VS{13}Il n'est point aussi de l'autre côté de la mer pour dire~: Qui passera de l'autre côté de la mer pour nous, et nous l'apportera, et nous le fera entendre pour que nous le fassions~?
\VS{14}Car cette parole est fort près de toi, dans ta bouche et dans ton cœur, afin que tu la pratiques\FTNT{Ro. 10:6.}.
\VS{15}Regarde, je mets aujourd'hui devant toi la vie et le bien, la mort et le mal.
\VS{16}Car je t'ordonne aujourd'hui d'aimer Yahweh, ton Dieu, de marcher dans ses voies, de garder ses commandements, ses lois, et ses ordonnances, afin que tu vives, que tu multiplies, et que Yahweh, ton Dieu, te bénisse dans le pays où tu vas entrer en possession.
\VS{17}Mais si ton cœur se détourne, si tu n'obéis point, et si tu te laisses entraîner à te prosterner devant d'autres dieux et à les servir,
\VS{18}je vous déclare aujourd'hui que vous périrez certainement, et que vous ne prolongerez point vos jours sur la terre dont vous allez entrer en possession, après avoir passé le Jourdain.
\VS{19}J'en prends aujourd'hui à témoin les cieux et la terre contre vous~: J'ai mis devant toi la vie et la mort, la bénédiction et la malédiction. Choisis donc la vie\FTNT{cp. Mt. 7:13-14.}, afin que tu vives, toi et ta postérité.
\VS{20}en aimant Yahweh, ton Dieu, en obéissant à sa voix, et en t'attachant à lui~: Car c'est lui qui est ta vie et la longueur de tes jours, afin que tu demeures sur la terre que Yahweh a juré à tes pères, Abraham, Isaac, et Jacob, de leur donner.
\Chap{31}
\TextTitle{Moïse encourage et affermit le peuple}
\VerseOne{}Puis Moïse s'en alla, et dit ces paroles à tout Israël~:
\VS{2}Aujourd'hui, leur dit-il, je suis âgé de cent vingt ans, je ne pourrai plus sortir ni entrer, et Yahweh m'a dit~: Tu ne passeras point ce Jourdain.
\VS{3}Yahweh, ton Dieu, passera lui-même devant toi, il détruira ces nations devant toi, et tu les posséderas. Josué passera aussi devant toi, comme Yahweh l'a dit.
\VS{4}Et Yahweh leur fera comme il a fait à Sihon et à Og, rois des Amoréens, qu'il a détruits avec leurs pays.
\VS{5}Et Yahweh les livrera devant vous, et vous leur ferez selon tout le commandement que je vous ai ordonné.
\VS{6}Fortifiez-vous donc et prenez courage~! Ne craignez point et ne soyez point effrayés devant eux~; car Yahweh, ton Dieu, marchera avec toi, il ne te délaissera point et ne t'abandonnera point.
\VS{7}Et Moïse appela Josué, et lui dit en présence de tout Israël~: Fortifie-toi et prends courage, car tu entreras avec ce peuple dans le pays que Yahweh a juré à leurs pères de leur donner, et c'est toi qui les en mettras en possession.
\VS{8}Yahweh est celui qui marchera devant toi, il sera lui-même avec toi, il ne te délaissera point, il ne t'abandonnera point~; ne crains point, et ne t'effraie point.
\VS{9}Or Moïse écrivit cette loi, et il la donna aux prêtres, fils de Lévi, qui portaient l'arche de l'alliance de Yahweh, et à tous les anciens d'Israël.
\VS{10}Moïse leur ordonna, en disant~: Tous les sept ans, au temps fixé de l'année du relâche, à la fête des tabernacles,
\VS{11}quand tout Israël viendra se présenter devant Yahweh, ton Dieu, dans le lieu qu'il aura choisi, tu liras alors cette loi devant tout Israël, à leurs oreilles.
\VS{12}Tu rassembleras le peuple, les hommes, les femmes, les enfants et l'étranger qui sera dans tes portes, pour qu'ils t'entendent, et qu'ils apprennent à craindre Yahweh, votre Dieu, et qu'ils prennent garde de faire toutes les paroles de cette loi.
\VS{13}Et leurs fils qui ne la connaîtront point l'entendront, et ils apprendront à craindre Yahweh, votre Dieu, tous les jours que vous vivrez sur cette terre que vous allez posséder après avoir passé le Jourdain.
\TextTitle{Yahweh annonce les événements à venir}
\VS{14}Alors Yahweh dit à Moïse~: Voici, le jour où tu vas mourir est proche. Appelle Josué, et tenez-vous dans la tente d'assignation. Je lui donnerai mes ordres. Moïse et Josué allèrent et se présentèrent dans la tente d'assignation.
\VS{15}Et Yahweh apparut dans la tente, dans une colonne de nuée~; et la colonne de nuée s'arrêta à l'entrée de la tente.
\VS{16}Yahweh dit à Moïse~: Voici, tu vas te coucher avec tes pères. Et ce peuple se lèvera et se prostituera après les dieux étrangers du pays au milieu duquel il va entrer. Il m'abandonnera et violera mon alliance que j'ai traitée avec lui.
\VS{17}En ce jour-là ma colère s'enflammera contre lui. Je les abandonnerai, et je leur cacherai ma face. Il sera dévoré, une multitude de maux et d'angoisses l'atteindront, et il dira en ce jour-là~: N'est-ce pas parce que mon Dieu n'est point au milieu de moi, que ces maux m'ont atteint~?
\VS{18}En ce jour-là, dis-je, je cacherai entièrement ma face, à cause de tout le mal qu'il aura fait, parce qu'il se sera tourné vers d'autres dieux.
\VS{19}Maintenant donc, écrivez ce cantique. Enseigne-le aux enfants d'Israël, mets-le dans leur bouche, afin que ce cantique me serve de témoignage contre les fils d'Israël.
\VS{20}Car je le conduirai sur la terre que j'ai juré à ses pères, où coulent le lait et le miel~; il mangera, se rassasiera, et s'engraissera~; puis il se tournera vers d'autres dieux, et il les servira, il m'irritera par mépris et violera mon alliance~;
\VS{21}et il arrivera qu'il sera atteint par une multitude de maux et d'angoisses, ce cantique, qui ne sera point oublié et qui sera dans la bouche de la postérité, répondra comme témoin contre eux. Je connais ses desseins, qu'il a déjà préparés aujourd'hui, avant même que je l'aie fait entrer dans le pays que j'ai juré.
\VS{22}Ainsi Moïse écrivit ce cantique en ce jour-là, et l'enseigna aux enfants d'Israël.
\VS{23}Et Yahweh commanda à Josué, fils de Nun, en disant~: Fortifie-toi et prends courage, car c'est toi qui feras entrer les enfants d'Israël dans le pays que je leur ai juré~; et je serai avec toi.
\VS{24}Et il arrivera que quand Moïse eut achevé d'écrire dans un livre les paroles de cette loi jusqu'à ce qu'elle soit complète,
\VS{25}Moïse ordonna aux Lévites qui portaient l'arche de l'alliance de Yahweh, en disant~:
\VS{26}Prenez ce livre de la loi, et mettez-le à côté de l'arche de l'alliance de Yahweh, votre Dieu, et il sera là comme témoin contre toi.
\VS{27}Car je connais ta rébellion et ton cou raide. Voici, déjà aujourd'hui étant en vie avec vous, vous avez été rebelles contre Yahweh, combien plus le serez-vous après ma mort~?
\VS{28}Faites assembler devant moi tous les anciens de vos tribus, et vos officiers, et je dirai ces paroles en leur présence, et j'appellerai à témoin contre eux les cieux et la terre.
\VS{29}Car je sais qu'après ma mort vous vous corromprez, et que vous vous détournerez de la voie que je vous ai ordonnée~; mais à la fin, le malheur vous atteindra, parce que vous aurez fait ce qui déplaît aux yeux de Yahweh, en l'irritant par les œuvres de vos mains.
\VS{30}Ainsi Moïse prononça entièrement les paroles de ce cantique-ci, en présence de toute l'assemblée d'Israël.
\Chap{32}
\TextTitle{Cantique de Moïse}
\VerseOne{}Cieux~! Prêtez l'oreille, et je parlerai. Terre~! écoute les paroles de ma bouche.
\VS{2}Que mon enseignement tombe comme la pluie, que ma parole se répande comme la rosée, comme une pluie fine sur l'herbe naissante, et comme une averse sur la verdure~!
\VS{3}Car j'invoquerai le Nom de Yahweh~; attribuez la grandeur à notre Dieu.
\VS{4}L'œuvre du rocher\FTNT{Voir commentaire en Es. 8:13-17.} est parfaite, car toutes ses voies sont justes. C'est un Dieu fidèle et sans iniquité, il est juste et droit.
\VS{5}Ils se sont corrompus, à lui n'est point la faute~; la faute est à ses fils, c'est une génération fausse et tortueuse.
\TextTitle{Israël, le choix de Yahweh}
\VS{6}Est-ce ainsi que tu récompenses Yahweh, peuple insensé et dépourvu de sagesse~? N'est-il pas ton père, celui qui t'a acquis~? Il t'a fait et t'a façonné.
\VS{7}Souviens-toi des anciens jours, considère les années, de génération en génération, interroge ton père, et il te l'apprendra, et tes anciens, et ils te le diront.
\VS{8}Quand le Très-Haut laissa un héritage aux nations, quand il sépara les enfants des hommes, il fixa les limites des peuples selon le nombre des fils d'Israël~;
\VS{9}car la portion de Yahweh, c'est son peuple, Jacob est le lot de son héritage.
\VS{10}Il l'a trouvé dans un pays désert, dans la désolation des hurlements d'une solitude, il l'a entouré, il l'a dirigé, il l'a gardé comme la prunelle de son œil,
\VS{11}comme l'aigle éveille sa nichée, couve ses petits, étend ses ailes, les prend, les porte sur ses ailes.
\VS{12}Yahweh seul l'a conduit, et il n'y a point eu avec lui de dieu étranger.
\VS{13}Il l'a fait monter à cheval sur les hauteurs du pays, et il a mangé les fruits des champs~; il lui a donné à sucer le miel du rocher, l'huile du rocher le plus dur,
\VS{14}la crème des vaches, le lait des brebis, et la graisse des agneaux, des béliers de Basan, et des boucs, et la fleur du froment~; et tu as bu le vin qui était le sang de la grappe.
\TextTitle{Condamnation de l'apostasie d'Israël}
\VS{15}Jeshurun\FTNT{Littéralement «~Jeshurun~» en hébreu~: «~celui qui est droit~». Nom symbolique donné à Israël pour décrire son caractère idéal.} s'est engraissé, et a regimbé~; tu es devenu gras, gros et épais~! Et il a abandonné Dieu qui l'a fait, et il a méprisé le rocher de son salut.
\VS{16}Ils ont provoqué sa jalousie par des dieux étrangers, ils l'ont irrité par des abominations.
\VS{17}Ils ont sacrifié à des démons, qui ne sont point Dieu~; aux dieux qu'ils ne connaissaient point, dieux nouveaux, venus depuis peu, et que vos pères n'ont point redoutés.
\VS{18}Tu as oublié le rocher qui t'a engendré, et tu as oublié le Dieu qui t'a fait naître.
\VS{19}Yahweh l'a vu, et a été irrité, parce que ses fils et ses filles l'ont provoqué à la colère.
\VS{20}Et il a dit~: Je cacherai ma face, je verrai quelle sera leur fin~; car ils sont une génération perverse, des fils infidèles.
\VS{21}Ils ont excité ma jalousie par ce qui n'est point Dieu, ils m'ont irrité par leurs vanités~; ainsi je provoquerai leur jalousie par ce qui n'est point un peuple, et je les offenserai par une nation insensée.
\VS{22}Car le feu de ma colère s'est allumé, et brûlera jusqu'au fond du scheol, dévorera la terre et son fruit, et embrasera les fondements des montagnes.
\VS{23}Je rassemblerai sur eux des maux, et je détruirai toutes mes flèches sur eux.
\VS{24}Ils seront consumés par la famine, rongés par des charbons ardents, et par une destruction amère~; j'enverrai contre eux la dent des bêtes et le venin des serpents qui rampent sur la poussière.
\VS{25}L'épée venant de dehors les privera les uns des autres~; et au-dedans, la terreur les privera d'enfants. Il en sera du jeune homme comme de la vierge, de l'enfant à la mamelle comme de l'homme aux cheveux blancs.
\TextTitle{A Yahweh la vengeance et la rétribution}
\VS{26}Je dirais~: Je les détruirai, et je ferai disparaître leur mémoire d'entre les hommes~!
\VS{27}Si je ne craignais la colère de l'ennemi, de peur que leurs adversaires ne se méprennent, et ne disent~: Notre main est élevée, et ce n'est pas Yahweh qui a fait tout ceci.
\VS{28}Car c'est une nation qui se perd par ses conseils, et il n'y a en eux aucune intelligence.
\VS{29}Ô s'ils étaient sages, ils comprendraient ceci, et ils considéreraient leur fin.
\VS{30}Comment un seul en poursuivrait-il mille, et deux en mettraient-ils dix mille en fuite, si ce n'était que leur Rocher les avait vendus, et que Yahweh ne les avait enserrés~?
\VS{31}Car leur rocher n'est pas comme notre Rocher, nos ennemis en sont juges.
\VS{32}Car leur vigne est du plant de Sodome, et du terroir de Gomorrhe~; leurs raisins sont des raisins empoisonnés, leurs grappes sont amères.
\VS{33}Leur vin est un venin de dragon, et du poison cruel d'aspic.
\VS{34}Cela n'est-il pas caché près de moi, scellé dans mes trésors~?
\VS{35}A moi la vengeance et la rétribution, le temps où leur pied glissera~! Car le jour de leur calamité est près, et les choses qui doivent leur arriver se hâtent.
\VS{36}Mais Yahweh jugera son peuple~; et il se repentira en faveur de ses serviteurs, quand il verra que leur force a disparu, et qu'il n'y a personne de retenu ni d'abandonné.
\VS{37}Et il dira~: Où sont leurs dieux, le rocher en qui ils se confiaient,
\VS{38}qui mangeaient la graisse de leurs sacrifices, qui buvaient le vin de leurs libations~? Qu'ils se lèvent, qu'ils vous aident, et qu'ils vous servent de refuge~!
\VS{39}Voyez maintenant que moi,JE SUIS\FTNT{«~JE SUIS. Il s'agit ici du Nom que Dieu a révélé à Moïse et à Esaïe (Ex. 3:14) et sous lequel Jésus s'est présenté (Jn. 18:5-8).}, et il n'y a point de dieu avec moi\FTNT{Ce verset confirme que Dieu est un puisqu'il n'y a pas d'autres dieux à ses côtés.}~; je fais mourir et je fais vivre, je blesse et je guéris~; et il n'y a personne qui puisse délivrer de ma main.
\VS{40}Car je lève ma main au ciel, et je dis~: Je vis éternellement.
\VS{41}Si j'aiguise l'éclair de mon épée, et si ma main saisit la justice, je rendrai la vengeance à mes adversaires et je rétribuerai ceux qui me haïssent.
\VS{42} J'enivrerai mes flèches de sang et mon épée dévorera la chair, j'enivrerai, dis-je, mes flèches du sang des tués et des captifs, de la tête des chefs de l'ennemi.
\VS{43}Nations, réjouissez-vous avec son peuple~! Car il venge le sang de ses serviteurs, il tire vengeance de ses ennemis, et fait propitiation pour sa terre et pour son peuple.
\TextTitle{Fin du cantique, invitation à demeurer fidèle}
\VS{44}Moïse donc vint et prononça toutes les paroles de ce cantique, à l'oreille du peuple, lui et Josué, fils de Nun.
\VS{45}Et quand Moïse eut achevé de prononcer toutes ces paroles à tout Israël,
\VS{46}il leur dit~: Appliquez votre cœur à toutes ces paroles que je vous conjure aujourd'hui d'ordonner à vos fils, afin qu'ils prennent garde de faire toutes les paroles de cette loi.
\VS{47}Car ce n'est pas une parole vaine pour vous, mais c'est votre vie~; et par cette parole vous prolongerez vos jours sur la terre que vous posséderez, après avoir passé le Jourdain.
\TextTitle{Moïse, invité à monter sur le mont Nebo}
\VS{48}En ce même jour-là, Yahweh parla à Moïse, en disant~:
\VS{49}Monte sur cette montagne d'Abarim, sur le mont Nebo, au pays de Moab, vis-à-vis de Jéricho~; et regarde le pays de Canaan, que je donne en possession aux enfants d'Israël.
\VS{50}Tu mourras sur la montagne où tu vas monter, et tu seras recueilli vers ton peuple, comme Aaron, ton frère, est mort sur la montagne d'Hor, et a été recueilli vers son peuple,
\VS{51}parce que vous avez péché contre moi au milieu des fils d'Israël, aux eaux de Meriba, à Kadès, dans le désert de Tsin~; car vous ne m'avez point sanctifié au milieu des enfants d'Israël.
\VS{52}Tu verras le pays devant toi, mais tu n'entreras point dans le pays, que je donne aux enfants d'Israël.
\Chap{33}
\TextTitle{Moïse bénit les tribus d'Israël}
\VerseOne{}Or c'est ici la bénédiction dont Moïse, homme de Dieu, bénit les enfants d'Israël avant sa mort.
\VS{2}Il dit donc~: Yahweh est venu de Sinaï, il s'est levé sur eux de Séir, il a resplendi de la montagne de Paran, et il est sorti d'entre les dix milliers des saints, et de sa droite le feu de la loi est sorti vers eux.
\VS{3}En effet, il aime les peuples~; tous ses saints sont dans ta main. Ils se sont mis à tes pieds pour recevoir tes paroles.
\VS{4}Moïse nous a donné la loi, héritage de l'assemblée de Jacob.
\VS{5}Il était roi de Jeshurun\FTNT{Voir commentaire en De. 32:15.}, quand les chefs du peuple s'assemblaient ensemble, avec les tribus d'Israël.
\VS{6}Que Ruben vive et qu'il ne meure point, encore que ses hommes soient en petit nombre.
\VS{7}Et voici ce qu'il dit pour Juda~: Ô Yahweh~! Ecoute la voix de Juda, et ramène-le vers son peuple. Que ses mains soient puissantes, et sois-lui en aide contre ses ennemis.
\VS{8}Il dit aussi touchant Lévi~: Tes thummim et tes urim sont à l'homme fidèle que tu as éprouvé à Massa, et avec qui tu as contesté aux eaux de Meriba.
\VS{9}Il dit de son père et de sa mère~: Je ne les ai point vus~! Il ne reconnait point ses frères, et ne connait point ses fils. Car ils gardent tes paroles, et ils gardent ton alliance.
\VS{10}Ils enseignent tes ordonnances à Jacob, et ta loi à Israël~; ils mettent l'encens sous tes narines, et l'holocauste sur ton autel.
\VS{11}Ô Yahweh, bénis sa force~! Agrée l'œuvre de ses mains~! Brise les reins de ceux qui s'élèvent contre lui, et que ceux qui le haïssent ne se relèvent plus~!
\VS{12}Il dit de Benjamin~: Le bien-aimé de Yahweh habitera en sécurité avec lui~; il le protégera toujours, et demeurera entre ses épaules.
\VS{13}Il dit de Joseph~: Son pays est béni par Yahweh, de ce qu'il y a de plus précieux au ciel, de la rosée, et de l'abîme qui est en bas,
\VS{14}et du plus précieux des produits du soleil, et du plus précieux des produits de la lune, 
\VS{15}et de ce qui croît sur le sommet des montagnes d'ancienneté, du plus précieux des collines éternelles,
\VS{16}et du plus précieux de la terre et de sa plénitude. Que la grâce de celui qui demeura dans le buisson vienne sur la tête de Joseph, sur le sommet, sur le sommet de la tête de celui qui est consacré d'entre ses frères~!
\VS{17}Sa majesté est comme le premier-né de son taureau~; et ses cornes comme les cornes du buffle~; il poussera tous les peuples ensemble jusqu'aux extrémités de la terre~: Ce sont les dix milliers d'Ephraïm, et ce sont les milliers de Manassé.
\VS{18}Il dit de Zabulon~: Réjouis-toi, Zabulon, dans ta sortie, et toi, Issacar, dans tes tentes.
\VS{19}Ils appelleront les peuples sur la montagne, ils y offriront des sacrifices de justice, car ils suceront l'abondance des mers, et les trésors cachés dans le sable.
\VS{20}Il dit aussi de Gad~: Béni soit celui qui élargit Gad~! Il habite comme un lion, et il déchire le bras et la tête.
\VS{21}Il a choisi les prémices, parce que c'était là qu'était cachée la portion du législateur, et il est venu en tête du peuple~; il a exécuté la justice de Yahweh et ses jugements envers Israël.
\VS{22}Et il dit de Dan~: Dan est un jeune lion, il s'élance de Basan.
\VS{23}Il dit de Nephthali~: Nephthali, rassasié de faveur, et rempli de la bénédiction de Yahweh, possède l'occident et le sud.
\VS{24}Il dit aussi d'Aser~: Aser sera béni entre les fils~; il sera agréable à ses frères, et il trempera son pied dans l'huile.
\VS{25}Tes verrous seront de fer et d'airain, et ta force durera autant que tes jours.
\VS{26}Nul n'est comme le Dieu de Jeshurun\FTNT{Voir commentaire en De. 32:15.}, porté sur les cieux pour te venir en aide, et sur les nuées dans sa majesté.
\VS{27}Le Dieu d'éternité est un refuge, et au-dessous de toi sont ses bras éternels~; car il a chassé de devant toi tes ennemis, et il a dit~: Extermine.
\VS{28}Israël donc habitera en sécurité, la source de Jacob est à part dans un pays de blé et de vin, et ses cieux distilleront la rosée.
\VS{29}Ô que tu es heureux, Israël~! Qui est le peuple semblable à toi, qui ait été sauvé par Yahweh, le bouclier de ton secours et l'épée de ta majesté~? Tes ennemis dissimuleront devant toi, et tu fouleras de tes pieds leurs lieux élevés.
\Chap{34}
\TextTitle{Moïse voit le pays mais n'y entre pas}
\VerseOne{}Alors Moïse monta des plaines de Moab sur le mont Nebo, au sommet du Pisga, vis-à-vis de Jéricho. Et Yahweh lui fit voir tout le pays~: De Galaad jusqu'à Dan,
\VS{2}tout Nephthali, le pays d'Ephraïm et de Manassé, tout le pays de Juda, jusqu'à la Mer Occidentale,
\VS{3}le sud, les environs du Jourdain, la plaine de Jéricho, la ville des palmiers, jusqu'à Tsoar.
\VS{4}Yahweh lui dit~: C'est ici le pays que j'ai juré à Abraham, à Isaac, et à Jacob, en disant~: Je le donnerai à ta postérité. Je te l'ai fait voir de tes yeux~; mais tu n'y entreras point.
\TextTitle{Mort de Moïse}
\VS{5}Ainsi Moïse, serviteur de Yahweh, mourut là, dans le pays de Moab, selon la parole de Yahweh.
\VS{6}Et il l'ensevelit dans la vallée, au pays de Moab, vis-à-vis de Beth-Peor. Personne n'a connu son sépulcre jusqu'à aujourd'hui\FTNT{Jud. 1:9.}.
\VS{7}Or Moïse était âgé de cent vingt ans quand il mourut~; sa vue n'était point affaiblie, et sa vigueur n'était point passée.
\VS{8}Les enfants d'Israël pleurèrent Moïse trente jours dans les plaines de Moab~; et ces jours de pleurs et de deuil sur Moïse furent accomplis.
\TextTitle{Josué, successeur de Moïse}
\VS{9}Et Josué, fils de Nun, fut rempli de l'Esprit de sagesse, parce que Moïse lui avait imposé les mains\FTNT{Jos. 1:9.}. Les enfants d'Israël lui obéirent, et firent ce que Yahweh avait ordonné à Moïse.
\VS{10}Et il ne s'est plus levé en Israël de prophète comme Moïse, que Yahweh connaissait face à face.
\VS{11}Selon tous les signes et les miracles que Yahweh l'envoya faire au pays d'Egypte, devant Pharaon, et tous ses serviteurs, et tout son pays,
\VS{12}et selon toute cette main forte, et tous ces terribles prodiges, que Moïse fit sous les yeux de tout Israël.
\PPE{}
\end{multicols}

%\addcontentsline{toc}{section}{Nevi'im (Prophètes)}\clearpage
%\clearpage\ShortTitle{Josué}\BookTitle{Josué}\BFont
\noindent\hrulefill
{\footnotesize
\textit{
\bigskip
{\centering{}
\\Auteur : Probablement Josué
\\(Heb. : Yehowshuwa)
\\Signification : Yahweh est salut
\\Thème : La conquête de Canaan
\\Date de rédaction : 14\up{ème} siècle av. J.-C.\\}
}
%\bigskip
\textit{
\\Né en Egypte, Josué, fils de Nun, originaire de la tribu d'Ephraïm, servit Moïse de la sortie d'Egypte jusqu'à sa mort. Choisi par Dieu pour succéder au prophète, il fut le seul de l'ancienne génération, avec Caleb, à avoir survécu à la longue épreuve du désert. Ce livre relate les étapes du voyage du peuple et sa conquête de la terre promise. Il présente par ailleurs les victoires acquises par la puissance de Yahweh sous la conduite de Josué. C'est l'histoire de la prise de Canaan et de son partage aux douze tribus d'Israël.\bigskip
}
}
\par\nobreak\noindent\hrulefill
\begin{multicols}{2}
\Chap{1}
\TextTitle{Josué succède à Moïse à sa mort\FTNTT{De. 34:9.}}
\VerseOne{}Or, il arriva après la mort de Moïse, serviteur de Yahweh, que Yahweh parla à Josué, fils de Nun, qui avait servi Moïse, en disant :
\VS{2}Moïse, mon serviteur est mort ; maintenant donc, lève-toi, passe ce Jourdain, toi et tout ce peuple, pour entrer dans le pays que je donne aux enfants d'Israël.
\VS{3}Tout lieu que foulera la plante de votre pied, je vous l'ai donné, comme je l'ai déclaré à Moïse\FTNT{De. 11:24.}.
\VS{4}Vos frontières seront depuis ce désert et le Liban, jusqu'au grand fleuve, le fleuve de l'Euphrate, tout le pays des Héthiens jusqu'à la grande mer, vers le soleil couchant.
\VS{5}Nul ne tiendra devant toi, tous les jours de ta vie. Je serai avec toi comme j'ai été avec Moïse ; je ne te délaisserai point, et je ne t'abandonnerai point\FTNT{De. 31:6 ; Hé. 13:5-6.}.
\VS{6}Fortifie-toi et prends courage, car c'est toi qui mettras ce peuple en possession du pays dont j'ai juré à leurs pères de leur donner.
\VS{7}Seulement fortifie-toi et renforce-toi de plus en plus, afin que tu prennes garde de faire selon toute la loi que Moïse mon serviteur t'a ordonnée ; ne t'en détourne point ni à droite ni à gauche, afin que tu prospères partout où tu iras.
\VS{8}Que ce livre de la loi ne s'éloigne point de ta bouche, mais médite-le jour et nuit, pour agir fidèlement selon tout ce qui y est écrit\FTNT{La clé d'une vie chrétienne épanouie est la Parole de Dieu. Méditer signifie : 
\\- Murmurer la Parole de Dieu : Partout où nous sommes, nous pouvons dans nos cœurs murmurer les promesses du Seigneur (Ps. 63:5-8 ; Ps. 119:11).
\\- Proclamer à haute voix : Il est intéressant de noter que le mot hébreu traduit dans Jos. 1:8 par méditer est traduit par « proclamer » ou « dire » dans Pr. 8:7 ; Ps. 35:8 ; Ps. 77:13. 
\\- Réfléchir profondément : Il faut être dans le lieu secret (Mt. 6:5-6). En Israël il est de coutume d'aller étudier la Torah à l'ombre d'un figuier. Voir Jn. 1:43-51.} ; car c'est alors que tu auras du succès dans tes entreprises, c'est alors que tu réussiras.
\VS{9}Ne t'ai-je pas donné cet ordre, fortifie-toi et prends courage ? Ne t'épouvante point et ne t'effraie point ; car Yahweh ton Dieu est avec toi partout où tu iras.
\TextTitle{Josué prend la direction du peuple}
\VS{10}Après cela, Josué donna cet ordre aux officiers du peuple, en disant :
\VS{11}Passez par le camp, ordonnez au peuple et dites-lui : Préparez-vous des provisions, car dans trois jours vous passerez ce Jourdain pour aller prendre possession du pays que Yahweh, votre Dieu, vous donne afin que vous le possédiez.
\VS{12}Josué parla aussi aux Rubénites, aux Gadites et à la demi-tribu de Manassé, en disant :
\VS{13}Souvenez-vous de la parole que Moïse, serviteur de Yahweh, vous a prescrite, en disant : Yahweh votre Dieu vous a accordé du repos, et vous a donné ce pays.
\VS{14}Vos femmes, vos petits-enfants, et vos bêtes resteront dans le pays que Moïse vous a donné de l'autre côté du Jourdain ; mais vous tous, hommes vaillants, vous passerez en armes devant vos frères, et vous les aiderez\FTNT{Ex. 13:18.} ;
\VS{15}jusqu'à ce que Yahweh ait accordé du repos à vos frères comme à vous, et qu'ils soient aussi en possession du pays que Yahweh, votre Dieu, leur donne. Puis vous reviendrez prendre possession du pays qui est votre propriété, et que vous a donné Moïse, serviteur de Yahweh, de l'autre côté du Jourdain, vers l'orient.
\VS{16}Ils répondirent à Josué, en disant : Nous ferons tout ce que tu nous as ordonné, et nous irons partout où tu nous enverras.
\VS{17}Nous t'obéirons comme nous avons obéi à Moïse ; seulement que Yahweh ton Dieu soit avec toi, comme il a été avec Moïse.
\VS{18}Tout homme qui sera rebelle à ton ordre, et qui n'obéira point à tes paroles dans tout ce que tu lui commanderas, sera mis à mort ; seulement, fortifie-toi, et sois courageux !
\Chap{2}
\TextTitle{Josué envoie deux espions à Jéricho ; ils sont reçus par Rahab\FTNTT{Ja. 2:25.}}
\VerseOne{}Or, Josué fils de Nun, envoya secrètement de Sittim deux hommes, pour épier secrètement le pays, et il leur dit : Allez, examinez le pays, et Jéricho. Ils partirent donc et entrèrent dans la maison d'une femme prostituée, nommée Rahab\FTNT{Rahab avait entendu parler du Dieu des Hébreux et avait placé son espérance de salut en lui (Ro. 10 :11). Par cet acte de foi, sa destinée a changé. Cette femme qui était vouée à une double condamnation du fait de sa condition de prostituée (De.23:17) et de son appartenance à une nation païenne qui devait être dévouée à la façon de l'interdit (Jos. 6), a été sauvée avec sa famille (Ac. 2:21 ; Ac. 16:31 ). Ainsi, bien des siècles plus tard, on ne la mentionnera plus comme une prostituée, mais comme une ancêtre du Sauveur et une héroïne de la foi (Mt. 1:5 ; Hé. 11 :21). Rahab est donc l'archétype des païens qui sont rentrés dans l'alliance de Dieu par la foi.}, et ils y couchèrent.
\VS{2}Alors on dit au roi de Jéricho : Voici, des hommes sont venus ici cette nuit de la part des enfants d'Israël pour explorer le pays.
\VS{3}Et le roi de Jéricho envoya dire à Rahab : Fais sortir les hommes qui sont venus chez toi et qui sont entrés dans ta maison ; car ils sont venus pour explorer tout le pays.
\VS{4}Or la femme prit les deux hommes et les cacha ; et elle dit : Il est vrai que des hommes sont venus chez moi, mais je ne savais pas d'où ils étaient ;
\VS{5}et comme on fermait la porte sur le soir, ces hommes sont sortis ; je ne sais pas où ces hommes sont allés ; poursuivez-les bien vite car vous les atteindrez.
\VS{6}Or elle les avait fait monter sur le toit et les avait cachés sous des tiges de lin qu'elle avait arrangées sur le toit.
\VS{7}Et quelques gens les poursuivirent par le chemin du Jourdain jusqu'aux passages ; et on ferma la porte après que ceux qui les poursuivaient furent sortis.
\VS{8}Or, avant qu'ils se couchent, elle monta vers eux sur le toit ;
\VS{9}et leur dit : Je sais que Yahweh vous a donné ce pays, et que la terreur de votre nom nous a saisis, et que tous les habitants du pays perdent courage à cause de vous\FTNT{Ex. 23:27.}.
\VS{10}Car nous avons entendu que Yahweh a mis à sec devant vous les eaux de la Mer Rouge à votre sortie du pays d'Egypte ; et ce que vous avez fait aux deux rois des Amoréens qui étaient de l'autre côté du Jourdain, à Sihon et à Og, que vous avez détruits complètement en les dévouant par le moyen de l'interdit.
\VS{11}Nous l'avons entendu, et notre cœur a fondu, et depuis aucun homme n'a eu le courage à cause de vous. Car Yahweh, votre Dieu, est le Dieu des cieux en haut et de la terre\FTNT{De. 4:39.} en bas.
\VS{12}Maintenant donc, je vous prie, jurez-moi par Yahweh, que puisque j'ai usé de bonté envers vous, vous userez aussi de bonté envers la maison de mon père, 
\VS{13}et que vous me donnerez un signe de votre fidélité\FTNT{La couleur cramoisi s'obtient grâce à la femelle cochenille aptère qui contient dans son corps et dans ses œufs un pigment rouge à base d'acide carminique qui permet à l'insecte et à ses larves de se protéger des prédateurs. Au moment de la ponte, cette dernière fixe fermement son corps au tronc d'un arbre puis libère ses œufs qui demeurent ainsi protégés en dessous d'elle jusqu'à leur éclosion. Ensuite, l'insecte meurt en libérant cette substance rouge qui se propage sur tout son corps et sur le bois hôte. C'est ce fluide que l'homme récupère pour en faire un colorant à la couleur caractéristique. Une subtile analogie peut être faire entre la cochenille et le Seigneur qui a versé son sang à la croix pour nous donner la vie. « Et moi, je suis un ver, et non un homme, l'opprobre des hommes et le méprisé du peuple » (Ps. 22 :7).} d'une ferme assurance que vous laisserez vivre mon père, ma mère, mes frères, mes sœurs, et tous ceux qui leur appartiennent, et que vous sauverez nos âmes de la mort.
\VS{14}Et ces hommes lui répondirent : Nos personnes répondront pour vous jusqu'à la mort, pourvu que vous ne divulguez pas cette affaire ; et quand Yahweh nous aura donné le pays nous userons envers toi de bonté et de vérité. 
\TextTitle{Les espions s'enfuient aidés par Rahab}
\VS{15}Elle les fit donc descendre avec une corde par la fenêtre ; car sa maison était sur la muraille de la ville, et elle habitait sur la muraille de la ville. 
\VS{16}Et elle leur dit : Allez à la montagne, de peur que ceux qui vous poursuivent ne vous rencontrent, et cachez-vous là pendant trois jours jusqu'à ce qu'ils soient de retour. Après cela vous suivrez votre chemin.
\VS{17}Et ces hommes lui dirent : Voici comment nous serons quittes de ce serment que tu nous as fait faire.
\VS{18}Voici, quand nous entrerons dans le pays, tu lieras ce cordon de fil d'écarlate à la fenêtre par laquelle tu nous auras fait descendre, et tu recueilleras chez toi, dans cette maison, ton père et ta mère, tes frères, et toute la famille de ton père.
\VS{19}Et quiconque sortira hors de la porte de ta maison, son sang sera sur sa tête, et nous en serons quittes ; mais quiconque sera avec toi, dans la maison, son sang sera sur notre tête si quelqu'un met la main sur lui.
\VS{20}Et si tu divulgues cette affaire, nous serons quittes du serment que tu nous as fait faire.
\VS{21}Et elle répondit : Que cela soit ainsi que vous l'avez dit. Alors elle les laissa aller. Ils s'en allèrent et elle lia le cordon de fil d'écarlate à la fenêtre.
\VS{22}Et ils marchèrent et arrivèrent à la montagne, où ils restèrent trois jours, jusqu'à ce que ceux qui les poursuivaient soient de retour. Ceux qui les poursuivaient les cherchèrent par tout le chemin, mais ils ne les trouvèrent pas.
\VS{23}Ainsi ces deux hommes s'en retournèrent, descendirent de la montagne, passèrent le Jourdain. Ils vinrent auprès de Josué, fils de Nun. Ils lui racontèrent toutes les choses qui leur étaient arrivées.
\VS{24}Et ils dirent à Josué : Certainement, Yahweh a livré tout le pays entre nos mains, et même tous les habitants ont perdu le courage à notre vue.
\Chap{3}
\TextTitle{Israël traverse le Jourdain à sec}
\VerseOne{}Or Josué se leva de bon matin, lui et tous les enfants d'Israël partirent de Sittim, ils vinrent jusqu'au Jourdain, et ils logèrent là cette nuit, avant de le traverser.
\VS{2}Et au bout de trois jours les officiers traversèrent le milieu du camp,
\VS{3}et donnèrent cet ordre au peuple en disant : Dès que vous verrez l'arche de l'alliance de Yahweh, votre Dieu, portée par les prêtres, les Lévites, vous partirez de votre quartier, et vous marcherez après elle.
\VS{4}Et afin que vous n'approchez pas d'elle, il y aura entre vous et elle une distance de la mesure d'environ deux mille coudées. Elle vous fera connaître le chemin par lequel vous devez marcher ; car vous n'avez pas encore passé par ce chemin.
\VS{5}Josué dit au peuple : Sanctifiez-vous, car Yahweh fera demain des choses merveilleuses au milieu de vous\FTNT{Ex. 19:10-11.}.
\VS{6}Josué parla aussi aux prêtres, en disant : Portez l'arche de l'alliance, et passez devant le peuple. Ainsi ils portèrent l'arche de l'alliance, et marchèrent devant le peuple.
\VS{7}Or Yahweh dit à Josué : Aujourd'hui je commencerai à t'élever aux yeux de tout Israël, afin qu'ils sachent que je serai aussi avec toi, comme j'ai été avec Moïse.
\VS{8}Tu donneras cet ordre aux prêtres qui portent l'arche de l'alliance, en leur disant : Dès que vous arriverez au bord des eaux du Jourdain, vous vous arrêterez dans le Jourdain.
\VS{9}Et Josué dit aux enfants d'Israël : Approchez-vous d'ici, et écoutez les paroles de Yahweh, votre Dieu.
\VS{10}Puis Josué dit : Vous reconnaîtrez à ceci que le Dieu vivant est au milieu de vous et qu'il chassera et déshéritera devant vous les Cananéens, les Héthiens, les Héviens, les Phéréziens, les Guirgasiens, les Amoréens et les Jébusiens.
\VS{11}Voici, l'arche de l'alliance du Seigneur de toute la terre va passer devant vous dans le Jourdain.
\VS{12}Maintenant, prenez douze hommes des tribus d'Israël, un homme de chaque tribu.
\VS{13}Et il arrivera qu'aussitôt que les plantes des pieds des prêtres qui portent l'arche de Yahweh, le Seigneur de toute la terre, seront posés dans les eaux du Jourdain, les eaux du Jourdain seront coupées, les eaux, dis-je, qui descendent d'en haut, et elles s'arrêteront en un monceau\FTNT{Ps. 114:3.}.
\VS{14}Et il arriva que le peuple étant parti de ses tentes pour passer le Jourdain, et les prêtres qui portaient l'arche de l'alliance, étaient devant le peuple.
\VS{15}Aussitôt que ceux qui portaient l'arche furent arrivés au Jourdain, et que les pieds des prêtres qui portaient l'arche furent mouillés au bord de l'eau. Le Jourdain regorge par-dessus toutes ses rives durant tout le temps de la moisson\FTNT{1 Ch. 12:15}.
\VS{16}Les eaux qui descendent d'en haut, s'arrêtèrent, et s'élevèrent en un monceau, à une très grande distance, depuis la ville d'Adam, qui est à côté de Tsarthan ; et celles d'en bas, qui descendaient vers la mer de la plaine, qui est la mer salée, furent totalement coupées. Le peuple passa vis-à-vis de Jéricho.
\VS{17}Mais les prêtres qui portaient l'arche de l'alliance de Yahweh, s'arrêtèrent de pied ferme sur le sec, au milieu du Jourdain, pendant que tout Israël passait à sec, jusqu'à ce que tout le peuple ait achevé de passer le Jourdain.
\Chap{4}
\TextTitle{Josué dresse un monument de pierres en souvenir de la traversée}
\VerseOne{}Or il arriva que quand tout le peuple eut achevé de passer le Jourdain, que Yahweh parla à Josué et dit :
\VS{2}Prenez douze hommes parmi le peuple, un homme de chaque tribu.
\VS{3}et donnez-leur cet ordre, en disant : Prenez ici, du milieu du Jourdain, de la place où les prêtres se sont arrêtés de pied ferme, douze pierres, que vous emporterez avec vous, et vous les poserez au lieu où vous passerez cette nuit.
\VS{4}Josué appela les douze hommes qu'il choisit parmi les enfants d'Israël, un homme de chaque tribu.
\VS{5}Et il leur dit : Passez devant l'arche de Yahweh, votre Dieu, au milieu du Jourdain, et que chacun de vous charge une pierre sur son épaule, selon le nombre des tribus des enfants d'Israël ;
\VS{6}afin que cela soit un signe au milieu de vous. Et quand vos fils interrogeront à l'avenir leurs pères, en disant : Que signifient ces pierres-ci ?
\VS{7}Alors vous leur répondrez : Les eaux du Jourdain ont été coupées devant l'arche de l'alliance de Yahweh ; lorsqu'elle passa le Jourdain, les eaux du Jourdain ont été arrêtées ; c'est pourquoi ces pierres-là seront à jamais un souvenir pour les enfants d'Israël.
\VS{8}Les enfants d'Israël firent donc comme Josué leur avait ordonné. Ils prirent douze pierres du milieu du Jourdain, comme Yahweh l'avait ordonné à Josué, selon le nombre des tribus des enfants d'Israël. Ils les emportèrent avec eux et les posèrent au lieu où ils devaient passer la nuit.
\VS{9}Josué dressa aussi douze pierres au milieu du Jourdain, à l'endroit où les pieds des prêtres qui portaient l'arche de l'alliance s'étaient arrêtés ; et elles y sont restées jusqu'à ce jour.
\VS{10}Les prêtres donc qui portaient l'arche se tinrent debout au milieu du Jourdain, jusqu'à ce que tout ce que Yahweh avait ordonné à Josué de dire au peuple soit accompli, selon tout ce que Moïse avait prescrit à Josué. Et le peuple se hâta de passer.
\VS{11}Et quand tout le peuple eut achevé de passer, alors l'arche de Yahweh et les prêtres passèrent devant le peuple.
\VS{12}Et les fils de Ruben, les fils de Gad, et la demi-tribu de Manassé passèrent en armes devant les enfants d'Israël, comme Moïse le leur avait dit\FTNT{No. 32:20-29.}.
\VS{13}Ils passèrent, dis-je, dans les plaines de Jérico environ quarante mille hommes en équipage de guerre, devant Yahweh, pour combattre. 
\VS{14}Ce jour-là, Yahweh éleva Josué à la vue de tout Israël, et ils le craignirent, comme ils avaient craint Moïse, tous les jours de sa vie.
\VS{15}Yahweh parla à Josué, et dit :
\VS{16}Ordonne aux prêtres qui portent l'arche du témoignage qu'ils montent hors du Jourdain.
\VS{17}Et Josué donna cet ordre aux prêtres, en disant : Montez hors du Jourdain.
\VS{18}Or sitôt que les prêtres, qui portaient l'arche de l'alliance de Yahweh furent montés hors du milieu du Jourdain, et qu'ils eurent mis la plante de leurs pieds sur le sec, les eaux du Jourdain retournèrent à leur place, et coulèrent comme auparavant sur tous les rivages.
\VS{19}Le peuple donc monta hors du Jourdain le dixième jour du premier mois, et il campa à Guilgal, à l'orient de Jéricho.
\VS{20}Josué aussi dressa à Guilgal les douze pierres qu'ils avaient prises du Jourdain.
\VS{21}Et il parla aux enfants d'Israël et leur dit : Quand vos enfants interrogeront à l'avenir leurs pères, et leur diront : Que signifient ces pierres-ci ?
\VS{22}Vous l'apprendrez à vos enfants, en leur disant : Israël a passé ce Jourdain à sec.
\VS{23}Car Yahweh, votre Dieu, a fait tarir les eaux du Jourdain devant vous jusqu'à ce que vous eussiez passé, comme Yahweh, votre Dieu, l'avait fait à la Mer Rouge, qu'il mit à sec devant nous, jusqu'à ce que nous eussions passé,
\VS{24}afin que tous les peuples de la terre sachent que la main de Yahweh est puissante, et afin que vous ayez toujours la crainte de Yahweh, votre Dieu.
\Chap{5}
\TextTitle{La crainte s'empare des Amoréens}
\VerseOne{}Or il arriva qu'aussitôt que tous les rois des Amoréens qui étaient au-delà du Jourdain, vers l'occident, et tous les rois des Cananéens qui étaient près de la mer, apprirent que Yahweh avait mis à sec les eaux du Jourdain devant les enfants d'Israël, jusqu'à ce que nous eussions passé, leur cœur fut fondu, et il n'y avait plus de courage en eux à cause des enfants d'Israël.
\TextTitle{Israël circoncis à nouveau ; la fin de la manne}
\VS{2}En ce temps-là, Yahweh dit à Josué : Fais-toi des couteaux de pierre tranchants, et circoncis de nouveau les enfants d'Israël, une seconde fois.
\VS{3}Et Josué se fit des couteaux de pierre tranchants, et circoncit les enfants d'Israël sur la colline d'Araloth.
\VS{4}Or la raison pour laquelle Josué les circoncit, c'est que tout le peuple sorti d'Egypte, tous les mâles, dis-je, hommes de guerre étaient morts en chemin dans le désert, après leur sortie d'Egypte.
\VS{5}Et tout le peuple sorti d'Egypte était circoncis, mais aucun du peuple né dans le désert en chemin n'avait été circoncis, après leur sortie d'Egypte.
\VS{6}Car les enfants d'Israël avaient marché dans le désert quarante ans jusqu'à ce que soit consummée toute la nation des hommes de guerre qui étaient sortis d'Egypte, et qui n'avaient point écouté la voix de Yahweh ; auxquels Yahweh avait juré qu'il ne leur laisserait point voir le pays qu'il avait juré à leurs pères de nous donner, pays où coulent le lait et le miel\FTNT{No. 14:32-33.}.
\VS{7}Et il a suscité à leur place leurs enfants que Josué circoncit, parce qu'ils étaient incirconcis ; car on ne les avait pas circoncis pendant le voyage.
\VS{8}Et quand on eut achevé de circoncire tout le peuple, ils restèrent dans leur camp, jusqu'à ce qu'ils soient guéris.
\VS{9}Et Yahweh dit à Josué : Aujourd'hui j'ai roulé de dessus vous l'opprobre de l'Egypte. Et ce lieu-là fut appelé Guilgal jusqu'à ce jour.
\VS{10}Ainsi les enfants d'Israël campèrent à Guilgal, et célébrèrent la Pâque le quatorzième jour du mois, sur le soir, dans les plaines de Jéricho\FTNT{Ex. 12:6.}.
\VS{11}Et dès le lendemain de la Pâque, ils mangèrent du blé du pays, savoir, des pains sans levain et du grain rôti, en ce même jour\FTNT{Ex. 12:39 ; Lé. 2:14}.
\VS{12}Et la manne cessa dès le lendemain de la Pâque, après qu'ils eurent manger du blé du pays ; les enfants d'Israël n'eurent plus de manne, mais ils mangèrent les récoltes de la terre de Canaan cette année-là\FTNT{Ex. 16:35.}.
\TextTitle{Rencontre avec le chef de l'armée de Yahweh}
\VS{13}Or il arriva, comme Josué était près de Jéricho, qu'il leva les yeux et regarda. Voici, un homme qui avait son épée nue à la main, se tenait debout devant lui. Josué alla vers lui et lui dit : Es-tu des nôtres ou de nos ennemis ?
\VS{14}Et il répondit : Non, mais je suis le Chef de l'armée de Yahweh, je viens maintenant. Josué tomba à terre sur son visage, l'adora, et lui dit : Qu'est-ce que mon Seigneur dit à son serviteur ?
\VS{15}Et le Chef de l'armée de Yahweh dit à Josué : Délie tes souliers de tes pieds ; car le lieu sur lequel tu te tiens est saint\FTNT{Ex. 3:5.}. Et Josué fit ainsi.
\Chap{6}
\TextTitle{Jéricho miraculeusement livré à Israël ; Rahab sauvée}
\VerseOne{}Or Jéricho était barricadée et fermée soigneusement, à cause des enfants d'Israël. Personne ne sortait, et personne n'entrait.
\VS{2}Et Yahweh dit à Josué : Regarde, j'ai livré entre tes mains Jéricho et son roi, ses hommes vaillants.
\VS{3}Vous tous donc, hommes de guerre, vous ferez le tour de la ville, en tournant une fois autour d'elle. Tu feras ainsi durant six jours.
\VS{4}Et Sept prêtres porteront sept shofars retentissants devant l'arche. Mais au septième jour, vous ferez sept fois le tour de la ville et les prêtres sonneront des shofars.
\VS{5}Et quand ils sonneront avec la corne de bélier, aussitôt que vous entendrez le son du shofar retentissant, tout le peuple poussera un grand cri de joie et la muraille de la ville tombera sur elle. Et le peuple montera, les hommes devant lui.
\VS{6}Josué donc, fils de Nun, appela les prêtres et leur dit : Portez l'arche de l'alliance et que sept prêtres portent sept shofars devant l'arche de Yahweh.
\VS{7}Il dit aussi au peuple : Passez et faites le tour de la ville, que tous ceux qui seront armés passent devant l'arche de Yahweh.
\VS{8}Et quand Josué eut parlé au peuple, les sept prêtres qui portaient les sept cornes de béliers devant Yahweh passèrent et sonnèrent des cornes. Et l'arche de l'alliance de Yahweh les suivait.
\VS{9}Et les hommes qui étaient armés marchaient devant les prêtres qui sonnaient des shofars ; mais l'arrière-garde suivait derrière l'arche ; on sonnait des shofars en marchant.
\VS{10}Or Josué avait donné cet ordre au peuple, en disant : Vous ne pousserez point de cris de joie et vous ne ferez point entendre votre voix. Et il ne sortira point un seul mot de votre bouche, jusqu'au jour où je vous dirai : Poussez des cris de joie ! Alors vous crierez.
\VS{11}L'arche de Yahweh fit ainsi le tour de la ville, en tournant tout autour une fois, puis on revint au camp, et on y passa la nuit.
\VS{12}Ensuite Josué se leva de bon matin, et les prêtres portèrent l'arche de Yahweh.
\VS{13}Et les sept prêtres qui portaient les sept cornes de bélier devant l'arche de Yahweh se mirent en marche et sonnèrent du shofar. Et les hommes armés allaient devant eux ; puis l'arrière-garde suivait l'arche de Yahweh ; on sonnait des shofars en marchant.
\VS{14}Ainsi ils firent une fois le tour de la ville le deuxième jour, et ils retournèrent au camp. Ils firent de même durant six jours.
\VS{15}Mais quand le septième jour fut venu, ils se levèrent dès le matin à l'aube du jour, et ils firent sept fois le tour de la ville de la même manière ; ce fut le seul jour où ils firent sept fois le tour de la ville.
\VS{16}Et à la septième fois, comme les prêtres sonnaient des shofars, Josué dit au peuple : Poussez des cris de joie, car Yahweh vous a donné la ville !
\VS{17}La ville sera dévouée par le moyen de l'interdit à Yahweh, elle et toutes les choses qui y sont ; seulement Rahab, la prostituée\FTNT{Rahab sauva sa famille par sa foi en Dieu (Ac. 16:31). Voir Josué 2.}, vivra, elle et tous ceux qui seront avec elle dans la maison, parce qu'elle a caché soigneusement les messagers que nous avions envoyés.
\VS{18}Mais quoi qu'il en soit gardez-vous de l'interdit, de peur que vous ne vous mettiez en interdit, et que vous ne mettriez le camp d'Israël en interdit et que vous le troubliez\FTNT{De. 7:26.}.
\VS{19}Mais tout l'argent et tout l'or, tous les objets d'airain et de fer seront consacrés à Yahweh, ils entreront dans le trésor de Yahweh\FTNT{No. 31:54.}.
\VS{20}Le peuple donc poussa de cris de joie et on sonna des shofars. Et quand le peuple entendit le son des shofars, il poussa de grands cris de joie et la muraille tomba sur elle-même\FTNT{Hé. 11:30.}. Alors le peuple monta dans la ville, les hommes devant le peuple. Et ils prirent la ville. 
\VS{21}Et ils la dévouèrent entièrement par le moyen de l'interdit, et passèrent au fil de l'épée tout ce qui était dans la ville, depuis l'homme jusqu'à la femme, depuis l'enfant jusqu'au vieillard, même jusqu'aux bœufs, aux brebis et aux ânes.
\VS{22}Mais Josué dit aux deux hommes qui avaient espionné le pays : Entrez dans la maison de cette femme prostituée, et faites-la sortir de là, avec tous ceux qui lui appartiennent, selon que vous lui avez juré.
\VS{23}Les jeunes hommes donc qui avaient espionné le pays, entrèrent et firent sortir Rahab, et son père, et sa mère et ses frères, avec tous ceux qui lui appartenaient ; ils firent aussi sortir toutes les familles qui lui appartenaient, et les mirent hors du camp d'Israël.
\VS{24}Puis ils allumèrent le feu et brûlèrent la ville et tout ce qui s'y trouvait ; seulement ils mirent l'argent et l'or, les objets d'airain et de fer dans le trésor de la maison de Yahweh.
\VS{25}Ainsi Josué sauva la vie à Rahab la prostituée, la maison de son père, et tous ceux qui lui appartenaient ; et elle a habité au milieu d'Israël jusqu'à ce jour, parce qu'elle avait caché les messagers que Josué avait envoyés pour explorer Jéricho.
\VS{26}Et en ce temps-là Josué jura, en disant : Maudit soit devant Yahweh l'homme qui se mettra à rebâtir cette ville de Jéricho ! Il la fondera sur son premier-né, et il posera ses portes sur son plus jeune fils\FTNT{Cette parole s'est accomplie en 1 R. 16:34.}.
\VS{27}Yahweh fut avec Josué, et sa renommée se répandit dans tout le pays.
\Chap{7}
\TextTitle{Israël battu à Aï suite au péché d'Acan}
\VerseOne{}Mais les enfants d'Israël se rendirent coupables au sujet de l'interdit. Car Acan, fils de Carmi, fils de Zabdi, fils de Zérach, de la tribu de Juda, prit de l'interdit, et la colère de Yahweh s'enflamma contre les enfants d'Israël.
\VS{2}Car Josué envoya de Jéricho des hommes vers Aï, qui est près de Beth-Aven, à l'orient de Béthel. Il leur parla, et dit : Montez, et reconnaissez le pays. Ces hommes donc montèrent et reconnurent Aï.
\VS{3}Et étant retournés vers Josué, ils lui dirent : Que tout le peuple n'y monte point mais qu'environ deux mille ou trois mille hommes y montent, et ils battront Aï. Ne fatigue pas tout le peuple en l'envoyant là, car ils sont en petit nombre.
\VS{4}Ainsi, environ trois mille hommes du peuple y montèrent, mais ils s'enfuirent devant les gens d'Aï.
\VS{5}Et les gens d'Aï leur tuèrent environ trente-six hommes ; car ils les poursuivirent depuis la porte jusqu'à Schebarim, et les battirent à la descente. Le cœur du peuple se fondit et devint comme de l'eau.
\VS{6}Alors Josué déchira ses vêtements, et se jeta sur le visage contre terre, devant l'arche de Yahweh jusqu'au soir, lui et les anciens d'Israël, et ils jettèrent de la poussière sur leur tête.
\VS{7}Et Josué dit : Helas ! Seigneur Yahweh, pourquoi as-tu fait si magnifiquement passer le Jourdain à ce peuple, pour nous livrer entre les mains des Amoréens, et nous faire périr ? Oh ! Que n'avons-nous eu dans l'esprit de demeurer de l'autre côté du Jourdain !
\VS{8}Hélas ! Seigneur, que dirai-je, puisqu'Israël a tourné le dos devant ses ennemis ?
\VS{9}Les Cananéens et tous les habitants du pays l'entendront ; ils nous envelepperont, et ils retrancheront notre nom de dessus la terre. Et que feras-tu à ton grand Nom ?
\VS{10}Alors Yahweh dit à Josué : Lève-toi ! Pourquoi te jettes-tu ainsi le visage contre terre ?
\VS{11}Israël a péché ; ils ont transgressé mon alliance que je leur avais prescrite, même ils ont pris de l'interdit, même ils en ont dérobé, même ils ont menti, et même ils l'ont caché parmi leurs objets\FTNT{Il est impossible de remporter une victoire contre Satan en ayant avec soi des choses qui lui appartiennent (Jn. 14:30). Celui qui pèche est du diable nous dit la Parole de Dieu (1 Jn. 3:4-10). Les grandes victoires sont remportées par ceux qui se sanctifient et invoquent le Nom de Jésus-Christ.}.
\VS{12}C'est pourquoi les enfants d'Israël ne pourront subsister devant leurs ennemis ; ils tourneront le dos devant leurs ennemis ; car ils sont devenus un interdit. Je ne serai plus avec vous si vous ne détruisez pas l'interdit du milieu de vous.
\VS{13}Lève-toi, sanctifie le peuple, et dis : Sanctifiez-vous pour demain ; car ainsi parle Yahweh, le Dieu d'Israël : Il y a de l'interdit au milieu de toi, Israël ! Tu ne pourras subsister et faire face à tes ennemis jusqu'à ce que vous ayez ôté l'interdit du milieu de vous.
\VS{14}Vous vous approcherez donc le matin selon vos tribus ; et la tribu que Yahweh aura saisi s'approchera selon les familles, et la famille que Yahweh aura saisie s'approchera selon les maisons, et la maison que Yahweh aura saisie s'approchera selon les hommes.
\VS{15}Alors celui qui aura été saisi avec l'interdit sera brûlé au feu, lui et tout ce qui lui appartient parce qu'il a transgressé l'alliance de Yahweh, et qu'il a commis une infamie en Israël.
\VS{16}Josué donc se leva de bon matin, et fit approcher Israël selon ses tribus, et la tribu de Juda fut saisie.
\VS{17}Puis il fit approcher les familles de Juda, et la famille de Zérach fut saisie. Puis il fit approcher les hommes de la famille de ceux qui étaient descendants de Zérach, et Zabdi fut saisie.
\VS{18}Et quand il fit approcher la maison de Zabdi par hommes, Acan fils de Carmi, fils de Zabdi, fils de Zérach, de la tribu de Juda, fut saisi.
\VS{19}Josué dit à Acan : Mon fils, je te prie donne gloire à Yahweh, le Dieu d'Israël, et fais-lui confession. Déclare-moi je te prie ce que tu as fait, ne me le cache point.
\VS{20}Et Acan répondit à Josué, et dit : J'ai péché il est vrai, contre Yahweh, le Dieu d'Israël, et voici ce que j'ai fait.
\VS{21}J'ai vu parmi le butin un beau manteau de Schinear\FTNT{Ge. 10:6-12.}, deux cents sicles d'argent et un lingot d'or du poids de cinquante sicles ; je les ai convoités, je les ai pris et voilà, ces choses sont cachées dans la terre au milieu de ma tente, et l'argent est sous le manteau.
\VS{22}Alors Josué envoya des messagers qui coururent à cette tente ; et voici, le manteau était caché dans la tente d'Acan, et l'argent sous le manteau.
\VS{23}Ils les tirèrent donc du milieu de la tente et les apportèrent à Josué et à tous les enfants d'Israël, et ils les déposèrent devant Yahweh.
\VS{24}Alors Josué et tout Israël avec lui, prirent Acan, fils de Zérach, l'argent, le manteau, le lingot d'or, ses fils et ses filles, ses bœufs, ses ânes et ses brebis, sa tente et tout ce qui lui appartenait, et ils les firent monter dans la vallée d'Acor.
\VS{25}Et Josué dit : Pourquoi nous as-tu troublés ? Yahweh te troublera aujourd'hui. Et tout Israël le lapida avec des pierres, et les brûlèrent au feu, après les avoir lapidés avec des pierres.
\VS{26}Et ils dressèrent sur lui un grand monceau de pierres, qui dure jusqu'à ce jour. Et Yahweh apaisa l'ardeur de sa colère. C'est pourquoi ce lieu-là a été appelé jusqu'à aujourd'hui, la vallée d'Acor\FTNT{2 S. 18:17.}.
\Chap{8}
\TextTitle{Victoire d'Israël à Aï}
\VerseOne{}Puis Yahweh dit à Josué : Ne crains point, et ne t'effraie de rien\FTNT{De. 1:21 ; De. 7:18.} ! Prends avec toi tout le peuple propre à la guerre et lève-toi, et monte contre Aï. Regarde, j'ai livré entre tes mains le roi d'Aï et son peuple, sa ville et son pays.
\VS{2}Et tu traiteras Aï et son roi, comme tu as fais Jéricho et son roi : Seulement vous pillerez pour vous le butin et les bêtes. Place des gens en embuscade derrière la ville.
\VS{3}Josué donc se leva avec tout le peuple propre à la guerre, pour monter contre Aï. Josué choisit trente mille vaillants hommes armés, et les envoya de nuit.
\VS{4}Et il leur donna cet ordre en disant : Voyez, vous qui serez en embuscade derrière la ville ; ne vous éloignez pas beaucoup de la ville, mais tenez-vous prêts.
\VS{5}Et moi et tout le peuple qui est avec moi, nous nous approcherons de la ville. Et quand ils sortiront à notre rencontre, comme ils ont fait la première fois, nous nous enfuirons devant eux.
\VS{6}Ainsi ils sortiront après nous, jusqu'à ce que nous les ayons attirés hors de la ville ; car ils diront : Ils fuient devant nous comme la première fois ; parce que nous fuirons devant eux.
\VS{7}Alors vous vous lèverez de l'embuscade, et vous vous saisirez de la ville ; car Yahweh, votre Dieu, la livrera entre vos mains.
\VS{8}Et quand vous aurez pris la ville, vous y mettrez le feu ; vous agirez selon la parole de Yahweh. Regardez, je vous l'ai ordonné.
\VS{9}Josué donc les envoya, et ils allèrent se mettre en embuscade, et se tinrent entre Béthel et Aï, à l'occident d'Aï. Mais Josué resta cette nuit-là au milieu du peuple.
\VS{10}Puis Josué se leva de bon matin, et dénombra le peuple ; et il monta lui et les anciens d'Israël, devant le peuple vers Aï.
\VS{11}Et tout le peuple propre à la guerre qui étaient avec lui, monta et s'approcha ; et ils vinrent en face de la ville et campèrent du côté du nord d'Aï ; et la vallée était entre lui et Aï.
\VS{12}Il prit aussi environ cinq mille hommes, et les mit en embuscade entre Béthel et Aï, à l'occident de la ville.
\VS{13}Après que tout le camp eut pris position au nord de la ville, et l'embuscade à l'occident de la ville, cette nuit-là, Josué s'avança au milieu de la vallée.
\VS{14}Or il arriva qu'aussitôt que le roi de Aï l'eut vu, les hommes de la ville se hâtèrent, et se levèrent de bon matin, et au temps marqué, le Roi et tout son peuple sortirent à la campagne contre Israël pour le combattre. Or il ne savait pas qu'il y eût des gens en embuscade contre lui derrière la ville.
\VS{15}Alors Josué et tout Israël feignirent d'être battus devant eux, et ils s'enfuirent par le chemin du désert.
\VS{16}Alors tout le peuple qui était dans la ville d'Aï, fut assemblé à grand cri pour les poursuivre. Ils poursuivirent Josué, et ils furent ainsi attirés loin de la ville.
\VS{17}Il ne resta pas un seul homme dans Aï ni dans Béthel qui ne sortit contre Israël. Ils laissèrent la ville ouverte, et ils poursuivirent Israël.
\VS{18}Alors Yahweh dit à Josué : Etends vers Aï l'étandard qui est dans ta main, car je la livrerai entre tes mains. Et Josué étendit vers la ville l'étandard qui était dans sa main.
\VS{19}Aussitôt qu'il eut étendu sa main, les hommes qui étaient en embuscade se levèrent précipitamment du lieu où ils étaient ; ils pénétrèrent dans la ville, la prirent, et se hâtèrent de mettre le feu dans la ville.
\VS{20}Et les gens d'Aï, se tournant derrière eux, regardèrent ; et voici, la fumée de la ville montait vers le ciel, et ils n'y eut en eux aucune force pour fuir ça ou là. Et le peuple qui fuyait vers le désert se tourna contre ceux qui le poursuivaient ;
\VS{21}Et Josué et tout Israël, voyant que ceux qui étaient en embuscade avaient pris la ville, et que la fumée de la ville montait, se retournèrent, et frappèrent les gens d'Aï.
\VS{22}Les autres aussi sortirent de la ville contre eux, et ils furent enveloppés par les Israélites ayant les uns d'un côté et les autres de l'autre. Ils furent tellement battus qu'il n'en laissa aucun qui resta en vie ou qui échappât\FTNT{De 7:2.} ;
\VS{23}ils prirent aussi vivant le roi d'Aï, et le présentèrent à Josué.
\VS{24}Et quand les Israélites eurent achevé de tuer tous les habitants d'Aï dans la campagne, dans le désert, où ils les avaient poursuivis, et que tous furent tombés sous le tranchant de l'épée, jusqu'à être entièrement défaits, tous les Israélites revinrent vers Aï, et la frappèrent au tranchant de l'épée.
\VS{25}Et tous ceux qui tombèrent ce jour-là, tant des hommes que des femmes, furent au nombre de douze mille, tous gens d'Aï.
\VS{26}Et Josué ne retira point sa main qu'il tenait étendue avec l'étandard, jusqu'à ce que tous les habitants d'Aï aient été entièrement dévoués par le moyen de l'interdit.
\VS{27}Seulement les Israélites pillèrent pour eux les bêtes et le butin de cette ville-là, suivant ce que Yahweh avait prescrit à Josué\FTNT{No. 31:22-26.}.
\VS{28}Josué donc brûla Aï, et en fit un monceau perpétuel de ruines, jusqu'à aujourd'hui.
\VS{29}Puis il fit pendre le roi d'Aï à un arbre jusqu'au temps du soir. Et comme le soleil se couchait, Josué ordonna qu'on descende de l'arbre son cadavre ; on le jeta à l'entrée de la porte de la ville, puis on dressa sur lui un grand amas de pierres, qui subsiste encore aujourd'hui.
\TextTitle{Sacrifices offerts à Yahweh et lecture de la loi de Moïse}
\VS{30}Alors Josué bâtit un autel à Yahweh, le Dieu d'Israël, sur la montagne d'Ebal,
\VS{31}comme Moïse, serviteur de Yahweh, l'avait ordonné aux enfants d'Israël, ainsi qu'il est écrit dans le livre de la loi de Moïse : Il fit cet autel de pierres brutes sur lesquelles personne ne porta le fer\FTNT{L'autel devait être construit avec des pierres taillées par Dieu lui-même dans la nature (Ex. 20:25). L'Eglise du Seigneur est construite avec des pierres vivantes, taillées par Dieu et non par les hommes (Mt. 16:18). Babylone est construite avec des briques, œuvre des hommes (Ge. 11:1-3).} ; et ils offrirent dessus des holocaustes à Yahweh, et sacrifièrent des sacrifices d'offrande de paix\FTNT{Voir commentaire en Lé. 3:1.}.
\VS{32}Il écrivit aussi là, sur les pierres une copie de la loi que Moïse avait mise par écrit devant les enfants d'Israël.
\VS{33}Et tout Israël, ses anciens, ses officiers et ses juges étaient des deux côtés de l'arche, en face des prêtres qui sont de la race de Lévi, qui portaient l'arche de l'alliance de Yahweh, les étrangers comme les Hébreux naturels, une moitié du côté du mont Garizim\FTNT{Voir Jn. 4:19-24.}, et l'autre moitié du côté du mont Ebal, selon l'ordre qu'avait précédemment donné Moïse, serviteur de Yahweh, de bénir le peuple d'Israël.
\VS{34}Et après cela, il lut tout haut toutes les paroles de la loi, tant les bénédictions que les malédictions, selon tout ce qui est écrit dans le livre de la loi.
\VS{35}Il n'y eut rien de tout ce que Moïse avait prescrit, que Josué ne lise tout haut devant toute l'assemblée d'Israël, des femmes et des petits-enfants, et des étrangers qui marchaient au milieu d'eux.
\Chap{9}
\TextTitle{Josué tombe dans la ruse des Gabaonites}
\VerseOne{}Or, dès que tous les rois qui étaient au-delà du Jourdain, dans la montagne et dans la plaine, et sur toute la côte de la grande mer, jusque près du Liban, les Héthiens, les Amoréens, les Cananéens, les Phéréziens, les Héviens et les Jébusiens, eurent appris ces choses,
\VS{2}ils s'assemblèrent tous d'un commun accord pour faire la guerre à Josué et à Israël.
\VS{3}Mais les habitants de Gabaon\FTNT{Les Gabaonites étaient rusés. Ils poussèrent les Hébreux à faire alliance avec eux, comme le font les faux chrétiens aujourd'hui (Esd. 4 ; Es. 30:1). Il n'y a pas de rapport entre la lumière et les ténèbres (2 Co. 6:14-18). Combien de chrétiens ne se font-ils pas avoir par des loups ravisseurs dans le domaine du mariage ?}, ayant entendu ce que Josué avait fait à Jéricho et à Aï,
\VS{4}usèrent de ruse, car ils se mirent en chemin et contrefirent les ambassadeurs et prirent de vieux sacs pour leurs ânes, et de vieilles outres de vin déchirées et recousues,
\VS{5}Et ils avaient à leurs pieds de vieux souliers raccommodés et de vieux habits sur eux ; et tout le pain qu'ils avaient pour nourriture était sec et moisi.
\VS{6}Et ils arrivèrent auprès de Josué au camp de Guilgal, et lui dirent, ainsi qu'à tous les hommes d'Israël : Nous sommes venus d'un pays éloigné, maintenant donc traitez alliance avec nous.
\VS{7}Et les hommes d'Israël répondirent à ces Héviens : Peut-être que vous habitez au milieu de nous, et comment traiterions-nous alliance avec vous ?
\VS{8}Mais ils dirent à Josué : Nous sommes tes serviteurs. Alors Josué leur dit : Qui êtes-vous ? Et d'où venez-vous ?
\VS{9}Ils lui répondirent : Tes serviteurs sont venus d'un pays très éloigné, sur la renommée de Yahweh, ton Dieu ; car nous avons entendu sa renommée, et toutes les choses qu'il a faites en Egypte,
\VS{10}et tout ce qu'il a fait aux deux rois des Amoréens, qui étaient au-delà du Jourdain, Sihon, roi de Hesbon, et Og, roi de Basan, qui demeurait à Aschtaroth.
\VS{11}Et nos anciens et tous les habitants de notre pays nous ont dit : Prenez avec vous des provisions pour le chemin, et allez au-devant d'eux, et dites-leur : Nous sommes vos serviteurs, et maintenant traitez alliance avec nous.
\VS{12}Voci notre pain : Nous l'avons pris dans nos maisons tout chaud pour notre provision, le jour où nous sommes partis pour venir vers vous, mais maintenant voici, il est devenu sec et moisi.
\VS{13}Et voici aussi les outres de vin neuves que nous avons remplies, elles se sont déchirées ; nos habits et nos souliers sont usés à cause de la longueur de la marche.
\VS{14}Les hommes d'Israël prirent de leur provision, et aucun d'eux ne consulta la bouche de Yahweh\FTNT{Josué et les chefs ne consultèrent pas Yahweh avant de traiter alliance avec les Gabaonites. Prenez le temps dans la prière afin de connaître le cœur de la personne avec laquelle vous voulez marcher.}.
\VS{15}Car Josué fit la paix avec eux, et traita avec eux une alliance par laquelle il devait leur laisser la vie, et les chefs de l'assemblée le leur jurèrent.
\TextTitle{Les Gabaonites démasqués}
\VS{16}Mais il arriva, trois jours après l'alliance traitée avec eux, qu'ils apprirent que c'étaient leurs voisins et qu'ils habitaient parmi eux.
\VS{17}Car les enfants d'Israël partirent, et arrivèrent à leurs villes le troisième jour. Leurs villes étaient Gabaon, Kephira, Beéroth, et Kirjath-Jearim.
\VS{18}Et les enfants d'Israël ne les frappèrent point, parce que les chefs de l'assemblée leur avaient juré par Yahweh, le Dieu d'Israël. Mais toute l'assemblée murmura contre les chefs.
\VS{19}Alors tous les chefs dirent à toute l'assemblée : Nous leur avons juré par Yahweh, le Dieu d'Israël, c'est pourquoi maintenant nous ne pouvons pas les frapper.
\VS{20}Faisons-leur ceci, et qu'on les laisse vivre afin qu'il n'y ait pas de colère contre nous, à cause du serment que nous leur avons fait.
\VS{21}Ils vivront, leur dirent les chefs. Mais ils furent employés à couper le bois et à puiser l'eau pour toute l'assemblée, comme les chefs le leur avaient dit\FTNT{2 S. 21:1-14. La présence des Gabaonites en plein centre de Canaan tendait à isoler les tribus du nord de celles du sud, favorisant ainsi le schisme des deux royaumes (1 R. 12).}.
\VS{22}Car Josué les fit appeler, et leur parla, en disant : Pourquoi nous avez-vous trompés, en nous disant : Nous sommes très éloignés de vous, alors que vous habitez au milieu de nous ?
\VS{23}Maintenant vous êtes maudits ; il y aura toujours des esclaves parmi vous, des coupeurs de bois et des puiseurs d'eau pour la maison de mon Dieu.
\VS{24}Et ils répondirent à Josué, et dirent : Après qu'il ait été exactement rapporté à tes serviteurs les ordres que Yahweh, ton Dieu, avait ordonnés à Moïse, son serviteur, pour vous donner tout le pays et pour en exterminer tous les habitants devant vous ; nous avons extrêmement crains pour nos personnes à cause de vous et nous avons fait ceci. 
\VS{25}Et maintenant nous voici entre tes mains ; fais-nous comme il te semblera bon et juste de nous faire.
\VS{26}Il leur fit donc ainsi et il les délivra de la main des enfants d'Israël, de sorte qu'il ne les tuèrent point.
\VS{27}Et en ce jour-là, Josué les établit coupeurs de bois et puiseurs d'eau pour l'assemblée, et pour l'autel de Yahweh, jusqu'à aujourd'hui, dans le lieu qu'il choisirait.
\Chap{10}
\TextTitle{Josué secoure Gabaon des cinq rois des Amoréens}
\VerseOne{}Or quand Adoni-Tsédek, roi de Jérusalem, entendit que Josué avait pris Aï, et qu'il l'avait entièrement détruite par le moyen de l'interdit, ayant fait à Aï et à son roi, comme il avait fait à Jéricho et à son roi, et que les habitants de Gabaon avaient fait la paix avec Israël, et étaient au milieu d'eux.
\VS{2}Il eut une grande frayeur, parce que Gabaon était une grande ville, comme une ville royale, et elle était plus grande qu'Aï, et parce que tous ses hommes étaient vaillants.
\VS{3}C'est pourquoi Adoni-Tsédek, roi de Jérusalem, envoya dire à Hoham, roi d'Hébron, et à Piream, roi de Jarmuth, et à Japhia, roi de Lakis, et à Debir, roi d'Eglon :
\VS{4}Montez vers moi, et aidez-moi afin que nous frappions Gabaon, car elle a fait la paix avec Josué et avec les enfants d'Israël.
\VS{5}Ainsi cinq rois des Amoréens, savoir, le roi de Jérusalem, le roi d'Hébron, le roi de Jarmuth, le roi de Lakis, et le roi d'Eglon, s'assemblèrent et montèrent avec toutes leurs armées ; et ils campèrent près de Gabaon, et lui firent la guerre.
\VS{6}Alors les gens de Gabaon dirent à Josué au camp de Guilgal : Ne retire point tes mains de tes serviteurs, monte rapidement vers nous, délivre-nous, et donne-nous du secours ; car tous les rois des Amoréens qui habitent aux montagnes se sont rassemblés contre nous.
\VS{7}Josué donc monta de Guilgal, et avec lui tout le peuple qui était propre à la guerre, et tous les hommes forts et vaillants.
\TextTitle{Yahweh accorde à Israël une grande victoire à Makkéda}
\VS{8}Et Yahweh dit à Josué : Ne les crains point, car je les ai livré entre tes mains, et aucun d'eux ne tiendra devant toi.
\VS{9}Josué arriva subitement sur eux, après avoir marché toute la nuit depuis Guilgal.
\VS{10}Yahweh les mit en déroute devant Israël, qui en fit un grand carnage près de Gabaon, et les poursuivit par le chemin de la montagne de Beth-Horon, les battit jusqu'à Azéka, et jusqu'à Makkéda.
\VS{11}Et comme ils s'enfuyaient devant Israël, et qu'ils étaient à la descente de Beth-Horon, Yahweh fit tomber du ciel sur eux de grosses pierres jusqu'à Azéka, et ils périrent ; ceux qui moururent des pierres de grêle furent plus nombreux que ceux qui furent tués avec l'épée par les enfants d'Israël.
\VS{12}Alors Josué parla à Yahweh, le jour où Yahweh livra les Amoréens aux enfants d'Israël, et dit en présence d'Israël : Soleil, arrête-toi sur Gabaon, et toi lune, sur la vallée d'Ajalon !
\VS{13}Et le soleil s'arrêta, et la lune aussi s'arrêta, jusqu'à ce que le peuple ait tiré vengeance de ses ennemis. Cela n'est-il pas écrit dans le livre du Juste ? Le soleil s'arrêta au milieu du ciel et ne se hâta point de se coucher environ un jour entier\FTNT{Ha. 3:11.}.
\VS{14}Et il n'y a point eu de jour semblable à celui-là, ni avant ni après, où Yahweh exauça la voix d'un homme ; car Yahweh combattait pour Israël.
\VS{15}Et Josué, et tout Israël avec lui, retourna au camp à Guilgal.
\VS{16}Au reste, ces cinq rois restants s'enfuirent, et se cachèrent dans une caverne à Makkéda.
\VS{17}Et on le rapporta à Josué, en disant : On a trouvé les cinq rois cachés dans une caverne à Makkéda.
\VS{18}Et Josué dit : Roulez de grosses pierres à l'entrée de la caverne et mettez près d'elle quelques hommes pour les garder.
\VS{19}Mais vous, ne vous arrêtez pas, poursuivez vos ennemis, attaquez-les par-derrière jusqu'au dernier, ne les laissez pas entrer dans leurs villes, car Yahweh, votre Dieu, les a livrés entre vos mains.
\VS{20}Et quand Josué et les enfants d'Israël eurent achevé d'en faire une très grande boucherie, jusqu'à les détruire entièrement, ceux d'entre eux qui s'étaient échappés se retirèrent dans les villes fortifiées,
\VS{21}tout le peuple revint en paix au camp vers Josué à Makkéda, et personne ne remua sa langue contre les enfants d'Israël.
\VS{22}Alors Josué dit : Ouvrez l'entrée de la caverne, et amenez-moi ces cinq rois hors de la caverne.
\VS{23}Et ils firent ainsi, et ils lui amenèrent hors de la caverne ces cinq rois : Le roi de Jérusalem, le roi d'Hébron, le roi de Jarmuth, le roi de Lakis et le roi d'Eglon.
\VS{24}Et après qu'ils eurent amené à Josué ces cinq rois hors de la caverne, Josué appela tous les hommes d'Israël, et dit aux chefs des gens de guerre qui étaient allés avec lui : Approchez-vous, mettez vos pieds sur les cous de ces rois. Ils s'approchèrent, et mirent leurs pieds sur leurs cous\FTNT{Ps. 110:1.}.
\VS{25}Alors Josué leur dit : Ne craignez point, et ne soyez point effrayés, fortifiez-vous, et ayez du courage, car Yahweh traitera ainsi tous vos ennemis contre lesquels vous combattez.
\VS{26}Et après cela, Josué les frappa et les fit mourir, il les fit pendre à cinq arbres, et ils restèrent pendus à ces arbres jusqu'au soir.
\VS{27}Et comme le soleil se couchait, Josué ordonna qu'on les descende de ces arbres, et on les jeta dans la caverne où ils s'étaient cachés, et on mit à l'entrée de la caverne de grosses pierres qui y sont demeurées jusqu'à ce jour\FTNT{De. 21:23.}.
\VS{28}Josué prit aussi Makkéda le même jour, la frappa du tranchant de l'épée, et dévoua à la façon de l'interdit son roi et ses habitants, et ne laissa échapper personne qui était dans cette ville. Et il fit au roi de Makkéda comme il fait au roi de Jéricho.
\TextTitle{Conquête des territoires du sud}
\VS{29}Après cela, Josué, et tout Israël avec lui, passa de Makkéda à Libna, et fit la guerre à Libna.
\VS{30}Et Yahweh la livra aussi entre les mains d'Israël, avec son roi, et il la frappa du tranchant de l'épée, elle et tous ceux qui s'y trouvaient ; il n'en laissa échapper aucune personne qui était dans cette ville ; et il fit à son roi comme il avait fait au roi de Jéricho.
\VS{31}Ensuite Josué, et tout Israël avec lui, passa de Libna à Lakis, campa devant elle, et lui fit la guerre.
\VS{32}Et Yahweh livra Lakis entre les mains d'Israël, qui la prit le deuxième jour, et la frappa du tranchant de l'épée, et toutes les personnes qui s'y trouvaient, comme il avait fait à Libna.
\VS{33}Alors Horam, roi de Guézer, monta pour secourir Lakis. Josué le frappa, lui et son peuple, de sorte qu'il n'en laissa pas échapper un seul homme.
\VS{34}Après cela Josué, et tout Israël avec lui, passa de Lakis à Eglon ; ils campèrent devant elle, et lui firent la guerre.
\VS{35}Ils la prirent le jour même, la frappèrent du tranchant de l'épée ; et Josué dévoua à la façon de l'interdit ce jour-là toutes les personnes qui y étaient, comme il avait fait à Lakis.
\VS{36}Puis Josué, et tout Israël avec lui, monta d'Eglon à Hébron, et ils lui firent la guerre.
\VS{37}Et ils la prirent, et la frappèrent du tranchant de l'épée, avec son roi, toutes ses villes, et toutes les personnes qui y étaient ; il n'en laissa échapper aucune, comme il avait fait à Eglon ; et il dévoua à la façon de l'interdit, toutes les personnes qui y étaient.
\VS{38}Ensuite Josué, et tout Israël avec lui, retourna vers Debir, et ils lui firent la guerre.
\VS{39}Et il la prit, avec son roi et toutes ses villes ; et ils les frappèrent du tranchant de l'épée, et dévouèrent à la façon de l'interdit toutes les personnes qui y étaient ; il n'en laissa échapper aucune. Il fait à Debir et son roi comme il avait fait à Hébron, et comme il avait fait à Libna et à son roi.
\VS{40}Josué donc frappa tout ce pays, la montagne et le midi, la plaine et les coteaux, et tous leurs rois ; il n'en laissa échapper aucun, et il dévoua par le moyen de l'interdit toutes les personnes qui y respiraient, comme Yahweh, le Dieu d'Israël, l'avait ordonné\FTNT{De. 20:16-17.}.
\VS{41}Ainsi Josué les battit depuis Kadès-Barnéa jusqu'à Gaza, et tout le pays de Gosen jusqu'à Gabaon.
\VS{42}Josué prit tous ces rois en même temps et leur pays, parce que Yahweh, le Dieu d'Israël, combattait pour Israël.
\VS{43}Après quoi Josué, et tout Israël avec lui, retourna au camp à Guilgal.
\Chap{11}
\TextTitle{Conquête des territoires du nord}
\VerseOne{}Et aussitôt que Jabin, roi de Hatsor, eut appris ces choses, il envoya des messagers à Jobab, roi de Madon, au roi de Schimron, et au roi d'Acschaph,
\VS{2}et aux rois qui habitaient vers le nord, aux montagnes et dans la plaine, vers le midi de Kinnéreth, dans la vallée, et sur les hauteurs de Dor vers l'occident,
\VS{3}aux Cananéens qui étaient à l'orient et à l'occident, aux Amoréens, aux Héthiens, aux Phéréziens, aux Jébusiens dans les montagnes, et aux Héviens au pied de la montagne de l'Hermon, dans le pays de Mitspa.
\VS{4}Ils sortirent donc avec toutes leurs armées, un grand peuple par leur grand nombre, comme le sable qui est sur le bord de la mer, il y avait aussi des chevaux et des chars en très grand nombre.
\VS{5}Tous ces rois se réunirent, et campèrent ensemble près des eaux de Mérom, pour combattre contre Israël.
\VS{6}Et Yahweh dit à Josué : Ne les crains point, car demain, à cette même heure, je les livrerai tous, blessés à mort, devant Israël. Tu couperas les jarrets à leurs chevaux, et brûleras au feu leurs chars\FTNT{2 S. 8:4.}.
\VS{7}Josué donc, et tous les gens de guerre avec lui vinrent subitement sur eux près des eaux de Mérom, et ils se précipitèrent au milieu d'eux.
\VS{8}Et Yahweh les livra entre les mains d'Israël ; ils les battirent, et les poursuivirent jusqu'à Sidon la grande, jusqu'aux eaux de Misrephoth-Maïm, et jusqu'à la vallée de Mitspa vers l'orient, et ils les battirent tellement qu'ils ne laissèrent aucun survivant.
\VS{9}Et Josué leur fit comme Yahweh lui avait dit ; il coupa les jarrets de leurs chevaux, et brûla au feu leurs chars.
\VS{10}A son retour, et dans le même temps, Josué prit Hatsor, et frappa son roi avec l'épée ; car Hatsor avait été auparavant la capitale de tous ces royaumes.
\VS{11}On frappa aussi du tranchant de l'épée et l'on dévoua à la façon de l'interdit tous ceux qui s'y trouvaient, il ne resta rien de ce qui respirait, et l'on brûla au feu Hatsor.
\VS{12}Josué prit aussi toutes les villes de ces rois, et tous leurs rois, et les frappa du tranchant de l'épée, et il les dévoua à la façon de l'interdit, comme Moïse, serviteur de Yahweh, l'avait ordonné.
\VS{13}Mais Israël ne brûla aucune des villes situées sur des collines, excepté de Hatsor seule, que Josué brûla.
\VS{14}Et les enfants d'Israël pillèrent pour eux tout le butin de ces villes et le bétail ; mais ils frappèrent du tranchant de l'épée tous les hommes, jusqu'à ce qu'ils les aient exterminés, ils n'y laissèrent aucun qui respirait.
\VS{15}Comme Yahweh l'avait ordonné à Moïse son serviteur, ainsi Moïse l'avait ordonné à Josué ; et Josué le fit ainsi ; de sorte qu'il n'omit rien de tout ce que Yahweh avait ordonné à Moïse. 
\TextTitle{Josué s'empare de tout le pays}
\VS{16}Josué donc prit tout ce pays-là, la montagne et tout le pays du midi, avec tout le pays de Gosen, la vallée et la plaine, la montagne d'Israël et ses vallées.
\VS{17}Depuis la montagne de Halak, qui s'élève vers Séir, jusqu'à Baal-Gad dans la vallée du Liban, au pied de la montagne d'Hermon. Il prit aussi tous leurs rois, les battit et les fit mourir.
\VS{18} Josué fit la guerre plusieurs jours contre tous ces rois.
\VS{19}Il n'y eut aucune ville qui fit la paix avec les enfants d'Israël, excepté les Héviens qui habitaient à Gabaon ; ils les prirent toutes par la guerre.
\VS{20}Car cela venait de Yahweh, qu'ils endurcissent leur cœur pour qu'ils sortent en bataille contre Israël, afin qu'il les dévoue à la façon de l'interdit, sans qu'il y ait pour eux de miséricorde, et qu'il les extermine, comme Yahweh l'avait ordonné à Moïse\FTNT{Ex. 4:21 ; De. 2:30 ; 1 R. 12:15.}.
\VS{21}En ce même temps-là aussi, Josué se mit en marche, et il extermina les Anakim des montagnes d'Hébron, de Debir, d'Anab, et de toute la montagne de Juda, et de toute la montagne d'Israël ; Josué, dis-je, les dévoua à la façon de l'interdit avec leurs villes.
\VS{22}Il ne resta aucun Anakim dans le pays des enfants d'Israël ; il n'en resta seulement qu'à Gaza, à Gath et à Asdod\FTNT{2 S. 21:20.}.
\VS{23}Josué donc prit tout le pays, suivant tout ce que Yahweh avait dit à Moïse. Et Josué le donna en héritage à Israël, selon leurs portions, et leurs tribus. Et le pays fut en repos et sans avoir guerre.
\Chap{12}
\TextTitle{Liste des rois vaincus par Moïse et Josué}
\VerseOne{}Voici les rois du pays que les enfants d'Israël frappèrent, et dont ils possédèrent le pays de l'autre côté du Jourdain, vers l'orient, depuis le torrent de l'Arnon jusqu'à la montagne de l'Hermon, et toute la plaine vers l'orient.
\VS{2}Savoir, Sihon, roi des Amoréens, qui habitait à Hesbon, et qui dominait depuis Aroër, qui est sur le bord du torrent de l'Arnon, et depuis le milieu du torrent, sur la moitié de Galaad, jusqu'au torrent de Jabbok, qui est la frontière des enfants d'Ammon\FTNT{De. 3:8-16.} ;
\VS{3}et depuis la plaine jusqu'à la mer de Kinnéreth vers l'orient, et jusqu'à la mer de la plaine, qui est la mer salée, vers l'orient, au chemin de Beth-Jeschimoth ; et depuis le midi sur le pied du Pisga.
\VS{4}Et les contrées d'Og, roi de Basan, qui était seul reste des Rephaïm, et qui habitait à Aschtaroth et à Edréï.
\VS{5}Et sa domination s'étendait sur la montagne de l'Hermon, sur Salca, et sur tout Basan, jusqu'à la frontière des Gueschuriens et des Maacathiens, et sur la moitié de Galaad, frontière de Sihon, roi de Hesbon.
\VS{6}Moïse, serviteur de Yahweh, et les enfants d'Israël, les battirent ; et Moïse, serviteur de Yahweh, en donna la possession aux Rubénites, aux Gadites, et à la demi-tribu de Manassé\FTNT{No. 32:33.}.
\VS{7}Voici les rois du pays que Josué et les enfants d'Israël frappèrent de ce côté-ci du Jourdain vers l'occident, depuis Baal-Gad, dans la vallée du Liban, jusqu'à la montagne de Halak qui monte vers Séir, et que Josué donna aux tribus d'Israël en possession, selon leurs portions,
\VS{8}pays consistant en montagnes et en vallées, en plaines et en collines, en pays de désert et de midi: Les Héthiens, les Amoréens, les Cananéens, les Phéréziens, les Héviens et les Jébusiens.
\VS{9}Le roi de Jéricho, un ; le roi d'Aï, près de Béthel, un ;
\VS{10}le roi de Jérusalem, un ; le roi d'Hébron, un ;
\VS{11}le roi de Jarmuth, un ; le roi de Lakis, un ;
\VS{12}le roi d'Eglon, un ; le roi de Guézer, un ;
\VS{13}le roi de Debir, un ; le roi de Guéder, un ;
\VS{14}le roi de Horma, un ; le roi d'Arad, un ;
\VS{15}le roi de Libna, un ; le roi d'Adullam, un ;
\VS{16}le roi de Makkéda, un ; le roi de Béthel, un ;
\VS{17}le roi de Tappuach, un ; le roi de Hépher, un ;
\VS{18}le roi d'Aphek, un ; le roi de Lascharon, un ;
\VS{19}le roi de Madon, un ; le roi de Hatsor, un ;
\VS{20}le roi de Schimron-Meron, un ; le roi d'Acschaph, un ;
\VS{21}le roi de Taanac, un ; le roi de Meguiddo, un ;
\VS{22}le roi de Kédesch, un ; le roi de Jokneam, au Carmel, un ;
\VS{23}le roi de Dor, sur les hauteurs de Dor, un ; le roi de Gojim, près de Guilgal, un ;
\VS{24}le roi de Thirtsa, un ; en tout trente et un rois.
\Chap{13}
\TextTitle{Les territoires de Ruben, de Gad et de la demi-tribu de Manassé}
\VerseOne{}Or, quand Josué fut devenu vieux, fort avancé en âge, Yahweh lui dit : Tu es devenu vieux, fort avancé en âge, et il te reste encore un très grand pays à posséder.
\VS{2}Voici le pays qui reste, toutes les contrées des Philistins, et des Gueschuriens,
\VS{3}depuis le Schichor, qui coule devant l'Egypte, jusqu'à la frontière d'Ekron au nord, contrée qui doit être tenue pour Cananéenne, et qui est occupée par les cinq princes des Philistins, celui de Gaza, celui d'Asdod, celui d'Askalon, celui de Gath, celui d'Ekron, et par les Avviens ;
\VS{4}du côté du midi, tout le pays des Cananéens, et Meara qui est aux Sidoniens, jusqu'à Aphek, jusqu'à la frontière des Amoréens ;
\VS{5}le pays qui appartient aux Guibliens, et tout le Liban, vers l'orient, depuis Baal-Gad, au pied de la montagne d'Hermon, jusqu'à l'entrée de Hamath ;
\VS{6}tous les habitants de la montagne, depuis le Liban jusqu'aux eaux de Misrephoth-Maïm, tous les Sidoniens. Je les chasserai moi-même devant les fils d'Israël. Donne seulement ce pays en héritage par le sort à Israël, comme je te l'ai prescrit.
\VS{7}Maintenant donc divise ce pays en héritage aux neuf tribus, et à la demi-tribu de Manassé.
\VS{8}Avec l'autre moitié de laquelle les Rubénites et les Gadites ont pris leur héritage, lequel Moïse leur a donné au delà du Jourdain, vers l'orient, selon que Moïse, serviteur de Yahweh, le leur a donné ;
\VS{9}depuis Aroër, qui est sur le bord du torrent de l'Arnon, et la ville qui est au milieu de la vallée, et toute la plaine de Médeba, jusqu'à Dibon ;
\VS{10}et toutes les villes de Sihon, roi des Amoréens, qui régnait à Hesbon, jusqu'à la frontière des enfants d'Ammon ;
\VS{11}et Galaad, et les territoires des Gueschuriens et des Maacathiens, toute la montagne de l'Hermon, et tout Basan jusqu'à Salca ;
\VS{12}tout le royaume d'Og en Basan, qui régnait à Aschtaroth, et à Edréï, et qui était resté le seul reste des Rephaïm ; Moïse battit ces rois, et les chassa.
\VS{13}Or les fils d'Israël ne chassèrent point les Gueschuriens et les Maacathiens, mais les Gueschuriens et les Maacathiens ont habité au milieu d'Israël jusqu'à ce jour.
\VS{14}Seulement il ne donna point d'héritage à la tribu de Lévi ; les sacrifices consumés par le feu devant Yahweh, le Dieu d'Israël, tel fut son héritage, comme il le lui avait dit\FTNT{No. 18:20-24 ; De. 10:9 ; De. 18:2 ; Ez. 44:28.}.
\VS{15}Moïse donc donna un héritage à la tribu des fils de Ruben selon leurs familles.
\VS{16}Et leurs frontières furent depuis Aroër qui est sur le bord du torrent d'Arnon, et de la ville qui est au milieu du torrent, et toute la plaine qui est près de Médeba.
\VS{17}Hesbon et toutes ses villes, qui étaient dans la plaine, Dibon, Bamoth-Baal, Beth-Baal-Meon,
\VS{18}Jahats, Kedémoth et Méphaath,
\VS{19}Kirjathaïm, Sibma, Tséreth-Haschachar sur la montagne de la vallée,
\VS{20}Beth-Peor, les coteaux du Pisga et Beth-Jeschimoth,
\VS{21}et toutes les villes de la plaine, et tout le royaume de Sihon, roi des Amoréens qui régnait à Hesbon ; Moïse l'avait battu, lui et les princes de Madian, Evi, Rékem, Tsur, Hur, et Réba, princes qui relevaient de Sihon, et qui habitaient dans le pays.
\VS{22}Les enfants d'Israël firent passer aussi par l'épée Balaam\FTNT{Voir No. 22. Balaam était l'exemple type du prophète corrompu, soucieux de tirer profit de son service.}, fils de Beor, le devin, avec les autres qui y furent tués.
\VS{23}Et les frontières des enfants d'Israël fut le Jourdain et sa frontière. Tel fut l'héritage des fils de Ruben, selon leurs familles ; savoir, ces villes-là et leurs villages\FTNT{No. 34:14-15.}.
\VS{24}Moïse donna aussi un héritage à la tribu de Gad, pour les fils de Gad, selon leurs familles.
\VS{25}Et leur pays fut Jaezer, et toutes les villes de Galaad et la moitié du pays des enfants d'Ammon, jusqu'à Aroër, qui est vis-à-vis de Rabba,
\VS{26}et depuis Hesbon jusqu'à Ramath-Mitspé, et Bethonim, et depuis Mahanaïm jusqu'à la frontière de Debir,
\VS{27}et, dans la vallée, Beth-Haram, Beth-Nimra, Succoth et Tsaphon, reste du royaume de Sihon, roi de Hesbon, ayant le Jourdain pour frontière jusqu'à l'extrémité de la mer de Kinnéreth, de l'autre côté du Jourdain, vers l'orient.
\VS{28}Tel fut l'héritage des fils de Gad, selon leurs familles ; savoir, les villes et leurs villages.
\VS{29}Moïse donna aussi à la demi-tribu de Manassé un héritage, qui est resté à la demi-tribu des fils de Manassé, selon leurs familles.
\VS{30}Leur pays fut depuis Mahanaïm, tout Basan, et tout le royaume d'Og, roi de Basan, et tous les villages de Jaïr qui sont en Basan, soixante villes.
\VS{31}Et la moitié de Galaad, Aschtaroth et Edréï, villes du royaume d'Og en Basan, furent aux fils de Makir, fils de Manassé, à la moitié des enfants de Makir, selon leurs familles.
\VS{32}Ce sont là les pays que Moïse avait donnés en héritage, lorsqu'il était dans les plaines de Moab, de l'autre côté du Jourdain, vis-à-vis de Jéricho, à l'orient.
\VS{33}Mais Moïse ne donna point d'héritage à la tribu de Lévi ; car Yahweh, le Dieu d'Israël, fut leur héritage, comme il le lui avait dit.
\Chap{14}
\TextTitle{Caleb reçoit Hébron}
\VerseOne{}Voici les terres que les enfants d'Israël eurent pour héritage dans le pays de Canaan, ce que partagèrent entre eux le prêtre Eléazar, Josué, fils de Nun, et les chefs des familles des tribus des enfants d'Israël.
\VS{2}Selon le sort de leur héritage ; comme Yahweh l'avait ordonné par le moyen de Moïse ; savoir, à neuf tribus et à la demi-tribu\FTNT{No. 26:55.}.
\VS{3}Car Moïse avait donné un héritage aux deux tribus et à la demi-tribu de l'autre côté du Jourdain, mais il n'avait point donné de part aux Lévites parmi eux.
\VS{4}Parce que les fils de Joseph, savoir, Manassé et Ephraïm, formaient deux tribus ; et l'on ne donna point de part aux Lévites dans le pays, excepté des villes pour habitation, et les faubourgs pour leurs troupeaux, et pour le reste de leurs biens.
\VS{5}Les enfants d'Israël firent comme Yahweh l'avait ordonné à Moïse, et ils partagèrent le pays.
\VS{6}Or les fils de Juda s'approchèrent de Josué à Guilgal ; et Caleb, fils de Jephunné, le Kenizien, lui dit : Tu sais la parole que Yahweh a déclarée à Moïse, homme de Dieu, à mon sujet et au tien à Kadès-Barnéa\FTNT{No. 14:24 ; No. 32:12 ; De. 1:36.}.
\VS{7}J'étais âgé de quarante ans quand Moïse, serviteur de Yahweh, m'envoya à Kadès-Barnéa pour espionner le pays, et je lui fis un rapport avec droiture de cœur.
\VS{8}Et mes frères qui étaient montés avec moi découragèrent le cœur du peuple, mais moi je persévérai à suivre Yahweh, mon Dieu.
\VS{9}Et ce jour-là Moïse jura, en disant : La terre que ton pied a foulée sera ton héritage à perpétuité, pour toi et pour tes fils, parce que tu as persévéré à suivre Yahweh, mon Dieu.
\VS{10}Or maintenant voici, Yahweh m'a fait vivre comme il l'a dit. Il y a déjà quarante-cinq ans que Yahweh déclarait cette parole à Moïse, lorsqu'Israël marchait dans le désert. Et maintenant voici, je suis aujourd'hui âgé de quatre-vingt-cinq ans.
\VS{11}Et je suis encore aujourd'hui aussi vigoureux que j'étais le jour où Moïse m'envoya ; et j'ai maintenant la même force que j'avais alors pour le combat, soit pour sortir et pour entrer.
\VS{12}Maintenant, donne-moi donc cette montagne, dont Yahweh a parlé ce jour-là ; car tu as appris en ce jour qu'il s'y trouve des Anakim, et qu'il y a de grandes villes fortifiées. Yahweh sera peut-être avec moi, et je les chasserai, comme Yahweh a dit.
\VS{13}Josué donc bénit Caleb, fils de Jephunné, et lui donna Hébron pour héritage.
\VS{14}C'est ainsi que Caleb, fils de Jephunné, le Kenizien, a eu jusqu'à ce jour Hébron pour héritage, parce qu'il avait persévéré à suivre Yahweh, le Dieu d'Israël.
\VS{15}Or Hébron s'appelait autrefois Kirjath-Arba ; et Arba avait été le plus grand homme parmi les Anakim. Le pays fut en repos et sans guerre.
\Chap{15}
\TextTitle{Le territoire de Juda}
\VerseOne{} Ce sont ici la part échue par le sort à la tribu des enfants de Juda, selon leurs familles ; à la frontière d'Edom, au désert de Tsin, vers le midi, fut la dernière extrémité de leurs pays vers le midi ;
\VS{2}tellement que leur frontière, du côté du midi, fut la dernière extrémité de la mer Salée, depuis le bras qui regarde vers le midi. 
\VS{3}Et elle devait sortir vers le midi de la montée d'Akrabbim, et passer vers Tsin ; et montant du midi de Kadès-Barnéa, passer à Hetsron ; puis montant vers Addar, se tourner vers Karkaa ; 
\VS{4}puis, passant vers Atsmon, sortir au torrent d'Egypte ; tellement que les extrémités de cette frontière devaient se rendre à la mer. Ce sera là, dit Josué, votre frontière, du côté du midi.
\VS{5}Et la frontière vers l'orient était la mer salée jusqu'à l'embouchure du Jourdain. La frontière du côté du nord sera depuis la langue de mer, qui est à l'embouchure du Jourdain.
\VS{6}Et cette frontière montera jusqu'à Beth-Hogla, et passera du côté du nord de Beth-Araba ; et cette frontière montera jusqu'à la pierre de Bohan, fils de Ruben.
\VS{7}Puis cette frontière montera vers Debir, depuis la vallée d'Acor, et même vers le nord, du côté de Guilgal, qui est vis-à-vis de la montée d'Adummim, au sud du torrent. Puis cette frontière passera près des eaux d'En-Schémesch, et ses extrémités se prolongeront à En-Roguel.
\VS{8}Puis cette frontière montera de là par la vallée de Ben-Hinnom, au côté du midi de Jebus, qui est Jérusalem, puis cette frontière montera jusqu'au sommet de la montagne, qui est vis-à-vis de la vallée de Hinnom, à l'occident, et à l'extrémité de la vallée des Rephaïm, au nord.
\VS{9}Et cette frontière s'alignera, depuis le sommet de la montagne jusqu'à la source des eaux de Nephthoach, et continuera vers les villes de la montagne d'Ephron, puis cette frontière s'alignera à Baala, qui est Kirjath-Jearim.
\VS{10}Et cette frontière se tournera depuis Baala, vers l'occident, jusqu'à la montagne de Séir, puis elle traversera le côté nord de la montagne de Jearim, à Kesalon, puis descendait à Beth-Schémesch, et passera par Thimna.
\VS{11}Et cette frontière sortira jusqu'au côté d'Ekron, vers le nord et cette frontière s'alignera vers Schicron, puis ayant passé la montagne de Baala, elle se sortira jusqu'à Jabneel ; tellement que les extrémités de cette frontière se rendront à la mer. 
\VS{12}Or la frontière du côté de l'occident sera ce qui est vers la grande mer et ses limites. Telles furent de tous les côtés les frontières des fils de Juda, selon leurs familles.
\VS{13}Au reste, on donna à Caleb, fils de Jephunné, une part au milieu des fils de Juda, comme Yahweh l'avait ordonné à Josué ;savoir, Kirjath-Arba, or Arba était père d'Anak ; et Kirjath-Arba c'est Hébron.
\VS{14}Et Caleb chassa de là les trois fils d'Anak : Schéschaï, Ahiman, et Talmaï, fils d'Anak.
\VS{15}Et de là il monta contre les habitants de Debir ; Debir s'appelait autrefois Kirjath-Sépher.
\VS{16}Et Caleb dit : Je donnerai ma fille Acsa pour femme à celui qui battra Kirjath-Sépher, et la prendra\FTNT{Jg. 1:12-14.}.
\VS{17}Et Othniel, fils de Kenaz, frère de Caleb, la prit ; et Caleb lui donna sa fille Acsa pour femme.
\VS{18}Et il arriva que comme elle s'en allait, elle l'incita à demander à son père un champ ; puis elle descendit impétueusement de dessus son âne, et Caleb lui dit : Qu'as-tu ? 
\VS{19}Elle répondit : Donne-moi un présent, puisque tu m'as donné une terre du sud, donne-moi aussi des sources d'eau. Et il lui donna les sources supérieures et les sources inférieures.
\VS{20}Tel fut l'héritage de la tribu des fils de Juda, selon leurs familles.
\VS{21}Les villes situées dans la contrée du midi, à l'extrémité de la tribu des fils de Juda, près de la frontière d'Edom, étaient : Kabtseel, Eder, Jagur,
\VS{22}Kina, Dimona, Adada,
\VS{23}Kédesch, Hatsor, Ithnan,
\VS{24}Ziph, Thélem, Bealoth,
\VS{25}Hatsor-Hadattha, Kerijoth-Hetsron qui est Hatsor,
\VS{26}Amam, Schema, Molada,
\VS{27}Hatsar-Gadda, Heschmon, Beth-Paleth,
\VS{28}Hatsar-Schual, Beer-Schéba, Bizjothja,
\VS{29}Baala, Ijjim, Atsem,
\VS{30}Eltholad, Kesil, Horma,
\VS{31}Tsiklag, Madmanna, Sansanna,
\VS{32}Lebaoth, Schilhim, Aïn et Rimmon. Total des villes : Vingt-neuf villes, et leurs villages.
\VS{33}Dans la plaine : Eschthaol, Tsorea, Aschna,
\VS{34}Zanoach, En-Gannim, Tappuach, Enam,
\VS{35}Jarmuth, Adullam, Soco, Azéka,
\VS{36}Schaaraïm, Adithaïm, Guedéra et Guedérothaïm ; quatorze villes, et leurs villages.
\VS{37}Tsenan, Hadascha, Migdal-Gad,
\VS{38}Dilean, Mitspé, Joktheel,
\VS{39}Lakis, Botskath, Eglon,
\VS{40}Cabbon, Lachmas, Kithlisch,
\VS{41}Guedéroth, Beth-Dagon, Naama, et Makkéda ; seize villes, et leurs villages.
\VS{42}Libna, Ether, Aschan,
\VS{43}Jiphtach, Aschna, Netsib,
\VS{44}Keïla, Aczib et Maréscha ; neuf villes, et leurs villages.
\VS{45}Ekron, et les villes de son ressort, et ses villages.
\VS{46}Depuis Ekron et à l'occident, toutes les villes près d'Asdod, et leurs villages.
\VS{47}Asdod, les villes de son ressort, et ses villages, Gaza, les villes de son ressort, et ses villages, jusqu'au torrent d'Egypte, et à la grande mer, qui sert de limite.
\VS{48}Dans la montagne : Schamir, Jatthir, Soco,
\VS{49}Danna, Kirjath-Sanna, qui est Debir,
\VS{50}Anab, Eschthemo, Anim,
\VS{51}Gosen, Holon, et Guilo ; onze villes et leurs villages.
\VS{52}Arab, Duma, Eschean,
\VS{53}Janum, Beth-Tappuach, Aphéka,
\VS{54}Humta, Kirjath-Arba, qui est Hébron, et Tsior ; neuf villes, et leurs villages.
\VS{55}Maon, Carmel, Ziph, Juta,
\VS{56}Jizreel, Jokdeam, Zanoach,
\VS{57}Kaïn, Guibea, et Thimna ; dix villes, et leurs villages.
\VS{58}Halhul, Beth-Tsur, Guedor,
\VS{59}Maarath, Beth-Anoth, et Elthekon ; six villes, et leurs villages.
\VS{60}Kirjath-Baal, qui est Kirjath-Jearim, et Rabba ; deux villes, et leurs villages.
\VS{61}Au désert : Beth-Araba, Middin, Secaca,
\VS{62}Nibschan, Ir-Hammélach, et En-Guédi : Six villes et leurs villages.
\VS{63}Au reste, les fils de Juda ne purent pas chasser les Jébusiens qui habitaient à Jérusalem, c'est pourquoi les Jébusiens ont habité avec les fils de Juda à Jérusalem jusqu'à ce jour.
\Chap{16}
\TextTitle{Le territoire d'Ephraïm}
\VerseOne{}La part échue par le sort aux fils de Joseph depuis le Jourdain près de Jéricho, aux eaux de Jéricho, vers l'orient qui est le désert ; montant de Jéricho par la montagne jusqu'à Béthel.
\VS{2}Et cette frontière devait sortir de Béthel à Luz, puis passer vers la frontière des Arkiens jusqu'à Atharoth.
\VS{3}Et elle devait descendre tirant vers l'occident, vers la frontière des Japhléthiens, jusqu'à celle de Beth-Horon la basse et jusqu'à Guézer, de sorte que ses extrémités aboutissent à la mer.
\VS{4}Ainsi les fils de Joseph, savoir, Manassé et Ephraïm, reçurent leur héritage.
\VS{5}Or la frontière des fils d'Ephraïm, selon leurs familles, la frontière de leur héritage était à l'orient, Atharoth-Addar, jusqu'à Beth-Horon la haute.
\VS{6}Et cette frontière devait sortir vers la mer à Micmethath, du côté du nord ; et cette frontière devait se tourner vers l'orient jusqu'à Thaanath-Silo, et passant du côté d'orient, se rendre à Janoach.
\VS{7}Puis descendre de Janoach à Atharoth et à Naaratha, se rencontrer à Jéricho, et sortir au Jourdain.
\VS{8}Et cette frontière devait aller de Tappuach, vers l'occident, jusqu'au torrent de Kana, tellement que ses extrémités devaient se rendre à la mer. Ce fut là l'héritage de la tribu des fils d'Ephraïm, selon leurs familles.
\VS{9}Les fils d'Ephraïm avaient aussi des villes séparées au milieu de l'héritage des fils de Manassé, toutes ces villes, avec leurs villages.
\VS{10}Or ils ne chassèrent point les Cananéens qui habitaient à Guézer, c'est pourquoi les Cananéens ont habité parmi Ephraïm jusqu'à ce jour, mais ils furent réduits à la servitude et assujettis à un tribut\FTNT{Jg. 1:29 ; 1 R. 9:16.}.
\Chap{17}
\TextTitle{Le territoire de Manassé}
\VerseOne{}Il y eut aussi une part échut par le sort à la tribu de Manassé qui était le premier-né de Joseph. Quant à Makir, premier-né de Manassé, et père de Galaad, il avait eu Galaad et Basan parce qu'il était un homme de guerre.
\VS{2}Puis on jeta donc le sort pour les autres enfants de Manassé, selon ses familles ; aux fils d'Abiézer, aux fils de Hélek, aux fils d'Asriel, aux fils de Sichem, aux fils de Hépher, et aux fils de Schemida. Ce sont là les enfants mâles de Manassé fils de Joseph, selon leurs familles.
\VS{3}Or Tselophchad, fils de Hépher, fils de Galaad, fils de Makir, fils de Manassé, n'eut point de fils, mais il eut des filles dont voici les noms : Machla, Noa, Hogla, Milca et Thirtsa.
\VS{4}Elles vinrent se présenter devant le prêtre Eléazar, devant Josué, fils de Nun, et devant les princes, en disant : Yahweh a ordonné à Moïse de nous donner un héritage parmi nos frères. C'est pourquoi on leur donna un héritage parmi les frères de leur père, selon l'ordre de Yahweh\FTNT{No. 27:7 ; No. 36:2.}.
\VS{5}Et dix portions échurent à Manassé, outre le pays de Galaad et de Basan, qui est de l'autre côté du Jourdain.
\VS{6}Car les filles de Manassé eurent un héritage parmi ses fils, et le pays de Galaad fut pour les autres des fils de Manassé.
\VS{7}Or la frontière de Manassé fut du côté d'Aser, venant à Micmethath, qui est près de Sichem ; puis cette frontière devait aller à main droite vers les habitants d'En-Tappuach.
\VS{8}Or le pays de Tappuach appartenait à Manassé, mais Tappuach qui était près de la frontière de Manassé, appartenait aux fils d'Ephraïm.
\VS{9}De là, cette frontière devait descendre au torrent de Kana, au midi du torrent. Ces villes étaient à Ephraïm parmi les villes de Manassé. La frontière de Manassé était au côté du nord du torrent, et ses extrémités devaient se rendre à la mer.
\VS{10}Ce qui était vers le midi était à Ephraïm, et celui qui était vers le nord était à Manassé, et la mer leur servait de frontière ; et du côté du nord, les frontières se rencontraient à Aser, à Issacar, vers l'orient.
\VS{11}Car Manassé possédait dans Issacar et dans Aser : Beth-Schean et les villes de son ressort, Jibleam et les villes de son ressort, les habitants de Dor et les villes de son ressort, les habitants d'En-Dor, et les villes de son ressort, les habitants de Thaanac et les villes de son ressort, les habitants de Meguiddo et les villes de son ressort, qui sont trois contrées.
\VS{12}Au reste, les fils de Manassé ne purent pas chasser les habitants de ces villes, et les Cananéens voulurent rester dans le même pays.
\VS{13}Mais lorsque les fils d'Israël furent assez forts, ils assujettirent les Cananéens à un tribut, mais ils ne les chassèrent pas entièrement.
\VS{14}Or les fils de Joseph parlèrent à Josué, et dirent : Pourquoi nous as-tu donné en héritage un seul lot, et une seule part, vu que nous sommes un peuple nombreux, et que Yahweh nous a bénis jusqu'à présent ?
\VS{15}Et Josué leur dit : Si vous êtes un peuple nombreux, montez à la forêt, et vous l'abattrez, pour vous y faire de la place dans le pays des Phéréziens et des Rephaïm, si la montagne d'Ephraïm est trop étroite pour vous.
\VS{16}Et les fils de Joseph répondirent : Cette montagne ne sera pas suffisante pour nous, et tous les Cananéens qui habitent la vallée ont des chars de fer, et ceux qui sont à Beth-Schean, et dans les villes de son ressort, et ceux qui habitent dans la vallée de Jizreel\FTNT{Jg. 1:19 ; Jg. 4:3.}.
\VS{17}Donc Josué parla à la maison de Joseph, à Ephraïm et à Manassé, et dit : Vous êtes un peuple nombreux, et vous avez de grandes forces, vous n'aurez pas qu'une seule part.
\VS{18}Mais vous aurez la montagne, car c'est une forêt que vous abattrez et dont les extrémités vous appartiendront, et vous chasserez les Cananéens, quoiqu'ils aient des chars de fer, et qu'ils soient puissants.
\Chap{18}
\TextTitle{La tente d'assignation à Silo}
\VerseOne{}Or toute l'assemblée des enfants d'Israël s'assembla à Silo\FTNT{Silo fut pendant la période des Juges le centre religieux d'Israël car c'est dans cette ville que l'on avait déposé l'arche jusqu'à ce que le roi David l'amène à Jérusalem (Jos. 18:1 ; 2 S. 6 ; 1 Ch. 15:3). Durant le schisme, Silo, située en Samarie, fit office de capitale du royaume de sud. La ville fut finalement détruite par les Philistins aux alentours de 1050 av. J.-C.}, et ils y posèrent la tente d'assignation, après que le pays leur ait été assujetti. 
\VS{2}Mais il restait sept tribus des enfants d'Israël qui n'avaient pas encore reçu leur héritage.
\VS{3}Josué dit aux enfants d'Israël : Jusqu'à quand négligerez-vous de prendre possession du pays que Yahweh, le Dieu de vos pères, vous a donné ?
\VS{4}Prenez trois hommes de chaque tribu, que j'enverrai. Ils se lèveront, traverseront le pays, traceront un plan en vue de l'héritage, puis ils reviendront auprès de moi.
\VS{5}Ils le diviseront en sept parts ; Juda restera dans ses limites au midi, et la maison de Joseph restera dans ses limites au nord.
\VS{6}Vous donc faites-vous un plan du pays en sept parts, et apportez-le-moi ici. Puis je jetterai pour vous le sort devant Yahweh, notre Dieu.
\VS{7}Et il n'y aura point de part pour les Lévites au milieu de vous, parce que le sacerdoce de Yahweh est leur héritage. Quant à Gad et à Ruben, et à la demi-tribu de Manassé, ils ont reçu leur héritage de l'autre côté du Jourdain, vers l'orient, que Moïse, serviteur de Yahweh, leur a donné.
\VS{8}Ces hommes-là donc se levèrent et s'en allèrent pour tracer un plan du pays, Josué leur donna cet ordre en disant : Allez et traversez le pays, et tracez-en un plan, puis revenez auprès de moi, et je jetterai ici le sort pour vous devant Yahweh, à Silo.
\VS{9}Ces hommes-là donc s'en allèrent, parcoururent le pays, et en tracèrent un plan dans un livre en sept parts selon les villes ; puis ils revinrent auprès de Josué dans le camp à Silo.
\VS{10}Et Josué jeta le sort pour eux à Silo devant Yahweh, et Josué fit le partage du pays entre les enfants d'Israël, selon leurs parts.
\TextTitle{Le territoire de Benjamin}
\VS{11}Et le sort tomba sur la tribu des fils de Benjamin selon leurs familles, et la part qui leur échut par le sort avait ses frontières entre les fils de Juda et les fils de Joseph.
\VS{12}Et leur frontière du côté du nord fut depuis le Jourdain ; et cette frontière devait monter à côté de Jéricho vers le nord, puis monter en la montagne tirant vers l'Occident ; de sorte que ses extrémités devaient se rendre au désert de Beth-aven. 
\VS{13}Puis cette frontière devait passer de là vers Luz, à côté de Luz, qui est Béthel tirant vers le midi ; et cette frontière devait descendre à Hatroth-addar, près de la montagne qui est du côté du midi de Beth-horon la basse. 
\VS{14}Et cette frontière devait s'aligner et tourner du côté occidental qui regarde vers le midi, depuis la montagne qui est vis-à-vis de Beth-horon, vers le midi ; tellement que ses extrémités devaient se rendre à Kirjath Baal, qui est Kirjath Jearim, ville des enfants de Juda. C'est là le côté d'occident. 
\VS{15}Mais le côté méridional est l'extrémité de Kirjath Jearim ; et cette frontière devait sortir vers l'Occident, puis elle devait sortir à la fontaine des eaux de Nephtoah. 
\VS{16}Et cette frontière devait descendre à l'extrémité de la montagne qui est vis-à-vis de la vallée de Ben-Hinnom, dans la vallée des Rephaïm, vers le nord, et descendre par la vallée de Hinnom, sur le côté méridional des Jébusiens, puis descendre jusqu'à En-Roguel.
\VS{17}Et elle devait s'aligner vers le nord, et sortir à En-Schémesch, de là à Gueliloth, qui est vis-à-vis de la montée d'Adummim, et descendre à la pierre de Bohan, fils de Ruben,
\VS{18}et passer sur le côté nord en face d'Araba, et descendre à Araba,
\VS{19}puis cette frontière devait passer à côté de Beth-Hogla vers le nord ; de sorte que les extrémités de cette frontière aboutissent à la langue de la mer salée vers le nord, à l'embouchure du Jourdain vers le midi. C'était la frontière du midi.
\VS{20}Et le Jourdain devait borner du côté de l'orient. Ce fut là l'héritage des fils de Benjamin avec ses frontières tout autour, selon leurs familles.
\VS{21}Les villes de la tribu des fils de Benjamin, selon leurs familles, étaient : Jéricho, Beth-Hogla, Emek-Ketsits,
\VS{22}Beth-Araba, Tsemaraïm, Béthel,
\VS{23}Avvim, Para, Ophra,
\VS{24}Kephar-Ammonaï, Ophni et Guéba ; douze villes et leurs villages.
\VS{25}Gabaon, Rama, Beéroth,
\VS{26}Mitspé, Kephira, Motsa,
\VS{27}Rékem, Jirpeel, Thareala,
\VS{28}Tséla, Eleph, Jebus, qui est Jérusalem, Guibeath et Kirjath ; quatorze villes et leurs villages. Tel fut l'héritage des fils de Benjamin selon leurs familles.
\Chap{19}
\TextTitle{Le territoire de Siméon}
\VerseOne{}La deuxième part échut par le sort à Siméon, pour la tribu des fils de Siméon, selon leurs familles. Leur héritage était parmi l'héritage des fils de Juda\FTNT{Ge. 49:5-7.}.
\VS{2}Ils eurent dans leur héritage Beer-Schéba, Schéba, Molada,
\VS{3}Hatsar-Schual, Bala, Atsem,
\VS{4}Eltholad, Bethul, Horma,
\VS{5}Tsiklag, Beth-Marcaboth, Hatsar-Susa,
\VS{6}Beth-Lebaoth et Scharuchen ; treize villes et leurs villages.
\VS{7}Aïn, Rimmon, Ether, et Aschan ; quatre villes et leurs villages ;
\VS{8}et tous les villages qui étaient autour de ces villes-là jusqu'à Baalath-Beer, qui est Ramath du midi. Tel fut l'héritage de la tribu des fils de Siméon, selon leurs familles.
\VS{9}L'héritage des fils de Siméon fut pris sur la portion des fils de Juda ; car la portion des fils de Juda était trop grande pour eux ; c'est pourquoi les fils de Siméon reçurent leur héritage parmi le leur.
\TextTitle{Le territoire de Zabulon}
\VS{10}La troisième part échut par le sort aux fils de Zabulon, selon leurs familles.
\VS{11}Et leur frontière devait monter vers le quartier devers la mer, même jusqu'à Mareala, puis se rencontrer à Dabbéscheth, et de là au torrent qui est vis-à-vis de Jokneam.
\VS{12}Or cette frontière devait retourner vers Sarid à l'orient, vers le soleil levant, jusqu'à la frontière de Kisloth-Thabor, puis continuer à Dabrath, et monter à Japhia.
\VS{13}De là passer à l'orient, par Guittha-Hépher, par Ittha-Katsin, puis continuer à Rimmon, jusqu'à Néa.
\VS{14}Puis cette frontière devait tourner du côté du nord vers Hannathon, et ses extrémités devaient se rendre à la vallée de Jiphthach-El.
\VS{15}Avec Katthath, Nahalal, Schimron, Jideala, et Bethléhem ; il y avait douze villes et leurs villages.
\VS{16}Tel fut l'héritage des fils de Zabulon selon leurs familles, ces villes-là, et leurs villages.
\TextTitle{Le territoire d'Issacar}
\VS{17}La quatrième part échut par le sort à Issacar, aux fils d'Issacar, selon leurs familles.
\VS{18}Et leur frontière devaient passer par Jizreel, Kesulloth, Sunem,
\VS{19}Hapharaïm, Schion, Anacharath,
\VS{20}Rabbith, Kischjon, Abets,
\VS{21}Rémeth, En-Gannim, En-Hadda et Beth-Patsets ;
\VS{22}elle devait se rencontrer à Thabor, et vers Schachatsima et Beth-Schémesch, et les extrémités de leur frontière devaient se rendre au Jourdain. Seize villes et leurs villages.
\VS{23}Tel fut l'héritage de la tribu des fils d'Issacar, selon leurs familles, ces villes-là et leurs villages.
\TextTitle{Le territoire d'Aser}
\VS{24}La cinquième part échut par le sort à la tribu des fils d'Aser, selon leurs familles.
\VS{25}Et leur frontière fut Helkath, Hali, Béthen, Acschaph,
\VS{26}Allammélec, Amead et Mischeal ; et elle devait se rencontrer à Carmel, au quartier vers la mer, et à Schichor-Libnath.
\VS{27}Puis elle devait retourner vers l'orient, à Beth-Dagon, et se rencontrer à Zabulon, et à la vallée de Jiphthach-El, vers le nord de Beth-Emek et de Neïel, puis sortir vers Cabul, à gauche,
\VS{28}et vers Ebron, Rehob, Hammon et Kana, jusqu'à Sidon la grande.
\VS{29}Puis la frontière devait retourner à Rama, jusqu'à la ville forte de Tyr, et cette frontière devait retourner à Hosa ; de sorte que ses extrémités se rencontrent au quartier qui est vers la mer, par la contrée d'Aczib.
\VS{30}Avec Umma, Aphek et Rehob ; vingt-deux villes et leurs villages.
\VS{31}Tel fut l'héritage de la tribu des fils d'Aser, selon leurs familles ; ces villes-là et leurs villages.
\TextTitle{Le territoire de Nephthali}
\VS{32}La sixième part échut par le sort aux fils de Nephthali, selon leurs familles.
\VS{33}Leur frontière fut depuis Héleph, depuis Allon par Tsaanannim, Adami-Nékeb et Jabneel, jusqu'à Lakkum, et ses extrémités devaient se rendre au Jourdain.
\VS{34}Puis cette frontière devait retourner du côté d'occident, vers Aznoth-Thabor, et sortir de là à Hukkok ; de sorte que du côté du midi elle devait se rencontrer à Zabulon, et du côté d'occident elle devait se rencontrer à Aser et à Juda ; le Jourdain était du côté au soleil levant.
\VS{35}Au reste, les villes fortifiées étaient : Tsiddim, Tser, Hammath, Rakkath, Kinnéreth,
\VS{36}Adama, Rama, Hatsor,
\VS{37}Kédesch, Edréï, En-Hatsor,
\VS{38}Jireon, Migdal-El, Horem, Beth-Anath et Beth-Schémesch ; dix-neuf villes et leurs villages.
\VS{39}Tel fut l'héritage de la tribu des fils de Nephthali, selon leurs familles ; ces villes-là, et leurs villages.
\TextTitle{Le territoire de Dan}
\VS{40}La septième part échut par le sort à la tribu des fils de Dan selon leurs familles.
\VS{41}La limite de leur héritage fut, Tsorea, Eschthaol, Ir-Schémesch,
\VS{42}Schaalabbin, Ajalon, Jithla,
\VS{43}Elon, Thimnatha, Ekron,
\VS{44}Eltheké, Guibbethon, Baalath,
\VS{45}Jehud, Bené-Berak, Gath-Rimmon,
\VS{46}Mé-Jarkon et Rakkon, avec le territoire qui est vis-à-vis de Japho.
\VS{47}Le territoire échu aux fils de Dan était trop petit pour eux. C'est pourquoi les fils de Dan montèrent, et combattirent contre Léschem ; ils s'en emparèrent et la frappèrent du tranchant de l'épée ; ils en prirent possession, s'y établirent, et l'appelèrent Léschem, Dan, du nom de Dan leur père.
\VS{48}Tel fut l'héritage de la tribu des fils de Dan selon leurs familles ; ces villes-là et leurs villages.
\TextTitle{Josué reçoit Thimnath-Sérach}
\VS{49}Après qu'on eut achevé de partager le pays selon ses frontières, les enfants d'Israël donnèrent à Josué, fils de Nun, une possession au milieu d'eux.
\VS{50}Selon l'ordre de Yahweh, ils lui donnèrent la ville qu'il demanda, Thimnath-Sérach, dans la montagne d'Ephraïm. Il rebâtit la ville, et y habita.
\VS{51}Ce sont là les héritages que le prêtre Eléazar, Josué, fils de Nun, et les chefs de pères des tribus des enfants d'Israël partagèrent par le sort à Silo, devant Yahweh, à l'entrée de la tente d'assignation, et ils achevèrent ainsi le partage du pays.
\Chap{20}
\TextTitle{Les six villes de refuge\FTNTT{No. 35.}}
\VerseOne{}Puis Yahweh parla à Josué et dit :
\VS{2}Parle aux enfants d'Israël et dis : Etablissez-vous des villes de refuge comme je vous l'ai ordonné par le moyen de Moïse,
\VS{3}où pourra s'enfuir le meurtrier qui aura tué quelqu'un involontairement, sans intention, et elles vous serviront de refuge devant celui qui a le droit de venger le sang.
\VS{4}Et le meurtrier s'enfuira dans l'une de ces villes, s'arrêtera à l'entrée de la porte de la ville, et il exposera son affaire aux anciens de cette ville-là, ils l'écouteront, et le recevront chez eux dans la ville, et lui donneront une demeure, afin qu'il habite avec eux.
\VS{5}Et quand celui qui a le droit de venger le sang le poursuivra, ils ne livreront pas le meurtrier entre ses mains ; puisque c'est involontairement qu'il a tué son prochain, et qu'il ne le haïssait point auparavant.
\VS{6}Mais il demeurera dans cette ville-là, jusqu'à ce qu'il comparaisse devant l'assemblée pour être jugé, jusqu'à la mort du grand prêtre qui sera en fonction en ce temps-là. Alors le meurtrier s'en retournera, et reviendra dans sa ville et dans sa maison, dans la ville d'où il s'était enfui\FTNT{Ex. 21:13 ; No. 35:9-34 ; De. 19.}.
\VS{7}Ils consacrèrent donc Kédesch, en Galilée, dans la montagne de Nephthali ; Sichem dans la montagne d'Ephraïm ; et Kirjath-Arba, qui est Hébron, dans la montagne de Juda.
\VS{8}Et de l'autre côté du Jourdain, à l'orient de Jéricho, ils choisirent Betser, dans la tribu de Ruben, dans le désert, dans la plaine ; Ramoth en Galaad, dans la tribu de Gad ; et Golan en Basan, dans la tribu de Manassé\FTNT{De. 4:43.}.
\VS{9}Telles furent les villes désignées pour tous les enfants d'Israël et pour l'étranger en séjour au milieu d'eux, afin que quiconque aurait tué quelqu'un involontairement puisse s'y réfugier, et qu'il ne meure pas de la main de celui qui a le droit de venger le sang, avant d'avoir comparu devant l'assemblée.
\Chap{21}
\TextTitle{Les quarante-huit villes des Lévites}
\VerseOne{}Or les chefs des pères de famille des Lévites s'approchèrent d'Eléazar, le prêtre, de Josué, fils de Nun, et des chefs des pères de famille des tribus des enfants d'Israël.
\VS{2}Et leur parlèrent à Silo, dans le pays de Canaan, et dirent : Yahweh a ordonné par Moïse qu'on nous donne des villes pour habiter, et leurs faubourgs pour nos bêtes\FTNT{No. 35:2-3.}.
\VS{3}Alors les enfants d'Israël donnèrent aux Lévites, sur leur héritage, les villes suivantes et leurs faubourgs, d'après l'ordre de Yahweh.
\VS{4}Et on tira au sort pour les familles des Kehathites ; et les Lévites, fils d'Aaron, le prêtre eurent par le sort treize villes de la tribu de Juda, de la tribu de Siméon, et de la tribu de Benjamin.
\VS{5}Les autres fils de Kehath eurent par le sort dix villes des familles de la tribu d'Ephraïm, de la tribu de Dan, et de la demi-tribu de Manassé.
\VS{6}Et les fils de Guerschon eurent par le sort treize villes, des familles de la tribu d'Issacar, de la tribu d'Aser, de la tribu de Nephthali, et de la demi-tribu de Manassé en Basan.
\VS{7}Et les fils de Merari selon leurs familles, eurent douze villes, de la tribu de Ruben, de la tribu de Gad, et de la tribu de Zabulon.
\VS{8}Les enfants d'Israël donnèrent donc par le sort aux Lévites ces villes-là avec leurs faubourgs, comme Yahweh l'avait ordonné par Moïse.
\VS{9}Ils donnèrent donc de la tribu des fils de Juda et de la tribu des fils de Siméon, ces villes, qui vont être nommées par leurs noms,
\VS{10}et qui furent pour les fils d'Aaron, qui étaient des familles des Kehathites, et des fils de Lévi, car le sort les avait indiqués les premiers.
\VS{11}Ils leur donnèrent Kirjath-Arba, qui est Hébron, dans la montagne de Juda, avec ses faubourgs tout autour : Arba était le père d'Anak.
\VS{12}Mais quant au territoire de la ville, et à ses villages, on les donna à Caleb, fils de Jephunné, pour sa possession.
\VS{13}Ils donnèrent donc aux fils d'Aaron, le prêtre, les villes de refuge pour les meurtriers, Hébron, avec ses faubourgs, et Libna avec ses faubourgs.
\VS{14}Jatthir, avec ses faubourgs, Eschthemoa, avec ses faubourgs,
\VS{15}Holon, avec ses faubourgs, Debir, avec ses faubourgs,
\VS{16}Aïn, avec ses faubourgs, Jutta, avec ses faubourgs ; et Beth-Schémesch, avec ses faubourgs ; neuf villes de ces deux tribus-là ;
\VS{17}et de la tribu de Benjamin, Gabaon, avec ses faubourgs, et Guéba, avec ses faubourgs,
\VS{18}Anathoth, avec ses faubourgs, et Almon, avec ses faubourgs ; quatre villes.
\VS{19}Toutes les villes des prêtres, fils d'Aaron, furent treize villes, avec leurs faubourgs.
\VS{20}Quant aux Lévites, appartenant aux familles des autres fils de Kehath, ils eurent par le sort des villes de la tribu d'Ephraïm.
\VS{21}On leur donna donc les villes de refuge pour les meurtriers, Sichem, avec ses faubourgs, dans la montagne d'Ephraïm, et Guézer avec ses faubourgs ;
\VS{22}Kibtsaïm, avec ses faubourgs, et Beth-Horon, avec ses faubourgs ; quatre villes ;
\VS{23}et de la tribu de Dan, Eltheké, avec ses faubourgs ; Guibbethon, avec ses faubourgs,
\VS{24}Ajalon, avec ses faubourgs, Gath-Rimmon, avec ses faubourgs ; quatre villes.
\VS{25}Et de la demi-tribu de Manassé, Thaanac, avec ses faubourgs ; et Gath-Rimmon, avec ses faubourgs, deux villes.
\VS{26}Total des villes : Dix villes avec leurs faubourgs, pour les familles des autres fils de Kehath.
\VS{27}On donna aussi aux fils de Guerschon, d'entre les familles des Lévites : De la demi-tribu de Manassé les villes de refuge pour les meurtriers, Golan en Basan, avec ses faubourgs, et Beeschthra, avec ses faubourgs ; deux villes ;
\VS{28}et de la tribu d'Issacar, Kischjon, avec ses faubourgs, Dabrath, avec ses faubourgs,
\VS{29}Jarmuth, avec ses faubourgs, En-Gannim, avec ses faubourgs ; quatre villes ;
\VS{30}et de la tribu d'Aser, Mischeal, avec ses faubourgs, Abdon, avec ses faubourgs,
\VS{31}Helkath, avec ses faubourgs, et Rehob, avec ses faubourgs ; quatre villes ;
\VS{32}et de la tribu de Nephthali, les villes de refuge pour les meurtriers, Kédesch en Galilée avec ses faubourgs, Hammoth-Dor, avec ses faubourgs, et Karthan, avec ses faubourgs ; trois villes.
\VS{33}Total des villes des Guerschonites, selon leurs familles : Treize villes, et leurs faubourgs.
\VS{34}On donna aussi au reste des Lévites, qui appartenaient aux familles des fils de Merari : De la tribu de Zabulon, Jokneam, avec ses faubourgs, Kartha, avec ses faubourgs,
\VS{35}Dimna, avec ses faubourgs, et Nahalal, avec ses faubourgs ; quatre villes ;
\VS{36}et de la tribu de Ruben, Betser, avec ses faubourgs, et Jahtsa, avec ses faubourgs ;
\VS{37}Kedémoth, avec ses faubourgs, et Méphaath, avec ses faubourgs ; quatre villes ;
\VS{38}et de la tribu de Gad, les villes de refuge pour les meurtriers, Ramoth en Galaad, avec ses faubourgs, et Mahanaïm, avec ses faubourgs,
\VS{39}Hesbon, avec ses faubourgs, et Jaezer, avec ses faubourgs ; en tout quatre villes.
\VS{40}Total des villes qui échurent par le sort aux fils de Merari, selon leurs familles, formant le reste des familles des Lévites : Douze villes.
\VS{41}Total des villes des Lévites qui étaient parmi la possession des enfants d'Israël : Quarante-huit villes, et leurs faubourgs.
\VS{42}Chacune de ces villes avait ses faubourgs autour d'elle ; il en était ainsi de toutes ces villes-là.
\TextTitle{Yahweh accomplit sa promesse}
\VS{43}Yahweh donna donc à Israël tout le pays qu'il avait juré de donner à leurs pères ; ils le possédèrent, et y habitèrent\FTNT{Dieu accomplit toujours ses promesses (Jé. 1:12).}.
\VS{44}Yahweh leur accorda un parfait repos tout autour, selon tout ce qu'il avait juré à leurs pères ; aucun de leurs ennemis ne put leur résister, car Yahweh les livra entre leurs mains.
\VS{45}Il ne tomba pas un seul mot de toutes les bonnes paroles que Yahweh avait dites à la maison d'Israël : Toutes s'accomplirent.
\Chap{22}
\TextTitle{Ruben, Gad et la demi-tribu de Manassé retournent sur leurs terres}
\VerseOne{}Alors Josué appela les Rubénites, les Gadites et la demi-tribu de Manassé.
\VS{2}Et il leur dit : Vous avez gardé tout ce que Moïse, serviteur de Yahweh, vous a prescrit, et vous avez obéi à ma voix dans tout ce que je vous ai ordonné.
\VS{3}Vous n'avez pas abandonné vos frères, depuis une très longue période jusqu'à ce jour ; et vous avez gardé les ordres, les commandements de Yahweh votre Dieu.
\VS{4}Maintenant que Yahweh, votre Dieu, a donné du repos à vos frères, comme il le leur avait dit, retournez et allez dans vos tentes, dans le pays qui vous appartient, et que Moïse, serviteur de Yahweh, vous a donné de l'autre côté du Jourdain\FTNT{No. 32:33 ; De. 3:13 ; De. 29:8.}.
\VS{5}Prenez seulement bien garde d'observer les ordonnances et les lois que Moïse, serviteur de Yahweh, vous a prescrites : Aimez Yahweh votre Dieu, marchez dans toutes ses voies, gardez ses commandements, attachez-vous à lui, et servez-le de tout votre cœur et de toute votre âme\FTNT{De. 10:12.}.
\VS{6}Puis Josué les bénit et les renvoya ; et ils s'en allèrent vers leurs tentes.
\VS{7}Moïse avait donné à la moitié de la tribu de Manassé son héritage en Basan ; et Josué donna à l'autre moitié son héritage avec leurs frères de l'autre côté du Jourdain vers l'occident. Josué les renvoya dans leurs tentes, et les bénit.
\VS{8}Et il leur parla et dit : Vous retournez à vos tentes avec de grandes richesses, une très nombreuse quantité de bétail, avec une quantité considérable d'argent, d'or, d'airain, de fer, et de vêtements. Partagez avec vos frères le butin de vos ennemis.
\VS{9}Ainsi donc les fils de Ruben, les fils de Gad, et la demi-tribu de Manassé s'en retournèrent, et partirent de Silo, dans le pays de Canaan, après avoir quitté les enfants d'Israël, pour s'en aller dans le pays de Galaad, sur la terre de leur possession et où ils s'établirent, suivant ce que Yahweh avait ordonné par Moïse.
\TextTitle{L'autel Ed, sujet d'incompréhension}
\VS{10}Quand ils furent arrivés aux frontières du Jourdain, qui appartiennent au pays de Canaan, les fils de Ruben, les fils de Gad, et la demi-tribu de Manassé y bâtirent un autel, près du Jourdain, un autel dont la grandeur frappait les regards.
\VS{11}Les enfants d'Israël apprirent que l'on disait : Voici, les fils de Ruben, les fils de Gad, et la demi-tribu de Manassé ont bâti un autel en face du pays de Canaan, sur les frontières du Jourdain, du côté des enfants d'Israël.
\VS{12}Lorsque les enfants d'Israël entendirent cela, toute l'assemblée des enfants d'Israël se réunit à Silo, pour monter en guerre contre eux.
\VS{13}Cependant les enfants d'Israël envoyèrent vers les fils de Ruben, vers les fils de Gad, et vers la demi-tribu de Manassé, au pays de Galaad, Phinées, fils du prêtre Eléazar,
\VS{14}et avec lui dix princes, un prince par maison paternelle pour chacune des tribus d'Israël ; tous étaient chefs de maison paternelle parmi les milliers d'Israël.
\VS{15}Ils se rendirent auprès des fils de Ruben, des fils de Gad et de la demi-tribu de Manassé au pays de Galaad, et leur parlèrent, en disant :
\VS{16}Ainsi parle toute l'assemblée de Yahweh : Quelle est cette infidélité que vous avez commise contre le Dieu d'Israël, et pourquoi vous détournez-vous aujourd'hui de Yahweh, en vous bâtissant un autel, pour vous rebeller aujourd'hui contre Yahweh ?
\VS{17}Regardons-nous comme peu de chose l'iniquité de Peor\FTNT{Peor : No. 25:1-9.}, dont nous ne nous sommes pas encore bien purifiés jusqu'à présent, malgré la plaie qu'il attira sur l'assemblée de Yahweh ?
\VS{18}Et vous vous détournez aujourd'hui de Yahweh ! Si vous vous rebellez aujourd'hui contre Yahweh, demain il s'irritera contre toute l'assemblée d'Israël.
\VS{19}Si vous tenez pour impure la terre qui est votre propriété, passez sur la terre qui est la possession de Yahweh, où est fixé le tabernacle de Yahweh, ayez votre possession parmi nous, mais ne vous révoltez point contre Yahweh, et ne soyez point rebelles contre nous, en vous bâtissant un autel, outre l'autel de Yahweh notre Dieu.
\VS{20}Acan\FTNT{Acan : Jos. 7:1-26.}, fils de Zérach, ne commit-il pas une infidélité en prenant des choses dévouées par le moyen de l'interdit, et la colère de Yahweh ne s'enflamma-t-elle pas contre toute l'assemblée d'Israël ? Cependant, cet homme ne fut pas le seul qui périt à cause de son iniquité.
\VS{21}Mais les fils de Ruben, les fils de Gad, et la demi-tribu de Manassé répondirent, et dirent aux chefs des milliers d'Israël :
\VS{22}Dieu\FTNT{Dieu : de l'hébreu « El » : puissant, etc.}, Dieu\FTNT{Dieu : de l'hébreu « elohim » : juge, ange.} Yahweh, Dieu\FTNT{Dieu : de l'hébreu « El » : puissant, etc.}, Dieu\FTNT{Dieu : de l'hébreu « elohim » : juge, ange.}Yahweh, le sait, et Israël lui-même le saura ! Si c'est par rébellion et par infidélité envers Yahweh, alors qu'il ne nous vienne point en aide aujourd'hui.
\VS{23}Si nous nous sommes bâti un autel pour nous détourner de Yahweh, si c'est pour y offrir des holocaustes, ou des offrandes, ou si c'est pour y faire des sacrifices d'offrande de paix, que Yahweh lui-même nous en demande compte !
\VS{24}C'est bien plutôt par une sorte d'inquiétude que nous avons fait cela, en pensant que vos fils pourraient un jour parler à nos fils et leur dire : Qu'y a-t-il de commun entre vous et Yahweh, le Dieu d'Israël ?
\VS{25}Puisque Yahweh a mis le Jourdain pour frontière entre nous et vous, fils de Ruben, et fils de Gad ; vous n'avez point de part à Yahweh ! Et ainsi vos fils feraient qu'un jour nos fils cesseraient de craindre Yahweh\FTNT{Né. 2:20 ; Ac. 8:21.}.
\VS{26}C'est pourquoi nous avons dit : Mettons-nous maintenant à bâtir un autel, non pour des holocaustes ni pour des sacrifices ;
\VS{27}mais afin qu'il serve de témoignage entre nous et vous, et entre nos descendants et les vôtres, que nous voulons servir Yahweh devant sa face par nos holocaustes et nos sacrifices d'expiation et d'offrande de paix, afin que vos fils ne disent pas un jour à nos fils : Vous n'avez point de part à Yahweh\FTNT{Ge. 31:48.} !
\VS{28}C'est pourquoi nous avons dit : Lorsqu'ils nous tiendront ce discours, ou à nos descendants, nous leur dirons : Voyez la forme de l'autel de Yahweh qu'ont fait nos pères, non pour des holocaustes, ni pour des sacrifices, mais afin qu'il soit témoin entre nous et vous.
\VS{29}A Dieu ne plaise que nous nous révoltions contre Yahweh et que nous nous détournions aujourd'hui de Yahweh, en bâtissant un autel pour des holocaustes, pour des offrandes, et pour des sacrifices, outre l'autel de Yahweh notre Dieu, qui est devant son tabernacle !
\VS{30}Or, après que le prêtre Phinées, et les princes de l'assemblée, les chefs des milliers d'Israël qui étaient avec lui, eurent entendu les paroles que les fils de Ruben, les fils de Gad, et les fils de Manassé leur dirent, ils furent satisfaits.
\VS{31}Et Phinées, fils du prêtre Eléazar, dit aux fils de Ruben, aux fils de Gad, et aux fils de Manassé : Nous reconnaissons aujourd'hui que Yahweh est au milieu de nous, puisque vous n'avez point commis cette infidélité contre Yahweh ; vous avez ainsi délivré les enfants d'Israël de la main de Yahweh.
\VS{32}Ainsi Phinées, fils du prêtre Eléazar, et les princes, quittèrent les fils de Ruben, les fils de Gad, et revinrent du pays de Galaad dans le pays de Canaan, auprès des enfants d'Israël, auxquels ils firent un rapport.
\VS{33}Et la chose plut aux enfants d'Israël ; ils bénirent Dieu, et ne parlèrent plus de monter en armes contre eux pour détruire le pays où habitaient les fils de Ruben, et les fils de Gad.
\VS{34}Les fils de Ruben, et les fils de Gad appelèrent l'autel Ed ; car, dirent-ils, il est témoin entre nous que Yahweh est Dieu.
\Chap{23}
\TextTitle{Avertissements de Josué}
\VerseOne{}Or il arriva, plusieurs jours après, que Yahweh ayant donné du repos à Israël de tous les ennemis qui l'entouraient, Josué était vieux, fort avancé en âge.
\VS{2}Et Josué convoqua tout Israël, ses anciens, ses chefs, ses juges, ses officiers, et leur dit : Je suis devenu vieux, fort avancé en âge.
\VS{3}Vous avez vu tout ce que Yahweh, votre Dieu, a fait à toutes ces nations devant vous ; car Yahweh, votre Dieu, est celui qui combat pour vous.
\VS{4}Voyez, je vous ai donné en héritage par le sort, selon vos tribus, ces nations qui sont restées, depuis le Jourdain, et toutes les nations que j'ai exterminées, jusqu'à la grande mer vers le soleil couchant.
\VS{5}Yahweh, votre Dieu, les repoussera devant vous et les chassera ; et vous posséderez leur pays en héritage, comme Yahweh, votre Dieu, vous l'a dit\FTNT{Ex. 14:14 ; Ex. 23:27 ; No. 33:53 ; De. 6:18-19.}.
\VS{6}Appliquez-vous avec force à observer et à mettre en pratique tout ce qui est écrit dans le livre de la loi de Moïse, sans vous en détourner ni à droite ni à gauche\FTNT{De. 5:32 ; De. 28:14.}.
\VS{7}Ne vous mêlez point avec ces nations qui sont restées parmi vous ; et ne faites point mention du nom de leurs dieux, et ne faites jurer personne par eux, ne les servez point, et ne vous prosternez point devant eux\FTNT{Ex. 23:13 ; De. 12:3 ; Jé. 5:7 ; De. 6:14.}.
\VS{8}Mais attachez-vous à Yahweh, votre Dieu, comme vous l'avez fait jusqu'à ce jour\FTNT{De. 11:22.}.
\VS{9}C'est pour cela que Yahweh a chassé devant vous des nations grandes et puissantes ; nul n'a pu vous résister jusqu'à ce jour.
\VS{10}Un seul homme d'entre vous en poursuivait mille ; car Yahweh votre Dieu est celui qui combat pour vous, comme il vous l'a dit\FTNT{Lé. 26:8 ; De. 32:30.}.
\VS{11}Veillez donc attentivement sur vos âmes, afin d'aimer Yahweh, votre Dieu.
\VS{12}Autrement, si vous vous détournez et que vous vous attachez au reste de ces nations qui sont demeurées parmi vous, si vous faites alliance par des mariages avec elles, et si vous formez ensemble des relations,
\VS{13}sachez certainenement que Yahweh, votre Dieu, ne continuera pas à chasser ces nations devant vous ; mais elles seront pour vous un piège et un filet, un fouet dans vos côtés et des épines dans vos yeux, jusqu'à ce que vous ayez péri de dessus cette bonne terre que Yahweh, votre Dieu, vous a donnée\FTNT{Ex. 23:33 ; De. 7:16 ; Jg. 2:3.}.
\VS{14}Voici, je m'en vais aujourd'hui par le chemin de toute la terre. Reconnaissez de tout votre cœur et de toute votre âme qu'aucune de toutes les bonnes paroles prononcées sur vous par Yahweh, votre Dieu, n'est restée sans effet ; toutes se sont accomplies pour vous, aucune n'est restée sans effet\FTNT{Jos. 21:45 ; 2 R. 10:10.}.
\VS{15}Et il arrivera que comme toutes les bonnes paroles que Yahweh, votre Dieu, vous a dites vous sont arrivées ; ainsi Yahweh fera venir sur vous toutes les paroles mauvaises, jusqu'à ce qu'il vous ait exterminés de dessus cette bonne terre que Yahweh, votre Dieu, vous a donnée.
\VS{16}Si vous transgressez l'alliance que Yahweh, votre Dieu, vous a prescrite, et si vous allez servir d'autres dieux et vous prosterner devant eux, la colère de Yahweh s'enflammera contre vous, et vous périrez promptement de dessus cette bonne terre qu'il vous a donnée.
\Chap{24}
\TextTitle{Josué rappelle à Israël son histoire}
\VerseOne{}Josué assembla toutes les tribus d'Israël à Sichem, et il convoqua les anciens d'Israël, ses chefs, ses juges, et ses officiers, qui se présentèrent devant Dieu.
\VS{2}Et Josué dit à tout le peuple : Ainsi parle Yahweh, le Dieu d'Israël : Vos pères, Térach père d'Abraham, et père de Nachor, ont anciennement habité de l'autre côté du fleuve, où ils servaient d'autres dieux.
\VS{3}Mais j'ai pris votre père Abraham de l'autre côté du fleuve, je lui fis parcourir tout le pays de Canaan, je multipliai sa postérité, et lui donnai Isaac\FTNT{Ge. 12 ; Ge. 21:2.}.
\VS{4}Je donnai à Isaac, Jacob et Esaü ; et je donnai à Esaü le mont de Séir, pour le posséder ; mais Jacob et ses fils descendirent en Egypte\FTNT{Ge. 25:24 ; Ge. 36:6.}.
\VS{5}Puis j'envoyai Moïse et Aaron, et je frappai l'Egypte, par les prodiges que j'opérai au milieu d'elle ; puis je vous en fis sortir\FTNT{Ex. 3:10.}.
\VS{6}Je fis donc sortir vos pères hors de l'Egypte, et vous arrivâtes à la mer. Les Egyptiens poursuivirent vos pères avec des chars et des cavaliers, jusqu'à la Mer Rouge\FTNT{Ex. 14:9.}.
\VS{7}Alors ils crièrent à Yahweh. Et il mit des ténèbres entre vous et les Egyptiens, et ramena sur eux la mer, qui les couvrit. Vos yeux ont vu ce que j'ai fait aux Egyptiens. Puis vous restâtes longtemps dans le désert.
\VS{8}Ensuite je vous conduisis dans le pays des Amoréens, qui habitaient de l'autre côté du Jourdain, et ils combattirent contre vous. Mais je les livrai entre vos mains ; vous prîtes possession de leur pays, et je les détruisis devant vous.
\VS{9}Balak\FTNT{Balaak : Voir No. 22:2-14.} aussi, fils de Tsippor, roi de Moab, se leva, et fit la guerre à Israël. Il fit appeler Balaam\FTNT{Balaam : Voir No. 22.}, fils de Beor, pour qu'il vous maudisse.
\VS{10}Mais je ne voulus point écouter Balaam ; il s'agenouilla et vous bénit, et je vous délivrai de la main de Balak.
\VS{11}Et vous passâtes le Jourdain, et arrivâtes près de Jéricho. Les habitants de Jéricho, les Amoréens, les Phéréziens, les Cananéens, les Héthiens, les Guirgasiens, les Héviens et les Jébusiens vous firent la guerre. Je les livrai entre vos mains,
\VS{12}et j'envoyai devant vous des frelons qui les chassèrent loin de votre face, comme les deux rois des Amoréens : Ce ne fut ni par ton épée, ni par ton arc\FTNT{Ex. 23:28 ; De. 7:20.}.
\VS{13}Je vous donnai une terre que vous n'aviez point cultivée, des villes que vous n'aviez point bâties, et que vous habitez, et vous mangez les fruits des vignes et des oliviers que vous n'avez point plantés\FTNT{De. 6:10 ; Ps. 105:44 ; Né. 9:25.}.
\TextTitle{Le peuple choisit de servir Yahweh}
\VS{14}Maintenant, craignez Yahweh, et servez-le avec intégrité et avec fidélité. Ôtez les dieux que vos pères ont servis de l'autre côté du fleuve et en Egypte, et servez Yahweh\FTNT{1 S. 12:23-24 ; Ez. 20:7-44.}.
\VS{15}Et s'il vous déplaît de servir Yahweh, choisissez aujourd'hui qui vous voulez servir, ou les dieux que servaient vos pères au-delà du fleuve, ou les dieux des Amoréens dans le pays desquels vous habitez. Mais moi et ma maison, nous servirons Yahweh.
\VS{16}Alors le peuple répondit, et dit : Que Dieu nous garde d'abandonner Yahweh pour servir d'autres dieux !
\VS{17}Car Yahweh, notre Dieu, est celui qui nous a fait monter, nous et nos pères, hors du pays d'Egypte, de la maison de servitude, qui a fait devant nos yeux ces grands signes, qui nous a gardés dans tout le chemin par lequel nous avons marché, et entre tous les peuples parmi lesquels nous avons passé.
\VS{18}Yahweh a chassé devant nous tous les peuples, et même les Amoréens qui habitaient ce pays. Nous servirons aussi Yahweh, car il est notre Dieu.
\VS{19}Josué dit au peuple : Vous ne pourrez pas servir Yahweh, car c'est un Dieu Saint, qui est jaloux, il ne pardonnera point votre rébellion et vos péchés.
\VS{20}Lorsque vous abandonnerez Yahweh et que vous servirez les dieux des étrangers, il reviendra vous faire du mal, et il vous consumera après vous avoir fait du bien.
\VS{21}Le peuple dit à Josué : Non ! Car nous servirons Yahweh.
\VS{22}Et Josué dit au peuple : Vous êtes témoins contre vous-mêmes que c'est vous qui avez choisi Yahweh pour le servir. Et ils répondirent : Nous en sommes témoins.
\VS{23}Maintenant donc ôtez les dieux étrangers qui sont au milieu de vous, et tournez votre cœur vers Yahweh, le Dieu d'Israël.
\VS{24}Et le peuple répondit à Josué : Nous servirons Yahweh notre Dieu et nous obéirons à sa voix.
\VS{25}Ce jour-là, Josué traita alliance avec le peuple, et lui donna des lois et des ordonnances à Sichem.
\VS{26}Josué écrivit ces paroles dans le livre de la loi de Dieu. Il prit aussi une grande pierre\FTNT{Cette Pierre entend selon Josué, elle est également appelée « témoin ». Jésus-Christ, la Pierre angulaire (Es. 8:13-16) est le témoin fidèle (Ap. 19:11). Cette Pierre suivait les Hébreux dans le désert (1 Co. 10:1-3).}, qu'il dressa là sous le chêne qui était dans le lieu consacré à Yahweh.
\VS{27}Josué dit à tout le peuple : Voici, cette pierre servira de témoin contre nous, car elle a entendu toutes les paroles que Yahweh nous a déclarées ; elle servira de témoin contre vous, afin que vous ne reniiez pas votre Dieu.
\VS{28}Puis Josué renvoya le peuple, chacun dans son héritage.
\TextTitle{Mort de Josué et d'Eléazar ; ensevelissement des os de Joseph (Ge. 50 :26)}
\VS{29}Or il arriva, après ces choses, que Josué, fils de Nun, serviteur de Yahweh, mourut, âgé de cent dix ans.
\VS{30}Et on l'ensevelit dans le territoire de son héritage, à Thimnath-Sérach, dans la montagne d'Ephraïm, du côté du nord de la montagne de Gaasch.
\VS{31}Et Israël servit Yahweh tout le temps de Josué, et tout le temps des anciens qui survécurent à Josué, qui avaient connu toutes les œuvres que Yahweh avait faites pour Israël.
\VS{32}Les os de Joseph\FTNT{(Ge. 50:25 ; Ex. 13:19 ; Hé. 11:22).}, que les enfants d'Israël avaient rapportés d'Egypte, furent ensevelis à Sichem, dans la portion du champ que Jacob avait achetée des fils de Hamor, père de Sichem, pour cent kesita, et qui appartint à l'héritage des fils de Joseph.
\VS{33}Et Eléazar, fils d'Aaron, mourut, on l'enterra à Guibeath-Phinées, qui avait été donnée à son fils Phinées, dans la montagne d'Ephraïm.
\PPE{}
\end{multicols}

%\clearpage\ShortTitle{Juges}\BookTitle{Juges}\BFont
\noindent\hrulefill
{\footnotesize
\textit{
\bigskip
{\centering{}
\\Auteur : Inconnu
\\(Heb. : Shoftim)
\\Signification : Être juge, prononcer, punir
\\Thème : Défaites et délivrances
\\Date de rédaction : Environ 1100 av. J.-C.\\}
}
%\bigskip
\textit{
\\A la mort de Josué et des anciens, il s’éleva en Israël une nouvelle génération qui n'avait pas connu l’expérience du désert. Elle fit ce qui est mal aux yeux de Dieu, l’abandonna et tomba dans l’idolâtrie. Ainsi, la colère de Yahweh s’abattit sur Israël et il livra le peuple entre les mains de ses ennemis. Dans ces temps de troubles, Dieu suscita des juges - douze hommes et une femme - pour délivrer Israël de ses oppresseurs. Aussi longtemps que le juge était en vie, Israël était en paix. Mais dès qu’il venait à mourir, le peuple se corrompait de nouveau et ses oppressions recommençaient.\bigskip
}
}
\par\nobreak\noindent\hrulefill
\begin{multicols}{2}
\Chap{1}
\TextTitle{Poursuite de la conquête de Canaan}
\VerseOne{}Or il arriva qu'après la mort de Josué, les enfants d'Israël consultèrent Yahweh, en disant : Qui de nous montera le premier contre les Cananéens pour leur faire la guerre ?
\VS{2}Et Yahweh répondit : Juda montera ; voici, j'ai livré le pays entre ses mains.
\VS{3}Juda dit à Siméon son frère : Monte avec moi dans mon lot et nous ferons la guerre aux Cananéens ; et j'irai aussi avec toi dans ton lot. Ainsi Siméon alla avec lui.
\TextTitle{Victoires de Juda ; Caleb prend possession d’Hébron}
\VS{4}Juda monta, et Yahweh livra les Cananéens et les Phéréziens entre leurs mains ; ils battirent dix mille hommes à Bézek.
\VS{5}Et ils trouvèrent Adoni-Bézek à Bézek ; ils l'attaquèrent et frappèrent les Cananéens et les Phéréziens.
\VS{6}Adoni-Bézek s'enfuit mais ils le poursuivirent ; et l'ayant pris, ils lui coupèrent les pouces des mains et des pieds.
\VS{7}Alors Adoni-Bézek dit : Soixante-dix rois, dont les pouces des mains et des pieds avaient été coupés, ramassaient du pain sous ma table ; Dieu me rend ce que j’ai fait. On l’amena à Jérusalem et il y mourut\FTNT{Es. 33:1}.
\VS{8}Les fils de Juda firent la guerre contre Jérusalem et la prirent, ils frappèrent ses habitants du tranchant de l'épée et mirent le feu à la ville.
\VS{9}Puis les fils de Juda descendirent pour faire la guerre aux Cananéens, qui habitaient la montagne, la contrée du midi et la plaine.
\VS{10}Juda marcha contre les Cananéens qui habitaient à Hébron ; or le nom d'Hébron était auparavant Kirjath-Arba ; et il battit Schéschaï, Ahiman et Talmaï\FTNT{Jos. 15:14.}.
\VS{11}De là, il marcha contre les habitants de Debir ; Debir s’appelait auparavant Kirjath-Sépher\FTNT{Jos. 15:15.}.
\VS{12}Caleb dit : Je donnerai ma fille Acsa pour femme à celui qui frappera Kirjath-Sépher et qui la prendra\FTNT{Jos. 15:16.}.
\VS{13}Othniel, fils de Kenaz, frère cadet de Caleb, s’en empara ; et Caleb lui donna sa fille Acsa pour femme.
\VS{14}Et il arriva que comme elle s'en allait, elle l'incita à demander à son père un champ. Puis elle descendit  impétueusement de dessus son âne ; et Caleb lui dit : Qu'as-tu ?\FTNT{Jos. 15:18.}
\VS{15}Elle lui répondit : Donne-moi un présent, puisque tu m'as donné une terre du midi ; donne-moi aussi des sources d'eau. Et Caleb lui donna les sources supérieures et les sources inférieures.
\VS{16}Les fils du Kénien, beau-père de Moïse, montèrent de la ville des palmiers avec les fils de Juda, dans le désert de Juda, qui est au midi d'Arad, et ils allèrent et demeurèrent avec le peuple\FTNT{Jg. 4:11}.
\VS{17}Puis Juda se mit en marche avec Siméon son frère et ils frappèrent les Cananéens qui habitaient à Tsephath ; et ils détruisirent la ville par le moyen de l'interdit, c'est pourquoi on appela la ville du nom de Horma.
\VS{18}Juda prit aussi Gaza avec ses territoires ; Askalon avec ses territoires ; et Ekron avec ses territoires.
\TextTitle{Des victoires en demi-teintes}
\VS{19}Yahweh fut avec Juda et il se rendit maître de la montagne, mais il ne pût chasser les habitants de la vallée, parce qu'ils avaient des chars de fer.
\VS{20}On donna Hébron à Caleb, comme Moïse l'avait dit ; et il en chassa les trois fils d'Anak\FTNT{No. 14:24}.
\VS{21}Quant aux fils de Benjamin, ils ne chassèrent pas les Jébusiens qui habitaient à Jérusalem ; c'est pourquoi les Jébusiens ont habité avec les fils de Benjamin à Jérusalem jusqu'à ce jour.
\VS{22}Ceux de la maison de Joseph montèrent aussi contre Béthel, et Yahweh fut avec eux.
\VS{23}Ceux de la maison de Joseph firent explorer Béthel, dont le nom était auparavant Luz.
\VS{24}Les espions virent un homme qui sortait de la ville, et ils dirent : Nous te prions de nous montrer un endroit par où l’on puisse entrer dans la ville, et nous te ferons grâce.
\VS{25}Il leur montra par où ils pourraient entrer dans la ville. Et ils frappèrent la ville du tranchant de l'épée ; mais ils laissèrent aller cet homme et toute sa famille.
\VS{26}Puis cet homme se rendit dans le pays des Héthiens ; il bâtit une ville et lui donna le nom de Luz, nom qu’elle a porté jusqu'à ce jour.
\VS{27}Manassé ne chassa pas les habitants de Beth-Schean et des villes de son ressort, de Thaanac et des villes de son ressort, de Dor et des villes de son ressort, les habitants de Jibleam et des villes de son ressort, les habitants de Meguiddo et des villes de son ressort ; et les Cananéens persistèrent à habiter dans ce pays-là.
\VS{28}Il est vrai qu’il arriva que quand Israël fut devenu plus fort, il assujettit les Cananéens à un tribut mais il ne les chassa pas entièrement.
\VS{29}Ephraïm ne chassa pas les Cananéens qui habitaient à Guézer, et les Cananéens habitèrent avec lui à Guézer.
\VS{30}Zabulon ne chassa pas les habitants de Kitron, ni les habitants de Nahalol ; et les Cananéens habitèrent avec lui et lui furent assujettis à un tribut.
\VS{31}Aser ne chassa pas les habitants d’Acco, ni les habitants de Sidon, ni ceux d’Achlal, ni d'Aczib, ni d'Helba, ni d'Aphik, ni de Rehob ;
\VS{32}Mais ceux d'Aser habitèrent parmi les Cananéens, habitants du pays ; car ils ne les chassèrent pas.
\VS{33}Nephthali ne chassa pas les habitants de Beth-Schémesch, ni les habitants de Beth-Anath, mais il habita parmi les Cananéens habitants du pays ; et les habitants de Beth-Schémesch, et de Beth-Anath lui furent assujettis au tribut.
\VS{34}Les Amoréens repoussèrent les enfants de Dan dans la montagne et ne les laissèrent pas descendre dans la vallée.
\VS{35}Les Amoréens voulurent encore habiter à Har-Hérès, à Ajalon et à Schaalbim ; mais la main de la maison de Joseph étant devenue plus forte, ils furent assujettis au tribut.
\VS{36}Le territoire des Amoréens s'étendait depuis la montée d’Akrabbim, depuis Séla et en dessus.
\Chap{2}
\TextTitle{Le peuple repris pour sa désobéissance}
\VerseOne{}Or l'Ange de Yahweh monta de Guilgal à Bokim, et dit : Je vous ai fait monter hors d'Egypte, et je vous ai fait entrer dans le pays que j’avais juré à vos pères, et j’ai dit : Je n’enfreindrai jamais mon alliance que j’ai traitée avec vous\FTNT{Ge. 17:7.} ;
\VS{2}Et vous aussi vous ne traiterez pas alliance avec les habitants de ce pays, vous démolirez leurs autels. Mais vous n'avez pas obéi à ma voix. Pourquoi avez-vous fait cela\FTNT{Ex. 23:32 ; De. 7:2 ; De. 12:3.} ?
\VS{3}J’ai dit alors : Je ne les chasserai pas devant vous, mais ils seront à vos côtés, et leurs dieux vous seront un piège\FTNT{Ex. 23:33 ; Jos. 23:13.}.
\VS{4}Et il arriva que, comme l'Ange de Yahweh disait ces paroles à tous les enfants d'Israël, le peuple éleva la voix et pleura.
\VS{5}C'est pourquoi ils appelèrent ce lieu Bokim et ils y offrirent des sacrifices à Yahweh.
\VS{6}Josué renvoya le peuple, et les enfants d'Israël allèrent chacun dans son héritage pour prendre possession du pays\FTNT{Jos. 24:28–32.}.
\VS{7}Le peuple servit Yahweh tout le temps de Josué, et tout le temps des anciens qui survécurent à Josué et qui avaient vu toutes les grandes œuvres que Yahweh avait faites en faveur d’Israël\FTNT{Jos. 24:31.}.
\VS{8}Puis Josué, fils de Nun, serviteur de Yahweh, mourut, âgé de cent dix ans\FTNT{Jos. 24:29.}.
\VS{9}On l’ensevelit dans le territoire qu’il avait eu en partage à Thimnath-Hérès, dans la montagne d'Ephraïm, au nord de la montagne de Gaasch\FTNT{Jos. 24:30.}.
\TextTitle{La nouvelle génération abandonne Yahweh}
\VS{10}Toute cette génération fut recueillie auprès de ses pères, puis il s’éleva après elle une autre génération, qui ne connaissait pas Yahweh ni les œuvres qu'il avait faites en faveur d’Israël.
\VS{11}Les enfants d'Israël firent alors ce qui est mal aux yeux de Yahweh et ils servirent les Baals\FTNT{Baal est un dieu phénicien qui, sous les Ramessides, était assimilé dans la mythologie égyptienne à Seth et à Montou. Baal est un dieu d’origine sémite. Il est le dieu de la pluie. Son nom – «~le maître~» ou «~l’époux~»- se retrouve partout dans le Moyen-Orient, depuis les zones peuplées par les sémites jusqu’aux colonies phéniciennes, dont Carthage. Il était invariablement accompagné d’une divinité féminine (Astarté, Ishtar, Tanit...). Voir Jg. 3:7 ; Jg. 8:33 ; Jg. 10:6.}.
\VS{12}Et ils abandonnèrent Yahweh, le Dieu de leurs pères, qui les avait fait sortir du pays d’Égypte, ils allèrent après d'autres dieux, d'entre les dieux des peuples qui les entouraient ; et ils se prosternèrent devant eux, irritant ainsi Yahweh.
\VS{13}Ils abandonnèrent donc Yahweh, et servirent Baal et les Astartés\FTNT{Astarté ou Ashtart en punico-phénicien, ou Ishtar, dérivé de la déesse de Babylone, était  généralement assimilée à la déesse Mésopotamienne Innana. Déesse phénicienne présentant un caractère belliqueux, elle était souvent représentée à califourchon sur son cheval, accompagnant et protégeant le souverain. Elément féminin du couple suprême qu’elle formait avec Baal, celle-ci assumait des fonctions variées : protectrice du souverain et de sa dynastie ou encore des marins.  Comme pour la plupart des divinités féminines primordiales de l’antiquité (et de la proto-histoire), son culte était  lié à la fertilité et à la fécondité. Parfois vénérée sous le nom de Tanit, elle sera assimilée à Vénus par les Romains sous le nom officiel de Venere Ericina.}.
\VS{14}La colère de Yahweh s'enflamma contre Israël. Il les livra entre les mains de pillards\FTNT{Lorsqu’un enfant de Dieu ouvre la porte au péché, il s’expose aux pillards, c’est-à-dire à Satan et ses démons (Jn. 10:10).} qui les pillèrent, il les vendit entre les mains de leurs ennemis d'alentour, de sorte qu'ils ne purent plus résister face à leurs ennemis\FTNT{Ps. 44:12-13 ; Es. 50:1.}.
\VS{15}Partout où ils allaient, la main de Yahweh était contre eux pour leur faire du mal, comme Yahweh l’avait dit et leur avait juré. Ils furent dans une grande détresse\FTNT{Lé. 26:25 ; De. 28:25.}.
\TextTitle{Yahweh suscite des libérateurs : Les juges}
\VS{16}Yahweh leur suscita des juges\FTNT{Les Juges étaient principalement des libérateurs de l’oppression des ennemis d’Israël.} et ils les délivrèrent de la main de ceux qui les pillaient.
\VS{17}Mais ils ne voulurent pas écouter leurs juges, ils se prostituèrent auprès d'autres dieux, se prosternèrent devant eux. Ils se détournèrent promptement du chemin qu’avaient suivi leurs pères et ils n’obéirent pas comme eux aux commandements de Yahweh.
\VS{18}Quand Yahweh leur suscitait des juges, Yahweh était avec le juge, et il les délivrait de la main de leurs ennemis pendant tout le temps de la vie du juge ; car Yahweh se repentait à cause de leurs gémissements contre ceux qui les opprimaient et les tourmentaient.
\VS{19}Puis il arrivait que quand le juge mourrait, ils se corrompaient de nouveau plus que leurs pères en allant après d'autres dieux pour les servir et se prosterner devant eux, et ils persévéraient dans la même conduite et dans la même voie obstinée\FTNT{Jg. 3:12.}.
\TextTitle{IYahweh éprouve Israël et ne chasse pas ses ennemis}
\VS{20}C'est pourquoi la colère de Yahweh s'enflamma contre Israël, et il dit : Puisque cette nation a transgressé mon alliance que j'avais prescrite à leurs pères et puisqu’ils n'ont pas obéi à ma voix,
\VS{21}aussi je ne chasserai plus devant eux aucune des nations que Josué laissa quand il mourut\FTNT{Jos. 23:13.},
\VS{22}afin d'éprouver par elles Israël, pour savoir s'ils prendront garde ou non de suivre la voie de Yahweh, comme leurs pères y ont pris garde.
\VS{23}Yahweh laissa en repos ces nations qu'il n'avait pas livrées entre les mains de Josué et il ne se hâta pas de les chasser\FTNT{Jg. 3:1-3.}.
\Chap{3}
\VerseOne{}Voici les nations que Yahweh laissa pour éprouver par elles Israël, tous ceux qui n'avaient pas connu toutes les guerres de Canaan\FTNT{Jg. 2:21-23.} ; 
\VS{2}afin qu’au moins les générations des enfants d'Israël connaissent et apprennent la guerre, ceux qui ne l’avaient pas connue auparavant.
\VS{3}Ces nations étaient : Les cinq princes des Philistins, tous les Cananéens, les Sidoniens et les Héviens qui habitaient la montagne du Liban depuis la montagne de Baal-Hermon, jusqu'à l'entrée de Hamath\FTNT{No. 13:22.}.
\VS{4}Ces nations, dis-je, servirent à éprouver Israël pour voir s'ils obéiraient aux commandements que Yahweh avait donnés à leurs pères par le moyen de Moïse.
\TextTitle{Israël se mélange aux nations païennes}
\VS{5}Ainsi les enfants d'Israël habitèrent parmi les Cananéens, les Héthiens, les Amoréens, les Phéréziens, les Héviens et les Jébusiens.
\VS{6}Ils prirent leurs filles pour femmes, ils donnèrent leurs filles à leurs fils et servirent leurs dieux.
\VS{7}Les enfants d'Israël firent ce qui est mal aux yeux de Yahweh, ils oublièrent Yahweh et servirent les Baals et les Astartés\FTNT{Jg. 2:11.}.
\TextTitle{Othniel, premier juge suscité par Yahweh}
\VS{8}C'est pourquoi la colère de Yahweh s'enflamma contre Israël, et il les vendit entre la main de Cuschan-Rischeathaïm, roi de Mésopotamie. Et les enfants d'Israël furent asservis à Cuschan-Rischeathaïm durant huit ans.
\VS{9}Puis les enfants d'Israël crièrent à Yahweh, et Yahweh leur suscita un libérateur qui les délivra, Othniel, fils de Kenaz, frère cadet de Caleb.
\VS{10}L’Esprit de Yahweh fut sur lui. Il devint juge en Israël, et il sortit pour la guerre. Yahweh livra entre ses mains Cuschan-Rischeathaïm, roi de Mésopotamie ; et sa main fut puissante contre Cuschan-Rischeathaïm.
\VS{11}Le pays fut en repos pendant quarante ans. Puis Othniel, fils de Kenaz, mourut.
\TextTitle{Ehud, juge en Israël}
\VS{12}Les enfants d'Israël firent encore ce qui est mal aux yeux de Yahweh ; et Yahweh fortifia Eglon, roi de Moab, contre Israël, parce qu'ils avaient fait ce qui est mauvais aux yeux de Yahweh.
\VS{13}Eglon réunit auprès de lui les fils d'Ammon et les Amalécites et il se mit en marche. Il battit Israël et ils s'emparèrent de la ville des palmiers\FTNT{Palmiers: un autre nom de Jéricho.}.
\VS{14}Et les enfants d'Israël furent asservis à Eglon, roi de Moab, durant dix-huit ans.
\VS{15}Puis les enfants d'Israël crièrent à Yahweh, et Yahweh leur suscita un libérateur, Ehud, fils de Guéra, Benjamite, qui ne se servait pas de sa main droite. Les enfants d'Israël envoyèrent par lui un présent à Eglon, roi de Moab.
\VS{16}Ehud se fit une épée à deux tranchants, de la longueur d'une coudée\FTNT{Une coudée correspond environ à 45 cm.} et il la ceignit sous ses vêtements, sur sa cuisse droite.
\VS{17}Il offrit le présent à Eglon, roi de Moab ; et Eglon était un homme fort gras.
\VS{18}Or il arriva que lorsqu’il eut achevé d’offrir le présent, il renvoya le peuple qui avait apporté le présent.
\VS{19}Mais Ehud revint depuis les idoles de pierre, qui étaient près de Guilgal et il dit : Ô roi ! J’ai quelque chose de secret à te dire. Et il lui répondit : Tais-toi ! Et tous ceux qui étaient auprès de lui sortirent de là.
\VS{20}Ehud s'approcha de lui, comme il était assis seul dans sa chambre d'été, et il dit : J'ai un mot à te dire de la part de Dieu, alors le roi se leva du trône.
\VS{21}Et Ehud avança sa main gauche, tira l'épée de son côté droit et la lui enfonça dans le ventre.
\VS{22}Et la poignée entra après la lame, et la graisse serra tellement la lame, qu’il ne pouvait retirer l’épée du ventre, et il en sortit de l’excrément.
\VS{23}Après cela, Ehud sortit par le portique, ferma après lui les portes de la chambre et tira le verrou.
\VS{24}Quand il fut sorti, les serviteurs d'Eglon vinrent et regardèrent ; et voici, les portes de la chambre étaient fermées au verrou. Ils dirent : Sans doute il se couvre les pieds dans sa chambre d’été.
\VS{25}Et ils attendirent tant qu'ils en furent déconcertés ; et voyant qu'il n'ouvrait pas les portes de la chambre, ils prirent la clef et ouvrirent ; et voici, leur maître était mort, étendu à terre.
\VS{26}Mais Ehud s'échappa pendant qu’ils hésitaient ; et il dépassa les carrières de pierre et se sauva à Seïra.
\VS{27}Dès qu’il fut arrivé, il sonna du shofar dans la montagne d'Ephraïm. Les enfants d'Israël descendirent avec lui de la montagne et il marchait à leur tête.
\VS{28}Il leur dit : Suivez-moi, car Yahweh a livré entre vos mains les Moabites, vos ennemis. Ainsi ils descendirent après lui, s’emparèrent des passages du Jourdain vis-à-vis de Moab et ne laissèrent passer personne.
\VS{29}Ils battirent dans ce temps-là environ dix mille hommes de Moab, tous robustes, tous vaillants et il n'en échappa aucun.
\VS{30}En ce jour, Moab fut humilié sous la main d'Israël. Et le pays fut en repos pendant quatre-vingts ans.
\TextTitle{Schamgar, juge en Israël}
\VS{31}Après lui, il y eut Schamgar, fils d'Anath. Il battit six cents Philistins avec un aiguillon à bœufs et délivra Israël.
\Chap{4}
\TextTitle{Débora et Barak, juges en Israël}
\VerseOne{}Mais les enfants d'Israël firent encore ce qui est mal aux yeux de Yahweh après qu'Ehud fut mort.
\VS{2}C'est pourquoi Yahweh les vendit entre la main de Jabin, roi de Canaan, qui régnait à Hatsor. Le chef de son armée était Sisera, qui habitait à Haroscheth-Goïm\FTNT{Jg. 3:8-16 ; Jos. 11:11-13 ; 1 S. 12:9.}.
\VS{3}Les enfants d'Israël crièrent à Yahweh car Jabin avait neuf cents chars de fer, et il avait violemment opprimé les enfants d'Israël durant vingt ans\FTNT{Jg. 1:19.}.
\VS{4}Dans ce temps-là, Débora, prophétesse, femme de Lappidoth, était juge en Israël.
\VS{5}Débora se tenait sous un palmier, entre Rama et Béthel, dans la montagne d'Ephraïm ; et les enfants d'Israël montaient vers elle pour être jugés.
\VS{6}Elle envoya appeler Barak, fils d'Abinoam, de Kédesch-Nephthali et elle lui dit : Yahweh, le Dieu d'Israël, n'a-t-il pas donné cet ordre ? En disant : Va, et dirige-toi sur la montagne de Thabor et prends avec toi dix mille hommes des enfants de Nephthali, et des enfants de Zabulon\FTNT{Hé. 11:32.} ;
\VS{7}J’attirerai vers toi, au torrent de Kison, Sisera, chef de l'armée de Jabin, avec ses chars et ses troupes et je le livrerai entre tes mains\FTNT{Ps. 83:9-10.}.
\VS{8}Barak lui dit : Si tu viens avec moi, j'irai ; mais si tu ne viens pas avec moi, je n’irai pas.
\VS{9}Elle répondit : J'irai, j'irai avec toi, mais tu n'auras pas d'honneur sur le chemin où tu marches ; car Yahweh livrera Sisera entre les mains d'une femme. Débora se leva et elle alla avec Barak à Kédesch.
\VS{10}Barak convoqua Zabulon et Nephthali à Kédesch ; dix mille hommes marchèrent à sa suite ; et Débora monta avec lui.
\VS{11}Héber, le Kénien, s’était séparé des fils de Hobab, beau-père de Moïse et il avait dressé ses tentes jusqu'au chêne de Tsaannaïm, près de Kédesch\FTNT{No. 10:29.}.
\TextTitle{Yahweh accorde la victoire à Israël}
\VS{12}On rapporta à Sisera que Barak, fils d'Abinoam, s’était dirigé sur la montagne de Thabor.
\VS{13}Et Sisera rassembla tous ses chars, neuf cents chars de fer, et tout le peuple qui était avec lui, depuis Haroscheth-Goïm, jusqu'au torrent de Kison.
\VS{14}Alors Débora dit à Barak : Lève-toi, car voici le jour où Yahweh livre Sisera entre tes mains. Yahweh ne marche-t-il pas devant toi ? Barak descendit de la montagne de Thabor, ayant dix mille hommes à sa suite.
\VS{15}Yahweh mit en déroute devant Barak, Sisera, tous ses chars et toute l'armée, par le tranchant de l'épée. Sisera descendit du char et s'enfuit à pied\FTNT{Ps. 83:9-10.}.
\VS{16}Barak poursuivit les chars et l'armée jusqu'à Haroscheth-Goïm ; et toute l'armée de Sisera fut passée au fil de l'épée ; il n'en resta pas un seul.
\VS{17}Sisera se sauva à pied dans la tente de Jaël, femme de Héber, le Kénien ; car il y avait paix entre Jabin, roi de Hatsor et la maison de Héber, le Kénien.
\VS{18}Jaël étant sortie au-devant de Sisera, lui dit : Entre, mon seigneur, entre chez moi, ne crains pas. Il entra donc chez elle dans la tente et elle le cacha sous une couverture.
\VS{19}Puis il lui dit : Je te prie, donne-moi un peu d'eau à boire, car j'ai soif. Et elle ouvrit une outre de lait, lui donna à boire et le couvrit\FTNT{Jg. 5:25.}.
\VS{20}Il lui dit encore : Tiens-toi à l'entrée de la tente et si l’on vient t’interroger, en disant : Y a-t-il ici quelqu'un ? Alors tu répondras : Non.
\VS{21}Jaël, femme de Héber, saisit un pieu de la tente, prit en sa main un marteau, s’approcha de lui doucement, et lui enfonça dans la tempe le pieu, qui pénétra en terre, pendant qu'il dormait profondément, car il était accablé de fatigue. Et ainsi il mourut.
\VS{22}Et voici, Barak poursuivait Sisera, Jaël sortit au-devant de lui et lui dit : Viens, et je te montrerai l'homme que tu cherches. Barak entra chez elle, et voici, Sisera était étendu mort, et le pieu était dans sa tempe.
\VS{23}En ce jour-là, Dieu humilia Jabin, roi de Canaan, devant les enfants d'Israël.
\VS{24}Et la main des enfants d'Israël s’appesantit et se renforça de plus en plus sur Jabin, roi de Canaan, jusqu'à ce qu'ils aient exterminé Jabin, roi de Canaan.
\Chap{5}
\TextTitle{Cantique à la gloire de Yahweh, le Dieu qui délivre}
\VerseOne{}En ce jour-là, Débora chanta ce cantique avec Barak, fils d'Abinoam, en disant :
\VS{2}Bénissez Yahweh de ce qu’il a fait de telles vengeances en Israël et de ce que le peuple s’est offert volontairement.
\VS{3}Vous, rois, écoutez ! Vous, princes, prêtez l'oreille ! Moi, je chanterai à Yahweh, je chanterai un hymne à Yahweh, le Dieu d'Israël.
\VS{4}Ô Yahweh ! Quand tu sortis de Séir, quand tu t’avanças des champs d'Edom, la terre trembla, les cieux se fondirent, les nuées fondirent en eaux ;
\VS{5}Les montagnes s'ébranlèrent devant Yahweh, ce Sinaï devant Yahweh, le Dieu d'Israël\FTNT{Ps. 68:8-9}.
\VS{6}Aux jours de Schamgar, fils d’Anath, aux jours de Jaël, les grandes routes étaient délaissées, et ceux qui voyageaient prenaient des chemins détournés.
\VS{7}Les villes non murées n’étaient plus habitées en Israël, elles n’étaient point habitées, jusqu’à ce que je me suis levée, moi Débora, jusqu’à ce que je me suis levée pour être mère en Israël.
\VS{8}Israël choisissait-il des dieux nouveaux aussitôt la guerre était aux portes. On ne voyait ni bouclier ni lance chez quarante milliers en Israël.
\VS{9}J’ai mon cœur vers les chefs d'Israël, qui se sont portés volontairement d’entre le peuple. Bénissez Yahweh !
\VS{10}Vous qui montez sur les ânesses blanches, vous qui avez pour sièges des tapis et vous qui marchez sur le chemin, méditez !
\VS{11}Le bruit des archers ayant cessé dans les abreuvoirs, qu’on s’y entretienne des justices de Yahweh et des justices de ses villes non murées en Israël ; alors le peuple de Dieu descendra aux portes.
\VS{12}Réveille-toi, réveille-toi, Débora !  Réveille-toi, réveille-toi, dit le cantique, lève-toi Barak et emmène en captivité ceux que tu as faits captifs, toi fils d'Abinoam\FTNT{Jg. 4:6.}.
\VS{13}Yahweh a fait dominer un reste du peuple sur les puissants ; Yahweh m'a fait dominer sur les héros.
\VS{14}Leur racine est depuis Ephraïm jusqu’à Amalek. A ta suite marcha Benjamin parmi ta troupe. De Makir descendirent les chefs, et de Zabulon ceux qui manient la plume du scribe.
\VS{15}Et les chefs d’Issacar ont été avec Débora, et Issacar ainsi que Barak ; il a été envoyé avec sa suite dans la vallée ; il y a eu aux ruisseaux de Ruben, de grandes considérations dans leur cœur.
\VS{16}Pourquoi es-tu resté entre les barres des étables, à écouter le bêlement des troupeaux ? Aux ruisseaux de Ruben, grandes furent les résolutions du cœur !
\VS{17}Galaad est resté au-delà du Jourdain ; et pourquoi Dan est-il resté sur ses navires ? Aser s'est tenu sur le rivage de la mer, et s’est reposé dans ses ports.
\VS{18}Mais pour Zabulon, c'est un peuple qui a exposé son âme à la mort ; et Nephthali de même, sur les hauteurs des champs.
\VS{19}Les rois vinrent, ils combattirent. Alors combattirent les rois de Canaan, à Thaanac, près des eaux de Meguiddo ; mais ils ne remportèrent nul butin, nul argent.
\VS{20}On a combattu des cieux, les étoiles, dis-je, ont combattu du lieu de leur cours contre Sisera\FTNT{Jg. 4:7.}.
\VS{21}Le torrent de Kison les a emportés, le torrent des anciens temps, le torrent de Kison. Mon âme tu as foulé aux pieds les héros.
\VS{22}Alors les talons des chevaux battirent le sol à cause de la course rapide, de la course rapide de ses puissants chevaux.
\VS{23}Maudissez Méroz, dit l'Ange de Yahweh ; maudissez, maudissez ses habitants, car ils ne sont pas venus au secours de Yahweh, au secours de Yahweh, avec les héros.
\VS{24}Bénie soit par-dessus toutes les femmes Jaël, femme de Héber, le Kénien ! Qu'elle soit bénie entre les femmes qui habitent sous les tentes !
\VS{25}Il demanda de l'eau, elle lui a donné du lait ; elle lui a présenté de la crème dans la coupe des chefs.
\VS{26}Elle a saisi de sa main gauche le pieu et de sa main droite le marteau des ouvriers ; elle a frappé Sisera et lui a fendu la tête ; elle a fracassé et transpercé ses tempes.
\VS{27}Il s'est affaissé aux pieds de Jaël, il est tombé, il s’est couché aux pieds de Jaël ; il s'est affaissé, il est tombé ; là où il s'est affaissé, il est tombé là tout défiguré.
\VS{28}La mère de Sisera regardait par la fenêtre et s'écriait en regardant par les treillis : Pourquoi son char tarde-t-il à venir ? Pourquoi ses chars vont-ils si lentement ?
\VS{29}Les plus sages de ses dames lui répondent, et elle se répond à elle-même :
\VS{30}N’ont-ils pas trouvé ? ils partagent le butin ; une fille, deux filles à chacun par tête. Le butin des vêtements de couleurs est à Sisera, le butin de couleurs de broderie ; couleur de broderie à deux endroits, autour du cou de ceux du butin.
\VS{31}Périssent ainsi, tous tes ennemis ô Yahweh ! Et que ceux qui t'aiment soient comme le soleil quand il sort dans sa force. Et le pays fut en repos pendant quarante ans.
\Chap{6}
\TextTitle{Israël assujetti par Madian}
\VerseOne{}Or, les enfants d'Israël firent ce qui est mal aux yeux de Yahweh ; et Yahweh les livra entre les mains de Madian pendant sept ans.
\VS{2}La main de Madian fut puissante contre Israël. Pour échapper aux Madianites, les enfants d'Israël se retiraient dans les ravins des montagnes, dans des cavernes et sur les rochers fortifiés.
\VS{3}Car il arrivait que quand Israël avait semé, Madian montait avec Amalek et les fils de l’orient, et ils montaient contre lui.
\VS{4}Ils faisaient un camp contre lui, ravageaient les fruits du pays jusqu'à Gaza et ne laissaient en Israël ni vivres, ni brebis, ni bœufs, ni ânes.
\VS{5}Car ils montaient avec leurs troupeaux et leurs tentes, ils arrivaient comme une multitude de sauterelles, ils étaient innombrables, eux et leurs chameaux et ils venaient dans le pays pour le ravager.
\VS{6}Israël fut très appauvri par Madian, et les enfants d'Israël crièrent à Yahweh.
\VS{7}Lorsque les enfants d'Israël crièrent à Yahweh au sujet de Madian,
\VS{8}Yahweh envoya un prophète aux enfants d'Israël, qui leur dit : Ainsi parle Yahweh, le Dieu d'Israël : Je vous ai fait monter hors d’Égypte et je vous ai retirés de la maison de servitude.
\VS{9}Je vous ai délivrés de la main des Egyptiens et de la main de tous ceux qui vous opprimaient ; je les ai chassés devant vous et je vous ai donné leur pays.
\VS{10}Je vous ai dit : Je suis Yahweh, votre Dieu ; vous ne craindrez pas les dieux des Amoréens, dans le pays desquels vous habitez. Mais vous n'avez pas obéi à ma voix.
\TextTitle{Gédéon rencontre l’Ange de Yahweh}
\VS{11}Puis l'Ange de Yahweh vint et s'assit sous le térébinthe d’Ophra, qui appartenait à Joas, de la famille d'Abiézer. Gédéon, son fils, battait du froment au pressoir pour le mettre à l'abri de Madian.
\VS{12}Alors l'Ange de Yahweh lui apparut et lui dit : Très fort et vaillant héros, Yahweh est avec toi !
\VS{13}Gédéon lui répondit : Hélas mon Seigneur ! Est-il possible que Yahweh soit avec nous ? Pourquoi donc toutes ces choses nous sont-elles arrivées ? Et où sont tous ces prodiges que nos pères nous ont racontés, en disant : Yahweh ne nous a-t-il pas fait monter hors d'Egypte ? Car maintenant Yahweh nous a abandonnés et nous a livrés entre les mains des Madianites.
\VS{14}Yahweh le regarda et lui dit : Va avec cette force que tu as et tu délivreras Israël de la main des Madianites ; ne t'ai-je pas envoyé\FTNT{Hé.11:32}?
\VS{15}Et il lui répondit : Hélas, mon Seigneur ! Avec quoi délivrerai-je Israël ? Voici, mon millier de bétail est le plus pauvre en Manassé et je suis le plus petit de la maison de mon père\FTNT{1 S. 9:21 ; 1 S. 16:11.}.
\VS{16}Yahweh lui dit : Parce que je serai avec toi, tu frapperas les Madianites comme s'ils n'étaient qu'un seul homme.
\VS{17}Et il lui répondit : Je te prie, si j'ai trouvé grâce à tes yeux, donne-moi un signe pour montrer que c'est toi qui me parles.
\VS{18}Je te prie, ne t’éloigne pas d’ici jusqu'à ce que je revienne auprès de toi, que j'apporte mon offrande et que je la dépose devant toi. Yahweh dit : Je resterai jusqu'à ce que tu reviennes.
\VS{19}Alors Gédéon rentra et apprêta un chevreau de lait, et fit avec un épha de farine des pains sans levain. Il mit la chair dans un panier, le jus dans un pot et il les lui apporta sous le térébinthe, et les présenta.
\VS{20}L'Ange de Dieu lui dit : Prends la chair et les pains sans levain et pose-les sur ce rocher\FTNT{Voir commentaire en  Es. 8:13-17} et répands le jus. Et il fit ainsi.
\VS{21}Alors l'Ange de Yahweh avança l’extrémité du bâton qu'il avait à la main, et toucha la chair et les pains sans levain. Le feu monta du rocher, et consuma la chair et les pains sans levain. Puis l'Ange de Yahweh disparut à ses yeux.
\VS{22}Gédéon, voyant que c'était l'Ange de Yahweh, dit : Ah, malheur à moi, Seigneur Yahweh ! Car j'ai vu l’Ange de Yahweh face à face.
\VS{23}Et Yahweh lui dit : Sois en paix, ne crains pas, tu ne mourras pas.
\VS{24}Gédéon bâtit là un autel à Yahweh, et lui donna pour nom Yahweh-Shalom. Cet autel, qui appartenait à la famille d'Abiézer, existe encore aujourd'hui à Ophra.
\TextTitle{Gédéon détruit les idole ; Yahweh lui confirme sa mission}
\VS{25}Or il arriva dans cette nuit-là que Yahweh lui dit : Prends un jeune taureau d'entre les bœufs qui sont à ton père et un deuxième taureau de sept ans ; et démolis l'autel de Baal qui est à ton père, et abats l’idole d'Astarté qui est dessus.
\VS{26}Tu bâtiras ensuite et tu disposeras, sur le haut de ce rocher, un autel à Yahweh, ton Dieu. Tu prendras ce deuxième taureau, et tu l'offriras en holocauste avec le bois de l’emblème d’Astarté que tu auras démoli.
\VS{27}Gédéon ayant pris dix hommes parmi ses serviteurs, fit comme Yahweh lui avait dit ; et parce qu'il craignait la maison de son père et les gens de la ville, il l’exécuta de nuit et non de jour.
\VS{28}Lorsque les gens de la ville se levèrent de bon matin, voici, l'autel de Baal avait été démoli, et l'idole d'Astarté qui est dessus était abattue, et le deuxième taureau était offert en holocauste sur l'autel qui avait été bâti.
\VS{29}Ils se dirent les uns aux autres : Qui a fait cela ? Et ils s’informèrent et firent des recherches. On leur dit : C’est Gédéon, fils de Joas, qui a fait cela.
\VS{30}Puis les gens de la ville dirent à Joas : Fais sortir ton fils et qu'il meure ; car il a démoli l'autel de Baal et abattu l'idole d'Astarté qui est dessus.
\VS{31}Joas répondit à tous ceux qui s'adressèrent à lui : Est-ce à vous de prendre parti pour Baal, est-ce à vous de venir à son secours ? Quiconque prendra parti pour Baal sera mis à mort avant le matin. Si Baal est un dieu, qu'il défende lui-même sa cause puisqu'on a démoli son autel.
\VS{32}Et en ce jour on donna à Gédéon le nom de Jerubbaal, en disant : Que Baal défende sa cause, puisque Gédéon a démoli son autel.
\VS{33}Tout Madian, Amalek, et les fils de l’orient se rassemblèrent ;  ils passèrent le Jourdain et campèrent dans la vallée de Jizréel.
\VS{34}Gédéon fut revêtu de l'Esprit de Yahweh ; il sonna du shofar et Abiézer fut convoqué pour marcher à sa suite\FTNT{Jg. 11:29 ; Jg. 13:25.}.
\VS{35}Il envoya des messagers dans tout Manassé qui fut aussi convoqué pour marcher à sa suite. Puis il envoya des messagers dans Aser, dans Zabulon et dans Nephthali, qui montèrent à leur rencontre.
\VS{36}Gédéon dit à Dieu : Si tu veux délivrer Israël par ma main, comme tu l'as dit,
\VS{37}voici, je vais mettre une toison de laine dans l'aire de battage ; si la toison seule se couvre de rosée et que tout le terrain reste sec, je connaîtrai que tu délivreras Israël par ma main, comme tu l’as dit.
\VS{38}Et il arriva ainsi. Le jour suivant, il se leva de bon matin, pressa la toison et en fit sortir la rosée qui donna de l’eau plein une coupe.
\VS{39}Gédéon dit encore à Dieu : Que ta colère ne s'enflamme pas contre moi, et je ne parlerai plus que cette fois : Je te prie, je voudrais seulement faire encore une épreuve avec la toison : Que la toison seule reste sèche et que tout le terrain se couvre de rosée.
\VS{40}Et Dieu fit ainsi cette nuit-là. La toison seule resta sèche, et tout le terrain se couvrit de rosée.
\Chap{7}
\TextTitle{Yahweh sélectionne un petit nombre pour le combat}
\VerseOne{}Jerubbaal qui est Gédéon, et tout le peuple qui était avec lui, se levèrent de bon matin et campèrent près de la source de Harod. Le camp de Madian était au nord, vers la colline de Moré, dans la vallée.
\VS{2}Yahweh dit à Gédéon : Le peuple qui est avec toi est trop nombreux pour que je livre Madian entre ses mains, de peur qu'Israël ne se glorifie contre moi, en disant : C’est ma main qui m'a délivré.
\VS{3}Maintenant donc fais plublier ceci aux oreilles du peuple, et qu'on dise : Que celui qui est craintif et qui a peur s’en retourne et s’éloigne de la montagne de Galaad. Vingt-deux mille hommes parmi le peuple s'en retournèrent et il en resta dix mille\FTNT{De. 20:8.}.
\VS{4}Yahweh dit à Gédéon : Le peuple est encore trop nombreux. Fais-les descendre vers l'eau et là je les épurerai\FTNT{C’est Dieu qui qualifie ses ouvriers, il les éprouve et les épure pour les rendre inébranlables. Voir le test de l’épreuve des Hébreux dans le désert de Sinaï (De. 8).} ; et celui dont je te dirai : Que celui-ci aille avec toi, ira avec toi ; et celui dont je te dirai : Que celui-ci n’aille pas avec toi, n’ira pas avec toi.
\VS{5}Il fit donc descendre le peuple vers l'eau ; et Yahweh dit à Gédéon : Tous ceux qui laperont l'eau avec la langue comme lape le chien, tu les sépareras de tous ceux qui se mettront à genoux pour boire\FTNT{Ps. 110:7.}.
\VS{6}Ceux qui lapèrent l’eau en la portant à la bouche avec leur main furent au nombre de trois cents hommes et tout le reste du peuple se mit à genoux pour boire.
\VS{7}Alors Yahweh dit à Gédéon : C’est par les trois cents hommes qui ont lapé, que je vous délivrerai et que je livrerai Madian entre tes mains. Que tout le reste du peuple s'en aille donc chacun chez soi.
\VS{8}Ainsi le peuple prit entre ses mains des provisions et ses shofars. Gédéon renvoya tous les hommes d'Israël chacun dans sa tente et il retint les trois cents hommes. Or le camp de Madian était au-dessous de lui, dans la vallée.
\TextTitle{Victoire de Gédéon sur Madian}
\VS{9}Et il arriva cette nuit-là que Yahweh lui dit : Lève-toi, descends au camp car je l'ai livré entre tes mains.
\VS{10}Si tu crains de descendre, descends-y avec Pura, ton serviteur.
\VS{11}Tu écouteras ce qu'ils diront et après cela, tes mains seront fortifiées ; descends donc au camp. Il descendit avec Pura, son serviteur, jusqu'aux avant-postes du camp.
\VS{12}Or Madian, Amalek et tous les fils de l'orient étaient répandus dans la vallée comme des sauterelles, tant il y en avait, et leurs chameaux étaient sans nombre, comme le sable qui est sur le bord de la mer, tant il y en avait\FTNT{Jg. 6:3-33.}.
\VS{13}Gédéon arriva ; et voici, un homme racontait à son compagnon un songe. Il lui disait : Voici, j'ai eu un songe ; il me semblait qu'un gâteau de pain d'orge roulait dans le camp de Madian ; et il est venu heurter jusqu’à la tente et elle est tombée ; il l’a retournée sens dessus dessous et elle a été renversée.
\VS{14}Alors son compagnon répondit et dit : Ce n'est pas autre chose que l'épée de Gédéon, fils de Joas, homme d'Israël ; Dieu a livré Madian et tout le camp entre ses mains.
\VS{15}Lorsque Gédéon eut entendu le récit du songe et son interprétation, il se prosterna, revint au camp d'Israël et dit : Levez-vous car Yahweh a livré le camp de Madian entre vos mains.
\VS{16}Puis il divisa les trois cents hommes en trois corps et il leur donna à chacun des shofars à la main et des cruches vides, avec des flambeaux dans les cruches.
\VS{17}Il leur dit : Regardez-moi et faites comme je ferai. Dès que je serai arrivé à l’extrémité du camp, vous ferez comme je ferai.
\VS{18}Quand je sonnerai du shofar, moi et tous ceux qui sont avec moi, alors vous sonnerez aussi du shofar tout autour du camp et vous direz : Pour Yahweh et pour Gédéon !
\VS{19}Gédéon et les cent hommes qui étaient avec lui arrivèrent à l’extrémité du camp, au commencement de la veille de la nuit, comme on venait de placer les gardes. Ils sonnèrent du shofar et  brisèrent les cruches qu'ils avaient à la main.
\VS{20}Ainsi les trois corps sonnèrent du shofar, et brisèrent les cruches ; ils saisirent de la main gauche les flambeaux et de la main droite les shofars pour sonner et ils s’écrièrent : L'épée de Yahweh et de Gédéon !
\VS{21}Ils restèrent chacun à sa place autour du camp, et tout le camp se mit à courir ça et là, à pousser des cris et à prendre la fuite.
\VS{22}Car comme les trois cents hommes sonnèrent encore du shofar, Yahweh leur fit tourner l'épée les uns contre les autres. Le camp s'enfuit jusqu'à Beth-Schitta, vers Tseréra, jusqu'au bord d'Abel-Mehola, près de Tabbath\FTNT{1 S. 14:20 ; Ez. 38:21.}.
\VS{23}Les hommes d'Israël, à savoir ceux de Nephthali, d'Aser et de tout Manassé, se rassemblèrent, et ils poursuivirent Madian.
\VS{24}Alors Gédéon envoya des messagers dans toute la montagne d'Ephraïm, pour leur dire : Descendez pour aller à la rencontre de Madian, et coupez-leur les premiers le passage des eaux jusqu'à Beth-Bara et celui du Jourdain. Tous les hommes d'Ephraïm se rassemblèrent, et ils s’emparèrent du passage des eaux jusqu’à Beth-Bara et de celui du Jourdain.
\VS{25}Ils saisirent deux des chefs de Madian, Oreb et Zeeb ; ils tuèrent Oreb au rocher d’Oreb, et ils tuèrent Zeeb au pressoir de Zeeb. Ils poursuivirent Madian, et ils apportèrent les têtes de Oreb et de Zeeb à Gédéon, de l’autre côté du Jourdain\FTNT{Ps. 83:11 ; Es. 10:26.}.
\Chap{8}
\TextTitle{Poursuite de Zébach et Tsalmunna ; exécution des rois de Madian}
\VerseOne{}Alors les hommes d'Ephraïm dirent à Gédéon : Que signifie cette manière d’agir envers nous ? Pourquoi ne pas nous avoir appelés quand tu es allé à la guerre contre Madian ? Et ils s'emportèrent fortement contre lui\FTNT{Jg. 12:1.}.
\VS{2}Et il leur répondit : Qu'ai-je fait maintenant au prix de ce que vous avez fait ? Les grappillages d'Ephraïm ne sont-ils pas meilleurs que la vendange d'Abiézer ?
\VS{3}Dieu a livré entre vos mains les chefs de Madian, Oreb et Zeeb. Qu'ai-je pu faire au prix de ce que vous avez fait ? Et leur esprit fut apaisé envers lui lorsqu’il eut ainsi parlé.
\VS{4}Gédéon arriva au Jourdain, et il le passa, lui et les trois cents hommes qui étaient avec lui, fatigués, mais poursuivant toujours l'ennemi.
\VS{5}C'est pourquoi il dit aux gens de Succoth : Donnez, je vous prie, quelques pains aux hommes qui m’accompagnent, car ils sont fatigués, et ainsi je poursuivrai Zébach et Tsalmunna, rois de Madian.
\VS{6}Mais les chefs de Succoth répondirent : La main de Zébach et celle de Tsalmunna sont-elles déjà en ton pouvoir, pour que nous donnions du pain à ton armée ?
\VS{7}Et Gédéon dit : Eh bien ! Quand Yahweh aura livré Zébach et Tsalmunna entre mes mains, je foulerai au pied votre chair avec des épines du désert et avec des chardons.
\VS{8}Puis de là il monta à Penuel, et il fit la même demande aux gens de Penuel. Les gens de Penuel lui répondirent comme avaient répondu ceux de Succoth.
\VS{9}Et il dit aussi aux gens de Penuel : Quand je reviendrai en paix, je démolirai cette tour.
\VS{10}Zébach et Tsalmunna étaient à Karkor et leurs armées avec eux, environ quinze mille hommes, tous ceux qui étaient restés de l'armée entière des fils de l’orient ; cent vingt mille hommes tirant l’épée avaient été tués.
\VS{11}Gédéon monta par le chemin de ceux qui habitent sous les tentes, à l’orient de Nobach et de Jogbeha, et il battit l'armée, qui se croyait en sûreté.
\VS{12}Et comme Zébach et Tsalmunna s'enfuyaient, il les poursuivit, et prit les deux rois de Madian, Zébach et Tsalmunna, et mit en déroute toute l'armée\FTNT{Ps. 83:11.}.
\TextTitle{Vengeance sur Succoth et Penuel ; exécution de Zébach et Tsalmunna}
\VS{13}Puis Gédéon, fils de Joas, revint de la bataille par la montée de Hérès.
\VS{14}Il saisit un garçon d’entre les hommes de Succoth, il l'interrogea, et ce garçon lui donna par écrit le nom des chefs et des anciens de Succoth, au nombre de soixante-dix-sept hommes.
\VS{15}Et il vint auprès de gens de Succoth, et leur dit : Voici Zébach et Tsalmunna, au sujet desquels vous m'avez insulté, en disant : La main de Zébach et celle de Tsalmunna sont-elles déjà en ton pouvoir, pour que nous donnions du pain à tes hommes fatigués ?
\VS{16}Il prit donc les anciens de la ville et châtia les hommes de Succoth avec des épines du désert et des chardons.
\VS{17}Il démolit la tour de Penuel, et tua les gens de la ville.
\VS{18}Puis il dit à Zébach et à Tsalmunna : Comment étaient les hommes que vous avez tués à Thabor ? Ils répondirent : Ils étaient entièrement comme toi, chacun d'eux avait l'air d'un fils de roi.
\VS{19}Il leur dit : C'étaient mes frères, fils de ma mère. Yahweh est vivant, si vous les aviez laissés vivre, je ne vous tuerais pas.
\VS{20}Puis il dit à Jéther, son premier-né : Lève-toi, tue-les ! Mais le jeune garçon ne tira pas son épée, car il avait peur, car il était encore un enfant.
\VS{21}Et Zébach et Tsalmunna dirent : Lève-toi toi-même, et jette-toi sur nous ! Car tel est l'homme, telle est sa force. Et Gédéon se leva, et tua Zébach et Tsalmunna. Il prit ensuite les croissants qui étaient aux cous de leurs chameaux.
\TextTitle{Gédéon recommande au peuple le règne de Yahweh}
\VS{22}Les hommes d'Israël dirent tous d'un commun accord à Gédéon : Domine sur nous, tant toi que ton fils, et le fils de ton fils, car tu nous as délivrés de la main de Madian.
\VS{23}Gédéon leur répondit : Je ne dominerai pas sur vous, et mon fils ne dominera pas sur vous ; c'est Yahweh qui dominera sur vous\FTNT{De. 17:15.}.
\TextTitle{Gédéon introduit une occasion de chute en Israël}
\VS{24}Mais Gédéon leur dit : J’ai une demande à vous faire : Donnez-moi chacun les anneaux que vous avez eus pour butin. Les ennemis avaient des anneaux d'or, car ils étaient Ismaélites.
\VS{25}Ils répondirent : Nous les donnerons volontiers. Et ils étendirent un manteau sur lequel chacun jeta les anneaux de son butin.
\VS{26}Le poids des anneaux d'or que Gédéon demanda fut de mille sept cents sicles d'or, sans les croissants, les pendants d'oreilles, et les vêtements de pourpre que portaient les rois de Madian, et sans les colliers qui étaient aux cous de leurs chameaux.
\VS{27}Puis Gédéon en fit un  éphod\FTNT{Sous Moïse, il y avait deux sortes d'éphods, le premier était de simple lin pour les sacrificateurs, et le deuxième de broderie pour le souverain sacrificateur. Comme celui des simples sacrificateurs n'avait rien de particulier, Moïse ne s'est pas arrêté à le décrire. Mais il décrit longuement celui du souverain sacrificateur. (Ex. 28:6-9). Il était composé d'or, d'hyacinthe, de pourpre, de cramoisi, de coton retors ; c'était un tissu de différentes couleurs. Il y avait à l'endroit de l'éphod qui venait sur les deux épaules du souverain sacrificateur, deux grosses pierres précieuses, qui étaient chargées du nom des douze tribus d’Israël, six noms sur chaque pierre. A l'endroit où l'éphod se croisait sur la poitrine du grand prêtre, il y avait un ornement carré, nommé le rational, en hébreu «~choschen~», dans lequel étaient enchâssées douze pierres précieuses, où l'on avait gravé les noms des douze tribus d'Israël ; un sur chacune des pierres.}, et le mit dans sa ville, à Ophra, où il devint un objet de prostitution pour tout Israël ; il fut un piège pour Gédéon et pour sa maison.
\TextTitle{Fin de la vie de Gédéon ; rechute d’Israël après sa mort}
\VS{28}Ainsi Madian fut humilié devant les enfants d'Israël, et il ne leva plus la tête. Le pays fut en repos pendant quarante ans, durant les jours de Gédéon.
\VS{29}Jerubbaal, fils de Joas s’en retourna dans sa ville, et demeura dans sa maison.
\VS{30}Gédéon eut soixante-dix fils, issus de ses reins, car il eut plusieurs femmes.
\VS{31}Sa concubine, qui était à Sichem, lui enfanta aussi un fils, et il lui donna le nom d’Abimélec.
\VS{32}Puis Gédéon, fils de Joas, mourut après une heureuse vieillesse ; et il fut enseveli dans le sépulcre de Joas, son père, à Ophra, qui appartenait à la famille d’Abiézer.
\TextTitle{Rechute dans l'idolâtrie}
\VS{33}Et il arriva après que Gédéon fut mort, que les enfants d'Israël se détournèrent et se prostituèrent aux Baals, et ils établirent Baal-Berith pour leur dieu\FTNT{Jg. 2:11-17 ; 10:6.}.
\VS{34}Ainsi les enfants d'Israël ne se souvinrent pas de Yahweh, leur Dieu, qui les avait délivrés de la main de tous leurs ennemis qui les entouraient.
\VS{35}Et ils n'usèrent d'aucune loyauté envers la maison de Jerubbaal, de Gédéon, après tout le bien qu'il avait fait à Israël.
\Chap{9}
\TextTitle{Conspiration d’Abimélec pour régner sur Israël}
\VerseOne{}Et Abimélec, fils de Jerubbaal, s'en alla à Sichem vers les frères de sa mère, et leur parla, ainsi qu'à toute la maison du père de sa mère :
\VS{2}Je vous prie, faites entendre ces paroles à tous les seigneurs de Sichem : Lequel vous semble le meilleur, que soixante-dix hommes, tous fils de Jerubbaal, dominent sur vous, ou qu'un seul homme domine sur vous ? Et souvenez-vous que je suis votre os et votre chair\FTNT{Ge. 29:14.}.
\VS{3}Les frères de sa mère dirent de sa part toutes ces paroles aux oreilles de tous les seigneurs de Sichem, et leur cœur se tourna après Abimélec, car ils disaient : C'est notre frère.
\VS{4}Ils lui donnèrent soixante-dix sicles d'argent de la maison de Baal-Berith. Abimélec s'en servit pour acheter des hommes misérables et turbulents, qui allèrent après lui.
\VS{5}Et il vint dans la maison de son père à Ophra, et tua sur une seule pierre ses frères, fils de Jerubbaal, qui étaient soixante-dix hommes. Il ne resta que Jotham, le plus jeune fils de Jerubbaal, parce qu'il s'était caché.
\VS{6}Et tous les seigneurs de Sichem s'assemblèrent avec toute la maison de Millo ; ils vinrent, et firent d'Abimélec leur roi près du chêne à Sichem.
\VS{7}On le rapporta à Jotham, qui alla se tenir au sommet de la montagne de Garizim, et les appelant, il dit en élevant la voix : Écoutez-moi, seigneurs de Sichem, et que Dieu vous entende !
\VS{8}Les arbres allèrent pour oindre un roi, et ils dirent à l'olivier : Règne sur nous.
\VS{9}Mais l'olivier leur répondit : Renoncerai-je à mon huile, par laquelle Dieu et les hommes sont honorés, pour aller m'agiter sur les arbres\FTNT{Ps. 104:15.} ?
\VS{10}Puis les arbres dirent au figuier : Viens, toi, règne sur nous.
\VS{11}Mais le figuier leur répondit : Renoncerai-je à ma douceur, et à mon bon fruit, pour aller m'agiter sur les arbres ?
\VS{12}Puis les arbres dirent à la vigne : Viens, toi, et règne sur nous.
\VS{13}Mais la vigne répondit : Renoncerai-je à mon vin, qui réjouit Dieu et les hommes, pour aller m'agiter sur les arbres ?
\VS{14}Alors tous les arbres dirent à l'épine : Viens, toi, et règne sur nous.
\VS{15}Et l'épine répondit aux arbres : Si c'est en vérité que vous m'oignez pour roi, venez, et réfugiez-vous sous mon ombrage ; sinon, que le feu sorte de l'épine, et qu'il dévore les cèdres du Liban.
\VS{16}Maintenant donc, est-ce en vérité et avec intégrité que vous avez agi en établissant Abimélec pour roi ? Avez-vous bien fait envers Jerubbaal et sa maison ? L'avez-vous fait selon les bienfaits qu'il a rendus de sa main ?
\VS{17}Car mon père a combattu pour vous, il a exposé sa vie devant vous, et vous a délivrés de la main de Madian ;
\VS{18}Mais vous vous êtes levés aujourd'hui contre la maison de mon père, et avez tué sur une pierre ses fils, soixante-dix hommes, et avez établi pour roi Abimélec, fils de sa servante, sur les habitants de Sichem, parce qu'il est votre frère.
\VS{19}Si, dis-je, vous avez agi aujourd'hui en vérité et avec intégrité envers Jerubbaal, et sa maison, réjouissez-vous d'Abimélec, et qu'il se réjouisse aussi de vous !
\VS{20}Sinon, que le feu sorte d'Abimélec et qu'il dévore les seigneurs de Sichem, et la maison de Millo ; et que le feu sorte des seigneurs de Sichem, et de la maison de Millo, et qu'il dévore Abimélec !
\VS{21}Puis Jotham s'enfuit rapidement ; il s'en alla à Beer, où il demeura loin d'Abimélec, son frère.
\TextTitle{Sichem se retourne contre Abimélec}
\VS{22}Abimélec gouverna sur Israël durant trois ans.
\VS{23}Alors Dieu envoya un mauvais esprit entre Abimélec et les seigneurs de Sichem, et les seigneurs de Sichem furent infidèles à Abimélec.
\VS{24}Afin que la violence faite aux soixante-dix fils de Jerubbaal vienne et que leur sang se tourne contre Abimélec, leur frère, qui les avait tués, et sur les seigneurs de Sichem, qui l'avaient aidé par leur main à tuer ses frères.
\VS{25}Les seigneurs de Sichem mirent des embûches sur le sommet des montagnes, des gens pillaient tous ceux qui passaient près d'eux sur le chemin. Cela fut rapporté à Abimélec.
\VS{26}Alors Gaal, fils d'Ebed, vint avec ses frères, et ils passèrent à Sichem. Les seigneurs de Sichem eurent confiance en lui.
\VS{27}Puis étant sortis aux champs, ils vendangèrent leurs vignes, foulèrent les raisins, et se donnèrent à des réjouissances ; ils entrèrent dans la maison de leur dieu, ils mangèrent et burent, et ils maudirent Abimélec.
\VS{28}Alors Gaal, fils d'Ebed, dit : Qui est Abimélec, et qui est Sichem pour que nous servions Abimélec ? N'est-il pas le fils de Jerubbaal et Zebul n'est-il pas son commissaire ? Servez plutôt les hommes de Hamor, père de Sichem ; mais pour quelle raison servirions-nous Abimélec ?
\VS{29}Plaise à Dieu ! Qu'on mette ce peuple sous mon pouvoir, et je chasserais Abimélec. Et il disait d'Abimélec : Multiplie ton armée, et sors !
\VS{30}Zebul, gouverneur de la ville, entendit les paroles de Gaal, fils d'Ebed, et sa colère s'enflamma.
\VS{31}Puis il envoya astucieusement des messagers vers Abimélec, pour lui dire : Voici, Gaal fils d'Ebed, et ses frères, sont entrés dans Sichem, et voici, ils assiègent la ville contre toi.
\VS{32}Maintenant donc, lève-toi de nuit, toi et le peuple qui est avec toi, et mets-toi en embuscade dans les champs.
\VS{33}Et le matin, au lever du soleil, tu te lèveras et tu te jetteras sur la ville. Gaal et le peuple qui est avec lui sortiront contre toi, ta main lui fera selon les forces que tu trouveras.
\VS{34}Abimélec et tout le peuple qui était avec lui se levèrent de nuit, et ils se mirent en embuscade contre Sichem, divisés en quatre bandes.
\VS{35}Alors Gaal, fils d'Ebed, sortit, et il se tint à l'entrée de la porte de la ville. Abimélec et tout le peuple qui était avec lui se levèrent de l'embuscade.
\VS{36}Gaal voyant le peuple, dit à Zebul : Voici un peuple qui descend du sommet des montagnes. Zebul lui dit : Tu vois l'ombre des montagnes comme des hommes.
\VS{37}Gaal, parla encore, et dit : C'est bien un peuple qui descend des hauteurs du pays, et une bande vient du chemin du chêne des devins.
\VS{38}Et Zebul lui dit : Où est donc ta bouche, toi qui disais : Qui est Abimélec, pour que nous le servions ? N'est-ce pas ici ce peuple que tu méprisais ? Sors maintenant, je te prie, et combats !
\VS{39}Alors, Gaal sortit conduisant les seigneurs de Sichem, et combattit contre Abimélec.
\VS{40}Abimélec le poursuivit, et il s'enfuit de devant lui, et plusieurs tombèrent morts jusqu'à l'entrée de la porte.
\VS{41}Abimélec s'arrêta à Aruma. Zebul repoussa Gaal et ses frères, afin qu'ils ne restent plus à Sichem.
\VS{42}Et il arriva, dès le lendemain, que le peuple sortit aux champs. Cela fut rapporté à Abimélec,
\VS{43}qui prit son peuple, et le divisa en trois bandes, et les mit en embuscade dans les champs. Ayant vu que le peuple sortait de la ville, il se leva contre eux, et les battit.
\VS{44}Abimélec et la bande qui était avec lui se répandirent, et se tinrent à l'entrée de la porte de la ville ; mais les deux autres bandes se jetèrent sur tous ceux qui étaient aux champs, et les battirent.
\VS{45}Ainsi Abimélec combattit contre la ville toute la journée ; il prit la ville, et tua le peuple qui y était. Il la rasa, et y sema du sel.
\VS{46}Ayant appris cela, tous les seigneurs de la tour de Sichem entrèrent dans la forteresse de la maison du dieu Baal-Berith\FTNT{Jg. 9:4 ; 8:33.}.
\VS{47}On rapporta à Abimélec que tous les seigneurs de la tour de Sichem s'étaient assemblés dans la forteresse.
\VS{48}Alors Abimélec monta sur la montagne de Tsalmon, lui et tout le peuple qui était avec lui. Il prit en main une hache, coupa une branche d'arbre, et l'ayant mise sur son épaule, la porta, et dit au peuple qui était avec lui : Avez-vous vu ce que j'ai fait ? Hâtez-vous de faire comme moi.
\VS{49}Chacun donc de tout le peuple coupa une branche, et ils marchèrent derrière Abimélec ; ils mirent ces branches tout autour de la forteresse, et y mirent le feu. Il brûlèrent la forteresse, et toutes les personnes de la tour de Sichem moururent ; au nombre d'environ mille, tant hommes que femmes.
\TextTitle{Abimélec meurt}
\VS{50}Puis Abimélec marcha contre Thébets, y mit son camp, et la prit.
\VS{51}Il y avait au milieu de la ville une forte tour, où s'enfuirent tous les hommes et toutes les femmes, et tous les seigneurs de la ville, et ayant fermé les portes après eux, ils montèrent sur le toit de la Tour.
\VS{52}Alors Abimélec alla jusqu'à la tour, l'attaqua, et s'approcha jusqu'à la porte pour la brûler par le feu.
\VS{53}Mais une femme jeta une pièce de meule de moulin sur la tête d'Abimélec, et lui brisa le crâne\FTNT{2 S. 11:21.}.
\VS{54}Rapidement, il appela le garçon qui portait ses armes, et lui dit : Tire ton épée, et tue-moi, de peur qu'on ne dise de moi : C'est une femme qui l'a tué. Le garçon le transperça, et il mourut\FTNT{1 S. 31:4.}.
\VS{55}Quand les hommes d'Israël virent qu'Abimélec était mort, ils s'en allèrent chacun en son lieu.
\VS{56}Ainsi Dieu rendit à Abimélec le mal qu'il avait fait contre son père, en tuant ses soixante-dix frères,
\VS{57}Et toute la méchanceté des hommes de Sichem ; Dieu, dis-je, la fit retourner sur leurs têtes ; et ainsi la malédiction de Jotham, fils de Jérubbaal, vint sur eux.
\Chap{10}
\TextTitle{Thola, juge en Israël}
\VerseOne{}Après Abimélec, Thola fils de Pua, fils de Dodo, homme d'Issacar, se leva pour délivrer Israël ; il habitait à Schamir, dans la montagne d'Ephraïm.
\VS{2}Il fut juge en Israël pendant vingt-trois ans ; puis il mourut, et fut enterré à Schamir.
\TextTitle{Jaïr, juge en Israël}
\VS{3}Après lui se leva Jaïr, le Galaadite, qui fut juge en Israël pendant vingt-deux ans.
\VS{4}Il avait trente fils, qui montaient sur trente ânons, et qui avaient trente villes, qu'on appelle jusqu'à ce jour bourgs de Jaïr, lesquelles sont situées au pays de Galaad\FTNT{Jg. 5:10.}.
\VS{5}Et Jaïr mourut, et fut enterré à Kamon.
\TextTitle{Idolâtrie d’Israël et oppression par ses ennemis}
\VS{6}Puis les enfants d'Israël firent encore ce qui est mal aux yeux de Yahweh ; et servirent les Baals et les Astartés, les dieux de Syrie, les dieux de Sidon, les dieux de Moab, les dieux des fils d'Ammon, et les dieux des Philistins, et ils abandonnèrent Yahweh, et ne le servirent plus\FTNT{Jg. 2:11 ; 3:7 ; 8:33.}.
\VS{7}Alors la colère de Yahweh s'enflamma contre Israël, et il les vendit entre les mains des Philistins, et des fils d'Ammon.
\VS{8}Ils opprimèrent et écrasèrent les enfants d'Israël cette année-là, et pendant dix-huit ans tous les enfants d'Israël qui étaient au-delà du Jourdain, au pays des Amoréens en Galaad.
\VS{9}Même les fils d'Ammon passèrent le Jourdain pour combattre contre Juda, contre Benjamin, et contre la maison d'Ephraïm. Israël fut dans une grande détresse.
\VS{10}Alors les enfants d'Israël crièrent à Yahweh, en disant : Nous avons péché contre toi, et certes, nous avons abandonné notre Dieu et nous avons servi les Baals.
\VS{11}Mais Yahweh répondit aux enfants d'Israël : N'avez-vous pas été opprimés par les Egyptiens, les Amoréens, les fils d'Ammon et les Philistins ?
\VS{12}Et lorsque les Sidoniens, Amalek et Maon, vous opprimèrent, et que vous criâtes à moi, ne vous ai-je pas délivrés de leurs mains ?
\VS{13}Mais vous, vous m'avez abandonné, et vous avez servi d'autres dieux. C'est pourquoi je ne vous délivrerai plus.
\VS{14}Allez et criez vers les dieux que vous avez choisis ; qu'ils vous délivrent au temps de votre détresse !
\VS{15}Mais les enfants d'Israël répondirent à Yahweh : Nous avons péché ; traite-nous comme tu le trouveras bon. Nous te prions seulement que tu nous délivres aujourd'hui !
\VS{16}Alors ils ôtèrent du milieu d'eux les dieux des étrangers, et servirent Yahweh, qui fut affligé des souffrances d'Israël.
\VS{17}Les fils d'Ammon se rassemblèrent et campèrent en Galaad, et les enfants d'Israël se rassemblèrent et campèrent à Mitspa.
\VS{18}Le peuple, les chefs de Galaad se dirent l'un à l'autre : Qui sera l'homme qui commencera à combattre contre les fils d'Ammon ? Il sera chef de tous les habitants de Galaad.
\Chap{11}
\TextTitle{Jephté, juge en Israël}
\VerseOne{}Or Jephthé, le Galaadite, était un fort et vaillant homme. Il était le fils d'une femme prostituée ; et c'est Galaad qui l'avait engendré.
\VS{2}La femme de Galaad lui enfanta des fils ; et quand les fils de cette femme furent grands, ils chassèrent Jephthé, en lui disant : Tu n'auras pas d'héritage dans la maison de notre père, car tu es fils d'une autre femme.
\VS{3}Jephthé s'enfuit donc de devant ses frères, et habita au pays de Tob. Des misérables se rassemblèrent auprès de Jephthé, et ils sortirent dehors avec lui\FTNT{Jg. 9:4 ; 1 S. 22:2 ; 1 S 10:6-8.}.
\VS{4}Et il arriva, quelque temps après, les fils d'Ammon firent la guerre à Israël.
\VS{5}Et comme les fils d'Ammon faisaient la guerre à Israël, les anciens de Galaad s'en allèrent pour emmener Jephthé du pays de Tob.
\VS{6}Ils dirent à Jephthé : Viens, et sois notre chef, afin que nous combattions contre les fils d'Ammon.
\VS{7}Jephthé répondit aux anciens de Galaad : N'est-ce pas vous qui m'avez haï et chassé de la maison de mon père ? Pourquoi êtes-vous venus à moi maintenant que vous êtes dans la détresse ?
\VS{8}Alors les anciens de Galaad dirent à Jephthé : La raison pour laquelle nous retournons à toi maintenant, c'est afin que tu viennes avec nous, que tu combattes contre les fils d'Ammon, et que tu sois notre chef, celui de tous les habitants de Galaad.
\VS{9}Jephthé répondit aux anciens de Galaad : Si vous me ramenez pour combattre contre les fils d'Ammon, et que Yahweh les livre devant moi, je serai votre chef.
\VS{10}Les anciens de Galaad dirent à Jephthé : Que Yahweh nous entende, et qu'il juge, si nous ne faisons pas ce que tu dis.
\VS{11}Jephthé donc s'en alla avec les anciens de Galaad. Le peuple le mit à sa tête et l'établit pour chef, et Jephthé déclara devant Yahweh, à Mitspa, toutes les paroles qu'il avait dites.
\VS{12}Puis Jephthé envoya des messagers au roi des fils d'Ammon, pour lui dire : Qu'y a-t-il entre toi et moi, que tu viennes contre moi pour faire la guerre à mon pays ?
\VS{13}Le roi des fils d'Ammon répondit aux messagers de Jephthé : C'est parce qu'Israël a pris mon pays quand il est monté d'Egypte, depuis l'Arnon jusqu'à Jabbok, et même jusqu'au Jourdain. Maintenant rends-le de bon gré.
\VS{14}Mais Jephthé envoya encore des messagers au roi des fils d'Ammon,
\VS{15}qui lui dirent : Ainsi parle Jephthé : Israël n'a rien pris du pays de Moab, ni du pays des fils d'Ammon.
\VS{16}Mais lorsqu’Israël est monté d'Egypte, il est venu par le désert jusqu'à la Mer Rouge et il a atteint Kadès.
\VS{17}Alors Israël envoya des messagers au roi d'Edom, pour lui dire : Que je passe, je te prie, par ton pays. Le roi d'Edom ne voulut pas l'entendre. Il en envoya aussi au roi de Moab, qui ne voulut pas non plus l'entendre. Et Israël demeura à Kadès.
\VS{18}Puis il marcha par le désert, tourna le pays d'Edom et le pays de Moab, et vint à l'orient du pays de Moab ; il campa au-delà de l'Arnon, et n'entra pas sur les frontières de Moab, car l'Arnon est la frontière de Moab.
\VS{19}Mais Israël envoya des messagers à Sihon, roi des Amoréens, roi de Hesbon, auquel Israël dit : Laisse-nous passer par ton pays jusqu'au lieu où nous allons.
\VS{20}Mais Sihon n'eut pas assez confiance en Israël pour le laisser passer sur son territoire ; il rassembla tout son peuple, ils campèrent vers Jahats, et combattirent contre Israël.
\VS{21}Et Yahweh, le Dieu d'Israël, livra Sihon et tout son peuple entre les mains d'Israël, qui les battit. Israël prit possession de tout le pays des Amoréens qui habitaient cette terre.
\VS{22}Ils conquirent donc tout le pays des Amoréens, depuis l'Arnon jusqu'à Jabbok, et depuis le désert jusqu'au Jourdain.
\VS{23}Et maintenant que Yahweh, le Dieu d'Israël, a dépossédé les Amoréens de devant son peuple d'Israël, aurais-tu la possession de leur pays ?
\VS{24}Ce que ton dieu Kemosch te donne à posséder, ne le posséderais-tu pas ? Et tout ce que Yahweh, notre Dieu, a mis en notre possession devant nous, nous ne le posséderions pas !
\VS{25}Or maintenant vaux-tu mieux en quelque sorte que ce soit que Balak, fils de Tsippor, roi de Moab ? A-t-il contesté et combattu contre Israël ?
\VS{26}Voilà trois cents ans qu'Israël demeure à Hesbon, et dans les villes de son ressort, à Aroër, et dans les villes de son ressort, et dans toutes les villes qui sont le long de l'Arnon : Pourquoi ne les avez-vous pas saisies pendant ce temps-là ?
\VS{27}Je ne t'ai pas offensé, mais tu fais mal de me faire la guerre. Que Yahweh, qui est le juge, juge aujourd'hui entre les enfants d'Israël et les fils d'Ammon !
\VS{28}Le roi des fils d'Ammon n'écouta pas les paroles que Jephthé lui fit dire.
\VS{29}L'Esprit de Yahweh fut sur Jephthé. Il passa au travers de Galaad et de Manassé ; il passa jusqu'à Mitspé de Galaad, et de Mitspé de Galaad, il passa jusqu'aux fils d'Ammon.
\TextTitle{Jephté fait un vœu ; Ammon livré entre ses mains}
\VS{30}Jephthé fit un vœu à Yahweh, et dit : Si tu livres les fils d'Ammon entre mes mains,
\VS{31}alors tout ce qui sortira des portes de ma maison au-devant de moi, quand je retournerai en paix chez les fils d'Ammon, sera consacré à Yahweh, et je l'offrirai en holocauste.
\VS{32}Jephthé passa jusqu'où étaient les fils d'Ammon, et Yahweh les livra entre ses mains.
\VS{33}Il les battit par une grande défaite, depuis Aroër jusqu'à Minnith, espace qui renfermait vingt villes, et jusqu'à Abel-Keramim. Et les fils d'Ammon furent humiliés devant les fils d'Israël.
\VS{34}Puis comme Jephthé retourna à Mitspa dans sa maison, voici, sa fille, qui était seule et unique, sans qu'il eût d'autres fils ou filles, sortit au-devant de lui avec des tambourins et des danses.
\VS{35}Et il arriva qu'aussitôt qu'il l'eut aperçue, il déchira ses vêtements, et dit : Ha ! Ma fille ! Tu m'as entièrement abaissé, tu es du nombre de ceux qui me troublent ! J'ai ouvert ma bouche à Yahweh, et je ne puis le révoquer.
\VS{36}Elle répondit : Mon père, si tu as ouvert ta bouche à Yahweh, fais-moi selon ce qui est sorti de ta bouche, puisque Yahweh t'a fait vengeance de tes ennemis, des fils d'Ammon.
\VS{37}Toutefois, elle dit à son père : Que ceci me soit fait : Laisse-moi pendant deux mois ! Je m'en irai, je descendrai par les montagnes, et je pleurerai ma virginité, avec mes compagnes.
\VS{38}Il répondit : Va ! Et il la laissa aller pour deux mois. Elle s'en alla donc avec ses compagnes, et pleura sa virginité dans les montagnes.
\VS{39}Et au bout de deux mois, elle retourna vers son père ; et il lui fit selon le vœu qu'il avait fait\FTNT{Yahweh interdit les sacrifices humains (Lé. 20:2-5 ; Lé. 21 ; De. 12:31 ; De. 18:10).}. Elle n'avait pas connu d'homme. Dès lors, ce fut une coutume en Israël,
\VS{40}tous les ans les filles d'Israël allaient pour célébrer la fille de Jephthé, le Galaadite, quatre jours par an.
\Chap{12}
\TextTitle{Querelle entre Jephté et Ephraïm}
\VerseOne{}Or les hommes d'Ephraïm se rassemblèrent, passèrent par le nord, et dirent à Jephthé : Pourquoi es-tu passé pour combattre contre les enfants d'Ammon, sans nous avoir appelés pour aller avec toi ? Nous brûlerons ta maison, et toi aussi\FTNT{Jg. 8:1.}.
\VS{2}Et Jephthé leur dit : J’ai eu un grand différend avec les enfants de Ammon, moi et mon peuple, et quand je vous ai appelés, vous ne m’avez point délivré de leurs mains.
\VS{3}Voyant que vous ne me délivriez pas, j'ai exposé ma vie, et je suis passé jusqu'où étaient les fils d'Ammon. Yahweh les a livrés entre mes mains. Pourquoi donc aujourd'hui montez-vous vers moi pour me faire la guerre ?
\VS{4}Puis Jephthé assembla tous les hommes de Galaad, et combattit contre Ephraïm. Les hommes de Galaad battirent Ephraïm, parce qu'ils disaient : Vous êtes des fugitifs d'Ephraïm ! Galaad est au milieu d'Ephraïm, au milieu de Manassé !
\VS{5}Les Galaadites se saisirent des gués du Jourdain du côté d'Ephraïm. Et quand l'un des fuyards d'Ephraïm disait : Que je passe ! Les hommes de Galaad lui disaient : Es-tu Ephraïmite ? Il répondait : Non.
\VS{6}Alors ils lui disaient : dis un peu « Schibboleth ». Et il disait « Sibboleth », car il ne pouvait pas le prononcer. Sur quoi, se saisissant de lui, ils le tuaient aux gués du Jourdain. En ce temps-là quarante-deux mille hommes d'Ephraïm périrent.
\VS{7}Jephthé fut juge en Israël pendant six ans ; puis Jephthé le Galaadite mourut, et fut enterré dans l'une des villes de Galaad.
\TextTitle{Ibstan, juge en Israël}
\VS{8}Après lui, Ibtsan de Bethléhem fut juge en Israël.
\VS{9}Il eut trente fils, il envoya trente filles au-dehors, et il fit venir du dehors trente filles pour ses fils. Il fut juge en Israël pendant sept ans.
\VS{10}Puis Ibtsan mourut, et fut enterré à Bethléhem.
\TextTitle{Le juge Elon}
\VS{11}Après lui, Elon de Zabulon fut juge en Israël pendant dix ans.
\VS{12}Puis Elon de Zabulon mourut, et fut enterré à Ajalon, dans le pays de Zabulon.
\TextTitle{Le juge Abdon}
\VS{13}Après lui, Abdon, fils d'Hillel, le Pirathonite, fut juge en Israël.
\VS{14}Il eut quarante fils et trente petits-fils, qui montaient sur soixante-dix ânons. Il fut juge en  Israël pendant huit ans\FTNT{Jg. 10:4. }.
\VS{15}Puis Abdon, fils d'Hillel, le Pirathonite, mourut, et fut enterré à Pirathon, dans le pays d'Ephraïm, sur la montagne des Amalécites.
\Chap{13}
\TextTitle{Israël à nouveau asservi pas les Philistins}
\VerseOne{}Et les enfants d’Israël recommencèrent à faire ce qui est mauvais aux yeux de Yahweh ; et Yahweh les livra entre les mains des Philistins, pendant quarante ans.
\TextTitle{Naissance du juge Samson}
\VS{2}Or il y avait un homme de Tsorea, de la famille des Danites, dont le nom était Manoach. Sa femme était stérile, et n'enfantait pas.
\VS{3}L’Ange de Yahweh apparut à la femme, et lui dit : Voici, tu es stérile, et tu n'as jamais eu d'enfants ; mais tu concevras, et tu enfanteras un fils.
\VS{4}Prends donc bien garde dès maintenant de ne boire ni vin ni liqueur forte, et de ne manger aucune chose impure.
\VS{5}Car voici tu vas être enceinte et tu enfanteras un fils. Le rasoir ne s'élèvera pas sur sa tête, parce que l'enfant sera Naziréen\FTNT{Naziréen vient du mot  «~nazir~» qui signifie «~consacré~» ou «~séparé~». Voir No. 6.} pour Dieu dès le ventre de sa mère ; et ce sera lui qui commencera à délivrer Israël de la main des Philistins.
\VS{6}Et La femme vint, et parla à son mari en disant : Un homme de Dieu est venu vers moi et il avait l'aspect d'un ange de Dieu, un aspect fort redoutable. Je ne lui ai pas demandé d'où il était, et il ne m'a pas déclaré son nom.
\VS{7}Mais il m'a dit : Tu vas être enceinte, et tu enfanteras un fils ; maintenant donc ne bois ni vin ni liqueur forte, et ne mange aucune chose impure, car cet enfant sera Naziréen pour Dieu dès le ventre de sa mère jusqu'au jour de sa mort.
\TextTitle{Prière de Manoach}
\VS{8}Et Manoach pria instamment Yahweh, et dit : Ah ! Seigneur, que l'homme de Dieu que tu as envoyé vienne encore vers nous, et qu'il nous enseigne ce que nous devons faire à l'enfant quand il naîtra !
\VS{9}Et Dieu exauça la prière de Manoach, et l'Ange de Dieu vint encore vers la femme lorsqu'elle était assise dans un champ ; mais Manoach, son mari, n'était pas avec elle.
\VS{10}Et la femme courut vite le rapporter à son mari, en lui disant : Voici, l'homme qui était venu vers moi l'autre jour m'est apparu.
\VS{11}Manoach se leva, suivit sa femme, et venant vers l'homme, il lui dit : Es-tu cet homme qui a parlé à cette femme ? Il répondit : C'est moi.
\VS{12}Manoach dit : Tout ce que tu as dit arrivera, quelle conduite faudra-t-il tenir envers l'enfant, et que lui faudra-t-il faire ?
\VS{13}L'Ange de Yahweh répondit à Manoach : La femme se gardera de tout ce que je lui ai dit.
\VS{14}Elle ne mangera rien qui sorte de la vigne, elle ne boira ni vin ni liqueur forte, et ne mangera aucune chose impure ; elle prendra garde à tout ce que je lui ai ordonné.
\VS{15}Alors Manoach dit à l'Ange de Yahweh : Permets que nous te retenions, et que nous apprêtions un chevreau en ta présence.
\VS{16}Et l'Ange de Yahweh répondit à Manoach : Quand tu me retiendrais, je ne mangerai pas de ton mets ; mais si tu fais un holocauste, tu l'offriras à Yahweh. Manoach ne savait pas que ce fût un Ange de Yahweh.
\VS{17}Et Manoach dit à l'Ange de Yahweh : Quel est ton nom, afin que nous te rendions les honneurs lorsque ta parole viendra ?
\VS{18}Et l'Ange de Yahweh lui répondit : Pourquoi demandes-tu mon nom ? Il est merveilleux.
\VS{19}Alors Manoach prit un chevreau, et une offrande, et les offrit à Yahweh sur le rocher. Il se produisit une chose merveilleuse à la vue de Manoach et de sa femme.
\VS{20}Comme la flamme montait de dessus l'autel vers les cieux, l'Ange de Yahweh monta aussi avec la flamme de l'autel. A cette vue, Manoach et sa femme tombèrent la face contre terre.
\VS{21}L'Ange de Yahweh n'apparut plus à Manoach ni à sa femme. Alors Manoach sut que c'était l'Ange de Yahweh.
\VS{22}Et Manoach dit à sa femme : Certainement nous mourrons, car nous avons vu Dieu.
\VS{23}Mais sa femme lui répondit : Si Yahweh avait voulu nous faire mourir, il n'aurait pas pris de nos mains l'holocauste ni l'offrande, il ne nous aurait pas fait voir toutes ces choses ni fait entendre les choses que nous avons entendues.
\VS{24}Puis cette femme enfanta un fils, et elle l'appela du nom de Samson. L'enfant devint grand, et Yahweh le bénit.
\VS{25}Et l'Esprit de Yahweh commença à l'agiter à Machané-Dan, entre Tsorea et Eschthaol.
\Chap{14}
\TextTitle{Yahweh, le maître des évènements}
\VerseOne{}Samson descendit à Thimna, et il y vit une femme d'entre les filles des Philistins.
\VS{2}Etant remonté dans sa maison, il le déclara à son père et à sa mère, en disant : J'ai vu une femme à Thimna d'entre les filles des Philistins ; prenez-la maintenant, afin qu'elle soit ma femme.
\VS{3}Son père et sa mère lui dirent : N'y a-t-il pas de femme parmi les filles de tes frères et parmi tout notre peuple, pour que tu ailles prendre une femme d'entre les Philistins, ces incirconcis ? Et Samson dit à son père : Prenez-la pour moi, car elle est droite à mes yeux.
\VS{4}Mais son père et sa mère ne savaient pas que cela venait de Yahweh : Car Samson cherchait une occasion de dispute de la part des Philistins. Or en ce temps-là, les Philistins dominaient sur Israël.
\TextTitle{L'énigme de Samson}
\VS{5}Samson descendit avec son père et sa mère à Thimna. Ils allèrent jusqu'aux vignes de Thimna, et voici, un jeune lion rugissant vint à sa rencontre.
\VS{6}Et l'Esprit de Yahweh saisit Samson ; sans avoir rien en sa main, il déchira le lion comme on déchire un chevreau. Il ne déclara pas à son père ni à sa mère ce qu'il avait fait\FTNT{1 S. 17:34-35.}.
\VS{7}Il descendit et parla à la femme, et elle fut trouvée droite à ses yeux.
\VS{8}Puis quelque temps après, il retourna à Thimna pour la prendre, et se détourna pour voir la carcasse du lion. Et voici, il y avait dans la carcasse du lion un essaim d'abeilles et du miel.
\VS{9}Il en prit entre ses mains, et s'en alla en mangeant ; et lorsqu'il fut arrivé vers son père et sa mère, il leur en donna, et ils en mangèrent. Mais il ne leur déclara pas qu'il avait pris ce miel dans la carcasse du lion.
\VS{10}Son père descendit chez la femme. Samson fit là un festin ; car c'est ainsi que les jeunes gens faisaient.
\VS{11}Dès qu'on le vit, on prit trente compagnons qui furent avec lui.
\VS{12}Samson leur dit : Je vous propose une énigme. Si vous me l'expliquez au cours des sept jours du festin, et si vous la trouvez, je vous donnerai trente chemises et trente vêtements de rechange.
\VS{13}Mais si vous ne pouvez pas me l'expliquer, vous me donnerez trente chemises et trente vêtements de rechange. Ils lui répondirent : Propose ton énigme, et nous l'écouterons.
\VS{14}Et il leur dit : De celui qui mange est sorti ce qui se mange, et du fort est sorti le doux. Pendant trois jours, ils ne purent pas expliquer l'énigme.
\VS{15}Et au septième jour, ils dirent à la femme de Samson : Persuade ton mari de nous expliquer l'énigme ; de peur que nous ne te brûlions au feu, toi et la maison de ton père. C'est pour nous déposséder que vous nous avez appelés ici, n'est-ce pas ?
\VS{16}La femme de Samson pleurait auprès de lui, et disait : Certainement tu me hais, et tu ne m'aimes pas ; tu as proposé une énigme aux enfants de mon peuple, et tu ne me l'as pas expliquée ! Et il lui répondait : Je ne l'ai expliquée ni à mon père ni à ma mère ; est-ce à toi que je l'expliquerais ?
\VS{17}Elle pleura ainsi auprès de lui durant les sept jours du festin ; mais au septième jour, il la lui expliqua, parce qu'elle le tourmentait. Puis elle l'expliqua aux enfants de son peuple.
\VS{18}Les gens de la ville lui dirent au septième jour, avant le coucher du soleil : Qu'y a-t-il de plus doux que le miel, et qu'y a-t-il de plus fort que le lion ? Et il leur dit : Si vous n'aviez pas labouré avec ma génisse vous n'auriez pas trouvé mon énigme.
\VS{19}L'Esprit de Yahweh le saisit, et il descendit à Askalon. Il tua trente hommes, il prit leurs dépouilles, et donna les vêtements de rechange à ceux qui avaient expliqué l'énigme. Sa colère s'enflamma, et il monta à la maison de son père.
\VS{20}Et la femme de Samson fut donnée à son compagnon, avec lequel il était lié.
\Chap{15}
\TextTitle{Samson utilisé pour le jugement des Philistins}
\VerseOne{}Et il arriva quelque jours après, au jour de la moisson des blés, que Samson alla visiter sa femme, et lui porta un chevreau. Il dit : J'entrerai vers ma femme dans sa chambre. Mais le père de sa femme ne lui permit pas d'y entrer.
\VS{2}Car il lui dit : J'ai cru que tu avais de la haine pour elle, c'est pourquoi je l'ai donnée à ton compagnon. Sa jeune sœur n'est-elle pas plus belle qu'elle ? Prends-la donc à sa place.
\VS{3}Samson leur dit : Cette fois je serai innocent à l'égard des Philistins si je leur fais du mal.
\VS{4}Samson s'en alla donc. Il prit trois cents renards, il prit aussi des torches ; puis il tourna les renards queue contre queue, et mit une torche entre les deux queues, au milieu.
\VS{5}Puis il mit le feu aux torches, et lâcha les renards dans les blés des Philistins, et brûla le tas de gerbes, le blé sur pied, jusqu'aux plantations d'oliviers.
\VS{6}Les Philistins dirent : Qui a fait cela ? On répondit : Samson, le gendre du Thimnien, parce qu'il lui a pris sa femme et l'a donnée à son compagnon. Les Philistins montèrent, et ils la brûlèrent au feu, elle avec son père.
\VS{7}Alors Samson leur dit : Est-ce donc ainsi que vous faites ? Je ne cesserai qu'après m'être vengé de vous.
\VS{8}Il les battit par une grande défaite, dos et ventre ; puis il descendit, et demeura dans une caverne du rocher d'Etam.
\VS{9}Alors les Philistins montèrent, campèrent en Juda, et s'étendirent jusqu'à Léchi.
\VS{10}Les hommes de Juda dirent : Pourquoi êtes-vous montés contre nous ? Ils répondirent : Nous sommes montés pour lier Samson, afin que nous lui fassions comme il nous a fait.
\VS{11}Alors trois mille hommes de Juda descendirent à la caverne du rocher d'Etam, et dirent à Samson : Ne sais-tu pas que les Philistins dominent sur nous ? Que nous as-tu donc fait ? Il leur répondit : Je leur ai fait comme ils m'ont fait.
\VS{12}Ils lui dirent : Nous sommes descendus pour te lier, afin de te livrer entre les mains des Philistins. Samson leur dit : Jurez-moi que vous ne me tuerez pas.
\VS{13}Ils lui répondirent, en disant : Non ; mais nous te lierons, afin de te livrer entre leurs mains, mais nous ne te tuerons pas. Ils le lièrent avec deux cordes neuves, et le firent monter hors du rocher.
\VS{14}Lorsqu'il entra à Léchi, les Philistins poussèrent des cris de joie à sa rencontre. Alors l'Esprit de Yahweh le saisit. Les cordes qui étaient sur ses bras devinrent comme du lin brûlé par le feu, et les liens tombèrent de ses mains.
\VS{15}Il trouva une mâchoire d'âne fraîche, il étendit sa main, la prit, et il en tua mille hommes.
\VS{16}Puis Samson dit : Avec une mâchoire d'âne, un monceau, deux monceaux ; avec une mâchoire d'âne, j'ai tué mille hommes.
\VS{17}Quand il cessa de parler, il jeta de sa main la mâchoire. On appela ce lieu Ramath-Léchi.
\VS{18}Il eut extrêmement soif, et invoqua Yahweh en disant : Tu as accordé par la main de ton serviteur cette grande délivrance ; et maintenant mourrais-je de soif, et tomberais-je entre les mains des incirconcis\FTNT{1 S. 17:26.} ?
\VS{19}Alors Dieu fendit la cavité du rocher qui est à Léchi, et il en sortit de l'eau. Samson but, l'Esprit lui revint, et il reprit vie. C'est pourquoi on a appelé cette source du nom d'En-Hakkoré ; elle existe encore aujourd'hui  à Léchi.
\VS{20}Samson fut juge en Israël, au temps des Philistins, pendant vingt ans\FTNT{Jg. 16:31.}.
\Chap{16}
\TextTitle{Faiblesse de Samson}
\VerseOne{}Or Samson s'en alla à Gaza ; il y vit une femme prostituée, et il entra chez elle.
\VS{2}On dit aux gens de Gaza : Samson est venu ici. Ils l'entourèrent, et se tinrent en embuscade toute la nuit à la porte de la ville. Ils restèrent tranquilles toute la nuit, en disant : Au point du jour, nous le tuerons.
\VS{3}Samson demeura couché jusqu'à minuit. Au milieu de la nuit, il se leva ; et il saisit les battants des portes de la ville et les deux poteaux, les retira avec la barre, les mit sur ses épaules, et les porta sur le sommet de la montagne qui est en face d'Hébron.
\VS{4}Après cela, il aima une femme dans la vallée de Sorek. Elle se nommait Delila.
\VS{5}Les princes des Philistins montèrent vers elle, et lui dirent : Séduis-le, jusqu'à ce que tu saches de lui en quoi consiste sa grande force, et comment pourrions-nous le vaincre ; afin que nous le lions pour l'abattre, et nous te donnerons chacun mille cent sicles d'argent.
\VS{6}Delila dit à Samson : Dis-moi, je te prie, en quoi consiste ta grande force, et avec quoi il faudrait te lier pour t'abattre.
\VS{7}Samson lui répondit : Si on me liait avec sept cordes fraîches, qui ne soient pas encore sèches, je deviendrais faible et je serais comme un autre homme.
\VS{8}Les princes des Philistins emmenèrent à Delila sept cordes fraîches, qui n'étaient pas encore sèches. Et elle le lia.
\VS{9}Or il y avait chez elle, dans une chambre, des gens qui se tenaient en embuscade. Elle lui dit : Les Philistins sont sur toi, Samson ! Alors il rompit les cordes comme se romprait un cordon d'étoupe dès qu'il sent le feu. Et l'on ne connut pas d'où lui venait sa force.
\VS{10}Puis Delila dit à Samson : Voici, tu t'es moqué de moi, car tu m'as dit des mensonges. Je te prie, déclare-moi maintenant avec quoi il faut te lier.
\VS{11}Il lui répondit : Si on me liait avec des cordes neuves, dont on ne se serait jamais servi pour un quelconque ouvrage, je deviendrais faible, et je serais comme un autre homme.
\VS{12}Delila prit des cordes neuves avec lesquelles elle le lia. Puis elle lui dit : Les Philistins sont sur toi, Samson ! Or il y avait des gens en embuscade dans une chambre. Et il rompit les cordes comme un fil.
\VS{13}Puis Delila dit à Samson : Tu t'es moqué de moi, jusqu'ici tu m'as dit des mensonges. Déclare-moi avec quoi il faut te lier. Il lui dit : Tu n'as qu'à tresser les sept tresses de ma tête avec la chaîne du tissu.
\VS{14}Et elle les fixa par la cheville. Puis elle dit : Les Philistins sont sur toi, Samson ! Alors il se réveilla de son sommeil, et il retira la chaîne du tissu.
\TextTitle{Samson révèle son secret}
\VS{15}Alors elle lui dit : Comment peux-tu dire : Je t'aime ! Puisque ton cœur n'est pas avec moi ? Tu t'es moqué de moi par trois fois, et tu ne m'as pas déclaré en quoi consiste ta grande force.
\VS{16}Comme elle le tourmentait et l'importunait tous les jours par ses paroles, son âme en fut affligée jusqu'à la mort,
\VS{17}alors il lui ouvrit tout son cœur, et lui dit : Le rasoir n'est jamais passé sur ma tête, car je suis Naziréen de Dieu dès le ventre de ma mère. Si j'étais rasé, ma force partirait, je me trouverais faible, et je serais comme tous les autres hommes.
\VS{18}Delila, voyant qu'il lui avait ouvert tout son cœur, envoya appeler les princes des Philistins, et leur fit dire : Montez cette fois, car il m'a ouvert tout son cœur. Les princes des Philistins montèrent vers elle, et emmenèrent l'argent dans leurs mains.
\VS{19}Elle l'endormit sur ses genoux. Et ayant appelé un homme, elle rasa les sept tresses de la tête de Samson, et commença à le dompter. Sa force partit.
\VS{20}Alors elle dit : Les Philistins sont sur toi, Samson ! Et il se réveilla de son sommeil, et dit : Je m'en sortirai comme les autres fois, et je me dégagerai. Mais il ne savait pas que Yahweh s'était retiré de lui\FTNT{L’immoralité sexuelle de Samson et sa désobéissance à Yahweh, dues à son manque de caractère, ont ruiné à jamais son ministère et compromis l’avenir du peuple d’Israël qu’il devait diriger (Jg. 16). Cet homme avait reçu un appel puissant dès le sein de sa mère, mais il ne vivait pas dans la crainte de Dieu. Le manque de discernement de Samson lui coûta ainsi toutes les grâces que le Seigneur lui avait accordées : La sainteté symbolisée par ses sept tresses, la force ou l’onction,  la vision, la liberté (Jg. 16:21).}.
\VS{21}Les Philistins donc le saisirent, et lui crevèrent les yeux ; ils le descendirent à Gaza, et le lièrent de deux chaînes d'airain. Il tournait la meule dans la prison\FTNT{2 S. 3:34.}.
\VS{22}Les cheveux de sa tête commencèrent à repousser, depuis qu'il avait été rasé.
\TextTitle{Samson achève le jugement des Philistins}
\VS{23}Or les princes des Philistins s'assemblèrent pour offrir un grand sacrifice à Dagon, leur dieu, et pour se réjouir. Ils disaient : Notre dieu a livré en nos mains Samson, notre ennemi.
\VS{24}Et quand le peuple le vit, il loua son dieu, en disant : Notre dieu a livré entre nos mains notre ennemi, celui qui ravageait notre pays, et qui multipliait nos morts.
\VS{25}Comme ils avaient le cœur joyeux, ils dirent : Qu'on appelle Samson, afin qu'il nous fasse rire ! Ils appelèrent Samson et le tirèrent de la prison ; et il joua devant eux. Ils le firent tenir entre les colonnes.
\VS{26}Alors Samson dit au garçon qui le tenait par la main : Laisse-moi afin que je puisse toucher les colonnes sur lesquelles repose la maison pour que je m'y appuie.
\VS{27}Or la maison était remplie d'hommes et de femmes ; tous les princes des Philistins y étaient, et il y avait même sur le toit près de trois mille personnes, hommes et femmes, qui regardaient Samson jouer.
\VS{28}Alors Samson invoqua Yahweh, et dit : Seigneur Yahweh ! Je te prie, souviens-toi de moi ; ô Dieu ! Fortifie-moi seulement cette fois, et que par un coup je me venge des Philistins pour mes deux yeux\FTNT{Hé. 11:32.} !
\VS{29}Samson embrassa les deux colonnes du milieu sur lesquelles reposait la maison, et il s'appuya contre elles ; l'une à sa droite, et l'autre à sa gauche.
\VS{30}Et il dit : Que mon âme meure avec les Philistins ! Il se pencha donc de toute sa force, et la maison tomba sur les princes et sur tout le peuple qui y était. Et il fit mourir beaucoup plus de gens à sa mort, qu'il n'en avait fait mourir pendant sa vie.
\VS{31}Ensuite ses frères et toute la maison de son père descendirent, et le transportèrent. Lorsqu'ils furent montés, ils l'enterrèrent entre Tsorea et Eschthaol dans le sépulcre de Manoach, son père. Il avait été juge en Israël pendant vingt ans\FTNT{Jg. 13:2.}.
\Chap{17}
\TextTitle{Confusion en Israël}
\VerseOne{}Il y avait un homme de la montagne d'Ephraïm, nommé Mica.
\VS{2}Il dit à sa mère : Les mille cent sicles d'argent qu'on t'a pris, et pour lesquels tu as fait des imprécations même à mes oreilles, voici, j'ai cet argent, c'est moi qui l'avais pris. Alors sa mère dit : Béni soit mon fils par Yahweh !
\VS{3}Et il rendit à sa mère les mille cent sicles d'argent ; sa mère dit : Je consacre de ma main cet argent à Yahweh, afin d'en faire pour mon fils une image taillée, et une image en métal fondu ; et c'est ainsi que je te le rendrai.
\VS{4}Et il rendit l'argent à sa mère. Elle prit deux cents sicles d'argent et les donna au fondeur, qui en fit une image taillée, et une image en métal fondu.  On les plaça dans la maison de Mica.
\VS{5}Ainsi cet homme, savoir Mica, avait une maison de Dieu ; il fit un éphod et des téraphim, et il consacra par sa main l'un de ses fils, qui lui servit de sacrificateur.
\VS{6}En ce temps-là, il n'y avait pas de roi en Israël. Chacun faisait ce qui lui semblait être droit à ses yeux\FTNT{Jg. 18:1}.
\VS{7}Or il y avait un jeune homme de Bethléhem de Juda, de la famille de la tribu de Juda ; il était Lévite, et il séjournait là.
\VS{8}Cet homme partit de la ville de Bethléhem de Juda, pour trouver une demeure qui lui convienne.  En chemin, il entra dans la montagne d'Ephraïm jusqu'à la maison de Mica.
\VS{9}Mica lui dit : D'où viens-tu ? Il lui répondit : Je suis Lévite, de Bethléhem de Juda, et je voyage pour trouver une demeure qui me convienne.
\VS{10}Mica lui dit : Demeure avec moi ; tu me serviras de père et de sacrificateur, et je te donnerai dix sicles d'argent par an, les vêtements d'ordre dont tu auras besoin, et ton entretien. Et le Lévite vint\FTNT{Jg. 18:19.}.
\VS{11}Ainsi le Lévite convint de demeurer avec cet homme, qui regarda le jeune homme comme l'un de ses fils.
\VS{12}Mica consacra\FTNT{"Consacrer" signifie litteralement "remplir la main".} le Lévite, qui lui servit de sacrificateur, et qui demeura dans sa maison.
\VS{13}Mica dit : Maintenant je sais que Yahweh me fera du bien, parce que j'ai un Lévite pour sacrificateur.
\Chap{18}
\TextTitle{Dan recherche un territoire}
\VerseOne{}En ce temps-là, il n'y avait pas de roi en Israël ; et en ce même temps la tribu des Danites cherchait un héritage afin de pouvoir s'établir, car jusqu'à ce jour il ne lui était pas échu d'héritage au milieu des tribus d'Israël\FTNT{Jg. 17:6.}.
\VS{2}C'est pourquoi les fils de Dan envoyèrent de leur famille cinq hommes vaillants, de Tsorea et d'Eschthaol, pour explorer le pays et l'examiner. Ils leur dirent : Allez examiner le pays. Ils entrèrent dans la montagne d'Ephraïm jusqu'à la maison de Mica, et ils y passèrent la nuit.
\VS{3}Comme ils étaient près de la maison de Mica, ils reconnurent la voix du jeune homme Lévite et lui dirent : Qui t'a amené ici ? Qu'y fais-tu ? Que fais-tu ici ?
\VS{4}Il leur répondit : Mica fait pour moi telle et telle chose, il me donne un salaire, et je lui sers de sacrificateur.
\VS{5}Ils lui dirent : Nous te prions consulte Dieu, afin que nous sachions si le voyage que nous entreprenons prospérera.
\VS{6}Et le sacrificateur leur répondit : Allez en paix ; Yahweh a sous ses yeux le voyage que vous mènerez.
\VS{7}Ces cinq hommes s'en allèrent, et entrèrent à Laïs. Ils virent le peuple qui y habitait en sécurité selon les coutumes des Sidoniens, tranquille et en confiance ; il n'y avait personne au pays qui les humiliait en quelque chose en dominant sur eux ; ils étaient éloignés des Sidoniens, et ils n'avaient aucune affaire avec d'autres hommes.
\VS{8}Puis ils vinrent auprès de leurs frères à Tsorea et à Eschthaol, et leurs frères leur dirent : Quelle nouvelle rapportez-vous ?
\VS{9}Et ils répondirent : Allons ! Montons contre eux ; car nous avons vu le pays, et nous l'avons trouvé très bon. Quoi ! Vous restez sans rien faire ? Ne soyez pas paresseux pour aller posséder ce pays.
\VS{10}Quand vous y entrerez, vous irez vers un peuple en sécurité. Le pays est vaste, Dieu l'a livré entre vos mains ; c'est un lieu où il ne manque rien de tout ce qui est sur la terre.
\VS{11}Il partit de Tsorea et d'Eschthaol, six cents hommes de la famille de Dan, munis de leurs armes de guerre.
\VS{12}Ils montèrent, et campèrent à Kirjath-Jearim en Juda ; c'est pourquoi on a appelé ce lieu qui est derrière Kirjath-Jearim jusqu'à ce jour, Machané-Dan.
\VS{13}Puis ils passèrent par la montagne d'Ephraïm, et ils entrèrent dans la maison de Mica.
\TextTitle{Campagnes de la tribu de Dan}
\VS{14}Alors les cinq hommes qui étaient allés explorer le pays de Laïs prirent la parole et dirent à leurs frères : Savez-vous qu'il y a dans ces maisons-là un éphod, des théraphim, une image taillée et une image en métal fondu ? Voyez maintenant ce que vous avez à faire.
\VS{15}Alors ils se détournèrent de ce lieu, et entrèrent dans la maison où était le jeune homme Lévite, dans la maison de Mica, et lui demandèrent comment il se portait.
\VS{16}Et les six cents hommes d'entre les fils de Dan, qui étaient munis de leurs armes de guerre, se tenaient à l'entrée de la porte.
\VS{17}Mais les cinq hommes, qui étaient allés explorer le pays, montèrent et entrèrent dans la maison ; ils prirent l'image taillée, l'éphod, les théraphim, et l'image en métal fondu, pendant que le sacrificateur était à l'entrée de la porte avec les six cents hommes munis de leurs armes de guerre.
\VS{18}Etant entrés dans la maison de Mica, ils prirent l'image taillée, l'éphod, les théraphim, et l'image en métal fondu. Le sacrificateur leur dit : Que faites-vous ?
\VS{19}Ils lui répondirent : Tais-toi, mets ta main sur ta bouche, et viens avec nous ; sois pour nous un père et un sacrificateur. Vaut-il mieux que tu serves de sacrificateur à la maison d'un homme seul, ou que tu serves de sacrificateur à une tribu et à une famille en Israël\FTNT{Jg. 17:10.} ?
\VS{20}Le sacrificateur eut de la joie dans son cœur ; il prit l'éphod, les théraphim, et l'image taillée, et vint au milieu du peuple.
\VS{21}Après quoi ils se retournèrent et marchèrent, en mettant devant eux les petits enfants, le bétail, et les bagages.
\VS{22}Comme ils étaient loin de la maison de Mica, les gens qui  habitaient les maisons voisines de celle de Mica furent assemblés à grand cri ; et poursuivirent les fils de Dan.
\VS{23}Et ils crièrent aux fils de Dan, qui se tournèrent de face et dirent à Mica : Qu’as-tu, que tu te sois ainsi écrié pour rassembler ces gens ?
\VS{24}Il répondit : Vous avez enlevé mes dieux que j'avais faits, vous avez pris le sacrificateur, et vous vous en êtes allés : Que me reste-t-il ? Comment pouvez-vous me dire : Qu'as-tu\FTNT{Ge. 31:30.} ?
\VS{25}Les fils de Dan lui dirent : Ne fais pas entendre ta voix après nous, de peur que des hommes exaspérés ne se jettent sur vous, et que vous n’y laissiez la vie, toi, et tous ceux de ta famille.
\VS{26}Les fils de Dan firent leur chemin. Mica, voyant qu'ils étaient plus forts que lui, s'en retourna et revint dans sa maison.
\VS{27}Ainsi ils prirent les choses que Mica avait faites, et le sacrificateur qu'il avait, et ils entrèrent à Laïs, vers un peuple tranquille et en sécurité ; ils les firent passer au fil de l'épée, et ils brûlèrent la ville.
\VS{28}Et il n'y eut personne qui la délivrât, car elle était éloignée de Sidon, et ses habitants n'avaient pas d'affaires avec les autres hommes : Elle était située dans la vallée qui appartenait au pays de Beth-Rehob. Les fils de Dan rebâtirent la ville, et y demeurèrent.
\VS{29}Ils appelèrent la ville Dan, selon le nom de Dan, leur père qui était né à Israël ; mais la ville s'appelait auparavant Laïs\FTNT{Jos. 19:47.}.
\VS{30}Et les fils de Dan dressèrent l'image taillée ; et Jonathan, fils de Guerschom, fils de Manassé, lui et ses fils, furent sacrificateurs pour la tribu des Danites, jusqu'au jour de la captivité du pays.
\VS{31}Ils y dressèrent donc l'image taillée que Mica avait faite, pendant tout le temps que la maison de Dieu fut à Silo.
\Chap{19}
\TextTitle{Dégradation morale}
\VerseOne{}Il arriva aussi en ce temps-là, où il n'y avait pas de roi en Israël, qu'un Lévite qui habitait aux côtés de la montagne d'Ephraïm, prit pour concubine une femme de Bethléhem de Juda\FTNT{Jg. 17:6 ; 21:25.}.
\VS{2}Mais sa concubine se prostitua chez lui, et elle s'en alla pour aller dans la maison de son père à Bethléhem de Juda, où elle resta pendant quatre mois.
\VS{3}Puis son mari se leva et alla après elle, pour parler à son cœur, et la ramener. Il avait avec lui son serviteur et deux ânes. Elle le fit entrer dans la maison de son père ; et quand le père de la jeune femme le vit, il s'approcha avec joie.
\VS{4}Son beau-père, le père de la jeune femme, le retint avec grande instance, de sorte qu'il demeura trois jours avec lui. Ils mangèrent et burent, et logèrent là.
\VS{5}Le quatrième jour, ils se levèrent de bon matin, et le Lévite se levait pour s'en aller. Mais le père de la jeune femme dit à son gendre : Fortifie ton cœur avec un morceau de pain, et vous partirez ensuite.
\VS{6}Ils s'assirent, et ils mangèrent et burent eux deux ensemble. Puis le père de la jeune femme dit au mari : Je te prie consens à passer encore ici cette nuit, et que ton cœur se réjouisse.
\VS{7}Le mari se levait pour s'en aller ; mais son beau-père le pressa tellement, qu'il s'en retourna, et y passa encore la nuit.
\VS{8}Le cinquième jour, il se leva de bon matin pour s'en aller. Alors le père de la jeune femme dit : Fortifie ton cœur ; et attendez le déclin du jour. Et ils mangèrent eux deux.
\VS{9}Puis le mari se levait pour s'en aller, avec sa concubine et son serviteur ; mais son beau-père, le père de la jeune femme, lui dit : Voici, maintenant le jour baisse, il se fait tard, je vous prie passez ici la nuit ; voici le jour est sur son déclin, passe ici la nuit, et que ton cœur se réjouisse ; demain matin vous vous mettrez en route, et tu t'en iras à ta tente.
\VS{10}Mais le mari ne voulut pas y passer la nuit, il se leva, et s'en alla.  Il vint jusque vis-à-vis de Jébus, qui est Jérusalem, avec les deux ânes bâtés et sa concubine.
\VS{11}Comme ils étaient près de Jébus, le jour avait beaucoup baissé. Le serviteur dit à son maître : Allons, détournons-nous vers cette ville des Jébusiens, afin que nous y passions la nuit.
\VS{12}Son maître lui répondit : Nous ne nous détournerons pas vers une ville d'étrangers, où il n'y a pas d'enfants d'Israël, mais nous passerons par Guibea.
\VS{13}Il dit aussi à son serviteur : Allons, approchons-nous de l'un de ces lieux, Guibea ou Rama, et passons-y la nuit.
\VS{14}Ils continuèrent à marcher, et le soleil se coucha quand ils furent près de Guibea, qui appartient à Benjamin.
\VS{15}Alors ils se détournèrent vers Guibea, et y entrèrent pour passer la nuit. Le Lévite entra, et il s'assit sur la place de la ville. Il n'y eut aucun homme qui les reçut dans sa maison afin qu'ils y passent la nuit.
\VS{16}Et voici, sur le soir, un vieil homme venait de travailler dans les champs ; cet homme était de la montagne d'Ephraïm, il séjournait à Guibea, et les gens du lieu étaient Benjamites.
\VS{17}Et levant ses yeux, il vit le voyageur sur la place de la ville. Le vieil homme lui dit : Où vas-tu, et d'où viens-tu ?
\VS{18}Il lui répondit : Nous passons de Bethléhem de Juda vers les côtés de la montagne d'Ephraïm, d'où je suis. J'étais allé jusqu'à Bethléhem de Juda, mais maintenant je m'en vais à la maison de Yahweh. Mais il n'y a aucun homme qui me reçoive dans sa maison.
\VS{19}Nous avons pourtant de la paille et du fourrage pour nos ânes ; du pain et du vin pour moi,  pour ta servante, et pour le garçon qui est avec tes serviteurs. Nous n'avons besoin d'aucune chose.
\VS{20}Le vieil homme dit : Pourvu que la paix soit ! Quoi qu'il en soit, je me charge de tous tes besoins, je te prie seulement de ne pas passer la nuit sur la place.
\VS{21}Alors il les fit entrer dans sa maison, et il donna du fourrage aux ânes. Les voyageurs se lavèrent les pieds ; puis ils mangèrent et burent\FTNT{Ge. 43:24.}.
\VS{22}Comme ils se réjouissaient, voici, les hommes de la ville, fils d'hommes pervers, environnèrent la maison, frappèrent à la porte, et dirent au vieil homme, maître de la maison : Fais sortir l'homme qui est entré dans ta maison, afin que nous le connaissions\FTNT{Jg. 20:13 ; Os. 9:9 ; 10:9 ; Ge. 19:4.}.
\VS{23}Mais cet homme, savoir le maître de la maison, sortit vers eux, et leur dit : Non, mes frères, ne lui faites pas de mal, je vous prie ; puisque cet homme est entré dans ma maison, ne faites pas une telle infamie.
\VS{24}Voici, j'ai une fille vierge, et cet homme a une concubine ; je vous les amènerai dehors ; vous les déshonorerez, et vous ferez d'elles comme il semblera bon à vos yeux. Mais ne faites pas cette action infâme à l'égard de cet homme.
\VS{25}Mais ces gens ne voulurent pas l'écouter. C'est pourquoi l'homme saisit sa concubine, et la leur amena dehors. Ils la connurent, et abusèrent d'elle toute la nuit jusqu'au matin ; puis ils la renvoyèrent au lever de l'aurore.
\VS{26}Vers le matin, cette femme alla tomber à la porte de la maison de l'homme où était son mari, et elle y demeura jusqu'au jour.
\VS{27}Et le matin, son mari se leva, et ayant ouvert la porte de la maison, il sortit pour poursuivre son chemin. Mais voici, la femme concubine était tombée à la porte de la maison, et avait les mains sur le seuil.
\VS{28}Il lui dit : Lève-toi, et allons-nous-en. Mais elle ne répondit pas. Alors il l'emmena sur un âne, se mit en chemin, et s'en alla dans sa demeure.
\VS{29}En entrant en sa maison, il prit un couteau, et saisissant sa concubine, il la coupa avec ses os en douze morceaux, qu'il envoya dans tout le territoire d'Israël.
\VS{30}Et il arriva que tous ceux qui virent cela dirent : Une telle chose n'a été faite ni vue depuis le jour où les enfants d'Israël sont montés hors du pays d'Egypte, jusqu'à ce jour ; prenez la chose à cœur, consultez-vous, et parlez !
\Chap{20}
\TextTitle{Israël devant Yahweh à Mitspa}
\VerseOne{}Alors tous les fils d'Israël sortirent, et toute l'assemblée se réunit comme un seul homme, depuis Dan jusqu'à Beer-Schéba et jusqu'au pays de Galaad, devant Yahweh, à Mitspa.
\VS{2}Les chefs de tout le peuple, toutes les tribus d'Israël, se présentèrent à l'assemblée du peuple de Dieu, au nombre de quatre cent mille hommes de pied, tirant l'épée.
\VS{3}Les fils de Benjamin entendirent que les fils d'Israël étaient montés à Mitspa. Les fils d'Israël dirent : Parlez, comment ce mal est arrivé ?
\VS{4}Alors le Lévite, mari de la femme tuée, répondit, et dit : J'étais venu à Guibea de Benjamin, avec ma concubine, pour y passer la nuit.
\VS{5}Les seigneurs de Guibea se sont élevés contre moi, et ont encerclé de nuit la maison où j'étais. Ils avaient l'intention de me tuer, et ils ont tellement violé ma concubine qu'elle en est morte.
\VS{6}C'est pourquoi j'ai saisi ma concubine, je l'ai coupée en morceaux, et je les ai envoyés dans tout le territoire de l'héritage d'Israël ; car ils ont fait un crime et une infamie en Israël.
\VS{7}Vous voici tous, fils d'Israël ; consultez-vous sur la question, et prenez ici une décision !
\VS{8}Tout le peuple se leva comme un seul homme, et ils dirent : Aucun homme n'ira dans sa tente, et aucun homme ne se retirera dans sa maison.
\VS{9}Et maintenant voici ce que nous ferons à Guibea : Nous marcherons contre elle d'après le sort.
\VS{10}Nous prendrons dans toutes les tribus d'Israël dix hommes sur cent, cent sur mille, et mille sur dix mille ; nous prendrons des provisions pour le peuple, afin qu'en entrant à Guibea de Benjamin, on leur fasse selon toute l'infamie qu'elle a commise en Israël.
\VS{11}Ainsi tous les hommes d'Israël s'assemblèrent contre la ville, unis comme un seul homme.
\VS{12}Alors les tribus d'Israël envoyèrent des hommes vers la maison de Benjamin, pour dire : Quelle méchanceté a été faite parmi vous ?
\VS{13}Maintenant donc livrez-nous les fils des hommes pervers qui sont à Guibea, afin que nous les fassions mourir et que nous ôtions le mal du milieu d'Israël. Mais les fils de Benjamin ne voulurent pas écouter la voix de leurs frères, les enfants d'Israël.
\VS{14}Et les fils de Benjamin s'assemblèrent à Guibea pour sortir en guerre contre les fils d'Israël.
\VS{15}En ce jour-là, on fit le dénombrement des fils de Benjamin qui étaient dans ces villes, et il se trouva vingt-six mille hommes, tirant l'épée, sans compter les habitants de Guibea formant sept cents hommes d'élite.
\VS{16}De tout ce peuple, il y avait sept cents hommes d'élite qui ne se servaient pas de la main droite ; tous tirant la pierre avec la fronde,  à un cheveu près,  ils n'y manquaient pas.
\VS{17}On fit aussi le dénombrement des hommes d'Israël, excepté ceux de Benjamin, et l'on en trouva quatre cent mille hommes tirant l'épée, tous gens de guerre.
\TextTitle{Coalition pour monter contre Benjamin}
\VS{18}Et les fils d'Israël se levèrent, montèrent vers Dieu à Béthel pour le consulter, en disant : Qui d'entre nous montera le premier pour faire la guerre aux fils de Benjamin ? Yahweh répondit : Juda montera le premier.
\VS{19}Puis les fils d'Israël se levèrent de bon matin, et campèrent près de Guibea.
\VS{20}Et les hommes d'Israël sortirent pour combattre ceux de Benjamin, et se rangèrent en bataille près de Guibea.
\VS{21}Les fils de Benjamin sortirent de Guibea, et ils tuèrent ce jour-là vingt-deux mille hommes d'Israël.
\VS{22}Toutefois le peuple, les hommes d'Israël, se fortifièrent et se rangèrent de nouveau en bataille au lieu où ils s'étaient rangés le premier jour.
\VS{23}Et les fils d'Israël montèrent, et ils pleurèrent devant Yahweh jusqu'au soir ; ils consultèrent Yahweh en disant : M'approcherai-je encore pour combattre contre les fils de Benjamin, mon frère ? Yahweh dit : Montez contre lui.
\VS{24}Le second jour, les fils d'Israël s'approchèrent des fils de Benjamin.
\VS{25}Ce même jour, les Benjamites sortirent de Guibea à leur rencontre, et ils tuèrent encore dix-huit mille hommes des fils d'Israël, tous tirant l'épée.
\VS{26}Alors tous les fils d'Israël et tout le peuple montèrent et vinrent vers Dieu à Béthel ; ils pleurèrent, et restèrent là devant Yahweh. Ce jour-là ils jeûnèrent jusqu'au soir, et ils offrirent des holocaustes, et des sacrifices de paix devant Yahweh.
\VS{27}Ensuite les fils d'Israël consultèrent Yahweh, c'était là que se trouvait l'arche de l'alliance de Dieu ;
\VS{28}et Phinées, fils d'Eléazar, fils d'Aaron, se tenait devant Yahweh en ce temps-là en disant : Sortirai-je encore en guerre contre les fils de Benjamin, mon frère, ou dois-je m'en abstenir ? Yahweh répondit : Montez, car demain je les livrerai entre vos mains.
\VS{29}Alors Israël mit une embuscade autour de Guibea.
\VS{30}Le troisième jour, les fils d'Israël montèrent contre les fils de Benjamin, et ils se rangèrent en bataille contre Guibea, comme les autres fois.
\VS{31}Alors les fils de Benjamin sortirent à la rencontre du peuple, et ils furent attirés hors de la ville. Ils commencèrent à frapper à mort quelques-uns du peuple comme les autres fois, environ trente hommes d'Israël, sur les routes dont l'une monte à Béthel et l'autre à Guibea, par les champs.
\VS{32}Les fils de Benjamin disaient : Ils tombent battus devant nous, comme la première fois ! Mais les fils d'Israël disaient : Fuyons, et attirons-les hors de la ville dans les chemins.
\VS{33}Tous les hommes d'Israël se levant de leur lieu, se rangèrent à Baal-Thamar ; et l'embuscade sortit du lieu où ils étaient, de Maaré-Guibea.
\VS{34}Dix mille hommes choisis sur tout Israël vinrent contre Guibea. La bataille fut rude, et les Benjamites ne surent pas que le mal les atteindrait.
\VS{35}Yahweh battit Benjamin devant Israël, et les fils d'Israël tuèrent ce jour-là vingt-cinq mille cent hommes de Benjamin, tous tirant l'épée.
\VS{36}Les fils de Benjamin regardaient comme battus les hommes d'Israël, qui cédaient du terrain à Benjamin et se reposaient sur l'embuscade qu'ils avaient mise près de Guibea.
\VS{37}Ceux qui étaient en embuscade se jetèrent promptement sur Guibea, ils se portèrent en avant et frappèrent toute la ville au tranchant de l'épée.
\VS{38}Et le signal convenu entre les hommes d’Israël et l’embuscade était qu’ils fassent monter beaucoup de fumée de la ville.
\VS{39}Les hommes d’Israël avaient donc tourné le dos dans la bataille, et les Benjamites avaient commencé de frapper et de blesser à mort environ trente hommes de ceux d’Israël ; et ils disaient : Certainement ils tombent devant nous comme à la première bataille !
\VS{40}Mais quand l'épaisse colonne de fumée commençait à monter de la ville, les Benjamites se tournèrent ; et voici, derrière eux toute la ville disparaissait montant en feu vers le ciel.
\VS{41}Les hommes d'Israël tournèrent le visage ; et ceux de Benjamin furent épouvantés en voyant le mal qui allait les  atteindre.
\VS{42}Ils tournèrent le dos devant les hommes d'Israël par le chemin du désert. Mais les assaillants s'attachaient à leurs pas, et ils détruisirent ceux qui étaient sortis des villes.
\VS{43}Ils environnèrent Benjamin, le poursuivirent, l'écrasèrent dès qu'il voulut se reposer jusqu'en face de Guibea, du côté du soleil levant.
\VS{44}Il tomba dix-huit mille hommes de Benjamin, tous des vaillants hommes.
\VS{45}Et parmi ceux de Benjamin qui tournèrent le dos pour s'enfuir vers le désert au rocher de Rimmon, les hommes d'Israël en firent périr cinq mille hommes sur les routes ; et les poursuivant de près jusqu'à Guideom, ils frappèrent deux mille hommes.
\TextTitle{La tribu de Benjamin décimée}
\VS{46}En ce jour-là, le nombre de Benjamites qui tombèrent fut de vingt-cinq mille hommes tirant l'épée, et tous étaient des vaillants hommes.
\VS{47}Et il y eut six cents hommes de ceux qui avaient tourné le dos, qui s'échappèrent vers le désert au rocher de Rimmon, et qui demeurèrent au rocher de Rimmon pendant quatre mois.
\VS{48}Les hommes d'Israël retournèrent vers les fils de Benjamin, et ils les frappèrent du tranchant de l'épée, depuis les hommes des villes jusqu'aux bêtes, et tout ce qui s'y trouva. Ils brûlèrent toutes les villes qu'ils trouvaient.
\Chap{21}
\TextTitle{Deuil national}
\VerseOne{}Les hommes d'Israël avaient juré à Mitspa, en disant : Aucun homme ne donnera sa fille pour femme à un Benjamite.
\VS{2}Puis le peuple vint vers Dieu à Béthel, jusqu'au soir. Ils élevèrent leurs voix, et pleurèrent grandement,
\VS{3}Et ils dirent : Ô Yahweh, Dieu d'Israël, pourquoi est-il arrivé en Israël qu'une tribu d'Israël ait été aujourd'hui punie ?
\VS{4}Le lendemain, le peuple se leva de bon matin ; ils bâtirent là un autel, et ils offrirent des holocaustes et des sacrifices d'offrande de paix.
\VS{5}Alors les fils d'Israël dirent : Quel est celui d'entre toutes les tribus d'Israël qui n'est pas monté à l'assemblée vers Yahweh ? Car on avait fait un grand serment contre tout homme qui ne monterait pas vers Yahweh à Mitspa, en disant : Il sera puni de mort.
\VS{6}Les fils d'Israël se repentaient de ce qui était arrivé à Benjamin, leur frère, et ils disaient : Aujourd'hui une tribu a été retranchée d'Israël.
\VS{7}Comment ferons-nous pour donner des femmes à ceux qui ont survécu, puisque nous avons juré par Yahweh que nous ne leur donnerions pas nos filles pour femmes ?
\TextTitle{Avenir de la tribu de Benjamin}
\VS{8}Ils dirent donc : Y a-t-il quelqu'un d'entre les tribus d'Israël qui ne soit pas monté vers Yahweh à Mitspa ? Et voici, aucun homme de Jabès en Galaad n'était venu au camp, à l'assemblée.
\VS{9}Quand on fit le dénombrement du peuple, il n'y avait aucun des hommes habitant à Jabès en Galaad.
\VS{10}C'est pourquoi l'assemblée envoya contre eux douze mille hommes des fils vaillants, en leur donnant cet ordre : Allez, et frappez du tranchant de l'épée les habitants de Jabès en Galaad, tant les femmes que les enfants.
\VS{11}Voici les choses que vous ferez : Vous détruirez par le moyen de l'interdit tout mâle et toute femme qui a connu la couche d'un homme.
\VS{12}Ils trouvèrent parmi les habitants de Jabès en Galaad quatre cents filles vierges, qui n'avaient pas connu d'homme en couchant avec lui, et ils les amenèrent au camp de Silo, qui est sur la terre de Canaan.
\VS{13}Alors toute l'assemblée envoya parler aux fils de Benjamin qui étaient au rocher de Rimmon,  pour leur proclamer la paix.
\VS{14}En ce temps-là, les Benjamites revinrent, et on leur donna pour femmes celles qui avaient été conservées en vie d'entre les femmes de Jabès en Galaad. Mais ils n’en trouvèrent pas assez pour eux.
\VS{15}Le peuple se repentit de ce qui avait été fait à Benjamin, car Yahweh avait fait une brèche dans les tribus d'Israël.
\VS{16}Les anciens de l'assemblée dirent : Comment ferons-nous pour donner des femmes à ceux qui restent, car les femmes de Benjamin ont été détruites ?
\VS{17}Et ils dirent : Que ceux qui sont réchappés de Benjamin possèdent leur héritage, afin qu'une tribu d'Israël ne soit pas effacée.
\VS{18}Cependant, nous ne pouvons pas leur donner des femmes d'entre nos filles, car les fils d'Israël ont juré, en disant : Maudit soit celui qui donnera une femme à un Benjamite !
\VS{19}Et ils dirent : Voici, il y a chaque année une fête de Yahweh à Silo, qui est au nord de Béthel, à l'orient qui monte à Béthel, à Sichem, et au midi de Lebona.
\VS{20}Puis ils ordonnèrent aux fils de Benjamin : Allez, et placez-vous en embuscade dans les vignes.
\VS{21}Vous verrez, et voici, lorsque les filles de Silo sortiront pour danser, alors vous sortirez des vignes, vous enlèverez chacun une des filles de Silo pour en faire votre femme, et vous vous en irez dans le pays de Benjamin.
\VS{22}Si leurs pères ou leurs frères viennent se plaindre auprès de nous, nous leur dirons : Accordez-nous cette faveur, puisque nous n'avons pas pris de femmes pour chaque homme dans cette guerre.  Ce n'est pas vous qui les leur avez données ; sinon vous en seriez coupables en ce temps.
\VS{23}Les fils de Benjamin firent ainsi ; ils prirent des femmes selon leur nombre, parmi les danseuses qu'ils saisirent, puis ils s'en allèrent et retournèrent dans leur héritage ; ils rebâtirent les villes, et y habitèrent.
\VS{24}Ainsi en ce temps-là chacun des enfants d’Israël s’en alla de là dans sa tribu, et dans sa famille, et ils se retirèrent de là chacun dans son héritage.
\VS{25}En ce temps-là, il n'y avait pas de roi en Israël. L'homme faisait ce qui lui semblait être droit à ses yeux.
\PPE{}
\end{multicols}

%\clearpage\ShortTitle{1 Samuel}\BookTitle{1 Samuel}\BFont
\noindent\hrulefill
{\footnotesize
\textit{
\bigskip
{\centering{}
\\(Shemouel)
\\Signifie : Entendu, Exaucé de Dieu
\\Thème : Samuel, Saül et David
\\Auteur : Inconnu
\\Date de rédaction : 10ème siècle av. J.-C.\\}
}
%\bigskip
\textit{
\\Samuel naquit de l’union entre Elkana, de la montagne d’Ephraïm, et Anne.  Sa mère, que Yahweh avait rendu stérile, fit une alliance avec Dieu et lui promit de lui consacrer son premier fils. Ainsi, Samuel fut dès son plus jeune âge amené à la maison de Dieu où il grandit aux côtés d’Eli, le sacrificateur. A la mort de ce dernier, Samuel exerça les fonctions de juge, sacrificateur et prophète sur Israël. C’est en son temps qu’Israël manifesta le désir d’avoir un roi, marquant ainsi la fin de l’ère des juges et le début de la monarchie en Israël.
%\bigskip
\\Ce livre relate l’histoire de Saül, premier roi de l’histoire d’Israël, à qui Yahweh accorda de puissantes victoires notamment sur les philistins, grand ennemi du peuple de Dieu. Le parcours de Saul ne fut pas sans erreur, aussi Yahweh le disqualifia et choisit pour lui succéder sur le trône un homme de la tribu de Juda, David fils d’Isaï. L’accès à la royauté de ce dernier ne fut pas immédiat comme en témoignent ces écrits. David dut faire preuve de patience, de courage et de confiance en Dieu au milieu de nombreuses persécutions.
%\bigskip
\\Au travers de la vie des deux premiers rois d’Israël, est mise en évidence l’importance de l’obéissance à Dieu ; au travers de la vie de Samuel est mis en exergue l’impact de la prière dans une vie, une nation.\bigskip
}
}
\par\nobreak\noindent\hrulefill
\begin{multicols}{2}
\TextTitle{[Stérilité d'Anne la mère de Samuel]}
\Chap{1}
\VerseOne{}Il y avait un homme de Ramathaïm-Tsophim, de la montagne d'Ephraïm, nommé Elkana, fils de Jeroham, fils d'Elihu, fils de Thohu, fils de Tsuph, Ephratien.
\VS{2}Il avait deux femmes, dont l'une s'appelait Anne, et l'autre Peninna. Peninna avait des enfants, mais Anne n'en avait pas.
\VS{3}Or cet homme-là montait tous les ans, de sa ville à Silo\FTNT{Jos. 18:1.}, pour adorer Yahweh des armées, et lui offrir des sacrifices. Là étaient les deux fils d’Eli, Hophni et Phinées, sacrificateurs de Yahweh.
\VS{4}Le jour où Elkana offrait son sacrifice, il donnait des portions à Peninna, sa femme, à tous les fils et à toutes les filles qu'il avait d'elle.
\VS{5}Mais il donnait à Anne une portion double ; car il aimait Anne, mais Yahweh avait fermé sa matrice\FTNT{Dieu est celui qui ferme et ouvre les portes des bénédictions.}.
\VS{6}Sa rivale lui portait envie et la mortifiait fort aigrement afin de l’irriter, car Yahweh avait fermé sa matrice.
\VS{7}Et Elkana faisait donc ainsi tous les ans. Mais quand Anne montait à la maison de Yahweh, Peninna la mortifiait de la même manière, et Anne pleurait et ne mangeait pas.
\VS{8}Elkana, son mari, lui disait : Anne, pourquoi pleures-tu ? Et pourquoi ne manges-tu pas ? Pourquoi ton cœur est-il triste ? Est-ce que je ne vaux pas pour toi mieux que dix fils ?
\TextTitle{[Prière et voeu d'Anne à Yahweh]}
\VS{9}Anne se leva, après avoir mangé et bu à Silo. Et le sacrificateur Eli était assis sur un siège, près de l’un des poteaux du temple de Yahweh.
\VS{10}Elle donc, ayant le coeur rempli d'amertume, pria Yahweh en pleurant abondamment.
\VS{11}Et elle fit un vœu, en disant : Yahweh des armées ! Si tu regardes attentivement l'affliction de ta servante, et si tu te souviens de moi, et n'oublies pas ta servante, et que tu donnes à ta servante un enfant mâle, je le donnerai à Yahweh pour tous les jours de sa vie ; et aucun rasoir ne passera sur sa tête.
\VS{12}Il arriva, comme elle continuait à prier devant Yahweh, Eli observait sa bouche.
\VS{13}Or Anne parlait dans son cœur, elle ne faisait que remuer ses lèvres et on n'entendait pas sa voix. C’est pourquoi Eli estima qu'elle était ivre,
\VS{14}et Eli lui dit : Jusqu'à quand seras-tu ivre ? Eloigne-toi du vin.
\VS{15}Mais Anne répondit et dit : Je ne suis pas ivre, mon seigneur, je suis une femme affligée en son esprit, je n'ai bu ni vin ni boisson forte, mais je répandais mon âme devant Yahweh.
\VS{16}Ne mets pas ta servante au rang d'une femme pervertie, car c'est l’excès de ma douleur et de mon affliction qui m’a fait parler jusqu'à présent.
\VS{17}Alors Eli répondit et dit : Va en paix, et que le Dieu d'Israël veuille t’accorder la demande que tu lui as faite.
\VS{18}Et elle dit : Que ta servante trouve grâce à tes yeux ! Puis cette femme poursuivit son voyage. Elle mangea, et son visage ne fut plus le même.
\TextTitle{[Naissance de Samuel]}
\VS{19}Après cela, ils se levèrent de bon matin, et se prosternèrent devant Yahweh, puis ils s'en retournèrent et revinrent dans leur maison à Rama. Elkana connut Anne, sa femme, et Yahweh se souvint d'elle.
\VS{20}Il arriva donc, quelque temps après, qu'Anne conçut et enfanta un fils ; elle le nomma Samuel, parce que dit-elle, je l'ai demandé à Yahweh.
\VS{21}Puis Elkana, son mari, monta avec toute sa maison, pour offrir à Yahweh le sacrifice annuel et son vœu.
\VS{22}Mais Anne n'y monta pas, car elle dit à son mari : Je n’irai pas jusqu'à ce que le petit enfant soit sevré, et alors je le mènerai afin qu'il soit présenté devant Yahweh et qu'il demeure toujours-là.
\VS{23}Elkana, son mari, lui dit : Fais ce qui te semblera bon, reste jusqu'à ce que tu l'aies sevré. Seulement que Yahweh accomplisse sa parole. Ainsi cette femme resta et allaita son fils, jusqu'à ce qu'elle l’ait sevré.
\TextTitle{[Samuel chez Elie, Anne accomplie son voeu]}
\VS{24}Et dès qu'elle l'eut sevré, elle le fit monter avec elle, et ayant pris trois taureaux, un épha de farine et une outre de vin, elle le mena dans la maison de Yahweh à Silo ; l'enfant était très jeune.
\VS{25}Puis ils égorgèrent le veau, et ils amenèrent l'enfant à Eli.
\VS{26}Elle dit : Pardon, mon seigneur ! Aussi vrai que ton âme vit, mon seigneur, je suis cette femme qui me tenais en ta présence pour prier Yahweh.
\VS{27}J'ai prié pour avoir cet enfant, et Yahweh m’a accordé la demande que je lui ai faite.
\VS{28}C'est pourquoi je le prête à Yahweh ; il sera prêté à Yahweh pour tous les jours de sa vie. Et ils se prosternèrent là devant Yahweh.
\TextTitle{[Prière et prophésie d'Anne]}
\Chap{2}
\VerseOne{}Alors Anne pria, et dit : Mon cœur se réjouit en Yahweh ; ma force a été relevée par Yahweh ; ma bouche s'est ouverte contre mes ennemis, parce que je me suis réjouie de ton salut\FTNT{Le mot « salut » vient  de l’hébreu « yeshuw`ah » c’est-à-dire « Jésus ». Voir commentaire en Es. 26:1.}.
\VS{2}Nul n’est saint comme Yahweh ; car il n'y en a pas d'autre que toi, et il n'y a pas de rocher\FTNT{Voir commentaire en Es. 8:13-17.} tel que notre Dieu.
\VS{3}Ne proférez pas tant de paroles hautaines ; qu'il ne sorte pas de votre bouche des paroles arrogantes ; car Yahweh est le Dieu qui sait tout ; c'est lui qui pèse toutes les actions.
\VS{4}L'arc des puissants est brisé, mais ceux qui chancèlent ont la force pour ceinture.
\VS{5}Ceux qui étaient rassasiés, se louent pour du pain, mais les affamés ont cessé de l'être ; même la stérile en a enfanté sept et celle qui avait beaucoup de fils est devenue languissante.
\VS{6}Yahweh est celui qui fait mourir et qui fait vivre, qui fait descendre au scheol et qui en fait remonter.
\VS{7}Yahweh appauvrit et il enrichit, il abaisse et il élève.
\VS{8}De la poussière il retire le pauvre, du fumier il relève l’indigent, pour le faire asseoir avec les nobles, avec les nobles de son peuple, et il leur donne en héritage un trône de gloire ; car les colonnes de la terre sont à Yahweh, et il a posé le monde sur elles.
\VS{9}Il gardera les pieds de ses bien-aimés, et les méchants se tairont dans les ténèbres. Car l'homme ne triomphera pas par sa force.
\VS{10}Ceux qui contestent contre Yahweh seront effrayés ; des cieux il lancera son tonnerre sur chacun d'eux ; Yahweh jugera les extrémités de la terre ; et il donnera la force à son Roi\FTNT{Le Roi dont il est question ici est le Seigneur Jésus-Christ, le Roi des rois (Za. 14:9 ; Ap. 19:16). }, et élèvera la corne de son Messie\FTNT{Anne a annoncé la glorification ou la résurrection du Seigneur Jésus, le Messie (Jn. 3:14).}.
\VS{11}Puis Elkana s'en alla à Rama dans sa maison, et le jeune garçon vaquait au service de Yahweh, en présence du sacrificateur Eli.
\TextTitle{[Corruption des fils d'Eli]}
\VS{12}Or les fils d'Eli\FTNT{Les fils d’Eli, Hophni et Phinées étaient corrompus. Ils volaient les offrandes de Dieu, couchaient avec les femmes qui venaient adorer Dieu. L’esprit qui animait ces sacrificateurs  n’a pas disparu après leur mort, mais il opère  encore dans beuacoup d’institutions religieuses actuelles. Beaucoup de dirigeants d’églises continuent à s’aproprier ce qui appartient à Dieu (l’adoration, les âmes... ) Ils ne craignent pas Yahweh. Ils abusent de leur position et de leur autorité pour contraindre leurs fidèles à leur donner la dîme et toutes sortes d’offrandes. Ils font payer les entretiens, les prières, et les divers dons qu’ils peuvent avoir. Non seulement l'esprit qui animait les fils d'Eli existe encore, mais il s'est accru en ces temps actuels.} étaient des fils de Bélial\FTNT{Les fils d’Eli étaient qualifiés de « fils de Bélial ». Ce mot vient de l’hébreu « beliya`al » qui signifie « indigne », « bon à rien », « méchant », « ruine », « destruction ». Il est à noter que Bélial est aussi  un nom de Satan. (2 Co. 6:15). Les fils d’Eli servaient Dieu sans le connaître. En fait, ils étaient au service de Satan. Ce terme est également utilisé au sujet des méchants qui incitèrent les Israélites à servir les dieux étrangers (De. 13:14), les hommes iniques de Guivéa (Jg. 19:22 ;  Jg. 20:13), les deux vauriens qui accusèrent Naboth (1 R. 21:10-13) et les individus qui s’opposèrent à la monarchie (1 S. 10:27 ; 2 S. 20:1 ; 2 Ch. 13:7). Voir aussi  De. 13:13 ; De. 15:9 ; Job. 34:18 ; Ps. 18:4 ;  Ps. 34:8 ; Ps. 111:3 ; Pr. 6:12 ; Pr. 16:27 ; Pr. 19:28 ; Na. 1:11 ; Na. 1:18.} et ils ne connaissaient pas Yahweh,
\VS{13}et voici la coutume de ces sacrificateurs envers le peuple : Lorsque quelqu'un faisait quelque sacrifice, le serviteur du sacrificateur venait lorsqu'on faisait bouillir la chair, ayant à la main une fourchette à trois dents,
\VS{14}avec laquelle il piquait dans la chaudière, dans le chaudron, dans la marmite, dans le pot ; et le sacrificateur prenait pour lui tout ce que la fourchette enlevait. C’est ainsi qu’ils agissaient envers tous ceux d'Israël qui venaient à Silo.
\VS{15}Même avant qu'on fasse brûler la graisse, le serviteur du sacrificateur venait et disait à l'homme qui sacrifiait : Donne-moi de la chair à rôtir pour le sacrificateur ; car il ne prendra pas de toi de chair bouillie, mais de la chair crue.
\VS{16}Et si l'homme lui répondait : On va d’abord faire brûler la graisse, et après cela tu prendras ce que ton âme souhaitera, alors le serviteur lui disait : Quoi qu'il en soit, tu en donneras maintenant, sinon j'en prendrai de force.
\VS{17}Et le péché de ces jeunes hommes fut très grand devant Yahweh, car ils méprisaient l’offrande de Yahweh.
\TextTitle{[Samuel au service de Yahweh]}
\VS{18}Samuel faisait le service en présence de Yahweh, étant jeune garçon, vêtu d'un éphod de lin.
\VS{19}Sa mère lui faisait une petite tunique, qu'elle lui apportait tous les ans, quand elle montait avec son mari pour offrir le sacrifice annuel.
\VS{20}Eli bénit Elkana, et sa femme, et dit : Que Yahweh te donne des enfants de cette femme, pour le prêt qu’elle a fait à Yahweh. Et ils s'en retournèrent chez eux.
\VS{21}Et Yahweh visita Anne, elle conçut et enfanta trois fils et deux filles ; et le jeune garçon Samuel grandissait en présence de Yahweh.
\TextTitle{[Eli avertit ses fils]}
\VS{22}Or Eli était très vieux, il apprit tout ce que faisaient ses fils à tout Israël, et qu'ils couchaient avec les femmes qui s'assemblaient à la porte de la tente d'assignation.
\VS{23}Et il leur dit : Pourquoi commettez-vous de telles choses ? Car j'apprends vos méchantes actions de tout le peuple.
\VS{24}Ne faites pas ainsi, mes fils, car ce que j'entends dire de vous n'est pas bon ; vous faites pécher le peuple de Yahweh.
\VS{25}Si un homme a péché contre un autre homme, le juge interviendra ; mais si quelqu'un pèche contre Yahweh, qui interviendra pour lui ? Mais ils n'obéirent pas à la voix de leur père parce que Yahweh voulait les faire mourir.
\VS{26}Cependant le jeune garçon Samuel croissait et il était agréable à Yahweh et aux hommes.
\TextTitle{[Yahweh annonce un jugement sur la maison d'Eli]}
\VS{27}Or un homme de Dieu vint auprès d’Eli, et lui dit : Ainsi parle Yahweh : Ne me suis-je pas clairement manifesté à la maison de ton père, quand ils étaient en Egypte, dans la maison de Pharaon ?
\VS{28}Je l'ai choisie parmi toutes les tribus d'Israël pour être mon sacrificateur, afin d'offrir sur mon autel, et faire brûler les parfums, et porter l'éphod devant moi, et j'ai donné à la maison de ton père toutes les offrandes des enfants d'Israël consumées par le feu.
\VS{29}Pourquoi avez-vous foulé aux pieds mes sacrifices et mes offrandes que j'ai ordonné de faire dans ma demeure ? Et pourquoi as-tu honoré tes fils plus que moi, afin de vous engraisser du meilleur de toutes les offrandes d'Israël mon peuple ?
\VS{30}C'est pourquoi voici ce que dit Yahweh, le Dieu d'Israël : J'avais dit et promis que ta maison et la maison de ton père marcheraient devant moi éternellement. Et maintenant, dit Yahweh : Il n’en sera pas ainsi ; car j'honorerai ceux qui m'honorent, mais ceux qui me méprisent seront méprisés.
\VS{31}Voici, les jours viennent où je couperai ton bras, et le bras de la maison de ton père, de telle sorte qu'il n'y ait plus de vieillard dans ta maison.
\VS{32}Et tu verras un adversaire dans ma demeure, au temps où Dieu enverra toutes sortes de biens à Israël ; et il n'y aura plus jamais de vieillard dans ta maison.
\VS{33}Celui de tes descendants que je n'aurai pas retranché d'auprès de mon autel, subsistera pour consumer tes yeux et affliger ton âme ; et tous les enfants de ta maison mourront dans la fleur de l'âge.
\VS{34}Et ceci sera pour toi un signe, à savoir ce qui arrivera à tes deux fils, Hophni et Phinées, ils mourront tous les deux le même jour.
\VS{35}Et je m'établirai un sacrificateur fidèle\FTNT{Hé. 2:17 ; Hé 7:26-28). }, qui agira selon mon cœur, et selon mon âme ; et je lui édifierai une maison stable\FTNT{La maison stable fait premièrement allusion à Israël (Mi. 4) et ensuite à l’Eglise (Mt. 16:18). Cette prophétie sera pleinement réalisée lors du millénium (Za. 14).}, et il marchera à toujours devant mon Messie.
\VS{36}Et quiconque restera de ta maison, viendra se prosterner devant lui pour avoir une pièce d'argent et un morceau de pain et dira : Attache-moi, je te prie, à l’une des fonctions du sacerdoce pour manger un morceau de pain.
\TextTitle{[Yahweh appelle Samuel]}
\Chap{3}
\VerseOne{}Le jeune garçon Samuel servait Yahweh en présence d'Eli. La parole de Yahweh était rare en ce temps-là, et les visions n’étaient pas fréquentes.
\VS{2}Il arriva en ce temps qu'Eli était couché à sa place, ses yeux commençaient à se ternir et il ne pouvait plus voir.
\VS{3}Et avant que les lampes\FTNT{Le chandelier d'or à sept branches du tabernacle et du temple de Jérusalem a été décrit avec une extrême minutie dans plusieurs passages de la Bible. Il a été réalisé selon le modèle imposé par Dieu à Moïse au Sinaï (Ex. 25:31-40 ; Ex. 37:17-24 ; No. 8:4).} de Dieu soient éteintes, Samuel était aussi couché dans le temple de Yahweh, où était l'arche de Dieu.
\VS{4}Yahweh appela Samuel. Et il répondit : Me voici !
\VS{5}Et il courut vers Eli, et lui dit : Me voici, car tu m'as appelé ; mais Eli dit : Je ne t'ai pas appelé, retourne te coucher. Et il s'en alla, et se coucha.
\VS{6}Yahweh appela encore Samuel. Et Samuel se leva, et s'en alla vers Eli, et lui dit : Me voici, car tu m'as appelé ! Et Eli dit : Mon fils, je ne t'ai pas appelé, retourne, et couche-toi.
\VS{7}Or Samuel ne connaissait pas encore Yahweh, et la parole de Yahweh ne lui avait pas encore été révélée.
\VS{8}Et Yahweh appela encore Samuel pour la troisième fois ; et Samuel se leva, et s'en alla vers Eli, et dit : Me voici, car tu m'as appelé. Eli reconnut que Yahweh appelait ce jeune garçon.
\VS{9}Alors Eli dit à Samuel : Va et couche-toi ; et si on t'appelle, tu diras : Parle Yahweh , car ton serviteur écoute. Samuel donc s'en alla, et se coucha à sa place.
\VS{10}Yahweh donc vint, et se tint là ; et appela comme les autres fois : Samuel, Samuel ! Et Samuel dit : Parle, car ton serviteur écoute.
\TextTitle{[Autre avertissement de Yahweh à Eli par Samuel]}
\VS{11}Alors Yahweh dit à Samuel : Voici, je vais faire une chose en Israël, qui étourdira les oreilles de quiconque l’entendra.
\VS{12}En ce jour-là, j’accomplirai sur Eli tout ce que j’ai déclaré contre sa maison ; je commencerai, et j’achèverai.
\VS{13}Car je l'ai averti que je vais punir sa maison à perpétuité, à cause de l'iniquité dont il a connaissance, par laquelle ses fils se sont rendus infâmes, sans qu’ils les ait réprimés.
\VS{14}C'est pourquoi j'ai juré contre la maison d'Eli que jamais l’iniquité de la la maison d'Eli, ne sera expiée ni par des sacrifices ni par des offrandes.
\VS{15}Et Samuel resta couché jusqu'au matin, puis il ouvrit les portes de la maison de Yahweh. Or Samuel craignait de rapporter cette vision à Eli.
\VS{16}Mais Eli appela Samuel, et lui dit : Samuel mon fils ! Il répondit : Me voici !
\VS{17}Et Eli dit : Quelle est la parole qui t'a été adressée ? Je te prie ne me la cache pas. Que Dieu te traite avec rigueur, si tu me caches un seul mot de tout ce qui t'a été dit.
\VS{18}Samuel lui déclara donc toutes ces paroles, et ne lui en cacha rien. Et Eli répondit : C'est Yahweh, qu'il fasse ce qui lui semblera bon !
\TextTitle{[Samuel, prophète de Yahweh]}
\VS{19}Samuel grandissait. Et Yahweh était avec lui, il ne laissa pas tomber à terre une seule de ses paroles.
\VS{20}Tout Israël, depuis Dan jusqu'à Beer-Schéba, reconnut que Samuel était établi prophète de Yahweh.
\VS{21}Yahweh continuait de se manifester dans Silo ; car Yahweh se manifestait à Samuel dans Silo par la parole de Yahweh.
\TextTitle{[Les philistins prennent l'arche de Yahweh, jugement sur la maison d'Eli]}
\Chap{4}
\VerseOne{}La parole de Samuel s’adressait à tout Israël. Car Israël sortit en bataille pour aller à la rencontre des Philistins. Ils campèrent près d'Eben-Ezer, et les Philistins campaient à Aphek.
\VS{2}Les Philistins se rangèrent en bataille contre d'Israël, et le combat s’engagea, Israël fut battu par les Philistins, qui en tuèrent environ quatre mille hommes sur le champ de bataille.
\VS{3}Quand le peuple rentra au camp, les anciens d'Israël dirent : Pourquoi Yahweh nous a-t-il laissé battre aujourd'hui par les Philistins ? Ramenons de Silo l'arche de l'alliance de Yahweh, et qu'elle vienne au milieu de nous, et nous délivre de la main de nos ennemis.
\VS{4}Le peuple envoya donc à Silo, d’où l’on apporta l'arche de l'alliance de Yahweh des armées, qui habite entre les chérubins. Les deux fils d’Eli, Hophni et Phinées étaient là, avec l'arche de l'alliance de Dieu.
\VS{5}Et comme l'arche de Yahweh entrait dans le camp, tout Israël poussa de grands cris de joie et la terre en fut ébranlée.
\VS{6}Les Philistins entendirent le bruit de ces cris de joie, et ils dirent : Que veut dire ce bruit, et que signifient ces grands cris de joie dans le camp de ces Hébreux ? Et ils apprirent que l'arche de Yahweh était arrivée dans le camp.
\VS{7}Les Philistins eurent peur, car ils disaient : Dieu est entré dans le camp. Et ils dirent : Malheur à nous ! Car il n’en a pas été ainsi auparavant.
\VS{8}Malheur à nous ! Qui nous délivrera de la main de ces dieux puissants\FTNT{Le terme hébreu « elohim », généralement traduit par « dieu » ou « dieux », signifie également « dirigeants », « juges » ou encore « anges ».  Dans les textes bibliques, « Elohim » est employé pour désigner Moïse, qui a été fait « dieu » (« Elohim ») pour Pharaon (Ex. 7:1), ainsi que pour  les dieux païens  Baal, Kemosh et Dagaon (Jg. 6:31 ; Jg. 11:24 ; 1 S. 5:7)  Les Philistins avaient une vision polythéiste de la divinité et n’avaient pas la révélation du Dieu des hébreux qui est Un (Dt. 6:4).} ? C’est le Dieu qui a frappé les Egyptiens de toutes sortes de plaies dans le désert.
\VS{9}Philistins prenez courage, et agissez en hommes, de peur que vous ne soyez esclaves des Hébreux, comme ils vous ont été asservis ; agissez en hommes, et combattez !
\VS{10}Les Philistins donc combattirent, et Israël fut battu. Et chacun s’enfuit dans sa tente. La défaite fut très grande, trente mille hommes de pied d'Israël périrent .
\VS{11}L'arche de Dieu fut prise, et les deux fils d'Eli, Hophni et Phinées moururent.
\VS{12}Un homme de Benjamin s'enfuit de la bataille, et arriva à Silo ce même jour, ayant ses vêtements déchirés et la tête recouverte de terre.
\VS{13}Au moment où il arriva, Eli était dans l’attente, assis sur un siège au bord du chemin ; car son cœur tremblait à cause de l'arche de Dieu. Cet homme entra donc dans la ville, et donna les nouvelles , et toute la ville se mit à crier.
\VS{14}Eli, entendant les cris, dit : Que veut dire ce grand tumulte ? Et aussitôt cet homme vint à Eli, et lui raconta tout.
\VS{15}Or Eli était âgé de quatre-vingt-dix-huit ans, ses yeux étaient fixes, il ne pouvait plus voir.
\VS{16}L’homme dit à Eli : Je viens de la bataille, car je me suis enfui aujourd'hui de la bataille. Et Eli dit : Qu'est-il arrivé, mon fils ?
\VS{17}Celui qui apportait les nouvelles répondit : Israël a fui devant les Philistins, e il y a eu une grande défaite du peuple ; tes deux fils, Hophni et Phinées sont morts et l'arche de Dieu a été prise.
\VS{18}Et dès qu'il eut fait mention de l'arche de Dieu, Eli tomba à la renverse, de dessus son siège, à côté de la porte, se rompit le cou et mourut ; car cet homme était vieux et pesant. Il avait été juge en Israël pendant quarante ans.
\VS{19}Sa belle-fille, femme de Phinées, qui était enceinte, et sur le point d'accoucher. Lorsqu’elle apprit la nouvelle de la prise de l'arche de Dieu, de la mort de son beau-père et de son mari, elle se coucha et enfanta, car les douleurs la surprirent.
\VS{20}Comme elle mourait, celles qui l'assistaient lui dirent : Ne crains pas, car tu as enfanté un fils ; mais elle ne répondit rien, et n'en tint pas compte.
\VS{21}Mais elle appela l'enfant I-Kabod, en disant : La gloire s’en est allée d'Israël parce que l'arche de Yahweh était prise à cause de son beau-père et de son mari.
\VS{22}Elle dit donc : La gloire s’en allée d'Israël, car l'arche de Dieu est prise !
\TextTitle{[Jugements de Yahweh sur les philistins]}
\Chap{5}
\VerseOne{}Les Philistins prirent l'arche de Dieu, et l'emmenèrent d'Eben-Ezer à Asdod.
\VS{2}Les Philistins donc prirent l'arche de Dieu, et l'emmenèrent dans la maison de Dagon\FTNT{L'étymologie du nom Dagon  avait justifié la représentation qu’on faisait de ce dieu : une sorte de sirène mâle ou un homme avec queue de poisson. En effet, «  dâg », en hébreu signifie « poisson ». Il était le dieu des semences et de l'agriculture chez les peuples d’origine sémites, mais également l’un des principaux dieux des Philistins.}, et la posèrent auprès de Dagon.
\VS{3}Le lendemain les Asdodiens s'étant levés de bon matin, trouvèrent Dagon le visage contre terre, devant l'arche de Yahweh ; mais ils le prirent et le remirent à sa place.
\VS{4}Ils se levèrent encore le lendemain de bon matin, et voici, Dagon était tombé le visage contre terre, devant l'arche de Yahweh ; la tête de Dagon et les deux paumes de ses mains découpées étaient sur le seuil, et il ne lui restait que le tronc.
\VS{5}C'est pour cela que les sacrificateurs de Dagon, et tous ceux qui entrent dans la maison de Dagon, à Asdod, ne marchent pas sur le seuil jusqu'à aujourd'hui.
\VS{6}Puis la main de Yahweh s'appesantit sur le Asdodiens et les dévasta ; et il les frappa d’hémorroïdes à Asdod et dans tout son territoire.
\VS{7}Ceux donc d'Asdod, voyant qu'il en allait ainsi, dirent : L'arche du Dieu d'Israël ne demeurera pas chez nous ; car sa main s’est appesantie sur nous, et sur Dagon, notre dieu.
\VS{8}Et ils firent appeler et assemblèrent auprès d’eux tous les princes des Philistins, et dirent : Que ferons-nous de l'arche du Dieu d'Israël ? Et ils répondirent : Qu'on transporte à Gath l'arche du Dieu d'Israël. Ainsi on transporta l'arche du Dieu d'Israël.
\VS{9}Mais il arriva après qu'on l'eut transportée, la main de Yahweh fut sur la ville et il y eut une très grande terreur ; et il frappa les gens de la ville depuis le plus petit jusqu'au plus grand, par une éruption d’hémorroïdes.
\VS{10}Ils envoyèrent donc l'arche de Dieu à Ekron. Or comme l'arche de Dieu entrait à Ekron, ceux d’Ekron s'écrièrent, en disant : Ils ont transporté vers nous l'arche du Dieu d'Israël, pour nous faire mourir, nous et notre peuple !
\VS{11}C'est pourquoi ils firent appeler, et assemblèrent tous les princes des Philistins, en disant : Renvoyez l'arche du Dieu d'Israël, et qu'elle retourne en son lieu, afin qu'elle ne nous fasse pas mourir, nous et notre peuple. Car il y régnait une terreur mortelle dans toute la ville, et la main de Dieu s’y appesantissait fortement.
\VS{12}Les hommes qui n'en mouraient pas étaient frappés d’hémorroïdes, de sorte que le cri de la ville montait jusqu'au ciel.
\TextTitle{[L'arche de Yahweh revient en Israël]}
\Chap{6}
\VerseOne{}L'arche de Yahweh ayant été pendant sept mois dans le pays des Philistins.
\VS{2}Les Philistins appelèrent les sacrificateurs et les devins, et leur dirent : Que ferons-nous de l'arche de Yahweh ? Dites-nous comment nous devons la renvoyer en son lieu.
\VS{3}Ils répondirent : Si vous renvoyez l'arche du Dieu d'Israël, ne la renvoyez pas à vide, et n’oubliez pas de lui payer une offrande de culpabilité ; alors vous serez guéris, et vous saurez pourquoi sa main ne s’est pas retirée de dessus vous.
\VS{4}Et ils dirent : Quelle offrande lui payerons-nous pour le péché ? Et ils répondirent : Selon le nombre des princes des Philistins, vous donnerez cinq hémorroïdes d'or, et cinq souris d'or ; car une même plaie a été sur vous tous, et sur vos princes.
\VS{5}Vous ferez donc des figures de vos hémorroïdes, et des figures des souris qui ravagent le pays, et vous donnerez gloire au Dieu d'Israël. Peut-être retirera-t-il sa main de dessus vous, et de dessus vos dieux, et de dessus votre pays.
\VS{6}Et pourquoi endurciriez-vous votre cœur, comme l'Egypte et Pharaon ont endurci leur cœur ? Après qu'il eut fait de merveilleux exploits parmi eux, ne les laissèrent-ils pas partir et s’en aller ?
\VS{7}Maintenant, donc prenez de quoi faire un char tout neuf, et deux jeunes vaches qui allaitent leurs veaux et qui n’aient point porté le joug ; et attelez au char les deux jeunes vaches, et ramenez leurs petits à la maison.
\VS{8}Vous prendrez l'arche de Yahweh et vous la mettrez sur le char ; et vous déposerez dans un coffre, à côté de l’arche, les objets d'or que vous donnez à Yahweh en offrande pour le péché ; vous la renverrez, et elle s'en ira.
\VS{9}Et vous observerez ; si l'arche monte vers Beth-Schémesch, par le chemin de sa frontière, c'est Yahweh qui nous a fait tout ce grand mal ; si elle n'y va pas, nous saurons alors que sa main ne nous a pas touchés, mais que ceci nous est arrivé par hasard.
\VS{10}Ces gens firent ainsi. Ils prirent donc deux jeunes vaches qui allaitaient, ils les attelèrent au char, et ils enfermèrent leurs petits dans l'étable.
\VS{11}Ils mirent sur le char l'arche de Yahweh, et le coffre avec les souris d'or, et les figures de leurs hémorroïdes.
\VS{12}Alors les jeunes vaches prirent tout droit le chemin de Beth-Schémesch, elles suivirent toujours le même chemin en marchant et en mugissant ; et elles ne se détournèrent ni à droite ni à gauche. Les princes des Philistins allèrent après elles jusqu'à la frontière de Beth-Schémesch.
\VS{13}Or ceux de Beth-Schémesch, moissonnaient les blés dans la vallée ; et ayant élevé leurs yeux, ils virent l'arche, et se réjouirent en la voyant.
\VS{14}Le char arriva dans le champ de Josué de Beth-Schémesch, et s'arrêta là. Or il y avait là une grande pierre, et on fendit le bois du char, et on offrit les jeunes vaches en holocauste à Yahweh.
\VS{15}Les Lévites descendirent l'arche de Yahweh, et le coffre dans lequel étaient les objets d'or, et ils les mirent sur cette grande pierre. En ce même jour, ceux de Beth-Schémesch offrirent des holocaustes et des sacrifices à Yahweh.
\VS{16}Les cinq princes des Philistins, après avoir vu cela, retournèrent le même jour à Ekron.
\VS{17}Voici les hémorroïdes d'or que les Philistins donnèrent à Yahweh en offrande pour le péché ; un pour Asdod, un pour Gaza, un pour Askalon, un pour Gath, un pour Ekron.
\VS{18}Les souris d’or, selon le nombre de toutes les villes des Philistins, appartenant aux cinq princes, tant des villes fortifiées, que des villages sans murailles. Et ils les amenèrent jusqu'à la grande pierre sur laquelle on posa l'arche de Yahweh, et qui jusqu'à ce jour est dans le champ de Josué de Beth-Schémesch.
\VS{19}Yahweh frappa des gens de Beth-Schémesch parce qu'ils avaient regardé dans l'arche de Yahweh ; il frappa (cinquante mille) et soixante-dix hommes\FTNT{Ce nombre est généralement considéré comme une erreur des copistes.} et le peuple mena le deuil parce que Yahweh l'avait frappé d'une grande plaie.
\VS{20}Alors ceux de Beth-Schémesch dirent : Qui pourrait subsister en présence de Yahweh, ce Dieu Saint ? Et vers qui montera-t-il en s'éloignant de nous ?
\VS{21}Et ils envoyèrent des messagers aux habitants de Kirjath-Jearim, en disant : Les Philistins ont ramené l'arche de Yahweh ; descendez, et faites-la monter vers vous.
\TextTitle{[Un réveil après l'apostasie]}
\Chap{7}
\VerseOne{}Ceux donc de Kirjath-Jearim vinrent et firent monter l'arche de Yahweh, et la mirent dans la maison d'Abinadab sur la colline ; et ils consacrèrent Eléazar, son fils, pour garder l'arche de Yahweh.
\VS{2}Il s’écoula un long moment, depuis le jour où l'arche de Yahweh fut déposée à Kirjath-Jearim. Vingt années s’étaient écoulées. Toute la maison d'Israël soupira après Yahweh.
\VS{3}Et Samuel parla à toute la maison d'Israël, en disant : Si vous revenez à Yahweh de tout votre cœur, ôtez du milieu de vous les dieux étrangers, et les Astartés, dirigez votre cœur vers Yahweh, et servez-le lui seul ; et il vous délivrera de la main des Philistins.
\VS{4}Alors les enfants d'Israël ôtèrent les Baals, et les Astartés, et ils servirent Yahweh seul\FTNT{Jg. 2:13.}.
\VS{5}Samuel dit : Assemblez tout Israël à Mitspa, et je prierai Yahweh pour vous.
\VS{6}Ils s'assemblèrent donc à Mitspa ; ils puisèrent de l'eau qu'ils répandirent devant Yahweh et ils jeûnèrent ce jour-là, en disant : Nous avons péché contre Yahweh ! Et Samuel jugea les enfants d'Israël à Mitspa.
\VS{7}Or quand les Philistins eurent appris que les enfants d'Israël étaient assemblés à Mitspa, les princes des Philistins montèrent contre Israël. Les enfants d'Israël l’apprirent et ils eurent peur des Philistins.
\VS{8}Les enfants d'Israël dirent à Samuel : Ne cesse pas de crier pour nous à Yahweh, notre Dieu, afin qu'il nous délivre de la main des Philistins.
\TextTitle{[Victoire d'Israël contre les Philistins]}
\VS{9}Alors Samuel prit un agneau de lait, et l'offrit tout entier à Yahweh en holocauste. Et Samuel cria à Yahweh pour Israël, et Yahweh l'exauça.
\VS{10}Comme Samuel offrait l'holocauste, les Philistins s'approchèrent pour combattre contre Israël, mais Yahweh fit gronder, en ce jour-là, un grand tonnerre sur les Philistins, et les mit en déroute, et ils furent battus devant Israël.
\VS{11}Les hommes d'Israël sortirent de Mitspa, et poursuivirent les Philistins, et les frappèrent jusqu'au-dessous de Beth-Car.
\VS{12}Alors Samuel prit une pierre, et la mit entre Mitspa et Schen, et il appela ce lieu Eben-Ezer, en disant : Yahweh nous a secourus jusqu'en ce lieu-ci.
\VS{13}Les Philistins furent humiliés, et ils ne vinrent plus sur le territoire d'Israël. La main de Yahweh fut contre les Philistins durant la vie de Samuel.
\VS{14}Les villes que les Philistins avaient prises sur Israël, retournèrent à Israël, depuis Ekron jusqu'à Gath, avec leurs territoires. Israël les délivra donc de la main des Philistins, et il y eut paix entre Israël et les Amoréens.
\VS{15}Samuel fut juge en Israël tous les jours de sa vie.
\VS{16}Il allait tous les ans faire le tour de Béthel, de Guilgal et de Mitspa, et il jugeait Israël dans tous ces lieux.
\VS{17}Puis il revenait à Rama, où était sa maison ; et là il jugeait Israël, et il y bâtit un autel à Yahweh.
\TextTitle{[Israël veut un roi]}
\Chap{8}
\VerseOne{}Lorsque Samuel devint vieux, il établit ses fils juges sur Israël.
\VS{2}Son fils premier-né s’appelait Joël, et le second Abija ; ils jugeaient à Beer-Schéba.
\VS{3}Mais ses fils ne marchèrent pas dans ses voies, ils s’en détournèrent pour les profits acquis par la violence ; ils recevaient des présents et violaient la justice.
\VS{4}C'est pourquoi tous les anciens d'Israël s'assemblèrent, et vinrent auprès de Samuel à Rama.
\VS{5}Ils lui dirent : Voici, tu es devenu vieux, et tes fils ne suivent pas tes voies ; maintenant, établis sur nous un roi pour nous juger comme il y en a chez toutes les nations.
\TextTitle{[Prière de Samuel et réponse de Yahweh]}
\VS{6}Samuel fut affligé de ce qu'ils lui avaient dit : Etablis sur nous un roi pour nous juger. Et Samuel pria Yahweh.
\VS{7}Yahweh dit à Samuel : Obéis à la voix du peuple dans tout ce qu'il te dira, car ce n'est pas toi qu'ils ont rejeté, mais c'est moi qu'ils ont rejeté, afin que je ne règne plus sur eux.
\VS{8}Ils agissent à ton égard comme ils ont agi depuis le jour où je les ai fait monter hors d'Egypte jusqu’à ce jour ; ils m’ont abandonné, pour servir d'autres dieux.
\VS{9}Maintenant donc, obéis à leur voix ; mais ne manque pas de les avertir, en leur déclarant comment le roi qui régnera sur eux, les traitera.
\TextTitle{[Avertissements aux enfants d'Israël qui demandent un roi]}
\VS{10}Ainsi Samuel dit toutes les paroles de Yahweh, au peuple qui lui avait demandé un roi.
\VS{11}Il leur dit donc : Voici comment vous traitera le roi qui régnera sur vous. Il prendra vos fils et les mettra sur ses chars et parmi ses cavaliers, afin qu’ils courent devant son char ;
\VS{12}il en établira des chefs de mille, et des chefs de cinquante, pour labourer ses terres, pour récolter ses moissons, et pour fabriquer ses armes de guerre et l’équipement de ses chars.
\VS{13}Il prendra aussi vos filles pour en faire des parfumeuses, des cuisinières, et des boulangères.
\VS{14}Il prendra ce qu’il y a de meilleur parmi vos champs, vos vignes et vos oliviers, et il les donnera à ses serviteurs.
\VS{15}Il prélèvera la dîme de ce que vous aurez semé et de ce que vous aurez vendangé, et il la donnera à ses eunuques, et à ses serviteurs.
\VS{16}Il prendra vos serviteurs et vos servantes, l'élite de vos jeunes gens, vos ânes, et les emploiera à ses ouvrages.
\VS{17}Il prélèvera la dîme de vos troupeaux, et vous serez ses esclaves.
\VS{18}En ce jour-là, vous crierez à cause du roi que vous vous serez choisi, mais Yahweh ne vous exaucera pas.
\VS{19}Mais le peuple refusa d’écouter la voix de Samuel, et ils dirent : Non ! Mais il y aura un roi sur nous.
\VS{20}Nous serons aussi comme toutes les nations ; et notre roi nous jugera, il sortira devant nous, et il conduira nos guerres.
\VS{21}Samuel entendit donc toutes les paroles du peuple, et les rapporta à Yahweh.
\VS{22}Et Yahweh dit à Samuel : Obéis à leur voix, et établis un roi sur eux. Et Samuel dit aux hommes d'Israël : Allez-vous-en chacun dans sa ville.
\TextTitle{[Dieu leur donne un roi : Saül]}
\Chap{9}
\VerseOne{}Il y avait un homme de Benjamin, nommé Kis, fort et vaillant, fils d Abiel, fils de Tseror, fils de Becorath, fils d'Aphiach, fils d'un Benjamite.
\VS{2}Il avait un fils nommé Saül, jeune et beau, et aucun des enfants d'Israël n’était plus beau que lui, des épaules en haut, il dépassait tout le peuple.
\VS{3}Les ânesses de Kis, père de Saül, s’égarèrent; et Kis dit à Saül, son fils : Prends maintenant avec toi un des serviteurs et lève-toi, et va chercher les ânesses.
\VS{4}Il passa donc par la montagne d'Ephraïm et traversa le pays de Schalischa ; mais ils ne les trouvèrent pas ; puis ils passèrent par le pays de Schaalim, mais elles n'y étaient pas ; ils passèrent ensuite par le pays de Benjamin, mais ils ne les trouvèrent pas.
\VS{5}Quand ils furent arrivés dans le pays de Tsuph, Saül dit à son serviteur qui était avec lui : Viens, et retournons, de peur que mon père oublie les ânesses, et s’inquiète pour nous.
\VS{6}Le serviteur lui dit : Voici, je te prie, il y a dans cette ville un homme de Dieu, qui est un homme très honoré ; tout ce qu'il déclare ne manque pas d’arriver ; allons y maintenant, peut-être nous renseignera-t-il sur le chemin que nous devons prendre.
\VS{7}Et Saül dit à son serviteur : Mais si nous y allons, que porterons-nous à l'homme de Dieu, nous n’avons plus de provisions, et nous n'avons aucun présent pour l'homme de Dieu ? Qu’est-ce que nous avons ?
\VS{8}Le serviteur reprit la parole et dit à Saül : Voici j'ai encore entre mes mains le quart d'un sicle d'argent, et je le donnerai à l'homme de Dieu, et il nous indiquera notre chemin.
\VS{9}Autrefois en Israël quand on allait consulter Dieu, on se disait l'un à l'autre : Venez, allons vers le voyant ! Car le prophète, s'appelait autrefois le voyant.
\VS{10}Saül dit à son serviteur : Tu as bien dit ; viens, allons ! Et ils s'en allèrent dans la ville où était l'homme de Dieu.
\VS{11}Et comme ils montaient à la ville, ils trouvèrent de jeunes filles qui sortaient pour puiser de l'eau, et ils leur dirent : Le voyant n'est-il pas ici ?
\VS{12}Elles leur répondirent, et dirent : Il y est, le voilà devant toi ; hâte-toi maintenant, car il est venu aujourd'hui à la ville, parce qu'il y a aujourd'hui un sacrifice pour le peuple sur le haut lieu.
\VS{13}Quand vous entrerez dans la ville, vous le trouverez avant qu'il monte au haut lieu pour manger ; car le peuple ne mangera pas jusqu'à ce qu'il soit venu, parce qu'il doit bénir le sacrifice ; après quoi, les conviés mangeront. Montez donc maintenant, car vous le trouverez aujourd'hui.
\VS{14}Ils montèrent donc à la ville. Comme ils entraient dans la ville, Samuel, qui sortait pour monter au haut lieu, les rencontra.
\VS{15}Or, un jour avant l’arrivée de Saül, Yahweh avait fait une révélation à Samuel, en disant :
\VS{16}Demain, à cette même heure, je t'enverrai un homme du pays de Benjamin, et tu l'oindras pour être le conducteur de mon peuple d'Israël. Il délivrera mon peuple de la main des Philistins ; car j'ai regardé mon peuple parce que son cri est venu jusqu'à moi.
\VS{17}Et dès que Samuel eut aperçu Saül, Yahweh lui dit : Voici l'homme dont je t'ai parlé ; c'est lui qui dominera sur mon peuple.
\VS{18}Et Saül s'approcha de Samuel au milieu de la porte, et dit : Indique-moi je te prie, où est la maison du voyant.
\VS{19}Et Samuel répondit à Saül, et dit : Je suis le voyant. Monte devant moi au haut lieu, et vous mangerez aujourd'hui avec moi. Je te laisserai partir demain, et je te dirai tout ce que tu as sur le cœur.
\VS{20}Mais quant aux ânesses que tu as perdues il y a trois jours, ne t'en inquiète pas, parce qu'elles ont été retrouvées. Et vers qui tend tout le désir d’Israël ? N’est-ce pas vers toi, et vers toute la maison de ton père ?
\VS{21}Saül répondit : Ne suis-je pas de Benjamin, l’une des moindres tribus d'Israël, et ma famille n'est-elle pas la plus petite de toutes tribus de Benjamin ? Pourquoi m’as-tu tenu de tels discours ?
\VS{22}Samuel prit Saül et son serviteur, et les fit entrer dans la salle, et les plaça à la tête des conviés, qui étaient environ trente hommes.
\VS{23}Et Samuel dit au cuisinier : Apporte la portion que je t'ai donnée, en te disant : Mets-la à part.
\VS{24}Le cuisinier prit l’épaule, et ce qui l’entoure, et il la servit à Saül. Et Samuel dit : Voici ce qui a été réservé, mets-le devant toi, et mange, car il t’a été gardé expressément pour cette heure, lorsque j'ai résolu de convier le peuple ; et Saül mangea avec Samuel ce jour-là.
\VS{25}Puis ils descendirent du haut lieu dans la ville, et Samuel parla avec Saül sur le toit.
\VS{26}Puis ils se levèrent de bon matin ; et, dès l’aurore, Samuel appela Saül sur le toit, et lui dit : Lève-toi, et je te laisserai aller. Saül donc se leva, et ils sortirent tous deux dehors, lui et Samuel.
\VS{27}Et comme ils descendaient à l’extrémité de la ville, Samuel dit à Saül : Dis au serviteur de passer devant nous, et le serviteur passa devant. Arrête-toi maintenant, afin que je te fasse entendre la parole de Dieu.
\TextTitle{[Samuel oint Saül comme roi]}
\Chap{10}
\VerseOne{}Or, Samuel prit une fiole d'huile, qu’il répandit sur la tête de Saül. Il l’embrassa, et lui dit : Yahweh ne t'a-t-il pas oint, pour être le conducteur de son héritage?
\VS{2}Aujourd’hui, après m’avoir quitté, tu trouveras deux hommes près du sépulcre de Rachel, sur la frontière de Benjamin à Tseltsach, qui te diront : Les ânesses que tu étais allé chercher sont retrouvées ; et voici, ton père ne pense plus aux ânesses, mais il s’inquiète pour vous, disant : Que dois-je faire à propos de mon fils ?
\VS{3}En allant plus loin, tu arriveras au chêne de Thabor, où tu seras rencontré par trois hommes qui montent vers Dieu, à Béthel, et l'un porte trois chevreaux, l'autre trois pains, et l'autre une outre de vin.
\VS{4}Ils te demanderont comment tu te portes, et ils te donneront deux pains, que tu recevras de leurs mains.
\VS{5}Après cela tu arriveras à Guibea-Elohim, où se trouve une garnison des Philistins. Et il arrivera qu’en entrant dans la ville, tu rencontreras une troupe de prophètes descendant du haut lieu, précédés du luth, du tambourin, de la flûte, et de la harpe, et qui prophétisent.
\VS{6}Alors l'Esprit de Yahweh te saisira, et tu prophétiseras avec eux, et tu seras changé en un autre homme.
\VS{7}Et quand ces signes te seront arrivés, fais avec force ce que tu trouveras, car Dieu est avec toi.
\VS{8}Puis tu descendras devant moi à Guilgal, et voici, je descendrai vers toi pour offrir des holocaustes, et des sacrifices d’offrande de paix, tu m'attendras là sept jours, jusqu'à ce que je vienne, et que je te déclare ce que tu devras faire.
\VS{9}Aussitôt que Saül eut tourné le dos pour se séparer de Samuel, Dieu changea son cœur, et tous ces signes s’accomplir le même jour.
\VS{10}Quand ils arrivèrent à Guibea, voici une troupe de prophètes vint à sa rencontre. L'Esprit de Dieu le saisit, et il prophétisa au milieu d'eux.
\VS{11}Tous ceux qui le connaissaient de longue date, le virent prophétiser avec les prophètes. Il se dirent l'un à l'autre : Qu'est-il arrivé au fils de Kis ? Saül est-il aussi parmi les prophètes ?
\VS{12}Un homme répondit : Et qui est leur père ? De là le proverbe : Saül est-il aussi parmi les prophètes ?
\VS{13}Lorsqu’il eut cessé de prophétiser, il se rendit au haut lieu.
\VS{14}L'oncle de Saül dit à Saül et à son serviteur : Où êtes-vous allés ? Et il répondit : Chercher les ânesses, mais ne les trouvant pas nous sommes allés vers Samuel.
\VS{15}Et l’oncle de Saül dit : Déclare-moi, je te prie, ce que vous a dit Samuel.
\VS{16}Saül répondit à son oncle : Il nous a assuré que les ânesses étaient retrouvées ; mais il ne lui déclara rien concernant la royauté dont Samuel lui avait parlé.
\VS{17}Samuel convoqua le peuple devant Yahweh, à Mitspa.
\VS{18}Et il dit aux enfants d'Israël : Ainsi parle Yahweh, le Dieu d'Israël : J'ai fait monter Israël hors d'Egypte, et je vous ai délivrés de la main des Egyptiens, et de la main de tous les royaumes qui vous opprimaient.
\VS{19}Mais aujourd'hui, vous avez rejeté votre Dieu, celui qui vous a délivrés de tous vos malheurs, et de vos afflictions, et vous avez dit : Non, établis-nous un roi. Présentez-vous donc maintenant, devant Yahweh, par tribus, et par familles.
\VS{20}Ainsi Samuel fit approcher toutes les tribus d'Israël ; et la tribu de Benjamin fut désignée.
\VS{21}Après il fit approcher la tribu de Benjamin selon ses familles ; et la famille de Matri fut désignée; puis Saül fils de Kis fut désigné, on le chercha, mais on ne le trouva pas.
\VS{22}On consulta de nouveau Yahweh : Est-il encore venu quelqu’un ici ? Yahweh répondit : Il est caché parmi les bagages.
\VS{23}Ils coururent donc le chercher, et il se présenta au milieu du peuple, et il était plus grand que tout le peuple, depuis les épaules en haut.
\VS{24}Et Samuel dit à tout le peuple : Voyez-vous celui que Yahweh a choisi, il n'y a personne dans tout le peuple qui soit semblable à lui. Et le peuple poussa des cris de joie, et dit : Vive le roi !
\VS{25}Alors Samuel fit connaître au peuple les règles de la royauté, et les écrivit dans un livre, qu’il déposa devant Yahweh. Puis Samuel renvoya le peuple, chacun dans sa maison.
\VS{26}Saül aussi s'en alla chez lui à Guibea. Il fut accompagné par des vaillants hommes dont Dieu avait touché le cœur.
\VS{27}Mais il y eut des fils de Bélial\FTNT{1 S. 2:12.} qui dirent : Comment celui-ci nous délivrerait-il ? Et ils le méprisèrent, et ne lui apportèrent pas de présent. Mais Saül fit le sourd.
\TextTitle{[Saül vainqueur des Ammonites]}
\Chap{11}
\VerseOne{}Nachasch, l’Ammonite, vint et assiégea Jabès, en Galaad. Les habitants de Jabès dirent à Nachasch : Traite alliance avec nous et nous te servirons.
\VS{2}Mais Nachasch, l’Ammonite, leur répondit : Je traiterai avec vous à la condition que je vous crève à tous l’œil droit, et que je mette cet opprobre sur tout Israël.
\VS{3}Les anciens de Jabès lui dirent : Donne-nous sept jours de trêve, et nous enverrons des messagers dans tout le territoire d'Israël, et s'il n'y a personne qui nous délivre, nous nous rendrons à toi.
\VS{4}Les messagers arrivèrent à Guibea de Saül, et dirent ces paroles devant le peuple. Tout le peuple éleva sa voix, et pleura.
\VS{5}Et voici, Saül revenait des champs derrière ses bœufs, et il dit : Qu'est-ce qu'a ce peuple pour pleurer ainsi ? Et on lui raconta ce qu'avaient dit ceux de Jabès.
\VS{6}Et l'Esprit de Dieu saisit Saül, lorsqu'il entendit ces paroles, et sa colère s’enflamma fortement.
\VS{7}Il prit une paire de bœufs, et les coupa en morceaux qu’il envoya dans tous le territoire d'Israël, par des messagers, en disant : Les boeufs de tous ceux qui ne sortiront pas pour suivre Saül et Samuel, seront traités de la même manière. Et la frayeur de Yahweh tomba sur le peuple, et ils sortirent comme un seul homme.
\VS{8}Saül en fit la revue à Bézek ; les enfants d'Israël étaient trois cents mille et ceux de Juda trente mille.
\VS{9}Puis, ils dirent aux messagers qui étaient venus : Vous parlerez ainsi à ceux de Jabès en Galaad : Vous serez délivrés demain, quand le soleil sera dans sa force. Les messagers rapportèrent donc cela à ceux de Jabès, qui s'en réjouirent ;
\VS{10}et ils dirent aux Ammonites : Demain nous nous rendrons à vous, et vous nous traiterez selon votre bon plaisir.
\VS{11}Le lendemain, Saül disposa le peuple en trois corps. Ils entrèrent dans le camp des Ammonites à la veille du matin, et ils les battirent jusqu’à la chaleur du jour. Ceux qui échappèrent furent dispersés si bien qu'il n'en resta pas deux ensemble.
\TextTitle{[Le peuple reconnait Saül comme roi]}
\VS{12}Le peuple dit à Samuel : Qui est-ce qui dit : Saül régnera-t-il sur nous ? Donnez-nous ces hommes-là, et nous les ferons mourir.
\VS{13}Saül répondit : Personne ne sera mis à mort en ce jour, car Yahweh a délivré Israël aujourd'hui .
\VS{14}Et Samuel dit au peuple : Venez, allons à Guilgal, et nous y renouvellerons la royauté.
\VS{15}Et tout le peuple se rendit à Guilgal, et là, ils établirent Saül pour roi devant Yahweh, à Guilgal. Et ils offrirent des sacrifices d’offrande de paix, devant Yahweh ; Saül et tous ceux d'Israël se réjouirent beaucoup.
\TextTitle{[Le peuple atteste l'intégrité de Samuel]}
\Chap{12}
\VerseOne{}Alors Samuel dit à tout Israël : Voici, j'ai obéi à votre voix dans tout ce que vous m'avez dit, et j'ai établi un roi sur vous.
\VS{2}Et maintenant, voici le roi qui marchera devant vous. Car moi, je suis vieux et tout blanc, et voici, mes fils aussi sont avec vous ; pour moi j'ai marché devant vous, depuis ma jeunesse jusqu’à ce jour.
\VS{3}Me voici, témoignez contre moi, devant Yahweh, et devant son oint. De qui ai-je pris le bœuf ? Et de qui ai-je pris l'âne ? Qui ai-je opprimé ? Qui ai-je traité durement ? Et de la main de qui ai-je reçu des présents, afin de fermer les yeux sur lui ? Et je vous le rendrai.
\VS{4}Et ils répondirent : Tu ne nous as pas opprimés, tu ne nous as pas traités durement et tu n'as rien reçu de la main de personne.
\VS{5}Il leur dit encore : Yahweh est témoin contre vous, et son oint aussi est témoin aujourd'hui, que vous n'avez rien trouvé entre mes mains. Et ils répondirent : Il en est témoin.
\TextTitle{[Exhortation de Samuel]}
\VS{6}Alors Samuel dit au peuple : Yahweh est celui qui a établi Moïse et Aaron, et qui a fait monter vos pères hors du pays d'Egypte.
\VS{7}Maintenant donc, présentez-vous, et je vous jugerai devant Yahweh sur tous les bienfaits que Yahweh vous a accordés, à vous et à vos pères.
\VS{8}Après que Jacob fut entré en Egypte, vos pères crièrent à Yahweh, et Yahweh envoya Moïse et Aaron qui firent sortir vos pères hors d'Egypte, et les firent habiter en ce lieu.
\VS{9}Mais ils oublièrent Yahweh, leur Dieu, et il les livra entre les mains de Sisera, chef de l'armée de Hatsor, et entre les mains des Philistins, et entre les mains du roi de Moab, qui leur firent la guerre.
\VS{10}Ils crièrent encore à Yahweh, et dirent : Nous avons péché ; car nous avons abandonné Yahweh, et nous avons servi les Baals et les Astartés. Maintenant donc, délivre-nous de la main de nos ennemis, et nous te servirons.
\VS{11}Et Yahweh envoya Jerubbaal, Bedan, Jephthé et Samuel, et il vous délivra de la main de tous vos ennemis d'alentour, et vous demeurâtes en sécurité.
\VS{12}Mais voyant que Nachasch, roi des fils d’Ammon, marchait contre vous, vous m'avez dit : Non ! Mais un roi régnera sur nous. Alors que Yahweh, votre Dieu, était votre Roi.
\VS{13}Maintenant donc, voici le roi que vous avez choisi, que vous avez demandé, et voici Yahweh l'a établi roi sur vous.
\VS{14}Si vous craignez Yahweh, si vous le servez, et obéissez à sa voix, et que vous n’êtes pas rebelles au commandement de Yahweh, alors vous et votre roi qui règne sur vous, vous serez sous la conduite de Yahweh, votre Dieu.
\VS{15}Mais si vous n'obéissez pas à la voix de Yahweh, et si vous êtes rebelles au commandement de Yahweh, la main de Yahweh sera aussi contre vous, comme elle a été contre vos pères.
\VS{16}Maintenant, préparez-vous, et voyez cette grande chose que Yahweh va opérer sous vos yeux.
\VS{17}N'est-ce pas aujourd'hui la moisson des blés ? Je crierai à Yahweh, et il enverra des tonnerres et de la pluie. Sachez alors et voyez combien vous avez mal agi aux yeux de Yahweh en demandant un roi.
\VS{18}Alors Samuel cria à Yahweh, et Yahweh envoya des tonnerres et de la pluie ce même jour. Tout le peuple eut une grande crainte de Yahweh, et de Samuel.
\VS{19}Et tout le peuple dit à Samuel : Prie Yahweh, ton Dieu, pour tes serviteurs, afin que nous ne mourions pas ; car nous avons ajouté à nos péchés, celui d'avoir demandé un roi.
\VS{20}Alors Samuel dit au peuple : Ne craignez pas ! Vous avez fait tout ce mal, néanmoins ne vous détournez pas de Yahweh, mais servez Yahweh de tout votre cœur.
\VS{21}Ne vous en détournez pas car vous iriez après des choses de néant, qui ne vous apportent ni profit ni délivrance ; puisque ce sont des choses de néant.
\VS{22}Car Yahweh n’abandonne pas son peuple, pour l'amour de son grand Nom, car Yahweh a résolu de faire de vous son peuple.
\VS{23}Et pour moi, Dieu me garde de pécher contre Yahweh, et de cesser de prier pour vous ! Je vous enseignerai le bon et le droit chemin.
\VS{24}Craignez seulement Yahweh, et servez-le en vérité, de tout votre cœur ; car vous avez vu les choses magnifiques qu'il a faites pour vous.
\VS{25}Mais si vous persévérez à faire le mal, vous serez détruits vous et votre roi.
\TextTitle{[Saül pèche contre Yahweh en offrant l'holocauste]}
\Chap{13}
\VerseOne{}Saül régna un an sur Israël et après deux années, 
\VS{2} Saül choisit trois mille hommes d'Israël, deux mille avec lui à Micmasch, et sur la montagne de Béthel, et mille étaient avec Jonathan à Guibea de Benjamin. Il renvoya le reste du peuple, chacun à sa tente.
\VS{3}Et Jonathan battit le poste des Philistins qui était à Guéba, et les Philistins en furent informés ; et Saül fit sonner le shofar dans tout le pays, en disant : Que les Hébreux écoutent !
\VS{4}Tout Israël apprit donc que Saül avait battu le poste des Philistins, et Israël se rendit odieux aux Philistins. Et le peuple fut convoqué auprès de Saül, à Guilgal.
\VS{5}Les Philistins s'assemblèrent pour combattre Israël, ayant trente mille chars et six mille cavaliers ; et le peuple était aussi nombreux que le sable au bord de la mer, tant il était en grand nombre ; ils allèrent prendre position à Micmasch, à l'orient de Beth-Aven.
\VS{6}Les hommes d'Israël furent pris d’une grande angoisse ; car ils étaient oppressés, c'est pourquoi le peuple se cacha dans les cavernes, dans les buissons, dans les rochers, dans les tours et dans des citernes.
\VS{7}Les Hébreux passèrent le Jourdain pour aller au pays de Gad, et de Galaad. Saül était encore à Guilgal, aussi tout le peuple effrayé le rejoignit.
\VS{8}Il attendit sept jours selon le terme fixé par Samuel ; mais Samuel ne venait pas à Guilgal et le peuple se dispersait.
\VS{9}Et Saül dit : Amenez-moi un holocauste et des sacrifices d’offrande de paix, et il offrit l'holocauste.
\VS{10}Comme il achevait d'offrir l'holocauste, Samuel arriva, et Saül sortit au-devant de lui pour le saluer.
\VS{11}Et Samuel lui dit : Qu'as-tu fait ? Saül répondit : Lorsque j’ai vu que le peuple se dispersait, que tu ne venais pas au jour fixé, et que les Philistins étaient assemblés à Micmasch ;
\VS{12}J'ai dit : Les Philistins descendront maintenant contre moi à Guilgal, et je n'ai pas supplié Yahweh ! Je me suis maîtrisé un temps, mais j'ai fini par offrir l'holocauste.
\VS{13}Samuel répondit à Saül : C’est en insensé que tu as agi, car tu n'as pas gardé le commandement que Yahweh, ton Dieu t'avait donné ; car Yahweh aurait maintenu à jamais ta royauté sur Israël.
\VS{14}Et maintenant ta royauté ne subsistera pas ; Yahweh s'est choisi un homme selon son cœur, et Yahweh l’a destiné à être le chef de son peuple parce que tu n'as pas respecté le commandement de Yahweh.
\VS{15}Puis Samuel se leva, et monta de Guilgal à Guibea de Benjamin. Et Saül passa en revue le peuple qui se trouvait avec lui, qui fut d'environ six cents hommes.
\VS{16}Or Saül vint s’établir avec son fils Jonathan, et le peuple qui était sous ses ordres à Guibea de Benjamin, et les Philistins étaient campés à Micmasch.
\VS{17}Les Philistins sortirent du camp en trois divisions pour ravager ; l'une de ces divisions prit le chemin d’Ophra, vers le pays de Schual ;
\VS{18}l'autre division prit le chemin de Beth-Horon ; et la troisième prit le chemin de la frontière qui regarde vers la vallée de Tseboïm, du côté du désert.
\VS{19}Or dans tout le pays d'Israël, il ne se trouvait aucun forgeron ; car les Philistins avaient dit : Empêchons les Hébreux de faire des épées ou des lances.
\VS{20}C'est pourquoi chaque homme descendait vers les Philistins, pour aiguiser son soc, son hoyau, sa hache, et sa bêche ;
\VS{21}lorsque le tranchant des bêches, des hoyaux, des tridents, et des haches était émoussé, même pour redresser un aiguillon.
\VS{22}De sorte qu’il arriva qu’au jour du combat, nul n’avait d’ épée ni de lance dans toute l’armée qui était avec Saül et Jonathan ; si ce n’est Saül lui-même et Jonathan, son fils.
\VS{23}Un poste de Philistins s’établit au passage de Micmasch.
\TextTitle{[Courage de Jonathan]}
\Chap{14}
\VerseOne{}Jonathan, fils de Saül, dit un jour au garçon qui portait ses armes : Viens et allons jusqu’au poste de garde des Philistins qui est au-delà de ce lieu-là ; mais il ne dit rien à son père.
\VS{2}Saül se tenait à l'extrémité de Guibea sous un grenadier, à Migron, entouré d'environ six cents hommes.
\VS{3}Achija, fils d'Achithub, frère d'I-Kabnod, fils de Phinées, fils d'Eli, sacrificateur de Yahweh à Silo, portait l'éphod ; et le peuple ignorait que Jonathan s'en était allé.
\VS{4}Or entre les passages par lesquels Jonathan voulait arriver au poste de garde des Philistins, il y avait une dent de rocher d’un côté, et une dent de rocher de l’autre ; l'une s’appelait Botsets et l'autre Séné.
\VS{5}L'une de ces dents était située du côté nord vis-à-vis de Micmasch ; et l'autre, du côté sud vis-à-vis de Guéba.
\VS{6}Jonathan dit au garçon qui portait ses armes : Viens, poursuivons jusqu’au poste de garde de ces incirconcis ; peut-être que Yahweh agira-t-il pour nous : car on ne saurait empêcher Yahweh de délivrer avec peu ou beaucoup de gens.
\VS{7}Et celui qui portait ses armes lui dit : Fais tout ce que tu as dans le cœur, vas-y , voici je serai avec toi où tu voudras.
\VS{8}Et Jonathan lui dit : Allons vers ces hommes, et montrerons-nous à eux.
\VS{9}S'ils nous disent : Attendez jusqu'à ce que nous venions à vous, alors nous resterons sur place, et nous ne monterons pas vers eux.
\VS{10}Mais s'ils disent : Montez vers nous, nous irons ; car Yahweh les aura livrés entre nos mains. Que cela soit pour nous un signe.
\VS{11}Ils se montrèrent donc tous deux au poste de garde des Philistins, et les Philistins dirent : Voici, les Hébreux sortent des trous où ils s'étaient cachés.
\VS{12}Et ceux du poste de garde dirent à Jonathan, et à celui qui portait ses armes : Montez vers nous, nous avons quelque chose à vous apprendre. Alors Jonathan dit à celui qui portait ses armes : Monte avec moi ; car Yahweh les a livrés entre les mains d'Israël.
\VS{13}Et Jonathan monta en s’aidant des mains et des pieds ; celui qui portait ses armes le suivit. Puis ceux du poste de garde tombèrent sous les coups de Jonathan, et celui qui portait ses armes les tuait à sa suite.
\VS{14}Dans cette première victoire, Jonathan et celui qui portait ses armes, tuèrent environ vingt hommes, dans un espace d'environ une moitié d’un arpent de terre.
\VS{15}Et il y eut un grand effroi au camp, à la campagne, et parmi tout le peuple ; le poste de garde aussi, et ceux qui avaient ravagé furent effrayés et le pays fut tellement troublé que cela fut comme une frayeur de Dieu.
\TextTitle{[Victoire d'Israël]}
\VS{16}Les sentinelles de Saül qui étaient à Guibea de Benjamin virent que la multitude se dispersait et courait éperdue.
\VS{17}Alors Saül dit au peuple qui était avec lui : Faites donc la revue et voyez qui s’en est allé du milieu de nous. Ils firent donc la revue, et voici Jonathan n'y était pas, ni celui qui portait ses armes.
\VS{18}Et Saül dit à Achija : Fais approcher l'arche de Dieu ; - car l'arche de Dieu était en ce jour-là avec les enfants d'Israël.-
\VS{19}Pendant que Saül parlait au sacrificateur, le tumulte venant du camp des Philistins augmentait de plus en plus ; et Saül dit au sacrificateur : Retire ta main !
\VS{20}Saül et tout le peuple se rassemblèrent ; et vinrent au champ de bataille, les Philistins tournaient les épées les uns contre les autres, la confusion était extrême.
\VS{21}Les Hébreux, qui étaient montés auparavant dans le camp des Philistins et qui étaient dispersés, se joignirent aux Israëlites qui étaient avec Saül et Jonathan.
\VS{22}Et tous les Israëlites qui s'étaient cachés dans la montagne d'Ephraïm, ayant appris que les Philistins s'enfuyaient, les poursuivirent aussi pour les combattre.
\VS{23}Ce jour-là, Yahweh délivra Israël, et le combat s’étendit jusqu'à Beth-Aven.
\TextTitle{[Jonathan épargné des conséquences du vœu de Saül]}
\VS{24}Les hommes d'Israël furent épuisés cette journée-là. Mais Saül avait fait jurer le peuple, en disant : Maudit soit l'homme qui prendra de la nourriture avant le soir, avant que je me sois vengé de mes ennemis ! Et le peuple n’avait pris de pain.
\VS{25}Tout le peuple arriva dans une forêt, où il y avait du miel à la surface du sol.
\VS{26}Lorsque le peuple entra dans la forêt, il vit le miel qui coulait, mais nul ne porta la main à sa bouche ; car le peuple craignait le serment.
\VS{27}Or Jonathan n'avait pas entendu son père lorsqu'il avait fait faire le serment au peuple, il étendit le bout du bâton qu'il avait à la main, le trempa dans un rayon de miel et porta sa main à sa bouche et ses yeux furent éclaircis.
\VS{28}Alors quelqu'un du peuple lui dit : Ton père a fait jurer le peuple en disant : Maudit soit l'homme qui mangera aujourd'hui quelque chose ; quoique le peuple soit très fatigué.
\VS{29}Et Jonathan dit : Mon père trouble le peuple ; voyez comment mes yeux sont éclaircis après avoir goûté un peu de ce miel ;
\VS{30}combien plus si le peuple s’était aujourd'hui restauré du butin de ses ennemis ; la défaite des Philistins n'en aurait-elle pas été plus considérable ?
\VS{31}En ce jour-là donc ils frappèrent les Philistins de Micmasch à Ajalon. Le peuple était très fatigué.
\VS{32}Puis il se jeta sur le butin, il prit des brebis, des bœufs, et des veaux, et les égorgea sur la terre ; et le peuple les mangeait avec le sang.
\VS{33}On le rapporta à Saül, en disant : Voici, le peuple pèche contre Yahweh, en mangeant, avec le sang ; et il dit : Vous avez péché, roulez-moi ici une grosse pierre.
\VS{34}Allez parmi le peuple, ajouta-t-il, et dites à chacun d’amener son bœuf et ses brebis ; vous les égorgerez ici, vous les mangerez, et vous ne pécherez plus contre Yahweh, en mangeant avec le sang. Et chacun amena cette nuit-là son bœuf à la main, et ils les égorgèrent.
\VS{35}Saül bâtit un autel à Yahweh ; ce fut le premier autel qu'il bâtit à Yahweh.
\VS{36}Puis Saül dit : Descendons et poursuivons de nuit les Philistins, afin de les piller jusqu'au matin, et n’en laissons pas un homme de reste. Ils lui répondirent : Fais tout ce qui te semble bon ; mais le sacrificateur dit : Approchons-nous d’abord de Dieu.
\VS{37}Saül consulta donc Dieu : Descendrai-je à la poursuite des Philistins ? Les livreras-tu entre les mains d'Israël ? Mais il ne lui répondit pas.
\VS{38}Et Saül dit : Approchez ici, vous tous les chefs du peuple, recherchez et voyez par qui ce péché est arrivé aujourd'hui.
\VS{39}Car Yahweh est vivant, lui qui délivre Israël, quand il s’agirait de mon fils Jonathan, il en mourrait. Mais du peuple, personne ne répondit.
\VS{40}Puis il dit à tout Israël : Mettez-vous d'un côté, et nous serons de l'autre, moi et mon fils, Jonathan. Le peuple répondit à Saül : Fais ce qui te semble bon.
\VS{41}Et Saül dit à Yahweh, le Dieu d'Israël : Fais connaître la vérité. Jonathan et Saül furent désignés; et le peuple fut écarté.
\VS{42}Et Saül dit : Jetez le sort entre moi et Jonathan, mon fils. Et Jonathan fut désigné.
\VS{43}Alors Saül dit à Jonathan : Déclare-moi ce que tu as fait. Et Jonathan lui déclara et dit : Il est vrai que j'ai goûté un peu de miel avec le bout de mon bâton que j'avais à la main ; me voici, je mourrai.
\VS{44}Et Saül dit : Que Dieu agisse à mon égard comme il le veut, si tu ne meurs pas, Jonathan.
\VS{45}Mais le peuple dit à Saül : Jonathan qui a accompli cette grande délivrance en Israël, mourrait-il ? Garde-toi bien ! Yahweh est vivant, il ne tombera pas à terre un seul des cheveux de sa tête ; car c’est avec Dieu qu’il a agi en ce jour. Le peuple délivra Jonathan de la mort.
\VS{46}Saül renonça à poursuivre les Philistins, qui regagnèrent leur pays.
\TextTitle{[Les guerres sous le règne de Saül]}
\VS{47}Après que Saül eut pris possession de la royauté sur Israël, il fit la guerre de tous côtés contre ses ennemis, Moab, les enfants d’Ammon, Edom, les rois de Tsoba et les Philistins ; partout où il se tournait, il était vainqueur.
\VS{48}Il manifesta sa puissance en frappant Amalek et délivra Israël de la main de ceux qui le pillaient.
\VS{49}Les fils de Saül étaient Jonathan, Jischvi et Malkischua ; et quant aux noms de ses deux filles, le nom de l'aînée était Mérab, et la plus jeune, Mical.
\VS{50}Et le nom de la femme de Saül était Achinoam, fille d'Achimaats ; et le nom du chef de son armée était Abner, fils de Ner, oncle de Saül.
\VS{51}Kis, père de Saül, et Ner père d'Abner étaient fils d'Abiel.
\VS{52}La guerre contre les Philistins fut violente durant toute la vie de Saül ; et chaque fois que Saül remarquait un homme fort et vaillant, il le prenait auprès de lui.
\TextTitle{[Saül désobéit à Yahweh]}
\Chap{15}
\VerseOne{}Samuel dit à Saül : Yahweh m'a envoyé pour t'oindre afin que tu sois roi sur son peuple, sur Israël ; maintenant donc, écoute les paroles de Yahweh.
\VS{2}Ainsi parle Yahweh des armées : Je me rappelle de ce qu'Amalek a fait à Israël, comment il s'opposa à lui sur le chemin, à sa sortie d'Egypte.
\VS{3}Va maintenant, et frappe Amalek, et dévouez par interdit tout ce qui lui appartient ; ne l’épargne pas, mais fais mourir hommes et femmes, enfants et nourrissons, bœufs et menu bétail, chameaux et ânes.
\VS{4}Saül donc convoqua le peuple, et en fit la revue à Thelaïm, il y avait deux cent mille hommes de pied, et de dix mille hommes de Juda.
\VS{5}Et Saül marcha jusqu'à la ville d’Amalek, et mit une embuscade dans la vallée.
\VS{6}Et Saül dit aux Kéniens : Allez retirez-vous, séparez-vous des Amalécites, de peur que je ne vous détruise avec eux ; car vous avez agi avec bonté envers tous les enfants d'Israël, quand ils montèrent d'Egypte. Et les Kéniens se séparèrent des Amalécites.
\VS{7}Et Saül frappa les Amalécites depuis Havila jusqu'à Schur, qui est face à l'Egypte.
\VS{8}Il fit passer tout le peuple au fil de l'épée, le dévouant par interdit ; mais il épargna Agag, roi d'Amalek.
\VS{9}Saül et le peuple épargnèrent Agag, les meilleures brebis, les meilleurs boeufs, les bêtes grasses, les agneaux, ce qu’il y avait de meilleur ; ils ne voulurent pas les dévouer par interdit ; détruisant seulement tout ce qui est chétif et méprisable.
\VS{10}Alors la parole de Yahweh fut adressée à Samuel en disant :
\VS{11}Je me repens d'avoir établi Saül pour roi car il s’est détourné de moi et n'a pas exécuté mes paroles. Samuel fut très irrité, et il cria à Yahweh toute la nuit.
\TextTitle{[Samuel annonce à Saül que Yahweh le rejette]}
\VS{12}Puis Samuel se leva de bon matin pour aller rencontrer Saül. On lui rapporta que Saül venu à Carmel, s'est érigé un monument, puis s'en est retourné, pour enfin descendre à Guilgal.
\VS{13}Samuel se rendit auprès de Saül, et Saül lui dit : Sois béni de Yahweh ! J’ai exécuté la parole de Yahweh.
\VS{14}Samuel dit : Quel est donc ce bêlement de brebis qui parvient à mes oreilles, et ce mugissement de bœufs que j'entends ?
\VS{15}Et Saül répondit : Ils les ont amenés de chez les Amalécites ; car le peuple a épargné les meilleures brebis et les meilleurs bœufs, pour les sacrifier à Yahweh, ton Dieu ; et nous avons détruit le reste, nous l’avons dévoué par interdit.
\VS{16}Samuel dit à Saül : Laisse-moi te déclarerai ce que Yahweh m'a dit cette nuit ; et il lui répondit : Parle !
\VS{17}Samuel dit : N'est-il pas vrai que, quand tu étais petit à tes yeux, tu as été fait chef des tribus d'Israël, et Yahweh t'a oint pour roi sur Israël ?
\VS{18}Yahweh t'avait envoyé dans cette expédition, et t'avait dit : Va, et détruis ces pécheurs, les Amalécites, et fais-leur la guerre, jusqu'à ce qu'ils soient exterminés.
\VS{19}Pourquoi n'as-tu pas obéi à la voix de Yahweh, tu t'es jeté sur le butin, et as fait ce qui déplaît à Yahweh ?
\VS{20}Et Saül répondit à Samuel : J'ai pourtant obéi à la voix de Yahweh, et je suis allé par le chemin par lequel Yahweh m'a envoyé, et j'ai amené Agag, roi des Amalécites, et j'ai dévoué les Amalécites, par interdit.
\VS{21}Mais le peuple a pris des brebis, des bœufs, du butin, comme prémices de ce qui devait être dévoué, pour le sacrifier à Yahweh, ton Dieu à Guilgal.
\VS{22}Samuel répondit : Yahweh prend-il plaisir aux holocaustes et aux sacrifices, autant qu’à l’obéissance à sa voix ? Voici, l'obéissance vaut mieux que les sacrifices, et l’observation de sa parole vaut mieux que la graisse des béliers.
\VS{23}Car la rébellion est un péché autant que la divination, et la résistance ne l’est pas moins que l’idolâtrie et les théraphim. Puisque tu as rejeté la parole de Yahweh, il te rejette aussi afin que tu ne sois plus roi.
\VS{24}Et Saül répondit à Samuel : J'ai péché parce que j'ai transgressé le commandement de Yahweh, ainsi que tes paroles ; car je craignais le peuple et j'ai obéi à sa voix.
\VS{25}Mais maintenant, je te prie, pardonne-moi mon péché, et reviens avec moi, que je me prosterne devant Yahweh.
\VS{26}Et Samuel dit à Saül : Je n’irai pas avec toi ; parce que tu as rejeté la parole de Yahweh, Yahweh te rejette afin que tu sois plus roi d’Israël.
\VS{27}Comme Samuel se détournait pour s'en aller, Saül le saisit par le pan de son manteau qui se déchira.
\VS{28}Alors Samuel lui dit : Yahweh déchire aujourd'hui le royaume d'Israël de dessus toi, et le donne à un autre, qui est meilleur que toi.
\VS{29}En effet, le Puissant d'Israël ne ment pas, il ne se repent pas ; car il n'est pas un homme pour se repentir.
\VS{30}Et Saül répondit : J'ai péché ; mais honore-moi maintenant, je te prie, en présence des anciens de mon peuple, et en présence d'Israël, et reviens avec moi, et je me prosternerai devant Yahweh ton Dieu.
\VS{31}Samuel retourna et suivit Saül ; et Saül se prosterna devant Yahweh.
\VS{32}Puis Samuel dit : Amenez-moi Agag, roi d'Amalek. Et Agag s’avança vers lui, faisant le gracieux ; car Agag disait : certainement l'amertume de la mort est passée.
\VS{33}Mais Samuel dit : Comme ton épée a privé les femmes de leurs enfants, ainsi ta mère entre les femmes sera privée d'enfants. Et Samuel mit Agag en pièces devant Yahweh à Guilgal.
\VS{34}Puis il s'en alla à Rama ; et Saül monta dans sa maison à Guibea de Saül.
\VS{35}Et Samuel n'alla plus voir Saül jusqu'au jour de sa mort ; car Samuel pleurait sur Saül, de ce que Yahweh s'était repenti d'avoir établi Saül, roi sur Israël.
\TextTitle{[Yahweh envoie Samuel à Bethléhem pour oindre David]}
\Chap{16}
\VerseOne{}Yahweh dit à Samuel : Jusqu'à quand mèneras-tu deuil sur Saül, vu que je l'ai rejeté, afin qu'il ne règne plus sur Israël ? Remplis ta corne d'huile, et viens ; je t’enverrai chez Isaï, Bethléhémite ; car je me suis pourvu d'un de ses fils pour roi.
\VS{2}Et Samuel dit : Comment irai-je ? Car Saül l’apprendra et il me tuera. Et Yahweh répondit : Tu emmèneras avec toi une jeune vache du troupeau ; et tu diras : Je suis venu pour sacrifier à Yahweh.
\VS{3}Et tu inviteras Isaï au sacrifice, et je te ferai savoir ce que tu auras à faire, et tu m'oindras celui que je te dirai.
\VS{4}Samuel fit donc comme Yahweh lui avait dit, et il alla à Bethléhem. Les anciens de la ville tout effrayés accoururent au-devant de lui et lui dirent : Ton arrivée annonce-t-elle la paix ?
\VS{5}Et il répondit : Soyez en paix ; je suis venu pour sacrifier à Yahweh, sanctifiez-vous, et venez avec moi au sacrifice. Il fit sanctifier aussi Isaï et ses fils, et les invita au sacrifice.
\VS{6}A son entrée, il remarqua Eliab, et se dit : L'oint de Yahweh est certainement devant lui.
\VS{7}Mais Yahweh dit à Samuel : Ne prête pas attention à son apparence, ni à la hauteur de sa taille, car je l'ai rejeté ; Yahweh ne considère pas ce que l'homme considère ; car l'homme considère ce que voient ses yeux ; mais Yahweh regarde au cœur.
\VS{8}Isaï appela Abinadab, et le fit passer devant Samuel, et Samuel dit : Yahweh n'a pas non plus choisi celui-ci.
\VS{9}Isaï fit passer Schamma, et Samuel dit : Yahweh n'a pas non plus choisi celui-ci.
\VS{10}Ainsi Isaï fit passer ses sept fils devant Samuel et Samuel dit à Isaï : Yahweh n’a pas choisi ceux-ci.
\VS{11}Puis Samuel dit à Isaï : Sont-ce là tous tes garçons? Et il dit : Il reste encore le plus jeune, seulement, il fait paître les brebis. Alors Samuel dit à Isaï : Envoie-le chercher ; car nous ne retournerons pas avant qu’il ne soit venu ici.
\VS{12}Il le fit donc venir. Il était roux, avec de beaux yeux et une belle apparence. Et Yahweh dit à Samuel : Lève-toi, et oins-le ; car c'est lui !
\VS{13}Alors Samuel prit la corne d'huile, et l'oignit au milieu de ses frères ; et depuis ce jour-là l'Esprit de Yahweh saisit David. Et Samuel se leva, et s'en alla à Rama.
\TextTitle{[David chez Saül]}
\VS{14}L'Esprit de Yahweh se retira de Saül, et un mauvais esprit\FTNT{Saül a été frappé d’un esprit d’égarement (2 Th. 2:9-12)} envoyé par Yahweh le terrifiait.
\VS{15}Les serviteurs de Saül lui dirent : Voici, un mauvais esprit envoyé de Dieu te tourmente.
\VS{16}Que le roi notre seigneur parle ! Tes serviteurs sont devant toi. Ils chercheront un homme qui sache jouer de la harpe ; et quand le mauvais esprit envoyé par Dieu sera sur toi, il jouera de sa main, et tu seras soulagé.
\VS{17}Saül répondit à ses serviteurs : Trouvez-moi un homme qui sache bien jouer et amenez-le-moi.
\VS{18}L'un des serviteurs répondit : Voici, j'ai vu l’un des fils d'Isaï, le Bethléhémite, qui sait jouer des instruments, il est fort et vaillant, c’est un guerrier qui parle bien, bel homme, et Yahweh est avec lui.
\VS{19}Alors Saül envoya des messagers à Isaï, pour lui dire : Envoie-moi David, ton fils, qui est avec les brebis.
\VS{20}Isaï prit un âne, qu’il chargea de pain, et une outre de vin, et un jeune chevreau, et les envoya par David, son fils, à Saül.
\VS{21}David arrivé chez Saül, se présenta devant lui ; et Saül l'aima beaucoup, et il lui servit à porter ses armes.
\VS{22}Saül fit dire à Isaï : Je te prie que David demeure à mon service ; car il a trouvé grâce devant moi.
\VS{23}Il arrivait donc que quand le mauvais esprit envoyé de Dieu, était sur Saül, David prenait la harpe, et en jouait de sa main ; et Saül en était soulagé, parce que le mauvais esprit se retirait de lui.
\TextTitle{[Goliath défie Israël]}
\Chap{17}
\VerseOne{}Les Philistins réunirent leurs armées pour faire la guerre, et ils se rassemblèrent à Soco, qui est de Juda ; et ils campèrent entre Soco et Azéka, à Ephès-Dammim.
\VS{2}Saül et ceux d'Israël se rassemblèrent aussi ; et ils campèrent dans la vallée du chêne, et ils se mirent en ordre de bataille contre les Philistins.
\VS{3}Les Philistins étaient sur une montagne d’un côté, et les Israëlites sur une montagne de l’autre côté ; de sorte que la vallée les séparait.
\VS{4}Il sortit du camp des Philistins un homme qui se présentait entre les deux armées, il s’appelait Goliath, de la ville de Gath, haut de six coudées et d'un empan.
\VS{5}Il avait un casque d'airain sur sa tête, et était armé d'une cuirasse à écailles pesant cinq mille sicles d'airain.
\VS{6}Il avait aussi des jambières d'airain, et un javelot d'airain entre ses épaules.
\VS{7}Le bois de sa lance était comme une ensouple d'un tisserand, et le fer de sa lance pesait six cents sicles de fer. Celui qui portait son bouclier marchait devant lui.
\VS{8}Il se présenta donc, et cria aux troupes d'Israël rangées en bataille, il leur disait : Pourquoi sortez-vous pour vous ranger en bataille ? Ne suis-je pas Philistin, et n'êtes-vous pas esclaves de Saül ? Choisissez l'un d'entre vous, et qu'il descende contre moi.
\VS{9}S’il peut me battre et qu'il me tue, nous serons vos esclaves ; mais si j'ai l'avantage sur lui, et que je le tue, vous serez nos esclaves, et vous nous serez asservis.
\VS{10}Le Philistin disait : Je jette un défi en ce jour aux troupes rangées d'Israël : Donnez-moi un homme, et nous combattrons ensemble.
\VS{11}Saül et tous les Israëlites ayant entendu les paroles du Philistin furent épouvantés et saisis d’une grande frayeur.
\VS{12}David, était le fils d'un homme Ephratien, de Bethléhem de Juda, nommé Isaï, qui avait huit fils, et qui du temps de Saül, était vieux et mis au rang de personnes de qualité.
\VS{13}Et les trois fils aînés d'Isaï avaient suivi Saül à la guerre. Les noms de ses trois fils qui s'en étaient allés à la guerre, étaient Eliab, le premier-né ; Abinadab, le second ; et Schamma, le troisième.
\VS{14}David était le plus jeune, et les trois plus grands suivaient Saül.
\VS{15}David allait et revenait d'auprès de Saül, pour paître les brebis de son père à Bethléhem.
\VS{16}Et le Philistin s'approchant le matin et le soir, se présenta pendant quarante jours.
\TextTitle{[David veut combattre contre Goliath]}
\VS{17}Isaï dit à David, son fils : Prends maintenant pour tes frères un épha de ce blé rôti, et ces dix pains, et porte-les promptement au camp, à tes frères.
\VS{18}Tu porteras aussi ces dix fromages au chef de leur millier, tu t’informeras du bien-être de tes frères et tu m'en apporteras des nouvelles sûres.
\VS{19}Or Saül était avec eux et les hommes d'Israël, combattant les Philistins dans la vallée du chêne.
\VS{20}David se leva de bon matin, et laissa les brebis aux soins d’un gardien ; puis ayant pris sa charge, s'en alla, comme son père Isaï le lui avait ordonné. Lorsqu’il arriva au lieu où était le camp, l'armée sortait pour se ranger en bataille, et on poussait des cris de guerre.
\VS{21}Car les Israëlites et les Philistins se rangèrent armée contre armée.
\VS{22}Alors David se déchargea de son bagage, le laissant entre les mains de celui qui gardait le bagage, et courut vers les rangs de l’armée. Aussitôt arrivé, il demanda à ses frères s'ils se portaient bien.
\VS{23}Et comme il parlait avec eux, le Philistin de Gath, nommé Goliath, sortit des rangs de l'armée des Philistins, se présenta entre les deux armées et proféra les mêmes paroles qu'il avait proférées auparavant et David les entendit.
\VS{24}A la vue de cet homme, tous ceux d'Israël s'enfuirent devant lui, saisis d’une grande frayeur.
\VS{25}Et les Israélites disaient : Avez-vous vu s’avancer cet homme ? Il est monté pour jeter un défi à Israël, mais si quelqu'un le tue, le roi le comblera de richesses, et lui donnera sa fille, et affranchira la maison de son père en Israël.
\VS{26}Alors David parla aux personnes qui étaient là avec lui, en disant : Quel bien fera-t-on à l'homme qui frappera ce Philistin, et qui ôtera l'opprobre de dessus Israël ? Car qui est ce Philistin, cet incirconcis, pour insulter l’armée du Dieu vivant ?
\VS{27}Et le peuple lui répéta ces mêmes paroles et lui dit : C'est le bien qu'on fera à l'homme qui l'aura tué.
\VS{28}Et quand Eliab son frère aîné entendit qu'il parlait à ces personnes, sa colère s'enflamma contre David, et il lui dit : Pourquoi es-tu descendu, et à qui as-tu laissé ce peu de brebis au désert ? Je connais ton orgueil et la malice de ton cœur, car tu es descendu pour voir la bataille.
\VS{29}Et David répondit : Qu'ai-je donc fait ? Ne puis-je pas parler ainsi ?
\VS{30}Puis il se détourna de lui vers un autre, et lui posa les mêmes questions ; et le peuple lui répondit comme la première fois.
\VS{31}Les paroles que David avait dites furent entendues et rapportées devant Saül qui le fit venir.
\VS{32}David dit à Saül : Que personne ne perde courage à cause de ce Philistin ! Ton serviteur ira et se battra contre lui.
\VS{33}Mais Saül dit à David : Tu ne peux aller te battre contre ce Philistin, car tu n'es qu'un enfant, et il est un homme de guerre depuis sa jeunesse.
\VS{34}David répondit à Saül : Ton serviteur faisait paître les brebis de son père, quand un lion ou un ours venait emporter une brebis du troupeau,
\VS{35}je le poursuivais, je le frappais, et j’arrachais la brebis de la gueule, s’il se jetait sur moi, je le saisissais par la mâchoire, je le frappais, et je le tuais.
\VS{36}Ton serviteur a tué et le lion, et l’ours ; et ce Philistin, cet incirconcis, sera comme l'un d'eux ; car il a déshonoré l’armée du Dieu vivant.
\VS{37}David dit encore : Yahweh qui m'a délivré de la griffe du lion, et de la patte de l'ours, me délivrera de la main de ce Philistin. Alors Saül dit à David : Va, et que Yahweh soit avec toi.
\TextTitle{[David tue Goliath]}
\VS{38}Saül fit revêtir David de ses vêtements, et lui mit son casque d'airain sur sa tête, et lui fit endosser une cuirasse.
\VS{39}Puis David ceignit l'épée par-dessus ses vêtements, et voulut marcher, car il n’avait pas encore essayé. Et David dit à Saül : Je ne saurais marcher ainsi, je ne l’ai jamais essayé. Et il s’en débarrassa.
\VS{40}Alors il prit en main son bâton, et se choisit dans le torrent cinq pierres bien polies, et les mit dans sa mallette de berger et dans sa poche, puis sa fronde en main, il s'approcha du Philistin.
\VS{41}Le Philistin aussi s'approcha lentement de David, précédé de l'homme qui portait son bouclier.
\VS{42}Le Philistin regarda, et lorsqu’il vit David, il le méprisa, car ce n'était qu'un jeune garçon, roux et beau de figure.
\VS{43}Le Philistin dit à David : Suis-je un chien, pour que tu viennes contre moi avec des bâtons ? Et le Philistin maudit David par ses dieux.
\VS{44}Le Philistin ajouta : Viens vers moi et je donnerai ta chair aux oiseaux du ciel, et aux bêtes des champs.
\VS{45}Et David dit au Philistin : Tu marches contre moi avec l'épée, la lance, et le javelot ; mais moi, je marche contre toi au Nom de Yahweh des armées, le Dieu de l’armée d'Israël, que tu as blasphémé.
\VS{46}Aujourd'hui Yahweh te livrera entre mes mains, je t’abattrai, je te couperai la tête ; aujourd'hui je donnerai les cadavres du camp des Philistins aux oiseaux du ciel, et aux animaux de la terre ; et toute la terre saura qu'Israël a un Dieu.
\VS{47}Et toute cette assemblée saura que Yahweh ne délivre pas par l'épée ni par la lance ; car la victoire est à Yahweh, qui vous livrera entre nos mains.
\VS{48}Voyant le Philistin se mettre en mouvement et s'approcher de lui, David s’élança, et courut au milieu du champ de bataille en direction du Philistin.
\VS{49}Il mit la main à sa mallette, prit une pierre, et la lança avec sa fronde ; il frappa le Philistin au front, tellement que la pierre s'enfonça dans son front, il tomba le visage contre terre.
\VS{50}Ainsi avec une fronde et une pierre, David fut plus fort que le Philistin, il le frappa, et le tua, sans avoir une épée à la main.
\VS{51}Alors David courut, se jeta sur le Philistin, prit son épée, la tira de son fourreau, le tua, et lui coupa la tête. Les Philistins, voyant que leur héros était mort, prirent la fuite.
\VS{52}Alors les hommes d'Israël et de Juda se levèrent, et poussèrent des cris de joie, et poursuivirent les Philistins, jusqu'à la vallée, et jusqu'aux portes d’Ekron. Les Philistins blessés à mort tombèrent dans le chemin de Schaaraïm, jusqu'à Gath, et jusqu'à Ekron.
\VS{53}Et les enfants d'Israël revinrent de la poursuite des Philistins, et pillèrent leurs camps.
\VS{54}David prit la tête du Philistin et la porta à Jérusalem, et il mit aussi dans sa tente les armes du Philistin.
\VS{55}Quand Saül vit David sortant à la rencontre du Philistin, il dit à Abner, chef de l'armée : Abner, de qui ce jeune homme est-il le fils ? Abner répondit : Que ton âme vive, ô roi! Je n'en sais rien.
\VS{56}Le roi lui dit : Informe-toi de qui ce jeune garçon est fils.
\VS{57}Et quand David fut de retour après avoir tué le Philistin, Abner le prit, et le mena devant Saül. David avait la tête du Philistin à la main.
\VS{58}Et Saül lui dit : Jeune garçon, de qui es-tu fils ? David répondit : Je suis fils d'Isaï Bethléhémite, ton serviteur.
\TextTitle{[Jonathan et David font alliance]}
\Chap{18}
\VerseOne{}Dès que David eut achevé de parler à Saül, l'âme de Jonathan fut attachée à l'âme de David, et Jonathan l'aima comme son âme.
\VS{2}Ce jour-là donc Saül le retint, et ne lui permit plus de retourner à la maison de son père.
\VS{3}Alors Jonathan fit alliance avec David, parce qu'il l'aimait comme son âme.
\VS{4}Jonathan se dépouilla du manteau qu'il portait, et le donna à David, avec ses habits, jusqu'à son épée, son arc, et sa ceinture.
\TextTitle{[Saül jaloux veut tuer David]}
\VS{5}David envoyé par Saül, réussissait partout où il allait, de sorte que Saül l'établit sur son armée, et il plaisait à tout le peuple, même aux serviteurs de Saül.
\VS{6}Or quand ils rentraient, lors du retour de David après qu’il eut tué le Philistin, des femmes sortirent de toutes les villes d'Israël, en chantant et dansant devant le roi Saül, avec des tambourins, des triangles et en poussant des cris de joie.
\VS{7}Les femmes chantaient, se répondant les unes aux autres, en disant : Saül a frappé ses mille, et David ses dix mille.
\VS{8}Saül fut très irrité, car cette parole lui déplut. Il dit : Elles en ont donné dix mille à David, et à moi, mille ! Il ne lui manque plus que le royaume.
\VS{9}Depuis ce jour-là, Saül regardait David d’un mauvais œil.
\VS{10}Dès le lendemain, le mauvais esprit envoyé de Dieu saisit Saül qui prophétisait dans sa maison, et David joua de sa main, comme les autres jours, et Saül avait une lance à la main.
\VS{11}Saül jeta sa lance, se disant : Je frapperai David, contre le mur ; mais David l’évita deux fois.
\VS{12}Saül craignait la présence de David, parce que Yahweh était avec David, et qu'il s'était retiré de Saül.
\VS{13}C'est pourquoi Saül éloigna David de lui, et l'établit chef de mille ; et David allait et venait devant le peuple.
\VS{14}David réussissait dans tout ce qu'il entreprenait, car Yahweh était avec lui.
\VS{15}Saül, voyant que David réussissait beaucoup, avait peur de sa présence.
\VS{16}Mais tout Israël et Juda aimaient David, parce qu'il allait et venait devant eux.
\TextTitle{[David épouse Mérab]}
\VS{17}Saül dit à David : Voici, je te donnerai Mérab, ma fille aînée pour femme ; sois pour moi un fils vaillant, et conduis les guerres de Yahweh ; car Saül disait : Que ma main ne le touche pas, mais que ce soit celle des Philistins.
\VS{18}David répondit à Saül : Qui suis-je, et quelle est ma vie, et la famille de mon père en Israël, pour que je devienne gendre du roi ?
\VS{19}Or, au temps où l’on devait donner Mérab fille de Saül à David, elle fut donnée pour femme à Adriel de Mehola.
\VS{20}Mais Mical, fille de Saül, aima David ; ce qu'on rapporta à Saül, et la chose lui plut.
\VS{21}Et Saül dit : Je la lui donnerai afin qu'elle soit pour lui un piège, et que par ce moyen la main des Philistins l’atteigne. Saül donc dit à David pour la seconde fois : Tu seras aujourd'hui mon gendre.
\VS{22}Et Saül ordonna à ses serviteurs de parler à David en secret, et de lui dire : Voici, le roi prend plaisir en toi, et tous ses serviteurs t'aiment ; sois donc maintenant gendre du roi.
\VS{23}Les serviteurs de Saül répétèrent toutes ces paroles à David, et David répondit : Pensez-vous qu’il soit facile de devenir le gendre du roi, moi qui suis un homme pauvre, et peu important ?
\VS{24}Et les serviteurs de Saül lui rapportèrent ce que David avait répondu.
\VS{25}Saül dit : Vous parlerez ainsi à David : Le roi ne désire pas de dot, mais cent prépuces de Philistins, afin d’être vengé de ses ennemis. Or Saül avait pour but de faire tomber David aux mains des Philistins.
\VS{26}Les serviteurs de Saül rapportèrent tous ces discours à David à qui il plut de devenir gendre du roi. Le temps n’était pas encore écoulé,
\VS{27}que David se leva, et s'en alla, lui et ses gens, et tua deux cents hommes parmi les Philistins ; il apporta leurs prépuces, et on les livra au complet au roi, afin qu'il devienne gendre du roi. Alors Saül lui donna pour femme Mical, sa fille.
\VS{28}Saül vit et comprit que Yahweh était avec David et Mical, fille de Saül l'aimait.
\VS{29}Saül craignait David de plus en plus, et devint son ennemi toute sa vie durant.
\VS{30}Les chefs des Philistins firent des incursions, mais chaque fois qu’ils sortaient, David remportait du succès mieux que tous les serviteurs de Saül et son nom devint célèbre.
\TextTitle{[David échappe aux assauts de Saül]}
\Chap{19}
\VerseOne{}Saül parla à Jonathan, son fils, et à tous ses serviteurs de faire mourir David.
\VS{2}Mais Jonathan, fils de Saül, avait une grande affection pour David. C'est pourquoi Jonathan le fit savoir à David, et lui dit : Saül, mon père, cherche à te faire mourir ; maintenant donc, tiens-toi sur tes gardes jusqu'au matin, demeure dans un lieu secret, et cache-toi.
\VS{3}Je me tiendrai auprès de mon père, je sortirai dans le champ où tu seras ; car je parlerai de toi à mon père ; je verrai ce qu'il en sera, et je te le rapporterai.
\VS{4}Jonathan parla favorablement de David à Saül, son père, et lui dit : Que le roi ne pèche pas contre son serviteur David, car il n'a pas péché contre toi ; au contraire, il a agi pour ton bien.
\VS{5}Car il a exposé sa vie, il a tué le Philistin, et Yahweh a opéré une grande délivrance pour tout Israël, tu l'as vu, et tu t'en es réjoui ; pourquoi donc pécherais-tu contre le sang innocent en faisant mourir David sans cause ?
\VS{6}Saül écouta la voix de Jonathan et jura : Yahweh est vivant, il ne mourra pas.
\VS{7}Alors Jonathan appela David, et lui répéta toutes ces choses. Jonathan l’introduisit auprès de Saül, et il fut à son service comme auparavant.
\VS{8}La guerre ayant recommencé, David se mit en campagne et frappa les Philistins, et leur infligea une grande défaite, de sorte qu'ils prirent la fuite.
\VS{9}Le mauvais esprit envoyé de Yahweh fut sur Saül, comme il était assis dans sa maison, ayant sa lance à la main, et David jouait de sa main.
\VS{10}Saül voulut frapper David avec sa lance contre le mur ; mais il se glissa de devant Saül, qui frappa le mur de la lance, David s'enfuit et s'échappa cette nuit-là.
\VS{11}Saül envoya des messagers à la maison de David pour le garder, et le faire mourir au matin. Mical, femme de David l’en informa, en disant : Si tu ne te sauves pas demain on te fera mourir.
\VS{12}Mical fit descendre David par une fenêtre, et ainsi il s'en alla et s'enfuit.
\VS{13}Ensuite Mical prit un théraphim, qu’elle plaça dans le lit ; elle mit une peau de chèvre à son chevet et l’enveloppa d'une couverture.
\VS{14}Lorsque Saül envoya des gens pour prendre David, elle dit : Il est malade.
\VS{15}Saül envoya encore des gens pour prendre David, en leur disant : Apportez-le-moi dans son lit, afin que je le fasse mourir.
\VS{16}Ces gens donc vinrent, et voici, un théraphim était au lit, et la peau de chèvre à son chevet.
\VS{17}Saül dit à Mical : Pourquoi m'as-tu trompé de la sorte, et as-tu laissé aller mon ennemi, de sorte qu'il s’est échappé ? Et Mical, répondit à Saül : Il m'a dit : Laisse-moi aller ou je te tue !
\VS{18}C’est ainsi que David prit la fuite et qu’il s’échappa. Il se rendit auprès de Samuel à Rama, et lui raconta tout ce que Saül lui avait fait. Puis il s'en alla avec Samuel, et ils demeurèrent à Najoth.
\VS{19}On le rapporta à Saül, en lui disant : Voici, David est à Najoth près de Rama.
\VS{20}Alors Saül envoya des gens pour s’emparer de David. Ils virent une assemblée de prophètes qui prophétisaient, et Samuel à leur tête, se tenait là. L'Esprit de Dieu saisit les envoyés de Saül, qui prophétisèrent aussi.
\VS{21}On le rapporta à Saül, qui envoya d'autres gens, et eux aussi prophétisèrent. Saül en envoya encore pour la troisième fois et ils prophétisèrent également.
\VS{22}Alors il alla lui-même à Rama. Arrivé à la grande citerne qui est à Sécou, il s'informa disant : Où sont Samuel et David ? Et on lui répondit : Ils sont à Najoth, près de Rama.
\VS{23}Il se dirigea vers Najoth près de Rama ; et l'Esprit de Dieu le saisit à son tour, et il continua son chemin en prophétisant, jusqu'à son arrivée à Najoth près de Rama.
\VS{24}Il se dépouilla lui aussi de ses vêtements et prophétisa devant Samuel ; et il se jeta à terre nu, tout ce jour-là et toute la nuit. C'est pourquoi on dit : Saül est-il aussi parmi les prophètes ?
\TextTitle{[David et Jonathan renouvellent leur serment]}
\Chap{20}
\VerseOne{}David s'enfuit de Najoth près de Rama. Il alla voir Jonathan et lui dit : Qu'ai-je fait ? Quelle est mon iniquité, et quel est mon péché devant ton père, pour qu'il en veuille à ma vie ?
\VS{2}Jonathan lui dit : Loin de là ! Tu ne mourras pas. Voici, mon père ne fait aucune chose, ni grande, ni petite, qu'il ne m’en informe ; pourquoi mon père me cacherait-il cette chose-là ? Il n'en est rien.
\VS{3}Alors David jurant, dit encore : Ton père sait certainement que j’ai trouvé grâce à tes yeux, et il aura dit : Que Jonathan ne sache rien de ceci, de peur qu'il n'en soit attristé ; mais Yahweh est vivant, et ton âme vit ! Qu'il n'y a qu'un pas entre moi et la mort.
\VS{4}Alors Jonathan dit à David : Que désires-tu que je fasse ? Et je le ferai pour toi.
\VS{5}Et David dit à Jonathan : Voici, c'est demain la nouvelle lune, et je devrais m'asseoir auprès du roi pour manger, laisse-moi donc aller et je me cacherai aux champs, jusqu'au troisième soir.
\VS{6}Si ton père me cherche, tu lui répondras : David m'a demandé la permission de courir Bethléhem sa ville, parce que toute sa famille fait un sacrifice annuel.
\VS{7}S'il dit ainsi : C’est bien ! Ton serviteur n’a rien à craindre. Mais s'il se met en colère, sache qu’il a résolu mon malheur.
\VS{8}Use donc de bonté envers ton serviteur, puisque tu as conclu une alliance avec ton serviteur devant Yahweh. S'il y a de l’iniquité en moi, tue-moi toi-même ; car pourquoi me mènerais-tu jusqu’à ton père ?
\VS{9}Jonathan lui dit : Loin de toi cette pensée ! Si je savais ta perte arrêtée dans la pensée de mon père, ne t’en informerais-je pas ?
\VS{10}David répondit à Jonathan : Qui m’avertira si la réponse que t'aura faite ton père est sévère ?
\VS{11}Et Jonathan dit à David : Viens et sortons dans les champs. Ils sortirent donc eux deux dans les champs.
\VS{12}Alors Jonathan dit à David : Par Yahweh, le Dieu d'Israël, je sonderai mon père demain, environ à cette heure ou après demain, et s’il est favorable envers David, et que je n'envoie personne vers toi pour t’en informer,
\VS{13}que Yahweh traite Jonathan dans toute sa rigueur ! Si mon père a résolu de te faire du mal, je t’en informerai, et je te laisserai aller, et tu t'en iras en paix, de sorte que Yahweh sera avec toi comme il a été avec mon père.
\VS{14}Si je vis encore, tu useras de la bonté de Yahweh envers moi, en sorte que je ne meure pas.
\VS{15}Ne retire jamais ta bonté de ma maison, pas même quand Yahweh retranchera tous les ennemis de David de dessus la surface de la terre.
\VS{16}Ainsi Jonathan traita alliance avec la maison de David, en disant : Que Yahweh tire vengeance des ennemis de David.
\VS{17}Jonathan se lia encore par serment à David pour l'amour qu'il lui portait ; car il l'aimait comme son âme.
\VS{18}Puis Jonathan lui dit : C'est demain la nouvelle lune, et on s'informera sur toi ; car ta place sera vide.
\VS{19}Le troisième jour au soir, tu descendras en hâte, jusqu’au fond du lieu où tu t’étais caché le jour de l’affaire et tu resteras près de la pierre d'Ezel.
\VS{20}Je tirerai trois flèches à côté de cette pierre, comme si je visais un but.
\VS{21}Et voici, j'enverrai un jeune homme, et je lui dirai : Va, trouve les flèches. Si je dis au jeune homme : Voici, les flèches sont au deçà de toi, prends-les ! Alors viens, car la paix est avec toi et tu n’as rien à craindre ; Yahweh est vivant.
\VS{22}Mais si je dis ainsi au jeune homme : Voici, les flèches sont au-delà de toi ; va-t'en, car Yahweh te renvoie.
\VS{23}Et quant à la parole que nous nous sommes donnée toi et moi ; voici, Yahweh est entre moi et toi à jamais.
\TextTitle{[Saül en colère contre Jonathan]}
\VS{24}David donc se cacha dans le champ. La nouvelle lune étant venue, le roi s'assit pour prendre son repas.
\VS{25}Et le roi s’assit à sa place, comme à l’ordinaire, sur son siège près du mur, Jonathan se leva, et Abner s'assit à côté de Saül ; mais la place de David resta vide.
\VS{26}Saül ne dit rien ce jour-là, car il se disait : Il lui est arrivé quelque chose ; il n'est pas pur, certainement il n'est pas pur.
\VS{27}Mais le lendemain, le second jour de la nouvelle lune, la place de David était encore vide. Et Saül dit à Jonathan, son fils : Pourquoi le fils d'Isaï n'a-t-il été ni hier ni aujourd'hui au repas ?
\VS{28}Et Jonathan répondit à Saül : David m'a instamment demandé la permission d’aller à Bethléhem.
\VS{29}Même il m'a dit : Je te prie, laisse-moi aller ; car notre famille fait un sacrifice dans la ville, et mon frère m'a ordonné de m'y trouver ; maintenant donc si je suis dans tes bonnes grâces, je te prie que j'y aille, afin que je voie mes frères. C'est pour cela qu'il n'est pas venu à la table du roi.
\VS{30}Alors la colère de Saül s'enflamma contre Jonathan et il lui dit : Fils perfide et rebelle, ne sais-je pas que tu as choisi le fils d'Isaï à ta honte et à la honte de ta mère ?
\VS{31}Car aussi longtemps que le fils d'Isaï sera vivant sur la terre, tu ne seras pas stable, ni toi, ni ta royauté ; c'est pourquoi maintenant amène-le-moi, car il est digne de mort.
\VS{32}Et Jonathan répondit à Saül son père, et lui dit : Pourquoi le ferait-on mourir ? Qu'a-t-il fait ?
\VS{33}Et Saül lança sa lance contre lui pour le frapper. Alors Jonathan reconnut que son père avait résolu la mort de David.
\VS{34}Jonathan se leva de table dans une ardente colère, et ne mangea pas le pain le deuxième jour de la nouvelle lune ; car il était affligé à cause de David, parce que son père l'avait insulté.
\VS{35}Le matin venu, Jonathan sortit dans les champs, au lieu convenu avec David, et il amena avec lui un petit garçon.
\VS{36}Et il dit à son garçon : Cours, trouve maintenant les flèches que je m'en vais tirer. Et le garçon courut, et Jonathan tira une flèche qui le dépassa.
\VS{37}Lorsque le garçon arriva au lieu où était la flèche que Jonathan avait tirée, Jonathan cria après lui, et lui dit : La flèche n'est-elle pas plus loin de toi ?
\VS{38}Jonathan cria encore après le garçon : Hâte-toi, ne t'arrête pas ; et le garçon ramassa les flèches, et revint vers son maître.
\VS{39}Le garçon ne savait rien de cette affaire ; seuls David et Jonathan le savaient.
\VS{40}Jonathan remit ses armes au garçon et lui dit : Va, porte-les à la ville.
\VS{41}Le garçon parti, David se leva du côté du midi, se jeta le visage contre terre et se prosterna à trois reprises. Ils s’embrassèrent et pleurèrent ensemble, David versa d’abondantes larmes.
\VS{42}Jonathan dit à David : Va en paix ; comme nous l’avons juré au nom de Yahweh, en disant : Que Yahweh soit entre moi et toi, entre ma postérité et ta postérité.
\VS{43}David donc se leva, s'en alla et Jonathan rentra dans la ville.
\TextTitle{[David s'enfuit]}
\Chap{21}
\VerseOne{}David se rendit à Nob, vers Achimélec le sacrificateur, qui tout effrayé courut au-devant de David, et lui dit : Pourquoi es-tu seul et n'y a-t-il personne avec toi ?
\VS{2}David répondit au sacrificateur Achimélec : Le roi m'a donné un ordre et m'a dit : Que personne ne sache rien de l'affaire pour laquelle je t'envoie, ni de l’ordre que je t'ai donné. J’ai donné rendez-vous à mes hommes en un certain lieu.
\VS{3}Maintenant donc qu'as-tu sous la main ? Donne-moi cinq pains ou ce qui se trouvera.
\VS{4}Le sacrificateur répondit à David et dit : Je n'ai pas de pain ordinaire sous la main, mais du pain sacré\FTNT{Mt. 12:4.} ; pourvu que tes gens se soient abstenus de femmes !
\VS{5}David répondit au sacrificateur : Il est vrai que depuis que je suis parti, il y a trois jours, les femmes ont été éloignées de nous, et les vases des serviteurs sont restés purs, et si c’est là un acte profane, à plus forte raison, il sera aujourd’hui sanctifié par les vases.
\VS{6}Alors le sacrificateur lui donna du pain sacré, car il n'y avait pas là d'autre pain que les pains de proposition qui avaient été ôtés de devant Yahweh, pour le remplacer par du pain chaud le jour où on l’avait pris.
\VS{7}Or il y avait là un homme d'entre les serviteurs de Saül, retenu ce jour-là devant Yahweh ; il s’appelait Doëg, un Edomite, le plus puissant des bergers de Saül.
\VS{8}David dit à Achimélec : Mais n'as-tu pas ici sous la main quelque lance, ou quelque épée ? Car je n’ai pas pris mon épée ni mes armes sur moi, parce que l’ordre du roi était pressant.
\VS{9}Et le sacrificateur dit : Voici l'épée de Goliath, le Philistin, que tu as tué dans la vallée du chêne, elle est enveloppée d'un drap, derrière l'éphod ; si tu veux la prendre pour toi, prends-la ; car il n'y en a pas ici d'autre que celle-là. Et David dit : Il n'y en a pas de pareille ; donne-la-moi.
\TextTitle{[David se rend à Gad]}
\VS{10}Alors David se leva, et s'enfuit ce jour-là, loin de Saül, et s'en alla vers Akisch, roi de Gath.
\VS{11}Et les serviteurs d'Akisch lui dirent : N'est-ce pas là David, roi du pays ? N’est-ce pas celui duquel on chantait et répondait en dansant : Saül a tué ses mille, et David ses dix mille ?
\VS{12}David mit ces paroles dans son cœur, et eut une grande crainte d'Akisch, roi de Gath.
\VS{13}Il se montra comme un insensé à leurs yeux, il agit devant eux comme un fou ; et il faisait des marques sur les battants des portes, et laissait couler sa salive sur sa barbe.
\VS{14}Et Akisch dit à ses serviteurs : Vous voyez que cet homme a perdu la raison. Pourquoi me l'avez-vous amené ?
\VS{15}Est-ce que je manque de fous, pour que vous m’ameniez celui-ci pour faire l'insensé devant moi ? Faudrait-il qu’il entre dans ma maison ?
\TextTitle{[David se réfugie dans la caverne d'Adullam]
\\(1 Ch. 12:16-18)}
\Chap{22}
\VerseOne{}David partit de là, et se sauva dans la caverne d'Adullam. Ses frères et toute la maison de son père l’ayant appris, ils descendirent vers lui.
\VS{2}Tous ceux qui étaient dans la détresse, qui avaient des créanciers, et qui avaient le cœur rempli d'amertume, se rassemblèrent auprès de lui, et il devint leur chef. Ainsi se joignirent à lui environ quatre cents hommes.
\VS{3}David s'en alla de là à Mitspé dans le pays de Moab. Il dit au roi de Moab : Permets, je te prie à mon père et ma mère de se retirer chez vous jusqu'à ce que je sache ce que Dieu fera de moi.
\VS{4}Il les amena devant le roi de Moab, et ils demeurèrent chez lui, tout le temps que David fut dans cette forteresse.
\VS{5}Gad, le prophète, dit à David : Ne demeure pas dans cette forteresse, va-t'en, et entre dans le pays de Juda. David donc s'en alla, et vint dans la forêt de Héreth.
\TextTitle{[Saül tue les sacrificateurs]}
\VS{6}Saül apprit qu'on avait découvert David et ses gens. Or Saül était assis sous le tamaris, à Guibea, sur la hauteur ; il avait sa lance à la main, et tous ses serviteurs se tenaient devant lui.
\VS{7}Saül dit à ses serviteurs qui se tenaient près de lui : Ecoutez Benjamites ! Le fils d'Isaï vous donnera-t-il à vous tous des champs et des vignes ? Vous établira-t-il tous chefs de mille, et chefs de cent ?
\VS{8}Pourquoi avez-vous tous conspiré contre moi, et n'y a-t-il personne qui m’informe de l’alliance que mon fils a faite avec le fils d'Isaï ? Pourquoi n’y a-t-il personne de vous qui souffre à mon sujet et qui m'avertisse que mon fils a suscité mon serviteur contre moi pour me dresser des embûches, comme il le fait aujourd'hui.
\VS{9}Alors Doëg, l’Edomite, qui était établi sur les serviteurs de Saül, répondit et dit : J'ai vu le fils d'Isaï venir à Nob, auprès d’Achimélec, fils d'Achithub.
\VS{10}Il a consulté Yahweh pour lui, il lui a donné des vivres ainsi que l'épée de Goliath, le Philistin.
\VS{11}Alors le roi envoya appeler Achimélec, le sacrificateur, fils d'Achithub, la maison de son père, et les sacrificateurs qui étaient à Nob ; et ils vinrent tous vers le roi.
\VS{12}Saül dit : Ecoute, fils d'Achithub ! Il répondit : Me voici, mon seigneur.
\VS{13}Saül lui dit : Pourquoi avez-vous conspiré contre moi, toi et le fils d'Isaï ? Pourquoi lui as-tu donné du pain et une épée, et as-tu consulté Dieu pour lui, pour qu'il s'élève contre moi comme il le fait aujourd'hui, pour me dresser des embûches ?
\VS{14}Achimélec répondit au roi et dit : Entre tous tes serviteurs y en a-t-il un comme David, fidèle et gendre du roi, qui est parti sur ton commandement, et honoré dans ta maison ?
\VS{15}Est-ce d’aujourd'hui que j’ai commencé à consulter Dieu pour lui ? Loin de moi ! Que le roi n’impute aucun tort à son serviteur, à personne de la maison de mon père ; car ton serviteur ne sait rien de tout cela, petite ou grande.
\VS{16}Le roi lui dit : Tu mourras, Achimélec, toi et toute la maison de ton père.
\VS{17}Alors le roi dit aux coureurs qui se tenaient devant lui : Approchez-vous et mettez à mort les sacrificateurs de Yahweh ; car leur main est avec David, parce qu'ils savaient qu'il s'enfuyait, et qu'ils ne m'ont pas averti. Mais les serviteurs du roi ne voulurent pas étendre la main pour frapper les sacrificateurs de Yahweh.
\VS{18}Le roi dit à Doëg : Approche-toi, et frappe les sacrificateurs. Et Doëg, l’Edomite se tourna, et frappa les sacrificateurs ; il tua en ce jour-là quatre-vingt-cinq hommes qui portaient l'éphod de lin.
\VS{19}Il frappa encore du tranchant de l’épée Nob, ville des sacrificateurs ; hommes et femmes, enfants et nourrissons, bœufs, ânes, et brebis, tombèrent sous le tranchant de l'épée.
\VS{20}Toutefois un des fils d'Achimélec, fils d'Achithub, qui s’appelait Abiathar, se sauva, et s'enfuit auprès de David.
\VS{21}Abiathar rapporta à David que Saül avait tué les sacrificateurs de Yahweh.
\VS{22}David dit à Abiathar : Je savais bien ce jour-là, que Doëg, l’Edomite qui était présent, ne manquerait pas d’informer Saül. Je suis la cause de la mort de toutes les personnes de la maison de ton père.
\VS{23}Reste avec moi, ne crains rien, car celui qui cherche ma vie, cherche la tienne ; avec moi, tu seras bien gardé.
\TextTitle{[David libère Kéïla]}
\Chap{23}
\VerseOne{}On fit ce rapport à David, en disant : Voici, les Philistins font la guerre à Keïla, et pillent les aires.
\VS{2}David consulta Yahweh\FTNT{La clé du succès de David était Yahweh. Il consultait régulièrement Dieu avant de s’engager dans une guerre (Ps. 60:14).} en disant : Irai-je, et frapperai-je ces Philistins ? Et Yahweh répondit à David : Va, et tu frapperas les Philistins, et tu délivreras Keïla.
\VS{3}Les gens de David lui dirent : Voici, nous avons peur ici en Juda ; que sera-ce donc quand nous irons à Keïla contre les troupes des Philistins ?
\VS{4}C'est pourquoi David consulta encore Yahweh, et Yahweh lui répondit et dit : Lève-toi, descends à Keïla, car je livre les Philistins entre tes mains.
\VS{5}Alors David s'en alla avec ses gens à Keïla, et combattit contre les Philistins, et emmena leur bétail, et fit un grand carnage ; ainsi David délivra les habitants de Keïla.
\VS{6}Lorsque Abiathar, fils d'Achimélec, s'était enfui vers David à Keïla, il avait en main l'éphod.
\VS{7}On rapporta à Saül que David était venu à Keïla ; Saül dit : Dieu l'a livré entre mes mains car il s'est enfermé en entrant dans une ville qui a des portes et des barres.
\VS{8}Saül convoqua tout le peuple pour aller à la guerre, afin de descendre à Keïla, et d'assiéger David et ses gens.
\VS{9}David ayant eu connaissance des mauvais desseins de Saül à son égard, dit au sacrificateur Abiathar : Apporte l'éphod.
\VS{10}Puis David dit : Yahweh, Dieu d'Israël ! Ton serviteur apprend que Saül cherche à venir à Keïla, pour détruire la ville à cause de moi.
\VS{11}Les chefs de Keïla me livreront-ils entre ses mains ? Saül descendra-t-il comme ton serviteur l'a entendu dire ? Yahweh, Dieu d'Israël ! Je te prie, révèle-le à ton serviteur. Et Yahweh répondit : Il descendra.
\VS{12}David dit encore : Les chefs de Keïla me livreront-ils, moi et mes gens, entre les mains de Saül ? Et Yahweh répondit : Ils te livreront.
\TextTitle{[David échappe encore à Saül]}
\VS{13}Alors David se leva avec ses gens au nombre d’environ six cents hommes ; et ils sortirent de Keïla, et s'en allèrent où ils purent. On rapporta à Saül que David s'était sauvé de Keïla, c'est pourquoi il cessa sa marche.
\VS{14}David resta au désert, dans des lieux forts, et il se tint sur la montagne au désert de Ziph. Et Saül le cherchait tous les jours, mais Dieu ne le livra pas entre ses mains.
\VS{15}David sachant que Saül était sorti pour attenter à sa vie, se tint au désert de Ziph, dans la forêt.
\VS{16}Alors Jonathan, fils de Saül, se leva, et s'en alla dans la forêt vers David, et fortifia son autorité en Dieu.
\VS{17}Et lui dit : Ne crains pas, car Saül mon père ne t’atteindra pas, mais tu régneras sur Israël, et moi je serai le second après toi ; et même Saül mon père le sait bien.
\VS{18}Ils firent tous les deux, alliance devant Yahweh ; et David resta dans la forêt, mais Jonathan retourna dans sa maison.
\VS{19}Or les Ziphiens montèrent auprès de Saül à Guibea, et lui dirent : David ne se tient-il pas caché parmi nous dans des lieux forts, dans la forêt, sur la colline de Hakila, qui est au midi du désert ?
\VS{20}Maintenant donc, ô roi ! Puisque tout le désir de ton âme est de descendre, descends, et ce sera à nous de le livrer entre les mains du roi.
\VS{21}Et Saül dit : Que Yahweh vous bénisse de ce que vous avez eu pitié de moi !
\VS{22}Allez donc, je vous prie, assurez-vous encore davantage pour savoir et trouver le lieu où il a dirigé ses pas et qui l’a vu ; car m’a-t-on dit, il est fort rusé.
\VS{23}Examinez donc et reconnaissez tous les lieux où il se tient caché, puis retournez vers moi quand vous en serez assurés, et j'irai avec vous. S'il est dans le pays, je le chercherai parmi tous les milliers de Juda.
\VS{24}Ils se levèrent donc et s'en allèrent à Ziph avant Saül. David et ses gens étaient dans le désert de Maon, dans la plaine, au midi du désert.
\VS{25}Saül et ses gens partirent à la recherche de David. Et l’on en informa David, qui descendit le rocher, et resta dans le désert de Maon. Saül l’ayant appris, poursuivit David au désert de Maon.
\VS{26}Saül marchait d’un côté de la montagne, et David et ses gens de l'autre côté de la montagne. David fuyait précipitamment pour échapper à Saül. Mais Saül et ses gens entouraient David et ses gens pour s’emparer d’eux.
\VS{27}Lorsqu’un messager vint à Saül, en disant : Hâte-toi de venir, car les Philistins envahissent le pays.
\VS{28}Alors Saül cessa de poursuivre David, et s'en retourna au-devant des Philistins : C'est pourquoi on appela ce lieu Séla-Hammachlekoth.
\TextTitle{[David épargne la vie de Saül à En-Guédi]}
\Chap{24}
\VerseOne{}Puis David monta de là et demeura dans les lieux forts d'En-Guédi.
\VS{2}Lorsque Saül fut revenu de la poursuite des Philistins, on lui fit ce rapport disant : David est dans le désert d’En-Guédi.
\VS{3}Saül prit trois mille hommes d'élite de tout Israël, et il s'en alla chercher David et ses gens jusque sur le rocher des boucs sauvages.
\VS{4}Saül arriva à des parcs de brebis qui étaient près du chemin, où il y avait une caverne dans laquelle il entra pour se couvrir les pieds. David et ses gens se tenaient au fond de la caverne.
\VS{5}Et les gens de David lui dirent : Voici le jour où Yahweh te dit : Je te livre ton ennemi entre tes mains, afin que tu lui fasses selon ce qu'il te semblera bon. David se leva et coupa tout doucement le pan du manteau de Saül.
\VS{6}Après cela, le cœur de David battit, parce qu'il avait coupé le pan du manteau de Saül.
\VS{7}Et il dit à ses gens : Que Yahweh me garde de commettre une telle action contre mon seigneur, l'oint de Yahweh, en mettant ma main sur lui ; car il est l'oint de Yahweh\FTNT{David épargne Saül parce qu'il fait confiance à Yahweh. David laisse Dieu agir plutôt que d'agir lui-même. C'est ce que Paul, apôtre du Seigneur Jésus-Christ, a écrit en Ro. 11 :17. }.
\VS{8}Ainsi David détourna ses gens par ses paroles, et il ne leur permit pas de s'élever contre Saül. Puis Saül se leva de la caverne et poursuivit son chemin.
\VS{9}Après cela, David se leva, sortit de la caverne, et cria après Saül, en disant : Mon seigneur le roi ! Saül regarda derrière lui, et David s'inclina le visage contre terre et se prosterna.
\VS{10}David dit à Saül : Pourquoi écouterais-tu les paroles des gens qui te disent : Voici, David cherche ton malheur ?
\VS{11}Aujourd'hui, tes yeux ont vu que Yahweh t'avait livré entre mes mains dans la caverne, et on m'a dit de te tuer ; mais je t'ai épargné, et j'ai dit : Je ne porterai pas la main sur mon seigneur ; car il est l'oint de Yahweh.
\VS{12}Regarde donc, mon père, regarde le pan de ton manteau dans ma main. Car, j’ai coupé le pan de ton manteau et je ne t'ai pas tué. Sache et reconnais qu'il n'y a ni mal ni injustice dans ma conduite ; et que je n'ai pas péché contre toi. Mais cependant tu me dresses des embûches pour me tuer.
\VS{13}Yahweh sera juge entre moi et toi, et Yahweh me vengera de toi, mais ma main ne sera pas sur toi.
\VS{14}Des méchants vient la méchanceté, dit l’ancien proverbe. C'est pourquoi je ne porterai pas la main sur toi.
\VS{15}Contre qui est sorti le roi d'Israël ? Qui poursuis-tu ? Un chien mort, une puce ?
\VS{16}Yahweh sera donc juge, et jugera entre moi et toi ; il regardera et plaidera ma cause, il me rendra justice en me délivrant de ta main.
\VS{17}Dès que David eut achevé d’adresser ces paroles à Saül, Saül dit : N'est-ce pas là ta voix, mon fils David ? Et Saül éleva la voix, et pleura.
\VS{18}Et il dit à David : Tu es plus juste que moi ; car tu m'as rendu le bien pour le mal que je t'ai fait,
\VS{19}et tu m'as fait connaître aujourd'hui comment tu as usé de bonté envers moi, car Yahweh m'avait livré entre tes mains, et cependant tu ne m'as pas tué.
\VS{20}Si quelqu’un rencontre son ennemi le laisse-t-il poursuivre tranquillement son chemin ? Que Yahweh donc te récompense pour la grâce que tu m'as faite aujourd'hui !
\VS{21}Et maintenant voici, je sais que tu régneras certainement et que le royaume d'Israël restera entre tes mains.
\VS{22}C'est pourquoi maintenant, jure-moi par Yahweh, que tu ne détruiras pas ma race après moi, et que tu n'extermineras pas mon nom de la maison de mon père.
\VS{23}Et David le jura à Saül. Puis Saül s'en alla dans sa maison, et David et ses gens montèrent au lieu fort.
\TextTitle{[Israël pleure la mort de Samuel]}
\Chap{25}
\VerseOne{}Samuel mourut, et tout Israël s'assembla, et le pleura, et on l'enterra dans sa maison à Rama. David se leva, et descendit au désert de Paran.
\TextTitle{[Ingratitude de Nabal, Abigail une femme de bon sens]}
\VS{2}Il y avait à Maon un homme qui avait ses biens à Carmel, et cet homme-là était très puissant, il avait trois mille brebis, et mille chèvres ; et il se trouvait à Carmel quand on tondait ses brebis.
\VS{3}Cet homme s’appelait Nabal, sa femme Abigaïl, elle était une femme de bon sens, et belle de visage, mais l’homme était cruel et méchant dans toutes ses actions. Il était de la race de Caleb.
\VS{4}David apprit au désert, que Nabal tondait ses brebis.
\VS{5}Il envoya dix jeunes gens, et leur dit : Montez à Carmel, et rendez-vous auprès de Nabal. Vous le saluerez en mon nom,
\VS{6}et vous lui direz : Puisses-tu faire autant l’année prochaine à la même saison, et que la paix soit avec ta maison et tout ce qui est à toi.
\VS{7}Et maintenant j'ai appris que tu as les tondeurs. Or tes bergers ont été avec nous, et nous ne leur avons fait aucune injure, et ils n’ont subi aucune perte pendant tout le temps qu'ils ont été à Carmel.
\VS{8}Demande-le à tes serviteurs, et ils te le diront. Que ces jeunes gens trouvent donc grâce à tes yeux, puisque nous venons dans un jour favorable. Nous te prions de donner à tes serviteurs, et à David, ton fils, ce que tu trouveras sous ta main.
\VS{9}Les gens de David arrivèrent et dirent à Nabal, au nom de David, toutes ces paroles ; puis ils se turent.
\VS{10}Nabal répondit aux serviteurs de David, et dit : Qui est David, et qui est le fils d'Isaï ? Aujourd'hui le nombre des serviteurs qui s’échappent de leurs maîtres se multiplie.
\VS{11}Et prendrais-je mon pain, mon eau, et la viande que j'ai apprêtée pour mes tondeurs, afin de les donner à des gens qui viennent je ne sais d'où ?
\VS{12}Ainsi les gens de David rebroussèrent chemin. Ils s'en retournèrent, et firent leur rapport à David.
\VS{13}Et David dit à ses gens : Que chacun de vous ceigne son épée. Et ils ceignirent chacun leur épée. David aussi ceignit son épée et environ quatre cents hommes montèrent avec David. Il en resta deux cents près des bagages.
\VS{14}Or un des serviteurs de Nabal fit ce rapport à Abigaïl, femme de Nabal, et lui dit : Voici, David a envoyé du désert des messagers pour saluer notre maître, qui les a traités rudement.
\VS{15}Cependant ces hommes ont été très bon envers nous, et ne nous ont fait aucune injure, et rien ne nous été enlevé, tout le temps que nous avons été avec eux lorsque nous étions dans les champs.
\VS{16}Ils nous ont servi de muraille nuit et jour, tout le temps que nous avons été avec eux, faisant paître les troupeaux.
\VS{17}Sache maintenant, et vois ce que tu as à faire, car le mal est résolu contre notre maître, et contre toute sa maison, et il est si méchant qu'on n'ose lui parler.
\VS{18}Abigaïl se hâta donc, et prit deux cents pains, deux outres de vin, cinq pièces de menu bétail, cinq mesures de grain rôti, cent paquets de raisins secs, deux cents de figues sèches, et les mit sur des ânes.
\VS{19}Puis elle dit à ses gens : Passez devant moi, je vais vous suivre. Elle n'en dit rien à Nabal, son mari.
\VS{20}Et étant montée sur un âne, elle descendait de la montagne par un chemin couvert ; voici, David et ses gens descendaient en face d’elle, et elle les rencontra.
\VS{21}David avait dit : C'est en vain que j'ai gardé tout ce que cet homme a dans le désert, en sorte qu'il ne s'est rien perdu de tout ce qu’il possède ; il m'a rendu le mal pour le bien.
\VS{22}Que Dieu traite son serviteur David dans toute sa rigueur, si d'ici au matin je laisse subsister de tout ce qui appartient à Nabal.
\VS{23}Lorsque Abigaïl aperçut David, elle se hâta de descendre de son âne, et tomba sur sa face devant David, et se prosterna contre terre.
\VS{24}Elle se jeta donc à ses pieds et lui dit : A moi la faute, mon seigneur ! Permets à ta servante de parler devant toi, et écoute les paroles de ta servante.
\VS{25}Que mon seigneur ne prenne pas garde à ce méchant homme, à Nabal, car il est comme son nom ; Nabal est son nom, et il y a de la folie chez lui. Et moi, ta servante, je n'ai pas vu les gens que mon seigneur a envoyés.
\VS{26}Maintenant, mon seigneur, aussi vrai que Yahweh est vivant, et que ton âme vit, Yahweh t'a empêché d'en venir au sang, et il a retenu ta main. Or que tes ennemis, et ceux qui cherchent à nuire à mon seigneur, soient comme Nabal.
\VS{27}Voici un présent, que ta servante a apporté à mon seigneur, afin qu'on le donne aux gens qui sont à la suite de mon seigneur.
\VS{28}Pardonne, je te prie, le crime de ta servante ; vu que Yahweh ne manquera pas d'établir une maison ferme à mon seigneur ; car mon seigneur conduit les batailles de Yahweh, et il ne s'est trouvé en toi aucun mal pendant toute ta vie.
\VS{29}Si les hommes se lèvent pour te persécuter, et pour chercher ton âme, l'âme de mon seigneur sera liée au faisceau des vivants auprès de Yahweh ton Dieu ; mais il lancera au loin, avec la fronde, l'âme de tes ennemis.
\VS{30}Lorsque Yahweh fera à mon seigneur selon tout le bien qu'il t'a prédit, et qu’il t'établira conducteur d'Israël,
\VS{31}ceci ne sera pas un obstacle, ni un sujet de regret dans l'âme de mon seigneur, pour avoir répandu le sang inutilement, et pour s'être vengé lui-même. Aussi lorsque Yahweh aura fait du bien à mon seigneur, tu te souviendras de ta servante.
\VS{32}Alors David dit à Abigaïl : Béni soit Yahweh, le Dieu d'Israël, qui t'a aujourd'hui envoyée à ma rencontre !
\VS{33}Et béni soit ton bon sens, et bénie sois-tu, toi qui m'as aujourd'hui empêché d'en venir au sang, et qui as retenu ma main !
\VS{34}Car Yahweh, le Dieu d'Israël qui m'a empêché de te faire du mal, est vivant ! Si tu ne t’étais hâtée de venir à ma rencontre, il ne serait resté qui que ce soit à Nabal d'ici au matin.
\VS{35}David prit donc de sa main ce qu'elle lui avait apporté, et lui dit : Remonte en paix dans ta maison ; regarde, j'ai écouté ta voix, et j'ai répondu favorablement à ta demande.
\TextTitle{[Mort de Nabal]}
\VS{36}Alors Abigaïl revint auprès de Nabal ; et voici, il faisait un festin dans sa maison, comme un festin de roi ; et Nabal avait le cœur joyeux, et il était complètement ivre ; c'est pourquoi elle ne lui dit aucune chose petite ou grande, jusqu'au matin.
\VS{37}Mais le matin, l’ivresse de Nabal étant dissipée, sa femme lui raconta toutes ces choses. Le cœur de Nabal reçut un coup mortel, de sorte qu'il devint comme une pierre.
\VS{38}Environ dix jours après, Yahweh frappa Nabal, et il mourut.
\VS{39}Lorsque David apprit que Nabal était mort, il dit : Béni soit Yahweh, qui m'a vengé de l'outrage que j'avais reçu de la main de Nabal, et qui a préservé son serviteur de faire du mal, et a fait retomber le mal de Nabal sur sa tête ! Puis David envoya des gens pour parler à Abigaïl, afin de la prendre pour sa femme.
\VS{40}Les serviteurs de David vinrent auprès d'Abigaïl à Carmel, et lui parlèrent, en disant : David nous a envoyés vers toi, afin de te prendre pour femme.
\VS{41}Alors elle se leva, et se prosterna le visage contre terre, et dit : Voici, ta servante sera à ton service afin de laver les pieds des serviteurs de mon seigneur.
\VS{42}Aussitôt, Abigaïl se leva et monta sur un âne, accompagnée de cinq jeunes filles ; elle suivit les messagers de David, et fut sa femme.
\VS{43}Or David avait pris aussi Achinoam, de Jizreel, et toutes les deux furent ses femmes.
\VS{44}Et Saül avait donné Mical, sa fille, femme de David, à Palthi, fils de Laïsch, qui était de Gallim.
\TextTitle{[David épargne encore la vie de Saül]}
\Chap{26}
\VerseOne{}Les Ziphiens allèrent encore auprès de Saül à Guibea, en disant : David ne se tient-il pas caché sur la colline de Hakila, en face du désert ?
\VS{2}Saül se leva, et descendit au désert de Ziph, avec trois mille hommes de l'élite d'Israël, pour chercher David dans le désert de Ziph.
\VS{3}Saül campa sur la colline de Hakila, en face du désert, près du chemin. David se tenait dans le désert, et il aperçut que Saül marchait à sa poursuite au désert,
\VS{4}alors il envoya des espions et apprit avec certitude que Saül était arrivé.
\VS{5}Alors David se leva, et alla au lieu où Saül campait, et David vit la place où couchait Saül, avec Abner, fils de Ner, chef de son armée. Saül couchait au milieu du camp, et le peuple campait autour de lui.
\VS{6}David prit la parole et dit à Achimélec, Héthien, et à Abischaï, fils de Tseruja et frère de Joab, il dit : Qui veut descendre avec moi dans le camp vers Saül ? Et Abischaï répondit : J'y descendrai avec toi.
\VS{7}David et Abischaï allèrent de nuit vers le peuple, et voici, Saül dormait étant couché au milieu du camp, et sa lance était plantée en terre à son chevet ; et Abner, et le peuple étaient couchés autour de lui.
\VS{8}Alors Abischaï dit à David : Aujourd'hui, Dieu a livré ton ennemi entre tes mains ; laisse-moi donc le frapper avec la lance, jusqu'en terre d'un seul coup, et je n'y retournerai pas une seconde fois.
\VS{9}Et David dit à Abischaï : Ne le tues pas ! Car qui porterait impunément sa main sur l'oint de Yahweh ?
\VS{10}David dit encore : Yahweh est vivant ! C’est Yahweh seul qui le frappera, soit que son jour vienne, soit qu'il descende au combat et qu'il y périsse.
\VS{11}Que Yahweh me garde de mettre ma main sur l'oint de Yahweh ; mais prends maintenant la lance qui est à son chevet et la cruche d’eau, et allons-nous-en.
\VS{12}David donc prit la lance et la cruche d’eau qui étaient au chevet de Saül, puis ils s'en allèrent. Personne ne les vit, ni ne s’aperçut de rien, ni ne se réveilla ; car ils dormaient tous d’un profond sommeil dans lequel Yahweh les avait plongés.
\VS{13}David passa de l'autre côté, et s'arrêta au loin sur le sommet de la montagne, et il y avait une grande distance entre eux.
\VS{14}Et il cria au peuple, et à Abner, fils de Ner, en disant : Ne répondras-tu pas, Abner ? Abner répondit, et dit : Qui es-tu toi qui cries vers le roi ?
\VS{15}Alors David dit à Abner : N'es-tu pas un vaillant homme ? Qui est semblable à toi en Israël ? Pourquoi donc n'as-tu pas gardé le roi ton seigneur ? Car quelqu'un du peuple est venu pour tuer le roi ton seigneur.
\VS{16}Ce que tu as fait n’est pas bien ; Yahweh est vivant ! Vous méritez la mort, pour avoir si mal gardé votre seigneur, l'oint de Yahweh. Et maintenant regarde où sont la lance du roi et la cruche d’eau qui était à son chevet.
\VS{17}Alors Saül reconnut la voix de David, et dit : N'est-ce pas là ta voix, mon fils David ? Et David dit : C'est ma voix, ô roi mon seigneur.
\VS{18}Il dit encore : Pourquoi mon seigneur poursuit-il son serviteur ? Car qu'ai-je fait, de quoi suis-je coupable ?
\VS{19}Maintenant donc je te prie, que le roi mon seigneur écoute les paroles de son serviteur. Si c'est Yahweh qui te pousse contre moi, que ton offrande lui soit agréable ; mais si ce sont les hommes, qu’ils soient maudits devant Yahweh ; car aujourd'hui ils m'ont chassé, afin que je ne puisse me joindre à l'héritage de Yahweh, et ils m'ont dit : Va, sers les dieux étrangers.
\VS{20}Que mon sang ne tombe pas en terre loin de la face de Yahweh ! Car le roi d’Israël est sorti pour chercher une puce, comme on poursuivrait une perdrix dans les montagnes.
\TextTitle{[Saül se repend devant David]}
\VS{21}Saül dit : J'ai péché, reviens mon fils David ; car je ne te ferai plus de mal, parce qu'aujourd'hui ma vie t'a été précieuse. Voici, j'ai agi en insensé, et j'ai commis une très grande faute.
\VS{22}David répondit et dit : Voici la lance du roi que l'un de tes gens vienne la prendre.
\VS{23}Que Yahweh rende à chacun selon sa justice et selon sa fidélité ; car il t'avait livré aujourd'hui entre mes mains, mais je n'ai pas voulu mettre ma main sur l'oint de Yahweh.
\VS{24}Voici, comme ta vie a été aujourd'hui de grand prix à mes yeux, ainsi ma vie sera de grand prix aux yeux de Yahweh, et il me délivrera de toutes les angoisses.
\VS{25}Saül dit à David : Béni sois-tu, mon fils David ! Tu auras du succès dans tes entreprises. Alors David continua son chemin, et Saül s'en retourna chez lui.
\TextTitle{[David se réfugie dans le pays des Philistins]}
\Chap{27}
\VerseOne{}David dit en son cœur : Certes je périrai un jour par les mains de Saül ; ne vaut-il pas mieux que je me sauve en hâte au pays des Philistins, afin que Saül renonce à me chercher encore dans tout le territoire d'Israël ? Ainsi j’échapperai à sa main.
\VS{2}David se leva, lui et les six cents hommes qui étaient avec lui, et il passa chez Akisch, fils de Maoc, roi de Gath.
\VS{3}David et ses gens restèrent à Gath auprès d’Akisch ; ils avaient chacun leur famille, David et ses deux femmes, Achinoam de Jizreel, et Abigaïl, femme de Nabal, qui était de Carmel.
\VS{4}Alors on informa Saül que David s'était enfui à Gath ; et il cessa de le chercher.
\VS{5}David dit à Akisch : Si j'ai trouvé grâce à tes yeux, qu'on me donne dans l'une des villes du pays, un lieu où je puisse habiter. Car pourquoi ton serviteur habiterait-il dans la ville royale avec toi ?
\VS{6}Akisch lui donna ce même jour, Tsiklag. C'est pourquoi Tsiklag appartient aux rois de Juda jusqu'à ce jour.
\VS{7}Le temps que David demeura dans le pays des Philistins fut d’un an et quatre mois.
\VS{8}David montait avec ses gens faire des incursions chez les Gueschuriens, les Guirziens, et les Amalécites ; car ces nations habitaient dans le territoire dès les temps anciens, depuis Schur jusqu'au pays d'Egypte.
\VS{9}David ravageait ce territoire, il ne laissait en vie ni homme ni femme, et il prenait les brebis, les bœufs, les ânes, les chameaux, et les vêtements, puis il s'en retournait, et allait chez Akisch.
\VS{10}Akisch disait : Où avez-vous fait vos incursions aujourd'hui ? Et David répondait : Vers le midi de Juda, vers le midi des Jerachmeélites, et vers le midi des Kéniens.
\VS{11}Mais David ne laissait en vie ni homme ni femme pour les amener à Gath, de peur, disait-il, qu'ils ne rapportent quelque chose contre nous, disant : Ainsi a fait David. Et il agit ainsi tout le temps qu'il demeura dans le pays des Philistins.
\VS{12}Akisch croyait David, et il disait : Il se rend odieux à Israël, son peuple ; c'est pourquoi il sera mon serviteur à jamais.
\TextTitle{[Les Philistins vont en guerre contre Saül]}
\Chap{28}
\VerseOne{}En ces temps-là, les Philistins rassemblèrent leurs armées pour faire la guerre, pour combattre Israël. Akisch dit à David : Sache certainement que vous viendrez avec moi au camp, toi et tes gens.
\VS{2}David répondit à Akisch : Certainement tu verras ce que ton serviteur fera. Et Akisch dit à David : C'est pour cela que je te confierai toujours la garde de ma personne.
\VS{3}Or Samuel était mort, et tout Israël avait fait le deuil, et on l'avait enseveli à Rama qui était sa ville. Saül avait ôté du pays ceux qui avaient des esprits de Python\FTNT{L’esprit de python possède les faux prophètes (Ac. 16:16-19).}, et les médiums.
\VS{4}Les Philistins se rassemblèrent et vinrent camper à Sunem ; Saül aussi rassembla tout Israël, et ils campèrent à Guilboa.
\VS{5}A la vue du camp des Philistins, Saül eut peur, et son cœur fut saisi de crainte.
\VS{6}Saül consulta Yahweh ; mais Yahweh ne lui répondit rien, ni par des songes, ni par l'urim, ni par les prophètes.
\TextTitle{[Saül consulte une femme qui évoque les morts]}
\VS{7}Saül dit à ses serviteurs : Cherchez-moi une femme qui ait un esprit de Python, et j'irai vers elle, et je la consulterai. Ses serviteurs lui dirent : Voilà, il y a une femme à En-Dor qui évoque les morts.
\VS{8}Alors Saül se déguisa, prit d'autres vêtements, et il partit avec deux hommes. Ils arrivèrent de nuit chez cette femme et Saül lui dit : Je te prie devine-moi par l’esprit de Python, et fais-moi monter vers moi celui que je te dirai.
\VS{9}Mais la femme lui répondit : Voici, tu sais ce que Saül a fait, et comment il a exterminé du pays ceux qui ont l'esprit de Python et les médiums ; pourquoi donc dresses-tu un piège à mon âme pour me faire mourir ?
\VS{10}Saül lui jura par Yahweh, et lui dit : Yahweh est vivant ! Il ne t’arrivera pas de mal pour cela.
\VS{11}Alors la femme dit : Qui veux-tu que je te fasse monter ? Et il répondit : Fais-moi monter Samuel.
\VS{12}Et la femme voyant Samuel s'écria à haute voix, en disant à Saül : Pourquoi m'as-tu trompée ? Car tu es Saül.
\VS{13}Et le roi lui répondit : Ne crains pas, mais que vois-tu ? La femme dit à Saül : Je vois un dieu qui monte de la terre.
\VS{14}Il lui dit encore : Comment est-il fait ? Elle répondit : C'est un vieillard qui monte, et il est couvert d'un manteau. Et Saül comprit que c'était Samuel, il s’inclina le visage contre terre et se prosterna.
\VS{15}Samuel dit à Saül : Pourquoi m'as-tu troublé en me faisant monter ? Et Saül répondit : Je suis dans une grande angoisse ; car les Philistins me font la guerre, et Dieu s'est retiré de moi, et ne m'a plus répondu ni par les prophètes ni par des songes ; c'est pourquoi je t'ai invoqué\FTNT{}, afin que tu me fasses entendre ce que j'aurai à faire.
\VS{16}Samuel dit : Pourquoi donc me consultes-tu, puisque Yahweh s'est retiré de toi, et qu'il est devenu ton ennemi ?
\VS{17}Yahweh te traite comme je te l’avais annoncé de sa part ; car Yahweh a déchiré le royaume d'entre tes mains, et l'a donné à un autre, à David.
\VS{18}Parce que tu n'as pas obéi à la voix de Yahweh, et que tu n'as pas exécuté l'ardeur de sa colère contre Amalek, à cause de cela, Yahweh te traite de cette manière aujourd'hui.
\VS{19}Yahweh livrera Israël avec toi entre les mains des Philistins, et vous serez demain avec moi, toi et tes fils; Yahweh livrera aussi le camp d'Israël entre les mains des Philistins.
\VS{20}Saül s’écroula à terre tout étendu, très effrayé des paroles de Samuel, les forces lui manquèrent parce qu'il n'avait rien mangé ce jour, ni toute cette nuit.
\VS{21}Alors la femme vint auprès de Saül, et voyant qu'il avait été très effrayé, elle lui dit : Voici, ta servante a obéi à ta voix, j'ai exposé ma vie, et j'ai obéi aux paroles que tu m'as dites.
\VS{22}Maintenant, je te prie, écoute toi aussi ce que ta servante te dira : Laisse-moi de servir avant un morceau de pain, afin que tu manges pour avoir la force de te remettre en route.
\VS{23}Et il le refusa et dit : Je ne mangerai pas. Mais ses serviteurs et la femme aussi le pressèrent tellement qu'il écouta leur voix. Il se leva de terre, et s'assit sur un lit.
\VS{24}Cette femme avait dans sa maison un veau qu'elle engraissait ; et elle se hâta de le tuer, puis elle prit de la farine, et la pétrit, et en cuisit des pains sans levain.
\VS{25}Elle les mit devant Saül et devant ses serviteurs. Et ils mangèrent. Puis s'étant levés, ils s'en allèrent cette nuit-là.
\TextTitle{[Les philistins refusent que David combattent contre Israël]}
\Chap{29}
\VerseOne{}Les Philistins rassemblèrent toutes leurs armées à Aphek, et Israël campa près de la fontaine de Jizreel.
\VS{2}Les princes des Philistins s’avancèrent avec leurs centaines et leurs milliers, et David et ses gens marchèrent à l'arrière-garde avec Akisch.
\VS{3}Les princes des Philistins dirent : Que font ici ces Hébreux ? Et Akisch répondit aux princes des Philistins : N'est-ce pas David, serviteur de Saül, roi d'Israël, il y a longtemps qu’il est avec moi, même quelques années, et je n'ai pas trouvé quelque chose à lui reprocher depuis son arrivée, jusqu'à ce jour.
\VS{4}Mais les princes des Philistins se mirent en colère contre lui, et lui dirent : Renvoie cet homme, et qu'il retourne dans le lieu où tu l'as établi, et qu'il ne descende pas avec nous dans la bataille, de peur qu'il ne se tourne contre nous dans la bataille ; car comment pourrait-il se remettre en grâce auprès de son maître ? Ne serait-ce pas par le moyen des têtes de nos hommes ?
\VS{5}N'est-ce pas ce David, pour qui l’on chantait et répondait en dansant : Saül a frappé ses mille, et David ses dix mille ?
\VS{6}Akisch appela David, et lui dit : Yahweh est vivant ! Tu es certainement un homme droit, et ta conduite dans le camp m'a paru bonne, car je n'ai pas trouvé de mal en toi, depuis le jour où tu es arrivé auprès de moi jusqu'à ce jour ; mais tu ne plais pas aux princes.
\VS{7}Maintenant retourne, et va-t'en en paix, afin que tu ne fasses aucune chose qui déplaise aux princes des Philistins.
\VS{8}David dit à Akisch : Mais qu'ai-je fait ? Et qu'as-tu trouvé en ton serviteur depuis que je suis avec toi jusqu'à ce jour, pour que je n'aille pas combattre contre les ennemis du roi, mon seigneur ?
\VS{9}Akisch répondit et dit à David : Je le sais, car tu es agréable à mes yeux, comme un ange de Dieu ; mais c'est seulement les chefs des Philistins qui disent : Il ne montera pas avec nous dans la bataille.
\VS{10}C'est pourquoi lève-toi de bon matin, avec les serviteurs de ton maître qui sont venus avec toi ; levez-vous de bon matin, et partez dès que vous verrez le jour, allez-vous-en.
\VS{11}Ainsi David se leva de bonne heure, lui et ses gens, pour partir dès le matin, et retourner dans le pays des Philistins. Et les Philistins montèrent à Jizreel.
\TextTitle{[David libère Tsiklag]}
\Chap{30}
\VerseOne{}Lorsque David et ses gens arrivèrent à Tsiklag, le troisième jour, les Amalécites avaient fait une invasion dans le midi, et à Tsiklag, et ils avaient frappé et brûlé Tsiklag.
\VS{2}Après avoir fait prisonniers les femmes et tous ceux qui étaient là, petits et grands. Ils n’avaient tué personne, mais ils les avaient emmenés, et s’étaient remis en chemin.
\VS{3}David et ses gens revinrent dans la ville et voici, elle était brûlée, et leurs femmes, leurs fils, et leurs filles avaient été faits prisonniers.
\VS{4}C’est pourquoi David et le peuple qui était avec lui élevèrent leur voix, et pleurèrent tellement qu’il n’y avait plus en eux de force pour pleurer.
\VS{5}Les deux femmes de David avaient été emmenées, Achinoam, de Jizreel, et Abigaïl, de Carmel, femme de Nabal.
\VS{6}David fut dans une grande angoisse, parce que le peuple parlait de le lapider ; car tout le peuple avait de l’amertume dans l’âme à cause de leurs fils et de leurs filles ; toutefois David se fortifia en Yahweh, son Dieu.
\VS{7}Et il dit au sacrificateur Abiathar, fils d’Achimélec : Apporte-moi, je te prie, l’éphod ! Abiathar apporta l'éphod à David.
\VS{8}Et David consulta Yahweh, en disant : Poursuivrai-je cette troupe ? L’atteindrai-je ? Et il lui répondit : Poursuis, car tu l’atteindras, et tu délivreras.
\VS{9}David s’en alla avec les six cents hommes qui étaient avec lui, et ils arrivèrent au torrent de Besor, où s’arrêtèrent ceux qui restaient en arrière.
\VS{10}Ainsi David et quatre cents hommes, continuèrent la poursuite, mais deux cents hommes s’arrêtèrent, trop fatigués pour pouvoir passer le torrent de Besor.
\VS{11}Ayant trouvé un homme Egyptien dans les champs, ils l’amenèrent à David, et lui donnèrent du pain, il mangea, puis ils lui donnèrent de l’eau à boire.
\VS{12}Ils lui donnèrent aussi quelques figues sèches, et deux grappes de raisins secs, et il mangea, et le coeur lui revint ; car cela faisait trois jours et trois nuits qu’il n’avait pas mangé de pain, ni bu d’eau.
\VS{13}Et David lui dit : A qui es-tu ? Et d’où es-tu ? Et il répondit : Je suis un garçon égyptien, serviteur d’un homme amalécite ; et mon maître m’a abandonné, parce que j’étais malade il y a trois jours.
\VS{14}Nous avons envahi le midi des Kéréthiens, et sur ce qui est à Juda, et sur le midi de Caleb, et nous avons mis le feu et brûlé Tsiklag.
\VS{15}David lui dit : Me conduiras-tu vers cette troupe ? Et il répondit : Jure-moi par le Nom de Dieu que tu ne me feras point mourir, et que tu ne me livreras point entre les mains de mon maître, et je te conduirai vers cette troupe.
\VS{16}Et il le conduisit. Et voici, ils étaient dispersés sur toute la contrée, mangeant, buvant, et dansant, à cause de ce grand butin qu’ils avaient pris au pays des Philistins, et au pays de Juda.
\VS{17}Et David les frappa depuis l’aube du jour, jusqu’au soir du lendemain, et il n’en échappa aucun d’eux, hormis quatre cents jeunes hommes qui montèrent sur des chameaux, et s’enfuirent.
\VS{18}David recouvra tout ce que les Amalécites avaient emporté ; il délivra aussi ses deux femmes.
\VS{19}Il ne leur manqua personne, depuis le plus petit jusqu’au plus grand, ni fils ni filles, ni butin, ni rien de ce qu’ils leur avaient emporté ; David ramena tout.
\VS{20}David reprit aussi tout le gros et menu bétail, qu’on mena devant les troupeaux ; et on disait : C’est ici le butin de David.
\TextTitle{[David partage le butin]}
\VS{21}Puis David arriva auprès de deux cents hommes qui avaient été tellement fatigués qu’ils n’avaient pu suivre David, et qu’on avait laissés au torrent de Besor. Ils sortirent au-devant de David, et au-devant du peuple qui était avec lui. David s’étant approché du peuple, il les salua aimablement.
\VS{22}Mais tous les mauvais et méchants hommes qui étaient allés avec David, prirent la parole, et dirent : Puisqu’ils ne sont point venus avec nous, nous ne leur donnerons rien du butin que nous avons récupéré, sinon à chacun sa femme et ses enfants, et qu’ils les emmènent, et s’en aillent.
\VS{23}Mais David dit : Mes frères, n’agissez pas ainsi au sujet de ce que Yahweh nous a donné, il nous a gardés, et a livré entre nos mains la troupe qui était venue contre nous.
\VS{24}Qui vous écouterait dans cette affaire ? Car celui qui est resté près des bagages doit avoir autant que celui qui est descendu sur le champ de bataille ; ils partageront ensemble.
\VS{25}Il en fut ainsi depuis ce jour et dans la suite, il en fut fait de même une ordonnance et une loi en Israël.
\VS{26}David revint à Tsiklag, et envoya une partie du butin aux anciens de Juda, à ses amis, en disant : Voici, un présent pour vous, du butin des ennemis de Yahweh.
\VS{27}Il en envoya à ceux de Béthel, à ceux qui étaient à Ramoth du midi, à ceux de Jatthir,
\VS{28}à ceux d’Aroër, à ceux de Siphmoth, à ceux de Eschthemoa,
\VS{29}à ceux de Racal, à ceux des villes des Jerachmeélites, à ceux des villes des Kéniens,
\VS{30}à ceux d’Horma, à ceux de Cor-Aschan, à ceux d’Athac,
\VS{31}à ceux d’Hébron, et dans tous les lieux où David avait demeuré, lui et ses gens.
\TextTitle{[Mort de Jonathan et Saül à Guilboa]
\\(1 Ch. 10:1-14)}
\Chap{31}
\VerseOne{}Les Philistins livrèrent bataille à Israël, et les hommes d'Israël s'enfuirent devant les Philistins, et furent tués sur la montagne de Guilboa.
\VS{2}Les Philistins atteignirent Saül et ses fils, et tuèrent Jonathan, Abinadab et Malkischua, fils de Saül.
\VS{3}L’effort du combat se porta sur Saül, et les archers l’atteignirent et le blessèrent grièvement.
\VS{4}Alors Saül dit à celui qui portait ses armes : Tire ton épée, et transperce-moi, de peur que ces incirconcis ne viennent, ne me transpercent, et ne m’outragent. Mais celui qui portait ses armes refusa, parce qu'il était saisi de crainte. Saül prit l'épée, et se jeta dessus.
\VS{5}Alors celui qui portait les armes de Saül, voyant que Saül était mort, se jeta aussi sur son épée, et mourut avec lui.
\VS{6}Ainsi périrent en ce jour, Saül et ses trois fils, celui qui portait ses armes, et tous ses gens.
\VS{7}Ceux d'Israël qui étaient de ce côté de la vallée et de ce côté du Jourdain, ayant vu que les Israëlites s'étaient enfuis que Saül et ses fils étaient morts, abandonnèrent les villes et s'enfuirent ; de sorte que les Philistins y entrèrent et s’y établirent.
\VS{8}Le lendemain, les Philistins vinrent pour dépouiller les morts, et ils trouvèrent Saül et ses trois fils, étendus sur la montagne de Guilboa.
\VS{9}Ils coupèrent la tête de Saül et le dépouillèrent de ses armes. Ils firent annoncer ces bonnes nouvelles par tout le pays des Philistins, dans les maisons de leurs idoles et parmi le peuple.
\VS{10}Ils déposèrent les armes de Saül dans le temple d’Astarté, et ils attachèrent son cadavre sur les murs de Beth-Schan.
\VS{11}Lorsque les habitants de Jabès en Galaad apprirent ce que les Philistins avaient fait à Saül,
\VS{12}tous les vaillants hommes, se levèrent et marchèrent toute la nuit, et ils enlevèrent des murs de Beth-Schan le cadavre de Saül et les cadavres de ses fils. Ils revinrent à Jabès, où ils les brûlèrent.
\VS{13}Puis ils prirent leurs os, les ensevelirent sous un tamaris près de Jabès, et ils jeûnèrent sept jours.
\PPE{}
\end{multicols}

%\clearpage\ShortTitle{2 Samuel}\BookTitle{2 Samuel}\BFont
\noindent\hrulefill
{\footnotesize
\textit{
\bigskip
{\centering{}
\\Auteur : Inconnu
\\(Heb. : Shemuw'el)
\\Signification : Entendu, exaucé de Dieu
\\Thème : Le règne de David
\\Date de rédaction : 10\up{ème} siècle av. J.-C.\\}
}
%\bigskip
\textit{
\\Suite du premier livre de Samuel, ce livre commence par le récit de la mort de Saül et l'accession progressive à la royauté de David. La faveur de Dieu dans sa vie lui donna du succès et lui permit d'étendre son royaume jusqu'au nord de Damas. Au détriment de sa piété et de son alliance avec Dieu, David commit de lourdes erreurs. Il s'en repentit sincèrement, mais il dut en assumer les conséquences…\bigskip
}
}
\par\nobreak\noindent\hrulefill
\begin{multicols}{2}
\Chap{1}
\TextTitle{Attitude de David à la mort de Saül}
\VerseOne{}Et il arriva qu'après la mort de Saül, David, qui était revenu vainqueur des Amalécites, resta deux jours à Tsiklag.
\VS{2}Le troisième jour, un homme arriva du camp de Saül, avec ses vêtements déchirés et de la terre sur sa tête. Il se présenta à David, se jeta par terre et se prosterna.
\VS{3}David lui dit : D'où viens-tu ? Il lui répondit : Je me suis échappé du camp d'Israël.
\VS{4}David lui dit : Qu'est-il arrivé ? Je te prie, raconte-le-moi! Il répondit : Le peuple s'est enfui de la bataille, et il y a eu beaucoup du peuple qui sont tombés morts ; Saül aussi et Jonathan, son fils, sont morts.
\VS{5}David dit à ce jeune garçon qui lui disait ces nouvelles : Comment sais-tu que Saül et Jonathan, son fils, sont morts ?
\VS{6}Le jeune garçon qui lui disait ces nouvelles lui répondit : Je me trouvais par hasard sur la montagne de Guilboa ; et voici, Saül s'appuyait sur sa lance, et voici, les chars et quelques chefs des cavaliers le poursuivaient.
\VS{7}S'étant retourné, il m'aperçut et m'appela. Je lui répondis : Me voici !
\VS{8}Il me dit : Qui es-tu ? Je lui répondis : Je suis Amalécite.
\VS{9}Et il dit : Approche-toi de moi et tue-moi ; car je suis dans une grande angoisse, et ma vie est encore toute en moi.
\VS{10}Je m'approchai de lui et je lui donnai la mort\FTNT{Cet homme Amalécite a peut-être menti afin de gagner la faveur de David (1 S. 31:3-5).}, sachant bien qu'il ne survivrait après s'être jeté sur sa lance. J'ai pris la couronne qu'il avait sur sa tête, et le bracelet qui était à son bras, et je les apporte ici à mon seigneur.
\VS{11}Alors David saisit ses vêtements et les déchira, et tous les hommes qui étaient avec lui firent de même.
\VS{12}Ils furent dans le deuil, ils pleurèrent et ils jeûnèrent jusqu'au soir, à cause de Saül, de Jonathan, son fils, à cause du peuple de Yahweh, et de la maison d'Israël, parce qu'ils étaient tombés par l'épée.
\VS{13}David dit au jeune garçon qui lui avait apporté ces nouvelles : D'où es-tu ? Et il répondit : Je suis fils d'un homme étranger, d'un Amalécite.
\VS{14}David lui dit : Comment n'as-tu pas craint d'avancer ta main pour tuer l'oint de Yahweh ?
\VS{15}Et David appela l'un de ses serviteurs et lui dit : Approche-toi, jette-toi sur lui ! Ce dernier le frappa et il mourut.
\VS{16}Et David lui dit : Que ton sang retombe sur ta tête, car ta bouche a témoigné contre toi, en disant : J'ai fait mourir l'oint de Yahweh !
\TextTitle{Chant funèbre de David}
\VS{17}Alors David composa sur Saül et sur Jonathan, son fils, un chant funèbre,
\VS{18}qu'il ordonna d'enseigner aux fils de Juda. C'est le cantique de l'arc : Il est écrit dans le livre du Juste.
\VS{19}L'élite d'Israël a succombé sur tes collines ! Comment des héros sont-ils tombés ?
\VS{20}Ne l'annoncez pas dans Gath et n'en publiez point la nouvelle dans les rues d'Askalon, de peur que les filles des Philistins ne se réjouissent, de peur que les filles des incirconcis n'en tressaillent de joie.
\VS{21}Montagnes de Guilboa ! Qu'il n'y ait sur vous ni rosée, ni pluie, ni champs qui donnent des prémices pour les offrandes ! Car là ont été jetés les boucliers des héros, le bouclier de Saül ; l'huile a cessé de les oindre.
\VS{22}L'arc de Jonathan ne revenait jamais sans être teint du sang des blessés et de la graisse des hommes forts ; et l'épée de Saül ne retournait jamais sans effet.
\VS{23}Saül et Jonathan, aimables et agréables pendant leur vie, n'ont point été séparés dans leur mort ; ils étaient plus légers que les aigles, ils étaient plus forts que des lions.
\VS{24}Filles d'Israël ! Pleurez sur Saül, qui vous revêtait magnifiquement de cramoisi, qui mettait des ornements d'or à vos habits.
\VS{25}Comment les héros sont-ils tombés au milieu du combat ? Comment Jonathan a-t-il été tué sur tes collines ?
\VS{26}Jonathan, mon frère, je suis dans la douleur à cause de toi ! Tu faisais tout mon plaisir ; l'amour que j'avais pour toi était plus grand que celui qu'on a pour les femmes.
\VS{27}Comment sont tombés les héros ? Comment se sont perdus les instruments de guerre ?
\Chap{2}
\TextTitle{David oint roi de Juda}
\VerseOne{}Et il arriva après cela que David consulta Yahweh, en disant : Monterai-je dans l'une des villes de Juda ? Yahweh lui répondit : Monte. David dit : Où monterai-je ? Yahweh répondit : A Hébron.
\VS{2}David y monta, avec ses deux femmes, Achinoam de Jizreel, et Abigaïl de Carmel, qui avait été femme de Nabal.
\VS{3}David fit monter aussi les hommes qui étaient avec lui, chacun avec sa famille ; et ils habitèrent dans les villes d'Hébron.
\VS{4}Les hommes de Juda vinrent, et là ils oignirent David pour roi sur la maison de Juda. On fit un rapport à David en disant : Les gens de Jabès en Galaad ont enseveli Saül.
\VS{5}Alors David envoya des messagers vers les gens de Jabès en Galaad, pour leur dire : Soyez bénis de Yahweh, puisque vous avez montré de la bienveillance envers Saül, votre seigneur, et que vous l'avez enseveli.
\VS{6}Maintenant donc que Yahweh use envers vous de bonté et de fidélité. Moi aussi je vous ferai du bien, parce que vous avez agi de la sorte.
\VS{7}Que maintenant vos mains se fortifient, et soyez des vaillants hommes ; car Saül, votre maître, est mort, et la maison de Juda m'a oint pour être roi sur elle.
\TextTitle{Isch-Boscheth établi roi d'Israël}
\VS{8}Cependant Abner, fils de Ner, chef de l'armée de Saül, prit Isch-Boscheth, fils de Saül, et le fit passer à Mahanaïm.
\VS{9}Il l'établit roi sur Galaad, sur les Gueschuriens, sur Jizreel, sur Ephraïm, sur Benjamin, et sur tout Israël.
\VS{10}Isch-Boscheth, fils de Saül, était âgé de quarante ans quand il commença à régner sur Israël, et il régna deux ans. Il n'y eut que la maison de Juda qui suivit David.
\VS{11}Le nombre de jours pendant lesquels David régna à Hébron sur la maison de Juda fut de sept ans et six mois.
\TextTitle{Guerre entre Juda et Israël}
\VS{12}Abner, fils de Ner, et les gens d'Isch-Boscheth, fils de Saül, sortirent de Mahanaïm pour marcher vers Gabaon.
\VS{13}Joab, fils de Tseruja, et les gens de David, sortirent aussi. Ils se rencontrèrent ensemble près de l'étang de Gabaon, et les uns se tinrent d'un côté de l'étang, et les autres du côté opposé de l'étang.
\VS{14}Alors Abner dit à Joab : Que ces jeunes gens se lèvent maintenant, et qu'ils se battent devant nous ! Et Joab répondit : Qu'ils se lèvent !
\VS{15}Ils se levèrent donc, et s'avancèrent en nombre égal, douze pour Benjamin et pour Isch-Boscheth, fils de Saül, et douze des gens de David.
\VS{16}Alors, chacun saisissant son adversaire par la tête, lui enfonça son épée dans le flanc, et ils tombèrent tous ensemble. Et l'on donna à ce lieu, qui est près de Gabaon, le nom de Helkath-Hatsurim.
\VS{17}Il y eut ce jour-là un combat très rude, dans lequel Abner et les hommes d'Israël furent battus par les gens de David.
\VS{18}Les trois fils de Tseruja, Joab, Abischaï et Asaël étaient là. Asaël avait les pieds légers comme une gazelle des champs.
\VS{19}Asaël poursuivit Abner, sans se détourner de lui ni à droite ni à gauche.
\VS{20}Abner regarda derrière lui, et dit : Est-ce toi, Asaël ? Et il répondit : C'est moi.
\VS{21}Abner lui dit : Détourne-toi à droite ou à gauche ; saisis-toi de l'un de ces jeunes gens, et prends sa dépouille. Mais Asaël ne voulut point se détourner de lui.
\VS{22}Et Abner dit encore à Asaël : Détourne-toi de moi ; pourquoi te frapperais-je et t'abattrais-je à terre ? Comment ensuite lèverais-je le visage devant ton frère Joab ?
\VS{23}Mais Asaël refusa de se détourner. Abner le frappa au ventre avec l'extrémité inférieure de sa lance, qui sortit par-derrière. Il tomba là, raide mort sur place. Tous ceux qui arrivaient au lieu où Asaël était tombé mort, s'y arrêtaient.
\VS{24}Joab et Abischaï poursuivirent Abner, et le soleil se couchait quand ils arrivèrent au coteau d'Amma, qui est en face de Guiach, sur le chemin du désert de Gabaon.
\VS{25}Les fils de Benjamin s'assemblèrent auprès d'Abner et formèrent un corps de troupe, et ils s'arrêtèrent sur le sommet d'une colline.
\VS{26}Alors Abner appela Joab, et dit : L'épée dévorera-t-elle sans cesse ? Ne sais-tu pas qu'il y aura de l'amertume à la fin ? Jusqu'à quand tarderas-tu à dire au peuple qu'il cesse de poursuivre ses frères ?
\VS{27}Joab répondit : Dieu est vivant ! Si tu n'avais parlé ainsi, le peuple n'aurait pas cessé avant le matin de poursuivre ses frères.
\VS{28}Joab sonna du shofar, et tout le peuple s'arrêta ; ils ne poursuivirent plus Israël, et ils ne continuèrent plus à se battre.
\VS{29}Ainsi, Abner et ses gens marchèrent toute la nuit dans la plaine ; ils passèrent le Jourdain, traversèrent tout le Bithron, et arrivèrent à Mahanaïm.
\VS{30}Joab aussi revint de la poursuite d'Abner, et rassembla tout le peuple ; il manquait dix-neuf hommes des gens de David, et Asaël.
\VS{31}Mais les gens de David avaient frappé à mort trois cent soixante hommes de Benjamin, et des gens d'Abner.
\VS{32}Ils emportèrent Asaël, et l'ensevelirent dans le sépulcre de son père à Bethléhem. Joab et ses gens marchèrent toute la nuit et arrivèrent à Hébron au point du jour.
\Chap{3}
\TextTitle{L'autorité de David s'accroît\FTNTT{1 Ch. 3:1-4.}}
\VerseOne{}Or il y eut une longue guerre entre la maison de Saül et la maison de David. David devenait de plus en plus fort, et la maison de Saül allait en s'affaiblissant.
\VS{2}Il naquit à David des fils à Hébron. Son premier-né fut Amnon, d'Achinoam de Jizreel ;
\VS{3}le second, Kileab, d'Abigaïl de Carmel, femme de Nabal ; le troisième, Absalom, fils de Maaca, fille de Talmaï, roi de Gueschur ;
\VS{4}le quatrième, Adonija, fils de Haggith ; le cinquième, Schephathia, fils d'Abithal ;
\VS{5}et le sixième, Jithream, d'Egla, femme de David. Ce sont là ceux qui naquirent à David à Hébron.
\TextTitle{Abner fait alliance avec David}
\VS{6}Et il arriva que pendant la guerre entre la maison de Saül et la maison de David, Abner tint ferme pour la maison de Saül.
\VS{7}Or Saül avait eu une concubine, nommée Ritspa, fille d'Ajja. Et Isch-Boscheth dit à Abner : Pourquoi es-tu venu vers la concubine de mon père ?
\VS{8}Abner fut très irrité à cause du discours d'Isch-Boscheth, et il lui dit : Suis-je une tête de chien, au service de Juda ? Je fais aujourd'hui preuve de bienveillance envers la maison de Saül, ton père, envers ses frères et ses amis, je ne t'ai pas livré entre les mains de David, et c'est aujourd'hui que tu me reproches une faute avec cette femme ?
\VS{9}Que Dieu punisse sévèrement Abner, si je n'agis pas avec David selon ce que Yahweh a juré à David,
\VS{10}en disant qu'il ferait passer la royauté de la maison de Saül à la sienne, et qu'il établirait le trône de David sur Israël et sur Juda depuis Dan jusqu'à Beer-Schéba.
\VS{11}Isch-Boscheth n'osa pas répondre un seul mot à Abner, parce qu'il le craignait.
\VS{12}Abner envoya des messagers à David pour lui dire de sa part : A qui est le pays ? Fais alliance avec moi, et voici, ma main sera avec toi, pour tourner vers toi tout Israël.
\VS{13}David répondit : Je le veux bien ! Je ferai alliance avec toi ; je te demande seulement une chose, c'est que tu ne voies point ma face, à moins que tu n'amènes d'abord Mical, fille de Saül, quand tu viendras me voir.
\VS{14}Et David envoya des messagers à Isch-Boscheth, fils de Saül, pour lui dire : Rends-moi ma femme Mical, que j'ai épousée pour cent prépuces des Philistins.
\VS{15}Isch-Boscheth envoya et l'ôta à son mari Palthiel, fils de Laïsch.
\VS{16}Et son mari la suivit, marchant et pleurant continuellement après elle jusqu'à Bachurim. Alors Abner lui dit : Va, retourne-t'en ! Et il s'en retourna.
\VS{17}Abner parla aux anciens d'Israël, et leur dit : Vous désiriez autrefois avoir David pour roi ;
\VS{18}établissez-le maintenant, car Yahweh a parlé de David et a dit : C'est par David, mon serviteur, que je délivrerai mon peuple d'Israël de la main des Philistins et de la main de tous ses ennemis.
\VS{19}Abner parla aussi aux oreilles de ceux de Benjamin, puis il alla faire entendre expressément à David, qui était à Hébron, ce qui semblait bon aux yeux d'Israël et aux yeux de toute la maison de Benjamin.
\VS{20}Abner vint donc vers David à Hébron, accompagné de vingt hommes ; et David fit un festin à Abner et aux hommes qui étaient avec lui.
\VS{21}Abner dit à David : Je me lèverai, et je partirai pour rassembler tout Israël auprès du roi, mon seigneur ; ils feront alliance avec toi, et tu régneras selon le désir de ton âme. David renvoya Abner, qui s'en alla en paix.
\VS{22}Voici, les gens de David et Joab revinrent d'une excursion, et amenèrent avec eux un grand butin. Abner n'était plus avec David à Hébron, car David l'avait renvoyé, et il s'en était allé en paix.
\VS{23}Lorsque Joab et toute l'armée qui était avec lui revinrent, on fit ce rapport à Joab en ces mots : Abner, fils de Ner, est venu auprès du roi, qui l'a renvoyé, et il s'en est allé en paix.
\VS{24}Joab vint vers le roi, et dit : Qu'as-tu fait ? Voici, Abner est venu vers toi ; pourquoi l'as-tu ainsi renvoyé, en sorte qu'il s'en est allé ?
\VS{25}Tu connais Abner, fils de Ner ! C'est pour te tromper qu'il est venu, pour épier tes démarches, tes allées et venues, et pour savoir tout ce que tu fais.
\VS{26}Puis Joab, après avoir quitté David, envoya sur les traces d'Abner des messagers, qui le ramenèrent de la fosse de Sira, sans que David n'en sache rien.
\TextTitle{Mort d'Abner}
\VS{27}Lorsque Abner revint à Hébron, Joab le tira à l'écart au milieu de la porte, comme pour lui parler en secret ; et là il le frappa à la cinquième côte ; et ainsi Abner mourut à cause du sang d'Asaël, frère de Joab.
\VS{28}David apprit ce qui était arrivé et dit : Je suis à jamais innocent, mon royaume et moi, devant Yahweh, du sang d'Abner, fils de Ner.
\VS{29}Que ce sang retombe sur la tête de Joab, et sur toute la maison de son père ! Que soit retranchée la maison de Joab, qu'il y ait toujours un homme qui soit atteint d'un flux ou de la lèpre, ou qui s'appuie sur un bâton, ou qui tombe par l'épée, ou qui manque de pain !
\VS{30}Ainsi Joab et Abischaï, son frère, tuèrent Abner, parce qu'il avait tué Asaël, leur frère, à Gabaon, dans la bataille.
\VS{31}David dit à Joab et à tout le peuple qui était avec lui : Déchirez vos vêtements, ceignez-vous de sacs, et menez le deuil en marchant devant Abner ! Et le roi David marcha derrière le cercueil.
\VS{32}On ensevelit Abner à Hébron. Le roi éleva la voix et pleura sur la tombe d'Abner, et tout le peuple pleura.
\VS{33}Le roi fit une complainte sur Abner, et dit : Abner devait-il mourir comme meurt un insensé ?
\VS{34}Tes mains n'étaient pas liées, et tes pieds n'étaient pas mis dans des chaînes ! Tu es tombé comme on tombe devant les méchants. Et tout le peuple recommença à pleurer sur Abner.
\VS{35}Puis tout le peuple vint pour faire prendre quelque nourriture à David, pendant qu'il était encore jour ; mais David jura, en disant : Que Dieu me punisse sévèrement, si je goûte du pain ou quelque chose d'autre avant le coucher du soleil !
\VS{36}Tout le peuple l'entendit, et l'approuva, et tout le peuple trouva bon tout ce qu'avait fait le roi.
\VS{37}En ce jour, tout le peuple et tout Israël surent que ce n'était pas par ordre du roi qu'Abner, fils de Ner, avait été tué.
\VS{38}Le roi dit à ses serviteurs : Ne savez-vous pas qu'un chef, un grand homme, est tombé aujourd'hui en Israël ?
\VS{39}Je suis encore faible aujourd'hui, bien que j'aie été oint roi ; et ces gens, les fils de Tseruja, sont trop puissants pour moi. Que Yahweh rende à celui qui fait le mal selon sa méchanceté !
\Chap{4}
\TextTitle{Mort d'Ish-Boscheth}
\VerseOne{}Quand le fils de Saül apprit qu'Abner était mort à Hébron, ses mains restèrent sans force, et tout Israël fut dans l'épouvante.
\VS{2}Le fils de Saül avait deux chefs de bandes, dont l'un s'appelait Baana et l'autre Récab ; ils étaient fils de Rimmon de Beéroth, d'entre les fils de Benjamin. Car Beéroth était regardée comme appartenant à Benjamin,
\VS{3}et les Beérothiens s'étaient enfuis à Guitthaïm, où ils y ont habité jusqu'à ce jour.
\VS{4}Jonathan, fils de Saül, avait un fils perclus des pieds ; il était âgé de cinq ans lorsque la nouvelle de la mort de Saül et de Jonathan arriva de Jizreel ; sa nourrice le prit et s'enfuit, et comme elle se hâtait de fuir, il tomba et devint boiteux ; son nom était Mephiboscheth.
\VS{5}Les fils de Rimmon de Beéroth, Récab et Baana, se rendirent pendant la chaleur du jour à la maison d'Isch-Boscheth, qui était couché pour son repos du midi.
\VS{6}Ils pénétrèrent jusqu'au milieu de la maison, comme pour y prendre du froment, et ils le frappèrent à la cinquième côte ; puis Récab et Baana, son frère, se sauvèrent.
\VS{7}Ils entrèrent donc dans la maison lorsqu'Isch-Boscheth était couché sur son lit dans la chambre à coucher où il dormait, ils le frappèrent et le tuèrent, puis ils lui coupèrent la tête. Ils prirent sa tête, et ils marchèrent toute la nuit au travers de la plaine.
\VS{8}Ils apportèrent la tête d'Isch-Boscheth à David dans Hébron, et ils dirent au roi : Voici la tête d'Isch-Boscheth, fils de Saül, ton ennemi, qui en voulait à ta vie ; Yahweh venge aujourd'hui le roi mon seigneur de Saül et de sa race.
\VS{9}Mais David répondit à Récab et à Baana, son frère, fils de Rimmon de Beéroth, et leur dit : Yahweh, qui a délivré mon âme de toute angoisse est vivant !
\VS{10}J'ai saisi celui qui est venu m'annoncer et me dire : Voilà, Saül est mort, et qui pensait m'apprendre de bonnes nouvelles, je l'ai fait saisir et tuer à Tsiklag, pour lui donner le salaire de ses bonnes nouvelles ;
\VS{11}combien plus, quand des méchants ont tué un homme juste dans sa maison et sur sa couche, ne redemanderai-je pas maintenant son sang de vos mains et ne vous exterminerai-je pas de la terre ?
\VS{12}David ordonna à ses gens de les tuer ; ils leur coupèrent les mains et les pieds, et les pendirent près de l'étang d'Hébron. Ils prirent la tête d'Isch-Boscheth, et l'ensevelirent dans le sépulcre d'Abner à Hébron.
\Chap{5}
\TextTitle{David oint roi sur tout Israël\FTNTT{1 Ch. 11:1-3.}}
\VerseOne{}Alors toutes les tribus d'Israël vinrent auprès de David, à Hébron, et dirent : Voici, nous sommes tes os et ta chair.
\VS{2}Autrefois déjà, quand Saül était roi sur nous, c'est toi qui conduisais et qui ramenais Israël. Yahweh t'a dit : Tu paîtras mon peuple d'Israël, et tu seras le chef d'Israël.
\VS{3}Tous les anciens d'Israël vinrent donc vers le roi à Hébron, et le roi David fit alliance avec eux à Hébron, devant Yahweh. Ils oignirent David pour roi sur Israël.
\VS{4}David était âgé de trente ans lorsqu'il commença à régner ; il régna quarante ans.
\VS{5}Il régna sur Juda à Hébron sept ans et six mois, puis il régna trente-trois ans à Jérusalem sur tout Israël et Juda.
\TextTitle{Jérusalem, capitale de tout Israël\FTNTT{1 Ch. 11:4-9.}}
\VS{6}Le roi marcha avec ses gens sur Jérusalem contre les Jébusiens qui habitaient ce pays. Ils dirent à David : Tu n'entreras point ici, car les aveugles mêmes et les boiteux te repousseront ! Ce qui voulait dire : David n'entrera point ici.
\VS{7}Mais David s'empara de la forteresse de Sion : C'est la cité de David.
\VS{8}David avait dit en ce jour-là : Quiconque battra les Jébusiens et atteindra le canal, ces aveugles et ces boiteux, qui sont haïs de l'âme de David, sera récompensé… C'est pourquoi l'on dit : Aucun aveugle ni boiteux n'entrera dans cette maison.
\VS{9}Et David habita dans la forteresse, et l'appela la cité de David. Il bâtit tout autour, depuis Millo jusqu'au-dedans.
\VS{10}David devenait de plus en plus grand, et Yahweh, le Dieu des armées, était avec lui.
\TextTitle{Yahweh affermit le règne de David}
\VS{11}Hiram, roi de Tyr, envoya des messagers à David, du bois de cèdre, des charpentiers et des tailleurs de pierres à bâtir, et ils bâtirent la maison de David.
\VS{12}David reconnut que Yahweh l'affermissait comme roi sur Israël, et qu'il élevait son royaume à cause de son peuple d'Israël.
\TextTitle{Fils de David nés à Jérusalem\FTNTT{2 S. 3:2-5 ; 1 Ch. 3:1-4.}}
\VS{13}David prit encore des concubines et des femmes de Jérusalem, après qu'il fut venu d'Hébron, et il lui naquit encore des fils et des filles.
\VS{14}Voici les noms de ceux qui lui naquirent à Jérusalem : Schammua, Schobad, Nathan, Salomon,
\VS{15}Jibhar, Elischua, Népheg, Japhia,
\VS{16}Elischama, Eliada et Eliphéleth.
\TextTitle{Yahweh livre les Philistins à David\FTNTT{2 S. 23:13-17 ; 1 Ch. 14:8-17 ; 11:15-19 ; 12:8-15.}}
\VS{17}Or quand les Philistins apprirent qu'on avait oint David pour roi sur Israël, ils montèrent tous pour chercher David. Et David l'ayant appris, il descendit vers la forteresse.
\VS{18}Les Philistins arrivèrent et se répandirent dans la vallée des Rephaïm.
\VS{19}Alors David consulta Yahweh, en disant : Monterai-je contre les Philistins ? Les livreras-tu entre mes mains ? Et Yahweh parla à David : Monte, car certainement je livrerai les Philistins entre tes mains.
\VS{20}Alors David vint à Baal-Peratsim, où il les battit. Puis il dit : Yahweh a dispersé mes ennemis devant moi, comme des eaux qui s'écoulent. C'est pourquoi il nomma ce lieu-là Baal-Peratsim.
\VS{21}Ils laissèrent là leurs faux dieux que David et ses gens emportèrent.
\VS{22}Les Philistins montèrent encore une autre fois, et se répandirent dans la vallée des Rephaïm.
\VS{23}David consulta Yahweh. Et Yahweh dit : Tu ne monteras pas ; contourne-les par-derrière, et tu les atteindras vis-à-vis des mûriers.
\VS{24}Quand tu entendras un bruit comme des gens qui marchent au sommet des mûriers, alors hâte-toi, car c'est Yahweh qui sort devant toi pour battre l'armée des Philistins.
\VS{25}David fit ce que Yahweh lui avait ordonné, et il battit les Philistins depuis Guéba jusqu'à Guézer.
\Chap{6}
\TextTitle{Désobéissance dans le transport de l'arche\FTNTT{1 Ch. 13:1-14.}}
\VerseOne{}David rassembla encore toute l'élite d'Israël, au nombre de trente mille hommes.
\VS{2}Puis David se leva, ainsi que tout le peuple qui était avec lui, et se mit en marche de Baalé-Juda, pour faire monter de là l'arche de Dieu, qui est appelée du Nom, du Nom de Yahweh des armées, qui siège entre les chérubins.
\VS{3}Ils mirent l'arche\FTNT{L'arche ne devait être portée que par les Lévites. Les meilleures intentions pour le service de Yahweh ne suffisent pas pour que le Seigneur nous agrée. Nous devons nous conformer à la Parole de Dieu (1 R. 18:36-39).} de Dieu sur un char tout neuf, et l'emmenèrent de la maison d'Abinadab qui était sur la colline ; Uzza et Achjo, fils d'Abinadab, conduisaient le char neuf.
\VS{4}Ils l'emportèrent donc de la maison d'Abinadab sur la colline ; et Achjo allait devant l'arche.
\VS{5}David et toute la maison d'Israël jouaient devant Yahweh de toutes sortes d'instruments faits de bois de cyprès, de harpes, de luths, de tambourins, de sistres et de cymbales.
\VS{6}Quand ils furent arrivés à l'aire de Nacon, Uzza étendit la main vers l'arche de Dieu et la saisit parce que les bœufs la faisaient pencher.
\VS{7}La colère de Yahweh s'enflamma contre Uzza et Dieu le frappa là à cause de sa faute. Il mourut là, près de l'arche de Dieu.
\VS{8}David fut irrité de ce que Yahweh avait fait une brèche en la personne d'Uzza. C'est pourquoi on a appelé ce lieu jusqu'à ce jour Pérets-Uzza.
\VS{9}David eut peur de Yahweh en ce jour-là, et il dit : Comment l'arche de Yahweh entrerait-elle chez moi ?
\VS{10}David ne voulut pas déposer l'arche de Yahweh chez lui dans la cité de David, mais il la fit conduire dans la maison d'Obed-Edom de Gath.
\VS{11}L'arche de Yahweh resta trois mois dans la maison d'Obed-Edom de Gath, et Yahweh bénit Obed-Edom et toute sa maison.
\TextTitle{Accueil de l'arche à Jérusalem\FTNTT{1 Ch. 15:26-16:1.}}
\VS{12}Puis on vint dire au roi David : Yahweh a béni la maison d'Obed-Edom et tout ce qui lui appartient, pour l'amour de l'arche de Dieu. Alors David s'y rendit, et il fit monter l'arche de Dieu depuis la maison d'Obed-Edom jusqu'à la cité de David, au milieu des réjouissances.
\VS{13}Et il arriva que quand ceux qui portaient l'arche de Dieu eurent fait six pas, on sacrifia des taureaux et des béliers gras.
\VS{14}David dansait de toute sa force devant Yahweh, et il était ceint d'un éphod de lin.
\VS{15}Ainsi, David et toute la maison d'Israël firent monter l'arche de Yahweh avec des cris de joie et au son du shofar.
\VS{16}Comme l'arche de Yahweh entrait dans la cité de David, Mical, fille de Saül, regardait par la fenêtre, et voyant le roi David sauter et danser devant Yahweh, elle le méprisa en son cœur.
\VS{17}Ils amenèrent l'arche de Yahweh, et la posèrent au milieu de la tente que David avait dressée pour elle ; et David offrit des holocaustes et des sacrifices d'offrande de paix\FTNT{Voir commentaire en Lé. 3:1.} devant Yahweh.
\VS{18}Quand David eut achevé d'offrir des holocaustes et des sacrifices d'offrande de paix, il bénit le peuple au Nom de Yahweh des armées.
\VS{19}Et il partagea à tout le peuple, à toute la multitude d'Israël, tant aux hommes qu'aux femmes, à chacun un pain, une portion de viande, un gâteau de raisins, et une ration de vin. Puis tout le peuple s'en alla, chacun dans sa maison.
\VS{20}David s'en retourna pour bénir aussi sa maison, et Mical, fille de Saül, sortit à sa rencontre. Elle dit : Quel honneur s'est fait aujourd'hui le roi d'Israël, en se découvrant aux yeux des servantes et de ses serviteurs, comme se découvrirait un homme de néant sans en avoir honte !
\VS{21}David répondit à Mical : C'est devant Yahweh, qui m'a choisi plutôt que ton père et toute sa maison pour m'établir chef sur le peuple de Yahweh, sur Israël, c'est devant Yahweh que je me suis réjoui.
\VS{22}Je me rendrai encore plus insignifiant que je n'ai été cette fois, et je m'estimerai encore moins à mes propres yeux ; malgré cela, je serai en honneur auprès des servantes dont tu parles.
\VS{23}Or Mical, fille de Saül, n'eut point d'enfants jusqu'au jour de sa mort.
\Chap{7}
\TextTitle{David veut construire une maison à Yahweh\FTNTT{1 Ch. 17:1-2.}}
\VerseOne{}Et il arriva, lorsque le roi fut établi dans sa maison, et que Yahweh lui eut donné du repos de tous ses ennemis qui l'entouraient,
\VS{2}qu'il dit à Nathan le prophète : Regarde maintenant ! J'habite dans une maison de cèdres, et l'arche de Dieu habite sous des tapis\FTNT{Dans la plupart des versions, on a traduit ce mot par « tente », alors que le terme hébreu est « yeriy'ah », ce qui signifie « rideau », « drap », « tapis ». Voir Ex. 26.}.
\VS{3}Alors Nathan répondit au roi : Va, fais tout ce qui est dans ton cœur, car Yahweh est avec toi.
\TextTitle{Yahweh traite alliance avec David et sa postérité\FTNTT{1 Ch. 17:3-15.}}
\VS{4}Mais il arriva cette nuit-là que la parole de Yahweh fut adressée à Nathan, en disant :
\VS{5}Va, et dis à David, mon serviteur : Ainsi parle Yahweh : Me bâtirais-tu une maison afin que j'y habite ?
\VS{6}Puisque je n'ai point habité dans une maison depuis le jour où j'ai fait monter les enfants d'Israël hors d'Egypte jusqu'à ce jour ; mais j'ai marché ça et là sous une tente et dans un tabernacle.
\VS{7}Partout où j'ai marché avec tous les enfants d'Israël, ai-je dit un seul mot à quelqu'une des tribus d'Israël à qui j'avais ordonné de paître mon peuple d'Israël, ai-je dit : Pourquoi ne me bâtissez-vous pas une maison de cèdres ?
\VS{8}Maintenant tu diras à David, mon serviteur : Ainsi parle Yahweh des armées : Je t'ai pris d'une cabane, d'auprès des brebis, afin que tu sois le conducteur de mon peuple, Israël ;
\VS{9}j'ai été avec toi partout où tu as marché, j'ai exterminé tous tes ennemis devant toi, et j'ai rendu ton nom grand, comme le nom des grands qui sont sur la terre ;
\VS{10}j'ai établi une demeure à mon peuple, à Israël, et je l'ai planté pour qu'il y habite et ne soit plus agité, pour que les méchants ne l'affligent plus comme auparavant,
\VS{11}et comme du temps où j'avais établi des juges sur mon peuple d'Israël. Je t'ai accordé du repos face à tous tes ennemis. Et Yahweh t'annonce qu'il te bâtira une maison.
\VS{12}Quand tu seras endormi avec tes pères, je susciterai après toi, ton fils, qui sera sorti de tes entrailles, et j'affermirai son règne.
\VS{13}Ce sera lui qui bâtira une maison à mon Nom, et j'affermirai pour toujours le trône de son règne\FTNT{Le royaume millénaire était promis à David et à sa postérité. Il fut proclamé par Jean-Baptiste (Mt. 3:1-12), le Messie (Mt. 4:17) et les apôtres (Mt. 10:5-7) comme étant proche. Présentement, le royaume de Dieu se manifeste par la vie sanctifiée des saints en Christ (Lu. 17:20 ; Jn. 3:1-8 ; Ro. 14:17). Il n'apparaîtra pas de manière visible avant la «moisson», c'est-à-dire le jugement des nations (Mt. 13:39-50). En effet, ce n'est qu'après cette moisson que le royaume sera installé ici-bas, lorsque le Messie rétablira la monarchie et la dynastie de David en sa propre personne. Il rassemblera alors les enfants d'Israël dispersés dans le monde entier et établira sa domination sur toute la terre pendant mille ans. Ce royaume sera remis au Père par le Messie après avoir vaincu le dernier ennemi, c'est-à-dire la mort (1 Co. 15:24-26). De ce fait, personne ne mourra pendant le millénium. Toutes les nations monteront tous les ans à Jérusalem pour adorer Yahweh et célébrer la fête des tabernacles qui sera restaurée (Za. 14). Le gouvernement théocratique en Israël sera alors restauré (Es. 1:26).}.
\VS{14}Je serai pour lui un père, et il sera pour moi un fils. S'il fait le mal, je le châtierai avec une verge d'hommes et avec des plaies des fils des hommes ;
\VS{15}mais ma grâce ne se retirera point de lui, comme je l'ai retirée de Saül, que j'ai ôté de devant toi.
\VS{16}Ainsi ta maison et ton règne seront assurés à jamais devant tes yeux, et ton trône sera pour toujours affermi.
\VS{17}Nathan rapporta à David toutes ces paroles et toute cette vision.
\TextTitle{Louange et reconnaissance de David envers Yahweh\FTNTT{1 Ch. 17:16-27.}}
\VS{18}Alors le roi David alla se présenter devant Yahweh, et dit : Qui suis-je, Seigneur Yahweh, et quelle est ma maison, que tu m'aies fait arriver au point où je suis ?
\VS{19}C'est encore peu de choses à tes yeux, ô Seigneur Yahweh ! Car tu as même parlé sur la maison de ton serviteur pour les temps éloignés. Est-ce là la manière d'agir des hommes, ô Seigneur Yahweh ?
\VS{20}Et que pourrait dire de plus David ? Car, Seigneur Yahweh, tu connais ton serviteur !
\VS{21}Tu as fait toutes ces grandes choses pour l'amour de ta parole, et selon ton cœur, pour les révéler à ton serviteur.
\VS{22}C'est pourquoi tu t'es montré grand, ô Yahweh Dieu ! Car nul n'est semblable à toi, et il n'y a point d'autre Dieu que toi, d'après tout ce que nous avons entendu de nos oreilles.
\VS{23}Et qui est comme ton peuple, comme Israël, la seule nation de la terre que Dieu est venu racheter pour en faire son peuple, y mettre son Nom et pour accomplir dans ton pays, devant ton peuple que tu t'es racheté d'Egypte, des choses grandes et terribles contre les nations et contre leurs dieux ?
\VS{24}Tu as affermi ton peuple d'Israël pour qu'il soit ton peuple pour toujours ; et toi, Yahweh, tu es devenu son Dieu.
\VS{25}Maintenant donc, ô Yahweh Dieu, confirme pour toujours la parole que tu as prononcée sur ton serviteur et sur sa maison, et agis selon ta parole.
\VS{26}Que ton Nom soit à jamais glorifié, et que l'on dise : Yahweh des armées est le Dieu d'Israël ! Et que la maison de David, ton serviteur, demeure stable devant toi !
\VS{27}Car toi, Yahweh des armées, Dieu d'Israël, tu as révélé ces choses à l'oreille de ton serviteur, en disant : Je te bâtirai une maison ! C'est pourquoi ton serviteur a pris courage pour t'adresser cette prière.
\VS{28}Maintenant, Seigneur Yahweh, tu es Dieu, tes paroles sont vérité, et tu as promis cette grâce à ton serviteur.
\VS{29}Veuille donc bénir la maison de ton serviteur, afin qu'elle soit éternellement devant toi ! Car c'est toi, Seigneur Yahweh, qui a parlé, et par ta bénédiction la maison de ton serviteur sera comblée de bénédictions éternellement.
\Chap{8}
\TextTitle{Yahweh donne à David la victoire sur ses ennemis\FTNTT{1 Ch. 18:1-17.}}
\VerseOne{}Après cela, il arriva que David battit les Philistins et les humilia, et il prit Métheg-Amma de la main des Philistins.
\VS{2}Il battit aussi les Moabites, et les mesura au cordeau, en les faisant coucher par terre ; il en mesura deux cordeaux pour les faire mourir, et un plein cordeau pour leur laisser la vie. Et les Moabites furent assujettis à David, et lui payèrent un tribut.
\VS{3}David battit aussi Hadadézer, fils de Rehob, roi de Tsoba, lorsqu'il alla rétablir sa domination sur le fleuve de l'Euphrate.
\VS{4}David lui prit mille sept cents cavaliers, et vingt mille hommes de pied ; il coupa les jarrets aux chevaux de tous les chars, et ne conserva que cent attelages.
\VS{5}Les Syriens de Damas vinrent au secours d'Hadadézer, roi de Tsoba, et David battit vingt-deux mille Syriens.
\VS{6}David mit des garnisons dans la Syrie de Damas. Et les Syriens furent assujettis à David, et lui payèrent un tribut. Yahweh protégeait David partout où il allait.
\VS{7}Et David prit les boucliers d'or qui étaient aux serviteurs d'Hadadézer, et les apporta à Jérusalem.
\VS{8}Le roi David emporta aussi une grande quantité d'airain de Béthach, et de Bérothaï, villes d'Hadadézer.
\VS{9}Thoï, roi de Hamath, apprit que David avait battu toute l'armée d'Hadadézer,
\VS{10}et il envoya Joram, son fils, vers le roi David, pour le saluer et pour le féliciter d'avoir fait la guerre contre Hadadézer et de l'avoir battu. Car Hadadézer était continuellement en guerre avec Thoï. Joram apporta des vases d'argent, des vases d'or, et des vases d'airain.
\VS{11}Le roi David les consacra à Yahweh, avec l'argent et l'or qu'il avait déjà consacrés du butin de toutes les nations qu'il s'était assujetties,
\VS{12}de la Syrie, de Moab, des fils d'Ammon, des Philistins, d'Amalek, et du butin d'Hadadézer, fils de Rehob, roi de Tsoba.
\VS{13}Au retour de la défaite des Syriens, David se fit encore un nom, en battant dans la vallée du sel dix-huit mille Edomites.
\VS{14}Il mit des garnisons dans Edom, il mit des garnisons dans tout Edom. Et tout Edom fut assujetti à David. Yahweh protégeait David partout où il allait.
\VS{15}Ainsi David régna sur tout Israël, et il faisait droit et justice à tout son peuple.
\VS{16}Joab, fils de Tseruja, commandait l'armée ; Josaphat, fils d'Achilud, était archiviste ;
\VS{17}Tsadok, fils d'Achithub, et Achimélec, fils d'Abiathar, étaient sacrificateurs ; Seraja était secrétaire ;
\VS{18}Benaja, fils de Jehojada, était chef des Kéréthiens et des Péléthiens ; et les fils de David étaient ministres d'Etat.
\Chap{9}
\TextTitle{Mephiboscheth à la table de David}
\VerseOne{}Alors David dit : Ne reste-t-il donc personne de la maison de Saül, afin que je lui fasse du bien pour l'amour de Jonathan ?
\VS{2}Il y avait dans la maison de Saül un serviteur nommé Tsiba, que l'on fit venir auprès de David. Le roi lui dit : Es-tu Tsiba ? Et il répondit : Je suis ton serviteur !
\VS{3}Le roi dit : N'y a-t-il plus personne de la maison de Saül, pour que j'use envers lui de la bonté de Dieu ? Tsiba répondit au roi : Il y a encore un des fils de Jonathan, qui est perclus des pieds.
\VS{4}Le roi lui dit : Où est-il ? Et Tsiba répondit au roi : Il est dans la maison de Makir, fils d'Ammiel, à Lodebar.
\VS{5}Alors le roi David l'envoya chercher dans la maison de Makir, fils d'Ammiel, à Lodebar.
\VS{6}Quand Mephiboscheth, fils de Jonathan, fils de Saül, vint auprès de David, il tomba sur sa face et se prosterna. David dit : Mephiboscheth ! Et il répondit : Voici ton serviteur.
\VS{7}David lui dit : Ne crains point, car certainement je te ferai du bien pour l'amour de Jonathan, ton père. Je te restituerai toutes les terres de Saül, ton père, et tu mangeras toujours du pain à ma table.
\VS{8}Il se prosterna, et dit : Qui suis-je, moi ton serviteur, pour que tu regardes un chien mort tel que moi ?
\VS{9}Le roi appela Tsiba, serviteur de Saül, et lui dit : Je donne au fils de ton maître tout ce qui appartenait à Saül et à toute sa maison.
\VS{10}Tu cultiveras pour lui ces terres, toi, tes fils, et tes serviteurs, et tu en recueilleras les fruits, afin que le fils de ton maître ait du pain à manger ; et Mephiboscheth, fils de ton maître, mangera toujours du pain à ma table. Or Tsiba avait quinze fils et vingt serviteurs.
\VS{11}Tsiba dit au roi : Ton serviteur fera tout ce que le roi, mon seigneur, ordonne à son serviteur. Et Mephiboscheth mangea à la table de David comme l'un des fils du roi.
\VS{12}Mephiboscheth avait un jeune fils, nommé Mica, et tous ceux qui demeuraient dans la maison de Tsiba étaient serviteurs de Mephiboscheth.
\VS{13}Mephiboscheth habitait à Jérusalem parce qu'il mangeait toujours à la table du roi. Il était boiteux des deux pieds.
\Chap{10}
\TextTitle{Double bataille contre les Ammonites et les Syriens}
\VerseOne{}Or il arriva après cela que le roi des fils d'Ammon mourût, et Hanun, son fils, régna à sa place.
\VS{2}Et David dit : J'userai de bonté envers Hanun, fils de Nachasch, comme son père en a usé envers moi. Ainsi David lui envoya ses serviteurs pour le consoler au sujet de son père. Lorsque les serviteurs de David arrivèrent dans le pays des fils d'Ammon,
\VS{3}les chefs des fils d'Ammon dirent à Hanun, leur maître : Penses-tu que ce soit pour honorer ton père que David t'envoie des consolateurs ? N'est-ce pas pour reconnaître exactement la ville et pour l'épier, afin de la détruire, que David envoie ses serviteurs auprès de toi ?
\VS{4}Alors Hanun saisit les serviteurs de David, et fit raser la moitié de leur barbe, et couper la moitié de leurs habits jusqu'aux hanches. Puis il les renvoya.
\VS{5}David en fut informé et envoya des gens à leur rencontre, car ces hommes étaient accablés de honte ; et le roi leur fit dire : Restez à Jéricho jusqu'à ce que votre barbe ait repoussé, et revenez ensuite.
\VS{6}Les fils d'Ammon, voyant qu'ils s'étaient rendus odieux à David, firent enrôler à leur solde vingt mille hommes de pied chez les Syriens de Beth-Rehob, et chez les Syriens de Tsoba, mille hommes chez le roi de Maaca, et douze mille hommes chez les gens de Tob.
\VS{7}David l'ayant appris, envoya Joab et toute l'armée, les hommes les plus vaillants.
\VS{8}Les fils d'Ammon sortirent et se rangèrent en bataille à l'entrée de la porte ; les Syriens de Tsoba de Rehob, et les hommes de Tob et de Maaca étaient à part dans la campagne.
\VS{9}Joab, voyant que leur armée était tournée contre lui devant et derrière, choisit alors des hommes d'élite parmi tous ceux d'Israël, et les rangea contre les Syriens ;
\VS{10}et il donna le commandement du reste du peuple à Abischaï, son frère, pour le ranger en bataille contre les fils d'Ammon.
\VS{11}Il dit : Si les Syriens sont plus forts que moi, tu viendras à mon secours ; et si les fils d'Ammon sont plus forts que toi, j'irai te secourir.
\VS{12}Sois vaillant, et portons-nous vaillamment pour notre peuple et pour les villes de notre Dieu, et que Yahweh fasse ce qu'il lui semblera bon !
\VS{13}Alors Joab et le peuple qui était avec lui s'approchèrent pour livrer bataille aux Syriens, et ils s'enfuirent devant lui.
\VS{14}Quand les fils d'Ammon virent que les Syriens avaient pris la fuite, ils s'enfuirent aussi devant Abischaï et rentrèrent dans la ville. Joab s'éloigna des fils d'Ammon et revint à Jérusalem.
\VS{15}Les Syriens, voyant qu'ils avaient été battus par Israël, se rassemblèrent.
\VS{16}Hadarézer envoya chercher les Syriens qui étaient de l'autre côté du fleuve ; et ils arrivèrent à Hélam, et Schobac, chef de l'armée d'Hadarézer, les conduisait.
\VS{17}Cela fut rapporté à David, qui assembla tout Israël, passa le Jourdain, et vint à Hélam. Les Syriens se rangèrent en bataille contre David, et combattirent contre lui.
\VS{18}Mais les Syriens s'enfuirent devant Israël. Et David défit sept cents chars des Syriens et quarante mille cavaliers ; il frappa aussi Schobac, le chef de leur armée, qui mourut sur place.
\VS{19}Tous les rois soumis à Hadarézer, se voyant battus par Israël, firent la paix avec Israël et lui furent assujettis. Et les Syriens craignirent désormais de secourir les fils d'Ammon.
\Chap{11}
\TextTitle{Péché de David avec Bath-Schéba}
\VerseOne{}Et il arriva, l'année suivante, au temps où les rois partaient en guerre, que David envoya Joab, avec ses serviteurs et tout Israël, pour détruire les fils d'Ammon et assiéger Rabba. Mais David resta à Jérusalem\FTNT{Au lieu d'aller en guerre et de diriger les troupes, David resta à Jérusalem. Cette négligence l'a conduit à la convoitise, à l'adultère et au meurtre d'Urie. La distraction peut conduire à la mort. Il y a un temps pour toutes choses (Ec. 3).}.
\VS{2}Et il arriva, sur le soir, que David se leva de sa couche ; et comme il se promenait sur le toit de la maison royale, il aperçut de là une femme qui se baignait, et cette femme était très belle de figure.
\VS{3}David envoya demander qui était cette femme, et on lui dit : N'est-ce pas Bath-Schéba, fille d'Eliam, femme d'Urie, le Héthien ?
\VS{4}Et David envoya des messagers pour la chercher. Elle vint vers lui, et il coucha avec elle. Après s'être purifiée de sa souillure, elle retourna dans sa maison.
\VS{5}Cette femme devint enceinte, et elle fit dire à David : Je suis enceinte.
\VS{6}Alors David envoya dire à Joab : Envoie-moi Urie, le Héthien. Et Joab envoya Urie à David.
\VS{7}Urie se rendit auprès de David, qui l'interrogea sur l'état de Joab, sur l'état du peuple, et sur l'état de la guerre.
\VS{8}Puis David dit à Urie : Descends dans ta maison, et lave tes pieds. Urie sortit de la maison du roi, et on fit porter après lui un présent royal.
\VS{9}Mais Urie se coucha à la porte de la maison du roi, avec tous les serviteurs de son maître, et il ne descendit point dans sa maison.
\VS{10}On le rapporta à David, et on lui dit : Urie n'est pas descendu dans sa maison. David dit à Urie : N'arrives-tu pas de voyage ? Pourquoi n'es-tu pas descendu dans ta maison ?
\VS{11}Urie répondit à David : L'arche et Israël et Juda habitent sous des tentes, mon seigneur Joab et les serviteurs de mon seigneur campent aux champs, et moi j'entrerais dans ma maison pour manger et boire et pour coucher avec ma femme ! Tu es vivant, et ton âme est vivante, je ne ferai point une telle chose.
\VS{12}David dit à Urie : Reste ici encore aujourd'hui, et demain je te renverrai. Urie resta donc ce jour-là et le lendemain à Jérusalem.
\VS{13}David l'invita à manger et à boire en sa présence, et il l'enivra ; néanmoins le soir, Urie sortit pour dormir sur sa couche, avec tous les serviteurs de son maître, et il ne descendit point dans sa maison.
\VS{14}Le lendemain matin, David écrivit une lettre à Joab, et l'envoya par la main d'Urie.
\VS{15}Il écrivit en ces termes : Placez Urie à l'endroit où sera le plus fort de la bataille et éloignez-vous de lui, afin qu'il soit frappé et qu'il meure.
\VS{16}Joab, en observant la ville, plaça Urie à l'endroit qu'il savait défendu par de vaillants soldats.
\VS{17}Les hommes de la ville sortirent et combattirent contre Joab ; et quelques-uns du peuple qui étaient des serviteurs de David moururent, et Urie, le Héthien, mourut aussi.
\VS{18}Alors Joab envoya un messager à David pour lui faire savoir tout ce qui était arrivé dans ce combat.
\VS{19}Il donna cet ordre au messager : Quand tu auras achevé de raconter au roi tout ce qui est arrivé au combat,
\VS{20}peut-être se mettra-t-il en fureur et te dira : Pourquoi vous êtes-vous approchés de la ville pour combattre ? Ne savez-vous pas bien qu'on tire de dessus la muraille ?
\VS{21}Qui a tué Abimélec, fils de Jerubbéscheth ? N'est-ce pas une femme qui lança sur lui de dessus la muraille une pièce de meule de moulin, et n'en est-il pas mort à Thébets ? Pourquoi vous êtes-vous approchés de la muraille ? Alors tu lui diras : Ton serviteur Urie, le Héthien, est mort aussi.
\VS{22}Le messager partit. A son arrivée, il fit savoir à David tout ce pourquoi Joab l'avait envoyé.
\VS{23}Le messager dit à David : Ces gens ont été plus forts que nous; ils avaient fait une sortie contre nous dans les champs, mais nous les avons repoussés jusqu'à l'entrée de la porte ;
\VS{24}les archers ont tiré sur tes serviteurs du haut de la muraille, et plusieurs des serviteurs du roi ont été tués, ton serviteur Urie, le Héthien, est mort aussi.
\VS{25}David dit au messager : Tu diras ainsi à Joab : Ne sois point peiné de cette affaire, car l'épée dévore tantôt l'un, tantôt l'autre ; attaque vigoureusement la ville, et détruis-la. Et toi, encourage-le !
\VS{26}La femme d'Urie apprit qu'Urie, son mari, était mort, et elle pleura son mari.
\VS{27}Quand le deuil fut passé, David l'envoya chercher et la recueillit dans sa maison. Elle devint sa femme, et lui enfanta un fils. Ce que David avait fait était mal aux yeux de Yahweh.
\Chap{12}
\TextTitle{Le prophète Nathan envoyé pour reprendre David}
\VerseOne{}Yahweh envoya Nathan vers David. Nathan vint à lui, et lui dit : Il y avait deux hommes dans une ville, l'un riche et l'autre pauvre.
\VS{2}Le riche avait des brebis et des bœufs en très grand nombre.
\VS{3}Le pauvre n'avait rien du tout sauf une petite brebis, qu'il avait achetée ; il la nourrissait et elle grandissait chez lui avec ses enfants ; elle mangeait de son pain, buvait dans sa coupe, dormait sur son sein et elle était comme sa fille.
\VS{4}Un voyageur arriva chez l'homme riche. Ce riche a épargné ses brebis et ses bœufs, pour préparer un repas au voyageur qui était venu chez lui ; il a pris la brebis du pauvre homme, et l'a apprêtée pour l'homme qui était venu chez lui.
\VS{5}Alors la colère de David s'enflamma violemment contre cet homme, et il dit à Nathan : Yahweh est vivant ! L'homme qui a fait cela mérite la mort.
\VS{6}Parce qu'il a fait cela et qu'il n'a pas épargné cette brebis, pour une brebis il en rendra quatre.
\VS{7}Alors Nathan dit à David : Tu es cet homme-là ! Ainsi parle Yahweh, le Dieu d'Israël : Je t'ai oint pour roi sur Israël, et je t'ai délivré de la main de Saül ;
\VS{8}je t'ai même donné la maison de ton maître, et les femmes de ton maître dans ton sein, et je t'ai donné la maison d'Israël, et de Juda. Et si cela avait été peu, j'y aurais encore ajouté.
\VS{9}Pourquoi donc as-tu méprisé la parole de Yahweh, en faisant ce qui est mal à ses yeux ? Tu as frappé de l'épée Urie, le Héthien ; tu as pris sa femme pour en faire ta femme, et tu l'as tué par l'épée des fils d'Ammon.
\VS{10}Maintenant, l'épée ne s'éloignera jamais de ta maison, parce que tu m'as méprisé, et que tu as pris la femme d'Urie, le Héthien, pour en faire ta femme.
\VS{11}Ainsi parle Yahweh : Voici, je vais faire sortir de ta propre maison le malheur contre toi, et je vais prendre sous tes yeux tes propres femmes pour les donner à un homme de ta maison, qui couchera avec elles à la vue de ce soleil.
\VS{12}Car tu as agi en secret ; mais moi, je le ferai en présence de tout Israël et à la face du soleil.
\TextTitle{Repentance de David}
\VS{13}David dit à Nathan : J'ai péché contre Yahweh ! Et Nathan dit à David : Yahweh passe par-dessus ton péché, tu ne mourras point.
\VS{14}Toutefois, parce qu'en commettant cela, tu as donné l'occasion aux ennemis de Yahweh de le blasphémer, à cause de cela le fils qui t'est né mourra certainement.
\VS{15}Et Nathan retourna dans sa maison. Yahweh frappa l'enfant que la femme d'Urie avait enfanté à David, et il devint gravement malade.
\VS{16}David pria Dieu pour l'enfant, et David jeûna ; et quand il rentra, il passa la nuit couché par terre.
\VS{17}Les anciens de sa maison se levèrent et vinrent vers lui pour le faire lever de terre ; mais il ne voulut point, et il ne mangea rien avec eux.
\VS{18}Et il arriva que l'enfant mourut le septième jour. Les serviteurs de David craignaient de lui annoncer que l'enfant était mort. Car ils disaient : Voici, quand l'enfant vivait encore, nous lui avons parlé, et il n'a pas écouté notre voix ; comment donc lui dirions-nous : L'enfant est mort ? Il s'affligera bien davantage.
\VS{19}David vit que ses serviteurs parlaient à voix basse, et il comprit que l'enfant était mort. David dit à ses serviteurs : L'enfant est-il mort ? Ils répondirent : Il est mort.
\VS{20}Alors David se leva de terre. Il se lava, s'oignit, et changea de vêtements ; il alla dans la maison de Yahweh, et se prosterna. De retour chez lui, il demanda à manger ; on mit de la viande devant lui et il mangea.
\VS{21}Ses serviteurs lui dirent : Qu'est-ce que tu fais ? Tu jeûnais et pleurais pour l'amour de l'enfant lorsqu'il vivait encore ; et maintenant que l'enfant est mort, tu te lèves et tu manges !
\VS{22}Mais il répondit : Quand l'enfant vivait encore, je jeûnais et pleurais, car je disais : Qui sait si Yahweh n'aura pas pitié de moi et si l'enfant ne vivra pas ?
\VS{23}Maintenant qu'il est mort, pourquoi jeûnerais-je ? Puis-je le faire revenir ? J'irai vers lui, mais il ne reviendra pas vers moi.
\TextTitle{Naissance de Salomon}
\VS{24}David consola sa femme Bath-Schéba, et il alla auprès d'elle et coucha avec elle. Elle lui enfanta un fils qu'il nomma Salomon et qui fut aimé de Yahweh.
\VS{25}Il le remit entre les mains de Nathan, le prophète, qui lui donna le nom de Jedidja, à cause de Yahweh.
\TextTitle{Le pays et le roi de Rabba livrés à Joab et David (1 Ch. 20:1-3)}
\VS{26}Joab combattait contre Rabba, qui appartenait aux fils d'Ammon, il s'empara de la ville royale,
\VS{27}et envoya des messagers à David pour lui dire : J'ai attaqué Rabba, et j'ai pris la ville des eaux ;
\VS{28}rassemble maintenant le reste du peuple, campe contre la ville, et prends-la, de peur que je ne m'en empare et que la gloire m'en soit attribuée.
\VS{29}David rassembla tout le peuple et marcha contre Rabba ; il l'attaqua et la prit.
\VS{30}Il enleva la couronne de dessus la tête de son roi ; elle pesait un talent d'or et était garnie de pierres précieuses. On la mit sur la tête de David, qui emporta de la ville un très grand butin.
\VS{31}Il fit sortir aussi le peuple qui s'y trouvait, et il les plaça sous des scies, des herses de fer, des haches de fer et le fit passer par un fourneau où l'on cuit les briques ; il traita ainsi toutes les villes des fils d'Ammon. Puis David retourna avec tout le peuple à Jérusalem.
\Chap{13}
\TextTitle{David subit les conséquences de son péché}
\VerseOne{}Or il arriva après cela qu'Absalom, fils de David, avait une sœur qui était belle et qui se nommait Tamar ; et Amnon, fils de David, l'aima.
\TextTitle{Inceste au sein de la famille royale}
\VS{2}Et Amnon fut si tourmenté qu'il tomba malade à cause de Tamar sa sœur, car elle était vierge ; et il paraissait trop difficile à Amnon d'obtenir la moindre chose d'elle.
\VS{3}Amnon avait un ami, nommé Jonadab, fils de Schimea, frère de David, et Jonadab était un homme très rusé.
\VS{4}Il lui dit : Fils de roi, pourquoi maigris-tu ainsi de jour en jour ? Ne veux-tu pas me le dire ? Amnon lui dit : J'aime Tamar, la sœur de mon frère, Absalom.
\VS{5}Jonadab lui dit : Couche-toi dans ton lit et fais le malade. Quand ton père viendra te voir, tu lui diras : Permets à Tamar, ma sœur, de venir pour me donner à manger ; qu'elle prépare un mets sous mes yeux, afin que je le voie et que je le prenne de sa main.
\VS{6}Amnon se coucha et fit le malade. Le roi vint le voir, et Amnon dit au roi : Je te prie, que ma sœur Tamar vienne faire deux beignets sous mes yeux, et que je les mange de sa main.
\VS{7}David envoya dire à Tamar dans la maison : Va dans la maison de ton frère Amnon, et prépare-lui quelque chose d'appétissant.
\VS{8}Tamar alla dans la maison de son frère Amnon, qui était couché. Elle prit de la pâte, la pétrit, et en fit devant lui des beignets et les fit cuire.
\VS{9}Puis elle prit la poêle, et elle les versa devant lui. Mais Amnon refusa d'en manger. Il dit : Faites sortir tous ceux qui sont auprès de moi. Et tout le monde se retira.
\VS{10}Alors Amnon dit à Tamar : Apporte-moi le mets dans la chambre, et que je le mange de ta main. Tamar prit les beignets qu'elle avait faits, et les apporta à Amnon, son frère dans la chambre.
\VS{11}Comme elle les lui présentait pour qu'il en mange, il se saisit d'elle et lui dit : Viens, couche avec moi, ma sœur !
\VS{12}Elle lui répondit : Non, mon frère, ne me déshonore pas, car cela ne se fait point en Israël ; ne commets pas cette infamie.
\VS{13}Et moi, où irais-je avec mon opprobre ? Et toi, tu serais comme l'un des infâmes en Israël. Maintenant, je te prie, parle au roi, et il ne s'opposera pas à ce que je sois ta femme.
\VS{14}Mais il ne voulut pas écouter sa parole ; il fut plus fort qu'elle, lui fit violence et coucha avec elle\FTNT{Le viol et l'inceste que commit Amnon, fils de David, sur Tamar, sa demi-sœur, furent les conséquences du péché de David avec Bath-Schéba.}.
\VS{15}Après cela, Amnon eut pour elle une très grande haine, en sorte que la haine qu'il lui portait était plus grande que l'amour qu'il avait eu pour elle. Ainsi, Amnon lui dit : Lève-toi, va-t'en !
\VS{16}Elle lui répondit : Tu n'as aucune raison de me faire ce mal, que de me chasser, ce mal est plus grand que l'autre que tu m'as fait.
\VS{17}Mais il ne voulut point l'écouter, et appelant le garçon qui le servait, il dit : Qu'on chasse cette femme loin de moi, qu'on la mette dehors. Et ferme la porte après elle !
\VS{18}Elle était habillée d'une tunique de couleurs ; car les filles du roi, qui étaient encore vierges, s'habillaient ainsi. Le serviteur d'Amnon la mit dehors, et ferma la porte après elle.
\VS{19}Alors Tamar répandit de la cendre sur sa tête, et déchira sa tunique de couleurs ; elle mit la main sur sa tête, et s'en alla en poussant des cris.
\VS{20}Et son frère Absalom lui dit : Ton frère, Amnon, a-t-il été avec toi ? Maintenant, ma sœur, tais-toi, c'est ton frère ; ne prends pas cette affaire à cœur. Et Tamar, désolée, demeura dans la maison d'Absalom, son frère.
\VS{21}Quand le roi David eut appris toutes ces choses, il fut très irrité.
\VS{22}Absalom ne parla ni en bien ni en mal avec Amnon ; mais il le prit en haine, parce qu'il avait déshonoré Tamar, sa sœur.
\TextTitle{Vengeance d'Absalom sur Amnon}
\VS{23}Et il arriva au bout de deux années entières, qu'Absalom ayant les tondeurs à Baal-Hatsor, près d'Ephraïm, invita tous les fils du roi.
\VS{24}Absalom alla vers le roi, et dit : Voici, ton serviteur a les tondeurs ; je te prie que le roi et ses serviteurs viennent avec ton serviteur.
\VS{25}Et le roi dit à Absalom : Non, mon fils, nous n'irons pas tous, de peur que nous ne te soyons à charge. Absalom le pressa ; mais le roi ne voulut point aller, et il le bénit.
\VS{26}Absalom dit : Permets au moins à Amnon, mon frère, de venir avec nous. Le roi lui répondit : Pourquoi irait-il ?
\VS{27}Absalom le pressa tellement qu'il laissa aller Amnon et tous les fils du roi avec lui.
\VS{28}Or Absalom avait donné cet ordre à ses serviteurs, en disant : Prenez bien garde, je vous prie, quand le cœur d'Amnon sera égayé par le vin et que je vous dirai : Frappez Amnon ! Tuez-le ; ne craignez point, n'est-ce pas moi qui vous l'ordonne ? Fortifiez-vous et portez-vous en vaillants hommes !
\VS{29}Les serviteurs d'Absalom traitèrent Amnon comme Absalom l'avait ordonné. Et tous les fils du roi se levèrent, montèrent chacun sur son mulet, et s'enfuirent.
\VS{30}Et il arriva, comme ils étaient en chemin, que le bruit parvint à David qu'Absalom avait tué tous les fils du roi, et qu'il n'en était pas resté un seul d'entre eux.
\VS{31}Le roi se leva, déchira ses vêtements, et se coucha par terre ; et tous ses serviteurs étaient là, avec leurs vêtements déchirés.
\VS{32}Jonadab, fils de Schimea, frère de David, prit la parole, et dit : Que mon seigneur ne dise point que tous les jeunes hommes, fils du roi, ont été tués, car seul Amnon est mort ; car c'était là le dessein d'Absalom, depuis le jour où Amnon a violé Tamar, sa sœur ; car il a été exécuté selon son commandement.
\VS{33}Maintenant donc, que le roi mon seigneur ne prenne point la chose à cœur, en disant que tous les fils du roi sont morts, car Amnon seul est mort.
\VS{34}Absalom prit la fuite. Or le jeune homme placé en sentinelle leva les yeux et regarda. Et voici, un grand peuple venait par le chemin qui était derrière lui, du côté de la montagne.
\VS{35}Jonadab dit au roi : Voici les fils du roi qui arrivent ! Ainsi se confirme ce que disait ton serviteur.
\VS{36}Comme il achevait de parler, voici, les fils du roi arrivèrent. Ils élevèrent la voix et pleurèrent ; le roi aussi et tous ses serviteurs versèrent d'abondantes larmes.
\TextTitle{Absalom s'enfuit loin de son père}
\VS{37}Absalom s'était enfui, et il alla chez Talmaï, fils d'Ammihur, roi de Gueschur\FTNT{Absalom s'était réfugié chez Talmaï, roi de Gueschur (Transjordanie, au nord de la Syrie), qui était le père de Maaca, sa mère (2 S. 3:3). Il est donc allé chez son grand-père maternel.}. Et David pleurait tous les jours son fils.
\VS{38}Absalom resta trois ans à Gueschur, où il était allé, après avoir pris la fuite.
\VS{39}Le roi David cessa de poursuivre Absalom, car il était consolé de la mort d'Amnon.
\Chap{14}
\TextTitle{Joab convainc le roi de faire revenir Absalom}
\VerseOne{}Alors Joab, fils de Tseruja, s'aperçut que le cœur du roi était pour Absalom.
\VS{2}Il envoya chercher à Tekoa une femme habile, et il lui dit : Fais semblant de te lamenter, et revêts des habits de deuil ; ne t'oins pas d'huile, mais sois comme une femme qui depuis longtemps pleure un mort.
\VS{3}Ensuite va vers le roi, et tu lui parleras de cette manière. Joab lui mit dans la bouche ce qu'elle devait dire.
\VS{4}La femme de Tekoa alla parler au roi. Elle tomba la face contre terre, se prosterna et dit : Ô roi, sauve-moi !
\VS{5}Le roi lui dit : Qu'as-tu ? Elle répondit : Certainement, je suis une femme veuve, et mon mari est mort !
\VS{6}Or ta servante avait deux fils ; ils se sont tous deux querellés dans les champs, et il n'y avait personne pour les séparer ; l'un a frappé l'autre et l'a tué.
\VS{7}Et voici, toute la famille s'est élevée contre ta servante, en disant : Donne-nous le meurtrier de son frère ! Nous voulons le faire mourir, pour la vie de son frère qu'il a tué ; et que nous exterminions même l'héritier ! Ils veulent ainsi éteindre le charbon vif qui me restait, pour ne laisser à mon mari ni nom ni survivant sur la face de la terre.
\VS{8}Le roi dit à la femme : Va-t-en dans ta maison, et je donnerai des ordres en ta faveur.
\VS{9}Alors la femme de Tekoa dit au roi : Mon seigneur et mon roi ! Que l'iniquité soit sur moi et sur la maison de mon père, et que le roi et son trône en soient innocents.
\VS{10}Et le roi répondit : Si quelqu'un parle contre toi, amène-le-moi, et jamais il ne lui arrivera de te toucher.
\VS{11}Et elle dit : Je te prie, que le roi se souvienne de Yahweh, son Dieu, afin que le vengeur de sang n'augmente pas la ruine et qu'on ne fasse pas périr mon fils. Et il répondit : Yahweh est vivant ! Il ne tombera pas à terre un seul des cheveux de ton fils.
\VS{12}La femme dit : Je te prie que ta servante dise un mot au roi, mon seigneur. Et il répondit : Parle !
\VS{13}La femme dit : Mais pourquoi as-tu pensé une chose comme celle-ci contre le peuple de Dieu ? Puisqu'en tenant ce discours, le roi se déclare coupable en ce qu'il n'a pas fait revenir celui qu'il a banni ?
\VS{14}Car nous mourrons certainement, et nous sommes comme l'eau versée sur la terre qu'on ne peut recueillir. Dieu n'ôte pas la vie, mais il médite les moyens de ne pas repousser loin de lui celui qui est banni de sa présence.
\VS{15}Maintenant, si je suis venue pour tenir ce discours au roi, mon seigneur, c'est parce que le peuple m'a effrayée. Et ta servante a dit : Je veux parler maintenant au roi ; peut-être que le roi fera ce que sa servante lui dira.
\VS{16}Oui, car le roi écoutera sa servante pour la délivrer de la main de celui qui veut nous exterminer, moi et mon fils, de l'héritage de Dieu.
\VS{17}Ta servante a dit : Que la parole du roi, mon seigneur, nous apporte du repos. Car le roi mon seigneur est comme un ange de Dieu, pour entendre le bien et le mal. Que Yahweh, ton Dieu, soit avec toi !
\VS{18}Le roi répondit, et dit à la femme : Je te prie, ne me cache rien de ce que je vais te demander. Et la femme dit : Que le roi mon seigneur parle !
\VS{19}Et le roi dit : La main de Joab n'est-elle pas avec toi dans tout ceci ? Et la femme répondit et dit : Ton âme vit, ô mon seigneur, qu'on ne saurait se détourner ni à droite ni à gauche de tout ce que dit le roi mon seigneur. C'est en effet ton serviteur Joab qui m'a donné des ordres et qui a mis dans la bouche de ta servante toutes ces paroles.
\VS{20}C'est ton serviteur Joab qui a fait que j'ai ainsi tourné ce discours. Mais mon seigneur est sage comme un ange de Dieu, pour savoir tout ce qui se passe sur la terre.
\TextTitle{Retour d'Absalom à Jérusalem}
\VS{21}Alors le roi dit à Joab : Voici, maintenant c'est toi qui as conduit cette affaire ; va donc, et fais revenir le jeune homme Absalom.
\VS{22}Et Joab tomba la face contre terre et se prosterna, et il bénit le roi. Puis il dit : Aujourd'hui, ton serviteur sait qu'il a trouvé grâce à tes yeux, ô roi mon seigneur, puisque le roi agit selon ce que son serviteur lui a dit.
\VS{23}Joab se leva et partit pour Gueschur, et il ramena Absalom à Jérusalem.
\VS{24}Mais le roi dit : Qu'il se retire dans sa maison, et qu'il ne voie point ma face. Et Absalom se retira dans sa maison, et ne vit point la face du roi.
\VS{25}Il n'y avait point d'homme dans tout Israël aussi renommé qu'Absalom pour sa beauté ; depuis la plante des pieds jusqu'au sommet de la tête, il n'y avait point en lui de défaut.
\VS{26}Et quand il faisait couper ses cheveux, or il arrivait tous les ans qu'il les faisait couper, parce que sa chevelure lui pesait trop, le poids de sa chevelure était de deux cents sicles, poids du roi.
\VS{27}Il naquit à Absalom trois fils, et une fille nommée Tamar, qui était une femme belle de figure.
\VS{28}Et Absalom demeura deux ans entiers à Jérusalem, sans voir la face du roi.
\VS{29}Absalom fit demander Joab, pour l'envoyer vers le roi ; mais Joab ne voulut pas venir vers lui ; il le fit demander encore pour la seconde fois ; mais Joab ne voulut point venir.
\VS{30}Absalom dit alors à ses serviteurs : Voyez le champ de Joab qui est à côté du mien ; il y a de l'orge ; allez et mettez-y le feu. Et les serviteurs d'Absalom mirent le feu au champ.
\VS{31}Alors Joab se leva et vint vers Absalom dans sa maison. Il lui dit : Pourquoi tes serviteurs ont-ils mis le feu à mon champ?
\VS{32}Et Absalom répondit à Joab : Voici, je t'ai fait dire : Viens ici, et je t'enverrai vers le roi, afin que tu lui dises : Pourquoi suis-je revenu de Gueschur ? Il vaudrait mieux pour moi que j'y fusse encore. Je désire maintenant voir la face du roi ; et s'il y a de l'iniquité en moi, qu'il me fasse mourir.
\VS{33}Joab alla vers le roi, et lui rapporta cela. Et le roi appela Absalom, qui vint vers lui et se prosterna le visage contre terre devant le roi. Le roi embrassa Absalom.
\Chap{15}
\TextTitle{Mauvaises intentions d'Absalom}
\VerseOne{}Or il arriva qu'après cela, Absalom se procura des chars et des chevaux, et il avait cinquante hommes qui couraient devant lui\FTNT{La révolte d'Absalom était une autre conséquence du péché de David avec Bath-Schéba.}.
\VS{2}Absalom se levait de bon matin et se tenait au bord du chemin de la porte. Et chaque fois qu'un homme ayant une contestation se rendait auprès du roi pour obtenir justice, Absalom l'appelait, et lui disait : De quelle ville es-tu ? Et il répondait : Ton serviteur est de l'une des tribus d'Israël.
\VS{3}Absalom lui disait : Vois, ta cause est bonne et droite ; mais personne de chez le roi ne t'écoutera.
\VS{4}Absalom disait encore : Qui m'établira juge dans le pays ? Tout homme qui aurait une contestation et un procès viendrait vers moi, et je lui ferais justice.
\VS{5}Et il arrivait aussi que quand quelqu'un s'approchait de lui pour se prosterner, il lui tendait sa main, le saisissait, et l'embrassait.
\VS{6}Absalom faisait ainsi à tous ceux d'Israël qui venaient vers le roi pour demander justice. Et Absalom gagnait les cœurs des hommes d'Israël.
\TextTitle{Conspiration d'Absalom}
\VS{7}Et il arriva qu'au bout de quarante ans, Absalom dit au roi : Permets que j'aille à Hébron, pour accomplir le vœu que j'ai fait à Yahweh.
\VS{8}Car quand ton serviteur demeurait à Gueschur en Syrie, il fit un vœu, en disant : Si Yahweh me ramène à Jérusalem, j'en témoignerai ma reconnaissance à Yahweh.
\VS{9}Et le roi lui répondit : Va en paix. Et Absalom se leva et s'en alla à Hébron.
\VS{10}Absalom envoya des espions dans toutes les tribus d'Israël, pour dire : Aussitôt que vous entendrez le son du shofar, vous direz : Absalom est établi roi à Hébron !
\VS{11}Deux cents hommes de Jérusalem, qui avaient été invités, s'en allèrent avec Absalom ; ils y allèrent en toute simplicité de coeur, ne sachant rien de cette affaire.
\VS{12}Pendant qu'Absalom offrait les sacrifices, il envoya chercher à la ville de Guilo, Achitophel, le Guilonite, conseiller de David. Il se forma une puissante conspiration, parce que le peuple était de plus en plus nombreux auprès d'Absalom.
\TextTitle{David fuit son fils Absalom}
\VS{13}Un messager se rendit auprès de David, et lui dit : Le cœur des hommes d'Israël s'est tourné vers Absalom.
\VS{14}Et David dit à tous ses serviteurs qui étaient avec lui à Jérusalem : Levez-vous, fuyons, car nous ne pourrons échapper à Absalom. Hâtez-vous de partir ; sinon, il ne tarderait pas à nous atteindre, et il nous précipiterait dans le malheur et frapperait la ville du tranchant de l'épée.
\VS{15}Les serviteurs du roi lui répondirent : Tes serviteurs feront tout ce que le roi, notre seigneur, voudra.
\VS{16}Le roi sortit, et toute sa maison le suivait, mais le roi laissa dix femmes, des concubines, pour garder la maison.
\VS{17}Le roi sortit, et tout le peuple le suivait, et ils s'arrêtèrent à Beth-Merkhak.
\VS{18}Tous ses serviteurs marchaient à côté de lui ; tous les Kéréthiens, tous les Péléthiens, et tous les Gathiens, qui étaient six cents hommes venus de Gath, pour être à sa suite, marchaient devant le roi.
\VS{19}Mais le roi dit à Ittaï de Gath : Pourquoi viendrais-tu aussi avec nous ? Retourne et reste avec le roi, car tu es étranger, et même tu vas retourner bientôt en ton lieu.
\VS{20}Tu es arrivé hier, et te ferais-je aujourd'hui errer çà et là avec nous ? Quant à moi, je m'en vais où je pourrai ! Retourne et emmène tes frères avec toi. Que la bonté et la vérité t'accompagnent !
\VS{21}Mais Ittaï répondit au roi, et dit : Yahweh est vivant, et le roi mon seigneur est vivant ! Quel que soit le lieu où le roi mon seigneur sera, soit pour mourir, soit pour vivre, ton serviteur y sera aussi.
\VS{22}David donc dit à Ittaï : Viens, et marche ! Alors Ittaï de Gath marcha avec tous ses gens et tous les enfants qui étaient avec lui.
\VS{23}Et tout le pays pleurait à grands cris et tout le peuple passait plus avant. Puis le roi passa le torrent de Cédron, et tout le peuple passa en face du chemin qui mène au désert.
\TextTitle{L'arche de l'alliance à Jérusalem}
\VS{24}Tsadok était aussi là, et avec lui tous les Lévites portant l'arche de l'alliance de Dieu ; et ils posèrent là l'arche de Dieu, et Abiathar montait, pendant que tout le peuple achevait de sortir de la ville.
\VS{25}Le roi dit à Tsadok : Rapporte l'arche de Dieu dans la ville. Si je trouve grâce aux yeux de Yahweh, il me ramènera, et il me fera voir l'arche et sa demeure.
\VS{26}Mais s'il dit : Je ne prends point de plaisir en toi ! Me voici, qu'il fasse de moi ce qui lui semblera bon.
\VS{27}Le roi dit encore au sacrificateur Tsadok : N'es-tu pas le voyant ? Retourne en paix dans la ville, avec Achimaats, ton fils, et Jonathan, fils d'Abiathar, vos deux fils.
\VS{28}Voyez, j'attendrai dans les plaines du désert, jusqu'à ce qu'on vienne m'apporter des nouvelles de votre part.
\VS{29}Ainsi, Tsadok et Abiathar rapportèrent l'arche de Dieu à Jérusalem, et ils y restèrent.
\VS{30}David monta par la montée des oliviers. Il montait en pleurant, la tête couverte, et marchait pieds nus ; tout le peuple qui était avec lui se couvrit aussi la tête, et il montait en pleurant.
\VS{31}Alors on vint dire à David : Achitophel est parmi ceux qui ont conspiré avec Absalom. Et David dit : Je te prie, ô Yahweh, abolis les conseils d'Achitophel !
\TextTitle{Huschaï, espion pour David dans la cour d'Absalom}
\VS{32}Et il arriva que quand David fut arrivé au sommet de la montagne, où il se prosterna devant Dieu, Huschaï, l'Arkien, vint au-devant de lui, la tunique déchirée et de la terre sur sa tête.
\VS{33}David lui dit : Tu me seras à charge si tu viens avec moi.
\VS{34}Et au contraire, tu anéantiras en ma faveur les conseils d'Achitophel, si tu retournes à la ville, et que tu dis à Absalom : Ô roi, je serai ton serviteur, comme je fus autrefois le serviteur de ton père ; mais maintenant je serai ton serviteur.
\VS{35}Les sacrificateurs Tsadok et Abiathar ne seront-ils pas là avec toi ? Tout ce que tu entendras de la maison du roi, tu le rapporteras aux sacrificateurs Tsadok et Abiathar.
\VS{36}Voici, ils ont là avec eux leurs deux fils, Achimaats, fils de Tsadok, et Jonathan, fils d'Abiathar ; c'est par eux que vous me ferez savoir tout ce que vous aurez entendu.
\VS{37}Huschaï, l'ami de David, retourna donc dans la ville, et Absalom entra à Jérusalem.
\Chap{16}
\TextTitle{Tsiba retrouve David en fuite}
\VerseOne{}Quand David eut un peu dépassé le sommet, voici, Tsiba, serviteur de Mephiboscheth, vint au-devant de lui avec deux ânes bâtés, sur lesquels il y avait deux cents pains, cent paquets de raisins secs, cent de fruits d'été, et une outre de vin.
\VS{2}Le roi dit à Tsiba : Que veux-tu faire de cela ? Et Tsiba répondit : Les ânes serviront de montures pour la maison du roi, le pain et les autres fruits d'été sont pour nourrir les jeunes gens, et le vin pour désaltérer ceux qui se seront fatigués dans le désert.
\VS{3}Le roi lui dit : Mais où est le fils de ton maître ? Et Tsiba répondit au roi : Voici, il est resté à Jérusalem, car il a dit : Aujourd'hui, la maison d'Israël me rendra le royaume de mon père.
\VS{4}Alors le roi dit à Tsiba : Voici, tout ce qui est à Mephiboscheth est à toi. Et Tsiba dit : Je me prosterne ! Que je trouve grâce à tes yeux, ô roi, mon seigneur !
\TextTitle{Schimeï maudit le roi David}
\VS{5}Le roi David était arrivé jusqu'à Bachurim. Et voici, il sortit de là un homme de la famille et de la maison de Saül, nommé Schimeï, fils de Guéra. Il s'avança en prononçant des malédictions,
\VS{6}il jeta des pierres contre David, contre tous ses serviteurs, et contre tout le peuple ; tous les hommes vaillants étaient à la droite et à la gauche du roi.
\VS{7}Schimeï parlait ainsi en le maudissant : Sors, sors, homme de sang, méchant homme !
\VS{8}Yahweh fait retomber sur toi tout le sang de la maison de Saül, à la place duquel tu régnais, et Yahweh a mis le royaume entre les mains de ton fils, Absalom ; et voilà, tu souffres le mal que tu as fait, parce que tu es un homme de sang !
\VS{9}Alors Abischaï, fils de Tseruja, dit au roi : Pourquoi ce chien mort maudit-il le roi, mon seigneur ? Permets que je m'avance et que je lui ôte la tête.
\VS{10}Mais le roi répondit : Qu'ai-je à faire avec vous, fils de Tseruja ? S'il maudit, c'est que Yahweh lui a dit : Maudis David ! Qui donc lui dira : Pourquoi agis-tu ainsi ?
\VS{11}Et David dit à Abischaï et à tous ses serviteurs : Voici, mon propre fils, qui est sorti de mes entrailles, en veut à ma vie ; à plus forte raison ce Benjamite ! Laissez-le, et qu'il maudisse, car Yahweh lui a parlé.
\VS{12}Peut-être Yahweh regardera mon affliction, et que Yahweh me rendra le bien au lieu des malédictions d'aujourd'hui.
\VS{13}David donc, et ses gens, continuèrent leur chemin. Et Schimeï marchait sur le flanc de la montagne vis-à-vis de lui, continuant à maudire, jetant des pierres contre lui et de la poussière en l'air.
\VS{14}Le roi David et tout le peuple qui était avec lui arrivèrent fatigués et là ils se rafraîchirent.
\TextTitle{Abominations d'Absalom à Jérusalem}
\VS{15}Absalom, et tout le peuple, les hommes d'Israël, étaient entrés dans Jérusalem ; et Achitophel était avec lui.
\VS{16}Quand Huschaï, l'Arkien, ami de David, fut arrivé auprès d'Absalom, il lui dit : Vive le roi ! Vive le roi !
\VS{17}Et Absalom dit à Huschaï : Est-ce donc là l'affection que tu as pour ton ami ? Pourquoi n'es-tu pas allé avec ton ami ?
\VS{18}Huschaï répondit à Absalom : Non, mais je serai à celui qui a été choisi par Yahweh, par ce peuple et par tous les hommes d'Israël, et je demeurerai avec lui.
\VS{19}D'ailleurs, qui servirai-je ? Ne sera-ce pas son fils ? Je serai ton serviteur, comme j'ai été le serviteur de ton père.
\VS{20}Absalom dit à Achitophel : Donnez un conseil sur ce que nous ferons.
\VS{21}Achitophel dit à Absalom : Va vers les concubines que ton père a laissées pour garder la maison ; ainsi tout Israël saura que tu t'es rendu odieux envers ton père, et les mains de tous ceux qui sont avec toi se fortifieront.
\VS{22}On dressa une tente pour Absalom sur le toit et Absalom alla vers les concubines de son père, aux yeux de tout Israël\FTNT{2 S. 12:11-12.}.
\VS{23}Les conseils que donnait Achitophel en ce temps-là étaient autant estimés que si l'on eût demandé la parole de Dieu. C'est ainsi qu'on considérait tous les conseils qu'Achitophel donnait, tant à David qu'à Absalom.
\Chap{17}
\TextTitle{Schimeï maudit le roi David}
\VerseOne{}Après cela, Achitophel dit à Absalom : Je choisirai maintenant douze mille hommes, et je me lèverai, et je poursuivrai David cette nuit.
\VS{2}Je l'atteindrai pendant qu'il est fatigué, et que ses mains sont affaiblies ; je l'épouvanterai tellement que tout le peuple qui est avec lui s'enfuira, et je frapperai seulement le roi ;
\VS{3}et je ramènerai à toi tout le peuple ; car l'homme que tu cherches vaut autant que si tous retournaient à toi ; ainsi tout le peuple sera en paix.
\VS{4}Cette parole plut à Absalom, et à tous les anciens d'Israël.
\VS{5}Cependant Absalom dit : Qu'on appelle maintenant aussi Huschaï, l'Arkien, et que nous entendions aussi son avis.
\VS{6}Huschaï vint vers Absalom et Absalom lui dit : Achitophel a donné un tel avis ; devons-nous faire ce qu'il a dit ou non ? Parle, toi aussi.
\VS{7}Alors Huschaï dit à Absalom : Cette fois, le conseil qu'Achitophel a donné n'est pas bon.
\VS{8}Huschaï dit encore : Tu connais ton père et ses gens, ce sont des hommes forts, et ils ont l'amertume dans l'âme comme une ourse des champs privée de ses petits. Ton père est un homme de guerre, il ne passera pas la nuit avec le peuple.
\VS{9}Voici, il est maintenant caché dans quelque fosse, ou dans quelque autre lieu ; et si, dès le commencement, il en est qui tombent sous leurs coups, on ne tardera pas à l'apprendre et l'on dira : Il y a une défaite parmi le peuple qui suit Absalom !
\VS{10}Alors le plus vaillant, celui-là même qui avait le cœur comme un lion, se découragera ; car tout Israël sait que ton père est un homme de cœur, et que ceux qui sont avec lui sont vaillants.
\VS{11}Je conseille donc que tout Israël se rassemble auprès de toi, depuis Dan jusqu'à Beer-Schéba, multitude pareille au sable qui est sur le bord de la mer, et qu'en personne tu marches au combat.
\VS{12}Alors nous viendrons à lui en quelque lieu que nous le trouvions, et nous nous jetterons sur lui, comme la rosée tombe sur la terre ; et il ne lui restera aucun de tous les hommes qui sont avec lui.
\VS{13}S'il se retire dans une ville, tout Israël portera des cordes vers cette ville-là, et nous la traînerons jusqu'au torrent, jusqu'à ce qu'on n'en trouve plus une pierre.
\VS{14}Alors Absalom et tous les hommes d'Israël dirent : Le conseil de Huschaï, l'Arkien, est meilleur que le conseil d'Achitophel. Car Yahweh avait résolu de dissiper le conseil d'Achitophel, qui était bon, afin de faire venir le mal sur Absalom.
\TextTitle{Huschaï avertit David du danger}
\VS{15}Alors Huschaï dit aux sacrificateurs Tsadok et Abiathar : Achitophel a donné tel et tel conseil à Absalom, et aux anciens d'Israël ; mais moi, j'ai conseillé telle et telle chose.
\VS{16}Maintenant donc, envoyez tout de suite informer David, en disant : Ne passe point la nuit dans les plaines du désert, mais va plus loin, de peur que le roi et tout le peuple qui est avec lui ne soient exposés au péril.
\VS{17}Jonathan et Achimaats se tenaient à En-Roguel (la fontaine du foulon). Une servante vint leur dire d'aller informer le roi David ; car ils n'osaient pas se montrer et entrer dans la ville.
\VS{18}Mais un garçon les aperçut, et le rapporta à Absalom. Et ils partirent tous deux en hâte et ils arrivèrent à Bachurim, à la maison d'un homme qui avait un puits dans sa cour, dans lequel ils descendirent.
\VS{19}La femme de cet homme prit une couverture, qu'elle étendit sur l'ouverture du puits, et y répandit dessus du grain pilé en sorte qu'on ne s'aperçut de rien.
\VS{20}Les serviteurs d'Absalom entrèrent dans la maison auprès de cette femme, et lui dirent : Où sont Achimaats et Jonathan ? La femme leur répondit : Ils ont passé le ruisseau. Ils cherchèrent, et ne les trouvant pas, ils retournèrent à Jérusalem.
\VS{21}Après leur départ, Achimaats et Jonathan remontèrent du puits et allèrent informer le roi David. Ils lui dirent : Levez-vous, et hâtez-vous de passer l'eau, car Achitophel a conseillé telle chose contre vous.
\VS{22}Alors David et tout le peuple qui était avec lui se levèrent et ils passèrent le Jourdain ; à la lumière du matin, il n'en manqua pas un qui n'eût passé le Jourdain.
\VS{23}Or Achitophel voyant qu'on n'avait point fait ce qu'il avait conseillé, fit seller son âne, se leva, et s'en alla en sa maison, dans sa ville. Après avoir donné des ordres à sa maison, il s'étrangla et mourut. On l'enterra dans le sépulcre de son père.
\TextTitle{Absalom et Israël en marche contre David}
\VS{24}David arriva à Mahanaïm. Et Absalom passa le Jourdain, lui et tous les hommes d'Israël avec lui.
\VS{25}Absalom établit Amasa sur l'armée, à la place de Joab. Or Amasa était fils d'un homme nommé Jithra, l'Israélite, qui était allé vers Abigaïl, fille de Nachasch, et soeur de Tseruja, mère de Joab.
\VS{26}Israël et Absalom campèrent dans le pays de Galaad.
\TextTitle{Mahanaïm bienveillant envers David}
\VS{27}Or il arriva qu'aussitôt que David fut arrivé à Mahanaïm, Schobi, fils de Nachasch de Rabba, des fils d'Ammon, Makir, fils d'Ammiel de Lodebar, et Barzillaï, le Galaadite de Roguelim,
\VS{28}apportèrent des lits, des bassins, des vases de terre, du froment, de l'orge, de la farine, du grain rôti, des fèves, des lentilles, des pois rôtis,
\VS{29}du miel, de la crème, des brebis, et des fromages de vache. Ils apportèrent ces choses à David et au peuple qui était avec lui, afin qu'ils mangent, car ils disaient : Ce peuple a dû souffrir de la faim, de la fatigue et de la soif dans le désert.
\Chap{18}
\TextTitle{Bataille dans la forêt d'Ephraïm ; instructions de David sur Absalom}
\VerseOne{}David fit le dénombrement du peuple qui était avec lui, et il établit sur eux des chefs de milliers et des chefs de centaines.
\VS{2}David envoya le peuple, un tiers sous le commandement de Joab, un tiers sous le commandement d'Abischaï, fils de Tseruja, frère de Joab, et un tiers sous le commandement d'Ittaï, de Gath. Et le roi dit au peuple : Moi aussi, je veux sortir avec vous.
\VS{3}Mais le peuple lui dit : Tu ne sortiras point ! Car si nous prenons la fuite, ce n'est pas sur nous que l'attention se portera ; et même quand la moitié d'entre nous y serait tuée, on n'y ferait pas attention ; mais toi, tu es comme dix mille de nous, et maintenant il vaut mieux que de la ville tu puisses venir à notre secours.
\VS{4}Le roi leur répondit : Je ferai ce qui est bon à vos yeux. Le roi s'arrêta donc à la place de la porte, pendant que tout le peuple sortait par centaines et par milliers.
\VS{5}Le roi donna cet ordre à Joab, à Abischaï, et à Ittaï, et dit : Epargnez-moi le jeune homme Absalom ! Et tout le peuple entendit ce que le roi commandait à tous les chefs au sujet d'Absalom.
\VS{6}Ainsi le peuple sortit dans les champs à la rencontre d'Israël, et la bataille eut lieu dans la forêt d'Ephraïm.
\VS{7}Là, le peuple d'Israël fut battu par les serviteurs de David, et il y eut en ce jour-là dans ce même lieu, une grande défaite de vingt mille hommes.
\VS{8}La bataille s'étendit sur toute la contrée, et la forêt dévora ce jour-là beaucoup plus de peuple que l'épée.
\TextTitle{Joab tue Absalom}
\VS{9}Absalom se retrouva devant les serviteurs de David. Il était monté sur un mulet. Le mulet entra sous les branches entrelacées d'un grand chêne, et la tête d'Absalom fut prise dans le chêne ; il demeura suspendu entre le ciel et la terre, et le mulet qui était sous lui passa outre.
\VS{10}Un homme ayant vu cela, le rapporta à Joab, et lui dit : Voici, j'ai vu Absalom suspendu à un chêne.
\VS{11}Et Joab répondit à l'homme qui lui rapportait cela : Tu l'as vu ! Pourquoi ne l'as-tu pas tué là, le jetant par terre ? Je t'aurais donné dix sicles d'argent et une ceinture.
\VS{12}Mais cet homme dit à Joab : Quand je pèserais dans ma main mille pièces d'argent, je ne mettrais pas ma main sur le fils du roi ; car nous avons entendu ce que le roi vous a ordonné, à toi, à Abischaï et à Ittaï, en disant : Prenez garde chacun au jeune homme Absalom !
\VS{13}Autrement j'aurais commis une lâcheté au péril de ma vie, car rien ne serait caché au roi, et toi-même tu te lèverais contre moi.
\VS{14}Joab répondit : Je ne m'attarderai pas auprès de toi ! Et il prit en sa main trois javelots, et les enfonça dans le cœur d'Absalom qui était encore vivant au milieu du chêne.
\VS{15}Puis dix jeunes hommes, qui portaient les armes de Joab, entourèrent Absalom, le frappèrent et le firent mourir\FTNT{La mort d'Absalom fut une conséquence du péché de David avec Bath-Schéba. Le péché a donc des conséquences graves et cause beaucoup de souffrances.}.
\VS{16}Alors Joab fit sonner la trompette ; et le peuple cessa de poursuivre Israël, parce que Joab le retint.
\VS{17}Ils prirent Absalom, le jetèrent dans la forêt dans une grande fosse, et mirent sur lui un très grand monceau de pierres. Tout Israël s'enfuit, chacun dans sa tente.
\VS{18}Or Absalom s'était fait ériger, de son vivant, un monument dans la vallée du roi ; car il disait : Je n'ai point de fils pour conserver la mémoire de mon nom. Et il donna son propre nom au monument, qu'on appelle encore aujourd'hui la place d'Absalom.
\TextTitle{David apprend la mort d'Absalom}
\VS{19}Et Achimaats, fils de Tsadok, dit : Laisse-moi courir, et porter au roi la bonne nouvelle que Yahweh lui a rendu justice en jugeant ses ennemis.
\VS{20}Joab lui répondit : Tu ne seras pas aujourd'hui porteur de bonnes nouvelles ; tu le seras un autre jour ; car aujourd'hui tu ne porterais pas de bonnes nouvelles, puisque le fils du roi est mort.
\VS{21}Et Joab dit à Cuschi : Va, et annonce au roi ce que tu as vu. Cuschi se prosterna devant Joab, puis il se mit à courir.
\VS{22}Achimaats, fils de Tsadok, dit encore à Joab : Quoi qu'il arrive, laisse-moi courir après Cuschi. Joab lui dit : Pourquoi veux-tu courir, mon fils, puisque tu n'as pas de bonnes nouvelles à apporter ?
\VS{23}Quoiqu'il arrive, je veux courir, reprit Achimaats. Et Joab lui dit : Cours ! Achimaats courut par le chemin de la plaine, et il devança Cuschi.
\VS{24}David était assis entre les deux portes. La sentinelle alla sur le toit de la porte vers la muraille ; elle leva les yeux et elle regarda. Et voici un homme qui courait tout seul.
\VS{25}Alors la sentinelle cria, et avertit le roi. Le roi dit : S'il est seul, il apporte des bonnes nouvelles. Et cet homme marchait incessamment et approchait.
\VS{26}Puis la sentinelle vit un autre homme qui courait ; et elle cria au portier : Voici un homme qui court tout seul. Le roi dit : Il apporte aussi des bonnes nouvelles.
\VS{27}La sentinelle dit : La manière de courir du premier me paraît celle d'Achimaats, fils de Tsadok. Et le roi dit : C'est un homme de bien, il vient quand il y a des bonnes nouvelles.
\VS{28}Achimaats cria, et il dit au roi : Tout va bien ! Et il se prosterna devant le roi, le visage contre terre, et dit : Béni soit Yahweh, ton Dieu, qui a livré les hommes qui levaient leurs mains contre le roi, mon seigneur !
\VS{29}Le roi dit : Le jeune homme Absalom se porte-t-il bien ? Achimaats lui répondit : J'ai vu s'élever un grand tumulte au moment où Joab envoya le serviteur du roi et moi ton serviteur ; mais je ne sais pas exactement ce que c'était.
\VS{30}Et le roi lui dit : Mets-toi là de côté. Et Achimaats se tint de côté.
\VS{31}Aussitôt arriva Cuschi. Et il dit : Que le roi, mon seigneur apprenne, ces bonnes nouvelles ! Aujourd'hui, Yahweh t'a rendu justice en jugeant tous ceux qui s'élevaient contre toi.
\VS{32}Le roi dit à Cuschi : Le jeune homme Absalom se porte-t-il bien ? Et Cuschi lui répondit : Que les ennemis du roi, mon seigneur, et tous ceux qui s'élèvent contre toi pour te faire du mal soient comme ce jeune homme !
\VS{33}Alors le roi, saisi d'émotion, monta à la chambre haute de la porte, et alla pleurer. Il disait ainsi en marchant : Mon fils Absalom ! Mon fils, mon fils Absalom ! Plaise à Dieu que je sois moi-même mort à ta place ! Absalom, mon fils, mon fils !
\Chap{19}
\TextTitle{Souffrance de David ; indignation de Joab}
\VerseOne{}Et on fit ce rapport à Joab : Voici, le roi pleure et se lamente à cause d'Absalom.
\VS{2}Ainsi, la victoire fut en ce jour-là changée en deuil pour tout le peuple, car en ce jour-là le peuple entendait dire : Le roi est affligé à cause de son fils.
\VS{3}Ce même jour, le peuple rentra dans la ville à la dérobée, comme l'auraient fait des gens honteux d'avoir pris la fuite dans la bataille.
\VS{4}Le roi s'était couvert le visage, et il criait à haute voix : Mon fils Absalom ! Absalom, mon fils, mon fils !
\VS{5}Joab entra dans la chambre où était le roi, et lui dit : Tu couvres aujourd'hui de confusion les faces de tous tes serviteurs, qui ont en ce jour sauvé ta vie, celle de tes fils et de tes filles, celle de tes femmes et de tes concubines.
\VS{6}Tu aimes ceux qui te haïssent, et tu hais ceux qui t'aiment, car tu montres aujourd'hui que tes chefs et tes serviteurs ne te sont rien ; et je sais maintenant que si Absalom vivait, et que nous tous fussions morts aujourd'hui, cela serait agréable à tes yeux.
\VS{7}Maintenant donc lève-toi, sors, et parle selon le coeur de tes serviteurs ! Car je jure par Yahweh que si tu ne sors pas, il ne restera pas un seul homme avec toi cette nuit ; et ce mal sera pire que tous ceux qui te sont arrivés depuis ta jeunesse jusqu'à présent.
\TextTitle{Retour du roi David à Jérusalem}
\VS{8}Alors le roi se leva et s'assit à la porte. On fit dire à tout le peuple : Voici, le roi est assis à la porte. Et tout le peuple vint devant le roi. Cependant, Israël s'était enfui, chacun dans sa tente.
\VS{9}Et dans toutes les tribus d'Israël, tout le peuple était en contestation, disant : Le roi nous a délivrés de la main de nos ennemis, c'est lui qui nous a sauvés de la main des Philistins, et maintenant il a dû fuir du pays devant Absalom.
\VS{10}Or Absalom, que nous avions oint pour roi sur nous, est mort dans la bataille. Maintenant donc, pourquoi ne parlez-vous pas de faire revenir le roi ?
\VS{11}Le roi David envoya dire aux sacrificateurs Tsadok et Abiathar : Parlez aux anciens de Juda, et dites-leur : Pourquoi seriez-vous les derniers à ramener le roi en sa maison ? Car les discours que tout Israël avait tenus étaient parvenus jusqu'au roi dans sa maison.
\VS{12}Vous êtes mes frères, vous êtes mes os et ma chair ; pourquoi seriez-vous les derniers à ramener le roi ?
\VS{13}Dites même à Amasa : N'es-tu pas mon os et ma chair ? Que Dieu me traite dans toute sa rigueur si tu ne deviens pas devant moi pour toujours chef de l'armée à la place de Joab !
\VS{14}Ainsi David fléchit le cœur de tous les hommes de Juda, comme s'ils n'eussent été qu'un seul homme ; et ils envoyèrent dire au roi : Reviens, toi, et tous tes serviteurs.
\VS{15}Le roi revint et arriva jusqu'au Jourdain ; et Juda se rendit jusqu'à Guilgal, pour aller à la rencontre du roi afin de lui faire repasser le Jourdain.
\VS{16}Et Schimeï, fils de Guéra, Benjamite, qui était de Bachurim, se hâta de descendre avec les hommes de Juda à la rencontre du roi David.
\VS{17}Il avait avec lui mille hommes de Benjamin, et Tsiba, serviteur de la maison de Saül, ses quinze enfants, et ses vingt serviteurs étaient aussi avec lui. Ils passèrent le Jourdain en présence du roi.
\VS{18}Le bateau, mis à la disposition du roi, faisait la traversée pour transporter sa maison ; et au moment où le roi allait passer le Jourdain, Schimeï, fils de Guéra, se prosterna devant lui.
\VS{19}Et il dit au roi : Que mon seigneur ne m'impute pas mon iniquité, et ne se souvienne pas de ce que ton serviteur a fait de mal le jour où le roi mon seigneur sortait de Jérusalem, et que le roi ne le prenne point à cœur !
\VS{20}Car ton serviteur sait qu'il a péché. Et voici, je viens aujourd'hui le premier de toute la maison de Joseph à la rencontre du roi, mon seigneur.
\VS{21}Mais Abischaï, fils de Tseruja, répondit et dit : A cause de cela, ne fera-t-on pas mourir Schimeï, puisqu'il a maudit l'oint de Yahweh ?
\VS{22}Et David dit : Qu'ai-je à faire avec vous, fils de Tseruja ? Et pourquoi vous montrez-vous aujourd'hui mes adversaires ? Ferait-on mourir aujourd'hui quelqu'un en Israël ? Ne sais-je donc pas que je règne aujourd'hui sur Israël ?
\VS{23}Et le roi dit à Schimeï : Tu ne mourras point ! Et le roi le lui jura.
\VS{24}Après cela, Mephiboscheth, fils de Saül, descendit aussi à la rencontre du roi. Il n'avait point lavé ses pieds, ni fait sa barbe, ni lavé ses vêtements, depuis que le roi s'en était allé, jusqu'au jour où il revenait en paix.
\VS{25}Il se trouva donc au-devant du roi comme il entrait dans Jérusalem, et le roi lui dit : Pourquoi n'es-tu pas venu avec moi, Mephiboscheth ?
\VS{26}Et il lui répondit : Ô roi, mon seigneur, mon serviteur m'a trompé, car ton serviteur qui est boiteux avait dit : Je ferai seller mon âne, je monterai dessus, et j'irai avec le roi.
\VS{27}Et il a calomnié ton serviteur auprès du roi, mon seigneur. Mais le roi mon seigneur est comme un ange de Dieu. Fais donc ce qui semblera bon à tes yeux.
\VS{28}Car bien que tous ceux de la maison de mon père n'ont été que des gens dignes de mort devant le roi mon seigneur ; cependant tu as mis ton serviteur parmi ceux qui mangent à ta table. Quel droit puis-je encore avoir, pour me plaindre encore au roi ?
\VS{29}Et le roi lui dit : Pourquoi toutes ces paroles ? Je l'ai dit : Toi et Tsiba, vous partagerez les terres.
\VS{30}Et Mephiboscheth dit au roi : Qu'il prenne même tout, puisque le roi mon seigneur rentre en paix dans sa maison.
\VS{31}Barzillaï, le Galaadite, descendit de Roguelim, et passa le Jourdain avec le roi, pour l'accompagner jusqu'au-delà du Jourdain.
\VS{32}Barzillaï était très vieux, âgé de quatre-vingts ans. Il avait nourri le roi pendant qu'il avait séjourné à Mahanaïm, car c'était un homme fort riche.
\VS{33}Le roi dit à Barzillaï : Viens avec moi, je te nourrirai chez moi à Jérusalem.
\VS{34}Mais Barzillaï répondit au roi : Combien d'années vivrai-je encore pour que je monte avec le roi à Jérusalem ?
\VS{35}Je suis aujourd'hui âgé de quatre-vingts ans. Puis-je encore discerner ce qui est bon de ce qui est mauvais ? Ton serviteur peut-il savourer ce qu'il mange et ce qu'il boit ? Puis-je encore entendre la voix des chanteurs et des chanteuses ? Et pourquoi ton serviteur serait-il encore à charge à mon seigneur, le roi ?
\VS{36}Ton serviteur ira un peu au-delà du Jourdain avec le roi. Pourquoi le roi voudrait-il me donner une telle récompense ?
\VS{37}Je te prie que ton serviteur s'en retourne, et que je meure dans ma ville, près du sépulcre de mon père et de ma mère ! Mais voici ton serviteur Kimham, passera avec le roi mon seigneur ; fais-lui ce qui semblera bon à tes yeux.
\VS{38}Le roi dit : Que Kimham passe avec moi, et je lui ferai ce qui sera bon à tes yeux ; et tout ce que tu voudras de moi, je te l'accorderai.
\VS{39}Tout le peuple passa donc le Jourdain avec le roi. Puis le roi embrassa Barzillaï et le bénit. Et Barzillaï retourna dans sa demeure.
\TextTitle{Juda et Israël se disputent le roi}
\VS{40}De là, le roi passa à Guilgal, et Kimham passa avec lui. Ainsi, tout le peuple de Juda, et même la moitié du peuple d'Israël ramenèrent le roi.
\VS{41}Mais voici, tous les hommes d'Israël vinrent vers le roi, et lui dirent : Pourquoi nos frères, les hommes de Juda, t'ont-ils enlevé, et ont-ils fait passer le Jourdain au roi et à sa maison, et à tous les gens de David ?
\VS{42}Alors tous les hommes de Juda répondirent aux hommes d'Israël : Parce que le roi nous est plus proche ; pourquoi vous fâchez-vous de cela ? Avons-nous vécu aux dépens du roi ? Nous a-t-il fait des présents ?
\VS{43}Les hommes d'Israël répondirent aux hommes de Juda, et dirent : Le roi nous appartient dix fois autant, et David même plus qu'à vous. Pourquoi nous avez-vous méprisés ? N'avons-nous pas parlé les premiers de ramener notre roi ? Mais les hommes de Juda parlèrent avec plus de violence que les hommes d'Israël.
\Chap{20}
\TextTitle{Juda reste fidèle au roi David}
\VerseOne{}Et il se trouvait là un méchant\FTNT{Littéralement « beliya`al » : « méchant, pervers », « ruine, destruction ». Voir commentaire en 1 S. 2:12.} homme, nommé Schéba, fils de Bicri, Benjamite. Il sonna du shofar et dit : Nous n'avons point de part avec David ni d'héritage avec le fils d'Isaï ! Israël, chacun à ses tentes !
\VS{2}Ainsi tous les hommes d'Israël se séparèrent de David, et suivirent Schéba, fils de Bicri. Mais les hommes de Juda s'attachèrent à leur roi, et l'accompagnèrent depuis le Jourdain jusqu'à Jérusalem.
\VS{3}David rentra dans sa maison à Jérusalem. Il prit les dix femmes concubines qu'il avait laissées pour garder sa maison, et les mit en un lieu où elles étaient gardées ; il pourvut à leur entretien, mais il n'alla point vers elles. Ainsi, elles furent enfermées jusqu'au jour de leur mort, vivant dans le veuvage.
\TextTitle{Bataille contre Schéba ; Joab tue Amasa}
\VS{4}Puis le roi dit à Amasa : Rassemble-moi dans trois jours les hommes de Juda ; et toi, sois ici présent.
\VS{5}Amasa donc s'en alla pour rassembler Juda ; mais il tarda au-delà du temps que le roi lui avait fixé.
\VS{6}Alors David dit à Abischaï : Maintenant Schéba, fils de Bicri, nous fera plus de mal qu'Absalom. Prends toi-même les serviteurs de ton maître et poursuis-le, de peur qu'il ne trouve des villes fortes, et que nous ne le perdions de vue.
\VS{7}Et Abischaï partit, suivi des gens de Joab, des Kéréthiens et des Péléthiens, et de tous les hommes forts ; ils sortirent de Jérusalem, pour poursuivre Schéba, fils de Bicri.
\VS{8}Et comme ils furent près de la grande pierre qui est à Gabaon, Amasa vint au-devant d'eux. Joab était ceint d'une épée par-dessus les habits dont il était revêtu ; elle était attachée à ses reins dans le fourreau, et comme il s'avançait, elle tomba.
\VS{9}Joab dit à Amasa : Te portes-tu bien, mon frère ? Puis Joab prit de sa main droite la barbe d'Amasa pour l'embrasser.
\VS{10}Amasa ne prit point garde à l'épée qui était dans la main de Joab ; et Joab l'en frappa au ventre et répandit ses entrailles à terre, sans le frapper une seconde fois. Et il mourut. Après cela, Joab et Abischaï, son frère, poursuivirent Schéba, fils de Bicri.
\VS{11}Un des serviteurs de Joab resta près d'Amasa, et il disait : Qui aime Joab et qui est pour David ? Qu'il suive Joab !
\VS{12}Amasa était vautré dans son sang au milieu de la route ; et cet homme-là, ayant vu que tout le peuple s'arrêtait, poussa Amasa hors de la route dans un champ, et jeta un vêtement sur lui, lorsqu'il vit que tous ceux qui arrivaient près de lui s'arrêtaient.
\VS{13}Quand il fut ôté de la route, tous les hommes qui suivaient Joab passaient au-delà, afin de poursuivre Schéba, fils de Bicri.
\TextTitle{La révolte de Schéba}
\VS{14}Joab passa par toutes les tribus d'Israël jusqu'à Abel-Beth-Maaca, avec tous les Bériens, qui s'étaient assemblés et qui l'avaient suivi.
\VS{15}Les gens donc de Joab vinrent assiéger Schéba dans Abel-Beth-Maaca, et ils élevèrent contre la ville une terrasse qui atteignait le rempart. Tout le peuple qui était avec Joab rompait la muraille pour la faire tomber.
\VS{16}Lorsqu'une femme sage de la ville se mit à crier : Ecoutez, écoutez ! Dites, je vous prie, à Joab : Approche jusqu'ici, je veux te parler !
\VS{17}Il s'approcha d'elle, et la femme dit : Es-tu Joab ? Il répondit : Je le suis. Elle lui dit : Ecoute les paroles de ta servante. Il répondit : J'écoute.
\VS{18}Et elle dit : Autrefois on avait coutume de dire : Que l'on consulte Abel ! Et tout se terminait ainsi.
\VS{19}Je suis une des cités paisibles et fidèles en Israël ; tu cherches à détruire une ville qui est une mère en Israël ! Pourquoi détruirais-tu l'héritage de Yahweh ?
\VS{20}Joab lui répondit : A Dieu ne plaise, à Dieu ne plaise que je détruise et que je ruine !
\VS{21}La chose n'est pas ainsi. Mais un homme de la montagne d'Ephraïm, nommé Schéba, fils de Bicri, a levé sa main contre le roi David ; livrez-le, lui seul, et je m'éloignerai de la ville. La femme dit à Joab : Voici, sa tête te sera jetée par-dessus la muraille.
\VS{22}Et la femme alla vers tout le peuple, et leur parla sagement ; et ils coupèrent la tête de Schéba, fils de Bicri, et la jetèrent à Joab. Alors il sonna du shofar ; et on se dispersa loin de la ville, et chacun s'en alla dans sa tente. Puis Joab retourna vers le roi à Jérusalem.
\VS{23}Joab était le chef de toute l'armée d'Israël ; Benaja, fils de Jehojada, était à la tête des Kéréthiens et des Péléthiens ;
\VS{24}et Adoram était préposé aux impôts ; Josaphat, fils d'Achilud, était archiviste.
\VS{25}Scheja était le secrétaire ; Tsadok et Abiathar étaient les sacrificateurs ;
\VS{26}et Ira de Jaïr était ministre d'Etat de David.
\Chap{21}
\TextTitle{Vengeance des Gabaonites sur la maison de Saül}
\VerseOne{}Or il y eut du temps de David, une famine qui dura trois ans de suite. David chercha la face de Yahweh, et Yahweh lui répondit : C'est à cause de Saül et de sa maison sanguinaire, parce qu'il a fait mourir les Gabaonites.
\VS{2}Alors le roi appela les Gabaonites pour leur parler. Or les Gabaonites n'étaient point des enfants d'Israël, mais un reste des Amoréens ; les enfants d'Israël leur avaient juré de les laisser vivre\FTNT{Jos. 9.}, mais Saül dans son zèle pour les enfants d'Israël et de Juda, avait cherché à les faire mourir.
\VS{3}Et David dit aux Gabaonites : Que ferais-je pour vous, et par quel moyen vous apaiserai-je, afin que vous bénissiez l'héritage de Yahweh ?
\VS{4}Les Gabaonites lui répondirent : Il ne s'agit pas pour nous d'argent ou d'or avec Saül et avec sa maison, et ce n'est pas à nous de faire mourir un homme en Israël. Le roi leur dit : Que voulez-vous donc que je fasse pour vous ?
\VS{5}Ils répondirent au roi : Puisque cet homme nous a consumés, et qu'il avait résolu de nous exterminer pour nous faire disparaître de tout le territoire d'Israël,
\VS{6}qu'on nous livre sept hommes d'entre ses fils, et nous les pendrons devant Yahweh à Guibea de Saül, l'élu de Yahweh. Et le roi dit : Je vous les livrerai.
\VS{7}Le roi épargna Mephiboscheth, fils de Jonathan, fils de Saül, à cause du serment que David et Jonathan, fils de Saül, avaient fait entre eux, devant Yahweh.
\VS{8}Mais le roi prit les deux fils que Ritspa, fille d'Ajja, avait enfantés à Saül, Armoni et Mephiboscheth, et les cinq fils que Mérab, fille de Saül, avait enfantés à Adriel de Mehola, fils de Barzillaï,
\VS{9}et il les livra entre les mains des Gabaonites, qui les pendirent sur la montagne, devant Yahweh. Tous les sept furent tués ensemble ; on les fit mourir dans les premiers jours de la moisson, au commencement de la moisson des orges.
\VS{10}Alors Ritspa, fille d'Ajja, prit un sac et l'étendit sous elle au-dessus d'un rocher, depuis le commencement de la moisson jusqu'à ce que l'eau du ciel tombât sur eux ; et elle ne permit pas aux oiseaux du ciel de s'approcher d'eux pendant le jour, ni aux bêtes des champs pendant la nuit.
\VS{11}On informa David de ce qu'avait fait Ritspa, fille d'Ajja, concubine de Saül.
\VS{12}Et David alla prendre les os de Saül et les os de Jonathan, son fils, chez les habitants de Jabès en Galaad, qui les avaient enlevés de la place de Beth-Schan, où les Philistins les avaient pendus lorsqu'ils tuèrent Saül à Guilboa.
\VS{13}Il emporta de là les os de Saül et les os de Jonathan, son fils ; on recueillit aussi les os de ceux qui avaient été pendus.
\VS{14}On les enterra avec les os de Saül et de Jonathan, son fils, au pays de Benjamin, à Tséla, dans le sépulcre de Kis, père de Saül. Et l'on fit tout ce que le roi avait ordonné. Après cela, Dieu fut apaisé envers le pays.
\TextTitle{Nouvelles batailles contre les Philistins}
\VS{15}Il y eut encore une guerre entre les Philistins et Israël. David y était allé, et ses serviteurs avec lui, et ils combattirent tellement contre les Philistins que David défaillait.
\VS{16}Et Jischbi-Benob, qui était un des enfants de Rapha, eut l'intention de tuer David ; il avait une lance dont le fer pesait trois cents sicles d'airain, et il était ceint d'une armure neuve.
\VS{17}Mais Abischaï, fils de Tseruja, vint au secours de David, frappa le Philistin, et le tua. Alors les gens de David jurèrent, en disant : Tu ne sortiras plus avec nous à la bataille, de peur que tu n'éteignes la lampe d'Israël.
\VS{18}Après cela, il y eut encore une autre guerre à Gob avec les Philistins. Sibbecaï, le Huschatite, tua Saph, qui était un des enfants de Rapha.
\VS{19}Il y eut encore une autre guerre à Gob avec les Philistins. Et Elchanan, fils de Jaaré-Oreguim, de Bethléhem, tua Goliath de Gath, qui avait une lance dont le bois était comme une ensouple de tisserand.
\VS{20}Il y eut encore une guerre à Gath. Il s'y trouva un homme de haute taille, qui avait six doigts à chaque main, et six orteils à chaque pied, en tout vingt-quatre, lequel était aussi issu de Rapha.
\VS{21}Il jeta un défi à Israël ; et Jonathan, fils de Schimea, frère de David, le tua.
\VS{22}Ces quatre-là étaient nés à Gath, de la race de Rapha. Ils moururent par les mains de David, ou par les mains de ses serviteurs.
\Chap{22}
\TextTitle{Louange à Yahweh, le Dieu qui délivre}
\VerseOne{}Après cela, David adressa à Yahweh les paroles de ce cantique, le jour où Yahweh l'eut délivré de la main de tous ses ennemis, et de la main de Saül.
\VS{2}Il dit : Yahweh est mon rocher, ma forteresse, mon libérateur.
\VS{3}Dieu est mon rocher, où je trouve un abri, mon bouclier et la force qui me sauve, ma haute retraite et mon refuge. Ô mon Sauveur ! Tu me délivres de la violence.
\VS{4}Je m'écrie : Loué soit Yahweh! Et je suis délivré de mes ennemis\FTNT{Ps. 18:4.}.
\VS{5}Car les flots de la mort m'avaient environné, les torrents des méchants m'avaient épouvanté ;
\VS{6}les liens du scheol m'avaient entouré, les filets de la mort m'avaient surpris.
\VS{7}Dans ma détresse, j'ai invoqué Yahweh, j'ai crié à mon Dieu ; de son palais, il a entendu ma voix, et mon cri est parvenu à ses oreilles.
\VS{8}Alors la terre fut ébranlée et trembla, les fondements des cieux s'agitèrent, et ils furent ébranlés, parce qu'il était irrité.
\VS{9}Une fumée montait de ses narines, et de sa bouche sortait un feu dévorant : Il en jaillissait des charbons embrasés.
\VS{10}Il abaissa les cieux, et descendit : Il y avait une épaisse nuée sous ses pieds.
\VS{11}Il était monté sur un chérubin, et il volait, il paraissait sur les ailes du vent.
\VS{12}Il mit autour de lui les ténèbres pour tabernacle, des amas d'eaux, des nuées épaisses.
\VS{13}Des charbons de feu étaient embrasés de la splendeur qui le précédait.
\VS{14}Yahweh tonna des cieux, et le Très-Haut fit retentir sa voix ;
\VS{15}il lança des flèches, et dispersa mes ennemis ;  il lança des éclairs, et les mit en déroute.
\VS{16}Alors le fond de la mer apparut, et les fondements de la terre habitable furent mis à découvert, par la menace de Yahweh, par le souffle du vent de sa colère.
\VS{17}Il étendit sa main d'en haut, il me saisit, il me retira des grandes eaux ;
\VS{18}il me délivra de mon ennemi puissant, de ceux qui me haïssaient, car ils étaient plus forts que moi.
\VS{19}Ils m'avaient surpris au jour de ma détresse, mais Yahweh fut mon appui.
\VS{20}Il m'a mis au large, il m'a sauvé, parce qu'il a pris son plaisir en moi.
\VS{21}Yahweh m'a traité selon ma droiture, il m'a rendu selon la pureté de mes mains ;
\VS{22}parce que j'ai gardé les voies de Yahweh, et que je ne me suis point détourné de mon Dieu.
\VS{23}Toutes ses ordonnances ont été devant moi, et je ne me suis point écarté de ses lois.
\VS{24}J'ai été intègre envers lui, et je me suis gardé de mon iniquité.
\VS{25}Yahweh donc m'a rendu selon ma droiture, selon ma pureté devant ses yeux.
\VS{26}Avec celui qui est bon tu es bon, avec l'homme intègre tu es intègre,
\VS{27}avec celui qui est pur tu te montres pur, mais avec le pervers tu agis selon sa perversité.
\VS{28}Tu sauves le peuple qui s'humilie, et de ton regard, tu abaisses les orgueilleux.
\VS{29}Tu es ma lampe, ô Yahweh ! Et Yahweh éclaire mes ténèbres.
\VS{30}Avec toi je me précipite sur une troupe en armes, avec mon Dieu je franchis une muraille.
\VS{31}La voie de Dieu est parfaite, la parole de Yahweh est éprouvée ; il est le bouclier de tous ceux qui se confient en lui.
\VS{32}Car qui est Dieu, si ce n'est Yahweh ? Et qui est un rocher, si ce n'est notre Dieu ?
\VS{33}C'est Dieu qui est ma puissante forteresse, et qui me conduit dans la voie droite.
\VS{34}Il a rendu mes pieds semblables à ceux des biches, et il me fait tenir debout sur mes lieux élevés.
\VS{35}Il exerce mes mains au combat, et mes bras tendent l'arc d'airain.
\VS{36}Tu me donnes le bouclier de ton salut, et ta bonté me fait devenir plus grand.
\VS{37}Tu élargis le chemin sous mes pas, et mes pieds ne chancellent point.
\VS{38}Je poursuis mes ennemis, et je les détruis ; je ne reviens qu'après les avoir exterminés.
\VS{39}Je les anéantis, je les transperce, et ils ne se relèvent plus ; ils tombent sous mes pieds.
\VS{40}Tu me ceins de force pour le combat, tu fais plier sous moi mes adversaires.
\VS{41}Tu fais tourner le dos à mes ennemis devant moi, et j'extermine ceux qui me haïssent.
\VS{42}Ils regardent autour d'eux, et il n'y a point de sauveur ! Ils crient à Yahweh, mais il ne leur répond pas !
\VS{43}Je les broie comme la poussière de la terre, je les écrase, je les foule, comme la boue des rues.
\VS{44}Tu me délivres des dissensions de mon peuple ; tu me gardes pour être chef des nations ; un peuple que je ne connaissais pas m'est asservi.
\VS{45}Les fils de l'étranger me flattent, dès qu'ils ont entendu parler de moi, ils se sont rendus obéissants.
\VS{46}Les fils de l'étranger défaillent, et sortent tremblants de leurs forteresses.
\VS{47}Yahweh est vivant, et béni soit mon rocher ! Que Dieu, le rocher de mon salut, soit exalté,
\VS{48}le Dieu qui me donne vengeance, qui m'assujettit les peuples,
\VS{49}et qui me fait échapper à mes ennemis ! Tu m'élèves au-dessus de mes adversaires, tu me délivres de l'homme violent.
\VS{50}C'est pourquoi, ô Yahweh, je te louerai parmi les nations, et je chanterai des psaumes à ton Nom.
\VS{51}C'est lui qui est la tour de délivrance de son roi, et qui fait miséricorde à son oint, à David, et à sa postérité, à jamais.
\Chap{23}
\TextTitle{Paroles prophétiques de David}
\VerseOne{}Voici les dernières paroles de David. Parole de David, fils d'Isaï, parole de l'homme qui a été élevé, de l'oint du Dieu de Jacob, du chantre agréable d'Israël :
\VS{2}L'Esprit de Yahweh parle par moi, et sa parole est sur ma langue.
\VS{3}Le Dieu d'Israël a parlé, le Rocher\FTNT{Voir commentaire Es. 8 :13-17.} d'Israël m'a dit : Celui qui règne parmi les hommes avec justice, celui qui règne dans la crainte de Dieu,
\VS{4}est comme la lumière du matin quand le soleil se lève, un matin sans nuage ; son éclat fait germer de la terre la verdure après la pluie.
\VS{5}N'en est-il pas ainsi de ma maison devant Dieu, puisqu'il a traité avec moi une alliance éternelle, bien ordonnée, et gardée ? Tout mon salut et tout mon plaisir, ne les fera-t-il pas germer ?
\VS{6}Mais les méchants sont tous comme des épines que l'on jette au loin, parce qu'on ne les prend pas avec la main ;
\VS{7}celui qui les touche, s'arme du fer ou du bois d'une lance, et on les brûle au feu sur place.
\TextTitle{Les vaillants hommes de David\FTNTT{1 Ch. 11:10-47.}}
\VS{8}Voici les noms des vaillants hommes qui étaient au service de David. Joscheb-Basschébeth, le Tachkemonite, était l'un des principaux chefs. C'était Hadino le Hetsnite, qui eut le dessus sur huit cents hommes qu'il tua en une seule fois.
\VS{9}Après lui, Eléazar, fils de Dodo, fils d'Achochi. Il était l'un des trois vaillants hommes qui étaient avec David lorsqu'ils défièrent les Philistins rassemblés pour combattre, tandis que les hommes d'Israël se retiraient.
\VS{10}Il se leva, et frappa les Philistins jusqu'à ce que sa main fut lasse et qu'elle restât attachée à l'épée. Ce jour-là, Yahweh opéra une grande délivrance. Le peuple revint après Eléazar, seulement pour prendre les dépouilles.
\VS{11}Après lui, Schamma, fils d'Agué d'Harar. Les Philistins s'étaient rassemblés en troupe. Il y avait là une parcelle de champ pleine de lentilles ; et le peuple fuyait devant les Philistins.
\VS{12}Schamma se mit au milieu de cette parcelle, la défendit, et frappa les Philistins. Et Yahweh opéra une grande délivrance.
\VS{13}Trois des trente chefs descendirent au temps de la moisson et vinrent vers David, dans la caverne d'Adullam, lorsqu'une troupe de Philistins était campée dans la vallée des Rephaïm.
\VS{14}David était alors dans la forteresse, et la garnison des Philistins était en ce temps-là à Bethléhem.
\VS{15}Et David eut un désir, et dit : Qui est-ce qui me fera boire de l'eau de la citerne qui est à la porte de Bethléhem ?
\VS{16}Alors ces trois vaillants hommes passèrent au travers du camp des Philistins, et puisèrent de l'eau de la citerne qui est à la porte de Bethléhem. Ils l'apportèrent, et ils la présentèrent à David ; mais il ne voulut pas la boire, et il la répandit devant Yahweh.
\VS{17}Car il dit : Loin de moi, ô Yahweh, de faire une telle chose ! N'est-ce pas le sang de ces hommes qui sont allés au péril de leur vie ? Il ne voulut pas la boire. Voilà ce que firent ces trois vaillants hommes.
\VS{18}Il y avait aussi Abischaï, frère de Joab, fils de Tseruja, qui était le chef des trois. Il brandit sa lance sur trois cents hommes, les blessa à mort ; et il eut du renom parmi les trois.
\VS{19}Il était le plus considéré des trois, et il fut leur chef ; cependant il n'égala point les trois premiers.
\VS{20}Benaja, fils de Jehojada, fils d'un vaillant homme de Kabtseel, rempli de force, avait fait de grands exploits. Il frappa deux des plus puissants hommes de Moab. Il descendit au milieu d'une fosse, où il frappa un lion, un jour de neige.
\VS{21}Il frappa aussi un Egyptien d'un aspect formidable et ayant une lance à la main ; Benaja descendit contre lui avec un bâton, arracha la lance de la main de l'Egyptien, et s'en servit pour le tuer.
\VS{22}Benaja, fils de Jehojada, fit ces choses-là ; et fut illustre parmi les trois vaillants hommes.
\VS{23}Il était le plus considéré des trente ; mais il n'égala pas les trois premiers. C'est pourquoi David l'établit dans son conseil secret.
\VS{24}Asaël, frère de Joab, était des trente. Elchanan, fils de Dodo, de Bethléhem.
\VS{25}Schamma, de Harod. Elika, de Harod.
\VS{26}Hélets, de Péleth. Ira, fils d'Ikkesch, de Tekoa.
\VS{27}Abiézer, d'Anathoth. Mebunnaï, de Huscha.
\VS{28}Tsalmon, d'Achoach. Maharaï, de Nethopha.
\VS{29}Héleb, fils de Baana, de Nethopha. Ittaï, fils de Ribaï, de Guibea des fils de Benjamin.
\VS{30}Benaja, de Pirathon. Hiddaï, de Nachalé-Gaasch.
\VS{31}Abi-Albon, d'Araba. Azmaveth, de Barchum.
\VS{32}Eliachba, de Schaalbon. Bené-Jaschen. Jonathan.
\VS{33}Schamma, d'Harar. Achiam, fils de Scharar, d'Arar.
\VS{34}Eliphéleth, fils d'Achasbaï, fils d'un Maacathien. Eliam, fils d'Achitophel, de Guilo.
\VS{35}Hetsro, de Carmel. Paaraï, d'Arab.
\VS{36}Jigueal, fils de Nathan, de Tsoba. Bani, de Gad.
\VS{37}Tsélek, l'Ammonite. Naharaï, de Beéroth, qui portait les armes de guerre de Joab, fils de Tseruja.
\VS{38}Ira, de Jéther. Gareb, de Jéther.
\VS{39}Urie, le Héthien. En tout, trente-sept.
\Chap{24}
\TextTitle{Péché de David ; plaie mortelle sur Israël\FTNTT{1 Ch. 21:1-17.}}
\VerseOne{}La colère de Yahweh s'enflamma encore contre Israël, parce que David fut incité contre eux, en disant : Va, fais le dénombrement d'Israël et de Juda\FTNT{1 Ch. 21.}.
\VS{2}Le roi dit donc à Joab, chef de l'armée qui se trouvait près de lui : Parcours toutes les tribus d'Israël, depuis Dan jusqu'à Beer-Schéba ; et dénombre le peuple, afin que je sache le nombre du peuple.
\VS{3}Joab dit au roi : Que Yahweh, ton Dieu, veuille augmenter ton peuple cent fois plus, et que les yeux du roi mon seigneur le voient ! Mais pourquoi le roi mon seigneur prend-il plaisir à cela ?
\VS{4}Néanmoins, la parole du roi l'emporta sur Joab, et sur les chefs de l'armée ; et Joab et les chefs de l'armée sortirent de la présence du roi pour dénombrer le peuple d'Israël.
\VS{5}Ils passèrent le Jourdain, et ils campèrent à Aroër, à droite de la ville qui est au milieu de la vallée du torrent de Gad, et vers Jaezer.
\VS{6}Ils allèrent en Galaad et dans le territoire de ceux qui habitent vers le bas du pays de Thachthim-Hodschi. Ils allèrent à Dan-Jaan, et aux environs de Sidon.
\VS{7}Ils vinrent jusqu'à la forteresse de Tyr, et dans toutes les villes des Héviens et des Cananéens. Ils sortirent vers le midi de Juda à Beer-Schéba.
\VS{8}Ainsi ils parcoururent tout le pays, et arrivèrent à Jérusalem au bout de neuf mois et vingt jours.
\VS{9}Et Joab donna au roi le rôle du dénombrement du peuple : Il y avait en Israël huit cent mille hommes de guerre tirant l'épée, et en Juda cinq cent mille hommes.
\VS{10}Alors David sentit battre son cœur, après qu'il eut fait ainsi dénombrer le peuple. Et David dit à Yahweh : J'ai commis un grand péché en faisant cela ! Mais, je te prie, ô Yahweh, de pardonner l'iniquité de ton serviteur, car j'ai agi en insensé !
\VS{11}Après cela, David se leva dès le matin, et la parole de Yahweh fut adressée à Gad le prophète, qui était le voyant de David :
\VS{12}Va dire à David : Ainsi parle Yahweh : J'apporte trois choses contre toi ; choisis l'une d'elles afin que je te la fasse.
\VS{13}Gad alla vers David, et lui rapporta cela en disant : Que veux-tu qu'il t'arrive : Sept ans de famine sur ton pays, ou que durant trois mois tu fuies devant tes ennemis qui te poursuivront, ou que durant trois jours la peste soit dans ton pays ? Choisis maintenant, et regarde ce que tu veux que je réponde à celui qui m'a envoyé.
\VS{14}David répondit à Gad : Je suis dans une très grande détresse ! Tombons entre les mains de Yahweh, car ses compassions sont en grand nombre ; mais que je ne tombe pas entre les mains des hommes !
\VS{15}Yahweh envoya donc la peste en Israël, depuis le matin jusqu'au temps fixé ; et depuis Dan jusqu'à Beer-Schéba, il mourut soixante-dix mille hommes parmi le peuple.
\VS{16}Mais quand l'ange étendait sa main sur Jérusalem pour la ravager, Yahweh se repentit de ce mal et dit à l'ange qui ravageait le peuple : C'est assez ! Retire maintenant ta main. Or l'Ange de Yahweh était près de l'aire d'Aravna, le Jébusien.
\VS{17}Car David voyant l'ange qui frappait le peuple, parla à Yahweh, et dit : Voici, c'est moi qui ai péché ! C'est moi qui ai commis l'iniquité ; mais ces brebis, qu'ont-elles fait ? Je te prie que ta main soit contre moi et contre la maison de mon père !
\TextTitle{Sacrifice de David ; Yahweh met fin à la plaie\FTNTT{1 Ch. 21:18-30.}}
\VS{18}Ce jour-là, Gad vint vers David, et lui dit : Monte, et dresse un autel à Yahweh dans l'aire d'Aravna, le Jébusien.
\VS{19}Et David monta, selon la parole de Gad, comme Yahweh l'avait ordonné.
\VS{20}Aravna regarda, et vit le roi et ses serviteurs qui venaient vers lui ; et Aravna sortit, et se prosterna devant le roi, le visage contre terre.
\VS{21}Aravna dit : Pourquoi le roi mon seigneur vient-il vers son serviteur ? Et David répondit : Pour acheter ton aire, et y bâtir un autel à Yahweh, afin que cette plaie se retire de dessus le peuple.
\VS{22}Aravna dit à David : Que le roi mon seigneur prenne et offre ce qu'il lui plaira ; vois les bœufs seront pour l'holocauste, et les chars avec l'attelage de bœufs serviront de bois.
\VS{23}Aravna donna tout cela au roi. Et Aravna dit au roi : Que Yahweh, ton Dieu, te soit favorable !
\VS{24}Mais le roi répondit à Aravna : Non ! Je veux l'acheter de toi pour un certain prix, et je n'offrirai point à Yahweh, mon Dieu, des holocaustes qui ne me coûtent rien. Ainsi, David acheta l'aire et les bœufs pour cinquante sicles d'argent.
\VS{25}David bâtit là un autel à Yahweh, et offrit des holocaustes et des sacrifices d'offrande de paix. Alors Yahweh fut apaisé envers le pays, et la plaie se retira d'Israël.
\PPE{}
\end{multicols}

%\clearpage\ShortTitle{1 Rois}\BookTitle{1 Rois}\BFont
\noindent\hrulefill
{\footnotesize
\textit{
\bigskip
{\centering{}
\\Auteur : Inconnu
\\(Heb. : Melakhim)
\\Signification : Roi, Règne
\\Thème : Unité du royaume après le schisme
\\Date de rédaction : 6ème siècle av. J.-C.\\}
}
%\bigskip
\textit{
\\Ce livre relate la vie de Salomon : son accession à la royauté après la mort de son père David, son alliance avec Dieu qui lui accorda une sagesse exceptionnelle ainsi que la construction du temple de Yahweh et du palais royal.
%\bigskip
\\Les premières années du règne de Salomon furent exemplaires. Malheureusement, il ne fit pas preuve de la même piété
que son père et développa une affection particulière pour les femmes étrangères qui l’entrainèrent dans l’idolâtrie. A sa
mort, son fils Roboam accéda au pouvoir et provoqua la division du royaume en deux : d’un côté les dix tribus du nord qui gardèrent le nom d’Israël, gouvernées par Jéroboam, et de l’autre côté les deux tribus du sud, Juda et Benjamin, qui demeurèrent sous l’autorité de Roboam.
%\bigskip
\\Ce livre raconte également le règne et la conduite parfois abominable des rois d’Israël et de Juda jusqu’à Achab et Josaphat.
Il présente la puissance de l’appel prophétique d’Elie, le Tischbite, que Dieu suscita pour ramener son peuple à lui et
montrer sa souveraineté.\bigskip
}
}
\par\nobreak\noindent\hrulefill
\begin{multicols}{2}
\Chap{1}
\TextTitle{Fin de la vie de David}
\VerseOne{}Le roi David était vieux et avancé en âge ; on le couvrait de vêtements parce qu’il ne parvenait point à se réchauffer.
\VS{2}Ses serviteurs lui dirent : Que l'on cherche pour le roi notre seigneur, une jeune fille vierge ; qu’elle se tienne devant le roi, qu’elle le soigne et qu'elle dorme en son sein, afin que le roi notre seigneur, se réchauffe.
\VS{3}On chercha donc dans toutes les contrées d'Israël une jeune et belle femme, et on trouva Abischag la Sunamite, que l’on amena auprès du roi.
\VS{4}Cette jeune femme était fort belle. Elle prit soin du roi et le servit, mais le roi ne la connut point.
\TextTitle{Conspiration d’Adonija pour régner sur Israël}
\VS{5}Alors Adonija, fils de Haggith, se laissa emporter par l’orgueil en disant : Je suis le roi ! Il se procura un char, des cavaliers et cinquante hommes qui couraient devant lui.
\VS{6}Son père ne lui avait jamais fait un reproche jusqu’à ce jour-là, en disant : Pourquoi agis-tu ainsi ? Adonija était très beau de figure, il était né après Absalom.
\VS{7}Il s’entendit avec Joab, fils de Tseruja, et avec le sacrificateur Abiathar, qui embrassèrent son parti.
\VS{8}Mais le sacrificateur Tsadok, Benaja fils de Jehojada, Nathan le prophète, Schimeï, Reï et les vaillants hommes de David ne furent point du parti d'Adonija.
\VS{9}Or, Adonija fit tuer des brebis, des bœufs et des veaux gras près de la pierre de Zohéleth, qui est auprès d’En-Roguel ; il invita tous ses frères, fils du roi, et tous les hommes de Juda qui étaient au service du roi.
\VS{10}Mais il ne convia point Nathan le prophète, ni Benaja, ni les vaillants hommes, ni Salomon, son frère.
\TextTitle{Opposition de Nathan et Bath-Schéba}
\VS{11}Alors Nathan parla à Bath-Schéba, mère de Salomon, en disant : N'as-tu pas entendu qu'Adonija, fils de Haggith, a été fait roi ? Et David notre Seigneur n'en sait rien.
\VS{12}Maintenant donc viens, je t’en donne le conseil afin que tu sauves ta vie et la vie de ton fils Salomon.
\VS{13}Va, entre chez le roi David et dis-lui : Ô roi, mon seigneur, n'as-tu pas fait serment à ta servante, en disant : ton fils Salomon régnera après moi et sera assis sur mon trône ? Pourquoi donc Adonija règne-t-il ?
\VS{14}Et voici, lorsque tu seras encore là et que tu parleras avec le roi, je viendrai après toi et je confirmerai tes dires.
\VS{15}Bath-Schéba se rendit dans la chambre du roi. Or, le roi était très vieux et Abischag, la Sunamite, le servait.
\VS{16}Bath-Schéba s'inclina et se prosterna devant le roi. Et le roi lui dit : Qu'as-tu ?
\VS{17}Et elle lui répondit : Mon seigneur, tu as juré par Yahweh, ton Dieu à ta servante, en lui disant : Ton fils Salomon régnera après moi et s’assiéra sur mon trône.
\VS{18}Mais maintenant voici, Adonija est proclamé roi ! Et tu ne le sais pas, ô roi, mon seigneur !
\VS{19}Il a fait tuer des bœufs, des veaux gras et des brebis en grand nombre, il a convié tous les fils du roi, avec Abiathar, le sacrificateur, et Joab, chef de l'armée, mais il n'a point convié ton serviteur Salomon.
\VS{20}Ô roi mon seigneur ! Les yeux de tout Israël sont sur toi, afin que tu lui fasses connaître qui s’assiéra sur le trône du roi mon seigneur après lui.
\VS{21}Aussi, lorsque le roi mon seigneur sera endormi avec ses pères, nous serons traités comme des coupables, moi et mon fils Salomon.
\VS{22}Tandis qu’elle parlait encore avec le roi, Nathan le prophète se présenta.
\VS{23}On l’annonça au roi en disant : Voici Nathan le prophète ! Il se présenta devant le roi et se prosterna devant lui, le visage contre terre.
\VS{24}Et Nathan dit : Ô roi mon seigneur ! Tu as dit : Adonija régnera après moi et sera assis sur mon trône !
\VS{25}Car il est descendu aujourd'hui, il a sacrifié des bœufs, des veaux gras et des brebis en grand nombre. Il a convié tous les fils du roi, les chefs de l'armée et le sacrificateur Abiathar. Et voici, ils mangent et boivent devant lui ; ils disent : Vive le roi Adonija !
\VS{26}Mais il n'a convié ni moi, ton serviteur, ni le sacrificateur Tsadok, ni Benaja, fils de Jehojada, ni Salomon ton serviteur.
\VS{27}Est-ce bien par ordre de mon seigneur le roi que cette chose a lieu et sans que tu aies fait connaître à ton serviteur quel est celui qui doit s'asseoir sur le trône du roi mon seigneur après lui ?
\VS{28}Et le roi David répondit, en disant : Appelez-moi Bath-Schéba ; elle entra et se présenta devant le roi.
\VS{29}Alors le roi jura et dit : Yahweh, qui m'a délivré de toute détresse, est vivant !
\VS{30}Comme je te l'ai juré par Yahweh, le Dieu d'Israël, en disant : Ton fils Salomon régnera après moi et sera assis sur mon trône à ma place ; ainsi ferai-je aujourd'hui.
\VS{31}Alors Bath-Schéba s'inclina le visage contre terre et se prosterna devant le roi en disant : Que le roi David mon seigneur vive éternellement !
\VS{32}Et le roi David dit : Appelez-moi le sacrificateur Tsadok, le prophète Nathan et Benaja, fils de Jehojada ; et ils se présentèrent devant le roi.
\VS{33}Le roi leur dit : Prenez avec vous les serviteurs de votre seigneur, faites monter mon fils Salomon sur ma mule, et faites-le descendre à Guihon.
\VS{34}Que Tsadok le sacrificateur et Nathan le prophète, l'oignent en ce lieu-là pour roi sur Israël, puis vous sonnerez du shofar et vous direz : Vive le roi Salomon !
\VS{35}Vous monterez après lui et il viendra, il s'assiéra sur mon trône et il régnera à ma place ; car j'ai ordonné qu'il soit le chef d'Israël et de Juda.
\VS{36}Et Benaja fils de Jehojada répondit au roi : Amen ! Ainsi parle Yahweh, le Dieu de mon seigneur le roi !
\VS{37}Comme Yahweh a été avec mon seigneur le roi, qu'il soit aussi avec Salomon, et qu'il élève son trône encore plus que le trône du roi David mon seigneur !
\TextTitle{Salomon oint roi d’Israël par Tsadok\FTNTT{cp. 1 Ch. 29:22}}
\VS{38}Puis Tsadok le sacrificateur descendit avec Nathan le prophète et Benaja, fils de Jehojada, les Kéréthiens et les Péléthiens ; ils firent monter Salomon sur la mule du roi David et le menèrent à Guihon.
\VS{39}Tsadok le sacrificateur prit du tabernacle une corne d'huile dont il oignit Salomon. On sonna du shofar et tout le peuple dit : Vive le roi Salomon !
\VS{40}Et tout le monde monta après lui et le peuple jouait de la flûte, en se livrant à une grande joie, au point que la terre se fendait par leurs cris.
\VS{41}Ce bruit fut entendu d’Adonija et de tous les conviés qui étaient avec lui comme ils achevaient de manger ; et Joab entendant le son du shofar, dit : Pourquoi ce bruit de la ville en tumulte ?
\VS{42}Et comme il parlait encore, voici Jonathan, fils du sacrificateur Abiathar, arriva et Adonija lui dit : Entre, car tu es un vaillant homme et tu apportes de bonnes nouvelles.
\VS{43}Oui ! répondit Jonathan à Adonija : Le roi David, notre seigneur, a établi Salomon roi.
\VS{44}Et le roi a envoyé avec lui Tsadok le sacrificateur, Nathan le prophète, Benaja, fils de Jehojada, les Kéréthiens, et les Péléthiens, et ils l'ont fait monter sur la mule du roi.
\VS{45}Tsadok le sacrificateur, et Nathan le prophète l'ont oint pour roi à Guihon, d'où ils sont remontés avec joie, et la ville est ainsi émue ; c'est là le bruit que vous avez entendu.
\VS{46}Salomon s'est même assis sur le trône royal.
\VS{47}Et les serviteurs du roi sont venus pour bénir le roi David notre seigneur, en disant : Que ton Dieu rende le nom de Salomon encore plus grand que ton nom, et qu'il élève son trône encore plus que ton trône ! Et le roi s'est prosterné sur son lit.
\VS{48}Le roi a ainsi parlé : Béni soit Yahweh, le Dieu d'Israël, qui a aujourd’hui établi sur mon trône un successeur, et qui m’a permis de le voir !
\VS{49}Alors tous les conviés d’Adonija furent saisis de frayeur, ils se levèrent et s'en allèrent chacun son chemin.
\VS{50}Adonija eut peur de Salomon ; il se leva aussi et s'en alla empoigner les cornes de l'autel.
\VS{51}On vint l’apprendre à Salomon, en disant : Voici Adonija a peur du roi Salomon et il a saisi les cornes de l'autel, en disant : Que le roi Salomon me jure aujourd'hui qu'il ne fera point mourir son serviteur par l'épée.
\VS{52}Et Salomon dit : A l’avenir, s’il se comporte en homme de bien il ne tombera pas un seul de ses cheveux à terre ; mais s'il se trouve du mal en lui, il mourra.
\VS{53}Alors le roi Salomon envoya des personnes qui le firent descendre de l'autel. Il vint et se prosterna devant le roi Salomon, et Salomon lui dit : Va dans ta maison.
\Chap{2}
\TextTitle{Dernières paroles de David à Salomon}
\VerseOne{}David approchait du moment de sa mort, et il donna ses ordres à Salomon, son fils, en disant :
\VS{2}Je m'en vais par le chemin de toute la terre, fortifie-toi et comporte-toi en homme.
\VS{3}Observe les commandements de Yahweh, ton Dieu, en marchant dans ses voies, en gardant ses statuts, ses commandements, ses ordonnances et ses préceptes, selon ce qui est écrit dans la loi de Moïse, afin que tu réussisses dans tout ce que tu feras et dans tout ce que tu entreprendras ;
\VS{4}et afin que s’accomplisse cette parole de Yahweh déclarée sur moi : Si tes fils prennent garde à leur voie, pour marcher devant moi dans la vérité, de tout leur cœur et de toute leur âme, tu ne manqueras jamais de successeur sur le trône d'Israël.
\VS{5}Tu sais ce que m'a fait Joab, fils de Tseruja et ce qu'il a fait aux deux chefs des armées d'Israël, Abner, fils de Ner, et à Amasa, fils de Jéther, qu'il a tués, en versant pendant la paix le sang de la guerre ; il a mis de ce sang sur la ceinture qu'il avait sur ses reins et sur les chaussures qu'il avait aux pieds.
\VS{6}Tu agiras selon ta sagesse, en sorte que tu ne laisseras point ses cheveux blancs descendre en paix dans le scheol.
\VS{7}Tu traiteras avec bienveillance les fils de Barzillaï, le Galaadite, et ils seront du nombre de ceux qui mangent à ta table ; car ils se sont approchés de moi quand je fuyais Absalom, ton frère.
\VS{8}Voici, tu as avec toi Schimeï, fils de Guéra, le benjamite de Bachurim, qui proféra contre moi des malédictions violentes le jour où je m'en allais à Mahanaïm. Mais il descendit au-devant de moi vers le Jourdain et je lui jurai par Yahweh, en disant : Je ne te ferai point mourir par l'épée.
\VS{9}Maintenant donc tu ne le laisseras point impuni, car tu es sage, pour savoir comment tu dois le traiter ; et tu feras descendre ses cheveux blancs ensanglantés au scheol.
\TextTitle{Mort de David ; début du règne de Salomon\FTNTT{1 Ch. 29:23-30}}
\VS{10}Ainsi David se coucha avec ses pères, il fut enseveli dans la cité de David.
\VS{11}Et le temps que David régna sur Israël fut quarante ans. Il régna sept ans à Hébron et il régna trente-trois ans à Jérusalem.
\VS{12}Et Salomon s'assit sur le trône de David, son père, et son règne fut très affermi.
\TextTitle{Mort d'Adonija}
\VS{13}Alors Adonija, fils de Haggith, vint vers Bath-Schéba, mère de Salomon et elle dit : Amènes-tu la paix ? Et il répondit : Je viens en paix.
\VS{14}Il ajouta : J'ai un mot à te dire. Elle répondit : Parle !
\VS{15}Et il dit : Tu sais bien que le royaume m'appartenait et que tout Israël s'attendait à ce que je règne. Mais la royauté s’est détournée de moi, elle est échue à mon frère parce que Yahweh la lui a donnée.
\VS{16}Maintenant donc je te demande une chose, ne me la refuse point. Elle lui répondit : Parle !
\VS{17}Et il dit : Je te prie, dis au roi Salomon, car il ne te refusera rien, qu'il me donne Abischag, la Sunamite, pour femme.
\VS{18}Bath-Schéba répondit : Et bien, je parlerai pour toi au roi.
\VS{19}Bath-Schéba se rendit auprès du roi Salomon pour lui parler en faveur d’Adonija ; et le roi se leva pour aller au-devant d’elle, il se prosterna devant elle, puis il s'assit sur son trône. On plaça un siège pour la mère du roi, et elle s'assit à sa droite.
\VS{20}Elle dit alors : J'ai une petite demande à te faire : ne me la refuse pas ! Et le roi lui répondit : Demande, ma mère, car je ne te la refuserai point.
\VS{21}Et elle dit : Qu'on donne Abischag, la Sunamite, pour femme à Adonija, ton frère.
\VS{22}Mais le roi Salomon répondit à sa mère et dit : Et pourquoi demandes-tu Abischag, la Sunamite, pour Adonija ? Demande plutôt le royaume pour lui, parce qu'il est mon frère aîné ; demande-le pour lui, pour Abiathar, le sacrificateur, et pour Joab, fils de Tseruja !
\VS{23}Alors le roi Salomon jura par Yahweh, en disant : Que Dieu me traite dans toute sa rigueur, si Adonija n'a dit cette parole contre sa propre vie !
\VS{24}Maintenant Yahweh est vivant, lui qui m'a établi, qui m'a fait asseoir sur le trône de David, mon père, et qui m'a donné une maison, selon sa promesse ! Aujourd’hui Adonija mourra.
\VS{25}Et le roi Salomon envoya Benaja, fils de Jehojada, qui le frappa, et Adonija mourut.
\TextTitle{Abiathar dépouillé de ses fonctions au temple}
\VS{26}Puis le roi dit à Abiathar, le sacrificateur : Va-t'en à Anathoth sur tes terres, car tu mérites la mort ; toutefois je ne te ferai point mourir aujourd'hui, parce que tu as porté l'arche du Seigneur Yahweh devant David, mon père ; et parce que tu as eu part à toutes les afflictions de mon père.
\VS{27}Ainsi Salomon dépouilla Abiathar de ses fonctions, afin qu'il ne fût plus sacrificateur de Yahweh pour accomplir la parole de Yahweh, qu'il avait prononcée à Silo contre la maison d'Eli.
\TextTitle{Mort de Joab ; Benaja à la tête de l’armé}
\VS{28}Le bruit en parvint à Joab, qui avait suivi le parti d’Adonija, quoiqu'il n’eût pas suivi le parti d’Absalom. Joab s'enfuit au tabernacle de Yahweh et empoigna les cornes de l'autel.
\VS{29}On alla l’apprendre au roi Salomon, en disant : Joab s'en est enfui dans la tente de Yahweh et il est auprès de l'autel. Salomon envoya Benaja, fils de Jehojada, et lui dit : Va et frappe-le.
\VS{30}Benaja entra dans la tente de Yahweh et dit à Joab : Ainsi a parlé le roi : Sors de là ! Mais il répondit : Non ! Je veux mourir ici. Et Benaja rapporta la chose au roi en disant : Joab m'a parlé ainsi et c’est ainsi qu’il m’a répondu.
\VS{31}Et le roi dit à Benaja : Fais comme il t'a dit, frappe-le et enterre-le ; tu ôteras ainsi de dessus moi et de dessus la maison de mon père le sang que Joab a répandu sans cause.
\VS{32}Et Yahweh fera retomber son sang sur sa tête, car il a frappé deux hommes plus justes et meilleurs que lui et les a tués par l'épée, sans que mon père David n’en sût rien : Abner, fils de Ner, chef de l'armée d'Israël, et Amassa, fils de Jéther, chef de l'armée de Juda.
\VS{33}Leur sang retombera sur la tête de Joab et sur la tête de sa postérité à perpétuité ; mais il y aura paix à toujours de par Yahweh, pour David, pour sa postérité, pour sa maison et pour son trône.
\VS{34}Donc Benaja, fils de Jehojada, monta, et il frappa Joab à mort. On l'ensevelit dans sa maison, dans le désert.
\VS{35}Alors le roi établit Benaja, fils de Jehojada, sur l'armée à la place de Joab ; le roi établit aussi Tsadok sacrificateur à la place d'Abiathar.
\TextTitle{Mort de Schimeï}
\VS{36}Puis le roi fit appeler Schimeï et lui dit : Bâtis-toi une maison à Jérusalem, et demeures-y, et n'en sors point pour aller de côté ou d'autre.
\VS{37}Car sache que le jour où tu en sortiras et que tu passeras le torrent de Cédron, tu mourras certainement ; ton sang sera sur ta tête.
\VS{38}Schimeï répondit au roi : Cette parole est bonne ! Ton serviteur fera tout ce que le roi mon Seigneur a dit. Ainsi Schimeï demeura à Jérusalem plusieurs jours.
\VS{39}Mais il arriva qu'au bout de trois ans, deux serviteurs de Schimeï s'enfuirent vers Akisch, fils de Maaca, roi de Gath, et on le rapporta à Schimeï en disant : Voilà tes serviteurs sont à Gath.
\VS{40}Alors Schimeï se leva, sella son âne, et s'en alla à Gath vers Akisch pour chercher ses serviteurs. Schimeï s'en alla donc et ramena de Gath ses serviteurs.
\VS{41}On rapporta à Salomon que Schimeï était allé de Jérusalem à Gath, et qu'il était de retour.
\VS{42}Et le roi envoya appeler Schimeï, et lui dit : Ne t'avais-je pas fait jurer par Yahweh, et ne t'avais-je pas fait cette déclaration formelle : Sache-le, sache bien que le jour que tu sortiras pour aller de côté ou d’autre, tu mourras ? Et ne me répondis-tu pas : La parole que j'ai entendue est bonne ?
\VS{43}Pourquoi donc n'as-tu pas observé le serment que tu as fait par Yahweh et le commandement que je t'avais donné ?
\VS{44}Le roi dit aussi à Schimeï : Tu sais en ton cœur tout le mal que tu as fait à David, mon père ; c'est pourquoi Yahweh a fait retomber ta méchanceté sur ta tête.
\VS{45}Mais le roi Salomon sera béni, et le trône de David sera affermi devant Yahweh à jamais.
\VS{46}Et le roi donna commission à Benaja fils de Jehojada, qui sortit, et frappa Schimeï et Schimeï mourut. La royauté fut ainsi affermie entre les mains de Salomon.
\Chap{3}
\TextTitle{Salomon s'allie à Pharaon}
\VerseOne{}Or, Salomon s'allia avec Pharaon roi d'Egypte. Il prit pour femme la fille de Pharaon, et l'amena en la cité de David, jusqu'à ce qu'il eût achevé de bâtir sa maison, la maison de Yahweh, et la muraille de Jérusalem tout alentour.
\VS{2}Seulement le peuple sacrifiait dans les hauts lieux, parce que jusqu’alors on n’avait pas bâti de maison au nom de Yahweh.
\Chap{3}
\TextTitle{Salomon demande la sagesse à Yahweh\FTNTT{2 Ch. 1:2-10}}
\VS{3}Salomon aimait Yahweh, il marchait selon les ordonnances de David son père. Seulement, c’était sur les hauts lieux qu’il offrait des sacrifices et des parfums.
\VS{4}Le roi se rendit à Gabaon pour y sacrifier, car c'était le plus grand des hauts lieux. Et Salomon offrit mille holocaustes sur cet autel.
\VS{5}Et Yahweh apparut de nuit à Salomon à Gabaon dans un songe, et Dieu lui dit : Demande ce que tu veux que je te donne.
\VS{6}Et Salomon répondit : Tu as usé d'une grande bienveillance envers ton serviteur David, mon père, parce qu’il a marché devant toi fidèlement, dans la justice, et dans la droiture de cœur envers toi. Tu as gardé cette grande bienveillance envers lui en lui donnant un fils qui est assis sur son trône, comme on le voit aujourd'hui.
\VS{7}Or, maintenant, ô Yahweh mon Dieu ! Tu as fait régner ton serviteur à la place de David, mon père, et je ne suis qu'un jeune homme, je ne sais comment me conduire.
\VS{8}Ton serviteur est parmi ce peuple que tu as choisi, un peuple nombreux qui ne peut être compté ni dénombré à cause de sa multitude.
\VS{9}Accorde donc à ton serviteur un cœur intelligent pour juger ton peuple, pour discerner le bien du mal ! Car qui pourrait juger ce peuple si grand ?
\TextTitle{Yahweh exauce Salomon\FTNTT{2 Ch. 1:11-13}}
\VS{10}Cette demande de Salomon plut à Yahweh.
\VS{11}Et Dieu lui dit : Puisque c’est là ta demande et que tu n'as point demandé une longue vie, ni les richesses, ni la mort de tes ennemis, mais que tu as demandé de l'intelligence pour rendre justice,
\VS{12}voici, je fais selon ta parole. Voici, je te donne un cœur sage et intelligent, de sorte qu'il n'y aura eu personne de semblable avant toi et qu’il n'y en aura jamais de semblable après toi.
\VS{13}Et même, je te donne ce que tu n'as point demandé, les richesses et la gloire, de sorte qu'il n'y aura point de roi semblable à toi entre les rois, tant que tu vivras.
\VS{14}Et si tu marches dans mes voies pour garder mes ordonnances et mes commandements, comme David, ton père, je prolongerai tes jours.
\VS{15}Salomon s’éveilla. Et voilà le songe. Puis il s'en retourna à Jérusalem et se tint devant l'Arche de l'alliance de Yahweh. Là, il offrit des holocaustes et des offrandes de paix et fit un festin à tous ses serviteurs.
\VS{16}Alors deux femmes prostituées vinrent au roi et se présentèrent devant lui.
\VS{17}Et l'une de ces femmes dit : Hélas, mon Seigneur ! Nous demeurions cette femme-ci et moi dans une même maison et j'ai accouché près d’elle dans cette maison-là.
\VS{18}Trois jours après, cette femme a aussi accouché. Et nous étions ensemble, il n'y avait aucun étranger avec nous dans cette maison, il n’y avait que nous deux.
\VS{19}Or, l'enfant de cette femme est mort la nuit, parce qu'elle s'était couchée sur lui.
\VS{20}Elle s'est levée au milieu de la nuit, et a pris mon fils à mes côtés pendant que ta servante dormait, et l'a couché dans son sein. Et son fils mort, elle l’a couché dans mon sein.
\VS{21}Le matin, je me suis levée pour allaiter mon fils. Et voici, il était mort. Je l’ai regardé attentivement ce matin-là ; et voici, ce n'était point mon fils que j'avais enfanté.
\VS{22}L’autre femme dit : Non, c’est mon fils qui est vivant, et c’est ton fils qui est mort. Mais la première répliqua : Nullement ! Celui qui est mort est ton fils, et c’est mon fils qui vit. Elles parlaient ainsi devant le roi.
\VS{23}Et le roi dit : L’une dit : C’est mon fils qui est vivant, et c’est ton fils qui est mort ; l’autre dit : Nullement ! C’est ton fils qui est mort, et c’est mon fils qui est vivant.
\VS{24}Alors le roi dit : Apportez-moi une épée ! Et on apporta une épée devant le roi.
\VS{25}Puis le roi dit : Partagez en deux l'enfant qui vit, et donnez-en la moitié à l'une et la moitié à l'autre.
\VS{26}Alors la femme dont le fils était vivant sentit ses entrailles s’émouvoir pour son fils, et elle dit au roi : Ah ! Mon seigneur, qu'on donne à celle-ci l'enfant qui vit et qu'on ne le fasse pas mourir ! Mais l'autre dit : Il ne sera ni à moi ni à toi ; qu'on le partage.
\VS{27}Alors le roi répondit et dit : Donnez à la première l'enfant qui vit, et ne le faites pas mourir. C’est elle qui est sa mère.
\VS{28}Tout Israël entendit parler du jugement que le roi avait prononcé. Et l’on craignit le roi, car l’on reconnut que la sagesse divine était en lui pour rendre justice.
\Chap{4}
\TextTitle{Salomon établit onze chefs et douze intendants}
\VerseOne{}Le roi Salomon était roi sur tout Israël.
\VS{2}Voici les chefs qu’il avait à son service. Azaria, fils du sacrificateur Tsadok,
\VS{3}Elihoreph et Achija, enfants de Schischa, secrétaires ; Josaphat, fils d'Achilud, archiviste ;
\VS{4}Benaja, fils de Jehojada, commandait l'armée ; Tsadok et Abiathar étaient sacrificateurs ;
\VS{5}Azaria, fils de Nathan était chef des intendants ; Zabud, fils de Nathan, était le ministre d’état, favori du roi ;
\VS{6}Achischar, chef de la maison du roi ; et Adoniram, fils d’Abda, préposé sur les impôts.
\VS{7}Or, Salomon avait douze intendants sur tout Israël, qui veillaient à l’entretien du roi et de sa maison ; et chacun pendant un mois de l'année.
\VS{8}Voici leurs noms : Le fils de Hur, sur la montagne d'Ephraïm.
\VS{9}Le fils de Déker, sur Makats, sur Saalbim, sur Beth-Schémesch, à Elon de Beth-Hanan.
\VS{10}Le fils de Hésed, à Arubboth ; il avait Soco et tout le pays de Hépher.
\VS{11}Le fils d'Abinadab avait toute la contrée de Dor ; il avait Thaphath, fille de Salomon, pour femme.
\VS{12}Baana, fils d'Achilud, avait Thaanac et Meguiddo, et tout le pays de Beth-Schean qui est près de Tsarthan au-dessous de Jizreel, depuis Beth-Schean jusqu'à Abel-Mehola et jusqu'au-delà de Jokmeam.
\VS{13}Le fils de Guéber, à Ramoth en Galaad ; il avait les bourgs de Jaïr, fils de Manassé, en Galaad ; il avait aussi toute la contrée d'Argob en Basan, soixante grandes villes à murailles et garnies de barres d'airain.
\VS{14}Achinadab, fils d’Iddo, à Mahanaïm.
\VS{15}Achimaats, qui avait pour femme Basmath, fille de Salomon, en Nephthali.
\VS{16}Baana, fils de Huschaï, en Aser et sur Bealoth.
\VS{17}Josaphat, fils de Paruach, à Issacar.
\VS{18}Schimeï, fils d'Ela, en Benjamin.
\VS{19}Guéber, fils d'Uri, dans le pays de Galaad, le pays de Sihon, roi des Amoréens, et d’Og, roi de Basan ; et il était seul intendant de ce pays-là.
\TextTitle{L'étendue de la domination du royaume}
\VS{20}Juda et Israël étaient en grand nombre, semblable au sable sur le bord de la mer ; ils mangeaient, buvaient et se réjouissaient.
\VS{21}Et Salomon dominait sur tous les royaumes depuis le fleuve jusqu'au pays des Philistins et jusqu'à la frontière d'Egypte ; ils apportaient des présents, et lui furent assujettis pendant toute sa vie.
\VS{22}Or, les vivres de Salomon pour chaque jour étaient de trente cors de fine farine et soixante d'autre farine,
\VS{23}dix bœufs gras, vingt bœufs de pâturages, et cent moutons, outre les cerfs, les daims et les volailles engraissées.
\VS{24}Il dominait sur toutes les contrées de l’autre côté du fleuve, depuis Thiphsach jusqu'à Gaza, sur tous les rois qui étaient de l’autre côté du fleuve. Il était en paix avec tous les pays alentour.
\VS{25}Juda et Israël habitèrent en sécurité chacun sous sa vigne et sous son figuier, depuis Dan jusqu'à Beer-Schéba, durant toute la vie de Salomon.
\VS{26}Salomon avait aussi quarante mille crèches pour les chevaux destinés à ses chars et douze mille hommes de cheval.
\VS{27}Or, les intendants pourvoyaient à l’entretien du roi Salomon et de tous ceux qui s'approchaient de sa table, chacun en son mois ; ils ne les laissaient manquer de rien.
\VS{28}Ils faisaient aussi venir de l'orge et de la paille pour les chevaux et les coursiers dans le lieu où se trouvait le roi, chacun selon les ordres qu'il avait reçus.
\TextTitle{La sagesse de Salomon connue de toute la terre}
\VS{29}Dieu donna à Salomon de la sagesse, une très grande intelligence, et des connaissances multipliées comme le sable qui est sur le bord de la mer.
\VS{30}La sagesse de Salomon surpassait la sagesse de tous les fils de l’orient et toute la sagesse des égyptiens.
\VS{31}Il était plus sage qu’aucun homme, plus qu'Ethan, l’Ezrachite, plus qu'Héman, Calcol et Darda, les fils de Machol ; et sa renommée était répandue parmi toutes les nations d'alentour.
\VS{32}Il a prononcé trois mille paraboles et composa cinq mille cantiques.
\VS{33}Il a aussi parlé des arbres, depuis le cèdre du Liban jusqu'à l'hysope qui sort de la muraille ; il a aussi parlé sur les animaux, sur les oiseaux, sur les reptiles et sur les poissons.
\VS{34}Il venait des gens d'entre tous les peuples pour entendre la sagesse de Salomon, de la part de tous les rois de la terre qui avaient entendu parler de sa sagesse.
\Chap{5}
\TextTitle{Salomon prépare la construction du temple\FTNTT{2 Ch. 2:1 ; 13:16}}
\VerseOne{}Hiram, roi de Tyr, envoya ses serviteurs vers Salomon, car il apprit qu'on l'avait oint pour roi à la place de son père, car Hiram avait toujours aimé David.
\VS{2}Et Salomon fit dire à Hiram :
\VS{3}Tu sais que David, mon père, n'a pu bâtir une maison à Yahweh, son Dieu, à cause des guerriers qui l'ont encerclé, jusqu'à ce que Yahweh les ait mis sous la plante de ses pieds.
\VS{4}Maintenant Yahweh, mon Dieu, m'a donné du repos de toutes parts, et je n'ai plus d’adversaires, plus de calamités !
\VS{5}Voici donc j’ai l’intention de bâtir une maison au nom de Yahweh, mon Dieu, comme Yahweh l’a promis à David, mon père, en disant : Ton fils que je mettrai à ta place sur ton trône sera celui qui bâtira une maison à mon nom.
\VS{6}Ordonne maintenant que l’on coupe des cèdres du Liban pour moi. Mes serviteurs seront avec les tiens, et je donnerai pour tes serviteurs le salaire que tu auras fixé ; car tu sais qu'il n'y a personne parmi nous qui sache couper le bois comme les Sidoniens.
\VS{7}Lorsque Hiram eut entendu les paroles de Salomon, il eut une grande joie et il dit : Béni soit aujourd'hui Yahweh, qui a donné à David un fils sage pour chef de ce grand peuple !
\VS{8}Hiram fit répondre à Salomon : J'ai entendu ce que tu m'as envoyé dire et je ferai tout ce qui te plaira au sujet des bois de cèdre et des bois de cyprès.
\VS{9}Mes serviteurs les descendront du Liban à la mer, puis je les expédierai sur la mer par radeaux jusqu'au lieu que tu m'auras indiqué ; là je les ferai délier, et tu les prendras. Ce que je désire en retour, c’est que tu fournisses des vivres à ma maison.
\VS{10}Hiram donna du bois de cèdre et du bois de cyprès à Salomon autant qu'il en voulait.
\VS{11}Et Salomon donna à Hiram vingt mille cors de froment pour la nourriture de sa maison et vingt cors d'huile d’olives concassées ; Salomon en donna autant à Hiram chaque année.
\VS{12}Et Yahweh donna de la sagesse à Salomon, comme il le lui avait promis ; et il y eut paix entre Hiram et Salomon, et ils firent alliance ensemble.
\TextTitle{Les hommes de corvée\FTNTT{2 Ch. 2:2 ; 17:18}}
\VS{13}Le roi Salomon leva sur tout Israël des hommes de corvée ; ils étaient au nombre de trente mille hommes.
\VS{14}Il en envoya dix mille au Liban chaque mois, tour à tour, ils étaient un mois au Liban, et deux mois chez eux. Adoniram était préposé sur les hommes de corvée.
\VS{15}Salomon avait aussi soixante-dix mille hommes qui portaient les fardeaux et quatre-vingt mille qui taillaient les pierres dans la montagne,
\VS{16}sans compter les chefs au nombre de trois mille trois cents, préposés par Salomon sur le suivi des travaux, et chargés de surveiller les ouvriers.
\VS{17}Le roi ordonna d’extraire de grandes et précieuses pierres, pour faire le fondement de la maison, qui soient toutes taillées,
\VS{18}de sorte que les maçons de Salomon et ceux d'Hiram, taillèrent les pierres et préparèrent le bois et les pierres pour bâtir la maison.
\Chap{6}
\TextTitle{Construction du temple de Yahweh\FTNTT{2 Ch. 3:1-14}}
\VerseOne{}Ce fut la quatre cent quatre-vingtième année après la sortie des enfants d'Israël du pays d'Egypte que Salomon bâtit la maison de Yahweh\FTNTT{Voir les annexes «~Le temple de Salomon~»}, la quatrième année du règne de Salomon sur Israël, au mois de Ziv, qui est le second mois.
\VS{2}La maison que le roi Salomon bâtit à Yahweh avait soixante coudées de long, vingt de large, et trente de haut.
\VS{3}Le portique devant le temple de la maison avait vingt coudées de longueur, répondant à la largeur de la maison, et il avait dix coudées de profondeur sur le devant de la maison.
\VS{4}Il fit placer des fenêtres à la maison, fenêtres solidement grillées.
\VS{5}Il bâtit contre la muraille de la maison, à l’entour, des étages qui entouraient les murs de la maison, le temple et le sanctuaire ainsi il fit des chambres latérales tout autour.
\VS{6}L’étage inférieur était large de cinq coudées, celui du milieu de six coudées et le troisième de sept coudées ; car il avait aménagé des retraites à la maison tout autour en dehors, afin que la charpente n'entrât pas dans les murailles de la maison.
\VS{7}Pour bâtir la maison, on se servit de pierres déjà taillées, de sorte qu'en bâtissant la maison on n'entendit ni marteau, ni hache, ni aucun outil de fer.
\VS{8}L'entrée des chambres de l’étage inférieur était au côté droit de la maison, et on montait à l’étage du milieu par un escalier tournant, et de l’étage du milieu au troisième.
\VS{9}Après avoir achevé de bâtir la maison, Salomon couvrit la maison de planches et de poutres de cèdre.
\VS{10}Et il bâtit les étages joignant toute la maison, avec chacun cinq coudées de haut, et il les lia à la maison par des bois de cèdre.
\VS{11}Alors la parole de Yahweh fut adressée à Salomon, en ces termes :
\VS{12}Quant à cette maison que tu bâtis, si tu marches dans mes statuts, si tu pratiques mes ordonnances et que tu gardes tous mes commandements pour y marcher, j’accomplirai en ta faveur la parole que j'ai dite à David, ton père.
\VS{13}Et j'habiterai au milieu des enfants d'Israël, et je n'abandonnerai point mon peuple d'Israël.
\VS{14}Ainsi Salomon bâtit la maison et l'acheva.
\VS{15}Il revêtit de cèdre les murs de la maison, depuis le sol jusqu'au plafond ; il revêtit ainsi de bois l’intérieur, et il couvrit le sol de la maison de planches de cyprès.
\VS{16}Il revêtit aussi l'espace de vingt coudées de planches de cèdre à partir du fond de la maison, depuis le sol jusqu'au haut des murailles, et il bâtit cet espace au dedans pour en faire le sanctuaire, le saint des saints.
\VS{17}Les quarante coudées sur le devant formaient la maison, c’est-à-dire le temple.
\VS{18}Le bois de cèdre à l’intérieur de la maison était sculpté en coloquintes et en fleurs épanouies ; tout l’intérieur était de cèdre, on ne voyait aucune pierre.
\VS{19}Salomon disposa aussi le sanctuaire, au dedans de la maison vers le fond, pour y mettre l'arche de l'alliance de Yahweh.
\VS{20}Le sanctuaire avait par devant vingt coudées de long, vingt coudées de large, et vingt coudées de haut, et on le couvrit d’or pur ; on en couvrit aussi l'autel, fait de planches de bois de cèdre.
\VS{21}Salomon couvrit d’or pur l’intérieur de la maison, et fit passer un voile avec des chaînes d'or au-devant du sanctuaire, qu’il couvrit également d'or.
\VS{22}Ainsi il couvrit d'or la maison tout entière. Il couvrit aussi d'or tout l'autel qui était devant le sanctuaire.
\VS{23}Et il fit dans le sanctuaire deux chérubins de bois d'olivier sauvage, qui avaient chacun dix coudées de haut.
\VS{24}Chacune des ailes de l'un des chérubins avait cinq coudées et les ailes de l’autre chérubin avaient aussi cinq coudées ; depuis le bout d'une aile jusqu'au bout de l'autre aile il y avait donc dix coudées.
\VS{25}Le second chérubin était aussi de dix coudées. Les deux chérubins étaient d'une même mesure et taillés l'un comme l'autre.
\VS{26}La hauteur de chacun des deux chérubins était de dix coudées.
\VS{27}Salomon plaça les chérubins à l’intérieur, au milieu de la maison. Les ailes des chérubins étaient déployées : l'aile de l'un touchait à l’un des murs, l'aile de l'autre chérubin touchait à l'autre mur ; et leurs autres ailes se rencontraient par l’extrémité au milieu de la maison.
\VS{28}Salomon couvrit d'or les chérubins.
\VS{29}Il fit sculpter sur tout le pourtour des murs de la maison, à l’intérieur et à l’extérieur, des sculptures en relief de chérubins, des palmes et des fleurs épanouies.
\VS{30}Il couvrit aussi d'or le sol de la maison, tant à l’intérieur qu’au-dehors.
\VS{31}A l'entrée du sanctuaire, il fit une porte à deux battants de bois d'olivier sauvage, dont les linteaux avec les poteaux équivalaient à un cinquième du mur.
\VS{32}Les deux battants étaient de bois d'olivier sauvage. Il y fit sculpter des chérubins, des palmes et des fleurs épanouies qu’il couvrit d'or, étendant également l'or sur les chérubins et sur les palmes.
\VS{33}Il fit aussi, à l'entrée du temple, des poteaux de bois d'olivier sauvage, du quart de la dimension du mur.
\VS{34}Les deux battants étaient de bois de sapin ; chacun des battants était formé de deux planches brisées.
\VS{35}Il y fit sculpter des chérubins, des palmes et des fleurs épanouies, et les couvrit d'or, proprement posé sur la sculpture.
\VS{36}Il bâtit aussi le parvis de l’intérieur de trois rangées de pierres de taille et d'une rangée de poutres de cèdre.
\VS{37}La quatrième année, au mois de Ziv, les fondements de la maison de Yahweh furent posés.
\VS{38}Et la onzième année, au mois de Bul, qui est le huitième mois, la maison fut achevée dans toutes ses parties et telle qu’elle devait être. Salomon la construisit en l’espace de sept années.
\Chap{7}
\TextTitle{Construction du palais royal}
\VerseOne{}Salomon bâtit aussi sa maison, et l'acheva complètement en treize ans.
\VS{2}Il bâtit d’abord la maison de la forêt du Liban, de cent coudées de long, de cinquante coudées de large, et de trente coudées de haut, sur quatre rangées de colonnes de cèdre ; et sur les colonnes il y avait des poutres de cèdre.
\VS{3}On couvrit de bois de cèdre les chambres qui portaient sur les colonnes qui étaient au nombre de quarante-cinq, quinze par étages.
\VS{4}Et il y avait trois rangées de fenêtrages ; et une fenêtre répondait à l'autre en trois endroits.
\VS{5}Toutes les portes et tous les poteaux étaient formés de poutres carrées, avec les fenêtres ; et à chacun des trois étages, les ouvertures étaient en vis-à-vis les unes des autres.
\VS{6}Il fit aussi le portique de colonnes, long de cinquante coudées, et large de trente coudées ; et un autre portique en avant avec des colonnes et des degrés sur leur front.
\VS{7}Il fit aussi le portique du trône sur lequel il rendait ses jugements, appelé le portique du jugement ; on le couvrit de cèdre depuis un bout du sol jusqu'à l'autre.
\VS{8}La maison où il demeurait fut construite de la même manière, dans une autre cour, derrière le portique. Salomon fit une maison bâtie comme ce portique à la fille de Pharaon, qu'il avait prise pour femme.
\VS{9}Toutes ces constructions étaient de pierres de prix, taillées d’après des mesures, sciées à la scie, en dedans et en dehors, depuis les fondements jusqu'aux corniches, et par dehors jusqu'au grand parvis.
\VS{10}Le fondement était en pierres magnifiques et de grand prix, de grandes pierres, des pierres de dix coudées et des pierres de huit coudées.
\VS{11}Et par-dessus il y avait des pierres de prix, taillées d’après des mesures, et du bois de cèdre.
\VS{12}Et le grand parvis avait aussi tout alentour trois rangées de pierres de taille et une rangée de poutres de cèdre, comme le parvis intérieur de la maison de Yahweh, et le portique de la maison.
\TextTitle{Hiram, artisan spécialiste en airain\FTNTT{2 Ch. 2:12-13}}
\VS{13}Or, le roi Salomon fit venir de Tyr Hiram ;
\VS{14}fils d'une femme veuve de la tribu de Nephthali, et d’un père tyrien, Hiram travaillait le cuivre ; fort expert, intelligent et savant pour faire toutes sortes d'ouvrages d'airain ; il arriva auprès du roi Salomon, et il fit tout son ouvrage.
\TextTitle{Les colonnes du temple\FTNTT{2 Ch. 3:15-17}}
\VS{15}Il fit les deux colonnes d'airain, la première avait dix-huit coudées de hauteur ; et un cordon de douze coudées mesurait le tour de la seconde.
\VS{16}Il fit aussi deux chapiteaux d'airain fondu pour mettre sur les sommets des colonnes ; le premier chapiteau était de cinq coudées de hauteur, le second était aussi de cinq coudées.
\VS{17}Il fit des treillis en forme de maillages, des festons façonnés en forme de chaînes, pour les chapiteaux qui étaient sur le sommet des colonnes, sept pour le premier des chapiteaux, et sept pour le second.
\VS{18}Il fit deux rangs de grenades autour de l’un des treillis, pour couvrir le chapiteau qui était sur le sommet d'une des colonnes ; et il fit de même pour l'autre chapiteau.
\VS{19}Dans le portique, les chapiteaux qui étaient sur le sommet des colonnes figuraient des fleurs de lis hautes de quatre coudées au porche.
\VS{20}Ces chapiteaux placés sur les deux colonnes étaient entourés de deux cents grenades, en haut, depuis le renflement qui était au-delà du treillis ; il y avait aussi deux cents grenades, disposées par rangs, autour du second chapiteau.
\VS{21}Il dressa donc les colonnes au portique du temple. Il dressa la colonne de droite qu’il nomma Jakin ; puis il dressa la colonne de gauche qu’il nomma Boaz.
\VS{22}Et l’on mit sur le chapiteau des colonnes l'ouvrage figurant des fleurs de lis ; ainsi l'ouvrage des colonnes fut achevé.
\TextTitle{La mer de fonte\FTNTT{2 Ch. 4:2-5}}
\VS{23}Il fit aussi la mer de fonte. Elle avait dix coudées d'un bord à l'autre, ronde tout autour, avec cinq coudées de haut ; et un cordon de trente coudées en mesurait le tour.
\VS{24}Au-dessous de son bord, des coloquintes l'environnaient, dix à chaque coudée, lesquelles faisaient tout le tour de la mer. Il y avait deux rangées de coloquintes, jetées en fonte.
\VS{25}Et elle était posée sur douze bœufs, dont trois regardaient le nord et trois regardaient l'occident, trois regardaient le sud et trois regardaient l'orient. La mer était sur eux et toute la partie postérieure de leur corps était tournée en dedans.
\VS{26}Son épaisseur était d'une paume, et son bord était comme le bord d'une coupe en fleur de lis ; elle contenait deux mille baths.
\TextTitle{Les dix socles d'airain}
\VS{27}Il fit aussi dix socles d'airain, ayant chacun quatre coudées de long, quatre coudées de large et trois coudées de haut.
\VS{28}Ces socles étaient réalisés de telle manière qu'il y avait des panneaux enchâssés entre leurs bordures.
\VS{29}Sur les panneaux qui étaient entre les bordures, il y avait des lions, des bœufs et des chérubins. Et sur les bordures, au-dessus et en dessous des lions et des bœufs, il y avait des ornements qui pendaient en festons.
\VS{30}Chaque socle avait quatre roues d'airain avec des essieux d'airain. Ses quatre pieds leur servaient d’appuis. Ces appuis étaient fondus au-dessous de la cuve, et au-dessus étaient les festons.
\VS{31}Le couronnement offrait à son intérieur une ouverture avec un prolongement d'une coudée vers le haut ; cette ouverture était arrondie comme pour les ouvrages de ce genre et elle avait une coudée et demie de largeur. Il s’y trouvait aussi des sculptures ; les panneaux étaient carrés, et non arrondis.
\VS{32}Les quatre roues étaient sous les panneaux, et les essieux des roues fixés à la base ; chaque roue était haute d'une coudée et demie.
\VS{33}Les roues étaient faites comme les roues de chars ; leurs essieux, leurs jantes, leurs rais et leurs moyeux étaient tous de fonte.
\VS{34}Il y avait aux quatre angles de chaque socle quatre consoles d’une même pièce que la base.
\VS{35}La partie supérieure de la base se terminait par un cercle d’une demi-coudée de hauteur, et elle avait ses appuis et ses panneaux de la même pièce.
\VS{36}Puis, on sculpta sur la surface de ses appuis et sur ses panneaux, des chérubins, des lions et des palmes, selon les espaces libres, et des ornements tout autour.
\VS{37}Ainsi les dix socles étaient tous d’une même fonte, d’une même mesure et d’une même forme.
\TextTitle{Les dix cuves d'airain\FTNTT{2 Ch. 4:6}}
\VS{38}Il fit aussi dix cuves d'airain, dont chacune contenait quarante baths, et chaque cuve était de quatre coudées, chaque cuve était sur l’un des dix socles.
\VS{39}Il mit cinq socles au côté droit de la maison, et cinq au côté gauche de la maison ; quant à la mer, il l’a mis au côté droit de la maison, vers l'orient du côté sud.
\TextTitle{Totalité de l’œuvre d’Hiram}
\VS{40}Ainsi Hiram fit les cuves, les pelles et les bassins, et il acheva tout l'ouvrage qu'il faisait au roi Salomon pour la maison de Yahweh.
\VS{41}Savoir, deux colonnes avec les deux chapiteaux qui étaient sur le sommet des colonnes ; et deux maillages pour couvrir les deux bourrelets des chapiteaux qui étaient sur le sommet des colonnes ;
\VS{42}les quatre cents grenades pour les deux maillages, deux rangs de grenades pour chaque réseau, pour couvrir les deux renflements des chapiteaux, qui étaient sur les colonnes ;
\VS{43}les dix socles ; et les dix cuves pour mettre sur les socles ;
\VS{44}la mer avec les douze bœufs sous la mer ;
\VS{45}les pots, les pelles et les bassins. Tous ces ustensiles que Hiram fit au roi Salomon pour la maison de Yahweh étaient d'airain poli.
\VS{46}Le roi les fit fondre dans la plaine du Jourdain, dans un sol argileux, entre Succoth et Tsarthan.
\VS{47}Et Salomon ne pesa aucun de ces ustensiles, parce qu'ils étaient en trop grand nombre, de sorte qu'on ne rechercha point le poids de l’airain.
\TextTitle{Divers ustensiles d’or pour la maison de Yahweh}
\VS{48}Salomon fit aussi tous les ustensiles pour la maison de Yahweh, savoir l'autel d'or, et les tables d'or, sur lesquelles étaient les pains de proposition ;
\VS{49}les chandeliers d’or pur, cinq à droite et cinq à gauche devant le sanctuaire, avec les fleurs, les lampes et les mouchettes d'or ;
\VS{50}les coupes, les couteaux, les bassins, les tasses et les brasiers d’or pur. Les gonds, même des portes de la maison, à l’entrée du saint des saints, à la porte de la maison et à l’entrée du temple, étaient d'or.
\VS{51}Ainsi fut achevé tout l'ouvrage que le roi Salomon fit pour la maison de Yahweh ; puis il y fit apporter l'or, l’argent et les ustensiles que David, son père, avait consacrés ; il les mit dans les trésors de la maison de Yahweh.
\Chap{8}
\TextTitle{L’arche de l’alliance placée dans le saint des saints ; la gloire de Yahweh remplit le temple \FTNTT{2 Ch. 5:2-14}}
\VerseOne{}Alors le roi Salomon convoqua près de lui à Jérusalem les anciens d'Israël, tous les chefs des tribus et les chefs de famille des fils d'Israël, pour transporter l'arche de l'alliance de Yahweh de la cité de David, qui est Sion.
\VS{2}Tous les hommes d'Israël s’assemblèrent auprès du roi Salomon, au mois d'Ethanim, qui est le septième mois, pendant la fête.
\VS{3}Une fois tous les anciens d'Israël arrivés, les sacrificateurs portèrent l'arche.
\VS{4}Ils transportèrent l'arche de Yahweh, la tente d'assignation, et tous les ustensiles qui étaient dans le tabernacle ; les sacrificateurs et les Lévites les emportèrent.
\VS{5}Le roi Salomon et toute l'assemblée d'Israël convoquée auprès de lui se tinrent devant l'arche. Ils sacrifièrent du gros et du menu bétail en si grand nombre, qu'on ne pouvait ni nombrer ni compter.
\VS{6}Et les sacrificateurs portèrent l'arche de l'alliance de Yahweh à sa place, dans le sanctuaire de la maison, dans le saint des saints, sous les ailes des chérubins.
\VS{7}Car les chérubins avaient les ailes étendues sur l’emplacement de l'arche, et ils couvraient l'arche et ses barres par-dessus.
\VS{8}On avait donné aux barres une longueur telle que leurs extrémités se voyaient du lieu saint devant le sanctuaire, mais elles ne se voyaient point du dehors. Elles sont demeurées là jusqu'à ce jour.
\VS{9}Il n'y avait rien dans l'arche que les deux tables de pierre que Moïse y déposa en Horeb, lorsque Yahweh fit alliance avec les enfants d'Israël à leur sortie du pays d'Egypte.
\VS{10}Au moment où les sacrificateurs sortirent du lieu saint, la nuée remplit la maison de Yahweh.
\VS{11}Les sacrificateurs ne purent pas y rester pour faire le service, à cause de la nuée ; car la gloire de Yahweh remplissait la maison de Yahweh.
\TextTitle{Discours de Salomon\FTNTT{2 Ch. 6:1-11}}
\VS{12}Alors Salomon dit : Yahweh veut habiter dans l'obscurité !
\VS{13}J'ai achevé de bâtir une maison pour ta demeure ô Yahweh ! Ce sera une demeure, un lieu où tu résideras éternellement.
\VS{14}Le roi tourna son visage, et bénit toute l'assemblée d'Israël ; car toute l'assemblée d'Israël se tenait là debout.
\VS{15}Et il dit : Béni soit Yahweh, le Dieu d'Israël, qui a parlé de sa propre bouche à David, mon père, et qui a accompli par sa puissance ce qu’il avait déclaré en disant :
\VS{16}Depuis le jour où je fis sortir mon peuple d'Israël hors d'Egypte, je n'ai choisi aucune ville d'entre toutes les tribus d'Israël pour y bâtir une maison afin que mon nom y fût, mais j'ai choisi David pour qu’il règne sur mon peuple d'Israël.
\VS{17}David, mon père, avait à cœur de bâtir une maison au nom de Yahweh, le Dieu d'Israël.
\VS{18}Et Yahweh dit à David, mon père : Puisque tu as eu à cœur de bâtir une maison à mon nom, tu as bien fait d’avoir eu cette intention.
\VS{19}Néanmoins, tu ne bâtiras point cette maison, mais ton fils qui sortira de tes entrailles sera celui qui bâtira cette maison à mon Nom.
\VS{20}Yahweh a donc accompli la parole qu'il avait prononcée. Je me suis élevé à la place de David, mon père, et me suis assis sur le trône d'Israël, comme Yahweh l’avait annoncé, et j'ai bâti cette maison au Nom de Yahweh, le Dieu d'Israël.
\VS{21}J'y ai établi ici un lieu pour l'arche, dans lequel est l'alliance de Yahweh, qu'il traita avec nos pères quand il les fit sortir hors du pays d'Egypte.
\TextTitle{Prière de Salomon\FTNTT{2 Ch. 6:12-42}}
\VS{22}Ensuite Salomon se tint devant l'autel de Yahweh en la présence de toute l'assemblée d'Israël, et étendant ses mains vers les cieux,
\VS{23}il dit : Ô Yahweh, Dieu d'Israël ! Il n'y a point de Dieu semblable à toi en haut dans les cieux, ni en bas sur la terre ; tu gardes l'alliance et la miséricorde envers tes serviteurs qui marchent devant ta face de tout cœur !
\VS{24}Ainsi tu as tenu parole à ton serviteur David, mon père, car ce que tu as déclaré de ta bouche, tu l'as accompli en ce jour par ta main puissante.
\VS{25}Maintenant donc, ô Yahweh, Dieu d'Israël, prête attention à la promesse faite à ton serviteur David, mon père, en lui disant : Tu ne manqueras jamais devant moi d’un successeur assis sur le trône d'Israël, pourvu seulement que tes fils prennent garde à leur voie et qu’ils marchent devant ma face, comme tu y as marché.
\VS{26}Et maintenant, ô Dieu d'Israël ! Je te prie, que s’accomplisse la promesse que tu as faite à ton serviteur David, mon père.
\VS{27}Mais Dieu habiterait-il véritablement sur la terre ? Voilà, les cieux, même les cieux des cieux ne peuvent te contenir ; combien moins cette maison que j'ai bâtie !
\VS{28}Toutefois, ô Yahweh, mon Dieu, sois attentif à la prière que t’adresse ton serviteur et à sa supplication, pour entendre le cri et la prière que ton serviteur t’adresse aujourd'hui.
\VS{29}Que tes yeux soient ouverts jour et nuit sur cette maison, sur le lieu dont tu as dit : Là sera mon Nom ! Ecoute la prière que ton serviteur fait en ce lieu.
\VS{30}Daigne exaucer la supplication de ton serviteur et de ton peuple d'Israël lorsqu’ils te prieront en ce lieu ; exauce du lieu de ta demeure. Des cieux, exauce, et pardonne !
\VS{31}Si quelqu'un pèche contre son prochain et qu’on lui impose un serment pour le faire jurer, et que le serment aura été fait devant ton autel dans cette maison ;
\VS{32}écoute-le des cieux, et agis. Juge tes serviteurs, condamne le coupable en lui rendant selon sa conduite ; rends justice à l’innocent, et traite-le selon son innocence !
\VS{33}Quand ton peuple d'Israël sera battu par l'ennemi, pour avoir péché contre toi, s’il revient à toi et rend gloire à ton Nom, en t’adressant des prières et des supplications dans cette maison,
\VS{34}exauce-le des cieux, et pardonne le péché de ton peuple d'Israël, et ramène-le dans la terre que tu as donnée à leurs pères.
\VS{35}Quand les cieux seront fermés et qu'il n'y aura point de pluie, à cause de ses péchés contre toi, s'il te fait une prière en ce lieu-ci, qu’il loue ton Nom, et s’il se détourne de ses péchés, parce que tu les auras affligés,
\VS{36}exauce-le des cieux, pardonne le péché de tes serviteurs et de ton peuple d'Israël, à qui tu enseigneras quel est le chemin par lequel ils doivent marcher et envoie-leur la pluie sur la terre que tu as donnée à ton peuple pour héritage !
\VS{37}Quand il y aura dans le pays, famine, peste, jaunisse, nielle, sauterelles d’une espèce ou d’une autre, même quand les ennemis assiégeront ton peuple dans son propre pays, quand il y aura un fléau ou une maladie quelconque ;
\VS{38}si un homme, si tout ton peuple d'Israël fait entendre des prières et des supplications, que chacun reconnaisse la plaie de son cœur et étende les mains vers cette maison,
\VS{39}exauce-le des cieux, du lieu de ta demeure, pardonne, et agis. Rends à chacun selon toutes ses voies, parce que tu auras connu leurs cœurs ; car toi seul connais le cœur de tous les fils des hommes ;
\VS{40}et ils te craindront toute leur vie dans le pays que tu as donné à nos pères !
\VS{41}Et même lorsque l'étranger, qui n’est pas de ton peuple d'Israël, viendra d'un pays éloigné à cause de ton Nom,
\VS{42}car on saura que ton Nom est grand, ta main puissante et ton bras étendu, quand il viendra prier dans cette maison,
\VS{43}exauce-le des cieux, du lieu de ta demeure, et fais à cet étranger selon ce qu’il t’aura demandé, afin que tous les peuples de la terre connaissent ton Nom pour te craindre, comme ton peuple d'Israël ; et pour connaître que ton Nom est invoqué sur cette maison que j'ai bâtie !
\VS{44}Quand ton peuple sortira pour combattre son ennemi, par la voie par laquelle tu l’auras envoyé, s'ils prient Yahweh en regardant vers cette ville que tu as choisie et vers cette maison que j'ai bâtie à ton Nom !
\VS{45}Exauce des cieux leurs prières et leurs supplications, et fais leur justice !
\VS{46}Quand ils pécheront contre toi, car il n'y a point d'homme qui ne pèche, et que tu seras irrité contre eux et que tu les auras livrés à leurs ennemis, qui les emmènera captifs dans un pays ennemi, lointain ou proche ;
\VS{47}si dans le pays où ils auront été menés captifs, ils reviennent à toi et t’adressent des supplications, se repentent et te prient au pays de ceux qui les auront emmenés captifs, en disant : Nous avons péché, nous avons commis l’iniquité, nous avons fait le mal !
\VS{48}S'ils reviennent à toi de tout leur cœur et de toute leur âme, dans le pays de leurs ennemis, qui les auront emmenés captifs, et s'ils t'adressent leurs prières, les regards tournés vers le pays que tu as donné à leurs pères, vers la ville que tu as choisie, vers la maison que j'ai bâtie à ton Nom,
\VS{49}exauce des cieux, du lieu de ta demeure, leurs prières et leurs supplications, et fais-leur justice.
\VS{50}Pardonne à ton peuple ses offenses et ses péchés envers toi, et fais que ceux qui les auront emmenés captifs aient pitié d'eux et leur fassent grâce,
\VS{51}car ils sont ton peuple et ton héritage, et tu les as fait sortir hors d'Egypte, du milieu d'une fournaise de fer !
\VS{52}Que tes yeux donc soient ouverts sur la supplication de ton serviteur et celle de ton peuple d'Israël, pour les exaucer dans tout ce pourquoi ils crieront à toi !
\VS{53}Car tu les as séparés de tous les autres peuples de la terre pour être ton héritage, comme tu l’as déclaré par Moïse, ton serviteur, quand tu fis sortir nos pères hors d'Egypte, ô Seigneur Yahweh !
\TextTitle{Bénédictions et réjouissances\FTNTT{2 Ch. 7:4-10}}
\VS{54}Lorsque Salomon eut achevé de faire cette prière et cette supplication à Yahweh, il se leva de devant l'autel de Yahweh où il était agenouillé et les mains étendues vers les cieux.
\VS{55}Il se tint debout, et bénit toute l'assemblée d'Israël à haute voix, en disant :
\VS{56}Béni soit Yahweh, qui a donné du repos à son peuple d'Israël, comme il l’avait annoncé ! De toutes les paroles qu'il avait prononcées par le moyen de Moïse, son serviteur, aucune n’est restée sans effet.
\VS{57}Que Yahweh, notre Dieu, soit avec nous, comme il a été avec nos pères ; qu'il ne nous abandonne point et qu'il ne nous délaisse point,
\VS{58}mais qu'il incline nos cœurs vers lui, afin que nous marchions dans toutes ses voies, et que nous observions ses commandements, ses statuts et ses ordonnances, qu'il a prescrits à nos pères !
\VS{59}Que ces paroles, par lesquelles j'ai fait supplication à Yahweh, soient présentes devant Yahweh, notre Dieu, jour et nuit ; afin qu'il fasse justice à son serviteur et à son peuple d’Israël en tout temps,
\VS{60}afin que tous les peuples de la terre reconnaissent que c'est Yahweh qui est Dieu et qu'il n'y en a point d'autre !
\VS{61}Que votre cœur soit intègre envers Yahweh, notre Dieu, comme aujourd'hui, pour marcher dans ses statuts et pour garder ses commandements.
\VS{62}Le roi et tout Israël avec lui offrirent des sacrifices devant Yahweh.
\VS{63}Salomon offrit un sacrifice d'offrande de paix à Yahweh, savoir vingt-deux mille bœufs et cent vingt mille brebis. Ainsi le roi et tous les enfants d'Israël firent la dédicace de la maison de Yahweh.
\VS{64}En ce jour-là, le roi consacra le milieu du parvis, qui est devant la maison de Yahweh ; car il offrit là les holocaustes, les offrandes et les graisses des sacrifices d’offrandes de paix, parce que l'autel d'airain qui est devant Yahweh, était trop petit pour contenir les holocaustes, les offrandes et les graisses des offrandes de paix.
\VS{65}Et en ce temps-là, Salomon célébra une fête solennelle ; et tout Israël avec lui, venu en grande multitude depuis les environs de Hamath jusqu'au torrent d'Egypte, devant Yahweh, notre Dieu, pendant sept jours, et sept autres jours, soit quatorze jours.
\VS{66}Le huitième jour, il renvoya le peuple. Et ils bénirent le roi, et s'en allèrent dans leurs demeures, en se réjouissant, et le cœur heureux pour tout le bien que Yahweh avait fait à David, son serviteur, et à Israël, son peuple.
\Chap{9}
\TextTitle{Yahweh apparaît à Salomon une seconde fois\FTNTT{2 Ch. 7:11-22}}
\VerseOne{}Lorsque Salomon eut achevé de bâtir la maison de Yahweh, la maison royale, et tout ce que Salomon prit plaisir à faire,
\VS{2}Yahweh apparut à Salomon une seconde fois, comme il lui était apparu à Gabaon.
\VS{3}Et Yahweh lui dit : J'exauce ta prière, et la supplication que tu as faite devant moi, j'ai sanctifié cette maison que tu as bâtie pour y mettre mon Nom à jamais, et mes yeux et mon cœur seront toujours là.
\VS{4}Quant à toi, si tu marches devant moi comme David, ton père, a marché, avec intégrité et de cœur et avec droiture, en faisant tout ce que je t'ai commandé, et si tu gardes mes statuts et mes ordonnances,
\VS{5}j’affermirai le trône de ton royaume sur Israël à jamais, comme je l’ai déclaré à David, ton père, en disant : Tu ne manqueras jamais d’un successeur sur le trône d'Israël.
\VS{6}Mais si vous et vos fils, vous vous détournez de moi et que vous ne gardiez pas mes commandements, mes lois que je vous ai prescrites, et si vous allez servir d'autres dieux et vous prosterner devant eux,
\VS{7}je retrancherai Israël de la terre que je lui ai donnée, je rejetterai loin de moi cette maison que j'ai consacrée à mon Nom et Israël sera un sujet de sarcasme et de moquerie parmi tous les peuples.
\VS{8}Et si haut placée qu’ait été cette maison, quiconque passera auprès d'elle sera étonné et sifflera. Et on dira : Pourquoi Yahweh a-t-il ainsi traité ce pays et cette maison ?
\VS{9}Et on répondra : Parce qu'ils ont abandonné Yahweh, leur Dieu, qui avait tiré leurs pères hors du pays d'Egypte, qu'ils se sont attachés à d'autres dieux, se sont prosternés devant eux et les ont servis, voilà pourquoi Yahweh a fait venir sur eux tous ces maux.
\TextTitle{Les réalisations de Salomon\FTNTT{2 Ch. 8:1-18}}
\VS{10}Au bout de vingt ans, Salomon avait bâti les deux maisons, la maison de Yahweh et la maison royale.
\VS{11}Hiram, roi de Tyr, avait fourni à Salomon du bois de cèdre, du bois de sapin et de l'or, autant qu'il en avait voulu, le roi Salomon donna à Hiram vingt villes dans le pays de Galilée.
\VS{12}Hiram sortit de Tyr, pour voir les villes que Salomon lui avait données. Mais elles ne lui plurent point,
\VS{13}et il dit : Quelles villes m'as-tu assignées, mon frère ? Et il les appela, pays de Cabul, nom qu’elles ont conservé jusqu'à ce jour.
\VS{14}Hiram avait aussi envoyé au roi cent vingt talents d'or.
\VS{15}Voici ce qui concerne les hommes de corvée que le roi Salomon leva pour bâtir la maison de Yahweh, sa maison, Millo, la muraille de Jérusalem, Hatsor, Meguiddo et Guézer.
\VS{16}Pharaon, roi d'Egypte, était venu s’emparer de Guézer et l'avait incendiée, il avait tué les Cananéens qui habitaient dans la ville. Puis il la donna pour dot à sa fille, femme de Salomon.
\VS{17}Salomon donc bâtit Guézer, et Beth-Horon la basse,
\VS{18}Baalath et Thadmor, dans le désert qui est au pays,
\VS{19}toutes les villes servant de magasins et lui appartenant, les villes pour les chars et les villes pour la cavalerie, et tout ce qu’il plut à Salomon de bâtir à Jérusalem, au Liban, et dans tout le pays dont il était le souverain.
\VS{20}Tout le peuple qui était resté des Amoréens, des Héthiens, des Phéréziens, des Héviens et des Jébusiens ne faisaient point partie des fils d'Israël,
\VS{21}leurs descendants qui étaient demeurés après eux dans le pays et que les fils d'Israël n'avaient pu dévouer par le moyen de l'interdit, Salomon les fit placer à son service comme gens de corvée à toujours.
\VS{22}Mais Salomon n’employa aucun des fils d'Israël comme esclaves ; car ils étaient ses hommes de guerre, ses serviteurs, ses chefs, ses officiers, les chefs de ses chars et ses hommes d'armes.
\VS{23}Les chefs préposés aux travaux par Salomon étaient au nombre de cinq cent cinquante, lesquels géraient l'intendance des ouvriers.
\VS{24}La fille de Pharaon monta de la cité de David dans la maison que Salomon lui avait bâtie. Ce fut alors qu’il bâtit Millo.
\VS{25}Trois fois par an, Salomon offrait des holocaustes et des offrandes de paix sur l'autel qu'il avait bâti à Yahweh, et il brûlait des parfums sur celui qui était devant Yahweh. Et il acheva la maison.
\VS{26}Le roi Salomon construisit des navires à Etsjon-Guéber, près d'Eloth, sur le rivage de la Mer Rouge, au pays d'Edom.
\VS{27}Et Hiram envoya sur ces navires, auprès des serviteurs de Salomon, ses propres serviteurs, des hommes connaissant la mer.
\VS{28}Ils allèrent en Ophir, et ils prirent de là quatre cent vingt talents d'or qu’ils apportèrent au roi Salomon.
\Chap{10}
\TextTitle{La reine de Séba chez Salomon\FTNTT{2 Ch. 7:1-12}}
\VerseOne{}Or, la reine de Séba ayant appris la renommée de Salomon, à cause du Nom de Yahweh, vint l’éprouver par des énigmes.
\VS{2}Elle entra dans Jérusalem avec une suite fort nombreuse, et avec des chameaux qui portaient des aromates, une grande quantité d'or, et des pierres précieuses. Elle se rendit auprès de Salomon, et lui parla de tout ce qu'elle avait dans le cœur.
\VS{3}Salomon répondit à toutes ses questions, et il n’y eut aucune parole à laquelle le roi ne put fournir une explication.
\VS{4}La reine de Séba vit toute la sagesse de Salomon et la maison qu'il avait bâtie,
\VS{5}les mets de sa table, la demeure de ses serviteurs, l’ordre de service, leurs vêtements, ses échansons, et les holocaustes qu'il offrait dans la maison de Yahweh.
\VS{6}Elle fut toute ravie en elle-même, elle parla ainsi au roi : Ce que j'ai entendu dire dans mon pays au sujet de ta sagesse était donc vrai !
\VS{7}Je ne croyais pas ce qu’on en disait avant d’être venue et que mes yeux ne l'aient vu. Et voici, on ne m'en avait point rapporté la moitié. Ta sagesse et ta prospérité surpassent tout ce que j'en avais entendu.
\VS{8}Heureux sont tes gens ! Heureux tes serviteurs qui se tiennent continuellement devant toi, et qui entendent ta sagesse !
\VS{9}Béni soit Yahweh, ton Dieu, qui t’a accordé la faveur de t’établir sur le trône d'Israël ! Car Yahweh a aimé Israël à toujours ; et t'a établi roi pour faire droit et justice.
\VS{10}Puis elle donna au roi cent vingt talents d'or, une très grande quantité d’aromates et des pierres précieuses. Il ne vint jamais depuis une aussi grande abondance d’aromates que la reine de Séba en donna au roi Salomon.
\VS{11}Et les navires de Hiram, qui amenèrent de l'or d'Ophir, amenèrent aussi d’Ophir une grande quantité de bois de santal et de pierres précieuses.
\VS{12}Le roi fit des supports de ce bois de santal pour la maison de Yahweh et pour la maison royale ; il en fit aussi des harpes et des luths pour les chantres ; il ne vint plus de ce bois de santal et on n’en a plus vu jusqu'à ce jour-là.
\VS{13}Le roi Salomon donna à la reine de Séba tout ce qu'elle désira et répondit à tout et ce qu'elle lui demanda. Il lui fit en outre des présents dignes d'un roi tel que Salomon. Puis elle s'en retourna et alla dans son pays, elle et ses serviteurs.
\TextTitle{Les richesses de Salomon\FTNTT{2 Ch. 9:13-28}}
\VS{14}Le poids de l'or qui revenait à Salomon chaque année, était de six cent soixante-six talents d'or,
\VS{15}outre ce qui lui revenait des négociants, du trafic des marchands, de tous les rois d'Arabie, et des gouverneurs de ce pays-là.
\VS{16}Le roi Salomon fit aussi deux cents grands boucliers d'or battu au marteau, employant six cents sicles d'or pour chaque bouclier,
\VS{17}et trois cents autres boucliers d'or battu au marteau, pour chacun desquels il employa trois mines d'or ; et le roi les mit dans la maison de la forêt du Liban.
\VS{18}Le roi fit aussi un grand trône d'ivoire, qu'il couvrit d’or pur.
\VS{19}Ce trône avait six degrés, et la partie supérieure, le haut du trône était arrondi par derrière. Il y avait des accoudoirs de chaque côté du siège et deux lions se tenaient auprès des accoudoirs.
\VS{20}Il y avait aussi douze lions sur les six degrés du trône, de part et d'autre. Il ne s'est rien fait de tel dans aucun royaume.
\VS{21}Toute la vaisselle du buffet du roi Salomon était d'or, et toutes les coupes de la maison de la forêt du Liban étaient d’or pur. Il n'y en avait point en argent ; on n’en faisait aucun cas du temps de Salomon.
\VS{22}Car le roi avait en mer des navires de Tarsis avec la flotte d'Hiram ; et tous les trois ans la flotte de Tarsis revenait, apportant de l'or, de l'argent, de l'ivoire, des singes et des paons.
\VS{23}Le roi Salomon fut plus grand que tous les rois de la terre, tant en richesses qu'en sagesse.
\VS{24}Tous les habitants de la terre cherchaient à voir la face de Salomon, pour écouter la sagesse que Dieu avait mise en son cœur.
\VS{25}Et chacun d'eux lui apportait son présent, des vases d’or et d'argent, des vêtements, des armes, des aromates, des chevaux et des mulets, tous les ans.
\VS{26}Salomon rassembla ses chars et sa cavalerie ; il y avait mille quatre cents chars et douze mille chevaliers, qu'il plaça dans les villes où il tenait ses chars et à Jérusalem près du roi.
\VS{27}Le roi rendit l'argent aussi commun à Jérusalem que les pierres ; et les cèdres que les sycomores qui croissent dans les plaines, tant il y en avait.
\VS{28}C’est d’Egypte que provenaient les chevaux de Salomon ; une caravane de marchands du roi allait les chercher par troupes, à un prix fixe :
\VS{29}Un char montait et sortait d'Egypte pour six cents sicles d'argent et chaque cheval pour cent cinquante sicles ; ils en amenaient de même avec eux pour tous les rois des Héthiens et pour les rois de Syrie.
\Chap{11}
\TextTitle{Salomon détourne son cœur de Yahweh}
\VerseOne{}Le roi Salomon aima plusieurs femmes étrangères, outre la fille de Pharaon ; savoir des Moabites, des Ammonites, des Edomites, des Sidoniennes et des Héthiennes.
\VS{2}Elles étaient d'entre les nations dont Yahweh avait dit aux enfants d'Israël : Vous n'irez point vers elles, et elles ne viendront point vers vous ; car certainement elles feraient détourner vos cœurs pour suivre leurs dieux. Salomon s'attacha à elles et les aima.
\VS{3}Il eut donc pour femmes sept cents princesses et trois cents concubines ; et ses femmes détournèrent son cœur.
\VS{4}Au temps de la vieillesse de Salomon, ses femmes firent détourner son cœur vers d'autres dieux ; et son cœur ne fut point intègre devant Yahweh, son Dieu, comme David, son père.
\VS{5}Salomon alla après Astarté, la divinité des Sidoniens, et après Milcom, l'abomination des Ammonites.
\VS{6}Ainsi Salomon fit ce qui est mal aux yeux de Yahweh, et il ne persévéra point à suivre Yahweh, comme David, son père.
\VS{7}Et Salomon bâtit un haut lieu à Kemosch, l'abomination des Moabites, sur la montagne qui est vis-à-vis de Jérusalem ; et à Moloc, l'abomination des fils d’Ammon.
\VS{8}Il en fit de même pour toutes ses femmes étrangères, qui offraient des parfums et des sacrifices à leurs dieux.
\VS{9}C'est pourquoi Yahweh fut irrité contre Salomon, parce qu'il avait détourné son cœur de Yahweh, le Dieu d'Israël, qui lui était apparu deux fois.
\VS{10}Il lui avait donné cet ordre de ne point aller après d'autres dieux ; mais il ne garda point ce que Yahweh lui avait ordonné.
\VS{11}Et Yahweh dit à Salomon : Puisque tu as agi de la sorte, et que tu n'as pas observé l’alliance et les ordonnances que je t'avais prescrites, je déchirerai le royaume afin qu'il ne soit plus à toi et je le donnerai à ton serviteur.
\VS{12}Toutefois je ne le ferai point en ton temps, pour l’amour de David, ton père. Ce sera d'entre les mains de ton fils que je déchirerai le royaume.
\VS{13}Néanmoins je ne déchirerai pas tout le royaume, j'en donnerai une tribu à ton fils, pour l'amour de David, mon serviteur, et pour l'amour de Jérusalem, que j'ai choisie.
\TextTitle{Dieu suscite des ennemis à Salomon}
\VS{14}Yahweh donc suscita un ennemi à Salomon, savoir Hadad, l’Edomite, qui était de la race royale d'Edom.
\VS{15}Car il était arrivé qu'au temps que David était en Edom, Joab, chef de l'armée, étant monté pour ensevelir les morts, tua tous les mâles qui étaient en Edom ;
\VS{16}Joab demeura là six mois avec tout Israël, jusqu'à ce qu'il eût exterminé tous les mâles d'Edom.
\VS{17}Ce fut alors qu’Hadad prit la fuite avec des Edomites d'entre les serviteurs de son père, pour se retirer en Egypte. Hadad était alors un jeune garçon.
\VS{18}Une fois partis de Madian, ils allèrent à Paran, prirent avec eux des hommes de Paran, et arrivèrent en Egypte auprès de Pharaon, roi d'Egypte, qui lui donna une maison, pourvut à sa subsistance et lui donna aussi une terre.
\VS{19}Et Hadad trouva grâce aux yeux de Pharaon, de sorte que Pharaon lui donna pour femme la sœur de sa propre femme, la sœur de la reine Thachpenès.
\VS{20}Et la sœur de Thachpenès lui enfanta son fils Guenubath. Thachpenès le sevra dans la maison de Pharaon. Ainsi Guenubath fut dans la maison de Pharaon, parmi les fils de Pharaon.
\VS{21}Lorsque Hadad apprit en Egypte que David s'était endormi avec ses pères, et que Joab, chef de l'armée, était mort, il dit à Pharaon : Laisse-moi partir dans mon pays.
\VS{22}Et Pharaon lui répondit : Que te manque-t-il auprès de moi, pour désirer ainsi t'en aller dans ton pays ? Et il répondit : Je n’ai besoin de rien, mais cependant laisse-moi partir.
\VS{23}Dieu suscita aussi un autre ennemi à Salomon, savoir Rezon, fils d'Eliada, qui s'était enfui de chez son maître Hadadézer, roi de Tsoba,
\VS{24}Il avait rassemblé des gens auprès de lui, et était devenu chef de bandes, lorsque David les fit périr ; et ils s'en allèrent à Damas, s’y établirent et y régnèrent.
\VS{25}Rezon fut ennemi d'Israël au temps de Salomon, en même temps qu’Hadad le mettait à mal, il avait en aversion Israël et il régna sur la Syrie.
\VS{26}Jéroboam aussi, serviteur de Salomon, s'éleva également contre le roi. Il était fils de Nebath, Ephratien, de Tseréda, dont la mère s’appelait Tserua, femme veuve.
\VS{27}Voici à quelle occasion il s'éleva contre le roi. Salomon bâtissait Millo, et fermait la brèche de la cité de David, son père.
\VS{28}Jéroboam était un homme fort et vaillant ; et Salomon, voyant ce jeune homme à l’ouvrage, lui assigna la charge de toute la maison de Joseph.
\VS{29}Dans ce même temps, Jéroboam, étant sorti de Jérusalem, rencontra en chemin le prophète Achija de Silo, revêtu d'un manteau neuf, et ils étaient eux deux tout seuls dans les champs.
\VS{30}Et Achija prit le manteau neuf qu'il avait sur lui et le déchira en douze morceaux,
\VS{31}et il dit à Jéroboam : Prends-en pour toi dix morceaux ! Car ainsi parle Yahweh, le Dieu d'Israël : Voici, je vais arracher le royaume d'entre les mains de Salomon, et je t'en donnerai dix tribus.
\VS{32}Mais il aura une tribu, pour l'amour de David, mon serviteur, et pour l'amour de Jérusalem, qui est la ville que j'ai choisie d'entre toutes les tribus d'Israël.
\VS{33}Parce qu'ils m'ont abandonné, et se sont prosternés devant Astarté, la déesse des Sidoniens, devant Kemosch, dieu de Moab, et devant Milcom, le dieu des fils d’Ammon, et qu'ils n'ont point marché dans mes voies, pour faire ce qui est droit à mes yeux et garder mes statuts, et mes ordonnances, comme l’a fait David, père de Salomon.
\VS{34}Toutefois, je n'ôterai pas de sa main tout le royaume, car pendant toute sa vie je le maintiendrai prince, pour l'amour de David, mon serviteur, que j'ai choisi et qui a observé mes commandements et mes lois.
\VS{35}Mais j'ôterai le royaume d'entre les mains de son fils, je t'en donnerai dix tribus ;
\VS{36}j'en donnerai une tribu à son fils, afin que David, mon serviteur, ait une lampe à toujours devant moi dans Jérusalem, qui est la ville que j'ai choisie pour y mettre mon Nom.
\VS{37}Je te prendrai donc, tu régneras sur tout ce que ton âme désirera, tu seras roi sur Israël.
\VS{38}Et il arrivera que si tu m'obéis en tout ce que je te commanderai, que tu marches dans mes voies, en faisant tout ce qui est droit à mes yeux, en gardant mes statuts et mes commandements, comme l’a fait David, mon serviteur, je serai avec toi, je te bâtirai une maison qui sera stable, comme j'en ai bâti une à David, et je te donnerai Israël.
\VS{39}Ainsi j’humilierai la postérité de David à cause de cela, mais non pas à toujours.
\VS{40}Salomon chercha à faire mourir Jéroboam, mais Jéroboam se leva et s'enfuit en Egypte vers Schischak, roi d'Egypte ; et il demeura en Egypte jusqu'à la mort de Salomon.
\TextTitle{Mort de Salomon\FTNTT{2 Ch. 9:29-31}}
\VS{41}Or, le reste des faits de Salomon, tout ce qu'il a fait et sa sagesse, cela n'est-il pas écrit dans le livre des actes de Salomon ?
\VS{42}Salomon régna à Jérusalem sur tout Israël pendant quarante ans.
\VS{43}Ainsi Salomon s'endormit avec ses pères, il fut enseveli dans la cité de David, son père. Et Roboam, son fils, régna en sa place.
\Chap{12}
\TextTitle{Règne de Roboam\FTNTT{2 Ch. 10:1 ; cp. Ec. 2:18-19}}
\VerseOne{}Roboam se rendit à Sichem, parce que tout Israël était venu à Sichem pour l'établir roi.
\VS{2}Or, Jéroboam, fils de Nebath, était encore en Egypte, où il s'était enfui de devant le roi Salomon, quand il l'apprit, et c’était en Egypte qu’il habitait.
\VS{3}On l'envoya appeler. Ainsi Jéroboam et toute l'assemblée d'Israël vinrent, ils parlèrent à Roboam, en disant :
\VS{4}Ton père a mis sur nous un pesant joug ; mais toi allège maintenant cette rude servitude de ton père et ce pesant joug qu'il a mis sur nous ; et nous te servirons.
\VS{5}Il leur répondit : Allez, et dans trois jours revenez vers moi. Et le peuple s'en alla.
\VS{6}Le roi Roboam consulta les vieillards qui avaient été auprès de Salomon, son père, pendant sa vie et leur dit : Que me conseillez-vous de répondre à ce peuple ?
\VS{7}Et ils lui répondirent, en disant : Si aujourd'hui tu rends service à ce peuple et que tu leur cèdes, et si tu leur réponds avec des paroles bienveillantes, ils seront tes serviteurs à toujours.
\VS{8}Mais Roboam laissa le conseil que les vieillards lui avaient donné et consulta les jeunes gens qui avaient grandi avec lui et qui se tenaient près de lui.
\VS{9}Il leur dit : Que me conseillez-vous de répondre à ce peuple qui m'a parlé, en disant : Allège le joug que ton père a mis sur nous ?
\VS{10}Alors les jeunes gens qui avaient grandi avec lui, lui dirent : Tu parleras ainsi à ce peuple qui t'est venu dire : Ton père a mis sur nous un pesant joug, mais toi allège-le-nous ! Tu leur parleras ainsi : Mon petit doigt est plus gros que les reins de mon père.
\VS{11}Or, mon père a mis sur vous un pesant joug, mais moi je rendrai votre joug encore plus pesant ; mon père vous a châtiés avec des fouets, mais moi je vous châtierai avec des scorpions.
\VS{12}Or, trois jours après, Jéroboam avec tout le peuple vint vers Roboam, selon que le roi leur avait dit : Retournez vers moi dans trois jours.
\VS{13}Mais le roi répondit durement au peuple, laissant le conseil que les anciens lui avaient donné.
\VS{14}Il leur parla selon le conseil des jeunes gens, en leur disant : Mon père a mis sur vous un pesant joug, mais moi, je rendrai votre joug plus pesant encore ; mon père vous a châtiés avec des fouets, mais moi, je vous châtierai avec des scorpions.
\VS{15}Le roi donc n'écouta point le peuple ; car cela était ainsi conduit par Yahweh, en vue d’accomplir la parole qu'il avait prononcée par le ministère d'Achija de Silo, à Jéroboam, fils de Nebath.
\TextTitle{Schisme du royaume ; Jéroboam devient roi d’Israël\FTNTT{2 Ch. 10:12-19 ; 11:1-4}}
\VS{16}Et quand tout Israël vit que le roi ne les avait point écoutés, le peuple fit cette réponse au roi, en disant : Quelle part avons-nous avec David ? Nous n'avons point de propriété avec le fils d'Isaï ! A tes tentes, Israël ! Et toi David, pourvois maintenant à ta maison ! Ainsi Israël s'en alla dans ses tentes.
\VS{17}Les fils d'Israël qui habitaient dans les villes de Juda furent les seuls sur qui Roboam régna.
\VS{18}Or, le roi Roboam envoya Adoram, qui était préposé aux impôts, mais tout Israël le lapida, et il mourut. Alors le roi Roboam se hâta de monter sur un char pour s'enfuir à Jérusalem.
\VS{19}C’est ainsi qu’Israël s’est détaché de la maison de David jusqu'à ce jour.
\VS{20}Tout Israël apprit que Jéroboam était de retour, ils l'envoyèrent appeler dans l'assemblée, et l'établirent roi sur tout Israël. La tribu de Juda fut la seule qui suivit la maison de David.
\VS{21}Roboam arriva à Jérusalem, il rassembla toute la maison de Juda et la tribu de Benjamin, savoir cent quatre-vingt mille hommes d’élite choisis et disposés à faire la guerre, pour combattre contre la maison d'Israël, et ramener la domination à Roboam, fils de Salomon.
\VS{22}Mais la parole de Dieu fut ainsi adressée à Schemaeja, homme de Dieu, disant :
\VS{23}Parle à Roboam, fils de Salomon, roi de Juda, et à toute la maison de Juda, et de Benjamin, et au reste du peuple, en disant :
\VS{24}Ainsi parle Yahweh : Vous ne monterez point et vous ne combattrez point contre vos frères, les fils d'Israël ! Que chacun de vous retourne dans sa maison, car ceci a été fait de par moi. Ils obéirent à la parole de Yahweh, et s'en retournèrent, selon la parole de Yahweh.
\TextTitle{Idolâtrie de Jéroboam}
\VS{25}Or, Jéroboam bâtit Sichem sur la montagne d'Ephraïm, et y demeura, puis il en sortit et bâtit Penuel.
\VS{26}Et Jéroboam dit en son cœur : Maintenant le royaume pourrait bien retourner à la maison de David.
\VS{27}Si ce peuple monte à Jérusalem pour faire des sacrifices dans la maison de Yahweh, le cœur de ce peuple se tournera vers son seigneur, Roboam, roi de Juda, et ils me tueront, et ils retourneront à Roboam, roi de Juda.
\VS{28}Sur quoi le roi ayant pris conseil, fit deux veaux d'or et dit au peuple : Vous êtes longtemps montés à Jérusalem ! Voici ton dieu, ô Israël, qui t'a fait sortir hors du pays d'Egypte.
\VS{29}Il plaça un de ces veaux à Béthel, et il mit l'autre à Dan.
\VS{30}Et cela fut une occasion de péché, car le peuple allait jusqu'à Dan, pour se prosterner devant l'un des veaux.
\VS{31}Il fit aussi des maisons dans les hauts lieux, et établit des sacrificateurs pris parmi tout le peuple, qui n'étaient point des enfants de Lévi.
\VS{32}Jéroboam ordonna aussi une fête solennelle au huitième mois, le quinzième jour du mois, à l'imitation de la fête solennelle qu'on célébrait en Juda, et il offrait des sacrifices sur un autel. Il fit ainsi à Béthel, sacrifiant aux veaux qu'il avait faits, et il établit à Béthel des sacrificateurs des hauts lieux qu'il avait élevés.
\VS{33}Or, le quinzième jour du huitième mois, savoir au mois qu'il avait choisi lui-même, il monta sur l'autel qu'il avait fait à Béthel, et célébra cette fête solennelle pour les enfants d'Israël ; et fit brûler des parfums sur l'autel.
\Chap{13}
\TextTitle{Un homme de Dieu envoyé vers Jéroboam}
\VerseOne{}Et voici, un homme de Dieu vint de Juda à Béthel avec la parole de Yahweh, pendant que Jéroboam se tenait près de l'autel pour brûler des parfums.
\VS{2}Et il cria contre l'autel selon la parole de Yahweh, et dit : Autel ! Autel ! Ainsi parle Yahweh : Voici, un fils naîtra à la maison de David, qui aura pour nom Josias ; il immolera sur toi les sacrificateurs des hauts lieux qui brûlent des parfums sur toi, et on brûlera sur toi des ossements d’hommes !
\VS{3}Le même jour il donna un signe, en disant : C'est ici le signe dont Yahweh a parlé : Voici, l'autel se fendra, et la cendre qui est dessus sera répandue.
\VS{4}Lorsque le roi entendit la parole que l'homme de Dieu avait criée contre l'autel de Béthel, Jéroboam étendit sa main de l'autel, en disant : Saisissez-le ! Et la main qu'il étendit contre lui devint sèche, et il ne put la ramener à lui.
\VS{5}L'autel aussi se fendit, et la cendre qui était sur l'autel fut répandue, selon le signe que l'homme de Dieu avait donné par la parole de Yahweh.
\VS{6}Alors le roi prit la parole et dit à l'homme de Dieu : Implore Yahweh, ton Dieu, et prie pour moi, afin que ma main revienne à moi. L'homme de Dieu implora Yahweh, et la main du roi put revenir à lui et elle fut comme auparavant.
\VS{7}Alors le roi dit à l'homme de Dieu : Entre avec moi dans la maison, tu prendras quelque nourriture et je te donnerai un présent.
\VS{8}Mais l'homme de Dieu répondit au roi : Quand tu me donnerais la moitié de ta maison, je n'entrerais point chez toi, je ne mangerais point de pain, ni ne boirais d'eau en ce lieu.
\VS{9}Car cela m'a été ordonné par Yahweh, qui m'a dit : Tu ne mangeras point de pain, tu ne boiras point d'eau et tu ne t'en retourneras point par le chemin par lequel tu y seras allé.
\VS{10}Il s'en alla donc par un autre chemin, et ne s'en retourna point par le chemin par lequel il était venu à Béthel.
\TextTitle{L’homme de Dieu séduit par un vieux prophète}
\VS{11}Or, il y avait un vieux prophète qui demeurait à Béthel. Ses fils vinrent raconter toutes les choses que l'homme de Dieu avait faites ce jour-là à Béthel, et les paroles qu'il avait dites au roi ; et comme les fils de ce prophète les rapportaient à leur père,
\VS{12}il leur demanda : Par quel chemin s'en est-il allé ? Or, ses fils avaient vu le chemin par lequel l'homme de Dieu qui était venu de Juda s'en était allé.
\VS{13}Et il dit à ses fils : Sellez-moi un âne. Ils lui sellèrent, puis il monta dessus.
\VS{14}Et il s'en alla après l'homme de Dieu, et le trouva assis sous un chêne. Et il lui dit : Es-tu l'homme de Dieu qui est venu de Juda ? Et il lui répondit : C'est moi.
\VS{15}Alors il lui dit : Viens avec moi dans la maison, et tu prendras de quoi te nourrir.
\VS{16}Mais il répondit : Je ne puis retourner avec toi, ni entrer chez toi et je ne mangerai point de pain, ni ne boirai d'eau avec toi en ce lieu ;
\VS{17}Car il m'a été dit de la part de Yahweh : Tu ne mangeras point de pain, tu ne boiras point d'eau, et tu ne t'en retourneras point par le chemin par lequel tu seras allé.
\VS{18}Et il lui dit : Et moi aussi je suis prophète comme toi ; et un ange m'a parlé de la part de Yahweh, en disant : Ramène-le avec toi dans ta maison, qu'il mange du pain, et qu'il boive de l'eau ; mais il lui mentait.
\VS{19}Il s'en retourna donc avec lui, il mangea du pain et but de l'eau dans sa maison.
\VS{20}Et il arriva que comme ils étaient assis à table, la parole de Yahweh fut adressée au prophète qui l'avait ramené.
\VS{21}Et il cria à l'homme de Dieu qui était venu de Juda, en disant : Ainsi a parlé Yahweh : Parce que tu as été rebelle au commandement de Yahweh et que tu n'as point gardé l’ordre que Yahweh, ton Dieu, t'avait donné ;
\VS{22}mais tu t'en es retourné, tu as mangé du pain et bu de l'eau dans le lieu dont Yahweh t'avait dit : N'y mange point de pain et n'y bois point d'eau, ton cadavre n'entrera point au sépulcre de tes pères.
\VS{23}Et quand le prophète qu'il avait ramené eut mangé du pain et bu de l’eau, il sella l’âne pour lui.
\VS{24}L’homme de Dieu s'en alla, et un lion le rencontra dans le chemin, et le tua. Son corps était étendu dans le chemin, l'âne resta auprès du corps, et le lion aussi resta à côté du cadavre.
\VS{25}Et voici des passants virent le corps étendu dans le chemin et le lion qui se tenait auprès du corps ; et ils vinrent le dire dans la ville où le vieux prophète demeurait.
\VS{26}Et le prophète qui avait ramené du chemin l'homme de Dieu, l'ayant appris, dit : C'est l'homme de Dieu qui a été rebelle au commandement de Yahweh, c'est pourquoi Yahweh l'a livré au lion, qui l'aura déchiré après l'avoir tué, selon la parole que Yahweh avait dite à ce prophète.
\VS{27}Et il parla à ses fils, en disant : Sellez-moi un âne. Ils le lui sellèrent,
\VS{28}Et il s'en alla et trouva le corps de l'homme de Dieu étendu dans le chemin, l'âne et le lion qui se tenaient auprès du corps. Le lion n'avait pas dévoré le cadavre, ni déchiré l'âne.
\VS{29}Alors le prophète leva le corps de l'homme de Dieu, le plaça sur l'âne et le ramena ; et ce vieux prophète revint dans la ville pour le pleurer et l'enterrer.
\VS{30}Il mit le corps de ce prophète dans le sépulcre, et il pleura sur lui, en disant : Hélas, mon frère !
\VS{31}Après l’avoir enterré, il parla à ses fils, en disant : Quand je serai mort, enterrez-moi au sépulcre où est enterré l'homme de Dieu, et vous déposerez mes os à côté de ses os.
\VS{32}Car elle s’accomplira, la parole qu’il a criée de la part de Yahweh, contre l'autel qui est à Béthel et contre toutes les maisons des hauts lieux qui sont dans les villes de Samarie.
\TextTitle{Jéroboam continue dans le mal}
\VS{33}Néanmoins, Jéroboam ne se détourna point de sa mauvaise voie, mais il établit de nouveau des sacrificateurs de hauts lieux pris parmi tout le peuple ; quiconque le voulait, Jéroboam le consacrait sacrificateur des hauts lieux.
\VS{34}Cela fut une occasion de péché pour la maison de Jéroboam, qui fut effacée et exterminée de dessus la terre.
\Chap{14}
\TextTitle{Maladie et mort du fils de Jéroboam}
\VerseOne{}En ce temps-là, Abija, fils de Jéroboam, devint malade.
\VS{2}Et Jéroboam dit à sa femme : Lève-toi maintenant et déguise-toi, en sorte qu'on ne reconnaisse point que tu es la femme de Jéroboam, et va à Silo. Voici, là est Achija, le prophète, qui m'a dit que je serais roi sur ce peuple.
\VS{3}Emmène avec toi dix pains, des gâteaux et un vase de miel, et entre chez lui ; il te dira ce qui arrivera à l’enfant.
\VS{4}La femme de Jéroboam fit donc ainsi ; elle se leva et s'en alla à Silo puis elle entra dans la maison d'Achija. Or, Achija ne pouvait plus voir, parce qu’il avait les yeux figés à cause de sa vieillesse.
\VS{5}Et Yahweh dit à Achija : Voici, la femme de Jéroboam, qui vient te consulter concernant l’état de son fils, parce qu'il est malade. Tu lui parleras de telle et de telle manière. Quand elle arrivera, elle se sera déguisée.
\VS{6}Lorsque Achija eut entendu le bruit de ses pas, comme elle franchissait la porte, il dit : Entre, femme de Jéroboam. Pourquoi fais-tu semblant d'être quelqu’un d’autre ? Je suis chargé de t’annoncer des choses dures.
\VS{7}Va, dis à Jéroboam : Ainsi parle Yahweh, le Dieu d'Israël : Parce que je t'ai élevé du milieu du peuple et que je t'ai établi pour chef sur mon peuple d'Israël,
\VS{8}j'ai arraché le royaume de la maison de David et je te l'ai donné ; mais parce que tu n'as point été comme David, mon serviteur, qui a gardé mes commandements et qui a marché après moi de tout son cœur, ne faisant que ce qui est droit à mes yeux.
\VS{9}Tu as fait pire que tous ceux qui ont été devant toi, tu es allé te faire d'autres dieux et des images de fonte, pour m'irriter, et tu m'as rejeté derrière ton dos !
\VS{10}A cause de cela, voici, je vais faire venir le malheur sur la maison de Jéroboam ; je retrancherai ce qui appartient à Jéroboam, ce qu’il détient et ce qu’il néglige en Israël, et je brûlerai la maison de Jéroboam, comme on brûle les ordures, jusqu'à ce qu'il n'en reste plus.
\VS{11}Celui de la maison de Jéroboam qui mourra dans la ville, les chiens le mangeront, et celui qui mourra aux champs, les oiseaux du ciel le mangeront. Car Yahweh a parlé.
\VS{12}Toi donc lève-toi, va dans ta maison. Dès que tes pieds entreront dans la ville, l'enfant mourra.
\VS{13}Tout Israël le pleurera et on l’enterrera ; car lui seul de la famille de Jéroboam entrera au sépulcre, parce que Yahweh, le Dieu d'Israël, a trouvé quelque chose de bon en lui seul dans toute la maison de Jéroboam.
\VS{14}Yahweh s'établira un roi sur Israël qui retranchera la maison de Jéroboam. Ce jour-là, n’est-ce pas déjà ce qui arrive ?
\VS{15}Yahweh frappera Israël, l'agitant comme le roseau est agité dans l'eau ; et il arrachera Israël de ce bon pays qu'il a donné à leurs pères, et les dispersera au-delà du fleuve, parce qu'ils se sont fait des idoles, irritant Yahweh.
\VS{16}Il livrera Israël à cause des péchés que Jéroboam a commis et qu’il a fait commettre à Israël.
\VS{17}Alors la femme de Jéroboam se leva et s'en alla, elle vint à Thirtsa : et comme elle franchit le seuil de la maison, le jeune garçon mourut.
\VS{18}Il fut enseveli et tout Israël le pleura, selon la parole de Yahweh, proférée par son serviteur Achija, le prophète.
\TextTitle{Règne de Nadab sur Israël\FTNTT{cp. 2 Ch. 13:20}}
\VS{19}Quant au reste des faits de Jéroboam, comment il a fait la guerre et comment il a régné, cela est écrit dans le livre des Chroniques des rois d'Israël.
\VS{20}Jéroboam régna vingt-deux ans, puis il s'endormit avec ses pères. Et Nadab, son fils, régna à sa place.
\TextTitle{Juda dans l'apostasie\FTNTT{2 Ch. 12:1}}
\VS{21}Roboam, fils de Salomon, régna en Juda. Il avait quarante et un ans quand il devint roi, et il régna dix-sept ans à Jérusalem, la ville que Yahweh avait choisie d'entre toutes les tribus d'Israël pour y mettre son nom. Sa mère s’appelait Naama, l’Ammonite.
\VS{22}Juda fit ce qui est mal aux yeux de Yahweh ; et par les péchés qu'ils commirent, ils excitèrent sa jalousie plus que leurs pères ne l'avaient jamais fait.
\VS{23}Ils se bâtirent, eux aussi, des hauts lieux avec des statues et des idoles sur toute colline élevée, et sous tout arbre verdoyant.
\VS{24}Il y avait dans le pays des prostitués. Et ils firent selon toutes les abominations des nations que Yahweh avait chassées devant les enfants d'Israël.
\TextTitle{Le roi d’Egypte emporte les trésors de Juda ; mort de Roboam\FTNTT{2 Ch. 12:2-16}}
\VS{25}La cinquième année du roi Roboam, Schischak, roi d'Egypte, monta contre Jérusalem.
\VS{26}Il prit les trésors de la maison de Yahweh et les trésors de la maison royale, et il emporta tout. Il prit aussi tous les boucliers d'or que Salomon avait faits.
\VS{27}Le roi Roboam fit des boucliers d'airain au lieu de ceux-là, et les mit entre les mains des chefs des coureurs, qui gardaient l’entrée de la maison du roi.
\VS{28}Toutes les fois où le roi entrait dans la maison de Yahweh, les coureurs les portaient, et ensuite ils les rapportaient dans la chambre des coureurs.
\VS{29}Le reste des actions de Roboam, et tout ce qu'il a fait, n'est-il pas écrit au livre des Chroniques des rois de Juda ?
\VS{30}Il y eut toujours guerre entre Roboam et Jéroboam.
\VS{31}Roboam s'endormit avec ses pères et fut enseveli avec eux dans la cité de David. Sa mère avait pour nom Naama, l’Ammonite. Et Abijam, son fils, régna à sa place.
\Chap{15}
\TextTitle{Règne d'Abijam (ou Abija) sur Juda\FTNTT{2 Ch. 13:1-2}}
\VerseOne{}La dix-huitième année du roi Jéroboam, fils de Nebath, Abijam commença à régner sur Juda.
\VS{2}Il régna trois ans à Jérusalem. Sa mère s’appelait Maaca et était fille d'Abisalom.
\VS{3}Il marcha dans tous les péchés que son père avait commis avant lui ; son cœur ne fut point intègre envers Yahweh, son Dieu, comme l'avait été le cœur de David, son père.
\VS{4}Mais pour l'amour de David, Yahweh, son Dieu, lui donna une lampe dans Jérusalem, lui suscitant son fils après lui et laissant subsister Jérusalem ;
\VS{5}Parce que David avait fait ce qui est droit devant Yahweh, et que pendant toute sa vie il ne s'était point détourné d’aucun de ses commandements, hormis dans l'affaire d'Urie, le Héthien.
\VS{6}Or, il y eut toujours guerre entre Roboam et Jéroboam, pendant toute la vie de Roboam.
\VS{7}Le reste des actions d'Abijam, et même tout ce qu'il fit, n'est-il pas écrit au livre des Chroniques des rois de Juda ? Il y eut aussi guerre entre Abijam et Jéroboam.
\VS{8}Ainsi Abijam s'endormit avec ses pères, et on l'enterra dans la cité de David. Et Asa, son fils, régna à sa place.
\TextTitle{Règne d’Asa sur Juda\FTNTT{2 Ch. 14:1-5 ; 15:1-19}}
\VS{9}La vingtième année de Jéroboam, roi d'Israël, Asa commença à régner sur Juda.
\VS{10}Il régna quarante et un ans à Jérusalem. Sa mère avait pour nom Maaca, elle était fille d'Abisalom.
\VS{11}Asa fit ce qui est droit devant Yahweh, comme David, son père.
\VS{12}Il ôta du pays les prostitués, et ôta toutes les idoles que ses pères avaient faites.
\VS{13}Et même il ôta la dignité de reine à sa mère Maaca, parce qu'elle avait fait une idole pour Astarté. Asa mit en pièces l’idole qu'elle avait faite, et la brûla au torrent de Cédron.
\VS{14}Mais les hauts lieux ne furent point ôtés. Néanmoins, le cœur d'Asa fut intègre envers Yahweh pendant toute sa vie.
\TextTitle{Guerre entre Juda et Israël ; Asa s’allie avec la Syrie\FTNTT{1 Ch. 14:6-15 ; 16:1-10}}
\VS{15}Il remit dans la maison de Yahweh les choses qui avaient été consacrées par son père et par lui-même, de l'argent, de l'or et les ustensiles.
\VS{16}Or, il y eut guerre entre Asa et Baescha, roi d'Israël, pendant toute leur vie.
\VS{17}Baescha, roi d'Israël, monta contre Juda, et bâtit Rama, pour empêcher quiconque de sortir et entrer vers Asa, roi de Juda.
\VS{18}Asa prit tout l'argent et l'or qui était resté dans les trésors de Yahweh et dans les trésors de la maison royale, et les donna à ses serviteurs ; le roi Asa les envoya vers Ben-Hadad, fils de Thabrimmon, fils de Hezjon, roi de Syrie, qui demeurait à Damas, pour lui dire :
\VS{19}Qu’il y ait alliance entre moi et toi, comme entre mon père et le tien. Voici, je t'envoie un présent en argent et en or. Va, romps l'alliance que tu as avec Baescha, roi d'Israël, afin qu'il se retire de moi.
\VS{20}Et Ben-Hadad écouta le roi Asa ; il envoya les chefs de son armée contre les villes d'Israël, et il battit Ijjon, Dan, Abel-Beth-Maaca, tout Kinneroth, et tout le pays de Nephthali.
\VS{21}Lorsque Baescha l’apprit, il cessa de bâtir Rama et demeura à Thirtsa.
\VS{22}Alors le roi Asa fit publier par tout Juda que tous, sans en excepter aucun, eussent à emporter les pierres et le bois de Rama, que Baescha faisait bâtir, et le roi Asa s’en servit pour bâtir Guéba de Benjamin, et Mitspa.
\TextTitle{Mort d’Asa ; Josaphat règne sur Juda\FTNTT{1 Ch. 16:11-17:1}}
\VS{23}Le reste de toutes les actions d'Asa, tous ses exploits, tout ce qu'il fit, et les villes qu'il a bâties, cela n'est-il pas écrit au livre des Chroniques des rois de Juda ? Au reste, il fut malade de ses pieds au temps de sa vieillesse.
\VS{24}Et Asa s'endormit avec ses pères, avec lesquels il fut enseveli en la cité de David, son père. Et, son fils, Josaphat, régna à sa place.
\TextTitle{Baescha tue Nadab et devient roi d'Israël}
\VS{25}Or, Nadab, fils de Jéroboam, régna sur Israël la seconde année d'Asa, roi de Juda, et il régna deux ans sur Israël.
\VS{26}Il fit ce qui est mal aux yeux de Yahweh ; et il marcha dans la voie de son père, se livrant aux péchés que son père avait fait commettre à Israël.
\VS{27}Et Baescha, fils d'Achija, de la maison d'Issacar, fit une conspiration contre lui. Il le tua devant Guibbethon, qui était aux Philistins, lorsque Nadab et tout Israël assiégeaient Guibbethon.
\VS{28}Baescha le fit donc mourir la troisième année d'Asa, roi de Juda, et il régna à sa place.
\VS{29}Une fois proclamé roi, il frappa toute la maison de Jéroboam et ne laissa échapper aucune âme vivante, il détruisit tout ce qui respirait, selon la parole de Yahweh qu'il avait proférée par son serviteur Achija, Silonite,
\VS{30}A cause des péchés que Jéroboam avait commis et fait commettre à Israël, irritant ainsi Yahweh, le Dieu d'Israël.
\VS{31}Le reste des faits de Nadab, et même tout ce qu'il a fait, n'est-il pas écrit au livre des Chroniques des rois d'Israël ?
\VS{32}Or, il y eut guerre entre Asa et Baescha, roi d'Israël, pendant toute leur vie.
\VS{33}La troisième année d'Asa, roi de Juda, Baescha, fils d'Achija, commença à régner sur tout Israël à Thirtsa, il régna vingt-quatre ans.
\VS{34}Et il fit ce qui est mal aux yeux de Yahweh, et marcha dans la voie de Jéroboam, en se livrant aux péchés que Jéroboam avait fait commettre à Israël.
\Chap{16}
\TextTitle{Yahweh avertit Baescha avant sa mort}
\VerseOne{}Alors la parole de Yahweh fut adressée à Jéhu, fils de Hanani, contre Baescha, en ces mots :
\VS{2}Je t'ai élevé de la poussière et je t'ai établi chef de mon peuple d'Israël ; malgré cela tu as suivi la voie de Jéroboam et fait pécher mon peuple d'Israël, pour m'irriter par leurs péchés.
\VS{3}Voici, je m'en vais entièrement consumer Baescha et sa maison, et je rendrai ta maison semblable à la maison de Jéroboam, fils de Nebath.
\VS{4}Celui de la maison de Baescha qui mourra dans la ville, les chiens le mangeront, et celui des siens qui mourra aux champs, les oiseaux du ciel le mangeront.
\VS{5}Le reste des faits de Baescha, ce qu'il a fait et ses exploits, n'est-il pas écrit au livre des Chroniques des rois d'Israël ?
\VS{6}Ainsi Baescha s'endormit avec ses pères et fut enseveli à Thirtsa. Ela, son fils, régna à sa place.
\VS{7}La parole de Yahweh fut aussi adressée par le moyen de Jéhu, fils d'Hanani, le prophète, contre Baescha et contre sa maison, tant à cause de tout le mal qu'il avait fait devant Yahweh, en l'irritant par l'œuvre de ses mains et en devenant comme la maison de Jéroboam, que parce qu'il l'avait détruite.
\TextTitle{Ela puis Zimri règnent sur Israël}
\VS{8}La vingt-sixième année d'Asa, roi de Juda, Ela, fils de Baescha, commença à régner sur Israël et il régna deux ans à Thirtsa.
\VS{9}Son serviteur, Zimri, capitaine de la moitié des chars, fit une conspiration contre Ela, lorsqu'il était à Thirtsa, buvant et s'enivrant dans la maison d'Artsa, chef de la maison du roi à Thirtsa.
\VS{10}Alors, Zimri vint, le frappa et le tua, la vingt-septième année d'Asa, roi de Juda, et il régna à sa place.
\VS{11}Dès qu’il fut roi et qu'il fut assis sur son trône, il frappa toute la maison de Baescha, il n'en laissa échapper personne qui lui appartint, ni parent, ni ami.
\VS{12}Ainsi Zimri extermina toute la maison de Baescha, selon la parole que Yahweh avait proférée contre Baescha, par Jéhu, le prophète,
\VS{13}A cause de tous les péchés de Baescha, et des péchés d'Ela, son fils, qu’ils avaient commis et qu’ils avaient fait commettre à Israël, irritant Yahweh, le Dieu d'Israël, par leurs idoles.
\VS{14}Le reste des faits d'Ela, et même tout ce qu'il a fait, n'est-il pas écrit au livre des Chroniques des rois d'Israël ?
\VS{15}La vingt-septième année d'Asa, roi de Juda, Zimri régna sept jours à Thirtsa. Or, le peuple était campé contre Guibbethon qui appartenait aux Philistins.
\VS{16}Et le peuple qui était campé là entendit que l'on disait : Zimri a fait une conspiration, et il a même tué le roi ! En ce même jour, tout Israël établit dans le camp pour roi d’Israël Omri, chef de l'armée d'Israël.
\VS{17}Omri et tout Israël avec lui partirent de Guibbethon, et assiégèrent Thirtsa.
\VS{18}Mais dès que Zimri vit que la ville était prise, il entra au palais de la maison royale et brûla sur lui la maison royale, il mourut ainsi,
\VS{19}A cause des péchés qu’il avait commis, faisant ce qui est mal aux yeux de Yahweh, en suivant la voie de Jéroboam et le péché qu'il avait fait commettre à Israël.
\VS{20}Le reste des actions de Zimri et la conspiration qu'il forma, cela n’est-il pas écrit dans le livre des Chroniques des rois d'Israël ?
\TextTitle{Omri règne sur Israël}
\VS{21}Alors le peuple d'Israël se divisa en deux partis : la moitié du peuple voulait faire roi Thibni, fils de Guinath ; et l'autre moitié suivait Omri.
\VS{22}Mais le peuple qui suivait Omri, fut plus fort que le peuple qui suivait Thibni, fils de Guinath. Thibni mourut et Omri régna.
\VS{23}La trente et unième année d'Asa, roi de Juda, Omri commença à régner sur Israël et il régna douze ans après avoir régné six ans à Thirtsa.
\VS{24}Puis il acheta de Schémer la montagne de Samarie, deux talents d'argent ; il bâtit une ville sur cette montagne et nomma la ville qu'il bâtit, du nom de Schémer, seigneur de la montagne.
\VS{25}Omri fit ce qui est mal aux yeux de Yahweh ; il agit même plus mal que tous ceux qui avaient été avant lui.
\VS{26}Il marcha dans la voie de Jéroboam, fils de Nebath, et se livra aux péchés que Jéroboam avait fait commettre à Israël, irritant Yahweh, le Dieu d'Israël, par leurs idoles.
\VS{27}Le reste des actions d’Omri, tout ce qu'il a fait et ses exploits, cela n’est-il pas écrit au livre des Chroniques des rois d'Israël ?
\VS{28}Ainsi Omri s'endormit avec ses pères et fut enseveli à Samarie. Achab, son fils, régna à sa place.
\TextTitle{Achab règne sur Israël et épouse Jézabel}
\VS{29}Achab, fils d’Omri, régna sur Israël la trente-huitième année d'Asa, roi de Juda. Et Achab, fils d’Omri, régna sur Israël à Samarie vingt-deux ans.
\VS{30}Et Achab, fils d’Omri, fit ce qui est mal aux yeux de Yahweh, plus que tous ceux qui avaient été avant lui.
\VS{31}Et il arriva que, comme si ce lui eût été peu de chose de marcher dans les péchés de Jéroboam, fils de Nebath, il prit pour femme Jézabel, fille d'Ethbaal, roi des Sidoniens, puis il alla servir Baal et se prosterna devant lui.
\VS{32}Il dressa un autel à Baal, dans la maison de Baal, qu'il bâtit à Samarie.
\VS{33}Et Achab fit une idole d’Astarté. De sorte qu'Achab fit plus encore que tous les rois d'Israël qui avaient été avant lui, pour irriter Yahweh, le Dieu d'Israël.
\VS{34}En son temps, Hiel de Béthel bâtit Jéricho ; il en jeta les fondements au prix d’Abiram, son premier-né, et posa ses portes sur Segub, son plus jeune fils, selon la parole que Yahweh avait proférée par le moyen de Josué, fils de Nun.
\Chap{17}
\TextTitle{Elie annonce trois ans de sécheresse\FTNTT{1 R. 17-2 R. 1}}
\VerseOne{}Alors Elie, Thischbite, l’un des habitants de Galaad, dit à Achab : Yahweh, le Dieu d'Israël, en la présence duquel je me tiens, est vivant ! Il n'y aura ces années-ci ni rosée ni pluie, sinon à ma parole.
\TextTitle{Elie au torrent de Kerith}
\VS{2}Puis la parole de Yahweh fut adressée à Elie, en disant :
\VS{3}Va-t'en d'ici et tourne-toi vers l'orient ; cache-toi près du torrent de Kerith, qui est en face du Jourdain.
\VS{4}Tu boiras de l’eau du torrent, et j'ai commandé aux corbeaux de t'y nourrir.
\VS{5}Il partit donc et fit selon la parole de Yahweh, il s'en alla et demeura au torrent de Kerith, vis-à-vis du Jourdain.
\VS{6}Les corbeaux lui apportaient du pain et de la viande le matin, et du pain et de la viande le soir, et il buvait de l’eau du torrent.
\VS{7}Mais il arriva qu'au bout d’un certain temps le torrent tarit, parce qu'il n'y avait point eu de pluie dans le pays.
\TextTitle{Elie chez la veuve de Sarepta}
\VS{8}Alors la parole de Yahweh lui fut adressée, en ces mots :
\VS{9}Lève-toi, va à Sarepta, qui appartient à Sidon, et demeure-là. Voici, j'ai commandé là à une femme veuve de t'y nourrir.
\VS{10}Il se leva donc et s'en alla à Sarepta. Et comme il fut arrivé à l’entrée de la ville, voici, une femme veuve était là, qui ramassait du bois. Et il l'appela et lui dit : Apporte-moi, je te prie, un peu d'eau dans un vase et que je boive.
\VS{11}Elle alla en chercher. Il l’appela de nouveau et dit : Apporte-moi, je te prie, un morceau de pain de ta main.
\VS{12}Mais elle répondit : Yahweh, ton Dieu, est vivant ! Je n'ai rien de cuit, je n'ai qu’une poignée de farine dans un pot et un peu d'huile dans une cruche. Et voici, j'amasse deux morceaux de bois, puis je rentrerai, je l'apprêterai pour moi et pour mon fils, nous le mangerons, après quoi nous mourrons.
\VS{13}Et Elie lui dit : Ne crains point, va, fais comme tu dis. Seulement, fais-moi d’abord avec cela un petit gâteau et tu me l’apporteras, tu en feras ensuite pour toi et pour ton fils.
\VS{14}Car ainsi parle Yahweh, le Dieu d'Israël : La farine qui est dans le pot ne finira point et l'huile qui est dans la cruche ne diminuera point, jusqu'à ce que Yahweh donne de la pluie sur la terre.
\VS{15}Elle s'en alla donc, et fit selon la parole d'Elie. Et elle eut à manger, elle et sa famille, ainsi qu’Elie pendant plusieurs jours.
\VS{16}La farine du pot ne finit point, et l'huile de la cruche ne diminua point, selon la parole que Yahweh avait prononcée par le moyen d'Elie.
\TextTitle{Résurrection du fils de la veuve de Sarepta}
\VS{17}Après ces choses, il arriva que le fils de la femme, maîtresse de la maison, devint malade ; et la maladie fut si forte, qu'il expira.
\VS{18}Et elle dit à Elie : Qu'y a-t-il entre moi et toi, homme de Dieu ? Es-tu venu chez moi pour rappeler le souvenir de mon iniquité, et pour faire mourir mon fils ?
\VS{19}Et il lui dit : Donne-moi ton fils. Et il le prit du sein de cette femme, le porta dans la chambre haute où il demeurait, et le coucha sur son lit.
\VS{20}Puis il cria à Yahweh, et dit : Yahweh, mon Dieu ! Affligeras-tu cette veuve au point de faire mourir son fils, elle qui a m’a reçu comme un hôte ?
\VS{21}Et il s'étendit sur l'enfant par trois fois, et cria à Yahweh, en disant : Yahweh, mon Dieu ! Je te prie que l'âme de cet enfant revienne au-dedans de lui.
\VS{22}Et Yahweh écouta la voix d'Elie, l'âme de l'enfant revint au-dedans de lui, et il fut rendu à la vie.
\VS{23}Elie prit l'enfant, le descendit de la chambre haute dans la maison et le donna à sa mère, en lui disant : Regarde, ton fils est vivant.
\VS{24}Et la femme dit à Elie : Je reconnais maintenant, que tu es un homme de Dieu et que la parole de Yahweh, qui est dans ta bouche, est vérité.
\Chap{18}
\TextTitle{Elie à la rencontre d’Abdias puis d’Achab}
\VerseOne{}Et il arriva, après bien des jours, que la parole de Yahweh fut adressée à Elie, dans la troisième année, en disant : Va, montre-toi à Achab et je ferai tomber de la pluie sur la terre.
\VS{2}Et Elie s'en alla pour se présenter devant Achab. Il y avait alors une grande famine en Samarie.
\VS{3}Achab avait appelé Abdias, chef de sa maison ; or, Abdias craignait beaucoup Yahweh ;
\VS{4}quand Jézabel exterminait les prophètes de Yahweh, Abdias prit cent prophètes et les cacha, cinquante dans une caverne et cinquante dans une autre, et il les y nourrit de pain et d'eau.
\VS{5}Achab dit alors à Abdias : Va par le pays vers toutes les sources d'eaux et vers tous les torrents ; peut-être que nous trouverons de l'herbe, nous garderons ainsi en vie les chevaux et les mulets, et nous n’aurons pas besoin d’abattre du bétail.
\VS{6}Ils se partagèrent donc entre eux le pays pour le parcourir ; Achab allait seul par un chemin et Abdias allait seul par un autre chemin.
\VS{7}Comme Abdias était en chemin, voici, Elie le rencontra. Abdias reconnut Elie, il tomba sur son visage et lui dit : N'es-tu pas mon seigneur Elie ?
\VS{8}Il lui répondit : C'est moi ; va et dis à ton seigneur : Voici Elie !
\VS{9}Et Abdias dit : Quel péché ai-je commis, pour que tu livres ton serviteur entre les mains d'Achab pour me faire mourir ?
\VS{10}Yahweh, ton Dieu, est vivant ! Il n'y a ni nation, ni royaume, où mon seigneur n'ait envoyé pour te chercher ; et quand on répondait que tu n'y étais pas, il faisait jurer aux rois et au peuple que l'on ne t’avait pas trouvé.
\VS{11}Et maintenant tu dis : Va, dis à ton seigneur, voici Elie !
\VS{12}Puis, lorsque je t’aurai quitté, l'Esprit de Yahweh te transportera je ne sais où et j’irai informer Achab qui ne te trouvera pas et qui me tuera. Or, ton serviteur craint Yahweh dès sa jeunesse.
\VS{13}N'a-t-on point dit à mon seigneur ce que je fis quand Jézabel tuait les prophètes de Yahweh, comment j'en cachai cent, cinquante dans une caverne et cinquante dans une autre et les y ai nourris de pain et d'eau ?
\VS{14}Et maintenant tu dis : Va, dis à ton seigneur : Voici Elie ! Il me tuera !
\VS{15}Mais Elie lui répondit : Yahweh des armées, devant lequel je me tiens, est vivant ! Aujourd'hui, je me montrerai à Achab.
\VS{16}Abdias étant allé à la rencontre d'Achab, l’informa de la chose ; puis Achab alla au-devant d'Elie.
\VS{17}Et aussitôt qu'Achab eut vu Elie, il lui dit : Est-ce toi qui jettes le trouble en Israël ?
\VS{18}Et Elie lui répondit : Je n'ai point troublé Israël ; c'est toi et la maison de ton père, puisque vous avez abandonné les commandements de Yahweh et que vous êtes allés après les Baals.
\VS{19}Fais maintenant se rassembler tout Israël auprès de moi, sur le mont Carmel, les quatre cent cinquante prophètes de Baal et les quatre cents prophètes d’Astarté qui mangent à la table de Jézabel.
\TextTitle{Confrontation entre Elie et les prophètes de Baal sur le mont Carmel}
\VS{20}Ainsi Achab envoya des messagers vers tous les fils d'Israël et, il rassembla les prophètes sur le mont Carmel.
\VS{21}Alors Elie s'approcha de tout le peuple et dit : Jusqu'à quand clocherez-vous des deux côtés ? Si Yahweh est Dieu, suivez-le ; mais si Baal est dieu, suivez-le. Et le peuple ne lui répondit pas un seul mot.
\VS{22}Alors Elie dit au peuple : Je suis demeuré seul prophète de Yahweh ; et voici quatre cent cinquante prophètes de Baal.
\VS{23}Que l’on nous donne deux veaux, qu'ils en choisissent l'un pour eux, qu'ils le coupent en pièces et qu'ils le mettent sur du bois ; mais qu'ils n'y mettent point de feu ; et je préparerai l'autre veau, je le mettrai sur du bois, sans y mettre le feu.
\VS{24}Puis invoquez le nom de vos dieux, et moi j'invoquerai le nom de Yahweh ; que le dieu qui répondra par le feu, soit reconnu pour être Dieu. Et tout le peuple répondit et dit : C'est bien !
\VS{25}Et Elie dit aux prophètes de Baal : Choisissez un veau et préparez-le les premiers, car vous êtes en plus grand nombre et invoquez le nom de vos dieux ; mais n'y mettez point de feu.
\VS{26}Ils prirent donc un veau qu'on leur donna, ils l'apprêtèrent et ils invoquèrent le nom de Baal depuis le matin jusqu'à midi, en disant : Baal exauce-nous ! Mais il n'y avait ni voix ni réponse et ils sautaient devant l'autel qu'ils avaient fait.
\VS{27}A midi, Elie se moqua d'eux et dit : Criez à haute voix, puisqu’il est dieu ; mais il pense à quelque chose, ou il est occupé, ou il est en voyage ; peut-être qu'il dort et il se réveillera.
\VS{28}Ils criaient donc à haute voix ; ils se faisaient des incisions avec des couteaux et des lances, selon leur coutume, en sorte que le sang coulait sur eux.
\VS{29}Lorsque midi fut passé et qu'ils eurent fait les prophètes jusqu'au temps où l’on offre l'oblation, sans qu'il y eût ni voix, ni réponse, ni signe d’attention.
\VS{30}Elie dit alors à tout le peuple : Approchez-vous de moi ! Et tout le peuple s'approcha de lui et il répara l'autel de Yahweh, qui avait été renversé.
\VS{31}Puis Elie prit douze pierres, selon le nombre des tribus des fils de Jacob, auquel la parole de Yahweh avait été adressée, en disant : Israël sera ton nom.
\VS{32}Et il rebâtit de ces pierres l'autel au nom de Yahweh. Puis il fit un fossé de la capacité de deux mesures de semence autour de l'autel.
\VS{33}Il rangea le bois, il coupa le veau en pièces, et il le plaça sur le bois.
\VS{34}Puis il dit : Remplissez quatre cruches d'eau, puis versez-les sur l'holocauste et sur le bois. Puis il dit : Faites-le encore une seconde fois. Et ils le firent une seconde fois. Il dit : Faites-le une troisième fois. Et ils le firent pour la troisième fois ;
\VS{35}de sorte que les eaux allaient à l'entour de l'autel ; et il remplit aussi d’eau le fossé.
\VS{36}Et au moment de la présentation de l’offrande, Elie, le prophète, s'approcha et dit : Ô Yahweh ! Dieu d'Abraham, d'Isaac et d'Israël ! Que l’on sache aujourd'hui que tu es Dieu en Israël et que je suis ton serviteur ; et que j'ai fait toutes ces choses par ta parole !
\VS{37}Réponds-moi, Ô Yahweh ! Réponds-moi, afin que ce peuple connaisse que c’est toi, Yahweh, qui es Dieu et que c'est toi qui ramènes leur cœur.
\VS{38}Alors le feu de Yahweh tomba et consuma l'holocauste, le bois, les pierres et la terre, et il absorba toute l'eau qui était dans le fossé.
\VS{39}Quand tout le peuple vit cela, ils tombèrent sur leur visage et dirent : C'est Yahweh qui est Dieu ! C'est Yahweh qui est Dieu !
\VS{40}Et Elie leur dit : Saisissez les prophètes de Baal et qu'il n'en échappe aucun ! Ils les saisirent. Elie les fit descendre au torrent de Kison, où il les fit égorger là.
\TextTitle{Retour de la pluie selon la parole d’Elie\FTNTT{Ja. 5:17-18}}
\VS{41}Puis Elie dit à Achab : Monte, mange et bois ; car il se fait un bruit qui annonce la pluie.
\VS{42}Ainsi Achab monta pour manger et pour boire tandis qu’Elie monta au sommet du Carmel ; et, se penchant contre terre, il mit son visage entre ses genoux ;
\VS{43}Et il dit à son serviteur : Monte maintenant et regarde vers la mer. Le serviteur monta, il regarda et dit : Il n'y a rien. Elie dit par sept fois : Retournes-y.
\VS{44}A la septième fois, il dit : Voici un petit nuage qui s’élève de la mer et qui est comme la paume de la main d'un homme, laquelle monte de la mer. Elie dit : Monte et dis à Achab : Attelle ton char et descends de peur que la pluie ne t’arrête.
\VS{45}Ici et là, les cieux s'obscurcirent de nuages accompagnés de vent et il y eut une forte pluie. Achab monta sur son char et partit pour Jizreel.
\VS{46}Et la main de Yahweh fut sur Elie, qui se ceignit les reins et courut devant Achab, jusqu'à l'entrée de Jizreel.
\Chap{19}
\TextTitle{Fuite d’Elie devant les menaces de Jézabel}
\VerseOne{}Achab rapporta à Jézabel tout ce qu'Elie avait fait, et comment il avait tué par l'épée tous les prophètes.
\VS{2}Et Jézabel envoya un messager vers Elie, pour lui dire : Que les dieux me traitent dans toute leur rigueur, si demain, à cette heure-ci, je ne fais de ta vie ce que tu as fait de la vie de chacun d'eux !
\VS{3}Elie, voyant cela, se leva et s'en alla pour sauver sa vie. Il arriva à Beer-Schéba, qui appartient à Juda ; et il laissa là son serviteur.
\TextTitle{L’ange de Yahweh fortifie Elie}
\VS{4}Mais lui s'en alla dans le désert où, après une journée de marche, il s'assit sous un genêt et demanda la mort, en disant : C'en est assez, Ô Yahweh ! Prends mon âme, car je ne suis pas meilleur que mes pères.
\VS{5}Puis il se coucha et s'endormit sous un genêt. Voici un ange le toucha et lui dit : Lève-toi, mange.
\VS{6}Et il regarda, et voici à son chevet, un gâteau cuit sur des pierres chauffées et une cruche d'eau. Il mangea et but, puis se recoucha.
\VS{7}Et l'ange de Yahweh vint une seconde fois, le toucha et lui dit : Lève-toi, mange, car le chemin est trop long pour toi.
\TextTitle{Elie à Horeb, visitation et instructions de Yahweh}
\VS{8}Il se leva donc, mangea et but ; puis avec la force que lui donna cette nourriture, il marcha quarante jours et quarante nuits jusqu'à Horeb, la montagne de Dieu.
\VS{9}Et là, il entra dans une caverne et y passa la nuit. Et voici, la parole de Yahweh lui fut adressée en ces mots : Que fais-tu ici, Elie ?
\VS{10}Et il répondit : J'ai déployé mon zèle pour Yahweh, le Dieu des armées, parce que les enfants d'Israël ont abandonné ton alliance, ils ont renversé tes autels, ils ont tué tes prophètes par l'épée ; je suis resté, moi seul et ils me cherchent pour m'ôter la vie.
\VS{11}Yahweh lui dit : Sors et tiens-toi sur la montagne devant Yahweh. Et voici, Yahweh passa. Et devant Yahweh, il y eut un grand vent impétueux qui déchirait les montagnes et brisait les rochers, mais Yahweh n'était point dans ce vent. Après le vent, ce fut un tremblement de terre ; mais Yahweh n'était point dans ce tremblement de terre.
\VS{12}Après le tremblement de terre, un feu ; mais Yahweh n'était pas dans le feu. Et après le feu vint un murmure doux et léger.
\VS{13}Quand Elie l'entendit, il s’enveloppa le visage de son manteau, il sortit et se tint à l'entrée de la caverne. Et voici, une voix lui fit entendre ces paroles : Que fais-tu ici Elie ?
\VS{14}Et il répondit : J'ai déployé mon zèle pour Yahweh, le Dieu des armées, parce que les enfants d'Israël ont abandonné ton alliance, ils ont renversé tes autels, ils ont tué par l'épée tes prophètes ; je suis resté moi seul, et ils cherchent ma vie pour me l'ôter.
\VS{15}Yahweh lui dit : Va, retourne-t'en par ton chemin vers le désert de Damas ; et quand tu seras arrivé, tu oindras Hazaël pour roi de Syrie.
\VS{16}Tu oindras aussi Jéhu, fils de Nimschi, pour roi d’Israël ; et tu oindras Elisée, fils de Schaphath, d'Abel-Mehola, pour prophète à ta place.
\VS{17}Et il arrivera que quiconque échappera de l'épée de Hazaël, Jéhu le fera mourir ; et quiconque échappera de l'épée de Jéhu, Elisée le fera mourir.
\VS{18}Mais je me suis réservé sept mille hommes de reste en Israël, tous ceux qui n'ont point fléchi les genoux devant Baal, et dont la bouche ne l'a point baisé.
\TextTitle{Elisée devient disciple d’Elie}
\VS{19}Elie partit donc de là, et il trouva Elisée, fils de Schaphath, qui labourait. Il y avait douze paires de bœufs devant soi et il était avec la douzième. Quand Elie passa près de lui, il jeta sur lui son manteau.
\VS{20}Elisée laissa ses bœufs et courut après Elie, en disant : Je t’en prie, laisse-moi embrasser mon père et ma mère, et je te suivrai. Elie lui répondit : Va, et reviens ; car pense à ce que je t'ai fait.
\VS{21}Après s’être éloigné d’Elie, il revint prendre une paire de bœufs qu’il offrit en sacrifice ; et avec l'attelage des bœufs, il en fit bouillir la chair, et la donna au peuple ; ils mangèrent ; puis il se leva et suivit Elie. Dès lors, il fut à son service.
\Chap{20}
\TextTitle{Achab monte contre Ben-Hadad}
\VerseOne{}Alors Ben-Hadad, roi de Syrie rassembla toute son armée ; il avait avec lui trente-deux rois, des chevaux et des chars. Puis il monta, assiégea Samarie et il lui fit la guerre.
\VS{2}Il envoya des messagers à Achab, roi d'Israël, dans la ville ;
\VS{3}Et il lui fit dire : Ainsi parle Ben-Hadad : Ton argent et ton or sont à moi, tes femmes aussi et tes beaux enfants sont à moi.
\VS{4}Et le roi d'Israël répondit, et dit : Mon seigneur, je suis à toi, comme tu le dis, avec tout ce que j'ai.
\VS{5}Ensuite les messagers retournèrent, et dirent : Ainsi parle Ben-Hadad : Puisque je t'ai envoyé dire : Donne-moi ton argent et ton or, ta femme et tes enfants ;
\VS{6}A la même heure demain, j'enverrai chez toi mes serviteurs, ils fouilleront ta maison et les maisons de tes serviteurs, et se saisiront de tout ce que tu as de précieux, et ils l'emporteront.
\VS{7}Alors le roi d'Israël appela tous les anciens du pays, et il dit : Sachez et considérez, je vous prie, combien cet homme nous veut du mal ; car il m’a envoyé demander mes femmes, mes enfants, mon argent et mon or, et je ne lui avais rien refusé.
\VS{8}Et tous les anciens et tout le peuple lui dirent : Ne l'écoute point et ne consens pas.
\VS{9}Il répondit donc aux messagers de Ben-Hadad : Dites au roi, mon seigneur : Je ferai tout ce que tu as envoyé demander la première fois à ton serviteur, mais je ne pourrai faire ceci. Les messagers s'en allèrent et lui rapportèrent cette réponse.
\VS{10}Et Ben-Hadad envoya dire à Achab : Que les dieux me traitent dans toute leur rigueur, si la poudre de Samarie suffit pour remplir le creux de la main de tout le peuple qui me suit.
\VS{11}Mais le roi d'Israël répondit, et dit : Dites-lui : Que celui qui revêt une armure ne se glorifie point comme celui qui la dépose.
\VS{12}Lorsque Ben-Hadad entendit cette réponse, il était à boire avec les rois sous les tentes et il dit à ses serviteurs : Rangez-vous en bataille ! Et ils se rangèrent en bataille contre la ville.
\TextTitle{Victoire Achab}
\VS{13}Alors voici, un prophète s’approcha d’Achab, roi d'Israël et lui dit : Ainsi parle Yahweh : N'as-tu pas vu cette grande multitude ? Voilà, je m'en vais la livrer aujourd'hui entre tes mains, et tu sauras que je suis Yahweh.
\VS{14}Et Achab dit : Par qui ? Et il lui répondit : Ainsi parle Yahweh : Ce sera par les serviteurs des chefs des provinces. Et Achab dit : Qui engagera le combat ? Et il lui répondit : Toi.
\VS{15}Alors il passa en revue les serviteurs des chefs des provinces, qui furent deux cent trente-deux ; et après eux, il dénombra tout le peuple de tous les enfants d'Israël qui furent sept mille.
\VS{16}Ils firent une sortie en plein midi, lorsque Ben-Hadad buvait et s'enivrait dans les tentes, lui et les trente-deux rois qui étaient ses auxiliaires.
\VS{17}Les serviteurs des chefs des provinces sortirent les premiers et Ben-Hadad envoya quelques-uns qui le lui rapportèrent en disant : Des hommes sont sortis de Samarie.
\VS{18}Et il dit : Qu’ils soient sortis pour la paix, ou qu'ils soient sortis pour faire la guerre, saisissez-les tous vivants.
\VS{19}Les serviteurs des chefs de province sortirent de la ville puis l'armée qui était après eux.
\VS{20}Chacun d'eux frappa son homme, de sorte que les Syriens s'enfuirent et Israël les poursuivit. Ben-Hadad, roi de Syrie, se sauva sur un cheval, avec des cavaliers.
\VS{21}Et le roi d'Israël sortit et frappa les chevaux et les chars, en sorte qu'il fit éprouver une grande défaite aux Syriens.
\TextTitle{Achab monte de nouveau contre les Syriens}
\VS{22}Alors le prophète s’approcha du roi d'Israël, et lui dit : Va, fortifie-toi ; considère et vois ce que tu auras à faire ; car l’année révolue, le roi de Syrie montera contre toi.
\VS{23}Or, les serviteurs du roi de Syrie lui dirent : Leur dieu est un dieu de montagnes, c'est pourquoi ils ont été plus forts que nous. Mais combattons contre eux dans la plaine, et certainement, nous serons plus forts qu'eux.
\VS{24}Fais donc ceci : Ote chacun de ces rois de sa place, et remplace-les par des chefs ;
\VS{25}Puis lève une armée pareille à celle que tu as perdue, avec autant de chevaux et de chars, puis nous les combattrons dans la plaine et l’on verra si nous ne sommes pas plus forts qu'eux. Il les écouta, et fit ainsi.
\VS{26}L’année suivante, Ben-Hadad dénombra les Syriens et monta à Aphek pour combattre contre Israël.
\VS{27}On fit aussi le dénombrement des enfants d'Israël ; ils reçurent des vivres, et ils marchèrent à la rencontre des Syriens. Les enfants d'Israël campèrent vis-à-vis d'eux ; semblables à deux petits troupeaux de chèvres, tandis que les Syriens remplissaient le pays.
\VS{28}Alors l'homme de Dieu vint, et dit au roi d'Israël : Ainsi parle Yahweh : Parce que les Syriens ont dit : Yahweh est un dieu des montagnes et non un dieu des vallées, je livrerai entre tes mains toute cette grande multitude, et vous saurez que je suis Yahweh.
\VS{29}Sept jours durant ils campèrent vis-à-vis les uns des autres. Le septième jour, ils entrèrent en bataille, et les enfants d'Israël tuèrent en un seul jour cent mille hommes de pied des Syriens.
\VS{30}Le reste s'enfuit à la ville d'Aphek, où la muraille tomba sur vingt-sept mille hommes demeurés de reste. Ben-Hadad s'était réfugié dans la ville où il allait de chambre en chambre.
\TextTitle{Faute d'Achab qui épargne Ben-Hadad}
\VS{31}Ses serviteurs lui dirent : Voici maintenant, nous avons appris que les rois de la maison d'Israël sont des rois miséricordieux ; maintenant donc mettons des sacs sur nos reins et des cordes à nos têtes, sortons vers le roi d'Israël, peut-être qu'il te laissera la vie sauve.
\VS{32}Ils se mirent donc des sacs autour des reins et des cordes autour de leurs têtes. Ils allèrent auprès du roi d'Israël. Ils lui dirent : Ton serviteur Ben-Hadad dit : Laisse-moi la vie ! Achab répondit : Est-il encore vivant ? Il est mon frère.
\VS{33}Ces hommes tirèrent de là un bon augure, ils se hâtèrent de le prendre au mot et ils dirent : Ben-Hadad est-il ton frère ! Et il répondit : Allez, amenez-le. Ben-Hadad vint vers lui, et il le fit monter sur son char.
\VS{34}Et Ben-Hadad lui dit : Je te rendrai les villes que mon père avait prises à ton père ; et tu te feras des rues en Damas comme mon père avait fait en Samarie. Et moi, répondit Achab, je te laisserai aller en faisant alliance. Il traita donc alliance avec lui, et le laissa aller.
\VS{35}Alors un homme d'entre les fils des prophètes dit à son compagnon, sur l’ordre de Yahweh : Frappe-moi, je te prie ! Mais celui-là refusa de le frapper.
\VS{36}Et il lui dit : Parce que tu n'as point obéi à la parole de Yahweh, voilà, quand tu m’auras quitté, un lion te frappera. Quand il se fut séparé de lui, un lion survint et le frappa.
\VS{37}Puis il trouva un autre homme, et lui dit : Frappe-moi, je te prie. Cet homme-là le frappa et il le blessa.
\VS{38}Après cela le prophète s'en alla, et se plaça sur le chemin du roi ; il se déguisa avec un bandeau sur ses yeux.
\VS{39}Lorsque le roi passa, il cria vers lui, et dit : Ton serviteur était allé au milieu de la bataille ; et voici quelqu'un s'étant retiré, m'a amené un homme, en disant : Garde cet homme, s'il vient à s'échapper, ta vie en répondra, ou tu paieras un talent d'argent.
\VS{40}Et pendant que ton serviteur faisait quelques affaires çà et là, cet homme a disparu. Et le roi d'Israël lui répondit : Telle est ta condamnation, tu l’as toi-même prononcée.
\VS{41}Alors le prophète ôta promptement le bandeau de dessus ses yeux et le roi d'Israël reconnut que c'était l’un des prophètes.
\VS{42}Et il dit : Ainsi parle Yahweh : Parce que tu as laissé échapper de tes mains l'homme que j'avais dévoué par la voie de l'interdit, ta vie répondra de sa vie, et ton peuple de son peuple.
\VS{43}Mais le roi d'Israël se retira en sa maison, triste et irrité ; et il arriva en Samarie.
\Chap{21}
\TextTitle{Achab convoite la vigne de Naboth}
\VerseOne{}Après ces choses, voici ce qui arriva. Naboth de Jizreel, avait une vigne à Jizreel, près du palais d'Achab, roi de Samarie.
\VS{2}Achab parla à Naboth et lui dit : Cède-moi ta vigne, afin que j'en fasse un jardin potager, car elle est proche de ma maison et je te donnerai à la place une vigne meilleure ; ou, si cela te semble bon, je te paierai l'argent qu'elle vaut.
\VS{3}Mais Naboth répondit à Achab : Que Yahweh me garde de te donner l'héritage de mes pères !
\VS{4}Et Achab vint en sa maison tout triste et irrité, à cause de cette parole que lui avait dite Naboth de Jizreel, en disant : Je ne te donnerai point l'héritage de mes pères ! Il se coucha sur son lit, détourna son visage, et ne mangea rien.
\TextTitle{Manigance meurtrière de Jézabel}
\VS{5}Alors Jézabel, sa femme, vint auprès de lui, et lui dit : D'où vient que ton esprit est si triste ? Et pourquoi ne manges-tu point ?
\VS{6}Et il lui répondit : J’ai parlé à Naboth de Jizreel, et je lui ai dit : Donne-moi ta vigne pour de l'argent, ou si tu le désires, je te donnerai une autre vigne pour celle-là, mais il m'a dit : Je ne te céderai point ma vigne !
\VS{7}Alors Jézabel, sa femme, lui dit : Est-ce bien toi maintenant qui exerces la royauté sur Israël ? Lève-toi, prends un repas et que ton cœur se réjouisse ; je te ferai avoir la vigne de Naboth de Jizreel.
\VS{8}Et elle écrivit au nom d'Achab des lettres qu’elle scella du sceau du roi, et elle envoya aux anciens et magistrats qui habitaient avec Naboth, dans sa ville.
\VS{9}Voici ce qu’elle écrivit dans ces lettres : Publiez un jeûne et placez Naboth à la tête du peuple.
\VS{10}Mettez face à lui deux méchants hommes et qu'ils témoignent contre lui, en disant : Tu as maudit Dieu et le roi ! Puis vous le mènerez dehors et le lapiderez afin qu'il meure.
\VS{11}Les gens donc de la ville de Naboth, les anciens et les magistrats qui habitaient dans sa ville, agirent comme Jézabel le leur avait dit, et d’après ce qui était écrit dans les lettres qu'elle leur avait envoyées.
\VS{12}Ils publièrent un jeûne et ils placèrent Naboth à la tête du peuple.
\VS{13}Les deux méchants hommes vinrent et se mirent face à lui, et ces méchants hommes déclarèrent contre Naboth en la présence du peuple : Naboth a maudit Dieu et le roi ! Puis ils le menèrent hors de la ville, ils le lapidèrent, et il mourut.
\VS{14}Après cela, ils envoyèrent dire à Jézabel : Naboth a été lapidé, et il est mort.
\VS{15}Lorsque Jézabel apprit que Naboth avait été lapidé et qu'il était mort, elle dit à Achab : Lève-toi, mets-toi en possession de la vigne de Naboth de Jizreel, qu’il avait refusé de te donner pour de l'argent ; car Naboth n'est plus en vie, il est mort.
\VS{16}Ainsi dès qu'Achab eut entendu que Naboth était mort, il se leva pour descendre à la vigne de Jizreel et pour s'en mettre en possession.
\TextTitle{Jugement d’Achab et de Jézabel ; Achab s’humilie devant Dieu}
\VS{17}Alors la parole de Yahweh fut adressée à Elie, le Thischbite, en ces mots :
\VS{18}Lève-toi, descends au-devant d'Achab, roi d'Israël, lorsqu'il sera à Samarie. Le voilà dans la vigne de Naboth, où il est descendu pour en prendre possession.
\VS{19}Et tu lui diras : Ainsi parle Yahweh : N’es-tu pas un meurtrier et un voleur ? Puis tu lui diras : Ainsi parle Yahweh : Comme les chiens ont léché le sang de Naboth, les chiens lécheront aussi ton propre sang.
\VS{20}Et Achab dit à Elie : M'as-tu trouvé mon ennemi ? Mais il lui répondit : Oui, je t'ai trouvé, parce que tu t'es vendu pour faire ce qui est mal aux yeux de Yahweh.
\VS{21}Voici je vais faire venir le malheur sur toi, et je te consumerai, j’exterminerai quiconque appartient à Achab, tant celui qui est esclave, que celui qui est libre en Israël.
\VS{22}Je rendrai ta maison semblable à la maison de Jéroboam, fils de Nebath, et la maison de Baescha, fils d'Achija, parce que tu m'as irrité et fait pécher Israël.
\VS{23}Yahweh parla aussi contre Jézabel en disant : Les chiens mangeront Jézabel près du rempart de Jizreel.
\VS{24}Celui de la maison d’Achab qui mourra dans la ville, les chiens le mangeront, et celui qui mourra aux champs, les oiseaux des cieux le mangeront.
\VS{25}En effet, il n'y en avait point eu de personne comme Achab, qui se soit vendu pour faire ce qui est mal aux yeux de Yahweh, et sa femme Jézabel l’y excitait ;
\VS{26}de sorte qu'il se rendit fort abominable, allant après les idoles, comme l'avaient fait les Amoréens, que Yahweh avait chassés de devant les enfants d'Israël.
\VS{27}Après avoir entendu les paroles d’Elie, Achab déchira ses vêtements, il mit un sac sur son corps, et jeûna. Il se tenait couché avec ce sac, et il marchait lentement.
\VS{28}Et la parole de Yahweh fut adressée à Elie, le Thischbite, en disant :
\VS{29}As-tu vu comment Achab s'est humilié devant moi ? Parce qu'il s'est humilié devant moi, je ne ferai pas venir le malheur pendant sa vie, ce sera aux jours de son fils que je ferai venir le malheur sur sa maison.
\Chap{22}
\TextTitle{Josaphat aide Achab contre les Syriens}
\VerseOne{}Et on resta trois ans sans qu'il y eût guerre entre la Syrie et Israël.
\VS{2}Puis il arriva, dans la troisième année, que Josaphat, roi de Juda, descendit vers le roi d'Israël.
\VS{3}Le roi d'Israël dit à ses serviteurs : Ne savez-vous pas que Ramoth de Galaad nous appartient ? Et nous ne nous inquiétons pas de la reprendre des mains du roi de Syrie !
\VS{4}Puis il dit à Josaphat : Viendras-tu avec moi à la guerre contre Ramoth de Galaad ? Et Josaphat répondit au roi d'Israël : Nous irons, moi comme toi, mon peuple comme ton peuple, et mes chevaux comme tes chevaux.
\TextTitle{Les prophètes de mensonge\FTNTT{2 Ch. 18:4-5, 9-11}}
\VS{5}Josaphat dit encore au roi d'Israël : Consulte aujourd'hui, je te prie, la parole de Yahweh.
\VS{6}Et le roi d'Israël assembla les prophètes, au nombre de quatre cents environ, auxquels il dit : Irai-je à la guerre contre Ramoth de Galaad, ou dois-je y renoncer ? Et ils répondirent : Monte, car le Seigneur la livrera entre les mains du roi.
\VS{7}Mais Josaphat dit : N'y a-t-il point ici encore quelque prophète de Yahweh, afin que nous le consultions ?
\VS{8}Et le roi d'Israël dit à Josaphat : Il y a encore un homme par qui l’on puisse consulter Yahweh, mais je le hais, car il ne prophétise rien de bon, mais seulement du mal, c'est Michée, fils de Jimla. Josaphat dit : Que le roi ne parle point ainsi !
\VS{9}Alors le roi d'Israël appela un eunuque auquel il dit : Fais venir promptement Michée, fils de Jimla.
\VS{10}Or, le roi d'Israël et Josaphat, roi de Juda, étaient assis chacun sur son trône, revêtus de leurs habits, dans la place, vers l'entrée de la porte de Samarie ; et tous les prophètes prophétisaient en leur présence.
\VS{11}Sédécias, fils de Kenaana, s'était fait des cornes de fer et il dit : Ainsi parle Yahweh : De ces cornes-ci tu heurteras les Syriens, jusqu'à les détruire.
\VS{12}Et tous les prophètes prophétisaient de même, en disant : Monte à Ramoth de Galaad et tu réussiras ; et Yahweh la livrera entre les mains du roi.
\TextTitle{Michée annonce la défaite et la mort d'Achab\FTNTT{2 Ch. 18:6-8, 12-27, 28-34}}
\VS{13}Le messager qui était allé appeler Michée, lui parla ainsi : Voici, les prophètes parlent d'un commun accord au sujet du roi ; je te prie que ta parole soit semblable à celle de chacun d’eux ! Annonce du bien !
\VS{14}Mais Michée lui répondit : Yahweh est vivant ! J’annoncerai ce que Yahweh me dira.
\VS{15}Il vint donc vers le roi, et le roi lui dit : Michée, irons-nous à la guerre contre Ramoth de Galaad, ou devons-nous y renoncer ? Et il lui dit : Monte et tu réussiras, et Yahweh la livrera entre les mains du roi.
\VS{16}Et le roi lui dit : Jusqu'à combien de fois te conjurerai-je de ne me dire que la vérité au nom de Yahweh ?
\VS{17}Et il répondit : J'ai vu tout Israël dispersé par les montagnes, comme un troupeau de brebis qui n'a point de berger ; et Yahweh a dit : Ces gens n’ont point de maître, que chacun retourne en paix dans sa maison !
\VS{18}Alors le roi d'Israël dit à Josaphat : Ne t'ai-je pas bien dit que quand il est question de moi il ne prophétise rien de bon, mais seulement du mal ?
\VS{19}Et Michée lui dit : Ecoute néanmoins la parole de Yahweh ! J'ai vu Yahweh assis sur son trône, et toute l'armée des cieux se tenant devant lui, à sa droite et à sa gauche.
\VS{20}Et Yahweh a dit : Quel est celui qui séduira Achab, afin qu'il monte et qu'il périsse en Ramoth de Galaad ? Et ils répondaient, l'un parlait d'une manière et l'autre d'une autre.
\VS{21}Alors un esprit s'avança et se tint devant Yahweh, il déclara : Je le séduirai. Et Yahweh lui dit : Comment ?
\VS{22}Et il répondit : Je sortirai et je serai un esprit de mensonge dans la bouche de tous ses prophètes. Et Yahweh dit : Tu le séduiras et même tu en viendras à bout ; sors et fais ainsi !
\VS{23}Et maintenant, voici, Yahweh a mis un esprit de mensonge dans la bouche de tous tes prophètes que voilà et Yahweh a prononcé du mal contre toi.
\VS{24}Alors Sédécias, fils de Kenaana, s'approcha et frappa Michée sur la joue et dit : Par où l'Esprit de Yahweh est-il sorti de moi pour s'adresser à toi ?
\VS{25}Et Michée répondit : Voici, tu le verras le jour où tu iras de chambre en chambre pour te cacher.
\VS{26}Alors le roi d'Israël dit : Qu'on prenne Michée et qu'on le mène vers Amon, capitaine de la ville et vers Joas, le fils du roi.
\VS{27}Et tu diras : Ainsi a parlé le roi : Mettez cet homme en prison, nourrissez-le de pain et de l’eau d’affliction, jusqu'à ce que je revienne en paix.
\VS{28}Et Michée répondit : Si tu reviens en paix, Yahweh n'a point parlé par moi. Il dit aussi : Vous tous, peuples, entendez !
\VS{29}Le roi d'Israël monta avec Josaphat, roi de Juda, contre Ramoth de Galaad.
\VS{30}Et le roi d'Israël dit à Josaphat : Que je me déguise et que j'aille à la bataille ; mais toi, revêts-toi de tes habits. Le roi d'Israël donc se déguisa et alla au combat.
\VS{31}Or, le roi de Syrie avait donné un ordre aux trente-deux chefs de ses chars, en disant : Vous n’attaquerez ni petits ni grands, mais seulement contre le roi d'Israël.
\VS{32}Quand les chefs des chars aperçurent Josaphat, ils dirent : C'est certainement le roi d'Israël. Et ils s’approchèrent de lui pour le combattre, mais Josaphat s'écria.
\VS{33}Et quand les chefs des chars virent que ce n'était pas le roi d'Israël, ils se détournèrent de lui.
\TextTitle{Mort d’Achab}
\VS{34}Alors un homme tira de son arc au hasard, et frappa le roi d'Israël entre les jointures de la cuirasse. Et le roi dit à son conducteur de char : Tourne et fais-moi sortir du champ de bataille, car je suis blessé.
\VS{35}Or, le combat devint acharné ce jour-là. Le roi d'Israël fut arrêté dans son char en face des Syriens et il mourut sur le soir. Le sang de sa blessure coulait à l’intérieur du char.
\VS{36}Au coucher du soleil, on cria par tout le camp, en disant : Que chacun se retire en sa ville et chacun en son pays !
\VS{37}Ainsi mourut le roi, qui fut ramené à Samarie ; et l’on enterra le roi à Samarie.
\VS{38}Lorsqu’on lava le char à l’étang de Samarie, les chiens léchèrent le sang d’Achab, et les prostituées s’y baignèrent, selon la parole que Yahweh avait prononcée.
\VS{39}Le reste des actions d'Achab, tout ce qu’il a fait, la maison d'ivoire qu'il construisit et toutes les villes qu'il a bâties, toutes ces choses ne sont-elles pas écrites au livre des Chroniques des rois d'Israël ?
\TextTitle{Règne de Josaphat sur Juda\FTNTT{2 Ch. 17:19-20}}
\VS{40}Ainsi Achab se coucha avec ses pères. Et Achazia, son fils, régna à sa place.
\VS{41}Josaphat, fils d'Asa, régna sur Juda, la quatrième année d'Achab, roi d'Israël.
\VS{42}Josaphat avait trente-cinq ans lorsqu’il devint roi, et il régna vingt-cinq ans à Jérusalem. Sa mère s’appelait Azuba, fille de Schilchi.
\VS{43}Il suivit entièrement la voie d'Asa, son père, et ne s'en détourna point, faisant tout ce qui est droit aux yeux de Yahweh.
\VS{44}Toutefois les hauts lieux ne disparurent pas ; le peuple offrait encore des sacrifices et offrait encore des parfums sur les hauts lieux.
\VS{45}Josaphat fit aussi la paix avec le roi d'Israël.
\VS{46}Le reste des actions de Josaphat, ses exploits et les guerres qu'il mena ne sont-elles pas écrites au livre des Chroniques des rois de Juda ?
\VS{47}Il extermina du pays le reste des prostitués, qui étaient demeurés là depuis le temps d'Asa, son père.
\VS{48}Il n'y avait point alors de roi en Edom : c’était un intendant qui gouvernait.
\VS{49}Josaphat construisit des navires de Tarsis pour aller chercher de l'or à Ophir ; mais il n'y alla point, parce que les navires se brisèrent à Etsjon-Guéber.
\VS{50}Alors Achazia, fils d'Achab, dit à Josaphat : Que mes serviteurs aillent sur les navires avec les tiens, mais Josaphat ne le voulut point.
\TextTitle{Joram règne sur Juda\FTNTT{2 Ch. 21:1}}
\VS{51}Et Josaphat s’endormit avec ses pères et fut enterré avec eux en la cité de David, son père. Et Joram, son fils, régna à sa place.
\TextTitle{Achazia règne sur Israël}
\VS{52}Achazia, fils d'Achab, régna sur Israël à Samarie, la dix-septième année de Josaphat, roi de Juda. Et il régna deux ans sur Israël.
\VS{53}Il fit ce qui est mal aux yeux de Yahweh : il marcha dans la voie de son père, de sa mère et celle de Jéroboam, fils de Nebath, qui avait fait pécher Israël.
\VS{54}Il servit Baal, il se prosterna devant lui et il irrita Yahweh, le Dieu d'Israël, comme l’avait fait son père.
\PPE{}
\end{multicols}

%\clearpage\ShortTitle{2 R.}\BookTitle{2 Rois}\BFont
\noindent\hrulefill
{\footnotesize
\textit{
\bigskip
{\centering{}
\\Auteur~: Inconnu
\\(Heb.~: Melakhim)
\\Signification~: Roi, Règne
\\Thème~: Suite de l'histoire d'Israël et de Juda
\\Date de rédaction~: 6\up{ème} siècle av. J.-C.\\}
}
\textit{
\\Le second livre des rois s'articule autour de la vie d'Elisée, serviteur d'Elie, devenu dorénavant son successeur. On y découvre le service prophétique au travers duquel Dieu se révéla comme le Tout-Puissant, le Dieu compatissant, le Maître des temps et des circonstances, le Libérateur, le Dieu de la résurrection, le Puissant Guerrier et aussi le Juge.
\\Ce livre relate l'histoire des derniers rois, la chute d'Israël et sa captivité, la destruction de Jérusalem par Nebudcanetsar, roi de Babylone, en 586 av. J.-C., et la captivité de Juda.\bigskip
}
}
\par\nobreak\noindent\hrulefill
\begin{multicols}{2}
\Chap{1}
\TextTitle{Jugement de Yahweh sur Achazia, roi d'Israël}
\VerseOne{}Or après la mort d'Achab, Moab se révolta contre Israël.
\VS{2}Or Achazia tomba par le treillis de sa chambre haute qui était à Samarie, et il en fut malade. Il envoya des messagers et leur dit~: Allez, consultez Baal-Zebub\FTNT{Baal-Zebub était une divinité des Philistins adorée à Ekron qui se nommait aussi Béelzébul (Mt. 10:25).}, dieu d'Ekron, pour savoir si je guérirai de cette maladie.
\VS{3}Mais l'Ange de Yahweh dit à Elie\FTNT{Elie~: Voir 1 R. 17.}, le Thischbite~: Lève-toi, monte à la rencontre des messagers du roi de Samarie, et dis-leur~: N'y a-t-il point de Dieu en Israël pour que vous alliez consulter Baal-Zebub, dieu d'Ekron~?
\VS{4}C'est pourquoi ainsi parle Yahweh~: Tu ne descendras pas du lit sur lequel tu es monté, mais tu mourras, tu mourras\FTNT{Voir en Gn. 2:16.}. Et Elie s'en alla.
\VS{5}Les messagers retournèrent vers Achazia. Et il leur dit~: Pourquoi revenez-vous~?
\VS{6}Ils lui répondirent~: Un homme est monté à notre rencontre et nous a dit~: Allez, retournez vers le roi qui vous a envoyés et dites-lui~: Ainsi parle Yahweh~: N'y a-t-il point de Dieu en Israël, pour que tu envoies consulter Baal-Zebub, dieu d'Ekron~? A cause de cela, tu ne descendras pas du lit sur lequel tu es monté, mais certainement tu mourras.
\VS{7}Achazia leur dit~: Comment était cet homme qui est monté à votre rencontre et qui vous a dit ces paroles~?
\VS{8}Ils lui répondirent~: C'était un homme vêtu de poil, ayant une ceinture de cuir, ceinte sur ses reins. Et Achazia dit~: C'est Elie, le Thischbite.
\TextTitle{Affirmation de l'autorité d'Elie}
\VS{9}Alors il envoya vers lui un chef de cinquante avec ses cinquante hommes. Ce chef monta auprès d'Elie, qui demeurait au sommet d'une montagne, et il lui dit~: Homme de Dieu, le roi a dit~: Descends~!
\VS{10}Mais Elie répondit et dit au chef de cinquante~: Si je suis un homme de Dieu, que le feu descende du ciel et te consume, toi et tes cinquante hommes~! Et le feu descendit du ciel et le consuma, lui et ses cinquante hommes.
\VS{11}Achazia envoya encore un autre chef de cinquante avec ses cinquante hommes. Ce chef prit la parole et dit à Elie~: Homme de Dieu, ainsi parle le roi~: Hâte-toi de descendre~!
\VS{12}Mais Elie répondit, et leur dit~: Si je suis un homme de Dieu, que le feu descende du ciel et te consume, toi et tes cinquante hommes~! Et le feu de Dieu descendit du ciel et le consuma, lui et ses cinquante hommes.
\VS{13}Achazia envoya encore un troisième chef de cinquante avec ses cinquante hommes. Ce troisième chef de cinquante hommes monta, et vint se mettre à genoux devant Elie, le suppliant, en disant~: Homme de Dieu, je te prie, que ma vie et la vie de ces cinquante hommes, tes serviteurs, soit précieuse à tes yeux~!
\VS{14}Voici, le feu est descendu du ciel et a consumé les deux premiers chefs de cinquante, avec leurs cinquante hommes~; mais maintenant, je te prie, que ma vie soit précieuse à tes yeux~!
\VS{15}Et l'Ange de Yahweh dit à Elie~: Descends avec lui, n'aie pas peur de lui. Elie se leva donc et descendit avec lui vers le roi.
\VS{16}Il lui dit~: Ainsi parle Yahweh~: Parce que tu as envoyé des messagers pour consulter Baal-Zebub, dieu d'Ekron, comme s'il n'y avait point de Dieu en Israël, pour consulter sa parole, tu ne descendras pas du lit sur lequel tu es monté, mais certainement tu mourras.
\TextTitle{Mort d'Achazia~; Joram règne sur Israël}
\VS{17}Achazia mourut, selon la parole de Yahweh prononcée par Elie. Et Joram régna à sa place, la seconde année de Joram, fils de Josaphat, roi de Juda, parce qu'Achazia n'avait point de fils.
\VS{18}Le reste des actions d'Achazia et ce qu'il a fait, cela n'est-il pas écrit dans le livre des Chroniques des rois d'Israël~?
\Chap{2}
\TextTitle{Enlèvement d'Elie au ciel}
\VerseOne{}Or il arriva lorsque Yahweh enleva Elie au ciel dans un tourbillon, Elie et Elisée partaient de Guilgal.
\VS{2}Elie dit à Elisée~: Je te prie, reste ici, car Yahweh m'envoie jusqu'à Béthel. Mais Elisée répondit Yahweh est vivant et ton âme est vivante~! Je ne te quitterai pas~! Ainsi ils descendirent à Béthel.
\VS{3}Les fils des prophètes qui étaient à Béthel sortirent vers Elisée, et lui dirent~: Ne sais-tu pas qu'aujourd'hui Yahweh va enlever ton maître au-dessus de ta tête~? Et il répondit~: Je le sais aussi~; taisez-vous~!
\VS{4}Elie lui dit~: Elisée, je te prie, reste ici, car Yahweh m'envoie à Jéricho. Mais Elisée lui répondit~: Yahweh est vivant et ton âme est vivante~! Je ne te quitterai pas~! Ainsi, ils arrivèrent à Jéricho.
\VS{5}Les fils des prophètes qui étaient à Jéricho s'approchèrent d'Elisée, et lui dirent~: Ne sais-tu pas qu'aujourd'hui Yahweh va enlever ton maître au-dessus de ta tête~? Et il répondit~: Je le sais aussi~; taisez-vous~!
\VS{6}Elie lui dit~: Elisée, je te prie demeure ici, car Yahweh m'envoie jusqu'au Jourdain. Mais Elisée répondit~: Yahweh est vivant et ton âme est vivante~! Je ne te quitterai pas~! Ainsi, ils s'en allèrent tous les deux.
\VS{7}Cinquante hommes d'entre les fils des prophètes arrivèrent et s'arrêtèrent à distance vis-à-vis d'eux, et eux deux s'arrêtèrent au bord du Jourdain.
\VS{8}Alors Elie prit son manteau, le roula et en frappa les eaux, qui se divisèrent çà et là, et ils passèrent tous deux à sec.
\VS{9}Quand ils furent passés, Elie dit à Elisée~: Demande ce que tu veux que je fasse pour toi, avant que je sois enlevé d'avec toi. Elisée répondit~: Je te prie, que j'aie, une double portion\FTNT{Le fils aîné recevait une double portion par rapport aux autres fils (De. 21:15-17).} de ton esprit~!
\VS{10}Elie lui dit~: Tu demandes une chose difficile. Mais si tu me vois pendant que je serai enlevé d'avec toi, cela te sera accordé~; mais si tu ne me vois pas, cela ne te sera pas accordé.
\VS{11}Comme ils continuaient à marcher en parlant, voici, un char de feu et des chevaux de feu les séparèrent l'un de l'autre, et Elie monta au ciel dans un tourbillon.
\TextTitle{La double portion de l'esprit d'Elie sur Elisée}
\VS{12}Elisée le regardait et criait~: Mon père~! Mon père~! Char d'Israël et sa cavalerie~! Et il ne le vit plus. Puis saisissant ses vêtements, il les déchira en deux morceaux.
\VS{13}Il releva le manteau qu'Elie avait laissé tomber. Puis il retourna et s'arrêta sur le bord du Jourdain.
\VS{14}Ensuite il prit le manteau qu'Elie avait laissé tomber et il en frappa les eaux, et dit~: Où est Yahweh, le Dieu d'Elie, Yahweh lui-même~? Lui aussi frappa les eaux qui se divisèrent en deux~; et Elisée passa.
\TextTitle{Le service d'Elisée est reconnu par les hommes}
\VS{15}Quand les fils des prophètes qui étaient à Jéricho, vis-à-vis, l'eurent vu, ils dirent~: L'esprit d'Elie repose sur Elisée~! Ils vinrent à sa rencontre et se prosternèrent contre terre devant lui.
\VS{16}Ils lui dirent~: Voici, il y a parmi tes serviteurs cinquante hommes vaillants~; veux-tu qu'ils aillent chercher ton maître, de peur que l'Esprit de Yahweh ne l'ait enlevé et ne l'ait jeté sur quelque montagne ou dans quelque vallée~? Elisée répondit~: Ne les envoyez pas.
\VS{17}Mais ils le pressèrent tant par leurs paroles, qu'il en était embarrassé. Il leur dit donc~: Envoyez-les. Ils envoyèrent cinquante hommes, qui pendant trois jours cherchèrent Elie, mais ils ne le trouvèrent point.
\VS{18}Puis ils retournèrent vers Elisée, qui était à Jéricho, et il leur dit~: Ne vous avais-je pas dit~: N'y allez pas~?
\VS{19}Les gens de la ville dirent à Elisée~: Voici, le séjour dans cette ville est bon, comme mon seigneur le voit~; mais les eaux sont mauvaises et le pays est stérile.
\VS{20}Il dit~: Apportez-moi un vase neuf et mettez-y du sel. Et ils le lui apportèrent.
\VS{21}Puis il alla vers la source des eaux, et il y jeta le sel, et dit~: Ainsi parle Yahweh~: J'assainis ces eaux~; elles ne causeront plus ni mort ni stérilité.
\VS{22}Les eaux furent assainies, jusqu'à ce jour, selon la parole qu'Elisée avait prononcée.
\TextTitle{Jugement des moqueurs}
\VS{23}Elisée monta de là à Béthel~; et comme il montait par le chemin, des petits garçons sortirent de la ville et se moquèrent de lui. Ils lui disaient~: Monte chauve~! Monte chauve~!
\VS{24}Il se retourna pour les regarder, et il les maudit au nom de Yahweh. Alors deux ours sortirent de la forêt et déchirèrent quarante-deux de ces enfants.
\VS{25}De là il alla sur la montagne de Carmel, d'où il retourna à Samarie.
\Chap{3}
\TextTitle{Joram règne sur Israël}
\VerseOne{}La dix-huitième année de Josaphat, roi de Juda, Joram, fils d'Achab, régna sur Israël à Samarie. Il régna douze ans.
\VS{2}Il fit ce qui est mal aux yeux de Yahweh, non pas toutefois comme son père et sa mère, car il ôta la statue de Baal que son père avait faite~;
\VS{3}mais il s'attacha aux péchés de Jéroboam, fils de Nebath, qui avait fait pécher Israël et il ne s'en détourna point.
\TextTitle{Rébellion de Moab~; Israël et Juda s'allient pour combattre}
\VS{4}Or Méscha, roi de Moab, possédait des troupeaux, et il payait au roi d'Israël un tribut de cent mille agneaux et cent mille béliers avec leur laine.
\VS{5}Mais aussitôt qu'Achab mourut, le roi de Moab se révolta contre le roi d'Israël.
\VS{6}C'est pourquoi le roi Joram sortit ce jour-là de Samarie et passa en revue tout Israël.
\VS{7}Il se mit en marche et fit dire à Josaphat, roi de Juda~: Le roi de Moab s'est rebellé contre moi~; veux-tu venir avec moi faire la guerre à Moab~? Josaphat répondit~: Je monterai, moi comme toi, mon peuple comme ton peuple, mes chevaux comme tes chevaux.
\VS{8}Ensuite il dit~: Par quel chemin monterons-nous~? Joram répondit~: Par le chemin du désert d'Edom.
\TextTitle{Les rois d'Israël, de Juda et d'Edom en marche~; ils consultent Elisée}
\VS{9}Ainsi, le roi d'Israël, le roi de Juda et le roi d'Edom, partirent~; ils firent un détour, et après une marche de sept jours, ils manquèrent d'eau pour l'armée et pour les bêtes qui la suivaient.
\VS{10}Alors le roi d'Israël dit~: Hélas~! Yahweh a appelé ces trois rois pour les livrer entre les mains de Moab.
\VS{11}Et Josaphat dit~: N'y a-t-il ici aucun prophète de Yahweh, par qui nous puissions consulter Yahweh~? Et un des serviteurs du roi d'Israël répondit, et dit~: Il y a ici Elisée, fils de Schaphath, qui versait de l'eau sur les mains d'Elie.
\VS{12}Alors Josaphat dit~: La parole de Yahweh est avec lui. Le roi d'Israël, Josaphat et le roi d'Edom descendirent vers lui.
\VS{13}Mais Elisée dit au roi d'Israël~: Qu'y a-t-il entre moi et toi~? Va-t'en vers les prophètes de ton père et vers les prophètes de ta mère. Et le roi d'Israël lui répondit~: Non~! Car Yahweh a appelé ces trois rois pour les livrer entre les mains de Moab.
\VS{14}Elisée dit~: Yahweh des armées, devant lequel je me tiens, est vivant~! Si je n'avais de la considération pour Josaphat, roi de Juda, je ne ferais aucune attention à toi et je ne te regarderais même pas.
\VS{15}Mais maintenant, amenez-moi un joueur d'instruments à cordes. Et comme le joueur jouait des instruments à cordes, la main de Yahweh fut sur Elisée.
\TextTitle{Prophétie sur la défaite de Moab}
\VS{16}Et il dit~: Ainsi parle Yahweh~: Faites des tranchées dans toute cette vallée.
\VS{17}Car ainsi parle Yahweh~: Vous ne verrez ni vent, ni pluie, et néanmoins cette vallée sera remplie d'eaux, et vous boirez, vous et vos bêtes.
\VS{18}Mais cela est peu de chose aux yeux de Yahweh. Il livrera Moab entre vos mains~;
\VS{19}Vous frapperez toutes les villes fortes et toutes les villes d'élite, vous abattrez tous les bons arbres, vous boucherez toutes les sources d'eau et vous ruinerez avec des pierres tous les meilleurs champs.
\VS{20}Il arriva donc au matin, environ à l'heure de l'offrande, que l'eau arriva du chemin d'Edom, en sorte que ce pays fut rempli d'eau.
\VS{21}Cependant, tous les Moabites ayant appris que ces rois étaient montés pour leur faire la guerre s'étaient assemblés. On convoqua tous ceux qui étaient en âge de porter les armes, et même au-dessus, et ils se tinrent sur la frontière.
\VS{22}Et le lendemain, ils se levèrent de bon matin, et comme le soleil se levait sur les eaux, les Moabites virent en face d'eux les eaux rouges comme du sang.
\VS{23}Ils dirent~: C'est du sang~! Certainement, ces rois-là se sont entretués, et chacun a frappé son compagnon~; maintenant, Moabites, au butin~!
\VS{24}Ainsi ils marchèrent contre le camp d'Israël. Mais Israël se leva et frappa Moab, qui prit la fuite devant eux. Puis ils pénétrèrent dans le pays et frappèrent Moab.
\VS{25}Ils détruisirent les villes, et chacun jetait des pierres dans les meilleurs champs, de sorte qu'ils les en remplirent, ils bouchèrent toutes les sources d'eaux et abattirent tous les bons arbres~; et les frondeurs entourèrent et frappèrent Kir-Haréseth, dont on ne laissa que les pierres.
\VS{26}Le roi de Moab, voyant qu'il n'était pas le plus fort dans la bataille, prit avec lui sept cents hommes tirant l'épée pour se frayer un passage jusqu'au roi d'Edom~; mais ils ne purent pas.
\VS{27}Alors il prit son fils premier-né, qui devait régner à sa place, et l'offrit en holocauste sur la muraille. Et il y eut une grande indignation en Israël~; ainsi ils se retirèrent du roi de Moab et retournèrent dans leur pays.
\Chap{4}
\TextTitle{Miracle~: Le vase d'huile de la veuve}
\VerseOne{}Or une femme d'un des fils des prophètes cria à Elisée, en disant~: Ton serviteur mon mari est mort, et tu sais que ton serviteur craignait Yahweh~; or son créancier est venu pour prendre mes deux enfants, afin qu'ils soient ses esclaves.
\VS{2}Elisée lui répondit~: Que puis-je faire pour toi~? Dis-moi ce que tu as à la maison. Et elle dit~: Ta servante n'a rien dans toute la maison qu'un vase d'huile.
\VS{3}Alors il lui dit~: Va, demande des vases dans la rue à tous tes voisins, des vases vides, et n'en demande pas un petit nombre.
\VS{4}Puis rentre et ferme la porte sur toi et sur tes enfants, et verse dans tous ces vases, et tu mettras de côté ceux qui seront pleins.
\VS{5}Alors elle le quitta. Ayant fermé la porte sur elle et sur ses enfants~; ils lui présentaient les vases, et elle versait.
\VS{6}Lorsqu'elle eut rempli les vases, elle dit à son fils~: Présente-moi encore un vase. Mais il répondit~: Il n'y a plus de vase. Et l'huile s'arrêta.
\VS{7}Elle alla le raconter à l'homme de Dieu, qui lui dit~: Va, vends l'huile, et paye ta dette~; et vous vivrez, toi et tes fils, de ce qui restera.
\TextTitle{Yahweh se souvient de la Sunamite}
\VS{8}Et il arriva un jour qu'Elisée passait par Sunem, où il y avait une femme importante~; elle le retint avec grande instance à manger du pain chez elle. Et toutes les fois qu'il passait, il s'y retirait pour manger du pain.
\VS{9}Elle dit à son mari~: Voilà, je sais que cet homme qui passe souvent chez nous est un saint homme de Dieu.
\VS{10}Faisons-lui, je te prie, une petite chambre haute avec des murs, et mettons-y pour lui un lit, une table, un siège et un chandelier, afin que quand il viendra chez nous, il s'y retire.
\VS{11}Un jour, Elisée étant revenu à Sunem, il se retira dans cette chambre haute et s'y coucha.
\VS{12}Puis il dit à Guéhazi, son serviteur~: Appelle cette Sunamite. Guéhazi l'appela, et elle se présenta devant lui.
\VS{13}Et Elisée dit à Guéhazi~: Dis maintenant à cette femme~: Voici, tu nous as montré tout cet empressement~; que pourrait-on faire pour toi~? Faut-il parler pour toi au roi ou au chef de l'armée~? Elle répondit~: J'habite au milieu de mon peuple.
\VS{14}Et il dit~: Que faudrait-il faire pour elle~? Guéhazi répondit~: Mais elle n'a point de fils et son mari est vieux.
\VS{15}Et il dit~: Appelle-la. Guéhazi l'appela, et elle se présenta à la porte.
\VS{16}Elisée lui dit~: L'année prochaine, à cette même époque, tu embrasseras un fils. Elle répondit~: Mon seigneur, homme de Dieu, ne trompe pas, ne trompe pas ta servante~!
\VS{17}Cette femme devint enceinte et enfanta un fils un an après, à la même époque, comme Elisée lui avait dit.
\TextTitle{Foi de la Sunamite, résurrection de son fils}
\VS{18}L'enfant grandit. Il sortit un jour pour aller trouver son père vers les moissonneurs.
\VS{19}Et il dit à son père~: Ma tête~! Ma tête~! Et le père dit au serviteur~: Porte-le à sa mère.
\VS{20}Il le porta donc et l'amena à sa mère. Et l'enfant resta sur les genoux de sa mère jusqu'à midi, puis il mourut.
\VS{21}Elle monta et le coucha sur le lit de l'homme de Dieu~; et ayant fermé la porte sur lui, elle sortit.
\VS{22}Elle appela son mari et dit~: Je te prie envoie-moi un des serviteurs et une ânesse~; j'irai chez l'homme de Dieu et je reviendrai.
\VS{23}Et il dit~: Pourquoi vas-tu vers lui aujourd'hui~? Ce n'est point la nouvelle lune ni le sabbat. Elle répondit~: Tout va bien~!
\VS{24}Elle fit donc seller l'ânesse, et dit à son serviteur~: Conduis-moi et ne m'arrête pas en route sans que je te le dise.
\VS{25}Ainsi elle s'en alla et se rendit vers l'homme de Dieu sur la montagne de Carmel. L'homme de Dieu, l'ayant aperçue, dit à Guéhazi son serviteur~: Voilà la Sunamite~!
\VS{26}Va, cours à sa rencontre et dis-lui~: Te portes-tu bien~? Ton mari se porte-t-il bien~? L'enfant se porte-t-il bien~? Et elle répondit~: Nous nous portons bien.
\VS{27}Dès qu'elle fut arrivée auprès de l'homme de Dieu sur la montagne, elle embrassa ses pieds. Guéhazi s'approcha pour la repousser, mais l'homme de Dieu lui dit~: Laisse-la, car son âme est dans l'amertume, et Yahweh me l'a caché, et ne me l'a pas révélé.
\VS{28}Alors elle dit~: Ai-je demandé un fils à mon seigneur~? N'ai-je pas dit~: Ne me trompe pas~?
\VS{29}Et Elisée dit à Guéhazi~: Ceins tes reins, prends mon bâton dans ta main, et pars. Si tu rencontres quelqu'un, ne le salue pas~; et si quelqu'un te salue, ne lui réponds pas. Tu mettras mon bâton sur le visage de l'enfant.
\VS{30}Mais la mère de l'enfant dit~: Yahweh est vivant, et ton âme est vivante~! Je ne te quitterai point. Il se leva donc et la suivit.
\VS{31}Or Guéhazi les avait devancés et il avait mis le bâton sur le visage de l'enfant~; mais il n'y eut ni voix ni signe d'attention. Guéhazi retourna à la rencontre d'Elisée et l'en informa en disant~: L'enfant ne s'est pas réveillé.
\VS{32}Lorsqu'Elisée entra dans la maison, l'enfant, mort, était couché sur son lit.
\VS{33}Il ferma la porte sur eux deux et pria Yahweh.
\VS{34}Puis, il monta et se coucha sur l'enfant~; il mit sa bouche sur la bouche de l'enfant, ses yeux sur ses yeux, ses mains sur ses mains, et il s'étendit sur lui. La chair de l'enfant se réchauffa.
\VS{35}Puis il s'éloigna et marcha dans la maison, tantôt dans un lieu, tantôt dans un autre, et il remonta et s'étendit encore sur lui. L'enfant éternua sept fois et ouvrit ses yeux.
\VS{36}Alors, Elisée appela Guéhazi, et lui dit~: Appelle cette Sunamite. Guéhazi l'appela, et elle vint vers Elisée qui lui dit~: Prends ton fils~!
\VS{37}Elle se jeta à ses pieds et se prosterna contre terre. Puis elle prit son fils et sortit.
\TextTitle{Les coloquintes sauvages}
\VS{38}Après cela, Elisée revint à Guilgal. Or il y avait une famine\FTNT{Par le passé, Israël a connu plusieurs famines, dont celle relatée en 2 R. 4:38-41.. Dans ce passage, l'un des fils des prophètes trouva une vigne sauvage dans un champ et y cueillit des coloquintes sauvages. Il les ajouta au potage qui mijotait dans un pot, ne sachant pas que c'était du poison. Le pot est l'image des églises de Laodicée dans lesquelles il y a un mélange mortel de fausses doctrines et de préceptes mondains qui viennent altérer la vérité de la parole de Dieu. Ce mélange impur est absorbé par des millions de personnes ignorantes à travers le monde. Celles-ci se rendent compte qu'elles ont été empoisonnées spirituellement, et une fois le mélange ingéré, elles constatent les effets pervers et dévastateurs souvent tardivement. Le champ tout comme la vigne sauvage, selon Mt. 13:38 et Ro. 11:17, symbolise le monde. Il est par ailleurs intéressant de noter que le mot herbe, «~owrah~» en hébreu, signifie aussi lumière (Ps. 139:12). Cette histoire n'est pas sans nous rappeler le feu étranger introduit par les fils d'Aaron dans le tabernacle, et ce, malgré l'interdiction formelle de Yahweh (Ex. 30:9~; Lé. 10:1-5). C'est exactement ce qui se passe de nos jours. Les églises importent de plus en plus en leur sein la lumière luciférienne du monde (musique, marketing, philosophie, etc.). Beaucoup de pasteurs et de musiciens cherchent malheureusement leur inspiration dans le monde à cause de la famine qui sévit dans les églises. Ce feu étranger représente la plupart des doctrines et pratiques promues par l'église de Laodicée.} dans le pays, et les fils des prophètes étaient assis devant lui~; et il dit à son serviteur~: Mets le grand pot et fais cuire du potage pour les fils des prophètes.
\VS{39}Mais quelqu'un étant sorti dans les champs pour cueillir des herbes, trouva de la vigne sauvage, et cueillit des coloquintes sauvages plein sa robe, et étant revenu, il les coupa en morceaux dans le pot où était le potage, car on ne savait pas ce que c'était.
\VS{40}Et on servit à manger de ce potage à quelques-uns~; mais aussitôt qu'ils eurent mangé de ce potage, ils s'écrièrent et dirent~: Homme de Dieu, la mort est dans le pot~! Et ils ne purent en manger.
\VS{41}Et il dit~: Apportez-moi de la farine~; et il en jeta dans le pot, puis il dit~: Qu'on en verse à ce peuple, afin qu'il mange~; et il n'y avait plus rien de mauvais dans le pot.
\TextTitle{Multiplication de pains}
\VS{42}Un homme venant de Baal-Schalischa apporta à l'homme de Dieu du pain des prémices, à savoir vingt pains d'orge et des épis nouveaux. Elisée dit~: Donne cela à ces gens, et qu'ils mangent.
\VS{43}Son serviteur répondit~: Comment pourrais-je en donner à cent hommes~? Mais Elisée lui répondit~: Donne-les à ces gens, et qu'ils mangent~; car ainsi parle Yahweh~: Ils mangeront et il en restera encore.
\VS{44}Il mit donc les pains devant eux. Ils mangèrent et en eurent de reste, selon la parole de Yahweh.
\Chap{5}
\TextTitle{Guérison miraculeuse de Naaman}
\VerseOne{}Or Naaman, chef de l'armée du roi de Syrie, était un homme puissant et très considéré aux yeux de son maître~; car c'était par lui que Yahweh avait délivré les Syriens. Mais cet homme fort et vaillant était lépreux.
\VS{2}Et les Syriens étaient sortis par troupes, et ils avaient emmené prisonnière une petite fille du pays d'Israël, qui était au service de la femme de Naaman.
\VS{3}Elle dit à sa maîtresse~: Oh~! Si mon seigneur se présentait devant le prophète qui est à Samarie, il le guérirait de sa lèpre~!
\VS{4}Naaman le rapporta à son maître, en disant~: La fille qui est du pays d'Israël a dit telle et telle chose.
\VS{5}Et le roi de Syrie dit à Naaman~: Va, rends-toi à Samarie et j'enverrai une lettre au roi d'Israël. Naaman donc s'en alla et prit avec lui dix talents d'argent et six mille pièces d'or, et dix vêtements de rechange.
\VS{6}Il porta au roi d'Israël la lettre, où il était dit~: Dès que cette lettre te sera parvenue, sache que je t'ai envoyé Naaman, mon serviteur, afin que tu le guérisses de sa lèpre.
\VS{7}Et dès que le roi d'Israël eut lu la lettre, il déchira ses vêtements et dit~: Suis-je Dieu pour faire mourir et pour rendre la vie, pour qu'il s'adresse à moi afin que je guérisse un homme de sa lèpre~? Voyez et comprenez qu'il cherche certainement une occasion de dispute avec moi.
\VS{8}Et il arriva qu'aussitôt qu'Elisée, homme de Dieu, apprit que le roi d'Israël avait déchiré ses vêtements, il envoya dire au roi~: Pourquoi as-tu déchiré tes vêtements~? Laisse-le venir vers moi et il saura qu'il y a un prophète en Israël.
\VS{9}Naaman vint avec ses chevaux et son char, et il s'arrêta à la porte de la maison d'Elisée.
\VS{10}Elisée envoya un messager vers lui, pour lui dire~: Va, et lave-toi sept fois dans le Jourdain, et ta chair redeviendra saine, et tu seras pur.
\VS{11}Mais Naaman se mit dans une grande colère, et s'en alla en disant~: Voilà, je me disais~: Il sortira et viendra vers moi, il se présentera lui-même, il invoquera le Nom de Yahweh, son Dieu, puis, il agitera sa main sur la plaie, et guérira le lépreux.
\VS{12}Les fleuves de Damas, l'Abana et le Parpar ne sont-ils pas meilleurs que toutes les eaux d'Israël~? Ne pourrais-je pas m'y laver et devenir pur~? Ainsi donc, il s'en retourna et s'en alla furieux.
\VS{13}Mais ses serviteurs s'approchèrent et lui parlèrent en disant~: Mon père, si le prophète t'avait imposé quelque chose de difficile, ne l'aurais-tu pas fait~? Combien plus dois-tu faire ce qu'il t'a dit~: Lave-toi, et tu deviendras pur~!
\VS{14}Alors il descendit et se plongea sept fois dans le Jourdain, selon la parole de l'homme de Dieu~; et sa chair redevint comme la chair d'un petit enfant~; et il fut pur.
\VS{15}Il retourna vers l'homme de Dieu, lui et tout son camp, et il vint se présenter devant lui et dit~: Voici, maintenant je sais qu'il n'y a point d'autre Dieu sur toute la terre, si ce n'est en Israël. Maintenant donc, je te prie, accepte ce présent de ton serviteur.
\VS{16}Elisée répondit~: Yahweh, devant lequel je me tiens, est vivant~! Je ne l'accepterai pas~! Naaman le pressa fort de l'accepter, mais Elisée refusa~!
\VS{17}Alors, Naaman dit~: Je te prie, permets que l'on donne de la terre à ton serviteur, une charge de deux mulets~; car ton serviteur ne fera plus d'holocauste ni de sacrifice à d'autres dieux, mais seulement à Yahweh.
\VS{18}Voici toutefois, que Yahweh pardonne ceci à ton serviteur. Quand mon maître entre dans la maison de Rimmon pour s'y prosterner et qu'il s'appuie sur ma main, je me prosterne aussi dans la maison de Rimmon~: Que Yahweh me pardonne, quand je me prosternerai dans la maison de Rimmon.
\TextTitle{Convoitise et mensonge de Guéhazi~; jugement de Dieu}
\VS{19}Elisée lui dit~: Va en paix. Lorsque Naaman eut quitté Elisée et qu'il fut à une certaine distance,
\VS{20}Guéhazi\FTNT{Guéhazi, dont le nom hébreu signifie «~vallée de la vision~».}, le serviteur d'Elisée, homme de Dieu, se dit en lui-même~: Voici, mon maître a ménagé Naaman, ce Syrien, et n'a pas accepté de sa main ce qu'il avait apporté~; Yahweh est vivant~! Je vais courir après lui et j'en obtiendrai quelque chose.
\VS{21}Et Guéhazi courut après Naaman. Naaman, le voyant courir après lui, descendit de son char pour aller à sa rencontre. Il dit~: Tout va bien~?
\VS{22}Guéhazi répondit~: Tout va bien. Mon maître m'envoie te dire~: Voici, il vient d'arriver chez moi deux jeunes hommes de la montagne d'Ephraïm, d'entre les fils des prophètes. Je te prie donne-leur un talent d'argent et deux vêtements de rechange.
\VS{23}Et Naaman dit~: Consens à prendre deux talents. Il insista, puis il serra deux talents d'argent dans deux sacs avec deux vêtements de rechange et les fit porter devant Guéhazi par deux de ses serviteurs.
\VS{24}Et quand il fut arrivé dans un lieu secret, il les prit de leurs mains, et les déposa dans la maison, et il renvoya ces gens qui s'en allèrent.
\VS{25}Puis il entra et se présenta devant son maître. Elisée lui dit~: D'où viens-tu, Guéhazi~? Et il répondit~: Ton serviteur n'est allé nulle part.
\VS{26}Mais Elisée lui dit~: Mon cœur n'est-il pas allé là, lorsque cet homme a quitté son char pour venir à ta rencontre~? Est-ce le temps de prendre de l'argent, de prendre des vêtements, des oliviers, des vignes, du menu et du gros bétail, des serviteurs et des servantes~?
\VS{27}C'est pourquoi la lèpre de Naaman s'attachera à toi et à ta postérité à jamais. Et Guéhazi sortit de la présence d'Elisée avec une lèpre comme de la neige.
\Chap{6}
\TextTitle{Miracle du fer de hache}
\VerseOne{}Les fils des prophètes dirent à Elisée~: Voici, le lieu où nous sommes assis devant toi est trop étroit pour nous.
\VS{2}Allons jusqu'au Jourdain~; nous prendrons là chacun une poutre et nous y ferons un lieu d'habitation. Elisée répondit~: Allez~!
\VS{3}Et l'un d'eux dit~: Veuille, je te prie, venir avec tes serviteurs. Il répondit~: J'irai.
\VS{4}Il partit donc avec eux. Arrivés au Jourdain, ils coupèrent du bois.
\VS{5}Mais il arriva que comme l'un d'eux abattait une poutre, le fer de sa cognée tomba dans l'eau. Il s'écria et dit~: Ah~! Mon seigneur~! Je l'avais emprunté~!
\VS{6}L'homme de Dieu dit~: Où est-il tombé~? Et il lui montra l'endroit. Alors Elisée coupa un morceau de bois, le jeta au même endroit, et fit surnager le fer.
\VS{7}Et il dit~: Retire-le~! Et cet homme étendit sa main et le prit.
\TextTitle{Yahweh révèle à Elisée les plans militaires des Syriens}
\VS{8}Le roi de Syrie était en guerre avec Israël, et, dans un conseil qu'il tint avec ses serviteurs, il dit~: Mon camp sera dans un tel lieu.
\VS{9}L'homme de Dieu envoya dire au roi d'Israël~: Garde-toi de passer dans ce lieu, car les Syriens y descendent.
\VS{10}Et le roi d'Israël envoya des gens, pour s'y tenir en observation, vers le lieu que l'homme de Dieu lui avait mentionné et signalé. Et il y était sur ses gardes. Et cela n'arriva pas seulement une fois ni deux fois.
\VS{11}Le roi de Syrie en eut le cœur troublé~; et il appela ses serviteurs et leur dit~: Ne voulez-vous pas me déclarer lequel de vous est pour le roi d'Israël~?
\VS{12}Et l'un de ses serviteurs répondit~: Personne~! Ô roi, mon seigneur~! Mais Elisée, le prophète qui est en Israël, révèle au roi d'Israël les paroles même que tu déclares dans ta chambre à coucher.
\VS{13}Et il dit~: Allez et voyez où il est, et je le ferai prendre. On vint lui dire~: Voici, il est à Dothan.
\VS{14}Il envoya là des chevaux et des chars, et une grande armée, qui arrivèrent de nuit, et qui entourèrent la ville.
\TextTitle{L'armée de Yahweh plus grande que celle des Syriens}
\VS{15}Le serviteur de l'homme de Dieu se leva de grand matin et sortit~; et voici, une armée, entourait la ville, avec des chevaux et des chars. Le serviteur dit à l'homme de Dieu~: Ah~! Mon seigneur, comment ferons-nous~?
\VS{16}Il lui répondit~: Ne crains point, car ceux qui sont avec nous sont en plus grand nombre que ceux qui sont avec eux.
\VS{17}Elisée pria et dit~: Je te prie, ô Yahweh~! Ouvre ses yeux, afin qu'il voie. Et Yahweh ouvrit les yeux du serviteur et il vit. Et voici la montagne était pleine de chevaux et de chars de feu autour d'Elisée.
\TextTitle{Dieu aveugle les Syriens à la prière d'Elisée}
\VS{18}Les Syriens descendirent vers Elisée. Il adressa alors cette prière à Yahweh~: Je te prie, frappe ces gens d'aveuglement~! Et Dieu les frappa d'aveuglement, selon la parole d'Elisée.
\VS{19}Elisée leur dit~: Ce n'est pas ici le chemin, et ce n'est pas ici la ville~; suivez-moi et je vous conduirai vers l'homme que vous cherchez. Et il les conduisit à Samarie.
\VS{20}Et il arriva qu'aussitôt qu'ils furent entrés dans Samarie, Elisée dit~: Ô Yahweh ouvre leurs yeux afin qu'ils voient. Et Yahweh ouvrit leurs yeux et ils virent qu'ils étaient au milieu de Samarie.
\VS{21}Et dès que le roi d'Israël le vit, il dit à Elisée~: Frapperai-je, frapperai-je, mon père~?
\VS{22}Et Elisée répondit~: Tu ne frapperas point~; frapperais-tu de ton épée et de ton arc ceux que tu as fait prisonniers~? Sers-leur du pain et de l'eau afin qu'ils mangent et boivent~; et après cela, qu'ils s'en aillent vers leur maître.
\VS{23}Le roi d'Israël leur fit servir un grand repas et ils mangèrent et burent~; puis il les renvoya et ils s'en allèrent vers leur maître. Alors, les armées de Syrie ne revinrent plus au pays d'Israël.
\TextTitle{Siège des Syriens et famine en Samarie}
\VS{24}Et il arriva après cela que Ben-Hadad, roi de Syrie, rassembla toute son armée, monta et assiégea Samarie.
\VS{25}Il y eut une grande famine\FTNT{Cette histoire est riche en enseignements pour notre génération. Le siège de la Samarie par les étrangers, la famine qui frappait les Hébreux, le cannibalisme de certaines femmes, la cherté des produits alimentaires, la consommation d'excréments d'animaux à cause de la famine, sont des conséquences du péché. Aujourd'hui, beaucoup d'églises sont assiégées par les choses du monde, les démons, les fausses doctrines, etc.} dans Samarie~; ils l'assiégèrent tellement qu'une tête d'âne se vendait quatre-vingts pièces d'argent, et le quart d'un kab de fiente de pigeon cinq pièces d'argent.
\VS{26}Et comme le roi d'Israël passait sur la muraille, une femme lui cria~: Ô roi, mon seigneur~! Sauve-moi.
\VS{27}Il répondit~: Si Yahweh ne te sauve pas, comment pourrais-je te sauver~? Serait-ce avec le produit de l'aire ou de la cuve~?
\VS{28}Il lui dit encore~: Qu'as-tu~? Elle répondit~: Cette femme-là m'a dit~: Donne ton fils, et mangeons-le aujourd'hui, et nous mangerons mon fils demain\FTNT{Lé. 26:29~; De. 28:53-57.}.
\VS{29}Ainsi nous avons fait bouillir mon fils et l'avons mangé. Et le jour suivant, je lui ai dit~: Donne ton fils et nous le mangerons. Mais elle a caché son fils.
\VS{30}Dès que le roi entendit les paroles de cette femme, il déchira ses vêtements et passa sur la muraille. Le peuple vit qu'il avait en dessous un sac sur son corps.
\VS{31}C'est pourquoi le roi dit~: Que Dieu me traite dans toute sa rigueur, si aujourd'hui la tête d'Elisée, fils de Schaphath, reste sur lui.
\VS{32}Or Elisée était assis dans sa maison, et les anciens étaient assis avec lui. Le roi envoya un homme devant lui. Mais avant que le messager soit arrivé, Elisée dit aux anciens~: Ne voyez-vous pas que le fils de ce meurtrier envoie quelqu'un pour m'ôter la tête~? Lorsque le messager viendra, fermez la porte et repoussez-le avec la porte. N'entendez-vous pas le bruit des pas de son maître derrière lui~?
\VS{33}Et comme il parlait encore avec eux, voici le messager descendit vers lui et dit~: Voici, ce mal vient de Yahweh~; qu'ai-je à espérer encore de Yahweh~?
\Chap{7}
\TextTitle{Prophétie d'Elisée~; les lépreux dans le camp des Syriens}
\VerseOne{}Alors Elisée dit~: Ecoutez la parole de Yahweh~! Ainsi parle Yahweh~: Demain, à cette heure, on aura une mesure de fleur de farine pour un sicle, et deux mesures d'orge pour un sicle, à la porte de Samarie.
\VS{2}Mais l'officier sur la main duquel le roi s'appuyait répondit à l'homme de Dieu et dit~: Quand Yahweh ferait des fenêtres au ciel, cela arriverait-il~? Et Elisée dit~: Tu le verras de tes yeux, mais tu n'en mangeras pas.
\VS{3}Or il y avait à l'entrée de la porte quatre hommes lépreux\FTNT{Dieu s'est servi de ces quatre lépreux comme messagers de bonnes nouvelles. Le Seigneur utilise souvent les personnes rejetées et déconsidérées (1 Co. 1:26-31).}, et ils se dirent l'un à l'autre~: Pourquoi resterions-nous ici jusqu'à ce que nous mourions~?
\VS{4}Si nous pensons à entrer dans la ville, la famine est dans la ville et nous y mourrons~; et si nous restons ici, nous mourrons également. Allons-nous jeter dans le camp des Syriens~; s'ils nous laissent vivre, nous vivrons, et s'ils nous font mourir, nous mourrons.
\VS{5}Ils se levèrent donc au crépuscule pour entrer au camp des Syriens. Lorsqu'ils furent arrivés à l'extrémité du camp, voici, il n'y avait personne.
\VS{6}Car le Seigneur avait fait entendre dans le camp des Syriens un bruit de chars, et un bruit de chevaux, et un bruit d'une grande armée~; de sorte qu'ils s'étaient dit l'un à l'autre~: Voici, le roi d'Israël a payé les rois des Héthiens et les rois des Egyptiens pour venir contre nous.
\VS{7}C'est pourquoi ils s'étaient levés au crépuscule et s'étaient enfuis. Ils avaient abandonné leurs tentes, leurs chevaux, leurs ânes, et le camp tel qu'il était, et ils s'étaient enfuis pour sauver leur vie.
\VS{8}Les lépreux donc arrivèrent jusqu'à l'extrémité du camp. Ils entrèrent dans une tente, mangèrent, burent, emportèrent de l'argent, de l'or, des vêtements, et ils s'en allèrent et les cachèrent. Ils revinrent et entrèrent dans une autre tente et emportèrent de là aussi des objets, s'en allèrent et les cachèrent.
\VS{9}Alors ils se dirent l'un à l'autre~: Nous n'agissons pas bien~! Ce jour est un jour de bonnes nouvelles~; si nous gardons le silence et si nous attendons jusqu'à lumière du matin, le châtiment nous atteindra. Venez maintenant et allons informer la maison du roi.
\VS{10}Ils partirent et appelèrent les portiers de la ville, et leur racontèrent, en disant~: Nous sommes entrés dans le camp des Syriens, et voici, il n'y a personne. On n'y entend aucune voix d'homme~; il n'y a que des chevaux attachés, des ânes attachés et les tentes sont comme elles étaient.
\VS{11}Alors les portiers crièrent et transmirent ce rapport à la maison du roi.
\TextTitle{Accomplissement de la prophétie d'Elisée}
\VS{12}Le roi se leva de nuit et dit à ses serviteurs~: Je veux vous dire ce que les Syriens ont préparé contre nous. Ils savent que nous sommes affamés et ils sont sortis du camp pour se cacher dans les champs, disant~: Quand ils sortiront hors de la ville, nous les saisirons vivants et nous entrerons dans la ville.
\VS{13}L'un des serviteurs du roi répondit et dit~: Qu'on prenne cinq des chevaux qui restent encore dans la ville~; c'est presque tout ce qui est resté du grand nombre des chevaux d'Israël~; ils sont comme toute la multitude d'Israël, qui est consumée. Envoyons voir ce qui se passe.
\VS{14}Ils prirent donc deux chars avec les chevaux, et le roi envoya des messagers après l'armée des Syriens, en disant~: Allez et voyez.
\VS{15}Et ils allèrent après eux jusqu'au Jourdain~; et voici, le chemin était plein de vêtements et d'objets que les Syriens avaient jetés dans leur précipitation. Les messagers revinrent et le rapportèrent au roi.
\VS{16}Alors le peuple sortit et pilla le camp des Syriens, de sorte qu'il eut une mesure de fleur de farine pour un sicle, et deux mesures d'orge pour un sicle, selon la parole de Yahweh.
\VS{17}Le roi donna à l'officier, sur la main duquel il s'appuyait, la charge de garder la porte. Mais cet officier fut écrasé à la porte par le peuple et il en mourut selon la parole qu'avait prononcée l'homme de Dieu, quand le roi était descendu vers lui.
\VS{18}Car lorsque l'homme de Dieu avait parlé au roi, en disant~: Demain matin, à cette heure-ci, on donnera à la porte de Samarie deux mesures d'orge pour un sicle et une mesure de fleur de farine pour un sicle~;
\VS{19}cet officier avait répondu à l'homme de Dieu~: Quand Yahweh ferait des fenêtres au ciel, ce que tu dis pourrait-il arriver~? Et l'homme de Dieu avait dit~: Voici, tu le verras de tes yeux, mais tu n'en mangeras pas.
\VS{20}C'est en effet ce qui lui arriva~; car le peuple l'écrasa à la porte et il mourut.
\Chap{8}
\TextTitle{Elisée annonce une famine de sept ans}
\VerseOne{}Elisée parla à la femme dont il avait fait revivre le fils, en disant~: Lève-toi et va-t'en, toi et ta famille, et séjourne où tu pourras~; car Yahweh a appelé la famine, et même elle vient sur le pays pour sept ans.
\VS{2}La femme se leva et elle fit selon la parole de l'homme de Dieu. Elle s'en alla, elle et sa famille, et séjourna sept ans au pays des Philistins.
\TextTitle{La Sunamite retrouve ses terres}
\VS{3}Mais il arriva qu'au bout des sept ans, la femme revint du pays des Philistins, et alla implorer le roi au sujet de sa maison et de ses champs.
\VS{4}Le roi parlait à Guéhazi\FTNT{Voir 2 R. 5.}, serviteur de l'homme de Dieu, en disant~: Je te prie raconte-moi toutes les grandes choses qu'Elisée a faites.
\VS{5}Et il arriva que comme il racontait au roi comment Elisée avait rendu la vie à un mort, la femme dont Elisée avait fait revivre le fils vint implorer le roi au sujet de sa maison et de ses champs. Guéhazi dit~: Ô roi, mon seigneur, voici la femme et voici son fils, à qui Elisée a rendu la vie.
\VS{6}Alors le roi interrogea la femme, et elle lui raconta ce qui s'était passé. Le roi lui donna un eunuque, auquel il dit~: Fais restituer tout ce qui lui appartenait, même tous les revenus de ses champs, depuis le jour où elle a quitté le pays jusqu'à maintenant.
\TextTitle{Prophétie sur le règne d'Hazaël sur la Syrie}
\VS{7}Elisée se rendit à Damas. Ben-Hadad, roi de Syrie, était malade et on lui fit ce rapport~: L'homme de Dieu est venu ici.
\VS{8}Le roi dit à Hazaël~: Prends avec toi un présent et va au-devant de l'homme de Dieu, et consulte par lui Yahweh, en disant~: Guérirai-je de cette maladie~?
\VS{9}Et Hazaël s'en alla au-devant d'Elisée, ayant pris avec lui un présent, à savoir quarante chameaux chargés de tout ce qu'il y avait de meilleur à Damas. Il vint se présenter devant Elisée et dit~: Ton fils, Ben-Hadad, roi de Syrie, m'a envoyé vers toi, pour te dire~: Guérirai-je de cette maladie~?
\VS{10}Et Elisée lui répondit~: Va, dis-lui~: Tu guériras~! Tu guériras~! Toutefois, Yahweh m'a révélé qu'il mourra, qu'il mourra.
\VS{11}L'homme de Dieu arrêta son regard sur Hazaël et le fixa longtemps, puis il pleura.
\VS{12}Hazaël dit~: Pourquoi mon seigneur pleure-t-il~? Et il répondit~: Parce que je sais le mal que tu feras aux enfants d'Israël~; tu mettras le feu à leurs villes fortes, tu tueras avec l'épée leurs jeunes gens, tu écraseras leurs petits-enfants et tu fendras le ventre de leurs femmes enceintes.
\VS{13}Hazaël dit~: Mais qu'est-ce que ton serviteur, ce chien, pour faire de si grandes choses~? Et Elisée répondit~: Yahweh m'a révélé que tu seras roi de Syrie.
\VS{14}Alors Hazaël quitta Elisée et revint vers son maître, qui lui demanda~: Que t'a dit Elisée~? Et il répondit~: Il m'a dit que tu guériras~! Tu guériras~!
\VS{15}Mais le lendemain, Hazaël prit une couverture et l'ayant plongé dans l'eau, il l'étendit sur le visage de Ben-Hadad, qui mourut. Et Hazaël régna à sa place.
\TextTitle{Joram règne sur Juda\FTNTT{2 Ch. 21:1-7.}}
\VS{16}La cinquième année de Joram, fils d'Achab, roi d'Israël, Josaphat était encore roi de Juda et Joram, fils de Josaphat, roi de Juda, commença à régner sur Juda.
\VS{17}Il était âgé de trente-deux ans lorsqu'il commença à régner. Il régna huit ans à Jérusalem.
\VS{18}Il marcha dans la voie des rois d'Israël comme avait fait la maison d'Achab, car il avait pour femme la fille d'Achab\FTNT{Le mariage de Joram, fils de Josaphat, avec Athalie, fille d'Achab, était une grande erreur. Cette union qui était contractée dans le but de favoriser la paix entre les deux royaumes, entraîna le déclin de Juda~; or Dieu est contre les alliances contre nature. Voir Es. 30-31.}, et il fit ce qui est mal aux yeux de Yahweh.
\VS{19}Mais Yahweh ne voulut point détruire Juda, par amour pour David, son serviteur, selon la promesse qu'il lui avait faite de lui donner toujours une lampe parmi ses fils.
\TextTitle{Révoltes contre l'autorité de Juda}
\VS{20}De son temps, Edom se révolta contre l'autorité de Juda et se donna un roi.
\VS{21}Joram passa à Tsaïr, avec tous ses chars~; il se leva de nuit, et frappa les Edomites qui l'entouraient, et les chefs des chars, mais le peuple s'enfuit dans ses tentes.
\VS{22}Néanmoins, les Edomites ont été rebelles à Juda jusqu'à ce jour. En ce même temps, Libna aussi se révolta.
\TextTitle{Achazia règne sur Juda\FTNTT{2 Ch. 21:18-22:4.}}
\VS{23}Le reste des actions de Joram et tout ce qu'il a fait, cela n'est-il pas écrit dans le livre des Chroniques des rois de Juda~?
\VS{24}Joram se coucha avec ses pères et il fut enterré avec ses pères dans la cité de David. Et Achazia, son fils, régna à sa place.
\VS{25}La douzième année de Joram, fils d'Achab, roi d'Israël, Achazia, fils de Joram, roi de Juda, commença à régner.
\VS{26}Achazia était âgé de vingt-deux ans lorsqu'il commença à régner. Il régna un an à Jérusalem. Sa mère s'appelait Athalie, fille d'Omri, roi d'Israël.
\VS{27}Il marcha dans la voie de la maison d'Achab et il fit ce qui est mal aux yeux de Yahweh, comme avait fait la maison d'Achab, car il était gendre de la maison d'Achab.
\VS{28}Il alla avec Joram, fils d'Achab, à la guerre contre Hazaël, roi de Syrie, à Ramoth en Galaad. Et les Syriens blessèrent Joram.
\VS{29}Le roi Joram s'en retourna pour se faire guérir à Jizreel des blessures que les Syriens lui avaient faites à Rama, lorsqu'il se battait contre Hazaël, roi de Syrie. Achazia, fils de Joram, roi de Juda, descendit pour voir Joram, fils d'Achab, à Jizreel, parce qu'il était malade.
\Chap{9}
\TextTitle{Jéhu oint roi d'Israël}
\VerseOne{}Alors Elisée, le prophète, appela l'un des fils des prophètes et lui dit~: Ceins tes reins, prends cette fiole d'huile dans ta main, et va à Ramoth en Galaad.
\VS{2}Quand tu y seras entré, vois Jéhu, fils de Josaphat, fils de Nimschi. Tu iras le faire lever du milieu de ses frères et tu le conduiras dans une chambre secrète.
\VS{3}Tu prendras la fiole d'huile, tu la verseras sur sa tête et tu diras~: Ainsi parle Yahweh~: Je t'ai oint pour être roi sur Israël. Après quoi tu ouvriras la porte, tu t'enfuiras et tu ne t'arrêteras pas.
\VS{4}Le jeune homme, serviteur du prophète, s'en alla à Ramoth en Galaad.
\VS{5}Quand il arriva, voici, les chefs de l'armée étaient là assis. Il dit~: Chef, j'ai à te parler. Et Jéhu répondit~: Auquel de nous parles-tu~? Et il répondit~: A toi, chef.
\VS{6}Alors Jéhu se leva, et entra dans la maison, et le jeune homme répandit l'huile sur la tête, et lui dit~: Ainsi parle Yahweh, le Dieu d'Israël~: Je t'ai oint pour être roi sur Israël, le peuple de Yahweh.
\VS{7}Tu frapperas la maison d'Achab, ton maître, et je vengerai sur Jézabel\FTNT{1 R. 16:31~; 1 R. 17,18,19.} le sang de mes serviteurs les prophètes, et le sang de tous les serviteurs de Yahweh.
\VS{8}Et toute la maison d'Achab périra, et je retrancherai quiconque appartient à Achab, celui qui est esclave et celui qui est libre en Israël.
\VS{9}Je rendrai la maison d'Achab semblable à la maison de Jéroboam, fils de Nebath, et à la maison de Baescha, fils d'Achija.
\VS{10}Les chiens mangeront Jézabel dans le champ de Jizreel, et il n'y aura personne pour l'enterrer. Puis il ouvrit la porte et s'enfuit.
\VS{11}Jéhu sortit pour rejoindre les serviteurs de son maître et on lui dit~: Tout va bien~? Pourquoi ce fou est-il venu vers toi~? Jéhu leur répondit~: Vous connaissez l'homme et ses rêveries.
\VS{12}Mais ils répliquèrent~: Mensonge~! Réponds-nous donc. Et il dit~: Il m'a parlé de telle et telle manière, disant~: Ainsi parle Yahweh, je t'ai oint pour être roi sur Israël.
\VS{13}Alors ils se hâtèrent, et prirent chacun leurs vêtements, et les mirent sous lui au plus haut des degrés. Ils sonnèrent du shofar et dirent~: Jéhu a été fait roi~!
\TextTitle{Mort de Joram}
\VS{14}Ainsi Jéhu, fils de Josaphat, fils de Nimschi, forma une conspiration contre Joram. Or Joram et tout Israël défendaient Ramoth en Galaad contre Hazaël, roi de Syrie.
\VS{15}Le roi Joram s'en était retourné pour se faire guérir à Jizreel des blessures que les Syriens lui avaient faites, lorsqu'il se battait contre Hazaël, roi de Syrie. Jéhu dit~: Si vous le trouvez bon, que personne ne sorte ni ne s'échappe de la ville pour aller porter cette nouvelle à Jizreel.
\VS{16}Alors, Jéhu monta à cheval et s'en alla à Jizreel, car Joram était là, malade, et Achazia, roi de Juda, y était descendu pour le visiter.
\VS{17}Or il y avait une sentinelle sur une tour à Jizreel, qui voyant venir la troupe de Jéhu dit~: Je vois une troupe de gens. Et Joram dit~: Prends un cavalier et envoie-le à leur rencontre, et qu'il dise~: Est-ce la paix~?
\VS{18}Le cavalier s'en alla à sa rencontre, et dit~: Ainsi parle le roi~: Est-ce la paix~? Et Jéhu répondit~: Qu'as-tu à faire de la paix~? Mets-toi derrière moi. La sentinelle le rapporta, en disant~: Le messager est allé jusqu'à eux et il ne revient pas.
\VS{19}Joram envoya un second cavalier, qui arriva jusqu'à eux et dit~: Ainsi parle le roi~: Est-ce la paix~? Et Jéhu répondit~: Qu'as-tu à faire de la paix~? Mets-toi derrière moi.
\VS{20}La sentinelle le rapporta et dit~: Il est arrivé jusqu'à eux et il ne revient pas~; mais la manière de conduire le char est comme celle de Jéhu, fils de Nimschi~; car il le conduit avec furie.
\VS{21}Alors Joram dit~: Attelle~! Et on attela son char. Ainsi Joram, roi d'Israël, sortit avec Achazia, roi de Juda, chacun dans son char, et ils allèrent à la rencontre de Jéhu, et ils le trouvèrent dans le champ de Naboth de Jizreel\FTNT{1 R. 21.}.
\VS{22}Dès que Joram vit Jéhu, il dit~: Est-ce la paix, Jéhu~? Jéhu répondit~: Quelle paix~! Tant que durent les prostitutions de Jézabel, ta mère, et la multitude de ses enchantements~!
\VS{23}Alors Joram tourna sa main et s'enfuit, et il dit à Achazia~: Trahison, Achazia~!
\VS{24}Mais Jéhu saisit l'arc de sa main, et il frappa Joram entre ses épaules, de sorte que la flèche transperça son cœur, et il tomba sur ses genoux dans son char.
\VS{25}Jéhu dit à Bidkar, son officier~: Prends-le et jette-le dans le champ de Naboth de Jizreel~; car souviens-toi, lorsque nous étions à cheval moi et toi, ensemble, derrière Achab, son père, Yahweh prononça cette sentence contre lui~:
\VS{26}N'ai-je pas vu hier le sang de Naboth et le sang de ses fils, dit Yahweh~? Et je te le rendrai dans ce champ-ci, dit Yahweh~! C'est pourquoi prends-le donc, et jette-le dans ce champ, selon la parole de Yahweh.
\TextTitle{Mort d'Achazia\FTNTT{2 Ch. 22:7,9.}}
\VS{27}Achazia, roi de Juda, ayant vu cela, s'enfuit par le chemin de la maison du jardin~; mais Jéhu le poursuivit et dit~: Frappez-le sur le char~! Et on le frappa à la montée de Gur, près de Jibleam. Puis il se réfugia à Meguiddo, et il y mourut.
\VS{28}Ses serviteurs le transportèrent sur un char à Jérusalem, et ils l'enterrèrent dans son sépulcre avec ses pères, dans la cité de David.
\VS{29}Achazia avait commencé à régner sur Juda la onzième année de Joram, fils d'Achab.
\TextTitle{Mort de Jézabel}
\VS{30}Jéhu entra dans Jizreel. Jézabel, l'ayant appris, mit du fard à ses yeux, orna sa tête et regarda par la fenêtre.
\VS{31}Comme Jéhu franchissait la porte, elle dit~: Est-ce la paix, Zimri, assassin de son maître~?
\VS{32}Il leva sa tête vers la fenêtre et dit~: Qui est avec moi~? Qui~? Alors deux ou trois des eunuques regardèrent vers lui.
\VS{33}Et il leur dit~: Jetez-la en bas~! Et ils la jetèrent, de sorte qu'il rejaillit de son sang sur la muraille et sur les chevaux. Jéhu la foula aux pieds~;
\VS{34}puis il entra, mangea et but, et il dit~: Allez voir maintenant cette maudite et enterrez-la, car elle est fille de roi.
\VS{35}Ils allèrent donc pour l'enterrer~; mais ils ne trouvèrent d'elle que le crâne, les pieds et les paumes des mains.
\VS{36}Ils retournèrent l'annoncer à Jéhu, qui dit~: C'est la parole que Yahweh avait déclarée par son serviteur Elie\FTNT{1 R. 21:23.}, le Thischbite, en disant~: Dans le champ de Jizreel les chiens mangeront la chair de Jézabel~;
\VS{37}et le cadavre de Jézabel sera comme du fumier sur la face des champs, dans le champ de Jizreel, de sorte qu'on ne pourra dire~: C'est Jézabel.
\Chap{10}
\TextTitle{Accomplissement du jugement de Dieu sur la maison d'Achab}
\VerseOne{}Achab avait soixante-dix fils dans Samarie. Jéhu écrivit des lettres qu'il envoya à Samarie aux chefs de Jizreel, aux anciens et aux gouverneurs d'Achab. Il y était dit~:
\VS{2}Dès que cette lettre vous sera parvenue, puisque vous avez avec vous les fils de votre maître, avec vous les chars et les chevaux, la ville forte et les armes,
\VS{3}choisissez qui est le plus considérable et le plus sincère parmi les fils de votre maître, mettez-le sur le trône de son père et combattez pour la maison de votre maître.
\VS{4}Ils eurent une très grande peur et ils dirent~: Voici, deux rois n'ont point pu tenir contre lui, comment donc résisterions-nous~?
\VS{5}Et le chef de la maison, le chef de la ville, les anciens et les gouverneurs envoyèrent dire à Jéhu~: Nous sommes tes serviteurs, nous ferons tout ce que tu nous diras~; nous n'établirons personne roi, fais ce qui te semblera bon.
\VS{6}Jéhu leur écrivit une seconde lettre, où il était dit~: Si vous êtes pour moi et si vous obéissez à ma voix, prenez les têtes des fils de votre maître et venez auprès de moi demain à cette heure-ci, à Jizreel. Or les soixante-dix hommes, fils du roi, étaient avec les plus grands de la ville qui les élevaient.
\VS{7}Aussitôt que la lettre leur fut parvenue, ils prirent les fils du roi et ils égorgèrent ces soixante-dix hommes~; et ayant mis leurs têtes dans des corbeilles, ils les envoyèrent à Jéhu, à Jizreel.
\VS{8}Un messager vint l'en informer, en disant~: Ils ont apporté les têtes des fils du roi. Et il répondit~: Mettez-les en deux tas à l'entrée de la porte, jusqu'au matin.
\VS{9}Le matin, il sortit~; et se présentant à tout le peuple, il dit~: Vous êtes justes~! Voici, j'ai conspiré contre mon maître et je l'ai tué~; mais qui a frappé tous ceux-ci~?
\VS{10}Sachez maintenant qu'il ne tombera rien à terre de la parole de Yahweh\FTNT{1 R. 21:19-24.}, de la parole que Yahweh a prononcée contre la maison d'Achab~; Yahweh accomplit ce qu'il avait déclaré par son serviteur Elie.
\VS{11}Jéhu tua aussi tous ceux qui restaient de la maison d'Achab à Jizreel, tous ses grands, ses familiers et ses prêtres, sans en laisser échapper un seul.
\TextTitle{Mise à mort des frères d'Achazia et de la lignée d'Achab\FTNTT{2 Ch. 22:8.}}
\VS{12}Puis il se leva et partit pour aller à Samarie. Et comme il était près d'une maison de bergers sur le chemin,
\VS{13}Jéhu trouva les frères d'Achazia, roi de Juda, et leur dit~: Qui êtes-vous~? Ils répondirent~: Nous sommes les frères d'Achazia et nous sommes descendus pour saluer les fils du roi et les fils de la reine.
\VS{14}Jéhu dit~: Saisissez-les vivants. Ils les saisirent vivants et les égorgèrent, à savoir quarante-deux hommes, auprès du puits de la maison des bergers, sans en laisser échapper un seul.
\VS{15}Jéhu étant parti de là, il rencontra Jonadab, fils de Récab, qui venait au-devant de lui. Il le salua, et lui dit~: Ton cœur est-il aussi droit envers moi comme mon cœur l'est à ton égard~? Et Jonadab répondit~: Il l'est. Donne-moi ta main répliqua Jéhu. Et Jonadab lui donna sa main, et Jéhu le fit monter auprès de lui dans son char.
\VS{16}Puis il dit~: Viens avec moi et tu verras le zèle que j'ai pour Yahweh. Il l'emmena ainsi dans son char.
\VS{17}Et quand Jéhu fut arrivé à Samarie, il tua tous ceux qui restaient de la maison d'Achab à Samarie, et il les extermina entièrement, selon la parole que Yahweh avait dite à Elie.
\TextTitle{Mise à mort de tous les prophètes de Baal}
\VS{18}Puis Jéhu assembla tout le peuple, et leur dit~: Achab a peu servi Baal\FTNT{Jg. 2:13.}, mais Jéhu le servira beaucoup.
\VS{19}Maintenant donc, convoquez-moi tous les prophètes de Baal, tous ses serviteurs, et tous ses prêtres, sans qu'il en manque un seul, car je veux offrir un grand sacrifice à Baal~: Quiconque manquera ne vivra pas. Jéhu agissait avec ruse, pour faire périr les serviteurs de Baal.
\VS{20}Jéhu dit~: Publiez une fête solennelle en l'honneur de Baal. Et ils la publièrent.
\VS{21}Jéhu envoya des messagers dans tout Israël~; et tous les serviteurs de Baal arrivèrent, il n'y en eut pas un qui ne vînt~; et ils entrèrent dans le temple de Baal, qui fut rempli d'un bout à l'autre.
\VS{22}Alors Jéhu dit à celui qui avait la charge du vestiaire~: Sors des vêtements pour tous les serviteurs de Baal. Et cet homme sortit des vêtements.
\VS{23}Alors Jéhu, et Jonadab, fils de Récab, entrèrent dans le temple de Baal, et Jéhu dit aux serviteurs de Baal~: Cherchez et regardez afin qu'il n'y ait pas ici de serviteurs de Yahweh. Prenez garde qu'il n'y ait seulement que les serviteurs de Baal.
\VS{24}Ils entrèrent donc pour offrir des sacrifices et des holocaustes. Or Jéhu avait placé dehors quatre-vingts hommes, et leur avait dit~: Celui qui laissera échapper un de ces hommes que je remets entre vos mains, sa vie répondra de la sienne.
\VS{25}Et il arriva que dès qu'on eut achevé d'offrir l'holocauste, Jéhu dit aux gardes et aux officiers~: Entrez, tuez-les, et que nul n'échappe. Les gardes et les officiers les frappèrent du tranchant de l'épée, et les jetèrent là~; puis ils allèrent jusqu'à la ville du temple de Baal.
\VS{26}Ils tirèrent dehors les statues de la maison de Baal, et les brûlèrent.
\VS{27}Et ils démolirent la statue de Baal. Ils démolirent aussi la maison de Baal, et ils en firent un cloaque qui subsiste jusqu'à ce jour.
\VS{28}Ainsi Jéhu extermina Baal d'Israël.
\TextTitle{L'idolâtrie dans la vie de Jéhu}
\VS{29}Toutefois, Jéhu ne se détourna point des péchés que Jéroboam, fils de Nebath, avait fait commettre à Israël, à savoir les veaux d'or\FTNT{1 R. 12:28-29.} qui étaient à Béthel et à Dan.
\VS{30}Yahweh dit à Jéhu~: Parce que tu as fort bien exécuté ce qui était droit à mes yeux, et que tu as fait à la maison d'Achab tout ce qui était conforme à ma volonté, tes fils seront assis sur le trône d'Israël jusqu'à la quatrième génération.
\VS{31}Mais Jéhu ne prit point garde à marcher de tout son cœur dans la loi de Yahweh, le Dieu d'Israël~; il ne se détourna point des péchés que Jéroboam avait fait commettre à Israël.
\TextTitle{Hazaël règne sur la Syrie}
\VS{32}Dans ce temps-là, Yahweh commença à entamer le territoire d'Israël, et Hazaël battit les Israélites sur toutes les frontières.
\VS{33}Depuis le Jourdain, jusqu'au soleil levant, il battit tout le pays de Galaad, les Gadites, les Rubénites et ceux de Manassé, depuis Aroër sur le torrent de l'Arnon, jusqu'à Galaad et à Basan.
\TextTitle{Joachaz règne sur Israël}
\VS{34}Le reste des actions de Jéhu, tout ce qu'il a fait, et tous ses exploits, ne sont-ils pas écrits dans le livre des Chroniques des rois d'Israël~?
\VS{35}Jéhu se coucha avec ses pères, et on l'enterra à Samarie. Et Joachaz, son fils, régna à sa place.
\VS{36}Jéhu avait régné vingt-huit ans sur Israël à Samarie.
\Chap{11}
\TextTitle{Athalie fait périr la race royale de Juda\FTNTT{2 Ch. 22:9-12.}}
\VerseOne{}Athalie, mère d'Achazia, ayant vu que son fils était mort, se leva et extermina toute la race royale.
\VS{2}Mais Joschéba, fille du roi Joram, sœur d'Achazia, prit Joas, fils d'Achazia, et l'enleva du milieu des fils du roi, quand on les fit mourir~: Elle le mit avec sa nourrice dans la chambre des lits. Il fut ainsi dérobé aux regards d'Athalie, de sorte qu'on ne le fit point mourir.
\VS{3}Il resta caché six ans avec Joschéba dans la maison de Yahweh. Cependant Athalie régnait sur le pays.
\TextTitle{Joas devient roi de Juda\FTNTT{2 Ch. 23:1-11.}}
\VS{4}La septième année, Jehojada envoya chercher les chefs de centaines des Kéréthiens et des archers, et il les fit venir auprès de lui dans la maison de Yahweh. Il traita alliance avec eux, les fit jurer dans la maison de Yahweh, et leur montra le fils du roi.
\VS{5}Puis il leur donna cet ordre, en disant~: Voici ce que vous ferez. Parmi ceux d'entre vous qui entrent en service le jour du sabbat, un tiers doit monter la garde à la maison du roi,
\VS{6}un tiers sera à la porte de Sur, et un tiers à la porte derrière les archers~; ainsi vous veillerez à la garde de la maison, afin que personne n'y entre par force.
\VS{7}Vos deux autres compagnies, tous ceux qui sortent de service le jour du sabbat feront la garde de la maison de Yahweh, auprès du roi~:
\VS{8}Et vous entourerez le roi de toutes parts, chacun ayant ses armes à la main, et l'on mettra à mort quiconque s'avancera dans les rangs~; vous serez avec le roi quand il sortira et quand il entrera.
\VS{9}Les chefs de centaines firent donc tout ce que Jehojada, le prêtre, avait ordonné. Ils prirent chacun leurs gens, ceux qui entraient en service et ceux qui sortaient de service le jour du sabbat, et ils se rendirent vers le prêtre Jehojada.
\VS{10}Le prêtre donna aux chefs de centaine les lances et les boucliers qui provenaient du roi David, et qui étaient dans la maison de Yahweh.
\VS{11}Les archers, chacun les armes à la main, entourèrent le roi, en se plaçant depuis le côté droit de la maison, jusqu'au côté gauche, près de l'autel et près de la maison.
\VS{12}Jehojada fit amener le fils du roi, et il mit sur lui la couronne\FTNT{Couronne ou consacrer.} et le témoignage. Ils l'établirent roi et l'oignirent, et frappant des mains, ils dirent~: Vive le roi~!
\TextTitle{Mort d'Athalie\FTNTT{2 Ch. 23:12-15,21.}}
\VS{13}Athalie entendit le bruit des archers et du peuple, et elle vint vers le peuple à la maison de Yahweh.
\VS{14}Elle regarda. Et voici, le roi se tenait sur l'estrade, selon la coutume des rois. Les chefs et les trompettes étaient près du roi~: Tout le peuple du pays éclatait de joie, et on sonnait des trompettes. Alors Athalie déchira ses vêtements, et cria~: Conspiration~! Conspiration~!
\VS{15}Alors le prêtre Jehojada donna cet ordre aux chefs de centaines, qui avaient la charge de l'armée~: Faites-la sortir hors des rangs, et que celui qui la suivra soit mis à mort par l'épée. Car le prêtre avait dit~: Qu'elle ne soit pas mise à mort dans la maison de Yahweh~!
\VS{16}Ils lui firent donc place, et elle retourna dans la maison du roi par le chemin de l'entrée des chevaux~: C'est là qu'elle fut tuée.
\TextTitle{Alliance entre Jehojada, Yahweh et le peuple~; réveil sous le règne de Joas\FTNTT{2 Ch. 23:16-21.}}
\VS{17}Jehojada traita entre Yahweh, le roi et le peuple l'alliance par laquelle ils devaient être le peuple de Yahweh~; il traita aussi l'alliance entre le roi et le peuple.
\VS{18}Alors tout le peuple du pays entra dans la maison de Baal, et ils la démolirent avec ses autels~; et ils brisèrent entièrement ses images~; ils tuèrent aussi Matthan, prêtre de Baal, devant les autels. Le prêtre Jehojada établit des gardes dans la maison de Yahweh.
\VS{19}Il prit les chefs de centaines, les Kéréthiens et les archers, et tout le peuple du pays~; et ils firent descendre le roi de la maison de Yahweh, et ils entrèrent dans la maison du roi par le chemin de la porte des archers, et Joas s'assit sur le trône des rois.
\VS{20}Tout le peuple du pays fut dans la joie, et la ville fut en repos, après qu'on eût mis à mort Athalie par l'épée dans la maison du roi.
\VS{21}Joas était âgé de sept ans lorsqu'il commença à régner.
\Chap{12}
\TextTitle{Joas ordonne des réparations dans le temple\FTNTT{2 Ch. 24:2.}}
\VerseOne{}La septième année de Jéhu, Joas, commença à régner. Il régna quarante ans à Jérusalem. Sa mère s'appelait Tsibja, elle était de Beer-Schéba.
\VS{2}Joas fit ce qui est droit aux yeux de Yahweh pendant tout le temps qu'il suivit les instructions de Jehojada, le prêtre.
\VS{3}Toutefois, les hauts lieux ne disparurent point~; le peuple offrait encore des sacrifices et des parfums sur les hauts lieux.
\VS{4}Joas dit aux prêtres~: Tout l'argent consacré qu'on apporte dans la maison de Yahweh, l'argent ayant cours, à savoir l'argent pour l'évaluation des personnes d'après l'estimation qui en est faite, et tout l'argent que chacun apporte volontairement à la maison de Yahweh,
\VS{5}que les prêtres le prennent, chacun de la part des gens de sa connaissance, et qu'ils l'emploient à réparer ce qui est à réparer dans la maison, partout où l'on trouvera quelque chose à réparer.
\VS{6}Mais il arriva que, la vingt-troisième année du roi Joas, les prêtres n'avaient point encore réparé les brèches de la maison.
\VS{7}Le roi Joas appela le prêtre Jehojada et les autres prêtres, et il leur dit~: Pourquoi n'avez-vous pas réparé ce qui était à réparer dans la maison~? Maintenant, vous ne prendrez plus l'argent de vos connaissances, mais vous le livrerez pour les réparations de la maison.
\VS{8}Les prêtres convinrent de ne plus prendre l'argent du peuple et de ne pas être chargés des réparations de la maison.
\TextTitle{Offrandes volontaires pour réparer le temple\FTNTT{2 Ch. 24:8-14.}}
\VS{9}Alors le prêtre Jehojada prit un coffre, et le perça dans son couvercle, et le plaça à côté de l'autel, à droite, à l'endroit par lequel on entrait à la maison de Yahweh. Les prêtres qui avaient la garde du seuil y mettaient tout l'argent qu'on apportait à la maison de Yahweh.
\VS{10}Et dès qu'ils voyaient qu'il y avait beaucoup d'argent dans le coffre, le secrétaire du roi montait avec le grand-prêtre, et ils mettaient dans des sacs l'argent qui se trouvait dans la maison de Yahweh, puis ils le comptaient.
\VS{11}Ils remettaient cet argent bien compté entre les mains de ceux qui étaient chargés de faire exécuter l'ouvrage dans la maison de Yahweh. Et l'on employait cet argent pour les charpentiers et pour les architectes qui travaillaient à la maison de Yahweh,
\VS{12}pour les maçons et les tailleurs de pierres, pour acheter du bois et des pierres de taille, afin de réparer les brèches de la maison de Yahweh, et pour acheter tout ce qu'il fallait pour la réparation de la maison.
\VS{13}Mais, avec l'argent qu'on apportait dans la maison de Yahweh, on ne fit pour la maison de Yahweh ni bassins d'argent ni de couteaux, ni coupes, ni trompettes, ni aucun autre ustensile d'or, ou ustensile d'argent~;
\VS{14}on le distribuait à ceux qui avaient la charge de l'ouvrage et qui réparaient la maison de Yahweh.
\VS{15}On ne demandait pas de comptes aux hommes entre les mains desquels on remettait l'argent pour qu'ils le donnent à ceux qui faisaient l'ouvrage, car ils le faisaient fidèlement.
\VS{16}L'argent des sacrifices pour la culpabilité et l'argent des sacrifices pour les expiations n'était point apporté dans la maison de Yahweh~: Car il était pour les prêtres.
\TextTitle{Invasion syrienne évitée~; mort de Joas}
\VS{17}Alors Hazaël\FTNT{Hazaël envahit Juda à deux reprises. Ce passage fait mention de la première invasion~; la deuxième invasion est relatée en 2 Ch. 24:23.}, roi de Syrie, monta et fit la guerre à Gath, dont il s'empara. Hazaël avait l'intention de monter contre Jérusalem.
\VS{18}Mais Joas, roi de Juda, prit tout ce qui était consacré, que Josaphat, Joram, et Achazia, ses pères, rois de Juda, avaient consacré, tout ce que lui-même avait consacré, tout l'or qui se trouva dans les trésors de la maison de Yahweh et de la maison du roi~; et il envoya le tout à Hazaël, roi de Syrie, qui ne monta pas contre Jérusalem.
\VS{19}Le reste des actions de Joas, tout ce qu'il a fait, cela n'est-il pas écrit dans le livre des Chroniques des rois de Juda~?
\VS{20}Ses serviteurs se soulevèrent et se liguèrent~; ils frappèrent Joas dans la maison de Millo, qui est à la descente de Silla.
\VS{21}Jozacar, fils de Schimeath, et Jozabad fils de Schomer, ses serviteurs, le frappèrent, et il mourut. On l'enterra avec ses pères dans la cité de David. Et Amatsia, son fils, régna à sa place.
\Chap{13}
\TextTitle{Joachaz règne sur Israël}
\VerseOne{}La vingt-troisième année de Joas, fils d'Achazia, roi de Juda, Joachaz, fils de Jéhu, commença à régner sur Israël à Samarie. Il régna dix-sept ans.
\VS{2}Il fit ce qui est mal aux yeux de Yahweh~; car il suivit les péchés de Jéroboam, fils de Nebath, par lesquels il avait fait pécher Israël, et il ne s'en détourna point.
\TextTitle{L'idolâtrie perdure dans le pays}
\VS{3}La colère de Yahweh s'enflamma contre Israël, et il les livra entre les mains de Hazaël, roi de Syrie, et entre les mains de Ben-Hadad, fils de Hazaël, tout le temps que ces rois vécurent.
\VS{4}Mais Joachaz implora Yahweh. Et Yahweh l'exauça, parce qu'il vit l'oppression sous laquelle le roi de Syrie tenait Israël.
\VS{5}Yahweh donna donc un libérateur à Israël, et ils échappèrent aux mains des Syriens~; ainsi les enfants d'Israël habitèrent dans leurs tentes comme auparavant.
\VS{6}Mais ils ne se détournèrent point des péchés de la maison de Jéroboam, par lesquels il avait fait pécher Israël~; ils s'y livrèrent, et même l'idole d'Asherah\FTNT{Voir commentaire Jg. 2:13.} resta debout à Samarie.
\VS{7}De tout le peuple de Joachaz, Dieu ne lui avait laissé que cinquante cavaliers, dix chars, et dix mille hommes de pied~; car le roi de Syrie les avait fait périr et les avait rendus semblables à la poussière qu'on foule aux pieds.
\TextTitle{Mort de Joachaz~; Joas règne sur Israël}
\VS{8}Le reste des actions de Joachaz, tout ce qu'il a fait, et ses exploits, cela n'est-il pas écrit dans le livre des Chroniques des rois d'Israël~?
\VS{9}Ainsi Joachaz se coucha avec ses pères, et on l'ensevelit à Samarie. Et Joas, son fils, régna à sa place.
\VS{10}La trente-septième année de Joas, roi de Juda, Joas, fils de Joachaz, commença à régner sur Israël à Samarie. Il régna seize ans.
\VS{11}Et il fit ce qui est mal aux yeux de Yahweh~; il ne se détourna d'aucun des péchés de Jéroboam, fils de Nebath, par lesquels il avait fait pécher Israël, il s'y livra comme lui.
\TextTitle{Mort de Joas}
\VS{12}Le reste des actions de Joas, tout ce qu'il a fait, ses exploits, et la guerre qu'il eut avec Amatsia, roi de Juda, tout cela n'est-il pas écrit dans le livre des Chroniques des rois d'Israël~?
\VS{13}Joas se coucha avec ses pères, et Jéroboam s'assit sur son trône. Joas fut enterré à Samarie avec les rois d'Israël.
\TextTitle{Fin de la vie d'Elisée~; récit de la visite de Joas roi d'Israël}
\VS{14}Elisée était atteint de la maladie dont il mourut~; et Joas, roi d'Israël, descendit vers lui, pleura sur son visage, en disant~: Mon père~! Mon père~! Char d'Israël et sa cavalerie~!
\VS{15}Elisée lui dit~: Prends un arc et des flèches. Il prit donc un arc et des flèches.
\VS{16}Puis Elisée dit au roi d'Israël~: Bande l'arc avec ta main. Mets ta main sur l'arc. Et quand il y eut mis sa main, Elisée mit ses mains sur les mains du roi,
\VS{17}et il lui dit~: Ouvre la fenêtre à l'orient. Et il l'ouvrit. Elisée lui dit~: Tire. Après qu'il eut tiré, il lui dit~: C'est la flèche de la délivrance de la part de Yahweh, la flèche de la délivrance contre les Syriens~; tu frapperas les Syriens à Aphek, jusqu'à leur extermination.
\VS{18}Elisée lui dit encore~: Prends les flèches. Et il les prit. Elisée dit au roi d'Israël~: Frappe contre terre. Et le roi frappa trois fois, puis il s'arrêta.
\VS{19}Et l'homme de Dieu se mit dans une très grande colère contre lui, et lui dit~: Il fallait frapper cinq ou six fois~; alors tu aurais battu les Syriens jusqu'à leur extermination~; mais maintenant tu ne les frapperas que trois fois.
\TextTitle{Mort d'Elisée~; ses os rendent la vie à un mort}
\VS{20}Elisée mourut, et on l'ensevelit. L'année suivante, quelques troupes de Moabites entrèrent dans le pays.
\VS{21}Et comme on enterrait un homme, voici, on aperçut l'une des troupes de soldats, et l'on jeta l'homme dans le sépulcre d'Elisée. L'homme alla toucher les os d'Elisée, il reprit vie et se leva sur ses pieds.
\TextTitle{Fin de l'oppression syrienne}
\VS{22}Pendant toute la vie de Joachaz, Hazaël, roi de Syrie, avait opprimé Israël.
\VS{23}Mais Yahweh eut compassion d'eux, leur fit miséricorde, il tourna sa face vers eux par amour pour son alliance avec Abraham, Isaac et Jacob, de sorte qu'il ne voulut point les exterminer, et il ne les rejeta pas de sa face, jusqu'à maintenant.
\VS{24}Puis Hazaël, roi de Syrie, mourut, et Ben-Hadad, son fils, régna à sa place.
\VS{25}Joas, fils de Joachaz, reprit des mains de Ben-Hadad, fils d'Hazaël, les villes enlevées par Hazaël, à Joachaz, son père, pendant la guerre. Joas le battit trois fois et recouvra les villes d'Israël.
\Chap{14}
\TextTitle{Amatsia règne sur Juda\FTNTT{2 Ch. 25:1-4.}}
\VerseOne{}La deuxième année de Joas, fils de Joachaz, roi d'Israël, Amatsia, fils de Joas, roi de Juda, commença à régner.
\VS{2}Il était âgé de vingt-cinq ans lorsqu'il commença à régner, et il régna vingt-neuf ans à Jérusalem. Sa mère s'appelait Joaddan, elle était de Jérusalem.
\VS{3}Il fit ce qui est droit aux yeux de Yahweh, non pas toutefois comme David, son père~; il agit entièrement comme avait agi Joas, son père.
\VS{4}Seulement, les hauts lieux ne furent point ôtés~; le peuple offrait encore des sacrifices et des parfums sur les hauts lieux.
\VS{5}Et il arriva que dès que le royaume fut affermi entre ses mains, il frappa ses serviteurs qui avaient tué le roi, son père.
\VS{6}Mais il ne fit point mourir les fils des meurtriers, suivant ce qui est écrit dans le livre de la loi de Moïse, où Yahweh donne ce commandement~: On ne fera point mourir les pères pour les enfants, et l'on ne fera pas mourir les enfants pour les pères~; mais on fera mourir chacun pour son péché\FTNT{De. 24:16~; Ez. 18:4,20.}.
\VS{7}Il frappa dix mille hommes d'Edom dans la vallée du sel~; et il prit Séla durant la guerre, et l'appela Joktheel, nom qu'elle a conservé jusqu'à ce jour.
\VS{8}Alors Amatsia envoya des messagers vers Joas, fils de Joachaz, fils de Jéhu, roi d'Israël, pour lui dire~: Viens, voyons-nous en face~!
\VS{9}Et Joas, roi d'Israël, envoya dire à Amatsia, roi de Juda~: L'épine du Liban envoya dire au cèdre du Liban~: Donne ta fille en mariage à mon fils~! Et les bêtes sauvages qui sont au Liban passèrent et foulèrent l'épine.
\VS{10}Parce que tu as frappé et ravagé Edom, ton cœur s'est élevé. Contente-toi de ta gloire et reste dans ta maison. Pourquoi exciterais-tu le mal par lequel tu tomberas, toi et Juda avec toi~?
\VS{11}Mais Amatsia ne l'écouta pas. Et Joas, roi d'Israël, monta~: Et ils s'affrontèrent, lui et Amatsia, roi de Juda, à Beth-Schémesch, qui est à Juda.
\VS{12}Juda fut battu par Israël, et ils s'enfuirent chacun dans leurs tentes.
\VS{13}Joas, roi d'Israël, prit Amatsia, roi de Juda, fils de Joas, fils d'Achazia, à Beth-Schémesch. Puis il vint à Jérusalem et fit une brèche de quatre cents coudées dans la muraille de Jérusalem, depuis la porte d'Ephraïm, jusqu'à la porte de l'angle.
\VS{14}Il prit tout l'or et tout l'argent et tous les vases qui se trouvaient dans la maison de Yahweh et dans les trésors de la maison royale~; il prit aussi des enfants en otages, et il retourna à Samarie.
\TextTitle{Jéroboam II règne sur Israël}
\VS{15}Le reste des actions de Joas, ses exploits, et comment il combattit contre Amatsia, tout cela n'est-il pas écrit dans le livre des Chroniques des rois d'Israël~?
\VS{16}Et Joas se coucha avec ses pères et fut enseveli à Samarie avec les rois d'Israël. Et Jéroboam, son fils, régna à sa place.
\TextTitle{Mort d'Amatsia~; Azaria (Ozias) règne sur Juda (2 Ch. 25:26-28)}
\VS{17}Amatsia, fils de Joas, roi de Juda, vécut quinze ans après la mort de Joas, fils de Joachaz, roi d'Israël.
\VS{18}Le reste des actions d'Amatsia n'est-il pas écrit dans le livre des Chroniques des rois de Juda~?
\VS{19}On forma une conspiration contre lui à Jérusalem, et il s'enfuit à Lakis~; mais on le poursuivit à Lakis, où on le fit mourir.
\VS{20}On le transporta sur des chevaux, et il fut enseveli à Jérusalem avec ses pères, dans la cité de David.
\VS{21}Alors tout le peuple de Juda prit Azaria, âgé de seize ans, et ils l'établirent roi à la place d'Amatsia, son père.
\VS{22}Azaria bâtit Elath et la fit rentrer sous la puissance de Juda, après que le roi se coucha avec ses pères.
\TextTitle{Prophétie de Jonas accomplie par Jéroboam II}
\VS{23}La quinzième année d'Amatsia, fils de Joas, roi de Juda, Jéroboam, fils de Joas, commença à régner sur Israël à Samarie, et il régna quarante et un ans.
\VS{24}Il fit ce qui est mal aux yeux de Yahweh, et ne se détourna d'aucun des péchés de Jéroboam, fils de Nebath, par lesquels il avait fait pécher Israël.
\VS{25}Il rétablit les frontières d'Israël depuis l'entrée de Hamath, jusqu'à la mer de la plaine, selon la parole de Yahweh, le Dieu d'Israël, qu'il avait prononcée par son serviteur Jonas\FTNT{Jon. 1:1.}, fils d'Amitthaï, le prophète, de Gath-Hépher.
\VS{26}Car Yahweh vit que l'affliction d'Israël était à son comble, et l'extrémité à laquelle se trouvaient réduits esclaves et hommes libres, sans qu'il n'y ait personne pour venir au secours d'Israël.
\VS{27}Or Yahweh n'avait point résolu d'effacer le nom d'Israël de dessous les cieux, à cause de cela, il les délivra par les mains de Jéroboam, fils de Joas.
\TextTitle{Zacharie règne sur Israël}
\VS{28}Le reste des actions de Jéroboam, tout ce qu'il a fait, ses exploits de guerre, et comment il reconquit pour Israël, Damas et Hamath qui avaient appartenu à Juda, cela n'est-il pas écrit dans le livre des Chroniques des rois d'Israël~?
\VS{29}Puis Jéroboam se coucha avec ses pères, avec les rois d'Israël. Et Zacharie, son fils, régna à sa place.
\Chap{15}
\TextTitle{Juda demeure dans l'idolâtrie sous le règne d'Azaria (Ozias)\FTNTT{2 R. 14:21-22~; 2 Ch. 26:1-15.}}
\VerseOne{}La vingt-septième année de Jéroboam, roi d'Israël, Azaria\FTNT{Azaria (Ozias, selon 2 Ch. 26:1-15~; à ne pas confondre avec le prophète du même nom que son grand-père avait fait assassiner) fut couronné à l'âge de seize ans et mourut à l'âge de soixante-huit ans.}, fils d'Amatsia, roi de Juda, régna.
\VS{2}Il était âgé de seize ans lorsqu'il commença à régner, et il régna cinquante-deux ans à Jérusalem. Sa mère s'appelait Jecolia, elle était de Jérusalem.
\VS{3}Il fit ce qui est droit aux yeux de Yahweh, entièrement comme avait fait Amatsia, son père.
\VS{4}Seulement, les hauts lieux ne disparurent pas~; le peuple offrait encore des sacrifices et des parfums sur les hauts lieux.
\TextTitle{Jugement de Yahweh sur Ozias par la lèpre\FTNTT{2 Ch. 26:16-21.}}
\VS{5}Alors Yahweh frappa le roi, qui fut lépreux jusqu'au jour de sa mort, et il demeura dans une maison à l'écart. Et Jotham, fils du roi, avait la charge de la maison, jugeant le peuple du pays.
\VS{6}Le reste des actions d'Azaria, tout ce qu'il a fait, cela n'est-il pas écrit dans le livre des Chroniques des rois de Juda~?
\VS{7}Azaria se coucha avec ses pères, et fut enseveli avec ses pères dans la cité de David, et Jotham, son fils, régna à sa place.
\TextTitle{Conspiration de Schallum contre Zacharie, roi d'Israël}
\VS{8}La trente-huitième année d'Azaria, roi de Juda, Zacharie, fils de Jéroboam, commença à régner sur Israël à Samarie, et il régna six mois.
\VS{9}Il fit ce qui est mal aux yeux de Yahweh, comme avaient fait ses pères~; il ne se détourna point des péchés de Jéroboam, fils de Nebath, par lesquels il avait fait pécher Israël.
\VS{10}Schallum, fils de Jabesch, fit une conspiration contre lui, et le frappa devant le peuple. Il le tua, et régna à sa place.
\VS{11}Quant au reste des actions de Zacharie, voilà, elles sont écrites dans le livre des Chroniques des rois d'Israël.
\VS{12}Ainsi s'accomplit la parole que Yahweh avait déclarée à Jéhu, en disant~: Tes fils seront assis sur le trône d'Israël jusqu'à la quatrième génération, et il en fut ainsi\FTNT{2 R. 10:30.}
\TextTitle{Schallum règne sur Israël~; sa mort}
\VS{13}Schallum, fils de Jabesch, commença à régner la trente-neuvième année d'Ozias, roi de Juda. Il régna pendant un mois à Samarie.
\VS{14}Menahem, fils de Gadi, monta de Thirtsa et vint dans Samarie, et frappa à Samarie, Schallum, fils de Jabesch, et le fit mourir~; et il régna à sa place.
\VS{15}Le reste des actions de Schallum, et la conspiration qu'il forma, cela est écrit dans le livre des Chroniques des rois d'Israël.
\TextTitle{Menahem règne sur Israël}
\VS{16}Alors Menahem frappa Thiphsach et tous ceux qui y étaient, avec son territoire depuis Thirtsa~; il la frappa parce qu'elle ne lui avait point ouvert ses portes. Il fendit le ventre de toutes les femmes enceintes.
\VS{17}La trente-neuvième année d'Azaria, roi de Juda, Menahem, fils de Gadi, commença à régner sur Israël. Il régna dix ans à Samarie.
\VS{18}Il fit ce qui est mal aux yeux de Yahweh~; il ne se détourna point des péchés de Jéroboam, fils de Nebath, par lesquels il avait fait pécher Israël.
\TextTitle{Invasion d'Israël par le roi d'Assyrie\FTNTT{1 Ch. 5:26.}}
\VS{19}Alors Pul, roi d'Assyrie, vint contre le pays~; et Menahem donna mille talents d'argent à Pul, afin qu'il l'aide à affermir son royaume entre ses mains.
\VS{20}Menahem leva cet argent sur tous ceux d'Israël qui avaient de la richesse pour le donner au roi d'Assyrie~; chacun cinquante sicles d'argent. Ainsi, le roi d'Assyrie s'en retourna, et ne s'arrêta point dans le pays.
\VS{21}Le reste des actions de Menahem, tout ce qu'il a fait, cela n'est-il pas écrit dans le livre des Chroniques des rois d'Israël~?
\TextTitle{Mort de Menahem~; Pekachia règne sur Israël}
\VS{22}Menahem se coucha avec ses pères, et Pekachia, son fils, régna à sa place.
\VS{23}La cinquantième année d'Azaria, roi de Juda, Pekachia, fils de Menahem, commença à régner sur Israël à Samarie. Il régna deux ans.
\VS{24}Il fit ce qui est mal aux yeux de Yahweh~; il ne se détourna point des péchés de Jéroboam, fils de Nebath, par lesquels il avait fait pécher Israël.
\TextTitle{Pékach tue Pekachia et devient roi d'Israël}
\VS{25}Pékach, fils de Remalia, son officier, conspira contre lui~; il le frappa à Samarie, dans le palais de la maison royale, de même qu'Argob et Arié~; il avait avec lui cinquante hommes d'entre les fils des Galaadites. Il fit ainsi mourir Pekachia, et il régna à sa place.
\VS{26}Le reste des actions de Pekachia tout ce qu'il a fait, cela est écrit dans le livre des Chroniques des rois d'Israël.
\VS{27}La cinquante-deuxième année d'Azaria, roi de Juda, Pékach, fils de Remalia, commença à régner sur Israël à Samarie. Il régna vingt ans.
\VS{28}Il fit ce qui est mal aux yeux de Yahweh et ne se détourna point des péchés de Jéroboam, fils de Nebath, par lesquels il avait fait pécher Israël.
\VS{29}Du temps de Pékach, roi d'Israël, Tiglath-Piléser, roi d'Assyrie, vint et prit Ijjon, Abel-Beth-Maaca, Janoach, Kédesch, Hatsor, Galaad et la Galilée, et même tout le pays de Nephthali, et il emmena captifs les habitants en Assyrie.
\TextTitle{Osée conspire contre Pékach et règne sur Israël}
\VS{30}Osée, fils d'Ela, forma une conspiration contre Pékach, fils de Remalia, le frappa et le fit mourir. Il régna à sa place la vingtième année de Jotham, fils d'Ozias.
\VS{31}Le reste des actions de Pékach, tout ce qu'il a fait, cela est écrit dans le livre des Chroniques des rois d'Israël.
\TextTitle{Jotham règne sur Juda~; sa mort\FTNTT{2 R. 15:2~; 2 Ch. 26:23~; 27:1-9.}}
\VS{32}La seconde année de Pékach, fils de Remalia, roi d'Israël, Jotham, fils d'Ozias, roi de Juda, commença à régner.
\VS{33}Il était âgé de vingt-cinq ans lorsqu'il commença à régner. Il régna seize ans à Jérusalem. Sa mère s'appelait Jeruscha, fille de Tsadok.
\VS{34}Il fit ce qui est droit aux yeux de Yahweh~; il agit entièrement comme avait agi Ozias, son père.
\VS{35}Seulement, les hauts lieux ne disparurent point~; et le peuple offrait encore des sacrifices et des parfums sur les hauts lieux. Jotham bâtit la porte supérieure de la maison de Yahweh.
\VS{36}Le reste des actions de Jotham, tout ce qu'il a fait, cela n'est-il pas écrit dans le livre des Chroniques des rois de Juda~?
\VS{37}Dans ce temps-là, Yahweh commença à envoyer contre Juda, Retsin, roi de Syrie, et Pékach, fils de Remalia.
\VS{38}Jotham se coucha avec ses pères, et il fut enseveli dans la cité de David, son père. Et Achaz, son fils, régna à sa place.
\Chap{16}
\TextTitle{Achaz règne sur Juda\FTNTT{2 R. 15:38~; 2 Ch. 28:1-4.}}
\VerseOne{}La dix-septième année de Pékach, fils de Remalia, Achaz, fils de Jotham, roi de Juda, commença à régner.
\VS{2}Achaz était âgé de vingt ans lorsqu'il commença à régner. Il régna seize ans à Jérusalem. Il ne fit point ce qui est droit aux yeux de Yahweh, son Dieu, comme avait fait David, son père.
\VS{3}Mais il suivit la voie des rois d'Israël et il fit même passer son fils par le feu, selon les abominations des nations que Yahweh avait chassées devant les enfants d'Israël.
\VS{4}Il offrait aussi des sacrifices et des parfums sur les hauts lieux, sur les coteaux et sous tout arbre vert.
\TextTitle{Juda envahi par les rois d'Assyrie et d'Israël\FTNTT{2 Ch. 28:5-19.}}
\VS{5}Alors Retsin, roi de Syrie, et Pékach, fils de Remalia, roi d'Israël, montèrent contre Jérusalem pour lui faire la guerre. Ils assiégèrent Achaz~; mais ne purent en venir à bout par les armes.
\VS{6}Dans ce même temps, Retsin, roi de Syrie, fit rentrer Elath au pouvoir des Syriens~; il expulsa les Juifs d'Elath, et les Syriens vinrent à Elath, où ils ont demeuré jusqu'à ce jour.
\TextTitle{Le roi d'Assyrie vient en aide à Achaz et s'empare de Damas\FTNTT{2 Ch. 28:16-25.}}
\VS{7}Achaz envoya des messagers à Tiglath-Piléser, roi d'Assyrie, pour lui dire~: Je suis ton serviteur et ton fils~; monte et délivre-moi de la main du roi des Syriens, et de la main du roi d'Israël, qui s'élèvent contre moi.
\VS{8}Alors Achaz prit l'argent et l'or qui se trouvaient dans la maison de Yahweh, et dans les trésors de la maison royale, et il les envoya en présent au roi d'Assyrie.
\VS{9}Le roi d'Assyrie l'écouta~; il monta contre Damas, la prit, emmena les habitants en captivité à Kir et fit mourir Retsin.
\VS{10}Alors le roi Achaz s'en alla à la rencontre de Tiglath-Piléser, roi d'Assyrie, à Damas. Et ayant vu l'autel\FTNT{Achaz, roi de Juda, se rendit chez le roi d'Assyrie et il fut fasciné par l'autel de son dieu au point de le convoiter. Il demanda au prêtre Urie de fabriquer un autel identique, dont le modèle n'était pas celui que Yahweh avait décrit à Moïse. Il introduisit un objet de culte d'origine païenne dans le temple de Jérusalem, sous prétexte d'honorer Yahweh. Certains «~Pères de l'Eglise~», comme les empereurs Constantin I (285-337) et Théodose I (347-395), se sont comportés exactement comme Achaz en adoptant les pratiques païennes. Les historiens s'accordent pour dire que la diffusion de la Parole de Dieu sous l'empire de Constantin I (285-337), empereur de Rome, avait des fins strictement politiques. Cette politique a eu deux conséquences essentielles concernant l'influence de l'Eglise chrétienne et son fonctionnement de plus en plus éloigné de la Parole de Dieu~:\\- Les peuples païens ont introduit leurs rites idolâtres au sein de l'Eglise. En effet, les dogmes de l'institution devaient plaire à la majorité.\\- L'Eglise chrétienne cessant d'être persécutée, son fonctionnement intimiste fondé sur l'implication de chaque croyant et l'exercice de la prêtrise universelle des chrétiens, a changé à cause de l'effet de masse. Devenant numériquement très importante, il a fallu imposer une autorité capable de contenir un nombre de fidèles de plus en plus élevé. Mais à cause de cette augmentation numérique et de la présence de «~faux convertis~» lié au fait que l'adhésion au christianisme (religion chrétienne fondée par les hommes) devenait une obligation, l'étude de la Parole, la fraction du pain et la prière ne pouvaient plus perdurer. C'est ainsi que beaucoup d'églises ont commencé à subir l'influence du monde.} qui était à Damas, le roi Achaz envoya au prêtre Urie, la forme et le modèle exact de cet autel.
\VS{11}Le prêtre Urie construisit un autel entièrement d'après le modèle envoyé de Damas par le roi Achaz, et le prêtre Urie le fit avant que le roi Achaz soit de retour de Damas.
\VS{12}Quand le roi Achaz revint de Damas et vit l'autel, il s'en approcha et y monta~;
\VS{13}Il fit brûler son holocauste et son sacrifice, versa ses libations et répandit sur l'autel le sang de ses sacrifices d'offrande de paix\FTNT{Voir commentaire en Lé. 3:1.}.
\VS{14}Il éloigna de la face de la maison l'autel d'airain qui était devant Yahweh, afin qu'il ne soit pas entre le nouvel autel et la maison de Yahweh~; et il le plaça à côté du nouvel autel, vers le nord.
\VS{15}Et le roi Achaz donna cet ordre au prêtre Urie~: Fais brûler l'holocauste du matin et l'offrande du soir, l'holocauste du roi et son offrande, les holocaustes de tout le peuple du pays et leurs offrandes, verses-y leurs libations, et répands-y tout le sang des holocaustes et tout le sang des sacrifices~; mais pour ce qui concerne l'autel d'airain, je m'en occuperai.
\VS{16}Le prêtre Urie, exécuta tout ce que le roi Achaz lui avait ordonné.
\VS{17}Le roi Achaz brisa les panneaux des bases et en ôta les cuves qui étaient dessus. Il descendit la mer de dessus les bœufs d'airain qui étaient sous elle et il la posa sur un pavé de pierre.
\VS{18}Il changea aussi dans la maison de Yahweh, à cause du roi d'Assyrie, le portique du sabbat qu'on y avait bâti et l'entrée extérieure du roi.
\TextTitle{Mort d'Achaz~; Ezéchias devient roi de Juda\FTNTT{2 Ch. 28:26-27.}}
\VS{19}Le reste des actions d'Achaz et tout ce qu'il a fait, cela n'est-il pas écrit dans le livre des Chroniques des rois de Juda~?
\VS{20}Achaz se coucha avec ses pères, et il fut enseveli avec ses pères dans la cité de David. Et Ezéchias, son fils, régna à sa place.
\Chap{17}
\TextTitle{Osée devient le dernier roi d'Israël}
\VerseOne{}La douzième année d'Achaz, roi de Juda, Osée, fils d'Ela, régna à Samarie sur Israël. Il régna neuf ans.
\VS{2}Il fit ce qui est mal aux yeux de Yahweh, non pas toutefois comme les rois d'Israël qui avaient été avant lui.
\TextTitle{Osée tente de s'affranchir du joug de l'Assyrie}
\VS{3}Salmanasar\FTNT{Le royaume d'Israël a été détruit en 722 av. J.-C., par l'empereur assyrien Salmanasar V (règne~: 727-722 av. J.-C.), après avoir assiégé trois ans le roi Osée (règne~: 732-722 av. J.-C.) dans sa capitale Samarie. Celui-ci ne payait plus le tribut et essayait d'obtenir l'appui de l'Egypte pour retrouver l'indépendance. Le royaume d'Israël a disparu au début du 8ème siècle av. J.-C., provoquant la dispersion dans le monde de plusieurs juifs issus des dix tribus. L'origine des Samaritains remonte à cette déportation, après que le royaume du Nord soit tombé aux mains de Salmanasar, roi d'Assyrie. Malgré les déportations, les Assyriens n'avaient pas laissé déserte cette région appelée «~Samarie~»~; plusieurs Israélites y étaient restés et des colons d'autres provinces assyriennes vinrent s'y établir. Les Samaritains sont issus du mélange de ces populations, et leur religion est un mélange entre le culte à Yahweh avec celui des dieux étrangers.}, roi d'Assyrie, monta contre lui~; et Osée lui fut assujetti et lui paya un tribut.
\VS{4}Mais le roi d'Assyrie découvrit une conspiration chez Osée, qui avait envoyé des messagers vers So, roi d'Egypte, et qui ne payait plus le tribut tous les ans au roi d'Assyrie. C'est pourquoi le roi d'Assyrie le fit enfermer et enchaîner dans une prison.
\TextTitle{Siège de Samarie par le roi d'Assyrie}
\VS{5}Le roi d'Assyrie parcourut tout le pays et monta contre Samarie qu'il assiégea pendant trois ans.
\TextTitle{Les causes de la captivité d’Israël par l'Assyrie}
\VS{6}La neuvième année d'Osée, le roi d'Assyrie prit Samarie et emmena captifs les Israélites en Assyrie. Il les fit habiter à Chalach, et sur le Chabor, fleuve de Gozan, et dans les villes des Mèdes.
\VS{7}Cela arriva parce que les enfants d'Israël péchèrent contre Yahweh, leur Dieu, qui les avait fait monter hors du pays d'Egypte, de dessous la main de Pharaon, roi d'Egypte, et parce qu'ils craignirent d'autres dieux.
\VS{8}Ils suivirent les coutumes des nations que Yahweh avait chassées devant les enfants d'Israël, et celles des rois d'Israël qu'ils avaient établis.
\VS{9}Les enfants d'Israël firent en secret des choses qui n'étaient point droites, contre Yahweh, leur Dieu. Ils se bâtirent des hauts lieux dans toutes leurs villes, depuis la tour des gardes jusqu'aux villes fortes.
\VS{10}Ils se dressèrent des statues et des Asherah sur toutes les hautes collines et sous tout arbre vert.
\VS{11}Et là, ils brûlèrent des parfums sur tous les hauts lieux, comme les nations que Yahweh avait chassées devant eux, et ils firent des choses mauvaises pour irriter Yahweh.
\VS{12}Ils servirent les idoles, au sujet desquelles Yahweh leur avait dit~: Vous ne ferez pas cela\FTNT{1 R. 12:28.}.
\VS{13}Yahweh fit avertir Israël et Juda par tous ses prophètes, tous les voyants, en disant~: Détournez-vous de toutes vos mauvaises voies, revenez, et gardez mes commandements et mes ordonnances, en suivant entièrement la loi que j'ai prescrite à vos pères et que je vous ai envoyée par mes serviteurs les prophètes.
\VS{14}Mais ils n'écoutèrent point et raidirent leur cou, comme leurs pères avaient raidi leur cou, et n'avaient pas cru en Yahweh, leur Dieu.
\VS{15}Ils rejetèrent ses lois, et son alliance qu'il avait traitée avec leurs pères, et ses avertissements, qu'il leur avait adressés. Ils allèrent après des choses de néant et ne furent eux-mêmes que néant, après les nations qui les entouraient et que Yahweh leur avait défendu d'imiter.
\VS{16}Ils abandonnèrent tous les commandements de Yahweh, leur Dieu, ils firent deux veaux en métal fondu, ils fabriquèrent des idoles d'Asherah\FTNT{Voir commentaire en Jg. 2:13.}, ils se prosternèrent devant toute l'armée des cieux et ils servirent Baal.
\VS{17}Ils firent aussi passer leurs fils et leurs filles par le feu, ils s'adonnèrent à la divination et aux enchantements, et ils se vendirent pour faire ce qui est mal aux yeux de Yahweh afin de l'irriter.
\VS{18}C'est pourquoi, Yahweh fut très irrité contre Israël et il les rejeta~; il n'est resté que la seule tribu de Juda.
\VS{19}Même Juda n'avait pas gardé les commandements de Yahweh, son Dieu, mais ils avaient suivi les ordonnances qu'Israël avait établies.
\VS{20}C'est pourquoi Yahweh rejeta toute la race d'Israël~; il les a humiliés, il les a livrés entre les mains des pillards, il a fini par les chasser loin de sa face.
\VS{21}Car Israël s'était détaché de la maison de David, et avait établi roi Jéroboam, fils de Nebath. Jéroboam avait détourné Israël de Yahweh, afin qu'il ne le suive plus, et lui avait fait commettre un grand péché.
\VS{22}C'est pourquoi les enfants d'Israël s'étaient livrés à tous les péchés que Jéroboam avait commis~; ils ne s'en détournèrent pas,
\VS{23}jusqu'à ce que Yahweh ait chassé Israël de devant sa face, comme il l'avait annoncé par tous ses serviteurs les prophètes. Et Israël fut emmené captif loin de son pays en Assyrie, jusqu'à ce jour.
\TextTitle{Jugement sur les étrangers occupant les villes d'Israël}
\VS{24}Le roi d'Assyrie fit venir des gens de Babylone, de Cutha, d'Avva, de Hamath et de Sepharvaïm. Il les fit habiter dans les villes de Samarie, à la place des enfants d'Israël. Ils prirent possession de la Samarie et habitèrent dans ses villes.
\VS{25}Lorsqu'ils commencèrent à y habiter, ils ne craignirent point Yahweh, et Yahweh envoya contre eux des lions, qui les tuaient.
\VS{26}Et on dit au roi d'Assyrie~: Les nations que tu as transportées et fait habiter dans les villes de Samarie ne connaissent pas la manière de servir le dieu du pays, c'est pourquoi il a envoyé contre eux des lions, et voilà, ces lions les tuent, parce qu'ils ne connaissent pas la manière de servir le dieu du pays.
\TextTitle{L'idolâtrie dans les villes occupées}
\VS{27}Alors le roi d'Assyrie donna cet ordre, en disant~: Faites-y aller quelqu'un des prêtres que vous avez emmenés de là en captivité~; qu'il parte pour s'y établir et qu'il leur enseigne la manière de servir le dieu du pays.
\VS{28}Alors l'un des prêtres, qui avaient été emmenés captifs de Samarie vint s'établir à Béthel et leur enseigna comment ils devaient craindre Yahweh.
\VS{29}Mais les nations firent chacune leurs dieux dans les villes qu'elles habitaient et les placèrent dans les maisons des hauts lieux bâties par les Samaritains.
\VS{30}Les gens de Babylone firent Succoth-Benoth, les gens de Cuth firent Nergal, et les gens de Hamath firent Aschima.
\VS{31}Ceux d'Avva firent Nibchaz et Thartak~; ceux de Sepharvaïm brûlaient leurs enfants par le feu à Adrammélec et Anammélec, les dieux de Sepharvaïm.
\VS{32}Toutefois, ils redoutaient Yahweh et ils établirent des prêtres des hauts lieux pris parmi tout le peuple~; ces prêtres offraient pour eux des sacrifices dans les maisons des hauts lieux.
\VS{33}Ils redoutaient Yahweh et en même temps, ils servaient leurs dieux à la manière des nations d'où on les avait transportés.
\VS{34}Et jusqu'à ce jour, ils font encore selon leurs premières coutumes~: Ils ne craignirent point Yahweh, et ils ne se conforment ni à leurs lois et à leurs ordonnances, ni à la loi et aux commandements prescrits par Yahweh Dieu aux enfants de Jacob, qu'il appela du nom d'Israël.
\VS{35}Yahweh avait traité alliance avec eux et leur avait donné cet ordre, en disant~: Vous ne craindrez point d'autres dieux~; et ne vous prosternerez point devant eux~; vous ne les servirez point et vous ne leur offrirez point de sacrifices.
\VS{36}Mais vous craindrez Yahweh, qui vous a fait monter hors du pays d'Egypte avec une grande puissance et à bras étendu~; et vous vous prosternerez devant lui, et vous lui offrirez des sacrifices.
\VS{37}Vous observerez et mettrez toujours en pratique les statuts, les ordonnances, la loi et les commandements, qu'il a écrits pour vous, et vous ne craindrez pas d'autres dieux.
\VS{38}Vous n'oublierez pas l'alliance que j'ai traitée avec vous et vous ne craindrez point d'autres dieux.
\VS{39}Mais vous craindrez Yahweh, votre Dieu, et il vous délivrera de la main de tous vos ennemis.
\VS{40}Ils n'écoutèrent pas et ils firent selon leurs premières coutumes.
\VS{41}Ainsi ces nations-là redoutaient Yahweh et servaient leurs images~; leurs enfants et les enfants de leurs enfants font jusqu'à ce jour ce que firent leurs pères.
\Chap{18}
\TextTitle{Ezéchias règne sur Juda\FTNTT{2 R. 16:20~; 2 Ch. 29:1-31:21.}}
\VerseOne{}La troisième année d'Osée, fils d'Ela, roi d'Israël, Ezéchias, fils d'Achaz, roi de Juda, commença à régner.
\VS{2}Il était âgé de vingt-cinq ans lorsqu'il commença à régner, il régna vingt-neuf ans à Jérusalem. Sa mère s'appelait Abi, fille de Zacharie.
\VS{3}Il fit ce qui est droit aux yeux de Yahweh, entièrement comme avait fait David, son père.
\TextTitle{Mouvement de réveil sous Ezéchias\FTNTT{2 Ch. 29:3-31:21.}}
\VS{4}Il fit disparaître les hauts lieux, mit en pièces les statues, abattit les idoles d'Asherah, et il brisa le serpent d'airain que Moïse avait fait, car les enfants d'Israël avaient jusqu'alors brûlé des parfums devant lui~; ils l'appelaient Nehuschtan.
\VS{5}Il se confia en Yahweh, le Dieu d'Israël~; et parmi tous les rois de Juda qui vinrent après ou qui le précédèrent, il n'y en eut point de semblable à lui.
\VS{6}Il s'attacha à Yahweh, il ne se détourna point de lui et il observa les commandements que Yahweh avait prescrits à Moïse.
\TextTitle{Révolte contre l'Assyrie~; victoire sur les Philistins}
\VS{7}Et Yahweh fut avec Ezéchias, qui réussit dans toutes ses entreprises. Il se révolta contre le roi d'Assyrie et ne lui fut plus assujetti.
\VS{8}Il frappa les Philistins jusqu'à Gaza et ravagea leur territoire depuis les tours des gardes jusqu'aux villes fortes.
\TextTitle{Captivité d'Israël par l'Assyrie\FTNTT{2 R. 17:4-6.}}
\VS{9}La quatrième année du roi Ezéchias, qui était la septième du règne d'Osée, fils d'Ela, roi d'Israël, Salmanasa, roi d'Assyrie, monta contre Samarie et l'assiégea.
\VS{10}Il la prit au bout de trois ans~; la sixième année du règne d'Ezéchias, qui était la neuvième d'Osée, roi d'Israël, Samarie fut prise.
\VS{11}Le roi d'Assyrie emmena Israël en Assyrie et il les établit à Chalach, sur le Chabor, fleuve de Gozan, et dans les villes des Mèdes,
\VS{12}parce qu'ils n'avaient point obéi à la voix de Yahweh, leur Dieu, et qu'ils avaient transgressé son alliance, parce qu'ils n'avaient ni écouté ni mis en pratique tout ce qu'avait ordonné Moïse, serviteur de Yahweh.
\TextTitle{Invasion de Juda par Sanchérib\FTNTT{2 Ch. 32:1-15,30~; Es. 36:1-10.}}
\VS{13}La quatorzième année du roi Ezéchias, Sanchérib, roi d'Assyrie, monta contre toutes les villes fortes de Juda et les prit.
\VS{14}Ezéchias, roi de Juda, envoya dire au roi d'Assyrie à Lakis~: J'ai commis une faute~! Eloigne-toi de moi. Je payerai tout ce que tu m'imposeras. Et le roi d'Assyrie imposa à Ezéchias, roi de Juda, trois cents talents d'argent et trente talents d'or.
\VS{15}Ezéchias donna tout l'argent qui se trouvait dans la maison de Yahweh et dans les trésors de la maison royale.
\VS{16}En ce temps-là, Ezéchias enleva les lames d'or dont il avait couvert les portes et les linteaux du temple de Yahweh, pour les livrer au roi d'Assyrie.
\VS{17}Puis le roi d'Assyrie envoya de Lakis à Jérusalem, vers le roi Ezéchias, Tharthan, Rab-Saris et Rabschaké avec une puissante armée. Ils montèrent et arrivèrent à Jérusalem. Lorsqu'ils furent montés et arrivés~; ils s'arrêtèrent à l'aqueduc de l'étang supérieur, qui est sur le chemin du champ du foulon.
\VS{18}Ils appelèrent le roi tout haut~; alors Eliakim, fils de Hilkija, chef de la maison du roi, Schebna, le secrétaire et Joach, fils d'Asaph, l'archiviste, se rendirent auprès d'eux.
\VS{19}Rabschaké leur dit~: Dites maintenant à Ezéchias~: Ainsi parle le grand roi, le roi d'Assyrie~: Quelle est cette confiance sur laquelle tu t'appuies~?
\VS{20}Tu as dit~: Il faut pour la guerre le conseil et la force. Mais ce ne sont que des paroles. Mais en qui donc as-tu placé ta confiance, pour te rebeller contre moi~?
\VS{21}Voici maintenant, tu l'as placée dans l'Egypte, dans ce roseau cassé, qui pénètre et perce la main de quiconque s'appuie dessus~: Tel est Pharaon, roi d'Egypte, pour tous ceux qui se confient en lui.
\VS{22}Peut-être me direz-vous~: Nous nous confions en Yahweh, notre Dieu, mais n'est-ce pas celui dont Ezéchias a détruit les hauts lieux et les autels, en disant à Juda et à Jérusalem~: Vous vous prosternerez devant cet autel à Jérusalem~?
\VS{23}Maintenant, donne des otages au roi d'Assyrie, mon maître, et je te donnerai deux mille chevaux, si tu peux donner autant de cavaliers pour les monter.
\VS{24}Comment donc repousserais-tu un seul gouverneur d'entre les serviteurs de mon maître~? Mais tu mets ta confiance dans l'Egypte, à cause des chars et des cavaliers.
\VS{25}D'ailleurs, est-ce sans l'ordre de Yahweh que je suis monté contre ce lieu, pour le détruire~? Yahweh m'a dit~: Monte contre ce pays et détruis-le.
\TextTitle{Menaces de Rabschaké\FTNTT{2 Ch. 32:16,18-19~; Es. 36:11-21.}}
\VS{26}Alors Eliakim, fils de Hilkija, Schebna et Joach dirent à Rabschaké~: Nous te prions de parler en araméen à tes serviteurs, car nous le comprenons~; et ne nous parle pas en langue judaïque, aux oreilles du peuple qui est sur la muraille.
\VS{27}Rabschaké leur répondit~: Est-ce à ton maître et à toi que mon maître m'a envoyé dire ces paroles~? Ne m'a-t-il pas envoyé vers les hommes qui se tiennent sur la muraille pour leur dire qu'ils mangeront leurs propres excréments et qu'ils boiront leur urine avec vous~?
\VS{28}Rabschaké, s'étant avancé, cria à haute voix en langue judaïque, il parla et dit~: Ecoutez la parole du grand roi, le roi d'Assyrie.
\VS{29}Ainsi parle le roi~: Qu'Ezéchias ne vous trompe pas, car il ne pourra pas vous délivrer de ma main.
\VS{30}Qu'Ezéchias ne vous amène pas à vous confier en Yahweh, en disant~: Yahweh nous délivrera certainement et cette ville ne sera pas livrée entre les mains du roi d'Assyrie.
\VS{31}N'écoutez pas Ezéchias~; car ainsi parle le roi d'Assyrie~: Faites la paix avec moi et rendez-vous à moi~; et chacun de vous mangera de sa vigne, et de son figuier, et chacun boira de l'eau de sa citerne,
\VS{32}jusqu'à ce que je vienne et que je vous emmène dans un pays comme le vôtre, dans un pays de blé et de bon vin, un pays de pain et de vignes, un pays d'oliviers qui portent de l'huile, et de miel, et vous vivrez et vous ne mourrez pas. Mais n'écoutez pas Ezéchias~; car il pourrait vous séduire, en disant~: Yahweh nous délivrera.
\VS{33}Les dieux des nations ont-ils délivré chacun leur pays de la main du roi d'Assyrie~?
\VS{34}Où sont les dieux de Hamath et d'Arpad~? Où sont les dieux de Sépharvaïm, d'Héna et d'Ivva~? Et même ont-ils délivré Samarie de ma main~?
\VS{35}Parmi tous les dieux de ces pays, quels sont ceux qui ont délivré leur pays de ma main, pour dire que Yahweh délivrera Jérusalem de ma main~?
\VS{36}Le peuple se tut et on ne lui répondit pas un mot~; car le roi avait donné cet ordre~: Vous ne lui répondrez point.
\VS{37}Après cela, Eliakim, fils de Hilkija, chef de la maison du roi, et Schebna le secrétaire, et Joach, fils d'Asaph, l'archiviste, vinrent auprès d'Ezéchias, les vêtements déchirés, et ils lui rapportèrent les paroles de Rabschaké.
\Chap{19}
\TextTitle{Ezéchias demande à Esaïe de consulter Yahweh\FTNTT{2 Ch. 32:20-22~; Es. 36:22-37:5.}}
\VerseOne{}Et il arriva qu'aussitôt que le roi Ezéchias entendit ces choses, il déchira ses vêtements, se couvrit d'un sac et entra dans la maison de Yahweh.
\VS{2}Puis il envoya Eliakim, chef de la maison du roi, et Schebna le secrétaire, et les plus anciens des prêtres, couverts de sacs, vers Esaïe, le prophète, fils d'Amots.
\VS{3}Ils lui dirent~: Ainsi parle Ezéchias~: Ce jour est un jour d'angoisse, de châtiment et d'opprobre~; car les enfants sont près du sein maternel, mais il n'y a point de force pour enfanter.
\VS{4}Peut-être Yahweh, ton Dieu, a-t-il entendu toutes les paroles de Rabschaké, que le roi d'Assyrie, son maître, a envoyé pour blasphémer le Dieu vivant, et peut-être Yahweh, ton Dieu, exercera-t-il ses châtiments à cause des paroles qu'il a entendues. Fais donc une prière pour le reste qui subsiste encore.
\VS{5}Les serviteurs du roi Ezéchias vinrent donc vers Esaïe.
\TextTitle{Réponse de Yahweh\FTNTT{Es. 37:6-7.}}
\VS{6}Et Esaïe leur dit~: Voici ce que vous direz à votre maître~: Ainsi parle Yahweh~: Ne t'effraie point des paroles que tu as entendues, par lesquelles les serviteurs du roi d'Assyrie m'ont blasphémé.
\VS{7}Voici, je vais mettre en lui un esprit, tel que sur une nouvelle qu'il recevra, il retournera dans son pays~; et je le ferai tomber par l'épée dans son pays.
\TextTitle{Défi du roi d'Assyrie au Dieu d'Israël\FTNTT{2 Ch. 32:17~; Es. 37:8-13.}}
\VS{8}Rabschaké s'étant retiré, trouva le roi d'Assyrie qui attaquait Libna, car il avait appris qu'il était parti de Lakis.
\VS{9}Le roi d'Assyrie reçut une nouvelle au sujet de Tirhaka, roi d'Ethiopie~; on lui dit~: Voici, il est sorti pour te combattre. C'est pourquoi le roi d'Assyrie retourna dans son pays, mais il envoya des messagers à Ezéchias, en leur disant~:
\VS{10}Vous parlerez ainsi à Ezéchias, roi de Juda, et lui direz~: Que ton Dieu, en qui tu te confies, ne t'abuse pas en te disant~: Jérusalem ne sera point livrée entre les mains du roi d'Assyrie.
\VS{11}Voici, tu as entendu ce que les rois d'Assyrie ont fait à tous les pays, et comment ils les ont détruits entièrement~; et tu échapperais~?
\VS{12}Les dieux des nations que mes ancêtres ont détruites, savoir de Gozan, de Charan, de Retseph, et des fils d'Eden, qui sont en Telassar, les ont-ils délivrées~?
\VS{13}Où sont le roi de Hamath, le roi d'Arpad, et le roi de la ville de Sepharvaïm, d'Héna et d'Ivva~?
\TextTitle{Ezéchias dans le temple, sa prière à Yawheh\FTNTT{2 Ch. 32:20~; Es. 37:14-20.}}
\VS{14}Quand Ezéchias reçut la lettre de la main des messagers, il la lut. Puis il monta à la maison de Yahweh et la déploya devant Yahweh~;
\VS{15}puis Ezéchias lui adressa cette prière et dit~: Ô Yahweh, Dieu d'Israël~! Qui est assis entre les chérubins, c'est toi qui es le seul Dieu de tous les royaumes de la terre, c'est toi qui as fait les cieux et la terre.
\VS{16}Ô Yahweh~! Incline ton oreille et écoute. Ouvre tes yeux et regarde. Ecoute les paroles de Sanchérib, et de celui qu’il a envoyé pour blasphémer le Dieu vivant.
\VS{17}Il est vrai, ô Yahweh~! Que les rois d'Assyrie ont détruit ces nations et ravagé leurs pays,
\VS{18}et qu'ils ont jeté dans le feu leurs dieux~; mais ils n'étaient pas des dieux, mais des ouvrages de mains d'homme, du bois, et de la pierre, c'est pourquoi ils les ont détruits.
\VS{19}Maintenant donc, ô Yahweh, notre Dieu~! Je te prie, délivre-nous de la main de Sanchérib, afin que tous les royaumes de la terre sachent que c'est toi, ô Yahweh, qui es le seul Dieu.
\TextTitle{Yahweh répond au travers d'Esaïe\FTNTT{Es. 37:21-35.}}
\VS{20}Alors Esaïe, fils d'Amots, envoya dire à Ezéchias~: Ainsi parle Yahweh, le Dieu d'Israël~: Je t'ai exaucé dans ce que tu m'as demandé au sujet de Sanchérib, roi d'Assyrie.
\VS{21}Voici la parole que Yahweh a prononcée contre lui~: Elle te méprise, elle se moque de toi, la fille, vierge de Sion~; elle hoche la tête après toi, la fille de Jérusalem.
\VS{22}Qui as-tu outragé et blasphémé~? Contre qui as-tu élevé la voix~? Tu as porté tes yeux en haut, vers le Saint d'Israël~!
\VS{23}Tu as insulté le Seigneur par le moyen de tes messagers et tu as dit~: J'ai gravi le sommet des montagnes avec la multitude de mes chars, les extrémités du Liban~; je couperai les plus hauts de ses cèdres et les plus beaux de ses cyprès, et j'atteindrai sa dernière cime, la forêt de son verger.
\VS{24}J'ai creusé des sources, après avoir bu les eaux étrangères et je tarirai avec la plante de mes pieds tous les fleuves de l'Egypte.
\VS{25}N'as-tu pas appris que j'ai préparé cette ville déjà dès longtemps, et que dès les temps anciens je l'ai ainsi formée~? Et maintenant l'aurais-je conservée pour être réduite en désolation, et les villes fortes, en monceaux de ruines~?
\VS{26}Il est vrai que leurs habitants sont impuissants, épouvantés et confus~; ils sont devenus comme l'herbe des champs et la tendre verdure, comme le gazon des toits et le blé brûlé avant la formation de sa tige.
\VS{27}Mais je connais ta demeure, ta sortie et ton entrée, et comment tu es furieux contre moi.
\VS{28}Parce que tu es furieux contre moi et que ton insolence est montée à mes oreilles, je mettrai ma boucle à tes narines, et mon mors entre tes lèvres, et je te ferai retourner par le chemin par lequel tu es venu.
\VS{29}Que ceci soit un signe pour toi, ô Ezéchias~: On mangera cette année le produit du grain tombé, et la deuxième année, ce qui croît de soi-même~; mais la troisième année, vous sèmerez et vous moissonnerez, vous planterez des vignes et vous en mangerez le fruit.
\VS{30}Ce qui aura été épargné de la maison de Juda, ce qui sera resté poussera encore des racines par-dessous et produira du fruit par-dessus.
\VS{31}Car il sortira de Jérusalem un reste, et de la montagne de Sion des réchappés. Voilà ce que fera le zèle de Yahweh des armées.
\VS{32}C'est pourquoi ainsi parle Yahweh, sur le roi d'Assyrie~: Il n'entrera point dans cette ville, il n'y lancera aucune flèche, il ne se présentera point contre elle avec le bouclier et il n'élèvera point des retranchements contre elle.
\VS{33}Il s'en retournera par le chemin par lequel il est venu et il n'entrera point dans cette ville, dit Yahweh.
\VS{34}Car je protègerai cette ville, afin de la délivrer, par amour pour moi et par amour pour David, mon serviteur.
\TextTitle{L'ange de Yahweh dans le camp des Assyriens\FTNTT{Es. 37:36-38.}}
\VS{35}Il arriva nuit-là que l'Ange de Yahweh sortit et frappa cent quatre-vingt-cinq mille hommes dans le camp des Assyriens. Et quand on se leva de bon matin, voici, ils étaient tous morts.
\TextTitle{Mort de Sanchérib, roi d'Assyrie\FTNTT{Es. 37:37-38~; 2 Ch. 32:21.}}
\VS{36}Alors Sanchérib, roi d'Assyrie, leva son camp, partit et s'en retourna~; et il resta à Ninive.
\VS{37}Il arriva, comme il était prosterné dans la maison de Nisroc, son dieu, qu'Adrammélec et Scharetser, ses fils, le tuèrent avec l'épée, puis ils se sauvèrent au pays d'Ararat~; et Esar-Haddon, son fils, régna à sa place.
\Chap{20}
\TextTitle{Ezéchias malade puis guéri par Yahweh\FTNTT{2 Ch. 32:24~; Es. 38.}}
\VerseOne{}En ce temps-là, Ezéchias fut malade à la mort. Le prophète Esaïe, fils d'Amots, vint auprès de lui, et lui dit~: Ainsi parle Yahweh~: Donne tes ordres à ta maison, car tu vas mourir et tu ne vivras plus.
\VS{2}Alors Ezéchias tourna son visage contre le mur et fit sa prière à Yahweh, en disant~:
\VS{3}Je te prie, ô Yahweh~! Souviens-toi que j'ai marché devant toi avec fidélité et intégrité de cœur, et que j'ai fait ce qui est agréable à tes yeux~! Et Ezéchias pleura abondamment.
\VS{4}Esaïe n'était pas encore sorti de la cour du milieu, que la parole de Yahweh lui fut adressée, en disant~:
\VS{5}Retourne et dis à Ezéchias, chef de mon peuple~: Ainsi parle Yahweh, le Dieu de David, ton père~: J'ai exaucé ta prière, j'ai vu tes larmes. Voici je te guérirai~; dans trois jours tu monteras à la maison de Yahweh.
\VS{6}J'ajouterai quinze ans à tes jours, je te délivrerai, toi et cette ville, de la main du roi d'Assyrie~; et je protégerai cette ville, par amour pour moi et par amour pour David, mon serviteur.
\VS{7}Puis Esaïe dit~: Prenez une masse de figues sèches. Et ils la prirent et l'appliquèrent sur l'ulcère. Et Ezéchias fut guéri.
\VS{8}Ezéchias avait dit à Esaïe~: A quel signe connaîtrai-je que Yahweh me guérira et qu'au troisième jour, je monterai à la maison de Yahweh~?
\VS{9}Esaïe répondit~: Voici, de la part de Yahweh, le signe auquel tu connaîtras que Yahweh accomplira la parole qu'il a prononcée~: L'ombre s'avancera-t-elle de dix degrés, ou reculera-t-elle en arrière de dix degrés~?
\VS{10}Ezéchias dit~: C'est peu de chose que l'ombre s'avance de dix degrés~; mais plutôt que l'ombre recule en arrière de dix degrés.
\VS{11}Alors Esaïe, le prophète, invoqua Yahweh, qui fit reculer l'ombre de dix degrés sur les degrés d'Achaz, où elle était descendue.
\TextTitle{Visite des ambassadeurs babyloniens~; prophétie sur la captivité babylonienne\FTNTT{2 Ch. 32:25-31~; Es. 39.}}
\VS{12}En ce temps-là, Berodac-Baladan, fils de Baladan, roi de Babylone, envoya une lettre avec un présent à Ezéchias, parce qu'il avait appris la maladie d'Ezéchias.
\VS{13}Et Ezéchias, donna audience aux envoyés et il leur montra tous les lieux où étaient ses objets les plus précieux, l'argent, l'or, les aromates, l'huile précieuse, tout son arsenal et tout ce qui se trouvait dans ses trésors. Il n'y eut rien qu'Ezéchias ne leur montra dans sa maison et dans tous ses domaines.
\VS{14}Esaïe, le prophète, vint ensuite auprès du roi Ezéchias, et lui dit~: Qu'ont dit ces gens-là~? Et d'où sont-ils venus vers toi~? Ezéchias répondit~: Ils sont venus d'un pays très éloigné, ils sont venus de Babylone.
\VS{15}Esaïe dit~: Qu'ont-ils vu dans ta maison~? Et Ezéchias répondit~: Ils ont vu tout ce qui est dans ma maison~; il n'y a rien dans mes trésors que je ne leur aie montré.
\VS{16}Alors Esaïe dit à Ezéchias~: Ecoute la parole de Yahweh~:
\VS{17}Voici, les jours viendront où tout ce qui est dans ta maison et ce que tes pères ont amassé dans leurs trésors jusqu'à ce jour, sera emporté à Babylone~; il n'en restera rien dit Yahweh\FTNT{La déportation des juifs à Babylone~: Voir 2 R. 24-25.}.
\VS{18}On prendra même de tes fils\FTNT{2 R. 24:12~; 2 Ch. 33:11~; Da. 1.} qui seront sortis de toi, que tu auras engendrés, afin qu'ils soient eunuques dans le palais du roi de Babylone.
\VS{19}Ezéchias répondit à Esaïe~: La parole de Yahweh que tu as prononcée est bonne. Et il ajouta~: N'y aura-t-il pas paix et sécurité pendant mes jours~?
\TextTitle{Mort d'Ezéchias~; Manassé règne sur Juda\FTNTT{2 Ch. 32:32-33.}}
\VS{20}Le reste des actions d'Ezéchias, tous ses exploits, et comment il fit l'étang et l'aqueduc par lequel il fit entrer les eaux dans la ville, cela n'est-il pas écrit dans le livre des Chroniques des rois de Juda~?
\VS{21}Ezéchias se coucha avec ses pères. Et Manassé, son fils, régna à sa place.
\Chap{21}
\TextTitle{Abominations et idolâtrie de Manassé\FTNTT{2 Ch. 33:1-9.}}
\VerseOne{}Manassé était âgé de douze ans, lorsqu'il commença à régner. Il régna cinquante-cinq ans à Jérusalem. Sa mère s'appelait Hephtsiba.
\VS{2}Il fit ce qui est mal aux yeux de Yahweh, selon les abominations des nations que Yahweh avait chassées devant les enfants d'Israël.
\VS{3}Car il rebâtit les hauts lieux qu'Ezéchias, son père, avait détruits, et redressa des autels à Baal, il fit une idole d'Asherah\FTNT{Voir commentaire en Jg. 2:13.}, comme avait fait Achab, roi d'Israël, il se prosterna devant toute l'armée des cieux et les servit.
\VS{4}Il bâtit aussi des autels dans la maison de Yahweh, quoique Yahweh ait dit~: C'est dans Jérusalem que j'établirai mon nom.
\VS{5}Il bâtit des autels à toute l'armée des cieux dans les deux parvis de la maison de Yahweh.
\VS{6}Il fit aussi passer son fils par le feu, il pratiquait l'astrologie et la divination, il établit des gens qui évoquaient les esprits des morts et qui prédisaient l'avenir. Il fit de plus en plus ce qui est mal aux yeux de Yahweh pour l'irriter.
\VS{7}Il plaça aussi l'idole d'Asherah qu'il avait faite, dans la maison de laquelle Yahweh avait dit à David, et à Salomon, son fils~: C'est dans cette maison, et c'est dans Jérusalem, que j'ai choisie parmi toutes les tribus d'Israël, que je veux à toujours établir mon nom.
\VS{8}Je ne ferai plus errer le pied d'Israël hors de cette terre que j'ai donnée à leurs pères, pourvu seulement qu'ils aient soin de mettre en pratique tout ce que je leur ai ordonné et toute la loi que Moïse, mon serviteur, leur a prescrite.
\VS{9}Mais ils n'obéirent point~; car Manassé les fit s'égarer, jusqu'à faire le mal plus que les nations que Yahweh avait exterminées devant les enfants d'Israël.
\TextTitle{Jugement de Yahweh contre Juda~; mort d'Ezéchias\FTNTT{2 Ch. 33:10-20.}}
\VS{10}Alors Yahweh parla par ses serviteurs les prophètes, en disant~:
\VS{11}Parce que Manassé, roi de Juda, a commis ces abominations, parce qu'il a fait pire que tout ce qu'avaient fait avant lui les Amoréens, et parce qu'il a aussi fait pécher Juda par ses idoles,
\VS{12}à cause de cela, Yahweh, le Dieu d'Israël, dit~: Voici, je vais faire venir sur Jérusalem et sur Juda des malheurs qui étourdiront les oreilles de quiconque en entendra parler.
\VS{13}Car j'étendrai sur Jérusalem le cordeau de Samarie, et le niveau de la maison d'Achab, et je nettoierai Jérusalem comme un plat qu'on nettoie, et qu’on renverse sur son fond après l'avoir nettoyé.
\VS{14}J'abandonnerai le reste de mon héritage, et je les livrerai entre les mains de leurs ennemis~; et ils seront le butin et la proie de tous leurs ennemis~;
\VS{15}parce qu'ils ont fait ce qui est mal à mes yeux, et qu'ils m'ont irrité depuis le jour où leurs pères sont sortis d'Egypte, jusqu'à ce jour.
\TextTitle{Meurtres de Manassé~; sa mort\FTNTT{2 Ch. 33:11-20.}}
\VS{16}Manassé répandit aussi beaucoup de sang innocent, jusqu'à en remplir Jérusalem d'un bout à l'autre, outre son péché par lequel il fit pécher Juda en faisant ce qui est mal aux yeux de Yahweh.
\VS{17}Le reste des actions de Manassé, tout ce qu'il a fait~; et les péchés auxquels il se livra, cela n'est-il pas écrit dans le livre des Chroniques des rois de Juda~?
\VS{18}Manassé se coucha avec ses pères, et il fut enseveli dans le jardin de sa maison, dans le jardin d'Uzza. Amon, son fils, régna à sa place.
\TextTitle{Amon règne sur Juda~; sa mort\FTNTT{2 Ch. 33:20-25.}}
\VS{19}Amon était âgé de vingt-deux ans lorsqu'il commença à régner. Il régna deux ans à Jérusalem. Sa mère s'appelait Meschullémeth, fille de Haruts, de Jotba.
\VS{20}Il fit ce qui est mal aux yeux de Yahweh, comme avait fait Manassé, son père.
\VS{21}Car il marcha dans toute la voie où avait marché son père, il servit les idoles que son père avait servies et se prosterna devant elles.
\VS{22}Il abandonna Yahweh, le Dieu de ses pères et il ne marcha point dans la voie de Yahweh.
\TextTitle{Josias, roi de Juda\FTNTT{2 Ch. 33:24-25.}}
\VS{23}Les serviteurs d'Amon firent une conspiration contre lui et le tuèrent dans sa maison.
\VS{24}Mais le peuple du pays frappa tous ceux qui avaient conspiré contre le roi Amon~; et ils établirent Josias, son fils, roi à sa place.
\VS{25}Le reste des actions d'Amon, ce qu'il a fait, cela n'est-il pas écrit dans le livre des Chroniques des rois de Juda~?
\VS{26}On l'ensevelit dans son sépulcre, dans le jardin d'Uzza. Et Josias, son fils, régna à sa place.
\Chap{22}
\TextTitle{Droiture de Josias~; réparations dans le temple\FTNTT{2 Ch. 34:2-13.}}
\VerseOne{}Josias était âgé de huit ans lorsqu'il commença à régner. Il régna trente et un ans à Jérusalem. Sa mère s'appelait Jedida, fille d'Adaja, de Botskath.
\VS{2}Il fit ce qui est droit aux yeux de Yahweh et il marcha dans toute la voie de David, son père~; il ne s'en détourna ni à droite ni à gauche.
\VS{3}La dix-huitième année du roi Josias, le roi envoya dans la maison de Yahweh, Schaphan, le secrétaire, fils d'Atsalia, fils de Meschullam.
\VS{4}Il lui dit~: Monte vers Hilkija, le grand-prêtre, et dis-lui d'amasser l'argent qui a été apporté dans la maison de Yahweh et que ceux qui ont la garde du seuil ont recueilli du peuple.
\VS{5}On remettra cet argent entre les mains de ceux qui sont chargés de faire exécuter l'ouvrage dans la maison de Yahweh. Et ils l'emploieront pour ceux qui travaillent dans la maison de Yahweh, pour réparer les brèches de la maison,
\VS{6}pour les charpentiers, les architectes et les maçons, pour les achats du bois et des pierres de taille pour réparer la maison.
\VS{7}Mais on ne leur demandera pas de comptes pour l'argent remis entre leurs mains, parce qu'ils agissent fidèlement.
\TextTitle{Découverte et lecture du livre de la loi\FTNTT{2 Ch. 34:14-19.}}
\VS{8}Alors Hilkija, le grand-prêtre, dit à Schaphan, le secrétaire~: J'ai trouvé le livre de la loi dans la maison de Yahweh. Et Hilkija donna ce livre à Schaphan qui le lut.
\VS{9}Schaphan, le secrétaire, alla vers le roi et lui rapporta la chose, et dit~: Tes serviteurs ont amassé l'argent qui se trouvait dans la maison et l'ont remis entre les mains de ceux qui sont chargés de faire l'ouvrage dans la maison de Yahweh.
\VS{10}Schaphan, le secrétaire, dit aussi au roi~: Le prêtre Hilkija m'a donné un livre. Et Schaphan le lut devant le roi.
\VS{11}Lorsque le roi eut entendu les paroles du livre de la loi, il déchira ses vêtements.
\TextTitle{Annonce du jugement de Yahweh par Hulda\FTNTT{2 Ch. 34:20-28.}}
\VS{12}Il donna cet ordre au prêtre Hilkija, à Achikam, fils de Schaphan, à Acbor, fils de Michée, à Schaphan, le secrétaire, et à Asaja, serviteur du roi~:
\VS{13}Allez, consultez Yahweh pour moi, pour le peuple et pour tout Juda, au sujet des paroles de ce livre qui a été trouvé~; car grande est la colère de Yahweh, qui s'est enflammée contre nous, parce que nos pères n'ont point obéi aux paroles de ce livre et n'ont pas mis en pratique tout ce qui nous y est prescrit.
\VS{14}Le prêtre Hilkija, Achikam, Acbor, Schaphan et Asaja, allèrent auprès de la prophétesse Hulda, femme de Schallum, fils de Thikva, fils de Harhas, gardien des vêtements. Elle habitait dans un autre quartier de Jérusalem.
\TextTitle{Yahweh rassure Josias par la prophétesse Hulda\FTNTT{2 Ch. 34:22-28.}}
\VS{15}Après qu'ils eurent parlé avec elle, elle leur répondit~: Ainsi parle Yahweh, le Dieu d'Israël~: Dites à l'homme qui vous a envoyé vers moi~:
\VS{16}Ainsi parle Yahweh~: Voici, je vais faire venir le malheur sur cette ville et sur ses habitants, selon toutes les paroles du livre que le roi de Juda a lu.
\VS{17}Parce qu'ils m'ont abandonné et qu'ils ont offert des parfums à d'autres dieux, pour m'irriter par toutes les actions de leurs mains, ma colère s'est enflammée contre cette ville et elle ne s'éteindra point.
\VS{18}Mais quant au roi de Juda qui vous a envoyé pour consulter Yahweh, vous lui direz~: Ainsi parle Yahweh, le Dieu d'Israël, au sujet des paroles que tu as entendues~:
\VS{19}Parce que ton cœur a été touché, et que tu t'es humilié devant Yahweh en entendant ce que j'ai prononcé contre cette ville et contre ses habitants, qui seront un objet d'épouvante et de malédiction, et parce que tu as déchiré tes vêtements, et que tu as pleuré devant moi, je t'ai exaucé, dit Yahweh.
\VS{20}C'est pourquoi voici, je vais te recueillir auprès de tes pères, et tu seras recueilli dans ton sépulcre en paix, et tes yeux ne verront point tout ce mal que je vais faire venir sur cette ville. Ils rapportèrent toutes ces paroles au roi.
\Chap{23}
\TextTitle{Le livre de la loi lu au peuple\FTNTT{2 Ch. 34:29-30.}}
\VerseOne{}Alors, le roi Josias, fit assembler auprès de lui tous les anciens de Juda et de Jérusalem.
\VS{2}Le roi monta à la maison de Yahweh, avec tous les hommes de Juda, tous les habitants de Jérusalem, les prêtres, les prophètes, et tout le peuple, depuis le plus petit jusqu'au plus grand. Il lut devant eux toutes les paroles du livre de l'alliance, qui avait été trouvé dans la maison de Yahweh.
\TextTitle{Engagement de Josias et du peuple à suivre la loi de Yahweh\FTNTT{2 Ch. 34:31-32.}}
\VS{3}Le roi se tenait sur l'estrade et il traita alliance devant Yahweh, s'engageant à suivre Yahweh, à observer ses ordonnances, ses préceptes et ses lois, de tout son cœur, à persévérer dans les paroles de cette alliance, écrites dans ce livre. Et tout le peuple entra dans cette alliance.
\TextTitle{Josias débarrasse Juda de tous ses faux dieux\FTNTT{2 Ch. 34:33.}}
\VS{4}Alors le roi donna cet ordre à Hilkija, le grand-prêtre, aux prêtres du second ordre et à ceux qui gardaient le seuil, de sortir hors du temple de Yahweh tous les ustensiles qui avaient été faits pour Baal\FTNT{Voir commentaire en Jg. 2:12.}, pour Asherah\FTNT{Voir commentaire en Jg. 2:13.}, et pour toute l'armée des cieux~; et il les brûla hors de Jérusalem, dans les champs de Cédron, et en fit porter la poussière à Béthel.
\VS{5}Il chassa les prêtres des idoles, que les rois de Juda avaient établis pour brûler des parfums sur les hauts lieux, dans les villes de Juda et aux environs de Jérusalem, et ceux qui offraient des parfums à Baal, au soleil, à la lune, au zodiaque et à toute l'armée des cieux.
\VS{6}Il sortit de la maison de Yahweh l'idole d'Asherah, qu'il transporta hors de Jérusalem vers le torrent de Cédron~; il la brûla au torrent de Cédron et la réduisit en poudre, et il en jeta la poussière sur le sépulcre des fils du peuple.
\VS{7}Ensuite, il démolit les maisons des prostituées qui étaient dans la maison de Yahweh, où les femmes tissaient des tentes pour Asherah.
\VS{8}Il fit venir des villes de Juda tous les prêtres~; il profana les hauts lieux où les prêtres brûlaient des parfums, depuis Guéba jusqu'à Beer-Schéba~; il renversa les hauts lieux des portes, celui qui était à l'entrée de la porte de Josué, chef de la ville, et celui qui était à gauche de la porte de la ville.
\VS{9}Toutefois, les prêtres des hauts lieux ne montaient pas à l'autel de Yahweh à Jérusalem, mais ils mangeaient des pains sans levain parmi leurs frères.
\VS{10}Le roi profana aussi Topheth, dans la vallée des fils de Hinnom, afin que personne ne fasse plus passer son fils ou sa fille par le feu, en l'honneur de Moloc\FTNT{Lé. 20:2-3.}.
\VS{11}Il fit disparaître de l'entrée de la maison de Yahweh les chevaux que les rois de Juda avaient consacrés au soleil, près de la chambre de l'eunuque Nethan-Mélec, situé à Parvarim, et il brûla au feu les chars du soleil.
\VS{12}Le roi démolit les autels qui étaient sur le toit de la chambre haute d'Achaz, que les rois de Juda avaient faits et les autels que Manassé avait faits dans les deux parvis de la maison de Yahweh~; après les avoir brisés et enlevés de là, il en jeta la poussière dans le torrent de Cédron.
\VS{13}Le roi profana aussi les hauts lieux qui étaient en face de Jérusalem, sur la droite de la montagne de perdition, que Salomon, roi d'Israël, avait bâtis à Astarté, l'abomination des Sidoniens, à Kemosch, l'abomination des Moabites, et à Milcom, l'abomination des fils d'Ammon.
\VS{14}Il brisa aussi les statues, et abattit les Asherah, et il remplit d'ossements d'hommes les lieux où elles étaient.
\VS{15}Il renversa l'autel qui était à Béthel et le haut lieu qu'avait fait Jéroboam, fils de Nebath, qui avait fait pécher Israël~; il brûla le haut lieu et le réduisit en poudre, et il brûla l'Asherah.
\VS{16}Josias s'étant tourné et ayant vu les sépulcres qui étaient là dans la montagne, envoya prendre les ossements des sépulcres, et il les brûla sur l'autel et le profana, selon la parole de Yahweh prononcée à haute voix par l'homme de Dieu.
\VS{17}Le roi dit~: Quel est ce monument que je vois~? Et les hommes de la ville lui répondirent~: C'est le sépulcre de l'homme de Dieu qui est venu de Juda qui a crié contre l'autel de Béthel ces choses que tu as accomplies.
\VS{18}Et il dit~: Laissez-le~; que personne ne remue ses os~! Ils conservèrent ainsi ses os, avec les os du prophète qui était venu de Samarie.
\VS{19}Josias fit encore disparaître toutes les maisons des hauts lieux, qui étaient dans les villes de Samarie, et qu'avaient faites les rois d'Israël pour irriter Yahweh~; et il fit à leur égard entièrement comme il avait fait à Béthel.
\VS{20}Il immola sur les autels tous les prêtres des hauts lieux qui étaient là, et il y brûla des ossements d'hommes. Puis il retourna à Jérusalem.
\TextTitle{Josias rétablit la fête de la Pâque\FTNTT{2 Ch. 35:1-19.}}
\VS{21}Alors le roi donna cet ordre à tout le peuple, en disant~: Célébrez la Pâque en l'honneur de Yahweh, votre Dieu, comme il est écrit dans le livre de cette alliance\FTNT{Jésus-Christ est notre Pâque. Voir Ex. 12 et 1 Co. 5:7.}.
\VS{22}Aucune Pâque pareille à celle-ci n'avait été célébrée depuis le temps où les juges jugeaient Israël et pendant tous les jours des rois d'Israël et des rois de Juda.
\VS{23}Ce fut la dix-huitième année du roi Josias qu'on célébra cette Pâque en l'honneur de Yahweh à Jérusalem.
\VS{24}Josias, extermina aussi, ceux qui évoquaient les esprits des morts et les devins, les théraphim, les idoles, et toutes les abominations qui se voyaient dans le pays de Juda et à Jérusalem, afin de mettre en pratique les paroles de la loi, écrites dans le livre que Hilkija, le prêtre, avait trouvé dans la maison de Yahweh.
\TextTitle{Témoignage de Josias~; confirmation du jugement de Yahweh}
\VS{25}Avant Josias, il n'y eut point de roi qui, comme lui, revienne à Yahweh de tout son cœur, de toute son âme et de toute sa force, selon toute la loi de Moïse~; et après lui, il n'en a point paru de semblable.
\VS{26}Toutefois, Yahweh ne se détourna point de l'ardeur de sa grande colère dont il était enflammé contre Juda, à cause de tout ce que Manassé avait fait pour l'irriter.
\VS{27}Et Yahweh dit~: J'ôterai Juda de devant ma face, comme j'ai ôté Israël, et je rejetterai cette ville de Jérusalem que j'avais choisie, et la maison de laquelle j'avais dit~: Là sera mon Nom.
\VS{28}Le reste des actions de Josias, tout ce qu'il a fait, cela n'est-il pas écrit dans le livre des Chroniques des rois de Juda~?
\TextTitle{Mort de Josias~; Joachaz règne sur Juda\FTNTT{2 Ch. 35:20-27~; 2 Ch. 36:1-2.}}
\VS{29}De son temps, Pharaon Néco, roi d'Egypte, monta contre le roi d'Assyrie, vers le fleuve d'Euphrate. Le roi Josias s'en alla au-devant de lui~; mais dès que pharaon le vit, il le tua à Meguiddo.
\VS{30}Ses serviteurs l'emportèrent mort sur un char~; ils l'amenèrent de Meguiddo à Jérusalem et l'ensevelirent dans son sépulcre. Et le peuple du pays prit Joachaz, fils de Josias, ils l'oignirent et l'établirent roi à la place de son père.
\TextTitle{Joachaz mis en prison par Pharaon\FTNTT{2 Ch. 36:3.}}
\VS{31}Joachaz était âgé de vingt-trois ans, lorsqu'il commença à régner. Il régna trois mois à Jérusalem. Sa mère s'appelait Hamuthal, fille de Jérémie, de Libna.
\VS{32}Il fit ce qui est mal aux yeux de Yahweh, entièrement comme avaient fait ses pères.
\VS{33}Et pharaon Néco l'emprisonna à Ribla, dans le pays de Hamath, afin qu'il ne règne plus à Jérusalem~; et il imposa sur le pays un tribut de cent talents d'argent et d'un talent d'or.
\TextTitle{Pharaon établit Jojakim roi de Juda\FTNTT{2 Ch. 36:4-5.}}
\VS{34}Puis pharaon Néco établit roi Eliakim, fils de Josias, à la place de Josias, son père, et il changea son nom en celui de Jojakim. Il prit Joachaz, qui alla en Egypte, où il mourut.
\VS{35}Jojakim donna cet argent et cet or à pharaon~; mais il taxa le pays pour fournir cet argent, selon l'ordre de pharaon~; il détermina la part de chacun et exigea du peuple du pays l'argent et l'or qu'il devait livrer à pharaon Néco.
\VS{36}Jojakim était âgé de vingt-cinq ans lorsqu'il commença à régner. Il régna onze ans à Jérusalem. Sa mère s'appelait Zebudda, fille de Pedaja, de Ruma.
\VS{37}Il fit ce qui est mal aux yeux de Yahweh, entièrement comme avaient fait ses pères.
\Chap{24}
\TextTitle{Asservissement de Jojakim au roi de Babylone~; destruction de Juda\FTNTT{2 Ch. 36:6-7.}}
\VerseOne{}De son temps, Nebucadnetsar, roi de Babylone, monta contre Jojakim, et Jojakim lui fut asservi pendant trois ans~; mais il se révolta de nouveau contre lui.
\VS{2}Alors Yahweh envoya contre Jojakim des troupes de Chaldéens, des armées de Syriens, des troupes de Moabites et des troupes des fils d'Ammon~; il les envoya contre Juda, pour le détruire, selon la parole que Yahweh avait prononcée par ses serviteurs les prophètes.
\VS{3}Cela arriva uniquement sur l'ordre de Yahweh, qui voulait ôter Juda de devant sa face, à cause de tous les péchés commis par Manassé,
\VS{4}et à cause aussi du sang innocent qu'il avait répandu, et dont il avait rempli Jérusalem. C'est pourquoi Yahweh ne voulut point lui pardonner.
\TextTitle{Mort de Jojakim~; Jojakin règne sur Juda\FTNTT{2 Ch. 36:8-9.}}
\VS{5}Le reste des actions de Jojakim et tout ce qu'il a fait, cela n'est-il pas écrit dans le livre des Chroniques des rois de Juda~?
\VS{6}Ainsi Jojakim se coucha avec ses pères. Et Jojakin, son fils, régna à sa place.
\VS{7}Le roi d'Egypte ne sortit plus de son pays, parce que le roi de Babylone avait pris tout ce qui était au roi d'Egypte, depuis le torrent d'Egypte jusqu'au fleuve d'Euphrate.
\VS{8}Jojakin était âgé de dix-huit ans lorsqu'il commença à régner. Il régna trois mois à Jérusalem. Sa mère s'appelait Nehuschtha, fille d'Elnathan, de Jérusalem.
\VS{9}Il fit ce qui est mal aux yeux de Yahweh, entièrement comme avait fait son père.
\TextTitle{Jérusalem et son roi en captivité à Babylone~; les pauvres restent\FTNTT{2 Ch. 36:10.}}
\VS{10}En ce temps-là, les serviteurs de Nebucadnetsar, roi de Babylone, montèrent contre Jérusalem, et la ville fut assiégée.
\VS{11}Nebucadnetsar, roi de Babylone, arriva devant la ville, pendant que ses serviteurs l'assiégeaient.
\VS{12}Alors Jojakin, roi de Juda, se rendit vers le roi de Babylone, avec sa mère, ses serviteurs, ses chefs et ses eunuques. Et le roi de Babylone le fit prisonnier, la huitième année de son règne.
\VS{13}Il emporta de là, tous les trésors de la maison de Yahweh et les trésors de la maison royale~; et il mit en pièces tous les ustensiles d'or que Salomon, roi d'Israël, avait faits pour le temple de Yahweh, comme Yahweh l'avait ordonné.
\VS{14}Il emmena en captivité tout Jérusalem\FTNT{Première déportation~: 2 R. 24:1-4 et 2 Ch. 36:6-7. La première déportation eut lieu en 597 av. J.-C. pendant le règne de Jojakim, roi de Juda. Les premiers exilés furent installés dans la région du fleuve Kebar (Ez. 1:1-3), un canal de 90 km de long reliant l'Euphrate au nord de Babylone au même fleuve au sud d'Ur en Chaldée. Jérémie savait que leur séjour à l'étranger serait long. Il avait prophétisé qu'il durerait soixante-dix ans (Jé. 25:1~; Jé. 25:11-12) et leur conseilla de se construire des maisons, de cultiver des jardins et de se multiplier (Jé. 29). Daniel et ses compagnons furent déportés à Babylone lors de la première déportation (Da. 1). Daniel fut déporté environ huit ans avant Ezéchiel.}, à savoir, tous les chefs, et tous les vaillants hommes de guerre, au nombre de dix mille captifs, avec les charpentiers et les serruriers, de sorte qu'il ne resta plus que le peuple pauvre du pays.
\VS{15}Ainsi il transporta Jojakin à Babylone, avec la mère du roi, les femmes du roi et ses eunuques. Il emmena captifs à Babylone tous les grands du pays, de Jérusalem à Babylone,
\VS{16}avec tous les guerriers au nombre de sept mille, les charpentiers, les serruriers au nombre de mille, tous les hommes vaillants et propres à la guerre. Le roi de Babylone les emmena captifs à Babylone.
\TextTitle{Nebucadnetsar établit Sédécias roi de Juda\FTNTT{2 Ch. 36:10-12.}}
\VS{17}Et le roi de Babylone établit roi, à la place de Jojakin, Matthania, son oncle, et il changea son nom en celui de Sédécias.
\VS{18}Sédécias était âgé de vingt et un ans lorsqu'il commença à régner. Il régna onze ans à Jérusalem. Sa mère s'appelait Hamuthal, fille de Jérémie, de Libna.
\VS{19}Il fit ce qui est mal aux yeux de Yahweh, entièrement comme avait fait Jojakim.
\TextTitle{Sédécias se révolte\FTNTT{2 Ch. 36:13-16.}}
\VS{20}Cela arriva à cause de la colère de Yahweh contre Jérusalem et contre Juda, qu'il voulait rejeter de devant sa face. Et Sédécias se révolta contre le roi de Babylone.
\Chap{25}
\TextTitle{Siège de Jérusalem\FTNTT{Jé. 39:1.}}
\VerseOne{}Et il arriva dans la neuvième année du règne de Sédécias, le dixième jour du dixième mois, que Nebucadnetsar\FTNT{Jérusalem fut assiégée pendant deux ans. Lors de ce siège, des femmes juives faisaient cuire leurs enfants pour les consommer (La. 2:20~; La. 4:10).}, roi de Babylone, vint avec toute son armée contre Jérusalem~; il campa devant elle et éleva des retranchements tout autour.
\VS{2}La ville fut assiégée jusqu'à la onzième année du roi Sédécias.
\VS{3}Le neuvième jour du 4ème mois, la famine\FTNT{La. 4:10.} augmenta dans la ville, de sorte qu'il n'y avait pas de pain pour le peuple du pays.
\TextTitle{Sédécias lié et emmené à Babylone\FTNTT{Jé. 39:2-7.}}
\VS{4}Alors la brèche fut faite à la ville~; et tous les gens de guerre s'enfuirent de nuit par le chemin de la porte entre les deux murailles près du jardin du roi, pendant que les Chaldéens environnaient la ville. Les fuyards et le roi prirent le chemin de la plaine.
\VS{5}Mais l'armée des Chaldéens poursuivit le roi et l'atteignit dans les plaines de Jéricho, et toute son armée se dispersa loin de lui.
\VS{6}Ils saisirent donc le roi et le firent monter vers le roi de Babylone à Ribla~; et l'on prononça contre lui un jugement.
\VS{7}Et on égorgea les fils de Sédécias en sa présence~; puis on creva les yeux à Sédécias, et on le lia de doubles chaînes d'airain, et on le mena à Babylone.
\TextTitle{Destruction de Jérusalem, du temple et des murailles\FTNTT{2 Ch. 36:17-21~; Jé. 39~:8-10.}}
\VS{8}Le septième jour du cinquième mois, c'était la dix-neuvième année du roi Nebucadnetsar, roi de Babylone, Nebuzaradan, chef des gardes, serviteur du roi de Babylone,\FTNT{Troisième déportation~: Le temple fut brûlé, la ville de Jérusalem fut totalement rasée et ses habitants furent déportés (De. 28:49-68). Contrairement à ce que l'on pense, il y a eu d'autres déportations. Voir Jé. 52.} entra dans Jérusalem.
\VS{9}Il brûla la maison de Yahweh, la maison royale et toutes les maisons de Jérusalem~; il brûla par le feu toutes les grandes maisons.
\VS{10}Toute l'armée des Chaldéens, qui était avec le chef des gardes, démolit les murailles qui entouraient Jérusalem.
\VS{11}Et Nebuzaradan, chef des gardes, emmena captifs le reste du peuple, ceux qui étaient restés dans la ville, ceux qui s'étaient rendus au roi de Babylone et le reste de la multitude.
\VS{12}Cependant le chef des gardes laissa quelques-uns des plus pauvres du pays comme vignerons et comme laboureurs.
\VS{13}Les Chaldéens brisèrent les colonnes d'airain qui étaient dans la maison de Yahweh, les bases, la mer d'airain qui était dans la maison de Yahweh, et ils en emportèrent l'airain à Babylone.
\VS{14}Ils prirent aussi les cendriers, les pelles, les couteaux, les tasses et tous les ustensiles d'airain avec lesquels on faisait le service.
\VS{15}Le chef des gardes emporta aussi les encensoirs et les coupes, ce qui était d'or et ce qui était d'argent.
\VS{16}Les deux colonnes, la mer et les bases, que Salomon avait faits pour la maison de Yahweh, tous ces ustensiles d'airain avaient un poids inconnu.
\VS{17}La hauteur d'une colonne était de dix-huit coudées, et il y avait au-dessus un chapiteau d'airain dont la hauteur était de trois coudées~; autour du chapiteau il y avait un treillis et des grenades, le tout d'airain~; il en était de même pour la seconde colonne avec le treillis.
\VS{18}Le chef des gardes emmena aussi Seraja, le premier prêtre, et Sophonie, le second prêtre, et les trois gardiens du seuil.
\VS{19}Et dans la ville, il prit un eunuque qui avait sous son commandement des hommes de guerre, cinq hommes de ceux qui voyaient la face du roi et qui furent trouvés dans la ville, il prit aussi le secrétaire du chef de l'armée qui était chargé d'enrôler le peuple du pays, et soixante hommes du peuple du pays qui se trouvaient dans la ville.
\VS{20}Nebuzaradan, chef des gardes, les prit et les conduisit vers le roi de Babylone à Ribla.
\VS{21}Le roi de Babylone les frappa, et les fit mourir à Ribla, dans le pays de Hamath. Ainsi Juda fut transporté captif hors de sa terre.
\TextTitle{Guedalia nommé gouverneur de Juda\FTNTT{Jé. 40:7-11.}}
\VS{22}Nebucadnetsar, roi de Babylone, plaça le reste du peuple, qu'il laissa dans le pays de Juda, sous le commandement de Guedalia, fils d'Achikam, fils de Schaphan.
\VS{23}Lorsque tous les chefs des troupes et leurs hommes, eurent appris que le roi de Babylone avait établi Guedalia pour gouverneur, ils allèrent trouver Guedalia à Mitspa, à savoir Ismaël, fils de Nethania, Jochanan, fils de Karéach, Seraja, fils de Thanhumeth, de Nethopha, Jaazania, fils du Maacathien, eux et leurs hommes.
\VS{24}Guedalia leur jura, à eux et à leurs hommes, et leur dit~: Ne craignez pas d'être serviteurs des Chaldéens~; demeurez dans le pays et servez le roi de Babylone, et vous vous en trouverez bien.
\TextTitle{Fuite du peuple en Egypte\FTNTT{Jé. 41:1-3~; Jé. 43:4-7.}}
\VS{25}Mais il arriva au septième mois, qu'Ismaël, fils de Nethania, fils d'Elischama, qui était de race royale, vint, accompagné de dix hommes, et ils frappèrent mortellement Guedalia, ainsi que les Juifs et les Chaldéens qui étaient avec lui à Mitspa.
\VS{26}Alors tout le peuple, depuis le plus petit jusqu'au plus grand, avec les chefs des troupes, se levèrent et s'en allèrent en Egypte, parce qu'ils avaient peur des Chaldéens.
\TextTitle{Jojakin à la table du roi de Babylone\FTNTT{Jé. 52:31-34.}}
\VS{27}La trente-septième année de la captivité de Jojakin, roi de Juda, le vingt-septième jour du douzième mois, Evil-Merodac, roi de Babylone, dans la première année de son règne, releva la tête de Jojakin, roi de Juda et le tira de prison.
\VS{28}Il lui parla avec bonté et il mit son trône au-dessus du trône des rois qui étaient avec lui à Babylone.
\VS{29}Il lui fit changer ses vêtements de prison, et Jojakin mangea du pain tout le temps de sa vie en sa présence.
\VS{30}Et quant à son entretien, un entretien perpétuel, lui fut accordé par le roi pour chaque jour, tous les jours de sa vie.
\PPE{}
\end{multicols}

%\clearpage
\ShortTitle{Esaïe}\BookTitle{Esaïe}\BFont
\noindent\hrulefill
{\footnotesize
\textit{
\bigskip
{\centering{}
\\Auteur : Esaïe
\\(Heb. : Yesha'yah)
\\Signification : YAHWEH a sauvé
\\Thème : Le Messie d'Israël
\\Date de rédaction : 8\up{ème} siècle av. J.-C.\\}
}
%\bigskip
\textit{
\\Prophète en Israël, Esaïe fut une figure marquante en raison du contenu et de l'impact de son message. Véritable porte-parole de Dieu, il parla de la ruine morale d'Israël, de la déportation à Babylone et des jugements de Dieu sur son peuple. Il prophétisa également sur le retour de l'exil, la restauration finale et la reconstruction de Jérusalem. Plus qu'aucun autre livre, les écrits d'Esaïe annoncent clairement la naissance du Messie, son service, sa mission rédemptrice, son sacrifice et son futur règne millénaire. 
%\bigskip
\\L'autorité et l'exactitude de ses prophéties ont été une source d'édification au fil des siècles.\bigskip
}
}
\par\nobreak\noindent\hrulefill
\begin{multicols}{2}
\Chap{1}
\TextTitle{Prophéties concernant Juda}
\VerseOne{}La vision d'Esaïe, fils d'Amots, qu'il a vue touchant Juda et Jérusalem, au jour d'Ozias, de Jotham, d'Achaz, et d'Ezéchias, rois de Juda.
\VS{2}Cieux, écoutez ! Et toi, terre, prête l'oreille ! Car Yahweh parle. J'ai nourri des enfants, je les ai élevés, mais ils se sont rebellés contre moi.
\VS{3}Le bœuf connaît son possesseur, et l'âne la crèche de son maître, mais Israël n'a point de connaissance, mon peuple n'a point d'intelligence.
\VS{4}Ah! Nation pécheresse, peuple chargé d'iniquités, race de gens méchants, enfants qui ne font que se corrompre ! Ils ont abandonné Yahweh, ils ont irrité par leur mépris le Saint d'Israël, ils se sont retirés en arrière.
\VS{5}Pourquoi serez-vous encore frappés ? Vous ajouterez la révolte ! La tête entière est malade, et tout le cœur est languissant.
\VS{6}Depuis la plante du pied jusqu'à la tête, il n'y a rien de sain en lui : Il n'y a que blessures, meurtrissures et plaies pourries, qui n'ont été ni nettoyées, ni bandées, et dont aucune n'a été adoucie par l'huile.
\VS{7}Votre pays n'est que désolation, et vos villes sont en feu ; des étrangers dévorent votre terre sous vos yeux, et cette désolation est comme un bouleversement fait par des étrangers.
\VS{8}Car la fille de Sion est restée comme une cabane dans une vigne, comme une cabane dans un champ de concombres, comme une ville assiégée.
\VS{9}Si Yahweh des armées ne nous avait pas laissé un petit reste, qui est même bien peu, nous serions comme Sodome, nous ressemblerions à Gomorrhe.
\TextTitle{Yahweh rejette la religiosité et recherche la justice}
\VS{10}Ecoutez la parole de Yahweh, chefs de Sodome, prêtez l'oreille à la loi de notre Dieu, peuple de Gomorrhe !
\VS{11}Qu'ai-je à faire, dit Yahweh, de la multitude de vos sacrifices ? Je suis rassasié des holocaustes de béliers et de la graisse des veaux ; je ne prends point plaisir au sang des taureaux, ni des agneaux, ni des boucs\FTNT{1 S. 15:22 ; Os. 8:13 ; Mt. 9:13.}.
\VS{12}Quand vous entrez pour vous présenter devant ma face, qui a requis cela de votre main, que vous fouliez de vos pieds mes parvis ?
\VS{13}Ne continuez plus à m'apporter de vaines offrandes : Le parfum m'est en abomination, quant aux nouvelles lunes, aux sabbats et à la publication de vos convocations ; je ne puis plus supporter votre méchanceté ni vos assemblées solennelles.
\VS{14}Mon âme hait vos nouvelles lunes et vos fêtes solennelles ; elles me sont fâcheuses, je suis las de les supporter.
\VS{15}C'est pourquoi, quand vous étendez vos mains, je cache mes yeux de vous ; quand vous multipliez vos prières, je ne les exauce pas ; vos mains sont pleines de sang\FTNT{Es. 59:1-3 ;Mi. 3:4.}.
\VS{16}Lavez-vous, purifiez-vous, ôtez de devant mes yeux la méchanceté de vos actions ; cessez de faire le mal.
\VS{17}Apprenez à bien faire, recherchez la droiture, redressez celui qui est foulé ; faites justice à l'orphelin, défendez la cause de la veuve.
\TextTitle{Mise en garde ; appel à la justice de Yahweh}
\VS{18}Venez maintenant, dit Yahweh, et débattons nos droits. Si vos péchés sont comme l'écarlate, ils seront blanchis comme la neige ; s'ils sont rouges comme le vermillon ils seront blanchis comme la laine.
\VS{19}Si vous obéissez volontairement, vous mangerez le meilleur du pays.
\VS{20}Mais si vous refusez d'obéir et si vous êtes rebelles, vous serez dévorés par l'épée, car la bouche de Yahweh a parlé.
\VS{21}Comment la cité fidèle est-elle devenue une prostituée ? Elle était pleine de droiture et la justice y habitait ; mais maintenant elle est pleine de meurtriers !
\VS{22}Ton argent s'est changé en scories; ton breuvage est mêlé d'eau. 
\VS{23}Les chefs de ton peuple sont rebelles et compagnons des voleurs ; chacun d'eux aime les présents, ils courent après les récompenses ; ils ne font point droit à l'orphelin, et la cause de la veuve ne vient point devant eux. 
\VS{24}C'est pourquoi le Seigneur, Yahweh des armées, le Puissant d'Israël dit : Ah ! Je me satisferai en punissant mes adversaires, et je me vengerai de mes ennemis. 
\VS{25}Et je remettrai ma main sur toi, je refondrai tes scories comme avec la potasse, et j'ôterai tout ton étain ;
\VS{26}mais je rétablirai tes juges, tels qu'ils étaient autrefois, et tes conseillers, tels qu'ils étaient au commencement\FTNT{Dans le royaume messianique, le gouvernement théocratique sera restauré et la fonction des juges sera rétablie (voir livre des Juges ; Mt. 19:28 ; 1 Co. 6:2-3).}. Après cela, on t'appellera cité de la justice, ville fidèle.
\VS{27}Sion sera rachetée par la droiture et ceux qui s'y convertiront seront rachetés par la justice.
\VS{28}Mais les rebelles et les pécheurs seront détruits ensemble, et ceux qui abandonnent Yahweh seront consumés.
\VS{29}Car on sera honteux à cause des térébinthes que vous avez désirés, et vous rougirez à cause des jardins que vous avez choisis\FTNT{Des cultes idolâtres avaient lieu autour des térébinthes et dans des jardins (De. 16:21 ; Es. 57:4-5 ; Es. 65:3 ; Jé. 2:20 ; Ez. 20:28 ; Os. 4:13).}.
\VS{30}Car vous serez comme le térébinthe dont le feuillage tombe, et comme un jardin qui n'a pas d'eau.
\VS{31}Et le fort sera de l'étoupe, et son œuvre une étincelle ; et tous deux brûleront ensemble, et il n'y aura personne pour éteindre le feu.
\Chap{2}
\TextTitle{Vision du règne messianique}
\VerseOne{}La parole qu'Esaïe, fils d'Amots a vue touchant Juda et Jérusalem.
\VS{2}Or il arrivera, dans les derniers jours\FTNT{Voir Ge. 49:1-2.}, que la montagne de la maison de Yahweh sera affermie au  sommet des montagnes, qu'elle sera élevée par-dessus les collines et que toutes les nations y afflueront.
\VS{3}Et plusieurs peuples iront et diront : Venez, et montons à la montagne de Yahweh, à la maison du Dieu de Jacob ; et il nous instruira ses voies, et nous marcherons dans ses sentiers ; car la loi sortira de Sion, et la parole de Yahweh sortira de Jérusalem. 
\VS{4}Il exercera le jugement parmi les nations, et reprendra plusieurs peuples. De leurs épées ils forgeront des hoyaux, et de leurs lances des serpes ; une nation ne lèvera plus l'épée contre une autre et ils ne s'adonneront plus à la guerre.
\VS{5}Venez, ô maison de Jacob, et marchons dans la lumière de Yahweh.
\TextTitle{L'orgueilleux abaissé au jour de Yahweh}
\VS{6}Certes tu as rejeté ton peuple, la maison de Jacob, parce qu'ils se sont remplis d'orient et adonnés à la divination comme les Philistins, et parce qu'ils s'allient aux enfants des étrangers\FTNT{De. 18:8-13 ; Os. 13:2 ; Mi. 5:11-13.}.
\VS{7}Son pays est rempli d'argent et d'or, et il n'y a pas de fin à ses trésors ; son pays est rempli de chevaux, et il n'y a pas de fin à ses chars.
\VS{8}Son pays est rempli d'idoles ; ils se prosternent devant l'ouvrage de leurs mains et devant ce que leurs doigts ont fabriqué.
\VS{9}Et ceux du commun sont abattus, et les personnes de qualité sont abaissées ; ne leur pardonne donc point.
\VS{10}Entre dans les rochers et cache-toi dans la poussière, à cause de la frayeur de Yahweh, et à cause de la gloire de sa majesté\FTNT{Ap. 6:15-16.}.
\VS{11}Les yeux hautains des hommes seront abaissés et les hommes qui s'élèvent seront humiliés, Yahweh sera seul haut élevé en ce jour-là.
\VS{12}Car il y a un jour assigné par Yahweh des armées contre tout homme orgueilleux et hautain, et contre tout homme qui s'élève, afin qu'il soit abaissé ;
\VS{13}contre tous les cèdres du Liban, hauts et élevés, et contre tous les chênes de Basan ;
\VS{14}contre toutes les hautes montagnes, et contre toutes les collines élevées ;
\VS{15}contre toutes les hautes tours, et contre toutes les murailles fortes ;
\VS{16}contre tous les navires de Tarsis, et contre toutes les peintures de plaisance.
\VS{17}Et l'arrogance des hommes sera humiliée, et les hommes qui s'élèvent seront abaissés :
\VS{18}Yahweh seul sera élevé en ce jour-là. Quant aux idoles, elles tomberont toutes.
\VS{19}Et les hommes entreront dans les cavernes des rochers et dans les trous de la terre, à cause de la frayeur de Yahweh et à cause de sa gloire magnifique, lorsqu'il se lèvera pour faire trembler la terre.
\VS{20}En ce jour-là, les hommes jetteront aux taupes et aux chauves-souris leurs idoles d'argent et leurs idoles d'or, qu'ils s'étaient faites pour se prosterner devant elles ;
\VS{21}et ils entreront dans les fentes des rochers et dans les creux des rochers, à cause de la frayeur de Yahweh, et à cause de sa gloire magnifique, quand il se lèvera pour punir la terre.
\VS{22}Retirez-vous de l'homme, dans les narines duquel il n'y a qu'un souffle : Car quel cas mérite-t-il qu'on en fasse ?
\Chap{3}
\TextTitle{Le péché, cause de dissolution nationale}
\VerseOne{}Car voici, le Seigneur, Yahweh des armées, va ôter de Jérusalem et de Juda tout appui et toute ressource, toute ressource de pain et toute ressource d'eau.
\VS{2}L'homme fort et l'homme de guerre, le juge et le prophète, le devin et l'ancien,
\VS{3}le chef de cinquante et l'homme d'autorité, le conseiller, l'expert d'entre les artisans et l'habile enchanteur.
\VS{4}Et je leur donnerai de jeunes gens pour chefs, et des enfants domineront sur eux.
\VS{5}Le peuple sera opprimé ; l'un opprimera l'autre, chacun son prochain. Le jeune homme se portera arrogamment contre le vieillard, et l'homme de rien contre l'honorable.
\VS{6}Même un homme ira jusqu'à saisir son frère dans la maison paternelle et lui dira : Tu as un manteau, sois notre chef ! Et prends en main ces ruines !
\VS{7}Ce jour même il répondra : Je ne suis pas médecin, et dans ma maison il n'y a ni pain ni manteau ; ne m'établissez donc pas chef du peuple.
\VS{8}Certes Jérusalem est renversée, et Juda est tombée, parce que leurs langues et leurs actions sont contre Yahweh, pour braver les regards de sa gloire.
\VS{9}L'aspect de leur visage témoigne contre eux, ils publient leur péché comme Sodome, ils ne le cachent pas. Malheur à leur âme, car ils ont attiré le mal sur eux !
\VS{10}Dites au juste que du bien lui arrivera, car il mangera le fruit de ses œuvres.
\VS{11}Malheur au méchant qui ne cherche qu'à faire le mal, car la rétribution de ses mains lui sera rendue.
\VS{12}Quant à mon peuple, il a pour oppresseur des enfants, et des femmes dominent sur lui. Mon peuple, ceux qui te conduisent t'égarent, ils corrompent le chemin dans lequel tu marches.
\VS{13}Yahweh se présente pour plaider, il se tient debout pour juger les peuples.
\VS{14}Yahweh entre en jugement avec les anciens de son peuple et avec ses chefs ; car vous avez brouté la vigne, et ce que vous avez ravi au pauvre est dans vos maisons.
\VS{15}Que vous revient-il de fouler mon peuple, et d'écraser le visage des affligés ? Dit le Seigneur, Yahweh des armées.
\TextTitle{Les filles hautaines de Sion}
\VS{16}Yahweh dit aussi : Parce que les filles de Sion sont hautaines, et qu'elles marchent le cou tendu et les yeux pleins de convoitise, parce qu'elles marchent avec une fière démarche faisant du bruit avec leurs pieds,
\VS{17}Yahweh rendra chauve le sommet de la tête des filles de Sion, Yahweh découvrira leur nudité.
\VS{18}En ce temps-là, le Seigneur ôtera l'ornement de leurs anneaux de cheville, et les filets et les croissants ;
\VS{19}les pendants d'oreilles, les bracelets et les voiles ;
\VS{20}les parures de la tête, les chaînettes des pieds et les ceintures, les boîtes à parfum et les amulettes ;
\VS{21}les anneaux et les bagues qui leur pendent sur le nez ;
\VS{22}les vêtements de fête et les larges tuniques, les manteaux et les gibecières ;
\VS{23}les miroirs et les chemises fines, les tiares et les voiles légers.
\VS{24}Et il arrivera qu'au lieu du parfum, il y aura de la puanteur ; au lieu de ceintures, des cordes ; au lieu de cheveux bouclés, des têtes chauves ; au lieu de robes flottantes, des sacs étroits ; et au lieu d'un beau teint, un teint tout hâlé.
\VS{25}Tes hommes tomberont par l'épée et ta force par la guerre.
\VS{26}Et ses portes gémiront et mèneront deuil ; désolée, elle s'assiéra par terre.
\Chap{4}
\TextTitle{Vision du règne messianique\FTNTT{Es. 11:1-16.}}
\VerseOne{}Et en ce jour sept femmes saisiront un seul homme, et diront : Nous mangerons notre pain, et nous nous vêtirons de nos habits ; seulement fais-nous porter ton nom ; ôte notre opprobre.
\VS{2}En ce temps-là, le germe de Yahweh\FTNT{Jésus est le « germe » de Yahweh (Es. 4:2) et le germe de David (Jé. 23:5 ; Za. 3:8 ; Za. 6:12). Ce germe a été placé par la vertu du Saint-Esprit dans le sein d'une vierge (Es. 7:14 ; Lu. 1:34-35) et l'enfant qui naquit d'elle fut appelé « Fils de Dieu » tout en étant le Dieu Tout-Puissant. Il existe de toute éternité en forme de Dieu (Jn. 1:1 ; Es. 9:5), mais il a été fait chair pour nous sauver (Jn. 1:14. 1 Ti. 3:16).
C'est le plus grand des miracles et la démonstration de sa divinité, de sa sagesse et de son amour envers les hommes.
} sera plein de noblesse et de gloire, et le fruit de la terre plein de grandeur et d'excellence pour les réchappés d'Israël.
\VS{3}Et il arrivera que les restes de Sion, et les restes de Jérusalem, seront appelés saints; et ceux de Jérusalem seront inscrits parmi les vivants\FTNT{Es. 10:20-22 ; Ro. 9:27 ; Ro. 11:5 ;}.
\VS{4}Quand le Seigneur aura lavé la souillure des filles de Sion, et purifié Jérusalem du sang qui est au milieu d'elle, par l'esprit de jugement et par l'esprit qui consume;
\VS{5}aussi Yahweh créera, sur toute l'étendue du mont Sion et sur ses assemblées, une nuée avec une fumée pendant le jour, et une splendeur de feu flamboyant pendant la nuit, car la gloire se répandra partout.
\VS{6}Et il y aura un tabernacle pour donner de l'ombre contre la chaleur du jour, pour servir de refuge et d'asile contre la tempête et la pluie\FTNT{Ap. 21:3.}.
\Chap{5}
\TextTitle{Israël, vigne de Yahweh}
\VerseOne{}Je chanterai maintenant pour mon bien-aimé le cantique de mon bien-aimé sur sa vigne. Mon bien-aimé avait une vigne sur un coteau fertile.
\VS{2}Il l'environna d'une haie, en ôta les pierres, et y planta des ceps exquis ; il bâtit une tour au milieu d'elle, et il y creusa aussi une cuve. Puis il espéra qu'elle produirait des raisins, mais elle a produit des grappes sauvages\FTNT{Lu. 13:6-9.}.
\VS{3}Maintenant donc, vous habitants de Jérusalem et vous hommes de Juda, jugez, je vous prie, entre moi et ma vigne.
\VS{4}Qu'y avait-il encore à faire à ma vigne que je ne lui aie fait ? Pourquoi, quand j'ai attendu qu'elle produirait des raisins, a-t-elle produit des grappes sauvages ?
\VS{5}Maintenant donc je vous dirai ce que je vais faire à ma vigne : J'ôterai sa haie, et elle sera broutée ; je romprai sa clôture et elle sera foulée.
\VS{6}Et je la réduirai en désert, elle ne sera plus taillée, ni cultivée ; les ronces et les épines y croîtront ; et je commanderai aux nuées qu'elles ne laissent plus tomber de pluie sur elle.
\VS{7}Or la maison d'Israël est la vigne de Yahweh des armées, et les hommes de Juda sont la plante en laquelle il prenait plaisir. Il en attendait de la droiture, et voici du saccagement ! De la justice, et voici des cris de détresse !
\TextTitle{Six malheurs en punition de l'infidélité d'Israël}
\VS{8}Malheur à ceux qui ajoutent maison à maison, et qui joignent champ à champ, jusqu'à ce qu'il n'y ait plus d'espace et qu'ils habitent seuls au milieu du pays.
\VS{9}Yahweh des armées m'a fait entendre : Certainement, ces maisons nombreuses seront réduites en désolation, ces maisons grandes et belles seront sans habitants.
\VS{10}Même dix arpents de vigne ne produiront qu'un bath, et un homer de semence ne produira qu'un épha.
\VS{11}Malheur à ceux qui se lèvent de bon matin, qui recherchent les boissons fortes, qui demeurent jusqu'au soir, et jusqu'à ce que le vin les échauffe !
\VS{12}La harpe et le luth, le tambourin, la flûte et le vin sont dans leurs festins ; mais ils ne regardent pas l'œuvre de Yahweh, et ils ne voient pas l'ouvrage de ses mains.
\VS{13}C'est pourquoi mon peuple sera emmené captif, parce qu'il n'a pas de connaissance\FTNT{2 R. 24:14-16 ; Os. 4:6.} ; et les plus honorables parmi eux seront des pauvres qui mourront de faim, et leur multitude sera asséchée par la soif.
\VS{14}C'est pourquoi le scheol s'élargit, il ouvre sa gueule outre mesure ; et sa magnificence y descend, sa multitude, sa pompe et tous ceux qui s'y réjouissent.
\VS{15}Ceux du commun seront abattus, les personnes de qualité seront humiliées, et les yeux des hautains seront humiliés.
\VS{16}Et Yahweh des armées sera haut élevé en jugement, et le Dieu saint sera sanctifié dans la justice.
\VS{17}Les agneaux paîtront selon qu'ils seront parqués, et les étrangers dévoreront les champs désolés des riches.
\VS{18}Malheur à ceux qui tirent l'iniquité avec des cordes de vanité, et le péché avec les traits d'un char,
\VS{19}et qui disent : Qu'il hâte et qu'il fasse venir son œuvre bientôt, afin que nous la voyions ! Que le conseil du Saint d'Israël s'avance et vienne, afin que nous le connaissions !
\VS{20}Malheur à ceux qui appellent le mal bien et le bien mal\FTNT{Mi. 7:2.} ; qui font les ténèbres lumière, et la lumière ténèbres ; qui font l'amertume douceur, et la douceur amertume.
\VS{21}Malheur à ceux qui sont sages à leurs yeux, en se considérant eux-mêmes intelligents !
\VS{22}Malheur à ceux qui sont forts pour boire le vin et vaillants pour mêler des boissons fortes ;
\VS{23}qui justifient le méchant pour des présents, et qui ôtent à chacun des justes sa justice.
\VS{24}C'est pourquoi, comme le flambeau de feu consume le chaume, et la flamme consume l'herbe sèche, ainsi leur racine sera comme la pourriture, et leur fleur sera détruite comme la poussière ; parce qu'ils ont rejeté la loi de Yahweh des armées, et ils ont méprisé la parole du Saint d'Israël.
\VS{25}C'est pourquoi la colère de Yahweh s'enflamme contre son peuple, il étend sa main sur lui, et il le frappe ; les montagnes tremblent, et leurs cadavres ont été mis en pièces au milieu des rues. Malgré tout cela, sa colère ne se détourne pas, mais sa main est encore étendue.
\VS{26}Il élève une bannière pour les nations éloignées, et il siffle à chacune d'elles depuis les extrémités de la terre ; et voici chacune viendra promptement et légèrement.
\VS{27}Nul n'est fatigué, nul ne chancelle de lassitude, personne ne sommeille ni ne dort ; et la ceinture de leurs reins ne sera point déliée, et la courroie de leurs souliers ne sera point rompue.
\VS{28}Leurs flèches sont aiguës et tous leurs arcs tendus ; les sabots de leurs chevaux ressemblent à des cailloux, et les roues de leurs chars à un tourbillon.
\VS{29}Leur rugissement est comme celui d'un vieux lion ; ils rugissent comme des lionceaux ; ils grondent et saisissent la proie, il l'emportent et personne ne vient à son secours.
\VS{30}En ce jour-là, on mènera un bruit sur lui, semblable au mugissement de la mer ; en regardant la terre, on ne verra que ténèbres et détresse ; la lumière sera obscurcie dans le ciel.
\Chap{6}
\TextTitle{Révélation de Yahweh à Esaïe}
\VerseOne{}L'année de la mort du roi Ozias, je vis le Seigneur assis sur un trône haut et élevé, et les pans de sa robe remplissaient le temple\FTNT{2 Ch. 26:23.}.
\VS{2}Les séraphins se tenaient au-dessus de lui ; et chacun d'eux avait six ailes ; deux dont ils se couvraient la face, deux dont ils se couvraient les pieds et deux dont ils se servaient pour voler.
\VS{3}Et ils criaient l'un à l'autre, et disaient : Saint, saint, saint est Yahweh des armées ! Toute la terre est pleine de sa gloire !
\VS{4}Et les poteaux des seuils furent ébranlés dans leurs fondements par la voix de celui qui criait ; et la maison fut remplie de fumée.
\VS{5}Alors je dis : Malheur à moi ! Je suis perdu, car je suis un homme dont les lèvres sont impures, j'habite au milieu d'un peuple dont les lèvres sont impures et mes yeux ont vu le Roi, Yahweh des armées\FTNT{Jg. 13:21-22.}.
\VS{6}Mais l'un des séraphins vola vers moi, tenant à la main un charbon ardent, qu'il avait pris sur l'autel avec des pincettes.
\VS{7}Il en toucha ma bouche, et dit : Voici, ceci a touché tes lèvres, c'est pourquoi ton iniquité est ôtée, et la propitiation est faite pour ton péché.
\VS{8}Puis j'entendis la voix du Seigneur, disant : Qui enverrai-je et qui marchera pour nous ? Je répondis : Me voici, envoie-moi.
\TextTitle{Mission d'Esaïe}
\VS{9}Et il dit : Va et dis à ce peuple : En entendant vous entendrez, mais vous ne comprendrez point ; et en voyant vous verrez, mais vous n'apercevrez point.
\VS{10}Engraisse le cœur de ce peuple, et rends ses oreilles pesantes, et bouche-lui les yeux ; de peur qu'il ne voie de ses yeux, et qu'il n'entende de ses oreilles, et que son cœur ne comprenne, et qu'il ne se convertisse, et qu'il ne recouvre la santé\FTNT{Mt. 13:15 ; Mc. 4:12 ; Jn. 12:40 ; Ac. 28:27.}.
\VS{11}Je dis : Jusqu'à quand, Seigneur ? Et il répondit : Jusqu'à ce que les villes soient dévastées, jusqu'à ce qu'il n'y ait plus d'habitants, ni d'hommes dans les maisons, et que la terre soit mise en entière désolation ;
\VS{12} et que Yahweh ait dispersé au loin les hommes, et que l'abandon ait été grand au milieu du pays.
\VS{13}Toutefois s'il y reste un dixième des habitants, ils reviendront pour être la proie des flammes. Mais comme le térébinthe et le chêne conservent leur tronc quand ils sont abattus, une sainte postérité renaîtra de ce peuple\FTNT{Ro. 11:17-25.}.
\Chap{7}
\TextTitle{Retsin et Pékach complote contre Juda}
\VerseOne{}Or il arriva du temps d'Achaz, fils de Jotham, fils d'Ozias, roi de Juda, que Retsin, roi de Syrie, et Pékach, fils de Remalia, roi d'Israël, montèrent contre Jérusalem pour lui faire la guerre ; mais ils ne purent l'assiéger.
\VS{2}Et on rapporta à la maison de David : La Syrie s'est reposée sur Ephraïm. Et le cœur d'Achaz, et le cœur de son peuple furent ébranlés comme les arbres des forêts qui sont ébranlés par le vent.
\VS{3}Alors Yahweh dit à Esaïe : Sors maintenant au devant d'Achaz, toi et Schear-Jaschub, ton fils, vers l'extrémité de l'aqueduc de l'étang supérieur, sur la route du champ du foulon.
\VS{4}Et dis-lui : Prends garde à toi, et demeure tranquille, ne crains point, et que ton cœur ne devienne point lâche à cause des deux queues de ces tisons fumants, à cause de l'ardeur, dis-je, de la colère de Retsin et de la Syrie, et du fils de Remalia,
\VS{5}de ce que la Syrie délibère avec Ephraïm et le fils de Remalia de te faire du mal, en disant :
\VS{6}Montons contre Juda, assiégeons la ville, battons-la en brèche, et établissons pour roi le fils de Tabeel au milieu d'elle.
\VS{7}Ainsi parle le Seigneur, Yahweh : Cela n'aura point d'effet, et cela ne se fera point.
\VS{8}Car la tête de la Syrie c'est Damas, et le chef de Damas c'est Retsin. Encore soixante-cinq ans, Ephraïm sera froissé pour n'être plus un peuple.
\VS{9}Et la tête d'Ephraïm c'est la Samarie, et le chef de la Samarie c'est le fils de Remalia. Si vous ne croyez pas, certainement vous ne serez point affermis.
\TextTitle{Annonce de la naissance d'Emmanuel}
\VS{10}Et Yahweh parla de nouveau à Achaz, en disant :
\VS{11}Demande pour toi un signe à Yahweh ton Dieu, demande-le, soit dans les bas lieux, soit dans les lieux élevés.
\VS{12}Et Achaz répondit : Je ne demanderai rien, et je ne tenterai point Yahweh.
\VS{13}Alors Esaïe dit : Ecoutez maintenant, ô maison de David ! Est-ce trop peu pour vous de lasser les hommes, que vous lassiez aussi mon Dieu ?
\VS{14}C'est pourquoi le Seigneur lui-même vous donnera un signe : Voici, une vierge sera enceinte, et elle enfantera un fils, et elle lui donnera le nom d'Emmanuel\FTNT{Le nom « Emmanuel » est dérivé de l'hébreu « Immanuw'el » qui signifie « Dieu est avec nous ». Jésus a dit aux disciples dans Mt. 28:20 : « Et moi, je suis avec vous tous les jours jusqu'à la fin des temps ». Jésus est Emmanuel, Dieu avec nous jusqu'à la fin des temps.}.
\VS{15}Il mangera du lait et du miel, jusqu'à ce qu'il sache rejeter le mal et choisir le bien.
\VS{16}Mais avant que l'enfant sache rejeter le mal et choisir le bien, la terre que tu as en détestation sera abandonnée par ses deux rois.
\TextTitle{Prophétie sur l'imminente invasion de Juda\FTNTT{2 Ch. 28:1-20.}}
\VS{17}Yahweh fera venir sur toi, sur ton peuple et sur la maison de ton père, par le roi d'Assyrie, des jours tels qu'il n'y en a point eu de semblable depuis le jour où Ephraïm s'est séparé de Juda.
\VS{18}Et il arrivera qu'en ce jour-là, Yahweh sifflera aux mouches qui sont à l'extrémité des ruisseaux d'Egypte, et aux abeilles qui sont au pays d'Assyrie.
\VS{19}Elles viendront, et se poseront dans toutes les vallées désertes, et dans les fentes des rochers, et par tous les buissons, et par tous les halliers.
\VS{20}En ce jour-là, le Seigneur rasera avec le rasoir pris à louage au-delà du fleuve, avec le roi d'Assyrie, la tête et les poils des pieds, et il enlèvera aussi la barbe\FTNT{2 R. 16:5-9.}.
\VS{21}Et il arrivera, en ce jour-là, qu'un homme nourrira une jeune vache et deux brebis.
\VS{22}Et il arrivera que de l'abondance du lait qu'elles rendront, il mangera du beurre ; car tous ceux qui seront restés dans le pays mangeront du beurre et du miel.
\VS{23}Et il arrivera, en ce jour-là, que tout lieu où il y aura mille vignes, valant mille sicles d'argent, sera réduit en ronces et en épines.
\VS{24}On y entrera avec des flèches et avec l'arc, car tout le pays ne sera que ronces et épines.
\VS{25}Et dans toutes les montagnes que l'on cultivait avec la bêche, on ne craindra plus de voir des ronces et des épines ; mais on y lâchera les bœufs, et la brebis en foulera le sol.
\Chap{8}
\TextTitle{Annonce de la défaite de Damas et de la Samarie}
\VerseOne{}Et Yahweh me dit : Prends un grand rouleau et écris dessus en grosses lettres : Qu'on se dépêche de butiner, qu'on se hâte de piller.
\VS{2}Et je pris avec moi des témoins fidèles : Urie, le sacrificateur, et Zacharie, fils de Bérékia.
\VS{3}Puis je m'étais approché de la prophétesse ; elle conçut et elle enfanta un fils. Et Yahweh me dit : Donne-lui pour nom Maher-Schalal-Chasch-Baz\FTNT{« Maher-Schalal-Chasch-Baz » signifie « rapide au butin, rapide sur la proie ».}.
\VS{4}Car avant que l'enfant sache dire : Mon père ! Ma mère ! On enlèvera la puissance de Damas et le butin de Samarie, devant le roi d'Assyrie.
\VS{5}Et Yahweh continua encore de me parler, en disant :
\VS{6}Parce que ce peuple a rejeté les eaux de Siloé qui coulent doucement, et qu'il s'est réjoui au sujet de Retsin, et du fils de Remalia,
\VS{7}à cause de cela, voici, le Seigneur va faire monter contre eux les puissantes et grandes eaux du fleuve : Le roi d'Assyrie et toute sa gloire. Il s'élèvera partout au-dessus de son lit, et il se répandra sur toutes ses rives.
\VS{8}Et il pénétrera dans Juda, il débordera et inondera, il atteindra jusqu'au cou. Et les étendues de ses ailes rempliront la largeur de ton pays, ô Emmanuel !
\TextTitle{Exhortation aux disciples de Yahweh à rester fidèles}
\VS{9}Alliez-vous, peuples ! Et vous serez brisés ; prêtez l'oreille, vous tous qui êtes d'un pays éloigné ! Equipez-vous, et vous serez brisés ; équipez-vous, et vous serez brisés.
\VS{10}Prenez conseil, et il sera dissipé ; dites la parole, et elle sera sans effet : Car Dieu est avec nous.
\VS{11}Car ainsi m'a parlé Yahweh, avec une main forte, et il m'instruisit de ne point aller par le chemin de ce peuple-ci, en me disant :
\VS{12}Ne dites point : Conjuration, toutes les fois que ce peuple dit conjuration ; ne craignez point ce qu'il craint, et ne vous en épouvantez point.
\VS{13}Sanctifiez Yahweh des armées, lui-même, c'est lui que vous devez craindre et redouter.
\VS{14}Et il sera un sanctuaire, mais aussi une pierre d'achoppement\FTNT{Yahweh s'est présenté comme une pierre d'achoppement et un rocher de scandale. En Es. 44:8 il affirme d'ailleurs ne pas connaître d'autre rocher que lui. Esaïe n'est pas le seul prophète à qui le Seigneur s'est révélé comme étant une pierre et un rocher. Dans le Ps. 118:22-23, il est dit : « La pierre qu'ont rejetée ceux qui bâtissaient est devenue la principale de l'angle ». Daniel et Zacharie ont également prophétisé au sujet de cette pierre : « Tu regardais, lorsqu'une pierre se détacha sans le secours d'aucune main, frappa les pieds de fer et d'argile de la statue, et les mit en pièces. Mais la pierre qui avait frappé la statue devint une grande montagne, et remplit toute la terre » (Da. 2:34-35). « Car voici, pour ce qui est de la pierre que j'ai placée devant Josué, il y a sept yeux sur cette seule pierre ; voici, je graverai moi-même ce qui doit y être gravé, dit Yahweh des armées; et j'enlèverai l'iniquité de ce pays, en un jour » (Za. 3:9). Ces prophéties se sont accomplies en Jésus-Christ, l'Agneau de Dieu qui ôte le péché du monde (Jn. 1:29). Le Seigneur s'est d'ailleurs clairement identifié à la pierre angulaire, affirmant ainsi sa divinité (Lu. 20:17-19). En Mt. 16:18, il s'est présenté comme le rocher inébranlable sur lequel il allait bâtir son Eglise. De plus, il est à noter que dans le livre de l'Apocalypse, l'Agneau possède sept yeux comme la pierre vue par Zacharie (Ap. 5:6). Ces sept yeux sont aussi les sept lampes du chandelier d'or que Zacharie et Jean avaient également vues (Za. 4:2 ; Ap. 4:5 ). Or le chiffre sept symbolise la plénitude et la perfection divines. Esaïe prophétisa encore en ces termes : « Voici, j'ai mis pour fondement en Sion une pierre, une pierre éprouvée, une pierre angulaire de prix, solidement posée; celui qui la prendra pour appui n'aura point hâte de fuir » (Es. 28:16). Les écrits de la Nouvelle Alliance attestent l'accomplissement de cette prophétie en Jésus-Christ, notamment par la bouche de Paul et de Pierre : « Vous avez été édifiés sur le fondement des apôtres et des prophètes, Jésus-Christ lui-même étant la pierre angulaire » (Ep. 2:20). « Car personne ne peut poser un autre fondement que celui qui a été posé, savoir Jésus-Christ » (1 Co. 3:11). « Approchez-vous de lui, pierre vivante, rejetée par les hommes, mais choisie et précieuse devant Dieu ; et vous-mêmes, comme des pierres vivantes, édifiez-vous pour former une maison spirituelle, un saint sacerdoce, afin d'offrir des victimes spirituelles, agréables à Dieu, par Jésus-Christ ». (1 Pi. 2:4-5).}, un rocher de scandale pour les deux maisons d'Israël, un filet et un piège pour les habitants de Jérusalem.
\VS{15}Plusieurs d'entre eux trébucheront, ils tomberont et se briseront, ils seront enlacés et pris.
\VS{16}Enveloppe ce témoignage, scelle cette loi\FTNTT{"towrah" ou "torah" en hébreu.} parmi mes disciples.
\VS{17}Je m'attends à Yahweh, qui cache sa face à la maison de Jacob, et je regarde à lui.
\VS{18}Me voici, avec les enfants que Yahweh m'a donnés, pour être un signe et un miracle en Israël, de la part de Yahweh des armées, qui habite sur la montagne de Sion.
\VS{19}Si l'on vous dit : Consultez ceux qui évoquent les morts et les diseurs de bonne aventure, qui poussent des sifflements et des soupirs, répondez : Un peuple ne consultera-t-il pas son Dieu ? S'adressera-t-il aux morts en faveur des vivants ?
\VS{20}A la loi et au témoignage ! Si l'on ne parle pas ainsi, il n'y aura certainement point d'aurore pour le peuple.
\VS{21}Et il sera errant dans le pays, accablé et affamé ; et il arrivera que dans sa faim, il s'irritera, maudira son roi et son Dieu, et tournera les yeux en haut ;
\VS{22}puis il regardera vers la terre, et voici, il n'y aura que détresse, ténèbres et de sombres angoisses : Il sera enfoncé dans l'obscurité.
\VS{23}Mais l'obscurité ne sera pas autant qu'elle avait été dans son humiliation ; quand au commencement, il affligea légèrement le pays de Zabulon et le pays de Nephthali, et ensuite, l'affligea plus sévèrement près de la mer, au-delà du Jourdain, dans la Galilée des Gentils.
\Chap{9}
\TextTitle{Annonce de la naissance et du règne du Messie}
\VerseOne{}Le peuple qui marchait dans les ténèbres voit une grande lumière, et la lumière resplendit sur ceux qui habitaient le pays de l'ombre de la mort\FTNT{Mt. 4:15-16.}.
\VS{2}Tu multiplies la nation, tu lui accordes de grandes joies, ils se réjouissent devant toi, comme on se réjouit à la moisson, comme on s'égaye quand on partage le butin.
\VS{3}Car tu as mis en pièces le joug dont il était chargé, et le bâton dont on lui battait ordinairement les épaules, et la verge de celui qui l'opprimait, comme au jour de Madian.
\VS{4}Parce que toute bataille de guerrier se fait dans un bruit confus, et que le vêtement est vautré dans le sang ; mais ceci sera comme un embrasement, quand le feu dévore quelque chose.
\VS{5}Car un enfant nous est né, un Fils nous a été donné\FTNT{Jésus-Christ est 100\% Dieu  et 100\% homme. Il  existe depuis toute éternité en tant que Dieu. Il est devenu homme au moment de son incarnation (Ph. 2 :5-7).}, et l'empire reposera sur son épaule : On l'appellera l'Admirable, le Conseiller, le Dieu Puissant, le Père d'éternité\FTNT{Philippe, disciple de Jésus-Christ voulait rencontrer le Père. Il posa au Seigneur cette question « Seigneur, montre-nous le Père, et cela nous suffit » (Jn. 14:8). Jésus lui répondit : « Il y a si longtemps que je suis avec vous, et tu ne m'as pas connu, Philippe ! » (Jn. 14:9).}, le Prince de paix,
\VS{6}pour accroître l'empire, et une paix sans fin au trône de David et à son royaume, pour l'affermir et le soutenir par le droit et par la justice, dès maintenant et à toujours\FTNT{Lu. 1:32-33.}. Voilà ce que fera le zèle de Yahweh des armées.
\TextTitle{Jugement sur le royaume du nord}
\VS{7}Le Seigneur envoie une parole à Jacob, et elle tombe sur Israël\FTNT{Ge. 32:28.}.
\VS{8}Et tout le peuple en aura connaissance, Ephraïm et les habitants de Samarie, qui disent avec orgueil et avec un cœur hautain :
\VS{9}Des briques sont tombées, mais nous bâtirons en pierres de taille ; des sycomores ont été coupés, mais nous les changerons en cèdres.
\VS{10}Yahweh élèvera contre eux les ennemis de Retsin, et il armera les ennemis d'Israël ;
\VS{11}la Syrie à l'orient, et les Philistins à l'occident ; et ils dévoreront Israël à gueule ouverte. Malgré tout cela, sa colère ne s'apaise point, et sa main est encore étendue.
\VS{12}Parce que le peuple ne revient pas à celui qui le frappe, et il ne cherche pas Yahweh des armées.
\VS{13}A cause de cela Yahweh retranchera d'Israël en un seul jour la tête et la queue, la branche de palmier et le roseau.
\VS{14}L'ancien et le magistrat, c'est la tête ; et le prophète qui enseigne le mensonge, c'est la queue.
\VS{15}Ceux donc qui font croire à ce peuple qu'il est heureux sont des séducteurs\FTNT{1 Ti. 4:1 ; Tit. 1:10.}; et ceux qui se laissent diriger par eux se perdent.
\VS{16}C'est pourquoi le Seigneur ne saurait prendre plaisir à leurs jeunes hommes ni avoir pitié de leurs orphelins et de leurs veuves, car tous sont des hypocrites et des méchants, et toute bouche ne profère que des infamies. Malgré tout cela, sa colère ne s'apaise point et sa main est encore étendue.
\VS{17}Car la méchanceté consume comme un feu, elle dévore les ronces et les épines ; elle embrase l'épaisseur de la forêt, d'où s'élèvent des colonnes de fumée.
\VS{18}A cause de la fureur de Yahweh des armées, la terre est obscurcie, et le peuple est comme la proie du feu ; nul n'a compassion de son frère.
\VS{19}On pille à droite, et l'on a faim ; on dévore à gauche, et l'on n'est pas rassasié ; chacun mange la chair de son bras.
\VS{20}Manassé dévore Ephraïm, Ephraïm dévore Manassé, et ensemble ils fondent sur Juda. Malgré tout cela, sa colère ne s'apaise point, et sa main est encore étendue.
\Chap{10}
\VerseOne{}Malheur à ceux qui décrètent des ordonnances iniques, et à ceux qui écrivent pour ordonner l'oppression,
\VS{2}pour refuser la justice aux pauvres et ravir leur droit aux malheureux de mon peuple, afin d'avoir les veuves pour leur butin, et de piller les orphelins !
\VS{3}Et que ferez-vous au jour de la visitation, et de la ruine éclatante qui viendra de loin ? Vers qui fuirez-vous pour avoir du secours et où laisserez-vous votre gloire\FTNT{Os. 9:7 ; Mt. 24:17-21 ; Lu. 19:41-44.} ?
\VS{4}Les uns seront courbés parmi les prisonniers, les autres tomberont parmi les morts. Malgré tout cela, sa colère ne s'apaise point, et sa main est encore étendue.
\TextTitle{Jugement sur l'Assyrie}
\VS{5}Malheur à l'Assyrie, verge de ma colère ! La verge dans leur main c'est l'instrument de ma colère.
\VS{6}Je l'ai envoyé contre une nation impie, et je l'ai fait marcher contre le peuple de ma fureur, afin qu'il se livre au pillage et fasse du butin, pour qu'il le foule aux pieds comme la boue des rues.
\VS{7}Mais il n'en juge pas ainsi, et ce n'est pas là la pensée de son cœur ; il ne songe qu'à détruire, qu'à exterminer beaucoup de nations.
\VS{8}Car il dit : Mes princes ne sont-ils pas autant de rois ?
\VS{9}Calno n'est-elle pas comme Carkemisch ? Hamath n'est-elle pas comme Arpad ? Et Samarie n'est-elle pas comme Damas ?
\VS{10}Puisque ma main a soumis les royaumes qui avaient des idoles, où il y avait plus d'images taillées qu'à Jérusalem et à Samarie,
\VS{11}ne ferai-je pas aussi à Jérusalem et à ses dieux, comme j'ai fait à Samarie et à ses idoles ?
\VS{12}Mais il arrivera que, quand le Seigneur aura achevé toute son œuvre sur la montagne de Sion et à Jérusalem, je punirai le roi d'Assyrie pour le fruit de son cœur orgueilleux, et pour la gloire de ses regards hautains.
\VS{13}Parce qu'il dit : C'est par la force de ma main que j'ai agi, c'est par ma sagesse, car je suis intelligent ; j'ai reculé les bornes des peuples, et j'ai pillé ce qu'ils avaient de plus précieux ; et comme un homme vaillant, j'ai fait descendre ceux qui étaient assis.
\VS{14}Ma main a trouvé les richesses des peuples, comme on trouve un nid ; comme on rassemble des œufs délaissés, ainsi ai-je rassemblé toute la terre ; nul n'a remué l'aile, ni ouvert le bec, ni poussé un cri.
\VS{15}La hache se glorifie-t-elle envers celui qui s'en sert ? Ou la scie s'élève-t-elle au-dessus de celui qui la manie ? Comme si la verge faisait mouvoir celui qui la lève, et que le bâton se levait comme s'il n'était pas du bois !
\VS{16}C'est pourquoi le Seigneur, Yahweh des armées, enverra la maigreur sur ses hommes gras ; et sous sa gloire éclatera l'embrasement d'un feu.
\VS{17}Car la lumière d'Israël deviendra un feu, et son Saint une flamme qui embrasera et consumera ses épines et ses ronces tout en un jour ;
\VS{18}et il consumera la gloire de sa forêt et de ses campagnes, depuis l'âme jusqu'à la chair. Il en sera comme quand celui qui porte l'étendard est défait.
\VS{19}Le reste des arbres de sa forêt pourra être compté, et un enfant en écrirait le nombre.
\TextTitle{Conversion et délivrance du reste d'Israël}
\VS{20}Et il arrivera en ce jour-là, que le reste d'Israël et les réchappés de la maison de Jacob ne s'appuieront plus sur celui qui les frappait, mais ils s'appuieront avec confiance sur Yahweh, le Saint d'Israël.
\VS{21}Le reste se convertira, le reste, dis-je, de Jacob se convertira au Dieu puissant.
\VS{22}Car quand ton peuple, ô Israël, serait comme le sable de la mer, un reste seulement se convertira ; la destruction est résolue, elle fera déborder la justice.
\VS{23}Car la destruction qu'il a résolue, le Seigneur, Yahweh des armées, va l'exécuter au milieu de toute la terre.
\VS{24}C'est pourquoi ainsi parle le Seigneur, Yahweh des armées : Mon peuple qui habites en Sion, ne crains pas le roi d'Assyrie ; il te frappe de la verge, et il lève son bâton sur toi comme faisait l'Egypte.
\VS{25}Mais encore un peu de temps, un peu de temps, et le châtiment cessera, puis ma colère se tournera contre lui pour l'exterminer.
\VS{26}Et Yahweh des armées lèvera le fouet contre lui, comme il frappa Madian au rocher d'Oreb ; et de même qu'il leva son bâton sur la mer, il le lèvera aussi comme contre les Egyptiens.
\VS{27}En ce jour-là, son fardeau sera ôté de dessus ton épaule et son joug de dessus ton cou ; et l'onction fera rompre le joug.
\TextTitle{Défaite des Assyriens\FTNTT{Es. 35-36 ; 37.7.}}
\VS{28}Il marche sur Ajjath, traverse Migron et il met ses bagages à Micmasch.
\VS{29}Ils passent le défilé, ils couchent à Guéba ; Rama est effrayée ; Guibea de Saül prend la fuite.
\VS{30}Pousse des cris, fille de Gallim ! Malheur à toi Anathoth ! Prends garde Laïs !
\VS{31}Madména se disperse, les habitants de Guébim se sauvent en foule.
\VS{32}Encore un jour d'arrêt à Nob, et il menace de sa main la montagne de la fille de Sion, la colline de Jérusalem.
\VS{33}Voici, le Seigneur, Yahweh des armées, brise les rameaux avec force ; et ceux qui sont les plus hauts élevés sont coupés, et les hauts montés sont abaissés.
\VS{34}Et il taille avec le fer les lieux les plus épais de la forêt, et le Liban tombe sous le Puissant.
\Chap{11}
\TextTitle{Rétablissement du règne de David par le Messie}
\VerseOne{}Mais il sortira un rameau du tronc d'Isaï, et un rejeton naîtra de ses racines\FTNT{Mt. 1:6-16 ; Lu. 1:31-32 ; Ro. 15:12 ; Ap. 5:5.}.
\VS{2}L'Esprit de Yahweh reposera sur lui, Esprit de sagesse et d'intelligence, Esprit de conseil et de force, Esprit de connaissance et de crainte de Yahweh\FTNT{Es. 61:1 ; Lu. 4:18}.
\VS{3}Il respirera la crainte de Yahweh, il ne jugera point sur l'apparence et il ne reprendra point sur un ouï-dire\FTNT{Jé. 11:20 ; Mt. 22:16 ; Ap. 2:23.}.
\VS{4}Mais il jugera les pauvres avec justice, et il prononcera avec droiture un jugement sur les malheureux de la terre, et il frappera la terre par la verge de sa bouche, et il fera mourir le méchant par le souffle de ses lèvres\FTNT{Job. 4:9 ; Job. 15:30 ; 2 Thess. 2:8.}.
\VS{5}La justice sera la ceinture de ses reins, et la fidélité, la ceinture de ses flancs\FTNT{Ep. 6:14.}.
\VS{6}Le loup habitera avec l'agneau, et le léopard se couchera avec le chevreau ; le veau, et le lionceau, et le bétail qu'on engraisse seront ensemble, et un petit enfant les conduira.
\VS{7}La jeune vache paîtra avec l'ourse, leurs petits auront un même gîte, et le lion, comme le bœuf, mangera de la paille\FTNT{Es. 65:25.}.
\VS{8}Le nourrisson s'ébattra sur l'antre de l'aspic, et l'enfant sevré mettra sa main dans la caverne du vipère.
\VS{9}Il ne se fera ni tort ni dommage sur toute ma montagne sainte, car la terre sera remplie de la connaissance de Yahweh, comme le fond de la mer des eaux qui le couvrent.
\VS{10}En ce jour-là, les nations rechercheront le rejeton d'Isaï qui sera comme une bannière\FTNT{Jésus, notre bannière doit être élevé afin que les pécheurs soient sauvés (Jn. 3:14-15). Le livre du Cantique des cantiques est une superbe image de l'amour de Dieu manifesté en Jésus-Christ, pour nous son Église, qui sommes sa bien-aimée. « Il m'a fait entrer dans la maison du vin ; et la bannière qu'il déploie sur moi, c'est l'amour » (Ca. 2:4). Le vin dont il est question c'est le Saint-Esprit que le Seigneur déverse sur nous jour après jour et qui nous désaltère spirituellement. « Moïse bâtit un autel, et lui donna pour nom : Yahweh ma bannière » (Ex. 17:15). Autrefois, lors des combats, les différentes armées portaient bien haut leur bannière en tête des troupes pour témoigner de leur appartenance et pour indiquer pour quel pays elles combattaient. En portant en nous Jésus, nous proclamons notre appartenance à Dieu et à son Royaume. Le Ps. 20:6 dit ceci : « Nous nous réjouirons de ton salut, nous lèverons l'étendard au nom de notre Dieu ; Yahweh exaucera tous tes vœux ». Et dans le Ps. 60:6 il est dit : « Tu as donné à ceux qui te craignent une bannière pour qu'elle s'élève à cause de la vérité ». La vérité se trouve en Jésus-Christ qui est le chemin, la vérité et la vie (Jn. 14:6). En élevant Jésus comme notre bannière, nous proclamons l'œuvre parfaite accomplie à la croix. Jésus, notre bannière, est le point de rassemblement des chrétiens de tous horizons. Ce rassemblement forme le corps de Christ, l'Eglise dont nous sommes les membres. Tout comme les douze tribus d'Israël se réunissaient pour combattre, nous nous réunissons tous sous la même bannière. Jésus, notre bannière, est le signe de la victoire contre les puissances des ténèbres. En élevant le nom de Jésus comme une bannière, nous faisons fuir toute l'armée de Satan. Dans Jn. 12:32 le Seigneur nous dit : « Et moi, quand j'aurai été élevé de la terre, j'attirerai tous les hommes à moi ». Jésus a été élevé comme étant la bannière qui a réconcilié Dieu avec les pécheurs. Lorsque cette bannière est élevée, les pécheurs sont attirés vers Dieu, ils passent des ténèbres à la lumière, de la mort à la vie.} pour les peuples, et son séjour ne   sera que gloire.
\TextTitle{Etablissement du règne du Messie}
\VS{11}Et il arrivera en ce jour-là, que le Seigneur mettra encore sa main une seconde fois pour acquérir le reste de son peuple dispersé en Assyrie, en Egypte, à Pathros, en Ethiopie, à Elam, à Schinear, à Hamath et dans les îles de la mer.
\VS{12}Il élèvera une bannière parmi les nations, il rassemblera les exilés d'Israël qui auront été chassés, et il recueillera les dispersés de Juda des quatre extrémités de la terre.
\VS{13}Et la jalousie d'Ephraïm sera ôtée, et les oppresseurs de Juda seront retranchés ; Ephraïm ne sera plus jaloux de Juda, et Juda n'opprimera plus Ephraïm.
\VS{14}Mais ils voleront sur l'épaule des Philistins vers la mer ; ils pilleront ensemble les fils de l'orient ; Edom et Moab seront la proie de leurs mains et les enfants d'Ammon leur obéiront.
\VS{15}Yahweh exterminera aussi à la façon de l'interdit la langue de la mer d'Egypte, et il lèvera sa main contre le fleuve par la force de son vent, et il le frappera sur les sept rivières, et fera qu'on y marche avec des souliers.
\VS{16}Et il y aura un chemin pour le reste de son peuple, qui sera échappé de l'Assyrie, comme il y en eut un pour Israël le jour où il remonta du pays d'Egypte.
\Chap{12}
\TextTitle{Louange au sein du royaume}
\VerseOne{}Tu diras en ce jour-là : Je te loue, ô Yahweh ! Car tu as été irrité contre moi, ta colère s'est apaisée, et tu m'as consolé.
\VS{2}Voici, Dieu est ma délivrance, j'aurai confiance et je ne craindrai rien ; car Yahweh, Yahweh est ma force et ma louange ; il est mon Sauveur.
\VS{3}Et vous puiserez de l'eau avec joie aux sources du salut\FTNT{Jn. 4:10-14.},
\VS{4}et vous direz en ce jour-là : Louez Yahweh, invoquez son Nom, publiez ses œuvres parmi les peuples, rappelez que son Nom est une haute retraite !
\VS{5}Psalmodiez à Yahweh car il a fait des choses magnifiques : Cela est connu dans toute la terre !
\VS{6}Habitante de Sion, égaye-toi, et réjouis-toi avec chant de triomphe ! Car le Saint d'Israël est grand au milieu de toi.
\Chap{13}
\TextTitle{Yahweh lève une armée}
\VerseOne{}Prophétie sur Babylone, révélé à Esaïe, fils d'Amots.
\VS{2}Elevez la bannière sur la haute montagne, élevez la voix vers eux, faites des signes avec la main, et qu'on entre dans les portes des magnifiques !
\VS{3}C'est moi qui ai donné des ordres à ceux qui me sont consacrés, j'ai appelé mes hommes forts pour exécuter ma colère, ceux qui se réjouissent de ma grandeur.
\VS{4}Il y a sur les montagnes un bruit d'une multitude, comme celui d'un grand peuple ; on entend un tumulte de royaumes, de nations rassemblées : Yahweh des armées passe en revue l'armée pour le combat.
\VS{5}D'un pays éloigné, de l'extrémité des cieux, Yahweh vient avec les instruments de sa colère pour détruire tout le pays.
\TextTitle{Jugement de Yahweh sur Babylone}
\VS{6}Hurlez, car le jour de Yahweh est proche, il vient comme un ravage du Tout-Puissant.
\VS{7}C'est pourquoi toutes les mains deviennent lâches, et tout cœur d'homme se fond.
\VS{8}Ils sont épouvantés ; les détresses et les douleurs les saisissent ; ils sont en travail comme celle qui enfante ; ils se regardent les uns les autres avec stupeur, leurs visages sont comme des visages enflammés.
\VS{9}Voici, le jour de Yahweh arrive, jour cruel, jour de colère et d'ardente fureur\FTNT{Mal. 4:1 ; Ap. 19:15}, qui réduira le pays en désolation, et en exterminera les pécheurs.
\VS{10}Même les étoiles des cieux et leurs astres ne feront plus briller leur lumière ; le soleil s'obscurcira dès son lever, et la lune ne fera plus resplendir sa lueur\FTNT{Joë. 2:31 ; Mt. 24:29 ; Mc. 13:24.}.
\VS{11}Je punirai le monde habitable à cause de sa malice, et les méchants à cause de leur iniquité ; je ferai cesser l'orgueil des hautains et j'abaisserai l'arrogance des tyrans.
\VS{12}Je ferai qu'un homme sera plus précieux que l'or fin, et une personne plus que l'or d'Ophir.
\VS{13}C'est pourquoi j'ébranlerai les cieux, et la terre sera secouée de sa base\FTNT{Ag. 2:6}, à cause de la fureur de Yahweh des armées, et à cause du jour de son ardente colère.
\VS{14}Et chacun sera comme un chevreuil qui est chassé, et comme une brebis que personne ne retire, chacun se tournera vers son peuple, chacun fuira vers son pays.
\VS{15}Quiconque sera trouvé, sera transpercé ; et quiconque s'y sera joint, tombera par l'épée.
\VS{16}Et leurs petits enfants seront écrasés sous leurs yeux\FTNT{Na. 3:10.}, leurs maisons seront pillées, et leurs femmes violées.
\TextTitle{Yahweh envoie les Mèdes contre Babylone}
\VS{17}Voici, je vais susciter contre eux les Mèdes, qui ne font point cas de l'argent, et qui ne convoitent point l'or.
\VS{18}Leurs arcs écraseront les jeunes gens, et ils seront sans pitié pour le fruit des entrailles, leur œil n'épargnera point les enfants.
\VS{19}Ainsi Babylone, l'ornement des royaumes, la parure et l'orgueil des Chaldéens, sera comme Sodome et Gomorrhe que Dieu détruisit.
\VS{20}Elle ne sera plus jamais habitée, elle ne sera point habitée de génération en génération ; même les Arabes n'y dresseront point leurs tentes, et les bergers n'y feront plus reposer leurs troupeaux.
\VS{21}Mais les bêtes sauvages des déserts y prendront leur gîte, et les hiboux rempliront ses maisons, les autruches en feront leur demeure, et les boucs y sauteront.
\VS{22}Les chacals hurleront dans ses palais, et les dragons dans ses maisons de plaisance. Son temps est près d'arriver et ses jours ne se prolongeront pas.
\Chap{14}
\TextTitle{Chant d'Israël après la chute de Babylone}
\VerseOne{}Car Yahweh aura pitié de Jacob, il choisira encore Israël, et il les rétablira dans leur terre ; les étrangers se joindront à eux et s'attacheront à la maison de Jacob.
\VS{2}Et les peuples les prendront, et les ramèneront à leur demeure, et la maison d'Israël les possédera en droit d'héritage sur la terre de Yahweh, comme serviteurs et comme servantes ; ils retiendront captifs ceux qui les avaient tenus captifs, et ils domineront sur leurs oppresseurs.
\VS{3}Et il arrivera qu'au jour que Yahweh fera cesser ton travail, ton tourment, et la dure servitude qui te fut imposée,
\VS{4}alors tu prononceras ce proverbe sur le roi de Babylone, et tu diras : Comment a-t-il fini le tyran ? Comment se repose celle qui était si avide de richesses ?
\VS{5}Yahweh a brisé le bâton des méchants, et la verge des dominateurs.
\VS{6}Celui qui frappait avec fureur les peuples de coups qu'on ne pouvait point détourner, qui dominait sur les nations avec colère, est poursuivi sans ménagement.
\VS{7}Toute la terre jouit du repos et de la paix ; on éclate en chants de triomphe à gorge déployée.
\VS{8}Même les cyprès et les cèdres du Liban se réjouissent de toi en disant : Depuis que tu es tombé, personne n'est monté pour nous abattre.
\TextTitle{Le roi de Babylone dépouillé de sa gloire}
\VS{9}Le scheol s'émeut jusque dans ses profondeurs, pour t'accueillir à ton arrivée ; il réveille à cause de toi les morts, et il fait lever de leurs sièges tous les principaux de la terre.
\VS{10}Tous prennent la parole pour te dire : Toi aussi, tu es sans force comme nous, tu es devenu semblable à nous !
\VS{11}Ta hauteur est descendue dans le scheol, avec le son de tes luths ; tu es couché sur une couche de vers, et la vermine est ta couverture.
\TextTitle{Orgueil, rébellion et chute de Satan}
\VS{12}Comment es-tu tombé du ciel, astre brillant, fils de l'aurore ? Toi qui foulais les nations, tu es abattu jusqu'à terre !
\VS{13}Tu disais en ton cœur : Je monterai aux cieux, je placerai mon trône au-dessus des étoiles de Dieu ; je m'assiérai sur la montagne de l'assemblée, du côté d'Aquilon\FTNT{Aquilon est un dieu des vents septentrionaux, froids et violents, dans la mythologie romaine.} ;
\VS{14}je monterai au dessus des hauts lieux des nuées, je serai semblable au Très-Haut.
\VS{15}Et cependant tu as été précipité dans le scheol, dans les profondeurs de la fosse\FTNT{Voir commentaire Ge. 1:1-2.}.
\VS{16}Ceux qui te voient fixent sur toi leurs regards, ils te considèrent attentivement, en disant : N'est-ce pas celui qui faisait trembler la terre, qui ébranlait les royaumes,
\VS{17}qui réduisait le monde habitable en désert, qui détruisait les villes, et ne relâchait pas ses prisonniers, ni ne les renvoyait chez eux ?
\TextTitle{Babylone anéantie}
\VS{18}Tous les rois des nations, oui, tous, reposent avec honneur, chacun dans sa maison.
\VS{19}Mais toi, tu as été jeté loin de ton sépulcre, comme un rejeton pourri, comme une dépouille de gens tués, transpercés avec l'épée, qu'on jette sous les pierres d'une fosse, comme un cadavre foulé aux pieds.
\VS{20}Tu ne seras point rangé comme eux dans le sépulcre, car tu as ravagé ta terre, tu as tué ton peuple. La race des méchants ne sera point renommée à toujours.
\VS{21}Préparez la tuerie pour ses enfants, à cause de l'iniquité de leurs pères ; afin qu'ils ne se relèvent point, et qu'ils n'héritent point la terre, et ne remplissent point de villes le dessus de la terre habitable.
\VS{22}Je m'élèverai contre eux, dit Yahweh des armées, et je retrancherai à Babylone le nom et le reste qu'elle a, ses descendants et sa postérité\FTNT{Ap. 14:8 ; Ap. 18:2.}, dit Yahweh.
\VS{23}J'en ferai l'habitation du butor et un marécage, et je la balayerai avec le balai de la destruction, dit Yahweh des armées.
\TextTitle{Jugement sur le roi d'Assyrie}
\VS{24}Yahweh des armées l'a juré, en disant : Certainement ce que j'ai décidé arrivera, ce que j'ai résolu s'accomplira.
\VS{25}Je briserai le roi d'Assyrie dans ma terre, je le foulerai aux pieds sur mes montagnes ; et son joug leur sera ôté, et son fardeau sera ôté de dessus leurs épaules.
\VS{26}C'est là le conseil arrêté contre toute la terre, c'est là la main étendue sur toutes les nations.
\TextTitle{Jugement sur le pays des Philistins}
\VS{27}Car Yahweh des armées l'a arrêté en son conseil : Qui l'empêchera ? Sa main est étendue : Qui la détournera\FTNT{Ec. 7:13.} ?
\VS{28}L'année de la mort du roi Achaz, cette prophétie fut prononcée : 
\VS{29}Ne te réjouis pas, toi pays des Philistins, de ce que la verge de celui qui te frappait est brisée ! Car de la racine du serpent sortira un vipère, et son fruit sera un serpent brûlant qui vole.
\VS{30}Alors les plus misérables seront repus, et les pauvres reposeront en assurance ; mais je ferai mourir de faim ta racine, et ce qui restera de toi sera tué.
\VS{31}Porte, hurle ! Ville, crie ! Tremble, pays tout entier des Philistins ! Car d'Aquilon, vient une fumée, et il ne restera pas un homme dans ses habitations.
\VS{32}Et que répondra-t-on aux envoyés de cette nation ? On répondra que Yahweh a fondé Sion, et que les affligés de son peuple y trouvent un refuge.
\Chap{15}
\TextTitle{Jugement sur Moab}
\VerseOne{}Prophétie sur Moab. La nuit même où elle est ravagée, Ar-Moab est détruite ! La nuit même où elle est saccagée, Kir-Moab est détruite !
\VS{2}Il monte à Bajith et à Dibon, dans les hauts lieux, pour pleurer ; Moab est en lamentations sur Nebo et sur Médeba : Toutes les têtes sont rasées et toutes les barbes sont coupées.
\VS{3}On sera couvert de sacs dans les rues ; chacun hurle, fondant en larmes sur ses toits et dans ses places\FTNT{Jé. 48:38.}.
\VS{4}Hesbon et Elealé poussent des cris, et l'on entend leur voix jusqu'à Jahats ; c'est pourquoi les guerriers de Moab se lamentent, ils ont l'effroi dans l'âme.
\VS{5}Mon cœur crie à cause de Moab, dont les fugitifs s'enfuient jusqu'à Tsoar, comme une génisse de trois ans ; car ils montent par la montée de Luchith avec des pleurs, et ils jettent des cris de détresse sur le chemin de Choronaïm.
\VS{6}Même les eaux de Nimrim ne sont que désolations, même le foin est déjà séché, l'herbe est consumée, et il n'y a point de verdure.
\VS{7}C'est pourquoi ils surveillent les richesses abondantes qu'ils ont acquises, afin que ce qu'ils ont réservé soit porté dans la vallée des saules.
\VS{8}Car les cris environnent les frontières de Moab, ses lamentations retentissent jusqu'à Eglaïm, ses lamentations retentissent jusqu'à Beer-Elim.
\VS{9}Même les eaux de Dimon sont pleines de sang ; car j'ajouterai un surcroît sur Dimon : Des lions contre les réchappés de Moab, et le reste du pays.
\Chap{16}
\TextTitle{Lamentation sur Moab}
\VerseOne{}Envoyez l'agneau au souverain du pays, envoyez-le du rocher du désert, à la montagne de la fille de Sion.
\VS{2}Car il arrivera que les filles de Moab seront au passage de l'Arnon, comme un oiseau volant ça et là, comme une nichée chassée de son nid.
\VS{3}Mets en avant le conseil, fais l'ordonnance, sers d'ombre comme une nuit au milieu de midi ; cache ceux qui ont été chassés, et ne trahis pas ceux qui sont errants.
\VS{4}Que ceux de mon peuple qui ont été chassés séjournent chez toi, ô Moab ! Sois pour eux un refuge contre le dévastateur ! Car celui qui use d'extorsion cessera, la dévastation finira, celui qui foule le pays sera consumé de dessus la terre.
\VS{5}Et le trône s'affermira par la clémence ; et sur ce trône sera assis en vérité, dans le tabernacle de David, un juge recherchant le droit, et se hâtant de faire justice\FTNT{Mi. 4:7 ; Da. 7:14 ; Lu. 1:33 ; Ap. 11:15}.
\VS{6}Nous avons entendu l'orgueil de Moab, le peuple extrêmement orgueilleux, sa fierté, son orgueil, son arrogance et ses vains discours.
\VS{7}C'est pourquoi Moab gémit sur Moab, chacun gémit ; vous soupirez pour les fondements de Kir-Haréseth, il n'y aura que des gens blessés à mort.
\VS{8}Car les campagnes de Hesbon et le vignoble de Sibma languissent ; les maîtres des nations ont foulé ses meilleurs ceps, qui s'étendaient jusqu'à Jaezer, qui couraient ça et là par le désert ; ses rameaux s'étendaient et passaient au-delà de la mer.
\VS{9}C'est pourquoi je pleure sur la vigne de Sibma, comme sur Jaezer ; je vous arrose de mes larmes, ô Hesbon et Elealé ! Car l'ennemi avec des cris s'est jeté sur tes fruits d'été et sur ta moisson.
\VS{10}Et la joie et l'allégresse se sont retirées du champ fertile ; on ne se réjouit plus et on ne s'égaye plus dans les vignes, le vendangeur ne foule plus dans les cuves, j'ai fait cesser la chanson de la vendange\FTNT{Jé. 48:31-34.}.
\VS{11}C'est pourquoi mes entrailles gémissent sur Moab, comme une harpe, et mon intérieur sur Kir-Harès.
\VS{12}Et on voit Moab qui se fatigue sur les hauts lieux ; il entre dans son sanctuaire pour prier mais il ne peut rien obtenir.
\VS{13}Telle est la parole que Yahweh a prononcée depuis longtemps sur Moab.
\VS{14}Et maintenant Yahweh a parlé, en disant : Dans trois ans, comme les années d'un mercenaire, la gloire de Moab sera avilie, avec toute cette grande multitude ; et le reste sera petit, ce sera peu de chose, ce ne sera rien de considérable.
\Chap{17}
\TextTitle{Prophétie sur la chute de Damas et de ses alliés}
\VerseOne{}Prophétie sur Damas. Voici, Damas est détruite pour ne plus être une ville, et elle ne sera qu'un monceau de ruines\FTNT{Jé. 49:23-27.}.
\VS{2}Les villes d'Aroër sont abandonnées, elles sont livrées aux troupeaux qui s'y reposent, et il n'y a personne qui les effraie.
\VS{3}Il n'y aura plus de forteresse en Ephraïm, ni de royaume à Damas et dans le reste de la Syrie ; ils seront comme la gloire des enfants d'Israël, dit Yahweh des armées.
\VS{4}Et il arrivera en ce jour-là que la gloire de Jacob sera affaiblie et la graisse de sa chair sera fondue.
\VS{5}Il en sera comme quand le moissonneur cueille les blés, et qu'il moissonne les épis avec son bras\FTNT{Joë. 3:13 ; Mt. 13:24-30.} ; comme quand on ramasse les épis dans la vallée de Rephaïm.
\VS{6}Mais il en restera quelques grappillages, comme quand on secoue l'olivier, et qu'il reste deux ou trois olives en haut de la cime, et qu'il y en a quatre ou cinq que l'olivier a produites dans ses branches fruitières, dit Yahweh, le Dieu d'Israël.
\VS{7}En ce jour-là, l'homme regardera vers celui qui l'a fait, et ses yeux se tourneront vers le Saint d'Israël.
\VS{8}Et il ne regardera plus vers les autels, qui sont l'ouvrage de ses mains, et il ne regardera plus ce que ses doigts ont fabriqué, ni les images d'Asherah, ni les statues du soleil.
\VS{9}En ce jour-là, ses villes fortes seront abandonnées à cause des enfants d'Israël, ils seront comme un bois taillis et des rameaux abandonnés, et ce sera un désert.
\VS{10}Parce que tu as oublié le Dieu de ton salut, et que tu ne t'es pas souvenue du rocher\FTNT{Voir commentaire Es. 8:13-14.} de ta force, à cause de cela tu as transplanté des plantes de plaisance, et tu as planté des ceps étrangers.
\VS{11}De jour tu as fais croître ce que tu as planté, et le matin tu as fait levé ta semence; mais la moisson a été enlevée au jour que l'on voulait en jouir, et il y a eu une douleur désespérée.
\VS{12}Malheur à la multitude de peuples nombreux, qui font un bruit comme le bruit des mers ; et à la tempête éclatante des nations, qui font du bruit comme une tempête éclatante d'eaux impétueuses !
\VS{13}Les nations font un bruit comme une tempête éclatante de grosses eaux, mais il les menace, et elles s'enfuient ; elles seront poursuivies comme la balle des montagnes chassée par le vent, et comme une boule poussée par un tourbillon.
\VS{14}Au temps du soir, voici une terreur soudaine ; mais avant le matin, ils ne sont plus ! C'est là le partage de ceux qui nous dépouillent, et le lot de ceux qui nous pillent.
\Chap{18}
\TextTitle{Jugement sur l'Ethiopie}
\VerseOne{}Malheur à la terre qui fait ombre avec des ailes, qui est au-delà des fleuves de l'Ethiopie ;
\VS{2}qui envoie par mer des messagers, dans des navires de jonc, voguant à la surface des eaux ! Allez, messagers rapides, vers la nation robuste et vigoureuse, vers le peuple redoutable, depuis là où il est et par delà ; nation puissante et qui écrase tout, et dont les fleuves ravagent son pays.
\VS{3}Vous tous, habitants du monde, et vous qui habitez dans le pays, quand la bannière sera élevée sur les montagnes, regardez ; et quand le shofar sonnera, écoutez !
\VS{4}Car ainsi m'a parlé Yahweh : Je me tiens tranquillement, et je regarde de ma demeure, par la chaleur de la lumière, et par la vapeur de la rosée, au temps de la chaude moisson.
\VS{5}Car avant la moisson, quand le bourgeon vient en sa perfection, et que la fleur devient un raisin qui mûrit, il coupe les sarments avec des serpes, il enlève les sarments, les ayant retranchés.
\VS{6}Ils seront tous ensemble abandonnés aux oiseaux de proie qui demeurent dans les montagnes, et aux bêtes de la terre ; les oiseaux de proie seront sur eux tout le long de l'été, et toutes les bêtes de la terre y passeront l'hiver.
\VS{7}En ce temps-là, un présent sera apporté à Yahweh des armées, par le peuple robuste et vigoureux, de la part, dis-je, du peuple terrible depuis là où il est et au-delà, nation puissante et qui écrase tout, et dont le pays est ravagé par ses fleuves ; il sera apporté dans la demeure du Nom de Yahweh des armées, sur la montagne de Sion.
\Chap{19}
\TextTitle{Chute de l'Egypte}
\VerseOne{}Prophétie sur l'Egypte. Voici, Yahweh est monté sur une nuée rapide, il entre en Egypte ; et les idoles d'Egypte s'enfuient de toutes parts devant sa face, et le cœur des Egyptiens se fond au milieu d'elle\FTNT{Jé. 43:12.}.
\VS{2}Et je ferai venir pêle-mêle l'Egyptien contre l'Egyptien, et chacun fera la guerre contre son frère, et chacun contre son ami, ville contre ville, et royaume contre royaume.
\VS{3}L'esprit de l'Egypte disparaîtra du milieu d'elle, et je dissiperai son conseil ; et ils consulteront les idoles et les enchanteurs, ceux qui évoquent les morts et ceux qui prédisent l'avenir.
\VS{4}Et je livrerai l'Egypte entre les mains d'un maître sévère ; et un roi cruel dominera sur eux, dit le Seigneur, Yahweh des armées.
\VS{5}Les eaux de la mer tariront, le fleuve séchera et tarira\FTNT{Jé. 51:36.}.
\VS{6}Et on fera détourner les fleuves ; les ruisseaux des digues s'abaisseront et sécheront ; les roseaux et les joncs seront coupés.
\VS{7}Les prairies qui sont près des ruisseaux, et sur l'embouchure du fleuve, tout ce qui aura été semé le long des ruisseaux, séchera, sera jeté au loin, et ne sera plus.
\VS{8}Et les pêcheurs gémiront, tous ceux qui jettent l'hameçon dans le fleuve mèneront deuil, et ceux qui étendent des filets sur les eaux languiront.
\VS{9}Ceux qui travaillent en fin lin et en fin crêpe, et ceux qui tissent les filets seront confus.
\VS{10}Les fondements du pays seront rompus, et tous ceux qui font des écluses de viviers auront l'âme attristée.
\VS{11}Certes les chefs de Tsoan ne sont que des insensés, les sages d'entre les conseillers de Pharaon forment un conseil stupide. Comment osez-vous dire à Pharaon : Je suis fils des sages, fils des anciens rois ?
\VS{12}Où sont-ils maintenant? Où sont, dis-je, tes sages ? Qu'ils t'annoncent, je te prie, s'ils le savent, ce que Yahweh des armées a décrété contre l'Egypte.
\VS{13}Les chefs de Tsoan sont devenus insensés, les chefs de Noph se sont trompés, les chefs des tribus font égarer l'Egypte.
\VS{14}Yahweh a versé au milieu d'elle un esprit de vertige\FTNT{1 R. 22:18-22.}, pour qu'ils fassent chanceler les Egyptiens dans toutes leurs actions, comme un homme ivre se vautre dans son vomissement.
\VS{15}Et l'Egypte sera hors d'état de faire ce que font la tête et la queue, la branche de palmier et le roseau.
\TextTitle{L'Egypte et l'Assyrie dans le royaume du Messie}
\VS{16}En ce jour-là, l'Egypte sera comme des femmes : Elle sera étonnée et épouvantée à cause de la main de Yahweh des armées, quand il élèvera la main contre elle.
\VS{17}Et la terre de Juda sera pour l'Egypte un objet d'effroi ; quiconque fera mention d'elle, en sera épouvanté en lui-même, à cause du conseil décrété contre elle par Yahweh des armées.
\VS{18}En ce jour-là, il y aura cinq villes au pays d'Egypte, qui parleront la langue de Canaan, et qui jureront par Yahweh des armées ; l'une sera appelée ville de la destruction.
\VS{19}En ce jour-là, il y aura un autel à Yahweh au milieu du pays d'Egypte, et un monument dressé à Yahweh sur la frontière.
\VS{20}Et ce sera un signe et un témoignage pour Yahweh des armées dans le pays d'Egypte ; car ils crieront à Yahweh à cause des oppresseurs, et il leur enverra un sauveur, quelqu'un de grand, et il les délivrera\FTNT{Es. 43:11.}.
\VS{21}Et Yahweh se fera connaître aux Egyptiens, et les Egyptiens connaîtront Yahweh en ce jour-là ; ils le serviront, ils offriront des sacrifices et des offrandes, et ils feront des vœux à Yahweh et les accompliront.
\VS{22}Ainsi Yahweh frappera les Egyptiens, il les frappera, mais il les guérira ; et ils retourneront à Yahweh, qui les exaucera et les guérira.
\VS{23}En ce jour-là, il y aura un chemin battu de l'Egypte en Assyrie ; et l'Assyrie viendra en Egypte, et l'Egypte en Assyrie, et l'Egypte servira avec l'Assyrie.
\VS{24}En ce même temps, Israël sera, lui troisième, uni à l'Egypte et à l'Assyrie, et la bénédiction sera au milieu de la terre.
\VS{25}Yahweh des armées les bénira, en disant : Bénis soit l'Egypte mon peuple, et l'Assyrie œuvre de mes mains, et Israël mon héritage !
\Chap{20}
\TextTitle{Conquête de l'Egypte et de l'Ethiopie}
\VerseOne{}L'année où Tharthan, envoyé par Sargon, roi d'Assyrie, vint et combattit contre Asdod, et la prit.
\VS{2}En ce temps-là, Yahweh parla par Esaïe, fils d'Amots, et lui dit : Va, délie le sac de dessus tes reins et ôte tes souliers de tes pieds. Il fit ainsi, marchant nu et déchaussé.
\VS{3}Puis Yahweh dit : De même que mon serviteur Esaïe marche nu et déchaussé, ce qui sera dans trois ans un signe et un prodige contre l'Egypte et contre l'Ethiopie,
\VS{4}de même le roi d'Assyrie emmènera de l'Egypte et de l'Ethiopie prisonniers et captifs les jeunes et les vieux, nus et déchaussés, ayant les hanches découvertes, ce qui sera l'opprobre de l'Egypte\FTNT{2 S. 10:4 ; Es. 3:17 ; Jé. 13:22-26.}.
\VS{5}Ils seront effrayés, et ils seront honteux à cause de l'Ethiopie à qui ils s'attendaient, et à cause de l'Egypte dont ils se glorifiaient.
\VS{6}Et les habitants de cette côte diront en ce jour-là : Voilà ce qu'est devenu le peuple à qui nous nous attendions, celui vers qui nous courions chercher du secours, afin d'être délivrés du roi d'Assyrie ! Comment pourrons-nous échapper ?
\Chap{21}
\TextTitle{Annonce de la conquête de Babylone}
\VerseOne{}Prophétie sur le désert de la mer. Il vient du désert, de la terre redoutable, comme des tourbillons qui s'élèvent au pays du midi pour traverser.
\VS{2}Une vision terrible m'a été révélée. Le traître demeure traître, celui qui saccage, saccage toujours. Monte, Elam ! Assiège, Médie ! Je fais cesser tous les soupirs.
\VS{3}C'est pourquoi mes reins sont remplis de douleur ; les angoisses me saisissent comme les douleurs de celle qui enfante ; je suis tourmenté à cause de ce que j'ai entendu, et j'ai été tout troublé à cause de ce que j'ai vu.
\VS{4}Mon cœur est agité de toutes parts, la terreur s'empare de moi ; la nuit de mes plaisirs devient une nuit de crainte.
\VS{5}Qu'on dresse la table, que la sentinelle veille, qu'on mange, qu'on boive! Levez-vous, chefs ! Oignez le bouclier !
\VS{6}Car ainsi m'a parlé le Seigneur : Va, place la sentinelle, et qu'elle rapporte ce qu'elle verra\FTNT{Ez. 33:1-19.}.
\VS{7}Et elle vit un char, un couple de cavaliers, un char tiré par des ânes, un char tiré par des chameaux ; et elle les considéra fort attentivement.
\VS{8}Et elle s'écria : C'est un lion ! Seigneur, je me tiens en sentinelle toute la journée et je suis à mon poste toutes les nuits ;
\VS{9}et voici venir le char d'un homme et un couple de cavaliers ! Alors elle parla et dit : Elle est tombée, elle est tombée, Babylone\FTNT{Prophétie sur la chute de Babylone. Voir Jé. 50 et 51 ; Ap.18.}, et toutes les images taillées de ses dieux sont brisées par terre.
\VS{10}C'est ce que j'ai foulé, et le grain que j'ai battu dans mon aire. Je vous ai annoncé ce que j'ai entendu de Yahweh des armées, du Dieu d'Israël.
\VS{11}Prophétie sur Duma. On me crie de Séir : Ô sentinelle ! Qu'en est-il de la nuit ? Ô sentinelle ! Qu'en est-il de la nuit ?
\VS{12}La sentinelle répond : Le matin vient et la nuit aussi. Si vous demandez, demandez. Retournez, venez.
\TextTitle{Jugement sur l'Arabie}
\VS{13}Prophétie contre l'Arabie. Vous passerez pêle-mêle la nuit dans la forêt, caravanes de Dedan !
\VS{14}Les habitants du pays de Théma portent de l'eau à ceux qui ont soif ; ils  viennent au-devant du fugitif avec du pain pour lui.
\VS{15}Car ils fuient devant les épées, devant l'épée dégainée, devant l'arc tendu, devant le fort de la bataille.
\VS{16}Car ainsi m'a parlé le Seigneur : Encore une année, comme les années d'un mercenaire, et toute la gloire de Kédar prendra fin.
\VS{17}Et le reste du nombre des forts archers des fils de Kédar sera diminué, car Yahweh, le Dieu d'Israël, a parlé.
\Chap{22}
\TextTitle{Malédiction sur la vallée des visions, Jérusalem}
\VerseOne{}Prophétie sur la vallée des visions. Qu'as-tu maintenant, que tu sois toute montée sur les toits ?
\VS{2}Toi ville bruyante, pleine de tumulte, ville joyeuse ! Tes blessés à morts ne seront pas blessés à mort par l'épée, et ils ne mourront pas par la guerre.
\VS{3}Tous tes chefs fuient ensemble, ils sont liés par les archers ; tous ceux des tiens qui sont trouvés sont liés ensemble tandis qu'ils s'enfuient au loin.
\VS{4}C'est pourquoi je dis : Détournez de moi vos regards, que je pleure amèrement. Ne vous empressez pas pour me consoler du désastre de la fille de mon peuple.
\VS{5}Car c'est le jour de trouble, d'oppression et de confusion\FTNT{Lam. 1:5 ; Lam. 2:2.}, envoyé par le Seigneur, Yahweh des armées, dans la vallée des visions. Il démolit la muraille et les cris retentissent jusqu'à la montagne.
\VS{6}Même Elam prend son carquois, il y a des hommes montés sur des chars et des cavaliers ; Kir découvre le bouclier.
\VS{7}Et tes plus belles vallées sont remplies de chars, et les cavaliers se rangent tous en bataille à tes portes.
\VS{8}Et on découvre ce qui couvrait Juda, et en ce jour là tu regardes vers les armes de la maison de la forêt.
\VS{9}Vous voyez que les brèches de la cité de David sont nombreuses ; et vous assemblez les eaux de l'étang inférieur.
\VS{10}Vous faites le dénombrement des maisons de Jérusalem, et vous démolissez les maisons pour fortifier la muraille.
\VS{11}Et vous faites aussi un réservoir d'eau entre les deux murailles, pour les eaux de l'ancien étang. Mais vous ne regardez pas à celui qui a fait ces choses, qui les a formées il y a longtemps.
\VS{12}Le Seigneur, Yahweh des armées, vous appelle ce jour-là aux pleurs et au deuil, à vous raser la tête, et à ceindre le sac\FTNT{Ez. 7:18 ; Joë 1:13}.
\VS{13}Et voici il y a de la joie et de l'allégresse ! On égorge des bœufs et l'on tue des moutons, on mange la viande et l'on boit du vin ; puis on dit : Mangeons et buvons, car demain nous mourrons\FTNT{Es. 56:12 ; 1 Co. 15:32.} !
\VS{14}Or il m'a été révélé à l'oreille, par Yahweh des armées : Sûrement cette iniquité ne vous sera pas pardonnée jusqu'à ce que vous mouriez, a dit le Seigneur, Yahweh des armées.
\TextTitle{Eliakim succède à Schebna}
\VS{15}Ainsi parle le Seigneur, Yahweh des armées : Va, entre chez ce trésorier, chez Schebna, gouverneur du palais et dis-lui :
\VS{16}Qu'as-tu à faire ici, et qu'as-tu ici qui t'appartienne, que tu te tailles ici un sépulcre ? Il taille un sépulcre en hauteur, il se taille une demeure dans le rocher.
\VS{17}Voici, ô homme ! Yahweh te chassera au loin d'un bras vigoureux ; il t'enveloppera entièrement.
\VS{18}Il te fera rouler fort vite, comme une balle sur une terre large et spacieuse ; là tu mourras, là seront les chars de ta gloire, ô toi qui es la honte de la maison de ton Seigneur !
\VS{19}Je te jetterai hors de ton rang, et on t'arrachera de ton service.
\VS{20}Et il arrivera en ce jour-là que j'appellerai mon serviteur Eliakim, fils de Hilkija.
\VS{21}Je le revêtirai de ta tunique, je le ceindrai de ta ceinture, et je remettrai ton autorité entre ses mains, il sera un père pour les habitants de Jérusalem et pour la maison de Juda.
\VS{22}Et je mettrai la clef de la maison de David sur son épaule ; et il ouvrira, et il n'y aura personne qui ferme ; et il fermera, et il n'y aura personne qui ouvre\FTNT{La clé de David est le symbole de l'autorité du Messie (Es. 9:5 ; Mt. 28:18 ; Ap. 3:7-8)}.
\VS{23}Je l'enfoncerai comme un clou dans un lieu sûr, et il sera un trône de gloire pour la maison de son père.
\VS{24}Et on y pendra toute la gloire de la maison de son père, de ses parents et de celles qui lui appartiennent ; tous les ustensiles des plus petites choses,  des bassins comme des vases. 
\VS{25}En ce jour-là, dit Yahweh des armées, le clou enfoncé dans un lieu sûr sera ôté ; et étant retranché il tombera, et le fardeau qui était sur lui sera retranché, car Yahweh a parlé.
\Chap{23}
\TextTitle{Effondrement de Tyr}
\VerseOne{}Prophétie sur Tyr. Hurlez, navires de Tarsis ! Car elle est détruite, il n'y a plus de maisons, on n'y entre plus ! Ceci leur a été révélé du pays de Kittim.
\VS{2}Vous qui habitez dans l'île, taisez-vous ! Toi qui étais remplie de marchands de Sidon, et de ceux qui traversaient la mer !
\VS{3}A travers les grandes eaux, les grains de Shichor, la moisson du Nil était pour elle son revenu ; elle était le marché des nations\FTNT{Ez. 27.}.
\VS{4}Sois honteuse, ô Sidon ! Car la mer, la forteresse de la mer, a parlé en disant : Je n'ai point eu de douleurs, je n'ai point enfanté, je n'ai point nourri de jeunes gens ni élevé aucune vierge.
\VS{5}Selon la nouvelle qui a été touchant l'Egypte, ainsi sera-t-on en travail quand on entendra la nouvelle touchant Tyr.
\VS{6}Passez à Tarsis, hurlez, vous qui habitez dans l'île !
\VS{7}N'est-ce pas ici votre ville joyeuse ? Elle avait une origine antique et ses propres pieds la mènent séjourner dans un pays étranger.
\VS{8}Qui a pris ce conseil contre Tyr, celle qui couronnait les siens, dont les marchands étaient des princes, et dont les trafiquants étaient les plus honorables de la terre\FTNT{Ap. 18:9-18.} ?
\VS{9}Yahweh des armées a pris ce conseil, pour flétrir l'orgueil de toute la noblesse, et pour avilir tous les honorables de la terre.
\VS{10}Traverse ton pays, comme une rivière, ô fille de Tarsis ! Il n'y a plus de ceinture.
\VS{11}Il a étendu sa main sur la mer, il a fait trembler les royaumes ; Yahweh a ordonné la destruction des forteresses de Canaan.
\VS{12}Il a dit : Tu ne te livreras plus à la joie, vierge opprimée, fille de Sidon ! Lève-toi, passe au pays de Kittim ! Même là, il n'y aura pas de repos pour toi.
\VS{13}Voilà le pays des Chaldéens ; ce peuple-là n'était pas autrefois ; Assur\FTNT{Assur : Le second fils de Sem (Ge. 10:22). L'ancêtre des Assyriens.} l'a fondé pour les gens du désert ; on a dressé ses forteresses, on a élevé ses palais, et il l'a mis en ruines.
\VS{14}Hurlez, navires de Tarsis ! Car votre force est détruite !
\VS{15}Et il arrivera en ce jour-là que Tyr tombera dans l'oubli durant soixante-dix ans, selon les jours d'un roi. Mais au bout de soixante-dix ans\FTNT{Jé. 25 : 11-12.}, on chantera une chanson à Tyr comme à une femme prostituée :
\VS{16}Prends la harpe, fais le tour de la ville, ô prostituée qu'on oublie ! Sonne avec force, chante et rechante, afin qu'on se ressouvienne de toi!
\VS{17}Et il arrivera au bout de soixante-dix ans que Yahweh visitera Tyr, mais elle retournera au salaire de sa prostitution, et elle se prostituera avec tous les royaumes de la terre, sur le dessus de la terre.
\VS{18}Mais son trafic et son salaire seront sanctifiés à Yahweh ; il n'en sera rien réservé, ni serré ; car son trafic sera pour ceux qui habitent dans la présence de Yahweh, pour en manger à satiété, et pour avoir des vêtements durables.
\Chap{24}
\TextTitle{Désastre après l'invasion babylonienne}
\VerseOne{}Voici, Yahweh s'en va rendre le pays vide et l'épuiser, il en renverse le dessus, et disperse ses habitants\FTNT{Ge. 11:1-8.}.
\VS{2}Et il en est du sacrificateur comme du peuple, du maître comme de son serviteur, de la dame comme de sa servante, du vendeur comme de l'acheteur, de celui qui prête comme de celui qui emprunte, du créancier comme du débiteur.
\VS{3}Le pays est entièrement vidé et entièrement pillé, car Yahweh a prononcé cet arrêt.
\VS{4}La terre mène le deuil, elle est déchue ; le pays habité est devenu languissant, il est déchu ; les plus distingués du peuple de la terre sont languissants.
\VS{5}Le pays était profané par ses habitants qui marchent sur lui ; car ils ont transgressé les lois, ils ont changé les ordonnances et ont enfreint l'alliance éternelle\FTNT{Da. 7:25.}.
\VS{6}C'est pourquoi la malédiction dévore le pays, et ses habitants portent la peine de leurs crimes ; c'est pourquoi les habitants du pays sont brûlés et il n'en reste qu'un petit nombre.
\VS{7}Le vin excellent pleure, la vigne languit, et tous ceux qui avaient le cœur joyeux soupirent.
\VS{8}La joie des tambours a cessé ; le bruit de ceux qui s'égayent a pris fin, la joie de la harpe a cessé.
\VS{9}On ne boit plus de vin en chantant ; les boissons fortes sont amères à ceux qui les boivent.
\VS{10}La ville confuse est en ruines ; toutes les maisons sont fermées, on n'y entre plus.
\VS{11}On crie dans les rues parce que le vin manque ; toute la joie est tournée en obscurité, l'allégresse du pays s'en est allée.
\VS{12}La désolation est restée dans la ville et la porte est frappée d'une ruine éclatante.
\VS{13}Car il arrivera au milieu de la terre et parmi les peuples, comme quand on secoue l'olivier, et comme quand on grappille après la vendange.
\TextTitle{Un reste de rescapés célèbre Yahweh}
\VS{14}Ils élèvent leur voix, ils se réjouissent avec chant de triomphe ; et s'égayent du côté de la mer, ils célèbrent la majesté de Yahweh.
\VS{15}C'est pourquoi glorifiez Yahweh dans les vallées, le Nom de Yahweh, le Dieu d'Israël, dans les îles de la mer !
\VS{16}De l'extrémité de la terre, nous entendons des cantiques à la gloire du Juste ; mais moi je dis : Maigreur sur moi ! Maigreur sur moi ! Malheur à moi ! Les perfides ont agi perfidement ; et ils ont imité la mauvaise foi des perfides.
\TextTitle{Manifestation des jugements de Yahweh}
\VS{17}La frayeur, la fosse, et le piège sont sur toi, habitant du pays !
\VS{18}Et il arrivera que celui qui fuit à cause du bruit de la frayeur tombe dans la fosse, et celui qui remonte hors de la fosse se prend au filet ; car les écluses d'en haut s'ouvrent et les fondements de la terre tremblent.
\VS{19}La terre est entièrement brisée, la terre s'écrase entièrement, la terre se remue de sa place.
\VS{20}La terre chancelle entièrement comme un homme ivre, elle est transportée comme une cabane ; son péché pèse sur elle, elle tombe et ne se relève plus.
\VS{21}Et il arrivera en ce jour là, que Yahweh punira dans le lieu élevé l'armée d'en haut, et sur la terre les rois de la terre.
\VS{22}Ils seront assemblés en troupes comme des prisonniers dans une fosse, et ils seront enfermés dans une prison, et après plusieurs jours ils seront visités.
\VS{23}La lune rougira et le soleil sera honteux quand Yahweh des armées régnera sur la montagne de Sion et à Jérusalem, resplendissant de gloire en présence de ses anciens\FTNT{Mt. 24:29-30 ; 2 Pi. 3:10-12 ; Ap. 6:12.}.
\Chap{25}
\TextTitle{Le royaume de Yahweh}
\VerseOne{}Ô Yahweh, tu es mon Dieu ; je t'exalterai, je célébrerai ton nom, car tu as fait des choses merveilleuses ; tes conseils conçus d'avance sont fidèlement accomplis.
\VS{2}Car tu as fait de la ville un monceau de pierres, et de la cité forte une ruine ; le palais des étrangers qui était dans la ville ne sera jamais rebâti.
\VS{3}C'est pourquoi le peuple fort te glorifie, la ville des nations redoutables te révère.
\VS{4}Parce que tu as été la force du faible, la force du misérable dans sa détresse, le refuge contre la tempête, l'ombrage contre la chaleur ; car le souffle des tyrans est comme la tempête qui abat une muraille.
\VS{5}Tu as rabaissé la tempête éclatante des étrangers ; comme la chaleur, dis-je, dans un pays sec, comme la chaleur par l'ombre d'une nuée, le branchage des tyrans sera abattu.
\VS{6}Et Yahweh des armées prépare à tous les peuples sur cette montagne un banquet de choses grasses, un banquet de vins vieux, un banquet, dis-je, de choses grasses et moelleuses, et de vins vieux bien purifiés\FTNT{Mt. 22:2 ; Ap. 3:20.}.
\VS{7}Et il détruit sur cette montagne l'enveloppe redoublée qu'on voit sur tous les peuples, et la couverture qui est étendue sur toutes les nations.
\VS{8}Il détruit la mort par sa victoire\FTNT{1 Co. 15:54.} ; et le Seigneur Yahweh essuie les larmes de tous les visages\FTNT{Ap. 7:17.}, et il ôte l'opprobre de son peuple de toute la terre\FTNT{Lu. 1:25.}, car Yahweh a parlé.
\VS{9}Et l'on dira en ce jour-là : Voici, c'est ici notre Dieu, auquel nous nous attendons, aussi c'est lui qui nous sauve ; c'est ici Yahweh, auquel nous nous attendons ; soyons dans l'allégresse, et réjouissons-nous de son salut !
\VS{10}Car la main de Yahweh repose sur cette montagne ; mais Moab est foulé aux pieds sous lui, comme on foule la paille pour en faire du fumier.
\VS{11}Et il étend ses mains au milieu d'eux, comme le nageur étend ses mains pour nager ; et Yahweh abat son orgueil, ainsi que l'artifice de ses mains.
\VS{12}Il abaisse la forteresse des plus hautes retraites de tes murailles, il les renverse, il les fait crouler à terre, et les réduit en poussière.
\Chap{26}
\TextTitle{Adoration à Yahweh}
\VerseOne{}En ce jour-là, ce cantique sera chanté dans le pays de Juda : Nous avons une ville forte ; le salut\FTNT{Le mot salut vient du mot « Yeshuw'ah ». Cette même racine a donné le prénom Jésus qui signifie Yahweh sauve. Jésus est notre muraille et notre rempart. Dans Ex. 15:2, Moïse identifie Yahweh à « Yeshuw'ah » c'est-à-dire à Jésus. Dans 1 Ch. 16:23, il est dit que « Yeshuw'ah » doit être annoncé tous les jours. Dans Ps. 62:2, il est présenté comme Dieu et le Rocher. Dans Es. 12:2, il est le Dieu qui sauve. Jacob et David avaient mis en lui leur espoir (Ge. 49:18 ; Ps. 119:166). Dans Es. 49:6, il est dit que le salut (« Yeshuw'ah » ou Jésus) doit être annoncé aux extrémités de la terre, et cela est répété et confirmé en Mt. 28:18-20. Es. 56:1 nous apprend que celui qui vient s'appelle « Yeshuw'ah ». Es. 59:17 le présente comme notre casque, ce qui fait écho au casque du salut en Ep. 6:17. Les murs de la Nouvelle Jérusalem portent son Nom (Es. 60:18). Ha. 3:8 nous dit que « Yeshuw'ah » montera sur ses chevaux, corroborant le récit de son retour en gloire dans Ap. 19:11-20. « Yeshuw'ah » est notre flambeau selon Es. 62:1 et Ap. 21:23.} y sera mis pour muraille et pour rempart.
\VS{2}Ouvrez les portes, et la nation juste, celle qui garde la fidélité, y entrera.
\VS{3}Tu gardes dans une paix parfaite celui dont l'esprit s'appuie sur toi, parce qu'il se confie en toi\FTNT{Es. 57:19 ; Ph. 4:6-7.}.
\VS{4}Confiez-vous en Yahweh à perpétuité, car le Rocher \FTNT{Voir commentaire en Es. 8:13-14. } des siècles est en Yahweh Dieu.
\VS{5}Car il a abaissé ceux qui habitaient aux lieux haut élevés, il a renversé la ville de haute retraite, il l'a renversée jusqu'à terre, il l'a réduite jusqu'à la poussière.
\VS{6}Le pied marchera dessus ; les pieds, dis-je, des pauvres, les plantes des misérables marcheront dessus.
\VS{7}Le sentier du juste est la droiture ; toi qui est juste, tu dresses au niveau le chemin du juste.
\VS{8}Aussi t'avons-nous attendu, ô Yahweh, dans le sentier de tes jugements ! Ton Nom et ton souvenir sont le désir de notre âme.
\VS{9}De nuit, je te désire de mon âme, et dès le point du jour, mon esprit qui est en moi te recherche ; car lorsque tes jugements s'exercent sur la terre, les habitants du monde apprennent la justice.
\VS{10}Est-il fait grâce au méchant ? Il n'en apprend point la justice, mais il agit méchamment sur la terre de la droiture, et il ne regarde pas à la majesté de Yahweh.
\VS{11}Yahweh, quand ta main est élevée, ils ne le voient pas. Mais ils verront et seront honteux à cause de leur jalousie pour ton peuple ; et le feu dont tu punis tes ennemis les dévorera.
\VS{12}Yahweh, tu ordonnes la paix pour nous, car aussi tout ce que nous faisons, c'est toi qui l'accomplis en nous.
\VS{13}Yahweh, notre Dieu, d'autres seigneurs que toi nous ont maîtrisés, mais c'est par toi seul que nous pouvons faire mention de ton Nom.
\VS{14}Ils sont morts, ils ne revivront plus, ils sont trépassés, ils ne se relèveront pas ; car tu les as châtiés et exterminés, et tu as fait périr toute mémoire d'eux\FTNT{Ec. 9:5.}.
\VS{15}Yahweh, tu avais accru la nation, tu avais accru la nation, tu as été glorifié, mais tu les as jetés loin dans toutes les extrémités de la terre.
\TextTitle{Un reste épargné de la colère de Yahweh}
\VS{16}Yahweh, étant en détresse ils se sont rendus auprès de toi ; ils se sont répandus en prières quand ton châtiment a été sur eux.
\VS{17}Comme celle qui est enceinte est en travail, et crie dans ses tranchées, lorsqu'elle est prête d'enfanter, ainsi avons-nous été devant ta face ô Yahweh !
\VS{18}Nous avons conçu et nous avons éprouvé des douleurs et nous avons comme enfanté du vent. Nous ne saurions en aucune manière délivrer le pays et les habitants de la terre habitable ne tomberaient point par notre force.
\VS{19}Tes morts vivront ! Même mon corps mort vivra ! Ils se relèveront. Réveillez-vous et réjouissez-vous avec des chants de triomphe, vous, habitants de la poussière ; car ta rosée est comme la rosée des herbes, et la terre jettera dehors les morts\FTNT{Os. 13:14 ; Da. 12:2 ; 1 Co. 15:52.}.
\VS{20}Va, mon peuple, entre dans tes cabinets et ferme ta porte derrière toi\FTNT{Mt. 6:6.} ; cache-toi pour un petit moment, jusqu'à ce que l'indignation soit passée.
\VS{21}Car voici, Yahweh s'en va sortir de son lieu pour visiter l'iniquité des habitants de la terre, commise contre lui ; alors la terre découvrira le sang qu'elle aura reçu et ne couvrira plus ceux qu'on a mis à mort.
\Chap{27}
\TextTitle{Israël rétabli}
\VerseOne{}En ce jour-là, Yahweh frappera de sa dure, grande et forte épée le Léviathan\FTNT{Ps. 104:26 ; Job. 40:20}, le serpent fuyard, le Léviathan, dis-je, le serpent tortueux, et il tuera le monstre qui est dans la mer.
\VS{2}En ce jour-là, chantez sur la vigne désirable\FTNT{Esaïe annonce ici le rétablissement d'Israël. Voir également Ro. 11:1-24.}.
\VS{3}C'est moi Yahweh qui la garde, je l'arrose à chaque instant, je la garde nuit et jour, afin que personne ne lui fasse du mal.
\VS{4}Il n'y a point de fureur en moi ; qu'on me donne des ronces, des épines pour les combattre ! Je marcherai contre elles, je les brulerai toutes ensemble.
\VS{5}Ou bien, qu'il saisisse ma force, qu'il fasse la paix avec moi, qu'il fasse la paix avec moi.
\VS{6}Il fera que Jacob prendra racine, Israël  fleurira, et s'épanouira ; et il remplira de fruits le dessus de la terre habitable.
\VS{7}L'a-t-il frappé comme il a frappé celui qui le frappaient ? L'a-t-il tué comme il a tué ceux qui le tuaient ?
\VS{8}Tu as plaidé avec elle modérément, quand tu l'as renvoyée ; en l'emportant par le vent rude au jour du vent d'orient.
\VS{9}C'est pourquoi l'expiation de l'iniquité de Jacob sera faite par ce moyen, et ceci en sera le fruit entier, que son péché sera ôté ; quand il aura transformé toutes les pierres des autels comme des pierres de chaux réduites en poussière ; et lorsque les idoles d'Asherah et les statues consacrées au soleil ne seront plus debout.
\VS{10}Car la ville fortifiée est désolée, la demeure agréable est abandonnée et délaissée comme le désert. Là pâture le veau, il y gîte et broute les branches.
\VS{11}Quand son branchage est sec, il est brisé ; et les femmes y venant en allument un feu. Car c'est un peuple sans intelligence\FTNT{De. 32:28 ; Es. 1:3.},  c'est pourquoi celui qui l'a fait n'a point eu pitié de lui, et celui qui l'a formé ne lui a point fait grâce.
\VS{12}Il arrivera en ce jour-là que Yahweh secouera, depuis le cours du fleuve jusqu'au torrent d'Egypte ; mais vous serez glanés un à un, ô enfants d'Israël.
\VS{13}Et il arrivera en ce jour-là qu'on sonnera du grand shofar, et ceux qui étaient exilés au pays d'Assyrie, et ceux qui avaient été chassés au pays d'Egypte, reviendront et se prosterneront devant Yahweh, sur la sainte montagne, à Jérusalem.
\Chap{28}
\TextTitle{Malheur et captivité d'Ephraïm en Assyrie}
\VerseOne{}Malheur à la couronne de fierté des ivrognes d'Ephraïm, la noblesse de la gloire qui n'est qu'une fleur qui tombe ; ceux qui sont sur le sommet de la grasse vallée sont étourdis de vin !
\VS{2}Voici, le Seigneur a dans sa main un homme fort et puissant, semblable à une tempête de grêle, à un tourbillon destructeur, à une tempête de grosses eaux débordées ; il la fera tomber à terre avec la main.
\VS{3}Elle seront foulées aux pieds, la couronne de fierté et les ivrognes d'Ephraïm.
\VS{4}Et la noblesse de sa gloire qui est sur le sommet de la fertile vallée, ne sera qu'une fleur qui tombe ; ils seront comme les fruits précoces avant l'été, aussitôt que celui qui regarde les voit, à peine ils sont dans sa main, il les dévore.  
\VS{5}En ce jour-là, Yahweh des armées sera une couronne de noblesse et un diadème de gloire pour le reste de son peuple ;
\VS{6}et un esprit de jugement pour celui qui sera assis au siège de jugement, et une force à ceux qui dans le combat repousseront l'ennemi jusqu'à la porte.
\VS{7}Mais eux aussi, s'oublient dans le vin, et se fourvoient dans les boissons fortes ; le sacrificateur et le prophète s'oublient dans les boissons fortes ; ils sont engloutis par le vin, ils se fourvoient à cause des boissons fortes ; ils s'oublient dans la vision, ils vacillent dans le jugement.
\VS{8}Car toutes leurs tables sont couvertes de vomissements et d'ordures ; aussi il n'y a plus de place !
\VS{9}A qui enseigne-t-on la connaissance ? A qui fait-on comprendre l'enseignement ? Est-ce à ceux qu'on vient de sevrer et de retirer de la mamelle ?
\VS{10}Car il faut leur donner précepte après précepte, précepte après précepte, règle après règle, règle après règle, un peu ici, un peu là\FTNT{Hé. 5:12.}.
\VS{11}C'est pourquoi, il parlera à ce peuple par des lèvres qui balbutient et une langue étrangère.
\VS{12}Il leur disait : Voici le repos, donnez du repos à celui qui est fatigué ;  voici le soulagement ! Mais ils n'ont point voulu écouter.
\VS{13}Ainsi la parole de Yahweh sera pour eux précepte après précepte, précepte après précepte, règle après règle, règle après règle, un peu ici, un peu là ; afin qu'ils aillent et tombent à la renverse, et qu'ils soient brisés, et afin qu'ils tombent dans le piège et qu'ils soient pris.
\TextTitle{Yahweh rompt le pacte du scheol par une pierre angulaire}
\VS{14}C'est pourquoi écoutez la parole de Yahweh, vous hommes moqueurs, qui dominez sur ce peuple qui est à Jérusalem !
\VS{15}Car vous dites : Nous avons fait un pacte avec la mort, et nous avons un accord avec le scheol ; quand le fléau débordé passera, il ne viendra pas sur nous, car nous avons le mensonge pour refuge et nous nous sommes cachés sous la fausseté.
\VS{16}C'est pourquoi ainsi parle le Seigneur Yahweh : Voici, je mettrai pour fondement en Sion une pierre\FTNT{Voir commentaire en Es. 8:13-16.}, une pierre éprouvée, la pierre angulaire la plus précieuse, pour être un fondement solide ; celui qui croira ne se hâtera point.
\VS{17}Et je mettrai le jugement à l'équerre, et la justice au niveau ; et la grêle détruira le refuge du mensonge, et les eaux inonderont le lieu où l'on se retirait.
\VS{18}Et votre pacte avec la mort sera détruit, votre accord avec le scheol ne tiendra pas ; quand le fléau débordé passera, vous en serez foulés.
\VS{19}Dès qu'il passera, il vous emportera. Or il passera tous les matins, le jour et la nuit ; et dès qu'on en entendra le bruit, il n'y aura que terreur.
\VS{20}Car le lit sera trop court, et on ne pourra pas s'y étendre, et la couverture trop étroite pour s'en envelopper.
\VS{21}Car Yahweh se lèvera comme à la montagne de Peratsim, et il sera ému comme dans la vallée de Gabaon, pour faire son œuvre, son œuvre extraordinaire, et pour faire son travail, son travail non accoutumé.
\VS{22}Maintenant donc, ne vous moquez plus, de peur que vos liens ne soient renforcés, car j'ai entendu de par le Seigneur, Yahweh des armées, que la destruction est déterminée sur tout le pays. 
\VS{23}Prêtez l'oreille, et écoutez ma voix ; soyez attentifs, et écoutez mon discours !
\VS{24}Celui qui laboure pour semer, laboure-t-il tous les jours ? Ne casse-t-il pas et ne rompt-il pas les mottes de sa terre ? 
\VS{25}Quand il en aura aplani la surface, ne sèmera-t-il pas la vesce\FTNT{La vesce est un genre de plante herbacées de la famille des légumineuse} ; ne répandra-t-il pas le cumin, ne mettra-t-il pas le froment au meilleur endroit, et l'orge en son lieu assigné, et l'épeautre\FTNT{L'épeautre est une espèce de blé} en son quartier ?
\VS{26}Parce que son Dieu l'a instruit, et lui a enseigné ce qu'il faut faire.
\VS{27}Car on ne foule pas la vesce avec la herse\FTNT{La herse est un instrument agricole permettant de travailler la terre en surface}, et on ne tourne point la roue du chariot sur le cumin ; mais on bat la vesce avec la verge, et le cumin avec le bâton.
\VS{28}Le blé avec lequel on fait le pain se menuise, car le laboureur ne le foule pas entièrement ; et quoiqu'il l'écrase avec la roue de son chariot, néanmoins il ne le menuisera pas avec ses chevaux.
\VS{29}Cela aussi vient de Yahweh des armées qui est admirable en conseil et magnifique en moyens.
\Chap{29}
\TextTitle{Avertissement d'un châtiment imminent}
\VerseOne{}Malheur à Ariel\FTNT{Ariel : Lion de Dieu, nom appliqué à Jérusalem.}, à Ariel, la ville dont David fit sa demeure ! Ajoutez année à année, qu'on égorge des victimes pour les fêtes.
\VS{2}Mais je mettrai Ariel à l'étroit, il n'y aura que tristesse et deuil ; et elle sera pour moi comme Ariel.
\VS{3}Car je camperai en rond contre toi, et je t'assiégerai avec des tours, et je dresserai contre toi des retranchements.
\VS{4}Et tu seras abaissée, et tu parleras depuis la terre, et ta parole sortira étouffée par la poussière ; et ta voix sortira de terre comme celle d'un esprit de Python , et ta parole marmottera comme si elle sortait de la poussière.
\VS{5}La multitude de tes étrangers sera comme une fine poussière ; et la multitude des guerriers sera comme la balle qui passe, et cela sera pour un petit moment.
\VS{6}Elle sera visitée par Yahweh des armées avec des tonnerres, des tremblements de terre, et un grand bruit\FTNT{Za. 14:13-14 ; Ap. 16:18-19.} ; avec la tempête, le tourbillon, et avec la flamme d'un feu dévorant.
\VS{7}Et la multitude de toutes les nations qui feront la guerre à Ariel, et tous ceux qui la combattront, et ceux qui la serreront de près seront comme un songe d'une vision de nuit.
\VS{8}Et il arrivera que comme celui qui a faim rêve qu'il mange, mais quand il se réveille son âme est vide ; et comme celui qui a soif rêve qu'il boit, mais quand il se réveille il est épuisé, et son âme est altérée ; ainsi sera-t-il de la multitude de toutes les nations qui combattront contre la montagne de Sion.
\TextTitle{Yahweh donne les raisons du châtiment}
\VS{9}Arrêtez-vous et soyez étonnés ! Ecriez-vous et criez ! Ils sont ivres, mais non de vin ; ils chancellent, mais non pas à cause des boissons fortes.
\VS{10}Car Yahweh a répandu sur vous un esprit d'un profond sommeil\FTNT{Ro. 11:8.} ; il a fermé vos yeux, il a bandé ceux de vos prophètes et de vos principaux voyants.
\VS{11}Et toute vision est pour vous comme les paroles d'un livre cacheté que l'on donne à un homme de lettres en lui disant : Nous te prions, lis donc cela ! Et qui répond : Je ne le puis, car il est cacheté ;
\VS{12}puis si on le donne à quelqu'un qui n'est pas un homme de lettres, en lui disant : Nous te prions, lis donc cela ! Et qui répond : Je ne sais pas lire.
\VS{13}C'est pourquoi le Seigneur dit : Parce que ce peuple s'approche de moi de sa bouche et qu'il m'honore de ses lèvres, mais que son cœur est éloigné de moi ; et parce que la crainte qu'il a de moi lui a été enseigné par un commandement d'hommes\FTNT{Mt. 15:8-9 ; Mc. 7:6-7.}.
\VS{14}A cause de cela, voici, je continuerai de faire à l'égard de ce peuple-ci des merveilles et des prodiges étranges ; et la sagesse de ses sages périra, et l'intelligence de ses hommes intelligents disparaîtra.
\VS{15}Malheur à ceux qui cachent profondément leurs desseins, pour les dissimuler à Yahweh, et dont les œuvres sont dans les ténèbres, et qui disent : Qui nous voit, et qui nous connaît\FTNT{Es. 47:10 ; Ez. 8:12 ; Ps. 10:11 ; Ps. 94:7.} ?
\VS{16}Ce que vous renversez ne sera-t-il pas réputé comme l'argile d'un potier ? Même l'ouvrage dira-t-il de celui qui l'a fait : Il ne m'a point fait ? Et la chose formée dira-t-elle de celui qui l'a formée : Il n'a point d'intelligence\FTNT{Ps. 100:3.} ?
\TextTitle{Yahweh rachète Jacob}
\VS{17}Le Liban ne sera-t-il pas encore dans très peu de temps changé en un Carmel ? Et Carmel ne sera-t-il pas considéré comme une forêt ?
\VS{18}En ce jour-là, les sourds entendront les paroles du livre, et les yeux des aveugles, étant délivrés de l'obscurité et des ténèbres, verront\FTNT{Mt. 11:5 ; Lu. 7:22.}.
\VS{19}Les humbles auront joie sur joie en Yahweh, et les pauvres d'entre les hommes se réjouiront dans le Saint d'Israël\FTNT{Mt. 5:3-11.}.
\VS{20}Car l'oppresseur prendra fin, le moqueur sera consumé, et tous ceux qui veillaient pour commettre l'iniquité seront retranchés\FTNT{Ap. 20:10.},
\VS{21}ceux qui rendaient coupable les hommes pour une parole, qui tendaient des pièges à celui qui les reprenait à la porte, et qui faisaient tomber le juste en confusion. 
\VS{22}C'est pourquoi ainsi parle Yahweh, lui qui a racheté Abraham, à la maison de Jacob : Jacob ne sera plus honteux, et sa face ne pâlira plus.
\VS{23}Car quand il verra ses fils, ouvrage de mes mains, au milieu de lui, ils sanctifieront mon Nom ; ils sanctifieront, dis-je, le Saint de Jacob, et ils craindront le Dieu d'Israël.
\VS{24}Et ceux dont l'esprit s'était fourvoyé deviendront intelligents, et ceux qui murmuraient apprendront la doctrine.
\Chap{30}
\TextTitle{Mise en garde contre les alliances étrangères}
\VerseOne{}Malheur aux enfants rebelles, dit Yahweh, qui prennent des conseils, et non pas de moi, et qui se forgent des idoles de métal où mon esprit n'est point, afin d'ajouter péché sur péché.
\VS{2}Qui sans avoir interrogé ma bouche, marchent pour descendre en Egypte, afin de se fortifier de la force de Pharaon et se retirer sous l'ombre de l'Egypte\FTNT{Jé. 42:19}.
\VS{3}Car la force de Pharaon sera pour vous une honte, et le refuge sous l'ombre de l'Egypte votre confusion.
\VS{4}Car ses princes sont à Tsoan, et ses messagers ont atteint Hanès.
\VS{5}Tous seront rendus honteux par un peuple qui ne leur profitera de rien, ils n'en recevront aucun secours ni aucun avantage, il sera leur honte et leur opprobre.
\VS{6}Les bêtes sont chargées pour aller au midi, ils portent leurs richesses sur les dos des ânons, et leurs trésors sur la bosse des chameaux, vers le peuple qui ne leur profitera point dans le pays de détresse et d'angoisse, d'où viennent le vieux lion et le lion, la vipère et le serpent volant ; .
\VS{7}Car le secours de l'Egypte n'est que vanité et néant ; c'est pourquoi je crie ceci : Leur force est de se tenir tranquille.
\VS{8}Va maintenant, et écris-le en leur présence sur une table, et rédige-le par écrit dans un livre, afin que cela demeure pour le temps à venir, à perpétuité, à jamais ;
\VS{9}que c'est ici un peuple rebelle, des enfants menteurs, des enfants qui ne veulent point écouter la loi de Yahweh\FTNT{No. 20: 3-5 ; De. 9:7 ; Ac. 7:51.} ;
\VS{10}qui disent aux voyants : Ne voyez pas ! Et aux prophètes : Ne nous prophétisez pas des choses droites, mais dites-nous des choses agréables, voyez des choses trompeuses\FTNT{2 Ti. 4:3-4 ; Mi. 2:6.} !
\VS{11}Retirez-vous du chemin, détournez-vous du sentier, éloignez de notre présence le Saint d'Israël\FTNT{Jn. 14:6.}.
\VS{12}C'est pourquoi ainsi dit le Saint d'Israël : Parce que vous rejetez cette parole et que vous vous confiez dans l'oppression et dans les détours, et que vous vous êtes appuyés sur ces choses,
\VS{13}à cause de cela, cette iniquité sera pour vous comme la fente d'une muraille qui va tomber, un renflement dans un mur élevé, dont la ruine vient soudainement, et en un instant.
\VS{14}Il la brise donc comme on brise un vase de terre, que l'on n'épargne point, et de ses pièces, il ne se trouve pas un tesson pour prendre du feu au foyer, ou pour puiser de l'eau à la citerne.
\TextTitle{La confiance en Yahweh, la vraie force}
\VS{15}Car ainsi a parlé le Seigneur Yahweh, le Saint d'Israël : En vous tenant tranquille et en repos vous serez sauvés ; votre force sera en vous tenant en repos et en espérance. Mais vous ne l'avez point voulu.
\VS{16}Et vous avez dit : Non, mais nous nous enfuirons sur des chevaux ; à cause de cela vous vous enfuirez. Et vous avez dit : Nous monterons sur des chevaux rapides ; à cause de cela ceux qui vous poursuivront seront rapides.
\VS{17}Mille d'entre vous s'enfuiront à la menace d'un seul ; vous vous enfuirez à la menace de cinq ; jusqu'à ce que vous soyez abandonnés comme un arbre tout ébranché au sommet d'une montagne, et comme un étendard sur la colline.
\VS{18}Cependant Yahweh attend pour vous faire grâce, et ainsi il sera exalté pour vous faire miséricorde ; car Yahweh est le Dieu de jugement : Ô bienheureux sont tous ceux qui se confient en lui !
\VS{19}Car le peuple demeurera dans Sion et dans Jérusalem. Tu ne pleureras point ! Certes, il te fera grâce dès qu'il entendra ton cri ; dès qu'il aura entendu, il t'exaucera.
\VS{20}Le Seigneur vous donnera du pain de détresse, et de l'eau d'angoisse, mais tes enseignants ne s'envoleront plus, et tes yeux verront tes enseignants.
\VS{21}Et tes oreilles entendront la parole de celui qui sera derrière toi, disant : Voici le chemin, marchez-y ,soit que vous tiriez à droite, soit que vous tiriez à gauche !
\VS{22}Et vous tiendrez pour souillés les chapiteaux des images taillées faites d'argent, et les ornements faits d'or fondu ; tu les jetteras au loin comme un sang impur, et tu leur diras : Hors d'ici ! 
\VS{23}Alors il donnera la pluie sur la semence que tu auras semées en terre, et le grain du revenu de la terre sera abondant et bien nourri ; en ce jour-là, ton bétail paîtra dans un pâturage spacieux\FTNT{Jn. 14:6.}.
\VS{24}Les bœufs et les ânes qui labourent la terre mangeront le pur fourrage de ce qui aura été vanné avec la pelle et le van.
\VS{25}Et il y aura des ruisseaux d'eau courante sur toute haute montagne, et sur toute colline haut élevée, au jour de la grande tuerie, quand les tours tomberont.
\VS{26}Et la lumière de la lune sera comme la lumière du soleil ; et la lumière du soleil sera sept fois plus grande, comme si c'était la lumière de sept jours, le jour où Yahweh bandera la blessure de son peuple, et qu'il guérira la blessure de sa plaie.
\TextTitle{Jugement de Yahweh sur les Assyriens}
\VS{27}Voici, le Nom de Yahweh vient de loin, sa colère est ardente, et une pesante charge ; ses lèvres sont pleines d'indignation, et sa langue est comme un feu dévorant.
\VS{28}Son Esprit est comme un torrent qui déborde et atteint jusqu'au milieu du cou, pour disperser les nations d'une telle dispersion qu'elles seront réduites à néant, et il est comme une bride aux mâchoires des peuples, qui les fera errer.
\VS{29}Vous aurez un cantique comme la nuit où l'on célèbre une fête solennelle ; vous aurez le cœur joyeux comme celui qui marche au son de la flûte, pour aller à la montagne de Yahweh, vers le Rocher d'Israël.
\VS{30}Et Yahweh fera entendre sa voix, pleine de majesté, et il montrera où aura assené son bras dans l'indignation de sa colère, avec une flamme de feu dévorant, avec éclat, tempête, et pierres de grêle.
\VS{31}Car l'Assyrien, qui frappait du bâton, sera effrayé par la voix de Yahweh.
\VS{32}Et partout où passe le bâton dont Yahweh l'a assené, et par lequel il combattra dans les batailles à bras élevé, on entendra les tambourins et les harpes.
\VS{33}Car Topheth\FTNT{Topheth : Lieu pour brûler. Un lieu à l'extrémité sud-est de la vallée de Hinnom au sud de Jérusalem.} est déjà préparée, et même elle est apprêtée pour le roi ; on a fait son bûcher profond et large ; son bûcher c'est du feu et du bois en abondance ; le souffle de Yahweh l'allume comme un torrent de soufre.
\Chap{31}
\TextTitle{Le secours de Yahweh préférable à celui de l'Egypte}
\VerseOne{}Malheur à ceux qui descendent en Egypte pour avoir de l'aide, et qui s'appuient sur les chevaux, et qui mettent leur confiance dans leurs chars parce qu'ils sont nombreux, et en leurs cavaliers quand ils sont bien forts, mais qui ne regardent pas vers le Saint d'Israël, et ne recherchent pas Yahweh.
\VS{2}Et cependant, c'est lui qui est sage, et il fait venir le malheur et ne révoque point sa parole ; il s'élève contre la maison des méchants et contre ceux qui aident les ouvriers d'iniquité.
\VS{3}Or les Egyptiens sont des hommes et non Dieu ; et leurs chevaux sont chair et non esprit. Quand Yahweh étendra sa main, et celui qui donne du secours sera renversé ; et celui à qui le secours est donné tombera ; et eux tous ensemble seront consumés.
\VS{4}Mais ainsi m'a dit Yahweh : Comme le lion, comme le lionceau rugit sur sa proie, et quoiqu'on appelle contre lui un grand nombre de bergers, il ne se laisse ni effrayer par leur cri, ni abaisser par leur bruit ; ainsi Yahweh des armées descendra pour combattre en faveur de la montagne de Sion et de sa colline.
\VS{5}Comme les oiseaux volent, ainsi Yahweh des armées défendra Jérusalem, la défendant et la délivrant, passant outre et la sauvant\FTNT{De. 32:11 ; Ps. 91:4 ; Mt. 23:37.}.
\VS{6}Retournez vers celui de qui les enfants d'Israël se sont étrangement éloignés.
\VS{7}Car en ce jour-là, chacun rejettera ses idoles d'argent et ses idoles d'or que vos propres mains ont fabriquées pour vous faire pécher.
\VS{8}Et l'Assyrien tombera par l'épée qui n'est pas celle d'un vaillant homme, et l'épée qui n'est pas celle d'un homme le dévorera ; et il s'enfuira devant l'épée, et ses jeunes hommes seront rendus tributaires.
\VS{9}Et saisi de frayeur, il s'enfuira à sa forteresse, et ses chefs seront effrayés à cause de la bannière, dit Yahweh, qui a son feu dans Sion et son fourneau dans Jérusalem.
\Chap{32}
\TextTitle{La venue de l'Esprit annonce la paix et la justice}
\VerseOne{}Voici, un roi régnera selon la justice, et les princes gouverneront avec équité.
\VS{2}Et un homme sera comme le lieu où l'on se cache du vent et comme un asile contre la tempête ; comme des ruisseaux d'eau dans un pays sec, et l'ombre d'un grand rocher dans une terre altérée.
\VS{3}Alors les yeux de ceux qui voient ne seront point retenus, et les oreilles de ceux qui entendent seront attentives.
\VS{4}Et le cœur des étourdis entendra la science, et la langue de ceux qui balbutient parlera aisément et nettement.
\VS{5}Le chiche ne sera plus appelé libéral, et l'avare trompeur ne sera plus nommé magnifique.
\VS{6}Car l'homme vil dira des choses viles, et son cœur ne machine qu'iniquité, pour exécuter son hypocrisie et pour proférer des faussetés contre Yahweh, pour rendre vide l'âme de celui qui a faim, et faire tarir la boisson de celui qui a soif\FTNT{Jn. 10:10.}.
\VS{7}Les instruments de l'avare sont pernicieux ; il prend des conseils pleins de machinations, pour attraper par des paroles de mensonge les affligés, même quand la cause du pauvre est juste\FTNT{2 Pi. 2:3.}.
\VS{8}Mais le libéral forme des conseils de libéralité et se lève pour user de libéralité.
\VS{9}Femmes qui êtes à votre aise, levez-vous, écoutez ma voix ! Filles qui vous tenez assurées, prêtez l'oreille à ma parole !
\VS{10}Dans un an et quelques jours, vous qui vous tenez assurées serez troublées ; car la vendange a manqué, la récolte n'arrivera plus.
\VS{11}Vous qui êtes à votre aise, tremblez ! Vous qui vous tenez assurées, soyez troublées ! Dépouillez-vous, quittez vos habits et ceignez de sacs vos reins !
\VS{12}On se frappe la poitrine à cause de la vigne abondante en fruits.
\VS{13}Les épines et les ronces montent sur la terre de mon peuple, même sur toutes les maisons où il y a de la joie et sur la ville joyeuse.
\VS{14}Car le palais est abandonné, la multitude de la cité est délaissée ; les lieux inaccessibles du pays et les forteresses serviront de cavernes à toujours ; les ânes sauvages y joueront, et les troupeaux y paîtront,
\VS{15}jusqu'à ce que l'Esprit soit répandu d'en haut sur nous\FTNT{Joë. 2:28 ; Za.12:10 ; Ac. 2:17-18.}, et que le désert devienne un Carmel et que Carmel  soit considéré comme une forêt.
\VS{16}Le jugement habitera dans le désert et la justice se tiendra en Carmel.
\VS{17}La justice produira de la paix, et le fruit de la justice sera le repos et la sécurité pour toujours.
\VS{18}Mon peuple habitera dans une demeure paisible, et dans des habitations assurées, et dans un repos fort tranquille.
\VS{19}Mais la grêle tombera sur la forêt, et la ville sera entièrement abaissée.
\VS{20}Heureux vous qui semez sur toutes les eaux, et qui laissez sans entraves le pied du bœuf et de l'âne !
\Chap{33}
\TextTitle{Yahweh se lève}
\VerseOne{}Malheur à toi qui dépouilles et qui n'as pas été dépouillé ! Qui pilles et qu'on n'a pas encore pillé ! Quand tu auras fini de dépouiller, tu seras dépouillé ; et quand tu auras achevé de piller, on te pillera.
\VS{2}Yahweh, aie pitié de nous ! Nous nous attendons à toi ! Sois leur bras dès le matin et notre délivrance au temps de la détresse !
\VS{3}Au son du tumulte, les peuples s'enfuient ; quand tu te lèves, les nations se dispersent.
\VS{4}Et votre butin est recueilli comme on rassemble les sauterelles ; on saute  dessus comme sautellent les sauterelles.
\VS{5}Yahweh est élevé, car il habite dans les lieux élevés ; il remplit Sion de jugement et de justice\FTNT{Ps. 97:9.}.
\VS{6}Et la sagesse et la science seront la certitude de ta durée, et la force de ton salut ; la crainte de Yahweh est son trésor.
\VS{7}Voici, leurs hérauts poussent des cris au-dehors, et les messagers de paix pleurent amèrement.
\VS{8}Les routes sont réduites en désolation, les passants n'y passent plus. Il a rompu l'alliance, il rejette les villes, il ne fait plus cas des hommes.
\VS{9}On mène le deuil, la terre languit. Le Liban est honteux et flétri. Le Saron est comme un désert. Le Basan et le Carmel secouent leur feuillage.
\VS{10}Maintenant je me lèverai, dit Yahweh, maintenant je serai exalté, maintenant je serai élevé.
\VS{11}Vous avez conçu du foin, et vous enfanterez de la paille ; votre souffle vous dévorera comme le feu.
\VS{12}Et les peuples seront des fourneaux de chaux ; ils seront brûlés au feu comme des épines coupées.
\VS{13}Vous qui êtes loin, écoutez ce que j'ai fait ! Et vous qui êtes près, connaissez ma force !
\TextTitle{Yahweh assure la paix aux justes}
\VS{14}Les pécheurs sont effrayés dans Sion, et le tremblement saisit les hypocrites, tellement qu'ils disent : Qui de nous pourra séjourner avec le feu dévorant\FTNT{Hé. 12:29.} ? Qui de nous pourra séjourner avec les flammes éternelles ?
\VS{15}Celui qui observe la justice et qui profère des choses droites ; celui qui rejette le gain déshonnête d'extorsion, et qui secoue ses mains pour ne pas accepter un présent ; celui qui bouche ses oreilles pour ne pas entendre des propos sanguinaires, et qui ferme ses yeux pour ne pas voir le mal,
\VS{16}Celui-là habitera dans des lieux élevés, des forteresses assises sur des rochers seront sa haute retraite ; son pain lui sera donné, et ses eaux ne lui manqueront point\FTNT{Jn. 4:14 ; Jn. 6:33-35 ; Ap 21:6.}.
\VS{17}Tes yeux contempleront le roi dans sa beauté ; et ils regarderont la terre éloignée.
\VS{18}Ton cœur méditera-il la frayeur, en disant : Où est le secrétaire, où est le trésorier ? Où est celui qui tient le compte des tours ?
\VS{19}Tu ne verras plus le peuple fier, le peuple au langage inconnu qu'on n'entend pas, et de langue bégayante qu'on ne comprend pas.
\VS{20}Regarde Sion, la ville de nos fêtes solennelles ! Que tes yeux voient Jérusalem, séjour tranquille, tabernacle qui ne sera pas transportée, et dont les pieux ne seront jamais ôtés, et dont les cordages ne seront point rompus\FTNT{Ap. 21:2.}.
\VS{21}C'est là que Yahweh nous est glorieux ; c'est le lieu de fleuves, de vastes rivières, où n'ira pas de navire à rame et où aucun gros navire passera.
\VS{22}Parce que Yahweh est notre Juge, Yahweh est notre Législateur, Yahweh est notre Roi\FTNT{Jésus-Christ exerce toutes les fonctions gouvernementales : législatives, exécutives et judiciaires.} ; c'est lui qui vous sauvera.
\VS{23}Tes cordages sont lâchés ; et ainsi ils ne tiennent point ferme leur mât et on n'étendra point la voile. Alors la dépouille d'un grand butin est partagé ; même les boiteux pillent le butin.
\VS{24}Et celui qui fait sa demeure dans la maison ne dit point : Je suis malade ! Le peuple qui habite en elle reçoit le pardon de ses iniquités.
\Chap{34}
\TextTitle{Le jugement des nations\FTNTT{Ap. 19:17-21.}}
\VerseOne{}Approchez-vous nations, pour écouter ! Et vous peuples, soyez attentifs ! Que la terre et tout ce qui la remplit écoute ! Que le monde habitable et tout ce qui y est produit écoute !
\VS{2}Car l'indignation de Yahweh est sur toutes les nations, et sa fureur sur toute leur armée ; il les voue à l'interdit, il les livre pour être tuées.
\VS{3}Leurs blessés à morts sont jetés là, et la puanteur de leurs corps morts se répand et les montagnes découlent de leur sang.
\VS{4}Et toute l'armée des cieux se fond ; les cieux sont roulés comme un livre\FTNT{Ap. 6:14.}, et toute leur armée tombe, comme tombe la feuille de la vigne, et comme tombe celle du figuier\FTNT{Mt. 24:28 ; Mc. 13:25.}.
\VS{5}Parce que mon épée s'est enivrée dans les cieux, voici, elle va descendre en jugement contre Edom, et contre le peuple que j'ai voué à l'interdit.
\VS{6}L'épée de Yahweh est pleine de sang ; engraissée de graisse, et du sang des agneaux et des boucs, et de la graisse des reins de béliers ; car il y a des sacrifices de Yahweh à Botsra, et une grande tuerie dans le pays d'Edom.
\VS{7}Les licornes descendent avec eux, et les bœufs avec les taureaux ; leur terre est enivrée de sang, et leur poussière engraissée de graisse.
\VS{8}Car c'est un jour de vengeance pour Yahweh, une année de rétribution pour maintenir la cause de Sion\FTNT{Jé. 46:10 ; Joë. 2:2 ; So. 1:15.}.
\VS{9}Et ces torrents d'Edom seront changés en poix, et sa poussière en soufre, et sa terre deviendra de la poix ardente.
\VS{10}Elle ne sera point éteinte ni jour ni nuit ; sa fumée montera éternellement, elle sera désolée de génération en génération ; il n'y aura personne qui passe par elle à jamais.
\VS{11}Le pélican et le hérisson la posséderont, la chouette et le corbeau  y habiteront ; et on étendra sur elle la ligne de la désolation et le niveau de désordre.
\VS{12}Ses magistrats crieront qu'il n'y a plus là de royaume, et tous ses princes seront réduits à néant.
\VS{13}Les épines croîtront dans ses palais, les chardons et les buissons dans ses forteresses; elle sera la demeure des dragons, et le parvis des hiboux.
\VS{14}Les bêtes sauvages des déserts rencontreront les bêtes sauvages des îles ; et les boucs s'y appelleront les uns les autres ; là aussi, la Lilith\FTNT{Lilith est le nom d'une déesse de la nuit connue pour être un démon nocturne qui hantait les lieux déserts d'Edom.} aura sa demeure et trouvera son lieu de repos ;
\VS{15}là le martinet fera son nid, déposera ses œufs, les couvera, et recueillera ses petits à son ombre ; et là aussi se rassembleront tous les vautours.
\VS{16}Consultez le livre de Yahweh et lisez : Il n'en manquera pas un seul point ; ni l'un ni l'autre ne manqueront ; car c'est ma bouche qui l'a ordonné, et son Esprit qui les rassemblera.
\VS{17}Car il leur a jeté le sort, et sa main leur a partagé cette terre au cordeau, ils la posséderont toujours, ils l'habiteront d'âge en âge.
\Chap{35}
\TextTitle{Yahweh se révèle et sauve son peuple}
\VerseOne{}Le désert et le lieu aride seront dans la joie ; le lieu solitaire se réjouira et fleurira comme une rose.
\VS{2}Il fleurira abondamment, et se réjouira, se réjouissant même et chantant en triomphe. La gloire du Liban lui est donnée, avec la magnificence de Carmel et de Saron ; ils verront la gloire de Yahweh et la magnificence de notre Dieu.
\VS{3}Renforcez les mains lâches, et fortifiez les genoux tremblants\FTNT{Hé. 12:12.}.
\VS{4}Dites à ceux qui ont le cœur troublé : Prenez courage et ne craignez plus\FTNT{Jn. 14:1 ; Jn. 16:33.} ; voici votre Dieu, la vengeance viendra, la rétribution de Dieu ; il viendra lui-même et vous délivrera.
\VS{5}Alors les yeux des aveugles seront ouverts, et les oreilles des sourds seront débouchées.
\VS{6}Alors le boiteux sautera comme un cerf, et la langue du muet chantera en triomphe\FTNT{Esaïe a annoncé la venue de Yahweh lui-même. Cette prophétie s'est parfaitement accomplie en Jésus-Christ qui a réalisé tout ce qui avait été prédit. « Allez rapporter à Jean ce que vous entendez et ce que vous voyez : Les aveugles voient, les boiteux marchent, les lépreux sont purifiés, les sourds entendent, les morts ressuscitent, et l'Evangile est annoncé aux pauvres » (Mt. 11:4-5).}. Car des eaux jailliront dans le désert, et des torrents dans le lieu solitaire.
\VS{7}Et les lieux secs deviendront des étangs, et la terre desséchée deviendra des sources d'eaux ; et dans les repaires où des dragons faisaient leur gîte, il y aura un parvis à roseaux et à joncs.
\VS{8}Il y aura là un sentier et un chemin, qu'on appellera le chemin de sainteté ; celui qui est souillé n'y passera point, mais il sera pour ceux-là ; celui qui va son chemin, et les insensés ne s'y égareront point\FTNT{Mt. 7:13-14 ; Jn. 14:6.}.
\VS{9}Là il n'y aura point de lion ; et aucune des bêtes qui ravissent les autres, n'y montera, et ne s'y trouvera ; mais les rachetés y marcheront.
\VS{10}Ceux dont Yahweh a payé la rançon\FTNT{Jésus-Christ est Yahweh qui a payé notre rançon (Mc. 10:45).}, retourneront, et viendront en Sion avec chant de triomphe, et une joie éternelle sera sur leur tête ; ils obtiendront la joie et l'allégresse ; la douleur et le gémissement s'enfuiront.
\Chap{36}
\TextTitle{Invasion de Sanchérib, menaces de Rabschaké\FTNTT{2 R. 18:9-37 ; 2 Ch. 32:1-19.}}
\VerseOne{}La quatorzième année du roi Ezéchias, Sanchérib, roi d'Assyrie, monta contre toutes les villes fortes de Juda et les prit\FTNT{2 R. 18:17.}.
\VS{2}Puis le roi d'Assyrie envoya de Lakis à Jérusalem, vers le roi Ezéchias, Rabschaké avec une puissante armée. Rabschaké s'arrêta à l'aqueduc de l'étang supérieur, sur le chemin du champ du foulon.
\VS{3}Alors Eliakim, fils de Hilkija, chef de la maison du roi, Schebna, le secrétaire, et Joach, fils d'Asaph, l'archiviste, sortirent vers lui.
\VS{4}Rabschaké leur dit : Dites maintenant à Ezéchias : Ainsi parle le grand roi, le roi d'Assyrie : Quelle est cette confiance que tu as ?
\VS{5}Je te le dis, ce ne sont là que des paroles ; mais il faut pour la guerre de la prudence et de la force. Or maintenant en qui t'es tu confié pour t'être rebellé contre moi ?
\VS{6}Voici, tu t'es confié sur ce bâton qui n'est qu'un roseau cassé, sur l'Egypte, qui perce et traverse la main de celui qui s'appuie dessus ; tel est Pharaon, roi d'Egypte, à tous ceux qui se confient en lui.
\VS{7}Que si tu me dis : Nous nous confions en Yahweh, notre Dieu. Mais n'est-ce pas lui dont Ezéchias a ôté les hauts lieux et les autels, en disant à Juda et à Jérusalem : Vous vous prosternerez devant cet autel-ci ?
\VS{8}Maintenant donc, donne des otages au roi d'Assyrie, mon maître ; et je te donnerai deux mille chevaux, si tu peux donner autant d'hommes pour monter dessus.
\VS{9}Et comment ferais-tu tourner le visage à un seul gouverneur d'entre les moindres serviteurs de mon maître ? Mais tu te confies en l'Egypte pour les chars et pour les cavaliers.
\VS{10}Mais suis-je monté sans Yahweh dans ce pays pour le détruire ? Yahweh m'a dit : Monte contre ce pays et détruis-le.
\VS{11}Alors Eliakim, Schebna et Joach dirent à Rabschaké : Nous te prions de parler en langue araméenne à tes serviteurs, car nous la comprenons ; mais ne parle pas en langue judaïque, pendant que le peuple qui est sur la muraille l'écoute.
\VS{12}Et Rabschaké répondit : Mon maître m'a-t-il envoyé vers ton maître ou vers toi, pour dire ces paroles là ? Ne m'a-t-il pas envoyé vers les hommes qui se tiennent sur la muraille, pour leur dire qu'ils mangeront leur propre fiente, et qu'ils boiront leur urine avec vous ?
\VS{13}Puis Rabschaké se dressa et s'écria à haute voix en langue judaïque, et dit : Ecoutez les paroles du grand roi, du roi d'Assyrie !
\VS{14}Ainsi parle le roi : Qu'Ezéchias ne vous séduise pas, car il ne pourra pas vous délivrer.
\VS{15}Qu'Ezéchias ne vous fasse pas confier en Yahweh, en disant : Yahweh nous délivrera certainement ; cette ville ne sera point livrée entre les mains du roi d'Assyrie.
\VS{16}N'écoutez point Ezéchias ; car ainsi parle le roi d'Assyrie : Faites un accord avec moi pour votre bien, et sortez vers moi, et vous mangerez chacun  de sa vigne, et chacun de son figuier, et vous boirez chacun de l'eau de sa citerne,
\VS{17}jusqu'à ce que je vienne, et que je vous emmène dans un pays qui est comme votre pays, un pays de blé et de bon vin, un pays de pain et de vignes.
\VS{18}Qu'Ezéchias donc ne vous séduise point, en disant : Yahweh nous délivrera. Les dieux des nations ont-ils délivré chacun leur pays de la main du roi d'Assyrie ?
\VS{19}Où sont les dieux de Hamath et d'Arpad ? Où sont les dieux de Sepharvaïm ? Ont-ils délivré Samarie de ma main ?
\VS{20}Qui sont ceux d'entre tous les dieux de ces pays qui aient délivré leur pays de ma main, pour que Yahweh délivre Jérusalem de ma main ?
\VS{21}Mais ils se turent et ne lui répondirent pas un mot ; car le roi avait donné cet ordre, disant : Vous ne lui répondrez pas.
\TextTitle{Ezéchias informé des menaces}
\VS{22}Après cela, Eliakim fils de Hilkija, chef de la maison du roi, Schebna, le secrétaire, et Joach, fils d'Asaph l'archiviste, s'en revinrent auprès d'Ezéchias, les vêtements déchirés, et lui rapportèrent les paroles de Rabschaké.
\Chap{37}
\TextTitle{Ezéchias recherche Yahweh auprès d'Esaïe\FTNTT{2 R. 19:1-7 ; 2 Ch. 32:20.}}
\VerseOne{}Et il arriva qu'aussitôt que le roi Ezéchias eut entendu ces choses, il déchira ses vêtements, se couvrit d'un sac, et entra dans la maison de Yahweh\FTNT{2 R. 19:1-7 ; 2 Ch. 32:20.}.
\VS{2}Puis il envoya Eliakim, chef de la maison du roi, et Schebna, le secrétaire, et les plus anciens des sacrificateurs couverts de sacs, vers Esaïe, le prophète, fils d'Amots.
\VS{3}Et ils lui dirent : Ainsi parle Ezéchias : Ce jour est un jour d'angoisse, de répréhension et de blasphème ; car les enfants sont près de sortir du sein maternel, mais il n'y a point de force pour enfanter.
\VS{4}Peut-être que Yahweh, ton Dieu, a-t-il entendu les paroles de Rabschaké, que le roi d'Assyrie, son maître, a envoyé pour blasphémer le Dieu vivant et lui faire outrage ; selon les paroles que Yahweh, ton Dieu, a entendues ; fais donc requête pour le reste qui subsiste encore.
\VS{5}Les serviteurs du roi Ezéchias vinrent vers Esaïe.
\VS{6}Et Esaïe leur dit : Voici ce que vous direz à votre maître : Ainsi parle Yahweh : Ne crains point pour les paroles que tu as entendues, par lesquelles les serviteurs du roi d'Assyrie m'ont blasphémé.
\VS{7}Voici, je vais mettre en lui un esprit tel qu'ayant entendu une certaine rumeur, il retournera dans son pays, et je le ferai tomber par l'épée dans son pays.
\TextTitle{Provocation et menace de Sanchérib\FTNTT{2 R. 19:8-13 ; 2 Ch. 32:17-19.}}
\VS{8}Or quand Rabschaké s'en fut retourné, il alla trouver le roi d'Assyrie qui attaquait Libna, car il avait appris qu'il était parti de Lakis.
\VS{9}Alors le roi d'Assyrie ayant entendu dire au sujet de Tirhaka, roi d'Ethiopie : Il est sorti pour te faire la guerre. Dès qu'il eut entendu cela, il envoya des messagers à Ezéchias, en leur disant :
\VS{10}Vous parlerez ainsi à Ezéchias, roi de Juda : Que ton Dieu, auquel tu te confies, ne te séduise point, en disant : Jérusalem ne sera point livrée entre les mains du roi d'Assyrie.
\VS{11}Voilà, tu as entendu ce que les rois d'Assyrie ont fait à tous les pays, en les détruisant entièrement ; et toi, tu échapperais ?
\VS{12}Les dieux des nations que mes ancêtres ont détruites, à savoir Gozan, Charan, Retseph, et les fils d'Eden, qui sont à Telassar, les ont-ils délivrées ?
\VS{13}Où sont le roi de Hamath, le roi d'Arpad, et le roi de la ville de Sepharvaïm, d'Héna et d'Ivva ?
\TextTitle{Prière d'Ezéchias à Yahweh\FTNTT{2 R. 19:14-19 ; 2 Ch. 32:20.}}
\VS{14}Et quand Ezéchias reçut les lettres de la main des messagers et les lut, il monta à la maison de Yahweh, et Ezéchias les déploya devant Yahweh.
\VS{15}Puis Ezéchias fit sa prière à Yahweh, en disant :
\VS{16}Ô Yahweh des armées ! Dieu d'Israël qui es assis entre les chérubins ! C'est toi qui es le seul Dieu de tous les royaumes de la terre, c'est toi qui as fait les cieux et la terre.
\VS{17}Ô Yahweh ! Incline ton oreille et écoute ! Ô Yahweh ! Ouvre tes yeux et regarde ! Ecoute les paroles de Sanchérib, qu'il m'a envoyé dire pour blasphémer le Dieu vivant.
\VS{18}Il est bien vrai, ô Yahweh, que les rois d'Assyrie ont détruit tous les pays et leurs contrées ;
\VS{19}et qu'ils ont jeté dans le feu leurs dieux ; mais ce n'étaient point des dieux, mais un ouvrage de mains d'homme, du bois et de la pierre ; c'est pourquoi ils les ont détruits.
\VS{20}Maintenant donc, ô Yahweh notre Dieu ! Délivre-nous de la main de Sanchérib, afin que tous les royaumes de la terre sachent que toi seul es Yahweh.
\TextTitle{Esaïe transmet la réponse de Yahweh\FTNTT{2 R. 19:20-34.}}
\VS{21}Alors Esaïe, fils d'Amots, envoya dire à Ezéchias : Ainsi parle Yahweh, le Dieu d'Israël : J'ai entendu la prière que tu m'as faite au sujet de Sanchérib, roi d'Assyrie.
\VS{22}C'est ici la parole que Yahweh a prononcée contre lui : La vierge, fille de Sion, te méprise et se moque de toi ; la fille de Jérusalem hoche la tête après toi.
\VS{23}Contre qui as-tu élevé ta voix, et levé tes yeux en haut ? C'est contre le Saint d'Israël.
\VS{24}Tu as outragé le Seigneur par le moyen de tes serviteurs, et tu as dit : Je suis monté avec la multitude de mes chars sur le haut des montagnes, aux côtés du Liban, je couperai les plus hauts cèdres, et les plus beaux cyprès qui y soient, et j'entrerai jusqu'en son plus haut bout, et en la forêt de son Carmel.
\VS{25}J'ai creusé des sources, et j'en ai bu les eaux, et je tarirai avec la plante de mes pieds tous les fleuves de l'Egypte.
\VS{26}N'as-tu pas appris, qu'il y a déjà longtemps, j'ai fait cette ville, et que dès les temps anciens je l'ai ainsi formée ? Et maintenant l'aurais-je conservée pour être réduite en désolation, et les villes fortes en monceaux de ruines ?
\VS{27}Or leurs habitants, étant dénués de force, ont été épouvantés et confus ; ils sont devenus comme l'herbe des champs ; et l'herbe verte, comme le foin des toits, et le blé brûlé avant la formation de sa tige.
\VS{28}Mais je sais quand tu t'assieds, quand tu sors et quand tu entres, et comment tu es furieux contre moi\FTNT{Ps. 139:2.}.
\VS{29}Parce que tu es furieux contre moi, et que ton insolence est montée à mes oreilles, je mettrai ma boucle à tes narines, et mon mors en ta bouche, et je te ferai retourner par le chemin par lequel tu es venu.
\VS{30}Et ceci te sera pour signe, ô Ezéchias, c'est qu'on mangera cette année ce qui viendra de soi-même aux champs ; et en la deuxième année ce qui croîtra encore sans semer ; mais la troisième année, vous sèmerez, vous moissonnerez, vous planterez des vignes, et vous en mangerez le fruit.
\VS{31}Et ce qui est réchappé, et demeuré de reste dans la maison de Juda, étendra sa racine par-dessous, et elle produira du fruit par-dessus.
\VS{32}Car il sortira de Jérusalem un reste, et de la montagne de Sion quelques réchappés, la jalousie de Yahweh des armées fera cela.
\VS{33}C'est pourquoi ainsi parle Yahweh sur le roi d'Assyrie : Il n'entrera point dans cette ville, il n'y jettera aucune flèche, il ne se présentera point contre elle avec le bouclier, et il ne dressera point de retranchements contre elle.
\VS{34}Il s'en retournera par le chemin par lequel il est venu, et il n'entrera point dans cette ville, dit Yahweh.
\VS{35}Car je protégerai cette ville pour la délivrer pour l'amour de moi, et pour l'amour de David, mon serviteur.
\TextTitle{Yahweh frappe Sanchérib\FTNTT{2 R. 19:35-37 ; 2 Ch. 32:21.}}
\VS{36}L'ange de Yahweh\FTNT{Ge. 16:7.} sortit et frappa cent quatre-vingt-cinq mille hommes dans le camp des Assyriens. Et quand on se leva le matin, voici, ils étaient tous morts.
\VS{37}Alors Sanchérib, roi d'Assyrie, partit de là ; il s'en alla et s'en retourna, et il se tint à Ninive.
\VS{38}Et il arriva qu'étant prosterné dans la maison de Nisroc\FTNT{Le nom Nisroc signifie « le grand aigle ». C'était une idole de Ninive adorée par Sanchérib, symbolisée par un aigle à figure humaine.}, son dieu, Adrammélec et Scharetser, ses fils, le tuèrent avec l'épée ; puis ils s'enfuirent au pays d'Ararat. Et Esar-Haddon, son fils, régna à sa place.
\Chap{38}
\TextTitle{Maladie et guérison d'Ezéchias\FTNTT{2 R. 20:1-11 ; 2 Ch. 32:24-30.}}
\VerseOne{}En ces jours-là, Ezéchias fut malade à la mort\FTNT{2 R. 20:1-11 ; 2 Ch. 32:24-30.}. Et Esaïe le prophète, fils d'Amots, vint auprès de lui, et lui dit : Ainsi parle Yahweh : Donne tes ordres à ta maison, car tu vas mourir et tu ne vivras plus.
\VS{2}Alors Ezéchias tourna sa face contre la muraille et fit sa prière à Yahweh,
\VS{3}et dit : Ô Yahweh, souviens-toi maintenant je te prie que j'ai marché devant toi en vérité et en intégrité de cœur, et que j'ai fait ce qui est agréable à tes yeux ! Et Ezéchias pleura abondamment.
\VS{4}Puis la parole de Yahweh fut adressée à Esaïe, en disant :
\VS{5}Va, et dis à Ezéchias ainsi parle Yahweh, le Dieu de David, ton père : J'ai exaucé ta prière, j'ai vu tes larmes. Voici, j'ajouterai à tes jours quinze années.
\VS{6}Et je te délivrerai de la main du roi d'Assyrie, toi et cette ville, et je défendrai cette ville.
\VS{7}Et ce signe t'est donné par Yahweh, pour voir que Yahweh accomplira la parole qu'il a prononcée.
\VS{8}Voici, je ferai retourner de dix degrés en arrière avec le soleil l'ombre des degrés qui est descendue sur les degrés d'Achaz. Et le soleil retourna de dix degrés par les degrés par lesquels il était descendu.
\VS{9}Or c'est ici l'écrit d'Ezéchias, roi de Juda, sur sa maladie et sur son rétablissement.
\VS{10}J'avais dit dans le retranchement de mes jours : Je m'en irai aux portes du scheol, je suis privé de ce qui restait de mes années.
\VS{11}Je disais : Je ne contemplerai plus Yahweh, Yahweh sur la terre des vivants ; je ne verrai plus aucun homme parmi les habitants du monde !
\VS{12}Ma durée s'en est allée, et a été transportée loin de moi, comme une cabane de berger ; ma vie est coupée je suis retranché comme la toile que le tisserand détache de sa trame. Du matin au soir tu m'auras enlevé\FTNT{Aux versets 12 et 13, le mot qui a été traduit par « enlevé » est « shalam » : « être dans une alliance de paix, être en paix ».} !
\VS{13}Je pensais en moi-même jusqu'au matin ; comme un lion, qui briserait ainsi tous mes os ; du matin au soir tu m'auras enlevé !
\VS{14}Je grommelais comme la grue et l'hirondelle ; je gémissais comme la colombe ; mes yeux défaillaient à force de regarder en haut : Ô Yahweh, je suis opprimé, sois mon garant !
\VS{15}Que dirai-je ? Il m'a parlé et lui-même l'a fait. Je m'en irai tout doucement tous les ans de ma vie, dans l'amertume de mon âme.
\VS{16}Seigneur, par ces choses-là on a la vie, et dans toutes ces choses est la vie de mon esprit. Ainsi tu me rétabliras et me feras revivre.
\VS{17}Voici, dans ma paix, une grande amertume m'est survenue, mais tu as embrassé mon âme afin qu'elle ne tombe pas dans la fosse de la pourriture, car tu as jeté tous mes péchés derrière ton dos.
\VS{18}Car le scheol ne te loue point,  la mort ne te célèbre point ; ceux qui sont descendus dans la fosse ne s'attendent plus à ta vérité\FTNT{Ps. 115:17.}.
\VS{19}Mais le vivant, le vivant est celui qui te célèbre, comme moi aujourd'hui ; le père conduira ses enfants à la connaissance de ta vérité\FTNT{Pr. 22:6 ; Ep. 6:4.}.
\VS{20}Yahweh est venu me délivrer, et à cause de cela, nous jouerons sur les instruments mes cantiques, tous les jours de notre vie dans la maison de Yahweh.
\VS{21}Or Esaïe avait dit : Qu'on prenne une masse de figues sèches et qu'on en fasse un emplâtre sur l'ulcère ; et Ezéchias guérira.
\VS{22}Et Ezéchias avait dit : Quel est le signe que je monterai à la maison de Yahweh ?
\Chap{39}
\TextTitle{Ezéchias montre toutes ses richesses aux Babyloniens\FTNTT{2 R. 20:12-19}}
\VerseOne{}En ce temps-là\FTNT{2 R. 20:12-19.}, Mérodac-Baladan, fils de Baladan, roi de Babylone, envoya des lettres avec un présent à Ezéchias, parce qu'il avait entendu qu'il avait été malade, et qu'il était guéri.
\VS{2}Et Ezéchias en eut de la joie, et il leur montra les cabinets où étaient ses choses précieuses, l'argent, l'or, et les aromates, et l'huile précieuse, tout son arsenal, et tout ce qui se trouvait dans ses trésors ; il n'y eut rien qu'Ezéchias ne leur montra dans sa maison et dans tous ses domaines.
\VS{3}Puis le prophète Esaïe vint vers le roi Ezéchias, et lui dit : Qu'ont dit ces hommes-là, et d'où sont-ils venus vers toi ? Et Ezéchias répondit : Ils sont venus vers moi d'un pays éloigné, de Babylone.
\VS{4}Puis Esaïe dit : Qu'ont-ils vu dans ta maison ? Ezéchias répondit : Ils ont vu tout ce qui est dans ma maison ; il n'y a rien dans mes trésors que je ne leur aie montré.
\VS{5}Et Esaïe dit à Ezéchias : Ecoute la parole de Yahweh des armées :
\VS{6}Voici, les jours viennent où l'on emportera à Babylone tout ce qui est dans ta maison, et ce que tes pères ont amassé dans leurs trésors jusqu'à aujourd'hui ; il n'en restera rien, dit Yahweh\FTNT{2 R. 24:13 ; 2 R. 25:13-15 ; Jé. 20:5.}.
\VS{7}Même on prendra de tes fils qui sortiront de toi, et que tu auras engendrés afin qu'ils soient eunuques dans le palais du roi de Babylone\FTNT{Da. 1:3-4.}.
\VS{8}Et Ezéchias répondit à Esaïe : La parole de Yahweh, que tu as prononcée, est bonne ; et, il ajouta, au moins qu'il y ait paix et sécurité pendant mes jours.
\Chap{40}
\TextTitle{Un nouveau message pour Esaïe}
\VerseOne{}Consolez, consolez mon peuple, dit votre Dieu.
\VS{2}Parlez à Jérusalem selon son cœur, et criez-lui que son temps marqué est accompli, que son iniquité est tenue pour acquittée, qu'elle a reçu de la main de Yahweh le double pour tous ses péchés.
\TextTitle{Mission de Jean-Baptiste\FTNTT{Mt. 3:3.}}
\VS{3}La voix de celui qui crie au désert\FTNT{L'accomplissement de cette prophétie se trouve en Mt 3:3, où il nous est dit que la voix qui devait crier ces choses était celle de Jean-Baptiste (voir aussi Mal. 3:1 ; Mal. 4:5-6 ; Mt. 17:10-13).} est : Préparez le chemin de Yahweh\FTNT{Les évangiles nous enseignent que Jean-Baptiste a été envoyé pour préparer le chemin du Seigneur Jésus (Jn. 1:19-27 ; Jn. 1:29-34 ; Jn. 3:28-31).}, aplanissez parmi les lieux arides un chemin pour notre Dieu.
\VS{4}Toute vallée sera comblée, toute montagne et toute colline seront abaissées, et les lieux tortueux seront redressés, et les lieux raboteux seront aplanis.
\VS{5}Alors la gloire de Yahweh sera manifestée, et toute chair en même temps la verra, car la bouche de Yahweh a parlé.
\TextTitle{La grandeur de Dieu échappe à l'homme}
\VS{6}La voix dit : Crie ! Et on a répondu : Que crierai-je ? Toute chair est comme l'herbe, et toute sa grâce est comme la fleur d'un champ\FTNT{Ja. 1:10 ; 1 Pi. 1:24-25.}.
\VS{7}L'herbe sèche, et la fleur tombe, parce que le vent de Yahweh souffle dessus. Certainement le peuple est comme l'herbe.
\VS{8}L'herbe sèche, et la fleur tombe, mais la parole de notre Dieu demeure éternellement.
\VS{9}Sion, qui annonce de bonnes nouvelles, monte sur une haute montagne ; Jérusalem, qui annonce de bonnes nouvelles, élève ta voix avec force ; élève-la, ne crains point ; dis aux villes de Juda : Voici votre Dieu !
\VS{10}Voici, le Seigneur Yahweh\FTNT{Jésus-Christ est Yahweh qui vient (Es. 35:4 ; Es. 40:10-11 ; Es. 60:1 ; Es. 62:11-12 ; Es. 66:15-16 ; Za. 14:1-7 ; Mt. 24 ; Jn. 14:1-3; Ac. 1:10-12 ; Ap. 3:11 ; Ap. 19:11-12 ; Ap. 22:7 ; Ap. 22:12 ; Ap. 22:20).} viendra contre le fort, et son bras dominera sur lui ; voici son salaire est avec lui, et ses rétributions sont devant lui.
\VS{11}Il paîtra son troupeau comme un berger, il rassemblera les agneaux dans ses bras, il les placera dans son sein ; il conduira celles qui allaitent\FTNT{Jn. 10.}.
\VS{12}Qui est celui qui a mesuré les eaux avec le creux de sa main, et qui a pris les dimensions des cieux avec la paume, qui a rassemblé toute la poussière de la terre dans un boisseau, et qui a pesé au crochet les montagnes et les collines à la balance ?
\VS{13}Qui a dirigé l'Esprit de Yahweh, ou qui a été son conseiller pour l'enseigner\FTNT{1 Co. 2:16 ; Ro. 11:34.} ?
\VS{14}Avec qui a-t-il pris conseil, et qui l'a instruit, et lui a enseigné le sentier de jugement ? Qui lui a enseigné la science, et lui a montré le chemin de l'intelligence ?
\VS{15}Voilà, les nations sont comme une goutte qui tombe d'un seau, et elles sont réputée comme la menue poussière d'une balance ; voila, il a jeté çà et là les îles comme de la poudre.
\VS{16}Et le Liban ne suffirait pas pour faire le feu, et les bêtes qui y sont ne seraient pas suffisantes pour l'holocauste.
\VS{17}Toutes les nations sont devant lui comme un rien, et il ne les considère que comme de la poussière, et comme un néant.
\VS{18}A qui donc ferez-vous ressembler Dieu ? Et à quelle ressemblance l'égalerez-vous ?
\VS{19}L'ouvrier fond l'image, et l'orfèvre la couvre d'or, et y soude des chaînettes d'argent.
\VS{20}Celui qui est si pauvre qu'il n'a pas de quoi faire une offrande, choisit un bois qui ne pourrisse point ; il se cherche un habile ouvrier pour faire une image taillée qui ne bouge pas\FTNT{Es. 44:9-20.}.
\VS{21}Ne le savez-vous pas ? Ne l'avez-vous pas entendu ? Cela ne vous a-t-il pas été déclaré dès le commencement ? Ne l'avez-vous pas entendu dès les fondements de la terre ?
\VS{22}C'est lui qui est assis au-dessus du globe de la terre, et à qui ses habitants sont comme des sauterelles ; c'est lui qui étend les cieux comme un voile, il les déploie même comme une tente pour y demeurer.
\VS{23}C'est lui qui réduit les princes à rien, et qui fait des chefs de la terre une chose de néant.
\VS{24}Ils ne sont pas même plantés, pas même semés, même leur tronc n'a point de racine en terre ; il souffle sur eux, et ils sèchent, et le tourbillon les emporte comme de la paille.
\VS{25}A qui donc me ferez-vous ressembler, et à qui serais-je égalé ? Dit le Saint.
\VS{26}Elevez vos yeux en haut et regardez ! Qui a créé ces choses ? C'est lui qui fait sortir leur armée par ordre, et qui les appelle toutes par leur nom ; il n'y en a pas une qui fait défaut, à cause de la grandeur de sa force, et parce qu'il excelle en puissance.
\VS{27}Pourquoi donc dis-tu, ô Jacob, pourquoi dis-tu, ô Israël : Ma voie est cachée à Yahweh, et mon jugement passe inaperçu devant mon Dieu ?
\VS{28}Ne sais-tu pas ? N'as-tu pas entendu que le Dieu d'éternité, Yahweh, a créé les extrémités de la terre ; il ne se fatigue point, il ne se lasse point, et il n'y a pas moyen de sonder son intelligence.
\VS{29}C'est lui qui donne de la force à celui qui est las, et il multiplie la force de celui qui n'a aucune vigueur.
\VS{30}Les jeunes gens se lassent et se fatiguent, même les jeunes hommes tombent sans force.
\VS{31}Mais ceux qui s'attendent à Yahweh renouvellent leur force. Ils s'élèvent avec des ailes, comme des aigles ; ils courent, et ne se fatiguent point ; ils marchent, et ne se lassent point.
\Chap{41}
\TextTitle{Dénonciation des idoles}
\VerseOne{}Iles, faites moi silence ! Que les peuples renouvellent leurs forces ; qu'ils s'approchent et qu'alors ils parlent ; allons ensemble en jugement.
\VS{2}Qui a fait levé l'homme droit de l'orient ? Qui l'a appelé à sa suite ? Qui a soumis à son commandement les nations ? Qui lui a donné la domination sur les rois ? Qui les a livrés à son épée comme de la poussière, et à son arc comme de la paille poussée par le vent ?
\VS{3}Il les a poursuivis, il est passé en paix par le chemin que son pied n'avait jamais foulé.
\VS{4}Qui est celui qui a opéré et fait ces choses ? C'est celui qui a appelé les âges dès le commencement. Moi, Yahweh, JE SUIS le premier, et JE SUIS avec les derniers\FTNT{Ap. 1:8 ; Ap. 21:6 ; Ap. 22:13.}.
\VS{5}Les îles voient, et sont dans la crainte, les extrémités de la terre sont effrayées, ils s'approchent, ils viennent.
\VS{6}Chacun aide son prochain, et chacun dit à son frère : Fortifie-toi.
\VS{7}L'ouvrier encourage le fondeur ; celui qui frappe doucement du marteau encourage celui qui frappe sur l'enclume, et il dit : Cela est bon pour souder, puis il fixe l'idole avec des clous, afin qu'elle ne bouge pas.
\VS{8}Mais toi, Israël, tu es mon serviteur, et toi, Jacob, tu es celui que j'ai élu, la race d'Abraham qui m'a aimé !
\VS{9}Car je t'ai pris aux extrémités de la terre, je t'ai appelé en te préférant aux plus excellents qui sont en elle, et je t'ai dit : C'est toi qui est mon serviteur, je t'ai élu, et je ne te rejette point\FTNT{De. 7:6 ; Ps. 77:8.}.
\VS{10}Ne crains rien, car je suis avec toi ; ne sois pas étonné, car je suis ton Dieu ; je te fortifie, et je t'aide, même je te soutiens par la droite de ma justice.
\VS{11}Voici, tous ceux qui sont indignés contre toi seront honteux et confus ; ils seront réduits à néant, et les hommes qui ont querelle avec toi périront.
\VS{12}Tu les chercheras, et tu ne les trouveras plus, ceux qui te suscitaient querelle ; ils seront réduits à néant, et ceux qui te font la guerre seront comme ce qui n'est plus.
\VS{13}Car je suis Yahweh, ton Dieu, qui soutient ta main droite, et te dis : Ne crains rien, c'est moi qui te secours.
\VS{14}Ne crains point, vermisseau de Jacob, hommes mortels d'Israël ; je viens à ton secours dit Yahweh, et ton défenseur, le Saint d'Israël.
\VS{15}Voici, je fais de toi un traîneau aigu, tout neuf, ayant des dents ; tu fouleras les montagnes et les menuiseras, et tu rendras les collines semblables à de la balle.
\VS{16}Tu les vanneras, et le vent les emportera, et le tourbillon les dispersera. Mais toi, tu te réjouiras en Yahweh, tu te glorifiera au Saint d'Israël.
\VS{17}Quant aux affligés et aux misérables qui cherchent des eaux, et n'en ont point ; dont la langue est tellement altérée qu'elle n'en peut plus ; moi, Yahweh, je les exaucerai ; moi, le Dieu d'Israël, je ne les abandonnerai pas\FTNT{Ge. 28:15 ; Jos. 1:5 ; Hé. 13:5.}.
\VS{18}Je ferai jaillir des fleuves sur les hauteurs et des fontaines au milieu des vallées ; et je ferai du désert des étang d'eaux et de la terre sèche des sources d'eaux.
\VS{19}Je ferai croître au désert le cèdre, l'acacia, le myrte et l'olivier ; je mettrai dans les lieux stériles le cyprès, l'orme et le buis ensemble,
\VS{20}afin qu'on voit, qu'on sache, qu'on pense, et qu'on comprenne que la main de Yahweh a fait cela, et que le Saint d'Israël a créé cela.
\VS{21}Plaidez votre cause, dit Yahweh ; et mettez en avant les fondements de votre cause, dit le Roi de Jacob.
\VS{22}Qu'ils les amènent et qu'ils nous déclarent ce qui doit arriver. Déclarez-nous que veulent dire les choses qui ont été auparavant et nous y prendrons garde, et nous saurons leur issue, ou faites-nous entendre ce qui est prêt à arriver. 
\VS{23}Déclarez les choses qui doivent arriver dorénavant, et nous saurons que vous êtes des dieux ; faites aussi du bien ou du mal, et nous en serons tout étonnés puis nous regarderons ensemble.
\VS{24}Voici, vous n'êtes rien, et votre œuvre est le néant ; celui qui vous choisit n'est qu'abomination.
\VS{25}Je l'ai suscité du nord, et il est venu ; il invoque mon Nom de devant le soleil levant ; et marche sur les princes comme sur le mortier, et les foule comme le potier foule la boue.
\VS{26}Qui est celui qui a manifesté ces choses dès le commencement, afin que nous le connaissions ? Et longtemps d'avance, que nous puissions dire : Il est juste. Mais il n'y a personne qui les annonce, même il n'y a personne qui les donne à entendre, même il n'y a personne qui entende vos paroles.
\VS{27}Le premier sera pour Sion, disant : Voici, les voici ! Et je donnerai quelqu'un à Jérusalem qui annoncera de bonnes nouvelles\FTNT{Es. 52:7 ; Ap. 14:6.}.
\VS{28}Je regarde, et il n'y a point d'homme même entre ceux-là, et il n'y a aucun homme de conseil ; je les interroge aussi afin qu'il réponde quelque chose. 
\VS{29}Voici, quant à eux tous, leurs œuvres ne sont que vanité, leurs idoles de fonte sont du vent et de la confusion.
\Chap{42}
\TextTitle{Le messie, serviteur de Yahweh}
\VerseOne{}Voici mon serviteur, que je soutiens, c'est mon élu, en qui mon âme prend son bon plaisir ; j'ai mis mon Esprit sur lui, il manifestera le jugement aux nations\FTNT{Mt. 3:17 ; Mt. 17:5 ; Mc. 9:7.}.
\VS{2}Il ne criera point, et il ne haussera, ni ne fera entendre sa voix dans les rues.
\VS{3}Il ne brisera point le roseau cassé, et il n'éteindra point le lumignon qui fume\FTNT{Mt. 12:18-20.} ; il mettra en avant le jugement en vérité.
\VS{4}Il ne se retirera point et ne s'affaiblira point, jusqu'à ce qu'il ait établi la justice sur la terre, et que les îles s'attendent à sa loi.
\VS{5}Ainsi parle Dieu, Yahweh, qui a créé les cieux, et qui les a étendus, qui a aplani la terre avec ce qu'elle produit, qui donne la respiration au peuple qui est sur elle, et l'esprit à ceux qui y marchent.
\VS{6}Moi Yahweh, je t'ai appelé en justice, et je prendrai ta main et te garderai, et je te ferai être l'alliance du peuple et la lumière des nations\FTNT{Voir commentaire en Ge. 1:3-5.},
\VS{7}afin d'ouvrir les yeux des aveugles, et de faire sortir les prisonniers hors du lieu où on les tient enfermés, et ceux qui habitent dans les ténèbres hors de la prison.
\TextTitle{Israël n'a pas été attentif à Yahweh}
\VS{8}Je suis Yahweh, c'est là mon Nom ; et je ne donnerai pas ma gloire à un autre, ni ma louange aux images taillées\FTNT{Es. 48:11.}.
\VS{9}Voici, les choses qui ont été prédites auparavant se sont accomplies. Et je vous en annonce de nouvelles ; et je vous les fait entendre avant qu'elles arrivent.
\VS{10}Chantez à Yahweh un cantique nouveau, et que sa louange éclate aux extrémités de la terre, vous qui descendez en la mer, et tout ce qui est en elle,  les îles et leurs habitants !
\VS{11}Que le désert et ses villes élèvent la voix ! Que les villages où habite Kédar et ceux qui habitent dans les rochers éclatent en chant de triomphe ! Qu'ils s'écrient du sommet des montagnes !
\VS{12}Qu'on donne gloire à Yahweh, et qu'on publie sa louange dans les îles !
\VS{13}Yahweh sort comme un homme vaillant, il réveille sa jalousie comme un homme de guerre, il jette, dis-je, des cris de joie, il jette de grands cris, et il prévaut sur ses ennemis.
\VS{14}Je me suis tu dès longtemps ; me tiendrais-je en repos ? Me retiendrais-je ? Je crierai comme celle qui enfante, je détruirai, et j'engloutirai tout à la fois.
\VS{15}Je réduirai les montagnes et les collines en désert, et j'en dessécherai toute la verdure, je réduirai les fleuves en îles, et je ferai tarir les étangs.
\VS{16}Je conduirai les aveugles sur un chemin qu'ils ne connaissent pas, je les ferai marcher par des sentiers qu'ils ne connaissent pas ; je réduirai devant eux les ténèbres en lumière, et les choses tortues en choses droites ; voilà ce que je ferai, et je ne les abandonnerai point.
\VS{17}Ils se retireront en arrière, et ils seront tout honteux, ceux qui se confient aux images taillées, et qui disent aux images de fonte : Vous êtes nos dieux !
\VS{18}Sourds, écoutez ! Et vous aveugles, regardez et voyez !
\VS{19}Qui, dis-je, est aveugle, sinon mon serviteur ? Et qui est sourd, comme mon messager que j'envoie ? Qui est aveugle, comme celui que j'ai comblé de grâces ? Qui est aveugle, comme le serviteur de Yahweh ?
\VS{20}Vous voyez beaucoup de choses, mais vous ne prenez garde à rien ; vous avez les oreilles ouvertes, mais vous n'entendez rien.
\VS{21}Yahweh a plaisir en lui à cause de sa justice ; il a magnifié la loi et l'a  rendu honorable. 
\VS{22}Mais c'est ici un peuple pillé et dépouillé ! Ils sont enlacés dans les cavernes, et sont cachés dans des prisons ; ils sont un butin, et il n'y a personne qui les délivre ; une proie, et il n'y a personne qui dise : Restituez !
\VS{23}Qui est celui d'entre vous qui prêtera l'oreille à ces choses ? Qui s'y rendra attentif et l'écoutera à l'avenir ?
\VS{24}Qui est-ce qui a livré Jacob au pillage, et Israël aux pillards\FTNT{Jg. 2:13-16.} ? N'est-ce pas Yahweh, contre lequel nous avons péché ? Car on n'a point voulu marcher dans ses voies et on n'a point obéi à sa loi.
\VS{25}C'est pourquoi il a répandu sur lui la fureur de sa colère, et une forte guerre ; et il l'a embrasé tout alentour, mais Israël ne l'a point connu ; et il l'a brûlé, mais il n'y a point pris garde.
\Chap{43}
\TextTitle{Yahweh veut racheter Israël}
\VerseOne{}Mais maintenant ainsi parle Yahweh, qui t'a créé, ô Jacob ! Celui qui t'a formé, ô Israël ! Ne crains point, car je te rachète, je t'appelle par ton nom, tu es à moi !
\VS{2}Si tu passes par les eaux, je serai avec toi ; et si tu passes par les fleuves, ils ne te noieront pas ; si tu marches dans le feu, tu ne seras pas brûlé, et la flamme ne t'embrasera pas.
\VS{3}Car je suis Yahweh, ton Dieu, le Saint d'Israël, ton Sauveur. Je donne l'Egypte pour ta rançon, l'Ethiopie et Saba à ta place.
\VS{4}Parce que tu es précieux à mes yeux, tu es rendu honorable et je t'aime, je donne des hommes à ta place, et des peuples pour ta vie.
\VS{5}Ne crains point, car je suis avec toi ; je ferai venir ta postérité de l'orient, et je t'assemblerai de l'occident.
\VS{6}Je dirai au nord : Donne ! Et au midi : Ne retiens point ! Fais venir mes fils de loin, et mes filles du bout de la terre,
\VS{7}savoir tous ceux qui s'appellent de mon Nom\FTNT{Dans les Ecritures, le Nom de Dieu le plus cité est YHWH. Jésus, dont le nom signifie « YHWH est salut » correspond au nom et à l'identité que Dieu a révélé à tous ceux qui l'ont rencontré quand il était sur cette terre. Dans sa dernière prière à Gethsémané, Jésus dit : « J'ai fait connaître ton Nom » (Jn. 17:6), et « Je leur ai fait connaître ton Nom » (Jn. 17:26). Ce nom n'est autre que le sien puisque Jésus (YHWH est salut) était et est le Nom de Dieu. Moïse n'avait pas reçu la révélation de ce Nom (Ex. 3:13-14) car cette révélation était réservée à l'Eglise. En tant qu'épouse de Christ, l'Eglise porte le Nom du Seigneur et bénéficie de l'autorité qu'il confère. Ainsi, Jésus est le seul Nom par lequel nous pouvons être sauvés (Ac. 4:12). C'est aussi en son Nom que nous devons être baptisés (Ac. 8:16 ; Ac. 19:5), que nous recevons l'exaucement de nos prières (Jn. 14:13-14 ; Jn. 16:24), que nous sommes délivrés de l'ennemi et que nous obtenons la victoire sur le camp de l'ennemi (Mc. 16:17 ; Ph. 2:9-11).} ; car je les ai créés pour ma gloire ; je les ai formés et les ai faits.
\TextTitle{Yahweh appelle ses témoins}
\VS{8}Amène dehors le peuple aveugle qui a des yeux, et les sourds qui ont des oreilles.
\VS{9}Que toutes les nations soient ramassées ensemble, et que les peuples soient assemblés. Lequel d'entre eux a annoncé ces choses-là ? Et qui sont ceux qui nous ont fait entendre les choses qui ont été ci-devant ? Qu'ils produisent leurs témoins et qu'ils se justifient ; qu'on les entende et qu'on dise : C'est vrai !
\VS{10}Vous êtes mes témoins\FTNT{Ac. 1:8.}, dit Yahweh, et mon serviteur que j'ai élu, afin que vous connaissiez, que vous me croyiez et que vous compreniez que JE SUIS. Avant moi il n'a pas été formé de Dieu, et il n'y en aura point après moi.
\VS{11}Moi, JE SUIS Yahweh, et à part moi il n'y a point de Sauveur\FTNT{Yahweh dit qu'à part lui, il n'y a pas d'autres sauveurs. Or les écrits de la nouvelle alliance affirment que Jésus-Christ est le seul Sauveur (Lu. 1:67-80 ; Ac. 4:11-12).}.
\VS{12}C'est moi qui ai prédit ce qui devait arriver, qui vous ai sauvés, et qui vous ai fait entendre l'avenir, quand il n'y avait point de dieu étranger parmi vous ; et vous êtes mes témoins, dit Yahweh, que je suis Dieu.
\VS{13}Et même avant que le jour fût, JE SUIS, et il n'y a personne qui puisse délivrer de ma main ; je ferai l'œuvre, qui m'en empêchera ?
\TextTitle{Yahweh fera une chose nouvelle car Jacob ne l'a pas honoré}
\VS{14}Ainsi parle Yahweh, votre Rédempteur\FTNT{Es. 60:16 ; 1 Co. 1:30 ; Ro. 3:24 ; Ep. 1:7.}, le Saint d'Israël : J'envoie pour l'amour de vous contre Babylone, et je les fais descendre tous fugitifs, et le cri des Chaldéens sera dans les navires.
\VS{15}Je suis Yahweh, votre Saint, le Créateur d'Israël, votre Roi.
\VS{16}Ainsi parle Yahweh, qui fraya un chemin dans la mer, et un sentier parmi les eaux impétueuses ;
\VS{17}qui amena des chars et des chevaux, et de grandes forces ; ils ont été étendus ensemble, et ils ne se relèveront point, ils ont été étouffés, ils ont été éteints comme un lumignon :
\VS{18}Ne pensez plus aux choses passées, et ne considérez point les choses anciennes.
\VS{19}Voici, je m'en vais faire une chose nouvelle\FTNT{2 Co. 5:17.}, qui paraîtra bientôt, ne la connaîtrez-vous pas ? Je mettrai un chemin dans le désert, et des fleuves dans le lieu de désolation.
\VS{20}Les bêtes des champs me glorifieront, les serpents et les autruches, parce que j'aurai mis des eaux dans le désert, et des fleuves dans la solitude, pour abreuver mon peuple que j'ai élu.
\VS{21}Ce peuple que je me suis formé racontera mes louanges.
\VS{22}Mais toi, Jacob, tu ne m'as pas invoqué, car tu t'es lassé de moi, ô Israël !
\VS{23}Tu ne m'as pas offert le menu bétail de tes holocaustes, et tu ne m'as pas glorifié dans tes sacrifices ; je ne t'ai point asservi pour me faire des offrandes, et je ne t'ai point fatigué pour de l'encens.
\VS{24}Tu ne m'as pas acheté à prix d'argent du roseau aromatique, et tu ne m'as pas rassasié de la graisse de tes sacrifices ; mais tu m'as asservi par tes péchés, et tu m'as peiné par tes iniquités.
\VS{25}Moi, JE SUIS celui qui efface tes transgressions pour l'amour de moi, et je ne me souviendrai plus de tes péchés.
\VS{26}Réveille ma mémoire, et plaidons ensemble ; toi, déclare pour que tu puisses être justifié.
\VS{27}Ton premier père a péché, et tes docteurs se sont rebellés contre moi.
\VS{28}C'est pourquoi j'ai profané les chefs du lieu saint, et j'ai livré Jacob à la destruction, et Israël à l'opprobre.
\Chap{44}
\TextTitle{Promesse de l'Esprit, folie de l'idolâtrie}
\VerseOne{}Ecoute maintenant, ô Jacob, mon serviteur, et toi Israël que j'ai choisi !
\VS{2}Ainsi parle Yahweh, qui t'a fait et formé dès le ventre, celui qui te soutient : Ne crains point, ô Jacob, mon serviteur ! Et toi Jeshurun que j'ai élu.
\VS{3}Car je répandrai des eaux sur celui qui est altéré, et des rivières sur la terre sèche ; je répandrai mon esprit sur ta postérité, et ma bénédiction sur ta descendance.
\VS{4}Et ils germeront comme au milieu de l'herbe, comme les saules auprès des courants d'eau.
\VS{5}L'un dira : Je suis à Yahweh ; et l'autre se réclamera du nom de Jacob ; et un autre écrira de sa main : Je suis à Yahweh, et se nommera du nom d'Israël.
\VS{6}Ainsi parle Yahweh, le Roi d'Israël et son Rédempteur, Yahweh des armées : Je suis le premier, et je suis le dernier ; et à part moi il n'y a point de Dieu.
\VS{7}Et qui, comme moi, a appelé, déclaré et ordonné cela, depuis que j'ai établi le peuple ancien ? Qu'ils déclarent les choses à venir, les choses qui arriveront ci-après !
\VS{8}Ne soyez point effrayés et ne soyez point troublés ; ne te l'ai-je pas fait entendre et déclarer dès ce temps-là ? Vous êtes mes témoins ; y a-t-il un autre Dieu que moi ? Certes il n'y a pas d'autre Rocher\FTNT{Yahweh dit qu'il ne connaît pas d'autre rocher. Jésus-Christ est ce rocher qui suivait les Hébreux dans le désert (Mt. 16:18 ; 1 Co. 10:1-4. Voir aussi commentaire en Es. 8:14). }, je n'en connais pas.
\VS{9}Les ouvriers d'images taillées ne sont tous que vanité, et leurs choses les plus désirables ne sont d'aucun profit ; elles le témoignent elles-mêmes, elles ne voient point, et ne connaissent point, afin qu'ils soient honteux.
\VS{10}Mais qui est-ce qui fabrique un dieu, ou fond une image taillée, pour n'en avoir aucun profit ?
\VS{11}Voici, tous ses compagnons seront honteux, car ces ouvriers-là sont d'entre les hommes. Qu'ils s'assemblent tous, qu'ils se tiennent là ! Ils seront effrayés et rendus honteux tous ensemble.
\VS{12}Le forgeron fait une hache, et il travaille avec le charbon, et il le forme à coups de marteau ; il le fait à force de bras, même il a faim et il est sans force, il ne boit point d'eau, et il est tout fatigué.
\VS{13}Le charpentier étend sa règle, il trace sa forme au crayon avec de la craie ; il le fait avec des équerres, et le forme au compas, et le fait à la ressemblance d'un homme, selon la beauté d'un homme, afin qu'il demeure dans la maison.
\VS{14}Il se coupe des cèdres, et prend un cyprès, ou un chêne, qu'il a laissé croître parmi les arbres de la forêt ; il plante des pins, et la pluie les fait croître.
\VS{15}Ces arbres servent à l'homme pour brûler, car il en prend et il s'en chauffe. Il en fait du feu, dis-je, et en cuit du pain ; et il en fait aussi un dieu et se prosterne devant lui ; il en fait une image taillée et l'adore.
\VS{16}Il en brûle au feu une partie, et d'une autre partie il mange sa chair, laquelle il rôtit, et s'en rassasie ; il s'en chauffe aussi, et il dit : Ah ! Ah ! Je me chauffe, je vois la flamme !
\VS{17}Puis avec le reste il fait un dieu pour être son image taillée ; il se prosterne devant elle, il l'adore, il lui fait sa requête et dit : Délivre-moi, car tu es mon dieu !
\VS{18}Ils ne savent et n'entendent rien, car on leur a plâtré les yeux afin qu'ils ne voient point, et les cœurs pour qu'ils ne comprennent point.
\VS{19}Nul ne rentre en lui-même\FTNT{So. 2:1 ; 2 Co. 13:5.}, et il n'a ni la connaissance ni l'intelligence pour dire : J'en ai brûlé une partie au feu, et même j'ai cuit du pain sur les charbons, j'ai rôti de la viande et je l'ai mangée ; et avec le reste ferais-je une abomination ? Adorerais-je une branche de bois ?
\VS{20}Il se repaît de cendres, et son cœur abusé l'égare, et il ne délivrera point son âme, et ne dira point : N'est-ce pas du mensonge que j'ai dans ma main droite ?
\TextTitle{Yahweh rachète son peuple}
\VS{21}Souviens-toi de ces choses, ô Jacob ! Ô Israël, car tu es mon serviteur ; je t'ai formé, tu es mon serviteur, ô Israël ! Je ne t'oublierai pas.
\VS{22}J'efface tes transgressions comme une nuée épaisse, et tes péchés comme une nuée ; reviens à moi, car je t'ai racheté.
\VS{23}Ô cieux ! Réjouissez-vous avec chants de triomphe, car Yahweh a opéré ; profondeurs de la terre, jetez des cris de réjouissance ! Montagnes, éclatez de joie avec chant de triomphe ! Et vous aussi forêts et tous les arbres qui êtes en elles ! Parce que Yahweh a racheté Jacob, et s'est manifesté glorieusement en Israël.
\VS{24}Ainsi parle Yahweh, ton Rédempteur, celui qui t'a formé dès le ventre : Je suis Yahweh qui ai fait toutes choses, qui seul ai étendu les cieux, et qui ai par moi-même étendu la terre ;
\VS{25}qui dissipe les signes des menteurs, qui rends insensés les devins ; qui renverse l'esprit des sages, et qui fait que leur science devient une folie.
\VS{26}C'est lui qui confirme la parole de son serviteur, et accomplit le conseil  de ses messagers ; qui dit à Jérusalem : Tu seras encore habitée ! Et aux villes de Juda : Vous serez rebâties ! Et je redresserai ses lieux déserts.
\VS{27}Qui dit à l'abîme : Sois asséchée, et je tarirai tes fleuves.
\TextTitle{Prophétie sur le rétablissement d'Israël par Cyrus}
\VS{28}Qui dit de Cyrus\FTNT{Esaïe prophétisa la destruction de Babylone deux siècles avant la réalisation de cet événement  le 5 octobre 539 av. J.-C. Fait remarquable : il précisa même le nom du commandant Cyrus qui dompta le lion babylonien. L’historien Hérodote donnera  par ailleurs raison au prophète sur le déroulement de la prise de Babylone.} : Il est mon berger, et il accomplira tout mon bon plaisir ; disant même à Jérusalem : Tu seras rebâtie ! Et au temple : Tu seras fondé.
\Chap{45}
\TextTitle{Cyrus suscité par Yahweh}
\VerseOne{}Ainsi parle Yahweh à son oint, à Cyrus\FTNT{Cyrus le Grand (580 av. J.-C. - 530 av. J.-C.). Voir Esd. 1.},
\VS{2}que je tiens par la main droite, pour terrasser les nations devant lui, et pour délier les ceintures des rois, pour ouvrir devant lui les portes, afin qu'elles ne soient point fermées.
\VS{3}J'irai devant toi, et j'aplanirai les lieux tortueux ; je romprai les portes d'airain, et je mettrai en pièces les barres de fer. Et je te donnerai des trésors cachés, et des richesses le plus secrètement gardées, afin que tu saches que je suis Yahweh, le Dieu d'Israël, qui t'appelle par ton nom.
\VS{4}Pour l'amour de Jacob, mon serviteur, et d'Israël mon élu ; je t'ai, dis-je, appelé par ton nom, et je t'ai surnommé avant que tu me connaisses.
\TextTitle{Yahweh, le seul Dieu}
\VS{5}Je suis Yahweh, et il n'y en a point d'autre ; à part moi, il n'y a point de Dieu. Je t'ai ceint avant que tu me connaisses,
\VS{6}afin que l'on sache, du soleil levant au soleil couchant, qu'à part moi, il n'y a point de Dieu. Je suis Yahweh, et il n'y en a point d'autre.
\VS{7}Je forme la lumière, et je crée les ténèbres ; je fais la paix et je crée l'adversité ; moi, Yahweh, je fais toutes ces choses.
\VS{8}Ô cieux ! Répandez la rosée d'en haut, et que les nuées laissent couler la justice ! Que la terre s'ouvre, qu'elle produise le salut, qu'elle fasse également germer la justice ! Moi, Yahweh, je crée ces choses.
\VS{9}Malheur à celui qui conteste avec celui qui l'a façonné ! Vase parmi des vases de terre ! L'argile dit-elle à celui qui la façonne : Que fais-tu ? Et l'œuvre dit-elle à l'ouvrier : Tu n'as point de mains\FTNT{Jé. 18:6 ; Ro. 9:21.} ?
\VS{10}Malheur à celui qui dit à son père : Qu'engendres-tu ? Et à sa mère : Qu'enfantes-tu ?
\VS{11}Ainsi parle Yahweh, le Saint d'Israël, qui est son Créateur : Interrogez-moi sur les choses à venir, mes fils ; me commanderez-vous sur l'œuvre de mes mains ?
\VS{12}C'est moi qui ai fait la terre et qui ai créé l'homme sur elle ; c'est moi qui ai étendu les cieux de mes mains, et qui ai donné la loi à toute leur armée.
\VS{13}C'est moi qui ai suscité Cyrus dans ma justice, et j'aplanirai toutes ses voies ; il rebâtira ma ville, et libérera mes captifs\FTNT{Cyrus le grand libéra les Juifs après 70 ans de captivité (Esd. 1).}, sans rançon ni présents, dit Yahweh des armées.
\TextTitle{Les autres peuples reconnaîtront la main de Yahweh sur Israël}
\VS{14}Ainsi parle Yahweh : Le travail de l'Egypte, et le trafic de l'Ethiopie, et ceux des Sabéens, gens de grande stature, passeront chez toi Jérusalem, et seront à toi ; ils marcheront à ta suite, ils passeront enchaînés, ils se prosterneront devant toi, ils te diront en suppliant : Certainement, Dieu est au milieu de toi, et il n'y a point d'autre Dieu que lui.
\VS{15}En vérité, tu es le Dieu qui te caches, le Dieu d'Israël, le Sauveur.
\VS{16}Ils sont tous honteux et confus, ils s'en vont tous avec ignominie, les fabricants d'idoles.
\VS{17}Mais Israël a été sauvé par Yahweh, d'un salut éternel ; vous ne serez ni honteux ni confus jusque dans l'éternité.
\VS{18}Car ainsi parle Yahweh qui a créé les cieux, Dieu lui-même qui a formé la terre, qui l'a faite et qui l'a affermie ; qui l'a créée pour qu'elle ne soit pas informe\FTNT{Informe : de l'hébreu « tohuw » qui signifie « informe, confusion, solitude, désert, néant ». On retrouve ce mot dès Ge. 1:2.}, qui l'a formée pour qu'elle soit habitée ; je suis Yahweh, et il n'y en a point d'autre.
\VS{19}Je n'ai point parlé en secret ni dans quelque lieu ténébreux de la terre ; je n'ai point dit à la postérité de Jacob : Cherchez-moi vainement ! Je suis Yahweh, qui prononce ce qui est juste, qui déclare ce qui est droit.
\VS{20}Assemblez-vous et venez, approchez-vous ensemble, vous les réchappés des nations ! Ceux qui portent le bois de leur image taillée ne savent rien, et invoquent un dieu qui ne sauve pas.
\VS{21}Déclarez-le, et faites-les approcher ! Qu'ils prennent conseil ensemble ! Qui a fait entendre ces choses dès l'origine, et les a déclarées dès longtemps ? N'est-ce pas moi, Yahweh ? Or il n'y a point d'autre Dieu à part moi ; un Dieu juste et un Sauveur, il n'y en a pas d'autre à part moi.
\VS{22}Vous tous qui êtes aux extrémités de la terre, regardez vers moi, et soyez sauvés ; car je suis Dieu, et il n'y en a point d'autre.
\VS{23}Je le jure par moi-même, la parole sort en justice de ma bouche, et elle ne sera point révoquée : Tout genou fléchira devant moi, et toute langue jurera par moi\FTNT{Ph. 2:9-11.}.
\VS{24}Certainement, on dira de moi : En Yahweh seul sont la justice et la force ; à lui viendront, pour être confondus, tous ceux qui étaient irrités contre lui.
\VS{25}Toute la postérité d'Israël sera justifiée, et elle se glorifiera en Yahweh.
\Chap{46}
\TextTitle{La puissance de Yahweh, l'incapacité des idoles}
\VerseOne{}Bel s'incline sur ses genoux, Nebo est renversé ; leurs faux dieux sont mis sur leurs bêtes et leur bétail ; les idoles que vous portiez, ont été chargées, elles sont un fardeau pour la bête fatiguée !
\VS{2}Elles se sont courbées, elles se sont inclinées ensemble sur leurs genoux, et ne peuvent échapper au fardeau, et elles-mêmes s'en vont en captivité.
\VS{3}Ecoutez-moi, maison de Jacob, et vous tous, tout le reste de la maison d'Israël, dont je me suis chargé dès le ventre, et que j'ai porté dès le sein maternel.
\VS{4}Jusqu'à votre vieillesse, JE SUIS ; et je vous chargerai sur moi jusqu'à votre blanche vieillesse, je l'ai fait, et je vous porterai encore, je vous chargerai sur moi et vous sauverai.
\VS{5}A qui me ferez-vous ressembler, et à qui m'égalerez-vous ? A qui me comparerez-vous pour que nous soyons semblable ? 
\VS{6}Ils tirent l'or de la bourse, et pèsent l'argent à la balance, et ils engagent un orfèvre pour en faire un dieu ; ils l'adorent, et se prosternent devant lui.
\VS{7}On le porte sur les épaules, on s'en charge ; on le pose en sa place où il se tient debout et ne bouge point de son lieu, puis on crie à lui, mais il ne répond pas, et il ne délivre pas de la détresse ceux qui crient vers lui.
\VS{8}Souvenez-vous de cela, et montrez-vous des hommes ; rappelez-le à votre pensée, ô vous transgresseurs !
\VS{9}Souvenez-vous des premières choses d'autrefois ; je suis Dieu, et il n'y en a point d'autre, je suis Dieu et il n'y en a point comme moi ;
\VS{10}qui déclare dès le commencement ce qui doit arriver à la fin, et longtemps auparavant, les choses qui n'ont pas encore été faites ; qui dis : Mon conseil tiendra, et j'exécuterai tout mon bon plaisir ;
\VS{11}qui appelle de l'orient l'oiseau de proie, et d'une terre éloignée un homme pour exécuter mon conseil. Oui, j'ai parlé, aussi je ferai venir la chose ; je l'ai formé, aussi je l'accomplirai. 
\VS{12}Ecoutez-moi, vous qui avez le cœur endurci et qui êtes éloignés de la justice.
\VS{13}Je fais approcher ma justice, elle ne s'éloignera point loin ; et mon salut, il ne tardera pas. Je mettrai le salut en Sion pour Israël, qui est ma gloire.
\Chap{47}
\TextTitle{Jugement sur Babylone}
\VerseOne{}Descends, et assieds-toi dans la poussière, vierge, fille de Babylone ! Assieds-toi à terre, il n'y a plus de trône pour la fille des Chaldéens ! Car tu ne te feras plus appeler la délicate et la voluptueuse.
\VS{2}Mets la main aux meules, et fais moudre la farine ; délie tes tresses, déchausse-toi, découvre tes jambes et traverse les fleuves !
\VS{3}Ta honte sera découverte et ton opprobre sera vue ; je prendrai vengeance, je n'irai point contre toi en homme.
\VS{4}Quant à notre Rédempteur, son Nom est Yahweh des armées, le Saint d'Israël.
\VS{5}Assieds-toi sans dire mot, et entre dans les ténèbres, fille des Chaldéens, car tu ne te feras plus appeler la dame des royaumes.
\VS{6}J'ai été embrasé de colère contre mon peuple, j'ai profané mon héritage, c'est pourquoi je les ai livrés entre tes mains, mais tu n'as point usé de miséricorde envers eux, tu as durement appesanti ton joug sur le vieillard.
\VS{7}Et tu as dit : Je serai dame à toujours ! De sorte que tu n'as point mis ces choses-là dans ton cœur, tu ne t'es point souvenue ce qu'en serait la fin.
\VS{8}Maintenant donc écoute ceci, toi voluptueuse qui habite avec assurance, et qui dis en ton cœur : C'est moi, et il n'y en a point d'autre que moi ; je ne deviendrai point veuve, et je ne saurai point ce que c'est que d'être privée d'enfants.
\VS{9}Mais ces deux choses t'arriveront en un moment, en un même jour, la privation d'enfants et le veuvage ; elles viendront sur toi dans leur perfection, pour le  grand nombre de tes sortilèges, et pour la grande abondance de tes enchantements\FTNT{Ap. 18:7-8.}.
\VS{10}Et tu t'es confiée dans ta méchanceté, et disais : Personne ne me voit ! Ta sagesse et ta science t'ont pervertie, et tu disais en ton cœur : C'est moi, et il n'y en a point d'autre que moi.
\VS{11} C'est pourquoi le mal viendra sur toi, et tu ne sauras pas quand il sera près d'arriver, et le malheur qui tombera sur toi sera tel, que tu ne pourras pas le détourner ; et la ruine éclatante que tu n'as pas soupçonnée viendra sur toi subitement.
\VS{12}Tiens-toi maintenant avec tes enchantements, et avec le grand nombre de tes sortilèges, après lesquels tu as travaillé dès ta jeunesse ; peut-être pourras-tu en tirer quelque profit ; peut-être en seras-tu renforcé.
\VS{13}Tu t'es lassée à force de demander des conseils. Que les spectateurs des cieux qui contemplent les étoiles, et qui font leurs prédictions selon les lunes, comparaissent maintenant, et qu'ils te délivrent des choses qui viendront sur toi.
\VS{14}Voici, ils sont devenus comme de la paille, le feu les consume, ils ne délivreront pas leur vie du pouvoir de la flamme ; il n'y a point de charbon pour se chauffer et il n'y a point de lueur de feu pour s'asseoir vis-à-vis. 
\VS{15}Tels te sont devenus ceux avec lesquels tu as travaillé et avec lesquels tu as trafiqué dès ta jeunesse, chacun s'en est fui en son quartier comme un vagabon ; il n'y a personne pour te sauver.
\Chap{48}
\TextTitle{Yahweh rappelle ses promesses}
\VerseOne{}Ecoutez ceci, maison de Jacob, qui êtes appelés du nom d'Israël, et qui êtes sortis des eaux de Juda ; qui jurez par le nom de Yahweh, et qui faites mention du Dieu d'Israël, mais non pas conformément à la vérité et à la justice\FTNT{Jé. 5:2.}.
\VS{2}Car ils prennent leur nom de la sainte cité, et ils s'appuient sur le Dieu d'Israël, dont le nom est Yahweh des armées\FTNT{Ex. 20:7.}.
\VS{3}J'ai déclaré les premières choses dès le commencement, elles sont sorties de ma bouche et je les ai publiées ; je les ai faites subitement et elles se sont accomplies.
\VS{4}Parce que j'ai connu que tu es obstiné, que ton cou est une barre de fer, et que ton front est d'airain,
\VS{5}je t'ai déclaré ces choses dès lors, et je les ai faites entendre avant qu'elles arrivent, de peur que tu ne dises : Mes dieux ont fait ces choses ; mon image taillée, et mon image de fonte les ont ordonnées.
\VS{6}Tu l'entends ! Vois tout ceci ! Et vous, ne l'annoncerez-vous pas ? Je te fais entendre dès maintenant des choses nouvelles, et qui étaient en réserve et que tu ne savais pas.
\VS{7}Elles sont créées maintenant, et non pas depuis le commencement ; et avant ce jour-ci tu n'en avais rien entendu, afin que tu ne dises pas : Voici, je les savais bien.
\VS{8}Oui, tu n'en avais pas entendu parler, oui, tu ne savais pas ; oui, depuis ce temps ton oreille n'a pas été ouverte ; car j'ai connu que tu agirais perfidement; aussi tu as été appelé transgresseur dès le ventre.
\VS{9}Pour l'amour de mon Nom, je diffère ma colère ; et pour l'amour de ma louange, je retiens mon courroux contre toi, afin de ne pas te retrancher.
\VS{10}Voici, je t'ai épuré, mais non pas comme on épure l'argent ; je t'ai éprouvé au creuset de l'affliction.
\VS{11}Pour l'amour de moi, pour l'amour de moi, je le ferai, car comment mon Nom serait-il profané ? Certes, je ne donnerai pas ma gloire à un autre
\VS{12}Ecoute-moi, Jacob ! Et toi Israël, mon appelé ; moi, JE SUIS le premier, JE SUIS aussi le dernier.
\VS{13}Ma main aussi a fondé la terre et ma droite a étendu les cieux ; quand je les appelle, ils comparaissent ensemble.
\VS{14}Vous tous, assemblez-vous et écoutez ! Lequel parmi eux a déclaré ces choses ? Yahweh l'aime et exécutera son bon plaisir contre Babylone, et son bras sera contre sur les Chaldéens.
\VS{15}Moi, JE SUIS celui qui ai parlé, je l'ai aussi appelé, je l'ai amené, et ses desseins réussiront.
\VS{16}Approchez-vous de moi et écoutez ceci ! Dès le commencement, je n'ai point parlé en secret, depuis l'origine de ces choses, JE SUIS. Or maintenant, le Seigneur, Yahweh, et son Esprit m'ont envoyé.
\VS{17}Ainsi parle Yahweh, ton Rédempteur, le Saint d'Israël : Je suis Yahweh, ton Dieu, qui t'enseigne pour ton profit, et qui te guide dans le chemin où tu dois marcher.
\VS{18}Ô ! Si tu étais attentif à mes commandements, ta paix serait comme un fleuve, et ta justice comme les flots de la mer\FTNT{Jos. 1:8 ; Ps. 1:2 ; Jn. 14:21 ; Ja. 1:22.},
\VS{19}ta postérité serait comme le sable, et ceux qui sortent de tes entrailles comme les grains de sable\FTNT{Ge. 15:5 ; Ge. 22:17 ; Ge. 32:12.} ; son nom ne serait point retranché ni effacé de devant ma face.
\VS{20}Sortez de Babylone, fuyez loin des Chaldéens ! Publiez ceci avec une voix de chant de triomphe, annoncez le, portez ceci jusqu'aux extrémités de la terre, dites : Yahweh a racheté son serviteur Jacob !
\VS{21}Et ils n'auront pas soif quand il les fera marcher dans les déserts ; il fera découler pour eux l'eau hors du rocher, même il leur fendra le rocher, et les eaux couleront.
\VS{22}Il n'y a point de paix pour les méchants, dit Yahweh.
\Chap{49}
\TextTitle{Le Messie, la lumière de tous les peuples}
\VerseOne{}Iles, écoutez-moi ! Soyez attentifs, vous peuples éloignés ! Yahweh m'a appelé dès le ventre, il a fait mention de mon nom dès les entrailles de ma mère\FTNT{Jé. 1:5 ; Ps. 139:16.}.
\VS{2}Et il a rendu ma bouche semblable à une épée aiguë ; il m'a caché dans l'ombre de sa main, et m'a rendu semblable à une flèche bien polie, il m'a serré dans son carquois.
\VS{3}Et il m'a dit : Tu es mon serviteur, ô Israël, en qui je serai glorifié.
\VS{4}Et moi j'ai dit : J'ai travaillé en vain, j'ai consumé ma force pour néant et sans fruit ; toutefois mon jugement est auprès de Yahweh, et ma récompense est auprès de mon Dieu.
\VS{5}Maintenant donc, Yahweh, qui m'a formé dès le ventre pour être à son service, m'a dit que je lui ramène Jacob, mais Israël ne se rassemble point ; toutefois je serai honoré aux yeux de Yahweh, et mon Dieu sera ma force.
\VS{6}Il me dit : C'est peu de chose que tu sois serviteur pour relever les tribus de Jacob et pour ramener les restes d'Israël ; c'est pourquoi je te donne pour lumière aux nations, afin que tu sois mon salut jusqu'aux extrémités de la terre.
\VS{7}Ainsi parle Yahweh, le Rédempteur, le Saint d'Israël, à celui qu'on méprise, à celui qui est abominable au peuple, au serviteur de ceux qui dominent ; les rois le verront, et se lèveront, et les princes aussi, et ils se prosterneront devant lui, pour l'amour de Yahweh, qui est fidèle, et du Saint d'Israël qui t'a élu.
\VS{8}Ainsi parle Yahweh : Je t'ai exaucé au temps de la bienveillance, et je t'ai aidé au jour du salut ; je te garderai, et je te donnerai pour être l'alliance du peuple, pour relever la terre, afin que tu possèdes les héritages désolés ;
\VS{9}disant à ceux qui sont emprisonnés : Sortez ! Et à ceux qui sont dans les ténèbres : Montrez-vous ! Ils paîtront sur les chemins, et leurs pâturages seront sur tous les lieux élevés.
\VS{10}Ils n'auront pas faim et ils n'auront pas soif ; la chaleur et le soleil ne les frapperont plus, car celui qui a pitié d'eux sera leur guide, et les conduira vers des sources d'eaux\FTNT{Ps. 121:6 ; Lu. 1:67-79.}.
\VS{11}Et je réduirai toutes mes montagnes en chemins, et mes sentiers seront relevés.
\VS{12}Voici, ceux-ci viennent de loin, et voici ceux-là viennent du nord et de l'occident, et les autres du pays de Sinim.
\VS{13}Ô cieux, réjouissez-vous avec des chants de triomphe! Et toi, ô terre, sois dans l'allégresse ! Et vous, ô montagnes, éclatez de joie avec des chants de triomphe ! Car Yahweh console son peuple, il a compassion de ceux qu'il a affligés.
\VS{14}Mais Sion disait : Yahweh me délaisse, le Seigneur m'oublie !
\VS{15}Une femme peut-elle oublier son enfant qu'elle allaite de sorte qu'elle n'ait pas pitié du fils de ses entrailles ? Mais quand les femmes les oublieraient, moi je ne t'oublierai point.
\VS{16}Voici, je t'ai gravé sur les paumes de mes mains ; tes murs sont continuellement devant moi.
\VS{17}Tes enfants viennent à grande hâtent, mais ceux qui te détruisaient et ceux qui te réduisaient en désert, sortiront du milieu de toi.
\VS{18}Elève tes yeux autour de toi, et regarde : Tous ceux-ci s'assemblent, ils viennent à toi. Je suis vivant, dit Yahweh, tu te revêtiras de tous comme d'une parure, et tu t'en orneras comme une épouse.
\VS{19}Car tes déserts, tes ruines, et ton pays détruit seront désormais trop étroits pour ses habitants, et ceux qui t'engloutissaient s'éloigneront.
\VS{20}Les enfants que tu auras après avoir perdu les autres diront encore, à tes oreilles : Le lieu est trop étroit pour moi, fais-moi de la place pour que je puisse y demeurer.
\VS{21}Et tu diras en ton cœur : Qui m'a engendré ceux-ci vu que j'avais perdu mes enfants et que j'étais stérile, emmenée en captivité et agitée ? Et qui m'a nourri ceux-ci ? Voici, j'étais restée toute seule, et ceux-ci où étaient-ils ?
\VS{22}Ainsi parle le Seigneur Yahweh : Voici, je lèverai ma main vers les nations et je dresserai ma bannière vers les peuples ; et ils ramèneront tes fils entre leurs bras, et ils porteront tes filles sur les épaules.
\VS{23}Et les rois seront tes nourriciers et leurs princesses, leurs femmes, tes nourrices ; ils se prosterneront devant toi le visage contre terre, et ils lécheront la poussière de tes pieds ; et tu sauras que je suis Yahweh, et que ceux qui se confient en moi ne seront point confus\FTNT{Ps. 22:5-6 ; Ps. 69:7 ; Ro. 9:33 ; 1 Pi. 2:6.}.
\VS{24}Le butin sera-t-il ôté à l'homme puissant ? Et les captifs du juste seront-ils délivrés ?
\VS{25}Car ainsi parle Yahweh : Même les captifs pris par l'homme puissant lui seront ôtés, et le butin de l'homme fort lui sera enlevé ; car je plaiderai moi-même avec ceux qui plaident contre toi, et je délivrerai tes enfants.
\VS{26}Et je ferai manger leur propre chair à ceux qui t'oppriment ; et ils s'enivreront de leur sang comme du moût, et toute chair connaîtra que je suis Yahweh, ton Sauveur, ton Rédempteur, le Puissant de Jacob.
\Chap{50}
\TextTitle{Avertissements de Yahweh par son serviteur}
\VerseOne{}Ainsi parle Yahweh : Où est la lettre de divorce par laquelle j'ai répudié votre mère\FTNT{De. 24:1 ; Jé. 3:8 ; Mt. 5:31.} ? Ou bien, auquel de mes créanciers vous ai-je vendus ? Voici, vous avez été vendus à cause de vos iniquités, et votre mère a été répudiée à cause de vos transgressions.
\VS{2}Je suis venu : Pourquoi ne s'est-il trouvé personne ? J'ai appelé : Pourquoi personne n'a-t-il répondu ? Ma main est-elle trop courte pour racheter\FTNT{No. 11:23 ; Es. 59:1.} ? Ou n'y a-t-il plus de force en moi pour délivrer ? Voici, par ma menace, je dessèche la mer, je réduis les fleuves en désert ; leurs poissons se corrompent faute d'eau, et ils meurent de soif.
\VS{3}Je revêts les cieux de noirceur, et je fais d'un sac leur couverture.
\VS{4}Le Seigneur, Yahweh, m'a donné la langue des savants, pour que je sache soutenir par la parole celui qui est accablé de maux\FTNT{Job. 6:14 ; 1 Th. 5:14. } ; chaque matin il me réveille soigneusement afin que je prête l'oreille aux discours des sages.
\VS{5}Le Seigneur Yahweh m'a ouvert l'oreille et je n'ai pas été rebelle, et je ne me suis pas retiré en arrière.
\VS{6}J'ai exposé mon dos à ceux qui me frappaient et mes joues à ceux qui me tiraient le poil ; je n'ai pas caché mon visage aux opprobres et aux crachats\FTNT{Mt. 5:39 ; Mt. 26:67 ; Lu. 6:29 ; Lu. 18:32.}.
\VS{7}Mais le Seigneur, Yahweh m'a aidé, c'est pourquoi je n'ai point été confus, et ainsi j'ai rendu mon visage semblable à un caillou\FTNT{Ez. 3:8-9.}, car je sais que je ne serais point rendu honteux.
\VS{8}Celui qui me justifie est proche ; qui plaidera contre moi ? Comparaissons ensemble ! Qui est mon adversaire ? Qu'il s'approche de moi.
\VS{9}Voici, le Seigneur, Yahweh m'aidera, qui est celui qui me condamnera ? Voici, tous seront usés comme un vêtement, la teigne les dévorera.
\VS{10}Qui est celui d'entre vous qui craint Yahweh, et qui obéit à la voix de son serviteur ! Que celui qui marche dans les ténèbres, et qui n'a pas de clarté, se confie dans le Nom de Yahweh, et qu'il s'appuie sur son Dieu.
\VS{11}Voici, vous tous qui allumez le feu, et qui vous ceignez d'étincelles, marchez à la lueur de votre feu et des étincelles que vous avez embrasées ; voici ce que vous aurez de ma main ; vous vous coucherez dans les tourments.
\Chap{51}
\TextTitle{Exhortation à ceux qui recherchent Yahweh}
\VerseOne{}Ecoutez-moi, vous qui poursuivez la justice et qui cherchez Yahweh ! Regardez au rocher d'où vous avez été taillés, et au creux de la citerne dont vous avez été tirés.
\VS{2}Regardez à Abraham, votre père, et à Sara qui vous a enfantés ; car lui seul je l'ai appelé, je l'ai béni et multiplié\FTNT{Ro. 4:1-16 ; Hé. 11:8-12.}.
\VS{3}Car Yahweh console Sion, il console de toutes ses désolations, il rendra son désert semblable à Eden, et sa terre aride à un jardin de Yahweh. En elle sera trouvée la joie et l'allégresse, la reconnaissance et la voie de mélodie.
\VS{4}Ecoutez-moi donc attentivement, mon peuple, et prêtez-moi l'oreille, vous ma nation ; car la loi sortira de moi, et j'établirai mon jugement pour être la lumière des peuples.
\VS{5}Ma justice est proche, mon salut va paraître, et mes bras jugeront les peuples ; les îles espéreront en moi, elles se confieront en mon bras.
\VS{6}Levez les yeux vers les cieux et regardez en bas sur la terre ! Car les cieux s'évanouiront comme la fumée, et la terre tombera en lambeaux comme un vêtement, et ses habitants périront pareillement ; mais mon salut demeurera éternellement, et ma justice ne sera point anéantie.
\VS{7}Ecoutez-moi, vous qui connaissez la justice, peuple dans le cœur duquel est ma loi ! Ne craignez point l'opprobre des hommes et ne soyez point effrayés devant leurs outrages.
\VS{8}Car la teigne les rongera comme un vêtement\FTNT{Mt. 6:19 ; Lu. 12:33 ; Ja. 5:2.}, et la gerce les dévorera comme de la laine ; mais ma justice demeurera toujours, et mon salut d'âge en âge.
\VS{9}Réveille-toi, réveille-toi, revêts-toi de force, bras de Yahweh ! Réveille-toi comme aux jours anciens, aux siècles passés. N'es-tu pas celui qui tailla en pièce l'Egypte, et qui blessa mortellement le dragon ?
\VS{10}N'est-ce pas toi qui fis tarir la mer, les eaux du grand abîme ? Qui réduisit les lieux les plus profonds de la mer en un chemin afin que les rachetés y passent ?
\VS{11}Ainsi ceux dont Yahweh aura payé la rançon, retourneront, ils iront à Sion avec chants de triomphe ; et une allégresse éternelle couronnera leurs têtes ; ils obtiendront la joie et l'allégresse ; la douleur et le gémissement s'enfuiront.
\VS{12}C'est moi qui suis celui qui vous console. Qui es-tu pour avoir peur de l'homme mortel qui mourra, et du fils de l'homme qui deviendra comme du foin ?
\VS{13}Et tu oublierais Yahweh qui t'a fait, qui a étendu les cieux et fondé la terre ; et chaque jour tu tremblerais continuellement à cause de la fureur de ton oppresseur parce qu'il s'apprête à détruire ! Et où est maintenant la fureur de ton oppresseur ?
\VS{14}Il se hâtera de faire que celui qui aura été transporté d'un lieu à l'autre, soit mis en liberté, afin qu'il ne meure point dans la fosse, et que son pain ne lui manque pas.
\VS{15}Car je suis Yahweh, ton Dieu, qui fend la mer, et les flots rugissants. Yahweh des armées est son Nom.
\VS{16}Or je mets mes paroles dans ta bouche, et je te couvre de l'ombre de ma main, afin que j'affermisse les cieux, que je fonde la terre, et que je dise à Sion : Tu es mon peuple !
\VS{17}Réveille-toi, réveille-toi ! Lève-toi, Jérusalem, qui as bu de la main de Yahweh la coupe de sa fureur ; tu as bu, tu as sucé la lie de la coupe d'étourdissement\FTNT{Ps. 60:5 ; Ap. 14:10.} !
\VS{18}Il n'y a pas un de tous les enfants qu'elle a enfantés qui te conduise, et de tous les enfants qu'elle a nourris, il n'y en a pas un qui la prenne par la main.
\VS{19}Ces deux choses te sont arrivées ; qui te plaindra ? Le ravage et la ruine, la famine et l'épée ; par qui te consolerai-je ?
\VS{20}Tes enfants en défaillance gisaient aux carrefours de toutes les rues, comme un bœuf sauvage pris dans les filets, pleins de la fureur de Yahweh, de la répréhension de ton Dieu.
\VS{21}C'est pourquoi, écoute maintenant ceci, ô affligée, ivre, mais non pas de vin.
\VS{22}Ainsi parle Yahweh, ton Seigneur et ton Dieu, qui plaide la cause de son peuple : Voici, je prends de la main la coupe d'étourdissement, la lie de la coupe de ma fureur, tu n'en boiras plus désormais !
\VS{23}Car je la mettrai dans la main de ceux qui t'ont affligée, et qui disaient à ton âme : Courbe-toi, et nous passerons ! C'est pourquoi tu as exposé ton corps  comme la terre, comme une rue pour les passants.
\Chap{52}
\TextTitle{Le réveil de Jérusalem, la ville sainte}
\VerseOne{}Réveille-toi, réveille-toi, Sion ! Revêts-toi de ta force ! Jérusalem, ville sainte ! Revêts-toi de tes vêtements magnifiques ! Car l'incirconcis et le souillé ne passeront plus désormais parmi toi. 
\VS{2}Jérusalem, secoue ta poussière, lève-toi, et assieds-toi ! Détache les liens de ton cou, captive, fille de Sion !
\VS{3}Car ainsi parle Yahweh : Vous avez été vendus pour rien, et vous serez aussi rachetés sans argent.
\VS{4}Car ainsi parle le Seigneur, Yahweh : Mon peuple descendit jadis en Egypte pour y séjourner ; mais les Assyriens l'opprimèrent sans cause.
\VS{5}Et maintenant, qu'ai-je à faire ici, dit Yahweh, quand mon peuple a été enlevé pour rien ? Ceux qui dominent sur lui le font hurler, dit Yahweh, et mon Nom est blasphémé continuellement chaque jour.
\VS{6}C'est pourquoi mon peuple connaîtra mon Nom ; c'est pourquoi il saura, en ce jour-là, que JE SUIS parle : Voici JE SUIS !
\VS{7}Combien sont beaux sur les montagnes les pieds de celui qui apporte de bonnes nouvelles, qui publie la paix\FTNT{Na. 2:1 ; Ro. 10:15.}, qui apporte de bonnes nouvelles concernant le bien, qui publie le salut, qui dit à Sion : Ton Dieu règne !
\VS{8}Tes sentinelles élèvent leurs voix, elles se réjouissent ensemble avec chants de triomphe ; car de leurs propres yeux elles voient comment Yahweh ramène Sion.
\VS{9}Déserts de Jérusalem, éclatez, réjouissez-vous ensemble avec chants de triomphe ! Car Yahweh console son peuple, il rachète Jérusalem.
\VS{10}Yahweh manifeste le bras de sa sainteté aux yeux de toutes les nations\FTNT{Es. 53:1.}, et toutes les extrémités de la terre verront le salut\FTNT{Toutes les extrémités de la terre verront le salut de Yahweh, c'est-à-dire Jésus (Mt. 28:18-20). } de notre Dieu.
\VS{11}Retirez-vous, retirez-vous, sortez de là ! Ne touchez rien d'impur ! Sortez du milieu d'elle\FTNT{Jé. 51:45 ; 2 Co. 6:17 ; Ap. 18:4.} ! Nettoyez-vous, vous qui portez les vases de Yahweh.
\VS{12}Car vous ne sortirez pas en hâte, et vous ne marcherez pas en fuyant, car Yahweh ira devant vous, et le Dieu d'Israël sera votre arrière-garde.
\TextTitle{Le serviteur de Yahweh}
\VS{13}Voici, mon serviteur prospérera, il sera fort exalté, élevé et glorifié.
\VS{14}Comme plusieurs ont été étonnés en te voyant, son visage était défiguré plus que celui d'aucun homme, et son apparence plus que celle d'aucun fils d'homme ;
\VS{15}ainsi, il aspergera plusieurs nations, et les rois fermeront la bouche sur lui ; car ceux auxquels on n'en avait point parlé le verront ; et ceux qui ne l'avait point entendu l'entendront.
\Chap{53}
\TextTitle{Le sacrifice du Messie, serviteur de Yahweh}
\VerseOne{}Qui a cru à notre prédication ? Et à qui le bras de Yahweh\FTNT{Jésus-Christ homme est le bras de Yahweh. Le bras de Yahweh est le symbole de la puissance divine. Cette puissance s'est manifestée dans l'œuvre du Messie accomplissant le salut du monde. Le prophète est transporté au moment où le peuple juif, après avoir rejeté son Messie, ouvrira enfin les yeux et acceptera celui qu'il a percé (Za. 12:10 ; Ap. 1:7). Voir aussi Jé. 27:4-5 ; Jé. 32:17.} a-t-il été révélé ?
\VS{2}Toutefois il s'est élevé devant lui comme une jeune plante, comme un rejeton qui sort d'une terre desséchée ; il n'y avait en lui ni beauté, ni splendeur, quand nous le regardions, ni apparence qui nous le fasse désirer.
\VS{3}Il était le méprisé et le rejeté des hommes\FTNT{Ps. 22:6-7 ; Mt. 27:27-31 ; Mc. 9:12 ; Jn. 16:32.}, homme de douleur, et sachant ce que c'est que la maladie ; et nous avons comme caché notre visage arrière de lui, tant il était méprisé ; et nous ne l'avons pas estimé.
\VS{4}En vérité, il a porté nos maladies, et il s'est chargé de nos douleurs\FTNT{Mt. 8:17 ; 1 Pi. 2:24.} ; et nous l'avons considéré comme frappé, battu par Dieu et humilié.
\VS{5}Mais il était transpercé pour nos péchés, brisé pour nos iniquités, le châtiment qui nous apporte la paix est tombé sur lui, et c'est par ses meurtrissures que nous avons la guérison.
\VS{6}Nous avons tous été errants\FTNT{Pierre, apôtre de l'Agneau, confirme que le Messie est bel et bien le Bon Berger (1 Pi. 2:25).} comme des brebis, nous nous sommes détournés, chacun suivait son propre chemin, et Yahweh a fait venir sur lui l'iniquité de nous tous.
\VS{7}Opprimé et humilié, il n'a point ouvert sa bouche\FTNT{Mt. 26:62-63 ; Mc. 15:3-5 ; Jn. 19:9 ; Ac. 8:32-33.}, semblable à un agneau qu'on mène à la boucherie, à une brebis muette devant celui qui la tond, et il n'a point ouvert sa bouche.
\VS{8}Il a été enlevé de la force de l'angoisse et de la condamnation ; mais qui racontera sa durée ? Car il a été retranché de la terre des vivants, et la plaie lui a été faite pour les péchés de mon peuple.
\VS{9}On a mis son sépulcre parmi les méchants, et dans sa mort, il a été avec le riche, quoiqu'il n'ait point commis de violence, et qu'il n'y ait point eu de fraude dans sa bouche\FTNT{Mc. 15:28 ; Lu. 23:32-33.}.
\VS{10}Toutefois il a plu à Yahweh de le briser ; il l'a mis dans la souffrance. Après avoir mis son âme en sacrifice pour le péché, il verra une postérité et prolongera ses jours ; et le bon plaisir de Yahweh prospérera en sa main\FTNT{Jé. 23:5.}.
\VS{11}Il jouira du travail de son âme et en sera rassasié ; mon serviteur juste justifiera beaucoup d'hommes par la connaissance qu'ils auront de lui ; et lui-même portera leurs iniquités.
\VS{12}C'est pourquoi je lui donnerai sa part parmi les grands ; il partagera le butin avec les puissants, parce qu'il a livré son âme à la mort, qu'il a été mis au rang des transgresseurs, et que lui-même a porté les péchés de plusieurs, et qu'il a intercédé pour les transgresseurs.
\Chap{54}
\TextTitle{Yahweh réhabilite Israël la délaissée}
\VerseOne{}Réjouis-toi avec chants de triomphe, stérile, toi qui n'enfantes point, toi qui n'a pas connu les douleurs de l'accouchement! Eclate de joie avec chant de triomphe et réjouis-toi  ! Car les enfants de la délaissée seront plus nombreux que les enfants de celle qui est mariée, dit Yahweh.
\VS{2}Elargis l'espace de ta tente, et qu'on étende les couvertures de ton tabernacle : Ne retiens rien ! Allonge tes cordages et affermis tes pieux !
\VS{3}Car tu te répandras à droite et à gauche, et ta postérité possédera les nations et peuplera les villes désertes.
\VS{4}Ne crains pas, car tu ne seras point honteuse, ni confuse, et tu ne rougiras pas ; mais tu oublieras la honte de ta jeunesse, et tu ne te souviendras plus de l'opprobre de ton veuvage.
\VS{5}Car ton Créateur est ton époux : Yahweh des armées est son Nom ; et ton Rédempteur est le Saint d'Israël : Il sera appelé le Dieu de toute la terre.
\VS{6}Car Yahweh t'appelle comme une femme délaissée et à l'esprit affligé, comme une femme qu'on a épousée dans la jeunesse, et qui a été répudiée, dit ton Dieu.
\VS{7}Je t'avais délaissée pour un petit moment, mais je te rassemblerai avec de grandes compassions.
\VS{8}Dans une courte colère, je t'avais un moment caché ma face, mais j'aurai compassion de toi avec une bonté éternelle, dit Yahweh, ton Rédempteur.
\VS{9}Car il en sera pour moi comme les eaux de Noé : De même que j'avais juré que les eaux de Noé ne se répandraient plus sur la terre\FTNT{Ge. 9:11 ; Ge. 8:21.} ; je jure de ne plus m'irriter contre toi, et de ne plus te menacer.
\VS{10}Car quand les montagnes s'en iraient, quand les collines chancelleraient, ma bonté ne s'en ira point de toi, et mon alliance de paix ne chancellera point, dit Yahweh, qui a compassion de toi.
\VS{11}Ô affligée, agitée de la tempête, dénuée de consolation, voici, je coucherai tes pierres d'antimoine, et je te fonderai sur des saphirs ;
\VS{12}et je ferai tes fenêtrages d'agates, et tes portes de rubis, et toute ton enceinte de pierres précieuses.
\VS{13}Aussi tous tes enfants seront enseignés de Yahweh, et grande sera la paix de tes fils.
\VS{14}Tu seras établie en justice, tu seras loin de l'oppression, et tu ne craindras rien ; tu seras, dis-je, loin de la frayeur, car elle n'approchera pas de toi.
\VS{15}Voici, on ne manquera pas de comploter contre toi, cela ne viendra pas de moi ; quiconque complotera contre toi tombera pour l'amour de toi\FTNT{Ps. 91:7 ; Ge. 37.}.
\VS{16}Voici, c'est moi qui ai créé le forgeron soufflant le charbon au feu, et formant un instrument pour son travail, et j'ai créé aussi le destructeur pour détruire.
\VS{17}Aucune arme forgée contre toi ne réussira, et toute langue qui se lèvera en jugement contre toi, tu la condamneras\FTNT{Ps. 23:4.}. Tel est l'héritage des serviteurs de Yahweh, et telle est la justice qui leur viendra de moi, dit Yahweh.
\Chap{55}
\TextTitle{Le salut gratuit par la grâce de Dieu}
\VerseOne{}Vous tous qui avez soif, venez aux eaux, et vous qui n'avez pas d'argent, venez, achetez et mangez ; venez, dis-je, achetez du vin et du lait sans argent, et sans rien payer !
\VS{2}Pourquoi dépensez-vous de l'argent pour ce qui ne nourrit pas ? Pourquoi travaillez-vous pour ce qui ne rassasie pas\FTNT{Ro. 14:17.} ? Ecoutez-moi attentivement, et vous mangerez de ce qui est bon, et votre âme se délectera de la graisse.
\VS{3}Inclinez l'oreille, et venez à moi\FTNT{Mt. 11:28.}, écoutez, et votre âme vivra ; et je traiterai avec vous une alliance éternelle, les miséricordes immuables promises à David.
\VS{4}Voici, je l'ai donné comme témoin auprès des peuples, comme chef et dominateur des peuples.
\VS{5}Voici, tu appelleras des nations que tu ne connais pas, et les nations qui ne te connaissent pas accourront vers toi, à cause de Yahweh, ton Dieu, et du Saint d'Israël, qui t'auras glorifié.
\VS{6}Cherchez Yahweh pendant qu'il se trouve, invoquez-le tandis qu'il est près.
\VS{7}Que le méchant abandonne sa voie, et l'homme injuste ses pensées ; et qu'il retourne à Yahweh, qui aura pitié de lui, et à notre Dieu qui pardonne abondamment\FTNT{Jé. 18:11 ; Ez. 33:11 ; Jon. 3:10 ; 1 Ti. 2:1-4 ; 2 Pi. 3:9.}.
\VS{8}Car mes pensées ne sont pas vos pensées, et mes voies ne sont pas vos voies, dit Yahweh.
\VS{9}Mais autant les cieux sont élevés au-dessus de la terre, autant mes voies sont élevées au-dessus de vos voies, et mes pensées au-dessus de vos pensées.
\VS{10}Car comme la pluie et la neige descendent des cieux et n'y retournent plus, mais arrosent la terre, et la font produire et germer, afin de donner de la semence au semeur, et du pain à celui qui mange,
\VS{11}ainsi en est-il de ma parole qui sort de ma bouche, elle ne retourne point vers moi sans effet, mais elle fait tout ce en quoi je prends plaisir, et prospérera dans l'œuvre pour laquelle je l'ai envoyée.
\VS{12}Car vous sortirez avec joie, et vous serez conduits en paix ; les montagnes et les collines éclateront de joie avec chants de triomphe devant vous, et tous les arbres des champs battront des mains.
\VS{13}Au lieu de l'épine s'élèvera le cyprès, au lieu de la ronce croîtra le myrte ; et ceci fera connaître le nom de Yahweh, et ce sera un signe perpétuel, qui ne sera jamais retranché.
\Chap{56}
\TextTitle{Exhortation à s'attacher à Yahweh}
\VerseOne{}Ainsi parle Yahweh : Observez le jugement, faites ce qui est juste, car mon salut ne tardera pas à venir, et ma justice à être révélée.
\VS{2}Bienheureux l'homme qui fait cela, et le fils de l'homme qui s'y tient, observant le sabbat pour ne pas le profaner, et gardant ses mains pour ne faire aucun mal.
\VS{3}Et que l'enfant de l'étranger qui se joint à Yahweh ne parle pas en disant : Yahweh me séparera entièrement de son peuple ! Et que l'eunuque ne dise pas : Voici, je suis un arbre sec.
\VS{4}Car ainsi parle Yahweh touchant les eunuques : Ceux qui garderont mes sabbats, et qui choisiront ce en quoi je prends plaisir, et qui tiendront dans mon alliance,
\VS{5}je leur donnerai dans ma maison et dans mes murailles une place et un nom meilleur que le nom de fils ou de filles ; je leur donnerai à chacun un nom éternel qui ne périra jamais\FTNT{Ap. 2:17.}.
\VS{6}Et les enfants des étrangers qui se joindront à Yahweh pour le servir, pour aimer le Nom de Yahweh, pour être ses serviteurs, savoir tous ceux qui garderont le sabbat pour ne pas le profaner et qui tiendront dans mon alliance\FTNT{Ex. 31:14.},
\VS{7}je les amènerai sur ma montagne sainte, et je les réjouirai dans ma maison de prière ; leurs holocaustes et leurs sacrifices seront agréés sur mon autel, car ma maison sera appelée une maison de prière\FTNT{Mt. 21:13 ; Mc. 11:17 ; Lu. 19:46.} pour tous les peuples.
\VS{8}Le Seigneur, Yahweh, parle, lui qui rassemble les exilés d'Israël : Je réunirai d'autres peuples à lui, outre ceux déjà rassemblés.
\VS{9}Bêtes des champs, bêtes des forêts, venez toutes pour manger !
\VS{10}Toutes ses sentinelles sont aveugles, elles ne connaissent rien ; ce sont tous des chiens muets, qui ne peuvent aboyer, dormant et demeurant couchés, et aimant à sommeiller.
\VS{11}Ce sont des chiens voraces et insatiables ; ce sont des pasteurs qui ne savent rien comprendre ; tous suivent leur propre voie, chacun à son gain injuste dans son quartier, en disant\FTNT{Mt. 23:24 ; Tit. 1:7-11 ; 1 Pi. 5:2.} :
\VS{12}Venez, je vais chercher du vin, et nous nous enivrerons de boissons fortes ! Nous en ferons autant demain, et même beaucoup plus encore !
\Chap{57}
\TextTitle{Yahweh expose la fausseté et défend le juste}
\VerseOne{}Le juste périt, et nul ne le prend à cœur ; et les gens de bien sont recueillis, sans qu'on y soit attentif, sans qu'on considère que le juste a été recueilli devant le mal\FTNT{Mi. 7:2 ; Ec. 7:15.}.
\VS{2}Il entrera en paix, il reposera sur sa couche, celui qui aura marché dans la droiture\FTNT{Mt. 25:23 ; Lu. 19:17.}.
\VS{3}Mais vous, approchez ici, enfants de l'enchanteresse, race de l'adultère et de la prostituée !
\VS{4}De qui vous êtes-vous moqués ? Contre qui avez-vous ouvert la bouche et tirez-vous la langue ? N'êtes-vous pas des enfants de rébellion, une race de mensonge ?
\VS{5}S'échauffant près des faux dieux, sous tout arbre vert ; égorgeant les enfants dans les vallées, sous les fentes des rochers\FTNT{Lé. 18:21 ; 1 R. 14:23 ; Jé. 2:20 ; Jé. 32:35.}.
\VS{6}Parmi les pierres polies des torrents est ta portion, ce sont elles, ce sont elles qui sont ton lot ; tu leur a aussi répandu ton aspersion, tu leur as aussi offert des offrandes ; puis-je être content de ces choses ?
\VS{7}Tu dresses ta couche sur les montagnes hautes et élevées ; c'est aussi là que tu montes pour offrir des sacrifices.
\VS{8}Et tu mets ton souvenir derrière la porte et les poteaux ; car tu te découvres loin de moi et tu montes, tu élargis ta couche, et tu te l'est taillé plus grande que n'ont fait ceux-là ; tu as aimé leur couche, tu as pris garde aux belles places.
\VS{9}Tu voyages vers le roi avec de l'huile précieuse, et tu ajoutes parfums sur parfums ; tu envoies au loin tes ambassades, tu t'abaisses jusqu'au scheol.
\VS{10}Tu te fatigues par la longueur du chemin, et tu ne dis pas : C'est sans espoir ! Tu trouves encore de la vigueur dans ta main ; c'est pourquoi tu n'as pas été languissante.
\VS{11}Et qui redoutais-tu, qui craignais-tu pour que tu me mentes, pour ne pas te souvenir et te soucier de moi ? N'ai-je pas gardé le silence, et même depuis longtemps, et tu ne me crains pas.
\VS{12}Je vais déclarer ta justice et tes œuvres, qui ne te profiteront pas.
\VS{13}Quand tu crieras, que ceux que tu assembles te délivrent ! Mais le vent les emmènera tous, la vanité les enlèvera ; mais celui qui met sa confiance en moi, héritera la terre et possédera ma montagne sainte\FTNT{Es. 2:3 ; Ps. 2:6 ;  Hé. 12:22.}.
\VS{14}On dira : Frayez, frayez, préparez le chemin, enlevez tout obstacle loin du chemin de mon peuple !
\TextTitle{Yahweh aime l'homme contrit}
\VS{15}Car ainsi parle celui qui est haut et élevé, qui habite dans l'éternité et dont le nom est le Saint : J'habiterai dans les lieux hauts et saints, avec celui qui a le cœur brisé et qui est humble d'esprit, afin de vivifier l'esprit des humbles, et afin de vivifier ceux qui ont le cœur brisé\FTNT{Ps. 34:19 ; Ps. 51:19.}.
\VS{16}Parce que je ne veux pas contester à toujours, et que je ne serai pas irrité à jamais ;  car devant moi tombent en défaillance les esprits, et les âmes que j'ai faites\FTNT{Mi. 7:18 ; Ps. 85:6 ; Ps. 103:9.}.
\VS{17}A cause de l'iniquité de ses gains déshonnêtes, je me suis irrité et je l'ai frappé, je me suis caché ma dans ma colère ; et le rebelle a suivi la voie de son cœur.
\VS{18}J'ai vu ses voies, et toutefois je le guérirai ; je le conduirai et je le restaurerai, lui et ceux qui mènent deuil avec lui.
\VS{19}Je crée les fruits des lèvres. Paix, paix à celui qui est loin et à celui qui est près ! dit Yahweh, car je le guérirai.
\VS{20}Mais les méchants sont comme la mer agitée, quand elle ne peut se calmer, et dont les eaux rejettent la boue et le bourbier.
\VS{21}Il n'y a point de paix pour les méchants, dit mon Dieu.
\Chap{58}
\TextTitle{Le vrai et le faux jeûne}
\VerseOne{}Crie à plein gosier, ne te retiens pas, élève ta voix comme un shofar, et annonce à mon peuple ses iniquités et à la maison de Jacob ses péchés !
\VS{2}Car ils me cherchent tous les jours, ils prennent plaisir à connaître mes voies ; comme une nation qui aurait pratiqué la justice, et qui n'aurait pas abandonné les ordonnances de son Dieu ; ils me demandent des jugements justes, ils prennent plaisir à s'approcher de Dieu, et puis ils disent :
\VS{3}Pourquoi jeûnons-nous, et tu ne le vois pas ? Pourquoi affligeons-nous nos âmes, si tu n'y as point connaissance ? Voici, le jour de votre jeûne, vous trouvez votre plaisir, et vous oppressez tous vos travailleurs.
\VS{4}Voici, vous jeûnez pour faire des querelles et vous disputer, et pour frapper du poing méchamment ; vous ne jeûnez pas comme le veut ce jour, pour que votre voix soit exaucée d'en haut.
\VS{5}Est-ce là le jeûne que j'ai choisi, que l'homme afflige son âme un jour ? Est-ce en courbant sa tête comme le jonc et en étendant le sac et la cendre ? Appelleras-tu cela un jeûne et un jour agréable à Yahweh ?
\VS{6}N'est-ce pas plutôt ici le jeûne que j'ai choisi : Que tu détaches les liens de la méchanceté, que tu délies les cordages du joug, que tu laisses aller libres les opprimés, et que l'on rompe toute espèce de joug ?
\VS{7}N'est ce-pas que tu partages ton pain avec celui qui a faim ? Et que tu fasses venir dans ta maison les affligés errants ? Quand tu vois un homme nu, que tu le couvres, et que tu ne te caches pas de ta propre chair ?
\TextTitle{Bénédiction pour ceux qui pratiquent le bien}
\VS{8}Alors ta lumière éclatera comme l'aurore, et ta guérison germera rapidement ; ta justice ira devant toi, et la gloire de Yahweh sera ton arrière-garde.
\VS{9}Alors tu prieras, et Yahweh t'exaucera ; tu crieras, et il dira : Me voici ! Si tu ôtes du milieu de toi le joug, si tu cesses de lever le doigt et de dire des outrages ;
\VS{10}si tu ouvres ton âme à celui qui a faim, si tu rassasies l'âme affligée ; ta lumière se lèvera sur les ténèbres, et l'obscurité sera comme le midi.
\VS{11}Et Yahweh te conduira continuellement, il rassasiera ton âme dans les grandes sécheresses, il fortifiera tes os, et tu seras comme un jardin arrosé, et comme une source dont les eaux ne tarissent pas\FTNT{Jn. 4:14 ; Ap. 21:6.}.
\VS{12}Et ceux qui sortiront de toi rebâtiront les lieux déserts depuis longtemps, tu rétabliras les fondements ruinés depuis plusieurs générations ; et on t'appellera le réparateur des brèches et le restaurateur des chemins, afin qu'on habite au pays.
\VS{13}Si tu détournes ton pied pendant le sabbat pour ne pas faire ta volonté en mon saint jour ; si tu appelles le sabbat tes délices, et honorable ce qui est saint à Yahweh, et si tu l'honores en ne suivant point tes voies, en ne te livrant pas à tes désirs et à des vains discours,
\VS{14}alors tu prendras plaisir en Yahweh, et je te ferai monter comme à cheval par-dessus les lieux haut élevés de la terre, et je te donnerai à manger de l'héritage de Jacob, ton père ; car la bouche de Yahweh a parlé.
\Chap{59}
\TextTitle{Le péché sépare de Yahweh}
\VerseOne{}Voici, la main de Yahweh n'est pas trop courte pour pouvoir sauver, ni son oreille trop pesante pour pouvoir entendre.
\VS{2}Mais ce sont vos iniquités qui mettent une séparation entre vous et votre Dieu ; ce sont vos péchés qui vous cachent sa face, afin qu'il ne vous entende point\FTNT{De. 31:17-18 ; Ez. 39:23-24.}.
\VS{3}Car vos mains sont souillées de sang, et vos doigts d'iniquité ; vos lèvres profèrent le mensonge, et votre langue déclare la perversité.
\VS{4}Nul ne crie pour la justice, nul ne plaide pour la vérité ; ils s'appuient sur des choses vaines et disent des faussetés, ils conçoivent le mal et enfantent l'iniquité.
\VS{5}Ils font éclore des œufs de vipère, et ils tissent des toiles d'araignée ; celui qui mange de leurs œufs meurt ; et si on les écrase, il en sort une vipère.
\VS{6}Leurs toiles ne servent point à faire des vêtements, et on ne se couvre pas de leurs ouvrages ; car leurs ouvrages sont des ouvrages d'iniquité, et il y a en leurs mains des actions de violence.
\VS{7}Leurs pieds courent au mal, et se hâtent pour répandre le sang innocent ; leurs pensées sont des pensées d'iniquité ; le ravage et la ruine sont sur leurs voies.
\VS{8}Ils ne connaissent point le chemin de la paix, et il n'y a point de jugement dans leurs voies, ils se sont pervertis dans leurs sentiers, tous ceux qui y marchent ignorent la paix\FTNT{Pr. 1:16 ; Pr. 6:16-19.}.
\VS{9}C'est pourquoi le jugement s'est éloigné de nous, et la justice ne parvient pas jusqu'à nous ; nous attendions la lumière, et voici les ténèbres, la clarté, et nous marchons dans l'obscurité.
\VS{10}Nous tâtonnons comme des aveugles le long du mur, nous tâtonnons comme ceux qui sont sans yeux ; nous chancelons en plein midi comme la nuit, et nous sommes dans les lieux abondants comme y sont des morts.
\VS{11}Nous rugissons tous comme des ours, et nous ne cessons de gémir comme des colombes ; nous attendons le jugement, et il n'y en a point, la délivrance, et elle est éloignée de nous.
\VS{12}Car nos transgressions se sont multipliées devant toi, et chacun de nos péchés témoignent contre nous ; parce que nos transgressions sont avec nous, et nous connaissons nos iniquités ;
\VS{13}qui sont de pécher et de mentir contre Yahweh, de s'éloigner de notre Dieu, de proférer l'oppression et la révolte, de concevoir et prononcer du cœur des paroles de mensonge.
\VS{14}C'est pourquoi le jugement s'est éloigné et la justice se tient éloignée ; car la vérité est tombée par les rues, et la droiture ne peut y entrer.
\VS{15}Même la vérité a disparu, et quiconque se retire du mal est exposé au pillage ; Yahweh voit, et cela lui a déplu, parce qu'il n'y a plus de droiture.
\TextTitle{Yahweh cherche un homme, il suscite le Messie}
\VS{16}Il voit aussi qu'il n'y a aucun homme, il s'étonne que personne ne se tienne à la brèche ; c'est pourquoi son bras lui vient en aide, et sa propre justice lui sert d'appui\FTNT{Es. 53:1 ; Es. 63:5 ; Ps. 77:15-16 ; Ac. 13:17.}.
\VS{17}Car il se revêt de la justice comme d'une cuirasse, et le casque du salut est sur sa tête\FTNT{Ep. 6:14-17.} ; il se revêt de la vengeance comme d'un vêtement, et se couvre de la jalousie comme d'un manteau.
\VS{18}Selon leurs actes, il rendra à chacun la pareille\FTNT{Jé. 17:10 ; Job. 34:11 ; Mt. 16:27 ; Ap. 2:23 ; Ap. 20:13.}, la fureur à ses adversaires, la rétribution à ses ennemis ; il rendra ainsi la rétribution aux îles.
\VS{19}Et on craindra le Nom de Yahweh depuis l'occident, et sa gloire depuis le soleil levant ; car l'ennemi viendra comme un fleuve, mais l'Esprit de Yahweh lèvera la bannière\FTNT{En hébreu « Yahweh Nissi », c'est-à-dire « Yahweh est ma bannière ». C'est le nom donné par Moïse à l'autel qu'il construisit pour célébrer la défaite d'Amalek (Ex. 17:15). En No. 21:8-9, Moïse éleva une bannière sur laquelle il avait fixé un serpent d'airain pour la guérison des malades. } contre lui.
\VS{20}Et le Rédempteur\FTNT{Le Rédempteur qui viendra pour Sion est le Seigneur Jésus-Christ (Ro. 11:26). Voir aussi Es. 60 : 16. } viendra en Sion, et vers ceux de Jacob qui se convertiront de leur péché, dit Yahweh.
\VS{21}Et quant à moi, c'est ici mon alliance que je ferai avec eux, dit Yahweh : Mon Esprit qui est sur toi, et mes paroles que j'ai mises dans ta bouche, ne se retireront point de ta bouche, ni de la bouche de ta postérité, ni de la bouche de la postérité de ta postérité, dit Yahweh, dès maintenant et à jamais.
\Chap{60}
\TextTitle{La gloire de Yahweh se lèvera sa gloire sur Sion}
\VerseOne{}Lève-toi, sois illuminée, car ta lumière arrive, et la gloire de Yahweh se lève sur toi.
\VS{2}Car voici, les ténèbres couvrent la terre, et l'obscurité couvre les peuples ; mais Yahweh se lève sur toi, et sa gloire apparaît sur toi.
\VS{3}Des nations marchent à ta lumière, et des rois à la splendeur qui se lève sur toi\FTNT{Ap. 21:24.}.
\VS{4}Elève tes yeux alentour, et regarde : Tous ceux-ci s'assemblent, ils viennent vers toi ; tes fils viennent de loin, et tes filles sont nourries par des nourriciers, étant portées sur les côtés.
\VS{5}Alors tu verras et tu seras éclairée, et ton cœur s'étonnera et s'épanouira de joie, quand l'abondance de la mer se sera tournée vers toi, et que la puissance des nations sera venue chez toi.
\VS{6}Tu seras couverte d'une foule de chameaux, des dromadaires de Madian et d'Epha ; et tous ceux de Séba viendront, ils apporteront de l'or et de l'encens, et publieront les louanges de Yahweh.
\VS{7}Toutes les brebis de Kédar seront assemblées vers toi, les béliers de Nebajoth seront à ton service ; ils seront agréables étant offerts sur mon autel, et je rendrai magnifique la maison de ma gloire.
\VS{8}Qui sont ceux-là qui volent comme des nuées, comme des colombes vers leur colombier ?
\VS{9}Car les îles s'attendent à moi, et les navires de Tarsis les premiers, afin d'amener de loin tes enfants, avec leur argent et leur or, à cause du Nom de Yahweh, ton Dieu, et du Saint d'Israël qui te glorifie.
\VS{10}Les fils des étrangers rebâtiront tes murailles, et leurs rois seront employés à ton service ; car je t'ai frappée dans ma colère, mais j'ai eu pitié de toi au temps de mon bon plaisir.
\VS{11}Tes portes seront continuellement ouvertes, elles ne seront fermées ni nuit ni jour, afin que les forces des nations te soient amenées et que leur roi y soient conduits\FTNT{Ap. 21:25-26.}.
\VS{12}Car la nation et le royaume qui ne te serviront pas périront, et ces nations-là seront réduites en une entière désolation.
\VS{13}La gloire du Liban viendra vers toi, le cyprès, l'orme, et le buis, tous ensemble pour rendre honorable le lieu de mon sanctuaire ; et je rendrais glorieux le lieu de mes pieds.
\VS{14}Mais les enfants de tes oppresseurs viendront vers toi en se courbant, et tous ceux qui te méprisaient se prosterneront à tes pieds et t'appelleront la ville de Yahweh, la Sion du Saint d'Israël.
\VS{15}Au lieu d'avoir été délaissée et haïe, si bien que personne ne passait par toi, je te mettrai dans une élévation éternelle et dans une joie qui sera de génération en génération.
\VS{16}Et tu suceras le lait des nations, et tu suceras la mamelle des rois, et tu sauras que je suis Yahweh, ton Sauveur, ton Rédempteur\FTNT{Le verbe « ga'al » et le nom correspondant « go'el », ont été traduits respectivement en français par « racheter » et « rédempteur ». Selon la loi de Moïse, si quelqu'un perdait son héritage à cause d'une dette ou s'il se vendait comme esclave, lui et ses biens pouvaient être rachetés par un proche parent qui devait payer le prix de la rédemption (Lé. 25:23-55). Yahweh se présente comme le Rédempteur par excellence (Es. 49:26 ; Es. 60:16 ; Ps. 78:35 ; Ps. 130:7; Job. 19:25). Or Jésus-Christ «[…] a été fait pour nous sagesse, justice, sanctification et rédemption » (1 Co. 1:30). Les épîtres nous révèlent la rédemption qu'il a acquise pour nous : « nous avons la rédemption par son sang » (Ep. 1:7). La rédemption est le paiement d'une rançon, or il est écrit : « Jésus-Christ s'est donné en rançon pour nous tous » (1 Ti. 2:6). « Vous avez été rachetés à grand prix » (1 Co. 6:20).}, le Puissant de Jacob.
\VS{17}Je ferai venir de l'or au lieu de l'airain, et de l'argent au lieu du fer, et de l'airain au lieu du bois, et du fer au lieu des pierres ; et je ferai régner la paix et dominer la justice.
\VS{18}On n'entendra plus parler de violence dans ton pays ni de ravage et de ruine dans ton territoire ; mais tu appelleras tes murailles : Salut ; et tes portes : Louange.
\VS{19}Tu n'auras plus le soleil pour la lumière du jour, et la lueur de la lune ne t'éclairera plus, mais Yahweh sera pour toi la lumière éternelle\FTNT{Voir le commentaire en Ge 1:3.}, et ton Dieu sera ta gloire.
\VS{20}Ton soleil ne se couchera plus, et ta lune ne se retirera plus, car Yahweh te sera pour lumière perpétuelle, et les jours de ton deuil seront finis.
\VS{21}Quant à ton peuple, ils seront tous justes, ils posséderont la terre à toujours ; savoir le germe de mes plantes, l'œuvre de mes mains pour y être glorifié\FTNT{Es. 11:1 ; Ro. 15:12 ; Ap. 5:5 ; Ap. 22:16.}.
\VS{22}La petite famille deviendra un millier de personnes, et la moindre deviendra une nation puissante. Je suis Yahweh, je hâterai ces choses en leur temps.
\Chap{61}
\TextTitle{La mission du Messie}
\VerseOne{}L'Esprit du Seigneur Yahweh est sur moi, car Yahweh m'a oint pour évangéliser les malheureux ; il m'a envoyé pour guérir ceux qui ont le cœur brisé, pour proclamer aux captifs la liberté, et aux prisonniers l'ouverture de la prison ;
\VS{2}pour publier une année de grâce de Yahweh, et le jour de vengeance de notre Dieu ; pour consoler tous ceux qui mènent deuil\FTNT{Lu. 4:14-19.} ;
\VS{3}pour annoncer à ceux de Sion qui mènent deuil, que la magnificence leur sera donnée au lieu de la cendre, une huile de joie au lieu du deuil, un manteau de louange au lieu d'un esprit abattu\FTNT{Job. 29:14 ; Ja. 1:12 ; 1 Co.9:25 ; 2 Ti. 4:8.}, afin qu'on les appelle des térébinthes de la justice, une plantation de Yahweh, pour servir à sa gloire.
\VS{4}Et ils rebâtiront les ruines antiques, ils relèveront les lieux qui étaient auparavant désolés, et ils renouvelleront des villes ravagées, et les choses désolées d'âge en âge.
\VS{5}Et des étrangers s'y tiendront là et feront paître vos troupeaux, et les enfants de l'étranger seront vos laboureurs et vos vignerons.
\VS{6}Mais vous, vous serez appelés sacrificateurs de Yahweh, et on vous nommera serviteurs de notre Dieu\FTNT{Ap. 1:6 ; Ap. 5:10.} ; vous mangerez les richesses des nations, et vous vous glorifierez de leur gloire.
\VS{7}Au lieu de la honte que vous avez eue, les nations en auront le double, et elles crieront tout haut que la confusion est leur portion ; c'est pourquoi ils posséderont le double dans leur pays, et leur joie sera éternelle.
\VS{8}Car je suis Yahweh qui aime le jugement et qui hait la rapine pour l'holocauste ; j'établirai leur œuvre dans la vérité et je traiterai avec eux une alliance éternelle.
\VS{9}Et leur race sera connue parmi les nations, et ceux qui seront sortis d'eux seront connus parmi les peuples ; tous ceux qui les verront connaîtront qu'ils sont la race que Yahweh aura bénie.
\VS{10}Je me réjouirai extrêmement en Yahweh, et mon âme se réjouira en mon Dieu ; car il m'a revêtu des vêtements du salut, il m'a couvert du manteau de la justice, comme un époux qui se pare de magnificence, et comme une épouse qui s'orne de ses joyaux\FTNT{Os. 2:21-22 ; Ap. 19:7-8.}.
\VS{11}Car comme la terre fait éclore son germe, et comme un jardin fait germer ses semences, ainsi le Seigneur Yahweh fera germer la justice, et la louange en présence de toutes les nations.
\Chap{62}
\TextTitle{Yahweh proclamme la restauration d'Israël}
\VerseOne{}Pour l'amour de Sion, je ne me tiendrai pas tranquille, et pour l'amour de Jérusalem je ne prendrai point de repos, jusqu'à ce que sa justice sorte dehors comme une splendeur, et que sa délivrance ne soit allumée comme une lampe.
\VS{2}Alors les nations verront ta justice, et tous les rois ta gloire ; et on t'appellera d'un nouveau nom\FTNT{Ap. 2:17.}, que la bouche de Yahweh aura expressément déclaré.
\VS{3}Tu seras une couronne de gloire dans la main de Yahweh, un turban royal dans la main de ton Dieu.
\VS{4}On ne te nommera plus la délaissée, et on ne nommera plus ta terre la désolation ; mais on t'appellera mon bon plaisir en elle ; et on appellera ta terre l'épouse ; car Yahweh prend son bon plaisir en toi, et ta terre aura un époux.
\VS{5}Car comme le jeune homme épouse la vierge, comme tes enfants se marient chez toi, ainsi ton Dieu se réjouira en toi, de la joie qu'un époux a de son épouse.
\VS{6}Jérusalem, j'ai placé des gardes sur tes murailles tout le jour et toute la nuit, et ils ne se tairont point. Vous qui faites mention de Yahweh, ne gardez point le silence !
\VS{7}Et ne vous arrêtez pas de l'invoquer jusqu'à ce qu'il rétablisse Jérusalem et lui rende sa renommée sur la terre.
\VS{8}Yahweh l'a juré par sa droite et par son bras puissant : Je ne donnerai plus ton froment pour nourriture à tes ennemis, et les enfants des étrangers ne boiront plus ton vin excellent pour lequel tu as travaillé.
\VS{9}Mais ceux qui auront amassé le froment le mangeront et loueront Yahweh, et ceux qui auront récolté le vin le boiront dans les parvis de ma sainteté.
\VS{10}Passez, passez les portes ! Disant : Préparez le chemin du peuple ! Frayez, frayez la route, et ôtez-en les pierres ! Elevez une bannière vers les peuples.
\VS{11}Voici ce que Yahweh proclame aux extrémités de la terre : Dites à la fille de Sion : Voici, ton Sauveur vient\FTNT{De nombreux passages, notamment dans le livre d'Esaïe, présentent Dieu comme le sauveur, le seul sauveur (Es. 43:3 ; Es. 43:11 ; Os. 13:4) qui viendra pour délivrer son peuple (Es. 35:4 ; Es. 60:1 ; Za. 14:1-7). Jésus-Christ a accompli en tous points les prophéties relatives à la venue de Yahweh. Dieu est bel et bien venu sur terre il y a plus de 2000 ans et ce même Dieu revient bientôt (Ac. 1:11 ; Ap. 1:7).} ; voici, son salaire est avec lui, et sa récompense marche devant lui.
\VS{12}Et on les appellera le peuple saint, les rachetés de Yahweh\FTNT{1 Pi. 2:9 ; Ap. 5:9.} ; et toi, on t'appellera la recherchée, la ville non abandonnée.
\Chap{63}
\TextTitle{Le jour de vengeance du Messie\FTNTT{Es. 2:10-22 ; Ap. 19:11-21.}}
\VerseOne{}Qui est celui-ci qui vient d'Edom, de Botsra, en habits rouges, magnifiquement paré en son vêtement, marchant selon la grandeur de sa force ? C'est moi qui parle en justice et qui ai tout pouvoir de sauver.
\VS{2}Pourquoi tes vêtements sont-ils rouges, et pourquoi tes habits sont comme les habits de ceux qui foulent dans la cuve ?
\VS{3}J'ai été seul à fouler au pressoir, et nul homme d'entre les peuples n'était avec moi. Cependant, j'ai marché sur eux dans ma colère, et je les ai foulés dans ma fureur ; et leur sang a rejailli sur mes vêtements, et j'ai souillé tous mes habits.
\VS{4}Car le jour de la vengeance était dans mon cœur, et l'année de mes rachetés est venue.
\VS{5}Je regardais donc, il n'y avait personne pour m'aider ; et j'étais étonné, et  il n'y avait personne pour me soutenir ; mais mon bras m'a sauvé et ma fureur m'a soutenu.
\VS{6}Ainsi j'ai foulé des peuples dans ma colère, et je les ai enivrés dans ma fureur ; et j'ai abattu leur force par terre.
\TextTitle{Esaïe confesse les péchés du peuple}
\VS{7}Je ferai mention des bontés de Yahweh, qui sont les louanges de Yahweh, pour tous les bienfaits que Yahweh nous a faits ; car grande est la bonté envers la maison d'Israël, qu'il a traitée selon ses compassions et la richesse de sa miséricorde.
\VS{8}Car il a dit : Certainement, ils sont mon peuple, des enfants qui ne tricheront pas ! Et il a été pour eux un Sauveur.
\VS{9}Et dans toutes leurs détresses, il a été en détresse, et l'ange qui est devant sa face les a délivrés\FTNT{Ge. 16:7-10 ; Jg. 6:11-14 ; Za.1:11.} ; lui-même les a rachetés dans son amour et sa miséricorde, et il les a soutenus et portés, tous les jours d'autrefois.
\VS{10}Mais ils ont été rebelles, et ils ont attristé son Esprit saint\FTNT{Ep. 4:30.}, c'est pourquoi il est devenu leur ennemi, et il a lui-même combattu contre eux.
\VS{11}Et on se souvint des anciens jours de Moïse et de son peuple. Où est celui, a-t-on dit, qui les fit monter de la mer, avec les pasteurs de son troupeau ? Où est celui qui mit au milieu d'eux son Esprit saint ;
\VS{12}qui les dirigea par la droite de Moïse et par son bras glorieux ; qui fendit les eaux devant eux pour se faire un nom éternel ?
\VS{13}Qui les dirigea à travers les flots, comme un cheval dans le désert, sans qu'ils ne bronchent ?
\VS{14}L'Esprit de Yahweh les a menés au repos comme on mène une bête qui descend dans la vallée. C'est ainsi que tu as conduit ton peuple, afin de t'acquérir un nom glorieux.
\VS{15}Regarde du ciel et vois de ta demeure sainte et glorieuse : Où sont ton zèle et ta puissance ? Le son de tes entrailles et de tes compassions se retiennent-ils envers moi ?
\VS{16}Certes tu es notre Père, encore qu'Abraham ne nous connaisse pas, et qu'Israël ne nous reconnaisse pas ; Yahweh, c'est toi qui es notre Père, et ton Nom est notre Rédempteur de tout temps.
\VS{17}Pourquoi nous as-tu fait égarer loin de tes voies, ô Yahweh, et endurcis-tu notre cœur contre ta crainte ? Reviens, pour l'amour de tes serviteurs, des tribus de ton héritage !
\VS{18}Ton peuple saint n'a possédé le pays que peu de temps ; nos ennemis ont foulé ton sanctuaire.
\TextTitle{Prière du reste d'Israël à Yahweh pour sa délivrance}
\VS{19}Nous sommes comme ceux sur lesquels tu ne domines pas depuis longtemps, et sur lesquels ton Nom n'est point réclamé. Ô ! Si tu fendais les cieux, et si tu descendais, les montagnes s'ébranleraient devant toi !
\Chap{64}
\VerseOne{}Comme un feu de fonte est ardent, le feu fait bouillir l'eau, afin de faire connaître ton Nom à tes ennemis, et que les nations tremblent en ta présence.
\VS{2}Lorsque tu fis les choses redoutables que nous n'attendions pas, tu descendis et les montagnes tremblèrent devant toi.
\VS{3}Jamais on n'a appris ni entendu dire, et jamais l'œil n'a vu qu'un autre dieu que toi fît de telles choses pour ceux qui s'attendent à lui\FTNT{1 Co. 2:9.}.
\VS{4}Tu viens à la rencontre de celui qui se réjouit et qui agit avec justice, et  se souviennent de toi dans tes voies. Voici tu as été irrité parce que nous avons péché ; tes compassions sont éternelles, c'est pourquoi nous serons sauvés.
\VS{5}Or nous sommes tous devenus comme une chose souillée, et toute notre justice est comme le linge le plus souillé\FTNT{Ap. 19:8.} ; nous sommes tous flétris comme la feuille, et nos iniquités nous emportent comme le vent.
\VS{6}Il n'y a personne qui invoque ton Nom, qui se réveille pour s'attacher fortement à toi ; c'est pourquoi tu nous as caché ta face, et tu nous fais fondre par l'effet de nos iniquités.
\VS{7}Cependant, ô Yahweh, tu es notre Père ; nous sommes l'argile, et c'est toi qui nous as formés, et nous sommes tous l'ouvrage de ta main\FTNT{Es. 29:16 ; Es. 45:9 ; Jé. 18:6 ; Ro. 9:20-21.}.
\VS{8}Ne t'irrite pas à l'extrême, ô Yahweh, et ne te souviens pas à toujours de notre iniquité. Voici, regarde, nous te prions, nous sommes tous ton peuple.
\VS{9}Tes villes saintes sont devenues un désert ; Sion est devenue un désert, et Jérusalem une désolation.
\VS{10}Notre maison sainte et glorieuse, où nos pères te louaient, a été brûlée par le feu ; tout ce que nous avions de précieux a été dévasté.
\VS{11}Après cela, ô Yahweh, ne te retiendras-tu pas ? Ne cesseras-tu pas, et nous affligeras-tu à l'excès ?
\Chap{65}
\TextTitle{Réponse de Yahweh}
\VerseOne{}Je me suis fait recherché de ceux qui ne me demandaient point, et je me suis laissé trouver par ceux qui ne me cherchaient pas\FTNT{Mt. 7:7 ; Lu. 11:9.} ; j'ai dit à la nation qui ne s'appelait pas de mon Nom : Me voici, me voici !
\VS{2}J'ai tendu mes mains tous les jours vers un peuple rebelle, à ceux qui marche dans une mauvaise voie, au gré de ses pensées ;
\VS{3}vers un peuple qui m'irrite continuellement en face, qui sacrifie dans les jardins, et qui fait des parfums sur les autels de briques,
\VS{4}qui habite les sépulcres et passe la nuit dans les lieux désolés, qui mangent la chair de porc, et ayant dans ses vases le jus des choses abominables.
\VS{5}Qui dit : Retire-toi, ne m'approche pas, car je suis plus saint que toi ! Ceux-là sont une fumée dans mes narines, un feu ardent tout le jour.
\VS{6}Voici, ceci est écrit devant moi, je ne me tairai point, mais je leur ferai porter la peine, oui je leur ferai porter la peine
\VS{7}de vos iniquités, dit Yahweh, et les iniquités de vos pères ensemble, qui ont brûlé de l'encens sur les montagnes, et qui m'ont blasphémé sur les collines ; c'est pourquoi je leur mesurerai aussi dans leur sein le salaire de ce qu'ils ont fait au commencement.
\VS{8}Ainsi parle Yahweh : Comme quand on trouve du vin dans une grappe, on dit : Ne la détruis pas, car il y a là une bénédiction ! J'agirai de même à cause de mes serviteurs, afin de ne pas tous les détruire.
\VS{9}Je ferai sortir de Jacob une postérité, et de Juda celui qui héritera de mes montagnes ; et mes élus hériteront le pays, et mes serviteurs y habiteront.
\VS{10}Et Saron servira de pâturage au menu bétail, et la vallée d'Acor sera le gîte du gros bétail, pour mon peuple qui m'aura recherché.
\VS{11}Mais vous, qui abandonnez Yahweh et qui oubliez ma montagne sainte, qui dressez la table pour Gad\FTNT{Gad : Dieu de la fortune.}, et qui remplissez une coupe pour Meni\FTNT{Meni : divinité païenne assimilée à la lune et dont le nom signifie « destin, sort ou fortune ».},
\VS{12}je vous destine aussi à l'épée, et vous serez tous courbés pour être égorgés ; parce que j'ai appelé, et vous n'avez point répondu ; j'ai parlé, et vous n'avez point écouté ; mais vous avez fait ce qui me déplaît, et vous avez choisi les choses auxquelles je ne prends pas plaisir.
\VS{13}C'est pourquoi, ainsi parle le Seigneur, Yahweh : Voici, mes serviteurs mangeront, et vous aurez faim ; voici, mes serviteurs boiront, et vous aurez soif ; voici mes serviteurs se réjouiront, et vous serez honteux.
\VS{14}Voici, mes serviteurs se réjouiront avec chants de triomphe pour la joie qu'ils auront au cœur ; mais vous, vous crierez pour la douleur que vous aurez au cœur, et vous crierez à cause de l'accablement de votre esprit.
\VS{15}Et vous laisserez votre nom à mes élus comme malédiction ; et le Seigneur Yahweh vous fera mourir ; et il donnera à ses serviteurs un autre nom.
\VS{16}Celui qui se bénira sur la terre, se bénira par le Dieu de vérité ; et celui qui jurera sur la terre jurera par le Dieu de vérité ; car les détresses du passé seront oubliées, et même elles seront cachées devant mes yeux.
\TextTitle{De nouveaux cieux et une nouvelle terre}
\VS{17}Car voici, je vais créer de nouveaux cieux et une nouvelle terre\FTNT{Es. 66:22 ; 2 Pi. 3:13 ; Ap. 21:1.} ; et on ne se souviendra plus des choses précédentes, elles ne reviendront plus au cœur.
\VS{18}Réjouissez-vous plutôt et soyez à toujours dans l'allégresse, à cause de ce que je vais créer ; car voici je vais créer Jérusalem pour n'être que joie, et son peuple pour n'être qu'allégresse.
\VS{19}Je ferai de Jérusalem mon allégresse, et de mon peuple ma joie ; on n'y entendra plus le bruit des pleurs et le bruit des clameurs.
\VS{20}Il n'y aura plus désormais ni nourrisson ni vieillard qui n'accomplissent leurs jours ; car celui qui mourra âgé de cent ans sera encore jeune ; mais le pécheur âgé de cent ans sera maudit.
\VS{21}Ils bâtiront des maisons et y habiteront ; ils planteront des vignes et ils en mangeront le fruit.
\VS{22}Ils ne bâtiront pas des maisons pour qu'un autre y habite ; ils ne planteront pas des vignes pour qu'un autre en mange le fruit ; car les jours de mon peuple seront comme les jours des arbres ; et mes élus jouiront de l'œuvre de leurs mains.
\VS{23}Ils ne travailleront plus en vain, et ils n'engendreront plus des enfants pour être exposés à la frayeur ; car ils seront la postérité des bénis de Yahweh, et ceux qui sortiront d'eux seront avec eux.
\VS{24}Et il arrivera qu'avant qu'ils crient, je les exaucerai ; et lorsqu'encore ils parleront, je les aurai déjà entendus.
\VS{25}Le loup et l'agneau paîtront ensemble, le lion comme le bœuf mangeront de la paille, et la poussière sera la nourriture du serpent\FTNT{Es. 2:4 ; Es. 11:6-7.}. On ne nuira point et on ne fera aucun dommage sur toute ma montagne sainte, dit Yahweh.
\Chap{66}
\TextTitle{Yahweh réprouve l'hypocrisie et agrée ceux qui le craignent}
\VerseOne{}Ainsi parle Yahweh : Le ciel est mon trône, et la terre est le marchepied de mes pieds\FTNT{Mt. 5:34-35 ; Ac. 7:49.}. Quelle maison me bâtiriez-vous, et quel serait le lieu de mon repos ?
\VS{2}Car ma main a fait toutes ces choses, et c'est par moi que toutes ces choses ont eu leur être, dit Yahweh. Mais à qui regarderai-je ? A celui qui est affligé, qui a l'esprit abattu, et qui tremble à ma parole.
\VS{3}Celui qui égorge un bœuf est comme celui qui tuerait un homme ; celui qui sacrifie une brebis est comme celui qui romprait la nuque à un chien ; celui qui présente une offrande est comme celui qui offrirait le sang d'un pourceau ; celui qui fait un parfum d'encens est comme celui qui bénirait une idole ; tous ceux-là ont choisi leurs voies, et leur âme trouve du plaisir dans leurs abominations.
\VS{4}Moi aussi je ferai attention à leurs tromperies, et je ferai venir sur eux les choses qu'ils craignent ; parce que j'ai appelé, et personne n'a répondu, parce que j'ai parlé, et qu'ils n'ont point écouté ; mais ils ont fait ce qui est mal à mes yeux, et ils ont choisi les choses auxquelles je ne prends pas de plaisir. 
\VS{5}Ecoutez la parole de Yahweh, vous qui tremblez à sa parole ; vos frères, qui vous haïssent et qui vous repoussent comme une chose abominable, à cause de mon Nom disent : Que Yahweh montre sa gloire ! Il sera donc vu à votre joie mais eux seront honteux. 
\VS{6}Un son éclatant sort de la ville, un son sort du temple, le son de Yahweh, qui rend à ses ennemis selon leurs œuvres.
\TextTitle{Israël renaît en un jour}
\VS{7}Elle a enfanté, avant d'éprouver les douleurs de l'enfantement ; elle a donné naissance à un enfant mâle, avant que les souffrances lui viennent.
\VS{8}Qui a jamais entendu une telle chose ? Qui en a jamais vu de semblable ? Ferait-on qu'un pays naisse en un jour ? Ou une nation naîtrait-elle d'un seul coup\FTNT{Cette prophétie fait allusion à la création de l'Etat d'Israël le 14 mai 1948.} ? Car dès que Sion a été en travail, elle a enfanté ses enfants !
\VS{9}Moi qui fais enfanter les autres, ne ferais-je point enfanter Sion ? Dit Yahweh. Moi qui donne de la postérité aux autres, l'empêcherais-je d'enfanter ? Dit ton Dieu.
\TextTitle{Réjouissance à Jérusalem et consolation}
\VS{10}Réjouissez-vous avec Jérusalem, faites d'elle le sujet de votre allégresse, vous tous qui l'aimez ; vous tous qui menez deuil sur elle, réjouissez-vous avec elle d'une grande joie ;
\VS{11}afin que vous soyez allaités et rassasiés de la mamelle de ses consolations, afin que vous suciez le lait et que vous jouissiez à plaisir de la plénitude de sa gloire.
\VS{12}Car ainsi parle Yahweh : Voici, je ferai couler vers elle la paix comme un fleuve, et la gloire des nations comme un torrent débordé, et vous serez allaités, vous serez portés sur les côtés et caressés sur les genoux.
\VS{13}Je vous consolerai pour vous apaiser, comme quelqu'un que sa mère caresse pour l'apaiser, vous serez consolés dans Jérusalem.
\VS{14}Vous le verrez et votre cœur se réjouira, et vos os germeront comme l'herbe ; et la main de Yahweh sera connue de ses serviteurs ; mais il sera indigné contre ses ennemis.
\TextTitle{Jugement de Yahweh}
\VS{15}Car voici, Yahweh viendra avec le feu, et ses chars seront comme la tempête ; afin qu'il tourne sa colère en fureur, et sa menace en flamme de feu.
\VS{16}Car Yahweh exercera jugement contre toute chair par le feu et avec son épée ; et le nombre de ceux qui seront mis à mort par Yahweh sera grand.
\VS{17}Ceux qui se sanctifient et se purifient au milieu des jardins, l'un après l'autre, qui mangent de la chair de porc et des choses abominables, comme des souris, seront ensemble consumés, dit Yahweh.
\VS{18}Mais pour moi, voyant leurs œuvres et leurs pensées, le temps est venu de rassembler toutes les nations et les langues ; ils viendront et verront ma gloire.
\TextTitle{Toutes les nations adoreront Yahweh}
\VS{19}Car je mettrai un signe en eux, et j'enverrai ceux d'entre eux qui seront réchappés, vers les nations, à Tarsis, à Pul, à Lud, gens tirant de l'arc, à Tubal et à Javan, et vers les îles lointaines, qui n'ont point entendu ma renommée, et qui n'ont pas vu ma gloire ; et ils annonceront ma gloire parmi les nations.
\VS{20}Et ils amèneront tous vos frères d'entre toutes les nations, sur des chevaux, sur des chars et dans des litières, sur des mulets et sur des dromadaires, en offrande à Yahweh, à la montagne sainte, à Jérusalem, dit Yahweh, comme lorsque les enfants d'Israël apportent l'offrande dans un vase pur, à la maison de Yahweh.
\VS{21}Et même je prendrai aussi parmi eux des sacrificateurs, des Lévites, dit Yahweh.
\VS{22}Car comme les nouveaux cieux et la nouvelle terre que je vais faire subsisteront devant moi, dit Yahweh, ainsi subsistera votre postérité et votre nom.
\VS{23}Et il arrivera que de nouvelle lune en nouvelle lune, et de sabbat en sabbat, toute chair viendra se prosterner devant ma face, dit Yahweh.
\VS{24}Et quand ils sortiront dehors, ils verront les cadavres des hommes qui se sont rebellés contre moi ; car leur ver ne mourra point, et leur feu ne s'éteindra point\FTNT{Mc. 9:48.} ; et ils seront méprisés de tout le monde.
\PPE{}
\end{multicols}

%\clearpage\ShortTitle{Jérémie}\BookTitle{Jérémie}\BFont
\noindent\hrulefill
{\footnotesize
\textit{
\bigskip
{\centering{}
\\Auteur : Jérémie
\\(Heb. : Yirmeyah)
\\Signification : Celui que Yahweh a désigné
\\Thème : Avertissements et jugements
\\Date de rédaction : 7\up{ème} siècle av J.C\\}
}
%\bigskip
\textit{
\\Issu d'une famille de sacrificateurs, Jérémie fut appelé dès son plus jeune âge au service de Yahweh et exerça un ministère prophétique avant et pendant les premières années de déportation. Outre son message à Israël et aux nations, le livre de Jérémie révèle sa personnalité. On découvre alors que l'opposition de ses pairs fut l'une de ses expériences les plus douloureuses. En effet, ce récit raconte ses combats contre les faux prophètes et met en évidence les signes accompagnant les prophètes authentiques, à savoir la souffrance, la solitude, l'incompréhension et le rejet.
%\bigskip
\\Son message annonçait le jugement imminent de Dieu et invitait le peuple à la repentance pour éviter le châtiment de Yahweh. Après la chute de Jérusalem, alors que Nebucadnetsar lui avait laissé le choix, Jérémie décida de rester avec les plus pauvres plutôt que de partir pour Babylone. Cependant, des Israélites décidèrent de s'expatrier en Egypte et l'entraînèrent avec eux de force. En terre étrangère, Jérémie continua de porter le fardeau de son peuple, l'exhortant à réformer ses voies. 
%\bigskip
\\Parmi les prophéties de Jérémie, figure le retour du peuple d'Israël sur la terre promise avant la seconde venue de Christ.\bigskip
}
}
\par\nobreak\noindent\hrulefill
\begin{multicols}{2}
\Chap{1}
\TextTitle{Yahweh appelle Jérémie à son service}
\VerseOne{}Les Paroles de Jérémie, fils de Hilkija, d'entre les sacrificateurs qui étaient à Anathoth, dans le pays de Benjamin;
\VS{2}auquel fut adressée la parole de Yahweh aux jours de Josias, fils d'Amon, roi de Juda, la treizième année de son règne,
\VS{3}laquelle lui fut aussi adressée aux jours de Jojakim, fils de Josias, roi de Juda, jusqu'à la fin de la onzième année de Sédécias, fils de Josias, roi de Juda; savoir jusqu'au temps où  Jérusalem fut transportée, ce qui arriva au cinquième mois.
\VS{4}La parole de Yahweh me fut adressée, en disant :
\VS{5}Avant que je t'aie formé dans le ventre de ta mère, je te connaissais, et avant que tu sois sorti de son sein, je t'avais consacré, je t'avais établi prophète pour les nations\FTNT{Es. 49:5 ; Ga. 1:15.}.
\VS{6}Je répondis : Ah ! Seigneur Yahweh ! Voici, je ne sais pas parler, car je suis un enfant\FTNT{Ex. 4:10-11.}.
\VS{7}Et Yahweh me dit : Ne dis pas : Je suis un enfant. Car tu iras partout où je t'enverrai, et tu diras tout ce que je t'ordonnerai.
\VS{8}Ne crains pas de te montrer devant eux, car je suis avec toi pour te délivrer, dit Yahweh.
\VS{9}Puis Yahweh avança sa main et toucha ma bouche ; et Yahweh me dit : Voici, je mets mes paroles dans ta bouche.
\VS{10}Regarde, je t'établis aujourd'hui sur les nations et sur les royaumes, pour que tu arraches et que tu démolisses, pour que tu ruines et que tu détruises, pour que tu bâtisses et que tu plantes\FTNT{Jérémie devait d'abord arracher, démolir, ruiner et détruire avant de bâtir et de planter. Il y avait dans le temple de Jérusalem les autels de Baal et le pieu d'Asherah (2 R. 21). De même, avant de planter la Parole de Dieu qui est une semence plantée dans les cœurs (Mc. 4 : 3-17), il est nécessaire au préalable d'arracher et de renverser les fausses doctrines et le péché en les dénonçant.}.
\TextTitle{Yahweh confirme la mission de Jérémie et l'établit sur Juda}
\VS{11}Puis la parole de Yahweh me fut adressée, en disant : Que vois-tu, Jérémie ? Et je répondis : Je vois une branche d'amandier.
\VS{12}Et Yahweh me dit : Tu as bien vu ; car je me hâte d'exécuter ma parole.
\VS{13}La parole de Yahweh me fut adressée pour la seconde fois, en disant : Que vois-tu ? Et je répondis : Je vois un pot bouillant dont le devant est tourné vers le nord.
\VS{14}Et Yahweh me dit : le mal se découvrira du côté du nord sur tous les habitants de ce pays-ci.
\VS{15}Car voici, je vais appeler toutes les familles des royaumes du nord, dit Yahweh ; elles viendront et mettront chacune leur trône à l'entrée des portes de Jérusalem, contre toutes ses murailles à l'entour, et contre toutes les villes de Juda.
\VS{16}Et je prononcerai mes jugements contre eux, à cause de toute leur méchanceté, par laquelle ils m'ont délaissé, et ont fait des parfums à d'autres dieux, et se sont prosternés devant l'ouvrage de leurs mains. 
\VS{17}Toi donc, ceins tes reins, lève-toi, et dis-leur tout ce que je t'ordonnerai. Ne crains pas de te montrer devant eux, de peur que je ne te mette en pièces en leur présence.
\VS{18}Car voici, je t'établis aujourd'hui sur tout le pays comme une ville forte, une colonne de fer, et un mur d'airain, contre les rois de Juda, contre les chefs du pays, contre ses sacrificateurs, et contre le peuple du pays.
\VS{19}Et ils combattront contre toi, mais ils ne seront pas plus forts que toi ; car je suis avec toi, dit Yahweh, pour te délivrer.
\Chap{2}
\TextTitle{Yahweh dénonce l'attitude d'Israël et l'avertit}
\VerseOne{}La parole de Yahweh me fut adressée, en disant :
\VS{2}Va et crie aux oreilles de Jérusalem, et dis : Ainsi parle Yahweh : Je me souviens de la fidélité de ta jeunesse, de l'amour de tes fiançailles, quand tu me suivais au désert, dans une terre qu'on n'ensemence pas. 
\VS{3}Israël était une chose sainte à Yahweh, il était les prémices de son revenu\FTNT{Lé. 23:20 ; Pr. 3:9 ; Né. 10:35.}; tous ceux qui le dévoraient étaient coupables, il leur en arrivait du mal dit Yahweh.
\VS{4}Ecoutez la parole de Yahweh, maison de Jacob, et vous toutes, familles de la maison d'Israël !
\VS{5}Ainsi parle Yahweh : Quelle iniquité vos pères ont-ils trouvée en moi, pour qu’ils se soient éloignés de moi, et qu’ils aient marché après la vanité et soient devenus vains ?
\VS{6}Ils n'ont pas dit : Où est Yahweh qui nous a fait remonter du pays d'Egypte, qui nous a conduits par un désert, par un pays de landes et montagneux, par un pays aride et d'ombre de mort, par un pays où aucun homme n'avait passé, et où personne n'avait habité ? 
\VS{7}Je vous ai fait entrer dans un pays de verger, pour que vous en mangiez les fruits et les biens ; mais sitôt vous y êtes entrés, vous avez souillé mon pays, et vous avez rendu abominable mon héritage.
\VS{8}Les sacrificateurs n'ont pas dit : Où est Yahweh ? Les dépositaires de la loi ne m'ont pas connu, les pasteurs se sont révoltés contre moi, les prophètes ont prophétisé par Baal\FTNT{Baal. Voir Jg. 2:13.}, et sont allés après ce qui n'est d'aucun profit.
\VS{9}A cause de cela, je veux encore contester avec vous, dit Yahweh, je veux contester avec les fils de vos fils.
\VS{10}Passez par les îles de Kittim et voyez ! Envoyez quelqu'un à Kédar ; observez bien, et voyez s'il n'y a rien de semblable !
\VS{11}Y a-t-il une nation qui change ses dieux, quoiqu'ils ne soient pas des dieux ? Et mon peuple a changé sa gloire contre ce qui n'est d'aucun profit\FTNT{Ro. 1:23.} !
\VS{12}Cieux, soyez étonnés de cela ; frémissez d'horreur et soyez stupéfaits ! dit Yahweh.
\VS{13}Car mon peuple a commis doublement le mal : Ils m'ont abandonné, moi qui suis la source d'eaux vives\FTNT{Yahweh est la Source d'eaux vives. Jésus-Christ se présente aussi comme la Source d'eau vive (Jn. 4:13-14 ; Ap. 21:6).}, pour se creuser des citernes, des citernes crevassées qui ne peuvent pas retenir l'eau.
\VS{14}Israël est-il un esclave, ou un esclave né dans la maison ? Pourquoi donc est-il mis au pillage ?
\VS{15}Les lionceaux rugissent, poussent leurs cris contre lui, et ils mettent son pays en désolation ; ses villes sont brûlées, de sorte que personne n'y habite.
\VS{16}Même les fils de Noph et de Tachpanès te casseront le sommet de la tête.
\VS{17}Cela ne t'arrive-t-il pas parce que tu as abandonné Yahweh, ton Dieu, à l'époque où il te conduisait par le chemin ?
\VS{18}Et maintenant, qu'as-tu à faire d'aller en Egypte, pour boire l'eau du Schichor\FTNT{Schichor : Sombre, noir, boueux. Le Nil, une rivière ou un canal affluent du fleuve. Les Israélites préféraient ces eaux à Yahweh.} ? Qu'as-tu à faire d'aller en Assyrie, pour boire l'eau du fleuve ?
\VS{19}Ta méchanceté te châtiera, et tes débauches te jugeront, tu sauras et tu verras que c'est une chose mauvaise et amère d'abandonner Yahweh, ton Dieu, et de n'avoir de moi aucune crainte, dit le Seigneur, Yahweh des armées.
\VS{20}Tu as dès longtemps brisé ton joug, rompu tes liens, et tu as dit : Je ne veux plus être dans la servitude ! Mais sur toute haute colline et sous tout arbre vert tu t'es incliné, tu t'es prostitué.
\VS{21}Je t'avais moi-même plantée comme une vigne exquise, dont tout le plant était franc ; comment t'es-tu changée en sarments d'une vigne étrangère ?
\VS{22}Quand tu te laverais avec du nitre, et que tu prendrais beaucoup de savon, ton iniquité resterait encore marquée devant moi, dit le Seigneur, Yahweh.
\VS{23}Comment dirais-tu : Je ne me suis pas souillée, je ne suis pas allée après les Baals ? Regarde tes pas dans la vallée, reconnais ce que tu as fait, dromadaire à la course légère et vagabonde !
\VS{24}Anesse sauvage, accoutumée au désert, humant le vent à son plaisir. Qui l'arrêtera dans son ardeur ? Tous ceux qui la cherchent n'ont pas à se fatiguer ; ils la trouvent pendant son mois.
\VS{25}Garde ton pied de se déchausser, ton gosier d'avoir soif ! Mais tu dis : C'est en vain, non ! Car j'aime les dieux étrangers, et j'irai après eux.
\VS{26}Comme un voleur est confus quand il est surpris, ainsi seront confus ceux de la maison d'Israël, eux, leurs rois, leurs chefs, leurs sacrificateurs et leurs prophètes.
\VS{27}Ils disent au bois : Tu es mon père ! Et à la pierre : Tu m'as enfanté ! Car ils me tournent le dos, et non la face. Et ils disent dans le temps de leur malheur : Lève-toi, et sauve-nous !
\VS{28}Où donc sont tes dieux que tu t'es faits ? Qu'ils se lèvent, s'ils peuvent te sauver au temps de ton malheur ! Car tu as autant de dieux que de villes, ô Juda !
\VS{29}Pourquoi contesteriez-vous avec moi ? Vous vous êtes tous rebellés contre moi, dit Yahweh.
\VS{30}En vain ai-je frappé vos fils ; ils n'ont pas reçu d'instruction ; votre épée a dévoré vos prophètes comme un lion destructeur.
\VS{31}Hommes de cette génération, considérez la parole de Yahweh ! Ai-je été un désert pour Israël, ou un pays de ténèbres ? Pourquoi mon peuple dit-il : Nous sommes libres, nous ne viendrons plus à toi ?
\VS{32}La vierge oublie-t-elle ses ornements, la fiancée sa ceinture ? Mais mon peuple m'a oublié depuis des jours sans nombre.
\VS{33}Comme tu es habile dans tes voies pour chercher ce que tu aimes ! C'est pourquoi aussi tu accoutumes tes voies aux crimes.
\VS{34}Même sur les pans de ta robe se trouve le sang des pauvres, des innocents que tu n'as pas trouvés en effraction.
\VS{35}Malgré cela, tu dis : Oui, je suis innocent ! Certainement sa colère s'est détournée de moi ! Voici, je vais entrer en jugement avec toi, sur ce que tu as dit : Je n'ai pas péché.
\VS{36}Pourquoi tant te précipiter pour changer ton chemin ? Tu auras autant de confusion de l'Egypte que tu en as eu de l'Assyrie.
\VS{37}Tu sortiras même d'ici, ayant tes mains sur la tête ; car Yahweh rejette ceux en qui tu te confies, et tu n'auras aucune prospérité par eux.
\Chap{3}
\TextTitle{Israël comparé à une prostituée}
\VerseOne{}Il dit : Si un homme répudie sa femme, qu'elle le quitte et se joigne à un autre, cet homme retourne-t-il encore vers elle\FTNT{Lé. 21:7 ; De. 24:2.} ? Le pays même n'en serait-il pas entièrement souillé ? Or toi, tu t'es prostituée à plusieurs amants, et tu reviendrais à moi ! dit Yahweh.
\VS{2}Lève tes yeux vers les lieux élevés et regarde ! Où ne t'es-tu pas prostituée ! Tu te tenais sur les chemins, comme un Arabe dans le désert, et tu as souillé le pays par tes prostitutions et par ta méchanceté.
\VS{3}Aussi les pluies ont été retenues, et il n'y a pas eu de pluie de l'arrière-saison ; mais tu as eu le front d'une femme prostituée, tu n'as pas voulu avoir honte.
\VS{4}Maintenant, n'est-ce pas ? Tu cries vers moi : Mon père ! Tu as été l'ami de ma jeunesse !
\VS{5}Gardera-t-il à toujours sa colère ? La conservera-t-il à jamais\FTNT{Es. 57:16 ; Ps.103:9.} ? Voici, tu as ainsi parlé, tu as fait ces maux-là autant que tu as pu.
\TextTitle{Yahweh appelle Israël à la repentance}
\VS{6}Yahweh me dit au temps du roi Josias : As-tu vu ce qu'a fait Israël, l'infidèle ? Elle est allée sur toute haute colline et sous tout arbre vert, et elle s'y est prostituée.
\VS{7}Je disais : Après avoir fait toutes ces choses, elle reviendra à moi. Mais elle n'est pas revenue. Et sa sœur Juda, la perfide, l'a vu.
\VS{8}Quoique j'aie répudié Israël, l'infidèle, à cause de tous ses adultères, et que je lui aie donné sa lettre de divorce, j'ai vu que la perfide Juda, sa sœur, n'a pas eu de crainte, mais elle s'en est allée et s'est aussi prostituée.
\VS{9}Par le bruit de sa prostitution, elle a souillé le pays, elle a commis un adultère avec la pierre et le bois.
\VS{10}Malgré tout cela, sa sœur Juda, la perfide, n'est pas revenue à moi de tout son cœur ; c'est avec fausseté qu'elle l'a fait, dit Yahweh.
\VS{11}Et Yahweh me dit : Israël, l'infidèle, se montre plus juste que Juda, la perfide.
\VS{12}Va, crie ces paroles vers le nord, et dis : Reviens, Israël, l'infidèle, dit Yahweh. Je ne jetterai pas sur vous un regard sévère ; car je suis miséricordieux, dit Yahweh, je ne garde pas ma colère à toujours.
\VS{13}Reconnais seulement ton iniquité, que tu t'es rebellée contre Yahweh, ton Dieu, que tu as tourné çà et là tes pas vers les étrangers, sous tout arbre vert, et que tu n'as pas écouté ma voix, dit Yahweh.
\VS{14}Fils rebelles, convertissez-vous, dit Yahweh, car je suis votre maître. Je vous prendrai, un d'une ville, deux d'une famille, et je vous ferai entrer dans Sion.
\VS{15}Je vous donnerai des pasteurs selon mon cœur, qui vous paîtront avec intelligence et avec sagesse\FTNT{Jé. 23:5.}.
\VS{16}Lorsque vous aurez multiplié et fructifié dans le pays, en ces jours-là, dit Yahweh, on ne parlera plus de l'arche de l'alliance de Yahweh, elle ne viendra plus à la pensée ; on ne s'en souviendra plus, on ne s'apercevra plus de son absence, et l'on n'en fera pas une autre.
\VS{17}En ce temps-là, on appellera Jérusalem le trône de Yahweh ; toutes les nations s'assembleront à Jérusalem, au Nom de Yahweh, et elles ne marcheront plus suivant les penchants de leur mauvais cœur.
\VS{18}En ces jours-là, la maison de Juda marchera avec la maison d'Israël ; elles viendront ensemble du pays du nord au pays que j'ai donné en héritage à vos pères.
\VS{19}Je disais : Comment te mettrai-je parmi mes fils et te donnerai-je un pays désirable, le plus bel héritage des armées des nations ? Je disais : Tu m'appelleras : Mon père ! Et tu ne te détourneras pas de moi.
\VS{20}Mais, comme une femme est infidèle à son compagnon, ainsi vous m'avez été infidèles, maison d'Israël, dit Yahweh.
\VS{21}Une voix se fait entendre sur les lieux élevés ; ce sont les pleurs, les supplications des fils d'Israël ; car ils ont perverti leur voie, ils ont oublié Yahweh, leur Dieu.
\VS{22}Fils rebelles, convertissez-vous, je guérirai vos infidélités. Nous voici, nous venons à toi, car tu es Yahweh, notre Dieu.
\VS{23}Certainement, on s'attend en vain aux collines et à la multitude des montagnes ; mais c'est en Yahweh, notre Dieu, qu'est la délivrance d'Israël.
\VS{24}Car la honte a dévoré dès notre jeunesse le travail de nos pères, leurs brebis et leurs bœufs, leurs fils et leurs filles.
\VS{25}Nous serons gisants dans notre honte, et notre ignominie nous couvrira ; parce que nous avons péché contre Yahweh, notre Dieu, nous et nos pères, dès notre jeunesse jusqu'à ce jour, et nous n'avons pas obéi à la voix de Yahweh, notre Dieu.
\Chap{4}
\TextTitle{Prophétie sur l'invasion du pays}
\VerseOne{}Israël, si tu reviens, dit Yahweh, si tu reviens à moi, si tu ôtes tes abominations de devant moi, tu ne seras plus errant ça et là.
\VS{2}Alors tu jureras avec vérité, avec droiture et avec justice : Yahweh est vivant ! Et les nations seront bénies en lui, et se glorifieront en lui.
\VS{3}Car ainsi parle Yahweh aux hommes de Juda et de Jérusalem : Labourez pour vous une terre arable et ne semez pas parmi les épines\FTNT{Mt. 13:7 ; Mt. 13:22 ; Mc. 4:7 ; Mc. 4:18 ; Lu. 8:14.}.
\VS{4}Hommes de Juda, et vous habitants de Jérusalem, circoncisez-vous pour Yahweh, circoncisez vos cœurs\FTNT{Ro. 2:29.}, de peur que ma fureur ne sorte comme un feu et qu'elle ne brûle sans qu'on puisse l'éteindre, à cause de la méchanceté de vos actions.
\VS{5}Annoncez en Juda, publiez dans Jérusalem, et dites : Sonnez du shofar dans le pays ! Criez à pleine voix et dites : Assemblez-vous et nous entrerons dans les villes fortes !
\VS{6}Elevez une bannière vers Sion, fuyez, ne vous arrêtez pas ! Car je fais venir du nord le malheur et une grande calamité.
\VS{7}Le lion\FTNT{Lion est ici une allusion à Nebucadnetsar, roi de Babylone. Voir 2 R. 24 et 25 ; Da. 7:4.} est sorti de la caverne, le destructeur des nations est en marche, il est sorti de son lieu, pour réduire ton pays en désert ; tes villes seront ruinées, il n'y aura personne pour y habiter.
\VS{8}C'est pourquoi ceignez-vous de sacs, lamentez-vous et gémissez ; car l'ardeur de la colère de Yahweh ne se détourne pas de nous.
\VS{9}Et il arrivera ce jour-là, dit Yahweh, que le cœur du roi et le cœur des chefs seront épouvantés et que les sacrificateurs seront étonnés, et que les prophètes seront stupéfaits.
\VS{10}C'est pourquoi je dis : Ah ! Seigneur Yahweh ! Oui certainement tu as abusé ce peuple et Jérusalem, en disant : Vous aurez la paix ! Et cependant l'épée est venue jusqu'à l'âme.
\VS{11}En ce temps-là on dira à ce peuple et à Jérusalem : Un vent brûlant souffle des lieux élevés du désert sur le chemin de la fille de mon peuple, non pas pour vanner ni pour nettoyer.
\VS{12}C'est un vent impétueux qui vient de là jusqu'à moi et je leur ferai maintenant leur procès. 
\VS{13}Voici, il monte comme des nuées ; ses chars sont comme un tourbillon, ses chevaux sont plus légers que les aigles. Malheur à nous, car nous sommes détruits !
\VS{14}Jérusalem, lave ton cœur du mal afin que tu sois délivrées ! Jusqu'à quand séjourneront-tu au-dedans de toi les pensées de ton injustice ?
\VS{15}Car une voix apporte des nouvelles de Dan, elle publie depuis la montagne d'Ephraïm le tourment.
\VS{16}Rappelez-le aux nations, faites-le entendre à Jérusalem : Des observateurs viennent d'un pays éloigné ; ils poussent des cris contre les villes de Juda.
\VS{17}Ils se sont mis tout autour d'elle comme ceux qui gardent un champ, parce qu'elle s'est rebellée contre moi, dit Yahweh.
\VS{18}Ta conduite et tes actions t'ont produit ces choses, telle a été ta méchanceté, parce que cela a été une chose amère, certainement elle t'atteindra jusqu'à ton cœur.
\VS{19}Mes entrailles ! Mes entrailles : Je suis dans la douleur au-dedans de mon cœur, mon cœur bat, je ne puis me taire ; car, ô mon âme, tu entends le son du shofar, la clameur de la guerre.
\VS{20}On annonce brèche sur brèche, car tout le pays est dévasté ; mes tentes sont détruites tout à coup, mes pavillons en un moment.
\VS{21}Jusqu'à quand verrai-je la bannière et entendrai-je le son du shofar ?
\VS{22}Car mon peuple est insensé ; ils ne m'ont pas reconnu, ce sont des enfants insensés qui n'ont pas d'intelligence ; ils sont habiles pour faire le mal, et ils ne savent pas faire le bien.
\VS{23}Je regarde la terre, et voici, elle est informe et vide\FTNT{Dieu n'a pas créé la terre informe et vide, mais elle l'est devenue à cause du péché. Voir le commentaire en Gn. 1:2.} ; les cieux et leur lumière ne sont plus.
\VS{24}Je regarde les montagnes, et voici, elles sont ébranlées ; et toutes les collines sont renversées.
\VS{25}Je regarde, et voici, il n'y a pas un seul homme et tous les oiseaux des cieux se sont enfuis.
\VS{26}Je regarde, et voici, le Carmel est un désert ; et toutes ses villes sont détruites, devant Yahweh, devant l'ardeur de sa colère.
\VS{27}Car ainsi parle Yahweh : Tout le pays sera dévasté, mais je ne ferai pas une entière destruction.
\VS{28}C'est pourquoi le pays mènera deuil et les cieux en haut seront obscurcis, parce que je l'ai dit, je l'ai résolu, et je ne m'en repentirai pas et je le révoquerai pas.
\VS{29}Toute la ville s'enfuit à cause du bruit des cavaliers et des archers ; ils entrent dans les bois fourrés et montent sur les rochers ; toute la ville est abandonnée, et aucun homme n'y habite.
\VS{30}Et quand tu auras été détruite que fais-tu ? Quoique tu te revêtes de pourpre, que tu te pares d'ornements d'or, et que tu bordes tes yeux de fard, tu t'embellis en vain: tes amants t'ont méprisée; c'est ta vie qu'ils cherchent.
\VS{31}Car j'entends un cri comme celui d'une femme qui est en travail, et une angoisse comme celle d'une femme qui est en travail de son premier-né ; c'est le cri de la fille de Sion ; elle soupire, elle étend ses mains, en disant : Malheur maintenant à moi, car mon âme a défailli à cause des meurtriers. 
\Chap{5}
\TextTitle{Raisons du jugement de Yahweh}
\VerseOne{}Parcourez les rues de Jérusalem et regardez maintenant, sachez et cherchez dans les places, si vous y trouvez un homme de bien, s'il y a quelqu'un qui fasse ce qui est droit, qui cherche la vérité, et je pardonne à Jérusalem\FTNT{Es. 59:15 ; Mi. 7:2 ; Pr. 20:6.}.
\VS{2}Même s'ils disent : Yahweh est vivant ! En cela, ils jurent faussement.
\VS{3}Yahweh, tes yeux ne regardent-ils pas à la fidélité ? Tu les frappes, et ils ne sentent pas de douleur ; tu les consumes, et ils refusent de recevoir l'instruction ; ils endurcissent leurs faces plus qu'un rocher, ils refusent de se convertir.
\VS{4}Je disais : Certainement ce ne sont que les plus petits ; ils se montrent insensés parce qu'ils ne connaissent pas la voie de Yahweh, le droit de leur Dieu.
\VS{5}J'irai donc vers les plus grands, et je leur parlerai ; car cela connaissent la voie de Yahweh, le droit de leur Dieu ; mais ceux-là même ont brisés le joug et ont rompu les liens.
\VS{6}C'est pourquoi le lion de la forêt les tue, le loup du soir les détruit, et le léopard est aux aguets contre leurs villes ; quiconque en sortira sera déchiré ; car leurs transgressions sont nombreuses, et leurs infidélités se sont renforcées.
\VS{7}Comment te pardonnerais-je en cela ? Tes fils m'ont abandonné, et ils jurent par ce qui ne sont pas dieux. Je les ai rassasiés, mais ils commettent l'adultère et ils se pressent en foule dans la maison de la prostituée.
\VS{8}Ils sont comme des chevaux bien nourris, quand ils se lèvent le matin, chacun hennit après la femme de son prochain.
\VS{9}Ne punirais-je pas ces choses-là, dit Yahweh ? Et mon âme ne se vengerait-elle pas d'une telle nation ?
\VS{10}Montez sur ses murailles et détruisez-les, mais ne les achevez pas entièrement ! Otez ses sarments car ils ne sont pas à Yahweh\FTNT{Jn. 15:5.} !
\VS{11}Car la maison d'Israël et la maison de Juda m'ont été infidèles, dit Yahweh.
\VS{12}Ils démentent Yahweh, et disent : Cela n'arrivera pas, et le malheur ne viendra pas sur nous, nous ne verrons ni l'épée ni la famine.
\VS{13}Et les prophètes sont légers comme le vent, et la parole n'est pas en eux. Qu'il leur soit fait ainsi !
\VS{14}C'est pourquoi ainsi parle Yahweh, le Dieu des armées : Parce que vous avez prononcé cette parole-là, voici, je vais mettre mes paroles dans ta bouche pour y être comme une feu, et ce peuple sera comme le bois, et ce feu les consumera.
\VS{15}Maison d'Israël, voici, je fais venir contre vous une nation d'un pays éloigné\FTNT{Il s'agit de Babylone. Voir 2 R. 24 et 25.}, dit Yahweh, une nation puissante, une nation ancienne, une nation dont tu ne connais pas la langue, et dont tu ne comprendras pas ce qu'elle dira.
\VS{16}Son carquois est comme un sépulcre ouvert, et ils sont tous des hommes vaillants.
\VS{17}Et elle dévorera ta moisson et ton pain, que tes fils et tes filles devaient manger; elle dévorera tes brebis et tes bœufs; elle dévorera les fruits ta vigne et ton figuier et réduira à la pauvreté par l'épée tes villes fortes dans lesquelles tu te confies.
\VS{18}Toutefois en ces jours-là, dit Yahweh, je ne vous achèverai pas entièrement.
\VS{19}Et il arrivera que vous direz : Pourquoi Yahweh, notre Dieu, nous a-t-il fait toutes ces choses ? Tu leur diras ainsi : Comme vous m'avez abandonné et que vous avez servi les dieux étrangers dans votre pays, ainsi vous servirez des étrangers dans un pays qui n'est pas le vôtre.
\VS{20}Annoncez ceci dans la maison de Jacob, et publiez-le dans Juda, en disant :
\VS{21}Ecoutez maintenant ceci, peuple insensé, et qui n'avez pas d'intelligence; qui avez des yeux et ne voyez pas; et qui avez des oreilles et n'entendez pas\FTNT{Ez. 12:2 ; Jn. 12:40.}.
\VS{22}Ne me craindrez-vous pas, dit Yahweh, ne tremblerez-vous pas devant ma face ? C'est moi qui ai mis le sable pour limite à la mer, par une ordonnance perpétuelle et qui ne passera pas ; ses vagues s'agitent, mais elles sont impuissantes ; elles grondent, mais elles ne la passent pas\FTNT{Pr. 8:29 ; Job. 38:8.}.
\VS{23}Mais ce peuple-ci a un cœur indocile et rebelle ; ils reculent en arrière et s'en vont.
\VS{24}Et ils ne disent pas dans leur cœur : Craignons maintenant Yahweh, notre Dieu, qui nous donne la pluie en son temps, de la première et de l'arrière-saison, et qui nous réserve les semaines ordonnées pour la moisson.
\VS{25}Vos iniquités ont détourné ces choses, vos péchés retiennent loin de vous le bien.
\VS{26}Car il se trouve parmi mon peuple des méchants ; ils épient comme l'oiseleur qui dresse des pièges, ils tendent des filets et prennent des hommes\FTNT{Ps. 91:3 ; Ps. 124:7.}.
\VS{27}Comme la cage est remplie d'oiseaux, ainsi leurs maisons sont remplies de fraude ; c'est par ce moyen qu'ils deviennent grands et riches.
\VS{28}Ils s'engraissent, ils sont brillants ; ils surpassent les actions des méchants, ils ne jugent pas la cause, la cause de l'orphelin, et ils prospèrent ; ils ne font pas droit aux pauvres.
\VS{29}Ne punirais-je pas ces choses-là, dit Yahweh ? Et mon âme ne se vengerait-elle pas d'une telle nation ?
\VS{30}Il est arrivé dans le pays une chose étonnante et horrible :
\VS{31}C'est que les prophètes prophétisent le mensonge, et les sacrificateurs dominent par leur moyen, et mon peuple prend plaisir à cela. Que ferez-vous donc quand elle prendra fin ?
\Chap{6}
\TextTitle{Jérusalem dans la confusion}
\VerseOne{}Fils de Benjamin, fuyez par troupes du milieu de Jérusalem, et sonnez du shofar à Tekoa, et élevez un signal de feu à Beth-Hakkérem ! Car on voit venir du nord un malheur et une grande ruine.
\VS{2}La belle et la délicate, la fille de Sion, je la détruis !
\VS{3}Les pasteurs avec leurs troupeaux viennent contre elle ; ils plantent leurs tentes autour d'elle, chacun paîtra en son quartier.
\VS{4}Préparez le combat contre elle ! Levez-vous, et montons en plein midi !… Malheur à nous, car le jour décline, les ombres du soir s'étendent.
\VS{5}Levez-vous ! Montons de nuit, et ruinons ses palais !
\VS{6}Car ainsi parle Yahweh des armées : Coupez des arbres, élevez des terrasses contre Jérusalem ! C'est la ville qui doit être visitée ; tout est oppression au milieu d'elle.
\VS{7}Comme le puits fait jaillir ses eaux, ainsi elle fait jaillir sa méchanceté ; on n'entend continuellement en elle, devant moi, que violence et ruine, avec des maladies et des plaies.
\VS{8}Jérusalem, reçois l'instruction, de peur que mon âme ne se retire de toi, et que je ne fasse de toi un désert, et une terre inhabitée !
\VS{9}Ainsi parle Yahweh des armées : On grappillera entièrement comme une vigne les restes d'Israël. Remets ta main dans les paniers, comme un vendangeur.
\VS{10}A qui parlerai-je, et qui prendrai-je à témoin, pour qu'ils écoutent ? Voici, leur oreille est incirconcise, et ils ne peuvent entendre ; voici, la parole de Yahweh leur est en opprobre, ils n'y prennent point de plaisir.
\VS{11}C'est pourquoi je suis plein de la fureur de Yahweh, et je suis las de la contenir. Répands-la sur les enfants dans la rue, et sur les assemblées des jeunes gens. Car tant le mari que la femme seront pris, le vieillard et celui qui est chargé de jours.
\VS{12}Et leurs maisons passeront à d'autres, les champs et les femmes aussi, quand j'étendrai ma main sur les habitants du pays, dit Yahweh.
\VS{13}Car depuis le plus petit d'entre eux jusqu'au plus grand, chacun s'adonne au gain déshonnête, tant le prophète que le sacrificateur, tous se agissent faussement.
\VS{14}Et ils pansent à la légère la plaie de la fille de mon peuple, disant : Paix ! Paix ! et il n'y a pas de paix\FTNT{1 Th. 5:3.}.
\VS{15}Sont-ils confus d'avoir commis des abominations ? Ils n'en ont même aucune honte, et ils ne savent pas ce que c'est que de rougir ; c'est pourquoi ils tomberont parmi ceux qui tombent, ils seront renversés au temps où je les visiterai, dit Yahweh.
\VS{16}Ainsi parle Yahweh : Tenez-vous sur les chemins, regardez et enquérez-vous des sentiers des siècles passés, quel est le bon chemin ; et marchez-y, et vous trouverez le repos de vos âmes ! Et ils répondent : Nous n'y marcherons pas.
\VS{17}J'ai aussi établi sur vous des sentinelles\FTNT{Es. 21:6 ; Ez. 33:1-19.} qui disent : Soyez attentifs au son du shofar ! Mais ils répondent : Nous n'y serons pas attentifs.
\VS{18}Vous donc, nations, écoutez, et toi assemblée, connais ce qui est entre eux.
\VS{19}Ecoute, terre ! Voici, je fais venir un mal sur ce peuple, à savoir le fruit de leurs pensées ; car ils n'ont pas été attentifs à mes paroles, et qu'ils ont rejeté ma loi.
\VS{20}Pourquoi m'offrir de l'encens venu de Séba, et le bon roseau aromatique du pays éloigné ? Vos holocaustes ne me plaisent pas, et vos sacrifices ne me sont pas agréables.
\VS{21}C'est pourquoi ainsi parle Yahweh : Voici, je mettrai devant ce peuple des pierres d'achoppement, auxquels les pères et les fils, le voisin et son compagnon, se heurteront ensemble et ils périront.
\VS{22}Ainsi parle Yahweh : Voici, un peuple vient du pays du nord, et une grande nation se réveille des extrémités de la terre.
\VS{23}Ils prendront l'arc et le javelot ; ils sont cruels et n'ont pas de pitié ; leur voix gronde comme la mer ; ils sont montés sur des chevaux, ils sont rangés comme un seul homme en bataille contre toi, fille de Sion !
\VS{24}Nous en entendons le bruit, nos mains en deviennent lâches, l'angoisse nous saisit, et une douleur comme celle d'une femme qui enfante.
\VS{25}Ne sortez pas dans les champs, n'allez pas par les chemins ; car l'épée de l'ennemi, la terreur est partout.
\VS{26}Fille de mon peuple, ceins-toi d'un sac et roule-toi dans la cendre, prends le deuil comme pour un fils unique, fais une lamentation très amère ! Car le dévastateur vient subitement sur nous.
\VS{27}Je t'avais établi en observateur au milieu de mon peuple, comme une forteresse, pour que tu connaisses et que tu éprouves leur voie.
\VS{28}Ils sont tous rebelles et plus que rebelles, des calomniateurs, ils sont comme de l'airain et du fer ; ils sont tous corrompus.
\VS{29}Le soufflet est brûlant, le plomb est consumé par le feu ; c'est en vain que l'on fond et refond, car les mauvais ne sont pas séparés.
\VS{30}On les appelle de l'argent réprouvé, car Yahweh les a réprouvés.
\Chap{7}
\TextTitle{Hypocrisie de Juda}
\VerseOne{}La parole fut adressée à Jérémie de la part de Yahweh, en disant :
\VS{2}Tiens-toi debout à la porte de la maison de Yahweh, et là, crie cette parole, et dis : Ecoutez la parole de Yahweh, vous tous, hommes de Juda, qui entrez par ces portes, pour vous prosterner devant Yahweh !
\VS{3}Ainsi parle Yahweh des armées, le Dieu d'Israël : Amendez vos voies et vos actions, et je vous ferai habiter en ce lieu-ci.
\VS{4}Ne vous confiez pas en des paroles trompeuses, en disant : C'est ici le temple de Yahweh, le temple de Yahweh, le temple de Yahweh !
\VS{5}Mais amendez sérieusement vos voies et vos actions, et appliquez-vous à faire droit à ceux qui plaident l'un contre l'autre,
\VS{6}et ne faites pas de tort à l'étranger, ni à l'orphelin, ni à la veuve, et ne répandez pas en ce lieu-ci le sang innocent, et ne marchez pas après les dieux étrangers, pour votre malheur.
\VS{7}Et je vous ferai habiter depuis un siècle jusqu'à l'autre siècle en ce lieu-ci, dans le pays que j'ai donné à vos pères.
\VS{8}Voici, vous vous confiez en des paroles trompeuses, sans aucun profit.
\VS{9}Ne dérobez-vous pas ? Ne tuez-vous pas ? Ne commettez-vous pas adultère ? Ne jurez-vous pas faussement ? Ne faites-vous pas des encensements à Baal ? N'allez-vous pas après les dieux étrangers, que vous ne connaissez point ?
\VS{10}Toutefois vous venez et vous vous présentez devant moi, dans cette maison sur laquelle mon Nom est invoqué, et vous dites : Nous sommes délivrés !… Pour faire toutes ces abominations !
\VS{11}N'est-elle plus à vos yeux qu'une caverne de voleurs\FTNT{Mt. 21:13 ; Mc. 11:17 ; Lu. 19:46.}, cette maison sur laquelle mon Nom est invoqué ? Et voici, moi-même je le vois, dit Yahweh.
\VS{12}Mais allez maintenant à mon lieu qui était à Silo, où j'avais fait demeurer mon Nom au commencement. Et regardez ce que je lui ai fait, à cause de la méchanceté de mon peuple d'Israël.
\VS{13}Maintenant donc, puisque vous avez fait toutes ces actions, dit Yahweh, puisque je vous ai parlé, parlé dès le matin, et que vous n'avez pas écouté, puisque je vous ai appelés et que vous n'avez pas répondu ;
\VS{14}je ferai à cette maison sur laquelle mon Nom est invoqué, et sur laquelle vous vous confiez, et à ce lieu que je vous ai donné à vous et à vos pères, comme j'ai fait à Silo ;
\VS{15}et je vous chasserai de devant ma face, comme j'ai chassé tous vos frères, avec toute la postérité d'Ephraïm.
\VS{16}Toi donc ne prie pas pour ce peuple, et n'élève pour eux ni cri ni prière, et n'intercède pas auprès de moi\FTNT{Ez. 3:26-27.} ; car je ne t'écouterai pas.
\VS{17}Ne vois-tu pas ce qu'ils font dans les villes de Juda et dans les rues de Jérusalem ?
\VS{18}Les fils ramassent le bois, et les pères allument le feu, et les femmes pétrissent la pâte pour faire des gâteaux à la reine des cieux\FTNT{La reine des cieux est une déesse qui change de nom en fonction des pays. Asherah, Astarté, Isis, Junon, Cybèle, Diane ou encore la vierge Marie, proclamée mère de Dieu en 431 au concile d'Ephèse. Voir De. 16:2-3.}, et pour faire des libations aux dieux étrangers, afin de m'irriter.
\VS{19}Est-ce moi qu'ils irritent ? dit Yahweh ; n'est-ce pas contre eux-mêmes, à la confusion de leurs faces ?
\VS{20}C'est pourquoi ainsi parle le Seigneur Yahweh : Voici, ma colère et ma fureur se répandent sur ce lieu-ci, sur les hommes et sur les bêtes, sur les arbres des champs et sur le fruit de la terre ; ma colère brûlera et ne s'éteindra pas.
\VS{21}Ainsi parle Yahweh des armées, le Dieu d'Israël : Ajoutez vos holocaustes à vos sacrifices, et mangez-en la chair !
\VS{22}Car je n'ai pas parlé avec vos pères et je ne leur ai pas donné d'ordre au sujet des holocaustes et des sacrifices, le jour où je les ai fait sortir du pays d'Egypte.
\VS{23}Mais voici la parole que je leur ai commandée, disant : Ecoutez ma voix, et je serai votre Dieu, et vous serez mon peuple ; marchez dans toutes les voies que je vous ordonne, afin que vous soyez heureux\FTNT{Ex. 15:26.}.
\VS{24}Mais ils n'ont pas écouté, et n'ont pas prêté l'oreille ; mais ils ont suivi d'autres conseils, les penchants de leur mauvais cœur ; ils se sont éloignés et ne sont pas revenus à moi.
\VS{25}Depuis le jour où vos pères sont sortis du pays d'Egypte, jusqu'à ce jour, je vous ai envoyé tous mes serviteurs les prophètes, je les ai envoyés chaque jour, dès le matin.
\VS{26}Mais ils ne m'ont pas écouté, et ils n'ont pas prêté l'oreille ; mais ils ont raidi leur cou, ils ont fait le mal plus que leurs pères.
\VS{27}Tu leur diras toutes ces paroles, mais ils ne t'écouteront pas ; et tu crieras après eux, mais ils ne te répondront pas.
\VS{28}C'est pourquoi tu leur diras : C'est ici la nation qui n'écoute pas la voix de Yahweh, son Dieu, et qui ne reçoit pas d'instruction ; la vérité a disparu, elle s'est retirée de leur bouche.
\VS{29}Coupe ta chevelure, ô Jérusalem ! Et jette-la au loin, et prononce à haute voix ta complainte sur les lieux élevés ! Car Yahweh rejette et abandonne la génération qui a provoqué sa fureur.
\VS{30}Car les fils de Juda ont fait ce qui est mal à mes yeux, dit Yahweh ; ils ont mis leurs abominations dans cette maison sur laquelle mon Nom est invoqué, afin de la souiller.
\VS{31}Et ils ont bâti les hauts lieux de Topheth, qui est dans la vallée de Ben-Hinnom\FTNT{Voir commentaire en Ap. 16:16.}, pour brûler au feu leurs fils et leurs filles\FTNT{Lé. 18:21. Voir commentaire en Lé. 20:2.} : Ce que je n'avais pas ordonné, et à quoi je n'ai jamais pensé.
\VS{32}C'est pourquoi voici, les jours viennent, dit Yahweh, qu'elle ne sera plus appelée Topheth, ni la vallée de Ben-Hinnom, mais la vallée de la tuerie ; et on enterrera les morts à Topheth, à cause qu'il n'y aura plus d'autre lieu.
\VS{33}Et les cadavres de ce peuple seront la pâture des oiseaux des cieux et des bêtes de la terre ; sans qu'il n'y ait personne qui les effraye.
\VS{34}Je ferai aussi cesser dans les villes de Juda et dans les rues de Jérusalem les cris de joie et les cris d'allégresse, la voix de l'époux et la voix de l'épouse ; car le pays sera un désert.
\Chap{8}
\TextTitle{Juda dans l'égarement}
\VerseOne{}En ce temps-là, dit Yahweh, on sortira les os des rois de Juda, et les os de ses chefs, les os des sacrificateurs, et les os des prophètes, et les os des habitants de Jérusalem, hors de leurs sépulcres.
\VS{2}Et on les étendra devant le soleil, et devant la lune, et devant toute l'armée des cieux, qui sont des choses qu'ils ont aimées, qu'ils ont servies et après lesquelles ils ont marché ; des choses qu'ils ont recherchées, et devant lesquelles ils se sont prosternés ; ils ne seront pas recueillis ni ensevelis, ils seront comme du fumier sur la face du sol.
\VS{3}Et la mort sera plus désirable que la vie pour tous ceux qui resteront de cette race mauvaise, ceux, dis-je, qui seront restés dans tous les lieux où je les aurai chassés, dit Yahweh des armées.
\VS{4}Dis-leur donc : Ainsi parle Yahweh : Si on tombe, ne se relève-t-on pas ? Et si on se détourne, ne revient-on pas ?
\VS{5}Pourquoi donc ce peuple de Jérusalem s'abandonne-t-il à de perpétuels égarements ? Ils tiennent ferme à la tromperie, et ils refusent de convertir.
\VS{6}Je suis attentif et j'écoute, mais nul ne parlent selon la justice ; il n'y a personne qui se repente de sa méchanceté, disant : Qu'ai-je fait ? Ils retournent tous vers les objets qui les entraînent, comme le cheval qui se jette avec impétuosité parmi la bataille.
\VS{7}Même la cigogne connaît dans les cieux ses saisons ; la tourterelle et l'hirondelle, et la grue observent le temps où elles doivent venir ; mais mon peuple ne connaît pas les ordonnances de Yahweh.
\VS{8}Comment dites-vous : Nous sommes les sages, et la loi de Yahweh est avec nous ? Voilà, certes on a agi faussement, et la plume des scribes est une plume de fausseté.
\VS{9}Les sages sont confus, ils sont épouvantés et pris ; car ils ont rejeté la parole de Yahweh, et quelle sagesse ont-ils ?
\VS{10}C'est pourquoi je donnerai leurs femmes à d'autres, et leurs champs à des gens qui les posséderont en héritage. Car depuis le plus petit jusqu'au plus grand, chacun s'adonne au gain déshonnête, tant le prophète que le sacrificateur, tous agissent faussement.
\VS{11}Ils pansent à la légère la plaie de la fille de mon peuple, en disant : Paix ! Paix ! Et il n'y a pas de paix.
\VS{12}Sont-ils confus d'avoir commis des abominations ? Ils n'en ont même aucune honte, et ils ne savent pas ce que c'est que de rougir ; c'est pourquoi ils tomberont parmi ceux qui tombent, ils seront renversés au temps où je les visiterai, dit Yahweh.
\VS{13}Je les ramasserai, j'en finirai avec eux, dit Yahweh ; il n'y aura plus de raisins à la vigne, et il n'y aura plus de figues au figuier, les feuilles se flétriront ; et ce que je leur avais donné sera transporté avec eux.
\VS{14}Pourquoi restons-nous assis ? Assemblez-vous et entrons dans les villes fortes, et nous serons là en repos ! Car Yahweh, notre Dieu, nous réduit au silence, et il nous fait boire des eaux empoisonnées, parce que nous avons péché contre Yahweh.
\VS{15}On attendait la paix, et il n'y a rien de bon ; on attend le temps de guérison, et voici la terreur !
\VS{16}Le hennissement de ses chevaux se fait entendre de Dan, et tout le pays tremble au bruit des hennissements de ses puissants chevaux ; ils viennent et dévorent le pays et ce qu'il contient, la ville et ceux qui l'habitent.
\VS{17}Qui plus est, voici, j'envoie contre vous des serpents, des basilics, contre lesquels il n'y a pas d'enchantement, et ils vous mordront, dit Yahweh.
\VS{18}J'ai voulu prendre des forces pour soutenir la douleur, mais mon cœur est languissant au dedans de moi.
\VS{19}Voici la voix du cri de la fille de mon peuple, qui crie d'un pays éloigné : Yahweh n'est-il plus à Sion ? Son Roi n'est-il plus au milieu d'elle ? Pourquoi m'ont-ils irrité par leurs images taillées, par les vanités\FTNT{Idoles que Dieu appelle vanité, vapeur ou souffle} (2) étrangères ?
\VS{20}La moisson est passée, l'été est fini, et nous ne sommes pas sauvés !
\VS{21}Je suis brisé par la blessure de la fille de mon peuple, je suis sombre, l'épouvante me saisit.
\VS{22}N'y a-t-il pas de baume en Galaad ? N'y a-t-il pas là de médecin ? Pourquoi donc la guérison de la fille de mon peuple ne s'opère-t-elle pas ?
\Chap{9}
\TextTitle{Jérémie pleure sur son peuple}
\VerseOne{}Plaise à Dieu que ma tête soit comme un réservoir d'eau, et que mes yeux soient une vive fontaine de larmes, et je pleurerais jour et nuit les blessés à mort de la fille de mon peuple !
\VS{2}Plaise à Dieu que j'aie au désert une cabane de voyageurs, j'abandonnerais mon peuple, je m'en irais loin de lui ! Car ils sont tous des adultères, et une assemblée de perfides.
\VS{3}Ils ont tendu leur langue, qui a été comme leur arc pour décrocher le mensonge\FTNT{Ps. 64:3-4.} ; et ils se sont renforcés dans la terre contre la fidélité ; car ils sont allés de méchanceté en méchanceté, et ne m'ont pas reconnu, dit Yahweh.
\VS{4}Gardez-vous chacun de son intime ami, et ne vous confiez en aucun frère\FTNT{Mi. 7:5.} ; car tout frère fait métier de supplanter, et tout intime ami marche dans la calomnie.
\VS{5}Et chacun se moque de son intime ami, et on ne parle pas selon la vérité ; ils ont instruit leur langue à dire le mensonge, ils se tourmentent extrêmement pour faire le mal.
\VS{6}Ta demeure est au milieu de la tromperie ; ils refusent, à cause de la tromperie, de me connaître, dit Yahweh.
\VS{7}C'est pourquoi, ainsi parle Yahweh des armées : Voici, je vais les fondre, je les éprouverai\FTNT{Mal. 3:3.}. Car comment en agirais-je autrement à l'égard de la fille de mon peuple ?
\VS{8}Leur langue est une flèche meurtrière, elle profère des tromperies ; chacun de sa bouche parle de la paix avec son ami, mais au-dedans il lui dresse des embûches\FTNT{Ps. 12:3; Ps. 28:3.}.
\VS{9}Ne les punirais-je pas pour ces choses-là, dit Yahweh ? Mon âme ne se vengerait-elle pas d'une telle nation ?
\VS{10}J'élèverai ma voix avec larmes, et je prononcerai à haute voix une lamentation à cause des montagnes, et une complainte à cause des cabanes du désert, parce qu'elles sont brûlées, de sorte que personne n'y passe et qu'on n'y entend plus la voix des troupeaux ; les oiseaux des cieux et le bétail ont fui, ils s'en sont allés.
\VS{11}Et je ferai de Jérusalem des monceaux de ruines, elle sera un repaire de serpents, et je ferai des villes de Juda un désert sans habitants.
\VS{12}Qui est l'homme sage qui comprenne ceci ? Qui est celui à qui la bouche de Yahweh a parlé ? Qu'il le déclare et qu'il dise pourquoi le pays est-il détruit, brûlé comme un désert, sans que personne y passe ?
\VS{13}Yahweh donc dit : Parce qu'ils ont abandonné ma loi que j'avais mise devant eux ; parce qu'ils n'ont pas écouté ma voix, et qu'ils n'ont pas marché selon elle ;
\VS{14}mais parce qu'ils ont marché suivant les penchants de leur cœur, et après les Baals, comme leurs pères le leur ont enseigné.
\VS{15}C'est pourquoi, ainsi parle Yahweh des armées, le Dieu d'Israël : Voici, je vais faire manger de l'absinthe à ce peuple-ci, et je leur ferai boire des eaux empoisonnées.
\VS{16}Je les disperserai parmi les nations que n'ont connues ni eux ni leurs pères, et j'enverrai après eux l'épée, jusqu'à ce que je les aie exterminés.
\VS{17}Ainsi parle Yahweh des armées : Considérez, et appelez des pleureuses, afin qu'elles viennent, et mandez les femmes sages, et qu'elles viennent !
\VS{18}Qu'elles se hâtent, et qu'elles prononcent à haute voix une lamentation sur nous ! Et que nos larmes tombent de nos yeux et que l'eau coule de nos paupières !
\VS{19}Car une voix de lamentation se fait entendre de Sion, disant : Eh quoi ! Nous sommes dévastés ! Nous sommes couverts de honte ! Car nous avons abandonné le pays, car nos demeures nous ont jetés dehors !
\VS{20}C'est pourquoi, vous, femmes, écoutez la parole de Yahweh, et que votre oreille reçoive la parole de sa bouche ! Enseignez vos filles à se lamenter, et chacune sa compagne à faire des complaintes !
\VS{21}Car la mort est montée par nos fenêtres, elle est entrée dans nos palais, pour exterminer les enfants dans les rues, et les jeunes hommes dans les places.
\VS{22}Dis : Ainsi parle Yahweh : Même les cadavres des hommes tomberont comme du fumier sur le dessus des champs, et comme une gerbe après le moissonneur, sans que personne les ramasse !
\VS{23}Ainsi parle Yahweh : Que le sage ne se glorifie pas de sa sagesse, que le fort ne se glorifie pas de sa force, et que le riche ne se glorifie pas de sa richesse.
\VS{24}Mais que celui qui se glorifie, se glorifie d'avoir de l'intelligence et de me connaître, car je suis Yahweh, qui fais miséricorde, droit et justice sur la terre ; car je prends plaisir en ces choses-là, dit Yahweh\FTNT{Ps. 62:10 ; 1 Co. 1:31 ; 2 Co. 10:17 ; 1 Ti. 6:17.}.
\VS{25}Voici, les jours viennent, dit Yahweh, où je punirai tout circoncis incirconcis,
\VS{26}l'Egypte, Juda, Edom, les fils d'Ammon, Moab, et tous ceux qui se coupent les coins de leur barbe et qui habitent dans le désert ; car toutes les nations sont incirconcises, et toute la maison d'Israël a le cœur incirconcis.
\Chap{10}
\TextTitle{Dénonciation de l'idolâtrie en Israël}
\VerseOne{}Ecoutez la parole que Yahweh vous adresse, maison d'Israël !
\VS{2}Ainsi parle Yahweh : N'apprenez pas les façons de faire des nations\FTNT{Lé. 18:3 ; De. 12:30.}, et ne craignez pas les signes des cieux, parce que les nations les craignent.
\VS{3}Car les lois des peuples ne sont que vanité\FTNT{Les lois des peuples, ou encore statuts, coutumes, ordonnances ne sont que vanité. Nous devons nous soumettre aux lois des nations tant que celles-ci ne s'opposent pas à la Loi de Dieu (1 Pi. 2 :13). Quand celles-ci sont contraires aux règles morales établies par le Seigneur, nous devons obéir à Dieu, car il vaut mieux obéir à Dieu plutôt qu'aux hommes (Ac. 4:19 ; Ac. 5:29).}. On coupe le bois dans la forêt ; la main de l'ouvrier le travaille avec la hache\FTNT{Es. 40 :20 ; Es. 44 :12-18.} ;
\VS{4}on l'embellit avec de l'argent et de l'or, on le fait tenir avec des clous et à coups de marteau, afin qu'il ne vacille pas.
\VS{5}Ils sont façonnés tout droits comme des colonnes massives, et ils ne parlent pas ; on les porte par nécessité, parce qu'ils ne peuvent pas marcher. Ne les craignez pas, car ils ne sauraient faire aucun mal, et aussi ils sont incapables de faire du bien.
\VS{6}Nul n'est semblable à toi, ô Yahweh ! Tu es grand, et ton Nom est grand par ta puissance.
\VS{7}Qui ne te craindrait, Roi des nations ? Car cela t'est dû ; car, parmi tous les sages des nations et dans tous leurs royaumes, nul n'est semblable à toi\FTNT{Ap. 15:4.}.
\VS{8}Et ils sont tous ensemble stupides et insensés ; le bois ne leur enseigne que des vanités\FTNT{Ha. 2:18.}.
\VS{9}L'argent qui est étendu en plaques est apporté de Tarsis, et l'or d'Uphaz, pour être mis en œuvre par l'ouvrier et par les mains du fondeur ; et la pourpre et l'écarlate sont leur vêtement ; toutes ces choses sont l'ouvrage de gens habiles.
\VS{10}Mais Yahweh est le Dieu de vérité, c'est le Dieu vivant et le Roi éternel ; la terre tremble devant sa colère, et les nations ne supportent pas sa fureur.
\VS{11}Vous leur parlerez ainsi : Les dieux qui n'ont pas fait les cieux et la terre périront de la terre et de dessous les cieux.
\VS{12}Mais Yahweh est celui qui a fait la terre par sa puissance, qui a fondé le monde habitable par sa sagesse, et qui a étendu les cieux par son intelligence.
\VS{13}Sitôt qu'il fait retentir sa voix, il y a un tumulte d'eaux dans les cieux ; il fait monter les vapeurs des extrémités de la terre, il fait les éclairs et la pluie, et il fait sortir le vent de ses réservoirs.
\VS{14}Tout homme devient stupide par sa connaissance, tout fondeur est honteux par les images taillées ; car les idoles en métal fondu ne sont que mensonge, il n'y a pas de souffle en elles ;
\VS{15}elles ne sont que vanité, une œuvre de tromperie ; elles périront au temps de leur châtiment.
\VS{16}La portion de Jacob n'est pas comme ces choses-là ; car c'est lui qui a tout formé, et Israël est la tribu de son héritage. Son Nom est Yahweh des armées.
\VS{17}Toi qui es assise dans la détresse, rassemble du pays tes paquets !
\VS{18}Car ainsi parle Yahweh : Voici, cette fois je vais lancer au loin, comme avec une fronde, les habitants du pays ; je vais les mettre à l'étroit, afin qu'on les atteigne.
\VS{19}Malheur à moi, diront-ils, à cause de ma blessure ! Ma plaie est douloureuse ! Mais moi, je dis : Quoi qu'il en soit, c'est une maladie qu'il faut que je supporte.
\VS{20}Ma tente est dévastée, tous mes cordages sont rompus ; mes fils m'ont quittée, et ils ne sont plus ; il n'y a plus personne qui dresse ma tente, qui relève mes pavillons.
\VS{21}Car les pasteurs ont été stupides, ils n'ont pas cherché Yahweh ; c'est pour cela qu'ils n'ont pas réussi et que tous leurs troupeaux s'éparpillent.
\VS{22}Voici, une rumeur se fait entendre ; avec une grande secousse qui vient du pays du nord, pour faire des villes de Juda un désert, un repaire de serpents.
\VS{23}Yahweh ! Je sais que la voie de l'homme ne dépend pas de lui\FTNT{Pr. 16:1.}, et qu'il n'est pas au pouvoir de l'homme qui marche de diriger ses pas.
\VS{24}Ô Yahweh ! Châtie-moi, mais avec équité, et non dans ta colère, de peur que tu ne me réduises à rien\FTNT{Es. 27:8 ; Ps. 38:2.}.
\VS{25}Répands ta fureur sur les nations qui ne te connaissent pas, et sur les familles qui n'invoquent pas ton Nom ! Car ils ont dévoré Jacob, ils l'ont, dis-je, dévoré et consumé, et ils ont mis en désolation son agréable demeure.
\Chap{11}
\TextTitle{Yahweh dénonce la prostitution de Juda}
\VerseOne{}La parole fut adressée à Jérémie de la part de Yahweh, en disant :
\VS{2}Ecoutez les paroles de cette alliance, et parlez aux hommes de Juda et aux habitants de Jérusalem !
\VS{3}Dis-leur : Ainsi parle Yahweh, le Dieu d'Israël : Maudit soit l'homme qui n'écoute pas les paroles de cette alliance\FTNT{De. 27:26 ; Ga. 3:10.},
\VS{4}que j'ai ordonnée à vos pères, le jour où je les ai fait sortir du pays d'Egypte, de la fournaise de fer, en disant : Ecoutez ma voix et faites toutes les choses que je vous ordonnerai ; alors vous serez mon peuple, et je serai votre Dieu\FTNT{Lé. 26:12 ; De. 4:20.},
\VS{5}afin que j'accomplisse le serment que j'ai juré à vos pères, de leur donner un pays où coulent le lait et le miel, comme vous le voyez aujourd'hui. Et je répondis et dis : Amen ! Ô Yahweh !
\VS{6}Puis Yahweh me dit : Crie toutes ces paroles dans les villes de Juda et dans les rues de Jérusalem, en disant : Ecoutez les paroles de cette alliance et observez-les !
\VS{7}Car j'ai averti vos pères, depuis le jour où je les ai fait monter du pays d'Egypte jusqu'à ce jour, je les ai avertis dès le matin, en disant : Ecoutez ma voix !
\VS{8}Mais ils n'ont pas écouté, ils n'ont pas prêté l'oreille, ils ont marché chacun suivant les penchants de leur mauvais cœur ; c'est pourquoi j'ai fait venir sur eux toutes les paroles de cette alliance, que je leur avais donné l'ordre d'observer, et qu'ils n'ont pas observée.
\VS{9}Yahweh me dit : Il y a une conspiration entre les hommes de Juda et entre les habitants de Jérusalem.
\VS{10}Ils sont retournés aux iniquités de leurs premiers pères, qui ont refusé d'écouter mes paroles, et ils sont allés après d'autres dieux pour les servir. La maison d'Israël et la maison de Juda ont rompu mon alliance, que j'avais faite avec leurs pères.
\VS{11}C'est pourquoi ainsi parle Yahweh : Voici, je fais venir sur eux un mal dont ils ne pourront sortir. Ils crieront vers moi, et je ne les écouterai pas\FTNT{Es. 1:15 ; Ez. 8:18 ; Mi. 3:4 ; Pr. 1:28.}.
\VS{12}Et les villes de Juda et les habitants de Jérusalem s'en iront et crieront vers les dieux auxquels ils brûlent de l'encens, mais ces dieux-là ne les sauveront pas au temps de leur malheur.
\VS{13}Car, ô Juda ! Tu as eu autant de dieux que de villes ; et toi, Jérusalem, tu as dressé autant d'autels aux choses honteuses que tu as de rues, des autels, dis-je, pour brûler de l'encens à Baal\FTNT{Ez. 16:24-31 ; Ac. 17:23.}…
\VS{14}Toi donc, n'intercède pas pour ce peuple, et n'élève pour eux ni cri ni prière ; car je ne les écouterai pas au temps où ils crieront vers moi dans leur malheur.
\VS{15}Qu'est-ce que mon bien-aimé a à faire dans ma maison, que tant de gens se servent d'elle pour y faire leurs complots? la chair sainte est transportée loin de toi, et encore quand tu fais le mal, c'est alors que tu triomphes !
\VS{16}Yahweh avait appelé ton nom Olivier verdoyant et beau par la forme de ton fruit ; mais au bruit d'un grand fracas, il y a mis le feu, et ses rameaux sont brisés.
\VS{17}Yahweh des armées, qui t'a plantée, prononce le mal contre toi, à cause de la méchanceté de la maison d'Israël et de la maison de Juda, qui ont agi pour m'irriter, en brûlant de l'encens à Baal.
\TextTitle{Jugement des ennemis de Jérémie}
\VS{18}Et Yahweh me l'a fait savoir, et je l'ai su ; alors tu m'as fait voir leurs actions.
\VS{19}Mais moi, comme un agneau, ou comme un bœuf qu'on mène pour être égorgé, je ne savais pas qu'ils projetaient de mauvais desseins contre moi, en disant : Détruisons l'arbre avec son fruit ! Exterminons-le de la terre des vivants, et qu'on ne se souvienne plus de son nom !
\VS{20}Mais toi, Yahweh des armées, qui juges justement, et qui éprouve les reins et le cœur ! Fais que je voie ta vengeance s'exercer contre eux, car je t'ai découvert ma cause\FTNT{1 S. 16:7 ; Ps. 26:2 ; 1 Ch. 28:9 ; Ap. 2:23.}.
\VS{21}C'est pourquoi ainsi parle Yahweh contre les gens d'Anathoth, qui cherchent ta vie et qui disent : Ne prophétise plus au Nom de Yahweh, et tu ne mourras pas par nos mains\FTNT{Es. 30:10 ; Mi. 2:6.} !
\VS{22}C'est pourquoi donc ainsi parle Yahweh des armées : Voici, je vais les punir ; les jeunes hommes mourront par l'épée, leurs fils et leurs filles mourront par la famine.
\VS{23}Et il ne restera rien d'eux ; car je ferai venir le mal sur les gens d'Anathoth, l'année de leur châtiment.
\Chap{12}
\TextTitle{Prière de Jérémie et réponse de Yahweh}
\VerseOne{}Yahweh, quand je contesterai avec toi, tu seras trouvé juste ; mais toutefois j'entrerai en contestation avec toi : Pourquoi la voie des méchants est-elle prospère ? Pourquoi tous les perfides vivent-ils en paix\FTNT{Job. 21:7-9 ; Ro. 3:4.} ?
\VS{2}Tu les as plantés, et ils ont pris racine, ils s'avancent, et ils portent du fruit. Tu es près de leur bouche, mais tu es loin de leurs cœurs\FTNT{Es. 29:13 ; Job. 21:7-8.}.
\VS{3}Mais, ô Yahweh, tu me connais, tu me vois, tu éprouves mon cœur qui est avec toi. Traîne-les comme des brebis qu'on mène pour être égorgées, et mets-les à part pour le jour de la tuerie !
\VS{4}Jusqu'à quand le pays mènera-t-il deuil, et l'herbe de tous les champs séchera-t-elle à cause de la méchanceté des habitants qui sont en la terre ? Les bêtes et les oiseaux ont été consumés par la disette, parce que ces méchants ont dit : On ne verra pas notre dernière fin. 
\VS{5}Si tu cours avec des piétons et qu'ils te fatiguent, comment lutteras-tu avec les chevaux ? Et si tu te crois en sûreté dans une terre de paix, que feras-tu devant l'orgueil du Jourdain ?
\VS{6} Certainement, mêmes tes frères et la maison de ton père, ceux-là mêmes ont agi perfidement contre toi, eux-mêmes ont crié après toi à plein gosier ; ne les crois point, quoiqu'ils te parlent amicalement\FTNT{Pr. 26:25.}.
\VS{7}J'ai abandonné ma maison, j'ai quitté mon héritage, ce que mon âme aimait le plus je l'ai livré aux mains de ses ennemis.
\VS{8}Mon héritage a été pour moi comme un lion dans la forêt, il a poussé contre moi ses rugissements ; c'est pourquoi je l'ai pris en haine.
\VS{9}Mon héritage a-t-il donc été pour moi comme un oiseau de proie tacheté ? Les oiseaux de proie ne sont-ils pas autour de lui ? Venez, assemblez-vous, vous tous les animaux des champs, venez pour le dévorer\FTNT{Es. 56:9.} !
\VS{10}Plusieurs pasteurs ravagent ma vigne, ils foulent mon champ ; ils réduisent le champ de mes délices en un désert, en une désolation.
\VS{11}Ils le réduisent en un désert ; il est en deuil, il est désolé devant moi. Tout le pays est ravagé, car nul n'y prend garde.
\VS{12}Les destructeurs viennent sur tous les lieux élevés du désert, car l'épée de Yahweh dévore le pays d'un bout à l'autre ; il n'y a de paix pour aucune chair.
\VS{13}Ils ont semé du froment, et ils moissonnent des épines, ils se sont fatigués sans profit. Soyez honteux de vos récoltes, à cause de l'ardeur de la colère de Yahweh\FTNT{Lé. 26:16.}.
\VS{14}Ainsi parle Yahweh contre tous mes mauvais voisins, qui mettent la main sur l'héritage que j'ai donné à mon peuple d'Israël : Voici, je les arracherai de leur pays, et j'arracherai la maison de Juda du milieu d'eux.
\VS{15}Mais il arrivera qu'après que je les avoir arrachés, j'aurai encore compassion d'eux, et je les ramènerai chacun dans son héritage, chacun dans son pays\FTNT{De. 30:3.}.
\VS{16}Et il arrivera que s'ils apprennent bien les voies de mon peuple, pour jurer par mon Nom, en disant : Yahweh est vivant ! Comme ils ont enseigné à mon peuple à jurer par Baal, ils seront édifiés au milieu de mon peuple.
\VS{17}Mais s'ils n'écoutent pas, j'arracherai entièrement une telle nation, et je la ferai périr, dit Yahweh\FTNT{Es. 60:12.}.
\Chap{13}
\TextTitle{La ceinture pourrie, illustration du jugement}
\VerseOne{}Ainsi m'a parlé Yahweh : Va, et achète-toi une ceinture de lin et mets-la sur tes reins ; et ne la mets pas dans l'eau.
\VS{2}J'achetai donc une ceinture, selon la parole de Yahweh, et je la mis sur mes reins.
\VS{3}Et la parole de Yahweh me fut adressée pour la seconde fois, en disant :
\VS{4}Prends la ceinture que tu as achetée et qui est sur tes reins ; lève-toi, va-t'en vers l'Euphrate, et là, cache-la dans la fente d'un rocher.
\VS{5}J'allai donc et je la cachai près de l'Euphrate, comme Yahweh me l'avait ordonné.
\VS{6}Et il arriva que plusieurs jours après Yahweh me dit : Lève-toi, va vers l'Euphrate et reprends la ceinture que je t'avais ordonné d'y cacher.
\VS{7}Et j'allai vers l'Euphrate, je creusai, et je pris la ceinture dans le lieu où je l'avais cachée ; mais voici, la ceinture était pourrie, elle n'était plus bonne à rien.
\VS{8}Alors la parole de Yahweh me fut adressée, en disant :
\VS{9}Ainsi parle Yahweh : Je ferai ainsi pourrir l'orgueil de Juda et le grand orgueil de Jérusalem.
\VS{10}L'orgueil de ce peuple très méchant, qui refuse d'écouter mes paroles, qui marche selon les penchants de son cœur, et qui va après d'autres dieux, pour les servir et pour se prosterner devant eux, qu'il devienne comme cette ceinture qui n'est plus bonne à rien !
\VS{11}Car comme une ceinture est attachée aux reins d'un homme, ainsi je m'étais attaché toute la maison d'Israël et toute la maison de Juda, dit Yahweh, afin qu'elles soient mon peuple, mon Nom, ma louange, et ma gloire. Mais ils ne m'ont pas écouté.
\VS{12}Tu leur diras donc cette parole-ci : Ainsi parle Yahweh, le Dieu d'Israël : Toute outre sera remplie de vin. Et ils te diront : Ne savons-nous pas que toute outre sera remplie de vin ?
\VS{13}Mais tu leur diras : Ainsi parle Yahweh : Voici, je vais remplir d'ivresse tous les habitants de ce pays, les rois qui sont assis sur le trône de David, les sacrificateurs, les prophètes, et tous les habitants de Jérusalem.
\VS{14}Et je les briserai les uns contre les autres, les pères et les fils ensemble, dit Yahweh\FTNT{Es. 51:17-20 ; Ps. 60:5.} ; je n'aurai pas de compassion, je n'épargnerai pas, et je n'aurai pas de miséricorde ; rien ne m'empêchera de les détruire.
\VS{15}Écoutez et prêtez l'oreille ! Ne vous élevez pas ! Car Yahweh parle.
\VS{16}Donnez gloire à Yahweh, votre Dieu, avant qu'il fasse venir les ténèbres, avant que vos pieds se heurtent contre les montagnes du crépuscule ; vous attendrez la lumière, et il la changera en ombre de la mort, il la réduira en obscurité profonde\FTNT{Es. 59:9 ; Jn. 12:35.}.
\VS{17}Que si vous n'écoutez pas ceci, mon âme pleurera en secret, à cause de votre orgueil ; mes yeux verseront des larmes en abondance, ils se fondront en larmes, parce que le troupeau de Yahweh sera emmené captif\FTNT{La. 1:2-16.}.
\VS{18}Dis au roi et à la reine : Humiliez-vous et asseyez-vous sur la cendre ! Car elle est tombée de vos têtes, la couronne de votre gloire.
\VS{19}Les villes du midi sont fermées, il n'y a personne qui les ouvre ; tout Juda est transporté en captivité, il est transporté entièrement.
\VS{20}Levez vos yeux et voyez ceux qui viennent du nord. Où est le troupeau qui t'avait été donné, le troupeau qui faisait ta gloire ?
\VS{21}Que diras-tu quand il te punit ? Car tu les as enseignés à dominer en maîtres sur toi. Les douleurs ne te saisiront-elles pas, comme elles saisissent une femme qui enfante ?
\VS{22}Que si tu dis en ton cœur : Pourquoi cela m'arrive-t-il ? C'est à cause de la multitude de tes iniquités que les pans de ta robe sont relevés, et que tes talons sont violemment mis à nu\FTNT{Es. 47:2-3.}.
\VS{23}L'éthiopien peut-il changer sa peau et le léopard ses taches ? Pourriez-vous, aussi, faire quelque bien, vous qui êtes accoutumés à faire le mal ?
\VS{24}C'est pourquoi je les disperserai, comme du chaume, qui est emporté çà et là par le vent du désert.
\VS{25}Voilà ton sort, la portion que je te mesure, dit Yahweh, parce que tu m'as oublié, et que tu as mis ta confiance dans le mensonge.
\VS{26}A cause de cela, je relèverai les pans de ta robe sur ton visage, et ta honte se verra.
\VS{27}Tes adultères et tes hennissements, l'énormité de tes prostitutions sur les collines et dans les champs, tes abominations, je les ai vues. Malheur à toi, Jérusalem ! Ne seras-tu pas purifiée ? Jusqu'à quand cela durera-t-il ?
\Chap{14}
\TextTitle{Le pays frappé par la sécheresse}
\VerseOne{}La parole de Yahweh, qui fut adressée à Jérémie, à l'occasion de la sécheresse.
\VS{2}Juda est dans le deuil, et ses portes sont dans un état pitoyable. Ils sont tous en deuil, gisant par terre ; et les cris de Jérusalem montent au ciel.
\VS{3}Et les personnes distinguées envoient les petits chercher de l'eau, et les petits vont aux citernes, ne trouvent pas d'eau, et reviennent leurs vases vides ; ils sont honteux et confus, ils couvrent leur tête.
\VS{4}Parce que la terre est crevassée, parce qu'il n'y a pas eu de pluie dans le pays, les laboureurs sont honteux, ils se couvrent la tête.
\VS{5}Même la biche met bas son faon dans le champ et l'abandonne, parce qu'il n'y a pas d'herbe.
\VS{6}Et les ânes sauvages se tiennent sur les lieux élevés, humant l'air comme des serpents ; leurs yeux se consument, parce qu'il n'y a pas d'herbe.
\VS{7}Si nos iniquités témoignent contre nous, agis à cause de ton Nom, ô Yahweh\FTNT{Es. 59:12.} ! Car nos infidélités sont nombreuses, c'est contre toi que nous avons péché.
\VS{8}Toi qui es l'espérance d'Israël, son sauveur au temps de la détresse, pourquoi serais-tu dans le pays comme un étranger, comme un voyageur qui se détourne pour passer la nuit ?
\VS{9}Pourquoi serais-tu comme un homme stupéfait, et comme un héros qui ne peut sauver ? Or tu es au milieu de nous, ô Yahweh, et ton Nom est invoqué sur nous : Ne nous abandonne pas !
\VS{10}Voici ce que Yahweh dit de ce peuple : Parce qu'ils aiment à errer ainsi çà et là, et qu'ils ne savent retenir leurs pieds, Yahweh ne prend pas plaisir en eux, il se souvient maintenant de leurs iniquités, et il punit leurs péchés\FTNT{Os. 8:13.}.
\VS{11}Puis Yahweh me dit : N'intercède pas en faveur de ce peuple.
\VS{12}Quand ils jeûnent, je n'écouterai pas leurs cris ; et quand ils offrent des holocaustes et des offrandes, je n'y prendrai pas plaisir ; mais je les consumerai par l'épée, par la famine et par la peste.
\VS{13}Et je répondis : Ah ! ah ! Seigneur Yahweh ! Voici, les prophètes leur disent : Vous ne verrez pas l'épée, et vous n'aurez pas de famine ; mais je vous donnerai dans ce lieu-ci une paix assurée.
\VS{14}Et Yahweh me dit : C'est le mensonge ce que ces prophètes prophétisent en mon Nom ; je ne les ai pas envoyés, je ne leur ai pas donné d'ordre, je ne leur ai pas parlé ; ils vous prophétisent des visions de mensonge, des divinations, de l'idolâtrie et des tromperies de leur cœur\FTNT{De. 18:20-22 ; Ez. 13:2-3.}.
\VS{15}C'est pourquoi ainsi parle Yahweh sur les prophètes qui prophétisent en mon Nom, sans que je les ai envoyés, et qui disent : Il n'y aura ni épée ni la famine dans ce pays : Ces prophètes-là seront consumés par l'épée et par la famine.
\VS{16}Et le peuple à qui ils prophétisent sera jeté dans les rues de Jérusalem à cause de la famine et de l'épée ; et il n'y aura personne pour les enterrer, ni eux, ni leurs femmes, ni leurs fils, ni leurs filles ; je répandrai sur eux leur méchanceté.
\VS{17}Tu leur diras donc cette parole-ci : Que mes yeux se fondent en larmes nuit et jour, et qu'ils ne cessent pas\FTNT{La. 1:16.} ; car la vierge, fille de mon peuple, a été frappée d'un grand coup, d'une plaie très douloureuse.
\VS{18}Si je sors dans les champs, voici les gens tués par l'épée ; si j'entre dans la ville, voici les gens consumés par la faim ; même le prophète et le sacrificateur parcourent le pays, sans savoir où ils vont.
\VS{19}As-tu entièrement rejeté Juda, et ton âme a-t-elle Sion en horreur ? Pourquoi nous frappes-tu sans qu'il y ait pour nous de guérison ? On attend la paix, mais il n'y a rien de bon, un temps de guérison, et voici la terreur !
\VS{20}Yahweh, nous reconnaissons notre méchanceté, l'iniquité de nos pères ; car nous avons péché contre toi\FTNT{Ps. 106:6 ; Da. 9:8.}.
\VS{21}Ne nous rejette pas, à cause de ton Nom, et ne déshonore pas le trône de ta gloire ! Souviens-toi de ton alliance avec nous, et ne la romps pas !
\VS{22}Parmi les vanités\FTNT{Ce terme veut aussi dire « idole ».} des nations, y en a-t-il qui fassent pleuvoir, et les cieux donnent-ils des ondées\FTNT{Es. 30:23 ; Ac. 14:17.} ? N'est-ce pas toi, ô Yahweh, notre Dieu ? C'est pourquoi nous nous attendons à toi, car c'est toi qui as fait toutes ces choses.
\Chap{15}
\TextTitle{Yahweh fermement décidé à juger son peuple}
\VerseOne{}Et Yahweh me dit : Quand Moïse et Samuel se tiendraient devant moi, je n'aurais pourtant point d'affection pour ce peuple ; chasse-les de devant ma face, et qu'ils sortent.
\VS{2}Que s'ils te disent : Où irons-nous ? Tu leur répondras : Ainsi parle Yahweh : Ceux qui sont destinés à la mort iront à la mort ; et ceux qui sont destinés à l’épée iront à l’épée ; et ceux qui sont destinés à la famine, iront à la famine ; et ceux qui sont destinés à la captivité iront en captivité\FTNT{Za. 11:9.} !
\VS{3}J'établirai aussi sur eux quatre espèces de punitions, dit Yahweh, l'épée pour tuer, et les chiens pour traîner, et les oiseaux des cieux, et les bêtes de la terre pour dévorer et pour détruire. 
\VS{4}Et je les livrerai à être agités par tous les Royaumes de la terre, à cause de Manassé, fils d'Ezéchias, Roi de Juda, pour les choses qu'il a faites dans Jérusalem. 
\VS{5}Car qui aurait compassion de toi, Jérusalem, ou qui te plaindrait ? Ou qui se détournerait pour s'informer de ta paix ? 
\VS{6}Tu m'as abandonné, dit Yahweh, et tu t'en es allée en arrière ; c'est pourquoi j'étends ma main sur toi, et je te détruis, je suis las d'avoir compassion.
\VS{7}Je les vanne avec un van aux portes du pays\FTNT{Mt. 3:12.} ; je prive d'enfants, je fais périr mon peuple, et ils ne se sont pas détournés de leurs voies.
\VS{8}Je multiplie ses veuves plus que le sable de la mer ; je fais venir sur eux, sur la mère du jeune homme, le dévastateur en plein midi ; je fais tomber subitement sur elle l'angoisse et les frayeurs.
\VS{9}Celle qui en avait enfanté sept languit, elle rend l'âme ; son soleil se couche pendant qu'il est encore jour\FTNT{Am. 8:9.} ; elle est confuse, couverte de honte. Ceux qui restent, je les livre à l'épée devant leurs ennemis, dit Yahweh.
\VS{10}Malheur à moi, ô ma mère, de ce que tu m'as enfanté\FTNT{Job. 3:1-2.} pour être un homme de contestation et un homme de dispute pour tout le pays ! Je n'emprunte ni ne prête, et néanmoins tous me maudissent et me méprisent.
\VS{11}Alors Yahweh dit : En vérité tout ira bien pour ton reste ; en vérité je ferai que l’ennemi te traite bien au temps du malheur, et au temps de la détresse.
\VS{12}Le fer brisera-t-il le fer du nord et l'airain ?
\VS{13}Je livre au pillage, sans en faire le prix, tes richesses et tes trésors, et cela à cause de tous tes péchés, sur tout ton territoire.
\VS{14}Je te fais passer avec tes ennemis dans un pays que tu ne connais pas, car le feu de ma colère s'est allumé, il brûle sur vous\FTNT{De. 32:22.}.
\TextTitle{La mise à part de Jérémie}
\VS{15}Yahweh ! Tu sais tout, souviens-toi de moi, visite-moi, venge-moi de ceux qui me persécutent\FTNT{Ps. 106:4.} ! Ne m'enlève pas, tandis que tu te montres lent à la colère ! Sache que je supporte l'opprobre à cause de toi.
\VS{16}J'ai trouvé tes paroles, je les ai aussitôt dévorées\FTNT{Ez. 3:3 ; Ap. 10:9.} ; tes paroles ont fait la joie et l'allégresse de mon cœur ; car ton Nom est invoqué sur moi, ô Yahweh, Dieu des armées !
\VS{17}Je ne me suis pas assis dans l'assemblée des moqueurs, et je ne m'y suis pas réjoui ; mais je me suis assis tout seul à cause de ta main, car tu me remplissais d'indignation.
\VS{18}Pourquoi ma douleur est-elle continuelle ? Pourquoi ma plaie est-elle incurable et refuse-t-elle d'être guérie ? Serais-tu pour moi comme une source trompeuse, comme des eaux qui ne durent pas ?
\VS{19}C'est pourquoi ainsi parle Yahweh : Si tu reviens, je te ramènerai, et tu te tiendras devant moi ; et si tu sépares la chose précieuse de la méprisable, tu seras comme ma bouche. Qu'ils reviennent vers toi, mais toi, ne retourne pas vers eux.
\VS{20}Je ferai que tu sois pour ce peuple une muraille d'airain bien forte ; ils combattront contre toi, mais il n'auront pas le dessus contre toi ; car je suis avec toi pour te sauver et te délivrer, dit Yahweh.
\VS{21}Et je te délivrerai de la main des malins, et te rachèterai de la main des méchants.
\Chap{16}
\TextTitle{Célibat de Jérémie, illustration du jugement sur Juda}
\VerseOne{}Puis la parole de Yahweh me fut adressée, en disant :
\VS{2}Tu ne prendras pas de femme, et tu n'auras pas de fils ni de filles dans ce lieu-ci.
\VS{3}Car ainsi parle Yahweh sur les fils et les filles qui naîtront en ce lieu-ci, sur leurs mères qui les auront enfantés, et sur leurs pères qui les auront engendrés dans ce pays :
\VS{4}Ils mourront de maladie mortelle ; ils ne seront ni pleurés ni enterrés ; ils seront comme du fumier sur la face du sol ; ils seront consumés par l'épée et par la famine ; et leurs cadavres seront la pâture des oiseaux des cieux et des bêtes de la terre.
\VS{5}Car ainsi parle Yahweh : N'entre pas dans une maison de deuil, ne vas pas te lamenter ni te plaindre avec eux ; car j'ai retiré de ce peuple dit Yahweh, ma paix, ma miséricorde et mes compassions.
\VS{6}Et les grands et petits mourront dans ce pays ; ils ne seront pas enterrés ; on ne les pleurera pas, on ne se fera pas d'incision, et on ne se rasera pas pour eux\FTNT{Lé. 19:28 ; De. 14:1 ; Ez. 7:11}.
\VS{7}On ne rompra pas le pain dans le deuil pour consoler quelqu'un au sujet d'un mort, et on ne leur donnera pas à boire de la coupe de consolation pour leur père ou pour leur mère.
\VS{8}Aussi n'entre pas non plus dans une maison de festin pour t'asseoir avec eux, pour manger et pour boire.
\VS{9}Car ainsi parle Yahweh des armées, le Dieu d'Israël : Voici, je vais faire cesser dans ce lieu-ci, devant vos yeux et en vos jours, les cris de joie et les cris d'allégresse, la voix de l'époux et la voix de l'épouse.
\VS{10}Et il arrivera que quand tu annonceras à ce peuple toutes ces paroles-là, ils te diront : Pourquoi Yahweh parle-t-il de tout ce grand mal contre nous ? Quelle est notre iniquité ? Quel est le péché que nous avons commis contre Yahweh, notre Dieu ?
\VS{11}Et tu leur diras : Parce que vos pères m'ont abandonné, dit Yahweh, et sont allés après d'autres dieux, et les ont servis, et se sont prosternés devant eux, et m'ont abandonné et n'ont pas gardé ma loi ; 
\VS{12}et que vous avez fait le mal plus encore que vos pères. Car voici, chacun de vous marche selon les penchants de son mauvais cœur pour ne pas m'écouter.
\VS{13}A cause de cela, je vous jetterai de ce pays dans un pays que vous n'avez pas connu, ni vous ni vos pères ; et là, vous servirez jour et nuit les autres dieux, car je ne vous aurai pas fait grâce\FTNT{De. 28:64-65.}.
\VS{14}Néanmoins voici, les jours viennent, dit Yahweh, qu'on ne dira plus : Yahweh est vivant, lui qui a fait monter les fils d'Israël du pays d'Egypte !
\VS{15}Mais on dira : Yahweh est vivant, lui qui a fait monter les fils d'Israël du pays du nord et de tous les pays où il les avait chassés ; après que je les aurai ramenés dans leur pays, que j'avais donné à leurs pères.
\VS{16}Voici, j'envoie plusieurs pêcheurs, dit Yahweh, et ils les pêcheront ; et ensuite, j'enverrai plusieurs chasseurs, et ils les chasseront de toutes les montagnes et de toutes les collines, et des fentes des rochers.
\VS{17}Car mes yeux sont sur toutes leurs voies, elles ne sont pas cachées devant ma face, et leur iniquité n'est pas couverte devant mes yeux\FTNT{Pr. 5:21 ; Job. 34:21.}.
\VS{18}Mais premièrement je leur rendrai le double de leur iniquité et de leur péché, parce qu'ils ont souillé mon pays par les cadavres de leurs idoles, et parce qu'ils ont rempli mon héritage de leurs abominations.
\VS{19}Yahweh, qui est ma force et ma forteresse, et mon refuge au jour de la détresse ! Les nations viendront à toi des extrémités de la terre, et diront : Certes nos pères ont hérité le mensonge et la vanité, et les choses auxquelles il n'y a pas de profit.
\VS{20}L'homme se fera-t-il lui-même des dieux, qui ne sont pas dieux ?
\VS{21}C'est pourquoi voici, je leur fais connaître, cette fois, je leur fais connaître ma main et ma force ; et ils sauront que mon Nom est Yahweh.
\Chap{17}
\TextTitle{Le caractère sinueux du cœur}
\VerseOne{}Le péché de Juda est écrit avec un burin de fer, et avec une pointe de diamant ; il est gravé sur la table de leur cœur, et sur les cornes de leurs autels.
\VS{2}De sorte que leurs fils se souviennent de leurs autels, et de leurs poteaux d'Asherah, auprès des arbres verts sur les hautes collines.
\VS{3}Ma montagne, je livre par les champs tes richesses et tous tes trésors au pillage ; tes hauts lieux sont pleins de péché sur tout ton territoire.
\VS{4}Et toi, et ceux qui sont avec toi, vous laisserez vacant l'héritage que je t'avais donné ; et je t'asservirai à tes ennemis dans un pays que tu ne connais pas ; car vous avez allumé le feu de ma colère, et il brûlera à toujours.
\VS{5}Ainsi parle Yahweh : Maudit soit l'homme qui se confie dans l'homme, et qui fait de la chair sa force, et dont le cœur se retire de Yahweh !
\VS{6}Car il sera comme la bruyère dans le désert, et il ne voit pas venir le bien ; mais il demeure dans des lieux brûlés du désert, dans une terre salée et inhabitable.
\VS{7}Béni soit l'homme qui se confie en Yahweh, et dont Yahweh est l'espérance !
\VS{8}Il est comme un arbre planté près des eaux\FTNT{Ps. 23.}, et qui étend ses racines le long d'une eau courante ; quand la chaleur vient, il ne s'en aperçoit pas, et sa feuille reste verte ; il n'est pas en peine dans l'année de la sécheresse, et ne cesse de porter du fruit.
\VS{9}Le cœur est rusé et désespérément malin par-dessus tout : Qui peut le connaître\FTNT{Ps. 64:7.} ?
\VS{10}Je suis Yahweh, qui sonde le cœur, et qui éprouve les reins ; même pour rendre à chacun selon sa voie, et selon le fruit de ses actions.
\VS{11}Celui qui acquiert des richesses, sans observer la justice, est une perdrix qui couve ce qu'elle n'a pas pondu ; il les laissera au milieu de ses jours, et à la fin il sera trouvé insensé\FTNT{Ec. 4:8}.
\VS{12}Le lieu de notre sanctuaire est un trône de gloire, un lieu haut élevé dès le commencement.
\VS{13}Yahweh, qui es l'espérance d'Israël ! Tous ceux qui t'abandonnent seront honteux : Ceux qui se détournent de moi seront écrits sur la terre, car ils abandonnent la source des eaux vives, Yahweh\FTNT{Es. 1:28 ; Ps. 73:28.}.
\VS{14}Yahweh, guéris-moi, et je serai guéri ; sauve-moi, et je serai sauvé ; car tu es ma louange.
\VS{15}Voici, ceux-ci me disent : Où est la parole de Yahweh ? Qu'elle vienne présentement\FTNT{Es. 5:19 ; Ez. 12:23 ; 2 Pi. 3:3-4.} !
\VS{16}Mais je ne me suis pas avancé plus qu’un pasteur après toi, je n'ai pas non plus désiré le jour du malheur, tu le sais ; et ce qui est sorti de mes lèvres est présent devant toi.
\VS{17}Ne sois pas pour moi un sujet d'effroi, toi, mon refuge au jour du malheur !
\VS{18}Que ceux qui me persécutent soient honteux, mais que je ne sois pas honteux ; qu'ils soient brisés, mais que je ne sois pas brisé ! Fais venir sur eux le jour du malheur, frappe-les d'une double plaie !
\TextTitle{Message à propos du sabbat}
\VS{19}Ainsi m'a parlé Yahweh : Va, et tiens-toi debout à la porte des fils du peuple, par laquelle les rois de Juda entrent et par laquelle ils sortent, et à toutes les portes de Jérusalem.
\VS{20}Tu leur diras : Ecoutez la parole de Yahweh, rois de Juda, et vous tous homme de Juda, et vous tous habitants de Jérusalem qui entrez par ces portes !
\VS{21}Ainsi parle Yahweh : Prenez garde à vos âmes ; ne portez aucun fardeau le jour du sabbat, et ne les faites pas passer par les portes de Jérusalem\FTNT{Né. 13:19.}.
\VS{22}Ne faites sortir de vos maisons aucun fardeau le jour du sabbat, et ne faites aucune œuvre ; mais sanctifiez le jour du sabbat, comme je l'ai ordonné à vos pères\FTNT{Ex. 20:8 ; Ex. 23:12.}.
\VS{23}Mais ils n'ont pas écouté, ils n'ont pas prêté l'oreille ; ils ont raidi leur cou, pour ne pas écouter et ne pas recevoir d'instruction.
\VS{24}Il arrivera donc, si vous m'écoutez attentivement, dit Yahweh, pour ne faire passer aucun fardeau par les portes de cette ville le jour du sabbat, et si vous sanctifiez le jour du sabbat, en ne faisant aucune œuvre ce jour-là,
\VS{25}que les rois et les chefs, ceux qui sont assis sur le trône de David, montés sur des chars et sur des chevaux, eux et les chefs d'entre eux, les hommes de Juda et les habitants de Jérusalem, entreront par les portes de cette ville, et cette ville sera habitée à toujours.
\VS{26}On viendra aussi des villes de Juda et des environs de Jérusalem, et du pays de Benjamin, et du bas pays, des montagnes et du midi, pour apporter des holocaustes, des sacrifices, des offrandes et de l'encens ; pour apporter aussi des sacrifices de louanges dans la maison de Yahweh.
\VS{27}Mais si vous ne m'écoutez pas pour sanctifier le jour du sabbat, pour ne porter aucun fardeau, et n'en faire entrer aucun par les portes de Jérusalem le jour du sabbat, je mettrai le feu à ses portes, et il consumera les palais de Jérusalem et ne s'éteindra pas\FTNT{2 R. 25:9.}.
\Chap{18}
\TextTitle{La maison du potier ; appel à la repentance et avertissement}
\VerseOne{}Cette parole fut adressée à Jérémie de la part de Yahweh, disant :
\VS{2}Lève-toi et descends dans la maison d'un potier ; et là, je te ferai entendre mes paroles.
\VS{3}Je descendis donc dans la maison d'un potier, et voici, il faisait son ouvrage, assis sur sa selle.
\VS{4}Et le vase qu'il faisait avec l'argile qu'il tenait dans sa main, fut gâté ; et il en fit encore un autre vase, comme il lui sembla bon de le faire.
\VS{5}Alors la parole de Yahweh me fut adressée, en disant :
\VS{6}Maison d'Israël, ne puis-je pas faire de vous comme a fait ce potier ? Dit Yahweh. Voici, comme l'argile est dans la main d'un potier, ainsi vous êtes dans ma main, maison d'Israël !
\VS{7}En un instant je parle contre une nation et contre un royaume, pour arracher, pour démolir, et pour détruire ;
\VS{8}mais si cette nation, contre laquelle j'ai parlé, revient de sa méchanceté, je me repentirai aussi du mal que j'avais pensé de lui faire\FTNT{Jon. 3:6-10.}.
\VS{9}Et si en un instant je parle d'une nation et d'un royaume, pour l'édifier et pour le planter ;
\VS{10} et que cette nation fasse ce qui est mal à mes yeux, en sorte qu'elle n'écoute pas ma voix, je me repentirai aussi du bien que j'avais dit que je lui ferais.
\VS{11}Or donc, parle maintenant aux hommes de Juda et aux habitants de Jérusalem, en disant : Ainsi parle Yahweh : Voici, je projette du mal contre vous, et je forme un dessein contre vous. Détournez-vous donc chacun de votre mauvaise voie, et amendez votre voie et vos actions !
\VS{12}Et ils répondent : Il n'y a plus d'espérance ; c'est pourquoi nous suivrons nos pensées, chacun de nous fera selon les penchants de son mauvais cœur.
\VS{13}C'est pourquoi ainsi parle Yahweh : Demandez maintenant aux nations ! Qui a entendu de telles choses ? La vierge d'Israël a fait une chose très horrible\FTNT{1 Co. 5:1.}.
\VS{14}La neige du Liban abandonnerait-elle le rocher du champ ? Ou les eaux fraîches et ruisselantes qui viennent de loin tariraient-elles ?
\VS{15}Mais mon peuple m'a oublié, et il brûle de l'encens à ce qui n'est que vanité, et qui les a fait chanceler leurs voies, pour les faire retirer des anciens sentiers, afin de marcher dans les sentiers d'un chemin non frayé ;
\VS{16}pour faire venir sur leur pays une désolation et un opprobre perpétuel ; quiconque passera par là, en sera étonné et secouera la tête.
\VS{17}Je les disperserai devant l'ennemi, comme par le vent d'orient ; je leur tournerai le dos, et non pas la face, au jour de leur calamité\FTNT{Es. 27:8.}.
\VS{18}Et ils ont dit : Venez, et faisons des complots contre Jérémie ! Car la loi ne périra pas chez le sacrificateur, ni le conseil chez le sage, ni la parole chez le prophète. Venez, et tuons-le avec la langue, et ne soyons pas attentifs à ses discours !
\VS{19}Yahweh ! Fais attention à moi, et écoute la voix de ceux qui contestent avec moi !
\VS{20}Le mal sera-t-il rendu pour le bien\FTNT{Ps. 35:12 ; Ps. 109:5.} ? Car ils ont creusé une fosse pour mon âme. Souviens-toi que je me suis tenu devant toi, afin de parler pour leur bien, et afin de détourner d'eux ta grande colère.
\VS{21}C'est pourquoi livre leurs fils à la famine, et fais couler leur sang à coups d’épée ; que leurs femmes soient privées d'enfants, et deviennent veuves, et que leurs maris soient enlevés par la mort ; et leurs jeunes gens frappés par l'épée dans la bataille\FTNT{Ps. 109:9-13.} !
\VS{22}Qu'on entende le cri de leurs maisons, quand tu feras venir subitement des troupes contre eux ! Car ils ont creusé une fosse pour me prendre, et ils ont caché des pièges pour mes pieds.
\VS{23}Or tu sais, ô Yahweh ! Que tout leur conseil est contre moi pour me mettre à mort ; ne sois pas apaisé à l'égard de leur iniquité, et n'efface pas leur péché de devant ta face, mais qu'on les fasse tomber en ta présence ; agis contre eux au temps de ta colère.
\Chap{19}
\TextTitle{Le vase brisé : Image de Juda}
\VerseOne{}Ainsi a parlé Yahweh : Va, et achète un vase de terre d'un potier, et prends avec toi des anciens du peuple et des anciens des sacrificateurs.
\VS{2}Et sors à la vallée de Ben-Hinnom, qui est auprès de l'entrée de la porte de la poterie, et crie là les paroles que je te dirai.
\VS{3}Dis donc : Rois de Juda, et vous, habitants de Jérusalem, écoutez la parole de Yahweh ! Ainsi parle Yahweh des armées, le Dieu d'Israël : Voici, je vais faire venir sur ce lieu-ci un mal, tel que quiconque l'entendra, les oreilles lui tinteront\FTNT{1 S. 3:11 ; 2 R. 21:12.} ; 
\VS{4}parce qu'ils m'ont abandonné, et qu'ils ont profané ce lieu, et y ont brûlé de l'encens à d'autres dieux, que ni eux, ni leurs pères, ni les rois de Juda n'ont connus, et parce qu'ils ont rempli ce lieu du sang des innocents ;
\VS{5}et qu'ils ont bâti des hauts lieux à Baal, afin de brûler au feu leurs fils pour en faire des holocaustes à Baal : Ce que je n'avais pas ordonné, et dont je n'avais pas parlé, et qui ne m'était pas monté à cœur.
\VS{6}A cause de cela, voici les jours viennent, dit Yahweh, que ce lieu-ci ne sera plus appelé Topheth, ni la vallée de Ben-Hinnom, mais la vallée de la tuerie.
\VS{7}Et j'anéantirai dans ce lieu-ci le conseil de Juda et de Jérusalem ; et je les ferai tomber par l'épée devant leurs ennemis et par la main de ceux qui cherchent leur vie ; et je donnerai leurs cadavres en pâture aux oiseaux des cieux et aux bêtes de la terre.
\VS{8}Je ferai de cette ville un objet de désolation et de moquerie ; quiconque passera près d'elle sera étonné et sifflera à cause de toutes ses plaies.
\VS{9}Et je leur ferai manger la chair de leurs fils et la chair de leurs filles ; et chacun mangera la chair de son compagnon durant le siège, et dans la détresse où les réduiront leurs ennemis et ceux qui cherchent leur vie\FTNT{Lé. 26:29 ; De. 28:53 ; La. 2:20.}.
\VS{10}Puis tu briseras le vase, sous les yeux des hommes qui seront allés avec toi.
\VS{11}Et tu leur diras : Ainsi parle Yahweh des armées : Je briserai ce peuple et cette ville, de même qu'on brise un vase de potier, qui ne peut être réparé. Et ils seront enterrés à Topheth parce qu'il n'y aura plus d'autre lieu pour les enterrer.
\VS{12}Je ferai ainsi à ce lieu-ci, dit Yahweh, et à ses habitants, et je rendrai cette ville semblable à Topheth ;
\VS{13}et les maisons de Jérusalem, et les maisons des rois de Juda, seront impures comme le lieu de Topheth, à cause de toutes les maisons sur les toits desquelles ils brûlaient de l'encens à toute l'armée des cieux, et faisaient des libations à d'autres dieux.
\VS{14}Puis Jérémie revint de Topheth, là où Yahweh l'avait envoyé pour prophétiser. Et il se tint debout dans le parvis de la maison de Yahweh, et il dit à tout le peuple :
\VS{15}Ainsi parle Yahweh des armées, le Dieu d'Israël : Voici, je vais faire venir sur cette ville et sur toutes ses villes tout le mal que j'ai prononcés contre elle, parce qu'ils ont raidi leur cou pour ne pas écouter mes paroles.
\Chap{20}
\TextTitle{Paschhur outrage Jérémie}
\VerseOne{}Alors Paschhur, fils d'Immer, qui était sacrificateur et inspecteur en chef dans la maison de Yahweh, entendit Jérémie qui prophétisait ces choses.
\VS{2}Et Paschhur frappa le prophète Jérémie, et le mit dans la prison qui était à la porte supérieure de Benjamin, dans la maison de Yahweh.
\VS{3}Et il arriva que dès le lendemain, que Paschhur tira Jérémie hors de la prison. Et Jérémie lui dit : Yahweh ne t'appelle pas du nom de Paschhur, mais Magor-Missabib\FTNT{« Magor-Missabib » veut dire « terreur de chaque côté ».}.
\VS{4}Car ainsi parle Yahweh : Voici, je vais te livrer à la terreur, toi et tous tes amis qui tomberont par l'épée de leurs ennemis, et tes yeux le verront. Je livrerai tous ceux de Juda entre les mains du roi de Babylone, qui les transportera à Babylone et les frappera de l'épée.
\VS{5}Et je livrerai toutes les richesses de cette ville, et tout son travail, et tout ce qu'elle a de précieux, je livrerai, dis-je, tous les trésors des rois de Juda entre les mains de leurs ennemis, qui les pilleront, les enlèveront et les conduiront à Babylone.
\VS{6}Et toi, Paschhur, et tous ceux qui demeurent dans ta maison, vous irez en captivité ; tu iras à Babylone, tu y mourras, et y seras enterré, toi et tous tes amis auxquels tu as prophétisé le mensonge.
\TextTitle{Jérémie gémit auprès de Yahweh}
\VS{7}Ô Yahweh ! Tu m'as persuadé, et je me suis laissé persuader ; tu m'as saisi, et tu m'as vaincu. Je suis un objet de moquerie chaque jour, chacun se moque de moi.
\VS{8}Car depuis que je parle, je crie, je crie violence et dévastation ! Et la parole de Yahweh est pour moi un sujet d'opprobre et de moquerie chaque jour\FTNT{Es. 57:4.}.
\VS{9}C'est pourquoi j'ai dit : Je ne ferai plus mention de lui, je ne parlerai plus en son Nom, mais il y a eu dans mon cœur comme un feu ardent, renfermé dans mes os ; je me fatigue à le contenir, et je ne le puis.
\VS{10}Car j'entends les mauvais propos de plusieurs, la frayeur m'a saisi de tous côtés ; rapportez, disent-ils, et nous le rapporterons ! Tous ceux qui étaient en paix avec moi observent si je bronche, et disent : Peut-être se laissera-t-il séduire, et nous le vaincrons, nous tirerons vengeance de lui !
\VS{11}Mais Yahweh est avec moi comme un héros puissant ; c'est pourquoi ceux qui me persécutent seront renversés, ils ne me vaincront pas ; ils seront honteux, car ils n'ont pas réussi : Ce sera une honte éternelle qui ne s'oubliera jamais.
\VS{12}Yahweh des armées qui éprouve les justes, qui voit les reins et les cœurs, fais que je voie ta vengeance s'exercer contre eux, car je t'ai découvert ma cause.
\VS{13}Chantez à Yahweh, louez Yahweh ! Car il délivre l'âme des pauvres de la main des méchants.
\VS{14}Maudit soit le jour où je suis né ! Que le jour où ma mère m'a enfanté ne soit pas béni !
\VS{15}Maudit soit l'homme qui porta cette nouvelle à mon père, en lui disant : Un fils mâle t'est né, et qui le combla de joie !
\VS{16}Que cet homme-là soit comme les villes que Yahweh a renversées sans s'en repentir ! Qu'il entende la clameur le matin, et le cri de guerre au temps du midi\FTNT{Ge. 19:24-25 ; So. 2:4.} !
\VS{17}Que ne m'a-t-on fait mourir dans le sein de ma mère ! Pourquoi ma mère ne m'a-t-elle pas servi de sépulcre ? Et pourquoi n'est-elle pas restée éternellement enceinte ?
\VS{18}Pourquoi suis-je sorti de son sein pour ne voir que peine et douleur, et pour consumer mes jours dans la honte ?
\Chap{21}
\TextTitle{Prophétie sur les rois de Juda : Sédécias}
\VerseOne{}La parole qui fut adressée à Jérémie de la part de Yahweh, lorsque le roi Sédécias envoya vers lui Paschhur, fils de Malkija, et Sophonie, fils de Maaséja, le sacrificateur, pour lui dire :
\VS{2}Consulte maintenant Yahweh pour nous ; car Nebucadnetsar, roi de Babylone, combat contre nous ; peut-être que Yahweh fera-t-il en notre faveur un de ses miracles, afin qu'il se retire de nous.
\VS{3}Et Jérémie leur dit : Vous direz ainsi à Sédécias :
\VS{4}Ainsi parle Yahweh, le Dieu d'Israël : Voici, je vais détourner les armes de guerre qui sont dans vos mains, et avec lesquelles vous combattez en dehors des murailles contre le roi de Babylone et contre les Chaldéens qui vous assiègent, et je les rassemblerai au milieu de cette ville.
\VS{5}Et je combattrai contre vous, avec une main étendue, et avec un bras puissant, avec colère, avec fureur, et avec une grande indignation.
\VS{6}Et je frapperai les habitants de cette ville, les hommes, et les bêtes ; et ils mourront d'une grande peste.
\VS{7}Et après cela, dit Yahweh, je livrerai Sédécias, roi de Juda, et ses serviteurs, et le peuple, et ceux qui dans cette ville survivront à la peste, à l'épée et à la famine, entre les mains de Nebucadnetsar\FTNT{2 R. 24 et 25. }, roi de Babylone, et entre les mains de leurs ennemis, et entre les mains de ceux qui cherchent leur vie ; et il les frappera au tranchant de l'épée, il ne les épargnera pas, il n'en aura pas de compassion, il n'en aura pas de pitié.
\VS{8}Tu diras aussi à ce peuple : Ainsi parle Yahweh : Voici, je mets devant vous le chemin de la vie et le chemin de la mort\FTNT{De. 30:19.}.
\VS{9}Quiconque restera dans cette ville mourra par l'épée, ou par la famine, ou par la peste, mais celui qui en sortira, et se rendra aux Chaldéens qui vous assiègent vivra et aura sa vie pour butin.
\VS{10}Car je dresse ma face en mal et non en bien contre cette ville, dit Yahweh ; elle sera livrée entre les mains du roi de Babylone, et il la brûlera par le feu.
\VS{11}Et quant à la maison du roi de Juda : Ecoutez la parole de Yahweh !
\VS{12}Maison de David ! Ainsi parle Yahweh : Rendez la justice dès le matin, et délivrez celui qui aura été pillé d'entre les mains de l'oppresseur, de peur que ma fureur ne sorte comme un feu, et qu'elle ne brûle sans qu'on puisse l'éteindre, à cause de la méchanceté de vos actions.
\VS{13}Voici, j'en veux à toi qui habites dans la vallée, et qui es le rocher de la plaine, dit Yahweh ; à vous qui dites : Qui descendra contre nous, et qui entrera dans nos demeures ?
\VS{14}Et je vous punirai selon le fruit de vos actions, dit Yahweh ; et je mettrai le feu dans sa forêt qui consumera tout ce qui est autour d'elle\FTNT{Ez. 21:2-3.}.
\Chap{22}
\TextTitle{Sédécias averti de la destruction de Jérusalem}
\VerseOne{}Ainsi parle Yahweh : Descends dans la maison du roi de Juda, et là prononce cette parole.
\VS{2}Tu diras donc : Ecoute la parole de Yahweh, ô roi de Juda qui es assis sur le trône de David, toi et tes serviteurs, et ton peuple, qui entrez par ces portes !
\VS{3}Ainsi parle Yahweh : Faites droit et justice ; et délivrez celui qui aura été pillé d'entre les mains de l'oppresseur ; ne maltraitez pas l'orphelin, ni l'étranger, ni la veuve ; et n'usez d'aucune violence, et ne répandez pas le sang innocent dans ce lieu-ci.
\VS{4}Car si vous mettez exactement en effet cette parole, alors les rois qui sont assis à la place de David sur son trône, montés sur des chars et sur des chevaux, entreront par les portes de cette maison, eux et leurs serviteurs, et leur peuple.
\VS{5}Mais si vous n'écoutez pas ces paroles, je le jure par moi-même\FTNT{Es. 45:23 ; Hé. 6:13.}, dit Yahweh, que cette maison deviendra une ruine.
\VS{6}Car ainsi parle Yahweh sur la maison du roi de Juda : Tu es pour moi un Galaad, et le sommet du Liban ; mais certainement, je ferai de toi un désert, une ville sans habitants.
\VS{7}Je prépare contre toi des destructeurs, chacun avec ses armes, qui couperont tes cèdres de choix, et les jetteront au feu.
\VS{8}Et plusieurs nations passeront près de cette ville, et chacun dira à son compagnon : Pourquoi Yahweh a-t-il fait ainsi à cette grande ville\FTNT{De. 29:24-28 ; 1 R. 9:8.} ?
\VS{9}Et on dira : C'est parce qu'ils ont abandonné l'alliance de Yahweh, leur Dieu, et qu'ils se sont prosternés devant d'autres dieux et les ont servis.
\TextTitle{Prophétie sur les rois de Juda : Joachaz (Schallum)}
\VS{10}Ne pleurez pas celui qui est mort, et ne vous lamentez pas sur lui ; mais pleurez amèrement celui qui s'en va, car il ne reviendra plus, il ne reverra plus le pays de sa naissance.
\VS{11}Car ainsi parle Yahweh sur Schallum, fils de Josias, roi de Juda, qui régnait à la place de Josias, son père, et qui est sorti de ce lieu : Il n'y reviendra plus ;
\VS{12}mais il mourra dans le lieu où on l'a transporté, et ne verra plus ce pays.
\TextTitle{Prophétie sur les rois de Juda : Jojakim}
\VS{13}Malheur à celui qui bâtit sa maison par l'injustice, et ses chambres hautes sans droiture ; qui fait travailler son prochain pour rien, sans lui donner le salaire de son travail\FTNT{Lé. 19:13 ; De. 24:14-15 ; Ha. 2:9.}.
\VS{14}Qui dit : Je me bâtirai une grande maison et des chambres spacieuses, et qui s'y fait percer des fenêtres ; elle est lambrissée de cèdre, et peinte de vermillon.
\VS{15}Régneras-tu, parce que tu t’enfermes dans du cèdre ? Ton père n'a-t-il pas mangé et bu ? Quand il a fait jugement et justice, alors il a prospéré.
\VS{16}Il jugeait la cause du pauvre et de l'indigent, alors il a prospéré. N'est-ce pas là me connaître ? Dit Yahweh.
\VS{17}Mais tes yeux et ton cœur ne sont adonnés qu'à ton gain déshonnête, qu'à répandre le sang innocent, qu'à faire du tord et qu'à opprimer.
\VS{18}C'est pourquoi ainsi parle Yahweh sur Jojakim, fils de Josias, roi de Juda : On ne le pleurera pas en disant : Hélas, mon frère ! et hélas, ma sœur ! On ne le pleurera pas en disant : Hélas, seigneur ! Et hélas, sa majesté !
\VS{19}Il sera enterré de la sépulture d'un âne, étant traîné et jeté hors des portes de Jérusalem.
\TextTitle{Prophétie sur les rois de Juda : Jojakin}
\VS{20}Monte sur le Liban, et crie ! Donne de la voix sur le Basan ! Crie du haut d'Abarim ! A cause que tous ceux qui t'aiment sont brisés.
\VS{21}Je t'ai parlé durant ta grande prospérité, mais tu disais : Je n'écouterai pas ; telle est ta voie depuis ta jeunesse, que tu n'as pas écouté ma voix.
\VS{22}Tous tes pasteurs seront la pâture du vent, et ceux qui t'aiment iront en captivité ; certainement tu seras alors honteuse et confuse, à cause de toute ta méchanceté.
\VS{23}Toi qui habites sur le Liban, et qui fais ton nid dans les cèdres, que tu seras à plaindre quand les douleurs t'atteindront, les douleurs comme celles d'une femme qui enfante.
\VS{24}Je suis vivant, dit Yahweh, que quand Jéconia, fils de Jojakim, roi de Juda, serait une bague à ma main droite, je t'arracherais de là.
\VS{25}Je te livrerai entre les mains de ceux qui cherchent ta vie, et entre les mains devant qui tu es craintif, et entre les mains de Nebucadnetsar, roi de Babylone, et entre les mains des Chaldéens\FTNT{2 R. 24:14 ; Ez. 17:12 ; 2 Ch. 36:10.}.
\VS{26}Et je te jetterai, toi et ta mère qui t'a enfanté, dans un autre pays où vous n'êtes pas nés, et vous y mourrez.
\VS{27}Et quant au pays qu'ils désirent pour y retourner, ils n'y retourneront pas.
\VS{28}Cet homme, Jéconia, est-il un vase méprisé et brisé ? Est-il un objet qui ne fait plus plaisir ? Pourquoi sont-ils jetés là, lui et sa postérité, lancés, dis-je, dans un pays qu'ils ne connaissent pas\FTNT{Os. 8:8.} ?
\VS{29}Ô terre, terre, terre ! Ecoute la parole de Yahweh !
\VS{30}Ainsi parle Yahweh : Ecrivez que cet homme-là est privé d'enfant, que c'est un homme qui ne prospérera pas pendant ses jours, et que même il n'y aura pas d'homme de sa postérité qui prospère, et qui soit assis sur le trône de David, ni qui domine plus en dominer sur Juda\FTNT{2 R. 24:8-16.}.
\Chap{23}
\TextTitle{Israël sera rassemblé par le Messie}
\VerseOne{}Malheur aux pasteurs qui détruisent et dispersent le troupeau de mon pâturage ! Dit Yahweh.
\VS{2}C'est pourquoi ainsi parle Yahweh, le Dieu d'Israël, sur les pasteurs qui paissent mon peuple : Vous avez dispersé\FTNT{Les brebis du Seigneur sont dispersées ou éparpillées par les faux pasteurs (Ez. 34).} mes brebis, et vous les avez chassées, et ne vous en êtes pas occupés ; voici, je vous punirai à cause de la méchanceté de vos actions, dit Yahweh.
\VS{3}Mais je rassemblerai le reste de mes brebis de tous les pays où je les ai chassées ; et je les ramènerai à leur pâturage, et elles seront fécondes et multiplieront.
\VS{4}Je susciterai aussi sur elles des pasteurs qui les paîtront, et elles n'auront plus de peur, et ne s'épouvanteront plus, et il n'en manquera aucune, dit Yahweh.
\VS{5}Voici, les jours viennent, dit Yahweh, où je susciterai à David un Germe juste, qui régnera en Roi ; il prospérera, et exercera le droit et la justice dans le pays\FTNT{Es. 4:2 ; Za. 6:12-13 ; Ps. 96:13 ; Lu. 1:32-33.}.
\VS{6}En son temps, Juda sera sauvé, Israël demeurera en sécurité ; et c'est ici le nom dont on l'appellera : Yahweh notre justice.
\VS{7}C'est pourquoi, voici, les jours viennent, dit Yahweh, qu'on ne dira plus : Yahweh est vivant, lui qui a fait monter les fils d'Israël du pays d'Egypte !
\VS{8}Mais : Yahweh est vivant, lui qui a fait monter et qui a ramené la postérité de la maison d'Israël, du pays du nord et de tous les pays où je les avais chassés, et ils habiteront dans leur pays.
\TextTitle{Jugement sur les faux prophètes}
\VS{9}A cause des prophètes mon cœur est brisé au-dedans de moi, tous mes os se relâchent ; je suis comme un homme ivre, et comme un homme que le vin a surmonté, à cause de Yahweh, et à cause des paroles de sa sainteté.
\VS{10}Car le pays est rempli d'hommes qui commettent l'adultère ; et le pays est en deuil à cause de la malédiction : Les pâturages du désert sont desséchés, leur course ne va qu'au mal, et leur force à ce qui n'est pas droit.
\VS{11}Car le prophète et le sacrificateur sont corrompus ; j'ai même trouvé dans ma maison leur méchanceté, dit Yahweh.
\VS{12}C'est pourquoi leur chemin sera comme des lieux glissants dans l'obscurité, ils y seront poussés et ils tomberont\FTNT{Ps. 35:6 ; Pr. 4:19.} ; car je ferai venir du mal sur eux, dans l'année de leur châtiment, dit Yahweh.
\VS{13}Or j'ai vu de la folie dans les prophètes de Samarie, car ils prophétisaient par Baal, et faisaient égarer mon peuple Israël.
\VS{14}Mais j'ai vu des choses horribles dans les prophètes de Jérusalem car ils commettent des adultères, et ils marchent dans le mensonge ; ils fortifient les mains de ceux qui font le mal, afin qu'aucun ne se détourne de sa méchanceté ; ils me sont tous comme Sodome, et les habitants de la ville comme Gomorrhe\FTNT{Es. 1:9.}.
\VS{15}C'est pourquoi, ainsi parle Yahweh des armées sur les prophètes : Voici, je vais leur faire manger de l'absinthe, et leur ferai boire des eaux empoisonnées ; car c'est par les prophètes que la profanation est venue dans tout le pays.
\VS{16}Ainsi parle Yahweh des armées : N'écoutez pas les paroles des prophètes qui vous prophétisent ! Ils vous font devenir vains, ils disent les visions de leur cœur, et ils ne les tiennent pas de la bouche de Yahweh.
\VS{17}Ils ne cessent de dire à ceux qui me méprisent : Yahweh a dit : Vous aurez la paix ; et ils disent à tous ceux qui marchent suivant les penchants de leur cœur : Il ne vous arrivera aucun mal\FTNT{Ez. 13:10.}.
\VS{18}Car qui s'est trouvé au conseil secret de Yahweh ? Et qui a aperçu et entendu sa parole\FTNT{Es. 40:13 ; Job. 15:8 ; 1 Co. 2:16.} ? Qui a été attentif à sa parole, et l'a entendue ?
\VS{19}Voici la tempête de Yahweh, sa fureur va se montrer, et le tourbillon prêt à fondre tombera sur la tête des méchants.
\VS{20}La colère de Yahweh ne se détournera pas jusqu'à ce qu'il ait accompli, exécuté les desseins de son cœur. Vous aurez une claire intelligence de ceci dans les derniers jours\FTNT{Gn. 49:1-2.}.
\VS{21}Je n'ai pas envoyé ces prophètes-là, et ils ont couru ; je ne leur ai pas parlé, et ils ont prophétisé.
\VS{22}S'ils s'étaient trouvés dans mon conseil secret, ils auraient aussi fait entendre mes paroles à mon peuple, et ils les auraient ramenés de leur mauvaise voie, de la méchanceté de leurs actions.
\VS{23}Suis-je un Dieu de près, dit Yahweh, et ne suis-je pas aussi un Dieu de loin ?
\VS{24}Quelqu'un se cachera-t-il dans un lieu secret sans que je ne le voie ? Dit Yahweh. Ne remplis-je pas, moi, les cieux et la terre ? Dit Yahweh\FTNT{Ps. 139:7-8 ; Am. 9:2-3.}.
\VS{25}J'ai entendu ce que les prophètes disent, prophétisant le mensonge en mon nom, et disant : J'ai eu un songe ! J'ai eu un songe !
\VS{26}Jusqu'à quand ceci sera-t-il au cœur des prophètes qui prophétisent le mensonge, et qui prophétisent la tromperie de leur cœur ?
\VS{27}Qui pensent comment ils feront oublier mon nom à mon peuple, par les songes que chacun d'eux raconte à son compagnon, comme leurs pères ont oublié mon Nom pour Baal\FTNT{Jg. 2:13.}.
\VS{28}Que le prophète qui a eu un songe raconte ce songe ; et que celui qui a ma parole proclame ma parole en vérité. Quelle convenance y a-t-il entre la paille et le froment ? Dit Yahweh.
\VS{29}Ma parole n'est-elle pas comme un feu, dit Yahweh, et comme un marteau qui brise le roc ?
\VS{30}C'est pourquoi voici, dit Yahweh, j'en veux aux prophètes qui se dérobent mes paroles l'un à l'autre.
\VS{31}Voici, dit Yahweh, j'en veux aux prophètes qui accommodent leurs langues, et qui disent : Il dit.
\VS{32}Voici, dit Yahweh, j'en veux à ceux qui prophétisent des songes faux, et qui les racontent, et font égarer mon peuple par leurs mensonges, et par leur témérité, quoique je ne les ai pas envoyés, et que je ne leur aie pas donné d'ordre ; c'est pourquoi ils ne sont d'aucune utilité à ce peuple, dit Yahweh\FTNT{So. 3:4.}.
\VS{33}Si donc ce peuple t'interroge, ou qu'il interroge le prophète, ou le sacrificateur, en disant : Quel est l'oracle de Yahweh ? Tu leur diras : Quel est cet oracle ? Je vous abandonnerai, dit Yahweh.
\VS{34}Et quant au prophète, et au sacrificateur, et au peuple qui dira : Oracle de Yahweh ; je punirai cet homme-là et sa maison.
\VS{35}Vous direz ainsi chacun à son compagnon, et chacun à son frère : Qu'a répondu Yahweh ? Qu'a dit Yahweh ?
\VS{36}Et vous ne mentionnerez plus : Oracle de Yahweh ; car la parole de chacun sera pour lui un oracle ; parce que vous tordez les paroles du Dieu vivant\FTNT{2 Pi. 3:15-16.}, les paroles de Yahweh des armées, notre Dieu.
\VS{37}Tu diras au prophète : Que t'a répondu Yahweh et que t'a dit Yahweh ?
\VS{38}Et si vous dites : Oracle de Yahweh ; à cause de cela, parle Yahweh, parce que vous dites cette parole : Oracle de Yahweh ; et que j'ai envoyé vers vous pour dire : Ne dites plus : Oracle de Yahweh !
\VS{39}A cause de cela, me voici, et je vous oublierai entièrement, et je vous rejetterai loin de ma présence, vous et la ville que j'ai donnée à vous et à vos pères.
\VS{40}Et je mettrai sur vous un opprobre éternel et une honte éternelle, qui ne s'oublieront pas.
\Chap{24}
\TextTitle{Bonnes figues: Futur retour en Juda des captifs de Babylone ; mauvaises figues: Jugement sur Jérusalem}
\VerseOne{}Yahweh me fit voir une vision, et voici deux paniers de figues posés devant le temple de Yahweh, après que Nebucadnetsar, roi de Babylone, eut transporté de Jérusalem, Jéconia, fils de Jojakim, roi de Juda, les chefs de Juda, avec les charpentiers et les serruriers, et les eut conduits à Babylone.
\VS{2}L'un des paniers avait de très bonnes figues, comme les figues de la première récolte ; et l'autre panier avait de très mauvaises figues, qu'on ne pouvait manger à cause de leur mauvaise qualité.
\VS{3}Et Yahweh me dit : Que vois-tu, Jérémie ? Et je répondis : Des figues. Les bonnes figues sont très bonnes, et les mauvaises sont très mauvaises et ne peuvent être mangées à cause de leur mauvaise qualité.
\VS{4}Alors la parole de Yahweh me fut adressée, en disant :
\VS{5}Ainsi parle Yahweh, le Dieu d'Israël : Comme tu distingues ces bonnes figues, ainsi je me souviendrai, pour leur faire du bien, des captifs de Juda, que j'ai envoyés hors de ce lieu dans le pays des Chaldéens.
\VS{6}Et je les regarderai d'un œil favorable, et je les ramènerai dans ce pays, je les y rétablirai et je ne les détruirai plus, je les planterai et je ne les arracherai pas.
\VS{7}Et je leur donnerai un cœur pour me connaître, pour connaître, dis-je, que je suis Yahweh ; et ils seront mon peuple, et je serai leur Dieu : Car ils reviendront à moi de tout leur cœur\FTNT{De. 30:6 ; Ez. 11:19.}.
\VS{8}Et comme les mauvaises figues, qui ne peuvent être mangées à cause de leur mauvaise qualité ; ainsi certainement, dit Yahweh, je ferai devenir Sédécias, roi de Juda, ses chefs, et le reste de Jérusalem, ceux qui sont restés dans ce pays, et ceux qui habitent dans le pays d'Egypte.
\VS{9}Et je les livrerai pour être agités pour leur malheur par tous les royaumes de la terre, et pour être en opprobre, en de proverbe, en raillerie, et en malédiction, par tous les lieux où je les aurai chassé\FTNT{De. 28:37.}.
\VS{10}Et j'enverrai sur eux l'épée, la famine et la peste, jusqu'à ce qu'ils soient consumés du pays que j'avais donné à eux et à leurs pères.
\Chap{25}
\TextTitle{Prophétie sur les soixante-dix ans de captivité babylonienne\FTNTT{Da.9:2.}}
\VerseOne{}La parole qui fut adressée à Jérémie touchant tout le peuple de Juda, la quatrième année de Jojakim, fils de Josias, roi de Juda, qui est la première année de Nebucadnetsar, roi de Babylone,
\VS{2}parole que Jérémie, le prophète, prononça à tout le peuple de Juda, et à tous les habitants de Jérusalem, en disant :
\VS{3}Depuis la treizième année de Josias, fils d'Amon, roi de Juda, jusqu'à ce jour, qui est la vingt-troisième année, la parole de Yahweh m'a été adressée ; et je vous ai parlé, me levant dès le matin et parlant, et vous n'avez pas écouté.
\VS{4}Et Yahweh vous a envoyé tous ses serviteurs, les prophètes, se levant dès le matin et les envoyant ; et vous ne les avez pas écoutés, vous n'avez pas prêté l'oreille pour écouter.
\VS{5}Lorsqu'ils disaient : Détournez-vous maintenant chacun de sa mauvaise voie et de la méchanceté de vos actions, et vous habiterez d'un siècle à l'autre dans le pays que Yahweh a donné à vous et à vos pères\FTNT{Jon. 3:8 ; 2 R. 17:13.} ;
\VS{6}et n'allez pas après d'autres dieux pour les servir et pour vous prosterner devant eux, ne m'irritez pas par les œuvres de vos mains, et je ne vous ferai aucun mal.
\VS{7}Mais vous ne m'avez désobéi, dit Yahweh, pour m'irriter par les œuvres de vos mains, pour votre malheur.
\VS{8}C'est pourquoi ainsi parle Yahweh des armées : Parce que vous n'avez pas écouté mes paroles,
\VS{9}voici j'enverrai et j'assemblerai toutes les familles du nord, dit Yahweh, et j'enverrai, dis-je, vers Nebucadnetsar, roi de Babylone, mon serviteur ; et je les ferai venir contre ce pays et contre ses habitants, et contre toutes ces nations d'alentour, je les détruirai à la façon de l'interdit, je les mettrai en désolation, en opprobre et en ruines éternelles\FTNT{De. 28:37 ; Es. 10:6.}.
\VS{10}Et je ferai cesser parmi eux les cris de joie et les cris d'allégresse, la voix de l'époux et la voix de l'épouse, le bruit des meules et la lumière des lampes\FTNT{Es. 24:7 ; Ez. 26:13.}.
\VS{11}Et tout ce pays sera une ruine jusqu'à s'en étonner, et ces nations seront asservies au roi de Babylone pendant soixante-dix ans\FTNT{Voir Jé. 29 : 10. Les soixante-dix ans se rapportent également au temps de la domination mondiale babylonienne. Le peuple avait une dette envers Yahweh de 70 ans de sabbats (Lé. 26 34-43 ; 2 Ch. 36:21).}.
\TextTitle{Jugement sur Babylone et les nations impies}
\VS{12}Et il arrivera que quand ces soixante-dix ans seront accomplis, je punirai le roi de Babylone et cette nation, dit Yahweh, à cause de leurs iniquités ; je punirai le pays des Chaldéens, que je mettrai en désolations éternelles\FTNT{Da. 9:2.}.
\VS{13}Et je ferai venir sur ce pays-là toutes mes paroles que j'ai prononcées contre lui, toutes les choses qui sont écrites dans ce livre, ce que Jérémie a prophétisé contre toutes ces nations.
\VS{14}Car de grands rois aussi et de grandes nations se serviront d'eux, et je leur rendrai selon leurs actions et selon l'œuvre de leurs mains.
\VS{15}Car ainsi m'a parlé Yahweh, le Dieu d'Israël : Prends de ma main cette coupe du vin, savoir de cette fureur-ci, et fais-la boire à toutes les nations vers lesquelles je t'enverrai\FTNT{Ab. 16.}.
\VS{16}Ils en boiront, et ils chancelleront et seront comme fous, à cause de l'épée que j'enverrai parmi eux.
\VS{17}Je pris donc la coupe de la main de Yahweh, et je la fis boire à toutes les nations vers lesquelles Yahweh m'envoyait :
\VS{18}Savoir : A Jérusalem et aux villes de Juda, à ses rois et à ses chefs, pour les mettre en désolation, en étonnement, en opprobre et en malédiction, comme il paraît aujourd'hui ;
\VS{19}à Pharaon, roi d'Egypte, à ses serviteurs, à ses chefs, et à tout son peuple ;
\VS{20}et à tout le mélange des peuples d'Arabie, à tous les rois du pays d'Uts, à tous les rois du pays des Philistins, à Askalon, à Gaza, à Ekron, et au reste d'Asdod ;
\VS{21}à Edom, à Moab, et aux fils d'Ammon ;
\VS{22}à tous les rois de Tyr, à tous les rois de Sidon, et aux rois des îles qui sont au-delà de la mer ;
\VS{23}à Dedan, à Théma, à Buz, et à tous ceux qui se coupent les coins de la barbe ;
\VS{24}à tous les rois d'Arabie, et à tous les rois des Arabes qui habitent au désert ;
\VS{25}à tous les rois de Zimri, à tous les rois d'Elam, et à tous les rois de Médie ;
\VS{26}à tous les rois du nord, tant proches qu'éloignés l'un de l'autre, et à tous les royaumes du monde qui sont sur la face de la terre. Et le roi de Schéschac boira après eux.
\VS{27}Et tu leur diras : Ainsi parle Yahweh des armées, le Dieu d'Israël : Buvez et soyez enivrés, même vomissez, et tombez sans vous relever, à cause de l'épée que j'enverrai parmi vous !
\VS{28}Or il arrivera qu'ils refuseront de prendre la coupe de ta main pour boire; mais tu leur diras : Ainsi parle Yahweh des armées : Vous en boirez certainement !
\VS{29}Car voici, je commence à envoyer du mal sur la ville sur laquelle mon nom est invoqué ; et vous, en seriez exempts en quelque sorte ? Vous n'en serez pas exempts ; car je m'en vais appeler l'épée sur tous les habitants de la terre, dit Yahweh des armées\FTNT{1 Pi. 4:17-18.}.
\VS{30}Tu prophétiseras donc contre eux toutes ces paroles-là, et tu leur diras : Yahweh rugira d'en haut ; il fera entendre sa voix de la demeure de sa sainteté ; il rugira, il rugira contre son son agréable demeure ; il poussera un cri  contre tous les habitants de la terre\FTNT{Joë. 3:16 ; Am. 1:2.}, comme ceux qui foulent au pressoir,
\VS{31}le son éclatant est parvenu jusqu'à l'extrémité de la terre ; car Yahweh plaide avec les nations, et il conteste contre toute chair. On livrera les méchants à l'épée, dit Yahweh.
\VS{32}Ainsi parle Yahweh des armées : Voici, le mal va sortir d'une nation à l'autre, et une grande tempête se se lèvera des extrémités de la terre.
\VS{33}Et en ce jour-là, ceux qui auront été mis à mort par Yahweh seront étendus d'un bout de la terre à l'autre bout ; ils ne seront ni pleurés, ni recueillis, ni enterrés, mais ils seront comme du fumier sur la face du sol.
\VS{34}Vous, pasteurs, hurlez et criez ! Et vous, les nobles du troupeau, roulez-vous dans la cendre ; car les jours pour vous massacrer sont accomplis. Je vous disperserai et vous tomberez comme un vase précieux.
\VS{35}Et les pasteurs n'auront aucun moyen de s'enfuir, ni les nobles d'échapper. 
\VS{36}Il y aura la voix du cri des bergers, les hurlements des nobles du troupeau ; parce que Yahweh s'en va ravager leur pâturage.
\VS{37}Et les demeures paisibles seront abattues, à cause de l'ardeur de la colère de Yahweh.
\VS{38}Il a abandonné son tabernacle comme un lionceau ; car leur pays est réduit en désert, à cause de l'ardeur du destructeur et à cause, dis-je, de l'ardeur de sa colère.
\Chap{26}
\TextTitle{Avertissement dans le parvis du temple}
\VerseOne{}Au commencement du règne de Jojakim, fils de Josias, roi de Juda, cette parole fut adressée à Jérémie part Yahweh, en disant :
\VS{2}Ainsi parle Yahweh : Tiens-toi debout dans le parvis de la maison de Yahweh, et prononce à toutes les villes de Juda qui viennent pour se prosterner dans la maison de Yahweh toutes les paroles que je t'ordonne de leur dire ; n'en retranche pas un mot.
\VS{3}Peut-être qu'ils écouteront et qu'ils se détourneront chacun de sa mauvaise voie ; et je me repentirai du mal que j'avais pensé leur faire à cause de la méchanceté de leurs actions.
\VS{4}Tu leur diras donc : Ainsi parle Yahweh : Si vous ne m'écoutez pas pour marcher selon ma loi que je vous ai proposée,
\VS{5}pour obéir aux paroles des prophètes, mes serviteurs, que je vous envoie, me levant dès le matin, et les envoyant, lesquels que vous n'avez pas écoutés,
\VS{6}je mettrai cette maison dans le même état que Silo, et je livrerai cette ville à la malédiction, à toutes les nations de la terre.
\VS{7}Or les sacrificateurs, les prophètes, et tout le peuple, entendirent Jérémie prononcer ces paroles dans la maison de Yahweh.
\TextTitle{Jérémie menacé de mort par les sacrificateurs et les prophètes}
\VS{8}Et il arrivera qu'aussitôt que Jérémie eut achevé prononcer tout ce que Yahweh lui avait ordonné de dire à tout le peuple, les sacrificateurs, les prophètes, et tout le peuple, le saisirent en disant : Tu mourras, tu mourras\FTNT{Gn. 2:17.} !
\VS{9}Pourquoi as-tu prophétisé au nom de Yahweh, en disant : Cette maison sera comme Silo, et cette ville sera déserte tellement que personne n'y habitera ? Et tout le peuple s'assembla autour de Jérémie dans la maison de Yahweh.
\VS{10}Et les les chefs de Juda ayant entendu toutes ces choses, montèrent de la maison du roi à la maison de Yahweh, et s'assirent à l'entrée de la porte neuve de la maison de Yahweh.
\VS{11}Et les sacrificateurs et les prophètes parlèrent aux chefs et à tout le peuple, en disant : Cet homme mérite d'être condamné à la mort ; car il a prophétisé contre cette ville, comme vous l'avez entendu de vos oreilles.
\VS{12}Et Jérémie parla à tous les chefs et à tout le peuple, en disant : Yahweh m'a envoyé pour prophétiser contre cette maison et contre cette ville toutes les paroles que vous avez entendues.
\VS{13}Maintenant donc, amendez votre conduite et vos actions, écoutez la voix de Yahweh, votre Dieu, et Yahweh se repentira du mal qu'il a prononcé contre vous.
\VS{14}Pour moi, me voici entre vos mains ; faites-moi ce qui vous semblera bon et juste.
\VS{15}Mais sachez comme une chose certaine, que si vous me faites mourir, vous mettrez du sang innocent sur vous, sur cette ville et sur ses habitants ; car en vérité Yahweh m'a envoyé vers vous pour prononcer à vos oreilles toutes ces paroles.
\VS{16}Alors les chefs et tout le peuple dirent aux sacrificateurs et aux prophètes : Cet homme ne mérite pas d'être condamné à la mort, car il nous a parlé au nom de Yahweh, notre Dieu.
\VS{17}Et quelques-uns des anciens du pays se levèrent et parlèrent à toute l'assemblée du peuple, en disant :
\VS{18}Michée, de Moréscheth, prophétisait aux jours d'Ezéchias, roi de Juda, et il parlait à tout le peuple de Juda, en disant : Ainsi parle Yahweh des armées : Sion sera labourée comme un champ, Jérusalem sera réduite en un monceau de pierres, et la montagne du temple en des hauts lieux d'une forêt\FTNT{Mi. 1:1 ; Mi. 3:12.}.
\VS{19}Ezéchias, roi de Juda, et tous ceux de Juda l'ont-ils fait mourir ? Ezéchias ne craignit-il pas Yahweh ? N'implora-t-il pas Yahweh ? Et Yahweh se repentit du mal qu'il avait prononcé contre eux. Et nous, nous ferions donc un grand mal contre nos âmes\FTNT{2 Ch. 32:26.} !
\VS{20}Mais aussi dirent les autres, y eut aussi un homme qui prophétisait au nom de Yahweh, savoir, Urie, fils de Schemaeja, de Kirjath-Jearim. Il prophétisa contre cette même ville et contre ce même pays, de la même manière que Jérémie.
\VS{21}Et le roi Jojakim, tous ses vaillants hommes, et tous ses chefs entendirent ses paroles, et le roi chercha à le faire mourir. Urie, qui en fut informé, eut peur, prit la fuite, et se retira en Egypte.
\VS{22}Et le roi Jojakim envoya des hommes en Egypte, savoir, Elnathan, fils d'Acbor, et quelques hommes avec lui, qui allèrent en Egypte.
\VS{23}Et ils firent sortir d'Egypte Urie et l'amenèrent au roi Jojakim qui le frappa avec l'épée et jeta son cadavre sur les sépulcres des fils du peuple.
\VS{24}Toutefois la main d'Achikam, fils de Schaphan, fut avec Jérémie, et empêcha qu'il ne soit livré au peuple pour être mis à mort.
\Chap{27}
\TextTitle{Prophétie : Les nations seront asservies à Nebucadnetsar}
\VerseOne{}Au commencement du règne de Jojakim\FTNT{Il est probable que ce soit une erreur de copiste, car bien que l'hébreu dit « Jojakim », le contexte se rapporte à Sédécias. Voir Jé. 27:3 ; Jé. 27:12 ; Jé. 27:20 ; Jé 28:1.}, fils de Josias, roi de Juda, cette parole fut adressée à Jérémie de la part de Yahweh, en disant :
\VS{2}Ainsi m'a parlé Yahweh : Fais-toi des liens et des jougs, et mets-les sur ton cou\FTNT{Ez. 7:23.}.
\VS{3}Et envoie-les au roi d'Edom, et au roi de Moab, et au roi des fils d'Ammon, et au roi de Tyr et au roi de Sidon, par les mains des messagers qui sont venus à Jérusalem vers Sédécias, roi de Juda ;
\VS{4}et tu leur donneras mes ordres pour leurs maîtres, en disant : Ainsi parle Yahweh des armées, le Dieu d'Israël : Vous direz ainsi à vos maîtres :
\VS{5}J'ai fait la terre, les hommes et les bêtes qui sont sur la terre, par ma grande force et par mon bras étendu, et je la donne à qui cela me plaît\FTNT{De. 32:8.}.
\VS{6}Et maintenant j'ai livré tous ces pays entre les mains de Nebucadnetsar, roi de Babylone, mon serviteur ; et même je lui ai donné les bêtes des champs pour qu'elles lui soient asservies\FTNT{Da. 2:38.}.
\VS{7}Et toutes les nations lui seront asservies, à lui, à son fils, et au fils de son fils, jusqu'à ce que le temps de son pays vienne aussi, et que plusieurs nations et de grands rois l'asservissent.
\VS{8}Et il arrivera que la nation ou le royaume qui ne se soumettra pas à lui, à Nebucadnetsar, roi de Babylone, et qui ne soumettra pas son cou au joug du roi de Babylone, je punirai cette nation par l'épée, par la famine et par la peste, dit Yahweh, jusqu'à ce que je les aie consumés par sa main.
\VS{9}Vous donc, n'écoutez pas vos prophètes, ni vos devins, ni vos songeurs, ni vos augures, ni vos magiciens, qui vous parlent, en disant : Vous ne serez pas asservis au roi de Babylone.
\VS{10}Car ils vous prophétisent le mensonge pour vous faire aller loin de votre pays, afin que je vous chasse et que vous périssiez.
\VS{11}Mais la nation qui livrera son cou au joug du roi de Babylone, et qui le servira, je la laisserai dans son pays, dit Yahweh, pour qu'elle le cultive et qu'elle y demeure.
\VS{12}Puis j'ai parlé à Sédécias, roi de Juda, selon toutes ces paroles-là, en disant : Soumettez votre cou au joug du roi de Babylone, et rendez-vous sujets, à lui et son peuple, et vous vivrez.
\VS{13}Pourquoi mourriez-vous, toi et ton peuple, par l'épée, par la famine et par la peste, selon que Yahweh a parlé contre la nation qui ne sera pas soumise au roi de Babylone ?
\VS{14}N'écoutez donc pas les paroles des prophètes qui vous parlent en disant : Vous ne serez pas asservis au roi de Babylone ! Car ils vous prophétisent le mensonge.
\VS{15}Même je ne les ai pas envoyés, dit Yahweh, et ils vous prophétisent faussement en mon nom, afin que je vous rejette et que vous périssiez, vous et les prophètes qui vous prophétisent.
\VS{16}J'ai aussi parlé aussi aux sacrificateurs et à tout ce peuple, en disant : Ainsi parle Yahweh : N'écoutez pas les paroles de vos prophètes qui vous prophétisent, en disant : Voici, les ustensiles de la maison de Yahweh seront bientôt rapportés de Babylone ! Car ils vous prophétisent le mensonge.
\VS{17}Ne les écoutez donc pas, rendez-vous sujets au roi de Babylone, et vous vivrez. Pourquoi cette ville serait-elle réduite en un désert ?
\VS{18}Et s'ils sont prophètes et si la parole de Yahweh est en eux, qu'ils intercèdent maintenant auprès de Yahweh des armées, afin que les ustensiles qui restent dans la maison de Yahweh, dans la maison du roi de Juda, et dans Jérusalem, n'aillent pas à Babylone.
\VS{19}Car ainsi parle Yahweh des armées au sujet des colonnes, de la mer, des bases, et des autres ustensiles qui sont restés dans cette ville,
\VS{20}que Nebucadnetsar, roi de Babylone, n'a pas emportés, quand il a transporté de Jérusalem à Babylone, Jéconia, fils de Jojakim, roi de Juda, et tous les nobles de Juda et de Jérusalem,
\VS{21}Yahweh, dis-je, des armées le Dieu d'Israël, parle ainsi au sujet des ustensiles qui restent dans la maison de Yahweh, dans la maison du roi de Juda et dans Jérusalem :
\VS{22}Ils seront emportés à Babylone, et ils y demeureront jusqu'au jour où je les visiterai, dit Yahweh, et où je les ferai remonter et revenir dans ce lieu\FTNT{2 R. 24:14-15 ; Esd. 1:7-11 ; 2 Ch. 25:13-16 ; 2 Ch. 36:18.}.
\Chap{28}
\TextTitle{Hanania meurt suite à sa prophétie mensongère}
\VerseOne{}Il arriva aussi, en cette même année, au commencement du règne de Sédécias, roi de Juda, savoir, au cinquième mois de la quatrième année, que Hanania, fils d'Azzur, prophète de Gabaon, me parla dans la maison de Yahweh, aux yeux des sacrificateurs et de tout le peuple, en disant :
\VS{2}Ainsi parle Yahweh des armées, le Dieu d'Israël : Je romps le joug du roi de Babylone !
\VS{3}Dans deux années accomplis, et je ferai rapporter dans ce lieu tous les ustensiles de la maison de Yahweh, que Nebucadnetsar, roi de Babylone, a pris de ce lieu, et qu'il a transportés à Babylone.
\VS{4}Et je ferai revenir dans ce lieu, dit Yahweh, Jéconia, fils de Jojakim, roi de Juda, et tous les captifs de Juda qui sont allés à Babylone ; car je romprai le joug du roi de Babylone.
\VS{5}Alors Jérémie, le prophète, répondit à Hanania, le prophète, aux yeux des sacrificateurs, et aux yeux de tout le peuple qui se tenait dans la maison de Yahweh.
\VS{6}Et Jérémie, le prophète, dit : Ainsi soit-il ! Que Yahweh fasse ainsi ! Que Yahweh accomplisse les paroles que tu as prophétisées, et qu'il fasse revenir de Babylone dans ce lieu-ci les ustensiles de la maison de Yahweh, et tous les captifs de Babylone !
\VS{7}Toutefois, écoute maintenant cette parole que je prononce, à tes oreilles et aux oreilles de tout le peuple :
\VS{8}Les prophètes qui ont été avant moi et avant toi, dès les temps anciens, ont prophétisé contre plusieurs pays et de grands royaumes, la guerre, le malheur et la peste ;
\VS{9}Le prophète qui aura prophétisé la paix, quand la parole de ce prophète sera accomplie, ce prophète-là sera reconnu pour avoir été véritablement envoyé par Yahweh.
\VS{10}Alors Hanania, le prophète, prit le joug de dessus le cou de Jérémie, le prophète, et le rompit.
\VS{11}Puis Hanania parla aux yeux de tout le peuple, en disant : Ainsi parle Yahweh : C'est ainsi que dans deux années, je romprai le joug de Nebucadnetsar, roi de Babylone, de dessus le cou de toutes les nations. Et Jérémie, le prophète, alla au loin par la route.
\VS{12}Mais la parole de Yahweh fut adressée à Jérémie, après que Hanania, le prophète, eut rompu le joug de dessus le cou de Jérémie, le prophète, en disant :
\VS{13}Va, et parle à Hanania, en disant : Ainsi parle Yahweh : Tu as rompu les jougs de bois, et tu auras à la place un joug de fer.
\VS{14}Car ainsi parle Yahweh des armées, le Dieu d'Israël : Je mets un joug de fer sur le cou de toutes ces nations, afin qu'elles servent Nebucadnetsar, roi de Babylone, et elles le serviront ; et je lui donne aussi les bêtes des champs\FTNT{De. 28:48.}.
\VS{15}Puis Jérémie, le prophète, dit à Hanania, le prophète : Ecoute maintenant, ô Hanania ! Yahweh ne t'a pas envoyé, et tu as fait que ce peuple se confie au mensonge\FTNT{Ez. 13:3-9.}.
\VS{16}C'est pourquoi ainsi parle Yahweh : Voici, je te chasse de la face de la terre ; et tu mourras cette année ; car tu as parlé de révolte contre Yahweh.
\VS{17}Et Hanania, le prophète, mourut cette année-là, dans le septième mois.
\Chap{29}
\TextTitle{Message à l'attention des Juifs captifs à Babylone}
\VerseOne{}Or ce sont ici les paroles de la lettre que Jérémie, le prophète, envoya de Jérusalem au reste des anciens en captivité, aux sacrificateurs et aux prophètes, et à tout le peuple, que Nebucadnetsar avait transportés de Jérusalem à Babylone,
\VS{2}après que le roi Jéconia fut sorti de Jérusalem, avec la reine, et les eunuques, et les chefs de Juda et de Jérusalem, et les charpentiers et les serruriers\FTNT{2 R. 24:12.}.
\VS{3}C'est par la main d'Eleasa, fils de Schaphan, et Guemaria, fils de Hilkija, que Sédécias, roi de Juda, l'envoya à Babylone vers Nebucadnetsar, roi de Babylone. La lettre disait :
\VS{4}Ainsi parle Yahweh des armées, le Dieu d'Israël, à tous les captifs que j'ai fait transporter de Jérusalem à Babylone.
\VS{5}Bâtissez des maisons, et habitez-les ; plantez des jardins, et mangez-en les fruits.
\VS{6}Prenez des femmes, et engendrez des fils et des filles ; prenez aussi des femmes pour vos fils, et donnez vos filles à des hommes, afin qu'elles enfantent des fils et des filles ; multipliez-vous là, et ne diminuez pas.
\VS{7}Et cherchez la paix de la ville où je vous ai transporté, et priez Yahweh pour elle ; parce que dans sa paix vous aurez la paix.
\VS{8}Car ainsi parle Yahweh des armées, le Dieu d'Israël : Que vos prophètes qui sont au milieu de vous, et vos devins, ne vous séduisent pas, et n'écoutez pas vos songes que vous vous songez\FTNT{Tous les songes ne viennent pas toujours du Seigneur. Les visions et les songes doivent être en accord avec la Parole de Dieu.}.
\VS{9}Parce qu'ils vous prophétisent faussement en mon nom. Je ne les ai pas envoyés, dit Yahweh.
\VS{10}Car ainsi parle Yahweh : Lorsque les soixante-dix ans seront accomplis pour Babylone, je vous visiterai, et j'accomplirai ma bonne parole à votre égard, pour vous faire revenir dans ce lieu.
\VS{11}Car je sais que les pensées que j'ai pour vous, dit Yahweh, sont des pensées de paix et non pas d'adversité, pour vous donner une fin telle que vous espérez\FTNT{Jos. 1:8.}.
\VS{12}Alors vous m'invoquerez, et vous partirez ; vous me prierez, et je vous exaucerai\FTNT{Os. 5:15.}.
\VS{13}Vous me chercherez, et vous me trouverez, après que vous m'aurez recherché de tout votre cœur\FTNT{Mt. 7:7.}.
\VS{14}Car je me laisserai trouver par vous, dit Yahweh, je ramènerai vos captifs ; et je vous rassemblerai d'entre toutes les nations et de tous les lieux où je vous ai chassés, dit Yahweh, et je vous ramènerai dans le lieu d'où je vous ai transportés.
\VS{15}Cependant si vous dites : Yahweh nous a suscité des prophètes à Babylone !
\VS{16} A cause de cela, ainsi parle Yahweh sur le roi qui est assis sur le trône de David, sur tout le peuple qui habite dans cette ville, sur vos frères qui ne sont pas allés avec vous en captivité ;
\VS{17}ainsi parle Yahweh des armées : Voici, je vais envoyer sur eux l'épée, la famine, et la peste, et je les ferai devenir comme des figues affreuses qui ne peuvent être mangées à cause de leur mauvaise qualité.
\VS{18}Et je les poursuivrai par l'épée, par la famine et par la peste, je les abandonnerai pour être agités par tous les royaumes de la terre, et pour être une malédiction, un étonnement, une moquerie et un opprobre parmi toutes les nations où je les chasserai\FTNT{De. 28:25-37.},
\VS{19}parce qu'ils n'ont pas écouté mes paroles, dit Yahweh, eux à qui j'ai envoyé mes serviteurs, les prophètes, en me levant dès le matin ; et ils n'ont pas écouté, dit Yahweh.
\VS{20}Vous tous donc, écoutez la parole de Yahweh, vous les captifs que j'ai envoyés de Jérusalem à Babylone !
\VS{21}Ainsi parle Yahweh des armées, le Dieu d'Israël sur Achab, fils de Kolaja, et sur Sédécias, fils de Maaséja, qui vous prophétisent faussement en mon nom : Voici, je vais les livrer entre les mains de Nebucadnetsar, roi de Babylone ; et il les frappera sous vos yeux.
\VS{22}Et on se servira d'eux comme une formule de malédiction, parmi tous les captifs de Juda qui sont à Babylone, en disant : Que Yahweh te mette dans un tel état, comme Sédécias et comme Achab, que le roi de Babylone a fait rôtir au feu !
\VS{23}parce qu'ils ont commis des impuretés en Israël, parce qu'ils ont commis l'adultère avec les femmes de leurs prochains, et qu'ils ont dit en mon nom des paroles fausses, alors que je ne leur avais pas commandées. Je le sais, et j'en suis témoin, dit Yahweh.
\VS{24}Parle aussi à Schemaeja, Néchélamite, en disant :
\VS{25}Ainsi parle Yahweh des armées, le Dieu d'Israël : Tu as envoyé en ton nom une lettre à tout le peuple de Jérusalem, à Sophonie, fils de Maaséja, le sacrificateur, et à tous les sacrificateurs, en disant :
\VS{26}Yahweh t'a établi sacrificateur à la place de Jehojada, le sacrificateur, afin qu'il y ait dans la maison de Yahweh des inspecteurs pour surveiller tout homme qui est fou et se donne pour prophète, et afin que tu le mettes en prison et dans les fers.
\VS{27}Et maintenant, pourquoi n'as-tu pas réprimé pas Jérémie d'Anathoth, qui prophétise parmi vous,
\VS{28}car à cause de cela il nous a envoyé dire à Babylone : La captivité sera longue ; bâtissez des maisons, et habitez-les ; plantez des jardins, et mangez-en les fruits !
\VS{29}Or Sophonie, le sacrificateur, lut cette lettre aux oreilles de Jérémie, le prophète.
\VS{30}C'est pourquoi la parole de Yahweh fut adressée à Jérémie, en disant :
\VS{31}Envoie dire à tous les captifs : Ainsi parle Yahweh sur Schemaeja, Néchélamite : Parce que Schemaeja vous prophétise, quoique que je ne l'aie envoyé, et qu'il vous a fait vous confier dans le mensonge,
\VS{32}à cause de cela, dit Yahweh : Je vais punir Schemaeja, Néchélamite, et sa postérité ; il n'y aura personne de sa race qui  habite au milieu de ce peuple, et il ne verra pas le bien que je ferai à mon peuple, dit Yahweh ; car il a parlé de révolte contre Yahweh.
\Chap{30}
\TextTitle{Le jour de Yahweh}
\VerseOne{}La parole qui fut adressée à Jérémie de la part de Yahweh, en disant :
\VS{2}Ainsi parle Yahweh, le Dieu d'Israël : Ecris pour toi dans un livre toutes les paroles que je t'ai dites.
\VS{3}Car voici, les jours viennent, dit Yahweh, où je ramènerai les captifs de mon peuple d'Israël et de Juda, dit Yahweh ; je les ramènerai dans le pays que j'ai donné à leurs pères, et ils le posséderont.
\VS{4}Ce sont ici les paroles que Yahweh a prononcées sur Israël et Juda.
\VS{5}Ainsi parle Yahweh : Nous entendons des cris d'effroi et de terreur, il n'y a pas de paix.
\VS{6}Informez-vous, je vous prie et voyez si un mâle enfante ! Pourquoi vois-je les hommes les mains sur leurs reins, comme une femme qui enfante ? Pourquoi tous les visages sont-ils devenus pâles ?
\VS{7}Malheur ! Que ce jour est grand ; il n'y en a pas eu de semblable. Il sera un temps de détresse pour Jacob ; mais il en sera pourtant délivré\FTNT{Joë. 2:11 ; So. 1:15 ; Da. 12:1 ; Mt. 24:21.}.
\VS{8}Et il arrivera en ce jour-là, dit Yahweh des armées, que je briserai son joug de dessus ton cou, je romprai tes liens, et les étrangers ne t'asserviront plus.
\VS{9}Mais ils serviront Yahweh, leur Dieu, et David, leur roi, que je leur susciterai\FTNT{Ez. 34:23-24.}.
\VS{10}Toi donc, mon serviteur Jacob, ne crains pas, dit Yahweh, et ne t'épouvante pas, ô Israël ! Car, voici, je te délivrerai du pays éloigné, et ta postérité du pays de leur captivité ; et Jacob reviendra, il sera en repos et sera en paix, et il n'y aura personne qui lui fasse peur\FTNT{Es. 41:13.}.
\VS{11}Car je suis avec toi, dit Yahweh, pour te délivrer ; et même je consumerai entièrement toutes les nations parmi lesquelles je t'ai dispersé, mais quand à toi, je ne te consumerai pas entièrement; je te châtierai avec équité, je ne te tiens pas entièrement pour innocent\FTNT{Es. 27:7-8.}.
\VS{12}Ainsi parle Yahweh : Ta blessure est incurable, ta plaie est très douloureuse\FTNT{Mi. 1:9 ; 2 Ch. 36:16.}.
\VS{13}Il n'y a personne qui défende ta cause, pour panser ta plaie ; il n'y a pour toi aucun remède, aucun moyen de guérison.
\VS{14}Tous tes amoureux t'oublient, ils ne te recherchent pas ; car je t'ai frappée d'une plaie d'ennemi, d'un châtiment d'homme cruel, à cause de la multitude de tes iniquités, tes péchés se sont renforcés\FTNT{La. 1:2.}.
\VS{15}Pourquoi cries-tu à cause de ta plaie ? Ta douleur est hors d'espérance ; je t'ai fait ces choses à cause de la grandeur de tes iniquités, du grand nombre de tes péchés.
\VS{16}Néanmoins tous ceux qui te dévorent seront dévorés, et tous ceux qui te mettent dans la détresse, iront en captivité ; ceux qui te dépouillent seront dépouillés, et je livrerai au pillage tous ceux qui te pillent\FTNT{Es. 41:11 ; Ab. 15.}.
\VS{17}Même je guérirai tes plaies, et je te guérirai tes blessures, dit Yahweh. Car ils t'appellent la repoussée, cette Sion que personne ne recherche.
\TextTitle{Israël délivré par Yahweh}
\VS{18}Ainsi parle Yahweh : Voici, je ramène les captifs des tentes de Jacob, j'ai compassion de ses demeures ; la ville sera rebâtie sur le monceau de ses ruines et le palais sera rétabli comme il était.
\VS{19}Et il en sortira des remerciements et des cris de joie ; et je les multiplierai, et ils ne diminueront pas ; je les honorerai, et ils ne seront pas amoindris.
\VS{20}Et ses enfants seront comme autrefois, son assemblée sera affermie devant moi, et je punirai tous ceux qui l'oppriment.
\VS{21}Et son chef sera tiré de son sein, son dominateur sortira du milieu de lui ; je le ferai approcher, et il viendra vers moi ; car qui disposerait son cœur pour venir vers moi ? dit Yahweh.
\VS{22}Et vous serez mon peuple, et je serai votre Dieu.
\VS{23}Voici, la tempête de Yahweh, la fureur éclate, un tourbillon qui s'entasse ; il tombera sur la tête des méchants. 
\VS{24}L'ardeur de la colère de Yahweh ne se détournera pas, jusqu'à ce qu'il ait exécuté, accompli les desseins de son cœur ; vous le comprendrez dans les derniers jours.
\Chap{31}
\TextTitle{Communion retrouvée : La paix et et la joie}
\VerseOne{}En ce temps-là, dit Yahweh, je serai le Dieu de toutes les familles d'Israël, et ils seront mon peuple.
\VS{2}Ainsi parle Yahweh : Le peuple survivant à l'épée a trouvé grâce dans le désert ; Israël marche vers son lieu de repos.
\VS{3}De loin Yahweh m'est apparu, et m'a dit : Je t'aime d'un amour éternel, c'est pourquoi j'ai prolongé ma bonté envers toi.
\VS{4}Je te rétablirai encore, et tu seras rétablie, ô vierge d'Israël ! Tu te pareras encore de tes tambours, et tu sortiras au milieu des danses joyeuses.
\VS{5}Tu planteras encore des vignes sur les montagnes de Samarie ; les vignerons planteront et recueilleront les fruits pour leur usage\FTNT{Es. 65:21.}.
\VS{6}Car il y a un jour où les gardes crieront sur la montagne d'Ephraïm : Levez-vous, et montons à Sion, vers Yahweh, notre Dieu !
\VS{7}Car ainsi parle Yahweh : Réjouissez-vous avec chant de triomphe, et avec allégresse à cause de Jacob, et vous égayez à cause du chef des nations ! Faites-le entendre, chantez des louanges, et dites : Yahweh, délivre ton peuple, le reste d’Israël !
\VS{8}Voici, je vais les faire venir du pays du nord, et je les rassemblerai des extrémités de la terre ; l'aveugle et le boiteux, la femme enceinte et celle qui enfante seront ensemble parmi eux ; une grande assemblée qui reviendra ici.
\VS{9}Ils y seront allés en pleurant, mais je les ferai retourner avec des supplications, et je les conduirai aux torrents d'eaux, et par un droit chemin, où ils ne broncheront pas ; car je suis un père pour Israël, et Ephraïm est mon premier-né\FTNT{Ex. 4:22.}.
\VS{10}Nations, écoutez la parole de Yahweh, et annoncez-la aux îles éloignées ! Dites : Celui qui a dispersé Israël le rassemblera, et il le gardera comme un berger garde son troupeau.
\VS{11}Car Yahweh rachète Jacob, et le retire de la main d'un ennemi plus fort que lui.
\VS{12}Ils viendront donc, et se réjouiront avec des chants de triomphe sur les hauteurs de Sion ; ils afflueront vers les biens de Yahweh, le blé, le vin, l'huile, et le fruit du gros et du menu bétail ; et leur âme sera comme un jardin arrosé, et ils ne seront plus dans la souffrance\FTNT{Es. 61:11.}.
\VS{13}Alors la vierge se réjouira à la danse, les jeunes hommes et les anciens ensemble ; je changerai leur deuil en joie, et je les consolerai ; et je les réjouirai en les délivrant de leur douleur.
\VS{14}Je rassasierai aussi de graisse l'âme des sacrificateurs, et mon peuple sera rassasié de mes biens, dit Yahweh.
\VS{15}Ainsi parle Yahweh : On entend des cris à Rama, des lamentations, des larmes amères ; Rachel pleure ses fils ; elle refuse d'être consolée sur ses fils, car ils ne sont plus\FTNT{Mt. 2:17-18.}.
\VS{16}Ainsi parle Yahweh : Retiens ta voix de pleurer, et tes yeux de verser des larmes, car ton œuvre aura son salaire, dit Yahweh ; et ils reviendront des terres de l'ennemi.
\VS{17}Et il y a de l'espérance pour tes derniers jours, dit Yahweh ; et tes fils reviendront dans leur territoire.
\VS{18}J'ai très bien entendu Ephraïm se plaignant, et disant : Tu m'as châtié, et j'ai été châtié comme un veau qui n'est pas dompté. Fais-moi revenir, et je reviendrai, car tu es Yahweh, mon Dieu\FTNT{Ps. 119:67-71.}.
\VS{19}Certes, après m'être détourné, je me repens ; et après avoir reconnu mes fautes, je frappe sur ma cuisse ; je suis honteux et confus, car je porte l'opprobre de ma jeunesse\FTNT{Ez. 21:17.}.
\VS{20}Ephraïm est-il donc pour moi un cher fils, un fils qui fait mes délices ? Car plus je parle de lui, plus encore son souvenir est en moi ; aussi mes entrailles sont émues en sa faveur : J'aurai certainement pitié de lui, dit Yahweh\FTNT{Es. 5:7.}.
\VS{21}Dresse-toi des signes sur les chemins, place des poteaux, prends garde à la route, au chemin par lequel tu es venue… Reviens, vierge d'Israël, reviens dans tes villes !
\VS{22}Jusqu'à quand seras-tu errante, fille rebelle ? Car Yahweh crée une chose nouvelle sur la terre : La femme entourera l'homme.
\VS{23}Ainsi parle Yahweh des armées, le Dieu d'Israël : On dira encore cette parole-ci dans le pays de Juda et dans ses villes, quand j'aurai ramené leurs captifs : Que Yahweh te bénisse, ô agréable demeure de la justice, montagne de sainteté !
\VS{24}Juda et toutes ses villes ensemble, les laboureurs, et ceux qui conduisent les troupeaux, y habiteront.
\VS{25}Car j'abreuverai l'âme épuisée par le travail, et je remplirai toute âme languissante.
\VS{26}C'est pourquoi je me suis réveillé, et j'ai regardé ; mon sommeil m'avait été agréable.
\TextTitle{Promesse d'une nouvelle alliance}
\VS{27}Voici, les jours viennent, dit Yahweh, où j'ensemencerai la maison d'Israël et la maison de Juda d'une semence d'hommes et d'une semence de bêtes.
\VS{28}Et il arrivera que comme j'ai veillé sur eux pour arracher et démolir, pour détruire, pour perdre et pour faire du mal ; ainsi je veillerai sur eux pour bâtir et pour planter, dit Yahweh.
\VS{29}En ces jours-là, on ne dira plus : Les pères ont mangé des raisins verts, et les dents des fils en ont été agacées\FTNT{Ez. 18:2-3.}.
\VS{30}Mais chacun mourra pour son iniquité ; tout homme qui mangera des raisins verts, ses dents en seront agacées.
\VS{31}Voici, les jours viennent, dit Yahweh, où je traiterai une nouvelle alliance\FTNT{Il s'agit de l'alliance du sang que Jésus, notre Messie, est venu inaugurer en prenant sur lui tous nos péchés et en mourant sur la croix à notre place (Mt. 26:27-29 ; Hé. 8:7-13).} avec la maison d'Israël et avec la maison de Juda,
\VS{32}non comme l'alliance que je traitai avec leurs pères, le jour où je les pris par la main, pour les faire sortir du pays d'Egypte, mon alliance qu'ils ont violée ; et toutefois j’avais été pour eux un mari, dit Yahweh.
\VS{33}Car c'est ici l'alliance que je traiterai avec la maison d'Israël, après ces jours-là, dit Yahweh, je mettrai ma loi au-dedans d'eux, je l'écrirai dans leur cœur ; et je serai leur Dieu, et ils seront mon peuple.
\VS{34}Aucun homme parmi eux n'enseignera plus son prochain, ni personne son frère, en disant : Connaissez Yahweh ! Car tous me connaîtront, depuis le plus petit jusqu'au plus grand, dit Yahweh ; parce que je pardonnerai leur iniquité, et que je ne me souviendrai plus de leur péché\FTNT{Es. 54:13 ; Ha. 2:14 ; Jn. 6:45.}.
\VS{35}Ainsi parle Yahweh, qui a donné le soleil pour être la lumière du jour, et qui a réglé la lune et les étoiles pour être la lumière de la nuit, qui remue la mer, et fait gronder ses flots, lui dont le nom est Yahweh des armées\FTNT{Ge. 1:16 ; Es. 51:15.} :
\VS{36}Si ces lois\FTNT{Les lois de l'univers ont été établies par Yahweh. Ces lois sont : la loi de la gravité, la loi de l'attraction et la loi de la résonance (voir Ps. 148:5-6 ; Job. 38:33).} viennent à cesser devant moi, dit Yahweh, la race d'Israël aussi cessera d'être à jamais une nation devant moi.
\VS{37}Ainsi parle Yahweh : Si les cieux en haut peuvent être mesurés, si les fondements de la terre en bas peuvent être sondés, alors je rejetterai toute la race d'Israël, à cause de toutes les choses qu'ils ont faites, dit Yahweh.
\VS{38}Voici, les jours viennent, dit Yahweh, où cette ville sera rebâtie à Yahweh, depuis la tour de Hananeel, jusqu'à la porte de l'angle\FTNT{Za. 14:10 ; Né. 3:1 ; 2 Ch. 26:9.}.
\VS{39}Le cordeau à mesurer sera encore tiré vis-à-vis d'elle, sur la colline de Gareb, et tournera vers Goath.
\VS{40}Et toute la vallée des cadavres et des cendres, et tous les champs jusqu'au torrent de Cédron, jusqu'à l'angle de la porte des chevaux à l'orient, seront consacrés à Yahweh, et ne seront plus jamais arrachés ni détruits.
\Chap{32}
\TextTitle{Le champ de Hanameel : La pérennité d'Israël}
\VerseOne{}La parole qui fut adressée à Jérémie de la part de Yahweh, la dixième année de Sédécias, roi de Juda. C'était la dix-huitième année de Nebucadnetsar.
\VS{2}Or l'armée du roi de Babylone assiégeait alors Jérusalem ; et Jérémie le prophète était enfermé dans la cour de la prison, qui était dans la maison du roi de Juda ;
\VS{3}car Sédécias, roi de Juda, l'avait fait enfermer, et lui avait dit : Pourquoi prophétises-tu, en disant : Ainsi parle Yahweh : Voici, je vais livrer cette ville entre les mains du roi de Babylone, et il la prendra ;
\VS{4}et Sédécias, roi de Juda, n'échappera pas aux mains des Chaldéens ; mais il sera livré entre les mains du roi de Babylone, et lui parlera bouche à bouche, et ses yeux verront les yeux de ce roi ;
\VS{5}il emmènera Sédécias à Babylone, qui y demeurera jusqu'à ce que je le visite, dit Yahweh ; si vous combattez contre les Chaldéens, vous ne prospérerez pas.
\VS{6}Jérémie donc dit : La parole de Yahweh m'a été adressée, en disant :
\VS{7}Voici Hanameel, fils de Schallum, ton oncle, qui vient vers toi pour te dire : Achète mon champ qui est à Anathoth, car tu as le droit de rachat pour l'acquérir\FTNT{Lé. 25:48 ; Ru. 3:12.}.
\VS{8}Hanameel donc, fils de mon oncle, vint à moi, selon la parole de Yahweh, dans la cour de la prison, et me dit : Achète, je te prie, mon champ, qui est à Anathoth, dans le pays de Benjamin, car tu as le droit d'héritage et de rachat, achète-le ! Et je connus alors que c'était la parole de Yahweh.
\VS{9}Ainsi j'achetai de Hanameel, fils de mon oncle, le champ qui est à Anathoth, et je lui pesai l'argent, qui fut dix-sept sicles d'argent.
\VS{10}Puis j'écrivis le contrat, que je cachetai, je pris des témoins après avoir pesé l'argent sur la balance.
\VS{11}Et je pris le contrat d'acquisition, celui qui était cacheté, selon les ordonnances et les statuts, et celui qui était ouvert ;
\VS{12}Et je remis le contrat d'acquisition à Baruc, fils de Nérija, fils de Machséja, sous les yeux de Hanameel, fils de mon oncle, des témoins qui avaient signé le contrat d'acquisition, et sous les yeux de tous les juifs qui étaient assis dans la cour de la prison.
\VS{13}Puis je donnai sous leurs yeux cet ordre à Baruc, en disant :
\VS{14}Ainsi parle Yahweh des armées, le Dieu d'Israël : Prends ces contrats-ci, à savoir, ce contrat d'acquisition, celui qui est scellé, et celui qui est ouvert, et mets-les dans un vase de terre, afin qu'ils se conservent longtemps.
\VS{15}Car ainsi parle Yahweh des armées, le Dieu d'Israël : On achètera encore des maisons, des champs et des vignes, dans ce pays.
\TextTitle{Promesse du retour des Juifs en Israël}
\VS{16}Et après que j'eus donné à Baruc, fils de Nérija, le contrat d'acquisition, je fis cette prière à Yahweh, en disant :
\VS{17}Ah ! Ah ! Seigneur Yahweh, voici, tu as fait les cieux et la terre par ta grande puissance et par ton bras étendu : Aucune chose n'est étonnante de ta part.
\VS{18}Tu fais miséricorde jusqu'à la millième génération, et tu punis l'iniquité des pères dans le sein de leurs fils après eux\FTNT{Ex. 34:7 ; Es. 65:7 ; Ps. 79:12.}. Tu es le Dieu, le Grand, le Puissant, dont le nom est Yahweh des armées.
\VS{19}Tu es grand en conseil et puissant en actions ; tes yeux sont ouverts sur toutes les voies des fils des hommes, pour rendre à chacun selon ses voies, et selon le fruit de ses œuvres.
\VS{20}Tu as fait dans le pays d'Egypte des miracles et des prodiges qui sont connus jusqu'à ce jour, et en Israël et parmi les hommes, tu t'es fait un nom tel qu'il est aujourd'hui.
\VS{21}Car tu as fait sortir du pays d'Egypte ton peuple d'Israël, avec des miracles et des prodiges, et avec une main forte, et avec un bras étendu, et en répandant partout une grande terreur ;
\VS{22}Et tu leur as donné ce pays, que tu avais juré à leurs pères de leur donner, pays où coulent le lait et le miel.
\VS{23}Et ils y sont entrés, ils l'ont possédé ; mais ils n'ont pas obéi à ta voix, et n'ont pas marché dans ta loi, et n'ont pas fait tout ce que tu leur avais ordonné de faire. C'est pourquoi tu as fait arriver sur eux tout ce mal-ci !
\VS{24}Voilà, les terrasses sont élevées, on est venu contre la ville pour la prendre, et à cause de l’épée, de la famine, et de la peste, la ville est livrée entre la main des Chaldéens qui combattent contre elle ; et ce que tu as dit est arrivé, et voici, tu le vois.
\VS{25}Et cependant, Seigneur Yahweh ! Tu m'as dit : Achète-toi ce champ à prix d'argent, et prends-en des témoins, quoique la ville soit livrée entre les mains des Chaldéens.
\VS{26}Mais la parole de Yahweh fut adressée à Jérémie, en disant :
\VS{27}Voici, je suis Yahweh, le Dieu de toute chair. Y a-t-il quelque chose d'étonnant de ma part ?
\VS{28}C'est pourquoi ainsi parle Yahweh : Voici, je vais livrer cette ville entre les mains des Chaldéens, et entre les mains de Nebucadnetsar, roi de Babylone, qui la prendra.
\VS{29}Et les Chaldéens qui combattent contre cette ville, y entreront, et mettront le feu à cette ville, et la brûleront, avec les maisons sur les toits desquelles on a brûlé de l'encens à Baal, et où l'on a fait des libations à d'autres dieux pour m'irriter.
\VS{30}Car les fils d'Israël et les fils de Juda n'ont fait, dès leur jeunesse, que ce qui est mal à mes yeux ; les fils d'Israël n'ont fait que m'irriter par les œuvres de leurs mains, dit Yahweh.
\VS{31}Car cette ville a été portée à provoquer ma colère et ma fureur, depuis le jour qu'ils l'ont bâtie, jusqu'à ce jour, afin que je l'ôte de devant ma face ;
\VS{32}à cause de tout le mal que les fils d'Israël et les fils de Juda ont fait pour m'irriter, eux, leurs rois, leurs chefs, leurs sacrificateurs et leurs prophètes, les hommes de Juda et les habitants de Jérusalem.
\VS{33}Ils m'ont tourné le dos, et non la face ; je les ai enseignés, je les ai enseignés dès le matin, mais ils n'ont pas écouté pour recevoir l'instruction.
\VS{34}Mais ils ont mis leurs abominations dans la maison sur laquelle mon Nom est invoqué, pour la souiller.
\VS{35}Et ils ont bâti les hauts lieux de Baal, qui sont dans la vallée de Ben-Hinnom, pour faire passer par le feu leurs fils et leurs filles à Moloc\FTNT{Voir commentaire en Lé. 20:2.} ; ce que je ne leur avais pas ordonné, et il ne m'était pas monté à la pensée qu'ils feraient cette abomination pour faire pécher Juda.
\VS{36}Et maintenant, à cause de cela Yahweh, le Dieu d'Israël, ainsi parle sur cette ville dont vous dites qu’elle est livrée entre les mains du roi de Babylone, à cause que l’épée, la famine, et la peste sont en elle :
\VS{37}Voici, je vais les rassembler de tous les pays où je les ai chassés, dans ma colère, dans ma fureur et dans mon grand courroux ; et je les ramènerai dans ce lieu-ci, et je les y ferai habiter en sécurité.
\VS{38}Et Ils seront mon peuple, et je serai leur Dieu.
\VS{39}Et je leur donnerai un même cœur et une même voie, afin qu'ils me craignent à toujours, pour leur bien et celui de leurs fils après eux.
\VS{40}Et je traiterai avec eux une alliance éternelle, à savoir, que je ne me détournerai plus d'eux pour leur faire du bien ; et je mettrai ma crainte dans leur cœur, afin qu'ils ne se détournent pas de moi\FTNT{Es. 54:10.}.
\VS{41}Et je me réjouirai à leur faire du bien, et je les planterai dans ce pays-ci solidement, de tout mon cœur, et de toute mon âme.
\VS{42}Car ainsi parle Yahweh : Comme j'ai fait venir tous ce grand mal sur ce peuple, ainsi je ferai venir sur eux tout le bien que je prononce en leur faveur.
\VS{43}Et on achètera des champs dans ce pays, duquel vous dites que ce n'est que désolation, sans hommes ni bêtes, et qui est livré entre les mains des Chaldéens.
\VS{44}On achètera, dis-je, des champs à prix d'argent, et on en écrira les contrats, et on les cachettera, et on en prendra des témoins dans le pays de Benjamin, et aux environs de Jérusalem, dans les villes de Juda, tant dans les villes des montagnes, que dans les villes de la plaine, et dans les villes du midi. Car je ramènerai leurs captifs, dit Yahweh.
\Chap{33}
\TextTitle{Jésus, le Germe appelé à régner\FTNTT{Voir 2 S. 7:8-16.}}
\VerseOne{}Et la parole de Yahweh fut adressée une seconde fois à Jérémie, quand il était encore enfermé dans la cour de la prison, en disant :
\VS{2}Ainsi parle Yahweh, qui fait ces choses, Yahweh qui les forme et les établit, lui dont le nom est Yahweh :
\VS{3}Crie vers moi\FTNT{Yahweh, qui demandait qu'on l'invoque, n'est autre que Jésus-Christ, notre Seigneur (Joë. 2:32 ; 1 Co. 1:2 ; Ro. 10:13).}, je te répondrai, et je t'annoncerai des choses grandes, des choses cachées, que tu ne connais pas.
\VS{4}Car ainsi parle Yahweh, le Dieu d'Israël, touchant les maisons de cette ville-ci et les maisons des rois de Juda ; elles seront abattues par les terrasses et par l'épée.
\VS{5}Ils sont venus pour combattre contre les Chaldéens, mais ça été pour remplir leurs maisons des cadavres des hommes que j'ai frappé dans ma colère et dans ma fureur, et parce que j'ai caché ma face de cette ville à cause toute leur méchanceté.
\VS{6}Voici, je vais lui donner la santé et la guérison, je les guérirai, et je leur ferai découvrir une abondance de paix et de fidélité\FTNT{Ap. 22:1-2.}.
\VS{7}Et je ramènerai les captifs de Juda, et les captifs d'Israël, et je les rétablirai comme autrefois.
\VS{8}Et je les purifierai de toute leur iniquité, par laquelle ils ont péché contre moi ; et je pardonnerai toutes leurs iniquités par lesquelles ils ont péché contre moi, et par lesquelles ils se sont révoltés contre moi\FTNT{Ez. 37:23.}.
\VS{9}Et cette ville sera pour moi un sujet de joie, de louange et de gloire, parmi toutes les nations de la terre qui entendront parler de tout le bien que je leur ferai, et elles seront dans la crainte et trembleront à cause de tout le bien et de toute la prospérité que je vais lui donner.
\VS{10}Ainsi parle Yahweh : Dans ce lieu-ci duquel vous dites : Il est désert, il n'y a plus d'hommes, plus de bêtes, dans les villes de Juda, et dans les rues de Jérusalem, qui sont désolées, privées d'hommes, d'habitants, de bêtes,
\VS{11}on y entendra encore les cris de joie et les cris d'allégresse, la voix de l'époux et la voix de l'épouse, et la voix de ceux qui disent : Louez Yahweh des armées ; car Yahweh est bon, parce que sa miséricorde demeure à toujours, lorsqu'ils offriront des offrandes de reconnaissance dans la maison de Yahweh ; car je ramènerai les captifs de ce pays, et je les rétablirai comme autrefois, dit Yahweh.
\VS{12}Ainsi parle Yahweh des armées : Dans ce lieu désert, où il n'y a ni hommes ni bêtes, et dans toutes ses villes, il y aura encore des demeures de bergers qui y feront reposer leurs troupeaux ;
\VS{13}dans les villes des montagnes, et dans les villes de la plaine, dans les villes du midi, dans le pays de Benjamin et aux environs de Jérusalem, et dans les villes de Juda ; les brebis passeront encore sous les mains de celui qui les compte, dit Yahweh.
\VS{14}Voici, les jours viennent, dit Yahweh, où j'accomplirai la bonne parole que j'ai prononcée sur la maison d'Israël et la maison de Juda.
\VS{15}En ces jours et en ce temps-là, je ferai germer à David le Germe de justice, qui exercera le jugement et la justice dans le pays.
\VS{16}En ces jours-là, Juda sera sauvé, Jérusalem habitera en sécurité ; et voici comment on l'appellera : Yahweh notre justice.
\VS{17}Car ainsi parle Yahweh : David ne manquera jamais d'un successeur assis sur le trône de la maison d'Israël ;
\VS{18}et d'entre les sacrificateurs Lévites, il ne manquera jamais d'y avoir devant moi d'homme offrant des holocaustes, brûlant de l'encens avec les offrandes, et faisant des sacrifices tous les jours.
\VS{19}La parole de Yahweh fut encore adressée à Jérémie, en disant :
\VS{20}Ainsi parle Yahweh : Si vous pouvez rompre mon alliance avec le jour et mon alliance avec la nuit, de sorte que le jour et la nuit ne soient plus en leur temps,
\VS{21}alors aussi mon alliance avec David, mon serviteur, sera rompue ; de sorte qu'il n'aura plus de fils régnant sur son trône ; et avec les Lévites sacrificateurs, faisant mon service.
\VS{22}Car comme on ne peut compter l'armée des cieux, ni mesurer le sable de la mer, ainsi je multiplierai la postérité de David mon serviteur, et les Lévites qui font mon service\FTNT{Ge. 2:1 ; Ge. 15:5.}.
\VS{23}La parole de Yahweh fut encore adressée à Jérémie, en disant :
\VS{24}N'as-tu pas vu ce que ce peuple prononce, en disant : Yahweh a rejeté les deux familles qu'il avait élues ? Ainsi ils méprisent mon peuple, ils ne sont plus une nation devant eux.
\VS{25}Ainsi parle Yahweh : Si je n'ai pas fait mon alliance avec le jour et la nuit, et si je n'ai pas établi les ordonnances des cieux et de la terre ;
\VS{26}aussi rejetterai-je la postérité de Jacob, et celle de David mon serviteur, pour ne plus prendre de sa postérité des gens qui dominent sur les descendants d'Abraham, d'Isaac et de Jacob ; car je ramènerai leurs captifs, et j'aurai compassion d'eux.
\Chap{34}
\TextTitle{Désobéissance du peuple : Jérusalem dévastée}
\VerseOne{}La parole qui fut adressée à Jérémie de la part de Yahweh, lorsque Nebucadnetsar, roi de Babylone, et toute son armée, et tous les royaumes de la terre, et tous les peuples qui étaient sous la puissance de sa main, combattaient contre Jérusalem, et contre toutes ses villes, en disant\FTNT{2 R. 25:1-2.} :
\VS{2}Ainsi parle Yahweh, le Dieu d'Israël : Va, et parle à Sédécias, roi de Juda, et dis-lui : Ainsi parle Yahweh : Voici, je vais livrer cette ville entre les mains du roi de Babylone, et il la brûlera par le feu.
\VS{3}Et tu n'échapperas pas de sa main, car certainement tu seras pris et tu seras livré entre ses mains, et tes yeux verront les yeux du roi de Babylone, et il te parlera bouche à bouche, et tu iras à Babylone.
\VS{4}Toutefois écoute la parole de Yahweh, ô Sédécias, roi de Juda ! Ainsi parle Yahweh sur toi : Tu ne mourras pas par l'épée ;
\VS{5}mais tu mourras en paix, et on brûlera pour toi des parfums aromatiques, comme on en a brûlé pour tes pères, les rois précédents qui ont été avant toi ; et on te pleurera, en disant : Hélas, Seigneur ! Car j'ai prononcé cette parole, dit Yahweh\FTNT{2 Ch. 16:14.}.
\VS{6}Jérémie, le prophète, dit toutes ces paroles à Sédécias, roi de Juda, à Jérusalem.
\VS{7}Et l'armée du roi de Babylone combattait contre Jérusalem et contre toutes les villes de Juda qui restaient, à savoir, contre Lakis et contre Azéka, car c'étaient les seules villes fortifiées qui restaient parmi les villes de Juda\FTNT{2 R. 18:13.}.
\TextTitle{Jérusalem deviendra une désolation à cause de la désobéissance}
\VS{8}La parole fut adressée à Jérémie de la part de Yahweh, après que le roi Sédécias eut traité une alliance avec tout le peuple de Jérusalem, pour proclamer la liberté,
\VS{9}afin que chacun renvoie libre son esclave et chacun sa servante, l'hébreu ou la femme de l'hébreu, et qu'aucun juif ne soit l'esclave de son frère.
\VS{10}Tous les chefs et tout le peuple, qui étaient entrés dans cette alliance, entendirent que chacun devait renvoyer libre son serviteur et chacun sa servante, sans plus les asservir ; ils obéirent et les renvoyèrent.
\VS{11}Mais ensuite, ils changèrent d'avis ; ils firent revenir leurs esclaves et leurs servantes, qu'ils avaient renvoyés libres, et les assujettirent pour être leurs esclaves et leurs servantes.
\VS{12}Et la parole de Yahweh fut adressée à Jérémie en disant :
\VS{13}Ainsi parle Yahweh, le Dieu d'Israël : J'ai traité une alliance avec vos pères, le jour où je les ai fait sortir du pays d'Egypte, de la maison de servitude, en disant :
\VS{14}A la fin de la septième année, chacun renverra libre son frère hébreu qui aura été vendu ; il te servira six années, puis tu le renverras libre de chez toi. Mais vos pères ne m'ont pas écouté, ils n'ont pas prêté l'oreille\FTNT{Ex. 21:2 ; Lé. 25:10-15 ; De. 15:12.}.
\VS{15}Et vous, qui aujourd'hui étiez revenus à vous-mêmes, et vous aviez fait ce qui était droit à mes yeux, en publiant la liberté chacun pour son prochain, vous aviez traité une alliance devant moi, dans la maison sur laquelle mon Nom est invoqué.
\VS{16}Mais vous êtes revenus en arrière, et vous avez souillé mon Nom ; vous avez fait revenir chacun ses esclaves et ses servantes, que vous aviez renvoyés libres, rendus à eux-mêmes, et vous les avez assujettis, afin qu'ils soient pour vous des serviteurs et des servantes.
\VS{17}C'est pourquoi ainsi parle Yahweh : Vous ne m'avez pas obéi, en publiant la liberté chacun à son frère, et chacun à son prochain. Voici, je vais publier contre vous, dit Yahweh, la liberté contre vous à l'épée, à la peste, et à la famine ;  et je vous livrerai pour être transportés par tous les royaumes de la terre.
\VS{18}Et je livrerai les hommes qui ont transgressé mon alliance, et qui n'ont pas observé les paroles de l'alliance qu'ils avaient traitée devant moi, lorsqu'ils sont passés entre les morceaux du veau qu'ils ont coupé en deux ;
\VS{19}les chefs de Juda, et les chefs de Jérusalem, les eunuques, et les sacrificateurs, et tout le peuple du pays, qui sont passés au travers des morceaux du veau ;
\VS{20}je les livrerai, dis-je, entre les mains de leurs ennemis, entre les mains de ceux qui cherchent leur vie ; et leurs cadavres seront la pâture des oiseaux des cieux et des bêtes de la terre.
\VS{21}Je livrerai aussi Sédécias, roi de Juda, et les chefs de sa cour, entre les mains de leurs ennemis, entre les mains de ceux qui cherchent leur vie, entre les mains de l'armée du roi de Babylone, qui s'est retiré de devant vous.
\VS{22}Voici, je vais leur donner mes ordres, dit Yahweh, et je les ramènerai contre cette ville-ci ; et ils combattront contre elle, et la prendront, et la brûleront au feu ; et je ferai des villes de Juda un désert sans habitants.
\Chap{35}
\TextTitle{L'obéissance des Récabites}
\VerseOne{}C'est ici la parole qui fut adressée à Jérémie de la part de Yahweh, au temps de Jojakim, fils de Josias, roi de Juda, en disant :
\VS{2}Va à la maison des Récabites, et parle-leur, et fais les venir à la maison de Yahweh, dans l'une des chambres, et présente-leur du vin à boire\FTNT{2 S. 4:2 ; 1 Ch. 2:55.}.
\VS{3}Je pris donc Jaazania, fils de Jérémie, fils de Habazinia, et ses frères, et tous ses fils, et toute la maison des Récabites,
\VS{4}et je les fis venir dans la maison de Yahweh, dans la chambre des fils de Hanan, fils de Jigdalia, homme de Dieu, qui était près de la chambre des chefs, au-dessus de la chambre de Maaséja, fils de Schallum, garde du seuil.
\VS{5}Et je mis devant les fils de la maison des Récabites des coupes pleines de vin, et des calices, et je leur dis : Buvez du vin !
\VS{6}Et ils répondirent : Nous ne buvons pas de vin ; car Jonadab, fils de Récab, notre père, nous a donné cet ordre en disant : Vous ne boirez jamais de vin, ni vous ni vos fils\FTNT{Lé. 10:9 ; No. 6:2-4.} ;
\VS{7}vous ne bâtirez aucune maison, vous ne sèmerez aucune semence, vous ne planterez aucune vigne, et vous n'en aurez pas ; mais vous habiterez sous des tentes toute votre vie, afin que vous viviez longtemps sur la terre où vous êtes étrangers.
\VS{8}Nous avons donc obéi à la voix de Jonadab, fils de Récab, notre père dans toutes les choses qu'il nous a ordonnées, de sorte que nous n'avons pas bu de vin tous les jours de notre vie, ni nous, ni nos femmes, ni nos fils, ni nos filles.
\VS{9}Nous n'avons bâti aucune maison pour notre demeure, et nous n'avons eu ni vigne, ni champ, ni semence.
\VS{10}Mais nous avons habité sous des tentes, et nous avons obéi, et nous avons fait selon toutes les choses que Jonadab, notre père, nous a ordonnées.
\VS{11}Mais il est arrivé que quand Nebucadnetsar, roi de Babylone, est monté au pays, nous avons dit : Venez, et entrons dans Jérusalem, pour fuir de devant l'armée des Chaldéens, et de devant l'armée de Syrie. C'est ainsi que nous habitons à Jérusalem.
\VS{12}Alors la parole de Yahweh fut adressée à Jérémie, en disant :
\VS{13}Ainsi parle Yahweh des armées, le Dieu d'Israël : Va, et dis aux hommes de Juda, et aux habitants de Jérusalem : Ne recevrez-vous pas d'instruction pour obéir à mes paroles ? Dit Yahweh.
\VS{14}Toutes les paroles de Jonadab, fils de Récab, qui a ordonné à ses fils de ne pas boire de vin, ont été observées, et ils n'en ont pas bu jusqu'à ce jour ; mais ils ont obéi au commandement de leur père ; mais moi, je vous ai parlé, je vous ai parlé dès le matin, et vous ne m'avez pas obéi.
\VS{15}Car je vous ai envoyé tous les prophètes, mes serviteurs, je les ai envoyés dès le matin, pour vous dire : Revenez maintenant chacun de votre mauvaise voie, et amendez vos actions, et n'allez pas après d'autres dieux pour les servir, afin que vous demeureriez dans le pays que j'ai donné à vous et à vos pères. Mais vous n'avez pas prêté l'oreille, et vous ne m'avez pas écouté.
\VS{16}Parce que les fils de Jonadab, fils de Récab, ont observé le commandement que leur avait donné leur père, et que ce peuple ne m'écoute pas ;
\VS{17}à cause de cela, Yahweh le Dieu des armées, le Dieu d'Israël, parle ainsi : Voici, je vais faire venir sur Juda et sur tous les habitants de Jérusalem tout le mal que j'ai prononcés contre eux ; parce que je leur ai parlé, et ils n'ont pas écouté ; et que je les ai appelés, et ils n'ont pas répondu.
\VS{18}Et Jérémie dit à la maison des Récabites: Ainsi parle Yahweh des armées, le Dieu d'Israël : Parce que vous avez obéi au commandement de Jonadab, votre père, et que vous avez gardé tous ses commandements, et avez fait selon tout ce qu'il vous a ordonné\FTNT{Les Récabites furent bénis parce qu'ils obéirent aux commandements de leur père (Ep. 6:1-3)} ;
\VS{19}c'est pourquoi, ainsi parle Yahweh des armées, le Dieu d'Israël : Jonadab, fils de Récab, ne manquera jamais de descendants qui se tiennent debout devant moi.
\Chap{36}
\TextTitle{Le roi Jojakim brûle le manuscrit de Jérémie}
\VerseOne{}Or il arriva, dans la quatrième année de Jojakim, fils de Josias, roi de Juda, que cette parole fut adressée à Jérémie de la part de Yahweh, en disant :
\VS{2}Prends-toi un rouleau de livre, et tu y écriras toutes les paroles que je t'ai dites contre Israël et contre Juda, et contre toutes les nations, depuis le jour où je t'ai parlé, c'est à dire, depuis le jours de Josias, jusqu'à ce jour.
\VS{3}Peut-être que la maison de Juda entendra tout le mal que je pense de leur faire, afin que chaque homme se détourne de sa mauvaise voie, et que je leur pardonne leur iniquité, et leur péché.
\VS{4}Jérémie donc appela Baruc, fils de Nérija, et Baruc écrivit, sous la dictée de Jérémie, dans le rouleau de livre, toutes les paroles que Yahweh lui avait dites.
\VS{5}Puis Jérémie donna cet ordre à Baruc, en disant : Je suis retenu, et je ne peux pas entrer dans la maison de Yahweh.
\VS{6}Tu y entreras donc et tu liras dans le rouleau que tu as écrit sous ma dictée, toutes les paroles de Yahweh, aux oreilles du peuple dans la maison de Yahweh le jour du jeûne ; tu les liras, dis-je, aussi aux oreilles de tous ceux de Juda qui seront venus de leurs villes.
\VS{7}Peut-être que Yahweh écoutera leur supplication et qu'ils reviendront chacun de leur mauvaise voie ; car grande est la colère, la fureur que Yahweh a déclarée contre ce peuple.
\VS{8}Baruc donc, fils de Nérija, fit selon tout ce que lui avait ordonné Jérémie le prophète, lisant dans le rouleau les paroles de Yahweh, dans la maison de Yahweh.
\VS{9}Or il arriva dans la cinquième année de Jojakim, fils de Josias, roi de Juda, le neuvième mois, qu'on publia le jeûne devant Yahweh à tout le peuple de Jérusalem et à tout le peuple venu des villes de Juda à Jérusalem.
\VS{10}Et Baruc lut dans le livre les paroles de Jérémie, aux oreilles de tout le peuple, dans la maison de Yahweh, dans la chambre de Guemaria, fils de Schaphan, le secrétaire, dans le parvis supérieur, à l'entrée de la porte neuve de la maison de Yahweh.
\VS{11}Et quand Michée fils de Guemaria, fils de Schaphan, eut entendu toutes les paroles de Yahweh contenues dans le livre ;
\VS{12}il descendit dans la maison du roi, vers la chambre du secrétaire, et voici tous les chefs y étaient assis, à savoir, Elischama le secrétaire, Delaja, fils de Schemaeja, Elnathan, fils de Acbor, et Guemaria, fils de Schaphan, et Sédécias, fils de Hanania, et tous les chefs.
\VS{13}Et Michée leur rapporta toutes les paroles qu'il avait entendues, quand Baruc lisait dans le livre, aux oreilles du peuple.
\VS{14}C'est pourquoi tous les chefs envoyèrent vers Baruc, Jehudi, fils de Nethania, fils de Schélémia, fils de Cuschi, pour lui dire : Prends en ta main le rouleau que tu as lu aux oreilles du peuple, et viens ici ! Baruc donc, fils de Nérija, prit le rouleau en sa main, et vint vers eux.
\VS{15}Et ils lui dirent : Assieds-toi maintenant, et lis-le à nos oreilles ; et Baruc le lut à leurs oreilles.
\VS{16}Et il arriva que sitôt qu'ils eurent entendu toutes les paroles, ils furent effrayés entre eux, et dirent à Baruc : Nous ne manquerons pas de rapporter au roi toutes ces paroles.
\VS{17}Et ils interrogèrent Baruc, en disant : Dis-nous comment tu as écrit toutes ces paroles sous sa dictée.
\VS{18}Et Baruc leur dit : Il me dictait de sa bouche toutes ces paroles, et je les écrivais avec de l'encre dans le livre.
\VS{19}Alors les chefs dirent à Baruc : Va, et cache-toi, ainsi que Jérémie, et que personne ne sache où vous serez.
\VS{20}Puis ils s'en allèrent vers le roi dans la cour, mais ils prirent soin de laisser le rouleau dans la chambre d'Elischama le secrétaire ; et ils racontèrent toutes ces paroles aux oreilles du roi.
\VS{21}Et le roi envoya Jehudi pour prendre le rouleau ; et quand Jehudi l'eut prit de la chambre d'Elischama le secrétaire, et il le lut aux oreilles du roi et de tous les chefs qui étaient autour de lui.
\VS{22}Or le roi était assis dans la maison d'hiver, au neuvième mois, et un brasier était allumé devant lui.
\VS{23}Et il arriva qu'aussitôt que Jehudi en eut lu trois ou quatre feuilles, le roi déchira le rouleau avec le canif du secrétaire, et le jeta au feu du brasier, jusqu'à ce que tout le rouleau fut consumé au feu du brasier.
\VS{24}Et ni le roi ni tous ses serviteurs qui entendirent toutes ces paroles, n'en furent pas effrayés, et ne déchirèrent pas leurs vêtements.
\VS{25}Toutefois Elnathan, et Delaja et Guemaria intercédèrent envers le roi, afin qu'il ne brûle pas le rouleau ; mais il ne les écouta pas.
\VS{26}Même le roi ordonna à Jerachmeel, fils de Hammélec, et à Seraja, fils d'Azriel, et à Schélémia, fils de Abdeel, de saisir Baruc, le secrétaire, et Jérémie le prophète ; mais Yahweh les cacha.
\TextTitle{Remplacement du manuscrit brûlé ; jugement sur Jojakim}
\VS{27}Et la parole de Yahweh fut adressée à Jérémie, après que le roi eut brûlé le rouleau contenant les paroles que Baruc avait écrites sous la dictée de Jérémie, en disant :
\VS{28}Prends encore un autre rouleau, et tu y écriras toutes les premières paroles qui étaient dans le premier rouleau que Jojakim, roi de Juda, a brûlé.
\VS{29}Et tu diras à Jojakim, roi de Juda : Ainsi parle Yahweh : Tu as brûlé ce rouleau, et tu as dit : Pourquoi y as-tu écrit ces paroles : Le roi de Babylone viendra certainement, il détruira ce pays, et il exterminera les hommes et les bêtes ?
\VS{30}C'est pourquoi ainsi parle Yahweh sur Jojakim, roi de Juda : Aucun des siens ne sera assis sur le trône de David, et son cadavre sera jeté de jour à la chaleur et de nuit à la gelée.
\VS{31}Je le punirai, lui, sa postérité, et ses serviteurs, à cause de leur iniquité ; et je ferai venir sur eux, et sur les habitants de Jérusalem, et sur les hommes de Juda, tout le mal que je leur ai prononcé, et qu'ils n'ont pas écouter.
\VS{32}Jérémie donc prit un autre rouleau, et le donna à Baruc, fils de Nérija secrétaire, lequel y écrivit, sous la dictée de Jérémie, toutes les paroles du rouleau que Jojakim, roi de Juda, avait brûlé au feu. Beaucoup de paroles semblables y furent encore ajoutées.
\Chap{37}
\TextTitle{Sédécias sollicite l'intercession de Jérémie}
\VerseOne{}Or le roi Sédécias, fils de Josias, régna à la place de Jéconia, fils de Jojakim, et il fut établi roi dans le pays de Juda par Nebucadnetsar, roi de Babylone.
\VS{2}Mais, ni lui, ni ses serviteurs, ni le peuple du pays, n'obéirent pas aux paroles que Yahweh prononça par Jérémie le prophète.
\VS{3}Toutefois le roi Sédécias envoya Jucal, fils de Schélémia, et Sophonie, fils de Maaséja sacrificateur, vers Jérémie le prophète, pour lui dire : Intercède pour nous auprès de Yahweh, notre Dieu.
\VS{4}Car Jérémie allait et venait parmi le peuple, parce qu'on ne l'avait pas encore mis en prison.
\VS{5}Alors l'armée de Pharaon sortit d'Egypte, et quand les Chaldéens qui assiégeaient Jérusalem en entendirent cette nouvelle, ils se retirèrent de devant Jérusalem.
\VS{6}Et la parole de Yahweh fut adressée à Jérémie le prophète, en disant :
\VS{7}Ainsi parle Yahweh, le Dieu d'Israël : Vous direz ainsi au roi de Juda, qui vous a envoyés me consulter : Voici, l'armée de Pharaon, qui était sortie à votre secours, retourne dans son pays, en Egypte ;
\VS{8}et les Chaldéens reviendront, et combattront contre cette ville, et la prendront, et la brûleront au feu.
\VS{9}Ainsi parle Yahweh : Ne vous abusez pas vous-mêmes, en disant : Les Chaldéens s'en iront loin de nous ; car ils ne s'en iront pas.
\VS{10}Même quand vous auriez battu toute l'armée des Chaldéens qui combattent contre vous, et qu'il n'y aurait de reste entre eux que des hommes percés de blessures, ils se relèveront pourtant chacun dans sa tente, et brûleront cette ville au feu.
\TextTitle{Jérémie calomnié et emprisonné}
\VS{11}Or il arriva que quand l'armée des Chaldéens se fut retirée de Jérusalem, à cause de l'armée de Pharaon,
\VS{12}Jérémie sortit de Jérusalem, pour s'en aller dans le pays de Benjamin, se glissant hors de là au milieu du peuple.
\VS{13}Mais quand il fut à la porte de Benjamin, il y avait là un commandant de la garde, nommé Jireija, fils de Schélémia, fils de Hanania, qui saisit Jérémie le prophète, en lui disant : Tu vas te rendre aux Chaldéens !
\VS{14}Et Jérémie répondit : C'est un mensonge ! Je ne vais pas me rendre aux Chaldéens. Mais il ne l'écouta pas, et Jireija prit Jérémie, et l'amena vers les chefs.
\VS{15}Et les chefs se mirent en colère contre Jérémie, et le frappèrent et le mirent en prison dans la maison de Jonathan le secrétaire, car ils en avaient fait une prison.
\VS{16}Et ainsi Jérémie entra dans la fosse de la maison et dans les cachots ; et Jérémie y demeura plusieurs jours.
\VS{17}Mais le roi Sédécias y envoya, et l'en tira, et il l'interrogea en secret dans sa maison, et lui dit : Y a-t-il une parole de la part de Yahweh ? Et Jérémie répondit : Il y en a une ; Et lui dit : Tu seras livré entre les mains du roi de Babylone.
\VS{18}Puis Jérémie dit au roi Sédécias : Quel péché ai-je commis contre toi, contre tes serviteurs, et contre ce peuple, pour que vous m'ayez mis en prison ?
\VS{19}Mais où sont vos prophètes qui vous prophétisaient, en disant : Le roi de Babylone ne reviendra pas contre vous, ni contre ce pays ?
\VS{20}Or écoute maintenant, je te prie, ô roi, mon seigneur ! Et que maintenant ma supplication soit reçue devant ta face, et ne me renvoie pas dans la maison de Jonathan le secrétaire, de peur que je n'y meure !
\VS{21}C'est pourquoi le roi Sédécias ordonna qu'on garde Jérémie dans la cour de la prison, et qu'on lui donne chaque jour un pain de la rue des boulangers, jusqu'à ce que tout le pain de la ville soit épuisé. Ainsi Jérémie demeura dans la cour de la prison.
\Chap{38}
\TextTitle{Jérémie jeté dans la fosse puis délivré par Ebed-Mélec l'éthiopien}
\VerseOne{}Mais Schephathia, fils de Matthan, et Guedalia, fils de Paschhur, et Jucal, fils de Schélémia, et Paschhur, fils de Malkija, entendirent les paroles que Jérémie prononçait à tout le peuple, en disant :
\VS{2}Ainsi parle Yahweh : Celui qui restera dans cette ville mourra par l'épée, par la famine, ou par la peste ; mais celui qui sortira vers les Chaldéens vivra, et sa vie sera son butin, et il vivra.
\VS{3}Ainsi parle Yahweh : Cette ville sera livrée certainement aux mains de l'armée du roi de Babylone, qui la prendra.
\VS{4}Et les chefs dirent au roi : Qu'on fasse mourir cet homme ! Car il décourage les mains des hommes de guerre qui restent dans cette ville, et les mains de tout le peuple, en leur disant de telles paroles ; parce que cet homme ne cherche pas le bien de ce peuple, mais le mal.
\VS{5}Et le roi Sédécias dit : Voici, il est entre vos mains ; car le roi ne peut rien contre vous.
\VS{6}Ils prirent donc Jérémie, et le jetèrent dans la fosse de Malkija, fils de Hammélec, laquelle était dans la cour de la prison, et ils descendirent Jérémie avec des cordes dans cette fosse où Il n'y avait pas d'eau mais de la boue ; et ainsi Jérémie enfonça dans la boue.
\VS{7}Mais Ebed-Mélec l'éthiopien, eunuque, qui était dans la maison du roi, apprit qu'ils avaient mis Jérémie dans cette fosse ; et le roi était assis à la porte de Benjamin.
\VS{8}Et Ebed-Mélec sortit de la maison du roi, et parla au roi, en disant :
\VS{9}Ô roi, mon seigneur ! Ces hommes-là ont mal fait dans tout ce qu'ils ont fait contre Jérémie le prophète, en le jetant dans la fosse, car il serait déjà mort de faim dans le lieu où il était parce qu'il n'y a plus de pain dans la ville.
\VS{10}C'est pourquoi le roi donna cet ordre à Ebed-Mélec l'éthiopien, en disant : Prends ici trente hommes avec toi, et fais remonter hors de la fosse Jérémie le prophète, avant qu'il meure.
\VS{11}Ebed-Mélec donc prit des hommes avec lui, et entra dans la maison du roi, dans un lieu au-dessous du trésor, d'où il prit de vieux lambeaux et de vieux chiffons, et les descendit avec des cordes à Jérémie dans la fosse.
\VS{12}Et Ebed-Mélec l'éthiopien dit à Jérémie : Mets ces vieux lambeaux et ces chiffons sous les aisselles de tes bras, au-dessous des cordes. Et Jérémie fit ainsi.
\VS{13}Ainsi ils tirèrent Jérémie dehors avec les cordes, et le firent remonter hors de la fosse ; et Jérémie demeura dans la cour de la prison.
\TextTitle{Jérémie appelle Sédécias à la repentance}
\VS{14}Et le roi Sédécias envoya chercher Jérémie le prophète, et le fit amener vers lui à la troisième entrée qui était dans la maison de Yahweh. Et le roi dit à Jérémie : Je vais te demander une chose, ne me cache rien.
\VS{15}Et Jérémie répondit à Sédécias : Quand je te l'aurais déclarée, n'est-il pas vrai que tu me feras mourir ? Et quand je t'aurai donné conseil, tu ne m'écouteras pas.
\VS{16}Alors le roi Sédécias jura secrètement à Jérémie, en disant : Yahweh est vivant, qui nous a fait cette âme, je ne te ferai pas mourir, et que je ne te livrerai pas entre les mains de ces hommes qui cherchent ta vie.
\VS{17}Alors Jérémie dit à Sédécias : Ainsi parle Yahweh, le Dieu des armées, le Dieu d'Israël : Si tu sors volontairement pour aller vers les chefs du roi de Babylone, tu auras la vie, et cette ville ne sera pas brûlée par le feu ; et tu vivras toi et ta maison.
\VS{18}Mais si tu ne sors pas vers les chefs du roi de Babylone, cette ville sera livrée entre les mains des Chaldéens, qui la brûleront par le feu ; et tu n'échapperas pas à leurs mains.
\VS{19}Et le roi Sédécias dit à Jérémie : Je crains à cause des Juifs qui se sont rendus aux Chaldéens, je crains qu'on ne me livre entre leurs mains et qu’ils ne se moquent de moi.
\VS{20}Et Jérémie lui répondit : On ne te livrera pas à eux. Je te prie, écoute la voix de Yahweh dans ce que je te dis ; tu t'en trouveras bien, et tu auras la vie.
\VS{21}Que si tu refuses de sortir, voici ce que Yahweh m'a fait voir :
\VS{22}C'est que, voici toutes les femmes qui restent dans la maison du roi de Juda seront menées aux chefs du roi de Babylone, et elles diront : Tu as été séduit, vaincu, par les hommes qui te prédisaient la paix ; et quand tes pieds sont enfoncés dans la boue, ils se sont retirés en arrière.
\VS{23}Toutes tes femmes et tes fils seront menés dehors aux Chaldéens ; et tu n'échapperas pas à leurs mains, mais tu seras pris, pour être livré entre les mains du roi de Babylone, et à cause de toi, cette ville sera brûlée par le feu.
\VS{24}Alors Sédécias dit à Jérémie : Que personne ne sache rien de ces paroles, et tu ne mourras pas.
\VS{25}Et si les chefs entendent que je t'ai parlé, et qu'ils viennent vers toi, et te dise : Déclare-nous maintenant ce que tu as dit au roi, et ce que le roi t'a dit, ne nous en cache rien, et nous ne te ferons pas mourir ;
\VS{26}tu leur diras : J'ai présenté ma supplication devant le roi afin qu'il ne me renvoie pas dans la maison de Jonathan, pour y mourir.
\VS{27}Tous les chefs donc vinrent vers Jérémie, et l'interrogèrent ; mais il leur répondit exactement comme le roi lui avait ordonné ; et ils gardèrent le silence, car l'affaire n'avait pas été divulguée.
\VS{28}Ainsi Jérémie demeura dans la cour de la prison, jusqu'au jour où Jérusalem fut prise, et il y était lorsque Jérusalem fut prise.
\Chap{39}
\TextTitle{Prise de Jérusalem ; Sédécias déporté à Babylone\FTNTT{2 R. 25:1-7; Jé. 52:4-17 ; 2 Ch. 36:17-21.}}
\VerseOne{}La neuvième année de Sédécias, roi de Juda, au dixième mois, Nebucadnetsar, roi de Babylone, vint avec toute son armée contre Jérusalem, et ils l'assiégèrent.
\VS{2}Et la onzième année de Sédécias, le neuvième jour du quatrième mois, une brèche fut faite à la ville.
\VS{3}Et tous les chefs du roi de Babylone y entrèrent, et s'assirent à la porte du milieu, à savoir, Nergal-Scharetser, Samgar-Nebu, Sarsekim, chef des eunuques, Nergal-Scharetser, chef des devins et tous les autres chefs du roi de Babylone.
\VS{4}Or il arriva qu'aussitôt que Sédécias, roi de Juda, et tous les hommes de guerre les eurent vus, ils s'enfuirent et sortirent de nuit hors de la ville, par le chemin du jardin du roi, par la porte entre les deux murailles, et ils s'en allèrent par le chemin de la plaine.
\VS{5}Mais l'armée des Chaldéens les poursuivit et atteignit Sédécias dans les plaines de Jéricho. Ils le prirent, et le firent monter vers Nebucadnetsar, roi de Babylone, à Ribla, dans le pays de Hamath, où il prononça contre lui une sentence.
\VS{6}Et le roi de Babylone fit égorger à Ribla les fils de Sédécias sous ses yeux ; le roi de Babylone fit aussi égorger tous les nobles de Juda.
\VS{7}Puis il fit crever les yeux à Sédécias, et le fit lier de doubles chaînes d'airain, pour le conduire à Babylone.
\VS{8}Les Chaldéens brûlèrent par le feu la maison royale et les maisons du peuple, et démolirent\FTNT{Ici débute le « temps des nations » (587-586 av J.-C.), Jérusalem est foulée aux pieds par les nations. Voir aussi 2 R. 25:8-24 ; 2 Ch. 36:17-21.} les murailles de Jérusalem.
\VS{9}Et Nebuzaradan, chef des gardes, transporta à Babylone le reste du peuple qui était resté dans la ville, et ceux qui s'étaient rendus à lui, le reste, dis-je, du peuple qui avait été épargné.
\VS{10}Mais Nebuzaradan, chefs des gardes, laissa dans le pays de Juda les plus pauvres du peuple qui n'avaient rien ; et en ce jour-là, il leur donna des vignes et des champs.
\TextTitle{Jérémie libéré de prison}
\VS{11}Or Nebucadnetsar, roi de Babylone, avait donné cet ordre au sujet de Jérémie, à Nebuzaradan, chef des gardes, en disant :
\VS{12}Prends cet homme et veille sur lui ; ne lui fais aucun mal, mais fais pour lui tout ce qu'il te dira.
\VS{13}Nebuzaradan donc chefs des gardes, envoya, et aussi Nebuschazban, Rabsaris, chef des eunuques, Nergal-Scharetser, Rabmag, chef des devins, et tous les chefs du roi de Babylone ;
\VS{14}ils envoyèrent, dis-je, chercher Jérémie dans la cour de la prison, et le remirent à Guedalia, fils d'Achikam, fils de Schaphan, pour qu'il le conduise dans sa maison. Ainsi il demeura au milieu du peuple.
\TextTitle{Yahweh épargne Ebed-Mélec}
\VS{15}Or la parole de Yahweh fut adressée à Jérémie pendant qu'il était enfermé dans la cour de la prison, en disant :
\VS{16}Va, et parle à Ebed-Mélec l'éthiopien, et dis-lui : Ainsi parle Yahweh des armées, le Dieu d'Israël : Voici, je vais faire venir mes paroles sur cette ville pour son malheur et non pas pour son bien, et elles s'accompliront en ce jour-là devant toi.
\VS{17}Mais je te délivrerai en ce jour-là, dit Yahweh, et tu ne seras pas livré entre les mains des hommes que tu crains.
\VS{18}Car certainement je te ferai échapper, et tu ne tomberas pas sous l'épée ; mais ta vie sera ton butin, parce que tu as eu confiance en moi, dit Yahweh.
\Chap{40}
\TextTitle{Assassinat de Guedalia et meurtres en série d'Ismaël}
\VerseOne{}La parole qui fut adressée à Jérémie de la part de Yahweh, quand Nebuzaradan, chef des gardes, l'eut renvoyé de Rama, après l'avoir pris lorsqu'il était lié de chaînes parmi tous les captifs de Jérusalem et de Juda qu'on transportait à Babylone.
\VS{2}Quand donc le chef des gardes prit Jérémie, et il lui dit : Yahweh, ton Dieu, a prononcé ce mal contre ce lieu-ci ;
\VS{3}et Yahweh l'a fait venir et a fait comme il avait dit, parce que vous avez péché contre Yahweh, et que vous n'avez pas écouté sa voix, à cause de cela ceci vous est arrivé.
\VS{4}Maintenant donc voici, je t'affranchis aujourd'hui des chaînes que tu as aux mains ; s'il est bon à tes yeux de venir avec moi à Babylone, viens, et j'aurai les yeux sur toi ; mais s'il est mauvais de venir avec moi à Babylone, ne viens pas ; regarde, tout le pays est à ta disposition, va où il te semblera bon et convenable d'aller.
\VS{5}Or Guedalia ne retournera plus ici ; retourne, dit-il, vers Guedalia, fils d'Achikam, fils de Schaphan, que le roi de Babylone a établi sur les villes de Juda, et demeure avec lui parmi le peuple ; ou bien, va partout où il conviendra à tes yeux d'aller. Et le chef des gardes lui donna des vivres et quelques présents, et le renvoya.
\VS{6}Jérémie donc alla vers Guedalia, fils d'Achikam, à Mitspa, et demeura avec lui parmi le peuple qui était resté dans le pays.
\VS{7}Et tous les chefs des armées qui étaient dans les champs, eux et leurs hommes, entendirent que le roi de Babylone avait établi Guedalia, fils d'Achikam, sur le pays, et qu'il lui avait commis les hommes, et les femmes, et les enfants, et ceux-là d'entre les plus pauvres du pays, à savoir, de ceux qui n'avaient pas été transportés à Babylone.
\VS{8}Alors ils allèrent vers Guedalia à Mitspa ; à savoir, Ismaël fils de Nethania, et Jochanan et Jonathan fils de Karéach, et Seraja fils de Thanhumeth, et les fils d'Ephaï de Nethopha, et Jezania fils du Maacatite, eux et leurs hommes.
\VS{9}Et Guedalia, fils d'Achikam, fils de Schaphan, leur jura, à eux et à leurs hommes, en disant : Ne craignez pas de servir les Chaldéens ; demeurez dans le pays, et servez le roi de Babylone, et vous vous en trouverez bien.
\VS{10}Et pour moi, voici, je resterai à Mitspa, pour me tenir prêt à recevoir les ordres des Chaldéens qui viendront vers nous ; mais vous, recueillez le vin, les fruits d'été et l'huile, et mettez-les dans vos vases, et demeurez dans vos villes que vous avez prises pour votre demeure.
\VS{11}Pareillement aussi tous les Juifs qui étaient au pays de Moab, et parmi les Ammonites, et au pays d'Edom, et dans toutes ces contrées, quand il eurent entendu que le roi de Babylone avait laissé quelque reste à Juda, et qu'il avait établi sur eux Guedalia, fils d'Achikam, fils de Schaphan ;
\VS{12}tous ces juifs-là retournèrent de tous les lieux où ils avaient été chassés, et vinrent dans le pays de Juda vers Guedalia à Mitspa, et recueillirent du vin et des fruits d'été en grande abondance.
\VS{13}Mais Jochanan, fils de Karéach, et tous les chefs des armées qui étaient dans les champs, vinrent vers Guedalia à Mitspa,
\VS{14}et lui dirent : Ne sais-tu pas certainement que Baalis, roi des Ammonites, a envoyé Ismaël, le fils de Nethania, pour t'ôter la vie ? Mais Guedalia, fils d'Achikam, ne les crut pas.
\VS{15}Et Jochanan, fils de Karéach parla en secret à Guedalia à Mitspa, en disant : Laisse-moi aller et frapper Ismaël, fils de Nethania, et personne ne le saura. Pourquoi t'ôterait-il la vie, afin que tous les Juifs qui se sont rassemblés vers toi soient dissipés, et que les restes de Juda périssent ?
\VS{16}Mais Guedalia, fils d'Achikam, dit à Jochanan, fils de Karéach : Ne fais pas cela, car tu parles faussement d'Ismaël.
\Chap{41}
\TextTitle{Assassinat de Guedalia}
\VerseOne{}Or il arriva, au septième mois, qu'Ismaël, fils de Nethania, fils d'Elischama, de la race royale, et l'un des grands du roi et dix hommes avec lui, vinrent vers Guedalia, fils d'Achikam, à Mitspa ; et ils mangèrent là du pain ensemble à Mitspa\FTNT{2 R. 25:25.}.
\VS{2}Mais Ismaël, fils de Nethania, se leva, et les dix hommes qui étaient avec lui, et ils frappèrent avec l'épée Guedalia, fils d'Achikam, fils de Schaphan, et on le fit mourir, lui que le roi de Babylone avait établi sur le pays.
\VS{3}Ismaël frappa aussi tous les juifs qui étaient avec Guedalia à Mitspa, et les Chaldéens, gens de guerre, qui se trouvaient là.
\VS{4}Et il arriva que le second jour après après qu'on eut fait mourir Guedalia, avant que personne le sût,
\VS{5}quelques hommes de Sichem, de Silo et de Samarie, au nombre de quatre-vingts hommes, ayant la barbe rasée et les vêtements déchirés, et s'étant fait des incisions, vinrent avec des dons et de l'encens dans leurs mains pour les apporter dans la maison de Yahweh.
\VS{6}Alors Ismaël, fils de Nethania, sortit de Mitspa au-devant d'eux, et il marchait en pleurant, et quand il les rencontra, il leur dit : Venez vers Guedalia, fils d'Achikam.
\VS{7}Mais sitôt qu'ils arrivèrent au milieu de la ville, Ismaël, fils de Nethania, accompagné des hommes qui étaient avec lui, les égorgea et les jeta dans une fosse.
\VS{8}Mais il se trouva parmi eux dix hommes, qui dirent à Ismaël : Ne nous fais pas mourir, car nous avons dans les champs des provisions cachées de froment, d'orge, d'huile et de miel ; et il les laissa, et ne les fit pas mourir avec leurs frères.
\VS{9} Et la fosse dans laquelle Ismaël jeta les cadavres des hommes qu'il tua, à l'occasion de Guedalia, est celle que le roi Asa avait faite, lorsqu'il craignait Baescha, roi d'Israël ; et Ismaël, fils de Nethania, la remplit de gens tués\FTNT{1 R. 15:22.}.
\VS{10}Et Ismaël emmena captif tout le reste du peuple qui était à Mitspa, les filles du roi et tous ceux du peuple qui demeuraient à Mitspa, que Nebuzaradan, chef des gardes, avait commis à Guedalia, fils d'Achikam ; Ismaël, fils de Nethania, les emmena captifs, et s'en alla pour passer vers les Ammonites.
\TextTitle{Jochanan délivre le peuple ; fuite d'Ismaël}
\VS{11}Mais Jochanan, fils de Karéach, et tous les chefs des armées qui étaient avec lui, entendirent tout le mal qu'Ismaël, fils de Nethania, avait fait ;
\VS{12}et ils prirent tous les hommes, et s'en allèrent pour combattre contre Ismaël, fils de Nethania. Ils le trouvèrent près des grandes eaux qui sont à Gabaon.
\VS{13}Et il arriva qu'aussitôt que tout le peuple qui était avec Ismaël vit Jochanan, fils de Karéach, et tous les chefs des armées qui étaient avec lui, ils s'en réjouirent ;
\VS{14}et tout le peuple qu'Ismaël avait emmené captif de Mitspa tourna visage, et revenant sur leur pas, il s'en alla vers Jochanan, fils de Karéach.
\VS{15}Mais Ismaël, fils de Nethania, échappa avec huit hommes devant Jochanan, et s'en alla vers les Ammonites.
\VS{16}Et Jochanan, fils de Karéach, et tous les chefs des armées qui étaient avec lui, prirent tout le reste du peuple qu'ils avaient retiré des mains d'Ismaël, fils de Nethania, qu'il emmenait captif de Mitspa, après avoir tué Guedalia, fils d'Achikam, à savoir, les vaillants hommes de guerre, et les femmes, et les enfants et les eunuques ; et les ramenèrent de Gabaon.
\VS{17}Et ils s'en allèrent et demeurèrent à l'hôtellerie de Kimham, près de Bethléhem, pour se retirer ensuite en Egypte,
\VS{18}à cause des Chaldéens ; car ils avaient peur d'eux, parce qu'Ismaël, fils de Nethania, avait tué Guedalia, fils d'Achikam, qui avait été établi sur le pays par le roi de Babylone.
\Chap{42}
\TextTitle{Yahweh défend au reste du peuple de se réfugier en Egypte }
\VerseOne{}Alors tous les chefs des armées, et Jochanan, fils de Karéach, et Jezania, fils d'Hosée, et tout le peuple, depuis le plus petit jusqu'au plus grand, s'approchèrent,
\VS{2}et dirent à Jérémie le prophète : Que notre supplication soit favorable devant toi ! Intercède auprès de Yahweh, ton Dieu, pour nous, à savoir, pour tout ce reste-ci ; car de beaucoup de monde que nous étions, nous sommes restés peu, comme tes yeux nous voient ;
\VS{3}et que Yahweh, ton Dieu, nous déclare le chemin par lequel nous aurons à marcher, et ce que nous avons à faire !
\VS{4}Et Jérémie le prophète, leur répondit : J'ai entendu votre demande ; voici, je vais prier Yahweh, votre Dieu, selon vos paroles ; et il arrivera que je vous déclarerai tout ce que Yahweh vous répondra, et je ne vous en cacherai pas un mot.
\VS{5}Et ils dirent à Jérémie : Yahweh soit entre nous un témoin véritable et fidèle, si nous ne faisons pas selon toutes les paroles que Yahweh, ton Dieu, t'enverra vers nous !
\VS{6}Soit bien, soit mal, nous obéirons à la voix de Yahweh, notre Dieu, vers qui nous t'envoyons, afin qu'il nous arrive du bien, quand nous aurons obéi à la voix de Yahweh, notre Dieu.
\VS{7}Et il arriva, au bout de dix jours, que la parole de Yahweh fut adressée à Jérémie.
\VS{8}Et il appela Jochanan, fils de Karéach, tous les chefs des armées qui étaient avec lui, et tout le peuple, depuis le plus petit jusqu'au plus grand ;
\VS{9}et leur dit : Ainsi parle Yahweh, le Dieu d'Israël, vers qui vous m'avez envoyé, pour présenter votre supplication devant lui :
\VS{10}Si vous continuez à demeurer dans ce pays, je vous rétablirai et je ne vous détruirai pas ; je vous y planterai et je ne vous arracherai pas, car je me repens du mal que je vous ai fait.
\VS{11}Ne craignez pas le roi de Babylone, dont vous avez peur, ne craignez pas, dit Yahweh, car je suis avec vous pour vous sauver et pour vous délivrer de sa main.
\VS{12}Même je vous ferai obtenir miséricorde, tellement qu'il aura pitié de vous, et vous fera retourner dans votre pays.
\VS{13}Que si vous dites : Nous ne demeurerons pas dans ce pays, et nous n'écouterons pas la voix de Yahweh, notre Dieu,
\VS{14}en disant : Non ; mais nous irons au pays d'Egypte, afin que nous ne voyons pas de guerre, et que nous n'entendions pas le son du shofar, et que nous ne manquions pas de pain, et nous demeurerons là.
\VS{15}A cause de cela écoutez maintenant la parole de Yahweh, vous les restes de Juda ! Ainsi parle Yahweh des armées, le Dieu d'Israël : Si vous tournez le visage pour aller en Egypte, et que vous y entriez pour y demeurer ;
\VS{16}il arrivera que l'épée dont vous avez peur vous attrapera là au pays d'Egypte ; et la famine que vous craignez si fort vous suivra en Egypte, et vous y mourrez\FTNT{Ez. 30:9-11.}.
\VS{17}Et il arrivera que tous les hommes qui tourneront le visage pour aller en Egypte afin d'y demeurer, mourront par l'épée, par la famine et par la peste ; et il n'y aura ni survivant ni réchappé devant le mal que je vais faire venir sur eux.
\VS{18}Car ainsi parle Yahweh des armées, le Dieu d'Israël : Comme ma colère et ma fureur se sont répandues sur les habitants de Jérusalem, ainsi ma fureur sera versée sur vous, quand vous serez entrés en Egypte ; et vous serez un sujet d'exécration, d'épouvante, de malédiction et d'opprobre, et vous ne verrez plus ce lieu-ci.
\VS{19}Vous, les restes de Juda, Yahweh dit contre vous : N'allez pas en Egypte ! Sachez certainement que je vous ai avertis aujourd'hui.
\VS{20}Car vous vous êtes séduits vous-mêmes dans vos âmes, quand vous m'avez envoyé vers Yahweh, votre Dieu, en me disant : Intercède pour nous auprès de Yahweh, notre Dieu, et déclare-nous tout ce que Yahweh, notre Dieu, te dira, et nous le ferons.
\VS{21}Et je vous l'ai déclaré aujourd'hui ; mais vous n'écoutez pas la voix de Yahweh, votre Dieu, ni rien de tout ce pour quoi il m'a envoyé vers vous.
\VS{22}Maintenant donc sachez certainement que vous mourrez par l'épée, par la famine et par la peste, dans le lieu où vous avez désiré d’aller pour y demeurer.
\Chap{43}
\TextTitle{Désobéissance des Hébreux ; jugement sur l'Egypte}
\VerseOne{}Or il arriva qu'aussitôt que Jérémie eut achevé de prononcer à tout le peuple toutes les paroles de Yahweh, leur Dieu, pour lesquelles Yahweh, leur Dieu, l'avait envoyé vers eux, à savoir, toutes ces choses-là ;
\VS{2}Azaria, fils d'Hosée, et Jochanan, fils de Karéach, et tous ces hommes orgueilleux, dirent à Jérémie : Tu dis un mensonge ; Yahweh, notre Dieu, ne t'a pas envoyé nous dire : N'allez pas en Egypte pour y demeurer.
\VS{3}Mais Baruc, fils de Nérija, t'incite contre nous, afin de nous livrer entre les mains des Chaldéens, pour nous faire mourir, et pour nous faire transporter à Babylone.
\VS{4}Ainsi Jochanan, fils de Karéach, et tous les chefs des armées, et tout le peuple, n'obéirent pas à la voix de Yahweh, pour demeurer dans le pays de Juda.
\VS{5}Car Jochanan, fils de Karéach, et tous les chefs des armées, prirent tous les restes de Juda qui étaient revenus de toutes les nations, parmi lesquelles ils avaient été chassés, pour demeurer dans le pays de Juda ;
\VS{6}les hommes, et les femmes, et les enfants, et les filles du roi, et toutes les personnes que Nebuzaradan, chef des gardes, avait laissées avec Guedalia, fils d'Achikam, fils de Schaphan ; ils prirent aussi Jérémie le prophète et Baruc, fils de Nérija.
\VS{7}Et ils entrèrent dans le pays d'Egypte, car ils n'obéirent pas à la voix de Yahweh, et ils vinrent jusqu'à Tachpanès.
\VS{8}Alors la parole de Yahweh fut adressée à Jérémie, à Tachpanès, en disant :
\VS{9}Prends dans ta main de grandes pierres, et cache-les dans l'argile, dans le four à briques qui est à l'entrée de la maison de Pharaon à Tachpanès, sous les yeux des Juifs ;
\VS{10}et dis-leur : Ainsi parle Yahweh des armées, le Dieu d'Israël : Voici, j'enverrai chercher Nebucadnetsar, roi de Babylone, mon serviteur, et je mettrai son trône sur ces pierres que j'ai cachées, et il étendra son dais sur elles ;
\VS{11}et Il viendra et frappera le pays d'Egypte. Ceux qui sont destinés à la mort, iront à la mort ; et ceux qui sont destinés à la captivité, iront en captivité ; et ceux qui sont destinés à l’épée, seront livrés à l’épée\FTNT{Ez. 29:9.} !
\VS{12}Et j'allumerai le feu dans les maisons des dieux d'Egypte, Nebucadnetsar les brûlera, et il emmènera captifs ceux d'Egypte, et il se parera des richesses du pays d'Egypte, comme le pasteur s'enveloppe de son vêtement, et il en sortira en paix\FTNT{Es. 19:1 ; Ez. 30:13.}.
\VS{13}Il brisera aussi les statues de Beth-Schémesch, qui est au pays d'Egypte, et il brûlera par le feu les maisons des dieux d'Egypte.
\Chap{44}
\TextTitle{Yahweh avertit les Juifs d'Egypte\FTNTT{Jé. 43:8-13}}
\VerseOne{}La parole qui fut adressée à Jérémie sur tous les Juifs qui demeuraient au pays d'Egypte, qui habitaient à Migdol, à Tachpanès, à Noph, et au pays de Pathros, en disant :
\VS{2}Ainsi parle Yahweh des armées, le Dieu d'Israël : Vous avez vu tous les malheurs que j'ai fait venir sur Jérusalem et sur toutes les villes de Juda : Voici, elles ne sont plus aujourd'hui que des ruines, et personne n'y habite,
\VS{3}à cause des méchancetés qu'ils ont faites pour m'irriter, en allant brûler de l'encens pour servir d'autres dieux, qu'ils n'ont pas connu, ni eux, ni vous, ni vos pères.
\VS{4}Et je vous ai envoyé tous mes serviteurs, les prophètes, me levant dès le matin, et les envoyant, pour vous dire : Ne commettez pas maintenant cette chose abominable que je hais.
\VS{5}Mais ils n'ont pas écouté, ils n'ont pas prêté l'oreille pour se détourner de leur méchanceté, afin de ne pas faire brûler de l'encens à d'autres dieux.
\VS{6}C'est pourquoi ma fureur et ma colère se sont répandues sur eux, et ont embrasé les villes de Juda et les rues de Jérusalem, qui ne sont réduites en désert et en désolation, comme il paraît aujourd'hui.
\VS{7}Maintenant donc, ainsi parle Yahweh, le Dieu des armées, le Dieu d'Israël : Pourquoi faites-vous ce grand mal contre vos âmes, pour vous faire exterminer du milieu de Juda, hommes et femmes, petits enfants et ceux qui tètent, afin qu'on ne vous laisse aucun reste ?
\VS{8}En m'irritant par les œuvres de vos mains, en brûlant de l'encens à d'autres dieux au pays d'Egypte, où vous êtes venus pour y demeurer, afin de vous faire exterminer et d'être un objet de malédiction et d'opprobre parmi toutes les nations de la terre ?
\VS{9}Avez-vous oublié les crimes de vos pères, les crimes des rois de Juda, les crimes de leurs femmes, vos propres crimes et les crimes de vos femmes, commis dans le pays de Juda et dans les rues de Jérusalem ?
\VS{10}Jusqu'à ce jour, ils ne se sont pas humiliés, ils n'ont pas eu de crainte, ils n'ont pas marché dans ma loi ni dans mes ordonnances, que j'ai mises devant vous et devant vos pères.
\VS{11}C'est pourquoi ainsi parle Yahweh des armées, le Dieu d'Israël : Voici, je tourne ma face contre vous pour vous nuire et vous retrancher tout Juda\FTNT{Am. 9:4.}.
\VS{12}Et je prendrai les restes de ceux de Juda qui ont tourné le visage pour aller au pays d'Egypte afin d'y demeurer ; ils seront tous consumés, ils tomberont dans le pays d'Egypte ; ils seront consumés par l'épée, par la famine, depuis le plus petit jusqu'au plus grand ; ils mourront par l'épée et par la famine ; et ils seront en exécration, en étonnement, en malédiction et en opprobre.
\VS{13}Et je punirai ceux qui demeurent au pays d'Egypte, comme j'ai puni Jérusalem, par l'épée, par la famine, et par la peste.
\VS{14}Il n'y aura personne du reste de Juda qui est venu dans le pays d’Égypte pour y séjourner, échappera ou restera, pour retourner dans le pays de Juda, où ils aspirent à retourner pour y demeurer ; car pas un ne retournera, sinon les rescapés.
\VS{15}Mais les tous hommes qui savaient que leurs femmes brûlaient de l'encens à d'autres dieux, toutes les femmes qui se tenaient là en grande compagnie, et tout le peuple qui demeurait au pays d'Egypte, à Pathros, répondirent à Jérémie, en disant :
\VS{16}Quant à la parole que tu nous as dite au nom de Yahweh, nous ne t'écouterons pas.
\VS{17}Mais nous ferons assurément selon toute parole qui est sortie de notre bouche, brûler de l'encens à la reine des cieux\FTNT{Voir commentaire en Jé. 7:18.}, et lui faire des libations, comme nous l'avons fait, nous et nos pères, nos rois et nos chefs, dans les villes de Juda et dans les rues de Jérusalem. Alors nous étions rassasiés de pain, nous étions heureux, et nous ne voyions pas le malheur\FTNT{Ez. 16:24 ; Ez. 20:32.}.
\VS{18}Mais depuis le temps que nous avons cessé de brûler de l'encens à la reine des cieux et de lui faire des libations, nous avons manqué de tout, et nous avons été consumés par l'épée et par la famine…
\VS{19}Quand nous brûlions de l'encens à la reine des cieux et que nous lui faisions des libations, est-ce à l'insu de nos maris que nous lui faisons des gâteaux sur lesquels elle est représentée et que nous lui faisons des libations ?
\VS{20}Alors Jérémie parla à tout le peuple, aux hommes, aux femmes, et à tous ceux qui lui avaient donné cette réponse, et leur dit :
\VS{21}Yahweh ne s'est-il pas souvenu, ne lui est-il pas monté à cœur l'encens que vous avez brûlé dans les villes de Juda et dans les rues de Jérusalem, vous et vos pères, vos rois et vos chefs, et le peuple du pays ?
\VS{22}Yahweh n'a pas pu le supporter davantage, à cause de la méchanceté de vos actions, à cause des abominations que vous avez faites ; et votre pays est devenu une ruine, un désert et un objet de malédiction, sans que personne y habite, comme on le voit aujourd'hui.
\VS{23}C'est parce que vous avez brûlé de l'encens et que vous avez péché contre Yahweh, parce que vous n'avez pas écouté la voix de Yahweh, et que vous n'avez pas marché dans sa loi, ni dans ses ordonnances, ni dans ses témoignages, c'est pour cela que ces malheurs vous sont arrivés, comme on le voit aujourd'hui.
\VS{24}Jérémie dit à tout le peuple et à toutes les femmes : Vous tous de Juda, qui êtes au pays d'Egypte, écoutez la parole de Yahweh !
\VS{25}Ainsi parle Yahweh des armées, le Dieu d'Israël, en disant : Vous et vos femmes, vous avez parlé de vos bouches et accompli de vos mains, en disant : Certainement, nous accomplirons nos vœux que nous avons faits, brûler de l'encens à la reine des cieux, et lui faire des libations. Vous avez entièrement accompli vos vœux, vous les avez effectués très exactement.
\VS{26}C'est pourquoi, écoutez la parole de Yahweh, vous tous de Juda, qui demeurez au pays d'Egypte ! Voici, je le jure par mon grand Nom, dit Yahweh, mon Nom ne sera plus invoqué par la bouche d'aucun homme de Juda, et dans tout le pays d'Egypte aucun ne dira : Le Seigneur Yahweh est vivant !
\VS{27}Voici, je veille sur eux pour leur mal et non pour leur bien ; et tous les hommes de Juda qui sont dans le pays d'Egypte seront consumés par l'épée et par la famine, jusqu'à ce qu'ils soient exterminés\FTNT{Da. 9:14.}.
\VS{28}Et ceux qui seront échappés à l'épée, retourneront du pays d'Egypte au pays de Juda en fort petit nombre. Mais tout le reste de Juda, tous ceux qui sont venus dans le pays d'Egypte pour y demeurer, sauront quelle est la parole qui s'accomplira, la mienne ou la leur.
\VS{29}Et ceci sera pour vous le signe, dit Yahweh, que je vous punirai dans ce lieu, afin que vous sachiez que mes paroles s'accompliront infailliblement pour votre malheur.
\VS{30}Ainsi parle Yahweh : Voici, je livrerai Pharaon Hophra, roi d'Egypte, entre les mains de ses ennemis, entre les mains de ceux qui cherchent sa vie, comme j'ai livré Sédécias, roi de Juda, entre les mains de Nebucadnetsar, roi de Babylone, son ennemi, et qui cherchait sa vie.
\Chap{45}
\TextTitle{Yahweh explique son dessein à Baruc}
\VerseOne{}La parole que Jérémie, le prophète, adressa à Baruc, fils de Nérija, quand il écrivit dans un livre ces paroles, sous la dictée de Jérémie, la quatrième année de Jojakim, fils de Josias, roi de Juda. Il dit :
\VS{2}Ainsi parle Yahweh, le Dieu d'Israël, sur toi, Baruc :
\VS{3}Tu dis : Malheur à moi ! Car Yahweh ajoute la tristesse à ma douleur ; je me suis lassé dans mon gémissement, et je ne trouve pas de repos.
\VS{4}Tu lui diras : Ainsi parle Yahweh : Voici, je vais détruire ce que j'ai bâti, et arracher ce que j'ai planté, à savoir tout ce pays.
\VS{5}Et toi, chercherais-tu de grandes choses ? Ne les cherche pas ! Car voici, je vais faire venir du mal sur toute chair, dit Yahweh ; je te donnerai ta vie pour butin, dans tous les lieux où tu iras.
\Chap{46}
\TextTitle{Prophétie contre l'Egypte}
\VerseOne{}La parole de Yahweh qui fut adressée à Jérémie, le prophète, sur les nations.
\VS{2}A l'égard de l'Egypte, contre l'armée de Pharaon Neco, roi d'Egypte, qui était près du fleuve de l'Euphrate, à Carkemisch, et qui fut battue par Nebucadnetsar, roi de Babylone, la quatrième année de Jojakim, fils de Josias, roi de Juda\FTNT{2 R. 24:7.}.
\VS{3}Préparez le bouclier et l'écu et approchez-vous pour la bataille !
\VS{4}Attelez les chevaux, montez, cavaliers ! Présentez-vous avec vos casques, polissez vos lances, revêtez l'armure !
\VS{5}D'où vient que je vois ceci ? Ils sont effrayés, ils reviennent en arrière ; leurs hommes vaillants sont battus ; ils s'enfuient avec précipitation sans regarder derrière eux… La frayeur les environne, dit Yahweh.
\VS{6}Que l'homme léger à la course ne s'enfuie pas, et que le fort ne se sauve pas\FTNT{Am. 2:14-16.} ! Ils sont renversés et tombés vers le nord, auprès du rivage du fleuve de l'Euphrate.
\VS{7}Qui est celui-ci qui s'élève comme le Nil et dont les eaux sont agitées comme les fleuves ?
\VS{8}C'est l'Egypte. Elle s'élève comme le Nil et ses eaux agitées comme les fleuves ; et elle dit : Je m'élèverai et je couvrirai la terre ; je détruirai la ville et ceux qui y habitent.
\VS{9}Montez, chevaux ! Agissez en insensés, chars ! Que les hommes vaillants sortent, ceux d'Ethiopie et de Puth qui manient le bouclier, et ceux de Lud qui manient et tendent l'arc\FTNT{Ez. 30:5-9 ; Na. 3:9-10.} !
\VS{10}Car c'est le jour du Seigneur,  Yahweh des armées ; c'est un jour de vengeance, où il se venge de ses ennemis. L'épée dévore, elle se rassasie, elle s'enivre de leur sang. Car il y a des sacrifices pour le Seigneur, Yahweh des armées, dans le pays du nord, sur le fleuve de l'Euphrate\FTNT{Es. 34:5-6 ; Ez. 39:17 ; So. 1:7.}.
\VS{11}Monte en Galaad, prends du baume, vierge, fille de l'Egypte ! En vain tu multiplies les remèdes, il n'y a pas de guérison pour toi\FTNT{Ez. 30:21-25 ; Na. 3:19.}.
\VS{12}Les nations apprennent ta honte, et tes cris remplissent la terre, car les hommes forts chancellent l'un sur l'autre, ils tombent tous deux ensemble.
\VS{13}La parole que Yahweh prononça à Jérémie, le prophète, sur la venue de Nebucadnetsar, roi de Babylone, pour frapper le pays d'Egypte :
\VS{14}Déclarez-le en Egypte, et publiez-le à Migdol, à Noph, et à Tachpanès et Dites : Présente-toi, tiens-toi prêt car l'épée dévore ce qui est autour de toi !
\VS{15}Pourquoi tes vaillants hommes sont-ils emportés ? Ils ne tiennent pas ferme, parce que Yahweh les pousse.
\VS{16}Il en a terrassé un grand nombre, et même chacun tombe sur son compagnon, et ils disent : Levons-nous, retournons vers notre peuple, au pays de notre naissance, loin de l'épée de l'oppresseur !
\VS{17}Là, ils s'écrient : Pharaon, roi d'Egypte, n'est qu'un bruit ; il a laissé passer le temps fixé.
\VS{18}Je suis vivant ! dit le Roi, dont le Nom est Yahweh des armées ; comme le Thabor entre les montagnes, comme le Carmel qui s'avance dans la mer, ainsi viendra-t-il.
\VS{19}O fille, habitante de l'Egypte, fais tes bagages pour la captivité ! Car Noph sera un désert, elle sera brûlée, elle n'aura plus d'habitants.
\VS{20}L'Egypte est une très belle génisse… la destruction vient, elle vient du nord.
\VS{21}Memes les mercenaires aussi sont au milieu d'elle comme des veaux engraissés. Et eux aussi tournent le dos, ils fuient tous sans résister. Car le jour de leur malheur, le temps de leur châtiment est venu sur eux.
\VS{22}Elle sifflera comme un serpent ; car ils marcheront avec une puissante armée, ils viendront contre elle avec des haches, comme des bûcherons.
\VS{23}Ils couperont sa forêt, dit Yahweh, quoiqu'elle soit impénétrable ; parce que leur armée est en plus grand nombre que les sauterelles, on ne saurait la compter.
\VS{24}La fille de l'Egypte est confuse, elle est livrée entre les mains du peuple du nord.
\VS{25}Yahweh des armées, le Dieu d'Israël, dit : Voici, je vais punir Amon de No, Pharaon, l'Egypte, ses dieux et ses rois, Pharaon et ceux qui se confient en lui.
\VS{26}Et je les livrerai entre les mains de ceux qui cherchent leur vie, entre les mains, dis-je, de Nebucadnetsar, roi de Babylone, et entre les mains de ses serviteurs ; mais après cela, l'Egypte sera habitée comme aux temps passés, dit Yahweh.
\VS{27}Et toi, Jacob, mon serviteur, ne crains pas ; ne t'épouvante pas Israël ! Car voici, je te sauverai de la terre lointaine, je sauverai ta postérité du pays de leur captivité ; Jacob reviendra, il sera en repos et en paix, et il n'y aura personne qui lui fasse peur.
\VS{28}Toi donc,Jacob, mon serviteur, ne crains pas ! dit Yahweh ; car je suis avec toi. Et même je consumerai entièrement  toutes les nations parmi lesquelles je t'ai chassé, mais je ne te consumerai pas entièrement ; et je te châtierai avec justice, je ne te tiendrai pas tout à fait pour innocent.
\Chap{47}
\TextTitle{Prophétie contre la Philistie et la Phénicie}
\VerseOne{}La parole de Yahweh, qui fut adressée à Jérémie, le prophète, contre les Philistins, avant que Pharaon frappe Gaza.
\VS{2}Ainsi parle Yahweh : Voici des eaux montent du nord, elles sont comme un torrent qui déborde ; elles inondent le pays et ce qu'il contient, les villes et leurs habitants. Les hommes poussent des cris, et tous les habitants du pays se lamentent,
\VS{3}à cause du bruit des battements de sabots de ses puissants chevaux, du bruit de ses chars et au son de ses roues ; les pères ne se tournent pas vers leurs fils, tant les mains sont affaiblies,
\VS{4}parce que le jour vient où seront détruits tous les Philistins, exterminés tout le reste de ceux qui servaient de secours à Tyr et à Sidon ; car Yahweh va détruire les Philistins, les restes de l'île de Caphtor.
\VS{5}Gaza est devenue chauve, Askalon est perdue, le reste de leur plaine aussi. Jusqu'à quand te feras-tu des incisions ?
\VS{6}Ah ! Epée de Yahweh, quand te reposeras-tu ? Rentre dans ton fourreau, repose-toi, et sois tranquille !
\VS{7}Mais comment te reposerais-tu ? Car Yahweh lui donne ses ordres, il l'a assignée contre Askalon et contre le rivage de la mer.
\Chap{48}
\TextTitle{Prophétie sur Moab}
\VerseOne{}Sur Moab. Ainsi parle Yahweh des armées, le Dieu d'Israël : Malheur à Nebo, car elle est dévastée ! Kirjathaïm est honteuse, elle est prise ; Misgab est honteuse et brisée.
\VS{2} Moab ne se glorifiera plus à Hesbon, car on a machiné du mal contre elle en disant : Allons, exterminons-la, qu'elle ne soit plus une nation ! Toi aussi, Madmen, tu seras détruite ; l'épée te poursuivra.
\VS{3}Il y a un bruit de clameur qui vient de Choronaïm ; c'est un ravage, une grande ruine.
\VS{4}Moab est brisé ! On entend les cris des plus jeunes.
\VS{5}Pleurs sur pleurs s’élèveront à la montée de Luchith, car on entendra à la descente de Choronaïm\FTNT{Es. 15:5.}. ceux qui crieront à cause des plaies que les ennemis leur auront faites.
\VS{6}Fuyez, dira-t-on, sauvez vos vies, et soyez comme un misérable dans le désert !
\VS{7}Car, parce que tu as eu confiance dans tes ouvrages, et dans tes trésors, tu seras pris, et Kemosch sortira pour être transporté, avec ses sacrificateurs et ses chefs\FTNT{Es. 46:1-7.}.
\VS{8}Et le dévastateur entrera dans toutes les villes, et aucune ville n'échappera ; la vallée périra et la plaine sera détruite, comme Yahweh l'a dit.
\VS{9}Donnez des ailes à Moab, et qu'il parte en volant ! Ses villes seront réduites en désert, elles n'auront plus d'habitants.
\VS{10}Maudit soit celui qui fait l'œuvre de Yahweh avec paresse, maudit soit celui qui garde son épée pour répandre le sang !
\VS{11}Moab était tranquille depuis sa jeunesse, il reposait sur sa lie, il n'était pas vidé de vase en vase, et il n'allait pas en captivité. C'est pourquoi son goût lui est resté, et son odeur ne s'est pas changée.
\VS{12}Mais voici, les jours viennent, dit Yahweh, où je lui enverrai des gens qui le transvaseront, qui videront ses vases, et qui briseront ses outres.
\VS{13}Moab aura honte à cause de Kemosch, comme la maison d'Israël a eu honte à cause de Béthel, qui était sa confiance.
\VS{14}Comment dites-vous : Nous sommes de vaillants hommes, des soldats prêts à combattre ?
\VS{15}Moab est dévasté, et chacune de ses villes monte en fumée, l'élite de sa jeunesse est descendue pour être égorgée, dit le Roi, dont le nom est Yahweh des armées.
\VS{16}La calamité de Moab est proche, son malheur avance à grands pas.
\VS{17}Vous tous qui êtes autour de lui, soyez-en émus à compassion, et vous tous qui connaissez son nom, dites : Comment a été rompue cette forte verge, et ce sceptre d'honneur ? 
\VS{18}Toi qui te tiens chez la fille de Dibon, descends de ta gloire, et assieds-toi dans un lieu desséché ! Car le dévastateur de Moab monte contre toi, il détruit tes forteresses.
\VS{19}Habitante d'Aroër, tiens-toi sur le chemin, et regarde ! Interroge celui qui s'enfuit, celui qui s'échappe, et dis : Qu'est-il arrivé ?
\VS{20}Moab est rendu honteux, car il est brisé. Poussez des gémissements et des cris\FTNT{Es. 15:5 ; Es. 16:7.} ! Rapportez dans Arnon que Moab est dévasté !
\VS{21}Et que jugement est venu sur le pays de la plaine, sur Holon, sur Jahats, sur Méphaath,
\VS{22}Et sur Dibon, sur Nebo, sur Beth-Diblathaïm,
\VS{23}Et sur Kirjathaïm, sur Beth-Gamul, sur Beth-Meon,
\VS{24}Et sur Kerijoth, sur Botsra, sur toutes les villes du pays de Moab, éloignées et proches.
\VS{25}La force de Moab est abattue, et son bras est brisé, dit Yahweh.
\VS{26}Enivrez-le, car il s'est élevé contre Yahweh ! Moab se vautrera dans le vin qu'il aura rendu et deviendra aussi un sujet de moquerie !
\VS{27}Car ô Moab! Israël n'a-t-il pas été pour toi un objet de moquerie ? Avait-il été trouvé parmi les voleurs, pour que tu ne dises des paroles qu'en secouant la tête ?
\VS{28}Habitants de Moab, quittez les villes, et demeurez dans les rochers ! Soyez comme les colombes qui font leur nid aux côtés de l'entrée des cavernes !
\VS{29}Nous avons appris l'extrême orgueil de Moab, son arrogance, sa fierté, et son cœur hautain\FTNT{Es. 16:6 ; So. 2:9-10.}.
\VS{30}J'ai connu son orgueil, dit Yahweh ; mais il n'en sera pas ainsi ; j'ai connu ceux sur lesquels il s'appuie ; ils n'ont rien fait de droit. 
\VS{31}Je hurlerai donc à cause de Moab, même je crierai à cause de Moab tout entier ; on gémira sur les gens de Kir-Hérès.
\VS{32}O vignoble de Sibma, je pleurerai sur toi du pleur de Jaezer ; tes rameaux allaient au-delà de la mer, ils atteignaient la mer de Jaezer ; le dévastateur s'est jeté sur tes fruits d'été et sur ta vendange.
\VS{33}L'allégresse ausi et la et la joie se sont retirées loin des campagnes et du pays de Moab ; j'ai fait cesser le vin des cuves ; on ne foule plus gaîment au pressoir ; il y a des cris de guerre, et non des cris de joie\FTNT{Es. 16:10.}.
\VS{34}A cause de cris de Hesbon qui est parvenu jusqu'à Elealé, et ils font entendre leurs cris jusqu'à Jahats, même depuis Tsoar jusqu'à Choronaïm, jusqu'à Eglath-Schelischija ; car les eaux de Nimrim seront aussi réduites en désolation.
\VS{35}Je ferai cesser en Moab, dit Yahweh, celui qui offre sur les hauts lieux et celui qui brûle de l’encens à ses dieux. 
\VS{36}C’est pourquoi mon cœur mènera un bruit sur Moab, comme des flûtes ; mon cœur mènera un bruit comme des flûtes sur les hommes de Kir-Hérès, parce que tous les biens qu'ils ont acquis ont péri.
\VS{37}Car toutes les têtes sont chauves, toutes les barbes sont coupées ; et il y a des incisions sur toutes les mains, et sur les reins des sacs.
\VS{38}Il y aura des lamentations sur tous les toits de Moab et dans ses places, parce que j'aurai brisé Moab comme un vase auquel on ne prend nul plaisir, dit Yahweh.
\VS{39}Hurlez, en disant : Comment a-t-il été mis en pièces ? brisé ! Comment Moab a-t-il tourné honteusement le dos ! Car Moab sera un objet de moquerie et de frayeur pour tous ceux qui sont autour de lui.
\VS{40}Car ainsi parle Yahweh : Voici, il volera comme un aigle, et il étendra ses ailes sur Moab.
\VS{41}Kerijoth est prise, les forteresses sont saisies, et le cœur des hommes forts de Moab est en ce jour comme le cœur d'une femme qui est en travail.
\VS{42}Et Moab sera exterminé, il ne sera plus un peuple, parce qu'il s'est élevé contre Yahweh.
\VS{43}Habitant de Moab, la frayeur, la fosse, et le filet sont sur toi ! dit Yahweh.
\VS{44}Celui qui s'enfuira à cause de la frayeur tombera dans la fosse, et celui qui remontera de la fosse sera au filet ; car je fera venir sur lui, sur Moab, l'année de son châtiment, dit Yahweh\FTNT{Es. 24:18.}.
\VS{45}Ils se sont arrêtés à l'ombre de Hesbon, voulant éviter la force ; mais le feu sort de Hesbon, une flamme du milieu de Sihon ; elle dévore les flancs de Moab, et le sommet de la tête des fils du tumulte\FTNT{No. 21:28.}.
\VS{46}Malheur à toi, Moab ! Le peuple de Kemosch est perdu ! Car tes fils sont enlevés et emmenés captifs, et tes filles ont été emmenées captives.
\VS{47}Toutefoi je ramènerai et mettrai en repos les captifs de Moab, aux derniers jours\FTNT{Gn. 49:1-2.}, dit Yahweh. Là est le jugement de Moab.
\Chap{49}
\TextTitle{Prophétie sur Ammon}
\VerseOne{}Sur les fils d'Ammon. Ainsi parle Yahweh : Israël n'a-t-il pas de fils ? N'a-t-il pas d'héritier ? Pourquoi donc Malcom hérite-t-il de Gad, et pourquoi son peuple demeure-t-il dans ses villes ?
\VS{2}C'est pourquoi, les jours viennent, dit Yahweh, où je ferai entendre le cri de guerre contre Rabbath des fils d'Ammon ; elle sera réduite en un monceau de ruines, et les villes de son ressort seront brûlées par le feu ; Israël possédera ceux qui l'auront possédé, dit Yahweh.
\VS{3}Hurle, ô Hesbon, car Aï est dévastée ! Poussez des cris, filles de Rabba, ceignez-vous de sacs, lamentez-vous, courez ça et là le long des murailles ! Car Malcom s'en va en captivité avec ses sacrificateurs et ses chefs.
\VS{4}Pourquoi te glorifies-tu de tes vallées ? Ta vallée se fond, fille rebelle, qui te confiais dans tes trésors : Qui viendra contre moi ?
\VS{5}Voici, je fais venir sur toi la terreur, dit le Seigneur, Yahweh des armées, de tous les alentours ; vous serez chassés chacun çà et là, et il n'y aura personne qui rassemblera les fuyards.
\VS{6}Mais après cela, je ramènerai les captifs des fils d'Ammon, dit Yahweh.
\TextTitle{Prophétie sur Edom}
\VS{7}Sur Edom. Ainsi parle Yahweh des armées : N'y a-t-il plus de sagesse dans Théman ? Le conseil a-t-il manqué aux hommes intelligents ? Leur sagesse s'est-elle évanouie\FTNT{Ab. 1:8.} ?
\VS{8}Fuyez, tournez le dos, demeurez dans les cavernes, habitants de Dedan ! Car je fais venir la détresse sur Esaü, le temps de son châtiment.
\VS{9}Si des vendangeurs entrent chez toi, ne laissent-ils rien à grappiller ? Si des voleurs viennent de nuit, ils ne pillent que ce qu'ils peuvent.
\VS{10}Mais je dépouillerai Esaü, je découvrirai ses lieux secrets, il ne pourra se cacher ; sa postérité, ses frères, et ses voisins, périront, et il ne sera plus.
\VS{11}Laisse tes orphelins, je les ferai vivre, et que tes veuves se confient en moi !
\VS{12}Car ainsi parle Yahweh : Voici, ceux dont le jugement n'était pas de boire la coupe, la boiront certainement ; et toi, tu resterais impuni ! Tu ne resteras pas impuni, tu la boiras.
\VS{13}Car je le jure par moi-même, dit Yahweh, que Botsra sera un objet de désolation, d'opprobre, de dévastation, et de malédiction, et que toutes ses villes deviendront des ruines éternelles.
\VS{14}J'ai entendu de Yahweh une nouvelle, et un messager a été envoyé parmi les nations : Assemblez-vous, et venez contre elle ! Levez-vous pour la guerre !
\VS{15}Car voici, je te rendrai petit entre les nations, méprisé entre les hommes.
\VS{16}Mais ta présomption, l'orgueil de ton cœur t'a séduit, toi qui habites dans le creux des rochers, et qui occupes le sommet des collines. Quand tu aurais élevé ton nid comme l'aigle, je t'en ferai descendre, dit Yahweh.
\VS{17}Edom sera un objet de désolation ; quiconque passera près de lui sera étonné, et sifflera à cause de toutes ses plaies.
\VS{18}Comme Sodome et Gomorrhe et les villes voisines qui furent détruites, dit Yahweh, il ne sera plus habité par des hommes, il ne sera le séjour d'aucun fils d'homme\FTNT{Ge. 19:25 ; Am. 4:11.}…
\VS{19}Voici, il monte comme un lion des rives orgueilleuses du Jourdain, vers la demeure forte ; soudain, j'en ferai fuir Edom, et j'établirai sur elle celui que j'ai choisi. Car qui est semblable à moi ? Qui me donnera des ordres ? Et quel est le chef qui me résistera en face\FTNT{Job. 41:1.} ?
\VS{20}C'est pourquoi écoutez le conseil que Yahweh a donné contre Edom, et les desseins qu'il a projetés contre les habitants de Théman ! Certainement, on les traînera comme les plus petits du troupeau, certainement on dévastera leur demeure.
\VS{21}La terre tremble au bruit de leur chute ; le bruit de leur cri se fait entendre jusqu'à la Mer Rouge…
\VS{22}Voici, il monte comme un aigle, il vole, il étend ses ailes sur Botsra, et le cœur des hommes forts d'Edom est en ce jour comme le cœur d'une femme en travail.
\TextTitle{Prophétie sur Damas}
\VS{23}Sur Damas. Hamath et Arpad sont honteuses parce qu'elles ont entendu des mauvaises nouvelles, elles tremblent ; il y a une tourmente dans la mer qui ne peut se calmer.
\VS{24}Damas est défaillante, elle se tourne pour fuir, et la panique la saisit ; l'angoisse et les douleurs la saisissent comme une femme qui enfante.
\VS{25}Ah ! Elle n'est pas abandonnée, la ville glorieuse, ma ville de plaisance !
\VS{26}C'est pourquoi ses jeunes gens tomberont dans les places, et tous ses hommes de guerre périront en ce jour, dit Yahweh des armées.
\VS{27}Je mettrai le feu à la muraille de Damas, qui dévorera les palais de Ben-Hadad.
\TextTitle{Prophétie sur Kédar (les Arabes) et Hatsor}
\VS{28}Sur Kédar et les royaumes de Hatsor, que Nebucadnetsar, roi de Babylone, frappa. Ainsi parle Yahweh : Levez-vous, montez vers Kédar, et détruisez les fils d'orient !
\VS{29}On prendra leurs tentes et leurs troupeaux, on prendra leurs tentes, tous leurs bagages et leurs chameaux, et l'on jettera de toutes parts contre eux des cris de terreur.
\VS{30}Fuyez, fuyez de toutes vos forces, cherchez une demeure dans les cavernes, vous habitants de Hatsor ! dit Yahweh ; car Nebucadnetsar, roi de Babylone, a pris une résolution contre vous, il a imaginé un plan contre vous.
\VS{31}Levez-vous, montez vers la nation tranquille qui habite en sécurité, dit Yahweh ; elle n'a ni portes ni barres, elle habite seule\FTNT{Ez. 38:11.}.
\VS{32}Leurs chameaux seront au pillage, et la multitude de leur bétail sera une proie ; je les disperserai à tout vent, vers ceux qui se coupent le coin de la barbe, et je ferai venir de tous les côtés leur détresse, dit Yahweh.
\VS{33}Hatsor sera le repaire des serpents, un désert pour toujours ; personne n'y habitera, et aucun fils d'homme n'y séjournera.
\TextTitle{Prophétie sur Elam}
\VS{34}La parole de Yahweh fut adressée à Jérémie, le prophète, sur Elam, au commencement du règne de Sédécias, roi de Juda, en disant :
\VS{35}Ainsi parle Yahweh des armées : Voici, je vais briser l'arc d'Elam, qui est sa principale force\FTNT{Ez. 32:24-27.}.
\VS{36}Je ferai venir contre Elam les quatre vents des quatre extrémités des cieux, je les disperserai par tous ces vents ; et il n'y aura pas une nation où ne viennent ceux qui seront chassés d'Elam éternellement.
\VS{37}Je ferai trembler les habitants d'Elam devant leurs ennemis, et devant ceux qui cherchent leur vie, je ferai venir le malheur sur eux, l'ardeur de ma colère, dit Yahweh, et j'enverrai l'épée après eux, jusqu'à ce que je les aie consumés.
\VS{38}Je mettrai mon trône dans Elam, et j'en détruirai les rois et les chefs, dit Yahweh.
\VS{39}Mais dans les derniers jours\FTNT{Gn. 49:1-2.}, je ramènerai les captifs d'Elam, dit Yahweh.
\Chap{50}
\TextTitle{Prophétie sur Babylone}
\VerseOne{}La parole que Yahweh prononça sur Babylone, sur le pays des Chaldéens, par le moyen de Jérémie, le prophète :
\VS{2}Annoncez-le parmi les nations, entendez-le, levez une bannière ! Entendez-le, ne le cachez pas ! Dites : Babylone est prise ! Bel est confus, Merodac est brisé ! Ses idoles sont confuses et brisées\FTNT{Es. 46:1.} !
\VS{3}Car une nation monte contre elle du nord, qui mettra son pays en ruines, il n'y aura plus personne qui y habite ; les hommes et les bêtes fuient, ils s'en vont.
\VS{4}En ces jours, et en ce temps-là, dit Yahweh, les fils d'Israël et les fils de Juda viendront ensemble ; ils marcheront en pleurant, et en cherchant Yahweh, leur Dieu.
\VS{5}Ils demanderont la route de Sion, ils tourneront leur visage vers elle : Venez, attachez-vous à Yahweh, par une alliance éternelle qui ne soit jamais oubliée !
\VS{6}Mon peuple était comme un troupeau de brebis perdues ; leurs bergers les égaraient, les rendaient errantes par les montagnes ; elles allaient de montagne en colline, oubliant leur bercail\FTNT{Ez. 34:5-6 ; Za. 10:2 ; Mt. 9:36.}.
\VS{7}Tous ceux qui les trouvaient les dévoraient, et leurs ennemis disaient : Nous ne sommes coupables d'aucun mal, parce qu'ils ont péché contre Yahweh, contre la demeure de la justice, contre Yahweh, l'espérance de leurs pères.
\VS{8}Fuyez du milieu de Babylone, sortez du pays des Chaldéens, et soyez comme les boucs qui vont devant le troupeau\FTNT{Es. 48:20 ; 2 Co. 6:17 ; Ap. 18:4.} !
\VS{9}Car voici, je vais susciter et faire monter contre Babylone une multitude de grandes nations du pays du nord ; elles se rangeront en bataille contre elle, de sorte qu'elle sera prise ; leurs flèches font des ravages comme celles d'un habile guerrier qui ne retourne pas à vide\FTNT{Es. 13:18.}.
\VS{10}Et la Chaldée sera abandonnée au pillage ; tous ceux qui la pilleront seront rassasiés, dit Yahweh.
\VS{11}Oui, réjouissez-vous, soyez dans l'allégresse, vous qui avez pillé mon héritage ! Oui, bondissez comme une génisse qui est dans l'herbe, hennissez comme de puissants chevaux !
\VS{12}Votre mère est fort honteuse, celle qui vous a enfantés rougit de honte ; voici, elle est la dernière entre les nations, c'est un désert, un pays sec et aride.
\VS{13}Elle ne sera plus habitée à cause de la colère de Yahweh, elle ne sera plus qu'une désolation. Quiconque passera près de Babylone sera étonné et sifflera à cause de toutes ses plaies.
\VS{14}Rangez-vous en bataille contre Babylone, mettez-vous tout alentour vous tous qui tendez l'arc ! Tirez contre elle, n'épargnez pas les flèches ! Car elle a péché contre Yahweh.
\VS{15}Poussez des cris de guerre contre elle tout alentour ! Elle tend les mains ; ses fondements tombent ; ses murs sont renversés. Car c'est ici la vengeance de Yahweh. Vengez-vous sur elle ! Faites-lui comme elle a fait\FTNT{Ab. 1:15 ; Ps. 137:8 ; Lu. 6:38.} !
\VS{16}Retranchez de Babylone le semeur, et celui qui manie la faucille au temps de la moisson ! Devant l'épée de l'oppresseur, que chacun se tourne vers son peuple, que chacun s'enfuie vers son pays.
\VS{17}Israël est comme une brebis égarée que les lions ont chassée ; le roi d'Assyrie l'a dévorée le premier ; mais ce dernier, Nebucadnetsar, roi de Babylone, lui a brisé les os.
\VS{18}C'est pourquoi ainsi parle Yahweh des armées, le Dieu d'Israël : Voici, je punirai le roi de Babylone et son pays, comme j'ai puni le roi d'Assyrie\FTNT{Es. 37:36 ; 2 R. 19:35.}.
\VS{19}Je ramènerai Israël dans sa demeure ; il paîtra au Carmel et au Basan, et son âme se rassasiera sur la montagne d'Ephraïm et de Galaad.
\VS{20}En ces jours, et en ce temps-là, dit Yahweh, on cherchera l'iniquité d'Israël, mais il n'y en aura pas, les péchés de Juda ne seront pas trouvés ; car je pardonnerai au reste que j'aurai fait demeurer.
\VS{21}Monte contre ce pays doublement rebelle, contre les habitants, et châtie-les ! Massacre, extermine-les ! dit Yahweh, fais selon toutes les choses que je t'ai ordonnées.
\VS{22}Des cris de guerre retentissent dans le pays, et la ruine est grande.
\VS{23}Eh quoi ! Il est rompu, brisé, le marteau de toute la terre ! Babylone est réduite en une désolation parmi les nations !
\VS{24}Je t'ai tendu un piège, et tu as été prise, Babylone, et tu n'en savais rien ; tu as été trouvée, et même attrapée, parce que tu as lutté contre Yahweh.
\VS{25}Yahweh a ouvert son arsenal et en a sorti les armes de sa colère ; c'est là une œuvre du Seigneur, de Yahweh des armées, dans le pays des Chaldéens.
\VS{26}Venez de toutes parts dans Babylone, ouvrez ses greniers, faites-y des monceaux, comme des tas de gerbes, et détruisez-la ! Qu'il ne reste plus rien d'elle !
\VS{27}Tuez tous ses taureaux et qu'ils descendent à l'abattage ! Malheur à eux ! Car le jour est venu, le temps de leur châtiment.
\VS{28}Ecoutez la voix de ceux qui s'enfuient, de ceux qui sont échappés du pays de Babylone pour annoncer dans Sion la vengeance de Yahweh, notre Dieu, la vengeance de son temple !
\VS{29}Appelez les archers contre Babylone, vous tous qui tendez l'arc ! Campez-vous contre elle tout alentour, que personne n'échappe, rendez-lui selon ses œuvres, faites-lui selon tout ce qu'elle a fait ! Car elle s'est fièrement élevée contre Yahweh, contre le Saint d'Israël\FTNT{Es. 13:11 ; Joë. 3:4-9 ; La. 1:22.}.
\VS{30}C'est pourquoi ses jeunes gens tomberont dans les places, et tous ses hommes de guerre périront en ce jour, dit Yahweh.
\VS{31}Voici, j'en veux à toi, orgueilleuse ! dit le Seigneur, Yahweh des armées ; car ton jour est venu, le temps de ton châtiment.
\VS{32}L'orgueilleuse chancellera et tombera, et il n'y aura personne pour la relever ; je mettrai le feu à ses villes, et il dévorera tous ses environs.
\VS{33}Ainsi parle Yahweh des armées : Les fils d'Israël et les fils de Juda sont ensemble opprimés ; tous ceux qui les ont emmenés captifs les retiennent, et refusent de les laisser aller.
\VS{34}Leur Rédempteur est fort, son nom est Yahweh des armées ; il défendra certainement leur cause, pour donner du repos au pays, et pour faire trembler les habitants de Babylone.
\VS{35}L'épée est sur les Chaldéens ! dit Yahweh, sur les habitants de Babylone, ses chefs, et ses sages !
\VS{36}L'épée est tirée contre ses devins de mensonges ! Qu'ils soient comme des insensés ! L'épée contre ses hommes forts ! Qu'ils soient épouvantés !
\VS{37}L'épée est sur ses chevaux et sur ses chars ! Contre les foules de toute espèce qui sont au milieu d'elle ! Qu'ils deviennent comme des femmes ! L'épée est sur ses trésors ! Qu'ils soient pillés !
\VS{38}La sécheresse est sur ses eaux ! Qu'elles soient mises à sec ! Parce que c'est un pays d'idoles ; ils agissent en insensés à l'égard de leurs idoles\FTNT{Es. 2:8.}.
\VS{39}C'est pourquoi les animaux du désert y habiteront avec les chacals, et les autruches y habiteront aussi ; elle ne sera plus jamais habitée, et on n'y demeurera plus jamais.
\VS{40}Comme Sodome et Gomorrhe, et les villes voisines que Dieu détruisit, dit Yahweh, elle ne sera plus habitée par des hommes, elle ne sera le séjour d'aucun fils d'homme.
\VS{41}Voici, un peuple vient du nord, une grande nation et plusieurs rois se lèvent des extrémités de la terre.
\VS{42}Ils saisissent l'arc et le javelot ; ils sont cruels, et ils n'ont pas de compassion ; leur voix mugit comme la mer ; ils sont montés sur des chevaux, chacun d'eux est rangé en bataille comme un seul homme, contre toi, fille de Babylone !
\VS{43}Le roi de Babylone entend la nouvelle, et ses mains s'affaiblissent, l'angoisse le saisit comme la douleur de celle qui enfante…
\VS{44}Voici, il monte comme un lion des rives orgueilleuses du Jourdain vers la demeure forte ; soudain, je les ferai courir, et je désignerai sur elle celui que j'ai choisi. Car qui est semblable à moi ? Qui me donnera des ordres ? Et quel est le chef qui me résistera en face ?
\VS{45}C'est pourquoi écoutez le conseil que Yahweh a donné contre Babylone, et les desseins qu'il a projetés contre le pays des Chaldéens ! Certainement, on les traînera comme les plus petits du troupeau, certainement on dévastera leur demeure.
\VS{46}La terre tremble au bruit de la prise de Babylone, et le cri se fait entendre parmi les nations.
\Chap{51}
\TextTitle{Le jugement de Babylone par Yahweh}
\VerseOne{}Ainsi parle Yahweh : Voici, je fais lever un vent destructeur contre Babylone et contre ceux qui habitent au cœur du royaume.
\VS{2}J'envoie contre Babylone des vanneurs qui la vanneront, qui videront son pays ; car de tous côtés ils seront contre elle, au jour du malheur.
\VS{3}Qu'on bande l'arc contre celui qui bande son arc, contre celui qui s'élève dans son armure ! N'épargnez pas ses jeunes hommes ! Exterminez toute son armée !
\VS{4}Qu'ils tombent les blessés à mort dans le pays des Chaldéens, percés de coups dans les rues de Babylone !
\VS{5}Car Israël et Juda ne sont pas abandonnés de leur Dieu, de Yahweh des armées, quoique leur pays ai été trouvé par le Saint d'Israël plein de crimes.
\VS{6}Fuyez hors de Babylone, et que chacun sauve sa vie, ne soyez point exterminés dans son iniquité ; car c'est le temps de la vengeance de Yahweh ; il lui rend ce qu'elle a mérité.
\VS{7}Babylone était comme une coupe d'or dans la main de Yahweh, enivrant toute la terre ; les nations ont bu de son vin : C'est pourquoi les nations ont agi comme des insensées.
\VS{8}Babylone est tombée\FTNT{Ap. 18.} en un instant, elle est brisée ! Gémissez sur elle, prenez du baume pour sa douleur : Peut-être qu'elle guérira.
\VS{9}Nous avons pansé Babylone, mais elle n'a pas guéri. Laissons-la et allons-nous-en chacun dans son pays ; car son jugement atteint les cieux et s'élève jusqu'aux nues.
\VS{10}Yahweh a rendu la justice de notre cause ; venez et racontons dans Sion l'œuvre de Yahweh, notre Dieu.
\VS{11}Aiguisez les flèches, remplissez vos mains avec les boucliers ! Yahweh a réveillé l'esprit des rois de Médie, car sa pensée est de détruire Babylone ; c'est ici la vengeance de Yahweh, la vengeance de son temple.
\VS{12}Elevez une bannière contre les murs de Babylone ! Fortifiez les postes, levez des gardes, préparez des embuscades ! Car Yahweh a formé un projet, il fait ce qu'il a dit contre les habitants de Babylone.
\VS{13}Toi qui habites près des grandes eaux, abondantes en trésors, ta fin est venue, ta cupidité est à son terme !
\VS{14}Yahweh des armées a juré par lui-même, en disant : Je te remplirai d'hommes comme de sauterelles, et ils pousseront contre toi des cris de guerre.
\VS{15}C'est lui qui a fait la terre par sa puissance, qui a fondé le monde habitable par sa sagesse, et qui a étendu les cieux par son intelligence\FTNT{Ge. 1:1 ; Es. 40:22 ; Ps. 104:2; Job. 9:8.}.
\VS{16}Lorsqu'il donne de la voix, il y a un tumulte d'eaux dans les cieux, il fait monter les vapeurs des extrémités de la terre, il fait les éclairs et la pluie, il fait sortir le vent de ses réservoirs.
\VS{17}Tout homme devient stupide par sa connaissance, tout fondeur est honteux par les images taillées ; car ses idoles en métal fondu ne sont que mensonge, il n'y a pas de souffle en elles.
\VS{18}Elles ne sont que vanité, une œuvre de tromperie ; elles périront au temps de leur châtiment.
\VS{19}La portion de Jacob n'est pas comme ces choses-là ; car c'est lui qui a tout formé, et Israël est la tribu de son héritage. Son nom est Yahweh des armées.
\VS{20}Tu as été pour moi un marteau, un instrument de guerre. Par toi j'ai brisé des nations, par toi j'ai détruit des royaumes.
\VS{21}Par toi j'ai brisé le cheval et son cavalier ; par toi j'ai brisé le char et celui qui était monté dessus.
\VS{22}Par toi j'ai brisé l'homme et la femme ; par toi j'ai brisé le vieillard et le jeune garçon ; par toi j'ai brisé le jeune homme et la jeune fille.
\VS{23}Par toi j'ai brisé le berger et son troupeau ; par toi j'ai brisé le laboureur et ses bœufs ; par toi j'ai brisé les gouverneurs et les chefs.
\VS{24}Mais je rendrai à Babylone et à tous les habitants de la Chaldée tout le mal qu'ils ont fait à Sion sous vos yeux, dit Yahweh\FTNT{La. 1:21.}.
\VS{25}Voici, j'en veux à toi, montagne de destruction, dit Yahweh, à toi qui détruisais toute la terre ! J'étendrai ma main sur toi, je te roulerai du haut des rochers, je ferai de toi une montagne embrasée.
\VS{26}On ne pourra prendre de toi aucune pierre pour la placer à l'angle de l'édifice ni aucune pierre pour servir de fondement ; car tu seras une ruine éternelle, dit Yahweh\FTNT{Es. 13:19-20.}…
\VS{27}Elevez une bannière dans le pays ! Sonnez du shofar parmi les nations ! Préparez les nations contre elle, appelez contre elle les royaumes d'Ararat, de Minni et d'Aschkenaz ! Établissez contre elle des chefs ! Faites monter ses chevaux comme des sauterelles hérissées !
\VS{28}Préparez contre elle les nations, les rois de Médie, ses gouverneurs et tous ses chefs, et tout le pays sous leur domination\FTNT{Es. 13:17.} !
\VS{29}La terre tremble, elle se tord ; car la pensée de Yahweh se dresse contre Babylone ; il va faire du pays de Babylone un désert sans habitants\FTNT{Es. 13:14 ; Joë. 3:16.}.
\VS{30}Les hommes forts de Babylone cessent de combattre, ils demeurent dans les forteresses ; leur force est épuisée, ils sont comme des femmes. On met le feu aux demeures, on brise les barres.
\VS{31}Le courrier rencontre le courrier, et le messager rencontre le messager, pour annoncer au roi de Babylone que sa ville est prise par tous les côtés,
\VS{32}que les gués sont saisis, les marais brûlés par le feu, et les hommes de guerre épouvantés.
\VS{33}Car ainsi parle Yahweh des armées, le Dieu d'Israël : La fille de Babylone est comme une aire dans le temps où on la foule ; encore un peu de temps, et le moment de la moisson sera venu pour elle.
\VS{34}Nebucadnetsar, roi de Babylone, m'a dévorée, m'a détruite ; il a fait de moi un vase vide ; il m'a engloutie tel un dragon, il a rempli son ventre de mes délices ; il m'a chassée au loin.
\VS{35}Que la violence envers moi et ma chair déchirée retombe sur Babylone ! dit l'habitante de Sion. Que mon sang retombe sur les habitants de la Chaldée ! dit Jérusalem.
\VS{36}C'est pourquoi ainsi parle Yahweh : Voici, je défendrai ta cause, je te vengerai ! Je dessécherai la mer de Babylone, et je ferai tarir sa source.
\VS{37}Babylone sera un monceau de ruines, un repaire de serpents, un objet d'épouvante et de moquerie ; sans que personne n'y habite.
\VS{38}Ils rugiront ensemble comme des lions, ils pousseront des cris comme des lionceaux.
\VS{39}Quand ils seront échauffés, je les ferai boire, et les enivrerai, afin qu'ils se réjouissent et qu'ils dorment d'un sommeil éternel et qu'ils ne se réveillent plus, dit Yahweh.
\VS{40}Je les ferai descendre comme des agneaux à la boucherie, comme des béliers et des boucs.
\VS{41}Eh quoi ! Schéschac est prise ! Celle dont la louange remplissait toute la terre est conquise ! Eh quoi ! Babylone est réduite en désolation parmi les nations !
\VS{42}La mer est montée sur Babylone, elle a été couverte de la multitude de ses flots\FTNT{Es. 8:8 ; Ez. 26:3-19 ; Lu. 21:25.}.
\VS{43}Ses villes sont en ruines, une terre sèche et déserte ; c'est un pays où personne ne demeure, et où il ne passe aucun fils d'homme.
\VS{44}Je punirai aussi Bel à Babylone, je sortirai de sa bouche ce qu'il a englouti, et les nations n'aborderont plus vers lui. Le mur même de Babylone est tombé !
\VS{45}Mon peuple, sortez du milieu d'elle, et que chacun sauve sa vie de l'ardeur de la colère de Yahweh.
\VS{46}Que votre cœur ne se trouble pas, et ne craignez pas les nouvelles qu'on entendra dans tout le pays ; car cette année viendra une nouvelle, et l'année d'après une autre nouvelle, et il y aura violence dans le pays, et un dominateur s'élèvera contre un autre dominateur.
\VS{47}C'est pourquoi voici, les jours viennent où je punirai les idoles de Babylone, et tout son pays sera honteux ; tous ses morts tomberont au milieu d'elle.
\VS{48}Les cieux, la terre, et tout ce qui y est, pousseront des cris de joie contre Babylone ; car du nord les dévastateurs viendront contre elle, dit Yahweh.
\VS{49}Babylone tombera, ô morts d'Israël, comme Babylone a fait tomber les morts de tout le pays.
\VS{50}Vous qui avez échappé à l'épée, allez, ne vous arrêtez pas ! Souvenez-vous de Yahweh dans ces pays éloignés, et que Jérusalem revienne à vos cœurs !
\VS{51}Nous étions honteux des reproches que nous entendions ; la honte couvrait nos visages, quand les étrangers sont venus dans le sanctuaire de la maison de Yahweh.
\VS{52}C'est pourquoi, voici, les jours viennent, dit Yahweh, où je châtierai ses idoles ; et les blessés gémiront dans tout son pays.
\VS{53}Quand Babylone monterait jusqu'aux cieux et qu'elle rendrait inaccessible le plus haut de sa forteresse, alors les dévastateurs viendront contre elle, dit Yahweh\FTNT{Am. 9:2 ; Ab. 1:4.}…
\VS{54}Un grand cri s'entend de Babylone, et la ruine est grande dans le pays des Chaldéens.
\VS{55}Parce que Yahweh dévaste Babylone, il en fait échapper de grands cris ; les flots des dévastateurs mugissent comme de grandes eaux, le bruit du mugissement s'étend.
\VS{56}Car le destructeur est venu contre elle, contre Babylone ; ses hommes forts sont pris, leurs arcs sont brisés. Car Yahweh est un Dieu qui rend à chacun selon ses œuvres, qui paie à chacun son salaire.
\VS{57}J'enivrerai ses princes et ses sages, ses gouverneurs, ses chefs, et ses hommes forts ; ils dormiront d'un sommeil éternel, et ils ne se réveilleront plus, dit le Roi dont le nom est Yahweh des armées.
\VS{58}Ainsi parle Yahweh des armées : Les larges murs de Babylone seront renversés, ses portes qui sont si hautes, seront brûlées par le feu ; ainsi les peuples auront travaillé en vain, les nations se seront fatiguées pour le feu.
\VS{59}C'est ici l'ordre que Jérémie, le prophète, donna à Seraja, fils de Nérija, fils de Machséja, lorsqu'il alla avec Sédécias, roi de Juda, la quatrième année du règne de Sédécias. Or Seraja était premier chambellan.
\VS{60}Jérémie écrivit dans un livre tous les malheurs qui devaient venir sur Babylone, toutes ces paroles qui sont écrites contre Babylone.
\VS{61}Jérémie dit à Seraja : Lorsque tu seras venu à Babylone, et que tu auras vu, tu liras toutes ces paroles,
\VS{62}et tu diras : Yahweh, c'est toi qui as déclaré que ce lieu serait exterminé, en sorte qu'il n'y ait aucun habitant, depuis l'homme jusqu'à la bête, mais qu'il deviendrait un désert pour toujours.
\VS{63}Et quand tu auras achevé de lire ce livre, tu le lieras à une pierre et tu le jetteras dans l'Euphrate,
\VS{64}et tu diras : Ainsi Babylone sera submergée, elle ne se lèvera pas des malheurs que je ferai venir sur elle ; ils seront épuisés. Jusqu'ici sont les paroles de Jérémie.
\Chap{52}
\TextTitle{Chute de Jérusalem et destruction du temple ; Juda déporté à Babylone\FTNTT{2 R. 25:1-26; Jé. 39:1-10.}}
\VerseOne{}Sédécias avait vingt et un ans lorsqu'il devint roi, et il régna onze ans à Jérusalem. Sa mère se nommait Hamuthal, fille de Jérémie, de Libna\FTNT{2 R. 24 et 25.}.
\VS{2}Il fit ce qui est mal aux yeux de Yahweh, comme avait fait Jojakim.
\VS{3}Et cela arriva à cause de la colère de Yahweh contre Jérusalem et Juda, qu'il voulait chasser de devant sa face. Sédécias se rebella contre le roi de Babylone.
\VS{4}La neuvième année de son règne, le dixième jour du dixième mois, Nebucadnetsar, roi de Babylone, vint contre Jérusalem, lui et toute son armée ; ils campèrent devant elle et construisirent des retranchements tout alentour.
\VS{5}La ville fut assiégée jusqu'à la onzième année du roi Sédécias.
\VS{6}Le neuvième jour du quatrième mois, la famine était forte dans la ville, et il n'y avait pas de pain pour le peuple du pays\FTNT{La. 2:11-12.}.
\VS{7}Alors la brèche fut faite à la ville ; et tous les gens de guerre s'enfuirent et sortirent de nuit hors de la ville par le chemin de la porte entre les deux murailles près du jardin du roi, tandis que les Chaldéens entouraient la ville. Ils s'en allèrent par le chemin de la plaine.
\VS{8}Mais l'armée des Chaldéens poursuivit le roi, et ils atteignirent Sédécias dans les plaines de Jéricho ; et toute son armée se dispersa loin de lui.
\VS{9}Ils prirent le roi, et le firent monter vers le roi de Babylone à Ribla, dans le pays de Hamath ; et il prononça contre lui une sentence.
\VS{10}Le roi de Babylone fit égorger les fils de Sédécias sous ses yeux ; il fit aussi égorger tous les chefs de Juda à Ribla.
\VS{11}Puis il fit crever les yeux à Sédécias, et le fit lier avec des chaînes d'airain ; le roi de Babylone l'emmena à Babylone, et le mit en prison jusqu'au jour de sa mort.
\VS{12}Le dixième jour du cinquième mois, c'était la dix-neuvième année du règne de Nebucadnetsar, roi de Babylone, Nebuzaradan, chef des gardes, qui se tenait devant le roi de Babylone, entra dans Jérusalem.
\VS{13}Il brûla la maison de Yahweh, la maison du roi, et toutes les maisons de Jérusalem ; il brûla toutes les maisons des personnes considérées.
\VS{14}Toute l'armée des Chaldéens, qui était avec le chef des gardes, renversa toutes les murailles qui entouraient Jérusalem.
\VS{15}Nebuzaradan, chef des gardes, transporta à Babylone une partie des plus pauvres du peuple, le reste du peuple qui était demeuré dans la ville, ceux qui s'étaient rendus au roi de Babylone, et le reste de la multitude.
\VS{16}Toutefois, Nebuzaradan, chef des gardes, laissa quelques-uns des plus pauvres du pays pour être vignerons et laboureur.
\VS{17}Les Chaldéens brisèrent les colonnes d'airain qui étaient dans la maison de Yahweh, les bases, la mer d'airain qui était dans la maison de Yahweh, et ils emmenèrent tout l'airain à Babylone.
\VS{18}Ils prirent les cendriers, les pelles, les couteaux, les coupes, les tasses, et tous les ustensiles d'airain avec lesquels on faisait le service.
\VS{19}Le chef des gardes prit aussi les coupes, les encensoirs, les cendriers, les chandeliers, les tasses et les calices, ce qui était d'or et ce qui était d'argent.
\VS{20}Les deux colonnes, la mer, et les douze bœufs d'airain qui servaient de base, et que le roi Salomon avait faits pour la maison de Yahweh, tous ces ustensiles d'airain ne pouvaient être pesés.
\VS{21}La hauteur de l'une des colonnes était de dix-huit coudées, et un cordon de douze coudées l'entourait ; elle était épaisse de quatre doigts, et creuse;
\VS{22}il y avait par-dessus un chapiteau d'airain, et la hauteur d'un des chapiteaux était de cinq coudées ; il y avait aussi un treillis et des grenades tout autour du chapiteau, le tout d'airain ; il en était de même pour la seconde colonne avec des grenades\FTNT{1 R. 7:15-20.}.
\VS{23}Il y avait quatre-vingt-seize grenades de chaque côté, et les grenades qui étaient autour du treillis étaient au nombre de cent.
\VS{24}Le chef des gardes prit Seraja, qui était le premier sacrificateur, Sophonie, qui était le second sacrificateur, et les trois gardiens du seuil.
\VS{25}Il prit de la ville un eunuque qui avait sous son commandement des hommes de guerre, sept hommes de ceux qui voyaient la face du roi et qui furent trouvés dans la ville, le secrétaire du chef de l'armée qui enrôlait le peuple du pays, et soixante hommes du peuple du pays, qui se trouvèrent dans la ville.
\VS{26}Nebuzaradan, chef des gardes, les prit et les emmena vers le roi de Babylone à Ribla.
\VS{27}Le roi de Babylone les frappa et les fit mourir à Ribla, dans le pays de Hamath. Ainsi Juda fut transporté hors de son pays.
\VS{28}Et c'est ici le peuple que Nebucadnetsar emmena en captivité : La septième année, trois mille vingt-trois juifs ;
\VS{29}la dix-huitième année de Nebucadnetsar, il emmena de Jérusalem huit cent trente-deux personnes ;
\VS{30}La vingt-troisième année de Nebucadnetsar, Nebuzaradan, chef des gardes, transporta en exil sept cent quarante-cinq personnes des Juifs ; en tout quatre mille six cents personnes.
\VS{31}La trente-septième année de la captivité de Jojakin, roi de Juda, le vingt-cinquième jour du douzième mois, Evil-Merodac, roi de Babylone, dans la première année de son règne, leva la tête de Jojakin, roi de Juda, et le fit sortir de prison.
\VS{32}Il lui parla avec bonté, et mit son trône au-dessus du trône des autres rois qui étaient avec lui à Babylone.
\VS{33}Il fit changer ses vêtements de prison, il mangea du pain tous les jours de sa vie en présence du roi.
\VS{34}Le roi de Babylone lui donna continuellement des vivres pour chaque jour, jusqu'au jour de sa mort, tout le temps de sa vie.
\PPE{}
\end{multicols}

%\clearpage\ShortTitle{Ezéchiel}\BookTitle{Ezéchiel}\BFont
\noindent\hrulefill
{\footnotesize
\textit{
\bigskip
{\centering{}
\\Auteur : Ezéchiel
\\(Heb. : Yechezqe'l)
\\Signification : Dieu fortifie
\\Thème : Jugements et gloire
\\Date de rédaction : 6\up{ème} siècle av. J.-C.\\}
}
%\bigskip
\textit{
\\Déporté à Babylone alors qu'il remplissait la fonction de sacrificateur, Ezéchiel eut la particularité d'exercer le ministère prophétique hors de la terre d'Israël. Sa mission était en partie d'affermir la foi des déportés par la promesse du jugement de leurs ennemis et du rétablissement de la nation. Il leur rappela aussi que les péchés de leurs pères étaient la raison de leur captivité et que l'occasion leur était donnée de réformer leurs voies.
%\bigskip
\\Ezéchiel reçut aussi des oracles concernant ceux qui étaient restés à Jérusalem et dont la condition n'était guère meilleure que celle des exilés et pour lesquels le pire était à venir. Ses prophéties s'exprimaient en songes et en visions, c'est donc ainsi qu'il vit la gloire de Yahweh quittant le temple de Jérusalem. Il reçut plusieurs prophéties sur les derniers temps, notamment la promesse d'un cœur nouveau en vue de la conversion, le retour de la gloire de Dieu lors du règne millénaire et aussi le rétablissement total d'Israël.\bigskip
}
}
\par\nobreak\noindent\hrulefill
\begin{multicols}{2}
\Chap{1}
\TextTitle{Vision de la gloire de Yahweh}
\VerseOne{}Or il arriva en la trentième année, au cinquième jour du quatrième mois, comme j'étais parmi ceux qui avaient été transportés sur le fleuve de Kebar, que les cieux furent ouverts, et je vis des visions de Dieu.
\VS{2}Au cinquième jour du mois de cette année, qui fut la cinquième après que le roi Jojakin\FTNT{2 R. 24:12-16.}ait été mené en captivité,
\VS{3}la parole de Yahweh fut adressée expressément à Ezéchiel, le sacrificateur, fils de Buzi, dans le pays des Chaldéens\FTNT{Ezéchiel était en exil à Babylone.}, sur le fleuve de Kebar, et la main de Yahweh fut là sur lui.
\VS{4}Je regardai donc, et voici, un vent impétueux vint du nord, une grosse nuée, et un feu qui prenait de tous côtés. Il y avait autour de la nuée une splendeur, et au milieu de la nuée paraissait comme de l'airain poli, lorsqu'il sort du milieu du feu.
\VS{5}Et du milieu aussi paraissait une ressemblance de quatre animaux\FTNT{Les quatre animaux représentent quatre aspects de Jésus. La face de l'homme correspond à l'humanité du Seigneur mise en exergue dans l'évangile de Luc. La face de lion symbolise la royauté de Christ, mise en évidence dans l'évangile de Matthieu. La face de bœuf fait écho à l'évangile de Marc où le Seigneur y est présenté comme serviteur. La face d'aigle symbolise la divinité du Messie mise en évidence dans l'évangile de Jean. Le Seigneur y est présenté comme le Fils de Dieu et le Dieu véritable.} et voici leur forme: Ils avaient une ressemblance humaine.
\VS{6}Et chacun d'eux avait quatre faces, et chacun avait quatre ailes.
\VS{7}Et leurs pieds étaient des pieds droits, et la plante de leurs pieds était comme la plante d'un pied de veau, ils étincelaient comme la couleur d'un airain poli.
\VS{8}Et il y avait des mains d'homme sous leurs ailes à leurs quatre côtés ; et tous quatre avaient leurs faces et leurs ailes.
\VS{9}Leurs ailes étaient jointes l'une à l'autre ; ils ne se tournaient point quand ils marchaient, mais chacun marchait droit devant soi.
\VS{10}Leurs faces ressemblaient à la face d'un homme, à la face de lion à la main droite, à la face de bœuf à la gauche des quatre, et à la face d'aigle à tous les quatre.
\VS{11}Leurs faces et leurs ailes étaient divisées par le haut ; chacun avait des ailes qui se joignaient l'une à l'autre, et deux couvraient leurs corps.
\VS{12}Chacun marchait droit devant soi ; ils allaient partout où l'Esprit les poussait à aller, et ils ne se tournaient point lorsqu'ils marchaient.
\VS{13}Et quant à l'aspect des animaux, leur regard était comme des charbons de feu ardent, comme des torches ; le feu courait parmi les animaux ; et le feu avait une splendeur, et de ce feu sortait un éclair.
\VS{14}Et les animaux couraient et revenaient selon que l'éclair paraissait.
\VS{15}Je regardais les animaux, et voici, une roue apparut sur la terre auprès des animaux, devant leurs quatre faces.
\VS{16}Et l'aspect et la forme des roues étaient comme la couleur d'un chrysolithe, et toutes les quatre avaient un même aspect ; leur aspect et leur structure étaient comme si chaque roue avait été au milieu d'une autre roue.
\VS{17}En marchant, elles allaient de leurs quatre côtés, et elles ne se tournaient point quand elles marchaient.
\VS{18}Elles avaient des jantes, elles étaient si hautes qu'elles faisaient peur, et leurs jantes étaient pleines d'yeux tout autour des quatre roues.
\VS{19}Quand ils marchaient, elles marchaient auprès d'eux ; et quand ils s'élevaient au-dessus de la terre, elles aussi s'élevaient.
\VS{20}Ils allaient partout où l'Esprit les poussait à aller ; l'Esprit tendait-il là, ils y allaient, et les roues s'élevaient avec eux, car l'esprit des animaux était dans les roues.
\VS{21}Quand ils marchaient, elles marchaient ; et quand ils s'arrêtaient, elles s'arrêtaient ; et quand ils s'élevaient au-dessus de la terre, les roues aussi s'élevaient avec eux, car l'esprit des animaux était dans les roues.
\VS{22}L'aspect de ce qui était au-dessus des têtes des animaux, était une étendue semblable à un cristal étincelant et terrible à voir, laquelle s'étendait au-dessus de leurs têtes.
\VS{23}Sous l'étendue, leurs ailes se tenaient droites l'une contre l'autre ; ils avaient chacun deux ailes dont ils se couvraient, chacun, dis-je, en avait deux qui couvraient leurs corps.
\VS{24}Puis j'entendis le bruit que faisaient leurs ailes quand ils marchaient, semblable au bruit des grandes eaux, et au bruit du Tout-Puissant, un bruit éclatant comme le bruit d'une armée ; quand ils s'arrêtaient, ils baissaient leurs ailes.
\VS{25}Et lorsqu'ils s'arrêtaient et laissaient tomber leurs ailes, il se faisait un bruit au-dessus de l'étendue qui était sur leurs têtes.
\VS{26}Et au-dessus de cette étendue, qui était sur leurs têtes, il y avait quelque chose de semblable à une pierre de saphir, en forme de trône ; et sur cette forme de trône apparaissait comme une figure d'homme\FTNT{Il est question ici de la manifestation du Messie} placée dessus en hauteur.
\VS{27}Je vis encore comme de l'airain poli semblable à un feu, au dedans duquel était cet homme, et qui l'environnait ; depuis la forme de ses reins jusqu'en haut et depuis la forme de ses reins jusqu'en bas, je vis comme du feu, et il y avait une lumière éclatante autour de lui.
\VS{28}Et la splendeur qui se voyait autour de lui, était comme l'arc qui se fait dans la nuée en un jour de pluie. C'est là la vision de la représentation de la gloire de Yahweh. A sa vue je tombai sur ma face, et j'entendis une voix qui parlait.
\Chap{2}
\TextTitle{Mandat d'Ezechiel}
\VerseOne{}Il me dit : Fils de l'homme, tiens-toi sur tes pieds, et je parlerai avec toi.
\VS{2}Alors l'Esprit entra en moi, après qu'il m'eut parlé, et il me releva sur mes pieds, et j'entendis celui qui me parlait.
\VS{3}Il me dit : Fils de l'homme, je t'envoie vers les fils d'Israël, vers des nations rebelles qui se sont rebellées contre moi. Eux et leurs pères ont péché contre moi jusqu'à ce jour même\FTNT{Jé. 3:25.}.
\VS{4}Ce sont des enfants à la face dure et au cœur obstiné, vers lesquels je t'envoie vers eux ; c'est pourquoi tu leur diras que le Seigneur Yahweh a ainsi parlé.
\VS{5}Et soit qu'ils écoutent, ou qu'ils n'en fassent rien, car ils sont une maison rebelle ; ils sauront pourtant qu'il y aura eu un prophète parmi eux\FTNT{Es. 6:9-10.}.
\VS{6}Mais toi, fils de l'homme, ne les crains point, et ne crains point leurs paroles ; quoique des gens rebelles et dont les langues sont perçantes comme des épines soient avec toi, et que tu habites parmi des scorpions ; ne crains point leurs paroles, et ne t'effraie point à cause d'eux, quoiqu'ils soient une maison rebelle\FTNT{Jé. 1:8 ; 1 Pi. 3:14.}.
\VS{7}Tu leur prononceras mes paroles, qu'ils écoutent ou qu'ils n'en fassent rien, car ils ne sont que rébellion.
\VS{8}Mais toi, fils de l'homme, écoute ce que je te dis, et ne sois point rebelle comme cette maison rebelle ; ouvre ta bouche et mange ce que je vais te donner\FTNT{Ap. 10:9 ; Jé. 15:16.}.
\VS{9}Alors je regardai, et voici, une main fut envoyée vers moi, et voici, elle avait un livre en rouleau.
\VS{10}Et elle l'ouvrit devant moi, et voici, il était écrit dedans et dehors ; des lamentations, des soupirs, et des gémissements y étaient écrits.
\Chap{3}
\TextTitle{Yahweh établit Ezéchiel comme sentinelle}
\VerseOne{}Puis il me dit : Fils de l'homme, mange ce que tu trouveras, mange ce rouleau, et va, parle à la maison d'Israël !
\VS{2}J'ouvris donc ma bouche, et il me fit manger ce rouleau.
\VS{3}Il me dit : Fils de l'homme, nourris ton ventre et remplis tes entrailles de ce rouleau que je te donne ! Je le mangeai, et il fut doux dans ma bouche comme du miel\FTNT{Ps. 119:103.}.
\VS{4}Puis il me dit : Fils de l'homme, lève-toi et va vers la maison d'Israël, et prononce-leur mes paroles !
\VS{5}Car tu n'es point envoyé vers un peuple au langage inconnu, ou à la langue barbare ; c'est vers la maison d'Israël ;
\VS{6}ni vers plusieurs peuples ayant un langage inconnu ou une langue barbare, dont tu ne puisses comprendre les paroles. Si je t'envoyais vers eux, ils t'écouteraient.
\VS{7}Mais la maison d'Israël ne voudra pas t'écouter, parce qu'ils ne veulent point m'écouter ; car toute la maison d'Israël a le front dur et le cœur obstiné.
\VS{8}Voici, j'endurcirai ta face contre leurs faces, et j'endurcirai ton front contre leurs fronts\FTNT{Jé. 1:18 ; Mi. 3:8.}.
\VS{9}Et j'ai rendu ton front semblable à un diamant, plus dur que le roc. Ne les crains donc point, et ne t'effraie point à cause d'eux, quoiqu'ils soient une maison rebelle.
\VS{10}Puis il me dit : Fils de l'homme, reçois dans ton cœur et écoute de tes oreilles toutes les paroles que je te dirai.
\VS{11}Lève-toi donc, va vers ceux qui ont été emmenés captifs, vers les enfants de ton peuple, parle-leur et dis-leur que le Seigneur Yahweh a ainsi parlé ; soit qu'ils écoutent ou qu'ils n'en fassent rien.
\VS{12}Puis l'Esprit m'enleva, et j'entendis derrière moi le bruit d'un grand tremblement, disant : Bénie soit la gloire de Yahweh du lieu de sa demeure !
\VS{13}Et j'entendis le bruit des ailes des animaux, qui s'entre-touchaient les unes les autres, et le bruit des roues auprès d'eux, et le bruit d'un grand tremblement.
\VS{14}L'Esprit donc m'enleva, et me prit, et j'allai, l'esprit rempli d'amertume et de colère, mais la main de Yahweh me fortifia.
\VS{15}Je vins donc vers ceux qui avaient été transportés à Thel-Abib, vers ceux qui demeuraient auprès du fleuve de Kebar ; et je me tins là où ils se tenaient, même je me tins là parmi eux sept jours, tout étonné.
\VS{16}Et au bout de sept jours, la parole de Yahweh me fut adressée, en disant :
\VS{17}Fils de l'homme, je t'établis pour être sentinelle sur la maison d'Israël ; tu écouteras donc la parole de ma bouche, et tu les avertiras de ma part\FTNT{Es. 52:8 ; Es. 62:6 ; Jé. 6:17.}.
\VS{18}Quand je dirai au méchant : Tu mourras, tu mourras ! Si tu ne l'avertis pas, et si tu ne parles pas pour l'avertir de se détourner de ses mauvaises voies, afin de lui sauver la vie ; ce méchant-là mourra dans son iniquité, mais je redemanderai son sang de ta main.
\VS{19}Et si tu avertis le méchant, et qu'il ne se détourne pas de sa méchanceté ni de ses mauvaises voies, il mourra dans son iniquité, mais toi, tu sauveras ton âme\FTNT{Ez. 18:23-24 ; Ez. 33:6.}.
\VS{20}Pareillement, si le juste se détourne de sa justice et commet l'iniquité, lorsque j'aurai mis quelque obstacle devant lui, il mourra ; parce que tu ne l'auras point averti, il mourra dans son péché, et il ne sera point fait mention de ses justices qu'il aura faites ; mais je te redemanderai son sang de ta main.
\VS{21}Et si tu avertis le juste de ne point pécher, et qu'il ne pèche point, il vivra, il vivra parce qu'il aura été averti, et toi pareillement tu sauveras ton âme.
\VS{22}Et la main de Yahweh fut sur moi, et il me dit : Lève-toi, et sors vers la vallée, et là je te parlerai.
\VS{23}Je me levai donc, et sortis dans la vallée ; voici, la gloire de Yahweh se tenait là, telle que je l'avais vue près du fleuve de Kebar, et je tombai sur ma face.
\VS{24}Alors l'Esprit entra en moi et me releva sur mes pieds ; il me parla et me dit : Entre, et enferme-toi dans ta maison.
\VS{25}Fils de l'homme, voici, on mettra des cordes sur toi, on te liera, et tu ne sortiras point pour aller parmi eux.
\VS{26} Et j'attacherai ta langue à ton palais, tu seras muet, et tu ne les reprendras point ; parce qu'ils sont une maison  rebelle\FTNT{Jn. 1:20-22.}.
\VS{27}Mais quand je te parlerai, j'ouvrirai ta bouche, et tu leur diras : Ainsi parle le Seigneur Yahweh : Que celui qui écoute, écoute ; et que celui qui n'écoute pas, n'écoute pas ; car ils sont une maison rebelle.
\Chap{4}
\TextTitle{Signes annonciateurs du jugement de Jérusalem : 
\\La brique, la plaque de fer et les cordes}
\VerseOne{}Toi, fils de l'homme, prends une brique et place-la devant toi, et traces-y la ville de Jérusalem.
\VS{2}Puis tu mettras le siège contre elle, tu bâtiras contre elle des retranchements, tu élèveras contre elle des terrasses, tu mettras des camps contre elle, et tu mettras autour d'elle des béliers pour la battre\FTNT{2 R. 25:1.}.
\VS{3}Tu prendras aussi une plaque de fer, et tu la mettras comme un mur de fer entre toi et la ville ; tu dresseras ta face contre elle, elle sera assiégée, et tu l'assiégeras ; ce sera un signe pour la maison d'Israël.
\VS{4}Après, tu dormiras sur ton côté gauche, mets-y l'iniquité de la maison d'Israël, et tu porteras leur iniquité autant de jours que tu seras couché sur ce côté.
\VS{5}Et je t'ai assigné un nombre de jours égal à celui des années de leur iniquité : Trois cent quatre-vingt-dix jours ; ainsi tu porteras l'iniquité de la maison d'Israël.
\VS{6}Et quand tu auras accompli ces jours-là, tu dormiras une seconde fois sur ton côté droit, et tu porteras l'iniquité de la maison de Juda pendant quarante jours ; un jour pour chaque année, car je t'ai assigné un jour pour chaque année.
\VS{7}Tu tourneras ta face et ton bras nu vers Jérusalem assiégée, et tu prophétiseras contre elle.
\VS{8}Et voici, j'ai mis sur toi des cordes, afin que tu ne puisses pas te tourner d'un côté sur l'autre, jusqu'à ce que tu aies accompli les jours de ton siège.
\TextTitle{Le pain impur}
\VS{9}Tu prendras aussi du froment, de l'orge, des fèves, des lentilles, du millet, et de l'épeautre ; tu les mettras dans un vase, et tu en feras du pain autant de jours que tu seras couché sur ton côté ; tu en mangeras pendant trois cent quatre-vingt-dix jours.
\VS{10}La viande que tu mangeras sera du poids de vingt sicles par jour ; et tu la mangeras de temps à autre.
\VS{11}Et tu boiras de l'eau par mesure; savoir la sixième de hin ; tu la boiras de temps à autre.
\VS{12}Et tu mangeras aussi des gâteaux d'orge, que tu feras cuire avec des excréments humains en leur présence.
\VS{13}Puis Yahweh dit : Les fils d'Israël mangeront ainsi leur pain souillé parmi les nations vers lesquelles je les chasserai\FTNT{Os. 9:3 ; Da. 1:8.}.
\VS{14}Et je dis : Ah ! Seigneur Yahweh, voici, mon âme n'a point été souillée, et je n'ai mangé d'aucune bête morte d'elle-même, ou déchirée par les bêtes sauvages, depuis ma jeunesse jusqu'à présent, et aucune chair impure n'est entrée dans ma bouche\FTNT{Lé 17:15 ; De. 14:3 ; Ac. 10:14.}.
\VS{15}Il me répondit : Voici, je te donne des excréments de bœuf au lieu d'excréments humains, et tu y feras cuire ton pain.
\VS{16}Puis il me dit : Fils de l'homme, voici, je m'en vais rompre le bâton du pain dans Jérusalem ; et ils mangeront leur pain au poids et avec chagrin ; et ils boiront de l'eau par mesure et avec horreur\FTNT{Lé. 26:26 ; Es. 3:1 ; Ps. 105:16 ; La. 5:4.}.
\VS{17}Parce que le pain et l'eau leur manqueront, ils seront épouvantés, se regardant les uns les autres, et ils se décomposeront à cause de leur iniquité.
\Chap{5}
\TextTitle{Les cheveux coupés et divisés en trois}
\VerseOne{}Et toi, fils de l'homme, prends un couteau tranchant, prends un rasoir de barbier, et fais-le passer sur ta tête et sur ta barbe. Puis, tu prendras une balance à peser, et tu partageras ce que tu auras rasé\FTNT{Lé. 21:5 ; Ez. 44:20.}.
\VS{2}Brûles-en un tiers dans le feu, au milieu de la ville, lorsque les jours du siège seront accomplis ; prends-en un tiers, et frappe-le avec l'épée tout autour de la ville ; disperses-en un tiers au vent, car je tirerai l'épée derrière eux\FTNT{Lé. 26:25 ; La. 1:20.}.
\VS{3}Tu en prendras une petite quantité que tu serreras aux pans de ton manteau.
\VS{4}De ceux-là, tu en prendras encore, les jetteras au milieu du feu, et les brûleras au feu. De là sortira un feu contre toute la maison d'Israël.
\VS{5}Ainsi parle le Seigneur Yahweh : C'est là cette Jérusalem que j'avais placée au milieu des nations et des pays qui sont autour d'elle.
\VS{6}Elle a changé mes ordonnances et s'est rendue plus coupable que les nations et les pays d'alentour ; car ils ont rejeté mes ordonnances, et n'ont point marché dans mes ordonnances.
\VS{7}C'est pourquoi ainsi parle le Seigneur Yahweh : Parce que vous avez multiplié vos méchancetés plus que les nations qui vous entourent, et que vous n'avez point suivi mes ordonnances et observé mes lois, et que vous n'avez pas agi selon les ordonnances des nations qui vous entourent ;
\VS{8}à cause de cela, ainsi parle le Seigneur Yahweh : Voici, j'en veux à toi et j'exécuterai au milieu de toi mes jugements, sous les yeux des nations.
\VS{9}Je te ferai, à cause de toutes tes abominations, des choses que je n'ai jamais faites, et ce que je ne ferai jamais\FTNT{Da. 9:12 ; Mt. 24:21.}.
\VS{10}Des pères mangeront leurs fils au milieu de toi, et des fils mangeront leurs pères ; j'exécuterai mes jugements sur toi, et je disperserai à tous les vents tout ce qui restera de toi\FTNT{Lé. 26:33 ; De. 28:64 ; Jé. 9:16 ; Za. 2:6.}.
\VS{11}Je suis vivant, dit le Seigneur Yahweh, parce que tu as souillé mon lieu saint par toutes tes infamies, et par toutes tes abominations, moi-même je te raserai, et mon oeil ne t'épargnera point, et je n'aurai point de compassion\FTNT{Jé. 7:9-11.}.
\VS{12}Un tiers d'entre vous mourra de la peste, et sera consumé par la famine au milieu de toi ; un tiers tombera par l'épée autour de toi ; et je disperserai à tous les vents l'autre tiers, je tirerai l'épée derrière eux.
\VS{13}Car ma colère sera portée à son comble, je ferai reposer ma fureur sur eux, et je me donnerai satisfaction ; ils sauront que moi, Yahweh, j'ai parlé dans ma jalousie, quand j'aurai consumé ma fureur sur eux.
\VS{14}Je ferai de toi un désert, un sujet d'opprobre parmi les nations qui sont autour de toi, aux yeux de tous les passants\FTNT{Lé. 26:31-32 ; Né. 2:17.}.
\VS{15}Tu seras en opprobre, en ignominie, un exemple et un sujet d'étonnement pour les nations qui t'entourent, quand j'aurai exécuté mes jugements sur toi, avec colère, avec fureur, et par des châtiments pleins de fureur ; moi, Yahweh, j'ai parlé\FTNT{De. 28:37 ; 1 R. 9:7 ; Ps. 79:4 ; Jé. 24:9 ; Es. 26:9.}.
\VS{16}Quand je lancerai sur eux les flèches douloureuses de la famine, qui seront mortelles, quand je les enverrai pour vous détruire, j'ajouterai la famine sur vous, et romprai pour vous le bâton du pain\FTNT{De. 32:24.}.
\VS{17}Je vous enverrai la famine, et des bêtes féroces, qui te priveront d'enfants ; la peste et le sang passeront au milieu de toi, et je ferai venir l'épée sur toi. Moi,Yahweh, j'ai parlé.
\Chap{6}
\TextTitle{Grâce de Yahweh pour quelques réchappés d'Israël}
\VerseOne{}La parole de Yahweh me fut encore adressée, en disant :
\VS{2}Fils de l'homme, tourne ta face contre les montagnes d'Israël, et prophétise contre elles !
\VS{3}Et dis : Montagnes d'Israël, écoutez la parole du Seigneur Yahweh. Ainsi parle le Seigneur Yahweh aux montagnes et aux collines, aux cours des rivières, et aux vallées : Me voici, je vais faire venir l'épée sur vous, et je détruirai vos hauts lieux\FTNT{Lé. 26:30.}.
\VS{4}Vos autels seront dévastés, vos autels d'encens seront brisés, et je ferai tomber vos morts devant vos idoles.
\VS{5}Car je mettrai les cadavres des fils d'Israël devant leurs idoles, et je disperserai vos os autour de vos autels\FTNT{2 R. 23:14-20.}.
\VS{6}Les villes seront désertes, là où sont vos demeures, et les hauts lieux seront dévastés, vos autels seront délaissés et abandonnés, et vos idoles seront brisées et ne seront plus ; vos autels d'encens abattus, et vos ouvrages seront nettoyés.
\VS{7}Les tués tomberont parmi vous ; et vous saurez que je suis Yahweh.
\VS{8}Mais je laisserai quelques restes d'entre vous, afin que vous ayez quelques réchappés de l'épée parmi les nations, quand vous serez dispersés parmi les pays.
\VS{9}Vos réchappés se souviendront de moi\FTNT{Jé. 51:50.} parmi les nations où ils seront captifs, parce que j'aurai brisé leur cœur adonné à la fornication, qui s'est détourné de moi, et à cause de leurs yeux qui se sont livrés à la prostitution après leurs idoles ; ils se prendront eux-mêmes en dégoût, à cause du mal qu'ils ont commis, à cause de leurs abominations.
\VS{10}Ils sauront que je suis Yahweh, que ce n'est point en vain que je les ai menacés.
\TextTitle{Sentence envers les idolâtres}
\VS{11}Ainsi parle le Seigneur Yahweh : Frappe de ta main et bats de ton pied, et dis : Hélas ! A cause de toutes les abominations, des maux de la maison d'Israël ; car ils tomberont par l'épée, par la famine, et par la peste.
\VS{12}Celui qui sera loin mourra de la peste, et celui qui sera près tombera par l'épée ; et celui qui restera et sera assiégé, mourra par la famine, ainsi je consumerai ma fureur sur eux\FTNT{Am. 4:10.}.
\VS{13}Vous saurez que je suis Yahweh quand les blessés à morts seront au milieu de leurs idoles, autour de leurs autels, sur toute colline élevée, sur tous les sommets des montagnes, sous tout arbre vert, et sous tout chêne touffu, là où ils offraient des parfums de bonne odeur à toutes leurs idoles\FTNT{Os. 4:13.}.
\VS{14}J'étendrai donc ma main sur eux, et je rendrai leur pays désolé et désert dans toutes leurs demeures, plus que le désert qui est vers Dibla. Et ils sauront que je suis Yahweh.
\Chap{7}
\TextTitle{Attaque babylonienne imminente}
\VerseOne{}Puis la parole de Yahweh me fut adressée, en disant :
\VS{2}Et toi, fils de l'homme, écoute : Ainsi parle le Seigneur Yahweh à la terre d'Israël : La fin, la fin vient sur les quatre coins de la terre !
\VS{3}Maintenant la fin vient sur toi, j'enverrai sur toi ma colère, et je te jugerai selon ta voie, et je mettrai sur toi toutes tes abominations\FTNT{Ro. 2:6.}.
\VS{4}Et mon œil ne t'épargnera point, et je n'aurai point de compassion ; mais je te chargerai de tes voies, et tes abominations seront au milieu de toi ; et vous saurez que je suis Yahweh.
\VS{5}Ainsi parle le Seigneur Yahweh : Voici un mal, un seul mal qui vient !
\VS{6}La fin vient, la fin vient, elle se réveille contre toi ; voici, le mal vient !
\VS{7}Ton tour arrive, habitant du pays ! Le temps vient, le jour est près de toi, il ne sera que frayeur, et non pas une invitation des montagnes\FTNT{So. 1:14-15.} à s'entre-réjouir.
\VS{8}Maintenant, je répandrai bientôt ma fureur sur toi, et je consumerai ma colère sur toi ; je te jugerai selon ta voie, je mettrai sur toi toutes tes abominations.
\VS{9}Mon œil ne t'épargnera point, et je n'aurai point de compassion, je te punirai selon ta voie, et tes abominations seront au milieu de toi ; et vous saurez que je suis Yahweh qui frappe.
\VS{10}Voici le jour, voici il vient, le matin paraît, la verge fleurit, l'orgueil bourgeonne.
\VS{11}La violence s'élève pour servir de verge à la méchanceté ; il ne restera rien d'eux, ni de leur multitude, ni de leur tumulte, et on ne se lamentera point sur eux.
\VS{12}Le temps vient, le jour est tout proche : Que celui donc qui achète ne se réjouisse point, et que celui qui vend ne se lamente point ; car il y a une ardente colère sur toute leur multitude.
\VS{13}Car le vendeur ne recouvrera pas ce qu'il a vendu, serait-il encore parmi les vivants ; car la vision touchant toute leur multitude ne sera point révoquée ; et à cause de son iniquité, nul ne conservera sa vie.
\VS{14}On sonne de la trompette, tout est prêt, mais il n'y a personne pour aller au combat, parce que l'ardeur de ma colère est sur toute leur multitude.
\VS{15}L'épée est au-dehors, la peste et la famine au-dedans ! Celui qui est aux champs mourra par l'épée ; et celui qui est dans la ville, la famine et la peste le dévoreront.
\VS{16}Les réchappés s'enfuiront et seront sur les montagnes comme les pigeons des vallées, tous gémissant, chacun sur son iniquité.
\VS{17}Toutes les mains deviendront lâches, et tous les genoux se fondront en eau\FTNT{Es. 13:7 ; Jé. 6:24.}.
\VS{18}Ils se ceindront de sacs, et le tremblement les couvrira, la confusion sera sur tous leurs visages, et leurs têtes deviendront chauves\FTNT{Es. 3:24 ; Jé. 48:37 ; Am. 8:10.}.
\VS{19}Ils jetteront leur argent par les rues, et leur or s'en ira au loin ; leur argent et leur or ne pourront pas les délivrer au jour de la grande colère de Yahweh\FTNT{Pr. 11:4 ; So. 1:18.} ; ils ne rassasieront point leurs âmes, et ne rempliront point leurs entrailles, parce que leur iniquité aura été leur ruine.
\TextTitle{Violation du temple}
\VS{20}Ils étaient fiers de leur magnifique parure ; mais ils y ont placé des images de leurs abominations et de leurs infamies, c'est pourquoi je la rendrai pour eux un objet d'horreur.
\VS{21}Je l'ai livrée au pillage dans la main des étrangers, et en proie aux méchants de la terre qui la profaneront\FTNT{Jé. 20:5.}.
\VS{22}Je détournerai aussi ma face d'eux, et on violera mon lieu secret, et des furieux entreront et le profaneront.
\VS{23}Fais une chaîne ! Car le pays est plein de crimes, de meurtre, et la ville est pleine de violence.
\VS{24}C'est pourquoi je ferai venir les plus méchants des nations, qui possèderont leurs maisons, et je ferai cesser l'orgueil des puissants, et leurs saints lieux seront profanés.
\VS{25}La destruction vient, et ils chercheront la paix, mais il n'y en aura point.
\VS{26}Il viendra malheur sur malheur, et il y aura rumeur sur rumeur ; ils demanderont la vision aux prophètes\FTNT{La. 2:9.} ; la loi périra chez le sacrificateur, et le conseil chez les anciens.
\VS{27}Le roi se lamentera, les princes se vêtiront de désolation, et les mains du peuple du pays tomberont de frayeur. Je les traiterai selon leur voie, je les jugerai comme ils le méritent et ils sauront que je suis Yahweh.
\Chap{8}
\TextTitle{Visions divines}
\VerseOne{}Puis il arriva dans la sixième année, au cinquième jour du sixième mois, comme j'étais assis dans ma maison, et que les anciens de Juda étaient assis devant moi, que la main du Seigneur Yahweh tomba là sur moi.
\VS{2}Je regardai, et voici c'était une figure ayant l'aspect d'un feu qui frappe les regards ; depuis ses reins jusqu'en bas c'était du feu, et depuis ses reins jusqu'en haut, c'était d'un aspect brillant comme de l'airain poli.
\VS{3}Il étendit une forme de main et me prit par les cheveux de ma tête. L'Esprit m'enleva entre la terre et le ciel et me transporta à Jérusalem, dans des visions de Dieu, à l'entrée de la porte intérieure, du côté nord, où était posée l'idole de jalousie\FTNT{L'idole de la jalousie : Dans le temple de Jérusalem à l'époque d'Ezéchiel, l'idolâtrie s'y développait sans retenue (2 R. 21, 22 et 23). Il y avait dans ce temple les idoles d'Astarté et les autels de Baal. Le temple était souillé.} qui provoque la jalousie.
\VS{4}Voici, la gloire du Dieu d'Israël était là, telle que je l'avais vue en vision dans la vallée.
\TextTitle{Abominations dans le temple}
\VS{5}Il me dit : Fils de l'homme, lève maintenant tes yeux vers le chemin qui tend vers le nord ! J'élevai mes yeux vers le chemin qui tend vers le nord, et voici du côté nord, à la porte de l'autel, était cette idole de jalousie, à l'entrée.
\VS{6}Il me dit : Fils de l'homme, ne vois-tu pas ce qu'ils font, les grandes abominations que la maison d'Israël commet ici, pour que je me retire de mon lieu saint ? Mais tourne-toi encore, tu verras de grandes abominations.
\VS{7}Il me conduisit donc à l'entrée du parvis. Je regardai, et voici, il y avait un trou dans le mur.
\VS{8}Il me dit : Fils de l'homme, perce maintenant le mur ; et quand je perçai le mur, il y avait une porte.
\VS{9}Puis il me dit : Entre et regarde les méchantes abominations qu'ils commettent ici.
\VS{10}J'entrai donc et je regardai ; et voici, toutes sortes de figures de reptiles et de bêtes abominables, et toutes les idoles de la maison d'Israël étaient peintes sur le mur tout autour\FTNT{Ex. 20:4 ; De. 4:16-18 ; Ro. 1:23.}.
\VS{11}Soixante-dix hommes des anciens de la maison d'Israël, au milieu desquels était Jaazania, fils de Schaphan, se tenaient debout devant ces idoles, chacun l'encensoir à la main, d'où s'élevait une épaisse nuée d'encens.
\VS{12}Alors il me dit : Fils de l'homme, n'as-tu pas vu ce que les anciens de la maison d'Israël font dans les ténèbres, chacun dans sa chambre pleine de figures ? Car ils disent : Yahweh ne nous voit point, Yahweh a abandonné le pays\FTNT{Es. 29:15.}.
\VS{13}Puis il me dit : Tourne-toi encore, et tu verras les grandes abominations qu'ils commettent.
\VS{14}Il me conduisit donc à l'entrée de la porte de la maison de Yahweh qui est vers le nord. Et voici, il y avait là des femmes assises qui pleuraient Thammuz\FTNT{Thammuz ou Adonis.}.
\VS{15}Il me dit : Fils de l'homme, n'as-tu pas vu ? Tourne-toi encore, et tu verras des abominations plus grandes que celles-ci.
\VS{16}Il me fit donc entrer dans le parvis intérieur de la maison de Yahweh. Et voici, à l'entrée du temple de Yahweh, entre le portique et l'autel, environ vingt-cinq hommes avaient le dos tourné contre le temple de Yahweh, leurs visages tournés vers l'orient ; et ils se prosternaient vers l'orient, devant le soleil\FTNT{De. 4:19.}.
\VS{17}Alors il me dit : Fils de l'homme, n'as-tu pas vu ? Est-ce une chose légère à la maison de Juda de commettre ces abominations qu'ils commettent ici ? Car ils ont rempli le pays de violence, et ils se sont ainsi tournés pour m'irriter ; mais voici ils approchent le rameau de leurs nez.
\VS{18}Et moi, j'agirai dans ma fureur ; mon œil ne les épargnera point, et je n'aurai point de compassion ; quand ils crieront à haute voix à mes oreilles, je ne les exaucerai point\FTNT{Pr. 1:28 ; Es. 1:15 ; Jé. 11:11 ; Mi. 3:4 ; Za. 7:13.}.
\Chap{9}
\TextTitle{Marque de Yahweh sur les justes ; extermination des impies}
\VerseOne{}Puis il cria d'une voix forte à mes oreilles : Faites approcher ceux qui châtient la ville, chacun avec son instrument de destruction à la main !
\VS{2}Et voici, six hommes venaient par le chemin de la haute porte qui regarde vers le nord, et chacun avait dans sa main son instrument de destruction. Il y avait au milieu d'eux un homme vêtu de lin, qui avait une écritoire sur ses reins ; ils entrèrent et se tinrent près de l'autel d'airain.
\VS{3}Alors la gloire du Dieu d'Israël s'éleva du chérubin sur lequel elle était, et vint sur le seuil de la maison. Il cria à l'homme qui était vêtu de lin et qui avait l'écritoire sur ses reins.
\VS{4}Yahweh lui dit : Passe par le milieu de la ville, par le milieu de Jérusalem, et marque la lettre Thau sur les fronts des hommes qui gémissent et qui soupirent à cause de toutes les abominations qui s'y commettent\FTNT{Ap. 7:3 ; Ap. 9:4 ; Ap. 13:16-17 ; Ap. 20:4 ; Ex. 12:7-23.}.
\VS{5}Et s'adressant aux autres en ma présence, il dit : Passez dans la ville après lui, et frappez ; que votre oeil soit sans pitié et n'ayez point de compassion !
\VS{6}Tuez-les tous, les vieillards, les jeunes gens, les vierges, les enfants et les femmes\FTNT{2 Ch. 36:17.} ; mais n'approchez pas de ceux qui ont la lettre Thau\FTNT{La lettre Thau ou Tav est la marque, le signe, le symbole ou le sceau Divin. La lettre Tav est formée par la réunion des lettres Daleth et Nun. Ces deux lettres forment le mot « dan » qui veut dire « juge ». Selon la Bible, la marque des chrétiens est représentée par : Le Saint-Esprit, le nom de Jésus-Christ (Ep. 1:13-14 ; Ep. 4:30 ; Ap. 14:1), le nom de la Nouvelle Jérusalem (Ap. 3 : 12) et le nom du Père. Les chrétiens fidèles à Dieu sont marqués par l'Esprit de Dieu qui est notre sceau. Le Saint-Esprit est saint, la sainteté est donc la marque des chrétiens (1 Pi. 1:2). Il est aussi l'Esprit de vérité, donc la vérité est aussi la marque des chrétiens (Jn. 16:13). Il est aussi amour, l'amour étant également la marque distinctive des véritables chrétiens (Ro. 5:5).}, et commencez par mon lieu saint\FTNT{Le jugement commence par la maison de Dieu (1 Pi. 4:17-18).}. Ils commencèrent donc par les vieillards qui étaient devant la maison.
\VS{7}Il leur dit : Profanez la maison, et remplissez de morts les parvis !… Sortez !… Et ils sortirent, et ils frappèrent dans la ville.
\VS{8}Or il arriva que comme ils frappaient, je restai là, et m'étant prosterné le visage contre terre, je criai et dis : Ah ! Seigneur Yahweh ! Vas-tu donc détruire tous les restes d'Israël en répandant ta fureur sur Jérusalem ?
\VS{9}Il me dit : L'iniquité de la maison d'Israël et de Juda est excessivement grande, le pays est rempli de meurtres et la ville remplie de crimes ; car ils ont dit : Yahweh a abandonné le pays, Yahweh ne nous voit point.
\VS{10}Quant à moi, mon oeil aussi ne les épargnera point, et je n'en aurai point compassion ; je mettrai leur voie sur leur tête.
\VS{11}Et voici, l'homme vêtu de lin, qui avait une écritoire sur ses reins, rapporta ce qui avait été fait, et il dit : J'ai fait comme tu m'as ordonné.
\Chap{10}
\TextTitle{La gloire de Yahweh quitte le temple}
\VerseOne{}Je regardai, et voici, sur l'étendue qui était au-dessus de la tête des chérubins, parut comme une pierre de saphir ; on voyait au-dessus d'eux quelque chose de semblable à un trône.
\VS{2}On parla à l'homme vêtu de lin, et on lui dit : Va entre les roues, sous les chérubins, et remplis tes mains de charbons ardents que tu prendras entre les chérubins, et répands-les sur la ville\FTNT{Es. 6:6 ; Ap. 8:5.} ; il y entra devant mes yeux.
\VS{3}Les chérubins étaient à la droite de la maison quand l'homme entra ; et une nuée remplit le parvis intérieur\FTNT{1 R. 8:10-11.}.
\VS{4}Puis la gloire de Yahweh s'éleva de dessus les chérubins pour venir sur le seuil de la maison, et la maison fut remplie d'une nuée, et le parvis fut rempli de la splendeur de la gloire de Yahweh.
\VS{5}On entendit le bruit des ailes des chérubins jusqu'au parvis extérieur, pareil à la voix du Dieu Tout-Puissant lorsqu'il parle.
\VS{6}Ainsi Yahweh donna cet ordre à l'homme qui était vêtu de lin : Prends du feu d'entre les roues des chérubins ; il entra et se tint auprès des roues.
\VS{7}L'un des chérubins étendit sa main entre les chérubins, vers le feu qui était entre les chérubins ; il en prit et le mit entre les mains de l'homme vêtu de lin. Et cet homme le prit et sortit.
\VS{8}On voyait aux chérubins une forme de main d'homme sous leurs ailes.
\VS{9}Puis je regardai, et voici, il y avait quatre roues près des chérubins, une roue près de chaque chérubin ; et ces roues avaient l'aspect d'une pierre de chrysolithe.
\VS{10}A leur aspect, toutes les quatre avaient la même forme ; chaque roue paraissait être au milieu d'une autre roue.
\VS{11}Quand elles marchaient, elles allaient de leurs quatre côtés, et elles ne se tournaient point dans leur marche ; mais elles allaient dans la direction de la tête, sans se tourner dans leur marche.
\VS{12}Tout le corps des chérubins, leur dos, leurs mains, leurs ailes, étaient remplis d'yeux, aussi bien que les roues tout autour, les quatre roues\FTNT{Ap. 4:6-8.}.
\VS{13}J'entendis qu'on appela les roues tourbillon.
\VS{14}Chaque animal avait quatre faces : La première face était la face d'un chérubin ; la seconde face était la face d'un homme ; la troisième était la face d'un lion ; et la quatrième la face d'un aigle\FTNT{Ez. 1 ; Ap. 4:7.}.
\VS{15}Puis les chérubins s'élevèrent. Ce sont là les animaux que j'avais vus près du fleuve de Kebar.
\VS{16}Lorsque les chérubins marchaient, les roues aussi marchaient à côté d'eux ; et quand les chérubins élevaient leurs ailes pour s'élever de terre, les roues ne se détournaient point d'eux.
\VS{17}Lorsqu'ils s'arrêtaient, elles s'arrêtaient ; et lorsqu'ils s'élevaient, elles s'élevaient ; car l'esprit des animaux était dans les roues.
\VS{18}Puis la gloire de Yahweh se retira de dessus le seuil de la maison, et se tint au-dessus des chérubins.
\VS{19}Les chérubins élevant leurs ailes, s'élevèrent de terre sous mes yeux quand ils partirent ; les roues s'élevèrent aussi. Et chacun d'eux s'arrêta à l'entrée de la porte orientale de la maison de Yahweh ; la gloire du Dieu d'Israël était sur eux en haut.
\VS{20}C'étaient les animaux que j'avais vus sous le Dieu d'Israël près du fleuve de Kebar ; et je reconnus que c'étaient des chérubins.
\VS{21}Chacun avait quatre faces, et chacun quatre ailes, une forme de main d'homme était sous leurs ailes.
\VS{22}Quant à l'aspect de leurs faces, c'étaient les faces que j'avais vues près du fleuve de Kebar, c'était le même aspect, c'étaient eux-mêmes. Et chacun marchait droit devant soi.
\Chap{11}
\TextTitle{Sentences sur les princes infidèles}
\VerseOne{}Puis l'Esprit m'enleva et me transporta à la porte orientale de la maison de Yahweh, à celle qui regarde vers l'orient. Et il y avait vingt-cinq hommes à l'entrée de la porte, et je vis au milieu d'eux Jaazania, fils d'Azzur, et Pelathia, fils de Benaja, les princes du peuple.
\VS{2}Il me dit : Fils de l'homme, ce sont les hommes qui ont des pensées d'iniquité, et qui donnent un mauvais conseil dans cette ville\FTNT{Mi. 2:1.}.
\VS{3}Ils disent : Ce n'est pas le moment ! Bâtissons des maisons ! La ville est la chaudière et nous sommes la viande.
\VS{4}C'est pourquoi prophétise contre eux, prophétise, fils de l'homme !
\VS{5}L'Esprit de Yahweh tomba sur moi. Et il me dit : Ainsi parle Yahweh : Vous parlez de la sorte, maison d'Israël, et je connais toutes les pensées de votre esprit.
\VS{6}Vous avez multiplié les meurtres dans cette ville, et vous avez rempli ses rues de gens que vous avez tués.
\VS{7}C'est pourquoi, ainsi parle le Seigneur Yahweh : Les gens que vous avez tués, et que vous avez mis au milieu d'elle, sont la viande, et elle est la chaudière, mais je vous tirerai hors du milieu d'elle\FTNT{Mi. 3:3.}.
\VS{8}Vous avez eu peur de l'épée, mais je ferai venir l'épée sur vous, dit le Seigneur Yahweh\FTNT{Jé. 42:16.}.
\VS{9}Je vous tirerai hors de la ville, je vous livrerai entre les mains des étrangers, et j'exécuterai mes jugements contre vous.
\VS{10}Vous tomberez par l'épée ; je vous jugerai dans le pays d'Israël, et vous saurez que je suis Yahweh.
\VS{11}Elle ne sera point une chaudière pour vous, et vous ne serez point au dedans d'elle comme la viande ; je vous jugerai dans le pays d'Israël.
\VS{12}Et vous saurez que je suis Yahweh ; car vous n'avez point suivi mes ordonnances, et vous n'avez pas observé mes lois, mais vous avez agi selon les ordonnances des nations qui sont autour de vous.
\VS{13}Or il arriva comme je prophétisais, que Pelathia, fils de Benaja, mourut. Alors je me prosternai sur mon visage, je criai à haute voix, et dis : Ah ! Seigneur Yahweh ! Vas-tu consumer entièrement le reste d'Israël ?
\TextTitle{Restauration d'Israël et de ses exilés}
\VS{14}La parole de Yahweh me fut adressée, en disant :
\VS{15}Fils de l'homme, tes frères, tes frères, les hommes de ta parenté, et la maison d'Israël tout entière, à qui les habitants de Jérusalem ont dit : Eloignez-vous de Yahweh, la terre nous a été donnée en héritage.
\VS{16}C'est pourquoi dis-leur : Ainsi parle le Seigneur Yahweh : Quoique je les aie éloignés des nations, et que je les aie dispersés dans divers pays, je serai pour eux quelque temps un lieu saint\FTNT{Alors que le lieu saint ou maison terrestre était souillée, Yahweh se présente comme le Lieu Sacré pour son peuple.} dans les pays où ils sont venus.
\VS{17}C'est pourquoi dis-leur : Ainsi parle le Seigneur Yahweh : Je vous rassemblerai du milieu des peuples, et je vous recueillerai des pays auxquels vous avez été dispersés, et je vous donnerai la terre d'Israël\FTNT{Es. 11:11-16 ; Jé. 24:6 ; Ez. 28:25 ; Ez. 34:13 ; Ez. 36:24.}.
\VS{18}C'est là qu'ils iront, et ils ôteront hors d'elle toutes ses infamies et toutes ses abominations.
\VS{19}Je leur donnerai un même cœur, et je mettrai en eux un esprit nouveau ; j'ôterai de leur corps le cœur de pierre, et je leur donnerai un cœur de chair\FTNT{Il s'agit d'une allusion à la nouvelle alliance (Jé. 31:31-34 ; Hé. 8).},
\VS{20}afin qu'ils suivent mes ordonnances, et qu'ils gardent et observent mes lois ; ils seront mon peuple, et je serai leur Dieu.
\VS{21}Quant à ceux dont le cœur se plaît à leurs idoles et à leurs abominations, quant à ceux-là, je ferai tomber sur leur tête les peines que mérite leur conduite, dit le Seigneur Yahweh.
\TextTitle{La gloire de Dieu en mouvement vers le Mont des Oliviers\FTNTT{Cp. Ez. 43:1-4.}}
\VS{22}Puis les chérubins élevèrent leurs ailes, accompagnés des roues ; et la gloire du Dieu d'Israël était sur eux, en haut.
\VS{23}La gloire de Yahweh s'éleva du milieu de la ville\FTNT{Le départ de la gloire de Dieu du temple de Jérusalem marque la fin de la théocratie (règne de Dieu) en Israël. Cet événement, comparable au retrait de l'Esprit de Dieu en Ge. 6:3, fut consécutif à la décadence morale d'Israël (voir Ez. 8) qui fut désormais livré aux nations. Certains estiment que la théocratie a cessé au moment où les israélites ont demandé un roi (voir 1 S. 8). Or bien que cette demande déplut à Yahweh, il continua néanmoins à diriger Israël au travers des souverains tels que David, qu'il établissait à la tête de son peuple. Les Hébreux avaient déjà reçu un sérieux avertissement avec la destruction du temple lors de la première déportation babylonienne (2 R. 24). Cet événement, bien que traumatisant pour beaucoup, n'avait cependant pas provoqué une réelle repentance, c'est pourquoi les israélites retombèrent rapidement dans leurs travers. Ainsi, comme en témoigne Mal. 2 :17 qui rapporte les propos de certains juifs : « Où est le Dieu de la justice ? », en dépit de la reconstruction du temple sous Néhémie et Esdras, la gloire de Dieu ne s'y manifestait plus depuis longtemps. Ezéchiel ne fait donc qu'assister à la conséquence de plusieurs siècles d'infidélité des juifs à l'égard de leur Dieu.}, et s'arrêta sur la montagne qui est à l'orient de la ville.
\VS{24}Puis l'Esprit m'enleva et me transporta en Chaldée, vers ceux qui avaient été emmenés captifs, le tout en vision par l'Esprit de Dieu ; et la vision que j'avais vue disparut au-dessus de moi.
\VS{25}Alors je dis à ceux qui avaient été emmenés captifs toutes les paroles que Yahweh m'avait révélées.
\Chap{12}
\TextTitle{Fuite d'Ezéchiel, un signe pour Israël}
\VerseOne{}La parole de Yahweh me fut encore adressée en ces mots :
\VS{2}Fils de l'homme : Tu habites au milieu d'une maison rebelle, au milieu de gens qui ont des yeux pour voir, et ne voient point ; et qui ont des oreilles pour entendre, et n'entendent point ; parce qu'ils sont une maison de rebelles\FTNT{Es. 6:9 ; Es. 49:19-20 ; Jé. 5:21 ; Ac. 28:26.}.
\VS{3}Toi donc fils d’homme, fais-toi des bagages d’un homme qui s'exile et pars en exil de jour, sous leurs yeux, pars en exil, dis-je de ton lieu pour aller dans un autre lieu, sous leurs yeux. Peut-être qu'ils y prendront garde, quoi qu’ils soient une maison rebelle.
\VS{4}Tu mettras donc dehors pendant le jour tes bagages comme les bagages d'un homme qui s'exile sous leurs yeux, et le soir, tu sortiras sous leurs yeux, comme quand on sort pour s'exiler.
\VS{5}Perce-toi le mur sous leurs yeux et sors par là tes bagages.
\VS{6}Tu les porteras sur tes épaules, sous leurs yeux, et tu sortiras tes bagages pendant l'obscurité. Tu couvriras aussi ton visage, afin que tu ne voies point la terre ; car je t'ai mis pour être un signe pour la maison d'Israël.
\VS{7}Je fis donc ce qui m'avait été ordonné : Je portai dehors pendant le jour mes bagages comme des bagages d'exil ; le soir je perçai le mur avec la main et je les sortis pendant l'obscurité, je les portai sur l'épaule, sous leurs yeux.
\VS{8}Au matin, la parole de Yahweh me fut adressée en ces mots :
\VS{9}Fils de l'homme, la maison d'Israël, maison rebelles, ne t'a-t-elle pas dit : Qu'est-ce que tu fais ?
\VS{10}Dis-leur : Ainsi parle le Seigneur Yahweh : Cet ordre dont je suis chargé s'adresse au prince qui est à Jérusalem et à toute la maison d'Israël qui s'y trouve.
\VS{11}Dis : Je suis pour vous un signe ; comme j'ai fait, ainsi il leur sera fait ; ils iront en exil, en captivité.
\VS{12}Et le prince qui est parmi eux, mettra son bagage sur l'épaule et sortira ; on percera le mur pour le tirer dehors ; il couvrira son visage, afin qu'il ne voie point de ses yeux la terre\FTNT{2 R. 25:4.}.
\VS{13}J'étendrai mon rets sur lui, et il sera pris dans mes filets ; je le ferai entrer dans Babylone, au pays des Chaldéens, mais il ne la verra point, et il y mourra.
\VS{14}Je disperserai à tout vent tout ce qui est autour de lui, son secours, et tous ses corps d'armées ; et je tirerai l'épée sur eux.
\VS{15}Ils sauront que je suis Yahweh, quand je les aurai répandus parmi les nations, et que je les aurai dispersés dans divers pays.
\VS{16}Je laisserai un reste d'entre eux, quelques hommes, préservés de l'épée, de la famine, et de la peste, afin qu'ils racontent toutes leurs abominations parmi les nations où ils iront ; et ils sauront que je suis Yahweh.
\TextTitle{La captivité du peuple imminente\FTNTT{Cp. 2 R. 25:1-10.}}
\VS{17}Puis la parole de Yahweh me fut adressée en ces mots :
\VS{18}Fils de l'homme, mange ton pain dans l'agitation, et bois ton eau en tremblant et avec inquiétude.
\VS{19}Puis tu diras au peuple du pays : Ainsi parle le Seigneur Yahweh, sur les habitants de Jérusalem, à la terre d'Israël : Ils mangeront leur pain avec chagrin, et ils boiront leur eau avec frayeur, parce que son pays sera désolé, étant privé de son abondance, à cause de la violence de tous ceux qui y habitent.
\VS{20}Les villes peuplées seront désertes, et le pays ne sera que désolation ; et vous saurez que je suis Yahweh.
\VS{21}La parole de Yahweh me fut encore adressée en ces mots :
\VS{22}Fils de l'homme, que signifient ces discours moqueurs que vous tenez sur la terre d'Israël, en disant : Les jours seront prolongés, et toute vision périra\FTNT{Es. 5:19 ; Am. 6:3 ; 2 Pi. 3:3.} ?
\VS{23}C'est pourquoi dis-leur : Ainsi parle le Seigneur Yahweh : Je ferai cesser ce proverbe, et on ne s'en servira plus comme proverbe en Israël ; et dis-leur : Les jours approchent, et toutes les visions s'accompliront.
\VS{24}Car il n'y aura plus désormais aucune vision de vanité ni aucune divination de flatteur, au milieu de la maison d'Israël.
\VS{25}Car moi, Yahweh, je parlerai, et la parole que j'aurai prononcée sera mis en exécution, elle ne sera plus différée ; mais ô maison rebelle! Je prononcerai en vos jours la parole, et je l'exécuterai dit le Seigneur Yahweh.
\VS{26}La parole de Yahweh me fut encore adressée en ces mots :
\VS{27}Fils de l'homme, voici, ceux de la maison d'Israël disent : La vision que celui-ci voit n'arrivera pas avant longtemps, et il prophétise pour des temps qui sont encore éloignés.
\VS{28}C'est pourquoi dis-leur : Ainsi parle le Seigneur Yahweh : Aucune de mes paroles ne sera plus différée, mais la parole que j'aurai prononcée sera exécutée incessament, dit le Seigneur Yahweh.
\Chap{13}
\TextTitle{Jugement sur ceux qui égarent le peuple de Dieu}
\VerseOne{}La parole de Yahweh me fut encore adressée en ces mots :
\VS{2}Fils de l'homme, prophétise contre les prophètes d'Israël qui prophétisent, et dis à ces prophètes qui prophétisent selon leur propre cœur : Ecoutez la parole de Yahweh !
\VS{3}Ainsi parle le Seigneur Yahweh : Malheur aux prophètes insensés qui suivent leur propre esprit, et qui n'ont point eu de vision.
\VS{4}Israël, tes prophètes ont été comme des renards dans les déserts.
\VS{5}Vous n'êtes point montés devant les brèches, et vous n'avez point réparé les murs pour la maison d'Israël, afin de vous tenir debout pour le combat au jour de Yahweh.
\VS{6}Ils ont eu des visions vaines et des divinations de mensonge, ils disent : Yahweh a dit ; et toutefois Yahweh ne les a point envoyés ; et ils font espérer que leur parole s'accomplira\FTNT{Les faux prophètes (Jé. 23) ; Jé. 14:14 ; Jé. 28:15.}.
\VS{7}N'avez-vous pas vu des visions de vanité, et prononcé des divinations de mensonge ? Cependant vous dites : Yahweh a parlé ; et je n'ai point parlé.
\VS{8}C'est pourquoi ainsi parle le Seigneur Yahweh : Parce que vous avez prononcé des choses vaines, et que vous avez eu des visions de mensonge, à cause de cela j'en veux à vous, dit le Seigneur Yahweh.
\VS{9}Et ma main sera sur les prophètes qui ont des visions de vanité et des divinations de mensonge ; ils ne seront plus admis dans le conseil de mon peuple, ils ne seront plus écrits dans les registres de la maison d'Israël, ils n'entreront plus dans la terre d'Israël ; et vous saurez que je suis le Seigneur Yahweh.
\VS{10}Parce, oui parce qu'ils ont abusé mon peuple, en disant : Paix ! Et il n'y avait point de paix\FTNT{Jé. 6:14 ; Jé. 8:11.}. L'un bâtissait le mur, et les autres l'induissaient de mortier mal lié.
\VS{11}Dis à ceux qui enduisent le mur de mortier mal lié, qu'il tombera ; il y aura une pluie débordante, et vous, pierres de grêle, vous tomberez sur lui, et un vent de tempête le fendra.
\VS{12}Et voici, le mur est tombé ; ne vous sera-t-il donc pas dit : Où est l'enduit dont vous l'avez couvert ?
\VS{13}C'est pourquoi, ainsi parle le Seigneur Yahweh : Je ferai dans ma fureur éclater un vent impétueux, et dans ma colère, il surviendra une pluie débordante et des pierres de grêle dans ma fureur, pour détruire entièrement.
\VS{14}Je démolirai le mur que vous avez enduit de mortier mal lié, je le jetterai par terre, tellement que son fondement sera découvert, et il tombera ; vous serez consumés au milieu de lui, et vous saurez que je suis Yahweh.
\VS{15}Ainsi j'accomplirai ma colère contre le mur, et contre ceux qui l'enduisent de mortier mal lié ; et je vous dirai : Le mur n'est plus ni ceux qui l'ont enduit ;
\VS{16}à savoir les prophètes d'Israël, qui prophétisent sur Jérusalem et qui voient pour elle des visions de paix ; et néanmoins il n'y a point de paix, dit le Seigneur Yahweh.
\VS{17}Aussi, toi, fils de l'homme, tourne ta face contre les filles de ton peuple qui prophétisent selon leur propre cœur, prophétise contre elles !
\VS{18} Et dis : ainsi parle le Seigneur Yahweh : Malheur à celles qui cousent des coussins\FTNT{Le mot « coussin » vient du terme hébreu « keceth », et signifie : bande, filet, faux phylactères,  tissu utilisé par les fausses prophétesses en Israël pour étayer leurs plans démoniaques de diseuses de bonne aventure. } pour s'accouder le long du bras jusqu'aux mains, et qui font des voiles pour mettre sur la tête des personnes de toute taille, pour séduire les âmes. Séduiriez-vous les âmes de mon peuple\FTNT{Ge. 10:9.} ; et conserveriez-vous vos âmes ?
\VS{19}Et me profaneriez-vous envers mon peuple pour des poignées d'orge et pour des morceaux de pain, en faisant mourir les âmes qui ne devaient point mourir et en faisant vivre les âmes qui ne devaient point vivre, en mentant à mon peuple qui écoute le mensonge ?
\VS{20}C'est pourquoi ainsi parle le Seigneur Yahweh : Voici, j'en veux à vos coussins, par lesquels vous séduisez les âmes pour les faire voler vers vous ; et je déchirerai ces coussins de vos bras, et je ferai échapper les âmes que vous avez attirées afin qu'elles volent vers vous\FTNT{Ap. 18:11-13 ; 1 Co. 6:10 ; 2 Pi. 2:14}.
\VS{21}Je déchirerai aussi vos voiles, et je délivrerai mon peuple d'entre vos mains, et ils ne seront plus entre vos mains pour en faire votre proie ; et vous saurez que je suis Yahweh.
\VS{22}Parce que vous avez affligé sans raison le cœur du juste, quand moi-même je ne l'ai point attristé, et que vous avez renforcé les mains du méchant, afin qu'il ne se détourne point de son mauvais chemin, et que je lui sauve la vie.
\VS{23}C'est pourquoi, vous n'aurez plus aucune vision de vanité ni aucune divination, mais je délivrerai mon peuple d'entre vos mains ; et vous saurez que je suis Yahweh.
\Chap{14}
\TextTitle{Jugement sur les anciens idolâtres}
\VerseOne{}Or quelques-uns des anciens d'Israël vinrent auprès de moi et s'assirent devant moi.
\VS{2}Et la parole de Yahweh me fut adressée en ces mots :
\VS{3}Fils de l'homme, ces gens élèvent leurs idoles dans leurs cœurs, et ils attachent les regards sur ce qui les fait tomber dans l'iniquité. Serais-je consulté par eux sérieusement ?
\VS{4}C'est pourquoi parle-leur et dis-leur : Ainsi parle le Seigneur Yahweh. Quiconque de la maison d'Israël aura élevé ses idoles dans son cœur, et aura mis devant sa face ce qui l'a fait tomber dans son iniquité, s'il vient vers le prophète, je suis Yahweh, je lui répondrai puisqu'il vient avec la multitude de ses idoles,
\VS{5}afin que je saisisse la maison d'Israël par leur propre cœur; car eux tous se sont éloignés de moi par leurs idoles.
\VS{6}C'est pourquoi dis à la maison d'Israël : Ainsi parle le Seigneur Yahweh : Revenez, et détournez-vous de vos idoles, détournez les regards de toutes vos abominations\FTNT{Es. 55:6-7.}.
\VS{7}Car quiconque de la maison d'Israël, ou des étrangers qui séjournent en Israël, qui s'est séparé de moi, qui éleve ses idoles dans son cœur, et attache ses regards sur ce qui l'a fait tomber dans l'iniquité, s'il vient vers le prophète pour me consulter par de lui, je suis Yahweh, on lui répondra tout ce qu'on a à lui répondre.
\VS{8}Je me tournerai contre cet homme\FTNT{Lé. 17:10 ; Lé. 20:3-6 ; Jé. 44:11.}, et je ferai de lui un signe, et un sujet de sarcasme\FTNT{No. 26:10 ; De. 28:37}. Je le retrancherai du milieu de mon peuple ; et vous saurez que je suis Yahweh.
\VS{9}S'il arrive que le prophète soit séduit, et qu'il profère quelque parole, moi, Yahweh, je séduirai ce prophète-là\FTNT{1 R. 22:23 ; Job. 12:16 ; 2 Th. 2:11.} ; et j'étendrai ma main sur lui, et je l'exterminerai du milieu de mon peuple d'Israël.
\VS{10}Et ils porteront la peine de leur iniquité ; la peine de l'iniquité du prophète sera comme la peine de celui qui l'aura interrogé ;
\VS{11}afin que la maison d'Israël ne s'éloigne plus de moi, et qu'ils ne se souillent plus par tous leurs crimes\FTNT{Jé. 31:18-19 ; Hé. 12:11 ; Ja. 1:1-3.} ; alors ils seront mon peuple, et je serai leur Dieu, dit le Seigneur Yahweh.
\TextTitle{Châtiments d'Israël ; Yahweh épargne un reste}
\VS{12}Puis la parole de Yahweh me fut adressée en ces mots :
\VS{13}Fils de l'homme, lorsqu'un pays aura péché contre moi, en commettant une infidélité, et que j'aurai étendu ma main contre lui, et que je lui aurai rompu le bâton du pain, envoyé la famine et retranché du milieu de lui tant les hommes que les bêtes,
\VS{14}et que ces trois hommes, Noé, Daniel et Job s'y trouvent, ils sauveraient leurs âmes par leur justice, dit le Seigneur Yahweh.
\VS{15}Si je fais passer les bêtes féroces par ce pays-là et qu'elles le privent d'enfants, tellement qu'il soit devenu un désert où personne ne passe à cause des bêtes,
\VS{16}et que ces trois hommes-là s'y trouvent, je suis vivant, dit le Seigneur Yahweh, ils ne sauveraient ni fils ni filles, eux seulement seraient sauvés, et le pays sera un désert.
\VS{17}Si je faisais venir l'épée sur ce pays-là et si je disais : Que l'épée passe par le pays, et qu'elle en retranche les hommes et les bêtes !
\VS{18}Si ces trois hommes-là se trouvent au milieu du pays, je suis vivant, dit le Seigneur, ils ne sauveraient ni fils ni filles ; mais eux seulement seraient sauvés.
\VS{19}Ou si j'envoyais la peste dans ce pays, et que je répandais ma colère contre lui jusqu'à faire ruisseler le sang, au point de retrancher du milieu de lui les hommes et les bêtes,
\VS{20}et que Noé\FTNT{Ge. 6:8.}, Daniel\FTNT{Da. 1:8-12.} et Job\FTNT{Job. 1:8.}, s'y trouvent, je suis vivant, dit le Seigneur Yahweh, ils ne sauveraient ni fils ni filles ; mais ils sauveraient leurs âmes par leur justice.
\VS{21}Car ainsi parle le Seigneur Yahweh : J'envoie mes quatre plaies mortelles, l'épée, la famine, les bêtes féroces, et la peste, contre Jérusalem, pour en retrancher les hommes et les bêtes\FTNT{Jé. 15:2-3.} ;
\VS{22}Et toutefois, il y aura un reste qui échappera, qui en sortira, des fils et des filles. Voici, ils viennent vers vous, et vous verrez leur conduite et leurs actions, et vous serez consolés du malheur que je fais venir contre Jérusalem, de tout ce que je fais venir sur elle.
\VS{23}Vous serez consolés, lorsque vous verrez leur conduite et leurs actions ; et vous reconnaîtrez que ce n'est pas sans raison que je fais tout ce que je lui fais, dit le Seigneur Yahweh\FTNT{Jé. 22:8-9.}.
\Chap{15}
\TextTitle{Infidélités d'Israël\FTNTT{Cp. Es. 5:1-24.}}
\VerseOne{}La parole de Yahweh me fut encore adressée, en disant :
\VS{2}Fils de l'homme, que vaut le bois de la vigne de plus que les autres bois ? Et les sarments de plus que les branches des arbres d'une forêt ?
\VS{3}Et prendra-t-on du bois pour en faire quelque ouvrage ? Ou prendra-t-on un clou pour y pendre quelque chose ?
\VS{4}Voici, on le met au feu pour être consumé ; le feu consume aussitôt ses deux bouts, et le milieu est en feu ; serait-il bon pour quelque ouvrage ?
\VS{5}Voici, quand il est entier, on n'en fait aucun ouvrage ; à plus forte raison quand le feu l'aura consumé et qu'il sera brûlé, sera-t-il bon pour quelque ouvrage ?
\VS{6}C'est pourquoi ainsi parle le Seigneur Yahweh : Comme le bois de la vigne est parmi les arbres d'une forêt, que j'ai assigné au feu pour être consumé, ainsi je livrerai les habitants de Jérusalem.
\VS{7}Je me tournerai contre eux, seront-ils sortis du feu ? Encore le feu les consumera ; et vous saurez que je suis Yahweh, quand je tournerai ma face contre eux.
\VS{8}Je ferai de ce pays une désolation, parce qu'ils ont commis une infidélité, dit le Seigneur Yahweh.
\Chap{16}
\TextTitle{Bonté de Yahweh, prostitutions d'Israël}
\VerseOne{}La parole de Yahweh me fut aussi adressée en ces mots :
\VS{2}Fils de l'homme, fais connaître à Jérusalem ses abominations.
\VS{3}Et dis : Ainsi parle le Seigneur Yahweh à Jérusalem : Tu as tiré ton origine et ta naissance du pays de Canaan ; ton père était Amoréen, et ta mère Héthienne.
\VS{4}Quant à ta naissance, le jour où tu naquis, ton cordon ombilical n'a pas été coupé, tu n'as pas été lavée dans l'eau pour être nettoyée ; tu n'as pas été salée de sel ni emmaillotée.
\VS{5}Il n'y a pas eu d'œil qui ait eu pitié de toi pour te faire une seule de ces choses, en ayant compassion pour toi ; mais tu as été jetée sur la face des champs le jour de ta naissance, parce qu'on avait horreur de toi.
\VS{6}Et passant près de toi, je te vis gisante par terre, dans ton sang, et je te dis : Vis dans ton sang ! Et je te redis encore : Vis dans ton sang !
\VS{7}Je t'ai fait croître par millions comme l'herbe des champs. Et tu pris de l'accroissement et tu devins grande, tu parvins à une parfaite bauté, tes seins se formèrent, ta chevelure poussa, tu devins nubile; mais tu étais abandonnée et sans habits.
\VS{8}Je passai près de toi, je te regardai, et voici, le temps était là, le temps des amours. J'étendis sur toi le pan de ma robe, et je couvris ta nudité. Je te jurai, j'entrai en alliance avec toi, dit le Seigneur Yahweh, et tu devins mienne.
\VS{9}Je te lavai dans l'eau en t'y plongeant, j'ôtai le sang de dessus toi, et je t'oignis d'huile.
\VS{10}Je te revêtis de vêtements brodés, je te chaussai de fourrure, je te ceignis de fin lin, et je te couvris de soie.
\VS{11}Je te parai d'ornements : Je mis des bracelets sur tes mains, et un collier à ton cou.
\VS{12}Je mis un anneau à ton nez, des pendants à tes oreilles, et une couronne de gloire sur ta tête.
\VS{13}Tu fus donc parée d'or et d'argent, et ton vêtement était de fin lin, de soie, et de broderie ; tu mangeas la fleur de farine, le miel, et l'huile ; tu devins extrêmement belle, et tu prospéras jusqu'à régner.
\VS{14}Ta renommée se répandit parmi les nations à cause de ta beauté, car elle était parfaite, à cause de ma gloire que j'avais mise sur toi, dit le Seigneur Yahweh.
\VS{15}Mais tu t'es confiée dans ta beauté, et tu t'es prostituée à cause de ta renommée, tu t'es abandonnée à tous les passants\FTNT{Es. 1:21 ; Jé. 2:20 ; Jé. 3:2-6 ; Os. 1:2.}.
\VS{16}Tu as pris tes vêtements pour t'en faire des hauts lieux de diverses couleurs, tels qu'il n'y en a point eu et n'en aura jamais, et tu t'y es prostituée.
\VS{17}Tu as pris ta magnifique parure d'or et d'argent, que je t'avais donnée, et tu t'en es fait des images d'hommes, tu as commis la fornication avec elles.
\VS{18}Tu as pris tes vêtements brodés, tu les en as couvertes, et tu as mis mon huile et mon encens devant elles.
\VS{19}Mon pain que je t'avais donné, la fleur de farine, l'huile, et le miel que je t'avais donné à manger, tu as mis cela devant elle en sacrifice de bonne odeur ; il a été fait ainsi, dit le Seigneur Yahweh.
\VS{20}Tu as aussi pris tes fils et tes filles que tu m'avais enfantés, et tu les as sacrifiés pour être mangés\FTNT{Lé. 18:21 ; Lé. 20:2 ; Es. 57:5 ; Jé. 19:5, Jé. 32:35.}. N'était-ce pas assez de tes prostitutions ?
\VS{21}Tu as égorgé mes fils, et tu les as livrés pour les faire passer par le feu, en l'honneur de ces idoles\FTNT{2 R. 17:17.}.
\VS{22}Et parmi toutes tes abominations et tes adultères, tu ne t'es point souvenue du temps de ta jeunesse, quand tu étais sans habits et toute nue, gisante par terre dans ton sang.
\VS{23}Après toutes tes méchancetés, malheur, malheur à toi ! dit le Seigneur Yahweh.
\VS{24}Tu t'es bâti un lieu éminent, et tu t'es fait des hauts lieux dans toutes les places.
\VS{25}A l'entrée de chaque chemin tu as bâti un haut lieu, et tu as rendu ta beauté abominable, tu t'es prostituée à tous les passants, tu as multiplié tes adultères.
\VS{26}Tu t'es abandonnée aux fils d'Egypte, tes voisins au corps avantageux ; tu as multiplié tes adultères pour m'irriter.
\VS{27}Et voici, j'ai étendu ma main sur toi, j'ai diminué la portion que je t'avais prescrite, et je t'ai abandonnée à la volonté de celles qui te haïssaient, des filles des Philistins, lesquelles ont honte de tes voies qui ne sont que méchanceté.
\VS{28}Tu t'es aussi abandonnée aux fils des Assyriens\FTNT{2 R. 16:7-10 ; Jé. 2:18-36.}, parce que tu n'étais pas encore rassasiée ; et après avoir commis l'adultère avec eux, tu n'as point encore été rassasiée.
\VS{29}Tu as multiplié tes adultères dans le pays de Canaan jusqu'en Chaldée, et avec cela tu n'as pas encore été rassasiée.
\VS{30}Quelle faiblesse de cœur tu as eue, dit le Seigneur, Yahweh, d'avoir fait toutes ces choses-là, qui sont les actions d'une femme qui se prostitue avec arrogance.
\VS{31}De t'être bâti un lieu éminent à chaque entrée de chemin, et d'avoir fait ton haut lieu dans toutes les places. Et tu n'as pas été comme la prostituée, car tu n'as point tenu compte du salaire.
\VS{32}Femme adultère, tu prends des étrangers au lieu de ton mari.
\VS{33}On donne un salaire à toutes les prostituées, mais toi tu as donné à tous tes amants des présents\FTNT{Es. 57:8-9 ; Os. 8:9-10.}, tu les as gagnés par des présents, afin que de toutes parts ils viennent vers toi, pour se plonger avec toi dans le crime.
\VS{34}Tu as été le contraire des autres prostituées, parce qu'on ne te recherchait pas ; et en donnant un salaire au lieu d'en recevoir un, tu as été le contraire des autres.
\TextTitle{Conséquences de l'infidélité de Jérusalem}
\VS{35}C'est pourquoi, ô adultère, écoute la parole de Yahweh !
\VS{36}Ainsi parle le Seigneur, Yahweh : Parce que ton venin s'est répandu, et que dans tes excès tu t'es abandonnée à ceux que tu aimais, à tes abominables idoles, et que tu as mis à mort tes fils que tu leur as donnés ;
\VS{37}à cause de cela, voici, je vais rassembler tous tes amants, avec lesquels tu te plaisais, et tous ceux que tu as aimés, avec tous ceux que tu as haïs ; je les assemblerai de toutes parts contre toi, et je découvrirai ta honte à leurs yeux et ils verront ton infamie.
\VS{38}Et je te jugerai comme on juge les femmes adultères, et celles qui répandent le sang\FTNT{Lé. 20:10 ; De. 22:22-30.} ; je te livrerai pour être mise à mort selon ma fureur et ma jalousie.
\VS{39}Je te livrerai, dis-je, entre leurs mains ; et ils détruiront tes maisons de prostitution, et ils détruiront tes hauts lieux ; ils te dépouilleront de tes vêtements, emporteront ta magnifique parure, et ils te laisseront sans habits et entièrement nue.
\VS{40}Et on fera monter contre toi une foule de gens qui te lapideront de pierres, et qui te perceront avec leurs épées.
\VS{41}Puis ils brûleront tes maisons, et feront justice de toi aux yeux d'un grand nombre de femmes, je ferai cesser tes prostitutions et tu ne donneras plus de salaires.
\VS{42}J'abandonnerai alors ma colère contre toi, et ma jalousie se retirera de toi ; je serai en repos, et je ne m'irriterai plus.
\VS{43}Parce que tu ne t'es point souvenue du temps de ta jeunesse, et que tu m'as provoqué par toutes ces choses-là ; à cause de cela, voici, j'ai fait tomber la peine de tes crimes sur ta tête, dit le Seigneur,Yahweh ; et tu ne feras plus de mauvais projets avec toutes tes abominations.
\VS{44}Voici, tous ceux qui usent de proverbes feront un proverbe de toi, en disant : Telle mère, telle fille !
\VS{45}Tu es la fille de ta mère, qui a dédaigné son mari et ses fils ; et tu es la sœur de chacune de tes sœurs, qui ont dédaigné leurs maris et leurs fils. Votre mère était Héthienne, et votre père était Amoréen.
\VS{46}Ta grande sœur qui demeure à ta gauche, c'est Samarie avec ses filles ; et ta petite sœur qui demeure à ta droite, c'est Sodome avec ses filles.
\VS{47}Et tu n'as pas seulement marché dans leurs voies et fait selon leurs abominations, c'était fort peu ; mais tu t'es corrompue plus qu'elles dans toutes tes voies.
\VS{48}Je suis vivant ! dit le Seigneur Yahweh, Sodome, ta sœur et tes filles, n'ont point fait comme tu as fait, toi et tes filles.
\VS{49}Voici quel a été le crime de Sodome, ta sœur : Elle avait de l'orgueil, elle vivait dans l'abondance de pain, et dans une insouciante tranquillité, elle et ses filles, elle ne fortifiait pas la main du pauvre et de l'indigent.
\VS{50}Elles se sont élevées, elles ont commis des abominations devant moi, et je me suis détourné quand j'ai vu cela.
\VS{51}Quant à Samarie, elle n'a pas commis la moitié de tes péchés ; car tu as multiplié tes abominations plus qu'elle, et tu as justifié tes sœurs par toutes les abominations que tu as commises.
\VS{52}Porte ta honte, toi qui as jugé chacune de tes sœurs, à cause de tes péchés, par lesquels tu as été rendue plus abominable qu'elles ; elles sont plus justes que toi ; c'est pourquoi sois honteuse, et porte ta confusion, vu que tu as justifié tes sœurs.
\VS{53}Quand je ramènerai leurs captifs, les captifs, dis-je, de Sodome et des villes de son ressort; et les captifs de Samarie et de villes de son ressort; je ramenèrai aussi les captifs de la captivité parmi elles!
\VS{54}afin que tu portes ta honte, et que tu sois confuse à cause de tout ce que tu as fait, et que tu les consoles.
\VS{55}Quand ta sœur Sodome, et les villes de son ressort retourneront à leur état précédent ; Samarie et les villes de son ressort retourneront à leur état précédent ; toi aussi, et les villes de ton ressort retournerez à votre état précédent.
\VS{56}Or ta bouche n'a point fait mention de ta sœur, Sodome, au jour de tes fiertés,
\VS{57}avant que ta méchanceté soit découverte ; lorsque tu as reçu les outrages des filles de Syrie, et de tous ses alentours, des filles des Philistins, qui te pillèrent de tous côtés !
\VS{58}Tu portes sur toi tes méchancetés et tes abominations, dit Yahweh.
\VS{59}Car ainsi parle le Seigneur Yahweh : Je te ferai comme tu as fait, quand tu as méprisé le serment en rompant l'alliance.
\TextTitle{Fidélité de Yahweh à son alliance}
\VS{60}Mais je me souviendrai de l'alliance que j'ai traitée avec toi dans les jours de ta jeunesse, et j'établirai avec toi une alliance éternelle\FTNT{Lé. 26:42-45 ; Ps. 106:45.}.
\VS{61}Et tu te souviendras de tes voies, et en seras confuse, lorsque tu recevras tes sœurs, tant tes plus grandes que tes plus petites, et je te les donnerai pour filles ; mais pas selon ton alliance.
\VS{62}Car j'établirai mon alliance avec toi, et tu sauras que je suis Yahweh,
\VS{63}afin que tu te souviennes de ta vie passée, que tu en sois honteuse, et que tu n'ouvres plus la bouche, à cause de ta confusion, après que j'aurai été apaisé envers toi, pour tout ce que tu auras fait, dit le Seigneur Yahweh.
\Chap{17}
\TextTitle{Enigme de Yahweh}
\VerseOne{}La parole de Yahweh me fut adressée en ces mots :
\VS{2}Fils de l'homme, propose une énigme, une parabole à la maison d'Israël.
\VS{3}Tu diras : Ainsi parle le Seigneur Yahweh : Un grand aigle à grandes ailes, aux ailes déployées, couvert de plumes de toutes les couleurs, vint au Liban, et enleva la cime d'un cèdre.
\VS{4}Il arracha la tête de ses rameaux, l'emmena dans un pays de commerce, et la mit dans une ville marchande.
\VS{5}Il prit de la semence du pays, et la mit dans un champ propre à semer, l'apporta près des grosses eaux, la planta comme un saule.
\VS{6}Cette semence poussa et devint un cep de vigne étendu, mais de peu d'élévation ; ses rameaux étaient tournés vers l'aigle, et ses racines étaient sous lui ; il devint une vigne, donna des jets, et produisit des branches.
\VS{7}Mais il y avait un autre grand aigle, aux longues ailes, et au plumage épais. Et voici, cette vigne serra vers lui ses racines, et étendit ses rameaux vers lui, afin qu'il l'arrose des eaux qui coulent des terrasses.
\VS{8}Elle était donc plantée dans une bonne terre, près des grosses eaux, en sorte qu'il y sortait des sarments et portait du fruit\FTNT{Mt. 13:8-23 ; Mc. 4:8-20 ; Lu. 8:8-15.}. Elle était devenue une vigne magnifique.
\VS{9}Dis : Ainsi parle le Seigneur Yahweh, prospèrera-t-elle ? N'arrachera-t-il pas ses racines, et ne coupera-t-il pas ses fruits pour qu'ils deviennent secs ? Tous les sarments qu'il a jetés sécheront, et il ne faudra pas un grand effort et beaucoup de monde pour l'enlever de dessus ses racines.
\VS{10}Mais voici, quoique plantée, prospèrera-t-elle ? Quand le vent d'orient l'aura touchée, ne séchera-t-elle pas entièrement ? Elle séchera sur le terrain où elle était plantée.
\TextTitle{Jugement de Dieu sur Sédécias\FTNTT{2 R. 24:17-20 ; 25:1-10.}}
\VS{11}Puis la parole de Yahweh me fut adressée en ces mots :
\VS{12}Parle maintenant à la maison rebelle : Ne savez-vous pas ce que veulent dire ces choses ? Dis : Voici, le roi de Babylone est venu à Jérusalem. Il a pris le roi, et les princes, et les a emmenés avec lui à Babylone.
\VS{13}Il a pris un de la race royale, il a traité alliance avec lui, il lui a fait prêter serment, et il a retenu les puissants du pays,
\VS{14}afin que le royaume soit tenu dans l'abaissement, et qu'il ne s'élève point, mais qu'en gardant son alliance, il subsiste.
\VS{15}Mais celui-ci s'est rebellé contre lui, envoyant ses messagers en Egypte, pour qu'on lui donne des chevaux et un grand peuple. Celui qui fait de telles choses prospérera-t-il, échappera-t-il ? Ayant enfreint l'alliance, échappera-t-il ?
\VS{16}Je suis vivant, dit le Seigneur Yahweh, c'est dans le pays du roi qui l'a établi pour roi, envers qui il a violé son serment et dont il a rompu l'alliance, c'est près de lui, au milieu de Babylone, qu'il mourra\FTNT{Référence à Nebucadnetsar. Sédécias eut les yeux crevés avant d'être emmené captif (2 R. 25:7 ; Jé. 34:3 ; Jé. 52:11).}.
\VS{17}Pharaon n'ira pas avec une grande armée et un peuple nombreux pour le secourir dans cette guerre, lorsque l'ennemi élèvera des terrasses et fera des retranchements pour exterminer beaucoup d'âmes.
\VS{18}Car il a méprisé le serment en violant l'alliance ; car voici, après avoir donné sa main, il a fait néanmoins toutes ces choses-là ; il n'échappera point !
\VS{19}C'est pourquoi ainsi parle le Seigneur Yahweh : Je suis vivant, si je ne fais tomber sur sa tête mon serment qu'il a méprisé, et mon alliance qu'il a enfreinte.
\VS{20}Et j'étendrai mon rets sur lui, et il sera pris dans mes filets, je le ferai entrer dans Babylone, et là j'entrerai en jugement contre lui pour le crime qu'il a commis contre moi.
\VS{21}Et tous ses fugitifs avec toutes ses troupes tomberont par l'épée, et ceux qui resteront seront dispersés à tout vent ; et vous saurez que moi, Yahweh, j'ai parlé.
\VS{22}Ainsi parle le Seigneur Yahweh : Je prendrai aussi un rameau de la cime de ce haut cèdre, et je le planterai ; je couperai, dis-je, du bout de ses jeunes branches, un tendre rameau, et je le planterai sur une montagne haute et éminente.
\VS{23}Je le planterai sur la haute montagne d'Israël, et là il produira des branches et produira du fruit, et il deviendra un excellent cèdre ; et des oiseaux de tout plumage demeureront sous lui, et habiteront sous l'ombre de ses branches.
\VS{24}Et tous les bois des champs connaîtront que moi, Yahweh, j'aurai abaissé le grand arbre, et élevé le petit arbre, fait sécher le bois vert, et fait reverdir le bois sec ; moi, Yahweh, j'ai parlé, et je le ferai.
\Chap{18}
\TextTitle{Chacun responsable de son péché}
\VerseOne{}La parole de Yahweh me fut encore adressée, en disant :
\VS{2}Que voulez-vous dire, vous qui usez ordinairement de ce proverbe touchant le pays d'Israël, en disant : Les pères ont mangé des raisins verts et les dents des enfants ont été agacées\FTNT{Jé. 31:29 ; La. 5:7.} ?
\VS{3}Je suis vivant, dit le Seigneur Yahweh et vous n'userez plus de ce proverbe en Israël.
\VS{4}Voici, toutes les âmes sont à moi ; l'âme du fils est à moi comme l'âme du père ; l'âme qui pèche sera celle qui mourra.
\VS{5}Mais l'homme qui est juste, et qui pratique la droiture et la justice,
\VS{6}qui ne mange pas sur les montagnes, et qui ne lève pas ses yeux vers les idoles de la maison d'Israël, et qui ne souille pas la femme de son prochain et ne s'approche pas de la femme dans son état d'impureté\FTNT{Lé 18:18 ; Lé. 20:18.},
\VS{7}qui n'opprime personne, qui rend le gage à son débiteur\FTNT{Ex. 22:26 ; De. 24:12-13.}, qui ne ravit pas le bien d'autrui, qui donne son pain à celui qui a faim et qui couvre d'un vêtement celui qui est nu\FTNT{De. 15:11 ; Es. 58:7.},
\VS{8}qui ne prête pas à intérêt, et ne tire pas d'usure, qui détourne sa main de l'iniquité et qui juge selon la vérité entre les parties qui plaident ensemble\FTNT{Ex. 22:25 ; Lé. 25:35-37 ; De. 23:19.},
\VS{9}qui suit mes lois et garde mes ordonnances pour agir avec fidélité, celui-là est juste, certainement il vivra, dit le Seigneur Yahweh.
\VS{10}Et s'il a engendré un fils qui soit un meurtrier, répandant le sang, et commettant des choses semblables ;
\VS{11}et qui ne fasse aucune de ces choses que j'ai ordonnées, s'il mange sur les montagnes, s'il déshonore la femme de son prochain,
\VS{12}s'il opprime le malheureux et le pauvre, s'il ravit le bien d'autrui, s'il ne rend pas le gage, s'il lève ses yeux vers les idoles et commet des abominations,
\VS{13}s'il prête à intérêt, et tire une usure, ce fils-là, vivrait ? Il ne vivra pas, quand il aura commis toutes ces abominations, on le fera mourir, et son sang retombera sur lui.
\VS{14}Mais s'il engendre un fils qui voie tous les péchés que commet son père, qui les voie et n'agisse pas de la même manière ;
\VS{15}S'il ne mange pas sur les montagnes et qu'il ne lève point ses yeux vers les idoles de la maison d'Israël, s'il ne déshonore pas la femme de son prochain,
\VS{16}s'il n'opprime personne, s'il ne prend point de gages, s'il ne ravit point le bien d'autrui, s'il donne de son pain à celui qui a faim et couvre celui qui est nu,
\VS{17}s'il retire sa main du pauvre, s'il n'exige ni usure ni intérêt, s'il garde mes ordonnances, et s'il suit mes lois ; il ne mourra point pour l'iniquité de son père, mais certainement il vivra.
\VS{18}Mais son père, parce qu'il a usé de fraude, et qu'il a ravi ce qui était à son frère, et fait parmi son peuple ce qui n'est pas bon, voici, il mourra pour son iniquité.
\VS{19}Mais, direz-vous : Pourquoi le fils ne porte-t-il pas l'iniquité de son père\FTNT{Ex. 20:5 ; De. 5:9.} ? Parce que le fils a fait ce qui était juste et droit, et qu'il a gardé toutes mes lois et les a observées, certainement il vivra.
\VS{20}L'âme qui pèche est celle qui mourra. Le fils ne portera point l'iniquité du père, et le père ne portera point l'iniquité du fils. La justice du juste sera sur le juste, et la méchanceté du méchant sera sur le méchant.
\VS{21}Si le méchant se détourne de tous ses péchés qu'il aura commis, et qu'il garde toutes mes lois, et fasse ce qui est juste et droit, certainement il vivra, il ne mourra point.
\VS{22}Il ne lui sera point fait mention de tous ses crimes qu'il aura commis, mais il vivra pour sa justice, à laquelle il se sera adonné.
\VS{23}Ce que je désire, est-ce que le méchant meure ? dit le Seigneur Yahweh. N'est-ce pas qu'il se détourne de ses mauvaises voies et qu'il vive ?
\VS{24}Mais si le juste se détourne de sa justice, et commet l'iniquité, selon toutes les abominations que le méchant a l'habitude de commettre, vivra-t-il ? Il ne sera point fait mention de toutes ses justices qu'il aura faites, à cause de son crime qu'il aura commis, et à cause de son péché qu'il aura fait ; il mourra à cause de ces choses-là.
\VS{25}Et vous, vous dites : La voie du Seigneur n'est pas bien réglée. Ecoutez maintenant maison d'Israël, ma voie n'est-elle pas bien réglée ? Ne sont-ce pas plutôt vos voies qui ne sont pas bien réglées ?
\VS{26}Si le juste se détourne de sa justice, et commet l'iniquité, il mourra à cause de ces choses-là ; il mourra à cause de son iniquité qu'il aura commise.
\VS{27}Si le méchant se détourne de sa méchanceté qu'il aura commise, et pratique ce qui est juste et droit, il fera vivre son âme.
\VS{28}Ayant donc considéré sa conduite, et s'étant détourné de tous ses crimes qu'il aura commis, certainement il vivra, il ne mourra point.
\VS{29}La maison d'Israël dit : La voie du Seigneur Yahweh n'est pas bien réglée. Ô maison d'Israël ! Mes voies ne sont-elles pas bien réglées ? Ne sont-ce pas plutôt vos voies qui ne sont pas bien réglées ?
\VS{30}C'est pourquoi je jugerai chacun de vous selon ses voies, ô maison d'Israël ! dit le Seigneur. Revenez, et détournez-vous de tous vos péchés, et l'iniquité ne vous ruinera pas.
\VS{31}Rejetez loin de vous tous les crimes par lesquels vous avez péché ; et faites-vous un nouveau cœur et un esprit nouveau ; pourquoi mourriez-vous, ô maison d'Israël ?
\VS{32}Car je ne désire pas la mort de celui qui meurt, dit le Seigneur Yahweh. Convertissez-vous donc, et vivez\FTNT{Ac. 3:19-20.}.
\Chap{19}
\TextTitle{Complaintes sur les dirigeants d'Israël}
\VerseOne{}Et toi, prononce à haute voix une complainte touchant les princes d'Israël.
\VS{2}Et dis : Ta mère, qu'était-ce ? C'était une lionne couchée parmi les lions, et qui a élevé ses petits parmi les jeunes lions.
\VS{3}Elle fit croître un de ses petits, qui devint un jeune lion, et qui apprit à déchirer la proie et a dévorer les hommes.
\VS{4}Les nations en entendirent parler, il fut attrapé dans leur fosse ; et elles l'emmenèrent avec des boucles au pays d'Egypte\FTNT{2 R. 23 : 33-34.}.
\VS{5}Puis ayant vu qu'elle attendait en vain, qu'elle était trompée dans son espérance, elle prit un autre de ses petits, et en fit un jeune lion.
\VS{6}Il marcha parmi les lions et devint un jeune lion, il apprit à déchirer la proie et a dévorer les hommes.
\VS{7}Il désola leurs palais, il ravagea leurs villes, de sorte que le pays, et tout ce qui y est, fut épouvanté par le cri de son rugissement.
\VS{8}Les nations s'armèrent contre lui de toutes les provinces, elles étendirent leurs rets contre lui, et il fut attrapé dans leur fosse\FTNT{2 R. 24:2.}.
\VS{9}Puis ils l'enfermèrent et l'enchaînèrent, pour l'amener au roi de Babylone, et le mettre dans une forteresse, afin que sa voix ne soit plus entendue sur les montagnes d'Israël.
\VS{10}Ta mère était comme une vigne dans ton sang plantée auprès des eaux, et elle est devenue chargée de fruits et de rameaux, à cause des grandes eaux.
\VS{11}Elle avait de puissantes branches pour en faire des sceptres de souverains ; son tronc s'était élevé jusqu'à ses branches touffues, et on la voyait dans sa hauteur avec la multitude de ses rameaux.
\VS{12}Mais elle a été arrachée avec fureur, et jetée par terre ; et le vent d'orient a séché son fruit ; ses puissantes branches se sont rompues et ont séché ; le feu les a consumées.
\VS{13}Maintenant elle est plantée dans le désert, dans une terre sèche et aride.
\VS{14}Le feu est sorti de ses branches, et a consumé son fruit ; et il n'y a plus en elle de puissantes branches pour un sceptre de souverain. C'est là une complainte, et cela servira de complainte.
\Chap{20}
\TextTitle{Compassions de Yahweh face aux infidélités d'Israël}
\VerseOne{}Or il arriva la septième année, au dixième jour du cinquième mois, que quelques-uns des anciens d'Israël vinrent pour consulter Yahweh, et s'assirent devant moi.
\VS{2}La parole de Yahweh me fut adressée en ces mots :
\VS{3}Fils de l'homme, parle aux anciens d'Israël, et dis-leur : Ainsi parle le Seigneur Yahweh : Est-ce pour me consulter que vous venez ? Je suis vivant, dit le Seigneur Yahweh, si vous me consultez.
\VS{4}Ne les jugeras-tu pas, ne les jugeras-tu pas, fils de l'homme ? Donne-leur à connaître les abominations de leurs pères.
\VS{5}Et dis-leur : Ainsi parle le Seigneur Yahweh : Le jour où j'ai choisi Israël, j'ai levé ma main vers la postérité de la maison de Jacob, et je me suis fait connaître à eux dans le pays d'Egypte, et j'ai levé ma main vers eux, en disant : Je suis Yahweh, votre Dieu.
\VS{6}En ce jour, j'ai levé ma main vers eux, pour les faire sortir du pays d'Egypte, pour les amener dans un pays que j'avais cherché pour eux, pays où coulent le lait et le miel, et qui est la noblesse de tous les pays\FTNT{Ex. 3:8 ; Ex. 6:7.}.
\VS{7}Alors je leur dis : Que chacun de vous rejette les abominations qui attirent ses regards, et ne vous souillez point par les idoles d'Egypte ! Je suis Yahweh, votre Dieu\FTNT{Jos. 24:14-23.}.
\VS{8}Mais ils se rebellèrent contre moi, et ils ne voulurent point m'écouter. Aucun ne rejeta les abominations qui attiraient ses regards, et ils n'abandonnèrent point les idoles de l'Egypte. Et je dis que je répandrais ma fureur sur eux, que je consumerais ma colère sur eux au milieu du pays d'Egypte.
\VS{9}Mais je les ai tirés hors du pays d'Egypte, je l'ai fait pour l'amour de mon Nom, afin qu'il ne soit point profané aux yeux des nations parmi lesquelles ils se trouvaient, et aux yeux desquelles je m'étais fait connaître à eux, pour les faire sortir du pays d'Egypte.
\VS{10}Je les fit donc sortir du pays d'Egypte, et je les conduisis dans le désert.
\VS{11}Je leur donnai mes lois et leur fis connaître mes ordonnances, que l'homme doit mettre en pratique, afin de vivre par elles\FTNT{Lé. 18:5 ; Ro. 10:5 ; Ga. 3:12.}.
\VS{12}Je leur donnai aussi mes sabbats, pour être un signe entre moi et eux, afin qu'ils sachent que je suis Yahweh qui les sanctifie\FTNT{Ex. 20:8 ; Ex. 31:13.}.
\VS{13}Mais ceux de la maison d'Israël se rebellèrent contre moi dans le désert. Ils ne suivirent point mes lois, et ils rejetèrent mes ordonnances que l'homme doit mettre en pratique, afin de vivre par elles, et ils profanèrent à l'excès mes sabbats. C'est pourquoi je dis que je répandrais sur eux ma fureur dans le désert pour les consumer\FTNT{Ex. 16:28.}.
\VS{14}Je l'ai fait pour l'amour de mon Nom, afin qu'il ne soit point profané devant les nations, en présence desquelles je les avais fait sortir d'Egypte\FTNT{Ex. 32:12 ; No. 14:13-14 ; De. 9:28 ; Jos. 7:9.}.
\VS{15}Je levai ma main vers eux dans le désert pour ne pas les amener dans le pays que je leur avais donné, pays où coulent le lait et le miel, et qui est la noblesse de tous les pays,
\VS{16}parce qu'ils ont rejeté mes ordonnances, qu'ils n'ont point suivi mes lois, et qu'ils ont profané mes sabbats, car leur cœur ne s'est pas éloigné de leurs idoles.
\VS{17}Toutefois, j'eus pour eux un regard de pitié pour ne point les détruire, et je ne les consumai point entièrement dans le désert.
\VS{18}Mais je dis à leurs fils dans le désert : Ne marchez point dans les statuts de vos pères, et ne gardez point leurs ordonnances, et ne vous souillez point par leurs idoles.
\VS{19}Je suis Yahweh, votre Dieu. Marchez dans mes statuts, et gardez mes ordonnances et accomplissez-les.
\VS{20}Sanctifiez mes sabbats, et ils seront un signe entre moi et vous, afin que vous reconnaissiez que je suis Yahweh, votre Dieu.
\VS{21}Mais les fils se rebellèrent aussi contre moi, et ils ne marchèrent point dans mes statuts, et ne gardèrent point mes ordonnances pour les faire ; ce que l'homme doit accomplir, pour vivre par elles. Ils profanèrent mes sabbats ; c'est pourquoi je dis que je répandrais ma fureur sur eux, et que je consumerais ma colère sur eux dans le désert.
\VS{22}Toutefois, je retirai ma main, et je le fis pour l'amour de mon Nom, afin qu'il ne soit point profané devant les nations, en présence desquelles je les avais sortis d'Egypte.
\VS{23}Néanmoins, je levai ma main vers eux dans le désert, pour les répandre parmi les nations, et les disperser dans les pays\FTNT{Lé. 26:13-33.},
\VS{24}parce qu'ils n'ont point accompli mes ordonnances, et qu'ils ont rejeté mes statuts, profané mes sabbats, et que leurs yeux se sont attachés aux idoles de leurs pères.
\VS{25}A cause de cela, je leur donnai des statuts qui n'étaient pas bons, et des ordonnances par lesquelles ils ne vivraient point.
\VS{26}Je les souillai par leurs dons, quand ils firent passer par le feu tous les premiers-nés, afin de les punir, et que l'on sache que je suis Yahweh.
\VS{27}C'est pourquoi, toi fils de l'homme, parle à la maison d'Israël, et dis-leur : Ainsi parle le Seigneur Yahweh : Vos pères m'ont encore outragé, car ils ont commis un crime contre moi.
\VS{28}Je les ai conduits dans le pays que j'avais juré de leur donner, et ils ont regardé toute haute colline, et tout arbre touffu, ils y ont fait leurs sacrifices, ils y ont posé leur oblation pour m'irriter, ils y ont mis leurs parfums, et ils y ont répandu leurs aspersions.
\VS{29}Je leur ai dit : Que veulent dire ces hauts lieux où vous allez ? Et le nom de hauts lieux leur a été donné jusqu'à ce jour.
\VS{30}C'est pourquoi dis à la maison d'Israël : Ainsi parle le Seigneur Yahweh : Ne vous souillez-vous pas selon les voies de vos pères, et ne vous prostituez-vous point à leurs idoles abominables,
\VS{31}en offrant vos dons, en faisant passer vos fils par le feu, en vous souillant par toutes vos idoles jusqu'à ce jour ? Est-ce ainsi que vous me consultez, ô maison d'Israël ? Je suis vivant, dit le Seigneur Yahweh, vous ne me consultez point.
\VS{32}Ce que vous pensez n'arrivera nullement, quand vous dites : Nous serons comme les nations, et comme les familles des pays, en servant le bois et la pierre.
\TextTitle{Restauration future d'Israël}
\VS{33}Je suis vivant ! dit le Seigneur Yahweh. Je règnerai sur vous avec une main forte, et un bras étendu, et avec effusion de colère.
\VS{34}Je vous sortirai du milieu des peuples, et vous rassemblerai hors des pays dans lesquels vous êtes dispersés, avec une main forte, et un bras étendu et avec effusion de colère.
\VS{35}Je vous ferai venir dans le désert des peuples, et je contesterai là contre vous, face à face,
\VS{36}comme j'ai contesté contre vos pères dans le désert du pays d'Egypte, ainsi je contesterai contre vous, dit le Seigneur Yahweh.
\VS{37}Je vous ferai passer sous la verge, et vous ramènerai au lieu de l'alliance\FTNT{Es. 65:12.}.
\VS{38}Je séparerai de vous les rebelles, et ceux qui se révoltent contre moi ; je les ferai sortir du pays dans lequel ils séjournent, mais ils n'entreront point dans la terre d'Israël ; et vous saurez que je suis Yahweh.
\VS{39}Vous donc, ô maison d'Israël, ainsi parle le Seigneur Yahweh : Allez, servez chacun vos idoles, puisque vous ne voulez pas m'écouter ! Ainsi vous ne profanerez plus mon saint Nom par vos dons et par vos idoles.
\VS{40}Mais ce sera sur ma sainte montagne, sur la haute montagne d'Israël, dit le Seigneur Yahweh, que toute la maison d'Israël me servira, dans le pays\FTNT{Jn. 4:21-24.}. Là, je prendrai plaisir en eux, et là je demanderai vos offrandes et les prémices de vos dons, et tout ce que vous me consacrerez.
\VS{41}Je prendrai plaisir en vous par vos parfums d'une agréable odeur, quand je vous aurai fait sortir du milieu des peuples, et que je vous aurai rassemblés des pays dans lesquels vous êtes dispersés ; je serai sanctifié par vous, aux yeux des nations.
\VS{42}Vous saurez que je suis Yahweh, quand je vous aurai fait revenir dans le pays d'Israël, dans le pays où j'ai levé ma main pour le donner à vos pères.
\VS{43}Et là, vous vous souviendrez de vos voies, et de toutes vos actions, par lesquelles vous vous êtes souillés ; et vous vous prendrez vous-mêmes en dégoût à cause de tous vos maux que vous aurez faits.
\VS{44}Vous saurez que je suis Yahweh, par tout ce que j'aurai fait pour vous, à cause de mon Nom, et non pas selon vos méchantes voies et vos actions corrompues, ô maison d'Israël ! dit le Seigneur Yahweh.
\Chap{21}
\TextTitle{L'épée de Yahweh}
\VerseOne{}La parole de Yahweh me fut encore adressée en ces mots :
\VS{2}Fils de l'homme, tourne ta face vers Jérusalem, parle en direction du sud, et prophétise contre la forêt du champ du sud.
\VS{3}Dis à la forêt du sud : Ecoute la parole de Yahweh. Ainsi parle le Seigneur Yahweh : Voici, je m'en vais allumer au dedans de toi un feu qui consumera tout bois vert et tout bois sec au dedans de toi ; la flamme de l'embrasement ne s'éteindra point, et tout le dessus en sera brûlé, depuis le sud jusqu'au nord\FTNT{Jé. 21:14 ; Jé. 22:7 ; Jé. 46:23 ; Lu. 23:31.}.
\VS{4}Toute chair verra que moi, Yahweh, j'ai allumé le feu ; et il ne s'éteindra point.
\VS{5}Je dis : Ah ! Seigneur Yahweh, ils disent de moi : N'est-il pas vrai que celui-ci ne fait que mettre en avant des similitudes ?
\TextTitle{Parabole de l'épée de Yahweh}
\VS{6}La parole de Yahweh me fut adressée en ces mots :
\VS{7}Fils de l'homme, tourne ta face vers Jérusalem, et parle en direction du lieu saint, et prophétise contre la terre d'Israël.
\VS{8}Dis à la terre d'Israël : Ainsi parle Yahweh : Voici, j'en veux à toi, je tirerai mon épée de son fourreau, et je retrancherai du milieu de toi le juste et le méchant.
\VS{9}Parce que je retrancherai du milieu de toi le juste et le méchant, à cause de cela mon épée sortira de son fourreau contre toute chair, depuis le sud jusqu'au nord.
\VS{10}Toute chair saura que moi, Yahweh, j'ai tiré mon épée de son fourreau, et elle n'y retournera plus.
\VS{11}Aussi, toi, fils de l'homme, gémis en te rompant les reins de douleur, et soupire avec amertume dans leur présence.
\VS{12}Quand ils te diront : Pourquoi gémis-tu ? Alors tu répondras : C'est à cause d'une nouvelle, car elle vient, et tout cœur se fondra, et toutes les mains seront baissées, tout esprit sera affaibli, et tous les genoux se fondront en eau ; voici, elle vient, elle arrive, dit le Seigneur Yahweh\FTNT{Jé. 6:24 ; Jé. 49:23.}.
\VS{13}Puis la parole de Yahweh me fut adressée en ces mots :
\VS{14}Fils de l'homme, prophétise, et dis : Ainsi parle Yahweh : Dis : L'épée ! L'épée a été aiguisée, elle est polie !
\VS{15}Elle a été aiguisée pour faire un grand carnage, elle a été polie afin qu'elle brille… Nous réjouirons-nous ? C'est la verge de mon fils, elle dédaigne tout bois.
\VS{16}Yahweh l'a donnée à polir, afin qu'on la tienne à la main ; l'épée a été aiguisée, et elle a été polie pour la mettre dans la main du destructeur.
\VS{17}Crie et hurle, fils de l'homme, car elle est contre mon peuple, elle est contre tous les princes d'Israël ; ils sont livrés à l'épée à cause de mon peuple. C'est pourquoi frappe sur ta cuisse !
\VS{18}Oui l'épreuve sera faite ; et que sera-ce si ce sceptre qui méprise tout est anéanti ? dit le Seigneur Yahweh.
\VS{19}Toi donc, fils de l'homme, prophétise, et frappe d'une main contre l'autre, que les coups de l'épée soient doublés, soient triplés, c'est l'épée du carnage, l'épée du grand carnage, l'épée qui doit les poursuivre.
\VS{20}J'ai mis à toutes leurs portes l'épée étincelante, afin que le cœur se fonde, et que les ruines soient multipliées. Ah ! Elle est faite pour briller et réservée pour tuer.
\VS{21}Joins-toi épée, frappe à la droite ! Avance-toi, frappe à la gauche, à tous côtés que tu rencontres !
\VS{22}Je frapperai aussi d'une main contre l'autre, je donnerai du repos à ma colère. Moi, Yahweh, j'ai parlé.
\VS{23}La parole de Yahweh me fut adressée en ces mots :
\VS{24}Toi, fils de l'homme, pose deux chemins où l'épée du roi de Babylone pourrait venir ; que les deux chemins sortent d'un même pays, et forme-les, forme-les de ta main à l'endroit où commence le chemin de la ville.
\VS{25}Tu poseras le chemin par lequel l'épée doit venir contre Rabbath des fils d'Ammon, et le chemin qui va en Judée, et à Jérusalem, ville forte.
\VS{26}Car le roi de Babylone se tient au carrefour, à l'entrée des deux chemins, pour consulter les devins ; il aiguise les flèches, il interroge les théraphim, il examine le foie.
\VS{27}Dans sa main droite est la divination contre Jérusalem, pour y dresser des béliers, pour publier le carnage, pour pousser des cris de guerre, pour ranger les béliers contre les portes, pour élever des terrasses et construire des remparts.
\VS{28}Mais ce sera pour eux, à leurs yeux, une divination vaine ; il y a de grands serments entre eux. Mais lui, il se souvient de leur iniquité, en sorte qu'ils seront pris.
\VS{29}C'est pourquoi ainsi parle le Seigneur Yahweh : Parce que vous avez fait revenir le souvenir de votre iniquité, lorsque vos crimes se sont découverts, au point de voir vos péchés dans toutes vos actions ; parce que vous avez fait qu'on se souvienne de vous, vous serez saisis par sa main.
\TextTitle{Quand l'iniquité arrive à son terme\FTNTT{Ap. 19:11-20:6.}}
\VS{30}Et toi, profane, méchant, prince d'Israël, dont le jour arrive au temps où l'iniquité est à son terme !
\VS{31}Ainsi parle le Seigneur Yahweh : Qu'on ôte cette tiare, et qu'on enlève cette couronne. Ce ne sera plus celle-ci ; j'élèverai ce qui est bas, et j'abaisserai ce qui est haut\FTNT{Job. 5:11 ; 1 Co. 1:27.}.
\VS{32}J'en ferai une ruine, une ruine, une ruine, et elle ne sera plus. Mais cela n'aura lieu qu'à la venue de celui à qui appartient le jugement et à qui je le lui donnerai.
\VS{33}Toi, fils de l'homme, prophétise, et dis : Ainsi parle le Seigneur Yahweh, au sujet des fils d'Ammon, et de leur opprobre : Dis donc, épée, épée dégainée, polie pour le massacre, pour dévorer avec son éclat !
\VS{34}Au milieu de tes visions vaines et de tes oracles menteurs, elle te fera tomber parmi les cadavres des méchants, dont le jour arrive au temps où l'iniquité est à son terme.
\VS{35}La remettrait-on dans son fourreau ? Je te jugerai sur le lieu où tu as été créé, au pays de ta naissance.
\VS{36}Je répandrai ma colère sur toi, j'allumerai sur toi le feu de ma fureur, et je te livrerai entre les mains d'hommes brutaux, qui ne travaillent qu'à détruire\FTNT{Jé. 25:11 ; Jé. 52:30.}.
\VS{37}Tu seras destiné au feu pour être dévoré ; ton sang sera au milieu de la terre : On ne se souviendra plus de toi, car c'est moi, Yahweh, qui parle.
\Chap{22}
\TextTitle{Les péchés d'Israël}
\VerseOne{}La parole de Yahweh me fut encore adressée en ces mots :
\VS{2}Et toi, fils de l'homme, ne jugeras-tu pas, ne jugeras-tu pas la ville sanguinaire, et ne lui donneras-tu pas à connaître toutes ses abominations\FTNT{Na. 3:1-4 ; Ha. 1:13 ; Ez. 24:6-9.} ?
\VS{3}Tu diras donc, ainsi parle le Seigneur Yahweh : Ville qui répands le sang au milieu de toi, afin que ton temps vienne, et qui as fait des idoles à ton préjudice, pour en être souillée.
\VS{4}Tu t'es rendue coupable par ton sang que tu as répandu, et tu t'es souillée par tes idoles que tu as faites ; tu as fait approcher tes jours, et tu es venue au terme de tes années ; c'est pourquoi je t'ai exposée en opprobre aux nations, et en dérision dans tous les pays\FTNT{2 R. 21:16 ; Jé. 26:21-23.}.
\VS{5}Celles qui sont près de toi, et celles qui en sont loin, se moqueront de toi, infâme de réputation, et remplie de troubles.
\VS{6}Voici, les princes d'Israël ont contribué au dedans de toi, chacun selon sa force, à répandre le sang.
\VS{7}Au dedans de toi, on méprise père et mère, on use de tromperie à l'égard de l'étranger, on opprime l'orphelin et la veuve.
\VS{8}Tu méprises ma sainteté, et profanes mes sabbats.
\VS{9}Des gens médisants sont au milieu de toi pour répandre le sang, ceux qui sont chez toi mangent sur les montagnes ; on commet des actions énormes au milieu de toi\FTNT{Es. 57:7 ; Jé. 2:20.}.
\VS{10}L'enfant découvre la nudité du père au milieu de toi, et on humilie au milieu de toi la femme dans le temps de son impureté\FTNT{Lé. 18:6-9 ; Ge. 9:22-23.}.
\VS{11}L'un commet l'abomination avec la femme de son prochain ; et l'autre se souille par l'inceste avec sa belle-fille ; chacun humilie sa sœur, fille de son père\FTNT{Ge. 19:32-36 ; Lé. 18:15-20 ; Jé. 5:8.}.
\VS{12}Chez toi, on reçoit des présents pour répandre le sang ; tu exiges un intérêt et une usure, tu dépouilles ton prochain par l'extorsion, et tu m'oublies, dit le Seigneur Yahweh\FTNT{Ex. 23:8 ; De. 27:25.}.
\VS{13}Voici, je frappe de mes mains l'une contre l'autre à cause de ton gain déshonnête que tu fais, et à cause de ton sang qui se répand au milieu de toi.
\VS{14}Ton cœur pourra-t-il tenir ferme, tes mains seront-elles fortes dans les jours où j'agirai contre toi ? Moi, Yahweh, j'ai parlé, et je le ferai.
\VS{15}Je te disperserai parmi les nations, je t'éparpillerai en divers pays, et je consumerai ta souillure, jusqu'à ce qu'il n'y en ait plus en toi.
\VS{16}Tu seras souillée par toi-même aux yeux des nations, et tu sauras que je suis Yahweh.
\TextTitle{La fureur de Yahweh}
\VS{17}Puis la parole de Yahweh me fut adressée en ces mots :
\VS{18}Fils de l'homme, la maison d'Israël m'est devenue comme de l'écume ; eux tous sont de l'airain, de l'étain, du fer et du plomb dans un creuset ; ils sont devenus comme une écume d'argent.
\VS{19}C'est pourquoi ainsi parle le Seigneur Yahweh : Parce que vous êtes tous devenus comme de l'écume, voici, je vais à cause de cela vous rassembler au milieu de Jérusalem,
\VS{20}comme on assemble de l'argent, de l'airain, du fer, du plomb, et de l'étain dans un creuset, afin d'y souffler le feu pour les fondre ; je vous rassemblerai ainsi dans ma colère et dans ma fureur, et je vous fondrai.
\VS{21}Je vous assemblerai, je soufflerai contre vous le feu de ma fureur, et vous serez fondus au milieu de Jérusalem.
\VS{22}Comme l'argent se fond dans le creuset, ainsi vous serez fondus au milieu d'elle, et vous saurez que moi,Yahweh, j'ai répandu ma fureur sur vous.
\TextTitle{Se tenir à la brèche devant Yahweh}
\VS{23}La parole de Yahweh me fut encore adressée en ces mots :
\VS{24}Fils de l'homme, dis-lui : Tu es une terre qui n'est pas purifiée ni arrosée de pluie au jour de la colère.
\VS{25}Il y a un complot de ses prophètes au milieu d'elle ; ils seront comme des lions rugissants, qui ravissent la proie : Ils dévorent les âmes, ils emportent les richesses et la gloire, ils multiplient les veuves au milieu d'elle\FTNT{Mt. 23:13 ; 1 Pi. 5:8.}.
\VS{26}Ses sacrificateurs ont fait violence à ma loi et ont profané mes choses saintes ; ils ne font pas de différence entre la chose sainte et profane; ils ne donnent pas à connaître la diffrence qu'il y a entre la chose impure et la pure, et ils cachent leurs yeux de mes sabbats, et je suis profané au milieu d'eux.
\VS{27}Ses princes sont au milieu d'elle comme des loups qui ravissent la proie, pour répandre le sang et pour détruire les âmes, pour s'adonner au gain déshonnête\FTNT{2 Pi. 2:16 ; Mt. 10:16 ; Mi. 3:11.}.
\VS{28}Ses prophètes ont pour eux des enduits de plâtre, des visions fausses, et des oracles menteurs, en disant : Ainsi parle le Seigneur Yahweh ; et cependant Yahweh n'a point parlé.
\VS{29}Le peuple du pays use de tromperies, ravit le bien d'autrui, opprime l'affligé et le pauvre, et foule l'étranger contre tout droit.
\VS{30}Je cherche parmi eux un homme\FTNT{Dieu n'a pas besoin d'une foule de gens avant d'agir. Une seule personne suffit. } qui élève un mur, et qui se tient à la brèche devant moi pour le pays, afin que je ne le détruise point ; mais je n'en trouve pas.
\VS{31}C'est pourquoi je répandrai sur eux ma colère, et je les consumerai par le feu de ma fureur ; je mettrai leur voie sur leur tête, dit le Seigneur Yahweh.
\Chap{23}
\TextTitle{Prostitutions d'Israël et de Juda}
\VerseOne{}La parole de Yahweh me fut encore adressée en ces mots :
\VS{2}Fils de l'homme, il y a eu deux femmes, filles d'une même mère,
\VS{3}qui se sont prostituées en Egypte, elles se sont prostituées dans leur jeunesse. Là, leur sein fut déshonoré et leur virginité touchée.
\VS{4}Et c'était ici leurs noms, celui de la plus grande était Ohola, et celui de sa sœur Oholiba\FTNT{Ohola signifie « sa propre tente », et Oholiba « la femme de la tente ». 2 R. 17:23-24.} ; elles étaient à moi, et elles ont enfanté des fils et des filles ; leurs noms donc étaient Ohola, qui était Samarie, et Oholiba, qui est Jérusalem.
\VS{5}Or Ohola a commis adultère étant ma femme, et s'est rendue amoureuse de ses amoureux, c'est-à-dire, des Assyriens ses voisins,
\VS{6}vêtus de pourpre, gouverneurs et magistrats, tous jeunes et aimables, tous cavaliers, montés sur des chevaux.
\VS{7}Elle a commis ses adultères avec toute l'élite des fils des Assyriens, et avec tous ceux pour qui elle s'était enflammée, et s'est souillée avec toutes leurs idoles.
\VS{8}Elle n'a pas abandonné ses fornications d'Egypte, car ils avaient couché avec elle dans sa jeunesse, ils avaient déshonoré sa virginité et s'étaient livrés à l'impureté avec elle\FTNT{Ac. 7:42.}.
\VS{9}C'est pourquoi je l'ai livrée entre les mains de ses amoureux, entre les mains des fils des Assyriens, dont elle s'était rendue amoureuse.
\VS{10}Ils l'ont couverte d'opprobre, ils ont enlevé ses fils et ses filles, et l'ont tuée elle-même avec l'épée ; elle a été en renom parmi les femmes, après avoir exercé des jugements sur elle.
\VS{11}Quand sa sœur Oholiba a vu cela, et fut plus déréglée qu'elle dans ses passions ; ses prostitutions dépassèrent celles de sa sœur.
\VS{12}Elle s'enflamma pour les fils des Assyriens, des gouverneurs et des magistrats, ses voisins, vêtus magnifiquement, et des cavaliers montés sur des chevaux, tous jeunes et bien faits.
\VS{13}J'ai vu qu'elle s'était souillée, et que l'une et l'autre avaient suivi la même voie.
\VS{14}Elle alla même plus loin dans ses prostitutions. Elle aperçut contre des murailles des peintures d'hommes, des images de Chaldéens peints en couleur rouge,
\VS{15}avec des ceintures autour des reins, avec des turbans de couleurs variées flottant sur la tête. Tous ayant l'apparence de princes, et figurant des fils de Babylone.
\VS{16}Elle s'enflamma pour eux au premier regard, et leur envoya des messagers en Chaldée.
\VS{17}Les fils de Babylone vinrent vers elle au lit de ses prostitutions, et la souillèrent par leurs adultères ; elle s'est aussi souillée avec eux, et après cela son cœur s'est détaché d'eux.
\VS{18}Elle a manifesté ses fornications et fait connaître son opprobre ; et mon cœur s'est détaché d'elle, comme mon cœur s'était détaché de sa sœur.
\VS{19}Car elle a multiplié ses adultères, jusqu'à rappeler le souvenir des jours de sa jeunesse, lorsqu'elle s'était abandonnée au pays d'Egypte.
\VS{20}Elle s'est enflammée pour des impudiques dont la chair était comme celle des ânes, et dont la force égale celle des chevaux.
\VS{21}Tu as donc repris les méchancetés de ta jeunesse, lorsque tu as été déshonorée, depuis que tu étais en Egypte, à cause du sein de ta jeunesse.
\VS{22}C'est pourquoi, Oholiba, ainsi parle le Seigneur Yahweh : Voici, je m'en vais réveiller contre toi tous tes amants, ceux dont ton cœur s'est détaché, et je les amènerai contre toi de toutes parts.
\VS{23}Les fils de Babylone, et tous les Chaldéens, Pekod, Shoa, Koa, et tous les Assyriens avec eux, tous jeunes gens d'élite, gouverneurs et magistrats, grands seigneurs et renommés, tous montant à cheval.
\VS{24}Ils viendront contre toi avec des armes, des chars, et des roues, avec une multitude de peuples, avec le grand bouclier et le petit bouclier, avec les casques, et je leur mettrai le jugement en main, ils te jugeront selon leur jugement.
\VS{25}Je mettrai ma jalousie contre toi, et ils agiront contre toi avec fureur ; ils te retrancheront le nez et les oreilles, et ce qui restera de toi tombera par l'épée. Ils enlèveront tes fils et tes filles, et ce qui restera de toi sera dévoré par le feu.
\VS{26}Ils te dépouilleront de tes vêtements, et t'enlèveront les ornements dont tu te pares.
\VS{27}Je mettrai un terme à tes méchancetés et tes prostitutions du pays d'Egypte ; tu ne lèveras plus tes yeux vers eux, et tu ne te souviendras plus de l'Egypte.
\VS{28}Car ainsi parle le Seigneur Yahweh : Voici, je te livre entre les mains de ceux que tu hais, entre les mains de ceux dont ton cœur s'est détaché.
\VS{29}Ils te traiteront avec haine ; ils enlèveront tout ton travail, et te laisseront sans habits et découverte ; et la turpitude de tes adultères, de ton énormité, et de tes fornications, sera découverte.
\VS{30}On te fera ces choses-là parce que tu t'es prostituée aux nations, avec lesquelles tu t'es souillée par leurs idoles.
\VS{31}Tu as marché dans la voie de ta sœur, c'est pourquoi je mets sa coupe dans ta main.
\VS{32}Ainsi parle le Seigneur Yahweh : Tu boiras la coupe profonde et large de ta sœur ; elle sera une coupe d'une grande mesure ; tu seras un sujet de risée et de moquerie\FTNT{Ps. 75:9 ; Es. 51:17 ; Jé. 25:15.}.
\VS{33}Tu seras remplie d'ivresse et de douleur, par la coupe de désolation et de dégât, qui est la coupe de ta sœur Samarie.
\VS{34}Tu la boiras et la videras, tu briseras ce pot de terre et tu déchireras ton sein. Car j'ai parlé, dit le Seigneur Yahweh.
\VS{35}C'est pourquoi ainsi parle le Seigneur Yahweh : Parce que tu m'as oublié, et que tu m'as jeté derrière ton dos, aussi porteras-tu la peine de ta méchanceté, et de tes prostitutions.
\TextTitle{Jugement sur Israël et Juda}
\VS{36}Puis Yahweh me dit : Fils de l'homme, ne jugeras-tu pas Ohola et Oholiba ? Déclare-leur donc leurs abominations.
\VS{37}Déclare-leur comment elles ont commis l'adultère et comment il y a du sang dans leurs mains ; comment, dis-je, elles ont commis l'adultère avec leurs idoles, et ont même fait passer par le feu leurs fils pour les consumer, ces enfants qu'elles m'avaient enfantés.
\VS{38}Voici encore ce qu'elles m'ont fait : Elles ont souillé mon lieu saint ce même jour, et ont profané mes sabbats.
\VS{39}Car après avoir égorgé leurs fils à leurs idoles, elles sont entrées ce même jour-là dans mon lieu saint pour le profaner ; et voilà, comment elles ont fait au milieu de ma maison\FTNT{2 R. 21:4.}.
\VS{40}Et qui plus est, elles ont fait chercher des hommes venant de loin, elles leur ont envoyé des messagers, et voici, ils sont venus. Pour eux tu t'es lavée, tu as fardé ton visage, et t'es parée d'ornements.
\VS{41}Tu t'es assise sur un lit magnifique, devant lequel a été apprêtée une table, sur laquelle tu as mis mon encens et mon huile.
\VS{42}On entendait le bruit d'une multitude tranquille ; et parmi cette foule d'hommes, on a fait venir du désert des Sabéens, qui ont mis des bracelets aux mains des deux sœurs, et de superbes couronnes sur leurs têtes.
\VS{43}J'ai dit au sujet de celle qui avait vieilli dans l'adultère : Maintenant ses impudicités prendront fin, et elle aussi.
\VS{44}Toutefois on est venu vers elle comme on vient vers une femme prostituée ; ils sont ainsi venus vers Ohola, et vers Oholiba, femmes pleines de méchanceté.
\VS{45}Les hommes justes donc les jugeront comme on juge les femmes adultères, et comme on juge celles qui répandent le sang ; car elles sont adultères, et le sang est dans leurs mains.
\VS{46}C'est pourquoi ainsi parle le Seigneur Yahweh : Qu'on fasse monter l'assemblée contre elles, et qu'elles soient abandonnées au tumulte et au pillage.
\VS{47}Que l'assemblée les lapide de pierres et les taille en pièces avec leurs épées ; qu'ils tuent leurs fils et leurs filles, et qu'ils brûlent au feu leurs maisons.
\VS{48}Et ainsi je ferai cesser la méchanceté dans le pays, et toutes les femmes seront enseignées à ne point faire selon votre méchanceté.
\VS{49}On mettra votre méchanceté sur vous, et vous porterez les péchés de vos idoles ; et vous saurez que je suis le Seigneur Yahweh.
\Chap{24}
\TextTitle{Malheur à la ville sanguinaire}
\VerseOne{}La neuvième année, au dixième jour du dixième mois, la parole de Yahweh me fut adressée en ces mots :
\VS{2}Fils de l'homme, mets par écrit la date de ce jour, de ce jour-ci ! Car en ce même jour le roi de Babylone s'approche contre Jérusalem\FTNT{2 R. 25:1.}.
\VS{3}Propose une parabole à la famille de rebelles, et dis-leur : Ainsi parle le Seigneur Yahweh : Mets, mets la chaudière, et verse de l'eau dedans.
\VS{4}Mets-y les morceaux, tous les bons morceaux, la cuisse, et l'épaule, et remplis-la des meilleurs os.
\VS{5}Prends la meilleure bête du troupeau, et fais brûler des os sous la chaudière, fais-la bouillir à gros bouillons, et que les os cuisent au-dedans.
\VS{6}Car ainsi parle le Seigneur Yahweh : Malheur à la ville sanguinaire, à la chaudière pleine de rouille, et de laquelle la rouille n'est point sortie ! Vide-la morceau par morceau, et que le sort ne soit point jeté sur elle.
\VS{7}Parce que son sang est au milieu d'elle, qu'elle l'a mis sur le rocher brillant, et qu'elle ne l'a point répandu sur la terre pour le couvrir de poussière,
\VS{8}j'ai mis son sang sur un rocher brillant, afin qu'il ne soit point couvert, pour faire monter la fureur, et pour me venger.
\VS{9}C'est pourquoi ainsi parle le Seigneur Yahweh : Malheur à la ville sanguinaire ! J'en ferai aussi un grand tas de bois à brûler !
\VS{10}Amasse beaucoup de bois, allume le feu, fais cuire la chair entièrement, et fais-la consumer, et que les os soient brûlés.
\VS{11}Puis mets sur les charbons ardents la chaudière toute vide, afin qu'elle s'échauffe et que son airain se brûle, et que sa souillure soit fondue à l'intérieur, et que sa rouille soit consumée.
\VS{12}Les efforts sont inutiles, sa rouille dont elle est pleine n'est point sortie d'elle ; sa rouille ne s'en ira que par le feu.
\VS{13}L'impureté est dans ta souillure ; car je t'avais purifiée, et tu n'as point été pure ; tu ne seras pas encore nettoyée de ta souillure, jusqu'à ce que j'aie assouvi sur toi ma fureur.
\VS{14}Moi, Yahweh, j'ai parlé, cela arrivera, et je le ferai ; et je ne me retirerai point en arrière, je n'épargnerai point, et je ne serai point apaisé. On t'a jugée selon tes voies et selon tes actions, dit le Seigneur Yahweh.
\TextTitle{La vie d'Ezéchiel, un signe pour Israël}
\VS{15}La parole de Yahweh me fut adressée en ces mots :
\VS{16}Fils de l'homme, voici, je vais t'ôter par une plaie ce que tes yeux voient avec le plus de plaisir. Ne mène point de deuil, ne pleure point, ne fais point couler tes larmes\FTNT{Jé. 16:6-7.}.
\VS{17}Garde-toi de gémir, et ne fais pas le deuil des morts ; attache ton turban sur ta tête, mets tes souliers à tes pieds, ne te couvre pas la barbe, et ne mange pas le pain des autres\FTNT{Lé. 10:6.}.
\VS{18}Je parlai au peuple le matin, et ma femme mourut le soir ; le lendemain matin je fis comme il m'avait été ordonné.
\VS{19}Le peuple me dit : Ne nous déclareras-tu point ce que nous signifient ces choses-là que tu fais ?
\VS{20}Je leur répondis : La parole de Yahweh m'a été adressée en ces mots :
\VS{21}Parle à la maison d'Israël : Ainsi parle le Seigneur Yahweh : Voici, je m'en vais profaner mon lieu saint, la magnificence de votre force, ce qui est le plus agréable à vos yeux, ce que vous voudriez épargner sur toutes choses ; et vos fils et vos filles, que vous aurez laissés, tomberont par l'épée.
\VS{22}Vous ferez alors comme j'ai fait ; vous ne couvrirez point vos barbes, et vous ne mangerez point le pain des autres.
\VS{23}Vos turbans seront sur vos têtes, et vos souliers à vos pieds ; vous ne mènerez point de deuil ni ne pleurerez ; mais vous pourrirez à cause de vos iniquités, et vous gémirez les uns avec les autres.
\VS{24}Ezéchiel sera pour vous un signe ; vous ferez selon toutes les choses qu'il a faites ; et quand cela sera arrivé, vous saurez que je suis le Seigneur Yahweh.
\VS{25}Quant à toi, fils de l'homme, au jour que je leur ôterai leur force, la joie de leur ornement, l'objet le plus agréable à leurs yeux, et l'objet de leurs cœurs, leurs fils et leurs filles,
\VS{26}ce jour-là un fuyard ne viendra-t-il pas vers toi pour te le raconter ?
\VS{27}En ce jour-là ta bouche sera ouverte envers celui qui sera échappé, et tu parleras, et ne seras plus muet ; ainsi tu seras pour eux un signe, et ils sauront que je suis Yahweh.
\Chap{25}
\TextTitle{Jugement de Dieu sur Ammon}
\VerseOne{}Puis la parole de Yahweh me fut adressée en ces mots :
\VS{2}Fils de l'homme, tourne ta face vers les fils d'Ammon, et prophétise contre eux\FTNT{Jé. 49:1.}.
\VS{3}Dis aux fils d'Ammon : Ecoutez la parole du Seigneur Yahweh : Parce que vous avez dit : Ah ! Ah ! contre mon lieu saint, parce qu'il était profané ; et contre la terre d'Israël, parce qu'elle était désolée ; et contre la maison de Juda, parce qu'ils allaient en captivité\FTNT{Am. 1:13 ; So. 2:8.};
\VS{4}à cause de cela, voici, je m'en vais te donner en héritage aux fils d'orient, et ils bâtiront des palais dans tes villes, et ils demeureront chez toi ; ils mangeront tes fruits et boiront ton lait.
\VS{5}Je livrerai Rabba pour être le repaire des chameaux, et le pays des fils d'Ammon pour être le gîte des brebis, et vous saurez que je suis Yahweh.
\VS{6}Car ainsi parle le Seigneur Yahweh : Parce que tu as frappé des mains, que tu as battu des pieds, et que tu t'es réjoui de bon cœur avec tout le mépris que tu as eu pour la terre d'Israël,
\VS{7}à cause de cela voici, j'ai étendu ma main sur toi, et je te livrerai pour être pillée par les nations, et je te retrancherai du milieu des peuples, je te ferai périr d'entre les pays ; je te détruirai ; et tu sauras que je suis Yahweh.
\TextTitle{Jugement sur Moab}
\VS{8}Ainsi parle le Seigneur Yahweh : Parce que Moab et Séir ont dit : Voici, la maison de Juda est comme toutes les autres nations ;
\VS{9}à cause de cela voici, j'ouvre le territoire de Moab du côté des villes, de ses villes frontières, la beauté du pays de Beth-Jeschimoth, de Baal-Meon et de Kirjathaïm\FTNT{Jos. 12:3 ; No. 32:38.},
\VS{10}je l'ouvre aux fils d'orient, qui sont au-delà du pays des fils d'Ammon, je leur donne en possession, afin qu'on ne se souvienne plus des fils d'Ammon parmi les nations.
\VS{11}J'exercerai aussi des jugements contre Moab, et ils sauront que je suis Yahweh.
\TextTitle{Jugement sur Edom}
\VS{12}Ainsi parle le Seigneur Yahweh : A cause de ce qu'Edom a fait quand il s'est vengé de la maison de Juda, et parce qu'il s'en est rendu coupable en se vengeant d'eux\FTNT{Ps. 137:7.},
\VS{13}à cause de cela, le Seigneur Yahweh dit : J'étendrai ma main sur Edom, j'en retrancherai les hommes et les bêtes, et j'en ferai un désert ; depuis Théman à Dedan ils tomberont par l'épée\FTNT{Jé. 49:7-9 ; Am. 1:12 ; Ab. 1:9.}.
\VS{14}J'exercerai ma vengeance sur Edom à cause de mon peuple d'Israël, et on traitera Edom selon ma colère, et selon ma fureur, et ils reconnaîtront ma vengeance, dit le Seigneur Yahweh.
\TextTitle{Jugement sur les Philistins}
\VS{15}Ainsi parle le Seigneur Yahweh : Puisque les Philistins ont agi par vengeance, et qu'ils se sont vengés avec mépris et du fond de leur âme, voulant tout détruire dans leur haine éternelle ;
\VS{16}à cause de cela le Seigneur Yahweh dit : Voici, je m'en vais étendre ma main sur les Philistins, j'exterminerai les Kéréthiens, et je ferai périr le reste sur le rivage de la mer.
\VS{17}J'exercerai sur eux de grandes vengeances par des châtiments de fureur ; et ils sauront que je suis Yahweh, quand j'aurai exécuté sur eux ma vengeance\FTNT{Es. 14:29 ; Jé. 25:20 ; So. 2:7.}.
\Chap{26}
\TextTitle{Jugement sur Tyr}
\VerseOne{}Il arriva dans la onzième année, le premier jour du mois, que la parole de Yahweh me fut adressée en ces mots :
\VS{2}Fils de l'homme, parce que Tyr a dit au sujet de Jérusalem : Ah ! Ah ! Celle qui était la porte des peuples a été rompue, elle s'est réfugiée chez moi, je serai remplie parce qu'elle a été rendue déserte\FTNT{Am. 1:9 ; Za. 9:2-3.} !
\VS{3}A cause de cela, ainsi parle le Seigneur Yahweh : Voici, j'en veux à toi, Tyr, et je ferai monter contre toi plusieurs nations, comme la mer fait monter ses flots\FTNT{Jé. 51:42.}.
\VS{4}Elles détruiront les murailles de Tyr, et démoliront ses tours ; j'en raclerai sa poussière, et la rendrai semblable à un rocher nu\FTNT{ Es. 23:15.}.
\VS{5}Elle servira à étendre les filets au milieu de la mer ; car j'ai parlé, dit le Seigneur Yahweh, et elle sera en pillage aux nations.
\VS{6}Ses filles sur sa terre seront tuées par l'épée, et elles sauront que je suis Yahweh.
\VS{7}Car ainsi parle le Seigneur Yahweh : Voici, je m'en vais faire venir du nord contre Tyr, Nebucadnetsar, roi de Babylone, le roi des rois, avec des chevaux, des chars, des cavaliers, et un grand peuple assemblé de toutes parts.
\VS{8}Il tuera par l'épée tes filles sur ta terre, il fera des remparts contre toi, il dressera des terrasses contre toi, et il lèvera les boucliers contre toi.
\VS{9}Il donnera des coups de béliers contre tes murs, et renversera tes tours avec ses épées.
\VS{10}La multitude de ses chevaux te couvrira de poussière, tes murs trembleront au bruit des cavaliers, des roues, et des chars, quand il entrera par tes portes, comme on entre dans une ville qu'on a divisée.
\VS{11}Il foulera toutes tes rues avec les sabots de ses chevaux, il tuera ton peuple avec l'épée, et les trophées de ta force tomberont par terre\FTNT{Jé. 47:3 ; Es. 5:8.}.
\VS{12}Puis ils retireront tes biens, et pilleront ta marchandise ; ils renverseront tes murs, et renverseront tes maisons de plaisance ; et ils mettront tes pierres, ton bois et ta poussière au milieu des eaux.
\VS{13}Je ferai cesser le bruit de tes chansons, et le son de tes harpes ne sera plus entendu.
\VS{14}Je te rendrai semblable à un rocher nu ; tu seras un lieu pour étendre les filets, et tu ne seras plus rebâtie, parce que moi,Yahweh, j'ai parlé, dit le Seigneur Yahweh.
\VS{15}Ainsi parle le Seigneur Yahweh, à Tyr : Les îles ne trembleront-elles pas du bruit de ta ruine, quand ceux qui seront blessés à mort gémiront, quand le carnage se fera au milieu de toi ?
\VS{16}Tous les princes de la mer descendront de leurs trônes, ôteront leurs manteaux, dépouilleront leurs vêtements brodés, et s'envelopperont de frayeur ; ils s'assiéront sur la terre, ils seront effrayés à chaque instant, et seront désolés à cause de toi.
\VS{17}Ils prononceront à haute voix une complainte sur toi, et te diront : Comment as-tu péri, toi qui étais fréquentée par ceux qui vont sur la mer, ville renommée, qui étais forte dans la mer, toi et tes habitants qui inspiraient la terreur à tous ceux qui habitent chez elle\FTNT{Es. 23:15-16 ; Ap. 18:9.}?
\VS{18}Maintenant les îles seront effrayées au jour de ta ruine, et les îles qui sont dans la mer seront terrifiées à cause de ta fuite.
\VS{19}Car ainsi parle le Seigneur Yahweh : Quand je ferai de toi une ville désolée, comme sont les villes qui ne sont point habitées, quand j'aurai fait tomber sur toi l'abîme, et que les grosses eaux t'auront couverte ;
\VS{20}alors je te ferai descendre avec ceux qui descendent dans la fosse, vers le peuple d'autrefois, et je te placerai aux lieux les plus bas de la terre, aux endroits désolés depuis longtemps, avec ceux qui descendent dans la fosse, afin que tu ne sois plus habitée, mais je donnerai la gloire pour la terre des vivants.
\VS{21}Je ferai qu'on sera épouvanté à cause de toi, de ce que tu n'es plus ; et quand on te cherchera, on ne te trouvera plus jamais, dit le Seigneur Yahweh.
\Chap{27}
\TextTitle{Lamentation sur Tyr\FTNTT{Cp. Ap. 18:1-24.}}
\VerseOne{}La parole de Yahweh me fut encore adressée en ces mots :
\VS{2}Toi donc, fils de l'homme, prononce à haute voix une complainte sur Tyr.
\VS{3}Tu diras à Tyr : Toi qui demeures au bord de la mer, qui trafiques avec les peuples dans plusieurs îles ; ainsi parle le Seigneur Yahweh : Tyr, tu disais : Je suis parfaite en beauté !
\VS{4}Ton territoire est au cœur de la mer, ceux qui t'ont bâtie t'ont rendue parfaite en beauté.
\VS{5}Ils t'ont bâti de tous les côtés des navires de sapins de Senir ; ils ont pris les cèdres du Liban pour te faire des mâts.
\VS{6}Ils ont fait tes rames de chênes de Basan, et la troupe des Assyriens a fait tes bancs d'ivoire, apporté des îles de Kittim.
\VS{7}Le fin lin d'Egypte, avec des broderies, te servait de voiles et de pavillon ; des étoffes teintes en bleu et en pourpre des îles d'Elischa formaient tes couvertures.
\VS{8}Les habitants de Sidon et d'Arvad étaient tes rameurs, ô Tyr ! Les plus sages du milieu de toi étaient tes pilotes.
\VS{9}Les anciens de Guebal et ses hommes experts furent parmi toi, réparant tes brèches ; tous les navires de la mer, et leurs mariniers étaient chez toi, pour faire l'échange de tes marchandises.
\VS{10}Ceux de Perse, de Lud, et de Puth servaient dans ton armée. C'étaient des hommes de guerre, ils suspendaient chez toi le bouclier et le casque ; ils t'ont rendue magnifique.
\VS{11}Les fils d'Arvad avec ton armée étaient autour de tes murs, et des hommes braves étaient dans tes tours ; ils ont suspendu leurs boucliers à tous tes murs, ils ont achevé de te rendre parfaite en beauté.
\VS{12}Ceux de Tarsis ont trafiqué avec toi de toutes sortes de richesses, d'argent, de fer, d'étain et de plomb.
\VS{13}Javan, Tubal, et Méschec trafiquaient avec toi ; ils donnaient des personnes et des ustensiles d'airain en échange de tes marchandises.
\VS{14}Ceux de la maison de Togarma pourvoyaient tes marchés de chevaux, de cavaliers, et de mulets.
\VS{15}Les fils de Dedan trafiquaient avec toi ; tu avais dans ta main le commerce de plusieurs îles ; et on t'a rendu en échange des dents d'ivoire et de l'ébène.
\VS{16}La Syrie trafiquait avec toi, en quantité d'ouvrages faits pour toi ; elle pourvoyait tes marchés d'escarboucles, d'écarlate, de broderie, de fin lin, de corail, et d'agate.
\VS{17}Juda et le pays d'Israël trafiquaient avec toi, faisant valoir ton commerce en blé de Minnith, en pâtisseries, en miel, en huile, et en baume.
\VS{18}Damas trafiquait avec toi en quantité d'ouvrages faits pour toi, en toutes sortes de richesses, en vin de Helbon, et en laine blanche.
\VS{19}Vedan, et Javan depuis Uzal, pourvoyaient tes marchés ; le fer luisant, la casse et le roseau aromatique furent dans ton commerce.
\VS{20}Dedan trafiquait avec toi en couvertures pour s'asseoir à cheval.
\VS{21}Les arabes, et tous les princes de Kédar, étaient des marchands dans ta main, trafiquant avec toi en agneaux, en moutons, et en boucs.
\VS{22}Les marchands de Séba et de Raema trafiquaient avec toi de tous les meilleurs aromates, de toute sorte de pierres précieuses et d'or.
\VS{23}Charan, Canné, et Eden, les marchands de Séba, d'Assyrie, de Kilmad, trafiquaient avec toi.
\VS{24}Ils trafiquaient avec toi toutes sortes de belles choses, des manteaux teints en bleu, en broderie, en riches étoffes contenues dans des coffres attachés avec des cordes, faits en bois de cèdre, et amenés sur tes marchés.
\VS{25}Les navires de Tarsis naviguaient pour ton commerce ; tu étais au comble de la force et de la richesse, au cœur des mers.
\VS{26}Tes rameurs t'ont amenée dans de grosses eaux, le vent d'orient t'a brisée au cœur de la mer.
\VS{27}Tes richesses, tes marchés et tes marchandises, tes mariniers et tes pilotes, ceux qui réparaient tes brèches, et ceux qui s'occupaient de ton commerce, tous tes hommes de guerre qui étaient chez toi, et toute ta multitude au milieu de toi, tomberont dans le cœur de la mer au jour de ta ruine\FTNT{Ap. 18:9.}.
\VS{28}Les faubourgs trembleront au bruit du cri de tes pilotes.
\VS{29}Tous ceux qui manient la rame descendront de leurs navires, les mariniers, et tous les pilotes de la mer ; ils se tiendront sur la terre ;
\VS{30}ils feront entendre leur voix, et crieront amèrement ; ils jetteront de la poussière sur leurs têtes, et se vautreront dans la cendre ;
\VS{31}ils arracheront leurs cheveux, et rendront leur tête chauve à cause de toi, ils se ceindront de sacs, et te pleureront avec l'amertume dans leur âme, en menant un deuil amer.
\VS{32}Ils prononceront à haute voix sur toi une complainte dans leur lamentation, et feront leur complainte sur toi, en disant : Qui fut jamais comme Tyr, comme cette ville détruite au cœur de la mer ?
\VS{33}Tu as rassasié plusieurs peuples par la traite des marchandises qu'on apportait de tes marchés au-delà des mers ; et tu as enrichi les rois de la terre par la multitude de tes richesses et de ton commerce.
\VS{34}Quand tu as été brisée par la mer au fond des eaux, ton commerce et toute ta multitude sont tombés avec toi.
\VS{35}Tous les habitants des îles sont désolés à cause de toi ; et leurs rois sont saisis d'épouvante, et leur visage pâlit.
\VS{36}Les marchands parmi les peuples t'insultent, tu es réduite au néant, tu ne seras plus à jamais !
\Chap{28}
\TextTitle{Yahweh réprime l'arrogance du roi de Tyr}
\VerseOne{}La parole de Yahweh me fut encore adressée en ces mots :
\VS{2}Fils de l'homme, dis au prince de Tyr : Ainsi parle le Seigneur Yahweh : Parce que ton cœur s'est élevé et que tu as dit : Je suis Dieu, je suis assis sur le siège de Dieu, au cœur de la mer, quoique tu sois un homme, et non Dieu, et parce que tu as élevé ton cœur comme si tu étais un Dieu.
\VS{3}Voici, tu es plus sage que Daniel, rien de caché ne t'a été rendu obscur.
\VS{4}Tu t'es acquis de la puissance par ta sagesse et par ton intelligence ; et tu as amassé de l'or et de l'argent dans tes trésors\FTNT{Za. 9:2-3.}.
\VS{5}Tu as multiplié ta puissance par la grandeur de ta sagesse dans ton commerce, puis ton cœur s'est élevé à cause de ta puissance.
\VS{6}C'est pourquoi ainsi parle le Seigneur Yahweh : Parce que tu as élevé ton cœur, comme si tu étais un Dieu,
\VS{7}à cause de cela voici, je m'en vais faire venir contre toi des étrangers, les plus terribles parmi les nations, qui tireront leurs épées sur la beauté de ta sagesse, et souilleront ta splendeur.
\VS{8}Ils te feront descendre dans la fosse, et tu mourras comme ceux qui tombent percés de coups, au milieu de la mer.
\VS{9}En face de ton meurtrier diras-tu : Je suis Dieu ? Tu seras homme et non Dieu sous la main de celui qui te tuera.
\VS{10}Tu mourras de la mort des incirconcis par la main des étrangers ; car j'ai parlé, dit le Seigneur Yahweh.
\TextTitle{Chute du roi de Tyr représentant satan\FTNTT{Cp. Es. 14:12-17.}}
\VS{11}La parole de Yahweh me fut encore adressée en ces mots :
\VS{12}Fils de l'homme, prononce à haute voix une complainte sur le roi de Tyr, et dis-lui : Ainsi parle le Seigneur Yahweh : Toi à qui rien ne manquait, plein de sagesse, et parfait en beauté ;
\VS{13}tu étais en Eden, le jardin de Dieu ; ta couverture était de pierres précieuses de toutes sortes, de sardoine, de topaze, de diamant, de chrysolithe, d'onyx, de jaspe, de saphir, d'escarboucle, d'émeraude, et d'or ; tes tambourins et tes flûtes étaient à ton service ; préparés pour le jour où tu fus créé.
\VS{14}Tu étais un chérubin, oint pour servir de protection ; je t'avais établi, et tu étais sur la sainte montagne de Dieu ; tu marchais entre les pierres éclatantes.
\VS{15}Tu étais parfait dans tes voies dès le jour où tu fus créé, jusqu'à celui où l'injustice fut trouvée en toi.
\VS{16}Selon la grandeur de ton trafic\FTNT{Satan est le premier commerçant. Il avait transformé ses sanctuaires célestes en un lieu de trafic, en un marché. Il avait reçu gratuitement du Seigneur, notre Dieu, plusieurs dons : la beauté, des pierres précieuses, des instruments de musique, la sagesse, un sanctuaire. Au lieu de les utiliser pour la gloire de Dieu, il en fit un trafic pour son propre profit égoïste. Il est le père de tous ceux qui vendent les dons de Dieu pour s'enrichir, de tous ceux qui font du commerce avec l'Evangile. Or Jésus nous a donné cet ordre formel : « Vous avez reçu gratuitement, donnez gratuitement » (Mt. 10 : 8). De même que le temple de Dieu était devenu une caverne de voleurs, plusieurs pasteurs ont transformé les bâtiments de leurs églises en véritables boutiques pour vendre toutes sortes de produits dérivés qui ne servent pas à l'avancement du Royaume de Dieu mais à enrichir des dirigeants cupides esclaves du dieu Mammon (Jn. 2:13-17 ; Mt. 6:24 ; Lu. 16:13 ; 1 Ti. 6:10 ; Hé. 13:5).}, tu as été rempli de violence, et tu as péché ; c'est pourquoi je te jette comme une chose souillée hors de la montagne de Dieu\FTNT{Ap. 12:1-12.}, et je te détruis d'entre les pierres éclatantes, ô chérubin protecteur !
\VS{17}Ton cœur s'est élevé à cause de ta beauté, tu as corrompu ta sagesse à cause de ton éclat ; je te jette par terre, je te donne en spectacle aux rois, afin qu'ils te regardent.
\VS{18}Tu as profané tes sanctuaires par la multitude de tes iniquités, par l'injustice de ton commerce ; et je fais sortir du milieu de toi un feu qui te consume, je te réduis en cendres sur la terre, dans la présence de tous ceux qui te regardent.
\VS{19}Tous ceux qui te connaissent parmi les peuples sont désolés à cause de toi ; tu es réduit à néant, tu ne seras plus à jamais.
\TextTitle{Jugement sur Sidon}
\VS{20}Puis la parole de Yahweh me fut adressée en ces mots :
\VS{21}Fils de l'homme, tourne ta face vers Sidon, et prophétise contre elle.
\VS{22}Tu diras : Ainsi parle le Seigneur Yahweh : Voici j'en veux à toi, Sidon ! Je serai glorifié au milieu de toi ; et on saura que je suis Yahweh, quand j'aurai exercé des jugements contre elle et que je serai sanctifié.
\VS{23}J'enverrai la peste dans son sein, je ferai couler le sang dans ses rues. Les morts tomberont au milieu d'elle par l'épée qui viendra de toutes parts sur elle ; et ils sauront que je suis Yahweh.
\VS{24}Elle ne sera plus pour la maison d'Israël une épine qui blesse, une ronce déchirante, parmi tous ceux qui l'entourent et qui la méprisent. Et ils sauront que je suis le Seigneur Yahweh.
\TextTitle{Rétablissement d'Israël}
\VS{25}Ainsi parle le Seigneur Yahweh : Quand j'aurai rassemblé la maison d'Israël d'entre les peuples parmi lesquels ils auront été dispersés, je manifesterai en elle ma sainteté, aux yeux des nations, et ils habiteront sur leur terre que j'ai donnée à mon serviteur Jacob.
\VS{26}Ils y habiteront en sûreté, ils bâtiront des maisons, ils planteront des vignes ; ils y habiteront, dis-je, en sûreté, lorsque j'aurai exercé des jugements contre ceux qui les auront pillés de toutes parts ; et ils sauront que je suis Yahweh leur Dieu.
\Chap{29}
\TextTitle{Jugement sur l'Egypte}
\VerseOne{}La dixième année, au douzième jour du dixième mois, la parole de Yahweh me fut adressée en ces mots :
\VS{2}Fils de l'homme, tourne ta face contre Pharaon, roi d'Egypte, prophétise contre lui, et contre toute l'Egypte\FTNT{Jé. 43:8-11.}.
\VS{3}Parle, et dis : Ainsi parle le Seigneur Yahweh : Voici, j'en veux à toi, Pharaon, roi d'Egypte, grand serpent couché au milieu de tes fleuves, qui dis : Mes fleuves sont à moi, et je me les suis faits\FTNT{Ps. 74:13-14 ; Es. 27:1.} !
\VS{4}C'est pourquoi je mettrai des crocs dans ta mâchoire, j'attacherai à tes écailles les poissons de tes fleuves ; je te tirerai hors de tes fleuves, avec tous les poissons de tes fleuves, qui seront attachés à tes écailles.
\VS{5}Et t'ayant tiré dans le désert, je te laisserai là, toi, et tous les poissons de tes fleuves ; tu tomberas sur la face des champs, tu ne seras point recueilli ni ramassé ; je te livrerai aux bêtes de la terre, et aux oiseaux des cieux, pour en être dévoré.
\VS{6}Et tous les habitants d'Egypte sauront que je suis Yahweh ; parce qu'ils ont été un soutien de roseau pour la maison d'Israël\FTNT{2 R. 18:21 ; Es. 36:6.}.
\VS{7}Quand ils t'ont pris par la main, tu t'es rompu, et tu leur as percé toute l'épaule ; et quand ils se sont appuyés sur toi, tu t'es cassé, et tu les as fait tomber à la renverse.
\VS{8}C'est pourquoi ainsi parle le Seigneur Yahweh : Voici, je m'en vais faire venir l'épée sur toi, et j'exterminerai du milieu de toi les hommes et les bêtes.
\VS{9}Le pays d'Egypte sera dans la désolation et dans le désert, et ils sauront que je suis Yahweh, parce que le roi d'Egypte a dit : Les fleuves sont à moi, et je les ai faits !
\VS{10}C'est pourquoi voici, j'en veux à toi, et à tes fleuves, et je réduirai le pays d'Egypte en désert de sécheresse et de désolation, depuis Migdol jusqu'à Syène, et aux frontières de l'Ethiopie.
\VS{11}Nul pied d'homme ne passera par là, et il n'y passera non plus aucun pied d'animal, elle sera quarante ans sans être habitée.
\VS{12}Car je réduirai le pays d'Egypte en désolation entre les pays désolés, et ses villes entre les villes réduites en désert ; elles seront en désolation durant quarante ans, je disperserai les Egyptiens parmi les nations, et je les répandrai parmi les pays.
\VS{13}Toutefois, ainsi parle le Seigneur Yahweh : Au bout de quarante ans, je ramasserai les Egyptiens d'entre les peuples parmi lesquels ils auront été dispersés ;
\VS{14}je ramènerai les captifs d'Egypte, et les ferai retourner au pays de Pathros, au pays de leur origine, mais ils seront là un royaume rabaissé.
\VS{15}Il sera le plus bas des royaumes, et il ne s'élèvera plus au-dessus des nations, je le diminuerai, afin qu'il ne domine point sur les nations.
\VS{16}Ce royaume ne sera plus pour la main d'Israël un sujet de confiance ; il lui rappellera son iniquité, quand elle se tournait vers eux ; et ils sauront que je suis le Seigneur Yahweh.
\VS{17}Il arriva la vingt-septième année, au premier jour du premier mois, que la parole de Yahweh me fut adressée en ces mots :
\VS{18}Fils de l'homme, Nebucadnetsar, roi de Babylone, a fait servir son armée dans un service pénible contre Tyr ; toute tête en est devenue chauve, et toute épaule en a été foulée, mais il n'a point eu de salaire, ni lui ni son armée, à cause de Tyr, pour le service qu'il a fait contre elle.
\VS{19}C'est pourquoi ainsi parle le Seigneur Yahweh : Voici, je m'en vais donner à Nebucadnetsar, roi de Babylone, le pays d'Egypte ; il enlèvera la multitude, il emportera le butin et fera le pillage ; ce sera là le salaire de son armée.
\VS{20}Pour prix du service qu'il a fait contre Tyr, je lui ai donné le pays d'Egypte, parce qu'ils ont travaillé pour moi, dit le Seigneur Yahweh.
\VS{21}En ce jour-là, je ferai germer la corne de la maison d'Israël, et j'ouvrirai ta bouche au milieu d'eux, et ils sauront que je suis Yahweh.
\Chap{30}
\TextTitle{Disgrâce de l'Egypte}
\VerseOne{}La parole de Yahweh me fut encore adressée en ces mots :
\VS{2}Fils de l'homme, prophétise, et dis : Ainsi parle le Seigneur Yahweh : Hurlez, et dites : Malheureux jour !
\VS{3}Car le jour est proche, oui le jour de Yahweh est proche, c'est un jour ténébreux ; ce sera le temps des nations.
\VS{4}L'épée viendra sur l'Egypte, et il y aura de l'effroi en Ethiopie, quand ceux qui seront blessés à mort tomberont dans l'Egypte, et quand on enlèvera la multitude de son peuple, et que ses fondements seront ruinés.
\VS{5}L'Ethiopie, Puth, Lud, toute l'Arabie, Cub, et les fils du pays allié tomberont par l'épée avec eux\FTNT{Jé. 46:9 ; Na. 3:9-10.}.
\VS{6}Ainsi parle Yahweh : Ceux qui soutiendront l'Egypte, tomberont ; et l'orgueil de sa force sera renversé ; ils tomberont par l'épée de Migdol à Syène, dit le Seigneur Yahweh.
\VS{7}Ils seront désolés au milieu des pays désolés, et ses villes seront au milieu des villes désertes.
\VS{8}Ils sauront que je suis Yahweh, quand j'aurai mis le feu en Egypte ; et tous ceux qui lui donneront du secours, seront brisés.
\VS{9}En ce jour-là, des messagers sortiront de ma part sur des navires pour effrayer l'Ethiopie dans sa sécurité, et il y aura entre eux un tourment au jour de l'Egypte ; car voici, il vient.
\VS{10}Ainsi parle le Seigneur Yahweh : Je ferai périr la multitude d'Egypte par la puissance de Nebucadnetsar, roi de Babylone.
\VS{11}Lui et son peuple avec lui, les plus terribles d'entre les nations, seront amenés pour ruiner le pays, et ils tireront leurs épées contre les Egyptiens, et rempliront la terre de morts.
\VS{12}Je mettrai à sec les fleuves et je livrerai le pays entre les mains des méchants ; je désolerai le pays, et tout ce qui y est, par la puissance des étrangers ; moi, Yahweh, j'ai parlé.
\VS{13}Ainsi parle le Seigneur Yahweh : Je détruirai aussi les idoles, j'anéantirai les faux dieux de Noph, et il n'y aura point de prince qui soit du pays d'Egypte ; je mettrai la frayeur dans le pays d'Egypte\FTNT{Es. 19:1-13 ; Jé. 43:12 ; Jé. 46:13.}.
\VS{14}Je désolerai Pathros, je mettrai le feu à Tsoan, et j'exercerai mes jugements sur No\FTNT{Jé. 44:1.}.
\VS{15}Je répandrai ma fureur sur Sin, qui est la place forte de l'Egypte, et j'exterminerai la multitude qui est à No.
\VS{16}Quand je mettrai le feu en Egypte, Sin sera grièvement tourmentée, et No sera rompue par diverses brèches, et il n'y aura à Noph que détresses en plein jour.
\VS{17}Les jeunes hommes d'On et de Pi-Béseth tomberont par l'épée, et ces villes iront en captivité.
\VS{18}Le jour s'obscurcira à Tachpanès, lorsque j'y romprai le joug de l'Egypte, et que l'orgueil de sa force aura cessé ; un nuage la couvrira, et les villes de son ressort iront en captivité.
\VS{19}J'exercerai des jugements en Egypte ; et ils sauront que je suis Yahweh.
\TextTitle{Chute et dispersion de l'Egypte}
\VS{20}Dans la onzième année, au septième jour du premier mois, la parole de Yahweh me fût adressée en ces mots :
\VS{21}Fils de l'homme, j'ai rompu le bras de Pharaon, roi d'Egypte ; et voici on ne l'a point bandé pour le guérir, on ne lui a point mis de linges pour le bander, et pour le fortifier, afin qu'il puisse manier l'épée.
\VS{22}C'est pourquoi ainsi parle le Seigneur Yahweh : Voici, j'en veux à Pharaon, roi d'Egypte, et je romprai ses bras, tant celui qui est fort que celui qui est rompu, et je ferai tomber l'épée de sa main.
\VS{23}Je disperserai les Egyptiens parmi les nations, et les répandrai parmi les pays.
\VS{24}Je fortifierai les bras du roi de Babylone, je lui mettrai mon épée dans la main ; mais je romprai les bras de Pharaon, et il gémira devant lui comme gémissent les mourants.
\VS{25}Je fortifierai donc les bras du roi de Babylone, mais les bras de Pharaon tomberont ; et on saura que je suis Yahweh, quand j'aurai mis mon épée dans la main du roi de Babylone, et qu'il l'a tournera contre le pays d'Egypte.
\VS{26}Je disperserai les Egyptiens parmi les nations, les répandrai parmi les pays ; et ils sauront que je suis Yahweh.
\Chap{31}
\TextTitle{Avertissement contre l'arrogance de Pharaon}
\VerseOne{}Il arriva aussi dans la onzième année, au premier jour du troisième mois, que la parole de Yahweh me fut adressée en ces mots :
\VS{2}Fils de l'homme, parle à Pharaon, roi d'Egypte, et à la multitude de son peuple : A qui ressembles-tu dans ta grandeur ?
\VS{3}Voici, le roi d'Assyrie a été comme un cèdre du Liban, ayant de belles branches, et des rameaux qui faisaient une grande ombre, et qui étaient d'une grande hauteur ; sa cime était fort touffue.
\VS{4}Les eaux l'ont fait croître, l'abîme l'a fait pousser en hauteur, ses fleuves ont coulé autour de ses plantes, et il a envoyé ses eaux abondantes vers tous les arbres des champs.
\VS{5}C'est pourquoi il s'est élevé au-dessus de tous les autres arbres des champs, ses branches se sont multipliées, et ses rameaux croissaient par les grandes eaux qui faisaient pousser ses branches.
\VS{6}Tous les oiseaux des cieux ont fait leurs nids dans ses branches, toutes les bêtes des champs ont fait leurs petits sous ses rameaux, et toutes les grandes nations ont habité sous son ombre.
\VS{7}Il était beau par sa grandeur, et par l'étendue de ses branches, parce que sa racine était sur de grandes eaux.
\VS{8}Les cèdres du jardin de Dieu ne le surpassaient point ; les cyprès n'égalaient point ses branches, et les platanes n'égalaient point comme ses rameaux ; aucun arbre du jardin de Dieu ne lui était comparable en beauté.
\VS{9}Je l'avais embelli par la multitude de ses rameaux, au point que tous les arbres d'Eden, qui étaient dans le jardin de Dieu, lui portaient envie.
\VS{10}C'est pourquoi le Seigneur Yahweh dit : Parce qu'il s'est élevé, parce qu'il lançait sa cime au milieu d'épais rameaux et que son cœur était fier de sa hauteur,
\VS{11}je l'ai livré entre les mains du plus fort des nations, qui l'a traité comme il fallait, et je l'ai chassé à cause de sa méchanceté.
\VS{12}Les étrangers les plus terrifiants parmi les nations l'ont coupé et l'ont laissé là, ses branches sont tombées sur les montagnes et sur toutes les vallées ; ses rameaux se sont rompus dans tous les ravins de la terre, et tous les peuples de la terre se sont retirés de dessous son ombre, et l'ont laissé là.
\VS{13}Tous les oiseaux des cieux se sont tenus sur ses ruines, et toutes les bêtes des champs se sont retirées vers ses rameaux,
\VS{14}afin que tous les arbres près des eaux n'élèvent plus leur hauteur, et qu'ils ne lancent plus leur cime au milieu d'épais rameaux, afin que tous les chênes arrosés d'eau ne gardent plus leur hauteur ; car tous sont livrés à la mort, aux profondeurs de la terre, parmi les fils des hommes, avec ceux qui descendent dans la fosse.
\VS{15}Ainsi parle le Seigneur Yahweh : Le jour qu'il descendit dans le scheol, j'ai répandu deuil sur lui, j'ai couvert l'abîme devant lui, j'ai empêché ses fleuves de couler, et les grosses eaux ont été retenues ; j'ai fait que le Liban soit en deuil à cause de lui, et tous les arbres des champs ont été desséchés.
\VS{16}J'ai ébranlé les nations par le bruit de sa ruine, quand je l'ai fait descendre dans le scheol, avec ceux qui descendent dans la fosse\FTNT{Es. 14:9.} ; et tous les arbres d'Eden, les plus beaux et les plus agréables du Liban, tous arrosés par les eaux, ont été consolés dans les profondeurs de la terre.
\VS{17}Eux aussi sont descendus avec lui dans le scheol, vers ceux qui ont péri par l'épée ; ils étaient son bras et ils habitaient sous son ombre parmi les nations.
\VS{18}A qui ressembles-tu ainsi en gloire et en grandeur parmi les arbres d'Eden ? Tu seras précipité avec les arbres d'Eden dans les profondeurs de la terre, tu seras gisant au milieu des incirconcis, avec ceux qui ont péri par l'épée. Voilà Pharaon et toute sa multitude ! dit le Seigneur Yahweh.
\Chap{32}
\TextTitle{Lamentation sur le pays d'Egypte}
\VerseOne{}Dans la douzième année, le premier jour du douzième mois, la parole de Yahweh me fut adressée en ces mots :
\VS{2}Fils de l'homme, prononce à haute voix une complainte sur Pharaon, roi d'Egypte, et dis-lui : Tu as été parmi les nations semblable à un lionceau, et comme un serpent dans les mers ; tu t'élançais dans tes fleuves, et tu troublais les eaux avec tes pieds, et remplissais de bourbe leurs fleuves.
\VS{3}Ainsi parle le Seigneur Yahweh : J'étendrai mon rets sur toi dans une assemblée nombreuse de peuples qui te tireront dans mes filets.
\VS{4}Je te laisserai à l'abandon sur la terre ; je te jetterai sur le dessus des champs, et je ferai demeurer sur toi tous les oiseaux des cieux, et rassasierai de toi les bêtes de toute la terre.
\VS{5}Car je mettrai ta chair sur les montagnes, et je remplirai les vallées de tes débris.
\VS{6}J'arroserai de ton sang jusqu'aux montagnes, la terre où tu nages, et les lits des eaux seront remplis de toi.
\VS{7}Quand je t'éteindrai, je couvrirai les cieux et j'obscurcirai leurs étoiles, je couvrirai le soleil de nuages, et la lune ne donnera plus sa lumière\FTNT{Es. 13:10 ; Joë. 2:31 ; Mt. 24:29.}.
\VS{8}J'obscurcirai à cause de toi tous les luminaires des cieux, et je répandrai les ténèbres sur ton pays, dit le Seigneur Yahweh.
\VS{9}J'affligerai le cœur de beaucoup de peuples, quand j'annoncerai ta ruine parmi les nations, à des pays que tu ne connaissais pas.
\VS{10}Je frapperai de stupeur beaucoup de peuples à cause de toi, et leurs rois seront saisis d'épouvante à cause de toi, quand je ferai luire mon épée à leurs yeux ; ils seront effrayés à chaque instant, chacun pour sa vie, au jour de ta ruine.
\VS{11}Car ainsi parle le Seigneur Yahweh : L'épée du roi de Babylone viendra sur toi.
\VS{12}J'abattrai ta multitude par les épées des hommes forts, qui tous sont les plus terribles d'entre les nations ; ils détruiront l'orgueil de l'Egypte, et toute la multitude de son peuple sera ruinée.
\VS{13}Je ferai périr tout son bétail près des grandes eaux, et aucun pied d'homme ne les troublera plus, ni aucun pied d'animaux ne les agitera plus.
\VS{14}Alors je rendrai profondes leurs eaux, et je ferai couler leurs fleuves comme de l'huile, dit le Seigneur Yahweh.
\VS{15}Quand j'aurai réduit le pays d'Egypte en désolation, et que le pays sera dénué des choses dont il était rempli ; quand je frapperai tous ceux qui y habitent, ils sauront alors que je suis Yahweh.
\VS{16}C'est ici la complainte qu'on fera sur elle, les filles des nations feront cette complainte sur elle ; elles feront cette complainte sur l'Egypte et sur toute la multitude de son peuple, dit le Seigneur.
\VS{17}Il arriva aussi dans la douzième année, le quinzième jour du mois, que la parole de Yahweh me fut adressée en ces mots :
\VS{18}Fils de l'homme, dresse une lamentation sur la multitude d'Egypte, et fais-la descendre, elle et les filles des nations magnifiques, aux plus bas lieux de la terre, avec ceux qui descendent dans la fosse\FTNT{Jé. 1:10 ; Jé. 18:7.}.
\VS{19}Qui surpasses-tu en beauté ? Descends, et couche-toi avec les incirconcis !
\VS{20}Ils tomberont au milieu de ceux qui seront tués par l'épée. L'épée a déjà été donnée : Entraînez l'Egypte et toute sa multitude !
\VS{21}Les plus forts d'entre les puissants lui parleront du milieu du scheol, avec ceux qui lui donnaient du secours, et diront : Ils sont descendus, ils sont couchés, les incirconcis, tués par l'épée.
\VS{22}Là est l'Assyrien, et toute son assemblée ; ses sépulcres sont autour de lui, eux tous, mis à mort, sont tombés par l'épée.
\VS{23}Car ses sépulcres sont posés au fond de la fosse et son assemblée autour de sa sépulture ; eux tous qui avaient répandu leur terreur sur la terre des vivants sont tombés morts par l'épée.
\VS{24}Là est Elam et toute sa multitude autour de son sépulcre ; eux tous sont tombés morts par l'épée, ils sont descendus incirconcis dans les plus bas lieux de la terre ; et après avoir répandu leur terreur sur la terre des vivants, ils ont porté leur ignominie avec ceux qui descendent dans la fosse.
\VS{25}On a mis sa couche parmi ceux qui ont été tués, avec toute sa multitude ; ses sépulcres sont autour de lui ; eux tous incirconcis, tués par l'épée, quoiqu'ils aient répandu leur terreur sur la terre des vivants, toutefois ils ont porté leur ignominie avec ceux qui descendent dans la fosse ; ils ont été placés parmi les morts.
\VS{26}Là est Méschec, Tubal, et toute leur multitude ; leurs sépulcres sont autour d'eux ; eux tous incirconcis, tués par l'épée, quoiqu'ils aient répandu leur terreur sur la terre des vivants.
\VS{27}Ils ne se sont point couchés avec les hommes vaillants qui sont tombés d'entre les incirconcis, lesquels sont descendus dans le scheol avec leurs armes de guerre, dont on a mis les épées sous leurs têtes, et dont les iniquités ont reposé sur leurs os ; parce que la terreur des hommes forts est dans la terre des vivants.
\VS{28}Toi aussi tu seras brisé au milieu des incirconcis, et tu seras couché avec ceux qui sont tués par l'épée.
\VS{29}Là est Edom, ses rois, et tous ses princes, qui ont été placés malgré leur force avec ceux qui sont tués par l'épée ; ils seront couchés avec les incirconcis, et avec ceux qui sont descendus dans la fosse.
\VS{30}Là sont tous les princes du nord, et tous les Sidoniens, qui sont descendus avec ceux qui sont tués, malgré la terreur qu'inspirait leur force ; ils sont couchés incirconcis avec ceux qui sont tués par l'épée ; ils ont porté leur ignominie avec ceux qui sont descendus dans la fosse.
\VS{31}Pharaon les verra, et il se consolera au sujet de toute la multitude de son peuple ; Pharaon, dit le Seigneur Yahweh, verra les blessés par l'épée et toute son armée.
\VS{32}Car je mettrai ma terreur dans la terre des vivants, c'est pourquoi Pharaon avec toute la multitude de son peuple se couchera au milieu des incirconcis, avec ceux qui sont tués par l'épée, dit le Seigneur Yahweh.
\Chap{33}
\TextTitle{Ezéchiel établi comme sentinelle pour avertir le pécheur}
\VerseOne{}La parole de Yahweh me fut encore adressée en ces mots :
\VS{2}Fils de l'homme, parle aux fils de ton peuple, et dis-leur : Quand je ferai venir l'épée sur un pays, et que le peuple du pays aura choisi quelqu'un d'entre eux, et l'aura établi pour leur servir de sentinelle,
\VS{3}et que voyant venir l'épée sur le pays, il sonnera du shofar et avertira le peuple,
\VS{4}si le peuple ayant bien entendu le son du shofar, ne se tient pas sur ses gardes, et qu'ensuite l'épée vienne le prendre, son sang sera sur sa tête.
\VS{5}Car il a entendu le son du shofar, et ne s'est point tenu sur ses gardes ; son sang sera sur lui ; mais s'il se tient sur ses gardes, il sauvera sa vie.
\VS{6}Si la sentinelle voit venir l'épée, et qu'elle ne sonne point du shofar, en sorte que le peuple ne se tienne point sur ses gardes, et qu'ensuite l'épée survienne et ôte la vie à l'un d'entre eux, celui-ci sera emmené en captivité à cause de son iniquité, mais je redemanderai son sang de la main de la sentinelle.
\VS{7}Toi donc, fils de l'homme, je t'ai établi pour sentinelle sur la maison d'Israël ; tu écouteras donc la parole qui sort de ma bouche, et tu les avertiras de ma part.
\VS{8}Quand j'aurai dit au méchant : Méchant, tu mourras ! et que tu n'auras point parlé au méchant pour l'avertir de se détourner de sa voie, ce méchant mourra dans son iniquité ; mais je redemanderai son sang de ta main.
\VS{9}Mais si tu as averti le méchant de se détourner de sa voie, et qu'il ne se détourne pas de sa voie, il mourra dans son iniquité ; mais toi tu auras délivré ton âme.
\VS{10}Toi donc, fils de l'homme, dis à la maison d'Israël : Vous avez parlé ainsi, en disant : Puisque nos crimes et nos péchés sont sur nous, et que nous périssons à cause d'eux, comment pourrions-nous vivre\FTNT{Lé. 26:39.} ?
\VS{11}Dis-leur : Je suis vivant, dit le Seigneur Yahweh, je ne prends point plaisir dans la mort du méchant, mais que le méchant se détourne de sa voie et qu'il vive. Détournez-vous, détournez-vous de votre méchante voie ! Pourquoi mourriez-vous, maison d'Israël ?
\VS{12}Toi donc, fils de l'homme, dis aux fils de ton peuple : La justice du juste ne le délivrera point au jour de son péché, le méchant ne tombera point par sa méchanceté au jour où il s'en détournera ; et le juste ne pourra pas vivre par sa justice au jour de son péché.
\VS{13}Quand j'aurai dit au juste qu'il vivra certainement, et que lui, se confiant sur sa justice, aura commis l'iniquité, on ne se souviendra plus d'aucune de ses justices, mais il mourra dans son iniquité qu'il aura commise.
\VS{14}Aussi quand j'aurai dit au méchant : Tu mourras ! s'il se détourne de son péché, et qu'il fasse ce qui est juste et droit ;
\VS{15}si le méchant rend le gage et qu'il restitue ce qu'il aura ravi, et qu'il marche dans les statuts de la vie, sans commettre d'iniquité, certainement il vivra, il ne mourra point.
\VS{16}On ne se souviendra plus des péchés qu'il aura commis ; il a fait ce qui est juste et droit ; certainement il vivra.
\VS{17}Or les fils de ton peuple ont dit : La voie du Seigneur n'est pas bien réglée ; mais c'est plutôt leur voie qui n'est pas bien réglée.
\VS{18}Quand le juste se détournera de sa justice, et qu'il commettra l'iniquité, il mourra à cause de cela.
\VS{19}Quand le méchant se détournera de sa méchanceté, et qu'il fera ce qui est juste et droit, il vivra à cause de cela.
\VS{20}Vous avez dit : La voie du Seigneur n'est pas bien réglée ! Je vous jugerai, maison d'Israël, chacun selon sa voie.
\TextTitle{Exécution du jugement de Yahweh}
\VS{21}Or il arriva dans la douzième année de notre captivité, au cinquième jour du dixième mois, qu'un homme qui s'était échappé de Jérusalem vint vers moi, en disant : La ville est prise !
\VS{22}La main de Yahweh fut sur moi le soir, avant l'arrivée du fugitif, et Yahweh ouvrit ma bouche lorsqu'il vint auprès de moi le matin. Ma bouche était ouverte et je n'étais plus muet.
\TextTitle{Ne pas se contenter d'écouter la Parole de Dieu}
\VS{23}La parole de Yahweh me fut adressée en ces mots :
\VS{24}Fils de l'homme, ceux qui habitent dans ces ruines, sur la terre d'Israël, discourent en disant : Abraham était seul, et il a possédé le pays\FTNT{Ge. 15:7.}; mais nous sommes un grand nombre de gens, et le pays nous a été donné en héritage.
\VS{25}C'est pourquoi tu leur diras : Ainsi parle le Seigneur Yahweh : Vous mangez la chair avec le sang, et vous levez vos yeux vers vos idoles, vous répandez le sang ; et vous posséderiez le pays\FTNT{Ge. 9:4 ; Lé. 3:17 ; Lé. 17:10.} ?
\VS{26}Vous vous appuyez sur votre épée, vous commettez des abominations, vous souillez chacun de vous la femme de son prochain ; et vous posséderiez le pays ?
\VS{27}Tu leur diras : Ainsi parle le Seigneur Yahweh : Je suis vivant, ceux qui sont dans ces ruines tomberont par l'épée, et je livrerai aux bêtes celui qui est dans les champs, afin qu'elles le mangent ; et ceux qui sont dans les forteresses et dans les cavernes mourront par la peste.
\VS{28}Ainsi je réduirai le pays en désolation et en désert, l'orgueil de sa force sera aboli, et les montagnes d'Israël seront désolées, en sorte qu'il n'y passera plus personne.
\VS{29}Ils sauront que je suis Yahweh, quand j'aurai réduit leur pays en désolation et en désert, à cause de toutes leurs abominations qu'ils ont commises.
\VS{30}Quant à toi, fils de l'homme, les fils de ton peuple parlent de toi près des murs et aux entrées des maisons, et parlent l'un à l'autre, chacun avec son prochain, en disant : Venez maintenant, et écoutez la parole qui vient de Yahweh.
\VS{31}Ils viennent vers toi en foule, et mon peuple s'assied devant toi, ils écoutent tes paroles, mais ils ne les mettent point en pratique ; ils les répètent comme si c'était une chanson profane, mais leur cœur marche toujours après leur gain déshonnête.
\VS{32}Voici tu es pour eux comme un homme qui leur chante une chanson profane avec une belle voix, qui résonne bien ; car ils écoutent bien tes paroles, mais ils ne les mettent point en pratique.
\VS{33}Mais quand ces choses arriveront, et voici, elles arrivent, ils sauront qu'il y avait un prophète au milieu d'eux.
\Chap{34}
\TextTitle{Jugement de Dieu sur les faux bergers}
\VerseOne{}La parole de Yahweh me fut encore adressée en ces mots :
\VS{2}Fils de l'homme, prophétise contre les pasteurs d'Israël\FTNT{Les faux pasteurs prennent en otage les brebis du Seigneur (Jé. 23). La véritable fonction pastorale consiste au service envers les frères et sœurs et non le contraire. Un vrai pasteur sert les autres, il n'aime pas être servi comme un roi. Il ne dit pas aux autres de faire les choses, mais il les fait et les autres l'imitent (Jn. 10).} ! Prophétise, et dis à ces pasteurs : Ainsi parle le Seigneur Yahweh : Malheur aux pasteurs d'Israël ! Qui ne paissent qu'eux-mêmes ! Les pasteurs ne paissent-ils pas le troupeau ?
\VS{3}Vous en mangez la graisse, et vous vous habillez de laine ; vous tuez ce qui est gras, vous ne paissez point le troupeau !
\VS{4}Vous n'avez point fortifié les brebis languissantes, vous n'avez point donné de remède à celle qui était malade, vous n'avez point bandé la plaie de celle qui avait la jambe rompue, vous n'avez point ramené celle qui était chassée, et vous n'avez point cherché celle qui était perdue\FTNT{Lu. 15:4-6 ; 1 Pi. 5:1-3.} ; mais vous les avez maîtrisées avec dureté et rigueur.
\VS{5}Elles se sont dispersées, parce qu'elles n'avaient pas de pasteurs, et elles se sont exposées à toutes les bêtes des champs, pour en être dévorées, étant dispersées.
\VS{6}Mes brebis sont errantes sur toutes les montagnes, et sur toutes les collines élevées ; mes brebis sont dispersées sur toute la surface de la terre ; et il n'y a personne qui les recherche, et il n'y a personne pour s'en soucier\FTNT{Mc. 14:27 ; Za. 13:7 ; Mt. 26:31.}.
\VS{7}C'est pourquoi pasteurs, écoutez la parole de Yahweh :
\VS{8}Je suis vivant, dit le Seigneur Yahweh, parce que mes brebis sont pillées, et que mes brebis sont la nourriture de toutes les bêtes des champs, parce qu'elles n'ont point de pasteur ; car mes pasteurs n'ont point recherché mes brebis, mais les pasteurs se sont nourris simplement eux-mêmes, et n'ont point fait paître mes brebis.
\VS{9}C'est pourquoi pasteurs, écoutez la parole de Yahweh !
\VS{10}Ainsi parle le Seigneur Yahweh : Voici, j'en veux à ces pasteurs-là, et je redemanderai mes brebis de leur main ; ils cesseront de paître les brebis, et les pasteurs ne se repaîtront plus eux-mêmes, mais je délivrerai mes brebis de leur bouche, et elles ne seront plus dévorées par eux.
\TextTitle{Yahweh, le bon berger qui restaure son troupeau\FTNT{Jn. 10:1-18}}
\VS{11}Car ainsi parle le Seigneur Yahweh : Me voici, je redemanderai mes brebis, et je les rechercherai.
\VS{12}Comme le pasteur prend soin de son troupeau quand il est au milieu de ses brebis dispersées, ainsi je rechercherai mes brebis, et les retirerai de tous les lieux où elles auront été dispersées au jour des nuages et de l'obscurité.
\VS{13}Je les retirerai d'entre les peuples et les rassemblerai des territoires, les ramènerai dans leur terre, et les nourrirai sur les montagnes d'Israël, auprès des cours d'eau et dans toutes les demeures du pays.
\VS{14}Je les paîtrai dans de bons pâturages, et leur demeure sera sur les hautes montagnes d'Israël ; et là elles coucheront dans une agréable demeure, et paîtront dans de gras pâturages, sur les montagnes d'Israël.
\VS{15}Moi-même je paîtrai mes brebis et les ferai reposer, dit le Seigneur Yahweh\FTNT{Ps. 23.}.
\VS{16}Je chercherai celle qui était perdue, et je ramènerai celle qui était chassée, je banderai la plaie de celle qui a la jambe rompue, et je fortifierai celle qui est malade ; mais je détruirai la grasse et la forte ; je les paîtrai avec justice.
\VS{17}Quant à vous, mes brebis, ainsi parle le Seigneur Yahweh : Voici, je m'en vais mettre à part les brebis, les béliers, et les boucs.
\VS{18}Et vous, est-ce peu de chose de vous faire paître dans de bons pâturages, pour que vous fouliez de vos pieds le reste de votre pâture ? Et de boire des eaux claires, pour que vous troubliez le reste avec vos pieds ?
\VS{19}Mais mes brebis sont nourries du pâturage que vous foulez de vos pieds, et boivent ce que vos pieds ont troublé.
\VS{20}C'est pourquoi le Seigneur Yahweh leur dit : Me voici, je mettrai moi-même à part la brebis grasse et la brebis maigre.
\VS{21}Parce que vous poussez du côté et de l'épaule, et que vous heurtez de vos cornes toutes celles qui sont languissantes, jusqu'à ce que vous les ayez chassées dehors,
\VS{22}je sauverai mes brebis, au point qu'elles ne seront plus au pillage. Voici, je jugerai entre brebis et brebis.
\VS{23}Je susciterai sur elles un pasteur qui les paîtra, mon serviteur David ; il les paîtra, et lui-même sera leur pasteur.
\VS{24}Moi, Yahweh, je serai leur Dieu, et mon serviteur David sera prince au milieu d'elles ; moi,Yahweh, j'ai parlé.
\VS{25}Je traiterai avec elles une alliance de paix ; et je détruirai dans le pays les mauvaises bêtes ; les brebis habiteront dans le désert en sécurité, et dormiront dans les forêts.
\VS{26}Je les comblerai de bénédictions, elles, et tous les environs de mes collines ; je ferai tomber la pluie en sa saison ; ce seront des pluies de bénédiction.
\VS{27}Les arbres des champs produiront leur fruit, et la terre rapportera son revenu ; elles seront dans leur terre en sécurité, et sauront que je suis Yahweh, quand j'aurai rompu les bois de leur joug, et que je les aurai délivrées de la main de ceux qui se les asservissent.
\VS{28}Elles ne seront plus au pillage parmi les nations, et les bêtes de la terre ne les dévoreront plus ; mais elles habiteront en sécurité, et il n'y aura personne pour les effrayer.
\VS{29}Je leur susciterai une plantation de renom ; elles ne mourront plus de faim sur la terre, et ne porteront plus l'opprobre des nations.
\VS{30}Ils sauront que moi, Yahweh, leur Dieu, suis avec eux, et qu'eux, la maison d'Israël, sont mon peuple, dit le Seigneur Yahweh.
\VS{31}Or vous êtes mes brebis, vous hommes, les brebis de mon pâturage, et je suis votre Dieu, dit le Seigneur Yahweh.
\Chap{35}
\TextTitle{Jugement sur Edom}
\VerseOne{}La parole de Yahweh me fut encore adressée en ces mots :
\VS{2}Fils de l'homme, tourne ta face contre la montagne de Séir, et prophétise contre elle\FTNT{Am. 1:11.}.
\VS{3}Dis-lui : Ainsi parle le Seigneur Yahweh : Voici, j'en veux à toi, montagne de Séir, et j'étendrai ma main contre toi, et te réduirai en désolation et en désert.
\VS{4}Je réduirai tes villes en désert, tu ne seras que désolation, et tu sauras que je suis Yahweh.
\VS{5}Parce que tu as eu une inimitié immortelle, et que tu as fait couler le sang des fils d'Israël à coups d'épée, au temps de leur détresse, au temps où l'iniquité était à son terme\FTNT{Ps. 137:7.}.
\VS{6}C'est pourquoi je suis vivant, dit le Seigneur Yahweh, je te mettrai à sang, et le sang te poursuivra ; parce que tu n'as point haï le sang, le sang aussi te poursuivra.
\VS{7}Je réduirai la montagne de Séir en désolation et en désert, et j'en éloignerai tous ceux qui la fréquentaient.
\VS{8}Je remplirai de morts ses montagnes ; tes hommes tués par l'épée tomberont sur tes collines, dans tes vallées, et dans tous tes courants d'eau.
\VS{9}Je te réduirai en désolations éternelles, et tes villes ne seront plus habitées ; vous saurez que je suis Yahweh.
\VS{10}Parce que tu as dit : Les deux nations, et les deux pays seront à moi, nous les posséderons, quand même Yahweh était là ;
\VS{11}à cause de cela, je suis vivant, dit le Seigneur Yahweh, j'agirai avec la colère et la jalousie que tu as montrées dans ta haine contre eux ; et je me ferai connaître au milieu d'eux, quand je te jugerai.
\VS{12}Tu sauras que moi, Yahweh, j'ai entendu toutes les paroles insultantes que tu as prononcées contre les montagnes d'Israël, en disant : Elles sont dévastées, elles nous sont livrées comme une proie.
\VS{13}Vous m'avez bravé par vos discours, et vous avez multiplié vos paroles contre moi ; je l'ai entendu.
\VS{14}Ainsi parle le Seigneur Yahweh : Quand toute la terre se réjouira, je te réduirai en désolation.
\VS{15}Comme tu t'es réjouie sur l'héritage de la maison d'Israël et de sa désolation, j'en ferai de même envers toi ; tu ne seras que désolation, ô montagne de Séir ! Ainsi qu'Edom tout entier ; et ils sauront que je suis Yahweh\FTNT{Ab. 1:11-16.}.
\Chap{36}
\TextTitle{Yahweh rétablit Israël}
\VerseOne{}Toi, fils de l'homme, prophétise sur les montagnes d'Israël, et dis : Montagnes d'Israël, écoutez la parole de Yahweh !
\VS{2}Ainsi parle le Seigneur Yahweh : Parce que l'ennemi a dit contre vous : Ah ! Ah ! Tous ces hauts lieux éternels sont devenus notre possession !
\VS{3}Prophétise, et dis : Ainsi parle le Seigneur Yahweh : Oui, parce qu'on vous a réduites en désolation, et que de toutes parts, on vous a englouties pour que vous soyez la propriété des autres nations, et qu'on vous a exposées à la langue et aux insultes des nations,
\VS{4}à cause de cela, montagnes d'Israël, écoutez la parole du Seigneur Yahweh : Ainsi parle le Seigneur Yahweh, aux montagnes, aux collines, aux courants d'eau, aux vallées, aux lieux détruits et désolés, et aux villes abandonnées qui sont pillées et sont un sujet de moquerie aux autres nations d'alentour ;
\VS{5}à cause de cela, ainsi parle le Seigneur Yahweh : Je parle dans le feu de ma jalousie contre les autres nations, et contre tous ceux d'Edom qui se sont attribués ma terre en possession, avec toute la joie de leur cœur et le mépris de leur âme, afin d'en piller le butin\FTNT{Lé. 25:23 ; Es. 14:2 ; Jé. 2:7.}.
\VS{6}C'est pourquoi prophétise sur la terre d'Israël, et dis aux montagnes et aux collines, aux courants d'eau et aux vallées : Ainsi parle le Seigneur Yahweh : Voici, j'ai parlé avec jalousie, et avec fureur, parce que vous avez porté l'ignominie des nations.
\VS{7}C'est pourquoi ainsi parle le Seigneur Yahweh : J'ai levé ma main, si les nations qui sont tout autour de vous ne portent leur ignominie.
\VS{8}Mais vous, montagnes d'Israël, vous pousserez vos branches, et vous porterez votre fruit pour mon peuple d'Israël ; car ils sont prêts à venir.
\VS{9}Car me voici, je viens à vous, et je retournerai vers vous, et vous serez labourées et semées.
\VS{10}Je mettrai sur vous des hommes en grand nombre, la maison d'Israël tout entière, et les villes seront habitées, les lieux déserts seront rebâtis.
\VS{11}Je multiplierai sur vous les hommes et les animaux, ils multiplieront et seront féconds ; je veux que vous soyez habitées comme auparavant, et je vous ferai plus de bien que vous n'en avez eu au commencement ; et vous saurez que je suis Yahweh.
\VS{12}Je ferai marcher sur vous des hommes, mon peuple d'Israël, qui vous posséderont, vous serez leur héritage, et vous ne les consumerez plus.
\VS{13}Ainsi parle le Seigneur Yahweh : Parce qu'on dit de vous : Tu es un pays qui dévore les hommes, et tu as consumé tes habitants ;
\VS{14}à cause de cela, tu ne dévoreras plus les hommes et ne consumeras plus tes habitants, dit le Seigneur Yahweh.
\VS{15}Je ne te ferai plus entendre l'ignominie des nations, tu ne porteras plus l'opprobre des peuples ; et tu ne feras plus périr tes habitants, dit le Seigneur Yahweh.
\VS{16}Puis la parole de Yahweh me fut adressée en ces mots :
\VS{17}Fils de l'homme, ceux de la maison d'Israël habitant sur leur terre l'ont souillée par leur voie et par leurs actions ; leur voie est devenue devant moi comme la souillure d'une femme pendant son impureté\FTNT{Lé. 12:2 ; Lé. 15:19.} ;
\VS{18}j'ai répandu ma fureur sur eux à cause du sang qu'ils ont répandu sur le pays, et parce qu'ils l'ont souillé par leurs idoles.
\VS{19}Je les ai dispersés parmi les nations, et ils ont été disséminés en divers pays ; je les ai jugés selon leur voie, et selon leurs actions.
\VS{20}Ils sont arrivés chez les nations où ils allaient, ils ont profané mon saint Nom en sorte qu'on disait d'eux : Ceux-ci sont le peuple de Yahweh, c'est de son pays qu'ils sont sortis\FTNT{Ro. 2:24.}.
\VS{21}Mais j'ai épargné mon saint Nom, que la maison d'Israël avait profané parmi les nations où elle est allée.
\VS{22}C'est pourquoi dis à la maison d'Israël : Ainsi parle le Seigneur Yahweh : Je ne le fais point à cause de vous, ô maison d'Israël ! Mais à cause de mon saint Nom, que vous avez profané parmi les nations où vous êtes allés\FTNT{De. 7:7 ; De. 9:5 ; Ps. 25:11 ; Es. 43:25.}.
\VS{23}Je sanctifierai mon grand nom, qui a été profané parmi les nations, et que vous avez profané au milieu d'elles ; et les nations sauront que je suis Yahweh, dit le Seigneur Yahweh, quand je serai sanctifié par vous, sous leurs yeux.
\VS{24}Je vous retirerai d'entre les nations, je vous rassemblerai de tous les pays, et je vous ramènerai dans votre terre.
\VS{25}Je répandrai sur vous une eau pure\FTNT{Il est question ici de la Nouvelle Alliance (Jé. 31:31-34 ; Hé. 8:7-13).}, et vous serez nettoyés ; je vous nettoierai de toutes vos souillures et de toutes vos idoles.
\TextTitle{Prophétie sur la naissance d'en haut}
\VS{26}Je vous donnerai un nouveau cœur, je mettrai au dedans de vous un Esprit nouveau ; j'ôterai de votre chair le cœur de pierre, et je vous donnerai un cœur de chair\FTNT{Jé. 32:39 ; 2 Co. 3:3 ; Ez. 11:19.}.
\VS{27}Je mettrai mon Esprit au dedans de vous, je ferai en sorte que vous suiviez mes ordonnances, et que vous observiez et pratiquiez mes lois.
\VS{28}Vous habiterez le pays que j'ai donné à vos pères, vous serez mon peuple, et je serai votre Dieu.
\VS{29}Je vous délivrerai de toutes vos souillures, j'appellerai le blé, je le multiplierai, et je ne vous enverrai plus la famine.
\VS{30}Je multiplierai le fruit des arbres et le revenu des champs, afin que vous ne portiez plus l'opprobre de la famine parmi les nations.
\VS{31}Vous vous souviendrez de votre mauvaise voie et de vos actions, qui n'étaient pas bonnes, et vous prendrez vous-mêmes en dégoût vos iniquités et vos abominations.
\VS{32}Je ne le fais point par amour pour vous, dit le Seigneur Yahweh ; sachez-le ! Soyez honteux et confus à cause de votre voie, ô maison d'Israël !
\VS{33}Ainsi parle le Seigneur Yahweh : Le jour où je vous aurai purifiés de toutes vos iniquités, je vous ferai habiter dans des villes, et les lieux déserts seront rebâtis.
\VS{34}La terre désolée sera cultivée, tandis qu'elle n'était que désolation aux yeux de tous les passants.
\VS{35}On dira : Cette terre-ci qui était désolée est devenue comme le jardin d'Eden ; et ces villes qui étaient désertes, désolées, et détruites, sont fortifiées et habitées\FTNT{Jé. 22:8-9 ; Es. 33:20.}.
\VS{36}Les nations qui resteront autour de vous sauront que moi, Yahweh, j'ai rebâti les lieux détruits et planté le pays désolé ; moi, Yahweh, j'ai parlé, et je le ferai.
\VS{37}Ainsi parle le Seigneur Yahweh : Je me laisserai rechercher par la maison d'Israël. Voici ce que je ferai pour eux : Je multiplierai les hommes comme un troupeau de brebis.
\VS{38}Les villes qui sont désertes seront remplies de troupeaux d'hommes, pareils aux troupeaux consacrés, aux troupeaux qu'on amène à Jérusalem pendant ses fêtes solennelles ; et ils sauront que je suis Yahweh.
\Chap{37}
\TextTitle{Vision des ossements desséchés, image de la restauration d'Israël}
\VerseOne{}La main de Yahweh fut sur moi, et Yahweh me transporta par son Esprit et me déposa au milieu d'une vallée remplie d'ossements\FTNT{Les ossements desséchés représentent les Israélites dispersés dans les nations.}.
\VS{2}Il me fit passer auprès d'eux, tout autour ; et voici, ils étaient fort nombreux à la surface de cette vallée et complètement secs.
\VS{3}Puis il me dit : Fils de l'homme, ces os pourront-ils revivre ? Et je répondis : Seigneur Yahweh, tu le sais.
\VS{4}Alors il me dit : Prophétise sur ces os, et dis-leur : Ossements desséchés, écoutez la parole de Yahweh !
\VS{5}Ainsi parle le Seigneur Yahweh à ces os : Voici, je ferai entrer un esprit en vous, et vous vivrez\FTNT{Ro. 8:11 ; Ps. 71:20.};
\VS{6}je mettrai des nerfs sur vous, je ferai croître de la chair sur vous, et j'étendrai la peau sur vous ; puis je mettrai un esprit en vous, et vous vivrez. Et vous saurez que je suis Yahweh.
\VS{7}Alors je prophétisai selon l'ordre que j'avais reçu. Et comme je prophétisais, il se fit un bruit, et voici, il se fit un mouvement, et ces os s'approchèrent les uns des autres.
\VS{8}Puis je regardai, et voici, il vint des nerfs sur eux, et il y crût de la chair, la peau fut étendue par dessus ; mais il n'y avait pas en eux d'esprit.
\VS{9}Alors il me dit : Prophétise à l'Esprit ! Prophétise, fils de l'homme ! Et dis à l'Esprit : Ainsi parle le Seigneur Yahweh : Esprit, viens des quatre vents, et souffle sur ces morts, et qu'ils revivent !
\VS{10}Je prophétisai donc selon l'ordre qu'il m'avait donné. Et l'Esprit entra en eux, ils reprirent vie, et se tinrent sur leurs pieds ; c'était une armée extrêmement grande.
\VS{11}Alors il me dit : Fils de l'homme, ces os sont toute la maison d'Israël ; voici, ils disent : Nos os sont desséchés, et notre attente est perdue, c'en est fait de nous !
\VS{12}C'est pourquoi prophétise, et dis-leur : Ainsi parle le Seigneur Yahweh : Mon peuple, voici, je m'en vais ouvrir vos sépulcres, je vous tirerai hors de vos sépulcres, et vous ferai entrer dans la terre d'Israël\FTNT{Les sépulcres représentent les nations dans lesquelles les Israélites se sont établis. Dieu annonce le retour de son peuple sur la terre d'Israël. Es. 26:19 ; Os. 13:14.}.
\VS{13}Et vous, mon peuple, vous saurez que je suis Yahweh quand j'aurai ouvert vos sépulcres, et que je vous aurai tirés hors de vos sépulcres.
\VS{14}Je mettrai mon Esprit en vous, et vous vivrez, je vous rétablirai sur votre terre ; et vous saurez que moi, Yahweh, j'ai parlé et que je l'ai fait, dit Yahweh.
\TextTitle{Prophétie sur l'unité d'Israël}
\VS{15}Puis la parole de Yahweh me fut adressée en ces mots :
\VS{16}Et toi, fils de l'homme, prends un bois et écris dessus : Pour Juda, et pour les fils d'Israël ses compagnons. Prends encore un autre bois, et écris dessus : Le bois d'Ephraïm et de toute la maison d'Israël, ses compagnons, pour Joseph.
\VS{17}Puis tu les joindras l'un à l'autre pour ne former qu'un même bois, ils seront unis dans ta main.
\VS{18}Quand les fils de ton peuple demanderont, en disant : Ne nous déclareras-tu pas ce que tu veux dire par ces choses ?
\VS{19}Dis-leur : Ainsi parle le Seigneur Yahweh : Voici, je m'en vais prendre le bois de Joseph qui est dans la main d'Ephraïm, et des tribus d'Israël, ses compagnons ; je les joindrai au bois de Juda, et j'en formerai un seul bois, ils ne seront qu'un seul bois dans ma main.
\VS{20}Ainsi les bois sur lesquels tu écriras seront dans ta main, sous leurs yeux.
\VS{21}Dis-leur : Ainsi parle le Seigneur Yahweh : Voici, je m'en vais prendre les fils d'Israël d'entre les nations parmi lesquelles ils sont allés, je les rassemblerai de toutes parts, et je les ferai entrer dans leur terre.
\VS{22}Je ferai d'eux une seule nation dans le pays, sur les montagnes d'Israël ; un seul roi sera leur roi à tous, ils ne seront plus deux nations, et ils ne seront plus divisés en deux royaumes\FTNT{Os. 2:2 ; Es. 11:12-13 ; Jn. 10:16.}.
\VS{23}Ils ne se souilleront plus par leurs idoles, ni par leurs infamies, ni par tous leurs crimes, et je les retirerai de toutes leurs demeures dans lesquelles ils ont péché, et je les purifierai ; ils seront mon peuple, et je serai leur Dieu\FTNT{Es. 1:18 ; Jé. 33:8 ; Jé. 24:7 ; Jé. 32:38 ; Za. 8:8 ; 2 Co. 6:16.}.
\VS{24}David, mon serviteur, sera leur roi, et ils auront tous un seul pasteur ; ils suivront mes ordonnances, ils garderont mes lois et les mettront en pratique.
\VS{25}Ils habiteront dans le pays que j'ai donné à Jacob, mon serviteur, dans lequel vos pères ont habité ; ils y habiteront, dis-je, eux, et leurs fils, et les fils de leurs fils, pour toujours ; et David mon serviteur sera leur prince pour toujours.
\VS{26}Je traiterai avec eux une alliance de paix, et il y aura une alliance éternelle avec eux ; je les établirai, et les multiplierai, je mettrai mon lieu saint au milieu d'eux pour toujours.
\VS{27}Ma demeure sera parmi eux ; je serai leur Dieu, et ils seront mon peuple.
\VS{28}Les nations sauront que je suis Yahweh qui sanctifie Israël, quand mon lieu saint sera au milieu d'eux pour toujours.
\Chap{38}
\TextTitle{Jugement sur Gog}
\VerseOne{}La parole de Yahweh me fut encore adressée en ces mots :
\VS{2}Fils de l'homme, tourne ta face vers Gog au pays de Magog\FTNT{Gog est un prince et Magog le pays. Ce chapitre doit être mis en parallèle avec Za. 12:1-4 ; Za. 14:1-9 ; Mt. 24:14-30 ; Ap. 14:14-20 ; Ap. 20:8.}, vers le prince de Rosch, de Méschec et de Tubal, et prophétise contre lui !
\VS{3}Tu diras : Ainsi parle le Seigneur Yahweh : Voici, j'en veux à toi, Gog, prince des chefs de Méschec et de Tubal !
\VS{4}Je te ferai retourner en arrière, et je mettrai des boucles dans tes mâchoires, et te ferai sortir avec toute ton armée, avec les chevaux, et les cavaliers, tous parfaitement bien équipés, une grande multitude portant le grand et le petit bouclier, et tous maniant l'épée ;
\VS{5}ceux de Perse, d'Ethiopie, et de Puth avec eux, qui tous ont des boucliers et des casques.
\VS{6}Gomer et toutes ses troupes, la maison de Togarma à l'extrême nord, avec toutes ses troupes, et plusieurs peuples avec toi.
\VS{7}Apprête-toi, tiens-toi prêt, toi, et toute la multitude assemblée autour de toi ! Sois leur chef !
\VS{8}Après plusieurs jours, tu seras à leur tête, et dans la suite des années, tu marcheras contre le pays dont les habitants, délivrés de l'épée, auront été rassemblés d'entre plusieurs peuples sur les montagnes d'Israël longtemps désertes ; retirés du milieu des peuples, ils seront en sécurité dans leurs demeures.
\VS{9}Tu monteras, tu viendras comme une dévastation, tu seras comme une nuée pour couvrir la terre, toi, toutes tes troupes, et plusieurs peuples avec toi\FTNT{Da. 11:40.}.
\VS{10}Ainsi parle le Seigneur Yahweh : Il arrivera dans ces jours-là que des pensées s'élèveront dans ton cœur, et que tu formeras un dessein pernicieux.
\VS{11}Car tu diras : Je monterai contre le pays dont les villes sont sans murailles ; j'envahirai ceux qui sont en repos, qui habitent en sécurité, qui demeurent tous dans des villes sans murs, lesquelles n'ont ni barres ni portes\FTNT{Jé. 49:31.} ;
\VS{12}pour enlever un grand butin et faire un grand pillage ; pour remettre ta main sur les déserts qui de nouveau étaient habités, et sur le peuple rassemblé d'entre les nations, ayant des troupeaux et des biens, et occupant les lieux élevés du pays.
\VS{13}Séba, et Dedan, les marchands de Tarsis, et tous ses lionceaux, te diront : Ne vas-tu pas pour faire du butin, et n'as-tu pas assemblé ta multitude pour faire un grand pillage, pour emporter de l'argent et de l'or, pour prendre le bétail et les biens, pour enlever un grand butin ?
\VS{14}Toi donc, fils de l'homme, prophétise, et dis à Gog : Ainsi parle le Seigneur Yahweh : En ce jour-là, quand mon peuple d'Israël habitera en sécurité, ne le sauras-tu pas ?
\VS{15}Ne viendras-tu pas de ton lieu, de l'extrême nord, toi, et plusieurs peuples avec toi, tous montés sur des chevaux, une grande multitude, et une grosse armée ?
\VS{16}Ne monteras-tu pas contre mon peuple d'Israël, comme une nuée pour couvrir la terre ? Dans la suite de ces jours, je te ferai venir sur ma terre, afin que les nations me connaissent, quand je serai sanctifié par toi sous leurs yeux, ô Gog !
\VS{17}Ainsi parle le Seigneur Yahweh : N'est-ce pas de toi que j'ai parlé autrefois par le ministère de mes serviteurs, les prophètes d'Israël, qui ont prophétisé dans ces jours-là pendant plusieurs années, qu'on te ferait venir contre eux ?
\VS{18}Mais il arrivera dans ce jour-là, au jour de la venue de Gog sur la terre d'Israël, dit le Seigneur Yahweh, que ma colère éclatera.
\VS{19}Je le déclare, dans ma jalousie, dans l'ardeur de ma fureur, en ce jour-là, il y aura une grande agitation sur la terre d'Israël.
\VS{20}Les poissons de la mer, les oiseaux des cieux, et les bêtes des champs, et tous les reptiles qui rampent sur la terre, et tous les hommes qui sont sur la surface de la terre seront épouvantés par ma présence ; les montagnes seront renversées, les parois des rochers tomberont, et tous les murs chuteront par terre.
\VS{21}J'appellerai contre lui l'épée sur toutes mes montagnes, dit le Seigneur Yahweh ; l'épée de chacun d'eux sera contre son frère.
\VS{22}J'entrerai en jugement avec lui par la peste, et par le sang ; je ferai pleuvoir sur lui, sur ses troupes, et sur les grands peuples qui seront avec lui, des torrents d'eau, des pierres de grêle, du feu et du soufre\FTNT{Ap. 8:7 ; Ps. 11:6 ; Ap. 16:21 ; Ap. 11:19.}.
\VS{23}Je me glorifierai, je me sanctifierai, je serai connu aux yeux de plusieurs nations ; et elles sauront que je suis Yahweh.
\Chap{39}
\TextTitle{Jugement sur Gog, suite}
\VerseOne{}Toi donc, fils de l'homme, prophétise contre Gog, et dis : Ainsi parle le Seigneur Yahweh : Voici, j'en veux à toi, Gog, prince des chefs de Méschec et de Tubal !
\VS{2}Je te ferai retourner en arrière, je te conduirai, je te ferai monter de l'extrême nord, et je t'amènerai sur les montagnes d'Israël.
\VS{3}Car je frapperai ton arc dans ta main gauche, et je ferai tomber tes flèches de ta main droite.
\VS{4}Tu tomberas sur les montagnes d'Israël, toi et toutes tes troupes, et les peuples qui seront avec toi ; je te livrerai aux oiseaux de proie, à tout ce qui a des ailes, et aux bêtes des champs, pour en être dévoré.
\VS{5}Tu tomberas sur la face des champs, parce que j'ai parlé, dit le Seigneur Yahweh.
\VS{6}Je mettrai le feu dans Magog, et parmi ceux qui demeurent en sécurité dans les îles ; et ils sauront que je suis Yahweh.
\VS{7}Je ferai connaître mon saint Nom au milieu de mon peuple d'Israël ; et je ne profanerai plus mon saint Nom ; les nations sauront que je suis Yahweh, le Saint d'Israël.
\VS{8}Voici, cela arrive et sera fait, dit le Seigneur Yahweh ; c'est ici le jour dont j'ai parlé.
\VS{9}Les habitants des villes d'Israël sortiront, allumeront le feu, brûleront les armes, les petits et les grands boucliers, les arcs, les flèches, les bâtons qu'on lance de la main, et les javelots ; ils en feront du feu pendant sept ans.
\VS{10}On n'apportera point du bois des champs, et on n'en coupera point dans les forêts, parce qu'ils feront du feu de ces armes, lorsqu'ils dépouilleront ceux qui les avaient dépouillés, et qu'ils pilleront ceux qui les avaient pillés, dit le Seigneur Yahweh.
\VS{11}Il arrivera ce jour-là que je donnerai à Gog dans ces quartiers-là un lieu pour sépulcre en Israël, à savoir la vallée des passants, qui est au-devant de la mer ; elle réduira les passants au silence ; on enterrera là Gog, et toute la multitude de son peuple, et on l'appellera la vallée d'Hamon-Gog\FTNT{La vallée d'Hamon-Gog : La vallée de la multitude de Gog.}.
\VS{12}Ceux de la maison d'Israël les enterreront, et cela durera sept mois, afin de purifier le pays.
\VS{13}Tout le peuple du pays les enterrera, et il en aura du renom, le jour où je serai glorifié, dit le Seigneur Yahweh.
\VS{14}Ils mettront à part des gens qui ne feront autre chose que parcourir le pays, et qui enterreront, avec l'aide des passants, les corps restés à la surface de la terre, pour la purifier, et ils seront à la recherche pendant sept mois.
\VS{15}Ils parcourront le pays, et celui qui verra l'os d'un homme, dressera auprès de lui un signal ; jusqu'à ce que les fossoyeurs l'aient enterré dans la vallée d'Hamon-Gog.
\VS{16}Il y aura aussi une ville nommée Hamona\FTNT{Hamona signifie « multitude ».}, et on nettoiera le pays.
\VS{17}Toi donc, fils de l'homme, ainsi parle le Seigneur Yahweh : Dis aux oiseaux de toutes espèces, et à toutes les bêtes des champs : Assemblez-vous et venez ; amassez-vous de toutes parts vers mon sacrifice que je fais pour vous, qui est un grand sacrifice sur les montagnes d'Israël ! Vous mangerez de la chair, et vous boirez du sang.
\VS{18}Vous mangerez la chair des hommes puissants, et vous boirez le sang des princes de la terre, le sang des moutons, des agneaux, des boucs, et des veaux engraissés sur le Basan\FTNT{Es. 34:6 ; Jé. 46:10 ; So. 1:7 ; Mt. 24:28 ; Job. 39:33.}.
\VS{19}Vous mangerez de la graisse jusqu'à en être rassasiés, et vous boirez du sang jusqu'à en être ivres, de la graisse et du sang de mon sacrifice, que j'aurai sacrifié pour vous.
\VS{20}Vous serez rassasiés à ma table, de chevaux et de bêtes d'attelage, d'hommes forts, et de tous hommes de guerre, dit le Seigneur Yahweh.
\VS{21}Je mettrai ma gloire parmi les nations, et toutes les nations verront mon jugement que j'aurai exercé, et comment j'aurai mis ma main sur eux.
\VS{22}La maison d'Israël connaîtra dès ce jour-là, et dans la suite, que je suis Yahweh, leur Dieu.
\VS{23}Les nations sauront que la maison d'Israël avait été emmenée en captivité à cause de son iniquité, parce qu'ils avaient péché contre moi, et que je leur avais caché ma face ; aussi je les avais livrés entre les mains de leurs ennemis pour qu'ils périssent par l'épée\FTNT{De. 31:17-18 ; Ps. 13:2.}.
\VS{24}Je leur avais fait selon leurs souillures, et selon leurs crimes, et je leur avais caché ma face.
\TextTitle{Rétablissement et conversion d'Israël}
\VS{25}C'est pourquoi, ainsi parle le Seigneur Yahweh : Maintenant, je ramènerai la captivité de Jacob, et j'aurai pitié de toute la maison d'Israël, et je serai jaloux de mon saint Nom,
\VS{26}après avoir porté leur ignominie, et tout leur crime, lorsqu'ils avaient péché contre moi, quand ils demeuraient en sûreté dans leur terre, sans qu'il y eût personne pour les effrayer.
\VS{27}Parce que je les ramènerai d'entre les peuples, que je les rassemblerai des pays de leurs ennemis, et que je serai sanctifié par eux, sous les yeux de plusieurs nations.
\VS{28}Ils sauront que je suis Yahweh, leur Dieu, lorsqu'après les avoir enlevés parmi les nations, je les rassemblerai sur leurs terres, et que je n'en laisserai chez elles aucun d'eux.
\VS{29}Je ne leur cacherai plus ma face, car je répandrai mon Esprit sur la maison d'Israël, dit le Seigneur Yahweh\FTNT{Joë. 2:28 ; Ac. 2:17.}.
\Chap{40}
\TextTitle{Mesures du futur temple}
\VerseOne{}Dans la vingt-cinquième année de notre captivité, au commencement de l'année, au dixième jour du mois, la quatorzième année après que la ville fut prise, en ce même jour, la main de Yahweh fut sur moi, et il m'amena là.
\VS{2}Il m'amena par des visions de Dieu, au pays d'Israël, et me posa sur une montagne fort élevée, sur laquelle du côté sud il y avait comme une ville construite.
\VS{3}Après qu'il m'y fît entrer, voici un homme, dont l'aspect était comme de l'airain, qui avait dans sa main un cordeau de lin, et une canne à mesurer, et qui se tenait debout à la porte.
\VS{4}Cet homme me parla ainsi : Fils de l'homme, regarde de tes yeux, écoute de tes oreilles, et applique ton cœur à toutes les choses que je m'en vais te faire voir, car tu as été amené ici afin que je te les fasse voir, et que tu fasses savoir à la maison d'Israël toutes les choses que tu vas voir.
\VS{5}Voici, un mur extérieur entourait la maison. Cet homme avait dans la main une canne à mesurer longue de six coudées, chaque coudée étant d'une coudée normale et une largeur de main en plus. Il mesura la largeur de ce mur bâti, laquelle était d'une canne, et sa hauteur d'une autre canne.
\VS{6}Puis il vint vers la porte orientale, et monta par ses étages. Il mesura l'un des poteaux de la porte d'une canne en largeur, et l'autre poteau d'une autre canne en largeur.
\VS{7}Puis il mesura chaque chambre d'une canne en longueur, et d'une canne en largeur. L'espace entre les deux chambres était de cinq coudées. Il mesura d'une canne chacun des poteaux de la porte près du vestibule qui menait à la porte la plus intérieure.
\VS{8}Puis il mesura d'une canne le vestibule qui menait à la porte la plus intérieure.
\VS{9}Il mesura de huit coudées le vestibule de la porte et ses poteaux, le vestibule de la porte était en dedans.
\VS{10}Les chambres de la porte orientale étaient au nombre de trois d'un côté et de trois de l'autre, toutes les trois avaient la même mesure.
\VS{11}Puis il mesura de dix coudées la largeur de l'ouverture de la première porte, et de treize coudées la longueur de la même porte.
\VS{12}Ensuite, il mesura d'un côté un espace limité au-devant des chambres d'une coudée, et une autre coudée d'espace limité de l'autre côté ; chaque chambre avait six coudées d'un côté, et six coudées de l'autre.
\VS{13}Après cela, il mesura le portail depuis le toit d'une chambre jusqu'au toit de l'autre, de la largeur de vingt-cinq coudées entre les deux ouvertures opposées.
\VS{14}Il compta soixante coudées pour les poteaux, près desquels était une cour, autour de la porte.
\VS{15}L'espace entre la porte d'entrée et le vestibule de la porte intérieure était de cinquante coudées.
\VS{16}Il y avait des fenêtres closes aux chambres et à leurs poteaux, à l'intérieur de la porte tout autour. Il y avait aussi des fenêtres dans les vestibules tout autour intérieurement, des palmes étaient sculptées sur les poteaux.
\VS{17}Il me mena dans le parvis extérieur, où se trouvaient des chambres et un pavé tout autour. Il y avait trente chambres sur ce pavé.
\VS{18}Le pavé était au côté des portes et répondait à la longueur des portes ; c'était le pavé inférieur.
\VS{19}Il mesura la largeur du parvis depuis la porte qui menait vers le bas jusqu'au parvis intérieur en dehors. Il y avait cent coudées à l'orient et au nord.
\VS{20}Après cela, il mesura la longueur et la largeur de la porte nord du parvis extérieur.
\VS{21}Quant aux chambres, au nombre de trois d'un côté et trois de l'autre, ses poteaux et ses vestibules avaient la même mesure que la première porte, cinquante coudées en longueur, et vingt-cinq coudées en largeur.
\VS{22}Ses fenêtres, son vestibule, et ses palmes avaient la même mesure que la porte orientale ; on y montait par sept étages, devant lesquels étaient son vestibule.
\VS{23}La porte du parvis intérieur était vis-à-vis de la première porte du nord, et vis-à-vis de la porte orientale. Il mesura depuis une porte jusqu'à l'autre cent coudées.
\VS{24}Après cela, il me conduisit du côté sud, où se trouvait la porte méridionale, il en mesura les poteaux et les vestibules qui avaient la même mesure.
\VS{25}Cette porte et ses vestibules avaient des fenêtres tout autour, comme les autres fenêtres, cinquante coudées de long, et vingt-cinq coudées de large.
\VS{26}On y montait par sept étages, devant lesquels était son vestibule ; il avait de chaque côté des palmes sur ses poteaux.
\VS{27}Pareillement, le parvis intérieur avait sa porte du côté sud ; il mesura d'une porte à l'autre au sud cent coudées.
\VS{28}Après cela il me fit entrer dans le parvis intérieur par la porte sud, et il mesura la porte sud, selon les mesures précédentes.
\VS{29}Ses chambres, ses poteaux et ses vestibules avaient la même mesure. Cette porte et ses vestibules avaient des fenêtres tout autour, cinquante coudées de long, et vingt-cinq coudées de large.
\VS{30}Il y avait tout autour des vestibules de vingt-cinq coudées de long, et cinq coudées de large.
\VS{31}Les vestibules de la porte aboutissaient au parvis extérieur ; il y avait des palmes sur ses poteaux, et huit étages pour y monter.
\VS{32}Il me conduisit dans le parvis intérieur, par l'entrée orientale. Il mesura la porte, qui avait la même mesure.
\VS{33}Ses chambres, ses poteaux et ses vestibules avaient la même mesure. Cette porte et ses vestibules avaient des fenêtres tout autour, cinquante coudées de long, et vingt-cinq de large.
\VS{34}Ses vestibules aboutissaient au parvis extérieur ; il y avait de chaque côté des palmes sur ses poteaux, et huit étages pour y monter.
\VS{35}Il me conduisit vers la porte nord. Il la mesura et trouva la même mesure.
\VS{36}Ainsi qu'à ses chambres, à ses poteaux et à ses vestibules ; elle avait des fenêtres tout autour, cinquante coudées de long, et vingt-cinq coudées de large.
\VS{37}Ses vestibules aboutissaient au parvis extérieur ; il y avait de chaque côté des palmes sur ses poteaux, et huit étages pour y monter.
\VS{38}Il y avait une chambre qui s'ouvrait vers les poteaux des portes, et où l'on devait laver les holocaustes.
\VS{39}Il y avait aussi dans le vestibule de la porte de chaque côté deux tables, pour y égorger les bêtes qu'on sacrifierait pour l'holocauste, et le sacrifice pour l'expiation et le sacrifice pour la culpabilité.
\VS{40}Vers l'un des côtés de la porte, au dehors, vers le lieu où l'on montait, à l'entrée de la porte nord, il y avait deux tables, et de l'autre côté, vers le vestibule de la porte, deux autres tables.
\VS{41}Il se trouvait ainsi, aux côtés de la porte, quatre tables d'une part, et quatre tables de l'autre, en tout huit tables, sur lesquelles on devait abattre les victimes.
\VS{42}Les quatre tables qui étaient pour l'offrande entièrement consumée, étaient en pierres de taille, de la longueur d'une coudée et demie, et de la largeur d'une coudée et demie, et de la hauteur d'une coudée ; et même on devait poser sur elles les instruments avec lesquels on tuait les victimes pour les offrandes entièrement consumées, et les autres sacrifices.
\VS{43}Il y avait aussi à l'intérieur de la maison tout autour, des chevilles pour accrocher, larges d'une paume, bien adaptées, d'où l'on apportait la chair des sacrifices sur les tables.
\TextTitle{Répartition des pièces du futur temple}
\VS{44}En dehors de la porte intérieure, il y avait des chambres pour les chantres dans le parvis intérieur, l'une était à côté de la porte nord et avait la face au sud, l'autre était à côté de la porte orientale et avait la face au nord.
\VS{45}Il me dit : Ces chambres, dont la face est au sud, sont pour les sacrificateurs qui ont la charge de la maison.
\VS{46}Mais ces chambres, dont la face est au nord, sont pour les sacrificateurs qui ont la charge de l'autel, qui sont les fils de Tsadok, qui, parmi les fils de Lévi, s'approchent de Yahweh pour faire son service.
\VS{47}Puis il mesura un parvis de la longueur et de la largeur de cent coudées, en carré ; et l'autel était devant la maison.
\VS{48}Ensuite, il me fit entrer dans le vestibule de la maison ; et il mesura les poteaux du vestibule de cinq coudées d'un côté, et de cinq coudées de l'autre, puis la largeur de la porte de trois coudées d'un côté, et de trois coudées de l'autre.
\VS{49}Le vestibule avait une longueur de vingt coudées, et une largeur de onze coudées ; on y montait par des étages. Il y avait des colonnes près des poteaux, l'une d'un côté, et l'autre de l'autre.
\Chap{41}
\TextTitle{Description du temple}
\VerseOne{}Puis il me fit entrer dans le temple, et il mesura des poteaux de six coudées de largeur d'un côté, et de six coudées de largeur de l'autre côté, largeur de la tente.
\VS{2}Ensuite il mesura la largeur de l'ouverture de la porte qui était de dix coudées, et les côtés de l'ouverture de cinq coudées, d'une part, et de cinq coudées de l'autre part. Puis il mesura la longueur du temple, quarante coudées, et la largeur, vingt coudées.
\VS{3}Il entra à l'intérieur, et il mesura un poteau d'une ouverture de porte, deux coudées, la hauteur de cette ouverture, six coudées, et la largeur de cette ouverture, sept coudées.
\VS{4}Puis il mesura une longueur de vingt coudées, et une largeur de vingt coudées en face du temple ; et il me dit : C'est ici le Saint des saints.
\VS{5}Il mesura l'épaisseur du mur de la maison, qui fut de six coudées, et la largeur des chambres qui étaient tout autour de la maison, de quatre coudées.
\VS{6}Les chambres latérales étaient les unes à côté des autres, au nombre de trente, et il y avait trois poutres ; elles entraient dans un mur construit pour ces chambres tout autour de la maison, elles y étaient appuyées sans entrer dans le mur même de la maison.
\VS{7}Les chambres occupaient plus d'espace, à mesure qu'elles s'élevaient, et l'on allait en tournant, car on montait autour de la maison par un escalier tournant. Il y avait plus d'espace dans le haut de la maison, et l'on montait de l'étage inférieur à l'étage supérieur par celui du milieu.
\VS{8}Je considérai la hauteur autour de la maison. Les chambres latérales, à partir de leur fondement avaient une canne pleine, six grandes coudées.
\VS{9}La largeur du mur extérieur des chambres latérales était de cinq coudées ; l'espace libre entre les chambres latérales de la maison,
\VS{10}et les chambres autour de la maison avait une largeur de vingt coudées.
\VS{11}L'ouverture des chambres latérales donnait sur l'espace libre, une ouverture au nord, et une autre ouverture au sud ; la largeur de l'espace libre était de cinq coudées tout autour.
\VS{12}Le bâtiment qui était devant la place vide, du côté de l'occident, avait une largeur de soixante-dix coudées, un mur de cinq coudées de largeur tout autour, et une longueur de quatre-vingt-dix coudées.
\VS{13}Il mesura la maison, qui avait cent coudées de longueur ; la place vide, le bâtiment et les murs avaient une longueur de cent coudées.
\VS{14}La largeur de la face de la maison et de la place vide, du côté oriental, était de cent coudées.
\VS{15}Il mesura la longueur du bâtiment devant la place vide, sur le derrière, et ses galeries de chaque côté : Il y avait cent coudées.
\VS{16}Les seuils, les fenêtres closes, les galeries du pourtour aux trois étages, en face des seuils, étaient recouverts de bois tout autour. Depuis le sol jusqu'aux fenêtres fermées,
\VS{17}jusqu'au-dessus des ouvertures, et jusqu'à la maison au-dedans comme au dehors, tout le mur du pourtour, à l'intérieur et à l'extérieur, tout était d'après la mesure,
\VS{18}et fait de chérubins et de palmes. Il y avait une palme entre deux chérubins, et chaque chérubin avait deux faces.
\VS{19}Une face d'homme était tournée vers la palme d'un côté, et une face de jeune lion était tournée vers la palme de l'autre côté ; il en était ainsi tout autour de la maison.
\VS{20}Depuis le sol jusqu'au-dessus des ouvertures il y avait des chérubins et des palmes et aussi sur le mur du temple.
\VS{21}Les poteaux du temple étaient carrés ; et la face du lieu saint avait la même apparence.
\VS{22}L'autel était de bois, de la hauteur de trois coudées, et de deux coudées de longueur ; ses angles, ses pieds et ses côtés étaient de bois. Puis il me dit : C'est ici la table qui est devant Yahweh.
\VS{23}Le temple et le lieu saint avaient deux portes.
\VS{24}Il y avait deux portes, deux battants, qui tous deux tournaient sur les portes, deux battants pour une porte et deux pour l'autre.
\VS{25}Il y avait aussi des chérubins et des palmes façonnés sur les portes du temple, comme sur les murs. Un entablement en bois était sur le front du vestibule en dehors.
\VS{26}Il y avait des fenêtres fermées, et des palmes de part et d'autre, ainsi qu'aux côtés du vestibule, aux chambres latérales de la maison, et aux entablements.
\Chap{42}
\TextTitle{Mesures supplémentaires du temple}
\VerseOne{}Après cela, il me fit sortir vers le parvis extérieur, du côté nord ; et il me conduisit vers les chambres qui étaient vis-à-vis de la place vide et vis-à-vis du bâtiment, au nord.
\VS{2}Sur la face où se trouvait une ouverture au nord, il y avait une longueur de cent coudées, et la largeur était de cinquante coudées.
\VS{3}C'était vis-à-vis des vingt coudées du parvis intérieur, et vis-à-vis du pavé extérieur, là où se trouvaient les galeries des trois étages.
\VS{4}Devant les chambres, il y avait une promenade large de dix coudées, et une voie d'une coudée ; leurs ouvertures donnaient au nord.
\VS{5}Les chambres supérieures étaient plus étroites que les inférieures et que celles du milieu du bâtiment parce que les galeries leur ôtaient de la place.
\VS{6}Car elles étaient à trois étages, et n'avaient point de colonnes, comme les colonnes des parvis ; c'est pourquoi à partir du sol, les chambres du haut étaient plus étroites que celles du bas et du milieu.
\VS{7}Le mur extérieur parallèle aux chambres, du côté du parvis extérieur devant les chambres, avait cinquante coudées de long.
\VS{8}Car la longueur des chambres du côté du parvis extérieur était de cinquante coudées. Mais sur la face du temple, il y avait cent coudées.
\VS{9}Au bas de ces chambres était l'entrée orientale quand on y venait du parvis extérieur.
\VS{10}Il y avait encore des chambres sur la largeur du mur du parvis du côté oriental, vis-à-vis de la place vide et vis-à-vis du bâtiment.
\VS{11}Devant elles, il y avait un chemin, comme devant les chambres qui étaient du côté nord. La longueur et la largeur étaient les mêmes ; leurs issues, leur disposition et leurs ouvertures étaient semblables.
\VS{12}Il en était de même pour les ouvertures des chambres du côté sud. Il y avait une ouverture à la tête du chemin, du chemin qui se trouvait droit devant le mur du côté oriental par où l'on y entrait.
\VS{13}Après cela, il me dit : Les chambres du parvis nord et les chambres du parvis sud, qui sont devant la place vide, ce sont les chambres du lieu saint où les sacrificateurs qui s'approchent de Yahweh, mangeront les choses très saintes. Ils déposeront là les choses très saintes, savoir les gâteaux, les offrandes pour l'expiation et les offrandes pour la culpabilité ; car ce lieu est saint.
\VS{14}Quand les sacrificateurs seront entrés, ils ne sortiront point du lieu saint pour venir au parvis extérieur, mais ils déposeront là leurs vêtements avec lesquels ils font le service ; car ces vêtements sont saints ; ils en mettront d'autres pour s'approcher du peuple.
\VS{15}Lorsqu'il eut achevé de mesurer la maison intérieure, il me fit sortir par la porte qui était du côté oriental, puis il mesura l'enceinte tout autour.
\VS{16}Il mesura le côté oriental avec la canne qui servait de mesure, et il y avait tout autour cinq cents cannes.
\VS{17}Ensuite il mesura le côté nord, avec la canne qui servait de mesure, et il y avait tout autour cinq cents cannes.
\VS{18}Puis il mesura le côté sud avec la canne qui servait de mesure, et il y avait cinq cents cannes.
\VS{19}Il se tourna du côté occidental, et mesura cinq cents cannes avec la canne qui servait de mesure.
\VS{20}Il mesura des quatre côtés le mur formant l'enceinte de la maison ; la longueur était de cinq cents cannes, et la largeur de cinq cents cannes, ce mur marquait la séparation entre le saint et le profane.
\Chap{43}
\TextTitle{La gloire de Yahweh remplit la maison\FTNTT{Cp. Ez. 11:22-24.}}
\VerseOne{}Puis il me ramena à la porte, à la porte qui était du côté oriental.
\VS{2}Et voici, la gloire du Dieu d'Israël s'avançait de l'orient, sa voix était pareille au bruit des grandes eaux, et la terre resplendissait de sa gloire\FTNT{Ap. 1:15.}.
\VS{3}La vision que j'eus alors était semblable à celle que j'avais vue lorsque j'étais venu pour détruire la ville, ces visions étaient comme la vision que j'avais vue sur le fleuve de Kebar ; et je me prosternai le visage contre terre.
\VS{4}Puis la gloire de Yahweh entra dans la maison par la porte qui était du côté oriental.
\VS{5}L'Esprit m'enleva et me fit entrer dans le parvis intérieur, et voici la gloire de Yahweh remplissait la maison.
\TextTitle{Le trône de Yahweh}
\VS{6}Je l'entendis s'adressant à moi depuis la maison, et l'homme qui me conduisait était debout près de moi.
\VS{7}Yahweh me dit : Fils de l'homme, c'est ici le lieu de mon trône, et le lieu des plantes de mes pieds, dans lequel je ferai ma demeure éternellement parmi les fils d'Israël ; et la maison d'Israël ne souillera plus mon saint Nom, ni eux, ni leurs rois, par leurs fornications ; mais ils souilleront leurs hauts lieux par les cadavres de leurs rois.
\VS{8}Car ils ont mis leur seuil près de mon seuil, et leur poteau près de mon poteau, il y avait un mur entre moi et eux ; ils ont souillé mon saint Nom par leurs abominations qu'ils ont faites, c'est pourquoi je les ai consumés dans ma colère.
\VS{9}Maintenant ils rejetteront loin de moi leurs adultères et les cadavres de leurs rois, et je ferai ma demeure éternellement parmi eux.
\VS{10}Toi donc, fils de l'homme, montre ce temple à la maison d'Israël ; et qu'ils soient confus à cause de leurs iniquités ; et qu'ils aient honte de leur iniquité.
\VS{11}S'ils rougissent de tout ce qu'ils ont fait, fais-leur connaître la forme de ce temple, sa disposition, avec ses sorties et ses entrées, toutes ses figures et toutes ses ordonnances, toutes ses formes, toutes ses lois, et écris-les sous leurs yeux, afin qu'ils gardent toutes ses formes, et toutes les ordonnances, et qu'ils les pratiquent.
\VS{12}Tel est la loi de la maison. Sur le sommet de la montagne, tout le territoire sera un lieu très saint tout autour. Voilà donc la loi de la maison.
\TextTitle{L'autel pour les holocaustes et les sacrifices}
\VS{13}Voici les mesures de l'autel, d'après les coudées dont chacune était d'une largeur de main plus longue que la coudée ordinaire. Le fond avait une coudée de hauteur et une coudée de largeur, et le rebord qui terminait son contour avait un empan de largeur ; c'était le dos de l'autel.
\VS{14}Depuis le fond sur le sol jusqu'à l'encadrement inférieur, il y avait deux coudées, et une coudée de largeur, et depuis le petit jusqu'au grand encadrement, il y avait quatre coudées et une coudée de largeur.
\VS{15}L'autel avait quatre coudées ; et quatre cornes s'élevaient de l'autel.
\VS{16}L'autel avait douze coudées de longueur, douze coudées de largeur, et formait un carré par ses quatre côtés.
\VS{17}L'encadrement avait quatorze coudées de longueur sur quatorze coudées de largeur à ses quatre côtés, le rebord qui terminait son contour avait une demi-coudée, le fond avait une coudée tout autour, et les étages étaient tournés vers l'orient.
\VS{18}Il me dit : Fils de l'homme, ainsi parle le Seigneur Yahweh : Ce sont ici les lois au sujet de l'autel pour le jour où on le fera, afin qu'on y offre l'holocauste, et qu'on y répande le sang.
\VS{19}C'est que tu donneras aux sacrificateurs, aux Lévites, qui sont de la race de Tsadok, et qui s'approchent de moi, dit le Seigneur Yahweh, afin qu'ils y fassent mon service, un jeune veau en sacrifice pour le péché.
\VS{20}Et tu prendras de son sang, et en mettras sur les quatre cornes de l'autel, et sur les quatre angles de l'encadrement et sur le rebord qui l'entoure, ainsi tu purifieras l'autel, et tu feras propitiation pour lui\FTNT{Ex. 29:36-39.}.
\VS{21}Tu prendras le jeune taureau expiatoire, et on le brûlera dans un lieu réservé de la maison, en dehors du lieu saint.
\VS{22}Le second jour, tu offriras en expiation un bouc, sans défaut, et on purifiera l'autel comme on l'aura purifié avec le jeune taureau.
\VS{23}Quand tu auras achevé de purifier l'autel, tu offriras un jeune taureau sans défaut, et un bélier du troupeau sans défaut.
\VS{24}Tu les offriras devant Yahweh, et les sacrificateurs jetteront du sel par dessus, et les offriront en holocauste à Yahweh\FTNT{Lé. 2:13.}.
\VS{25}Durant sept jours, tu sacrifieras chaque jour un bouc comme victime expiatoire, et les sacrificateurs sacrifieront un jeune taureau et un bélier du troupeau sans défaut.
\VS{26}Pendant sept jours, les sacrificateurs feront la propitiation pour l'autel, on le purifiera et chacun d'eux sera consacré\FTNT{Le terme « consacré » veut dire littéralement « remplir sa main ». Voir aussi Jg.17:5 et 17:12}
\VS{27}Lorsque ces jours seront accomplis, dès le huitième jour, et à l'avenir, les sacrificateurs offriront sur cet autel vos holocaustes et vos sacrifices d'offrande de paix. Et je serai apaisé envers vous, dit le Seigneur Yahweh.
\Chap{44}
\TextTitle{La porte fermée du sanctuaire}
\VerseOne{}Puis il me ramena vers la porte extérieure du lieu saint, du côté oriental, mais elle était fermée.
\VS{2}Yahweh me dit : Cette porte-ci sera fermée, et ne sera point ouverte, personne n'y passera, parce que Yahweh, le Dieu d'Israël, est entré par cette porte ; elle sera donc fermée\FTNT{Ap. 3:8.}.
\VS{3}Elle sera pour le prince ; le prince sera le seul qui s'y assiéra pour manger le pain devant Yahweh ; il entrera par le chemin du vestibule de la porte, et sortira par le même chemin.
\TextTitle{[La gloire dans la maison de Yahweh]}
\VS{4}Il me fit revenir par le chemin de la porte nord jusque sur le devant de la maison, je regardai, et voici, la gloire de Yahweh avait rempli la maison de Yahweh, et je me prosternai sur ma face.
\VS{5}Alors Yahweh me dit : Fils de l'homme, applique ton cœur, et regarde de tes yeux, écoute de tes oreilles tout ce dont je vais te parler, concernant toutes les ordonnances et toutes les lois qui concernent la maison de Yahweh. Applique ton cœur en ce qui concerne l'entrée de la maison et toutes les sorties du lieu saint.
\VS{6}Tu diras aux rebelles, à la maison d'Israël : Ainsi parle le Seigneur Yahweh : Maison d'Israël ! Assez de toutes vos abominations !
\VS{7}Vous avez fait entrer les fils de l'étranger, incirconcis de cœur et incirconcis de chair, pour être dans mon lieu saint, pour profaner ma maison. Vous avez offert mon pain, la graisse et le sang, à toutes vos abominations, vous avez enfreint mon alliance\FTNT{Lé. 3:11-16 ; Lé. 22:25 ; No. 28:2.}.
\VS{8}Vous n'avez pas observé l'office de mon lieu saint, mais vous les avez mis à votre place pour faire l'office dans mon lieu saint.
\TextTitle{Recommandations aux sacrificateurs du futur temple}
\VS{9}Ainsi parle le Seigneur Yahweh : Pas un de tous ceux qui seront fils d'étranger, incirconcis de cœur et incirconcis de chair, n'entrera dans mon lieu saint, pas même un d'entre tous les fils d'étrangers qui seront parmi les fils d'Israël.
\VS{10}Mais les Lévites qui se sont éloignés de moi, lorsque Israël s'est égaré, et qui se sont égarés de moi pour suivre leurs idoles, porteront la peine de leur iniquité.
\VS{11}Toutefois, ils seront employés dans mon lieu saint aux charges qui sont vers les portes de la maison, et ils feront le service de la maison ; ils égorgeront pour le peuple les bêtes pour l'holocauste, et pour les autres sacrifices, et se tiendront prêts devant lui pour le servir.
\VS{12}Parce qu'ils l'ont servi se présentant devant leurs idoles, et qu'ils ont fait tomber dans l'iniquité la maison d'Israël, à cause de cela j'ai levé ma main en jurant contre eux, dit le Seigneur Yahweh, qu'ils porteront la peine de leur iniquité.
\VS{13}Ils n'approcheront plus de moi pour exercer la sacrificature, ni pour approcher mes sanctuaires, mes lieux très saints ; mais ils porteront leur confusion et leurs abominations qu'ils ont commises.
\VS{14}C'est pourquoi je les établirai pour avoir la garde de la maison pour tout son service, et pour tout ce qui s'y fait.
\VS{15}Mais quant aux sacrificateurs et aux Lévites, fils de Tsadok, qui ont soigneusement administré ce qu'il fallait faire dans mon lieu saint, lorsque les fils d'Israël se sont éloignés de moi, ceux-là s'approcheront de moi pour faire mon service, et se tiendront devant moi pour m'offrir la graisse et le sang, dit le Seigneur Yahweh.
\VS{16}Ceux-là entreront dans mon lieu saint, et s'approcheront de ma table, pour faire mon service, et ils administreront soigneusement ce que j'ai ordonné de faire.
\VS{17}Lorsqu'ils franchiront les portes des parvis intérieurs, ils se vêtiront de robes de lin ; et il n'y aura point de laine sur eux pendant qu'ils feront le service aux portes des parvis intérieurs et dans la maison.
\VS{18}Ils auront des ornements de lin sur leur tête, et des caleçons de lin sur leurs reins, et ne se ceindront point de manière à provoquer la sueur.
\VS{19}Quand ils sortiront pour aller dans le parvis extérieur, dans le parvis extérieur, vers le peuple, ils se dévêtiront de leurs habits, avec lesquels ils font le service, et les poseront dans les chambres saintes, et se revêtiront d'autres habits, afin qu'ils ne sanctifient point le peuple avec leurs habits.
\VS{20}Ils ne se raseront point la tête, ni ne laisseront point croître leurs cheveux, mais simplement ils tondront leur tête\FTNT{Lé. 19:27.}.
\VS{21}Pas un des sacrificateurs ne boira du vin quand ils entreront au parvis intérieur.
\VS{22}Ils ne prendront point pour femme une veuve, ni une répudiée ; mais ils prendront des vierges, de la race de la maison d'Israël, ou une veuve qui soit veuve d'un sacrificateur\FTNT{Lé. 21:13-14.}.
\VS{23}Ils enseigneront à mon peuple la différence qu'il y a entre le saint et le profane, et leur feront entendre la différence qu'il y a entre ce qui est souillé et ce qui est pur.
\VS{24}Quand il surviendra quelque procès, ils assisteront au jugement, et jugeront suivant les lois que j'ai données ; et ils garderont mes lois et mes statuts dans toutes mes fêtes, et ils sanctifieront mes sabbats.
\VS{25}Un sacrificateur n'ira pas vers un mort, de peur d'en être souillé, il pourra se rendre impur que pour un père, pour une mère, pour un fils, pour une fille, pour un frère, et pour une sœur qui n'aura point eu de mari\FTNT{Lé. 21:1-2.}.
\VS{26}Et après que chacun d'eux se sera purifié, on lui comptera sept jours.
\VS{27}Le jour où il entrera dans le lieu saint, dans le parvis intérieur pour faire le service dans le lieu saint, il offrira son sacrifice pour son péché, dit le Seigneur Yahweh.
\VS{28}Et cela leur sera pour héritage. Ce sera moi leur héritage, car vous ne leur donnerez aucune possession en Israël, ce sera moi leur possession\FTNT{No. 18:20 ; De. 18:1-2.}.
\VS{29}Ils mangeront donc les gâteaux et ce qui s'offrira pour l'expiation, et ce qui s'offrira pour la culpabilité ; et tout interdit en Israël leur appartiendra.
\VS{30}Les prémices de tous les fruits et toutes les offrandes que vous présenterez par élévation, appartiendront aux sacrificateurs ; vous donnerez aussi les prémices de votre pâte aux sacrificateurs, afin que la bénédiction repose sur votre maison.
\VS{31}Les sacrificateurs ne mangeront aucune créature volante, aucun animal mort ou déchiré\FTNT{Ex. 22:31 ; Lé. 22:8.}.
\Chap{45}
\TextTitle{Zone réservée à Yahweh et aux sacrificateurs}
\VerseOne{}Quand vous partagerez au sort le pays en héritage, vous prélèverez comme une offrande en élévation pour Yahweh, une portion du pays longue de vingt-cinq mille cannes, et large de dix mille ; ce sera une chose sainte dans tous ses territoires et aux environs.
\VS{2}De cette portion, vous prendrez pour le lieu saint cinq cents cannes sur cinq cents en carré, et cinquante coudées tout autour pour ses faubourgs.
\VS{3}Sur cette étendue de vingt-cinq mille en longueur, et de dix mille en largeur, tu mesureras un emplacement pour le lieu saint, pour le Saint des saints.
\VS{4}C'est la portion sainte du pays, elle appartiendra aux sacrificateurs qui font le service du lieu saint, qui s'approchent de Yahweh pour le servir ; c'est là que seront leur maison, et ce sera un lieu saint pour le lieu saint.
\VS{5}Vingt-cinq mille cannes en longueur, et dix mille en largeur, formeront la propriété des Lévites, serviteurs de la maison, avec vingt chambres.
\VS{6}Vous donnerez pour la possession de la ville la largeur de cinq mille et la longueur de vingt-cinq mille, suivant la proportion de la portion sanctifiée, qui aura été levée pour toute la maison d'Israël.
\TextTitle{Zone réservée au prince}
\VS{7}Pour le prince vous réserverez un espace aux deux côtés de la portion sainte et de la propriété de la ville, le long de la portion sainte et le long de la propriété de la ville, du côté de l'occident vers l'occident, et du côté de l'orient vers l'orient, sur une longueur parallèle à l'une des parts, depuis la limite de l'occident jusqu'à la limite de l'orient.
\VS{8}Ce sera sa terre, sa propriété en Israël ; et mes princes que j'établirai ne fouleront plus mon peuple, mais ils distribueront le pays à la maison d'Israël, selon leurs tribus.
\TextTitle{Le prince, exemple au milieu du peuple ; prescriptions sur les offrandes}
\VS{9}Ainsi parle le Seigneur Yahweh : Assez, princes d'Israël ! Otez la violence et le pillage, et jugez avec justice ; ôtez vos extorsions de dessus mon peuple ! dit le Seigneur Yahweh.
\VS{10}Ayez la balance juste, l'épha juste, et le bath juste\FTNT{Lé. 19:35-36.}.
\VS{11}L'épha et le bath seront de même mesure ; on prendra un bath pour la dixième partie d'un homer, et l'épha sera la dixième partie d'un homer, la mesure de l'un et de l'autre se rapportera à l'homer.
\VS{12}Le sicle sera de vingt guéras ; vingt sicles, vingt-cinq sicles et quinze sicles feront la mine\FTNT{Ex. 30:13 ; Lé. 27:25.}.
\VS{13}Voici l'offrande que vous élèverez en offrande : La sixième partie d'un épha d'un homer de blé ; et vous donnerez la sixième partie d'un épha d'un homer d'orge.
\VS{14}Le bath est la mesure pour l'huile, l'offrande ordonnée pour l'huile sera la dixième partie d'un bath sur un cor, qui est égal à un homer de dix baths ; car dix baths feront un homer.
\VS{15}Pareillement l'offrande ordonnée des bêtes du menu bétail sera de deux cents l'une, même des meilleurs pâturages d'Israël ; toute cette offrande sera employée en gâteaux et en holocaustes, et en offrandes de paix, afin de faire propitiation pour vous, dit le Seigneur Yahweh.
\VS{16}Tout le peuple du pays sera tenu à cette offrande élevée, pour celui qui sera prince en Israël.
\VS{17}Mais le prince sera tenu de fournir les holocaustes, les offrandes et les libations qu'il faudra offrir aux fêtes solennelles, aux nouvelles lunes et aux sabbats, et dans toutes les solennités de la maison d'Israël. Il tiendra prêtes les bêtes qu'on sacrifiera pour l'expiation, et les gâteaux, et les bêtes qu'on sacrifiera pour l'holocauste, et les bêtes qu'on sacrifiera pour les offrandes de paix, afin de faire propitiation pour la maison d'Israël.
\VS{18}Ainsi parle le Seigneur Yahweh : Au premier mois, au premier jour du mois, tu prendras un jeune taureau sans défaut, et tu feras l'expiation du lieu saint.
\VS{19}Le sacrificateur prendra du sang de ce sacrifice offert pour le péché, et en mettra sur les poteaux de la maison, et sur les quatre angles de l'encadrement de l'autel, et sur les poteaux de la porte du parvis intérieur.
\VS{20}Tu en feras ainsi au septième jour du même mois, à cause des hommes qui pèchent involontairement et à cause des hommes simples ; et vous ferez ainsi propitiation pour la maison.
\VS{21}Au premier mois, au quatorzième jour du mois, vous aurez la Pâque, fête solennelle qui durera sept jours, pendant lesquels on mangera des pains sans levain\FTNT{Lé. 25:5 ; No. 9:3 ; Ex. 12.}.
\VS{22}En ce jour-là, le prince offrira un taureau pour le sacrifice d'expiation, tant pour lui que pour tout le peuple du pays.
\VS{23}Pendant les sept jours de cette fête solennelle, il offrira chaque jour sept taureaux et sept béliers sans défaut, pour l'holocauste qu'on offrira à Yahweh, et un bouc en sacrifice d'expiation, chaque jour.
\VS{24}Il offrira un épha pour chaque taureau, et un épha pour chaque bélier, avec un hin d'huile par épha.
\VS{25}Au septième mois, le quinzième jour du mois, à la fête solennelle, il offrira durant sept jours les mêmes choses, le même sacrifice expiatoire, le même holocauste, et la même offrande avec l'huile.
\Chap{46}
\TextTitle{Le service le jour du sabbat et les jours de fêtes}
\VerseOne{}Ainsi parle le Seigneur Yahweh : La porte du parvis intérieur, du côté oriental, sera fermée les six jours ouvrables, mais elle sera ouverte le jour du sabbat, elle sera aussi ouverte le jour de la nouvelle lune.
\VS{2}Et le prince y entrera par le chemin du vestibule de la porte du parvis extérieure, et se tiendra près de l'un des poteaux de l'autre porte, les sacrificateurs prépareront son holocauste et ses sacrifices d'offrande de paix ; il se prosternera sur le seuil de cette porte, et ensuite il sortira ; et la porte ne sera point fermée jusqu'au soir.
\VS{3}Tellement que le peuple du pays se prosternera devant Yahweh à l'entrée de cette porte, les jours de sabbat et des nouvelles lunes.
\VS{4}L'holocauste que le prince offrira à Yahweh le jour du sabbat sera de six agneaux sans défaut, et d'un bélier sans défaut.
\VS{5}L'offrande pour le bélier sera d'un épha, et l'offrande pour chacun des agneaux sera selon ce qu'il pourra donner ; mais il y aura un hin d'huile pour chaque épha.
\VS{6}Au jour de la nouvelle lune, son holocauste sera d'un jeune taureau, sans défaut, de six agneaux et d'un bélier, aussi sans défaut.
\VS{7}Son offrande pour le taureau sera d'un épha, pour l'offrande du bélier, un autre épha, et pour chacun des agneaux selon ce qu'il pourra donner ; mais il y aura un hin d'huile pour chaque épha.
\VS{8}Lorsque le prince entrera, il entrera par le chemin du vestibule de la porte, et il sortira par le même chemin.
\VS{9}Quand le peuple du pays entrera pour se présenter devant Yahweh, aux fêtes solennelles, celui qui y entrera par le chemin de la porte nord pour y adorer Yahweh, sortira par le chemin de la porte sud ; et celui qui y entrera par le chemin de la porte sud, sortira par le chemin de la porte nord ; personne ne retournera par le chemin de la porte par laquelle il sera entré, mais il sortira par celle qui lui est opposée.
\VS{10}Alors le prince entrera parmi eux, quand ils entreront ; et quand ils sortiront, ils sortiront ensemble.
\VS{11}Or,dans ces fêtes solennelles et dans ces solennités, l'offrande d'un taureau sera d'un épha, et l'offrande d'un bélier d'un autre épha, l'offrande de chacun des agneaux sera selon ce que le prince pourra donner, et il y aura un hin d'huile pour chaque épha.
\VS{12}Et si le prince offre un sacrifice volontaire, quelque un holocauste, soit quelques un sacrifices d'offrande de paix en offrande à Yahweh, on lui ouvrira la porte qui est du côté oriental, et il offrira son holocauste et ses sacrifices d'offrande de paix comme il les offre le jour du sabbat, puis il sortira, et après qu'il sera sorti, on fermera cette porte.
\VS{13}Tu sacrifieras chaque jour en holocauste à Yahweh un agneau d'un an sans défaut, tu le sacrifieras tous les matins.
\VS{14}Tu lui offriras tous les matins l'offrande, faite de la sixième partie d'un épha, et de la troisième d'un hin d'huile pour pétrir la farine ; c'est l'offrande à Yahweh qu'il faut offrir par ordonnances perpétuelles.
\VS{15}Ainsi on offrira tous les matins en holocauste perpétuel cet agneau et l'offrande avec cette huile.
\VS{16}Ainsi a dit le Seigneur Yahweh : quand le Prince aura fait un don de quelque pièce de son héritage à quelqu'un de ses fils, ce don appartiendra à ses fils ; parce qu'ils ont droit de possession en l'héritage.
\VS{17}Mais s'il fait un don pris de son héritage à l'un de ses serviteurs, le don lui appartiendra bien, mais seulement jusqu'à l'année de la liberté, puis il retournera au prince ; ses fils seuls posséderont ce qu'il leur donnera de son héritage\FTNT{Lé. 25:10.}.
\VS{18}Et le prince ne prendra pas de l'héritage du peuple en les opprimant, les chassant de leur possession : c'est de sa propre possession qu'il fera hériter ses fils, afin qu'aucun de mon peuple ne soit pas dispersé loin de sa possession.
\VS{19}Puis il me mena par l'entrée qui était vers le côté de la porte, aux chambres saintes qui appartenaient aux sacrificateurs, vers le nord. Et voici, il y avait un certain lieu dans le fond du côté occidental.
\VS{20}Il me dit : C'est là le lieu où les sacrificateurs feront bouillir le reste de la bête qu'on aura sacrifiée pour la culpabilité, et le reste de la bête qu'on aura sacrifiée pour l'expiation, et où ils feront cuire les offrandes, afin qu'ils ne les emportent point au parvis extérieur de manière à sanctifier le peuple\FTNT{No. 18:9.}.
\VS{21}Puis il me fit sortir vers le parvis extérieur, et me fit traverser vers les quatre angles du parvis, et voici, il y avait une cour à chacun des angles du parvis.
\VS{22}Aux quatre angles de ce parvis, il y avait des cours voûtées, longues de quarante coudées, et larges de trente ; et toutes les quatre avaient la même mesure dans les angles.
\VS{23}Un mur les entourait toutes les quatre, et des foyers étaient faits au bas du campement tout autour.
\VS{24}Et il me dit : Ce sont ici les cuisines, où ceux qui font le service de la maison cuiront les sacrifices du peuple.
\Chap{47}
\TextTitle{Les eaux pures du sanctuaire\FTNTT{Cp. Za. 14:8-9 ; Ap. 22:1-2.}}
\VerseOne{}Puis il me ramena vers l'entrée de la maison, et voici, des eaux sortaient sous le seuil de la maison, vers l'orient, car la face de la maison était vers l'orient ; et ces eaux-là descendaient du côté droit de la maison, du côté sud de l'autel\FTNT{Ps. 46:5 ; Joë. 3:18 ; Za. 13:1 ; Za. 14:8 ; Ap. 22:1.}.
\VS{2}Puis il me fit sortir par le chemin de la porte nord, et me fit faire le tour par dehors, jusqu'à la porte extérieure, du côté de l'orient, et voici, les eaux coulaient du côté droit.
\VS{3}Quand cet homme s'avança vers l'orient, il avait dans sa main un cordeau ; et il mesura mille coudées, puis il me fit traverser ces eaux-là, et j'avais de l'eau jusqu'aux chevilles.
\VS{4}Puis il mesura mille autres coudées, et me fit traverser les eaux, j'avais de l'eau jusqu'aux genoux ; puis il mesura mille autres coudées, et me fit traverser, et j'avais de l'eau jusqu' aux reins.
\VS{5}Il mesura mille autres coudées ; mais ces eaux-là étaient déjà un torrent que je ne pouvais traverser ; car ces eaux-là étaient si profondes qu'il fallait y traverser à la nage, c'était un torrent que l'on ne pouvait traverser.
\VS{6}Alors il me dit : Fils de l'homme, as-tu vu ? Puis il me fit aller et revenir vers le bord du torrent.
\VS{7}Quand je revins, il y avait un grand nombre d'arbres sur les deux bords du torrent.
\VS{8}Il me dit : Ces eaux couleront vers la Galilée orientale, et elles descendront à la campagne, puis elles entreront dans la mer, et quand elles se seront jetées dans la mer, les eaux deviendront saines.
\VS{9}Il arrivera que tout être vivant qui se meut vivra partout où les deux torrents couleront, et il y aura une grande quantité de poissons ; car là où ces eaux entreront, les eaux deviendront saines, et tout vivra là où ce torrent parviendra.
\VS{10}Il arrivera que des pêcheurs se tiendront le long de cette mer, depuis En-Guédi jusqu'à En-Eglaïm ; on étendra les filets, il y aura des poissons de diverses espèces, comme les poissons de la grande mer, et ils seront très nombreux.
\VS{11}Ses marais et ses fosses ne seront pas assainis, ils seront abandonnés au sel.
\VS{12}Auprès de ce torrent et sur ses deux bords, il croîtra des arbres fruitiers de toutes sortes. Leur feuillage ne se flétrira point, et l'on trouvera toujours du fruit. Tous les mois, ils produiront des fruits mûrs, parce que les eaux de ce torrent sortent du lieu saint, et à cause de cela leur fruit sera bon à manger, et leur feuillage servira de remède\FTNT{Ap. 22:2.}.
\TextTitle{Délimitations du pays\FTNTT{Cp. Ge. 15:18-21.}}
\VS{13}Ainsi parle le Seigneur Yahweh : Voici les frontières du pays que vous aurez en héritage, selon les douze tribus d'Israël. Joseph aura deux portions.
\VS{14}Vous en aurez la possession l'un comme l'autre de ce pays. J'ai levé ma main de le donner à vos pères ; et ce pays-là vous sera donc échu en héritage\FTNT{Ge. 12:7 ; Ge. 17:8.}.
\VS{15}Voici la frontière du pays, du côté nord, depuis la grande mer, le chemin de Hethlon jusqu'à Tsedad,
\VS{16}Hamath, Bérotha, et Sibraïm, entre la frontière de Damas et la frontière de Hamath, Hatzer-Hatthicon, vers la frontière de Havran.
\VS{17}La frontière sera depuis la mer, Hatsar-Enon ; la frontière de Damas, Tsaphon au nord et la frontière de Hamath : Ce sera le côté nord.
\VS{18}Le côté oriental sera le Jourdain entre Havran, Damas et Galaad, et le pays d'Israël ; vous mesurerez depuis la frontière nord jusqu'à la mer orientale : Ce sera le côté oriental.
\VS{19}Le côté méridional, au midi, ira depuis Thamar jusqu'aux eaux de Meriba à Kadès, jusqu'au torrent vers la grande mer : Ce sera le côté méridional.
\VS{20}Le côté occidental sera la grande mer, depuis la frontière jusque vis-à-vis de Hamath : Ce sera le côté occidental.
\VS{21}Vous partagerez ce pays entre vous, selon les tribus d'Israël.
\VS{22}Vous le diviserez en héritage par le sort pour vous et pour les étrangers qui séjourneront au milieu de vous, qui engendreront des fils au milieu de vous ; vous les regarderez comme natifs des fils d'Israël ; ils partageront au sort l'héritage avec vous parmi les tribus d'Israël.
\VS{23}Vous donnerez à l'étranger son héritage dans la tribu où il séjournera, dit le Seigneur Yahweh.
\Chap{48}
\TextTitle{Héritage de sept tribus\FTNTT{Cp. Jos. 13:1-19:51.}}
\VerseOne{}Voici les noms des tribus. Depuis l'extrémité qui regarde vers le nord, le long de la contrée du chemin de Hethlon du quartier par lequel on entre à Hamath, jusqu'à Hatsar-Enon, qui est la frontière de Damas, du côté qui regarde vers le nord, le long de la contrée de Hamath, tellement que cette extrémité est le canton de l'orient et celui de l'occident: il y aura une portion pour Dan.
\VS{2}Et tout joignant les frontières de Dan, depuis le canton de l'orient jusqu'au canton qui regarde vers l'occident : il y aura une partion pour Aser.
\VS{3}Et tout joignant les frontières d'Aser, depuis le canton qui regarde vers l'orient jusqu'au canton qui regarde vers l'occident : il y aura une partion pour Nephthali.
\VS{4}Et tout les frontières de Nephthali, depuis le canton qui regarde vers l'orient jusqu'au canton qui regarde vers  l'occident : il y aura une partion pour Manassé. 
\VS{5}Et tout joignant les frontières de Manassé, depuis le canton qui regarde vers de l'occident jusqu'au canton qui regarde vers  l'orient : il y aura une part pour Ephraïm. 
\VS{6}Et tout joignant les frontières d'Ephraïm, encore depuis le canton de l'orient,jusqu'au canton qui regarde vers l'occident: il y aura une partion pour Ruben. 
\VS{7}Et joignant les frontièrs de Ruben, depuis le canton de l'orient jusqu'au canton qui regarde vers l'occident : il y aura une partion pour Juda.
\VS{8}Et tout le long des frontières de Juda, depuis le canton de l'orient, jusqu'au canton qui regarde vers l'occident; il y aura une portion que vous prélèverez sur toute la masse du pays, comme une offrande élevée et elle aura vingt-cinq mille cannes de largueur et de longueur, autant que l'une des autres partions depuis le canton qui regarde vers l'orient jusqu'au canton qui regarde vers l'occident; de sorte que le sanctuaire sera au milieu.
\VS{9}La portion que vous lèverez pour Yahweh et offerte en offrande élevée, aura vingt-cinq mille cannes de longueur, et dix mille de largeur.
\TextTitle{Territoire réservé aux sacrificateurs et aux Lévites}
\VS{10}Et cette portion sainte sera pour les sacrificateurs, vingt-cinq mille cannes de longueur au nord, et dix mille de largeur, à l'occident, dix mille en largeur à l'orient, et vingt-cinq mille en longueur au sud, et le sanctuaire de Yahweh sera au milieu.
\VS{11}Elle sera pour les sacrificateurs, et quiconque aura été sanctifié d'entre les fils de Tsadok, qui ont fait ce que j'ai ordonné, et qui ne se sont point égarés quand les fils d'Israël se sont égarés, comme se sont égarés les autres Lévites,
\VS{12}Ceux-là auront une portion ainsi levée sur l'autre très sainte, prélevée sur la portion du pays qui aura été prélevée, à côté de la frontière des Lévites.
\VS{13}Les Lévites auront, parallèlement à la frontière des sacrificateurs, vingt-cinq mille cannes en longueur et dix mille de largeur, vingt-cinq mille pour toute la longueur et dix mille pour toute la largeur.
\VS{14}Ils n'en pourront ni vendre, ni échanger, et les prémices du pays ne seront point transgressées car elles sont mises à part pour Yahweh.
\VS{15}Les cinq mille cannes qui resteront en largeur sur les vingt-cinq mille cannes seront destinées à la ville, pour les habitations et le faubourg, et la ville sera au milieu.
\VS{16}En voici les mesures : Du côté nord, quatre mille cinq cents cannes, du côté sud, quatre mille cinq cents, du côté oriental, quatre mille cinq cents, et du côté occidental, quatre mille cinq cents.
\VS{17}Puis il y aura des faubourgs pour la ville, vers le nord. La ville aura un faubourg au nord de deux cent cinquante cannes, de deux cent cinquante au sud, de deux cent cinquante à l'orient, et de deux cent cinquante à l'occident.
\VS{18}Quant à ce qui sera de reste sur la longueur et qui sera tout joignant à la portion sanctifiée, et qui aura dix mille cannes à l'orient, et dix mille autres cannes à l'occident, parallèlement à la portion sanctifiée, le revenu qu'on en tirera sera pour nourriture de ceux qui feront le service qu'il faut dans la ville.
\VS{19}Le sol sera travaillé par ceux de toutes les tribus d'Israël qui travailleront pour la ville.
\VS{20}Toute la portion prélevée sera de vingt-cinq mille cannes en longueur sur vingt-cinq mille en largeur ; vous en séparerez un carré pour la propriété de la ville.
\VS{21}PUis le reste sera pour le prince aux deux côtés de la portion sainte et de la possession de la ville, des vingt-cinq mille coudées de la portion prélevée jusqu'à la frontière de l'orient, et à l'occident, des vingt-cinq mille coudées jusqu'à la frontière de l'occident, le long des parts, pour le prince, et la portion sainte et le sanctuaire de la Maison seront au milieu de tout le pays. 
\VS{22}Ce qui sera donc pour le prince sera l'espace compris depuis la possession des Lévites, et depuis la possession de la ville ; ce qui sera entre ces possessions là et la frontière de Juda, et la frontière de Benjamin, sera pour le prince.
\TextTitle{Héritage de cinq tribus}
\VS{23}Or ce qui sera de reste sera pour les autres tribus; depuis le canton de ce qui regarde vers l'orient, jusqu'au canton de ce qui regarde vers l'occident, il y aura une portion pour Benjamin.
\VS{24}Puis tout joignant les frontières de Benjamin, depuis le canton de ce qui regarde vers l'orient, jusqu'au canton de ce qui regarde vers l'occident, il y aura une autre portion pour Siméon.
\VS{25}Puis tout joignant les frontières de Siméon, depuis le canton de ce qui regarde vers l'orient, jusqu'au canton de ce qui regarde vers l'occident, il y aura une autre portion pour Issacar.
\VS{26}Puis tout joignant sur les frontières d'Issacar, depuis le canton de ce qui regarde vers l'orient, jusqu'au canton de ce qui regarde vers l'occident, il y aura une autre portion pour Zabulon.
\VS{27}Puis joignant les frontières de Zabulon, depuis le canton de ce qui regarde vers l'orient, jusqu'au canton de ce qui regarde vers l'occident, il y aura une autre portion pour Gad.
\VS{28}Or ce qui appartient au côté du midi qui regarde proprement le vent d'autant, et sur la frontière de Gad, et cette frontière sera depuis Thamar jusqu'aux eaux de contestation, à Kadès, le long du torrent jusqu'à la grande mer.
\VS{29}C'est là le pays que vous partagerez par le sort en héritage aux tribus d'Israël, et ce sont là leurs portions, dit le Seigneur Yahweh.
\VS{30}Et ce sont ici les sorties de la ville : Du côté du nord, il y aura quatre mille cinq cents mesures.
\VS{31}Et les portes de la ville seront selon les noms des tribus d'Israël : Trois portes vers le nord, une porte de Ruben, une porte de Juda, une porte de Lévi.
\VS{32}Et du côté de l'orient, quatre mille cinq cents mesures, et trois portes : Une porte de Joseph, une porte de Benjamin, une porte de Dan.
\VS{33}Et du côté sud, quatre mille cinq cents mesures, et trois portes : Une porte de Simeon, une porte d'Issacar, une porte de Zabulon.
\VS{34}Et du côté ouest, quatre mille cinq cents mesures, avec leurs trois portes : Une porte de Gad, une porte d'Aser, une porte de Nephthali.
\VS{35}Ainsi le circuit de la ville sera de dix-huit mille mesures ; et le nom de la ville depuis ce jour-là sera : Yahweh est ici.
\PPE{}
\end{multicols}

%\clearpage\ShortTitle{Osée}\BookTitle{Osée}\BFont
\noindent\hrulefill
{\footnotesize
\textit{
\bigskip
{\centering{}
\\Auteur : Osée
\\(Heb. : Hoswhéa')
\\Signification : Salut, Sauve
\\Thème : Israël sera rejeté à cause de son apostasie. D'autres nations seront appelées à sa place.
\\Date de rédaction : 8\up{ème} siècle av. J.-C.\\}
}
%\bigskip
\textit{
\\Osée, fils de Béeri, exerça son service dans le royaume du nord au temps de Joas, roi d'Israël. Il était contemporain des prophètes Amos, Michée et Esaïe.
%\bigskip
\\Yahweh demanda à Osée d'épouser une prostituée pour que le prophète puisse partager plus profondément son fardeau
et la tristesse qu'il subissait en raison de l'infidélité du peuple qu'il aimait tant. Malgré la rébellion et les mauvais
agissements d'Israël, Yahweh manifesta une fois de plus sa patience et utilisa Osée pour avertir et inviter les fils de Jacob à la repentance.\bigskip
}
}
\par\nobreak\noindent\hrulefill
\begin{multicols}{2}
\Chap{1}
\VerseOne{}La parole de Yahweh qui fut adressée à Osée fils de Béeri, au temps d'Ozias, de Jotham, d'Achaz et d'Ezéchias, rois de Juda, et au temps de Jéroboam, fils de Joas, roi d'Israël.
\TextTitle{Mariage d'Osée et naissance de Jizreel}
\VS{2}La première fois que Yahweh parla à Osée, Yahweh dit à Osée : Va, prends une femme prostituée, et aie d'elle des enfants de prostitution ; car le pays s'est entièrement prostitué en abandonnant Yahweh !
\VS{3}Il alla, et il prit Gomer, fille de Diblaïm. Elle conçut, et lui enfanta un fils.
\VS{4}Et Yahweh lui dit : Donne-lui le nom de Jizreel ; car encore un peu de temps, et je punirai la maison de Jéhu pour le sang versé à Jizreel, et je ferai cesser le royaume de la maison d'Israël\FTNT{Cette prophétie s'est accomplie en 722 av. J-C. Voir 2 R.17.}.
\VS{5}Et il arrivera qu'en ce jour-là, je briserai l'arc d'Israël dans la vallée de Jizreel.
\TextTitle{Naissance de Lo-Ruchama}
\VS{6}Elle conçut encore, et enfanta une fille. Et Yahweh lui dit : Donne-lui le nom de Lo-Ruchama\FTNT{Tout au long de ce livre, certains mots sont symboliquement utilisées pour nommer Israël. A savoir : 
\\- « ruchama » : miséricorde ;
\\- « Lo-Ruchama » : celle dont on ne fait pas miséricorde ; 
\\- « ammi » : mon peuple ;
\\- « Lo-Ammi » : pas mon peuple.
\\Ces noms illustrent ainsi l'infidélité d'Israël envers Yahweh.} ; car je ne continuerai plus à faire miséricorde à la maison d'Israël, mais je les enlèverai entièrement.
\VS{7}Mais je ferai miséricorde à la maison de Juda; et je les délivrerai par Yahweh, leur Dieu, et je ne les délivrerai ni par l'arc, ni par l'épée, ni par les combats, ni par les chevaux, ni par les cavaliers.
\TextTitle{Naissance de Lo-Ammi}
\VS{8}Puis quand elle sevra Lo-Ruchama ; elle conçut, et enfanta un fils.
\VS{9}Et Yahweh dit : Appelle-le du nom de Lo-Ammi ; car vous n'êtes point mon peuple, et je ne suis pas votre Dieu.
\Chap{2}
\TextTitle{Futur rétablissement d'Israël}
\VerseOne{}Cependant le nombre des fils d'Israël sera comme le sable de la mer, qui ne peut ni se mesurer ni se compter ; et dans la ville où il leur est dit : Vous n'êtes pas mon peuple ! On leur dira : Vous êtes les fils du Dieu vivant !
\VS{2}Aussi les fils de Juda et les fils d'Israël se rassembleront, et ils s'établiront un chef, et monteront hors du pays ; car la journée de Jizreel sera grande.
\VS{3}Dites à vos frères : Ammi ! Et à vos sœurs : Ruchama !
\TextTitle{Châtiment d'Israël, la prostituée\FTNTT{2 R. 17:1-18.}}
\VS{4}Plaidez, plaidez contre votre mère, car elle n'est point ma femme, et je ne suis point son mari ! Qu'elle ôte ses prostitutions de son visage, et ses adultères de son sein !
\VS{5}De peur que je ne la dépouille à nu, et que je ne l'expose comme au jour de sa naissance, et que je ne la rende semblable à un désert, à une terre aride, et ne la fasse mourir de soif ;
\VS{6}et je n'aurai point de miséricorde pour ses enfants, car ce sont des enfants de prostitution.
\VS{7}Car leur mère s'est prostituée, celle qui les a conçus s'est déshonorée, car elle a dit : Je m'en irai après mes amants, qui me donnent mon pain et mes eaux, ma laine et mon lin, mon huile, et mes boissons.
\VS{8}C'est pourquoi voici, je vais fermer ton chemin avec des épines, j'y élèverai un mur, afin qu'elle ne trouve plus ses sentiers.
\VS{9}Elle poursuivra ses amants, mais ne les atteindra pas ; elle les cherchera, mais elle ne les trouvera point. Puis elle dira : Je m'en irai, et je retournerai vers mon premier mari, car alors j'étais plus heureuse que maintenant.
\VS{10}Mais elle n'a pas reconnu que c'était moi qui lui donnais le blé, le vin et l'huile ; et l'on a fait des offrandes à Baal\FTNT{Baal : Voir commentaire en Jg. 2:13.} avec l'argent et l'or que je lui prodiguais.
\VS{11}C'est pourquoi je reprendrai mon froment en son temps, mon vin en sa saison, et je retirerai ma laine et mon lin qui couvraient sa nudité.
\VS{12}Et maintenant je découvrirai sa honte aux yeux de ses amants, et personne ne la délivrera de ma main.
\VS{13}Je ferai cesser toute sa joie, ses fêtes, ses nouvelles lunes, ses sabbats, et toutes ses solennités.
\VS{14}Je ravagerai ses vignes et ses figuiers, dont elle disait : Voici le salaire que mes amants m'ont donné ! Je les réduirai en une forêt, et les bêtes des champs les dévoreront.
\VS{15}Je la punirai pour les jours où elle encensait les Baals, où elle se parait de ses anneaux et de ses colliers, et s'en allait après ses amants, et m'oubliait, dit Yahweh.
\TextTitle{La femme adultère revient dans son foyer : Israël revient à Yahweh}
\VS{16}Néanmoins, voici, je veux l'attirer et la mener au désert, là je parlerai à son cœur.
\VS{17}Là, je lui accorderai ses vignes et la vallée d'Acor, telle une porte d'espérance, et là, elle chantera comme au temps de sa jeunesse, et comme au jour où elle remonta du pays d'Egypte.
\VS{18} Et il arrivera en ce jour-là, dit Yahweh, tu m'appelleras: Mon Mari ! Et tu ne m'appelleras plus: Mon Maître\FTNT{Littéralement : Baal.} !
\VS{19}Car j'ôterai de sa bouche les noms des Baals, et on ne fera plus mention de leurs noms.
\VS{20}Aussi en ce temps-là, je traiterai pour eux une alliance avec les bêtes des champs, avec les oiseaux du ciel, et avec les reptiles de la terre ; je briserai et j'ôterai du pays l'arc, l'épée et la guerre, et je les ferai se reposer en sécurité.
\VS{21}Et je te fiancerai pour moi à toujours ; je te fiancerai, dis-je, pour moi, par la justice, la droiture, la grâce et la miséricorde;
\VS{22}je serai ton fiancé par la fidélité, et tu reconnaîtras Yahweh.
\VS{23}Et il arrivera en ce jour-là, j'exaucerai, dit Yahweh, je témoignerai aux cieux, et les cieux exauceront la terre;
\VS{24}la terre exaucera le blé, le bon vin et l'huile, et ils exauceront Jizreel. 
\VS{25}Puis, je la sèmerai pour moi dans ce pays, et je ferai miséricorde à Lo-Ruchama ; je dirai à Lo-Ammi : Tu es mon peuple ! Et il me répondra : Mon Dieu !
\Chap{3}
\TextTitle{Soumission d'Israël à Yahweh}
\VerseOne{}Après cela Yahweh me dit : Va encore, et aime une femme aimée d'un ami, et adultère; aime-la comme Yahweh aime les enfants d'Israël, qui se tournent toutefois vers d'autres dieux et aiment les gâteaux de raisins.
\VS{2}J'achetai donc cette femme pour quinze pièces d'argent, un homer et demi d'orge\FTNT{Voir dans les annexes les tableaux des poids et mesures.}.
\VS{3}Et je lui dis : Assieds-toi avec moi pendant plusieurs jours, ne t'abandonne plus à la prostitution, ne sois à aucun homme, et je serai fidèle envers toi.
\VS{4}Car les enfants d'Israël demeureront plusieurs jours sans roi, sans chef, sans sacrifice, sans statue, sans éphod, et sans théraphim\FTNT{Cette prophétie s'est accomplie d'une manière extraordinaire à travers l'histoire du peuple d'Israël depuis la première venue de Jésus-Christ. Les Israélites étaient dispersés, sans unité politique faute de roi, empêchés d'offrir des sacrifices depuis la destruction du temple par Titus (39-81), fils de l'empereur romain Vespasien (9-79), en l'an 70.}.
\VS{5}Mais après cela, les enfants d'Israël se repentiront\FTNT{La repentance et la conversion nationale d'Israël auront lieu lors du retour du Messie (Es. 59:20-21 ; Ro. 11:26-27). Dieu n'a pas abandonné son peuple, il viendra lui-même le délivrer.}; et rechercheront Yahweh, leur Dieu, et David, leur roi; ils seront dans la crainte à la vue de Yahweh et de sa bonté, dans les derniers jours\FTNT{Les derniers jours : Voir commentaire en Ge. 49:1.}.
\Chap{4}
\TextTitle{Israël, la nation pécheresse}
\VerseOne{}Ecoutez la parole de Yahweh, fils d'Israël ! Car Yahweh a un procès avec les habitants du pays, parce qu'il n'y a ni de vérité, ni de miséricorde, ni de connaissance de Dieu dans le pays.
\VS{2}Il n'y a que parjures et mensonges, meurtres, vols et adultères ; on use de violence, et un meurtre touche l'autre.
\VS{3}C'est pourquoi le pays sera dans le deuil, et tous ceux qui l'habitent seront languissants, et avec eux toutes les bêtes des champs et tous les oiseaux du ciel ; même les poissons de la mer périront.
\VS{4}Mais que nul ne conteste, et que nul ne reprenne ; car ton peuple est comme ceux qui disputent avec le sacrificateur.
\VS{5}Tu tomberas donc en plein jour, et le prophète aussi tombera avec toi de nuit, et j'exterminerai ta mère.
\TextTitle{Israël dans l'ignorance}
\VS{6}Mon peuple est détruit, parce qu'il lui manque la connaissance\FTNT{Le verbe « détruire » vient de l'hébreu « damah » qui signifie aussi « égorger ». Satan est celui qui vient égorger, dérober et détruire, notamment avec ses faux prophètes (Jn. 10:10). Chaque disciple de Jésus-Christ doit avoir une vie de prière et de méditation quotidienne afin de résister aux attaques de l'ennemi.}. Parce que tu as rejeté la connaissance, je te rejetterai, afin que tu n'exerces plus la sacrificature; puisque tu as oublié la loi de ton Dieu, moi aussi j'oublierai tes enfants.
\VS{7}Plus ils se sont multipliés, plus ils ont péché contre moi : Je changerai leur gloire en ignominie.
\VS{8}Ils se nourrissent des péchés de mon peuple, leur âme soutient leur iniquité.
\VS{9}C'est pourquoi le sacrificateur sera traité comme le peuple; je le châtierai selon ses voies, et je lui rendrai selon ses œuvres.
\VS{10}Et ils mangeront mais ils ne seront point rassasiés, ils se prostitueront mais ils ne multiplieront point, parce qu'ils ont cessé de prendre garde à Yahweh.
\VS{11}La prostitution, le vin et le moût, font perdre l'entendement.
\TextTitle{Israël dans l'idolâtrie}
\VS{12}Mon peuple consulte son bois, et c'est son bâton qui lui répond ; car l'esprit de prostitution égare, et ils se prostituent loin de leur Dieu.
\VS{13}Ils sacrifient sur le sommet des montagnes, ils brûlent de l'encens sur les collines, sous les chênes, sous les peupliers, et les térébinthes, parce que leur ombrage est agréable. C'est pourquoi vos filles se prostituent, et vos belles-filles commettent l'adultère.
\VS{14}Je ne punirai pas vos filles parce qu'elles se prostituent, ni vos belles-filles parce qu'elles commettent l'adultère, car eux-mêmes se retirent avec des prostituées, et sacrifient avec des femmes débauchées. Ainsi le peuple qui est sans intelligence sera ruiné.
\VS{15}Si tu te prostitues, ô Israël, au moins que Juda ne se rende point coupable ! N'entrez donc point dans Guilgal, et ne montez pas à Beth-Aven, et ne jurez point : Yahweh est vivant !
\VS{16}Parce qu'Israël se révolte comme une vache indomptable, maintenant Yahweh les fera paître comme des agneaux dans de vastes plaines.
\VS{17}Ephraïm s'est associé aux idoles ; abandonne-le !
\VS{18}Leur breuvage est devenu aigre ; ils n'ont fait que se prostituer ; ils n'aiment qu'à dire : apportez ; ce n'est qu'ignominie que ses protecteurs.
\VS{19}Le vent l'a enfermé dans ses ailes, et ils auront honte de leurs sacrifices.
\Chap{5}
\TextTitle{Yahweh abandonne son peuple}
\VerseOne{}Ecoutez ceci, sacrificateurs ! Maison d'Israël, sois attentive ! Maison du roi, tendez l'oreille ! Car c'est à vous que s'adresse le jugement, parce que vous avez été un piège à Mitspa, et un filet tendu sur le Thabor.
\VS{2}Les infidèles s'enfoncent dans le crime. Et moi, je les châtierai tous.
\VS{3}Je connais Ephraïm, et Israël ne m'est point caché; car maintenant, Ephraïm, tu t'es prostitué, et Israël est souillé.
\VS{4}Leurs œuvres ne leur permettent pas de revenir à leur Dieu, parce que l'esprit de prostitution est au milieu d'eux, et parce qu'ils ne connaissent point Yahweh.
\VS{5}L'orgueil d'Israël témoigne contre lui; Israël et Ephraïm tomberont par leur iniquité ; Juda aussi tombera avec eux.
\VS{6}Ils iront avec leurs brebis et leurs bœufs chercher Yahweh, mais ils ne le trouveront point, il s'est retiré du milieu d'eux.
\VS{7}Ils se sont montrés infidèles envers Yahweh, car ils ont engendré des fils étrangers ; maintenant un mois suffira pour les dévorer avec leurs biens.
\VS{8}Sonnez du shofar à Guibea. Sonnez de la trompette à Rama ! Poussez des cris de guerre à Beth-Aven ! Derrière toi, Benjamin !
\VS{9}Ephraïm sera un sujet d'épouvante au jour du châtiment ; je le fais savoir parmi les tribus d'Israël comme une chose certaine.
\VS{10}Les chefs de Juda sont comme ceux qui déplacent les bornes; je répandrai sur eux ma fureur comme un torrent.
\VS{11}Ephraïm est opprimé, brisé par le jugement, car il a vécu selon les préceptes qui lui plaisaient.
\VS{12}Je serai comme une teigne pour Ephraïm, comme de la pourriture pour la maison de Juda.
\VS{13}Ephraïm voit sa maladie, et Juda ses plaies ; Ephraïm s'en est allé vers le roi d'Assyrie, et s'est adressé au roi Jareb. Mais ce roi ne pourra ni vous guérir, ni panser vos plaies.
\VS{14}Je serai comme un lion pour Ephraïm, comme un lionceau pour la maison de Juda. Moi, moi je déchirerai, puis je m'en irai, j'emporterai la proie, et nul ne me l'enlèvera.
\TextTitle{Israël revient à Yahweh}
\VS{15}Je m'en irai, je reviendrai dans ma demeure, jusqu'à ce qu'ils se reconnaissent coupables, et qu'ils cherchent ma face. Quand ils seront dans la détresse, dans leur angoisse, ils me chercheront.
\Chap{6}
\VerseOne{}Venez, retournons à Yahweh ! Car il a déchiré, mais il nous guérira ; il a frappé, mais il bandera nos plaies.
\VS{2}Il nous rendra la vie dans deux jours; et le troisième jour il nous relèvera, et nous vivrons en sa présence.
\VS{3}Car nous connaîtrons Yahweh, et nous continuerons à le connaître ; sa venue\FTNT{Il est question ici du retour du Seigneur Jésus-Christ : Voir  commentaire en Za. 14:1.} est aussi certaine que celle de l'aurore. Il viendra pour nous comme la pluie, comme la pluie de l'arrière-saison\FTNT{Voir commentaire en Joë. 2:23.} qui arrose la terre.
\TextTitle{Yahweh dénonce le péché d'Ephraïm}
\VS{4}Que te ferai-je, Ephraïm ? Que te ferai-je, Juda ? Votre piété est comme la nuée du matin, comme la rosée qui se dissipe dès le matin.
\VS{5}C'est pourquoi je les taillerai en pièces par mes prophètes, je les tuerai par les paroles de ma bouche\FTNT{Hé. 4:12 ; Ap. 1:16 ; Ap. 19:15.}, et mes jugements apporteront la lumière.
\VS{6}Car je prends plaisir à la miséricorde et non aux sacrifices, et à la connaissance de Dieu plus qu'aux holocaustes.
\VS{7}Mais ils ont transgressé l'alliance, comme si elle avait été d'un homme, en quoi ils se sont portés perfidement contre moi.
\VS{8}Galaad est une ville d'ouvriers d'iniquité, couverte de traces de sang.
\VS{9}Et comme les bandes des voleurs attendent quelqu'un, ainsi les sacrificateurs, après avoir comploté, tuent les gens sur le chemin du côté de Sichem ; car ils exécutent leurs méchants desseins.
\VS{10}J'ai vu des choses infâmes dans la maison d'Israël: Là Ephraïm se prostitue, Israël en est souillé.
\VS{11}A toi aussi Juda, une moisson est préparée, quand je ramènerai les captifs de mon peuple.
\Chap{7}
\TextTitle{Transgression d'Ephraïm}
\VerseOne{}Lorsque je guérissais Israël, l'iniquité d'Ephraïm et la méchanceté de Samarie se sont révélées, car ils ont agi frauduleusement ; le voleur vient tandis que la bande dépouille au-dehors.
\VS{2}Ils n'ont point pensé dans leur cœur que je me souviens de toute leur méchanceté ; maintenant leurs œuvres les entourent, elles sont devant ma face.
\VS{3}Ils réjouissent le roi par leur méchanceté, et les chefs par leurs mensonges.
\VS{4}Ils sont tous adultères, comme un four allumé par le boulanger : Il cesse d'attiser le feu depuis qu'il a pétri la pâte jusqu'à ce qu'elle soit levée.
\VS{5}Au jour de notre roi, les chefs se rendent malades par les excès de vin ; il tend la main aux moqueurs.
\VS{6}Lorsqu'ils dressent des embuscades, leur cœur s'embrase comme un four ; leur boulanger dort toute la nuit, le matin le four est embrasé comme un feu accompagné de flammes.
\VS{7}Ils sont tous ardents comme un four, et ils dévorent leurs chefs ; tous leurs rois tombent, et il n'y a aucun d'entre eux qui crie à moi.
\VS{8}Ephraïm se mêle avec les peuples, Ephraïm est un gâteau qui n'a pas été retourné.
\VS{9}Les étrangers ont dévoré sa force, et il ne s'en doute pas ; les cheveux gris sont aussi parsemés sur lui, et il ne s'en doute pas.
\VS{10}L'orgueil d'Israël rendra témoignage contre lui ; car ils ne reviennent pas à Yahweh, leur Dieu, et ils ne le recherchent pas malgré tout cela. 
\VS{11}Ephraïm est comme une colombe troublée, sans intelligence ; car ils appellent l'Egypte, et s'en vont vers le roi d'Assyrie.
\VS{12}Quand ils s'en iront, j'étendrai mon filet sur eux, et je les précipiterai comme les oiseaux du ciel ; je les châtierai, comme ils en ont été avertis au sein de leurs assemblées.
\VS{13}Malheur à eux, parce qu'ils me fuient ! Ruine sur eux, car ils se révoltent contre moi ! Je voudrais les sauver, mais ils profèrent contre moi des paroles mensongères.
\VS{14}Ils ne crient pas vers moi dans leur cœur, mais ils gémissent sur leurs couches ; ils se rassemblent pour le froment et le bon vin, et ils s'éloignent de moi.
\VS{15}Je les ai châtiés, et j'ai fortifié leurs bras, mais ils méditent le mal contre moi.
\VS{16}Ce n'est pas au Très-Haut qu'ils retournent ; ils sont comme un arc trompeur. Leurs chefs tomberont par l'épée, à cause de l'insolence de leur langue. C'est ce qui en fera un objet de moquerie dans le pays d'Egypte.
\Chap{8}
\TextTitle{Conséquences de la désobéissance}
\VerseOne{}Crie comme si tu avais un shofar dans ta bouche ! Il vient comme un aigle contre la maison de Yahweh, parce qu'ils ont transgressé mon alliance, et qu'ils ont agi méchamment contre ma loi.
\VS{2}Ils crieront à moi : Mon Dieu, nous te connaissons, dira Israël !
\VS{3}Israël a rejeté le bien ; l'ennemi le poursuivra.
\VS{4}Ils ont fait régner, mais non pas de ma part, ils ont établi des chefs, et je n'en ai rien su ; ils se sont fait des idoles avec leur argent et leur or ; c'est pourquoi ils seront retranchés.
\VS{5}Samarie, ton veau t'a chassée loin ! Ma colère s'est embrasée contre eux. Jusqu'à quand ne pourront-ils pas s'adonner à l'innocence ?
\VS{6}Car il vient d'Israël, c'est un orfèvre qui l'a fait, et il n'est pas Dieu ; c'est pourquoi le veau de Samarie sera mis en pièces.
\VS{7}Parce qu'ils ont semé du vent, ils moissonneront la tempête ; ils n'auront pas un épi de blé ; le grain qui poussera ne donnera point de farine, et s'il en faisait, les étrangers la dévoreraient.
\VS{8}Israël est dévoré ! Il est maintenant parmi les nations comme un vase dont on ne se soucie pas.
\VS{9}Car ils sont montés vers le roi d'Assyrie, qui est un âne sauvage se tenant seul à part ; Ephraïm a fait des présents à ceux qui l'aimait.
\VS{10}Et parce qu'ils ont fait des présents aux nations, je les rassemblerai maintenant ; et ils commenceront à être amoindris à cause de l'impôt pour le roi des princes.
\VS{11}Parce qu'Ephraïm a fait plusieurs autels pour pécher, ils auront des autels pour pécher.
\VS{12}Je lui ai écrit les grandes choses de ma loi, mais elles sont estimées comme des lois étrangères.
\VS{13}Quant aux sacrifices qui me sont offerts, ils sacrifient de la chair, et la mangent ; mais Yahweh ne les accepte point. Et maintenant il se souviendra de leur iniquité, et punira leurs péchés ; ils retourneront en Egypte.
\VS{14}Israël a oublié celui qui l'a fait, et il a bâti des palais ; et Juda a multiplié les villes fortes ; c'est pourquoi j'enverrai le feu dans les villes de celui-ci, quand il aura dévoré les palais de celui-là.
\Chap{9}
\TextTitle{Ephraïm châtié et rejeté}
\VerseOne{}Israël, ne te réjouis point, ne sois pas dans l'allégresse, comme les autres peuples, de ce que tu t'es prostitué en abandonnant ton Dieu, de ce que tu as obtenu un salaire de tes amants dans toutes les aires à blé !
\VS{2}L'aire et la cuve ne les nourriront pas, et le vin doux les trompera.
\VS{3}Ils ne resteront pas dans le pays de Yahweh; Ephraïm retournera en Egypte, et ils mangeront en Assyrie ce qui est impur.
\VS{4}Ils ne feront pas d'aspersions de vin à Yahweh : Elles ne lui seraient point agréables. Leurs sacrifices seront pour eux comme le pain de deuil ; tous ceux qui en mangeront se rendront impurs ; car leur pain ne sera que pour eux, il n'entrera point dans la maison de Yahweh.
\VS{5}Que ferez-vous aux jours des fêtes solennelles, aux jours des fêtes de Yahweh ?
\VS{6}Car voici, ils partent à cause de la dévastation ; l'Egypte les recueillera, Moph les enterrera ; ce qu'ils ont de précieux, leur argent, sera la proie des ronces, et l'épine sera dans leurs tentes.
\VS{7}Les jours du châtiment sont venus, les jours de la rétribution sont venus, et Israël le saura ! Les prophètes sont fous, les hommes de révélation sont insensés, à cause de la grandeur de ton iniquité, et de ta grande aversion.
\VS{8}Ephraïm est une sentinelle avec mon Dieu ; mais le prophète est un filet d'oiseleur sur toutes ses voies, en rébellion contre la maison de son Dieu.
\VS{9}Ils se sont profondément corrompus, comme aux jours de Guibea ; Yahweh se souviendra de leur iniquité, il punira leurs péchés.
\VS{10}J'ai, dira-t-il, trouvé Israël comme des raisins dans le désert ; j'ai vu vos pères comme les premiers fruits d'un figuier ; mais ils sont allés vers Baal-Peor\FTNT{Baal-Peor, « seigneur de la brèche », était une divinité adorée à Peor avec des rites licencieux (No. 23:28 ; No. 25:1-3 ; Ps. 106:28-29).}, ils se sont consacrés à l'infâme idole, et ils sont devenus abominables comme ce qu'ils ont aimé.
\VS{11}La gloire d'Ephraïm s'envolera comme un oiseau : Point d'enfantement, point de grossesse, point de conception.
\VS{12}Que s'ils élèvent leurs enfants, je les en priverai tellement, que pas un d'entre eux ne deviendra homme ; malheur à eux, quand je me retirerai d'eux !
\VS{13}Ephraïm était comme j'ai vu Tyr, plantée dans un lieu agréable ; mais Ephraïm mènera ses fils à celui qui les tuera.
\VS{14}Ô Yahweh, donne-leur ! Mais que leur donnerais-tu ? Donne-leur un sein qui avorte et des mamelles desséchées.
\VS{15}Toute leur méchanceté s'est manifestée à Guilgal ; c'est là que je les ai pris en aversion. Je les chasserai de ma maison à cause de la malice de leurs actions. Je ne les aimerai plus ; tous leurs chefs sont des rebelles.
\VS{16}Ephraïm est frappé, sa racine est devenue sèche ; ils ne porteront plus de fruit ; et s'ils engendrent des enfants, je mettrai à mort les fruits désirables de leur ventre.
\VS{17}Mon Dieu les rejettera, parce qu'ils ne l'ont point écouté, et ils seront vagabonds parmi les nations.
\Chap{10}
\TextTitle{Yahweh annonce la destruction du royaume d'Israël}
\VerseOne{}Israël était une vigne dévastée, elle ne fait de fruit que pour elle-même. Selon l'abondance de son fruit, il a multiplié les autels ; selon la beauté de son pays, il a rendu belles ses statues.
\VS{2}Leur cœur est partagé. Ils vont être déclarés coupables. Yahweh renversera leurs autels, il détruira leurs statues.
\VS{3}Car bientôt ils diront : Nous n'avons point de roi, parce que nous n'avons point craint Yahweh ; et le roi, que pourrait-il faire pour nous ?
\VS{4}Ils prononcent des paroles vaines, des faux serments, lorsqu'ils concluent une alliance. C'est pourquoi le châtiment germera dans les sillons des champs, comme une plante vénéneuse.
\VS{5}Les habitants de Samarie seront épouvantés à cause des jeunes vaches de Beth-Aven; car le peuple mènera deuil sur son idole ; et les prêtres de ses idoles, qui s'en étaient réjouis, mèneront deuil parce que sa gloire est transportée loin d'elle.
\VS{6}Elle sera transportée en Assyrie, pour en faire un présent au roi Jareb. Ephraïm sera dans la confusion, et Israël aura honte de ses desseins.
\VS{7}C'en est fait de Samarie, et de son roi, qui sera retranché comme l'écume qui est à la surface des eaux.
\VS{8}Les hauts lieux de Beth-Aven, qui sont le péché d'Israël, seront détruits ; l'épine et la ronce croîtront sur leurs autels. Et on dira aux montagnes : Couvrez-nous ! Et aux collines : Tombez sur nous !
\VS{9}Israël, tu as péché dès les jours de Guibea ! Là ils restèrent debout, la guerre contre les fils d'iniquité ne les atteignit pas à Guibea.
\VS{10}Je les châtierai selon ma volonté, et les peuples s'assembleront contre eux, lorsqu'on les enchaînera pour leur double iniquité.
\VS{11}Ephraïm est une génisse bien dressée, qui aime à fouler le blé, mais je m'approcherai de son superbe cou ; j'attellerai Ephraïm, Juda labourera, Jacob brisera ses mottes.
\VS{12}Semez selon la justice, moissonnez selon la miséricorde, défrichez-vous un champ nouveau ! Car il est temps de chercher Yahweh, jusqu'à ce qu'il vienne, et répande sur vous sa justice.
\VS{13}Vous avez cultivé la méchanceté, et vous avez moissonné l'iniquité, vous avez mangé le fruit du mensonge; parce que vous avez eu confiance dans vos voies, dans la multitude de vos vaillants hommes.
\VS{14}Il s'élèvera un tumulte parmi ton peuple, et on détruira toutes tes forteresses, comme Schalman a détruit Beth-Arbel, au jour de la bataille, où la mère fut écrasée avec les enfants.
\VS{15}Béthel vous fera de même, à cause de votre extrême méchanceté ; le roi d'Israël sera entièrement exterminé dès l'aurore.
\Chap{11}
\TextTitle{L'amour de Yahweh pour Israël}
\VerseOne{}Quand Israël était jeune enfant, je l'aimais, et j'appelai mon fils hors d'Egypte\FTNT{La sortie des Hébreux de l'Egypte sous Moïse était une préfiguration de celle de Jésus-Christ lorsqu'il fuyait le massacre décrété par Hérode (Mt. 2:15).}.
\VS{2}Lorsqu'on les appelait ils se sont éloignés ; ils ont sacrifié aux Baals, et offert de l'encens aux idoles.
\VS{3}J'appris à Ephraïm à marcher en le prenant par les bras; et ils n'ont pas vu que je les guérissais.
\VS{4}Je les tirai avec des liens d'humanité, et avec des cordages d'amour, et je fus pour eux comme ceux qui enlèveraient le joug de dessus leur mâchoire, et je leur présentai de la nourriture.
\VS{5}Ils ne retourneront pas au pays d'Egypte ; mais le roi d'Assyrie sera leur roi, parce qu'ils n'ont point voulu revenir à moi.
\VS{6}L'épée fondra sur leurs villes, les réduira à néant, consumera leurs forces, et les dévorera, à cause des desseins qu'ils ont eus.
\VS{7}Mon peuple tient à se détourner de moi ; on les appelle vers le Très-Haut, mais aucun d'eux ne l'exalte.
\VS{8}Que ferai-je de toi, Ephraïm ? Te livrerais-je, Israël ? Te traiterai-je comme Adma ? Te rendrai-je semblable à Tseboïm ? Mon cœur s'agite au-dedans de moi, mes compassions sont émues.
\VS{9}Je n'exécuterai pas l'ardeur de ma colère, je ne reviendrai pas pour détruire Ephraïm ; car je suis Dieu, et non pas un homme, je suis le Saint au milieu de toi; et je n'entrerai point dans la ville.
\VS{10}Ils marcheront après Yahweh, qui rugira comme un lion\FTNT{Yahweh rugit comme un lion : Jésus-Christ est le lion de la tribu de Juda, car selon la chair, il est issu de la postérité de Juda (Lu. 3:23-38 ; Ap. 5:5). Le lion est le roi des animaux, or Jacob fut le premier a avoir annoncé la venue du Schilo, c'est-à-dire celui à qui appartient le sceptre (Ge. 49:8-12).}, et quand il rugira, les enfants accourront en hâte de la mer.
\VS{11}Ils accourront en hâte hors d'Egypte, comme des oiseaux, et hors du pays d'Assyrie, comme des colombes. Et je les ferai habiter dans leurs maisons, dit Yahweh.
\Chap{12}
\TextTitle{Dénonciation du péché d'Ephraïm}
\VerseOne{}Ephraïm m'entoure avec des mensonges, et la maison d'Israël avec des tromperies ; lorsque Juda erre sans frein vis-à-vis du Dieu Puissant, vis-à-vis du Saint fidèle.
\VS{2}Ephraïm se repaît de vent, et poursuit le vent d'orient ; il multiplie chaque jour le mensonge et la violence, et il traite alliance avec l'Assyrie, et l'on porte des huiles de senteur en Egypte.
\VS{3}Yahweh a aussi un procès avec Juda, et il punira Jacob pour sa conduite, il lui rendra selon ses œuvres.
\VS{4}Dans le ventre Jacob saisit son frère par le talon\FTNT{Ge. 25:26}, puis dans sa vigueur, il lutta avec Dieu\FTNT{Ge. 32:24-28.}.
\VS{5}Il lutta avec l'Ange, et il fut vainqueur, il pleura, et lui demanda grâce. Jacob l'avait rencontré à Béthel, et c'est là que Dieu nous a parlé.
\VS{6}Yahweh est le Dieu des armées ; Yahweh est son mémorial.
\VS{7}Et toi donc, reviens à ton Dieu, garde la miséricorde et la justice, et espère toujours en ton Dieu.
\VS{8}Ephraïm est un marchand\FTNT{Ephraïm est appelé « marchand », littéralement « Canaan ». Notez que l'ange de Laodicée s'exprime comme Ephraïm : « Je suis riche, je me suis enrichi… » (Ap. 3:14-19).}, qui a dans sa main des balances fausses, il aime à frauder.
\VS{9}Et Ephraïm dit : Quoi qu'il en soit, je suis devenu riche ; je me suis acquis des richesses ; c'est entièrement le produit de mon travail ; on ne trouvera en moi aucune iniquité, rien qui soit un péché.
\VS{10}Et moi, je suis Yahweh, ton Dieu, dès le pays d'Egypte ; je te ferai encore habiter dans des tentes, comme aux jours des fêtes solennelles.
\VS{11}Je parlerai par les prophètes, et je multiplierai les visions, et par les prophètes, je proposerai des paraboles.
\VS{12}Certainement Galaad n'est qu'iniquité, certainement ils ne seront que vanité. Ils sacrifient des bœufs dans Guilgal ; même leurs autels seront comme des monceaux de pierres sur les sillons des champs.
\VS{13}Jacob s'enfuit au pays de Syrie, et Israël servit pour une femme, et pour une femme il garda les troupeaux.
\VS{14}Par un prophète, Yahweh fit monter Israël hors d'Egypte, et par un prophète, Israël fut gardé.
\VS{15}Mais Ephraïm a provoqué Yahweh à une amère colère ; son Seigneur laissera sur lui le sang qu'il a répandu, et lui rendra ses mépris.
\Chap{13}
\TextTitle{Ephraïm persiste dans sa méchanceté}
\VerseOne{}Quand Ephraïm parlait, c'était une terreur ; il s'éleva en Israël. Mais il se rendit coupable par Baal, et mourut.
\VS{2}Et maintenant ils continuent de pécher, et se sont fait avec leur argent des images de fonte, des idoles selon leurs pensées; toutes sont un travail d'artisans, desquelles ils disent : Que les hommes qui sacrifient embrassent\FTNT{C'est une expression d'hommage.} les veaux !
\VS{3}C'est pourquoi ils seront comme la nuée du matin, et comme la rosée qui bientôt disparaît ; comme la balle qui est emportée par le vent hors de l'aire, comme la fumée sortant de la cheminée.
\VS{4}Et moi, je suis Yahweh, ton Dieu, dès le pays d'Egypte. Et tu ne devrais reconnaître d'autre dieu que moi, et il n'y a pas d'autre Sauveur que moi\FTNT{Yahweh dit qu'il n'y a pas d'autre sauveur que lui (Es. 43:11). Et les Ecrits de la Nouvelle Alliance nous présentent clairement notre Sauveur : Jésus-Christ (Mt.1:21; Ac.13:23 ; 2 Ti. 1:10 ; Tit : 1:4).}.
\VS{5}Je t'ai connu dans le désert, dans une terre aride.
\VS{6}Ils se sont rassasiés dans leurs pâturages ; ils se sont rassasiés, et leur cœur s'est enflé ; alors ils m'ont oublié.
\VS{7}Je serai pour eux comme un lion ; je les épierai sur la route comme un léopard.
\VS{8}Je les attaquerai, comme une ourse à qui on a enlevé ses petits, et je déchirerai l'enveloppe de leur cœur ; et là, je les dévorerai comme un lion ; les bêtes des champs les mettront en pièces.
\TextTitle{Châtiment d'Ephraïm}
\VS{9}Ta ruine, ô Israël, c'est que tu as été contre moi, alors que moi seul pouvais te secourir !
\VS{10}Où donc est ton roi ? Qu'il te délivre dans toutes tes villes ! Où sont tes juges, au sujet desquels tu as dit : Donne-moi un roi et des princes ?
\VS{11}Je t'ai donné un roi\FTNT{Ce passage concerne Saül, premier roi d'Israël (1 S. 8, 9 et 10)} dans ma colère, et je l'ôterai dans ma fureur.
\VS{12}L'iniquité d'Ephraïm est enveloppée, et son péché est mis en réserve.
\VS{13}Les douleurs comme de celle qui enfante le surprendront ; c'est un enfant qui n'est pas sage, qui, au temps marqué, ne sort pas du sein maternel.
\VS{14}Je les rachèterai de la puissance du scheol, je les délivrerai de la mort\FTNT{L'auteur de l'épître aux Hébreux applique ce passage à la victoire que le Seigneur Jésus-Christ a remportée face à la mort lors de sa résurrection (Hé. 2 :14-18). Depuis la chute d'Adam, les hommes ont toujours eu peur de la mort. Cette peur est d'autant plus forte de nos jours, car la plupart des gens sont angoissés par son aspect imprévisible, inévitable et par son non-sens. Et bien que beaucoup ne croient pas à l'existence de la vie après la mort (au paradis ou à l'enfer), la mort associée à l'annihilation, au non-être, apparaît d'autant plus monstrueuse et insupportable. Or notre Seigneur Jésus-Christ a vaincu la mort et il promet la vie éternelle à ceux qui croient en lui (Jn. 3:16 ; Jn. 5:24-29 ; Ap. 1:18). En plaçant notre foi en lui, nous avons non seulement la victoire sur la mort, mais aussi sur l'angoisse qu'elle produit dans le cœur de tout homme.} ; ô mort, où est ta peste ? Scheol, où est ta destruction\FTNT{1 Co. 15:55-57} ? Mais le repentir se cache à mes yeux !
\VS{15}Ephraïm a beau être fertile au milieu de ses frères, le vent d'orient, le vent de Yahweh s'élèvera du désert, viendra, desséchera ses sources et tarira ses fontaines. On pillera le trésor de tous ses objets précieux.
\VS{16}Samarie sera châtiée, car elle s'est rebellée contre son Dieu. Ils tomberont par l'épée; leurs petits enfants seront écrasés, et l'on fendra le ventre de leurs femmes enceintes.
\Chap{14}
\TextTitle{Bénédiction future d'Israël}
\VerseOne{}Israël, reviens à Yahweh ton Dieu ; car tu es tombé par ton iniquité.
\VS{2}Apportez avec vous des paroles, et revenez à Yahweh. Dites-lui : Pardonne toutes nos iniquités, et reçois le bien, pour le mettre à sa place! Et nous t'offrirons pour sacrifices la louange de nos lèvres.
\VS{3}L'Assyrie ne nous sauvera pas, nous ne monterons pas sur des chevaux, et nous ne dirons plus à l'ouvrage de nos mains : Notre dieu ! Car c'est auprès de toi que l'orphelin trouve de la compassion.
\VS{4}Je guérirai leur rébellion, et les aimerai volontairement ; parce que ma colère s'est détournée d'eux.
\VS{5}Je serai comme la rosée pour Israël ; il fleurira comme le lis, et il poussera ses racines comme le Liban.
\VS{6}Ses branches s'étendront, et sa magnificence sera comme celle de l'olivier, avec un parfum comme celui du Liban.
\VS{7}Ils reviendront s'asseoir à son ombre, et ils redonneront la vie au froment, et ils fleuriront comme la vigne ; et l'odeur de chacun d'eux sera comme celle du vin du Liban.
\VS{8}Ephraïm dira : Qu'ai-je à faire encore avec les idoles ? Je l'exaucerai, je le regarderai, je serai pour lui comme un cyprès verdoyant. C'est de moi que tu recevras ton fruit.
\VS{9}Qui est celui qui est sage ? Qu'il entende ces choses ! Et qui est celui qui est prudent ? Qu'il les connaisse ! Car les voies de Yahweh sont droites ; aussi les justes y marcheront, mais les rebelles y tomberont.
\PPE{}
\end{multicols}

%\clearpage\ShortTitle{Joël}\BookTitle{Joël}\BFont
\noindent\hrulefill
{\footnotesize
\textit{
\bigskip
{\centering{}
\\Auteur : Joël
\\(Heb. : Yow'el)
\\Signification : Yahweh est Dieu
\\Thème : Le jour de Yahweh
\\Date de rédaction : 9ème ou 8ème siècle av. J.-C.\\}
}
%\bigskip
\textit{
\\Joël, fils de Pethuel, exerça son ministère dans le royaume de Juda. Son message faisait suite à deux fléaux qui s’étaient abattus sur Juda, à savoir une invasion de sauterelles et la sécheresse. Il s’agissait d’un avertissement de Yahweh qui appelait le peuple à revenir à lui avec la promesse de le restaurer dans tout ce qu’il avait perdu. Joël annonça en outre l’effusion de l’Esprit sur toute chair dans un avenir lointain, prophétie ayant trouvé son accomplissement à la naissance de l’Eglise lors de la Pentecôte.\bigskip
}
}
\par\nobreak\noindent\hrulefill
\begin{multicols}{2}
\Chap{1}
\VerseOne{}La parole de Yahweh qui fut adressée à Joël, fils de Pethuel.
\VS{2}Anciens écoutez ceci ! Et vous, tous les habitants du pays, prêtez l'oreille ! Rien de pareil est-il arrivé de votre temps, ou même du temps de vos pères ?
\VS{3}Racontez-le à vos enfants, et que vos enfants le racontent à leurs enfants, et leurs enfants à la génération suivante !
\TextTitle{Désolation  après  l'invasion des sauterelles}
\VS{4}La sauterelle a dévoré les restes du gazam, le jélek a dévoré les restes de la sauterelle, et le hasil a dévoré les restes du jélek.
\VS{5}Ivrognes, réveillez-vous, et pleurez ; et vous tous buveurs de vin hurlez à cause du vin nouveau, parce qu’il est retranché de votre bouche.
\VS{6}Car une nation puissante et innombrable est montée contre mon pays. Elle a les dents d’un lion et les mâchoires d’un vieux lion.
\VS{7}Elle a réduit ma vigne en désert ; et a ôté l’écorce de mes figuiers ; elle les a entièrement dépouillés, et les a abattus, leurs branches en sont devenues blanches.
\VS{8}Lamente-toi, comme une jeune fille qui se revêt d'un sac pour pleurer le mari de sa jeunesse !
\VS{9}L'offrande et la libation sont retranchées de la maison de Yahweh, et les sacrificateurs qui font le service de Yahweh mènent deuil.
\VS{10}Les champs sont ravagés, la terre est dans le deuil ; parce que le blé est détruit, le moût est tari, l'huile est desséchée.
\VS{11}Les laboureurs sont confus, les vignerons gémissent, à cause du froment et de l'orge, car la moisson des champs est perdue.
\VS{12}La vigne est desséchée, le figuier languissant ; le grenadier, le palmier, le pommier, tous les arbres des champs ont séché, c'est pourquoi la joie a cessé parmi les fils de l’homme !
\VS{13}Sacrificateurs, ceignez-vous et pleurez ! Poussez des gémissements, vous qui faites le service de l’autel, hurlez, vous qui faites le service de mon Dieu ; entrez, passez la nuit vêtus de sacs car il est défendu à l'offrande et à la libation d’entrer dans la maison de votre Dieu.
\TextTitle{Désolation après la sécheresse et la famine}
\VS{14}Sanctifiez le jeûne, publiez l’assemblée solennelle, assemblez les anciens, et tous les habitants du pays dans la maison de Yahweh votre Dieu, et criez à Yahweh en disant :
\VS{15}Hélas ! Quel jour ! Car le jour de Yahweh\FTNT{Jour de Yahweh : Voir commentaire en Za. 14:1.} est proche : Il vient comme un ravage fait par le Tout-Puissant.
\VS{16}La nourriture n’est-elle pas retranchée sous nos yeux ? Et la joie et l'allégresse de la maison de notre Dieu ?
\VS{17}Les semances sont pourries sous leurs mottes, les magasins sont dévastés, les greniers sont renversés parce que le blé a manqué.
\VS{18}Ô combien ont gémi les bêtes, et dans quelle peine ont été les troupeaux de bœufs, parce qu’ils n’ont point de pâturage ! Aussi les troupeaux de brebis sont dévastés.
\VS{19}Ô Yahweh, je crierai à toi, car le feu a consumé les pâturages du désert, et la flamme a brûlé tous les arbres des champs.
\VS{20}Même toutes les bêtes des champs crient aussi vers toi ; car les torrents d’eau sont à sec, et le feu a consumé les pâturages du désert.
\Chap{2}
\TextTitle{Le jour de Yahweh, invasion future}
\VerseOne{}Sonnez du shofar en Sion, et sonnez avec un retentissement bruyant dans la montagne de ma sainteté ; que tous les habitants du pays tremblent ; car le jour de Yahweh vient ; car il est proche,
\VS{2}jour de ténèbres et d'obscurité, jour de nuées et de brouillards, il vient comme l'aurore s'étend sur les montagnes. Voici un peuple nombreux et puissant, tel qu’il n’y en a jamais eu, et qu’il n’y en aura jamais dans la suite des siècles.
\VS{3}Devant lui est un feu dévorant, et derrière lui la flamme brûle ; le pays était, avant sa venue, comme le jardin d’Eden, et après qu’il sera parti il sera comme un désert affreux ; et même il n’y aura rien qui lui échappe.
\VS{4}Leur aspect est comme l’aspect des chevaux, et ils courent comme des cavaliers.
\VS{5}C’est comme le bruit de chariots, quand ils sautent au sommet des montagnes, comme le bruit d’une flamme de feu, qui dévore le chaume, comme un peuple puissant rangé en bataille.
\VS{6}Les peuples tremblent en le voyant ; tous les visages en deviennent pâles et livides.
\VS{7}Ils courent comme des hommes vaillants, et montent sur les murailles comme des gens de guerre ; chacun va son chemin, sans se détourner de son chemin.
\VS{8}Ils ne se pressent point les uns les autres, chacun va son chemin ; ils se jettent au travers des épées sans être blessés.
\VS{9}Ils courent çà et là dans la ville, se précipitent sur les murailles, montent sur les maisons, entrent par les fenêtres comme le voleur.
\VS{10}La terre tremble devant eux, les cieux sont ébranlés, le soleil et la lune s’obscurcissent, et les étoiles retirent leur éclat.
\VS{11}Aussi Yahweh fait entendre sa voix devant son armée ; parce que son camp est très grand, car l'exécuteur de sa parole est puissant. Certainement le jour de Yahweh est grand et terrible. Qui peut le supporter ?
\TextTitle{Repentance et miséricorde}
\VS{12}Maintenant encore, dit Yahweh, revenez à moi de tout votre cœur, avec des jeûnes, avec des pleurs et des lamentations !
\VS{13}Déchirez vos cœurs et non vos vêtements, et revenez à Yahweh, votre Dieu ; car il est compatissant et miséricordieux, lent à la colère et riche en bonté, et il se repent d’avoir affligé.
\VS{14}Qui sait si Yahweh, votre Dieu, ne reviendra pas et ne se repentira pas, et s'il ne laissera point après lui la bénédiction, des offrandes et des libations ?
\VS{15}Sonnez du shofar en Sion ! Sanctifiez le jeûne, publiez l'assemblée solennelle !
\VS{16}Assemblez le peuple, sanctifiez la congrégation ! Réunissez les anciens, assemblez les enfants, même les nourrissons à la mamelle ! Que l’époux sorte de sa demeure, et l’épouse de sa chambre nuptiale !
\VS{17}Que les sacrificateurs qui font le service de Yahweh pleurent entre le portique et l'autel, et qu'ils disent : Yahweh ! Epargne ton peuple ! N’expose pas ton héritage à l'opprobre, que les nations n’en fassent pas un sujet de railleries ! Pourquoi dirait-on parmi les peuples : Où est leur Dieu ?
\TextTitle{Promesse de restauration}
\VS{18}Or Yahweh est jaloux pour son pays, et il est ému de compassion envers son peuple.
\VS{19}Yahweh répond et il dit à son peuple : Voici, je vous enverrai du blé, du moût, et de l'huile, et vous en serez rassasiés ; et je ne vous exposerai plus à l'opprobre parmi les nations.
\VS{20}J'éloignerai de vous l’armée venue du nord, je la chasserai vers une terre aride et déserte, son avant-garde dans la mer orientale, son arrière-garde dans la mer occidentale ; et sa puanteur montera, et son infection s’élèvera, après avoir fait de grandes choses.
\VS{21}Terre, ne crains pas, sois dans l’allégresse et réjouis-toi, car Yahweh fait de grandes choses !
\VS{22}Ne craignez point, bêtes des champs, car les pâturages du désert ont poussé leur jet, et même les arbres portent leur fruit ; le figuier et la vigne ont poussé avec vigueur.
\VS{23}Et vous, enfants de Sion, soyez dans l’allégresse et réjouissez-vous en Yahweh, votre Dieu, car il vous donnera la pluie selon sa justice, il vous enverra la pluie de la première\FTNT{La pluie de la première saison : En Orient, la première pluie tombe au moment des semailles d’automne. Elle est nécessaire afin que la semence puisse germer. Sous l'influence des pluies fertilisantes, les tendres pousses sortent du sol.} et de l’arrière-saison\FTNT{La pluie de l’arrière-saison : Elle tombe vers la fin de la saison, mûrit le grain et le prépare pour la moisson. C’est la pluie du printemps. Voir Jé. 5:24 ; Os. 6:1-3 ; Za. 10:1.}, au premier mois.
\VS{24}Et les aires se rempliront de blé, et les cuves regorgeront de moût et d'huile.
\VS{25}Ainsi je vous rendrai les fruits des années qu'ont dévoré la sauterelle, le jélek, le hasil et le gazam, ma grande armée que j’avais envoyée contre vous.
\VS{26}Vous aurez donc abondamment de quoi manger et être rassasiés, et vous louerez le Nom de Yahweh votre Dieu, qui aura fait pour vous des choses merveilleuses ; et mon peuple ne sera plus jamais dans la confusion.
\VS{27}Et vous saurez que je suis au milieu d'Israël, que je suis Yahweh, votre Dieu, et qu'il n'y en a point d'autre, et mon peuple ne sera plus jamais dans la confusion.
\TextTitle{La promesse de l'Esprit}
\VS{28}Et il arrivera après cela, que je répandrai mon Esprit sur toute chair\FTNT{Cette promesse s’est réalisée dans Actes 2. Elle se réalise encore aujourd’hui dans la vie de chaque enfant de Dieu. Enfin, elle sera pleinement réalisée lors du retour du Messie en Israël (Za. 12:10-14) puisque cette prophétie annonce la repentance nationale d’Israël (Ro. 11:26-27).} ; et vos fils et vos filles prophétiseront ; vos vieillards songeront des songes, et vos jeunes gens verront des visions.
\VS{29}Et même en ces jours-là, je répandrai mon Esprit sur les serviteurs et sur les servantes.
\TextTitle{Prodiges précédant le jour de Yahweh\FTNTT{Es. 13:9-10 ; 24:21-23 ; Ez. 32:7-10 ; Mt. 24:29-30.}}
\VS{30}Je ferai des prodiges dans les cieux et sur la terre, du sang, et du feu, et des colonnes de fumée ;
\VS{31}Le soleil se changera en ténèbres, et la lune en sang, avant que le grand et terrible jour de Yahweh vienne.
\VS{32}Et il arrivera que quiconque invoquera le Nom de Yahweh\FTNT{Quiconque invoquera le Nom de Yahweh sera sauvé. Ce passage nous confirme que Jésus-Christ est vraiment Yahweh. En effet, Paul, apôtre des païens, attribue le Nom de Yahweh et cette prophétie à Jésus-Christ (Ro. 10:9-13). C’est bien le Nom de Jésus-Christ qu'il faut invoquer pour être sauvé (Ac. 4:12 ; Ac. 9:21 ; 1 Co. 1:2). Les éditeurs de la Traduction du Monde Nouveau (bible des témoins de Jéhovah) se sont permis de « restituer » le Nom divin YHWH qui apparaît près de 6000 fois dans le Tanakh, en 237 endroits dans les écrits de la nouvelle alliance, alors qu’aucun ancien manuscrit de la nouvelle alliance (testament de Jésus) ne le contient. Ils affirment, sur la base d’éléments de preuves indirectes, que les scribes du IIème siècle remplaçaient le Nom divin dans la Nouvelle Alliance par « Seigneur » ou « Dieu ». Pour restituer ce Nom (YHWH), ils se basent sur les citations du Tanakh où celui-ci figure et sur des versions hébraïques de la nouvelle alliance dont la plus ancienne date du XIVème siècle pour la plupart des copies de textes plus anciens. On constate cependant qu’ils n’ont pas restitué le Nom divin en 1 Pierre 2:3 qui est pourtant une citation du Psaumes 34:8. Pourquoi ? 
Parce que l’application de ce texte à Jésus-Christ, la pierre rejetée, est évidente. Si ce texte du Tanakh mentionnant Yahweh est appliqué à Jésus que penser des autres ? Jésus-Christ est vraiment Yahweh qui s’est incarné pour nous sauver. D'ailleurs, le Nom de Jésus veut dire « YHWH est Sauveur » (Es. 7:14 ; Es. 9:5 ; Mt. 1 ; Lu. 1 ; 1 Ti. 3:16).} sera sauvé ; car le salut sera sur la montagne de Sion et dans Jérusalem, comme l’a dit Yahweh, et parmi les réchappés que Yahweh appellera.
\Chap{3}
\TextTitle{Rétablissement d'Israël\FTNTT{Es. 11:10-12 ; Jé. 23:5-8 ; Ez. 37:21-28 ; Ac. 15:15-17.}}
\VerseOne{}Car voici, en ces jours-là, et en ce temps-là, quand je ramènerai les captifs de Juda et de Jérusalem,
\TextTitle{Jugements des nations étrangères\FTNTT{Za. 12:2-3.}}
\VS{2}J'assemblerai toutes les nations\FTNT{Dieu rassemblera les nations dans la vallée de Josaphat (de l'hébreu « Yehowshaphat », « Yahweh a jugé ») pour leur jugement. Cette vallée est peut-être celle où le roi Josaphat remporta une grande victoire, avec beaucoup de facilité, sur les Moabites, les Ammonites et les Maonites (2 Ch. 20). Cette vallée s'étend à l'orient de Jérusalem, entre la ville et le Mont des Oliviers, et traverse le torrent de Cédron.}, et je les ferai descendre dans la vallée de Josaphat ; là, j'entrerai en jugement avec elles, à cause de mon peuple, et d'Israël, mon héritage, lequel ils ont dispersé parmi les nations, et parce qu’ils ont partagé entre eux mon pays ;
\VS{3}et qu'ils ont tiré mon peuple au sort ; ils ont donné l’enfant pour une prostituée, ils ont vendu la jeune fille pour du vin, et ils ont bu.
\VS{4}Et qu’ai-je aussi affaire de vous, Tyr et Sidon, et de vous, toutes les limites de la Palestine, me rendrez-vous ma récompense, ou voulez-vous m'irriter ? Je vous rendrai promptement et sans délai votre récompense sur votre tête.
\VS{5}Car vous avez pris mon argent et mon or ; et vous avez emporté dans vos temples ce que j’avais de plus précieux et de plus beau.
\VS{6}Vous avez vendu les enfants de Juda et de Jérusalem aux enfants des Grecs, afin de les éloigner de leur territoire.
\VS{7}Voici, je les ferai lever\FTNT{Le verbe lever vient de l'hébreu «'uwr » qui signifie « se réveiller », « éveiller », « être éveillé » , « inciter », « veiller »,  « se lever », « sortir de l’assoupissement », «prendre courage». Yahweh annonce le réveil des hébreux depuis les nations, d'où ils sont établis. Ce réveil est une prise de conscience qui aboutira au retour à la terre sainte.} du lieu où ils ont été transportés après que vous les avez vendus ; et je ferai retourner votre récompense sur votre tête.
\VS{8}Je vendrai donc vos fils et vos filles entre les mains des enfants de Juda, et ils les vendront à ceux de Séba, qui les transporteront vers une nation éloignée ; car Yahweh a parlé.
\VS{9}Publiez ceci parmi les nations ! Préparez la guerre ! Réveillez les hommes vaillants ! Qu’ils s’approchent, et qu’ils montent, tous les hommes de guerre !
\VS{10}Forgez des épées de vos hoyaux, et des lances de vos serpes ! Et que le faible dise : Je suis fort !
\VS{11}Hâtez-vous et venez, vous toutes les nations d'alentour, et rassemblez-vous ! Là, ô Yahweh, fais descendre tes hommes vaillants !
\VS{12}Que les nations se réveillent, et qu'elles montent à la vallée de Josaphat ! Car là je siégerai pour juger toutes les nations d'alentour.
\VS{13}Saisissez la faucille, car la moisson est mûre ! Venez, et descendez, car le pressoir est plein, les cuves regorgent ! Car leur méchanceté est grande,
\VS{14}Des multitudes, des multitudes, dans la vallée du jugement ; car le jour de Yahweh est proche, dans la vallée du jugement.
\VS{15}Le soleil et la lune s’obscurcissent, et les étoiles retirent leur éclat.
\VS{16}De Sion Yahweh rugit, de Jérusalem il fait entendre sa voix ; les cieux et la terre sont ébranlés. Mais Yahweh est le refuge pour son peuple, et la forteresse\FTNT{Jésus-Christ est notre rocher (commentaire Es. 8:14 ; Ps. 78:35 ; 1 Co. 10:4).} pour les enfants d’Israël.
\VS{17}Et vous saurez que je suis Yahweh, votre Dieu, qui habite à Sion, ma sainte montagne. Jérusalem sera sainte, et les étrangers n'y passeront plus.
\TextTitle{Restauration finale et pleine bénédiction du royaume}
\VS{18}Et il arrivera en ce jour-là, le moût ruissellera des montagnes, le lait coulera des collines, il y aura de l’eau dans tous les torrents de Juda ; et une source\FTNT{Jésus est celui qui fait jaillir en nous une source d’eau qui étanche notre soif à jamais et nous donne la vie éternelle (Jé. 2:13 ; Jé. 17:13 ; Ez. 47:1-12 ; Za. 14:8 ; Jn. 4:14 ; Ap. 22:1).} sortira de la maison de Yahweh, et arrosera la vallée de Sittim.
\VS{19}L'Egypte sera dévastée, Edom sera réduit en désert de désolation, à cause de la violence faite aux enfants de Juda, dont ils ont répandu le sang innocent dans leur pays.
\VS{20}Mais là, Judas sera éternellement habitée, et Jérusalem, d’âge en âge.
\VS{21}Et je nettoierai leur sang que je n’avais point nettoyé ; car Yahweh habite en Sion.

\PPE{}
\end{multicols}

%\clearpage\ShortTitle{Amos}\BookTitle{Amos}\BFont
\noindent\hrulefill
{\footnotesize
\textit{
\bigskip
{\centering{}
\\Auteur : Amos
\\(Heb. : Amowc)
\\Signification : Fardeau, porteur de fardeau
\\Thème : Jugement sur le péché
\\Date de rédaction : 8ème siècle av. J.-C.\\}
}
%\bigskip
\textit{
\\Originaire de Tekoa, Amos exerça son ministère dans le royaume du nord, au temps d’Ozias,  roi de Juda, et Jéroboam II, roi d’Israël. Il fut aussi le contemporain des prophètes Osée, Michée, Jonas et Esaïe.
%\bigskip
\\Alors que le peuple juif jouissait d’une certaine prospérité, l’immoralité et les sacrilèges prirent place dans le royaume. Amos avertit le peuple de son péché et du jugement qu'il encourait. Il lui rappela la bonté de Dieu et l’invita à revenir à Yahweh et à lui rester fidèle.\bigskip
}
}
\par\nobreak\noindent\hrulefill
\begin{multicols}{2}
\Chap{1}
\VerseOne{}Paroles d'Amos, berger de Tekoa, qui prophétisa sur Israël, du temps d’Ozias, roi de Juda, et de Jéroboam, fils de Joas, roi d'Israël, deux ans avant le tremblement de terre\FTNT{Za. 14:5}.
\VS{2}Il dit : Yahweh rugit de Sion, et fait entendre sa voix de Jérusalem. Les habitations des bergers sont en deuil, et le sommet du Carmel est desséché\FTNT{Jé. 25:30 ; Joë 3:16}.
\TextTitle{Yahweh annonce ses jugements sur les villes et les pays d'alentour}
\VS{3}Ainsi parle Yahweh : A cause de trois crimes de Damas, et même de quatre, je ne rappellerai point cela, mais je le ferai\FTNT{Il est question du jugement de Dieu.}, parce qu'ils ont foulé Galaad avec des herses de fer\FTNT{Es. 17:1}.
\VS{4}J'enverrai le feu dans la maison de Hazaël, et il dévorera le palais de Ben-Hadad.
\VS{5}Je briserai aussi les verrous de Damas, j'exterminerai de Bikath-Aven ses habitants, et de Beth-Eden celui qui tient le sceptre. Et le peuple de Syrie sera mené captif à Kir, dit Yahweh.
\VS{6}Ainsi parle Yahweh : A cause de trois crimes de Gaza, et même de quatre, je ne rappellerai point cela, mais je le ferai\FTNT{Il est question du jugement de Dieu.} parce qu'ils ont emmené des captifs en grand nombre pour les livrer à Edom\FTNT{Ez. 25:13-17}.
\VS{7}J'enverrai le feu dans les murs de Gaza, et il dévorera ses palais.
\VS{8}J'exterminerai d'Asdod les habitants, et d'Askalon celui qui tient le sceptre ; je tournerai ma main contre Ekron, et le reste des Philistins périra, dit le Seigneur, Yahweh.
\VS{9}Ainsi parle Yahweh : A cause de trois crimes de Tyr, et même de quatre, je ne rappellerai point cela, mais je le ferai\FTNT{Il est question du jugement de Dieu.}, parce qu'ils ont livré à Edom des captifs en grand nombre sans se souvenir de l'alliance fraternelle\FTNT{Ez. 26:2}.
\VS{10}J'enverrai le feu dans les murs de Tyr, et il dévorera ses palais.
\VS{11}Ainsi parle Yahweh : A cause de trois crimes d'Edom, et même de quatre, je ne rappellerai point cela,, mais je le ferai\FTNT{Il est question du jugement de Dieu.}, parce qu'il a poursuivi son frère avec l'épée, refoulant toute compassion, parce que sa colère déchire continuellement et qu'il garde sa fureur éternellement.
\VS{12}J'enverrai le feu dans Théman, et il dévorera les palais de Botsra\FTNT{Jé. 49:7 ; Abd. 1:9}.
\VS{13}Ainsi parle Yahweh : A cause de trois crimes des enfants d'Ammon, et même de quatre, je ne rappellerai point cela, mais je le ferai\FTNT{Il est question du jugement de Dieu.}, parce qu’ils ont fendu le ventre des femmes enceintes de Galaad pour étendre leurs frontières\FTNT{Ez. 21:33 ; So. 2:8}.
\VS{14}J'allumerai le feu dans les murs de Rabba, et il dévorera les palais, au bruit des cris de guerre au jour du combat, et au milieu de l’ouragan au jour de la tempête.
\VS{15}Et leur roi ira en captivité, lui et ses chefs, dit Yahweh.
\Chap{2}
\TextTitle{Suite des jugements prononcés sur les villes et les pays d'alentour}
\VerseOne{}Ainsi parle Yahweh : A cause de trois crimes de Moab, et même de quatre, je ne rappellerai point cela, mais je le ferai, parce qu'il a brûlé les os du roi d'Edom jusqu’à les calciner.
\VS{2}J'enverrai le feu dans Moab, et il dévorera les palais de Kerijoth ; et Moab périra dans le tumulte, au milieu des cris de guerre et du bruit du shofar\FTNT{Ez. 25:8-9}.
\VS{3}J'exterminerai les juges de son pays, et je tuerai tous ses chefs, dit Yahweh.
\TextTitle{Juda et Israël jugés à cause de leurs iniquités}
\VS{4}Ainsi parle Yahweh : A cause de trois crimes de Juda, et même de quatre, je ne rappellerai point cela, mais je le ferai, parce qu'ils ont rejeté la loi de Yahweh et n'ont point gardé ses ordonnances ; parce qu’ils ont été égarés par les mensonges après lesquels leurs pères ont marché.
\VS{5}J'enverrai le feu dans Juda, et il dévorera les palais de Jérusalem.
\VS{6}Ainsi parle Yahweh : A cause de trois crimes d'Israël, et même de quatre, je ne rappellerai point cela, mais je le ferai, parce qu'ils ont vendu le juste pour de l'argent, et le pauvre pour une paire de souliers.
\VS{7}Ils aspirent à voir la poussière de la terre sur la tête des misérables, et ils pervertissent la voie des pauvres. Le fils et le père vont vers la même jeune fille, pour profaner mon Saint Nom.
\VS{8}Ils se couchent près de chaque autel, sur les vêtements qu'ils ont pris en gage, et boivent dans la maison de leurs dieux le vin de ceux qu’ils châtient.
\VS{9}Pourtant j'ai détruit devant eux les Amoréens qui étaient hauts comme les cèdres et forts comme les chênes ; j'ai détruit son fruit en haut, et ses racines en bas\FTNT{No. 21:24 ; Jos. 24:8}.
\VS{10}Je vous ai fait monter du pays d'Egypte et je vous ai conduits dans le désert quarante ans pour que vous possédiez le pays des Amoréens.
\VS{11}J'ai suscité quelques-uns d'entre vos fils pour être prophètes, et quelques-uns d'entre vos jeunes gens pour être nazaréens\FTNT{Le mot nazaréen vient de l'hébreu nâzîr, de la racine nâzar qui signifie séparer. Il y avait deux types de nazaréens. Premièrement ceux qui étaient appelés par Dieu. Par exemple : Samson Jg. 13:1-7 ; Samuel 1 S. 1:11 ; Jean-Baptiste Lu. 1:15. Deuxièmement les personnes qui voulaient se consacrer à Dieu. No. 6:13.}. N'en est-il pas ainsi, enfants d'Israël ? dit Yahweh.
\VS{12}Mais vous avez fait boire du vin aux nazaréens, et vous avez donné cet ordre aux prophètes disant : Ne prophétisez pas\FTNT{Es. 30:10 ; Jé. 11:21 ; Mi. 2:6} !
\VS{13}Voici, je m’en vais fouler le lieu où vous habitez, comme un chariot plein de gerbes foule tout par où il passe.
\VS{14}Tellement que l’homme agile ne pourra pas fuir, et le fort ne pourra pas se servir de sa vigueur, et l’homme vaillant ne sauvera pas sa vie\FTNT{Jé. 46:6}.
\VS{15}Celui qui manie l'arc, ne pourra pas tenir ferme, et celui qui a les pieds légers n'échappera pas, et le cavalier ne sauvera pas sa vie.
\VS{16}Le plus courageux d’entre les hommes vaillants s'enfuira tout nu en ce jour-là, dit Yahweh.
\Chap{3}
\TextTitle{La maison de Jacob coupable devant Yahweh}
\VerseOne{}Enfants d’Israël écoutez la parole que Yahweh prononce contre vous, contre toutes les familles que j'ai fait monter du pays d'Egypte.
\VS{2}Je vous ai connu vous seuls d'entre toutes les familles de la terre ; c'est pourquoi je vous châtierai pour toutes vos iniquités\FTNT{Ex. 19:5-6 ; Ps. 147:19-20}.
\VS{3}Deux hommes marchent-ils ensemble s’ils ne sont pas accordés ?
\VS{4}Le lion rugit-il dans la forêt sans qu’il n'ait de proie ? Le lionceau jette-t-il son cri de sa tanière sans qu’il n'ait rien attrapé ?
\VS{5}L'oiseau tombe-t-il dans le filet posé à terre sans que ce ne soit un piège ? Le filet est-il ramassé par terre sans qu’il n’y ait rien de capturé ?
\VS{6}Le shofar sonne-t-il dans une ville sans que le peuple en étant tout effrayé s’assemble ? Arrive-t-il un malheur dans une ville sans que Yahweh ne l’ait causé\FTNT{Es. 45:7 ; La. 3:37-38} ?
\VS{7}Car le Seigneur, Yahweh, ne fait aucune chose qu'il n'ait révélé son secret aux prophètes ses serviteurs.
\VS{8}Le lion rugit, qui ne serait pas effrayé ? Le Seigneur, Yahweh, parle, qui ne prophétiserait\FTNT{Lorsque Yahweh parle, des prophètes sont suscités : Jé. 20:9 ; Mi. 3:8 ; Ac. 4:20.} ?
\VS{9}Faites entendre votre voix dans les palais d'Asdod, et dans les palais du pays d'Egypte, et dites : Assemblez-vous sur les montagnes de Samarie, et voyez l’important tumulte interne et quelles oppressions dans son sein !
\VS{10}Ils ne savent pas faire ce qui est droit, dit Yahweh, ils amassent la violence et la rapine dans leurs palais.
\VS{11}C'est pourquoi ainsi parle le Seigneur, Yahweh : L'ennemi viendra, il cernera le pays, il t'ôtera ta force et tes palais seront pillés.
\VS{12}Ainsi parle Yahweh : Comme un berger arrache de la gueule d'un lion deux jambes ou un bout d'oreille, ainsi les enfants d'Israël qui habitent dans Samarie seront arrachés de l’angle d’un lit et de l’asile de Damas.
\VS{13}Ecoutez et soyez mes témoins contre la maison de Jacob, dit le Seigneur, Yahweh, le Dieu des armées :
\VS{14}Le jour où je punirai Israël pour ses péchés, j’exercerai mon châtiment sur les autels de Béthel ; les cornes de l'autel seront brisées, et tomberont à terre.
\VS{15}J’abattrai la maison d'hiver et la maison d'été ; les maisons d'ivoire seront détruites, et un grand nombre de maisons disparaîtront, dit Yahweh.
\Chap{4}
\TextTitle{Yahweh condamne les sacrifices du peuple}
\VerseOne{}Ecoutez cette parole, vaches de Basan, qui êtes sur la montagne de Samarie, vous qui opprimez les faibles, qui maltraitez les pauvres, qui dites à leurs maîtres : Apportez, et que nous buvions !
\VS{2}Le Seigneur, Yahweh, l’a juré par sa sainteté : Voici, les jours viennent sur vous, où l’on vous enlèvera avec des hameçons, et votre postérité avec des crochets de pêche\FTNT{Jé. 16:16 ; Ha. 1.14-16}.
\VS{3}Vous sortirez dehors par les brèches, chacune devant soi, et vous serez jetées dans la forteresse, dit Yahweh.
\VS{4}Entrez dans Béthel, et péchez ! Multipliez vos péchés dans Guilgal ! Amenez vos sacrifices dès le matin, et vos dîmes tous les trois ans\FTNT{Voir commentaires en Mal. 3:10 et No. 18:21.} !
\VS{5}Brûlez de l’encens avec du pain levé pour l’offrande de remerciement ; proclamez et publiez les offrandes volontaires ; car c’est là ce que vous aimez, enfants d'Israël, dit le Seigneur, Yahweh\FTNT{Lé. 2:1}.
\TextTitle{Endurcissement du peuple malgré les châtiments de Yahweh}
\VS{6}C'est pourquoi je vous ai envoyé la famine dans toutes vos villes, et la disette de pain dans toutes vos demeures ; mais malgré cela, vous n’êtes pas revenus vers moi, dit Yahweh.
\VS{7}Je vous ai aussi privés de pluie, alors qu’il restait encore trois mois jusqu'à la moisson ; j'ai fait pleuvoir sur une ville et je n'ai pas fait pleuvoir sur une autre ville ; une parcelle a été arrosée par la pluie, et l'autre parcelle, sur laquelle il n'a pas plu, est desséchée\FTNT{1 R. 8:35 ; 1 R. 17:1 ; Es. 5:6 ; Ag. 1:11}.
\VS{8}Et deux, même trois villes sont allées vers une autre ville pour boire de l'eau et n'ont pas été désaltérées, mais vous n’êtes pas revenus vers moi, dit Yahweh.
\VS{9}Je vous ai frappés par la rouille et par la nielle, et la sauterelle a brouté autant de jardins et de vignes, de figuiers et d'oliviers que vous aviez, mais vous n’êtes pas revenus vers moi, dit Yahweh\FTNT{De. 28:22-39 ; 1 R. 8:37 ; Ag. 2:17 ; 2 Ch. 6:28}.
\VS{10}J’ai envoyé parmi vous la peste comme celle en Egypte ; j'ai tué par l'épée vos jeunes hommes et vos chevaux en captivité ; j'ai fait remonter, jusque dans votre nez, la puanteur de vos camps ; mais vous n’êtes pas revenus vers moi, dit Yahweh\FTNT{Ez. 14:19}.
\VS{11}Je vous ai détruits comme Dieu détruisit Sodome et Gomorrhe, et vous avez été comme un tison arraché du feu, mais vous n’êtes pas revenus vers moi, dit Yahweh\FTNT{Ge. 19:24 ; Jé. 49:18 ; Za. 3:2},
\VS{12}C'est pourquoi je te traiterai de la même manière ô Israël ; et parce que je te traiterai ainsi, prépare-toi à la rencontre de ton Dieu, ô Israël !
\VS{13}Car voici celui qui a formé les montagnes et créé le vent, et qui déclare à l'homme quelle est sa pensée, qui fait l'aube et l'obscurité, et qui marche sur les hauteurs de la terre ; son nom est Yahweh, le Dieu des armées.
\Chap{5}
\TextTitle{Israël invité à revenir entièrement à Yahweh}
\VerseOne{}Ecoutez cette parole, cette complainte que je prononce sur vous, maison d'Israël !
\VS{2}Elle est tombée, elle ne se relèvera plus, la vierge d'Israël ; elle est couchée par terre, et personne ne la relève.
\VS{3}Car ainsi a parlé le Seigneur, Yahweh : La ville qui mettait en campagne mille hommes n'en aura de reste que cent ; et celle qui mettait en campagne cent hommes n'en aura de reste que dix dans la maison d’Israël.
\VS{4}Car ainsi a parlé Yahweh à la maison d'Israël : Cherchez-moi, et vous vivrez !
\VS{5}Ne cherchez pas Béthel, et n'allez pas à Guilgal, et ne passez point à Beer-Schéba. Car Guilgal sera transportée en captivité, et Béthel sera détruite\FTNT{Os. 4:15}.
\VS{6}Cherchez Yahweh, et vous vivrez, de peur qu'il ne saisisse comme un feu la maison de Joseph, et que ce feu ne la consume, sans qu'il y ait personne à Béthel pour l’éteindre.
\VS{7}Ils changent le jugement en absinthe, et ils foulent à terre la justice\FTNT{Es. 5:26-28 ; Ha. 1:1-3}.
\VS{8}Celui qui a créé les Pléiades et l'Orion, qui change les profonds ténèbres en aurore, et qui obscurcit le jour en nuit, qui appelle les eaux de la mer, et les répand sur la surface de la Terre, Yahweh est son nom\FTNT{Es. 58:8-10 ; Job 9:9 ; Job 38:31}.
\VS{9}Il fait éclater la ruine sur les puissants, et la ruine vient sur les forteresses.
\VS{10}Ils haïssent à la porte ceux qui les reprennent, et ils ont en abomination celui qui parle en intégrité.
\VS{11}C'est pourquoi, puisque vous opprimez le pauvre, et que vous prenez de lui du blé en présent, vous avez bâti des maisons en pierres de taille, mais vous n'y habiterez pas ; vous avez planté des vignes délicieuses, mais vous n'en boirez pas le vin.
\VS{12}Car j'ai connu vos crimes, ils sont en grand nombre, et vos péchés se sont multipliés : Vous opprimez le juste, vous recevez des présents, et vous violez à la porte le droit des pauvres.
\VS{13}C'est pourquoi, en ce temps-ci, le sage se tait, car les temps sont mauvais.
\VS{14}Recherchez le bien et non le mal, afin que vous viviez ; et qu’ainsi Yahweh, le Dieu des armées, soit avec vous, comme vous l'avez dit.
\VS{15}Haïssez le mal, et aimez le bien, faites régner la justice à la porte ; peut-être Yahweh, le Dieu des armées, aura pitié des restes de Joseph.
\TextTitle{Le jour de Yahweh}
\VS{16}C'est pourquoi ainsi parle Yahweh, le Dieu des armées, le Seigneur, parle ainsi : Dans toutes les places on se lamentera, dans toutes les rues on dira : Hélas ! Hélas ! On appellera au deuil le laboureur, et à la lamentation ceux qui en savent le métier.
\VS{17}Dans toutes les vignes on se lamentera, quand je passerai au milieu de toi, dit Yahweh.
\VS{18}Malheur à ceux qui désirent le jour de Yahweh\FTNT{L’expression «~le jour du Seigneur~» ou «~le jour de Yahweh~» est une période durant laquelle Jésus-Christ interviendra ouvertement dans les affaires des hommes. Elle est utilisé dix-neuf fois dans le Tanakh (Es. 2:12 ; Es. 13:6-9 ; Ez. 13:5 ; Ez. 30:3 ; Joë. 1:15 ; Joë. 2:1, 11, 31 ; Joë. 3:14 ; Am. 5:18, 20 ; Ab. 1:15 ; So. 1:7, 14 ; Za. 14:1 ; Mal. 4:5) et quatre fois dans le Testament de Jésus (Ac. 2:20 ; 2 Th. 2:2 ; 2 Pi. 3:10). On y fait également allusion dans d’autres passages (Ap. 6:17 ; Ap. 16:14).}. Qu’attendez-vous du jour de Yahweh ? Ce sont des ténèbres, et non pas une lumière.
\VS{19}Vous serez comme un homme qui fuit devant un lion et qui rencontre un ours, ou qui entre dans sa maison, appuie sa main sur le mur et un serpent le mord.
\TextTitle{Mépris du droit et de la justice}
\VS{20}Le jour de Yahweh n’est-il pas ténèbres et non lumière ? Obscurité et non clarté ?
\VS{21}Je hais, je méprise vos fêtes, je ne prends pas plaisir à vos assemblées solennelles.
\VS{22}Si vous me présentez des holocaustes, je n’agréerai pas vos offrandes, je ne regarderai pas les bêtes grasses de vos offrandes de paix.
\VS{23}Eloigne de moi le bruit de tes cantiques ; je n’écouterai pas la mélodie de tes luths.
\VS{24}Mais le jugement roule comme de l'eau, et la justice comme un torrent intarissable.
\VS{25}Est-ce à moi, maison d'Israël, que vous avez offert des sacrifices et des gâteaux dans le désert pendant quarante ans ? 
\VS{26}Au contraire, vous avez porté la tente de votre roi, et de vos idoles Kijun\FTNT{«~Kijun~» : Probablement une statue d'un dieu Assyro-Babylonien de la planète Saturne et utilisé pour symboliser l'apostasie d'Israél}, l'étoile de votre dieu que vous vous êtes fabriqué.
\VS{27}C'est pourquoi je vous transporterai au-delà de Damas, dit Yahweh, dont le nom est le Dieu des armées.
\Chap{6}
\TextTitle{Ceux qui prospèrent seront emmenés captifs}
\VerseOne{}Hélas vous qui êtes à votre aise en Sion, et qui vous confiez en la montagne de Samarie, lieux les plus renommés d’entre les principaux des nations, auprès desquels va la maison d'Israël.
\VS{2}Passez à Calné, et regardez ; allez de là à Hamath la grande, puis descendez à Gath chez les Philistins. Ces villes sont-elles plus prospères que vos deux royaumes, ou leur pays n'est-il pas plus étendu que votre pays ?
\VS{3}Vous qui éloignez le jour du malheur, et qui approchez le règne de la violence.
\VS{4}Vous qui vous couchez sur des lits d'ivoire, et qui sont étendus sur vos coussins ; qui mangez les agneaux du troupeau, et les veaux pris du lieu où on les engraisse ;
\VS{5}qui fredonnez au son du luth ; qui inventez des instruments de musique comme David.
\VS{6}Qui buvez le vin dans de grandes coupes, et qui parfumez des parfums les plus exquis, et qui n’êtes pas affligés pour la plaie de Joseph.
\VS{7}A cause de cela ils vont être emmenés à la tête des captifs, et les cris de joie de ces personnes voluptueuses prendront fin.
\VS{8}Le Seigneur, Yahweh, l’a juré par lui-même. Yahweh, Dieu des armées, dit : J'ai en détestation l'orgueil de Jacob, et j'ai en haine ses palais, c'est pourquoi je livrerai la ville, et tout ce qui est en elle.
\VS{9}Et si il reste dix hommes dans une maison, ils mourront.
\VS{10}Un proche parent prendra un mort et le brûlera pour emporter les os hors de la maison ; il dira à celui qui est au fond de la maison : Y a-t-il encore quelqu'un avec toi ? Et il répondra : Il n’y a plus personne. Puis il dira : Silence ! Ce n’est pas le moment de prononcer le nom de Yahweh.
\VS{11}Car voici, Yahweh ordonne : Et il frappera les grandes maisons par des débordements d'eau, et la petite maison en débris.
\VS{12}Les chevaux courent-ils sur les rochers, y laboure-t-on avec des bœufs, pour que vous ayez changé la droiture en poison, et le fruit de la justice en absinthe ?
\VS{13}Vous vous réjouissez de choses qui ne sont que néant, vous dites : N’est-ce pas par notre force que nous avons acquis de la puissance ?
\VS{14}Voici, je ferai lever contre vous, maison d’Israël, dit Yahweh, le Dieu des armées, une nation qui vous opprimera depuis l'entrée de Hamath jusqu'au torrent du désert.
\Chap{7}
\TextTitle{Avertissement\FTNTT{Am. 8:1 ; 9:10}}
\VerseOne{}Le Seigneur, Yahweh, me fit voir cette vision : Voici, il formait des sauterelles au temps où le regain commençait à croître ; et voici le regain poussait après les récoltes du roi.
\VS{2}Et quand elles eurent achevé de dévorer l'herbe de la terre, je dis : Seigneur Yahweh, pardonne, je te prie ! Comment Jacob subsistera-t-il ? Car il est faible.
\VS{3}Yahweh se repentit de cela. Cela n'arrivera pas, dit Yahweh.
\VS{4}Le Seigneur, Yahweh, me fit voir cette vision : Voici, le Seigneur, Yahweh, proclamait le jugement par le feu. Et le feu dévorait le grand abîme et dévorait les champs.
\VS{5}Et je dis : Seigneur Yahweh ! Arrête, je te prie ! Comment Jacob subsistera-t-il ? Car il est faible.
\VS{6}Yahweh se repentit de cela. Cela non plus n'arrivera pas, dit le Seigneur, Yahweh.
\VS{7}Il me fit voir cette vision : Voici, le Seigneur se tenait debout sur un mur fait au niveau, et il avait un niveau dans la main.
\VS{8}Et Yahweh me dit : Que vois-tu, Amos ? Et je répondis : Un niveau. Et le Seigneur me dit : Je mettrai le niveau au milieu de mon peuple d'Israël, je ne lui pardonnerai plus.
\VS{9}Et les hauts lieux d'Isaac seront ravagés, et les sanctuaires d'Israël seront détruits ; et je me lèverai contre la maison de Jéroboam avec l'épée.
\TextTitle{Amatsia accuse Amos devant Jéroboam}
\VS{10}Alors Amatsia, sacrificateur de Béthel, fit dire à Jéroboam roi d'Israël : Amos conspire contre toi au milieu de la maison d'Israël ; le pays ne saurait supporter toutes ses paroles.
\VS{11}Car voici ce que dit Amos : Jéroboam mourra par l'épée, et Israël sera emmené captif hors de son pays.
\VS{12}Et Amatsia dit à Amos : Voyant\FTNT{Voyant ou prophète.}, va-t’en, fuis dans le pays de Juda, et manges-y ton pain, et là tu prophétiseras.
\VS{13}Mais ne continue pas à prophétiser à Béthel\FTNT{Bethel, qui signifie «~maison de Dieu~», était devenue le sanctuaire du roi Jéroboam. De même aujourd’hui des églises du Seigneur sont devenues la propriété des hommes et les brebis de Dieu sont devenues la propriété des pasteurs.}, car c'est le sanctuaire du roi, et c'est une maison royale.
\TextTitle{Amos répond}
\VS{14}Amos répondit à Amatsia : Je n'étais ni prophète ni fils de prophète ; j'étais un berger, et je cueillais des figues sauvages.
\VS{15}Or Yahweh m’a pris derrière le troupeau, et Yahweh m’a dit : Va, prophétise à mon peuple d'Israël.
\VS{16}Ecoute maintenant la parole de Yahweh, tu dis : Ne prophétise pas contre Israël, et ne parle pas contre la maison d'Isaac.
\VS{17}C'est pourquoi ainsi parle Yahweh : Ta femme se prostituera dans la ville, tes fils et tes filles tomberont par l'épée, ton champ sera partagé au cordeau, et toi, tu mourras sur une terre souillée, et Israël sera emmené captif hors de son pays.
\Chap{8}
\TextTitle{Vision du panier de fruit, la fin pour le peuple d'Israël}
\VerseOne{}Le Seigneur, Yahweh, me fit voir cette vision : Voici, je vis un panier de fruits d'été.
\VS{2}Il dit : Que vois-tu, Amos ? Et je répondis : Un panier de fruits. Et Yahweh me dit : La fin est venue pour mon peuple d'Israël, je ne continuerai plus à lui pardonner.
\VS{3}En ce jour-là, les chants du palais seront des gémissements, dit le Seigneur, Yahweh ; en tout lieu, il y aura beaucoup de cadavres que l'on jettera en silence.
\VS{4}Ecoutez ceci vous qui dévorez les pauvres et qui faites périr les pauvres misérables du pays,
\VS{5}et qui dites : Quand la nouvelle lune sera-t-elle passée pour que nous vendions du blé ? Quand finira le sabbat pour que nous ouvrions les greniers ? Nous diminuerons l’épha, nous augmenterons le sicle, nous falsifierons les balances pour tromper,
\VS{6}nous achèterons les faibles pour de l’argent, et le pauvre pour une paire de souliers, et nous vendrons la criblure du froment.
\VS{7}Yahweh l’a juré par la gloire de Jacob : Jamais je n’oublierai toutes leurs actions !
\VS{8}La terre ne sera-t-elle point émue d'une telle chose, et tous ses habitants ne se lamenteront-ils point ? Le pays tout entier montera comme le fleuve. Il se soulèvera et s’affaissera comme le fleuve d'Egypte.
\VS{9}Il arrivera en ce jour-là, dit le Seigneur, Yahweh, que je ferai coucher le soleil à midi, et que j’obscurcirai la terre en plein jour.
\VS{10}Je changerai vos fêtes en deuil, et tous vos chants en lamentations ; je couvrirai de sacs tous les reins, et je rendrai chauves toutes les têtes ; je mettrai le pays dans le deuil comme pour un fils unique, et sa fin sera un jour d’amertume.
\VS{11}Voici, les jours viennent, dit le Seigneur, Yahweh, où j'enverrai la famine\FTNT{Nous sommes dans une époque où la Parole de Dieu, l’Evangile véritable a presque disparu au profit de l’évangile de prospérité. L’esprit de commerce a pris place au sein de beaucoup d’églises. C’est le temps de l’église de Laodicée, une église qui fait l’apologie de la richesse matérielle.} dans le pays ; non une famine de pain, ni une soif d'eau, mais d’entendre les paroles de Yahweh.
\VS{12}Ils erreront d’une mer jusqu'à l'autre, du nord à l'orient, ils iront çà et là pour chercher la parole de Yahweh, et ils ne la trouveront pas.
\VS{13}En ce jour-là, les belles vierges et les jeunes hommes mourront de soif.
\VS{14}Ceux qui jurent par le péché de Samarie disent : Vive ton Dieu, ô Dan ! Vive la voie de Beer-Schéba ! Mais ils tomberont et ne se relèveront plus.
\Chap{9}
\TextTitle{Prophétie annonçant la destruction\FTNTT{De. 28:63-68.}}
\VerseOne{}Je vis le Seigneur qui se tenait debout sur l'autel. Et il dit : Frappe le chapiteau et que les seuils s’ébranlent ; et brise-les sur leurs têtes à tous ! Je tuerai par l'épée ce qui restera d'eux. Il ne s’enfuira pas un fugitif, il ne s’échappera pas un fuyard.
\VS{2}S’ils pénètrent dans le séjour des morts, ma main les enlèvera de là ; s’ils montent aux cieux, je les en ferai descendre.
\VS{3}S’ils se cachent au sommet du Carmel, je les y rechercherai et je les enlèverai de là ; s’ils se dérobent à mes yeux dans le fond de la mer, là j’ordonnerai au serpent de les mordre.
\VS{4}Lorsqu'ils s'en iront en captivité devant leurs ennemis, là j’ordonnerai à l’épée de les tuer ; je fixerai mon regard sur eux pour leur faire du mal et non du bien.
\VS{5}Le Seigneur, Yahweh des armées, touche la terre, et elle tremble, et tous ses habitants sont dans le deuil ; elle monte tout entière comme le fleuve, et elle s’affaisse comme le fleuve d'Egypte.
\VS{6}Il a bâti sa demeure dans les cieux, et fondé sa voûte sur la terre ; il appelle les eaux de la mer, et les répand sur la surface de la Terre. Son nom est Yahweh.
\VS{7}N'êtes-vous pas pour moi comme les enfants des Ethiopiens, enfants d'Israël ? dit Yahweh. N'ai-je pas fait monter Israël du pays d'Egypte, les Philistins de Caphtor et les Syriens de Kir ?
\VS{8}Voici, les yeux du Seigneur, Yahweh, sont sur ce royaume pécheur. Je le détruirai de dessus la surface de la terre. Cependant, je ne détruirai pas entièrement la maison de Jacob, dit Yahweh.
\VS{9}Car voici, je donnerai mes ordres, et je secouerai la maison d'Israël parmi toutes les nations, comme on secoue le blé dans le crible, sans qu'il en tombe un grain à terre.
\VS{10}Tous les pécheurs de mon peuple mourront par l'épée, ceux qui disent : Le mal n'approchera pas, il ne nous atteindra pas.
\TextTitle{Yahweh relève la maison de David}
\VS{11}En ce temps-là, je relèverai le tabernacle de David qui est tombé, j’en réparerai les brèches, j’en redresserai les ruines, et je le rebâtirai comme il était autrefois,
\VS{12}afin qu'ils possèdent le reste d’Edom et toutes les nations sur lesquelles mon nom a été invoqué, dit Yahweh, qui accomplira cela.
\TextTitle{Restauration d'Isarël}
\VS{13}Voici, les jours viennent, dit Yahweh, où le laboureur suivra de près le moissonneur, et celui qui foule les raisins atteindra celui qui répand la semence ; et le moût ruissellera des montagnes et découlera de toutes les collines.
\VS{14}Je ramènerai les captifs de mon peuple d'Israël ; ils rebâtiront les villes dévastées, et y habiteront, ils planteront des vignes, et en boiront le vin ; ils feront des jardins et en mangeront les fruits.
\VS{15}Je les planterai sur leur terre, et ils ne seront plus arrachés du pays que je leur ai donné\FTNT{Cette prophétie annonce la restauration de la maison de David. Personne ne chassera Israël de sa terre, aucune nation n’a le pouvoir de le déloger, car c’est le Seigneur qui l’a établi.}, dit Yahweh, ton Dieu.
\PPE{}
\end{multicols}

%\clearpage\ShortTitle{Abdias}\BookTitle{Abdias}\BFont
\noindent\hrulefill
{\footnotesize
\textit{
\bigskip
{\centering{}
\\Auteur : Abdias
\\(Heb. : Obadyah)
\\Signification : Adorateur ou serviteur de Yahweh
\\Thème : Condamnation d'Edom
\\Date de rédaction : 6ème siècle av. J.-C.\\}
}
%\bigskip
\textit{
\\Prophète ayant exercé son ministère en Juda, Abdias reçut un message court, mais clair sur le jour de Yahweh, et plus particulièrement sur le jugement d’Edom à la suite de ses violences envers Israël.\bigskip
}
}
\par\nobreak\noindent\hrulefill
\begin{multicols}{2}
\TextTitle{[I. Malédiction d'Edom : Introduction]}
\Chap{1}
\VerseOne{}Vision d'Abdias. Ainsi parle le Seigneur, Yahweh, sur Edom : Nous l’avons entendu de la part de Yahweh, et un messager a été envoyé parmi les nations : Courage, levons-nous contre lui pour le combattre\FTNT{Jé. 49:14} !
\VS{2}Voici, je te rendrai petit parmi les nations, tu seras fort méprisé.
\VS{3}L'orgueil de ton coeur t'a égaré, toi qui habites dans le creux des rochers, qui sont ta haute demeure, et qui dis en toi-même : Qui me précipitera jusqu’à terre ?
\VS{4}Quand tu élèverais ton nid comme l'aigle, et quand bien même tu le mettrais entre les étoiles, je te précipiterai de là, dit Yahweh.
\VS{5}Si des voleurs entraient chez toi, ou des pillards de nuit, comme te voilà ruiné ! Mais ils ne prendraient que ce qui leur suffit. Si des vendangeurs entraient chez toi, ne laisseraient-ils pas des grappillages ?
\VS{6}Comme Esaü a été fouillé ! Comme ses trésors cachés ont été découverts !
\VS{7}Tous tes alliés t'ont chassé jusqu'à la frontière, ceux qui étaient en paix avec toi t'ont trompé et ont eu le dessus sur toi, ceux qui mangeaient ton pain t'ont tendu des pièges, et tu ne t’en es pas aperçu.
\VS{8}N’est-ce pas en ce jour-là, dit Yahweh, que je ferai périr les sages d'Edom, et l’intelligence de la montagne d'Esaü ?
\VS{9}Tes guerriers seront effrayés, ô Théman ! Afin qu’ils soient tous retranchés de la montagne d'Esaü par le carnage\FTNT{Ez. 25:13 ; Mi. 7:8 ; So. 2:8}.
\TextTitle{II. Causes de la malédiction}
\VS{10}A cause de la violence que tu as faite à ton frère Jacob, la honte te couvrira, et tu seras retranché à jamais.
\VS{11}Le jour où tu te tenais en face de lui, le jour où des étrangers emmenaient captive son armée, où des inconnus entraient dans ses portes et jetaient le sort sur Jérusalem, toi aussi, tu étais comme l'un d'eux.
\VS{12}Ne considère pas avec joie le jour de ton frère, le jour de son malheur, ne te réjouis pas sur les enfants de Juda au jour de leur ruine, et n’ouvre pas une grande bouche au jour de la détresse.
\VS{13}N’entre pas dans les portes de mon peuple au jour de sa ruine, ne considère pas avec joie son malheur, au jour de sa ruine, et que tes mains ne se portent pas sur ses richesses, au jour de sa ruine !
\VS{14}Ne te tiens pas aux carrefours pour exterminer ses fugitifs, et ne livre pas ses fuyards au jour de la détresse.
\TextTitle{III. Edom au jour de Yahweh}
\VS{15}Car le jour de Yahweh est proche pour toutes les nations ; on te fera comme tu as fait, tes actes retomberont sur ta tête\FTNT{Jé. 50:15-29 ; Ez. 35:15}.
\VS{16}Car comme vous avez bu sur ma montagne sainte, ainsi toutes les nations boiront continuellement ; elles boiront, elles avaleront, et elles seront comme si elles n'avaient jamais été\FTNT{Jé. 25:15-28}.
\TextTitle{Délivrance future de Jacob et jugement sur Edom}
\VS{17}Mais le salut\FTNT{Le salut sera sur la montagne de Sion. Cette prophétie fait allusion au Royaume messianique. Voir Ro. 11:26.} sera sur la montagne de Sion, elle sera sainte, et la maison de Jacob possédera ses possessions.
\VS{18}La maison de Jacob sera un feu, et la maison de Joseph une flamme, et la maison d'Esaü du chaume ; ils l'allumeront et la consumeront ; et il ne restera rien de la maison d'Esaü, car Yahweh a parlé.
\VS{19}Ceux du midi possèderont la montagne d'Esaü, ceux de la plaine le pays des Philistins ; ils posséderont le territoire d'Ephraïm et celui de Samarie ; et Benjamin possédera Galaad.
\VS{20}Les captifs de cette armée des enfants d'Israël posséderont le pays des Cananéens jusqu'à Sarepta, et ceux qui auront été transportés de Jérusalem, qui sont à Sepharad, posséderont les villes du midi.
\VS{21}Des libérateurs monteront sur la montagne de Sion, pour juger la montagne d'Esaü ; et la royauté sera à Yahweh.
\PPE{}
\end{multicols}

%\clearpage\ShortTitle{Jonas}\BookTitle{Jonas}\BFont
\begin{multicols}{2}
\TextTitle{[I. Désobéissance et fuite de Jonas
\\Introduction]}
\Chap{1}
\VerseOne{}La parole de Yahweh fut adressée à Jonas, fils d'Amitthaï, en ces mots :
\TextTitle{[Jonas fuit la face de Yahweh]}
\VS{2}Lève-toi, va à Ninive (1), la grande ville, et crie contre elle ! Car leur malice est montée jusqu'à moi.
\VS{3}Mais Jonas se leva pour s'enfuir à Tarsis, loin de la face de Yahweh. Il descendit à Japho, où il trouva un navire qui allait à Tarsis ; il paya le prix du transport et y entra pour aller à Tarsis, loin de la face de Yahweh.
\VS{4}Mais Yahweh fit lever un grand vent sur la mer, et il y eut une grande tempête sur la mer, de sorte que le navire semblait se briser.
\VS{5}Et les mariniers eurent peur, et ils crièrent chacun à son dieu, et jetèrent dans la mer les objets qui étaient dans le navire, pour l’alléger. Jonas descendit au fond du navire, se coucha et s’endormit profondément.
\VS{6}Le chef des marins s'approcha de lui, et lui dit : Qu’as-tu dormeur ? Lève-toi, invoque ton Dieu ! Peut-être ton Dieu pensera à nous et nous ne périrons pas.
\VS{7}Puis ils se dirent l'un à l'autre : Venez, tirons au sort pour savoir qui est la cause de ce malheur. Ils tirèrent au sort, et le sort tomba sur Jonas.
\VS{8}Alors ils lui dirent : Dis-nous quelle est la cause de ce malheur. Quel est ton métier, et d'où viens-tu ? Quel est ton pays, et de quel peuple es-tu ?
\VS{9}Il leur répondit : Je suis Hébreu, et je crains Yahweh, le Dieu des cieux, qui a fait la mer et la terre sèche.
\VS{10}Alors ces hommes furent saisis d'une grande crainte, et lui dirent : Pourquoi as-tu fait cela ? Car ces hommes savaient qu’il fuyait loin de la face de Yahweh, parce qu'il le leur avait déclaré.
\VS{11}Ils lui dirent : Que te ferons-nous pour que la mer se calme ? Car la mer était de plus en plus agitée.
\TextTitle{[II. Jonas et le poisson
\\Jonas englouti par le poisson]}
\VS{12}Il leur répondit : Prenez-moi, et jetez-moi dans la mer, et la mer se calmera ; car je sais que c’est moi la cause de cette grande tempête.
\VS{13}Ces hommes ramaient pour revenir sur la terre sèche, mais ils ne le purent, car la mer s'agitait toujours plus contre eux.
\VS{14}Alors ils invoquèrent Yahweh, et dirent : O Yahweh, ne nous fais pas périr à cause de la vie de cet homme, et ne mets pas sur nous le sang innocent ! Car toi, Yahweh, tu fais comme il te plait (2).
\VS{15}Alors ils prirent Jonas, et le jetèrent dans la mer. Et la fureur de la mer s'arrêta.
\VS{16}Ces hommes furent saisis d’une grande crainte envers Yahweh, et ils offrirent des sacrifices à Yahweh, et firent des vœux.
\VS{17}Yahweh ordonna à un grand poisson d’engloutir Jonas, et Jonas fut dans le ventre du poisson trois jours et trois nuits.
\Chap{2}
\VerseOne{}Jonas pria Yahweh, son Dieu, dans le ventre du poisson.
\TextTitle{[Prière de Jonas et l'exaucement de Yahweh]}
\VS{2}Il dit : Dans ma détresse j’ai invoqué Yahweh, et il m'a exaucé ; du sein du scheol j’ai crié, et tu as entendu ma voix (1).
\VS{3}Tu m'as jeté dans les profondeurs, au cœur de la mer, et le courant m'a environné ; tous tes flots et toutes tes vagues ont passé sur moi.
\VS{4}Je disais : Je suis chassé loin de tes yeux ! Cependant je verrai encore le temple de ta sainteté.
\VS{5}Les eaux m'ont environné jusqu'à l'âme. L'abîme m'a enveloppé, les roseaux ont lié ma tête.
\VS{6}Je suis descendu jusqu'aux bases des montagnes, la terre fermait sur moi ses barres pour toujours ; mais tu m’as fait remonter vivant de la fosse, Yahweh, mon Dieu !
\VS{7}Quand mon âme s’était affaiblie en moi, je me suis souvenu de Yahweh, et ma prière est parvenue jusqu’à toi, dans le temple de ta sainteté.
\VS{8}Ceux qui s’adonnent à des vanités mensongères abandonnent ta miséricorde.
\VS{9}Mais moi, je t’offrirai des sacrifices avec un cri de louange, j’accomplirai les vœux que j’ai faits : Car le salut vient de Yahweh (2).
\VS{10}Alors Yahweh parla au poisson, et le poisson vomit Jonas sur la terre sèche.
\TextTitle{[III. Le plus grand réveil de l'histoire
\\Ninive se repent et elle est épargnée]}
\Chap{3}
\VerseOne{}La parole de Yahweh fut adressée à Jonas une seconde fois, en disant :
\VS{2}Lève-toi, va à Ninive, la grande ville, et proclames-y à haute voix ce que je t'ordonne !
\VS{3}Jonas se leva, et alla à Ninive, suivant la parole de Yahweh. Or Ninive était une grande ville devant Dieu, de trois jours de marche.
\VS{4}Jonas commença dans la ville le chemin d'une journée de marche ; il criait et disait : Encore quarante jours, et Ninive sera renversée !
\VS{5}Les hommes de Ninive crurent à Dieu, ils publièrent un jeûne, et se vêtirent de sacs, depuis le plus grand d'entre eux jusqu'au plus petit.
\VS{6}Cette parole parvint au roi de Ninive ; il se leva de son trône, ôta de dessus lui son manteau, se couvrit d'un sac, et s'assit sur la cendre.
\VS{7}Puis il fit faire une proclamation, et publier dans Ninive par décret du roi et de ses grands : Que les hommes, les bêtes, les bœufs et les brebis, ne goûtent de rien, ne paissent point, et ne boivent point d'eau !
\VS{8}Que les hommes et les bêtes soient couverts de sacs, qu'ils crient à Dieu avec force, et que chacun revienne de sa mauvaise voie, et des actions violentes que ses mains ont commises !
\VS{9}Qui sait si Dieu ne reviendra pas et ne se repentira pas, et s'il ne se détournera pas de son ardente colère, en sorte que nous ne périssions point ?
\VS{10}Dieu vit ce qu’ils faisaient et comment ils revenaient de leur mauvaise voie. Alors Dieu se repentit du mal qu'il avait déclaré de leur faire, et il ne le fit point.
\TextTitle{[IV. Miséricorde infinie de Dieu
\\Mécontentement de Jonas]}
\Chap{4}
\VerseOne{}Mais cela déplut fortement à Jonas, et il fut furieux.
\VS{2}Il pria Yahweh, et dit : Oh ! Yahweh, n'est-ce pas là ce que je disais quand j'étais encore dans mon pays ? C'est pourquoi j'ai voulu m'enfuir à Tarsis. Car je savais que tu es un Dieu compatissant, miséricordieux, lent à la colère et riche en bonté, et qui te repens du mal (1).
\VS{3}Maintenant, Yahweh, prends-moi donc la vie, car la mort m'est meilleure que la vie.
\TextTitle{[Reproches de Yahweh à Jonas]}
\VS{4}Et Yahweh répondit : Fais-tu bien de te mettre en colère ?
\VS{5}Alors Jonas sortit de la ville, et s'assit à l'orient de la ville, là il se fit une cabane, et y resta à l'ombre, jusqu'à ce qu'il vît ce qui arriverait à la ville.
\VS{6}Yahweh Dieu ordonna à un ricin de croître au-dessus de Jonas, pour donner de l’ombre sur sa tête et pour le délivrer de son mal. Jonas éprouva une grande joie à cause de ce ricin.
\VS{7}Mais le lendemain, à l’aurore, Dieu ordonna à un ver d’attaquer le ricin, et le ricin sécha.
\VS{8}Au lever du soleil, Dieu ordonna à un vent chaud d’orient de souffler, et le soleil frappa la tête de Jonas, au point qu’il s’évanouit. Il demanda la mort, et dit : La mort m'est meilleure que la vie.
\VS{9}Dieu dit à Jonas : Fais-tu bien de te mettre en colère à cause du ricin ? Et il répondit : Je fais bien de m’irriter jusqu’à la mort.
\VS{10}Et Yahweh dit : Tu as pitié du ricin pour lequel tu n'as point travaillé et que tu n'as point fait croître, qui est né dans une nuit et qui a péri dans une nuit.
\VS{11}Et moi, je n’aurais pas pitié de Ninive, la grande ville, dans laquelle il y a plus de cent vingt mille personnes qui ne savent point distinguer leur main droite de leur main gauche, et des animaux en grand nombre !
\PPE{}
\end{multicols}

%\clearpage\ShortTitle{Michée}\BookTitle{Michée}\BFont
\noindent\hrulefill
{\footnotesize
\textit{
\bigskip
{\centering{}
\\(Mikha)
\\Signifie : Qui est semblable à Dieu ?
\\Thème : Le jugement et le royaume
\\Auteur : Michée 
\\Date de rédaction : 8ème siècle av. J.-C.\\}
}
%\bigskip
\textit{
\\Originaire de Moréscheth, Michée exerça son ministère dans le royaume du sud au temps d’Ezéchias, roi de Juda et fut contemporain d’Osée, Amos et Esaïe. Alors que la corruption et l’idolâtrie régnaient en Samarie et à Jérusalem, Michée appela le peuple à se détourner de ses iniquités et les prévint du danger qui les menaçait. Il prophétisa également le rétablissement final de la nation juive et mit en exergue la miséricorde divine.\bigskip
}
}
\par\nobreak\noindent\hrulefill
\begin{multicols}{2}
\TextTitle{[Jugement de Yahweh sur Israël infidèle]}
\Chap{1}
\VerseOne{}La Parole de Yahweh qui fut adressée à Michée, de Moréscheth, au temps de Jotham, Achaz, et Ezéchias, rois de Juda, laquelle lui fut adressée dans une vision contre Samarie et Jérusalem.
\VS{2}Vous tous, peuples, écoutez ! Et toi, terre, et tout ce qui est en elle, soyez attentifs ! Et que le Seigneur, Yahweh, soit témoin contre vous, le Seigneur, sortant du palais de sa sainteté.
\VS{3}Car voici, Yahweh sortira de son lieu, il descendra, et marchera sur les hauts lieux de la terre\FTNT{Cette prophétie annonce à la fois la destruction du royaume du nord par Salmanasar V (régna de 727-722 av. J.-C.) en 722 av. J.-C. (2 R. 17:1-23), l’invasion de Sanchérib (régna de 705 à 681 av. J.-C.), (2 R. 18:13 à 2 R. 19:37) ainsi que celle de Nebucadnetsar (régna de 604 à 562 av. J.-C.), (2 R. 24 et 25).} ;
\VS{4}et les montagnes se fondront sous lui, et les vallées se fendront, elles seront comme de la cire devant le feu et comme des eaux qui coulent sur une pente.
\VS{5}Tout ceci arrivera à cause du crime de Jacob, et à cause des péchés de la maison d'Israël. Or quel est le crime de Jacob ? N'est-ce pas Samarie ? Et quels sont les hauts lieux de Juda ? N'est-ce pas Jérusalem ?
\TextTitle{[Chutes futures de Samarie et Jérusalem]}
\VS{6}C'est pourquoi je réduirai Samarie en un monceau de pierres dans les champs, un lieu où l'on plante des vignes ; et je ferai rouler ses pierres dans la vallée, et je découvrirai ses fondements.
\VS{7}Toutes ses images taillées seront brisées, tous ses salaires de prostitution seront brûlés au feu, et je mettrai tous ses faux dieux en désolation ; parce qu'elle les a entassés par le moyen du salaire de sa prostitution, ils serviront de salaire à une prostituée.
\VS{8}C'est pourquoi je me plaindrai, et je hurlerai ; je m'en irai dépouillé et nu ; je ferai une lamentation comme celle des dragons, et je mènerai le deuil comme celui des autruches.
\VS{9}Car sa plaie est incurable, elle est venue jusqu'en Juda, et est parvenue jusqu'à la porte de mon peuple, jusqu'à Jérusalem.
\VS{10}Ne l'annoncez point dans Gath, et ne pleurez nullement ! Vautre-toi dans la poussière à Beth-Leaphra.
\VS{11}Passez habitants de Schaphir, dans la nudité et la honte ! Les habitants de Tsaanan ne sont point sortis ; le deuil de Beth-Haëtsel vous prive de son abri.
\VS{12}L’habitante de Maroth est dans l'angoisse à cause de son bien ; parce que le mal est descendu de par Yahweh sur la porte de Jérusalem.
\VS{13}Attelle le cheval au char, habitante de Lakisch ! Toi qui es le commencement du péché de la fille de Sion ; car en toi ont été trouvés les crimes d'Israël.
\VS{14}C'est pourquoi donne des présents à cause de Moréschet-Gath ; les maisons d'Aczib mentiront aux rois d'Israël.
\VS{15}Je t’amènerai un autre héritier, habitante de Maréscha ; et la gloire d'Israël s'en ira jusqu'à Adullam.
\VS{16}Arrache tes cheveux et fais-toi tondre, à cause de tes fils qui font tes délices ; arrache tout le poil de ton corps, comme un aigle qui mue, car ils sont emmenés prisonniers loin de toi.
\TextTitle{[Causes du jugement de Dieu sur Israël]}
\Chap{2}
\VerseOne{}Malheur à ceux qui pensent à faire outrage, qui forgent le mal sur leurs lits, et qui l'exécutent dès le point du jour, parce qu'ils ont le pouvoir en main.
\VS{2}S'ils convoitent des possessions, ils les ont aussitôt ravies, des maisons, ils les ont aussitôt prises ; ainsi ils oppriment l'homme et sa maison, l'homme et son héritage.
\VS{3}C'est pourquoi ainsi parle Yahweh : Voici, je médite contre cette famille-ci un mal duquel vous ne pourrez point préserver votre cou, et vous ne marcherez point la tête levée, car ce temps est mauvais.
\VS{4}En ce temps-là on fera de vous un proverbe lugubre, et l'on gémira d'un gémissement lamentable, en disant : Nous sommes entièrement détruits ; la part de mon peuple, il la change de mains ! Comment nous enlève-t-il et partage-t-il notre terre à l'infidèle ?
\VS{5}C'est pourquoi il n'y aura personne qui jettera le cordeau pour ton lot, dans l'assemblée de Yahweh.
\VS{6}Ne prophétisez point, disent-ils, ne prophétisez point de telles choses ; l'opprobre ne s'éloignera point.
\VS{7}Or toi qui es appelée maison de Jacob, l'Esprit de Yahweh est-il amoindri ? Sont-ce là ses actes ? Mes paroles ne sont-elles pas bonnes pour celui qui marche droitement ?
\VS{8}Mais celui qui était hier mon peuple, s'élève à la manière d'un ennemi ; vous dépouillez le manteau avec le vêtement à ceux qui passent en assurance, de retour de la guerre.
\VS{9}Vous chassez les femmes de mon peuple hors des maisons de leurs délices ; vous ôtez pour toujours ma gloire de dessus leurs enfants.
\VS{10}Levez-vous et marchez, car ce pays n'est plus pour vous un lieu de repos ; à cause de la souillure, il vous détruira d’une violente destruction.
\VS{11}S'il y a quelque homme qui court après le vent et le mensonge, et qui mente en disant : Je te prophétiserai sur du vin et sur les boissons fortes, ce sera le prophète de ce peuple.
\TextTitle{[Yahweh, le Dieu qui rassemble son peuple]}
\VS{12}Mais je t'assemblerai tout entier, ô Jacob ! Et je ramasserai entièrement le reste d'Israël, et le mettrai tout ensemble comme les brebis d'une bergerie, comme un troupeau au milieu de son pâturage ; il y aura un grand bruit pour la foule des hommes.
\VS{13}Celui qui fera la brèche\FTNT{«~Celui qui fera la brèche~» est Jean-Baptiste, qui fut suscité à une époque où la gloire de Dieu ne se manifestait plus depuis quatre cents ans. En effet, Dieu n’avait plus suscité de ministère prophétique ou de messagers pour parler à son peuple. Le ciel était comme fermé. Il est donc venu ouvrir une brèche, c’est-à-dire préparer le chemin du Seigneur selon Es. 40:3-5, Mal. 3:1, Mal. 4:5-6.} montera  devant eux, on brisera, et on passera outre, et ils sortiront par la porte ; et leur Roi marchera devant eux\FTNT{Le Roi qui marchera devant eux est bien Jésus-Christ. Le message de Jean-Baptiste était clair : «~Repentez-vous car le Royaume des cieux arrive~»,  or ce royaume est celui du Messie (Mt. 3:1-3).}, Yahweh sera à leur tête.
\TextTitle{[Corruption et méchanceté des chefs]}
\Chap{3}
\VerseOne{}C'est pourquoi je dis : Ecoutez, chefs de Jacob, et vous conducteurs de la maison d'Israël ! N'est-ce point à vous de connaître ce qui est juste ?
\VS{2}Ils haïssent le bien, et aiment le mal ; ils leur arrachent la peau et la chair de dessus les os.
\VS{3}Ils dévorent la chair de mon peuple, lui arrachent la peau, et lui brisent les os ; ils le mettent en pièces comme dans un pot, comme de la viande dans une chaudière.
\VS{4}Alors ils crieront à Yahweh, mais il ne les exaucera point, et il leur cachera sa face en ce temps-là, parce qu’ils se sont mal conduits dans leurs actions
\VS{5}Ainsi parle Yahweh contre les prophètes qui égarent mon peuple, qui annoncent la paix si leurs dents ont quelque chose à mordre, et qui publient la guerre si on ne leur met rien dans la bouche :
\VS{6}C'est pourquoi la nuit sera sur vous, afin que vous n'ayez plus de vision ; et elle s'obscurcira, afin que vous ne deviniez plus ; le soleil se couchera sur ces prophètes-là, et le jour leur sera ténébreux.
\VS{7}Les voyants seront honteux, et les devins seront confondus ; eux tous se couvriront la barbe, parce qu'il n'y aura aucune réponse de Dieu.
\VS{8}Mais moi, je suis rempli de force, de justice, et de courage, par l'Esprit de Yahweh, pour déclarer à Jacob son crime, et à Israël son péché.
\TextTitle{[Future destruction de Jérusalem]}
\VS{9}Ecoutez maintenant ceci, chefs de la maison de Jacob, et vous conducteurs de la maison d'Israël, qui avez la justice en abomination, et qui pervertissez tout ce qui est droit,
\VS{10}vous qui bâtissez Sion avec le sang, et Jérusalem avec l'injustice.
\VS{11}Ses chefs jugent pour des présents, ses sacrificateurs enseignent pour un salaire, et ses prophètes devinent pour de l'argent\FTNT{Ce passage est encore d’actualité de nos jours. En effet, nombreux sont les dirigeants chrétiens qui exigent un salaire, notamment par le moyen de la dîme pour la plupart, en échange de leurs prières, enseignements,  conseils, formations bibliques… Le même constat se fait avec les chantres qui font des concerts payants alors qu’ils ont reçu leurs grâces du Seigneur gratuitement. Voir commentaire en Mt. 10:8.}, puis ils s'appuient sur Yahweh, en disant : Yahweh n'est-il pas parmi nous ? Le mal ne nous atteindra pas.
\VS{12}C'est pourquoi, à cause de vous, Sion sera labourée comme un champ, et Jérusalem sera réduite en ruines, et la montagne du temple en hauts lieux de forêt.
\TextTitle{[Marcher au nom de Yahweh]}
\Chap{4}
\VerseOne{}Mais il arrivera dans les derniers jours\FTNT{Voir commentaire en Ge. 49:1-2.}, que  la montagne de la maison de Yahweh\FTNT{Dans les Ecritures, les montagnes symbolisent parfois une grande puissance terrestre, et les collines celles de moindre importance. Cette prophétie confirme l’établissement du Royaume messianique dont la capitale sera Jérusalem (2 S. 7:14-16). Esaïe avait reçu la même prophétie que l’on peut découvrir au chapitre 2 de son livre.} sera affermie au sommet des montagnes, et sera élevée par-dessus les collines ; les peuples y afflueront.
\VS{2}Et des nations nombreuses iront et diront : Venez, et montons à la montagne de Yahweh, à la maison du Dieu de Jacob ; il nous enseignera ses voies, et nous marcherons dans ses sentiers ; car la loi sortira de Sion, et la parole de Yahweh de Jérusalem.
\VS{3}Il exercera le jugement parmi des peuples nombreux, et il sera l'arbitre de nations puissantes et lointaines ; et de leurs épées elles forgeront des hoyaux ; et de leurs hallebardes, des serpes ; une nation ne lèvera plus l'épée contre une autre, et on n'apprendra plus la guerre.
\VS{4}Mais chacun demeurera sous sa vigne et sous son figuier, et il n'y aura personne qui les épouvante ; car la bouche de Yahweh des armées aura parlé.
\VS{5}Les peuples marchent chacun au nom de leur dieu ; mais nous, nous marchons au Nom de Yahweh, notre Dieu, à toujours et à perpétuité.
\TextTitle{[Future restauration d'Israël]}
\VS{6}En ce jour-là, dit Yahweh, j'assemblerai les boiteux, je recueillerai ceux que j'avais chassés et ceux que j'avais maltraités.
\VS{7}Je ferai de ceux qui boitent un reste, et de ceux qui étaient éloignés une nation robuste ; Yahweh régnera sur eux, à la montagne de Sion, dès lors et à toujours.
\VS{8}Et toi, tour du troupeau, citadelle de la fille de Sion, jusqu'à toi viendra, à toi arrivera la souveraineté première ; le royaume sera à la fille de Jérusalem.
\TextTitle{[Yahweh, le Dieu qui rachète son peuple]}
\VS{9}Pourquoi maintenant pousses-tu des cris ? N'y a-t-il point de roi au milieu de toi ? Ou ton conseiller est-il mort, que la douleur t'ait saisie comme celle qui enfante ?
\VS{10}Souffre et gémis, fille de Sion, comme celle qui enfante ; car tu sortiras bientôt de la ville, tu demeureras aux champs, et tu iras jusqu'à Babylone ; là tu seras délivrée ; là Yahweh te rachètera de la main de tes ennemis.
\TextTitle{[Nations rassemblés pour l'Harmaguedon]}
\VS{11}Maintenant plusieurs nations se sont rassemblées contre toi\FTNT{Il est ici question de la guerre d’Harmaguédon. Voir commentaire en Ap. 16:12-16.} et disent : Qu'elle soit profanée ! Et que notre œil voie en Sion ce qu’il y voudrait voir\FTNT{Jérusalem est une horloge de Dieu. Le Seigneur a fait en sorte que les nations aient les yeux tournés vers ce bout de terre, car c'est là que débutera la troisième guerre mondiale, l’Harmaguédon (Mt. 24:15-28 ; Ap. 16:12-16 ; Ap. 19:11-21). Là aura lieu le jugement des nations, dans la vallée de Josaphat (Joë. 3:2-12). Le Messie reviendra (Es. 59:20-21 ; Za. 14:1-8 ; Ac. 1:10-11) et gouvernera le monde depuis Jérusalem (Za. 14:9-21). L'actuel conflit israélo-palestinien nous confirme bien ces prophéties. Il ne se passe pas un jour sans que les informations nous rapportent des événements venant de cet endroit du monde. D'ailleurs le Seigneur lui-même nous invite à suivre de près ce qu'il s'y passe (Mt. 24:32-34).}.
\VS{12}Mais ils ne connaissent point les pensées de Yahweh, et ne comprennent pas ses desseins ; car il les a assemblées comme des gerbes dans l'aire.
\VS{13}Lève-toi, et foule, fille de Sion ! Car je te ferai une corne de fer, et te mettrai des ongles d'airain ; et tu écraseras des peuples nombreux, et tu consacreras par interdit leurs profits à Yahweh, et leurs richesses au Seigneur de toute la terre.
\VS{14}Maintenant, fille de troupes, rassemble tes troupes ; on a mis le siège contre nous, on frappera le juge d'Israël avec la verge sur la joue.
\TextTitle{[Naissance du roi: le Messie]
\\(cp. Mt. 2:1-6 ; 27:24-37)}
\Chap{5}
\VerseOne{}Mais toi, Bethléhem Ephrata, petite pour être entre les milliers de Juda, de toi sortira quelqu’un pour être dominateur en Israël, dont l'origine remonte aux temps anciens, aux jours de l'éternité\FTNT{Il est question ici de Jésus-Christ. Ce passage nous parle de sa préexistence éternelle. Les pharisiens, scribes et principaux sacrificateurs avaient la connaissance de cette prophétie concernant le Messie (Mt. 2:1-6).}.
\VS{2}C'est pourquoi il les livrera jusqu'au temps où enfantera celle qui doit enfanter ; et le reste de ses frères retournera avec les enfants d'Israël.
\VS{3}Et il se maintiendra et gouvernera par la force de Yahweh, avec la magnificence du Nom de Yahweh, son Dieu ; et ils auront une demeure assurée, car dès lors il sera élevé jusqu'aux extrémités de la terre.
\VS{4}C'est lui qui sera la paix. Lorsque l'Assyrien sera entré dans notre pays, et qu'il aura mis le pied dans nos palais, nous élèverons contre lui sept pasteurs et huit princes du peuple.
\VS{5}Ils ravageront le pays d'Assyrie avec l'épée, et le pays de Nimrod à ses portes. Il nous délivrera ainsi des Assyriens, quand ils seront entrés dans notre pays, et qu'ils auront mis le pied dans nos quartiers.
\VS{6}Et le reste de Jacob sera au milieu de peuples nombreux, comme une rosée qui vient de Yahweh, et comme une pluie qui tombe sur l'herbe, qui ne s’attend à aucun homme, et qui n'espère pas des enfants des hommes.
\VS{7}Aussi le reste de Jacob sera parmi les nations, et au milieu de peuples nombreux, comme un lion parmi les bêtes de la forêt, et comme un lionceau parmi les troupeaux de brebis ; qui, en passant, foule et déchire, sans que personne ne puisse les sauver.
\TextTitle{[Jugement de Dieu sur les ennemis d'Isrël]}
\VS{8}Ta main se lèvera sur tes adversaires, et tous tes ennemis seront exterminés.
\VS{9}Et il arrivera en ce temps-là, dit Yahweh, que j'exterminerai du milieu de toi tes chevaux, et ferai périr tes chars.
\VS{10}J'exterminerai les villes de ton pays et renverserai toutes tes forteresses.
\VS{11}J'exterminerai aussi de ta main les sorcelleries, et tu n'auras plus aucun devin.
\VS{12}Et j'exterminerai du milieu de toi tes idoles et tes statues, et tu ne te prosterneras plus devant l'ouvrage de tes mains.
\VS{13}J'arracherai aussi du milieu de toi les poteaux d'Asherah\FTNT{Ex. 34:13.}, et détruirai tes ennemis.
\VS{14}Et j'exercerai ma vengeance avec colère et avec fureur contre toutes les nations qui ne m'auront pas écouté.
\TextTitle{[Yahweh appelle son peuple à l'humilité]}
\Chap{6}
\VerseOne{}Ecoutez maintenant ce que dit Yahweh : Lève-toi, plaide devant les montagnes, et que les collines entendent ta voix !
\VS{2}Ecoutez, montagnes, le procès de Yahweh, vous solides fondements de la terre ! Car Yahweh a un procès avec son peuple, et il plaidera avec Israël.
\VS{3}Mon peuple, que t'ai-je fait, ou en quoi t'ai-je causé de la peine ? Réponds-moi !
\VS{4}Car je t'ai fait sortir du pays d'Égypte et t'ai délivré de la maison de servitude, et j'ai envoyé devant toi Moïse, Aaron et Marie.
\VS{5}Mon peuple, rappelle-toi quel conseil Balak, roi de Moab, avait pris contre toi, et de ce que Balaam, fils de Beor, lui répondit ; et de ce que j'ai fait depuis Sittim jusqu'à Guilgal, afin que tu connaisses les justices de Yahweh.
\TextTitle{[Pratiquer la justice]}
\VS{6}Avec quoi me présenterai-je devant Yahweh, et me prosternerai-je devant le Dieu Très-Haut ? Me présenterai-je avec des holocaustes, et avec des veaux d'un an ?
\VS{7}Yahweh prendra-t-il plaisir à des milliers de béliers ou à des myriades de torrents d'huile ? Donnerai-je pour mon crime mon premier-né\FTNT{Selon la loi (Ex. 13 : 2 ; Ex. 3 : 12), les premiers-nés de l’homme et des animaux appartenaient au Seigneur. Ceux des animaux étaient offerts en sacrifice alors que le sacrifice des enfants était formellement interdit sous peine de mort (Lé. 18:21 ; Lé. 20:2-5 ; De. 12:31 ; De. 18:10).}, le fruit de mes entrailles pour le péché de mon âme ?
\VS{8}Ô homme ! Il t'a fait connaître ce qui est bon, et ce que Yahweh exige de toi : Que tu fasses ce qui est juste, que tu aimes la miséricorde, et que tu marches en toute humilité avec ton Dieu.
\VS{9}La voix de Yahweh crie à la ville, et le sage reconnaît son Nom. Écoutez la verge, et celui qui la dirige !
\VS{10}Y a-t-il encore dans la maison du méchant des trésors iniques, et un épha court et détestable ?
\VS{11}Tiendrai-je pour pur celui qui a de fausses balances et de faux poids dans son sac ?
\VS{12}Ses riches sont pleins de violence, ses habitants usent de mensonge, et ils ont une langue trompeuse dans leur bouche.
\VS{13}C'est pourquoi je te rendrai languissante en te frappant, et te ravagerai à cause de tes péchés.
\VS{14}Tu mangeras, mais tu ne seras pas rassasiée, et la faim sera au-dedans de toi-même ; tu mettras de côté, mais tu ne sauveras point, et ce que tu auras sauvé je le livrerai à l'épée.
\VS{15}Tu sèmeras, mais tu ne moissonneras point ; tu presseras l'olive, mais tu ne feras pas d'onctions d'huile ; et tu presseras le moût, mais tu ne boiras pas le vin.
\VS{16}Car tu as gardé les ordonnances d'Omri, et toutes les œuvres de la maison d'Achab, et tu as marché dans leurs conseils. C'est pourquoi je te livrerai à la désolation, je ferai de tes habitants un objet de raillerie, et vous porterez l'opprobre de mon peuple.
\TextTitle{[Le mal appelé bien, et le bien appelé mal]}
\Chap{7}
\VerseOne{}Malheur à moi ! Car je suis comme quand on a cueilli les fruits d'été et les grappillages de la vendange : Il n'y a ni grappe pour manger ni les premiers fruits que mon âme désirait.
\VS{2}Le fidèle est exterminé du pays, et il n'y a plus de juste entre les hommes ; ils sont tous en embûche pour verser le sang, chacun chasse son frère avec des filets.
\VS{3}Leurs mains sont habiles à faire le mal : Le gouverneur exige, le juge demande un salaire, le grand déclare ce qu'il convoite, et ils s'unissent.
\VS{4}Le meilleur d'entre eux est comme une ronce, et le plus juste est pire qu'une haie d'épines. Le jour annoncé par tes sentinelles, ton châtiment arrive. C'est alors qu'ils seront dans la confusion.
\VS{5}Ne crois pas à ton ami intime, et ne te confie pas en tes conducteurs ; garde-toi d'ouvrir ta bouche devant la femme qui dort dans ton sein.
\VS{6}Car le fils déshonore le père, la fille s'élève contre sa mère, la belle-fille contre sa belle-mère, et chacun a pour ennemis les gens de sa maison.
\TextTitle{[Espérance en Yahweh, le Dieu de notre salut]}
\VS{7}Mais moi, je regarderai vers Yahweh, je m'attendrai au Dieu de mon salut ; mon Dieu m'exaucera.
\VS{8}Toi, mon ennemie, ne te réjouis pas sur moi ; si je suis tombée, je me relèverai ; si j'ai été gisante dans les ténèbres, Yahweh m'éclairera.
\VS{9}Je supporterai la colère de Yahweh, car j'ai péché contre lui, jusqu'à ce qu'il défende ma cause, et qu'il me fasse justice ; il me conduira à la lumière, je verrai sa justice.
\VS{10}Et mon ennemie le verra, et la honte la couvrira ; elle qui me disait : Où est Yahweh, ton Dieu ? Mes yeux la verront, et alors elle sera foulée aux pieds comme la boue des rues.
\VS{11}Le jour où il rebâtira tes murs, en ce jour-là tes limites seront reculées.
\VS{12}En ce jour-là on viendra jusqu'à toi d'Assyrie et des villes d'Égypte, et depuis les villes d'Égypte jusqu'au fleuve, et depuis une mer jusqu'à l'autre mer, et depuis une montagne jusqu'à l'autre montagne ;
\VS{13}après que le pays aura été en désolation à cause de ses habitants, et du fruit de leurs actions.
\VS{14}Pais ton peuple avec ta houlette, le troupeau de ton héritage, qui demeure seul dans les forêts au milieu de Carmel ! Et fais qu'ils paissent en Basan et en Galaad, comme aux temps anciens.
\VS{15}Je lui ferai voir des choses merveilleuses, comme au jour où tu sortis du pays d'Égypte.
\VS{16}Les nations le verront, et elles seront honteuses avec toute leur force ; elles mettront la main sur la bouche, et leurs oreilles seront sourdes.
\VS{17}Elles lécheront la poussière comme le serpent, comme les reptiles de la terre ; elles trembleront dans leurs forteresses et accourront toutes effrayées vers Yahweh, notre Dieu, et te craindront.
\VS{18}Quel dieu est semblable à toi, qui est un Dieu qui pardonne l'iniquité, et qui passe par-dessus les péchés du reste de son héritage ? Il ne garde pas à toujours sa colère, parce qu'il prend plaisir à la miséricorde.
\VS{19}Il aura encore compassion de nous ; il effacera nos iniquités, et jettera tous nos péchés au fond de la mer.
\VS{20}Tu feras voir ta fidélité à Jacob, et ta miséricorde à Abraham, comme tu l’as juré à nos pères dès les temps anciens\FTNT{Les versets 18 à 20 de Mi. 7 sont lus chaque année dans les synagogues le jour des expiations.}.
\PPE{}
\end{multicols}

%\clearpage\ShortTitle{Na.}\BookTitle{Nahum}\BFont
\noindent\hrulefill
{\footnotesize
\textit{
\bigskip
{\centering{}
\\Auteur~: Nahum
\\(Heb.~: Nachuwm)
\\Signification~: Consolation, qui a compassion
\\Thème~: La ruine de Ninive
\\Date de rédaction~: 7\up{ème} siècle av. J.-C.\\}
}
\textit{
\\Nahum d'Elkosch, contemporain d'Habakuk, exerça son service dans le royaume de Juda. Il fut chargé d'annoncer la chute de Ninive, qui s'était repentie quelques décennies plus tôt, suite à la prédication de Jonas, mais elle multiplia de nouveau ses actes d'injustice et de violences au point de terroriser tous les peuples des alentours.\bigskip
}
}
\par\nobreak\noindent\hrulefill
\begin{multicols}{2}
\Chap{1}
\VerseOne{}Prophétie sur Ninive\FTNT{Ninive était la capitale de l'ancien empire assyrien. Voir Jon. 1:1-2.}, qui est le livre de la vision de Nahum d'Elkosch.
\TextTitle{Jugement annoncé sur Ninive}
\VS{2}Yahweh est un Dieu jaloux, il se venge, Yahweh se venge, il est plein de fureur~; Yahweh se venge de ses adversaires, et il garde sa colère contre ses ennemis.
\VS{3}Yahweh est lent à la colère, et grand par sa force, mais il ne tient nullement le coupable pour innocent. Yahweh marche parmi les tourbillons et les tempêtes, les nuées sont la poussière de ses pieds.
\VS{4}Il réprimande la mer et la fait tarir, il dessèche tous les fleuves~; le Basan et le Carmel languissent, la fleur du Liban languit.
\VS{5}Les montagnes tremblent à cause de lui, et les collines se fondent~; la terre se soulève devant sa présence, dis-je, et tous ceux qui y habitent.
\VS{6}Qui subsistera devant son indignation~? Et qui demeurera ferme dans l'ardeur de sa colère~? Sa fureur se répand comme un feu, et les rochers se brisent devant lui.
\VS{7}Yahweh est bon, il est une forteresse au jour de la détresse, et il connaît ceux qui se confient en lui.
\VS{8}Il s'en va passer comme un débordement d'eaux~; il réduira son lieu à néant, et fera que les ténèbres poursuivront ses ennemis.
\TextTitle{Jugement contre les ennemis de Yahweh}
\VS{9}Que projetez-vous contre Yahweh~? C'est lui qui réduit à néant~; la détresse ne se lèvera pas deux fois~;
\VS{10}car entrelacés comme des épines, et ivres de leur vin, ils seront consumés entièrement comme la paille sèche.
\VS{11}De toi est sorti celui qui méditait du mal contre Yahweh, et qui avait de mauvais desseins.
\VS{12}Ainsi parle Yahweh~: Bien qu'ils soient en paix et en grand nombre, ils seront certainement retranchés, et on passera outre. Or je t'ai affligé, mais je ne t'affligerai plus.
\VS{13}Maintenant je briserai son joug de dessus toi, et je détacherai tes liens.
\VS{14}Voici ce qu'a ordonné Yahweh contre toi~: Tu n'auras plus de semence qui porte ton nom~; je retrancherai de la maison de tes dieux les images taillées et en fonte~; j'en ferai ta tombe parce que tu es insignifiant.
\Chap{2}
\TextTitle{Délivrance et réjouissance}
\VerseOne{}Voici sur les montagnes les pieds de celui qui apporte de bonnes nouvelles\FTNT{Es. 40:9~; 52:7~; Ro. 10:15.}, qui publie la paix~! Toi Juda, célèbre tes fêtes solennelles, accomplis tes vœux~; car à l'avenir les hommes violents ne passeront plus au milieu de toi, ils sont entièrement retranchés.
\TextTitle{Récit de la destruction de Ninive}
\VS{2}Le destructeur est monté contre toi~; garde la forteresse~! Veille sur la route~! Affermis tes reins~! Consolide toute ta force~!
\VS{3}Car Yahweh se détourne de la majesté de Jacob et d'Israël~; parce que les dévastateurs les ont vidés, et qu'ils ont détruit leurs sarments.
\VS{4}Le bouclier de ses hommes forts est rouge~; ses hommes puissants sont teints de pourpre~; le fer des chars étincelle au jour qu'il a fixé pour la bataille, et les lances sont agitées.
\VS{5}Les chars s'élancent avec rapidité dans les rues, ils se précipitent sur les places, ils sont comme des flambeaux, et courent comme des éclairs.
\VS{6}Il se souvient de ses hommes vaillants, mais ils chancellent dans leur marche. Ils se hâtent vers les murs, et ils se préparent à la défense.
\VS{7}Les portes des fleuves sont ouvertes, et le palais se fond.
\VS{8}C'est fixé~: Elle est découverte et emportée~; ses servantes gémissent de leur voix comme des colombes, frappant leurs poitrines comme un tambour.
\VS{9}Or Ninive, depuis qu'elle a été bâtie, a été comme un vivier d'eaux~; mais ils s'enfuient… Arrêtez-vous~! Arrêtez-vous~! Mais il n'y a personne qui tourne le visage…
\VS{10}Pillez l'argent~! Pillez l'or~! Il y a des dispositions sans fin, des richesses en objets précieux.
\VS{11}On pille, on dévaste, on ravage~! Et les cœurs se fondent, leurs genoux se heurtent l'un contre l'autre. Que le tourment soit dans les reins de tous~! Et que leurs visages deviennent noirs comme un pot qui a été mis sur le feu~!
\VS{12}Où est le repaire des lions, le pâturage des lionceaux, dans lequel se retiraient les lions, et où se tenaient le lion et la lionne, et le petit du lion, sans qu'aucun ne les effraie~?
\VS{13}Les lions ravissaient tout ce qu'il fallait pour leurs petits, et étranglaient les bêtes pour leurs lionnes, ils remplissaient leurs tanières de proies, et leurs repaires de dépouilles.
\VS{14}Voici, j'en veux à toi, dit Yahweh des armées, je brûlerai tes chars, et ils s'en iront en fumée, l'épée consumera tes lionceaux. Je retrancherai de la terre ta proie, et la voix de tes messagers ne sera plus entendue.
\Chap{3}
\TextTitle{Méchanceté de Ninive}
\VerseOne{}Malheur à la ville sanguinaire qui est pleine de mensonge, pleine de violence~; la rapine ne s'en retirera point,
\VS{2}ni le bruit du fouet, ni le bruit impétueux des roues, ni le galop des chevaux, ni le saut des chars~;
\VS{3}ni les cavaliers montant leurs chevaux, ni l'épée flamboyante, ni la lance étincelante, ni la multitude des blessés, ni le grand nombre de cadavres. Des corps morts à l'infini, on trébuche sur un grand nombre de corps morts~!
\VS{4}à cause de la multitude des prostitutions de cette prostituée, pleine de charmes, experte en sortilèges, qui vendait les nations par ses prostitutions, et les familles par ses enchantements.
\VS{5}Voici, j'en veux à toi, dit Yahweh des armées, je relèverai ta robe jusqu'à ton visage~; je manifesterai ta nudité aux nations, et ton ignominie aux royaumes.
\VS{6}Je ferai tomber sur ta tête la peine de tes abominations, je te consumerai et je te donnerai en spectacle.
\VS{7}Et il arrivera que quiconque te verra, s'éloignera de toi et dira~: Ninive est détruite~! Qui aura compassion d'elle~? D'où te chercherai-je des consolateurs~?
\VS{8}Vaux-tu mieux que No-Amon, qui est assise au milieu des fleuves, qui a des eaux autour d'elle, dont la mer est le rempart, et à qui la mer sert de murailles~?
\VS{9}L'Ethiopie et l'Egypte étaient sa force, et une infinité d'autres peuples~; Puth et les Lybiens sont allés à son secours.
\VS{10}Elle-même aussi est transportée hors de sa terre, elle s'en est allée en captivité~; ses enfants ont été écrasés aux carrefours de toutes les rues, et on a jeté le sort sur ses gens honorables, et tous ses grands ont été liés de chaînes.
\VS{11}Toi aussi, tu seras enivrée, tu te tiendras cachée, et tu chercheras un refuge contre l'ennemi.
\VS{12}Toutes tes forteresses seront comme des figues, et comme des premiers fruits qui étant secoués, tombent dans la bouche de celui qui veut les manger.
\VS{13}Voici, ton peuple sera comme autant de femmes au milieu de toi~; les portes de ton pays seront toutes ouvertes, elles seront ouvertes à tes ennemis~; le feu consumera tes verrous.
\VS{14}Puise-toi de l'eau pour le siège~! Fortifie tes remparts~! Enfonce le pied dans la boue, foule l'argile~! Et fortifie le four à brique~!
\VS{15}Là, le feu te consumera, l'épée te retranchera, elle te dévorera comme la sauterelle dévore les arbres. Multiplie-toi comme les sauterelles~! Multiplie-toi comme les sauterelles~!
\VS{16}Tu as multiplié le nombre de tes marchands plus que les étoiles des cieux~; les sauterelles s'étant répandues, ont tout ravagé, et puis se sont envolées.
\VS{17}Tes princes et leurs scribes sont comme des sauterelles qui campent dans les murs au temps de la froidure~: Le soleil paraît, elles s'envolent et on ne reconnaît plus le lieu où elles étaient.
\VS{18}Tes bergers se sont endormis, ô roi d'Assyrie~! Tes grands hommes se tiennent dans leurs tentes~; ton peuple est dispersé par les montagnes, et il n'y a personne qui le rassemble.
\VS{19}Il n'y a point de remède à ta blessure, ta plaie est douloureuse~; tous ceux qui entendront parler de toi battront des mains sur toi~; car qui n'a pas continuellement éprouvé les effets de ta méchanceté~?
\PPE{}
\end{multicols}

%\clearpage\ShortTitle{Habakuk}\BookTitle{Habakuk}\BFont
\noindent\hrulefill
{\footnotesize
\textit{
\bigskip
{\centering{}
\\Signifie : Embrasser, amour
\\Thème : Du doute à la foi
\\Auteur : Habakuk
\\Date de rédaction : 7ème siècle av. J.-C.\\}
}
%\bigskip
\textit{
\\Habakuk, contemporain de Nahum, Sophonie et Jérémie, exerça son ministère dans le royaume de Juda. Véritable sentinelle, il fut chargé d’annoncer le châtiment de Juda par les chaldéens. Ce récit, qui est en partie un dialogue entre Dieu et Habakuk, témoigne de la relation qui les liait. Il est aussi une invitation à la patience et la foi en Yahweh.\bigskip
}
}
\par\nobreak\noindent\hrulefill
\begin{multicols}{2}
\Chap{1}
\TextTitle{[Quand la méchanceté semble triompher de la justice]}
\VerseOne{}Oracle qu’Habakuk, le prophète a vu.
\VS{2}Ô Yahweh ! Jusqu’à quand crierai-je sans que tu m'écoutes ? Jusqu'à quand crierai-je vers toi ? On me traite avec violence sans que tu me délivres !
\VS{3}Pourquoi me fais-tu voir la méchanceté\FTNT{La perplexité d’Habakuk était la même que celle de Job (Job. 21:7), d’Asaph (Ps. 73) et de Jérémie (Jé. 12:1-2). Les méchants semblent prospérer tandis que les justes pleurent et sont persécutés (Mal. 3 : 12-15).}, et vois-tu la perversité ? Pourquoi y a-t-il de l’oppression et de la violence devant moi, et des gens qui excitent des procès et des querelles ?
\VS{4}Parce que la loi est sans force, et que la justice ne se fait jamais, à cause de cela le méchant environne le juste, et à cause de cela on rend des jugements corrompus\FTNT{Jé. 5:26 ; Am. 5:7.}.
\TextTitle{[La réponse de Yahweh]}
\VS{5}Regardez parmi les nations, et voyez, et soyez étonnés et stupéfaits ! Car je vais faire en vos jours une œuvre que vous ne croiriez pas si on vous la racontait\FTNT{Ac. 13:41.}.
\VS{6}Car voici, je vais susciter les Chaldéens, ce peuple cruel et impétueux, marchant sur l'étendue de la terre, pour posséder des demeures qui ne lui appartiennent pas.
\VS{7}Il est redoutable et terrible, son gouvernement et son autorité viennent de lui-même .
\VS{8}Ses chevaux sont plus légers que les léopards, et ils ont la vue plus aiguë que les loups du soir ; et ses cavaliers se répandront çà et là, même ses cavaliers viendront de loin ; ils voleront comme un aigle qui fond sur sa proie\FTNT{Jé. 5:6 ; So. 3:3.}.
\VS{9}Ils viendront tous pour la violence ; ce qu'ils engloutiront de leurs regards sera porté vers l'orient, et ils amasseront les prisonniers comme du sable.
\VS{10}Ce peuple se moque des rois, et les princes sont l’objet de ses railleries ; il se rit de toutes les forteresses ; il amoncelle de la terre, et il s’en empare.
\VS{11}Alors il traverse comme le vent, il passe outre et se rend coupable, car sa force est son dieu.
\TextTitle{[La souveraineté de Dieu]}
\VS{12}N'es-tu pas de toute éternité, ô Yahweh ! Mon Dieu, mon Saint ? Nous ne mourrons point ! Ô Yahweh, tu l'as établi pour exécuter tes jugements ; et toi, mon rocher\FTNT{Voir commentaire en  Es. 8:13-17.}, tu l'as fondé pour punir.
\VS{13}Tu as les yeux trop purs pour voir le mal, et tu ne saurais prendre plaisir à regarder le mal qu'on fait à autrui. Pourquoi regarderais-tu les perfides, et te tairais-tu quand le méchant dévore son prochain qui est plus juste que lui ?
\VS{14}Or tu as fait les hommes comme les poissons de la mer, et comme le reptile qui n'a point de maître.
\VS{15}Il a tout enlevé avec l'hameçon ; il l'a amassé avec son filet, et l'a assemblé dans son rets ; c'est pourquoi il se réjouira et s'égayera\FTNT{Am. 4:2.}.
\VS{16}A cause de cela, il sacrifie à son filet, et il offre de l’encens à ses rets, parce qu'il aura eu par leur moyen une grasse portion, et que sa viande est une chose moelleuse.
\VS{17}Videra-t-il à cause de cela son filet ? Et ne cessera-t-il jamais de faire le carnage des nations ?
\TextTitle{[S'attendre à Yahweh]}
\Chap{2}
\VerseOne{}Je me tenais en sentinelle, j'étais debout dans la forteresse et je faisais le guet, pour voir ce qu’il me dira, et ce que je répondrais après ma plainte\FTNT{Jé. 6:17 ; Es. 21:1-6 ; Ez. 33:1-19.}.
\TextTitle{[Le juste vivra par la foi]}
\VS{2}Et Yahweh m'a répondu et m'a dit : Ecris la vision, et grave-la sur des tablettes, afin qu'on la lise couramment.
\VS{3}Car la vision est encore différée jusqu'à un certain temps, et Yahweh parlera de ce qui arrivera à la fin, et il ne mentira point. S'il tarde, attends-le, car il ne manquera point de venir, et il ne tardera point\FTNT{Hé. 10:37.}.
\VS{4}Voici, l'âme de celui qui s'élève n'est point droite en lui ; mais le juste vivra de sa foi\FTNT{Ro. 1:17 ; Hé. 10:38.}.
\VS{5}Et combien plus l'homme adonné au vin est-il perfide, et l'homme puissant est-il orgueilleux, ne se tenant point tranquille chez lui ; il élargit son âme comme le scheol, et il est insatiable comme la mort, il rassemble vers lui toutes les nations, et réunit à lui tous les peuples.
\VS{6}Tous ceux-là ne feront-ils pas de lui un sujet de raillerie et d’énigmes ? Et ne dira-t-on pas : Malheur à celui qui accumule ce qui ne lui appartient point ; jusqu'à quand le fera-t-il, et entassera-t-il sur lui de la boue épaisse ?
\VS{7}Ne se lèveront-ils pas soudain, ceux qui le mordront ? Ne se réveilleront-ils pas pour te tourmenter ? Et tu deviendras leur proie.
\VS{8}Parce que tu as pillé beaucoup de nations, tout le reste des peuples te pillera, et à cause aussi des meurtres des hommes, et de la violence faite dans le pays, contre la ville, et contre tous ses habitants\FTNT{Es. 33:1 ; Na. 3:1.}.
\VS{9}Malheur à celui qui amasse pour sa maison des gains injustes, afin de placer son nid dans un lieu élevé, pour échapper à l’atteinte de la calamité !
\VS{10}C’est pour la confusion de ta maison que tu as pris conseil, en détruisant beaucoup de peuples, et c’est contre ton âme que tu as péché.
\VS{11}Car la pierre crie du milieu de la muraille, et de la charpente la poutre lui répond.
\VS{12}Malheur à celui qui bâtit des villes avec le sang et qui fonde des cités sur l'iniquité.
\VS{13}Voici, n'est-ce pas la volonté de Yahweh des armées que les peuples travaillent pour le feu, et que les peuples se lassent pour le néant ?
\VS{14}Car la terre sera remplie de la connaissance de la gloire de Yahweh\FTNT{Es. 11:9.}, comme le fond de la mer par les eaux qui le couvrent.
\VS{15}Malheur à celui qui fait boire son compagnon en lui approchant sa bouteille, et qui l’enivre afin qu'on voie sa nudité\FTNT{Es. 5:22 ; Ge. 9:21-24.}.
\VS{16}Tu seras rassasié de honte plutôt que de gloire ; toi aussi bois, et découvre-toi. La coupe de la droite de Yahweh fera le tour jusqu’à toi, et l'ignominie sera répandue sur ta gloire.
\VS{17}Car la violence faite au Liban retombera sur toi ; et les ravages des bêtes t’effrayeront, parce que tu as répandu le sang des hommes, et commis  des violences dans le pays, contre la ville et tous ses habitants.
\VS{18}A quoi sert l'image taillée  pour qu’un ouvrier la taille ? A quoi sert l’image de fonte, docteur de mensonge, a quoi sert-elle pour que l'ouvrier qui l’a faite place en elle sa confiance en fabriquant des idoles muettes ?
\VS{19}Malheur à ceux qui disent au bois : Réveille-toi ! Et à la pierre muette : Réveille-toi ! Enseignera-t-elle ? Voici, elle est couverte d'or et d'argent, et il n'y a aucun esprit au-dedans d’elle.
\VS{20}Mais Yahweh est dans le temple de sa sainteté. Que toute la terre fasse silence devant lui !
\TextTitle{[Habakuk reconnait et accepte la volonté de Dieu]}
\Chap{3}
\VerseOne{}Prière d'Habakuk, le prophète, sur le mode des chants lyriques.
\VS{2}Yahweh, j'ai entendu ce que tu m'as fait entendre, et j'ai été saisi de crainte, ô Yahweh ! Dans le cours des années, ravive ton œuvre ; dans le cours des années, fais-la connaître; dans ta colère souviens-toi de tes compassions.
\VS{3}Dieu vient de Théman, et le Saint vient du mont de Paran. Sélah. Sa majesté couvre les cieux, et la terre est remplie de sa louange.
\VS{4}Sa splendeur est comme la lumière même, et des rayons sortent de sa main ; c'est là où réside sa force.
\VS{5}La peste marche devant lui, et une  flamme ardente sort sous ses pieds.
\VS{6}Il s'arrête et mesure la terre ; il regarde et met en déroute les nations ; les montagnes antiques sont  brisées en éclats,  et les collines éternelles s’affaissent. Ses voies sont les voies anciennes.
\VS{7}Je vois les tentes de Cuschan accablées sous la punition ; les pavillons du pays de Madian sont ébranlés.
\VS{8}Est-ce contre les fleuves que s’irrite Yahweh ? Ta colère est-elle contre les fleuves, et ta fureur contre la mer, que tu sois monté sur tes chevaux et sur tes chars de délivrance ?
\VS{9}Ton arc est mis à nu et tire toutes les flèches, selon le serment fait aux tribus, à savoir ta parole. Sélah. Tu fends la terre et tu en fais sortir des fleuves\FTNT{Ps. 78:15-16 ; Ps. 105:41.}.
\VS{10}Les montagnes te voient et elles tremblent\FTNT{Ps. 114:4-7.}; des torrents d’eau se précipitent, l'abîme fait retentir sa voix de la profondeur, il élève ses mains en haut.
\VS{11}Le soleil et la lune s'arrêtent dans leur habitation\FTNT{Jos. 10:12 ; Ap. 22:5.}, ils marchent à la lueur de tes flèches, et à la splendeur de l'éclat de ta lance étincelante.
\VS{12}Tu marches sur la terre avec indignation, et foules les nations avec colère.
\VS{13}Tu sors pour la délivrance de ton peuple, tu sors avec ton Oint pour la délivrance ; tu transperces le chef, afin qu'il n'y en ait plus dans la maison du méchant, tu en découvres le fondement  jusqu’au fond. Sélah.
\VS{14}Tu perces avec ses flèches  la tête de ses chefs, quand ils viennent comme une tempête pour me dissiper ; ils s'égaient comme pour dévorer l'affligé dans sa retraite.
\VS{15}Tu marches avec tes chevaux par la mer, les grandes eaux ayant été amoncelées.
\VS{16}J'ai entendu ce que tu m'as déclaré, et mes entrailles en sont émues ; à ta voix le tremblement saisit mes lèvres ; la pourriture entre dans mes os, et je tremble en moi-même, car je serai en repos au jour de la détresse, lorsque montant vers le peuple, il le mettra en pièces.
\VS{17}Car le figuier ne fleurira pas, et il n'y aura point de fruit dans les vignes ; ce que l'olivier produit mentira, et aucun champ ne produira rien à manger ; les brebis seront retranchées du parc, et il n'y aura point de bœufs dans les étables.
\VS{18}Mais moi, je me réjouis en Yahweh, et je me réjouis dans le Dieu de ma délivrance.
\VS{19}Yahweh, le Seigneur, est ma force, et il rend mes pieds semblables à ceux des biches, et me fait marcher sur mes lieux élevés\FTNT{Ps. 18:33-34 ; De. 32:13.}. Au chef des chantres avec instruments à cordes.
\PPE{}
\end{multicols}

%\clearpage\ShortTitle{So.}\BookTitle{Sophonie}\BFont
\noindent\hrulefill
{\footnotesize
\textit{
\bigskip
{\centering{}
\\Auteur~: Sophonie
\\(Heb.~: Tsephanyah)
\\Signification~: Yahweh a caché, protégé
\\Thème~: Le jour de Yahweh
\\Date de rédaction~: 7\up{ème} siècle av. J.-C.\\}
}
\textit{
\\De lignée royale, Sophonie exerça son service dans le royaume de Juda au temps du roi Josias et fut contemporain de Jérémie, Habakuk, Ezéchiel et Abdias. A une époque où l'iniquité s'était accrue au point où les quelques personnes fidèles à Dieu étaient persécutées, Sophonie fut suscité par Yahweh pour annoncer le jugement de Juda, d'Israël et de quelques nations païennes.\bigskip
}
}
\par\nobreak\noindent\hrulefill
\begin{multicols}{2}
\Chap{1}
\TextTitle{Yahweh annonce son jugement sur Juda, conséquence de son idolâtrie}
\VerseOne{}C'est ici la parole de Yahweh qui fut adressée à Sophonie, fils de Cuschi, fils de Guedalia, fils d'Amaria, fils d'Ezéchias, du temps de Josias, fils d'Amon, roi de Juda.
\VS{2}Je ferai entièrement périr toutes choses de dessus cette terre, dit Yahweh.
\VS{3}Je ferai périr l'homme et le bétail~; je consumerai les oiseaux des cieux et les poissons de la mer~; et la ruine arrivera aux méchants, et je retrancherai les hommes de dessus cette terre, dit Yahweh.
\VS{4}J'étendrai ma main sur Juda, et sur tous les habitants de Jérusalem~; je retrancherai de ce lieu-ci le reste de Baal\FTNT{Voir commentaire en Jg. 2:13.}, les noms des prêtres des faux dieux, les prêtres,
\VS{5}ceux qui se prosternent sur les toits devant l'armée des cieux, ceux qui se prosternent devant Yahweh, qui jurent par lui, et qui jurent aussi par Malcom\FTNT{2 R. 17:33~; 2 R. 23:11-12~; Jé. 19:13.},
\VS{6}ceux qui se détournent de Yahweh, ceux qui n'ont point cherché Yahweh, qui ne l'ont point consulté.
\VS{7}Silence, à cause de la présence du Seigneur Yahweh, car le jour de Yahweh est proche\FTNT{Voir commentaire en Za. 14:1.}~; Yahweh a préparé le sacrifice, il a invité ses conviés.
\VS{8}Et il arrivera au jour du sacrifice de Yahweh que je punirai les chefs, et les enfants du roi, et tous ceux qui portent des vêtements étrangers.
\VS{9}Et je punirai, en ce jour-là, tous ceux qui sautent par-dessus le seuil, et ceux qui remplissent de violence et de fraude la maison de leurs maîtres.
\VS{10}Et en ce jour-là dit Yahweh, il y aura de grands cris vers la porte des poissons, et des hurlements vers la seconde partie de la ville, et une grande désolation sur les collines.
\VS{11}Vous qui habitez dans Macthesch\FTNT{Macthesch était un bas-quartier de Jérusalem où se trouvaient les marchés.}, hurlez~! Car tous ceux qui trafiquaient ont été détruits, et tous ceux qui apportaient de l'argent ont été retranchés.
\VS{12}Et il arrivera en ce temps-là que je fouillerai Jérusalem avec des lampes, que je punirai les hommes qui sont figés sur leurs lies, et qui disent dans leurs cœurs~: Yahweh ne nous fera ni bien ni mal.
\VS{13}Leurs biens seront au pillage et leurs maisons en désolation~; et ils auront bâti des maisons, mais ils ne les habiteront pas~; ils auront planté des vignes, mais ils n'en boiront pas le vin.
\VS{14}Le grand jour de Yahweh est proche, il est proche, et il se hâte beaucoup~; le jour de Yahweh n'est que bruit~; celui qui est dans l'amertume, crie de toute sa force. Là sont les hommes vaillants\FTNT{Jé. 30:7~; Joë. 2:11~; Am. 5:18.}.
\VS{15}Ce jour est un jour de fureur, un jour de détresse et d'angoisse, un jour de bruit éclatant et effrayant, un jour de ténèbres et d'obscurité, un jour de nuées et de brouillards~;
\VS{16}un jour de shofar et de cris de guerre contre les villes fortifiées, et contre les hautes tours.
\VS{17}Je mettrai les hommes dans la détresse, et ils marcheront comme des aveugles, parce qu'ils ont péché contre Yahweh~; et leur sang sera répandu comme de la poussière, et leur chair comme des ordures.
\VS{18}Ni leur argent ni leur or ne pourront les délivrer au jour de la fureur de Yahweh~; et tout ce pays sera dévoré par le feu de sa jalousie, car il se hâtera de consumer tous les habitants de ce pays\FTNT{Ez. 7:19~; Pr. 11:4.}.
\Chap{2}
\TextTitle{Yahweh invite Israël à la repentance}
\VerseOne{}Examinez-vous, examinez-vous avec soin ô nation non désirée\FTNT{1 Th. 5:21~; 2 Co. 13:5~; Ep. 5:10.}~!
\VS{2}Avant que le décret enfante, et que le jour passe comme la balle~; avant que l'ardeur de la colère de Yahweh vienne sur vous, avant que le jour de la colère de Yahweh vienne sur vous~!
\VS{3}Vous, tous les pauvres du pays, qui faites ce qu'il ordonne, cherchez Yahweh, cherchez la justice, cherchez l'humilité~; peut-être serez-vous protégés au jour de la colère de Yahweh\FTNT{Am. 5:15.}.
\VS{4}Mais Gaza sera abandonnée, et Askalon sera en désolation~; on chassera les habitants d'Asdod en plein midi, et Ekron sera arrachée\FTNT{Am. 8:9~; Za. 9:5.}.
\VS{5}Malheur aux habitants de la contrée maritime, à la nation des Kéréthiens~! La parole de Yahweh est contre vous~; Canaan, qui est le pays des Philistins, je te détruirai, si bien que, personne n'y habitera.
\VS{6}Et la contrée maritime sera des pâturages, des demeures pour les bergers, et des parcs pour les troupeaux.
\VS{7}Et cette contrée sera pour le reste de la maison de Juda~; ils paîtront dans ces lieux-là, et le soir ils feront leur gîte dans les maisons d'Askalon~; car Yahweh, leur Dieu, les visitera, et il ramènera leurs captifs.
\VS{8}J'ai entendu les insultes de Moab, et les outrages des fils d'Ammon, quand ils ont diffamé mon peuple, et l'ont bravé sur leur frontière\FTNT{Ez. 25:3-6.}.
\VS{9}C'est pourquoi, je suis vivant, dit Yahweh des armées, le Dieu d'Israël, Moab sera comme Sodome, et les fils d'Ammon comme Gomorrhe, un lieu couvert d'orties, et une carrière de sel et de désolation à jamais~; les restes de mon peuple les pilleront, et les restes de ma nation les posséderont.
\VS{10}Ceci leur arrivera en échange de leur orgueil, parce qu'ils ont usé d'insultes et d'arrogance, contre le peuple de Yahweh des armées\FTNT{Es. 16:6~; Jé. 48:29.}.
\VS{11}Yahweh sera terrible contre eux, car il anéantira tous les dieux du pays~; et on se prosternera devant lui, chacun de son lieu, même dans toutes les îles des nations\FTNT{Mal. 1:11~; Jn. 4:21.}.
\VS{12}Vous aussi, habitants de Cusch, vous serez blessés à mort par mon épée.
\VS{13}Il étendra aussi sa main sur le nord, et il détruira l'Assyrie, et il fera de Ninive une désolation, dans un lieu aride comme un désert.
\VS{14}Et les troupeaux feront leur gîte au milieu d'elle, et toutes les bêtes des nations, même le pélican et le hérisson, habiteront parmi les chapiteaux de ses colonnes~; la voix des oiseaux retentira à la fenêtre, la désolation sera au seuil, parce qu'il en aura abattu les cèdres\FTNT{Es. 14:23~; Es. 34:11.}.
\VS{15}C'est là cette ville remplie de joie, qui se tenait assurée, et qui disait en son cœur~: C'est moi, et il n'y en a point d'autre que moi~! Comment a-t-elle été réduite en désert, pour être le repère des bêtes~? Quiconque passera près d'elle sifflera et secouera sa main.
\Chap{3}
\TextTitle{Israël persiste dans l'immoralité}
\VerseOne{}Malheur à la ville immonde et souillée et qui ne fait qu'opprimer~!
\VS{2}Elle n'a point écouté la voix, elle n'a point reçu d'instruction, elle ne s'est point confiée en Yahweh, elle ne s'est point approchée de son Dieu.
\VS{3}Ses chefs au milieu d'elle sont des lions rugissants, et ses juges sont des loups du soir, qui ne gardent pas les os pour les ronger le matin\FTNT{Ez. 22:27~; Pr. 28:15.}.
\VS{4}Ses prophètes sont des téméraires, et des hommes infidèles~; ses prêtres ont souillé les choses saintes, ils ont fait violence à la loi\FTNT{Jé. 23:11-32.}.
\VS{5}Yahweh est juste au milieu d'elle, il ne commet point d'iniquité\FTNT{De. 32:4.}. Chaque matin il met en lumière son jugement, il n'y manque pas~; mais celui qui est inique ne sait ce que c'est que d'avoir honte.
\VS{6}J'ai exterminé les nations, et leurs forteresses ont été désolées~; j'ai rendu désertes leurs places, si bien que personne n'y passe~; leurs villes ont été détruites, sans qu'il y soit resté un seul homme, et sans qu'il y ait aucun habitant.
\VS{7}Et je disais~: Au moins tu me craindras, tu recevras instruction, et sa demeure ne sera pas retranchée, quelque soit la punition que je lui envoie. Mais ils se sont levés de bon matin, ils ont corrompu toutes leurs actions.
\VS{8}C'est pourquoi attendez-moi, dit Yahweh, au jour où je me lèverai pour le butin~; car j'ai résolu de rassembler les nations et de réunir les royaumes, pour répandre sur eux mon indignation, et toute l'ardeur de ma colère~; car tout le pays sera dévoré par le feu de ma jalousie.
\TextTitle{Un reste trouve refuge en Yahweh}
\VS{9}Alors je transformerai les langues\FTNT{Il est question ici de la conversion des peuples issus des nations (Ap. 7:9-17).} des nations en des langues pures, afin qu'elles invoquent toutes le Nom de Yahweh, pour qu'elles le servent d'un commun accord.
\VS{10}Mes adorateurs qui sont au-delà des fleuves de Cusch, à savoir la fille de mes dispersés, m'apporteront mes offrandes\FTNT{Es. 19:21~; Es. 27:13~; Ps. 68:31-32~; Ps. 72:10-11.}.
\VS{11}En ce jour-là, tu ne seras plus confuse à cause de toutes tes actions, par lesquelles tu as péché contre moi~; parce qu'alors j'aurai ôté du milieu de toi ceux qui se réjouissent de ton orgueil, et désormais tu ne t'enorgueilliras plus de la montagne de ma sainteté.
\VS{12}Et je laisserai au milieu de toi un peuple humble et faible, et il mettra sa confiance dans le Nom de Yahweh.
\VS{13}Les restes d'Israël ne commettront point d'iniquité, et ne proféreront point de mensonge, et il n'y aura point dans leur bouche de langue trompeuse~; aussi ils paîtront et se reposeront, et il n'y aura personne qui les épouvante.
\TextTitle{Israël délivré et restauré}
\VS{14}Réjouis-toi avec chant de triomphe, fille de Sion~! Pousse des cris de réjouissance, ô Israël~! Réjouis-toi et triomphe de tout ton cœur, fille de Jérusalem~!
\VS{15}Yahweh a aboli ta condamnation, il a éloigné ton ennemi. Le Roi d'Israël, Yahweh, est au milieu de toi~; tu ne verras plus de mal\FTNT{Ps. 46:5-6~; Col. 2:14.}.
\VS{16}En ce temps-là, on dira à Jérusalem~: Ne crains point Sion, que tes mains ne défaillent point~!
\VS{17}Yahweh, ton Dieu, est au milieu de toi comme le Puissant qui sauve~; il se réjouira à cause de toi d'une grande joie~; il se taira à cause de son amour, et se réjouira à cause de toi avec chant de triomphe.
\VS{18}Je rassemblerai ceux qui sont tristes à cause de l'assemblée solennelle, ils sont sortis de toi~; sur eux pèse l'opprobre.
\VS{19}Voici, je détruirai en ce temps-là tous ceux qui t'auront affligé~; je sauverai la boiteuse, je recueillerai celle qui avait été chassée, et je les ferai louer et devenir célèbres, dans tous les pays où ils auront été couverts de honte.
\VS{20}En ce temps-là, je vous ramènerai, et en ce temps-là je vous rassemblerai~; car je vous rendrai célèbres et un sujet de louange parmi tous les peuples de la terre, quand je ramènerai vos captifs sous vos yeux, dit Yahweh.
\PPE{}
\end{multicols}

%\clearpage\ShortTitle{Ag.}\BookTitle{Aggée}\BFont
\noindent\hrulefill
{\footnotesize
\textit{
\bigskip
{\centering{}
\\Auteur~: Aggée
\\(Heb.~: Chaggay)
\\Signification~: En fête, né un jour de fête
\\Thème~: Reconstruction du temple
\\Date de rédaction~: 6\up{ème} siècle av. J.-C.\\}
}
\textit{
\\Aggée, contemporain de Zacharie, exerça son service dans le royaume de Juda après le retour de l'exil. Alors que la reconstruction du temple était négligée, Aggée reçut un message rappelant au peuple quelles devaient être ses priorités et redéfinissant les exigences de Yahweh en matière de sainteté. Ce récit montre la bénédiction accompagnant celui qui oublie ses propres intérêts et qui prend véritablement à cœur l'œuvre de Dieu.\bigskip
}
}
\par\nobreak\noindent\hrulefill
\begin{multicols}{2}
\Chap{1}
\TextTitle{Israël coupable de négligence}
\VerseOne{}La seconde année du roi Darius, le premier jour du sixième mois, la parole de Yahweh vint par le moyen d'Aggée, le prophète, à Zorobabel, fils de Schealthiel, gouverneur de Juda, et à Josué, fils de Jotsadak, le grand-prêtre, en ces mots\FTNT{Esd. 4:24.}~:
\VS{2}Ainsi parle Yahweh des armées, en disant~:~Ce peuple dit : Le temps n'est pas encore venu, le temps de rebâtir la maison de Yahweh.
\VS{3}C'est pourquoi la parole de Yahweh a été adressée par le moyen d'Aggée, le prophète, en disant~:
\VS{4}Est-il temps pour vous d'habiter dans vos maisons lambrissées pendant que cette maison est en ruine~?
\VS{5}Maintenant donc, ainsi parle Yahweh des armées~: Considérez attentivement votre conduite~!
\VS{6}Vous avez semé beaucoup, mais vous avez récolté peu. Vous avez mangé, mais non pas jusqu'à être rassasiés. Vous avez bu, mais vous n'avez pas eu de quoi boire abondamment. Vous avez été vêtus, mais non pas jusqu'à en être échauffés. Et celui qui se loue, se loue pour mettre son salaire dans un sac percé\FTNT{Mi. 6:14-15.}.
\VS{7}Ainsi parle Yahweh des armées~: Considérez attentivement vos chemins~!
\VS{8}Montez à la montagne, apportez du bois, et bâtissez cette maison~; et j'y prendrai mon plaisir et je serai glorifié, a dit Yahweh.
\VS{9}Vous comptiez sur beaucoup, et voici, il y a eu peu~; vous l'avez apporté à la maison et j'ai soufflé dessus. Pourquoi~? A cause de ma maison, dit Yahweh des armées, parce qu'elle est en ruine pendant que vous vous empressez chacun pour sa maison.
\VS{10}A cause de cela, les cieux au-dessus de vous retiennent la rosée, et la terre a retenu ses fruits\FTNT{Lé. 26:19~; De. 28:23.}.
\VS{11}Et j'ai appelé la sécheresse sur la terre, et sur les montagnes, et sur le blé, et sur le moût, et sur l'huile, et sur tout ce que la terre produit, et sur les hommes et sur les bêtes, et sur tout le travail des mains\FTNT{Am. 4:7~; Ps. 105:16.}.
\TextTitle{Yahweh réveille son peuple}
\VS{12}Zorobabel donc, fils de Schealthiel, et Josué, fils de Jotsadak, le grand-prêtre, et tout le reste du peuple, entendirent la voix de Yahweh, leur Dieu, et les paroles d'Aggée, le prophète, ainsi que Yahweh, leur Dieu, l'avait envoyé~; et le peuple eut de la crainte devant Yahweh.
\VS{13}Et Aggée, messager de Yahweh, parla au peuple, suivant le message de Yahweh, en disant~: Je suis avec vous, dit Yahweh.
\VS{14}Et Yahweh réveilla l'esprit de Zorobabel, fils de Schealthiel, gouverneur de Juda, et l'esprit de Josué, fils de Jotsadak, le grand-prêtre, et l'esprit de tout le reste du peuple. Et ils vinrent et travaillèrent à la maison de Yahweh, leur Dieu,
\VS{15}le vingt-quatrième jour du sixième mois, de la seconde année du roi Darius.
\Chap{2}
\TextTitle{Encouragements à poursuivre la construction}
\VerseOne{}Le vingt et unième jour du septième mois, la parole de Yahweh vint par le moyen d'Aggée, le prophète, en disant~:
\VS{2}Parle maintenant à Zorobabel, fils de Schealthiel, gouverneur de Juda, et à Josué, fils de Jotsadak, le grand-prêtre, et au reste du peuple, en disant~:
\VS{3}Quel est parmi vous le survivant qui ait vu cette maison dans sa première gloire~? Et comment la voyez-vous maintenant~? N'est-elle pas comme un rien devant vos yeux, au prix de celle-là\FTNT{Esd. 3:12.}~?
\VS{4}Maintenant donc Zorobabel, fortifie-toi~! dit Yahweh. Toi aussi, Josué, fils de Jotsadak, grand-prêtre, fortifie-toi~! Vous aussi, tout le peuple du pays, fortifiez-vous~! dit Yahweh. Et travaillez, car je suis avec vous, dit Yahweh des armées.
\VS{5}La parole de l'Alliance que je traitai avec vous, quand vous sortîtes d'Egypte, et mon Esprit, demeurent au milieu de vous~; ne craignez point\FTNT{Za. 4:6.}~!
\VS{6}Car ainsi parle Yahweh des armées~: Encore un peu de temps, et j'ébranlerai les cieux et la terre, la mer et le sec\FTNT{Hé. 12:26.}~;
\VS{7}j'ébranlerai toutes les nations~; et les trésors de toutes les nations viendront, et je remplirai de gloire cette maison, dit Yahweh des armées.
\VS{8}L'argent est à moi, et l'or est à moi, dit Yahweh des armées.
\VS{9}La gloire de cette dernière maison sera plus grande que celle de la première, dit Yahweh des armées~; et je mettrai la paix en ce lieu, dit Yahweh des armées.
\TextTitle{Purification et sanctification du peuple}
\VS{10}Le vingt-quatrième jour du neuvième mois de la seconde année de Darius, la parole de Yahweh vint par le moyen d'Aggée le prophète, en disant~:
\VS{11}Ainsi parle Yahweh des armées~: Interroge maintenant les prêtres sur la loi en ces mots~:
\VS{12}Si quelqu'un porte de la chair consacrée dans le pan de son vêtement, et que ce vêtement touche du pain, ou un mets cuit, ou du vin, ou de l'huile, ou un aliment quelconque, cela devient-il sanctifié~? Et les prêtres répondirent et dirent~: Non~!
\VS{13}Alors Aggée dit~: Si celui qui est souillé pour un mort touche toutes ces choses-là, ne seront-elles pas souillées~? Et les prêtres répondirent et dirent~: Elles seront souillées\FTNT{Lé. 17:15~; No. 19:22~; Tit. 1:15.}.
\VS{14}Alors Aggée répondit et dit~: Tel est ce peuple et telle est cette nation devant ma face, dit Yahweh~; et telles sont toutes les œuvres de leurs mains~; même ce qu'ils offrent ici est souillé.
\VS{15}Maintenant donc mettez ceci, je vous prie, dans votre cœur, depuis ce jour et par la suite, avant qu'on ait mis pierre sur pierre au temple de Yahweh~!
\VS{16}Avant cela, dis-je, quand on venait à un monceau de blé, au lieu de vingt mesures, il n'y en avait que dix~; et quand on  venait au pressoir, au lieu de puiser de la cuve cinquante mesures, il n'y en avait que vingt.
\VS{17}Je vous ai frappés de brûlure, de rouille, de grêle, dans tout le travail de vos mains. Et vous n'êtes point revenus à moi, dit Yahweh\FTNT{De. 28:22~; 1 R. 8:37~; Am. 4:9~; 2 Ch. 6:28.}.
\VS{18}Mettez maintenant ceci dans votre cœur~; depuis ce jour-ci et dans la suite~; depuis, dis-je, le vingt-quatrième jour du neuvième mois, depuis le jour où les fondements du temple de Yahweh ont été posés, mettez ceci dans votre cœur~!
\VS{19}Ya-t-il encore de la semence dans les greniers~? Même jusqu'à la vigne, au figuier, au grenadier, et à l'olivier, rien n'a rapporté~; mais depuis ce jour-ci, je donnerai la bénédiction.
\TextTitle{Destruction des royaumes des nations}
\VS{20}Et la parole de Yahweh vint pour la seconde fois à Aggée, le vingt-quatrième jour du mois, en disant~:
\VS{21}Parle à Zorobabel, gouverneur de Juda, et dis-lui~: J'ébranlerai les cieux et la terre~;
\VS{22}je renverserai le trône des royaumes, je détruirai la force des royaumes des nations, je renverserai les chars et ceux qui les montent~; et les chevaux et ceux qui les montent seront abattus, chacun par l'épée de son frère.
\VS{23}En ce jour-là, dit Yahweh des armées, je te prendrai, ô Zorobabel, fils de Schealthiel, mon serviteur, dit Yahweh~; et je te mettrai comme un sceau\FTNT{Le sceau est un symbole d'autorité.}, car je t'ai choisi, dit Yahweh des armées.
\PPE{}
\end{multicols}

%\clearpage\ShortTitle{Zacharie}\BookTitle{Zacharie}\BFont
\noindent\hrulefill
{\footnotesize
\textit{
\bigskip
{\centering{}
\\Signifie : Yahweh se souvient
\\Thème : Les deux avènements du Messie
\\Auteur : Zacharie
\\Date de rédaction : 6ème siècle av. J.-C.\\}
}
%\bigskip
\textit{
\\Zacharie, contemporain d’Aggée, exerça son ministère en Juda au retour des exilés de Babylone, où il était né. Il annonça la venue du Messie et raconta de manière très précise différents épisodes de sa vie, également retrouvés dans le récit des évangiles. Il dévoila également quelques-uns des attributs du Sauveur, premièrement rejeté mais finalement accepté par le peuple juif pendant le millenium. On y découvre ainsi le Christ en tant que souverain sacrificateur, germe, serviteur, ange de l’Yahweh, roi de paix, fils de David…\bigskip
}
}
\par\nobreak\noindent\hrulefill
\begin{multicols}{2}
\Chap{1}
\TextTitle{Yahweh avertit son peuple}
\VerseOne{}Le huitième mois de la deuxième année de Darius, la parole de Yahweh fut adressée à Zacharie, le prophète, fils de Bérékia, fils d’Iddo, en ces mots :
\VS{2}Yahweh a été extrêmement irrité contre vos pères.
\VS{3}C'est pourquoi tu leur diras : Ainsi parle Yahweh des armées : Revenez à moi, dit Yahweh des armées, et je reviendrai à vous, dit Yahweh des armées\FTNT{Joë. 2:12 ; Es. 31:6 ; Jé. 3:12.}.
\VS{4}Ne soyez point comme vos pères, auxquels s’adressaient les premiers prophètes, en disant : Ainsi a dit Yahweh des armées : Détournez-vous maintenant de vos mauvaises voies et de vos mauvaises actions ! Mais ils n’écoutèrent pas, ils ne furent pas attentifs à ce que je leur disais, dit Yahweh\FTNT{2 Ch. 29:6 ; Esd. 9:7 ; Né. 9:16 ; La. 5:7.}.
\VS{5}Vos pères où sont-ils ? Et ces prophètes-là pouvaient-ils vivre éternellement ?
\VS{6}Cependant mes paroles et mes ordonnances que j'avais données aux prophètes, mes serviteurs, n'ont-elles pas atteint vos pères ? De sorte qu’étant revenus, ils ont dit : Yahweh des armées nous a traités comme il avait résolu de le faire, selon nos voies et nos actions.
\TextTitle{Le cavalier sur le cheval roux}
\VS{7}Le vingt-quatrième jour du onzième mois, qui est le mois de Schebat, la deuxième année de Darius, la parole de Yahweh fut adressée à Zacharie, le prophète, fils de Bérékia, fils d’Iddo, en ces mots :
\VS{8}Je voyais de nuit une vision, et voici, un homme était monté sur un cheval roux, et il se tenait parmi des myrtes qui étaient dans un lieu creux ; il y avait derrière lui des chevaux roux, fauves et blancs\FTNT{Ap. 6:2-4.}.
\VS{9}Je dis : Mon Seigneur ! Que signifient ces choses ? Et l’Ange qui me parlait me dit : Je te montrerai ce que signifient ces choses.
\VS{10}L’homme qui se tenait parmi les myrtes répondit et dit : Ce sont ceux que Yahweh a envoyés pour parcourir la terre.
\VS{11}Et ils répondirent à l'Ange de Yahweh\FTNT{Voir commentaire en Ge. 16:7.} qui se tenait parmi les myrtes, et dirent : Nous avons parcouru la terre ; et voici, toute la terre est en repos et tranquille.
\TextTitle{La compassion de Yahweh pour Jérusalem}
\VS{12}Alors l'Ange de Yahweh répondit et dit : Yahweh des armées, jusqu'à quand n'auras-tu pas compassion de Jérusalem et des villes de Juda, contre lesquelles tu es irrité depuis soixante-dix ans\FTNT{Jérémie prophétisa que la captivité babylonienne durerait soixante-dix ans (Jé. 25:11-12 ; Jé. 29:10). Les soixante-dix ans commencèrent à la déportation de la famille royale à Babylone en 605 av. J.-C. (2 R. 24 ; Da. 1) et se terminèrent avec la première vague de retours conduite par Zorobabel (Esd. 1). Les Israélites furent emmenés en captivité en plusieurs vagues. Le livre d’Esdras raconte les deux premières. En 538 av. J.-C., Zorobabel mena la première vague et fut nommé gouverneur (Ag. 1:1). Le sacrificateur Josué  (Esd. 3:2) et les prophètes Aggée et Zacharie (Es. 5:1-2) le secondaient. Leur plus grand défi fut de rebâtir le temple. Puisque la seule tribu à retourner en masse fut celle de Juda, dès lors, le reste du peuple fut appelé «~les Juifs~» (Esd. 4:23).} ?
\VS{13}Yahweh répondit à l'Ange qui me parlait, par de bonnes paroles, par des paroles de consolation.
\VS{14}Puis l'Ange qui me parlait me dit : Crie, en disant : Ainsi parle Yahweh des armées : Je suis ému d'une grande jalousie pour Jérusalem et pour Sion,
\VS{15}et je suis extrêmement irrité contre les nations qui sont à leur aise ; car je n’étais que peu irrité, mais elles ont contribué au mal.
\VS{16}C'est pourquoi ainsi parle Yahweh : Je reviens à Jérusalem avec compassion, et ma maison y sera rebâtie, dit Yahweh des armées ; et le cordeau sera étendu sur Jérusalem.
\VS{17}Crie encore, et dis : Ainsi parle Yahweh des armées : Mes villes regorgeront encore de biens, et Yahweh consolera encore Sion, il choisira encore Jérusalem.
\TextTitle{Les quatre cornes et les quatre forgerons}
\VS{18}Puis je levai les yeux et je regardai ; et voici, quatre cornes\FTNT{Da. 7:7-11 ; Da. 8:22 ; Ap. 13:1-11.}.
\VS{19}Alors je dis à l'Ange qui me parlait : Que veulent dire ces choses ? Et il me répondit : Ce sont les cornes qui ont dispersé Juda, Israël et Jérusalem.
\VS{20}Puis Yahweh me fit voir quatre forgerons.
\VS{21}Je dis : Que viennent-ils faire ? Et il répondit et dit : Ce sont les cornes qui ont dispersé Juda, au point que personne ne lève la tête ; et ces forgerons sont venus pour les effrayer, et pour abattre les cornes des nations qui ont levé la corne contre le pays de Juda, pour le disperser.
\Chap{2}
\TextTitle{L'homme tenant dans sa main le cordeau pour mésurer}
\VerseOne{}Je levai encore mes yeux et je regardai, et voici, il y avait un homme tenant dans la main un cordeau pour mesurer,
\VS{2}auquel je dis : Où vas-tu ? Et il me répondit : Je vais mesurer Jérusalem, pour voir quelle est sa largeur et quelle est sa longueur.
\VS{3}Et voici, l'Ange qui me parlait s’avança, et un autre ange sortit à sa rencontre.
\TextTitle{Yahweh, la gloire de Jérusalem}
\VS{4}Il lui dit : Cours, et parle à ce jeune homme, et dis : Jérusalem sera habitée comme les villes sans murailles, à cause de la multitude d'hommes et de bêtes qui seront au milieu d'elle\FTNT{Né. 1:3 ; Né. 2:13.}.
\VS{5}Mais je serai pour elle, dit Yahweh, une muraille de feu tout autour, et je serai sa gloire au milieu d'elle\FTNT{Es. 60:19.}.
\VS{6}Ha ! Fuyez, fuyez hors du pays du nord ! dit Yahweh. Car je vous ai dispersés aux quatre vents des cieux, dit Yahweh.
\VS{7}Ha ! Sauve-toi, Sion, toi qui habites chez la fille de Babylone\FTNT{Jé. 50:8 ; Es. 48:20 ; Es. 52:11 ; Jé. 51:6.} !
\VS{8}Car ainsi parle Yahweh des armées, lequel après la gloire, il m'a envoyé vers les nations qui ont fait de vous leur proie ; car celui qui vous touche, touche à la prunelle de son œil\FTNT{De. 32:10 ; Ps. 17:8.}.
\VS{9}Car voici, je vais lever ma main contre elles, et elles seront la proie de ceux qui leur étaient asservis. Et vous saurez que Yahweh des armées m'a envoyé.
\VS{10}Pousse des cris d’allégresse et réjouis-toi, fille de Sion ! Car voici, je viens\FTNT{Jésus-Christ est Yahweh qui vient ! C’est la seconde venue de Jésus-Christ qui est évoquée ici (Es. 40:10-11 ; Za. 12:10-14 ; Za. 14:1-10 ; Ac. 1:1-11 ; Ap. 1:7-8 ; Ap. 22:12-17).}, et j'habiterai au milieu de toi, dit Yahweh.
\VS{11}Beaucoup de nations se joindront à Yahweh en ce jour-là, et deviendront mon peuple ; et j'habiterai au milieu de toi ; et tu sauras que Yahweh des armées m'a envoyé vers toi.
\VS{12}Yahweh possédera Juda comme sa part dans la terre sainte, et il choisira encore Jérusalem.
\VS{13}Que toute chair fasse silence devant la face de Yahweh ! Car il s'est réveillé de sa demeure sainte.
\Chap{3}
\TextTitle{Yahweh enlève l'iniquité du pays}
\VerseOne{}Puis Yahweh me fit voir Josué, le souverain sacrificateur, se tenant debout devant l'Ange de Yahweh\FTNT{Voir commentaire en Ge. 16:7.}, et Satan qui se tenait debout à sa droite, pour l’accuser.
\VS{2}Yahweh dit à Satan : Que Yahweh te réprime, ô Satan ! Que Yahweh, dis-je, qui a choisi Jérusalem, te réprime ! N’est-ce pas là un tison qui a été retiré du feu\FTNT{Jud. 1:9 ; Am. 4:11.} ?
\VS{3}Or Josué était vêtu de vêtements sales, et il se tenait debout devant l'Ange.
\VS{4}L’Ange prit la parole et dit à ceux qui étaient debout devant lui : Otez-lui ces vêtements sales ! Et il dit à Josué : Regarde, je t’enlève ton iniquité, et je te revêts d’habits de fête.
\VS{5}Je dis : Qu'on mette sur sa tête un turban pur ! Et ils mirent un turban pur sur sa tête, puis ils lui mirent des vêtements\FTNT{Ap. 19:8.}. L’Ange de Yahweh était présent.
\VS{6}Alors l'Ange de Yahweh fit à Josué cette déclaration, en disant :
\VS{7}Ainsi parle Yahweh des armées : Si tu marches dans mes voies, et si tu observes mes commandements, tu jugeras ma maison, tu garderas mes parvis, et je te donnerai libre accès parmi ceux qui se tiennent devant moi.
\VS{8}Ecoute maintenant, Josué, souverain sacrificateur, toi, et tes compagnons qui sont assis devant toi ! Car ce sont des hommes qui serviront de signes. Certainement voici je ferai venir mon serviteur, le Germe\FTNT{Le Germe est un autre nom de Jésus-Christ, notre Seigneur (Es. 4:2).}.
\VS{9}Car voici, quant à la pierre\FTNT{Jésus-Christ est le Rocher des âges (Es. 8:13-17; Ap. 5:1-7).} que j'ai mise devant Josué, sur cette pierre, qui n'est qu'une\FTNT{Cette pierre est UNE (~E’had~) c’est-à-dire indivisible (De. 6:4).}, il y a sept yeux. Voici, je graverai moi-même ce qui doit y être gravé, dit Yahweh des armées ; et j'ôterai en un jour l'iniquité de ce pays.
\VS{10}En ce jour-là, dit Yahweh des armées, chacun de vous appellera son prochain sous la vigne et sous le figuier.
\Chap{4}
\TextTitle{Le peuple de Yahweh peut tout par son Esprit}
\VerseOne{}Puis l'Ange qui me parlait revint, et il me réveilla comme un homme que l’on réveille de son sommeil.
\VS{2}Il me dit : Que vois-tu ? Et je répondis : Je regarde, et voici, il y a un chandelier tout en or, surmonté d’un vase et portant ses sept lampes, avec sept conduits pour les sept lampes qui sont au sommet du chandelier\FTNT{Ap. 1:12-13.} ;
\VS{3}et il y a deux oliviers près de lui, l'un à la droite du vase, et l'autre à sa gauche.
\VS{4}Alors je pris la parole et je dis à l'Ange qui me parlait : Mon Seigneur, que signifient ces choses ?
\VS{5}L'Ange qui me parlait répondit et me dit : Ne sais-tu pas ce que signifient ces choses ? Je dis : Non, mon Seigneur !
\VS{6}Alors il reprit et me dit : C'est ici la parole que Yahweh adresse à Zorobabel : Ce n'est point par la puissance ni par la force, mais par mon Esprit, dit Yahweh des armées.
\VS{7}Qui es-tu, grande montagne, devant Zorobabel ? Tu seras aplanie. Il fera sortir la pierre principale ; il y aura des sons éclatants : Grâce, grâce pour elle !
\TextTitle{Yahweh encourage son peuple à achever l'oeuvre commencée}
\VS{8}Aussi la parole de Yahweh me fut adressée en ces mots :
\VS{9}Les mains de Zorobabel ont fondé cette maison, et ses mains l'achèveront ; et tu sauras que Yahweh des armées m'a envoyé vers vous.
\VS{10}Car qui est-ce qui a méprisé le jour des faibles commencements ? Ils se réjouiront en voyant le niveau dans la main de Zorobabel.  Ces sept\FTNT{Les sept yeux de Yahweh sont aussi les sept yeux de l’Agneau (Ap. 5 : 6). Ces yeux représentent l’omniscience et l’omniprésence de Jésus-Christ (Za. 3:9 ; Za 4:10. ; Za. 14:7 ; Jn. 16:30 ; Ac. 1:24 ; Ap. 21:17).} sont les yeux de Yahweh qui parcourent toute la terre.
\VS{11}Je pris la parole et je lui dis : Que signifient ces deux oliviers\FTNT{L’identité de ces deux individus est inconnue.  Selon Ap. 11:3, ces deux hommes recevront des pouvoirs incroyables pour les trois années et demi de la grande tribulation qui précéderont le retour du Christ. Si quiconque tente de leur faire du mal ou d’interférer dans leur ministère et leur témoignage, «~… du feu sortirait de leur bouche et dévorerait leurs ennemis~», (Ap. 11:5). Ils auront aussi le pouvoir de provoquer la sécheresse et la famine sur la terre, tout comme l’avait fait Elie (1 R. 17:1-7 ; 2 R. 1:9-15 ; Lu. 4:25). Ils auront également le pouvoir de frapper la Terre par des plaies diverses, semblables à celles provoquées par Moïse (Chapitres 7, 8, 9, 10, 11 d’Exode ; Ap. 11:6).}, à la droite et à la gauche du chandelier ?
\VS{12}Je pris la parole pour la seconde fois et je lui dis : Que signifient ces deux branches d'olivier qui sont près des deux conduits d'or, d’où l'or découle ?
\VS{13}Il me répondit et dit : Ne sais-tu pas ce que signifient ces choses ? Et je dis : Non, mon Seigneur.
\VS{14}Et il dit : Ce sont les  deux fils oints, qui se tiennent devant le Seigneur de toute la terre.
\Chap{5}
\TextTitle{La malédiction se répands sur Israël}
\VerseOne{}Puis je me retournai, et levai mes yeux pour regarder ; et voici, un rouleau qui volait.
\VS{2}Alors il me dit : Que vois-tu ? Je répondis : Je vois un rouleau qui vole, dont la longueur est de vingt coudées, et la largeur de dix coudées.
\VS{3}Et il me dit : C'est l’exécration du serment qui sort sur la face de tout le pays ; car selon elle, quiconque d'entre ce peuple-ci vole, sera puni comme elle ; et selon elle, quiconque d'entre ce peuple parjure, sera puni comme elle.
\VS{4}Je déploierai cette exécration dit Yahweh des armées, et elle entrera dans la maison du voleur, et dans la maison de celui qui jure faussement en mon Nom, et elle logera au milieu de leur maison, et la consumera avec son bois et ses pierres.
\TextTitle{L'épha au pays de Schinear}
\VS{5}L’Ange qui me parlait sortit, et me dit : Lève maintenant tes yeux, et regarde ce qui sort là.
\VS{6}Et je dis : Qu'est-ce ? Et il répondit : C'est l’épha\FTNT{L'épha était une unité de mesure utilisée dans le commerce des céréales, souvent à des fins frauduleuses (De. 25:14 ; Mi. 6:10 ; Am. 8:5).} qui sort dehors. Puis il dit : C'est ici leur aspect dans tout le pays.
\VS{7}Et voici, on portait une masse de plomb, et une femme était assise au milieu de l'épha\FTNT{Zacharie voit une femme assise au milieu de l'épha. L'ange déclare : «~C'est la méchanceté ou l’iniquité~». Elle représente la grande prostituée décrite en Ap. 17, avec sa coupe d'or pleine de ses abominations et des impuretés de sa fornication (v. 4). Cette femme est la figure du «~mystère de l'iniquité qui opère déjà~» (2 Th. 2:7).}.
\VS{8}Il dit : C'est là l’iniquité\FTNT{L’iniquité ou la méchanceté.} ; puis il la repoussa dans l'épha, et il jeta la masse de plomb sur l’ouverture.
\VS{9}Je levai les yeux et je regardai, et voici deux femmes sortirent\FTNT{Les deux femmes ayant «~des ailes de cigogne~» apparaissent portées par le vent. Sous Moïse, cet oiseau devait être considéré comme impur (Lé. 11 : 19). Dans les Ecritures, le vent est constamment en relation avec le jugement (Job. 27:20-22 ; Job. 30:22 ; Es. 7:2 ; Es. 26:6 ; Es. 41:16). Elles soulèvent l'épha et l'emportent dans son lieu d'origine, le pays de Schinear, c’est-à-dire Babylone, pour lui bâtir une maison, au siège même de l'idolâtrie et de la révolte contre Dieu. (Ge. 11:2-9 ; 2 R. 17:24).}. Le vent soufflait dans leurs ailes : Elles avaient des ailes comme les ailes de la cigogne. Et elles enlevèrent l'épha entre la terre et le ciel.
\VS{10}Je dis à l'Ange qui me  parlait : Où emportent-elles l'épha ?
\VS{11}Il me répondit : C'est pour lui bâtir une maison dans le pays de Schinear\FTNT{Schinear ou Babylone (Ge. 10:6-12).} ; et quand elle sera prête, il sera déposé là, sur sa base.
\Chap{6}
\TextTitle{Les quatre vents des cieux}
\VerseOne{}Je levai encore les yeux et je regardai, et voici quatre chars\FTNT{Dans les Ecritures, les chars et les chevaux représentent souvent la puissance de Dieu exerçant un jugement sur la terre (Jé. 46:9-10 ; Joë. 2:3-11). Ce jugement concerne le monde entier (Ap. 6:1-8).} sortaient d'entre deux montagnes ; et ces montagnes étaient des montagnes d'airain.
\VS{2}Au premier char, il y avait des chevaux roux ; au deuxième char, des chevaux noirs,
\VS{3}au troisième char, des chevaux blancs, et au quatrième char, des chevaux tachetés, rouges.
\VS{4}Je pris la parole et je dis à l'Ange qui me parlait : Mon Seigneur, que veulent dire ces choses ?
\VS{5}L’Ange répondit et me dit : Ce sont les quatre vents des cieux, qui sortent du lieu où ils se tenaient devant le Seigneur de toute la terre.
\VS{6}Quant au char où sont les chevaux noirs, ils se dirigent vers le pays du nord, et les blancs sortent après eux ; les tachetés se dirigent vers le pays du midi.
\VS{7}Ensuite les rouges sortirent et demandèrent à aller parcourir la terre. L’Ange leur dit : Allez, et parcourez la terre ! Et ils parcoururent la terre.
\VS{8}Puis il m'appela, et me parla, en disant : Voici, ceux qui se dirigent vers le pays du nord ont apaisé mon Esprit dans le pays du nord.
\TextTitle{Prophétie sur le règne du germe de Yahweh}
\VS{9}La parole de Yahweh me fut adressée en ces mots :
\VS{10}Tu recevras les dons de ceux qui sont de retour de la captivité : Heldaï, Tobija et Jedaeja. Et tu iras toi-même ce même jour-là, et tu iras dans la maison de Josias, fils de Sophonie, où ils se sont rendus en arrivant de Babylone.
\VS{11}Tu prendras de l'argent et de l'or, et tu en feras des couronnes que tu mettras sur la tête de Josué, fils de Jotsadak, le souverain sacrificateur.
\VS{12}Tu lui diras : Ainsi parle Yahweh des armées : Voici un homme, dont le nom est Germe\FTNT{Es. 4:2.}, germera dans son lieu, et bâtira le temple de Yahweh\FTNT{C’est Yahweh, c’est-à-dire Jésus-Christ lui-même, qui bâtit son temple (Ps. 127:1-2 ; Mt. 16:18).}.
\VS{13}Oui, lui-même bâtira le temple de Yahweh ; et lui-même sera rempli de majesté. Il s’assiéra et dominera sur son trône, il sera Sacrificateur\FTNT{Jésus-Christ est Souverain Sacrificateur (Hé. 6:20 ; Hé. 7:1-28).}, étant sur son trône ; et il y aura un conseil de paix entre les deux.
\VS{14}Les couronnes seront pour Hélem, Tobija et Jedaeja, et pour Hen, fils de Sophonie, un souvenir dans le temple de Yahweh.
\VS{15}Ceux qui sont éloignés viendront, et travailleront au temple de Yahweh ; et vous saurez que Yahweh des armées m'a envoyé vers vous. Cela arrivera, si vous écoutez attentivement la voix de Yahweh, votre Dieu.
\Chap{7}
\TextTitle{Yahweh dénonce le jeûne formaliste}
\VerseOne{}La quatrième année du roi Darius, la parole de Yahweh fut adressée à Zacharie, le quatrième jour du neuvième mois, qui est le mois de Kisleu.
\VS{2}On avait envoyé à Béthel Scharetser et Réguem-Mélec avec ses gens, pour supplier Yahweh,
\VS{3}et pour parler aux sacrificateurs de la maison de Yahweh des armées, et aux prophètes, en disant : Dois-je pleurer au cinquième mois, et faire abstinence, comme j'ai déjà fait pendant plusieurs années ?
\VS{4}La parole de Yahweh des armées me fut adressée en ces mots :
\VS{5}Parle à tout le peuple du pays et aux sacrificateurs, et dis-leur : Quand vous avez jeûné et pleuré au cinquième mois et au septième, et cela depuis soixante-dix ans, avez-vous célébré ce jeûne par amour pour moi ?
\VS{6}Et quand vous buvez et mangez, n'est-ce pas vous qui mangez et vous qui buvez\FTNT{Es. 58:3-4.} ?
\VS{7}Ne connaissez-vous pas les paroles qu’a proclamées Yahweh par les premiers prophètes, lorsque Jérusalem était habitée et paisible avec ses villes à l’entour, et que le midi et la plaine étaient habités ?
\TextTitle{Yahweh n'exauce pas les pécheurs}
\VS{8}Puis la parole de Yahweh fut adressée à Zacharie en ces mots :
\VS{9}Ainsi parlait Yahweh des armées, en disant : Rendez véritablement la justice, et exercez la miséricorde et la compassion chacun envers son frère.
\VS{10}N’opprimez pas la veuve et l'orphelin, l'étranger et le pauvre, et ne méditez aucun mal dans vos cœurs chacun contre son frère\FTNT{Ex. 22:21 ; Es. 1:23 ; Jé. 5:28 ; Pr. 22:22-23.}.
\VS{11}Mais ils refusèrent d’être attentifs, ils eurent l'épaule rebelle, et ils endurcirent leurs oreilles pour ne pas entendre.
\VS{12}Ils rendirent leur cœur dur comme le diamant, pour ne pas écouter la loi et les paroles que Yahweh des armées adressait par son Esprit, par les premiers prophètes. C’est pourquoi  Yahweh des armées s’enflamma d’une grande colère.
\VS{13}Quand il appelait, ils n'ont pas écouté. Aussi n’ai-je pas écouté, quand ils ont appelé, dit Yahweh des armées\FTNT{Pr. 1:28 ; Es. 1:15 ; Jé. 11:11.}.
\VS{14}Je les ai dispersés comme par un tourbillon parmi toutes les nations qu'ils ne connaissaient pas ; le pays a été dévasté derrière eux, il n’y a plus eu ni allants ni venants ; et d’un pays de délices ils ont fait un désert.
\Chap{8}
\TextTitle{Futur royaume d'Israël rétabli dans la justice}
\VerseOne{}La parole de Yahweh des armées me fut encore adressée en ces mots :
\VS{2}Ainsi parle Yahweh des armées : Je suis jaloux pour Sion d'une grande jalousie, et je suis jaloux pour elle d’une grande fureur.
\VS{3}Ainsi parle Yahweh : Je retourne à Sion, et j'habiterai au milieu de Jérusalem ; et Jérusalem sera appelée ville fidèle ; et la montagne de Yahweh des armées sera appelée montagne sainte\FTNT{Es. 1:26.}.
\VS{4}Ainsi parle Yahweh des armées : Il y aura encore des vieillards et des femmes âgées, assis dans les rues de Jérusalem, et chacun aura son bâton à la main, à cause du grand nombre de leurs jours.
\VS{5}Les rues de la ville seront remplies de fils et de filles, jouant dans les rues.
\VS{6}Ainsi parle Yahweh des armées : S’il semble difficile aux yeux du reste de ce peuple que cela arrive, en ces jours-là, sera-t-il de même difficile à mes yeux ? dit Yahweh des armées.
\VS{7}Ainsi parle Yahweh des armées : Voici, je délivre mon peuple du pays de l'orient et du pays du soleil couchant.
\VS{8}Je les ramènerai, et ils habiteront au milieu de Jérusalem ; ils seront mon peuple, et je serai leur Dieu avec vérité et droiture.
\TextTitle{Juger selon la vérité}
\VS{9}Ainsi parle Yahweh des armées : Que vos mains soient fortifiées, vous qui entendez aujourd’hui ces paroles de la bouche des prophètes qui parurent au jour où la maison de Yahweh fut fondée, et où le temple allait être bâti\FTNT{Ag. 2:4.}.
\VS{10}Car avant ces jours-là, il n'y avait pas de salaire pour l'homme ni de salaire pour la bête ; et il n'y avait pas de paix pour ceux qui entraient et sortaient, à cause de la détresse ; et je lâchais tous les hommes les uns contre les autres.
\VS{11}Mais maintenant je ne serai pas pour le reste de ce peuple comme les premiers jours,  dit Yahweh des armées.
\VS{12}Car les semailles prospéreront, la semence de paix sera là ; la vigne rendra son fruit, et la terre donnera ses produits ; les cieux donneront leur rosée, et je ferai hériter toutes ces choses au reste de ce peuple.
\VS{13}De même que vous avez été en malédiction parmi les nations, ô maison de Juda et maison d'Israël, de même je vous délivrerai, et vous serez en bénédiction. Ne craignez pas, mais que vos mains soient fortifiées\FTNT{Ge. 1:11 ; Ap. 22:2.}.
\VS{14}Car ainsi parle Yahweh des armées : Comme j'ai eu la pensée de vous affliger, lorsque vos pères ont provoqué ma colère, dit Yahweh des armées, et que je ne m'en suis point repenti,
\VS{15}ainsi je reviens en arrière et j’ai résolu en ces jours de faire du bien à Jérusalem, et à la maison de Juda. Ne craignez pas !
\VS{16}Voici les choses que vous devez faire : Que chacun dise la vérité à son prochain ; jugez selon la vérité et prononcez un jugement en vue de la paix dans vos portes\FTNT{Ep. 4:25 ; Ex. 20:16 ; Mt. 19:18 ; Lu. 18:20.} ;
\VS{17}que personne ne projette du mal dans son cœur contre son prochain ; et n'aimez point le faux serment, car ce sont là des choses que je hais, dit Yahweh\FTNT{Ps. 5:5 ; Ps. 11:5 ; Pr. 6:16-19.}.
\VS{18}Puis la parole de Yahweh des armées me fut adressée en ces mots :
\VS{19}Ainsi parle Yahweh des armées : Le jeûne du quatrième mois, le jeûne du cinquième, le jeûne du septième et le jeûne du dixième seront changés pour la maison de Juda en joie et en allégresse, et en fêtes solennelles de réjouissance. Aimez donc la vérité et la paix\FTNT{Ep. 4:15.}.
\TextTitle{Les nations reconnaissent que Yahweh est le seul Dieu}
\VS{20}Ainsi parle Yahweh des armées : Il viendra encore des peuples et des habitants de plusieurs villes.
\VS{21}Les habitants d’une ville  iront à l'autre, en disant : Allons, allons implorer Yahweh et chercher Yahweh des armées ! Nous irons aussi !
\VS{22}Et beaucoup de peuples et de puissantes nations viendront rechercher Yahweh des armées à Jérusalem\FTNT{Jérusalem est appelée à devenir le centre d’adoration de la terre et la capitale du monde à cause de la présence de Dieu (Es. 66:23 ; Za. 14:16-21).}, et implorer Yahweh.
\VS{23}Ainsi parle Yahweh des armées : En ce jour-là, dix hommes de toutes les langues des nations saisiront le pan de la robe d'un homme Juif, et diront : Nous irons avec vous, car nous avons entendu que Dieu est avec vous.
\Chap{9}
\TextTitle{Le jugement de Yahweh sur les nations}
\VerseOne{}Oracle, parole de Yahweh sur le pays de Hadrac. Elle s’arrête sur Damas, car Yahweh a l’œil sur les hommes et sur toutes les tribus d'Israël.
\VS{2}Il s’arrête aussi sur Hamath, à la frontière de Damas, sur Tyr, et Sidon, quoique chacune d'elles soit fort sage.
\VS{3}Car Tyr s'est bâti une forteresse ; elle a amassé l'argent comme la poussière, et l’or fin comme la boue des rues\FTNT{Ez. 28:3-17.}.
\VS{4}Voici, le Seigneur l'appauvrira, et en la frappant, il jettera sa puissance dans la mer, et elle sera consumée par le feu\FTNT{Ez. 26:3-4.}.
\VS{5}Askalon le verra, et elle sera dans la crainte ; Gaza aussi le verra, et un violent tremblement la saisira ; Ekron aussi, car son espoir sera confondu. Et il n'y aura plus de roi à Gaza, et Askelon ne sera plus habitée\FTNT{So. 2:4.}.
\VS{6}Et le bâtard habitera à Asdod ; et j’abattrai l'orgueil des Philistins.
\VS{7}J'ôterai le sang de la bouche de chacun d'eux, et leurs abominations d'entre leurs dents ; et lui aussi restera pour notre Dieu, il sera comme un chef en Juda, et Ekron sera comme le Jébusien.
\VS{8}Je camperai autour de ma maison, pour la défendre contre une armée, contre les allants et les venants, et l’oppresseur ne passera plus près d’eux ; car maintenant mes yeux sont fixés sur elle.
\TextTitle{Prophétie sur la première venue du Messie}
\VS{9}Sois transportée d’allégresse, fille de Sion ! Pousse des cris de joie, fille de Jérusalem ! Voici, ton Roi vient à toi ; il est juste et vainqueur, il est monté sur un âne, sur un âne, le petit d'une ânesse\FTNT{Cette prophétie s’est accomplie 500 ans après. Jésus est effectivement entré à Jérusalem monté sur un âne (Mt. 21:1-11 ; Lu. 19:28-40 ; Jn. 12:12-19).}.
\TextTitle{La vision du Messie pour Israël}
\VS{10}Je détruirai les chars d'Ephraïm, et les chevaux de Jérusalem ; et les arcs de guerre seront aussi retranchés.  Et le Roi parlera de paix aux nations ; et sa domination s'étendra d’une mer à l’autre, depuis le fleuve jusqu'aux extrémités de la terre\FTNT{Es. 57:19 ; Ps. 2:8 ; Ps. 72:8.}.
\VS{11}Quant à toi, à cause de ton alliance scellée par le sang, je retirerai tes captifs de la fosse où il n'y a pas d'eau.
\VS{12}Retournez à la forteresse, captifs pleins d’espérance ! Aujourd'hui même je le déclare, je te rendrai le double.
\VS{13}Car je bande Juda comme un arc, je m’arme d’Ephraïm comme d’un arc, et j’exciterai tes enfants, ô Sion, contre tes enfants, ô Javan ! Je te rendrai pareille à l’épée d’un vaillant homme.
\VS{14}Alors Yahweh au-dessus d’eux apparaîtra, et ses dards partiront comme l'éclair, et le Seigneur, Yahweh, sonnera du shofar, il s’avancera dans le tourbillon du midi.
\VS{15}Yahweh des armées sera leur protecteur ; ils dévoreront, après avoir subjugué ceux qui tirent les pierres de fronde ; ils boiront, ils seront bruyants comme des hommes ivres, ils se rempliront de vin comme un bassin, et comme les coins de l'autel.
\VS{16}Yahweh, leur Dieu, les sauvera en ce jour-là, comme le troupeau de son peuple ; car ils sont les pierres d’une couronne  qui brilleront dans son pays.
\VS{17}Car combien est grande sa bonté ! Quelle beauté ! Le froment fera croître les jeunes hommes, et le vin doux rendra ses vierges éloquentes.
\Chap{10}
\TextTitle{Yahweh rassemblera son peuple}
\VerseOne{}Demandez à Yahweh la pluie\FTNT{Les pluies en Israël : En Israël, la saison des pluies commence généralement vers la fin du mois d’octobre avec de légères pluies qui ramollissent la terre (Ps. 65:10), et se poursuit ensuite par de fortes précipitations intermittentes durant deux ou trois jours, tout au long des mois de novembre et de décembre. Ces fortes précipitations étaient appelées dans les écritures «~la pluie de la première saison~» (en hébreu «~yoreb~» ou «~moreh~). Les fermiers dépendaient de la pluie de la première saison pour que la terre dure comme le roc soit rendue apte au labour et à l’ensemencement. Quand ces fortes précipitations s’achèvent, des pluies plus fines continuent encore de façon intermittente. Toutefois, à l’approche de la moisson,  la forte pluie revenait gonfler le grain et le fruit en préparation. Celle-ci était connue comme étant «~la pluie de l’arrière-saison~» (Jé. 5:24; Joë. 2:23-24 ; Os. 6:3).}, la pluie au temps de la dernière saison ! Yahweh produira des éclairs, et il vous donnera une abondante pluie, il donnera à chacun de l'herbe dans son champ.
\VS{2}Car les théraphim ont des paroles vaines, et les devins prophétisent le mensonge, ils profèrent des songes vains et consolent par la vanité. C'est pourquoi ils sont errants comme des brebis, ils sont malheureux, parce qu'il n'y a point de pasteur\FTNT{Mt. 9:36 ; Ez. 34:2 ; Jé. 23:21-30.}.
\VS{3}Ma colère s'est enflammée contre ces pasteurs, et je châtierai ces boucs ; car Yahweh des armées visite son troupeau, la maison de Juda ; et il les a rangés en bataille comme son cheval d'honneur.
\VS{4}De lui sortira l’Angle\FTNT{De lui (Juda) sortira l’Angle ou la pierre angulaire (Jésus-Christ), (1 Pi. 2:7 ; Es. 8:13-17).}, de lui sortira le clou, de lui sortira l'arc de bataille, et de lui sortiront tous les chefs ensemble.
\VS{5}Ils seront comme des vaillants hommes foulant la boue des rues dans la bataille, et ils combattront, parce que Yahweh sera avec eux ; et les cavaliers seront confus.
\VS{6}Car je fortifierai la maison de Juda, et je sauverai la maison de Joseph ; je les ramènerai, et je les ferai habiter en repos, parce que j'aurai compassion d'eux, et ils seront comme si je ne les avais point rejetés ; car je suis Yahweh, leur Dieu, et je les exaucerai.
\VS{7}Et ceux d'Ephraïm seront comme un héros, et leur cœur se réjouira comme par le vin ; leurs fils le verront, et se réjouiront ; leur cœur se réjouira en Yahweh.
\VS{8}Je les sifflerai et les rassemblerai, car je les rachète ; et ils seront multipliés comme ils l'ont été auparavant.
\VS{9}Et après que je les aurai dispersés parmi les peuples, ils se souviendront de moi dans les pays éloignés, et ils vivront avec leurs enfants, et ils reviendront.
\VS{10}Ainsi je les ramènerai du pays d'Egypte, je les rassemblerai de l'Assyrie, je les ferai venir au pays de Galaad, et au Liban, et il n'y aura point assez d'espace pour eux.
\VS{11}Il passera la mer de détresse, et il frappera les flots de la mer ; et toutes les profondeurs du fleuve seront desséchées ; l'orgueil de l'Assyrie sera abattu, et le sceptre d'Egypte sera ôté.
\VS{12}Je les fortifierai en Yahweh, et ils marcheront en son Nom, dit Yahweh.
\Chap{11}
\TextTitle{Les houlettes du vrai berger}
\VerseOne{}Liban, ouvre tes portes, et que le feu dévore tes cèdres !
\VS{2}Cyprès, gémis, car le cèdre est tombé, parce que les choses magnifiques ont été ravagées ! Chênes de Basan, gémissez, car la forêt inaccessible est coupée !
\VS{3}Les pasteurs poussent des cris de lamentations, parce que leur magnificence est ravagée ; on entend le rugissement des lionceaux, parce que l'orgueil du Jourdain est abattu.
\VS{4}Ainsi parle Yahweh, mon Dieu : Pais les brebis exposées au carnage !
\VS{5}Leurs possesseurs les égorgent, sans qu'on les tienne pour coupables, et celui qui les vend dit : Béni soit Yahweh, car je m’enrichis !  Et leurs pasteurs ne les épargnent pas.
\VS{6}Car je n’ai plus de pitié pour les habitants du pays, dit Yahweh ; et voici, je livre les hommes aux mains les uns des autres et aux mains de leur roi ; ils ravageront le pays, et je ne le délivrerai pas de leur main.
\VS{7}Alors je me mis donc à paître les brebis exposées au carnage, qui sont véritablement les plus misérables du troupeau. Puis je pris deux verges : J'appelai l'une Grâce, et l'autre Cordon ; et je me mis à paître les brebis.
\VS{8}Et je supprimai les trois pasteurs en un mois ; car mon âme était impatiente à leur sujet, et leur âme aussi avait pour moi du dégoût.
\VS{9}Et je dis : Je ne vous paîtrai plus ; que celle qui va mourir meure, et que celle qui va périr périsse, et que celles qui restent se dévorent la chair les unes les autres.
\VS{10}Puis je pris ma verge, appelée Grâce, et je la brisai, pour rompre mon alliance que j'avais traitée avec tous ces peuples.
\VS{11}Elle fut rompue en ce jour-là ; et les plus malheureuses brebis\FTNT{Les plus malheureuses des brebis sont le reste d’Israël.}, qui prirent garde à moi, reconnurent ainsi que c'était la parole de Yahweh.
\VS{12}Je leur dis : S'il vous semble bon, donnez-moi mon salaire ; sinon, ne me le donnez pas.  Alors ils pesèrent\FTNT{Mt. 26:15 ; Mt. 27:9-10.}  pour mon salaire trente pièces d'argent\FTNT{Selon la loi de Moïse, pour racheter un mâle de 20 à 60 ans, ayant fait un vœu, il fallait payer cinquante sicles d'argent (Lé. 27:3). Pour dédommager un préjudice causé par un bœuf ayant frappé un esclave, on devait donner trente sicles d'argent au maître de l'esclave et lapider le bœuf (Ex. 21:32). Or le prix du Seigneur a été estimé à trente sicles d’argent, comme pour les esclaves.}.
\VS{13}Yahweh me dit : Jette-le au potier, ce prix honorable auquel ils m’ont estimé ! Alors je pris les trente pièces d'argent, et les jetai dans la maison de Yahweh, pour le potier.
\VS{14}Puis je brisai ma seconde verge, appelée Cordon, pour rompre la fraternité entre Juda et Israël.
\TextTitle{Caractéristiques du faux berger}
\VS{15}Yahweh me dit : Prends-toi encore l'équipage d'un berger insensé.
\VS{16}Car voici, je susciterai dans le pays un pasteur, qui ne visitera pas les brebis qui périssent ; il ne cherchera pas celles qui s’égarent, il ne guérira pas celles qui sont blessées, et il ne soutiendra pas celles qui sont saines, mais il dévorera la chair des plus grasses, et il déchirera jusqu’aux cornes de leurs pieds.
\VS{17}Malheur au pasteur inutile qui abandonne les brebis ! Que l’épée fonde sur son bras et sur son œil droit ! Que son bras se dessèche, et que son œil droit s’éteigne entièrement !
\Chap{12}
\TextTitle{Jérusalem, une coupe d'étourdissement pour les nations}
\VerseOne{}Oracle, parole de Yahweh, sur Israël. Ainsi parle Yahweh, qui a étendu les cieux et fondé la terre, et qui a formé l'esprit de l'homme au-dedans de lui :
\VS{2}Voici, je ferai de Jérusalem une coupe d'étourdissement pour tous les peuples d'alentour ; et aussi pour Juda dans le siège de Jérusalem\FTNT{Ap. 16:12-16.}.
\VS{3}En ce jour-là, je ferai de Jérusalem une pierre pesante pour tous les peuples ; tous ceux qui en porteront le poids seront entièrement écrasés, car toutes les nations de la terre s'assembleront contre elle.
\VS{4}En ce temps-là, dit Yahweh, je frapperai d'étourdissement tous les chevaux, et de folie ceux qui les monteront ; mais j’aurai les yeux ouverts sur la maison de Juda, et je frapperai d'aveuglement tous les chevaux des peuples.
\VS{5}Les chefs de Juda diront en leur cœur : Les habitants de Jérusalem sont notre force, par Yahweh des armées, leur Dieu.
\VS{6}En ce jour-là, je ferai des chefs de Juda comme un foyer de feu parmi du bois, et comme une torche enflammée parmi des gerbes ; ils dévoreront à droite et à gauche tous les peuples d'alentour ; et Jérusalem sera encore habitée à sa place, à Jérusalem.
\VS{7}Yahweh sauvera premièrement les tentes de Juda, afin que la gloire de la maison de David, la gloire des habitants de Jérusalem, ne s'élève point au-dessus de Juda.
\VS{8}En ce jour-là, Yahweh sera le protecteur des habitants de Jérusalem ; et le plus faible parmi eux sera en ce jour-là comme David ; la maison de David sera comme Dieu, comme l'Ange de Yahweh devant leur face.
\VS{9}En ce jour-là, je chercherai à détruire toutes les nations qui viendront contre Jérusalem.
\TextTitle{Repentance et délivrance d'Israël}
\VS{10}Et je répandrai sur la maison de David, et sur les habitants de Jérusalem, l'Esprit de grâce et de supplications\FTNT{Joë. 2:28-30.}, et ils regarderont vers moi\FTNT{Au retour du Messie, il y aura une repentance et une conversion nationale d’Israël (Ro. 11:26).}, celui qu’ils ont percé, et ils pleureront sur lui\FTNT{Celui qu’ils ont percé : Il est question ici du Seigneur Jésus, le Messie (Ap. 1:7).}, comme on pleure sur un fils unique, et ils pleureront amèrement sur lui, comme quand on pleure sur un premier-né.
\VS{11}En ce jour-là, il y aura un grand deuil à Jérusalem, comme le deuil d'Hadadrimmon dans la vallée de Meguiddon.
\VS{12}Le pays sera dans le deuil, chaque famille à part : La famille de la maison de David à part, et les femmes de cette maison-là à part ; la famille de la maison de Nathan à part, et les femmes de cette maison-là à part.
\VS{13}La famille de la maison de Lévi à part, et les femmes de cette maison-là à part ; la famille de Schimeï à part, et ses femmes à part.
\VS{14}Toutes les autres  familles, chaque famille à part, et leurs femmes à part.
\Chap{13}
\TextTitle{Dieu frappe les faux prophètes}
\VerseOne{}En ce jour-là, il y aura une source ouverte en faveur de la maison de David et des habitants de Jérusalem, pour le péché et pour la souillure.
\VS{2}En ce jour-là, dit Yahweh des armées, je retrancherai du pays les noms des faux dieux, et on n'en fera plus mention. J'ôterai aussi du pays les faux prophètes et l'esprit d'impureté.
\VS{3}Et il arrivera que si quelqu'un prophétise encore, son père et sa mère qui l’ont engendré, lui diront : Tu ne vivras plus ; car tu as prononcé des mensonges au Nom de Yahweh ; et son père et sa mère qui l’ont engendré, le transperceront quand il prophétisera.
\VS{4}En ce jour-là, les prophètes seront confus de leurs visions, quand ils prophétiseront ; et ils ne revêtiront plus un manteau de poil pour mentir.
\VS{5}Chacun d’eux dira : Je ne suis pas prophète, mais je suis laboureur, car on m'a appris à gouverner du bétail dès ma jeunesse.
\VS{6}Et si on lui demande : Que veulent donc dire ces blessures que tu as aux mains ? Et il répondra : C’est dans la maison de mes amis qu’on me les a faites.
\TextTitle{Prophétie sur le vrai berger, le Messie}
\VS{7}Epée, réveille-toi contre mon Berger\FTNT{Mon Berger : Il est question de Jésus-Christ, le Bon Berger (Ps. 23 ; Jn. 10:1-17).}, et sur l'homme qui est mon compagnon ! dit Yahweh des armées frappe le Berger, et les brebis seront dispersées\FTNT{Frappe le Berger : Cette prophétie fait référence à la crucifixion du Seigneur Jésus-Christ (Ge. 3:15 ; Mt. 26:31 ; Mc. 14:27 ; Mc. 14:50 ; Mc. 15:19).} ; et je tournerai ma main vers les faibles.
\TextTitle{Le reste de Yahweh épuré à travers l'épreuve}
\VS{8}Dans tout le pays, dit Yahweh, les deux tiers seront retranchées et périront, et l’autre tiers restera.
\VS{9}Je mettrai ce tiers dans le feu, et je le purifierai comme on purifie l'argent, je les éprouverai comme on éprouve l'or. Il invoquera mon Nom, et je l'exaucerai ; je dirai : C'est ici mon peuple ! Et il dira : Yahweh est mon Dieu\FTNT{1 Pi. 1:6-7 ; Ps. 50:15 ; Ps. 91:15 ; Ps. 144:15.} !
\Chap{14}
\TextTitle{Imminence du jour de Yahweh}
\VerseOne{}Voici, le jour de Yahweh\FTNT{L’expression «~le jour du Seigneur~» ou «~le jour de Yahweh~» est utilisée dix-neuf fois dans le Tanakh (Es. 2:12 ;  Es. 13:6 ; Es. 13:9 ; Ez. 13:5 ; Ez. 30:3 ; Joë. 1:15 ; Joë. 2:1 ; Joë. 2:11 ; Joë. 2:31 ;  Joë. 3:14 ; Am. 5:18-20 ; Ab. 1:15 ; So. 1:7 ; So. 1:14 ; Za. 14:1 ; Mal. 4:5) et quatre fois dans les textes de la nouvelle alliance (Ac. 2:20 ; 2 Th. 2:2 ; 2 Pi. 3:10 ; Ap. 6:17 ; Ap. 16:14). Cette expression désigne habituellement des événements qui se déroulent à la fin des temps (Es. 7:18-25). Elle désigne un espace de temps au cours duquel Dieu va intervenir personnellement dans l’histoire des hommes. Appelé  «~jour de colère~», «~jour de  visitation~», et «~grand jour du Dieu Tout-Puissant~» ; il se réfère ainsi à un accomplissement encore futur, quand la colère de Dieu viendra s’abattre sur l’Israël qui n’aura pas cru (Es. 22 ; Jé. 30:1-17 ; Joë. 1 et 2 ; Am. 5 ; So. 1) et sur tous les incrédules du monde (Ez. 38 et 39 ; Za. 14). Ce jour sera aussi un temps de salut puisque Dieu va délivrer «~le reste~» d’Israël, accomplissant ainsi sa promesse selon laquelle «~tout Israël sera sauvé~» (Ro. 11:26) : il pardonnera leurs péchés et restaurera le peuple qu’Il s’est choisi sur la terre promise à Abraham (Es. 10:27 ; Jé. 30:19-31 ; Mi. 4 ; Za. 13).} arrive, et tes dépouilles seront partagées au milieu de toi, Jérusalem.
\VS{2}Je rassemblerai toutes les nations à Jérusalem pour qu’elles lui fassent la guerre\FTNT{Joë. 3 ; Ap. 16:12-16.} ; la ville sera prise, les maisons pillées, et les femmes violées ; la moitié de la ville ira en captivité, mais le reste du peuple ne sera pas retranché de la ville.
\VS{3}Yahweh sortira, et il combattra contre ces nations, comme il a combattu au jour de la bataille.
\TextTitle{Retour visible et en gloire du Seigneur}
\VS{4}Ses pieds se poseront en ce jour sur la Montagne des Oliviers\FTNT{Ce sont les pieds de Jésus-Christ (Ac. 1:10-11).}, qui est vis-à-vis de Jérusalem, du côté de l’orient ; et la Montagne des Oliviers se fendra par le milieu, à l'orient et à l'occident, de sorte qu'il y aura une très grande vallée ; une moitié de la montagne reculera vers le nord, et l'autre moitié vers le midi.
\VS{5}Vous fuirez alors dans la vallée de mes montagnes ; car la vallée des montagnes s’étendra jusqu'à Atzel ; et vous fuirez comme vous avez fui devant le tremblement de terre, aux jours d’Ozias, roi de Juda. Alors Yahweh, mon Dieu, viendra, et tous les saints seront avec lui\FTNT{Ce passage confirme clairement que Jésus-Christ est Yahweh (1 Th. 3:13 ; Jud. 14-15 ; Es. 34:5 ; Es. 40:10-11 ; Es. 62:11-15).}.
\VS{6}Et il arrivera qu'en ce jour-là, la lumière précieuse ne sera pas mêlée de ténèbres.
\VS{7}Ce sera un jour unique, connu de Yahweh, et qui ne sera ni jour ni nuit ; mais au temps du soir il y aura de la lumière.
\VS{8}Et il arrivera qu'en ce jour-là, des eaux vives\FTNT{Ez. 47:1-12 ;  Ap. 22:1-2.} sortiront de Jérusalem, la moitié d'elles coulera vers la mer orientale, et l'autre moitié, vers la mer occidentale ; il en sera ainsi été et hiver.
\TextTitle{Le royaume messianique}
\VS{9}Yahweh sera Roi sur toute la terre ; en ce jour-là, Yahweh sera Un, et son nom sera Un\FTNT{Littéralement «~E’had~». Le jour du Seigneur est un comme le jour un de Ge. 1:5. Yahweh est Un et non trois (De. 6:4). Son Nom est Un (Ac. 4:12).}.
\VS{10}Toute la terre deviendra comme la plaine, depuis Guéba jusqu'à Rimmon, au midi de Jérusalem ; et Jérusalem sera exaltée et restera à sa place, depuis la porte de Benjamin, jusqu'à l'endroit de la première porte, jusqu'à la porte des angles, et depuis la tour de Hananeel, jusqu'aux pressoirs du roi.
\VS{11}On habitera dans son sein, et il n'y aura plus d'interdit, mais Jérusalem sera habitée en sûreté.
\VS{12}Voici la plaie dont Yahweh frappera tous les peuples qui auront fait la guerre contre Jérusalem ; il fera que la chair de chacun tombera en pourriture tandis qu’ils seront sur leurs pieds, leurs yeux tomberont en pourriture dans leurs orbites, et leur langue tombera en pourriture dans leur bouche.
\VS{13}Et il arrivera en ce jour-là que Yahweh produira un grand trouble parmi eux ; car chacun saisira la main de son prochain, et la main de l'un s'élèvera contre la main de l'autre.
\VS{14}Juda combattra aussi dans Jérusalem, et les richesses de toutes les nations d'alentour y seront amassées : L'or, l'argent, et des vêtements en très grand nombre.
\VS{15}Et la même plaie sera sur les chevaux, les mulets, les chameaux, les ânes et sur toutes les bêtes qui seront dans ces camps, cette plaie sera semblable à l’autre.
\TextTitle{Adoration de Yahweh des armées dans le royaume}
\VS{16}Et il arrivera que tous ceux qui resteront de toutes les nations venues contre Jérusalem, monteront en foule chaque année pour adorer le Roi, Yahweh des armées, et pour célébrer la fête des tabernacles.
\VS{17}S’il y a des familles de la terre qui ne montent pas à Jérusalem, pour adorer le Roi, Yahweh des armées, la pluie ne tombera pas sur elles.
\VS{18}Si la famille d'Egypte ne monte pas, si elle ne vient pas, la pluie ne tombera pas sur elle ; elle sera frappée de la plaie dont Yahweh frappera les nations qui ne monteront pas pour célébrer la fête des tabernacles.
\VS{19}Ce sera la peine du péché de l’Egypte, et du péché de toutes les nations qui ne monteront pas pour célébrer la fête des tabernacles.
\VS{20}En ce jour-là, il sera écrit sur les clochettes des chevaux : Sainteté à Yahweh ! Et les chaudières dans la maison de Yahweh seront comme les coupes devant l'autel.
\VS{21}Toute chaudière qui sera à Jérusalem et dans Juda, sera consacrée à Yahweh des armées ; et tous ceux qui offriront des sacrifices viendront, et s’en serviront pour cuire  les viandes ; et il n'y aura plus de marchands dans la maison de Yahweh des armées, en ce jour-là.
\PPE{}
\end{multicols}

%\clearpage\ShortTitle{Malachie}\BookTitle{Malachie}\BFont
\noindent\hrulefill
\textit{
\bigskip
{\centering{}
\\(Malakhi)
\\Signifie : Mon messager, mon ange
\\Thème : Message final de l'ancienne alliance à une nation désobéissante
\\Auteur : Malachie
\\Date de rédaction : Vème siècle av. J.-C.\\}
}
%\bigskip
\textit{
\\Dernier prophète de l’ancienne alliance, Malachie exerça son ministère en Juda après la reconstruction du temple et la reprise des cultes. Il annonça la venue du Messie et du messager qui devait le précéder, le nouvel Elie que Jésus-Christ reconnut en Jean-Baptiste. Ses écrits mettent en évidence l’importance de l’obéissance à la loi de Yahweh et la justice divine.
\bigskip
\\message. 
\bigskip
\\message.\bigskip
}
\par\nobreak\noindent\hrulefill
\begin{multicols}{2}
\TextTitle{[L'amour de Yahweh pour son peuple]}
\Chap{1}
\VerseOne{}Oracle, parole de Yahweh contre Israël, par le moyen de Malachie.
\VS{2}Je vous ai aimés, dit Yahweh ; et vous dites : En quoi nous as-tu aimés ? Esaü n'était-il pas frère de Jacob ? dit Yahweh. Or j'ai aimé Jacob,
\VS{3}Mais j'ai eu de la haine pour Esaü, et j'ai fait de ses montagnes une solitude, j’ai livré son héritage aux chacals du désert.
\VS{4}Si Edom dit : Nous sommes détruits, nous rebâtirons les lieux ruinés ! Ainsi parle Yahweh des armées : Ils rebâtiront, mais je détruirai, et on les appellera pays de méchanceté, peuple contre lequel Yahweh est irrité pour toujours.
\VS{5}Vos yeux le verront, et vous direz : Yahweh est grand par-delà les frontières d'Israël !
\TextTitle{[Le péché des sacrificateurs après le retour d'exil]}
\VS{6}Un fils honore son père, et un serviteur son maître. Si donc je suis Père, où est l'honneur qui m'appartient ? Si je suis maître, où est la crainte qu'on a de moi ? dit Yahweh des armées, à vous sacrificateurs, qui méprisez mon Nom, et qui dites : En quoi avons-nous méprisé ton Nom ?
\VS{7}Vous offrez sur mon autel des aliments souillés, et vous dites : En quoi t'avons-nous profané ? C'est en disant : La table de Yahweh est méprisable !
\VS{8}Et quand vous amenez une bête aveugle pour la sacrifier, n'y a-t-il point de mal en cela ? Quand vous en offrez une boiteuse ou malade, n’est-ce pas mal ? Offre-la à ton gouverneur ! T’agréera-t-il, te recevra-t-il favorablement ? dit Yahweh des armées.
\VS{9}Maintenant, donc suppliez Dieu, pour qu'il ait pitié de nous ! Cela vient de vos mains : Vous recevra-t-il favorablement ? dit Yahweh des armées.
\VS{10}Lequel de vous fermera les portes pour que vous n’allumiez pas en vain le feu sur mon autel ? Je ne prends aucun plaisir en vous, dit Yahweh des armées, et je n’agrée pas l'offrande de vos mains.
\VS{11}Car depuis le soleil levant jusqu'au soleil couchant, mon Nom est grand parmi les nations, et en tous lieux on brûle de l’encens en l'honneur de mon Nom, et des offrandes pures ; car mon Nom est grand parmi les nations, dit Yahweh des armées.
\VS{12}Mais vous, vous le profanez, en disant : La table de Yahweh est souillée, et ce qu’elle rapporte est un aliment méprisable.
\VS{13}Vous dites aussi : Quelle fatigue ! Et vous le dédaignez, dit Yahweh des armées ; vous amenez ce qui a été dérobé, ce qui est boiteux, et malade, ce sont là les offrandes que vous faites ! Accepterai-je cela de vos mains ? dit Yahweh.
\VS{14}C'est pourquoi, maudit soit l'homme trompeur, qui a dans son troupeau un mâle, et qui voue et sacrifie à Yahweh ce qui est corrompu ! Car je suis un grand roi, dit Yahweh des armées, et mon Nom est redoutable parmi les nations.
\TextTitle{[Mise en garde de Yahweh aux sacrificateurs]}
\Chap{2}
\VerseOne{}Maintenant, c’est à vous, sacrificateurs que s'adresse ce commandement :
\VS{2}Si vous n'écoutez pas, et que vous ne preniez point pas à cœur de donner gloire à mon Nom, dit Yahweh des armées, j'enverrai sur vous la malédiction, et je maudirai vos bénédictions ; et déjà même je les ai maudites, parce que vous ne prenez pas cela à cœur.
\VS{3}Voici, je vais détruire vos semences, et je répandrai les excréments de vos victimes sur vos visages, les excréments, dis-je, de vos solennités, et on vous emportera avec eux.
\VS{4}Alors vous saurez que je vous ai adressé ce commandement, afin que mon alliance avec Lévi subsiste, dit Yahweh des armées.
\VS{5}Mon alliance avec lui était la vie et la paix, c’est ce que je lui accordai pour qu’il me craigne ; il a eu pour moi de la crainte, et il a tremblé devant mon Nom.
\VS{6}La loi de la vérité était dans sa bouche, et il ne s'est point trouvé de perversité sur ses lèvres ; il a marché avec moi dans la paix et dans la droiture, et il en a détourné beaucoup de l'iniquité.
\VS{7}Car les lèvres du sacrificateur doivent garder la science, et c’est de sa bouche qu’on demande la loi, parce qu'il est un messager de Yahweh des armées.
\VS{8}Mais vous, vous vous êtes écartés de la voie, vous avez fait de la loi une occasion de chute pour beaucoup, et vous avez corrompu l'alliance de Lévi, dit Yahweh des armées.
\TextTitle{[Infidélités envers des frères et envers Yahweh]}
\VS{9}C'est pourquoi je vous rendrai méprisables et abjects aux yeux de tout le peuple. Parce que vous n’avez pas gardé mes voies, et vous avez égard à l'apparence des personnes quand vous enseignez la loi.
\VS{10}N'avons-nous pas tous un seul Père ? N’est-ce pas un seul Dieu qui nous a créés ? Pourquoi donc agissons-nous avec perfidie l’un avec l’autre, en violant l'alliance de nos pères ?
\VS{11}Juda s’est montré infidèle, et une abomination a été commise en Israël et à Jérusalem ; car Juda a profané ce qui est consacré à Yahweh, ce qu’il aime, il s'est marié à la fille d'un dieu étranger.
\VS{12}Yahweh retranchera l’homme qui fait cela, celui qui veille et qui répond, il le retranchera des tentes de Jacob, et il retranchera celui qui présente une offrande à Yahweh des armées.
\VS{13}Voici une autre chose que vous faites : Vous couvrez l'autel de Yahweh de larmes, de plaintes et de gémissements, en sorte qu’il n’a plus égard aux offrandes et qu’il ne peut rien agréer de vos mains.
\VS{14}Et vous dites : Pourquoi ?... C'est parce que Yahweh est intervenu comme témoin entre toi et la femme de ta jeunesse, envers laquelle tu as été infidèle, bien qu’elle soit ta compagne et la femme de ton alliance.
\VS{15}Nul n’a fait cela, avec un reste de bon esprit. Un seul l’a fait, et pourquoi ? Parce qu'il cherchait une postérité de Dieu. Prenez donc garde en votre esprit, et qu’aucun ne soit infidèle à la femme de sa jeunesse !
\VS{16}Car je hais la répudiation, dit Yahweh, le Dieu d'Israël, et celui qui couvre de violence son vêtement, dit Yahweh des armées. Prenez donc garde en votre esprit, et ne soyez pas infidèles !
\TextTitle{[Fausse profession religieuse]}
\VS{17}Vous fatiguez Yahweh par vos paroles, et vous dites : En quoi l'avons-nous fatigué ? C'est quand vous dites : Quiconque fait le mal plaît à Yahweh, et il prend plaisir à de tels gens ! Autrement : Où est le Dieu du jugement ?
\TextTitle{[Venue du précurseur du messie]}
\Chap{3}
\VerseOne{}Voici, j'enverrai mon messager\FTNT{Ce messager, ou Elie le prophète, est Jean-Baptiste (Es. 40:1-3 ; Mal. 3:1 ; Mt. 3:1-15 ; Mt. 11:14 ; Mt. 17:10-13 ; Mc. 1:1-11 ; Mc. 9:11-13 ; Lu. 1:17 ; Lu. 3:1-5).} ; il préparera le chemin devant moi. Et soudain entrera dans son temple le Seigneur que vous cherchez ; l’ange de l'alliance\FTNT{L’ange de l’alliance est le Seigneur Jésus-Christ. Voir aussi commentaire en Mt. 1:20}, que vous désirez, voici, il vient, dit Yahweh des armées.
\VS{2}Mais qui pourra soutenir le jour de sa venue ? Qui pourra subsister quand il paraîtra ? Car il sera comme le feu du fondeur et comme la potasse des foulons.
\VS{3}Et il sera assis comme celui qui raffine et purifie l'argent ; il nettoiera les fils de Lévi, il les épurera comme l’or et l'argent, et ils présenteront à Yahweh des offrandes avec justice.
\VS{4}Alors l’offrande de Juda et de Jérusalem sera agréable à Yahweh, comme aux anciens jours, comme aux années d'autrefois.
\VS{5}Je m'approcherai de vous pour le jugement, et je me hâterai de témoigner contre les enchanteurs et les adultères, contre ceux qui jurent faussement, et contre ceux qui retiennent le salaire du mercenaire, qui oppriment la veuve et l'orphelin, qui font tort à l'étranger, et qui ne me craignent point, dit Yahweh des armées.
\VS{6}Parce que je suis Yahweh et que je n'ai point changé ; à cause de cela, enfants de Jacob, vous n'avez point été consumés.
\TextTitle{[Le peuple infidèle qui vole Yahweh]}
\VS{7}Depuis le temps de vos pères, vous vous êtes écartés de mes ordonnances, vous ne les avez point observées. Revenez à moi, et je reviendrai à vous, dit Yahweh des armées. Et vous dites : En quoi nous convertirons-nous ?
\VS{8}L'homme pillera-t-il Dieu, que vous me pilliez ? Et vous dites : En quoi t'avons-nous pillé ? Vous l'avez fait dans les dîmes et dans les offrandes.
\VS{9}Vous êtes certainement maudits, parce que vous me pillez, vous, toute la nation !
\VS{10}Apportez toutes les dîmes\FTNT{Il est question ici de la dîme de la dîme que les levites donnaient aux sacrificateurs. Cette dîme était rapportée aux magasins, ou greniers (Né. 10:35-39), là aussi était stocké toute sorte de trésor. Pour les autres dîmes, voir le commentaire dans Dt. 14:22-29.} aux magasins, afin qu'il y ait provision dans ma maison ; et dès maintenant éprouvez moi en cela, a dit Yahweh des armées, si je ne vous ouvre pas les écluses des cieux, et si je ne répands pas en votre faveur la bénédiction, jusqu'à ce qu'il n'y ait plus assez de place.
\VS{11}Et je réprimerai pour l'amour de vous le dévorateur, et il ne vous ravagera pas les fruits de la terre, et vos vignes ne seront pas stériles dans vos campagnes, a dit Yahweh des armées.
\VS{12}Toutes les nations vous diront heureux, car vous serez un pays de délices, dit Yahweh des armées.
\VS{13}Vos paroles sont rudes contre moi, a dit Yahweh. Et vous dites : Qu'avons-nous donc dit contre toi ?
\VS{14}Vous avez dit : C'est en vain que l’on sert Dieu ; et qu'avons-nous gagné à observer ses ordonnances, et à marcher en pauvre état pour l'amour de Yahweh des armées ?
\VS{15}Et maintenant nous tenons pour heureux les orgueilleux ; et même ceux qui commettent la mechanceté, sont avancés ; et s'ils ont tenté Dieu, ils ont été délivrés !
\TextTitle{[Le "reste d'Israël" demeure fidèle à Yahweh"]}
\VS{16}Alors ceux qui craignent Yahweh se parlèrent l'un à l’autre ; et Yahweh fut attentif, et il écouta ; et un livre de souvenir fut écrit devant lui pour ceux qui craignent Yahweh et qui pensent à son Nom.
\VS{17}Ils seront à moi, a dit Yahweh des armées, le jour où je mettrai à part mes plus précieux joyaux, et je leur pardonnerai comme un homme pardonne à son fils qui le sert.
\VS{18}Convertissez-vous donc, et vous verrez la différence qu'il y a entre le juste et le méchant, entre celui qui sert Dieu et celui qui ne le sert pas.
\TextTitle{[Avènement du jour de Yahweh]}
\Chap{4}
\VerseOne{}Car voici, le jour vient, ardent comme une fournaise. Tous les orgueilleux et tous les méchants seront comme du chaume ; et ce jour qui vient, dit Yahweh des armées, les embrasera, il ne leur laissera ni racine ni rameau.
\VS{2}Mais pour vous qui craignez mon Nom, se lèvera le Soleil de justice\FTNT{Le Soleil de justice : Jésus-Christ est notre Soleil (Lu. 1:78-79). Cet aspect de Jésus-Christ nous parle de la grâce de Dieu : « il fait lever son soleil sur les méchants, et sur les gens de bien » (Mt. 5:45). Le soleil évoque aussi le jugement de Dieu. Ainsi, en plein midi, il est le feu de la justice et de la colère de Dieu. Son pardon et son amour pour nous sont alors comparés à une ombre fraîche qui nous sauve de sa chaleur ardente (Ps. 121 ; Es. 25:4). Dans Ps. 19:6, le soleil est comparé à un époux. Or Jésus-Christ est notre époux et le soleil qui nous apporte la guérison.}, et la guérison sera sous ses ailes ; vous sortirez, et bondirez comme les veaux d’une étable.
\VS{3}Et vous foulerez les méchants, car ils seront comme de la cendre sous les plantes de vos pieds, au jour où je ferai mon œuvre, dit Yahweh des armées.
\VS{4}Souvenez-vous de la loi de Moïse, mon serviteur, auquel j’ai prescrit en Horeb, pour tout Israël, des statuts et des ordonnances.
\TextTitle{[Retour d'Elie avant le jour de Yahweh]}
\VS{5}Voici, je vous enverrai Elie, le prophète\FTNT{Voir commentaire en Mal. 3:1.}, avant que le jour grand et redoutable de Yahweh vienne.
\VS{6}Il ramènera le cœur des pères à leurs enfants, et le cœur des enfants à leurs pères, de peur que je ne vienne et que je ne frappe la terre d’interdit.
\PPE{}
\end{multicols}

%\addcontentsline{toc}{section}{Ketouvim (Écrits)}\clearpage
%\clearpage\ShortTitle{Psaumes}\BookTitle{Psaumes}\BFont
\noindent\hrulefill
{\footnotesize
\textit{
\bigskip
{\centering{}
\\Auteurs : David essentiellement et d'autres écrivains
\\(Heb. : Tehilim)
\\Signification : Louanges 
\\Thème : La louange et l'adoration
\\Date de rédaction : A compter du 10\up{ème} siècle et au-delà\\}
}
%\bigskip
\textit{
\\Le terme « psaume » désigne un poème chanté avec l'accompagnement d'un instrument. C'est ainsi que furent initialement contés les récits de la création divine, la captivité ou encore la gloire de Jérusalem. Expressions de joie, de reconnaissance, de repentance, d'angoisse ou de vulnérabilité de l'homme, ces hymnes étaient des prières adressées à Dieu.
%\bigskip
\\Prophétiques, certains psaumes annoncent les événements de la fin des temps, notamment les souffrances de Christ. Utilisé comme recueil de chants, le livre des Psaumes exalte la grandeur de Dieu, sa souveraineté, sa miséricorde et son omniscience. Il est le fruit d'une grande variété d'expériences spirituelles du fait de la diversité de ses auteurs. De plus, il contient une richesse de styles considérable, ce qui en fait le chef-d'œuvre de la poésie hébraïque.\bigskip
}
}
\par\nobreak\noindent\hrulefill
\begin{multicols}{2}
\Chap{1}
\TextTitle{La voie du juste et du pécheur}
\VerseOne{}Heureux l'homme qui ne marche pas selon le conseil des méchants, et qui ne s'arrête pas sur la voie des pécheurs, qui ne s'assied pas dans l'assemblée des moqueurs\FTNT{Jé. 15:17 ; 1 Co. 15 : 33 ; Ep. 5:11.},
\VS{2}mais qui prend plaisir dans la loi de Yahweh, et qui médite sa loi jour et nuit\FTNT{De. 6:6 ; De. 17:19 ; Jos. 1:8.}.
\VS{3}Il est comme un arbre planté près des ruisseaux d'eaux, qui rend son fruit en sa saison, et dont le feuillage ne se flétrit point\FTNT{Jé. 17:7-8 ; Ez. 47:12 ; Jn. 15:8 ; Ap. 22:2.}. Et ainsi tout ce qu'il fera réussira.
\VS{4}Il n'en est pas ainsi des méchants : Ils sont comme la balle que le vent chasse au loin\FTNT{Job. 21:17-18 ; Os. 13:3.}.
\VS{5}C'est pourquoi les méchants ne résistent pas dans le jugement, ni les pécheurs dans l'assemblée des justes.
\VS{6}Car Yahweh connaît la voie des justes, mais la voie des méchants périra.
\Chap{2}
\TextTitle{Complot des nations contre le Messie}
\VerseOne{}Pourquoi cette agitation parmi les nations, et pourquoi les peuples projettent-ils des choses vaines ?
\VS{2}Pourquoi les rois de la terre se lèvent-ils en personne, et les princes se liguent-ils avec eux contre Yahweh, et contre son Messie\FTNT{Cette prophétie concerne le complot des Juifs, de Pilate et d'Hérode contre Jésus-Christ, notre Seigneur. Il est également question du gouvernement mondial dirigé par Satan. Mt. 12:14 ; Mt. 26:3-4 ; Mt. 26:59-66. ; Mt. 27:1-2 ; Mc. 3:6 ; Mc. 11:18 ; Ac. 4:23-29.}?
\VS{3}Rompons leurs liens et jetons loin de nous leurs cordes!
\VS{4}Celui qui habite dans les cieux se rit d'eux, le Seigneur se moque d'eux.
\VS{5}Il leur parle dans sa colère, et il les remplira de terreur par la grandeur de son courroux\FTNT{Pr. 1:26.}:
\VS{6}C'est moi qui ai consacré mon Roi sur Sion, la montagne de ma sainteté\FTNT{Mi. 4:7.}!
\VS{7}Je vous réciterai cette ordonnance ; Yahweh m'a dit : Tu es mon Fils ! Je t'ai engendré aujourd'hui\FTNT{Ac. 13:33 ; Hé. 1:5 ; Hé. 5:5.}.
\VS{8}Demande-moi, et je te donnerai les nations pour héritage, et les extrémités de la terre pour possession.
\VS{9}Tu les briseras avec un sceptre de fer et tu les mettras en pièces comme un vase de potier\FTNT{Da. 2:44 ; Ap. 2:27.}.
\VS{10}Maintenant donc, rois, ayez de l'intelligence ! Juges de la terre, recevez instruction !
\VS{11}Servez Yahweh avec crainte, et réjouissez-vous avec tremblement\FTNT{Ps. 19:10.}.
\VS{12}Embrassez le Fils, de peur qu'il ne s'irrite et que vous ne périssiez dans cette conduite, quand sa colère s'embrasera promptement. Heureux sont tous ceux qui se confient en lui !
\Chap{3}
\TextTitle{Yahweh, le véritable secours}
\VerseOne{}Psaume de David au sujet de sa fuite devant Absalom, son fils.
\VS{2}Ô Yahweh, que mes adversaires sont nombreux ! Beaucoup de gens se lèvent contre moi!
\VS{3}Plusieurs disent à mon âme : Plus de salut pour lui auprès de Dieu! Sélah\FTNT{Le mot hébreu « Sélah » signifie « élever, exalter ». Il peut aussi traduire une pause dans le cantique ou le texte. C'est sûrement un terme technique musical montrant probablement une accentuation, une pause, une interruption.}.
\VS{4}Mais toi, ô Yahweh ! Tu es un bouclier autour de moi, tu es ma gloire, et tu relèves ma tête.
\VS{5}De ma voix je crie à Yahweh, et il me répond de sa sainte montagne. Sélah.
\VS{6}Je me couche, je m'endors, je me réveille, car Yahweh me soutient\FTNT{Lé. 26:6.}.
\VS{7}Je ne crains pas les myriades de peuples quand ils se rangent contre moi de toutes parts.
\VS{8}Lève-toi, Yahweh, mon Dieu ! Délivre-moi ! Car tu frappes à la joue tous mes ennemis, tu brises les dents des méchants.
\VS{9}La délivrance vient de Yahweh\FTNT{Es. 43:11 ; Jé. 3:23 ; Pr. 21:31 ; Ap. 7:10.} ! Que ta bénédiction soit sur ton peuple! Sélah.
\Chap{4}
\TextTitle{Yahweh, la joie et la paix du juste}
\VerseOne{}Psaume de David, donné au chef des chantres pour le chanter sur Neguinoth.
\VS{2}Ô Dieu de ma justice, puisque je crie, réponds-moi ! Quand j'étais à l'étroit, tu m'as mis au large ! Aie pitié de moi, et exauce ma prière\FTNT{Ps. 28:1-2.} !
\VS{3}Fils des hommes, jusqu'à quand ma gloire sera-t-elle diffamée ? Jusqu'à quand aimerez-vous la vanité et chercherez-vous le mensonge ? Sélah.
\VS{4}Or sachez que Yahweh s'est choisi un bien-aimé. Yahweh m'exauce quand je crie à lui\FTNT{1 Jn. 5:14.}.
\VS{5}Tremblez et ne péchez point ; parlez en vos cœurs sur votre couche et taisez-vous. Sélah.
\VS{6}Offrez des sacrifices de justice\FTNT{Ps. 51:19.} et confiez-vous en Yahweh.
\VS{7}Plusieurs disent : Qui nous fera voir le bonheur ? Lève sur nous la lumière de ta face, ô Yahweh !
\VS{8}Tu mets plus de joie dans mon cœur qu'ils n'en ont, quand abondent leur froment et leur vin.
\VS{9}Je me couche et je m'endors en paix, car toi seul, ô Yahweh ! Tu me fais reposer en sécurité\FTNT{Pr. 3:24.}.
\Chap{5}
\TextTitle{Recours à la protection de Yahweh}
\VerseOne{}Psaume de David, donné au chef des chantres, pour le chanter sur Nehiloth.
\VS{2}Yahweh, prête l'oreille à mes paroles ! Ecoute ma méditation !
\VS{3}Mon Roi et mon Dieu ! Sois attentif à la voix de mon cri ; car c'est à toi que j'adresse ma requête.
\VS{4}Yahweh, le matin tu entends ma voix, dès le matin je me tourne vers toi, et je veille.
\VS{5}Car tu n'es point un Dieu qui prenne plaisir au mal ; le méchant n'a point sa demeure auprès de toi.
\VS{6}Les orgueilleux ne subsistent pas devant tes yeux ; tu hais tous ceux qui commettent l'iniquité\FTNT{Ps. 1:5 ; Ha. 1:13.}.
\VS{7}Tu fais périr les menteurs ; Yahweh a en abomination l'homme sanguinaire et le trompeur.
\VS{8}Mais moi, comblé de tes bienfaits, j'entrerai dans ta maison, je me prosternerai dans le palais de ta sainteté avec les sentiments d'une crainte respectueuse.
\VS{9}Yahweh, conduis-moi dans ta justice, à cause de mes ennemis, aplanis ta voie sous mes pas\FTNT{Ps. 25:4-5 ; Ps. 27:11.}.
\VS{10}Car il n'y a rien de droit dans leur bouche; leur cœur est rempli de malice, leur gosier est un sépulcre ouvert, ils flattent de leur langue\FTNT{Ps. 10:7 ; Ps. 12:3 ; Ro. 3:13.}.
\VS{11}Ô Dieu ! Fais-leur leur procès, qu'ils échouent dans leurs entreprises ! Chasse-les au loin, à cause du grand nombre de leurs transgressions ! Car ils se sont rebellés contre toi.
\VS{12}Mais que tous ceux qui se confient en toi se réjouiront, qu'ils soient dans la joie perpétuellement, et que tu sois leur protecteur ; et que ceux qui aiment ton Nom s'égayent en toi !
\VS{13}Car tu bénis le juste, ô Yahweh ! Et tu l'entoures de ta bienveillance comme d'un bouclier.
\Chap{6}
\TextTitle{La miséricorde de Yahweh}
\VerseOne{}Psaume de David, donné au chef des chantres, pour le chanter pour le chanter en Neguinoth, sur Sheminith.
\VS{2}Yahweh, ne me punis pas dans ta colère et ne me châtie pas dans ta fureur\FTNT{Jé. 10:24.}.
\VS{3}Yahweh, aie pitié de moi ! Car je suis sans aucune force. Guéris-moi, ô Yahweh ! Car mes os sont épouvantés.
\VS{4}Même mon âme est fort troublée ; et toi, ô Yahweh ! Jusqu'à quand ?
\VS{5}Reviens, Yahweh ! Délivre mon âme. Sauve-moi, à cause de ta miséricorde.
\VS{6}Car celui qui meurt n'a plus ton souvenir ; qui te célébrera dans le scheol\FTNT{Es. 38 : 18 ; Ps. 88 : 11 ; Ps. 115 : 17.} ?
\VS{7}Je m'épuise à force de gémir; chaque nuit ma couche est baignée de mes larmes\FTNT{Job 7 : 3-4.}, mon lit est arrosé de mes pleurs.
\VS{8}J'ai le visage usé par le chagrin\FTNT{Ps. 31 : 10.}; il vieillit à cause de tous ceux qui m'oppriment.
\VS{9}Retirez-vous loin de moi, vous tous ouvriers d'iniquité\FTNT{Mt. 7:23 ; Mt. 25 : 41 ; Lu. 13 : 27.}! Car Yahweh a entendu la voix de mes pleurs.
\VS{10}Yahweh a entendu ma supplication, Yahweh a reçu ma prière.
\VS{11}Tous mes ennemis sont confondus, saisis d'épouvante; ils reculent soudain, honteux.
\Chap{7}
\TextTitle{La délivrance se trouve auprès de Yahweh}
\VerseOne{}Shiggaïon de David, chantée à Yahweh, au sujet de Cusch, le Benjamite.
\VS{2}Yahweh, mon Dieu ! Je cherche en toi mon refuge. Sauve-moi de tous mes persécuteurs et délivre-moi,
\VS{3}afin qu'ils ne me déchirent pas, comme un lion qui dévore sans qu'il n'y ait personne qui me secoure.
\VS{4}Yahweh, mon Dieu ! Si j'ai commis une telle action, s'il y a de l'iniquité dans mes mains,
\VS{5}si j'ai rendu le mal à celui qui était paisible envers moi, si j'ai dépouillé celui qui m'opprimait sans cause,
\VS{6}que l'ennemi me poursuive et m'atteigne, qu'il foule à terre ma vie, et qu'il couche ma gloire dans la poussière ! Sélah.
\VS{7}Lève-toi, ô Yahweh ! Dans ta colère, lève-toi contre la fureur de mes adversaires. Réveille-toi pour me secourir, ordonne un jugement !
\VS{8}Que l'assemblée des peuples t'environne ! Monte au-dessus d'elle vers les lieux élevés !
\VS{9}Yahweh juge les peuples : Rends-moi justice, ô Yahweh\FTNT{Ps. 9 : 5.} ! Selon ma droiture et selon mon intégrité !
\VS{10}Que la malice des méchants prenne fin, et affermis le juste, toi qui sondes les cœurs et les reins\FTNT{Jé. 11:20 ; Jé. 17:10.}, ô Dieu juste !
\VS{11}Mon bouclier est en Dieu, qui délivre ceux qui sont droits de cœur.
\VS{12}Dieu est un juste juge, Dieu s'irrite en tout temps.
\VS{13}Si le méchant ne se convertit pas, Dieu aiguise son épée\FTNT{De. 32 : 41.}, il bande son arc, et vise.
\VS{14}Il dirige sur lui des traits meurtriers, il rend ses flèches brûlantes.
\VS{15}Voici, le méchant prépare le mal, il conçoit l'iniquité, et il enfante le mensonge\FTNT{Ja. 1:15.}.
\VS{16}Il fait une fosse, il la creuse, et il tombe dans la fosse qu'il a faite\FTNT{Ps. 9 : 16.}.
\VS{17}Son travail retourne sur sa tête, et sa violence redescend sur son front.
\VS{18}Je célébrerai Yahweh à cause de sa justice, je psalmodierai le Nom de Yahweh, du Très-Haut.
\Chap{8}
\TextTitle{Magnificence de Dieu et vanité de l'homme}
\VerseOne{}Psaume de David, donné au chef des chantres, pour le chanter sur guitthith.
\VS{2}Yahweh, notre Seigneur ! Que ton Nom est magnifique sur toute la terre ! Ta majesté s'élève au-dessus des cieux\FTNT{Es. 6 : 3.}.
\VS{3}Par la bouche des petits enfants et de ceux qui tètent\FTNT{Mt. 21 : 16.}, tu as fondé ta puissance, à cause de tes adversaires, afin de faire cesser l'ennemi et le vindicatif.
\VS{4}Quand je regarde tes cieux, l'ouvrage de tes doigts, la lune et les étoiles que tu as fixées:
\VS{5}Qu'est-ce que l'homme, pour que tu te souviennes de lui ? Et le fils de l'homme, pour que tu le visites\FTNT{Dans ce passage, il est question de l'incarnation de Yahweh afin de nous sauver (1 Co. 15:45-49 ; 1 Ti. 3:16 ; Hé. 2:14). Jésus-Christ s'est lui-même nommé « fils de l'homme » (Lu. 9:22-26), littéralement « fils d'Adam », d'ailleurs cette expression apparaît dans les évangiles plus de quatre-vingt fois.} ?
\VS{6}Tu l'as fait de peu inférieur aux anges, et tu l'as couronné de gloire et d'honneur.
\VS{7}Tu lui as donné la domination sur les œuvres de tes mains, tu as tout mis sous ses pieds\FTNT{1 Co. 15:27.},
\VS{8}les brebis comme les bœufs, les animaux des champs,
\VS{9}les oiseaux du ciel et les poissons de la mer, tout ce qui parcourt les sentiers des mers.
\VS{10}Yahweh, notre Seigneur ! Que ton Nom est magnifique sur toute la terre !
\Chap{9}
\TextTitle{Louange à Yahweh, l'auteur de nos victoires}
\VerseOne{}Psaume de David, donné au chef des chantres, pour le chanter sur Muth-Labben.
\VS{2}Je célébrerai de tout mon cœur Yahweh, je raconterai toutes tes merveilles.
\VS{3}Je me réjouirai et je m'égaierai en toi, je chanterai ton Nom, ô Très-Haut !
\VS{4}Mes ennemis reculent, ils trébuchent, ils périssent devant ta face.
\VS{5}Car tu soutiens mon droit et ma cause, tu sièges sur ton trône en juste juge.
\VS{6}Tu châties les nations, tu détruis le méchant, tu effaces leur nom pour toujours, et à perpétuité.
\VS{7}Plus d'ennemis ! Les désolations ont-elles pris fin ? As-tu aussi rasé les villes pour toujours ? Leur mémoire est perdue avec elles.
\VS{8}Mais Yahweh sera assis éternellement, il a établi son trône pour juger.
\VS{9}Il juge le monde avec justice, il juge les peuples avec droiture\FTNT{Ps. 96:13 ; Ps. 98:9.}.
\VS{10}Yahweh est un refuge pour l'opprimé, un refuge au temps de la détresse\FTNT{Ps. 37:39 ; Ps. 46:2 ; Ps. 91:2.}.
\VS{11}Ceux qui connaissent ton Nom se confient en toi\FTNT{Pr. 3:5.}. Car tu n'abandonnes point ceux qui te cherchent, ô Yahweh !
\VS{12}Chantez à Yahweh qui habite en Sion, annoncez ses exploits parmi les peuples !
\VS{13}Lorsqu'il recherche le sang versé, il se souvient des malheureux? il n'oublie pas le cris des affligés.
\VS{14}Aie pitié de moi, Yahweh ! Vois la misère où me réduisent mes ennemis, enlève-moi des portes de la mort,
\VS{15}afin que je raconte toutes tes louanges, dans les portes de la fille de Sion. Je me réjouirai de la délivrance\FTNT{Voir commentaire en Es. 26:1.} que tu m'auras donnée.
\VS{16}Les nations tombent dans la fosse qu'elles ont faite\FTNT{Ps. 10:2 ; Ps. 35:7.}, leur pied se prend au filet qu'elles ont caché.
\VS{17}Yahweh se fait connaître, il fait justice, le méchant est enlacé dans l'ouvrage de ses mains. Jeu d'instruments. Sélah.
\VS{18}Les méchants retournent dans le scheol, toutes les nations qui oublient Dieu.
\VS{19}Car le pauvre n'est point oublié à jamais, l'espérance des affligés ne périt pas à toujours.
\VS{20}Lève-toi, ô Yahweh ! Que l'homme mortel ne triomphe point ! Que les nations soient jugées devant ta face !
\VS{21}Frappe-les de terreur, ô Yahweh ! Que les peuples sachent qu'ils ne sont que des hommes mortels\FTNT{Es. 51:12.}! Sélah.
\Chap{10}
\TextTitle{Appel au jugement de Dieu sur les méchants}
\VerseOne{}Pourquoi, ô Yahweh ! Te tiens-tu éloigné ? Pourquoi te caches-tu au temps où nous sommes dans la détresse\FTNT{Ps. 13:2 ; Ps. 44:24.} ?
\VS{2}Le méchant par son orgueil poursuit ardemment les affligés, mais ils seront pris par les machinations qu'ils ont préméditées\FTNT{Ps. 7:15-16 ; Ps. 9:16 ; Ps. 35:8.}.
\VS{3}Car le méchant se glorifie du désir de son âme, il bénit l'avare et il méprise Yahweh.
\VS{4}Le méchant dit avec arrogance : Il ne fera pas d'enquête ! Il n'y a point de Dieu\FTNT{Ps. 14:1 ; Ps. 53:2.} ! Voilà toutes ses pensées.
\VS{5}Ses voies réussissent en tout temps; tes jugements sont éloignés de lui, il souffle contre tous ses adversaires.
\VS{6}Il dit en son cœur : Je ne chancelle pas, je suis pour toujours à l'abri du malheur !
\VS{7}Sa bouche est pleine de malédictions, de tromperies et de fraudes ; il n'y a sous sa langue qu'oppression et outrage\FTNT{Ps. 59:7-8 ; Ps. 64:3-4 ; Job. 20:12.}.
\VS{8}Il se tient aux embûches dans des villages, il tue l'innocent dans des lieux cachés, ses yeux épient le malheureux.
\VS{9}Il se tient aux aguets dans un lieu caché, comme un lion dans sa tanière, il se tient aux aguets pour attraper l'affligé ; il attrape l'affligé, l'attirant dans son filet.
\VS{10}Il se courbe, il se baisse, et les malheureux tombent dans ses griffes.
\VS{11}Il dit en son cœur : Dieu oublie ! Il cache sa face, il ne le verra jamais\FTNT{Ps 94:7.}!
\VS{12}Lève-toi, ô Yahweh ! Lève ta main ! N'oublie pas les malheureux !
\VS{13}Pourquoi le méchant méprise-t-il Dieu ? Il dit en son cœur que tu ne punis pas.
\VS{14}Tu regardes cependant, car tu vois la peine et la souffrance, pour prendre en main leur cause ; c'est toi qui viens en aide à l'orphelin.
\VS{15}Brise le bras du méchant, punis ses iniquités et qu'il disparaisse à tes yeux !
\VS{16}Yahweh est Roi à toujours et à perpétuité\FTNT{Ps. 29:10 ; Ps. 145:13 ; Ps. 146:10 ; La. 5:19.} ; les nations sont exterminées de sa terre.
\VS{17}Tu entends les vœux de ceux qui souffrent, ô Yahweh ! Tu affermis leur cœur; tu prêtes l'oreille
\VS{18}pour rendre justice à l'orphelin et à l'opprimé, afin que l'homme mortel tiré de la terre cesse d'inspirer l'effroi.
\Chap{11}
\TextTitle{Yahweh, le refuge des hommes droits}
\VerseOne{}Psaume de David, donné au chef des chantres. C'est en Yahweh que je cherche un refuge. Comment un homme peut-il dire à mon âme : Fuis dans vos montagnes, comme un oiseau ?
\VS{2}En effet, les méchants bandent l'arc\FTNT{Ps. 37:14.}, ils ajustent leur flèche sur la corde, pour tirer dans l'ombre sur ceux dont le cœur est droit.
\VS{3}Quand les fondements sont renversés, que fera le juste ?
\VS{4}Yahweh est dans son saint temple, Yahweh a son trône dans les cieux ; ses yeux contemplent, ses paupières sondent les fils des hommes.
\VS{5}Yahweh sonde le juste et le méchant ; et son âme hait celui qui aime la violence.
\VS{6}Il fait pleuvoir sur les méchants des charbons, du feu et du soufre\FTNT{Ez. 38:22.} ; un vent brûlant, c'est le calice qu'ils ont en partage.
\VS{7}Car Yahweh est juste, il aime la justice ; les hommes droits contemplent sa face.
\Chap{12}
\TextTitle{Le langage des lèvres arrogantes}
\VerseOne{}Psaume de David, donné au chef des chantres pour le chanter sur Sheminith.
\VS{2}Sauve, ô Yahweh ! Car les hommes pieux s'en vont, les fidèles disparaissent parmi les fils de l'homme.
\VS{3}Chacun dit des faussetés à son compagnon avec des lèvres flatteuses et ils parlent avec un cœur double.
\VS{4}Que Yahweh retranche toutes les lèvres flatteuses, la langue qui parle fièrement\FTNT{Ps. 17:10.},
\VS{5}parce qu'ils disent : Nous sommes puissants par nos langues, nous avons nos lèvres avec nous ; qui serait notre maître ?
\VS{6}A cause du mauvais traitement que l'on fait aux malheureux, à cause du gémissement des pauvres, je me lèverai maintenant, dit Yahweh, je mettrai en sûreté celui à qui l'on tend des pièges.
\VS{7}Les paroles de Yahweh sont des paroles pures, c'est un argent éprouvé sur terre au creuset\FTNT{Ps. 19:10 ; Ps. 119:140 ; Pr. 30:5.}, et sept fois épuré.
\VS{8}Toi, Yahweh ! Garde-les, préserve cette race à jamais.
\VS{9}Les méchants se promènent de toutes parts, tandis que des gens abjects sont élevés parmi les fils des hommes.
\Chap{13}
\TextTitle{Savoir attendre le secours de Dieu}
\VerseOne{}Psaume de David, donné au chef des chantres.
\VS{2}Yahweh, jusqu'à quand m'oublieras-tu ? Sera-ce pour toujours ? Jusqu'à quand me cacheras-tu ta face\FTNT{Ps. 10:1 ; Ps. 27:9.} ?
\VS{3}Jusqu'à quand consulterai-je mon âme, affligerai-je mon cœur tous les jours ? Jusqu'à quand mon ennemi s'élèvera-t-il contre moi ?
\VS{4}Yahweh, mon Dieu ! Regarde, exauce-moi, illumine mes yeux, de peur que je ne dorme du sommeil de la mort,
\VS{5}de peur que mon ennemi ne dise : J'ai eu le dessus! Que mes adversaires ne se réjouissent, si je venais à tomber\FTNT{Ps. 25:2.}.
\VS{6}Mais moi, je me confie en ta bonté, mon cœur se réjouira de la délivrance que tu m'auras donnée ; je chanterai à Yahweh, parce qu'il m'a fait du bien.
\Chap{14}
\TextTitle{L'insensé ne cherche pas Dieu}
\VerseOne{}Psaume de David, donné au chef des chantres. L'insensé dit en son cœur : Il n'y a point de Dieu\FTNT{Ceux qui ne croient pas en l'existence de Dieu sont appelés insensés. En effet, la création révèle l'existence du Créateur (Ro. 1:19-20).} ! Ils se sont corrompus, ils ont commis des actions abominables ; il n'y a personne qui fasse le bien.
\VS{2}Yahweh regarde des cieux les fils de l'homme, pour voir s'il y a quelqu'un qui soit intelligent, qui cherche Dieu\FTNT{Ps. 33:13 ; Job. 28:24.}.
\VS{3}Ils se sont tous égarés, ils se sont tous ensemble rendus odieux, il n'y a personne qui fasse le bien, pas même un seul\FTNT{Tous les hommes naissent pécheurs (Ro. 3:10-23).}.
\VS{4}Tous ces ouvriers d'iniquité n'ont-ils point de connaissance ? Ils dévorent mon peuple, ils le prennent pour nourriture ; ils n'invoquent point Yahweh.
\VS{5}Là, ils seront saisis d'une grande frayeur, car Dieu est avec la race des justes.
\VS{6}Jetez l'opprobre sur l'espérance du malheureux…Yahweh est son refuge.
\VS{7}Oh ! Qui fera partir de Sion la délivrance d'Israël\FTNT{C'est le Messie qui délivrera Israël (Ro. 11:25-27).} ? Quand Yahweh ramenera son peuple captif, Jacob se réjouira, Israël se réjouira.
\Chap{15}
\TextTitle{L'homme que Yahweh agrée}
\VerseOne{}Psaume de David. Yahweh, qui séjournera dans ta tente ? Qui demeurera sur ta montagne sainte\FTNT{Ps. 24:3-4.} ?
\VS{2}Celui qui marche dans l'intégrité, qui fait ce qui est juste, et qui profère la vérité telle qu'elle est dans son cœur,
\VS{3}qui ne calomnie point avec sa langue, qui ne fait point de mal à son ami, qui ne diffame point son prochain.
\VS{4}Il regarde avec dédain celui qui est méprisable, mais il honore ceux qui craignent Yahweh; il ne se rétracte point s'il fait un serment à son préjudice.
\VS{5}Il n'exige point d'intérêt de son argent, et il n'accepte point de présent contre l'innocent\FTNT{Lé. 25:36 ; De. 16:19 ; De. 27:25.}. Celui qui fait ces choses ne sera jamais ébranlé.
\Chap{16}
\TextTitle{Yahweh, la source de la vie}
\VerseOne{}Mictam de David. Garde-moi, ô Dieu ! Car je cherche en toi mon refuge.
\VS{2}Je dis à Yahweh : Tu es mon Seigneur, tu es mon bonheur!
\VS{3}Les saints qui sont dans le pays, les hommes pieux sont l'objet de toute mon affection.
\VS{4}On multiplie les peines, on court après les dieux étrangers : Je ne répands pas leurs libations de sang et je ne mets pas leurs noms sur mes lèvres.
\VS{5}Yahweh est la part de mon héritage et ma coupe ; tu maintiens mon lot;
\VS{6}un héritage délicieux m'est échu, une belle possession m'est accordée.
\VS{7}Je bénirai Yahweh qui me donne conseille; je le bénirai même durant les nuits dans lesquelles mes reins m'enseignent.
\VS{8}J'ai constamment Yahweh sous mes yeux ; quand il est à ma droite, je ne chancelle pas\FTNT{Ps. 109:31 ; Ps. 110:5 ; Ac. 2:25.}.
\VS{9}C'est pourquoi mon cœur se réjouit, mon esprit se réjouit et mon corps repose en sécurité.
\VS{10}Car tu n'abandonneras point mon âme au scheol, tu ne permettras point que ton bien-aimé voie la corruption\FTNT{Le roi David prophétise ici la résurrection du Messie.}.
\VS{11}Tu me feras connaître le chemin de la vie ; il y a d'abondantes joies devant ta face, des délices éternels à ta droite.
\Chap{17}
\TextTitle{L'assurance en Dieu}
\VerseOne{}Prière de David. Yahweh, écoute la droiture, sois attentif à mon cri, prête l'oreille à ma prière faite avec des lèvres sans tromperie !
\VS{2}Que ma justice paraisse devant ta face, que tes yeux contemplent mon intégrité !
\VS{3}Tu as sondé mon cœur\FTNT{Ps. 139:1 ; Jé. 12:3.}, tu l'as visité de nuit, tu m'as examiné, tu n'as rien trouvé : Ma pensée ne va point au-delà de ma parole.
\VS{4}Quant aux actions des hommes, selon la parole de tes lèvres, je me tiens en garde contre la voie du violent.
\VS{5}Mes pas sont fermes dans tes sentiers, mes pieds ne chancellent point.
\VS{6}Je t'invoque, car tu m'exauces, ô Dieu ! Incline ton oreille vers moi, écoute mes paroles !
\VS{7}Signale ta bonté, toi qui sauves ceux qui cherchent un refuge, et qui par ta droite les délivres de leurs adversaires !
\VS{8}Garde-moi comme la prunelle de l'œil, cache-moi à l'ombre de tes ailes\FTNT{Mt. 23:37.},
\VS{9}contre les méchants qui me traitent violemment, mes ardents ennemis qui m'entourent.
\VS{10}Ils sont enfermés dans leur propre graisse, leur bouche parle avec orgueil.
\VS{11}Maintenant, ils nous environnent à chaque pas que nous faisons ; ils jettent leur regard pour nous étendre par terre.
\VS{12}Ils ressemblent au lion qui ne demande qu'à déchirer, et au lionceau qui se tient dans les lieux cachés.
\VS{13}Lève-toi, ô Yahweh, devance-les, renverse-les ! Délivre mon âme du méchant par ton épée.
\VS{14}Yahweh, délivre-moi par ta main de ces gens, des gens de ce monde ! Leur part est dans cette vie et tu remplis leur ventre de tes biens ; leurs enfants sont rassasiés et ils laissent leurs restes à leurs petits-enfants.
\VS{15}Mais moi, dans mon innocence, je verrai ta face\FTNT{Job. 19:26-27 ; Ps. 16:10-11.}, et je me rassasierai de ton image, dès mon réveil.
\Chap{18}
\TextTitle{Louange à Dieu, le bouclier des saints}
\VerseOne{}Psaume de David, serviteur de Yahweh, qui adressa à Yahweh les paroles de ce cantique le jour où Yahweh l'eut délivré de la main de Saül. Au chef des chantres.
\VS{2}Il dit donc : Je t'aime, ô Yahweh, ma force !
\VS{3}Yahweh est mon rocher\FTNT{Yahweh est le rocher sur lequel s'appuyait David. Paul enseigne que ce rocher était Jésus-Christ (1 Co. 10:1-4). Voir commentaire en Es. 8:13-17.}, ma forteresse et mon libérateur ! Mon Dieu, mon rocher où je trouve un refuge ! Mon bouclier, la force qui me sauve, ma haute retraite !
\VS{4}Je crie : Loué soit Yahweh ! Et je suis délivré de mes ennemis.
\VS{5}Les liens de la mort m'avaient environné et des torrents de destruction m'avaient épouvanté.
\VS{6}Les liens du scheol m'avaient entouré, les filets de la mort m'avaient surpris\FTNT{Ps. 116:3.}.
\VS{7}Dans ma détresse, j'ai invoqué Yahweh, j'ai crié à mon Dieu ; il a entendu ma voix de son palais, mon cri est parvenu devant lui à ses oreilles.
\VS{8}La terre fut ébranlée et trembla, les fondements des montagnes croulèrent\FTNT{Es. 5:25 ; Es. 64:1-3 ; Jé. 4:24 ; Ps. 104:32.}, et ils furent ébranlés, parce qu'il était irrité.
\VS{9}Une fumée montait de ses narines, et de sa bouche sortait un feu dévorant, des charbons embrasés.
\VS{10}Il abaissa les cieux et descendit : Il y avait une épaisse nuée sous ses pieds.
\VS{11}Il était monté sur un chérubin, et il volait, il était porté sur les ailes du vent\FTNT{Ps. 104:3.}.
\VS{12}Il faisait des ténèbres sa demeure secrète, autour de lui était sa tente, il était enveloppé des eaux obscures et de sombres nuages.
\VS{13}De la splendeur qui le précédait s'échappaient les nuées, lançant de la grêle et des charbons de feu.
\VS{14}Yahweh tonna dans les cieux, le Très-Haut fit retentir sa voix avec de la grêle et des charbons de feu.
\VS{15}Il tira ses flèches, et écarta mes ennemis, il lança des éclairs et les mit en déroute\FTNT{Ps. 77:18.}.
\VS{16}Le fond des eaux parut, les fondements du monde furent découverts, par ta menace, ô Yahweh ! Par le souffle du vent de tes narines.
\VS{17}Il étendit la main d'en haut, il m'enleva et me retira des grandes eaux\FTNT{2 S. 22:17.} ;
\VS{18}il me délivra de mon puissant ennemi, et de ceux qui me haïssaient, car ils étaient plus forts que moi.
\VS{19}Ils m'avaient surpris au jour de ma détresse, mais Yahweh, fut mon appui.
\VS{20}Il m'a mis au large, il m'a délivré, parce qu'il m'aime.
\VS{21}Yahweh m'a rendu selon ma justice, il m'a traité selon la pureté de mes mains\FTNT{Ps. 18:25 ; Ps. 7:9.},
\VS{22}car j'ai observé les voies de Yahweh et je n'ai point été coupable envers mon Dieu.
\VS{23}Car j'ai eu devant moi toutes ses ordonnances et je ne me suis point écarté de ses lois.
\VS{24}J'ai été intègre envers lui, et je me suis tenu en garde contre mon iniquité.
\VS{25}Aussi Yahweh m'a rendu selon ma justice, selon la pureté de mes mains devant ses yeux,
\VS{26}avec celui qui est bon, tu te montres bon, avec l'homme droit tu agis selon la droiture.
\VS{27}Avec celui qui est pur, tu te montres pur, et avec le pervers tu agis selon sa perversité.
\VS{28}Car tu sauves le peuple affligé et tu abaisses les yeux hautains\FTNT{Es. 2:11 ; Es. 5:15.}.
\VS{29}Tu fais briller ma lumière ; Yahweh, mon Dieu, éclaire mes ténèbres.
\VS{30}Avec toi, je me précipite sur un corps d'armée, avec mon Dieu je franchis la muraille.
\VS{31}Les voies de Dieu sont sans défaut ; la parole de Yahweh est éprouvée\FTNT{De. 32:4 ; Ps. 19:8-9 ; Da. 4:37.} ; il est un bouclier pour tous ceux qui se confient en lui.
\VS{32}Car qui est Dieu, si ce n'est Yahweh ? Et qui est un rocher, si ce n'est notre Dieu ?\FTNT{1 S. 2:2 ; 2 S. 22:32.}
\VS{33}C'est le Dieu qui me ceint de force, et qui me conduit dans la voie droite.
\VS{34}Il rend mes pieds semblables à ceux des biches\FTNT{2 S. 2:18.}, et il me place sur mes lieux élevés.
\VS{35}Il exerce mes mains au combat, tellement qu'un arc d'airain a été rompu avec mes bras.\FTNT{Job. 20:24.}.
\VS{36}Tu me donnes le bouclier de ton salut, ta droite me soutient, et je deviens puissant par ta bonté.
\VS{37}Tu élargis le chemin sous mes pas, et mes pieds ne chancellent point.
\VS{38}Je poursuis mes ennemis, je les atteints, et je ne reviens pas avant de les avoir anéantis.
\VS{39}Je les brise et ils ne peuvent se relever ; ils tombent sous mes pieds.
\VS{40}Car tu m'as ceint de force pour le combat, tu fais plier sous moi ceux qui s'élevaient contre moi.
\VS{41}Tu fais tourner le dos à mes ennemis devant moi, et j'extermine ceux qui me haïssaient.
\VS{42}Ils crient, mais il n'y a point de libérateur! Ils crient à Yahweh, mais il ne leur répond point!
\VS{43}Je les brise comme la poussière qui est dispersée par le vent et je les foule comme la boue des rues.
\VS{44}Tu me délivres des séditions du peuple, tu m'établis chef des nations. Un peuple que je ne connais point m'est asservi.
\VS{45}Ils m'obéissent au premier ordre, les fils de l'étranger me flattent.
\VS{46}Les étrangers s'enfuient et ils tremblent de peur dans leurs forteresses.
\VS{47}Yahweh est vivant, et béni soit mon rocher ! Que le Dieu de mon salut soit exalté !
\VS{48}Le Dieu qui est mon vengeur et qui m'assujettit les peuples,
\VS{49}c'est lui qui me délivre de mes ennemis ! Tu m'élèves au-dessus de mes adversaires, tu me sauves de l'homme violent.
\VS{50}C'est pourquoi, ô Yahweh, je te célébrerai parmi les nations ! Et je chanterai des psaumes à ton Nom.
\VS{51}Il accorde de grandes délivrances à son roi, et il fait miséricorde à son oint, à David, et à sa postérité, pour toujours.
\Chap{19}
\TextTitle{La création exalte la grandeur de Dieu}
\VerseOne{}Psaume de David, donné au chef des chantres.
\VS{2}Les cieux racontent la gloire de Dieu, et l'étendue met en évidence l'oeuvre de ses mains.
\VS{3}Un jour en instruit un autre jour, et une nuit fait connaître sa science à l'autre nuit.
\VS{4}Ce n'est pas un langage, ce ne sont pas des paroles dont le cri ne soit point entendu :
\VS{5}Leur retentissement couvre toute la terre, et leur voix est allée jusqu'aux extrémités du monde\FTNT{Ro. 10:18.}. Il a dressé une tente pour le soleil.
\VS{6}Et le soleil est semblable à un époux sortant de sa chambre ; il s'élance sur le sentier avec la joie d'un homme vaillant;
\VS{7}il se lève à l'extrémité des cieux et achève sa course à l'autre extrémité\FTNT{Ec. 1:5.}: Rien ne se dérobe à sa chaleur.
\VS{8}La loi de Yahweh est parfaite, elle restaure l'âme ; le témoignage de Yahweh est fidèle, il donne la sagesse au simple\FTNT{2 S. 22:31 ; Ps. 18:31 ; Ps. 119:130.}.
\VS{9}Les ordonnances de Yahweh sont droites, elles réjouissent le cœur ; les commandements de Yahweh sont purs, ils éclairent les yeux.
\VS{10}La crainte de Yahweh est pure, elle subsiste à toujours ; les jugements de Yahweh sont vrais, et ils sont tous justes.
\VS{11}Ils sont plus précieux que l'or, que beaucoup d'or fin ; et plus doux que le miel, que celui qui coule des rayons de miel\FTNT{Ps. 119:103.}.
\VS{12}Ton serviteur aussi en reçoit l'éclairage ; pour qui les observe la récompense est grande.
\VS{13}Qui connaît ses fautes commises par erreur ? Purifie-moi de mes fautes cachées.
\VS{14}Eloigne aussi ton serviteur des actions commises par fierté, en sorte qu'elles ne dominent point sur moi, qu'elles cessent et que je sois nettoyé de mes grands péchés!
\VS{15}Que les propos de ma bouche et la méditation de mon cœur te soient agréables, ô Yahweh ! Mon rocher et mon rédempteur\FTNT{Voir commentaire en Es. 60:16.}!
\Chap{20}
\TextTitle{Recours à l'intervention de Dieu}
\VerseOne{}Psaume de David, donné au chef des chantres.
\VS{2}Que Yahweh te réponde au jour de la détresse, que le Nom du Dieu de Jacob te protège !
\VS{3}Qu'il envoie ton secours du saint lieu, et qu'il te soutienne de Sion !
\VS{4}Qu'il se souvienne de toutes tes offrandes, qu'il réduise en cendres ton holocauste ! Sélah.
\VS{5}Qu'il te donne ce que ton cœur désire, et qu'il fasse réussir tes desseins !
\VS{6}Nous triompherons dans ton salut, nous lèverons la bannière au Nom de notre Dieu ; Yahweh exaucera tous tes vœux.
\VS{7}Je sais déjà que Yahweh sauve son oint ; il l'exaucera des cieux, de sa sainte demeure, par le secours puissant de sa droite.
\VS{8}Les uns se vantent de leurs chars, et les autres de leurs chevaux ; mais nous, nous glorifierons le Nom de Yahweh, notre Dieu.
\VS{9}Eux ils plient, et ils tombent ; nous, nous tenons ferme, et restons debout.
\VS{10}Yahweh, sauve le roi ! Qu'il nous réponde quand nous crions à lui !
\Chap{21}
\TextTitle{La protection de Dieu sur le roi}
\VerseOne{}Psaume de David, donné au chef des chantres.
\VS{2}Yahweh, le roi se réjouit de ta puissance, ton secours le remplit d'allégresse !
\VS{3}Tu lui as donné ce que désirait son cœur et tu n'as point refusé ce que demandaient ses lèvres. Sélah.
\VS{4}Car tu l'as prévenu par les bénédictions de ta bonté, et tu as mis sur sa tête une couronne d'or pur.
\VS{5}Il t'avait demandé la vie, et tu la lui as donnée, une vie longue pour toujours et à perpétuité.
\VS{6}Sa gloire est grande à cause de ton salut, tu l'as couvert de majesté et d'honneur.
\VS{7}Tu le rends à jamais un objet de bénédictions, tu le combles de joie devant ta face\FTNT{Ps. 16:11.}.
\VS{8}Le roi se confie en Yahweh, et par la bonté du Très-Haut, il ne chancelle pas\FTNT{Ps. 16:8.}.
\VS{9}Ta main trouvera tous tes ennemis, ta droite trouvera tous ceux qui te haïssent.
\VS{10}Tu les rendras tels qu'une fournaise ardente le jour où l'on verra ta face ; Yahweh les engloutira dans sa colère, et le feu les consumera.
\VS{11}Tu feras périr leur fruit de la terre et leur race du milieu des fils des hommes.
\VS{12}Car ils ont projeté du mal contre toi et ils ont conçu de mauvais desseins dont ils ne pourront venir à bout.
\VS{13}Parce que tu leur feras tourner le dos, et avec ton arc tu tireras sur eux.
\VS{14}Elève-toi, Yahweh, par ta force ! Nous chanterons et célébrerons ta puissance.
\Chap{22}
\TextTitle{Les souffrances du Messie}
\VerseOne{}Psaume de David, donné au chef des chantres, pour le chanter sur Ajéleth-Hashakhar.
\VS{2}Mon Dieu ! Mon Dieu ! Pourquoi m'as-tu abandonné\FTNT{Le Ps. 22 est une description détaillée de la mort par crucifixion du Seigneur Jésus-Christ (Mt. 27:45-46).}, et t'éloignes-tu sans me secourir, sans écouter mes plaintes ?
\VS{3}Mon Dieu ! Je crie le jour, mais tu ne réponds point ; la nuit, et je n'ai point de repos.
\VS{4}Pourtant tu es le Saint, tu habites au milieu des louanges d'Israël.
\VS{5}Nos pères se sont confiés en toi ; ils se sont confiés, et tu les as délivrés.
\VS{6}Ils ont crié vers toi, et ils ont été délivrés ; ils se sont appuyés sur toi, et ils n'ont point été confus\FTNT{Es. 49:23 ; Ps. 25:3 ; Ps. 31:2.}.
\VS{7}Et moi, je suis un ver et non un homme, l'opprobre des hommes et le méprisé du peuple\FTNT{Es. 53:2-3.}.
\VS{8}Tous ceux qui me voient se moquent de moi, ils ouvrent les lèvres, secouent la tête\FTNT{Ps. 109:25 ; Mt. 27:39.} :
\VS{9}Recommande-toi à Yahweh ! Qu'il te délivre, et qu'il te sauve, puisqu'il prend plaisir en toi\FTNT{Mt. 27:43.} !
\VS{10}Cependant, c'est toi qui m'as tiré hors du ventre de ma mère, qui m'as mis en sûreté lorsque j'étais sur les mamelles de ma mère.
\VS{11}J'ai été sous ta garde, dès le sein maternel, tu as été mon Dieu dès le ventre de ma mère\FTNT{Es. 49:1.}.
\VS{12}Ne t'éloigne point de moi, car la détresse est près de moi, et il n'y a personne qui me secoure\FTNT{Ps. 69:21.} !
\VS{13}Plusieurs taureaux sont autour de moi, de puissants taureaux de Basan m'entourent.
\VS{14}Ils ouvrent leur gueule contre moi, comme un lion qui déchire et rugit.
\VS{15}Je suis comme de l'eau qui s'écoule, et tous mes os se séparent ; mon cœur est comme de la cire, il se fond dans mes entrailles.
\VS{16}Ma force se dessèche comme l'argile, et ma langue s'attache à mon palais ; tu me réduis à la poussière de la mort.
\VS{17}Car des chiens m'environnent, une assemblée de méchants m'entoure, ils ont percé mes mains et mes pieds.
\VS{18}Je pourrais compter tous mes os un par un. Eux, ils m'examinent, ils me regardent.
\VS{19}Ils se partagent mes vêtements et tirent au sort ma tunique\FTNT{Mt. 27:35 ; Mc. 15:24 ; Lu. 23:33.}.
\VS{20}Et toi, Yahweh, ne t'éloigne point ! Ma force, hâte-toi de me secourir !
\VS{21}Délivre ma vie de l'épée, ma vie contre le pouvoir des chiens !
\VS{22}Sauve-moi de la gueule du lion, délivre-moi des cornes du buffle !
\VS{23}Je déclarerai ton Nom à mes frères, je te louerai au milieu de l'assemblée\FTNT{Hé. 2:12.}.
\VS{24}Vous qui craignez Yahweh, louez-le ! Toute la race de Jacob, glorifiez-le ! Toute la race d'Israël, redoutez-le !
\VS{25}Car il n'a ni mépris ni dédain pour les peines du misérable, et il ne lui cache point sa face, mais il l'écoute quand il crie à lui.
\VS{26}Tu seras l'objet de mes louanges dans la grande assemblée ; j'accomplirai mes vœux en présence de ceux qui te craignent\FTNT{Ps. 56:13.}.
\VS{27}Les malheureux mangeront et seront rassasiés, ceux qui cherchent Yahweh le loueront. Votre cœur vivra à perpétuité !
\VS{28}Toutes les extrémités de la terre s'en souviendront, ils se convertiront à Yahweh, et toutes les familles des nations se prosterneront devant toi\FTNT{Ps. 72:8-11 ; Ps. 86:9.}.
\VS{29}Car le règne appartient à Yahweh : Il domine sur les nations.
\VS{30}Tous les gens de la terre mangeront et se prosterneront devant lui ; tous ceux qui descendent dans la poussière s'inclineront, même celui qui ne peut conserver sa vie.
\VS{31}La postérité le servira, on parlera du Seigneur de génération en génération\FTNT{Es. 59:21 ; Es. 65:23 ; Ps. 110:3.}.
\VS{32}Ils viendront et ils publieront sa justice au peuple qui naîtra, parce qu'il aura fait ces choses.
\Chap{23}
\TextTitle{Le bon Berger}
\VerseOne{}Psaume de David. Yahweh est mon berger\FTNT{Yahweh, le bon berger, est notre Seigneur Jésus-Christ. Es. 40:11 ; Jé. 23:4 ; Jn. 10:11.}: Je ne manquerai de rien.
\VS{2}Il me fait reposer dans de verts pâturages, il me dirige près des eaux paisibles.
\VS{3}Il restaure mon âme, et me conduit dans les sentiers de la justice, à cause de son Nom.
\VS{4}Quand je marche dans la vallée de l'ombre de la mort, je ne crains aucun mal\FTNT{Ps. 118:6.}, car tu es avec moi : Ton bâton et ta houlette me consolent.
\VS{5}Tu dresses devant moi une table, en face de mes adversaires; tu oins d'huile ma tête et ma coupe déborde.
\VS{6}Le bonheur et la grâce m'accompagneront tous les jours de ma vie, et j'habiterai dans la maison de Yahweh jusqu'à la fin de mes jours.
\Chap{24}
\TextTitle{Accueil de Yahweh, le Roi de gloire}
\VerseOne{}Psaume de David. La terre appartient à Yahweh, avec tout ce qui est en elle\FTNT{Ex. 19:5 ; De. 10:14 ; Ps. 50:12 ; Job. 41:2 ; 1 Co. 10:26.}, le monde et ceux qui y habitent!
\VS{2}Car il l'a fondée sur les mers, et affermie sur les fleuves.
\VS{3}Qui pourra monter à la montagne de Yahweh ? Qui s'élèvera jusqu'à son lieu saint\FTNT{Ps. 15:1-2 ; Ps. 118:19.} ?
\VS{4}Celui qui a les mains pures et le cœur pur, qui ne livre point son âme au mensonge, et qui ne jure pas pour tromper.
\VS{5}Il obtiendra la bénédiction de Yahweh et la justice du Dieu de son salut.
\VS{6}Voilà le partage de la génération qui l'invoque, de ceux qui cherchent ta face de Jacob ! Sélah.
\VS{7}Portes, élevez vos linteaux; élevez-vous portes éternelles ! Que le Roi de gloire fasse son entrée !
\VS{8}Qui est ce Roi de gloire ? C'est Yahweh fort et puissant, Yahweh puissant dans les combats.
\VS{9}Portes, élevez vos linteaux; élevez-les aussi, vous portes éternelles! Que le Roi de gloire fasse son entrée !
\VS{10}Qui est ce Roi de gloire ? Yahweh des armées : Voilà le Roi de gloire! Sélah.
\Chap{25}
\TextTitle{La crainte de Dieu mène à la voie de Yahweh}
\VerseOne{}Psaume de David. [Aleph.] Yahweh, j'élève mon âme à toi.
\VS{2}[Beth.] Mon Dieu ! Je me confie en toi : Que je ne sois point honteux\FTNT{Ps. 22:5 ; Ps. 31:2.} ! Que mes ennemis ne triomphent point de moi!
\VS{3}[Guimel.] Tous ceux qui espèrent en toi ne seront point confus\FTNT{Ro. 10:11.} ; ceux qui agissent avec tromperie sans cause seront honteux.
\VS{4}[Daleth.] Yahweh ! Fais-moi connaître tes voies, enseigne-moi tes sentiers\FTNT{Ps. 27:11 ; Ps. 86:11 ; Ps. 143:10.}.
\VS{5}[He. Vav.] Fais-moi marcher selon la vérité, et instruis-moi, car tu es le Dieu de ma délivrance, je m'attends à toi tous les jours.
\VS{6}[Zayin.] Yahweh ! Souviens-toi de ta miséricorde et de ta bonté, car elles sont éternelles\FTNT{Jé. 33:11 ; Ps. 103:17 ; Ps. 106:1 ; Ps. 107:1 ; Ps. 117:2 ; Ps. 136:1-2.}.
\VS{7}[Heth.] Ne te souviens point des péchés de ma jeunesse ni de mes transgressions ; souviens-toi de moi selon ta miséricorde, à cause de ta bonté, ô Yahweh !
\VS{8}[Teth.] Yahweh est bon et droit : C'est pourquoi il enseigne aux pécheurs la voie.
\VS{9}[Yod.] Il conduit les humbles dans la justice, et il leur enseigne sa voie.
\VS{10}[Kaf.] Tous les sentiers de Yahweh sont miséricorde et fidélité, pour ceux qui gardent son alliance et son témoignage.
\VS{11}[Lamed.] Pour l'amour de ton Nom, ô Yahweh ! Tu me pardonneras mon iniquité, car elle est grande\FTNT{2 S. 24:10.}.
\VS{12}[Mem.] Qui est l'homme qui craint Yahweh ? Yahweh lui enseignera la voie qu'il doit choisir.
\VS{13}[Nun.] Son âme demeurera dans le bonheur, et sa postérité possédera la terre en héritage.
\VS{14}[Samech.] Le secret de Yahweh est pour ceux qui le craignent, et son alliance leur donne le savoir.
\VS{15}[Ayin.] Mes yeux sont continuellement sur Yahweh, car c'est lui qui sortira mes pieds du filet.
\VS{16}[Pe.] Tourne ta face vers moi, et aie pitié de moi, car je suis seul et affligé.
\VS{17}[Tsade.] Les angoisses de mon cœur augmentent ; sors-moi de ma détresse.
\VS{18}[Resh.] Vois ma misère et ma peine, et pardonne tous mes péchés.
\VS{19}[Resh.] Vois combien mes ennemis sont nombreux, et me haïssent d'une haine pleine de violence\FTNT{Jn. 15:25.}.
\VS{20}[Shin.] Garde mon âme et délivre-moi ! Que je ne sois point confus, car je me suis réfugié en toi!
\VS{21}[Tav.] Que l'innocence et la droiture me protègent, car je m'attends à toi!
\VS{22}[Pe.] Ô Dieu ! Rachète Israël de toutes ses détresses !
\Chap{26}
\TextTitle{Demeurer dans l'intégrité}
\VerseOne{}Psaume de David. Yahweh, rends-moi justice\FTNT{Ps. 43:1 ; Ps. 54:3.} ! Car je marche dans l'intégrité, je me confie en Yahweh, je ne chancelle pas.
\VS{2}Sonde-moi et éprouve-moi\FTNT{Ps. 11:4-5 ; Ps. 17:3 ; Ps. 139:23.}, Yahweh ! Fais passer au creuset mes reins et mon cœur ;
\VS{3}car ta grâce est devant mes yeux, et je marche dans ta vérité.
\VS{4}Je ne m'assieds pas avec les hommes faux\FTNT{Ps. 1:1 ; 1 Co. 5:9-11 ; 1 Co. 15:33.}, et je ne vais point avec les gens dissimulés.
\VS{5}Je hais la compagnie de ceux qui font le mal\FTNT{Ps. 101:2-5 ; Ps. 119:113.}, et je ne m'assieds pas avec les méchants.
\VS{6}Je lave mes mains dans l'innocence et je fais le tour de ton autel\FTNT{Ps. 73:13.}, ô Yahweh !
\VS{7}Pour faire entendre le cri de reconnaissance, et pour raconter toutes tes merveilles.
\VS{8}Yahweh, j'aime la demeure de ta maison, le lieu dans lequel est le tabernacle de ta gloire.
\VS{9}N'enlève pas mon âme avec les pécheurs, ma vie avec les hommes de sang,
\VS{10}dont les mains sont criminelles, et la droite pleine de présents.
\VS{11}Moi, je marche dans l'intégrité ; délivre-moi et aie pitié de moi !
\VS{12}Mon pied se tient dans la droiture ; je bénirai Yahweh dans les assemblées.
\Chap{27}
\TextTitle{La foi qui triomphe des épreuves}
\VerseOne{}Psaume de David. Yahweh est ma lumière\FTNT{Es. 60:19-20 ; Mi. 7:8 ; Ps. 118:6 ; Jn. 8:12 ; Ap. 21:23.} et mon salut : De qui aurai-je peur ? Yahweh est le soutien de ma vie : De qui aurai-je peur ?
\VS{2}Lorsque les méchants s'avancent contre moi pour dévorer ma chair, ce sont mes adversaires et mes ennemis qui chancellent et tombent.
\VS{3}Si toute une armée campait contre moi, mon cœur ne craindrait point ; si une guerre s'élevait contre moi, je serai plein de confiance.
\VS{4}Je demande une chose à Yahweh, que je désire ardemment : C'est d'habiter dans la maison de Yahweh tous les jours de ma vie, pour contempler la beauté de Yahweh et pour admirer son temple.
\VS{5}Car il me cachera dans son tabernacle au jour du malheur, il me tiendra caché sous l'abri de sa tente; il m'élèvera sur un rocher.
\VS{6}Même maintenant ma tête s'élève par-dessus mes ennemis qui m'entourent ; et j'offrirai des sacrifices dans sa tente, au son de la trompette; je chanterai et célèbrerai Yahweh.
\VS{7}Yahweh ! Ecoute ma voix, je t'invoque : Aie pitié de moi et exauce-moi !
\VS{8}Mon cœur dit de ta part : Cherche ma face ! Je chercherai ta face, ô Yahweh !
\VS{9}Ne me cache point ta face, ne rejette point avec colère ton serviteur ! Tu es mon secours, ne me laisse pas, ne m'abandonne pas, Dieu de mon salut !
\VS{10}Car mon père et ma mère m'abandonnent, mais Yahweh me recueillera\FTNT{Es. 49:15.}.
\VS{11}Yahweh, enseigne-moi ta voie, et conduis-moi dans le sentier de la droiture, à cause de mes ennemis\FTNT{Ps. 5:9 ; Ps. 25:4-5.}.
\VS{12}Ne me livre pas au désir de mes adversaires, car s'élèvent contre moi de faux témoins et des gens qui ne respirent que la violence.
\VS{13}Oh ! Si je n'étais pas sûr de voir la bonté de Yahweh sur la terre des vivants…
\VS{14}Espère en Yahweh ! Fortifie-toi et que ton cœur s'affermisse\FTNT{Es. 33:2 ; Ps. 31:25.} ! Espère en Yahweh !
\Chap{28}
\TextTitle{Louange à Yahweh, le Rocher de son peuple}
\VerseOne{}Psaume de David. Je crie à toi, ô Yahweh ! Mon rocher! Ne te rends point sourd envers moi, de peur que si tu ne me réponds pas, je ne sois semblable à ceux qui descendent dans la fosse\FTNT{Ps. 4:2 ; Ps. 143:7. 
Voir commentaire en Es. 8:13-17.}.
\VS{2}Ecoute la voix de mes supplications, lorsque je crie à toi, quand j'élève mes mains vers ton saint sanctuaire.
\VS{3}Ne m'emporte pas avec les méchants ni avec les ouvriers d'iniquité, qui parlent de paix avec leur prochain pendant que la malice est dans leur cœur\FTNT{Jé. 9:8 ; Ps. 26:9.}.
\VS{4}Traite-les selon leurs œuvres et selon la malice de leurs actions, traite-les selon l'ouvrage de leurs mains, rends-leur ce qu'ils ont mérité\FTNT{2 Ti. 4:14.}.
\VS{5}Parce qu'ils ne prennent point garde aux œuvres de Yahweh, à l'œuvre de ses mains. Qu'il les renverse et ne les édifie point !
\VS{6}Béni soit Yahweh ! Car il exauce la voix de mes supplications.
\VS{7}Yahweh est ma force et mon bouclier ; mon cœur se confie en lui, et je suis secouru ; mon cœur se réjouit, c'est pourquoi je le loue par mes chants.
\VS{8}Yahweh est la force de son peuple, il est le refuge des délivrances de son oint.
\VS{9}Sauve ton peuple et bénis ton héritage ! Nourris-les et élève-les éternellement.
\Chap{29}
\TextTitle{La suprématie de Dieu}
\VerseOne{}Psaume de David. Fils de Dieu, rendez à Yahweh, rendez à Yahweh la gloire et la force\FTNT{Ps. 96:7-8.} !
\VS{2}Rendez à Yahweh la gloire due à son Nom ! Prosternez-vous devant Yahweh avec des ornements sacrés !
\VS{3}La voix de Yahweh est sur les eaux, le Dieu de gloire fait tonner ; Yahweh est sur les grandes eaux.
\VS{4}La voix de Yahweh est forte, la voix de Yahweh est majestueuse.
\VS{5}La voix de Yahweh brise les cèdres, Yahweh brise les cèdres du Liban,
\VS{6}il les fait sauter comme un veau, le Liban et le Sirion comme de jeunes buffles.
\VS{7}La voix de Yahweh fait jaillir des flammes de feu.
\VS{8}La voix de Yahweh fait trembler le désert; Yahweh fait trembler le désert de Kadès.
\VS{9}La voix de Yahweh fait naître les biches, et dépouille les forêts. Dans son palais tout s'écrie : Gloire !
\VS{10}Yahweh était assis lors du déluge ; Yahweh est assis comme roi éternellement\FTNT{Ps. 146:10.}.
\VS{11}Yahweh donne de la force à son peuple ; Yahweh bénit son peuple en paix.
\Chap{30}
\TextTitle{De la délivrance découle la louange}
\VerseOne{}Psaume. Cantique pour la dédicace de la maison de David.
\VS{2}Yahweh, je t'exalte parce que tu m'as relevé, tu n'as pas voulu que mes ennemis se réjouissent à mon sujet.
\VS{3}Yahweh, mon Dieu ! J'ai crié à toi, et tu m'as guéri.
\VS{4}Yahweh ! Tu as fait remonter mon âme du scheol, tu m'as rendu la vie, afin que je ne descende point dans la fosse.
\VS{5}Chantez à Yahweh, vous ses bien-aimés, et célébrez la mémoire de sa sainteté\FTNT{Ps. 97:12.} !
\VS{6}Car sa colère dure un instant, mais sa grâce toute la vie. Le soir arrivent les pleurs, et le matin les cris de louange.
\VS{7}Dans ma sécurité, je disais : Je ne serai jamais ébranlé\FTNT{Ps. 10:6.} !
\VS{8}Yahweh ! Par ta faveur tu avais affermi ma montagne… Tu cachas ta face, et je fus terrifié\FTNT{Ps. 13:2 ; Ps. 88:15 ; Ps. 102:3 ; Ps. 143:7.}.
\VS{9}Yahweh, j'ai crié à toi, j'ai présenté ma supplication à Yahweh :
\VS{10}Que gagnes-tu à verser mon sang si je descends dans la fosse ? La poussière te célébrera-t-elle ? Racontera-t-elle ta fidélité\FTNT{Es. 38:18.} ?
\VS{11}Yahweh, écoute, et aie pitié de moi ! Yahweh, secours-moi !
\VS{12}Tu as changé mon deuil en allégresse, tu as détaché mon sac, et tu m'as ceint de joie,
\VS{13}afin que ma langue te loue\FTNT{Ps. 57:10.} et ne se taise point. Yahweh, mon Dieu ! Je te célébrerai toujours.
\Chap{31}
\TextTitle{Recherche de la protection divine}
\VerseOne{}Psaume de David. Au chef des chantres.
\VS{2}Yahweh ! Tu es mon refuge : Que je ne sois jamais confus ! Délivre-moi par ta justice\FTNT{Ps. 25:2-20 ; Ps. 71:1-2.} !
\VS{3}Incline ton oreille vers moi, hâte-toi de me délivrer ! Sois pour moi un rocher protecteur, une forteresse, afin que je puisse m'y sauver !
\VS{4}Car tu es mon rocher, ma forteresse ; tu me dirigeras et tu me donneras du repos, à cause de ton Nom.
\VS{5}Tire-moi du filet qu'ils m'ont tendu en secret, car tu es ma vigueur.
\VS{6}Je remets mon esprit entre tes mains\FTNT{Lu. 23:46.} ; tu me rachèteras, Yahweh, Dieu de vérité !
\VS{7}Je hais ceux qui s'adonnent aux vanités trompeuses, et je me confie en Yahweh.
\VS{8}Je serai par ta bonté dans l'allégresse et dans la joie ; car tu vois mon affliction, tu sais les angoisses de mon âme,
\VS{9}tu ne m'as pas livré entre les mains de l'ennemi, mais tu feras tenir mes pieds au large.
\VS{10}Yahweh, aie pitié de moi! Car je suis dans la détresse. Mes yeux, mon âme et mon corps dépérissent de chagrin\FTNT{Ps. 6:8 ; Ps. 88:10.}.
\VS{11}Ma vie se consume dans la douleur, et mes années dans les soupirs ; ma force chancelle à cause de mon iniquité, et mes os sont consumés.
\VS{12}J'ai été un objet d'opprobre à cause de tous mes adversaires, de grand opprobre pour mes voisins, et de terreur pour ceux qui me connaissent ; ceux qui me voient dehors s'enfuient loin de moi\FTNT{Ps. 38:12 ; Job. 19:13-14.}.
\VS{13}Je suis oublié des cœurs comme un mort, je suis comme un vase détruit.
\VS{14}J'entends les calomnies de plusieurs, la crainte m'environne, quand ils se concertent unis contre moi : Ils projettent de m'ôter la vie\FTNT{Jé. 20:10.}.
\VS{15}Toutefois, je me confie en toi, ô Yahweh ! Je dis : Tu es mon Dieu !
\VS{16}Ma destinée est entre tes mains ; délivre-moi de la main de mes ennemis et de ceux qui me poursuivent !
\VS{17}Fais luire ta face sur ton serviteur\FTNT{Ps. 4:7 ; Ps. 67:2.}, délivre-moi par ta bonté !
\VS{18}Yahweh, que je ne sois point confus puisque je t'ai invoqué. Que les méchants soient confus, qu'ils soient couchés dans le scheol !
\VS{19}Que les lèvres menteuses soient muettes, elles profèrent des paroles dures contre le juste, avec orgueil et avec mépris!
\VS{20}Que ta bonté est grande\FTNT{Ps. 36:6.} ! Toi qui la réserves pour ceux qui te craignent, tu leur fais un refuge à la vue des fils de l'homme !
\VS{21}Tu les caches sous l'abri de ta face, loin du complot des hommes, tu les caches sous ton abri contre les langues querelleuses.
\VS{22}Béni soit Yahweh ! Car il a rendu merveilleuse sa bonté envers moi, comme si j'avais été dans une ville retranchée.
\VS{23}Je disais dans ma précipitation : Je suis retranché loin de ton regard ! Mais tu as entendu la voix de mes supplications quand j'ai crié vers toi.
\VS{24}Aimez Yahweh, vous tous, ses bien-aimés! Yahweh garde les fidèles, et il punit sévèrement les orgueilleux.
\VS{25}Fortifiez-vous et que votre esprit s'affermisse, espérez en Yahweh\FTNT{Ps. 27:14.} !
\Chap{32}
\TextTitle{La puissance du pardon}
\VerseOne{}Cantique de David. Heureux celui à qui la transgression est pardonnée, et dont le péché est couvert !
\VS{2}Heureux l'homme à qui Yahweh n'impute point son iniquité\FTNT{Ro. 4:6-8.}, et dans l'esprit duquel il n'y a point de fraude !
\VS{3}Quand je me suis tu, mes os se sont consumés, je n'ai fait que gémir tout le jour;
\VS{4}parce que jour et nuit ta main s'appesantissait sur moi\FTNT{Ps. 38:3.}, ma vigueur s'est changée en une sécheresse d'été. Sélah.
\VS{5}Je t'ai fait connaître mon péché et je n'ai point caché mon iniquité; j'ai dit : J'avouerai mes transgressions à Yahweh\FTNT{Pr. 28:13 ; 1 Jn 1:9.} ! Et tu as porté la peine de mon péché. Sélah.
\VS{6}Que tout fidèle te prie au temps convenable\FTNT{Es. 55:6 ; So. 2:3 ; Ps. 69:14.}! Si de grandes eaux débordent, elles ne l'atteindront point.
\VS{7}Tu es mon asile, tu me gardes de la détresse, tu m'environnes de chants de triomphe à cause de ta délivrance. Sélah.
\VS{8}Je te rendrai intelligent, je t'enseignerai la voie dans laquelle tu dois marcher ; je te guiderai, mon oeil sera sur toi.
\VS{9}Ne soyez pas comme le cheval ni comme le mulet qui sont sans intelligence ; il faut brider leur bouche avec un mors et un frein, de peur qu'ils ne s'approchent de toi\FTNT{Ja. 3:3.}.
\VS{10}Beaucoup de douleurs atteindront le méchant\FTNT{Pr. 19:29.}, mais la bonté environne l'homme qui se confie en Yahweh.
\VS{11}Vous justes, réjouissez-vous en Yahweh, soyez dans l'allégresse ! Criez de joie, vous tous qui êtes droits de cœur\FTNT{Ps. 33:1 ; Ps. 64:11.} !
\Chap{33}
\TextTitle{Louanges à Yahweh, le Dieu fidèle}
\VerseOne{}Vous justes, poussez un cri de joie à cause de Yahweh\FTNT{Ps. 32:11 ; Ps. 97:12 ; Ps. 147:1.} ! Sa louange sied aux hommes droits.
\VS{2}Célébrez Yahweh avec la harpe, chantez-le sur le luth à dix cordes!
\VS{3}Chantez-lui un cantique nouveau\FTNT{Ps. 40:4 ; Ps. 96:1 ; Ps. 98:1 ; Ps. 144:9 ; Ap. 5:9 ; Ap. 14:3.} ! Jouez de vos instruments avec un cri de réjouissance !
\VS{4}Car la parole de Yahweh est droite, et toutes ses œuvres s'accomplissent avec fidélité ;
\VS{5}il aime la justice et la droiture\FTNT{Ps. 45:8 ; Hé. 1:9.} ; la terre est remplie de la bonté de Yahweh.
\VS{6}Les cieux ont été faits par la parole de Yahweh, et toute leur armée par le souffle de sa bouche\FTNT{Ge. 2:1-2.}.
\VS{7}Il amoncelle en un tas les eaux de la mer, il met les abîmes dans des réservoirs.
\VS{8}Que toute la terre craigne Yahweh ! Que tous les habitants du monde le redoutent !
\VS{9}Car il dit, et la chose arrive ; il ordonne, et la chose se présente.
\VS{10}Yahweh rompt le conseil des nations, il anéantit les desseins des peuples ;
\VS{11}mais le conseil de Yahweh subsiste à toujours, les desseins de son cœur subsistent d'âge en âge\FTNT{Pr. 19:21.}.
\VS{12}Heureuse la nation dont Yahweh est le Dieu\FTNT{Ps. 144:15.} et le peuple qu'il s'est choisi pour héritage !
\VS{13}Yahweh regarde des cieux, il voit tous les fils des hommes\FTNT{Job. 28:24.};
\VS{14}du lieu de sa demeure, il observe tous les habitants de la terre.
\VS{15}C'est lui qui forme également leur cœur et qui prend garde à toutes leurs actions.
\VS{16}Le roi n'est point sauvé par une grande armée, l'homme puissant n'échappe point par sa grande force;
\VS{17}le cheval est impuissant pour sauver, et ne délivre point par la grandeur de sa force\FTNT{Ps. 147:10.}.
\VS{18}Voici, l'œil de Yahweh est sur ceux qui le craignent\FTNT{Ps. 34:16 ; 1 Pi. 3:12.}, sur ceux qui s'attendent à sa bonté,
\VS{19}afin qu'il les délivre de la mort, et les fasse vivre durant la famine.
\VS{20}Notre âme espère en Yahweh; il est notre aide et notre bouclier.
\VS{21}Notre cœur se réjouit en lui, car nous avons confiance en son saint Nom.
\VS{22}Que ta bonté soit sur nous, ô Yahweh ! Nous nous attendons à toi!
\Chap{34}
\TextTitle{Yahweh libère les siens}
\VerseOne{}Psaume de David, lorsqu'il contrefît l'insensé en présence d'Abimélec, qui s'en alla, chassé par lui.
\VS{2}[Aleph.] Je bénirai Yahweh en tout temps, sa louange sera continuellement dans ma bouche.
\VS{3}[Beth.] Mon âme se glorifie en Yahweh ! Que les pauvres écoutent et se réjouissent.
\VS{4}[Guimel.] Glorifiez Yahweh avec moi ! Elevons son Nom tous ensemble !
\VS{5}[Daleth.] J'ai cherché Yahweh et il m'a répondu ; il m'a délivré de toutes mes frayeurs.
\VS{6}[He. Vav.] Quand on le regarde, on est illuminé, et la face n'est point confuse.
\VS{7}[Zayin.] Cet affligé a crié et Yahweh l'a exaucé, et l'a délivré de toutes ses détresses.
\VS{8}[Heth.] L'ange de Yahweh campe tout autour de ceux qui le craignent, et les équipe.
\VS{9}[Teth.] Goûtez et voyez combien Yahweh est bon ! Heureux l'homme qui se confie en lui !
\VS{10}[Yod.] Craignez Yahweh vous ses saints ! Car rien ne manque à ceux qui le craignent.
\VS{11}[Kaf.] Les lionceaux éprouvent la disette et la faim, mais ceux qui cherchent Yahweh ne manquent d'aucun bien.
\VS{12}[Lamed.] Venez, mes fils, écoutez-moi ! Je vous enseignerai la crainte de Yahweh.
\VS{13}[Mem.] Qui est l'homme qui prend plaisir à la vie, qui aime la prolonger pour jouir du bonheur ?
\VS{14}[Nun.] Garde ta langue du mal et tes lèvres des paroles trompeuses\FTNT{1 Pi. 3:10.} ;
\VS{15}[Samech.] détourne-toi du mal et fais-le bien ; cherche la paix et poursuis-la\FTNT{Hé. 12:14.}.
\VS{16}[Ayin.] Les yeux de Yahweh sont sur les justes et ses oreilles sont attentives à leur cri.
\VS{17}[Pe.] La face de Yahweh est contre ceux qui font le mal, pour retrancher de la terre leur mémoire\FTNT{Jé. 44:11 ; Lé. 17:10.}.
\VS{18}[Tsade.] Quand les justes crient, Yahweh les exauce et il les délivre de toutes leurs détresses.
\VS{19}[Qof.] Yahweh est près de ceux qui ont le cœur déchiré par la douleur, et il délivre ceux qui ont l'esprit abattu.
\VS{20}[Resh.] Le juste a des maux en grand nombre, mais Yahweh le délivre de tous\FTNT{2 Ti. 3:11.}.
\VS{21}[Shin.] Il garde tous ses os, aucun d'eux n'est brisé.
\VS{22}[Tav.] Le mauvais tue le méchant, et ceux qui haïssent le juste sont détruits.
\VS{23}[Pe.] Yahweh rachète l'âme de ses serviteurs, et aucun de ceux qui se confient en lui ne sera détruit.
\Chap{35}
\TextTitle{Prière au juste Juge}
\VerseOne{}Psaume de David. Yahweh, défends-moi contre mes adversaires, combats ceux qui me combattent !
\VS{2}Prends le petit et le grand bouclier, et lève-toi pour me secourir !
\VS{3}Brandis la lance et le javelot contre mes persécuteurs ! Dis à mon âme : Je suis ta délivrance !
\VS{4}Que ceux qui en veulent à ma vie soient honteux et confus\FTNT{Jé. 17:18 ; Ps. 40:15 ; Ps. 70:3.} ! Que ceux qui méditent ma perte reculent et rougissent !
\VS{5}Qu'ils soient comme la balle emportée par le vent\FTNT{Es. 29:5 ; Os. 13:3.}, et que l'Ange de Yahweh les chasse !
\VS{6}Que leur chemin soit ténébreux et glissant, et que l'Ange de Yahweh les poursuive.
\VS{7}Car sans cause ils m'ont tendu leur filet sur une fosse, sans cause ils l'ont creusée pour m'ôter la vie\FTNT{Jé. 18:20 ; Ps. 57:7 ; Ps. 140:5 ; Ps. 141:9.}.
\VS{8}Que la ruine les atteigne sans qu'ils le sachent, qu'ils soient capturés dans le filet qu'ils ont caché. Qu'ils y tombent et soient ravagés !
\VS{9}Mon âme aura de la joie en Yahweh, de l'allégresse en sa délivrance.
\VS{10}Tous mes os diront : Yahweh ! Qui est semblable à toi ? Qui délivre l'affligé de la main de celui qui est plus fort que lui ? L'affligé et le pauvre de celui qui le pille ?
\VS{11}De faux témoins s'élèvent contre moi : On m'interroge sur ce que j'ignore.
\VS{12}Ils me rendent le mal pour le bien, tâchant de m'ôter la vie\FTNT{Ps. 38:21 ; Ps. 109:5.}.
\VS{13}Mais moi, quand ils étaient malades, je me couvrais d'un sac, j'affligeais mon âme par le jeûne, je priais dans mon sein,
\VS{14}comme pour un ami, pour un frère, j'étais abattu, en pleurs, comme pour le deuil d'une mère.
\VS{15}Mais quand je chancelle, ils se réjouissent et s'assemblent, ils s'assemblent contre moi sans que je le sache pour me frapper, ils me déchirent pour que je sois silencieux ;
\VS{16}avec les hypocrites d’entre les railleurs qui suivent les bonnes tables, et ils ont grincé les dents contre moi.
\VS{17}Seigneur ! Jusqu'à quand le verras-tu ? Détourne mon âme de leurs tempêtes, mon unique des lionceaux.
\VS{18}Je te célébrerai dans la grande assemblée, je te louerai parmi un peuple nombreux\FTNT{Ps. 111:1.}.
\VS{19}Que ceux qui sont mes ennemis par leur mensonge ne se réjouissent point de moi, que ceux qui me haïssent sans cause ne m'insultent point par leurs regards\FTNT{Jn. 15:25.}.
\VS{20}Car ils ne parlent point de paix, mais ils préméditent des choses pleines de fraudes contre les gens tranquilles de la terre.
\VS{21}Ils ont ouvert leur bouche autant qu'ils ont pu contre moi, et ont dit : Ah ! Ah ! Nos yeux l'ont vu !
\VS{22}Yahweh ! Tu le vois : Ne te tais point\FTNT{Ps. 83:2.} ! Seigneur, ne t'éloigne point de moi !
\VS{23}Réveille-toi, réveille-toi pour me rendre justice\FTNT{Ps. 44:24.} ! Mon Dieu et mon Seigneur, défends ma cause !
\VS{24}Juge-moi selon ta justice, Yahweh mon Dieu ! et qu'ils ne se réjouissent point de moi !
\VS{25}Qu'ils ne disent point dans leur cœur : Ah ! Notre âme ! Et qu'ils ne disent point : Nous l'avons englouti !
\VS{26}Que ceux qui se réjouissent de mon mal soient honteux et rougissent tous ensemble ! Que ceux qui s'élèvent contre moi soient couverts de honte et de confusion !
\VS{27}Mais que ceux qui prennent plaisir à ma justice se réjouissent avec des chants de triomphe, qu'ils disent sans cesse : Grand est Yahweh qui désire la paix de son serviteur !
\VS{28}Alors ma langue publiera ta justice et ta louange tous les jours.
\Chap{36}
\TextTitle{Opposition : Les justes et les méchants}
\VerseOne{}Psaume de David, serviteur de Yahweh, donné au chef des chantres.
\VS{2}La transgression du méchant me dit, au dedans de mon cœur, qu'il n'y a point de crainte de Dieu devant ses yeux.
\VS{3}Car il se flatte à ses propres yeux pour consumer, pour assouvir sa haine.
\VS{4}Les paroles de sa bouche ne sont que méchanceté et tromperie, il cesse d'être sage et de faire le bien.
\VS{5}Il projette le malheur sur sa couche, il se tient sur un chemin qui n'est pas bon, il ne rejette pas le mal.
\VS{6}Yahweh ! Ta bonté atteint jusqu'aux cieux, ta fidélité jusqu'aux nues\FTNT{Ps. 57:11 ; Ps. 108:5.}.
\VS{7}Ta justice est comme les montagnes de Dieu, tes jugements sont un grand abîme. Yahweh ! Tu sauves les hommes et les bêtes.
\VS{8}Ô Dieu ! Combien est précieuse ta bonté ! Aussi les fils des hommes se retirent à l'ombre de tes ailes\FTNT{Ps. 17:8 ; Ps. 57:2.}.
\VS{9}Ils seront abondamment rassasiés de la graisse de ta maison et tu les abreuveras au fleuve de tes délices.
\VS{10}Car la source de la vie est auprès de toi, et par ta lumière nous voyons la lumière.
\VS{11}Etends ta bonté sur ceux qui te connaissent, et ta justice sur ceux qui ont le cœur droit !
\VS{12}Que le pied de l'orgueilleux ne s'avance point sur moi, et que la main des méchants ne m'ébranle point !
\VS{13}Là sont tombés les ouvriers d'iniquité ; ils sont renversés et ne peuvent se relever.
\Chap{37}
\TextTitle{Se confier en la justice de Yahweh}
\VerseOne{}Psaume de David. [Aleph.] Ne t'irrite pas contre les méchants, ne jalouse pas ceux qui s'adonnent à la perversité\FTNT{Pr. 23:17 ; Pr. 24:19.}.
\VS{2}Car ils seront soudainement retranchés comme le foin, et ils se faneront comme l'herbe verte.
\VS{3}[Beth.] Confie-toi en Yahweh, et fais ce qui est bon ; aie le pays pour demeure et la fidélité pour pâture.
\VS{4}Fais de Yahweh tes délices et il t'accordera ce que ton cœur désire.
\VS{5}[Guimel.] Recommande tes voies à Yahweh, confie-toi en lui et il agira\FTNT{Ps. 22:9 ; Ps. 55:23 ; Pr. 16:3.}.
\VS{6}Il manifestera ta justice comme la lumière et ton droit comme le soleil à son midi\FTNT{Pr. 4:18.}.
\VS{7}[Daleth.] Garde le silence devant Yahweh et tremble devant lui ; ne t'irrite point contre celui qui réussit dans ses voies, contre celui qui vient à bout de ses mauvais desseins.
\VS{8}[He.] Laisse la colère et abandonne la rage\FTNT{Ep. 4:26.} ; ne t'irrite pas pour faire le mal.
\VS{9}Car les méchants seront retranchés, mais ceux qui se confient en Yahweh hériteront la terre.
\VS{10}[Vav.] Encore un peu de temps et le méchant ne sera plus ; tu regardes le lieu où il était et il n'y est plus\FTNT{Job. 7:10 ; Job. 20:9.}.
\VS{11}Les pauvres prennent possession du pays et jouissent abondamment de la paix.
\VS{12}[Zayin.] Le méchant complote contre le juste et grince ses dents contre lui.
\VS{13}Le Seigneur se rit de lui, car il voit que son jour approche.
\VS{14}[Heth.] Les méchants tirent leur épée et bandent leur arc pour faire tomber le malheureux et le pauvre, pour massacrer ceux qui marchent dans la droiture\FTNT{Ps. 11:2.}.
\VS{15}Mais leur épée entre dans leur propre cœur, et leurs arcs se brisent.
\VS{16}[Teth.] Mieux vaut au juste le peu qu'il a, que l'abondance de beaucoup de méchants\FTNT{Pr. 15:16-17 ; Ec. 4:6.} ;
\VS{17}car les bras des méchants seront brisés, mais Yahweh soutient les justes.
\VS{18}[Yod.] Yahweh connaît les jours de ceux qui sont intègres, et leur héritage demeure à jamais.
\VS{19}Ils ne sont pas honteux au jour du malheur, mais ils sont rassasiés au jour de la famine.
\VS{20}[Kaf.] Mais les méchants périssent, et les ennemis de Yahweh, comme les beaux pâturages, s'évanouissent, ils s'évanouissent en fumée.
\VS{21}[Lamed.] Le méchant emprunte et ne rend point ; mais le juste a compassion et donne.
\VS{22}Car les bénis de Yahweh hériteront la terre, mais ceux qu'il a maudits seront retranchés.
\VS{23}[Mem.] Yahweh affermit les pas de l'homme, et il prend plaisir à ses voies.
\VS{24}S'il tombe, il ne sera pas entièrement abattu, car Yahweh le soutient de sa main.
\VS{25}[Nun.] J'ai été jeune et j'ai vieilli ; et je n'ai point vu le juste abandonné, ni sa postérité mendiant son pain.
\VS{26}Il est compatissant tout le temps, et il prête ; et sa postérité est bénie.
\VS{27}[Samech.] Retire-toi du mal et fais le bien ; et tu auras une demeure éternelle.
\VS{28}Car Yahweh aime ce qui est juste, et il n'abandonne point ses fidèles ; c'est pourquoi ils sont sous sa garde pour toujours, mais la postérité des méchants est retranchée.
\VS{29}[Ayin.] Les justes hériteront la terre et y habiteront à perpétuité.
\VS{30}[Pe.] La bouche du juste prononce la sagesse et sa langue déclare la justice.
\VS{31}La loi de son Dieu est dans son cœur\FTNT{Ps. 40:8-9.}, aucun de ses pas ne chancellera.
\VS{32}[Tsade.] Le méchant épie le juste et cherche à le faire mourir.
\VS{33}Yahweh ne l'abandonne point entre ses mains et ne le laisse point condamner quand on le juge.
\VS{34}[Qof.] Espère en Yahweh et garde sa voie, et il t'élèvera pour que tu hérites la terre ; tu verras les méchants retranchés.
\VS{35}[Resh.] J'ai vu le méchant dans toute sa puissance, il s'étendait comme un arbre verdoyant.
\VS{36}Il a passé, et voici, il n'est plus ; je le cherche et il ne se trouve plus.
\VS{37}[Shin.] Observe l'homme intègre et considère l'homme droit, car il y a une issue pour l'homme de paix.
\VS{38}Mais les rebelles seront tous détruits et ce qui sera resté des méchants sera retranché.
\VS{39}[Tav.] Mais la délivrance des justes viendra de Yahweh, il sera leur force au temps de la détresse.
\VS{40}Yahweh les secourt et les délivre ; il les délivre des méchants et les sauve, parce qu'ils se confient en lui.
\Chap{38}
\TextTitle{La tristesse du péché mène à la repentance}
\VerseOne{}Psaume de David. Pour souvenir.
\VS{2}Yahweh ! Ne me juge pas dans ta colère et ne me châtie pas dans ta fureur.
\VS{3}Car tes flèches m'ont atteint et ta main s'est appesantie sur moi.
\VS{4}Il n'y a rien de sain dans ma chair, à cause de ta colère, ni de paix dans mes os, à cause de mon péché.
\VS{5}Car mes iniquités s'élèvent au-dessus de ma tête, elles se sont appesanties comme un pesant fardeau, au-delà de mes forces\FTNT{Ps. 40:13.}.
\VS{6}Mes plaies ont une mauvaise odeur et sont purulentes à cause de ma folie.
\VS{7}Je suis courbé et abattu outre mesure ; je marche en pleurs tout le jour.
\VS{8}Car un mal brûlant remplit mes reins, et dans ma chair il n'y a rien de sain.
\VS{9}Je suis affaibli et brisé, je rougis le cœur troublé.
\VS{10}Seigneur, tout mon désir est devant toi, et mon soupir ne t'est point caché.
\VS{11}Mon cœur est agité çà et là, ma force m'abandonne, et la lumière de mes yeux n'est plus avec moi.
\VS{12}Ceux qui m'aiment, et même mes amis intimes, se tiennent loin de ma plaie, et mes proches se tiennent loin de moi\FTNT{Job. 19:13-14.}.
\VS{13}Ceux qui en veulent à ma vie me tendent des pièges ; ceux qui cherchent ma perte parlent de calamités et méditent des tromperies tous les jours.
\VS{14}Mais moi je suis comme un sourd, comme un muet qui n'ouvre point sa bouche.
\VS{15}Je suis, dis-je, comme un homme qui n'entend pas et qui n'a point de réplique dans sa bouche.
\VS{16}Car je m'attends à toi, ô Yahweh ! Tu me répondras, Seigneur mon Dieu !
\VS{17}Je dis : Il faut prendre garde qu'ils ne triomphent de moi ; quand mon pied chancelle, ils s'élèvent contre moi\FTNT{Ps. 94:18.} !
\VS{18}Car je suis près de tomber et ma douleur est continuellement devant moi.
\VS{19}Car je reconnais mon iniquité et je suis dans la crainte à cause de mon péché.
\VS{20}Cependant mes ennemis qui sont vivants se renforcent, et ceux qui me haïssent à tort se multiplient.
\VS{21}Ceux qui me rendent le mal pour le bien sont mes adversaires, parce que je recherche le bien\FTNT{Ps. 109:5 ; Jé. 18:20.}.
\VS{22}Ne m'abandonne pas Yahweh ! Mon Dieu, ne t'éloigne pas de moi !
\VS{23}Hâte-toi de venir à mon secours, Seigneur, tu es ma délivrance !
\Chap{39}
\TextTitle{La faiblesse de l'homme}
\VerseOne{}Psaume de David, donné au chef des chantres, à Jeduthun.
\VS{2}J'ai dit : Je prends garde à mes voies, de peur de pécher par ma langue ; je mettrai un frein à ma bouche tant que le méchant sera devant moi.
\VS{3}Je suis resté muet, dans le silence ; je me suis tu, quoique malheureux ; et ma douleur n'était pas moins vive.
\VS{4}Mon cœur brûlait au-dedans de moi, un feu intérieur me consumait, et la parole est venue sur ma langue.
\VS{5}Yahweh ! Dis-moi quel est le terme de ma vie et quelle est la mesure de mes jours\FTNT{Ps. 119:84.} ; que je sache combien je suis fragile.
\VS{6}Voici, tu as réduit mes jours à la largeur de ma main, et ma vie est comme un rien devant toi. Oui, tout homme debout n'est qu'un souffle\FTNT{Ja. 4:14.}. Sélah.
\VS{7}Oui, l'homme se promène comme une ombre, il s'agite inutilement ; il amasse des biens et il ne sait pas qui les recueillera.
\VS{8}Maintenant que puis-je espérer, Seigneur ? Mon espérance est en toi.
\VS{9}Délivre-moi de toutes mes transgressions ! Ne permets pas l'opprobre des insensés.
\VS{10}Je me suis tu et je n'ai point ouvert ma bouche, parce que c'est toi qui agis.
\VS{11}Détourne de moi tes coups ! Je suis consumé par les attaques de ta main.
\VS{12}Aussitôt que tu châties quelqu'un, en le punissant à cause de son iniquité, tu détruis comme la teigne ce qu'il a de plus cher. Oui, tout homme est une vapeur. Sélah.
\VS{13}Yahweh, écoute ma prière et prête l'oreille à mon cri ! Ne sois point sourd à mes larmes ! Car je suis un voyageur et un étranger chez toi, comme tous mes pères\FTNT{Lé. 25:23 ; Ps. 119:19 ; 1 Pi. 2:11 ; Hé. 11:13.}.
\VS{14}Détourne ton regard de moi, afin que je reprenne mes forces, avant que je m'en aille et que je ne sois plus.
\Chap{40}
\TextTitle{Un cantique nouveau à Yahweh}
\VerseOne{}Psaume de David, donné au chef des chantres.
\VS{2}J'ai attendu patiemment Yahweh, et il s'est tourné vers moi et a entendu mon cri.
\VS{3}Il m'a retiré de la fosse de destruction, du fond de la boue ; il a mis mes pieds sur un roc et a assuré mes pas.
\VS{4}Il a mis dans ma bouche un cantique nouveau, qui est la louange de notre Dieu ; plusieurs verront cela et ils craindront et se confieront en Yahweh.
\VS{5}Heureux l'homme qui place sa confiance en Yahweh et qui ne se tourne pas vers les orgueilleux et les menteurs !
\VS{6}Yahweh, mon Dieu ! Tu as multiplié tes merveilles et tes desseins envers nous ; nul n'est comparable à toi ; je voudrais les annoncer et les déclarer, mais leur nombre est trop grand pour que je les raconte.
\VS{7}Tu ne désires ni sacrifice ni offrande. Tu m'as percé les oreilles ; tu ne demandes ni holocauste ni victime expiatoire pour le péché\FTNT{Hé. 10:5.}.
\VS{8}Alors je dis : Voici, je viens avec le rouleau du livre écrit pour moi.
\VS{9}Mon Dieu, je prends plaisir à faire ta volonté, et ta loi est au fond de mes entrailles\FTNT{Ps. 37:31 ; Es. 51:7.}.
\VS{10}J'annonce ta justice dans la grande assemblée ; voilà, je ne ferme pas mes lèvres, Yahweh, tu le sais !
\VS{11}Je ne cache pas ta justice, qui est dans mon cœur ; je déclare ta fidélité et ta délivrance ; je ne cache pas ta bonté ni ta vérité dans la grande assemblée.
\VS{12}Toi, Yahweh ! Ne m'épargne point tes compassions, que ta bonté et ta vérité me gardent continuellement.
\VS{13}Car des maux sans nombre m'environnent ; mes iniquités m'atteignent et je ne supporte pas leur vue ; elles surpassent en nombre les cheveux de ma tête, et mon cœur m'abandonne.
\VS{14}Yahweh, veuille me délivrer ! Yahweh, hâte-toi de venir à mon secours !
\VS{15}Que tous ensemble ils soient honteux et confus, ceux qui cherchent mon âme pour la perdre ; et que ceux qui prennent plaisir à mon malheur retournent en arrière et rougissent.
\VS{16}Que ceux qui disent de moi : Ah ! Ah ! Soient consumés, en récompense de la honte qu'ils m'ont faite.
\VS{17}Que tous ceux qui te cherchent soient dans l'allégresse et se réjouissent en toi\FTNT{Ps. 70:5.} ! Que ceux qui aiment ta délivrance disent continuellement : Grand est Yahweh !
\VS{18}Moi, je suis affligé et misérable, mais le Seigneur prend soin de moi. Tu es mon secours et mon libérateur : Mon Dieu ne tarde point\FTNT{Ps. 70:6.} !
\Chap{41}
\TextTitle{Secours de Yahweh dans le malheur}
\VerseOne{}Psaume de David, donné au chef des chantres.
\VS{2}Heureux celui qui s'intéresse au pauvre ! Yahweh le délivrera au jour du malheur ;
\VS{3}Yahweh le garde et lui conserve la vie. Il est heureux sur la terre, et tu ne le livres pas au bon plaisir de ses ennemis.
\VS{4}Yahweh le soutient sur son lit de douleur ; tu le soulages dans toutes ses maladies.
\VS{5}Je dis : Yahweh ! Aie pitié de moi, guéris mon âme, car j'ai péché contre toi.
\VS{6}Mes ennemis disent du mal de moi : Quand mourra-t-il ? Et quand périra son nom ?
\VS{7}Si quelqu'un vient me voir, il dit des mensonges, il recueille de mauvais desseins\FTNT{Ps. 5:10 ; Ps. 10:7 ; Ps. 12:3.}, il s'en va et il parle au-dehors.
\VS{8}Tous ceux qui m'ont en haine murmurent sourdement ensemble contre moi, et machinent du mal contre moi.
\VS{9}Quelque action criminelle\FTNT{Le mot « criminelle » donne en hébreu « Belial »} pèse sur lui ; le voilà couché, disent–ils, il ne se relèvera plus !
\VS{10}Même celui qui était en paix avec moi, qui avait ma confiance et qui mangeait mon pain, a levé le talon contre moi\FTNT{Il est question ici de la trahison du Messie par Judas (Jn. 13:18-19).}.
\VS{11}Mais toi, ô Yahweh ! Aie pitié de moi et relève-moi ! Et je leur rendrai ce qui leur est dû.
\VS{12}Je connaîtrai que tu prends plaisir en moi, si mon ennemi ne triomphe pas de moi.
\VS{13}Pour moi, tu m'as soutenu à cause de mon intégrité, et tu m'as établi pour toujours en ta présence.
\VS{14}Béni soit Yahweh, le Dieu d'Israël, d'éternité en éternité. Amen ! Amen !
\Chap{42}
\TextTitle{Avoir soif de Dieu}
\VerseOne{}Cantique des fils de Koré, donné au chef des chantres.
\VS{2}Comme une biche soupire après des courants d'eau, ainsi mon âme soupire ardemment après toi, ô Dieu !
\VS{3}Mon âme a soif de Dieu, du Dieu vivant\FTNT{Ps. 63:2 ; Ps. 84:3.} : Ô quand entrerai-je et me présenterai-je devant la face de Dieu ?
\VS{4}Mes larmes sont ma nourriture jour et nuit, quand on me dit chaque jour : Où est ton Dieu\FTNT{Ps. 80:6 ; Ps. 115:2.} ?
\VS{5}Je rappelais ces choses dans mon souvenir, en répandant mon âme au-dedans de moi, savoir que je marchais dans la foule, et que je m'en allais tout doucement en leur compagnie, avec une voix de triomphe et de louange, jusqu'à la maison de Dieu, et qu'une grande multitude de gens sautait alors de joie.
\VS{6}Mon âme, pourquoi t'abats-tu et murmures-tu au-dedans de moi ? Attends-toi à Dieu, car je le célébrerai encore ; sa face est la délivrance même.
\VS{7}Mon Dieu ! Mon âme est abattue au-dedans de moi, aussi je me souviens de toi depuis la terre du Jourdain, depuis l'Hermon, et depuis la montagne de Mitsear.
\VS{8}Un flot appelle un autre flot au bruit de tes ondées ; toutes tes vagues et tes flots passent sur moi.
\VS{9}Toutefois, Yahweh enverra sa bonté compatissante de jour, et de nuit son cantique sera avec moi, et ma prière sera au Dieu qui est ma vie.
\VS{10}Je dis à Dieu, mon rocher : Pourquoi m'oublies-tu ? Pourquoi marcherai-je dans la tristesse à cause de l'oppression de l'ennemi ?
\VS{11}Comme avec une épée dans mes os, mes ennemis m'outragent, tandis qu'ils me disent chaque jour : Où est ton Dieu ?
\VS{12}Mon âme, pourquoi t'abats-tu et pourquoi murmures-tu au-dedans de moi ? Attends-toi à Dieu, car je le célébrerai encore, il est ma délivrance et mon Dieu.
\Chap{43}
\TextTitle{Espérer dans la délivrance de Dieu}
\VerseOne{}Fais-moi justice, ô Dieu ! Et défends ma cause contre une nation infidèle\FTNT{Ps. 26:1 ; Ps. 35:1.} ! Délivre-moi de l'homme trompeur et pervers.
\VS{2}Toi, mon Dieu protecteur, pourquoi me repousses-tu ? Pourquoi marcherai-je dans la tristesse à cause de l'oppression de l'ennemi ?
\VS{3}Envoie ta lumière et ta vérité, afin qu'elles me conduisent et m'introduisent dans ta sainte montagne, et dans tes demeures.
\VS{4}Alors je viendrai à l'autel de Dieu, au Dieu de ma joie et mon allégresse, et je te célébrerai sur la harpe, ô Dieu ! Mon Dieu !
\VS{5}Mon âme, pourquoi t'abats-tu et pourquoi murmures-tu au-dedans de moi ? Attends-toi à Dieu, car je le célébrerai encore ; il est ma délivrance et mon Dieu\FTNT{Ps. 42:6.}.
\Chap{44}
\TextTitle{Prière des affligés}
\VerseOne{}Cantique des fils de Koré, donné au chef des chantres.
\VS{2}Ô Dieu ! Nous avons entendu de nos oreilles et nos pères nous ont raconté les exploits que tu as faits de leur temps, aux jours d'autrefois\FTNT{Jg. 6:13 ; Ps. 77:12.}.
\VS{3}Tu as de ta main chassé les nations, et tu as affermi nos pères, tu as affligé les peuples, et tu as fait prospérer nos pères.
\VS{4}Car ce n'est point par leur épée qu'ils ont conquis le pays, et ce n'est pas leur bras qui les a délivrés, mais c'est ta droite, c'est ton bras, c'est la lumière de ta face, parce que tu les aimais.
\VS{5}Ô Dieu ! Tu es mon Roi : Ordonne la délivrance de Jacob !
\VS{6}Avec toi nous battrons nos adversaires, par ton Nom nous foulerons ceux qui s'élèvent contre nous.
\VS{7}Car je ne me confie point en mon arc, et ce n'est pas mon épée qui me délivrera.
\VS{8}Mais tu nous délivreras de nos adversaires, et tu rendras confus ceux qui nous haïssent.
\VS{9}Nous nous glorifierons en Dieu chaque jour et nous célébrerons à jamais ton Nom. Sélah.
\VS{10}Mais tu nous rejettes, tu nous confonds, et tu ne sors plus avec nos armées.
\VS{11}Tu nous fais reculer devant l'adversaire, et ceux qui nous haïssent enlèvent nos dépouilles.
\VS{12}Tu nous livres comme des brebis destinées à être dévorées, et tu nous as dispersés parmi les nations.
\VS{13}Tu as vendu ton peuple pour rien, et tu ne l'estimes pas d'une grande valeur\FTNT{Es. 52:3 ; Jé. 15:13.}.
\VS{14}Tu nous as mis en opprobre chez nos voisins, en dérision, et en sujet de moquerie auprès de ceux qui habitent autour de nous\FTNT{Jé. 24:9 ; Ps. 79:4.}.
\VS{15}Tu fais de nous un objet de sarcasmes parmi les nations, et de hochement de tête parmi les peuples.
\VS{16}Ma confusion est tout le jour devant moi, et la honte couvre ma face,
\VS{17}à cause des discours de celui qui nous fait des reproches et qui nous injurie, et à cause de l'ennemi et du vindicatif.
\VS{18}Tout cela nous est arrivé, et cependant nous ne t'avons point oublié, et nous n'avons point violé ton alliance.
\VS{19}Notre cœur ne s'est point détourné, nos pas ne se sont point éloignés de tes sentiers,
\VS{20}pour que tu nous écrases dans le lieu du serpent, et que tu nous couvres de l'ombre de la mort\FTNT{Ps. 23:4.}.
\VS{21}Si nous avions oublié le Nom de notre Dieu et étendu nos mains vers un dieu étranger,
\VS{22}Dieu ne le saurait-il pas, lui qui connaît les secrets du cœur ?
\VS{23}Mais nous sommes tous les jours mis à mort pour l'amour de toi, nous sommes regardés comme des brebis destinées à la boucherie\FTNT{Es. 53:7.}.
\VS{24}Lève-toi, pourquoi dors-tu Seigneur ? Réveille-toi ! Ne nous rejette point à jamais !
\VS{25}Pourquoi caches-tu ta face, pourquoi oublies-tu notre affliction et notre oppression ?
\VS{26}Car notre âme est abattue dans la poussière et notre ventre est attaché à la terre.
\VS{27}Lève-toi pour nous secourir ! Délivre-nous à cause de ta bonté.
\Chap{45}
\TextTitle{La beauté du Roi}
\VerseOne{}Cantique des fils de Koré, qui est un chant nuptial donné au chef des chantres pour le chanter sur Shoshannim.
\VS{2}Des paroles agréables bouillonnent dans mon cœur, et j'ai dit : Mon œuvre est pour le roi ! Ma langue sera comme la plume d'un habile écrivain !\FTNT{Ca. 5:13 ; Ca. 5:16.}
\VS{3}Tu es le plus beau des fils de l'homme, la grâce est répandue sur tes lèvres : C'est pourquoi Dieu t'a béni éternellement.
\VS{4}Ô héros, ceins ton épée sur ta cuisse, ta majesté et ta magnificence,
\VS{5}et prospère dans ta magnificence. Sois porté sur la parole de vérité, de douceur, et de justice, et ta droite versera des choses terribles !
\VS{6}Tes flèches sont aiguës, les peuples tomberont sous toi, elles perceront le cœur des ennemis du roi.
\VS{7}Ô Dieu, ton trône est à toujours et à perpétuité ! Le sceptre de ton règne est un sceptre d'équité.
\VS{8}Tu aimes la justice et tu hais la méchanceté : C'est pourquoi, ô Dieu, ton Dieu t'a oint d'une huile de joie par privilège sur tes compagnons\FTNT{Hé. 1:8-9.}.
\VS{9}Tous tes vêtements sont parfumés de myrrhe, d'aloès et de casse. Dans les palais d'ivoire les instruments à cordes te réjouissent.
\VS{10}Des filles de rois sont parmi tes bien-aimées ; la reine est à ta droite, parée d'or d'Ophir.
\VS{11}Ecoute, jeune fille, vois et prête l'oreille ; oublie ton peuple et la maison de ton père.
\VS{12}Le roi porte ses désirs sur ta beauté ; puisqu'il est ton Seigneur, prosterne-toi devant lui.
\VS{13}La fille de Tyr et les plus riches des peuples te supplieront avec des présents.
\VS{14}La fille du roi est intérieurement pleine de gloire. Elle porte un vêtement tissé d'or.
\VS{15}Elle sera présentée au roi en vêtements de broderie, et les filles qui viennent après elle, et qui sont ses compagnes, seront amenées vers toi.
\VS{16}Elles te seront présentées avec réjouissance et allégresse, et elles entreront au palais du roi.
\VS{17}Tes fils seront au lieu de tes pères, tu les établiras pour princes sur toute la terre.
\VS{18}Je rendrai ton Nom mémorable dans tous les âges, et à cause de cela les peuples te célébreront pour toujours et à perpétuité\FTNT{Ps. 67:3-5.}.
\Chap{46}
\TextTitle{L'assurance du peuple de Dieu}
\VerseOne{}Cantique des fils de Koré, donné au chef des chantres pour le chanter sur Alamoth. Cantique.
\VS{2}Dieu est notre retraite, notre force, et notre secours qui ne manque jamais dans les détresses\FTNT{Ps. 9:10.}.
\VS{3}C'est pourquoi nous ne craindrons point quand la terre est bouleversée et que les montagnes chancellent au cœur des mers\FTNT{Es. 54:10.},
\VS{4}quand ses eaux mugissent et écument, se soulèvent jusqu'à faire trembler les montagnes. Sélah.
\VS{5}Il est un fleuve dont les courants réjouissent la cité de Dieu, le lieu saint des demeures du Très-Haut\FTNT{Ez. 47:1-2 ; Za. 14:8-9 ; Jn. 7:38 ; Ap. 22:1-2.}.
\VS{6}Dieu est au milieu d'elle : Elle n'est point ébranlée. Dieu la secourt dès le point du jour\FTNT{So. 3:16-17.}.
\VS{7}Les nations murmurent, les royaumes s'ébranlent ; il a fait entendre sa voix et la terre se fond.
\VS{8}Yahweh des armées est avec nous, le Dieu de Jacob est pour nous une haute retraite. Sélah.
\VS{9}Venez, contemplez les œuvres de Yahweh et voyez quels ravages il a faits sur la terre.
\VS{10}Il a fait cesser les guerres jusqu'au bout de la terre, il a brisé l'arc et rompu la lance, il a consumé par le feu les chars de guerre\FTNT{Es. 2:4.}.
\VS{11}Arrêtez, et sachez que je suis Dieu : Je suis élevé parmi les nations, je suis élevé sur toute la terre.
\VS{12}Yahweh des armées est avec nous, le Dieu de Jacob est pour nous une haute retraite. Sélah.
\Chap{47}
\TextTitle{Yahweh, le Dieu élevé}
\VerseOne{}Psaume des fils de Koré, donné au chef des chantres.
\VS{2}Peuples battez tous des mains ! Poussez vers Dieu des cris de joie avec une voix de triomphe !
\VS{3}Car Yahweh, le Très-Haut, est terrible. Il est un grand Roi sur toute la terre.
\VS{4}Il nous assujettit des peuples et des nations sous nos pieds.
\VS{5}Il nous choisit notre héritage, la gloire de Jacob qu'il aime. Sélah.
\VS{6}Dieu est monté avec un cri de réjouissance, Yahweh monte au son du shofar.
\VS{7}Chantez à Dieu, chantez ! Chantez à notre Roi, chantez !
\VS{8}Car Dieu est le Roi de toute la terre : Chantez un cantique !
\VS{9}Dieu règne sur les nations, Dieu est assis sur son saint trône.
\VS{10}Les princes des peuples se rassemblent vers le peuple du Dieu d'Abraham, car les boucliers de la terre sont à Dieu : Il est puissamment élevé.
\Chap{48}
\TextTitle{Sion, splendeur du grand Roi}
\VerseOne{}Cantique. Psaume des fils de Koré.
\VS{2}Yahweh est grand, il est l'objet de toutes les louanges dans la ville de notre Dieu, sur sa montagne sainte.
\VS{3}Belle est la colline, joie de toute la terre, la montagne de Sion, le côté nord, c'est la ville du grand Roi.
\VS{4}Dieu est connu dans ses palais pour une haute retraite.
\VS{5}Ils l'ont vue, et aussitôt ils ont été émerveillés ; ils ont été troublés et se sont enfuis à la hâte.
\VS{6}Ils ont regardé tout stupéfaits, ils ont eu peur et ont pris la fuite.
\VS{7}Là un tremblement les a saisis, une douleur comme celle de l'enfantement\FTNT{Es. 13:8.}.
\VS{8}Ils ont été chassés comme par le vent d'orient qui brise les navires de Tarsis.
\VS{9}Comme nous l'avions entendu, ainsi l'avons-nous vu dans la ville de Yahweh des armées, dans la ville de notre Dieu : Dieu l'établira à toujours. Sélah.
\VS{10}Ô Dieu ! Nous pensons à ta bonté au milieu de ton temple.
\VS{11}Ô Dieu ! Comme ton Nom, ta louange retentit jusqu'aux extrémités de la terre ; ta droite est pleine de justice.
\VS{12}La montagne de Sion se réjouit, et les filles de Juda sont dans la joie, à cause de tes jugements.
\VS{13}Entourez Sion, faites-en le tour, comptez ses tours.
\VS{14}Observez son rempart, examinez ses palais pour le raconter à la génération future.
\VS{15}Car ce Dieu-là est notre Dieu éternellement et à jamais ; il nous accompagnera jusqu'à la mort.
\Chap{49}
\TextTitle{Vanité des richesses terrestres}
\VerseOne{}Psaume des fils de Koré, au chef des chantres.
\VS{2}Vous tous peuples, entendez ceci, vous habitants du monde, prêtez l'oreille,
\VS{3}petits et grands, riches et pauvres !
\VS{4}Ma bouche prononcera des discours pleins de sagesse, et les pensées de mon cœur sont pleines de sens.
\VS{5}Je prête l'oreille aux sentences qui me sont inspirées, je poserai mes questions au son de la harpe.
\VS{6}Pourquoi craindrai-je au jour du malheur, quand l'iniquité de mes adversaires m'entoure ?
\VS{7}Ils mettent leur confiance dans leurs biens et se glorifient de l'abondance de leurs richesses.
\VS{8}Ils ne peuvent se racheter l'un l'autre ni donner à Dieu le prix du rachat\FTNT{Mt. 16:26 ; Mc. 8:36-37 ; Lu. 12:15-21.}.
\VS{9}Car le rachat de leur âme est trop considérable, et il ne se fera jamais ;
\VS{10}ils ne vivront pas toujours et n'éviteront pas la vue de la fosse.
\VS{11}Car on voit que les sages meurent, l'insensé et le stupide périssent également, et ils laissent à d'autres leurs biens\FTNT{Ec. 2:21 ; Ec. 6:2.}.
\VS{12}Leur intention est que leurs maisons durent éternellement, et que leurs habitations demeurent d'âge en âge, ils ont donné leurs noms à leurs terres.
\VS{13}Mais l'homme qui est en honneur n'a point de durée, il est semblable aux bêtes que l'on égorge.
\VS{14}Tel est leur chemin, leur folie, et ceux qui les suivent se plaisent à leurs discours. Sélah.
\VS{15}Ils seront mis dans le scheol comme des brebis, la mort en fait sa pâture, et au matin les hommes droits les foulent aux pieds, leur beau rocher s'use, le scheol est leur résidence\FTNT{Job. 24:19.}.
\VS{16}Mais Dieu rachètera mon âme du pouvoir du scheol, quand il m'enlèvera de sa captivité\FTNT{Ps. 68:19 ; Ep. 4:8-9.}. Sélah.
\VS{17}Ne crains point quand tu verras quelqu'un s'enrichir et quand les trésors de sa maison se multiplient.
\VS{18}Car lorsqu'il mourra, il n'emportera rien, ses trésors ne descendront point après lui\FTNT{Job. 27:16-19 ; 1 Ti. 6:7.}.
\VS{19}Il aura beau s'estimer heureux pendant sa vie, on aura beau te louer des jouissances que tu te donnes,
\VS{20}tu iras néanmoins au séjour de tes pères, qui jamais ne reverront la lumière.
\VS{21}L'homme qui est en honneur, qui n'a pas d'intelligence, est semblable aux bêtes que l'on égorge.
\Chap{50}
\TextTitle{Yahweh, le juste Juge}
\VerseOne{}Psaume d'Asaph. Le Dieu puissant, Dieu, Yahweh a parlé et il a appelé toute la terre, depuis le soleil levant jusqu'au soleil couchant.
\VS{2}De Sion, Dieu a fait luire sa splendeur qui est d'une beauté parfaite,
\VS{3}notre Dieu viendra, il ne se taira point : Il y aura devant lui un feu dévorant, et tout autour de lui une grosse tempête.
\VS{4}Il appellera les cieux d'en haut, et la terre pour juger son peuple :
\VS{5}Rassemblez-moi mes bien-aimés qui ont traité alliance avec moi par le sacrifice\FTNT{Mt. 24:29-31.}.
\VS{6}Les cieux aussi annonceront sa justice parce que Dieu est le juge. Sélah.
\VS{7}Ecoute, ô mon peuple ! Et je parlerai. Entends, Israël ! Et je t'avertirai. Moi je suis Dieu, ton Dieu.
\VS{8}Je ne te réprimande pas pour tes sacrifices, tes holocaustes sont continuellement devant moi.
\VS{9}Je ne prendrai point de taureaux de ta maison ni de boucs de tes bergeries\FTNT{Ps. 40:7.}.
\VS{10}Car tous les animaux des forêts sont à moi, toutes les bêtes qui paissent sur mille montagnes.
\VS{11}Je connais tous les oiseaux des montagnes, et tout ce qui se meut dans les champs m'appartient.
\VS{12}Si j'avais faim, je ne t'en dirais rien, car le monde est à moi et tout ce qu'il renferme.
\VS{13}Mangerais-je la chair des gros taureaux ? Et boirais-je le sang des boucs ?
\VS{14}Offre à Dieu la reconnaissance et accomplis tes vœux envers le Très-Haut.
\VS{15}Invoque-moi au jour de ta détresse, je te délivrerai, et tu me glorifieras\FTNT{Ps. 37:5.}.
\VS{16}Dieu dit au méchant : Quoi donc ? Tu énumères mes lois ! Et tu as mon alliance dans ta bouche !
\VS{17}Toi qui hais la correction, et qui jettes mes paroles derrière toi !
\VS{18}Si tu vois un voleur, tu te plais avec lui, et ta part est avec les adultères.
\VS{19}Tu livres ta bouche au mal, et ta langue est un tissu de tromperies.
\VS{20}Tu t'assieds et parles contre ton frère, tu couvres d'opprobre le fils de ta mère.
\VS{21}Tu as fait ces choses-là, et je me suis tu. Tu as estimé que je te ressemble, mais je vais te reprendre et tout mettre sous tes yeux.
\VS{22}Comprenez cela maintenant, vous qui oubliez Dieu, de peur que je ne déchire sans que personne ne vous délivre.
\VS{23}Celui qui offre la louange me glorifie, et à celui qui veille sur sa voie, je lui montrerai le salut de Dieu.
\Chap{51}
\TextTitle{Le cœur repentant, sacrifice agréable à Dieu}
\VerseOne{}Psaume de David, au chef des chantres.
\VS{2}Lorsque Nathan le prophète vint à lui, après que David fut allé vers Bath-Schéba\FTNT{2 S. 11 ; 2 S. 12.}.
\VS{3}Ô Dieu ! Aie pitié de moi dans ta bonté, selon ta grande miséricorde, efface mes transgressions ;
\VS{4}lave-moi parfaitement de mon iniquité et purifie-moi de mon péché.
\VS{5}Car je reconnais mes transgressions, et mon péché est continuellement devant moi\FTNT{Es. 59:12.}.
\VS{6}J'ai péché contre toi, contre toi seul, et j'ai fait ce qui déplaît à tes yeux : En sorte que tu seras juste dans ta sentence, sans reproche dans ton jugement.
\VS{7}Voici, je suis né dans l'iniquité, et ma mère m'a conçu dans le péché.
\VS{8}Mais tu prends plaisir à la vérité au fond du cœur, et tu me fais connaître la sagesse au-dedans de moi.
\VS{9}Purifie-moi de mon péché avec de l'hysope, et je serai pur ; lave-moi, et je serai plus blanc que la neige.
\VS{10}Fais-moi entendre la joie et l'allégresse, et les os que tu as brisés se réjouiront.
\VS{11}Détourne ta face de mes péchés, et efface toutes mes iniquités.
\VS{12}Ô Dieu ! Crée en moi un cœur pur et renouvelle en moi un esprit ferme\FTNT{Mt. 5:8.}.
\VS{13}Ne me rejette pas loin de ta face et ne m'ôte pas ton Esprit Saint.
\VS{14}Rends-moi la joie de ton salut et qu'un esprit bien disposé me soutienne.
\VS{15}J'enseignerai tes voies aux transgresseurs et les pécheurs reviendront à toi.
\VS{16}Ô Dieu, Dieu de mon salut ! Délivre-moi de tant de sang, et ma langue chantera hautement ta justice.
\VS{17}Seigneur, ouvre mes lèvres, et ma bouche annoncera ta louange.
\VS{18}Car tu ne prends point plaisir aux sacrifices, autrement je t'en donnerais ; l'holocauste ne t'est point agréable.
\VS{19}Les sacrifices à Dieu, c'est un esprit brisé. Ô Dieu ! Tu ne méprises point un cœur brisé et contrit.
\VS{20}Répands par ta grâce, tes bienfaits sur Sion, édifie les murs de Jérusalem.
\VS{21}Alors tu prendras plaisir aux sacrifices de justice, à l'holocauste, et aux sacrifices qui se consument entièrement par le feu ; alors on offrira des taureaux sur ton autel.
\Chap{52}
\TextTitle{Sort de l'homme qui se confie en ses richesses}
\VerseOne{}Cantique de David, donné au chef des chantres.
\VS{2}A l'occasion du rapport que Doëg, l'Edomite, vint faire à Saül, en lui disant : David s'est rendu dans la maison d'Achimélec.
\VS{3}Pourquoi te vantes-tu du mal, vaillant homme ? La bonté de Dieu dure à toujours.
\VS{4}Ta langue trame des méchancetés, elle est comme un rasoir affilé qui trompe.
\VS{5}Tu aimes plus le mal que le bien, le mensonge plutôt que de dire la vérité. Sélah.
\VS{6}Tu aimes tous les discours pernicieux, le langage trompeur.
\VS{7}Aussi Dieu te détruira pour toujours, il t'enlèvera et t'arrachera de ta tente ; il te déracinera de la terre des vivants. Sélah.
\VS{8}Les justes le verront et auront de la crainte, et ils se riront d'un tel homme, disant :
\VS{9}Voilà cet homme qui ne tenait point Dieu pour sa protection, mais qui se confiait en ses grandes richesses et qui mettait sa force dans ses mauvais désirs\FTNT{Es. 47:10 ; Lu. 12:15-21.}.
\VS{10}Mais moi, je serai dans la maison de Dieu comme un olivier verdoyant. Je me confie dans la bonté de Dieu pour toujours et à jamais.
\VS{11}Je te célébrerai à jamais, car tu agis ; et je mettrai mon espérance en ton Nom, parce qu'il est bon envers tes fidèles.
\Chap{53}
\TextTitle{Egarement des impies}
\VerseOne{}Cantique de David, donné au chef des chantres, pour le chanter sur la flûte.
\VS{2}L'insensé dit en son cœur : Il n'y a point de Dieu ! Ils se sont corrompus, ils ont commis des injustices abominables ; il n'y a personne qui fasse le bien\FTNT{Ps. 10:4 ; Ro. 1:20-21 ; Ro. 3:12.}.
\VS{3}Dieu a regardé des cieux les fils des hommes, pour voir s'il y a quelqu'un qui soit intelligent, qui cherche Dieu.
\VS{4}Ils se sont tous détournés et se sont tous rendus odieux. Il n'y a personne qui fasse le bien, pas même un seul.
\VS{5}Les ouvriers d'iniquité n'ont-ils point de connaissance ? Ils mangent mon peuple comme s'ils mangeaient du pain. Ils n'invoquent point Dieu.
\VS{6}Ils seront épouvantés sans qu'il y ait sujet d'épouvante, car Dieu a dispersé les os de celui qui campe contre toi. Tu les confondras, car Dieu les a rejetés.
\VS{7}Oh ! Qui fera partir de Sion les délivrances d'Israël ? Quand Dieu aura ramené son peuple captif, Jacob s'égayera, Israël se réjouira.
\Chap{54}
\TextTitle{La délivrance vient de Yahweh}
\VerseOne{}Cantique de David, donné au chef des chantres, pour le chanter avec instruments à cordes.
\VS{2}Lorsque les Ziphiens vinrent dire à Saül : David n'est-il pas caché parmi nous\FTNT{1 S. 23:19 ; 1 S. 26:1.} ?
\VS{3}Ô Dieu ! Délivre-moi par ton Nom et fais-moi justice par ta puissance.
\VS{4}Ô Dieu ! Ecoute ma prière, prête l'oreille aux paroles de ma bouche !
\VS{5}Car des étrangers se sont élevés contre moi, et des gens terribles qui ne mettent pas Dieu devant eux en veulent à ma vie. Sélah.
\VS{6}Voilà, Dieu m'accorde son secours, le Seigneur est de ceux qui soutiennent mon âme.
\VS{7}Il fera retourner le mal sur ceux qui m'épient ; détruis-les selon ta vérité.
\VS{8}Je t'offrirai de bon cœur des sacrifices ; Yahweh ! Je célébrerai ton Nom parce qu'il est bon.
\VS{9}Car il m'a délivré de toute détresse ; et mes yeux se réjouissent à la vue de mes ennemis.
\Chap{55}
\TextTitle{Se garder des méchants}
\VerseOne{}Cantique de David, donné au chef des chantres, pour le chanter avec instruments à cordes.
\VS{2}Ô Dieu ! Prête l'oreille à ma prière et ne te cache pas de mes supplications !
\VS{3}Ecoute-moi et réponds-moi ! J'erre çà et là dans ma méditation et je suis agité
\VS{4}à cause du bruit que fait l'ennemi, à cause de l'oppression du méchant ; car ils font tomber sur moi les outrages, et ils me haïssent jusqu'à la fureur.
\VS{5}Mon cœur tremble au-dedans de moi et les terreurs de la mort tombent sur moi.
\VS{6}La crainte et l'épouvante m'atteignent et le frisson m'habille.
\VS{7}Je dis : Qui me donnera des ailes de colombe ? Je m'envolerais et je trouverais ma demeure.
\VS{8}Voilà, je m'enfuirais bien loin et je me tiendrais au désert. Sélah.
\VS{9}Je m'échapperais en toute hâte, plus rapide que le vent impétueux, que la tempête.
\VS{10}Seigneur, réduis à néant, divise leur langue, car j'ai vu la violence et les querelles dans la ville.
\VS{11}Elles font jour et nuit le tour sur les murailles ; l'iniquité et la malice sont dans son sein.
\VS{12}Les calamités sont au milieu d'elle, et la tromperie et la fraude ne partent point de ses places.
\VS{13}Car ce n'est pas mon ennemi qui m'a diffamé, je le supporterais ; ce n'est point celui qui m'a en haine qui s'élève contre moi, je me cacherais de lui.
\VS{14}Mais c'est toi, ô homme ! Que j'estimais mon égal, mon confident et mon ami\FTNT{Ps. 41:10.} !
\VS{15}Nous prenions plaisir à communiquer nos secrets ensemble, nous allions avec la multitude dans la maison de Dieu.
\VS{16}Que la mort les séduise ! Qu'ils descendent vivants dans le scheol ! Car le mal est dans leur demeure, parmi eux dans leur assemblée.
\VS{17}Mais moi je crie à Dieu, et Yahweh me délivrera.
\VS{18}Le soir, le matin, et à midi je me plains et je gémis, et il entendra ma voix.
\VS{19}Il délivrera mon âme de la guerre et me rendra la paix ; car ils sont nombreux contre moi.
\VS{20}Dieu entendra et témoignera en ma faveur. Lui qui de toute éternité est assis sur son trône. Sélah. Car il n'y a point de changement en eux, et ils ne craignent point Dieu.
\VS{21}Chacun d'eux porte la main sur ceux qui vivaient en paix avec lui, et viole son alliance.
\VS{22}Les paroles de sa bouche sont plus douces que la crème, mais la guerre est dans son cœur ; ses paroles sont plus douces que l'huile, néanmoins elles sont tout autant d'épées nues.
\VS{23}Remets ton sort à Yahweh et il te soulagera, il ne permettra jamais que le juste tombe.
\VS{24}Mais toi, ô Dieu ! Tu les précipiteras au puits de la perdition ; les hommes sanguinaires et trompeurs ne parviendront point à la moitié de leurs jours. C'est en toi que je me confie.
\Chap{56}
\TextTitle{Se glorifier en la Parole de Yahweh}
\VerseOne{}Hymne de David, donné au chef des chantres, pour le chanter sur « Colombe des térébinthes lointains ». Lorsque les Philistins le saisirent à Gath\FTNT{1 S. 21:10-14.}.
\VS{2}Dieu ! Aie pitié de moi, car des hommes m'écrasent et m'oppriment, me faisant tout le jour la guerre, ils m'oppressent.
\VS{3}Mes adversaires me piétinent tout le jour ; car, ô Très-Haut, plusieurs me font la guerre comme des hautains.
\VS{4}Le jour où j'aurai peur, je me confierai en toi.
\VS{5}Je me glorifierai en Dieu, en sa parole ; je me confie en Dieu, je ne craindrai rien. Que peuvent me faire les hommes\FTNT{Ps. 118:6 ; Hé. 13:6.} ?
\VS{6}Tout le jour ils tordent mes propos, et toutes leurs pensées tendent à me nuire.
\VS{7}Ils s'assemblent, ils se tiennent cachés, ils observent mes pas, s'attendant à m'ôter la vie.
\VS{8}C'est par l'iniquité qu'ils espèrent échapper. Dans ta colère, ô Dieu, précipite les peuples !
\VS{9}Tu comptes mes allées et venues ; recueille mes larmes dans tes outres : Ne sont-elles pas écrites dans ton livre ?
\VS{10}Le jour où je crierai à toi, mes ennemis reculeront ; je sais que Dieu est pour moi.
\VS{11}Je me glorifierai en Dieu, en sa parole, je me glorifierai en Yahweh, en sa parole.
\VS{12}Je me confie en Dieu, je ne craindrai rien : Que me fera l'homme ?
\VS{13}Ô Dieu ! Les vœux que je t'ai fait s'accompliront, je te louerai.
\VS{14}Car tu as délivré mon âme de la mort, tu as garanti mes pieds de la chute, afin que je marche devant Dieu, à la lumière des vivants.
\Chap{57}
\TextTitle{Avoir confiance en Dieu dans les difficultés}
\VerseOne{}Hymne de David, donné au chef des chantres, pour le chanter sur Al-Thasheth\FTNT{Al-Thasheth signifie « Ne détruis pas »}. Lorsqu'il se réfugia dans la caverne, poursuivi par Saül\FTNT{1 S. 22:1.}.
\VS{2}Aie pitié de moi, ô Dieu, aie pitié de moi ! Car mon âme cherche un refuge ; je cherche un refuge à l'ombre de tes ailes, jusqu'à ce que les calamités soient passées\FTNT{Ps. 17:8.}.
\VS{3}Je crie au Dieu Très-Haut, au Dieu qui accomplit son œuvre pour moi.
\VS{4}Il m'enverra des cieux la délivrance, il rendra honteux celui qui veut me dévorer. Sélah. Dieu enverra sa bonté et sa vérité.
\VS{5}Mon âme est parmi des lions ; je suis couché au milieu de gens qui vomissent la flamme, parmi des hommes dont les dents sont des lances et des flèches, et dont la langue est une épée aiguë\FTNT{Ps. 59:8 ; Ps. 64:4 ; Ja. 3:5-12.}.
\VS{6}Ô Dieu, élève-toi sur les cieux ! Que ta gloire soit sur toute la terre !
\VS{7}Ils avaient tendu un filet sous mes pas : Mon âme se courbait. Ils avaient creusé une fosse devant moi, mais ils y sont tombés. Sélah.
\VS{8}Mon cœur est affermi, ô Dieu ! Mon cœur est affermi, je chanterai et je ferai retentir mes instruments.
\VS{9}Réveille-toi ma gloire ! Réveillez-vous mon luth et ma harpe ! Je me réveillerai à l'aube du jour.
\VS{10}Seigneur, je te célébrerai parmi les peuples, je te chanterai parmi les nations.
\VS{11}Car ta bonté est grande jusqu'aux cieux, et ta vérité jusqu'aux nues\FTNT{Ps. 118:4-5.}.
\VS{12}Ô Dieu ! Elève-toi sur les cieux ! Que ta gloire soit sur toute la terre !
\Chap{58}
\TextTitle{Yahweh rend justice sur la terre}
\VerseOne{}Hymne de David, donné au chef des chantres, pour le chanter sur Al-Thasheth « Ne détruis pas ».
\VS{2}En vérité, vous, gens de l'assemblée, prononcez-vous ce qui est juste ? Vous, fils des hommes, jugez-vous avec droiture ?
\VS{3}Au contraire, vous tramez des injustices dans votre cœur. Sur la terre, c'est la violence de vos mains que vous placez sur la balance.
\VS{4}Les méchants se sont égarés dès le sein maternel, ils ont erré dès le ventre de leur mère, en parlant faussement.
\VS{5}Ils ont un venin semblable au venin du serpent, ils sont comme l'aspic sourd, qui ferme son oreille,
\VS{6}qui n'entend pas la voix des enchanteurs, du magicien le plus sage.
\VS{7}Ô Dieu, brise-leur les dents dans leur bouche ! Yahweh, brise les mâchoires des lionceaux !
\VS{8}Qu'ils s'écoulent comme de l'eau, et qu'ils se fondent ! Que chacun d'eux bande son arc, mais que ses flèches soient comme si elles étaient rompues !
\VS{9}Qu'ils s'en aillent comme un limaçon qui se fond ! Qu'ils ne voient point le soleil comme l'avorton d'une femme !
\VS{10}Avant que vos chaudières aient senti le feu des épines, l'ardeur de la colère, semblable à un tourbillon, enlèvera chacun d'eux comme de la chair crue..
\VS{11}Le juste se réjouira quand il aura vu la vengeance, il lavera ses pieds avec le sang du méchant.
\VS{12}Et chacun dira : Quoi qu'il en soit, il y a une récompense pour le juste ; quoi qu'il en soit, il y a un Dieu qui juge sur la terre.
\Chap{59}
\TextTitle{Intervention divine}
\VerseOne{}Hymne de David, donné au chef des chantres, pour le chanter sur Al-Thasheth « Ne détruis pas ». Lorsque Saül envoya des gens qui épièrent sa maison afin de le tuer\FTNT{1 S. 19:11.}.
\VS{2}Mon Dieu ! Délivre-moi de mes ennemis, protège-moi de ceux qui s'élèvent contre moi !
\VS{3}Délivre-moi des ouvriers d'iniquité et garde-moi des hommes sanguinaires !
\VS{4}Car voici, ils m'ont dressé des embûches, et des gens robustes se sont assemblés contre moi, bien qu'il n'y ait point en moi de transgression ni de péché, ô Yahweh !
\VS{5}Ils courent çà et là, et se mettent en ordre, bien qu'il n'y ait point d'iniquité en moi. Réveille-toi pour venir au-devant de moi ! Et regarde !
\VS{6}Toi donc, ô Yahweh ! Dieu des armées, Dieu d'Israël, réveille-toi pour visiter toutes les nations ! Ne fais point de grâce à aucun de ceux qui me trahissent ! Sélah.
\VS{7}Ils reviennent chaque soir, ils hurlent comme des chiens, ils font le tour de la ville.
\VS{8}Voici, de leur bouche ils font jaillir le mal, il y a des épées sur leurs lèvres\FTNT{Ja. 3:5-12.} ; car, disent-ils, qui nous entend ?
\VS{9}Mais toi, Yahweh ! Tu te riras d'eux, tu te moqueras de toutes les nations\FTNT{Ps. 2:4.}.
\VS{10}Quelle que soit leur force, je m'attends à toi, car Dieu est ma haute retraite.
\VS{11}Dieu qui me favorise me préviendra, Dieu me fera voir mes adversaires\FTNT{Ps. 118:7.}.
\VS{12}Ne les tue pas, de peur que mon peuple ne l'oublie ; fais-les errer par ta puissance et abats-les ; Seigneur, notre bouclier !
\VS{13}Leur bouche pèche à chaque parole de leurs lèvres ; qu'ils soient pris par leur orgueil ! Ils ne tiennent que des discours de malédiction et de mensonge.
\VS{14}Consume-les avec fureur, consume-les de sorte qu'ils ne soient plus ! Qu'on sache que Dieu domine sur Jacob et jusqu'aux extrémités de la terre ! Sélah.
\VS{15}Qu'ils reviennent le soir, et qu'ils hurlent comme des chiens, et qu'ils fassent le tour de la ville.
\VS{16}Qu'ils errent çà et là cherchant leur nourriture, et qu'ils passent la nuit sans être rassasiés.
\VS{17}Mais moi je chanterai ta force, je louerai dès le matin à haute voix ta bonté\FTNT{Ps. 88:14.}. Car tu es pour moi une haute retraite, et mon asile au jour de ma détresse.
\VS{18}Ma force ! Je te chanterai ; car Dieu est ma haute retraite, le Dieu qui me favorise.
\Chap{60}
\TextTitle{Yahweh, le meilleur secours}
\VerseOne{}Hymne de David, pour enseigner, donné au chef des chantres, pour le chanter sur le lis lyrique,
\VS{2}lorsqu'il fit la guerre contre les Syriens de Mésopotamie, et contre les Syriens de Tsoba, et que Joab revint et défit douze mille Edomites dans la vallée du sel\FTNT{2 S. 8:3-13 ; 1 Ch. 18:3-12.}.
\VS{3}Ô Dieu ! Tu nous as rejetés, tu nous as dispersés, tu t'es irrité : Reviens vers nous !
\VS{4}Tu as ébranlé la terre et l'as mise en pièces ; répare ses brèches, car elle chancelle !
\VS{5}Tu as fait voir à ton peuple des choses dures, tu nous as abreuvés d'un vin d'étourdissement\FTNT{Es. 51:17-21 ; Ap. 14:10.}.
\VS{6}Mais tu as donné une bannière à ceux qui te craignent, afin de l'élever bien haut pour l'amour de ta vérité. Sélah.
\VS{7}Afin que ceux que tu aimes soient délivrés ; sauve-moi par ta droite et exauce-moi\FTNT{Ps. 108:6.}.
\VS{8}Dieu a parlé dans son lieu saint : Je me réjouirai, je partagerai Sichem, je mesurerai la vallée de Succoth ;
\VS{9}Galaad est à moi, Manassé aussi est à moi, et Ephraïm est la protection de ma tête et Juda mon sceptre.
\VS{10}Moab est le bassin où je me lave ; je jette mon soulier sur Edom ; pays des Philistins, pousse des cris de guerre à mon sujet\FTNT{2 S. 8:2 ; 1 Ch. 18:2.}.
\VS{11}Qui me conduira dans la ville forte ? Qui me conduira jusqu'en Edom ?
\VS{12}Ne sera-ce pas toi, ô Dieu, qui nous avais rejetés, et qui ne sortais plus, ô Dieu, avec nos armées ?
\VS{13}Donne-nous du secours pour sortir de la détresse ! Car la délivrance qu'on attend de l'homme est vanité\FTNT{Jé. 17:5 ; Ps. 118:8.}.
\VS{14}Avec le secours de Dieu, nous ferons des exploits, et il foulera nos ennemis.
\Chap{61}
\TextTitle{Dieu, le parfait Refuge}
\VerseOne{}Psaume de David, donné au chef des chantres, pour le chanter sur instruments à cordes.
\VS{2}Ô Dieu, je crie à toi, sois attentif à ma prière !
\VS{3}Je crie à toi du bout de la terre, le cœur abattu ; conduis-moi sur le rocher qui est trop haut pour moi !
\VS{4}Car tu es mon refuge, une tour forte au-devant de l'ennemi.
\VS{5}Je séjournerai éternellement dans ta tente, je me retirerai à l'ombre de tes ailes. Sélah.
\VS{6}Car, ô Dieu ! Tu exauces mes vœux, et tu me donnes l'héritage de ceux qui craignent ton Nom.
\VS{7}Tu ajoutes des jours aux jours du roi ; que ses années se prolongent à jamais !
\VS{8}Qu'il demeure toujours dans la présence de Dieu ! Que la bonté et la vérité le gardent !
\VS{9}Ainsi je chanterai ton Nom à perpétuité, en rendant mes vœux chaque jour.
\Chap{62}
\TextTitle{La confiance en Dieu}
\VerseOne{}Psaume de David, donné au chef des chantres, d'après Jeduthun.
\VS{2}Quoiqu'il en soit, mon âme se repose en Dieu ; c'est de lui que vient ma délivrance.
\VS{3}Quoiqu'il en soit, il est mon rocher, ma délivrance, et ma haute retraite ; je ne serai pas entièrement ébranlé.
\VS{4}Jusqu'à quand accablerez-vous de maux un homme ? Vous serez tous mis à mort, et vous serez comme le mur qui penche, comme une cloison qui a été ébranlée.
\VS{5}Ils ne font que consulter pour le faire déchoir de son élévation ; ils prennent plaisir au mensonge ; ils bénissent de leur bouche, mais au-dedans ils maudissent. Sélah.
\VS{6}Mais toi mon âme, demeure tranquille, regarde à Dieu, car mon espérance est en lui.
\VS{7}Quoiqu'il en soit, il est mon rocher, ma délivrance, et ma haute retraite ; je ne serai point ébranlé.
\VS{8}En Dieu est ma délivrance et ma gloire ; en Dieu est le rocher de ma force et ma retraite.
\VS{9}Peuples, confiez-vous en lui en tout temps, déchargez votre cœur sur lui ! Dieu est notre retraite. Sélah.
\VS{10}Oui, vanité, les fils de l'homme ! Mensonge, les fils de l'homme ! Dans une balance, ils monteraient tous ensemble, plus légers qu'un souffle.
\VS{11}Ne vous confiez pas dans la violence ni dans la rapine ; ne devenez point vains ; quand les richesses abonderont, n'y mettez point votre cœur.
\VS{12}Dieu a parlé une fois, j'ai entendu cela deux fois : C'est que la force est à Dieu.
\VS{13}Et c'est à toi, Seigneur, qu'appartient la bonté ; certainement tu rendras à chacun selon son œuvre\FTNT{Jé. 32:19 ; Pr. 24:12 ; Job. 34:11 ; Mt. 16:27 ; Ro. 2:6.}.
\Chap{63}
\TextTitle{Soif de la présence de Dieu}
\VerseOne{}Psaume de David, lorsqu'il était dans le désert de Juda\FTNT{1 S. 22:5 ; 1 S. 23:14-15.}.
\VS{2}Ô Dieu ! Tu es mon Dieu, je te cherche au point du jour ; mon âme a soif de toi, mon corps soupire après toi sur cette terre aride, desséchée, et sans eau\FTNT{Ps. 42:2 ; Ps. 84:3 ; Ps. 143:6.}.
\VS{3}Ainsi je te contemple dans ton lieu saint pour voir ta force et ta gloire.
\VS{4}Car ta bonté vaut mieux que la vie, mes lèvres te louent.
\VS{5}Et ainsi je te bénirai donc toute ma vie\FTNT{Ps. 104:33.}, j'élèverai mes mains en ton Nom.
\VS{6}Mon âme est rassasiée comme de mets gras et succulents, et ma bouche te loue avec un chant de réjouissance.
\VS{7}Quand je me souviens de toi dans mon lit, je médite sur toi durant les veilles de la nuit\FTNT{Ps. 16:7 ; Ps. 119:55.}.
\VS{8}Car tu m'as secouru, je me réjouirai à l'ombre de tes ailes.
\VS{9}Mon âme s'est attachée à toi pour te suivre, ta droite me soutient.
\VS{10}Mais ceux-ci qui demandent que mon âme tombe en ruine, entreront au plus bas de la terre.
\VS{11}On les détruira à coups d'épée, ils seront la proie des chacals.
\VS{12}Mais le roi se réjouira en Dieu ; quiconque jure par lui s'en glorifiera, car la bouche de ceux qui mentent sera fermée\FTNT{Ps. 107:42 ; Job. 5:16.}.
\Chap{64}
\TextTitle{Yahweh, le seul abri}
\VerseOne{}Psaume de David, donné au chef des chantres.
\VS{2}Ô Dieu ! Ecoute ma voix quand je m'écrie. Protège ma vie contre l'ennemi que je crains !
\VS{3}Cache-moi des complots des méchants, de l'assemblée tumultueuse des ouvriers d'iniquité !
\VS{4}Ils aiguisent leur langue comme une épée\FTNT{Jé. 9:3 ; Ps. 11:2 ; Ps. 59:8.}, ils tirent comme des flèches leurs paroles amères,
\VS{5}afin de tirer sur l'innocent dans sa cachette ; ils tirent soudainement sur lui et n'ont aucune crainte.
\VS{6}Ils se fortifient dans leur méchanceté, tiennent des discours pour tendre des pièges, ils disent : Qui les verra\FTNT{Job. 24:15.} ?
\VS{7}Ils cherchent curieusement des méchancetés ; ils ont sondé tout ce qui se peut sonder, même ce qui peut être au–dedans de l'homme, et au cœur le plus profond.
\VS{8}Mais Dieu lance contre eux ses traits, soudain les voilà frappés.
\VS{9}Leur langue a causé leur chute ; tous ceux qui les voient secouent leur tête.
\VS{10}Et tous les hommes craindront et raconteront l'œuvre de Dieu, et considéreront ce qu'il aura fait.
\VS{11}Le juste se réjouira en Yahweh, et se retirera vers lui, et tous ceux qui sont droits de cœur s'en glorifieront\FTNT{Ps. 63:12 ; Ps. 97:12.}.
\Chap{65}
\TextTitle{Le règne de Yahweh sur la nature}
\VerseOne{}Psaume de David. Cantique. Donné au chef des chantres.
\VS{2}Ô Dieu ! Dans le calme, on te louera dans Sion, et l'on accomplira nos vœux\FTNT{Ps. 50:14 ; Ps. 66:13.}.
\VS{3}Tu entends nos prières, toute chair viendra jusqu'à toi.
\VS{4}Les iniquités prévalent sur moi, mais tu feras la propitiation de nos transgressions.
\VS{5}Heureux celui que tu choisis et que tu admets dans ta présence pour qu'il habite dans tes parvis ! Nous serons rassasiés des biens de ta maison, des biens du saint lieu de ton temple.
\VS{6}Dans ta justice, tu nous réponds par des choses terribles, ô Dieu de notre salut, espoir de toutes les extrémités lointaines de la terre et de la mer.
\VS{7}Il affermit les montagnes par sa force, il est ceint de puissance.
\VS{8}Il apaise le mugissement de la mer, le mugissement de leurs flots, et le tumulte des peuples.
\VS{9}Ceux qui habitent aux extrémités de la terre ont peur de tes prodiges ; tu réjouis l'orient et l'occident.
\VS{10}Tu visites la terre, tu lui donnes l'abondance, tu la combles de richesses ; le ruisseau de Dieu est plein d'eau ; tu prépares le blé, quand tu l'établis ainsi.
\VS{11}Tu arroses ses sillons, et tu aplanis ses mottes ; tu l'amollis par la pluie, et tu bénis son germe\FTNT{Es. 55:10 ; Ps. 104:13-14.}.
\VS{12}Tu couronnes l'année de tes biens, et tes voies versent l'abondance.
\VS{13}Les plaines du désert sont abreuvées et les collines sont ceintes de joie.
\VS{14}Les pâturages se couvrent de brebis, et les vallées se revêtent de froments ; les cris de joie et les chants retentissent.
\Chap{66}
\TextTitle{Louange au Dieu de grâces}
\VerseOne{}Cantique. Psaume, donné au chef des chantres. Vous tous habitants de toute la terre, poussez des cris de triomphe à Dieu.
\VS{2}Chantez la gloire de son Nom, faites éclater sa gloire par vos louanges.
\VS{3}Dites à Dieu : Que tes œuvres sont redoutables ! Tes ennemis te mentiront à cause de la grandeur de ta force.
\VS{4}Toute la terre se prosterne devant toi et te chante ; elle chante ton Nom. Sélah.
\VS{5}Venez et voyez les œuvres de Dieu : Il est redoutable quand il agit sur les fils des hommes.
\VS{6}Il a fait de la mer une terre sèche ; on a passé le fleuve à pied sec ; là, nous nous sommes réjouis en lui\FTNT{Ex. 14:21 ; Jos. 3:14-17.}.
\VS{7}Il domine par sa puissance éternellement ; ses yeux prennent garde sur les nations\FTNT{Ps. 14:2 ; Ps. 33:13 ; Job. 28:24.} ; les rebelles ne pourront point s'élever. Sélah.
\VS{8}Peuples, bénissez notre Dieu, et faites retentir le son de sa louange.
\VS{9}C'est lui qui a remis notre âme en vie, et qui n'a point permis que nos pieds chancellent.
\VS{10}Car, ô Dieu, tu nous as éprouvés ! Tu nous a fait passer au creuset comme l'argent.
\VS{11}Tu nous as amenés dans le filet, tu as mis sur nos reins un pesant fardeau.
\VS{12}Tu as fait monter des hommes sur notre tête, et nous avons passé par le feu et par l'eau. Mais tu nous as fait entrer dans un lieu d'abondance.
\VS{13}J'entrerai dans ta maison avec des holocaustes, j'accomplirai mes vœux envers toi\FTNT{Ps. 22:26 ; Ps. 76:12 ; Ps. 116:14.}.
\VS{14}Pour eux, mes lèvres se sont ouvertes et ma bouche les a prononcés dans ma détresse.
\VS{15}Je t'offrirai en holocauste des brebis grasses, avec la graisse des béliers, je te sacrifierai des taureaux et des boucs. Sélah.
\VS{16}Vous tous qui craignez Dieu, venez, écoutez, et je raconterai ce qu'il a fait à mon âme.
\VS{17}Je l'ai invoqué de ma bouche, et la louange a été sur ma langue.
\VS{18}Si j'avais conçu l'iniquité dans mon cœur, le Seigneur ne m'aurait pas écouté\FTNT{Jn. 9:31.}.
\VS{19}Mais certainement Dieu m'a écouté, il a été attentif à la voix de ma prière.
\VS{20}Béni soit Dieu qui n'a point rejeté ma prière, et qui n'a point éloigné de moi sa bonté.
\Chap{67}
\TextTitle{Louange des peuples}
\VerseOne{}Psaume. Cantique donné au chef des chantres, pour le chanter avec instruments à cordes.
\VS{2}Que Dieu ait pitié de nous et qu'il nous bénisse, qu'il fasse luire sa face sur nous\FTNT{No. 6:25 ; Ps. 4:7 ; Ps. 31:17 ; Ps. 119:135.}. Sélah.
\VS{3}Afin que ta voie soit connue sur la terre et ta délivrance parmi toutes les nations.
\VS{4}Les peuples te célébreront, ô Dieu ! Tous les peuples te célébreront\FTNT{Ps. 22:27 ; Ps. 68:33.} !
\VS{5}Les peuples se réjouissent et chantent de joie, car tu juges les peuples avec droiture et tu conduis les nations sur la terre\FTNT{Ps. 96:10.}. Sélah.
\VS{6}Les peuples te célébreront, ô Dieu ! Tous les peuples te célébreront !
\VS{7}La terre produira son fruit ; Dieu, notre Dieu, nous bénira.
\VS{8}Dieu nous bénira, et toutes les extrémités de la terre le craindront.
\Chap{68}
\TextTitle{Yahweh, le Dieu glorieux}
\VerseOne{}Psaume. Cantique de David, donné au chef des chantres.
\VS{2}Que Dieu se lève, et ses ennemis seront dispersés, et ceux qui le haïssent s'enfuiront devant lui\FTNT{No. 10:35.}.
\VS{3}Tu les chasseras comme la fumée est chassée par le vent ; comme la cire se fond devant le feu, ainsi les méchants périront devant Dieu\FTNT{Ps. 37:20 ; Ps. 97:5.}.
\VS{4}Mais les justes se réjouiront et s'égayeront devant Dieu, et tressailliront de joie\FTNT{Ps. 67:4-5.}.
\VS{5}Chantez à Dieu, célébrez son Nom ! Exaltez celui qui est monté sur les cieux ! Son Nom est Yahweh ! Réjouissez-vous dans sa présence.
\VS{6}Il est le père des orphelins et le juge des veuves ; Dieu est dans sa demeure sainte\FTNT{Ps. 146:9}.
\VS{7}Dieu donne une famille à ceux qui étaient abandonnés, il délivre ceux qui étaient enchaînés, mais les rebelles habitent sur une terre déserte.
\VS{8}Ô Dieu ! Quand tu sortis devant ton peuple, quand tu marchais dans le désert ! Sélah.
\VS{9}La terre trembla et les cieux répandirent leurs eaux à cause de la présence de Dieu, le mont Sinaï trembla à cause de la présence de Dieu, du Dieu d'Israël\FTNT{Ex. 19:18 ; Jg. 5:5.}.
\VS{10}Ô Dieu ! Tu as fait tomber une pluie abondante sur ton héritage, et quand il était épuisé, tu l'as rétabli.
\VS{11}Ton troupeau établit sa demeure dans le pays, que par ta bonté tu avais préparé pour les malheureux, ô Dieu !
\VS{12}Le Seigneur donne une parole, et les messagères de bonnes nouvelles sont une grande armée.
\VS{13}Les rois des armées se sont enfuis, ils se sont enfuis, et celle qui se tenait à la maison a partagé le butin\FTNT{1 S. 30:16.}.
\VS{14}Tandis que vous vous couchez dans les étables, les ailes de la colombe sont couvertes d'argent, et son plumage est d'un jaune d'or.
\VS{15}Quand le Tout-Puissant dispersa les rois dans le pays, il devint blanc comme la neige du Tsalmon.
\VS{16}La montagne de Dieu est un mont de Basan ; une montagne élevée, un mont de Basan.
\VS{17}Pourquoi l'insultez-vous, montagnes dont le sommet est élevé ? Dieu a désiré cette montagne pour y habiter, et Yahweh y demeurera à jamais.
\VS{18}Les chars de Dieu se comptent par vingt-mille, par milliers et par milliers ; le Seigneur est au milieu d'eux ; le Sinaï est dans le sanctuaire.
\VS{19}Tu es monté dans les hauteurs, tu as emmené des captifs, tu as pris des dons pour les distribuer parmi les hommes, et même parmi les rebelles, afin qu'ils habitent dans le lieu de Yahweh Dieu\FTNT{Ep. 4:8-10. Cette prophétie concerne la résurrection du Seigneur Jésus-Christ.}.
\VS{20}Béni soit le Seigneur, qui tous les jours nous comble de ses biens ; Dieu est notre délivrance. Sélah.
\VS{21}Dieu est pour nous le Dieu de délivrance, et les issues de la mort sont à Yahweh le Seigneur.
\VS{22}Certainement, Dieu écrasera la tête de ses ennemis\FTNT{Ge. 3:15.}, le sommet de la tête chevelue de celui qui marche dans ses péchés.
\VS{23}Le Seigneur dit : Je les ramènerai de Basan\FTNT{No. 21:33-35.}, je les ramènerai du fond de la mer.
\VS{24}Afin que tu plonges ton pied dans le sang\FTNT{Ps. 58:11.}, et que la langue de tes chiens ait sa part de tes ennemis.
\VS{25}Ils voient ta marche, ô Dieu ! Ils ont vu ta marche dans le lieu saint, la marche de mon Dieu, mon Roi.
\VS{26}Les chantres allaient devant, ensuite les joueurs d'instruments, et au milieu les jeunes filles jouant du tambour\FTNT{Ex. 15:20 ; 1 S. 18:6.}.
\VS{27}Bénissez Dieu dans les assemblées, bénissez le Seigneur, vous qui êtes descendants d'Israël.
\VS{28}Là sont Benjamin, le plus jeune qui domine sur eux, les chefs de Juda et leur corps d'armée, les chefs de Zabulon, et les chefs de Nephthali.
\VS{29}Ton Dieu ordonne que tu sois puissant. Affermis, ô Dieu, ce que tu as fait pour nous.
\VS{30}Dans ton temple, à Jérusalem, les rois t'amèneront des présents\FTNT{1 R. 10:10 ; Ps. 72:10 ; 2 Ch. 32:23.}.
\VS{31}Epouvante les bêtes sauvages des roseaux, la troupe des taureaux, et les veaux des peuples, et ceux qui se prosternent avec des pièces d'argent. Disperse les peuples qui prennent plaisir à la guerre.
\VS{32}De grands seigneurs viendront d'Egypte ; l'Ethiopie se hâtera d'étendre ses mains vers Dieu.
\VS{33}Royaumes de la terre, chantez à Dieu, célébrez le Seigneur ! Sélah.
\VS{34}Chantez celui qui est monté dans les cieux des cieux, les cieux éternels ; voilà, il fait retentir de sa voix un son puissant.
\VS{35}Attribuez la force à Dieu ; sa majesté est sur Israël, et sa force est dans les nuées.
\VS{36}Dieu ! Tu es redouté à cause de ton lieu saint. Le Dieu d'Israël est celui qui donne la force et la puissance à son peuple. Béni soit Dieu !
\Chap{69}
\TextTitle{Dieu attentif à la prière de ceux qui s'humilient}
\VerseOne{}Psaume de David, donné au chef des chantres, pour le chanter sur les lis.
\VS{2}Délivre-moi, ô Dieu, car les eaux menacent ma vie\FTNT{Ps. 124:4 ; Ps. 144:7.}.
\VS{3}Je suis enfoncé dans un bourbier profond, sans appui ; je suis entré au plus profond des eaux, et les courants d'eau me submergent.
\VS{4}Je suis las de crier, mon gosier se dessèche, mes yeux se consument pendant que je m'attends à Dieu.
\VS{5}Ceux qui me haïssent sans cause\FTNT{Jn. 15:25.} dépassent en nombre les cheveux de ma tête ; ceux qui tâchent de me ruiner et qui sont mes ennemis à tort se sont renforcés ; je dois rendre ce que je n'avais point ravi.
\VS{6}Ô Dieu ! Tu connais ma folie et mes fautes ne te sont point cachées.
\VS{7}Ô Seigneur Yahweh des armées ! Que ceux qui se confient en toi ne soient point honteux à cause de moi ; et que ceux qui te cherchent ne soient point humiliés à cause de moi, ô Dieu d'Israël !
\VS{8}Car pour l'amour de toi j'ai souffert l'opprobre, la honte a couvert mon visage.
\VS{9}Je suis devenu un étranger pour mes frères, et un homme de dehors pour les fils de ma mère\FTNT{Ge. 31:14-15 ; Jn. 7:3-5}.
\VS{10}Car le zèle de ta maison me dévore\FTNT{Jn. 2:17 ; Ro. 15:3.}, et les outrages de ceux qui t'insultaient sont tombés sur moi.
\VS{11}Je pleure et je jeûne : C'est ce qui m'attire l'opprobre.
\VS{12}Je prends un sac pour vêtement, et je suis l'objet de leurs discours moqueurs.
\VS{13}Ceux qui sont assis à la porte parlent de moi, et les buveurs de boissons fortes me mettent en chanson\FTNT{Job. 30:9 ; La. 3:14.}.
\VS{14}Mais je t'adresse ma prière, ô Yahweh\FTNT{Ps. 102:2.} ! Que ce soit le temps favorable, ô Dieu ! Par ta grande bonté. Réponds-moi en m'assurant ta délivrance.
\VS{15}Délivre-moi de la boue, que je ne m'y enfonce point\FTNT{Ps. 40:3.}, et que je sois délivré de ceux qui me haïssent, et des eaux profondes.
\VS{16}Que les courants d'eau ne me submergent plus, que l'abîme ne m'engloutisse point, et que le puits ne ferme point sa bouche sur moi.
\VS{17}Yahweh ! Exauce-moi, car ta bonté est agréable ; dans tes grandes compassions, tourne ta face vers moi ;
\VS{18}et ne cache point ta face à ton serviteur, car je suis en détresse. Hâte-toi, exauce-moi !
\VS{19}Approche-toi de mon âme, rachète-la ; délivre-moi à cause de mes ennemis.
\VS{20}Tu connais toi-même mon opprobre, et ma honte, et mon ignominie ; tous mes ennemis sont devant toi.
\VS{21}L'opprobre m'a brisé le cœur, et je suis languissant ; j'ai attendu que quelqu'un ait compassion de moi, mais il n'y en a point eu. J'ai attendu des consolateurs, mais je n'en ai point trouvé.
\VS{22}Ils m'ont au contraire donné du fiel\FTNT{Mt. 27:34 ; Mt. 27:48.} pour mon repas ; et dans ma soif, ils m'ont abreuvé de vinaigre.
\VS{23}Que leur table soit pour eux un piège et un appât au sein de leur perfection.
\VS{24}Que leurs yeux soient tellement obscurcis, qu'ils ne puissent point voir ; et fais continuellement chanceler leurs reins.
\VS{25}Répands ton indignation sur eux, et que l'ardeur de ta colère les saisisse.
\VS{26}Que leur campement soit désolé, et qu'il n'y ait personne qui habite dans leurs tentes.
\VS{27}Car ils persécutent celui que tu avais frappé, et racontent les souffrances de ceux que tu blesses.
\VS{28}Mets des iniquités à leurs iniquités ; et qu'ils n'entrent point dans ta justice.
\VS{29}Qu'ils soient effacés du livre de vie, et qu'ils ne soient point inscrits avec les justes.
\VS{30}Mais pour moi, qui suis affligé et dans la douleur, ta délivrance, ô Dieu, m'élèvera en une haute retraite.
\VS{31}Je louerai le Nom de Dieu par des cantiques et je le glorifierai par des louanges.
\VS{32}Cela est agréable à Yahweh plus qu'un taureau avec des cornes et des sabots fendus.
\VS{33}Les malheureux le voient et ils se réjouissent ; que votre cœur vive, vous qui cherchez Dieu.
\VS{34}Car Yahweh exauce les misérables et ne méprise point ses prisonniers.
\VS{35}Que les cieux et la terre le louent ; que la mer et tout ce qui s'y meut le louent aussi\FTNT{Ps. 96:11.}.
\VS{36}Car Dieu délivrera Sion et bâtira les villes de Juda ; on y habitera et on la possèdera.
\VS{37}Et la postérité de ses serviteurs en fera son héritage, et ceux qui aiment son Nom y auront leur demeure.
\Chap{70}
\TextTitle{Le pauvre et l'indigent}
\VerseOne{}Psaume de David, pour souvenir, donné au chef des chantres.
\VS{2}Dieu ! Hâte-toi de me délivrer, ô Dieu ! Hâte-toi de venir à mon secours\FTNT{Ps. 40:14 ; Ps. 71:12.}.
\VS{3}Que ceux qui cherchent mon âme soient honteux et rougissent\FTNT{Ps. 35:4 ; Ps. 71:13.} ; et que ceux qui prennent plaisir à mon mal soient repoussés en arrière et soient confus.
\VS{4}Que ceux qui disent : Aha ! Aha ! Retournent en arrière par l'effet de leur honte.
\VS{5}Que tous ceux qui te cherchent exultent et se réjouissent en toi ; et que ceux qui aiment ta délivrance disent toujours : Glorifié soit Dieu !
\VS{6}Moi, je suis affligé et misérable, ô Dieu ! Hâte-toi de venir vers moi ; tu es mon secours et mon libérateur, ô Yahweh ! Ne tarde point.
\Chap{71}
\TextTitle{Demeurer en Dieu jusqu'au bout}
\VerseOne{}Yahweh ! Je cherche en toi mon refuge : Que je ne sois jamais confus !
\VS{2}Délivre-moi par ta justice et sauve-moi. Incline ton oreille vers moi, mets-moi en sûreté.
\VS{3}Sois pour moi le rocher de mon refuge, afin que je puisse toujours m'y retirer ; tu as donné l'ordre de me mettre en sûreté, car tu es mon rocher et ma forteresse.
\VS{4}Mon Dieu ! Délivre-moi de la main du méchant, de la main du pervers et de l'oppresseur.
\VS{5}Car tu es mon espérance, Seigneur Yahweh ! Ma confiance dès ma jeunesse.
\VS{6}Je m'appuie sur toi dès le ventre de ma mère ; c'est toi qui m'as tiré hors des entrailles de ma mère\FTNT{Ps. 22:10-11.} ; tu es le sujet continuel de mes louanges.
\VS{7}Je suis pour plusieurs comme un miracle, mais tu es mon puissant refuge.
\VS{8}Que ma bouche soit remplie de ta louange et de ta gloire chaque jour.
\VS{9}Ne me rejette point au temps de ma vieillesse ; ne m'abandonne point maintenant que ma force est consumée.
\VS{10}Car mes ennemis ont parlé de moi, et ceux qui épient mon âme ont pris conseil ensemble,
\VS{11}disant : Dieu l'a abandonné. Poursuivez-le et saisissez-le, car il n'y a personne qui le délivre.
\VS{12}Dieu, ne t'éloigne point de moi ! Mon Dieu hâte-toi de venir à mon secours !
\VS{13}Que ceux qui sont les ennemis de mon âme soient honteux et défaits ; et que ceux qui cherchent mon malheur soient enveloppés d'opprobre et de honte.
\VS{14}Mais moi, j'espèrerai toujours et je te louerai tous les jours davantage.
\VS{15}Ma bouche racontera chaque jour ta justice et ta délivrance, bien que je n'en sache point le nombre.
\VS{16}Je marcherai par la force du Seigneur Yahweh ; je raconterai ta seule justice.
\VS{17}Ô Dieu ! Tu m'as enseigné dès ma jeunesse et j'ai annoncé jusqu'à présent tes merveilles.
\VS{18}Ô Dieu ! Ne m'abandonne pas, même dans la blanche vieillesse. Afin que j'annonce ta force à cette génération présente, ta puissance à la génération à venir.
\VS{19}Car ta justice, ô Dieu, est haut élevée, car tu as fait de grandes choses. Ô Dieu, qui est semblable à toi ?
\VS{20}Tu m'as fait éprouver bien des détresses et des malheurs, mais tu me redonneras la vie et tu me feras remonter hors des abîmes de la terre.
\VS{21}Relève ma grandeur et console-moi encore.
\VS{22}Je te louerai au son du luth, je chanterai ta fidélité, mon Dieu, je te célèbrerai avec la harpe, Saint d'Israël !
\VS{23}Mes lèvres et mon âme, que tu as rachetée, pousseront des cris de joie quand je te chanterai.
\VS{24}Ma langue aussi publiera chaque jour ta justice, car ceux qui cherchent mon malheur seront honteux et rougiront.
\Chap{72}
\TextTitle{Le royaume messianique}
\VerseOne{}De Salomon. Ô Dieu, donne tes jugements au roi et ta justice au fils du roi.
\VS{2}Qu'il juge avec justice ton peuple, et tes malheureux avec équité.
\VS{3}Que les montagnes portent la paix pour le peuple, et que les collines la portent en justice.
\VS{4}Qu'il fasse droit aux malheureux du peuple, qu'il délivre les fils du misérable, et qu'il écrase l'oppresseur !
\VS{5}Ils te craindront tant que le soleil et la lune dureront d'âge en âge.
\VS{6}Il descendra comme la pluie sur l'herbe fauchée, comme les ondées qui arrosent la terre.
\VS{7}En son temps, le juste fleurira, et il y aura abondance de paix jusqu'à ce qu'il n'y ait plus de lune.
\VS{8}Il dominera depuis une mer jusqu'à l'autre, et depuis le fleuve jusqu'aux extrémités de la terre.
\VS{9}Les habitants des déserts se courberont devant lui, et ses ennemis lécheront la poussière.
\VS{10}Les rois de Tarsis et des îles lui rapporteront des dons ; les rois de Saba et de Séba lui apporteront des présents.
\VS{11}Tous les rois aussi se prosterneront devant lui, toutes les nations le serviront.
\VS{12}Car il délivrera le pauvre qui crie vers lui, l'affligé et celui qui n'a personne qui l'aide\FTNT{Ps. 34:18 ; Job. 29:12.}.
\VS{13}Il aura compassion du pauvre et du misérable, et il sauvera les âmes des misérables.
\VS{14}Il garantira leur âme de la fraude et de la violence, et leur sang sera précieux devant ses yeux.
\VS{15}Il vivra donc, et on lui donnera de l'or de Séba, et on fera des prières pour lui continuellement ; on le bénira chaque jour.
\VS{16}Les blés abonderont dans le pays, au sommet des montagnes, et leurs épis s'agiteront comme les arbres du Liban ; les hommes fleuriront dans les villes comme l'herbe de la terre.
\VS{17}Sa renommée durera à toujours ; sa renommée ira de père en fils tant que le soleil durera ; et on se bénira en lui ; toutes les nations le diront heureux.
\VS{18}Béni soit Yahweh Dieu, le Dieu d'Israël, qui seul fait des choses merveilleuses !
\VS{19}Béni soit éternellement son Nom glorieux, et que toute la terre soit remplie de sa gloire. Amen ! Oui, amen !
\VS{20}Fin des prières de David, fils d'Isaï.
\Chap{73}
\TextTitle{L'orgueil des méchants}
\VerseOne{}Psaume d'Asaph. Quoi qu'il en soit, Dieu est bon pour Israël, pour ceux qui ont le cœur pur\FTNT{Mt. 5:8.}.
\VS{2}Toutefois, mes pieds allaient fléchir, mes pas étaient sur le point de glisser.
\VS{3}Car j'ai porté envie aux insensés en voyant la prospérité des méchants.
\VS{4}Rien ne les tourmente jusqu'à leur mort, et leur corps est gras.
\VS{5}Ils n'ont point de part aux peines des humains, et ils ne sont point frappés avec les autres hommes.
\VS{6}C'est pourquoi l'orgueil les environne comme un collier, et un vêtement de violence les couvre.
\VS{7}Les yeux leur sortent dehors à force de graisse ; ils surpassent les desseins de leur cœur.
\VS{8}Ils sont pernicieux, et parlent méchamment d'opprimer ; ils parlent d'une manière hautaine.
\VS{9}Ils élèvent leur bouche jusqu'aux cieux et leur langue parcourt la terre.
\VS{10}C'est pourquoi son peuple se tourne de leur côté, il avale l'eau abondamment.
\VS{11}Ils disent : Comment Dieu saurait-il ? Comment le Très-Haut connaîtrait-il\FTNT{Es. 29:15 ; Ez. 8:12 ; Ps. 94:7 ; Job. 22:12-13.} ?
\VS{12}Voilà, ceux-ci sont méchants, ils prospèrent toujours dans ce monde et acquièrent de plus en plus de richesses.
\VS{13}Quoi qu'il en soit, c'est donc en vain que j'ai purifié mon cœur et que j'ai lavé mes mains dans l'innocence\FTNT{Mal. 3:14 ; Job. 35:3}.
\VS{14}Je suis frappé tous les jours, et tous les matins mon châtiment est là.
\VS{15}Si je disais : Je veux parler comme eux, voici je trahirais la génération de tes fils.
\VS{16}Toutefois, j'ai tâché de connaître cela, mais cela m'a paru fort difficile,
\VS{17}jusqu'à ce que je sois entré dans le sanctuaire de Dieu et que j'aie considéré la fin de telles gens.
\VS{18}Quoi qu'il en soit, tu les as mis sur des voies glissantes, tu les fais tomber dans des précipices.
\VS{19}Comment ont-ils été ainsi détruits en un instant ? Ont-ils défailli ? Ont-ils été consumés d'épouvante ?
\VS{20}Ils sont comme un songe lorsqu'on s'est réveillé. Seigneur, tu méprises leur image à ton réveil.
\VS{21}Quand mon cœur s'aigrissait et que je me sentais percé dans les entrailles,
\VS{22}j'étais alors stupide, et je n'avais aucune connaissance ; j'étais comme une bête dans ta présence.
\VS{23}Je serai donc toujours avec toi ; tu m'as pris par la main droite.
\VS{24}Tu me conduiras par ton conseil, et tu me recevras dans la gloire.
\VS{25}Quel autre ai-je au ciel ? Or sur la terre je ne prends plaisir qu'en toi seul.
\VS{26}Ma chair et mon cœur étaient consumés, mais Dieu est le rocher de mon cœur, et mon partage pour toujours.
\VS{27}Car voilà, ceux qui s'éloignent de toi périront ; tu retrancheras tous ceux qui se détournent de toi.
\VS{28}Mais pour moi, m'approcher de Dieu c'est mon bien ; j'ai mis toute mon espérance dans le Seigneur Yahweh, afin de raconter toutes tes œuvres.
\Chap{74}
\TextTitle{Appel au secours du peuple de Dieu}
\VerseOne{}Cantique d'Asaph. Ô Dieu, pourquoi nous as-tu rejetés pour toujours ? Et pourquoi ta colère fume-t-elle contre le troupeau de ton pâturage\FTNT{Ps. 79:5.} ?
\VS{2}Souviens-toi de ton assemblée que tu as acquise autrefois. Tu t'es approprié cette montagne de Sion, sur laquelle tu habitais, afin qu'elle soit la portion de ton héritage.
\VS{3}Elève tes pas vers les lieux constamment dévastés ; l'ennemi a tout renversé dans le lieu saint.
\VS{4}Tes adversaires ont rugi au milieu de ton assemblée ; ils ont mis leurs signes pour signes.
\VS{5}On les a vus pareils à celui qui lève la cognée dans une épaisse forêt.
\VS{6}Et maintenant, avec des haches et des marteaux, ils brisent les sculptures.
\VS{7}Ils ont mis le feu à ton lieu saint. Ils ont abattu à terre et profané la demeure dédiée à ton Nom\FTNT{2 R. 25:9.}.
\VS{8}Ils ont dit en leur cœur : Saccageons-les tous ensemble ! Ils ont brûlé dans le pays tous les lieux saints de Dieu.
\VS{9}Nous ne voyons plus nos signes ; il n'y a plus de prophètes ; et personne parmi nous qui sache jusqu'à quand\FTNT{La. 2:9-10.}.
\VS{10}Ô Dieu ! Jusqu'à quand l'adversaire te couvrira-t-il d'opprobres et l'ennemi méprisera-t-il ton Nom à jamais ?
\VS{11}Pourquoi retires-tu ta main, même ta droite ? Consume-les en la tirant du milieu de ton sein !
\VS{12}Or Dieu est mon Roi dès les temps anciens, faisant des délivrances au milieu de la terre.
\VS{13}Tu as fendu la mer par ta force ; tu as brisé les têtes des serpents sur les eaux.
\VS{14}Tu as brisé les têtes du léviathan, tu l'as donné pour nourriture au peuple du désert.
\VS{15}Tu as ouvert la fontaine et le torrent, tu as desséché les grosses rivières.
\VS{16}A toi est le jour, à toi aussi est la nuit ; tu as établi la lumière et le soleil.
\VS{17}Tu as posé toutes les limites de la terre ; tu as formé l'été et l'hiver.
\VS{18}Souviens-toi de ceci : Que l'ennemi a blasphémé Yahweh et qu'un peuple insensé a outragé ton Nom.
\VS{19}Ne livre pas aux vivants l'âme de la tourterelle, n'oublie pas à toujours la vie de tes affligés.
\VS{20}Regarde à ton alliance, car les lieux ténébreux de la terre sont remplis d'habitations de violence.
\VS{21}Ne permets pas que celui qui est foulé s'en retourne tout confus. Que l'affligé et le pauvre louent ton Nom !
\VS{22}Ô Dieu ! Lève-toi, défends ta cause, souviens-toi de l'opprobre qui t'est fait tous les jours par l'insensé !
\VS{23}N'oublie pas le cri de tes adversaires, le bruit de ceux qui s'élèvent contre toi monte continuellement !
\Chap{75}
\TextTitle{L'élevation vient de Yahweh}
\VerseOne{}Psaume d'Asaph. Cantique donné au chef des chantres, pour le chanter sur Al-Thasheth\FTNT{Voir Ps. 57:1.}.
\VS{2}Nous te célébrons, ô Dieu ! Nous te célébrons et ton Nom est près de nous ; nous racontons tes merveilles.
\VS{3}Au temps que j'aurai fixé, je jugerai avec droiture.
\VS{4}La terre se dissout avec tous ceux qui y habitent, mais j'affermis ses piliers. Sélah.
\VS{5}Je dis aux insensés : N'agissez point follement ; et aux méchants : N'élevez pas la tête.
\VS{6}N'élevez pas si haut votre tête, et ne parlez point avec fierté.
\VS{7}Car l'élévation ne vient point d'orient, ni d'occident ni du désert.
\VS{8}Car c'est Dieu qui gouverne ; il abaisse l'un, et élève l'autre\FTNT{1 S. 2:7.}.
\VS{9}Il y a une coupe dans la main de Yahweh\FTNT{Es. 51:17-22 ; Jé. 25:27-28 ; Ap. 14:10 ; Ap. 16:19.}, et le vin rougit dedans ; il est plein de mélange, et Dieu en verse ; certainement, tous les méchants de la terre en suceront et en boiront jusqu'à la lie.
\VS{10}Mais moi, je raconterai ces choses à jamais, je chanterai au Dieu de Jacob.
\VS{11}J'humilierai tous les méchants, mais les justes seront élevés.
\Chap{76}
\TextTitle{La Puissance du Dieu redoutable}
\VerseOne{}Psaume d'Asaph. Cantique donné au chef des chantres, pour le chanter avec instruments à cordes.
\VS{2}Dieu est connu en Judée, sa renommée est grande en Israël ;
\VS{3}sa tente est à Salem et sa demeure à Sion.
\VS{4}Là il a brisé les arcs étincelants, le bouclier, l'épée et les armes de guerre. Sélah.
\VS{5}Tu es resplendissant, plus magnifique que les montagnes des ravisseurs.
\VS{6}Les plus courageux sont étourdis, ils sont dans un profond assoupissement, et aucun de ces hommes vaillants n'a trouvé ses mains.
\VS{7}Ô Dieu de Jacob, les cavaliers et les chevaux se sont endormis quand tu les as menacés.
\VS{8}Tu es redoutable, toi. Qui peut se tenir devant toi quand ta colère éclate ?
\VS{9}Tu fais entendre des cieux le jugement ; la terre en a eu peur et s'est tenue dans le silence.
\VS{10}Quand tu te lèves, ô Dieu, pour faire jugement, pour délivrer tous les malheureux de la terre ! Sélah.
\VS{11}L'homme te célèbre, même dans sa fureur, quand tu te ceins de toute ta colère.
\VS{12}Faites vos vœux à Yahweh votre Dieu et accomplissez-les ! Que tous ceux qui l'environnent apportent des dons au Dieu terrible !
\VS{13}Il coupe le souffle des princes ; il est redoutable aux rois de la terre.
\Chap{77}
\TextTitle{Se souvenir des prodiges de Yahweh}
\VerseOne{}Psaume d'Asaph, donné au chef des chantres, d'après Jeduthun.
\VS{2}Ma voix s'élève à Dieu, et je crie ; ma voix s'adresse à Dieu, et il m'écoutera.
\VS{3}Je cherche le Seigneur au jour de ma détresse ; sans cesse mes mains s'étendent durant la nuit ; mon âme refuse d'être consolée.
\VS{4}Je me souviens de Dieu, et je gémis ; je médite, et mon esprit est affaibli. Sélah.
\VS{5}Tu empêches mes yeux de dormir ; je suis troublé, et ne peux parler.
\VS{6}Je pense aux jours d'autrefois et aux années des siècles passées\FTNT{Ps. 143:5.}.
\VS{7}Je me souviens de mes chants pendant la nuit, je médite en mon cœur, et mon esprit cherche diligemment.
\VS{8}Le Seigneur m'a-t-il rejeté pour toujours ? Ne me sera-t-il plus favorable ?
\VS{9}Sa bonté est-elle disparue pour toujours ? Sa parole a-t-elle pris fin pour l'éternité ?
\VS{10}Dieu a-t-il oublié d'avoir compassion ? A-t-il dans sa colère retiré sa miséricorde ? Sélah.
\VS{11}Je dis : Ce qui me fait devenir malade, je me souviendrai des années de la droite du Très–Haut.
\VS{12}Je me souviens des exploits de Yahweh ; je me suis, dis-je, souvenu de tes merveilles d'autrefois.
\VS{13}Je méditerai toutes tes œuvres, et je parlerai de tes œuvres.
\VS{14}Ô Dieu ! Tes voies sont saintes. Quel dieu est grand comme Dieu ?
\VS{15}Tu es le Dieu qui fait des merveilles ! Tu as fait connaître ta force parmi les peuples.
\VS{16}Tu as délivré par ton bras ton peuple, les fils de Jacob et de Joseph. Sélah.
\VS{17}Les eaux t'ont vu, ô Dieu ! Les eaux t'ont vu et ont tremblé, même les abîmes en ont été émus.
\VS{18}Les nuées ont versé un déluge d'eau, les nuées ont fait retentir leur son ; tes flèches ont volé de toutes parts.
\VS{19}La voix de ton tonnerre était dans le tourbillon, les éclairs ont éclairé le monde, la terre en a été émue et en a tremblé.
\VS{20}Tu te frayas un chemin par la mer, un sentier par les grosses eaux ; et tes traces ne furent plus reconnues.
\VS{21}Tu as mené ton peuple comme un corps d'armée sous la conduite de Moïse et d'Aaron\FTNT{Mi. 6:4.}.
\Chap{78}
\TextTitle{Les œuvres de Dieu dans l'histoire d'Israël}
\VerseOne{}Cantique d'Asaph. Mon peuple, écoute ma loi, prêtez vos oreilles aux paroles de ma bouche.
\VS{2}J'ouvrirai ma bouche en une parabole ; je proférerai les énigmes cachées des temps anciens\FTNT{Mt. 13:35.}.
\VS{3}Ce que nous avons entendu et connu, et que nos pères nous ont raconté\FTNT{Ps. 44:2.},
\VS{4}nous ne le cacherons point à leurs fils. Ils raconteront à la génération à venir les louanges de Yahweh, sa puissance et ses merveilles qu'il a faites.
\VS{5}Car il a établi le témoignage en Jacob, et il a mis la loi en Israël ; il a donné cet ordre à nos pères de la faire connaître à leurs fils\FTNT{De. 4:9.},
\VS{6}pour qu'elle soit connue de la génération future, des fils qui naîtraient, et pour que lorsqu'ils seront grands, ils la relatent à leurs fils,
\VS{7}afin qu'ils mettent leur confiance en Dieu, et qu'ils n'oublient point les œuvres de Dieu, et qu'ils gardent ses commandements.
\VS{8}Afin qu'ils ne soient point comme leurs pères, une génération revêche et rebelle, une génération insoumise de cœur, dont l'esprit est infidèle à Dieu\FTNT{Ex. 32:9 ; Ac. 7:51.}.
\VS{9}Les fils d'Ephraïm, armés et tirant de l'arc, tournèrent le dos le jour de la bataille.
\VS{10}Ils ne gardèrent point l'alliance de Dieu et refusèrent de marcher selon sa loi.
\VS{11}Ils oublièrent ses œuvres et ses merveilles qu'il leur avait fait voir.
\VS{12}Il avait fait des miracles en présence de leurs pères, dans le pays d'Egypte, dans le champ de Tsoan.
\VS{13}Il fendit la mer et les fit passer au travers ; et il fit arrêter les eaux comme un monceau de pierres.
\VS{14}Il les conduisit de jour par la nuée, et toute la nuit par une lumière de feu\FTNT{Ex. 13:21.}.
\VS{15}Il fendit les rochers au désert, et leur donna à boire d'abondantes eaux, comme s'il eût puisé des abîmes.
\VS{16}Il fit sortir des ruisseaux de la roche\FTNT{Ex. 17:6 ; No. 20:11 ; 1 Co. 10:4.} et fit couler des eaux comme des rivières.
\VS{17}Toutefois, ils continuèrent à pécher contre lui, irritant le Très-Haut dans le désert.
\VS{18}Ils tentèrent Dieu dans leurs cœurs, en demandant de la viande selon leur désir.
\VS{19}Ils parlèrent contre Dieu, disant : Dieu pourrait-il dresser une table dans ce désert\FTNT{No. 11:4.} ?
\VS{20}Voilà, dirent-ils, il a frappé le rocher, et les eaux ont coulé et des torrents ont débordé ; mais pourrait-il aussi nous donner du pain ? Fournirait-il de la viande à son peuple ?
\VS{21}C'est pourquoi, Yahweh les ayant entendus, se mit dans une grande colère, et le feu s'embrasa contre Jacob, et sa colère s'excita contre Israël.
\VS{22}Parce qu'ils n'avaient point cru en Dieu et ne s'étaient point confiés en sa délivrance.
\VS{23}Il ordonna aux nuées d'en haut et il ouvrit les portes des cieux ;
\VS{24}il fit pleuvoir la manne sur eux pour leur nourriture et il leur donna le blé du ciel\FTNT{Ex. 16:14 ; Jn. 6:31.}.
\VS{25}Ils mangèrent tous le pain des grands. Il leur envoya de la viande pour s'en rassasier.
\VS{26}Il excita dans les cieux le vent d'orient et il amena par sa puissance le vent du sud.
\VS{27}Il fit pleuvoir sur eux de la viande comme de la poussière, et comme le sable des mers des oiseaux ailées.
\VS{28}Il les fit tomber au milieu du camp, autour de leurs demeures.
\VS{29}Ils en mangèrent et en furent pleinement rassasiés, car il leur donna selon leur désir.
\VS{30}Mais ils ne furent pas encore dégoûtés de leur désir, et leur viande était encore dans leur bouche
\VS{31}quand la colère de Dieu s'excita contre eux, et qu'il mit à mort les plus gras d'entre eux, et abattit les gens d'élite d'Israël\FTNT{1 Co. 10:5.}.
\VS{32}Malgré cela, ils péchèrent encore et ne crurent point à ses prodiges\FTNT{No. 14:2.}.
\VS{33}C'est pourquoi il consuma leurs jours par la vanité et leurs années par une fin soudaine.
\VS{34}Quand il les mettait à mort, alors ils le recherchaient ; ils se repentaient et ils cherchaient Dieu dès le matin.
\VS{35}Ils se souvenaient que Dieu était leur rocher, et Dieu, le Très-Haut, était leur libérateur.
\VS{36}Mais ils le trompaient de leur bouche et ils lui mentaient de leur langue\FTNT{Es. 29:13 ; Jé. 12:2 ; Mt. 15:8.} ;
\VS{37}car leur cœur n'était point droit envers lui, et ils ne furent point fidèles à son alliance.
\VS{38}Toutefois, comme il est compatissant, il pardonna leur iniquité, au point qu'il ne les détruisit pas ; mais il détourna souvent sa colère et ne réveilla pas toute sa fureur.
\VS{39}Il se souvint qu'ils n'étaient que chair, qu'un vent qui passe et qui ne revient point.
\VS{40}Combien de fois l'ont–ils irrité au désert, et combien de fois l'ont–ils attristé dans ce lieu inhabitable ?
\VS{41}Ils ne cessèrent de tenter Dieu et de provoquer le Saint d'Israël.
\VS{42}Ils ne se souvinrent point de sa puissance, du jour où il les délivra de la main de l'ennemi,
\VS{43}des miracles qu'il accomplit en Egypte, et de ses merveilles dans les champs de Tsoan.
\VS{44}Il changea en sang leurs fleuves et leurs ruisseaux et ils ne purent en boire les eaux\FTNT{Ex. 7:20.}.
\VS{45}Il envoya contre eux des mouches qui les dévorèrent et des grenouilles qui les détruisirent\FTNT{Ex. 8:6-24.}.
\VS{46}Il livra leurs récoltes aux sauterelles, le produit de leur travail aux sauterelles\FTNT{Ex. 10:13.}.
\VS{47}Il détruisit leurs vignes par la grêle, et leurs sycomores par les orages\FTNT{Ex. 9:23.}.
\VS{48}Il livra leur bétail à la grêle, et leurs troupeaux aux foudres étincelantes.
\VS{49}Il envoya sur eux l'ardeur de sa colère, la fureur, la rage et la détresse, un corps d'armée de messagers de malheur.
\VS{50}Il donna libre cours à sa colère, et ne retira point leur âme de la mort ; il livra leur vie à la peste\FTNT{Ex. 9:6.}.
\VS{51}Il frappa tout premier-né en Egypte, les prémices de la vigueur dans les tentes de Cham\FTNT{Ex. 12:29.}.
\VS{52}Il fit partir son peuple comme des brebis, il les mena comme un corps d'armée dans le désert.
\VS{53}Il les conduisit sûrement, et sans qu'ils eussent aucune frayeur, là où la mer couvrit leurs ennemis.
\VS{54}Il les amena vers sa frontière sainte, vers cette montagne que sa droite a acquise\FTNT{Ex. 15:17.}.
\VS{55}Il chassa devant eux les nations, leur distribua le pays en héritage, et fit habiter les tribus d'Israël dans les tentes de ces nations.
\VS{56}Mais ils tentèrent et irritèrent le Dieu Très-Haut, et ne gardèrent point ses préceptes.
\VS{57}Et ils se retirèrent en arrière et furent infidèles comme leurs pères ; ils tournèrent comme un arc trompeur.
\VS{58}Ils le provoquèrent à la colère par leurs hauts lieux, et l'émurent à la jalousie par leurs images taillées\FTNT{De. 32:16-21.}.
\VS{59}Dieu l'entendit et se mit dans une grande colère, et il méprisa fortement Israël.
\VS{60}Il abandonna la demeure de Silo, la tente où il habitait parmi les hommes.
\VS{61}Il livra en captivité sa force et son ornement entre les mains de l'ennemi.
\VS{62}Il livra son peuple à l'épée et se mit dans une grande colère contre son héritage.
\VS{63}Le feu consuma leurs gens d'élite, et leurs vierges ne furent point louées.
\VS{64}Leurs sacrificateurs tombèrent par l'épée, et leurs veuves ne les pleurèrent point.
\VS{65}Puis le Seigneur se réveilla comme un homme qui se serait endormi, et comme un puissant homme qui s'écrie ayant encore le vin dans la tête.
\VS{66}Il frappa ses adversaires par derrière et les mit en opprobre perpétuel.
\VS{67}Mais il dédaigna la tente de Joseph, et ne choisit point la tribu d'Ephraïm.
\VS{68}Mais il choisit la tribu de Juda, la montagne de Sion, celle qu'il aime.
\VS{69}Il bâtit son lieu saint dans les lieux élevés, et l'établit comme la terre qu'il a fondée pour toujours.
\VS{70}Il choisit David, son serviteur, et le prit de la bergerie\FTNT{1 S. 16:11 ; 2 S. 7:8.} ;
\VS{71}il le prit derrière les brebis qui allaitent et l'amena pour paître Jacob, son peuple, et Israël, son héritage.
\VS{72}Aussi il les dirigea selon l'intégrité de son cœur, et les conduisit avec des mains intelligentes.
\Chap{79}
\TextTitle{Appel au jugement de Dieu}
\VerseOne{}Psaume d'Asaph. Ô Dieu ! Les nations sont entrées dans ton héritage ; on a profané ton saint temple, on a mis Jérusalem en monceaux de pierres.
\VS{2}On a livré les cadavres de tes serviteurs pour viande aux oiseaux du ciel, et la chair de tes fidèles aux bêtes de la terre.
\VS{3}On a répandu leur sang comme de l'eau autour de Jérusalem, et il n'y a eu personne pour les enterrer.
\VS{4}Nous sommes un sujet d'opprobre à nos voisins, de moquerie et de risée à ceux qui habitent autour de nous\FTNT{Ps. 44:14 ; Ps. 80:7.}.
\VS{5}Jusqu'à quand, ô Yahweh, t'irriteras-tu sans cesse et ta jalousie s'embrasera-t-elle comme un feu\FTNT{Ps. 89:47.} ?
\VS{6}Répands ta fureur sur les nations qui ne te connaissent point et sur les royaumes qui n'invoquent point ton Nom\FTNT{Jé. 10:25.}.
\VS{7}Car on a dévoré Jacob et on a ravagé ses demeures.
\VS{8}Ne rappelle point devant nous les iniquités passées. Que tes compassions viennent en hâte au-devant de nous, car nous sommes dans une extrême détresse.
\VS{9}Ô Dieu de notre délivrance ! Aide-nous pour l'amour de la gloire de ton Nom, et délivre-nous ! Pardonne-nous nos péchés pour l'amour de ton Nom !
\VS{10}Pourquoi les nations diraient-elles : Où est leur Dieu ? Que la vengeance du sang de tes serviteurs, qui a été répandu, soit manifestée parmi les nations en notre présence.
\VS{11}Que le gémissement des captifs parviennent jusqu'à toi. Par ton bras puissant sauve tes fils, ceux qui vont périr !
\VS{12}Et rends à nos voisins, dans leur sein, sept fois au double l'opprobre qu'ils t'ont fait, ô Yahweh !
\VS{13}Mais nous, ton peuple, et le troupeau de ton pâturage, nous te louerons pour toujours, et de génération en génération nous publierons tes louanges.
\Chap{80}
\TextTitle{Implorer Yahweh}
\VerseOne{}Psaume d'Asaph, donné au chef des chantres, pour le chanter Sosannim-héduth.
\VS{2}Toi qui pais Israël, prête l'oreille ! Toi qui mènes Joseph comme un troupeau, toi qui es assis entre les chérubins\FTNT{2 S. 6:2 ; Es. 37:16 ; Ps. 99:1.}, fais briller ta splendeur !
\VS{3}Réveille ta puissance au-devant d'Ephraïm, de Benjamin et de Manassé ; et viens pour notre délivrance !
\VS{4}Dieu, ramène-nous et fais briller ta face ! Et nous serons délivrés !
\VS{5}Ô Yahweh, Dieu des armées, jusqu'à quand seras-tu irrité contre la prière de ton peuple ?
\VS{6}Tu les nourris de pain de larmes et tu les abreuves de larmes à pleine mesure.
\VS{7}Tu fais de nous un sujet de dispute entre nos voisins, et nos ennemis se moquent de nous.
\VS{8}Ô Dieu des armées, ramène-nous et fais briller ta face ! Et nous serons délivrés.
\VS{9}Tu avais retiré une vigne hors d'Egypte, tu as chassé les nations, et tu l'as plantée\FTNT{Es. 5:1-7 ; Os. 10:1 ; Mt. 20:1 ; Mt. 21:28-33.}.
\VS{10}Tu as préparé une place devant elle, tu lui as fait prendre racine, et elle a rempli la terre.
\VS{11}Les montagnes étaient couvertes de son ombre, et ses rameaux étaient comme de hauts cèdres de Dieu.
\VS{12}Elle étendait ses branches jusqu'à la mer, et ses rejetons jusqu'au fleuve.
\VS{13}Pourquoi as-tu rompu ses clôtures, de sorte que tous les passants sur la route cueillent ses raisins ?
\VS{14}Les sangliers de la forêt l'ont détruite, et toutes les bêtes des champs en font leur pâture.
\VS{15}Ô Dieu des armées, reviens ! Regarde des cieux, vois, et visite cette vigne ;
\VS{16}et le plant que ta droite avait planté, et le fils que tu t'es choisi.
\VS{17}Elle est brûlée par le feu, elle est coupée ; ils périssent devant ta face menaçante.
\VS{18}Que ta main soit sur l'homme de ta droite, sur le fils de l'homme que tu t'es choisi.
\VS{19}Et nous ne nous éloignerons plus de toi. Rends-nous la vie, et nous invoquerons ton Nom.
\VS{20}Ô Yahweh ! Dieu des armées, ramène-nous, fais briller ta face, et nous serons délivrés !
\Chap{81}
\TextTitle{Se débarasser des dieux étrangers}
\VerseOne{}Psaume d'Asaph, donné au chef des chantres, pour le chanter sur la Guitthith.
\VS{2}Chantez avec allégresse à notre Dieu, notre force ! Poussez des cris de joie en l'honneur du Dieu de Jacob.
\VS{3}Sonnez du shofar, prenez le tambour, la harpe mélodieuse et le luth.
\VS{4}Sonnez du shofar à la nouvelle lune, à la pleine lune, au jour de notre fête\FTNT{No. 10:10.}.
\VS{5}Car c'est une loi pour Israël, une ordonnance du Dieu de Jacob.
\VS{6}Il établit un statut à Joseph, lorsqu'il marcha contre le pays d'Egypte, où j'entendis un langage que je ne connaissais pas.
\VS{7}J'ai retiré son épaule du fardeau, et ses mains ont lâché les corbeilles.
\VS{8}Tu as crié dans la détresse, et je t'ai sauvé ; je t'ai répondu dans le lieu caché du tonnerre ; je t'ai éprouvé auprès des eaux de Mériba. Sélah.
\VS{9}Ecoute mon peuple, je te relèverai. Israël, si tu m'écoutais !
\VS{10}Qu'il n'y ait point de dieu étranger au milieu de toi, et ne te prosterne point devant les dieux des étrangers.
\VS{11}Je suis Yahweh, ton Dieu, qui t'ai fait monter hors du pays d'Egypte. Ouvre ta bouche et je la remplirai.
\VS{12}Mais mon peuple n'a point écouté ma voix, et Israël ne m'a point obéi.
\VS{13}C'est pourquoi je les ai abandonnés aux penchants de leur cœur, et ils ont suivi leurs propres conseils\FTNT{Es. 63:17 ; Es. 65:2 ; 2 Pi. 3:3.}.
\VS{14}Ô si mon peuple m'écoutait ! Si Israël marchait dans mes voies !
\VS{15}J'abattrais en un instant leurs ennemis et je tournerais ma main contre leurs adversaires.
\VS{16}Ceux qui haïssent Yahweh le flatteraient, et le bonheur de mon peuple durerait toujours.
\VS{17}Dieu le nourrirait du meilleur froment ; et je le rassasierais du miel du rocher.
\Chap{82}
\TextTitle{Dieu dénonce l'injustice des hommes}
\VerseOne{}Psaume d'Asaph. Dieu se tient dans l'assemblée de Dieu, il juge au milieu des juges.
\VS{2}Jusqu'à quand jugerez-vous injustement et aurez-vous égard à l'apparence de la personne des méchants\FTNT{Ps. 58:2.} ? Sélah.
\VS{3}Faites droit à celui qu'on opprime et à l'orphelin ; faites justice à l'affligé et au pauvre ;
\VS{4}délivrez celui qu'on maltraite et le misérable, retirez-le de la main des méchants.
\VS{5}Ils ne connaissent ni n'entendent rien ; ils marchent dans les ténèbres, tous les fondements de la terre sont ébranlés.
\VS{6}J'ai dit : Vous êtes des dieux\FTNT{Jn. 10:34.}, et vous êtes tous fils du Très-Haut.
\VS{7}Toutefois, vous mourrez comme des hommes, et vous les princes vous tomberez comme les autres.
\VS{8}Ô Dieu ! Lève-toi, juge la terre ; car tu auras en héritage toutes les nations\FTNT{Ps. 2:8 ; Hé. 1:2.}.
\Chap{83}
\TextTitle{Dessein et confusion des ennemis d'Israël}
\VerseOne{}Cantique et psaume d'Asaph.
\VS{2}Ô Dieu ! Ne garde point le silence, ne te tais point, et ne te tiens point en repos, ô Dieu\FTNT{Ps. 35:22.} !
\VS{3}Car voici, tes ennemis s'agitent, et ceux qui te haïssent ont levé la tête.
\VS{4}Ils ont consulté finement en secret contre ton peuple, et ils ont tenu conseil contre ceux qui se sont retirés vers toi pour se cacher\FTNT{Ps. 2:2.}.
\VS{5}Ils disent : Venez et détruisons-les, en sorte qu'ils ne soient plus une nation, et qu'on ne fasse plus mention du nom d'Israël\FTNT{Ce passage fait allusion aux désirs qu'ont certaines nations de voir Israël détruite Mi. 4:11 ; Ap. 11:1-2.}.
\VS{6}Car ils consultent ensemble d'un même esprit ; ils font alliance contre toi.
\VS{7}Les tentes d'Edom et des Ismaélites, des Moabites et des Hagaréniens ;
\VS{8}de Guebal, d'Ammon, d'Amalek, les Philistins avec les habitants de Tyr.
\VS{9}L'Assyrie aussi se joint à eux ; ils ont servi de bras aux fils de Lot. Sélah.
\VS{10}Fais-leur comme tu fis à Madian\FTNT{Jg. 7:15.}, comme à Sisera\FTNT{Jg. 4:15.}, et comme à Jabin, auprès du torrent de Kison !
\VS{11}Ils furent détruits à En-Dor et servirent de fumier à la terre.
\VS{12}Que leurs chefs soient traités comme Oreb et comme Zeeb ; et que tous leurs princes soient comme Zébach et Tsalmunna\FTNT{Jg. 7:25.} ;
\VS{13}parce qu'ils ont dit : Prenons possession des habitations agréables de Dieu.
\VS{14}Mon Dieu ! Rends-les semblables au tourbillon et au chaume chassé par le vent,
\VS{15}comme le feu brûle une forêt, et comme la flamme embrase les montagnes.
\VS{16}Poursuis-les ainsi par ta tempête et épouvante-les par ton tourbillon !
\VS{17}Couvre leurs visages d'ignominie afin qu'on cherche ton Nom, ô Yahweh !
\VS{18}Qu'ils soient honteux et épouvantés à jamais, qu'ils rougissent, et qu'ils périssent ;
\VS{19}afin qu'on sache que toi seul, dont le nom est Yahweh, tu es le Très-Haut sur toute la terre.
\Chap{84}
\TextTitle{Délices pour ceux qui ont Yahweh comme appui}
\VerseOne{}Psaume des fils de Koré, donné au chef des chantres, pour le chanter sur la Guitthith.
\VS{2}Yahweh des armées, que tes demeures sont aimables !
\VS{3}Mon âme soupire et languit après les parvis de Yahweh ; mon cœur et ma chair poussent des cris de joie vers le Dieu vivant.
\VS{4}Le passereau même trouve sa maison, et l'hirondelle son nid où elle a mis ses petits… Tes autels, ô Yahweh des armées ! Mon Roi et mon Dieu !
\VS{5}Heureux ceux qui habitent ta maison et qui te louent sans cesse ! Sélah.
\VS{6}Heureux l'homme dont la force est en toi, ils trouvent dans leur cœur des chemins tout tracés !
\VS{7}Passant par la vallée de Baca, ils la réduisent en fontaine ; la pluie la couvre de bénédictions.
\VS{8}Ils marchent avec force pour se présenter devant Dieu à Sion.
\VS{9}Yahweh Dieu des armées, écoute ma prière, Dieu de Jacob, prête l'oreille. Sélah.
\VS{10}Ô Dieu, notre bouclier, vois et regarde la face de ton oint !
\VS{11}Car mieux vaut un jour dans tes parvis, que mille ailleurs. J'aimerais mieux me tenir à la porte dans la maison de mon Dieu, que de demeurer dans les tentes des méchants.
\VS{12}Car Yahweh Dieu est un soleil et un bouclier\FTNT{Ge. 15:1 ; Ps. 89:19 ; Ps. 144:2.} ; Yahweh donne la grâce et la gloire, et il ne refuse aucun bien à ceux qui marchent dans l'intégrité.
\VS{13}Yahweh des armées, heureux l'homme qui se confie en toi\FTNT{Ps. 2:12.} !
\Chap{85}
\TextTitle{Supplication des rescapés de l'exil}
\VerseOne{}Psaume des fils de Koré, donné au chef des chantres.
\VS{2}Yahweh, tu as été favorable à ta terre, tu as ramené et mis en repos les prisonniers de Jacob.
\VS{3}Tu as pardonné l'iniquité de ton peuple, tu as couvert tous leurs péchés. Sélah.
\VS{4}Tu as retiré toute ta colère, tu es revenu de l'ardeur de ton indignation.
\VS{5}Ô Dieu de notre délivrance, rétablis-nous et fais cesser la colère que tu as contre nous.
\VS{6}Seras-tu irrité à jamais contre nous ? Feras-tu durer ta colère de génération en génération ?
\VS{7}Ne reviendras–tu pas nous rendre la vie\FTNT{Ps. 71:20.}, afin que ton peuple se réjouisse en toi ?
\VS{8}Yahweh, fais-nous voir ta miséricorde et accorde-nous ta délivrance !
\VS{9}J'écouterai ce que dira Dieu, Yahweh ; car il parlera de paix à son peuple et à ses bien-aimés, pourvu que jamais ils ne retournent à leur folie.
\VS{10}Certainement sa délivrance est proche de ceux qui le craignent, la gloire habite dans notre pays.
\VS{11}La bonté et la vérité se rencontrent ; la justice et la paix s'embrassent\FTNT{Hé. 7:2.}.
\VS{12}La vérité germe de la terre et la justice regarde des cieux.
\VS{13}Yahweh aussi donne le bien et notre terre rendra son fruit.
\VS{14}La justice marchera devant lui, et il la mettra partout où il passera.
\Chap{86}
\TextTitle{Coeur disposé à la crainte de Dieu}
\VerseOne{}Prière de David. Yahweh, écoute, réponds-moi, car je suis affligé et misérable.
\VS{2}Garde mon âme, car je suis un de tes bien-aimés ; ô toi mon Dieu, délivre ton serviteur qui se confie en toi !
\VS{3}Seigneur, aie pitié de moi, car je crie à toi tout le jour.
\VS{4}Réjouis l'âme de ton serviteur, car j'élève mon âme à toi, Seigneur.
\VS{5}Yahweh ! Tu es bon et clément, et d'une grande bonté envers tous ceux qui t'invoquent\FTNT{Joë. 2:13.}.
\VS{6}Yahweh, prête l'oreille à ma prière, et sois attentif à la voix de mes supplications.
\VS{7}Je t'invoque au jour de ma détresse, car tu m'exauces\FTNT{Ps. 50:15.}.
\VS{8}Seigneur, nul n'est comme toi parmi les dieux, et rien ne ressemble à tes œuvres\FTNT{De. 3:24 ; Ps. 95:3.}.
\VS{9}Seigneur, toutes les nations que tu as faites viendront et se prosterneront devant toi, et glorifieront ton Nom,
\VS{10}car tu es grand, et tu fais des choses merveilleuses ; tu es Dieu, toi seul.
\VS{11}Yahweh ! Enseigne-moi tes voies et je marcherai dans ta vérité\FTNT{Ps. 25:4 ; Ps. 27:11.} ; lie mon cœur à la crainte de ton Nom.
\VS{12}Seigneur, mon Dieu, je te célébrerai de tout mon cœur, et je glorifierai ton Nom à toujours.
\VS{13}Car ta bonté est grande envers moi, et tu as retiré mon âme du profond scheol.
\VS{14}Ô Dieu ! Des gens orgueilleux se sont élevés contre moi, et un corps d'armée de méchants en veut à ma vie ; ils ne portent pas leurs pensées sur toi.
\VS{15}Mais toi, Seigneur, tu es le Dieu compatissant, miséricordieux, lent à la colère, riche en bonté et en vérité.
\VS{16}Tourne-toi vers moi, et aie pitié de moi ! Donne ta force à ton serviteur, délivre le fils de ta servante !
\VS{17}Accorde–moi un signe de ta faveur, et que ceux qui me haïssent le voient et soient honteux, parce que tu m'aideras, ô Yahweh ! Tu me consoleras !
\Chap{87}
\TextTitle{Sion, la cité de Dieu}
\VerseOne{}Psaume. Cantique des fils de Koré. Elle est fondée sur les montagnes saintes.
\VS{2}Yahweh aime les portes de Sion, plus que toutes les demeures de Jacob.
\VS{3}Ce qui se dit de toi, cité de Dieu, sont des choses glorieuses. Sélah.
\VS{4}Je ferai mention de Rahab et de Babylone parmi ceux qui me connaissent ; voici le pays des philistins, et Tyr avec l'Ethiopie. C'est dans Sion qu'ils sont nés.
\VS{5}Et de Sion il est dit : Un homme y est né ; le Très-Haut lui-même l'établira.
\VS{6}Yahweh compte en inscrivant les peuples : C'est là qu'ils sont nés. Sélah.
\VS{7}Et les chantres, de même que les joueurs de flûte, toutes mes sources sont en toi.
\Chap{88}
\TextTitle{Lamentation dans l'affliction}
\VerseOne{}Cantique. Psaume des fils de Koré, donné au chef des chantres. Pour chanter sur la flûte. Cantique d'Héman, l'Ezrahite.
\VS{2}Yahweh ! Dieu de ma délivrance, je crie jour et nuit devant toi\FTNT{Lu. 18:7.}.
\VS{3}Que ma prière parvienne en ta présence ; étends ton oreille à mon cri.
\VS{4}Car mon âme est rassasiée de maux, et ma vie atteint le scheol\FTNT{Lu. 16:23.}.
\VS{5}On m'a mis au rang de ceux qui descendent dans la fosse\FTNT{Ps. 28:1 ; Ps. 31:13.} ; je suis devenu comme un homme qui n'a plus de vigueur.
\VS{6}Je suis étendu parmi les morts, semblable à ceux qui sont tués et couchés dans la tombe, à ceux dont tu n'as plus le souvenir, et qui sont séparés par ta main.
\VS{7}Tu m'as jeté dans une fosse profonde, dans les ténèbres, dans les abîmes.
\VS{8}Ta fureur se pose sur moi, et tu m'as accablé de tous tes flots. Sélah.
\VS{9}Tu as éloigné de moi ceux de qui j'étais connu, tu m'as mis en abomination devant eux ; je suis enfermé et je ne peux sortir.
\VS{10}Mes yeux se consument dans la souffrance ; Yahweh ! Je crie à toi tout le jour ! J'étends mes mains vers toi\FTNT{Ex. 9:29 ; 1 R. 8:22. ; Job. 17:7} !
\VS{11}Est-ce pour les morts que tu fais des miracles ? Les morts se relèveront-ils pour te célébrer\FTNT{1 Th. 4:16 ; 1 Co. 15:12-13.} ? Sélah.
\VS{12}Parle-t-on de ta bonté dans le sépulcre, de ta fidélité dans le tombeau\FTNT{Ep. 4:9-10 ; 1 Pi. 3:18-20.} ?
\VS{13}Connaîtra-t-on tes merveilles dans les ténèbres et ta justice dans la terre de l'oubli ?
\VS{14}Mais moi, ô Yahweh ! J'implore ton secours, ma prière s'élève dès le matin.
\VS{15}Yahweh ! Pourquoi rejettes-tu mon âme, pourquoi me caches-tu ta face\FTNT{Mt. 27:46 ; Mc. 15:34.} ?
\VS{16}Je suis malheureux et moribond dès ma jeunesse ; j'ai été exposé à tes terreurs, et je ne sais pas où j'en suis.
\VS{17}Les ardeurs de ta colère sont passées sur moi et tes terreurs m'anéantissent\FTNT{Es. 53:5.}.
\VS{18}Elles m'environnent tout le jour comme des eaux, elles m'enveloppent toutes à la fois.
\VS{19}Tu as éloigné de moi mon ami et mon compagnon, mes connaissances ont disparu\FTNT{Mt. 26:56.}.
\Chap{89}
\TextTitle{«Heureux le peuple qui connaît le son de la trompette »}
\VerseOne{}Cantique d'Ethan, l'Ezrachite.
\VS{2}Je chanterai toujours les bontés de Yahweh ; je ferai connaître de ma bouche ta fidélité de génération en génération.
\VS{3}Car je dis : Ta bonté a des fondements éternels, tu établis ta fidélité dans les cieux quand tu dis :
\VS{4}J'ai traité alliance avec mon élu, j'ai fait serment à David mon serviteur :
\VS{5}J'affermirai ta postérité pour toujours, et j'établirai ton trône de génération en génération\FTNT{2 S. 7:8-16.}. Sélah.
\VS{6}Les cieux célèbrent tes merveilles, ô Yahweh ! Ta fidélité aussi est célébrée dans l'assemblée des saints.
\VS{7}Car qui dans le ciel peut se comparer à Yahweh ? Qui est semblable à Yahweh parmi les fils de Dieu ?
\VS{8}Dieu se rend extrêmement terrible dans le conseil secret des saints, il est plus redouté que tous ceux qui sont à l'entour de lui.
\VS{9}Ô Yahweh Dieu des armées ! Qui est semblable à toi, puissant Yahweh ? Aussi ta fidélité t'environne.
\VS{10}Tu domines l'élévation des flots de la mer ; quand ses vagues s'élèvent, tu les calmes\FTNT{Job. 26:12 ; Job. 38:8-12.}.
\VS{11}Tu écrasas Rahab\FTNT{Ce terme hébreu fait référence au nom emblèmatique de l'Egypte, il signifie « largeur », « arrogance ».} comme un homme blessé à mort ; tu dispersas tes ennemis par le bras de ta force.
\VS{12}A toi sont les cieux, à toi aussi est la terre ; tu as fondé le monde, et tout ce qui est en lui.
\VS{13}Tu as créé le nord et le sud ; le Thabor et l'Hermon se réjouissent en ton Nom.
\VS{14}Ton bras est puissant, ta main est forte, ta droite est haut élevée.
\VS{15}La justice et l'équité sont la base de ton trône ; la bonté et la vérité marchent devant ta face.
\VS{16}Heureux le peuple qui connaît le son de la trompette\FTNT{1 Co. 15:52 ; Ap. 10:7.} ! Il marche, ô Yahweh ! A la clarté de ta face.
\VS{17}Il se réjouit chaque jour en ton Nom, et il se glorifie de ta justice.
\VS{18}Parce que tu es la gloire de leur force ; et notre pouvoir est distingué par ta faveur.
\VS{19}Car notre bouclier est Yahweh, et notre Roi est le Saint d'Israël.
\VS{20}Tu as autrefois parlé en vision touchant ton bien-aimé, et tu as dit : J'ai ordonné mon secours en faveur d'un homme vaillant ; j'ai élevé l'élu du milieu du peuple.
\VS{21}J'ai trouvé David mon serviteur, je l'ai oint de ma sainte huile\FTNT{1 S. 16:13 ; Ac. 13:22.} ;
\VS{22}ma main sera ferme avec lui, et mon bras le renforcera.
\VS{23}L'ennemi ne le surprendra point, et l'inique ne l'affligera point ;
\VS{24}mais j'écraserai devant lui ses adversaires, et je détruirai ceux qui le haïssent.
\VS{25}Ma fidélité et ma bonté seront avec lui, et sa gloire sera élevée en mon Nom.
\VS{26}Je mettrai sa main sur la mer, et sa droite sur les fleuves.
\VS{27}Il m'invoquera : Tu es mon Père, mon Dieu, et le rocher\FTNT{Voir commentaire en Es. 8:14.} de ma délivrance.
\VS{28}Aussi je ferai de lui le premier-né\FTNT{Col. 1:15.}, le plus élevé des rois de la terre.
\VS{29}Je lui garderai ma bonté à toujours, et mon alliance lui sera assurée.
\VS{30}Je rendrai éternelle sa postérité, et son trône comme les jours des cieux.
\VS{31}Mais si ses fils abandonnent ma loi, et ne marchent point selon mes ordonnances ;
\VS{32}s'ils violent mes statuts, et qu'ils ne gardent point mes commandements ;
\VS{33}je punirai de la verge leur transgression, et de plaie leur iniquité.
\VS{34}Mais je ne retirerai point de lui ma bonté, et je ne lui trahirai point ma fidélité.
\VS{35}Je ne violerai point mon alliance, et je ne changerai point ce qui est sorti de mes lèvres.
\VS{36}J'ai une fois juré par ma sainteté : Mentirais-je à David\FTNT{Hé. 6:13.} ?
\VS{37}Sa postérité sera à toujours, et son trône sera devant moi comme le soleil.
\VS{38}Il aura une durée éternelle comme la lune ; le témoin qui est dans le ciel est fidèle. Sélah.
\VS{39}Néanmoins, tu l'as rejeté et dédaigné\FTNT{Es. 53:3.} ; tu t'es mis en grande colère contre ton oint.
\VS{40}Tu as rejeté l'alliance faite avec ton serviteur ; tu as souillé sa couronne en la jetant par terre.
\VS{41}Tu as rompu toutes ses murailles ; tu as mis en ruines ses forteresses.
\VS{42}Tous ceux qui passaient par le chemin l'ont pillé ; il a été mis en opprobre à ses voisins.
\VS{43}Tu as élevé la droite de ses adversaires, tu as réjoui tous ses ennemis.
\VS{44}Tu as fait reculer le tranchant de son épée, et tu ne l'as point élevé dans le combat.
\VS{45}Tu as fait cesser sa splendeur, et tu as jeté par terre son trône.
\VS{46}Tu as abrégé les jours de sa jeunesse et l'as couvert de honte. Sélah.
\VS{47}Jusqu'à quand, ô Yahweh ? Te cacheras-tu à jamais ? Ta fureur s'embrasera-t-elle comme un feu ?
\VS{48}Souviens-toi quelle est la durée de ma vie ; pourquoi aurais-tu créé en vain tous les fils des hommes ?
\VS{49}Qui est l'homme qui vivra et ne verra point la mort, et qui garantira son âme de la main du scheol\FTNT{1 Co. 15:54-57.} ? Sélah.
\VS{50}Seigneur, où sont tes bontés premières que tu juras à David dans ta fidélité ?
\VS{51}Seigneur ! Souviens-toi de l'opprobre de tes serviteurs, et comment je porte dans mon sein l'opprobre qui nous a été fait par tous les peuples nombreux.
\VS{52}Souviens-toi des outrages de tes ennemis, ô Yahweh ! Des outrages contre les pas de ton oint.
\VS{53}Béni soit à toujours Yahweh ; amen ! Oui, amen !
\Chap{90}
\TextTitle{Mortalité de l'homme}
\VerseOne{}Prière de Moïse, homme de Dieu\FTNT{De. 33:1.}. Seigneur ! Tu as été pour nous un refuge de génération en génération.
\VS{2}Avant que les montagnes soient nées et que tu aies formé la terre et le monde, d'éternité en éternité, tu es Dieu\FTNT{Ge. 17:1 ; Es. 40:28.}.
\VS{3}Tu fais revenir l'homme à la poussière, et tu dis : Fils des hommes, retournez\FTNT{Ge. 3:19 ; Ec. 12:7.} !
\VS{4}Car mille ans sont à tes yeux comme le jour d'hier qui est passé, et comme une veille de la nuit\FTNT{Ps. 39:5 ; 2 Pi. 3:8.}.
\VS{5}Tu les emportes semblables à un songe qui, le matin, passe comme l'herbe :
\VS{6}Elle fleurit au matin et reverdit ; le soir on la coupe et elle se fane\FTNT{1 Pi. 1:24.}.
\VS{7}Car nous sommes consumés par ta colère et nous sommes troublés par ta fureur.
\VS{8}Tu as mis devant toi nos iniquités, et à la lumière de ta face nos fautes cachées.
\VS{9}Car tous nos jours s'en vont par ta grande colère, et nos années se consument dans un soupir.
\VS{10}Les jours de nos années reviennent à soixante-dix ans, et pour les plus forts, à quatre-vingts ans ; l'orgueil qu'ils en tirent n'est que peine et misère ; car il passe vite, et nous nous envolons.
\VS{11}Qui connaît, selon ta crainte, la force de ton indignation et de ta grande colère ?
\VS{12}Enseigne-nous à compter nos jours, afin que nous puissions avoir un cœur rempli de sagesse.
\VS{13}Yahweh ! Reviens ! Jusqu'à quand ? Sois apaisé envers tes serviteurs.
\VS{14}Rassasie-nous chaque matin de ta bonté, afin que nous nous réjouissions et que nous soyons joyeux tout le long de nos jours.
\VS{15}Réjouis-nous autant de jours que tu nous as affligés, autant d'années que nous avons vu le malheur.
\VS{16}Que ton œuvre se voie sur tes serviteurs, et ta gloire sur leurs fils.
\VS{17}Que la grâce de Yahweh, notre Dieu, soit sur nous, et affermis l'œuvre de nos mains ; oui, affermis l'œuvre de nos mains !
\Chap{91}
\TextTitle{La sécurité et la fidélité de Yahweh}
\VerseOne{}Celui qui demeure sous la couverture\FTNT{Jésus est notre couverture spirituelle.} du Très-Haut, repose à l'ombre du Tout-Puissant.
\VS{2}Je dis à Yahweh : Tu es ma retraite et ma forteresse, tu es mon Dieu en qui je me confie.
\VS{3}Certes, il te délivre du filet de l'oiseleur, de la peste et de ses ravages.
\VS{4}Il te couvrira de ses plumes et tu trouveras un refuge sous ses ailes ; sa fidélité est un bouclier et une cuirasse.
\VS{5}Tu ne craindras ni les terreurs de la nuit, ni la flèche qui vole le jour\FTNT{Pr. 3:23-24.},
\VS{6}ni la peste qui marche dans les ténèbres, ni la destruction qui frappe en plein midi.
\VS{7}Que mille tombent à ton côté et dix mille à ta droite, tu ne seras pas atteint.
\VS{8}De tes yeux tu regarderas et tu verras la rétribution des méchants.
\VS{9}Car tu es mon refuge, ô Yahweh ! Tu fais du Très-Haut ta demeure.
\VS{10}Aucun malheur ne s'approchera de toi, aucun fléau n'approchera de ta tente\FTNT{Ex. 8:18-19 ; Ps. 121:6-8.}.
\VS{11}Car il ordonnera à ses anges de te garder dans toutes tes voies.
\VS{12}Ils te porteront sur les mains, de peur que ton pied ne heurte contre une pierre\FTNT{Mt. 4:5-6 ; Lu. 4:9-11.}.
\VS{13}Tu marcheras sur le lion et sur l'aspic, tu piétineras le lionceau et le dragon.
\VS{14}Puisqu'il m'aime, je le délivrerai ; je le mettrai sur les hauteurs, parce qu'il connaît mon Nom.
\VS{15}Il m'invoquera et je l'exaucerai ; je serai avec lui dans la détresse, je le délivrerai et le glorifierai.
\VS{16}Je le rassasierai de jours et je lui ferai voir ma délivrance.
\Chap{92}
\TextTitle{Proclamer la louange de Dieu}
\VerseOne{}Psaume. Cantique pour le jour du sabbat.
\VS{2}C'est une belle chose que de célébrer Yahweh, et de chanter ton Nom, ô Très-Haut\FTNT{Ps. 147:1.} !
\VS{3}Afin d'annoncer chaque matin ta bonté et ta fidélité toutes les nuits\FTNT{Ps. 59:17 ; Ps. 88:14 ; Ps. 89:2.}.
\VS{4}Sur l'instrument à dix cordes, sur le luth, et par un cantique prémédité sur la harpe.
\VS{5}Car, ô Yahweh ! Tu me réjouis par tes œuvres, je me réjouis des œuvres de tes mains.
\VS{6}Ô Yahweh ! Que tes œuvres sont magnifiques ! Tes pensées sont merveilleusement profondes\FTNT{Es. 55:8-9 ; Job. 5:9.}.
\VS{7}L'homme stupide n'y connaît rien et le fou n'y prend point garde\FTNT{Es. 5:12 ; Ro. 1:21.}.
\VS{8}Les méchants croissent comme l'herbe, et tous les ouvriers d'iniquité fleurissent pour être exterminés éternellement\FTNT{Jé. 12:1-2 ; Mal. 3:15 ; Ps. 37:2 ; Ps. 73:1-20.}.
\VS{9}Mais toi, ô Yahweh ! Tu es élevé à toujours.
\VS{10}Car voici tes ennemis, ô Yahweh ! Car voici, tes ennemis périssent, tous les ouvriers d'iniquité sont dispersés.
\VS{11}Mais tu élèveras ma corne comme celle d'un buffle, je serai oint d'une huile fraîche\FTNT{Ps. 23:5 ; Hé. 1:9.}.
\VS{12}Mes yeux se plaisent à regarder ceux qui m'épient, et mes oreilles à entendre les méchants qui s'élèvent contre moi.
\VS{13}Le juste fleurit comme le palmier, il croît comme le cèdre au Liban.
\VS{14}Etant plantés dans la maison de Yahweh, ils fleurissent dans les parvis de notre Dieu.
\VS{15}Ils portent encore des fruits dans la blanche vieillesse ; ils sont gras et verdoyants\FTNT{Os. 14:6 ; Ps. 1:3.},
\VS{16}afin d'annoncer que Yahweh est droit ; c'est mon rocher, et il n'y a point d'injustice en lui.
\Chap{93}
\TextTitle{Majesté et puissance de Yahweh}
\VerseOne{}Yahweh règne, il est revêtu de majesté ; Yahweh est revêtu de force, il s'en est ceint ; aussi le monde est ferme, tellement qu'il ne sera point ébranlé.
\VS{2}Ton trône est établi dès lors, tu es de toute éternité\FTNT{Ps. 9:8 ; Hé. 1:8.}.
\VS{3}Les fleuves élevés, ô Yahweh ! Les fleuves augmentent leur bruit, les fleuves élèvent leurs flots\FTNT{Ps. 46:4 ; Ps. 65:7-8.} ;
\VS{4}Yahweh, qui est dans les lieux élevés, est plus puissant que le bruit des grandes eaux, et que les fortes vagues de la mer\FTNT{Es. 57:15 ; Ac. 7:49.}.
\VS{5}Tes préceptes sont entièrement fidèles. Yahweh ! La sainteté orne ta maison pour une longue durée.
\Chap{94}
\TextTitle{A Dieu seul la vengeance}
\VerseOne{}Ô Yahweh ! Dieu des vengeances, Dieu des vengeances, fais briller ta splendeur !
\VS{2}Toi, juge de la terre, élève-toi ! Rends aux orgueilleux selon leurs œuvres.
\VS{3}Jusqu'à quand les méchants, ô Yahweh ! Jusqu'à quand les méchants se réjouiront-ils ?
\VS{4}Jusqu'à quand tous les ouvriers d'iniquité discourront-ils et diront-ils des paroles rudes et se vanteront-ils ?
\VS{5}Yahweh, ils écrasent ton peuple et affligent ton héritage.
\VS{6}Ils tuent la veuve et l'étranger, et ils mettent à mort les orphelins.
\VS{7}Ils disent : Yahweh ne le voit point, le Dieu de Jacob n'entend rien.
\VS{8}Vous les plus abrutis d'entre les peuples, prenez garde à ceci ; et vous insensés, quand serez-vous intelligents ?
\VS{9}Celui qui a planté l'oreille, n'entendrait-il point ? Celui qui a formé l'œil, ne verrait-t-il point\FTNT{Ex. 4:11 ; Pr. 20:12.} ?
\VS{10}Celui qui châtie les nations, celui qui enseigne la science aux hommes, ne réprimanderait-il point\FTNT{Ap. 19:15.} ?
\VS{11}Yahweh connaît les pensées des hommes qui ne sont que vanité.
\VS{12}Heureux l'homme que tu châties, ô Yahweh\FTNT{Hé. 12:6.} ! Que tu instruis par ta loi,
\VS{13}afin qu'il soit dans la paix aux jours du malheur, jusqu'à ce que la fosse soit creusée pour le méchant !
\VS{14}Car Yahweh ne délaisse point son peuple et n'abandonne point son héritage\FTNT{Es. 49:15 ; Ro. 11:2.}.
\VS{15}C'est pourquoi le jugement s'unira à la justice, et tous ceux qui sont droits de cœur le suivront.
\VS{16}Qui se lèvera pour moi contre les méchants\FTNT{Job. 19:25 ; Ro. 8:31.} ? Qui m'assistera contre les ouvriers d'iniquité ?
\VS{17}Si Yahweh n'était pas mon secours, mon âme serait bien vite dans la demeure du silence.
\VS{18}Quand je dis : Mon pied chancelle, ta bonté me soutient, ô Yahweh !
\VS{19}Quand j'ai beaucoup de pensées au-dedans de moi, tes consolations font les délices de mon âme.
\VS{20}Serais–tu l'allié du trône de méchanceté, qui forge des injustices contre les règles de la justice ?
\VS{21}Ils se rassemblent contre l'âme du juste et condamnent le sang innocent\FTNT{Mt. 27:1-4 ; Mt. 27:24.}.
\VS{22}Or Yahweh est pour moi une haute retraite ; mon Dieu est le rocher de mon refuge.
\VS{23}Il fera retourner sur eux leur iniquité et les détruira par leur propre méchanceté. Yahweh notre Dieu les détruira\FTNT{Mt. 13:30 ; Ap. 20:14-15.}.
\Chap{95}
\TextTitle{Adoration à Yahweh}
\VerseOne{}Venez, chantons à Yahweh, poussons des cris de réjouissance au rocher de notre salut.
\VS{2}Allons au-devant de lui en lui présentant nos louanges ; et poussons devant lui des cris de réjouissance en chantant des psaumes.
\VS{3}Car Yahweh est un grand Dieu, et il est un grand Roi au-dessus de tous les dieux.
\VS{4}Les lieux les plus profonds de la terre sont dans sa main, et les sommets des montagnes sont à lui.
\VS{5}C'est à lui qu'appartient la mer, car lui-même l'a faite, et ses mains ont formé la terre.
\VS{6}Venez, prosternons-nous, inclinons-nous, et mettons-nous à genoux devant Yahweh qui nous a faits\FTNT{Ps. 96:9 ; Ph. 2:10-11.}.
\VS{7}Car il est notre Dieu, et nous sommes le peuple de son pâturage, et les brebis que sa main conduit\FTNT{Ps. 23:1 ; Ps. 100:3 ; Jn. 10:11.}. Si vous entendez aujourd'hui sa voix,
\VS{8}n'endurcissez point votre cœur\FTNT{Hé. 3:8 ; Hé. 4:7.}, comme à Meriba, comme à la journée de Massa, au désert ;
\VS{9}là où vos pères m'ont tenté et éprouvé bien qu'ils virent mes œuvres\FTNT{Ex. 17:7.}.
\VS{10}J'ai eu cette génération en dégoût durant quarante ans, et j'ai dit : C'est un peuple dont le cœur s'égare ; et ils n'ont point connu mes voies ;
\VS{11}c'est pourquoi j'ai juré dans ma colère, ils n'entreront pas dans mon repos\FTNT{No. 14:22-23 ; Hé. 3:15-19 ; Hé. 4:3.}.
\Chap{96}
\TextTitle{La grandeur et la gloire de Dieu}
\VerseOne{}Chantez à Yahweh un cantique nouveau\FTNT{Es. 42:10 ; Ps. 98:1 ; Ap. 5:9 ; Ap. 14:3.} ! Vous tous habitants de la terre chantez à Yahweh !
\VS{2}Chantez à Yahweh, bénissez son Nom ! Prêchez de jour en jour sa délivrance !
\VS{3}Racontez sa gloire parmi les nations, ses merveilles parmi tous les peuples\FTNT{Ps. 67:5.}.
\VS{4}Car Yahweh est grand et digne d'être loué ; il est redoutable au-dessus de tous les dieux\FTNT{Ph. 2:9 ; Ap. 5:9.} ;
\VS{5}car tous les dieux des peuples ne sont que des idoles, mais Yahweh a fait les cieux.
\VS{6}La splendeur et la magnificence marchent devant lui, la force et la beauté sont dans son lieu saint.
\VS{7}Familles des peuples, rendez à Yahweh, rendez à Yahweh la gloire et la puissance !
\VS{8}Rendez à Yahweh la gloire due à son Nom ! Apportez des offrandes et entrez dans ses parvis !
\VS{9}Prosternez–vous devant Yahweh avec des ornements sacrés ; tremblez devant lui, vous toute la terre !
\VS{10}Dites parmi les nations : Yahweh règne ; même le monde est affermi, il ne sera point ébranlé ; il jugera les peuples avec équité.
\VS{11}Que les cieux se réjouissent et que la terre soit dans l'allégresse ! Que la mer tonne avec tout ce qui la remplit !
\VS{12}Que les champs s'égayent avec tout ce qui est en eux. Alors tous les arbres de la forêt chanteront de joie
\VS{13}devant Yahweh, car il vient ! Car il vient pour juger la terre ; il jugera avec justice le monde habitable et les peuples selon sa fidélité.
\Chap{97}
\TextTitle{Aimer Dieu, c'est haïr le mal}
\VerseOne{}Yahweh règne, que la terre soit dans l'allégresse et que les îles nombreuses s'en réjouissent\FTNT{Es. 42:10 ; Ps. 86:9 ; Ps. 93:1 ; Ps. 99:1} !
\VS{2}La nuée et l'obscurité sont autour de lui ; la justice et le jugement sont la base de son trône.
\VS{3}Le feu marche devant lui et embrase tout autour ses adversaires.
\VS{4}Ses éclairs éclairent le monde, et la terre le voit et tremble tout étonnée\FTNT{Job. 38:35 ; Ap. 4:5.}.
\VS{5}Les montagnes se fondent comme de la cire\FTNT{Mi. 1:4.}, à cause de la présence de Yahweh, à cause de la présence du Seigneur de toute la terre.
\VS{6}Les cieux annoncent sa justice et tous les peuples voient sa gloire.
\VS{7}Que tous ceux qui servent les images et qui se glorifient des idoles soient confus\FTNT{De. 4:25-26 ; 1 S. 5:1-5.} ; vous dieux, prosternez-vous tous devant lui.
\VS{8}Sion l'a entendu et s'en est réjouie ; et les filles de Juda se sont égayées pour l'amour de tes jugements, ô Yahweh !
\VS{9}Yahweh, tu es le Très-Haut sur toute la terre ; tu es élevé au-dessus de tous les dieux.
\VS{10}Vous qui aimez Yahweh, haïssez le mal\FTNT{Am. 5:14-15 ; Ro. 12:9.} ! Il garde les âmes de ses bien-aimés et les délivre de la main des méchants\FTNT{Ps. 34:8 ; Jn. 10:28-29.}.
\VS{11}La lumière est faite pour le juste\FTNT{Mt. 5:15-16.} et la joie pour ceux qui sont droits de cœur.
\VS{12}Justes, réjouissez-vous en Yahweh et célébrez la mémoire de sa sainteté.
\Chap{98}
\TextTitle{Invitation à la louange}
\VerseOne{}Psaume. Chantez à Yahweh un cantique nouveau, car il a fait des choses merveilleuses ; sa droite et le bras de sa sainteté l'ont délivré\FTNT{Es. 52:10 ; Es 53:1 ; Es. 63:3-5.}.
\VS{2}Yahweh a fait connaître son salut\FTNT{Il est question de la révélation de Jésus. Voir commentaire en Es. 26:1.}, il a révélé sa justice devant les yeux des nations.
\VS{3}Il s'est souvenu de sa bonté et de sa fidélité envers la maison d'Israël ; toutes les extrémités de la terre ont vu la délivrance de notre Dieu\FTNT{Es. 49:6 ; Lu. 1:72 ; Ac. 13:47.}.
\VS{4}Vous tous habitants de la terre, poussez des cris de réjouissance à Yahweh ! Faites retentir vos cris et chantez de joie !
\VS{5}Chantez à Yahweh avec la harpe, avec la harpe et avec une voix mélodieuse !
\VS{6}Poussez des cris de réjouissance avec le shofar au son du cor devant le Roi, Yahweh !
\VS{7}Que la mer tonne avec tout ce qu'elle contient, que la terre et ceux qui y habitent fassent éclater leurs cris !
\VS{8}Que les fleuves frappent des mains et que les montagnes chantent de joie
\VS{9}devant Yahweh ! Car il vient pour juger la terre\FTNT{Yahweh, qui vient pour juger la terre, est Jésus-Christ ( 
Za. 14:1-7 ; 2 Ti. 4:1 ; Ap. 19:15).} ; il jugera le monde habitable avec justice et les peuples avec équité.
\Chap{99}
\TextTitle{Grandeur, justice, et sainteté de Dieu}
\VerseOne{}Yahweh règne, que les peuples tremblent ; il est assis entre les chérubins, que la terre soit ébranlée\FTNT{Ex. 25:22 ; Es. 37:16.}.
\VS{2}Yahweh est grand en Sion, et il est élevé au-dessus de tous les peuples.
\VS{3}Ils célébreront ton Nom, grand et terrible, car il est saint.
\VS{4}Qu'on célèbre la force du roi qui aime la justice ! Tu as ordonné l'équité, tu as prononcé des jugements justes en Jacob.
\VS{5}Exaltez Yahweh notre Dieu et prosternez-vous devant son marchepied ! Il est saint !
\VS{6}Moïse et Aaron étaient parmi ses sacrificateurs\FTNT{Ex. 31:10 ; Lé. 2:2.} ; et Samuel parmi ceux qui invoquaient son Nom ; ils invoquaient Yahweh et il leur répondait\FTNT{1 S. 12:18-19.}.
\VS{7}Il leur parlait de la colonne de nuée ; ils ont gardé ses préceptes et l'ordonnance qu'il leur avait donnée.
\VS{8}Ô Yahweh, mon Dieu ! Tu les as exaucés, tu as été pour eux un Dieu qui pardonne\FTNT{Hé. 10:16-17.}, mais tu les as punis de leurs fautes.
\VS{9}Exaltez Yahweh notre Dieu ! Prosternez-vous sur la montagne de sa sainteté ! Car Yahweh, notre Dieu est saint !
\Chap{100}
\TextTitle{Célébrer et bénir le nom de Yahweh}
\VerseOne{}Psaume de louange. Vous tous habitants de la terre, poussez des cris de réjouissance à Yahweh !
\VS{2}Servez Yahweh avec allégresse, venez devant lui avec un chant de joie !
\VS{3}Sachez que Yahweh est Dieu. C'est lui qui nous a faits, ce n'est pas nous qui nous sommes faits ; nous sommes son peuple et le troupeau de son pâturage\FTNT{Ps. 79:13 ; Ps. 80:2 ; Ps. 95:6 ; Ps. 119:73.}.
\VS{4}Entrez dans ses portes avec des louanges ; et dans ses parvis, avec des cantiques. Célébrez-le, bénissez son Nom !
\VS{5}Car Yahweh est bon ; sa bonté demeure à toujours et sa fidélité de génération en génération.
\Chap{101}
\TextTitle{Appel à l'intégrité}
\VerseOne{}Psaume de David. Je chanterai la miséricorde et la justice ; Yahweh ! Je te chanterai.
\VS{2}Je me rendrai attentif à une conduite pure jusqu'à ce que tu viennes à moi ; je marcherai dans l'intégrité de mon cœur au milieu de ma maison.
\VS{3}Je ne mettrai point devant mes yeux des choses de Bélial\FTNT{Ps. 26:5 ; Ps. 119:115.}; j'ai en haine les actions de ceux qui se détournent ; elles ne s'attacheront pas à moi.
\VS{4}Le cœur mauvais s'éloignera de moi ; je ne connaîtrai pas le méchant.
\VS{5}Je retrancherai celui qui calomnie en secret son prochain ; je ne supporterai pas celui qui a les yeux élevés et le cœur enflé\FTNT{Pr. 6:16-17.}.
\VS{6}Je prendrai garde aux gens de bien du pays afin qu'ils demeurent avec moi ; celui qui marche dans la voie de l'intégrité me servira.
\VS{7}Celui qui usera de tromperie ne demeurera point dans ma maison ; celui qui profèrera des mensonges ne sera point affermi devant mes yeux.
\VS{8}Je retrancherai chaque matin tous les méchants du pays, afin d'exterminer de la cité de Yahweh tous les ouvriers d'iniquité.
\Chap{102}
\TextTitle{Yahweh, le Dieu immuable}
\VerseOne{}Prière de l'affligé étant dans l'angoisse et répandant sa plainte devant Yahweh.
\VS{2}Yahweh ! Ecoute ma prière, et que mon cri parvienne jusqu'à toi\FTNT{Ps. 69:14.}.
\VS{3}Ne cache pas ta face arrière de moi ; au jour où je suis en détresse, prête l'oreille à ma prière ; au jour où je t'invoque, hâte-toi de me répondre.
\VS{4}Car mes jours se sont évanouis comme la fumée et mes os brûlent comme dans un foyer.
\VS{5}Mon cœur est frappé et se dessèche comme l'herbe, car j'ai oublié de manger mon pain\FTNT{Mt. 4:4 ; Lu. 4:4.}.
\VS{6}Le gémissement de ma voix est tel que mes os s'attachent à ma chair\FTNT{Job. 19:20.}.
\VS{7}Je suis devenu semblable au pélican du désert ; et je suis comme la chouette des lieux sauvages.
\VS{8}Je veille et je suis semblable au passereau solitaire sur le toit.
\VS{9}Mes ennemis m'outragent tous les jours, et ceux qui sont furieux contre moi jurent contre moi.
\VS{10}Car j'ai mangé la cendre comme le pain et j'ai mêlé des larmes à ma boisson,
\VS{11}à cause de ta colère et de ta fureur ; car après m'avoir soulevé, tu m'as jeté par terre.
\VS{12}Mes jours sont comme l'ombre qui décline et je deviens sec comme l'herbe.
\VS{13}Mais toi, ô Yahweh ! Tu demeures éternellement, et ta mémoire est de génération en génération.
\VS{14}Tu te lèveras, tu auras compassion de Sion ; car il est temps d'en avoir pitié, parce que le temps assigné est échu.
\VS{15}Car tes serviteurs aiment ses pierres et chérissent sa poussière.
\VS{16}Alors les nations redouteront le Nom de Yahweh, et tous les rois de la terre ta gloire.
\VS{17}Quand Yahweh aura édifié Sion, quand il aura été vu dans sa gloire,
\VS{18}quand il aura eu égard à la prière du désolé et qu'il n'aura point méprisé leur supplication.
\VS{19}Cela sera enregistré pour la génération à venir, le peuple qui sera créé louera Yahweh !
\VS{20}Car il regarde du lieu élevé de sa sainteté. Du haut des cieux, Yahweh regarde la terre,
\VS{21}pour entendre le gémissement des prisonniers, pour délier ceux qui étaient voués à la mort\FTNT{Es. 42:6-7 ; Es. 61:1 ; Lu. 4:18-19.},
\VS{22}afin qu'on annonce le Nom de Yahweh dans Sion et sa louange dans Jérusalem,
\VS{23}quand les peuples se seront joints ensemble, et les royaumes aussi, pour servir Yahweh.
\VS{24}Il a brisé ma force en chemin, il a abrégé mes jours.
\VS{25}Je dis : Mon Dieu, ne m'enlève point au milieu de mes jours dont les années durent éternellement.
\VS{26}Tu as jadis fondé la terre, et les cieux sont l'ouvrage de tes mains.
\VS{27}Ils périront, mais tu subsisteras, ils s'useront tous comme un vêtement ; tu les changeras comme un habit, et ils seront changés.
\VS{28}Mais toi, tu es toujours le même et tes années ne seront jamais achevées.
\VS{29}Les fils de tes serviteurs habiteront près de toi et leur postérité sera établie devant toi.
\Chap{103}
\TextTitle{Yahweh, le Dieu miséricordieux et compatissant}
\VerseOne{}Psaume de David. Mon âme, bénis Yahweh, et que tout ce qui est en moi bénisse son saint Nom.
\VS{2}Mon âme, bénis Yahweh, et n'oublie pas un de ses bienfaits\FTNT{De. 6:12.}.
\VS{3}C'est lui qui pardonne toutes tes iniquités, qui guérit toutes tes infirmités\FTNT{Es. 33:24 ; Es. 53:5 ; Jé. 17:14 ; Ps. 130:3-4 ; Mt. 9:6 ; Lu. 7:47.} ;
\VS{4}qui garantit ta vie de la fosse\FTNT{Es. 59:20 ; Ps. 106:10.}, qui te couronne de bonté et de compassions ;
\VS{5}qui rassasie ta bouche de biens ; ta jeunesse est renouvelée comme celle de l'aigle\FTNT{Es. 40:31.}.
\VS{6}Yahweh fait justice et droit à tous les opprimés\FTNT{Ps. 146:7.}.
\VS{7}Il a fait connaître ses voies à Moïse, et ses exploits aux fils d'Israël\FTNT{Ex. 33:12-17.}.
\VS{8}Yahweh est compatissant, miséricordieux, lent à la colère, et riche en bonté.
\VS{9}Il ne conteste pas éternellement, et il ne garde point à toujours sa colère\FTNT{Es. 57:16 ; Jé. 3:5 ; Mi. 7:18.}.
\VS{10}Il ne nous traite pas selon nos péchés, et ne nous rend point selon nos iniquités\FTNT{Esd. 9:13.}.
\VS{11}Car autant les cieux sont élevés au-dessus de la terre, autant sa bonté est grande sur ceux qui le craignent.
\VS{12}Il éloigne de nous nos transgressions, autant que l'orient est éloigné de l'occident\FTNT{Es. 38:17.}.
\VS{13}Comme un père a compassion de ses fils, Yahweh a compassion de ceux qui le craignent\FTNT{Mal. 3:17 ; Lu. 11:11-13.}.
\VS{14}Car il sait bien de quoi nous sommes faits, se souvenant que nous ne sommes que poussière.
\VS{15}L'homme ! Ses jours sont comme l'herbe, il fleurit comme la fleur d'un champ.
\VS{16}Car le vent étant passé par-dessus, elle n'est plus, et son lieu ne la reconnaît plus.
\VS{17}Mais la miséricorde de Yahweh est de tout temps, et elle sera pour toujours en faveur de ceux qui le craignent ; et sa justice en faveur des fils de leurs fils ;
\VS{18}pour ceux qui gardent son alliance, et qui se souviennent de ses commandements pour les faire\FTNT{De. 7:9.}.
\VS{19}Yahweh a établi son trône dans les cieux, et son règne domine sur tout.
\VS{20}Bénissez Yahweh, vous ses anges puissants en force, qui faites ses affaires, en obéissant à la voix de sa parole.
\VS{21}Bénissez Yahweh, vous toutes ses armées, qui êtes ses serviteurs faisant sa volonté.
\VS{22}Bénissez Yahweh, vous toutes ses œuvres, par tous les lieux de sa domination. Mon âme, bénis Yahweh !
\Chap{104}
\TextTitle{Yahweh, le Dieu de toute la création}
\VerseOne{}Mon âme, bénis Yahweh. Ô Yahweh mon Dieu, tu es merveilleusement grand, tu es revêtu de majesté et de splendeur.
\VS{2}Il s'enveloppe de lumière comme d'un vêtement, il étend les cieux comme un voile\FTNT{Es. 40:22 ; Job. 9:8 ; 1 Ti. 6:16.}.
\VS{3}Avec les eaux, il va à la rencontre de sa demeure ; il fait des grosses nuées son char, il se promène sur les ailes du vent\FTNT{Es. 19:1 ; Ps. 18:10 ; Ap. 14:14.}.
\VS{4}Il fait des vents ses messagers, et des flammes de feu ses serviteurs\FTNT{Ps. 148:8 ; Hé. 1:7 ; Jn. 3:8.}.
\VS{5}Il a fondé la terre sur ses bases, elle ne sera jamais ébranlée\FTNT{Ps. 24:1-2 ; Ps. 78:69 ; Ps. 93:1 ; Job. 26:7 ; Job. 38:4-6.}.
\VS{6}Tu l'avais couverte de l'abîme comme d'un vêtement, les eaux se tenaient sur les montagnes\FTNT{Ge. 1:2.}.
\VS{7}Elles s'enfuirent à ta menace et se mirent promptement en fuite au son de ton tonnerre.
\VS{8}Les montagnes s'élevèrent et les vallées s'abaissèrent au même lieu que tu leur avais fixé.
\VS{9}Tu as posé une limite que les eaux ne doivent point franchir, afin qu'elles ne reviennent plus couvrir la terre\FTNT{Ge. 1:9 ; Jé. 5:2 ; Pr. 8:29 ; Job. 26:10.}.
\VS{10}C'est lui qui conduit les sources par les vallées, elles se promènent entre les monts.
\VS{11}Elles abreuvent toutes les bêtes des champs, les ânes sauvages y étanchent leur soif.
\VS{12}Les oiseaux des cieux se tiennent auprès d'elles, et font résonner leur voix parmi les rameaux.
\VS{13}Il abreuve les montagnes de ses chambres hautes ; la terre est rassasiée du fruit de tes œuvres.
\VS{14}Il fait germer l'herbe pour le bétail, et les plantes pour le besoin de l'homme, faisant sortir le pain de la terre,
\VS{15}et le vin qui réjouit le cœur de l'homme\FTNT{Jg. 9:11 ; Pr. 31:6-7.}, qui fait resplendir son visage avec l'huile, et qui soutient le cœur de l'homme avec le pain.
\VS{16}Les hauts arbres de Yahweh en sont rassasiés, ainsi que les cèdres du Liban qu'il a plantés,
\VS{17}afin que les oiseaux y fassent leurs nids. Quant à la cigogne, les sapins sont sa demeure.
\VS{18}Les hautes montagnes sont pour les chamois, et les rochers sont la retraite des lapins.
\VS{19}Il a fait la lune pour les saisons, et le soleil sait quand il doit se coucher\FTNT{Ge. 1:16.}.
\VS{20}Tu amènes les ténèbres, et il fait nuit ; alors toutes les bêtes de la forêt sont en mouvement.
\VS{21}Les lionceaux rugissent après la proie pour demander à Dieu leur nourriture.
\VS{22}Le soleil se lève-t-il ? Ils se retirent et se couchent dans leurs tanières.
\VS{23}Alors l'homme sort pour se rendre à son ouvrage, et à son travail jusqu'au soir.
\VS{24}Ô Yahweh, que tes œuvres sont en grand nombre ! Tu les as toutes faites avec sagesse ; la terre est pleine de tes richesses.
\VS{25}Cette mer grande et spacieuse, là où des animaux sans nombre se meuvent, des petites bêtes avec des grandes !
\VS{26}Là se promènent les navires, et ce léviathan que tu as formé pour jouer dans les flots.
\VS{27}Ils s'attendent tous à toi afin que tu leur donnes la nourriture en leur temps.
\VS{28}Quand tu la leur donnes, ils la recueillent, et quand tu ouvres ta main, ils sont rassasiés de biens.
\VS{29}Caches-tu ta face ? Ils sont troublés ; retires-tu leur souffle ? Ils défaillent et retournent dans leur poussière.
\VS{30}Tu envoies ton souffle, ils sont créés ; et tu renouvelles la face de la terre.
\VS{31}Que la gloire de Yahweh subsiste à toujours, que Yahweh se réjouisse dans ses œuvres !
\VS{32}Il jette son regard sur la terre et elle tremble ; il touche les montagnes et elles fument.
\VS{33}Je chanterai à Yahweh durant ma vie ; je chanterai à mon Dieu tant que j'existerai.
\VS{34}Ma méditation lui sera agréable, et je me réjouirai en Yahweh.
\VS{35}Que les pécheurs soient consumés de dessus la terre et qu'il n'y ait plus de méchants ! Mon âme, bénis Yahweh ! Louez Yahweh !
\Chap{105}
\TextTitle{Yahweh, le Dieu fidèle}
\VerseOne{}Célébrez Yahweh, invoquez son Nom, faites connaître parmi les peuples ses exploits.
\VS{2}Chantez-le, chantez-le, parlez de toutes ses merveilles !
\VS{3}Glorifiez-vous de son saint Nom, et que le cœur de ceux qui cherchent Yahweh se réjouisse.
\VS{4}Recherchez Yahweh et sa puissance ; cherchez continuellement sa face.
\VS{5}Souvenez-vous de ses merveilles qu'il a faites, de ses miracles, et des jugements de sa bouche.
\VS{6}La postérité d'Abraham sont ses serviteurs ; les enfants de Jacob sont ses élus !
\VS{7}Il est Yahweh notre Dieu, ses jugements sont sur toute la terre.
\VS{8}Il s'est souvenu pour toujours de son alliance, de la parole qu'il a ordonnée pour mille générations,
\VS{9}du traité qu'il a fait avec Abraham et du serment qu'il a fait à Isaac\FTNT{Ge. 17:2 ; Ge. 22:16 ; Ge. 26:3 ; Ge. 28:13 ; Ge. 33:11 ; Lu. 1:73.}.
\VS{10}Il l'a érigé pour être une ordonnance à Jacob, et à Israël pour être une alliance éternelle,
\VS{11}en disant : Je te donnerai le pays de Canaan, comme héritage qui vous est échu\FTNT{Ge. 13:15 ; Ge. 15:18.}.
\VS{12}Ils étaient alors un petit nombre de gens, très peu nombreux, et étrangers dans le pays.
\VS{13}Car ils allaient de nation en nation, et d'un royaume vers un autre peuple.
\VS{14}Il ne permit à personne de les opprimer, il châtia des rois à cause d'eux\FTNT{Ge. 35:5.},
\VS{15}disant : Ne touchez point à mes oints et ne faites point de mal à mes prophètes\FTNT{1 Ch. 16:22.} !
\VS{16}Il appela aussi la famine sur la terre, rompit le bâton du pain\FTNT{Lé. 26:26 ; Es. 3:1 ; Ez. 4:16.}.
\VS{17}Il envoya un homme devant eux ; Joseph fut vendu pour esclave\FTNT{Ge. 37:28-36.}.
\VS{18}On serra ses pieds dans des ceps, sa personne fut mise aux fers.
\VS{19}Jusqu'au temps où arriva ce qu'il avait annoncé, et où la parole de Yahweh l'éprouva.
\VS{20}Le roi le relâcha et le laissa aller ; le dominateur des peuples le délivra.
\VS{21}Il l'établit pour maître sur sa maison, et pour gouverneur sur tout son domaine\FTNT{Ge. 41:40.} ;
\VS{22}pour soumettre les princes à ses désirs, et pour instruire ses anciens.
\VS{23}Puis Israël entra en Egypte, et Jacob séjourna dans le pays de Cham\FTNT{Ge. 46:6 ; Ps. 78:51.}.
\VS{24}Yahweh rendit son peuple très fécond et le rendit plus puissant que ceux qui l'opprimaient.
\VS{25}Il changea leur cœur, au point qu'ils haïrent son peuple jusqu'à conspirer contre ses serviteurs\FTNT{Ex. 1:7-12.}.
\VS{26}Il envoya Moïse son serviteur, et Aaron, qu'il avait élu\FTNT{Ex. 4:14.}.
\VS{27}Ils accomplirent au milieu d'eux des prodiges et des miracles qu'ils avaient eu la charge de faire dans le pays de Cham.
\VS{28}Il envoya les ténèbres et fit venir l'obscurité ; et ils ne furent point rebelles à sa parole.
\VS{29}Il changea leurs eaux en sang et fit mourir leurs poissons.
\VS{30}Leur terre produisit en abondance des grenouilles jusque dans les chambres de leurs rois.
\VS{31}Il dit, et des mouches vinrent, des poux sur tout leur pays.
\VS{32}Il leur donna pour pluie de la grêle, et un feu flamboyant sur la terre.
\VS{33}Il frappa leurs vignes et leurs figuiers, et il brisa les arbres du pays.
\VS{34}Il ordonna et les sauterelles vinrent, des jeunes sauterelles sans nombre
\VS{35}qui dévorèrent toute l'herbe du pays, et qui dévorèrent le fruit de leur terroir.
\VS{36}Il frappa tous les premiers-nés du pays, les prémices de toute leur vigueur\FTNT{Lire Ex. 7 à 12.}.
\VS{37}Puis il les fit sortir avec de l'or et de l'argent, et nul ne chancela parmi ses tribus.
\VS{38}Les Egyptiens se réjouirent à leur départ, car la peur qu'ils avaient d'eux les avait saisis.
\VS{39}Il étendit la nuée pour couverture, et le feu pour éclairer la nuit.
\VS{40}Le peuple demanda et il fit venir des cailles ; et il les rassasia du pain des cieux\FTNT{Ex. 16:12-13.}.
\VS{41}Il ouvrit le rocher et les eaux en coulèrent ; elles se répandirent comme un fleuve dans les lieux arides\FTNT{Ex. 17:6.}.
\VS{42}Car il se souvint de sa parole sainte qu'il avait donnée à Abraham son serviteur\FTNT{Ge. 15:13-16.}.
\VS{43}Il fit sortir son peuple dans l'allégresse, ses élus au milieu des cris retentissants\FTNT{Ex. 15:1.}.
\VS{44}Il leur donna les terres des nations et ils possédèrent le fruit du travail des peuples,
\VS{45}afin qu'ils gardent ses statuts et qu'ils observent ses lois. Louez Yahweh !
\Chap{106}
\TextTitle{L'infidélité d'Israël}
\VerseOne{}Louez Yahweh ! Célébrez Yahweh car il est bon, car sa bonté demeure à toujours !
\VS{2}Qui pourrait réciter les exploits de Yahweh ? Qui pourrait faire retentir toute sa louange ?
\VS{3}Heureux ceux qui observent la justice, qui font en tout temps ce qui est juste !
\VS{4}Yahweh, souviens-toi de moi selon la bienveillance que tu portes à ton peuple, aie soin de moi selon ta délivrance !
\VS{5}Afin que je voie le bien de tes élus, que je me réjouisse dans la joie de ta nation, que je me glorifie avec ton héritage.
\VS{6}Nous avons péché avec nos pères, nous avons agi dans l'iniquité, nous avons fait le mal\FTNT{Da. 9:16 ; Esd. 9:7 ; Né. 1:6.}.
\VS{7}Nos pères n'ont point été attentifs à tes merveilles en Egypte ; ils ne se sont point souvenus de la multitude de tes faveurs ; mais ils furent rebelles près de la mer, vers la Mer Rouge\FTNT{Ex. 14:11.}.
\VS{8}Toutefois, il les délivra pour l'amour de son Nom, afin de faire connaître sa puissance.
\VS{9}Il menaça la Mer Rouge et elle se sécha ; et il les conduisit à travers les profondeurs de la mer comme un désert ;
\VS{10}il les délivra de la main de ceux qui les haïssaient et les racheta de la main de l'ennemi.
\VS{11}Les eaux couvrirent leurs oppresseurs, il n'en resta pas un seul\FTNT{Ex. 14:27.}.
\VS{12}Alors ils crurent à ses paroles et ils chantèrent sa louange.
\VS{13}Mais ils oublièrent vite ses œuvres et ne s'attendirent point à son conseil.
\VS{14}Ils furent épris de convoitise au désert et ils tentèrent Dieu dans le désert.
\VS{15}Alors il leur donna ce qu'ils avaient demandé, toutefois il leur envoya le dépérissement dans leur corps.
\VS{16}Ils jalousèrent dans le camp Moïse et Aaron, le saint de Yahweh.
\VS{17}La terre s'ouvrit et engloutit Dathan ; et recouvrit de terre Abiram\FTNT{No. 16.}.
\VS{18}Le feu s'alluma au milieu de leur assemblée, la flamme brûla les méchants.
\VS{19}Ils firent un veau en Horeb, et se prosternèrent devant une image de métal fondu\FTNT{Ex. 32.}.
\VS{20}Ils changèrent leur gloire contre la figure d'un bœuf qui mange l'herbe.
\VS{21}Ils oublièrent Dieu, leur libérateur, qui avait fait de grandes choses en Egypte,
\VS{22}des choses merveilleuses dans le pays de Cham, et des prodiges sur la Mer Rouge.
\VS{23}C'est pourquoi il dit qu'il les détruirait ; mais Moïse, son élu, se tint à la brèche devant lui pour détourner sa fureur, afin qu'il ne les détruisît point\FTNT{Ex. 32:11.}.
\VS{24}Ils méprisèrent le pays désirable et ne crurent point à sa parole.
\VS{25}Ils murmurèrent dans leurs tentes et n'obéirent point à la voix de Yahweh.
\VS{26}C'est pourquoi il leur jura la main levée de les faire tomber dans le désert,
\VS{27}d'accabler leur postérité parmi les nations, et de les disperser au milieu des pays\FTNT{No. 14:22.}.
\VS{28}Ils se joignirent aux adorateurs de Baal-Peor et mangèrent des victimes sacrifiées aux morts.
\VS{29}Ils irritèrent Dieu par leurs actions, au point qu'une plaie fit une brèche parmi eux.
\VS{30}Mais Phinées se présenta et fit justice ; et la plaie fut arrêtée.
\VS{31}Cela lui fut imputé à justice de génération en génération, pour toujours\FTNT{No. 25:3-8.}.
\VS{32}Ils excitèrent aussi sa colère près des eaux de Meriba, et Moïse fut puni à cause d'eux.
\VS{33}Car ils aigrirent son esprit et il parla avec légèreté de ses lèvres\FTNT{No. 20:12.}.
\VS{34}Ils ne détruisirent point les peuples que Yahweh leur avait dit de détruire,
\VS{35}mais ils se mêlèrent parmi ces nations et apprirent leurs manières de faire.
\VS{36}Ils servirent leurs faux dieux qui furent un piège pour eux.
\VS{37}Car ils sacrifièrent leurs fils et leurs filles aux démons\FTNT{Lé. 18:21 ; De. 12:31 ; 2 R. 16:3 ; Ez. 20:26.}.
\VS{38}Ils répandirent le sang innocent, le sang de leurs fils et de leurs filles, ils sacrifièrent aux faux dieux de Canaan ; et le pays fut souillé de sang\FTNT{No. 35:33.}.
\VS{39}Ils se souillèrent par leurs œuvres et se prostituèrent par leurs actions.
\VS{40}C'est pourquoi la colère de Yahweh s'embrasa contre son peuple et il eut en abomination son héritage.
\VS{41}Il les livra entre les mains des nations, et ceux qui les haïssaient dominèrent sur eux.
\VS{42}Leurs ennemis les opprimèrent et ils furent humiliés sous leur main.
\VS{43}Il les délivra souvent, mais ils se montrèrent rebelles dans leurs desseins et furent humiliés par leur iniquité.
\VS{44}Toutefois, il vit leur détresse lorsqu'il entendit leurs supplications.
\VS{45}Il se souvint en leur faveur de son alliance et se repentit selon la grandeur de ses compassions.
\VS{46}Il fit que ceux qui les avaient emmenés captifs eurent pitié d'eux.
\VS{47}Yahweh notre Dieu, délivre-nous et rassemble-nous du milieu des nations ! Afin que nous célébrions ton saint Nom et que nous mettions notre gloire à te louer !
\VS{48}Béni soit Yahweh, le Dieu d'Israël, d'éternité en éternité ! Et que tout le peuple dise amen ! Louez Yahweh.
\Chap{107}
\TextTitle{La grâce de Yahweh pour ses rachetés}
\VerseOne{}Célébrez Yahweh car il est bon, parce que sa bonté demeure à toujours.
\VS{2}Qu'ainsi disent les rachetés de Yahweh, ceux qu'il a rachetés de la main de l'oppresseur,
\VS{3}et qu'il a rassemblés de tous les pays, de l'orient et de l'occident, du nord et de la mer.
\VS{4}Ils erraient dans le désert, ils marchaient dans la solitude, sans trouver une ville où ils puissent habiter.
\VS{5}Ils étaient affamés et assoiffés, leur âme était languissante.
\VS{6}Alors ils crièrent vers Yahweh dans leur détresse et il les délivra de leurs angoisses ;
\VS{7}il les conduisit sur le droit chemin pour aller dans une ville habitée.
\VS{8}Qu'ils célèbrent Yahweh pour sa bonté et ses merveilles envers les fils des hommes !
\VS{9}Parce qu'il a désaltéré l'âme altérée et rassasié de ses biens l'âme affamée\FTNT{Ps. 146:7 ; Lu. 1:53.}.
\VS{10}Ceux qui avaient pour demeure les ténèbres et l'ombre de la mort, vivaient captifs dans l'affliction et dans les chaînes,
\VS{11}parce qu'ils furent rebelles aux paroles de Dieu, et parce qu'ils avaient rejeté le conseil du Très-Haut\FTNT{De. 31:20 ; La. 3:42.}.
\VS{12}Il humilia leur cœur par la souffrance, ils furent abattus ; et personne ne les secourut.
\VS{13}Alors ils crièrent vers Yahweh dans leur détresse, et il les délivra de leurs angoisses.
\VS{14}Il les fit sortir hors des ténèbres et de l'ombre de la mort ; et il rompit leurs liens\FTNT{Ps. 68:19 ; Ep. 4:8 ; Col. 1:12-13.}.
\VS{15}Qu'ils célèbrent Yahweh pour sa bonté et ses merveilles envers les fils des hommes !
\VS{16}Parce qu'il a brisé les portes d'airain et cassé les barreaux de fer.
\VS{17}Les insensés sont affligés à cause de leurs transgressions et à cause de leurs iniquités.
\VS{18}Leur âme avait en horreur toute nourriture, et ils touchaient aux portes de la mort.
\VS{19}Alors ils crièrent vers Yahweh dans leur détresse, et il les délivra de leurs angoisses\FTNT{Ps. 50:15 ; Os. 5:15.}.
\VS{20}Il envoya sa parole et les guérit ; et il les délivra de leurs tombeaux.
\VS{21}Qu'ils célèbrent Yahweh pour sa bonté et ses merveilles envers les fils des hommes !
\VS{22}Qu'ils offrent des sacrifices de remerciements, et qu'ils racontent ses œuvres avec des cris de joie.
\VS{23}Ceux qui descendaient sur la mer dans des navires, faisant commerce sur les grandes eaux,
\VS{24}ceux-là virent les œuvres de Yahweh et ses merveilles dans les lieux profonds,
\VS{25}car il dit, et il fit paraître la tempête qui souleva les vagues de la mer.
\VS{26}Ils montaient vers les cieux, ils descendaient dans l'abîme ; leur âme se fondait d'angoisse.
\VS{27}Saisis de vertiges, ils chancelaient comme un homme ivre ; et toute leur sagesse était anéantie\FTNT{Es. 51:17-21 ; Jé. 13:13.}.
\VS{28}Alors ils crièrent vers Yahweh dans leur détresse, et il les tira hors de leurs angoisses.
\VS{29}Il arrêta la tempête, la changeant en calme, et les ondes se turent.
\VS{30}Puis ils se réjouirent de ce qu'elles s'étaient apaisées, et il les conduisit au port qu'ils désiraient.
\VS{31}Qu'ils célèbrent Yahweh pour sa bonté et ses merveilles envers les fils des hommes !
\VS{32}Et qu'ils l'exaltent dans l'assemblée du peuple et le louent dans l'assemblée des anciens.
\VS{33}Il réduit les fleuves en désert, et les sources d'eaux en sécheresse ;
\VS{34}la terre fertile en terre salée, à cause de la méchanceté de ses habitants\FTNT{Jé. 12:4 ; Jé. 17:6.}.
\VS{35}Il transforme le désert en étangs d'eaux, et la terre sèche en des sources d'eaux\FTNT{Es. 41:18.} ;
\VS{36}il y établit ceux qui sont affamés, ils bâtissent des villes pour l'habiter.
\VS{37}Ils ensemencent des champs et plantent des vignes qui rendent du fruit tous les ans.
\VS{38}Il les bénit et ils se multiplient extrêmement ; et il ne laisse point diminuer leur bétail.
\VS{39}Puis ils sont amoindris et humiliés par l'oppression, le malheur et la souffrance.
\VS{40}Il répand le mépris sur les princes et les fait errer dans des lieux déserts sans chemin.
\VS{41}Mais il relève le pauvre et le délivre de la misère, il établit les familles comme des troupeaux\FTNT{1 S. 2:8 ; Ps. 113:7.}.
\VS{42}Les hommes droits le voient et se réjouissent, mais toute iniquité a la bouche fermée.
\VS{43}Quiconque est sage prendra garde à ces choses, afin qu'on considère les bontés de Yahweh.
\Chap{108}
\TextTitle{Yahweh, le secours}
\VerseOne{}Cantique. Psaume de David. Mon cœur est affermi, ô Dieu ! Je chante et je joue de mes instruments, c'est ma gloire !
\VS{2}Réveillez-vous, mon luth et ma harpe ! Je me réveillerai à l'aube du jour.
\VS{3}Yahweh, je te célébrerai parmi les peuples et je te chanterai parmi les nations.
\VS{4}Car ta bonté est grande par-dessus les cieux, et ta vérité atteint jusqu'aux nues.
\VS{5}Ô Dieu ! Elève-toi sur les cieux, et que ta gloire soit sur toute la terre !
\VS{6}Afin que ceux que tu aimes soient délivrés ; sauve-moi par ta droite et exauce-moi !
\VS{7}Dieu a dit dans sa sainteté : Je me réjouirai, je partagerai Sichem et mesurerai la vallée de Succoth.
\VS{8}Galaad sera à moi, Manassé sera à moi, et Ephraïm sera le sommet de ma forteresse, Juda mon législateur.
\VS{9}Moab sera le bassin où je me laverai, je jetterai mon soulier sur Edom, je triompherai des Philistins. 
\VS{10}Qui me conduira dans la ville forte ? Qui me conduira jusqu'en Edom ?
\VS{11}N'est-ce pas toi, ô Dieu, qui nous avais rejetés, et qui ne sortais plus, ô Dieu, avec nos armées ?
\VS{12}Donne–nous du secours pour sortir de la détresse ! Car la délivrance qu'on attend de l'homme est vaine.
\VS{13}Avec Dieu, nous ferons des exploits ; il foulera nos ennemis\FTNT{Ps. 60:5-14.}.
\Chap{109}
\TextTitle{La méchanceté de l'homme}
\VerseOne{}Psaume de David ; Donné au chef des chantres\FTNT{Les Psaumes d'imprécations (Ps. 35 ; 52 ; 55 ; 58, 59 ; 79 ; 109 ; 137) sont des demandes faites à Dieu pour qu'il punisse les méchants. Le Seigneur Jésus-Christ nous demande aujourd'hui de bénir nos ennemis (Lu. 6:27-37).}. Dieu de ma louange, ne te tais point !
\VS{2}Car la bouche du méchant et la bouche remplie de fraude se sont ouvertes contre moi ; ils parlent contre moi avec une langue mensongère !
\VS{3}Ils m'entourent de paroles pleines de haine et ils me font la guerre sans cause !
\VS{4}Tandis que je les aime, ils sont mes ennemis ; mais moi, je n'ai fait que prier en leur faveur !
\VS{5}Ils me rendent le mal pour le bien, et la haine pour l'amour que je leur porte.
\VS{6}Etablis le méchant sur lui, et que Satan se tienne à sa droite !
\VS{7}Quand il sera jugé, fais qu'il soit déclaré méchant, et que sa prière soit regardée comme un crime !
\VS{8}Que sa vie soit courte\FTNT{Il est question ici du suicide de Judas (Mt. 27:3-5).} et qu'un autre prenne sa charge\FTNT{Ce passage parle de Judas (Ac. 1:20).} !
\VS{9}Que ses enfants soient orphelins et sa femme veuve !
\VS{10}Que ses enfants soient entièrement vagabonds, et qu'ils mendient et quêtent en sortant de leurs maisons détruites\FTNT{Job. 20:10.} !
\VS{11}Que le créancier usant d'exaction attrape tout ce qui est à lui et que les étrangers butinent tout son travail !
\VS{12}Qu'il n'y ait personne qui étende sa compassion sur lui, et qu'il n'y ait personne qui ait pitié de ses orphelins !
\VS{13}Que sa postérité soit exposée à être retranchée ; que leur nom soit effacé dans la génération qui le suivra !
\VS{14}Que l'iniquité de ses pères revienne en mémoire à Yahweh, et que le péché de sa mère ne soit point effacé !
\VS{15}Qu'ils soient continuellement devant Yahweh, et qu'il retranche leur mémoire de la terre\FTNT{Ps. 34:17.},
\VS{16}parce qu'il ne s'est point souvenu d'user de miséricorde, mais il a persécuté l'homme affligé et misérable, dont le cœur est brisé, et cela pour le faire mourir !
\VS{17}Puisqu'il aime la malédiction, que la malédiction tombe sur lui ! Puisqu'il ne prend pas plaisir à la bénédiction, que la bénédiction aussi s'éloigne de lui !
\VS{18}Et qu'il soit revêtu de la malédiction comme de sa robe ; qu'elle entre dans son corps comme de l'eau, et dans ses os comme de l'huile !
\VS{19}Qu'elle lui soit comme un vêtement dont il se couvre, et comme une ceinture dont il se ceigne continuellement !
\VS{20}Telle soit, de la part de Yahweh, la récompense de mes adversaires, et de ceux qui parlent mal de moi !
\VS{21}Mais toi, Yahweh, Seigneur, agis avec moi pour l'amour de ton Nom ! Et parce que ta miséricorde est grande, délivre-moi !
\VS{22}Car je suis affligé et misérable, et mon cœur est blessé au-dedans de moi.
\VS{23}Je m'en vais comme l'ombre quand elle décline, et je suis chassé comme une sauterelle.
\VS{24}Mes genoux sont affaiblis par le jeûne, et mon corps est épuisé de maigreur au lieu d'être gras.
\VS{25}Je suis pour eux un objet d'opprobre ; quand ils me voient, ils secouent la tête.
\VS{26}Yahweh, mon Dieu ! Aide-moi, délivre-moi selon ta miséricorde.
\VS{27}Afin qu'on sache que c'est ta main, que c'est toi, ô Yahweh, qui l'as fait.
\VS{28}Ils maudiront, mais tu béniras ; ils s'élèveront, mais ils seront confus ; et ton serviteur se réjouira.
\VS{29}Que mes adversaires soient revêtus de confusion et couverts de leur honte comme d'un manteau.
\VS{30}Je célébrerai hautement de ma bouche Yahweh, et je le louerai au milieu de plusieurs nations.
\VS{31}De ce qu'il se tient à la droite du misérable pour le délivrer de ceux qui condamnent son âme.
\Chap{110}
\TextTitle{Yahweh, le Roi et le Sacrificateur}
\VerseOne{}Psaume de David. Yahweh a dit à mon Seigneur : Assieds-toi à ma droite, jusqu'à ce que je fasse de tes ennemis le marchepied de tes pieds\FTNT{Ce psaume affirme la divinité de Jésus-Christ (Mt. 22:41-46 ; Mc. 12:35-37 ; Lu. 20:41-44 ; Ac. 2:34-35 ; Hé. 1:13 ; Hé. 10:12-13).}.
\VS{2}Yahweh étendra de Sion le sceptre de ta puissance, en disant : Domine au milieu de tes ennemis\FTNT{Es. 2:2-3 ; Da. 7:14.} !
\VS{3}Ton peuple est plein d'ardeur quand tu rassembles ton armée ; avec des ornements sacrés, du sein de l'aurore, ta jeunesse vient à toi comme une rosée.
\VS{4}Yahweh l'a juré, et il ne s'en repentira point que tu es sacrificateur éternellement, à la manière de Melchisédek\FTNT{Hé. 5:6 ; Hé. 6:20 ; Hé. 7:17 ; Ge. 14:18.}.
\VS{5}Le Seigneur est à ta droite, il brisera les rois au jour de sa colère.
\VS{6}Il exercera le jugement sur les nations, il remplira tout de cadavres ; il brisera le chef qui domine sur un grand pays\FTNT{Ap. 14 ; Ap. 16.}.
\VS{7}Il boit au torrent pendant la marche : C'est pourquoi il lève haut la tête.
\Chap{111}
\TextTitle{Les oeuvres magnifiques de Dieu}
\VerseOne{}Louez Yahweh. [Aleph.] Je célébrerai Yahweh de tout mon cœur, [Beth.] dans la compagnie des hommes droits et dans l'assemblée.
\VS{2}[Guimel.] Les œuvres de Yahweh sont grandes, [Daleth.] elles sont recherchées par tous ceux qui y prennent plaisir.
\VS{3}[He.] Son œuvre n'est que majesté et magnificence, [Vav.] et sa justice demeure à perpétuité.
\VS{4}[Zayin.] Il a rendu ses merveilles mémorables. [Heth.] Yahweh est miséricordieux et compatissant.
\VS{5}[Teth.] Il a donné de la nourriture à ceux qui le craignent ; [Yod.] il s'est souvenu pour toujours de son alliance.
\VS{6}[Kaf.] Il a manifesté à son peuple la puissance de ses œuvres, [Lamed.] en leur donnant l'héritage des nations.
\VS{7}[Mem.] Les œuvres de ses mains ne sont que vérité et équité. [Nun.] Tous ses commandements sont véritables,
\VS{8}[Samech.] appuyés à perpétuité, éternellement, [Ayin.] faits avec fidélité et droiture.
\VS{9}[Pe.] Il a envoyé la rédemption à son peuple\FTNT{Ex. 6:6 ; Jn. 3:16.} ; [Tsade.] il lui a donné une alliance éternelle ; [Qof.] son nom est saint et redoutable.
\VS{10}[Resh.] Le commencement de la sagesse c'est la crainte de Yahweh : [Shin.] Tous ceux qui s'adonnent à faire ce qu'elle prescrit sont sages\FTNT{Pr. 1:7 ; Pr. 9:10 ; Pr. 8:13 ; De. 4:6.}. [Tav.] Sa louange demeure à perpétuité.
\Chap{112}
\TextTitle{La crainte de Yahweh enrichit et donne de l'assurance}
\VerseOne{}Louez Yahweh. [Aleph.] Heureux l'homme qui craint Yahweh [Beth.] et qui prend un grand plaisir à ses commandements !
\VS{2}[Guimel.] Sa postérité sera puissante sur la terre, [Daleth.] la génération des hommes droits sera bénie\FTNT{Pr. 20:7.}.
\VS{3}[He.] Il y aura des biens et des richesses dans sa maison ; [Vav.] et sa justice demeure à perpétuité.
\VS{4}[Zayin.] La lumière s'est levée dans les ténèbres sur ceux qui sont justes\FTNT{Pr. 4:18 ; Ps. 37:6.} ; [Heth.] il est compatissant, miséricordieux et juste.
\VS{5}[Teth.] Heureux l'homme de bien qui exerce la miséricorde et prête, [Yod.] qui règle ses actions avec justice.
\VS{6}[Kaf.] Il ne chancelle jamais. [Lamed.] La mémoire du juste dure toujours\FTNT{Pr. 10:7.}.
\VS{7}[Mem.] Il ne craint pas les mauvaises nouvelles ; [Nun.] son cœur est ferme, confiant en Yahweh.
\VS{8}[Samech.] Son cœur est bien affermi, il ne craint pas, [Ayin.] jusqu'à ce qu'il mette son plaisir à regarder ses adversaires.
\VS{9}[Pe.] Il fait des largesses, il donne aux pauvres ; [Tsade.] sa justice demeure à perpétuité ; [Qof.] sa corne s'élève en gloire.
\VS{10}[Resh.] Le méchant le voit et s'irrite. [Shin.] Il grince des dents et se consume ; [Tav.] les désirs des méchants périssent.
\Chap{113}
\TextTitle{Yahweh, le Dieu élevé au-dessus de tout}
\VerseOne{}Louez Yahweh\FTNT{Les psaumes disant « Alléluia » sont les Ps. 104 à 106, 111 à 113, 115 à 117, 135 à 136, 146 à 150. Parmi eux, les psaumes 135 et 146 à 150, étaient chantés durant le service quotidien d'adoration dans la synagogue. Les psaumes 115 à 118, appelés « le grand Hallel », étaient chantés lors des fêtes de Pâque. Alléluia veut dire « Louez Yahweh » (Ap. 19:1).} ! Louez, vous serviteurs de Yahweh, louez le Nom de Yahweh !
\VS{2}Que le Nom de Yahweh soit béni dès maintenant et à toujours !
\VS{3}Le Nom de Yahweh est digne de louanges depuis le soleil levant jusqu'au soleil couchant.
\VS{4}Yahweh est élevé par-dessus toutes les nations, sa gloire est au-dessus des cieux.
\VS{5}Qui est semblable à Yahweh notre Dieu, qui habite dans les lieux très hauts ?
\VS{6}Il s'abaisse pour regarder sur le ciel et sur la terre,
\VS{7}Il relève l'affligé de la poussière, et retire le pauvre\FTNT{1 S. 2:8 ; Ps. 107:41.} de dessus le fumier,
\VS{8}pour le faire asseoir avec les nobles, avec les nobles de son peuple\FTNT{Job. 36:7.}.
\VS{9}Il donne une maison à la femme stérile, il en fait une mère joyeuse au milieu de ses fils\FTNT{Ge. 17:17-21 ; 1 S. 2:5 ; Ps. 68:6.}. Louez Yahweh !
\Chap{114}
\TextTitle{La création tremble devant le Tout-Puissant}
\VerseOne{}Quand Israël sortit d'Egypte, quand la maison de Jacob s'éloigna d'un peuple barbare,
\VS{2}Juda devint son lieu saint, Israël son domaine\FTNT{Jé. 2:2-3.}.
\VS{3}La mer le vit et s'enfuit, le Jourdain retourna en arrière\FTNT{Jos. 3:13-16 ; Ps. 77:17.}.
\VS{4}Les montagnes sautèrent comme des béliers, les collines comme des agneaux\FTNT{Jg. 5:5 ; Ha. 3:10 ; Ps. 68:9.}.
\VS{5}Ô Mer ! Qu'avais-tu pour t'enfuir ? Jourdain, pour retourner en arrière ?
\VS{6}Et vous montagnes, pour sauter comme des béliers ? Et vous collines, comme des agneaux ?
\VS{7}Ô Terre ! Tremble devant la présence du Seigneur, devant la présence du Dieu de Jacob,
\VS{8}qui a changé le rocher en un étang d'eaux, la pierre très dure en une source d'eaux.
\Chap{115}
\TextTitle{Louange au Dieu de gloire}
\VerseOne{}Non point à nous, ô Yahweh ! Non point à nous, mais à ton Nom donne gloire, à cause de ta bonté, à cause de ta fidélité !
\VS{2}Pourquoi les nations diraient-elles : Où est maintenant leur Dieu ?
\VS{3}Certes notre Dieu est au ciel, il fait tout ce qu'il veut\FTNT{Ps. 135:6 ; Job. 23:13.}.
\VS{4}Leurs idoles sont des dieux d'or et d'argent, elles sont l'ouvrage de mains d'homme.
\VS{5}Elles ont une bouche, et ne parlent point ; elles ont des yeux, et ne voient point ;
\VS{6}elles ont des oreilles, et n'entendent point ; elles ont un nez, et ne sentent point ;
\VS{7}elles ont des mains, et elles ne touchent point ; elles ont des pieds, et elles ne marchent point ; et elles ne rendent aucun son de leur gosier\FTNT{Ex. 32:2-8 ; 1 R. 18:25-26 ; Es. 44:9 ; Ez. 8:8-12.}.
\VS{8}Ils leur ressemblent, ceux qui les fabriquent, tous ceux qui se confient en eux.
\VS{9}Israël confie-toi en Yahweh ; il est leur secours et leur bouclier de ceux qui se confient en lui.
\VS{10}Maison d'Aaron, confie-toi en Yahweh ; il est leur secours et leur bouclier.
\VS{11}Vous qui craignez Yahweh, confiez-vous en Yahweh ; il est leur secours et leur bouclier.
\VS{12}Yahweh s'est souvenu de nous, il bénira, il bénira la maison d'Israël, il bénira la maison d'Aaron.
\VS{13}Il bénira ceux qui craignent Yahweh, tant les petits que les grands.
\VS{14}Yahweh vous multipliera ses bénédictions, à vous et à vos fils.
\VS{15}Vous êtes bénis de Yahweh, qui a fait les cieux et la terre.
\VS{16}Les cieux, sont les cieux de Yahweh, mais il a donné la terre aux fils des hommes.
\VS{17}Ce ne sont pas les morts qui célèbrent Yahweh, ce n'est aucun de ceux qui descendent dans le lieu du silence\FTNT{Ps. 88:11 ; Es. 38:18-19 ; Ps. 6:6.}.
\VS{18}Mais nous, nous bénirons Yahweh dès maintenant et pour toujours. Louez Yahweh !
\Chap{116}
\TextTitle{Psaume des rachetés}
\VerseOne{}J'aime Yahweh, car il a entendu ma voix et mes supplications.
\VS{2}Car il a incliné son oreille vers moi, c'est pourquoi je l'invoquerai durant mes jours.
\VS{3}Les liens de la mort m'avaient environné, et les angoisses du scheol m'avaient trouvé\FTNT{2 S. 22:5 ; Ps. 18:5.} ; j'avais trouvé la détresse et la douleur.
\VS{4}Mais j'invoquai le Nom de Yahweh en disant : Je te prie, délivre mon âme, ô Yahweh !
\VS{5}Yahweh est compatissant et juste, et notre Dieu fait miséricorde.
\VS{6}Yahweh garde les simples ; j'étais devenu misérable et il m'a sauvé.
\VS{7}Mon âme, retourne dans ton repos, car Yahweh t'a fait du bien.
\VS{8}Parce que tu as retiré mon âme de la mort, mes yeux des larmes et mes pieds de la chute,
\VS{9}je marcherai dans la présence de Yahweh, sur la terre des vivants.
\VS{10}J'ai cru, c'est pourquoi j'ai parlé\FTNT{2 Co. 4:13.} ; j'ai été fort affligé.
\VS{11}Je disais dans ma précipitation : Tout homme est menteur\FTNT{Ro. 3:4.}.
\VS{12}Que rendrai-je à Yahweh ? Tous ses bienfaits envers moi ?
\VS{13}J'élèverai la coupe des délivrances et j'invoquerai le Nom de Yahweh.
\VS{14}J'accomplirai maintenant mes vœux à Yahweh, devant tout son peuple.
\VS{15}Elle a du prix aux yeux de Yahweh, la mort de ceux qu'il aime.
\VS{16}Ecoute-moi, ô Yahweh ! Car je suis ton serviteur, je suis ton serviteur, fils de ta servante. Tu as délié mes liens.
\VS{17}Je t'offrirai le sacrifice de remerciement et j'invoquerai le Nom de Yahweh.
\VS{18}J'accomplirai maintenant mes vœux à Yahweh, devant tout son peuple,
\VS{19}dans les parvis de la maison de Yahweh, au milieu de toi, Jérusalem ! Louez Yahweh !
\Chap{117}
\TextTitle{Toutes les nations louent Yahweh}
\VerseOne{}Toutes les nations, louez Yahweh ! Tous les peuples, célébrez-le !
\VS{2}Car sa miséricorde est grande envers nous, et sa fidélité dure à toujours. Louez Yahweh !
\Chap{118}
\TextTitle{Yahweh, le Dieu de mon secours}
\VerseOne{}Célébrez Yahweh, car il est bon, parce que sa bonté dure à toujours !
\VS{2}Qu'Israël dise maintenant : Car sa bonté dure à toujours !
\VS{3}Que la maison d'Aaron dise maintenant : Car sa bonté dure à toujours !
\VS{4}Que ceux qui craignent Yahweh disent maintenant : Car sa bonté dure à toujours !
\VS{5}Me trouvant dans la détresse, j'ai invoqué Yahweh\FTNT{Ps. 120:1.} ; et Yahweh m'a répondu et m'a mis au large.
\VS{6}Yahweh est pour moi, je ne craindrai point. Que me ferait l'homme ?
\VS{7}Yahweh est pour moi parmi ceux qui me secourent, c'est pourquoi je verrai en ceux qui me haïssent ce que je désire.
\VS{8}Mieux vaut se confier en Yahweh que se confier en l'homme\FTNT{Es. 2:22 ; Jé. 17:5 ; Ps. 62:9.}.
\VS{9}Mieux vaut se confier en Yahweh que se reposer sur les grands d'entre les peuples.
\VS{10}Toutes les nations m'avaient environné, mais au Nom de Yahweh je les taille en pièces.
\VS{11}Elles m'avaient environné, elles m'avaient, dis-je, environné ; mais au Nom de Yahweh je les taille en pièces.
\VS{12}Elles m'avaient environné comme des abeilles, elles s'éteignent comme un feu d'épines\FTNT{De. 1:44.}, car au Nom de Yahweh je les taille en pièces.
\VS{13}Tu me poussais violemment pour me faire tomber, mais Yahweh m'a secouru.
\VS{14}Yahweh est ma force et le sujet de mes louanges, et il a été ma délivrance\FTNT{Ex. 15:2 ; Es. 12:2.}.
\VS{15}Une voix de chant de triomphe et de délivrance retentit dans les tentes des justes : La droite de Yahweh exerce la puissance !
\VS{16}La droite de Yahweh est élevée, la droite de Yahweh exerce sa puissance.
\VS{17}Je ne mourrai pas, je vivrai et je raconterai les œuvres de Yahweh.
\VS{18}Yahweh m'a châtié sévèrement, mais il ne m'a point livré à la mort.
\VS{19}Ouvrez-moi les portes de la justice ; j'y entrerai et je célébrerai Yahweh.
\VS{20}C'est ici la porte de Yahweh, les justes y entreront.
\VS{21}Je te célébrerai parce que tu m'as exaucé et tu as été mon libérateur.
\VS{22}La Pierre que les architectes avaient rejetée, est devenue la principale de l'angle\FTNT{Le Messie est présenté comme la pierre ou le rocher. (cp. Es. 8:13-17 ; 1 Pi. 2:7).}.
\VS{23}Ceci a été fait par Yahweh, c'est un prodige à nos yeux.
\VS{24}C'est ici la journée que Yahweh a faite, qu'elle soit pour nous un sujet d'allégresse et de joie.
\VS{25}Yahweh, je te prie, délivre maintenant. Yahweh, je te prie, donne maintenant la prospérité !
\VS{26}Béni soit celui qui vient au Nom de Yahweh ! Nous vous bénissons de la maison de Yahweh.
\VS{27}Yahweh est Dieu, et il nous a éclairés. Liez avec des cordes la bête du sacrifice, et amenez-la jusqu'aux cornes de l'autel.
\VS{28}Tu es mon Dieu, c'est pourquoi je te célébrerai. Tu es mon Dieu, je t'exalterai.
\VS{29}Célébrez Yahweh car il est bon, parce que sa miséricorde demeure à toujours !
\Chap{119}
\TextTitle{La Parole de Yahweh éclaire}
\VerseOne{}[Aleph.] Heureux ceux qui sont intègres dans leur voie, qui marchent selon la loi de Yahweh.
\VS{2}Heureux sont ceux qui gardent ses préceptes et qui le cherchent de tout leur cœur\FTNT{Jos. 1:8.} ;
\VS{3}qui ne font point d'iniquité, qui marchent dans ses voies\FTNT{1 Jn. 3:9 ; 1 Jn. 5:18.}.
\VS{4}Tu as donné tes commandements afin qu'on les garde soigneusement.
\VS{5}Oh ! Que mes voies soient bien établies pour garder tes statuts !
\VS{6}Et je ne rougirai point de honte quand je regarderai à tous tes commandements.
\VS{7}Je te célébrerai avec droiture de cœur quand j'aurai appris les ordonnances de ta justice.
\VS{8}Je veux garder tes statuts, ne me délaisse point entièrement.
\VS{9}[Beth.] Par quel moyen le jeune homme rendra-t-il pure sa voie ? Ce sera en y prenant garde selon ta parole.
\VS{10}Je te recherche de tout mon cœur, ne me laisse pas m'égarer loin de tes commandements.
\VS{11}Je serre ta parole dans mon cœur afin de ne pas pécher contre toi.
\VS{12}Yahweh ! Tu es béni ; enseigne-moi tes statuts.
\VS{13}De mes lèvres je raconte toutes les ordonnances de ta bouche.
\VS{14}Je me réjouis dans le chemin de tes préceptes comme si je possédais toutes les richesses du monde.
\VS{15}Je médite tes commandements et j'observe tes voies.
\VS{16}Je prends plaisir à tes statuts et je n'oublie pas tes paroles.
\VS{17}[Guimel.] Fais du bien à ton serviteur afin que je vive, et je garderai ta parole\FTNT{Ps. 116:7.}.
\VS{18}Ouvre mes yeux afin que je regarde aux merveilles de ta loi\FTNT{Ep. 1:18.} !
\VS{19}Je suis voyageur sur la terre, ne me cache pas tes commandements.
\VS{20}Mon âme est brisée par le désir qui toujours me porte vers tes ordonnances.
\VS{21}Tu réprimandes les orgueilleux, ces maudits, qui se détournent de tes commandements.
\VS{22}Décharge-moi de l'opprobre et du mépris, car j'ai gardé tes préceptes\FTNT{Ps. 3:9.}.
\VS{23}Même les princes s'assoient et parlent contre moi pendant que ton serviteur médite tes statuts.
\VS{24}Tes préceptes font mes délices, ce sont mes conseillers.
\VS{25}[Daleth.] Mon âme est attachée à la poussière, fais-moi revivre selon ta parole\FTNT{Ps. 44:26 ; Ps. 143:11.}.
\VS{26}Je te raconte mes voies et tu me réponds ; enseigne-moi tes statuts.
\VS{27}Fais-moi entendre la voie de tes commandements, et je parlerai de tes merveilles\FTNT{Ps. 145:6.}.
\VS{28}Mon âme pleure de chagrin, relève-moi selon tes paroles.
\VS{29}Eloigne de moi la voie du mensonge et accorde–moi la grâce d'observer ta loi\FTNT{Le mot « loi » vient de l'hebreu « towrah » qui donne « Thora » en français}.
\VS{30}Je choisis la voie de la vérité et je place tes ordonnances sous mes yeux.
\VS{31}Je m'attache à tes préceptes, ô Yahweh ! Ne me fais point rougir de honte.
\VS{32}Je cours dans la voie de tes commandements car tu élargis mon cœur.
\VS{33}[He.] Yahweh, enseigne-moi la voie de tes statuts, et je la garderai jusqu'au bout.
\VS{34}Donne-moi de l'intelligence ; je garderai ta loi et je l'observerai de tout mon cœur\FTNT{Pr. 2:6 ; Ja. 1:5.}.
\VS{35}Fais-moi marcher sur le sentier de tes commandements car j'y prends plaisir.
\VS{36}Incline mon cœur à tes préceptes et non point au profit\FTNT{Ez. 33:31 ; Mc 7:21-22 ; Hé. 13:5.}.
\VS{37}Détourne mes yeux de la vue des choses vaines ; fais-moi vivre dans ta voie.
\VS{38}Accomplis ta parole envers ton serviteur, parole qui est pour ceux qui te craignent.
\VS{39}Eloigne de moi l'opprobre que je redoute, car tes ordonnances sont bonnes.
\VS{40}Voici, je désire pratiquer tes commandements, fais-moi vivre dans ta justice.
\VS{41}[Vav.] Que ta miséricorde vienne sur moi, ô Yahweh ! Et ta délivrance aussi, selon ta promesse !
\VS{42}Et je pourrai répondre à celui qui m'outrage, car je me confie en ta parole.
\VS{43}N'arrache pas de ma bouche la parole de vérité, car j'espère en tes jugements.
\VS{44}Je garderai continuellement ta loi, à toujours et à perpétuité.
\VS{45}Je marcherai au large parce que je recherche tes commandements.
\VS{46}Je parlerai de tes préceptes devant les rois et je ne rougirai pas de honte\FTNT{Ps. 138:1-4 ; Mt. 10:18-19 ; Ac. 26.}.
\VS{47}Je fais mes délices de tes commandements que j'aime ;
\VS{48}j'étends mes mains vers tes commandements que j'aime ; et je médite tes statuts.
\VS{49}[Zayin.] Souviens-toi de la parole donnée à ton serviteur, sur laquelle tu m'as fait espérer.
\VS{50}C'est ici ma consolation dans mon affliction, car ta parole me rend la vie.
\VS{51}Les orgueilleux se sont fort moqués de moi, mais je ne me suis pas détourné de ta loi.
\VS{52}Yahweh, je me souviens de tes jugements anciens et je me suis consolé en eux.
\VS{53}L'horreur me saisit à cause des méchants qui abandonnent ta loi.
\VS{54}Tes statuts sont le sujet de mes cantiques dans la maison où je suis étranger.
\VS{55}Yahweh, je me souviens de ton Nom pendant la nuit et je garde ta loi.
\VS{56}Cela m'arrive parce que je garde tes commandements.
\VS{57}[Heth.] Ô Yahweh ! J'en conclus que ma part est de garder tes paroles.
\VS{58}Je te supplie de tout mon cœur : Aie pitié de moi selon ta parole.
\VS{59}Je fais le compte de mes voies et je rebrousse chemin vers tes préceptes\FTNT{Os. 6:3 ; La. 3:40.}.
\VS{60}Je me hâte, je ne diffère point de garder tes commandements.
\VS{61}Une compagnie de méchants me pille, mais je n'oublie pas ta loi.
\VS{62}Je me lève au milieu de la nuit pour te célébrer à cause des ordonnances de ta justice.
\VS{63}Je suis l'ami de tous ceux qui te craignent et qui gardent tes commandements.
\VS{64}Yahweh, la terre est pleine de ta bonté ; enseigne-moi tes statuts.
\VS{65}[Teth.] Yahweh, tu fais du bien à ton serviteur selon ta parole.
\VS{66}Enseigne-moi le bon sens et la connaissance car je crois à tes commandements.
\VS{67}Avant d'avoir été humilié, je m'égarais, mais maintenant j'observe ta parole.
\VS{68}Tu es bon et bienfaisant, enseigne-moi tes statuts.
\VS{69}Les orgueilleux imaginent des faussetés contre moi, mais je garde de tout mon cœur tes commandements.
\VS{70}Leur cœur est insensible comme la graisse, mais moi, je prends plaisir dans ta loi\FTNT{De. 32:15 ; Jé. 5:28.}.
\VS{71}Il est bon que je sois humilié afin que j'apprenne tes statuts.
\VS{72}La loi que tu as prononcée de ta bouche m'est plus précieuse que mille pièces d'or ou d'argent\FTNT{Ps. 19:10-11 ; Job. 22:2.}.
\VS{73}[Yod.] Tes mains m'ont façonné, elles m'ont formé\FTNT{Jé. 1:5 ; Job. 10:9.} ; donne-moi l'intelligence afin que j'apprenne tes commandements.
\VS{74}Ceux qui te craignent me verront et se réjouiront, parce que j'espère en tes promesses.
\VS{75}Je reconnais, ô Yahweh, que tes jugements sont justes, et que tu m'as humilié par ta fidélité\FTNT{Hé. 12:10.}.
\VS{76}Que ta bonté soit ma consolation, comme tu l'as promis à ton serviteur.
\VS{77}Que tes compassions viennent sur moi et je vivrai ; car ta loi fait mes délices.
\VS{78}Que les orgueilleux rougissent de honte, de ce qu'ils m'oppriment sans cause ; mais moi, je médite sur tes ordonnances.
\VS{79}Que ceux qui te craignent et ceux qui connaissent tes préceptes reviennent vers moi.
\VS{80}Que mon cœur soit intègre dans tes statuts afin que je ne sois pas couvert de honte.
\VS{81}[Kaf.] Mon âme se consume en attendant ta délivrance ; j'espère en ta promesse.
\VS{82}Mes yeux s'épuisent en attendant ta promesse, lorsque je dis : Quand me consoleras-tu ?
\VS{83}Car je suis comme une outre dans la fumée, je n'oublie pas tes statuts.
\VS{84}Quel est le nombre de jours de ton serviteur ? Quand jugeras-tu ceux qui me poursuivent\FTNT{Ap. 6:10.} ?
\VS{85}Les orgueilleux me creusent des fosses, ils n'agissent pas selon ta loi.
\VS{86}Tous tes commandements ne sont que fidélité ; on me persécute sans cause, aide-moi\FTNT{Mt. 5:10.} !
\VS{87}On m'a presque réduit à rien et mis par terre ; mais je n'ai point abandonné tes commandements.
\VS{88}Fais-moi revivre selon ta miséricorde et je garderai les préceptes de ta bouche.
\VS{89}[Lamed.] Ô Yahweh ! Ta parole subsiste à toujours dans les cieux.
\VS{90}Ta fidélité dure d'âge en âge ; tu as établi la terre, et elle demeure ferme\FTNT{Pr. 1:4.}.
\VS{91}Ces choses subsistent aujourd'hui selon tes ordonnances, car toutes choses te servent.
\VS{92}Si ta loi n'avait pas fait mes délices, j'aurais déjà péri dans mon affliction.
\VS{93}Je n'oublierai jamais tes commandements car c'est par eux que tu m'as fait revivre.
\VS{94}Je suis à toi, sauve-moi ; car je recherche tes commandements.
\VS{95}Les méchants m'attendent pour me faire périr, mais je suis attentif à tes préceptes.
\VS{96}Je vois des bornes à tout ce qui est parfait, mais tes commandements n'ont point de limites.
\VS{97}[Mem.] Combien j'aime ta loi\FTNT{Ps. 1:2.} ! Elle est tout le jour l'objet de ma méditation.
\VS{98}Par tes commandements, tu m'as rendu plus sage que mes ennemis, parce que tes commandements sont toujours avec moi.
\VS{99}J'ai surpassé en prudence tous ceux qui m'avaient enseigné parce que tes préceptes sont l'objet de ma méditation.
\VS{100}Je suis devenu plus intelligent que les vieillards parce que j'observe tes commandements.
\VS{101}Je garde mes pieds de toute mauvaise voie afin d'observer ta parole.
\VS{102}Je ne me suis point détourné de tes ordonnances parce que tu me les enseignes.
\VS{103}Que ta parole est douce à mon palais ! Plus douce que le miel à ma bouche.
\VS{104}Je suis devenu intelligent par tes commandements, c'est pourquoi je hais toute voie de mensonge.
\VS{105}[Nun.] Ta parole est une lampe à mes pieds et une lumière sur mon sentier\FTNT{Pr. 6:23 ; 2 Pi. 1:19.}.
\VS{106}J'ai juré et je le tiendrai, d'observer les lois de ta justice\FTNT{Né. 10:29.}.
\VS{107}Yahweh, je suis extrêmement affligé, fais-moi revivre selon ta parole.
\VS{108}Yahweh, je te prie, agrée les sentiments que ma bouche exprime, et enseigne-moi tes ordonnances\FTNT{Os. 14:2 ; Hé. 13:15.}.
\VS{109}Ma vie est continuellement en danger, toutefois je n'oublie pas ta loi.
\VS{110}Les méchants m'ont tendu des pièges, toutefois je ne me suis point égaré de tes commandements.
\VS{111}J'ai pris pour héritage perpétuel tes préceptes car ils sont la joie de mon cœur.
\VS{112}J'ai incliné mon cœur à accomplir toujours tes statuts jusqu'au bout.
\VS{113}[Samech.] Je hais les hommes indécis\FTNT{1 R. 18:21 ; Ja. 1:6 ; Ja. 4:8.}, mais j'aime ta loi.
\VS{114}Tu es mon refuge et mon bouclier, je m'attends à ta parole.
\VS{115}Méchants, retirez-vous de moi\FTNT{Mt. 7:23 ; Ps. 6:9.} ! Et je garderai les commandements de mon Dieu.
\VS{116}Soutiens-moi suivant ta parole, et je vivrai ; et ne me fais point rougir de honte en me refusant ce que j'espérais.
\VS{117}Soutiens-moi, et je serai en sûreté ; et j'aurai continuellement les yeux sur tes statuts.
\VS{118}Tu as foulé aux pieds tous ceux qui se détournent de tes statuts, car le mensonge est le moyen dont ils se servent pour tromper.
\VS{119}Tu réduis à néant tous les méchants de la terre, comme de l'écume ; c'est pourquoi j'aime tes préceptes.
\VS{120}Ma chair frissonne de l'effroi que tu m'inspires et je crains tes jugements\FTNT{Ha. 3:16.}.
\VS{121}[Ayin.] J'ai exercé le jugement et la justice, ne m'abandonne pas à ceux qui me font tort.
\VS{122}Prends sous ta garantie le bien de ton serviteur et ne permets pas que je sois opprimé par les orgueilleux.
\VS{123}Mes yeux s'épuisent en attendant ta délivrance et la parole de ta justice.
\VS{124}Agis envers ton serviteur suivant ta miséricorde et enseigne-moi tes statuts.
\VS{125}Je suis ton serviteur, donne-moi l'intelligence, et je connaîtrai tes préceptes\FTNT{Pr. 1:4 ; Pr. 6:23.}.
\VS{126}Il est temps que Yahweh opère ; ils ont aboli ta loi.
\VS{127}C'est pourquoi j'aime tes commandements, plus que l'or et l'or fin.
\VS{128}C'est pourquoi je trouve justes tous tes commandements, je hais toute voie de mensonge.
\VS{129}[Pe.] Tes préceptes sont merveilleux, c'est pourquoi mon âme les garde.
\VS{130}La révélation de tes paroles éclaire, elle donne de l'intelligence aux simples.
\VS{131}J'ouvre ma bouche et je soupire, car je désire tes commandements.
\VS{132}Regarde-moi, et aie pitié de moi, selon tes jugements à l'égard de ceux qui aiment ton Nom.
\VS{133}Affermis mes pas sur ta parole, et que l'iniquité n'ait point d'emprise sur moi.
\VS{134}Délivre-moi de l'oppression des hommes afin que je garde tes commandements.
\VS{135}Fais luire ta face sur ton serviteur et enseigne-moi tes statuts.
\VS{136}Mes yeux répandent des torrents d'eau parce qu'on n'observe point ta loi.
\VS{137}[Tsade.] Tu es juste, ô Yahweh, et droit dans tes jugements.
\VS{138}Tu ordonnes tes préceptes avec justice et grande fidélité.
\VS{139}Mon zèle me consume parce que mes adversaires oublient tes paroles.
\VS{140}Ta parole est entièrement éprouvée, c'est pourquoi ton serviteur l'aime.
\VS{141}Je suis petit et méprisé, toutefois je n'oublie point tes commandements.
\VS{142}Ta justice est une justice éternelle, et ta loi est la vérité.
\VS{143}La détresse et l'angoisse m'atteignent, mais tes commandements font mes délices.
\VS{144}Tes préceptes ne sont que justice éternelle ; donne-moi l'intelligence afin que je vive.
\VS{145}[Qof.] Je crie de tout mon cœur, réponds-moi, ô Yahweh ! Je garde tes statuts.
\VS{146}Je crie vers toi, sauve-moi afin que j'observe tes préceptes.
\VS{147}Je devance l'aurore et je crie ; je m'attends à ta parole.
\VS{148}Mes yeux ont devancé les veilles de la nuit pour méditer ta parole.
\VS{149}Ecoute ma voix selon ta miséricorde, ô Yahweh ! Fais-moi revivre selon ton ordonnance.
\VS{150}Ceux qui poursuivent le crime s'approchent de moi, et ils s'éloignent de ta loi.
\VS{151}Yahweh, tu es près de moi ; et tous tes commandements ne sont que vérité.
\VS{152}Depuis longtemps, je sais par tes préceptes, que tu les as établis pour toujours.
\VS{153}[Resh.] Regarde mon affliction et sauve-moi, car je n'oublie pas ta loi.
\VS{154}Soutiens ma cause et rachète-moi ; fais-moi revivre selon ta parole.
\VS{155}La délivrance est loin des méchants parce qu'ils ne recherchent pas tes statuts.
\VS{156}Tes compassions sont en grand nombre, ô Yahweh ! Fais-moi revivre selon tes ordonnances.
\VS{157}Ceux qui me persécutent et qui me pressent sont en grand nombre, toutefois je ne me détourne pas de tes préceptes.
\VS{158}Je vois avec dégoût les traîtres et je suis rempli de tristesse car ils n'observent pas ta parole.
\VS{159}Regarde combien j'aime tes commandements, Yahweh ! Fais-moi revivre selon ta miséricorde !
\VS{160}Le fondement de ta parole est la vérité, et toutes les lois de ta justice sont éternelles.
\VS{161}[Shin.] Les princes du peuple me persécutent sans cause, mais mon cœur tremble à cause de ta parole.
\VS{162}Je me réjouis de ta parole comme ferait celui qui aurait trouvé un grand butin.
\VS{163}J'ai en haine et en abomination le mensonge ; j'aime ta loi.
\VS{164}Sept fois le jour je te loue à cause des ordonnances de ta justice.
\VS{165}Il y a une grande paix pour ceux qui aiment ta loi, et rien ne peut les renverser\FTNT{Es. 32:17 ; Ph. 4:7.}.
\VS{166}Yahweh, j'espère en ta délivrance et je pratique tes commandements.
\VS{167}Mon âme observe tes préceptes, je les aime beaucoup.
\VS{168}J'observe tes commandements et tes préceptes, car toutes mes voies sont devant toi.
\VS{169}[Tav.] Yahweh, que mon cri parvienne jusqu'à toi, donne-moi l'intelligence selon ta parole.
\VS{170}Que ma supplication vienne devant toi, délivre-moi selon ta parole.
\VS{171}Mes lèvres publieront ta louange quand tu m'auras enseigné tes statuts.
\VS{172}Ma langue ne s'entretiendra que de ta parole, parce que tous tes commandements ne sont que justice.
\VS{173}Que ta main me soit en aide, parce que j'ai choisi tes commandements.
\VS{174}Yahweh, je souhaite ta délivrance, et ta loi fait mes délices.
\VS{175}Que mon âme vive afin qu'elle te loue, et que tes ordonnances me soient en aide !
\VS{176}Je suis errant comme une brebis perdue\FTNT{Es. 53:6 ; Lu. 15:4 ; 1 Pi. 2:25.} ; cherche ton serviteur, car je n'oublie pas tes commandements.
\Chap{120}
\TextTitle{Cri de détresse}
\VerseOne{}Cantique des degrés\FTNT{Les psaumes 120 à 134 sont appelés « psaumes des degrés » ou de « l'ascension ». Ces psaumes furent chantés par les Israélites montant à Jérusalem au retour de la captivité de Babylone.}. J'ai invoqué Yahweh dans ma grande détresse, et il m'a exaucé.
\VS{2}Yahweh, délivre mon âme des lèvres mensongères et de la langue trompeuse.
\VS{3}Que te donne, que te rapporte la langue trompeuse ?
\VS{4}Ce sont des flèches aiguës tirées par un homme puissant et des charbons ardents du genêt\FTNT{Jé. 9:3 ; Ja. 3:5-6.}.
\VS{5}Malheureux que je suis de séjourner à Méschec et de demeurer aux tentes de Kédar !
\VS{6}Assez longtemps mon âme a demeuré auprès de ceux qui haïssent la paix !
\VS{7}Je ne cherche que la paix, mais lorsque j'en parle, ils sont pour la guerre.
\Chap{121}
\TextTitle{Yahweh ne dort ni ne sommeille}
\VerseOne{}Cantique des degrés. J'élève mes yeux vers les montagnes, d'où me viendra le secours.
\VS{2}Mon secours vient de Yahweh qui a fait les cieux et la terre\FTNT{Ps. 124:8.}.
\VS{3}Il ne permettra point que ton pied chancelle, celui qui te garde ne sommeillera point\FTNT{Es. 27:3 ; Pr. 3:23.}.
\VS{4}Voici, il ne sommeille ni ne dort celui qui garde Israël.
\VS{5}Yahweh est celui qui te garde, Yahweh est ton ombre à ta main droite\FTNT{Es. 25:4.}.
\VS{6}Pendant le jour, le soleil ne te frappera point, ni la lune pendant la nuit\FTNT{Es. 49:10. Ap. 7:16.}.
\VS{7}Yahweh te gardera de tout mal, il gardera ton âme.
\VS{8}Yahweh gardera ton départ et ton arrivée, dès maintenant et à jamais\FTNT{De. 28:6.}.
\Chap{122}
\TextTitle{Jérusalem, la ville de Yahweh}
\VerseOne{}Cantique des degrés de David. Je me réjouis à cause de ceux qui me disent : Allons à la maison de Yahweh\FTNT{Ps. 84:1-5.} !
\VS{2}Nos pieds s'arrêtent dans tes portes, ô Jérusalem !
\VS{3}Jérusalem, qui est bâtie comme une ville dont les édifices sont joints ensemble,
\VS{4}à laquelle montent les tribus, les tribus de Yahweh, selon le témoignage d'Israël, pour célébrer le Nom de Yahweh.
\VS{5}Car c'est là qu'ont été posés les trônes pour juger\FTNT{Mt. 19:28.}. Les trônes de la maison de David.
\VS{6}Demandez la paix de Jérusalem ; que ceux qui t'aiment jouissent du repos.
\VS{7}Que la paix soit dans tes murs et la tranquillité dans tes palais.
\VS{8}Pour l'amour de mes frères et de mes amis, je prie maintenant pour ta paix.
\VS{9}A cause de la maison de Yahweh notre Dieu, je fais une requête pour ton bonheur.
\Chap{123}
\TextTitle{Les regards fixés sur Yahweh}
\VerseOne{}Cantique des degrés. J'élève mes yeux vers toi qui habites dans les cieux.
\VS{2}Voici, comme les yeux des serviteurs regardent la main de leurs maîtres, comme les yeux de la servante regardent la main de sa maîtresse, ainsi nos yeux regardent à Yahweh notre Dieu, jusqu'à ce qu'il ait pitié de nous\FTNT{Ps. 25:15.}.
\VS{3}Aie pitié de nous, ô Yahweh ! Aie pitié de nous ! Car nous sommes assez rassasiés de mépris !
\VS{4}Notre âme est assez rassasiée des moqueries des orgueilleux, du mépris des hautains.
\Chap{124}
\TextTitle{Yahweh, le Dieu qui secours et protège son peuple}
\VerseOne{}Cantique des degrés, de David. Sans Yahweh, qui nous protégea, qu'Israël le dise !
\VS{2}Sans Yahweh, qui nous protégea, quand les hommes s'élevèrent contre nous ?
\VS{3}Ils nous auraient engloutis tous vivants quand leur colère s'enflamma contre nous.
\VS{4}Alors les eaux nous auraient submergés, les torrents auraient passé sur notre âme.
\VS{5}Alors les flots impétueux auraient passé sur notre âme.
\VS{6}Béni soit Yahweh qui ne nous a point livrés en proie à leurs dents !
\VS{7}Notre âme s'est échappée comme l'oiseau du filet des oiseleurs ; le filet a été rompu, et nous nous sommes échappés\FTNT{Pr. 6:5.}.
\VS{8}Notre secours est dans le Nom de Yahweh\FTNT{Ac. 4:11-12.} qui a fait les cieux et la terre.
\Chap{125}
\TextTitle{Yahweh entoure tous ceux qui se confient en lui}
\VerseOne{}Cantique des degrés. Ceux qui se confient en Yahweh sont comme la montagne de Sion : Elle ne chancelle point et est affermie pour toujours.
\VS{2}Quant à Jérusalem, il y a des montagnes autour d'elle, ainsi Yahweh entoure son peuple, dès maintenant et à jamais.
\VS{3}Car la verge de la méchanceté ne restera pas sur le lot des justes, de peur que les justes n'étendent leurs mains vers l'iniquité\FTNT{Es. 14:5.}.
\VS{4}Yahweh, répands tes bienfaits sur les bons et sur ceux dont le cœur est droit.
\VS{5}Mais ceux qui s'engagent dans des voies détournées, que Yahweh les fasse marcher avec les ouvriers d'iniquité\FTNT{Mt. 7:23.}. La paix sera sur Israël.
\Chap{126}
\TextTitle{Yahweh, le libérateur}
\VerseOne{}Cantique des degrés. Quand Yahweh ramena les captifs de Sion, nous étions comme ceux qui font un rêve.
\VS{2}Alors notre bouche était remplie de joie, et notre langue de chants de triomphe, alors on disait parmi les nations : Yahweh a fait de grandes choses pour eux !
\VS{3}Yahweh a fait de grandes choses pour nous ; nous sommes dans la joie.
\VS{4}Ô Yahweh ! Ramène nos captifs, comme des ruisseaux dans le midi\FTNT{Os. 6:11 ; Joë. 3:11.} !
\VS{5}Ceux qui sèment avec larmes moissonneront avec chants d'allégresse\FTNT{Ga. 6:9.}.
\VS{6}Celui qui marche en pleurant quand il porte la semence pour la mettre en terre, revient avec des chants d'allégresse quand il porte ses gerbes.
\Chap{127}
\TextTitle{Yahweh, le plus grand architecte}
\VerseOne{}Cantique des degrés, de Salomon. Si Yahweh ne bâtit la maison, ceux qui la bâtissent travaillent en vain ; si Yahweh ne garde la ville, celui qui la garde fait le guet en vain.
\VS{2}C'est en vain que vous vous levez de grand matin, que vous vous couchez tard, et que vous mangez le pain de douleurs ; certes c'est Dieu qui donne du repos à celui qu'il aime\FTNT{Ez. 20:20 ; Mc. 2:27.}.
\VS{3}Voici, les fils sont un héritage donné par Yahweh et le fruit du ventre est une récompense de Dieu\FTNT{Ps. 113:9 ; Ps. 128:3-6.}.
\VS{4}Telles sont les flèches dans la main d'un homme puissant, tels sont les fils de la jeunesse.
\VS{5}Heureux l'homme qui en a rempli son carquois ! Ils ne seront pas honteux quand ils parleront avec leurs ennemis à la porte.
\Chap{128}
\TextTitle{Yahweh assure la paix à celui qui le craint}
\VerseOne{}Cantique des degrés. Heureux tout homme qui craint Yahweh et marche dans ses voies !
\VS{2}Tu jouis du travail de tes mains ; tu es heureux et tu prospères\FTNT{Es. 3:10.}.
\VS{3}Ta femme est dans ta maison comme une vigne qui porte du fruit ; tes fils sont autour de ta table comme des plants d'oliviers.
\VS{4}Voici, certainement ainsi sera béni l'homme qui craint Yahweh.
\VS{5}Yahweh te bénira de Sion et tu verras le bien de Jérusalem tous les jours de ta vie.
\VS{6}Tu verras les fils de tes fils. La paix sera sur Israël.
\Chap{129}
\TextTitle{L'opprimé plus que vainqueur en Yahweh}
\VerseOne{}Cantique des degrés. Qu'Israël dise maintenant : Ils m'ont souvent tourmenté dès ma jeunesse.
\VS{2}Ils m'ont assez opprimé dès ma jeunesse, mais ils ne m'ont pas vaincu.
\VS{3}Des laboureurs ont labouré mon dos, ils y ont tracé de longs sillons.
\VS{4}Yahweh est juste, il a coupé les cordes des méchants.
\VS{5}Qu'ils soient honteux et qu'ils reculent, tous ceux qui haïssent Sion !
\VS{6}Qu'ils soient comme l'herbe des toits qui sèche avant qu'on l'arrache !
\VS{7}Le moissonneur n'en remplit point sa main, ni celui qui lie les gerbes n'en remplit point ses bras ;
\VS{8}et les passants ne disent pas : Que la bénédiction de Yahweh soit sur vous ! Nous vous bénissons au nom de Yahweh !
\Chap{130}
\TextTitle{La rédemption en abondance auprès de Yahweh}
\VerseOne{}Cantique des degrés. Ô Yahweh ! Je t'invoque du fond de l'abîme.
\VS{2}Seigneur, écoute ma voix ! Que tes oreilles soient attentives à la voix de mes supplications !
\VS{3}Yahweh ! si tu prends garde aux iniquités, Seigneur, qui subsistera ?
\VS{4}Mais le pardon se trouve auprès de toi, afin qu'on te craigne\FTNT{Mt. 26:28 ; Ro. 3:24 ; Col. 1:12-14.}.
\VS{5}J'espère en Yahweh, mon âme espère, et j'attends sa parole.
\VS{6}Mon âme attend le Seigneur plus que les sentinelles n'attendent le matin, plus que les sentinelles n'attendent le matin.
\VS{7}Israël, attends-toi à Yahweh, car Yahweh est miséricordieux et la rédemption est auprès de lui en abondance.
\VS{8}Lui-même rachètera Israël de toutes ses iniquités.
\Chap{131}
\TextTitle{Mettre son espoir en Yahweh seul}
\VerseOne{}Cantique des degrés, de David. Ô Yahweh ! Je n'ai ni un cœur qui s'élève ni un regard hautain\FTNT{Pr. 16:5 ; Pr. 6:17.} ; je ne m'occupe pas de choses trop grandes et trop extraordinaires pour moi.
\VS{2}J'ai l'âme calme et tranquille comme un enfant sevré de sa mère ; j'ai l'âme comme un enfant sevré.
\VS{3}Israël attends-toi à Yahweh dès maintenant et à jamais !
\Chap{132}
\TextTitle{Sion, le trône de Yahweh}
\VerseOne{}Cantique des degrés. Ô Yahweh ! Souviens-toi de David et de toute son affliction !
\VS{2}Il a juré à Yahweh et fait ce vœu au puissant de Jacob :
\VS{3}Je n'entrerai pas dans la tente où j'habite, je ne monterai pas sur le lit où je couche,
\VS{4}je ne donnerai pas du sommeil à mes yeux, je ne laisserai pas sommeiller mes paupières,
\VS{5}jusqu'à ce que j'aie trouvé un lieu pour Yahweh, une demeure pour le puissant de Jacob\FTNT{1 Ch. 15:1.}.
\VS{6}Voici, nous avons entendu parler d'elle à Ephrata, nous l'avons trouvée dans les champs de Jaar.
\VS{7}Entrons dans sa demeure, prosternons-nous devant son marchepied.
\VS{8}Lève-toi, ô Yahweh, pour venir à ton lieu de repos, toi et l'arche de ta force\FTNT{No. 10:35-36 ; 2 Ch. 6:41.}.
\VS{9}Que tes sacrificateurs soient revêtus de justice et que tes bien-aimés chantent de joie\FTNT{Es. 11:5 ; Ap. 19:8.} !
\VS{10}Pour l'amour de David, ton serviteur, ne permets pas que ton oint retourne en arrière !
\VS{11}Yahweh a juré la vérité à David, et il ne se rétractera pas, disant : Je mettrai le fruit de tes entrailles\FTNT{2 S. 7:12 ; 1 R. 8:25 ; 2 Ch. 6:16 ; Lu. 1:69 ; Ac. 2:30.} sur ton trône.
\VS{12}Si tes fils gardent mon alliance et mon témoignage que je leur enseignerai, leurs fils aussi seront assis à perpétuité sur ton trône.
\VS{13}Car Yahweh a choisi Sion, il l'a préférée pour être son trône :
\VS{14}Elle est mon lieu de repos à perpétuité, j'y habiterai parce que je l'ai désirée.
\VS{15}Je bénirai abondamment sa nourriture, je rassasierai de pain ses pauvres.
\VS{16}Je revêtirai de salut ses sacrificateurs, et ses bien-aimés chanteront avec des cris de joie.
\VS{17}Je ferai qu'en elle germera une corne à David ; je préparerai une lampe à mon oint,
\VS{18}je revêtirai de honte ses ennemis, et sur lui fleurira son diadème.
\Chap{133}
\TextTitle{La bénédiction dans la communion fraternelle}
\VerseOne{}Cantique des degrés. De David. Voici, oh ! Que c'est une chose bonne et que c'est une chose agréable que des frères demeurent unis ensemble\FTNT{Hé. 13:1 ; Ac. 2:46.} !
\VS{2}C'est comme cette huile précieuse, répandue sur la tête, qui coule sur la barbe d'Aaron\FTNT{Ex. 30:22-30.}, sur le bord de ses vêtements ;
\VS{3}comme la rosée de l'Hermon, celle qui descend sur les montagnes de Sion. Car c'est là que Yahweh a ordonné la bénédiction et la vie, pour l'éternité.
\Chap{134}
\TextTitle{Bénissez Yahweh, vous tous ses serviteurs}
\VerseOne{}Cantique des degrés. Voici, bénissez Yahweh ! Vous tous les serviteurs de Yahweh ! Qui vous tenez toutes les nuits dans la maison de Yahweh !
\VS{2}Elevez vos mains vers le lieu saint ! Et bénissez Yahweh !
\VS{3}Que Yahweh, qui a fait les cieux et la terre, te bénisse de Sion !
\Chap{135}
\TextTitle{La souveraineté de Dieu}
\VerseOne{}Louez le Nom de Yahweh ! Vous serviteurs de Yahweh ! Louez-le !
\VS{2}Vous qui vous tenez dans la maison de Yahweh, dans les parvis de la maison de notre Dieu,
\VS{3}louez Yahweh, car Yahweh est bon ! Chantez son Nom, car il est agréable !
\VS{4}Car Yahweh s'est choisi Jacob et Israël pour sa possession\FTNT{Ex. 19:5 ; De. 7:6 ; Tit. 2:14 ; 1 Pi. 2:9.}.
\VS{5}Certainement, je sais que Yahweh est grand et que notre Seigneur est au-dessus de tous les dieux.
\VS{6}Yahweh fait tout ce qu'il lui plaît, dans les cieux et sur la terre, dans la mer et dans tous les abîmes.
\VS{7}C'est lui qui fait monter les vapeurs des extrémités de la terre ; il fait les éclairs et la pluie ; il tire le vent hors de ses trésors.
\VS{8}C'est lui qui a frappé les premiers-nés d'Egypte, tant des hommes que des bêtes ;
\VS{9}qui a envoyé des prodiges et des miracles au milieu de toi, ô Egypte ! Contre Pharaon et contre tous ses serviteurs ;
\VS{10}qui a frappé plusieurs nations et tué les puissants rois ;
\VS{11}Sihon, roi des Amoréens, et Og, roi de Basan, et ceux de tous les royaumes de Canaan\FTNT{No. 21:33-35 ; De. 3:11.} ;
\VS{12}qui a donné leur pays en héritage, en héritage à Israël son peuple.
\VS{13}Yahweh, ton Nom est pour toujours ! Yahweh, ta mémoire de génération en génération !
\VS{14}Car Yahweh jugera son peuple et se repentira à l'égard de ses serviteurs.
\VS{15}Les dieux des nations ne sont que de l'or et de l'argent, un ouvrage de mains d'homme.
\VS{16}Ils ont une bouche, et ne parlent point ; ils ont des yeux, et ne voient point ;
\VS{17}ils ont des oreilles, et n'entendent point ; il n'y a point de souffle dans leur bouche.
\VS{18}Ils leur ressemblent ceux qui les font, et tous ceux qui s'y confient.
\VS{19}Maison d'Israël, bénissez Yahweh ! Maison d'Aaron, bénissez Yahweh !
\VS{20}Maison des Lévites, bénissez Yahweh ! Vous qui craignez Yahweh, bénissez Yahweh !
\VS{21}Béni soit de Sion Yahweh qui habite dans Jérusalem ! Louez Yahweh !
\Chap{136}
\TextTitle{La bonté de Yahweh demeure à toujours}
\VerseOne{}Célébrez Yahweh, car il est bon, car sa bonté demeure à toujours !
\VS{2}Célébrez le Dieu des dieux, car sa bonté demeure à toujours !
\VS{3}Célébrez le Seigneur des seigneurs, car sa bonté demeure à toujours !
\VS{4}Célébrez celui qui seul fait de grandes merveilles, car sa bonté demeure à toujours !
\VS{5}Celui qui a fait avec intelligence les cieux, car sa bonté demeure à toujours !
\VS{6}Celui qui a étendu la terre sur les eaux, car sa bonté demeure à toujours !
\VS{7}Celui qui a fait les grands luminaires, car sa bonté demeure à toujours !
\VS{8}Le soleil pour dominer sur le jour, car sa bonté demeure à toujours !
\VS{9}La lune et les étoiles pour dominer la nuit, car sa bonté demeure à toujours !
\VS{10}Celui qui a frappé l'Egypte dans leurs premiers-nés, car sa bonté demeure à toujours !
\VS{11}Qui a fait sortir Israël du milieu d'eux, car sa bonté demeure à toujours.
\VS{12}Et cela avec main forte et bras étendu, car sa bonté demeure à toujours !
\VS{13}Il a fendu la Mer Rouge en deux, car sa bonté demeure à toujours !
\VS{14}Il a fait passer Israël par le milieu d'elle, car sa bonté demeure à toujours !
\VS{15}Il a renversé Pharaon et son armée dans la Mer Rouge, car sa bonté demeure à toujours !
\VS{16}Il a conduit son peuple dans le désert, car sa bonté demeure à toujours !
\VS{17}Il a frappé les grands rois, car sa bonté demeure à toujours !
\VS{18}Qui a tué des grands rois, car sa bonté demeure à toujours !
\VS{19}Sihon, roi des Amoréens, car sa bonté demeure à toujours !
\VS{20}Og, roi de Basan, car sa bonté demeure à toujours !
\VS{21}Il a donné leur pays en héritage, car sa bonté demeure à toujours\FTNT{Jos. 12:7.} !
\VS{22}En héritage à Israël son serviteur, car sa bonté demeure à toujours !
\VS{23}Et qui, lorsque nous étions humiliés, s'est souvenu de nous, car sa bonté demeure à toujours !
\VS{24}Il nous a délivrés de la main de nos adversaires, car sa bonté demeure à toujours !
\VS{25}Il donne la nourriture à toute chair, car sa bonté demeure à toujours\FTNT{Ps. 104:21 ; Mt. 6:26 ; Ps. 147:9.} !
\VS{26}Célébrez le Dieu des cieux, car sa bonté demeure à toujours !
\Chap{137}
\TextTitle{Le coeur des captifs}
\VerseOne{}Sur les bords des fleuves de Babylone, nous étions assis et nous pleurions en nous souvenant de Sion.
\VS{2}Nous avions suspendu nos harpes au milieu des saules.
\VS{3}Là, ceux qui nous avaient emmenés en captivité, nous ont demandé des paroles de chants, et nos oppresseurs de la joie, en nous disant : Chantez-nous quelques cantiques de Sion ! Nous avons répondu :
\VS{4}Comment chanterions-nous les cantiques de Yahweh sur une terre étrangère ?
\VS{5}Si je t'oublie, Jérusalem, que ma droite s'oublie elle-même.
\VS{6}Que ma langue soit attachée à mon palais\FTNT{Ez. 3:26.}, si je ne me souviens pas de toi, si je ne fais pas de Jérusalem le sujet de ma réjouissance.
\VS{7}Ô Yahweh, souviens-toi des fils d'Edom, qui dans la journée de Jérusalem disaient : Rasez, rasez jusqu'à ses fondements\FTNT{Jé. 25:15-21 ; Jé. 49:7-8 ; Ez. 25:12 ; La. 4:21 ; Am. 1:11.} !
\VS{8}Fille de Babylone, qui va être détruite, heureux celui qui te rend la pareille de ce que tu nous as fait\FTNT{Jé. 50:15-29 ; Ap. 18:6.} !
\VS{9}Heureux celui qui saisit tes petits enfants et qui les écrase contre le rocher\FTNT{Es. 13:16.} !
\Chap{138}
\TextTitle{La renommée de Yahweh dans les nations}
\VerseOne{}Psaume de David. Je te célèbre de tout mon cœur, je te chante des louanges dans la présence de Dieu.
\VS{2}Je me prosterne dans ton saint temple, et je célèbre ton Nom à cause de ta bonté et de ta fidélité ; car ta renommée s'est accrue par l'accomplissement de ta promesse.
\VS{3}Le jour où je t'ai invoqué, tu m'as exaucé, tu m'as rassuré, tu m'as fortifié d'une nouvelle force en mon âme.
\VS{4}Yahweh ! Tous les rois de la terre te célèbrent, quand ils entendent les paroles de ta bouche.
\VS{5}Ils chantent les voies de Yahweh, car la gloire de Yahweh est grande.
\VS{6}Car Yahweh est haut élevé, il voit les humbles et il reconnaît de loin les orgueilleux.
\VS{7}Quand je marche au milieu de l'adversité, tu me rends la vie, tu avances ta main contre la colère de mes ennemis, et ta droite me délivre.
\VS{8}Yahweh achèvera ce qui me concerne. Yahweh, ta bonté demeure toujours ; tu n'abandonnes pas l'œuvre de tes mains\FTNT{Ph. 1:6.}.
\Chap{139}
\TextTitle{L'omniscience de Yahweh}
\VerseOne{}Psaume de David, donné au chef des chantres. Yahweh, tu me sondes et tu me connais\FTNT{Jé. 12:3 ; Ps. 17:3.}.
\VS{2}Tu sais quand je m'assieds et quand je me lève ; tu discernes de loin ma pensée.
\VS{3}Tu sais quand je marche et quand je me couche ; tu connais parfaitement toutes mes voies.
\VS{4}Avant que la parole soit sur ma langue, voici, ô Yahweh, tu la connais déjà !
\VS{5}Tu m'entoures par derrière et par devant, et tu mets ta main sur moi.
\VS{6}Ta science est trop merveilleuse pour moi, elle est si haut élevée que je ne saurais l'atteindre\FTNT{Job. 42:3 ; Ps. 92:6 ; Ro. 11:33.}.
\VS{7}Où irai-je loin de ton Esprit, et où fuirai-je loin de ta face\FTNT{Jé. 23:24 ; Am. 9:2-4 ; Jon. 1:3.} ?
\VS{8}Si je monte aux cieux, tu y es ; si je me couche dans le scheol, t'y voilà.
\VS{9}Si je prends les ailes de l'aurore et que je demeure à l'extrémité de la mer,
\VS{10}là aussi ta main me conduira et ta droite me saisira.
\VS{11}Si je dis : Au moins les ténèbres me couvriront, la nuit même sera une lumière tout autour de moi.
\VS{12}Même les ténèbres ne me cacheront point de toi, et la nuit resplendira comme le jour, et les ténèbres comme la lumière.
\VS{13}Tu as créé mes reins, tu me couvres du sein de ma mère.
\VS{14}Je te célèbre de ce que je suis une créature redoutée et merveilleuse ; tes œuvres sont merveilleuses, et mon âme le reconnaît très bien.
\VS{15}Mon corps n'était pas caché devant toi lorsque j'ai été fait dans un lieu secret et brodé dans les profondeurs de la terre\FTNT{Ps. 119:73 ; Ec. 11:5.}.
\VS{16}Tes yeux me voyaient quand je n'étais qu'un embryon, et sur ton livre étaient inscrits tous les jours qui m'étaient destinés\FTNT{Ph. 4:3 ; Ap. 3:5 ; Ap. 20:15.}.
\VS{17}Dieu ! Que tes pensées sont précieuses ! Que le nombre en est grand !
\VS{18}Si je les compte, elles sont plus nombreuses que les grains de sable. Je m'éveille et je suis encore avec toi.
\VS{19}Ô Dieu ! Ne tueras-tu pas le méchant ? C'est pourquoi, hommes sanguinaires, retirez-vous loin de moi !
\VS{20}Car ils ont parlé de toi en pensant à quelque méchanceté ; ils ont élevé tes ennemis en mentant.
\VS{21}Yahweh, n'aurais-je point en haine ceux qui te haïssent ; et ne serais-je point irrité contre ceux qui s'élèvent contre toi ?
\VS{22}Je les hais d'une parfaite haine ; ils sont pour moi des ennemis.
\VS{23}Ô Dieu ! Sonde-moi et considère mon cœur ! Eprouve-moi et considère mes discours !
\VS{24}Et regarde si je suis sur une mauvaise voie ; conduis-moi sur la voie de l'éternité.
\Chap{140}
\TextTitle{Yahweh, le protecteur}
\VerseOne{}Psaume de David, donné au chef des chantres. Yahweh, délivre-moi de l'homme méchant, garde-moi de l'homme violent.
\VS{2}Ils méditent des méchancetés dans leur cœur, tous les jours ils complotent des guerres.
\VS{3}Ils aiguisent leur langue comme un serpent, il y a du venin de vipère sous leurs lèvres. Sélah.
\VS{4}Yahweh, garde-moi de la main du méchant, préserve-moi de l'homme violent, de ceux qui méditent de me faire tomber.
\VS{5}Les orgueilleux me tendent un piège et des filets, et ils étendent des rets le long du chemin, ils me dressent des embûches. Sélah.
\VS{6}Je dis à Yahweh : Tu es mon Dieu, Yahweh ! Prête l'oreille à la voix de mes supplications !
\VS{7}Ô Yahweh ! Seigneur ! La force de mon salut ! Tu couvres ma tête au jour de la bataille.
\VS{8}Yahweh n'accorde point au méchant ses désirs ; qu'il n'apporte pas ses méchants desseins, ils s'élèveraient. Sélah.
\VS{9}Quant à la tête de ceux qui m'environnent, que la méchanceté de leurs lèvres les recouvre.
\VS{10}Que des charbons ardents soient jetés sur eux ! Qu'ils tombent sur eux ! Qu'il les fasse tomber dans le feu, et dans des fosses profondes, sans qu'ils se relèvent\FTNT{Pr. 25:21-22 ; Ro. 12:20.} !
\VS{11}Que l'homme à la langue méchante ne soit point affermi sur la terre ; quant à l'homme violent et mauvais, qu'on le chasse jusqu'à ce qu'il soit exterminé.
\VS{12}Je sais que Yahweh fera justice au malheureux et droit aux indigents.
\VS{13}Quoi qu'il en soit, les justes célébreront ton Nom, les hommes droits habiteront devant ta face.
\Chap{141}
\TextTitle{Yahweh, garde-moi du mal !}
\VerseOne{}Psaume de David. Yahweh, je t'invoque, hâte-toi de venir vers moi ; prête l'oreille à ma voix lorsque je crie à toi.
\VS{2}Que ma prière te soit agréable comme l'encens, et l'élévation de mes mains comme l'offrande du soir\FTNT{Ex. 30:1 ; Ap. 5:8 ; Ap. 8:3.}.
\VS{3}Yahweh, mets une garde à ma bouche, garde l'entrée de mes lèvres.
\VS{4}N'incline point mon cœur à des choses mauvaises, au point que je commette quelques méchantes actions par malice, avec les hommes qui font le mal ; et que je ne mange point de leurs délices.
\VS{5}Que le juste me frappe, ce me sera une faveur ; et qu'il me réprimande, ce sera pour moi un baume excellent\FTNT{Pr. 27:6 ; Ec. 7:5.} ; il ne blessera point ma tête ; car ma prière sera pour eux leur calamité.
\VS{6}Que leurs juges soient précipités le long des rochers, et l'on écoutera mes paroles, car elles sont agréables.
\VS{7}Nos os sont dispersés dans la bouche du scheol comme quand on laboure la terre et on fend le bois.
\VS{8}C'est pourquoi, ô Yahweh, Seigneur, mes yeux sont sur toi, je me suis retiré vers toi, n'abandonne point mon âme !
\VS{9}Garde-moi du piège qu'ils m'ont tendu et des filets de ceux qui font le mal.
\VS{10}Que tous les méchants tombent dans leurs filets, jusqu'à ce que je sois passé.
\Chap{142}
\TextTitle{Yahweh, mon refuge}
\VerseOne{}Cantique de David. Prière qu'il fit lorsqu'il était dans la caverne\FTNT{1 S. 24:4.}.
\VS{2}Je crie de ma voix à Yahweh, je supplie de ma voix Yahweh.
\VS{3}Je répands devant lui ma complainte, je déclare mon angoisse devant lui\FTNT{1 S. 1:15 ; La. 2:19.}.
\VS{4}Quand mon esprit est abattu en moi, toi, tu connais mon sentier. Ils me tendent un piège sur le chemin par lequel je marche.
\VS{5}Je contemple à ma droite et je regarde, et il n'y a personne qui me reconnaît ; tout refuge s'évanouit devant moi, il n'y a personne qui prend soin de mon âme.
\VS{6}Yahweh, je crie vers toi ; je dis : Tu es mon refuge, ma part sur la terre des vivants.
\VS{7}Sois attentif à mon cri car je suis devenu très affaibli. Délivre-moi de ceux qui me poursuivent car ils sont plus puissants que moi.
\VS{8}Retire mon âme de sa prison afin que je célèbre ton Nom ! Les justes viendront m'entourer quand tu m'auras fait du bien.
\Chap{143}
\TextTitle{Yahweh, enseigne-moi à faire ta volonté}
\VerseOne{}Psaume de David. Yahweh, écoute ma requête, prête l'oreille à mes supplications ! Exauce-moi dans ta fidélité, réponds-moi à cause de ta justice !
\VS{2}N'entre point en jugement avec ton serviteur, car aucun homme vivant n'est juste devant toi.
\VS{3}Car l'ennemi poursuit mon âme, il foule ma vie par terre ; il me fait habiter dans les ténèbres comme ceux qui sont morts depuis longtemps.
\VS{4}Et mon esprit est abattu au-dedans de moi, mon cœur est épouvanté en mon sein.
\VS{5}Je me souviens des jours anciens, je médite sur toutes tes œuvres, je médite sur l'ouvrage de tes mains\FTNT{Ps. 77:11-13.}.
\VS{6}J'étends mes mains vers toi ; mon âme s'adresse à toi comme une terre desséchée\FTNT{Ps. 28:1 ; Ps. 42:1-3.}. Sélah.
\VS{7}Ô Yahweh, hâte-toi, réponds-moi ! Mon esprit se consume ! Ne me cache point ta face au point que je devienne semblable à ceux qui descendent dans la fosse !
\VS{8}Fais-moi entendre dès le matin ta miséricorde, car je me confie en toi ; fais-moi connaître le chemin par lequel je dois marcher, car j'ai élevé mon cœur vers toi\FTNT{Ps. 25:1.}.
\VS{9}Yahweh, délivre-moi de mes ennemis, car je me suis réfugié auprès de toi !
\VS{10}Enseigne-moi à faire ta volonté, car tu es mon Dieu ! Que ton bon Esprit me conduise sur la voie de la droiture\FTNT{Jn. 16:13.} !
\VS{11}Yahweh, rends-moi la vie pour l'amour de ton Nom ! Retire mon âme de la détresse à cause de ta justice !
\VS{12}Et selon la bonté que tu as pour moi, retranche mes ennemis ! Détruis tous ceux qui tiennent mon âme oppressée, parce que je suis ton serviteur !
\Chap{144}
\TextTitle{Se confier en Yahweh, le Rocher}
\VerseOne{}Psaume de David. Béni soit Yahweh, mon rocher\FTNT{Voir commentaire en Es. 8:13-17.} qui exerce mes mains au combat et mes doigts à la bataille,
\VS{2}qui déploie sa bonté envers moi, qui est ma forteresse, ma haute retraite, mon libérateur\FTNT{Es. 59:20-21 ; Ro. 11:26.}, mon bouclier\FTNT{Ep. 6:16.}, mon refuge\FTNT{Ps. 91 ; Mt. 11:28-30.}, qui m'assujettit mon peuple.
\VS{3}Ô Yahweh ! Qu'est-ce que l'homme pour que tu aies soin de lui\FTNT{Ps. 8:5 ; Job. 7:17 ; Hé. 2:6-7.} ? Le fils de l'homme mortel pour que tu prennes garde à lui ?
\VS{4}L'homme est semblable à la vanité, ses jours sont comme une ombre qui passe\FTNT{Ps. 102:12 ; Job. 14:1-2 ; Ec. 6:12.}.
\VS{5}Yahweh abaisse tes cieux et descends ! Touche les montagnes et qu'elles soient fumantes\FTNT{Es. 63:19 ; Ps. 18:7-8.}.
\VS{6}Lance les éclairs et disperse mes ennemis ! Lance tes flèches et mets-les en déroute !
\VS{7}Etends tes mains d'en haut ; sauve-moi et délivre-moi des grandes eaux, de la main des fils de l'étranger,
\VS{8}dont la bouche profère le mensonge, et dont la droite est une droite trompeuse !
\VS{9}Ô Dieu ! Je chanterai un cantique nouveau ! Je te célèbrerai sur le luth à dix cordes !
\VS{10}Toi qui donnes la délivrance aux rois et qui délivres de l'épée meurtrière David, ton serviteur.
\VS{11}Retire-moi et délivre-moi de la main des fils de l'étranger, dont la bouche profère le mensonge et dont la droite est une droite trompeuse ;
\VS{12}afin que nos fils soient comme des plantes qui croissent dans leur jeunesse et nos filles comme des pierres angulaires taillées pour l'ornement d'un palais.
\VS{13}Que nos greniers soient pleins, fournissant toute espèce de provision ; que nos troupeaux multiplient par milliers, même par dix milliers dans nos rues.
\VS{14}Que nos bœufs soient chargés de graisse. Qu'il n'y ait ni brèche, ni sortie dans nos murailles, ni cri dans nos places.
\VS{15}Heureux le peuple pour qui il en est ainsi ! Heureux le peuple dont Yahweh est le Dieu !
\Chap{145}
\TextTitle{Louange à Yahweh pour tout ce qu'il est}
\VerseOne{}Psaume de louange, composé par David. [Aleph.] Mon Dieu, mon roi, je t'exalterai et je bénirai ton Nom à toujours, et à perpétuité !
\VS{2}[Beth.] Je te bénirai chaque jour, et je louerai ton Nom à toujours, et à perpétuité !
\VS{3}[Guimel.] Yahweh est grand et très digne de louanges, il n'est pas possible de sonder sa grandeur.
\VS{4}[Daleth.] Que chaque génération célèbre tes œuvres et publie tes hauts faits !
\VS{5}[He.] Je dirai la splendeur glorieuse de ta majesté et de tes faits merveilleux.
\VS{6}[Vav.] On parlera de ta puissance redoutable, et je raconterai ta grandeur.
\VS{7}[Zayin.] Ils proclameront le souvenir de ton immense bonté, et ils raconteront avec chants de triomphe ta justice.
\VS{8}[Heth.] Yahweh est miséricordieux et compatissant, lent à la colère et grand en bonté.
\VS{9}[Teth.] Yahweh est bon envers tous et ses compassions sont au-dessus de toutes ses œuvres.
\VS{10}[Yod.] Yahweh, toutes tes œuvres te célébreront, et tes fidèles te béniront.
\VS{11}[Kaf.] Ils diront la gloire de ton règne, et ils proclameront ta puissance
\VS{12}[Lamed.] pour faire connaître aux fils de l'homme ta puissance et la splendeur glorieuse de ton règne.
\VS{13}[Mem.] Ton règne est un règne de tous les siècles et ta domination subsiste dans tous les âges.
\VS{14}[Samech.] Yahweh soutient tous ceux qui tombent et redresse tous ceux qui sont courbés\FTNT{Ps. 146:8.}.
\VS{15}[Ayin.] Les yeux de tous les animaux s'attendent à toi et tu leur donnes leur nourriture en leur temps.
\VS{16}[Pe.] Tu ouvres ta main et tu rassasies à souhait toute créature vivante.
\VS{17}[Tsade.] Yahweh est juste dans toutes ses voies et plein de bonté dans toutes ses œuvres\FTNT{Da. 4:37.}.
\VS{18}[Qof.] Yahweh est près de tous ceux qui l'invoquent, de tous ceux qui l'invoquent avec vérité\FTNT{Ps. 34:18.}.
\VS{19}[Resh.] Il accomplit le désir de ceux qui le craignent, il entend leur cri et les délivre.
\VS{20}[Shin.] Yahweh garde tous ceux qui l'aiment, mais il exterminera tous les méchants.
\VS{21}[Tav.] Ma bouche racontera la louange de Yahweh, et toute chair bénira le Nom de sa sainteté à toujours, et à perpétuité\FTNT{Ps. 103:1.}.
\Chap{146}
\TextTitle{La fidélité de Yahweh dure à toujours}
\VerseOne{}Louez Yahweh ! Mon âme, loue Yahweh !
\VS{2}Je louerai Yahweh durant ma vie, je chanterai mon Dieu tant que je vivrai !
\VS{3}Ne vous confiez pas aux grands, ni en aucun fils de l'homme qui ne peuvent délivrer.
\VS{4}Son esprit s'en va et l'homme retourne dans sa terre, et ce même jour ses desseins périssent.
\VS{5}Heureux celui qui a pour secours le Dieu de Jacob, qui met son espoir en Yahweh, son Dieu !
\VS{6}Il a fait les cieux et la terre, la mer et tout ce qui s'y trouve. Il garde la vérité à toujours !
\VS{7}Il fait droit aux opprimés, il donne du pain aux affamés ; Yahweh délie ceux qui sont liés\FTNT{Jn. 11:43-44.}.
\VS{8}Yahweh ouvre les yeux des aveugles\FTNT{Les miracles de Jésus-Christ confirment sa divinité (Es. 35:4-6 ; Lu. 7:19-23).} ; Yahweh redresse ceux qui sont courbés\FTNT{Lu. 13:11-13.} ; Yahweh aime les justes.
\VS{9}Yahweh protège les étrangers, il soutient l'orphelin et la veuve, mais il renverse la voie des méchants.
\VS{10}Yahweh règne éternellement. Ô Sion ! Ton Dieu subsiste d'âge en âge. Louez Yahweh !
\Chap{147}
\TextTitle{Yahweh aime ceux qui le craignent et qui s'attendent à sa bonté}
\VerseOne{}Louez Yahweh ! Car il est beau de chanter à notre Dieu ! Car il est doux et bienséant de le louer !
\VS{2}Yahweh est celui qui bâtit Jérusalem ; il rassemblera ceux d'Israël qui sont dispersés çà et là.
\VS{3}Il guérit ceux qui ont le cœur brisé et il bande leurs plaies\FTNT{Ex. 15:26 ; De. 32:39 ; Job. 5:18.}.
\VS{4}Il compte le nombre des étoiles, il les appelle toutes par leur nom.
\VS{5}Notre Seigneur est grand, puissant par sa force, son intelligence n'a point de limites.
\VS{6}Yahweh soutient les malheureux, mais il abaisse les méchants jusqu'à terre.
\VS{7}Chantez à Yahweh avec reconnaissance ! Célébrez notre Dieu avec la harpe !
\VS{8}Il couvre les cieux de nuées, il prépare la pluie pour la terre ; il fait germer l'herbe sur les montagnes.
\VS{9}Il donne la nourriture au bétail et aux petits du corbeau qui crient.
\VS{10}Il ne prend point plaisir dans la force du cheval ; il ne fait point cas des jambes de l'homme.
\VS{11}Yahweh aime ceux qui le craignent, ceux qui s'attendent à sa bonté.
\VS{12}Jérusalem, loue Yahweh ! Sion, loue ton Dieu !
\VS{13}Car il a affermi les barres de tes portes, il a béni tes fils au milieu de toi.
\VS{14}Il rend la paix à son territoire et te rassasie du meilleur froment.
\VS{15}C'est lui qui envoie ses ordres sur la terre, sa parole court avec rapidité\FTNT{Es. 55:10-11.}.
\VS{16}C'est lui qui donne la neige comme des flocons de laine et qui répand la gelée blanche comme de la cendre.
\VS{17}C'est lui qui lance sa glace comme par morceaux, qui peut résister devant son froid ?
\VS{18}Il envoie sa parole, et il les fond ; il fait souffler son vent, et les eaux coulent\FTNT{Ps. 135:7.}.
\VS{19}Il déclare ses paroles à Jacob, ses statuts et ses ordonnances à Israël\FTNT{Ps. 78:5.}.
\VS{20}Il n'a pas agi de même pour toutes les nations, c'est pourquoi elles ne connaissent point ses ordonnances. Louez Yahweh !
\Chap{148}
\TextTitle{La création loue son Dieu}
\VerseOne{}Louez Yahweh ! Louez des cieux Yahweh ! Louez-le dans les lieux élevés !
\VS{2}Louez-le, vous tous anges ! Louez-le, vous toutes ses armées !
\VS{3}Louez-le, vous, soleil et lune ! Louez-le, vous toutes, étoiles lumineuses !
\VS{4}Louez-le, vous, cieux des cieux ! Et vous, eaux qui êtes au-dessus des cieux !
\VS{5}Qu'ils louent le Nom de Yahweh ! Car il a commandé et ils ont été créés\FTNT{Ge. 1:3-6 ; Jé. 31:35.}.
\VS{6}Il les a établis à perpétuité et à toujours ; il a donné des lois, et il ne les violera pas\FTNT{Ps. 104:5 ; Ps. 119:91 ; Job. 14:5.}.
\VS{7}De la terre, louez Yahweh ! Louez-le, monstres marins et tous les abîmes !
\VS{8}Feu et grêle, neige et brouillard, vent impétueux qui exécutez ses ordres,
\VS{9}montagnes et toutes les collines, arbres fruitiers et tous les cèdres,
\VS{10}bêtes sauvages et tout le bétail, reptiles et oiseaux ailés,
\VS{11}rois de la terre et tous les peuples, princes et tous les juges de la terre,
\VS{12}ceux qui sont à la fleur de leur âge, et les vierges aussi, les vieillards, et les jeunes gens !
\VS{13}Qu'ils louent le Nom de Yahweh ! Car son Nom seul est haut élevé ! Sa majesté est au-dessus de la terre et des cieux.
\VS{14}Il a relevé la force de son peuple, sujet de louange pour tous ses fidèles, pour les fils d'Israël, du peuple qui est près de lui. Louez Yahweh !
\Chap{149}
\TextTitle{Adorons Yahweh}
\VerseOne{}Louez Yahweh ! Chantez à Yahweh un cantique nouveau et louez-le dans l'assemblée de ses fidèles !
\VS{2}Qu'Israël se réjouisse en celui qui l'a fait ! Et que les fils de Sion soient dans l'allégresse à cause de leur Roi\FTNT{Ps. 100:3 ; Za. 9:9 ; Mt. 21:5.} !
\VS{3}Qu'ils louent son Nom avec des danses ! Qu'ils le chantent avec le tambourin et la harpe !
\VS{4}Car Yahweh prend plaisir à son peuple, il glorifie les pauvres en les délivrant.
\VS{5}Que les fidèles se réjouissent dans la gloire, qu'ils poussent des cris de joie sur leur couche.
\VS{6}Les louanges de Dieu sont dans leur bouche et les épées affilées à deux tranchants dans leur main,
\VS{7}pour se venger des nations, pour châtier les peuples,
\VS{8}pour lier leurs rois avec des chaînes, et les plus honorables parmi eux avec des ceps de fer,
\VS{9}pour exercer sur eux le jugement qui est écrit ! Cet honneur est pour tous ses fidèles. Louez Yahweh !
\Chap{150}
\TextTitle{Que tout ce qui respire loue Yahwe !}
\VerseOne{}Louez Yahweh ! Louez Dieu à cause de sa sainteté ! Louez-le dans l'étendue de toute sa puissance !
\VS{2}Louez-le pour ses hauts faits ! Louez-le selon la grandeur de sa magnificence !
\VS{3}Louez-le au son du shofar ! Louez-le avec le luth et la harpe !
\VS{4}Louez-le avec le tambour et avec des danses ! Louez-le avec des instruments à cordes et le chalumeau !
\VS{5}Louez-le avec les cymbales sonores ! Louez-le avec les cymbales de cri de joie !
\VS{6}Que tout ce qui respire loue Yahweh ! Louez Yahweh !
\PPE{}
\end{multicols}

%\clearpage\ShortTitle{Proverbes}\BookTitle{Proverbes}\BFont
\noindent\hrulefill
{\footnotesize
\textit{
\bigskip
{\centering{}
\\Auteurs : Salomon, Agur et Lemuel
\\(Heb. : Mishlei)
\\Signification : Paraboles
\\Thème : La sagesse
\\Date de rédaction : 10\up{ème} siècle av J.-C.\\}
}
%\bigskip
\textit{
\\Le mot « proverbe » désigne un genre littéraire appliqué à une sentence, une énigme, une comparaison, un oracle, une
parabole ou une parole de sagesse. Le livre des proverbes est donc un recueil de sentences dont la majeure partie
est attribuée à Salomon. Véritable collection de maximes morales et spirituelles, la sagesse, la crainte de Dieu et la
tempérance en sont les thèmes principaux.
%\bigskip
\\Ce livre met en évidence l'opposition entre la voie du méchant et celle du juste, entre la femme étrangère et la femme
vertueuse, entre l'orgueil et l'humilité, entre la sagesse et la folie, entre le chemin de la vie et celui de la mort. Comme il était coutume au Moyen-Orient, ces écrits s'adressaient particulièrement aux jeunes gens en vue de leur instruction.\bigskip
}
}
\par\nobreak\noindent\hrulefill
\begin{multicols}{2}
\Chap{1}
\TextTitle{[But du livre : connaître la sagesse}
\VerseOne{}Les Proverbes de Salomon, fils de David et roi d'Israël.
\VS{2}Pour connaître la sagesse et l'instruction, pour discerner les paroles d'intelligence ;
\VS{3}pour recevoir une leçon de bon sens, de justice, de jugement et d'équité.
\VS{4}Pour donner du discernement aux simples, aux jeunes gens de la connaissance et de la réflexion.
\VS{5}Le sage écoutera, et il augmentera son savoir, et l'homme intelligent acquerra de la prudence ;
\VS{6}afin d'entendre les paraboles et les énigmes ; les discours des sages et leurs énigmes.
\TextTitle{Le fondement de la sagesse : la crainte de Dieu}
\VS{7}La crainte de Yahweh\FTNT{Pr. 8:13.} est la principale de la science ; mais les fous méprisent la sagesse et l'instruction.
\VS{8}Mon fils, écoute l'instruction de ton père, et n'abandonne pas l'enseignement\FTNT{Loi vient de Torah (instruction, enseignement, direction etc.).} de ta mère.
\VS{9}Car se sont des grâces enfilées ensemble autour de ta tête, et des colliers autour de ton cou.
\VS{10}Mon fils, si les pécheurs veulent t'attirer, ne t'y accorde pas.
\VS{11}S'ils disent : Viens avec nous, dressons des embûches pour tuer; épions secrètement l'innocent, quoiqu'il ne nous en ai point donné de sujet aucune raison.
\VS{12}Engloutissons-les tout vifs, comme le scheol ; et tout entiers, comme ceux qui descendent dans la fosse ;
\VS{13}nous trouverons toutes sortes de biens précieux, nous remplirons nos maisons de butin ;
\VS{14}tu auras ta part avec nous, il n'y aura qu'une bourse pour nous tous.
\VS{15}Mon fils, ne te mets point en chemin avec eux ; retires ton pied de leur sentier ;
\VS{16}parce que leurs pieds courent au mal, et se hâtent pour répandre le sang\FTNT{Es. 59:7.}.
\VS{17}Car c'est en vain qu'on jette le filet devant les yeux de tout Baal ailé\FTNT{Ec. 10:20.} ;
\VS{18}ainsi ceux-ci dressent des embûches contre le sang de ceux-là et épient secrètement leurs vies.
\VS{19}Tel est le train de tout homme convoiteux de gain déshonnête, qui ôte la vie à ceux qui y sont adonnés.
\TextTitle{La sagesse crie}
\VS{20}La souveraine sagesse crie hautement au-dehors, elle fait retentir sa voix dans les rues.
\VS{21}Elle crie dans les carrefours, là où on fait le plus de bruit, aux entrées des portes, elle prononce ses paroles dans la ville :
\VS{22}Stupides, dit-elle, jusqu'à quand aimerez-vous la stupidité ? Et jusqu'à quand les moqueurs prendront-ils plaisir à la moquerie, et les haïront-ils la connaissance ?
\VS{23}Etant repris par moi, convertissez-vous ; voici, je vous donnerai de mon Esprit en abondance, et je vous ferai connaître mes paroles.
\VS{24}Parce que je crie, et que vous refusez d'entendre ; parce que j'étends ma main, et que personne n'y prend garde ;
\VS{25}et parce que vous rejetez tout mon conseil, et que vous n'avez accepté je vous reprenne ;
\VS{26}moi aussi je rirai quand vous serez dans le malheur, je me moquerai quand la terreur viendra sur vous.
\VS{27}Quand votre effroi surviendra comme une ruine, et que votre calamité viendra comme un tourbillon; vous enveloppera comme un tourbillon ; quand la détresse et l'angoisse viendront  sur vous ;
\VS{28}alors on criera vers moi, mais je ne répondrai point ; on me cherchera de grand matin, mais on ne me trouvera pas\FTNT{De. 31:18 ; Job. 35:12.}.
\VS{29}Parce qu'ils auront haï la connaissance, et qu'ils n'auront point choisi la crainte de Yahweh.
\VS{30}Ils n'ont point aimé mon conseil ; ils ont rejeté toutes mes réprimandes.
\VS{31}Qu'ils mangent donc le fruit de leur voie, et qu'ils se rassasient de leurs conseils.
\VS{32}Car l'égarement des sots les tue, et la prospérité des insensés les perd.
\VS{33}Mais celui qui m'écoute habitera en sécurité et sera tranquille, sans être effrayé d'aucun mal.
\Chap{2}
\TextTitle{La sagesse nous libère du mal}
\VerseOne{}Mon fils, si tu reçois mes paroles, et que tu gardes précieusement en toi mes commandements,
\VS{2}si tu rends ton oreille attentive à la sagesse, et que tu inclines ton cœur à l'intelligence ;
\VS{3}si tu appelles à toi la sagesse, et que tu adresses ta voix à l'intelligence,
\VS{4}si tu la cherches comme de l'argent, et si tu la recherches soigneusement comme des trésors,
\VS{5}alors tu connaîtras la crainte de Yahweh, et tu trouveras la connaissance de Dieu.
\VS{6}Car Yahweh donne la sagesse, et de sa bouche procède la connaissance et l'intelligence.
\VS{7}Il réserve le salut pour ceux qui sont droits, et il est le bouclier de ceux qui marchent dans l'intégrité,
\VS{8}pour garder les sentiers de la justice ; il gardera la voie de ses bien-aimés.
\VS{9}Alors tu comprendras la justice, le jugement, l'équité, et tout bon chemin.
\VS{10}Si la sagesse vient dans ton coeur, et si la connaissance est agréable à ton âme ;
\VS{11}la réflexion veillera sur toi, et l'intelligence te gardera,
\VS{12}pour te délivrer du mauvais chemin, et de l'homme qui tient de mauvais discours ,
\VS{13}de ceux qui abandonnent les voies de la droiture pour marcher dans les chemins ténébreux,
\VS{14}qui sont joyeux de mal faire, et qui se réjouissent dans la perversité des méchants.
\VS{15}Eux dont les sentiers sont tortueux, et qui dans leur conduite vont de travers.
\VS{16}Afin qu'il te délivre de la femme étrangère\FTNT{La femme étrangère est la prostituée ou l'esprit de Jézabel qui séduit les hommes. Voir Pr. 6:24 ; Pr. 7:5.}, et de la femme d'autrui, dont les paroles sont flatteuses ;
\VS{17}qui abandonne l'ami de sa jeunesse et qui oublie l'alliance de son Dieu.
\VS{18}Car sa maison penche vers la mort, et son chemin mène vers les morts.
\VS{19}Pas un de ceux qui vont vers elle n'en retourne, ni ne reprend les sentiers de la vie.
\VS{20}Ainsi tu marcheras dans la voie des gens de bien, et tu garderas les sentiers des justes.
\VS{21}Car ceux qui sont droits habiteront la terre, les hommes intègres y demeureront.
\VS{22}Mais les méchants seront retranchés de la terre, et ceux qui agissent perfidement seront arrachés.
\Chap{3}
\TextTitle{La sagesse bénit et protège}
\VerseOne{}Mon fils, ne mets pas en oubli mon enseignement, et que ton cœur garde mes commandements.
\VS{2}Car ils t'apportent de longs jours et des années de vie et de paix.
\VS{3}Que la bonté et la vérité ne t'abandonnent pas : Lie-les à ton cou, et écris-les sur la table de ton coeur ;
\VS{4}et tu trouveras la grâce et la prudence au yeux de Dieu et des hommes.
\VS{5}Confie-toi de tout ton coeur en Yahweh et ne t'appuie point sur ton intelligence.
\VS{6}Considère-le dans toutes tes voies et il dirigera tes sentiers.
\VS{7}Ne sois point sage à tes yeux ; crains Yahweh, et détourne-toi du mal.
\VS{8}Ce sera la guérison de ton nombril et un rafraîchissement pour tes os.
\VS{9}Honore Yahweh avec tes biens et les prémices de tout ton revenu\FTNT{De. 12:6.} :
\VS{10}Alors tes greniers seront remplis d'abondance, et tes cuves regorgeront de vin nouveau.
\VS{11}Mon fils, ne rebute pas l'instruction de Yahweh, et ne te fâche pas de ce qu'il te reprend.
\VS{12}Car Yahweh châtie\FTNT{Hé. 12:4-11.} celui qu'il aime, comme un père le fils auquel il prend plaisir.
\VS{13}Heureux l'homme qui a trouvé la sagesse, et l'homme qui possède l'intelligence !
\VS{14}Car le trafic qu'on peut faire d'elle est meilleur que le trafic l'argent; et le profit qu'on en tire est meilleur que l'or fin.
\VS{15}Elle est plus précieuse que les perles, et toutes tes choses désirables ne la valent point.
\VS{16}Il y a de longs jours dans sa main droite, des richesses et de la gloire en sa gauche.
\VS{17}Ses voies sont des voies agréables, et tous ses sentiers ne sont que paix.
\VS{18}Elle est l'arbre de vie pour ceux qui l'embrassent ; et tous ceux qui la tiennent sont heureux\FTNT{Ge. 2:9 ; Ap. 22:2.}.
\VS{19}Yahweh a fondé la terre par la sagesse, il a disposé les cieux par l'intelligence.
\VS{20}C'est par sa science que les abîmes se sont ouverts, et que les nuages distillent la rosée.
\VS{21}Mon fils, que ces enseignements ne s'écartent point de devant tes yeux ; garde la sagesse et la réflexion :
\VS{22}Elles seront la vie de ton âme et l'ornement de ton cou.
\VS{23}Alors tu marcheras avec assurance dans ton chemin, et ton pied ne bronchera pas.
\VS{24}Si tu te couches, tu seras sans crainte, et quand tu seras couché ton sommeil sera doux.
\VS{25}Ne crains ni une terreur soudaine, ni la ruine des méchants, quand elle arrivera.
\VS{26}Car Yahweh sera ton assurance, et il gardera ton pied de toute embûche.
\VS{27}Ne retiens pas le bien à ceux à qui il est dû, quand il est au pouvoir de ta main de le faire\FTNT{Ga. 6:10.}.
\VS{28}Ne dis pas à ton prochain : Va, et reviens, demain je te donnerai ! Quand tu as de quoi donner.
\VS{29}Ne médite pas le mal contre ton prochain, lorsqu'il demeure tranquillement près de toi.
\VS{30}Ne conteste pas sans motif avec quelqu'un, à moins qu'il ne t'ait causé quelque tort\FTNT{Ro. 12:18.}.
\VS{31}Ne porte pas envie à l'homme violent, et ne choisis aucune de ses voies.
\VS{32}Car celui qui va de travers est en abomination à Yahweh ; mais son intimité est pour ceux qui sont justes.
\VS{33}La malédiction de Yahweh est dans la maison du méchant ; mais il bénit la demeure des justes.
\VS{34}Certes il se moque des moqueurs, mais il fait grâce à ceux qui s'humilient.
\VS{35}Les sages hériteront la gloire ; mais la honte élève des insensés.
\Chap{4}
\TextTitle{Instructions et conseils d'un père}
\VerseOne{}Ecoutez, mes fils, l'instruction du père, et soyez attentifs pour connaître l'intelligence.
\VS{2}Car je vous donne une bonne doctrine, ne rejetez donc pas mon enseignement.
\VS{3}J'ai été un fils pour mon père. Un fils tendre et unique auprès de ma mère.
\VS{4}Il m'a enseigné, et m'a dit : Que ton coeur retienne mes paroles ; garde mes commandements et tu vivras.
\VS{5}Acquiers la sagesse, acquiers l'intelligence ; n'oublie pas les paroles de ma bouche, et ne t'en détourne pas.
\VS{6}Ne l'abandonne point, et elle te gardera ; aime-la, et elle te protégera.
\VS{7}La principale chose c'est la sagesse ; donc acquiers la sagesse ; et sur toutes tes acuisitions, acquiers la prudence.
\VS{8}Exalte-la, et elle t'élèvera ; elle te glorifiera quand tu l'auras embrassée.
\VS{9}Elle posera sur ta tête une couronne de grâce, et elle t'ornera d'un magnifique diadème.
\VS{10}Ecoute, mon fils, et reçois mes paroles, ainsi les années de ta vie te seront multipliées.
\VS{11}Je t'ai enseigné le chemin de la sagesse, et je t'ai conduit dans les sentiers de la droiture\FTNT{Ps. 23:3.}.
\VS{12}Quand tu y marcheras, ton pas ne sera pas gêné ; et si tu cours, tu ne chancelleras pas\FTNT{Ps. 121:3.}.
\VS{13}Embrasse l'instruction, ne la lâche pas ; garde-la ; car elle est ta vie.
\VS{14}N'entre pas dans le sentier des méchants, et ne marche pas dans la voie des hommes mauvais.
\VS{15}Détourne-t'en, ne passe pas par là, détourne-t'en, et passe outre.
\VS{16}Car ils ne dormiraient pas, s'ils n'avaient fait quelque mal, et le sommeil leur serait ôté, s'ils n'avaient fait tomber quelqu'un.
\VS{17}Parce qu'ils mangent le pain de méchanceté, et qu'ils boivent le vin de la violence.
\VS{18}Mais le sentier des justes est comme la lumière resplendissante, dont l'éclat augmente jusqu'à ce que le jour soit dans sa perfection.
\VS{19}La voie des méchants est comme l'obscurité ; ils n'aperçoivent pas ce qui les fera tomber.
\VS{20}Mon fils, sois attentif à mes paroles, incline ton oreille à mes discours.
\VS{21}Qu'ils ne s'écarte pas de tes yeux ; garde-les dans le fond de ton coeur.
\VS{22}Car ils sont la vie pour ceux qui les trouvent, et la santé de tout le corps de chacun d'eux.
\VS{23}Garde ton coeur de tout ce dont il faut se garder ; car de lui procèdent les sources de la vie\FTNT{Mt 12:35 ; Mt. 15:18-19.}.
\VS{24}Eloigne de toi la perversité de la bouche et la dépravation des lèvres.
\VS{25}Que tes yeux regardent droit et que tes paupières dirigent ton chemin devant toi.
\VS{26}Pèse le chemin de tes pieds, et que toutes tes voies soient bien stables.
\VS{27}Ne te tourne ni à droite ni à gauche ; détourne ton pied du mal.
\Chap{5}
\TextTitle{[Se garder de l'immoralité]}
\VerseOne{}Mon fils, sois attentif à ma sagesse, incline ton oreille à mon intelligence ;
\VS{2}afin que tu gardes mes avis, et que tes lèvres conservent la connaissance.
\VS{3}Car les lèvres de l'étrangère distillent des rayons de miel, et son palais est plus doux que l'huile.
\VS{4}Mais ce qui en provient est amère comme de l'absinthe, et aigu comme une épée à deux tranchants.
\VS{5}Ses pieds descendent à la mort, ses pas atteignent le scheol.
\VS{6}Afin que tu ne balance pas sur le chemin de la vie car, ses chemins en sont écartés; tu ne le connaîtras pas.
\VS{7}Maintenant donc, fils, écoutez-moi, et ne vous détournez pas des paroles de ma bouche.
\VS{8}Eloigne ton chemin de la femme étrangère et n'approche pas de l'entrée de sa maison.
\VS{9}De peur que tu ne donnes ton honneur à d'autres, et tes années à un homme cruel.
\VS{10}De peur que les étrangers ne se rassasient de tes biens, et que le fruit de ton travail ne soit dans la maison d'un étranger.
\VS{11}De peur que tu ne gémisses quand tu seras près de ta fin, quand ta chair et ton corps seront consumés ;
\VS{12}et que tu ne dises : Comment donc ai-je pu haïr la correction, et comment mon coeur a-t-il dédaigné les réprimandes ?
\VS{13}Et comment n'ai-je point obéi à la voix de ceux qui m'instruisaient, et n'ai-je point incliné mon oreille à ceux qui m'enseignaient ?
\VS{14}Peu s'en est fallu que je n'aie été dans toute sorte de mal, au milieu du peuple et de l'assemblée.
\VS{15}Bois des eaux de ta citerne et de ce qui coule du milieu de ton puits ;
\VS{16}que tes sources se répandent dehors, et les ruisseaux d'eau sur les rues ;
\VS{17}qu'elles soient à toi seul, et non aux étrangers avec toi.
\VS{18}Que ta source soit bénie, et réjouis-toi de la femme de ta jeunesse,
\VS{19}comme d'une biche des amours, et d'une chevrette gracieuse ; que ses mamelles te rassasient en tout temps, et sois continuellement épris de son amour.
\VS{20}Et pourquoi, mon fils, irais-tu errant après l'étrangère et embrasserais-tu le sein de l'inconnue ?
\VS{21}Vu que les voies de l'homme sont devant les yeux de Yahweh et qu'il pèse toutes ses voies\FTNT{Jé 16:17 ; Hé. 4:13.}.
\VS{22}Les iniquités du méchant l'attraperont, et il sera retenu par les cordes de son péché.
\VS{23}Il mourra faute d'instruction et il s'égarera par l'excès de sa folie.
\Chap{6}
\TextTitle{Recommandations diverses}
\VerseOne{}Mon fils, si tu t'es porté caution pour ton prochain, si tu as engagé ta main pour un étranger,
\VS{2}tu es enlacé par les paroles de ta bouche, tu es pris par les paroles de ta bouche.
\VS{3}Mon fils, fais maintenant ceci, et dégage-toi, puisque tu es tombé entre les mains de ton intime ami, va, prosterne-toi, et importune tes amis.
\VS{4}Ne donne point de sommeil à tes yeux et ne laisse point sommeiller tes paupières.
\VS{5}Dégage-toi comme la gazelle de la main du chasseur, et comme l'oiseau de la main de l'oiseleur.
\VS{6}Va, paresseux, vers la fourmi, regarde ses voies, et sois sage.
\VS{7}Elle n'a ni chef, ni directeur, ni gouverneur,
\VS{8}et cependant elle prépare en été son pain, et amasse durant la moisson de quoi manger.
\VS{9}Paresseux, jusqu'à quand resteras-tu couché ? Quand te lèveras-tu de ton sommeil ?
\VS{10}Un peu de sommeil, dis-tu, un peu d'assoupissement, un peu croiser les mains afin de rester couché ;
\VS{11}et ta pauvreté viendra comme un voyageur, et ta disette comme un soldat.
\VS{12}Celui qui marche, la fausseté dans sa bouche, est un homme de Bélial\FTNT{1 S. 2:12.}, un homme inique.
\VS{13}Il cligne des yeux, parle du pied, enseigne de ses doigts.
\VS{14}Il y a la perversité dans son cœur, il machine du mal en tout temps, il fait naître des querelles.
\VS{15}C'est pourquoi sa calamité viendra subitement, il sera subitement brisé, il n'y aura point de guérison.
\VS{16}Il y a six choses que Yahweh hait, et il y en a sept qui sont en abomination à son âme ;
\VS{17}savoir, les yeux hautains\FTNT{Ps. 101:5.}, la langue mensongère\FTNT{Ps. 120:2-3.}, les mains qui répandent le sang innocent\FTNT{Es. 1:15.},
\VS{18}le coeur qui médite des projets iniques\FTNT{Ps. 36:5.}, les pieds qui se hâtent de courir au mal\FTNT{Es. 59:7.},
\VS{19}le faux témoin qui profère des mensonges\FTNT{Ps. 27:12.}, et celui qui sème des querelles entre les frères\FTNT{Jud. 1:16-19.}.
\VS{20}Mon fils, garde le commandement de ton père, et n'abandonne pas l'enseignement de ta mère ;
\VS{21}attache-les continuellement à ton coeur, lie-les à ton cou.
\VS{22}Quand tu marcheras, il te conduira ; et quand tu te coucheras, il te gardera ; et quand tu te réveilleras, il s'entretiendra avec toi.
\VS{23}Car le commandement est une lampe ; et l'enseignement une lumière\FTNT{Ps. 119:105.} ; et les réprimandes propres à instruire sont le chemin de la vie.
\VS{24}Ils te préserveront de la mauvaise femme, de la langue doucereuse de l'étrangère.
\VS{25}Ne la convoite pas dans ton coeur pour sa beauté, et ne te laisse pas prendre par ses yeux\FTNT{Mt. 5:28.}.
\VS{26}Car pour l'amour de la femme prostituée on est réduit à un morceau de pain, et la femme adultère chasse après l'âme précieuse de l'homme.
\VS{27}Un homme peut-il prendre du feu dans son sein, sans que ses habits brûlent ?
\VS{28}Un homme marchera-t-il sur des charbons ardents, sans que ses pieds en soient brûlés ?
\VS{29}Il en est de même pour celui qui va vers la femme de son prochain ; quiconque la touchera ne restera pas impuni.
\VS{30}On ne méprise pas un voleur, s'il vole pour satisfaire son âme quand il a faim ;
\VS{31}si on le trouve, il rendra sept fois autant, il donnera tout ce qu'il a dans sa maison.
\VS{32}Mais celui qui commet un adultère avec une femme est dépourvu de sens ; et celui qui le fera, détruira son âme.
\VS{33}Il trouvera des plaies et de l'ignominie, et son opprobre ne sera pas effacé.
\VS{34}Car la jalousie d'un mari est une fureur, il n'épargnera pas l'adultère au jour de la vengeance.
\VS{35}Il n'aura égard à aucune rançon, et il n'acceptera rien, quand même tu multiplierais les présents.
\Chap{7}
\TextTitle{Mise en garde contre la femme prostituée}
\VerseOne{}Mon fils, observe mes paroles, et garde avec toi mes commandements.
\VS{2}Garde mes commandements, et tu vivras, garde mes enseignements comme la prunelle de tes yeux\FTNT{Lé. 18:5.}.
\VS{3}Lie-les sur tes doigts, écris-les sur la table de ton coeur.
\VS{4}Dis à la sagesse : Tu es ma soeur ; et appelle l'intelligence, ton amie.
\VS{5}Afin qu'elles te préservent de la femme étrangère, de l'étrangère qui emploie des paroles doucereuses.
\VS{6}Comme je regardais de la fenêtre de ma maison à travers mon treillis,
\VS{7}je vis parmi les stupides, et je remarquai parmi les jeunes gens un jeune homme dépourvu de sens.
\VS{8}Il passait dans la rue, près de l'angle où se tenait une de ces femmes, et qui suivait le chemin de sa maison,
\VS{9}au crépuscule, au soir du jour, au milieu de la nuit et de l'obscurité.
\VS{10}Et voici, il fut abordé par une femme, vêtue en tenue de prostituée, et pleine de ruse dans le cœur.
\VS{11}Elle était bruyante et rebelle, ses pieds ne restaient point dans sa maison ;
\VS{12}tantôt dehors, tantôt sur les places, elle était aux aguets à chaque coin de rue.
\VS{13}Elle le saisit, et l'embrassa ; et avec un visage effronté, lui dit :
\VS{14}J'ai chez moi des sacrifices d'offrande de paix ; j'ai aujourd'hui accompli mes voeux.
\VS{15}C'est pourquoi je suis sortie à ta rencontre pour chercher ton visage, et je t'ai trouvé.
\VS{16}J'ai orné mon lit de couvertures, d'étoffes de fil d'Egypte.
\VS{17}J'ai parfumé ma couche de myrrhe, d'aloès et de cinnamome.
\VS{18}Viens, enivrons-nous de plaisir jusqu'au matin, réjouissons-nous en amours.
\VS{19}Car mon mari n'est point à la maison, il est parti pour un voyage lointain.
\VS{20}Il a pris un sac d'argent dans sa main, il ne reviendra à la maison qu'à la nouvelle lune.
\VS{21}Elle l'a fait détourner par beaucoup de douces paroles, et l'a attiré par la flatterie de ses lèvres.
\VS{22}Il s'en alla aussitôt après elle, comme un boeuf qui va à la boucherie, comme le fou qu'on lie pour être châtié ;
\VS{23}jusqu'à ce que la flèche lui ait transpercé le foie ; comme l'oiseau qui se hâte vers le filet, sans savoir que c'est au prix de sa vie.
\VS{24}Maintenant donc, fils, écoutez-moi, et soyez attentifs aux paroles de ma bouche.
\VS{25}Que ton coeur ne se détourne pas vers les voies d'une telle femme, ne t' égare pas dans ses sentiers.
\VS{26}Car elle a fait tomber plusieurs blessés à mort, et tous ceux qu'elle a tués sont nombreux.
\VS{27}Sa maison est le chemin du scheol, qui descend vers les demeures de la mort.
\Chap{8}
\TextTitle{[La sagesse préférable aux richesses]}
\VerseOne{}La sagesse ne crie-t-elle pas ? Et l'intelligence ne fait-elle pas entendre sa voix ?
\VS{2}Elle s'est présentée sur le sommet des lieux élevés, sur le chemin, aux carrefours.
\VS{3}Elle crie près des portes, devant la ville, à l'entrée des portes,
\VS{4}ô vous ! Hommes de qualité, je vous appelle ; et ma voix s'adresse aussi aux fils des hommes.
\VS{5}Vous stupides, apprenez le discernement, et vous tous, devenez intelligents de coeur.
\VS{6}Écoutez, car je dirai des choses importantes : Et j'ouvrirai mes lèvres pour enseigner des choses droites.
\VS{7}Parce que ma bouche proclame la vérité, et mes lèvres ont en horreur le mensonge.
\VS{8}Tous les discours de ma bouche sont selon la justice, il n'y a rien en eux de faux, ni de déformé.
\VS{9}Ils sont tous clairs à l'homme intelligent, et droits pour ceux qui ont trouvé la connaissance.
\VS{10}Recevez mon instruction plutôt que de l'argent, la connaissance à l'or le plus précieux.
\VS{11}Car la sagesse vaut mieux que les perles, et tout ce qu'on pourrait souhaiter ne la vaut pas\FTNT{Ps. 19:11 ; Ps. 119:127 ; Job. 28:18.}.
\VS{12}Moi, la Sagesse, j'habite avec le discernement, et je possède la connaissance de la réflexion.
\VS{13}La crainte de Yahweh c'est la haine du mal. Je hais l'orgueil et l'arrogance, la voie du mal, et la bouche perverse.
\VS{14}A moi appartiennent le conseil et le succès ; je suis l'intelligence, à moi appartient la force.
\VS{15}Par moi règnent les rois, et par moi les princes décrètent ce qui est juste.
\VS{16}Par moi gouvernent les seigneurs, les princes, et tous les juges de la terre.
\VS{17}J'aime ceux qui m'aiment ; et ceux qui me cherchent soigneusement me trouveront\FTNT{Mt. 7:7 ; Lu. 11:9 ; Jn 14:23-24.}.
\VS{18}Avec moi sont la richesse et la gloire, les biens durables et la justice.
\VS{19}Mon fruit est meilleur que le fin or, même que l'or raffiné ; et mon revenu est meilleur que l'argent choisi.
\VS{20}Je marche dans le chemin de la justice, au milieu des sentiers de la droiture ;
\VS{21}pour donner des biens en héritage à ceux qui m'aiment, et pour remplir leurs trésors.
\VS{22}Yahweh m'a acquise dès le commencement de ses voies, avant ses œuvres les plus anciennes.
\VS{23}J'ai été déclarée princesse depuis l'éternité, dès le commencement, avant l'origine de la terre.
\VS{24}J'ai été engendrée lorsqu'il n'y avait point encore d'abîmes, ni de sources chargées d'eaux.
\VS{25}Avant que les montagnes soient affermies, avant que les collines existent, j'ai été engendrée.
\VS{26}Lorsqu'il n'avait pas encore fait la terre et les campagnes, et le commencement de la poussière du monde habitable.
\VS{27}Lorsqu'il disposa les cieux, j'étais là ; lorsqu'il traça un cercle à la surface de l'abîme ;
\VS{28}lorsqu'il fixa les nuages en haut ; et que les sources de l'abîme jaillirent avec force ;
\VS{29}lorsqu'il donna une limite à la mer, pour que les eaux ne franchissent pas les bords ; lorsqu'il posa les fondements de la terre,
\VS{30}j'étais à l'œuvre auprès de lui, je faisais ses délices tous les jours, et toujours j'étais en joie en sa présence.
\VS{31}Je me réjouissais dans la partie habitable de sa terre, trouvant mes délices avec les fils de l'homme.
\VS{32}Maintenant donc, mes fils, écoutez-moi : Heureux sont ceux qui observent mes voies.
\VS{33}Ecoutez l'instruction, et soyez sages, et ne la rejetez point.
\VS{34}Ô ! Heureux est l'homme qui m'écoute, qui veille chaque jour à mes portes. Et qui monte la garde aux montants de mes portes !
\VS{35}Car celui qui me trouve a trouvé la vie, et obtient la faveur de Yahweh.
\VS{36}Mais celui qui pèche contre moi nuit à son âme ; tous ceux qui me haïssent aiment la mort.
\Chap{9}
\TextTitle{[La sagesse, source de vie]}
\VerseOne{}La Souveraine Sagesse a bâti sa maison, elle a taillé ses sept colonnes.
\VS{2}Elle a apprêté sa viande, elle a mêlé son vin ; elle a aussi dressé sa table.
\VS{3}Elle a envoyé ses servantes, elle crie du haut des lieux les plus élevés de la ville, disant : 
\VS{4}Que celui qui est stupide, entre ici ; et elle dit à ceux qui sont dépourvus de sens :
\VS{5}Venez, mangez de mon pain, et buvez du vin que j'ai mêlé.
\VS{6}Abandonnez la stupidité, et vous vivrez ; et marchez droit dans la voie de l'intelligence.
\VS{7}Celui qui instruit le moqueur, en reçoit de l'ignominie ; et celui qui reprend le méchant en reçoit une tache.
\VS{8}Ne reprends point le moqueur, de crainte qu'il ne te haïsse ; reprends le sage, et il t'aimera\FTNT{Ps. 141:5.}.
\VS{9}Donne l'instruction au sage, et il deviendra encore plus sage ; enseigne le juste, et il croîtra en science.
\VS{10}Le commencement de la sagesse est la crainte de Yahweh\FTNT{Ps. 19:10.} ; et la connaissance des saints, c'est l'intelligence.
\VS{11}Car tes jours se multiplieront par moi, et les années de vie augmenteront.
\VS{12}Si tu es sage, tu es sage pour toi-même ; si tu es moqueur, tu en porteras seul la peine.
\VS{13}La femme folle est bruyante, stupide et elle ne connaît rien.
\VS{14}Et elle s'assied à la porte de sa maison sur un siège, dans les lieux élevés de la ville ;
\VS{15}pour appeler les passants qui vont droit leur chemin, disant :
\VS{16}Que celui qui est stupide entre ici ! Elle dit à celui qui est dépourvu de sens :
\VS{17}Les eaux dérobées sont douces, et le pain pris en secret est agréable.
\VS{18}Et il ne sait pas que là sont les défunts, et que ceux qu'elle a conviés sont dans le scheol.
\Chap{10}
\TextTitle{[La justice s'oppose à la méchanceté]}
\VerseOne{}Proverbes de Salomon. Le fils sage réjouit son père, mais le fils insensé est l'ennui de sa mère.
\VS{2}Les trésors de méchanceté ne profitent pas, mais la justice délivre de la mort.
\VS{3}Yahweh ne laisse pas l'âme du juste avoir faim, mais il repousse au loin l'avidité des méchants.
\VS{4}Celui qui agit d'une main nonchalante s'appauvrit, mais la main des diligents enrichit.
\VS{5}L'enfant prudent amasse en été, mais celui qui dort durant la moisson est un enfant qui fait honte.
\VS{6}Les bénédictions seront sur la tête du juste, mais la violence couvrira la bouche des méchants.
\VS{7}La mémoire du juste est en bénédiction\FTNT{Ps 112:6.}, mais la réputation des méchants tombe en pourriture.
\VS{8}Celui qui est sage de coeur reçoit les commandements, mais celui qui est insensé des lèvres, tombera.
\VS{9}Celui qui marche dans l'intégrité marche avec assurance, mais celui qui pervertit ses voies, sera connu.
\VS{10}Celui qui cligne de l'oeil cause du chagrin, et celui qui a les lèvres insensées sera renversé.
\VS{11}La bouche du juste est une source de vie, mais la cruauté couvre la bouche des méchants.
\VS{12}La haine excite les querelles, mais la charité couvre toutes les fautes\FTNT{1 Pi 4:8.}.
\VS{13}La sagesse se trouve sur les lèvres de l'homme intelligent, mais la verge est pour le dos de celui qui est dépourvu de sens.
\VS{14}Les sages tiennent la connaissance en réserve, mais la bouche de l'insensé est une ruine prochaine.
\VS{15}Les biens du riche sont la ville de sa force, mais la pauvreté des misérables est leur ruine.
\VS{16}L'oeuvre du juste est pour la vie, mais le revenu du méchant est pour le péché.
\VS{17}Celui qui garde l'instruction est dans le chemin de la vie, mais celui qui néglige la correction s'y égare.
\VS{18}Celui qui dissimule la haine a des lèvres menteuses, et celui qui répand la calomnie est un insensé.
\VS{19}Dans la multitude de paroles le péché ne manque pas, mais celui qui retient ses lèvres est prudent.
\VS{20}La langue du juste est un argent de choix, mais le coeur des méchants est bien peu de chose.
\VS{21}Les lèvres du juste en instruisent plusieurs, mais les insensés mourront faute de sens.
\VS{22}La bénédiction de Yahweh est celle qui enrichit, et il n'y ajoute aucune peine.
\VS{23}C'est comme un jeu à un insensé de pratiquer l'infamie, mais la sagesse appartient à l'homme intelligent.
\VS{24}Ce que redoute le méchant, c'est ce qui lui arrive ; mais Dieu accorde aux justes ce qu'ils désirent.
\VS{25}Comme le tourbillon passe, ainsi le méchant n'est plus ; mais le juste est un fondement perpétuel.
\VS{26}Ce qu'est le vinaigre aux dents et la fumée aux yeux, tel est le paresseux à ceux qui l'envoient.
\VS{27}La crainte de Yahweh augmente les jours, mais les années des méchants sont raccourcies.
\VS{28}L'espérance des justes n'est que joie, mais l'espérance des méchants périra.
\VS{29}La voie de Yahweh est le refuge de l'homme intègre, mais elle est la ruine pour ceux qui pratiquent l'iniquité.
\VS{30}Le juste ne sera jamais ébranlé, mais les méchants ne demeureront pas sur la terre.
\VS{31}La bouche du juste produit la sagesse, mais la langue perverse sera retranchée.
\VS{32}Les lèvres du juste connaissent ce qui est agréable ; mais la bouche des méchants n'est que perversité.
\Chap{11}
\TextTitle{[La justice s'oppose à la méchanceté (suite)]}
\VerseOne{}La fausse balance est une abomination à Yahweh, mais le poids juste lui est agréable\FTNT{Lé 19:35-36 ; De. 25:13-16.}.
\VS{2}Quand l'orgueil vient, la honte vient aussi ; mais la sagesse est avec ceux qui sont modestes.
\VS{3}L'intégrité des hommes droits les conduit, mais la perversité des perfides les détruit.
\VS{4}Les richesses ne servent à rien au jour de la colère, mais la justice délivre de la mort.
\VS{5}La justice de l'homme intègre rend droite sa voie, mais le méchant tombe par sa méchanceté.
\VS{6}La justice des hommes droits les délivre, mais les perfides sont pris par leur méchanceté.
\VS{7}Quand l'homme méchant meurt, son espoir périt ; et l'espérance des hommes iniques périt.
\VS{8}Le juste est délivré de la détresse, et le méchant y entre à sa place.
\VS{9}Par sa bouche l'impie corrompt son prochain, mais les justes en sont délivrés par la connaissance.
\VS{10}La ville se réjouit quand les justes sont heureux, et quand les méchants périssent, c'est un triomphe.
\VS{11}La ville est élevée par la bénédiction des hommes droits, mais elle est renversée par la bouche des méchants.
\VS{12}Celui qui méprise son prochain est dépourvu de sens, mais l'homme prudent se tait.
\VS{13}Celui qui va rapportant, révèle les secrets, mais celui qui a l'esprit qui supporte les paroles, les couvre.
\VS{14}Le peuple tombe par faute de prudence, mais la délivrance est dans la multitude de conseillers.
\VS{15}Celui qui se porte garant pour un étranger en souffrira, et celui qui hait le cautionnement est assuré.
\VS{16}La femme gracieuse obtient de l'honneur, et les hommes robustes obtiennent les richesses.
\VS{17}L'homme doux fait du bien à son âme, mais le cruel trouble sa chair.
\VS{18}Le méchant fait une oeuvre qui le trompe, mais la récompense est assurée à celui qui sème la justice\FTNT{Os. 10:12.}.
\VS{19}Ainsi la justice conduit à la vie, mais celui qui poursuit le mal aboutit à sa mort.
\VS{20}Ceux qui ont le cœur pervers sont en abomination à Yahweh, mais ceux qui sont intègres dans leurs voies lui sont agréables.
\VS{21}De main en main le méchant ne demeurera point impuni, mais la race des justes sera délivrée.
\VS{22}Une belle femme qui se détourne de la raison est comme un anneau d'or au nez d'un pourceau.
\VS{23}Le souhait des justes n'est que le bien, mais l'attente des méchants c'est l'indignation.
\VS{24}Tel, qui donne libéralement, devient plus riche ; et tel qui épargne à l'excès ne fait que s'appauvrir.
\VS{25}Celui qui bénit sera engraisssé ; et celui qui arrose abondamment sera lui-même arrosé.
\VS{26}Sera maudit du peuple, celui qui cache le froment, mais la bénédiction est sur la tête de celui qui le vend.
\VS{27}Qui recherche le bien cherche la faveur, mais le mal arrive à qui le recherche.
\VS{28}Celui qui se confie dans ses richesses tombera, mais les justes verdiront comme le feuillage\FTNT{Ps. 1:3 ; Jé 17: 8.}.
\VS{29}Celui qui ne gouverne pas sa maison avec ordre, aura le vent pour héritage, et le fou sera le serviteur de celui qui a le coeur sage.
\VS{30}Le fruit du juste est un arbre de vie, et celui qui gagne les âmes est sage.
\VS{31}Voici, le juste reçoit sur la terre sa rétribution, combien plus le méchant et le pécheur la recevront-ils ?
\Chap{12}
\TextTitle{[La justice s'oppose à la méchanceté (suite)]}
\VerseOne{}Celui qui aime la correction aime la connaissance, mais celui qui hait la réprimande est un stupide.
\VS{2}L'homme de bien obtient la faveur de Yahweh, mais Yahweh condamne l'homme qui a des mauvaises pensées.
\VS{3}L'homme ne sera point affermi par la méchanceté, mais la racine des justes ne sera point ébranlée.
\VS{4}La femme vertueuse est la couronne de son mari\FTNT{Pr. 31:10.}, mais celle qui fait honte est comme la pourriture dans ses os.
\VS{5}Les pensées des justes ne sont que jugement, mais les conseils des méchants ne sont que fraude.
\VS{6}Les paroles des méchants ne tendent qu'à dresser des embûches pour répandre le sang, mais la bouche des hommes droits les délivrera.
\VS{7}Les méchants sont renversés, et ils ne sont plus, mais la maison des justes se maintiendra.
\VS{8}L'homme est estimé en raison de sa prudence, mais celui qui a le coeur pervers est l'objet du mépris.
\VS{9}Mieux vaut l'homme qui ne fait pas cas de lui-même, bien qu'il ait des serviteurs, que celui qui se glorifie, et qui manque de pain.
\VS{10}Le juste a égard à la vie de sa bête, mais les entrailles des méchants sont cruelles.
\VS{11}Celui qui cultive son champ sera rassasié de pain, mais celui qui court après des futilités est dépourvu de sens.
\VS{12}Ce que le méchant désire, est un filet des hommes mauvais, mais la racine des justes donnera son fruit.
\VS{13}Il y a dans le péché des lèvres un piège pernicieux, mais le juste sortira de la détresse.
\VS{14}L'homme sera rassasié de biens par le fruit de sa bouche, et on rendra à l'homme la rétribution de ses mains.
\VS{15}La voie de l'insensé est droite à son opinion, mais celui qui écoute le conseil est sage.
\VS{16}Quand à l'insensé, sa colère est révélée le jour même, mais l'homme bien avisé couvre son ignominie.
\VS{17}Celui qui prononce des choses véritables rend un témoignage juste, mais le faux témoin fait des rapports trompeurs.
\VS{18}Il y a tel homme dont les paroles blessent comme des pointes d'épée, mais la langue des sages apporte la guérison.
\VS{19}La lèvre véridique est affermie pour toujours, mais la fausse langue n'est que pour un moment\FTNT{Ps. 52: 6-7.}.
\VS{20}Il y a de la tromperie dans le coeur de ceux qui méditent le mal, mais il y a de la joie pour ceux qui conseillent la paix.
\VS{21}Il n'arrivera aucun outrage aux justes, mais les méchants seront remplis de mal.
\VS{22}Les fausses lèvres sont une abomination à Yahweh\FTNT{Ap. 22:15.}, mais ceux qui agissent fidèlement lui sont agréables.
\VS{23}L'homme bien avisé cache sa connaissance, mais le coeur des insensés publie la folie.
\VS{24}La main des diligents dominera, mais la main paresseuse sera tributaire.
\VS{25}Le chagrin qui est au cœur de l'homme, l'accable ; mais la bonne parole le réjouit.
\VS{26}Le juste a plus de reste que son voisin, mais la voie des méchants les égare.
\VS{27}L'homme paresseux ne rôtit point son gibier ; mais les biens précieux de l'homme sont au diligent.
\VS{28}La vie est dans le chemin de la justice, et la voie de son sentier ne tend point à la mort.
\Chap{13}
\TextTitle{[La justice s'oppose à la méchanceté (suite)]}
\VerseOne{}Un fils sage écoute l'instruction de son père, mais le moqueur n'écoute pas la réprimande\FTNT{Ps. 1:1.}.
\VS{2}L'homme mange du bien par le fruit de sa bouche, mais l'âme de ceux qui agissent perfidement mangent l'injustice.
\VS{3}Celui qui garde sa bouche, garde son âme ; mais celui qui ouvre à tout propos ses lèvres, tombera en ruine\FTNT{Ps. 39:2.}.
\VS{4}L'âme du paresseux a des désirs qu'il ne peut satisfaire, mais l'âme des diligents sera engraissée.
\VS{5}Le juste hait la parole mensongère, mais elle rend le méchant odieux et le fait tomber dans la confusion.
\VS{6}La justice garde celui qui est intègre dans sa voie, mais la méchanceté renversera celui qui s'égare.
\VS{7}Tel fait le riche et n'a rien du tout, tel fait le pauvre et a de grandes fortunes.
\VS{8}Les richesses d'un homme servent de rançon pour sa vie, mais le pauvre n'entend pas des réprimandes.
\VS{9}La lumière des justes remplit de joie, mais la lampe des méchants s'éteint.
\VS{10}L'orgueil ne produit que querelle, mais la sagesse est avec ceux qui écoutent les conseils.
\VS{11}Les richesses provenues de la fraude seront diminuées, mais celui qui amasse peu à peu les augmentera.
\VS{12}Un espoir différé fait languir le cœur, mais un désir accompli est comme un arbre de vie.
\VS{13}Celui qui méprise la parole périra à cause d'elle, mais celui qui craint le commandement en sera récompensé.
\VS{14}L'enseignement du sage est une source de vie, pour se détourner des pièges de la mort.
\VS{15}Le bon sens donne de la grâce ; mais la voie de ceux qui agissent perfidement est raboteuse.
\VS{16}Tout homme bien avisé agira avec connaissance, mais l'insensé fera l'étalage de sa folie\FTNT{Da.11:32}.
\VS{17}Le méchant messager tombe dans le mal, mais l'ambassadeur fidèle apporte la guérison.
\VS{18}La pauvreté et l'ignominie arrivent à celui qui rejette l'instruction, mais celui qui garde la réprimande est honoré.
\VS{19}Le souhait accompli est une chose douce à l'âme, mais se détourner du mal est une abomination aux insensés.
\VS{20}Celui qui marche avec les sages deviendra sage, mais le compagnon des insensés sera accablé.
\VS{21}Le mal poursuit les pécheurs, mais le bien sera rendu aux justes.
\VS{22}L'homme de bien laissera de quoi hériter aux fils de ses fils, mais les richesses du pécheur sont réservées aux justes.
\VS{23}Il y a beaucoup à manger dans les terres défrichées des pauvres, mais il y a tel qui est consumé faute de règles.
\VS{24}Celui qui épargne sa verge hait son fils, mais celui qui l'aime se hâte de le châtier.
\VS{25}Le juste mangera jusqu'à être rassasié à son souhait, mais le ventre des méchants aura la disette.
\Chap{14}
\TextTitle{[La justice s'oppose à la méchanceté (suite)]}
\VerseOne{}Toute femme sage bâtit sa maison, mais la folle la ruine de ses mains.
\VS{2}Celui qui marche dans la droiture craint Yahweh, mais celui dont les voies sont perverses le méprise.
\VS{3}La verge d'orgueil est dans la bouche de l'insensé, mais les lèvres des sages les garderont.
\VS{4}Où il n'y a point de boeuf, la grange est vide ; et l'abondance du revenu provient de la force du boeuf.
\VS{5}Le témoin véritable ne ment jamais, mais le faux témoin avance volontiers des mensonges.
\VS{6}Le moqueur cherche la sagesse et ne la trouve pas, mais la connaissance est aisée à trouver pour l'homme intelligent.
\VS{7}Eloigne-toi de l'homme insensé, puisque tu n'as pas trouvé sur ses lèvres la connaissance.
\VS{8}La sagesse d'un homme avisé est de connaître les règles de sa voie, mais la folie des insensés est la tromperie.
\VS{9}Les insensés se moquent du péché, mais parmi les hommes droits se trouve la bienveillance.
\VS{10}Le cœur d'un chacun connaît l'amertume de son âme, et un autre ne saurait partager sa joie.
\VS{11}La maison des méchants sera abolie, mais la tente des hommes droits fleurira.
\VS{12}Il y a telle voie qui semble droite à l'homme, mais dont l'issue sont les voies de la mort.
\VS{13}Même en riant le coeur sera triste, et la joie finit par l'ennui.
\VS{14}Celui qui a un cœur hypocrite, sera rassasié de ses voies ; mais l'homme de bien de ce qui est en lui.
\VS{15}Le simple croit à toute parole ; mais l'homme bien avisé considère ses pas.
\VS{16}Le sage craint et se retire du mal, mais l'insensé se met en colère et est confiant.
\VS{17}Celui qui est prompt à la colère agit follement\FTNT{Ps. 37:8.}, et l'homme plein de ruse est haï.
\VS{18}Les naïfs hériteront la folie ; mais les prudents seront couronnés de connaissance.
\VS{19}Les malins seront humiliés devant les bons, et les méchants, devant les portes du juste.
\VS{20}Le pauvre est haï même de son ami, mais les amis du riche sont en grand nombre.
\VS{21}Celui qui méprise son prochain commet un péché, mais celui qui a pitié des pauvres affligés est heureux.
\VS{22}Ceux qui méditent le mal ne s'égarent-ils pas ? Mais la bonté et la vérité sont pour ceux qui méditent le bien.
\VS{23}En tout travail il y a quelque profit, mais les vains discours ne tournent qu'à la disette.
\VS{24}Les richesses des sages leur sont comme une couronne, mais la stupidité des insensés est toujours stupidité.
\VS{25}Le témoin fidèle délivre les âmes, mais celui qui prononce des mensonges est trompeur.
\VS{26}En la crainte de Yahweh il y a une ferme assurance, et une retraite pour ses fils.
\VS{27}La crainte de Yahweh est une source de vie pour se détourner des pièges de la mort.
\VS{28}La gloire d'un roi, c'est la multitude du peuple, mais quand le peuple manque, c'est la ruine du prince.
\VS{29}Celui qui est lent à la colère a une grande intelligence, mais celui qui est prompt à s'emporter excite la folie.
\VS{30}Un coeur sain est la vie de la chair, mais l'envie est la pourriture des os.
\VS{31}Celui qui fait tort au pauvre déshonore celui qui l'a fait, mais celui qui a pitié de l'indigent honore Yahweh\FTNT{De. 24:11 ; Ps. 107:41.}.
\VS{32}Le méchant est chassé par sa malice, mais le juste trouve un refuge même dans sa mort.
\VS{33}La sagesse repose au coeur de l'homme intelligent, et elle est même reconnue au milieu des insensés.
\VS{34}La justice élève une nation, mais le péché est l'ignominie des peuples.
\VS{35}Le roi prend plaisir au serviteur prudent, mais son indignation sera contre celui qui lui fait honte.
\Chap{15}
\TextTitle{[La justice s'oppose à la méchanceté (suite)]}
\VerseOne{}La réponse douce apaise la fureur ; mais la parole douloureuse excite la colère
\VS{2}La langue des sages se réjouit de la connaissance, mais la bouche des insensés profère la sottise.
\VS{3}Les yeux de Yahweh sont en tous lieux, observant les méchants et les bons.
\VS{4}La langue qui corrige le prochain est comme l'arbre de vie, mais celle où il y a de la perversité est comme une brèche dans l'esprit.
\VS{5}L'insensé méprise l'instruction de son père, mais celui qui prend garde à la réprimande agit avec prudence.
\VS{6}Il y a un grand trésor dans la maison du juste, mais il y a du trouble dans les revenus du méchant.
\VS{7}Les lèvres des sages répandent partout la connaissance, mais le coeur des insensés ne fait pas ainsi.
\VS{8}Le sacrifice des méchants est en abomination à Yahweh, mais la requête des hommes droits lui est agréable.
\VS{9}La voie du méchant est en abomination à Yahweh, mais il aime celui qui poursuit soigneusement la justice.
\VS{10}Le châtiment est fâcheux à celui qui quitte le droit chemin, mais celui qui hait d'être repris, mourra.
\VS{11}Le schéol et le gouffre sont devant Yahweh ; combien plus les coeurs des fils des hommes !
\VS{12}Le moqueur n'aime pas qu'on le reprenne, et il ne va pas vers les sages.
\VS{13}Le cœur joyeux rend le visage beau, mais l'esprit est abattu par l'ennui du cœur.
\VS{14}Le cœur de l'homme prudent cherche la science ; mais la bouche des insensés se repaît de folie.
\VS{15}Tous les jours de l'affligé sont mauvais, mais quand on a le coeur gai, c'est un festin perpétuel.
\VS{16}Un peu de bien vaut mieux avec la crainte de Yahweh, qu'un grand trésor avec lequel il y a du trouble\FTNT{Ps. 37:16.}.
\VS{17}Mieux vaut un repas d'herbes où il y a de l'amitié, qu'un repas de boeuf bien gras où il y a de la haine.
\VS{18}L'homme furieux excite la querelle, mais l'homme lent à la colère apaise la dispute.
\VS{19}La voie du paresseux est comme une haie d'épines, mais le chemin des hommes droits est aplani.
\VS{20}Un fils sage réjouit le père, et un homme insensé méprise sa mère.
\VS{21}La stupidité est la joie de celui qui est dépourvu de sens, mais un homme prudent dresse ses pas au chemin de la droiture.
\VS{22} Les résolutions deviennent inutiles où il n'y a point de conseil ; mais il y a de la fermeté dans la multitude des conseillers.
\VS{23}L'homme a de la joie dans les réponses de sa bouche ; et combien est bonne une parole dite en son temps !
\VS{24}Le chemin de la vie élève l'homme prudent, afin qu'il se détourne du scheol qui est en bas.
\VS{25}Yahweh renverse la maison des orgueilleux, mais il affermit la borne de la veuve.
\VS{26}Les pensées du malin sont en abomination à Yahweh, mais celles de ceux qui sont purs sont des paroles agréables à ses yeux.
\VS{27}Celui qui est entièrement adonné au gain déshonnête trouble sa maison, mais celui qui hait les présents vivra.
\VS{28}Le coeur du juste médite ce qu'il doit répondre, mais la bouche des méchants profère des choses mauvaises.
\VS{29}Yahweh est loin des méchants, mais il exauce la requête des justes.
\VS{30}La clarté des yeux réjouit le coeur ; et la bonne renommée fortifie les os.
\VS{31}L'oreille qui écoute la correction qui donne la vie habite parmi les sages.
\VS{32}Celui qui rejette l'instruction a en dédain son âme, mais celui qui écoute la réprimande s'acquiert du sens.
\VS{33}La crainte de Yahweh enseigne la sagesse, et l'humilité précède la gloire\FTNT{Ps. 19:10.}.
\Chap{16}
\TextTitle{[La justice s'oppose à la méchanceté (suite)]}
\VerseOne{}Les préparations du cœur sont à l'homme, mais le discours réponse de la langue est de par Yahweh.
\VS{2}Chacune des voies de l'homme lui semble pure à ses yeux; mais Yahweh pèse les esprits.
\VS{3}Recommande tes affaires à Yahweh, et tes pensées seront bien ordonnées.
\VS{4}Yahweh a fait toutes choses pour lui-même ; et même le méchant pour le jour de l'affliction.
\VS{5}Yahweh a en abomination tout homme hautain de coeur ; assurément, il ne demeurera pas impuni.
\VS{6}Il y aura propitiation de l'iniquité par la miséricorde et la vérité ; on se détourne du mal par la crainte de Yahweh. 
\VS{7}Quand Yahweh prend plaisir aux voies d'un homme, il apaise\FTNT{Apaiser vient de shalom qui signifie : être dans une alliance de paix, être en paix, apaiser, vivre dans la paix etc.} envers lui même ses ennemis.
\VS{8}Il vaut mieux un peu de bien avec justice, qu'un gros revenu là où on n'a pas de droit.
\VS{9}Le cœur de l'homme médite sur sa voie, mais Yahweh conduit ses pas.
\VS{10}La divination est sur les lèvres du roi : Sa bouche ne doit pas s'égarer du droit.
\VS{11}La balance et le poids justes sont à Yahweh, tous les poids du sachet sont aussi son oeuvre.
\VS{12}Commettre une injustice doit être en abomination aux rois, parce que le trône est affermi par la justice.
\VS{13}Les rois doivent prendre plaisir aux lèvres de justice, et aimer celui qui profère des paroles justes.
\VS{14}Ce sont autant de messagers de mort que la colère du roi, mais l'homme sage l'apaisera.
\VS{15}Le visage serein du roi c'est la vie, et sa faveur est comme la nuée portant la pluie de la dernière saison.
\VS{16}Combien est-il plus précieux que l'or fin, d'acquérir de la sagesse! Et combien est-il plus excellent que l'argent, d'acquérir de la prudence ! 
\VS{17}Le chemin aplani des hommes droits, c'est de se détourner du mal ; celui qui prend garde de sa voie garde son âme.
\VS{18}L'orgueil va devant l'écrasement, et la fierté d'esprit devant la ruine.
\VS{19}Mieux vaut être humilié d'esprit avec les débonnaires, que de partager le butin avec les orgueilleux.
\VS{20}Celui qui prend garde à la parole trouvera le bien, et celui qui se confie en Yahweh est heureux\FTNT{Ps. 2:12.}.
\VS{21}On appellera prudent le sage de cœur, et la douceur des lèvres augmente l'instruction.
\VS{22}La prudence est à ceux qui la possèdent une source de vie ; mais le l'instruction des fous c'est leur folie.
\VS{23}Celui qui est sage de coeur conduit prudemment sa bouche, et ajoute l'instruction sur ses lèvres.
\VS{24}Les paroles agréables sont des rayons de miel, douces à l'âme et santé pour les os.
\VS{25} II y a telle voie qui semble droite à l'homme, mais dont la fin sont les voies de la mort.
\VS{26}Celui qui travaille, travaille pour lui-même, parce que sa bouche se courbe devant lui\FTNT{Ec. 6:7.}.
\VS{27}L'homme méchant creuse le mal, et il y a comme un feu brûlant sur ses lèvres.
\VS{28}L'homme qui use de perversité sème des querelles, et le rapporteur divise les grands amis.
\VS{29}L'homme violent attire son compagnon et le fait marcher dans une voie qui n'est pas bonne.
\VS{30}Il fait signe des yeux pour méditer des choses perverses, et remuant ses lèvres il exécute le mal.
\VS{31}Les cheveux blancs sont une couronne d'honneur ; elle se trouvera dans la voie de la justice.
\VS{32}Celui qui est lent à la colère vaut mieux que l'homme fort, et celui qui est maître de son cœur, vaut mieux que celui qui prend des villes.
\VS{33}On jette le sort dans le pan de la robe, mais tout ce qui doit arriver est de part Yahweh.
\Chap{17}
\TextTitle{[La justice s'oppose à la méchanceté (suite)]}
\VerseOne{}Mieux vaut un morceau de pain sec là où il y a la paix, qu'une maison pleine de viandes, là où il y a des querelles.
\VS{2}Le serviteur prudent sera maître sur l'enfant qui fait honte, et il partagera l'héritage entre les frères.
\VS{3}Le creuset est pour éprouver l'argent, et le fourneau l'or ; mais Yahweh éprouve les coeurs\FTNT{Jé. 17:10 ; Mal. 3:3 ; Ps. 26:2.}.
\VS{4}L'homme mauvais est attentif à la lèvre trompeuse, et le menteur écoute la mauvaise langue.
\VS{5}Celui qui se moque du pauvre déshonore celui qui l'a fait ; et celui qui se réjouit de l'affliction ne demeurera pas impuni.
\VS{6}Les petits-fils sont la couronne des vieillards\FTNT{Ps. 127:3 ; Ps. 128:3.}, et les pères sont la gloire de leurs fils.
\VS{7}La parole distinguée ne convient pas à un fou ; combien moins aux principaux du peuple des paroles de mensonge!
\VS{8}Le présent est comme une pierre précieuse aux yeux de ceux qui y sont adonnés ; de quelque côté qu'ils se tournent, ils réussissent.
\VS{9}Celui qui couvre les fautes cherche l'amitié, mais celui qui rapporte la chose divise les plus grands amis.
\VS{1}La répréhension se fait mieux sentir sur l'homme prudent que cent coups au fou.
\VS{11}Le méchant ne cherche que rébellion, mais le messager cruel sera envoyé contre lui.
\VS{12}Que l'homme rencontre plutôt une ourse qui a perdu ses petits qu'un fou dans sa folie.
\VS{13}Le mal ne partira point de la maison de celui qui rend le mal pour le bien.
\VS{14}Le commencement d'une querelle est comme quand on lâche une l'eau; mais avant qu'on en vienne à la dispute, retire-toi.
\VS{15}Celui qui déclare juste le méchant et celui qui déclare méchant le juste, sont tous deux en abomination à Yahweh\FTNT{Ex. 23:7 ; Es. 5:23.}.
\VS{16}A quoi sert le prix dans la main du fou pour acheter la sagesse, vu qu'il n'a pas de sens?
\VS{17}L'ami intime aime en tout temps, et il naît comme un frère dans la détresse.
\VS{18}Celui là est dépourvu de sens qui touche à la main et se rend caution pour son ami.
\VS{19}Celui qui aime les querelles aime le péché ; celui qui élève sa porte cherche sa ruine.
\VS{20}Celui qui est pervers de coeur ne trouve pas le bien; et l'hypocrite tombe dans le malheur.
\VS{21}Celui qui engendre un sot en aura de l'ennui, et le père du sot ne se réjouira pas.
\VS{22}Le coeur joyeux est un remède, mais l'esprit abattu dessèche les os.
\VS{23}Le méchant rend les présents en secret, pour pervertir les voies du jugement.
\VS{24}La sagesse est en présence de l'homme prudent; mais les yeux du fou sont à l'extrémité de la terre.
\VS{25}Le fils fou est l'ennui de son père, et l'amertume de celle qui l'a enfanté.
\VS{26}Il n'est pas bon de condamner l'innocent à l'amende, ni que les principaux frappent quelqu'un pur avoir agi avec droiture.
\VS{27}L'homme retenu dans ses paroles sait ce qu'est la connaissance, et l'homme qui est d'un esprit calme est un homme intelligent.
\VS{28}Même le fou, quand il se tait, est réputé sage ; et celui qui ferme ses lèvres est réputé intelligent.
\Chap{18}
\TextTitle{[La justice s'oppose à la méchanceté (suite)]}
\VerseOne{}Celui qui se sépare cherche ce qui lui fait plaisir, et se mêle de savoir comment tout doit aller.
\VS{2}Le fou ne prend pas plaisir à l'intelligence, mais à ce que son cœur soit manifesté.
\VS{3}Quand le méchant vient, le mépris vient aussi, et le reproche avec l'ignominie.
\VS{4}Les paroles de la bouche d'un homme sont des eaux profondes ; et la source de la sagesse est un torrent qui bouillonne\FTNT{Jn. 4:14.}.
\VS{5}Il n'est pas bon d'avoir égard à l'apparence de la personne du méchant, pour renverser le juste en jugement.
\VS{6}La bouche du fou entrent en querelles, et sa bouche appelle les combats.
\VS{7}La bouche du fou lui est une ruine, et ses lèvres sont un piège à son âme.
\VS{8}Les paroles du flatteur sont de ceux qui font semblant d'y toucher ; mais elles pénètrent jusqu'au-dedans des entrailles.
\VS{9}Celui qui se relâche dans son ouvrage est frère de celui qui dissipe ce qu'il a.
\VS{10}Le Nom de Yahweh est une tour forte, le juste y court et y trouve une haute retraite.
\VS{11}Les biens du riche sont sa ville forte et comme une haute muraille de retraite, selon son imagination.
\VS{12}Le coeur de l'homme s'élève avant que la ruine arrive, mais l'humilité précède la gloire.
\VS{13}Celui qui répond à quelque propos avant de l'avoir entendu, agit en fou et s'attire le reproche.
\VS{14}L'esprit d'un homme fort soutiendra dans son infirmité ; mais l'esprit abattu, qui le relèvera ?
\VS{15}Le coeur de l'homme intelligent acquiert la connaissance, et l'oreille des sages cherche la connaissance.
\VS{16}Le présent d'un homme lui fait faire place, et le conduit devant les grands.
\VS{17}Celui qui plaide le premier paraît juste; mais sa partie adverse vient, et examine le tout.
\VS{18}Le sort fait cesser les procès et fait les partages entre les puissants.
\VS{19}Un frère offensé se rend plus difficile qu'une ville forte, et les discordes entre frères sont comme les verrous d'un palais.
\VS{20}Le ventre de chacun est rassasié du fruit de sa bouche, il se rassasie du revenu de ses lèvres.
\VS{21}La mort et la vie sont au pouvoir de la langue\FTNT{Mt. 12:37.}, et celui qui aime à parler mangera de ses fruits.
\VS{22}Celui qui trouve une femme vertueuse trouve le bonheur et il obtient une faveur de Yahweh.
\VS{23}Le pauvre ne prononce que des supplications, mais le riche ne répond que des paroles dures.
\VS{24}L'homme qui a des intimes amis se tiennent à leur amitié parce qu'il y a tel ami qui est plus attaché que le frère.
\Chap{19}
\TextTitle{[La justice s'oppose à la méchanceté (suite)]}
\VerseOne{}Le pauvre qui marche dans son intégrité, vaut mieux que celui qui pervertit ses lèvres et qui est fou.
\VS{2}La vie même sans connaissance n'est pas une bonne personne ; et celui qui hâte ses pas dans le péché, s'égare.
\VS{3}La folie de l'homme renverse son chemin ; et cependant, c'est contre Yahweh que son coeur s'irrite.
\VS{4}Les richesses attirent un grand nombre d'amis, mais celui qui est pauvre est abandonné même par son ami.
\VS{5}Le faux témoin ne restera pas impuni, et celui qui profère des mensonges n'échappera pas.
\VS{6}Plusieurs supplient celui qui est en état de faire du bien, et chacun est ami de celui qui donne.
\VS{7}Tous les frères du pauvre le haïssent ; combien plus ses amis se retirent-ils de lui ! Il les supplie, mais il n'y a que des paroles pour lui.
\VS{8}Celui qui acquiert du sens aime son âme, et celui qui prend garde à l'intelligence c'est pour trouve le bonheur.
\VS{9}Le faux témoin ne restera pas impuni, et celui qui profère des mensonges périra.
\VS{10}Il ne sied pas à un fou de vivre dans les délices ; combien moins sied-il à un esclave de dominer sur les personnes de distinction !
\VS{11}La prudence de l'homme retient à la colère ; c'est un honneur pour lui de passer par dessus le tort qu'on lui fait.
\VS{12}La colère du roi est comme le rugissement d'un jeune lion, mais sa faveur est comme la rosée sur l'herbe.
\VS{13}Un fils insensé est un grand malheur pour son père, et les querelles d'une femme sont une gouttière continuelle.
\VS{14}On peut hériter de ses pères une maison et des richesses, mais la femme prudente est un don de Yahweh.
\VS{15}La paresse fait venir le sommeil, et l'âme paresseuse a faim.
\VS{16}Celui qui garde le commandement garde son âme, mais celui qui méprise ses voies mourra.
\VS{17}Celui qui a pitié du pauvre prête à Yahweh, qui lui rendra son bienfait.
\VS{18}Châtie ton fils tandis qu'il y a de l'espérance, mais ne va pas jusqu'à le faire mourir.
\VS{19}Celui qui est de grande colère en porte la peine ; et si tu l'en retires, tu y ajoute davantage.
\VS{20}Ecoute le conseil et reçois l'instruction, afin que tu deviennes sage en ton dernier temps.
\VS{21}Il y a dans le cœur de l'homme plusieurs pensées, mais le conseil de Yahweh est\FTNT{Es. 46:10 ; Ps. 33:11.}.
\VS{22}Ce que l'homme doit désirer, c'est d'exercer la miséricorde ; et le pauvre vaut mieux qu'un menteur.
\VS{23}La crainte de Yahweh conduit à la vie, et celui qui l'a, passe la nuit étant rassasié, sans qu'il soit visité par aucun mal.
\VS{24}Le paresseux cache sa main dans le sein, et il ne daigne même pas la ramener à sa bouche.
\VS{25}Si tu bats le moqueur, le sot en rend garde ; et si tu reprends l'homme intelligent, il discernera ce qu'il faut savoir.
\VS{26}L'enfant qui fait honte et sème la confusion, détruit le père et met en fuite sa mère.
\VS{27}Mon fils, cesse d'écouter ce qui pourrait t'apprendre à te détourner des paroles de la connaissance.
\VS{28}Le témoin indigne\FTNT{Le mot « pervers » vient de l'hébreu « beliya'al » : « sans valeur », « vaurien » (Jg. 19:22 ; 1. S. 2:12). Bélial est aussi un autre nom de Satan (2 Co. 6:15).} se moque de la justice, et la bouche des méchants avale l'iniquité.
\VS{29}Les jugements sont préparés pour les moqueurs, et les grands coups pour le dos des fous.
\Chap{20}
\TextTitle{[La justice s'oppose à la méchanceté (suite)]}
\VerseOne{}Le vin est moqueur et les boissons fortes sont tumultueuses, quiconque en fait excès, n'est pas sage.
\VS{2}La terreur du roi est comme le rugissement d'un jeune lion, celui qui l'irrite pèche contre sa propre âme.
\VS{3}C'est une gloire à l'homme de s'abstenir des disputes, mais tout insensé s'y engage.
\VS{4}Le paresseux ne labourera pas à cause de l'hiver, lors de la moisson il mendiera et n'aura rien.
\VS{5}Les conseils dans le coeur d'un homme sage sont comme des eaux profondes, et l'homme intelligent sait y puiser.
\VS{6}Beaucoup de gens vantent leur bonté ; mais l'homme fidèle, qui le trouvera ?
\VS{7}Ô, que les fils du juste qui marchent dans son intégrité seront heureux après lui !
\VS{8}Le roi assis sur le trône de justice dissipe tout mal par son regard.
\VS{9}Qui est-ce qui peut dire : J'ai purifié mon cœur, je suis net de mon péché ?
\VS{10}Le double poids et la double mesure sont tous deux en abomination à Yahweh.
\VS{11}Un jeune enfant fait connaître par ses actions si son oeuvre sera pure et droite.
\VS{12}L'oreille qui entend et l'oeil qui voit, Yahweh les a faits tous les deux.
\VS{13}N'aime point le sommeil, de peur que tu ne deviennes pauvre ; ouvre tes yeux, et tu auras suffisamment de pain.
\VS{14}Il est mauvais, il est mauvais, dit l'acheteur ; puis il s'en va, et se vante.
\VS{15}Il y a de l'or et beaucoup de perles ; mais les lèvres qui gardent la connaissance sont un vase précieux.
\VS{16}Quand quelqu'un se porte garant pour l'étranger, prends son vêtement ; exige de lui des gages pour cet étranger.
\VS{17}Le pain acquis par la tromperie est doux à l'homme, mais ensuite sa bouche sera remplie de gravier.
\VS{18}Les projets s'affermissent par le conseil ; fais donc la guerre avec prudence.
\VS{19}Celui qui médit révèle les secrets ; ne te mêle donc pas avec celui qui séduit par ses lèvres.
\VS{20}Celui qui traite avec mépris son père ou sa mère, sa lampe s'éteindra au milieu des ténèbres les plus noires\FTNT{Ex. 21:17 ; Lé. 20:9 ; Mt. 15:4.}.
\VS{21}L'héritage pour lequel on s'est trop hâté dès l'origine, ne sera pas béni à la fin.
\VS{22}Ne dis point : Je rendrai le mal ; espère en Yahweh, et il te délivrera.
\VS{23}Le double poids est en horreur à Yahweh, et la balance fausse n'est pas une chose bonne.
\VS{24}Les pas de l'homme sont dirigés par Yahweh, comment donc l'homme comprendrait-il sa voie ?
\VS{25}C'est un piège à l'homme que prendre à la légère un engagement sacré, et de ne réfléchir qu'après avoir fait un vœu.
\VS{26}Un roi sage disperse les méchants et ramène la roue sur eux.
\VS{27}L'esprit de l'homme est une lampe de Yahweh, il pénètre jusqu'au fond des entrailles.
\VS{28}La bienveillance et la vérité protègent le roi, et il soutient son trône par la bienveillance.
\VS{29}La force est la gloire des jeunes gens, et les cheveux blancs sont l'honneur des vieillards.
\VS{30}Les meurtrissures et les plaies nettoient le mal, de même les coups qui pénètrent jusqu'au fond des entrailles.
\Chap{21}
\TextTitle{[La justice s'oppose à la méchanceté (suite)]}
\VerseOne{}Le coeur du roi est un courant d'eau dans la main de Yahweh ; il l'incline partout où il veut.
\VS{2}Toutes les voies de l'homme sont droites à ses yeux, mais c'est Yahweh qui pèse les coeurs.
\VS{3}Faire ce qui est juste et droit est une chose que Yahweh préfère aux sacrifices.
\VS{4}Des regards hautains et le coeur qui s'enfle sont la lampe des méchants, ce n'est que péché.
\VS{5}Les projets de l'homme diligent ne mènent qu'à l'abondance, mais celui qui agit avec précipitation ne court qu'à l'indigence.
\VS{6}Des trésors acquis par une langue mensongère, c'est une vanité qu'on ne peut retenir, un signe avant-coureur de la mort.
\VS{7}La violence des méchants les emporte, parce qu'ils refusent de faire ce qui est droit.
\VS{8}La voie d'un homme coupable est détournée, mais l'oeuvre de celui qui est innocent est droite.
\VS{9}Il vaut mieux habiter à l'angle d'un toit qu'avec une femme querelleuse dans une grande maison.
\VS{10}L'âme du méchant désire le mal, son prochain ne trouve pas de grâce à ses yeux.
\VS{11}Quand on punit le moqueur, le sot devient sage ; et quand on instruit le sage, il reçoit la connaissance.
\VS{12}Il y a un juste qui considère attentivement la maison du méchant, Yahweh renverse les méchants dans le malheur.
\VS{13}Celui qui bouche son oreille pour ne pas entendre le cri du pauvre, criera aussi lui-même, et on ne lui répondra point.
\VS{14}Un don fait en secret apaise la colère, et un présent fait en cachette calme une fureur violente.
\VS{15}C'est une joie pour le juste de pratiquer la justice, mais c'est la ruine pour les ouvriers d'iniquité.
\VS{16}L'homme qui s'écarte du chemin de la sagesse aura sa demeure dans l'assemblée des morts.
\VS{17}Celui qui aime les réjouissances reste dans l'indigence ; et celui qui aime le vin et l'huile ne s'enrichira pas.
\VS{18}Le méchant sert de rançon pour le juste, et le déloyal pour les hommes intègres.
\VS{19}Il vaut mieux habiter dans une terre déserte qu'avec une femme querelleuse et qui se dépite.
\VS{20}Des précieux trésors et l'huile sont dans la demeure du sage, mais l'homme insensé les engloutit.
\VS{21}Celui qui poursuit la justice et la bonté, trouve la vie, la justice et la gloire.
\VS{22}Le sage entre dans la ville des forts et il abat la force qui lui donnait de l'assurance.
\VS{23}Celui qui veille sur sa bouche et sur sa langue préserve son âme des angoisses.
\VS{24}On appelle moqueur un superbe arrogant, qui agit avec colère et orgueil.
\VS{25}Les désirs du paresseux le tuent, parce que ses mains refusent de travailler.
\VS{26}Tout le jour il désire avidement, mais le juste donne sans parcimonie.
\VS{27}Le sacrifice des méchants est une abomination ; combien plus quand ils l'apportent avec des mauvaises intentions\FTNT{1 S. 15:22.} ?
\VS{28}Le témoin menteur périra, mais l'homme qui écoute parlera avec gain de cause.
\VS{29}L'homme méchant prend un air effronté, mais l'homme droit règle sa conduite.
\VS{30}Il n'y a ni sagesse, ni intelligence, ni conseil, contre Yahweh.
\VS{31}Le cheval est équipé pour le jour de la bataille, mais la délivrance vient de Yahweh.
\Chap{22}
\TextTitle{[La justice s'oppose à la méchanceté (suite)]}
\VerseOne{}La renommée est préférable aux grandes richesses\FTNT{Ec.7:1.}, et la bonne grâce plus que l'argent et l'or.
\VS{2}Le riche et le pauvre se rencontrent ; celui qui les a faits l'un et l'autre, c'est Yahweh\FTNT{Lu. 16.}.
\VS{3}L'homme prudent voit le mal et se cache, mais les stupides passent et en portent la peine.
\VS{4}Les fruits de l'humilité et de la crainte de Yahweh sont les richesses, la gloire et la vie.
\VS{5}Il y a des épines et des pièges dans la voie de l'homme pervers ; celui qui aime son âme s'en retirera loin.
\VS{6}Instruis le jeune enfant selon la voie qu'il doit suivre, et quand il sera vieux, il ne s'en détournera pas.
\VS{7}Le riche domine sur les pauvres\FTNT{Ja. 2:6.}, et celui qui emprunte est l'esclave de celui qui prête.
\VS{8}Celui qui sème l'injustice moissonne le malheur\FTNT{Job. 4:8 ; Ga. 6:7.}, et la verge de sa fureur prendra fin.
\VS{9}Celui qui a l'oeil bienveillant sera béni, parce qu'il aura donné de son pain au pauvre.
\VS{10}Chasse le moqueur, et la querelle prendra fin ; les disputes et l'ignominie cesseront.
\VS{11}Le roi est ami de celui qui aime la pureté de coeur, et qui a de la grâce dans ses paroles.
\VS{12}Les yeux de Yahweh veillent sur la connaissance, mais il confond les paroles du perfide.
\VS{13}Le paresseux dit : Il y a un lion dehors ! Je serais tué dans les rues !
\VS{14}La bouche des courtisanes est une fosse profonde, celui contre qui Yahweh est irrité y tombera.
\VS{15}La folie est liée au coeur du jeune enfant, mais la verge de la correction l'éloignera de lui.
\VS{16}Celui qui fait tort au pauvre pour s'enrichir et pour donner au riche, ne peut manquer de tomber dans l'indigence.
\VS{17}Prête ton oreille et écoute les paroles des sages, et applique ton coeur à ma connaissance.
\VS{18}Car ce sera une chose agréable pour toi si tu les gardes au-dedans de toi, et qu'elles soient toutes présentes sur tes lèvres.
\VS{19}Je te les ai fait connaître à toi aujourd'hui, dis-je, afin que ta confiance soit en Yahweh.
\VS{20}N'ai-je pas déjà pour toi mis par écrit des choses qui conviennent à ceux qui gouvernent, des conseils et des réflexions,
\VS{21}pour te faire connaître la certitude des paroles vraies, afin que tu répondes par des paroles vraies à celui qui t'envoie ?
\VS{22}Ne dépouille pas le pauvre, parce qu'il est pauvre ; et n'opprime pas le malheureux à la porte.
\VS{23}Car Yahweh défendra leur cause et privera de la vie ceux qui les auront volés.
\VS{24}Ne fréquente pas quelqu'un de coléreux, ne va pas avec l'homme violent ;
\VS{25}de peur que tu n'apprennes ses manières, et qu'ils ne deviennent un piège pour ton âme.
\VS{26}Ne sois pas parmi ceux qui prennent des engagements ni de ceux qui cautionnent les dettes.
\VS{27}Si tu n'as pas de quoi payer, pourquoi prendrait-on ton lit de dessous toi ?
\VS{28}Ne déplace pas la borne ancienne, que tes pères ont posée.
\VS{29}As-tu vu un homme habile en son travail ? Il sera au service des rois, il ne se tiendra pas devant des gens obscurs.
\Chap{23}
\TextTitle{[La justice s'oppose à la méchanceté (suite)]}
\VerseOne{}Quand tu t'assieds pour manger avec un gouverneur, considère avec attention celui qui est devant toi.
\VS{2}Autrement tu te mettras le couteau à la gorge, si ton appétit te domine.
\VS{3}Ne convoite pas ses friandises, car c'est un pain trompeur.
\VS{4}Ne travaille pas en vue d'acquérir des richesses ; désiste-toi de la résolution que tu as prise.
\VS{5}Jetteras-tu tes yeux sur ce qui bientôt n'est plus ? Car certainement, il se fera des ailes, il s'envolera comme un aigle dans les cieux.
\VS{6}Ne mange pas le pain de celui dont le regard est envieux, et ne désire pas ses friandises.
\VS{7}Car il est tel qu'il pense dans son âme. Il te dira bien : Mange et bois, mais son coeur n'est pas avec toi.
\VS{8}Tu voudrais vomir le morceau que tu auras mangé, et tu auras perdu tes paroles agréables.
\VS{9}Ne parle pas aux oreilles de l'insensé, car il méprise le bon sens de ton discours.
\VS{10}Ne déplace pas la borne ancienne et n'entre pas dans les champs des orphelins :
\VS{11}Car leur vengeur est puissant, il défendra leur cause contre toi.
\VS{12}Applique ton coeur à l'instruction, et tes oreilles aux paroles de la connaissance.
\VS{13}Ne te retiens pas de corriger le jeune enfant ; quand tu l'auras frappé de la verge, il n'en mourra pas.
\VS{14}En le frappant de la verge, tu préserves son âme du scheol.
\VS{15}Mon fils, si ton coeur est sage, mon coeur s'en réjouira, oui, moi-même.
\VS{16}Certes, mes reins tressailliront de joie quand tes lèvres proféreront ce qui est droit.
\VS{17}Que ton coeur ne porte pas d'envie aux pécheurs, mais adonne-toi tout le jour à la crainte de Yahweh.
\VS{18}Car il y a véritablement un avenir, et ton espérance ne sera pas retranchée.
\VS{19}Toi, mon fils, écoute et sois sage, et dirige ton coeur dans la bonne voie.
\VS{20}Ne fréquente point les ivrognes ni les gourmands\FTNT{Ro. 13:13 ; Ep. 5:18 ; Ga. 5:18-21.}.
\VS{21}Car l'ivrogne et le gourmand s'appauvrissent ; et l'assoupissement fait porter des vêtements déchirés.
\VS{22}Ecoute ton père, c'est celui qui t'a engendré ; et ne méprise pas ta mère, quand elle est devenue vieille.
\VS{23}Acquiers la vérité, et ne la vends point, acquiers la sagesse, l'instruction et l'intelligence.
\VS{24}Le père du juste aura beaucoup de joie, et celui qui donne naissance à un sage se réjouira en lui.
\VS{25}Que ton père et ta mère se réjouissent, que celle qui t'a enfanté soit dans l'allégresse !
\VS{26}Mon fils, donne-moi ton coeur, et que tes yeux prennent plaisir à mes voies.
\VS{27}Car la femme prostituée est une fosse profonde, et la courtisane un puits de détresse.
\VS{28}Aussi se tient-elle en embûche comme un voleur, et elle augmente parmi les hommes le nombre des infidèles.
\VS{29}Pour qui les « ah » ? Pour qui les « malheur à moi ! » Pour qui les disputes ? Pour qui les plaintes ? Pour qui les blessures sans raison ? Pour qui les yeux rouges ?
\VS{30}Pour ceux qui s'attardent auprès du vin, pour ceux qui vont chercher des vins mélangés.
\VS{31}Ne regarde pas le vin parce qu'il est d'un beau rouge, qu'il donne son éclat dans la coupe, et qu'il coule aisément.
\VS{32}Il finit par mordre par derrière comme un serpent, et par piquer comme un basilic.
\VS{33}Ensuite tes yeux regarderont les femmes étrangères, et ton coeur parlera d'une manière perverse.
\VS{34}Tu seras comme un homme qui dort au milieu de la mer, et comme un homme couché sur le sommet d'un mât.
\VS{35}On m'a battu, diras-tu, et je n'en ai pas été malade, on m'a frappé, et je ne l'ai pas senti, quand me réveillerai-je ? Je me remettrai encore à chercher le vin.
\Chap{24}
\TextTitle{[La justice s'oppose à la méchanceté (suite)]}
\VerseOne{}N'envie pas les hommes qui font le mal, et ne désire pas être avec eux.
\VS{2}Car leur coeur médite la destruction, et leurs lèvres parlent d'iniquité.
\VS{3}C'est par la sagesse qu'une maison est bâtie, et par l'intelligence qu'elle s'affermit.
\VS{4}C'est par la connaissance que les chambres seront remplies de tous les biens précieux et agréables.
\VS{5}Un homme sage est accompagné de force, et celui qui a de la connaissance affermit sa vigueur.
\VS{6}Car avec de bonnes directives tu feras la guerre avantageusement, et le salut est dans le grand nombre des bons conseillers.
\VS{7}La sagesse est trop élevée pour l'insensé, il n'ouvrira pas sa bouche à la porte.
\VS{8}Celui qui médite de faire le mal s'appelle un homme plein de malice.
\VS{9}Le projet de la folie est un péché, et le moqueur est en abomination parmi les hommes.
\VS{10}Si tu perds courage au jour de la détresse, ta force n'est que détresse.
\VS{11}Ne te retiens pas de délivrer ceux qu'on traîne à la mort, ceux qu'on va égorger, agis pour qu'on les épargne !
\VS{12}Si tu dis : Ah ! nous n'en savions rien… Celui qui pèse les cœurs, lui, ne le considérera-t-il pas ? Celui qui garde ton âme, lui, le sait, et il rend à chacun selon son œuvre.
\VS{13}Mon fils, mange le miel, car il est bon ; un rayon de miel sera doux à ton palais.
\VS{14}Ainsi sera à ton âme la connaissance de la sagesse, quand tu l'auras trouvée ; il y a un avenir, et ton espérance ne sera pas anéantie.
\VS{15}Méchant, ne mets pas des embûches dans le domaine du juste, et ne dévaste pas le lieu où il se repose.
\VS{16}Car le juste tombera sept fois, et sera relevé\FTNT{Ps. 34:20. ; Job. 5:19.} ; mais les méchants trébuchent pour tomber dans le malheur.
\VS{17}Si ton ennemi tombe, ne t'en réjouis pas, et que ton coeur ne soit pas dans l'allégresse quand il chancelle,
\VS{18}de peur que Yahweh ne le voie et que cela ne lui déplaise, tellement qu'il détourne sa colère de dessus de lui sur toi.
\VS{19}Ne t'irrite pas à cause de ceux qui font le mal, n'envie pas les méchants,
\VS{20}car il n'y a pas d'avenir pour celui qui fait le mal, la lampe des méchants sera éteinte.
\VS{21}Mon fils, crains Yahweh et le roi ; et ne te mêle pas avec les gens agités.
\VS{22}Car leur ruine s'élèvera tout d'un coup ; et qui sait le malheur qui arrivera à l'un et à l'autre ?
\VS{23}Voici encore ce qui vient des sages : Il n'est pas bon d'avoir égard à l'apparence des personnes dans le jugement.
\VS{24}Celui qui dit au méchant : Tu es juste ! Les peuples le maudiront, et les nations seront indignées contre lui.
\VS{25}Mais pour ceux qui le reprennent, ils en retireront de la satisfaction. Et la bénédiction vient sur eux pour leur bonheur.
\VS{26}Celui qui répond avec justesse fait plaisir à celui qui l'écoute.
\VS{27}Prépare ton ouvrage au-dehors, et apprête ton champ, et après, tu bâtiras ta maison.
\VS{28}Ne témoigne pas sans cause contre ton prochain ; car voudrais-tu tromper de tes lèvres\FTNT{Ep. 4:25.} ?
\VS{29}Ne dis pas : Comme il m'a fait, ainsi je lui ferai ; je rendrai à cet homme selon ce qu'il m'a fait.
\VS{30}J'ai passé près du champ de l'homme de paresseux, et près de la vigne d'un homme dépourvu de sens ;
\VS{31}et voici, tout y était monté en chardons, et les ronces avaient couvert la surface, et le mur de pierres était écroulé.
\VS{32}Et j'ai regardé, j'y ai appliqué mon coeur, j'ai vu et j'en ai tiré instruction.
\VS{33}Un peu de sommeil, un peu d'assoupissement, un peu croiser les mains pour dormir ! …
\VS{34}Et la pauvreté te surprendra comme un rôdeur ; et la disette, comme un homme armé.
\Chap{25}
\TextTitle{[Avertissements et conseils]}
\VerseOne{}Ce sont ici aussi des proverbes de Salomon\FTNT{1 R. 4: 32.}, que les gens d'Ezéchias, roi de Juda, ont transcrits.
\VS{2}La gloire de Dieu est de cacher les choses, et la gloire des rois est de sonder les choses.
\VS{3}Les cieux dans leur hauteur, la terre dans sa profondeur, le coeur des rois sont impénétrables.
\VS{4}Ôte de l'argent les scories, et il en sortira un vase pour l'orfèvre.
\VS{5} De même, ôte le méchant de devant le roi, et son trône sera affermi par la justice.
\VS{6}Ne t'élève pas devant le roi et ne te tiens pas à la place des grands.
\VS{7}Car il vaut mieux qu'on te dise: Monte-ici ! Que si l'on t'abaisse devant un seigneur que tes yeux ont vu\FTNT{Lu. 14:8-11.}.
\VS{8}Ne te hâte pas d'entrer en contestation, de peur que tu ne saches que faire à la fin, lorsque ton prochain t'aura confondu.
\VS{9}Plaide ta cause contre ton prochain, mais ne révèle pas le secret d'autrui.
\VS{10}De peur que celui qui l'écoute ne te couvre de honte, et que ton opprobre ne s'efface pas.
\VS{11}Telles que sont des pommes d'or sur des ciselures d'argent, telle est une parole dite quand il faut.
\VS{12}Comme un anneau d'or ou comme un joyau d'or fin, ainsi est l'oreille obéissante pour le sage qui réprimande.
\VS{13}Le messager fidèle est à ceux qui l'envoient, comme la fraîcheur de la neige au temps de la moisson, il restaure l'âme de son maître.
\VS{14}Celui qui se glorifie faussement de ses libéralités, est comme les nuages et le vent sans pluie.
\VS{15}Un prince est fléchi par la patience, et la langue douce brise les os.
\VS{16}As-tu trouvé du miel, manges-en autant qu'il t'en faut, de peur que tu n'en sois rassasié, que tu ne le vomisses.
\VS{17}De même, mets rarement le pied dans la maison de ton prochain, de peur qu'étant rassasié de toi, il ne te haïsse.
\VS{18}Comme une massue, une épée et une flèche aiguë, ainsi est un homme qui porte un faux témoignage contre son prochain.
\VS{19}Comme une dent qui se rompt, un pied qui glisse, telle est la confiance qu'on met en un traître au jour de la détresse.
\VS{20}Celui qui chante des chansons à un coeur affligé est comme celui qui ôte sa robe dans un jour froid, et comme du vinaigre répandu sur du nitre.
\VS{21}Si celui qui te hait a faim, donne-lui à manger du pain ; et s'il a soif, donne-lui à boire de l'eau\FTNT{Mt 5:39-44.}.
\VS{22}Car ce sont des charbons ardents que tu lui mets sur sa tête, et Yahweh te le rendra.
\VS{23}Le vent du nord engendre les averses, et la langue qui médit en cachette un visage irrité.
\VS{24}Il vaut mieux habiter à l'angle d'un toit que de partager la demeure d'une femme querelleuse.
\VS{25}Comme de l'eau fraîche pour une personne fatiguée et lasse, ainsi est une bonne nouvelle venant d'une terre lointaine.
\VS{26}Le juste qui bronche devant le méchant est une fontaine troublée et une source gâtée.
\VS{27}Comme il n'est pas bon de manger beaucoup de miel, aussi n'y a-t-il pas de gloire pour les hommes de rechercher leur gloire avec ardeur.
\VS{28}L'homme qui n'est pas maître de son esprit est comme une ville où il y a une brèche, et qui est sans murailles.
\Chap{26}
\TextTitle{[Avertissements et conseils (suite)]}
\VerseOne{}Comme la neige en été et la pluie pendant la moisson, ainsi la gloire ne convient pas à un insensé.
\VS{2}Comme l'oiseau est prompt à s'échapper et l'hirondelle à s'envoler, ainsi la malédiction sans cause n'atteint pas.
\VS{3}Le fouet est pour le cheval, le mors pour l'âne, et la verge pour le dos des insensés.
\VS{4}Ne réponds pas à l'insensé selon sa folie, de peur que tu ne lui ressembles toi-même.
\VS{5}Réponds à l'insensé selon sa folie, de peur qu'il ne devienne sage à ses propres yeux.
\VS{6}Celui qui envoie des messages par l'intermédiaire d'un insensé, se coupe les pieds et boit la peine du tort qu'il s'est fait.
\VS{7}Faites marcher un homme boiteux, ainsi il en sera d'un proverbe dans la bouche des insensés.
\VS{8}Celui qui donne de la gloire à un insensé, c'est comme s'il jetait une pierre précieuse dans un monceau de pierres.
\VS{9}Comme une épine dans la main d'un homme ivre, ainsi est un proverbe dans la bouche des insensés.
\VS{10}Les puissants donnent de l'ennui à tout le monde, et prennent à leur service les insensés et les passants.
\VS{11}Comme le chien retourne à ce qu'il a vomi, ainsi l'insensé réitère sa folie\FTNT{2 Pi. 2:22.}.
\VS{12}As-tu vu un homme qui croit être sage ? Il y a plus à espérer d'un insensé que de lui.
\VS{13}Le paresseux dit : Il y a un lion rugissant sur le chemin, il y a un lion dans les rues.
\VS{14}Comme une porte tourne sur ses gonds, ainsi fait le paresseux sur son lit.
\VS{15}Le paresseux plonge sa main dans le plat, et il trouve fatigant de la ramener à sa bouche.
\VS{16}Le paresseux se croit plus sage que sept hommes qui répondent avec bon sens.
\VS{17}Celui qui en passant se met en colère pour une dispute qui ne le touche en rien, est comme celui qui prend un chien par les oreilles.
\VS{18}Tel est celui qui fait l'insensé, et qui cependant jette des feux, des flèches, et des choses propres à tuer,
\VS{19}tel est l'homme qui a trompé son ami, et qui après cela dit : N'était-ce pas pour plaisanter ?
\VS{20}Le feu s'éteint faute de bois, ainsi quand il n'y a pas de rapporteurs la querelle s'apaise.
\VS{21}Le charbon est pour faire de la braise, et le bois pour faire du feu, et l'homme querelleur pour exciter des querelles.
\VS{22}Les paroles d'un rapporteur sont comme des friandises, elles descendent jusqu'au fond des entrailles.
\VS{23}Les lèvres brûlantes de zèle et le coeur mauvais sont comme des scories d'argent appliquées sur un vase de terre.
\VS{24}Celui qui a de la haine se déguise par ses discours, mais au-dedans de lui il nourrit la trahison.
\VS{25}Lorsqu'il prend une voix douce, ne le crois pas, car il y a sept abominations dans son coeur.
\VS{26}S'il cache sa haine sous la dissimulation, sa méchanceté sera révélée dans l'assemblée.
\VS{27}Celui qui creuse la fosse y tombe ; et la pierre retourne sur celui qui la roule\FTNT{Ps. 7:16-17 ; Ps. 57:7 ; Ec. 10:8.}.
\VS{28}La fausse langue hait ceux qu'elle a écrasés ; et la bouche flatteuse prépare la ruine.
\Chap{27}
\TextTitle{[Avertissements et conseils (suite)]}
\VerseOne{}Ne te vante point du lendemain, car tu ne sais pas quelle chose le jour enfantera\FTNT{Ja. 4:13-15.}.
\VS{2}Qu'un autre te loue, et non pas ta propre bouche ; que ce soit l'étranger, et non pas tes lèvres.
\VS{3}La pierre est pesante, et le sable est lourd ; mais l'irritation de l'insensé est plus pesante que tous les deux.
\VS{4}Il y a de la cruauté dans la fureur, et du débordement dans la colère ; mais qui pourra subsister devant la jalousie ?
\VS{5}Mieux vaut une réprimande ouverte qu'une amitié cachée.
\VS{6}Les blessures d'un ami sont dignes de confiance, mais les baisers d'un ennemi sont à craindre\FTNT{Il est question ici de Judas}.
\VS{7}L'âme rassasiée foule aux pieds les rayons de miel ; mais l'âme qui a faim trouve doux même ce qui est amer.
\VS{8}Tel qu'est un oiseau errant loin de son nid, tel est l'homme qui s'écarte de son lieu.
\VS{9}L'huile et les parfums réjouissent le coeur, et il en est ainsi de la douceur d'un ami dont le fruit est un conseil qui vient du coeur.
\VS{10}Ne quitte point ton ami ni l'ami de ton père, et n'entre pas dans la maison de ton frère au jour de ta détresse ; car le voisin qui est proche vaut mieux que le frère qui est éloigné.
\VS{11}Mon fils sois sage, et réjouis mon coeur, afin que j'aie de quoi répondre à celui qui m'outrage.
\VS{12}L'homme prudent voit le malheur arriver et se cache ; mais les stupides passent outre et en payent l'amende.
\VS{13}Quand quelqu'un se portera garant pour l'étranger, prends son vêtement, exige de lui des gages, à cause des étrangers.
\VS{14}Celui qui bénit son ami à haute voix, se levant de grand matin, on le lui comptera comme une malédiction.
\VS{15}Une gouttière continuelle en un jour de grosse pluie, et une femme querelleuse, cela se ressemble.
\VS{16}Celui qui veut la retenir est comme s'il voulait arrêter le vent, et retenir dans sa main une huile qui s'écoule.
\VS{17}Comme le fer aiguise le fer, ainsi l'homme aiguise la personnalité de son prochain.
\VS{18}Comme celui qui garde le figuier mangera de son fruit, ainsi celui qui garde son maître sera honoré.
\VS{19}Comme dans l'eau le visage répond au visage, ainsi le coeur d'un homme répond à celui d'un autre homme.
\VS{20}Le scheol et le gouffre ne sont jamais rassasiés ; de même, les yeux des hommes sont insatiables\FTNT{Ec. 1:8 ; 2 Pi. 2:14.}.
\VS{21}Le fourneau est pour éprouver l'argent, et le creuset pour l'or ; mais un homme est jugé d'après sa renommée.
\VS{22}Quand tu pilerais un insensé dans un mortier, parmi du grain qu'on pile avec un pilon, sa stupidité ne se détacherait pas de lui.
\VS{23}Sois diligent à reconnaître l'état de chacune de tes brebis, et applique ton coeur aux troupeaux.
\VS{24}Car l'abondance ne dure pas à toujours, et une couronne dure-t-elle d'âge en âge ?
\VS{25}Le foin s'enlève et la verdure paraît, et les herbes des montagnes sont amassées.
\VS{26}Les agneaux sont pour te vêtir, et les boucs pour payer le champ ;
\VS{27}et l'abondance du lait des chèvres sera pour ta nourriture et celle de ta maison, et pour la subsistance de tes servantes.
\Chap{28}
\TextTitle{[Avertissements et conseils (suite)]}
\VerseOne{}Le méchant prend la fuite sans qu'on le poursuive, mais les justes seront assurés comme un jeune lion.
\VS{2}Il y a plusieurs chefs, à cause de la rébellion d'un pays, mais pour l'amour de l'homme avisé et intelligent, il y aura prolongation du même gouvernement.
\VS{3}L'homme qui est pauvre et qui opprime les pauvres, est comme une pluie violente qui cause la disette du pain.
\VS{4}Ceux qui abandonnent la loi louent le méchant, mais ceux qui gardent la loi leur font la guerre.
\VS{5}Les gens adonnés au mal n'entendent point ce qui est droit ; mais ceux qui cherchent Yahweh comprennent tout.
\VS{6}Le pauvre qui marche dans son intégrité vaut mieux que le pervers qui marche par deux chemins et qui est riche.
\VS{7}Celui qui garde la loi est un fils prudent, mais celui qui entretient les gourmands fait honte à son père.
\VS{8}Celui qui augmente ses biens par l'intérêt et l'usure, les amasse pour celui qui en fera des libéralités aux pauvres.
\VS{9}Celui qui détourne son oreille pour ne pas écouter la loi, sa prière même est une abomination\FTNT{La prière doit être faite selon la Parole de Dieu, en conformité avec sa volonté. Le Seigneur n'exauce que ceux qui obéissent à sa Parole (Mt. 6:9-10 ; Jn. 9:31 ; Jn. 15:7 ; 1 Jn. 5:14-15).}.
\VS{10}Celui qui égare les hommes droits dans le mauvais chemin tombera dans la fosse qu'il aura creusée, mais ceux qui sont intègres hériteront le bonheur.
\VS{11}L'homme riche pense être sage, mais le pauvre qui est intelligent le sondera.
\VS{12}Quand les justes se réjouissent, la gloire est grande, mais quand les méchants sont élevés, chacun se déguise.
\VS{13}Celui qui cache ses transgressions ne prospère point, mais celui qui les confesse et les délaisse, obtient miséricorde\FTNT{Ec. 1:8 ; 2 Pi. 2:14.}.
\VS{14}Heureux est l'homme qui est continuellement dans la crainte, mais celui qui endurcit son coeur tombera dans la calamité.
\VS{15}Le méchant qui domine sur un peuple pauvre est un lion rugissant et comme un ours quêtant sa proie.
\VS{16}Le conducteur qui manque d'intelligence fait beaucoup d'extorsions, mais celui qui hait le gain déshonnête prolonge ses jours.
\VS{17}L'homme chargé du sang d'une personne fuira jusqu'à la fosse sans qu'aucun ne le retienne.
\VS{18}Celui qui marche dans l'intégrité sera sauvé, mais le pervers qui suit deux chemins tombera tout à coup.
\VS{19}Celui qui laboure sa terre sera rassasié de pain, mais celui qui suit les fainéants sera accablé de misère.
\VS{20}L'homme fidèle abondera en bénédictions, mais celui qui se hâte de s'enrichir ne restera pas impuni.
\VS{21}Il n'est pas bon d'avoir égard à l'apparence des personnes, car pour un morceau de pain l'homme commet un crime.
\VS{22}L'homme qui a l'oeil malin se hâte pour avoir des richesses, et il ne sait pas que la disette lui arrivera.
\VS{23}Celui qui reprend les hommes obtient ensuite plus de faveur que celui qui flatte de sa langue.
\VS{24}Celui qui pille son père ou sa mère, et qui dit que ce n'est point un péché, est compagnon de l'homme dissipateur.
\VS{25}Celui qui a l'âme enflée excite les querelles, mais celui qui se confie en Yahweh sera rassasié.
\VS{26}Celui qui se confie dans son propre coeur est un fou, mais celui qui marche sagement sera délivré.
\VS{27}Celui qui donne au pauvre n'aura point de disette, mais celui qui en détourne ses yeux abondera en malédictions.
\VS{28}Quand les méchants s'élèvent, l'homme se cache ; mais quand ils périssent, les justes se multiplient.
\Chap{29}
\TextTitle{[Avertissements et conseils (suite)]}
\VerseOne{}L'homme qui étant repris, raidit son cou, sera subitement brisé et sans qu'il y ait de guérison.
\VS{2}Quand les justes sont nombreux, le peuple se réjouit ; mais quand le méchant domine, le peuple gémit.
\VS{3}L'homme qui aime la sagesse, réjouit son père, mais celui qui se plaît avec les femmes prostituées dissipe ses richesses.
\VS{4}Le roi affermit le pays par la justice, mais l'homme qui est adonné aux présents le ruinera.
\VS{5}L'homme qui flatte son prochain lui tend un piège sous ses pas.
\VS{6}Le péché de l'homme méchant lui tend un piège dangereux, mais le juste triomphe et se réjouit.
\VS{7}Le juste prend connaissance de la cause des pauvres, mais le méchant n'en prend pas connaissance.
\VS{8}Les hommes moqueurs troublent la ville, mais les sages apaisent la colère.
\VS{9}Un homme sage qui conteste avec un insensé, qu'il se fâche ou qu'il rie, la paix n'aura pas lieu.
\VS{10}Les hommes de sang ont en haine l'homme intègre, mais les hommes droits tiennent chère sa vie.
\VS{11}L'insensé pousse au-dehors tout ce qu'il a dans l'esprit, mais le sage le calme et le retient en arrière.
\VS{12}Tous les serviteurs d'un prince qui prêtent l'oreille à la parole de mensonge sont méchants.
\VS{13}Le pauvre et l'oppresseur se rencontrent, c'est Yahweh qui illumine les yeux de l'un et de l'autre.
\VS{14}Le trône du roi qui fait justice selon la vérité aux pauvres, sera établi à perpétuité.
\VS{15}La verge et la réprimande donnent la sagesse, mais l'enfant livré à lui-même fait honte à sa mère.
\VS{16}Quand les méchants se multiplient, les péchés s'accroissent, mais les justes verront leur ruine.
\VS{17}Corrige ton fils, et il te donnera du repos, et il procurera du plaisir à ton âme.
\VS{18}Lorsqu'il n'y a pas de vision\FTNT{Le manque de vision n'est bon pour personne. Dieu donne une vision aux personnes qu'il a appelées. La vision peut être un songe, une directive, une prophétie, etc. Il s'agit des objectifs à atteindre.}, le peuple est sans frein, mais heureux est celui qui garde la loi !
\VS{19}L'esclave ne se corrige pas par des paroles, même s'il comprend, il n'obéit pas.
\VS{20}As-tu vu un homme irréfléchi dans ses paroles ? Il y a plus à espérer d'un insensé que de lui.
\VS{21}Le serviteur qu'on a traité délicatement dès sa jeunesse finit par se croire un fils.
\VS{22}L'homme coléreux excite des querelles, et l'homme furieux commet beaucoup de péchés.
\VS{23}L'orgueil de l'homme l'abaisse, mais celui qui est humble d'esprit obtient la gloire\FTNT{Mt. 23:12 ; Lu. 14:11 ; 1 Pi. 5:5.}.
\VS{24}Celui qui partage avec un voleur hait son âme ; il entend la malédiction, et il ne révèle rien.
\VS{25}La crainte qu'on a des hommes tend un piège, mais celui qui se confie en Yahweh est élevé dans une haute retraite.
\VS{26}Plusieurs recherchent la face de celui qui domine, mais c'est de Yahweh que vient le jugement des hommes.
\VS{27}L'homme inique est en abomination aux justes, et celui dont la voie est droite est en abomination au méchant.
\Chap{30}
\TextTitle{[Proverbe d'Agur]}
\VerseOne{}Les paroles d'Agur, fils de Jaké, à savoir la sentence prononcée par cet homme pour Ithiel, pour Ithiel et Ucal.
\VS{2}Certainement je suis le plus stupide de tous les hommes, et il n'y a pas en moi l'intelligence humaine.
\VS{3}Et je n'ai pas appris la sagesse ; et je n'ai pas la connaissance des saints.
\VS{4}Qui est celui qui est monté aux cieux, ou qui en est descendu\FTNT{Jn. 3:13 ; Ro. 10:6-7.} ? Qui est celui qui a recueilli le vent dans le creux de sa main, qui a serré les eaux dans son manteau, qui a dressé toutes les bornes de la terre ? Quel est son nom, et quel est le nom de son fils, le sais-tu ?
\VS{5}Toute la parole de Dieu est éprouvée ; il est un bouclier pour ceux qui se réfugient en lui\FTNT{Ps. 18:31 ; Ps. 115:9-11.}.
\VS{6}N'ajoute rien à ses paroles, de peur qu'il ne te reprenne et que tu ne sois trouvé menteur.
\VS{7}Je te demande deux choses : Ne me les refuse pas durant ma vie.
\VS{8}Eloigne de moi la vanité et la parole mensongère ; ne me donne ni pauvreté ni richesse, nourris-moi du pain qui m'est nécessaire.
\VS{9}De peur que dans l'abondance je ne te renie, et que je ne dise : Qui est Yahweh ? Ou que dans la pauvreté, je ne dérobe et que je ne porte atteinte au Nom de mon Dieu.
\VS{10}N'accuse pas un serviteur devant son maître, de peur que ce serviteur ne te maudisse, et qu'il ne t'en arrive du mal.
\VS{11}Il est une race de gens qui maudit son père et qui ne bénit pas sa mère.
\VS{12}Il est une race de gens qui croit être pure, et qui toutefois n'est point lavée de son ordure.
\VS{13}Il est une race de gens dont les yeux sont fort hautains, et les paupières élevées.
\VS{14}Il est une race de gens dont les dents sont des épées, et les mâchoires sont des couteaux pour dévorer les malheureux sur la terre et les pauvres d'entre les hommes.
\VS{15}La sangsue a deux filles qui disent : Apporte ! Apporte ! Il y a trois choses qui sont insatiables, il y en a même quatre qui ne disent point : C'est assez !
\VS{16}Le scheol, la matrice stérile, la terre qui n'est pas rassasiée d'eau, et le feu qui ne dit jamais : C'est assez !
\VS{17}L'oeil de celui qui se moque de son père et qui méprise l'enseignement de sa mère, les corbeaux des torrents le crèveront, et les petits de l'aigle le mangeront.
\VS{18}Il y a trois choses qui sont trop merveilleuses pour moi, même quatre, que je ne connais point :
\VS{19}La trace de l'aigle dans le ciel, la trace du serpent sur un rocher, le chemin d'un navire au milieu de la mer, et la trace de l'homme chez la jeune femme.
\VS{20}Telle est la trace de la femme adultère : Elle mange, et s'essuie la bouche, puis elle dit : Je n'ai pas commis d'iniquité.
\VS{21}La terre tremble pour trois choses, même pour quatre, qu'elle ne peut supporter :
\VS{22}Pour l'esclave quand il vient à régner, pour l'insensé quand il est rassasié de pain,
\VS{23}pour la femme odieuse quand elle se marie, et pour la servante quand elle hérite de sa maîtresse.
\VS{24}Il y a quatre choses des plus petites de la terre qui toutefois sont bien sages entre les sages :
\VS{25}Les fourmis, qui sont un peuple sans force, et qui néanmoins préparent durant l'été leur nourriture ;
\VS{26}les damans, qui sont un peuple qui n'est pas puissant, et qui néanmoins font leurs maisons dans les rochers ;
\VS{27}les sauterelles, qui n'ont point de roi, et qui toutefois sortent toutes par divisions ;
\VS{28}les lézards que tu peux saisir avec les mains, et qui sont pourtant dans les palais des rois.
\VS{29}Il y a trois choses qui ont une belle allure, même quatre, qui ont une belle démarche :
\VS{30}Le lion, qui est le plus fort d'entre les animaux, et qui ne recule pas à la rencontre de qui que ce soit ;
\VS{31}le cheval, qui a les flancs bien troussés ; le bouc, et le roi devant qui personne ne résiste.
\VS{32}Si tu t'es conduit follement en t'emportant, et si tu as des mauvaises intentions, mets la main sur ta bouche.
\VS{33}Comme celui qui bat le lait en fait sortir le beurre, et comme celui qui presse le nez en fait sortir le sang, ainsi celui qui provoque la colère excite la querelle.
\Chap{31}
\TextTitle{Proverbe de Lemuel}
\VerseOne{}Les paroles du roi Lemuel et l'instruction que sa mère lui donna.
\VS{2}Quoi, mon fils ? Quoi, fils de mes entrailles ? Eh quoi, mon fils, pour lequel j'ai tant fait de voeux ?
\VS{3}Ne livre pas ta vigueur aux femmes et tes voies à celles qui perdent les rois.
\VS{4}Lemuel, ce n'est point aux rois, ce n'est point aux rois de boire le vin, ni aux princes de boire la cervoise\FTNT{La cervoise est une bière faite avec de l'orge ou d'autres céréales.}
\VS{5}de peur qu'ayant bu, ils n'oublient ce qui a été prescrit, et qu'ils n'altèrent le jugement de tous les pauvres affligés.
\VS{6}Donnez de la cervoise à celui qui va périr, et du vin à celui qui a l'amertume dans le coeur ;
\VS{7}afin qu'il en boive, et qu'il oublie sa pauvreté, et ne se souvienne plus de ses peines.
\VS{8}Ouvre ta bouche en faveur du muet, pour la cause de tous les fils délaissés qui vont périr.
\VS{9}Ouvre ta bouche, fais justice, et plaide pour le malheureux et l'indigent.
\TextTitle{[La femme vertueuse]}
\VS{10}[Aleph.] Qui est-ce qui trouvera une femme vertueuse ? Car son prix surpasse de beaucoup les perles.
\VS{11}[Beth.] Le coeur de son mari a confiance en elle, et il ne manquera point de dépouilles.
\VS{12}[Guimel.] Elle lui fera du bien tous les jours de sa vie, et jamais du mal.
\VS{13}[Daleth.] Elle cherche de la laine et du lin, et elle en fait ce qu'elle veut avec ses mains.
\VS{14}[He.] Elle est semblable aux navires d'un marchand, elle amène son pain de loin.
\VS{15}[Vav.] Elle se lève lorsqu'il est encore nuit, elle donne la nourriture nécessaire à sa maison et elle donne à ses servantes leur tâche.
\VS{16}[Zayin.] Elle pense à un champ, et l'acquiert ; et elle plante la vigne du fruit de ses mains.
\VS{17}[Heth.] Elle ceint ses reins de force, et affermit ses bras.
\VS{18}[Teth.] Elle sent que ce qu'elle gagne est bon ; sa lampe ne s'éteint pas pendant la nuit.
\VS{19}[Yod.] Elle met sa main à la quenouille, et ses doigts tiennent le fuseau.
\VS{20}[Kaf.] Elle tend sa main au malheureux, elle tend ses mains à l'indigent.
\VS{21}[Lamed.] Elle ne craint point la neige pour sa famille, car toute sa famille est vêtue de vêtements doubles.
\VS{22}[Mem.] Elle se fait des couvertures, le fin lin et l'écarlate sont ce dont elle s'habille.
\VS{23}[Nun.] Son mari est considéré aux portes, lorsqu'il siège avec les anciens du pays.
\VS{24}[Samech.] Elle fait des chemises et les vend, et elle livre des ceintures au marchand.
\VS{25}[Ayin.] Elle est revêtue de force et de gloire, elle se rit du jour à venir.
\VS{26}[Pe.] Elle ouvre sa bouche avec sagesse, et la loi de la charité est sur sa langue.
\VS{27}[Tsade.] Elle veille sur ce qui se passe dans sa maison, et elle ne mange pas le pain de la paresse.
\VS{28}[Qof.] Ses fils se lèvent et la disent bienheureuse ; son mari aussi, et il la loue, en disant :
\VS{29}[Resh.] Plusieurs filles se sont conduites vertueusement, mais toi, tu les surpasses toutes.
\VS{30}[Shin.] La grâce est trompeuse et la beauté vaine, mais la femme qui craint Yahweh est celle qui sera louée.
\VS{31}[Tav.] Récompensez-la du fruit de ses mains, et que ses oeuvres la louent aux portes.
\PPE{}
\end{multicols}

%\clearpage\ShortTitle{Job}\BookTitle{Job}\BFont
\noindent\hrulefill
{\footnotesize
\textit{
\bigskip
{\centering{}
\\Auteur : Inconnu
\\(Heb. : Iyov)
\\Signification : haï, ennemi et « je m'exclamerai »
\\Thème : La souffrance
\\Date de rédaction : Incertaine\\}
}
%\bigskip
\textit{
\\Job était un homme prospère et intègre auquel Dieu rendit témoignage. Il subit une succession de malheurs en très peu de temps en perdant tout ce qui lui était cher. Après avoir cherché à se justifier et subi les railleries de sa femme et les accusations de ses amis, Job s'humilia devant Dieu et comprit l'impuissance de sa propre justice. Cette histoire, dont on n'a aucune indication spatio-temporelle et qui pourtant parle à tous, est un encouragement pour le juste éprouvé.
%\bigskip
\\Rappelant que la souffrance peut être le moyen choisi par Dieu pour enseigner et se révéler, ce récit illustre la fidélité et la bonté de Yahweh envers ceux qui le craignent.\bigskip
}
}
\par\nobreak\noindent\hrulefill
\begin{multicols}{2}
\Chap{1}
\TextTitle{Job et sa famille}
\VerseOne{}Il y avait dans le pays d'Uts\FTNT{Ge. 36:28} un homme dont le nom était Job\FTNT{Ez. 14:14; Ja. 5:11.}. Cet homme était intègre\FTNT{1 R. 8:61.} et droit, craignant\FTNT{Ps. 19:10; Pr. 1:7.} Dieu et se détournant du mal.
\VS{2}Il eut sept fils et trois filles.
\VS{3}Et son bétail était de sept mille brebis, trois mille chameaux, cinq cents paires de bœufs, cinq cents ânesses, avec un très grand nombre de serviteurs\FTNT{Job 42:12-13.}; tellement que cet homme était le plus puissant de tous les Orientaux.
\VS{4}Or ses fils allaient et faisaient des festins les uns chez les autres chacun à son jour, et ils envoyaient appeler leurs trois sœurs pour manger et boire avec eux.
\VS{5}Quand les jours de festin étaient passés, Job envoyait chercher ses fils pour les sanctifier, et se levant de bon matin, il offrait un holocauste selon le nombre de ses enfants ; car Job disait : Peut-être mes fils ont-ils péché, et ont-ils blasphémé contre Dieu dans leurs cœurs. Job faisait toujours ainsi.\FTNT{Job 42:8.}
\VS{6}Or, il arriva un jour que les fils de Dieu\FTNT{Ps. 89:7 ; Job 38:7.} vinrent se présenter devant Yahweh, et Satan\FTNT{Es. 14:12; Ap. 12:9-10.} aussi vint au milieu d'eux.
\VS{7}Yahweh dit à Satan : D'où viens-tu ? Et Satan répondit à Yahweh : De courir çà et là sur la terre et de m'y promener\FTNT{1 Pi. 5:8.}.
\VS{8}Yahweh dit à Satan : N'as-tu point considéré mon serviteur Job, qui n'a point d'égal sur la terre ; homme intègre et droit, craignant Dieu, et se détournant du mal ?
\VS{9}Et Satan répondit à Yahweh : Est-ce en vain que Job craint Dieu ?
\VS{10}N'as-tu pas mis une haie \FTNT{La haie est une protection, une barrière végétale entretenue afin de protéger et de clôturer un terrain.} tout autour de lui, autour de sa maison, autour de tout ce qui lui appartient ? Tu as béni l'œuvre de ses mains, et ses troupeaux se répandent sur la terre.
\VS{11}Mais étends maintenant ta main, touche à tout ce qui lui appartient, et tu verras s'il ne te maudit pas en face.
\VS{12}Et Yahweh dit à Satan : Voilà, tout ce qui lui appartient est en ton pouvoir ; seulement ne porte pas la main sur lui. Et Satan sortit de devant la face de Yahweh\FTNT{1 R. 22:22.}.
\TextTitle{Première attaque de Satan}
\VS{13}Il arriva donc qu'un jour comme les fils et les filles de Job mangeaient et buvaient du vin dans la maison de leur frère aîné, un messager vint vers Job,
\VS{14}et lui dit : Les bœufs labouraient, et les ânesses paissaient à côté d'eux;
\VS{15}et ceux de Séba se sont jetés dessus, les ont pris, et ont frappé les serviteurs au fil de l'épée. Et je me suis échappé, moi seul, pour te l'annoncer.
\VS{16}Cet homme parlait encore, lorsqu'un autre vint et dit : Le feu de Dieu est tombé du ciel, il a brûlé les brebis et les serviteurs, et les a consumés\FTNT{2 R. 1:10-12.}. Et je me suis échappé moi seul, pour te l'annoncer.
\VS{17}Cet homme parlait encore, lorsqu'un autre vint et dit : Des Chaldéens\FTNT{Ge. 11:28.} ont fait trois bandes, se sont jetés sur les chameaux et les ont pris, ils ont frappé les serviteurs au fil de l'épée, et je me suis échappé moi seul, pour te l'annoncer.
\VS{18}Cet homme parlait encore, lorsqu'un autre vint et dit : Tes fils et tes filles mangeaient et buvaient du vin dans la maison de leur frère aîné ;
\VS{19}voici, un grand vent est venu de l'autre côté du désert et a frappé contre les quatre coins de la maison ; elle est tombée sur les jeunes gens, et ils sont morts. Et je me suis échappé, moi seul, pour te l'annoncer.
\VS{20}Alors Job se leva, déchira\FTNT{Job 2:12 ; Est. 4:1.} son manteau et  rasa la tête ; et se jetant par terre, se prosterna,
\VS{21}et dit : Je suis sorti nu du ventre de ma mère, et nu je retournerai dans le sein de la terre\FTNT{Ec. 5:14. ; 1 Ti. 6:7.} ; Yahweh a donné, Yahweh a enlevé\FTNT{1 S. 2:6.} ; que le nom de Yahweh soit béni !
\VS{22}En tout cela, Job ne pécha pas et n'attribua rien d'injuste à Dieu.
\Chap{2}
\TextTitle{Deuxième attaque de Satan}
\VerseOne{}Or il arriva un jour que les fils de Dieu vinrent un jour se présenter devant Yahweh, Satan\FTNT{Za. 3:1-2.} vint aussi au milieu d'eux se présenter devant Yahweh.
\VS{2}Yahweh dit à Satan : D'où viens-tu ? Satan répondit à Yahweh : De courir çà et là sur la terre et de m'y promener.
\VS{3}Yahweh dit à Satan: N'as-tu point considéré mon serviteur Job,qui n'a point d'égal sur la terre ; homme sincère et droit, craignant Dieu, et se détournant du mal ? Il demeure ferme dans son intégrité, quoique tu m'aies incité contre lui à le détruire sans cause\FTNT{Job 9:17.}.
\VS{4}Et Satan répondit à l’Eternel, en disant : Chacun donnera peau pour peau, et tout ce qu’il a, pour sa vie.
\VS{5}Mais étends maintenant ta main, et frappe à ses os et à sa chair\FTNT{Job 19:20.}, et tu verras s'il ne te maudit pas en face. 
\VS{6}Yahweh dit à Satan : Voici, il est en ta main : Seulement garde sa vie.
\VS{7}Ainsi Satan sortit de devant l’Eternel, et frappa Job d’un ulcère malin, depuis la plante de ses pieds jusqu’au sommet de la tête.
\VS{8}Job prit un tesson pour se gratter et s'assit au milieu de la cendre\FTNT{Jé. 6:26 ; Jon. 3:6.}.
\TextTitle{Réaction de Job et de sa femme}
\VS{9}Et sa femme lui dit : Conserveras-tu encore ton intégrité ?  Bénis\FTNT{Job 1:11.} Dieu, et meurs !
\VS{10}Et il lui dit : Tu parles comme une femme insensée ! Nous recevons le bien de la part de Dieu, et nous n'en recevrions pas le mal !\FTNT{Es. 45:7 ; Am. 3:6 ; La. 3:37.} En tout cela, Job ne pécha pas par ses lèvres.
\TextTitle{Job et ses trois amis}
\VS{11}Et trois des amis de Job, Eliphaz de Théman, Bildad de Schuach, et Tsophar de Naama, ayant appris tous les maux qui lui étaient arrivés, vinrent chacun du lieu de leur demeure, après s'être convenus ensemble d'un jour pour venir le plaindre et le consoler.
\VS{12}Ayant de loin levé les yeux sur lui, ils ne le reconnurent pas, alors ils élevèrent la voix et ils pleurèrent. Ils déchirèrent leurs manteaux, et jetèrent de la poussière vers le ciel au-dessus de leur tête.
\VS{13}Et ils s'assirent à terre avec lui, sept jours et sept nuits, et aucun d'eux ne lui dit une parole, car ils voyaient que sa douleur était fort grande.
\Chap{3}
\TextTitle{Lamentations de Job}
\VerseOne{}Après cela, Job ouvrit la bouche et maudit le jour de sa naissance.\FTNT{Jé. 20:14 ; Job 10:18.}
\VS{2}Car prenant la parole, il dit :
\VS{3}Périsse le jour où je suis né, et la nuit qui a dit : Un homme est conçu !
\VS{4}Que ce jour-là ne soit que ténèbres ; que Dieu ne le recherche point d'en haut, et qu'il ne soit point éclairé de la lumière ! 
\VS{5}Que les ténèbres et l'ombre de la mort\FTNT{Job 10:21-22.} s'en emparent, que les nuées demeurent sur lui, qu'il soit rendu terrible comme le jour de ceux à qui la vie est amère ! 
\VS{6}Que l'obscurité prenne cette nuit, qu'elle ne se réjouisse pas au milieu des jours de l'année, qu'elle n'entre pas dans le compte des mois !
\VS{7}Voici, que cette nuit soit stérile, et qu'aucun cri de joie n'y survienne !
\VS{8}Qu'ils la maudissent ceux qui maudissent les jours, ceux qui sont prêts à réveiller le Léviathan !
\VS{9}Que les étoiles de son crépuscule soient obscurcies ; qu'elle attende la lumière, mais qu'il n'y en ait point, et qu'elle ne voie point les rayons de l'aube du jour ! \FTNT{Job 41:9.} !
\VS{10}Parce qu'elle n'a pas fermé le sein qui me conçut ni caché la souffrance à mes yeux.
\VS{11}Pourquoi ne suis-je pas mort dans le sein de ma mère ? Pourquoi n'ai-je pas expiré aussitôt que je suis sorti de ses entrailles ?\FTNT{Job 10:18.}
\VS{12}Pourquoi des genoux m'ont-ils reçu? Pourquoi des mamelles m'ont-elles allaité ?
\VS{13}Je serais couché maintenant, je h serais tranquille, je dormirais, je me reposerais\FTNT{Job 17:16.},
\VS{14}avec les rois et les grands de la terre, qui se bâtirent des mausolées,
\VS{15}avec les princes qui possedèrent de l'or, et qui remplirent d'argent leurs maisons.
\VS{16}Ou comme l'avorton caché, je n'existerais pas\FTNT{Ps. 58:9.}, comme les petits enfants qui n'ont pas vu la lumière.
\VS{17}Là les méchants n'agitent plus personne, et là se reposent ceux qui sont fatigués. 
\VS{18}Pareillement ceux qui avaient été dans les liens, jouissent là du repos, et n'entendent plus la voix de l'oppresseur. 
\VS{19}le petit et le grand sont là, et l'esclave est délivré de son maître.
\VS{20}Pourquoi la lumière est-elle donnée au misérable, et la vie à ceux qui ont le cœur dans l'amertume ;
\VS{21}qui désirent en vain la mort, et qui la recherchent plus que le trésor,\FTNT{Ap. 9:6.}
\VS{22}qui seraient ravis de joie et seraient dans l'allégresse s'ils avaient trouvé le tombeau ?
\VS{23}Pourquoi, dis-je, la lumière est-elle donnée à l'homme à qui le chemin est caché, et que Dieu a enfermé de toutes parts\FTNT{Job 19:8 ; La. 3:7.} ?
\VS{24}Car avant que je mange, mon soupir vient, et mes cris se répandent comme de l'eau. 
\VS{25}Ce que je crains le plus, m'arrive, et ce que je redoute le plus, m'atteint. 
\VS{26}Je n'ai point eu de paix, je n'ai point eu de repos, ni de calme, depuis que ce trouble m'est arrivé. 
\Chap{4}
\TextTitle{Premier discours d'Eliphaz}
\VerseOne{}Alors Eliphaz de Théman prit la parole et dit :
\VS{2}Si l'on tente de te parler, en seras-tu peiné ? Mais qui pourrait retenir ses paroles ?
\VS{3}Voici, tu as souvent instruit les autres, et tu as fortifié les mains affaiblies\FTNT{Es. 35:3 ; Hé. 12:12.},
\VS{4}Tes paroles ont affermi ceux qui chancelaient, et tu as fortifié les genoux qui pliaient\FTNT{Job 16:5.}.
\VS{5}Et maintenant que le malheur t'arrive, tu faiblis ! Maintenant que tu es atteint, tu en es tout troublé !
\VS{6} Ta crainte de Yahweh n'a-t-elle pas été ton espérance ? Et l'intégrité de tes voies n'a-t-elle pas été ton attente ? 
\VS{7}Rappelle, je te prie, dans ton souvenir : Quel est l'innocent qui a péri ? Quels sont les justes qui ont été exterminés ?\FTNT{Job 8:20.}
\VS{8}Selon ce que j'ai vu, ceux qui labourent l'iniquité et qui sèment la peine en moissonnent les fruits ;\FTNT{Job 15:35 ; Ga. 6:7.}
\VS{9}ils périssent par le souffle de Dieu, et ils sont consumés par le vent de ses narines.\FTNT{Ex. 15:8 ; Es. 11:4 ; 30:33 ; Job 15:30 ; 2 Th. 2:8.}
\VS{10}Il étouffe le rugissement du lion, et le cri d'un grand lion, et il arrache les dents des lionceaux ;
\VS{11}le lion périt faute de proie, et les petits de la lionne sont dispersés.
\VS{12}Une parole m'est furtivement arrivée, et mon oreille en a saisi les sons légers.
\VS{13}Au moment où les visions de la nuit agitent la pensée, quand un profond sommeil tombe sur les hommes\FTNT{Job 33:15.},
\VS{14}une frayeur et un tremblement me saisirent, et tous mes os tremblèrent.
\VS{15}Un esprit passa devant moi, et mes cheveux en furent tout hérissés. 
\VS{16}Il se tint là et je ne reconnus pas son visage ; une figure était devant mes yeux. Et j'entendis un léger murmure et une voix :
\VS{17}L'homme serait-il juste devant Dieu ? L'homme serait-il pur devant celui qui l'a fait ?\FTNT{Job 25:4.}
\VS{18}Voici, il ne se fie pas à ses serviteurs, il trouve des erreurs à ses anges\FTNT{Job 15:15 ; Job 25:5 ; 2 Pi 2:4. }
\VS{19}combien plus chez ceux qui habitent des maisons d'argile, qui ont leurs fondements dans la poussière, qu'on écrase comme des vermisseaux !\FTNT{Job 25:6.}
\VS{20}Du matin au soir ils sont brisés, et, sans qu'on s'en aperçoive, ils périssent pour toujours. 
\VS{21}L'excellence qui était en eux, n'a-t-elle pas été emportée ? Ils meurent sans être sages. 
\Chap{5}
\VerseOne{}Crie maintenant ! Y aura-t-il quelqu'un qui te réponde ? Et vers quel saint te tourneras-tu ?\FTNT{Job 15:15.}
\VS{2}La colère tue l'insensé, et le fou meurt dans ses emportements.
\VS{3}J'ai vu l'insensé qui s'enracinait \FTNT{Jé. 12:1-2.}, mais j'ai aussitôt maudit sa demeure.
\VS{4}Ses fils sont loin de tout secours ; ils sont écrasés à la porte, et personne ne les délivre !\FTNT{Ps. 119:155.}
\VS{5} Sa moisson est dévorée par l'affamé, qui même la ravit d'entre les épines ; et le voleur convoite ses biens.
\VS{6}Le malheur ne sort pas de la poussière, et le travail ne germe pas de la terre ;
\VS{7}l'homme naît pour la peine\FTNT{Ge. 3:17-19 ; Job 14:1-5.}, comme l'étincelle pour voler et s'élever.
\VS{8}Mais moi, j'aurais recours à Dieu, et j'adresserais ma parole à Dieu.
\VS{9}Il fait de grandes choses qu'on ne peut sonder, de merveilleuses choses qu'on ne peut compter\FTNT{Ps. 72:18. Ps. 92:5 ; Job 9:10.}.
\VS{10}Il répand la pluie sur la face de la terre, et envoie les eaux sur les campagnes\FTNT{De. 28:12 ; Ps. 135:7 ; Job 28:26; Job 38:25-26 ; Ac. 14:17.};
\VS{11}il met en haut ceux qui sont abaissés, et délivre les affligés\FTNT{1 S. 2:7; Ez. 21:31 ; Ps. 113:7-8.} ;
\VS{12}il anéantit les projets des hommes rusés, de sorte qu'ils ne viennent pas à bout de leurs entreprises\FTNT{Es. 8:10 ; Ps. 33:10 ; Né. 4:15.} ;
\VS{13}il prend les sages dans leur propre ruse\FTNT{1 Co. 3:19.}, et les desseins des hommes pervers sont renversés :
\VS{14}De jour ils rencontrent les ténèbres, et ils marchent à tâtons en plein midi, comme dans la nuit.
\FTNT{De. 28:29.}.
\VS{15}Ainsi Dieu délivre le pauvre de l'épée de leur bouche, et le sauve de la main des puissants\FTNT{Ps. 12:3-4; Ps 52:2; Ps. 57:4.} ;
\VS{16}et l'espérance soutient le malheureux\FTNT{1S. 2:8.}, et la méchanceté a la bouche fermée\FTNT{Es. 52:15 ; Ps. 63: 11; Ps. 107: 42; Pr. 10:6.}.
\VS{17}Voici, heureux est celui que Yahweh châtie ! Ne rejette donc point le châtiment de Yahweh.
\FTNT{Ps 94:12 ; Pr.3:11-12 ; Hé. 12:5-6; Ap. 3:19.}.
\VS{18}Car c'est lui qui fait la plaie, et la bande ; il blesse et ses mains guérissent\FTNT{De. 32:39; 1S. 2: 6-7 ; Cp. Es. 30:26 ; Os. 6:1.}.
\VS{19}Six fois il te délivrera de l'angoisse, et sept fois le mal ne te touchera pas\FTNT{Ps 34:20; Ps. 91:3; Pr.24:16.}.
\VS{20}Il te sauvera de la mort pendant la famine, et du tranchant de l'épée pendant la guerre\FTNT{Ps. 33: 19; Ps. 37:19.}.
\VS{21}Tu seras à l'abri du fléau de la langue, et tu n’auras point peur de la dévastation, quand elle arrivera.
\FTNT{Ps. 31:21.}.
\VS{22}Tu riras de la dévastation et de la famine, et tu n'auras pas peur des bêtes de la terre\FTNT{Es. 65:25; Ez. 34:25; Os. 2:20.};
\VS{23}car tu feras une alliance avec les pierres des champs, et les bêtes des champs seront en paix avec toi\FTNT{Os. 2:20.}.
\VS{24}Tu jouiras en paix de la prospérité sous ta tente, tu pourvoiras à ta demeure et tu n'y seras point trompé ;
\VS{25}tu verras ta postérité s'accroître, et tes descendants se multiplier comme l'herbe de la terre\FTNT{Ps. 72:16; Ps. 127: 3-5; Ps. 128:6.}.
\VS{26}Tu entreras au tombeau dans ta vieillesse, comme une gerbe qu'on emporte en son temps\FTNT{Pr. 9:11; Pr. 10:27.}.
\VS{27}Voilà ce que nous avons examiné, voilà ce qui est ; à toi d'entendre et de choisir.
\Chap{6}
\TextTitle{Réponse de Job}
\VerseOne{}Job prit la parole et dit :
\VS{2}Oh ! si l’on pesait ma douleur, et si l’on mettait en même temps mes calamités dans la balance !
\VS{3}Car elle serait plus pesante que le sable de la mer ; c’est pourquoi mes paroles sont englouties.
\FTNT{Pr. 27:3.} !
\VS{4}Car les flèches du Tout-Puissant sont sur moi, mon âme en boit le venin ; les terreurs\FTNT{Job 30:15 ; Ps. 88:16-17.} de Dieu se rangent en bataille contre moi\FTNT{Job 19:12 ; Ps. 38:2-3.}.
\VS{5}L'âne sauvage\FTNT{Job 39:8.} brait-il auprès de l’herbe ? Le bœuf mugit-il auprès de son fourrage ?
\VS{6}Mange-t-on sans sel ce qui est fade ? Trouve-t-on du goût dans un blanc d’œuf ?
\VS{7}Ce que mon âme voudrait ne pas toucher, c'est là ma nourriture, si dégoûtante soit-elle !
\VS{8}Oh ! Puisse ma prière s'accomplir et Dieu me donner ce que j'attends !
\VS{9}Qu’il plaise à Dieu de me réduire en poussière, qu’il laisse aller sa main pour m’achever !
!\FTNT{Job 7:16; 9:21; 10:1; cp. No. 11:15; 1R. 19:4; Jon. 4:3, 8.}
\VS{10}Mais j’ai encore cette consolation, quoique la douleur me consume, et qu’elle ne m’épargne point, je n'ai pas transgressé les paroles du Saint.
\VS{11}Quelle est ma force pour que j’espère, et quelle est ma fin pour que je prenne patience ?
\VS{12}Ma force est-elle une force de pierre ? Ma chair est-elle d'airain ?
\VS{13}Ne suis-je pas sans secours, et le salut n'est-il pas loin de moi ?
\VS{14}A celui qui souffre, est due la compassion de son ami ; mais il a abandonné la crainte. \FTNT{Ps. 19:10.} du Tout-Puissant\FTNT{Pr. 17:17.}.
\VS{15}Mes frères m'ont trompé comme un torrent, comme le lit des torrents qui passent\FTNT{Ps. 38:12; Ps 41:10; Ps 69:9; Jé. 15:19.}.
\VS{16}Les glaçons en troublent le cours, la neige s'y cache ;
\VS{17}mais au temps de la sécheresse, ils tarissent, et dans les chaleurs, ils disparaissent de leur place.
\VS{18}Les caravanes se détournent de leur route, elles montent dans le désert et périssent.
\VS{19}Les caravanes de Théma\FTNT{Ge. 25:15.} fixent le regard, les voyageurs de Séba\FTNT{1R. 10:1; Ps. 72:10; Ez. 27:22-23.} s'attendent à eux;
\VS{20}ils sont honteux d'avoir eu cette confiance, ils restent confondus quand ils arrivent.
\VS{21}Certes, vous m'êtes devenus inutiles ; vous voyez mon angoisse, et vous en avez horreur !\FTNT{Job 19:13 ; Ps. 31:12.}
\VS{22}Mais vous ai-je dit : Donnez-moi quelque chose, et de vos biens, faites des présents en ma faveur ? 
\VS{23}délivrez-moi de la main de l'ennemi, et rachetez-moi de la main des violents ?
\VS{24}Instruisez-moi, et je me tairai ; faites-moi comprendre en quoi je me suis égaré.
\VS{25}Ô combien sont fortes les paroles de vérité ! Mais que veut censurer votre argumentation ?
\VS{26}Voulez-vous donc blâmer ce que j'ai dit, et ne voir que du vent dans les paroles d'un homme désespéré ?\FTNT{Ec. 9:16.}
\VS{27}Vous vous jetez même sur un orphelin, vous persécutez votre ami.
\VS{28}Regardez-moi, je vous prie ! Et voyez si je vous mens en face ?
\VS{29}Revenez\FTNT{Job 17:10.} donc, soyez sans injustice; revenez, et reconnaissez mon innocence\FTNT{Job 27:5-6 ; 34:5 ; cp. Job 23:10 ; 42:1-6.}.
\VS{30}Y a-t-il de l'injustice dans ma langue ? et mon palais ne sait-il pas discerner le mal ? 
\Chap{7}
\VerseOne{}N'y a-t-il pas un temps de guerre limité à l'homme sur la terre ? Et ses jours ne sont-ils pas comme les jours d'un mercenaire ?
\VS{2}Comme un esclave, il soupire après l'ombre, comme un mercenaire\FTNT{Es. 16:14.}, il attend son salaire\FTNT{Ps. 39:5.}.
\VS{3}Ainsi j'ai reçu en partage des mois en vain, et l'on m'a assigné des nuits de peine\FTNT{Ps. 6:6.}.
\VS{4}Si je suis couché, je dis : Quand me lèverai-je ? Quand finira la nuit ? Et je suis rassasié d'agitations jusqu'au point du jour\FTNT{De. 28:67.}.
\VS{5}Ma chair se couvre de vers et d'une croûte terreuse, ma peau se crevasse et coule.
\VS{6}Mes jours sont plus rapides que la navette du tisserand, ils se consument sans espoir !\FTNT{Es. 38:12 ; Job 9:25 ; 17:11; Ja. 4:14.}
\VS{7}Souviens-toi que ma vie est un souffle ! Et que mes yeux ne reverront plus le bonheur\FTNT{Es. 40:6 ; Ps. 78:39 ; Ps. 89:48 ; Ps. 102: 12 ; Ps. 103:15 ; Job 8:9 ; Job 14:1-2 ; 1P. 1:24.}.
\VS{8}L'œil de ceux qui me regarndent ne me verra plus ; tes yeux seront sur moi, et je ne serai plus.
\VS{9}La nuée se dissipe et s'en va, ainsi celui qui descend au scheol\FTNT{cp. Ha. 2:5 ; Lu. 16:23.} ne remontera pas\FTNT{Job 10:21-22 ; Job 14:7-14.};
\VS{10}il ne reviendra plus dans sa maison, et le lieu qu'il habitait ne le reconnaîtra plus\FTNT{Ps. 37:35-36 ; Ps. 103:16 ; Job 10:21.}.
\VS{11}C'est pourquoi, je ne retiendrai pas ma bouche, je parlerai dans l'angoisse de mon esprit, je me plaindrai dans l'amertume de mon âme\FTNT{Job 10:1.}.
\VS{12}Suis-je une mer ? Suis-je un monstre marin, pour que tu poses autour de moi des gardes ?
\VS{13}Quand je dis : Mon lit me consolera, ma couche calmera ma plainte,
\VS{14}alors tu me terrifies par des songes, et tu m'épouvantes par des visions.
\VS{15}C'est pourquoi je choisirais d'être étranglé, et de mourir, plutôt que de conserver mes os.
\VS{16}Je les méprise !… Je ne vivrai pas toujours… Laisse-moi car mes jours sont un souffle\FTNT{Job 10:20.}.
\VS{17}Qu'est-ce que l'homme pour que tu en fasses tant de cas, pour que tu poses ta main sur son cœur,\FTNT{Ps. 8:5 ; Ps. 144:3 ; Hé. 2:6.}
\VS{18}pour que tu le visites tous les matins, pour que tu l'éprouves\FTNT{Job 23:10.} à chaque instant ?
\VS{19}Quand finiras-tu de me regarder? Ne me lâcheras-tu pas, pour que j'avale ma salive ?\FTNT{Job 9:18.}
\VS{20}J'ai péché ; que te ferai-je, gardien des hommes ?  \FTNT{1 Ti. 4:10.}  Pourquoi m'as-tu mis en butte à tes coups, et pourquoi suis-je à charge à moi-même ?
\VS{21}Et pourquoi ne pardonnes-tu pas mon péché, et ne fais-tu pas passer mon iniquité ? Car je vais maintenant me coucher dans la poussière ; tu me chercheras, et je ne serai plus.
\Chap{8}
\TextTitle{Premier discours de Bildad}
\VerseOne{}Bildad de Schuach prit la parole et dit :
\VS{2}Jusqu'à quand parleras-tu ainsi, et les paroles de ta bouche seront-elles un vent impétueux ?\FTNT{Job 15:2.}
\VS{3}Dieu renverserait-il le droit, et le Tout-puissant renverserait-il la justice ? \FTNT{Cp. Ge. 18:25.} ?\FTNT{De. 32:4 ; Job 34:12 ; Da. 9:14 ; 2 Ch. 19:7.}
\VS{4}Si tes fils ont péché contre lui, il les a livrés à leur crime.
\VS{5}Mais toi si tu cherches Dieu, si tu demandes grâce au Tout-Puissant ;\FTNT{Cp. Job 5:17-27.}
\VS{6}si tu es pur et droit, il veillera certainement sur toi, il rendra le bonheur à la demeure de ta justice ;
\VS{7}tes commencements\FTNT{Za. 4:10.} auront été peu de chose, et ta fin sera bien plus grande.\FTNT{Job 42:12.}
\VS{8}Interroge ceux des générations précédentes, applique-toi à l'expérience de leurs pères.\FTNT{De. 4:32 ; De. 32:7.}
\VS{9}Car nous sommes d'hier, et nous ne savons rien, parce que nos jours sur la terre ne sont qu'une ombre.\FTNT{Ps. 102:12 ; Ps. 144:41 ; Ch. 29:15.}
\VS{10}Ils t'instruiront, ils te parleront, ils tireront de leur cœur ces discours :
\VS{11}Le roseau croît-il sans marais ? Le jonc pousse-t-il sans eau ?
\VS{12}Il est encore en sa verdure, sans qu'on le coupe, il sèche plus vite que toutes les herbes.\FTNT{Cp. Jé. 17:5-8 ; Ps. 129:6.}
\VS{13}Ainsi est la voie de tous ceux qui oublient Dieu\FTNT{Ps. 9:18.}, et l'espérance de l'impie périra\FTNT{Ps. 1:4 ; Ps. 112:10 ; Pr. 10:28 ; Job 11:20 ; Job 27:8.}.
\VS{14}Sa confiance est brisée, son soutien est une toile d'araignée.
\VS{15}Il s'appuie sur sa maison, et elle ne tient pas ; il s'y cramponne, et elle ne reste pas debout.
\VS{16}Dans toute sa vigueur, en plein soleil, il étend ses rameaux sur son jardin,
\VS{17}mais ses racines s'entrelacent parmi des monceaux de pierres, il pénètre dans les rochers.
\VS{18}S'Il l'ôte de sa place, celle-ci le renie, disant je ne t'ai pas connu ! 
\VS{19}Telle est la joie que ses voies lui procurent. Puis sur le même sol, d'autres s'élèvent après lui.
\VS{20}Dieu ne rejette pas l'homme intègre, il ne soutient pas la main des méchants.\FTNT{Job 4:7.}
\VS{21}Il remplira encore ta bouche de cris de joie, et tes lèvres de chants d'allégresse.\FTNT{Ps. 126:2.}
\VS{22}Ceux qui te haïssent seront revêtus de honte, et la tente des méchants ne sera plus. \FTNT{Ps. 35:26 ; Ps. 109:29.}
\Chap{9}
\TextTitle{Réponse de Job}
\VerseOne{}Job prit la parole et dit :
\VS{2}Certainement, je sais qu'il en est ainsi ; et comment l'homme mortel se justifierait-il devant Dieu ? \FTNT{Ha. 2:4 ; Ga. 3:11 ; Ro. 1:17 ; Hé. 10:38.} devant Dieu ?\FTNT{Ps. 25:4 ; Ps. 143:2 ; Job 15:14-16 ; Da. 9:11 ; Ro 3:19.}
\VS{3}S'il veut plaider avec lui, il ne lui répondra pas une fois sur mille. \FTNT{Es. 45:9-10.}
\VS{4} Dieu est sage de coeur, et puissant en force. Qui est-ce qui s'est opposé à lui, et s'en est bien trouvé ? \FTNT{Job 12:13 ; Job 36:5 ; Job 37:23.}
\VS{5}Il transporte les montagnes, et quand il les renverse dans sa fureur, elles n'en connaissent rien.\FTNT{Ps. 144:5.}
\VS{6}Il remue la terre de sa place, et ses piliers sont ébranlés.\FTNT{Ag. 2:6, 21 ; Hé. 12:26.}
\VS{7}Il commande au soleil, et le soleil ne se lève pas ; et il met un sceau sur les étoiles.\FTNT{Jos. 10:12.}
\VS{8} C'est lui seul qui étend les cieux\FTNT{Ge 1:6-8 ; Es. 44:24; Es. 51:13 ; Ps. 104:2.},qui marche sur les hauteurs de la mer\FTNT{Cp. Mt. 14:25.}.
\VS{9}Il a fait la grande ourse, l'orion, les pléiades, et les étoiles des régions australes.\FTNT{Ge. 1:16 ; Am. 5:8 ; Ps. 89:12 ; Job 38:31-32.}
\VS{10}Il fait de grandes choses qu'on ne peut sonder, des merveilles sans nombre.\FTNT{Ps. 86:10 ; Ps. 139:6, 17-18 ; Job 5:9 ; Job 37:5.}
\VS{11}Voici, il passe près de moi, et je ne le vois pas ; il passe encore, et je ne l'aperçois pas.\FTNT{Job 23:8-9 ; 35:14.}
\VS{12}S'il enlève, qui l'en détournera? Qui lui dira : Que fais-tu ?\FTNT{Es. 45: 9-10 ; Da. 4:35 ; Ro. 11:33-35.}
\VS{13}Dieu ne revient pas sur sa colère ; sous lui s'inclinent les appuis de l'orgueil.\FTNT{Job 26:12; Cp. Es. 30:7.}
\VS{14}Combien moins lui répondrais-je, moi et comment choisirais-je mes paroles contre lui ? 
\VS{15}Quand je serais juste, je ne répondrais pas ; je demanderais grâce à mon juge.\FTNT{Job 23:1-7.}
\VS{16}Si je l'invoque et qu'il me réponde, ne croirais-je pas qu'il ait écouté ma voix,
\VS{17}lui qui m'assaille comme par une tempête, qui multiplie mes plaies sans motif,\FTNT{Job 6:29.}
\VS{18}qui ne me permet pas de reprendre haleine ; qui me rassasie d'amertume.\FTNT{Job 7:19.}
\VS{19}S'il est question de savoir qui est le plus fort ; voilà, il est fort ; et s'il est question d'aller en justice, qui est-ce qui m'y fera comparaître ? 
\VS{20}Si je me justifie, ma propre bouche me condamnera ; si je me fais parfait, il me convaincra d'être coupable.
\VS{21}Je suis innocent ! Je ne me soucie pas de vivre, je méprise ma vie.\FTNT{Job 10:1.}
\VS{22}Tout se vaut! C'est pourquoi j'ai dit: Il détruit l'innocent comme l'impie. \FTNT{ Cp. Ez. 21:3 ; Ec. 9:2-3 ; Mt 5:45.}
\VS{23}Au moins si le fléau dont il frappe faisait mourir tout aussitôt ; mais il se rit de l'épreuve des innocents. 
\VS{24}[C'est par lui que] la terre est livrée entre les mains du méchant ; c'est lui qui couvre la face des juges de la [terre] ; et si ce n'est pas lui, qui est-ce donc ? 
\VS{25}Or mes jours vont plus vite qu'un courrier ; ils s'en fuient sans avoir vu le bonheur ;\FTNT{Job 7:6-7.}
\VS{26}ils passent comme les navires de roseaux, comme l'aigle qui fond sur sa proie.
\VS{27}i je dis : J'oublierai ma plainte, je renoncerai à ma colère, je me fortifierai; 
\VS{28}Je suis épouvanté de tous mes tourments. Je sais que tu ne me jugeras pas innocent. .\FTNT{Cp. Ps. 130:3.}
\VS{29}Je serai jugé coupable ; pourquoi travaillerais-je en vain ?
\VS{30}Quand je me laverais dans de l'eau de neige, et que je nettoierais mes mains dans la pureté, \FTNT{Jé. 2:22.}
\VS{31}tu me plongerais dans le fossé, et mes vêtements m'auraient en horreur.
\VS{32}Car il n'est pas comme moi un homme, pour que je lui réponde, [et] que nous allions ensemble en jugement. .\FTNT{Es. 45:9 ; Jé 49:19 ; Ec. 6:10; Ro. 9:20.}
\VS{33}Mais il n'y a personne qui prend connaissance de la cause qui serait entre nous, et qui pose la main sur nous deux. \FTNT{Cp. 1 S. 2:25.}
\VS{34}Qu'il ôte donc sa verge de dessus moi, et que la frayeur que j'ai de lui ne me trouble plus. ;
\VS{35}Je parlerai, et je ne le craindrai pas ; mais dans l'état où je suis je ne suis plus à moi-même. 
\Chap{10}
\VerseOne{}Mon âme a pris en dégoût la vie ! Je laisserai aller ma plainte, je parlerai dans l'amertume de mon âme.
\VS{2}Je dirai à Dieu : Ne me condamne pas ; montre-moi pourquoi tu plaides contre moi ?
\VS{3}Te plais-tu à m'opprimer, et à dédaigner l'ouvrage de tes mains, et à bénir les desseins des méchants\FTNT{Es. 64:7-8.} ?
\VS{4}As-tu des yeux de chair ? Vois-tu comme voit un homme mortel?
\VS{5}Tes jours sont-ils comme les jours de l'homme mortel ? Tes années sont-elles comme les jours de l'homme, 
\VS{6}Que tu recherches mon iniquité, et que tu t'informes de mon péché,
\VS{7}tu sais que je n'ai point commis de crime, et qu'il n'y a personne qui me délivre de ta main?
\VS{8}Tes mains m'ont formé, et elles ont rangé toutes les parties de mon corps; et tu me détruirais !\FTNT{Ge. 2:7 ; Ps. 119:73 ; Ps. 139:14-15.} !
\VS{9}Souviens-toi, je te prie, que tu m'as formé comme de la boue, et que tu me feras retourner en poudre?
\VS{10}Ne m'as-tu pas coulé comme du lait ? et ne m'as-tu pas fait cailler comme un fromage ?
\VS{11}Tu m'as revêtu de peau et de chair, et tu m'as composé d'os et de nerfs;
\VS{12}Tu m'as donné la vie, et tu as usé de miséricorde envers moi, et [par] tes soins continuels tu as gardé mon esprit.
\VS{13}Et cependant tu gardais ces choses en ton cœur ; mais je connais que cela était devant toi. 
\VS{14}Si j'ai pèche, tu m'observes, et tu ne me tiens pas pour innocent de mon iniquité.
\VS{15}Si j'agis méchamment, malheur à moi ! si je suis juste, je n'en lève pas la tête plus haut. Je suis rempli d'ignominie ; mais regarde mon affliction. 
\VS{16}Si je redresse la tête, tu me poursuis comme à un lion, et tu multiplie tes exploits contre moi; \FTNT{Zq. 38:13 ; La. 3:10.}.
\VS{17}Tu renouvelles tes témoins contre moi, et ton indignation augmente contre moi. De nouvelles troupes toutes fraîches viennent contre moi.
\VS{18}Mais pourquoi m'as-tu fait sortir du sein de ma mère? J'aurais expiré, et aucun œil ne m'aurait vu ;
\VS{19}et j'aurais été comme n'ayant jamais été, et j'aurais été porté du ventre de ma mère au tombeau.
\VS{20}Mes jours ne sont-ils pas en petit nombre ? Cesse donc et retire-toi de moi, et que je me renforce un peu. .
\VS{21}Avant que j'aille au lieu d'où je ne reviendrai plus, en la terre de ténèbres et de l'ombre de la mort,
\VS{22}terre d'une grande obscurité, comme les ténèbres de l'ombre de la mort, où il n'y a aucun ordre, et où rien ne luit que des ténèbres. 
\Chap{11}
\TextTitle{Première accusation de Tsophar}
\VerseOne{}Tsophar de Naama prit la parole et dit :
\VS{2}Ne répondra-t-on point à tant de discours, et suffira-t-il d'être un grand parleur pour être justifié ?
\VS{3}Tes vains feront-ils taire les gens ? Et quand tu te seras moqué, n'y aura-t-il personne qui te fasse honte ?
\VS{4}Car tu as dit : Ma doctrine est pure, et je suis sans tache devant tes yeux. 
\VS{5}Mais je voudrais que Dieu parle, et qu'il ouvre sa bouche pour te répondre; ,
\VS{6}Qu'il te montre les secrets de sa sagesse, de son immense sagesse; et que tu reconnaisse que Dieu oublie une partie de ton iniquité. .
\VS{7}Trouveras-tu Dieu en le sondant ? Connaîtras-tu parfaitement le Tout-puissant ? 
\VS{8}Ce sont les hauteurs des cieux : Qu'y feras-tu ? C'est plus profond que le scheol : Qu'y connaîtras-tu ?
\VS{9}Son étendue est plus longue que la terre, et plus large que la mer.
\VS{10}S'il remue, et qu'il resserre, ou qu'il rassemble, qui l'en détournera ?
\VS{11}Car il connaît les hommes vicieux, il discerne par le regard les coupables\FTNT{Ps. 10:11-14 ; Ps. 35:22.}.
\VS{12}Mais l'homme vide de sens devient intelligent, quoique l'homme naisse comme un ânon sauvage\FTNT{Ec. 3:18.}.
\VS{13}Si tu disposes ton cœur, et que tu étendes tes mains vers lui,
\VS{14}Si tu éloignes de toi l'iniquité qui est en ta main, et si tu ne permets pas que la méchanceté habite dans tes tentes ; 
\VS{15}Alors certainement tu pourras élever ton visage sans tache ; tu seras ferme et tu ne craindras rien;
\VS{16}tu oublieras tes peines, tu t'en souviendras comme des eaux écoulées.
\VS{17}La vie se lèvera pour toi plus brillante que le midi, et l'obscurité même sera comme le matin\FTNT{Ps. 37:6 ; Ps. 112:4.}.
\VS{18} Tu seras plein de confiance, parce qu'il y aura de l'espérance pour toi; tu creuseras, et tu reposeras sûrement. \FTNT{Lé. 26:6 ; Ps. 3:6 ; Pr. 3:24.}.
\VS{19}Tu te coucheras, et il n'y aura personne qui t'épouvante, et plusieurs te feront la cour. 
\VS{20}Mais les yeux des méchants seront consumés; tout refuge leur sera ôté et toute leur espérance sera de rendre l'âme !
\Chap{12}
\TextTitle{Réplique de Job}
\VerseOne{}Job reprit la parole, et dit :
\VS{2}On dirait vraiment que vous êtes tout un peuple, et qu'avec vous doit mourir la sagesse.
\VS{3}J'ai du bon sens aussi bien que vous, et je ne vous suis point inférieur ; et qui ne sait de telles choses ?
\VS{4}Je suis pour mes amis un objet de raillerie, quand je m'écrie à Dieu pour qu'il me réponde; on se moque d'un homme qui est juste et droit.
\VS{5} Mépris au malheur! telle est la pensée des heureux; le mépris est réservé à ceux dont le pied chancelle !
\VS{6}Elles sont en paix, les tentes des pillards, et toutes les sécurités sont pour ceux qui irritent Dieu, qui se font un dieu de leur bras. \FTNT{ Jé. 12:1 ; Ps. 73:12.}.
\VS{7}Mais interroge donc les bêtes, et elles t'instruiront, ou les oiseaux des cieux, et ils te l'annonceront ;
\VS{8}Ou parle à la terre, et elle t'enseignera ; même les poissons de la mer te le raconteront ; 
\VS{9}Qui est-ce qui ne sait toutes ces choses; que c'est la main de Yahweh qui a fait cela ?
\VS{10} Qu'il tient en sa main, l'âme de tout ce qui vit, et l'esprit de toute chair humaine,
\VS{11}L'oreille ne discerne-t-elle pas les discours, ainsi que le palais savoure les aliments ?
\VS{12}La sagesse est dans les vieillards, et l'intelligence est le fruit d'une longue vie.
\VS{13}Mais en Dieu est la sagesse et la force ; à lui appartient le conseil et l'intelligence. \FTNT{Da. 2:20.}.
\VS{14}Voici, il démolit, et on ne rebâtit pas ; il enferme un homme, et on ne lui ouvre pas\FTNT{Es. 22:22 ; Ap. 3:7.}.
\VS{15}Voilà, il retient les eaux, et tout devient sec ; il les lâche, et elles bouleversent la terre.
\VS{16} En lui résident la puissance et la sagesse; de lui dépendent celui qui s'égare et celui qui égare.
\VS{17} Il emmène dépouillés les conseillers, et il met hors de sens les juges. \FTNT{2 S. 15:31 ; 2 S. 17:14-23 ; Es. 19:12 ; Es. 29:14 ; 1 Co. 1:19.}.
\VS{18}Il rend impuissant le gouvernement des rois, et lie de chaînes leurs reins. 
\VS{19}Il fait marcher pieds nus les sacrificateurs ; et il renverse les puissants.
\VS{20}Il ôte la parole à ceux qui sont les plus assurés en leurs discours, et il prive de sens les anciens.
\VS{21}Il verse le mépris sur les nobles ; il relâche la ceinture des forts\FTNT{Es. 40:23.}.
\VS{22}Il met en évidence les choses qui étaient cachées dans les ténèbres, et il produit en lumière l'ombre de la mort. \FTNT{Ps. 139:11-12 ; Ec. 12:16 ; Mt. 10:26 ; 1 Co. 4:6.}.
\VS{23} Il multiplie les nations, et les fait périr ; il répand çà et là les nations, et puis il les ramène. 
\VS{24}Il ôte la raison aux Chefs des peuples de la terre, et les fait errer dans les déserts où il n'y a point de chemin;
\VS{25}ils tâtonnent dans les ténèbres, sans aucune clarté, et il les fait chanceler comme des gens ivres. 
\Chap{13}
\VerseOne{}Voici, mon œil a vu toutes ces choses, mon oreille l'a entendu et compris.
\VS{2}Comme vous les savez, je les sais aussi ; je ne vous suis pas inférieur. 
\VS{3}Mais je veux parler au Tout-Puissant, je veux plaider auprès de Dieu. .
\VS{4}Et certes vous inventez des mensonges ; vous êtes tous des médecins inutiles.
\VS{5}Plaît à Dieu que vous demeuriez entièrement dans le silence ; et cela vous sera réputé à sagesse. \FTNT{Pr. 17:28.}.
\VS{6}Ecoutez donc maintenant ma cause, et soyez attentifs à la défense de mes lèvres.
\VS{7}Tiendrez-vous des discours injustes en faveur de Dieu, et, pour le défendre, direz-vous des mensonges?
\VS{8}Ferez-vous acception de personnes en sa faveur? Prétendrez-vous plaider pour Dieu? 
\VS{9}S'il vous sonde, vous trouvera-t-il bon ? Comme on trompe un homme, le tromperez-vous? 
\VS{10}Certainement il vous reprendra, si même en secret vous faites acception de personnes.
\VS{11}Sa majesté ne vous épouvantera-t-elle pas ? Et sa frayeur ne tombera-t-elle pas sur vous ? 
\VS{12}Vos discours mémorables sont des sentences de cendre, et vos éminences sont des éminences de boue. 
\VS{13}Taisez-vous devant moi, et que je parle ; et il m'arrivera ce qui pourra. 
\VS{14}Pourquoi porterais-je ma chair entre mes dents, et tiendrais-je mon âme entre mes mains ? \FTNT{Jg. 12:3 ; 1 S. 19:5.}.
\VS{15}Voilà, qu'il me tue, je ne cesserai pas d'espérer en lui ; et je défendrai ma conduite en sa présence.
\VS{16}Et qui plus est, il sera lui-même mon salut ; mais l'hypocrite ne viendra point devant sa face. \FTNT{Ps. 1:5.}.
\VS{17}Ecoutez attentivement mes paroles, et prêtez l'oreille à ce que je vais vous déclarer. 
\VS{18}Voici, j'ai préparé ma cause. Je sais que je serai justifié.
\VS{19}Qui est-ce qui veut disputer contre moi ? car maintenant si je me tais, je mourrai. 
\VS{20}Seulement ne me fais pas ces deux choses, et alors je ne me cacherai point devant ta face :
\VS{21}Retire ta main de dessus moi, et que tes terreurs ne me troublent pas.
\VS{22}Puis appelle-moi, et je répondrai ; ou bien je parlerai, et tu me répondras. 
\VS{23}Combien ai-je d'iniquités et de péchés ? Montre-moi mon crime et mon péché. 
\VS{24}Pourquoi caches-tu ta face, et me tiens-tu pour ton ennemi ?
\VS{25}Déploieras-tu tes forces contre une feuille que le vent emporte ? Poursuivras-tu du chaume tout sec \FTNT{1 S. 24:15.} ?
\VS{26}Que tu écrives contre moi des choses amères, et que tu me fasses porter la peine des péchés de ma jeunesse? \FTNT{Ps. 25:7.} ?
\VS{27}Que tu mettes mes pieds aux ceps, et observes tous mes chemins, et que tu suives les traces de mes pieds,
\VS{28}Quand mon corps s'en va par pièces comme du bois vermoulu, et comme une robe que la teigne a rongée? 
\Chap{14}
\VerseOne{}L'homme né de la femme est de courte vie, et rassasié d'agitations. \FTNT{Ps. 102:12 ; Ps. 103:15 ; Ps. 144:4 ; Ja. 4:14.}.
\VS{2}Il sort comme une fleur, puis il est coupé, et il s'enfuit comme une ombre qui ne s'arrête pas. \FTNT{Es. 40:6 ; Ps. 90:61 ; 1 Pi. 1:24.}.
\VS{3}Cependant tu as ouvert tes yeux sur lui, et tu me conduis en justice avec toi.
\VS{4}Qui est-ce qui tirera le pur de l'impur ? Personne. \FTNT{Es. 48:8 ; Pr. 22:15.}.
\VS{5}Les jours de l'homme sont déterminés, le nombre de ses mois est entre tes mains, tu lui as prescrit ses limites, et il ne passera point au delà.
\VS{6}Retire-toi de lui, afin qu'il ait du relâche, jusqu'à ce que comme un mercenaire il ait achevé sa journée.
\VS{7}Car si un arbre est coupé, il y a de l'espérance, et il poussera encore, et ne manquera pas de rejetons ; 
\VS{8}quoique sa racine ait vieilli dans la terre, et que son tronc soit mort dans la poussière;
\VS{9}Dès qu'il sent l'eau il regerme, et produit des branches, comme un arbre nouvellement planté. 
\VS{10}Mais l'homme meurt et perd toute sa force; il expire et puis où est-il ?
\VS{11}Les eaux s'écoulent de la mer, et une rivière s'assèche, et tarit ;
\VS{12} Ainsi l'homme est couché par terre, et ne se relève plus ; jusqu'à ce qu'il n'y ait plus de cieux, ils ne se réveillera plus, et ne sera pas réveillé de son sommeil. 
\VS{13}Oh que tu me caches dans le scheol, que tu me gardes à l'abri jusqu'à ce que ta colère soit passée, que tu me donnes un temps arrêté, après lequel tu te souviendrais de moi !
\VS{14}Si un homme meurt, revivra-t-il ? Tous les jours de ma détresse, j'attendrais jusqu'à ce que mon état vînt à changer.
\VS{15} Tu appellerais, et moi je te répondrais, tu ne dédaignerais pas l'ouvrage de tes mains.
\VS{16}Mais maintenant tu comptes mes pas, et tu n'exceptes rien de mon péché. \FTNT{Ps. 56:9 ; Ps. 139:2-4 ; Pr. 5:21.} ;
\VS{17}Mes péchés sont scellés dans un sac, et tu as cousu ensemble mes iniquités. \FTNT{Os. 13:12.}.
\VS{18}Car comme une montagne s'éboule en tombant, et comme un rocher est transporté de sa place ; 
\VS{19} et comme les eaux minent les pierres, et entraînent par leur débordement la poussière de la terre, avec tout ce qu'elle a produit, tu fais ainsi périr l'attente de l'homme. 
\VS{20}Tu te montres toujours plus fort que lui, et il s'en va, et lui ayant défiguré le visage, tu le renvoies.
\VS{21}Quand ses fils sont honorés, il n'en sait rien ; et quand ils sont abaissés, il ne s'en aperçoit pas.
\VS{22}Seulement sa chair sur lui, a de la douleur, et son âme en lui s'afflige. 
\Chap{15}
\TextTitle{Deuxième discours d'Eliphaz}
\VerseOne{}Eliphaz de Théman prit la parole et dit :
\VS{2}Un homme sage profère-t-il dans ses réponses une science aussi légère que le vent, des opinions vaines ? Remplit-il son ventre du vent d'orient ?
\VS{3}Contestant avec des discours qui ne servent de rien, et avec des paroles dont on ne peut tirer aucun profit ?
\VS{4}Certainement tu abolis la crainte de Dieu, et tu anéantis peu à peu la prière qu'on doit présenter à Dieu. 
\VS{5} Car ta bouche fait connaître ton iniquité, et tu as choisi un langage trompeur. 
\VS{6}C'est ta bouche qui te condamne, et non pas moi ; et tes lèvres témoignent contre toi. 
\VS{7}Es-tu le premier homme né ? Ou as-tu été formé avant les montagnes ? \FTNT{Ps. 90:2 ; Pr. 8:25.} ?
\VS{8}As-tu été instruit dans le conseil secret de Dieu, et renfermes-tu seul la sagesse ? \FTNT{Es. 40:13 ; Jé. 23:18 ; Ro. 11:34.} ?
\VS{9}Que sais-tu que nous ne sachions pas ? Quelle connaissance as-tu que nous n'ayons pas ?
\VS{10}Parmi nous, il y a des hommes à cheveux blancs, et des gens d'une fort grande vieillesse, il y en a même de plus âgés que ton père. 
\VS{11}Les consolations du Dieu te semblent-elles trop petites ? As-tu quelque chose de caché par-devant toi ? …
\VS{12}Pourquoi ton coeur s'emporte-il et pourquoi tes yeux clignent-ils ?
\VS{13}C'est contre Dieu que tu tournes ta colère, et que tu fais sortir de ta bouche de tels discours! 
\VS{14}Qu'est-ce que de l'homme, pour qu'il soit pur, et celui qui est né de femme, pour qu'il soit juste ? \FTNT{Ps. 14:3 ; Pr. 20:9 ; Ec. 7:20.} ?
\VS{15}Si Voici, Dieu ne se fie pas à ses saints, et les cieux ne sont pas purs à ses yeux,
\VS{16}Combien plus est abominable et corrompu, l'homme qui boit l'iniquité comme l'eau !  
\VS{17}Je t'enseignerai, écoute-moi, et je te raconterai ce que j'ai vu ,
\VS{18} savoir ce que les sages ont déclaré, et qu'ils n'ont point caché ; ce qu'ils avaient [reçu] de leurs pères.
\VS{19}Eux à qui seuls la terre a été donnée, et parmi lesquels l'étranger n'est point passé.
\VS{20}Toute sa vie, le méchant est tourmenté, et un petit nombre d'années sont réservées au malfaiteur.\FTNT{Es. 48:22 ; Es. 57:21.}.
\VS{21}Un cri de frayeur est dans ses oreilles ; au milieu de la paix [il croit] que le destructeur se jette sur lui. \FTNT{1 Th. 5:3.} ;
\VS{22}Il ne croit pas pouvoir sortir des ténèbres, car il voit la menace de   l’épée;
\VS{23}il court çà et là pour chercher son pain, il sait que le jour des ténèbres lui est préparé \FTNT{Ps. 109:10.}.
\VS{24}La détresse et l'angoisse l'épouvantent, elles l'assaillent comme un roi prêt à combattre ;
\VS{25}Parce qu'il a élevé sa main contre Dieu, et qu'il s'est levé contre le Tout-puissant ;
\VS{26}Il lui a sauté au collet, et sur l'épaisseur de ses gros boucliers. 
\VS{27}Parce que la graisse a couvert son visage, et qu'elle a fait des replis sur son ventre;
\VS{28} il habite des villes détruites, des maisons désertes, tout près de n'être plus que des monceaux de pierres. 
\VS{29}Et il ne s'enrichira plus, car ses biens ne subsisteront pas, et ses richesses ne se répandront pas sur la terre. 
\VS{30}Il ne pourra pas se détourner des ténèbres, la flamme desséchera ses rejetons, et Dieu le fera disparaître par le souffle de sa bouche.
\VS{31}S'il a confiance dans la vanité, il se trompe, car la vanité sera sa récompense.
\VS{32}Ce sera fait de lui avant son temps, ses branches ne reverdiront plus. 
\VS{33}On arrachera ses fruits non mûrs, comme à une vigne; on jettera sa fleur, comme celle d'un olivier. 
\VS{34}Car la famille des hypocrites est stérile, et le feu dévore les tentes de l'homme corrompu.
\VS{35}Ils conçoivent le travail, et ils enfantent la misère, et machinent dans le cœur des fraudes. \FTNT{Es. 59:4 ; Os. 10:13.}.
\Chap{16}
\TextTitle{Réponse de Job}
\VerseOne{}Job répondit, et dit :
\VS{2}J'ai souvent entendu de pareils discours ; vous êtes tous des consolateurs fâcheux.
\VS{3}Y aura-t-il une fin à [ces] paroles de vent ? Qu'est-ce qui t'irrite, que tu répondes ?
\VS{4}Parlerais-je comme vous faites, si vous étiez en ma place ; accumulerais-je des paroles contre vous, ou secouerais-je ma tête contre vous ? 
\VS{5}Je vous fortifierais de ma bouche, et le mouvement de mes lèvres vous soulagerait.
\VS{6}Si je parle, ma douleur ne sera point soulagée. Si je me tais, en sera-t-elle diminuée?
\VS{7}Maintenant il m'a épuisé... Tu as dévasté toute ma famille, ;
\VS{8}Tu m'as tout couvert de rides, qui sont un témoignage des maux que je souffre ; et il s'est élevé en moi une maigreur qui en rend aussi témoignage sur mon visage. 
\VS{9}Sa fureur me déchire, il se déclare mon ennemi, il grince des dents contre moi, et étant devenu mon ennemi, il étincelle des yeux contre moi.
\VS{10}Ils ouvrent contre moi leur bouche; ils me frappent à la joue pour m'outrager; ils se réunissent tous ensemble contre moi. 
\VS{11}Dieu m'a livré à l'impie, et m'a jeté entre les mains des méchants. 
\VS{12}J'étais tranquille, et il m'a secoué, il m'a saisi par la nuque et m'a brisé, il m'a posé en butte à ses traits.
\VS{13}Ses archers m'ont environné, il me perce les reins, et ne m'épargne pas ; il répand mon fiel par terre. 
\VS{14}Il m'a brisé en me faisant plaie sur plaie, il a couru sur moi comme un homme fort.
\VS{15}J'ai cousu un sac sur ma peau, j'ai souillé ma tête dans la poussière\FTNT{Ps. 44 : 25 ; Ps. 119 : 25.},
\VS{16}J'ai le visage tout enflammé, à force de pleurer, et l'ombre de la mort est sur mes paupières, 
\VS{17}Quoiqu'il n'y ait point de violence dans mes mains, et que ma prière fut toujours pure.
\VS{18}Ô terre, ne cache pas mon sang, et qu'il n'y ait aucun lieu où s'arrête mon cri !
\VS{19}Mais maintenant voilà, mon témoin est aux cieux, mon témoin est dans les lieux élevés. \FTNT{Ap. 1:5 ; Ap. 3:14.}.
\VS{20}Mes amis se moquent de moi: c'est vers Dieu que mon œil se tourne en pleurant,
\VS{21}pour qu'il fasse justice entre l'homme et Dieu, entre le fils d'Adam et son semblable.
\VS{22}Car les années de mon compte arrivent à leur terme, et j'entre dans un sentier d'où je ne reviendrai plus. 
\Chap{17}
\VerseOne{}Mon souffle se perd, mes jours s'éteignent, le sépulcre m'attend.
\VS{2}Je suis environné de moqueurs, et mon œil veille toute la nuit au milieu de leurs insultes.
\VS{3}Dépose un gage, sois ma caution auprès de toi-même; car qui voudrait répondre pour moi? 
\VS{4}C'est pourquoi tu ne les élèveras pas\FTNT{De. 29:4 ; Mt. 11:25.}.
\VS{5}Celui qui trahit ses amis pour qu'ils soient pillés, les yeux de ses fils se consument.
\VS{6}On a fait de moi la fable des peuples, un être à qui l'on crache au visage.
\VS{7}Mon œil est obscurci par le chagrin, tous mes membres sont comme une ombre\FTNT{Ps. 6:7 ; Ps. 31:10.}.
\VS{8}Les hommes droits en sont consternés, et l'innocent est irrité contre l'impie. 
\VS{9}Toutefois le juste se tient ferme dans sa voie, et celui qui a les mains pures, se renforce.
\VS{10}Retournez donc vous tous, et revenez, je vous prie ; car je ne trouve pas de sages parmi vous. 
\VS{11}Mes jours sont passés; mes desseins, chers à mon cœur, sont renversés.…
\VS{12}On me change la nuit en jour, et on fait que la lumière se trouve proche des ténèbres !
\VS{13}Certes je n'ai plus à attendre que le sépulcre, qui va être ma maison ; j'ai dressé mon lit dans les ténèbres ;
\VS{14}J'ai crié à la fosse : tu es mon père ; et aux vers : vous êtes ma mère et ma Soeur. 
\VS{15}Où est donc mon espérance? Et mon espérance, qui pourrait la voir? \VS{16}Elle descendra au fond du sépulcre ; certes elle reposera avec moi dans la poussière. 
\Chap{18}
\TextTitle{Deuxième discours de Bildad}
\VerseOne{}Bildad de Schuach prit la parole et dit :
\VS{2}Quand finirez-vous ces discours ? Ecoutez, et puis nous parlerons.
\VS{3}Pourquoi sommes-nous regardés comme des bêtes, et sommes-nous stupides à vos yeux?
\VS{4}Ô toi qui déchires ton âme dans ta colère, la terre sera-t-elle abandonnée à cause de toi, et le rocher sera-t-il transporté de sa place ? 
\VS{5}Certainement, la lumière du méchant s'éteindra, et la flamme de son feu ne brillera pas\FTNT{Ps. 37:9-10.}.
\VS{6}La lumière sera ténèbres dans sa tente, et sa lampe sera éteinte au-dessus de lui. 
\VS{7}Les pas de sa force seront resserrés, et son propre conseil le renversera.
\VS{8}Car il est poussé dans le filet par ses propres pieds ; et il marche sur les mailles du filet.
\VS{9}Le piège le prend par le talon, et le filet s'empare de lui;
\VS{10}la corde est cachée dans la terre, et la trappe est sur son sentier.
\VS{11}Les terreurs l'assiègent de tous côtés, et le font courir ses pieds çà et là.\FTNT{Jé. 6:25 ; Jé. 46:5 ; Jé. 49:29.}.
\VS{12}Sa vigueur sera affamée, la détresse est à ses côtés.
\VS{13}Il dévorera les membres de son corps, il dévorera ses membres, le premier-né de la mort ! 
\VS{14} Les choses en quoi il mettait sa confiance seront arrachées de sa tente, et il sera conduit vers le Roi des épouvantements. 
\VS{15}On habitera dans sa tente, qui ne sera plus à lui; le soufre sera répandu sur sa demeure. 
\VS{16}Ses racines sèchent au dessous, et ses branches sont coupées en haut. 
\VS{17}Sa mémoire périt sur la terre, et on ne parle plus de son nom dans les places \FTNT{Ps. 109:13 ; Pr. 10:7.}.
\VS{18}Il est chassé de la lumière dans les ténèbres, et il est exterminé du monde. 
\VS{19}Il n'a ni lignée, ni descendance au milieu de son peuple, ni survivant dans ses habitations. \FTNT{Es. 14:20-22 ; Jé. 22:30 ; Ps. 37:28 ; Ps. 109:13.}. 
\VS{20}Ceux qui seront venus après lui, seront étonnés de sa ruine ; et ceux qui auront été avant lui en seront saisis d'horreur. 
\VS{21}Tel est le sort de l'injuste. Telle est la destinée de celui qui ne connaît pas Dieu. 
\Chap{19}
\TextTitle{Réponse de Job}
\VerseOne{}Job prit la parole et dit :
\VS{2}Jusqu'à quand affligerez-vous mon âme, et m'accablerez-vous de paroles ?
\VS{3} Voilà déjà dix fois que vous m'outragez: vous n'avez pas honte de me maltraiter? 
\VS{4}Vraiment si j'ai failli, ma faute demeure avec moi. 
\VS{5}Si réellement vous voulez vous élever contre moi et faire valoir mon opprobre contre moi, 
\VS{6}Sachez donc que c'est Dieu qui me renverse, et qui tend son filet autour de moi.
\VS{7}Voici je crie pour la violence qui m'est faite, et je ne suis pas exaucé ; je m'écrie, et il n'y a point de justice !
\VS{8}Il a fermé mon chemin, et je ne puis passer; il a mis des ténèbres sur mes sentiers. 
\VS{9}Il m'a dépouillé de ma gloire, il a ôté la couronne de ma tête.
\VS{10}Il m'a détruit de tous côtés, et je m'en vais ; il a arraché mon espérance comme un arbre.
\VS{11}Il s'est enflammé de colère contre moi, et m'a traité comme un de ses ennemis\FTNT{La. 2:5.}.
\VS{12}Ses troupes sont venues ensemble, et elles ont dressé leur chemin contre moi, et se sont campées autour de ma tente\FTNT{La. 2:22.}.
\VS{13}Il a éloigné de moi mes frères, et ceux qui me connaissaient se sont écartés comme des étrangers\FTNT{Ps. 88:9.} ;
\VS{14}mes proches m'ont abandonné, et ceux que je connaissais m'ont oublié.
\VS{15}Ceux qui séjournent dans ma maison et mes servantes m'ont traité comme un étranger; je suis devenu un inconnu pour eux. 
\VS{16}J'appelle mon serviteur, il ne me répond; de ma propre bouche, je le supplie en vain. 
\VS{17}Mon haleine est devenue dégoûtante à ma femme, et ma plainte aux fils de mes entrailles.
\VS{18}Je suis méprisé même par des enfants ; si je me lève, ils parlent contre moi.
\VS{19}Ceux que j'avais pour confidents m'ont en horreur, ceux que j'aimais se sont tournés contre moi\FTNT{Ps. 55:13-14.}.
\VS{20}Mes os sont attachés à ma peau et à ma chair ; et je me suis échappé avec la peau de mes dents\FTNT{La. 4:8.}.
\VS{21}Ayez pitié, ayez pitié de moi, vous, mes amis ! Car la main de Dieu m'a frappé.
\VS{22}Pourquoi, comme Dieu, me poursuivez-vous et n'êtes-vous pas rassasiés de ma chair \FTNT{Ps. 27:2.} ?
\VS{23}Oh! je voudrais que mes paroles fussent écrites quelque part, je voudrais qu'elles fussent inscrites dans un livre; 
\VS{24}Qu'avec un burin de fer et avec du plomb, elles fussent gravées sur le roc, pour toujours... 
\VS{25}Mais je sais que mon rédempteur est vivant, il demeurera le dernier sur la terre.
\VS{26}Et après que cette peau aura été détruite, hors de ma chair, je verrai Dieu \FTNT{Ps. 17:15.}.
\VS{27}Je le verrai moi-même, et mes yeux le verront, et non un autre. Mes reins se consument dans mon sein. 
\VS{28}Vous direz : Comment le poursuivrons-nous, et trouverons-nous en lui la cause de son malheur? 
\VS{29}Ayez peur de l'épée ; car la fureur [avec laquelle vous me persécutez], est [du nombre] des iniquités qui attirent l'épée ; c'est pourquoi sachez qu'il y a un jugement. 
\Chap{20}
\TextTitle{Dernier discours de Tsophar}
\VerseOne{}Tsophar de Naama prit la parole et dit :
\VS{2}C'est à cause de cela que mes pensées diverses me poussent à répondre, et que cette promptitude est en moi. 
\VS{3}J'ai entendu la correction dont tu veux me faire honte, mais mon esprit tirera de mon intelligence la réponse pour moi. 
\VS{4}Ne sais-tu pas que, de tout temps, depuis que Dieu a mis l'homme sur la terre, 
\VS{5}Le triomphe des méchants est de peu de durée, et la joie de l'hypocrite n'est que pour un moment \FTNT{Ps. 37:35-36.} ?
\VS{6}Quand son élévation monterait jusqu'aux cieux, et que sa tête atteindrait les nues,
\VS{7}il périra pour toujours comme ses ordures, et ceux qui le voyaient diront : Où est-il ?
\VS{8}Il s'envolera comme un songe, et on ne le trouvera plus ; il se  retirera comme une vision nocturne\FTNT{Ps. 73:19-20.} ;
\VS{9}l'œil qui le regardait ne le regardera plus, le lieu qu'il habitait ne le contemplera plus.
\VS{10}Ses fils rechercheront la faveur des pauvres, et ses mains restitueront ce que sa violence a ravi\FTNT{Ps. 109:10.}.
\VS{11}Ses os seront pleins de la punition, à  cause des péchés de sa jeunesse, et elle reposera avec lui dans la poussière.
\VS{12}Le mal était doux à sa bouche, il le cachait sous sa langue,
\VS{13}s'il l'épargne, et ne le rejette point, mais le retient dans son palais ; 
\VS{14}Ce qu'il mangera se changera dans ses entrailles en un fiel d'aspic.
\VS{15}Il a englouti des richesses, il les vomira ; Dieu les arrachera de son ventre.
\VS{16}Il a sucé du venin d'aspic, la langue de la vipère le tuera.
\VS{17}Il ne verra plus les ruisseaux, les fleuves, les torrents de miel et de lait.
\VS{18}Il rendra le fruit de son travail, et ne l'avalera pas; il restituera à proportion de ce qu'il aura amassé, et ne s'en réjouira pas\FTNT{So. 2:10.}.
\VS{19}Car il a opprimé, délaissé les pauvres; il a pillé des maisons et ne les a pas rebâties.
\VS{20}Certainement il ne sentira pas dans son ventre la satisfaction de son avidité, et il ne sauvera rien de ce qu'il aura tant convoité \FTNT{Ec. 5:12.}.
\VS{21}Rien n'échappait à sa voracité, mais son bonheur ne durera pas.
\VS{22}Après que la mesure de ses biens aura été remplie, il sera dans la misère ; toutes les mains de ceux qu'il aura opprimés se jetteront sur lui.
\VS{23}Il arrivera que pour lui remplir le ventre, Dieu enverra contre lui l'ardeur de sa colère; il la fera pleuvoir sur lui et entrer dans sa chair.
\VS{24}S’il s’enfuit de devant les armes de fer, l’arc d’airain le transpercera.
\VS{25}Il arrachera la flèche, et elle sortira de son corps, et le fer étincelant, de son foie; les frayeurs de la mort viendront sur lui.
\VS{26}Toutes les ténèbres sont renfermées dans ses demeures les plus secrètes ; un feu qu'on n'aura point soufflé, le consumera ; l'homme qui restera dans sa tente sera malheureux\FTNT{Ps. 12:6.}.
\VS{27}Les cieux découvriront son iniquité, et la terre s'élèvera contre lui. 
\VS{28}Le revenu de sa maison sera emporté. Tout s'écoulera au jour de la colère de Dieu.
\VS{29}C'est là la portion que Dieu réserve à l'homme méchant, et l'héritage qu'il aura de Dieu pour ses discours.
\Chap{21}
\TextTitle{Réponse de Job}
\VerseOne{}Job répondit, et dit :
\VS{2}Ecoutez, écoutez mes discours, donnez-moi seulement cette consolation.
\VS{3}Supportez-moi, et je parlerai ; et quand j'aurai parlé, tu pourras te moquer.
\VS{4}Mais est-ce contre un homme que s'adresse ma plainte ? Et pourquoi mon âme ne serait-elle pas impatiente ?
\VS{5}Regardez-moi, soyez étonnés, et mettez la main sur la bouche.
\VS{6}Quand j'y pense, cela m'épouvante, et un frisson saisit mon corps.
\VS{7}Pourquoi les méchants vivent-ils, vieillissent-ils, et croissent-ils en puissance\FTNT{Jé. 12:1 ; Ha. 1:3 ; Mal. 3:14-15.}?
\VS{8}Leur postérité s'établit avec eux et en leur présence, leurs rejetons prospèrent sous leurs yeux.
\VS{9}Dans leurs maisons règne la paix, loin de la crainte ; la verge de Dieu ne vient pas les frapper.
\VS{10}Leurs taureaux sont féconds, leurs génisses conçoivent et n'avortent pas\FTNT{Ps. 144:13-14.}.
\VS{11}Ils laissent courir leurs enfants comme un troupeau, et les enfants prennent leurs ébats.
\VS{12}Ils chantent au son du tambourin et de la harpe, ils se réjouissent au son du chalumeau.
\VS{13}Ils passent leurs jours dans le bonheur, et ils descendent en un instant au scheol.
\VS{14}Ils disaient pourtant à Dieu : Éloigne-toi de nous, nous ne voulons pas connaître tes voies.
\VS{15}Qu'est-ce que le Tout-Puissant pour que nous le servions ? Que gagnerions-nous à lui adresser nos prières\FTNT{Ex. 5:2.} ?
\VS{16}Quoi donc ! Ne sont-ils pas en possession du bonheur entre leurs mains ? Loin de moi le conseil des méchants\FTNT{Ps. 1:1-2.} !
\VS{17}Mais arrive-t-il que la lampe des méchants s'éteigne, que la ruine vienne sur eux, que Dieu leur distribue leur part dans sa colère\FTNT{Ps. 11:5-6 ; Pr. 13:9.},
\VS{18}qu'ils soient comme la paille face au vent, comme la balle enlevée par le tourbillon\FTNT{Ps. 1:4.}?
\VS{19}Dieu réservera-t-il aux enfants du méchant la punition de ses violences? Il la leur rendra, et il le connaîtra!
\VS{20}Il verra de ses propres yeux sa ruine, c'est lui qui devrait boire la colère du Tout-Puissant\FTNT{Es. 51:17-22 ; Jé. 25:15 ; Ez. 23:31-32 ; Ap. 14:10.}.
\VS{21}Car que lui importe sa maison après lui, quand le nombre de ses mois est achevé ?
\VS{22}Enseignerait-on la science à Dieu, lui qui juge les esprits élevés\FTNT{Ro. 11:34 ; 1 Co. 2:16.} ?
\VS{23}L'un meurt au sein du bien-être, tout à son aise et en joie,
\VS{24}les flancs chargés de graisse, et ses os comme abreuvés de mœlle ;
\VS{25}l'autre meurt l'amertume dans l'âme, n'ayant jamais mangé ce qui est bon.
\VS{26}Et tous deux se couchent dans la poussière, tous deux deviennent couverts de vers.
\VS{27}Je sais bien quelles sont vos pensées, quels jugements iniques vous portez sur moi.
\VS{28}Vous dites : Où est la maison de l'homme puissant ? Où est la tente, demeure des méchants ?
\VS{29}Mais quoi ! N'avez-vous pas interrogé les voyageurs, et n'avez-vous pas appris par les rapports qu'ils vous on faits ?
\VS{30}Au jour du malheur, le méchant est épargné ; au jour de la colère, il échappe\FTNT{Pr. 16:4 ; Ec. 9:12.}.
\VS{31}Qui lui dit en face sa conduite ? Qui lui rend ce qu'il a fait ?
\VS{32}Il est porté au tombeau, et il veille encore sur sa tombe.
\VS{33}Les mottes de la vallée lui sont légères ; et tous après lui suivront la même voie, comme une multitude l'a déjà suivie.
\VS{34}Pourquoi donc m'offrir de vaines consolations ? Ce qui reste de vos réponses n'est que transgression.
\Chap{22}
\TextTitle{Dernier discours d'Eliphaz}
\VerseOne{}Eliphaz de Théman prit la parole et dit :
\VS{2}Un homme peut-il être utile à Dieu ? Mais le sage n'est utile qu'à lui-même.
\VS{3}Si tu es juste, est-ce à l'avantage du Tout-Puissant ? Si tu es intègre dans tes voies, qu'y gagne-t-il ?
\VS{4}Est-ce par crainte de toi qu'il te reprend, qu'il entre en jugement avec toi?
\VS{5}Ta méchanceté n'est-elle pas grande ? Tes iniquités ne sont-elles pas sans fin ?
\VS{6}Tu a pris sans raison le gage de tes frères, tu privais de leurs vêtements ceux qui étaient nus\FTNT{Ex. 22:21.} ;
\VS{7}tu ne donnais pas d'eau à boire à l'homme altéré, tu refusais du pain à l'homme affamé.
\VS{8}Le pays était à l'homme le plus fort, et le puissant s'y établissait.
\VS{9}Tu renvoyais les veuves à vide, les bras des orphelins étaient brisés.
\VS{10}C'est pour cela que tu es entouré de pièges, et que la terreur t'a saisi tout à coup.
\VS{11}Ne vois-tu donc pas ces ténèbres, ces eaux débordées qui te couvrent ?
\VS{12}Dieu n'est-il pas là-haut dans les cieux ? Regarde le sommet des étoiles, comme il est élevé !
\VS{13}Et tu dis : Qu'est-ce que Dieu connaît ? Peut-il juger à travers l'obscurité\FTNT{So. 1:12 ; Ps. 10:11-13 ; Ps. 94:7.} ?
\VS{14}Les nuées l'enveloppent, et il ne voit rien ; il ne parcourt que la voûte des cieux.
\VS{15}Eh quoi ! N'as-tu pas pris garde à l'ancienne route qu'ont suivie les hommes d'iniquité ?
\VS{16}Ils ont été emportés avant le temps, ils ont eu la durée d'un torrent qui s'écoule.
\VS{17}Ils disaient à Dieu : Éloigne-toi de nous ; que peut faire pour nous le Tout-Puissant ?
\VS{18}Dieu cependant avait rempli leurs maisons de biens ! Loin de moi le conseil des méchants !
\VS{19}Les justes le verront, se réjouiront, et l'innocent se moquera d'eux\FTNT{Ps. 107:42.} :
\VS{20}Certainement, notre adversaire a été détruit, le feu a dévoré ce qui en restait\FTNT{Ps. 37:20 ; Ec. 8:12-13.} !
\VS{21}Attache-toi donc à Dieu, et tu auras la paix, tu atteindras ainsi le bonheur.
\VS{22}Reçois de sa bouche l'instruction, et mets ses paroles dans ton cœur\FTNT{Ps. 119:72.}.
\VS{23}Si tu reviens au Tout-Puissant, tu seras rétabli ; si tu éloignes l'iniquité de ta tente.
\VS{24}Jette l'or dans la poussière, l'or d'Ophir parmi les rochers des torrents ;
\VS{25}et le Tout-Puissant sera ton or, ton argent, ta richesse.
\VS{26}Alors tu feras du Tout-Puissant tes délices, tu élèveras vers Dieu ta face ;
\VS{27}tu le prieras, et il t'exaucera, et tu lui rendras tes vœux\FTNT{Ps. 50:14-15.}.
\VS{28}Quand tu prendras des résolutions elles s'accompliront, sur tes sentiers brillera la lumière\FTNT{Ps. 97:11.}.
\VS{29}Quand on aura abaissé quelqu'un et que tu auras dit qu'il soit élevé; alors Dieu délivrera celui qui tenait les yeux abaissés\FTNT{Pr. 29:23.}.
\VS{30}Il délivrera le coupable ; il sera délivré par la pureté de tes mains.
\Chap{23}
\TextTitle{Réponse de Job}
\VerseOne{}Job répondit, et dit :
\VS{2}Maintenant encore ma plainte est une révolte, et pourtant ma main appesantit mes soupirs.
\VS{3}Oh ! Si je savais où le trouver, j'irais jusqu'à son trône,
\VS{4}je disposerais en ordre ma cause devant lui, je remplirais ma bouche d'arguments,
\VS{5}je saurais ce qu'il peut avoir à répondre, je comprendrais ce qu'il peut avoir à me dire.
\VS{6}Contesterait-il avec moi dans la grandeur de sa force? Ne prendrait-il pas le temps de m'écouter ?
\VS{7}Ce serait un homme juste qui argumenterait avec lui, et je serais pour toujours absous par mon juge.
\VS{8}Mais, si je vais à l'orient, il n'y est pas ; si je vais à l'occident, je ne l'aperçois pas ;
\VS{9}est-il occupé au nord, je ne le vois pas ; se cache-t-il au midi, je ne l'aperçois pas.
\VS{10}Il connaît la voie que j'ai suivie ; et s'il m'éprouvait, j'en sortirai pur comme l'or\FTNT{1 Pi. 1:7.}.
\VS{11}Mon pied s'est attaché à ses pas ; j'ai gardé sa voie, et je ne m'en suis pas détourné.
\VS{12}Je n'ai pas abandonné les commandements de ses lèvres ; j'ai fait plier ma volonté aux paroles de sa bouche.
\VS{13}Mais il n'a qu'une pensée ; qui l'en fera revenir ? Ce que son âme désire, il le fait\FTNT{Ps. 115:3 ; Ps. 135:6.}.
\VS{14}Il achèvera donc ses desseins à mon égard, et il en concevra beaucoup d'autres encore.
\VS{15}C'est pourquoi je suis terrifié à cause de sa présence, et quand je le considère, je suis effrayé devant lui.
\VS{16}Dieu a brisé mon cœur, le Tout-Puissant m'a épouvanté.
\VS{17}Car ce n'est pas la présence des ténèbres qui m'anéantit, ce n'est pas l'obscurité dont ma face est couverte.
\Chap{24}
\VerseOne{}Pourquoi le Tout-Puissant ne met-il pas des temps en réserve, et pourquoi ceux qui le connaissent ne voient-ils pas ses jours ?
\VS{2}On déplace les bornes, on ravit des troupeaux, et on les fait paître\FTNT{De. 19:14 ; De. 27:17 ; Pr. 13:10 ; Pr. 22:28.} ;
\VS{3}on emmène l'âne de l'orphelin, on prend pour gage le bœuf de la veuve ;
\VS{4}on fait écarter les pauvres du chemin, on force tous les affligés du pays à se cacher.
\VS{5}Et voici, comme les ânes sauvages du désert, ils sortent le matin pour chercher de la nourriture, ils n'ont que le désert pour trouver le pain de leurs enfants ;
\VS{6}ils moissonnent le fourrage qui reste dans les champs, ils grappillent dans la vigne de l'impie ;
\VS{7}ils passent la nuit nus, sans vêtements, sans couverture contre le froid\FTNT{Lé. 19:13 ; De. 24:12-13.} ;
\VS{8}ils sont percés par la pluie des montagnes, et ils embrassent les rochers comme unique refuge.
\VS{9}On arrache l'orphelin à la mamelle, on prend des gages sur le pauvre.
\VS{10}Ils font aller sans habits l'homme qu'ils ont dépouillé; et ils enlèvent à ceux qui n'avaient pas de quoi manger, ce qu'ils avaient glâné.
\VS{11}Dans les enclos de l'impie, ils font de l'huile, ils foulent le pressoir à raisin et ils ont soif.
\VS{12}Ils font gémir les gens dans la ville, l'âme de ceux qu'ils ont fait mourir, crient; Dieu ne fait rien d'indigne de lui.
\VS{13}En voici d'autres qui se révoltent contre la lumière, ils n'en connaissent pas les voies, ils ne restent pas sur leurs sentiers.
\VS{14}Le meurtrier se lève au point du jour ; il tue le pauvre et l'indigent, et il dérobe pendant la nuit\FTNT{Ps. 10:8-9.}.
\VS{15}L'œil de l'adultère épie le crépuscule ; aucun œil ne me verra, dit-il, et il met un voile sur le visage\FTNT{Ps. 64:6 ; Pr. 7:7-10.}.
\VS{16}Ils percent durant les ténèbres les maisons, qu'ils avaient marquées le jour, ils haïssent la lumière.\FTNT{Jn. 3:20.}.
\VS{17}Pour eux, le matin c'est l'ombre de la mort ; si quelqu'un les reconnaît, ils ont des terreurs.
\VS{18}Eh quoi ! L'impie est d'un poids léger sur la surface de l'eau, il n'a sur la terre qu'un héritage maudit, il ne prend jamais le chemin des vignes !
\VS{19}Comme la sécheresse et la chaleur absorbent les eaux de la neige, ainsi le scheol engloutit ceux qui pèchent\FTNT{Ps. 49:15.} !
\VS{20}Quoi ! Le sein maternel l'oublie, les vers en font leurs délices, on ne se souvient plus de lui ! L'injuste est brisé comme du bois,
\VS{21}lui qui dépouille la femme stérile et sans enfants, lui qui ne répand aucun bien sur la veuve !…
\VS{22}Non ! Dieu par sa force prolonge les jours des violents, et les voilà s'élever quand ils ne croyaient plus en la vie.
\VS{23}Il leur donne de la sécurité et de la confiance, ses yeux sont sur leurs voies.
\VS{24}Ils se sont élevés ; et en un peu de temps ils ne sont plus, ils s'affaissent, ils meurent en chemin comme tous les hommes, ils sont coupés comme une tête d'épi.
\VS{25}S'il n'en est pas ainsi, qui me fera mentir, qui fera de mes paroles un rien ?
\Chap{25}
\TextTitle{Dernier discours de Bildad}
\VerseOne{}Bildad de Schuach prit la parole et dit :
\VS{2}La domination et la terreur appartiennent à Dieu ; il fait régner la paix dans ses hautes régions.
\VS{3}Ses armées peuvent-elles se compter ? Sur qui sa lumière ne se lève-t-elle pas\FTNT{Mt. 5:45.} ?
\VS{4}Comment l'homme serait-il juste devant Dieu ? Comment celui qui est né de la femme serait-il pur ?
\VS{5}Voici, la lune même n'est pas brillante, et les étoiles ne sont pas pures à ses yeux ;
\VS{6}combien moins l'homme qui n'est qu'un ver, le fils de l'homme qui n'est qu'un vermisseau\FTNT{Ps. 22:7.} !
\Chap{26}
\TextTitle{Réponse de Job}
\VerseOne{}Job répondit, et dit :
\VS{2}Comme tu as aidé celui qui était sans force ! Comme tu as secouru le bras sans force !
\VS{3}Quels bons conseils tu donnes à celui qui manque de sagesse ! Tu fais connaître l'abondance de ton intelligence !
\VS{4}A qui s'adressent tes paroles ? Et de qui est l'esprit qui est sorti de toi ?
\VS{5}Devant Dieu les ombres des morts tremblent au-dessous des eaux, et de leurs habitants ;
\VS{6}devant lui le scheol est nu, l'abîme est sans voile\FTNT{Ps. 139:8-12 ; Pr. 15:11 ; Hé. 4:13.}.
\VS{7}Il étend la direction nord sur le vide, il suspend la terre sur le néant.
\VS{8}Il renferme les eaux dans ses nuages, et la nuée n'éclate pas sous leur poids\FTNT{Ps. 104:2-3.}.
\VS{9}Il couvre la face de son trône, il répand sur lui sa nuée.
\VS{10}Il a tracé un cercle à la surface des eaux, comme limite entre la lumière et les ténèbres\FTNT{Ge. 1:9 ; Jé. 5:22 ; Ps. 33:7 ; Ps. 104:9 ; Pr. 8:29.}.
\VS{11}Les colonnes du ciel s'ébranlent et s'étonnent à sa menace.
\VS{12}Par sa force il soulève la mer, par son intelligence il en brise l'orgueil\FTNT{Ps. 89:10.}.
\VS{13}Il a orné les cieux par son Esprit, et  de sa main, il transperce le serpent fuyard.
\VS{14}Ce sont là les bords de ses voies, c'est le discours fait en chuchotant que nous entendons ; mais qui comprendra le tonnerre de sa puissance\FTNT{Ec. 3:10.} ?
\Chap{27}
\VerseOne{}Et Job continuant, reprit son discours sentencieux, et dit :
\VS{2}Dieu, qui met mon droit à l'écart, et le Tout-puissant qui remplit mon âme d'amertume, est vivant.
\VS{3}Aussi longtemps que j'aurai ma respiration et que l'esprit de Dieu sera dans mes narines,
\VS{4}mes lèvres ne prononceront rien d'injuste, et ma langue ne dira pas de chose fausse\FTNT{Es. 33:15 ; Ps. 15:2 ; Ps. 24:4.}.
\VS{5}Loin de moi la pensée de vous reconnaître pour justes ! Tant que je vivrai je n'abandonnerai pas mon intégrité.
\VS{6}Je conserve ma justice, et je ne l'abandonne pas ; et mon cœur ne me reproche rien en mes jours.
\VS{7}Qu'il en soit de mon ennemi comme du méchant ; et de celui qui se lève contre moi, comme de l'injuste !
\VS{8}Quelle espérance reste-t-il à l'hypocrite quand Dieu coupe le fil de sa vie, quand il lui retire son âme\FTNT{Mt. 16:26 ; Lu. 12:20.} ?
\VS{9}Est-ce que Dieu entend ses cris, quand l'angoisse vient sur lui\FTNT{ Es. 1:15 ; Jé. 14:12 ; Ez. 8:18 ; Mi. 3:4 ; Ps. 18:41 ; Pr. 1:28 ; Jn. 9:31 ; Ja. 4:3.} ?
\VS{10}Trouvera-t-il son plaisir dans le Tout-Puissant ? Invoque-t-il Dieu en tout temps ?
\VS{11}Je vous enseignerai comment la main de Dieu agit, je ne vous cacherai pas les desseins du Tout-Puissant.
\VS{12}Voilà, vous avez tous vu ces choses, et pourquoi vous laissez-vous aller à des pensées vaines ?
\VS{13}Voici la part que Dieu réserve à l'homme méchant, l'héritage que les violents reçoivent du Tout-Puissant.
\VS{14}S'il a des fils en grand nombre, c'est pour l'épée, et ses rejetons ne seront pas rassasiés de pain ;
\VS{15}ses survivants sont ensevelis par la peste, et leurs veuves ne les pleurent pas\FTNT{Ps. 78:64.}.
\VS{16}Parce qu’il entasse l'argent comme la poussière, et qu'il entasse des habits comme on amasse de la boue,
\VS{17}le riche tombe, et il n’est pas relevé ; il ouvre ses yeux, et il ne trouve rien.
\VS{18}Il se bâtit une maison comme celle de la teigne, comme la cabane que fait un gardien\FTNT{Ps. 49:18.}.
\VS{19}Il se couche riche, et il périt dépouillé ; il ouvre les yeux, et tout a disparu.
\VS{20}Les frayeurs l'atteignent comme des eaux ; le tourbillon l'enlève de nuit.
\VS{21}Le vent d'orient l'emporte, et il s'en va ; il l'arrache de sa demeure comme un tourbillon.
\VS{22}Dieu le précipite à terre et ne l'épargne pas, et le méchant voudrait fuir devant sa main.
\VS{23}On applaudit à sa chute, et on le siffle au lieu où il se tient.
\Chap{28}
\VerseOne{}Il y a pour l'argent une mine d'où on le fait sortir, et pour l'or un lieu d'où on le purifie pour l'affiner ;
\VS{2}le fer se tire de la poussière, et la pierre se fond pour produire l'airain.
\VS{3}L'homme fait cesser les ténèbres, il explore jusqu'aux extrêmes limites les pierres cachées dans l'obscurité et dans l'ombre de la mort.
\VS{4}Il creuse un puits, loin des lieux habités ; ne se souvenant plus de ses pieds, il est suspendu, balancé, loin des humains.
\VS{5}La terre, d'où sort le pain, est bouleversée dans ses entrailles comme par le feu.
\VS{6}Ses pierres sont la demeure du saphir, et l'on y trouve de la poudre d'or.
\VS{7}L'oiseau de proie n'en connaît pas le chemin, l'œil du vautour ne l'aperçoit pas ;
\VS{8}les plus jeunes et fiers animaux n'y ont pas marché, le lion n'y a jamais passé.
\VS{9}L'homme avance sa main sur le roc, il renverse les montagnes depuis la racine ;
\VS{10}il fend des tranchées dans les rochers, et son œil voit tout ce qu'il y a de précieux ;
\VS{11}il arrête l'écoulement des eaux, et il fait sortir ce qui est caché.
\VS{12}Mais la sagesse, où se trouve-t-elle ? Où est le lieu où se tient l'intelligence ?
\VS{13}L'homme n'en connaît pas le prix, elle ne se trouve pas dans la terre des vivants.
\VS{14}L'abîme dit : Elle n'est pas en moi ; et la mer dit : Elle n'est pas avec moi.
\VS{15}Elle ne se donne pas contre de l'or pur, elle ne s'achète pas au poids de l'argent\FTNT{Pr. 3:14 ; Pr. 8:11 ; Pr. 16:16.} ;
\VS{16}elle ne se pèse pas contre de l'or d'Ophir, ni contre le précieux onyx, ni contre le saphir.
\VS{17}Elle ne peut se comparer à l'or ni au verre, elle ne peut s'échanger pour un vase d'or fin.
\VS{18}On ne se souvient ni du corail ni du cristal auprès d'elle : La sagesse vaut plus que les perles.
\VS{19}On ne la compare pas avec la topaze d'Ethiopie ; on ne la met pas en balance avec l'or pur.
\VS{20}D'où vient donc la sagesse ? Où est la demeure de l'intelligence ?
\VS{21}Elle est cachée aux yeux de tous les vivants, elle est cachée aux oiseaux des cieux.
\VS{22}L'abîme et la mort disent : Nous en avons entendu parler de nos oreilles.
\VS{23}C'est Dieu qui en sait le chemin, c'est lui qui en connaît la demeure ;
\VS{24}car il regarde jusqu'aux extrémités de la terre, il voit tout sous les cieux\FTNT{Ps. 14:2 ; Ps. 33:13-14 ; Ps. 102:20.}.
\VS{25}Quand il façonna le poids du vent, et qu'il estima la mesure des eaux\FTNT{Pr. 8:29.},
\VS{26}quand il ordonna des lois à la pluie, et qu'il fit un chemin à l'éclair et au tonnerre,
\VS{27}alors il vit la sagesse et la manifesta ; il l'établit et la sonda.
\VS{28}Puis il dit à l'homme : Voici, la crainte du Seigneur, c'est la sagesse ; se détourner du mal, c'est l'intelligence\FTNT{De. 4:6 ; Jé. 9:24 ; Ps. 111:10 ; Pr. 1:7 ; Pr. 9:10 ; Ec. 12:15.}.
\Chap{29}
\TextTitle{La postérité passée de Job}
\VerseOne{}Job prit de nouveau la parole sous forme sentencieuse et dit :
\VS{2}Oh ! Que ne puis-je être comme aux mois du passé, comme aux jours où Dieu me gardait,
\VS{3}quand sa lampe brillait sur ma tête, quand je marchais à sa lumière dans les ténèbres !
\VS{4}Que ne suis-je comme aux jours de mon automne, où Dieu veillait en ami sur ma tente,
\VS{5}quand le Tout-Puissant était encore avec moi, et que mes serviteurs m'entouraient ;
\VS{6}quand je lavais mes pieds dans le lait, et que le rocher répandait près de moi des torrents d'huile\FTNT{De. 32:13.} !
\VS{7}Si je sortais pour aller à la porte de la ville, et si je me faisais préparer un siège dans la place,
\VS{8}les jeunes gens se retiraient en me voyant, les vieillards se levaient et se tenaient debout.
\VS{9}Les princes s'abstenaient de parler, et mettaient la main sur leur bouche ;
\VS{10}la voix des chefs se taisait, et leur langue s'attachait à leur palais.
\VS{11}L'oreille qui m'entendait me disait heureux, l'œil qui me voyait me rendait témoignage ;
\VS{12}car je délivrais l'affligé qui criait au secours, et l'orphelin qui n'avait personne pour le secourir\FTNT{Ps. 72:12 ; Pr. 21:13.}.
\VS{13}La bénédiction de celui qui allait périr venait sur moi ; je remplissais de joie le cœur de la veuve.
\VS{14}Je me revêtais de la justice et elle se revêtait de moi, j'avais ma droiture pour manteau et pour turban\FTNT{Es. 59:17 ; 1 Th. 5:8 ; Ep. 6:14-17.}.
\VS{15}J'étais les yeux de l'aveugle et les pieds du boiteux.
\VS{16}J'étais le père des pauvres, j'examinais la cause de l'inconnu\FTNT{Pr. 29:7.} ;
\VS{17}je brisais les mâchoires de l'injuste, et j'arrachais la proie d'entre ses dents\FTNT{Ps. 58:7.}.
\VS{18}Alors je disais : Je mourrai dans mon nid, mes jours seront aussi nombreux que le sable ;
\VS{19}l'eau pénétrera dans mes racines, la rosée passera la nuit sur mes branches\FTNT{Jé. 17:5-8 ; Ps. 1:3.} ;
\VS{20}ma gloire se renouvellera sans cesse en moi, et mon arc se renouvellera dans ma main.
\VS{21}On m'écoutait et l'on restait dans l'attente, on gardait le silence devant mes conseils.
\VS{22}Après mes discours, nul ne répliquait, et ma parole était pour tous une bienfaisante rosée ;
\VS{23}ils s'attendaient à moi comme à la pluie, ils ouvraient la bouche comme pour une pluie de printemps.
\VS{24}Je souriais quand ils perdaient confiance, et l'on ne pouvait faire tomber la sérénité de mon visage.
\VS{25}J'aimais à aller avec eux, et je m'asseyais à leur tête ; j'étais comme un roi au milieu de ses gardes, comme un consolateur auprès des affligés.
\Chap{30}
\TextTitle{Son humiliation}
\VerseOne{}Mais maintenant !… Chaque jour je suis la risée de plus jeunes que moi, de ceux dont je dédaignais de mettre les pères parmi les chiens de mon troupeau.
\VS{2}Mais à quoi me servirait la force de leurs mains ? En eux avait péri toute vigueur.
\VS{3}Desséchés par la disette et la faim, ils fuient dans les lieux arides, depuis longtemps abandonnés et déserts ;
\VS{4}ils arrachent près des buissons l'herbe sauvage, et la racine des genêts est leur nourriture.
\VS{5}On les chasse du milieu des hommes, on crie après eux comme après un voleur.
\VS{6}Ils habitent dans le creux des torrents, dans les trous de la terre et des rochers ;
\VS{7}ils hurlent parmi les buissons, ils se rassemblent sous les ronces.
\VS{8}Peuple insensé et sans nom, on les repousse du pays !
\VS{9}Et maintenant, je suis le sujet de leurs chansons, je suis en butte à leurs propos\FTNT{Ps. 69:12 ; La. 3:14.}.
\VS{10}Ils m'ont en horreur, ils s'éloignent de moi, ils ne se retiennent pas de me cracher leur salive au visage.
\VS{11}Ils n'ont aucune retenue et ils m'humilient, ils rejettent tout frein devant moi.
\VS{12}Ces misérables se lèvent à ma droite et me poussent les pieds, ils se fraient contre moi des routes pour ma ruine\FTNT{Ps. 35:15.} ;
\VS{13}ils détruisent mon propre sentier et travaillent à ma perte, eux à qui personne ne viendrait en aide ;
\VS{14}ils viennent contre moi comme par une brèche large, et ils se sont jetés sur moi à cause de ma désolation.
\VS{15}Toutes les terreurs se tournent contre moi ; ma gloire est emportée comme par le vent, mon bonheur a passé comme un nuage\FTNT{Os. 13:3.}.
\VS{16}Et maintenant, mon âme se répand en mon sein, les jours d'affliction m'ont saisi.
\VS{17}La nuit me perce et m'arrache les os, la douleur qui me ronge ne se donne aucun repos.
\VS{18}Par la violence du mal, mon vêtement se déforme, il se colle à mon corps comme ma tunique.
\VS{19}Dieu m'a jeté dans la boue, et je ressemble à la poussière et à la cendre.
\VS{20}Je crie vers toi, et tu ne me réponds pas ; je me tiens debout, et tu m'aperçois.
\VS{21}Tu deviens cruel contre moi, tu t'opposes à moi avec la force de ta main.
\VS{22}Tu me soulèves, tu me fais chevaucher sur le vent, et tu me fais fondre au bruit de la tempête.
\VS{23}Car, je le sais, tu me mènes à la mort, à la demeure fixée pour tous les vivants\FTNT{Hé. 9:27.}.
\VS{24}Mais celui qui va périr n'étend-il pas les mains ? Celui qui est dans le malheur n'implore-t-il pas du secours ?
\VS{25}Ne pleurais-je pas sur l'homme qui passait des jours difficiles ? Mon âme n'avait-elle pas pitié du pauvre\FTNT{Ro. 12:15.} ?
\VS{26}J'attendais le bonheur, et le malheur est arrivé ; j'espérais la lumière, et les ténèbres sont venues.
\VS{27}Mes entrailles bouillonnent sans repos, les jours d'affliction m'ont confronté.
\VS{28}Je marche noirci, mais non par le soleil ; je me lève en pleine assemblée, et je crie.
\VS{29}Je suis devenu le frère des serpents, le compagnon des autruches\FTNT{Ps. 102:7-8.}.
\VS{30}Ma peau noircit et tombe, mes os brûlent et se dessèchent\FTNT{La. 4:8 ; La. 5:10.}.
\VS{31}Ma harpe n'est plus qu'un instrument de deuil, et mon chalumeau ne peut rendre que des voix en pleurs.
\Chap{31}
\TextTitle{Job se justifie}
\VerseOne{}J'avais fait une alliance avec mes yeux, et je n'aurais pas regardé une vierge.
\VS{2}Quelle part Dieu m'eût-il réservée d'en haut ? Quel héritage le Tout-Puissant m'aurait-il envoyé des cieux ?
\VS{3}La ruine n'est-elle pas pour l'injuste, et le malheur pour ceux qui commettent l'iniquité ?
\VS{4}Dieu ne voit-il pas mes voies ? Ne compte-t-il pas tous mes pas\FTNT{Pr. 5:21 ; Pr. 15:3 ; 2 Ch. 16:9.} ?
\VS{5}Si j'ai marché dans le mensonge, si mon pied s'est hâté pour tromper,
\VS{6}que Dieu me pèse dans des balances justes, et il reconnaîtra mon intégrité !
\VS{7}Si mes pas se sont détournés du droit chemin, si mon cœur a suivi mes yeux, si quelque souillure s'est attachée à mes mains,
\VS{8}Que je sème et qu'un autre mange, et tout ce que j'aurais fait produire soit déraciné!
\VS{9}Si mon cœur a été séduit par une femme, si j'ai fait le guet à la porte de mon prochain\FTNT{Pr. 7.},
\VS{10}que ma femme broie le grain pour un autre, et que d'autres se penchent sur elle !
\VS{11}Car c'est un crime, une iniquité punie par les juges ;
\VS{12}c'est un feu qui dévore jusqu'à la destruction, et qui aurait détruit toutes mes récoltes dans leur racine.
\VS{13}Si j'ai méprisé le droit de mon serviteur ou de ma servante, lorsqu'ils étaient en contestation avec moi,
\VS{14}qu'ai-je à faire, quand Dieu se lève ? Qu'ai-je à répondre, quand il châtie ?
\VS{15}Celui qui m'a fait dans le ventre de ma mère ne l'a-t-il pas fait aussi ? Un même Dieu ne nous a-t-il pas formés dans le sein maternel\FTNT{Pr. 14:31 ; Pr. 17:5.} ?
\VS{16}Si j'ai refusé aux pauvres leur désir, si j'ai laissé se consumer les yeux de la veuve\FTNT{Es. 10:2 ; Lu. 18:2-3.},
\VS{17}si j'ai mangé seul mon morceau de pain, sans que l'orphelin en ait sa part,
\VS{18}moi qui l'ai dès ma jeunesse fait grandir près de moi comme un père, et qui dès le sein de ma mère, ai été le guide de la veuve ;
\VS{19}si j'ai vu le malheureux périr faute de vêtements, le pauvre manquer de couverture\FTNT{Mt. 25:41-45.},
\VS{20}sans que ses reins m'aient béni, sans qu'il ait été réchauffé par la toison de mes agneaux ;
\VS{21}si j'ai levé la main contre l'orphelin, parce que je me voyais comme un appui dans les portes\FTNT{Pr. 22:22.} ;
\VS{22}que mon épaule tombe de sa jointure, que mon bras tombe et qu'il se brise l'os !
\VS{23}Car les châtiments de Dieu m'épouvantent, et je ne pourrais pas prévaloir devant sa majesté.
\VS{24}Si j'ai mis dans l'or ma confiance, si j'ai dit à l'or fin : Tu es mon espoir\FTNT{Mc. 10:24 ; 1 Ti. 6:17.} ;
\VS{25}si je me suis réjoui de ma grande puissance, de la quantité des richesses que ma main a acquise\FTNT{Ps. 62:11.} ;
\VS{26}si j'ai regardé le soleil quand il brillait, la lune quand elle s'avançait de façon majestueuse,
\VS{27}et si mon cœur s'est laissé secrètement séduire, si ma main a envoyé des baisers de ma bouche ;
\VS{28}c'est encore une iniquité que doit punir le juge, et j'aurais renié le Dieu d'en haut !
\VS{29}Si je me suis réjoui du malheur de mon ennemi, si j'ai sauté d'allégresse quand le mal l'a atteint\FTNT{Mt. 5:43-44.},
\VS{30}moi qui n'ai pas permis à ma langue de pécher en demandant sa mort par des malédictions ;
\VS{31}si les gens de ma tente ne disaient pas : Où est celui qui n'a pas été rassasié de sa viande\FTNT{Ps. 27:2.} ?
\VS{32}Si l'étranger passait la nuit dehors, si je n'ouvrais pas ma porte au voyageur\FTNT{Ge. 19:1-2 ; De. 10:19 ; 1 Pi. 4:9 ; Hé. 13:2.} ;
\VS{33}si, comme les hommes, j'ai caché mes transgressions et mon crime dans mon sein\FTNT{Ge. 3:10-12 ; Pr. 28:13.},
\VS{34}parce que je craignais la multitude, et je craignais le mépris des familles, en sorte que je restais tranquille et n'osais franchir ma porte…
\VS{35}Oh ! Qui me fera trouver quelqu'un qui m'écoute ? Voilà ma défense toute signée : Que le Tout-Puissant me réponde ! Qui me donnera la plainte écrite par mon adversaire ?
\VS{36}Je porterai son écrit sur mon épaule, je l'attacherai sur mon front comme une couronne ;
\VS{37}je lui déclarerai le nombre de mes pas, je m'approcherai de lui comme un prince.
\VS{38}Si ma terre crie contre moi, et que ses sillons pleurent ;
\VS{39}si j'en ai mangé le produit sans l'avoir payée, et que j'aie attristé l'âme de ses anciens maîtres ;
\VS{40}Qu'elle en produise des épines au lieu du froment, et de l'ivraie au lieu de l'orge ! C'est ici la fin des paroles de Job.
\Chap{32}
\TextTitle{Discours d'Elihu : reproches à Job et à ses amis}
\VerseOne{}Ces trois hommes-là cessèrent de répondre à Job, parce qu'il se regardait comme juste.
\VS{2}Élihu, fils de Barakeel de Buz, de la famille de Ram, s'enflamma de colère contre Job, parce qu'il disait son âme juste devant Dieu.
\VS{3}Et sa colère s'enflamma contre ses trois amis, parce qu'ils ne trouvaient rien à répondre et que néanmoins ils condamnaient Job.
\VS{4}Comme ils étaient plus âgés que lui, Elihu avait attendu jusqu'à ce moment pour parler à Job.
\VS{5}Mais, voyant que ces trois hommes n'avaient plus aucune réponse à la bouche, Elihu se mit en colère.
\VS{6}Et Elihu, fils de Barakeel de Buz, prit la parole et dit : Je suis jeune et vous êtes des vieillards ; c'est pourquoi j'ai craint, j'ai eu peur de vous faire connaître mon sentiment.
\VS{7}Je disais: les jours parleront, et le grand nombre des années fera connaître la sagesse.
\VS{8}L'esprit dans l'homme, c'est l'esprit, le souffle du Tout-Puissant qui rend intelligent\FTNT{Da. 1:17 ; Da. 2:21 ; Pr. 2:6 ; Ec. 2:26.} ;
\VS{9}ce ne sont pas les aînés qui sont sages, ce ne sont pas les vieillards qui comprennent ce qui est juste.
\VS{10}C'est pourquoi je dis : Ecoute-moi ! Et je dirai aussi ma pensée.
\VS{11}J'ai attendu la fin de vos discours, j'ai écouté vos raisonnements, jusqu'à ce que vous ayez bien examiné les discours de Job.
\VS{12}J'ai pris le soin de vous écouter ; et voici, aucun de vous n'a convaincu Job, aucun n'a répondu à ses paroles.
\VS{13}Qu'il ne vous n'arrive pas de dire: nous avons trouvé la sagesse ; c'est Dieu qui le poursuit, et non pas l'homme !
\VS{14}Il n'a pas dirigé ses discours contre moi : Aussi je ne lui répondrai pas à votre manière.
\VS{15}Ils sont étonnés! Ils ne répondent plus rien! On leur a ôté la parole!
\VS{16}J'ai attendu jusqu'à ce qu'ils n'ont plus rien dit, car ils sont demeurés muets, et ils n'ont plus su que répondre.
\VS{17}A mon tour, je veux répondre pour moi, et je veux donner mon avis.
\VS{18}Car je suis rempli de discours, l'esprit qui est en mon sein me presse.
\VS{19}Mon sein est comme vin sans air, comme des outres neuves qui vont éclater\FTNT{Mt. 9:17 ; Mc. 2:22 ; Lu. 5:38.}.
\VS{20}Je parlerai pour respirer à l'aise, j'ouvrirai mes lèvres et je répondrai.
\VS{21}Je ne ferai pas acception de personnes, et je flatterai aucun homme.
\VS{22}Car je ne sais pas flatter : Mon Créateur m'enlèverait bien vite.
\Chap{33}
\TextTitle{Discours d'Elihu : la justice de Dieu}
\VerseOne{}C'est pourquoi Job, écoute mon discours, je te prie et prête l'oreille à toutes mes paroles !
\VS{2}Voici, j'ouvre la bouche, ma langue parle dans mon palais.
\VS{3}Mes paroles exprimeront la droiture de mon cœur, mes lèvres diront la vérité pure.
\VS{4}L'Esprit de Dieu m'a fait, et le souffle du Tout-Puissant me donne la vie\FTNT{Ge. 2:7.}.
\VS{5}Si tu peux, réponds-moi, dresse-toi contre moi, demeure ferme!
\VS{6}Voici, je suis pour le Dieu fort, selon que tu en as parlé; j'ai été formé de la terre tout comme toi.\FTNT{Ac. 14:15.} ;
\VS{7}Voici ma terreur ne te trouble pas, et ma main ne s'appesantit pas sur toi
\VS{8}Quoi qu'il en soit, tu as dit, moi l'entendant, j'ai entendu la voix de tes discours:
\VS{9}Je suis pur, sans péché, je suis net, il n'y a pas d'iniquité en moi.
\VS{10}Voici il cherche à rompre avec moi, il me considère comme son ennemi ;
\VS{11}il met mes pieds dans les ceps, il surveille tous mes chemins.
\VS{12}Je te répondrai qu'en cela tu n'as pas été juste, car Dieu sera toujours plus grand que l'homme.
\VS{13}Pourquoi as-tu donc plaidé contre lui ? Car il ne rend pas compte de toutes ses actions.
\VS{14}Car Dieu parle une première fois, et une seconde fois à celui qui n'aura pas pris garde à la première.
\VS{15}Par des songes, par des visions nocturnes, quand les hommes tombent dans un profond sommeil, quand ils dorment sur leur couche.
\VS{16}Alors il ouvre l'oreille de l'homme d'une mauvaise action et de rabaisser la fierté de l'homme.
\VS{17}afin de détourner l'homme de son œuvre et de le préserver de l'orgueil,
\VS{18}il garantit son âme de la fosse, et sa vie de l'épée.
\VS{19}L'homme est aussi châtié par des douleurs sur son lit, à cause d'une lutte perpétuelle en ses os\FTNT{Ps. 38:4.}.
\VS{20}Alors sa vie prend en horreur le pain et son âme les mets les plus désirés\FTNT{Ps. 107:18.} ;
\VS{21}Sa chair est tellement consumée qu'elle paraît plus, ses os sont tellement brisés, qu'on y connaît plus rien;
\VS{22}son âme s'approche de la fosse, et sa vie des messagers de la mort.
\VS{23}Mais s'il y a pour cet homme un messager qui interprète, un d'entre les mille, pour lui annoncer la voie de la droiture,
\VS{24}alors Dieu prend pitié de lui et dit : Garantis-le, afin qu'il ne descende pas dans la fosse ; j'ai trouvé la propitiation!
\VS{25}Sa chair devient plus délicate qu'elle n'était dan son enfance; il revient aux jours de sa jeunesse.
\VS{26}Il supplie Dieu par ses prières, et Dieu lui est favorable, il lui laisse voir sa face avec joie, et lui rend sa justice\FTNT{Es. 58:9.}.
\VS{27}Il regarde vers les hommes et dit : J'ai péché, j'ai violé la justice, et je n'ai pas été puni comme je le méritais ;
\VS{28}Dieu a racheté mon âme afin qu'elle ne passe pas dans la fosse, et ma vie voit encore la lumière !
\VS{29}Voilà ce que Dieu fait, deux fois, trois fois, envers l'homme\FTNT{Ps. 62:11.},
\VS{30}pour ramener son âme de la fosse, pour l'éclairer de la lumière des vivants\FTNT{Ps. 56:14.}.
\VS{31}Sois attentif, Job, écoute-moi ! Tais-toi, et je parlerai !
\VS{32}Si tu as quelque chose à dire, réponds-moi ! Parle, car je désire te justifier.
\VS{33}Sinon, écoute, tais-toi et je t'enseignerai la sagesse.
\Chap{34}
\TextTitle{Discours d'Elihu : il accuse Job de se révolter}
\VerseOne{}Elihu reprit la parole, et dit :
\VS{2}Sages, écoutez mes discours ! Vous qui avez la connaissance, prêtez-moi l'oreille !
\VS{3}Car l'oreille discerne les discours, comme le palais savoure ce qu'il mange.
\VS{4}Choisissons ce qui est juste, voyons entre nous ce qui est bon.
\VS{5}Job dit : Je suis juste, et Dieu a écarté ma justice;
\VS{6}mentirai-je à mon droit? Ma flèche est mortelle sans que j'aie commis de crime.
\VS{7}où y a-t-il un homme comme Job, qui boit le péché la moquerie comme l'eau,
\VS{8}qui marche en compagnie des ouvriers d'iniquité, et qui fréquente  avec les hommes marchant de pair avec les hommes méchants ?
\VS{9}Car il a dit : Il est inutile à l'homme de plaire à Dieu\FTNT{Mal. 3:14.}.
\VS{10} C'est pourquoi écoutez, vous qui avez de l'intelligence, écoutez-moi ! Loin de Dieu la méchanceté, loin du Tout-Puissant l'injustice\FTNT{De. 32:4 ; Ps. 92:16 ; Ro. 9:14.} !
\VS{11}Car il rend à l'homme selon son œuvre, il fait trouver à chacun selon sa voie\FTNT{Jé. 17:10 ; Jé. 32:19 ; Ez. 7:27 ; Pr. 24:12 ; Mt. 16:27 ; Ro. 2:6 ; 2 Co. 5:10 ; Ep. 6:8 ; Ap. 22:12.}.
\VS{12}Certes, Dieu ne commet pas l'injustice ; le Tout-Puissant ne renverse pas le droit.
\VS{13}Qui lui a donné la terre en charge ? Ou Qui a placé la terre habitable?
\VS{14}S'il ne pensait qu'à lui-même, s'il retirait à lui son esprit et son souffle\FTNT{Ps. 104:29.},
\VS{15}toute chair périrait ensemble, et l'homme retournerait dans la poussière\FTNT{Ge. 3:19 ; Ec. 3:20 ; Ec. 12:9.}.
\VS{16}Si donc tu as de l'intelligence, écoute ceci, prête l'oreille à ce que tu entendras de moi.
\VS{17}Comment celui qui n'aimerait pas à faire la justice jugerait-il le monde? Et condamneras-tu celui qui est souverainement juste ?
\VS{18}Dira-t-on à un roi, qu'il est un scélérat ? Et aux princes, qu'ils sont des méchants ?
\VS{19}Combien moins le dira-t-on à celui qui n'a point d'égard à la personne des grands, et qui ne connaît point les riches pour les préférer aux pauvres, parce qu'ils sont tous l'ouvrage de ses mains\FTNT{De. 10:17 ; 2 Ch. 19:7 ; Ac. 10:34 ; Ga. 2:6 ; Ro. 2:11 ; Ep. 6:9 ; Col. 3:25.} ?
\VS{20}En un moment, ils mourront ; au milieu de la nuit, un peuple est ébranlé et passe ; le puissant s'en va, sans la main d'aucun homme.
\VS{21}Car les yeux de Dieu sont sur les voies de l'homme, il regarde tous ses pas.
\VS{22}Il n'y a ni ténèbres ni ombre de la mort où puissent se cacher les ouvriers d'iniquité.
\VS{23}Dieu ne regarde pas à deux fois un homme, pour le faire aller en jugement avec lui.
\VS{24}Il brise les puissants par des voies incompréhensibles, et il n établit d'autres à leur place ;
\VS{25}car il connaît leurs œuvres. Il les renverse de nuit, et ils sont écrasés ;
\VS{26}il les frappe comme des impies, au lieu où se tiennent tous les regards.
\VS{27}Du fait qu'ils se sont détournés de lui, et qu'ils n'ont considéré aucune de ses voies.
\VS{28}ils ont fait monter à Dieu le cri du pauvre, et il a entendu le cri des affligés\FTNT{Ja. 5:4.}.
\VS{29}S'il donne le repos, qui est-ce qui causera du trouble? S'il cache sa face à quelqu'un, qui le regardera, qu'il s'agisse de toute une nation ou d'un seul un homme?
\VS{30}afin que l'hypocrite ne règne pas, de peur qu'il plus un piège pour le peuple.
\VS{31}Car a-t-il jamais dit à Dieu : J'ai été pardonné, je ne pécherai plus ;
\VS{32}montre-moi ce que je ne vois pas ; si j'ai fait le mal, je ne le ferai plus ?
\VS{33}Mais Dieu ne te le rendra-t-il pas, puisque tu as rejeté son châtiment, quand tu as fait le choix que tu as fait? Pour moi, je ne sais que dire à cela; mais toi, si tu as quelque chose à répondre, parle.
\VS{34}Les gens de bon sens diront avec moi, et tout homme sage en conviendra,
\VS{35}que Job ne parle pas avec connaissance, et ses paroles manquent d'intelligence.
\VS{36}Ah! Mon père, que Job soit éprouvé jusqu'à ce qu'il soit vaincu, puisqu'il vaincu, puisqu'il répond comme les impies.
\VS{37}Car il ajoute péché sur péché; il applaudit au milieu de nous ; il parle de plus en plus contre Dieu.
\Chap{35}
\TextTitle{Discours d'Elihu : il reproche à Job ses propos irréfléchis}
\VerseOne{}Elihu reprit la parole et dit :
\VS{2}Penses-tu avoir raison de dire : Je suis juste devant Dieu ?
\VS{3}Quand tu dis : Que me sert-il, et que gagnerais-je de plus sans pécher ?
\VS{4}Je te répondrai en ces termes, et à tes amis qui sont avec toi.
\VS{5}Regarde les cieux, et considère-les ! Vois les nuées, comme elles sont plus hautes que toi !
\VS{6}Si tu pèches, quel mal fais-tu à Dieu ? Et si tes péchés se multiplient, quel mal reçoit-il ?
\VS{7}Si tu es juste, que lui donnes-tu ? Que reçoit-il de ta main ?
\VS{8}C'est à un homme, comme tu es, que ta méchanceté peut seule nuire, et c'est au fils d'un homme que ta justice peut seule être utile.
\VS{9}On fait crier les opprimés par la grandeur des maux qu'on leur afflige; ils crient à cause de la violence des grands;
\VS{10}Et nul ne dit : Où est le Dieu qui m'a fait, qui donne de quoi chanter pendant la nuit ,
\VS{11}qui nous instruit plus que les animaux de la terre, et plus intelligent que les oiseaux des cieux ?
\VS{12}On crie donc à cause de la fierté des méchants; mais Dieu ne l'exauce pas.
\FTNT{Es. 1:15 ; Ez. 8:18 ; Mi. 3:4 ; Jn. 9:31.}.
\VS{13} Cependant que ce soit en vain; que Dieu n'écoute pas, et que le Tout-puissant n'y a pas égard.
\VS{14}Encore moins dois-tu lui dire, tu ne le vois pas; car le jugement est devant lui, attends-le donc!
\VS{15}Mais maintenant, ce n'est rien ce que sa colère exécute, et il n'a pas encore pris connaissance en profondeur toutes les choses que tu as faites.
\VS{16}Job ouvre donc sa bouche pour se plaindre, il multiplie les paroles sans intelligence.
\Chap{36}
\TextTitle{Discours d'Elihu : Dieu traite les hommes selon leurs oeuvres}
\VerseOne{}Elihu continua de parler, et dit :
\VS{2}Attends un peu, et je te montrerai qu'il y a encore d'autres raisons pour la cause de Dieu.
\VS{3}Je tirerai de loin mes raisons, et je défendrai la justice du Créateur.
\VS{4}Car certainement il n'y aura rien de faux en tout ce que je dirai, et celui qui est avec toi, est parfait dans sa connaissance.
\VS{5}Dieu est puissant, mais il ne méprise personne ; il est puissant par la force de son coeur.
\VS{6}Il ne laisse pas vivre le méchant, et il fait droit aux pauvres.
\VS{7}Il ne détourne pas ses yeux de dessus les justes, il les place sur le trône avec les rois, il les y fait asseoir pour toujours, afin qu'ils soient élevés\FTNT{Ps. 33:18 ; Ps. 34:16.}.
\VS{8}S'ils sont liés de chaînes, s'ils sont pris dans les liens de l'affliction,
\VS{9}il leur fait connaître leurs œuvres, leurs transgressions, leur orgueil.
\VS{10}Alors il ouvre leur oreille pour leur discipline, il leur dit de se détourner de l'iniquité.
\VS{11}S'ils écoutent, et s'ils le servent, ils achèvent leurs jours dans le bonheur, leurs années dans la joie.
\VS{12}S'ils n'écoutent pas, ils passent par l'épée, ils expirent dans leur aveuglement.
\VS{13}Ceux qui sont hypocrites dans leur cœur, ils ne crient pas à lui quand il les a liés ;
\VS{14}leur personne meurt dans sa jeunesse, leur vie s'éteint parmi les débauchés.
\VS{15}Mais Dieu sauve celui qui est affligé de son oppression, et c'est par la détresse qu'il lui ouvre les oreilles.
\VS{16}Il t'écartera aussi de la détresse, pour te mettre au large, loin de toute angoisse, et ta table sera chargée de viandes grasses\FTNT{Ps. 50:15 ; Ps. 63:6.}.
\VS{17}Or tu remplis le jugement du méchant, mais le jugement et le droit subsisteront.
\VS{18}Certainement Dieu et irrité; prends garde qu'il ne te  plonge dans l'affliction, car il n'y aura pas alors de rançon si grande pour te délivrer\FTNT{Ps. 49:8.} !
\VS{19}Tes cris valent-ils ton or, et même toutes les forces qui se trouvent dans tes richesses ?
\VS{20}Ne soupire pas après la nuit, qui enlève les peuples de leur place.
\VS{21}Garde-toi de te retourner vers l'iniquité, car la souffrance t'y dispose.
\VS{22}Dieu est élevé par sa puissance ; qui saurait enseigner comme lui ?
\VS{23}Qui lui prescrit le chemin qu'il devait tenir? Qui lui dit : Tu a fait une injustice ?
\VS{24}Souviens-toi de célébrer ses ouvrages, que tous les hommes voient.
\VS{25}Tout homme les voit, chacun les contemple de loin.
\VS{26}Dieu est grand, mais nous ne le connaissons pas, quant au nombre de ses années il est insondable\FTNT{Es. 63:16 ; Ps. 92:8 ; Ps. 93:2 ; Ps. 102:13 ; La. 5:19.}.
\VS{27}Parce qu'il met les eaux  en petites gouttes, elle répandent la pluie selon la vapeur d'eau qui la contient ;
\VS{28}les nuées la font dégoutter, elles coulent sur les hommes en abondance.
\VS{29}Et qui pourra comprendre l'étendue des nuages et le son éclatant de sa tente ?
\VS{30}Voici, il étend sa lumière sur elle, et il se cache jusque dans les profondeurs de la mer.
\VS{31} Or c'est par ces choses qu'il juge les peuples, qu' il donne la nourriture en abondance.
\VS{32} Il tient caché dans les paumes de ses mains le feu étincelant, et lui ordonne de frapper de ce qui se présente à sa rencontre.
\VS{33} Son bruit porte les nouvelles ; les troupeaux font connaître qu'il approche.
\Chap{37}
\TextTitle{Discours d'Elihu : conclusion}
\VerseOne{}Mon cœur même à cause de cela est tout tremblant, il sort de sa place.
\VS{2}Ecoutez attentivement et en tremblant le bruit de sa voix, le grondement qui sort de sa bouche\FTNT{Ps. 29:3-9.} !
\VS{3}Il le conduit dans toute l'étendue des cieux, et son éclair brille jusqu'aux extrémités de la terre\FTNT{Ps. 97:4.}.
\VS{4}Après lui de l'élève un grand bruit, il tonne de sa voix majestueuse; il ne tarde pas après que sa voix a été entendue\FTNT{Jé. 10:13.}.
\VS{5}Dieu tonne avec sa voix d'une manière étonnante ; il fait de grandes choses que nous ne comprenons pas.
\VS{6}Car il dit à la neige : Tombe sur la terre ! Il le dit à la pluie, même aux plus fortes pluies.
\VS{7}Il met un sceau à la main de tous les hommes pour reconnaître tous les hommes qui sont son ouvrage.
\VS{8}Les bêtes entrent dans leurs tanières, et elles demeurent dans leurs repaires.
\VS{9}L'ouragan vient du fond du sud, et le froid vient des vents du nord.
\VS{10}Par son souffle, Dieu donne la glace, et il réduit l'espace où se répondaient au large les eaux\FTNT{Ps. 147:17-18.}.
\VS{11}Il lasse les nuages à force d'arroser, il écarte les nuages par sa lumière.
\VS{12}Et ceux-ci font plusieurs tours pour faire ce qu'il a commandé, sur la face de la terre sur la face de la terre habitée ;
\VS{13}Il les fait venir pour s'en servir soit comme une verge pour la terre, soit pour répandre ses bienfaits\FTNT{Ex. 9:18-23 ; 1 S. 12:18-19.}.
\VS{14}Job, arrête-toi, prête l'oreille à ces choses ! Considère encore les merveilles de Dieu !
\VS{15}Sais-tu comment Dieu les dispose, et fait briller la lumière de ses nuages?
\VS{16}Connais-tu le balancement des nuages, les merveilles de celui dont la science est parfaite ?
\VS{17}Sais-tu pourquoi tes vêtements sont chauds quand la terre se repose par le vent du midi ?
\VS{18}Peux-tu étendre avec lui les cieux, aussi fermes qu'un miroir de fonte ?
\VS{19}Montre-nous ce que nous pouvons lui dire ; car nous ne saurions rien dire par ordre à cause de nos ténèbres. 
\VS{20}Lui racontera-t-on quand je parlerai ? S'il y a un homme qui en parle, certainement il en sera englouti ?
\VS{21}Et maintenant on ne voit pas la lumière du soleil qui resplendit dans les cieux, lorsque le vent passe et le nettoie ;
\VS{22}Le temps qui la reluit comme l'or vient du nord. Il y a en Dieu une majesté redoutable.
\VS{23}Nous ne saurions comprendre le Tout-Puissant, grand en puissance, en jugement et en abondante justice, il n'opprime personne !
\VS{24}C'est pourquoi les hommes le craignent ; mais il ne les voit pas tous sages de cœur\FTNT{Ps. 92:7 ; Ro. 1:21.}.
\Chap{38}
\TextTitle{Yahweh interroge Job}
\VerseOne{}Yahweh répondit à Job du milieu de le tourbillon et dit :
\VS{2}Qui est celui qui obscurcit mes décisions par des paroles sans connaissance ?
\VS{3}Ceins maintenant tes reins comme un vaillant homme ; je t'interrogerai, et tu me fera voir ta science.
\VS{4}Où étais-tu quand je fondais la terre ? Dis-le, si tu as de l'intelligence\FTNT{Pr. 8:29.}.
\VS{5}Qui en a réglé les mesures, le sais-tu ? Ou qui a appliqué sur elle le niveau ?
\VS{6}Sur quoi ses bases sont-elles plantées ? Ou qui en a posé la pierre angulaire pour la soutenir\FTNT{Ps. 104:5.}?
\VS{7}Quand les étoiles du matin se réjouissent ensemble, et que tous les fils de Dieu poussent des cris de joie \FTNT{Ps. 148:3.} ?
\VS{8}Qui a renfermé la mer dans ses bords, quand elle fut tirée de la matrice et qu'elle sortit? 
\VS{9}quand je lui donnai la nuée pour vêtement, et l'obscurité pour langes ;
\VS{10}que je lui imposai ma loi, et que je lui mis des barrières et des portes;
\VS{11} et quand je dis : Tu viendras jusqu'ici, tu n'iras pas plus loin ; ici s'arrêtera l'orgueil de tes flots ?
\VS{12}Depuis que tu es au monde, as-tu commandé au matin et as-tu montré à l'aube du jour le lieu où elle doit se lever,
\VS{13}pour qu'elle saisisse les extrémités de la terre, et que les méchants en soient chassés ;
\VS{14}pour que la terre prenne une forme comme l'argile qui reçoit un sceau, et qu'elle soit parée comme d'un vêtement nouveau ;
\VS{15}pour que la lumière soit ôtée aux méchants, et que le bras qui se lève soit brisé\FTNT{Ps. 10:15.} ?
\VS{16}As-tu pénétré jusqu'aux sources de la mer ? T'es-tu promené dans les profondeurs de l'abîme ?
\VS{17}Les portes de la mort se sont-elles découvertes à toi ? As-tu vu les portes de l'ombre de la mort ?
\VS{18}As-tu compris l'étendue de la terre ? Si tu sais tout cela, dis-le !
\VS{19}Où est la demeure de la lumière, et où est le lieu des ténèbres ?
\VS{20}Pour que tu les prennes à leur limite, et que tu connaisses le chemin de leur maison ?
\VS{21}Tu le sais, car alors tu étais né, et le nombre de tes jours est grand !
\VS{22}Es tu entré dans les trésors de la neige ? As-tu vu les trésors de grêle,
\VS{23}que je réserve pour les temps de détresse, pour les jours de guerre et de bataille\FTNT{Ex. 9:23 ; Jos. 10:11 ; Ap. 8 :7.} ?
\VS{24}Par quel chemin la lumière se partage la lumière, et le vent d'orient se répand-il sur la terre\FTNT{Jn. 3:8.} ?
\VS{25}Qui a ouvert un conduit aux inondations, et tracé la route de l'éclair et du tonnerre,
\VS{26}pour qu'elle pleuve sur une terre sans habitants, sur un désert sans hommes\FTNT{Ps. 104:13-14 ; PS. 147:8 ; Ac. 14:17.} ;
\VS{27}pour qu'elle abreuve les lieux solitaires et arides, et qu'elle fasse germer et sortir l'herbe ?
\VS{28}La pluie a-t-elle un père ? Qui enfante les gouttes de la rosée ?
\VS{29}De quel sein est sortie la glace ? Et qui enfante le givre du ciel,
\VS{30}pour que les eaux se cachent comme une pierre, et que le dessus de l'abîme soit enchaîné ?
\VS{31}Peux-tu resserer les liens des pléiades ou détacher les chaînes d'orient\FTNT{Am. 5:8.}?
\VS{32}Fais-tu sortir en leur temps les signes du zodiaque, et conduis-tu la Grande Ourse avec ses petits ?
\VS{33}Connais-tu les lois du ciel ? Disposes-tu de son pouvoir sur la terre\FTNT{Jé. 31:35-36 ; Ps. 104:4.} ?
\VS{34}Élèves-tu la voix jusqu'aux nuées, pour que des eaux abondantes te couvrent ?
\VS{35}Envoies-tu les éclairs ? Partent-ils ? Te disent-ils : Nous voici ?
\VS{36}Qui a mis la sagesse dans le cœur, ou qui a donné l'intelligence à l'esprit\FTNT{Ec. 2:26.} ?
\VS{37}Qui Est-ce qui peut avec intelligence compter les nuages, et pour placer les outres des cieux,
\VS{38}Quand la poussière est détrompée par les eaux qui l'arrosent, et que les mottes viennent à se joindre ?
\Chap{39}
\TextTitle{Yahweh démontre son omnipotence}
\VerseOne{}Chasses-tu de la proie pour la lionne, et apaises-tu la faim des lionceaux\FTNT{Ps. 104:21.},
\VS{2}Quand ils se tapissent dans leurs tanières et se tiennent aux aguets dans leur repaire ?
\VS{3}Qui est-ce qui apprête la nourriture au corbeau, quand ses petits crient à Dieu, et qu'ils vont errants, parce qu'ils n'ont point de quoi manger\FTNT{Ps. 104:27 ; Ps. 147:9 ; Mt. 6:26.} ?
\VS{4}Sais-tu quand les boucs de rochers mettent bas ? Observes-tu les biches de rochers quand elles font leurs petits\FTNT{Ps. 29:9.} ?
\VS{5}Comptes-tu les mois de leur gestation, et sais-tu le temps auquel elles font leurs petits ?
\VS{6}Et qu'elles se courbent pour mettre bas leurs petits et se délivrent de leurs douleurs ?
\VS{7}Leurs petits se fortifient, ils croissent en plein air, ils s’en vont et ne reviennent plus vers elles.
\VS{8}Qui a laissé aller libre l’âne sauvage ? et qui a délié les liens de l’âne farouche ?
\VS{9}Auquel j’ai donné le désert pour maison, et la terre inhabitée pour ses retraites\FTNT{Jé. 2:24.} ?
\VS{10}Il se rit du bruit des villes, il n'entend pas les cris d'un exacteur.
\VS{11}Les montagnes qu'il va épiant çà et là, sont ses pâturages, et il cherche toute sorte de verdure. 
\VS{12}Le buffle voudra-t-il te servir, ou demeurera-t-il à ta crèche ? 
\VS{13}Lies-tu le buffle avec son licou pour labourer ? ou rompra-t-il les mottes des vallées après toi ? 
\VS{14}Te fies-tu à lui parce que sa force est grande, et lui abandonnes-tu ton travail? 
\VS{15}Comptes-tu sur lui pour rentrer ta semence, et pour l'amasser sur ton aire ? 
\VS{16}As-tu donné aux paons ce plumage qui est si brillant, ou à l'autruche les ailes et les plumes ? 
\VS{17}Néanmoins elle abandonne ses oeufs à terre, et les fait échauffer sur la poussière ;
\VS{18}et elle oublie que le pied peut les écraser, ou que les bêtes des champs peuvent les fouler. 
\VS{19}Elle est dure envers ses petits, comme s’ils n’étaient pas siens. Son travail est vain, elle ne s’en inquiète pas.
\VS{20}Car Dieu l'a privée de sagesse et ne lui a pas donné l'intelligence.
\VS{21}A la première occasion elle se dresse en haut, et se moque du cheval et de celui qui le monte. 
\VS{22}As-tu donné la force au cheval ? et as-tu revêtu son cou d'un hennissement éclatant comme le tonnerre ? 
\VS{23}Fais-tu bondir le cheval comme la sauterelle ? le son magnifique de ses narines est effrayant.
\VS{24}Il creuse la terre de son pied, il s'égaie dans sa force, il va à la rencontre d'un homme armé ;
\VS{25}Il se rit de la frayeur, il ne s'épouvante de rien, et il ne se détourne point de devant l'épée.
\VS{26}Il n'a point peur des flèches qui sifflent tout autour de lui, ni du fer luisant de la lance et du javelot. 
\VS{27}Il creuse la terre, plein d'émotion et d'ardeur au son de la trompette, et il ne peut se retenir. 
\VS{28}Au son bruyant de la trompette, il dit : En avant ! En avant ! Il flaire de loin la bataille, le tonnerre des capitaines, et le cri de triomphe.
\VS{29}Est-ce par ta sagesse que l'épervier prend son vol, et qu'il étend ses ailes vers le midi ?
\VS{30}Est-ce par ton commandement que l'aigle s'élève, et qu'il place son nid sur les hauteurs\FTNT{Jé. 49:16 ; Abd. 1:4.} ?
\VS{31}Elle habite sur les rochers, et elle s'y tient ; même sur les sommets des rochers et dans des lieux forts. 
\VS{32}De là il découvre le gibier, ses yeux voient de loin.
\VS{33}Ses petits auss sucent le sang ; et là où sont des cadavres, il s'y trouve aussitôt\FTNT{Mt. 24:28 ; Lu. 17:37.}.
\TextTitle{Yahweh lui pose une question}
\VS{34}Yahweh prit encore la parole et dit à Job :
\VS{35}Celui qui conteste avec le Tout-puissant, lui apprendra-t-il quelque chose ? Que celui qui dispute avec Dieu réponde à ceci.
\TextTitle{Réponse de Job}
\VS{36}Alors Job répondit à Yahweh et dit :
\VS{37}Voici, je suis un homme vil ; que te répondrais-je ? Je mets ma main sur ma bouche\FTNT{Ps. 39:10.}.
\VS{38}J'ai parlé une fois, mais je ne répondrai plus ; j'ai même parlé deux fois mais je n'ajouterai plus.
\Chap{40}
\TextTitle{Yahweh questionne encore Job}
\VerseOne{}Et Yahweh répondit à Job du milieu d'un tourbillon, et lui dit :
\VS{2}Ceins maintenant tes reins comme un vaillant homme ; je t'interrogerai et tu m'enseigneras.
\VS{3}Anéantiras-tu mon jugement ? me condamneras-tu pour te justifier\FTNT{Ps. 51:6 ; Ro. 3:4.} ?
\VS{4}As-tu un bras comme celui de Dieu ; tonnes-tu de la voix, comme lui ?
\VS{5}Pare-toi maintenant de magnificence et de grandeur, et revêts-toi de majesté et de gloire.
\VS{6}Répands les fureurs de ta colère, d'un regard, humilie tous les orgueilleux
\VS{7}D'un regard humilie les orgueilleux, écrase sur place les méchants,
\VS{8}cache-les tous ensemble dans la poussière, enferme leur face dans les ténèbres !
\VS{9}Alors je rends hommage à mon sauveur qui me sauve par sa droite.
\VS{10}Voici le béhémoth, que j'ai façonné comme toi ! Il mange de l'herbe comme le bœuf.
\VS{11}Regarde donc, sa force est dans ses reins, et sa puissance dans les muscles de son ventre ;
\VS{12}il plie sa queue aussi ferme qu'un cèdre ; les tendons de ses cuisses sont entrelacés ;
\VS{13}ses os sont des tubes d'airain, ses membres sont comme des barres de fer.
\VS{14}C’est le chef-d’œuvre de Dieu ; celui qui l’a fait lui a donné son épée.
\VS{15}Car les montagnes lui apportent sa pâture, là où se jouent toutes les bêtes des champs.
\VS{16}Il se couche sous les lotus, caché dans les roseaux et les marécages ;
\VS{17}les lotus le couvrent de leur ombre, les saules du torrent l'enveloppent.
\VS{18}Voilà, il engloutit une rivière en buvant, et il ne s'en retire pas vite ; et il ne s'étonnerait pas quand le Jourdain se dégorgerait dans sa gueule. 
\VS{19}Il l'engloutit en le voyant, et son nez passe au travers des empêchements qu'il rencontre.  
\VS{20}Attireras-tu le léviathan à l'hameçon ? Saisiras-tu sa langue avec une corde ?
\VS{21}Mettras-tu un jonc dans ses narines ? Lui perceras-tu la mâchoire avec un crochet ?
\VS{22}Accumulera-t-il les supplications ? Te parlera-t-il d'une voix douce ?
\VS{23}Fera-t-il une alliance avec toi, pour te prendre pour toujours comme esclave ?
\VS{24}Joueras-tu avec lui comme avec un oiseau ? L'attacheras-tu pour amuser les jeunes filles ?
\VS{25}Les pêcheurs en trafiquent-ils ? Le partagent-ils entre les marchands ?
\VS{26}Couvriras-tu sa peau de dards, et sa tête de harpons ?
\VS{27}Mets ta main contre lui, et tu ne te souviendras plus de l'attaquer.
\VS{28}Voici, on est trompé dans son attente ; à sa vue n'est-on pas terrassé ?
\Chap{41}
\VerseOne{}Nul n'est assez féroce pour l'exciter ; qui donc me résisterait en face ?
\VS{2}De qui suis-je le débiteur ? Je le paierai. Sous le ciel tout m'appartient\FTNT{Ex. 19:5 ; De. 10:14 ; Ps. 24:1 ; Ps. 50:12 ; 1 Co. 10:26 ; Ro. 11:35.}.
\VS{3}Je veux encore parler de ses discours, et de sa force, et de la beauté de sa structure.
\VS{4}Qui découvrira son vêtement devant ma face ? Qui viendra freiner ses mâchoires par un mors ?
\VS{5}Qui ouvrira les portes devant sa face ? Autour du lion habite la terreur.
\VS{6}Ses magnifiques et puissants boucliers sont fermés comme un sceau ;
\VS{7}ils se serrent l'un contre l'autre, et l'air n'entrerait pas entre eux ;
\VS{8}ce sont des frères qui s'embrassent, se saisissent, demeurent inséparables.
\VS{9}Ses éternuements font briller la lumière, ses yeux sont comme les paupières de l'aurore.
\VS{10}Des flammes viennent de sa bouche, des étincelles de feu s'en échappent.
\VS{11}Une fumée sort de ses narines, comme d'un chaudron qui bout, d'une chaudière ardente.
\VS{12}Son souffle allume les charbons, de sa bouche sort la flamme.
\VS{13}La force a son cou pour demeure, et l'effroi bondit devant lui.
\VS{14}Ses parties charnues sont jointes ensemble, fondues sur lui, inébranlables.
\VS{15}Son cœur est dur comme la pierre, dur comme la meule inférieure.
\VS{16}Quand il se lève, les plus vaillants ont peur, et l'épouvante les fait quitter le droit chemin.
\VS{17}C'est en vain qu'on l'attaque avec l'épée ; la lance, le javelot, la cuirasse ne servent à rien.
\VS{18}Il regarde le fer comme de la paille, l'airain comme du bois pourri.
\VS{19}La flèche de l'arc ne le met pas en fuite, les pierres de la fronde sont pour lui changés en chaume.
\VS{20}Il ne voit dans la massue qu'un brin de paille, il rit au sifflement des dards.
\VS{21}Sous son ventre sont des pointes aiguës : On dirait une herse qu'il étend sur la boue.
\VS{22}Il fait bouillir les profondeurs de la mer comme une chaudière, il la traite comme un vase rempli de parfums.
\VS{23}Il laisse après lui un sentier lumineux ; l'abîme prend la chevelure d'un vieillard.
\VS{24}Sur la terre nul n'est son maître ; il a été façonné pour ne rien craindre.
\VS{25}Il regarde avec dédain tout ce qui est élevé, il est le roi des plus fiers animaux.
\Chap{42}
\TextTitle{Job reconnaît la souveraineté de Dieu et s'humilie}
\VerseOne{}Job répondit à Yahweh et dit :
\VS{2}Je sais que tu peux tout, et qu'on ne saurait t'empêcher de faire ce que tu penses.
\VS{3}Quel est celui qui a la folie d'obscurcir mes conseils ? Oui, j'ai parlé sans les comprendre, de merveilles qui me dépassent et que je ne connais pas\FTNT{Ps. 40:6 ; Ps. 131:1 ; Ps. 139:6.}.
\VS{4}Écoute-moi maintenant, et je parlerai ; je t'interrogerai et tu m'instruiras.
\VS{5}J'avais entendu parler de toi ; mais maintenant mon œil t'a vu.
\VS{6}C'est pourquoi je me condamne et je me repens d'avoir ainsi parlé et je m'en sur la poussière et sur la cendre.
\VS{7}Après que Yahweh eut ainsi parlé à Job, il dit à Eliphaz de Théman : Ma colère est embrasée contre toi et contre tes deux amis, parce que vous n'avez pas parlé de moi avec droiture comme Job, mon serviteur.
\VS{8} C'est pourquoi prenez maintenant sept taureaux et sept béliers, allez auprès de mon serviteur Job, et offrez un holocauste pour vous. Job, mon serviteur, priera pour vous, et certainement j'exaucerai sa prière, afin que je ne vous traite pas selon votre folie ; car vous n'avez pas parlé de moi avec droiture, comme mon serviteur Job.
\VS{9} Ainsi Eliphaz de Théman, Bildad de Schuach, et Tsophar de Naama allèrent et firent comme Yahweh leur avait commandé ; et Yahweh exauça la prière de Job.
\VS{10}Yahweh rétablit Job de sa captivité, quand il eut prié pour ses amis ; et Yahweh lui ajouta le double de tout ce qu'il avait possédé.
\VS{11}Ses frères, ses sœurs, et tous ceux qui l'avaient connu auparavant vinrent tous le visiter, et ils mangèrent avec lui dans sa maison. Ils compatirent et le consolèrent au sujet de tout le mal que Yahweh avait fait venir sur lui, et chacun lui donna une kesita et un anneau d'or.
\VS{12}Pendant ses dernières années, Job reçut de Yahweh plus de bénédictions qu'il n'en avait reçu dans les premières. Il posséda quatorze mille brebis, six mille chameaux, mille paires de bœufs, et mille ânesses.
\VS{13}Il eut aussi sept fils et trois filles :
\VS{14}Il donna à la première le nom de Jemima, à la seconde celui de Ketsia, à la troisième celui de Kéren-Happuc.
\VS{15}Et il ne se trouvait pas de femmes aussi belles que les filles de Job dans tout le pays. Leur père leur donna une part de l'héritage parmi leurs frères.
\VS{16}Job vécut, après ces choses, cent quarante ans, et il vit ses fils et les fils de ses fils jusqu'à la quatrième génération.
\VS{1} Job mourut âgé et rassasié de jours.
\PPE{}
\end{multicols}

%\clearpage\ShortTitle{Cantique des cantiques}\BookTitle{Cantique des cantiques}\BFont
\begin{multicols}{2}
\TextTitle{[La fiancée et le fiancé exprime leur amour mutuel.]}
\Chap{1}
\VerseOne{}Le Cantique des cantiques, de Salomon.
\VS{2}[La Sulamithe :] Qu'il me baise des baisers de sa bouche ! [Les filles de Jérusalem :] Car tes amours sont plus agréables que le vin,
\VS{3}à cause de l'odeur de tes excellents parfums, ton nom est comme un parfum qui se répand ; c'est pourquoi les filles t'aiment.
\VS{4}[La Sulamithe :] Entraine-moi après toi ! [Les filles de Jérusalem :] Nous courrons ! [La Sulamithe :] Le roi m'introduit dans ses appartements. [Les filles de Jérusalem :] Nous nous égaierons et nous nous réjouirons à cause de toi ; nous célébrerons ton amour plus que le vin. C’est avec raison que l’on t’aime.
\VS{5}[La Sulamithe :] Ô filles de Jérusalem, je suis noire, je suis belle. Je suis comme les tentes de Kédar, et comme les pavillons de Salomon.
\VS{6}Ne prenez pas garde à moi, de ce que je suis noire : Le soleil m'a brûlée. Les fils de ma mère se sont mis en colère contre moi, ils m'ont faite gardienne des vignes. Ma vigne à moi, je ne l’ai pas gardée.
\VS{7}Dis-moi, toi que mon âme aime, où tu fais paître ton troupeau, et où tu les fais reposer à midi ; car pourquoi serais-je comme une femme errante près des troupeaux de tes compagnons ?
\VS{8}[Salomon :] Si tu ne le sais pas, ô la plus belle des femmes, sors sur les traces du troupeau, et fais paître tes chevreaux près des demeures des bergers.
\VS{9}Ma grande amie, je te compare au plus beau couple de chevaux que j'ai aux chars de Pharaon.
\VS{10}Tes joues sont belles au milieu de tes colliers, et ton cou est beau au milieu des rangées de perles.
\VS{11}[Les filles de Jérusalem :] Nous te ferons des colliers d'or, avec des boutons d'argent.
\VS{12}[La Sulamithe :] Tandis que le roi est assis à table, mon nard répand son parfum.
\VS{13}Mon bien-aimé est pour moi comme un bouquet de myrrhe, il passe la nuit entre mes seins.
\VS{14}Mon bien-aimé m'est comme une grappe de troëne dans les vignes d'En-Guédi.
\VS{15}[Salomon :] Que tu es belle, ma grande amie, que tu es belle ! Tes yeux sont comme ceux des colombes.
\VS{16}[La Sulamithe :] Que tu es beau, mon bien-aimé, que tu es aimable ! Aussi, notre lit est verdoyant.
\VS{17}Les poutres de nos maisons sont faites de cèdre, et nos chevrons de cyprès.
\TextTitle{[l'amour entre la fiancée et le fiancé]}
\Chap{2}
\VerseOne{}[La Sulamithe :] Je suis la rose de Saron, et le lis des vallées.
\VS{2}[Salomon :] Comme un lis au milieu des épines (1), telle est ma grande amie parmi les filles.
\VS{3}[La Sulamithe :] Comme un pommier au milieu des arbres de la forêt, tel est mon bien-aimé parmi les jeunes hommes. J'ai désiré m’asseoir à son ombre, son fruit est doux à mon palais.
\VS{4}Il m'a fait entrer dans la salle du festin (2) ; et la bannière qu’il déploie sur moi c’est l’amour (3).
\VS{5}Soutenez-moi avec des gâteaux de raisins, fortifiez-moi avec des pommes ; car je suis malade d'amour.
\VS{6}Que sa main gauche soit sous ma tête et que sa droite m'embrasse !
\VS{7}[Salomon :] Filles de Jérusalem, je vous en conjure, par les gazelles et par les biches des champs, ne réveillez pas celle que j'aime, ne la réveillez pas jusqu'à ce qu'elle le veuille.
\TextTitle{[La Sulamithe parle de Salomon]}
\VS{8}[La Sulamithe :] C'est la voix de mon bien-aimé ! Le voici, il vient, sautant sur les montagnes et bondissant sur les collines.
\VS{9}Mon bien-aimé est semblable à la gazelle ou aux faons des biches. Le voici qui se tient derrière notre muraille, il regarde par les fenêtres, il se fait voir par les treillis.
\VS{10}[La Sulamithe rapporte les paroles de Salomon :] Mon bien-aimé parle et me dit : Lève-toi, ma grande amie, ma belle, et viens !
\VS{11}Car voici, l'hiver est passé ; la pluie a cessé, elle s'en est allée.
\VS{12}Les fleurs paraissent sur la terre, le temps des chansons est venu, et la voix de la tourterelle se fait déjà entendre dans notre contrée.
\VS{13}Le figuier produit ses premiers fruits, et les vignes en fleurs exhalent le parfum. Lève-toi, ma grande amie, ma belle, et viens !
\VS{14}Ma colombe qui te tiens dans les fentes du rocher (4), dans les lieux secrets (5) et escarpés, fais-moi voir ta figure, fais-moi entendre ta voix ; car ta voix est douce, et ta figure est belle.
\VS{15}Prenez-nous les renards, et les petits renards qui ravagent les vignes, car nos vignes produisent des grappes.
\VS{16}[La Sulamithe :] Mon bien-aimé est à moi, et je suis à lui ; il fait paître son troupeau parmi les lis.
\VS{17}Avant que le jour se rafraîchisse et que les ombres s'enfuient, reviens mon bien-aimé ! Sois comme la gazelle ou les faons des biches, sur les montagnes qui nous séparent.
\TextTitle{[La fiancée recherche son fiancé et le trouve.]}
\Chap{3}
\VerseOne{}[La Sulamithe :] J'ai cherché pendant les nuits, sur ma couche, celui que mon âme aime ; je l'ai cherché, mais je ne l'ai point trouvé.
\VS{2}Je me lèverai maintenant, et je ferai le tour de la ville, des carrefours et des places ; je chercherai celui que mon âme aime. Je l'ai cherché, mais je ne l'ai point trouvé.
\VS{3}Les gardes qui font la ronde dans la ville m'ont rencontrée : N'avez-vous pas vu, leur ai-je dit, celui que mon âme aime ?
\VS{4}A peine les avais-je passés, que j’ai trouvé celui que mon âme aime ; je l’ai saisi et je ne l’ai point lâché jusqu’à ce que je l'aie amené dans la maison de ma mère, dans la chambre de celle qui m'a conçue.
\VS{5}[Salomon :] Filles de Jérusalem, je vous en conjure par les gazelles et par les biches des champs, ne réveillez pas celle que j'aime, ne la réveillez pas, jusqu'à ce qu'elle le veuille.
\TextTitle{[Salomon entre avec sa fiancée dans Sion]}
\VS{6}[La Sulamithe :] Qui est celle qui monte du désert, comme des colonnes de fumée en forme de palmiers, parfumée de myrrhe et d'encens, et de tous les aromates du parfumeur ?
\VS{7}[Un officier des gardes du roi Salomon :] Voici la litière de Salomon, autour de laquelle il y a soixante vaillants hommes, des plus vaillants d'Israël.
\VS{8}Tous sont armés de l'épée et sont très bien exercés à la guerre ; chacun porte son épée sur sa hanche, à cause des frayeurs de la nuit.
\VS{9}Le roi Salomon s'est fait une litière de bois du Liban.
\VS{10}Il en a fait les piliers d’argent, le dossier d’or, le siège de pourpre ; et au milieu, il a placé un tissu que les filles de Jérusalem aiment.
\VS{11}[Les filles de Jérusalem :] Sortez, filles de Sion, et regardez le roi Salomon avec la couronne dont sa mère l'a couronné le jour de ses fiançailles, le jour de la joie de son cœur.
\TextTitle{[Le fiancé déclare son amour]}
\Chap{4}
\VerseOne{}[Salomon :] Que tu es belle, ma grande amie, que tu es belle ! Tes yeux sont comme ceux des colombes derrière ton voile. Tes cheveux sont comme le poil d'un troupeau de chèvres suspendu aux flancs de la montagne de Galaad.
\VS{2}Tes dents sont comme un troupeau de brebis tondues qui remontent de l’abreuvoir ; toutes sont des jumelles, il n’y en a pas une qui manque.
\VS{3}Tes lèvres sont comme un fil teint d’écarlate, et ta bouche est charmante ; ta joue est comme une moitié de grenade, derrière ton voile.
\VS{4}Ton cou est comme la tour de David, bâtie pour être un arsenal ; mille boucliers y sont suspendus, tous les grands boucliers des héros.
\VS{5}Tes deux seins sont comme deux faons, comme les jumeaux d'une gazelle qui paissent au milieu des lis.
\VS{6}Avant que le jour se rafraîchisse et que les ombres s'enfuient, je m'en irai à la montagne de myrrhe, et à la colline de l'encens.
\VS{7}Tu es toute belle, ma grande amie, et il n'y a point de défaut (1) en toi.
\VS{8}Viens du Liban avec moi, mon épouse, viens du Liban avec moi ! Regarde du sommet de l'Amana, du sommet du Senir et de l’Hermon, des repaires des lions et des montagnes des léopards.
\VS{9}Tu me ravis le cœur, ma sœur, mon épouse, tu me ravis le cœur, par l'un de tes regards et par l'un des colliers de ton cou.
\VS{10}Que de charmes dans ton amour, ma sœur, mon épouse ! Comme ton amour vaut mieux que le vin, et combien l'odeur de tes parfums exhale plus que tous les aromates !
\VS{11}Tes lèvres, mon épouse, distillent des rayons de miel ; le miel et le lait sont sous ta langue, et l'odeur de tes vêtements est comme l'odeur du Liban.
\VS{12}Ma sœur, mon épouse, tu es un jardin fermé, une source fermée, une fontaine scellée (1).
\VS{13}Tes jets forment un jardin, où sont des grenadiers, avec les fruits les plus excellents, les troënes avec le nard.
\VS{14}Le nard et le safran, le roseau aromatique et le cinnamome, avec tous les arbres qui donnent l'encens ; la myrrhe et l'aloès, avec tous les excellents aromates.
\VS{15}Ô fontaine des jardins ! Ô source d'eau vive ! Ruisseaux coulant du Liban !
\VS{16}Lève-toi, nord ! Et viens, vent du midi ! Soufflez sur mon jardin afin que ses parfums s’en exhalent ! [La Sulamithe :] Que mon bien-aimé entre dans son jardin et qu'il mange de ses fruits délicieux.
\Chap{5}
\VerseOne{}[Salomon :] J’entre dans mon jardin, ma sœur, mon épouse ; je cueille ma myrrhe avec mes aromates, je mange mes rayons de miel ; avec mon miel, je bois mon vin avec mon lait. Mes amis, mangez, buvez, enivrez-vous d’amour, mes bien-aimés !
\TextTitle{[La Sulamithe raconte son rêve]}
\VS{2}[La Sulamithe :] J'étais endormie, mais mon cœur veillait (1), c’est la voix de mon bien-aimé qui frappe, en disant : [La Sulamithe rapporte les propos de Salomon :] Ouvre-moi, ma sœur, ma grande amie, ma colombe, ma parfaite ! Car ma tête est pleine de rosée et mes cheveux des gouttes de la nuit.
\VS{3}[La Sulamithe :] J'ai ôté ma tunique, lui dis-je, comment la remettrais-je ? J'ai lavé mes pieds, comment les salirais-je ?
\VS{4}Mon bien-aimé a passé la main par la fenêtre, et mes entrailles se sont émues pour lui.
\VS{5}Je me suis levée pour ouvrir à mon bien-aimé, et la myrrhe a ruisselé de mes mains, et la myrrhe s’est répandue de mes doigts sur la poignée du verrou.
\VS{6}J'ai ouvert à mon bien-aimé, mais mon bien-aimé s'en était allé, il avait disparu ; mon âme était hors de moi quand il me parlait. Je l’ai cherché, mais je ne l’ai point trouvé ; je l'ai appelé, mais il ne m’a point répondu.
\VS{7}Les gardes qui font la ronde dans la ville m’ont rencontrée ; ils m’ont battue, ils m’ont blessée ; les gardes des murailles m'ont enlevé mon voile.
\VS{8}Filles de Jérusalem, je vous en conjure, si vous trouvez mon bien-aimé, que lui direz-vous ? Que je suis malade d’amour.
\VS{9}[Les filles de Jérusalem :] Qu'a ton bien-aimé de plus qu’un autre, ô la plus belle des femmes ? Qu'a ton bien-aimé de plus qu’un autre pour que tu nous conjures ainsi ?
\TextTitle{[La Sulamithe décrit Salomon]}
\VS{10}[La Sulamithe :] Mon bien-aimé est blanc et vermeil, il se distingue entre dix mille.
\VS{11}Sa tête est de l’or pur, ses cheveux sont bouclés et flottants, noirs comme le corbeau.
\VS{12}Ses yeux sont comme ceux des colombes au bord des ruisseaux, lavés dans du lait, reposant au sein de la plénitude.
\VS{13}Ses joues sont comme un parterre d’aromates, comme des fleurs parfumées ; ses lèvres sont comme des lis d’où découle la myrrhe.
\VS{14}Ses mains sont comme des anneaux d'or, garnis de chrysolithes ; son ventre est comme de l’ivoire bien poli, couvert de saphirs.
\VS{15}Ses jambes sont comme des piliers de marbre, posés sur des bases d’or fin. Son aspect est comme le Liban, il est précieux comme les cèdres.
\VS{16}Son palais n'est que douceur, et toute sa personne est pleine de charme (2). Tel est mon bien-aimé, tel est mon ami, filles de Jérusalem !
\TextTitle{[Les filles de Jérusalem aident la Sulamithe à chercher Salomon]}
\Chap{6}
\VerseOne{}[Les filles de Jérusalem :] Où est allé ton bien-aimé, ô la plus belle des femmes ? De quel côté est allé ton bien-aimé ? Nous le chercherons avec toi.
\VS{2}[La Sulamithe :] Mon bien-aimé est descendu à son jardin, au parterre d’aromates, pour faire paître son troupeau dans les jardins, et pour cueillir des lis.
\VS{3}Je suis à mon bien-aimé et mon bien-aimé est à moi ; il fait paître son troupeau parmi les lis.
\TextTitle{[Salomon à la Sulamithe]}
\VS{4}[Salomon :] Tu es belle, grande amie, comme Thirtsa ; agréable comme Jérusalem, redoutable comme des troupes sous leurs bannières.
\VS{5}Détourne de moi tes yeux, car ils me troublent. Tes cheveux sont comme un troupeau de chèvres suspendus aux flancs de Galaad.
\VS{6}Tes dents sont comme un troupeau de brebis qui remontent de l’abreuvoir ; toutes portent des jumeaux, et aucune d'elles n'est stérile.
\VS{7}Ta joue est comme une moitié de grenade, derrière ton voile.
\VS{8}Il y a soixante reines, quatre-vingts concubines, et des vierges sans nombre.
\VS{9}Une seule est ma colombe, ma parfaite ; elle est l’unique de sa mère, celle qui l'a enfantée. Les filles la voient et la disent heureuse ; les reines et les concubines la louent en disant :
\VS{10}Qui est celle qui apparaît comme l’aurore, belle comme la lune, brillante comme le soleil, redoutable comme des troupes sous leurs bannières ?
\VS{11}[La Sulamithe :] Je suis descendu au jardin des noyers pour voir les fruits de la vallée qui mûrissent, et pour voir si la vigne pousse, et si les grenadiers fleurissent.
\VS{12}Je ne me suis pas aperçue que mon affection m'a rendue semblable aux chars d’Amminadib (1).
\TextTitle{[Description de la beauté de la Sulamithe.]}
\Chap{7}
\VerseOne{}[Les filles de Jérusalem :] Reviens, reviens, ô Sulamithe ! Reviens, reviens, afin que nous te contemplions. [La Sulamithe :] Qu’avez-vous à contempler la Sulamithe comme une danse de deux armées ?
\VS{2}[Les filles de Jérusalem :] Que tes pieds sont beaux dans tes chaussures, fille de prince ! Les contours de ta hanche sont comme des colliers, œuvres des mains d'un excellent artisan.
\VS{3}Ton sein est comme une coupe arrondie, pleine d’un vin aromatisé, ton ventre est comme un tas de blé entouré de lis.
\VS{4}Tes deux seins sont comme deux faons, comme les jumeaux d'une gazelle.
\VS{5}Ton cou est comme une tour d'ivoire ; tes yeux sont comme les étangs de Hesbon, près de la porte de Bath-Rabbim ; ton nez est comme la tour du Liban qui regarde vers Damas.
\VS{6}Ta tête est élevée comme le Carmel, et les cheveux fins de ta tête sont comme le pourpre ; un roi est enchaîné par tes boucles pour te contempler.
\VS{7}[Salomon :] Que tu es belle, que tu es agréable, ô mon amour, au milieu des délices !
\VS{8}Ta taille est semblable à un palmier, et tes seins à des grappes.
\VS{9}Je me dis : Je monterai sur le palmier, je prendrai possession de ses rameaux ! Que tes seins soient comme les grappes de la vigne, l’odeur de tes narines comme celle des pommes,
\VS{10}et ta bouche comme un vin excellent… [La Sulamithe :] …qui coule droitement pour mon bien-aimé, et glisse sur les lèvres de ceux qui s’endorment.
\VS{11}Je suis à mon bien-aimé et ses désirs se portent vers moi.
\VS{12}Viens, mon bien-aimé, sortons dans les champs, passons la nuit dans les villages !
\VS{13}Levons-nous dès le matin pour aller aux vignes, nous verrons si la vigne pousse, si la fleur s’ouvre, si les grenadiers fleurissent. Là je te donnerai mes amours.
\VS{14}Les mandragores répandent leur parfum, et à nos portes il y a toutes sortes de fruits exquis, des fruits nouveaux, et des fruits anciens : Mon bien-aimé, je les ai gardés pour toi.
\Chap{8}
\VerseOne{}[La Sulamithe :] Oh ! Que n’es-tu pour moi comme un frère, allaité des seins de ma mère ! Je te rencontrerais dehors, je t’embrasserais, et on ne me mépriserait pas.
\VS{2}Je te conduirais, je t’introduirais dans la maison de ma mère ; tu m’instruiras, et je te ferai boire du vin parfumé d'aromates et du moût de mon grenadier.
\VS{3}Que sa main gauche soit sous ma tête, et que sa droite m'embrasse !
\VS{4}[La Sulamithe (citant Salomon) :] Je vous en conjure, filles de Jérusalem, ne réveillez pas celle que j'aime, ne la réveillez pas, jusqu'à ce qu'elle le veuille.
\VS{5}[Les frères de la Sulamithe :] Qui est celle qui monte du désert, mollement appuyée sur son bien-aimé ? [Salomon :] Je t'ai réveillée sous le pommier ; là où ta mère t'a enfantée, là où celle qui t'a conçue t'a donné le jour.
\VS{6}Mets-moi comme un sceau sur ton cœur (1), comme un sceau sur ton bras ; car l'amour est fort comme la mort, et la jalousie est cruelle comme le scheol ; leurs ardeurs sont des ardeurs de feu, une flamme de Yahweh.
\VS{7}[La Sulamithe (à Salomon) :] Les grandes eaux ne peuvent éteindre l’amour, même les fleuves ne pourraient le submerger ; quand un homme donnerait toutes les richesses de sa maison contre l’amour, il ne s’attirerait qu’un profond mépris.
\VS{8}[La Sulamithe (racontant ce que ses frères lui ont dit) :] Nous avons une petite sœur qui n'a pas encore de seins ; que ferons-nous à notre sœur le jour où on parlera d’elle ?
\VS{9}Si elle est comme une muraille, nous bâtirons sur elle un palais d'argent ; si elle est une porte, nous la renforcerons avec une planche de cèdre.
\VS{10}[La Sulamithe :] Je suis comme une muraille, et mes seins sont comme des tours ; j'ai été à ses yeux comme celle qui trouve la paix.
\VS{11}Salomon avait une vigne à Baal-Hamon ; il remit la vigne à des gardiens ; chacun apportait pour son fruit mille pièces d'argent.
\VS{12}Ma vigne, qui est à moi, je la garde. Ô Salomon, que les mille pièces d'argent soient à toi, et deux cents pour les gardiens du fruit de la vigne !
\VS{13}[Les frères de la Sulamithe :] Ô toi qui habites dans les jardins ! Les amis sont attentifs à ta voix. [Salomon :] Daigne me la faire entendre !
\VS{14}[La Sulamithe :] Fuis, mon bien-aimé ! Sois semblable à la gazelle ou au faon des biches, sur les montagnes des aromates !
\PPE{}
\end{multicols}

%\clearpage\ShortTitle{Ruth}\BookTitle{Ruth}\BFont
\noindent\hrulefill
{\footnotesize
\textit{
\bigskip
{\centering{}
\\(Routh)
\\Signifie : Amitié, une amie
\\Thème : Les origines de la famille messianique
\\Auteur : Inconnu
\\Date de rédaction : 11ème siècle av. J.-C.\\}
}
%\bigskip
\textit{
\\Au temps des juges, tout le pays fut frappé par une famine qui poussa Elimélec, sa femme Naomi, et ses deux fils à s’installer dans le pays de Moab. Ce pays tire son nom de son fondateur Moab, né de l’inceste entre Lot et sa fille ainée.
%\bigskip
\\Ils y rencontrèrent Ruth qui devint ensuite la belle fille d’Elimélec. Après la mort de son époux, cette moabite démontra  son attachement non seulement à cette famille mais également au Dieu de cette famille qui devint aussi le sien.
%\bigskip
\\Au prix de sa détermination, son obéissance et son humilité, la destinée de cette femme fut complètement bouleversée. Image du rachat des nations, elle entra dans la lignée de Jésus-Christ homme.\bigskip
}
}
\par\nobreak\noindent\hrulefill
\begin{multicols}{2}
\Chap{1}
\TextTitle{[Famine en Juda]}
\VerseOne{}Au temps où les juges gouvernaient, il y eut une famine dans le pays. Un homme de Bethléhem de Juda s'en alla, avec sa femme et ses deux fils, pour séjourner sur la terre de Moab.
\TextTitle{[Séjour en Moab]}
\VS{2}Le nom de cet homme était Elimélec, celui de sa femme Naomi, et les noms de ses deux fils Machlon et Kiljon ; ils étaient Ephratiens, de Bethléhem de Juda. Entrés sur la terre de Moab, ils s’y établirent.
\VS{3}Elimélec, mari de Naomi, mourut, et elle resta avec ses deux fils.
\VS{4}Ils prirent pour eux des femmes Moabites, dont l'une se nommait Orpa, et la seconde Ruth\FTNT{Ruth, la Moabite, dont l’ancêtre était issu d’une relation incestueuse (Ge. 19:36-37), est devenue l’ancêtre du Messie (Mt. 1:5-6).}, et ils demeurèrent là environ dix ans.
\VS{5}Machlon et Kiljon moururent aussi tous les deux, et cette femme resta privée de ses deux fils et de son mari.
\TextTitle{[Retour en Juda]}
\VS{6}Puis elle se leva avec ses belles-filles afin de retourner de la terre de Moab, car elle entendit, au pays de Moab, que Yahweh avait visité son peuple en lui donnant du pain.
\VS{7}Elle sortit du lieu où elle habitait, avec ses deux belles-filles, et elle marcha pour revenir sur la terre de Juda.
\VS{8}Naomi dit à ses deux belles-filles : Allez, retournez chacune dans la maison de sa mère ! Que Yahweh use de bonté envers vous, comme vous l'avez fait envers ceux qui sont morts et envers moi !
\VS{9}Que Yahweh vous donne de trouver chacune du repos dans la maison d'un mari ! Et elle les embrassa. Elles levèrent leur voix, et pleurèrent ;
\VS{10}et elles lui dirent : Non, nous retournerons avec toi vers ton peuple.
\TextTitle{[Décision loyale de Ruth]}
\VS{11}Naomi dit : Retournez, mes filles ! Pourquoi viendriez-vous avec moi ? Ai-je encore dans mon sein des fils qui puissent devenir vos maris ?
\VS{12}Retournez, mes filles, allez ! Je suis trop vieille pour me remarier. Et quand je dirais : J'ai de l'espérance, quand cette nuit même je serais avec un mari, et que j'enfanterais des fils,
\VS{13}Attendriez-vous donc qu'ils aient grandi, refuseriez-vous donc des maris ? Non, mes filles ! Je suis dans une plus grande amertume que vous, car la main de Yahweh s'est éloignée de moi.
\VS{14}Et elles levèrent leur voix, et pleurèrent encore. Orpa embrassa sa belle-mère, mais Ruth s’attacha à elle.
\VS{15}Naomi dit à Ruth : Voici, ta belle-sœur est retournée vers son peuple et vers ses dieux ; retourne, après ta belle-sœur.
\VS{16}Ruth répondit : Ne me prie pas de te laisser, de me retourner et de ne pas te suivre ! Où tu iras, j'irai, où tu demeureras je demeurerai ; ton peuple sera mon peuple, et ton Dieu sera mon Dieu ;
\VS{17}Où tu mourras je mourrai, et j'y serai enterrée. Que Yahweh me traite dans toute sa rigueur, si autre chose que la mort vient à me séparer de toi !
\VS{18}Naomi, la voyant déterminée à aller avec elle, arrêta de lui parler.
\TextTitle{[Arrivée à Bethléhem]}
\VS{19}Elles marchèrent toutes deux jusqu'à ce qu'elles entrent à Bethléhem. Lorsqu'elles entrèrent dans Bethléhem, toute la ville fut agitée à cause d'elles, et les femmes disaient : Est-ce là Naomi ?
\VS{20}Elle leur dit : Ne m'appelez pas Naomi ; appelez-moi Mara, car le Tout-Puissant m'a remplie de beaucoup d'amertume.
\VS{21}J’étais dans l'abondance à mon départ, et Yahweh me ramène à vide. Pourquoi m'appelleriez-vous Naomi, après que Yahweh s'est prononcé contre moi, et que le Tout-Puissant m'a affligée ?
\VS{22}Ainsi revinrent de la terre de Moab Naomi et sa belle-fille, Ruth, la Moabite. Elles entrèrent dans Bethléhem au commencement de la moisson des orges.
\Chap{2}
\TextTitle{[Boaz félicite Ruth des soins désintéressés dont elle entoure Naomi]}
\VerseOne{}Naomi avait un parent de son mari. C'était un homme puissant et riche, de la famille d'Elimélec, et qui s’appelait Boaz.
\VS{2}Ruth la Moabite dit à Naomi : Je te prie laisse-moi aller glaner des épis dans le champ de celui aux yeux duquel je trouverai grâce. Elle lui dit : Va, ma fille.
\VS{3}Elle s'en alla et entra dans un champ, pour glaner après les moissonneurs. Et elle arriva par hasard sur une parcelle de champ qui appartenait à Boaz, qui était de la famille d'Elimélec.
\VS{4}Or voici, Boaz vint de Bethléhem, et il dit aux moissonneurs : Que Yahweh soit avec vous ! Ils lui dirent : Que Yahweh te bénisse !
\VS{5}Et Boaz dit à son serviteur qui était établi sur les moissonneurs : A qui est cette jeune fille ?
\VS{6}Le serviteur qui était établi sur les moissonneurs répondit et dit : C'est une jeune femme Moabite, qui est revenue avec Naomi de la terre de Moab.
\VS{7}Elle nous a dit : Permettez-moi de glaner et de recueillir des épis entre les gerbes, après les moissonneurs. Depuis ce matin qu'elle est venue, elle est restée jusqu'à présent, et s'est à peine assise dans la maison.
\VS{8}Boaz dit à Ruth : Ecoute, ma fille, ne va pas glaner dans un autre champ ; ne pars pas au loin, et reste avec mes servantes.
\VS{9}Regarde où l'on moissonne dans le champ, et va après elles. J'ai défendu à mes serviteurs de te toucher. Et si tu as soif, va prendre des vases, et bois de ce que les serviteurs auront puisé.
\VS{10}Alors elle tomba sur sa face, et se prosterna contre terre, et elle lui dit : Comment ai-je trouvé grâce à tes yeux, pour que tu prêtes attention à moi, moi qui suis une étrangère ?
\VS{11}Boaz lui répondit et dit : On m'a raconté tout ce que tu as fait pour ta belle-mère depuis que ton mari est mort, comment tu as laissé ton père, ta mère, et le pays de ta naissance, pour aller vers un peuple que tu ne connaissais pas auparavant.
\VS{12}Que Yahweh te récompense pour ton œuvre, et que ton salaire soit entier de la part de Yahweh le Dieu d'Israël, sous les ailes duquel tu es venue te réfugier !
\VS{13}Et elle dit : Mon seigneur, que je trouve grâce à tes yeux ! Car tu m'as consolée, et tu as parlé au cœur de ta servante. Et pourtant je ne suis pas, moi, comme l'une de tes servantes.
\VS{14}Au moment du repas, Boaz dit à Ruth : Approche-toi ici, mange du pain, et trempe ton morceau dans le vinaigre. Elle s'assit à côté des moissonneurs. On lui donna du grain rôti ; elle mangea et se rassasia, et elle garda le reste.
\VS{15}Puis elle se leva pour glaner. Boaz ordonna à ses serviteurs : Qu'elle glane même entre les gerbes, et ne lui faites pas honte.
\VS{16}Et vous retirerez même pour elle quelques poignées de gerbes, que vous lui laisserez glaner, sans la réprimander.
\VS{17}Elle glana donc dans le champ jusqu'au soir, et elle battit ce qu'elle avait glané. Il y eut environ un épha d'orge.
\VS{18}Elle l'emporta, entra dans la ville, et sa belle-mère vit ce qu'elle avait glané. Elle sortit aussi les restes de son repas, et les lui donna.
\VS{19}Sa belle-mère lui dit : Où as-tu glané aujourd'hui, et où as-tu travaillé ? Béni soit celui qui t'a reconnue ! Et Ruth raconta à sa belle-mère chez qui elle avait travaillé : L'homme chez qui j'ai travaillé aujourd'hui s’appelle Boaz.
\VS{20}Naomi dit à sa belle-fille : Qu'il soit béni de Yahweh, puisqu'il a la même bonté pour les vivants, comme il en eut pour ceux qui sont morts ! Cet homme est un proche parent, lui dit encore Naomi, il est un de ceux qui ont sur nous le droit de rachat\FTNT{Le droit de rachat : Le rédempteur est celui qui rachète une personne moyennant le paiement d'une rançon. Sous la première alliance, le rachat se faisait soit par un frère, soit par un proche parent, pour la libération de celui qui s’était fait esclave ou qui avait aliéné sa propriété ou son bien (Lé. 25:25 et 48). Sous la nouvelle alliance, Jésus-Christ est désormais notre rédempteur. Romains 3:23-24 nous dit : «~ Car tous ont péché, et sont entièrement privés de la gloire de Dieu. Et ils sont gratuitement justifiés par sa grâce, par la rédemption qui est en Jésus-Christ «~». Christ nous a rachetés de la malédiction de la loi en se donnant lui-même pour nous afin de nous délivrer de toute iniquité (Ga. 3:13 ; Ti. 2:14). Dieu s’est fait homme (Hé. 2:14-17) afin de mieux nous libérer de l'esclavage du diable par sa mort à la croix de Golgotha (Es. 60:16).}.
\VS{21}Ruth la Moabite dit : Il m'a même dit : Reste avec mes serviteurs jusqu'à ce qu'ils aient achevé toute ma moisson.
\VS{22}Et Naomi dit à Ruth, sa belle-fille : Ma fille, il est bon que tu sortes avec ses servantes, et qu'on ne te rencontre pas dans un autre champ.
\VS{23}Elle resta donc avec les servantes de Boaz, pour glaner, jusqu'à la fin de la moisson des orges et la moisson des froments. Et elle demeurait avec sa belle-mère.
\Chap{3}
\TextTitle{[Ruth dans l'obéissance de la foi]}
\VerseOne{}Naomi, sa belle-mère, lui dit : Ma fille, je voudrais chercher ton repos, afin que tu sois heureuse.
\VS{2}Maintenant Boaz, avec les servantes duquel tu as été, n'est-il pas de notre parenté ? Voici, il doit vanner cette nuit les orges qui ont été foulées dans l'aire.
\VS{3}Lave-toi et oins-toi, puis mets tes habits, et descends dans l'aire. Ne te fais pas connaître à lui, jusqu'à ce qu'il ait achevé de manger et de boire.
\VS{4}Quand il se couchera, découvre le lieu où il se couche. Ensuite, entre, découvre ses pieds, et couche-toi. Il te dira ce que tu as à faire.
\VS{5}Elle lui répondit : Je ferai tout ce que tu as dit.
\VS{6}Elle descendit à l'aire, et fit tout ce que sa belle-mère lui avait ordonné.
\VS{7}Boaz mangea et but, et son cœur était joyeux. Il vint se coucher à l'extrémité d'un tas de gerbes. Ruth vint secrètement, découvrit ses pieds, et se coucha.
\VS{8}Au milieu de la nuit, cet homme eut peur ; il se retourna et retira ses pieds, car voici, une femme était couchée à ses pieds.
\VS{9}Il dit : Qui es-tu ? Elle répondit : Je suis Ruth, ta servante ; étends le pan de ta robe sur ta servante, car tu as droit de rachat.
\VS{10}Et il dit : Ma fille, que Yahweh te bénisse ! Ce dernier trait de bonté me réjouit plus que le premier, car tu n'es pas allée après des jeunes gens, pauvres ou riches.
\VS{11}Maintenant, ma fille, ne crains pas ; je te ferai tout ce que tu me diras ; car toute la porte de mon peuple sait que tu es une femme vertueuse.
\VS{12}Il est bien vrai que j'ai droit de rachat, mais il existe un autre plus proche que moi, qui a le droit de rachat.
\VS{13}Passe ici la nuit, et demain, si cet homme veut user envers toi du droit de rachat, à la bonne heure, qu'il te rachète ; mais s'il ne lui plaît pas de te racheter, moi je te rachèterai, Yahweh est vivant ! Couche-toi jusqu'au matin.
\VS{14}Elle se coucha à ses pieds jusqu'au matin, et elle se leva avant qu'on puisse se reconnaître l'un l'autre. Boaz dit : Qu'on ne sache pas qu'une femme est entrée dans l'aire.
\VS{15}Et il dit : Donne-moi le manteau qui est sur toi, et tiens-le. Elle le tint, et il mesura six mesures d'orge, qu'il posa sur elle. Puis il entra dans la ville.
\VS{16}Ruth revint auprès de sa belle-mère, et Naomi dit : Est-ce toi ma fille ? Ruth lui raconta tout ce que cet homme avait fait pour elle.
\VS{17}Elle dit : Il m'a donné ces six mesures d'orge, en disant : Tu n'iras pas à vide vers ta belle-mère.
\VS{18}Et Naomi dit : Ma fille, assieds-toi ici jusqu'à ce que tu saches ce que l'affaire deviendra, car cet homme ne se donnera pas de repos, qu'il n'ait achevé cette affaire aujourd'hui.
\Chap{4}
\TextTitle{[Ruth comblée par le mariage]}
\VerseOne{}Boaz monta à la porte, et s'y assit. Or voici, celui qui avait le droit de rachat, et dont Boaz avait parlé, passa. Boaz lui dit : Ah ! Détourne-toi, reste ici, toi un tel. Et il se détourna, et s'assit.
\VS{2}Boaz prit dix hommes d'entre les anciens de la ville, et leur dit : Asseyez-vous ici. Et ils s'assirent.
\VS{3}Puis il dit à celui qui avait le droit de rachat : Naomi qui est revenue de la terre de Moab, a vendu la parcelle du champ qui appartenait à notre frère Elimélec.
\VS{4}J'ai parlé à tes oreilles afin de te le faire savoir et te le dire : Acquiers-la en la présence de ceux qui sont assis ici et en présence des anciens de mon peuple. Si tu veux racheter par droit de rachat, rachète-la ; mais si tu ne veux pas la racheter, déclare-le-moi, afin que je le sache. Car il n'y a pas d'autre que toi qui ait le droit de rachat, et je l'ai après toi. Et il dit : je rachèterai.
\VS{5}Boaz dit : Le jour où tu acquerras le champ de la main de Naomi, tu l'acquerras aussi de Ruth la Moabite, femme du défunt, pour maintenir le nom du défunt dans son héritage.
\VS{6}Et celui qui avait le droit de rachat dit : Je ne puis pas racheter pour mon compte, de peur de détruire mon héritage ; prends pour toi le droit de rachat, car je ne puis pas le racheter.
\VS{7}Autrefois en Israël, pour confirmer une affaire quelconque relative à un rachat ou à un échange, l'homme ôtait son soulier et le donnait à son parent : C'était là, en Israël, un témoignage qu'on cédait son droit.
\VS{8}Celui qui avait le droit de rachat dit à Boaz : Acquiers-le pour toi ! Et il ôta son soulier.
\VS{9}Alors Boaz dit aux anciens et à tout le peuple : Vous êtes aujourd'hui témoins que j'ai acquis de la main de Naomi tout ce qui appartenait à Elimélec, à Kiljon et à Machlon.
\VS{10}Et que je me suis également acquis pour femme Ruth la Moabite, femme de Machlon, pour maintenir le nom du défunt dans son héritage, et afin que le nom du défunt ne soit pas retranché d'entre ses frères et de la porte de sa ville. Vous en êtes témoins aujourd'hui !
\VS{11}Tout le peuple qui était à la porte et les anciens dirent : Nous en sommes témoins ! Que Yahweh rende la femme qui entre dans ta maison semblable à Rachel et à Léa, qui ont bâti toutes deux, la maison d'Israël ! Montre ta puissance dans Ephrata et proclame ton nom dans Bethléhem !
\VS{12}Puisse la postérité que Yahweh te donnera de cette jeune femme, rendre ta maison semblable à la maison de Pérets, que Tamar enfanta à Juda !
\VS{13}Boaz prit Ruth, qui devint sa femme, et il alla vers elle. Yahweh lui fit la grâce de concevoir, et elle enfanta un fils.
\VS{14}Les femmes dirent à Naomi : Béni soit Yahweh qui ne t'a pas laissé manquer aujourd'hui d'un homme, ayant droit de rachat, et dont le nom sera proclamé en Israël !
\VS{15}Cet enfant restaurera ton âme, et sera le soutien de ta vieillesse ; car ta belle-fille, qui t'aime, l'a enfanté, et elle vaut mieux que sept fils.
\VS{16}Naomi prit l'enfant et le posa sur son sein, et elle fut sa nourrice.
\TextTitle{[Le fils de Ruth sera le grand-père de David]}
\VS{17}Les voisines lui donnèrent un nom, en disant : Un fils est né à Naomi ! Et elles l'appelèrent du nom de Obed. Ce fut le père d'Isaï, père de David.
\VS{18}Voici la généalogie de Pérets. Pérets engendra Hetsron ;
\VS{19}Hetsron engendra Ram ; Ram engendra Amminadab ;
\VS{20}Amminadab engendra Nachschon ; Nachschon engendra Salmon ;
\VS{21}Salmon engendra Boaz ; Boaz engendra Obed ;
\VS{22}Obed engendra Isaï, et Isaï engendra David.
\PPE{}
\end{multicols}

%\clearpage\ShortTitle{Lamentations de Jérémie}\BookTitle{Lamentations de Jérémie}\BFont
\noindent\hrulefill
{\footnotesize
\textit{
\bigskip
{\centering{}
\\Auteur : Jérémie
\\(Heb. : Eikha)
\\Signification : Où ?
\\Thème : Affliction pour Jérusalem
\\Date de rédaction : 6\up{ème} siècle av. J.-C\\}
}
%\bigskip
\textit{
\\Recueil de pièces poétiques, les lamentations de Jérémie furent composées selon un procédé visant à accentuer le caractère funèbre, de façon à ce qu'elles soient récitées avec gémissements. Ses complaintes exposent la profonde désolation du prophète face au fardeau du peuple qu'il portait dans ses entrailles tout comme la douleur et la tristesse de Yahweh face à Israël.
%\bigskip
\\Très différentes des prophéties retrouvées dans le livre de Jérémie, les Lamentations reflètent l'affliction convenant à la
gravité du châtiment subi : famine, pillage et ruine du temple, déportation, cessation du culte, diverses calamités… Jérémie rappelle ainsi les conséquences de l'endurcissement du cœur face aux appels à la repentance ; il présente aussi les bontés éternelles de Yahweh.\bigskip
}
}
\par\nobreak\noindent\hrulefill
\begin{multicols}{2}
\Chap{1}
\TextTitle{Pleurs et désolation de Jérusalem}
\VerseOne{}[Aleph.] Comment est-il arrivé que la ville si peuplée se trouve si solitaire ? Que celle qui était grande entre les nations est devenue comme une veuve ? Que celle qui était noble dame entre les provinces a été rendue tributaire ?
\VS{2}[Beth.] Elle ne cesse de pleurer pendant la nuit, et ses larmes sont sur ses joues ; il n'y a pas un de tous ses amis qui la console ; ses intimes amis ont agi perfidement contre elle, ils sont devenus ses ennemis.
\VS{3}[Guimel.] Juda a été emmenée captive tant elle est affligée, et tant est grande sa servitude ; elle demeure maintenant entre les nations, et ne trouve point de repos ; tous ses persécuteurs l'ont attrapée dans sa détresse\FTNT{Jé. 52:26.}.
\VS{4}[Daleth.] Les chemins de Sion mènent deuil de ce qu'il n'y a plus personne qui vienne aux fêtes solennelles ; toutes ses portes sont désolées, ses sacrificateurs sanglotent, ses vierges sont accablées de tristesse ; elle est remplie d'amertume. 
\VS{5}[He.] Ses adversaires sont établis pour chefs, ses ennemis prospèrent ; car Yahweh l'a humiliée à cause de la multitude de ses transgressions ; ses petits enfants ont marché captifs devant l'adversaire\FTNT{Jé. 30:14.}.
\VS{6}[Vav.] Et tout l'honneur de la fille de Sion s'est retiré d'elle ; ses chefs sont devenus semblables à des cerfs qui ne trouvent pas de pâture, et qui fuient sans force devant celui qui les poursuit.
\VS{7}[Zayin.] Jérusalem dans les jours de son affliction et de son pauvre état s'est souvenue de toutes ses choses précieuses qu'elle avait depuis si longtemps, lorsque son peuple est tombé par la main de l'ennemi, sans aucun secours ; les ennemis l'ont vue, et se sont moqués de ses sabbats.
\VS{8}[Heth.] Jérusalem a grièvement péché ; c'est pourquoi elle est devenue un objet de dégoût ; tous ceux qui l'honoraient l'ont méprisée parce qu'ils ont vu son ignominie ; elle en a aussi sangloté, et s'est retournée en arrière.
\VS{9}[Teth.] Sa souillure était dans les pans de sa robe, et elle ne s'est pas souvenue de sa dernière fin ; elle a été extraordinairement abaissée, et elle n'a pas de consolateur. Vois ma misère, ô Yahweh ! Car l'ennemi s'est élevé avec orgueil !
\VS{10}[Yod.] L'ennemi a étendu sa main sur toutes ses choses désirables ; car elle a vu entrer dans son sanctuaire les nations au sujet desquelles tu avais donné cet ordre : Elles n'entreront point dans ton assemblée\FTNT{De. 23:3.}.
\VS{11}[Kaf.] Tout son peuple gémit, cherchant du pain\FTNT{Jé. 52:6.} ; ils ont donné leurs choses désirables pour des aliments, afin ranimer leur vie. Vois, ô Yahweh ! Regarde combien je suis méprisée.
\VS{12}[Lamed.] Cela ne vous touche-t-il point ? Vous tous passants, contemplez, et voyez s'il est une douleur comme ma douleur, celle dont j'ai été frappée ! Moi que Yahweh a accablée de douleur au jour de l'ardeur de sa colère.
\VS{13}[Mem.] Il a envoyé d'en haut, dans mes os, un feu qui les domine ; il a tendu un filet sous mes pieds, et m'a fait revenir en arrière ; il m'a mise dans la désolation, dans une langueur de tous les jours.
\VS{14}[Nun.] Le joug de mes iniquités est lié par sa main ; elles sont entrelacées, et appliquées sur mon cou ; il a renversé ma force ; le Seigneur m'a livrée entre les mains de ceux contre qui je ne pourrai pas me lever.
\VS{15}[Samech.] Le Seigneur a abattu tous les hommes forts que j'avais au milieu de moi ; il a appelé contre moi, au temps fixé, une armée pour détruire mes jeunes hommes ; le Seigneur a foulé au pressoir la vierge, fille de Juda.
\VS{16}[Ayin.] À cause de ces choses, je pleure, mes yeux fondent en larmes ; car le consolateur qui restaurait ma vie est loin de moi. Mes fils sont dans la désolation parce que l'ennemi a été plus fort.
\VS{17}[Pe.] Sion a étendu les mains, et personne ne l'a consolée ; Yahweh a ordonné aux ennemis de Jacob de l'entourer de toutes parts. Jérusalem a été comme une impureté au milieu d'eux.
\VS{18}[Tsade.] Yahweh est juste car j'ai été rebelle à ses ordres. Ecoutez, vous tous, peuples, et voyez ma douleur ! Mes vierges et mes jeunes hommes sont allés en captivité.
\VS{19}[Qof.] J'ai appelé mes amis, mais ils m'ont trompé. Mes sacrificateurs et mes anciens sont morts dans la ville : Ils cherchaient de la nourriture afin de restaurer leur vie.
\VS{20}[Resh.] Regarde Yahweh ! car je suis dans la détresse ; mes entrailles bouillonnent, mon coeur palpite au dedans de moi, parce que je n'ai fait qu'être rebelle ; au dehors l’épée m’a privée d’enfants ; au dedans il y a comme la mort. 
\VS{21}[Shin.] On m'a entendu sangloter et je n'ai personne qui me console ; tous mes ennemis ont appris mon malheur, et s’en sont réjouis, parce que tu l’as fait ; tu amèneras le jour que tu as assigné, et ils seront dans mon état.
\VS{22}[Tav.]Que toute leur méchanceté vienne devant toi, et traite-leur comme tu m'as traitée à cause de tous mes péchés ; car mes sanglots sont en grand nombre et mon coeur est languissant. 
\Chap{2}
\TextTitle{Le jour de la colère de Yahweh}
\VerseOne{}[Aleph.] Comment est-il arrivé que le Seigneur a couvert de sa colère la fille de Sion tout à l'entour, comme d'une nuée, et qu'il a précipité du ciel sur la terre la beauté d'Israël, et ne s'est pas souvenu du marchepied de ses pieds\FTNT{Ez. 43:7.} au jour de sa colère ?
\VS{2}[Beth.] Le Seigneur a englouti sans épargner toutes les habitations de Jacob ; il a dans sa fureur renversé les forteresses de la fille de Juda, il les a jetées par terre ; il a profané le royaume et ses chefs.
\VS{3}[Guimel.] Il a retranché toute la force d'Israël par l'ardeur de sa colère ; il a retiré sa droite en arrière devant l'ennemi ; il s'est allumé dans Jacob comme un feu flamboyant qui le consume de toutes parts.
\VS{4}[Daleth.] Il a tendu son arc comme un ennemi ; sa droite s'est dressée comme celle d'un adversaire ; il a tué tout ce qui était agréable à l'œil dans la tente de la fille de Sion ; il a répandu sa fureur comme un feu.
\VS{5}[He.] Le Seigneur a été comme un ennemi ; il a englouti Israël, il a englouti tous ses palais, il a détruit toutes ses forteresses ; il a multiplié chez la fille de Juda le deuil et les afflictions.
\VS{6}[Vav.] Il a mis en pièces avec violence sa tente comme un jardin ; il a détruit le lieu de son assemblée ; Yahweh a fait oublier dans Sion la fête solennelle et le sabbat, et dans sa violente colère, il a rejeté le roi et le sacrificateur.
\VS{7}[Zayin.] Le Seigneur a rejeté au loin son autel, il a dédaigné son sanctuaire ; il a livré entre les mains de l'ennemi les murailles de ses palais ; ils ont poussé des cris dans la maison de Yahweh, comme aux jours des fêtes solennelles.
\VS{8}[Heth.] Yahweh avait projeté de détruire les murailles de la fille de Sion ; il a étendu le cordeau, il n'a pas fait revenir sa main sans les avoir engloutis ; il a plongé dans le deuil remparts et murailles, ils ont été ruinés tous ensemble.
\VS{9}[Teth.] Ses portes sont enfoncées dans la terre ; il en a détruit et brisé les barres. Son roi et ses chefs sont parmi les nations ; la loi n'est plus. Même les prophètes ne reçoivent plus aucune vision de Yahweh\FTNT{Ez. 7:26.}.
\VS{10}[Yod.] Les anciens de la fille de Sion sont assis à terre, ils sont muets ; ils ont couvert leur tête de poussière, ils se sont ceints de sacs ; les vierges de Jérusalem baissent leurs têtes vers la terre.
\VS{11}[Kaf.] Mes yeux se consument à force de larmes, mes entrailles bouillonnent, ma bile se répand sur la terre. À cause des ruines de la fille de mon peuple, des enfants et des nourrissons qui tombent en défaillance dans les rues de la ville.
\VS{12}[Lamed.] Ils disaient à leurs mères : Où y a-t-il du blé et du vin ? Et ils tombaient comme morts dans les rues de la ville, comme un homme blessé à mort, ils rendaient l'âme sur le sein de leurs mères.
\VS{13}[Mem] Qui dois-je prendre à témoin ? À qui te comparer, fille de Jérusalem ? Qui pourrait t'égaler, et quelle consolation te donner, vierge, fille de Sion ? Car ta ruine est grande comme une mer : Qui pourrait te guérir\FTNT{Es. 51:19-20.} ?
\VS{14}[Nun.] Tes prophètes ont eu pour toi des visions vaines et insensées ; ils n'ont pas découvert ton iniquité, afin de détourner ta captivité ; ils t'ont prophétisé des oracles mensongers et trompeurs\FTNT{Jé. 2:8 ; Jé. 5:31 ; Jé. 14:14.}.
\VS{15}[Samech.] Tous les passants applaudissent sur toi, ils sifflent, ils secouent leur tête contre la fille de Jérusalem : Est-ce ici la ville de laquelle on disait : La parfaite en beauté, la joie de toute la terre\FTNT{Na. 3:19.} ?
\VS{16}[Pe.] Tous tes ennemis ouvrent la bouche contre toi, ils sifflent, ils grincent des dents, ils disent : Nous l'avons engloutie ! C'est ici le jour que nous attendions, nous l'avons atteint, nous le voyons !
\VS{17}[Ayin.] Yahweh a fait ce qu'il avait projeté, il a accompli sa parole qu'il avait ordonnée depuis longtemps, il a détruit sans épargner, il a fait de toi la joie de l'ennemi, il a donné de la force à tes adversaires.
\VS{18}[Tsade.] Leur cœur crie au Seigneur… Muraille de la fille de Sion, fais couler des larmes jour et nuit, comme un torrent\FTNT{Jé. 14:17.} ! Ne te donne pas de repos ; et que la prunelle de tes yeux ne se repose pas !
\VS{19}[Qof.] Lève-toi, pousse des cris dès le commencement des veilles de la nuit ! Répands ton cœur comme de l'eau en présence du Seigneur ! Lève tes mains vers lui pour l'âme de tes enfants qui meurent de faim aux coins de toutes les rues !
\VS{20}[Resh.] Vois, ô Yahweh ! Regarde qui tu as traité avec sévérité ! Les femmes n'ont-elles pas mangé leur fruit : leurs petits enfants objets de leur tendresse ? Le sacrificateur et le prophète n'ont-ils pas été tués dans le sanctuaire du Seigneur\FTNT{Lé. 26:29 ; De. 28:53 ; Jé. 19:9.} ?
\VS{21}[Shin.] Les jeunes gens et les vieillards sont couchés par terre dans les rues ; mes vierges et mes jeunes hommes sont tombés par l'épée ; tu as tué au jour de ta colère, tu as massacré sans épargner.
\VS{22}[Tav.] Tu as convié comme pour un jour solennel mes frayeurs de toutes parts. Au jour de la colère de Yahweh, il n'y a eu ni réchappé ni survivant. Ceux que j'avais langés et élevés, mon ennemi les a consumés.
\Chap{3}
\TextTitle{Jérémie partage l'affliction des siens}
\VerseOne{}[Aleph.] Je suis l'homme qui a vu l'affliction par la verge de sa fureur\FTNT{Jé. 15:15-18.}.
\VS{2}Il m'a conduit, mené dans les ténèbres, et non dans la lumière.
\VS{3}Certes c'est contre moi qu'il a tout le jour tourné et retourné sa main.
\VS{4}[Beth.] Il a fait vieillir ma chair et ma peau, il a brisé mes os\FTNT{Es. 38:13.}.
\VS{5}Il a bâti autour de moi, il m'a environné de venin et de peine.
\VS{6}Il me fait habiter dans les lieux ténèbreux, comme ceux qui sont morts depuis longtemps.
\VS{7}[Guimel.] Il a fait une cloison autour de moi, afin que je ne sorte point ; il a appesanti mes chaînes.
\VS{8}Même quand je crie et que j'élève ma voix, il rejette ma prière.
\VS{9}Il a fait un mur de pierres de taille pour fermer mes chemins, il a renversé mes sentiers.
\VS{10}[Daleth.] Il a été pour moi un ours en embuscade, un lion qui se tient dans un lieu caché\FTNT{Os. 13:8.}.
\VS{11}Il a détourné mes chemins, il m'a mis en pièces, il m'a mis dans la désolation.
\VS{12}Il a tendu son arc, et il m'a placé comme une cible pour sa flèche.
\VS{13}[He.] Il a fait entrer dans mes reins les flèches de son carquois.
\VS{14}Je suis la risée pour tout mon peuple, et leur chanson\FTNT{Ps. 69:13 ; Job. 30:9.} tout le jour.
\VS{15}Il m'a rassasié d'amertume, il m'a enivré d'absinthe.
\VS{16}[Vav.] Il a brisé mes dents avec du gravier, il m'a couvert de cendres.
\VS{17}Tellement que la paix s'est éloignée de mon âme, j'ai oublié ce que c'est que d'être à son aise.
\VS{18}Et j'ai dit : Ma force est perdue, et mon espérance aussi que j'avais en Yahweh.
\VS{19}[Zayin.] Souviens-toi de mon affliction, et de mon pauvre état qui n'est qu'absinthe et que fiel ;
\VS{20}Mon âme s'en souvient sans cesse, et elle est abattue au-dedans de moi.
\VS{21}Mais je rappellerai ceci en mon coeur, et c'est pourquoi j'aurai de l'espérance :
\VS{22}[Heth.] C'est une grâce de Yahweh que nous n'avons point été consumés parce que ses compassions ne sont pas épuisées\FTNT{Ps. 103:10.} ;
\VS{23}elles se renouvellent chaque matin. C'est une chose grande que ta fidélité !
\VS{24}Yahweh est ma portion, dit mon âme ; c'est pourquoi j'aurai espérance en lui\FTNT{Ps. 16:5.}.
\VS{25}[Teth.] Yahweh est bon pour ceux qui s'attendent à lui, pour l'âme qui le cherche.
\VS{26}Il est bon d'espérer et d'attendre en silence la délivrance de Yahweh.
\VS{27}Il est bon pour l'homme de porter le joug dans sa jeunesse.
\VS{28}[Yod.] Il sera assis solitaire et silencieux parce qu'on le lui impose.
\VS{29}Il mettra sa bouche dans la poussière, peut-être y aura-t-il quelque espérance ?
\VS{30}Il présentera la joue à celui qui le frappe, il se rassasiera d'opprobres.
\VS{31}[Kaf.] Car le Seigneur ne rejette pas à toujours\FTNT{Es. 57:16 ; Ps. 77:8.}.
\VS{32}Mais s'il afflige quelqu'un, il a aussi compassion selon la grandeur de sa miséricorde.
\VS{33}Car ce n'est pas sa volonté d'affliger et d'humilier les fils des hommes.
\VS{34}[Lamed.] Lorsqu'on foule aux pieds tous les prisonniers de la terre,
\VS{35}lorsqu'on pervertit la justice humaine en la présence du Très-Haut,
\VS{36}lorsqu'on fait tort à quelqu'un dans son procès, le Seigneur ne le voit-il pas ?
\VS{37}[Mem.] Qui est-ce qui dit qu'une chose est arrivée sans que le Seigneur l'ait commandé ?
\VS{38}Les maux et les biens\FTNT{Es. 45:7 ; Am. 3:6 ; Job. 1:21.} ne procèdent-ils pas de la bouche du Très-Haut ?
\VS{39}Pourquoi un homme vivant se plaindrait-il, un homme, à cause de la peine de ses péchés ?
\TextTitle{Le peuple appelé à s'examiner pour revenir à Yahweh}
\VS{40}[Nun.] Recherchons nos voies, sondons-les, et retournons à Yahweh\FTNT{Ps. 119:59 ; 2 Co. 13:5.} ;
\VS{41}élevons nos cœurs et nos mains vers Dieu qui est au ciel :
\VS{42}Nous avons péché, nous avons été rebelles ! Tu n'as pas pardonné !
\VS{43}[Samech.] Tu nous as couverts de ta colère, et tu nous as poursuivis ; tu as tué sans épargner ;
\VS{44}tu t'es couvert d'une nuée pour que les prières ne te parviennent pas.
\VS{45}Tu nous as fait être la raclure et le rebut au milieu des peuples.
\VS{46}[Pe.] Tous nos ennemis ouvrent leur bouche contre nous.
\VS{47}La frayeur et la fosse, le dégât et la calamité nous sont arrivés\FTNT{Es. 24:18 ; Jé. 48:44.}.
\VS{48}De mes yeux coulent des torrents d'eau à cause de la ruine de la fille de mon peuple.
\VS{49}[Ayin.] Mon œil fond en larmes, sans repos, sans relâche,
\VS{50}jusqu'à ce que Yahweh regarde et voie des cieux\FTNT{Ps. 80:15 ; Ps. 102:20.} ;
\VS{51}mon œil fait souffrir mon âme à cause de toutes les filles de ma ville.
\TextTitle{Yahweh, le soutien de Jérémie dans la détresse}
\VS{52}[Tsade.] Ceux qui sont mes ennemis sans cause m'ont poursuivi à outrance, comme après un oiseau.
\VS{53}Ils ont voulu anéantir ma vie dans une fosse, et ils ont jeté une pierre sur moi.
\VS{54}Les eaux ont coulé par-dessus ma tête ; je disais : Je suis retranché !
\VS{55}[Qof.] J'ai invoqué ton nom, ô Yahweh, du fond de la fosse\FTNT{Jé. 38:6.}.
\VS{56}Tu as entendu ma voix : Ne ferme pas tes oreilles à mes soupirs, à mes cris !
\VS{57}Au jour où je t'ai invoqué, tu t'es approché, et tu as dit : Ne crains rien !
\VS{58}[Resh.] Ô Seigneur, tu as plaidé la cause de mon âme, tu as racheté ma vie.
\VS{59}Tu as vu, ô Yahweh ! le tort qu'on me fait, fais-moi justice !
\VS{60}Tu as vu toutes les vengeances dont ils ont usé, et toutes leurs machinations contre moi.
\VS{61}[Shin.] Yahweh, tu as entendu leurs outrages, toutes leurs machinations contre moi,
\VS{62}les discours de ceux qui se lèvent contre moi, et leur dessein qu'ils ont contre moi tout au long du jour.
\VS{63}Considère quand ils sont assis et quand ils se lèvent, car je suis leur chanson.
\VS{64}[Tav.] Rends-leur la pareille, ô Yahweh, selon l'œuvre de leurs mains ;
\VS{65}livre-les à l'endurcissement de leur cœur, à ta malédiction.
\VS{66}Poursuis-les dans ta colère, et extermine-les de dessous les cieux, ô Yahweh !
\Chap{4}
\TextTitle{Crimes et apostasie du peuple}
\VerseOne{}[Aleph.] Comment l'or est-il devenu obscur, et le fin or s'est-il altéré ? Comment les pierres du sanctuaire sont-elles répandues aux coins de toutes les rues ?
\VS{2}[Beth.] Comment les chers fils de Sion, qui étaient estimés à l'égal de l'or pur, sont-ils reputés comme des vases de terre, ouvrage des mains du potier !
\VS{3}[Guimel.] Il y a même des monstres marins qui présentent leurs mamelles et allaitent leurs petits ; mais la fille de mon peuple est devenue cruelle comme les autruches du désert.
\VS{4}[Daleth.] La langue de celui qui têtait s'est attachée à son palais dans sa soif ; les enfants demandent du pain, et personne ne leur en donne\FTNT{Jé. 52:6.}.
\VS{5}[He.] Ceux qui mangeaient des mets délicats sont en désolation dans les rues ; ceux qui étaient nourris sur l'étoffe écarlate embrassent le fumier.
\VS{6}[Vav.] L'iniquité de la fille de mon peuple est plus grande que le péché de Sodome, renversée en un instant, sans que personne n'ait tourné la main sur elle.
\VS{7}[Zayin.] Ses naziréens étaient plus purs que la neige, plus blancs que le lait ; leur teint était plus vermeil que les pierres précieuses ; ils étaient polis comme un saphir.
\VS{8}[Heth.] Leur apparence est plus sombre que le noir ; on ne les reconnaît pas dans les rues ; ils ont la peau collée sur les os ; elle est devenue sèche comme du bois\FTNT{Job. 30:30.}.
\VS{9}[Teth.]Ceux qui ont été mis à mort par l'épée, ont été plus heureux que ceux qui sont morts par la famine, qui eux sont consumés peu à peu, transpercés par le défaut du fruit des champs.
\VS{10}[Yod.] Les mains des femmes, naturellement tendres, font cuire leurs enfants ; ils leur servent de nourriture dans la ruine de la fille de mon peuple\FTNT{De. 28:57 ; 2 R. 6:29.}.
\VS{11}[Kaf.] Yahweh a accompli sa fureur, il a répandu l'ardeur de sa colère ; il a allumé dans Sion un feu qui en dévore les fondements.
\VS{12}[Lamed.] Les rois de la terre, et tous les habitants de la terre habitable n'auraient jamais cru que l'adversaire et l'ennemi entrerait dans les portes de Jérusalem.
\VS{13}[Mem.] Cela est arrivé à cause des péchés de ses prophètes, et des iniquités de ses sacrificateurs, qui répandaient le sang des justes au milieu d'elle\FTNT{Jé. 5:29-31. Le péché des conducteurs donne accès à l'ennemi pour les détruire ainsi que les biens qui leur ont été confiés (Lu. 11:21-22).}.
\VS{14}[Nun.] Ils erraient comme des aveugles dans les rues, souillés de sang, au point qu'on ne pouvait pas toucher leurs vêtements.
\VS{15}[Samech.] On leur criait : retirez-vous, souillés, retirez-vous, retirez-vous, ne nous touchez point. Quand ils se sont enfuis, ils ont erré ça et là ; on a dit parmi les nations : Ils n'auront plus leur demeure !
\VS{16}[Pe.] La face de Yahweh les a dispersés, il ne veut plus les regarder ; ils n'ont pas eu de respect pour les sacrificateurs, et n'ont pas été miséricordieux envers les vieillards.
\VS{17}[Ayin.] Pour nous, nos yeux se consumaient après un vain secours ; nous regardions du haut de nos lieux élevés vers une nation qui ne pouvait pas délivrer\FTNT{Jé. 18:15.}.
\VS{18}[Tsade.] Ils ont épié nos pas afin de nous empêcher d'aller sur nos places ; notre fin s'approchait, nos jours étaient accomplis… Notre fin est arrivée !
\VS{19}[Qof.] Nos persécuteurs étaient plus légers que les aigles des cieux ; ils nous ont poursuivis sur les montagnes, ils ont mis des embûches contre nous dans le désert.
\VS{20}[Resh.] Le souffle de nos narines, l'oint de Yahweh\FTNT{L'oint en question est le roi Josias (2 R. 21:24 ; 22 ; 23).}, a été pris dans leurs fosses, celui de qui nous disions : Nous vivrons sous son ombre parmi les nations.
\VS{21}[Shin.] Réjouis-toi, sois dans l'allégresse, fille d'Edom, habitante du pays d'Uts ! La coupe passera aussi vers toi ; tu en seras enivrée, et tu seras mise à nu\FTNT{Jé. 25:15-18 ; Ps. 137:7.}.
\VS{22}[Tav.] Fille de Sion, ton iniquité est expiée ; il ne t'enverra plus en exil. Fille d'Edom, il châtiera ton iniquité, il découvrira tes péchés.
\Chap{5}
\TextTitle{Supplications de Jérémie à Yahweh}
\VerseOne{}Souviens-toi, ô Yahweh, de ce qui nous est arrivé ! Regarde et vois notre opprobre !
\VS{2}Notre héritage a été renversé par des étrangers, nos maisons par des inconnus.
\VS{3}Nous sommes devenus comme des orphelins qui sont sans pères, et nos mères sont comme des veuves.
\VS{4}Nous buvons notre eau à prix d'argent, et notre bois nous est vendu.
\VS{5}Ceux qui nous poursuivent sont sur notre cou ; nous sommes épuisés, nous n'avons pas de repos.
\VS{6}Nous avons étendu la main vers l'Egypte, et vers l'Assyrie pour nous rassasier de pain.
\VS{7}Nos pères ont péché, ils ne sont plus, et c'est nous qui portons la peine de leurs iniquités\FTNT{Chaque homme naît pécheur et hérite de la nature pécheresse d'Adam (Ge. 3:20 ; Ac. 17:26 ; Ro. 5:12-21). De ce fait, les péchés commis par les parents ont des conséquences sur les enfants (Ex. 20:4-5 ). Jésus-Christ nous a délivrés du péché d'Adam et de celui de nos ancêtres à la croix (Col. 1:12). Lors de notre naissance d'en haut, les péchés de notre passé et de nos origines sont expiés (2 Co. 5:17 ; Ep. 1:7 ; Ep. 2:1-15 ; Col. 1:12-14 ; Col. 2:13-15 ; 1 Pi. 1:18-19 ; 1 Jn. 1:7 ; 1 Jn. 1:9). Les péchés des ancêtres et leurs conséquences touchent les personnes qui vivent dans les péchés de leurs ancêtres, c'est-à-dire ceux qui haïssent Dieu et ses commandements (De. 24:16 ; Jé. 31:29-34 ; Ez. 18:17-20).}.
\VS{8}Les esclaves dominent sur nous, et personne ne nous délivre de leurs mains.
\VS{9}Nous amenons notre pain au péril de notre vie, à cause de l'épée du désert.
\VS{10}Notre peau est brûlante comme un four, à cause l'ardeur de la faim.
\VS{11}Ils ont déshonoré les femmes dans Sion, les vierges dans les villes de Juda.
\VS{12}Des chefs ont été pendus par leurs mains ; et ils n'ont pas honoré la personne des vieillards.
\VS{13}Ils ont pris les jeunes gens pour moudre, et les enfants sont tombés sous le bois.
\VS{14}Les vieillards ont cessé de se trouver aux portes, et les jeunes gens de chanter.
\VS{15}La joie a disparu de notre cœur, et notre danse est changée en deuil.
\VS{16}La couronne de notre tête est tombée ! Malheur à nous, parce que nous avons péché !
\VS{17}C'est pourquoi notre coeur est languissant. À cause de ces choses, nos yeux sont obscurcis.
\VS{18}À cause de la montagne de Sion qui est désolée ; les renards s'y promènent.
\VS{19}Toi, ô Yahweh, tu demeures éternellement, et ton trône subsiste de génération en génération.
\VS{20}Pourquoi nous oublierais-tu à jamais ? pourquoi nous délaisserais-tu si longtemps ?
\VS{21}Convertis-nous à toi, ô Yahweh ! et nous serons convertis ; renouvelle nos jours comme ils étaient autrefois\FTNT{Jé. 30:20 ; Jé. 31:18 ; Ps. 80:3.}.
\VS{22}Ou bien, nous aurais-tu entièrement rejetés ? Serais-tu extrêmement courroucé contre nous ?
\PPE{}
\end{multicols}

\clearpage\ShortTitle{Ec.}\BookTitle{Ecclésiaste}\BFont
\noindent\hrulefill
{\footnotesize
\textit{
\bigskip
{\centering{}
\\Auteur~: Salomon
\\(Heb.~: Qohelet)
\\Signification~: Prédicateur
\\Thème~: Les raisonnements humains
\\Date de rédaction~: 10\up{ème} siècle av. J.-C.\\}
}
\textit{
\\Ce livre, du fait de sa date de rédaction, est généralement attribué à Salomon en raison de l'allusion faite au premier chapitre et du style adopté. L'Ecclésiaste figure d'ailleurs dans le canon des livres reconnus d'inspiration divine.
\\La problématique centrale du livre est de savoir si la vie vaut la peine d'être vécue ou non. L'auteur y répondit en connaissance de cause car il avait obtenu tout ce que l'homme pouvait désirer~: les richesses, le luxe, la volupté, la sagesse… Sans pour autant incriminer Dieu, il dressa le constat de ce qu'est l'expérience humaine. Selon lui, l'homme vit dans un cycle d'éternels recommencements où tout n'est que poursuite du vent et vanité.\bigskip
}
}
\par\nobreak\noindent\hrulefill
\begin{multicols}{2}
\Chap{1}
\TextTitle{Tout est vanité\FTNTT{Ec. 12:8.}}
\VerseOne{}Les Paroles de l'Ecclésiaste, fils de David, roi de Jérusalem.
\VS{2}Vanité des vanités, dit l'Ecclésiaste, vanité des vanités, tout est vanité.
\TextTitle{Le cycle du temps}
\VS{3}Quel avantage a l'homme de tout son travail auquel il s'occupe sous le soleil~?\FTNT{Ec. 2:22~; Ec. 3:9.}
\VS{4}Une génération passe et une autre génération vient, mais la terre demeure toujours ferme.
\VS{5}Le soleil aussi se lève et le soleil se couche~; il soupire après le lieu d'où il se lève.
\VS{6}Le vent va vers le midi, tourne vers le nord~; il va tournoyant çà et là, et il retourne après ses circuits.
\VS{7}Tous les fleuves vont à la mer, et la mer n'en est point remplie~; les fleuves retournent au lieu d'où ils étaient partis, pour revenir\FTNT{Job 38:8-11~; Ps. 104:9-10.} à la mer. 
\VS{8}Toutes les choses sont lassantes et l'homme ne peut en parler~; l'œil n'est jamais rassasié de voir\FTNT{Pr. 27:20.} et l'oreille ne se lasse pas d'entendre. 
\VS{9}Ce qui a été, c'est ce qui sera, et ce qui s'est fait, c'est ce qui se fera, il n'y a rien de nouveau sous le soleil\FTNT{Ec. 3:15.}.
\VS{10}Y~a-t-il quelque chose dont on puisse dire~: Regarde cela, il est nouveau~? Il a déjà été dans les siècles qui ont été avant nous.
\VS{11}On ne se souvient plus des choses d'autrefois~; de même on ne se souviendra point des choses à venir et ceux qui viendront n'en auront aucun souvenir. 
\TextTitle{La sagesse des hommes ne comble pas}
\VS{12}Moi l'Ecclésiaste, j'ai été roi sur Israël à Jérusalem.
\VS{13}Et j'ai appliqué mon cœur à rechercher et à sonder par la sagesse tout ce qui se fait sous les cieux~: C'est une occupation désagréable que Dieu a donnée aux hommes, afin qu'ils s'y occupent\FTNT{1 R. 4:30-34~; Ec. 7:25.}.
\VS{14}J'ai vu toutes les œuvres qui se font sous le soleil~; et voici tout est vanité et tourment d'esprit.
\VS{15}Ce qui est courbé ne peut se redresser, et ce qui manque ne peut être compté.
\VS{16}J'ai parlé en mon cœur, disant~: Voici, je suis devenu grand et j'ai surpassé en sagesse tous ceux qui ont été avant moi sur Jérusalem, et mon cœur a vu beaucoup de sagesse et de science.
\VS{17}Et j'ai appliqué mon cœur à connaître la sagesse, et à connaître les sottises et la stupidité~; et j'ai reconnu que cela aussi était un tourment d'esprit.
\VS{18}Car là où il y a beaucoup de sagesse, il y a beaucoup de chagrin, et celui qui augmente sa connaissance, augmente son chagrin.
\Chap{2}
\TextTitle{Les richesses ne comblent pas}
\VerseOne{}J'ai dit en mon cœur~: Allons, que je t'éprouve maintenant par la joie et prends du bon temps. Et voici, c'est encore une vanité\FTNT{Lu. 12:19.}.
\VS{2}J'ai dit concernant le rire~: Il est insensé~! Et concernant la joie~: A quoi sert-elle~?
\VS{3}J'ai recherché en moi-même le moyen de me traiter délicatement, de faire que mon cœur s'accoutume cependant à la sagesse, et qu'il comprenne ce que c'est que la folie, jusqu'à ce que je voie ce qu'il est bon aux hommes de faire sous les cieux, pendant les jours de leur vie. 
\VS{4}Je me suis fait des choses magnifiques~; je me suis bâti des maisons~; je me suis planté des vignes.
\VS{5}Je me suis fait des jardins et des vergers, et j'y plantai des arbres fruitiers de toutes sortes~;
\VS{6}je me suis fait des réservoirs d'eaux pour arroser la forêt où poussent les arbres.
\VS{7}J'ai acquis des hommes et des femmes esclaves~; et j'ai eu des esclaves nés dans ma maison, et j'ai eu plus de gros et de menu bétail que tous ceux qui ont été avant moi dans Jérusalem. 
\VS{8}Je me suis aussi amassé de l'argent et de l'or, et des plus précieux joyaux qui se trouvent chez les rois et dans les provinces\FTNT{1 R. 9:28~; 1 R. 10:10~; 2 Ch. 1:15.}. Je me suis acquis des chanteurs et des chanteuses, et les délices des hommes~; une harmonie d'instruments de musique, même plusieurs harmonies de toutes sortes d'instruments\FTNT{La plupart des bibles ont traduit la deuxième partie de ce verset par le mot «~femme~», or le sens du terme hébreu «~shiddah~» est incertain. Toutefois, le contexte de ce verset montre clairement que Salomon parlait des instruments de musique et des chanteurs et chanteuses qu'il a acquis, et non de ses conquêtes féminines.}. 
\VS{9}Je me suis aggrandi et me suis accru plus que tous ceux qui ont été avant moi dans Jérusalem. Et ma sagesse est demeurée avec moi.
\VS{10}Enfin je n'ai rien refusé à mes yeux de tout ce qu'ils ont demandé~; et je n'ai épargné aucune joie à mon cœur~; car mon cœur s'est réjoui de tout mon travail et c'est là tout ce que j'ai eu de tout mon travail.
\VS{11}Mais ayant considéré toutes mes œuvres que mes mains avaient faites, et tout le travail auquel je m'étais occupé en les faisant, voilà tout était vanité et tourment d'esprit~; tellement que l'homme n'a aucun avantage de ce qui est sous le soleil.
\TextTitle{Le sage et l'insensé ont le même sort}
\VS{12}Puis je me suis mis à considérer tant la sagesse que les sottises et la folie, (or qui est l'homme qui pourrait suivre le roi dans ce qui a été déjà fait~?).
\VS{13}Et j'ai vu que la sagesse a beaucoup d'avantage sur la folie, comme la lumière a beaucoup d'avantage sur les ténèbres.
\VS{14}Le sage a ses yeux à sa tête, et l'insensé marche dans les ténèbres. Mais j'ai aussi reconnu qu'un même sort leur arrive à tous\FTNT{Ps. 49:11~; Ec. 3:17~; Ec. 9:2.}.
\VS{15}C'est pourquoi j'ai dit en mon cœur~: Il m'arrivera le même sort que l'insensé~; de quoi donc me servira-t-il alors d'avoir été plus sage~? C'est pourquoi j'ai dit en mon cœur que cela aussi est une vanité. 
\VS{16}Car le souvenir du sage n'est pas plus éternel que celui de l'insensé, parce que ce qui est maintenant va être oublié dans les jours qui suivent. Le sage meurt aussi bien que l'insensé\FTNT{Ec. 8:10~; Ec. 9:5.}~!
\VS{17}C'est pourquoi j'ai haï cette vie, car les choses qui se sont faites sous le soleil m'ont déplu~; car tout est vanité, et tourment d'esprit.
\VS{18}J'ai aussi haï tout mon travail, auquel je me suis occupé sous le soleil, parce que je le laisserai à l'homme qui sera après moi\FTNT{Ec. 4:8.}.
\VS{19}Et qui sait s'il sera sage ou insensé~? Cependant il sera maître de tout mon travail, auquel je me suis occupé et de ce en quoi j'ai été sage sous le soleil. Cela aussi est une vanité.
\VS{20}C'est pourquoi j'ai fait en sorte que mon cœur perde toute espérance de tout le travail auquel je m'étais occupé sous le soleil.
\VS{21}Car il y a tel homme, dont le travail a été avec sagesse, science, et adresse, qui néanmoins le laisse à celui qui n'y a point travaillé comme étant sa part~; cela aussi est une vanité et un grand mal. 
\VS{22}Car qu'est-ce que l'homme a de tout son travail et du désir de son cœur, dont il souffre sous le soleil~?
\VS{23}Puisque tous ses jours ne sont que douleur, et son occupation n'est que chagrin~; même la nuit son cœur ne repose point. Cela aussi est une vanité\FTNT{Ps. 90:9~; Job. 14:1.}.
\VS{24}N'est-ce donc pas un bien pour l'homme de manger, et de boire, et de faire que son âme jouisse du bien dans son travail~? J'ai vu aussi que cela vient de la main de Dieu\FTNT{Ec. 3:12~; Ec. 3:22~; Ec. 5:18~; Ec. 8:15.}.
\VS{25}Car qui en mangera, qui s'est réjoui plus que moi~?
\VS{26}Parce que Dieu donne à celui qui lui est agréable, de la sagesse, de la science et de la joie~; mais il donne au pécheur de l'occupation à recueillir et à assembler, afin que cela soit donné à celui qui est agréable à Dieu. Cela aussi est une vanité et un tourment d'esprit\FTNT{Pr. 13:22~; Pr. 28:8~; Job 27:17.}.
\Chap{3}
\TextTitle{Il y a un temps pour toute chose}
\VerseOne{}A toute chose sa saison, et à toute affaire sous les cieux son temps.
\VS{2}Un temps pour naître et un temps pour mourir~; un temps pour planter et un temps pour arracher ce qui est planté~;
\VS{3}un temps pour tuer et un temps pour guérir~; un temps pour démolir et un temps pour bâtir~;
\VS{4}un temps pour pleurer et un temps pour rire~; un temps pour se lamenter et un temps pour sauter de joie~;
\VS{5}un temps pour jeter des pierres et un temps pour ramasser des pierres~; un temps pour embrasser et un temps pour s'éloigner des embrassements~;
\VS{6}un temps pour chercher et un temps pour perdre~; un temps pour garder et un temps pour jeter~;
\VS{7}un temps pour déchirer et un temps pour coudre~; un temps pour être silencieux et un temps pour parler~;
\VS{8}un temps pour aimer et un temps pour haïr~; un temps pour la guerre et un temps pour la paix.
\TextTitle{Dieu fait toute chose belle en son temps}
\VS{9}Quel avantage celui qui travaille a-t-il de sa peine~?
\VS{10}J'ai considéré cette occupation que Dieu a donnée aux fils des hommes pour s'y appliquer.
\VS{11}Il a fait que toutes choses sont belles en leur temps~; aussi a-t-il mis l'éternité dans leur cœur, sans toutefois que l'homme puisse comprendre du commencement à la fin\FTNT{Ec. 8:17.}l'œuvre que Dieu a faite. 
\VS{12}C'est pourquoi j'ai reconnu qu'il n'y a rien de meilleur aux hommes, que de se réjouir et de se faire du bien pendant leur vie. 
\VS{13}Et même si un homme mange et boit et jouit du bien-être de tout son travail, c'est un don de Dieu\FTNT{Ec. 5:18~; Ec. 8:15~; Ec. 9:7.}.
\VS{14}J'ai reconnu que tout ce que Dieu fait subsiste à toujours, il n'y a rien à y ajouter et rien à en retrancher, et Dieu le fait afin que devant lui, on le craigne.
\VS{15}Ce qui a été, est maintenant~; et ce qui doit être, a déjà été~; et Dieu rappelle ce qui est passé.
\VS{16}J'ai encore vu sous le soleil qu'au lieu établi pour juger, il y a de la méchanceté~; et qu'au lieu établi pour la justice, il y a de la méchanceté.
\VS{17}J'ai dit en mon cœur~: Dieu jugera le juste et le méchant~; car il y a là un temps pour toute chose et pour toute œuvre.
\VS{18}J'ai dit en mon cœur sur l'état des fils de l'homme, que Dieu les éprouverait, et qu'ils verraient qu'ils ne sont que des bêtes.
\VS{19}Car le sort des fils d'Adam et le sort de la bête est un même sort~; telle qu'est la mort de l'un, telle est la mort de l'autre. Tous ont un même souffle et la supériorité de l'homme sur la bête est nulle. Car tout est vanité.
\VS{20}Tout va dans un même lieu~; tout a été fait de la poussière, et tout retourne à la poussière\FTNT{Ge. 3:19~; Job 34:15~; Ec. 6:6~; Ec. 12:9.}.
\VS{21}Qui sait si l'esprit des fils de l'homme monte en haut, et si l'esprit de la bête descend en bas dans la terre\FTNT{Les animaux comme les hommes ont une âme et un esprit (Ge. 1:20~; Ez. 1:1-28). A leur mort, leurs esprits quittent leurs corps (Ja. 2:26). L'homme régénéré reçoit le Saint-Esprit, ce qui n'est pas le cas des animaux (Ro. 8:16). Les animaux, tout comme la création tout entière, attendent leur rédemption, car ils ont été soumis à la corruption à cause du péché de l'homme (Ro. 8:19-22).Dans le royaume millénaire, il y aura des animaux (Es. 11:6-9).}~?
\VS{22}J'ai donc vu qu'il n'y a rien de meilleur pour l'homme que de se réjouir de ses œuvres~: C'est là sa part. Car qui le ramènera pour voir ce qui sera après lui~?
\Chap{4}
\TextTitle{Un monde injuste}
\VerseOne{}Puis je me suis mis à regarder toutes les injustices qui se font sous le soleil~; et voici les larmes de ceux à qui on fait tort, et ils n'ont point de consolation. Et la force est du côté de ceux qui leur font tort, et ils n'ont point de consolateur. 
\VS{2}C'est pourquoi j'estime plus les morts qui sont déjà morts, que les vivants qui sont encore vivants\FTNT{Ec. 7:1.}~;
\VS{3}même j'estime celui qui n'a pas encore été, plus heureux que les uns et les autres~; car il n'a pas vu les mauvaises actions qui se font sous le soleil.
\VS{4}Puis j'ai vu que tout travail et tout succès dans le travail n'est que jalousie de l'un à l'égard de l'autre. Cela aussi est une vanité et un tourment d'esprit. 
\VS{5}L'insensé se croise les mains et dévore sa propre chair\FTNT{Pr. 6:10~; Pr. 19:24~; Pr. 24:33~; Pr. 26:15.}.
\VS{6}Mieux vaut le creux de la main pleine avec repos, que les deux mains pleines avec travail et tourment d'esprit\FTNT{Ps. 37:16~; Pr. 15:16-17~; Pr. 16:8.}. 
\VS{7}Puis je me suis mis à regarder une autre vanité sous le soleil. 
\VS{8}C'est qu'il y a tel qui est seul, et qui n'a point de second, qui aussi n'a ni fils ni frère et qui cependant ne met nulle fin à son travail~; même son œil ne voit jamais assez de richesses, et il ne se dit point en lui-même~: Pour qui est-ce que je travaille, et que je prive mon âme du bien~? Cela aussi est une vanité et une fâcheuse occupation\FTNT{Ec. 2:26~; Ps. 39:7~; Lu. 12:20.}.
\VS{9}Deux valent mieux qu'un, car ils ont un meilleur salaire de leur travail.
\VS{10}Même si l'un des deux tombe, l'autre relèvera son compagnon~; mais malheur à celui qui est seul~; parce qu'étant tombé, il n'aura personne pour le relever. 
\VS{11}Si deux aussi couchent ensemble, ils en auront plus de chaleur~; mais celui qui est seul, comment aura-t-il chaud~? 
\VS{12}Et si quelqu'un a le dessus sur l'un ou sur l'autre, les deux peuvent lui résister~; et la corde à trois cordons ne se rompt pas rapidement.
\VS{13}Un enfant pauvre et sage vaut mieux qu'un roi vieux et insensé qui ne sait ce que c'est que d'être averti.
\VS{14}Car tel qui sort de prison pour régner, et de même tel étant né roi, devient pauvre dans son royaume.
\VS{15}J'ai vu tous les vivants qui marchent sous le soleil suivre le fils qui est la seconde personne après le roi, et qui doit être à sa place. 
\VS{16}Il n'y a pas de fin à tout le peuple, à tous ceux qui ont été devant eux~; cependant ceux qui viendront après ne se réjouiront point en lui. Certainement cela aussi est une vanité, et un tourment d'esprit. 
\TextTitle{Le sacrifice des insensés}
\VS{17}Quand tu entres dans la maison de Dieu, prends garde à ton pied, et approche-toi pour écouter, plutôt que pour donner ce que donnent les insensés, car ils ne savent pas qu'ils font mal.
\Chap{5}
\VerseOne{}Ne te précipite point à parler, et que ton cœur ne se hâte point de parler devant Dieu~; car Dieu est au ciel, et toi sur la terre~; c'est pourquoi use de peu de paroles.
\VS{2}Car comme le songe vient de la multitude des occupations~; ainsi la voix des insensés sort de la multitude des paroles\FTNT{Pr. 10:19.}.
\VS{3}Quand tu as fais quelque vœu à Dieu, ne diffère point de l'accomplir~; car il ne prend point de plaisir aux insensés~; accomplis donc le vœu que tu as fait\FTNT{No. 30:3. De. 23:21}.
\VS{4}Il vaut mieux que tu ne fasses point de vœux que d'en faire et de ne pas les accomplir\FTNT{De. 23:21-22.}.
\VS{5}Ne permets pas à ta bouche de faire pécher ta chair, et ne dis point devant le messager de Dieu que c'est un péché involontaire. Pourquoi Yahweh s'irriterait-il de tes paroles, et détruirait-il l'œuvre de tes mains~?
\VS{6}Car comme dans la multitude des songes il y a des vanités, aussi y en a-t-il beaucoup dans la multitude des paroles~; mais crains Dieu\FTNT{Ec. 10:14~; Pr. 10:19.}.
\VS{7}Si tu vois dans la Province qu'on fasse tort au pauvre, et que le droit et la justice y soient violés, ne t'étonne point de cela~; car celui qui est plus élevé que les plus hauts élevés y prend garde, et il y en a de plus élevés qu'eux\FTNT{Es. 3:14-15.}.
\TextTitle{Vanité des richesses}
\VS{8}C'est un avantage pour le pays, un roi qui travaille dans les champs.
\VS{9}Celui qui aime l'argent n'est point rassasié par l'argent\FTNT{Jé. 6:13~; Pr. 22:7~; Pr. 28:16~; Mt. 6:33~; Mt. 7:7-11~; Lu. 12:13-20~; Ac. 20:33~; 2 Co. 9:5~; Ep. 4:19~; Ep. 5:5~; Col. 3:5~; 1 Ti. 6:10~; Hé. 13:5.}, et celui qui aime les richesses n'en est pas nourri~; cela aussi est une vanité. 
\VS{10}Où il y a beaucoup de bien, là il y a beaucoup de gens qui le mangent~; et quel avantage en revient-il à son maître, sinon qu'il le voit de ses yeux~? 
\VS{11}Le sommeil de celui qui travaille est doux, qu'il mange peu ou beaucoup~; mais le rassasiement du riche ne le laisse point dormir. 
\VS{12}Il y a un mal fâcheux que j'ai vu sous le soleil, c'est que des richesses sont conservées à leurs maîtres afin qu'ils en aient du mal. 
\VS{13}Ces richesses périssent par quelque fâcheux accident~; de sorte qu'on aura engendré un fils et il n'aura rien entre ses mains.
\VS{14}Et comme il est sorti du ventre de sa mère, il s'en retournera nu, s'en allant comme il était venu, et il n'emportera rien de son travail auquel il a employé ses mains\FTNT{1 Ti. 6:7.}.
\VS{15}Et c'est aussi un mal fâcheux, que comme il est venu, il s'en va de même~; et quel avantage a-t-il d'avoir travaillé pour du vent~?
\VS{16}Il mange aussi tous les jours de sa vie dans les ténèbres et se chagrine beaucoup, et son mal va jusqu'à la fureur.
\VS{17}Voilà donc ce que j'ai vu~; que c'est une chose bonne et agréable à l'homme de manger, de boire et de jouir du bien-être de tout son travail qu'il fait sous le soleil, pendant le nombre des jours de vie que Dieu lui a donnés~; car c'est là sa part.
\VS{18}Aussi ce que Dieu donne de richesses et de biens à un homme, quel qu'il soit~; ce dont il le fait maître pour en manger, pour en prendre sa part et pour se réjouir de son travail~; c'est là un don de Dieu. 
\VS{19}Car il ne se souviendra pas beaucoup des jours de sa vie, parce que Dieu lui répond par la joie de son cœur. 
\Chap{6}
\TextTitle{Vanité de la vie de l'homme}
\VerseOne{}Il y a un mal que j'ai vu sous le soleil, et qui est fréquent parmi les hommes.
\VS{2}C'est qu'il y a tel homme à qui Dieu a donné des richesses, des biens et des honneurs, en sorte qu'il ne manque rien pour son âme de tout ce qu'il peut souhaiter. Mais Dieu ne l'en fait pas le maître pour en manger, car un étranger le mangera. Cela est une vanité et un mal fâcheux. 
\VS{3}Quand un homme engendrerait cent fils, qu'il vivrait plusieurs années, en sorte que les jours de ses années se soient fort multipliés, cependant si son âme ne s'est point rassasiée de bien, et même s'il n'a point eu de sépulture, je dis qu'un avorton vaut mieux que lui.
\VS{4}Car il est venu en vain, et s'en va dans les ténèbres, et son nom est couvert de ténèbres~;
\VS{5}Il n'a même point vu le soleil~; il n'a rien connu~; il a plus de repos que cet homme-là\FTNT{Job 3:16.}.
\VS{6}Et s'il vivait deux fois mille ans, et qu'il ne jouit d'aucun bien, tous ne vont-ils pas dans un même lieu\FTNT{Ec. 3:20~; Job 3:13-19~; Job 30:23~; Ps. 89:48~; Hé. 9:27.}~?
\VS{7}Tout le travail de l'homme est pour sa bouche, et cependant son âme n'est jamais satisfaite\FTNT{Les richesses de ce monde ne peuvent jamais combler le vide de l'âme. Seul l'amour de Dieu peut réellement inonder nos âmes (Pr. 13:4).}.
\VS{8}Car qu'est-ce que le sage a de plus que l'insensé~? Ou quel avantage a le malheureux qui sait se conduire devant les vivants~?
\VS{9}Mieux vaut ce qu'on voit de ses yeux, que les grandes recherches que fait l'âme. Cela aussi est une vanité, et un tourment d'esprit\FTNT{1 Ti. 6:9.}.
\VS{10}Ce qui existe a déjà été appelé par son nom\FTNT{Ec. 1:9~; Ec. 3:15.}~; et savait-on ce que devait être l'homme, et qu'il ne pourrait plaider avec celui qui est plus fort que lui~? 
\VS{11}Quand on a beaucoup de choses, on a beaucoup de vanités. Quel avantage en a l'homme~? 
\VS{12}Car qui est-ce qui connaît ce qui est bon à l'homme dans sa vie, pendant les jours de la vie de sa vanité, lesquels il passe comme une ombre~? Et qui est-ce qui déclarera à l'homme ce qui sera après lui sous le soleil\FTNT{Ps. 144:4~; Ec. 8:7~; Ec. 8:13~; Ec. 10:14~; Ja. 4:13-14.}~?
\Chap{7}
\TextTitle{La sagesse qu'enseigne la vie de l'homme}
\VerseOne{}Une bonne réputation vaut mieux que le bon parfum, et le jour de la mort que le jour de la naissance\FTNT{Pr. 22:1.}.
\VS{2}Il vaut mieux aller dans une maison de deuil que d'aller dans une maison de festin~; car c'est là la fin de tout homme, et le vivant met cela dans son cœur.
\VS{3}Il vaut mieux le chagrin que le rire~; car par la tristesse du visage le cœur devient joyeux\FTNT{Ec. 8:1~; 2 Co. 7:10.}.
\VS{4}Le cœur des sages est dans la maison du deuil, mais le cœur des insensés est dans la maison de joie.
\VS{5}Il vaut mieux entendre la réprimande du sage, que d'entendre la chanson des hommes insensés\FTNT{Ps. 141:5~; Pr. 13:18~; Pr. 15:31-32.}.
\VS{6}Car tel qu'est le bruit des épines sous la chaudière, tel est le rire de l'insensé. Cela aussi est une vanité. 
\VS{7}Certainement l'oppression fait perdre le sens au sage~; et le don fait perdre l'entendement. 
\VS{8}Mieux vaut la fin d'une chose que son commencement. Mieux vaut l'homme qui est d'un esprit patient que l'homme qui est d'un esprit hautain. 
\VS{9}Ne te précipite point en ton esprit de t'irriter, car l'irritation repose dans le sein des insensés.
\VS{10}Ne dis point~: D'où vient que les jours passés ont été meilleurs que ceux-ci~? Car ce n'est pas par sagesse que tu demandes cela. 
\VS{11}La sagesse est bonne avec un héritage, elle est un avantage pour ceux qui voient le soleil.
\VS{12}Car on est à couvert à l'ombre de la sagesse, de même qu'à l'ombre de l'argent~; mais la science a cet avantage, que la sagesse fait vivre celui qui en est doué. 
\VS{13}Regarde l'œuvre de Dieu~: Qui pourra redresser ce qu'il a renversé~?
\VS{14}Au jour du bonheur, sois heureux, et au jour de l'adversité, prends-y garde~; car Dieu a fait l'un exactement comme l'autre, afin que l'homme ne trouve rien à redire après lui. 
\VS{15}J'ai vu tout ceci pendant les jours de ma vanité. Il y a tel juste qui périt dans sa justice, et il y a tel méchant qui prolonge ses jours dans sa méchanceté\FTNT{Ec. 8:14~; Job 21:7-8.}.
\VS{16}Ne te crois pas trop juste, et ne te fais pas plus sage qu'il ne faut~: Pourquoi t'exposer à la ruine\FTNT{Pr. 3:7~; Ro. 12:16.}~?
\VS{17}Ne sois point méchant à l'excès, et ne sois point insensé~: Pourquoi mourrais-tu avant ton temps\FTNT{Ec. 9:16.}~?
\VS{18}Il est bon que tu retiennes ceci, et que tu ne retires point ta main de cela~; car celui qui craint Dieu sort de tout.
\VS{19}La sagesse donne plus de force au sage que dix gouverneurs qui sont dans une ville.
\VS{20}Certainement il n'y a point d'homme juste sur la terre qui agisse toujours bien, et qui ne pèche point\FTNT{Ps. 14:3~; Pr. 20:9~; 2 Ch. 6:36~; Ja. 3:2~; Ro. 3:12~; 1 Jn. 1:8.}.
\VS{21}Ne mets point aussi ton cœur à toutes les paroles qu'on dira, afin que tu n'entendes pas ton serviteur médire de toi. 
\VS{22}Car ton cœur aussi a reconnu plusieurs fois que tu as pareillement mal parlé des autres. 
\VS{23}J'ai essayé tout ceci avec sagesse, et j'ai dit~: J'acquerrai de la sagesse~; mais elle s'est éloignée de moi. 
\VS{24}Ce qui est loin et ce qui est profond, qui le trouvera~?
\VS{25}Je me suis appliqué dans mon cœur à connaître, à sonder, et à chercher la sagesse et la raison de tout~; et à connaître la méchanceté de la folie, de la bêtise et des sottises. 
\VS{26}Et j'ai trouvé plus amère que la mort, la femme dont le cœur est un piège et un filet, et dont les mains sont des liens~; celui qui est agréable à Dieu lui échappera~; mais le pécheur sera pris par elle\FTNT{Pr. 5:3-4~; Pr. 6:26~; Pr. 7:13-27~; Pr. 9:13-16~; Pr. 22:14.}.
\VS{27}Voici, dit l'Ecclésiaste, ce que j'ai trouvé en cherchant la raison de toutes choses, l'une après l'autre~;
\VS{28}C'est que jusqu'à présent, mon âme a cherché, mais que je n'ai point trouvé, c'est que j'ai bien trouvé un homme entre mille~; mais pas une femme entre elles toutes. 
\VS{29}Seulement voici ce que j'ai trouvé~; c'est que Dieu a créé l'homme juste~; mais ils ont cherché beaucoup d'inventions.
\Chap{8}
\TextTitle{L'obéissance aux autorités}
\VerseOne{}Qui est tel que le sage~? Et qui sait ce que veulent dire les choses~? La sagesse de l'homme fait briller son visage, et son regard farouche en est changé\FTNT{Ec. 7:3~; Pr. 15:13.}.
\VS{2}Je te le dis~: Prends garde aux ordres du roi, et cela à cause du serment fait à Dieu.
\VS{3}Ne te précipite point de te retirer de devant sa face~; et ne persévère point dans une chose mauvaise~; car il fera tout ce qu'il lui plaira. 
\VS{4}En quelque lieu qu'est la parole du roi, là est la puissance~; et qui lui dira~: Que fais-tu~? 
\VS{5}Celui qui garde le commandement ne sentira aucun mal~; et le cœur du sage discerne le temps et ce qui est juste. 
\VS{6}Car dans toute affaire il y a un temps et un jugement, autrement mal sur mal tombe sur l'homme. 
\VS{7}Car il ne sait pas ce qui arrivera~; et même qui est-ce qui lui déclarera quand cela arrivera~? 
\VS{8}L'homme n'est point maître de son souffle\FTNT{Le souffle ou l'esprit de l'homme quitte son corps le jour de sa mort (Ps. 39:5~; Ja. 2:26).} pour pouvoir le retenir, il n'a aucune puissance sur le jour de la mort~; il n'y a point de délivrance dans ce combat, et la méchanceté ne délivrera point son maître.
\VS{9}J'ai vu tout cela, et j'ai appliqué mon cœur à toute œuvre qui se fait sous le soleil. Il y a un temps où l'homme domine sur l'autre pour son malheur.
\VS{10}Alors j'ai vu les méchants ensevelis et s'en aller~; et ceux qui avaient agi avec droiture s'en aller loin du lieu saint et être oubliés dans la ville. Cela aussi est une vanité\FTNT{Ec. 2:16~; Ec. 9:5.}.
\VS{11}Parce que la sentence contre les mauvaises œuvres ne s'exécute point promptement, à cause de cela le cœur des fils de l'homme se remplit en eux de l'envie de faire le mal\FTNT{Ec. 12:1.}.
\VS{12}Car bien que le pécheur fasse le mal cent fois, et qu'il y persévère longtemps, je sais aussi qu'il y aura du bonheur pour ceux qui craignent Dieu et qui révèrent sa face\FTNT{Job 22:21~; Pr. 1:33~; Es. 3:10.}.
\VS{13}Mais le bonheur n'est pas pour le méchant, et il ne prolongera point ses jours plus que l'ombre, parce qu'il n'a pas de crainte devant Dieu.
\VS{14}Il y a une vanité qui arrive sur la terre~: C'est qu'il y a des justes auxquels il arrive selon l'œuvre des méchants~; et il y a aussi des méchants auxquels il arrive selon l'œuvre des justes. Je dis que cela aussi est une vanité.
\VS{15}C'est pourquoi j'ai loué la joie, parce qu'il n'y a rien sous le soleil de meilleur à l'homme, que de manger et de boire et de se réjouir~; c'est aussi ce qui lui restera de son travail durant les jours de sa vie, que Dieu lui donne sous le soleil. 
\VS{16}Après avoir appliqué mon cœur à connaître la sagesse, et à regarder les occupations qu'il y a sur la terre, (car l'homme ne donne, ni jour ni nuit, de repos à ses yeux), 
\VS{17}après avoir vu, dis-je, toute l'œuvre de Dieu, j'ai vu que l'homme ne peut pas trouver l'œuvre qui se fait sous le soleil~; il a beau se fatiguer à chercher, il n'est pas capable de trouver~; et même si le sage dit la connaître, il ne peut la trouver.
\Chap{9}
\TextTitle{L'impuissance de la sagesse face à la mort}
\VerseOne{}Certainement j'ai appliqué mon cœur à tout cela~; et pour l'éclaircir à savoir que les justes, les sages et leurs actions sont dans la main de Dieu~; mais les hommes ne connaissent ni l'amour ni la haine de tout ce qui est devant eux. 
\VS{2}Tout arrive également à tous~; un même sort arrive au juste et au méchant~; au bon, au pur et au souillé~; à celui qui sacrifie et à celui qui ne sacrifie point~; le pécheur est comme l'homme de bien~; celui qui jure, comme celui qui craint de jurer. 
\VS{3}C'est un mal parmi tout ce qui se fait sous le soleil, c'est qu'il y a pour tous un même sort~; aussi le cœur des fils de l'homme est-il plein de méchanceté, et la folie est dans leur cœur pendant leur vie~; après cela, ils vont chez les morts. Qui est celui qui voudrait leur être associé\FTNT{Ec. 2:16~; Job 9:22}~?
\VS{4}Il y a de l'espérance pour tous ceux qui sont encore vivants~; et même un chien vivant vaut mieux qu'un lion mort.
\VS{5}Certainement les vivants savent qu'ils mourront, mais les morts ne savent rien, et ne gagnent plus rien~; car leur mémoire est mise en oubli. 
\VS{6}Aussi leur amour, leur haine, et leur envie ont déjà péri, et ils n'auront plus aucune part à tout ce qui se fait sous le soleil.
\VS{7}Va donc, mange ton pain avec joie, et bois gaiement ton vin~; car depuis longtemps Dieu prend plaisir à tes œuvres.
\VS{8}Que tes vêtements soient blancs en tout temps, et que le parfum ne manque point sur ta tête. 
\VS{9}Vis joyeusement tous les jours de ta vie de vanité avec la femme que tu aimes, qui t'a été donnée sous le soleil, tous les jours de ta vanité~; car c'est là ta part dans la vie, au milieu de ton travail que tu fais sous le soleil.
\VS{10}Tout ce que ta main trouve à faire, fais-le selon ton pouvoir~; car dans le scheol, où tu vas, il n'y a ni œuvre, ni pensée, ni connaissance, ni sagesse.
\VS{11}Je me suis tourné ailleurs, et j'ai vu sous le soleil que la course n'est point aux légers, ni la guerre aux héros, ni le pain aux sages, ni la richesse à ceux qui sont intelligents, ni la grâce aux savants~; mais que le temps et les circonstances décident de ce qui arrive à tous.
\VS{12}Car l'homme ne connaît pas son heure, comme les poissons qui sont pris au filet de malheur et les oiseaux qui sont pris au piège~; comme eux, les fils de l'homme sont enlacés au temps du malheur, lorsqu'il tombe subitement sur eux.
\VS{13}J'ai aussi vu cette sagesse sous le soleil, et elle m'a semblé grande.
\VS{14}Il y avait une petite ville, avec peu d'hommes dans son sein~; un roi puissant marcha contre elle, l'investit et bâtit de grands forts contre elle.
\VS{15}Mais il s'y trouvait un homme pauvre et sage qui délivra la ville par sa sagesse. Et personne ne s'est souvenu de cet homme pauvre.
\VS{16}Alors j'ai dit~: La sagesse vaut mieux que la force. Cependant, la sagesse du pauvre est méprisée, et ses paroles ne sont point écoutées.
\VS{17}Les paroles des sages doivent être écoutées plus paisiblement que le cri de celui qui domine parmi les insensés. 
\VS{18}Mieux vaut la sagesse que tous les instruments de guerre~; et un seul homme pécheur détruit beaucoup de bien.
\Chap{10}
\TextTitle{La sagesse vaut mieux que la folie}
\VerseOne{} Les mouches mortes font puer et fermenter les parfums du parfumeur~; et un peu de folie produit le même effet à l'égard de celui qui est estimé pour sa sagesse, et pour sa gloire.
\VS{2}Le cœur du sage est à sa droite, et le cœur de l'insensé est à sa gauche.
\VS{3}Et même quand l'insensé se met en chemin, le sens lui manque~; et il dit de chacun~: Il est insensé. 
\VS{4}Si l'esprit de celui qui domine s'élève contre toi, ne sors point de ta condition~; car la douceur fait pardonner de grandes fautes. 
\VS{5}Il y a un mal que j'ai vu sous le soleil, comme une erreur qui procède du prince~:
\VS{6}C'est que la folie est mise aux plus hauts lieux, et que les riches sont assis dans un lieu bas. 
\VS{7}J'ai vu des serviteurs sur des chevaux, et des princes marchant sur terre comme des serviteurs.
\VS{8}Celui qui creuse la fosse y tombera, et celui qui coupe la haie, le serpent le mordra\FTNT{Ps. 7:15~; Pr. 26:27~; Pr. 28:10.}.
\VS{9}Celui qui remue des pierres hors de leur place, en sera blessé, et celui qui fend du bois se met en danger.
\VS{10}Si le fer est émoussé, et qu'on n'en ait point aiguisé le tranchant, il devra redoubler de force~; mais la sagesse a l'avantage du succès.
\VS{11}Si le serpent mord sans faire du bruit, le médisant ne vaut pas mieux. 
\VS{12}Les paroles de la bouche du sage ne sont que grâce, mais les lèvres de l'insensé le réduisent à néant\FTNT{Pr. 10:21.}.
\VS{13}Le commencement des paroles de sa bouche est folie, et la fin de son discours est une méchante folie.
\VS{14}Or l'insensé multiplie les paroles. L'homme ne sait point ce qui arrivera, et qui lui déclarera ce qui sera après lui~?
\VS{15}Le travail de l'insensé le fatigue, parce qu'il ne sait pas aller à la ville.
\VS{16}Malheur à toi, pays dont le roi est un enfant, et dont les princes mangent dès le matin\FTNT{Es. 3:4.}~!
\VS{17}Que tu es béni, ô pays~! Si ton roi est de race illustre, et si tes gouverneurs mangent au temps convenable, pour leur réfection et non pour se livrer à la débauche~! 
\VS{18}A cause des mains paresseuses, la charpente s'affaisse~; et à cause des mains lâches, la maison a des gouttières.
\VS{19}On fait des pains pour se réjouir et le vin réjouit les vivants et l'argent répond à tout.
\VS{20}Ne maudis point le roi, même dans ta pensée, et ne maudis pas le riche dans la chambre où tu couches~; car l'oiseau du ciel emporterait ta voix, le Baal ailé\FTNT{Baal ailé~: Le terme sémitique «~baal~» (en hébreu ba'al) signifie à l'origine «~possesseur~», «~maître~» ou «~seigneur~». Le Baal ailé était une créature ailée. Utilisé au pluriel, l'expression «~baalim de flèches~» désignait des archers. Les écritures nous parlent de Baal-Zebub (seigneur des mouches), un démon adoré à Ekron, l'une des villes des Philistins (2 R. 1:1-16). Baal-Zebud à donné «~Béelzébul~» dans les Evangiles (Mt. 10:25~; Mt. 12:24~; Mt. 12:27~; Lu. 11:15-19). Ce passage nous enseigne clairement que les démons épient les enfants de Dieu et vont ensuite faire leurs rapports à Satan afin de mieux les attaquer. Ils agissent comme des espions. Ces esprits sont comme des mouches et essayent de s'infiltrer partout.} rapporterait tes paroles.
\Chap{11}
\TextTitle{L'homme travaille en tâtonnant}
\VerseOne{}Jette ton pain à la face des eaux, car avec le temps tu le retrouveras.
\VS{2}Donnes-en une part à sept et même à huit, car tu ne sais point quel mal viendra sur la terre.
\VS{3}Quand les nuages sont pleins, ils répandent la pluie sur la terre~; et quand un arbre tombe, au sud ou au nord, il reste à la place où il est tombé.
\VS{4}Celui qui prend garde au vent, ne sèmera point~; et celui qui regarde les nuées, ne moissonnera point. 
\VS{5}Comme tu ne sais point quel est le chemin du vent, ni comment se forment les os dans le ventre de celle qui est enceinte, ainsi tu ne connais pas l'œuvre de Dieu qui fait tout\FTNT{Ceux qui sont nés d'en-haut sont insaisissables comme le vent (Jn. 3:8).}.
\VS{6} Sème ta semence dès le matin, et ne laisse pas reposer tes mains le soir~; car tu ne sais point lequel sera le meilleur, ceci ou cela~; et si tous deux seront pareillement bons.
\VS{7}Il est vrai que la lumière est douce, et qu'il est agréable aux yeux de voir le soleil.
\VS{8}Mais si un homme vit de nombreuses années, qu'il se réjouisse, et qu'il se souvienne des jours de ténèbres qui seront en grand nombre, tout ce qui lui arrivera est vanité.
\Chap{12}
\TextTitle{Message à la jeunesse}
\VerseOne{}Jeune homme, réjouis-toi dans ton jeune âge, et que ton cœur te rende gai aux jours de ta jeunesse, et marche comme ton cœur te mène, et selon le regard de tes yeux~; mais sache que pour toutes ces choses Dieu t'amènera en jugement. 
\VS{2}Ôte le chagrin de ton cœur, et éloigne de toi le mal~; car le jeune âge et l'adolescence ne sont que vanité. 
\VS{3}Mais souviens-toi de ton Créateur pendant les jours de ta jeunesse, avant que les jours mauvais arrivent et que viennent les années où tu diras~: Je n'y prends point de plaisir~;
\VS{4}avant que le soleil et la lumière, la lune et les étoiles s'obscurcissent, et que les nuages reviennent après la pluie.
\VS{5}Lorsque les gardes de la maison tremblent\FTNT{«~Ceux qui gardent la maison~» représentent les mains.}, et que les hommes forts\FTNT{«~Les hommes forts~» sont les jambes.} se courbent, et que celles qui moulent\FTNT{Les dents sont celles qui moulent.} cessent de travailler parce qu'elles sont diminuées, et quand ceux qui regardent par les fenêtres\FTNT{Les yeux sont ceux qui regardent par les fenêtres.} sont obscurcis.
\VS{6}Et quand les deux battants de la porte\FTNT{Les oreilles sont les deux battants de la porte.} se ferment sur la rue quand s'abaisse le bruit de la meule, quand on se lève au chant de l'oiseau, et que toutes les chanteuses s'affaiblissent.
\VS{7}Quand aussi on craint ce qui est élevé, et qu'on tremble en chemin, quand l'amandier fleurit, et quand les cigales deviennent pesantes, et que l'appétit s'en ira, car l'homme s'en va vers sa maison éternelle\FTNT{La maison éternelle c'est la Nouvelle Jérusalem pour les chrétiens (Ap. 21.) et pour les païens le lac de feu (Ap. 20:11-15).}, et ceux qui pleurent font le tour des rues.
\VS{8}Avant que la corde d'argent\FTNT{Cette corde est comme le cordon ombilical, elle lie l'âme au corps. Lors de la mort, la corde d'argent est coupée.} se détache, que le vase d'or\FTNT{Le corps humain est comme un vase ou une tente qui renferme son esprit. Comme l'argile dans la main du potier, ainsi est l'homme dans celle de Dieu. Avec cette argile, il décide souverainement de fabriquer de la même masse un vase d'honneur et un autre pour un usage vil (Jé. 18:4-6~; Ro. 9:21~; 2 Ti. 2:20-21).} se brise, que la cruche se rompe sur la source, que la roue s'écrase sur la citerne~;
\VS{9}avant que la poussière retourne dans la terre, comme elle y avait été, et que l'esprit retourne à Dieu qui l'a donné.
\TextTitle{Conclusion}
\VS{10}Vanité des vanités, dit l'Ecclésiaste, tout est vanité.
\VS{11}Plus l'Ecclésiaste a été sage, plus il a enseigné la science au peuple~; il a fait entendre, il a recherché et mis en ordre plusieurs graves sentences. 
\VS{12}L'Ecclésiaste a cherché pour trouver des discours agréables~; mais ce qui en a été écrit ici, est la droiture même~; ce sont des paroles de vérité. 
\VS{13}Les paroles des sages sont comme des aiguillons, et les maîtres qui en ont fait des recueils, sont comme des clous plantés, et ces choses ont été données par un maître.
\VS{14}Mon fils, garde-toi de ce qui est au-delà de ceci~; car il n'y a point de fin à faire plusieurs livres, et tant d'étude n'est que travail qu'on se donne. 
\VS{15}Voici la conclusion de tout le discours qui a été entendu~: Crains Dieu, et garde ses commandements~; car c'est là le tout de l'homme.
\VS{16}Parce que Dieu amènera toute œuvre en jugement, au sujet de tout ce qui est caché, soit bien, soit mal.
\PPE{}
\end{multicols}

%\clearpage\ShortTitle{Esther}\BookTitle{Esther}\BFont
\noindent\hrulefill
{\footnotesize
\textit{
\bigskip
{\centering{}
\\(Ecter)
\\Signifie : Etoile (persan) ; Myrthe (hébreu)
\\Thème : Délivrance des juifs de l’extermination
\\Auteur : Inconnu
\\Date de rédaction : 5ème siècle av. J.C.\\}
}
%\bigskip
\textit{
\\Dernier livre à caractère historique du Tanahk, l’histoire d’Esther se déroula à Suse, capitale du royaume de Perse. En ce temps, le peuple d’Israël était dispersé et le roi Assuérus régnait sur un large territoire allant de l’Inde à l'Ethiopie.
%\bigskip
\\Ce livre raconte la vie d’Esther, son ascension au trône royal où elle succéda à la reine Vasthi et la manière dont elle fut utilisée pour éviter le génocide du peuple juif.  
%\bigskip
\\Bien que ne comportant pas le nom de Dieu ni d’allusion à une œuvre spirituelle, hormis le jeûne, ce récit met en évidence le secours divin.\bigskip
}
}
\par\nobreak\noindent\hrulefill
\begin{multicols}{2}
\TextTitle{[Un festin de sept jour au palais de Suse]}
\Chap{1}
\VerseOne{}Or il arriva qu’au temps d’Assuérus, de cet Assuérus qui régnait depuis les Indes jusqu'en Ethiopie, sur cent vingt-sept provinces ;
\VS{2}[Il arriva, dis-je], en ce temps-là, que le roi Assuérus était assis sur le trône royal à Suse, dans la capitale.
\VS{3}La troisième année de son règne, il fit un festin à tous les principaux princes de ses pays ; à ses serviteurs, à l’armée des Perses et de Mèdes, aux nobles et aux chefs des provinces qui furent réunis devant lui,
\VS{4}pour leur montrer la gloire de la richesse de son royaume et la splendeur de sa grandeur, durant plusieurs jours, pendant cent quatre-vingts jours.
\VS{5}Lorsque ces jours furent achevés, le roi fit pour tout le peuple qui se trouvait à Suse, la capitale, depuis le plus grand jusqu'au plus petit, un festin pendant sept jours, dans la cour du jardin du palais royal.
\VS{6}Des étoffes blanches, vertes et violettes, étaient attachées par des cordons de byssus et de pourpre à des anneaux d'argent et à des colonnes de marbre. Les lits étaient d'or et d'argent sur un pavé de porphyre, de marbre, de nacre, et de pierres noires.
\VS{7}On servait à boire dans des vases d'or, de différentes espèces, et il y avait du vin royal en abondance, selon la libéralité du roi.
\VS{8}On ne forçait personne à boire, car le roi avait ordonné à tous les chefs de sa maison de se conformer à la volonté de chacun.
\VS{9}La reine Vasthi fit aussi un festin aux femmes dans la maison royale du roi Assuérus.
\TextTitle{[Destitution de la reine Vasthi]}
\VS{10}Or le septième jour, le cœur du roi était réjoui par le vin, il ordonna à Mehuman, Biztha, Harbona, Bigtha, Abagtha, Zéthar, et Carcas, les sept eunuques qui servaient devant le roi Assuérus,
\VS{11}d’amener en sa présence la reine Vasthi, portant la couronne royale, afin de montrer sa beauté aux peuples et aux princes, car elle était belle de figure.
\VS{12}Les eunuques transmirent l’ordre du roi à la reine Vasthi, mais elle refusa de venir. Et le roi fut très irrité, et il s’enflamma de colère.
\VS{13}Alors le roi dit aux sages qui avaient la connaissance des temps. Car le roi traitait ainsi les affaires en présence de tous ceux qui connaissaient les lois et le droit.
\VS{14}Il avait auprès de lui, Carschena, Schéthar, Admatha, Tarsis, Mérès, Marsena, Memucan, sept princes de Perse et de Médie, qui voyaient la face du roi et qui occupaient le premier rang dans le royaume.
\VS{15}Que faut-il faire dit-il, selon les lois, à la reine Vasthi, pour n'avoir pas observé l’ordre que le roi Assuérus lui a ordonné par les eunuques ?
\VS{16}Alors Memucan répondit en présence du roi et des princes : La reine Vasthi n'a pas seulement mal agi contre le roi, mais aussi contre tous les princes et tous les peuples qui sont dans toutes les provinces du roi Assuérus.
\VS{17}Car l'action de la reine parviendra à la connaissance de toutes les femmes, et les portera à mépriser leurs maris ; elles diront : Le roi Assuérus avait ordonné qu'on fasse venir en sa présence la reine, et elle n'y est pas allée.
\VS{18}Dès ce jour, les princesses de Perse et de Médie qui auront appris l’action de la reine répondront de même à tous les princes du roi ; ce sera une marque de mépris et un sujet de colère.
\VS{19}Si le roi le trouve bon, qu'un édit royal soit publié de sa part, et qu'il soit écrit parmi les lois de Perse et de Médie, avec défense de la transgresser, que Vasthi ne vienne plus devant le roi Assuérus et le roi donnera sa royauté à une compagne, qui sera meilleure qu'elle.
\VS{20}L’édit du roi sera présenté et connu dans tout son royaume, quelque grand qu'il soit, toutes les femmes honoreront leurs maris\FTNT{Respect ou soumission de la femme à l’égard de son mari : Ep. 5 : 22 ; Col. 3 : 18 ; Ti. 2 : 5 ; 1 Pi. 3 : 1-5.}, depuis le plus grand jusqu'au plus petit.
\VS{21}Cette parole plut au roi et aux princes, et le roi fit selon la parole de Memucan.
\VS{22}Il envoya des lettres à toutes les provinces du royaume, à chaque province selon son écriture et à chaque peuple selon sa langue ; elles portaient que tout homme devait être le maître de sa maison\FTNT{L’homme, chef de la femme et maître de la maison : 1 Co. 11 : 3 ; Ep. 5 : 23.}, et qu’il parlerait la langue de son peuple.
\TextTitle{[Le roi choisit une autre reine]}
\Chap{2}
\VerseOne{}Après ces choses, quand la colère du roi Assuérus fut calmée, il se souvint de Vasthi, de ce qu'elle avait fait, et de ce qui avait été décrété à son sujet.
\VS{2}Les serviteurs qui servaient le roi dirent : Qu'on cherche pour le roi des jeunes filles, vierges, et belles de figure.
\VS{3}Que le roi désigne des commissaires dans toutes les provinces de son royaume chargés de rassembler toutes les jeunes filles, vierges et belles de figure, dans Suse, la capitale, dans la maison des femmes sous la charge d'Hégué, eunuque du roi et gardien des femmes, qu'on leur donne les parfums nécessaires pour leur toilette ;
\VS{4}et la jeune fille qui plaira au roi régnera à la place de Vasthi. Ce discours plût au roi, et il fit ainsi.
\VS{5}Or, il y avait à Suse, la capitale, un juif nommé Mardochée, fils de Jaïr, fils de Schimeï, fils de Kis, Benjamite,
\VS{6}qui avait été emmené de Jérusalem\FTNT{La captivité babylonienne : Voir 2 R. 24.}, parmi les captifs déportés avec Jeconia, roi de Juda, par Nebucadnetsar, roi de Babylone.
\VS{7}Il élevait Hadassa, qui est Esther, fille de son oncle ; car elle n'avait ni père ni mère. La jeune fille était belle de taille et très belle de figure. Après la mort de son père et de sa mère, Mardochée l'avait prise pour fille.
\VS{8}Lorsqu’on eut publié l’ordre du roi et son édit, un grand nombre de jeunes filles furent rassemblées à Suse, la capitale, sous la charge d'Hégaï. Esther fut aussi amenée dans la maison du roi, sous la charge d'Hégaï, gardien des femmes.
\VS{9}La jeune fille lui plut, et trouva grâce à ses yeux, il s’empressa de lui fournir les parfums nécessaires pour sa toilette, et pour sa subsistance, lui donna sept jeunes filles choisies, et établies dans la maison du roi, il lui fit changer d'appartement, et la logea, elle et ses servantes, dans le meilleur des appartements de la maison des femmes.
\VS{10}Esther ne fit connaître ni son peuple ni sa parenté, car Mardochée lui avait ordonné de ne rien raconter.
\VS{11}Tous les jours Mardochée allait et venait devant la cour de la maison des femmes, pour savoir comment se portait Esther, et comment on s’occupait d'elle.
\VS{12}Chaque jeune fille allait à son tour vers le roi Assuérus, après s’être conformée au décret concernant les femmes pendant douze mois\FTNT{Esther se soumit à une toilette particulière avant de rencontrer le roi. Le mot « toilette » vient de l’hébreu « tam-rook », qui signifie « grattement ». La racine de ce mot signifie « nettoyer », « purifier », « polir » (voir Lé. 6 : 28 ; Jé. 46 : 4). Ce grattage symbolise le dépouillement du vieil homme et le renoncement aux œuvres de la chair (Ep. 4 : 22).
Douze mois étaient nécessaires pour préparer Esther aux noces : Six mois avec de l’huile de myrrhe et six mois avec des aromates et des parfums. La myrrhe était l’une des composantes de l’onction sainte dont on s’est servi pour oindre notamment la tente d’assignation, l’arche du témoignage ainsi qu’Aaron et ses fils (Ex. 30 : 23-30). Cet aspect de la toilette d’Esther nous parle de la sanctification sans laquelle nul ne peut voir le Seigneur (Hé. 12 : 14). La myrrhe est par ailleurs citée à sept reprises dans le livre du Cantique des cantiques, véritable hymne de l’amour parfait qui lie Christ à son Eglise. Le parfum quant à lui symbolise les prières que nous devons faire en tout temps afin de maintenir notre communion avec Jésus, notre époux (Ap. 5 : 8 ; Ap. 8 : 4 ; Ep. 6 : 18 ; 1 Th. 5 : 17).
Ainsi, à l’instar d’Esther qui se préparait à rencontrer le roi, l’Eglise se prépare depuis deux mille ans pour les noces de l’agneau (Ap. 19 : 7-9).}. C'est ainsi que s'accomplissaient les jours de leurs préparatifs, six mois avec de l'huile de myrrhe, et six autres mois avec des aromates et des parfums en usage parmi les femmes.
\VS{13}C'est ainsi que la jeune fille entrait vers le roi ; et, quand elle passait de la maison des femmes à la maison du roi, on lui laissait prendre ce qu’elle voulait.
\VS{14}Elle y entrait le soir, et le matin elle retournait dans la seconde maison des femmes sous la charge de Schaaschgaz, eunuque du roi et gardien des concubines. Elle ne retournait plus vers le roi, à moins que le roi n’en ait le désir et qu'elle soit appelée par son nom.
\TextTitle{[Esther, reine de Suse]}
\VS{15}Quand son tour d’aller vers le roi fut arrivé, Esther, fille d'Abichaïl, oncle de Mardochée qui l’avait prise pour sa fille, ne demanda rien sinon ce qui fut ordonné par Hégaï, eunuque du roi et gardien des femmes. Esther trouva grâce aux yeux de tous ceux qui la voyaient.
\VS{16}Ainsi Esther fut amenée auprès du roi Assuérus, dans sa maison royale, le dixième mois, qui est le mois de Tébeth, la septième année de son règne.
\VS{17}Le roi aima Esther plus que toutes les autres femmes, elle obtint sa grâce et sa bienveillance plus que toutes les vierges. Il mit la couronne royale sur sa tête, et l'établit reine à la place de Vasthi.
\VS{18}Le roi fit alors un grand festin à tous les princes de ses pays, et à ses serviteurs, un festin en l’honneur d'Esther ; il donna du repos aux provinces, et fit des présents selon la puissance du roi.
\VS{19}Or pendant qu'on assemblait les vierges pour la seconde fois, Mardochée s’assit à la porte du roi.
\VS{20}Esther n’avait fait connaître ni sa parenté ni son peuple, car Mardochée le lui avait défendu. Elle faisait tout ce que lui disait Mardochée, comme à l’époque où elle était élevée par lui.
\TextTitle{[Mardochée sauve la vie du roi]}
\VS{21}En ces jours-là, Mardochée s’assit à la porte du roi, Bigthan et Théresch, deux eunuques du roi, gardes du seuil, s’irritèrent et cherchèrent à mettre la main sur le roi Assuérus.
\VS{22}Mardochée ayant eu connaissance de l’affaire, informa la reine Esther, qui le redit au roi de la part de Mardochée.
\VS{23}On vérifia l’affaire et on trouva que cela était exact, les deux eunuques furent pendus à un bois, et cela fut écrit dans le livre des chroniques en présence du roi.
\TextTitle{[Conspiration de Haman contre les Juifs]}
\Chap{3}
\VerseOne{}Après ces choses, le roi Assuérus fit de grands honneurs à Haman, fils d'Hammedatha, l’Agaguite ; il l'éleva en dignité et plaça son siège au-dessus de tous les princes qui étaient auprès de lui.
\VS{2}Tous les serviteurs du roi qui étaient à la porte du roi s'inclinaient et se prosternaient devant Haman, car le roi l’avait ainsi ordonné. Mais Mardochée ne s'inclinait pas et ne se prosternait pas devant lui.
\VS{3}Les serviteurs du roi, qui étaient à la porte du roi, disaient à Mardochée : Pourquoi transgresses-tu l’ordre du roi ?
\VS{4}Comme ils le lui répétaient chaque jour et qu'il ne les écoutait pas, ils le rapportèrent à Haman, pour voir si Mardochée tiendrait ferme dans sa résolution ; car il leur avait déclaré qu'il était juif.
\VS{5}Haman vit que Mardochée ne s'inclinait pas et ne se prosternait pas devant lui et il fut rempli de colère.
\VS{6}Mais il dédaigna de porter la main sur Mardochée seul, car on lui avait rapporté de quel peuple était Mardochée. Haman chercha à exterminer tous les juifs, le peuple de Mardochée qui se trouvait dans tout le royaume d'Assuérus.
\VS{7}Au premier mois, qui est le mois de Nissan, la douzième année du roi Assuérus, on jeta le pur, c'est-à-dire le sort, devant Haman, pour chaque jour et pour chaque mois, jusqu’au douzième mois, qui est le mois d'Adar.
\VS{8}Haman dit au roi Assuérus : Il y a un peuple dispersé dans toutes les provinces de ton royaume, qui se tient à part parmi les peuples. Leurs lois sont différentes de celles de tous les autres peuples, ils n’observent pas les lois du roi. Il n'est pas dans l’intérêt du roi de le laisser en repos.
\VS{9}S'il plaît au roi, qu'on écrive l’ordre de les faire périr, et je pèserai dix mille talents d'argent entre les mains de ceux qui s’occupent des affaires, pour les porter dans le trésor du roi.
\VS{10}Le roi ôta son anneau de sa main, et le donna à Haman fils de Hammedatha, l’Agaguite, l’adversaire des Juifs.
\VS{11}Outre cela, le roi dit à Haman : Cet argent t'est donné avec ce peuple ; fais-en ce que tu voudras.
\VS{12}Le treizième jour du premier mois, les secrétaires du roi furent appelés, et on écrivit selon l’ordre d'Haman, aux satrapes du roi, aux gouverneurs de chaque province et aux princes de chaque peuple, à chaque province selon son écriture et à chaque peuple selon sa langue. Ce fut au nom du roi Assuérus que l’on écrivit, et on scella avec l'anneau du roi.
\VS{13}Les lettres furent envoyées par des coureurs dans toutes les provinces du roi, afin qu'on extermine, qu’on tue et qu’on fasse périr tous les juifs, jeunes et vieux, petits enfants et femmes, en un seul jour, le treizième du douzième mois, qui est le mois d'Adar, et pour que leurs biens soient livrés au pillage.
\VS{14}Ces lettres qui furent écrites portaient une copie de l’édit, qui devait être publié dans chaque province, et invitaient publiquement tous les peuples, à se tenir prêts pour ce jour-là.
\VS{15}Ainsi les coureurs partirent en toute hâte d’après l’ordre du roi. L'édit fut aussi publié dans Suse, la capitale. Or le roi et Haman étaient assis pour boire, pendant que la ville de Suse était dans la confusion.
\TextTitle{[Esther avertie du complot d'Haman]}
\Chap{4}
\VerseOne{}Mardochée, ayant appris ce qui se passait, déchira ses vêtements et se couvrit d'un sac et de la cendre. Puis il alla au milieu de la ville en poussant avec force des cris amers,
\VS{2}et se rendit jusqu'à la porte du roi, or il était interdit d'entrer dans le palais du roi revêtu d'un sac.
\VS{3}Dans chaque province, partout où arrivait l’ordre du roi et son édit, il y eut une grande désolation parmi les juifs ; ils jeûnaient, pleuraient, gémissaient, et beaucoup se couchaient sur le sac et la cendre.
\VS{4}Les servantes d'Esther et ses eunuques vinrent lui raconter ces choses, et la reine fut très effrayée. Elle envoya des vêtements à Mardochée pour le couvrir et lui faire ôter son sac, mais il ne les prit pas.
\VS{5}Alors Esther appela Hathac, l'un des eunuques que le roi avait établi pour la servir, et elle le chargea de demander à Mardochée ce qui s’était passé et pourquoi il agissait ainsi.
\VS{6}Hathac sortit donc vers Mardochée sur la place de la ville, devant la porte du roi.
\VS{7}Mardochée lui raconta tout ce qui lui était arrivé, et la somme d'argent qu'Haman avait promis de payer comptant au trésor du roi, pour la destruction des juifs.
\VS{8}Il lui donna aussi une copie de l'édit publié dans Suse en vue de leur extermination, afin qu’il le montre à Esther et lui fasse tout connaître ; et il ordonna qu’Esther se rende chez le roi pour implorer sa miséricorde, et faire une requête en faveur de son peuple.
\VS{9}Hathac vint rapporter à Esther les paroles de Mardochée.
\TextTitle{[Mardochée incite Esther à risquer sa vie pour ses frères]}
\VS{10}Esther chargea Hathac de dire à Mardochée :
\VS{11}Tous les serviteurs du roi et le peuple des provinces du roi savent qu'il existe une loi prescrivant la peine de mort contre quiconque, homme ou femme, entre chez le roi, dans la cour intérieure sans avoir été appelé ; à moins que le roi ne lui tende le sceptre d'or, celui-là a la vie sauve. Or il y a déjà trente jours que je n'ai pas été appelée pour entrer chez le roi.
\VS{12}On rapporta les paroles d'Esther à Mardochée.
\VS{13}Mardochée fit cette réponse à Esther : Ne t’imagine pas que tu échapperas seule d'entre tous les juifs parce que tu es dans la maison du roi.
\VS{14}Mais si tu te tais et gardes le silence en ce temps-ci, les juifs seront secourus et délivrés par un autre moyen, mais toi et la maison de ton père vous périrez. Et qui sait si tu n'es pas arrivée à la royauté pour un temps comme celui-ci ?
\TextTitle{[Esther demande un jeûne]}
\VS{15}Esther fit cette réponse à Mardochée :
\VS{16}Va, rassemble tous les juifs qui se trouvent à Suse, et jeûnez pour moi, sans manger ni boire pendant trois jours, ni la nuit ni le jour. Moi aussi et mes servantes nous jeûnerons de même, puis j'entrerai chez le roi, malgré la loi ; et si je dois périr, je périrai.
\VS{17}Mardochée s'en alla, et fit comme Esther lui avait ordonné.
\TextTitle{[Esther se présente devant le roi]}
\Chap{5}
\VerseOne{}Le troisième jour, Esther mit des vêtements royaux et se présenta dans la cour intérieure de la maison du roi, devant la maison du roi. Le roi était assis sur le trône dans la maison royale, en face de l’entrée de la maison.
\VS{2}Dès que le roi vit la reine Esther debout dans la cour, elle trouva grâce à ses yeux ; le roi tendit à Esther le sceptre d'or qui était dans sa main. Esther s'approcha, et toucha le bout du sceptre.
\VS{3}Le roi lui dit : Qu'as-tu, reine Esther, et que demandes-tu ? Quand ce serait la moitié du royaume, elle te serait donnée.
\VS{4}Esther répondit : Si le roi le trouve bon, que le roi vienne aujourd'hui avec Haman au festin que je lui ai préparé.
\VS{5}Alors le roi dit : Qu'on fasse venir en toute hâte, Haman, pour accomplir la parole d'Esther. Le roi vint donc avec Haman au festin qu'Esther avait préparé.
\VS{6}Le roi dit à Esther, pendant qu’on buvait le vin : Quelle est ta demande ? Elle te sera accordée. Quelle est ta requête ? Quand ce serait la moitié du royaume, tu l’obtiendras.
\VS{7}Esther répondit et dit : Voici ce que je demande et ce que je désire.
\VS{8}Si j'ai trouvé grâce aux yeux du roi, et si le roi trouve bon d'accorder ma requête, que le roi et Haman viennent au festin que je leur préparerai, et je donnerai demain une réponse au roi selon sa parole.
\VS{9}Haman sortit ce jour-là, joyeux et le cœur content. Mais aussitôt qu'il vit, à la porte du roi, Mardochée, qui ne se levait ni ne tremblait devant lui, il fut rempli de colère contre Mardochée.
\VS{10}Il sut toutefois se contenir, et il alla dans sa maison. Puis il envoya chercher ses amis et Zéresch, sa femme.
\VS{11}Haman leur parla de la magnificence de ses richesses, du nombre de ses fils, et tout ce qu’avait fait le roi pour le rendre puissant, et comment il l'avait élevé au-dessus des princes et des serviteurs du roi.
\VS{12}Puis Haman ajouta : Même la reine Esther n'a fait venir que moi et le roi au festin qu'elle a fait, et je suis encore invité demain chez elle avec le roi.
\VS{13}Mais tout cela n’est d’aucun intérêt, aussi longtemps que je verrai Mardochée, le juif, assis à la porte du roi.
\VS{14}Zéresch sa femme, et tous ses amis lui répondirent : Qu'on prépare un bois haut de cinquante coudées, et demain matin dis au roi qu'on y pende Mardochée ; et tu iras joyeux au festin avec le roi. Cette parole plut à Haman, et il fit préparer le bois.
\TextTitle{[Le roi Assuérus se souvient de Mardochée]}
\Chap{6}
\VerseOne{}Cette nuit-là, le roi ne put dormir, il fit apporter le livre des annales, les chroniques. On les lut devant le roi,
\VS{2}et l’on trouva écrit ce que Mardochée avait rapporté au sujet de la conspiration de Bigthan et de Théresch, les deux eunuques du roi, gardes du seuil, qui avaient cherché à mettre la main sur le roi Assuérus.
\VS{3}Le roi dit : Quel honneur et quelle distinction a-t-on accordé à Mardochée pour cela ? Il n’a rien reçu répondirent les serviteurs du roi.
\VS{4}Le roi dit : Qui est dans la cour ? Haman était venu dans la cour extérieure de la maison du roi, pour demander au roi de pendre Mardochée au bois qu'il avait préparé.
\VS{5}Les serviteurs du roi répondirent : C’est Haman qui se tient dans la cour. Et le roi dit : Qu'il entre.
\VS{6}Haman entra, et le roi lui dit : Que faudrait-il faire à un homme que le roi désire honorer ? Haman se dit en lui-même : A qui le roi voudrait-il faire plus honneur qu'à moi ?
\VS{7}Haman répondit au roi : Pour un homme que le roi désire honorer,
\VS{8}qu’on lui apporte le vêtement royal, dont le roi se revêt, et qu'on lui amène le cheval que le roi monte, et qu'on lui mette la couronne royale sur la tête.
\VS{9}Et qu'ensuite on donne ce vêtement et ce cheval à quelqu'un des principaux et des plus grands chefs qui sont auprès du roi, et qu'on revête l'homme que le roi prend plaisir d'honorer, et qu'on le fasse aller à cheval par les rues de la ville ; et qu'on crie devant lui : C'est ainsi qu'on doit faire à l'homme que le roi prend plaisir d'honorer.
\VS{10}Alors le roi dit à Haman : Prends tout de suite le vêtement, et le cheval, comme tu l'as dit, et fais ainsi à Mardochée, le juif qui est assis à la porte du roi ; ne néglige rien de tout ce que tu as déclaré.
\VS{11}Et Haman prit le vêtement et le cheval, il revêtit Mardochée, il le promena à cheval à travers les rues de la ville, et il criait devant lui : C'est ainsi que l’on fait à l'homme que le roi désire honorer.
\VS{12}Mardochée retourna à la porte du roi, et Haman se retira en hâte dans sa maison, pleurant et ayant la tête voilée.
\VS{13}Haman raconta à Zéresch, sa femme, et à tous ses amis, tout ce qui lui était arrivé. Ses sages, et Zéresch, sa femme, lui répondirent : Si Mardochée devant lequel tu as commencé à tomber, est de la race des juifs, tu n'auras pas le dessus sur lui, mais tu tomberas certainement devant lui.
\VS{14}Comme ils parlaient encore avec lui, les eunuques du roi vinrent, et se hâtèrent d'amener Haman au festin qu'Esther avait préparé.
\TextTitle{[Esther plaide sa cause et celle de son peuple]}
\Chap{7}
\VerseOne{}Le roi et Haman allèrent au festin chez la reine Esther.
\VS{2}Le roi dit encore à Esther, ce second jour, pendant qu’on buvait le vin : Quelle est ta demande, reine Esther ? Elle te sera donnée. Que désires-tu ? Quand ce serait la moitié du royaume, cela te sera accordé.
\VS{3}Alors la reine Esther répondit, et dit : Si j'ai trouvé grâce à tes yeux, ô roi ! et si le roi le trouve bon, que ma vie me soit donnée à ma demande, et que mon peuple me soit donné à ma prière.
\VS{4}Car nous avons été vendus, mon peuple et moi, pour être détruits, tués, exterminés. Si nous avions été vendus pour être esclaves et serviteurs, j’aurais gardé le silence, bien que l'oppresseur ne saurait compenser le dommage fait au roi.
\VS{5}Le roi Assuérus parla et dit à la reine Esther : Qui est-il et où est l’homme dont le cœur est consacré à faire cela ?
\VS{6}Esther répondit : L'oppresseur, l'ennemi, c’est Haman, ce méchant ! Alors Haman fut terrifié en présence du roi et de la reine.
\TextTitle{[Haman pendu au gibet qu'il avait dressé]}
\VS{7}Le roi, dans sa colère, se leva et quitta le festin, il entra dans le jardin du palais. Haman resta pour demander grâce pour sa vie à la reine Esther, car il voyait bien que sa perte était résolue par le roi.
\VS{8}Puis le roi revint du jardin du palais dans la salle du festin, il vit Haman qui s’était précipité sur le lit où était Esther, et il dit : Serait-ce encore pour faire violence sous mes yeux à la reine dans cette maison ? Dès que la parole fut sortie de la bouche du roi, on voila le visage d'Haman.
\VS{9}Et Harbona, l'un des eunuques, dit en présence du roi : Voici, le bois préparé par Haman pour Mardochée, qui a parlé pour le bien du roi, est dressé dans la maison d'Haman, à une hauteur de cinquante coudées. Le roi dit : Qu’on y pende Haman !
\VS{10}On pendit Haman au bois qu'il avait préparé pour Mardochée. Et la colère du roi fut apaisée.
\TextTitle{[Un décret royal fait échouer le complot d'Haman]}
\Chap{8}
\VerseOne{}Ce même jour, le roi Assuérus donna à la reine Esther la maison d'Haman, l'oppresseur des juifs ; et Mardochée fut introduit devant le roi, car Esther avait déclaré quel était son lien de parenté avec elle.
\VS{2}Le roi ôta son anneau, qu'il avait repris à Haman, et le donna à Mardochée ; Esther établit Mardochée sur la maison d'Haman.
\VS{3}Esther parla encore en présence du roi. Elle se jeta à ses pieds, elle pleura, elle l’implora d’empêcher les effets de la méchanceté d'Haman, l’Agaguite, et la réussite de ses projets contre les juifs.
\VS{4}Le roi tendit le sceptre d'or à Esther qui se releva et resta debout devant le roi.
\VS{5}Elle dit : Si le roi le trouve bon, et si j'ai trouvé grâce devant lui, si mes paroles semblent convenables au roi et si je suis agréable à ses yeux, qu'on écrive pour révoquer les lettres conçues par Haman, fils d'Hammedatha, l’Agaguite, qu'il écrivit afin de détruire les juifs qui sont dans toutes les provinces du roi.
\VS{6}Car comment pourrais-je voir le mal qui atteindrait mon peuple, et comment pourrais-je voir la destruction de ma race ?
\VS{7}Le roi Assuérus dit à la reine Esther et au juif Mardochée : Voici, j'ai donné la maison d'Haman à Esther, et il a été pendu au bois pour avoir étendu sa main contre les juifs.
\VS{8}Ecrivez donc, au nom du roi, en faveur des juifs comme il vous plaira, et scellez l'écrit de l'anneau du roi ; car un édit écrit au nom du roi et scellé de l'anneau du roi ne peut être révoqué.
\VS{9}En ce temps, le vingt-troisième jour du troisième mois, qui est le mois de Sivan, les secrétaires du roi furent appelés, et on écrivit, comme Mardochée l’ordonna, aux juifs, aux satrapes, aux gouverneurs, et aux princes des cent vingt-sept provinces, de l’Inde jusqu'en Ethiopie, à chaque province selon son écriture, à chaque peuple selon sa langue, et aux juifs selon leur écriture et selon leur langue.
\VS{10}On écrivit les lettres au nom du roi Assuérus, et on les scella de l'anneau du roi. On les envoya par des coureurs, ayant pour montures des chevaux et des mulets nés de juments.
\VS{11}Par ces lettres, le roi accordait aux juifs, qui étaient dans chaque ville la permission de se rassembler et de défendre leur vie, de détruire, de tuer, et d’exterminer toute force armée du peuple et de quelque province que ce soit, qui prendraient les armes pour les attaquer, ainsi que leurs petits enfants et leurs femmes, et de piller leurs biens ;
\VS{12}et cela en un seul jour, dans toutes les provinces du roi Assuérus, le treizième jour du douzième mois, qui est le mois d'Adar.
\VS{13}Ces lettres écrites portaient une copie de l’édit qui devait être publié dans chaque province, et informaient tous les peuples que les juifs seraient prêts en ce jour à se venger de leurs ennemis.
\VS{14}Les coureurs, montés sur des chevaux et des mulets, partirent aussitôt et en toute hâte, d’après l’ordre du roi. L'édit fut aussi publié dans Suse, la capitale.
\TextTitle{[Mardochée honoré]}
\VS{15}Mardochée sortit de chez le roi, en vêtement royal violet et blanc, avec une grande couronne d'or, et une robe de byssus et de pourpre. La ville de Suse poussait des cris, et elle fut dans la joie.
\VS{16}Il eut pour les juifs du bonheur et de la joie, des réjouissances et des honneurs.
\VS{17}Dans chaque province et dans chaque ville, partout où arrivaient l’ordre du roi et son décret, il y eut pour les juifs de la joie, des réjouissances, des festins, et des fêtes. Et beaucoup de gens d'entre les peuples du pays se faisaient juifs, parce que la crainte des juifs les avait saisis.
\TextTitle{[Les juifs triomphent de leurs ennemis]}
\Chap{9}
\VerseOne{}Le douzième mois, qui est le mois d'Adar, le treizième jour du mois, où l’ordre du roi et son décret devaient être exécutés, au jour où les ennemis des juifs espéraient dominer, ce fut le contraire qui arriva, les juifs dominèrent sur leurs ennemis.
\VS{2}Les juifs se rassemblèrent dans leurs villes, dans toutes les provinces du roi Assuérus, pour mettre la main sur ceux qui cherchaient leur perte ; et personne ne put leur résister, car la crainte qu'on avait d'eux avait saisi tous les peuples.
\VS{3}Et tous les princes des provinces, les satrapes, les gouverneurs, et ceux qui s’occupaient des affaires du roi, soutenaient les juifs, à cause de la terreur que leur inspirait Mardochée.
\VS{4}Car Mardochée était puissant dans la maison du roi, et sa renommée se répandait dans toutes les provinces, parce qu’il devenait de plus en plus puissant.
\VS{5}Les juifs frappèrent tous leurs ennemis à coups d'épée, ils les tuèrent et les détruisirent ; ils traitèrent selon leurs désirs ceux qui les haïssaient.
\VS{6}Dans Suse, la capitale, les juifs tuèrent et firent périr cinq cents hommes.
\VS{7}Ils tuèrent aussi Parschandatha, Dalphon, Aspatha,
\VS{8}Poratha, Adalia, Aridatha,
\VS{9}Parmaschtha, Arizaï, Aridaï, et Vajezatha,
\VS{10}les dix fils d'Haman, fils d'Hammedatha, l'oppresseur des juifs. Mais ils ne mirent pas leurs mains au pillage.
\VS{11}Ce jour-là, on rapporta au roi le nombre de ceux qui avaient été tués dans Suse, la capitale.
\VS{12}Le roi dit à la reine Esther : Dans Suse, la capitale, les juifs ont tué et détruit cinq cents hommes, et les dix fils d'Haman, qu'auront-ils fait dans le reste des provinces du roi ? Quelle est ta demande ? Et elle te sera accordée. Que désires-tu encore ? Tu l’obtiendras.
\VS{13}Esther répondit : Si le roi le trouve bon qu'il soit permis aux juifs, qui sont à Suse, d’agir encore demain selon le décret d’aujourd'hui, et que l'on pende au bois les dix fils d'Haman.
\VS{14}Et le roi ordonna de faire ainsi. L'édit fut publié dans Suse. On pendit les dix fils d'Haman ;
\VS{15}et les juifs qui étaient dans Suse se rassemblèrent encore le quatorzième jour du mois d'Adar et tuèrent dans Suse trois cents hommes. Mais ils ne mirent pas la main au pillage.
\VS{16}Les autres juifs qui étaient dans les provinces du roi se rassemblèrent, et défendirent leur vie ; ils eurent du repos et furent délivrés de leurs ennemis, et ils tuèrent soixante-quinze mille hommes de ceux qui les haïssaient. Mais ils ne mirent pas la main au pillage.
\VS{17}Ces choses arrivèrent le treizième jour du mois d'Adar, et le quatorzième du même mois ils se reposèrent, et ils en firent un jour de festin et de joie.
\VS{18}Les juifs qui étaient dans Suse, s'assemblèrent le treizième et le quatorzième jour du même mois, et ils se reposèrent le quinzième jour, et ils en firent un jour de festin et de joie.
\VS{19}C'est pourquoi les juifs des campagnes qui habitent dans des villes sans murailles, font le quatorzième jour du mois d'Adar, un jour de réjouissance, de festin et de fête, où l’on s’envoie des portions les uns aux autres.
\TextTitle{[Esther confirme l'instauration la fête des Purim]}
\VS{20}Mardochée écrivit ces choses, et il envoya les lettres à tous les juifs qui étaient dans toutes les provinces du roi Assuérus, auprès et au loin.
\VS{21}Il leur prescrivait de célébrer chaque année le quatorzième jour et le quinzième jour du mois d'Adar.
\VS{22}Comme les jours où les juifs avaient obtenu du repos en se délivrant de leurs ennemis, de célébrer le mois où leur angoisse fut changée en joie, et leur deuil en jour heureux, et de faire de ces jours des jours de festin et de joie, où l’on s’envoie des portions les uns aux autres, et des dons aux pauvres.
\VS{23}Les juifs s’engagèrent à faire ce qu’ils avaient déjà commencé et ce que Mardochée leur prescrivit.
\VS{24}Car Haman, fils d'Hammedatha, l’Agaguite, l'oppresseur de tous les juifs, avait projeté de détruire les juifs, et il avait jeté le pur, c'est-à-dire le sort, afin de les détruire et de les tuer ;
\VS{25}mais Esther s’étant présentée devant le roi, le roi ordonna par écrit que le méchant projet qu'Haman avait imaginé contre les juifs, retombe sur sa tête, et qu'on le pende au bois, lui et ses fils.
\VS{26}C'est pourquoi on appelle ces jours-là purim, du nom de pur\FTNT{Pur ou purim : Ce terme signifie sort (Est. 3 : 7). La fête de purim a été instituée pour célébrer leur délivrance de l’extermination planifiée par Haman, à la suite de l’intervention héroïque d’Esther. Les juifs l’observent désormais chaque année le 14 du mois d’Adar (février ou mars) depuis le temps d’Esther jusqu’à ce jour.}. D’après tout le contenu de cette lettre, et selon ce qu’ils avaient eux-mêmes vu et ce qui leur était arrivé,
\VS{27}Les juifs établirent et adoptèrent pour eux, pour leur postérité, et pour tous ceux qui s’attacheraient à eux, l’engagement de ne pas manquer de célébrer chaque année ces deux jours, selon le mode prescrit et au temps fixé.
\VS{28}Ces jours devaient être rappelés et observés de génération en génération, dans chaque famille, dans chaque province et dans chaque ville ; et ces jours de Purim ne devaient jamais être abolis au milieu des juifs, ni le souvenir s’en effacer parmi leurs descendants.
\VS{29}La reine Esther, fille d'Abichaïl, écrivit aussi avec le juif Mardochée, de manière pressante pour la seconde fois, pour confirmer la lettre sur les Purim.
\VS{30}On envoya des lettres à tous les juifs, dans les cent vingt-sept provinces du royaume d'Assuérus. Elles contenaient des paroles de paix et de vérité,
\VS{31}pour établir ces jours de Purim au temps fixé, comme Mardochée le juif et la reine Esther les avaient établis pour eux, et comme ils les avaient établis pour eux-mêmes et pour leur postérité, à l’occasion de leur jeûne et de leurs cris.
\VS{32}Ainsi l'édit d'Esther confirma l’institution des Purim, et cela fut écrit dans le livre.
\TextTitle{[Mardochée établi dans la cours du roi]}
\Chap{10}
\VerseOne{}Le roi Assuérus imposa un tribut au pays, et aux îles de la mer.
\VS{2}Tous les faits concernant ses exploits, et les détails sur la grandeur à laquelle le roi éleva Mardochée, ne sont-ils pas écrits dans le livre des chroniques des rois de Médie et de Perse ?
\VS{3}Car Mardochée le juif était le premier après le roi Assuérus ; grand parmi les juifs et agréable à la multitude de ses frères, il chercha le bien-être de son peuple, et parla pour la paix de toute sa race.
\PPE{}
\end{multicols}

%\clearpage\ShortTitle{Daniel}\BookTitle{Daniel}\BFont
\noindent\hrulefill
{\footnotesize
\textit{
\bigskip
{\centering{}
\\Auteur : Daniel
\\(Heb. : Daniye'l)
\\Signification : Dieu est mon juge
\\Thème : Ascension et chute des royaumes
\\Date de rédaction : 6\up{ème} siècle av. J.-C.\\}
}
%\bigskip
\textit{
\\Issu d'une famille princière de Juda, Daniel fut déporté de Jérusalem à Babylone pendant sa jeunesse, sous le règne de Nebucadnestar. Lui et trois de ses amis – eux aussi de noble lignée - furent choisis pour être instruits selon la sagesse babylonienne en vue de servir le roi. Fervent dans sa foi en Yahweh, Daniel - imité ensuite par ses compagnons – résolut de ne point se souiller et obtint ainsi la faveur de son Dieu. Son intégrité et sa crainte de Dieu lui valurent de miraculeuses victoires, de nombreuses distinctions et une grande sagesse. Daniel avait reçu du discernement pour expliquer songes et visions et délivra plusieurs prophéties dont certaines se sont déjà accomplies, d'autres se réaliseront à la fin du temps des nations, au moment du retour de Christ.
%\bigskip
\\Dieu témoigna de la justice de Daniel au prophète Ezéchiel dont il fut contemporain.\bigskip
}
}
\par\nobreak\noindent\hrulefill
\begin{multicols}{2}
\Chap{1}
\TextTitle{Juda livré à la captivité babylonienne}
\VerseOne{}La troisième année du règne de Jojakim, roi de Juda, Nebucadnetsar, roi de Babylone, vint contre Jérusalem et l'assiégea.
\VS{2}Le Seigneur livra entre ses mains Jojakim\FTNT{En 597 av. J-C., la ville de Jérusalem tomba entre les mains des Babyloniens qui déportèrent le roi Jojakim et nommèrent comme roi à sa place son oncle Sédécias. Une petite partie de la population fut déportée à cette occasion. Cette première déportation ne concernait que l'élite administrative et sacerdotale : prêtres, scribes, hauts fonctionnaires, membres de la famille royale et artisans métallurgistes. Pour de nombreux historiens, il s'agissait moins d'une déportation que d'une constitution d'un groupe d'otages. Le roi, quelques membres de sa famille, et de diverses familles de notables, furent tenus en résidence surveillée à la cour babylonienne pour s'assurer que le royaume de Juda resterait pacifié.}, roi de Juda et une partie des vases de la maison de Dieu. Nebucadnetsar emporta les vases au pays de Schinear\FTNT{Schinear : « le pays des deux fleuves ». C'est l'ancien nom du territoire qui est devenu Babylonie ou Chaldée. C'est le pays de Nimrod (Ge. 10:6-12). C'est à Schinear qu'on tenta de construire la tour de Babel et de mettre en place le premier gouvernement mondial.}, dans la maison de son dieu, il les mit dans la maison du trésor de son dieu.
\VS{3}Le roi dit à Aschpenaz, capitaine de ses eunuques, d'amener quelques-uns des enfants d'Israël de race royale\FTNT{Daniel était de la race royale (2 R. 20:16-19 ; Es. 39:1-8).} et des principaux seigneurs,
\VS{4}quelques jeunes enfants en qui il n'y avait aucun défaut corporel, beaux de figure, instruits en toute sagesse, connaissant les sciences, pleins d'intelligence, et capables de se tenir dans le palais du roi ; et à qui l'on enseignerait les lettres et la langue des Chaldéens.
\VS{5}Le roi leur assigna pour provision chaque jour une portion de la viande royale et du vin dont il buvait, afin qu'on les nourrisse ainsi pendant trois ans au bout desquels ils se tiendraient devant le roi.
\VS{6}Il y avait parmi eux, d'entre les fils de Juda, Daniel, Hanania, Mischaël et Azaria.
\VS{7}Mais le capitaine des eunuques leur donna d'autres noms, il donna à Daniel le nom de Beltschatsar, à Hanania celui de Schadrac, à Mischaël celui de Méschac et à Azaria celui d'Abed-Nego.
\TextTitle{La fermeté de Daniel à Babylone}
\VS{8}Daniel résolut dans son cœur de ne pas se souiller par la portion de la viande du roi et par le vin dont le roi buvait; c'est pourquoi il supplia le chef des eunuques afin qu'il ne l'oblige pas à se souiller.
\VS{9}Et Dieu fit trouver à Daniel faveur et grâce auprès du chef des eunuques. 
\VS{10}Et le chef des eunuques dit à Daniel : Je crains le roi, mon seigneur, qui a fixé ce que vous devez manger et boire ; car pourquoi verrait-il vos visages plus défaits que ceux des jeunes gens de votre âge ? Vous exposeriez ma tête auprès du roi.
\VS{11}Mais Daniel dit à Meltsar, l'intendant à qui le chef des eunuques avait remis la surveillance de Daniel, Hanania, Mischaël et Azaria :
\VS{12}Eprouve, je te prie, tes serviteurs pendant dix jours, et qu'on nous donne des légumes à manger et de l'eau à boire.
\VS{13}Après cela, tu regarderas nos visages et ceux des jeunes enfants qui mangent la portion de la viande royale; puis tu feras à tes serviteurs selon ce que tu auras vu.
\VS{14}Et il les écouta dans cette affaire et les éprouva pendant dix jours.
\VS{15}Au bout des dix jours, leurs visages parurent en meilleur état et plus d'embonpoint que tous les jeunes gens qui mangeaient la portion de la viande royale.
\VS{16} Ainsi Meltsar prenait la portion de leur viande et le vin qu'ils devaient boire, et leur donnait des légumes.
\VS{17}Et Dieu donna à ces quatre jeunes gens de la science et de l'intelligence dans toutes les lettres, et de la sagesse ; et Daniel comprenait toutes les visions et tous les songes.
\VS{18}Et à la fin des jours fixés par le roi pour qu'on les lui amène, le chef des eunuques les présenta à Nebucadnetsar.
\VS{19}Le roi s'entretint avec eux ; mais entre eux tous il ne s'en trouva pas de tels que Daniel, Hanania, Mischaël et Azaria ; et ils entrèrent au service du roi.
\VS{20}Sur toutes les questions savantes qui réclamaient de la sagesse et de l'intelligence, et sur lesquelles le roi les interrogeait, il les trouva dix fois supérieurs à tous les magiciens et les astrologues qui étaient dans tout son royaume.
\VS{21}Et Daniel fut là jusqu'à la première année du roi Cyrus.
\Chap{2}
\TextTitle{Les sages de Babylone tous condamnés à mort}
\VerseOne{}La deuxième année du règne de Nebucadnetsar, Nebucadnetsar eut des songes, et son esprit fut agité, et son sommeil fut interrompu.
\VS{2}Alors le roi fit appeler les magiciens, les astrologues, les enchanteurs et les Chaldéens, pour qu'ils lui expliquent ses songes ; ils vinrent donc et se présentèrent devant le roi.
\VS{3}Le roi leur dit : J'ai eu un songe, mon esprit est agité, tâchant de connaître ce songe\FTNT{Ce songe annonce la future mise en place d'un gouvernement mondial. Voir commentaire en Da. 7:3.}.
\VS{4}Et les Chaldéens répondirent au roi en langue araméenne\FTNT{L'araméen : De Daniel 2:5 à 7:28, le livre est écrit en araméen.} : Ô roi, vis éternellement ! Dis le songe à tes serviteurs et nous en donnerons l'interprétation.
\VS{5}Mais le roi répondit et dit aux Chaldéens : La chose m'a échappé ; si vous ne me faites connaître le songe et son interprétation, vous serez mis en pièces et vos maisons seront réduites en un tas d'immondices.
\VS{6}Mais si vous me faites connaître le songe et son interprétation, vous recevrez de moi, des dons, des présents et un grand honneur. Quoi qu'il en soit faites-moi connaître le songe et son interprétation.
\VS{7}Ils répondirent pour la seconde fois et dirent : Que le roi dise le songe à ses serviteurs et nous en donnerons l'interprétation.
\VS{8}Le roi répondit et dit : Je m'aperçois en vérité, que vous ne cherchez qu'à gagner du temps, parce que vous voyez que la chose m'a échappée.
\VS{9}Mais si vous ne me faites pas connaître le songe, il y a une même sentence contre vous tous ; car vous vous êtes préparés à dire devant moi des mensonges et des faussetés en attendant que le temps soit changé. Quoi qu'il en soit, dites-moi le songe et je saurai que vous pouvez m'en donner l'interprétation.
\VS{10}Les Chaldéens répondirent au roi et dirent : Il n'y a aucun homme sur la terre qui puisse exécuter ce que le roi demande. Et aussi il n'y a ni roi, ni seigneur, ni gouverneur qui ait jamais demandé une telle chose à quelque magicien, astrologue ou Chaldéen que ce soit.
\VS{11}Car la chose que le roi demande est extrêmement difficile et il n'y a personne qui puisse le faire connaître au roi, excepté les dieux dont la demeure n'est pas parmi les hommes.
\VS{12}A cause de cela, le roi s'irrita et se mit dans une très grande colère, et ordonna qu'on fasse périr tous les sages de Babylone.
\VS{13}La sentence fut donc publiée; on mettait à mort les sages et l'on cherchait Daniel et ses compagnons pour les faire périr.
\TextTitle{Daniel implore la miséricorde de Dieu}
\VS{14}Alors Daniel détourna l'exécution du conseil et l'arrêt donné à Arjoc, chef des gardes du roi, qui était sorti pour tuer les sages de Babylone.
\VS{15}Et il demanda et dit à Arjoc, commandant du roi : Pourquoi la sentence du roi est-elle si sévère ? Arjoc exposa la chose à Daniel.
\VS{16}Et Daniel entra et pria le roi de lui accorder du temps pour donner l'interprétation au roi.
\VS{17}Alors Daniel alla dans sa maison et informa de cette affaire Hanania, Mischaël et Azaria, ses compagnons,
\VS{18}pour implorer la miséricorde du Dieu des cieux sur ce secret, afin qu'on ne mette pas à mort Daniel et ses compagnons avec le reste des sages de Babylone. 
\TextTitle{Le songe de la grande statue révélé à Daniel}
\VS{19}Et le secret fut révélé à Daniel dans une vision pendant la nuit. Et Daniel bénit le Dieu des cieux.
\VS{20}Daniel prit donc la parole et dit : Béni soit le nom de Dieu, d'éternité en éternité ! A lui appartiennent la sagesse et la force\FTNT{Job. 12:13 ; Ap. 5:12 ; Ap. 7:12.}.
\VS{21}C'est lui qui change les temps et les saisons, qui ôte et qui établit les rois, qui donne la sagesse aux sages et la connaissance à ceux qui ont de l'intelligence.
\VS{22}C'est lui qui révèle les choses profondes et cachées, il connaît les choses qui sont dans les ténèbres et la lumière demeure avec lui\FTNT{De. 29:29 ; Es. 48:6 ; Jé. 33:3 ; Lu. 12:2-3.}.
\VS{23}Ô Dieu de nos pères ! Je te glorifie et te loue de ce que tu m'as donné de la sagesse et de la force, et de ce que tu m'as maintenant fait connaître ce que nous t'avons demandé, en nous ayant fait connaître le secret du roi.
\VS{24}Après cela, Daniel alla auprès d'Arjoc, à qui le roi avait ordonné de faire périr les sages de Babylone. Il alla et lui parla ainsi : Ne fais pas périr les sages de Babylone, mais fais-moi entrer devant le roi et je donnerai au roi l'interprétation qu'il souhaite.
\VS{25}Alors Arjoc conduisit promptement Daniel devant le roi et lui parla ainsi : J'ai trouvé parmi les captifs de Juda un homme qui donnera au roi l'interprétation de son songe.
\VS{26}Le roi prit la parole et dit à Daniel, qu'on nommait Beltschatsar : Es-tu capable de me faire connaître le songe que j'ai eu et son interprétation ?
\VS{27}Daniel répondit en présence du roi et dit : Ce que le roi demande est un secret que les sages, les astrologues, les magiciens et les devins ne sont pas capables de révéler au roi.
\VS{28}Mais il y a dans les cieux un Dieu qui révèle les secrets et qui a fait connaître au roi Nebucadnetsar ce qui doit arriver dans les derniers jours\FTNT{« Les derniers jours » voir commentaires dans Ge. 49:1-2.}. Voici ton songe et les visions de ta tête que tu as eues sur ta couche.
\VS{29}Sur ta couche, ô roi, il t'est monté des pensées touchant ce qui arriverait après ce temps-ci ; et celui qui révèle les secrets t'a fait connaître ce qui doit arriver.
\VS{30}Si ce secret m'a été révélé, ce n'est point qu'il y ait en moi une sagesse supérieure à celle de tous les vivants, mais c'est afin de donner au roi l'interprétation de son songe et afin que tu connaisses les pensées de ton cœur.
\VS{31}Ô roi, tu regardais et tu voyais une grande statue\FTNT{Voir annexe « La statue de Nebucadnetsar ».} ; cette grande statue, dont la splendeur était extraordinaire, était debout devant toi et son apparence était terrible.
\VS{32}La tête de cette statue était d'un or très fin, sa poitrine et ses bras étaient d'argent ; son ventre et ses cuisses étaient d'airain ;
\VS{33}ses jambes étaient de fer et ses pieds étaient en partie de fer et en partie de terre.
\VS{34}Tu regardais cela, jusqu'à ce qu'une pierre se détacha sans main, frappa les pieds de fer et d'argile de la statue et les brisa.
\VS{35}Alors le fer, l'argile, l'airain, l'argent et l'or furent brisés ensemble et devinrent comme la paille de l'aire en été que le vent transporte çà et là ; et nulle trace n'en fut retrouvée. Mais la pierre qui avait frappé la statue devint une grande montagne et remplit toute la terre.
\TextTitle{Premier empire universel : Babylone\FTNT{Cp. Da. 7:4.}}
\VS{36}C'est là le songe. Nous en donnerons maintenant l'interprétation devant le roi.
\VS{37}Ô roi, tu es le roi des rois, parce que le Dieu des cieux t'a donné le royaume, la puissance, la force et la gloire.
\VS{38}Il a remis entre tes mains, en quelque lieu qu'ils habitent, les enfants des hommes, les bêtes des champs et les oiseaux du ciel, et il t'a fait dominer sur eux tous : C'est toi qui es la tête d'or\FTNT{Jé. 27:6-7.}.
\TextTitle{Deuxième et troisième empires : les Mèdes et les Perses\FTNTT{cp. Da. 7:5 ; 8:20} et la Grèce\FTNTT{cp. Da. 7:6 ; 8:21}}
\VS{39}Mais après toi, il s'élèvera un autre royaume, moindre que le tien ; et ensuite un troisième royaume qui sera d'airain et qui dominera sur toute la terre.
\TextTitle{Quatrième empire : Rome\FTNTT{Cp. Da. 7:7 ; 9:26.}}
\VS{40}Puis il y aura un quatrième royaume, fort comme du fer ; de même que le fer brise et rompt tout ainsi il brisera et rompra tout, comme le fer qui met tout en pièces.
\VS{41}Et quant à ce que tu as vu, que les pieds et les orteils étaient en partie d'argile de potier et en partie de fer, c'est que ce royaume sera divisé, mais il y aura en lui de la force du fer, parce que tu as vu le fer mêlé avec l'argile de potier.
\VS{42}Et comme les doigts des pieds étaient en partie de fer et en partie d'argile, ce royaume sera en partie fort et en partie fragile.
\VS{43} Quant à ce que tu as vu, le fer mêlé avec l'argile de potier, c'est qu'ils se mêleront par des alliances humaines\FTNT{Le mot « alliances » vient de l'araméen « zera » qui signifie « semence » et « descendant ».} ; mais ils ne seront point unis l'un à l'autre de même que le fer ne s'allie point avec l'argile.
\TextTitle{Le royaume du Messie}
\VS{44}Dans le temps de ces rois, le Dieu des cieux suscitera un Royaume qui ne sera jamais détruit, et ce Royaume ne passera point à un autre peuple ; il brisera et anéantira tous ces royaumes-là, et lui-même sera établi éternellement.
\VS{45}Selon que tu as vu que de la montagne une pierre a été coupée sans main et qu'elle a brisé le fer, l'airain, la terre, l'argent et l'or. Le grand Dieu a fait connaître au Roi ce qui arrivera ci-après ; or le songe est véritable et son interprétation est certaine.
\TextTitle{Yahweh, le Dieu qui revèle les secrets}
\VS{46}Alors le roi Nebucadnetsar tomba sur sa face et se prosterna devant Daniel et il ordonna qu'on lui offre des offrandes de bonne odeur et des parfums.
\VS{47}Le roi parla à Daniel et lui dit : Certainement, votre Dieu est le Dieu des dieux, et le Seigneur des rois, et il révèle les secrets, puisque tu as pu découvrir ce secret.
\VS{48}Alors le roi éleva Daniel en dignité et lui fit de nombreux et riches présents ; il l'établit gouverneur sur toute la province de Babylone et chef suprême de tous les sages de Babylone.
\VS{49}Daniel pria le roi de remettre l'intendance de la province de Babylone à Schadrac, Méschac et Abed-Négo. Et Daniel se tenait à la porte du roi.
\Chap{3}
\TextTitle{La statue d'or de Nebucadnetsar}
\VerseOne{}Le roi Nebucadnetsar fit une statue d'or\FTNT{Nebucadnetsar est un type de l'antéchrist qui s'oppose aux plans de Dieu. Contrairement à la statue composée de plusieurs métaux qu'il avait vue en songe, et où il est représenté par la tête en or (Da. 2:38), il s'est fait construire une statue entièrement en or, se déclarant ainsi symboliquement invincible et immortel. En agissant de la sorte, Nebucadnetsar se fait Dieu et exige d'être adoré (2 Th. 2:3-4). Cette statue annonçait prophétiquement la mise en place d'une religion mondiale, fruit d'un mélange entre la politique et la religion. Ces choses sont déjà bien installées, il ne manque plus que la révélation de l'impie. Nous vivons dans une époque où on oblige les chrétiens à adhérer à des organisations politiques, religieuses, sous contrôle de l'état, et cela dans le but de contrôler les individus et le message qu'ils entendent et diffusent. Ainsi, les dirigeants actuels doivent passer au préalable par des études théologiques, dont l'enseignement contredit de plus en plus la vérité biblique pour se conformer aux préceptes de ce monde. Une fois ordonnés, ils doivent affilier leurs églises à des fédérations qui sont sous contrôle de l'état. En échange des subventions, beaucoup accepteront de diluer l'évangile, privant ainsi les âmes de la vérité.}, dont la hauteur était de soixante coudées, et la largeur de six coudées. Il la dressa dans la vallée de Dura, dans la province de Babylone.
\VS{2}Puis le roi Nebucadnetsar envoya pour rassembler les satrapes, les intendants, les gouverneurs, les conseillers, les trésoriers, les jurisconsultes, les juges, et tous les magistrats des provinces, afin qu'ils se rendent à la dédicace de la statue que le roi Nebucadnetsar avait dressée.
\VS{3}Ainsi furent assemblés les satrapes, les intendants, les gouverneurs, les conseillers, les trésoriers, les jurisconsultes, les juges, et tous les magistrats des provinces, pour la dédicace de la statue que le roi Nebucadnetsar avait dressée. Ils s'assemblèrent devant la statue que le roi Nebucadnetsar avait dressée.
\VS{4}Alors un héraut cria à haute voix, en disant : On vous fait savoir, ô peuples, nations, et langues !
\VS{5}Au moment où vous entendrez le son du cor, du chalumeau, de la guitare, de la sambuque, du psaltérion, de la cornemuse, et de toutes sortes d'instruments de musique, vous vous jetterez à terre et vous adorerez la statue d'or que le roi Nebucadnetsar a dressée.
\VS{6}Quiconque ne se jettera pas à terre et n'adorera pas sera jeté à l'instant même au milieu de la fournaise de feu ardent.
\VS{7}C'est pourquoi, au moment où tous les peuples entendirent le son du cor, du chalumeau, de la guitare, de la sambuque, du psaltérion, et de toutes sortes d'instruments de musique, tous les peuples, les nations, et les hommes de toutes les langues, se prosternèrent et adorèrent la statue d'or que le roi avait dressée.
\TextTitle{Le refus de l'idolâtrie}
\VS{8}Alors à ce même moment, certains chaldéens s'approchèrent et accusèrent les Juifs.
\VS{9}Et ils parlèrent et dirent au roi Nebucadnetsar : Roi, vis éternellement !
\VS{10}Toi, ô roi, tu as donné un ordre d'après lequel tout homme qui entendrait le son du cor, du chalumeau, de la guitare, de la sambuque, du psaltérion, de la cornemuse, et de toutes sortes d'instruments de musique, devrait se prosterner et adorer la statue d'or,
\VS{11}et que quiconque ne se prosternerait pas et ne l'adorerait pas, serait jeté au milieu d'une fournaise ardente.
\VS{12}Or, il y a certains Juifs que tu as établis sur les affaires de la province de Babylone, Schadrac, Méschac, et Abed-Négo ; ces hommes-là, ô roi, ne tiennent aucun compte de toi ; ils ne servent pas tes dieux, et ils n'adorent pas la statue d'or que tu as dressée.
\VS{13}Alors le roi Nebucadnetsar, saisi de colère et de fureur, ordonna qu'on amène Schadrac, Méschac, et Abed-Négo. Et ces hommes furent amenés devant le roi.
\VS{14}Et le roi Nebucadnetsar prit la parole et leur dit : Est-il vrai, Schadrac, Méschac, et Abed-Négo, que vous ne servez pas mes dieux, et que vous ne vous prosternez pas devant la statue d'or que j'ai dressée ?
\VS{15}Maintenant si vous êtes prêts, au moment où vous entendrez le son du cor, du chalumeau, de la guitare, de la sambuque, du psaltérion, de la cornemuse, et de toutes sortes d'instruments de musique, vous vous prosternerez, et vous adorerez la statue que j'ai faite ; si vous ne l'adorez pas, vous serez jetés à l'instant au milieu de la fournaise de feu ardent. Et qui est le dieu qui vous délivrera de mes mains ?
\VS{16}Schadrac, Méschac et Abed-Négo répondirent et dirent au roi Nebucadnetsar : Nous n'avons pas besoin de te répondre sur ce sujet.
\VS{17}Voici, notre Dieu, que nous servons, peut nous délivrer de la fournaise de feu ardent, et il nous délivrera de ta main, ô roi !
\VS{18}Sinon, sache, ô roi, que nous ne servirons pas tes dieux, et que nous n'adorerons pas la statue d'or que tu as dressée.
\TextTitle{L'épreuve de la fournaise de feu ardent}
\VS{19}Alors Nebucadnetsar fut rempli de fureur, et il changea de visage en tournant ses regards contre Schadrac, Méschac, et Abed-Négo. Il prit la parole et ordonna de chauffer la fournaise sept fois plus qu'on avait coutume de la chauffer.
\VS{20}Puis il commanda aux hommes les plus forts et les plus vaillants qui étaient dans son armée de lier Schadrac, Méschac, et Abed-Négo, et de les jeter dans la fournaise de feu ardent.
\VS{21}Et en même temps ces hommes furent liés avec leurs caleçons, leurs chaussures, leurs tiares, et leurs vêtements, et furent jetés au milieu de la fournaise de feu ardent.
\VS{22}Et parce que l'ordre du roi était sévère, et que la fournaise était extraordinairement chauffée, la flamme tua les hommes qui y avaient jetés, Schadrac, Méschac, et Abed-Négo.
\VS{23}Et ces trois hommes, Schadrac, Méschac, et Abed-Négo, tombèrent tous liés au milieu de la fournaise ardente.
\TextTitle{La grandeur de Yahweh, le Dieu qui délivre}
\VS{24}Alors le roi Nebucadnetsar fut effrayé, et se leva précipitamment. Il prit la parole et il dit à ses conseillers : N'avons-nous pas jeté trois hommes liés au milieu du feu ? Ils répondirent et dirent au roi : Certainement, ô roi !
\VS{25}Il reprit et dit : Voici, je vois quatre hommes sans liens qui marchent au milieu du feu, et qui n'ont point de mal ; et la figure du quatrième est semblable à celle d'un fils de Dieu.
\VS{26}Alors Nebucadnetsar s'approcha vers la porte de la fournaise de feu ardent ; et prenant la parole, il dit : Schadrac, Méschac, et Abed-Négo, serviteurs du Dieu Très-Haut, sortez et venez ! Alors Schadrac, Méschac, et Abed-Négo sortirent du milieu du feu.
\VS{27}Puis les satrapes, les intendants, les gouverneurs, et les conseillers du roi s'assemblèrent pour contempler ces hommes-là, et le feu n'avait eu aucun pouvoir sur leurs corps, et aucun cheveu de leur tête n'était brûlé, et leurs caleçons n'étaient point endommagés, et l'odeur du feu n'avait pas passé sur eux.
\VS{28}Alors Nebucadnetsar prit la parole et dit : Béni soit le Dieu de Schadrac, Méschac, et Abed-Négo, lequel a envoyé son ange et délivré ses serviteurs qui ont eu confiance en lui, et qui ont violé l'ordre du roi et livré leur corps plutôt que de servir et d'adorer aucun autre dieu que leur Dieu\FTNT{Mt. 4:10 ; Ac. 4:19 ; Ac. 5:29.}.
\TextTitle{Schadrac, Méschac et Abed-Nego élevé par le roi}
\VS{29}Voici maintenant l'ordre que je donne : Tout homme, à quelque nation ou langue qu'il appartienne, qui parlera mal du Dieu de Schadrac, Méschac, et Abed-Négo, sera mis en pièces, et sa maison sera réduite en un tas d'immondices, parce qu'il n'y a aucun autre dieu qui puisse délivrer comme lui.
\VS{30}Alors le roi fit réussir Schadrac, Méschac, et Abed-Négo dans la province de Babylone.
\Chap{4}
\TextTitle{La suprématie de Yahweh déclarée aux nations}
\VerseOne{}Le roi Nebucadnetsar, à tous les peuples, aux nations, aux hommes de toutes langues qui habitent sur toute la terre : Que votre paix soit multipliée !
\VS{2}Il m'a semblé bon de vous déclarer les signes et les merveilles que le Dieu Très-Haut a opérés à mon égard.
\VS{3}Ô que ses signes sont grands, et ses merveilles pleines de force ! Son règne est un règne éternel, et sa domination subsiste de génération en génération\FTNT{Ps 102:12 ; La. 5:19 ; Lu. 1:33.}.
\TextTitle{La vision du grand arbre}
\VS{4}Moi, Nebucadnetsar, j'étais tranquille dans ma maison, et heureux dans mon palais.
\VS{5}J'ai eu un songe qui m'épouvanta ; et les pensées sur ma couche et les visions de ma tête me troublèrent.
\VS{6}J'ordonnai qu'on fasse venir devant moi tous les sages de Babylone, afin qu'ils me donnent l'interprétation du songe.
\VS{7}Alors vinrent les magiciens, les astrologues, les Chaldéens et les devins. Je leur dis le songe, mais ils ne purent m'en donner l'interprétation.
\VS{8}En dernier lieu, se présenta devant moi Daniel, nommé Beltschatsar, selon le nom de mon Dieu, et qui a en lui l'Esprit des dieux saints. Je lui dis le songe :
\VS{9}Beltschatsar, chef des magiciens, comme je sais que l'Esprit des dieux saints est en toi, et que nul secret ne t'est difficile, écoute les visions que j'ai eues en songe, et donne-moi son interprétation.
\VS{10}Voici les visions de ma tête, pendant que j'étais sur ma couche. Je regardais, et voici, il y avait un arbre au milieu de la terre d'une grande hauteur.
\VS{11}Cet arbre était devenu grand et fort, sa cime s'élevait jusqu'aux cieux, et on le voyait des extrémités de toute la terre.
\VS{12}Son feuillage était beau, et son fruit abondant, et il portait de la nourriture pour tous ; les bêtes des champs s'abritaient sous son ombre, les oiseaux du ciel habitaient dans ses branches, et tout être vivant tirait de lui sa nourriture.
\VS{13}Dans les visions de ma tête que j'avais sur ma couche, je regardais, et voici, un de ceux qui veillent et qui sont saints descendit des cieux.
\VS{14}Il cria à haute voix et parla ainsi : Abattez l'arbre, et coupez ses branches ! Secouez son feuillage, et dispersez son fruit ; que les bêtes s'enfuient de dessous, et les oiseaux du milieu de ses branches !
\VS{15}Mais laissez en terre le tronc où se trouvent ses racines, et liez-le avec des chaînes de fer et d'airain, qu'il soit parmi l'herbe des champs. Qu'il soit trempé de la rosée des cieux, et qu'il ait, comme les bêtes, l'herbe de la terre pour partage.
\VS{16}Que son cœur d'homme soit changé, et qu'un cœur de bête lui soit donné ; et que sept temps passent sur lui.
\VS{17}Cette sentence est le décret de ceux qui veillent, cette résolution est un ordre des saints, afin que les vivants sachent que le Très-Haut domine sur le royaume des hommes, qu'il le donne à qui il lui plaît, et qu'il y élève le plus vil des hommes\FTNT{Cette vérité est confirmée par l'apôtre Paul dans Ro. 13:1. C'est Dieu qui choisit souverainement qui il établit à la tête d'un pays. Selon les Ecritures, toute autorité vient de Dieu.}.
\VS{18}Voilà le songe que j'ai eu, moi, le roi Nebucadnetsar. Toi donc Beltschatsar, donnes-en l'interprétation, puisque tous les sages de mon royaume ne peuvent me la donner ; mais toi, tu le peux, parce l'Esprit des dieux saints est en toi.
\TextTitle{Interprétation de la vision}
\VS{19}Alors Daniel, nommé Beltschatsar, fut stupéfait environ une heure, et ses pensées le troublaient. Le roi reprit et dit : Beltschatsar, que le songe et son interprétation ne te troublent pas ! Et Beltschatsar répondit : Mon seigneur, que le songe soit pour ceux qui te haïssent, et son interprétation pour tes ennemis !
\VS{20}L'arbre que tu as vu, qui était devenu grand et fort, dont la cime s'élevait jusqu'aux cieux, et qu'on voyait de tous les points de la terre ;
\VS{21}cet arbre, dont le feuillage était beau, et les fruits abondants, qui portait de la nourriture pour tous, sous lequel s'abritaient les bêtes des champs, et parmi les branches duquel les oiseaux du ciel faisaient leur demeure,
\VS{22}c'est toi, ô roi, qui es devenu grand et fort, dont la grandeur s'est accrue et s'est élevée jusqu'aux cieux, et dont la domination s'étend jusqu'aux extrémités de la terre.
\VS{23}Le roi a vu un de ceux qui veillent et qui sont saints descendre des cieux et dire : Abattez l'arbre, et détruisez-le ! Toutefois, laissez en terre le tronc où se trouvent ses racines, et liez-le avec des chaînes de fer et d'airain, parmi l'herbe des champs, qu'il soit trempé de la rosée du ciel, et que son partage soit avec les bêtes des champs, jusqu'à ce que sept temps soient passés sur lui.
\VS{24}Voici l'interprétation, ô roi, voici le décret du Très-Haut, qui s'accomplira sur mon seigneur, le roi :
\VS{25}On te chassera du milieu des hommes, tu auras ta demeure avec les bêtes des champs, et l'on te donnera de l'herbe comme aux bœufs, et tu seras trempé de la rosée du ciel ; et sept temps passeront sur toi, jusqu'à ce que tu reconnaisses que le Très-Haut domine sur le royaume des hommes, et qu'il le donne à qui il lui plaît.
\VS{26}L'ordre de laisser le tronc où se trouvent les racines de cet arbre signifie que ton royaume te sera rendu, dès que tu auras reconnu que les cieux dominent.
\VS{27}C'est pourquoi, ô roi, que mon conseil te soit agréable : Rachète tes péchés par la justice, et tes iniquités en faisant miséricorde aux pauvres, et ta paix pourra se prolonger.
\TextTitle{Le roi déchu à cause de son orgueil}
\VS{28}Toutes ces choses se sont accomplies sur le roi Nebucadnetsar.
\VS{29}Au bout de douze mois, comme il se promenait dans le palais royal de Babylone,
\VS{30}le roi prit la parole et dit : N'est-ce pas ici Babylone la grande, que j'ai bâtie pour être la maison royale, par la puissance de ma force et pour la gloire de ma magnificence ?
\VS{31}La parole était encore dans la bouche du roi, qu'une voix descendit du ciel, disant : Roi Nebucadnetsar, on t'annonce que ton royaume va t'être ôté.
\VS{32}On te chassera du milieu des hommes, tu auras ta demeure avec les bêtes des champs ; on te donnera de l'herbe à manger comme aux bœufs ; et sept temps passeront sur toi, jusqu'à ce que tu reconnaisses que le Très-Haut domine sur le royaume des hommes, et qu'il le donne à qui il lui plaît.
\VS{33}Au même instant, la parole s'accomplit sur Nebucadnetsar. Il fut chassé du milieu des hommes, il mangea de l'herbe comme les bœufs, et son corps fut trempé de la rosée du ciel jusqu'à ce que ses cheveux croissent comme les plumes des aigles, et ses ongles comme ceux des oiseaux.
\TextTitle{Le roi est rétabli ; il loue le Dieu Très-Haut}
\VS{34}Mais à la fin de ces jours-là, moi Nebucadnetsar, je levai mes yeux vers le ciel, et la raison me revint. J'ai béni le Très-Haut, j'ai loué et glorifié celui qui vit éternellement, celui dont la domination est une domination éternelle, et dont le règne subsiste de génération en génération.
\VS{35}Tous les habitants de la terre ne sont à ses yeux que néant ; il agit comme il lui plaît avec l'armée des cieux et avec les habitants de la terre, et il n'y a personne qui empêche sa main, et qui lui dise : Que fais-tu\FTNT{Es. 45:9 ; Jé. 23:18-22 ; ps.115:3 ; Job. 9:12.} ?
\VS{36}En ce temps, la raison me revint, et je retournai à la gloire de mon royaume, ma magnificence et ma splendeur me furent rendues ; mes conseillers et mes grands me redemandèrent ; je fus rétabli dans mon royaume, et ma gloire fut augmentée.
\VS{37}Maintenant, moi, Nebucadnetsar, je loue, j'exalte, et je glorifie le Roi des cieux, dont toutes les œuvres sont véritables et ses voies justes, et qui peut abaisser ceux qui marchent avec orgueil\FTNT{De. 32:4 ; Es. 13:11 ; Ez. 17:24 ; Ps. 145:17.}.
\Chap{5}
\TextTitle{Les vases du temple souillés}
\VerseOne{}Le roi Belschatsar donna un grand festin à ses grands au nombre de mille, et il buvait le vin devant ces mille courtisans.
\VS{2}Et ayant goûté au vin, Belschatsar ordonna qu'on apporte les vases d'or et d'argent que Nebucadnetsar, son père, avait enlevés du temple de Jérusalem\FTNT{Ex. 27 ; Ex. 30 ; 2 Ch. 36:10.}, afin que le roi et ses grands, ses femmes et ses concubines, s'en servent pour boire.
\VS{3}Alors furent apportés les vases d'or qui avaient été enlevés du temple, de la maison de Dieu qui était à Jérusalem ; et le roi, et ses grands, ses femmes et ses concubines, s'en servirent pour boire.
\VS{4}Ils burent du vin, et ils louèrent leurs dieux d'or, d'argent, d'airain, de fer, de bois et de pierre.
\TextTitle{L'écriture sur la muraille}
\VS{5}Et à cette même heure-là sortirent de la muraille des doigts d'une main d'homme, qui écrivaient à l'endroit du chandelier, sur l'enduit de la muraille du palais royal ; et le Roi voyait cette partie de main qui écrivait.
\VS{6}Alors l'aspect du roi changea, et ses pensées l'effrayèrent, si bien que les jointures de ses reins se desserrèrent, et ses genoux se cognèrent l'un contre l'autre.
\VS{7}Puis le roi cria avec force qu'on fasse venir les astrologues, les Chaldéens et les devins ; et le roi prit la parole et dit aux sages de Babylone : Quiconque lira cette écriture, et m'en donnera l'interprétation, sera revêtu de pourpre, il aura un collier d'or à son cou, et sera le troisième dans le gouvernement du royaume.
\VS{8}Alors tous les sages du roi entrèrent, mais ils ne purent pas lire l'écriture et en donner au roi l'interprétation.
\VS{9}Sur quoi le roi Belschatsar fut très effrayé, il changea de couleur, et ses grands furent consternés.
\TextTitle{Interprétation de l'écriture: Division de l'empire babylonien}
\VS{10}La reine entra dans la maison du festin, à cause de ce qui était arrivé au roi et à ses grands. La reine prit la parole et dit : Ô roi, vis éternellement ! Que tes pensées ne te troublent pas, et que ton visage ne change pas de couleur !
\VS{11}Il y a dans ton royaume un homme qui a en lui l'Esprit des dieux saints ; et du temps de ton père, on trouva en lui une lumière, une intelligence, et une sagesse semblable à la sagesse des dieux. Aussi, le roi Nebucadnetsar, ton père, et le roi, ton père\FTNT{Belschatsar était le petit-fils de Nebucadnetsar qui avait régné conjointement avec son père, Nabonide, à partir de 552 av. J.-C.}, ô roi, l'établirent chef des magiciens, des astrologues, des Chaldéens et des devins,
\VS{12}parce qu'on trouva chez lui, chez Daniel, que le roi avait nommé Beltschatsar, un esprit supérieur, de la connaissance et de l'intelligence, pour interpréter les songes, pour expliquer les énigmes et résoudre les questions difficiles. Que Daniel soit donc appelé et il donnera l'interprétation que tu souhaites.
\VS{13}Alors Daniel fut introduit devant le roi. Le roi prit la parole et dit à Daniel : Es-tu ce Daniel, l'un des captifs de Juda, que le roi, mon père, a amenés de Juda ?
\VS{14}J'ai appris sur ton compte que tu as en toi l'Esprit des dieux, et qu'on trouve en toi une lumière, une intelligence et une sagesse extraordinaires.
\VS{15}On vient d'amener devant moi les sages et les astrologues, afin qu'ils lisent cette écriture et m'en donnent l'interprétation, mais ils n'ont pas pu donner l'interprétation de la chose.
\VS{16}J'ai appris que tu peux interpréter et résoudre les choses difficiles ; maintenant donc si tu peux lire cette écriture, et m'en donner l'interprétation, tu seras revêtu de pourpre, tu porteras à ton cou un collier d'or, et tu seras le troisième dans le gouvernement du royaume.
\VS{17}Alors Daniel répondit et dit en présence du roi : Que tes dons restent à toi, et donne tes présents à un autre ; toute fois je lirai l'écriture au roi, et je lui en donnerai l'interprétation.
\VS{18}Ô roi ! Le Dieu Très-Haut avait donné à Nebucadnetsar, ton père, le royaume, la magnificence, la gloire et l'honneur.
\VS{19}Et à cause de la grandeur qu'il lui avait donnée, tous les peuples, les nations, et les hommes de toutes langues tremblaient devant lui et le redoutaient. Il faisait mourir ceux qu'il voulait, et il laissait la vie à ceux qu'il voulait ; il élevait ceux qu'il voulait, et il abaissait ceux qu'il voulait.
\VS{20}Mais lorsque son cœur s'éleva et que son esprit s'endurcit jusqu'à l'arrogance, il fut renversé de son trône royal et dépouillé de sa gloire ;
\VS{21}il fut chassé du milieu des fils des hommes, son cœur fut rendu semblable à celui des bêtes, et sa demeure fut avec les ânes sauvages ; on lui donna comme aux bœufs de l'herbe à manger, et son corps fut trempé de la rosée du ciel, jusqu'à ce qu'il reconnaisse que le Dieu Très-Haut domine sur les royaumes des hommes, et qu'il y établit ceux qu'il lui plaît.
\VS{22}Et toi aussi, Belschatsar, son fils, tu n'as pas humilié ton cœur, quoique tu saches toutes ces choses.
\VS{23}Mais tu t'es élevé contre le Seigneur des cieux ; les vases de sa maison ont été apportés devant toi, et vous vous en êtes servis pour boire du vin, toi et tes grands, tes femmes et tes concubines ; tu as loué les dieux d'argent, d'or, d'airain, de fer, de bois et de pierre, qui ne voient point, qui n'entendent point, et qui ne savent rien, et tu n'as pas glorifié le Dieu dans la main duquel est ton souffle, et toutes tes voies\FTNT{Job. 12:10 et 33:4.}.
\VS{24}Alors de sa part a été envoyée cette partie de main, et cette écriture a été gravée.
\VS{25}Voici l'écriture qui a été gravée : Compté, compté, pesé et divisé.
\VS{26}Et voici l'interprétation de ces paroles. Compté : Dieu a compté ton règne, et y a mis la fin.
\VS{27}Pesé : Tu as été pesé dans la balance, et tu as été trouvé léger.
\VS{28}Mesuré : Ton royaume a été divisé, et donné aux Mèdes et aux Perses.
\VS{29}Aussitôt, Belschatsar donna des ordres, et l'on revêtit Daniel de pourpre, on lui mit un collier d'or au cou, et on publia qu'il serait le troisième dans le gouvernement du royaume.
\VS{30}Cette même nuit, Belschatsar, roi des Chaldéens, fut tué.
\VS{31}Et Darius, le Mède, reçut le royaume, étant âgé d'environ soixante-deux ans\FTNT{Es. 13:17 ; Es. 21:2 ; Jé. 51:11.}.
\Chap{6}
\TextTitle{Règne de Darius, le Mède}
\VerseOne{}Darius trouva bon d'établir sur le royaume cent vingt satrapes, qui devaient être répartis dans tout le royaume.
\VS{2}Il mit à leur tête trois chefs, au nombre desquels était Daniel, afin que ces satrapes leur rendent compte, et que le roi ne souffre aucun préjudice.
\VS{3}Daniel surpassait les autres chefs et satrapes, parce qu'il y avait en lui un Esprit supérieur ; et le roi pensait à l'établir sur tout le royaume.
\TextTitle{Daniel refuse l'idolâtrie et persévère dans la prière}
\VS{4}Alors les chefs et les satrapes cherchèrent une occasion d'accuser Daniel en ce qui concerne les affaires du royaume. Mais ils ne purent trouver en lui aucune occasion, ni aucune fausseté, parce qu'il était fidèle, et il ne se trouvait en lui ni faute ni vice.
\VS{5}Et ces hommes dirent : Nous ne trouverons aucune occasion d'accuser ce Daniel, à moins que nous n'en trouvions une dans la loi de son Dieu.
\VS{6}Alors ces chefs et ces satrapes se rendirent tumultueusement auprès du roi, et lui parlèrent ainsi : Roi Darius, vis éternellement !
\VS{7}Tous les chefs de ton royaume, les intendants, les satrapes, les conseillers, et les gouverneurs, sont d'avis d'établir un édit royal et une défense sévère, portant que quiconque, dans l'espace de trente jours, adressera des prières à quelque dieu ou à quelque homme, excepté à toi, ô roi, sera jeté dans la fosse aux lions.
\VS{8}Maintenant donc, ô roi, établis cette défense, et écris le décret afin qu'il soit irrévocable, selon la loi des Mèdes et des Perses, qui est immuable.
\VS{9}Là-dessus, le roi Darius écrivit le décret et la défense.
\VS{10}Lorsque Daniel sut que le décret était écrit, il entra dans sa maison, où les fenêtres de sa chambre étaient ouvertes dans la direction de Jérusalem ; et trois fois par jour, il se mettait à genoux, il priait, et il louait son Dieu, comme il le faisait auparavant\FTNT{1 R. 8:44 ; Ps. 55:17-18.}.
\VS{11}Alors ces hommes entrèrent tumultueusement, et ils trouvèrent Daniel qui priait et invoquait son Dieu.
\VS{12}Puis ils s'approchèrent du roi, et lui dirent au sujet de la défense royale : N'as-tu pas écrit une défense portant que tout homme dans l'espace de trente jours qui adresserait des prières à quelque dieu ou à quelque homme, excepté à toi, ô roi, serait jeté dans la fosse aux lions ? Le roi répondit : La chose est certaine, selon la loi des Mèdes et des Perses, qui est irrévocable.
\VS{13}Ils prirent de nouveau la parole et dirent au roi : Daniel, l'un des captifs de Juda, n'a tenu aucun compte de toi, ô roi, ni de la défense que tu as écrite, et il fait sa prière trois fois par jour.
\VS{14}Le roi fut très affligé quand il entendit cela ; il prit à cœur de délivrer Daniel, et jusqu'au coucher du soleil il s'efforça de le sauver.
\VS{15}Mais ces hommes se rendirent tumultueusement auprès du roi, et lui dirent : Sache, ô roi, que la loi des Mèdes et des Perses exige que toute défense ou tout décret établi par le roi soit irrévocable.
\TextTitle{Daniel demeure fidèle à Dieu face à la mort}
\VS{16}Alors le roi commanda qu'on amène Daniel, et qu'on le jette dans la fosse aux lions. Et le roi prenant la parole et dit à Daniel : Ton Dieu, lequel tu sers constamment, sera celui qui te délivrera.
\VS{17}On apporta une pierre, et on la mit sur l'ouverture de la fosse ; le roi la scella de son anneau, et de l'anneau de ses grands, afin que rien ne soit changé à l'égard de Daniel.
\TextTitle{Yahweh fait justice à Daniel}
\VS{18}Le roi se rendit ensuite dans son palais ; il passa la nuit à jeun, il ne fit point venir des danseuses\FTNT{Le mot « danseuse » vient de l'araméen « dachavah » qui signifie « divertissement », « instrument de musique », « danseuse », « concubine », « musique ».} auprès de lui, et il ne put se livrer au sommeil.
\VS{19}Puis le roi se leva au point du jour, avec l'aurore, et il alla précipitamment à la fosse aux lions.
\VS{20}En s'approchant de la fosse, il cria d'une voix triste : Daniel ! Le roi prit la parole et dit à Daniel : Daniel, serviteur du Dieu vivant, ton Dieu, que tu sers avec persévérance, a-t-il pu te délivrer des lions ?
\VS{21}Alors Daniel dit au roi : Ô roi, vis éternellement !
\VS{22}Mon Dieu a envoyé son ange, et a tellement fermé la gueule des lions, qu'ils ne m'ont fait aucun mal, parce que j'ai été trouvé innocent devant lui ; et même à ton égard, ô Roi ! je n'ai commis aucune faute. 
\VS{23}Alors le roi fut extrêmement heureux pour lui et il ordonna qu'on fasse retirer Daniel de la fosse. Ainsi Daniel fut retiré de la fosse, et on ne trouva sur lui aucune blessure, parce qu'il avait cru en son Dieu.
\VS{24}Le roi ordonna que ces hommes qui avaient accusé Daniel, soient amenés et jetés dans la fosse aux lions, eux, leurs enfants et leurs femmes, et avant qu'ils soient parvenus au fond de la fosse, les lions se saisirent d'eux, et leur brisèrent tous les os.
\TextTitle{Les merveilles de Yahweh proclamées aux nations}
\VS{25}Après cela, le roi Darius écrivit à tous les peuples, à toutes les nations, aux hommes de toutes les langues, qui habitent sur toute la terre : Que votre paix soit multipliée !
\VS{26}J'ordonne que dans toute l'étendue de mon royaume on ait de la crainte et de la frayeur pour le Dieu de Daniel, car c'est le Dieu vivant, et il subsiste éternellement ; son Royaume ne sera jamais détruit, et sa domination durera jusqu'à la fin\FTNT{Lu. 1:33 ; Es. 11.}.
\VS{27}Il sauve et délivre, il fait des prodiges et des merveilles dans les cieux et sur la terre, et il a délivré Daniel de la puissance des lions.
\VS{28}Ainsi Daniel prospéra sous le règne de Darius, et sous le règne de Cyrus, roi de Perse.
\Chap{7}
\TextTitle{Songe des quatre animaux ; Explication des visions de Daniel}
\VerseOne{}La première année de Belschatsar, roi de Babylone, Daniel eut un songe et des visions de sa tête, étant dans sa couche. Ensuite il écrivit le songe, et il relata les principales choses.
\VS{2}Daniel donc parla et dit : Je regardais dans ma vision nocturne, et voici, les quatre vents des cieux se levèrent avec impétuosité sur la grande mer.
\VS{3}Puis quatre grandes bêtes\FTNT{Les quatre bêtes représentent les quatre empires historiques. 
Le lion :
V. 4 : Le premier animal est un lion, il représente l'Empire néo-babylonien (625 – 539 av. J.-C.). Les ailes suggèrent la rapidité de la conquête babylonienne (Ha.1:6-8 ; Jé. 4:13). En 30 ans, l'Arabie, la Judée, la Syrie et la Phénicie furent conquises.
Les ailes arrachées annonçant l'arrêt des grandes conquêtes avec la mort de Nebucadnetsar. 
Le cœur d'homme donné au lion symbolise la conversion de Nebucadnetsar et le changement dans l'attitude des rois babyloniens (Da.4:30-31 ; 2 R. 25:27-30).
L'ours :
V. 5 : Le deuxième animal est un ours. Il représente l'empire médo-perse (539–331 av. J.-C.) qui succéda à l'Empire Babylonien.
Le fait que l'ours se tienne sur un côté indique que les Mèdes sont soumis aux Perses qui sont les véritables maîtres de l'empire. Les trois côtes dans la gueule de l'ours symbolisent trois grandes conquêtes médo-perses : la Lydie (546 av. J.-C.), la Babylonie (539 av. J.-C.) et l'Egypte (524 av. J.-C.).
Le léopard :
V. 6 : Le troisième animal est un léopard, qui représente l'Empire gréco-macédonien (331 – 146 av. J.-C.). En 331 av. J.-C., le coup de grâce est donné aux Médo-Perses à la bataille d'Arbèles.
Les quatre ailes symbolisent la grande rapidité des conquêtes. Quand Alexandre le Grand mourut à l'âge de 33 ans, il avait le plus grand Empire jamais vu jusqu'à l'époque. Ses conquêtes s'étendaient jusqu'en Inde !
Les quatre têtes symbolisent quatre de ses généraux qui, à la mort d'Alexandre, se partagèrent l'immense empire : Cassandre en Grèce et en Macédoine, Lysimaque en Thrace et en Asie Mineure, Séleucus en Syrie et en orient, Ptolémée en Égypte.
Très rapidement, la Palestine, qui se trouvait au croisement des routes, fut l'objet de rivalités entre les généraux et leurs successeurs. Après quelques années de stabilité, les généraux luttèrent entre eux jusqu'au maintien de deux dynasties : les Séleucides, au nord, et les Lagides, au sud, en Égypte. Cela dura jusqu'à l'apparition de l'Empire romain.
La quatrième bête, est différente des autres
V. 7, 19, 24 : La quatrième bête est extraordinaire, terrible, effrayante, elle ne porte même pas de nom ! Elle représente l'Empire romain qui succéda à l'Empire gréco-macédonien (146 av. J.-C. – 476 ap. J.-C.). En 168 av. J.-C., la Macédoine passa sous le contrôle de Rome, puis, en 146 av. J.-C., c'est au tour de la Grèce de devenir une province romaine.
 Le quatrième empire ne peut être que celui de Rome comme l'enseigne l'histoire de l'antiquité. 
Dès le quatrième siècle, l'Empire romain fut assailli par les tribus barbares venues du nord (Alamans, Wisigoths, Goths, Vandales, Burgondes, Ostrogoths, etc.) et, en 476, le dernier empereur romain d'occident, Romulus Augustule, fut chassé par le roi barbare Odoacre (Goth). L'Empire romain n'est plus.
Les orteils en partie de fer et en partie d'argile représentent les nations européennes issues de la fragmentation de l'Empire romain qui a eu lieu le 4 septembre 476.} montèrent de la mer, différentes les unes des autres.
\TextTitle{Premier empire universel: Babylone\FTNTT{Cp. Da. 2:37-38.}}
\VS{4}La première était semblable à un lion, et avait des ailes d'aigle ; je la regardai jusqu'à ce que les plumes de ses ailes furent arrachées ; elle fut enlevée de terre et dressée sur ses pieds comme un homme, et un cœur d'homme lui fut donné.
\TextTitle{Deuxième empire: Les Mèdes et les Perses\FTNTT{Cp. Da. 2:39 ; 8:20.}}
\VS{5}Et voici, une deuxième bête était semblable à un ours, et se tenait sur un côté ; il avait trois côtes dans la gueule entre ses dents ; et on lui disait ainsi : Lève-toi, mange beaucoup de chair.
\TextTitle{Troisième empire: La Grèce\FTNTT{Cp. Da. 2:39 ; 8:21-22 ; 11:2-4.}}
\VS{6}Après cela je regardai, et voici une autre bête, semblable à un léopard, qui avait sur son dos quatre ailes d'oiseau, et cette bête avait quatre têtes, et la domination lui fut donnée.
\TextTitle{Quatrième empire: Rome\FTNTT{Cp. Da. 2:40-43 ; 7:23-24 ; 9:26.}}
\VS{7}Après cela, je regardai dans mes visions nocturnes, et voici, il y avait une quatrième bête, terrible, épouvantable et extraordinairement forte ; elle avait de grandes dents de fer, elle mangeait, brisait, et elle foulait à ses pieds ce qui restait ; elle était différente de toutes les bêtes qui avaient été avant elle, et elle avait dix cornes.
\TextTitle{Les dix cornes et la petite corne\FTNTT{Da. 7:24-27.}}
\VS{8}Je considérai ses cornes, et voici, une autre petite corne sortit du milieu d'elles, et trois des premières cornes furent arrachées par elle ; et voici, elle avait des yeux comme des yeux d'homme, et une bouche qui proférait de grandes choses.
\TextTitle{Le règne de Yahweh, l'Ancien des jours\FTNTT{Cp. Mt. 24:27-30 ; 25:31-34 ; Ap. 19:11-21.}}
\VS{9}Je regardai jusqu'à ce que les trônes soient placés. Et l'Ancien des jours s'assit. Son vêtement était blanc comme la neige, et les cheveux de sa tête étaient comme de la laine pure ; son trône était des flammes de feu, et ses roues un feu ardent.
\VS{10}Un fleuve de feu coulait et sortait de devant lui. Mille milliers le servaient, et dix mille millions se tenaient en sa présence. Le jugement se tint, et les livres furent ouverts\FTNT{Ap. 5:11 ; Ps. 68:18 ; 1 R. 22:19.}.
\VS{11}Je regardai alors, à cause du bruit des paroles arrogantes que proférait la corne ; et tandis que je regardais, la bête fut tuée, et son corps fut détruit et livré pour être brûlé au feu.
\VS{12}Les autres bêtes furent dépouillées de leur domination, mais une prolongation de vie leur fut accordée jusqu'à un temps déterminé.
\TextTitle{La domination du Fils de l'homme est éternelle\FTNTT{Cp. Ap. 5:1-14.}}
\VS{13}Je regardai encore dans les visions nocturnes, et je vis, comme le Fils de l'homme, qui venait avec les nuées des cieux, et il vint jusqu'à l'Ancien des jours, et se tint devant lui\FTNT{Ap. 19:14 ; Jud. 1:14.}.
\VS{14}Et il lui donna la domination, la gloire et le règne ; et tous les peuples, les nations et les langues le serviront. Sa domination est une domination éternelle qui ne passera point, et son règne ne sera jamais détruit.
\TextTitle{Interprétation de la vision du quatrième animal}
\VS{15}Moi, Daniel, j'eus l'esprit troublé au-dedans de moi, et les visions de ma tête m'effrayèrent.
\VS{16}Je m'approchai de l'un des assistants, et lui demandai ce qu'il y avait de vrai dans toutes ces choses. Il me parla, et me donna l'interprétation de ces choses, en disant :
\VS{17}Ces quatre grandes bêtes sont quatre rois, qui s'élèveront de la terre.
\VS{18}Mais les saints du Très-Haut recevront le Royaume, et ils posséderont le Royaume éternellement, d'éternité en éternité.
\VS{19}Alors, je désirai savoir la vérité sur la quatrième bête, qui était différente de toutes les autres, extraordinairement terrible, qui avait des dents de fer et des ongles d'airain, qui mangeait, brisait, et foulait à ses pieds ce qui restait ;
\VS{20}et sur les dix cornes qu'elle avait à la tête, et sur l'autre corne qui était sortie et devant laquelle trois étaient tombées, sur cette corne qui avait une bouche parlant avec arrogance, et une plus grande apparence que celle de ses associées.
\VS{21}Je regardai comment cette corne faisait la guerre aux saints et l'emportait sur eux\FTNT{Ap. 13:2-7.},
\VS{22}jusqu'au moment où l'Ancien des jours vint donner droit aux saints du Très-Haut, et que le temps arriva où les saints furent en possession du Royaume.
\VS{23}Il me parla donc ainsi : La quatrième bête est un quatrième royaume qui sera sur la terre, différent de tous les royaumes, et qui dévorera toute la terre, la foulera, et la brisera.
\TextTitle{Règne de l'homme impie et jugement de Dieu}
\VS{24}Mais les dix cornes sont dix rois qui s'élèveront de ce royaume. Un autre s'élèvera après eux, il sera différent des premiers, et il abattra trois rois.
\VS{25}Il proférera des paroles contre le Très-Haut, il harcelera les saints du Très-Haut, et il aura l'intention de changer les temps et la loi ; et les saints seront livrés entre ses mains pendant un temps, des temps, et la moitié d'un temps.
\VS{26}Mais le jugement se tiendra, et on lui ôtera sa domination, en la détruisant et la faisant périr, jusqu'à en voir la fin..
\VS{27}Afin que le règne, la domination, et la grandeur de tous les royaumes qui sont sous les cieux, soient donnés au peuple des saints du Très-Haut. Son royaume est un royaume éternel, et tous les royaumes lui seront assujettis et lui obéiront.
\VS{28}Jusqu'ici est la fin de cette affaire. Quant à moi, Daniel, mes pensées m'effrayèrent beaucoup, et ma splendeur changea en moi, toutefois je gardais cette affaire dans mon cœur.
\Chap{8}
\TextTitle{Vision du bélier et du bouc}
\VerseOne{}La troisième année du règne du roi Belschatsar, moi, Daniel, j'eus cette vision, en plus de celle que j'avais eue auparavant.
\VS{2}Je vis cette vision, et il arriva, comme je regardais, que j'étais à Suse, la capitale, dans la province d'Élam, et dans ma vision, je me trouvais près du fleuve d'Ulaï.
\VS{3}Et je levai mes yeux, je regardai, et voici, un bélier se tenait devant le fleuve, et il avait deux cornes ; et les deux cornes étaient hautes, mais l'une était plus haute que l'autre, et la plus haute s'éleva sur la dernière.
\VS{4}Je vis ce bélier qui frappait de ses cornes à l'occident, au nord, et au midi ; aucune bête ne pouvait subsister devant lui,et il n'y avait personne qui puisse délivrer de sa puissance ; et il agissait selon sa volonté et devenait grand.
\VS{5}Comme je regardais attentivement, voici, un bouc d'entre les chèvres venait de l'occident, et parcourait toute la terre à sa surface, sans la toucher ; ce bouc avait entre les yeux une corne considérable.
\VS{6}Il arriva jusqu'au bélier qui avait deux cornes et que j'avais vu se tenant devant le fleuve, et il courut sur lui dans la fureur de sa force.
\VS{7}Je le vis qui s'approchait du bélier et s'irritait contre lui ; il frappa le bélier et lui brisa les deux cornes, et le bélier n'avait aucune force pour tenir ferme contre lui ; et quand il l'eut jeté par terre, il le foula; et il n'y eut personne pour délivrer le bélier de sa puissance.
\VS{8}Alors le bouc d'entre les chèvres grandit extrêmement ; mais lorsqu'il fut puissant, sa grande corne se brisa. Quatre grandes cornes s'élevèrent pour la remplacer, aux quatre vents des cieux.
\TextTitle{La petite corne renverse la vérité}
\VS{9}De l'une d'elles sortit une petite corne\FTNT{Antiochus IV Épiphane est le fils d'Antiochos III le Grand, né vers 215 av. J.-C. Il gouverna le royaume séleucide de 175 av. J.-C. à 164 av. J.-C., date de sa mort. Ce dernier avait profané le temple de Jérusalem en sacrifiant des porcs sur l'autel (Voir commentaire en Mt. 24:15). Cette petite corne, qui fait tomber par terre une partie de l'armée des étoiles, agit de même que Satan au ciel qui avait fait chuter un tiers des étoiles, soit des anges (Ap. 12:3-4).}, qui s'agrandit beaucoup vers le midi, et vers l'orient, et vers le pays de noblesse.
\VS{10}Elle s'éleva même jusqu'à l'armée des cieux, elle fit tomber à terre une partie de l'armée et des étoiles, et elle les foula\FTNT{Es. 14:12-15 ; Ez. 28:12-19.}.
\VS{11}Et elle s'éleva même jusqu'au chef de l'armée, lui enleva le sacrifice perpétuel, et renversa la demeure de son sanctuaire.
\VS{12}L'armée fut livrée avec le sacrifice perpétuel, à cause du péché ; la corne jeta la vérité par terre, et fit de grands exploits, et prospéra.
\VS{13}Alors j'entendis un saint qui parlait ; et un autre saint disait à celui qui parlait : Jusqu'à quand durera cette vision sur le sacrifice perpétuel et sur le péché qui cause la désolation ? Jusqu'à quand le sanctuaire et l'armée seront-ils foulés ?
\VS{14}Et il me dit : Deux mille trois cents soirs et matins ; puis le sanctuaire sera purifié.
\TextTitle{La vision du bélier et du bouc interprétée}
\VS{15}Et quand à moi, Daniel, j'avais cette vision et que je désirais la comprendre, voici, quelqu'un qui avait l'apparence d'un homme se tenait devant moi.
\VS{16}Et j'entendis la voix d'un homme au milieu du fleuve Ulaï ; il cria et dit : Gabriel, explique-lui la vision.
\VS{17}Puis Gabriel vint alors près du lieu où je me tenais ; et à son approche, je fus effrayé, et je tombai sur ma face. Il me dit : Comprends, fils de l'homme, car la vision est pour le temps de la fin.
\VS{18}Comme il me parlait, je restai frappé d'étourdissement, la face contre terre. Il me toucha, et me fit tenir debout à la place où je me trouvais.
\VS{19}Et il dit : Voici, je vais t'apprendre ce qui arrivera à la fin de la colère, car il y a un temps marqué pour la fin.
\VS{20}Le bélier que tu as vu qui avait deux cornes, ce sont les rois des Mèdes et des Perses ;
\VS{21}et le bouc velu, c'est le roi de Javan\FTNT{Javan ou Grèce.} ; et la grande corne entre ses yeux, c'est le premier roi.
\VS{22}Les quatre cornes qui se sont élevées pour remplacer cette corne brisée, ce sont quatre royaumes qui s'élèveront de cette nation, mais qui n'auront pas autant de force.
\TextTitle{Le roi impie, adversaire de Dieu ; la vision scellée}
\VS{23}A la fin de leur règne, lorsque les pécheurs seront consumés, il se lèvera un roi cruel et artificieux.
\VS{24}Sa puissance s'accroîtra, mais non par sa propre force ; il fera d'incroyables ravages, il réussira dans ses entreprises, il détruira les puissants et le peuple des saints.
\VS{25}Et par la subtilité de son esprit, il fera prospérer la fraude dans sa main. Il aura de l'arrogance dans le cœur, et fera périr beaucoup d'hommes qui vivaient dans la paix, et il s'élèvera contre le Prince des princes ; mais il sera brisé, sans l'effort d'aucune main.
\VS{26}Et la vision du soir et du matin, dont il s'agit, est véritable. Mais toi, scelle la vision, car elle se rapporte à un temps éloigné.
\VS{27}Moi Daniel, je fus tout défait et malade pendant quelques jours ; puis je me levai, et je m'occupai des affaires du roi. J'étais étonné de la vision, et personne n'en eut connaissance.
\Chap{9}
\TextTitle{Supplications de Daniel à Yahweh}
\VerseOne{}La première année de Darius, fils d'Assuérus, de la race des Mèdes, lequel était établi roi sur le royaume des Chaldéens.
\VS{2}La première année, dis-je, de son règne, moi Daniel, je discernai par les livres, que le nombre des années dont Yahweh avait parlé au prophète Jérémie\FTNT{Jé. 25:11.} pour finir les désolations de Jérusalem, était de soixante et dix ans. 
\VS{3}Et je tournai ma face vers le Seigneur Dieu, pour le chercher par la prière et des supplications, avec jeûne, et le sac et la cendre.
\VS{4}Je priai Yahweh, mon Dieu, et je lui fis ma confession : Ah ! Seigneur, Dieu grand et redoutable, toi qui gardes ton alliance et qui fais miséricorde à ceux qui t'aiment et qui gardent tes commandements !
\VS{5}Nous avons péché, nous avons commis l'iniquité, nous avons agi méchamment, nous avons été rebelles, et nous nous sommes détournés de tes commandements et de tes ordonnances.
\VS{6}Nous n'avons pas écouté tes serviteurs, les prophètes, qui ont parlé en ton Nom à nos rois, à nos chefs, à nos pères, et à tout le peuple du pays.
\VS{7}Ô Seigneur ! A toi est la justice, et à nous la confusion de face, en ce jour, aux hommes de Juda, aux habitants de Jérusalem, et à tout Israël, à ceux qui sont près et à ceux qui sont loin, dans tous les pays où tu les as dispersés, à cause des infidélités dont ils se sont rendus coupables envers toi\FTNT{Né. 9:30 ; Ps. 106:6 ; La. 3:42.}.
\VS{8}Seigneur, à nous est la confusion de face, à nos rois, à nos chefs, et à nos pères, parce que nous avons péché contre toi.
\VS{9}Auprès du Seigneur, notre Dieu, la miséricorde et le pardon, car nous avons été rebelles envers lui.
\VS{10}Nous n'avons pas écouté la voix de Yahweh, notre Dieu, pour marcher dans ses lois, qu'il a mises devant nous par le moyen de ses serviteurs, les prophètes.
\VS{11}Tout Israël a transgressé ta loi, et s'est détourné pour ne pas écouter ta voix. Alors se sont répandues sur nous les malédictions et les imprécations qui sont écrites dans la loi de Moïse, serviteur de Dieu, parce que nous avons péché contre Dieu\FTNT{Lé. 26:14-33 ; De. 27:15-33.}.
\VS{12}Il a accompli les paroles qu'il avait prononcées contre nous, et contre nos chefs qui nous ont gouvernés, et il a fait venir sur nous un grand mal, et il n'en est jamais arrivé sous le ciel entier un semblable à celui qui est arrivé à Jérusalem.
\VS{13}Comme cela est écrit dans la loi de Moïse, ce mal est venu sur nous ; et nous n'avons pas imploré Yahweh, notre Dieu, pour nous détourner de nos iniquités, et pour nous rendre attentifs à ta vérité.
\VS{14}Yahweh a veillé sur le mal que nous avons fait et il l'a fait venir sur nous ; car Yahweh, notre Dieu, est juste dans toutes les œuvres qu'il a faites, vu que nous n'avons point obéi à sa voix.
\VS{15}Or maintenant, Seigneur, notre Dieu ! Toi qui as tiré ton peuple du pays d'Egypte par ta main puissante, et qui t'es acquis un Nom comme il l'est aujourd'hui, nous avons péché, nous avons été méchants.
\VS{16}Seigneur, je te prie que selon ta justice, que ta colère et ton indignation se détournent de ta ville de Jérusalem, de la montagne de ta sainteté ; car à cause de nos péchés et des iniquités de nos pères, Jérusalem et ton peuple sont en opprobre à tous ceux qui nous entourent.
\VS{17}Maintenant donc, ô notre Dieu, écoute la prière et les supplications de ton serviteur, et pour l'amour du Seigneur, fais briller ta face sur ton sanctuaire dévasté.
\VS{18}Mon Dieu ! Prête l'oreille, et écoute ; ouvre tes yeux, et regarde nos ruines, et la ville sur laquelle ton Nom a été invoqué ; car ce n'est pas à cause de notre justice que nous te présentons nos supplications, c'est à cause de tes grandes compassions.
\VS{19}Seigneur, exauce, Seigneur pardonne, Seigneur sois attentif, et opère ; ne tarde pas, par amour pour toi, ô mon Dieu ! Car ton Nom a été invoqué sur ta ville, et sur ton peuple.
\VS{20}Je parlais encore, je priais, je confessais mon péché, et le péché de mon peuple d'Israël, et je présentais ma supplication à Yahweh, mon Dieu, en faveur de la sainte montagne de mon Dieu.
\TextTitle{Les soixante-dix semaines}
\VS{21}Je parlais encore dans ma prière, quand l'homme Gabriel, que j'avais vu précédemment dans une vision, s'approcha de moi d'un vol rapide au moment de l'offrande du soir.
\VS{22}Il m'instruisit, et s'entretint avec moi. Il me dit : Daniel, je suis venu maintenant pour ouvrir ton intelligence.
\VS{23}La parole est sortie dès le commencement de tes supplications, et je suis venu pour te la déclarer, car tu es un bien-aimé. Sois attentif à la parole, et comprends la vision.
\VS{24}Il y a soixante-dix semaines\FTNT{Le verset 24 concerne la chronologie de l'accomplissement de la prophétie de Jérémie (25 : 11). Les soixante-dix semaines auxquelles elle fait allusion représentent une période de 490 ans, conformément au principe biblique prophétique selon lequel un jour prophétique équivaut à une année (No. 14:33-34 ; Ez. 4:4-6). Dans les versets 25 et 27, les soixante-dix semaines sont divisées en trois périodes : 7 semaines (49 ans), 62 semaines (434 ans), et une semaine (7 ans). Les soixante-dix semaines devaient débuter au moment où la parole a annoncé que Jérusalem serai rebâtie (v. 25). En 445 avant notre ère, dans la vingtième année de son règne, le roi Artaxerxès publia un décret permettant à Esdras de retourner à Jérusalem pour achever la reconstruction de la ville (Esd. 7:6-10 ; Esd. 9:9 ; Né 2:5). Il est attesté par l'histoire profane que cette date est le point de départ de la soixante-dixième semaine de Daniel. Les 69 premières semaines vont jusqu'au Messie conducteur. La semaine qui reste (7 ans) concerne la période du règne de l'Antéchrist, celle-ci est divisée en deux période : trois ans et demi de fausse paix (1 Th. 5:3), et trois ans et demi concernent la Grande Tribulation (Ap. 7:9-17 ; Ap. 11:1-3 ; Ap. 12:6 ; Ap. 13:5).} fixées sur ton peuple et sur ta ville sainte, pour abolir la transgression et mettre fin aux péchés, faire la propitiation pour l'iniquité, pour amener la justice éternelle, pour mettre le sceau à la vision et à la prophétie et pour oindre le Saint des saints.
\VS{25}Tu sauras donc et tu comprendras, que depuis le moment où la parole a annoncé que Jérusalem sera rebâtie jusqu'au Messie, le Conducteur, il y a sept semaines et soixante-deux semaines ; et les places et les brèches seront rebâties, mais en des temps d'angoisse.
\VS{26}Et après ces soixante-deux semaines, le Messie sera retranché, mais non pas pour lui. Le peuple du chef qui viendra, détruira la ville et le sanctuaire, et sa fin arrivera comme par une inondation ; il est déterminé que les dévastations dureront jusqu'à la fin de la guerre.
\VS{27}Et il confirmera l'alliance à plusieurs pour une semaine, et à la moitié de cette semaine il fera cesser le sacrifice, et l'offrande ; puis par le moyen des ailes abominables, qui causeront la désolation, même jusqu'à une consomption déterminée, la désolation fondra sur le désolé.
\Chap{10}
\TextTitle{Daniel voit la gloire du Messie}
\VerseOne{}La troisième année de Cyrus, roi de Perse, une parole fut révélée à Daniel, qu'on nommait Beltschatsar. Cette parole est véritable et annonce une grande guerre. Il fut attentif à cette parole, et il eut l'intelligence de la vision.
\VS{2}En ce temps-là, moi Daniel, je fus dans le deuil pendant trois semaines entières.
\VS{3}Je ne mangeai aucun mets délicat, il n'entra ni viande ni vin dans ma bouche, et je ne m'oignis point, jusqu'à ce que ces trois semaines entières soient accomplies.
\VS{4}Le vingt-quatrième jour du premier mois, j'étais au bord du grand fleuve qui est Hiddékel.
\VS{5}Je levai les yeux, et je regardai, et voici, il y avait un homme vêtu de lin, et ayant sur les reins une ceinture d'or fin d'Uphaz.
\VS{6}Son corps était comme de chrysolithe, et son visage brillait comme l'éclair, ses yeux étaient comme des flammes de feu, ses bras et ses pieds ressemblaient à de l'airain poli, et le son de sa voix était comme le bruit d'une multitude de gens\FTNT{Ap. 1:13-15.}.
\VS{7}Moi, Daniel, je vis seul la vision, et les hommes qui étaient avec moi ne la virent point ; mais ils furent saisis d'une grande frayeur, et ils s'enfuirent pour se cacher.
\VS{8}Je restai seul, et je vis cette grande vision ; les forces me manquèrent, mon visage changea de couleur et fut tout défait, et je ne conservai aucune vigueur.
\VS{9}J'entendis le son de ses paroles ; et comme j'entendais le son de ses paroles, je tombai frappé d'étourdissement, la face contre terre.
\TextTitle{Le combat du monde spirituel}
\VS{10}Et voici, une main me toucha et me fit mettre sur mes genoux, et sur les paumes de mes mains.
\VS{11}Puis il me dit : Daniel, homme aimé de Dieu, sois attentif aux paroles que je vais te dire, et tiens-toi debout à la place où tu es ; car je suis maintenant envoyé vers toi. Lorsqu'il m'eut ainsi parlé, je me tins debout en tremblant.
\VS{12}Il me dit : Ne crains rien, Daniel, car dès le premier jour où tu as appliqué ton cœur à comprendre, et à t'humilier devant ton Dieu, tes paroles ont été exaucées, et c'est à cause de tes paroles que je viens.
\VS{13}Mais le chef du royaume de Perse m'a résisté vingt et un jours ; mais voici, Micaël, l'un des principaux chefs, est venu à mon secours, et je suis demeuré là auprès des rois de Perse.
\VS{14}Je viens maintenant pour te faire connaître ce qui doit arriver à ton peuple dans les derniers jours, car la vision s'étend jusqu'à ces temps-là.
\VS{15}Pendant qu'il m'adressait ces paroles, je mis mon visage contre terre, et je gardai le silence.
\VS{16}Et voici, quelqu'un qui avait l'apparence des fils de l'homme toucha mes lèvres. J'ouvris la bouche, je parlai, et je dis à celui qui se tenait devant moi : Mon seigneur ! La vision m'a rempli d'effroi, et j'ai perdu toute vigueur.
\VS{17}Comment le serviteur de mon seigneur pourrait-il parler avec mon seigneur ? Maintenant les forces me manquent, et je n'ai plus de souffle.
\VS{18}Alors celui qui avait l'apparence d'un homme me toucha encore, et me fortifia.
\VS{19}Puis il me dit : Ne crains rien, homme bien-aimé, que la paix soit avec toi ! Fortifie-toi, fortifie-toi ! Et comme il me parlait, je repris des forces, et je dis : Que mon seigneur parle, car tu m'as fortifié.
\VS{20}Il me dit : Ne sais-tu pas pourquoi je suis venu vers toi ? Maintenant je m'en retournerai pour combattre le chef de Perse ; et quand je partirai, voici, le chef de Javan viendra.
\VS{21}Mais je veux te faire connaître ce qui est écrit dans le livre de vérité. Et il n'y a personne qui me soutienne contre ceux-là, excepté Micaël, votre chef.
\Chap{11}
\TextTitle{Succession des monarques jusqu'à l'homme impie\FTNTT{Da. 11:1 - 12:13.}}
\VerseOne{}Et moi, dans la première année de Darius, le Mède, je me tenais auprès de lui pour l'aider et le fortifier.
\VS{2}et maintenant, je vais te faire connaître la vérité : Voici, il y aura encore trois rois en Perse. Le quatrième amassera plus de richesses que les autres ; et quand il sera puissant par ses richesses, il soulèvera tout le monde contre le royaume de Javan.
\VS{3}Mais il s'élèvera un vaillant roi\FTNT{Ce vaillant roi est Alexandre le Grand qui règna de 336 à 323 av. J.-C.}, qui dominera avec une grande puissance, et fera ce qu'il voudra.
\VS{4}Et sitôt qu'il sera élevé, son royaume sera brisé et sera divisé\FTNT{A la mort d'Alexandre le Grand, ses quatre principaux généraux se partagèrent l'empire : 
-Lysimaque régna sur l'Asie mineure. 
-Cassandre régna sur la Grèce et la Macédoine.
-Seleucos régna en Syrie, en Babylonie et sur toutes les régions à l'est jusqu'aux Indes. 
-Ptolémée régna sur l'Égypte, la Judée et une partie de la Syrie.} vers les quatre vents des cieux ; il ne passera point à ses descendants, et n'aura pas la même puissance qu'il a exercée, car son royaume sera déchiré, et il passera à d'autres qu'à eux.
\VS{5}Le roi du midi\FTNT{Le roi du midi est Ptolémée 1er Soter (règne : 323-285 av. J.-C.), le chef plus fort que lui est Séleucus 1er Nicator (règne : 305-281 av. J.-C.).} deviendra fort et puissant. Mais un de ses chefs [roi de Javan] sera plus puissant que lui et dominera ; sa domination sera puissante.
\VS{6}Au bout de quelques années, ils s'allieront, et la fille du roi du midi viendra vers le roi du nord pour redresser les affaires. Mais elle ne conservera pas la force de son bras, et il ne résistera pas, ni lui ni son bras ; elle sera livrée avec ceux qui l'auront amenée, avec son père et avec celui qui aura été son soutien dans ce temps-là.
\VS{7}Mais un rejeton de ses racines s'élèvera pour le remplacer\FTNT{Ptolémée III Evergète (règne : 246-222 av. J.-C.)} ; il viendra à l'armée, il entrera dans les forteresses du roi du nord, il en disposera à son gré, et il se rendra puissant.
\VS{8}Et même il emmènera captifs en Egypte leurs dieux, avec leurs images de fonte et avec leurs vases précieux d'argent et d'or. Puis il restera quelques années éloigné du roi du nord.
\VS{9}Et celui-ci marchera contre le royaume du roi du midi, et retournera dans son pays.
\VS{10}Ses fils\FTNT{Ses fils : Ce sont les deux rois de Syrie, Seuleucus III Ceraunus (règne : 225-223 av. J.-C.), et Antiochus III le Grand (223-187 av. J.-C.).} entreront en guerre et rassembleront une multitude nombreuse de troupes ; l'un d'eux s'avancera et se répandra comme un torrent, débordera, puis reviendra ; et il poussera la guerre jusqu'à la forteresse du roi du midi.
\VS{11}Et le roi du midi sera irrité, il sortira et combattra contre lui, savoir contre le roi du nord ; il soulèvera une grande multitude, et les troupes du roi du nord seront livrées entre les mains du roi du midi.
\VS{12}Et après avoir défait cette multitude, le cœur du roi s'élèvera ; il fera tomber des milliers, mais il ne triomphera pas.
\VS{13}Car le roi du nord reviendra et rassemblera une plus grande multitude que la première ; au bout de quelque temps, de quelques années, il viendra avec une grande armée et de grandes richesses.
\VS{14}Et en ce temps-là, plusieurs s'élèveront contre le roi du midi ; et des hommes violents parmi ton peuple se révolteront pour accomplir la vision, mais ils succomberont.
\VS{15}Le\FTNT{Le roi du nord est Séleucos IV Philopator (règne :187-175 av. J.-C.)} roi du nord viendra, il élèvera des terrasses, et prendra les villes fortes. Les bras du midi et l'élite du roi ne résisteront pas, ils manqueront de force pour résister.
\VS{16}Celui qui marchera contre lui fera ce qu'il voudra, et personne ne lui résistera ; il s'arrêtera dans le pays de noblesse, exterminant ce qui tombera sous sa main.
\VS{17}Puis il tournera sa face pour entrer avec la force de tout son royaume, et fera un accord avec le roi du midi, et il lui donnera sa fille pour femme, pour ruiner le royaume ; mais cela ne tiendra pas, et elle ne sera pas pour lui. 
\VS{18}Puis il tournera ses vues vers les îles, et il en prendra plusieurs ; mais un chef mettra fin à l'opprobre qu'il voulait lui attirer, et le fera retomber sur lui.
\VS{19}Il se dirigera ensuite vers les forteresses de son pays ; et il chancellera, il tombera, et on ne le trouvera plus.
\VS{20}Et un autre sera établi à sa place, qui fera passer un exacteur dans l'ornement du royaume, et en peu de jours il sera brisé, et ce ne sera ni par la colère ni par la guerre.
\TextTitle{Usage de la tromperie pour régner}
\VS{21}Et sa place il en sera établi un autre qui sera méprisé, auquel on ne donnera pas l'honneur royal ; mais il viendra en paix, et il s'emparera du royaume par des flatteries.
\VS{22}Les troupes qui se répandront comme un torrent seront submergées devant lui, et brisées, de même qu'un chef de l'alliance.
\VS{23}Mais après les accords faits avec lui, il usera de tromperie, et il montera, et il aura le dessus avec peu de gens.
\VS{24} Il entrera tranquillement dans les lieux les plus riches de la province, et il fera ce que n'avaient pas fait ses pères, ni les pères de ses pères ; il distribuera le butin, le pillage et les richesses ; et il formera des desseins contre les places fortes, et cela jusqu'à un certain temps.
\VS{25}Puis il réveillera sa force et son coeur contre le roi du midi avec une grande armée. Et le roi du midi s'avancera en bataille avec une très grande et très forte armée ; mais il ne résistera pas, car on formera des complots contre lui.
\VS{26}Ceux qui mangent les mets de sa table le mettront en pièces ; son armée se répandra comme un torrent, et beaucoup de gens tomberont blessés à mort.
\VS{27}Et les deux rois chercheront en leur cœur à se nuire, et à la même table ils parleront avec fausseté. Mais cela ne réussira pas ; car la fin ne viendra qu'au temps marqué.
\VS{28}Après quoi il retournera dans son pays avec de grandes richesses ; et son cœur sera contre la sainte alliance, il agira contre elle, puis retournera dans son pays.
\VS{29}Ensuite il retournera au temps fixé, et il viendra contre le midi ; mais cette dernière expédition ne sera pas comme la précédente.
\VS{30}Car les navires de Kittim viendront contre lui ; affligé, il rebroussera chemin. Puis irrité contre la sainte alliance, il agira contre elle, il retournera et s'entendra avec les apostats de la sainte alliance.
\VS{31}Et les forces seront de son côté, et on profanera le sanctuaire qui est la forteresse, et on fera cesser le sacrifice perpétuel, et on y dressera l'abomination qui causera la désolation.
\VS{32}Et il corrompra par des flatteries ceux qui agissent méchamment à l'égard de l'alliance. Mais ceux du peuple qui connaîtront leur Dieu agiront avec courage.
\VS{33}Et les plus intelligents parmi le peuple donneront instruction à plusieurs. Il en est qui succomberont pour un temps à l'épée et à la flamme, à la captivité et au pillage.
\VS{34}Dans le temps où ils succomberont, ils seront un peu secourus, et plusieurs se joindront à eux par hypocrisie.
\VS{35}Et quelques-uns des hommes intelligents succomberont, afin qu'ils soient épurés, purifiés et blanchis, jusqu'au temps de la fin, car elle n'arrivera qu'au temps marqué.
\TextTitle{Blasphème du roi contre Yahweh, le Dieu des dieux}
\VS{36}Le roi fera ce qu'il voudra, il s'élèvera, il se glorifiera au-dessus de tous les dieux ; il proférera des choses étranges contre le Dieu des dieux, il prospérera jusqu'à ce que la colère soit consommée, car ce qui est décrété sera exécuté.
\VS{37}Il n'aura égard ni aux dieux de ses pères, ni à l'objet du désir des femmes ; il n'aura égard à aucun dieu ; car il s'élèvera au-dessus de tout.
\VS{38}Mais, à la place, il honorera le dieu Mahuzzim ; ce dieu que ses pères n'ont pas connu, il rendra des hommages avec de l'or et de l'argent, et des pierres précieuses, et des objets de prix.
\VS{39}C'est avec le dieu étranger qu'il agira contre les lieux les plus fortifiés ; et il comblera d'honneurs ceux qui le reconnaîtront, il les fera dominer sur plusieurs, il leur partagera des terres à prix d'argent.
\VS{40}Au temps de la fin, le roi du midi se heurtera contre lui de ses cornes. Et le roi du nord fondra sur lui comme une tempête, avec des chars et des cavaliers, et avec de nombreux navires ; il s'avancera dans les terres, se répandra comme un torrent et débordera.
\VS{41}Il entrera dans le pays de noblesse, et plusieurs pays succomberont ; mais Edom, Moab et les principaux des enfants d'Ammon seront délivrés de sa main.
\VS{42}Il étendra sa main sur ces pays-là, et le pays d'Egypte n'échappera point.
\VS{43}Il se rendra maître des trésors d'or et d'argent, et de toutes les choses précieuses de l'Egypte ; les Libyens et les Ethiopiens seront à sa suite.
\VS{44}Mais des nouvelles de l'orient et du nord viendront le troubler, et il partira avec une grande fureur, pour détruire et exterminer beaucoup de gens.
\VS{45}Il dressera les tentes de son palais entre les mers, vers la glorieuse et sainte montagne. Puis il arrivera à la fin, et personne ne lui donnera du secours.
\Chap{12}
\TextTitle{La résurrection pour le jugement éternel}
\VerseOne{}Or, en ce temps-là Michaël, ce grand Chef qui tient ferme pour les enfants de ton peuple, tiendra ferme ; et ce sera un temps de détresse, tel qu'il n'y en a point eu de semblable depuis que les nations existent jusqu'à ce temps-là. En ce temps-là, ceux de ton peuple qui seront trouvés inscrits dans le livre seront sauvés.
\TextTitle{Les deux résurrections}
\VS{2}Plusieurs de ceux qui dorment dans la poussière de la terre se réveilleront\FTNT{Il est question ici de la résurrection. Tout d'abord, il y aura la résurrection des morts en Christ, lors du retour de Jésus-Christ (1 Th. 4:12-17). Ensuite, il y aura celle de tous les saints lors du retour de Christ avec l'Eglise (Ap. 19 et 20). Enfin, la dernière résurrection interviendra à l'issue du millénium. Il s'agit de la résurrection des impies (Ap. 20:11-15). Voir également Jn. 5 : 24-29 ; Jn. 11:25.}, les uns pour la vie éternelle, et les autres pour l'opprobre, pour l'infamie éternelle.
\VS{3}Ceux qui auront été intelligents, brilleront comme la splendeur du ciel, et ceux qui auront amené plusieurs à la justice brilleront comme les étoiles, à toujours et à perpétuité\FTNT{Mt. 13:43.}.
\TextTitle{Dernières paroles de Yahweh à Daniel ; le livre scellé jusqu'au temps de la fin}
\VS{4}Mais toi, Daniel, tiens secrètes ces paroles, et scelle le livre jusqu'au temps de la fin. Plusieurs le liront et la connaissance augmentera\FTNT{Ap. 10:4 ; Ap. 5:2.}.
\VS{5}Et moi, Daniel, je regardai, et voici, deux autres hommes se tenaient debout, l'un en deçà du bord du fleuve, et l'autre au-delà du bord du fleuve.
\VS{6}L'un d'eux dit à l'homme vêtu de lin, qui se tenait au-dessus des eaux du fleuve : Quand sera la fin de ces merveilles ?
\VS{7}Et j'entendis l'homme vêtu de lin, qui se tenait au-dessus des eaux du fleuve ; il leva sa main droite et sa main gauche vers les cieux, et il jura par celui qui vit éternellement que ce sera dans un temps, des temps, et la moitié d'un temps, et que toutes ces choses finiront quand la force du peuple saint sera entièrement brisée.
\VS{8}J'entendis, mais je ne compris pas ; et je dis : Mon seigneur, quelle sera l'issue de ces choses ?
\VS{9}Il répondit : Va, Daniel, car ces paroles sont tenues secrètes et scellées jusqu'au temps de la fin.
\VS{10}Plusieurs seront purifiés, blanchis et éprouvés ; mais les méchants agiront avec méchanceté, et aucun des méchants ne comprendra, mais les sages comprendront.
\VS{11}Depuis le temps où cessera le sacrifice perpétuel et où sera dressée l'abomination de la désolation, il y aura mille deux cent quatre-vingt-dix jours\FTNT{Mt. 24:15 ; Mc. 13:14 ; Lu. 21:20.}.
\VS{12}Heureux celui qui attendra et qui parviendra jusqu'à mille trois cent trente-cinq jours.
\VS{13}Mais toi, marche vers ta fin ; néanmoins tu te reposeras, et tu seras debout pour ton héritage à la fin des jours.
\PPE{}
\end{multicols}

%\clearpage\ShortTitle{Esdras}\BookTitle{Esdras}\BFont
\noindent\hrulefill
{\footnotesize
\textit{
\bigskip
{\centering{}
\\Auteur : Esdras
\\(Heb : Ezrah)
\\Signification : Secours
\\Thème : Edit de Cyrus et reconstruction du temple 
\\Date de rédaction : 5ème siècle av. J.-C.\\}
}
%\bigskip
\textit{
\\Conformément aux prophéties reçues par Esaïe et Jérémie, Yahweh toucha le cœur du roi Cyrus afin de renvoyer les fils d’Israël sur leur terre avec la mission de reconstruire le temple détruit quelques décennies auparavant. Ce livre montre comment Dieu ramena glorieusement son peuple à Jérusalem et retrace la reconstruction du temple ainsi que les épreuves ayant accompagné ce projet. Il traite des réformes sociales et religieuses mises en place dans le cadre d’un retour total à Yahweh.\bigskip
}
}
\par\nobreak\noindent\hrulefill
\begin{multicols}{2}
\Chap{1}
\TextTitle{Publication de Cyrus}
\VerseOne{}La première année de Cyrus\FTNT{538 av. J.-C.}, roi de Perse, afin que la parole de Yahweh prononcée par la bouche de Jérémie\FTNT{Jé. 25:12 ; 29:10 ; 33:7-10.} soit accomplie,  Yahweh réveilla l'esprit de Cyrus, roi de Perse, qui fit publier par écrit et de vive voix dans tout son royaume, en disant :
\VS{2}Ainsi parle Cyrus, roi de Perse : Yahweh, le Dieu des cieux, m'a donné tous les royaumes de la terre, et il m'a ordonné de lui bâtir une maison à Jérusalem, en Juda.
\VS{3}Qui d'entre vous est de son peuple, qui veut s'y employer ? Que son Dieu soit avec lui, qu'il monte à Jérusalem, en Juda, et qu'il rebâtisse la maison de Yahweh, le Dieu d'Israël ! C'est le Dieu qui est à Jérusalem.
\VS{4}Dans tout lieu où séjournent des restes du peuple,  les gens du lieu leur donneront de l’argent, de l’or, des biens, et du bétail, avec des offrandes volontaires pour la maison du Dieu qui est à Jérusalem.
\TextTitle{Cyrus rend les ustensiles}
\VS{5}Alors les chefs des familles de Juda et de Benjamin, les sacrificateurs et les Lévites, tous ceux dont Dieu réveilla l'esprit, se levèrent afin de monter pour rebâtir la maison de Yahweh à Jérusalem.
\VS{6}Tous ceux qui étaient autour d'eux les encouragèrent, leur fournissant des objets d'argent, d'or, des biens, du bétail, et des choses précieuses, outre toutes les offrandes volontaires.
\VS{7}Le roi Cyrus prit les ustensiles de la maison de Yahweh, que Nebucadnetsar avait emportés\FTNT{2 R. 24:13 ; 2 Ch 36:7.} de Jérusalem et mis dans la maison de son dieu.
\VS{8}Cyrus, roi de Perse, les fit sortir par Mithredath, le trésorier, qui les remit à Scheschbatsar, prince de Juda.
\VS{9}Et voici leur nombre : Trente bassins d'or, mille bassins d'argent, vingt-neuf couteaux,
\VS{10}trente coupes d'or, quatre cent dix coupes d'argent de second ordre, et d'autres ustensiles par milliers.
\VS{11}Tous les ustensiles d'or et d'argent étaient de cinq mille quatre cents. Scheschbatsar emporta le tout, lorsqu’on fit remonter de Babylone à Jérusalem ceux de la captivité.
\Chap{2}
\TextTitle{Dénombrement des Israélites revenus de captivité}
\VerseOne{}Voici ceux de la province qui revinrent de la captivité, d'entre ceux que Nebucadnetsar, roi de Babylone, avait transportés en exil à Babylone, et qui retournèrent à Jérusalem, et en Juda ; chacun dans sa ville\FTNT{Esd. 5:8 ; Né. 1:3 ; Né. 7:6.}.
\VS{2}Ils vinrent avec Zorobabel, Josué, Néhémie, Seraja, Reélaja, Mardochée, Bilschan, Mispar, Bigvaï, Rehum, Baana. Nombre des hommes du peuple d'Israël :
\VS{3}Les fils de Pareosch, deux mille cent soixante-douze\FTNT{Né. 7:8.} ;
\VS{4}les fils de Schephathia, trois cent soixante-douze ;
\VS{5}les fils d'Arach, sept cent soixante-quinze ;
\VS{6}les fils de Pachath-Moab, des fils de Josué, de Joab, deux mille huit cent douze ;
\VS{7}les fils d'Elam, mille deux cent cinquante-quatre ;
\VS{8}les fils de Zatthu, neuf cent quarante-cinq ;
\VS{9}les fils de Zaccaï, sept cent soixante ;
\VS{10}les fils de Bani, six cent quarante-deux\FTNT{Né. 7:15.} ;
\VS{11}les fils de Bébaï, six cent vingt-trois ;
\VS{12}les fils d'Hazgad, mille deux cent vingt-deux ;
\VS{13}les fils d'Adonikam, six cent soixante-six ;
\VS{14}les fils de Bigvaï, deux mille cinquante-six ;
\VS{15}les fils d’Adin, quatre cent cinquante-quatre ;
\VS{16}les fils d'Ather, de la famille d'Ezéchias, quatre-vingt-dix-huit ;
\VS{17}les fils de Betsaï, trois cent vingt-trois ;
\VS{18}les fils de Jora, cent douze ;
\VS{19}les fils de Haschum, deux cent vingt-trois ;
\VS{20}les fils de Guibbar, quatre-vingt-quinze ;
\VS{21}les fils de Bethléhem, cent vingt-trois ;
\VS{22}les gens de Netopha, cinquante-six ;
\VS{23}les gens d'Anathoth, cent vingt-huit ;
\VS{24}les fils d'Azmaveth, quarante-deux ;
\VS{25}les fils de Kirjath-Arim, de Kephira, et de Beéroth, sept cent quarante-trois ;
\VS{26}les fils de Rama et de Guéba, six cent vingt et un ;
\VS{27}les gens de Micmas, cent vingt-deux ;
\VS{28}les gens de Béthel et d’Aï, deux cent vingt-trois ;
\VS{29}les fils de Nebo, cinquante-deux ;
\VS{30}les fils de Magbisch, cent cinquante-six.
\VS{31}les fils d'un autre Elam, mille deux cent cinquante-quatre ;
\VS{32}les fils de Harim, trois cent vingt ;
\VS{33}les fils de Lod, de Hadid, et d'Ono, sept cent vingt-cinq ;
\VS{34}les fils de Jéricho, trois cent quarante-cinq ;
\VS{35}les fils de Senaa, trois mille six cent trente.
\TextTitle{Dénombrement des sacrificateurs revenus de captivité}
\VS{36}Des sacrificateurs : Les fils de Jedaeja, de la maison de Josué, neuf cent soixante-treize ;
\VS{37}les fils d'Immer, mille cinquante-deux ;
\VS{38}les fils de Paschhur, mille deux cent quarante-sept ;
\VS{39}les fils de Harim, mille dix-sept.
\TextTitle{Dénombrement des Lévites revenus de captivité}
\VS{40}Des Lévites : Les fils de Josué et de Kadmiel, d'entre les fils d’Hodavia, soixante-quatorze.
\VS{41}Des chantres : Les fils d'Asaph, cent vingt-huit.
\VS{42}Des fils des portiers : Les fils de Schallum, les fils d'Ather, les fils de Thalmon, les fils d’Akkub, les fils de Hathitha, les fils de Schobaï, en tout cent trente-neuf.
\VS{43}Des Néthiniens : Les fils de Tsicha, les fils de Hasupha, les fils de Tabbahoth\FTNT{Esd.8:17 ; Jos. 9:23.},
\VS{44}les fils de Kéros, les fils de Siaha, les fils de Padon,
\VS{45}les fils de Lebana, les fils de Hagaba, les fils d'Akkub,
\VS{46}les fils de Hagab, les fils de Schamlaï, les fils de Hanan,
\VS{47}les fils de Guiddel, les fils de Gachar, les fils de Reaja,
\VS{48}les fils de Retsin, les fils de Nekoda, les fils de Gazzam,
\VS{49}les fils d'Uzza, les fils de Paséach, les fils de Bésaï,
\VS{50}les fils d'Asna, les fils de Mehunim, les fils de Nephusim,
\VS{51}les fils de Bakbuk, les fils de Hakupha, les fils de Harhur,
\VS{52}les fils de Batsluth, les fils de Mehida, les fils de Harscha,
\VS{53}les fils de Barkos, les fils de Sisera, les fils de Thamach,
\VS{54}les fils de Netsiach, les fils de Hathipha.
\TextTitle{Dénombrement des serviteurs de Salomon revenus de captivité}
\VS{55}Des fils des serviteurs de Salomon : Les fils de Sothaï, les fils de Sophéreth, les fils de Peruda,
\VS{56}les fils de Jaala, les fils de Darkon, les fils de Guiddel,
\VS{57}les fils de Schephathia, les fils de Hatthil, les fils de Pokéreth-Hatsebaïm, les fils d'Ami.
\VS{58}Total des Néthiniens et des fils des serviteurs de Salomon : Trois cent quatre-vingt-douze.
\VS{59}Voici ceux qui montèrent de Thel-Mélach, de Thel-Harscha, de Kerub-Addan et qui ne purent pas faire connaître leur maison paternelle et leur race, pour prouver qu’ils étaient d'Israël :
\VS{60}Les fils de Delaja, les fils de Tobija, les fils de Nekoda, six cent cinquante-deux.
\TextTitle{Certains sacrificateurs rejetés de la sacrificature}
\VS{61}Des fils des sacrificateurs : Les fils de Habaja, les fils d'Hakkots, les fils de Barzillaï, qui avait pris pour femme une des filles de Barzillaï, le Galaadite, fut appelé de leur nom.
\VS{62}Ils cherchèrent leurs registres généalogiques, mais ils ne les trouvèrent point. C'est pourquoi ils furent rejetés pour ne pas souiller le sacerdoce,
\VS{63}et le gouverneur leur dit de ne pas manger des choses très saintes, en attendant qu'un sacrificateur ait consulté l'urim et le thummim.
\TextTitle{Nombre total des Israélites revenus de captivité}
\VS{64}L’assemblée tout entière était de quarante-deux mille trois cent soixante,
\VS{65}sans leurs serviteurs et leurs servantes, qui étaient sept mille trois cent trente-sept. Ils avaient deux cents chantres ou chanteuses.
\VS{66}Ils avaient sept cent trente-six chevaux, et deux cent quarante-cinq mulets,
\VS{67}quatre cent trente-cinq chameaux, et six mille sept cent vingt ânes.
\VS{68}Quelques-uns d'entre les chefs des pères, quand ils vinrent à la maison de Yahweh à Jérusalem, firent des offrandes volontaires pour la maison de Dieu, afin qu'on la rétablît sur son emplacement.
\VS{69}Ils donnèrent au trésor de l'ouvrage, selon leurs moyens, soixante et un mille drachmes d'or, et cinq mille mines d'argent, et cent tuniques de sacrificateurs.
\VS{70}Ainsi, les sacrificateurs, les Lévites, quelques-uns du peuple, les chantres, les portiers et les Néthiniens habitèrent dans leurs villes. Et tous ceux d'Israël dans leurs villes aussi.
\Chap{3}
\TextTitle{Rétablissement de l'autel et des sacrifices}
\VerseOne{}Le septième mois approcha, et les fils d'Israël étaient dans leurs villes. Le peuple s'assembla alors comme un seul homme à Jérusalem.
\VS{2}Alors\FTNT{Ag. 1:1 ; De.12:5-6.} Josué, fils de Jotsadak, avec ses frères les sacrificateurs, et Zorobabel, fils de Schealthiel, avec ses frères, se levèrent et bâtirent l'autel du Dieu d'Israël, pour y offrir des holocaustes, comme il est écrit dans la loi de Moïse, homme de Dieu.
\VS{3}Ils rétablirent l'autel de Dieu sur ses fondements, parce qu'ils avaient peur en eux-mêmes des peuples du pays, et ils y offrirent des holocaustes à Yahweh, les holocaustes du matin et du soir\FTNT{No. 28:3.}.
\VS{4}Ils célébrèrent aussi la fête des tabernacles, comme il est écrit, et ils offrirent des holocaustes, autant qu'il en fallait  chaque jour\FTNT{Lé. 23:34 ; No. 29:12.}.
\VS{5}Après cela, ils offrirent l'holocauste perpétuel, ceux des nouvelles lunes, de toutes les fêtes solennelles consacrées à Yahweh, et ceux de quiconque faisait des offrandes volontaires à Yahweh\FTNT{No. 28:11 ; Né.10:33.}.
\VS{6}Dès le premier jour du septième mois, ils commencèrent à offrir des holocaustes à Yahweh. Cependant, les fondements du temple de Yahweh n'étaient pas encore posés.
\VS{7}Ils donnèrent de l'argent aux tailleurs de pierres et aux charpentiers, et aussi de la nourriture, des boissons, de l'huile aux Sidoniens et aux Tyriens, afin qu'ils amènent du bois de cèdre du Liban par la mer de Japho, selon la permission que Cyrus, roi de Perse, leur en avait donnée.
\TextTitle{Les fondements du temple posés}
\VS{8}Et la deuxième année depuis leur arrivée à la maison de Dieu à Jérusalem, au deuxième mois, Zorobabel, fils de Schealthiel,  Josué, fils de Jotsadak, et le reste de leurs frères les sacrificateurs et les Lévites, et tous ceux qui étaient revenus de la captivité à Jérusalem, débutèrent l’œuvre et désignèrent des Lévites, depuis l'âge de vingt ans et au-dessus pour surveiller l'ouvrage de la maison de Yahweh.
\VS{9}Et Josué, avec ses fils et ses frères, Kadmiel, avec ses fils, fils de Juda, les fils de Hénadad, avec leurs fils et leurs frères les Lévites, se tenaient debout pour surveiller ceux qui faisaient l'ouvrage de la maison de Dieu.
\VS{10}Et lorsque ceux qui bâtissaient posèrent les fondements du temple de Yahweh, on fit assister les sacrificateurs revêtus de leurs habits, avec leurs trompettes, et les Lévites, fils d'Asaph, avec les cymbales, pour qu’ils célèbrent Yahweh, selon l’institution de David, roi d'Israël.
\VS{11}Et en louant et célébrant Yahweh, ils s'entre-répondaient : Il est bon, parce que sa miséricorde demeure à toujours sur Israël ! Et tout le peuple poussait de grands cris de joie en louant Yahweh, parce qu'on posait les fondements de la maison de Yahweh.
\VS{12}Mais plusieurs des sacrificateurs et des Lévites, et des chefs de familles âgés, qui avaient vu la première maison, pleuraient à grand bruit pendant qu'on posait sous leurs yeux les fondements de cette maison. Et beaucoup élevaient leur voix avec des cris de joie,
\VS{13}et le peuple ne pouvait distinguer le bruit des cris de joie  d'avec le bruit des pleurs du peuple, car le peuple poussait de grands cris de joie dont le son s’entendait de très loin.
\Chap{4}
\TextTitle{Les ennemis de Juda et de Benjamin découragent le peuple de Juda}
\VerseOne{}Les ennemis de Juda et de Benjamin entendirent que les fils de la captivité rebâtissaient un temple à Yahweh, le Dieu d'Israël.
\VS{2}Ils vinrent vers Zorobabel et vers les chefs des familles, et leur dirent : Nous bâtirons\FTNT{On ne doit jamais s’associer avec les impies pour bâtir l’œuvre du Seigneur. Satan essaie toujours de s’infiltrer dans les assemblées afin de nous éloigner de la vérité, c’est pour cela que nous devons faire preuve de discernement (2 Co. 6:14-16).} avec vous ; car nous invoquons votre Dieu comme vous ; et nous lui avons sacrifié depuis le temps d'Esar-Haddon, roi d'Assyrie, qui nous a fait monter ici.
\VS{3}Mais Zorobabel, Josué, et les autres chefs des familles d'Israël, leur répondirent : Il ne convient pas à vous de bâtir la maison de notre Dieu ; mais nous, qui sommes ici ensemble, nous la bâtirons à Yahweh, le Dieu d'Israël, comme nous l'a ordonné le roi Cyrus, roi de Perse\FTNT{Esd. 1:1,2,5.}.
\VS{4}Alors les gens du pays rendirent paresseuses les mains du peuple de Juda ; ils l’intimidèrent pour l'empêcher de bâtir,
\VS{5}ils avaient même engagé à prix d’argent des conseillers pour faire échouer leur projet, pendant toute la durée de vie de Cyrus, roi de Perse, jusqu'au règne de Darius, roi de Perse.
\VS{6}Et sous le règne d'Assuérus, au commencement de son règne, ils écrivirent une accusation contre les habitants de Juda et de Jérusalem.
\TextTitle{Lettre envoyé à Artaxerxès}
\VS{7}Et du temps d'Artaxerxès, Bischlam, Mithredath, Thabeel, et le reste de leurs collègues, écrivirent à Artaxerxès, roi de Perse. La lettre était écrite en caractères araméens, et elle était traduite en araméen.
\VS{8}Rehum, le gouverneur, et Schimschaï, le secrétaire, écrivirent au roi Artaxerxès la lettre suivante concernant Jérusalem :
\VS{9}Rehum, gouverneur, Schimschaï, secrétaire, et le reste de leurs collègues, ceux de Din, d'Arpharsathac, de Tharpel, d'Apharas, d'Erec, de Babylone, de Suse, de Déha, d'Elam,
\VS{10}et les autres peuples que le grand et illustre Osnappar a transportés et fait habiter dans la ville de Samarie et les autres régions au-delà du fleuve, à cette date.
\VS{11}Voici donc ici la copie de la lettre qu'ils envoyèrent au roi Artaxerxès : Tes serviteurs, les gens de ce côté du fleuve, à cette date.
\VS{12}Que le roi sache que les Juifs qui sont montés de chez lui et arrivés vers nous à Jérusalem rebâtissent la ville rebelle et méchante, et achèvent en finissant de poser et de réparer les fondements des murs.
\VS{13}Que le roi sache donc que si cette ville est rebâtie et si ses murs sont réparés, ils ne paieront plus de tribut, ni d’impôt, ni de droit de passage, et elle causera une grande nuisance aux revenus du roi.
\VS{14}Et parce que nous mangeons le sel du palais, il ne nous parait pas convenable de voir le roi déshonoré ; c'est pourquoi nous envoyons au roi ces informations.
\VS{15}Qu'on recherche dans le livre des mémoires de tes pères, et tu trouveras et tu apprendras dans ce livre des mémoires que cette ville est une ville rebelle, nuisible aux rois et aux provinces ; et qu’on s’y est livré à la révolte depuis toujours. Donc cette ville a été détruite à cause de cela.
\VS{16}Nous faisons donc savoir au roi que si cette ville est rebâtie et si ses murs sont relevés, il n'aura plus de possession de ce côté du fleuve.
\TextTitle{Réponse du roi Artaxerxès}
\VS{17}Et c'est ici le décret envoyé par le roi à Rehum, le gouverneur, à Schimschaï, le secrétaire, et au reste de leurs collègues demeurant à Samarie, et aux autres de l'autre côté du fleuve : Paix sur vous, à cette date.
\VS{18}La lettre que vous nous avez envoyée a été lue exactement en ma présence.
\VS{19}J'ai donné ordre de faire des recherches et l’on a trouvé  que depuis toujours cette ville s'est soulevée contre les rois, et qu'on s’y est livré à la sédition et à la révolte.
\VS{20}Il y eut aussi à Jérusalem des rois puissants, maîtres de tout le pays de l'autre côté du fleuve, et auxquels on payait  tribut, impôt et droit de passage\FTNT{2 S. 8:2,6 ; 1 R 4:21 ; 2 Ch. 17:11 ; 32:23.}.
\VS{21}A présent, donnez l’ordre de ne pas laisser continuer ces gens-là, afin que cette ville ne se rebâtisse point, jusqu’à ce que je l'ordonne par décret.
\VS{22}Gardez-vous de mettre en cela de la négligence, de peur que le mal  n’augmente au préjudice des rois.
\VS{23}Aussitôt que la copie de la lettre du roi Artaxerxès eut été lue en présence de Rehum, de Schimschaï,  le secrétaire, et de leurs collègues, ils allèrent en hâte à Jérusalem vers les Juifs, et ils les firent cesser leurs travaux avec violence et force.
\VS{24}Alors l’ouvrage de la maison de Dieu, à Jérusalem, cessa, et elle demeura dans cet état, jusqu'à la deuxième année du règne de Darius, roi de Perse\FTNT{Esd. 5:2}.
\Chap{5}
\TextTitle{Aggée et Zacharie prophétisent}
\VerseOne{}Aggée, le prophète, et Zacharie, fils d'Iddo, le prophète, prophétisèrent aux Juifs qui étaient en Juda et à Jérusalem, au nom du Dieu d'Israël, qui s’adressait à eux\FTNT{Ag. 1:4 ; Za. 1:1.}.
\VS{2}Alors Zorobabel, fils de Schealthiel, et Josué, fils de Jotsadak, se levèrent et commencèrent à rebâtir la maison de Dieu à Jérusalem. Et ils avaient avec eux les prophètes de Dieu qui les soutenaient\FTNT{Ag. 1:14 ; Esd. 6:14.}.
\VS{3}En ce temps-là, Thathnaï, gouverneur de ce côté du fleuve, et Schethar-Boznaï, et leurs collègues, vinrent à eux et leur parlèrent ainsi : Qui vous a donné l’ordre de rebâtir cette maison et de relever ces murs ?\FTNT{Esd. 5:9.}
\VS{4}Ils leur dirent alors : Quels sont les noms des hommes qui construisent cet édifice ?
\VS{5}Mais l’œil de Dieu était sur les anciens des Juifs. Et on ne les laissa continuer les travaux, pendant l’envoi d’un rapport à Darius, et jusqu'à la réception d’une lettre sur cet objet.
\TextTitle{Thathnaï, Schethar-Boznaï et leurs collègues d'Apharsac écrivent à Darius}
\VS{6}Copie de la lettre envoyée au roi Darius par Thathnaï, gouverneur de ce côté du fleuve, Schethar-Boznaï, et leurs collègues d'Apharsac, de l’autre côté du fleuve.
\VS{7}Ils lui envoyèrent un rapport ainsi écrit : Paix parfaite soit au roi Darius !
\VS{8}Que le roi sache que nous sommes allés dans la province de Juda, vers la maison du grand Dieu. Elle se bâtit avec des pierres de taille, et le bois se pose dans les murs ; ce travail se réalise complètement et prospère entre leurs mains\FTNT{Esd. 2:1.}.
\VS{9}Nous avons interrogé ces anciens, et nous leur avons parlé ainsi : Qui vous a donné l'autorisation de rebâtir cette maison et de finir ces murs ?\FTNT{Esd. 5:3.}
\VS{10}Nous leur avons aussi demandé leurs noms pour te les faire connaître, et nous avons mis par écrit les noms des hommes à leur tête.
\VS{11}Et ils nous ont répondu de cette manière, disant : Nous sommes les serviteurs du Dieu des cieux et de la terre, et nous rebâtissons la maison qui avait été bâtie il y a de nombreuses années ; un grand roi d'Israël l’avait bâtie et finie.
\VS{12}Mais après que nos pères eurent provoqué la colère du Dieu des cieux, il les livra entre les mains de Nebucadnetsar\FTNT{Voir 2 R. 24 et 25.}, roi de Babylone, Chaldéen, qui détruisit cette maison et qui emmena le peuple en exil à Babylone\FTNT{2 Ch. 36:7}.
\VS{13}Mais la première année de Cyrus, roi de Babylone, le roi Cyrus prit un décret pour rebâtir cette maison de Dieu\FTNT{Esd. 1:1-2.}.
\VS{14}Et même le roi Cyrus ôta du temple de Babylone les ustensiles d'or et d'argent de la maison de Dieu, que Nebucadnetsar avait sortis du temple qui était à Jérusalem et transportés dans le temple de Babylone, et il les fit remettre au nommé Scheschbatsar, qu’il établit gouverneur\FTNT{Esd. 1:8.},
\VS{15}et il lui dit : Prends ces ustensiles, et va les déposer dans le temple de Jérusalem ; et que la maison de Dieu soit rebâtie sur sa place.
\VS{16}Alors ce Scheschbatsar est venu, et il a posé les fondements de la maison de Dieu à Jérusalem ; et depuis ce temps-là jusqu'à présent, on la bâtit, et elle n'est point encore achevée.
\VS{17}Maintenant, s'il semble bon au roi, que l’on fasse des recherches dans la maison des trésors du roi à Babylone, pour voir s'il est vrai qu'il y a eu un ordre donné par Cyrus de rebâtir cette  maison de Dieu à Jérusalem. Puis, que le roi nous transmette sa volonté sur cet objet.
\Chap{6}
\TextTitle{Darius confirme l'édit de Cyrus}
\VerseOne{}Alors le roi Darius donna un ordre de faire des recherches dans la maison des livres où l'on déposait les  trésors à Babylone.
\VS{2}Et l’on trouva à Achmetha, dans un coffre, capitale de la province de Médie, un rouleau à l’intérieur duquel était écrit  le mémoire suivant :
\VS{3}La première année du roi Cyrus, le roi Cyrus prit un décret quant à la maison de Dieu à Jérusalem : Que cette maison soit rebâtie, afin d’être un lieu où l'on offre des sacrifices, et que ses fondements soient solides pour porter sa charge. La hauteur sera de soixante coudées, et la longueur de soixante coudées,
\VS{4}trois rangées de pierres de taille et une rangée de bois neuf.  La dépense sera payée par la maison du roi.
\VS{5}Aussi, les ustensiles d'or et d'argent de la maison de Dieu, que Nebucadnetsar avait enlevés du temple de Jérusalem et apportés à Babylone, seront remis et apportés dans le temple de Jérusalem, à leur place, et déposés dans la maison de Dieu.
\VS{6}Maintenant, Thathnaï, gouverneur de l'autre côté du fleuve, Schethar-Boznaï, et vos collègues d'Apharsac de l'autre côté du fleuve, tenez-vous loin de ce lieu.
\VS{7}Laissez le travail de cette maison de Dieu ; que le gouverneur des Juifs et les anciens des Juifs rebâtissent cette maison de Dieu à sa place.
\VS{8}En raison de ce décret pris, ce que vous aurez à exécuter, avec les anciens de ces Juifs pour rebâtir cette maison de Dieu : Sur les finances du roi provenant du tribut de l’autre côté du fleuve, les frais seront complètement payés à ces hommes, afin qu'il n'y ait pas d'interruption.
\VS{9}Et ce qui sera nécessaire pour les holocaustes du Dieu des cieux, veaux, béliers et agneaux, blé, sel, vin et huile, seront livrés, sur leur demande, aux sacrificateurs de Jérusalem, jour après jour, sans négligence,
\VS{10}afin qu'ils offrent des sacrifices de bonne odeur au Dieu des cieux et qu'ils prient pour la vie du roi et de ses fils.
\VS{11}Et voici l’ordre que je donne touchant quiconque changera cette parole : On arrachera de sa maison une pièce de bois, on la dressera, afin qu'il y soit exterminé, et l’on fera de sa maison un tas de déchets\FTNT{2 R. 10:27 ; Ez. 6:11 ; Da. 3:29.}.
\VS{12}Et que Dieu, qui fait résider en ce lieu son nom, renverse tout roi et tout peuple qui étendrait sa main pour changer et détruire cette maison de Dieu à Jérusalem ! Moi, Darius, j’ai donné cet ordre. Qu'il soit donc exécuté complètement.
\TextTitle{Achèvement et dédicace de la maison de Dieu}
\VS{13}Alors Thathnaï, gouverneur de l'autre côté du fleuve,  Schethar-Boznaï, et leurs collègues, firent exécuter ainsi complètement ce que le roi Darius leur envoya.
\VS{14}Et les anciens des Juifs bâtirent avec succès, selon les prophéties d'Aggée, le prophète, et de Zacharie, fils d’Iddo ; ils bâtirent et finirent, d'après l'ordre du Dieu d'Israël, et d'après l'ordre de Cyrus, de Darius, et d'Artaxerxès, roi de Perse.
\VS{15}Cette maison fut achevée le troisième jour du mois d'Adar, dans la sixième année du règne du roi Darius.
\VS{16}Les fils d'Israël, les sacrificateurs, les Lévites, et le reste des fils de la captivité, célébrèrent la dédicace de cette maison de Dieu avec joie.
\VS{17}Ils offrirent pour la dédicace de cette maison de Dieu, cent taureaux, deux cents béliers, quatre cents agneaux, et douze boucs comme victimes expiatoires pour tout Israël, selon le nombre des tribus d'Israël.
\VS{18}Ils établirent les sacrificateurs selon leurs classes et les Lévites selon leurs divisions, pour le service de Dieu à Jérusalem,  selon ce qui est écrit dans le livre de Moïse\FTNT{No. 3:6,32 ; No. 8:11}.
\TextTitle{Rétablissement de la Pâque}
\VS{19}Puis les fils de la captivité célébrèrent la Pâque le quatorzième jour du premier mois\FTNT{Lé. 23:5 ; No. 28:16 ; De. 16:2.}.
\VS{20}Les sacrificateurs et les Lévites s'étaient purifiés comme un seul homme, tous étaient purs ; c'est pourquoi ils immolèrent la Pâque pour tous les fils de la captivité, pour leurs frères les sacrificateurs, et pour eux-mêmes\FTNT{2 Ch. 30:15,17,21}.
\VS{21}Les fils d'Israël revenus de la captivité mangèrent la Pâque, avec tous ceux qui s'étaient séparés de l’impureté des nations du pays pour chercher Yahweh, le Dieu d'Israël.
\VS{22}Ils célébrèrent avec joie la fête des pains sans levain pendant sept jours, car Yahweh les avait réjouis en disposant  le cœur du roi d'Assyrie à fortifier leurs mains dans l’œuvre de la maison de Dieu, du Dieu d'Israël.
\Chap{7}
\TextTitle{Voyage d'Esdras jusqu'à Jérusalem}
\VerseOne{}Après ces choses, sous le règne d'Artaxerxès, roi de Perse, Esdras, fils de Seraja, fils d'Azaria, fils de Hilkija\FTNT{Esd. 6:14.},
\VS{2}fils de Schallum, fils de Tsadok, fils d'Achithub,
\VS{3}fils d'Amaria, fils d'Azaria, fils de Merajoth,
\VS{4}fils de Zerachja, fils d'Uzzi, fils de Bukki,
\VS{5}fils d'Abischua, fils de Phinées, fils d'Eléazar, fils d'Aaron, souverain sacrificateur.
\VS{6}Esdras monta de Babylone : C’était un scribe bien exercé dans la loi de Moïse, donnée par Yahweh, le Dieu d'Israël. Et comme la main de Yahweh, son Dieu, était sur lui, le roi lui accorda toute sa requête\FTNT{Vers. 9,28.}.
\VS{7}Des fils d'Israël, des sacrificateurs, des Lévites, des chantres, des portiers, et des Néthiniens, montèrent à Jérusalem, la septième année du roi Artaxerxès.
\VS{8}Il entra à Jérusalem le cinquième mois de la septième année du roi ;
\VS{9}il était parti de Babylone au premier jour du premier mois, et il entra à Jérusalem au premier jour du cinquième mois, selon que la main de son Dieu était bonne sur lui.
\VS{10}Car Esdras avait disposé son cœur à étudier la loi de Yahweh, à l’observer et à enseigner les lois et les ordonnances parmi le peuple d'Israël.
\TextTitle{Lettre d'Artaxerxès à Esdras}
\VS{11}Voici la copie de la lettre que le roi Artaxerxès donna à Esdras, sacrificateur et scribe, enseignant les paroles des commandements de Yahweh et ses ordonnances concernant Israël :
\VS{12}Artaxerxès, roi des rois, à Esdras, sacrificateur et scribe de la loi du Dieu des cieux, à cette date.
\VS{13}J’ai donné ordre de laisser aller tous ceux de mon royaume qui sont du peuple d'Israël, de ses sacrificateurs et Lévites, qui se présenteront volontairement pour aller avec toi à Jérusalem.
\VS{14}Tu es envoyé de la part du roi, et de ses sept conseillers, pour inspecter Juda et Jérusalem touchant la loi de ton Dieu, laquelle est entre tes mains,
\VS{15}et pour porter l'argent et l'or que le roi et ses conseillers ont offert volontairement au Dieu d'Israël, dont la demeure est à Jérusalem\FTNT{Esd. 8:24.},
\VS{16}tout l'argent et l'or que tu trouveras dans toute la province de Babylone, avec les offrandes volontaires du peuple et des sacrificateurs, qu'ils feront volontairement à la maison de leur Dieu à Jérusalem.
\VS{17} C'est pourquoi tu achèteras avec cet argent des taureaux, des béliers, des agneaux, avec leurs offrandes et leurs libations, et tu les offriras sur l'autel de la maison de votre Dieu à Jérusalem.
\VS{18}Vous ferez, selon la volonté de votre Dieu, ce qu'il te semblera bon à toi et à tes frères de faire du reste de l'argent et de l'or.
\VS{19}Et pour ce qui est des ustensiles qui te sont remis pour le service de la maison de ton Dieu, déposes-les en présence du Dieu de Jérusalem.
\VS{20}Quand au reste de ce qui sera nécessaire pour la maison de ton Dieu, autant qu'il t'en faudra employer, tu le prendras de la maison des trésors du roi.
\VS{21}Moi, le roi Artaxerxès, je donne l’ordre à tous les trésoriers qui sont de l'autre côté du fleuve de livrer exactement à Esdras, sacrificateur et scribe de la loi du Dieu des cieux, tout ce qu’il vous demandera,
\VS{22}jusqu'à cent talents d'argent, cent cors de froment, cent baths de vin, cent baths d'huile, et du sel sans nombre.
\VS{23}Que tout ce qui est ordonné par le Dieu des cieux se fasse exactement pour la maison du Dieu des cieux, afin que sa colère ne soit pas sur le royaume, sur le roi et sur ses fils.
\VS{24}Nous vous faisons savoir qu'on ne pourra imposer ni tribut, ni impôt, ni droit de passage sur aucun des sacrificateurs, des Lévites, des chantres, des portiers, des  Néthiniens, et des serviteurs de cette maison de Dieu.
\VS{25}Et toi, Esdras, établis des magistrats et des juges selon la sagesse de ton Dieu que tu possèdes, afin qu'ils rendent justice à tout ce peuple de l'autre côté du fleuve, à tous ceux qui connaissent les lois de ton Dieu ; afin que vous enseigniez celui qui ne les connaît point.
\VS{26}Et tous ceux qui n'observeront point la loi de ton Dieu et la loi du roi seront aussitôt jugés, soit à la mort, soit au bannissement, soit à une amende pécuniaire, ou à l'emprisonnement.
\VS{27}Béni soit Yahweh, le Dieu de nos pères, qui a mis cela au cœur du roi, pour honorer la maison de Yahweh, qui est à Jérusalem ;
\VS{28}et qui a fait que j'ai trouvé grâce devant le  roi, devant ses conseillers, et devant tous les puissants chefs ! Fortifié par la main de Yahweh, mon Dieu, qui était sur moi, j'ai rassemblé les chefs d'Israël, afin qu'ils montent avec moi.
\Chap{8}
\TextTitle{Dénombrement de ceux qui montèrent avec Esdras}
\VerseOne{}Voici les chefs des pères, avec le dénombrement fait selon les généalogies de ceux qui montèrent avec moi de Babylone, pendant le règne du roi Artaxerxès\FTNT{1 Ch. 4:33.}.
\VS{2}Des fils de Phinées, Guerschom ; des fils d'Ithamar, Daniel ; des fils de David, Hattusch ;
\VS{3}des fils de Schecania ; des fils de Pareosch, Zacharie, et avec lui, en faisant le dénombrement par leur généalogie selon les hommes, cent cinquante hommes ;
\VS{4}des fils de Pachat Moab, Eljoénaï, fils de Zerachja, et avec lui deux cents hommes;
\VS{5}des fils de Schecania, le fils de Jachaziel, et avec lui trois cents hommes;
\VS{6}des fils d'Adin, Ebed, fils de Jonathan, et avec lui cinquante hommes ;
\VS{7}des fils d'Elam, Esaïe, fils d'Athalia, et avec lui soixante-dix hommes;
\VS{8}des fils de Schephathia, Zebadia, fils de Micaël, et avec lui quatre-vingts hommes ;
\VS{9}des fils de Joab, Abdias, fils de Jehiel, et avec lui deux cent dix-huit hommes ;
\VS{10}des fils de Schelomith, le fils de Josiphia, et avec lui cent soixante hommes ;
\VS{11}des fils de Bébaï, Zacharie, fils de Bébaï, et avec lui vingt-huit hommes ;
\VS{12}des fils d'Azgad, Jochanan, fils d'Hakkathan, et avec lui cent-dix hommes ;
\VS{13}des fils d'Adonikam, les derniers, dont voici les noms: Eliphélet, Jeïel, et Schemaeja, et avec eux soixante hommes ;
\VS{14}des fils de Bigvaï, Uthaï, Zabbud, et avec eux soixante-dix hommes.
\VS{15}Je les rassemblai près du fleuve qui coule vers Ahava, et nous campâmes là trois jours. Puis je portai mon attention sur  le peuple et les sacrificateurs, et je n'y trouvai aucun des fils de Lévi.
\VS{16}Alors j'envoyai d'entre les chefs Eliézer, Ariel, Schemaeja, Elnathan, Jarib, Elnathan, Nathan, Zacharie et Meschullam, avec les docteurs Jojarib et Elnathan.
\VS{17}Je leur donnai des ordres pour le chef Iddo, demeurant à Casiphia, et je mis dans leur bouche les paroles qu'ils devaient dire à Iddo et à ses frères les Néthiniens, qui étaient à Casiphia, afin qu'ils nous amènent des serviteurs pour la maison de notre Dieu\FTNT{Esd. 2:43.}.
\VS{18}Et comme la bonne main de notre Dieu était sur nous, ils nous amenèrent Schérébia, un homme intelligent, d'entre les fils de Machli, fils de Lévi, fils d'Israël, et avec ses fils et ses frères, au nombre dix-huit\FTNT{Esd. 7:6,9,28.} ;
\VS{19}Haschabia, et avec lui Esaïe, d'entre les fils de Merari, ses frères, et leurs fils, au nombre vingt ;
\VS{20}et des Néthiniens, que David et les chefs du peuple avaient assignés pour le service des Lévites, deux cent vingt Néthiniens, tous désignés par leurs noms\FTNT{Esd. 2:43,58.}.
\TextTitle{Esdras publie un jeûne pour obtenir la protection de Dieu}
\VS{21}Et je publiai là un jeûne près de la rivière d'Ahava, afin de nous humilier devant notre Dieu, le priant de nous donner un heureux voyage, pour nos enfants, et pour tous nos biens.
\VS{22}Car j'aurais eu honte de demander au roi une armée et des cavaliers pour nous soutenir contre des ennemis pendant le chemin ; car nous avions dit au roi : La main de notre Dieu est favorable sur tous ceux qui le cherchent ; mais sa force et sa colère sont contre ceux qui l'abandonnent.
\VS{23}Nous jeûnâmes donc, et nous cherchâmes notre Dieu à cause de cela. Et il se laissa fléchir par nos prières.
\TextTitle{Trésors remis par Esdras entre les mains de douze sacrificateurs}
\VS{24}Alors je mis à part douze chefs des sacrificateurs, Schérébia, Haschabia, et dix de leurs frères.
\VS{25}Je pesai l'argent, l'or et les ustensiles donnés en offrandes pour la maison de notre Dieu par le roi, ses conseillers, ses chefs, et tous ceux d'Israël qu'on avait trouvés\FTNT{Esd. 7:14,15.}.
\VS{26}Je pesai donc, et je remis entre leurs mains six cent cinquante talents d'argent, des ustensiles d'argent pesant cent talents, cent talents d'or,
\VS{27}vingt coupes d'or valant mille drachmes, et deux ustensiles d’un bel airain poli, aussi précieux que de l'or.
\VS{28}Et je leur dis : Vous êtes consacrés à Yahweh ; et les ustensiles sont sanctifiés, et cet argent et cet or sont une offrande volontaire faite à Yahweh, le Dieu de vos pères.
\VS{29}Soyez vigilants et gardez-les, jusqu'à ce que vous les pesiez devant les chefs des sacrificateurs et les Lévites, et devant les chefs des pères d'Israël, à Jérusalem, dans les chambres de la maison de Yahweh.
\VS{30}Les sacrificateurs et les Lévites reçurent le poids de l'argent, de l'or, et des ustensiles, pour les apporter à Jérusalem, dans la maison de notre Dieu.
\TextTitle{Esdras arrive à Jérusalem}
\VS{31}Nous partîmes du fleuve d'Ahava pour aller à Jérusalem, le douzième jour du premier mois. La main de notre Dieu fut sur nous et nous délivra de la main des ennemis et des  embûches sur le chemin.
\VS{32}Puis nous arrivâmes à Jérusalem, et nous nous y reposâmes trois jours.
\VS{33}Le quatrième jour, nous pesâmes l'argent, l'or, et les ustensiles dans la maison de notre Dieu, et nous les remîmes à Merémoth, fils d'Urie, le sacrificateur - il était avec Eléazar, fils de Phinées, et avec eux les Lévites Jozabad, fils de Josué, et Noadia, fils de Binnuï-
\VS{34} selon tout le nombre et le poids de toutes ces choses, et tout le poids fut mis alors par écrit.
\VS{35}Et les fils de la captivité revenus de l’exil offrirent en holocauste au Dieu d'Israël douze taureaux, quatre-vingt-seize béliers, soixante-dix-sept agneaux, et douze boucs comme victimes expiatoires pour tout Israël, le tout en holocauste à Yahweh.
\VS{36}Ils transmirent les ordres du roi entre les mains des satrapes du roi et des gouverneurs qui étaient de ce côté du fleuve, lesquels favorisèrent le peuple et la maison de Dieu.
\Chap{9}
\TextTitle{La désobéissance}
\VerseOne{}Après que ces choses furent terminées, les chefs du peuple s'approchèrent de moi, en disant : Le peuple d'Israël,  les sacrificateurs et les Lévites ne se sont point séparés des peuples de ces pays, quant à leurs abominations, celles des Cananéens, des Héthiens, des Phéréziens, des Jébusiens, des Ammonites, des Moabites, des Egyptiens, et des Amoréens.
\VS{2}Car ils ont pris de leurs filles pour eux et pour leurs fils, et ont mêlé la semence sainte avec les peuples de ces pays ; et des chefs et des magistrats ont été les premiers à commettre ce péché\FTNT{Né. 13:3.}.
\VS{3}Lorsque j'entendis cela, je déchirai mes vêtements et mon manteau, j'arrachai les cheveux de ma tête et ma barbe, et je m'assis tout épouvanté.
\VS{4}Et tous ceux qui tremblaient aux paroles du Dieu d'Israël, s'assemblèrent auprès de moi, à cause de l’infidélité de ceux de la captivité ; et je demeurai assis tout épouvanté jusqu'à l'offrande du soir.
\TextTitle{Prière et confession d'Esdras}
\VS{5}Et au temps de l'offrande du soir, je me levai du sein de mon affliction, et ayant mes vêtements et mon manteau déchirés, je me mis à genoux, et j'étendis mes mains vers Yahweh, mon Dieu,
\VS{6}et je dis : Mon Dieu ! J’ai honte, et je suis trop confus, ô mon Dieu, pour lever ma face vers toi ; car nos iniquités se sont multipliées au-dessus de nos têtes, et notre péché s'est élevé jusqu’aux cieux.
\VS{7}Depuis les jours de nos pères jusqu'à ce jour, nous sommes grandement coupables, et c’est à cause de nos iniquités que nous avons été livrés, nous, nos rois et nos sacrificateurs entre les mains des rois des pays, à l'épée, à la captivité, au pillage, et à la honte, comme il paraît aujourd'hui.
\VS{8}Et cependant Yahweh, notre Dieu, nous a maintenant fait grâce, en épargnant un reste, et il nous a donné un clou dans son saint lieu, afin d'éclaircir nos yeux et nous donner un peu de répit dans notre servitude\FTNT{Es. 22:23.}.
\VS{9}Car nous sommes esclaves, mais notre Dieu ne nous a point abandonnés dans notre servitude. Il a incliné la bienveillance des rois de Perse pour nous accorder de préserver nos vies afin que nous puissions relever la maison de notre Dieu, et rétablir ces lieux en ruines, et pour nous donner une clôture en Juda et à Jérusalem.
\VS{10}Mais maintenant, ô notre Dieu ! Que dirons-nous après ces choses ? Car nous avons abandonné tes commandements,
\VS{11}que tu as ordonnés par tes serviteurs les prophètes, en disant : Le pays dans lequel vous entrez pour le posséder est un pays souillé par les impuretés des peuples de ces pays, à cause des abominations dont ils l'ont rempli d’un bout à l'autre par leurs impuretés\FTNT{Lé. 18:25-27.};
\VS{12}maintenant donc, ne donnez point vos filles à leurs fils, et ne prenez point leurs filles pour vos fils, ne cherchez jamais ni leur bonheur, ni leur paix, ainsi vous deviendrez forts,  vous mangerez les meilleurs productions du pays, et vous le laisserez hériter à vos fils pour toujours\FTNT{De. 7:3.}.
\VS{13}Après toutes les choses qui nous sont arrivées à cause de nos mauvaises actions et des grandes offenses que nous avons commises - quoi que tu ne nous aies pas, ô notre Dieu, punis en proportion de nos péchés et maintenant que  tu nous as conservé ces réchappés ; 
\VS{14}retournerions-nous à violer tes commandements, et à faire alliance avec ces peuples abominables ? Ne serais-tu pas en colère contre nous, jusqu'à nous exterminer, sans aucun reste ni aucun réchappé ?
\VS{15}Yahweh, Dieu d'Israël ! Tu es juste, car nous sommes aujourd'hui un reste de réchappés. Voici, nous sommes devant toi avec nos fautes, ne pouvant subsister à cause d’elles devant ta face.
\Chap{10}
\TextTitle{Confession et séparation}
\VerseOne{}Pendant qu’Esdras priait et faisait cette confession, pleurant et étant prosterné à terre devant la maison de Dieu, une grande multitude d'hommes, de femmes, et d’enfants d'Israël, s'assembla auprès de lui ; et le peuple se lamenta abondamment par des pleurs.
\VS{2}Alors Schecania, fils de Jehiel, d'entre les fils d’Elam, prit la parole, et dit à Esdras : Nous avons péché contre notre Dieu, en nous mariant avec des femmes étrangères d'entre les peuples de ce pays. Mais Israël ne reste pas pour cela sans espérance\FTNT{De. 7:22,23.}.
\VS{3}Faisons maintenant une alliance avec notre Dieu pour le renvoi de toutes ces femmes et de leurs enfants, selon le conseil de mon seigneur et de ceux qui tremblent devant les commandements de notre Dieu. Et qu'il en soit fait selon la loi\FTNT{Esd. 9:4 ; Mal. 3:16.}.
\VS{4}Lève-toi, car cette affaire te regarde. Nous serons avec toi. Prends donc courage et agis.
\VS{5}Esdras se leva, et il fit jurer aux chefs des sacrificateurs, des Lévites, et de tout Israël, de faire selon cette parole. Et ils le jurèrent.
\VS{6}Puis Esdras se retira de devant la maison de Dieu, et s'en alla dans la chambre de Jochanan, fils d'Eliaschib ; et quand il y fut entré, il ne mangea point de pain, ne but point d'eau, parce qu'il se lamentait à cause du péché de ceux de la captivité.
\VS{7}Alors on publia dans le pays de Juda et à Jérusalem que tous ceux qui étaient retournés de la captivité aient à s'assembler à Jérusalem,
\VS{8}et que quiconque ne s'y rendrait pas dans trois jours, selon l'avis des chefs et des anciens, aurait tous ses biens complètement détruits, et que lui-même serait séparé de l'assemblée de ceux de la captivité.
\VS{9}Ainsi tous ceux de Juda et de Benjamin s'assemblèrent à Jérusalem dans les trois jours. C’était le vingtième jour du neuvième mois. Tout le peuple se tenait sur la place de la maison de Dieu, tremblant au sujet de cette affaire et à cause des pluies\FTNT{1 S. 12:18.}.
\VS{10}Esdras, le sacrificateur, se leva et leur dit : Vous avez péché en vous mariant avec des femmes étrangères, de sorte que vous avez augmenté la culpabilité d'Israël\FTNT{De. 7:3.}.
\VS{11}Prononcez maintenant votre confession à Yahweh, le Dieu de vos pères, et faites sa volonté ! Séparez-vous des peuples du pays et des femmes étrangères.
\VS{12}Et toute l'assemblée répondit à haute voix : A nous de faire ce que tu as dit !
\VS{13}Mais le peuple est nombreux, le temps est pluvieux, et il n'y a pas moyen de se tenir dehors ; d’ailleurs, ce n’est pas l’affaire d’un jour ou de deux, car il y en a beaucoup parmi nous qui ont péché dans cette affaire.
\VS{14}Que tous nos chefs se présentent donc devant toute l'assemblée, et que tous ceux qui sont dans nos villes, et qui se sont mariés avec des femmes étrangères, viennent à un temps fixé, et que les anciens de chaque ville et ses juges soient avec eux, jusqu'à ce que nous détournions de nous l'ardente colère de notre Dieu à ce sujet.
\VS{15}Il n'y eut que Jonathan, fils d'Asaël, et Jachzia, fils de Thikva, qui s'opposèrent a cet avis ; et Meschullam et Schabthaï, Lévites, les appuyèrent ;
\VS{16}mais ceux qui étaient retournés de la captivité s’y conformèrent. On choisit Esdras, le sacrificateur, et des chefs de famille selon leurs maisons paternelles, tous désignés par leurs noms ; ils siégèrent le premier jour du dixième mois, pour suivre cette affaire.
\VS{17}Le premier jour du premier mois, ils en finirent avec tous les hommes qui s’étaient mariés à des femmes étrangères.
\VS{18}Parmi les fils des sacrificateurs qui s’étaient mariés à des femmes étrangères, il se trouva d'entre les fils de Josué, fils de Jotsadak et de ses frères, Maaséja, Eliézer, Jarib et Guedalia,
\VS{19}qui, en donnant leurs mains, renvoyèrent leurs femmes ; et offrirent un bélier comme sacrifice de culpabilité ;
\VS{20}des fils d'Immer, Hanani et Zebadia ;
\VS{21}des fils de Harim, Maaséja, Elie, Schemaeja, Jehiel et Ozias ;
\VS{22}des fils de Paschhur, Eljoénaï, Maaséja, Ismaël, Nethaneel, Jozabad et Eleasa.
\VS{23}Parmi les Lévites : Jozabad, Schimeï, Kélaja (ou Kelitha) Pethachja, Juda et Eliézer.
\VS{24}Parmi les chantres : Eliaschib. Et des portiers : Schallum, Thélem et Uri.
\VS{25}Parmi ceux d'Israël : Des fils de Pareosch, Ramia, Jizzija, Malkija, Mijamin, Eléazar, Malkija et Benaja ;
\VS{26}des fils d’Elam, Matthania, Zacharie, Jehiel, Abdi, Jérémoth et Elie ;
\VS{27}des fils de Zatthu, Eljoénaï, Eliaschib, Matthania, Jérémoth, Zabad et Aziza ;
\VS{28}des fils de Bébaï, Jochanan, Hanania, Zabbaï et Athlaï ;
\VS{29}des fils de Bani, Meschullam, Malluc, Adaja, Jaschub, Scheal et Ramoth ;
\VS{30}des fils de Pachath-Moab, Adna, Kelal, Benaja, Maaséja, Matthania, Betsaleel, Binnuï et Manassé ;
\VS{31}des fils de Harim, Eliézer, Jischija, Malkija, Schemaeja, Siméon,
\VS{32}Benjamin, Malluc et Schemaria ;
\VS{33}des fils de Haschum, Matthnaï, Matthattha, Zabad, Eliphéleth, Jerémaï, Manassé et Schimeï ;
\VS{34}des fils de Bani, Maadaï, Amram, Uel,
\VS{35}Benaja, Bédia, Keluhu,
\VS{36}Vania, Merémoth, Eliaschib,
\VS{37}Matthania, Matthnaï, Jaasaï,
\VS{38}Bani, Binnuï, Schimeï,
\VS{39}Schélémia, Nathan, Adaja,
\VS{40}Macnadbaï, Schaschaï, Scharaï,
\VS{41}Azareel, Schélémia, Schemaria,
\VS{42}Schallum, Amaria et Joseph ;
\VS{43}des fils de Nebo, Jeïel, Matthithia, Zabad, Zebina, Jaddaï, Joël et Benaja.
\VS{44}Tous ceux-là avaient pris des femmes étrangères ; et  quelques-uns avaient eu des fils avec ces femmes-là.
\PPE{}
\end{multicols}

%\clearpage\ShortTitle{Néhémie}\BookTitle{Néhémie}\BFont
\noindent\hrulefill
{\footnotesize
\textit{
\bigskip
{\centering{}
\\Auteur : Néhémie
\\(Heb. : Nechemyah)
\\Signification : Yahweh a consolé
\\Thème : Reconstruction des murailles de Jérusalem
\\Date de rédaction : 5\up{ème} siècle av. J.-C.\\}
}
%\bigskip
\textit{
\\En apprenant l'état de ruine dans lequel se trouvait Jérusalem, Néhémie, échanson du roi perse Artaxerxés Ier, fut profondément affecté. Après plusieurs jours dans la désolation et l'humiliation, le Seigneur toucha le cœur du roi qui lui donna l'autorisation et le matériel nécessaire pour rebâtir la muraille de Jérusalem. Malgré les nombreuses oppositions dont il fit l'objet au cours de son entreprise, Néhémie acheva l'œuvre qui lui avait été confiée. Dans le même temps, il mit en place de profondes réformes dans le cadre du retour à la loi de Yahweh.
%\bigskip
\\Complément du livre d'Esdras avec lequel il ne formait initialement qu'un ouvrage, le livre de Néhémie présente un homme de prière, un serviteur œuvrant pour, avec, et au Nom de Yahweh.\bigskip
}
}
\par\nobreak\noindent\hrulefill
\begin{multicols}{2}
\Chap{1}
\TextTitle{La détresse du peuple resté à Jérusalem est racontée à Néhémie}
\VerseOne{}Paroles de Néhémie, fils de Hacalia. Il arriva au mois de Kisleu, la vingtième année, comme j'étais à Suse, la capitale,
\VS{2}Hanani, l'un de mes frères et quelques hommes arrivèrent de Juda. Je les questionnai au sujet des Juifs réchappés qui étaient restés de la captivité et au sujet de Jérusalem.
\VS{3}Et ils me dirent : Ceux qui sont restés de la captivité sont là dans la province, dans une grande misère et dans l'opprobre ; et la muraille de Jérusalem demeure renversée et ses portes ont été consumées par le feu.
\TextTitle{Néhémie prie Yahweh et implore sa grâce}
\VS{4}Or il arriva que, dès que j'entendis ces paroles, je m'assis, je pleurai et je fus dans le deuil plusieurs jours. Je jeûnai et je priai devant le Dieu des cieux,
\VS{5} et je dis : Je te prie, ô Yahweh ! Dieu des cieux, Dieu grand et redoutable, qui garde l'alliance et la miséricorde de ceux qui t'aiment et qui observent tes commandements !
\VS{6}Je te prie que ton oreille soit attentive et que tes yeux soient ouverts pour entendre la prière que ton serviteur te présente en ce temps-ci, jour et nuit, pour tes serviteurs les enfants d'Israël, en confessant les péchés des enfants d'Israël, que nous avons commis contre toi ; même moi et la maison de mon père, nous avons péché.
\VS{7}Certainement nous sommes coupables devant toi, nous n'avons pas gardé les commandements, les lois et les ordonnances que tu prescrivis à Moïse, ton serviteur.
\VS{8}Mais, je te prie, souviens-toi de la parole que tu chargeas Moïse, ton serviteur, de dire : Vous pécherez et je vous disperserai parmi les peuples\FTNT{De. 28:63-67.} ;
\VS{9}mais si vous revenez à moi, et si vous gardez mes commandements et les observez ; et s'il y en a d'entre vous qui ont été chassés jusqu'à l'extrémité du ciel, je vous rassemblerai de là, et je vous ramènerai au lieu que j'aurai choisi pour y faire habiter mon Nom\FTNT{De. 30:1-10.}.
\VS{10}Ils sont tes serviteurs et ton peuple, que tu as rachetés par ta grande puissance et par ta main forte.
\VS{11}Je te prie donc, Seigneur, que ton oreille soit maintenant attentive à la prière de ton serviteur, et à la prière de tes serviteurs qui prennent plaisir à craindre ton Nom ! Je te prie, donne aujourd'hui du succès à ton serviteur, et fais-lui trouver grâce devant cet homme ! J'étais alors échanson du roi.
\Chap{2}
\TextTitle{Yahweh exauce Néhémie et lui donne la faveur du roi}
\VerseOne{}Et il arriva, au mois de Nisan, la vingtième année du roi Artaxerxès, comme le vin était devant lui, je pris le vin et le présentai au roi. Je n'avais jamais été triste devant lui\FTNT{Pr. 15:13.}.
\VS{2}Et le roi me dit : Pourquoi as-tu mauvais visage, puisque tu n'es point malade ? Cela ne peut être qu'une tristesse de cœur. Je fus alors saisi d'une grande crainte,
\VS{3}et je répondis au roi : Que le roi vive éternellement ! Comment n'aurais-je pas mauvais visage, puisque la ville où sont les sépulcres de mes pères demeure désolée et que ses portes ont été consumées par le feu ?
\VS{4}Et le roi dit : Que me demandes-tu ? Alors je priai le Dieu des cieux,
\VS{5}et je dis au roi : Si le roi le trouve bon, et si ton serviteur lui est agréable, envoie-moi en Juda, vers la ville des sépulcres de mes pères, pour la rebâtir\FTNT{La reconstruction de la ville de Jérusalem sous Néhémie date, selon certains, de l'an 445 av. J.-C., suite au décret d'Artaxerxès. Cette date marquerait le point de départ des soixante-dix semaines d'années annoncées par Daniel (Da. 9:24-27).}.
\VS{6}Le roi me dit, et sa femme aussi qui était assise auprès de lui : Combien ton voyage durera-t-il, et quand seras-tu de retour ? Je lui précisai le temps, et le roi trouva bon de m'envoyer.
\VS{7}Puis je dis au roi : Si le roi le trouve bon, qu'on me donne des lettres pour les gouverneurs de l'autre côté du fleuve, afin qu'ils me laissent passer, jusqu'à ce que j'arrive en Juda ;
\VS{8}et des lettres pour Asaph, le garde de la forêt du roi, afin qu'il me donne du bois pour la charpente des portes de la forteresse près de la maison, pour les murailles de la ville, et pour la maison dans laquelle j'entrerai. Et le roi me l'accorda, car la main de mon Dieu était bonne sur moi.
\TextTitle{Arrivée à Jérusalem, constat des murailles en ruines}
\VS{9}J'allai donc vers les gouverneurs qui sont de l'autre côté du fleuve et je leur donnai les lettres du roi. Le roi avait aussi envoyé avec moi des chefs de l'armée et des cavaliers.
\VS{10}Quand Sanballat, le Horonite, et Tobija, le serviteur Ammonite, l'ayant appris, ils eurent un très grand déplaisir de ce qu'il venait un homme pour procurer du bien aux enfants d'Israël.
\VS{11}Ainsi j'arrivai à Jérusalem et j'y passai trois jours.
\VS{12}Puis je me levai de nuit, avec quelques hommes ; mais je ne dis à personne ce que Dieu avait mis dans mon cœur de faire pour Jérusalem. Il n'y avait point d'autre bête avec moi que celle sur laquelle j'étais monté.
\VS{13}Je sortis donc de nuit par la porte de la vallée et me dirigeai vers la source du dragon, vers la porte du fumier ; et je considérai les murailles de Jérusalem qui étaient en ruines\FTNT{Jé. 39:8.}, et ses portes consumées par le feu.
\VS{14}Je passai près de la porte de la source et vers l'étang du roi ; et il n'y avait point de place par où je puisse passer avec ma monture.
\VS{15}Je montai de nuit par le torrent et je considérai la muraille. Puis en revenant, je rentrai par la porte de la vallée ; et ainsi je fus de retour.
\VS{16}Or les magistrats ne savaient pas où j'étais allé, ni ce que je faisais ; car je n'avais rien dit jusqu'à ce moment, ni aux Juifs, ni aux sacrificateurs, ni aux chefs, ni aux magistrats, ni au reste de ceux qui s'occupaient des affaires.
\TextTitle{Néhémie partage sa vision de rebâtir la muraille}
\VS{17}Alors je leur dis : Vous voyez la misère dans laquelle nous sommes ! Comment Jérusalem demeure désolée et ses portes brûlées par le feu ! Venez et rebâtissons les murailles de Jérusalem et nous ne serons plus dans l'opprobre.
\VS{18}Et je leur déclarai comment la main de mon Dieu avait été bonne sur moi, et quelles paroles le roi m'avait dites. Alors ils dirent : Levons-nous et bâtissons ! Ils fortifièrent leurs mains pour bien faire.
\TextTitle{Premières oppositions}
\VS{19}Mais Sanballat, le Horonite, Tobija, le serviteur Ammonite, et Guéschem, l'Arabe, l'ayant appris, se moquèrent de nous et nous méprisèrent. Ils dirent : Qu'est-ce que vous faites ? Ne vous rebellez-vous pas contre le roi ?
\VS{20}Et je leur répondis cette parole : Le Dieu des cieux lui-même nous donnera le succès ! Nous donc, qui sommes ses serviteurs, nous nous lèverons et nous bâtirons ; mais vous, vous n'avez aucune part, ni droit, ni souvenir, à Jérusalem.
\Chap{3}
\TextTitle{Les participants à la reconstruction de la muraille}
\VerseOne{}Eliaschib, le souverain sacrificateur, se leva donc avec ses frères, les sacrificateurs et ils rebâtirent la porte des brebis\FTNT{La première porte qui fut reconstruite fut la porte des brebis. Cette porte est très proche du temple, c'est par elle que l'on faisait entrer les brebis destinées aux sacrifices dans la cour du temple. Cette porte est la préfiguration de Jésus-Christ qui s'est lui-même présenté comme étant la « porte des brebis » (Jn. 10:7).}. Ils la sanctifièrent, ils y posèrent ses battants. Ils la sanctifièrent depuis la tour de Méa jusqu'à la tour de Hananeel.
\VS{2}Et les gens de Jéricho rebâtirent à son côté ; et à côté d'eux Zaccur, fils d'Imri, rebâtit aussi.
\VS{3}Les fils de Senaa rebâtirent la porte des poissons. Ils en firent la charpente et y mirent ses portes, ses serrures et ses barres.
\VS{4}Et à leur côté travailla aux réparations Merémoth, fils d'Urie, fils d'Hakkots ; et à leur côté travailla Meschullam, fils de Bérékia, fils de Meschézabeel, et à leur côté travailla Tsadok, fils de Baana.
\VS{5}A leur côté travaillèrent les Tekoïtes ; mais les chefs d'entre eux ne vinrent point au service de leur Seigneur.
\VS{6}Et Jojada, fils de Paséach, et Meschullam, fils de Besodia, réparèrent la vieille porte. Ils en firent la charpente, y mirent ses battants, ses serrures et ses barres.
\VS{7}A leur côté travaillèrent Melatia, le Gabaonite, Jadon, le Méronothite, et les hommes de Gabaon et de Mitspa, vers le siège du gouverneur de ce côté du fleuve.
\VS{8}A côté d'eux travailla Uzziel, fils de Harhaja, d'entre les orfèvres, et à côté de lui travailla Hanania, d'entre les parfumeurs. Et ainsi ils relevèrent Jérusalem jusqu'à la muraille large.
\VS{9}Et à leur côté travailla Rephaja, fils de Hur, chef d'un demi-quartier de Jérusalem.
\VS{10}Puis à leur côté travailla Jedaja, fils de Harumaph, devant sa maison ; et à son côté travailla Hattusch, fils de Haschabnia.
\VS{11}Et Malkija, fils de Harim, et Haschub, fils de Pachath-Moab, en réparèrent une seconde section, et la tour des fours.
\VS{12}Et à leur côté travailla, avec ses filles, Schallum, fils de d'Hallochesch, chef de la moitié du quartier de Jérusalem.
\VS{13}Hanun et les habitants de Zanoach réparèrent la porte de la vallée. Ils la rebâtirent et mirent ses battants, ses serrures, et ses barres, et ils bâtirent mille coudées de muraille, jusqu'à la porte du fumier.
\VS{14}Et Malkija, fils de Récab, chef du quartier de Beth-Hakkérem, répara la porte du fumier. Il la rebâtit et mit ses battants, ses serrures et ses barres.
\VS{15}Schallum, fils de Col-Hozé, chef du quartier de Mitspa, répara la porte de la source. Il la rebâtit et la couvrit, et mit ses portes, ses serrures, et ses barres. Il répara aussi la muraille de l'étang de Siloé, vers le jardin du roi, et jusqu'aux marches qui descendent de la cité de David.
\VS{16}Après lui travailla Néhémie, fils d'Azbuk, chef de la moitié du quartier de Beth-Tsur, jusqu'à l'endroit des sépulcres de David, et jusqu'à l'étang qui avait été refait, et jusqu'à la maison des hommes vaillants.
\VS{17}Après lui travaillèrent les Lévites, Rehum, fils de Bani ; et à son côté travailla Haschabia, chef de la moitié du quartier de Keïla, pour ceux de son quartier.
\VS{18}Après lui travaillèrent leurs frères, Bavvaï, fils de Hénadad, chef de la moitié du quartier de Keïla.
\VS{19}A son côté, Ezer, fils de Josué, chef de Mitspa, en répara autant, à l'endroit où l'on monte à l'arsenal, à l'angle.
\VS{20}Après lui Baruc, fils de Zabbaï, répara avec ardeur une seconde section, depuis l'angle jusqu'à la porte de la maison d'Eliaschib, le souverain sacrificateur.
\VS{21}Après lui Merémoth, fils d'Urie, fils d'Hakkots, répara une seconde section, depuis l'entrée de la maison d'Eliaschib, jusqu'à l'extrémité de la maison d'Eliaschib.
\VS{22}Et après lui travaillèrent les sacrificateurs, habitants des environs.
\VS{23}Après eux, Benjamin et Haschub travaillèrent devant leur maison. Après eux, Azaria, fils de Maaséja, fils d'Anania, travailla auprès de sa maison.
\VS{24}Après lui, Binnuï, fils de Hénadad, répara une seconde section, depuis la maison d'Azaria jusqu'à l'angle et jusqu'au coin.
\VS{25}Palal, fils d'Uzaï, travailla vis-à-vis de l'angle, et de la tour qui sort de la tour supérieure du roi, qui est auprès de la cour de la prison. Après lui travailla Pedaja, fils de Pareosch.
\VS{26}Les Néthiniens, qui demeuraient sur la colline, réparèrent vers l'orient, jusqu'à l'endroit de la porte des eaux, et vers la tour qui sort.
\VS{27}Après eux, les Tekoïtes réparèrent une seconde section, depuis l'endroit de la grande tour qui sort en dehors, jusqu'à la muraille de la colline.
\VS{28}Au-dessus de la porte des chevaux, les sacrificateurs travaillèrent, chacun devant de sa maison.
\VS{29}Après eux, Tsadok, fils d'Immer, travailla devant sa maison. Après lui répara Schemaeja, fils de Schecania, gardien de la porte orientale.
\VS{30}Après lui, Hanania, fils de Schélémia et Hanun le sixième fils de Tsalaph, en réparèrent une seconde section. Après eux, Meschullam, fils de Bérékia, travailla vis-à-vis de sa chambre.
\VS{31}Après lui, Malkija, fils de l'orfèvre, répara jusqu'à la maison des Néthiniens et des marchands, vis-à-vis de la porte de Miphkad, et jusqu'à la chambre haute du coin.
\VS{32}Et les orfèvres et les marchands travaillèrent entre la chambre haute du coin et la porte des brebis.
\Chap{4}
\TextTitle{La prière, solution pour faire face aux attaques et moqueries}
\VerseOne{}Or il arriva que Sanballat apprit que nous rebâtissions la muraille, il devint furieux et très fâché. Il se moqua des Juifs.
\VS{2}Et il dit en présence de ses frères, et des gens de guerre de Samarie : Que font ces faibles Juifs ? Les laissera-t-on faire ? Sacrifieront-ils ? Et achèveront-ils tout en un jour ? Pourront-ils faire revenir à la vie les pierres des monceaux de poussière, puisqu'elles sont brûlées ?
\VS{3}Et Tobija, l'Ammonite, qui était auprès de lui, dit : Qu'ils bâtissent encore ! Si un renard monte, il rompra leur muraille de pierre !
\VS{4}Ô notre Dieu, écoute comment nous sommes méprisés ! Fais retourner leurs insultes sur leur tête, et donne-les en pillage dans un pays de captivité.
\VS{5}Ne couvre point leur iniquité, et que leur péché ne soit point effacé de devant ta face ; car ils ont irrité les bâtisseurs.
\VS{6}Nous rebâtîmes donc la muraille, et tout le mur fut achevé jusqu'à sa moitié ; et le peuple avait le cœur au travail.
\VS{7}Mais quand Sanballat et Tobija, les Arabes, les Ammonites et les Asdodiens eurent appris que la muraille de Jérusalem avait été refaite, et qu'on avait commencé à fermer les brèches, ils s'enflammèrent de colère.
\VS{8}Et ils se liguèrent tous ensemble pour venir faire la guerre contre Jérusalem, et pour les faire échouer.
\VS{9}Alors nous priâmes notre Dieu, et ayant peur d'eux, nous établîmes une garde jour et nuit pour nous défendre contre leurs attaques.
\TextTitle{Persévérance du peuple prêt à se battre à tout moment}
\VS{10}Et Juda disait : La force des ouvriers est affaiblie, et il y a beaucoup de débris, en sorte que nous ne pourrons pas bâtir la muraille.
\VS{11}Et nos ennemis disaient : Qu'ils n'en sachent rien et qu'ils ne voient rien, jusqu'à ce que nous entrions au milieu d'eux ; nous les tuerons et ferons ainsi cesser l'ouvrage.
\VS{12}Mais il arriva que les Juifs, qui habitaient près d'eux, vinrent dix fois nous avertir, de tous les lieux d'où ils se rendaient vers nous.
\VS{13}C'est pourquoi je plaçai le peuple depuis le bas, derrière la muraille, et sur des lieux élevés, secs et lumineux, selon leurs familles, avec leurs épées, leurs lances et leurs arcs.
\VS{14}Puis je regardai et m'étant levé, je dis aux chefs, aux magistrats et au reste du peuple : N'ayez point peur d'eux ! Souvenez-vous du Seigneur, qui est grand et terrible, et combattez pour vos frères, pour vos fils et pour vos filles, pour vos femmes et pour vos maisons !
\VS{15}Et quand nos ennemis entendirent que nous étions avertis, Dieu fit échouer leur projet, et nous retournâmes tous aux murailles, chacun à son travail.
\VS{16}Depuis ce jour-là, la moitié de mes serviteurs travaillait, et l'autre moitié avait des lances, des boucliers, des arcs et des cuirasses. Les gouverneurs suivaient chaque maison de Juda.
\VS{17}Ceux qui bâtissaient la muraille, et ceux qui portaient ou chargeaient les fardeaux, travaillaient chacun d'une main, et de l'autre ils tenaient une arme.
\VS{18}Car chacun de ceux qui bâtissaient avait son épée ceinte autour des reins. Et celui qui sonnait du shofar se tenait près de moi.
\VS{19}Et je dis aux chefs, aux magistrats et au reste du peuple : L'ouvrage est grand et étendu, et nous sommes séparés sur la muraille, éloignés les uns des autres.
\VS{20}En quelque lieu donc d'où vous entendrez le son du shofar, courez-y vers nous ; notre Dieu combattra pour nous\FTNT{Ex. 14:14 ; De. 1:30 ; 2 Ch. 20:29.}.
\VS{21}C'est donc ainsi que nous accomplissions le travail ; la moitié tenait des lances, depuis le lever du jour jusqu'à l'apparition des étoiles.
\VS{22}En ce temps-là, je dis aussi au peuple : Que chacun passe la nuit dans Jérusalem avec son serviteur, afin de faire la garde la nuit et de travailler le jour.
\VS{23}Et nous ne quittions point nos vêtements, ni moi, ni mes frères, ni mes serviteurs, ni les hommes de garde qui me suivaient; chacun n'avait que ses armes et de l'eau.
\Chap{5}
\TextTitle{Cupidité des chefs dévoilée ; rétablissement de la justice}
\VerseOne{}Or il y eut un grand cri du peuple et de leurs femmes, contre les Juifs, leurs frères.
\VS{2}Les uns disaient : Nous, nos fils et nos filles, nous sommes nombreux; qu'on nous donne du blé, afin que nous mangions et que nous vivions.
\VS{3}Et d'autres disaient : Nous engageons nos champs, nos vignes et nos maisons, pour avoir du blé pendant la famine.
\VS{4}D'autres disaient : Nous avons emprunté de l'argent sur nos champs et sur nos vignes pour le tribut du roi.
\VS{5}Toutefois notre chair est comme la chair de nos frères, et nos fils sont comme leurs fils ; et voici, nous soumettons à la servitude nos fils et nos filles ; et quelques-unes de nos filles sont déjà esclaves et ne sont plus en notre pouvoir ; et nos champs et nos vignes sont à d'autres.
\VS{6}Je fus très en colère quand j'entendis leur cri et ces paroles-là.
\VS{7}Je résolus dans mon cœur de réprimander les chefs et les magistrats, et je leur dis : Vous prêtez avec intérêt à vos frères\FTNT{Ex.22:25 ; Lé. 25:36.} ! Et je fis convoquer autour d'eux une grande foule.
\VS{8}Et je leur dis : Nous avons racheté selon notre pouvoir nos frères Juifs vendus aux nations, et vous vendriez vous-mêmes vos frères, ou c'est à nous qu'ils seraient vendus ? Ils se turent, ne trouvant rien à dire.
\VS{9}Et je dis : Ce que vous faites n'est pas bien. Ne voulez-vous pas marcher dans la crainte de notre Dieu, plutôt que d'être insultés par les nations qui sont nos ennemies ?
\VS{10}Moi aussi, mes frères et mes serviteurs, nous leur avons prêté de l'argent et du blé. Abandonnons je vous prie, cette dette !
\VS{11}Rendez-leur, je vous prie, aujourd'hui leurs champs, leurs vignes, leurs oliviers et leurs maisons ; et outre cela, le centième de l'argent, du blé, du vin, et de l'huile que vous exigez d'eux.
\VS{12}Et ils répondirent : Nous les rendrons et nous ne leur demanderons rien ; nous ferons ce que tu dis. Alors j'appelai les sacrificateurs et je les fis jurer de tenir parole.
\VS{13}Et je secouai mon bras et je dis : Que Dieu secoue ainsi de sa maison et de son travail tout homme qui n'aura pas tenu parole, et qu'il soit ainsi secoué et vidé ! Et toute l'assemblée répondit : Amen ! Et ils louèrent Yahweh. Et le peuple fit selon cette parole.
\TextTitle{Néhémie, modèle de dévouement}
\VS{14}Et même, depuis le jour où le roi m'établit comme gouverneur au pays de Juda, depuis la vingtième année jusqu'à la trente-deuxième année du roi Artaxerxès, pendant douze ans, moi et mes frères, nous n'avons pas pris ce qui était assigné au gouverneur comme revenu.
\VS{15}Quoique, les premiers gouverneurs qui avaient été avant moi, chargeaient le peuple, et prenaient de lui du pain et du vin, outre quarante sicles d'argent, et leurs serviteurs tyrannisaient le peuple. Mais je n'ai point fait ainsi, à cause de la crainte de mon Dieu.
\VS{16}Et même, j'ai travaillé à la réparation d'une partie de cette muraille, et nous n'avons acheté aucun champ, et tous mes serviteurs étaient tous ensemble à l'ouvrage.
\VS{17}Et outre cela, j'avais aussi à ma table les Juifs et les magistrats, au nombre de cent cinquante hommes, et ceux qui venaient vers nous des nations d'alentour.
\VS{18}On m'apprêtait chaque jour un bœuf, six moutons choisis et aussi des volailles ; et tous les dix jours on me présentait toutes sortes de vins en abondance. Malgré cela, je n'ai point demandé le revenu qui était assigné au gouverneur ; parce que les travaux étaient à la charge de ce peuple.
\VS{19}Ô mon Dieu ! Souviens-toi de moi en bien, à cause de tout ce que j'ai fait pour ce peuple.
\Chap{6}
\TextTitle{Complot et mensonge contre Néhémie ; fermeté et confiance en Dieu}
\VerseOne{}Or il arriva que quand Sanballat, Tobija, et Guéschem l'Arabe, et le reste de nos ennemis apprirent que j'avais rebâti la muraille, et qu'il n'y restait aucune brèche. (bien que jusqu'à ce temps-là, je n'avais pas encore mis les battants aux portes.)
\VS{2}Alors Sanballat et Guéschem envoyèrent vers moi, pour dire : Viens, et ayons ensemble une rencontre dans les villages qui sont dans la vallée d'Ono. Or ils avaient comploté de me faire du mal.
\VS{3}Mais j'envoyai des messagers vers eux pour leur dire : J'ai un grand ouvrage à faire, et je ne puis descendre. Le travail serait interrompu pendant que je le quitterais pour aller vers vous.
\VS{4}Ils m'adressèrent la même chose quatre fois ; et je leur répondis la même réponse.
\VS{5}Alors Sanballat m'envoya son serviteur pour me tenir le même discours une cinquième fois ; et il avait dans sa main une lettre ouverte.
\VS{6}Il y était écrit : On entend dire parmi les nations, et Gaschmu le dit, que vous pensez, toi et les Juifs, à vous révolter, et que c'est pour cela que tu rebâtis la muraille. Et tu vas, dit-on, devenir leur roi ;
\VS{7}Même que tu as ordonné des prophètes pour te louer dans Jérusalem, et pour dire : Il est roi de Juda. Et maintenant, on fera entendre au roi ces mêmes choses. Viens donc afin que nous consultions ensemble.
\VS{8}Et je renvoyai vers lui pour lui dire : Ce que tu dis là n'est point, mais c'est toi qui l'inventes dans ton propre coeur !
\VS{9}Car tous ces gens voulaient nous effrayer, en disant : Leurs mains relâcheront le travail, de sorte qu'il ne se fera point. Maintenant donc, ô Dieu, fortifie-moi !
\VS{10}Je me rendis à la maison de Schemaeja, fils de Delaja, fils de Mehétabeel. Il s'était enfermé et il me dit : Assemblons-nous dans la maison de Dieu, au milieu du temple et fermons les portes du temple ; car ils doivent venir pour te tuer, et ils viendront pendant la nuit pour te tuer.
\VS{11}Mais je répondis : Un homme tel que moi s'enfuirait-il ? Et quel homme tel que moi pourrait entrer dans le temple pour sauver sa vie ? Je n'y entrerai point.
\VS{12}Et voilà, je reconnus bien que Dieu ne l'avait point envoyé, mais qu'il avait prononcé cette prophétie contre moi parce que Sanballat et Tobija lui avaient donné de l'argent.
\VS{13}Car il était leur pensionnaire pour m'épouvanter, et pour m'obliger à agir de la sorte, et à commettre cette faute, afin qu'ils aient quelque mauvaise chose à me reprocher.
\VS{14}Ô mon Dieu ! Souviens-toi de Tobija et de Sanballat, et de leurs actions et aussi de Noadia, la prophétesse, et du reste des prophètes qui cherchaient à m'effrayer !
\TextTitle{Achèvement de la muraille}
\VS{15}Néanmoins, la muraille fut achevée le vingt-cinquième jour du mois d'Elul, en cinquante-deux jours.
\VS{16}Quand donc tous nos ennemis l'apprirent et qu'ils la virent, toutes les nations qui étaient autour de nous furent dans la crainte ; elles éprouvèrent une grande humiliation, et ils reconnurent que cet ouvrage s'était accompli par le secours de notre Dieu.
\VS{17}Mais aussi en ce temps-là, les chefs de Juda adressaient fréquemment des lettres à Tobija, et celles de Tobija venaient à eux.
\VS{18}Car il y en avait plusieurs en Juda qui s'étaient liés à lui par serment, parce qu'il était gendre de Schecania, fils d'Arach, et que son fils Jochanan avait pris la fille de Meschullam, fils de Bérékia.
\VS{19}Ils racontaient même du bien de lui en ma présence, et lui rapportaient mes paroles. Et Tobija envoyait des lettres pour m'effrayer.
\Chap{7}
\TextTitle{Instructions spécifiques à Hanani et Hanania}
\VerseOne{}Or après que la muraille fut rebâtie, et que j'aie mis les portes, et qu'on ait fait la revue des portiers, des chantres et des Lévites ; 
\VS{2}je donnai cet ordre à Hanani, mon frère, et à Hanania, chef de la forteresse de Jérusalem ; car il était tel qu'un homme fidèle doit être, et il craignait Dieu plus que plusieurs autres ;
\VS{3}et je leur dis : Que les portes de Jérusalem ne s'ouvrent point avant la chaleur du soleil ; et pendant que les gardes seront encore là, que l'on ferme les portes, et qu'on y mette les barres ; que l'on place comme gardes les habitants de Jérusalem, chacun à son poste, et chacun devant de sa maison.
\VS{4}Or la ville était spacieuse et grande, mais il y avait peu de gens, et ses maisons n'étaient point bâties\FTNT{De. 4:27.}.
\TextTitle{Liste des familles revenues de captivité avec Zorobabel}
\VS{5}Et mon Dieu me mit à coeur d'assembler les chefs, les magistrats et le peuple, pour en faire le dénombrement selon leurs généalogies. Je trouvai le registre du dénombrement selon les généalogies de ceux qui étaient montés la première fois. Et j'y trouvai ainsi écrit :
\VS{6}Ce sont ici ceux de la province qui remontèrent de la captivité, d'entre ceux que Nebucadnetsar, roi de Babylone, avait transportés en exil, et qui retournèrent à Jérusalem et en Juda, chacun dans sa ville.
\VS{7}Ils vinrent avec Zorobabel\FTNT{Esd. 5:2.}, Josué, Néhémie, Azaria, Raamia, Nachamani, Mardochée, Bilschan, Mispéreth, Bigvaï, Nehum, et Baana. Nombre des hommes du peuple d'Israël :
\VS{8}Les fils de Pareosch, deux mille cent soixante-douze.
\VS{9}Les fils de Schephathia, trois cent soixante-douze.
\VS{10}Les fils d'Arach, six cent cinquante-deux.
\VS{11}Les fils de Pachath-Moab, des fils de Josué et de Joab, deux mille huit cent dix-huit.
\VS{12}Les fils d'Elam, mille deux cent cinquante-quatre.
\VS{13}Les fils de Zatthu, huit cent quarante-cinq.
\VS{14}Les fils de Zaccaï, sept cent soixante.
\VS{15}Les fils de Binnuï, six cent quarante-huit.
\VS{16}Les fils de Bébaï, six cent vingt-huit.
\VS{17}Les fils d'Azgad, deux mille trois cent vingt-deux.
\VS{18}Les fils d'Adonikam, six cent soixante-sept.
\VS{19}Les fils de Bigvaï, deux mille soixante-sept.
\VS{20}Les fils d'Adin, six cent cinquante-cinq.
\VS{21}Les fils d'Ather, issu d'Ezéchias, quatre-vingt-dix-huit.
\VS{22}Les fils de Haschum, trois cent vingt-huit.
\VS{23}Les fils de Betsaï, trois cent vingt-quatre.
\VS{24}Les fils de Hariph, cent douze.
\VS{25}Les fils de Gabaon, quatre-vingt-quinze.
\VS{26}Les gens de Bethléhem et de Netopha, cent quatre-vingt-huit.
\VS{27}Les gens d'Anathoth, cent vingt-huit.
\VS{28}Les gens de Beth-Azmaveth, quarante-deux.
\VS{29}Les gens de Kirjath-Jearim, de Kephira et de Beéroth, sept cent quarante-trois.
\VS{30}Les gens de Rama et de Guéba, six cent vingt et un.
\VS{31}Les gens de Micmas, cent vingt-deux.
\VS{32}Les gens de Béthel et d'Aï, cent vingt-trois.
\VS{33}Les gens de l'autre Nebo, cinquante-deux.
\VS{34}Les fils d'un autre Elam, mille deux cent cinquante-quatre.
\VS{35}Les fils de Harim, trois cent vingt.
\VS{36}Les fils de Jéricho, trois cent quarante-cinq.
\VS{37}Les fils de Lod, de Hadid et d'Ono, sept cent vingt et un.
\VS{38}Les fils de Senaa, trois mille neuf cent trente.
\TextTitle{Liste des sacrificateurs revenus de captivité}
\VS{39}Sacrificateurs : Les fils de Jedaeja, de la maison de Josué, neuf cent soixante-treize.
\VS{40}Les fils d'Immer, mille cinquante-deux.
\VS{41}Les fils de Paschhur, mille deux cent quarante-sept.
\VS{42}Les fils de Harim, mille dix-sept.
\TextTitle{Liste des Lévites revenus de captivité}
\VS{43}Lévites : Les fils de Josué et de Kadmiel, d'entre les fils de Hodva, soixante quatorze.
\VS{44}Chantres : Les fils d'Asaph, cent quarante-huit.
\VS{45}Portiers : Les fils de Schallum, les fils d'Ather, les fils de Thalmon, les fils d'Akkub, les fils de Hathitha, les fils de Schobaï, cent trente-huit.
\TextTitle{Liste des Néthiniens revenus de captivité}
\VS{46}Néthiniens : Les fils de Tsicha, les fils de Hasupha, les fils de Thabbaoth,
\VS{47}les fils de Kéros, les fils de Sia, les fils de Padon,
\VS{48}les fils de Lebana, les fils de Hagaba, les fils de Salmaï,
\VS{49}les fils de Hanan, les fils de Guiddel, les fils de Gachar,
\VS{50}les fils de Reaja, les fils de Retsin, les fils de Nekoda,
\VS{51}les fils de Gazzam, les fils d'Uzza, les fils de Paséach,
\VS{52}les fils de Bésaï, les fils de Mehunim, les fils de Nephischsim,
\VS{53}les fils de Bakbuk, les fils de Hakupha, les fils de Harhur,
\VS{54}les fils de Batslith, les fils de Mehida, les fils de Harscha,
\VS{55}les fils de Barkos, les fils de Sisera, les fils de Thamach,
\VS{56}les fils de Netsiach, les fils de Hathipha.
\TextTitle{Liste des fils des serviteurs de Salomon revenus de captivité}
\VS{57}Fils des serviteurs de Salomon : Les fils de Sothaï, les fils de Sophéreth, les fils de Perida,
\VS{58}les fils de Jaala, les fils de Darkon, les fils de Guiddel,
\VS{59}les fils de Schephathia, les fils de Hatthil, les fils de Pokéreth-Hatsebaïm, les fils d'Amon.
\VS{60}Tous les Néthiniens, et les fils des serviteurs de Salomon, étaient trois cent quatre-vingt-douze.
\VS{61}Voici ceux qui montèrent de Thel-Mélach, de Thel-Harscha, de Kerub-Addon et d'Immer, lesquels ne purent montrer la maison de leurs pères, ni leur race, pour prouver qu'ils étaient d'Israël.
\VS{62}Les fils de Delaja, les fils de Tobija, les fils de Nekoda, six cent quarante-deux.
\TextTitle{Liste des sacrificateurs exclus de la sacrificature}
\VS{63}Et les sacrificateurs : Les fils de Hobaja, les fils d'Hakkots, les fils de Barzillaï, qui prit pour femme une des filles de Barzillaï, le Galaadite, et qui fut appelé de leur nom.
\VS{64}Ils cherchèrent leur registre généalogique, mais ils n'y furent point trouvés ; c'est pourquoi ils furent exclus de la sacrificature.
\VS{65}Et le gouverneur leur dit de ne pas manger des choses très saintes, jusqu'à ce que le sacrificateur eût consulté l'urim et le thummim\FTNT{Ex. 28:30.}.
\TextTitle{Somme des Israélites revenus de captivité}
\VS{66}Toute l'assemblée réunie était de quarante-deux mille trois cent soixante ;
\VS{67}sans leurs serviteurs et leurs servantes, qui étaient sept mille trois cent trente-sept ; et ils avaient deux cent quarante-cinq chantres ou chanteuses.
\TextTitle{Dons des fils d'Israël pour le trésor}
\VS{68}Ils avaient sept cent trente-six chevaux, deux cent quarante-cinq mulets ;
\VS{69}quatre cent trente-cinq chameaux et six mille sept cent vingt ânes.
\VS{70}Or quelques-uns des chefs des pères firent des dons pour l'ouvrage. Le gouverneur donna au trésor mille drachmes d'or, cinquante coupes, cinq cent trente tuniques de sacrificateurs.
\VS{71}Quelques autres d'entre les chefs des pères donnèrent pour le trésor de l'ouvrage vingt mille drachmes d'or et deux mille deux cent mines d'argent.
\VS{72}Le reste du peuple donna vingt mille drachmes d'or, deux mille mines d'argent et soixante-sept tuniques de sacrificateurs.
\VS{73}Et ainsi les sacrificateurs, les Lévites, les portiers, les chantres, quelques-uns du peuple, les Néthiniens, et tous ceux d'Israël habitèrent dans leurs villes. Ainsi, quand le septième mois approcha, les enfants d'Israël étaient dans leurs villes.
\Chap{8}
\TextTitle{Lecture du livre de la loi, conviction de péché du peuple}
\VerseOne{}Or tout le peuple s'assembla, comme un seul homme, sur la place qui est devant la porte des eaux. Et ils dirent à Esdras, le scribe, d'apporter le livre de la loi de Moïse, que Yahweh avait ordonnée à Israël.
\VS{2}Et ainsi le premier jour du septième mois, Esdras, le sacrificateur, apporta la loi devant l'assemblée, composée d'hommes et de femmes, et de tous ceux qui étaient capables de l'entendre.
\VS{3}Et il lut dans le livre, sur la place qui est devant la porte des eaux, depuis le matin jusqu'au milieu du jour, en présence des hommes et des femmes, et de ceux qui étaient capables d'entendre. Et les oreilles de tout le peuple étaient attentives à la lecture du livre de la loi.
\VS{4}Ainsi Esdras, le scribe, était debout sur une tour bâtie de bois, qu'on avait dressée pour cela. Il avait auprès de lui, à sa droite, Matthithia, Schéma, Anaja, Urie, Hilkija et Maaséja ; et à sa gauche étaient Pedaja, Mischaël, Malkija, Haschum, Haschbaddana, Zacharie, et Meschullam.
\VS{5}Esdras ouvrit le livre devant les yeux de tout le peuple ; car il était au-dessus de tout le peuple ; et sitôt qu'il l'eut ouvert, tout le peuple se tint debout.
\VS{6}Puis Esdras bénit Yahweh, le grand Dieu ; et tout le peuple répondit en élevant leurs mains: Amen ! Amen ! Et ils s'inclinèrent et se prosternèrent devant Yahweh, le visage contre terre.
\VS{7}Aussi Josué, Bani, Schérébia, Jamin, Akkub, Schabbethaï, Hodija, Maaséja, Kelitha, Azaria, Jozabad, Hanan, Pelaja, et les Lévites, faisaient comprendre la loi au peuple, et le peuple se tenait à sa place.
\VS{8}Et ils lisaient dans le livre de la loi de Dieu, ils l'expliquaient et en donnaient l'intelligence, la faisant comprendre par l'Ecriture elle-même.
\VS{9}Or Néhémie, qui est le gouverneur, Esdras, le sacrificateur et le scribe, et les Lévites qui instruisaient le peuple dirent à tout le peuple : Ce jour est consacré à Yahweh, notre Dieu ; ne soyez pas dans les lamentations, et ne pleurez point ! Car tout le peuple pleurait en entendant les paroles de la loi.
\VS{10}Puis on leur dit : Allez, mangez des viandes grasses, et buvez du vin doux ; et envoyez-en des portions à ceux qui n'ont rien de prêt ; car ce jour est consacré à notre Seigneur. Ne soyez donc point tristes, car la joie de Yahweh est votre force.
\VS{11}Et les Lévites faisaient faire silence parmi tout le peuple, en disant : Taisez-vous, car ce jour est saint, et ne vous affligez point.
\VS{12}Ainsi tout le peuple s'en alla pour manger et pour boire, pour envoyer des portions, et pour faire une grande réjouissance, parce qu'ils avaient bien compris les paroles qu'on leur avait fait connaître.
\TextTitle{Célébration de la fête des tabernacles}
\VS{13}Et le second jour, les chefs des pères de tout le peuple, les sacrificateurs et les Lévites, s'assemblèrent auprès d'Esdras, le scribe, pour sagement comprendre les paroles de la loi.
\VS{14}Et ils trouvèrent écrit dans la loi que Yahweh avait ordonnée par Moïse, que les enfants d'Israël devaient habiter sous des tentes\FTNT{Voir les sept fêtes de Yahweh en Lé. 23.} pendant la fête solennelle au septième mois.
\VS{15}Ce qu'ils firent savoir et qu'ils publièrent dans toutes leurs villes et à Jérusalem, en disant : Allez sur la montagne, et apportez des rameaux d'oliviers, et des rameaux d'autres arbres huileux, des rameaux de myrte, des rameaux de palmier, et des rameaux d'arbres touffus, afin de faire des tentes, selon ce qui est écrit.
\VS{16}Alors le peuple alla et apporta des rameaux. Ils se firent des tentes, chacun sur son toit, dans les cours de leurs maisons, et dans les parvis de la maison de Dieu, sur la place de la porte des eaux, et sur la place de la porte d'Ephraïm.
\VS{17}Ainsi toute l'assemblée de ceux qui étaient revenus de la captivité fit des tentes, et ils habitèrent sous ces tentes. Or les enfants d'Israël n'en avaient point fait de telles depuis les jours de Josué, fils de Nun, jusqu'à ce jour ; et il y eut une très grande joie.
\VS{18}On lut dans le livre de la loi de Dieu chaque jour, depuis le premier jour jusqu'au dernier. On célébra la fête pendant sept jours, et il y eut une assemblée solennelle au huitième jour, comme cela est ordonné.
\Chap{9}
\TextTitle{Confession, jeune et prière du peuple}
\VerseOne{}Et le vingt-quatrième jour du même mois, les enfants d'Israël s'assemblèrent, jeûnant, revêtus de sacs, et ayant de la terre sur eux.
\VS{2}Et la race d'Israël se sépara de tous les étrangers, et ils se présentèrent confessant leurs péchés et les iniquités de leurs pères.
\VS{3}Ils se levèrent donc à leur place, et on lut dans le livre de la loi de Yahweh, leur Dieu, pendant un quart de la journée, et pendant un autre quart, ils faisaient confession, et se prosternaient devant Yahweh, leur Dieu.
\TextTitle{Prière des Lévites, alliance avec Yahweh}
\VS{4}Josué, Bani, Kadmiel, Schebania, Bunni, Schérébia, Bani et Kenani se levèrent sur le lieu qu'on avait élevé pour les Lévites, et crièrent à haute voix à Yahweh, leur Dieu.
\VS{5}Et les Lévites Josué, Kadmiel, Bani, Haschabnia, Schérébia, Hodija, Schebania et Pethachja, dirent : Levez-vous, bénissez Yahweh, votre Dieu, d'éternité en éternité ! Que l'on bénisse ton Nom glorieux, qui est au-dessus de toute bénédiction et de toute louange !
\VS{6}Toi seul, Yahweh, tu as fait les cieux, les cieux des cieux, et toute leur armée ; la terre, et tout ce qui y est ; les mers, et toutes les choses qui y vivent. Tu donnes la vie à toutes ces choses, et l'armée des cieux se prosterne devant toi.
\VS{7}Tu es Yahweh, notre Dieu, qui as choisi Abram, et qui l'as fait sortir d'Ur en Chaldée, et qui lui as donné le nom d'Abraham\FTNT{Ge. 11:31 ; Ge. 17:5.}.
\VS{8}Tu trouvas son coeur fidèle devant toi, et tu traitas avec lui cette alliance que tu donneras à sa postérité le pays des Cananéens, des Héthiens, des Amoréens, des Phéréziens, des Jébusiens, et des Guirgasiens. Et tu as accompli ce que tu as promis, parce que tu es juste.
\VS{9}Car tu vis l'affliction de nos pères en Egypte et tu entendis leurs cris près de la Mer Rouge\FTNT{Ex. 2:23-25.}.
\VS{10}Tu fis des miracles et des prodiges sur Pharaon et sur tous ses serviteurs, et sur tout le peuple de son pays ; parce que tu connus qu'ils s'étaient orgueilleusement élevés contre eux, et tu t'es acquis un renom, tel qu'il paraît aujourd'hui.
\VS{11}Tu fendis aussi la mer devant eux, et ils passèrent à sec au milieu de la mer ; et tu jetas dans l'abîme ceux qui les poursuivaient, comme une pierre dans les eaux violentes.
\VS{12}Tu les fis marcher de jour par la colonne de nuée, et de nuit par la colonne de feu, pour les éclairer dans le chemin par où ils devaient aller\FTNT{Ex. 13:21.}.
\VS{13}Tu descendis sur la montagne de Sinaï, tu parlas avec eux du haut des cieux, tu leur donnas des ordonnances justes et des lois de vérité, des statuts et des commandements bons.
\VS{14}Tu leur fis connaître ton saint sabbat\FTNT{Ge 2:1-3 ; Ex. 20:8-11.} ; et tu leur donnas les commandements, les statuts, et la loi par Moïse, ton serviteur.
\VS{15}Tu leur donnas aussi, du haut des cieux, du pain quand ils avaient faim, et tu fis sortir de l'eau du rocher quand ils avaient soif\FTNT{Ex. 16:13-36 ; No. 20 : 8.}. Et tu leur dis d'entrer et de posséder le pays que tu avais juré de leur donner.
\VS{16}Mais nos pères s'élevèrent orgueilleusement et raidirent leur cou. Ils n'écoutèrent point tes commandements.
\VS{17}Ils refusèrent d'écouter et ne se souvinrent point des merveilles que tu avais faites en leur faveur. Mais ils raidirent leur cou, et par leur rébellion, ils s'attribuèrent un chef pour retourner à leur servitude. Mais toi, tu es un Dieu qui pardonne, miséricordieux, compatissant, lent à la colère et abondant en bonté, et tu ne les abandonnas pas.
\VS{18}Et quand ils se firent un veau en métal fondu et qu'ils dirent : Voici ton Dieu qui t'a fait sortir hors d'Egypte, et qu'ils te firent de grands outrages\FTNT{Ex. 32:1-14.} ;
\VS{19}dans ton immense miséricorde, tu ne les abandonnas pourtant pas dans le désert ; et la colonne de nuée ne se retira point pour les conduire le jour par le chemin, ni la colonne de feu la nuit, pour les éclairer dans le chemin par lequel ils devaient aller.
\VS{20}Tu leur donnas ton bon Esprit pour les rendre sages ; tu ne retiras point ta manne de leur bouche, et tu leur donnas de l'eau pour leur soif.
\VS{21}Tu les nourris ainsi quarante ans au désert, en sorte que rien ne leur manqua. Leurs vêtements ne s'usèrent point, et leurs pieds ne s'enflèrent point.
\VS{22}Tu leur donnas les royaumes et les peuples, dont tu partageas entre eux les contrées ; et ils possédèrent le pays de Sihon, le pays du roi de Hesbon, et le pays d'Og, roi de Basan.
\VS{23}Et tu multiplias leurs fils comme les étoiles des cieux, et les fis entrer au pays dont tu avais dit à leurs pères qu'ils y entreraient pour le posséder.
\VS{24}Ainsi leurs fils y entrèrent et possédèrent le pays ; tu humilias devant eux les habitants du pays, les Cananéens, et les livras entre leurs mains, eux et leurs rois, et les peuples du pays, afin qu'ils en fissent selon leur volonté.
\VS{25}Ils prirent les villes fortifiées et la terre grasse, ils possédèrent les maisons remplies de toutes sortes de biens, les puits qu'on avait creusés, les vignes, les oliviers, et les arbres fruitiers en abondance ; ils mangèrent, ils se rassasièrent ; ils s'engraissèrent et ils vécurent dans les délices de ta grande bonté.
\VS{26}Mais ils se rebellèrent et se révoltèrent contre toi. Ils jetèrent ta loi derrière leur dos, ils tuèrent tes prophètes qui les avertissaient pour les ramener à toi, et ils te firent de grands outrages.
\VS{27}C'est pourquoi tu les donnas aux mains de leurs ennemis, qui les opprimèrent. Mais au temps de leur détresse, ils crièrent à toi, et tu les entendis des cieux ; et selon ta grande miséricorde, tu leur donnas des libérateurs qui les délivrèrent de la main de leurs ennemis.
\VS{28}Mais dès qu'ils eurent du repos, ils recommencèrent à faire le mal devant toi. Alors tu les abandonnas entre les mains de leurs ennemis, qui dominèrent sur eux. Puis ils revinrent et crièrent vers toi, et tu les entendis des cieux. Ainsi tu les délivras selon tes miséricordes, plusieurs fois, et en divers temps.
\VS{29}Et tu les exhortas à revenir à ta loi, mais ils s'élevèrent orgueilleusement et n'écoutèrent pas tes commandements ; ils péchèrent contre tes ordonnances, qui font vivre l'homme qui les observe. Ils tirèrent l'épaule en arrière, raidirent leur cou et n'écoutèrent pas.
\VS{30}Tu les supportas patiemment plusieurs années, et tu les avertissais par ton Esprit, par la main de tes prophètes ; mais ils ne prêtèrent point l'oreille. C'est pourquoi tu les livras entre les mains des peuples des pays étrangers.
\VS{31}Néanmoins, dans ta grande miséricorde, tu ne les anéantis pas et tu ne les abandonnas pas ; car tu es un Dieu compatissant et miséricordieux.
\VS{32}Et maintenant donc, ô notre Dieu ! Grand, puissant et terrifiant, qui garde ton alliance et la miséricorde ; ne regarde pas comme peu de chose cette affliction qui nous est arrivée, à nous, à nos rois, à nos chefs, à nos sacrificateurs, à nos prophètes, à nos pères et à tout ton peuple, depuis le temps des rois d'Assyrie jusqu'à aujourd'hui.
\VS{33}Tu as été juste dans toutes les choses qui nous sont arrivées ; car tu as agi avec fidélité, mais nous, nous avons agi méchamment.
\VS{34}Nos rois, nos chefs, nos sacrificateurs et nos pères n'ont point pratiqué ta loi et n'ont point été attentifs à tes commandements ni à tes témoignages par lesquels tu les as avertis.
\VS{35}Ils ne t'ont point servi durant leur règne ni durant les grands biens que tu leur as faits, même dans le pays vaste et riche que tu leur avais donné pour être à leur disposition, et ils ne se sont point détournés de leurs mauvaises oeuvres.
\VS{36}Voici, nous sommes aujourd'hui esclaves ! Sur la terre que tu as donnée à nos pères pour en manger le fruit et les biens ; voici, nous y sommes esclaves !
\VS{37}Elle rapporte ses produits en abondance pour les rois que tu as établis sur nous à cause de nos péchés, et qui dominent sur nos corps et sur nos bêtes, à leur volonté, de sorte que nous sommes dans une grande angoisse !
\VS{38}C'est pourquoi, à cause de toutes ces choses, nous contractâmes une alliance et nous l'écrivîmes ; et les chefs d'entre nous, nos Lévites et nos sacrificateurs y apposèrent leur sceau.
\Chap{10}
\TextTitle{Liste des contractants et termes de l'alliance}
\VerseOne{}Voici ceux qui apposèrent leur sceau. Néhémie, qui est le gouverneur, fils de Hacalia, et Sédécias.
\VS{2}Seraja, Azaria, Jérémie,
\VS{3}Paschhur, Amaria, Malkija,
\VS{4}Hattusch, Schebania, Malluc,
\VS{5}Harim, Merémoth, Abdias,
\VS{6}Daniel, Guinnethon, Baruc,
\VS{7}Meschullam, Abija, Mijamin,
\VS{8}Maazia, Bilgaï et Schemaeja. Ce sont les sacrificateurs.
\VS{9}Des Lévites : Josué, fils d'Azania, Binnuï d'entre les fils de Hénadad, et Kadmiel.
\VS{10}Et leurs frères, Schebania, Hodija, Kelitha, Pelaja, Hanan,
\VS{11}Michée, Rehob, Haschabia.
\VS{12}Zaccur, Schérébia, Schebania,
\VS{13}Hodija, Bani et Beninu.
\VS{14}Des chefs du peuple : Pareosch, Pachath-Moab, Elam, Zatthu, Bani,
\VS{15}Bunni, Azgad, Bébaï,
\VS{16}Adonija, Bigvaï, Adin,
\VS{17}Ather, Ezéchias, Azzur,
\VS{18}Hodija, Haschum, Betsaï,
\VS{19}Hariph, Anathoth, Nébaï,
\VS{20}Magpiasch, Meschullam, Hézir,
\VS{21}Meschézabeel, Tsadok, Jaddua,
\VS{22}Pelathia, Hanan, Anaja,
\VS{23}Hosée, Hanania, Haschub,
\VS{24}Hallochesch, Pilcha, Schobek,
\VS{25}Rehum, Haschabna, Maaséja,
\VS{26}Achija, Hanan, Anan,
\VS{27}Malluc, Harim et Baana.
\VS{28}Quant au reste du peuple, les sacrificateurs, les Lévites, les portiers, les chantres, les Néthiniens et tous ceux qui s'étaient séparés des peuples de ces pays pour suivre la loi de Dieu, leurs femmes, leurs fils et leurs filles, tous ceux qui étaient capables de connaissance et d'intelligence,
\VS{29}se joignirent à leurs frères les plus considérables d'entre eux. Ils s'engagèrent par serment et jurèrent de marcher dans la loi de Dieu, qui avait été donnée par Moïse, serviteur de Dieu ; de garder et faire tous les commandements de Yahweh, notre Seigneur, ses jugements et ses ordonnances ;
\VS{30}de ne pas donner nos filles aux peuples du pays, et de ne pas prendre leurs filles pour nos fils ;
\VS{31}de ne rien prendre le jour du sabbat, ou tel autre jour consacré, des peuples du pays qui apporteraient des marchandises et toutes sortes de denrées, le jour du sabbat, pour les vendre, d'abandonner la septième année et de faire remise de toute dette.
\VS{32}Nous fîmes aussi des ordonnances, nous chargeant de donner chaque année le tiers d'un sicle, pour le service de la maison de notre Dieu,
\VS{33}pour les pains de proposition, pour l'offrande perpétuelle et pour l'holocauste perpétuel ; pour ceux des sabbats, des nouvelles lunes et des fêtes ; pour les choses consacrées, pour les sacrifices d'expiation afin de faire propitiation pour Israël ; et pour toute l'oeuvre de la maison de notre Dieu.
\VS{34}Nous tirâmes au sort, pour l'offrande du bois, tant les sacrificateurs et les Lévites, que le peuple, afin de l'amener dans la maison de notre Dieu, selon les maisons de nos pères, et dans les temps fixés, d'année en année, pour le brûler sur l'autel de Yahweh, notre Dieu, ainsi qu'il est écrit dans la loi.
\VS{35}Nous décidâmes aussi d'apporter dans la maison de Yahweh, d'année en année, les premiers fruits de notre terre, et les prémices de tous les fruits de tous les arbres ;
\VS{36}d'amener les premiers-nés de nos fils, et de nos bêtes, comme il est écrit dans la loi ; et d'amener dans la maison de notre Dieu, aux sacrificateurs qui font le service dans la maison de notre Dieu, les premiers-nés de nos boeufs et de notre menu bétail;
\VS{37}d'apporter les prémices de notre pâte, nos offrandes, les fruits de tous les arbres, le vin, et l'huile aux sacrificateurs, dans les chambres de la maison de notre Dieu, et la dîme de notre terre aux Lévites, et que les Lévites prendraient les dîmes dans toutes les villes agricoles.
\VS{38}Le sacrificateur, fils d'Aaron, sera avec les Lévites, lorsque les Lévites paieront la dîme\FTNT{Il est question de la dîme des Lévites (No. 18:24 ; De. 14:28-29).}; et les Lévites apporteront la dîme de la dîme\FTNT{Il s'agit ici de la dîme de la dîme que les Lévites donnaient aux sacificateurs. Elle était apportée aux magasins du temple. Voir commentaires en No. 18:21 et Mal. 3:10.} à la maison de notre Dieu, dans les chambres de la maison où sont les magasins\FTNT{Le mot hébreu « owtsar » (trésor) signifie aussi magasin (Né. 12:44 ; Né. 13:12. Né. 13:13).}.
\VS{39}Car les enfants d'Israël et les fils de Lévi apporteront dans ces chambres les offrandes du blé, du vin et de l'huile ; là sont les ustensiles du sanctuaire, et les sacrificateurs qui font le service, les portiers, et les chantres. Et nous n'abandonnâmes point la maison de notre Dieu.
\Chap{11}
\TextTitle{Les habitants de Jérusalem}
\VerseOne{}Les chefs du peuple demeurèrent à Jérusalem. Mais tout le reste du peuple tira au sort, afin qu'un sur dix vînt habiter à Jérusalem, la ville sainte, et que les neuf autres parties demeurassent dans les autres villes.
\VS{2}Et le peuple bénit tous ceux qui se présentèrent volontairement pour habiter à Jérusalem.
\VS{3}Voici les chefs de la province qui habitèrent à Jérusalem ; les autres s'étant établis dans les villes de Juda, chacun dans sa propriété, selon sa ville, Israélites, sacrificateurs, Lévites, Néthiniens, et les fils des serviteurs de Salomon.
\VS{4}A Jérusalem habitèrent donc des fils de Juda et des fils de Benjamin. Des fils de Juda : Athaja, fils d'Ozias, fils de Zacharie, fils d'Amaria, fils de Schephathia, fils de Mahalaleel, d'entre les fils de Pérets,
\VS{5}et Maaséja, fils de Baruc, fils de Col-Hozé, fils de Hazaja, fils d'Adaja, fils de Jojarib, fils de Zacharie, fils de Schiloni.
\VS{6}Total des fils de Pérets, qui s'établirent à Jérusalem : Quatre cent soixante-huit vaillants hommes.
\VS{7}Voici les fils de Benjamin : Sallu, fils de Meschullam, fils de Joëd, fils de Pedaja, fils de Kolaja, fils de Maaséja, fils d'Ithiel, fils d'Esaïe,
\VS{8}et après lui, Gabbaï et Sallaï : Neuf cent vingt-huit.
\VS{9}Joël, fils de Zicri, était leur chef ; et Juda, fils de Senua, était le second chef de la ville.
\VS{10}Des sacrificateurs : Jedaeja, fils de Jojarib, Jakin,
\VS{11}Seraja, fils de Hilkija, fils de Meschullam, fils de Tsadok, fils de Merajoth, fils d'Achithub, prince de la maison de Dieu,
\VS{12}et leurs frères, faisant le service de la maison : Huit cent vingt-deux. Adaja, fils de Jerocham, fils de Pelalia, fils d'Amtsi, fils de Zacharie, fils de Paschhur, fils de Malkija,
\VS{13}et ses frères, chefs des pères : Deux cent quarante-deux ; et Amaschsaï, fils d'Azareel, fils d'Achzaï, fils de Meschillémoth, fils d'Immer,
\VS{14}et leurs frères, forts et vaillants: Cent vingt-huit. Zabdiel, fils de Guedolim, était leur chef.
\VS{15}Des Lévites : Schemaeja, fils de Haschub, fils d'Azrikam, fils de Haschabia, fils de Bunni,
\VS{16}Schabbethaï et Jozabad chargés des travaux extérieurs pour la maison de Dieu, étant d'entre les chefs des Lévites ;
\VS{17}Matthania, fils de Michée, fils de Zabdi, fils d'Asaph, était le chef qui commençait le premier à chanter les louanges dans la prière, et Bakbukia, le second parmi ses frères, puis Abda, fils de Schammua, fils de Galal, fils de Jeduthun.
\VS{18}Total des Lévites dans la ville sainte : Deux cent quatre-vingt-quatre.
\VS{19}Et les portiers : Akkub, Thalmon, et leurs frères qui gardaient les portes : Cent soixante-douze.
\TextTitle{Les habitants des autres villes}
\VS{20}Le reste d'Israël, des sacrificateurs et des Lévites, fut dans toutes les villes de Juda, chacun dans son héritage.
\VS{21}Mais les Néthiniens habitèrent sur la colline ; et Tsicha et Guischpa étaient leurs chefs.
\VS{22}Celui qui avait la charge des Lévites à Jérusalem était Uzzi, fils de Bani, fils de Haschabia, fils de Matthania, fils de Michée, d'entre les fils d'Asaph, chantres, pour l'ouvrage de la maison de Dieu ;
\VS{23}car il y avait un commandement du roi à leur égard, et il y avait chaque jour un salaire assuré pour les chantres.
\VS{24}Pethachja, fils de Meschézabeel, d'entre les fils de Zérach, fils de Juda, était commissaire du roi pour toutes les affaires du peuple.
\VS{25}Dans les villages et leurs territoires, quelques-uns des fils de Juda habitèrent à Kirjath-Arba, et dans les lieux de son ressort ; à Dibon, et dans les lieux de son ressort ; à Jekabtseel, et dans les villages de son ressort,
\VS{26}à Jéschua, à Molada, à Beth-Paleth,
\VS{27}à Hatsar-Schual, à Beer-Schéba, et dans les lieux de son ressort,
\VS{28}à Tsiklag, à Mecona, et dans les lieux de son ressort,
\VS{29}à En-Rimmon, à Tsorea, à Jarmuth,
\VS{30}à Zanoach, à Adullam, et dans leurs villages, à Lakis et dans ses territoires, à Azéka et dans les lieux de son ressort. Ils habitèrent depuis Beer-Schéba jusqu'à la vallée de Hinnom.
\VS{31}Et les fils de Benjamin habitèrent depuis Guéba à Micmasch, à Ajja, à Béthel, et dans les lieux de son ressort,
\VS{32}à Anathoth, à Nob, à Hanania,
\VS{33}à Hatsor, à Rama, à Guitthaïm,
\VS{34}à Hadid, à Tseboïm, à Neballath,
\VS{35}à Lod, et à Ono, la vallée des ouvriers.
\VS{36}D'entre les Lévites, des classes de Juda se rattachèrent à Benjamin.
\Chap{12}
\TextTitle{Les sacrificateurs et les Lévites montés avec Zorobabel}
\VerseOne{}Voici les sacrificateurs et les Lévites qui montèrent avec Zorobabel, fils de Schealthiel, et avec Josué : Seraja, Jérémie, Esdras,
\VS{2}Amaria, Malluc, Hattusch,
\VS{3}Schecania, Rehum, Merémoth,
\VS{4}Iddo, Guinnethoï, Abija,
\VS{5}Mijamin, Maadia, Bilga,
\VS{6}Schemaeja, Jojarib, Jedaeja,
\VS{7}Sallu, Amok, Hilkija, Jedaeja. Ce furent là les chefs des sacrificateurs, et de leurs frères, du temps de Josué.
\VS{8}Lévites : Josué, Binnuï, Kadmiel, Schérébia, Juda, Matthania, qui dirigeait les louanges, lui et ses frères.
\VS{9}Bakbukia et Unni, leurs frères, étaient avec eux pour la surveillance.
\TextTitle{Les fils des sacrificateurs}
\VS{10}Josué engendra Jojakim, Jojakim engendra Eliaschib, Eliaschib engendra Jojada,
\VS{11}Jojada engendra Jonathan, et Jonathan engendra Jaddua.
\VS{12}Au temps de Jojakim, étaient sacrificateurs, chefs des pères : Pour Seraja, Meraja ; pour Jérémie, Hanania ;
\VS{13}pour Esdras, Meschullam ; pour Amaria, Jochanan ;
\VS{14}pour Meluki, Jonathan ; pour Schebania, Joseph ;
\VS{15}pour Harim, Adna ; pour Merajoth, Helkaï ;
\VS{16}pour Iddo, Zacharie ; pour Guinnethon, Meschullam ;
\VS{17}pour Abija, Zicri ; pour Minjamin et Moadia, Pilthaï ;
\VS{18}pour Bilga, Schammua ; pour Schemaeja, Jonathan ;
\VS{19}pour Jojarib, Matthnaï ; pour Jedaeja, Uzzi ;
\VS{20}pour Sallaï, Kallaï ; pour Amok, Eber ;
\VS{21}pour Hilkija, Haschabia ; pour Jedaeja, Nethaneel.
\TextTitle{Les chefs des fils de Lévi}
\VS{22}Au temps d'Eliaschib, de Jojada, de Jochanan et de Jaddua, les Lévites, chefs de famille, et les sacrificateurs, furent inscrits sous le règne de Darius, le Perse.
\VS{23}Les fils de Lévi, chefs des pères, furent enregistrés dans le livre des Chroniques jusqu'au temps de Jochanan, fils d'Eliaschib.
\VS{24}Les chefs des Lévites, Haschabia, Schérébia, et Josué, fils de Kadmiel, et leurs frères, étaient vis-à-vis d'eux, pour louer et célébrer, selon l'ordre de David, homme de Dieu.
\VS{25}Matthania, Bakbukia, Abdias, Meschullam, Thalmon, et Akkub, les portiers, faisaient la garde au seuil des portes.
\VS{26}Ce fut du temps de Jojakim, fils de Josué, fils de Jotsadak, et du temps de Néhémie, le gouverneur, et d'Esdras, sacrificateur et scribe.
\TextTitle{La dédicace de la muraille de Jérusalem}
\VS{27}Lors de la dédicace de la muraille de Jérusalem, on envoya chercher les Lévites de tous les lieux où ils étaient, pour les faire venir à Jérusalem, afin de célébrer la dédicace avec joie, par des louanges, et par des chants sur des cymbales, des luths et des harpes.
\VS{28}Les fils des chantres se rassemblèrent des plaines aux alentours de Jérusalem, des villages des Nethophatiens,
\VS{29}de Beth-Guilgal, et des territoires de Guéba et d'Azmaveth ; car les chantres s'étaient bâtis des villages aux alentours de Jérusalem.
\VS{30}Les sacrificateurs et les Lévites se purifièrent, et ils purifièrent le peuple, les portes et la muraille.
\VS{31}Puis je fis monter sur la muraille les chefs de Juda, et j'établis deux grands chœurs. Le premier se mit en marche du côté droit sur la muraille, vers la porte du fumier.
\VS{32}Et après eux marchait Hosée, avec la moitié des chefs de Juda,
\VS{33}Azaria, Esdras, Meschullam,
\VS{34}Juda, Benjamin, Schemaeja et Jérémie,
\VS{35}des fils des sacrificateurs avec les trompettes, Zacharie, fils de Jonathan, fils de Schemaeja, fils de Matthania, fils de Michée, fils de Zaccur, fils d'Asaph,
\VS{36}et ses frères, Schemaeja, Azareel, Milalaï, Guilalaï, Maaï, Nethaneel, Juda, et Hanani, avec les instruments des cantiques de David, homme de Dieu. Esdras, le scribe, marchait devant eux.
\VS{37}A la porte de la source, qui était vis-à-vis d'eux, ils montèrent aux marches de la cité de David, par la montée de la muraille, depuis la maison de David, jusqu'à la porte des eaux, vers l'orient.
\VS{38}Le second choeur de ceux qui chantaient les louanges allait à l'opposé. J'allais après lui, avec l'autre moitié du peuple, allant sur la muraille. Passant par-dessus la tour des fours, jusqu'à la muraille large ;
\VS{39}puis vers la porte d'Ephraïm, vers la vieille porte, vers la porte des poissons, la tour de Hananeel, et la tour de Méa, jusqu'à la porte des brebis. Et l'on s'arrêta à la porte de la prison.
\VS{40}Les deux choeurs s'arrêtèrent dans la maison de Dieu ; moi aussi, avec les magistrats qui étaient avec moi,
\VS{41}et les sacrificateurs Eliakim, Maaséja, Minjamin, Michée, Eljoénaï, Zacharie, Hanania, avec les trompettes,
\VS{42}et Maaséja, Schemaeja, Eléazar, Uzzi, Jochanan, Malkija, Elam et Ezer. Puis les chantres, desquels Jizrachja avait la charge, se firent entendre.
\VS{43}On offrit ce jour-là de nombreux sacrifices, et on se réjouit, parce que Dieu leur avait donné un grand sujet de joie. Les femmes et les enfants se réjouirent aussi ; et la joie de Jérusalem fut entendue au loin.
\TextTitle{Les sacrificateurs et les Lévites à leur poste}
\VS{44}En ce jour-là, on établit des hommes sur les chambres des trésors, des offrandes, des prémices et des dîmes ; pour rassembler du territoire des villes les portions ordonnées par la loi aux sacrificateurs et aux Lévites. Car Juda se réjouissait de ce que les sacrificateurs et de ce que les Lévites étaient à leur poste,
\VS{45}et parce qu'ils avaient gardé la charge qui leur avait été donnée de la part de leur Dieu, et la charge de la purification. Les chantres et les portiers remplissaient aussi leurs fonctions, selon le commandement de David, et de Salomon, son fils.
\VS{46}Car autrefois, du temps de David et d'Asaph, on avait établi des chefs de chantres et des cantiques de louange et de reconnaissance à Dieu.
\VS{47}Tout Israël, du temps de Zorobabel et de Néhémie, donna les portions des chantres et des portiers, jour par jour, et les consacraient aux Lévites, et les Lévites les consacraient aux fils d'Aaron.
\Chap{13}
\TextTitle{Lecture du livre de Moïse, séparation d'avec les étrangers}
\VerseOne{}Dans ce temps-là, on lut en présence du peuple dans le livre de Moïse, et l'on y trouva écrit que les Ammonites et les Moabites ne devaient jamais entrer dans l'assemblée de Dieu,
\VS{2}parce qu'ils n'étaient pas venus au-devant des enfants d'Israël avec du pain et de l'eau ; et qu'ils avaient engagé à prix d'argent Balaam\FTNT{Balaam : voir No. 22,23 et 24.} contre eux pour qu'il les maudisse ; mais notre Dieu changea la malédiction en bénédiction.
\VS{3}Dès qu'on eut entendu la loi, on sépara d'Israël tous les étrangers.
\TextTitle{Purification des chambres du temple}
\VS{4}Avant cela, le sacrificateur Eliaschib, établi sur les chambres de la maison de notre Dieu, et parent de Tobija,
\VS{5}avait disposé pour lui une grande chambre, où on mettait auparavant les offrandes, l'encens, les ustensiles, les dîmes du blé, du vin et de l'huile, qui étaient ordonnées pour les Lévites, pour les chantres et pour les portiers, avec les contributions pour les sacrificateurs.
\VS{6}Je n'étais point à Jérusalem pendant tout cela, car j'étais retourné vers le roi la trente-deuxième année d'Artaxerxès, roi de Babylone. Et à la fin de l'année, j'obtins du roi la permission
\VS{7}de revenir à Jérusalem, et je m'aperçus du mal qu'Eliaschib avait fait, en disposant une chambre pour Tobija dans le parvis de la maison de Dieu.
\VS{8}J'en éprouvai un vif déplaisir, et je jetai tous les objets de Tobija hors de la chambre ;
\VS{9}j'ordonnai qu'on purifie les chambres, et j'y ramenai les ustensiles de la maison de Dieu, les offrandes et l'encens.
\TextTitle{Sur les portions des Lévites}
\VS{10}J'appris aussi que les portions des Lévites ne leur avaient point été données ; et que les Lévites et les chantres qui faisaient le service s'étaient enfuis chacun sur sa terre.
\VS{11}Je fis des réprimandes aux magistrats, leur disant : Pourquoi a-t-on abandonné la maison de Dieu ? Je rassemblai les Lévites et les chantres, et les rétablis à leur place.
\VS{12}Alors tous ceux de Juda apportèrent dans le trésor les dîmes du blé, du vin et de l'huile.
\VS{13}Je confiai la surveillance du trésor à Schélémia, le sacrificateur, et Tsadok, le scribe, et Pedaja, l'un des Lévites ; et pour les aider, Hanan, fils de Zaccur, fils de Matthania, parce qu'ils étaient considérés comme très fidèles. Ils furent chargés de faire les distributions à leurs frères.
\VS{14}Souviens-toi de moi, ô mon Dieu, à cause de cela et n'efface point ce que j'ai fait avec fidélité pour la maison de mon Dieu, et pour ce qu'il est ordonné d'y faire !
\TextTitle{Avertissement pour le respect du sabbat}
\VS{15}En ces jours-là, je vis quelques-uns de Juda fouler aux pressoirs le jour du sabbat, et d'autres apporter des gerbes, et charger sur des ânes du vin, des raisins, des figues, et toutes autres sortes de fardeaux, et les apporter à Jérusalem le jour du sabbat ; et je les avertis le jour où ils vendaient leurs denrées.
\VS{16}Les Tyriens, qui demeuraient aussi à Jérusalem, apportaient du poisson, et plusieurs autres marchandises, et les vendaient aux fils de Juda dans Jérusalem le jour du sabbat.
\VS{17}Je fis des réprimandes aux chefs de Juda, et leur dis : Quel mal ne faites-vous pas, en violant le jour du sabbat ?
\VS{18}Vos pères n'ont-ils pas fait la même chose, et n'est-ce pas pour cela que notre Dieu a fait venir tout ce mal sur nous et sur cette ville ? Et vous amenez de nouveau son ardente colère contre Israël, en violant le sabbat !
\VS{19}C'est pourquoi, dès que le soleil s'était retiré des portes de Jérusalem, avant le sabbat, par mon commandement, on ferma les portes ; j'ordonnai aussi qu'on ne les ouvre point jusqu'après le sabbat. Et je plaçai quelques-uns de mes serviteurs aux portes, afin d'empêcher l'entrée des fardeaux le jour du sabbat.
\VS{20}Alors les marchands et les vendeurs de toutes sortes de denrées passèrent une ou deux fois la nuit hors de Jérusalem.
\VS{21}Je les avertis et je leur dis : Pourquoi passez-vous la nuit devant la muraille ? Si vous le faites encore, je mettrai la main sur vous. Ainsi, depuis ce temps-là, ils ne vinrent plus le jour du sabbat.
\VS{22}J'ordonnai aussi aux Lévites de se purifier, et de venir garder les portes pour sanctifier le jour du sabbat. Souviens-toi de moi, ô mon Dieu, à cause de cela, et ai compassion de moi selon la grandeur de ta miséricorde !
\TextTitle{Condamnation des unions mixtes ; rétablissement des fonctions des sacrificateurs et des Lévites}
\VS{23}En ces jours-là, je vis des Juifs qui avaient pris des femmes Asdodiennes, Ammonites et Moabites.
\VS{24}La moitié de leurs fils parlaient en partie asdodien et ne savaient point parler l'hébreu ; mais ils parlaient la langue de divers peuples.
\VS{25}Je leur fis des réprimandes et les maudis ; j'en frappai même quelques-uns, leur arrachai les cheveux et les fis jurer par Dieu, qu'ils ne donneraient point leurs filles à leurs fils, et qu'ils ne prendraient point leurs filles pour leurs fils, ou pour eux.
\VS{26}Salomon, le roi d'Israël, n'avait-il point péché par ce moyen ? Il n'y avait point de roi semblable à lui parmi un grand nombre de nations, il était aimé de son Dieu, et Dieu l'avait établi pour roi sur tout Israël ; toutefois, les femmes étrangères l'amenèrent à pécher.
\VS{27}Faut-il donc apprendre que vous fassiez tout ce grand mal, de commettre ce péché contre notre Dieu, en prenant des femmes étrangères ?
\VS{28}Or un des fils de Jojada, fils d'Eliaschib, grand sacrificateur, était gendre de Sanballat, le Horonite. Je le chassai loin de moi.
\VS{29}Souviens-toi d'eux, ô mon Dieu, car ils ont souillé le sacerdoce et l'alliance contractée par les sacrificateurs et les Lévites.
\VS{30}Ainsi je les nettoyai de tous les étrangers et je rétablis les fonctions des sacrificateurs et des Lévites, chacun selon ce qu'il avait à faire,
\VS{31}et ce qui concernait l'offrande du bois aux temps fixés, de même que les prémices. Souviens-toi de moi en bien, ô mon Dieu !
\PPE{}
\end{multicols}

%\clearpage\ShortTitle{1 Chroniques}\BookTitle{1 Chroniques}\BFont
\noindent\hrulefill
{\footnotesize
\textit{
\bigskip
{\centering{}
\\Auteur : Probablement Esdras
\\(Heb. : Hayyamim dibre)
\\Signification : Actes des journées
\\Thème : Généalogies et Histoire
\\Date de rédaction : 5ème siècle av. J.-C.\\}
}
%\bigskip
\textit{
\\Les deux livres des Chroniques constituent des compléments aux livres des Rois dans la mesure où ils confirment les récits de ceux-ci.
%\bigskip
\\Après avoir établi la généalogie d’Adam à Jacob, puis une généalogie plus détaillée de la descendance de Jacob jusqu’au retour de la captivité babylonienne, le premier livre des Chroniques reprend l’histoire du roi David et met un accent particulier sur certains combats qu’il eut à mener, les rapports avec ses serviteurs, ainsi que les préparatifs de la construction du temple. Il présente aussi l’organisation du travail des sacrificateurs et des Lévites au service de
Dieu et du peuple.\bigskip
}
}
\par\nobreak\noindent\hrulefill
\begin{multicols}{2}
\Chap{1}
\TextTitle{Généalogie d'Adam à Noé\FTNTT{Ge. 5:1-32}}
\VerseOne{}Adam, Seth, Enosch\FTNT{Les généalogies se faisaient par les premiers-nés de chaque famille.}.
\VS{2}Kénan, Mahalaleel, Jéred ;
\VS{3}Hénoc, Metuschélah, Lémec.
\VS{4}Noé, Sem, Cham et Japhet\FTNT{Ge. 5:1-32.}.
\TextTitle{Les fils de Japhet\FTNTT{Ge. 10:2-5.}}
\VS{5}Les fils de Japhet furent : Gomer, Magog, Madaï, Javan, Tubal, Méschec et Tiras.
\VS{6}Les fils de Gomer furent : Aschkenaz, Diphat et Togarma.
\VS{7}Les fils de Javan furent : Elischa, Tarsisa, Kittim et Rodanim.
\TextTitle{Les fils de Cham\FTNTT{Ge. 10:6-20.}}
\VS{8}Les fils de Cham furent : Cusch, Mitsraïm, Puth et Canaan.
\VS{9}Les fils de Cusch furent  : Saba, Havila, Sabta, Raema et Sabteca. Les fils de Raema furent : Séba et Dedan.
\VS{10}Cusch engendra aussi Nimrod qui commença à être puissant sur la terre.
\VS{11}Mitsraïm engendra les Ludim, les Anamim, les Lehabim, les Naphtuhim,
\VS{12}les Patrusim, les Casluhim, desquels sont issus les Philistins et les Caphtorim.
\VS{13}Canaan engendra Sidon, son fils aîné, et Heth;
\VS{14}les Jébusiens, les Amoréens, les Guirgasiens,
\VS{15}les Héviens, les Arkiens, les Siniens,
\VS{16}les Arvadiens, les Tsemariens et les Hamathiens.
\TextTitle{Les fils de Sem\FTNTT{Ge. 10:21-31.}}
\VS{17}Les fils de Sem furent : Elam, Assur, Arpacschad, Lud, Aram, Uts, Hul, Guéter et Méschec.
\VS{18}Arpacschad engendra Schélach, et Schélach engendra Héber.
\VS{19}A Héber naquirent deux fils : L'un s'appelait Péleg, car en son temps la terre fut partagée; et son frère s’appelait Jokthan.
\VS{20}Jokthan engendra Almodad, Schéleph, Hatsarmaveth, Jérach,
\VS{21}Hadoram, Uzal, Dikla,
\VS{22}Ebal, Abimaël, Séba,
\VS{23}Ophir, Havila et Jobab; tous ceux-là furent des fils de Jokthan\FTNT{Ge. 10:2-31.}.
\TextTitle{De Sem aux fils d'Abraham\FTNTT{Ge. 11:10-26.}}
\VS{24}Sem, Arpacschad, Schélach\FTNT{Ge. 11:10-26},
\VS{25}Héber, Péleg, Rehu,
\VS{26}Serug, Nachor, Térach,
\VS{27}et Abram, qui est Abraham.
\VS{28}Les fils d’Abraham furent  Isaac et Ismaël.
\TextTitle{Les fils d’Ismaël\FTNTT{Ge. 25:12-18.}}
\VS{29}Voici leur postérité\FTNT{Ge 25:12-18} : Le premier-né d'Ismaël fut Nebajoth, puis Kédar, Adbeel, Mibsam,
\VS{30}Mischma, Duma, Massa, Hadad, Téma,
\VS{31}Jethur, Naphisch et Kedma; ce sont là les fils d'Ismaël.
\TextTitle{Les fils de Ketura\FTNTT{Ge. 25:1-4.}}
\VS{32}Quant aux fils de Ketura, concubine d'Abraham, elle enfanta Zimran, Jokschan, Medan, Madian, Jischbak et Schuach; et les fils de Jokschan furent Séba et Dedan.
\VS{33}Les fils de Madian furent  Epha, Epher, Hénoc, Abida et Eldaa. Tous ceux-là furent les fils de Ketura.
\TextTitle{Les fils d’Isaac\FTNTT{Ge. 25:19-26.}}
\VS{34}Or Abraham engendra Isaac; et les fils d'Isaac furent  Esaü et Israël.
\TextTitle{Les descendants d’Esaü \FTNTT{Ge. 36:1-14.}}
\VS{35}Les fils d’Esaü furent  Eliphaz, Reuel, Jeusch, Jaelam et Koré\FTNT{Ge. 36:1-14.}.
\VS{36}Les fils d’Eliphaz furent Théman, Omar, Tsephi, Gaetham et Kenaz; Thimna lui enfanta Amalek.
\VS{37}Les fils de Reuel furent Nahath, Zérach, Schamma et Mizza.
\VS{38}Les fils de Séir furent Lothan, Schobal, Tsibeon, Ana, Dischon, Etser et Dischan.
\VS{39}Les fils de Lothan furent  Hori et Homam; et Thimna fut la soeur de Lothan.
\VS{40}Les fils de Schobal furent Aljan, Manahath, Ebal, Schephi et Onam. Les fils de Tsibeon furent Ajja et Ana.
\VS{41}Ana eut un fils : Dischon.  Les fils de Dischon furent Hamran, Eschban, Jithran et Keran.
\VS{42}Les fils d’Etser furent Bilhan, Zaavan et Jaakan. Les fils de Dischon furent Uts et Aran.
\TextTitle{Les rois et les chefs d’Edom\FTNTT{Ge. 36:15-19, 25-43}}
\VS{43}Voici les rois qui ont régné au pays d'Edom, avant qu’un roi ne règne sur les fils d’Israël : Béla, fils de Beor, et le nom de sa ville était Dinhaba.
\VS{44}Béla mourut, et Jobab, fils de Zérach de Botsra, régna à sa place.
\VS{45}Jobab mourut, et Huscham, du pays des Thémanites, régna à sa place.
\VS{46}Huscham mourut, et Hadad, fils de Bedad, régna à sa place. C’est lui qui frappa Madian dans les champs de Moab. Le nom de sa ville était Avith.
\VS{47}Hadad mourut, et Samla de Masréka, régna à sa place.
\VS{48}Samla mourut, et Saül de Rehoboth, sur le fleuve, régna à sa place.
\VS{49}Saül mourut, et Baal-Hanan, fils de Acbor, régna à sa place.
\VS{50}Baal-Hanan mourut, et Hadad régna à sa place. Le nom de sa ville était Pahi, et le nom de sa femme Mehéthabeel, qui était fille de Mathred, et petite- fille de Mézahab.
\VS{51}Enfin Hadad mourut. Ensuite vinrent les chefs d'Edom, le chef Thimna, le chef Alja, le chef Jetheth.
\VS{52}Le chef Oholibama, le chef Ela, le chef Pinon.
\VS{53}Le chef Kenaz, le chef Théman, le chef Mibtsar.
\VS{54}Le chef Magdiel, et le chef Iram. Ce sont là les chefs d'Edom.
\Chap{2}
\TextTitle{Les douze fils de Jacob (Israël)\FTNTT{Ge. 29:31-35 ; 30:6-24 ; 35:16-18}}
\VerseOne{}Voici les fils d'Israël : Ruben, Siméon, Lévi, Juda, Issacar, Zabulon,
\VS{2}Dan, Joseph, Benjamin, Nephthali, Gad et Aser.
\TextTitle{Les descendants de Juda jusqu’aux fils d’Hetsron\FTNTT{Ge. 46:12 ; No 26:19-22}}
\VS{3}Les fils de Juda furent  Er, Onan, et Schéla. Ces trois lui naquirent de la fille de Schua, la Cananéenne. Mais Er, premier-né de Juda, fut méchant aux yeux de Yahweh, qui le fit mourir.
\VS{4}Et Tamar, belle-fille de Juda, lui enfanta Pérets et Zérach. Tous les fils de Juda furent cinq.
\VS{5}Les fils de Pérets furent Hetsron et Hamul.
\VS{6}Et les fils de Zérach furent Zimri, Ethan, Héman, Calcol et Dara, cinq en tout.
\VS{7}Carmi n'eut point d’autre fils qu'Acar qui troubla Israël et qui pécha en prenant de l'interdit.
\VS{8}Ethan eut un seul fils : Azaria.
\VS{9}Les fils qui naquirent à Hetsron furent Jerachmeel, Ram et Kelubaï.
\TextTitle{Les descendants de Ram jusqu’à David\FTNTT{Ru. 4:17-22}}
\VS{10}Ram engendra Amminadab et Amminadab engendra Nachschon, chef des fils de Juda.
\VS{11}Nachschon engendra Salma et Salma engendra Boaz.
\VS{12}Boaz engendra Obed et Obed engendra Isaï.
\VS{13}Isaï engendra son premier-né Eliab, le second Abinadab, le troisième Schimea,
\VS{14}le quatrième Nethaneel, le cinquième Raddaï,
\VS{15}le sixième Otsem, et le septième, David.
\VS{16}Tseruja et Abigaïl furent leurs soeurs. Tseruja eut trois fils : Abischaï, Joab, et Asaël.
\VS{17}Abigaïl enfanta Amasa, dont le père fut Jéther l’Ismaélite.
\TextTitle{Les descendants de Caleb}
\VS{18}Or Caleb, fils de Hetsron, eut des enfants d’Azuba sa femme, et aussi de Jerioth; et ses fils furent Jéscher, Schobab et Ardon.
\VS{19}Azuba mourut, et Caleb prit pour femme Ephrath, qui lui enfanta Hur.
\VS{20}Hur engendra Uri, et Uri engendra Betsaleel.
\VS{21}Après cela, Hetsron vint vers la fille de Makir, père de Galaad, et la prit pour sa femme ; il était âgé de soixante ans, et elle lui enfanta Segub.
\VS{22}Segub engendra Jaïr, qui eut vingt-trois villes au pays de Galaad.
\VS{23}Il prit sur Gueschur et sur la Syrie les bourgades de Jaïr, et Kenath, avec les villes de son ressort, au nombre de soixante. Tous ceux-là furent fils de Makir, père de Galaad.
\VS{24}Après la mort de Hetsron, à Caleb-Ephratha, la femme de Hetsron, Abija, lui enfanta Aschchur, père de Tekoa.
\VS{25}Les fils de Jerachmeel, premier-né de Hetsron furent : Ram, son fils aîné, puis Buna, Oren et Otsem, nés d'Achija.
\VS{26}Jerachmeel eut aussi une autre femme, dont le nom était Athara, qui fut mère d'Onam.
\VS{27}Les fils de Ram, premier-né de Jerachmeel, furent Maats, Jamin et Eker.
\VS{28}Les fils  d'Onam furent  Schammaï et Jada; et les fils de Schammaï furent  Nadab et Abischur.
\VS{29}Le nom de la femme d'Abischur fut Abichaïl, qui lui enfanta Achban et Molid.
\VS{30}Les fils de Nadab furent Séled et Appaïm; mais Séled mourut sans fils.
\VS{31}Appaïm eut un seul fils : Jischeï. Jischeï eut un seul fils : Schéschan. Schéschan n'eut qu' Achlaï.
\VS{32}Les fils de Jada, frère de Schammaï, furent Jéther et Jonathan ; mais Jéther mourut sans fils.
\VS{33}Les fils de Jonathan furent Péleth et Zara ; ce furent là les fils de Jerachmeel.
\VS{34}Schéschan n'eut point de fils, mais des filles ; or il avait un serviteur Egyptien, dont le nom était Jarcha ;
\VS{35}Schéschan donna sa fille pour femme à Jarcha, son serviteur, et elle lui enfanta Attaï.
\VS{36}Attaï engendra Nathan, et Nathan engendra Zabad ;
\VS{37}Zabad engendra Ephlal; et Ephlal engendra Obed ;
\VS{38} Obed engendra Jéhu; Jéhu engendra Azaria;
\VS{39}Azaria engendra Halets;  Halets engendra Elasa;
\VS{40}Elasa engendra Sismaï; Sismaï engendra Schallum;
\VS{41}Schallum engendra Jekamja; Jekamja engendra Elischama.
\TextTitle{Les autres fils de Caleb}
\VS{42}Les fils de Caleb, frère de Jerachmeel, furent Méscha, son premier-né, qui fut le père de Ziph, et les fils de Maréscha, père d'Hébron.
\VS{43}Les fils d'Hébron furent  Koré, Thappuach, Rékem et Schéma.
\VS{44}Schéma engendra Racham, père de Jorkeam, et Rékem engendra Schammaï.
\VS{45}Le fils de Schammaï fut Maon. Maon fut père de Beth-Tsur.
\VS{46}Et Epha, concubine de Caleb, enfanta Haran, Motsa et Gazez; Haran aussi engendra Gazez.
\VS{47}Les fils de Jahdaï furent Réguem, Jotham, Guéschan, Péleth, Epha et Schaaph.
\VS{48}Maaca, la concubine de Caleb, enfanta Schéber et Tirchana.
\VS{49}La femme de Schaaph, père de Madmanna, enfanta Scheva, père de Macbéna, et le père de Guibea, et la fille de Caleb fut Acsa.
\TextTitle{Les descendants de Hur, fils de Caleb\FTNTT{v. 19 ; cp. 1 Ch. 4:1}}
\VS{50}Ceux-ci furent les fils de Caleb, fils de Hur, premier-né d'Ephrata : Schobal, père de Kirjath-Jearim.
\VS{51}Salma, père de Bethléhem, Hareph, père de Beth-Gader.
\VS{52}Schobal, père de Kirjath-Jearim, eut des fils : Haroé et Hatsi-Hammenuhoth.
\VS{53}Les familles de Kirjath-Jearim furent les Jéthriens, les Puthiens, les Schumathiens et les Mischraïens, desquels sont sortis les Tsoreathiens et les Eschthaoliens.
\VS{54}Les fils de Salma : Bethléhem et les Nethophatiens, Athroth-Beth-Joab, Hatsi-Hammanachthi et les Tsoreïens.
\VS{55}Et les familles des scribes, qui habitaient à Jaebets : Les Thireathiens, les Schimeathiens, les Sucathiens ; ce sont les Kéniens, qui sont sortis de Hamath père de Récab.
\Chap{3}
\TextTitle{Les fils de David\FTNTT{2 S. 3:2-5 ; 5:13-16}}
\VerseOne{}Voici les fils de David, qui lui naquirent à Hébron\FTNT{2 S. 3:2-5.}. Le premier-né fut Amnon, fils d' Achinoam de Jizreel; le second Daniel, d'Abigaïl de Carmel.
\VS{2}Le troisième, Absalom, fils de Maaca, fille de Talmaï, roi de Gueschur ; le quatrième, Adonija, fils de Haggith ;
\VS{3}le cinquième, Schephatia, d'Abithal; le sixième, Jithream, d'Egla sa femme.
\VS{4}Ces six lui naquirent à Hébron, où il régna sept ans et six mois ; puis il régna trente-trois ans à Jérusalem.
\VS{5}Ceux-ci lui naquirent à Jérusalem : Schimea, Schobab, Nathan et Salomon, tous quatre de Bath-Schua, fille d'Ammiel ;
\VS{6}et Jibhar, Elischama, Eliphéleth,
\VS{7}Noga, Népheg, Japhia,
\VS{8}Elischama, Eliada et Eliphéleth, qui sont neuf.
\VS{9}Ce sont tous des fils de David, outre les fils de ses concubines. Et Tamar était leur sœur.
\TextTitle{De Salomon à Sédécias}
\VS{10}Le fils de Salomon fut  Roboam. Abija, son fils ; Asa, son fils ; Josaphat, son fils ;
\VS{11}Joram, son fils ; Achazia, son fils ; Joas, son fils ;
\VS{12}Amatsia, son fils ; Azaria, son fils ; Jotham, son fils ;
\VS{13}Achaz, son fils ; Ezéchias, son fils ; Manassé, son fils ;
\VS{14}Amon, son fils ; Josias, son fils.
\VS{15}Les fils de Josias furent Jochanan, son premier-né ; le deuxième, Jojakim ; le troisième Sédécias ; le quatrième, Schallum.
\VS{16}Les fils de Jojakim furent  Jéconias, son fils, qui eut pour fils Sédécias.
\TextTitle{Les fils de Jéconias}
\VS{17}Quant aux fils de Jéconias, Assir qui fut emmené en captivité, Schealthiel fut son fils ;
\VS{18}dont les fils furent Malkiram, Pedaja, Schénatsar, Jekamia, Hoschama et Nedabia.
\VS{19}Les fils de Pedaja furent Zorobabel et Schimeï ; et les fils de Zorobabel furent Meschullam et Hanania ; et Schelomith était leur soeur.
\VS{20}De Meschullam, Haschuba, Ohel, Bérékia, Hasadia et Juschab-Hésed, en tout cinq.
\VS{21}Les fils de Hanania furent  Pelathia et Esaïe ; les fils de Rephaja, les fils d'Arnan, les fils d’Abdias et les fils de Schecania.
\VS{22}De Schecania naquit Schemaeja ; et les fils de Schemaeja, Hattusch, Jigueal, Bariach, Nearia, Schaphath, en tout six.
\VS{23}Les fils de Nearia furent trois : Eljoénaï, Ezéchias et Azrikam.
\VS{24}Et les fils d' Eljoénaï furent  sept : Hodavia, Eliaschib, Pelaja, Akkub, Jochanan, Delaja, et Anani.
\Chap{4}
\TextTitle{Les autres fils de Hur\FTNTT{1 Ch. 2:50}}
\VerseOne{}Les fils de Juda furent Pérets, Hetsron, Carmi, Hur et Schobal.
\VS{2}Reaja, fils de Schobal, engendra Jachath ; et Jachath engendra Achumaï et Lahad. Ce sont les familles des Tsoreathiens.
\VS{3}Voici les descendants du père d’Etham : Jizreel, Jischma, et Jidbasch ; le nom de leur soeur était Hatselelponi.
\VS{4}Penuel, père de Guedor, et Ezer, père de Huscha, sont les fils de Hur, premier-né d'Ephrata, père de Bethléhem.
\TextTitle{Les descendants d’Aschchur\FTNTT{1 Ch. 2:24}}
\VS{5}Aschchur, père de Tekoa, eut deux femmes : Hélea et Naara.
\VS{6}Naara lui enfanta Achuzzam, Hépher, Thémeni et Achaschthari. Ce sont là les fils de Naara.
\VS{7}Les fils de Hélea furent Tséreth, Tsochar et Ethnan.
\VS{8}Kots engendra Anub, Hatsobéba et les familles Acharchel, fils de Harum.
\TextTitle{Jaebets invoque Dieu}
\VS{9}Jaebets était plus honoré que ses frères ; sa mère lui avait donné le nom de Jaebets, parce que, dit-elle, je l'ai enfanté avec douleur.
\VS{10}Jaebets invoqua le Dieu d'Israël, en disant : Ô, si tu me bénis abondamment et que tu étends mes limites, si ta main est avec moi, et si tu me mets à l'abri du mal, en sorte que je ne sois pas dans l’affliction !... Et Dieu lui accorda ce qu'il avait demandé.
\TextTitle{Les fils de Juda et de Caleb}
\VS{11}Kelub, frère de Schucha, engendra Mechir, qui fut père d'Eschthon.
\VS{12}Et Eschthon engendra la maison de Rapha, Paséach et Thechinna, père de la ville de Nachasch ; ce sont là les gens de Réca.
\VS{13}Les fils de Kenaz furent Othniel et Seraja. Et le fils d’Othniel, Hathath.
\VS{14}Meonothaï engendra Ophra ; et Seraja engendra Joab, père de la vallée des ouvriers ; car ils étaient ouvriers.
\VS{15}Les fils de Caleb, fils de Jephunné, furent Iru, Ela et Naam, et les fils d'Ela, Kenaz.
\VS{16}Les fils de Jehalléleel furent Ziph, Zipha, Thirja, et Asareel.
\VS{17}Les fils d'Esdras furent Jéther, Méred, Epher, et Jalon ; et la femme de Méred enfanta Miriam, Schammaï, et Jischbach, père d' Eschthemoa.
\VS{18}Sa femme, la Juive, enfanta Jéred, père de Guedor ;  Héber, père de Soco ;  Jekuthiel, père de Zanoach. Ceux-là sont les fils de Bithja, fille de Pharaon, que Méred prit pour femme.
\VS{19}Les fils de la femme de Hodija, soeur de Nacham : Le père de Kehila, le Garmien, et Eschthemoa, le Maacathien.
\VS{20}Et les fils de Simon furent Amnon, Rinna, Ben-Hanan et Thilon. Les fils de Jischeï furent Zocheth et Ben-Zocheth.
\TextTitle{Les fils de Juda par Schéla\FTNTT{1 Ch. 2:3}}
\VS{21}Les fils de Schéla, fils de Juda, furent Er, père de Léca ; Laeda, père de Maréscha ; et les familles de la maison où l'on travaille le byssus, qui sont de la maison d'Aschbéa.
\VS{22}Jokim, et les gens de Cozéba, Joas et Saraph dominèrent sur Moab, avec Jaschubi-Léchem. Mais ce sont là des choses anciennes.
\VS{23}C’étaient les potiers et les habitants des plantations  et des parcs. Ils cohabitaient là chez le roi et oeuvraient pour lui.
\TextTitle{Les descendants de Siméon ; leurs terres et leurs conquêtes}
\VS{24}Les fils de Siméon furent Nemuel, Jamin, Jarib, Zérach et Saül.
\VS{25}Schallum son fils, Mibsam son fils, et Mischma son fils.
\VS{26}Les fils de Mischma furent Hammuel son fils, Zaccur son fils, et Schimeï son fils.
\VS{27}Schimeï eut seize fils et six filles ; mais ses frères n'eurent pas beaucoup de fils, et toute leur famille ne put être aussi nombreuse que celle des fils de Juda.
\VS{28}Ils habitèrent à Beer-Schéba, à Molada, à Hatsar-Schual,
\VS{29}à Bilha, à Etsem, à Tholad,
\VS{30}à Bethuel, à Horma, à Tsiklag,
\VS{31}à Beth-Marcaboth, à Hatsar-Susim, à Beth-Bireï, et à Schaaraïm. Ce furent là leurs villes jusqu'au temps où David devint roi.
\VS{32}Leurs villages furent Etham, Aïn, Rimmon, Thoken, et Aschan, cinq villes ;
\VS{33}et tous leurs villages, qui étaient autour de ces villes-là, jusqu'à Baal. Ce sont là leurs habitations et leur généalogie :
\VS{34}Meschobab, Jamlec, Joscha fils d'Amatsia ;
\VS{35}Joël, Jéhu fils de Joschibia, fils de Seraja, fils d'Asiel ;
\VS{36}Eljoénaï, Jaakoba, Jeschochaja, Asaja, Adiel, Jesimiel, Benaja,
\VS{37}Ziza, fils de Schipheï, fils d'Allon, fils de Jedaja, fils de Schimri, fils de Schemaeja.
\VS{38}Ceux-là furent désignés pour être des chefs dans leurs familles, et les maisons de leurs pères s’étendirent abondamment.
\VS{39}Et ils allèrent pour entrer dans Guedor, jusqu'à l'orient de la vallée, cherchant des pâturages pour leurs troupeaux.
\VS{40}Ils trouvèrent des pâturages gras et bons, et un pays spacieux, paisible et fertile ; car ceux qui habitaient là auparavant étaient descendus de Cham.
\VS{41}Ceux-ci, dont les noms sont inscrits, vinrent du temps d'Ezéchias, roi de Juda, et abattirent leurs tentes ; et quant aux Maonites qui s’y trouvaient, et les détruisirent à la façon de l'interdit jusqu'à ce jour, et y habitèrent à leur place, car il y avait là des pâturages pour leurs troupeaux.
\VS{42}Cinq cents hommes d'entre eux, c'est-à-dire des fils de Siméon, s'en allèrent à la montagne de Séir, et ils avaient à leur tête Pelathia, Nearia, Rephaja, et Uziel, fils de Jischeï ;
\VS{43}ils frappèrent le reste des réchappés d'Amalek, et ils demeurèrent là jusqu'à ce jour.
\Chap{5}
\TextTitle{Les descendants de Ruben jusqu’au temps des captivités}
\VerseOne{}Les fils de Ruben, le premier-né d'Israël. - Car il était le premier-né ; mais après qu'il eut souillé le lit de son père, son droit d'aînesse fut donné aux fils de Joseph fils d'Israël ; cependant, Joseph ne fut pas enregistré  dans la généalogie selon le droit d'aînesse.
\VS{2}Car Juda fut le plus puissant parmi ses frères, et de lui est issu un chef ; mais le droit d'aînesse est à Joseph.
\VS{3}Les fils de Ruben, premier-né d'Israël, furent donc Hénoc, Pallu, Hetsron, et Carmi.-
\VS{4}Les fils de Joël furent  Schemaeja, son fils ; Gog, son fils ; Schimeï, son fils ;
\VS{5}Michée, son fils ; Reaja, son fils ; Baal,  son fils ;
\VS{6}Beéra, son fils, qui fut emmené captif par Tilgath- Pilnéser, roi d’Assyrie ; c'est lui qui était le principal chef des Rubénites.
\VS{7}Ses frères, selon leurs familles, d’après le registre généalogique et selon leurs générations, avaient pour chefs Jeïel et Zacharie.
\VS{8}Béla, fils d’Azaz, fils de Schéma, fils de Joël, habitait depuis Aroër jusqu'à Nebo et Baal-Meon.
\VS{9}Ensuite, il habita du côté de l’orient jusqu'à l'entrée du désert, depuis le fleuve d'Euphrate; car son bétail s'était multiplié dans le pays de Galaad.
\VS{10}Du temps de Saül, ils firent la guerre contre les Hagaréniens, qui tombèrent par leurs mains, et ils habitèrent dans leurs tentes, dans toute la partie orientale de Galaad.
\TextTitle{Les descendants de Gad et leurs villes}
\VS{11}Les fils de Gad habitaient près d'eux, au pays de Basan, jusqu'à Salca.
\VS{12}Joël fut le premier chef, et Schapham le deuxième après lui, puis Jaenaï, puis Schaphath en Basan.
\VS{13}Et leurs frères, selon la maison de leurs pères, furent sept : Micaël, Meschullam, Schéba, Joraï, Jaecan, Zia, et Eber.
\VS{14}Ceux-ci furent les fils d'Abichaïl, fils de Huri, fils de Jaroach, fils de Galaad, fils de Micaël, fils de Jeschischaï, fils de Jachdo, fils de Buz.
\VS{15}Achi, fils d'Abdiel, fils de Guni, fut le chef de la maison de leurs pères.
\VS{16}Ils habitèrent en Galaad, et en Basan, dans les villes de son ressort, et dans toutes les banlieues de Saron, jusqu’à leurs limites.
\VS{17}Tous ceux-ci furent inscrits dans la généalogie du temps de Jotham, roi de Juda, et du temps de Jéroboam, roi d'Israël.
\TextTitle{Captivité de Ruben, Gad et la demi-tribu de Manassé}
\VS{18}Il y eut des fils de Ruben, et de ceux de Gad, et de la demi-tribu de Manassé, d'entre les vaillants hommes, portant le bouclier et l'épée, tirant de l'arc, et exercés à la guerre, quarante-quatre mille sept cent soixante, en état d’aller à l’armée.
\VS{19}Ils firent la guerre contre les Hagaréniens, contre Jethur, Naphisch, et Nodab.
\VS{20}Et ils reçurent du secours contre eux, de sorte que les Hagaréniens, et tous ceux qui étaient avec eux furent livrés entre leurs mains, parce qu'ils crièrent à Dieu dans la bataille, et il les exauça parce qu'ils avaient mis leur confiance en lui.
\VS{21}Ainsi ils prirent leurs troupeaux, consistant en cinquante mille chameaux, deux cent cinquante mille brebis, deux mille ânes, avec cent mille personnes ;
\VS{22}car il y eut beaucoup de morts, parce que la bataille venait de Dieu.  Ils habitèrent là, à leur place, jusqu'au temps de la déportation.
\VS{23}Les fils de la demi-tribu de Manassé habitèrent aussi dans ce pays-là, et s'étendirent depuis Basan jusqu'à Baal-Hermon et à Sénir, à la montagne d’Hermon ; ils étaient nombreux.
\VS{24}Et voici les chefs de la maison de leurs pères : Epher, Jischeï, Eliel, Azriel, Jérémie, Hodavia, et Jachdiel, hommes forts et vaillants, gens de réputation, et chefs des maisons de leurs pères.
\VS{25}Mais ils péchèrent contre le Dieu de leurs pères, et se prostituèrent après les dieux des peuples du pays, que Dieu avait détruits devant eux.
\VS{26}Le Dieu d'Israël excita l'esprit de Pul, roi d’Assyrie, et l'esprit de Thilgath-Pilnéser, roi d’Assyrie,  qui emmena en captivité les Rubénites, les Gadites et la demi-tribu de Manassé, et les emmena à Chalach, à Chabor, à Hara, et au fleuve de Gozan, où ils sont restés jusqu'à ce jour.
\Chap{6}
\TextTitle{Les fils de Kehath le Lévite, jusqu'à la captivité}
\VerseOne{}Les fils de Lévi furent Guerschon, Kehath et Merari.
\VS{2}Les fils de Kehath furent  Amram, Jitsehar, Hébron, et Uziel.
\VS{3}Et les fils d'Amram furent Aaron, Moïse et Marie. Les fils d'Aaron furent  Nadab, Abihu, Eléazar et Ithamar.
\VS{4}Eléazar engendra Phinées, et Phinées engendra Abischua.
\VS{5}Abischua engendra Bukki, et Bukki engendra Uzzi.
\VS{6}Uzzi engendra Zerachja, et Zerachja engendra Merajoth.
\VS{7}Merajoth engendra Amaria, et Amaria engendra Achithub.
\VS{8}Achithub engendra Tsadok, et Tsadok engendra Achimaats.
\VS{9}Achimaats engendra Azaria, et Azaria engendra Jochanan.
\VS{10}Jochanan engendra Azaria, qui exerça la sacrificature au temple que Salomon bâtit à Jérusalem.
\VS{11}Azaria engendra Amaria, et Amaria engendra Achithub.
\VS{12}Achithub engendra Tsadok, et Tsadok engendra Schallum.
\VS{13}Schallum engendra Hilkija, et Hilkija engendra Azaria.
\VS{14}Azaria engendra Seraja, et Seraja engendra Jehotsadak,
\VS{15}Jehotsadak s'en alla, quand Yahweh emmena en exil Juda et Jérusalem par le moyen de Nebucadnetsar.
\TextTitle{Les fils de Guerschon, Kehath et Mérari}
\VS{16}Les fils de Lévi  furent donc  Guerschon, Kehath et Merari.
\VS{17}Voici les noms des fils de Guerschon : Libni et Schimeï.
\VS{18}Les fils de Kehath furent Amram, Jitsehar, Hébron et Uziel.
\VS{19}Les fils de Merari furent  Machli et Muschi. Ce sont là les familles des Lévites, selon les maisons de leurs pères.
\VS{20}De Guerschon, Libni son fils, Jachath son fils, Zimma son fils,
\VS{21}Joach son fils, Iddo son fils, Zérach son fils, Jeathraï son fils.
\VS{22}Des fils de Kehath, Amminadab son fils, Koré son fils, Assir son fils,
\VS{23}Elkana son fils, Ebjasaph son fils, Assir son fils,
\VS{24}Thachath son fils, Uriel son fils, Ozias son fils, et Saül son fils.
\VS{25}Les fils d’Elkana furent  Amasaï, Achimoth ;
\VS{26}Elkana, son fils ; les fils d’Elkana furent Elkana-Tsophaï, son fils, Nachath son fils,
\VS{27}Eliab son fils, Jerocham son fils, Elkana son fils.
\VS{28}Quant aux fils de Samuel, fils d'Elkana, son fils aîné fut Vaschni, puis Abija.
\VS{29}Les fils de Merari furent Machli, Libni son fils, Schimeï son fils, Uzza son fils,
\VS{30}Schimea son fils, Hagguija son fils, Asaja son fils.
\TextTitle{Les chefs des chantres}
\VS{31}Or voici ceux que David établit pour la direction de la musique dans la maison de Yahweh, depuis que l’arche fut en lieu de repos.
\VS{32}Ils faisaient le service comme chantres devant le tabernacle, devant la tente d'assignation, jusqu'à ce que Salomon eût bâti la maison de Yahweh à Jérusalem ; ils continuèrent dans leur service selon l'ordonnance qui était prescrite. Voici ceux qui firent le service avec leurs fils : D'entre les fils des Kehathites, Héman le chantre, fils de Joël, fils de Samuel,
\VS{34}fils d'Elkana, fils de Jerocham, fils d’Eliel, fils de Thoach,
\VS{35}fils de Tsuph, fils d'Elkana, fils de Machath, fils de Amasaï,
\VS{36}fils d'Elkana, fils de Joël, fils d’Azaria, fils de Sophonie,
\VS{37}fils de Thachath, fils d'Assir, fils de Ebjasaph, fils de Koré,
\VS{38}fils de Jitsehar, fils de Kehath, fils de Lévi, fils d'Israël.
\VS{39}Son frère Asaph, qui se tenait à sa droite. Asaph était fils de Bérékia, fils de Schimea,
\VS{40}fils de Micaël, fils de Baaséja, fils de Malkija,
\VS{41}fils d’Ethni, fils de Zérach, fils d’Adaja,
\VS{42}fils d'Ethan, fils de Zimma, fils de Schimeï,
\VS{43}fils de Jachath, fils de Guerschon, fils de Lévi.
\VS{44}Les fils de Merari, leurs frères étaient à la gauche ; à savoir Ethan, fils de Kischi, fils d’Abdi, fils de Malluc,
\VS{45}fils de Haschabia, fils d'Amatsia, fils de Hilkija,
\VS{46}fils d'Amtsi, fils de Bani, fils de Schémer,
\VS{47}fils de Machli, fils de Muschi, fils de Merari, fils de Lévi.
\VS{48}Et leurs autres frères Lévites furent ordonnés pour tout le service du tabernacle de la maison de Dieu.
\VS{49}Mais Aaron et ses fils offraient les parfums sur l'autel de l'holocauste et sur l'autel des parfums ; pour tout ce qu'il fallait faire dans le Saint des saints, et pour faire propitiation pour Israël ; comme Moïse, serviteur de Dieu, l'avait commandé.
\TextTitle{Les sacrificateurs d’Aaron à Achimaats}
\VS{50}Voici les fils d'Aaron : Eléazar son fils, Phinées son fils, Abischua son fils,
\VS{51}Bukki son fils, Uzzi son fils, Zerachja son fils,
\VS{52}Merajoth son fils, Amaria son fils, Achithub son fils,
\VS{53}Tsadok son fils, Achimaats son fils.
\TextTitle{Villes des fils d’Aaron et des Lévites}
\VS{54}Voici leurs lieux d’habitation, selon leurs demeures et leurs limites. Aux fils d'Aaron, qui appartiennent à la famille des Kehathites, désignés par le sort,
\VS{55}on leur donna Hébron dans le pays de Juda, et sa banlieue tout autour.
\VS{56}Mais on donna à Caleb, fils de Jephunné, le territoire de la ville et ses villages.
\VS{57}On donna donc aux fils d'Aaron, d'entre les villes de refuge, Hébron, Libna et sa banlieue, Jatthir et Eschthemoa, avec leurs banlieues,
\VS{58}Hilen, avec sa banlieue, Debir avec sa banlieue,
\VS{59}Aschan avec sa banlieue, et Beth-Schémesch avec sa banlieue.
\VS{60}De la tribu de Benjamin, Guéba, avec sa banlieue, Allémeth avec sa banlieue, et Anathoth avec sa banlieue. Toutes leurs villes, selon leurs familles, étaient treize en nombre.
\VS{61}On donna au reste des fils de Kehath, par le sort, dix villes des familles de la demi-tribu, c'est-à-dire de la demi-tribu de Manassé.
\VS{62}Et aux fils de Guerschon, selon leurs familles, de la tribu d'Issacar, de la tribu d'Aser, de la tribu de Nephthali, et de la tribu de Manassé en Basan, treize villes.
\VS{63}Aux fils de Merari, selon leurs familles, par le sort, douze villes, de la tribu de Ruben, de la tribu de Gad, et de la tribu de Zabulon.
\VS{64}Ainsi, les fils d’Israël donnèrent aux Lévites ces villes-là, avec leurs banlieues.
\VS{65}Et ils donnèrent, par le sort, de la tribu des fils de Juda, de la tribu des fils de Siméon, et de la tribu des fils de Benjamin, ces villes qu’ils désignèrent par leurs noms.
\VS{66}Et pour les autres familles des fils de Kehath, ils eurent pour territoire des villes de la tribu d'Ephraïm.
\VS{67}Car on leur donna entre les villes de refuge, Sichem avec sa banlieue, dans la montagne d'Ephraïm, Guézer avec sa banlieue,
\VS{68}Jokmeam avec sa banlieue, Beth-Horon avec sa banlieue,
\VS{69}Ajalon avec sa banlieue, et Gath-Rimmon avec sa banlieue.
\VS{70}De la demi-tribu de Manassé, Aner avec sa banlieue, et Bileam avec sa banlieue, on donna ces villes-là aux familles qui restaient des fils de Kehath.
\VS{71}Aux fils de Guerschon, on donna, des familles de la demi-tribu de Manassé, Golan en Basan avec sa banlieue, et Aschtaroth, avec sa banlieue.
\VS{72}De la tribu d'Issacar, Kédesch avec sa banlieue, Dobrath avec sa banlieue,
\VS{73}Ramoth avec sa banlieue, et Anem avec sa banlieue.
\VS{74}Et de la tribu d'Aser, Maschal, avec sa banlieue, Abdon, avec sa banlieue,
\VS{75}Hukok avec sa banlieue, et Rehob avec sa banlieue.
\VS{76}De la tribu de Nephthali, Kédesch en Galilée avec sa banlieue, Hammon avec sa banlieue, et Kirjathaïm avec sa banlieue.
\VS{77}Aux fils de Merari, qui étaient le reste d'entre les Lévites, on donna, de la tribu de Zabulon, Rimmono avec sa banlieue, et Thabor avec sa banlieue.
\VS{78}Au-delà du Jourdain, vis-à-vis de Jéricho, vers l’orient du Jourdain, de la tribu de Ruben, Betser au désert avec sa banlieue, Jahtsa avec sa banlieue,
\VS{79}Kedémoth avec sa banlieue, et Méphaath avec sa banlieue.
\VS{80}De la tribu de Gad, Ramoth en Galaad avec sa banlieue, Mahanaïm avec sa banlieue,
\VS{81}Hesbon avec sa banlieue, et Jaezer avec sa banlieue.
\Chap{7}
\TextTitle{Les descendants d'Issacar}
\VerseOne{}Les fils d'Issacar furent  Thola, Pua, Jaschub et Schimron, quatre.
\VS{2}Les fils de Thola furent Uzzi, Rephaja, Jeriel, Jachmaï, Jibsam et Samuel, chefs des maisons de leurs pères qui étaient de Thola, gens forts et vaillants dans leurs générations ; leur nombre, aux jours de David, était de vingt-deux mille six cents.
\VS{3}Le fils d’Uzzi : Jizrachja. Et les fils de Jizrachja : Micaël, Abdias, Joël, et Jischija, en tout cinq chefs.
\VS{4}Ils avaient avec eux, selon leurs générations, et selon les familles de leurs pères, trente-six mille hommes de troupe, armés pour la guerre, car ils eurent plusieurs femmes et plusieurs fils.
\VS{5}Leurs frères selon toutes les familles d'Issacar, hommes forts et vaillants, étant comptés tous selon leur généalogie, furent quatre-vingt-sept mille.
\TextTitle{Les descendants de Benjamin}
\VS{6}Les fils de Benjamin furent Béla, Béker et Jediaël, trois\FTNT{Benjamin avait encore d’autres fils (Ge. 46:21 ; No. 26:38-41 ; 1 Ch. 8:1-2).}.
\VS{7}Les fils de Béla furent  Etsbon, Uzzi, Uziel, Jerimoth et Iri, cinq chefs des familles de leurs pères, hommes forts et vaillants, et enregistrés dans la généalogie au nombre de vingt-deux mille trente-quatre.
\VS{8}Les fils de Béker furent  Zemira, Joasch, Eliézer, Eljoénaï, Omri, Jerémoth, Abija, Anathoth, et Alameth, tous ceux-là furent fils de Béker,
\VS{9}et enregistrés dans les généalogies, selon leurs générations, comme chefs des familles de leurs pères, hommes forts et vaillants au nombre de vingt mille deux cents.
\VS{10}Jediaël eut pour fils Bilhan. Et les fils de Bilhan furent Jeusch, Benjamin, Ehud, Kenaana, Zéthan, Tarsis, et Achischachar.
\VS{11}Tous ceux-là furent fils de Jediaël, comme chefs des familles de leurs pères, dix-sept mille deux cents hommes forts et vaillants, en état de porter les armes et d’aller à la guerre.
\VS{12}Schuppim et Huppim furent  des fils d’Ir ; et Huschim fut fils d'Acher.
\TextTitle{Les descendants de Nephtali}
\VS{13}Les fils de Nephthali furent Jahtsiel, Guni, Jetser, et Schallum, fils de Bilha.
\TextTitle{Les descendants de Manassé}
\VS{14}Les fils de Manassé : Asriel, qu’enfanta sa concubine Araméenne. Elle enfanta Makir, père de Galaad.
\VS{15}Makir prit une femme de la parenté de Huppim et de Schuppim ; car ils avaient une sœur dont le nom était Maaca. Et le nom d'un des petits-fils de Galaad fut Tselophchad ; et Tselophchad eut des filles.
\VS{16}Maaca, femme de Makir, enfanta un fils et l'appela Péresch, et le nom de son frère Schéresch, dont les fils furent Ulam et Rékem.
\VS{17}Le fils d'Ulam fut Bedan. Ce sont là les fils de Galaad, fils de Makir, fils de Manassé.
\VS{18}Mais sa soeur Hammoléketh enfanta Ischhod, Abiézer et Machla.
\VS{19}Les fils de Schemida furent Achjan, Sichem, Likchi et Aniam.
\TextTitle{Les descendants d'Ephraïm et leurs villes}
\VS{20}Or les fils d'Ephraïm furent  Schutélach ; Béred son fils, Tachath son fils, Eleada son fils, Tachath son fils.
\VS{21}Zabad son fils, Schutélach son fils, Ezer, et Elead. Mais ceux de Gath, nés dans le pays, les mirent à mort, parce qu'ils étaient descendus pour prendre leur bétail.
\VS{22}Ephraïm, leur père, fut dans le deuil plusieurs jours, et ses frères vinrent pour le consoler.
\VS{23}Puis il alla vers sa femme, qui conçut et enfanta un fils ; et elle l'appela du nom de Beria, parce que le malheur était dans sa maison.
\VS{24}Il eut pour fille Schééra, qui bâtit la basse et la haute Beth-Horon, et Uzzen-Schééra.
\VS{25}Son fils fut  Réphach, puis Réscheph, et Thélach son fils, Thachan son fils,
\VS{26}Laedan son fils, Ammihud son fils, Elischama son fils,
\VS{27}Nun son fils, Josué son fils.
\VS{28}Ils possédaient et habitaient Béthel ainsi que les villes de son ressort ; à l’orient Naaran, à l’occident Guézer, avec les villes de son ressort, et Sichem avec les villes de son ressort, jusqu'à Gaza avec les villes de son ressort.
\VS{29}Les lieux qui étaient aux fils de Manassé furent  Beth-Schean avec les villes de son ressort, Thaanac avec les villes de son ressort, Meguiddo avec les villes de son ressort, et Dor avec les villes de son ressort. Les fils de Joseph, fils d'Israël, habitèrent dans ces villes.
\TextTitle{Les descendants d'Aser}
\VS{30}Les fils d’Aser furent Jimna, Jischva, Jischvi, Beria, et Sérach leur soeur.
\VS{31}Les fils de Beria furent Héber et Malkiel, qui fut père de Birzavith.
\VS{32}Héber engendra Japhleth, Schomer, Hotham, et Schua leur soeur.
\VS{33}Les fils de Japhleth furent Pasac, Bimhal, et Aschvath. Ce sont là les fils de Japhlet.
\VS{34}Et les fils de Schamer furent Achi, Rohega, Hubba et Aram.
\VS{35}Les fils d'Hélem, son frère, furent  Tsophach, Jimna, Schélesch et Amal.
\VS{36}Les fils de Tsophach furent Suach, Harnépher, Schual, Béri, Jimra,
\VS{37}Betser, Hod, Schamma, Schilscha, Jithran, et Beéra.
\VS{38}Les fils de Jéther furent Jephunné, Pispa et Ara.
\VS{39}Les fils d'Ulla furent Arach, Hanniel et Ritsja.
\VS{40}Tous ceux-là furent fils d'Aser, chefs des maisons de leurs pères, gens d'élite, forts et vaillants, chefs des princes, enregistrés au nombre de vingt-six mille hommes, en état de porter les armes et d’aller en guerre.
\Chap{8}
\TextTitle{Les descendants de Benjamin}
\VerseOne{}Benjamin engendra Béla, qui fut son premier-né, Aschbel le deuxième, Achrach le troisième,
\VS{2}Nocha le quatrième, et Rapha le cinquième.
\VS{3}Les fils de Béla furent Addar, Guéra, Abihud,
\VS{4}Abischua, Naaman, Achoach,
\VS{5}Guéra, Schephuphan et Huram.
\VS{6}Voici les fils d'Echud, qui étaient chefs des maisons des pères des habitants de Guéba, et qui les transportèrent à Manachath :
\VS{7}Naaman, Achija, et Guéra. Guéra, qui les transporta et qui après engendra Uzza et Achichud.
\VS{8}Or Schacharaïm eut des enfants au pays de Moab, après avoir renvoyé Huschim et Baara, ses femmes.
\VS{9}Il engendra, de Hodesch sa femme, Jobab, Tsibja, Méscha, Malcam,
\VS{10}Jeuts, Schocja et Mirma. Ce sont là ses fils, chefs des pères.
\VS{11}Mais de Huschim, il engendra Abithub, Elpaal.
\VS{12}Les fils d'Elpaal furent Eber, Mischeam, et Schémer, qui bâtit Ono, Lod et les villes de son ressort.
\VS{13}Et Beria et Schéma furent chefs des pères des habitants d'Ajalon ; ils mirent en fuite les habitants de Gath.
\VS{14}Achjo, Schaschak, Jerémoth,
\VS{15}Zebadja, Arad, Eder,
\VS{16}Micaël, Jischpha, et Jocha, fils de Beria.
\VS{17}Zebadja, Meschullam, Hizki, Héber,
\VS{18}Jischmeraï, Jizlia, et Jobab, fils d' Elpaal.
\VS{19}Jakim, Zicri, Zabdi,
\VS{20}Eliénaï, Tsilthaï, Eliel,
\VS{21}Adaja, Beraja, et Schimrath, fils de Schimeï.
\VS{22}Jischpan, Eber, Eliel,
\VS{23}Abdon, Zicri, Hanan,
\VS{24}Hanania, Elam, Anthothija,
\VS{25}Jiphdeja et Penuel, fils de Schaschak.
\VS{26}Schamscheraï, Schecharia, Athalia,
\VS{27}Jaaréschia, Elija, et Zicri, fils de Jerocham.
\VS{28}Ce sont là les chefs des pères, selon leurs générations ; et ils habitèrent à Jérusalem.
\TextTitle{Les fils du père de Gabaon, ascendant de Saül}
\VS{29}Le père de Gabaon habita à Gabaon, sa femme avait pour nom Maaca.
\VS{30}Son fils premier-né fut Abdon, puis Tsur, Kis, Baal, Nadab,
\VS{31}Guedor, Achjo, et Zéker.
\VS{32}Mikloth engendra Schimea. Ils habitèrent aussi vis-à-vis de leurs frères à Jérusalem, avec leurs frères.
\VS{33}Ner engendra Kis, et Kis engendra Saül, et Saül engendra Jonathan, Malki-Schua, Abinadab, et Eschbaal.
\VS{34}Le fils de Jonathan fut  Merib-Baal ; et Merib-Baal engendra Michée.
\VS{35}Les fils de Michée furent Pithon, Mélec, Thaeréa, et Achaz.
\VS{36}Achaz engendra Jehoadda ; et Jehoadda engendra Alémeth, Azmaveth et Zimri ; Zimri engendra Motsa.
\VS{37}Motsa engendra Binea, qui eut pour fils Rapha, qui eut pour fils Eleasa, qui eut pour fils Atsel.
\VS{38}Atsel eut six fils, dont les noms sont : Azrikam, Bocru, Ismaël, Schearia, Abdias, et Hanan ; tous ceux-là furent fils d'Atsel.
\VS{39}Les fils d'Eschek, son frère, furent  Ulam son premier-né, Jéusch le second, Eliphéleth le troisième.
\VS{40}Et les fils d'Ulam furent des hommes forts et vaillants, tirant bien de l'arc, et ils eurent beaucoup de fils et de petits-fils, jusqu'à cent cinquante ; tous des fils de Benjamin.
\Chap{9}
\TextTitle{Les habitants de Jérusalem}
\VerseOne{}Ainsi, tous ceux d'Israël furent enregistrés par généalogie et inscrits dans le livre des rois d'Israël. Et ceux de Juda furent emmenés en captivité à Babylone à cause de leurs péchés\FTNT{La captivité babylonienne voir 2 R. 24 et 25.}.
\VS{2}Mais ce sont ici les premiers qui habitèrent dans leurs possessions, et dans leurs villes, tant d'Israël que des sacrificateurs, des Lévites, et des Néthiniens.
\VS{3}A Jérusalem habitaient les fils de Juda, les fils de Benjamin, et les fils d'Ephraïm et de Manassé.
\VS{4}Uthaï, fils d'Ammihud, fils d'Omri, fils d'Imri, fils de Bani, des fils de Pérets, fils de Juda.
\VS{5}Des Schilonites, Asaja le premier-né, et ses fils.
\VS{6}Des fils de Zérach, Jeuel, et ses frères, six cent quatre-vingt-dix.
\VS{7}Des fils de Benjamin, Sallu fils de Meschullam, fils de Hodavia, fils d'Assenua.
\VS{8}Jibneja, fils de Jerocham, et Ela fils d’Uzzi, fils de Micri ; et Meschullam fils de Schephathia, fils de Reuel, fils de Jibnija.
\VS{9}Leurs frères, selon leurs générations, furent neuf cent cinquante-six. Tous ces hommes-là furent chefs des pères dans les maisons de leurs  pères.
\VS{10}Des sacrificateurs : Jedaeja, Jehojarib, et Jakin.
\VS{11}Azaria fils de Hilkija, fils de Meschullam, fils de Tsadok, fils de Merajoth, fils d' Achithub, intendant de la maison de Dieu.
\VS{12}Adaja, fils de Jerocham, fils de Paschhur, fils de Malkija; et Maesaï, fils d'Adiel, fils de Jachzéra, fils de Meschullam, fils de Meschillémith, fils d'Immer.
\VS{13}Leurs frères, chefs de la maison de leurs pères, mille sept cent soixante hommes, forts et vaillants, occupés au service de la maison de Dieu.
\VS{14}Des Lévites : Schemaeja, fils de Haschub, fils d'Azrikam, fils de Haschabia, des fils de Merari,
\VS{15}Bakbakkar, Héresch, et Galal ; et Matthania, fils de Michée, fils de Zicri, fils d'Asaph,
\VS{16}Abdias fils de Schemaeja, fils de Galal, fils de Jeduthun ; et Bérékia, fils d'Asa, fils d'Elkana, qui habita dans les villages des Nethophathiens.
\VS{17}Et les portiers : Schallum, Akkub, Thalmon, et Achiman, et leurs frères ; mais Schallum était le chef.
\VS{18}Il l'a été jusqu'à maintenant, ayant la charge de la porte du roi vers l’orient. Ceux-là furent portiers pour le camp des fils de Lévi.
\VS{19}Schallum, fils de Koré, fils d'Ebiasaph, fils de Koré, et ses frères Koréites, de la maison de son père, remplissaient les fonctions de gardiens, gardant les seuils de la tente, comme leurs pères en avaient gardé l'entrée au camp de Yahweh ;
\VS{20}Phinées, fils d'Eléazar, fut établi chef sur eux en présence de Yahweh qui était avec lui.
\VS{21}Zacharie, fils de Meschélémia, était le portier de l'entrée de la tente d'assignation.
\VS{22}Ils étaient en tout deux cent douze, choisis pour être les portiers des seuils, et enregistrés selon les familles dans la généalogie, selon leurs villages ; David et Samuel, le voyant, les avaient établis dans leurs fonctions.
\VS{23}Eux, dis-je, et leurs fils furent établis sur les portes de la maison de Yahweh, qui est la maison du tabernacle, pour y faire la garde.
\VS{24}Il y avait des portiers aux quatre vents, à l'orient, à l'occident, au nord et au midi.
\VS{25}Et leurs frères, qui étaient dans leurs villages, devaient de temps à autre venir auprès d’eux pendant sept jours.
\VS{26}Car selon cette fonction, il y avait toujours quatre chefs des portiers, des Lévites, qui avaient la surveillance des chambres et des trésors de la maison de Dieu.
\VS{27}Ils se tenaient la nuit tout autour de la maison de Dieu, dont ils avaient la garde, et qu’ils devaient ouvrir tous les matins.
\VS{28}Certains d’entre eux prenaient soin des ustensiles du service; car on en faisait le compte lorsqu'on les rentrait et qu'on les sortait.
\VS{29}D’autres veillaient sur les ustensiles, sur tous les ustensiles du sanctuaire, sur la fleur de farine, sur le vin, sur l'huile, sur l'encens et sur les aromates.
\VS{30}Mais ceux qui composaient les parfums aromatiques étaient des fils de sacrificateurs.
\VS{31}Matthithia, d'entre les Lévites, premier-né de Schallum, Koréite, s’occupait des gâteaux cuits sur les plaques.
\VS{32}Et quelques-uns de leurs frères, parmi les fils des Kehathites, avaient la charge du pain de proposition\FTNT{Il y avait douze gâteaux de pain qu’on plaçait sur une table dans le tabernacle ou dans le temple et qu’on remplaçait chaque sabbat (Ex. 35:13 ;  Ex. 39:36 ; 1 R. 7:48 ; 2 Ch. 13:11 ; Né. 10:32-33). En hébreu, le pain de proposition signifie littéralement le « pain de la face ». Le mot rendu par « face » se rapporte parfois à la « présence » (2 R. 13:23). Le pain de proposition est en réalité l’image du Seigneur Jésus-Christ, notre pain de vie (Jn. 6:48-59).}  pour l'apprêter chaque sabbat.
\VS{33}Certains étaient des chantres, chefs des pères des Lévites, qui demeuraient dans les chambres, sans avoir d’autres charges, parce qu'ils devaient être en fonction le jour et la nuit.
\VS{34}Ce sont là les chefs des pères des Lévites, selon leurs familles ; ils furent chefs, et ils habitèrent à Jérusalem.
\TextTitle{De Jeïel au roi Saül, de Jonathan à Arsel\FTNTT{1 Ch. 10 ; 1 S. 1 ; 30}}
\VS{35}Or Jeïel, le père de Gabaon, habita à Gabaon ; et le nom de sa femme était Maaca.
\VS{36}Son fils premier-né, Abdon, puis Tsur, Kis, Baal, Ner, Nadab,
\VS{37}Guedor, Achjo, Zacharie, et Mikloth.
\VS{38}Mikloth engendra Schimeam ; et ils habitèrent vis-à-vis de leurs frères à Jérusalem, avec leurs frères.
\VS{39}Ner engendra Kis, et Kis engendra Saül, et Saül engendra Jonathan, Malki-Schua, Abinadab et Eschbaal.
\VS{40}Le fils de Jonathan fut  Merib-Baal; et Merib-Baal engendra Michée.
\VS{41}Et les fils de Michée furent Pithon, Mélec, Thachréa et Achaz.
\VS{42}Achaz engendra Jaera ; et Jaera engendra Alémeth, Azmaveth, et Zimri ; et Zimri engendra Motsa.
\VS{43}Motsa engendra Binea, qui eut pour fils Rephaja, qui eut pour fils Eleasa, qui eut pour fils Atsel.
\VS{44}Atsel eut six fils, dont les noms sont Azrikam, Bocru, Ismaël, Schearia, Abdias et Hanan. Ce furent là les fils d'Atsel.
\Chap{10}
\TextTitle{Mort de Saül\FTNTT{1 S. 31:1-10 ; 2 S. 1}}
\VerseOne{}Les Philistins combattirent contre Israël, et les hommes d'Israël s'enfuirent devant les Philistins, et tombèrent blessés à mort sur la montagne de Guilboa\FTNT{1 S. 31:1-10}.
\VS{2}Les Philistins poursuivirent et atteignirent Saül et ses fils, et tuèrent Jonathan, Abinadab et Malki-Schua, les fils de Saül.
\VS{3}L’effort du combat se porta sur Saül ; de sorte que les archers l'atteignirent, et il eut peur de ces archers.
\VS{4}Alors Saül dit à celui qui portait ses armes : Tire ton épée, et transperce-moi, de peur que ces incirconcis ne viennent et ne fassent de moi selon leur volonté ; mais celui qui portait ses armes ne voulut pas, parce qu’il avait très peur. Saül prit donc son épée, et se jeta dessus.
\VS{5}Alors celui qui portait les armes de Saül, ayant vu que Saül était mort, se jeta aussi sur son épée, et il mourut.
\VS{6}Ainsi mourut Saül, et ses trois fils, et toute sa maison périt avec lui.
\VS{7}Tous ceux d'Israël, qui étaient dans la vallée, ayant vu qu'on avait fui, et que Saül et ses fils étaient morts, abandonnèrent leurs villes et s'enfuirent, de sorte que les Philistins y entrèrent et y habitèrent.
\VS{8}Or il arriva que dès le lendemain, les Philistins vinrent pour dépouiller les morts, et ils trouvèrent Saül et ses fils étendus sur la montagne de Guilboa.
\VS{9}Ils le dépouillèrent et emportèrent sa tête et ses armes. Puis ils firent annoncer ces bonnes nouvelles par tout le pays des Philistins, et aux environs, pour en faire savoir les nouvelles à leurs dieux et au peuple.
\VS{10}Ils mirent ses armes dans la maison de leur dieu, et ils attachèrent sa tête dans la maison de Dagon\FTNT{1 S. 5:1-11.}.
\VS{11}Tous ceux de Jabès de Galaad, ayant appris tout ce que les Philistins avaient fait à Saül,
\VS{12}tous les vaillants hommes d’entre eux se levèrent et enlevèrent le corps de Saül et les corps de ses fils ; ils les apportèrent à Jabès, et ils ensevelirent leurs os sous un chêne à Jabès, et jeûnèrent pendant sept jours.
\VS{13}Saül mourut pour le crime qu'il avait commis contre Yahweh, en ce qu'il n'avait point gardé la parole de Yahweh, et qu'il avait même consulté ceux qui évoquent les morts\FTNT{1 S. 28:7-20.} pour savoir ce qui devait lui arriver.
\VS{14}Il ne consulta point Yahweh ; c'est pourquoi Yahweh le fit mourir, et transféra la royauté à David, fils d'Isaï.
\Chap{11}
\TextTitle{David règne sur Israël\FTNTT{2 S. 5:1-3 ; 2 S. 2-4}}
\VerseOne{}Tous ceux d'Israël s'assemblèrent auprès de David à Hébron, et lui dirent : Voici, nous sommes tes os et ta chair.
\VS{2}Autrefois déjà, quand Saül était roi, tu étais celui qui faisais sortir et qui ramenais Israël. Yahweh, ton Dieu, t'a dit : Tu paîtras mon peuple d'Israël, et tu seras le chef de mon peuple d'Israël.
\VS{3}Ainsi, tous les anciens d'Israël vinrent auprès du roi à Hébron ; et David traita alliance avec eux à Hébron, devant Yahweh. Ils oignirent David pour roi sur Israël, selon la parole de Yahweh, prononcée par Samuel\FTNT{2 S. 2, 3, 4 ; 2 S. 5:1-3.}.
\TextTitle{Jérusalem devient la cité de David\FTNTT{2 S. 5:6-10}}
\VS{4}David et tous ceux d'Israël s'en allèrent à Jérusalem, qui est Jébus. Là étaient  les Jébusiens qui habitaient le pays.
\VS{5}Ceux qui habitaient à Jébus dirent à David : Tu n'entreras point ici. Mais David prit la forteresse de Sion, qui est la cité de David.
\VS{6}Car David avait dit: Quiconque battra le premier les Jébusiens sera chef et prince. Joab, fils de Tseruja, monta le premier, et fut fait chef.
\VS{7}David s’établit dans la forteresse ; c'est pourquoi on l'appela la cité de David\FTNT{2 S. 5:6-10.}.
\VS{8}Il bâtit aussi la ville tout autour, depuis Millo et ses environs ; et Joab répara le reste de la ville.
\VS{9}David devenait de plus en plus grand, car Yahweh des armées était avec lui.
\TextTitle{Les vaillants hommes de David\FTNTT{2 S. 23:8-39}}
\VS{10}Voici les chefs des hommes vaillants qui étaient au service de David, qui l’aidèrent avec tout Israël à assurer sa royauté, afin de le faire régner selon la parole de Yahweh au sujet d’Israël.
\VS{11}Ceux-ci sont du nombre des vaillants hommes que David avait. Jaschobeam, fils de Hacmoni, chef entre les trois principaux. Il brandit sa lance contre trois cents hommes et les blessa à mort en une seule fois\FTNT{2 S. 23:8-39. }.
\VS{12}Après lui était Eléazar, fils de Dodo, l'Achochite, qui fut l’un des trois vaillants hommes.
\VS{13}Il se trouvait avec David à Pas-Dammim, lorsque les Philistins s'étaient assemblés pour combattre. Il y avait là une parcelle de terre remplie d'orge ; et le peuple fuyait devant les Philistins.
\VS{14}Ils s'arrêtèrent au milieu de cette parcelle de champ, la défendirent, et battirent les Philistins. Ainsi, Yahweh accorda une grande délivrance.
\VS{15}Il en descendit encore trois des trente chefs près du rocher, auprès de David, dans la caverne d'Adullam, lorsque l'armée des Philistins campait dans la vallée des Rephaïm.
\VS{16}David était alors dans la forteresse, et la garnison des Philistins était en ce même temps-là à Bethléhem.
\VS{17}David eut un désir, et dit : Qui est-ce qui me fera boire de l'eau du puits qui est à la porte de Bethléhem ?
\VS{18}Alors ces trois hommes passèrent au travers du camp des Philistins, et puisèrent de l'eau du puits qui était à la porte de Bethléhem ; et l'ayant apportée, la présentèrent à David, qui ne voulut point la boire, mais la répandit en l'honneur de Yahweh.
\VS{19}Car il dit : Que mon Dieu me garde de faire une telle chose ! Boirais-je le sang de ces hommes qui ont fait un tel voyage au péril de leur vie ? Car ils m'ont apporté cette eau au péril de leur vie. Ainsi, il ne voulut point la boire. Voilà ce que firent ces trois vaillants hommes.
\VS{20}Abischaï, frère de Joab, était chef des trois. Il sortit sa lance sur trois cents hommes, les blessa à mort ; et il eut du renom entre les trois.
\VS{21}Entre les trois, il fut plus honoré que les deux autres, et il fut leur chef ; cependant, il n'égala point ces trois premiers.
\VS{22}Benaja aussi, fils de Jehojada, fils d'un vaillant homme de Kabtseel, avait fait de grands exploits. Il tua deux des plus puissants hommes de Moab. Il descendit et frappa un lion au milieu d'une fosse en un jour de neige.
\VS{23}Il tua aussi un homme Egyptien qui était haut de cinq coudées. Cet Egyptien avait à la main une lance grosse comme une ensouple de tisserand ; mais il descendit contre lui avec un bâton, et arracha la lance de la main de l'Egyptien, et le tua avec sa propre lance.
\VS{24}Benaja, fils de Jehojada, fit ces choses-là, et fut célèbre entre ces trois vaillants hommes.
\VS{25}Voilà, il était le plus honoré des trente ; cependant, il n'égala point les trois premiers. David l'établit dans son conseil privé.
\VS{26}Et les plus vaillants d'entre les gens de guerre furent Asaël, frère de Joab ; et Elchanan fils de Dodo, de Bethléhem,
\VS{27}Schammoth d'Haror, Hélets de Palon,
\VS{28}Ira, fils d'Ikkesch, de Tekoa, Abiézer d'Anathoth,
\VS{29}Sibbecaï le Huschatite, Ilaï d'Achoach,
\VS{30}Maharaï de Nethopha, Héled fils de Baana de Nethopha,
\VS{31}Ithaï fils de Ribaï, de Guibea des fils de Benjamin, Benaja de Pirathon,
\VS{32}Huraï de Nachalé-Gaasch, Abiel d'Araba,
\VS{33}Azmaveth de Bacharum, Eliachba de Schaalbon,
\VS{34}Bené-Haschem de Guizon, Jonathan fils de Schagué d'Harar,
\VS{35}Achiam fils de Sacar d'Harar, Eliphal fils d'Ur,
\VS{36}Hépher de Mekéra, Achija de Palon,
\VS{37}Hetsro de Carmel, Naaraï fils d'Ezbaï,
\VS{38}Joël frère de Nathan, Mibchar fils d'Hagri,
\VS{39}Tsélek l'Ammonite, Nachraï de Béroth, qui portait les armes de Joab fils de Tseruja,
\VS{40}Ira de Jéther, Gareb de Jéther,
\VS{41}Urie le Héthien, Zabad fils d' Achlaï,
\VS{42}Adina fils de Schiza le Rubénite, chef des Rubénites, et trente avec lui.
\VS{43}Hanan fils de Maaca, et Josaphat de Mithni,
\VS{44}Ozias d'Aschtharoth, Schama et Jehiel fils de Hotham d'Aroër,
\VS{45}Jediaël fils de Schimri, et Jocha son frère, le Thitsite,
\VS{46}Eliel de Machavim, Jeribaï, et Joschavia fils d'Elnaam, et Jithma le Moabite,
\VS{47}Eliel, et Obed, et Jaasie-Metsobaja.
\Chap{12}
\TextTitle{Les guerriers venus chez David à Tsiklag\FTNTT{2 S. 5:17 ; 1 Ch. 12:8-15 ; 1 Ch. 14:8}}
\VerseOne{}Voici ceux qui allèrent trouver David à Tsiklag, lorsqu'il était encore éloigné de la présence de Saül, fils de Kis. Ils étaient parmi les vaillants hommes qui lui prêtèrent leur secours pendant la guerre.
\VS{2}Ils étaient équipés d'arcs, se servant de la main droite et de la gauche pour jeter des pierres, et pour tirer des flèches avec l'arc. Ils étaient frères de Saül, de Benjamin,
\VS{3}Achiézer, le chef, et Joas, fils de Schemaa, qui était de Guibea, Jeziel, Péleth, fils d'Azmaveth, Beraca et Jéhu d'Anathoth ;
\VS{4}Jischmaeja de Gabaon, vaillant entre les trente, et même au-dessus des trente, et Jérémie, Jachaziel, Jochanan et Jozabad de Guedéra ;
\VS{5}Eluzaï, Jerimoth, Bealia, Schemaria et Schephathia de Haroph ;
\VS{6}Elkana, Jischija, Azareel, Joézer et Jaschobeam Koréites ;
\VS{7}Joéla et Zebadia, fils de Jerocham de Guedor.
\TextTitle{Les guerriers venus chez David dans la forteresse de Moab\FTNTT{1 S. 22:2-4}}
\VS{8}Quelques-uns aussi des Gadites se retirèrent auprès de David, dans la forteresse, au désert, hommes forts et vaillants, experts à la guerre et maniant le bouclier et la lance. Leurs visages étaient comme des faces de lion, et aussi prompts que des gazelles sur les montagnes.
\VS{9}Ezer le premier, Abdias le second, Eliab le troisième ;
\VS{10}Mischmanna le quatrième, Jérémie le cinquième ;
\VS{11}Attaï le sixième ; Eliel le septième ;
\VS{12}Jochanan le huitième, Elzabad le neuvième ;
\VS{13}Jérémie le dixième, Macbannaï le onzième.
\VS{14}C’étaient des fils de Gath, qui furent chefs de l'armée ; le plus petit avait la charge de cent hommes, et le plus grand de mille.
\VS{15}Ce sont ceux qui passèrent le Jourdain au premier mois, quand il déborde sur tous ses rivages ; et ils chassèrent ceux qui demeuraient dans les vallées, vers l'orient et l'occident.
\VS{16}Il vint aussi des fils de Benjamin et de Juda vers David à la forteresse.
\VS{17}David sortit au-devant d'eux, et prenant la parole, il leur dit : Si vous êtes venus en paix vers moi pour m'aider, mon cœur s’unira à vous ; mais si c'est pour me trahir et me livrer à mes ennemis, quoique je ne sois coupable d'aucune violence, que le Dieu de nos pères le voie, et qu'il fasse justice !
\VS{18}Alors Amasaï, l’un des principaux officiers, fut revêtu de l’Esprit, et dit : Que la paix soit avec toi, ô David ! Qu'elle soit avec toi, fils d'Isaï ! Que la paix soit à ceux qui t'aident, puisque ton Dieu t'aide ! Et David les reçut, et les établit parmi les chefs de ses troupes.
\VS{19}Des hommes de Manassé se joignirent à David, lorsqu’il alla combattre Saül avec les Philistins. Mais David et ses gens ne les aidèrent pas, parce que les princes des Philistins, après en avoir délibéré entre eux, le renvoyèrent, en disant : Il se tournera vers son maître Saül, au péril de nos têtes.
\VS{20}Quand donc il retournait à Tsiklag, Adnach, Jozabad, Jediaël, Micaël, Jozabad, Elihu et Tsilthaï, chefs des milliers qui étaient en Manassé, se tournèrent vers lui.
\VS{21}Et ils aidèrent David contre la troupe des Amalécites, car ils étaient tous forts et vaillants, et ils furent faits chefs dans l'armée.
\VS{22}De jour en jour, il venait des gens auprès de David pour l'aider, de sorte qu'il eut une grande armée, comme une armée de Dieu\FTNT{1 S. 22:2-4}.
\TextTitle{Les guerriers venus chez David à Hébron\FTNTT{2 S. 5:1-3}}
\VS{23}Voici le nombre des hommes équipés pour la guerre, qui vinrent auprès de David à Hébron, afin de lui transférer la royauté de Saül, selon le commandement de Yahweh\FTNT{2 S. 5:1-3}.
\VS{24}Des fils de Juda, qui portaient le bouclier et la lance, six mille huit cents, équipés pour la guerre.
\VS{25}Des fils de Siméon, forts et vaillants pour la guerre, sept mille cent.
\VS{26}Des fils de Lévi, quatre mille six cents.
\VS{27}Et Jehojada, prince de ceux d'Aaron, et avec lui trois mille sept cents ;
\VS{28}et Tsadok, jeune homme fort et vaillant, et vingt-deux chefs de la maison de son père.
\VS{29}Des fils de Benjamin, parents de Saül, trois mille ; car jusqu'alors la plus grande partie d’entre eux soutenaient la maison de Saül.
\VS{30}Des fils d'Ephraïm, vingt mille huit cents, forts et vaillants, et hommes de renom dans la maison de leurs pères.
\VS{31}De la demi-tribu de Manassé, dix-huit mille, qui furent désignés par leur nom pour aller établir David roi.
\VS{32}Des fils d'Issacar\FTNT{Les fils d’Issacar avaient la connaissance des temps. Discerner les temps dans lesquels nous sommes n’a rien à voir avec le fait de chercher à connaître la date du retour du Seigneur. Seul le Père connaît la date du retour du Messie (Za. 14:7 ; Mt. 24:36). Comprendre les caractéristiques de notre époque nous aide à nous réveiller afin d’accomplir les oeuvres que le Seigneur nous confie. Cette prise de conscience nous aidera à éviter les pièges de Satan et à mieux nous préparer aux noces de l’Agneau. Voir Mt. 16:3 ;  Ro. 13:11-14 ;  2 Pi. 1:19.}, fort intelligents dans la connaissance des temps, pour savoir ce que devait faire Israël, deux cents de leurs chefs, et tous leurs frères sous leurs ordres.
\VS{33}De Zabulon, cinquante mille combattants, rangés en bataille avec toutes sortes d'armes, et prêts à livrer bataille d’un cœur assuré.
\VS{34}De Nephthali, mille capitaines, et avec eux trente-sept mille, portant le bouclier et la lance.
\VS{35}Des Danites, vingt-huit mille six cents, équipés pour la guerre.
\VS{36}D'Aser, quarante mille combattants, et prêts à combattre.
\VS{37}De l’autre côté du Jourdain, à savoir des Rubénites, des Gadites, et de la demi-tribu de Manassé, cent vingt mille, avec tous les instruments de guerre pour combattre.
\VS{38}Tous ces hommes, gens de guerre, prêts a combattre, vinrent tous de bon cœur à Hébron, pour établir David roi sur tout Israël. Et tout le reste d'Israël était aussi d'un même sentiment pour établir David roi.
\VS{39}Et ils furent là avec David, mangeant et buvant pendant trois jours ; car leurs frères leur avaient préparé des vivres.
\VS{40}Et même ceux qui étaient les plus proches d'eux, jusqu'à Issacar, Zabulon et Nephthali, apportaient du pain sur des ânes, sur des chameaux, sur des mulets et sur des bœufs, de la farine, des figues sèches, des raisins secs, du vin, et de l'huile ; et ils amenaient des bœufs et des brebis en abondance, car il y avait une joie en Israël.
\Chap{13}
\TextTitle{Retour de l'arche, Uzza frappé par Yahweh\FTNTT{2 S. 6:1-11}}
\VerseOne{}Or David tint conseil avec les chefs de milliers et de centaines, avec tous les princes du peuple.
\VS{2}Et il dit à toute l'assemblée d'Israël : Si vous l'approuvez, et que cela vient de Yahweh, notre Dieu, envoyons partout vers nos autres frères, qui sont dans toutes les contrées d'Israël, et avec lesquels sont les sacrificateurs et les Lévites, dans leurs villes et dans leurs banlieues, afin qu'ils se réunissent à nous,
\VS{3}et que nous ramenions auprès de nous l’arche de notre Dieu ; car nous ne nous en sommes pas occupés du temps de Saül.
\VS{4}Et toute l'assemblée répondit qu'on le fasse ainsi ; car la chose fut approuvée par tout le peuple.
\VS{5}David donc assembla tout Israël, depuis Schichor, le torrent d'Egypte, jusqu'à l'entrée du pays de Hamath, pour ramener de Kirjath-Jearim l’arche de Dieu.
\VS{6}Et David monta avec tout Israël vers Baala à Kirjath-Jearim, qui appartient à Juda, pour faire amener de là l’arche de Dieu, devant laquelle est invoqué le Nom de Yahweh, qui habite entre les chérubins.
\VS{7}Ils mirent l’arche de Dieu sur un char neuf, et l'emmenèrent de la maison d'Abinadab ; et Uzza et Achjo conduisaient le char.
\VS{8}Et David et tout Israël dansaient en présence de Dieu de toute leur force, en chantant des cantiques et en jouant sur des violons, des luths, des tambourins, des cymbales, et des trompettes.
\VS{9}Quand ils furent arrivés à l'aire de Kidon, Uzza\FTNT{L’arche devait être transportée grâce à  des barres faites spécialement à cet effet, qui ne devaient pas être enlevées (Ex 27:6-7 ;  No. 1:51). Selon la Loi, seuls les Lévites devaient préparer et déplacer tout ce qui concernait le tabernacle.  Et même parmi les Lévites, chaque famille avait une fonction spécifique (No. 3 ; No. 4). Les Kehathites n’étaient pas autorisés à toucher l’arche, leur rôle se limitait seulement à la transporter à l’aide des barres (No. 4:15). Uzza a étendu sa main sur l’arche, alors qu’il n’était certainement pas Lévite. Il était devenu trop familier avec les choses saintes et avait pris à la légère les principes de Dieu. Il a voulu aider le Seigneur. Or, il ne faut jamais chercher à servir Dieu sans être appelé par lui.}  étendit sa main pour retenir l’arche, parce que les boeufs avaient glissé.
\VS{10}Et la colère de Yahweh s'enflamma contre Uzza, et il le frappa, parce qu'il avait étendu sa main sur l’arche. Uzza mourut en présence de Dieu.
\VS{11}David fut irrité de ce que Yahweh avait fait une brèche en la personne de Uzza. On a appelé jusqu'à ce jour ce lieu-là Pérets-Uzza, brèche d'Uzza.
\VS{12}David eut peur de Dieu en ce jour-là, et il dit: Comment ferais-je entrer chez moi l’arche de Dieu ?
\VS{13}C'est pourquoi David ne la retira point chez lui, dans la cité de David, mais il la fit conduire dans la maison d'Obed- Edom de Gath.
\VS{14}Et l’arche de Dieu demeura trois mois avec la famille d'Obed-Edom, dans sa maison. Yahweh bénit la maison d'Obed-Edom, et tout ce qui lui appartenait.
\Chap{14}
\TextTitle{Rayonnement du règne de David\FTNTT{2 S. 5:11-25 ; 23:13-17 ; 1 Ch. 3:5-9 ; 11:15-19 ; 12:8-15}}
\VerseOne{}Hiram, roi de Tyr, envoya des messagers à David, et du bois de cèdre, des tailleurs de pierres et des charpentiers, pour lui bâtir une maison.
\VS{2}Alors David reconnut que Yahweh l'affermissait comme roi sur Israël, et que son règne était fort élevé, à cause de son peuple d'Israël.
\VS{3}David prit encore des femmes à Jérusalem, et il engendra encore des fils et des filles.
\VS{4}Voici les noms des fils qu'il eut à Jérusalem : Schammua, Schobab, Nathan, Salomon,
\VS{5}Jibhar, Elischua, Elphéleth,
\VS{6}Noga, Népheg, Japhia,
\VS{7}Elischama, Beéliada et Eliphéleth.
\VS{8}Or quand les Philistins apprirent que David avait été oint pour roi sur tout Israël, ils montèrent tous à sa recherche. David l'ayant appris, sortit au-devant d'eux.
\VS{9}Les Philistins vinrent et se répandirent dans la vallée des Rephaïm.
\VS{10}David consulta Dieu, en disant : Monterai-je contre les Philistins, et les livreras-tu entre mes mains? Yahweh lui répondit : Monte, et je les livrerai entre tes mains.
\VS{11}Alors ils montèrent à Baal-Peratsim\FTNT{Baal-Peratsim signifie « seigneur des brèches »}, où David les battit. Puis il dit : Dieu a fait une brèche au milieu de mes ennemis par ma main, comme une brèche faite par les eaux. C'est pourquoi on donna à ce lieu-là le nom de Baal- Peratsim.
\VS{12}Et ils laissèrent là leurs dieux, et David ordonna qu'on les brûle au feu.
\VS{13}Les Philistins se répandirent encore une autre fois dans cette même vallée.
\VS{14}David consulta encore Dieu ; et Dieu lui répondit : Tu ne monteras point vers eux, mais tu te détourneras d'eux, et tu iras contre eux vis-à-vis des mûriers.
\VS{15}Dès que tu auras entendu au sommet des mûriers un bruit comme des gens qui marchent, tu sortiras alors pour combattre, car c’est Dieu qui marche devant toi pour frapper le camp des Philistins.
\VS{16}David fit selon ce que Dieu lui avait ordonné, et on frappa le camp des Philistins, depuis Gabaon jusqu'à Guézer.
\VS{17}Ainsi, la renommée de David se répandit par tous ces pays-là, et Yahweh remplit de frayeur toutes ces nations-là, au seul nom de David.
\Chap{15}
\TextTitle{David coordonne avec minutie l’arrivé de l’arche à Jérusalem\FTNTT{2 S. 6:12}}
\VerseOne{}David se bâtit des maisons dans la cité de David ; il prépara un lieu pour l’arche de Dieu, et dressa pour elle une tente.
\VS{2}Et David dit : L’arche de Dieu ne doit être portée que par les Lévites, car Yahweh les a choisis pour porter l’arche de Dieu, et pour faire le service à toujours\FTNT{No. 4:15.}.
\VS{3}David donc assembla tous ceux d'Israël à Jérusalem, pour faire monter l’arche de Yahweh dans le lieu qu'il lui avait préparé.
\VS{4}David assembla aussi les fils d'Aaron, et les Lévites.
\VS{5}Des fils de Kehath : Uriel, le chef, et ses frères, cent vingt.
\VS{6}Des fils de Merari : Asaja, le chef, et ses frères, deux cent vingt.
\VS{7}Des fils de Guerschon : Joël, le chef, et ses frères, cent trente.
\VS{8}Des fils d'Elitsaphan : Schemaeja, le chef, et ses frères, deux cents.
\VS{9}Des fils de Hébron : Eliel, le chef, et ses frères, quatre-vingts.
\VS{10}Des fils de Uziel : Amminadab, le chef, et ses frères, cent douze.
\VS{11}David appela les sacrificateurs Tsadok et Abiathar, et les Lévites, à savoir Uriel, Asaja, Joël, Schemaeja, Eliel, et Amminadab ;
\VS{12}et il leur dit: Vous qui êtes les chefs des familles des Lévites, sanctifiez-vous, vous et vos frères ; et transportez l’arche de Yahweh, le Dieu d'Israël, au lieu que je lui ai préparé.
\VS{13}Parce que vous n'y étiez pas la première fois, Yahweh, notre Dieu, a fait une brèche parmi nous ; car nous ne l'avons pas cherché selon la loi.
\VS{14}Les sacrificateurs donc et les Lévites se sanctifièrent pour faire monter l’arche de Yahweh, le Dieu d'Israël.
\VS{15}Et les fils des Lévites portèrent l’arche de Dieu sur leurs épaules, avec les barres qu'ils avaient sur eux, comme Moïse l'avait ordonné selon la parole de Yahweh.
\VS{16}David dit aux chefs des Lévites d'établir quelques-uns de leurs frères chantres, avec des instruments de musique, des luths, des violons, et des cymbales qui feraient retentir des sons éclatants, en signe de réjouissance.
\VS{17}Les Lévites donc établirent Héman, fils de Joël, et parmi ses frères, Asaph, fils de Bérékia; et des fils de Merari, qui étaient leurs frères, Ethan, fils de Kuschaja ;
\VS{18}avec eux leurs frères pour être du second ordre : Zacharie, Ben, Jaaziel, Schemiramoth, Jehiel, Unni, Eliab, Benaja, Maaséja, Matthithia, Eliphelé, Miknéja, Obed-Edom, et Jeïel, les portiers.
\VS{19}Quant aux chantres : Héman, Asaph et Ethan, ils avaient des cymbales d'airain pour les faire retentir.
\VS{20}Zacharie, Aziel, Schemiramoth, Jehiel, Unni, Eliab, Maaséja, et Benaja jouaient des luths sur alamoth ;
\VS{21}et Matthithia, Eliphelé, Miknéja, Obed-Edom, Jeïel et Azazia jouaient des harpes à huit cordes, pour conduire le chant.
\VS{22}Mais Kenania, le chef des Lévites, avait la charge de faire porter l’arche, enseignant comment il fallait la porter, car il était un homme très intelligent.
\VS{23}Bérékia et Elkana étaient portiers de l’arche.
\VS{24}Schebania, Josaphat, Nethaneel, Amasaï, Zacharie, Benaja, Eliézer, les sacrificateurs, sonnaient des trompettes devant l’arche de Dieu, et Obed-Edom et Jechija étaient portiers de l’arche.
\TextTitle{L'arche transportée au milieu des réjouissances\FTNTT{2 S. 6:12}}
\VS{25}David et les anciens d'Israël, avec les gouverneurs de milliers, marchaient, amenant avec joie l’arche de l'alliance de Yahweh, de la maison d' Obed-Edom.
\VS{26}Dieu aidait les Lévites qui portaient l’arche de l'alliance de Yahweh, et l’on sacrifia sept veaux et sept béliers.
\VS{27}David était vêtu d'un manteau de fin lin ; et tous les Lévites aussi qui portaient l’arche, les chantres ;  et Kenania, qui avait la principale charge de faire porter l’arche, était avec les chantres ; et David avait un éphod de lin.
\VS{28}Ainsi tout Israël amena l’arche de l'alliance de Yahweh, avec de grands cris de joie, et au son du cor, des shofars et des cymbales, faisant retentir leur voix avec des luths et des harpes.
\VS{29}Mais il arriva, comme l’arche de l'alliance de Yahweh entrait dans la cité de David, que Mical, fille de Saül, regardant par la fenêtre, vit le roi David sautant et dansant, et elle le méprisa dans son coeur.
\Chap{16}
\TextTitle{L’arche placée dans une tente à Jérusalem ; sacrifices et cantiques pour Yahweh\FTNTT{2 S. 6:17-19}}
\VerseOne{}Ils amenèrent donc l’arche de Dieu et la posèrent au milieu de la tente que David avait dressée pour elle ; et on offrit devant Dieu des holocaustes et des sacrifices d’offrande de paix.
\VS{2}Quand David eut achevé d'offrir les holocaustes et les sacrifices d’offrande de paix, il bénit le peuple au Nom de Yahweh.
\VS{3}Et il distribua à chacun, tant aux hommes qu'aux femmes, un pain,  un morceau de viande et un gâteau de raisin.
\VS{4}Et il établit quelques-uns des Lévites pour faire le service devant l’arche de Yahweh, pour célébrer, remercier, et louer le Dieu d'Israël.
\VS{5}Asaph était le premier et Zacharie le second ; Jeïel, Schemiramoth, Jehiel, Matthithia, Eliab, Benaja, Obed-Edom, et Jeïel, qui avaient des instruments de musique, à savoir des luths et des harpes ; et Asaph faisait retentir sa voix avec des cymbales.
\VS{6}Benaja et Jachaziel, les sacrificateurs, étaient continuellement avec des trompettes devant l’arche de l'alliance de Dieu.
\VS{7}Et en ce même jour, David remit entre les mains d'Asaph et de ses frères, les Psaumes suivants, pour commencer à célébrer Yahweh :
\VS{8}Célébrez Yahweh, invoquez son Nom ! Faites connaître parmi les peuples ses exploits !
\VS{9}Chantez-le,  célébrez-le !  Parlez de toutes ses merveilles !
\VS{10}Glorifiez-vous de son saint Nom !  Que le cœur de ceux qui cherchent Yahweh se réjouisse !
\VS{11}Recherchez Yahweh et sa force, cherchez continuellement sa face !
\VS{12}Souvenez-vous des merveilles qu'il a faites, de ses miracles et des jugements de sa bouche.
\VS{13}Postérité d'Israël, son serviteur, fils de Jacob, ses élus !
\VS{14}Yahweh est notre Dieu ; ses jugements s’exercent sur toute la terre.
\VS{15}Souvenez-vous toujours de son alliance, de ses promesses établies pour mille générations ;
\VS{16}du traité qu'il a fait avec Abraham et du serment qu’il a fait à Isaac,
\VS{17}et qu’il a confirmé à Jacob et à Israël, pour être une loi et une alliance éternelle,
\VS{18}en disant : Je te donnerai le pays de Canaan, comme l’héritage qui vous est échu.
\VS{19}Ils étaient alors une  poignée de gens, peu nombreux, et étrangers dans le pays,
\VS{20}car ils étaient errants de nation en nation, et d'un royaume vers un autre peuple.
\VS{21}Il ne permit à personne de les opprimer ; il a même châtié des rois à cause d'eux.
\VS{22}Et il a dit : Ne touchez point à mes oints, et ne faites point de mal à mes prophètes\FTNT{L’expression « ne touchez pas à mes oints » signifie qu'il ne faut pas leur porter physiquement atteinte. C’est une expression associée à des mauvais traitements physiques. Il est donc clair que ce verset, qu’on trouve également dans le  Ps. 105 : 15, ne peut absolument pas concerner la remise en question des enseignements d'un quelconque pasteur, prophète ou apôtre. Dans le contexte de ce passage, il est question des rois, des prophètes et des sacrificateurs, car c’est sur eux que reposait l’onction. Aujourd’hui, tous les chrétiens sont oints de Dieu (Ep. 1:13 ; Ep. 4:30).} !
\VS{23}Habitants de la terre, chantez à Yahweh ! Racontez chaque jour sa délivrance.
\VS{24}Racontez sa gloire parmi les nations, et ses merveilles parmi tous les peuples !
\VS{25}Car Yahweh est grand et très digne de louanges, il est plus redoutable que tous les dieux.
\VS{26}Car tous les dieux des peuples sont des idoles\FTNT{Jésus-Christ est le seul et le véritable Dieu (1 Co. 8:6 ; 1 Jn. 5:20).}, mais Yahweh a fait les cieux.
\VS{27}La majesté et la magnificence marchent devant lui; la force et la joie sont dans le lieu où il habite.
\VS{28}Familles des peuples, donnez à Yahweh, donnez à Yahweh gloire et force !
\VS{29}Donnez à Yahweh la gloire due à son Nom ! Apportez des offrandes, et présentez-vous devant lui. Prosternez-vous devant Yahweh avec des ornements saints !
\VS{30}Tremblez, vous tous habitants de la terre tout étonnés devant sa face ! Car la terre habitable est affermie par lui, et elle ne chancelle point.
\VS{31}Que les cieux se réjouissent, que la terre soit dans l’allégresse ! Et que l’on dise parmi les nations : Yahweh règne !
\VS{32}Que la mer retentisse avec tout ce qu'elle contient ! Que la campagne se réjouisse avec tout ce qu’elle renferme !
\VS{33}Que les arbres de la forêt poussent des cris de joie au devant de Yahweh, parce qu'il vient juger la terre\FTNT{Yahweh vient juger la terre. Cette prophétie confirme de façon incontestable la divinité de Jésus-Christ. Voir Za. 14:1-7.}.
\VS{34}Célébrez Yahweh, car il est bon, car sa miséricorde demeure à jamais !
\VS{35}Et dites : Ô Dieu de notre salut, sauve-nous, et rassemble-nous, et retire-nous d'entre les nations, pour célébrer ton saint Nom, et que nous nous glorifions de ta louange !
\VS{36}Béni soit Yahweh, le Dieu d'Israël, de siècle en siècle ! Et tout le peuple dit : Amen ! Louez Yahweh !
\VS{37}On laissa donc là, devant l’arche de l'alliance de Yahweh, Asaph et ses frères, pour faire le service continuellement, remplissant leur tâche jour par jour devant l’arche.
\VS{38}On laissa Obed-Edom, et ses frères, au nombre de soixante-huit, Obed-Edom, dis-je, fils de Jeduthun, et Hosa comme portiers.
\VS{39}On établit le sacrificateur Tsadok, et les sacrificateurs ses frères, devant le tabernacle de Yahweh, dans le haut lieu qui était à Gabaon,
\VS{40}pour offrir des holocaustes à Yahweh continuellement sur l'autel de l'holocauste, matin et soir, selon tout ce qui est écrit dans la loi de Yahweh, qu’il ordonna à Israël.
\VS{41}Auprès d’eux étaient Héman et Jeduthun, et les autres qui furent choisis et désignés par leur nom, pour célébrer Yahweh, parce que sa miséricorde demeure éternellement.
\VS{42}Et Héman et Jeduthun étaient avec ceux-là; il y avait aussi des trompettes et des cymbales pour ceux qui les faisaient retentir, et des instruments pour chanter les cantiques de Dieu. Les fils de Jeduthun étaient portiers.
\VS{43}Puis tout le peuple s'en alla chacun dans sa maison, et David aussi s'en retourna pour bénir sa maison.
\Chap{17}
\TextTitle{David veut construire un temple à Yahweh\FTNTT{2 S. 7:1-3}}
\VerseOne{}Or il arriva après que David fut établi dans sa maison, qu'il dit à Nathan, le prophète : Voici,  j’habite dans une maison de cèdres, et l’arche de l'alliance de Yahweh est sous une tente.
\VS{2}Nathan dit à David : Fais tout ce que tu as dans le cœur, car Dieu est avec toi.
\TextTitle{Réponse de Yahweh à David\FTNTT{2 S. 7:4-17}}
\VS{3}Mais il arriva cette nuit-là que la parole de Dieu fut adressée à Nathan, en disant :
\VS{4}Va, et dis à David, mon serviteur : Ainsi parle Yahweh :  Tu ne me bâtiras point de maison pour y habiter.
\VS{5}Puisque je n'ai point habité dans une maison depuis le jour où j'ai fait monter les fils d’Israël hors d'Egypte jusqu'à ce jour ; mais j'ai été de tente en tente, et de tabernacle en tabernacle.
\VS{6}Partout où j’ai marché avec tout Israël, ai-je dit un mot à un seul des juges d'Israël, auxquels j'ai ordonné de paître mon peuple, ai-je dit : Pourquoi ne m'avez-vous point bâti une maison de cèdres ?
\VS{7}Maintenant donc tu diras ainsi à David, mon serviteur : Ainsi parle Yahweh des armées : Je t'ai pris d'une cabane, d'auprès des brebis, afin que tu sois le conducteur de mon peuple d'Israël ;
\VS{8}j'ai été avec toi partout où tu as marché, j'ai exterminé devant toi tous tes ennemis, et j’ai rendu ton nom semblable au nom des grands qui sont sur la terre.
\VS{9}J’ai établi un lieu pour mon peuple d'Israël, et je l’ai planté afin qu’il habite chez lui et ne soit plus agité.  Les fils d'iniquité ne le détruiront plus comme ils l’ont fait auparavant,
\VS{10}et comme à l’époque où j'ai établi des juges sur mon peuple d'Israël. J’ai humilié tous tes ennemis. Je t’informe que Yahweh te bâtira une maison.
\VS{11}Quand tes jours seront accomplis pour t'en aller avec tes pères, je ferai lever ta postérité après toi, l’un de tes fils, et j'affermirai son règne\FTNT{Cette prophétie est relative au Messie. Voir 2 S. 7:12-17.}.
\VS{12}Il me bâtira une maison, et j'affermirai son trône éternellement.
\VS{13}Je serai pour lui un père, et il sera pour moi un fils ; et je ne retirerai point de lui ma grâce, comme je l'ai retirée de celui qui a été avant toi.
\VS{14}Mais je l'établirai dans ma maison et dans mon royaume éternellement, et son trône sera affermi pour toujours.
\VS{15}Nathan récita à David toutes ces paroles, et toute cette vision.
\TextTitle{Adoration et reconnaissance de David à Yahweh\FTNTT{2 S. 7:18-29}}
\VS{16}Alors le roi David entra, et se tint devant Yahweh, et dit : Ô Yahweh Dieu ! Qui suis-je, et quelle est ma maison, que tu m'aies fait parvenir au point où je suis ?
\VS{17}Mais cela t'a semblé être peu de chose, ô Dieu ! Et tu as parlé de la maison de ton serviteur pour le temps à venir, et tu as porté les regards sur moi à la manière de l'homme, toi qui es élevé, ô Yahweh Dieu !
\VS{18}Que pourrait te dire encore David de l'honneur que tu fais à ton serviteur ? Car tu connais ton serviteur.
\VS{19}Ô Yahweh ! Pour l'amour de ton serviteur, et selon ton cœur, tu as fait toutes ces grandes choses, pour lui révéler toutes ces grandeurs.
\VS{20}Ô Yahweh ! Nul n’est semblable à toi, et il n'y a point d'autre Dieu que toi selon tout ce que nous avons entendu de nos oreilles.
\VS{21}Et qui est comme ton peuple d'Israël, la seule nation sur la terre que Dieu lui-même est venu racheter pour lui, afin qu'elle soit son peuple, et pour te faire un Nom et pour accomplir des miracles et des prodiges, en chassant les nations devant ton peuple que tu as racheté d'Egypte ?
\VS{22}Et tu as établi ton peuple d'Israël afin qu’il soit ton peuple à toujours ; et toi, ô Yahweh ! Tu as été son Dieu.
\VS{23}Maintenant donc, ô Yahweh ! Que la parole que tu as prononcée sur ton serviteur et sur sa maison, soit ferme à jamais, et agis selon ta parole !
\VS{24}Et que ton Nom subsiste et soit magnifié éternellement, de sorte qu'on dise : Yahweh des armées, le Dieu d'Israël, est Dieu pour Israël ; et que la maison de David, ton serviteur, soit affermie devant toi.
\VS{25}Car, ô mon Dieu ! Tu as révélé à ton serviteur que tu lui bâtirais une maison. C'est pourquoi ton serviteur a pris la hardiesse de te faire cette prière.
\VS{26}Maintenant, ô Yahweh ! Tu es Dieu, et tu as parlé de ce bien à ton serviteur.
\VS{27}Veuille donc maintenant bénir la maison de ton serviteur, afin qu'elle soit éternellement devant toi ; car tu l'as bénie, ô Yahweh ! Et elle sera bénie à jamais !
\Chap{18}
\TextTitle{Le règne de David affermi\FTNTT{2 S. 8:1-18}}
\VerseOne{}Et il arriva que David battit les Philistins, et les humilia, et il enleva de la main des Philistins Gath et les villes de son ressort\FTNT{2 S. 8.  }.
\VS{2}Il battit aussi les Moabites, et les Moabites furent asservis à David et lui payèrent un tribut.
\VS{3}David battit aussi Hadarézer, roi de Tsoba, vers Hamath, lorsqu’il alla établir sa domination sur le fleuve de l'Euphrate.
\VS{4}David lui prit mille chars, sept mille cavaliers, et vingt mille hommes de pied ; et il coupa les jarrets des chevaux de tous les chars, mais il réserva cent chars.
\VS{5}Les Syriens de Damas vinrent au secours d’Hadarézer, roi de Tsoba, et David battit vingt-deux mille Syriens.
\VS{6}Puis David mit des garnisons dans la Syrie de Damas. Et les Syriens furent assujettis à David et lui payèrent un tribut. Yahweh sauvait David partout où il allait.
\VS{7}Et David prit les boucliers d'or qui étaient aux serviteurs de Hadarézer, et les apporta à Jérusalem.
\VS{8}Il emporta aussi de Thibchath, et de Cun, villes de Hadarézer, une grande quantité d'airain, dont Salomon fit la mer d'airain, les colonnes et les ustensiles d'airain.
\VS{9}Thohu, roi de Hamath, apprit que David avait défait toute l'armée de Hadarézer, roi de Tsoba.
\VS{10}Et il envoya Hadoram, son fils, vers le roi David pour le saluer et le féliciter de ce qu'il avait combattu Hadarézer, et qu'il l'avait défait. Car Hadarézer était dans une guerre continuelle contre Thohu. Quant à tous les vases d'or, d'argent, et d'airain,
\VS{11}le roi David les consacra aussi à Yahweh, avec l'argent et l'or qu'il avait emporté de toutes les nations, à savoir d'Edom, de Moab, des fils d'Ammon, des Philistins, et d’Amalek.
\VS{12}Et Abischaï, fils de Tseruja battit dix-huit mille Edomites dans la vallée du sel.
\VS{13}Il mit une garnison dans Edom, et tous les Edomites furent asservis à David ; et Yahweh gardait David partout où il allait.
\VS{14}Ainsi, David régna sur tout Israël, rendant jugement et justice à tout son peuple.
\VS{15}Joab, fils de Tseruja, avait la charge de l'armée, et Josaphat, fils d'Achilud, était archiviste.
\VS{16}Tsadok, fils d'Achithub, et Abimélec, fils d'Abiathar, étaient les sacrificateurs; et Schavscha était le secrétaire.
\VS{17}Benaja, fils de Jehojada, était sur les Kéréthiens et les Péléthiens ; mais les fils de David étaient les premiers auprès du roi.
\Chap{19}
\TextTitle{David monte contre les Ammonites et les Syriens\FTNTT{2 S. 10}}
\VerseOne{}Or il arriva après cela que Nachasch, roi des fils d’Ammon, mourut ; et son fils régna à sa place.
\VS{2}David dit : J'userai de bonté envers Hanun, fils de Nachasch, car son père a usé de bonté envers moi. Ainsi, David envoya des messagers pour le consoler de la mort de son père ; et les serviteurs de David vinrent au pays des fils d’Ammon vers Hanun pour le consoler.
\VS{3}Mais les chefs d'entre les fils d’Ammon dirent à Hanun : Penses-tu que ce soit pour honorer ton père que David t'a envoyé des consolateurs ? N'est-ce pas pour examiner et épier le pays, afin de le détruire, que ses serviteurs sont venus vers toi ?
\VS{4}Alors Hanun prit les serviteurs de David, les fit raser, et les fit couper leurs habits par le milieu jusqu'aux hanches. Puis il les renvoya.
\VS{5}Ils s'en allèrent, et le firent savoir par le moyen de quelques personnes à David, qui envoya des gens à leur rencontre ; car ces hommes-là étaient fort confus. Et le roi leur fit dire : Restez à Jéricho jusqu'à ce que votre barbe ait repoussé, et revenez ensuite.
\VS{6}Or les fils d’Ammon voyant qu'ils s'étaient rendus odieux à David, Hanun et les fils d'Ammon envoyèrent mille talents d'argent pour prendre à leur solde des chars et des cavaliers de Mésopotamie, de Syrie, de Maaca et de Tsoba.
\VS{7}Ils prirent à leur solde trente-deux mille hommes et des chars, et le roi de Maaca avec son peuple, lesquels vinrent camper devant Médeba. Les fils d'Ammon aussi s'assemblèrent de leurs villes et vinrent pour combattre.
\VS{8}David l’ayant appris, envoya Joab et ceux de toute l'armée qui étaient les plus vaillants.
\VS{9}Les fils d’Ammon sortirent et rangèrent leur armée en bataille à l'entrée de la ville ; et les rois qui étaient venus étaient à part dans la campagne.
\VS{10}Joab, voyant que l'armée était tournée contre lui devant et derrière, prit de tous les gens d'élite d'Israël, et les rangea contre les Syriens.
\VS{11}Et il donna la conduite du reste du peuple à Abischaï, son frère; et on les rangea contre les fils d’Ammon.
\VS{12}Et Joab lui dit: Si les Syriens sont plus forts que moi, tu viendras me délivrer ; et si les fils d'Ammon sont plus forts que toi, je te délivrerai.
\VS{13}Sois ferme, et montrons-nous vaillants pour notre peuple, et pour les villes de notre Dieu ; et que Yahweh fasse ce qui lui semblera bon.
\VS{14}Alors Joab et le peuple qui était avec lui s'approchèrent pour livrer bataille aux Syriens qui s'enfuirent de devant lui.
\VS{15}Et les fils d'Ammon voyant que les Syriens s'étaient enfuis, eux aussi s'enfuirent devant Abischaï, frère de Joab, et rentrèrent dans la ville, et Joab revint à Jérusalem.
\VS{16}Mais les Syriens, qui avaient été battus par ceux d'Israël, envoyèrent des messagers et firent venir les Syriens qui étaient au-delà du fleuve ; et Schophach, chef de l'armée d'Hadarézer, les conduisait.
\VS{17}On le rapporta à David, qui assembla tout Israël, passa le Jourdain, alla au-devant d'eux, et se rangea en bataille contre eux. David donc rangea la bataille contre les Syriens, et ils combattirent contre lui.
\VS{18}Mais les Syriens s'enfuirent de devant Israël ; et David défit sept mille chars des Syriens et quarante mille hommes de pied ; et il tua Schophach, le chef de l'armée.
\VS{19}Alors les serviteurs d'Hadarézer, voyant qu'ils avaient été battus par ceux d'Israël, firent la paix avec David, et lui furent asservis ; et les Syriens ne voulurent plus secourir les fils d’Ammon.
\Chap{20}
\TextTitle{Conquête de Rabba\FTNTT{2 S. 11:1-12:25 ; 2 S. 12:26-31}}
\VerseOne{}L’année suivante, au temps où les rois se mettaient en campagne, Joab conduisit une forte armée et ravagea le pays des fils d’Ammon ; puis il alla assiéger Rabba, tandis que David resta à Jérusalem. Joab battit Rabba, et la détruisit\FTNT{2 S 12:26-31.}.
\VS{2}David enleva la couronne de dessus la tête de son roi, et il trouva qu'elle pesait un talent d'or : Elle était garnie de pierres précieuses. On la mit sur la tête de David, qui emmena un très grand butin de la ville.
\VS{3}Il emmena aussi le peuple qui y était, et les mit aux scies, aux pics de fer et aux haches de fer ; David traita de la sorte toutes les villes des fils d’Ammon ; puis il s'en retourna avec tout le peuple à Jérusalem.
\TextTitle{Guerre contre les Philistins\FTNTT{2 S. 21:15-22}}
\VS{4}Il arriva après cela que la guerre continua à Guézer contre les Philistins. Alors Sibbecaï, le Huschatite, frappa Sippaï, qui était des fils de Rapha, et ils furent humiliés\FTNT{2 S 21:15-22.}.
\VS{5}Il y eut encore une autre guerre contre les Philistins. Et Elchanan, fils de Jaïr, frappa Lachmi, frère de Goliath de Gath, qui avait une lance dont le bois était comme une ensouple de tisserand.
\VS{6}Il y eut encore une autre guerre à Gath, où se trouva un homme de grande stature, qui avait six doigts à chaque main, et six orteils à chaque pied, de sorte qu'il en avait en tout vingt-quatre ; et il était aussi issu de Rapha.
\VS{7}Et il défia Israël ; mais Jonathan, fils de Schimea, frère de David, le tua.
\VS{8}Ceux-là naquirent à Gath ; ils étaient des enfants de Rapha, et ils moururent par les mains de David, et par les mains de ses serviteurs.
\Chap{21}
\TextTitle{David fait le dénombrement contre la volonté de Yahweh\FTNTT{2 S. 24:1-17}}
\VerseOne{}Mais Satan s'éleva contre Israël, et il incita David à faire le dénombrement d'Israël.
\VS{2}Et David dit à Joab et aux chefs du peuple : Allez et faites le dénombrement d'Israël, depuis Beer-Schéba jusqu'à Dan, et rapportez-le-moi, afin que j’en connaisse le nombre.
\VS{3}Mais Joab répondit : Que Yahweh veuille augmenter son peuple cent fois encore plus qu'il ne l’est, ô roi, mon seigneur. Tous ne sont-ils pas serviteurs de mon seigneur ? Pourquoi mon seigneur cherche-t-il cela ? Et pourquoi cela serait-il imputé comme un crime à Israël ?
\VS{4}Mais la parole du roi l'emporta sur Joab. Et Joab partit et parcourut tout Israël ; puis il revint à Jérusalem.
\VS{5}Et Joab donna à David le rôle du dénombrement du peuple, et il se trouva dans tout Israël onze cent mille hommes tirant l'épée ; et dans Juda quatre cent soixante-dix mille hommes tirant l'épée.
\VS{6}Bien qu'il n'eût pas compté entre eux ni Lévi ni Benjamin, parce que Joab exécutait la parole du roi en l’ayant en abomination,
\VS{7}cette chose déplut à Dieu, c'est pourquoi il frappa Israël.
\VS{8}Et David dit à Dieu : J'ai commis un très grand péché d'avoir fait une telle chose ; je te prie, pardonne maintenant l'iniquité de ton serviteur, car j'ai agi en insensé.
\VS{9}Et Yahweh parla à Gad, le voyant de David, en disant :
\VS{10}Va, parle à David, et dis-lui : Ainsi parle Yahweh, je te propose trois choses ; choisis l'une d'elles, afin que je te la fasse.
\VS{11}Et Gad vint à David, et lui dit : Ainsi parle Yahweh :
\VS{12}Choisis soit la famine durant l'espace de trois ans ; soit d'être consumé durant trois mois, étant poursuivi par tes ennemis, en sorte que l'épée de tes ennemis t'atteigne ; ou que l'épée de Yahweh, la peste, soit durant trois jours sur le pays, et que l'Ange de Yahweh porte la destruction dans toutes les contrées d'Israël. Vois maintenant ce que j'aurai à répondre à celui qui m'a envoyé.
\VS{13}Alors David répondit à Gad : Je suis dans une très grande angoisse ! Que je tombe, je te prie, entre les mains de Yahweh, parce que ses compassions sont immenses ; mais que je ne tombe point entre les mains des hommes !
\VS{14}Yahweh envoya donc la peste sur Israël, et il tomba soixante-dix mille hommes d'Israël.
\VS{15}Dieu envoya aussi un ange à Jérusalem pour la détruire ; et comme il la détruisait, Yahweh regarda et se repentit de ce mal. Et il dit à l'Ange qui détruisait : C'est assez ! Retire à présent ta main. Et l'Ange de Yahweh se tenait près de l'aire d'Ornan, le Jébusien.
\VS{16}Or David leva les yeux,  et vit l'Ange de Yahweh\FTNT{Ge. 16:7.} qui était entre la terre et le ciel, ayant dans sa main son épée nue, tournée contre Jérusalem. Et David et les anciens, couverts de sacs, tombèrent sur leurs faces.
\VS{17}Et David dit à Dieu : N'est-ce pas moi qui ai ordonné qu'on fasse le dénombrement du peuple ? C'est donc moi qui ai péché et qui ai très mal agi ; mais ces brebis qu'ont-elles fait ? Yahweh, mon Dieu ! Je te prie que ta main soit contre moi, et contre la maison de mon père, mais qu'elle ne soit pas contre ton peuple, pour le détruire.
\TextTitle{Fin de la plaie après l’offrande de David\FTNTT{2 S. 24:18-25}}
\VS{18}Alors l'Ange de Yahweh ordonna à Gad de dire à David, qu'il monte pour dresser un autel à Yahweh, dans l'aire d'Ornan, le Jébusien.
\VS{19}David donc monta selon la parole que Gad lui avait dite au Nom de Yahweh.
\VS{20}Ornan s'étant retourné, et ayant vu l'Ange,  ses quatre fils se cachèrent avec lui. Or Ornan foulait du blé.
\VS{21}David vint jusqu'à Ornan, et Ornan regarda, et ayant vu David, il sortit de l'aire et se prosterna devant lui, le visage à terre.
\VS{22}Et David dit à Ornan : Donne-moi la place de cette aire, et j'y bâtirai un autel à Yahweh ; donne-la-moi pour le prix qu'elle vaut, afin que cette plaie soit arrêtée de dessus le peuple.
\VS{23}Et Ornan dit à David : Prends-la, et que le roi, mon seigneur, fasse tout ce qui lui semblera bon. Voici, je donne ces bœufs pour les holocaustes, et ces instruments à fouler du blé pour le bois, et ce blé pour l'offrande ; je donne toutes ces choses.
\VS{24}Mais le roi David lui répondit : Non, mais certainement j'achèterai tout cela au prix qu'il vaut ; car je ne présenterai point à Yahweh ce qui est à toi, et je n'offrirai point un holocauste qui ne me coûte rien.
\VS{25}David donna donc à Ornan pour cette place, six cents sicles d'or de poids.
\VS{26}Puis il bâtit là un autel à Yahweh, et il offrit des holocaustes et des sacrifices d’offrande de paix, et il invoqua Yahweh, qui l'exauça par le feu envoyé des cieux sur l'autel de l'holocauste.
\VS{27}Alors Yahweh parla à l'ange, et l'ange remit son épée dans son fourreau.
\VS{28}En ce temps-là, David, voyant que Yahweh l'avait exaucé dans l'aire d'Ornan, le Jébusien, y offrait des sacrifices.
\VS{29}Or le tabernacle de Yahweh, que Moïse avait construit au désert, et l'autel des holocaustes, étaient en ce temps-là dans le haut lieu de Gabaon.
\VS{30}Mais David ne pouvait pas aller devant cet autel pour invoquer Dieu, parce qu'il avait été épouvanté à cause de l'épée de l'Ange de Yahweh.
\Chap{22}
\TextTitle{Préparatifs de David pour la construction du temple}
\VerseOne{}Et David dit : C'est ici la maison de Yahweh Dieu, et c'est ici l'autel pour les holocaustes d'Israël.
\VS{2}David ordonna de rassembler les étrangers qui étaient dans le pays d'Israël, et il établit des tailleurs de pierres pour tailler des pierres de taille, pour la construction de la maison de Dieu.
\VS{3}David prépara aussi du fer en abondance, afin d'en faire des clous pour les battants des portes et pour les crampons,  de l’airain en quantité telle qu’il n’était pas possible de le peser,
\VS{4}et du bois de cèdre sans nombre, parce que les Sidoniens et les Tyriens amenaient à David du bois de cèdre en abondance.
\VS{5}David dit : Salomon, mon fils, est jeune et délicat, et la maison qu'il faut bâtir à Yahweh doit être magnifique en excellence, en réputation, et en gloire, dans tous les pays. Je lui préparerai donc maintenant de quoi la bâtir. Ainsi, David prépara, avant sa mort, ces choses en abondance.
\TextTitle{Recommandation de David à Salomon }
\VS{6}Puis il appela Salomon, son fils, et lui ordonna de bâtir une maison à Yahweh, le Dieu d'Israël.
\VS{7}David donc dit à Salomon : Mon fils, j’avais à cœur de bâtir une maison au Nom de Yahweh, mon Dieu.
\VS{8}Mais la parole de Yahweh m'a été adressée, en disant : Tu as répandu beaucoup de sang, et tu as fait de grandes guerres ; tu ne bâtiras point de maison à mon Nom, parce que tu as répandu beaucoup de sang sur la terre devant moi.
\VS{9}Voici, il te naîtra un fils, qui sera un homme de repos,  et à qui je donnerai du repos par rapport à tous ses ennemis tout autour, c'est pourquoi son nom sera Salomon. Et en son temps, je donnerai la paix et le repos à Israël.
\VS{10}Ce sera lui qui bâtira une maison à mon Nom ; et il sera un fils pour moi, et je serai un père pour lui ; et j'affermirai le trône de son règne sur Israël à jamais.
\VS{11}Maintenant donc, mon fils, Yahweh sera avec toi, et tu prospéreras, et tu bâtiras la maison de Yahweh, ton Dieu, ainsi qu'il l’a déclaré à ton égard.
\VS{12}Seulement, que Yahweh te donne de la sagesse et de l'intelligence, et qu'il t'instruise touchant le gouvernement d'Israël, et comment tu dois garder la loi de Yahweh, ton Dieu.
\VS{13}Tu prospéreras si tu as soin de mettre en pratique les lois et les ordonnances que Yahweh a prescrites à Moïse pour Israël. Fortifie-toi et prends courage ; ne crains point et ne t'effraie de rien.
\VS{14}Voici, selon ma petitesse, j'ai préparé pour la maison de Yahweh cent mille talents d'or et un million de talents d'argent. Quant à l'airain et au fer, il est d'un poids incalculable, car il est en abondance. J'ai aussi préparé le bois et les pierres ; et tu y ajouteras ce qu'il faudra .
\VS{15}Tu as avec toi beaucoup d'ouvriers, de maçons, de tailleurs de pierres, de charpentiers, et toutes sortes de gens experts dans toute espèce d’ouvrage.
\VS{16}Il y a de l'or et de l'argent, de l'airain et du fer sans nombre. Lève-toi et agis, et Yahweh sera avec toi.
\VS{17}David ordonna aussi à tous les chefs d'Israël d'aider Salomon, son fils ; et il leur dit :
\VS{18}Yahweh,votre Dieu, n'est-il pas avec vous, et ne vous a-t-il pas donné du repos de tous côtés ? Car il a livré entre mes mains les habitants du pays, et le pays a été soumis devant Yahweh, et devant son peuple.
\VS{19}Maintenant donc, appliquez vos cœurs et vos âmes à rechercher Yahweh, votre Dieu ; levez-vous et bâtissez le sanctuaire de Yahweh Dieu, afin d’amener l’arche de l'alliance de Yahweh, et les ustensiles consacrés à Dieu dans la maison qui doit être bâtie au Nom de Yahweh.
\Chap{23}
\TextTitle{David désigne Salomon comme son successeur\FTNTT{1 Ch. 28:1}}
\VerseOne{}David étant vieux et rassasié de jours, établit Salomon, son fils, pour roi sur Israël.
\VS{2}Et il assembla tous les principaux d'Israël, les sacrificateurs et les Lévites.
\VS{3}On fit le dénombrement des Lévites, depuis l'âge de trente ans et au-dessus ; et les mâles d' entre-eux étant comptés, chacun par tête, il y eut trente-huit mille hommes\FTNT{No. 3:25-37}.
\VS{4}Et David dit : Qu’il y en ait  parmi eux vingt-quatre mille pour vaquer ordinairement à l'œuvre de la maison de Yahweh, et six mille comme  magistrats et juges,
\VS{5}quatre mille portiers, et quatre autres mille pour louer Yahweh avec des instruments que j'ai faits pour le louer.
\TextTitle{Dénombrement des Lévites\FTNTT{No. 3:25-37}}
\VS{6}David les divisa en classes d'après les fils de Lévi, à savoir Guerschon, Kehath et Merari.
\VS{7}Des Guerschonites il y eut Laedan et Schimeï.
\VS{8}Les fils de Laedan furent ces trois : Jehiel le premier, puis Zétham, puis Joël.
\VS{9}Les fils de Schimeï furent ces trois : Schelomith, Haziel et Haran. Ce sont là les chefs des maisons paternelles de la famille de Laedan.
\VS{10}Et les fils de Schimeï furent Jachath, Zina, Jeusch et Beria. Ce sont là les quatre fils de Schimeï.
\VS{11}Jachath était le premier et Zina le second; mais Jeusch et Beria n'eurent pas beaucoup de fils, c'est pourquoi ils furent comptés pour une seule maison paternelle dans le dénombrement.
\VS{12}Des fils de Kehath il y eut Amram, Jitsehar, Hébron et Uziel, en tout quatre.
\VS{13}Les fils d’Amram furent Aaron et Moïse. Aaron fut séparé lui et ses fils à toujours, pour sanctifier le Saint des saints, pour faire brûler des parfums en présence de Yahweh, pour le servir, et pour bénir en son Nom à toujours.
\VS{14}Et quant à Moïse, homme de Dieu, ses fils devaient être comptés de la tribu de Lévi.
\VS{15}Les fils de Moïse furent Guerschom et Eliézer.
\VS{16}Des fils de Guerschom, Schebuel le premier.
\VS{17}Quant aux fils d' Eliézer, Rechabia fut le premier ; et Eliézer n'eut point d'autres fils, mais les fils de Rechabia furent très nombreux.
\VS{18}Des fils de Jitsehar, Schelomith était le premier.
\VS{19}Les fils de Hébron furent Jerija le premier, Amaria le second, Jachaziel le troisième, Jekameam le quatrième.
\VS{20}Les fils d’Uziel furent Michée le premier, Jischija le second.
\VS{21}Des fils de Merari il y eut Machli et Muschi. Les fils de Machli furent Eléazar et Kis.
\VS{22}Eléazar mourut, et n'eut point de fils, mais des filles ; et les fils de Kis, leurs frères les prirent pour femmes.
\VS{23}Les fils de Muschi furent Machli, Eder et Jerémoth, eux trois.
\TextTitle{Fonctions des Lévites\FTNTT{No. 3:5-12}}
\VS{24}Ce sont là les fils de Lévi, selon les maisons de leurs pères, chefs des maisons paternelles, selon leurs dénombrements qui furent faits en comptant leurs noms, étant comptés chacun par tête ; et ils faisaient l’œuvre du service de la maison de Yahweh, depuis l'âge de vingt ans et au-dessus.
\VS{25}Car David dit : Yahweh, le Dieu d'Israël, a donné du repos à son peuple, et il établira sa demeure dans Jérusalem  à toujours.
\VS{26}Quant aux Lévites, ils n'auront plus à porter le tabernacle ni tous les ustensiles pour son service.
\VS{27}C'est pourquoi, dans les derniers registres de David, les fils de Lévi furent dénombrés depuis l'âge de vingt ans et au-dessus.
\VS{28}Car leur charge était d'assister les fils d'Aaron pour le service de la maison de Yahweh, étant établis sur le parvis et sur les chambres, pour la purification de toutes les choses saintes, pour l'œuvre du service de la maison de Dieu,
\VS{29}pour les pains de proposition, de la fleur de farine pour l'offrande, des galettes sans levain, pour tout ce qui se cuit sur la plaque, pour tout ce qui est rissolé, et pour la petite et grande mesure,
\VS{30}pour se présenter tous les matins et tous les soirs, afin de célébrer et louer Yahweh,
\VS{31}et offrir tous les holocaustes qu'il fallait offrir à Yahweh les jours de sabbat, aux nouvelles lunes, et aux fêtes solennelles, continuellement devant Yahweh, selon le nombre et les usages prescrits.
\VS{32}Ils donnaient leurs soins à la tente d’assignation, au lieu saint, et aux fils d’Aaron, leurs frères, pour le service de la maison de Yahweh.
\Chap{24}
\TextTitle{Vingt-quatre classes de sacrificateurs}
\VerseOne{}Quant aux fils d'Aaron, voici leurs classes\FTNT{Les vingt-quatre classes de sacrificateurs qui se tenaient devant Yahweh dans le temple de Jérusalem étaient une représentation des vingt-quatre vieillards qui se tiennent devant le trône de Dieu (Ap. 4:4).}. Les fils d’Aaron furent Nadab, Abihu, Eléazar et Ithamar.
\VS{2}Mais Nadab et Abihu\FTNT{Lé. 10:1-4.} moururent en présence de leur père, et n'eurent point de fils ; et Eléazar et Ithamar exercèrent la sacrificature.
\VS{3}Or David les sépara, à savoir Tsadok, qui était des fils d'Eléazar, et Achimélec, qui était des fils d'Ithamar, en fonction de leurs charges dans le service qu'ils avaient à faire.
\VS{4}Il se trouva parmi les fils d'Eléazar plus de chefs que parmi les fils d'Ithamar, et on en fit la division ; les fils d'Eléazar avaient seize chefs, selon leurs maisons paternelles, et les fils d'Ithamar huit chefs de maisons paternelles.
\VS{5}Et on les classa par le sort, les entremêlant les uns avec les autres, car les chefs du sanctuaire et les chefs de la maison de Dieu furent tirés tant des fils d'Eléazar que des fils d'Ithamar.
\VS{6}Schemaeja, fils de Nethaneel, le scribe, qui était de la tribu de Lévi, les mit par écrit devant le roi, les princes du peuple, devant Tsadok, le sacrificateur, et Achimélec, fils d'Abiathar, et devant les chefs de maisons paternelles des sacrificateurs et des Lévites. On tira au sort une maison paternelle pour Eléazar, et une autre fut tirée pour Ithamar.
\VS{7}Le premier sort échut à Jehojarib, le second à Jedaeja,
\VS{8}le troisième à Harim, le quatrième à Seorim,
\VS{9}le cinquième à Malkija, le sixième à Mijamin,
\VS{10}le septième à Hakkots, le huitième à Abija,
\VS{11}le neuvième à Josué, le dixième à Schecania,
\VS{12}le onzième à Eliaschib, le douzième à Jakim,
\VS{13}le treizième à Huppa, le quatorzième à Jeschébeab,
\VS{14}le quinzième à Bilga, le seizième à Immer,
\VS{15}le dix-septième à Hézir, le dix-huitième à Happitsets,
\VS{16}le dix-neuvième à Pethachja, le vingtième à Ezéchiel,
\VS{17}le vingt et unième à Jakim, et le vingt-deuxième à Gamul,
\VS{18}le vingt-troisième à Delaja, le vingt-quatrième à Maazia.
\VS{19}Tel fut leur classement pour le service qu'ils avaient à faire, lorsqu'ils entraient dans la maison de Yahweh, selon qu'il leur avait été ordonné par Aaron, leur père, comme Yahweh, le Dieu d'Israël, le lui avait ordonné.
\TextTitle{Les chefs des lévites ; les fils de Kehath et de Merari}
\VS{20}Voici les chefs du reste des Lévites. Des fils de Amram : Schubaël ; et des fils de Shubaël, Jechdia.
\VS{21}De Rechabia, des fils de Rechabia, Jischija était le premier.
\VS{22}Des Jitseharites, Schelomoth ; des fils de Schelomoth, Jachath.
\VS{23}Des fils d’Hébron, Jerija, Amaria le second ; Jachaziel le troisième, Jekameam le quatrième.
\VS{24}Des fils d'Uziel, Michée ; des fils de Michée, Schamir.
\VS{25}Le frère de Michée était Jischija ; des fils de Jischija, Zacharie.
\VS{26}Des fils de Merari, Machli et Muschi. Des fils de Jaazija, son fils.
\VS{27}Des fils de Merari, de Jaazija, son fils : Schoham, Zaccur et Ibri.
\VS{28}De Machli, Eléazar, qui n'eut point de fils.
\VS{29}De Kis, les fils de Kis, Jerachmeel.
\VS{30}Et des fils de Muschi, Machli, Eder et Jerimoth. Ce sont là les fils des Lévites, selon les maisons de leurs pères.
\VS{31}Eux aussi, comme leurs frères, les fils d'Aaron, ils tirèrent au sort, devant le  roi David, Tsadok et Ahimélec, et les chefs des pères des sacrificateurs et des Lévites. Il en fut ainsi pour chaque chef de maison comme pour le moindre de ses frères.
\Chap{25}
\TextTitle{Dénombrement des musiciens et des chantres}
\VerseOne{}David et les chefs de l'armée mirent à part pour le service ceux des fils d'Asaph, d'Héman et de Jeduthun qui prophétisaient avec des harpes, des luths et des cymbales. Et voici le nombre des hommes employés pour le service qu’ils avaient à faire.
\VS{2}Des fils d'Asaph : Zaccur, Joseph, Nethania et Aschareéla, fils d'Asaph, sous la conduite d'Asaph, qui prophétisait selon les ordres du roi.
\VS{3}De Jeduthun, les six fils de Jeduthun : Guedalia, Tseri, Esaïe, Haschabia, Matthithia et Schimeï, jouaient de la harpe, sous la conduite de leur père Jeduthun, qui prophétisait en célébrant et louant Yahweh.
\VS{4}D'Héman, les fils d'Héman : Bukkija, Matthania, Uziel, Schebuel, Jerimoth, Hanania, Hanani, Eliatha, Guiddalthi, Romamthi-Ezer, Joschbekascha, Mallothi, Hothir, Machazioth.
\VS{5}Tous ceux-là étaient fils d'Héman, le voyant du roi, qui révélait les paroles de Dieu pour en exalter la puissance. Dieu donna à Héman quatorze fils et trois filles.
\VS{6}Tous ceux-là étaient employés, sous la conduite de leurs pères, aux cantiques de la maison de Yahweh, avec des cymbales, des luths, et des harpes, dans le service de la maison de Dieu, selon les ordres du roi donnés à Asaph, à Jeduthun et à Héman.
\VS{7}Et leur nombre avec leurs frères, auxquels on avait enseigné les cantiques de Yahweh, était de deux cent quatre-vingt-huit, tous très habiles.
\TextTitle{Les musiciens et chantres répartis en vingt-quatre classes}
\VS{8}Et ils tirèrent au sort pour leurs fonctions, petits et grands, maîtres et disciples.
\VS{9}Et le premier sort échut à Asaph, à savoir à Joseph. Le second à Guedalia, lui, ses frères et ses fils étaient douze.
\VS{10}Le troisième à Zaccur, lui, ses fils et ses frères étaient douze.
\VS{11}Le quatrième à Jitseri, lui, ses fils et ses frères étaient douze.
\VS{12}Le cinquième à Nethania, lui, ses fils et ses frères étaient douze.
\VS{13}Le sixième à Bukkija, lui, ses fils et ses frères étaient douze.
\VS{14}Le septième à Jesareéla, lui, ses fils et ses frères étaient douze.
\VS{15}Le huitième à Esaïe, lui, ses fils et ses frères étaient douze.
\VS{16}Le neuvième à Matthania, lui, ses fils et ses frères étaient douze.
\VS{17}Le dixième à Schimeï, lui, ses fils et ses frères étaient douze.
\VS{18}L'onzième à Azareel, lui, ses fils et ses frères étaient douze.
\VS{19}Le douzième à Haschabia, lui, ses fils, et ses frères étaient douze.
\VS{20}Le treizième à Schubaël, lui, ses fils et ses frères étaient douze.
\VS{21}Le quatorzième à Matthithia, lui, ses fils et ses frères étaient douze.
\VS{22}Le quinzième à Jerémoth, lui, ses fils et ses frères étaient douze.
\VS{23}Le seizième à Hanania, lui, ses fils et ses frères étaient douze.
\VS{24}Le dix-septième à Joschbekascha, lui, ses fils et ses frères étaient douze.
\VS{25}Le dix-huitième à Hanani, lui, ses fils et ses frères étaient douze.
\VS{26}Le dix-neuvième à Mallothi, lui, ses fils et ses frères étaient douze.
\VS{27}Le vingtième à Elijatha, lui, ses fils et ses frères étaient douze.
\VS{28}Le vingt et unième à Hothir, lui, ses fils et ses frères étaient douze.
\VS{29}Le vingt-deuxième à Guiddalthi, lui, ses fils et ses frères étaient douze.
\VS{30}Le vingt-troisième à Machazioth, lui, ses fils et ses frères étaient douze.
\VS{31}Le vingt-quatrième à Romamthi-Ezer, lui, ses fils et ses frères étaient douze.
\Chap{26}
\TextTitle{Les classes des portiers}
\VerseOne{}Et quant aux classes des portiers, il y eut pour les Koréites : Meschélémia, fils de Koré, d'entre les fils d'Asaph.
\VS{2}Les fils de Meschélémia furent Zacharie, le premier-né, Jediaël le second, Zebadia le troisième, Jathniel le quatrième,
\VS{3}Elam le cinquième, Jochanan le sixième et Eljoénaï le septième.
\VS{4}Les fils d’Obed-Edom furent Schemaeja le premier-né, Jozabad le second, Joach le troisième, Sacar le quatrième, Nethaneel le cinquième,
\VS{5}Ammiel le sixième, Issacar le septième, Peulthaï le huitième ; car Dieu l'avait béni.
\VS{6}A Schemaeja, son fils, naquirent des fils qui eurent le commandement sur la maison de leur père, parce qu'ils étaient des hommes forts et vaillants.
\VS{7}Les fils donc de Schemaeja furent Othni, Rephaël, Obed, Elzabad et ses frères, hommes vaillants, Elihu et Semaeja.
\VS{8}Tous ceux-là étaient des fils d' Obed-Edom, eux, leurs fils et leurs frères, étaient des hommes pleins de vigueur et de force pour le service ; ils étaient soixante-deux d'Obed-Edom.
\VS{9}Les fils de Meschélémia avec ses frères, vaillants hommes étaient au nombre de dix-huit.
\VS{10}Les fils de Hosa, d'entre les fils de Merari, furent  Schimri le chef, quoiqu'il ne fût pas l'aîné, néanmoins son père l'établit pour chef ;
\VS{11}Hilkija était le second, Thebalia le troisième, Zacharie le quatrième; tous les fils et frères de Hosa furent treize.
\VS{12}A ces classes de portiers, aux chefs de ces hommes et à leurs frères, fut remise la garde pour le service de la maison de Yahweh.
\VS{13}Ils tirèrent au sort pour chaque porte, autant pour le plus petit que pour le plus grand, selon leurs familles.
\VS{14}Et ainsi, le sort pour la porte vers l'orient échut à Schélémia. Puis on tira au sort pour Zacharie, son fils, qui était un sage conseiller, et la porte du côté du nord lui fut échue par le sort.
\VS{15}Le sort d'Obed-Edom lui échut pour la porte du côté du sud, et la maison des magasins échut à ses fils.
\VS{16}A Schuppim et à Hosa pour la porte vers l'occident, auprès de la porte de Schalléketh, au chemin montant ; une garde étant vis-à-vis de l'autre.
\VS{17}Il y avait vers l'orient six Lévites ; vers le nord, quatre par jour  ; vers le sud, quatre aussi par jour ; et vers la maison des magasins, deux de chaque côté ;
\VS{18}du côté de la banlieue vers l'occident, il y en avait quatre au chemin, et deux vers la banlieue.
\VS{19}Ce sont là les classes des portiers pour les fils des Koréites, et pour les fils de Merari.
\TextTitle{Les Lévites commis sur les trésors du temple}
\VS{20}Ceux-ci aussi étaient Lévites : Achija commis sur les trésors de la maison de Dieu et les trésors des choses consacrées.
\VS{21}Des fils de Laedan, qui étaient d'entre les fils des Guerschonites, du côté de Laedan, d'entre les chefs des maisons paternelles appartenant à Laedan le Guerschonite, Jehiéli.
\VS{22}D'entre les fils de Jehiéli : Zétham et Joël, son frère, commis sur les trésors de la maison de Yahweh.
\VS{23}Pour les Amramites, les Jitseharites, les Hébronites et les  Uziélites,
\VS{24}Schebuel, fils de Guerschom, fils de Moïse, était commis sur les autres trésors.
\VS{25}Et quant à ses frères issus d'Eliézer, dont Rechabia fut fils, dont le fils fut Esaïe, dont le fils fut Joram, dont le fils fut  Zicri, dont le fils fut Schelomith,
\VS{26}c’étaient Schelomith et ses frères qui gardaient tous les trésors des choses saintes que le roi David, les chefs des familles paternelles, les chefs de milliers et de centaines, et les chefs de l'armée avaient consacrées.
\VS{27}C'était le butin de guerre qu'ils avaient consacré, pour l’entretien de la maison de Yahweh.
\VS{28}Tout ce qu'avait consacré Samuel, le voyant, Saül, fils de Kis, Abner, fils de Ner et Joab, fils de Tseruja, toutes les choses consacrées étaient mises sous la main de Schelomith et de ses frères.
\TextTitle{Les magistrats et juges en Israël}
\VS{29}Parmi les Jitseharites, Kenania et ses fils étaient employés aux affaires extérieures sur Israël pour être magistrats et juges.
\VS{30}Quant aux Hébronites, Haschabia et ses frères, hommes vaillants, au nombre de mille sept cents, avaient la surveillance d'Israël de l’autre côté du Jourdain, vers l'occident, pour toute œuvre qui concernait Yahweh, et pour le service du roi.
\VS{31}Quant aux Hébronites, selon leurs générations dans les maisons paternelles, Jerija fut le chef des Hébronites. On fit une recherche au sujet des Hébronites à la quarantième année du règne de David, et on trouva parmi eux à Jaezer de Galaad, des hommes forts et vaillants.
\VS{32}Les frères de Jerija, hommes vaillants, furent deux mille sept cents, issus des maisons paternelles ; et le roi David les établit sur les Rubénites, sur les Gadites, et sur la demi-tribu de Manassé, pour toute œuvre qui concernait Dieu, et pour les affaires du roi.
\Chap{27}
\TextTitle{Les douze chefs de guerre de David}
\VerseOne{}Quant aux fils d'Israël, selon leur dénombrement, il y avait des chefs de maisons paternelles, des chefs de milliers et de centaines, et leurs officiers, qui servaient le roi pour tout ce qui concernait les divisions, leur arrivée et leur départ, mois par mois, pendant tous les mois de l'année, et chaque division était de vingt-quatre mille hommes.
\VS{2}Et Jaschobeam, fils de Zabdiel, présidait sur la première division, pour le premier mois ; et dans sa division il y avait vingt-quatre mille hommes.
\VS{3}Il était des fils de Pérets, chef de tous les capitaines de l'armée du premier mois.
\VS{4}Dodaï, l'Achochite, présidait sur la division du deuxième mois, Mikloth, l’un des chefs de sa division ; et il avait une division de vingt-quatre mille hommes.
\VS{5}Le chef de la troisième armée pour le troisième mois était Benaja, fils de Jehojada, le sacrificateur et le capitaine en chef ; et dans sa division il y avait vingt-quatre mille hommes.
\VS{6}C'est ce Benaja qui était fort entre les trente, et par dessus les trente ; et Ammizadab, son fils, était dans sa division.
\VS{7}Le quatrième pour le quatrième mois était Asaël, frère de Joab, et Zébadia son fils, après lui ; et il y avait dans sa division vingt-quatre mille hommes.
\VS{8}Le cinquième pour le cinquième mois était le capitaine Schamehuth, le Jizrachite ; et dans sa division il y avait vingt-quatre mille hommes.
\VS{9}Le sixième pour le sixième mois était Ira, fils d' Ikkesch le Tekoïte ; et dans sa division il y avait vingt-quatre mille hommes.
\VS{10}Le septième pour le septième mois était Hélets le Pelonite, des fils d'Ephraïm ; et il y avait dans sa division vingt-quatre mille hommes.
\VS{11}Le huitième pour le huitième mois était Sibbecaï le Huschatite, de la famille des Zérachites ; et il y avait dans sa division vingt-quatre mille hommes.
\VS{12}Le neuvième pour le neuvième mois était Abiézer d'Anathoth, des Benjamites ; et il y avait dans sa division vingt-quatre mille hommes.
\VS{13}Le dixième pour le dixième mois était Maharaï de Nethopha, de la famille des Zérachites ; et il y avait dans sa division vingt-quatre mille hommes.
\VS{14}Le onzième pour le onzième mois était Benaja de Pirathon, des fils d'Ephraïm; et il y avait dans sa division vingt-quatre mille hommes.
\VS{15}Le douzième pour le douzième mois était Heldaï de Nethopha, appartenant à Othniel; et il y avait dans sa division vingt-quatre mille hommes.
\TextTitle{Les douze chefs des tribus d’Israël}
\VS{16}Et ceux-ci présidaient sur les tribus d'Israël : Eliézer, fils de Zicri, était le conducteur des Rubénites. Des Siméonites : Schephathia, fils de Maaca.
\VS{17}Des Lévites, Haschabia, fils de Kemuel. De ceux d'Aaron : Tsadok.
\VS{18}De Juda : Elihu, qui était des frères de David. De ceux d'Issacar : Omri, fils de Micaël.
\VS{19}De ceux de Zabulon : Jischemaeja, fils d'Abdias. De ceux de Nephthali : Jerimoth, fils d'Azriel.
\VS{20}Des fils d'Ephraïm : Hosée, fils d'Azazia. De la demi-tribu de Manassé : Joël, fils de Pedaja.
\VS{21}De l'autre demi-tribu de Manassé en Galaad : Jiddo, fils de Zacharie. De ceux de Benjamin : Jaasiel, fils d'Abner.
\VS{22}De ceux de Dan : Azareel, fils de Jerocham. Ce sont là les chefs des tribus d'Israël.
\TextTitle{Dénombrement arrêté par Yahweh}
\VS{23}Mais David ne fit point le dénombrement des Israélites, depuis l'âge de vingt ans et au-dessous ; parce que Yahweh avait dit qu'il multiplierait Israël comme les étoiles du ciel.
\VS{24}Joab, fils de Tseruja, avait bien commencé à en faire le dénombrement, mais il n'acheva pas parce que la colère de Dieu s'était répandue à cause de cela sur Israël ; c'est pourquoi ce dénombrement ne fut point mis parmi les dénombrements enregistrés dans les Chroniques du roi David.
\TextTitle{Les gestionnaires de David}
\VS{25}Or Azmaveth, fils d'Adiel, était commis sur les finances du roi ; mais Jonathan, fils d'Ozias, était commis sur les provisions dans les champs, dans les villes, les villages et les châteaux.
\VS{26}Et Ezri, fils de Kelub, était commis sur ceux qui travaillaient dans la campagne et cultivaient la terre.
\VS{27}Et Schimeï de Rama sur les vignes, et Zabdi de Schepham sur ce qui provenait des vignes, et sur les celliers du vin.
\VS{28}Et Baal-Hanan de Guéder sur les oliviers et sur les figuiers qui étaient à la campagne ; et Joasch sur les celliers à huile.
\VS{29}Schithraï de Saron était commis sur le gros bétail qui paissait en Saron ; Schaphath, fils d'Adlaï, sur le gros bétail qui paissait dans les vallées.
\VS{30}Obil, l'Ismaélite, sur les chameaux; Jechdia de Méronoth, sur les ânesses.
\VS{31}Jaziz, l'Hagarénien, sur les troupeaux du menu bétail. Tous ceux-là avaient la charge des biens qui appartenaient au roi David.
\TextTitle{Les conseillers de David}
\VS{32}Mais Jonathan, oncle de David, était conseiller, homme très intelligent et scribe ; et Jehiel, fils de Hacmoni, était avec les fils du roi.
\VS{33}Achitophel était le conseiller du roi ; et Huschaï, l'Arkien, était l'intime ami du roi.
\VS{34}Après Achitophel était Jehojada, fils de Benaja et Abiathar; et Joab était le chef de l'armée du roi.
\Chap{28}
\TextTitle{Dernières paroles de David, la royauté remise à Salomon\FTNTT{1 Ch. 23:2}}
\VerseOne{}David convoqua à Jérusalem tous les chefs d'Israël, les chefs des tribus, et les chefs des divisions qui servaient le roi ; et les chefs de milliers et de centaines, et ceux qui avaient la charge de tous les biens du roi, et de tout ce qu’il possédait, ses fils avec ses eunuques, et les hommes puissants, et tous les  héros et tous les hommes vaillants.
\VS{2}Puis le roi David se leva sur ses pieds, et dit : Mes frères et mon peuple, écoutez-moi ! J’avais à cœur de bâtir une maison de repos pour l’arche de l'alliance de Yahweh, et pour le marchepied de notre Dieu, et j'ai fait les préparatifs pour la bâtir.
\VS{3}Mais Dieu m'a dit : Tu ne bâtiras point de maison à mon Nom, parce que tu es un homme de guerre, et que tu as répandu beaucoup de sang.
\VS{4}Or comme Yahweh, le Dieu d'Israël, m'a choisi dans toute la maison de mon père pour être roi sur Israël à toujours ; car il a choisi Juda pour conducteur, et de la maison de Juda la maison de mon père, et d'entre les fils de mon père il a pris son plaisir en moi, pour me faire régner sur tout Israël.
\VS{5}Aussi, entre tous mes fils, car Yahweh m'a donné beaucoup de fils, il a choisi Salomon, mon fils, pour le faire asseoir sur le trône du royaume de Yahweh, sur Israël.
\VS{6}Et il m'a dit : Salomon, ton fils, est celui qui bâtira ma maison et mes parvis ; car je me le suis choisi pour fils et je serai pour lui un père.
\VS{7}Et j'affermirai son règne à toujours s'il s'applique à pratiquer mes commandements et à observer mes ordonnances, comme il le fait aujourd'hui.
\VS{8}Maintenant donc, je vous somme en présence de tout Israël, qui est l'assemblée de Yahweh, et devant notre Dieu qui l'entend, que vous ayez à garder et à rechercher diligemment tous les commandements de Yahweh, votre Dieu, afin que vous possédiez ce bon pays, et que vous le fassiez hériter à vos fils après vous, à jamais.
\VS{9}Et toi, Salomon, mon fils, connais le Dieu de ton père, et sers-le avec un cœur droit et une bonne volonté ; car Yahweh sonde tous les cœurs, et connaît toutes les dispositions des pensées. Si tu le cherches, il se laissera trouver par toi ; mais si tu l'abandonnes, il te rejettera pour toujours.
\VS{10}Considère maintenant que Yahweh t'a choisi pour bâtir une maison pour son sanctuaire. Fortifie-toi donc et applique-toi à y travailler.
\VS{11}David donna à Salomon, son fils, le modèle\FTNT{David donna le modèle du temple qu’il  avait reçu de Dieu à Salomon. Beaucoup veulent servir Dieu sans modèle, tandis que d’autres vont chercher des modèles dans le monde (2 R. 16:10-18). Nous devons faire l’œuvre de Dieu uniquement selon le modèle biblique.} du portique, de ses maisons, des chambres du trésor, des chambres hautes, des chambres intérieures et du lieu du propitiatoire.
\VS{12}Il lui donna le modèle de toutes les choses qui lui avaient été inspirées par l'Esprit qui était avec lui, pour les parvis de la maison de Yahweh, pour les chambres d'alentour, pour les trésors de la maison de Yahweh et pour les trésors des choses saintes ;
\VS{13}pour les divisions des sacrificateurs et des Lévites, pour toute l'œuvre du service de la maison de Yahweh, et pour tous les ustensiles du service de la maison de Yahweh.
\VS{14}Il lui donna aussi de l'or, un certain poids, pour les choses qui devaient être d'or, à savoir pour tous les ustensiles de chaque service ; et de l'argent, un certain poids, pour tous les ustensiles d'argent, à savoir pour tous les ustensiles de chaque service.
\VS{15}Le poids des chandeliers d'or, et de leurs lampes d'or, selon le poids de chaque chandelier et de ses lampes ; et le poids des chandeliers d'argent, selon le poids de chaque chandelier et de ses lampes, selon l’usage de chaque chandelier.
\VS{16}Et le poids de l'or pesant ce qu'il fallait pour chaque table des pains de proposition; et de l'argent pour les tables d'argent.
\VS{17}Il lui donna le modèle pour les fourchettes, pour les bassins et pour les calices d’or pur ; le modèle pour les coupes d'or, selon le poids de chaque coupe,  et de l'argent pour les coupes d'argent, selon le poids de chaque coupe ;
\VS{18}et le modèle pour l'autel des parfums en or épuré, avec le poids. Il lui donna encore le modèle du char, des chérubins d’or qui étendent les ailes et qui couvrent l’arche de l'alliance de Yahweh.
\VS{19}C’est par un écrit de sa main, dit-il, que Yahweh m’a donné l'intelligence de tout cela, de tous les ouvrages de ce modèle.
\TextTitle{David demande à Salomon de  bâtir le temple}
\VS{20}C'est pourquoi David dit à Salomon, son fils : Fortifie-toi, prends courage et travaille ; ne crains point et ne t'effraie point. Car Yahweh Dieu, mon Dieu, sera avec toi, il ne te délaissera point, et il ne t'abandonnera point, jusqu'à ce que tu aies achevé tout l'ouvrage du service de la maison de Yahweh.
\VS{21}Et voici, j'ai fait les divisions des sacrificateurs et des Lévites pour tout le service de la maison de Dieu ; et il y a avec toi pour tout cet ouvrage toutes sortes de gens prompts et experts, pour toutes sortes de services ; et les chefs avec tout le peuple seront prêts pour exécuter tout ce que tu diras.
\Chap{29}
\TextTitle{Offrandes volontaires de David et de tout le peuple}
\VerseOne{}Puis le roi David dit à toute l'assemblée : Dieu a choisi un seul de mes fils, à savoir Salomon, qui est encore jeune et délicat, et l'ouvrage est considérable, car ce palais n'est point pour un homme, mais pour Yahweh Dieu.
\VS{2}Et moi, j'ai préparé de toutes mes forces pour la maison de mon Dieu, de l'or pour les choses qui doivent être d'or, de l'argent pour celles qui doivent être d'argent, de l'airain pour celles d'airain, du fer pour celles de fer, du bois pour celles de bois, des pierres d'onyx, et des pierres pour être enchâssées, des pierres d'escarboucle, et des pierres de diverses couleurs, des pierres précieuses de toutes sortes, et du marbre en abondance.
\VS{3}Et outre cela, parce que j'ai une grande affection pour la maison de mon Dieu, je donne pour la maison de mon Dieu, outre toutes les choses que j'ai préparées pour la maison du sanctuaire, l'or et l'argent que j'ai entre mes plus précieux joyaux :
\VS{4}Trois mille talents d'or, de l'or d'Ophir, et sept mille talents d'argent affiné, pour revêtir les murailles de la maison ;
\VS{5}afin qu'il y ait de l'or partout où il faut de l'or, et de l'argent partout où il faut de l'argent ; et pour tout l'ouvrage qui se fera par la main des ouvriers. Or qui est celui d'entre vous qui se disposera volontairement à offrir aujourd'hui libéralement à Yahweh ?
\VS{6}Alors les chefs des maisons paternelles, les chefs des tribus d'Israël, les chefs de milliers et de centaines et les intendants du roi offrirent volontairement.
\VS{7}Ils donnèrent pour le service de la maison de Dieu cinq mille talents et dix mille drachmes d'or, dix mille talents d'argent, dix-huit mille talents d'airain, et cent mille talents de fer.
\VS{8}Ils mirent aussi les pierres que chacun avait, pour le trésor de la maison de Yahweh, entre les mains de Jehiel, le Guerschonite.
\VS{9}Et le peuple offrait avec joie volontairement, car ils offraient de tout leur cœur leurs offrandes volontaires à Yahweh; et David en eut une très grande joie.
\TextTitle{Prières de David}
\VS{10}Puis David bénit Yahweh en présence de toute l'assemblée, et dit : Ô Yahweh, Dieu d'Israël, notre père ! Tu es béni de tout temps et à toujours.
\VS{11}Ô Yahweh ! C’est à toi qu'appartient la magnificence, la puissance, la gloire, l'éternité, et la majesté; car tout ce qui est aux cieux et sur la terre est à toi, ô Yahweh ! Le règne est à toi, et tu t'élèves en souverain au-dessus de toutes choses !
\VS{12}Les richesses et les honneurs viennent de toi, et tu as la domination sur toutes choses ; la force et la puissance sont dans ta main, et il est aussi du pouvoir de ta main d'agrandir et de fortifier toutes choses.
\VS{13}Maintenant donc, ô notre Dieu ! Nous te célébrons et nous louons ton Nom glorieux.
\VS{14}Mais qui suis-je, et qui est mon peuple, que nous ayons assez pour pouvoir t’offrir ces choses volontairement ? Car toutes choses viennent de toi, et les ayant reçues de ta main, nous te les présentons.
\VS{15}Et même nous sommes devant toi des étrangers et des habitants, comme ont été tous nos pères ; et nos jours sont comme l'ombre sur la terre, et il n'y a point d’espérance.
\VS{16}Yahweh, notre Dieu,  toute cette abondance que nous avons préparée pour bâtir une maison à ton saint Nom, est de ta main, et toutes ces choses sont à toi.
\VS{17}Et je sais, ô mon Dieu, que c'est toi qui sondes les cœurs, et que tu prends plaisir à la droiture. C'est pourquoi j'ai volontairement offert d'un cœur droit toutes ces choses, et j'ai vu maintenant avec joie que ton peuple, qui se trouve ici, t'a fait son offrande volontairement.
\VS{18}Ô Yahweh ! Dieu d'Abraham, d'Isaac et d'Israël, nos pères, conserve à toujours dans le cœur de ton peuple, ces dispositions et ces pensées, et affermis leurs cœurs en toi.
\VS{19}Donne aussi un cœur droit à Salomon, mon fils, afin qu'il garde tes commandements, tes préceptes et tes lois, et qu'il fasse tout ce qui est nécessaire et qu'il bâtisse le palais que j'ai préparé.
\TextTitle{Sacrifices en l’honneur de Yahweh ; Salomon oint roi\FTNTT{1 Ch. 23:1 ; 1 R. 2:12 ; 1 R. 1:32-37}}
\VS{20}Après cela, David dit à toute l'assemblée : Bénissez maintenant Yahweh, votre Dieu ! Et toute l'assemblée bénit Yahweh, le Dieu de leurs pères. Ils s'inclinèrent et se prosternèrent devant Yahweh et devant le roi.
\VS{21}Et le lendemain, ils offrirent des sacrifices à Yahweh, et des holocaustes ; à savoir mille veaux, mille moutons, et mille agneaux, avec leurs libations ; et des sacrifices en grand nombre pour tous ceux d'Israël.
\VS{22}Et ils mangèrent et burent ce jour-là devant Yahweh avec une grande joie ; et ils établirent roi pour la seconde fois Salomon, fils de David, et l'oignirent en l'honneur de Yahweh pour être leur conducteur, et Tsadok pour sacrificateur.
\VS{23}Salomon s'assit donc sur le trône de Yahweh pour être roi à la place de David, son père. Il prospéra, car tout Israël lui obéit.
\VS{24}Et tous les chefs et les héros, et même tous les fils du roi David consentirent d'être les sujets du roi Salomon.
\VS{25}Ainsi, Yahweh éleva souverainement Salomon, à la vue de tout Israël, et lui donna une majesté royale telle qu'aucun roi avant lui n'en avait eue en Israël.
\TextTitle{Fin du règne de David ; sa mort\FTNTT{2 S. 5:4-5 ; 1 R. 2:10-12 ; 1 Ch. 3:4}}
\VS{26}David donc, fils d'Isaï, régna sur tout Israël.
\VS{27}Et les jours qu'il régna sur Israël furent quarante ans ; il régna sept ans à Hébron et trente-trois ans à Jérusalem.
\VS{28}Puis il mourut dans une heureuse vieillesse, rassasié de jours, de richesses, et de gloire. Et Salomon, son fils, régna à sa place.
\VS{29}Les actions du roi David, tant les premières que les dernières, sont écrites dans le livre de Samuel le voyant, dans le livre de Nathan le prophète, et dans le livre de Gad le prophète,
\VS{30}avec tout son règne, ses exploits et ce qui se passa de son temps, tant sur Israël que sur tous les royaumes du territoire.
\PPE{}
\end{multicols}

%\clearpage\ShortTitle{2 Chroniques}\BookTitle{2 Chroniques}\BFont
\noindent\hrulefill
{\footnotesize
\textit{
\bigskip
{\centering{}
\\Auteur : Inconnu
\\(Heb. : Hayyamim dibre)
\\Signification : Actes des journées
\\Thème : La grandeur de Juda
\\Date de rédaction : 5\up{ème} siècle av. J.-C.\\}
}
%\bigskip
\textit{
\\Initialement, 1 et 2 Chroniques ne constituaient qu'un seul ouvrage. Ce livre raconte le règne de Salomon, la construction de la maison de Dieu et du palais. Il reprend ensuite l'histoire des royaumes d'Israël et de Juda, du schisme à la captivité babylonienne, mettant en exergue l'instabilité du peuple dont le cœur balançait entre Yahweh et les idoles.\bigskip
}
}
\par\nobreak\noindent\hrulefill
\begin{multicols}{2}
\Chap{1}
\TextTitle{Yahweh élève Salomon qui demande la sagesse\FTNTT{1 R. 2:12 ; 3:4-9 ; 1 Ch. 29:23-25}}
\VerseOne{}Or Salomon, fils de David, se fortifia dans son royaume ; Yahweh, son Dieu, fut avec lui, et l'éleva au plus haut.
\VS{2}Salomon parla à tout Israël, aux chefs de milliers et de centaines, aux juges et à tous les principaux de tout Israël, chefs des pères.
\VS{3}Salomon et toute l'assemblée avec lui allèrent au haut lieu qui était à Gabaon ; car là était la tente d'assignation de Dieu, que Moïse, serviteur de Yahweh, avait faite dans le désert.
\VS{4}Mais David avait fait monter l'arche de Dieu de Kirjath-Jearim au lieu qu'il avait préparé ; car il lui avait dressé une tente à Jérusalem.
\VS{5}L'autel d'airain que Betsaleel, fils d'Uri, fils de Hur, avait fait, était là devant le tabernacle de Yahweh. Et Salomon et l'assemblée y cherchèrent Yahweh\FTNT{Ex. 27:1-8 ; Ex. 36:1-2.}.
\VS{6}Salomon offrit là, devant Yahweh, mille holocaustes, sur l'autel d'airain qui était devant la tente d'assignation.
\VS{7}En cette nuit-là, Dieu apparut à Salomon, et lui dit : Demande ce que tu veux que je te donne.
\VS{8}Et Salomon répondit à Dieu : Tu as usé d'une grande bienveillance envers David, mon père, et tu m'as établi roi à sa place.
\VS{9}Maintenant, ô Yahweh Dieu ! Que ta parole à David, mon père, se confirme ; car tu m'as établi roi sur un peuple nombreux comme la poussière de la terre.
\VS{10}Donne-moi donc maintenant de la sagesse et de l'intelligence, afin que je sache me conduire devant ce peuple ; car qui pourrait juger ton peuple, ce peuple si grand ?
\TextTitle{Yahweh agrée la prière de Salomon et l'exauce\FTNTT{1 R. 3:10-28}}
\VS{11}Et Dieu dit à Salomon : Puisque c'est là ce qui est dans ton cœur, et que tu n'as demandé ni des richesses, ni des biens, ni de la gloire, ni la mort de ceux qui te haïssent, ni même des jours nombreux, mais que tu as demandé pour toi de la sagesse et de l'intelligence, afin de pouvoir juger mon peuple, sur lequel je t'ai établi roi,
\VS{12}la sagesse et l'intelligence te sont données. Je te donnerai aussi des richesses, des biens et de la gloire, comme n'en ont pas eu les rois qui ont été avant toi, et comme il n'en aura aucun après toi.
\VS{13}Puis Salomon s'en retourna à Jérusalem, du haut lieu qui était à Gabaon devant la tente d'assignation ; et il régna sur Israël.
\VS{14}Salomon rassembla des chars et des cavaliers ; il avait quatorze cents chars et douze mille cavaliers ; et il les plaça dans les villes où il tenait ses chars, et auprès du roi, à Jérusalem.
\VS{15}Et le roi fit que l'argent et l'or étaient aussi communs à Jérusalem que les pierres, et les cèdres que les sycomores de la plaine.
\VS{16}Le lieu d'où étaient issus les chevaux de Salomon était l'Egypte ; une caravane de marchands du roi allait les prendre par troupe à un prix convenu.
\VS{17}On faisait monter et sortir d'Egypte un char pour six cents sicles d'argent, et un cheval pour cent cinquante. On en amenait de même par eux pour tous les rois des Héthiens, et pour les rois de Syrie.
\Chap{2}
\TextTitle{La prière de Salomon exaucée\FTNTT{1 R. 5:1-18 ; 7:13,14}}
\VerseOne{}Or Salomon ordonna de bâtir une maison au Nom de Yahweh, ainsi qu'une maison royale.
\VS{2}Et il fit un dénombrement de soixante et dix mille hommes qui portaient les fardeaux, et de quatre vingt mille qui coupaient le bois sur la montagne, et de trois mille six cents qui étaient commis sur eux.
\VS{3}Puis Salomon envoya vers Huram, roi de Tyr, pour lui dire : Fais pour moi comme tu as fait pour David, mon père, à qui tu as envoyé des cèdres, pour se bâtir une maison afin d'y habiter.
\VS{4}Voici, je vais bâtir une maison au Nom de Yahweh, mon Dieu, pour la lui consacrer, pour faire brûler devant lui le parfum des aromates, pour présenter continuellement devant lui les pains de proposition, et pour offrir les holocaustes du matin et du soir, des sabbats, des nouvelles lunes, et des fêtes de Yahweh, notre Dieu, ce qui est perpétuel en Israël.
\VS{5}La maison que je vais bâtir sera grande ; car notre Dieu est plus grand que tous les dieux.
\VS{6}Mais qui aurait le pouvoir de lui bâtir une maison, puisque les cieux et les cieux des cieux ne sauraient le contenir ? Et qui suis-je pour lui bâtir une maison, si ce n'est pour faire brûler des parfums devant sa face ?
\VS{7}Maintenant, envoie-moi un homme habile pour travailler l'or, l'argent, l'airain et le fer, en écarlate, en cramoisi et en pourpre, sachant faire des sculptures, pour travailler avec les hommes habiles que j'ai avec moi en Juda et à Jérusalem, et que David, mon père, a préparés.
\VS{8}Envoie-moi aussi du Liban du bois de cèdre, de cyprès et de santal ; car je sais que tes serviteurs savent couper les bois du Liban. Voici, mes serviteurs seront avec les tiens.
\VS{9}Qu'on me prépare du bois en grande quantité ; car la maison que je vais bâtir sera grande et magnifique.
\VS{10}Et je donnerai à tes serviteurs qui couperont, qui abattront les bois, vingt mille cors de froment foulé, vingt mille cors d'orge, vingt mille baths de vin, et vingt mille baths d'huile.
\VS{11}Huram, roi de Tyr, répondit dans un écrit qu'il envoya à Salomon : C'est parce que Yahweh aime son peuple qu'il t'a établi roi sur eux.
\VS{12}Et Huram dit : Béni soit Yahweh, le Dieu d'Israël, qui a fait les cieux et la terre, de ce qu'il a donné au roi David un fils sage, prudent et intelligent, qui va bâtir une maison à Yahweh, et une maison royale !
\VS{13}Je t'envoie donc un homme habile et intelligent, Huram-Abi,
\VS{14}fils d'une femme d'entre les filles de Dan, et d'un père tyrien. Il sait travailler l'or, l'argent, l'airain et le fer, les pierres et le bois, en écarlate, en pourpre, en fin lin et en cramoisi ; il sait faire toutes sortes de sculptures et imaginer toutes sortes d'objets d'art qu'on lui donne à faire. Il travaillera avec tes hommes habiles et avec les hommes habiles de mon seigneur David, ton père.
\VS{15}Et maintenant, que mon seigneur envoie à ses serviteurs le froment, l'orge, l'huile et le vin comme il l'a dit.
\VS{16}Et nous couperons des bois du Liban autant que tu en auras besoin, et nous te les amènerons en radeaux, par la mer, jusqu'à Japho, et tu les feras monter à Jérusalem.
\VS{17}Alors Salomon compta tous les hommes étrangers qui étaient au pays d'Israël, d'après le dénombrement que David, son père, en avait fait. On en trouva cent cinquante-trois mille six cents.
\VS{18}Et il en établit soixante-dix mille qui portaient des fardeaux, quatre-vingt mille qui taillaient les pierres dans la montagne, et trois mille six cents surveillants pour faire travailler le peuple.
\Chap{3}
\TextTitle{Salomon commence la construction du temple\FTNTT{1 R. 6:1}}
\VerseOne{}Salomon commença donc à bâtir la maison de Yahweh à Jérusalem, sur la montagne de Morija, qui avait été indiquée à David, son père, au lieu même que David avait préparé dans l'aire d'Ornan, le Jébusien.
\VS{2}Il commença à bâtir, le second jour du second mois, la quatrième année de son règne.
\TextTitle{Les matériaux du temple et les dimensions\FTNTT{1 R. 6:2-38 ; 7:13-22}}
\VS{3}Or voici les fondements fixés par Salomon pour bâtir la maison de Dieu : La longueur, en coudées de l'ancienne mesure, était de soixante coudées, et la largeur de vingt coudées.
\VS{4}Le portique qui était sur le devant, et dont la longueur répondait à la largeur de la maison, avait vingt coudées, et cent vingt de hauteur. Il le revêtit intérieurement d'or pur.
\VS{5}Et il recouvrit la grande maison de bois de cyprès ; il la revêtit d'or fin, et y fit mettre des palmes et des chaînettes.
\VS{6}Il revêtit la maison de pierres précieuses, pour l'ornement ; et l'or était de l'or de Parvaïm.
\VS{7}Il revêtit d'or la maison, les poutres, les seuils, les parois et les portes ; et il fit sculpter des chérubins sur les parois.
\VS{8}Il fit le Saint des saints, dont la longueur était de vingt coudées, selon la largeur de la maison, et la largeur de vingt coudées ; et il le couvrit d'or fin, pour une valeur de six cents talents.
\VS{9}Et le poids des clous montait à cinquante sicles d'or. Il revêtit aussi d'or les chambres hautes.
\VS{10}Il fit dans le Saint des saints deux chérubins sculptés, et on les couvrit d'or ;
\VS{11}La longueur des ailes des chérubins était de vingt coudées. L'aile du premier, longue de cinq coudées, touchait la paroi de la maison, et l'autre aile, longue de cinq coudées, touchait une aile de l'autre chérubin.
\VS{12}Et une aile de l'autre chérubin, longue de cinq coudées, touchait la paroi de la maison ; et l'autre aile longue de cinq coudées, joignait l'aile de l'autre chérubin.
\VS{13}Les ailes étendues de ces chérubins faisaient vingt coudées. Ils se tenaient debout sur leurs pieds, leurs faces tournées vers la maison.
\VS{14}Il fit le voile de pourpre, d'écarlate, de cramoisi et de fin lin: Et il y représenta par dessus des chérubins.
\VS{15}Devant la maison, il fit deux colonnes de trente-cinq coudées de hauteur, et le chapiteau sur leur sommet était de cinq coudées.
\VS{16}Il fit des chaînes dans le sanctuaire ; et il en mit sur le sommet des colonnes ; et il fit cent grenades qu'il mit aux chaînes.
\VS{17}Il dressa les colonnes sur le devant du temple, l'une à droite, et l'autre à gauche ; il appela celle de droite Jakin, et celle de gauche Boaz.
\Chap{4}
\TextTitle{L'autel d'airain, la mer de fonte et les ustensiles du temple\FTNTT{1 R. 7:23-50}}
\VerseOne{}Il fit aussi un autel d'airain\FTNT{Voir l'annexe « Le temple de Salomon - extérieur »} long de vingt coudées, large de vingt coudées, et haut de dix coudées.
\VS{2}Il fit la mer de fonte de dix coudées d'un bord à l'autre, ronde tout autour, et haute de cinq coudées, et une circonférence que mesurait un cordon de trente coudées.
\VS{3}Des figures de bœufs l'entouraient en dessous, dix par coudée, faisant tout le tour de la mer ; il y avait deux rangées de bœufs fondus avec elle en une seule pièce.
\VS{4}Elle était posée sur douze bœufs, dont trois tournés vers le nord, trois tournés vers l'occident, trois tournés vers le sud, et trois tournés vers l'orient. La mer était sur eux, et toute la partie postérieure de leur corps était en dedans.
\VS{5}Son épaisseur était d'une paume ; et son bord était comme le bord d'une coupe en fleur de lis. Elle avait une contenance de trois mille baths\FTNT{Ex 25 ; Ex 27.}.
\VS{6}Il fit aussi dix cuves, et en mit cinq à droite et cinq à gauche, pour servir à la purification. On y lavait ce qui appartenait aux holocaustes, et la mer servait aux sacrificateurs pour s'y laver.
\VS{7}Il fit dix chandeliers d'or, d'après l'ordonnance, et les mit dans le temple, cinq à droite et cinq à gauche.
\VS{8}Il fit aussi dix tables, et il les mit dans le temple, cinq à droite et cinq à gauche. Il fit cent coupes d'or.
\VS{9}Il fit encore le parvis des sacrificateurs, le grand parvis et des portes pour ce parvis, et couvrit d'airain ces portes.
\VS{10}Il mit la mer du côté droit, vers l'orient, face au sud-est.
\VS{11}Et Huram fit les cuves, les pelles et les bassins. Huram acheva de faire l'ouvrage qu'il faisait pour le roi Salomon dans la maison de Dieu :
\VS{12}Deux colonnes, les bourrelets et les deux chapiteaux sur le sommet des colonnes ; les deux maillages pour couvrir les deux bourrelets des chapiteaux sur le sommet des colonnes ;
\VS{13}et les quatre cents grenades pour les deux maillages, deux rangs de grenades à chaque maille, pour couvrir les deux bourrelets des chapiteaux sur le sommet des colonnes.
\VS{14}Il fit aussi les bases, et il fit les cuves sur les bases ;
\VS{15}la mer et les douze bœufs sous elle ;
\VS{16}les pots, les pelles et les fourchettes et tous leurs ustensiles ; Huram-Abi les fit au roi Salomon, pour la maison de Yahweh, en airain poli.
\VS{17}Le roi les fit fondre dans la plaine du Jourdain, dans une terre grasse, entre Succoth et Tseréda.
\VS{18}Et Salomon fit tous ces ustensiles en si grand nombre qu'on ne rechercha point le poids de l'airain.
\VS{19}Salomon fit encore tous les ustensiles\FTNT{Voir l'annexe « Le temple de Salomon - intérieur »} qui étaient dans la maison de Yahweh : L'autel d'or, et les tables sur lesquelles on mettait le pain de proposition ;
\VS{20}les chandeliers et leurs lampes d'or fin, qu'on devait allumer devant le sanctuaire, selon l'ordonnance ;
\VS{21}les fleurs, les lampes, et les mouchettes d'or, d'un or parfaitement pur ;
\VS{22}et les mouchettes, les bassins, les tasses et les encensoirs d'or fin. Quant à l'entrée de la maison, les portes intérieures conduisant dans le Saint des saints, et les portes de la maison pour entrer au temple étaient d'or.
\Chap{5}
\TextTitle{L'arche dans le sanctuaire, Yahweh manifeste sa gloire\FTNTT{1 R. 7:51-8:11}}
\VerseOne{}Ainsi fut achevé tout l'ouvrage que Salomon fit pour la maison de Yahweh. Puis Salomon fit apporter ce que David, son père, avait consacré : L'argent, l'or et tous les ustensiles ; et il les mit dans les trésors de la maison de Dieu.
\VS{2}Alors Salomon assembla à Jérusalem les anciens d'Israël, et tous les chefs des tribus, les chefs des pères des fils d'Israël, pour transporter de la ville de David, qui est Sion, l'arche de l'alliance de Yahweh.
\VS{3}Et tous les hommes d'Israël s'assemblèrent auprès du roi pour la fête ; c'était le septième mois.
\VS{4}Tous les anciens d'Israël vinrent, et les Lévites portèrent l'arche.
\VS{5}Ils transportèrent l'arche, la tente d'assignation, et tous les ustensiles sacrés qui étaient dans la tente ; les sacrificateurs et les Lévites les emportèrent.
\VS{6}Or le roi Salomon et toute l'assemblée d'Israël réunie auprès de lui étaient devant l'arche, sacrifiant du menu et du gros bétail en si grand nombre qu'on ne pouvait ni dénombrer ni compter.
\VS{7}Les sacrificateurs portèrent l'arche de l'alliance de Yahweh à sa place, dans le sanctuaire de la maison, dans le Saint des saints, sous les ailes des chérubins.
\VS{8}Les chérubins étendaient les ailes sur l'endroit où devait être l'arche, et les chérubins couvraient l'arche et ses barres par-dessus.
\VS{9}Les barres avaient une longueur telle que leurs extrémités se voyaient en avant de l'arche, devant le sanctuaire ; mais elles ne se voyaient point du dehors. Et l'arche a été là jusqu'à ce jour.
\VS{10}Il n'y avait dans l'arche que les deux tables que Moïse y avait mises en Horeb, quand Yahweh traita alliance avec les enfants d'Israël à leur sortie d'Egypte.
\VS{11}Or il arriva que comme les sacrificateurs sortaient du lieu saint (car tous les sacrificateurs présents s'étaient sanctifiés, sans observer l'ordre des classes),
\VS{12}etque tous les Lévites qui étaient chantres, Asaph, Héman, Jeduthun, leurs fils et leurs frères, vêtus de fin lin, avec des cymbales, des luths et des harpes, se tenaient à l'orient de l'autel ; et il y avait avec eux cent vingt sacrificateurs sonnant des trompettes.
\VS{13}Il arriva, dis-je, que comme un seul homme, ceux qui sonnaient des trompettes et ceux qui chantaient firent entendre leur voix d'un même accord, pour célébrer et pour louer Yahweh, et firent retentir le son des trompettes, des cymbales et d'autres instruments de musique, et ils célébrèrent Yahweh, en disant : Car il est bon, car sa miséricorde demeure à toujours\FTNT{Jé. 33:11 ; Ps. 118:29 ;  Ps. 136} ! Il arriva que la maison de Yahweh fut remplie d'une nuée.
\VS{14}Les sacrificateurs ne purent s'y tenir pour faire le service, à cause de la nuée ; car la gloire de Yahweh remplissait la maison de Dieu.
\Chap{6}
\TextTitle{Salomon s'adresse à l'assemblée d'Israël\FTNTT{1 R. 8:12-21}}
\VerseOne{}Alors Salomon dit : Yahweh a dit qu'il habiterait dans l'obscurité\FTNT{Nous avons ici une prophétie concernant la venue du Messie. Dieu, qui est lumière, a accepté d'habiter dans les ténèbres afin de nous sauver (Mt. 4:16 ; Jn. 1:5).}.
\VS{2}Et moi, j'ai bâti une maison qui sera ta demeure, et un domicile afin que tu y résides à toujours !
\VS{3}Puis le roi tourna son visage, et bénit toute l'assemblée d'Israël ; et toute l'assemblée d'Israël était debout.
\VS{4}Et il dit : Béni soit Yahweh, le Dieu d'Israël, qui de sa bouche a parlé à David, mon père, et qui par sa main puissante accomplit ce qu'il avait déclaré en disant :
\VS{5}Depuis le jour où j'ai fait sortir mon peuple du pays d'Egypte, je n'ai point choisi de ville entre toutes les tribus d'Israël pour y bâtir une maison afin que mon Nom y réside, et je n'ai point choisi d'homme pour être chef de mon peuple d'Israël.
\VS{6}Mais j'ai choisi Jérusalem pour que mon Nom y réside, et j'ai choisi David pour qu'il règne sur mon peuple d'Israël.
\VS{7}Or David, mon père, avait à cœur de bâtir une maison au Nom de Yahweh, le Dieu d'Israël.
\VS{8}Mais Yahweh parla à David, mon père : Puisque tu as eu à cœur de bâtir une maison à mon Nom, tu as bien fait d'avoir eu cette intention.
\VS{9}Seulement, ce n'est pas toi qui bâtiras cette maison ; mais ce sera ton fils, qui sortira de tes entrailles, qui bâtira cette maison à mon Nom.
\VS{10}Yahweh a accompli la parole qu'il avait déclarée ; j'ai succédé à David, mon père, et je me suis assis sur le trône d'Israël, comme Yahweh l'avait dit, et j'ai bâti cette maison au Nom de Yahweh, le Dieu d'Israël.
\VS{11}J'y ai mis l'arche où est l'alliance de Yahweh, qu'il traita avec les enfants d'Israël.
\TextTitle{Prière de Salomon\FTNTT{1 R. 8:22-61}}
\VS{12}Puis il se plaça devant l'autel de Yahweh, en face de toute l'assemblée d'Israël, et il étendit ses mains.
\VS{13}Car Salomon avait fait une tribune d'airain, et il l'avait mise au milieu du grand parvis ; elle était longue de cinq coudées, large de cinq coudées, et haute de trois coudées. Il s'y plaça, se mit à genoux en face de toute l'assemblée d'Israël, et étendant ses mains vers les cieux, il dit :
\VS{14}Ô Yahweh, Dieu d'Israël ! Il n'y a ni dans les cieux ni sur la terre de Dieu semblable à toi, qui gardes l'alliance et la miséricorde envers tes serviteurs qui marchent de tout leur cœur devant ta face.
\VS{15}Toi qui as tenu parole à ton serviteur David, mon père. Ce que tu lui avais promis, et ce que tu as déclaré de ta bouche, tu l'as accompli de ta main puissante, comme il paraît aujourd'hui.
\VS{16}Maintenant, ô Yahweh, Dieu d'Israël ! Tiens la parole que tu as faite à ton serviteur David, mon père, en disant : Tu ne manqueras jamais devant moi d'un successeur assis sur le trône d'Israël, pourvu que tes fils prennent garde à leur voie pour marcher dans ma loi, comme tu as marché devant ma face.
\VS{17}Et maintenant, ô Yahweh, Dieu d'Israël ! Que ta parole, que tu as déclarée à David, ton serviteur, soit confirmée !
\VS{18}Mais Dieu habiterait-il véritablement sur la terre avec les hommes ? Voici, les cieux, même les cieux des cieux, ne peuvent te contenir, combien moins cette maison que j'ai bâtie !
\VS{19}Toutefois, ô Yahweh, mon Dieu, aie égard à la prière de ton serviteur et à sa supplication, pour écouter le cri et la prière que ton serviteur t'adresse.
\VS{20}Que tes yeux soient ouverts jour et nuit sur cette maison, sur le lieu où tu as promis de mettre ton Nom ! Ecoute la prière que ton serviteur te fait en ce lieu.
\VS{21}Exauce les supplications de ton serviteur et de ton peuple d'Israël, quand ils prieront en ce lieu. Exauce des cieux, du lieu de ta demeure ; exauce et pardonne !
\VS{22}Si quelqu'un pèche contre son prochain, et qu'on lui impose un serment pour le faire jurer, et qu'il vient prêter serment devant ton autel, dans cette maison ;
\VS{23}écoute-le des cieux, agis et juge tes serviteurs, en donnant au méchant son salaire, et fais retomber sa conduite sur sa tête, en justifiant le juste, et lui rendant selon sa justice.
\VS{24}Quand ton peuple d'Israël sera battu par l'ennemi, pour avoir péché contre toi ; s'ils retournent à toi, s'ils donnent gloire à ton Nom, s'ils t'adressent dans cette maison des prières et des supplications ;
\VS{25}toi, exauce-les des cieux, et pardonne le péché de ton peuple d'Israël, et ramène-les dans la terre que tu leur as donnée à eux et à leurs pères.
\VS{26}Quand les cieux seront fermés, et qu'il n'y aura point de pluie, parce qu'ils auront péché contre toi ; s'ils prient en ce lieu, s'ils donnent gloire à ton Nom, et s'ils se détournent de leurs péchés, parce que tu les auras affligés ;
\VS{27}toi, exauce-les des cieux, et pardonne le péché de tes serviteurs et de ton peuple d'Israël, après que tu leur auras enseigné le bon chemin, par lequel ils doivent marcher ; et envoie de la pluie sur la terre que tu as donnée en héritage à ton peuple.
\VS{28}Quand il y aura dans le pays la famine ou la peste, quand il y aura la rouille, la nielle, les sauterelles d'une espèce ou d'une autre, quand les ennemis les assiégeront dans leur pays, dans leurs portes, ou qu'il y aura un fléau, une maladie quelconque ;
\VS{29}si un homme, si tout ton peuple d'Israël fait entendre des prières et des supplications, et que chacun reconnaît sa plaie et sa douleur, et étend ses mains vers cette maison ;
\VS{30}exauce-le des cieux, du lieu de ta demeure, et pardonne. Rends à chacun selon toutes ses voies, toi qui connais leur cœur ; car seul tu connais le cœur des fils des hommes ;
\VS{31}afin qu'ils te craignent, pour marcher dans tes voies, tout le temps qu'ils vivront sur la terre que tu as donnée à nos pères.
\VS{32}Et l'étranger, qui ne sera pas de ton peuple d'Israël, mais qui viendra d'un pays éloigné, à cause de ton grand Nom, de ta main puissante, et de ton bras étendu ; quand il viendra prier dans cette maison,
\VS{33}exauce-le des cieux, du lieu de ta demeure, et accorde tout ce que cet étranger réclamera de toi ; afin que tous les peuples de la terre connaissent ton Nom pour te craindre comme ton peuple d'Israël, et sachent que ton Nom est invoqué sur cette maison que j'ai bâtie.
\VS{34}Quand ton peuple sortira en guerre contre ses ennemis, par la voie par laquelle tu l'auras envoyé ; s'ils te prient, en regardant vers cette ville que tu as choisie, et vers cette maison que j'ai bâtie à ton Nom,
\VS{35}exauce des cieux leur prière et leur supplication, et fais-leur droit.
\VS{36}Quand ils pécheront contre toi, car il n'y a point d'homme qui ne pèche, et qu'irrité contre eux, tu les auras livrés à leurs ennemis, et que ceux qui les auront pris les auront emmenés captifs en quelque pays, soit éloigné soit proche ;
\VS{37}si dans le pays où ils seront captifs, ils rentrent en eux-mêmes et s'ils se repentent, s'ils t'adressent des supplications dans le pays de leur captivité, en disant : Nous avons péché, nous avons commis l'iniquité, nous avons agi méchamment !
\VS{38}S'ils retournent à toi de tout leur cœur et de toute leur âme, dans le pays de leur captivité où ils ont été emmenés captifs, et s'ils t'adressent des prières, les regards tournés vers leur pays que tu as donné à leurs pères, vers cette ville que tu as choisie, et vers cette maison que j'ai bâtie à ton Nom ;
\VS{39}exauce des cieux, du lieu de ta demeure, leurs prières et leurs supplications, et fais-leur droit ; pardonne à ton peuple qui aura péché contre toi !
\VS{40}Maintenant, ô mon Dieu, que tes yeux soient ouverts et que tes oreilles soient attentives à la prière qu'on te fera en ce lieu !
\VS{41}Et maintenant, Yahweh Dieu ! Lève-toi, viens au lieu de ton repos, toi et l'arche de ta puissance. Yahweh Dieu, que tes sacrificateurs soient revêtus du salut, et que tes bien-aimés se réjouissent du bien que tu leur fais !
\VS{42}Yahweh Dieu, ne repousse la face pas ton oint ; souviens-toi des grâces accordées à David, ton serviteur.
\Chap{7}
\TextTitle{Yahweh répond par le feu : Sa gloire remplit la maison}
\VerseOne{}Lorsque Salomon eut achevé de prier, le feu descendit du ciel et consuma l'holocauste et les sacrifices\FTNT{Lé 9:24 ; 1 R 18:38.} ; et la gloire de Yahweh remplit la maison.
\VS{2}Les sacrificateurs ne pouvaient entrer dans la maison de Yahweh, parce que la gloire de Yahweh avait rempli la maison de Yahweh.
\VS{3}Tous les enfants d'Israël virent descendre le feu et la gloire de Yahweh sur la maison ; et ils se courbèrent, le visage contre terre, sur le pavé, se prosternèrent et louèrent Yahweh, en disant : Car il est bon, car sa miséricorde demeure éternellement !
\TextTitle{Salomon et le peuple offrent des sacrifices à Yahweh\FTNTT{1 R. 8:62-66}}
\VS{4}Or le roi et tout le peuple offraient des sacrifices devant Yahweh.
\VS{5}Le roi Salomon offrit un sacrifice de vingt-deux mille bœufs, et cent vingt mille brebis. Ainsi, le roi et tout le peuple firent la dédicace de la maison de Dieu.
\VS{6}Les sacrificateurs se tenaient à leurs fonctions, ainsi que les Lévites, avec les instruments de musique de Yahweh, que le roi David avait faits pour louer Yahweh en disant : Car sa miséricorde demeure éternellement ; ayant les Psaumes de David entre leurs mains. Et les sacrificateurs sonnaient des trompettes vis-à-vis d'eux, et tout Israël se tenait debout.
\VS{7}Salomon consacra le milieu du parvis, qui est devant la maison de Yahweh ; car il offrit là les holocaustes et les graisses des sacrifices d'offrande de paix\FTNT{Voir commentaire en Lé. 3:1.}, parce que l'autel d'airain que Salomon avait fait ne pouvait contenir les holocaustes, les offrandes et les graisses.
\VS{8}Ainsi Salomon célébra, en ce temps-là, la fête pendant sept jours, avec tout Israël. Il y avait une grande multitude, venue depuis l'entrée d'Hamath jusqu'au torrent d'Egypte.
\VS{9}Le huitième jour, ils firent une assemblée solennelle ; car ils firent la dédicace de l'autel pendant sept jours, et la fête pendant sept jours.
\VS{10}Le vingt-troisième jour du septième mois, il laissa aller le peuple dans ses tentes, se réjouissant et ayant le cœur plein de joie, à cause du bien que Yahweh avait fait à David, à Salomon, et à Israël, son peuple.
\TextTitle{Yahweh apparaît à Salomon\FTNTT{1 R. 9:1-9}}
\VS{11}Salomon acheva donc la maison de Yahweh et la maison du roi ; et Salomon réussit dans tout ce qui lui vint à cœur de faire dans la maison de Yahweh et dans sa maison.
\VS{12}Yahweh apparut à Salomon pendant la nuit, et lui dit : J'exauce ta prière, et je choisis ce lieu comme une maison de sacrifices.
\VS{13}Quand je fermerai les cieux, et qu'il n'y aura point de pluie, et quand j'ordonnerai aux sauterelles de consumer le pays, et quand j'enverrai la peste parmi mon peuple ;
\VS{14}si mon peuple, sur lequel mon Nom est invoqué, s'humilie, prie, et cherche ma face, et s'il se détourne de ses mauvaises voies, alors je l'exaucerai des cieux, je pardonnerai ses péchés, et je guérirai son pays.
\VS{15}Mes yeux seront désormais ouverts, et mes oreilles seront attentives à la prière faite en ce lieu.
\VS{16}Maintenant je choisis et je sanctifie cette maison, afin que mon Nom y soit à toujours ; mes yeux et mon cœur seront toujours là.
\VS{17}Et toi, si tu marches devant moi comme David, ton père, a marché, faisant tout ce que je t'ai ordonné, et si tu gardes mes lois et mes ordonnances,
\VS{18}j'affermirai le trône de ton royaume, comme je l'ai déclaré à David, ton père, en disant : Il ne te manquera point de successeur qui règne en Israël.
\VS{19}Mais si vous vous détournez, et si vous abandonnez mes lois et mes commandements que je vous ai prescrits, et si vous allez servir d'autres dieux et vous prosterner devant eux,
\VS{20}je vous arracherai de mon pays que je vous ai donné, je rejetterai loin de moi cette maison que j'ai consacrée à mon Nom, et j'en ferai un sujet de sarcasmes et de moqueries parmi tous les peuples.
\VS{21}Et quiconque passera près de cette maison qui aura été élevée, sera dans l'étonnement et dira : Pourquoi Yahweh a-t-il ainsi traité ce pays et cette maison?
\VS{22}Et on répondra : Parce qu'ils ont abandonné Yahweh, le Dieu de leurs pères, qui les a fait sortir du pays d'Egypte, et qu'ils se sont attachés à d'autres dieux, et qu'ils se sont prosternés devant eux, et les ont servis ; à cause de cela, il a fait venir sur eux tous ces maux.
\Chap{8}
\TextTitle{Les réalisations de Salomon\FTNTT{1 R. 9:15-28 ; 10:26-29}}
\VerseOne{}Au bout de vingt ans, pendant lesquels Salomon bâtit la maison de Yahweh et sa propre maison,
\VS{2}il bâtit les villes que Huram lui avait données et y fit habiter les enfants d'Israël.
\VS{3}Puis Salomon marcha contre Hamath de Tsoba, et la conquit.
\VS{4}Il bâtit Thadmor au désert, et toutes les villes servant de magasins qu'il bâtit dans le pays de Hamath.
\VS{5}Il bâtit Beth-Horon la haute, et Beth-Horon la basse, villes fortes de murailles, de portes et de barres ;
\VS{6}Baalath, et toutes les villes servant de magasins qu'avait Salomon, toutes les villes pour les chars, les villes pour la cavalerie, et tout ce que Salomon prit plaisir à bâtir à Jérusalem, au Liban, et dans tout le pays de sa domination.
\VS{7}Tout le peuple qui était resté des Héthiens, des Amoréens, des Phéréziens, des Héviens et des Jébusiens, qui n'étaient point d'Israël ;
\VS{8}leurs descendants, qui étaient restés après eux dans le pays, et que les enfants d'Israël n'avaient pas détruits, Salomon les leva comme des gens de corvée jusqu'à ce jour.
\VS{9}Salomon n'employa comme esclave pour ses travaux aucun des fils d'Israël ; car ils étaient des hommes de guerre, les chefs de ses officiers, les chefs de ses chars et de ses hommes d'armes.
\VS{10}Voici le nombre des chefs de ceux qui étaient préposés aux travaux du roi Salomon : Ils étaient deux cent cinquante, ayant autorité sur le peuple.
\VS{11}Salomon fit monter la fille de Pharaon de la cité de David dans la maison qu'il lui avait bâtie ; car il dit : Ma femme n'habitera point dans la maison de David, roi d'Israël, parce que les lieux où l'arche de Yahweh est entrée sont saints.
\VS{12}Alors Salomon offrit des holocaustes à Yahweh, sur l'autel de Yahweh qu'il avait bâti devant le portique.
\VS{13}Il offrait chaque jour ce qui était prescrit par Moïse pour les sabbats, pour les nouvelles lunes, et pour les fêtes, trois fois l'année, à la fête des pains sans levain, à la fête des semaines, et à la fête des tabernacles\FTNT{Ex. 14:17 ; Lé. 23:1-44.}.
\VS{14}Il établit, selon l'ordonnance de David, son père, les classes des sacrificateurs selon leur fonction, et les Lévites selon leurs charges, pour célébrer Yahweh et pour faire, jour par jour, le service en présence des sacrificateurs ; et les portiers, selon leurs classes, à chaque porte ; car tel était le commandement de David, homme de Dieu.
\VS{15}Et on ne s'écarta pas du commandement du roi à l'égard des Sacrificateurs et des Lévites, en aucune chose, ni à l'égard les trésors.
\VS{16}Ainsi fut préparé tout l'ouvrage de Salomon, jusqu'au jour de la fondation de la maison de Yahweh et jusqu'à ce qu'elle fut terminée. La maison de Yahweh fut donc achevée.
\VS{17}Alors Salomon alla à Etsjon-Guéber et à Eloth, sur le rivage de la mer, dans le pays d'Edom.
\VS{18}Et Huram lui envoya, sous la conduite de ses serviteurs, des navires et des serviteurs connaissant la mer. Ils allèrent avec les serviteurs de Salomon à Ophir, et ils y prirent quatre cent cinquante talents d'or, qu'ils apportèrent au roi Salomon.
\Chap{9}
\TextTitle{La reine de Séba chez Salomon\FTNTT{1 R. 10:1-13}}
\VerseOne{}Or la reine de Séba, ayant appris la renommée de Salomon, vint à Jérusalem pour éprouver Salomon par des énigmes. Elle avait une suite très nombreuse, et des chameaux portant des aromates, de l'or en grande quantité et des pierres précieuses. Elle vint auprès de Salomon, et elle lui parla de tout ce qu'elle avait dans le cœur.
\VS{2}Salomon lui expliqua tout ce qu'elle lui proposa ; il n'y eut rien que Salomon n'entendît et qu'il ne sût lui expliquer.
\VS{3}Alors, la reine de Séba vit toute la sagesse de Salomon, et la maison qu'il avait bâtie,
\VS{4}les mets de sa table, la demeure de ses serviteurs, l'ordre de service et les vêtements de ceux qui le servaient, ses échansons et leurs vêtements, et les marches par où l'on montait à la maison de Yahweh, et elle fut toute ravie hors d'elle-même.
\VS{5}Elle parla ainsi au roi : Ce que j'ai entendu dire dans mon pays de tes actions et de ta sagesse était donc vrai !
\VS{6}Je ne croyais pas ce qu'on en disait avant d'être venue et que mes yeux ne l'aient vu ; et voici, on ne m'avait pas rapporté la moitié de la grandeur de ta sagesse ; tu surpasses la rumeur que j'avais entendue.
\VS{7}Heureux tes gens ! Heureux tes serviteurs qui se tiennent continuellement devant toi, et qui entendent ta sagesse !
\VS{8}Béni soit Yahweh, ton Dieu, qui a pris plaisir en toi pour te placer sur son trône comme roi pour Yahweh, ton Dieu ! C'est parce que ton Dieu aime Israël et veut le faire subsister à jamais, qu'il t'a établi roi sur eux pour faire droit et justice.
\VS{9}Puis elle donna au roi cent vingt talents d'or, une très grande quantité d'aromates, et des pierres précieuses ; et il n'y eut plus d'aromates tels que ceux que la reine de Séba donna au roi Salomon.
\VS{10}Les serviteurs de Huram et les serviteurs de Salomon, qui amenèrent de l'or d'Ophir, amenèrent aussi du bois de santal et des pierres précieuses.
\VS{11}Le roi fit de ce bois de santal les chemins qui allaient à la maison de Yahweh et à la maison du roi, et des harpes et des luths pour les chantres. On n'en avait point vu auparavant de semblable dans le pays de Juda.
\VS{12}Le roi Salomon donna à la reine de Séba tout ce qu'elle désira, ce qu'elle demanda, plus qu'elle n'avait apporté au roi ; et elle s'en retourna, revint dans son pays, elle et ses serviteurs.
\TextTitle{Les richesses de Salomon\FTNTT{cp. 1 R. 4:1-34}}
\VS{13}Le poids de l'or qui arrivait à Salomon chaque année était de six cent soixante-six talents d'or,
\VS{14}outre ce qu'il retirait des négociants et des marchands qui en apportaient, et de tous les rois d'Arabie et des gouverneurs de ces pays-là, qui apportaient de l'or et de l'argent à Salomon.
\VS{15}Le roi Salomon fit deux cents grands boucliers d'or battu, employant six cents sicles d'or battu pour chaque bouclier ;
\VS{16}et trois cents autres boucliers plus petits d'or battu, employant trois cents sicles d'or pour chaque bouclier ; et le roi les mit dans la maison de la forêt du Liban.
\VS{17}Le roi fit aussi un grand trône d'ivoire, qu'il couvrit d'or pur.
\VS{18}Ce trône avait six marches et un marchepied d'or qui était accolé au trône ; et il avait des accoudoirs de l'un et de l'autre côté du siège ; et deux lions se tenaient auprès des accoudoirs.
\VS{19}Douze lions se tenaient là sur les six marches de part et d'autre. Rien de pareil n'avait été fait pour aucun royaume.
\VS{20}Et toutes les coupes à boire du roi Salomon étaient d'or, et toute la vaisselle de la maison de la forêt du Liban était d'or pur ; rien n'était d'argent ; on n'en faisait aucun cas du temps de Salomon.
\VS{21}Car les navires du roi allaient à Tarsis avec les serviteurs de Huram ; et une fois tous les trois ans arrivaient les navires de Tarsis, apportant de l'or, de l'argent, des dents d'éléphants, des singes et des paons.
\VS{22}Le roi Salomon fut plus grand que tous les rois de la terre, tant en richesses qu'en sagesse.
\VS{23}Tous les rois de la terre cherchaient à voir la face de Salomon, pour écouter la sagesse que Dieu avait mise dans son cœur.
\VS{24}Et chacun d'eux apportait son présent : Des ustensiles d'argent, des ustensiles d'or, des vêtements, des armes, des aromates, des chevaux et des mulets, et il en était ainsi année après année.
\VS{25}Salomon avait quatre mille écuries pour ses chevaux, avec des chars ; et douze mille cavaliers qu'il plaça dans les villes où il avait des chars et auprès du roi à Jérusalem.
\VS{26}Il dominait sur tous les rois depuis le fleuve jusqu'au pays des Philistins, et jusqu'à la frontière d'Egypte.
\VS{27}Et le roi fit que l'argent était aussi commun à Jérusalem que les pierres, et les cèdres aussi nombreux que les sycomores qui sont dans les plaines.
\VS{28}On tirait des chevaux pour Salomon de l'Egypte et de tous les pays.
\TextTitle{Mort de Salomon\FTNTT{1 R. 11:1-40}}
\VS{29}Le reste des actions de Salomon, les premières et les dernières, cela n'est-il pas écrit dans le livre de Nathan le prophète, dans la prophétie d'Achija de Silo, et dans la vision de Jéedo le voyant, touchant Jéroboam, fils de Nebath ?
\VS{30}Salomon régna quarante ans à Jérusalem sur tout Israël.
\VS{31}Puis Salomon s'endormit avec ses pères, et on l'ensevelit dans la cité de David, son père ; et Roboam, son fils, régna à sa place.
\Chap{10}
\TextTitle{Roboam règne sur Israël\FTNTT{1 R. 12:1-15}}
\VerseOne{}Roboam se rendit à Sichem, car tout Israël était venu à Sichem pour l'établir roi.
\VS{2}Quand Jéroboam, fils de Nebath, qui était en Egypte, où il s'était enfui de devant le roi Salomon, l'eut appris, il revint d'Egypte.
\VS{3}Or on l'envoya appeler. Ainsi Jéroboam et tout Israël vinrent et parlèrent à Roboam, en disant :
\VS{4}Ton père a mis sur nous un joug pesant. Allège maintenant cette rude servitude de ton père, et ce joug pesant qu'il a mis sur nous, et nous te servirons.
\VS{5}Alors il leur dit : Revenez vers moi dans trois jours. Et le peuple s'en alla.
\VS{6}Le roi Roboam demanda conseil aux vieillards qui avaient été auprès de Salomon, son père, pendant sa vie, et il leur parla ainsi : Comment, et quelle chose me conseillez-vous de répondre à ce peuple ?
\VS{7} Et ils lui répondirent en ces termes : Si tu es bon envers ce peuple, si tu es bienveillant envers eux, et que tu leur dises de bonnes paroles, ils seront tes serviteurs à toujours.
\VS{8}Mais il laissa le conseil que les vieillards lui avaient donné, et il demanda conseil aux jeunes gens qui avaient grandi avec lui, et qui se tenaient auprès de lui.
\VS{9}Et il leur dit : Que me conseillez-vous de répondre à ce peuple qui m'a parlé en disant : Allège le joug que ton père a mis sur nous ?
\VS{10}Et les jeunes gens qui avaient grandi avec lui, lui parlèrent en disant : Tu répondras en disant à ce peuple qui t'a parlé et t'a dit : Ton père a mis sur nous un joug pesant, mais toi, allège-le ; tu leur répondras donc : Mon petit doigt est plus gros que les reins de mon père.
\VS{11}Or mon père a mis sur vous un joug pesant, mais moi, je rendrai votre joug encore plus pesant. Mon père vous a châtiés avec des fouets, mais moi, je vous châtierai avec des scorpions.
\TextTitle{Roboam délaisse le conseil des anciens}
\VS{12}Trois jours après, Jéroboam, avec tout le peuple, vint vers Roboam, suivant ce qu'avait dit le roi : Revenez vers moi dans trois jours.
\VS{13}Mais le roi leur répondit durement. Le roi Roboam délaissa le conseil des anciens,
\VS{14}et leur parla suivant le conseil des jeunes gens, en disant : Mon père a mis sur vous un joug pesant ; mais moi, j'y ajouterai encore. Mon père vous a châtiés avec des fouets ; mais moi, je vous châtierai avec des scorpions.
\VS{15}Le roi n'écouta donc point le peuple ; cela était conduit par Dieu, afin que Yahweh accomplisse la parole qu'il avait déclarée par Achija de Silo, à Jéroboam, fils de Nebath.
\TextTitle{Israël se détache de la maison de David\FTNTT{1 R. 12:16-19}}
\VS{16}Quand tout Israël vit que le roi ne les écoutait pas, le peuple répondit au roi, en disant : Quelle part avons-nous avec David ? Nous n'avons point d'héritage avec le fils d'Isaï. Israël, chacun à ses tentes ! Et toi David, pourvois maintenant à ta maison. Ainsi, tout Israël s'en alla dans ses tentes.
\VS{17}Mais quant aux enfants d'Israël qui habitaient les villes de Juda, Roboam régna sur eux.
\VS{18}Alors le roi Roboam envoya Hadoram, qui était préposé aux impôts ; mais les enfants d'Israël le lapidèrent à coups de pierres et il mourut. Et le roi Roboam se hâta de monter sur un char pour s'enfuir à Jérusalem.
\VS{19}C'est ainsi qu'Israël s'est rebellé contre la maison de David, jusqu'à ce jour\FTNT{1 R. 12:16-19.}.
\Chap{11} 
\TextTitle{Yahweh interdit la guerre entre Juda et Israël\FTNTT{1 R. 12:21-24}}
\VerseOne{}Roboam, étant arrivé à Jérusalem, assembla la maison de Juda et de Benjamin, cent quatre-vingt mille hommes d'élite et de guerre, afin de combattre contre Israël, pour le ramener sous le règne de Roboam.
\VS{2}Mais la parole de Yahweh fut adressée à Schemaeja, homme de Dieu, en ces termes :
\VS{3}Parle à Roboam, fils de Salomon, roi de Juda, et à ceux d'Israël qui sont en Juda et en Benjamin, et dis-leur :
\VS{4}Ainsi parle Yahweh : Ne montez point, et ne combattez point contre vos frères. Retournez chacun dans sa maison ; c'est par moi que cette chose est arrivée. Et ils obéirent aux paroles de Yahweh, et ils s'en retournèrent sans aller contre Jéroboam\FTNT{1 R. 12:21-24.}.
\VS{5}Roboam demeura donc à Jérusalem, et il bâtit des villes fortes en Juda.
\VS{6}Il bâtit Bethléhem, Etham, Tekoa,
\VS{7}Beth-Tsur, Soco, Adullam,
\VS{8}Gath, Maréscha, Ziph,
\VS{9}Adoraïm, Lakis, Azéka,
\VS{10}Tsorea, Ajalon et Hébron, qui étaient en Juda et en Benjamin, et en fit des villes fortes.
\VS{11}Il les fortifia et y mit des gouverneurs, des provisions de vivres, d'huile et de vin.
\VS{12}Dans chacune de ces villes, il mit des boucliers et des lances, et il les rendit puissantes. Ainsi Juda et Benjamin lui furent soumis.
\TextTitle{Les sacrificateurs et les Lévites soutiennent Roboam}
\VS{13}Les sacrificateurs et les Lévites, qui étaient dans tout Israël, vinrent de toutes leurs contrées se joindre à lui.
\TextTitle{Jéroboam abandonne Yahweh\FTNTT{1 R. 12:26-30 ; 14:7-8}}
\VS{14}Car les Lévites abandonnèrent leurs faubourgs et leurs possessions et vinrent en Juda et à Jérusalem, parce que Jéroboam et ses fils les avaient rejetés des fonctions de sacrificateurs pour Yahweh.
\VS{15}Car il s'était établi des sacrificateurs pour les hauts lieux, pour les boucs, et pour les veaux qu'il avait faits.
\VS{16}Et à leur suite, ceux d'entre toutes les tribus d'Israël qui avaient appliqué leur cœur à chercher Yahweh, le Dieu d'Israël, vinrent à Jérusalem pour sacrifier à Yahweh, le Dieu de leurs pères.
\VS{17}Ils fortifièrent le royaume de Juda et affermirent Roboam, fils de Salomon, pendant trois ans ; car on suivit les voies de David et de Salomon pendant trois ans.
\TextTitle{Les femmes et les enfants de Roboam}
\VS{18}Or Roboam prit pour femme : Mahalath, fille de Jerimoth, fils de David et d'Abichaïl, fille d'Eliab, fils d'Isaï.
\VS{19}Elle lui enfanta des fils : Jeusch, Schemaria et Zaham.
\VS{20}Après elle, il prit Maaca, fille d'Absalom, qui lui enfanta Abija, Attaï, Ziza et Schelomith.
\VS{21}Roboam aima Maaca, fille d'Absalom, plus que toutes ses femmes et ses concubines. Car il prit dix-huit femmes et soixante concubines, et il engendra vingt-huit fils et soixante filles.
\VS{22}Roboam établit pour chef Abija, fils de Maaca, comme prince entre ses frères ; car il voulait le faire roi.
\VS{23}Il agit prudemment et dispersa tous ses fils dans toutes les contrées de Juda et de Benjamin, dans toutes les villes fortes ; il leur donna de quoi vivre en abondance, et demanda pour eux une multitude de femmes.
\Chap{12}
\TextTitle{Roboam affermi, il abandonne Yahweh\FTNTT{1 R. 14:21-24}}
\VerseOne{}Lorsque la royauté de Roboam fut affermie et qu'il eut acquis de la force, il abandonna la loi de Yahweh, et tout Israël avec lui\FTNT{1 R. 14:21-29.}.
\TextTitle{Yahweh veut livrer Juda à Schischak\FTNTT{1 R. 14:25-28}}
\VS{2}C'est pourquoi il arriva que la cinquième année du Roi Roboam, Schischak, roi d'Egypte, monta contre Jérusalem, parce qu'ils avaient péché contre Yahweh.
\VS{3}Il avait mille deux cents chars et soixante mille cavaliers, et le peuple qui vint avec lui d'Egypte, des Libyens, des Sukkiens et des Ethiopiens, était innombrable.
\VS{4}Il prit les villes fortes qui appartenaient à Juda, et vint jusqu'à Jérusalem.
\VS{5}Alors Schemaeja, le prophète, vint vers Roboam et les chefs de Juda, qui s'étaient assemblés à Jérusalem à cause de Schischak, et leur dit : Ainsi parle Yahweh : Vous m'avez abandonné ; moi aussi je vous abandonne aux mains de Schischak.
\VS{6}Alors les chefs d'Israël et le roi s'humilièrent, et dirent : Yahweh est juste !
\VS{7}Et quand Yahweh vit qu'ils s'humiliaient, la parole de Yahweh fut adressée à Schemaeja, et il lui dit : Ils se sont humiliés ; je ne les détruirai pas, mais je leur donnerai dans peu de temps un moyen d'échapper, et ma fureur ne se répandra point sur Jérusalem par la main de Schischak.
\VS{8}Toutefois, ils lui seront asservis, afin qu'ils sachent ce que c'est que de me servir ou de servir les royaumes de la terre.
\VS{9}Schischak, roi d'Egypte, monta donc contre Jérusalem, et prit les trésors de la maison de Yahweh et les trésors de la maison du roi ; il prit tout. Il prit les boucliers d'or que Salomon avait faits.
\VS{10}Le roi Roboam fit des boucliers d'airain à leur place, et il les mit entre les mains des chefs des coureurs qui gardaient la porte de la maison du roi.
\VS{11}Et toutes les fois que le roi entrait dans la maison de Yahweh, les coureurs venaient et les portaient ; puis ils les rapportaient dans la chambre des coureurs.
\VS{12}Ainsi comme il s'était humilié, la colère de Yahweh se détourna de lui, et ne le détruisit pas entièrement ; car il y avait encore de bonnes choses en Juda.
\TextTitle{Mort de Roboam\FTNTT{1 R. 14:21,29,31}}
\VS{13}Le roi Roboam se fortifia donc dans Jérusalem, et régna. Il avait quarante et un ans quand il devint roi, et il régna dix-sept ans à Jérusalem, la ville que Yahweh avait choisie de toutes les tribus d'Israël, pour y mettre son Nom. Sa mère s'appelait Naama, l'Ammonite.
\VS{14}Il fit le mal, car il ne disposa point son cœur pour chercher Yahweh.
\VS{15}Or les actions de Roboam, les premières et les dernières, ne sont-elles pas écrites dans les livres de Schemaeja le prophète, et d'Iddo le voyant, parmi les registres généalogiques ? Les guerres entre Roboam et Jéroboam furent continuelles.
\VS{16}Roboam s'endormit avec ses pères, et il fut enseveli dans la cité de David ; et Abija, son fils, régna à sa place.
\Chap{13}
\TextTitle{Abija règne sur Juda ; guerre entre Israël et Juda\FTNTT{1 R. 15:1-8}}
\VerseOne{}La dix-huitième année du roi Jéroboam, Abija commença à régner sur Juda.
\VS{2}Il régna trois ans à Jérusalem. Sa mère s'appelait Micaja, fille d'Uriel, de Guibea. Or il y eut guerre entre Abija et Jéroboam.
\VS{3}Abija engagea la guerre avec une armée de vaillants guerriers, quatre cent mille hommes d'élite ; et Jéroboam se rangea en bataille contre lui avec huit cent mille hommes d'élite, forts et vaillants.
\VS{4}Et Abija se leva du haut de la montagne de Tsemaraïm, parmi les montagnes d'Ephraïm, et dit : Jéroboam et tout Israël, écoutez-moi!
\VS{5}Ne savez-vous pas que Yahweh, le Dieu d'Israël, a donné pour toujours la royauté sur Israël à David, à lui et à ses fils, par une alliance de sel\FTNT{Sel : Voir commentaire en Lé. 2:13} !
\VS{6}Mais Jéroboam, fils de Nebath, serviteur de Salomon, fils de David, s'est élevé et s'est rebellé contre son seigneur.
\VS{7}Et des gens sans valeur, des fils de Belial, se sont assemblés avec lui et se sont fortifiés contre Roboam, fils de Salomon. Or Roboam était un jeune homme craintif et sans force devant eux.
\VS{8}Et maintenant, vous vous dites être forts devant la royauté de Yahweh, qui est aux mains des fils de David ; vous êtes une multitude, et vous avez avec vous les veaux d'or que Jéroboam vous a faits pour dieux.
\VS{9}N'avez-vous pas rejeté les sacrificateurs de Yahweh, les fils d'Aaron, et les Lévites ? Et ne vous êtes-vous pas faits des sacrificateurs comme les peuples des autres pays ? Quiconque venait, avec un jeune taureau et sept béliers pour être consacré, devenait sacrificateur de ce qui n'est pas Dieu.
\VS{10}Mais quant à nous, Yahweh est notre Dieu, et nous ne l'avons pas abandonné ; les sacrificateurs qui font le service de Yahweh sont fils d'Aaron, et ce sont les Lévites qui tiennent cette fonction.
\VS{11}Nous faisons brûler pour Yahweh, chaque matin et chaque soir, les holocaustes et le parfum d'aromates. Les pains de proposition sont rangés sur la table pure, et on allume le chandelier d'or avec ses lampes, chaque soir. Car nous gardons ce que Yahweh, notre Dieu,veut qu'on garde ; mais vous, vous l'avez abandonné.
\VS{12}Voici, Dieu est avec nous pour être notre chef, avec ses sacrificateurs, et les trompettes retentissantes, pour les faire sonner contre vous. Fils d'Israël, ne combattez pas contre Yahweh, le Dieu de vos pères ; car cela ne vous réussira pas.
\VS{13}Mais Jéroboam fit une embuscade par un détour, et arriva derrière eux. De sorte que les Israélites étaient en face de Juda, qui avait l'embuscade par-derrière.
\VS{14}Ceux de Juda se retournèrent et voici ils avaient la bataille par-devant et par-derrière. Alors ils crièrent à Yahweh, et les sacrificateurs sonnèrent des trompettes.
\TextTitle{Victoire de Juda sur Israël}
\VS{15}Les hommes de Juda poussèrent un cri, et au cri de guerre des hommes de Juda, Yahweh frappa Jéroboam et tout Israël devant Abija et Juda.
\VS{16}Les fils d'Israël s'enfuirent devant ceux de Juda, parce que Dieu les livra entre leurs mains.
\VS{17}Abija et son peuple leur firent un grand un carnage, et il tomba d'Israël cinq cent mille hommes d'élite blessés à mort.
\VS{18}Ainsi, les enfants d'Israël furent humiliés en ce temps-là ; et les enfants de Juda devinrent plus forts, parce qu'ils s'étaient appuyés sur Yahweh, le Dieu de leurs pères.
\VS{19} Abija poursuivit Jéroboam, et lui prit ces villes : Béthel et les villes de son ressort, Jeschana et les villes de son ressort, Ephron et les villes de son ressort.
\TextTitle{Mort de Jéroboam\FTNTT{1 R. 14:19,20}}
\VS{20}Et Jéroboam n'eut plus de force durant le temps d'Abija ; et Yahweh le frappa, et il mourut.
\TextTitle{Les femmes et les fils d'Abija\FTNTT{1 R. 15:7-8}}
\VS{21}Mais Abija se fortifia ; il prit quatorze femmes, et engendra vingt-deux fils et seize filles.
\VS{22}Le reste des actions d'Abija, sa conduite et ses paroles sont écrites dans les mémoires du prophète Iddo.
\VS{23}Abija s'endormit avec ses pères, et on l'ensevelit dans la cité de David ; et Asa, son fils, régna à sa place. De son temps, le pays fut en repos pendant dix ans.
\Chap{14}
\TextTitle{Asa règne sur Juda, il rétablit l'ordre de Yahweh\FTNTT{1 R. 15:11}}
\VerseOne{}Asa fit ce qui est bon et droit aux yeux de Yahweh, son Dieu.
\VS{2}Il ôta les autels étrangers et les hauts lieux ; il brisa les statues et mit en pièces les idoles d'Asherah.
\VS{3}Et il recommanda à Juda de rechercher Yahweh, le Dieu de leurs pères, et de pratiquer la loi et les commandements.
\VS{4}Il ôta de toutes les villes de Juda les hauts lieux et les colonnes consacrées au soleil. Et le royaume fut en repos devant lui.
\VS{5}Il bâtit des villes fortes en Juda, car le pays fut en repos. Et pendant ces années-là, il n'y eut point de guerre contre lui, parce que Yahweh lui donna du repos.
\VS{6}Et il dit à Juda : Bâtissons ces villes, et entourons-les de murailles, de tours, de portes et de barres ; le pays est encore devant nous, parce que nous avons recherché Yahweh, notre Dieu. Nous l'avons recherché, et il nous a donné du repos de toutes parts. Ainsi, ils bâtirent et prospérèrent.
\TextTitle{Asa s'appuie sur Yahweh et triomphe de Zérach\FTNTT{2 Ch. 16:1-10}}
\VS{7}Or Asa avait dans son armée trois cent mille hommes de Juda, portant le grand bouclier et la lance, et deux cent quatre-vingt mille de Benjamin, portant le bouclier et tirant de l'arc, tous vaillants guerriers.
\VS{8}Mais Zérach, l'Ethiopien, sortit contre eux avec une armée d'un million d'hommes, et de trois cents chars ; et il vint jusqu'à Maréscha.
\VS{9}Asa alla au-devant de lui, et ils se rangèrent en bataille dans la vallée de Tsephata, près de Maréscha.
\VS{10}Alors Asa cria à Yahweh, son Dieu, et dit : Yahweh ! Toi seul peux nous secourir, que l'on soit nombreux ou sans force ! Aide-nous, Yahweh, notre Dieu ! Car nous nous appuyons sur toi, et nous sommes venus en ton Nom contre cette multitude. Tu es Yahweh, notre Dieu : Que l'homme ne prévale pas contre toi !
\VS{11}Et Yahweh frappa les Ethiopiens devant Asa et devant Juda ; et les Ethiopiens s'enfuirent.
\VS{12}Asa et le peuple qui était avec lui les poursuivirent jusqu'à Guérar, et tant d'Ethiopiens tombèrent sans pouvoir sauver leur vie ; car ils furent brisés devant Yahweh et son armée, et on emporta un très grand butin.
\VS{13}Ils frappèrent aussi toutes les villes autour de Guérar, car la terreur de Yahweh était sur eux ; et ils pillèrent toutes ces villes, car il s'y trouvait un grand butin.
\VS{14}Ils frappèrent aussi les tentes des troupeaux, et emmenèrent des brebis et des chameaux en abondance ; puis ils retournèrent à Jérusalem.
\Chap{15}
\TextTitle{Azaria le prophète avertit Asa}
\VerseOne{}Alors l'Esprit de Dieu fut sur Azaria, fils d'Oded.
\VS{2}Et il sortit au-devant d'Asa, et lui dit : Asa, et tout Juda et Benjamin, écoutez-moi ! Yahweh est avec vous quand vous êtes avec lui. Si vous le cherchez, vous le trouverez ; mais si vous l'abandonnez, il vous abandonnera.
\VS{3}Pendant longtemps Israël a été sans vrai Dieu, sans sacrificateur qui l'enseignait, et sans loi.
\VS{4}Mais dans leur détresse, ils sont revenus vers Yahweh, le Dieu d'Israël ; ils l'ont cherché, et ils l'ont trouvé\FTNT{Ps. 107:19-20.}.
\VS{5}Dans ces temps-là, il n'y avait point de sûreté pour ceux qui allaient et venaient, car il y avait de grands troubles parmi tous les habitants du pays.
\VS{6}Une nation était écrasée par une autre nation, et une ville par une autre ville ; car Dieu les agitait par toutes sortes d'angoisses.
\VS{7}Mais vous, fortifiez-vous, et que vos mains ne se relâchent pas ; car il y a une récompense pour vos œuvres.
\TextTitle{Asa écoute les paroles d'Azaria\FTNTT{1 R. 15:12-15}}
\VS{8}Or dès qu'Asa eut entendu ces paroles et la prophétie d'Oded le prophète, il se fortifia ; et fit disparaître les abominations de tout le pays de Juda et de Benjamin, et des villes qu'il avait prises dans les montagnes d'Ephraïm ; et il rétablit l'autel de Yahweh, qui était devant le portique de Yahweh.
\VS{9}Puis il assembla tout Juda et Benjamin, et ceux d'Ephraïm, de Manassé et de Siméon, qui habitaient avec eux ; car un grand nombre de gens d'Israël passaient à lui, voyant que Yahweh, son Dieu, était avec lui.
\VS{10}Ils s'assemblèrent donc à Jérusalem, le troisième mois de la quinzième année du règne d'Asa ;
\VS{11}et ils sacrifièrent ce jour-là à Yahweh sept cents bœufs et sept mille brebis, du butin qu'ils avaient amené.
\VS{12}Et ils rentrèrent dans l'alliance pour chercher Yahweh, le Dieu de leurs pères, de tout leur cœur et de toute leur âme ;
\VS{13}de sorte qu'on devait faire mourir quiconque ne rechercherait pas Yahweh, le Dieu d'Israël, petit ou grand, homme ou femme.
\VS{14}Et ils jurèrent à Yahweh, à haute voix, avec des cris de joie, et au son des shofars et des cors.
\VS{15}Tout Juda se réjouit de ce serment, parce qu'ils avaient juré de tout leur cœur et qu'ils avaient recherché Yahweh de leur plein gré, et qu'ils l'avaient trouvé. Et Yahweh leur donna du repos de toutes parts.
\VS{16}Le roi Asa destitua même sa mère, Maaca, de son rang de reine, parce qu'elle avait fait une idole pour Astarté. Asa abattit l'idole, l'écrasa et la brûla près du torrent de Cédron.
\VS{17}Mais les hauts lieux ne furent point ôtés du milieu d'Israël. Néanmoins, le cœur d'Asa fut intègre tout le long de ses jours.
\VS{18}Il remit dans la maison de Dieu les choses que son père avait consacrées, avec ce qu'il avait lui-même consacré, l'argent, l'or et les ustensiles.
\VS{19}Et il n'y eut point de guerre jusqu'à la trente-cinquième année du règne d'Asa.
\Chap{16}
\TextTitle{Alliance d'Asa et du roi de Syrie contre le roi d'Israël\FTNTT{ 1 R. 15:16-22 ; cp. 1 R. 15:27 ; 16:7}}
\VerseOne{}La trente-sixième année du règne d'Asa, Baescha, roi d'Israël, monta contre Juda, et il bâtit Rama, pour empêcher quiconque de sortir et d'entrer vers Asa, roi de Juda.
\VS{2}Alors Asa sortit de l'argent et de l'or des trésors de la maison de Yahweh et de la maison royale, et il envoya dire à Ben-Hadad, roi de Syrie, qui habitait à Damas :
\VS{3}Il y a alliance entre nous, et entre mon père et ton père ; voici, je t'envoie de l'argent et de l'or ; va, romps l'alliance que tu as avec Baescha, roi d'Israël, afin qu'il s'éloigne de moi.
\VS{4}Ben-Hadad écouta le roi Asa, et il envoya les chefs de son armée contre les villes d'Israël, et ils frappèrent Ijjon, Dan, Abel-Maïm, et tous les magasins des villes de Nephthali.
\VS{5}Et aussitôt que Baescha l'apprit, il cessa de bâtir Rama et suspendit ses travaux.
\VS{6}Alors le roi Asa prit avec lui tout Juda, et ils emportèrent les pierres et le bois de Rama, que Baescha faisait bâtir ; et il en bâtit Guéba et Mitspa.
\TextTitle{Hanani condamne l'alliance d'Asa}
\VS{7}En ce temps-là, Hanani le voyant, vint vers Asa, roi de Juda, et lui dit : Parce que tu t'es appuyé sur le roi de Syrie, et que tu ne t'es point appuyé sur Yahweh, ton Dieu, l'armée du roi de Syrie a échappé de ta main.
\VS{8}Les Ethiopiens et les Libyens n'étaient-ils pas une grande armée, ayant des chars et une multitude de cavaliers ? Mais parce que tu t'étais appuyé sur Yahweh, il les livra entre tes mains.
\VS{9}Car les yeux de Yahweh parcourent toute la terre, pour soutenir ceux dont le cœur est tout entier à lui. Tu as agi follement en cela ; car désormais tu auras des guerres.
\VS{10}Asa fut irrité contre le voyant, et le mit en prison, car il était indigné contre lui à ce sujet. Asa opprima aussi, en ce temps-là, quelques-uns du peuple.
\TextTitle{Mort d'Asa\FTNTT{1 R. 15:23-24}}
\VS{11}Or voici, les actions d'Asa, les premières et les dernières, sont écrites dans le livre des rois de Juda et d'Israël.
\VS{12}Asa fut malade des pieds la trente-neuvième année de son règne, et sa maladie fut très grave. Toutefois, il ne chercha point Yahweh dans sa maladie, mais les médecins.
\VS{13}Puis Asa s'endormit avec ses pères, et il mourut la quarante et unième année de son règne.
\VS{14}On l'ensevelit dans le sépulcre qu'il s'était creusé dans la cité de David. On le coucha dans un lit qui était rempli de parfums et d'aromates, composés par le travail d'un parfumeur ; et on lui en brûla une quantité considérable.
\Chap{17}
\TextTitle{Josaphat règne sur Juda, il recherche Yahweh\FTNTT{1 R. 15:24}}
\VerseOne{}Josaphat son fils régna à sa place et se fortifia contre Israël.
\VS{2}Il mit des troupes dans toutes les villes fortes de Juda, et des garnisons dans le pays de Juda, et dans les villes d'Ephraïm qu'Asa, son père, avait prises.
\VS{3}Yahweh fut avec Josaphat, parce qu'il suivit les premières voies de David, son père, et qu'il ne rechercha point les Baals ;
\VS{4}car il rechercha le Dieu de son père, et il marcha dans ses commandements, et non pas selon ce que faisait Israël.
\VS{5}Yahweh affermit donc le royaume entre ses mains ; et tout Juda apportait des présents à Josaphat, et il eut en abondance des richesses et de la gloire.
\VS{6}Son cœur grandit dans les voies de Yahweh, et il ôta encore de Juda les hauts lieux et les idoles d'Astarté.
\VS{7}Puis, la troisième année de son règne, il envoya ses chefs Ben-Haïl, Abdias, Zacharie, Nethaneel et Michée, pour enseigner dans les villes de Juda ;
\VS{8}et avec eux les Lévites Schemaeja, Nethania, Zebadia, Asaël, Schemiramoth, Jonathan, Adonija, Tobija et Tob-Adonija, Lévites, et avec eux Elischama et Joram, les sacrificateurs.
\VS{9}Ils enseignèrent dans Juda, ayant avec eux le livre de la loi de Yahweh. Ils firent le tour de toutes les villes de Juda, et enseignèrent parmi le peuple.
\TextTitle{Affermissement du règne de Josaphat}
\VS{10}La terreur de Yahweh fut sur tous les royaumes des pays qui entouraient Juda, et ils ne firent point la guerre à Josaphat.
\VS{11}On apporta aussi à Josaphat des présents de la part des Philistins, et un impôt en argent ; et les Arabes lui amenèrent aussi du bétail, sept mille sept cents béliers et sept mille sept cents boucs.
\VS{12}Ainsi Josaphat s'élevait jusqu'au plus haut degré de gloire. Et il bâtit en Juda des châteaux et des villes pour servir de magasins.
\VS{13}Il fit de grands travaux dans les villes de Juda ; et il avait à Jérusalem des gens de guerre puissants et vaillants.
\VS{14}Voici leur dénombrement, selon les maisons de leurs pères. Les chefs de milliers de Juda furent Adna le chef, avec trois cent mille vaillants guerriers.
\VS{15}Et après lui, Jochanan le chef, avec deux cent quatre-vingt mille hommes.
\VS{16}A ses côtés, Amasia, fils de Zicri, qui s'était volontairement offert à Yahweh, avec deux cent mille vaillants guerriers.
\VS{17}De Benjamin, Eliada, vaillant guerrier, avec deux cent mille hommes, armés d'arcs et de boucliers,
\VS{18}à côté de lui Zozabad, avec cent quatre-vingt mille hommes équipés pour le combat.
\VS{19}Tels sont ceux qui étaient au service du roi, outre ceux que le roi avait placés dans toutes les villes fortes de Juda.
\Chap{18}
\TextTitle{Josaphat s'allie à Achab contre les Syriens\FTNTT{1 R. 22:2-4}}
\VerseOne{}Or Josaphat, ayant beaucoup de richesses et de gloire, s'allia par mariage avec Achab.
\VS{2}Et au bout de quelques années, il descendit vers Achab, à Samarie. Achab tua pour lui, et pour le peuple qui était avec lui, un grand nombre de brebis et de bœufs, et l'incita à monter contre Ramoth de Galaad\FTNT{1 R. 22:2-40.}.
\VS{3}Achab, roi d'Israël, dit à Josaphat, roi de Juda : Viendras-tu avec moi contre Ramoth de Galaad ? Et il lui répondit : Compte sur moi comme sur toi, et sur mon peuple comme sur ton peuple, nous irons avec toi à la guerre.
\TextTitle{Les prophètes de mensonge encouragent Achab\FTNTT{1 R. 22:5-12}}
\VS{4}Puis Josaphat dit au roi d'Israël : Consulte aujourd'hui, je te prie, la parole de Yahweh.
\VS{5}Le roi d'Israël assembla les prophètes, au nombre de quatre cents, et leur dit : Irons-nous à la guerre contre Ramoth de Galaad, ou dois-je y renoncer ? Ils répondirent : Monte, et Dieu la livrera entre les mains du roi.
\VS{6}Mais Josaphat dit : N'y a-t-il point encore ici quelque prophète de Yahweh, afin que nous l'interrogions ?
\VS{7}Le roi d'Israël dit à Josaphat : Il y a encore un homme par qui on peut consulter Yahweh ; mais je le hais parce qu'il ne me prophétise rien de bon, mais du mal ; c'est Michée, fils de Jimla. Josaphat dit : Que le roi ne parle pas ainsi !
\VS{8}Alors le roi d'Israël appela un eunuque, et dit : Fais promptement venir Michée, fils de Jimla.
\VS{9}Or le roi d'Israël et Josaphat, roi de Juda, étaient assis, chacun sur son trône, revêtus de leurs habits, et ils étaient assis dans la place, à l'entrée de la porte de Samarie ; et tous les prophètes prophétisaient en leur présence.
\VS{10}Alors Sédécias, fils de Kenaana, s'étant fait des cornes de fer, dit : Ainsi parle Yahweh : Avec ces cornes tu heurteras les Syriens jusqu'à les détruire.
\VS{11}Tous les prophètes prophétisaient de même, en disant : Monte à Ramoth de Galaad, et tu prospèreras ; Yahweh la livrera entre les mains du roi.
\TextTitle{Michée annonce la défaite et la mort d'Achab\FTNTT{1 R. 22:13-28 ; 1 R. 22:29-40}}
\VS{12}Or le messager qui était allé appeler Michée, lui parla et lui dit : Voici, tous les prophètes disent d'une même bouche du bien au roi ; je te prie que ta parole soit semblable à celle de chacun d'eux ! Annonce du bien !
\VS{13}Mais Michée répondit : Yahweh est vivant ! Je dirai ce que mon Dieu dira.
\VS{14}Il vint donc vers le roi, et le roi lui dit : Michée, irons-nous à la guerre contre Ramoth de Galaad, devons-nous y renoncer ? Et il répondit : Montez, vous prospérerez, et ils seront livrés entre vos mains.
\VS{15}Et le roi lui dit : Combien de fois devrais-je te faire jurer de ne me dire que la vérité au Nom de Yahweh ?
\VS{16}Et il répondit : J'ai vu tout Israël dispersé par les montagnes, comme un troupeau de brebis qui n'a point de berger ; et Yahweh a dit : Ces gens n'ont point de seigneur ; que chacun retourne en paix dans sa maison !
\VS{17}Alors le roi d'Israël dit à Josaphat : Ne t'ai-je pas dit qu'il ne prophétise rien de bon quand il s'agit de moi, mais seulement du mal ?
\VS{18}Et Michée dit : Ecoute la parole de Yahweh ! J'ai vu Yahweh assis sur son trône, et toute l'armée des cieux se tenant à sa droite et à sa gauche.
\VS{19}Et Yahweh dit : Qui est-ce qui séduira Achab, roi d'Israël, afin qu'il monte et qu'il tombe à Ramoth de Galaad ? Et l'un répondait d'une façon et l'autre d'une autre.
\VS{20}Alors un esprit s'avança et se tint devant Yahweh, et dit : Moi, je le séduirai. Yahweh lui dit : Comment ?
\VS{21}Il répondit : Je sortirai, dit-il, et je serai un esprit de mensonge\FTNT{Achab a été frappé de l'esprit d'égarement (2 Th. 2:9-11). Voir commentaires en Ge. 6:3 ; Mt. 12:31.} dans la bouche de tous ses prophètes. Et Yahweh dit : Tu le séduiras, et même tu en viendras à bout. Sors, et fais ainsi.
\VS{22}Maintenant voici, Yahweh a mis un esprit de mensonge dans la bouche de tes prophètes que voilà ; et Yahweh a prononcé du mal contre toi.
\VS{23}Alors Sédécias, fils de Kenaana, s'étant approché, frappa Michée sur la joue, et dit : Par quel chemin l'Esprit de Yahweh s'est-il retiré de moi pour te parler ?
\VS{24}Et Michée répondit : Voici, tu le verras au jour où tu iras de chambre en chambre pour te cacher !
\VS{25}Alors le roi d'Israël dit : Prenez Michée, et emmenez-le vers Amon, chef de la ville, et vers Joas, fils du roi.
\VS{26}Et vous direz : Ainsi parle le roi : Mettez cet homme en prison, et nourrissez-le du pain et de l'eau de l'affliction, jusqu'à ce que je revienne en paix.
\VS{27}Et Michée dit : Si jamais tu retournes et reviens en paix, Yahweh n'aura point parlé par moi. Et il dit : Entendez cela peuples, vous tous qui êtes ici !
\VS{28}Le roi d'Israël monta donc avec Josaphat, roi de Juda, à Ramoth de Galaad.
\VS{29}Le roi d'Israël dit à Josaphat : Je vais me déguiser pour aller au combat ; mais toi, revêts-toi de tes habits. Ainsi le roi d'Israël se déguisa ; et ils allèrent au combat.
\VS{30}Or le roi des Syriens avait donné cet ordre aux chefs de ses chars, disant : Vous ne combattrez ni petit ni grand, mais seulement le roi d'Israël.
\VS{31}Les chefs des chars aperçurent Josaphat, et dirent : C'est le roi d'Israël ! Et ils se tournèrent vers lui pour le combattre ; mais Josaphat poussa un cri, et Yahweh le secourut, et Dieu les éloigna de lui.
\VS{32}Quand les chefs des chars virent que ce n'était pas le roi d'Israël, ils se détournèrent de lui.
\VS{33}Alors quelqu'un tira de son arc au hasard, et frappa le roi d'Israël entre les jointures de la cuirasse ; et le roi dit à son conducteur de char : Tourne-toi, et sors-moi du camp ; car je suis blessé.
\VS{34}Or en ce jour-là, le combat fut très rude. Le roi d'Israël se posa dans son char, en face des Syriens, jusqu'au soir ; et il mourut vers le coucher du soleil.
\Chap{19}
\TextTitle{Jéhu dénonce l'alliance de Josaphat avec Achab}
\VerseOne{}Josaphat roi de Juda, revint en paix dans sa maison, à Jérusalem.
\VS{2}Mais Jéhu, fils de Hanani, le voyant, sortit au-devant du roi Josaphat, et lui dit : Faut-il donner du secours au méchant, ou aimer ceux qui haïssent Yahweh ? A cause de cela, Yahweh est irrité contre toi.
\VS{3}Mais il s'est trouvé de bonnes choses en toi, puisque tu as ôté du pays les idoles d'Astarté, et tu as appliqué ton cœur à rechercher Dieu.
\VS{4}Josaphat demeura à Jérusalem. Puis, il ressortit de nouveau parmi le peuple, depuis Beer-Schéba jusqu'à la montagne d'Ephraïm, et il les ramena à Yahweh, le Dieu de leurs pères.
\TextTitle{Josaphat organise la justice}
\VS{5}Il établit aussi des juges dans le pays, dans toutes les villes fortes de Juda, de ville en ville.
\VS{6}Et il dit aux juges : Veillez sur ce que vous ferez ; car vous n'exercez pas la justice de la part d'un homme, mais de la part de Yahweh, qui sera avec vous quand vous prononcerez les jugements.
\VS{7}Maintenant, que la crainte de Yahweh soit sur vous ; prenez garde à ce que vous ferez ; car il n'y a point d'iniquité chez Yahweh, notre Dieu, ni d'acception de personnes, ni d'acceptation de présents. 
\VS{8}Josaphat établit aussi à Jérusalem des Lévites, des sacrificateurs, et des chefs des pères d'Israël, pour le jugement de Yahweh, et pour les contestations ; car on revenait à Jérusalem.
\VS{9}Il leur donna des ordres, en disant : Vous agirez ainsi dans la crainte de Yahweh, avec fidélité et avec intégrité de cœur.
\VS{10}Dans toute contestation qui viendra devant vous, de la part de vos frères qui habitent dans leurs villes, soit d'un meurtre, d'une loi, d'un commandement, d'un statut ou d'une ordonnance, vous les instruirez, afin qu'ils ne se rendent pas coupables envers Yahweh, et que sa colère ne vienne pas sur vous et sur vos frères. Vous agirez ainsi afin de ne pas être coupables.
\VS{11}Et voici, Amaria, le souverain sacrificateur, sera au-dessus de vous pour toutes les affaires de Yahweh ; et Zebadia, fils d'Ismaël, prince de la maison de Juda, pour toutes les affaires du roi ; et pour secrétaires, vous avez devant vous les Lévites. Fortifiez-vous et faites ainsi ; et que Yahweh soit avec l'homme de bien !
\Chap{20}
\TextTitle{Menaces des ennemis de Juda, prière de Josaphat}
\VerseOne{}Après ces choses, les fils de Moab et les fils d'Ammon, et avec eux les Maonites, vinrent contre Josaphat pour lui faire la guerre.
\VS{2}On vint le rapporter à Josaphat, en disant : Il vient contre toi une grande multitude depuis l'autre bord de la mer, de Syrie ; et les voici à Hatsatson-Thamar, qui est En-Guédi.
\VS{3}Alors Josaphat craignit ; mais il se disposa à rechercher Yahweh, et publia un jeûne pour tout Juda.
\VS{4}Juda s'assembla donc pour rechercher Yahweh ; on vint même de toutes les villes de Juda pour chercher Yahweh.
\VS{5}Et Josaphat se tint au milieu de l'assemblée de Juda et de Jérusalem, dans la maison de Yahweh, devant le nouveau parvis.
\VS{6}Il dit : Yahweh, Dieu de nos pères ! N'es-tu pas Dieu dans les cieux, toi qui domines sur tous les royaumes des nations ? Ne tiens-tu pas dans ta main la force et la puissance, et à qui nul ne peut résister ?
\VS{7}N'est-ce pas toi, ô notre Dieu, qui as dépossédé les habitants de ce pays devant ton peuple d'Israël, et qui l'as donné pour toujours à la postérité d'Abraham, qui t'aimait ?
\VS{8}Ils y ont habité et t'y ont bâti un sanctuaire pour ton Nom, en disant :
\VS{9}S'il nous arrive quelque malheur, l'épée, le jugement, la peste, ou la famine, nous nous tiendrons devant cette maison, et en ta présence ; car ton Nom est en cette maison ; et nous crierons à toi dans notre détresse, et tu exauceras et tu délivreras !
\VS{10}Maintenant, voici les enfants d'Ammon et de Moab, et ceux de la montagne de Séir, chez lesquels tu ne permis pas à Israël d'entrer quand il venait du pays d'Egypte, car il se détourna d'eux, et ne les détruisit pas.
\VS{11}Voici, pour nous récompenser, ils viennent nous chasser de ton héritage, que tu nous as fait posséder.
\VS{12}Ô notre Dieu ! Ne seras-tu pas juge contre eux ? Car nous sommes sans force devant cette grande multitude qui vient contre nous, et nous ne savons que faire ; mais nos yeux sont sur toi.
\VS{13}Or tout Juda se tenait devant Yahweh, même avec leurs petits enfants, leurs femmes et leurs fils.
\TextTitle{Yahweh répond à Josaphat}
\VS{14}Alors l'Esprit de Yahweh saisit au milieu de l'assemblée Jachaziel, fils de Zacharie, fils de Benaja, fils de Jeïel, fils de Matthania, Lévite, d'entre les fils d'Asaph,
\VS{15}et il dit : Soyez attentifs, tout Juda et habitants de Jérusalem, et toi, roi Josaphat ! Ainsi parle Yahweh : Ne craignez point, et ne soyez point effrayés en face de cette grande multitude ; car ce ne sera pas à vous de combattre, mais à Dieu.
\VS{16}Descendez demain vers eux ; les voici qui montent par la montée de Tsits, et vous les trouverez à l'extrémité de la vallée, en face du désert de Jeruel.
\VS{17}Ce ne sera point à vous de combattre en cette bataille ; présentez-vous, tenez-vous là, et voyez la délivrance que Yahweh va vous donner. Juda et Jérusalem, ne craignez point, et ne soyez point effrayés ! Demain, sortez au-devant d'eux, et Yahweh sera avec vous.
\VS{18}Alors Josaphat s'inclina le visage contre terre, et tout Juda et les habitants de Jérusalem se jetèrent devant Yahweh, se prosternant devant Yahweh.
\VS{19}Et les Lévites, d'entre les fils des Kehathites et d'entre les fils des Koréites, se levèrent pour célébrer Yahweh, le Dieu d'Israël, d'une voix haute et forte.
\TextTitle{Yahweh délivre Juda des armées ennemies}
\VS{20}Puis, le matin, ils se levèrent de bonne heure et sortirent vers le désert de Tekoa. Et comme ils sortaient, Josaphat se tint debout et dit : Ecoutez-moi Juda et vous, habitants de Jérusalem ! Croyez en Yahweh, votre Dieu, et vous serez en sûreté ; croyez en ses prophètes, et vous réussirez.
\VS{21}Puis, ayant consulté le peuple, il établit des chantres de Yahweh, qui célébraient sa sainte magnificence ; et marchant devant l'armée, ils disaient : Louez Yahweh, car sa miséricorde dure à toujours\FTNT{Ps. 136} !
\VS{22}Et au moment où ils commencèrent le chant et la louange, Yahweh mit des embuscades contre les fils d'Ammon, de Moab, et ceux de la montagne de Séir, qui venaient contre Juda. Et ils furent battus.
\VS{23}Les fils d'Ammon et de Moab se levèrent contre les habitants de la montagne de Séir pour les dévouer par interdit et les exterminer ; et quand ils en eurent fini avec les habitants de Séir, ils s'aidèrent l'un l'autre à se détruire mutuellement.
\VS{24}Et quand Juda fut arrivé sur la hauteur d'où l'on voit le désert, ils regardèrent vers cette multitude, et voici, c'étaient des cadavres gisant à terre, et personne n'avait échappé.
\VS{25}Ainsi Josaphat et son peuple vinrent pour piller leurs dépouilles, et ils trouvèrent parmi les cadavres des biens en abondance, et des objets précieux ; et ils en saisirent tant qu'ils ne pouvaient tout porter ; et ils pillèrent le butin pendant trois jours, car il était considérable.
\VS{26}Le quatrième jour, ils s'assemblèrent dans la vallée de Beraca ; car ils bénirent là Yahweh ; c'est pourquoi on a appelé ce lieu, jusqu'à ce jour, la vallée de Beraca.
\VS{27}Et tous les hommes de Juda et de Jérusalem, et Josaphat à leur tête, s'en retournèrent, revenant à Jérusalem avec joie ; car Yahweh les avait réjouis au sujet de leurs ennemis. 
\VS{28}Ils entrèrent donc à Jérusalem, dans la maison de Yahweh, avec des luths, des harpes et des trompettes.
\VS{29}Et la crainte de Dieu fut sur tous les royaumes des autres pays, lorsqu'ils apprirent que Yahweh avait combattu contre les ennemis d'Israël.
\VS{30}Ainsi le royaume de Josaphat fut tranquille, et son Dieu lui donna du repos de toutes parts.
\TextTitle{Règne de Josaphat, son alliance coupable\FTNTT{1 R. 22:41-49}}
\VS{31}Josaphat régna donc sur Juda. Il était âgé de trente-cinq ans quand il devint roi, et il régna vingt-cinq ans à Jérusalem. Sa mère s'appelait Azuba, fille de Schilchi.
\VS{32}Il suivit les traces d'Asa, son père, et il ne s'en détourna point, faisant ce qui est droit aux yeux de Yahweh.
\VS{33}Seulement les hauts lieux ne furent pas ôtés, et le peuple n'avait pas encore le cœur fermement attaché au Dieu de ses pères.
\VS{34}Or le reste des actions de Josaphat, les premières et les dernières, voici, elles sont écrites dans les mémoires de Jéhu, fils de Hanani, insérées dans le livre des rois d'Israël.
\VS{35}Après cela, Josaphat, roi de Juda, s'associa avec Achazia, roi d'Israël, dont la conduite était impie.
\VS{36}Il s'associa avec lui pour faire des navires, afin d'aller à Tarsis ; et ils firent des navires à Etsjon-Guéber.
\VS{37}Alors Eliézer, fils de Dodava, de Maréscha, prophétisa contre Josaphat, en disant : Parce que tu t'es associé avec Achazia, Yahweh a détruit ton œuvre. Et les navires furent brisés, et ne purent aller à Tarsis.
\Chap{21}
\TextTitle{Joram règne sur Juda\FTNTT{1 R. 22:50 ; 2 R. 8:16-19}}
\VerseOne{}Puis Josaphat s'endormit avec ses pères, et il fut enseveli avec eux dans la cité de David. Et Joram, son fils, régna à sa place\FTNT{1 R. 22:51.}.
\VS{2}Il avait des frères, fils de Josaphat : Azaria, Jehiel, Zacharie, Azaria, Micaël et Schephathia. Tous ceux-là étaient fils de Josaphat, roi d'Israël.
\VS{3}Leur père leur avait fait de grands dons d'argent, d'or et de choses précieuses, avec des villes fortes en Juda ; mais il avait donné le royaume à Joram, parce qu'il était le premier-né.
\VS{4}Quand Joram fut élevé sur le royaume de son père, et s'y fut fortifié, il tua avec l'épée tous ses frères, et quelques-uns aussi des chefs d'Israël.
\VS{5}Joram était âgé de trente-deux ans quand il devint roi, et il régna huit ans à Jérusalem.
\VS{6}Il marcha dans la voie des rois d'Israël, comme avait fait la maison d'Achab ; car la fille d'Achab était sa femme, et il fit ce qui est mal aux yeux de Yahweh.
\VS{7}Toutefois, Yahweh, à cause de l'alliance qu'il avait traitée avec David, ne voulut pas détruire la maison de David, selon qu'il avait dit qu'il lui donnerait une lampe, à lui et à ses fils, pour toujours.
\TextTitle{Rébellion d'Edom et de Libna\FTNTT{1 R. 8:20-23}}
\VS{8}De son temps, Edom se rebella de l'autorité de Juda, et établit un roi sur lui\FTNT{2 R. 8:20-23}.
\VS{9}Joram se mit donc en marche avec ses chefs et tous ses chars ; et s'étant levé de nuit, il battit les Edomites qui l'entouraient, et tous les chefs des chars.
\VS{10}Néanmoins, Edom se rebella contre l'autorité de Juda jusqu'à ce jour. En ce même temps, Libna se rebella aussi contre son autorité, parce qu'il avait abandonné Yahweh, le Dieu de ses pères.
\VS{11}Il fit aussi des hauts lieux dans les montagnes de Juda ; il fit que les habitants de Jérusalem se prostituèrent, et il y entraîna ceux de Juda.
\TextTitle{Elie prononce un jugement sur Joram}
\VS{12}Alors il lui vint un écrit de la part d'Elie, le prophète, disant : Ainsi parle Yahweh, le Dieu de David, ton père : Parce que tu n'as point suivi le chemin de Josaphat, ton père, ni celui d'Asa, roi de Juda,
\VS{13}mais que tu as suivi les voies des rois d'Israël, et que tu as poussé à la prostitution Juda et les habitants de Jérusalem, comme s'est prostituée la maison d'Achab, et que tu as tué tes frères, meilleurs que toi, la maison même de ton père ;
\VS{14}voici, Yahweh frappera d'une grande plaie ton peuple, tes fils, tes femmes et tous tes biens.
\VS{15}Et toi, tu auras une grosse maladie, une maladie d'entrailles ; jusqu'à ce que, de jour en jour, tes entrailles sortent par la force de la maladie.
\TextTitle{Yahweh excite les Philistins et les Arabes contre Joram}
\VS{16}Yahweh souleva contre Joram l'esprit des Philistins et des Arabes, qui habitent près des Ethiopiens.
\VS{17}Ils montèrent donc contre Juda, et firent une brèche pour piller toutes les richesses qui furent trouvées dans la maison du roi ; et même, ils emmenèrent captifs ses fils et ses femmes, de sorte qu'il ne lui resta d'autre fils que Joachaz, le plus jeune de ses fils.
\TextTitle{Mort de Joram}
\VS{18}Après tout cela, Yahweh frappa ses entrailles d'une maladie sans remède.
\VS{19}Elle s'avança chaque jour, et vers la fin de la seconde année, ses entrailles sortirent par la force de son mal, et il mourut dans de grandes souffrances. Son peuple ne brûla pas sur lui de parfums, comme il l'avait fait pour ses pères.
\VS{20}Il était âgé de trente-deux ans quand il devint roi, et il régna huit ans à Jérusalem. Il s'en alla sans être regretté, et on l'ensevelit dans la cité de David, mais non dans les sépulcres des rois.
\Chap{22}
\TextTitle{Achazia règne sur Juda\FTNTT{2 R. 8:24-29}}
\VerseOne{}Les habitants de Jérusalem firent régner à sa place Achazia, le plus jeune de ses fils, parce que les troupes qui étaient venues au camp avec les Arabes avaient tué tous les plus âgés ; et Achazia, fils de Joram, roi de Juda, régna\FTNT{2 R. 8:24-29 ; 2 R. 9:16.}.
\VS{2}Achazia était âgé de quarante-deux ans quand il devint roi, et il régna un an à Jérusalem. Sa mère avait pour nom Athalie, fille d'Omri.
\VS{3}Il suivit aussi les voies de la maison d'Achab, car sa mère lui donnait des conseils impies.
\VS{4}Il fit donc ce qui est mal aux yeux de Yahweh, comme la maison d'Achab ; parce qu'ils furent ses conseillers après la mort de son père, pour sa ruine.
\TextTitle{Achazia livré aux mains de Jéhu\FTNTT{2 R. 8:28-29 ; 2 R. 9:1-30}}
\VS{5}Conduit par leurs conseils, il alla avec Joram, fils d'Achab, roi d'Israël, à la guerre à Ramoth de Galaad, contre Hazaël, roi de Syrie. Et les Syriens frappèrent Joram,
\VS{6}qui s'en retourna à Jizreel, pour guérir des blessures que les Syriens lui avaient faites à Rama, lorsqu'il faisait la guerre contre Hazaël, roi de Syrie. Azaria, fils de Joram, roi de Juda, descendit pour voir Joram, le fils d'Achab, à Jizreel, parce qu'il était malade.
\VS{7}Dieu fit pour sa ruine qu'Achazia vint auprès de Joram. En effet, quand il fut arrivé, il sortit avec Joram pour aller au-devant de Jéhu, fils de Nimschi, que Yahweh avait oint pour retrancher la maison d'Achab.
\VS{8}Et comme Jéhu faisait justice de la maison d'Achab\FTNT{2 R. 10:12-30.}, il trouva les chefs de Juda et les fils des frères d'Achazia, qui servaient Achazia, et il les tua.
\VS{9}Il chercha ensuite Achazia, qui s'était caché en Samarie. On le prit, et on l'amena vers Jéhu qui le fit mourir. Puis on l'ensevelit, car on dit : C'est le fils de Josaphat, qui cherchait Yahweh de tout son cœur. Et il n'y eut plus personne dans la maison d'Achazia qui fut capable de régner.
\TextTitle{Joas échappe au massacre de sa famille\FTNTT{2 R. 11:1-3}}
\VS{10}Or Athalie, mère d'Achazia, voyant que son fils était mort, se leva et fit périr toute la race royale de la maison de Juda\FTNT{1 R. 11:1-3.}.
\VS{11}Mais Joschabeath, fille du roi Joram, prit Joas, fils d'Achazia, en le dérobant d'entre les fils du roi qu'on faisait mourir. Elle le mit avec sa nourrice dans la salle des lits. Ainsi Joschabeath, fille du roi Joram et femme de Jehojada le sacrificateur, étant la sœur d'Achazia, le cacha de la vue d'Athalie, qui ne put le faire mourir.
\VS{12}Il fut ainsi caché avec eux dans la maison de Dieu six ans ; et c'est Athalie qui régnait sur le pays.
\Chap{23}
\TextTitle{Joas devient roi grâce à Jehojada\FTNTT{2 R. 11:4-12}}
\VerseOne{}Mais la septième année, Jehojada prit courage et traita alliance avec les chefs de centaines, Azaria, fils de Jerocham, Ismaël, fils de Jochanan, Azaria, fils d'Obed, Maaséja, fils d'Adaja, et Elischaphath, fils de Zicri.
\VS{2}Ils firent le tour de Juda, pour rassembler de toutes les villes de Juda les Lévites et les chefs des pères d'Israël ; puis ils vinrent à Jérusalem.
\VS{3}Et toute cette assemblée traita alliance avec le roi dans la maison de Dieu. Jehojada leur dit : Voici, c'est le fils du roi qui régnera, selon la parole de Yahweh au sujet des fils de David.
\VS{4}Vous ferez donc ceci : Le tiers qui parmi vous entre en service au sabbat, sacrificateurs et Lévites, fera la garde des seuils.
\VS{5}Un autre tiers se tiendra dans la maison du roi, et un tiers à la porte de Jesod ; et tout le peuple sera dans les parvis de la maison de Yahweh.
\VS{6}Que personne n'entre dans la maison de Yahweh, sauf les sacrificateurs et les Lévites de service : Ils entreront car ils sont sanctifiés ; et tout le reste du peuple gardera les ordres de Yahweh.
\VS{7}Les Lévites environneront le roi de toutes parts, tenant chacun leurs armes à la main, et donneront la mort à quiconque voudra entrer dans la maison ; vous serez avec le roi quand il entrera et quand il sortira.
\VS{8}Les Lévites et tout Juda firent tout ce que Jehojada le sacrificateur, avait ordonné. Ils prirent chacun leurs gens, tant ceux qui entraient en service que ceux qui en sortaient au sabbat ; car Jehojada, le sacrificateur, n'avait exempté aucune classe.
\VS{9}Et Jehojada le sacrificateur, donna aux chefs de centaines les lances, les grands et les petits boucliers qui provenaient du roi David, et qui étaient dans la maison de Dieu.
\VS{10}Puis il rangea tout le peuple autour du roi, chacun tenant ses armes à la main, du côté droit du temple jusqu'au côté gauche de la maison, près de l'autel et de la maison.
\VS{11}Alors ils firent sortir le fils du roi, et mirent sur lui la couronne et le témoignage. Ils l'établirent roi, et Jehojada et ses fils l'oignirent et dirent : Vive le roi !
\TextTitle{Mort d'Athalie\FTNTT{2 R. 11:13-16}}
\VS{12}Mais Athalie, entendant le bruit du peuple qui courait et célébrait le roi, vint vers le peuple, dans la maison de Yahweh.
\VS{13}Elle regarda, et voici, le roi se tenait près de la colonne, à l'entrée ; les chefs et les trompettes étaient près du roi, et tout le peuple du pays était dans la joie, et l'on sonnait des trompettes ; les chantres, avec des instruments de musique, dirigeaient les chants de louanges. Alors Athalie déchira ses vêtements et dit : Conspiration ! Conspiration !
\VS{14}Le sacrificateur Jehojada fit sortir les chefs de centaines qui étaient à la tête de l'armée, et leur dit : Faites-la sortir hors des rangs, et que celui qui la suivra soit mis à mort par l'épée ! Car le sacrificateur avait dit : Ne la mettez pas à mort dans la maison de Yahweh.
\VS{15}Ils mirent donc la main sur elle pour la faire entrer dans la maison du roi, par l'entrée de la porte des chevaux ; et là ils la firent mourir.
\TextTitle{Jehojada fait asseoir Joas sur le trône de Juda\FTNTT{2 R. 11:17-20}}
\VS{16}Puis Jehojada traita, avec tout le peuple et le roi, une alliance pour être le peuple de Yahweh.
\VS{17}Et tout le peuple entra dans la maison de Baal pour la détruire. Ils brisèrent ses autels et ses images et ils tuèrent devant les autels Matthan, sacrificateur de Baal.
\VS{18}Jehojada remit aussi les fonctions de la maison de Yahweh entre les mains des sacrificateurs, des Lévites, comme David les avait répartis dans la maison de Yahweh, afin qu'ils élèvent des holocaustes à Yahweh, comme cela est écrit dans la loi de Moïse, avec joie et avec des chants, selon les ordonnances de David.
\VS{19}Il établit aussi les portiers aux portes de la maison de Yahweh, afin qu'il n'y entrât aucune personne souillée de quelque manière que ce fût.
\VS{20}Il prit les chefs de centaines, hommes considérés, qui avaient de l'autorité parmi le peuple, et tout le peuple du pays. Il fit descendre le roi, de la maison de Yahweh à la maison du roi, en entrant par la porte supérieure ; et ils firent asseoir le roi sur le trône royal.
\VS{21}Alors tout le peuple du pays se réjouit, et la ville fut tranquille, bien qu'on eût mis à mort Athalie par l'épée.
\Chap{24}
\TextTitle{Joas règne sur Juda ; ses travaux sur le temple\FTNTT{2 R. 11:21-12:8}}
\VerseOne{}Joas était âgé de sept ans quand il devint roi, et il régna quarante ans à Jérusalem. Sa mère avait pour nom Tsibja, de Beer-Schéba\FTNT{2 R. 11:21 ; 2 R. 12:1-3.}.
\VS{2}Joas fit ce qui est droit aux yeux de Yahweh, pendant toute la vie de Jehojada, le sacrificateur.
\VS{3}Et Jehojada prit pour lui deux femmes, dont il eut des fils et des filles.
\VS{4}Après cela Joas eut la pensée de renouveler la maison de Yahweh\FTNT{2 R. 12:4-16.}.
\VS{5}Il assembla donc les sacrificateurs et les Lévites, et leur dit : Allez vers les villes de Juda, et recueillez de l'argent par tout Israël, suffisamment pour réparer la maison de votre Dieu d'année en année, et hâtez-vous pour cette affaire. Mais les Lévites ne se hâtèrent point.
\VS{6}Alors le roi appela Jehojada, leur chef, et lui dit : Pourquoi n'as-tu pas veillé à ce que les Lévites aient apporté de Juda et de Jérusalem, l'impôt sur l'assemblée d'Israël, selon Moïse, serviteur de Yahweh, pour la tente du témoignage ?
\VS{7}Car l'impie Athalie et ses fils ont ravagé la maison de Dieu ; et même ils ont employé pour les Baals toutes les choses consacrées à la maison de Yahweh.
\TextTitle{Offrandes volontaires pour la réparation du temple\FTNTT{2 R. 12:9-16}}
\VS{8}Et le roi ordonna qu'on fasse un seul coffre, et qu'on le mette à la porte de la maison de Yahweh, à l'extérieur.
\VS{9}Puis on publia dans Juda et dans Jérusalem, pour qu'on qu'on apportât à Yahweh l'impôt mis par Moïse, serviteur de Dieu, sur Israël dans le désert.
\VS{10}Tous les chefs et tout le peuple s'en réjouirent, et l'on apporta et jeta le tribut dans le coffre, jusqu'à ce qu'il fût plein.
\VS{11}Au moment venu, les Lévites apportaient le coffre aux inspecteurs du roi, car ceux-ci voyaient qu'il y avait beaucoup d'argent. Un secrétaire du roi et un commissaire du souverain sacrificateur venaient et vidaient le coffre ; puis ils le rapportaient et le remettaient à sa place. Ils faisaient ainsi jour après jour, et ils recueillaient de l'argent en abondance.
\VS{12}Le roi et Jehojada le donnaient à ceux qui étaient chargés de l'ouvrage pour le service de la maison de Yahweh, et ceux-ci engageaient des tailleurs de pierres et des charpentiers pour réparer la maison de Yahweh, et aussi des ouvriers pour le fer et l'airain, afin de réparer la maison de Yahweh.
\VS{13}Ceux qui étaient chargés de l'ouvrage travaillèrent donc ; et par leurs mains, les travaux s'exécutèrent, de sorte qu'ils rétablirent la maison de Dieu en son état, et l'affermirent.
\VS{14}Lorsqu'ils eurent achevé, ils apportèrent devant le roi et devant Jehojada le reste de l'argent ; et on fit faire des ustensiles pour la maison de Yahweh, des ustensiles pour le service et pour les holocaustes, des coupes et d'autres ustensiles d'or et d'argent. Et on offrit continuellement des holocaustes dans la maison de Yahweh, tant que vécut Jehojada.
\TextTitle{Mort de Jehojada, Joas abandonne Yahweh\FTNTT{2 R. 12:9-16}}
\VS{15}Or Jehojada devint vieux et rassasié de jours, et il mourut ; il était âgé de cent trente ans quand il mourut.
\VS{16}On l'ensevelit dans la cité de David avec les rois ; car il avait fait du bien à Israël, et à l'égard de Dieu et de sa maison.
\VS{17}Mais, après la mort de Jehojada, les chefs de Juda vinrent et se prosternèrent devant le roi ; et le roi les écouta.
\VS{18}Ils abandonnèrent la maison de Yahweh, le Dieu de leurs pères, et ils servirent les idoles d'Astarté et les faux dieux ; et la colère de Yahweh fut sur Juda et sur Jérusalem, parce qu'ils s'étaient ainsi rendus coupables.
\VS{19}Yahweh envoya parmi eux des prophètes, pour les faire retourner à lui par leurs avertissements ; mais ils ne voulurent point les écouter.
\VS{20}Alors l'Esprit de Dieu revêtit Zacharie, fils de Jehojada, le sacrificateur, et se tenant devant le peuple, il leur dit : Dieu m'a parlé ainsi : Pourquoi transgressez-vous les commandements de Yahweh ? Vous ne prospérerez point ; car vous avez abandonné Yahweh, et il vous abandonnera aussi.
\VS{21}Mais ils se liguèrent contre lui et le lapidèrent, par ordre du roi, dans le parvis de la maison de Yahweh.
\VS{22}Ainsi le roi Joas ne se souvint point de la bonté dont Jehojada, père de Zacharie, avait usé envers lui ; et il tua son fils, qui dit en mourant : Yahweh le voit, et il en demandera compte !
\TextTitle{Invasion des Syriens, conspiration et mort de Joas\FTNTT{2 R. 12:17-21 ; cp. 2 R. 13:7}}
\VS{23}A la fin de cette année-là, l'armée de Syrie monta contre Joas, et vint en Juda et à Jérusalem. Ils détruisirent, d'entre le peuple, tous les chefs du peuple, et ils envoyèrent au roi de Damas tout leur butin.
\VS{24}Et quoique l'armée venue de Syrie fût peu nombreuse, Yahweh livra entre leurs mains une armée très nombreuse, parce qu'ils avaient abandonné Yahweh, le Dieu de leurs pères. Ainsi les Syriens furent le châtiment de Joas.
\VS{25}Quand ils s'éloignèrent de lui, après l'avoir laissé dans de grandes souffrances, ses serviteurs conspirèrent contre lui, à cause du sang des fils de Jehojada, le sacrificateur ; ils le tuèrent sur son lit, et il mourut. On l'ensevelit dans la cité de David, mais on ne l'ensevelit pas dans les sépulcres des rois.
\VS{26}Ce sont ici ceux qui conspirèrent contre lui : Zabad, fils de Schimeath, femme ammonite, et Jozabad, fils de Schimrith, femme Moabite.
\VS{27}Quant à ses fils et à la grande charge qui reposa sur lui, et à la réparation de la maison de Dieu, voici, ces choses sont écrites dans les mémoires du livre des rois. Amatsia, son fils, régna à sa place.
\Chap{25}
\TextTitle{Amatsia règne sur Juda\FTNTT{2 R. 12:21 ; 2 R. 14:1-6}}
\VerseOne{}Amatsia devint roi à l'âge de vingt-cinq ans, et il régna vingt-neuf ans à Jérusalem. Sa mère avait pour nom Joaddan, de Jérusalem\FTNT{2 R. 12:21 ; 2 R. 14:1-20}.
\VS{2}Il fit ce qui est droit aux yeux de Yahweh, mais non d'un cœur entier.
\VS{3}Après qu'il fut affermi dans son règne, il fit mourir ses serviteurs qui avaient tué le roi, son père.
\VS{4}Mais il ne fit pas mourir leurs fils ; car il fit selon ce qui est écrit dans la loi, dans le livre de Moïse, où Yahweh a donné ce commandement : Les pères ne mourront point pour les fils, et les fils ne mourront point pour les pères ; mais chacun mourra pour son péché.
\TextTitle{Amatsia en guerre contre les Edomites, sa victoire\FTNTT{2 R. 14:7}}
\VS{5}Puis Amatsia rassembla ceux de Juda, et il les rangea selon les maisons paternelles, par chefs de milliers et par chefs de centaines, pour tout Juda et Benjamin ; il en fit le dénombrement depuis l'âge de vingt ans et au-dessus ; et il trouva trois cent mille hommes d'élite, propres à l'armée, maniant la lance et le bouclier.
\VS{6}Il prit aussi à sa solde, pour cent talents d'argent, cent mille vaillants hommes de guerre d'Israël.
\VS{7}Mais un homme de Dieu vint à lui, et lui dit : Ô roi ! Que l'armée d'Israël ne marche point avec toi ; car Yahweh n'est point avec Israël ni avec tous ces fils d'Ephraïm.
\VS{8}Si tu vas avec eux, quand bien même tu ferais de vaillants combats, Dieu te fera tomber devant l'ennemi ; car Dieu a la puissance d'aider et de faire tomber.
\VS{9}Amatsia dit à l'homme de Dieu : Mais que faire des cent talents que j'ai donnés à la troupe d'Israël ? L'homme de Dieu dit : Yahweh peut t'en donner beaucoup plus.
\VS{10}Ainsi Amatsia sépara les troupes qui lui étaient venues d'Ephraïm, et les fit retourner chez elles ; mais leur colère s'enflamma très ardemment contre Juda, et ces gens retournèrent chez eux dans une grande colère.
\VS{11}Alors Amatsia prit courage, conduisit son peuple et s'en alla dans la vallée du sel, où il battit dix mille hommes des fils de Séir.
\VS{12}Les fils de Juda prirent dix mille hommes vivants, et les ayant amenés sur le sommet d'une roche, ils les jetèrent du haut de la roche, de sorte qu'ils furent tous brisés.
\VS{13}Mais les gens de la troupe qu'Amatsia avait renvoyée, afin qu'ils n'aillent pas avec lui à la guerre, firent une incursion dans les villes de Juda, depuis Samarie jusqu'à Beth-Horon. Ils y tuèrent trois mille personnes et emportèrent un gros butin.
\TextTitle{Idolâtrie d'Amatsia\FTNTT{2 R. 14:7}}
\VS{14}Lorsqu'Amatsia fut de retour de la défaite des Edomites, et ayant apporté les dieux des fils de Séir, il se les établit pour dieux ; il se prosterna devant eux et leur brûla de l'encens.
\VS{15}Et la colère de Yahweh s'enflamma contre Amatsia, et il envoya vers lui un prophète qui lui dit : Pourquoi as-tu recherché les dieux d'un peuple qui n'ont point délivré leur peuple de ta main ?
\VS{16}Et comme il parlait au roi, il lui répondit : T'a-t-on établi conseiller du roi ? Cesse maintenant ! Pourquoi veux-tu qu'on te tue ? Et le prophète se retira, mais en disant : Je sais que Dieu a résolu de te détruire, parce que tu as fait cela, et que tu n'as point écouté mon conseil.
\TextTitle{Défaite d'Amatsia contre Israël\FTNTT{2 R. 14:8-14}}
\VS{17}Puis Amatsia, roi de Juda, ayant tenu conseil, envoya vers Joas, fils de Joachaz, fils de Jéhu, roi d'Israël, pour lui dire : Viens, voyons-nous en face !
\VS{18}Mais Joas, roi d'Israël, envoya dire à Amatsia, roi de Juda : L'épine du Liban envoya dire au cèdre du Liban : Donne ta fille pour femme à mon fils ! Et les bêtes sauvages qui sont au Liban passèrent et foulèrent l'épine.
\VS{19}Voici, tu dis que tu as frappé les Edomites, et ton cœur s'est élevé pour te glorifier. Maintenant, reste dans ta maison ! Pourquoi t'engagerais-tu dans un combat où tu tomberais, et Juda avec toi ?
\VS{20}Mais Amatsia ne l'écouta point ; Dieu avait résolu de le livrer aux mains de Joas parce qu'il eût recours aux dieux d'Edom.
\VS{21}Joas, roi d'Israël, monta ; et ils se virent en face, lui et Amatsia, roi de Juda, à Beth-Schémesch, qui est de Juda.
\VS{22}Juda fut battu en face d'Israël, et chacun s'enfuit dans sa tente.
\VS{23}Joas, roi d'Israël, prit Amatsia, roi de Juda, fils de Joas, fils de Joachaz, à Beth-Schémesch. Il l'emmena à Jérusalem et fit une brèche de quatre cents coudées dans la muraille de Jérusalem, depuis la porte d'Ephraïm jusqu'à la porte de l'angle.
\VS{24}Il prit l'or, l'argent, tous les vases qui se trouvaient dans la maison de Dieu sous la garde d'Obed-Edom, les trésors de la maison du roi ; il fit des otages et il retourna à Samarie.
\TextTitle{Assassinat d'Amatsia\FTNTT{2 R. 14:17-20}}
\VS{25}Amatsia, fils de Joas, roi de Juda, vécut quinze ans, après que Joas, fils de Joachaz, roi d'Israël, mourut.
\VS{26}Le reste des actions d'Amatsia, les premières et les dernières, voici cela n'est-il pas écrit dans le livre des rois de Juda et d'Israël ?
\VS{27}Or depuis le moment où Amatsia se détourna de Yahweh, on fit une conspiration contre lui à Jérusalem, et il s'enfuit à Lakis ; mais on le poursuivit à Lakis, et on le fit mourir.
\VS{28}Puis on le transporta sur des chevaux, et on l'ensevelit avec ses pères dans la ville de Juda.
\Chap{26}
\TextTitle{Ozias règne sur Juda ; il est fidèle à Yahweh\FTNTT{2 R. 14:21-15:4}}
\VerseOne{}Alors, tout le peuple de Juda prit Ozias, âgé de seize ans, et l'établit roi à la place de son père Amatsia\FTNT{2 R. 14:21 ; 2 R. 15:1-4.}.
\VS{2}Ce fut lui qui bâtit Eloth, et la ramena sous la puissance de Juda, après que le roi se fut endormi avec ses pères.
\VS{3}Ozias était âgé de seize ans quand il devint roi, et il régna cinquante-deux ans à Jérusalem. Sa mère avait pour nom Jecolia, de Jérusalem.
\VS{4}Il fit ce qui est droit aux yeux de Yahweh, comme avait fait Amatsia, son père.
\VS{5}Il s'appliqua à rechercher Dieu pendant les jours de Zacharie, qui avait une intelligence dans les visions de Dieu et pendant les jours où il rechercha Yahweh, Dieu le fit prospérer.
\VS{6}Il sortit et fit la guerre contre les Philistins. Il brisa la muraille de Gath, la muraille de Jabné, et la muraille d'Asdod ; et il bâtit des villes dans le pays d'Asdod et chez les Philistins.
\VS{7}Dieu le secourut contre les Philistins et contre les Arabes qui habitaient à Gur-Baal, et contre les Maonites.
\VS{8}Même les Ammonites faisaient des présents à Ozias, et sa renommée parvint jusqu'à l'entrée de l'Egypte ; car il était devenu très puissant.
\VS{9}Ozias bâtit des tours à Jérusalem, sur la porte de l'angle, sur la porte de la vallée, sur l'angle, et il les fortifia.
\VS{10}Il bâtit des tours dans le désert, et il creusa de nombreux puits, parce qu'il avait de nombreux troupeaux dans la plaine et dans la campagne, des laboureurs et des vignerons sur les montagnes, et au Carmel ; car il aimait l'agriculture.
\VS{11}Ozias avait une armée de gens de guerre, allant à la guerre par bandes, selon le compte de leur dénombrement fait par Jeïel le scribe, et Maaséja le commissaire, et sous la conduite de Hanania l'un des chefs du roi.
\VS{12}Le nombre total des chefs des maisons paternelles, des vaillants guerriers, était de deux mille six cents.
\VS{13}Il y avait sous leur conduite une armée de trois cent sept mille cinq cents combattants, tous gens de guerre, puissants et vaillants, capables de soutenir le roi contre l'ennemi.
\VS{14}Ozias leur procura, pour toute l'armée, des boucliers, des lances, des casques, des cuirasses, des arcs et des pierres de fronde.
\VS{15}Il fit faire à Jérusalem des machines inventées par un ingénieur, pour être placées sur les tours et sur les angles, pour lancer des flèches et de grosses pierres. Et sa renommée se répandit au loin ; car il fut extraordinairement soutenu, jusqu'à ce qu'il devienne fort puissant.
\TextTitle{Ozias pèche et est frappé de lèpre\FTNTT{2 R. 15:5-7, 32}}
\VS{16}Mais dès qu'il fut puissant, son cœur s'éleva pour le corrompre. Et il pécha contre Yahweh, son Dieu : Il entra dans le temple de Yahweh pour brûler des parfums sur l'autel des parfums\FTNT{2 R. 15:5-7.}.
\VS{17}Mais Azaria le sacrificateur, entra après lui, et avec lui quatre-vingts sacrificateurs de Yahweh, hommes vaillants,
\VS{18}qui s'opposèrent au roi Ozias, et lui dirent : Ce n'est pas à toi, Ozias, d'offrir le parfum à Yahweh, mais aux sacrificateurs, fils d'Aaron, qui sont consacrés pour cela. Sors du sanctuaire, car tu as péché ! Et cela ne sera pas à ta gloire devant Yahweh Dieu.
\VS{19}Alors Ozias, qui avait à la main un encensoir pour faire brûler le parfum, se mit en colère. Et comme il s'irritait contre les sacrificateurs, la lèpre parut sur son front, en présence des sacrificateurs, dans la maison de Yahweh, près de l'autel des parfums.
\VS{20}Azaria, le principal sacrificateur, le regarda ainsi que tous les sacrificateurs. Et voici, il avait de la lèpre sur le front. Ils le pressèrent et lui-même se hâta de sortir, parce que Yahweh l'avait frappé.
\VS{21}Le roi Ozias fut ainsi lépreux jusqu'au jour de sa mort ; il habita seul comme lépreux dans une maison écartée, car il était exclu de la maison de Yahweh. Et Jotham, son fils, avait la charge de la maison du roi, jugeant le peuple du pays.
\VS{22}Esaïe, fils d'Amots, le prophète, a écrit le reste des actions d'Ozias, les premières et les dernières.
\VS{23}Ozias s'endormit avec ses pères, et on l'ensevelit avec ses pères dans le champ de la sépulture des rois ; car on dit : Il est lépreux. Et Jotham, son fils, régna à sa place.
\Chap{27}
\TextTitle{Jotham règne sur Juda ; sa mort\FTNTT{2 R. 15:7, 32-38}}
\VerseOne{}Jotham était âgé de vingt-cinq ans quand il devint roi, et il régna seize ans à Jérusalem\FTNT{1 R. 15:7.}. Sa mère avait le nom de Jeruscha, fille de Tsadok.
\VS{2}Il fit ce qui est droit aux yeux de Yahweh, tout comme Ozias, son père, avait fait ; mais il n'entra pas dans le temple de Yahweh. Néanmoins, le peuple se corrompait encore.
\VS{3}Ce fut lui qui bâtit la porte supérieure de la maison de Yahweh, et il fit beaucoup de constructions sur les murs de la colline.
\VS{4}Il bâtit des villes sur les montagnes de Juda, des châteaux et des tours dans les forêts.
\VS{5}Il fut en guerre avec le roi des fils d'Ammon, et fut le plus fort. Cette année-là, les fils d'Ammon lui donnèrent cent talents d'argent, dix mille cors de froment, et dix mille d'orge. Les fils d'Ammon lui en donnèrent autant la seconde et la troisième année.
\VS{6}Jotham devint donc très puissant, parce qu'il avait affermi ses voies devant Yahweh, son Dieu.
\VS{7}Le reste des actions de Jotham, tous ses combats et sa conduite, voici, toutes ces choses sont écrites dans le livre des rois d'Israël et de Juda.
\VS{8}Il était âgé de vingt-cinq ans quand il devint roi, et il régna seize ans à Jérusalem.
\VS{9}Puis Jotham s'endormit avec ses pères, et on l'ensevelit dans la cité de David. Et Achaz, son fils, régna à sa place.
\Chap{28}
\TextTitle{Achaz règne sur Juda\FTNTT{2 R. 15:38-16:4}}
\VerseOne{}Achaz était âgé de vingt ans quand il devint roi, et il régna seize ans à Jérusalem\FTNT{2 R. 15:38 ; 2 R. 16:1-4.}. Il ne fit point ce qui est droit aux yeux de Yahweh, comme David, son père.
\VS{2}Il suivit la voie des rois d'Israël ; et il fit même des images de fonte pour les Baals.
\VS{3}Il brûla des parfums dans la vallée du fils de Hinnom, et il brûla ses fils au feu, suivant les abominations des nations que Yahweh avait chassées devant les enfants d'Israël.
\VS{4}Il offrait aussi des sacrifices et brûlait des parfums dans les hauts lieux, sur les collines, et sous tout arbre vert.
\TextTitle{La Syrie et Israël envahissent Juda\FTNTT{2 R. 16:5-6}}
\VS{5}C'est pourquoi Yahweh, son Dieu, le livra entre les mains du roi de Syrie. Les Syriens le battirent et lui prirent un grand nombre de prisonniers, qu'ils emmenèrent à Damas. Il fut livré aussi entre les mains du roi d'Israël, qui lui fit endurer une grande défaite\FTNT{2 R . 16:5-20.}.
\VS{6}Car Pékach, fils de Remalia, tua en un seul jour en Juda cent vingt mille hommes, tous vaillants, parce qu'ils avaient abandonné Yahweh, le Dieu de leurs pères.
\VS{7}Zicri, homme vaillant d'Ephraïm, tua Maaséja, fils du roi, et Azrikam, chef de la maison, et Elkana, le second après le roi.
\VS{8}Les fils d'Israël emmenèrent prisonniers deux cent mille de leurs frères, tant femmes que fils et filles ; ils firent aussi sur eux un gros butin. Ils emmenèrent le butin à Samarie.
\TextTitle{Les captifs de Juda libérés grâce à Obed}
\VS{9}Or il y avait un prophète de Yahweh nommé Oded. Il sortit au-devant de cette armée qui revenait à Samarie, et leur dit : Voici, Yahweh, le Dieu de vos pères, étant indigné contre Juda, les a livrés entre vos mains, et vous les avez tués avec une colère telle qu'elle est parvenue aux cieux.
\VS{10}Et maintenant, vous pensez assujettir les fils de Juda et de Jérusalem pour serviteurs et pour servantes ! Mais n'êtes-vous pas également coupables envers Yahweh, votre Dieu ?
\VS{11}Maintenant écoutez-moi, et ramenez les prisonniers que vous vous êtes faits parmi vos frères ; car la colère ardente de Yahweh est sur vous.
\VS{12}Alors quelques-uns des chefs des fils d'Ephraïm, Azaria, fils de Jochanan, Bérékia, fils de Meschillémoth, Ezéchias, fils de Schallum, et Amasa, fils de Hadlaï, s'élevèrent contre ceux qui retournaient de la guerre,
\VS{13}et leur dirent : Vous ne ferez point entrer ici ces captifs. C'est pour nous rendre coupables devant Yahweh, voulez-vous en rajouter à nos péchés et à notre culpabilité ; car nous sommes déjà grandement coupables, et une colère ardente est sur Israël.
\VS{14}Alors les soldats abandonnèrent les captifs et le butin devant les chefs et toute l'assemblée.
\VS{15}Et des hommes, désignés par leurs noms, se levèrent, prirent les captifs, utilisèrent le butin pour revêtir tous ceux d'entre eux qui étaient nus avec des vêtements et des chaussures. Ils leur donnèrent à manger et à boire, les oignirent et ils conduisirent sur des ânes tous ceux qui étaient affaiblis pour les emmener à Jéricho, la ville des palmiers, auprès de leurs frères ; puis ils s'en retournèrent à Samarie.
\TextTitle{Achaz fait appel aux Assyriens\FTNTT{2 R. 15:29 ; 16:7-18}}
\VS{16}En ce temps-là, le roi Achaz envoya demander du secours aux rois d'Assyrie.
\VS{17}Les Edomites étaient revenus, avaient battu Juda et avaient emmené des prisonniers.
\VS{18}Les Philistins s'étaient aussi jetés sur les villes de la plaine et du sud de Juda ; et ils avaient pris Beth-Schémesch, Ajalon, Guedéroth, Soco et les villes de son ressort, Thimna et les villes de son ressort, Guimzo et les villes de son ressort, et ils y demeurèrent.
\VS{19}Car Yahweh humilia Juda, à cause d'Achaz, roi d'Israël, parce qu'il avait mis le désordre en Juda, et qu'il avait commis des transgressions contre Yahweh.
\VS{20}Tilgath-Pilnéser, roi d'Assyrie, vint vers lui ; mais il l'assiégea, et ne le fortifia pas.
\VS{21}Or Achaz dépouilla la maison de Yahweh, la maison du roi et celle des chefs, pour faire des dons au roi d'Assyrie, mais sans avoir du secours.
\TextTitle{Achaz irrite Yahweh par ses péchés}
\VS{22}Dans le temps de sa détresse, il continua à pécher contre Yahweh, lui, le roi Achaz.
\VS{23}Il sacrifia aux dieux de Damas qui l'avaient battu, et il dit : Puisque les dieux des rois de Syrie leur viennent en aide, je leur sacrifierai, afin qu'ils me viennent en aide. Mais ils furent la cause de sa chute et de celle de tout Israël.
\VS{24}Or Achaz rassembla les ustensiles de la maison de Dieu, et il mit en pièces les ustensiles de la maison de Dieu. Il ferma les portes de la maison de Yahweh, et se fit des autels dans tous les coins de Jérusalem.
\VS{25}Il fit des hauts lieux dans chaque ville de Juda, pour offrir des parfums à d'autres dieux ; et il irrita Yahweh, le Dieu de ses pères.
\TextTitle{Mort d'Achaz\FTNTT{2 R. 16:19-20}}
\VS{26}Quant au reste de ses actions et de toutes ses voies, les premières et les dernières, voici, elles sont écrites dans le livre des rois de Juda et d'Israël.
\VS{27}Puis Achaz s'endormit avec ses pères, et on l'ensevelit dans la ville de Jérusalem ; car on ne le mit point dans les sépulcres des rois d'Israël. Et Ezéchias, son fils, régna à sa place.
\Chap{29}
\TextTitle{Ezéchias règne sur Juda ; le réveil du peuple\FTNTT{2 R. 18:1-7 ; cp. Es. 36-39}}
\VerseOne{}Ezéchias devint roi à l'âge de vingt-cinq ans, et il régna vingt-neuf ans à Jérusalem\FTNT{ Es. 36 ; Es. 37 ; Es. 38, Es. 39 ; 2 R. 18:1-7 ;.}. Sa mère avait pour nom Abija, fille de Zacharie.
\VS{2}Il fit ce qui est droit aux yeux de Yahweh, tout comme avait fait David, son père.
\VS{3}La première année de son règne, au premier mois, il ouvrit les portes de la maison de Yahweh, et il les répara.
\VS{4}Il fit venir les sacrificateurs et les Lévites, et les rassembla dans la place orientale.
\VS{5}Et il leur dit : Ecoutez-moi, Lévites ! Sanctifiez-vous et sanctifiez la maison de Yahweh, le Dieu de vos pères, et ôtez du sanctuaire tout ce qui est impur.
\VS{6}Car nos pères ont péché, ils ont fait ce qui est mal aux yeux de Yahweh, notre Dieu. Ils l'ont abandonné, ils ont détourné leurs faces du tabernacle de Yahweh et lui ont tourné le dos.
\VS{7}Ils ont même fermé les portes du portique et ont éteint les lampes, ils n'ont fait ni monter d'offrande, ni brûler du parfum et des holocaustes au Dieu d'Israël dans le sanctuaire.
\VS{8}C'est pourquoi la colère de Yahweh a été sur Juda et sur Jérusalem ; et il les a livrés à de grands troubles, à la ruine et à la moquerie, comme vous le voyez de vos yeux.
\VS{9}Car voici, nos pères sont tombés par l'épée, et nos fils, nos filles et nos femmes sont en captivité.
\VS{10}Maintenant donc j'ai à cœur de traiter alliance avec Yahweh, le Dieu d'Israël, pour que son ardente colère se détourne de nous.
\VS{11}Or mes fils, cessez d'être négligents ; car Yahweh vous a choisis, afin que vous vous teniez devant lui à son service, comme ses serviteurs, pour lui brûler des parfums.
\VS{12}Les Lévites se levèrent : Machath, fils d'Amasaï, Joël, fils d'Azaria, des fils des Kehathites ; et des fils des Merarites, Kis, fils d'Abdi, Azaria, fils de Jehalléleel ; et des Guerschonites, Joach, fils de Zimma, et Eden, fils de Joach ;
\VS{13}et des fils d'Elitsaphan, Schimri et Jeïel ; et des fils d'Asaph, Zacharie et Matthania ;
\VS{14}et des fils d'Héman, Jehiel et Schimeï, et des fils de Jeduthun, Schemaeja et Uzziel.
\VS{15}Ils assemblèrent leurs frères, et ils se sanctifièrent ; puis ils entrèrent selon l'ordre du roi, et d'après la parole de Yahweh, pour purifier la maison de Yahweh.
\VS{16}Ainsi les sacrificateurs entrèrent à l'intérieur de la maison de Yahweh pour la purifier. Ils firent sortir dans le parvis de la maison de Yahweh toutes les impuretés qu'ils trouvèrent dans le temple de Yahweh. Les Lévites les prirent pour les emporter dehors, au torrent de Cédron.
\VS{17}Ils commencèrent à sanctifier le temple le premier jour du premier mois. Le huitième jour du mois, ils entrèrent au portique de Yahweh, et ils sanctifièrent la maison de Yahweh pendant huit jours. Le seizième jour du premier mois, ils avaient achevé.
\VS{18}Puis ils se rendirent chez le roi Ezéchias, et dirent : Nous avons purifié toute la maison de Yahweh, l'autel des holocaustes et ses ustensiles, la table des pains de proposition et ses ustensiles\FTNT{Ex. 29.}.
\VS{19}Nous avons remis en état et sanctifié tous les ustensiles que le roi Achaz avait rendus odieux pendant son règne, par ses transgressions ; ils sont maintenant devant l'autel de Yahweh.
\TextTitle{Nouvelle consécration du temple}
\VS{20}Alors le roi Ezéchias se leva de bonne heure, rassembla les chefs de la ville, et monta à la maison de Yahweh.
\VS{21}Ils amenèrent sept taureaux, sept béliers, sept agneaux et sept boucs sans défaut, en sacrifice pour le péché, pour le royaume, pour le sanctuaire et pour Juda\FTNT{Lé 4:3-26}. Puis le roi dit aux sacrificateurs, fils d'Aaron, de les faire monter en offrande sur l'autel de Yahweh.
\VS{22}Ils égorgèrent donc les bœufs, et les sacrificateurs recueillirent le sang et en aspergèrent l'autel ; ils égorgèrent les béliers et aspergèrent le sang sur l'autel ; ils égorgèrent les agneaux et aspergèrent le sang sur l'autel.
\VS{23}Puis on fit approcher les boucs pour le sacrifice du péché, devant le roi et devant l'assemblée, et ils posèrent leurs mains sur eux\FTNT{Lé 8:14.}.
\VS{24}Alors les sacrificateurs les égorgèrent, et offrirent en expiation leur sang vers l'autel, afin de faire propitiation pour tout Israël ; car le roi avait ordonné cet holocauste et ce sacrifice d'expiation pour tout Israël.
\VS{25}Il plaça aussi les Lévites dans la maison de Yahweh, avec des cymbales, des luths et des harpes, comme l'avait ordonné David, Gad, le voyant du roi, et Nathan le prophète ; car c'était un commandement de Yahweh, par ses prophètes.
\VS{26}Les Lévites se tinrent donc là avec les instruments de David, et les sacrificateurs avec les trompettes.
\VS{27}Alors Ezéchias ordonna de faire monter en offrande l'holocauste sur l'autel ; et au moment où commença l'holocauste, le cantique de Yahweh commença aussi, avec les trompettes et les instruments de David, roi d'Israël.
\VS{28}Toute l'assemblée se prosterna en chantant le cantique, et les trompettes sonnèrent ; et cela continua jusqu'à ce que l'holocauste fût achevé.
\VS{29}Et quand on eut achevé de faire monter l'holocauste, le roi et tous ceux qui se trouvaient avec lui fléchirent les genoux et se prosternèrent.
\VS{30}Puis le roi Ezéchias et les chefs dirent aux Lévites de célébrer Yahweh par les paroles de David et d'Asaph le voyant ; et ils le célébrèrent dans des réjouissances, et s'inclinèrent pour se prosterner.
\VS{31}Alors Ezéchias prit la parole, et dit : Vous avez maintenant consacré vos mains à Yahweh. Approchez-vous, amenez des sacrifices et faites des sacrifices de reconnaissances dans la maison de Yahweh. Et l'assemblée amena des sacrifices et firent des sacrifices de reconnaissances, et tous ceux qui étaient d'un cœur volontaire offrirent des holocaustes.
\VS{32}Le nombre des holocaustes que l'assemblée offrit fut de soixante-dix taureaux, cent béliers, deux cents agneaux, le tout en holocauste à Yahweh.
\VS{33}Et les autres choses consacrées furent, six cents bœufs, et trois brebis moutons.
\VS{34}Mais ils étaient peu de sacrificateurs et ne purent pas dépouiller tous les holocaustes ; les Lévites, leurs frères, les aidèrent jusqu'à ce que cette œuvre fut achevée, et jusqu'à ce que les autres sacrificateurs se fussent sanctifiés ; car les Lévites avaient eu plus à cœur de se sanctifier que les sacrificateurs.
\VS{35}Il y eut aussi un grand nombre d'holocaustes, avec les graisses des offrande de paix et avec les libations des holocaustes. Ainsi, le service de la maison de Yahweh fut rétabli.
\VS{36}Ezéchias et tout le peuple se réjouirent de ce que Dieu avait ainsi disposé le peuple ; car les choses se firent instantanément.
\Chap{30}
\TextTitle{Rétablissement de la Pâque}
\VerseOne{}Puis Ezéchias envoya dire à tout Israël et à Juda ; et il écrivit aussi des lettres à Ephraïm et à Manassé, pour les faire venir à la maison de Yahweh à Jérusalem, pour célébrer la Pâque en l'honneur de Yahweh, le Dieu d'Israël.
\VS{2}Le roi, ses chefs et toute l'assemblée avaient tenu un conseil à Jérusalem afin de célébrer la Pâque au second mois\FTNT{No. 9:10-11.} ;
\VS{3}car on ne pouvait la célébrer au temps ordinaire, parce qu'il n'y avait pas un nombre suffisant de sacrificateurs sanctifiés, et que le peuple n'était pas rassemblé à Jérusalem.
\VS{4}Le roi vit cela d'un bon œil ainsi que toute l'assemblée.
\VS{5}Ils décidèrent de faire une publication dans tout Israël, depuis Beer-Schéba jusqu'à Dan, pour que l'on vienne à Jérusalem célébrer la Pâque à Yahweh, le Dieu d'Israël. Car elle n'était pas célébrée par la multitude depuis longtemps conformément à ce qui était écrit.
\VS{6}Les coureurs allèrent donc avec des lettres de la part du roi et de ses chefs, partout en Israël et en Juda. Selon que le roi l'avait ordonné, ils disaient : Enfants d'Israël, retournez à Yahweh, le Dieu d'Abraham, d'Isaac et d'Israël, et afin qu'il revienne vers vous, qui êtes le reste échappé de la main des rois d'Assyrie.
\VS{7}Ne soyez pas comme vos pères ni comme vos frères, qui ont péché contre Yahweh, le Dieu de leurs pères, c'est pourquoi il les a livrés à la désolation, comme vous le voyez.
\VS{8}Maintenant, ne raidissez pas votre cou comme vos pères. Tendez les mains vers Yahweh, venez à son sanctuaire consacré pour toujours, servez Yahweh, votre Dieu, et son ardente colère se détournera de vous.
\VS{9}Car si vous revenez à Yahweh, vos frères et vos fils trouveront grâce auprès de ceux qui les ont emmenés captifs, et ils reviendront en ce pays parce que Yahweh, votre Dieu, est compatissant et miséricordieux ; et il ne détournera point sa face de vous, si vous revenez à lui.
\VS{10}Les coureurs passaient ainsi de ville en ville, par le pays d'Ephraïm et de Manassé jusqu'à Zabulon ; mais on riait et on se moquait d'eux.
\VS{11}Toutefois, quelques-uns d'Aser, de Manassé et de Zabulon s'humilièrent, et vinrent à Jérusalem.
\VS{12}La main de Dieu fut aussi sur Juda, pour leur donner un même cœur, afin d'exécuter l'ordre du roi et des chefs, selon la parole de Yahweh.
\VS{13}C'est pourquoi il s'assembla un grand peuple à Jérusalem pour célébrer la fête des pains sans levain\FTNT{Ex. 12:15 ; Lé. 23:6.}, au second mois. Ce fut une très grande assemblée.
\VS{14}Ils se levèrent et ôtèrent les autels qui étaient à Jérusalem ; ils ôtèrent aussi tous ceux où l'on brûlait de l'encens, et ils les jetèrent dans le torrent de Cédron.
\VS{15}Puis on immola la Pâque, au quatorzième jour du second mois ; car les sacrificateurs et les Lévites avaient eu honte et s'étaient sanctifiés, et ils amenèrent les holocaustes dans la maison de Yahweh.
\VS{16}Ils se tinrent à leur poste, selon leur charge, d'après la loi de Moïse, homme de Dieu. Et les sacrificateurs répandaient le sang qu'ils recevaient des mains des Lévites.
\VS{17}Car il y en avait un grand nombre dans cette assemblée qui ne s'étaient pas sanctifiés ; c'est pourquoi les Lévites eurent la charge d'immoler la Pâque pour tous ceux qui n'étaient pas purs, afin de les consacrer à Yahweh.
\VS{18}Car une grande partie du peuple, à savoir la plupart de ceux d'Ephraïm, de Manassé, d'Issacar et de Zabulon, ne s'étaient pas purifiés et mangèrent la Pâque contrairement à ce qui est écrit. Mais Ezéchias pria pour eux, en disant : Que Yahweh, qui est bon, tienne la propitiation pour faite,
\VS{19}pour quiconque a disposé son cœur à rechercher Dieu Yahweh, le Dieu de leurs pères, bien qu'il ne soit pas purifié conformément au sanctuaire !
\VS{20}Yahweh exauça Ezéchias, et guérit le peuple.
\VS{21}Les enfants d'Israël qui se trouvèrent à Jérusalem célébrèrent donc la fête des pains sans levain, pendant sept jours, dans une grande réjouissance ; les Lévites et les sacrificateurs célébraient Yahweh chaque jour, avec les instruments qui retentissaient à la louange de Yahweh.
\VS{22}Ezéchias parla au cœur de tous les Lévites, qui prêtaient une grande attention et de l'intelligence au service de Yahweh. Ils mangèrent pendant la fête, sept jours durant, offrant des sacrifices d'offrande de paix, et louant Yahweh, le Dieu de leurs pères.
\TextTitle{Sept jours supplémentaires pour la Pâque}
\VS{23}Puis toute l'assemblée fut d'avis de célébrer sept autres jours. Et ils célébrèrent ces sept jours dans la joie.
\VS{24}Car Ezéchias, roi de Juda, offrit à l'assemblée mille taureaux et sept mille brebis ; et les chefs donnèrent à l'assemblée mille taureaux et dix mille brebis ; et des sacrificateurs en grand nombre s'étaient sanctifiés.
\VS{25}Toute l'assemblée de Juda, avec les sacrificateurs et les Lévites, et toute l'assemblée venue d'Israël, ainsi que les étrangers venus du pays d'Israël, et ceux qui habitaient en Juda, se réjouirent.
\VS{26}Il y eut une grande joie à Jérusalem ; car depuis le temps de Salomon, fils de David, roi d'Israël, il ne s'était pas fait une telle chose dans Jérusalem.
\VS{27}Puis les sacrificateurs et les Lévites se levèrent et bénirent le peuple, et leur voix fut entendue, leur prière parvint jusqu'aux cieux, jusqu'à la sainte demeure de Yahweh.
\Chap{31}
\TextTitle{Destruction des idoles et organisation des services du temple}
\VerseOne{}Lorsque tout cela fut achevé, tous ceux d'Israël qui s'étaient retrouvés là, allèrent dans les villes de Juda, et brisèrent les statues, abattirent les idoles d'Astarté et renversèrent les hauts lieux et les autels, dans tout Juda et Benjamin, dans Ephraïm et Manassé, jusqu'à détruire tout\FTNT{2 R. 18:4.}. Puis tous les enfants d'Israël retournèrent dans leurs villes, chacun dans sa possession.
\VS{2}Et Ezéchias rétablit les classes des sacrificateurs et des Lévites, selon leur partage, chacun suivant sa charge, tant les sacrificateurs que les Lévites, pour les holocaustes et les offrandes de paix, pour faire le service, célébrer et chanter les louanges aux portes du camp de Yahweh.
\VS{3}Le roi donna une portion de ses biens pour les holocaustes, pour les holocaustes du matin et du soir, pour les holocaustes des sabbats, des nouvelles lunes et des fêtes, comme cela est écrit dans la loi de Yahweh.
\VS{4}Il dit au peuple, aux habitants de Jérusalem, de donner la portion des sacrificateurs et des Lévites, afin de s'appliquer à la loi de Yahweh.
\VS{5}Dès que la chose fut publiée, les enfants d'Israël amenèrent en abondance les prémices du blé, du moût, de l'huile, du miel et de tous les produits des champs ; ils apportèrent les dîmes de tout, en abondance.
\VS{6}Les enfants d'Israël et de Juda, qui demeuraient dans les villes de Juda, apportèrent aussi les dîmes du gros et du menu bétail, et les dîmes des choses saintes, qui étaient consacrées à Yahweh, leur Dieu ; et ils les mirent par tas.
\VS{7}Ils commencèrent à former les tas au troisième mois, et ils les achevèrent au septième mois.
\VS{8}Alors Ezéchias et les chefs vinrent voir les tas, et ils bénirent Yahweh et son peuple d'Israël.
\VS{9}Ezéchias interrogea les sacrificateurs et les Lévites au sujet de ces tas.
\VS{10}Le souverain sacrificateur Azaria, de la maison de Tsadok, lui répondit, et parla ainsi : Depuis qu'on a commencé à apporter des offrandes à la maison de Yahweh, nous avons mangé et avons été rassasiés, et il est resté cette grande quantité ; car Yahweh a béni son peuple, et cette grande quantité est le reste.
\VS{11}Alors Ezéchias leur dit de préparer des chambres dans la maison de Yahweh ; et ils les préparèrent.
\VS{12}On y apporta fidèlement les offrandes et les dîmes, les choses consacrées. Conania, le Lévite, en eut l'intendance, et Schimeï, son frère, était son second.
\VS{13}Jehiel, Azazia, Nachath, Asaël, Jerimoth, Jozabad, Eliel, Jismakia, Machath, et Benaja, étaient commis sous l'autorité de Conania et de Schimeï, son frère, d'après l'indication du roi Ezéchias, et d'Azaria, chef de la maison de Dieu.
\VS{14}Koré, le Lévite, fils de Jimna, portier de l'orient, avait la charge des offrandes volontaires offertes à Dieu, pour distribuer l'offrande élevée à Yahweh, et les choses consacrées et saintes.
\VS{15}Il avait sous sa direction Eden, Minjamin, Josué, Schemaeja, Amaria, et Schecania, dans les villes des sacrificateurs, pour distribuer fidèlement les portions à leurs frères, grands et petits, suivant leurs divisions,
\VS{16}à ceux qui étaient enregistrés comme mâles, depuis l'âge de trois ans et au-delà ; à tous ceux qui entraient dans la maison de Yahweh, pour le service quotidien, pour servir dans leurs charges et suivant leurs divisions ;
\VS{17}aux sacrificateurs et aux Lévites enregistrés selon la maison de leurs pères, depuis ceux de vingt ans et au-delà, selon leurs charges et selon leurs divisions ;
\VS{18}à ceux de toute l'assemblée enregistrés avec leurs petits enfants, leurs femmes, leurs fils et leurs filles ; car ils se consacraient avec fidélité aux choses saintes ;
\VS{19}et pour les enfants d'Aaron, les sacrificateurs, qui étaient à la campagne et dans les faubourgs de leurs villes, dans chaque ville, il y avait des gens désignés par leur nom, pour distribuer les portions à tous les mâles des sacrificateurs, et à tous les Lévites enregistrés.
\VS{20}Ezéchias en fit ainsi dans tout Juda ; et il fit ce qui est bon, droit et véritable, devant Yahweh, son Dieu.
\VS{21}Il travailla de tout son cœur et il réussit dans tout l'ouvrage qu'il entreprit pour le service de la maison de Dieu, et pour la loi, et pour les commandements, en recherchant son Dieu.
\Chap{32}
\TextTitle{Menaces de Sanchérib, roi d'Assyrie\FTNTT{2 R. 19:17-37 ; 19:8-13 ; Es. 36:2-20}}
\VerseOne{}Après que ces choses furent bien établies, Sanchérib, roi d'Assyrie, vint et entra en Juda, et campa contre les villes fortes, dans l'intention de faire une brèche\FTNT{Es. 36:2-21 ; 2 R. 18:13-37.}.
\VS{2}Ezéchias, voyant que Sanchérib était venu, et qu'il se tournait vers Jérusalem pour lui faire la guerre,
\VS{3}tint conseil avec ses chefs et ses vaillants hommes pour boucher les sources d'eau qui étaient hors de la ville, et ils l'aidèrent.
\VS{4}Un peuple nombreux s'assembla, et ils bouchèrent toutes les sources et le torrent qui coule par le milieu de la contrée, en disant : Pourquoi les rois d'Assyrie trouveraient-ils à leur venue de l'eau en abondance ?
\VS{5}Il se fortifia et rebâtit toute la muraille où il y avait une brèche, et l'éleva jusqu'aux tours ; il bâtit une autre muraille en dehors ; il répara Millo, dans la cité de David, et il fit faire beaucoup d'armes et de boucliers.
\VS{6}Il donna des chefs de guerre au peuple, les assembla auprès de lui sur la place de la porte de la ville, et parla à leur cœur, en disant :
\VS{7}Fortifiez-vous, soyez forts ! Ne craignez point et ne soyez pas effrayés devant le roi d'Assyrie et toute la multitude qui est avec lui ; car avec nous il y a quelqu'un de plus puissant.
\VS{8}Avec lui est le bras de la chair, mais avec nous est Yahweh, notre Dieu, pour nous aider et pour combattre dans nos combats. Et le peuple s'appuya sur les paroles d'Ezéchias, roi de Juda.
\VS{9}Après cela, Sanchérib, roi d'Assyrie, pendant qu'il était devant Lakis, ayant avec lui toutes les forces de son royaume, envoya ses serviteurs à Jérusalem vers Ezéchias, roi de Juda, et vers tous ceux de Juda qui étaient à Jérusalem, pour leur dire :
\VS{10}Ainsi parle Sanchérib, roi d'Assyrie : Sur qui vous confiez-vous pour que vous restiez à Jérusalem pour y être assiégés ?
\VS{11}Ezéchias ne vous incite-t-il pas pour vous livrer à la mort, par la famine et par la soif, en vous disant : Yahweh, notre Dieu, nous délivrera de la main du roi d'Assyrie ?
\VS{12}Cet Ezéchias n'a-t-il pas ôté les hauts lieux et les autels, et n'a-t-il pas ordonné à Juda et à Jérusalem : Vous vous prosternerez devant un seul autel pour y brûler le parfum ?
\VS{13}Ne savez-vous pas ce que nous avons fait, moi et mes pères, à tous les peuples des autres pays ? Les dieux des nations de ces pays ont-ils pu de quelque manière que ce soit délivrer leur pays de ma main ?
\VS{14}Quel est celui de tous les dieux de ces nations, que mes pères ont entièrement détruites, qui ait pu délivrer son peuple de ma main, pour que votre Dieu puisse vous délivrer de ma main ?
\VS{15}Maintenant donc, qu'Ezéchias ne vous abuse point, et qu'il ne vous incite plus de cette manière, et ne le croyez pas ; car aucun dieu d'aucune nation ni d'aucun royaume n'a pu délivrer son peuple de ma main ni de la main de mes pères ; combien moins votre Dieu vous délivrerait-il de ma main ?
\VS{16}Ses serviteurs parlèrent encore contre Yahweh Dieu, et contre Ezéchias, son serviteur.
\VS{17}Il écrivit aussi une lettre pour blasphémer contre Yahweh, le Dieu d'Israël, en parlant ainsi : Comme les dieux des nations des autres pays n'ont pu délivrer leur peuple de ma main, ainsi le Dieu d'Ezéchias ne pourra délivrer son peuple de ma main\FTNT{2 R. 19:14-37.}.
\VS{18}Et ses serviteurs crièrent à haute voix en langue judaïque, au peuple de Jérusalem qui était sur la muraille, pour les effrayer et les épouvanter, afin de prendre la ville.
\VS{19}Ils parlèrent du Dieu de Jérusalem comme des dieux des peuples de la terre, qui ne sont qu'un ouvrage de mains d'homme.
\TextTitle{Prière d'Ezéchias et exaucement de Yahweh\FTNTT{2 R. 19:14-37 ; Es. 36:21-37:35}}
\VS{20}Alors le roi Ezéchias, et Esaïe, le prophète, fils d'Amots, prièrent à ce sujet et crièrent vers les cieux.
\VS{21}Et Yahweh envoya un ange, dans le camp du roi d'Assyrie, qui extermina tous les vaillants hommes, les princes et les chefs, en sorte qu'il retourna dans son pays, dans la honte. Il entra dans la maison de son dieu ; et là, ceux qui étaient sortis de ses entrailles le firent tomber par l'épée.
\VS{22}C'est ainsi que Yahweh sauva Ezéchias et les habitants de Jérusalem de la main de Sanchérib, roi d'Assyrie, et de la main de tout homme, et il les protégea de toutes parts.
\VS{23}Plusieurs apportèrent des offrandes à Yahweh, à Jérusalem, et des choses précieuses à Ezéchias, roi de Juda, qui après cela fut élevé aux yeux de toutes les nations.
\TextTitle{Maladie et guérison d'Ezéchias\FTNTT{2 R. 20:1-11}}
\VS{24}En ces jours-là, Ezéchias fut malade à en mourir, et il pria Yahweh, qui l'exauça et lui accorda un prodige\FTNT{2 R. 20:1-11}.
\VS{25}Mais Ezéchias ne fut pas reconnaissant du bienfait qu'il avait reçu ; car son cœur s'éleva, et il y eut des maux contre lui, et contre Juda et Jérusalem.
\VS{26}Mais Ezéchias s'humilia de l'élévation de son cœur, lui et les habitants de Jérusalem, et la colère de Yahweh ne vint plus sur eux durant les jours d'Ezéchias.
\TextTitle{Fin du règne d'Ezéchias, sa mort\FTNTT{2 R. 20:12-21 ; cp. Es. 39}}
\VS{27}Ezéchias eut de très grandes richesses et de la gloire, et il se fit des trésors d'argent, d'or, de pierres précieuses, d'aromates, de boucliers, et de toutes sortes d'objets précieux ;
\VS{28}des magasins pour les récoltes de blé, de moût et d'huile, des étables pour toutes sortes de bétail, avec des rangées dans les étables.
\VS{29}Il se fit aussi des villes, et il acquit des troupeaux du gros et du menu bétail en abondance ; car Dieu lui avait donné de très grandes richesses.
\VS{30}Ce fut Ezéchias, qui boucha le canal du haut des eaux de Guihon, et les conduisit directement en bas, vers l'occident de la cité de David. Ainsi Ezéchias réussit dans tout ce qu'il fit.
\VS{31}Toutefois, lorsque les princes de Babylone envoyèrent des messagers vers lui pour s'informer du prodige qui s'était produit dans le pays, Dieu l'abandonna pour le mettre à l'épreuve, afin de connaître tout ce qui était dans son cœur\FTNT{Es. 29.}.
\VS{32}Le reste des actions d'Ezéchias, ses bonnes œuvres, voici, elles sont écrites dans la vision d'Esaïe, le prophète, fils d'Amots, dans le livre des rois de Juda et d'Israël.
\VS{33}Puis Ezéchias s'endormit avec ses pères, et on l'ensevelit au plus haut des sépulcres des fils de David ; et tout Juda, et Jérusalem lui firent honneur à sa mort, et Manassé, son fils régna à sa place.
\Chap{33}
\TextTitle{Manassé, roi impie de Juda\FTNTT{2 R. 21:1-9}}
\VerseOne{}Manassé était âgé de douze ans quand il devint roi, et il régna cinquante-cinq ans à Jérusalem.
\VS{2}Il fit ce qui est mal aux yeux de Yahweh, suivant les abominations des nations que Yahweh avait chassées devant les enfants d'Israël.
\VS{3}Il rebâtit les hauts lieux qu'Ezéchias, son père, avait démolis, il redressa les autels aux Baals, il fit des idoles d'Astarté, et se prosterna devant toute l'armée des cieux et la servit.
\VS{4}Il bâtit aussi des autels dans la maison de Yahweh, de laquelle Yahweh avait parlé ainsi : Mon Nom sera dans Jérusalem à jamais.
\VS{5}Il bâtit des autels à toute l'armée des cieux, dans les deux parvis de la maison de Yahweh.
\VS{6}Il fit passer ses fils par le feu dans la vallée du fils de Hinnom ; il pratiquait la magie, les sorcelleries et la voyance ; il établit des gens qui évoquaient les esprits et des devins. Il s'adonna à faire à l'extrême ce qui est mal aux yeux de Yahweh, pour l'irriter.
\VS{7}Il posa aussi une image taillée, une idole qu'il avait faite, dans la maison de Dieu, de laquelle Dieu avait dit à David, et à Salomon, son fils : Je mettrai à perpétuité mon Nom dans cette maison et dans Jérusalem, que j'ai choisie entre toutes les tribus d'Israël ;
\VS{8}et je ne ferai plus sortir Israël de la terre que j'ai assignée à leurs pères, pourvu seulement qu'ils prennent garde à faire tout ce que je leur ai ordonné, selon toute la loi, les préceptes et les ordonnances prescrites par Moïse.
\VS{9}Manassé donc fit s'égarer Juda et les habitants de Jérusalem, jusqu'à faire pire que les nations que Yahweh avait exterminées de devant les enfants d'Israël.
\TextTitle{Yahweh avertit Manassé\FTNTT{2 R. 21:10-16}}
\VS{10}Yahweh parla à Manassé et à son peuple ; mais ils ne furent pas attentifs.
\TextTitle{Manassé emmené captif se repent\FTNTT{2 R. 21:17-18}}
\VS{11}Alors Yahweh fit venir contre eux les chefs de l'armée du roi d'Assyrie, qui mirent Manassé dans les fers ; ils le lièrent d'une double chaîne d'airain, et l'emmenèrent à Babylone.
\VS{12}Et dès qu'il fut dans l'angoisse, il supplia Yahweh, son Dieu, et il s'humilia profondément devant le Dieu de ses pères.
\VS{13}Il lui adressa ses supplications, et Dieu se laissa fléchir par sa prière, et exauça sa supplication. Il le fit retourner à Jérusalem, dans son royaume. Manassé reconnut alors que c'est Yahweh qui est Dieu.
\VS{14}Après cela, il bâtit une muraille extérieure à la cité de David, vers l'occident de Guihon, dans la vallée, jusqu'à l'entrée de la porte des poissons ; il environna la colline et l'éleva à une grande hauteur ; il établit aussi des chefs d'armée dans toutes les villes fortes de Juda.
\VS{15}Il ôta de la maison de Yahweh l'idole, et les dieux étrangers, et tous les autels qu'il avait bâtis sur la montagne de la maison de Yahweh et à Jérusalem, et les jeta hors de la ville.
\VS{16}Puis il rebâtit l'autel de Yahweh et y offrit des sacrifices d'offrande de paix et de reconnaissance ; et il ordonna à Juda de servir Yahweh, le Dieu d'Israël.
\VS{17}Toutefois, le peuple sacrifiait encore dans les hauts lieux, mais seulement à Yahweh, son Dieu.
\VS{18}Le reste des actions de Manassé, et la prière qu'il fit à son Dieu, et les paroles des voyants qui lui parlaient, au Nom de Yahweh, le Dieu d'Israël, voilà, toutes ces choses sont écrites dans les actes des rois d'Israël.
\VS{19}Sa prière, et comment Dieu se laissa fléchir par sa prière, ses péchés et ses infidélités, les lieux sur lesquels il bâtit des hauts lieux, et dressa des idoles d'Astarté et des images taillées, avant de s'être humilié, voici cela est écrit dans le livre de Hozaï.
\VS{20}Puis Manassé s'endormit avec ses pères, et on l'ensevelit dans sa maison. Et Amon, son fils, régna à sa place.
\TextTitle{Amon règne brièvement sur Juda\FTNTT{2 R. 21:18-26}}
\VS{21}Amon était âgé de vingt-deux ans quand il devint roi, et il régna deux ans à Jérusalem.
\VS{22}Il fit ce qui est mal aux yeux de Yahweh, comme avait fait Manassé, son père. Il sacrifia à toutes les images taillées que Manassé, son père, avait faites, et il les servit.
\VS{23}Mais il ne s'humilia point devant Yahweh, comme s'était humilié Manassé, son père, mais se rendit de plus en plus coupable.
\VS{24}Et ses serviteurs ayant fait une conspiration contre lui, le firent mourir dans sa maison.
\VS{25}Mais le peuple du pays frappa tous ceux qui avaient conspiré contre le roi Amon. Et le peuple du pays établit pour roi, à sa place, Josias, son fils.
\Chap{34}
\TextTitle{Josias règne sur Juda ; ses réformes\FTNTT{2 R. 22:1-2}}
\VerseOne{}Josias était âgé de huit ans quand il devint roi, et il régna trente et un ans à Jérusalem.
\VS{2}Il fit ce qui est droit aux yeux de Yahweh. Il marcha dans les voies de David, son père ; et ne s'en détourna ni à droite ni à gauche.
\VS{3}La huitième année de son règne, lorsqu'il était jeune, il commença à rechercher le Dieu de David, son père ; et à la douzième année, il commença à purifier Juda et Jérusalem des hauts lieux, des idoles d'Astarté, et des images taillées, et des images de fonte.
\VS{4}On démolit dans sa présence les autels des Baals, et il abattit les tentes solaires\FTNT{Tentes solaires : lieux d’idolâtrie} qui étaient par-dessus. Il brisa les idoles d'Astarté, les images taillées et les images de fonte ; et les ayant réduites en poudre, il la répandit sur les sépulcres de ceux qui leur avaient sacrifié.
\VS{5}Puis il brûla les os des prêtres sur leurs autels, et il purifia ainsi Juda et Jérusalem.
\VS{6}Il fit la même chose dans les villes de Manassé, d'Ephraïm et de Siméon, et jusqu'à Nephthali, dans leurs ruines et tout autour.
\VS{7}Il démolit les autels et mit en pièces les idoles d'Astarté et les images taillées, et il les réduisit en poussière ; il abattit toutes les tentes solaires dans tout le pays d'Israël. Puis il revint à Jérusalem.
\TextTitle{Restauration du temple\FTNTT{2 R. 22:3-7}}
\VS{8}La dix-huitième année de son règne, après avoir purifié le pays et le temple, il envoya Schaphan, fils d'Atsalia, et Maaséja, chefs de la ville, et Joach, fils de Joachaz, commis sur les registres, pour réparer la maison de Yahweh, son Dieu.
\VS{9}Ils vinrent vers Hilkija, le souverain sacrificateur ; et on livra l'argent qui avait été apporté dans la maison de Dieu et que les Lévites, gardes du seuil, avaient amassé des mains de Manassé, d'Ephraïm et de tout le reste d'Israël, et aussi de tout Juda et Benjamin ; puis ils s'en retournèrent à Jérusalem.
\VS{10}On le remit entre les mains de ceux qui avaient la charge de l'ouvrage, qui étaient préposés sur la maison de Yahweh. Et ceux qui avaient la charge de l'ouvrage et qui travaillaient dans la maison de Yahweh le distribuèrent pour restaurer et réparer la maison de Yahweh.
\VS{11}Ils le donnèrent aux charpentiers et aux maçons, pour acheter des pierres de taille et du bois pour les poutres et pour la charpente des maisons que les rois de Juda avaient détruites.
\VS{12}Ces hommes s'employaient fidèlement à cet ouvrage. Jachath et Abdias, Lévites d'entre les fils de Merari, étaient préposés sur eux, et Zacharie et Meschullam, d'entre les fils des Kehathites, pour les diriger. Ces Lévites avaient tous de l'intelligence pour les instruments de musique.
\VS{13}Ils surveillaient ceux qui portaient les fardeaux, et dirigeaient tous ceux qui faisaient l'ouvrage, dans quelque service que ce soit ; les scribes, les administrateurs et les portiers, d'entre les Lévites.
\TextTitle{Le livre de la loi redécouvert\FTNTT{2 R. 22:8-10}}
\VS{14}Au moment où l'on sortit l'argent qui avait été apporté dans la maison de Yahweh, Hilkija, le sacrificateur, trouva le livre de la loi de Yahweh, donné par Moïse.
\VS{15}Alors Hilkija, prenant la parole, dit à Schaphan, le secrétaire : J'ai trouvé le livre de la loi dans la maison de Yahweh. Et Hilkija donna le livre à Schaphan.
\VS{16}Schaphan apporta le livre au roi, et rapporta tout au roi, en disant : Les mains de tes serviteurs ont fait tout ce qui leur a été donné à faire.
\VS{17}Ils ont amassé l'argent qui se trouvait dans la maison de Yahweh, et l'ont livré entre les mains des administrateurs, et entre les mains de ceux qui ont la charge de l'ouvrage.
\VS{18}Schaphan, le secrétaire, raconta en disant au roi : Hilkija, le sacrificateur, m'a donné un livre ; et Schaphan le lut devant le roi.
\TextTitle{Lecture du livre de la loi\FTNTT{2 R. 22:11-13}}
\VS{19}Lorsque le roi entendit les paroles de la loi, il déchira ses vêtements.
\VS{20}Il ordonna à Hilkija, à Achikam, fils de Schaphan, à Abdon, fils de Michée, à Schaphan, le secrétaire, et à Asaja, serviteur du roi, en disant :
\VS{21}Allez, consultez Yahweh pour moi et pour ce qui reste en Israël et en Juda, touchant les paroles du livre qui a été trouvé ; car la colère de Yahweh est grande, et elle s'est déversée sur nous, parce que nos pères n'ont point gardé la parole de Yahweh, pour faire selon tout ce qui est écrit dans ce livre.
\TextTitle{Instruction de la prophétesse Hulda\FTNTT{2 R. 22:14-20}}
\VS{22}Hilkija et les gens du roi allèrent vers Hulda, la prophétesse, femme de Schallum, fils de Thokehath, fils de Hasra, garde des vêtements, laquelle demeurait à Jérusalem, dans un autre quartier, et lui en parlèrent.
\VS{23}Alors elle leur répondit : Ainsi parle Yahweh, le Dieu d'Israël : Dites à l'homme qui vous a envoyés vers moi :
\VS{24}Ainsi parle Yahweh : Voici, je vais faire venir le malheur sur ce lieu et sur ses habitants, à savoir toutes les malédictions du serment qui sont écrites dans le livre qu'on a lu devant le roi de Juda.
\VS{25}Parce qu'ils m'ont abandonné, et qu'ils ont fait brûler des parfums aux autres dieux, pour m'irriter par toutes les œuvres de leurs mains, ma colère s'est déversée sur ce lieu, et elle ne sera point éteinte.
\VS{26}Mais quant au roi de Juda, qui vous a envoyés pour consulter Yahweh, vous lui direz : Ainsi parle Yahweh, le Dieu d'Israël, au sujet des paroles que tu as entendues :
\VS{27}Parce que ton cœur a été touché, et que tu t'es humilié devant Dieu, quand tu as entendu ses paroles contre ce lieu et contre ses habitants, et que t'étant humilié devant moi, tu as déchiré tes vêtements et pleuré devant moi, je t'ai aussi entendu, dit Yahweh.
\VS{28}Voici, je vais te recueillir avec tes pères, et tu seras recueilli dans tes sépulcres en paix, et tes yeux ne verront point tout ce mal que je vais faire venir sur ce lieu et sur ses habitants. Et ils rapportèrent cette parole au roi.
\TextTitle{Renouvellement de l'alliance avec Yahweh\FTNTT{2 R. 23:1-3}}
\VS{29}Alors le roi envoya assembler tous les anciens de Juda et de Jérusalem.
\VS{30}Le roi monta à la maison de Yahweh avec tous les hommes de Juda et les habitants de Jérusalem, les sacrificateurs et les Lévites, et tout le peuple, depuis le plus grand jusqu'au plus petit ; et on lut devant eux toutes les paroles du livre de l'alliance, qui avait été trouvé dans la maison de Yahweh.
\VS{31}Et le roi se tint debout à sa place ; et traita devant Yahweh cette alliance qu'ils suivraient Yahweh, et qu'ils garderaient ses commandements, ses témoignages et ses lois, chacun de tout son cœur et de toute son âme, en pratiquant les paroles de l'alliance écrites dans ce livre.
\VS{32}Et il fit tenir debout tous ceux qui se trouvèrent à Jérusalem et en Benjamin ; et les habitants de Jérusalem firent selon l'alliance de Dieu, le Dieu de leurs pères.
\VS{33}Josias ôta de tous les pays qui appartenaient aux enfants d'Israël, toutes les abominations ; et il obligea tous ceux qui se trouvaient en Israël à servir Yahweh, leur Dieu. Pendant toute sa vie, ils ne se détournèrent point de Yahweh, le Dieu de leurs pères.
\Chap{35}
\TextTitle{Josias rétablit la Pâque\FTNTT{2 R. 23:21-27}}
\VerseOne{}Or Josias célébra la Pâque à Yahweh à Jérusalem, et on immola la Pâque le quatorzième jour du premier mois.
\VS{2}Il rétablit les sacrificateurs dans leurs charges, et les encouragea au service de la maison de Yahweh.
\VS{3}Il dit aussi aux Lévites qui enseignaient tout Israël et qui étaient consacrés à Yahweh : Mettez l'arche sainte dans la maison que Salomon, fils de David, roi d'Israël, a bâtie. Qu'elle ne soit plus une charge sur vos épaules. Maintenant, servez Yahweh, votre Dieu, et son peuple d'Israël.
\VS{4}Préparez-vous, selon les maisons de vos pères, selon vos divisions suivant l'écrit de David, roi d'Israël, et suivant l'écrit de Salomon, son fils.
\VS{5}Tenez-vous dans le sanctuaire pour vos frères, les fils du peuple, selon les classes des maisons des pères, et selon que chaque famille des Lévites est partagée.
\VS{6}Immolez la Pâque, sanctifiez-vous, et préparez-la pour vos frères, afin qu'ils puissent la faire selon la parole que Yahweh a donnée par Moïse.
\VS{7}Josias éleva une offrande pour les gens du peuple et pour tous ceux qui se trouvaient là, des troupeaux d'agneaux et de chevreaux, au nombre de trente mille, et trois mille bœufs, le tout pour la Pâque ; cela fut pris sur les biens du roi.
\VS{8}Ses chefs élevèrent une offrande de bon gré pour le peuple, aux sacrificateurs et aux Lévites. Hilkija, Zacharie et Jehiel, princes de la maison de Dieu, donnèrent aux sacrificateurs, pour la Pâque, deux mille six cents agneaux, et trois cents bœufs.
\VS{9}Conania, Schemaeja et Nethaneel, ses frères, et Haschabia, Jeïel et Jozabad, qui étaient les princes des Lévites, élevèrent une offrande de cinq mille agneaux aux Lévites pour faire la Pâque, et cinq cents bœufs.
\VS{10}Le service étant préparé, les sacrificateurs se tinrent à leurs postes, et les Lévites suivant leurs divisions, selon l'ordre du roi.
\VS{11}Puis on immola la Pâque ; et les sacrificateurs répandaient le sang reçu de leurs mains, et les Lévites les dépouillaient.
\VS{12}Ils mirent à part les holocaustes, pour les donner aux gens du peuple, suivant les divisions des maisons de leurs pères, afin de les offrir à Yahweh, selon ce qui est écrit au livre de Moïse ; ils firent de même pour les bœufs.
\VS{13}Ils firent cuire la Pâque au feu, selon l'ordonnance ; et ils firent cuire dans des chaudières, des chaudrons et des poêles, les choses consacrées ; et ils les apportèrent rapidement à tous les gens du peuple.
\VS{14}Puis ils apprêtèrent ce qui était pour eux et pour les sacrificateurs, car les sacrificateurs, fils d'Aaron, furent occupés jusqu'à la nuit à élever en offrande les holocaustes et les graisses ; c'est pourquoi les Lévites apprêtèrent ce qui était pour eux et pour les sacrificateurs, fils d'Aaron.
\VS{15}Les chantres, fils d'Asaph, étaient à leur place, selon l'ordre de David, d'Asaph, d'Héman et de Jeduthun, le voyant du roi. Les portiers étaient à chaque porte, ils n'eurent pas à se détourner de leur service, car leurs frères les Lévites apprêtaient ce qui était pour eux.
\VS{16}Ainsi, tout le service de Yahweh, en ce jour-là, fut réglé pour faire la Pâque et pour élever en offrande les holocaustes sur l'autel de Yahweh, selon l'ordre du roi Josias.
\VS{17}Les fils d'Israël qui s'y trouvèrent célébrèrent donc la Pâque en ce temps-là, et la fête des pains sans levain pendant sept jours.
\VS{18}Or on n'avait point célébré en Israël de Pâque semblable à celle-là depuis les jours de Samuel le prophète ; et aucun des rois d'Israël n'avait célébré une Pâque pareille comme le fit Josias, avec les sacrificateurs et les Lévites, et tout Juda et Israël, qui s'y étaient trouvés avec les habitants de Jérusalem.
\VS{19}Cette Pâque fut célébrée la dix-huitième année du règne de Josias.
\TextTitle{Blessure et mort de Josias\FTNTT{2 R. 23:28-30}}
\VS{20}Après tout cela, quand Josias eut réparé la maison de Yahweh, Néco, roi d'Egypte, monta pour faire la guerre à Carkemisch, sur l'Euphrate. Josias sortit à sa rencontre.
\VS{21}Mais Néco envoya vers lui des messagers pour lui dire : Qu'y a-t-il entre nous, roi de Juda ? Ce n'est pas à toi que j'en veux aujourd'hui, mais à une maison qui me fait la guerre ; et Dieu m'a dit de me hâter. Désiste-toi donc de venir contre Dieu, qui est avec moi, de peur qu'il ne te détruise.
\VS{22}Cependant Josias ne se détourna point de lui, mais se déguisa pour combattre contre lui et il n'écouta pas les paroles de Néco, qui venaient de la bouche de Dieu. Il vint donc pour combattre dans la vallée de Meguiddo.
\VS{23}Les archers tirèrent sur le roi Josias ; et le roi dit à ses serviteurs : Emportez-moi, car je suis très blessé.
\VS{24}Ses serviteurs l'ôtèrent du char, le mirent sur un second char qu'il avait, et le menèrent à Jérusalem. Il mourut et il fut enseveli dans les sépulcres de ses pères, et tous ceux de Juda et de Jérusalem menèrent le deuil de Josias.
\VS{25}Jérémie fit aussi des lamentations sur Josias ; et tous les chanteurs et toutes les chanteuses parlèrent dans leurs complaintes sur Josias jusqu'à ce jour ; et on en a fait une coutume en Israël. Voici, ces choses sont écrites dans les lamentations.
\VS{26}Le reste des actions de Josias, et ses œuvres de piété, selon ce qui est écrit dans la loi de Yahweh,
\VS{27}ses premières et ses dernières actions, sont écrites dans le livre des rois d'Israël et de Juda.
\Chap{36}
\TextTitle{Joachaz règne brièvement sur Juda\FTNTT{2 R. 23:31-33}}
\VerseOne{}Alors le peuple du pays prit Joachaz, fils de Josias, et on l'établit roi à Jérusalem, à la place de son père.
\VS{2}Joachaz était âgé de vingt-trois ans quand il devint roi, et il régna trois mois à Jérusalem.
\VS{3}Le roi d'Egypte le destitua à Jérusalem, et condamna le pays à une amende de cent talents d'argent et d'un talent d'or.
\TextTitle{Règne de Jojakim, déportation à Babylone\FTNTT{2 R. 23:34-24:4-9}}
\VS{4}Le roi d'Egypte établit pour roi sur Juda et Jérusalem Eliakim, frère de Joachaz ; et changea son nom en celui de Jojakim. Puis Néco prit Joachaz, son frère, et l'emmena en Egypte.
\VS{5}Jojakim était âgé de vingt-cinq ans quand il devint roi, et il régna onze ans à Jérusalem. Il fit ce qui est mal aux yeux de Yahweh, son Dieu.
\VS{6}Nebucadnetsar, roi de Babylone, monta contre lui et le lia de doubles chaînes d'airain pour le mener à Babylone.
\VS{7}Nebucadnetsar emporta aussi à Babylone des ustensiles de la maison de Yahweh, et il les mit dans son temple à Babylone.
\VS{8}Le reste des actions de Jojakim, et les abominations qu'il commit, et ce qui fut trouvé en lui, cela est écrit dans le livre des rois d'Israël et de Juda. Et Jojakin, son fils, régna à sa place.
\VS{9}Jojakin était âgé de huit ans quand il devint roi, et il régna trois mois et dix jours à Jérusalem. Il fit ce qui est mal aux yeux de Yahweh.
\TextTitle{Sédécias le dernier roi de Juda, autres déportations à Babylone\FTNTT{2 R.24:10-20 ; cp. 2 R. 25:1-21 ; Jé. 39:8-10}}
\VS{10}Et l'année suivante, le roi Nebucadnetsar envoya, et le fit emmener à Babylone avec les ustensiles précieux de la maison de Yahweh ; et il établit roi sur Juda et Jérusalem, Sédécias, son frère.
\VS{11}Sédécias était âgé de vingt et un ans quand il devint roi, et il régna onze ans à Jérusalem.
\VS{12}Il fit ce qui est mal aux yeux de Yahweh, son Dieu ; et il ne s'humilia point devant Jérémie le prophète, qui lui parlait de la part de Yahweh.
\VS{13}Et même il se rebella contre le roi Nebucadnetsar, qui l'avait fait prêter serment par le Nom de Dieu. Il raidit son cou, et il obstina son cœur pour ne point retourner à Yahweh, le Dieu d'Israël.
\VS{14}Pareillement, tous les chefs des sacrificateurs et le peuple furent infidèles et continuèrent de plus en plus à pécher, selon toutes les abominations des nations ; et ils souillèrent la maison que Yahweh avait sanctifiée dans Jérusalem.
\VS{15}Or Yahweh, le Dieu de leurs pères, les avait sommés par ses messagers qu'il envoya de bonne heure, car il voulait épargner son peuple et sa propre demeure.
\VS{16}Mais ils se moquèrent des messagers de Dieu, ils méprisèrent ses paroles et traitèrent ses prophètes de séducteurs, jusqu'à ce que la fureur de Yahweh montat contre son peuple au point qu'il n'y eut plus de remède.
\VS{17}C'est pourquoi il fit monter contre eux le roi des Chaldéens, qui tua par l'épée leurs jeunes gens dans la maison de leur sanctuaire ; il n'épargna ni le jeune homme, ni la vierge, ni le vieillard, ni l'homme à cheveux blancs ; il les livra tous entre ses mains.
\VS{18}Il fit apporter à Babylone tous les ustensiles de la maison de Dieu, grands et petits, les trésors de la maison de Yahweh, et les trésors du roi et ceux de ses chefs.
\VS{19}Ils brûlèrent la maison de Dieu, ils démolirent les murailles de Jérusalem, ils livrèrent au feu tous ses palais et détruisirent tout ce qu'il y avait comme objets précieux.
\VS{20}Puis le roi de Babylone transporta à Babylone le reste qui échappa à l'épée, et ils furent ses esclaves et ceux de ses fils, jusqu'à la domination du royaume de Perse,
\VS{21}afin que la parole de Yahweh, prononcée par la bouche de Jérémie, fût accomplie ; jusqu'à ce que la terre eût pris plaisir à ses sabbats et durant tous les jours qu'elle demeura dévastée ; elle se reposa pour accomplir les soixante-dix années.
\TextTitle{L'édit de Cyrus autorise les juifs à retourner dans leurs villes}
\VS{22}Mais la première année de Cyrus, roi de Perse, afin que la parole de Yahweh prononcée par Jérémie fût accomplie, Yahweh réveilla l'esprit de Cyrus, roi de Perse, qui fit publier dans tout son royaume, et même par écrit, en disant :
\VS{23}Ainsi parle Cyrus, roi de Perse : Yahweh, le Dieu des cieux, m'a donné tous les royaumes de la terre, et lui-même m'a ordonné de lui bâtir une maison à Jérusalem, qui est en Juda. Qui d'entre vous est de son peuple ? Que Yahweh, son Dieu, soit avec lui, et qu'il monte !
\PPE{}
\end{multicols}

%\addcontentsline{toc}{chapter}{Évangiles}\clearpage
%\clearpage\ShortTitle{Matthieu}\BookTitle{Matthieu}\BFont
\noindent\hrulefill
{\footnotesize
\textit{
\bigskip
{\centering{}
\\Auteur : Matthieu
\\(Gr. : Matthaios)
\\Signifie : Don de Yahweh
\\Thème : Jésus le Roi
\\Date de rédaction : Env. 50 ap. J.-C.\\}
}
%\bigskip
\textit{
\\Matthieu, également connu sous le nom de Lévi, était un juif percepteur d'impôts au service des Romains. Appelé par
Jésus-Christ à Capernaüm et choisi pour être l'un des douze disciples, il offrit un banquet en l'honneur de Jésus dans
sa maison, ce qui lui valut l'hostilité des pharisiens. Rédigé à Antioche de Syrie, son évangile était destiné à des juifs
convertis comme en témoignent ses nombreuses allusions à l'Ancienne Alliance.
%\bigskip
\\Matthieu, démontre l'hégémonie de Jésus, fils de David, fils d'Abraham, roi d'Israël. Il évoque son règne qui se manifestera un jour physiquement, lorsque le Roi jugera tous les hommes du « trône de sa gloire ». Il fait concorder les paroles et les événements de la vie de Jésus avec les prophéties de l'Ancienne Alliance.
%\bigskip
\\Son récit exalte la royauté de Jésus et expose l'Evangile du Royaume.\bigskip
}
}
\par\nobreak\noindent\hrulefill
\begin{multicols}{2}
\Chap{1}
\TextTitle{Présentation du Roi : La généalogie de Jésus-Christ}
\VerseOne{}Livre de la généalogie de Jésus-Christ, fils de David, fils d'Abraham.
\VS{2}Abraham engendra Isaac ; Isaac engendra Jacob ; Jacob engendra Juda et ses frères ;
\VS{3}Juda engendra Pérets et Zara, de Thamar ; Pérets engendra Esrom ; Esrom engendra Aram ;
\VS{4}Aram engendra Aminadab ; Aminadab engendra Naasson ; et Naasson engendra Salmon ;
\VS{5}Salmon engendra Boaz, de Rahab\FTNT{Rahab était une prostituée cananéenne qui est devenue l'ancêtre du Messie (Jos. 6).} ; Boaz engendra Obed, de Ruth\FTNT{Ruth était une Moabite, son peuple était issu de la relation incestueuse de Lot et sa fille aînée (Ge. 19:36-37). Elle est devenue l'ancêtre du Messie (Ru. 4:17).} ; Obed engendra Isaï ;
\VS{6}Isaï engendra le roi David ; le roi David engendra Salomon, de la femme d'Urie ;
\VS{7}Salomon engendra Roboam ; Roboam engendra Abia ; Abia engendra Asa ;
\VS{8}Asa engendra Josaphat ; Josaphat engendra Joram ; Joram engendra Ozias ;
\VS{9}Ozias engendra Joatham ; Joatham engendra Achaz ; Achaz engendra Ezéchias ;
\VS{10}Ezéchias engendra Manassé ; Manassé engendra Amon ; Amon engendra Josias ;
\VS{11}Josias engendra Jéchonias et ses frères, au temps de la déportation à Babylone.
\VS{12}Après la déportation à Babylone, Jéchonias engendra Salathiel ; Salathiel engendra Zorobabel ;
\VS{13}Zorobabel engendra Abiud ; Abiud engendra Eliakim ; Eliakim engendra Azor ;
\VS{14}Azor engendra Sadok ; Sadok engendra Achim ; Achim engendra Eliud ;
\VS{15}Eliud engendra Eléazar ; Eléazar engendra Matthan ; Matthan engendra Jacob ;
\VS{16}Jacob engendra Joseph, l'époux de Marie, de laquelle est né Jésus, qui est appelé Christ\FTNT{Christ : Du grec « christos », ce qui signifie « oint », est l'équivalent grec du mot hébreu « mashiyach », traduit par « messie » en français (Da. 9:25-26). C'est le titre officiel du Seigneur Jésus.}.
\VS{17}Ainsi, il y a en tout quatorze générations depuis Abraham jusqu'à David, quatorze générations depuis David jusqu'à la déportation à Babylone, et quatorze générations depuis la déportation à Babylone jusqu'au Christ.
\TextTitle{Naissance miraculeuse de Jésus-Christ\FTNTT{Lu. 1:26-38 ; 2:1-7 ; Jn. 1:1-2,14}}
\VS{18}Voici de quelle manière arriva la naissance de Jésus-Christ. Marie, sa mère, ayant été fiancée à Joseph, se trouva enceinte par l'opération du Saint-Esprit, avant qu'ils aient habité ensemble.
\VS{19}Joseph, son époux, qui était un homme juste et qui ne voulait pas la diffamer, se proposa de la répudier secrètement.
\VS{20}Mais comme il y pensait, voici, l'Ange du Seigneur lui apparut en songe et lui dit : Joseph, fils de David, ne crains point de prendre avec toi Marie, ta femme, car l'enfant qu'elle a conçu est du Saint-Esprit.
\VS{21}Elle enfantera un fils, et tu lui donneras le nom de Jésus. C'est lui qui sauvera son peuple de ses péchés.
\VS{22}Tout cela arriva afin que s'accomplisse ce que le Seigneur avait annoncé par le prophète :
\VS{23}Voici, la vierge deviendra enceinte, elle enfantera un fils ; et on lui donnera le nom d'Emmanuel\FTNT{Es. 7:14.}, ce qui signifie, Dieu est avec nous.
\VS{24}Joseph s'étant donc réveillé de son sommeil, fit ce que l'Ange du Seigneur lui avait ordonné, et il prit sa femme.
\VS{25}Mais il ne la connut point jusqu'à ce qu'elle ait enfanté son fils premier-né, auquel il donna le nom de Jésus.
\Chap{2}
\TextTitle{Les mages adorent Jésus}
\VerseOne{}Jésus étant né à Bethléhem, ville de Juda, au temps du roi Hérode, voici des mages d'orient arrivèrent à Jérusalem.
\VS{2}En disant : Où est le Roi des Juifs qui vient de naître ? Car nous avons vu son étoile en orient, et nous sommes venus l'adorer.
\VS{3}Le roi Hérode ayant entendu, fut troublé et tout Jérusalem avec lui.
\VS{4}Et ayant assemblé tous les principaux sacrificateurs et les scribes du peuple, il s'informa auprès d'eux où le Christ devait naître.
\VS{5}Et ils lui dirent : A Bethléhem, ville de Judée ; car voici ce qui a été écrit par le prophète :
\VS{6}Et toi, Bethléhem, terre de Juda, tu n'es nullement la plus petite parmi les gouverneurs de Juda, car de toi sortira le Chef qui paîtra mon peuple d'Israël\FTNT{Mi. 5:1.}.
\VS{7}Alors Hérode ayant appelé en secret les mages, s'informa soigneusement auprès d'eux depuis combien de temps brillait l'étoile.
\VS{8}Puis il les envoya à Bethléhem, en leur disant : Allez, et prenez des informations exactes sur le petit enfant ; et quand vous l'aurez trouvé, faites-le-moi savoir, afin que j'aille aussi moi-même l'adorer.
\VS{9}Après avoir entendu le roi, ils partirent. Et voici, l'étoile\FTNT{Le Seigneur Jésus-Christ s'est révélé à Jean comme l'étoile brillante du matin (Ap. 22:16).} qu'ils avaient vue en orient allait devant eux, jusqu'au moment où, arrivée au-dessus du lieu où était le petit enfant, elle s'arrêta.
\VS{10}Quand ils virent l'étoile, ils furent saisis d'une très grande joie.
\VS{11}Ils entrèrent dans la maison, virent le petit enfant avec Marie sa mère, se prosternèrent et l'adorèrent. Ils ouvrirent ensuite leurs trésors et lui offrirent des présents : De l'or, de l'encens et de la myrrhe.
\VS{12}Puis, divinement avertis en songe de ne pas retourner vers Hérode, ils regagnèrent leur pays par un autre chemin.
\TextTitle{Fuite en Egypte}
\VS{13}Lorsqu'ils furent partis, voici, l'Ange du Seigneur apparut dans un songe à Joseph et lui dit : Lève-toi et prends le petit enfant et sa mère, fuis en Egypte, et demeure là, jusqu'à ce que je te le dise ; car Hérode cherchera le petit enfant pour le faire mourir.
\VS{14}Joseph donc étant réveillé, prit de nuit le petit enfant et sa mère, et se retira en Egypte.
\VS{15}Il y resta là jusqu'à la mort d'Hérode ; afin que s'accomplisse ce que le Seigneur avait annoncé par le prophète : J'ai appelé mon Fils hors d'Egypte\FTNT{Os. 11:1}.
\TextTitle{Hérode envoie tuer des enfants innocents}
\VS{16}Alors Hérode voyant que les mages s'étaient moqués de lui, se mit dans une grande colère, et il envoya tuer tous les enfants qui étaient à Bethléhem, et dans tout son territoire ; depuis l'âge de deux ans, et au-dessous, selon la date dont il s'était exactement enquis auprès des mages.
\VS{17}Alors s'accomplit ce qui avait été annoncé par Jérémie le prophète :
\VS{18}On a entendu à Rama des cris, des lamentations, des plaintes, et des grands gémissements : Rachel pleure ses enfants et n'a pas voulu être consolée, parce qu'ils ne sont plus\FTNT{Jé. 31:15}.
\TextTitle{Joseph revient en Israël et s'installe à Nazareth\FTNTT{Lu. 2:39-52}}
\VS{19}Mais après qu'Hérode fut mort, voici, l'Ange du Seigneur apparut dans un songe à Joseph en Egypte,
\VS{20}et lui dit : Lève-toi, et prends le petit enfant et sa mère, et va dans le pays d'Israël ; car ceux qui cherchaient à ôter la vie au petit enfant sont morts.
\VS{21}Joseph donc s'étant réveillé, prit le petit enfant et sa mère, et alla dans le pays d'Israël.
\VS{22}Mais quand il eut appris qu'Archélaüs régnait en Judée, à la place d'Hérode son père, il craignit d'y aller ; et étant divinement averti dans un songe, il se retira dans le territoire de la Galilée,
\VS{23}et vint habiter dans la ville appelée Nazareth ; afin que s'accomplisse ce qui avait été dit par les prophètes : Il sera appelé Nazaréen.
\Chap{3}
\TextTitle{Ministère de Jean-Baptiste\FTNTT{Mc. 1:1-8 ; Lu. 3:1-20 ; Jn. 1:6-8,15-37}}
\VerseOne{}Or, en ce temps-là arriva Jean-Baptiste, prêchant dans le désert de la Judée.
\VS{2}Il disait : Repentez-vous, car le Royaume des cieux est proche.
\VS{3}Car c'est celui dont Esaïe le prophète a parlé, en disant : C'est ici la voix de celui qui crie dans le désert : Préparez le chemin du Seigneur, aplanissez ses sentiers.
\VS{4}Jean avait un vêtement de poils de chameau et une ceinture de cuir autour de ses reins. Et il se nourrissait de sauterelles et de miel sauvage.
\VS{5}Alors les habitants de Jérusalem, et de toute la Judée, et de tout le pays des environs du Jourdain, vinrent à lui ;
\VS{6}et confessant leurs péchés, ils se faisaient baptiser par lui dans le Jourdain.
\VS{7}Mais, voyant venir à son baptême beaucoup de pharisiens et de sadducéens, il leur dit : Race de vipères, qui vous a appris à fuir la colère à venir ?
\VS{8}Produisez donc des fruits convenables à la repentance
\VS{9}et ne prétendez pas dire en vous-mêmes : Nous avons Abraham pour père ! Car je vous dis que Dieu peut faire naître de ces pierres mêmes des enfants à Abraham.
\VS{10}Et déjà la cognée est mise à la racine des arbres ; c'est pourquoi tout arbre qui ne produit pas de bons fruits sera coupé et jeté au feu.
\VS{11}Pour moi, je vous baptise d'eau en signe de repentance ; mais celui qui vient après moi est plus puissant que moi, et je ne suis pas digne de porter ses souliers ; celui-là vous baptisera du Saint-Esprit et de feu\FTNT{Le baptême du Saint-Esprit ne doit pas être confondu avec la plénitude du Saint-Esprit. Le baptême est un acte définitif qui nous greffe au corps du Christ lors de la conversion (1 Co. 12:13). La plénitude consiste quant à elle en un constant renouvellement que nous devons impérativement rechercher (Ep. 5:18). Certains courants chrétiens charismatiques enseignent que le parler en langues est le signe distinctif du baptême du Saint-Esprit. Cette doctrine est basée sur au moins trois passages : Ac. 2:4 ; Ac. 10:44-46 et Ac. 19:1-7. Si cela était vraiment le cas, plusieurs chrétiens seraient encore dans leurs péchés et n'appartiendraient pas au Seigneur Jésus-Christ. En effet, Ro. 8:9 déclare ceci : « Si quelqu'un n'a pas l'Esprit de Christ, il ne lui appartient pas ». Or il est manifeste que bon nombre de chrétiens nés d'en haut ne parlent pas en langues, ce qui est d'ailleurs attesté par l'apôtre Paul (1 Co. 12:30). Il n'y a aucun verset dans les Ecritures qui nous ordonne de chercher le baptême du Saint-Esprit pour la bonne et simple raison que nous le recevons à la conversion.}.
\VS{12}Il a son van à la main, et il nettoiera entièrement son aire, et il assemblera son froment dans le grenier ; mais il brûlera la paille dans un feu qui ne s'éteint point.
\TextTitle{Jean baptise Jésus-Christ\FTNTT{Mc. 1:9-11 ; Lu. 3:21-22 ; Jn. 1:31-34}}
\VS{13}Alors Jésus vint de Galilée au Jourdain vers Jean pour être baptisé par lui.
\VS{14}Mais Jean l'en empêchait avec force en lui disant : J'ai besoin d'être baptisé par toi, et tu viens vers moi ?
\VS{15}Et Jésus répondit en disant : Laisse-moi faire pour le moment, car il nous est ainsi convenable d'accomplir tout ce qui est juste. Et alors il le laissa faire.
\VS{16}Dès que Jésus eut été baptisé, il sortit aussitôt hors de l'eau. Et voici, les cieux lui furent ouverts, et Jean vit l'Esprit de Dieu descendant comme une colombe et venant sur lui.
\VS{17}Et voici une voix du ciel déclara : Celui-ci est mon Fils bien-aimé en qui j'ai mis toute mon affection.
\Chap{4}
\TextTitle{La tentation\FTNTT{Ge. 3:6 ; Mc. 1:12-13 ; Lu. 4:1-13 ; 1 Jn. 2:16.}}
\VerseOne{}Alors Jésus fut emmené par l'Esprit dans le désert, pour être tenté par le diable.
\VS{2}Après avoir jeûné quarante jours et quarante nuits, finalement il eut faim.
\VS{3}Et le tentateur s'étant approché, lui dit : Si tu es le Fils de Dieu, ordonne que ces pierres deviennent des pains.
\VS{4}Mais Jésus répondit et dit : Il est écrit : L'homme ne vivra point de pain seulement, mais de toute parole qui sort de la bouche de Dieu\FTNT{De. 8:3.}.
\VS{5}Alors le diable le transporta dans la sainte ville et le mit sur le haut du temple ;
\VS{6}et il lui dit : Si tu es le Fils de Dieu, jette-toi en bas ; car il est écrit : Il ordonnera à ses anges de te porter sur leurs mains de peur que ton pied ne heurte contre une pierre\FTNT{Ps. 91:12-13.}.
\VS{7}Jésus lui dit : Il est aussi écrit : Tu ne tenteras point le Seigneur ton Dieu\FTNT{De. 6:16.}.
\VS{8}Le diable le transporta encore sur une forte haute montagne, et lui montra tous les royaumes du monde et leur gloire ;
\VS{9}et il lui dit : Je te donnerai toutes ces choses, si tu te prosternes et m'adores.
\VS{10}Mais Jésus lui dit : Retire-toi Satan ! Car il est écrit : Tu adoreras le Seigneur ton Dieu, et tu le serviras lui seul\FTNT{De. 6:13 ; De. 10:20.}.
\VS{11}Alors le diable le laissa. Et voici, des anges s'approchèrent et le servirent.
\TextTitle{Etablissement de Jésus à Capernaüm\FTNTT{Mc. 1:14-15 ; Lu. 4:14-15}}
\VS{12}Jésus, ayant appris que Jean avait été mis en prison, se retira dans la Galilée.
\VS{13}Et ayant quitté Nazareth, il alla demeurer à Capernaüm, ville maritime, sur les confins de Zabulon et de Nephthali ;
\VS{14}afin que s'accomplisse ce qui avait été annoncé par Esaïe, le prophète, en disant :
\VS{15}Le pays de Zabulon et le pays de Nephthali, de la contrée voisine de la mer, au-delà du Jourdain, et la Galilée des Gentils ;
\VS{16}Ce peuple, assis dans les ténèbres, a vu une grande lumière ; et à ceux qui étaient assis dans la région et l'ombre de la mort, la lumière elle-même s'est levée\FTNT{Es. 9:1.}.
\VS{17}Dès lors, Jésus commença à prêcher et à dire : Repentez-vous, car le Royaume des cieux est proche.
\TextTitle{Appel de Pierre, André, Jacques et Jean\FTNTT{Mc. 1:16-20 ; Lu. 5:1-11 ; Jn. 1:35-51}}
\VS{18}Comme Jésus marchait le long de la mer de Galilée, il vit deux frères, Simon, appelé Pierre, et André, son frère, qui jetaient leurs filets dans la mer ; car ils étaient pêcheurs.
\VS{19}Et il leur dit : Suivez-moi et je vous ferai pêcheurs d'hommes.
\VS{20}Et ayant aussitôt quitté leurs filets, ils le suivirent.
\VS{21}Et de là étant allé plus en avant, il vit deux autres frères, Jacques, fils de Zébédée, et Jean, son frère, dans une barque, avec Zébédée leur père, qui réparaient leurs filets, et il les appela.
\VS{22}Et ayant aussitôt quitté leur barque et leur père, ils le suivirent.
\TextTitle{Ministère de Jésus-Christ en Galilée}
\VS{23}Jésus allait par toute la Galilée, enseignant dans leurs synagogues, prêchant l'Evangile du Royaume, et guérissant toutes sortes de maladies, et toutes sortes d'infirmités parmi le peuple.
\VS{24}Et sa renommée se répandit par toute la Syrie ; et on lui présentait tous ceux qui se portaient mal, tourmentés de diverses maladies, des démoniaques, des lunatiques, des paralytiques ; et il les guérissait.
\VS{25}Une grande foule le suivit, de Galilée, de la Décapole, de Jérusalem, de Judée et au-delà du Jourdain.
\Chap{5}
\TextTitle{L'enseignement de Jésus sur la montagne\FTNTT{Lu. 6:20-49 ; Mt. 5:1-48}}
\VerseOne{}Voyant la foule, Jésus monta sur la montagne ; puis s'étant assis, ses disciples s'approchèrent de lui.
\VS{2}Puis, ayant ouvert la bouche, il les enseigna de la sorte :
\VS{3}Heureux les pauvres en esprit, car le Royaume des cieux est à eux.
\VS{4}Heureux ceux qui pleurent, car ils seront consolés.
\VS{5}Heureux les humbles, car ils hériteront la terre.
\VS{6}Heureux ceux qui ont faim et soif de la justice, car ils seront rassasiés.
\VS{7}Heureux les miséricordieux, car ils obtiendront miséricorde.
\VS{8}Heureux sont ceux qui sont purs de cœur, car ils verront Dieu.
\VS{9}Heureux ceux qui procurent la paix, car ils seront appelés enfants de Dieu.
\VS{10}Heureux ceux qui sont persécutés pour la justice, car le Royaume des cieux est à eux.
\VS{11}Heureux serez-vous lorsqu'on vous outragera, qu'on vous persécutera et qu'on dira faussement de vous toute sorte de mal à cause de moi.
\VS{12}Réjouissez-vous et soyez dans l'allégresse, parce que votre récompense sera grande dans les cieux ; car c'est ainsi qu'on a persécuté les prophètes qui ont été avant vous.
\TextTitle{Le sel de la terre et la lumière du monde\FTNTT{Mc. 4:21-23 ; Lu. 8:16-18 ; 11:33-36}}
\VS{13}Vous êtes le sel de la terre mais si le sel perd sa saveur, avec quoi le salera-t-on ? Il ne sert plus qu'à être jeté dehors, et foulé aux pieds par les hommes.
\VS{14}Vous êtes la lumière du monde. Une ville située sur une montagne ne peut être cachée,
\VS{15}et on n'allume point la lampe pour la mettre sous un boisseau, mais sur un chandelier et elle éclaire tous ceux qui sont dans la maison.
\VS{16}Ainsi, que votre lumière luise devant les hommes, afin qu'ils voient vos bonnes œuvres et qu'ils glorifient votre Père qui est dans les cieux.
\TextTitle{Le Messie et la loi}
\VS{17}Ne croyez pas que je sois venu abolir la loi ou les prophètes ; je ne suis pas venu les abolir, mais les accomplir.
\VS{18}Car, je vous le dis en vérité, tant que le ciel et la terre ne passeront point, il ne disparaîtra pas de la loi un seul iota ou un seul trait de lettre jusqu'à ce que tout soit arrivé.
\VS{19}Celui donc qui aura violé l'un de ces petits commandements, et qui aura enseigné les hommes à faire de même, sera appelé le plus petit au Royaume des cieux ; mais celui qui les observera et qui enseignera à les observer, celui-là sera appelé grand au Royaume des cieux.
\VS{20}Car je vous dis que si votre justice ne surpasse celle des scribes et des pharisiens, vous n'entrerez point dans le Royaume des cieux.
\VS{21}Vous avez entendu qu'il a été dit aux anciens : Tu ne tueras point, et celui qui tuera, sera puni par les juges.
\VS{22}Mais moi je vous dis que quiconque se met en colère sans cause contre son frère sera puni par les juges ; et celui qui dira à son frère : Raca\FTNT{Raca : Expression de mépris utilisée parmis les Juifs au temps de Jésus signifiant vide, indigne ou encore vaurien.} ! sera puni par le conseil ; et celui qui lui dira : Insensé ! sera puni par le feu de la géhenne\FTNT{La géhenne ou le lac de feu : Voir commentaire en Ap. 20:14.}.
\VS{23}Si donc tu apportes ton offrande à l'autel, et que là tu te souviennes que ton frère a quelque chose contre toi,
\VS{24}laisse là ton offrande devant l'autel et va te réconcilier d'abord avec ton frère, puis viens et offre ton offrande.
\VS{25}Accorde-toi rapidement avec ta partie adverse, tandis que tu es en chemin avec elle ; de peur que ta partie adverse ne te livre au juge, et que le juge ne te livre à l'officier de justice, et que tu ne sois mis en prison.
\VS{26}En vérité, je te dis, tu ne sortiras point de là, jusqu'à ce que tu n'aies payé le dernier quart de sou.
\TextTitle{Convoitise, adultère et divorce\FTNTT{Mt. 19:3-11 ; Mc. 10:2-12 ; 1 Co. 7:1-16}}
\VS{27}Vous avez entendu qu'il a été dit aux anciens : Tu ne commettras point d'adultère.
\VS{28}Mais moi, je vous dis que quiconque regarde une femme pour la convoiter a déjà commis dans son cœur un adultère avec elle.
\VS{29}Si ton œil droit est pour toi une occasion de chute, arrache-le et jette-le loin de toi ; car il est avantageux pour toi qu'un seul de tes membres périsse et que ton corps entier ne soit pas jeté dans la géhenne.
\VS{30}Si ta main droite est pour toi une occasion de chute, coupe-la et jette-la loin de toi ; car il est avantageux pour toi qu'un seul de tes membres périsse et que ton corps entier ne soit pas jeté dans la géhenne.
\VS{31}Il a été dit encore : Si quelqu'un répudie sa femme, qu'il lui donne une lettre de divorce.
\VS{32}Mais moi, je vous dis que celui qui répudie sa femme, si ce n'est pour cause d'adultère, l'expose à devenir adultère ; et que celui qui épouse une femme répudiée commet un adultère.
\TextTitle{Acquittement des promesses faites au Seigneur ; attitude face à son prochain}
\VS{33}Vous avez aussi appris qu'il a été dit aux anciens : Tu ne te parjureras point, mais tu rendras au Seigneur ce que tu auras promis par serment.
\VS{34}Mais moi, je vous dis de ne jurer aucunement, ni par le ciel, parce que c'est le trône de Dieu ;
\VS{35}ni par la terre, parce que c'est le marchepied de ses pieds, ni par Jérusalem, parce que c'est la ville du grand Roi.
\VS{36}Ne jure pas non plus par ta tête, car tu ne peux pas rendre blanc ou noir un seul cheveu.
\VS{37}Mais que votre parole soit : Oui, oui ; non, non ; car ce qui est de plus, vient du malin.
\VS{38}Vous avez appris qu'il a été dit : Œil pour œil et dent pour dent.
\VS{39}Mais moi, je vous dis : Ne résistez point au méchant. Si quelqu'un te frappe sur ta joue droite, présente-lui aussi l'autre.
\VS{40}Si quelqu'un veut plaider contre toi, et prendre ta tunique, laisse-lui encore ton manteau.
\VS{41}Si quelqu'un te force à faire un mille, fais-en deux avec lui.
\VS{42}Donne à celui qui te demande et ne te détourne point de celui qui veut emprunter de toi.
\TextTitle{Le standard de l'Amour\FTNTT{Lu. 6:27-36}}
\VS{43}Vous avez appris qu'il a été dit : Tu aimeras ton prochain et tu haïras ton ennemi.
\VS{44}Mais moi, je vous dis : Aimez vos ennemis, bénissez ceux qui vous maudissent, faites du bien à ceux qui vous haïssent, et priez pour ceux qui vous maltraitent et vous persécutent,
\VS{45}afin que vous soyez les fils de votre Père qui est dans les cieux ; car il fait lever son soleil sur les méchants et sur les gens de bien, et il envoie sa pluie sur les justes et sur les injustes.
\VS{46}Car si vous aimez seulement ceux qui vous aiment, quelle récompense en aurez-vous ? Les publicains aussi n'en font-ils pas tout autant ?
\VS{47}Et si vous faites accueil seulement à vos frères, que faites-vous de plus que les autres ? Les publicains aussi ne le font-ils pas de même ?
\VS{48}Soyez donc parfaits, comme votre Père qui est dans les cieux est parfait.
\Chap{6}
\TextTitle{Jésus condamne l'hypocrisie}
\VerseOne{}Gardez-vous de pratiquer votre justice devant les hommes, pour en être vus ; autrement, vous ne recevrez point la récompense de votre Père qui est dans les cieux.
\VS{2}Donc, lorsque tu fais ton aumône, ne fais point sonner la trompette devant toi, comme font les hypocrites dans les synagogues et dans les rues, afin d'être glorifiés par les hommes. Je vous le dis en vérité, ils reçoivent leur récompense.
\VS{3}Mais quand tu fais ton aumône, que ta main gauche ne sache pas ce que fait ta droite ;
\VS{4}afin que ton aumône se fasse en secret, et ton Père, qui voit ce qui se fait dans le secret, te récompensera publiquement.
\VS{5}Et quand tu pries, ne sois point comme les hypocrites ; car ils aiment à prier en se tenant debout dans les synagogues et aux coins des rues, pour être vus des hommes. Je vous le dis en vérité, ils reçoivent leur récompense.
\VS{6}Mais toi, quand tu pries, entre dans ta chambre, et ayant fermé ta porte, prie ton Père, qui est là dans ce lieu secret ; et ton Père qui te voit dans ce lieu secret, te récompensera publiquement.
\VS{7}Quand vous priez, ne multipliez pas de vaines paroles, comme font les Gentils ; car ils s'imaginent qu'à force de paroles ils seront exaucés.
\VS{8}Ne leur ressemblez donc point ; car votre Père sait de quoi vous avez besoin, avant que vous le lui demandiez.
\TextTitle{Instructions de Jésus sur la prière}
\VS{9}Voici donc comment vous devez prier : Notre Père qui es aux cieux, que ton Nom soit sanctifié.
\VS{10}Que ton règne vienne. Que ta volonté soit faite sur la terre comme au ciel.
\VS{11}Donne-nous aujourd'hui notre pain quotidien.
\VS{12}Et remets nous nos dettes\FTNT{Du grec « opheilema » : « ce qui est légalement dû, une dette ». Les Ecritures considèrent le péché comme une dette. Voir Mat. 18:21-35.}, comme nous aussi nous remettons les dettes à nos débiteurs.
\VS{13}Ne nous induis pas en tentation ; mais délivre-nous du mal. Car c'est à toi qu'appartiennent, dans tous les siècles, le règne, et la puissance et la gloire. Amen !
\VS{14}Car si vous pardonnez aux hommes leurs offenses, votre Père céleste vous pardonnera aussi les vôtres.
\VS{15}Mais si vous ne pardonnez point aux hommes leurs offenses, votre Père ne vous pardonnera point non plus vos offenses.
\TextTitle{Attitude pendant le jeûne}
\VS{16}Et quand vous jeûnerez, ne prenez pas un air triste, comme font les hypocrites ; car ils se rendent le visage tout défait, afin de montrer aux hommes qu'ils jeûnent. Je vous le dis en vérité, ils reçoivent leur récompense.
\VS{17}Mais toi, quand tu jeûnes, oint ta tête et lave ton visage,
\VS{18}afin qu'il ne paraisse pas aux hommes que tu jeûnes, mais à ton Père qui est présent dans ton lieu secret ; et ton Père qui te voit dans ton lieu secret, te récompensera publiquement.
\TextTitle{Le trésor selon Dieu}
\VS{19}Ne vous amassez point des trésors sur la terre, où les vers et la rouille détruisent, et où les voleurs percent et dérobent.
\VS{20}Mais amassez-vous des trésors dans le ciel, où les vers et la rouille ne détruisent point, et où les voleurs ne percent ni ne dérobent.
\VS{21}Car là où est ton trésor, là aussi sera ton cœur.
\VS{22}L'œil est la lampe du corps. Si donc ton œil est en bon état, tout ton corps sera éclairé.
\VS{23}Mais si ton œil est mal disposé, tout ton corps sera ténébreux. Si donc la lumière qui est en toi n'est que ténèbres, combien seront grandes les ténèbres même ?
\VS{24}Nul ne peut servir deux maîtres. Car, ou il haïra l'un et aimera l'autre ; ou il s'attachera à l'un et méprisera l'autre. Vous ne pouvez pas servir Dieu et Mammon\FTNT{Mammon : Mot d'origine araméenne signifiant « riche ». Certains le rapprochent de l'hébreu « matmon » signifiant « trésor, argent ». D'autres le rapprochent du phénicien « mommon » signifiant « bénéfice ». Dans les évangiles, il signifie « possession » (matérielle), mais il est parfois personnifié.}.
\TextTitle{Rechercher le Royaume}
\VS{25}C'est pourquoi je vous dis : Ne vous inquiétez pas pour votre vie, de ce que vous mangerez, et de ce que vous boirez ; ni pour votre corps, de quoi vous serez vêtus. La vie n'est-elle pas plus que la nourriture et le corps plus que le vêtement ?
\VS{26}Considérez les oiseaux du ciel ; car ils ne sèment, ni ne moissonnent, ni n'assemblent dans des greniers, et cependant votre Père céleste les nourrit. N'êtes vous pas beaucoup plus excellents qu'eux ?
\VS{27}Et qui est celui d'entre vous qui puisse, par ses inquiétudes, ajouter une coudée à sa taille ?
\VS{28}Et pourquoi vous inquiéter au sujet du vêtement ? Apprenez comment croissent les lis des champs : Ils ne travaillent ni ne filent ;
\VS{29}cependant je vous dis que Salomon même, dans toute sa gloire, n'a pas été vêtu comme l'un d'eux.
\VS{30}Si donc Dieu revêt ainsi l'herbe des champs, qui est aujourd'hui sur pied, et qui demain sera jetée au four, ne vous vêtira-t-il pas à plus forte raison, ô gens de petite foi ?
\VS{31}Ne vous inquiétez donc point en disant : Que mangerons-nous ? Ou, que boirons-nous ? Ou de quoi serons-nous vêtus ?
\VS{32}Vu que les païens recherchent toutes ces choses; car votre Père céleste sait que vous avez besoin de toutes ces choses.
\VS{33}Mais cherchez premièrement le Royaume de Dieu et sa justice, et toutes ces choses vous seront données par-dessus.
\VS{34}Ne vous inquiétez donc pas pour le lendemain ; car le lendemain prendra soin de lui-même. A chaque jour suffit sa peine.
\Chap{7}
\TextTitle{Le jugement hypocrite\FTNTT{Lu. 6:37-42}}
\VerseOne{}Ne jugez point afin que vous ne soyez point jugés.
\VS{2}Car de tel jugement que vous jugez, vous serez jugés ; et de telle mesure que vous mesurerez, on vous mesurera réciproquement.
\VS{3}Et pourquoi vois-tu la paille qui est dans l'œil de ton frère, et n'aperçois-tu pas la poutre qui est dans ton œil ?
\VS{4}Ou comment peux-tu dire à ton frère : Permets que j'ôte de ton œil cette paille et n'aperçois-tu pas la poutre dans ton œil ?
\VS{5}Hypocrite, ôte premièrement de ton œil la poutre, et après cela tu verras comment tu ôteras la paille de l'œil de ton frère.
\VS{6}Ne donnez point les choses saintes aux chiens et ne jetez point vos perles devant les pourceaux, de peur qu'ils ne les foulent aux pieds, ne se retournent et ne vous déchirent.
\TextTitle{Exhortation à la prière}
\VS{7}Demandez et il vous sera donné. Cherchez et vous trouverez. Frappez et l'on vous ouvrira.
\VS{8}Car quiconque demande, reçoit ; et celui qui cherche trouve ; et l'on ouvre à celui qui frappe.
\VS{9}Lequel de vous donnera une pierre à son fils, s'il lui demande du pain ?
\VS{10}Ou, s'il lui demande un poisson, lui donnera-t-il un serpent ?
\VS{11}Si donc vous, méchants comme vous l'êtes, savez donner à vos enfants de bonnes choses, à combien plus forte raison votre Père qui est dans les cieux, donnera-t-il des bonnes choses à ceux qui les lui demandent ?
\TextTitle{La règle d'or de la loi et des prophètes\FTNTT{Lu. 6:31; Ep.4:32}}
\VS{12}Tout ce que vous voulez que les hommes fassent pour vous, faites-le de même pour eux, car c'est la loi et les prophètes.
\TextTitle{Les deux chemins\FTNTT{Ps. 1}}
\VS{13}Entrez par la porte étroite, car c'est la porte large et le chemin spacieux qui mènent à la perdition, et il y en a beaucoup qui entrent par elle.
\VS{14}Mais étroite est la porte, resserré le chemin qui mènent à la vie, et il y en a peu qui les trouvent.
\TextTitle{Les faux prophètes, reconnaissables à leurs fruits\FTNTT{Lu. 6:43-45}}
\VS{15}Gardez-vous des faux prophètes, ils viennent à vous en habits de brebis, mais au-dedans ce sont des loups ravisseurs.
\VS{16}Vous les reconnaîtrez à leurs fruits. Cueille-t-on des raisins sur des épines, ou des figues sur des chardons ?
\VS{17}Ainsi tout bon arbre porte de bons fruits ; mais le mauvais arbre porte de mauvais fruits.
\VS{18}Un bon arbre ne peut porter de mauvais fruits ni le mauvais arbre porter de bons fruits.
\VS{19}Tout arbre qui ne porte pas de bons fruits est coupé et jeté au feu.
\VS{20}Vous les reconnaîtrez donc à leurs fruits.
\TextTitle{Fausse confession\FTNTT{Lu. 6:46}}
\VS{21}Ceux qui me disent : Seigneur ! Seigneur ! N'entreront pas tous dans le Royaume des cieux ; mais celui qui fait la volonté de mon Père qui est dans les cieux.
\VS{22}Plusieurs me diront en ce jour-là : Seigneur ! Seigneur ! N'avons-nous pas prophétisé en ton Nom ? N'avons-nous pas chassé les démons en ton Nom ? N'avons-nous pas fait beaucoup de miracles en ton Nom ?
\VS{23}Alors je leur dirai ouvertement : Je ne vous ai jamais connus. Retirez-vous de moi, vous qui commettez l'iniquité.
\TextTitle{Parabole des deux bâtisseurs et des deux fondements\FTNTT{Lu. 6:47-49}}
\VS{24}Quiconque entend ces paroles que je dis, et les met en pratique, je le comparerai à un homme prudent qui a bâti sa maison sur le roc.
\VS{25}La pluie est tombée, les torrents sont venus, les vents ont soufflé contre cette maison : Elle n'est point tombée parce qu'elle était fondée sur le roc\FTNT{Jésus-Christ le Rocher : voir Es. 8:13-17.}.
\VS{26}Mais quiconque entend ces paroles que je dis et ne les met point en pratique, sera semblable à un homme insensé qui a bâti sa maison sur le sable.
\VS{27}La pluie est tombée, les torrents sont venus, les vents ont soufflé contre cette maison : Elle est tombée et sa ruine a été grande.
\TextTitle{Effet de l'enseignement}
\VS{28}Or il arriva que quand Jésus eut achevé ce discours, la foule fut frappée de sa doctrine ;
\VS{29}car il les enseignait comme ayant de l'autorité et non comme les scribes.
\Chap{8}
\TextTitle{Le lépreux guérit\FTNTT{Mc. 1:40-45}}
\VerseOne{}Et quand il fut descendu de la montagne, de grandes foules le suivirent.
\VS{2}Et voici, un lépreux vint et se prosterna devant lui, en lui disant : Seigneur\FTNT{Seigneur : Du grec « kurios ». C'est la première fois que ce terme est appliqué à Jésus. Notez que c'est un lépreux qui a eu la révélation que Jésus-Christ est YHWH.}, si tu veux, tu peux me rendre pur.
\VS{3}Et Jésus étendit la main, le toucha, en disant : Je le veux, sois pur. A l'instant même il fut purifié de sa lèpre.
\VS{4}Puis Jésus lui dit : Prends garde de ne le dire à personne ; mais va te montrer au sacrificateur et offre l'offrande que Moïse a prescrite afin que cela leur serve de témoignage.
\TextTitle{Guérison du serviteur d'un centenier\FTNTT{Lu. 7:1-10}}
\VS{5}Et quand Jésus fut entré dans Capernaüm, un centenier vint à lui, le priant
\VS{6}et disant : Seigneur, mon serviteur qui est paralytique est couché à la maison et il souffre extrêmement.
\VS{7}Jésus lui dit : J'irai et je le guérirai.
\VS{8}Mais le centenier lui répondit : Seigneur, je ne suis pas digne que tu entres sous mon toit ; mais dis seulement une parole et mon serviteur sera guéri.
\VS{9}Car moi-même qui suis un homme soumis à l'autorité d'un autre, j'ai des soldats sous mes ordres, et je dis à l'un : Va ! et il va ; et à un autre : Viens ! et il vient ; et à mon serviteur : Fais cela ! et il le fait.
\VS{10}Après l'avoir entendu, Jésus fut étonné et dit à ceux qui le suivaient : Je vous le dis en vérité, même en Israël je n'ai pas trouvé une aussi grande foi.
\VS{11}Or, je vous dis que plusieurs viendront de l'orient et de l'occident, et seront à table dans le Royaume des cieux, avec Abraham, Isaac et Jacob.
\VS{12}Et les enfants du Royaume seront jetés dans les ténèbres du dehors, où il y aura des pleurs et des grincements de dents.
\VS{13}Alors Jésus dit au centenier : Va, et qu'il te soit fait selon ta foi. Et à l'heure même, son serviteur fut guéri.
\TextTitle{Guérison de la belle-mère de Pierre\FTNTT{Mc. 1:29-34 ; Lu. 4:38-41}}
\VS{14}Puis Jésus alla à la maison de Pierre, dont il vit la belle-mère couchée et ayant la fièvre.
\VS{15}Il toucha sa main, et la fièvre la quitta ; puis elle se leva, et les servit.
\VS{16}Et le soir étant venu, on lui amena plusieurs démoniaques. Et il chassa par sa parole les esprits malins, et guérit tous ceux qui étaient malades,
\VS{17}afin que s'accomplisse ce qui avait été annoncé par Esaïe le prophète, en disant : Il a pris nos faiblesses et a porté nos maladies\FTNT{Es. 53:4.}.
\TextTitle{Les disciples éprouvés dans leur consécration\FTNTT{Lu. 9:57-62}}
\VS{18}Or Jésus voyant autour de lui de grandes foules, donna l'ordre de passer à l'autre rive.
\VS{19}Et un scribe s'approchant, lui dit : Maître, je te suivrai partout où tu iras.
\VS{20}Jésus lui dit : Les renards ont des tanières, et les oiseaux du ciel ont des nids ; mais le Fils de l'homme n'a pas de place pour reposer sa tête.
\VS{21}Puis un autre de ses disciples lui dit : Seigneur, permets-moi d'aller d'abord ensevelir mon père.
\VS{22}Et Jésus lui dit : Suis-moi et laisse les morts ensevelir leurs morts.
\TextTitle{Autorité de Jésus face à la tempête\FTNTT{Mc. 4:35-41 ; Lu. 8:22-25}}
\VS{23}Il monta dans la barque et ses disciples le suivirent.
\VS{24}Et voici, il s'éleva sur la mer une si grande tempête que la barque était couverte de flots ; et Jésus dormait.
\VS{25}Et ses disciples vinrent le réveiller en lui disant : Seigneur, sauve-nous, nous périssons !
\VS{26}Et il leur dit : Pourquoi avez-vous peur, gens de peu de foi ? Alors s'étant levé, il menaça les vents et la mer, et il se fit un grand calme.
\VS{27}Et les gens qui étaient là furent étonnés, et dirent : Qui est celui-ci à qui obéissent même les vents et la mer ?
\TextTitle{Deux aveugles et un démoniaque guéris\FTNTT{Mc. 5:1-20 ; Lu. 8:26-40}}
\VS{28}Et quand il fut passé de l'autre côté, dans le pays des Gadaréniens, deux démoniaques sortant des sépulcres, vinrent le rencontrer. Ils étaient si dangereux que personne ne pouvait passer par ce chemin-là.
\VS{29}Et voici, ils s'écrièrent : Qu'y a-t-il entre nous et toi, Jésus Fils de Dieu ? Es-tu venu ici nous tourmenter avant le temps ?
\VS{30}Et il y avait loin d'eux un grand troupeau de pourceaux qui paissaient.
\VS{31}Et les démons le priaient en disant : Si tu nous chasses dehors, permets-nous d'entrer dans ce troupeau de pourceaux.
\VS{32}Et il leur dit : Allez ! Et ils sortirent et entrèrent dans le troupeau de pourceaux. Et voici, tout le troupeau de pourceaux se précipita des pentes escarpées dans la mer et ils périrent dans les eaux.
\VS{33}Ceux qui les gardaient s'enfuirent et allèrent dans la ville, ils racontèrent toutes ces choses et ce qui était arrivé aux démoniaques.
\VS{34}Et voici, toute la ville alla à la rencontre de Jésus, et l'ayant vu, ils le prièrent de se retirer de leur pays.
\Chap{9}
\TextTitle{Un paralytique guéri\FTNTT{Mc. 2:3-12 ; Lu. 5:18-26}}
\VerseOne{}Alors, étant monté dans une barque, il traversa la mer et vint dans sa ville.
\VS{2}Et voici, on lui présenta un paralytique couché sur un lit. Et Jésus voyant leur foi, dit au paralytique : Prends courage, mon enfant ! Tes péchés te sont pardonnés.
\VS{3}Et voici, quelques-uns des scribes disaient au dedans d'eux : Cet homme blasphème.
\VS{4}Mais Jésus, connaissant leurs pensées, leur dit : Pourquoi avez-vous de mauvaises pensées dans vos cœurs ?
\VS{5}Car lequel est le plus aisé de dire : Tes péchés te sont pardonnés ; ou de dire : Lève-toi et marche ?
\VS{6}Or afin que vous sachiez que le Fils de l'homme a le pouvoir sur la terre de pardonner les péchés : Lève-toi, dit-il au paralytique, prends ton lit et va dans ta maison.
\VS{7}Et il se leva et s'en alla dans sa maison.
\VS{8}Quand la foule vit cela, elle fut saisie d'étonnement, et elle glorifiait Dieu qui a donné aux hommes un tel pouvoir.
\TextTitle{L'appel de Matthieu\FTNTT{Mc. 2:14 ; Lu. 5:27-28}}
\VS{9}De là, étant allé plus loin, Jésus vit un homme nommé Matthieu, assis au bureau du péage et il lui dit : Suis-moi ; et il se leva et le suivit.
\TextTitle{L'appel des pécheurs\FTNTT{Mc. 2:15-20 ; Lu. 5:29-35}}
\VS{10}Comme Jésus était à table dans la maison de Matthieu, beaucoup de publicains et des gens de mauvaise vie, qui étaient venus là, se mirent à table avec Jésus et avec ses disciples.
\VS{11}Les pharisiens virent cela et ils dirent à ses disciples : Pourquoi votre Maître mange-t-il avec les publicains et les gens de mauvaise vie ?
\VS{12}Jésus l'ayant entendu, leur dit : Ce ne sont pas ceux qui sont en bonne santé qui ont besoin de médecin, mais les malades.
\VS{13}Mais allez et apprenez ce que veulent dire ces paroles : Je prends plaisir à la miséricorde et non aux sacrifices\FTNT{Os. 6:6.}. Car je ne suis pas venu appeler à la repentance les justes, mais les pécheurs.
\VS{14}Alors les disciples de Jean vinrent auprès de lui et lui dirent : Pourquoi nous et les pharisiens jeûnons-nous souvent, tandis que tes disciples ne jeûnent point ?
\VS{15}Et Jésus leur répondit : Les amis de l'époux peuvent-ils s'affliger pendant que l'époux est avec eux ? Mais les jours viendront où l'époux leur sera enlevé, alors ils jeûneront.
\TextTitle{Parabole du drap neuf et des outres neuves\FTNTT{Mc. 2:21-22 ; Lu. 5:36-39}}
\VS{16}Aussi personne ne met une pièce de drap neuf à un vieil habit car la pièce emporterait une partie de l'habit et la déchirure serait pire.
\VS{17}On ne met pas non plus du vin nouveau dans de vieilles outres ; autrement les outres se rompent, et le vin se répand, et les outres sont perdues ; mais on met le vin nouveau dans des outres neuves, et l'un et l'autre se conservent.
\TextTitle{Résurrection de la fille de Jaïrus et guérison de la femme à la perte de sang\FTNTT{Mc. 5:21-43 ; Lu. 8:41-56}}
\VS{18}Tandis qu'il leur disait ces choses, voici, arriva un chef qui se prosterna devant lui, en lui disant : Ma fille est morte il y a un instant, mais viens, et impose-lui ta main et elle vivra.
\VS{19}Et Jésus s'étant levé le suivit avec ses disciples.
\VS{20}Et voici, une femme atteinte d'une perte de sang depuis douze ans s'approcha par-derrière et toucha le bord de son vêtement.
\VS{21}Car elle disait en elle-même : Si je puis seulement toucher son vêtement, je serai guérie.
\VS{22}Et Jésus se retourna, et dit en la voyant : Prends courage, ma fille ! Ta foi t'a sauvée. Et cette femme fut guérie à l'heure même.
\VS{23}Lorsque Jésus fut arrivé à la maison du chef et qu'il vit les joueurs de flûte et une foule bruyante,
\VS{24}il leur dit : Retirez-vous car la jeune fille n'est pas morte, mais elle dort ; et ils se moquaient de lui.
\VS{25}Quand la foule eut été renvoyée, il entra, prit la main de la jeune fille et elle se leva.
\VS{26}Et le bruit s'en répandit dans toute la contrée.
\TextTitle{Deux aveugles et un démoniaque guéris}
\VS{27}Etant parti de là, Jésus fut suivi par deux aveugles qui criaient : Fils de David, aie pitié de nous !
\VS{28}Et quand il fut arrivé dans la maison, les aveugles s'approchèrent de lui et Jésus leur dit : Croyez-vous que je puisse faire ce que vous me demandez ? Ils lui répondirent : Oui, Seigneur !
\VS{29}Alors il toucha leurs yeux en disant : Qu'il vous soit fait selon votre foi.
\VS{30}Et leurs yeux s'ouvrirent. Alors Jésus leur dit sévèrement : Prenez garde que personne ne le sache.
\VS{31}Mais, dès qu'ils furent sortis, ils répandirent sa renommée dans tout le pays.
\VS{32}Comme ils s'en allaient, voici, on présenta à Jésus un homme muet et démoniaque.
\VS{33}Et le démon ayant été chassé, le muet parla ; et les foules étonnées disaient : Jamais pareille chose ne s'est vue en Israël.
\VS{34}Mais les pharisiens disaient : Il chasse les démons par le prince des démons.
\VS{35}Jésus allait dans toutes les villes et les villages, enseignant dans leurs synagogues, et prêchant l'Evangile du Royaume, et guérissant toutes sortes de maladies et toutes sortes d'infirmités parmi le peuple.
\TextTitle{Jésus ému de compassion pour la foule\FTNTT{Mc. 6:34}}
\VS{36}Et voyant les foules, il fut ému de compassion, parce qu'elles étaient dispersées et errantes comme des brebis qui n'ont point de pasteur.
\VS{37}Et il dit à ses disciples : La moisson est grande, mais il y a peu d'ouvriers.
\VS{38}Priez donc le Maître de la moisson d'envoyer des ouvriers dans sa moisson.
\Chap{10}
\TextTitle{Appel et mission des douze apôtres\FTNTT{Mc. 6:7-13 ; Lu. 9:1-6}}
\VerseOne{}Alors Jésus ayant appelé ses douze disciples, leur donna le pouvoir de chasser les esprits impurs et de guérir toutes sortes de maladies et toutes sortes d'infirmités.
\VS{2}Et voici les noms des douze apôtres : Le premier est Simon, nommé Pierre, et André son frère ; Jacques, fils de Zébédée, et Jean, son frère ;
\VS{3}Philippe et Barthélemy ; Thomas, et Matthieu le péager ; Jacques, fils d'Alphée, et Lebbée, surnommé Thaddée.
\VS{4}Simon le Cananite, et Judas Iscariot, celui qui le livra.
\VS{5}Tels sont les douze que Jésus envoya, et leur donna ses ordres en disant : N'allez point vers les Gentils et n'entrez point dans aucune ville des Samaritains ;
\VS{6}Mais allez plutôt vers les brebis perdues de la maison d'Israël.
\VS{7}Et quand vous serez partis, prêchez, en disant : Le Royaume des cieux est proche.
\VS{8}Guérissez les malades, rendez purs les lépreux, ressuscitez les morts, chassez les démons hors des possédés. Vous l'avez reçu gratuitement, donnez-le gratuitement\FTNT{Vous avez reçu gratuitement : Aucun chrétien, quel que soit son appel ou son don ne peut prétendre qu'il a payé pour avoir les talents qu'il a reçus du Seigneur. Dans 1 Co. 4:7 Paul nous pose une question : « Qu'as-tu que tu n'aies reçu et si tu l'as reçu pourquoi te glorifies-tu ? » Dieu interroge également Job : « De qui suis-je le débiteur ? » (Job 41:2). Vendre quelque chose qu'on a reçu gratuitement n'est rien d'autre que du vol. Donnez gratuitement : C'est la suite logique des choses, on reçoit gratuitement et on donne gratuitement. Si nous aimons Dieu, nous devons garder sa Parole et marcher comme lui a marché (Jn. 14:15 ; 1 Jn. 2:6). Il a donné ses enseignements et nourri les gens gratuitement. Dans Ap. 21:6 et 22:17, le Seigneur invite toutes les personnes qui ont soif à venir s'abreuver gratuitement. Alors pourquoi vendre la Parole qu'on a reçue gratuitement ? Le Seigneur a envoyé les douze en mission et leur a demandé d'apporter l'évangile du Royaume, de guérir les malades et de délivrer les possédés gratuitement (Ac. 8:18-24 ; Ac. 20:33-35 ; Ap. 21:6 ; Ap. 22:17).}.
\VS{9}Ne prenez ni or, ni argent, ni monnaie dans vos ceintures ;
\VS{10}ni de sac pour le voyage, ni deux tuniques, ni souliers, ni bâton ; car l'ouvrier mérite sa nourriture.
\VS{11}Et dans quelque ville ou village que vous entriez, informez-vous qui y est digne de vous loger ; et demeurez chez lui jusqu'à ce que vous partiez de là.
\VS{12}Et quand vous entrerez dans quelque maison, saluez-la.
\VS{13}Et si cette maison en est digne, que votre paix vienne sur elle ; mais si elle n'en est pas digne, que votre paix retourne à vous.
\VS{14}Mais lorsque quelqu'un ne vous recevra point et n'écoutera point vos paroles, secouez, en partant de cette maison ou de cette ville, la poussière de vos pieds.
\VS{15}Je vous dis en vérité que ceux du pays de Sodome et de Gomorrhe seront traités moins rigoureusement au jour du jugement que cette ville-là.
\TextTitle{La proclamation du Royaume avant le retour du Messie}
\VS{16}Voici, je vous envoie comme des brebis au milieu des loups ; soyez donc prudents comme des serpents et simples comme des colombes.
\VS{17}Et mettez-vous en garde contre les hommes ; car ils vous livreront aux tribunaux et vous battront de verges dans leurs synagogues.
\VS{18}Et vous serez menés devant des gouverneurs et même devant des rois, à cause de moi, pour rendre témoignage de moi devant eux et aux nations.
\VS{19}Mais, quand ils vous livreront, ne vous inquiétez pas de ce que vous aurez à dire, ni comment vous parlerez. Ce que vous aurez à dire vous sera donné à l'heure même.
\VS{20}Car ce n'est pas vous qui parlez, mais c'est l'Esprit de votre Père qui parlera en vous.
\VS{21}Le frère livrera son frère à la mort, et le père son enfant ; et les enfants s'élèveront contre leurs pères et leurs mères, et les feront mourir.
\VS{22}Et vous serez haïs de tous à cause de mon Nom ; mais celui qui persévérera jusqu'à la fin sera sauvé.
\VS{23}Quand ils vous persécuteront dans une ville, fuyez dans une autre. Je vous le dis en vérité, vous n'aurez pas achevé de parcourir toutes les villes d'Israël, que le Fils de l'homme sera venu.
\TextTitle{La consécration du disciple et sa récompense}
\VS{24}Le disciple n'est point au-dessus du maître, ni le serviteur au-dessus de son seigneur.
\VS{25}Il suffit au disciple d'être traité comme son maître, et au serviteur comme son seigneur. S'ils ont appelé le père de famille Béelzébul, à combien plus forte raison appelleront-ils ainsi ses domestiques ?
\VS{26}Ne les craignez donc point. Car il n'y a rien de caché qui ne doive être découvert, ni rien de secret qui ne doive être connu.
\VS{27}Ce que je vous dis dans les ténèbres, dites-le dans la lumière ; et ce que je vous dis à l'oreille, prêchez-le sur les toits.
\VS{28}Et ne craignez point ceux qui tuent le corps et qui ne peuvent tuer l'âme ; mais craignez plutôt celui qui peut faire périr et l'âme et le corps en les jetant dans la géhenne.
\VS{29}Ne vend-on pas deux passereaux pour un sou ? Cependant, il n'en tombe pas un à terre sans la volonté de votre Père.
\VS{30}Et même les cheveux de votre tête sont tous comptés.
\VS{31}Ne craignez donc point : Vous valez plus que beaucoup de passereaux.
\VS{32}Quiconque donc me confessera devant les hommes, je le confesserai aussi devant mon Père qui est aux cieux.
\VS{33}Mais quiconque me reniera devant les hommes, je le renierai aussi devant mon Père qui est dans les cieux.
\VS{34}Ne croyez pas que je sois venu apporter la paix sur la terre. Je ne suis pas venu apporter la paix, mais l'épée.
\VS{35}Car je suis venu mettre en division le fils contre son père, et la fille contre sa mère, et la belle-fille contre sa belle-mère.
\VS{36}Et les propres domestiques d'un homme seront ses ennemis.
\VS{37}Celui qui aime son père ou sa mère plus que moi, n'est pas digne de moi ; et celui qui aime son fils ou sa fille plus que moi, n'est pas digne de moi.
\VS{38}Et quiconque ne prend pas sa croix et ne vient pas après moi, n'est pas digne de moi.
\VS{39}Celui qui aura conservé sa vie la perdra ; mais celui qui aura perdu sa vie pour l'amour de moi la retrouvera.
\VS{40}Celui qui vous reçoit me reçoit, et celui qui me reçoit, reçoit celui qui m'a envoyé.
\VS{41}Celui qui reçoit un prophète en qualité de prophète, recevra la récompense d'un prophète ; et celui qui reçoit un juste en qualité de juste recevra la récompense d'un juste.
\VS{42}Et quiconque aura donné à boire seulement un verre d'eau froide à l'un de ces petits parce qu'il est mon disciple, je vous le dis en vérité qu'il ne perdra point sa récompense.
\Chap{11}
\TextTitle{Jean-Baptiste le plus grand des hommes\FTNTT{Lu. 7:19-35}}
\VerseOne{}Et il arriva que quand Jésus eut achevé de donner ses ordres à ses douze disciples, il partit de là pour aller enseigner et prêcher dans leurs villes.
\VS{2}Jean, ayant entendu parler dans sa prison des œuvres du Christ, envoya deux de ses disciples pour lui dire :
\VS{3}Es-tu celui qui devait venir, ou devons-nous en attendre un autre ?
\VS{4}Et Jésus leur répondit : Allez, et rapportez à Jean les choses que vous entendez et que vous voyez.
\VS{5}Les aveugles recouvrent la vue, les boiteux marchent, les lépreux sont purifiés, les sourds entendent, les morts sont ressuscités, et l'Evangile est annoncé aux pauvres\FTNT{Jésus-Christ est le Dieu véritable dont la venue était annoncée par Esaïe (Es. 35:4-6).}.
\VS{6}Mais, heureux est celui qui n'aura point été scandalisé en moi ;
celui pour qui je ne serai pas une occasion de chute !
\VS{7}Et comme ils s'en allaient, Jésus se mit à dire à la foule au sujet de Jean : Mais qu'êtes-vous allés voir dans le désert ? Un roseau agité par le vent ?
\VS{8}Mais qu'êtes-vous allés voir ? Un homme vêtu de précieux vêtements ? Voici, ceux qui portent des habits précieux sont dans les maisons des rois.
\VS{9}Mais qu'êtes-vous allés voir ? Un prophète ? Oui, vous dis-je, et plus qu'un prophète.
\VS{10}Car c'est celui dont il est écrit : Voici, j'envoie mon messager\FTNT{Mal. 3:1.} devant ta face, pour préparer ton chemin devant toi.
\VS{11}En vérité, je vous le dis, parmi ceux qui sont nés de femmes, il n'en a point paru de plus grand que Jean-Baptiste. Toutefois, le plus petit dans le Royaume des cieux, est plus grand que lui.
\VS{12}Or depuis le temps de Jean-Baptiste jusqu'à maintenant, le Royaume des cieux est forcé et ce sont les violents qui s'en emparent.
\VS{13}Car tous les prophètes et la loi ont prophétisé jusqu'à Jean.
\VS{14}Et si vous voulez recevoir mes paroles, c'est lui qui est l'Elie\FTNT{Mal. 4:5-6.} qui devait venir.
\VS{15}Que celui qui a des oreilles pour entendre, entende.
\VS{16}Mais à qui comparerai-je cette génération ? Elle est semblable aux petits-enfants qui sont assis sur les places publiques, et qui crient à leurs compagnons
\VS{17}et leur disent : Nous vous avons joué de la flûte et vous n'avez point dansé ; nous vous avons chanté des complaintes et vous ne vous êtes point lamentés.
\VS{18}Car Jean est venu ne mangeant ni ne buvant et ils disent : Il a un démon.
\VS{19}Le Fils de l'homme est venu mangeant et buvant et ils disent : C'est un mangeur et un buveur, un ami des publicains et des gens de mauvaise vie. Mais la sagesse a été justifiée par ses enfants.
\TextTitle{Jésus dénonce les indifférents}
\VS{20}Alors il se mit à faire des reproches aux villes où il avait fait beaucoup de miracles, parce qu'elles ne s'étaient point repenties.
\VS{21}Malheur à toi, Chorazin ! Malheur à toi, Bethsaïda ! Car si les miracles qui ont été faits au milieu de vous, avaient été faits dans Tyr et dans Sidon, il y a longtemps qu'elles se seraient repenties, en prenant le sac et la cendre.
\VS{22}C'est pourquoi je vous dis que Tyr et Sidon seront traitées moins rigoureusement que vous, au jour du jugement.
\VS{23}Et toi Capernaüm, qui as été élevée jusqu'au ciel, tu seras précipitée jusqu'à Hadès\FTNT{Voir commentaire Mt. 16:18} ; car si les miracles qui ont été faits au milieu de toi, avaient été faits dans Sodome, elle subsisterait encore.
\VS{24}C'est pourquoi je vous dis que ceux de Sodome seront traités moins rigoureusement que toi, au jour du jugement.
\TextTitle{La relation personnelle du disciple avec son Seigneur}
\VS{25}En ce temps-là, Jésus prenant la parole dit : Je te loue, ô mon Père ! Seigneur du ciel et de la terre, de ce que tu as caché ces choses aux sages et aux intelligents, et que tu les as révélées aux petits enfants.
\VS{26}Oui, Père, je te loue parce que telle a été ta bonne volonté.
\VS{27}Toutes choses m'ont été données par mon Père ! Et personne ne connaît le Fils si ce n'est le Père ; et personne ne connaît le Père si ce n'est le Fils, et celui à qui le Fils veut le révéler.
\VS{28}Venez à moi vous tous qui êtes fatigués et chargés, et je vous donnerai du repos.
\VS{29}Prenez mon joug sur vous et recevez mes instructions, car je suis doux et humble de cœur ; et vous trouverez le repos pour vos âmes.
\VS{30}Car mon joug est doux et mon fardeau est léger.
\Chap{12}
\TextTitle{Jésus, le Maître du sabbat\FTNTT{Mc. 2:23-28 ; Lu. 6:1-5}}
\VerseOne{}En ce temps-là, Jésus traversa des champs de blé un jour de sabbat. Et ses disciples qui avaient faim se mirent à arracher des épis et à les manger.
\VS{2}Les pharisiens voyant cela, lui dirent : Voici, tes disciples font ce qu'il n'est pas permis de faire le jour du sabbat.
\VS{3}Mais il leur dit : N'avez-vous pas lu ce que fit David quand il eut faim, lui et ceux qui étaient avec lui ?
\VS{4}Comment il entra dans la maison de Dieu, et mangea les pains de proposition, qu'il ne lui était pas permis de manger ni à lui, ni à ceux qui étaient avec lui, mais aux sacrificateurs seulement ?
\VS{5}Ou n'avez-vous pas lu dans la loi, qu'aux jours du sabbat, les sacrificateurs violent le sabbat dans le temple, sans se rendre coupables ?
\VS{6}Or, je vous le dis, qu'il y a ici quelqu'un de plus grand que le temple.
\VS{7}Si vous saviez ce que signifient ces paroles : Je veux la miséricorde, et non pas le sacrifice, vous n'auriez pas condamné ceux qui ne sont pas coupables\FTNT{1 S. 15:22 ; Os. 6:6.}.
\VS{8}Car le Fils de l'homme est Maître même du sabbat.
\TextTitle{Jésus guérit l'homme à la main sèche le jour du sabbat\FTNTT{Mc. 3:1-5 ; Lu. 6:6-11}}
\VS{9}Puis étant parti de là, il entra dans leur synagogue.
\VS{10}Et voici, il s'y trouvait un homme qui avait la main sèche. Et pour avoir sujet de l'accuser, ils l'interrogèrent en disant : Est-il permis de guérir les jours du sabbat ?
\VS{11}Et il répondit : Lequel d'entre vous s'il n'a qu'une brebis, et qu'elle vienne à tomber dans une fosse le jour du sabbat, ne la saisira-t-il pas pour l'en retirer ?
\VS{12}Combien un homme ne vaut-il pas plus qu'une brebis ! Il est donc permis de faire du bien les jours du sabbat.
\VS{13}Alors il dit à cet homme : Etends ta main. Il l'étendit et elle devint saine comme l'autre.
\TextTitle{Jésus accomplit de nombreuses guérisons}
\VS{14}Les pharisiens sortirent et ils se consultèrent sur les moyens de le faire périr.
\VS{15}Mais Jésus, l'ayant su, partit de là, et de grandes foules le suivirent. Il les guérit tous.
\VS{16}Et il leur défendit avec menaces de le faire connaître,
\VS{17}afin que s'accomplît ce qui avait été annoncé par Esaïe le prophète, en disant :
\VS{18}Voici mon serviteur que j'ai élu, mon bien-aimé, qui est l'objet de mon amour, je mettrai mon Esprit en lui et il annoncera le jugement aux nations.
\VS{19}Il ne contestera point, il ne criera point et personne n'entendra sa voix dans les rues.
\VS{20}Il ne brisera point le roseau cassé et n'éteindra point le lumignon qui fume, jusqu'à ce qu'il ait fait triompher la justice.
\VS{21}Et les nations espéreront en son nom\FTNT{Es. 42:1-4.}.
\VS{22}Alors on lui amena un homme tourmenté d'un démon, aveugle et muet, et il le guérit ; de sorte que celui qui avait été aveugle et muet, parlait et voyait.
\VS{23}Et toutes les foules en furent étonnées, et elles disaient : Celui-ci n'est-il pas le Fils de David ?
\TextTitle{Le blasphème contre le Saint-Esprit\FTNTT{Mc. 3:22-30 ; Lu. 11:15-23}}
\VS{24}Mais les pharisiens ayant entendu cela, disaient : Celui-ci ne chasse les démons que par Béelzébul, prince des démons.
\VS{25}Mais Jésus connaissant leurs pensées, leur dit : Tout royaume divisé contre lui-même sera réduit en désert ; et toute ville, ou maison, divisée contre elle-même ne subsistera point.
\VS{26}Or si Satan chasse Satan, il est divisé contre lui-même ; comment donc son royaume subsistera-t-il ?
\VS{27}Et si je chasse les démons par Béelzébul, par qui vos fils les chassent-ils ? C'est pourquoi ils seront eux-mêmes vos juges.
\VS{28}Mais si je chasse les démons par l'Esprit de Dieu, certes le Royaume de Dieu est donc venu jusqu'à vous.
\VS{29}Ou, comment quelqu'un peut-il entrer dans la maison d'un homme fort et piller ses biens, sans avoir auparavant lié cet homme fort ? Alors il pillera sa maison.
\VS{30}Celui qui n'est pas avec moi, est contre moi, et celui qui n'assemble pas avec moi disperse.
\VS{31}C'est pourquoi je vous dis, que tout péché et tout blasphème sera pardonné aux hommes ; mais le blasphème contre l'Esprit ne leur sera point pardonné.
\VS{32}Quiconque parlera contre le Fils de l'homme, il lui sera pardonné ; mais quiconque parlera contre le Saint-Esprit, il ne lui sera pardonné ni dans ce siècle, ni dans le siècle à venir\FTNT{Le blasphème contre le Saint-Esprit : Le blasphème est un outrage, une calomnie à l'encontre de Dieu. En attribuant l'œuvre de Dieu à Satan, les pharisiens ont commis l'impardonnable. Beaucoup de personnes craignent d'avoir commis ce péché par inadvertance, en ayant par exemple un doute sur l'origine d'un miracle. La Parole nous recommande de ne pas ajouter foi à tout esprit, mais d'éprouver les esprits pour savoir s'ils sont de Dieu (1 Jn. 4:1). On ne pèche donc pas lorsqu'on exerce son discernement. De plus, si l'on a commis une erreur de jugement par ignorance, le Seigneur ne nous en tiendra pas rigueur (Ac. 17:30). Le blasphème contre le Saint-Esprit est commis par des personnes qui, bien qu'ayant la connaissance et la capacité de différencier le bien du mal, font preuve de mauvaise foi. Ainsi, les pharisiens avaient constaté les bons fruits portés par Jésus, mais ils ont hypocritement qualifié de mal le bien qu'il faisait (Es. 5:20). Ceux qui blasphèment contre le Saint-Esprit sont loin d'être ignorants. Comme nous l'atteste Hé. 6:4-6, parmi ces gens, certains « ont goûté le don céleste », « ont eu part au Saint-Esprit, et ont goûté la bonne parole de Dieu, et les puissances du siècle à venir ». En choisissant sciemment de pécher, alors qu'ils ont expérimenté la grâce de Dieu, ils retournent à ce qu'ils ont vomi et outragent ainsi le Seigneur (2 Pi. 2:18-22). Leur cœur endurci à l'extrême rejette volontairement la vérité pour s'attacher au mensonge. Constatant leur refus définitif de se repentir, le Saint-Esprit finit par se retirer pour laisser la place à l'esprit d'égarement qui les maintiendra dans l'erreur (2 Th. 2:9-12). Enfin, il est à noter qu'en Ap. 14:9-11, ceux qui ont reçu la marque de la bête sont condamnés d'office. Il ne faut nullement conclure que le Seigneur leur a refusé son pardon, mais plutôt que les personnes ayant reçu cette marque ont aussi blasphémé contre le Saint-Esprit. Voir commentaire en Ap. 13:16.}.
\TextTitle{Toute parole proclamée appelle un jugement}
\VS{33}Ou dites que l'arbre est bon et son fruit est bon ; ou dites que l'arbre est mauvais et son fruit est mauvais ; car on connaît l'arbre par le fruit.
\VS{34}Race de vipères, comment pourriez-vous dire de bonnes choses, méchants comme vous l'êtes ? Car c'est de l'abondance du cœur que la bouche parle.
\VS{35}L'homme de bien\FTNT{Le mot « bien » dans ce passage vient du grec « Agathos » qui signifie : « de bonne constitution ou nature », « utile », « salutaire », « bon », « agréable », plaisant », « joyeux », « heureux », « excellent », « distingué », « droit », « honorable ».} tire de bonnes choses du bon trésor de son cœur ; et l'homme méchant tire de mauvaises choses du mauvais trésor de son cœur.
\VS{36}Je vous le dis : Les hommes rendront compte au jour du jugement, de toute parole vaine qu'ils auront proférée.
\VS{37}Car tu seras justifié par tes paroles et tu seras condamné par tes paroles.
\TextTitle{Le miracle du prophète Jonas\FTNTT{Jon. 2:1 ; Lu. 11:29-32.}}
\VS{38}Alors quelques-uns des scribes et des pharisiens lui dirent : Maître, nous voudrions bien te voir faire quelque miracle.
\VS{39}Mais il leur répondit et dit : Une génération méchante et adultère demande un miracle, mais il ne lui sera point donné d'autre miracle que celui de Jonas le prophète.
\VS{40}Car, de même que Jonas fut trois jours et trois nuits dans le ventre d'un grand poisson, de même le Fils de l'homme sera trois jours et trois nuits dans le sein de la terre.
\VS{41}Les Ninivites se lèveront au jour du jugement contre cette génération et la condamneront, parce qu'ils se repentirent à la prédication de Jonas ; et voici, il y a ici plus que Jonas.
\TextTitle{La condamnation de cette génération par la reine de Séba\FTNTT{2 Ch. 9:1-12}}
\VS{42}La reine du Midi se lèvera au jour du jugement contre cette nation et la condamnera, parce qu'elle vint des extrémités de la terre pour entendre la sagesse de Salomon ; et voici, il y a ici plus que Salomon.
\TextTitle{Le retour de l'esprit impur\FTNTT{Lu. 11:24-26}}
\VS{43}Lorsque l'esprit impur est sorti d'un homme, il va par des lieux arides, cherchant du repos, mais il n'en trouve point.
\VS{44}Alors il dit : Je retournerai dans ma maison, d'où je suis sorti ; et quand il arrive, il la trouve vide, balayée et ornée.
\VS{45}Puis il s'en va et prend avec lui sept autres esprits plus méchants que lui ; qui y étant entrés, habitent là ; et ainsi la dernière condition de cet homme est pire que la première. Il en sera de même pour cette génération perverse.
\TextTitle{La famille spirituelle\FTNTT{Mc. 3:31-35 ; Lu. 8:19-21}}
\VS{46}Et comme il parlait encore aux foules, voici, sa mère et ses frères se tenaient dehors, cherchant à lui parler.
\VS{47}Et quelqu'un lui dit : Voici, ta mère et tes frères sont là dehors, qui cherchent à te parler.
\VS{48}Mais il répondit à celui qui lui avait dit cela : Qui est ma mère et qui sont mes frères ?
\VS{49}Et étendant sa main sur ses disciples, il dit : Voici ma mère et mes frères.
\VS{50}Car, quiconque fera la volonté de mon Père qui est dans les cieux, celui-là est mon frère, et ma sœur, et ma mère.
\Chap{13}
\TextTitle{1. Parabole des quatre terrains\FTNTT{Mc. 4:1-20 ; Lu. 8:4-15}}
\VerseOne{}Ce même jour, Jésus sortit de la maison et s'assit au bord de la mer.
\VS{2}Une grande foule s'assembla auprès de lui, c'est pourquoi il monta dans une barque et il s'assit. Aussi, toute la foule se tenait sur le rivage.
\VS{3}Et il leur parla en paraboles sur beaucoup de choses et il dit : Un semeur sortit pour semer.
\VS{4}Et comme il semait, une partie de la semence tomba le long du chemin, et les oiseaux vinrent, et la mangèrent toute.
\VS{5}Et une autre partie tomba dans les endroits pierreux où elle n'avait pas beaucoup de terre : Elle leva aussitôt parce qu'elle n'entrait pas profondément dans la terre ;
\VS{6}mais, quand le soleil parut, elle fut brûlée et sécha parce qu'elle n'avait point de racines.
\VS{7}Une autre partie tomba parmi les épines ; et les épines montèrent et l'étouffèrent.
\VS{8}Une autre partie tomba dans la bonne terre : Et elle donna du fruit, un grain en donna cent, un autre, soixante, et un autre, trente.
\VS{9}Que celui qui a des oreilles pour entendre, qu'il entende.
\TextTitle{Explication aux disciples}
\VS{10}Alors les disciples s'approchèrent et lui dirent : Pourquoi leur parles-tu en paraboles ?
\VS{11}Il leur répondit et dit : Parce qu'il vous a été donné de connaître les mystères du Royaume des cieux, et que cela ne leur a pas été donné de les connaître.
\VS{12}Car on donnera à celui qui a, et il sera dans l'abondance, mais à celui qui n'a pas, on ôtera même ce qu'il a.
\VS{13}C'est pourquoi je leur parle en paraboles, parce qu'en voyant, ils ne voient point, et qu'en entendant, ils n'entendent point et ne comprennent point.
\VS{14}Et ainsi s'accomplit pour eux la prophétie d'Esaïe qui dit : Vous entendrez de vos oreilles et vous ne comprendrez point ; et vous regarderez des yeux, et vous ne verrez point.
\VS{15}Car le cœur de ce peuple est engraissé, et ils ont endurci leurs oreilles, et ils ont fermé leurs yeux de peur qu'ils ne voient de leurs yeux, qu'ils n'entendent de leurs oreilles, qu'ils ne comprennent de leur cœur, qu'ils ne se convertissent et que je ne les guérisse\FTNT{Es. 6:9-10.}.
\VS{16}Mais heureux sont vos yeux, car ils voient ; et vos oreilles, parce qu'elles entendent.
\VS{17}Je vous le dis en vérité, beaucoup de prophètes et de justes ont désiré voir les choses que vous voyez, et ils ne les ont point vues, entendre les choses que vous entendez, et ils ne les ont point entendues.
\VS{18}Vous donc, écoutez la signification de la parabole du semeur.
\VS{19}Lorsqu'un homme écoute la parole du Royaume et ne la comprend pas, le malin vient et ravit ce qui est semé dans son cœur ; cet homme est celui qui a reçu la semence le long du chemin.
\VS{20}Et celui qui a reçu la semence dans les endroits pierreux, c'est celui qui entend la parole et la reçoit aussitôt avec joie ;
\VS{21}mais il n'a point de racine en lui-même, il croit pour un temps, et dès que survient une tribulation ou une persécution à cause de la parole, il y trouve une occasion de chute.
\VS{22}Et celui qui a reçu la semence parmi les épines, c'est celui qui entend la parole de Dieu, mais en qui les soucis du siècle et la séduction des richesses étouffent la parole et la rendent infructueuse.
\VS{23}Mais celui qui a reçu la semence dans la bonne terre, c'est celui qui entend la parole et la comprend. Il porte du fruit, et un grain donne cent, un autre soixante, et un autre trente.
\TextTitle{2. Parabole du blé et de l'ivraie}
\VS{24}Il leur proposa une autre parabole et il dit\FTNT{La parabole du blé et de l'ivraie. En méditant cette parabole, nous remarquons que lorsque le blé eut poussé et donné du fruit, l'ivraie parut aussi. Il est vrai que lorsqu'il y a un réveil spirituel divin dans une assemblée ou dans un pays, l'ennemi suscite aussi un faux réveil avec des faux ouvriers et des fausses manifestations spirituelles. Voilà pourquoi l'ivraie côtoiera le blé jusqu'à la fin du monde. Le mot « ivraie » se dit « ebriacus » en latin, ce qui donne « ébriété » en français. Nous comprenons donc que l'un des rôles de l'ivraie est d'enivrer le blé (les enfants de Dieu). Dans les Ecritures, l'ivresse est synonyme de la débauche spirituelle ou physique. En grec l'ivraie se dit « zizanion » qui donne en français « zizanie ». Voir Mt. 12:25. La division est l'œuvre de l'ivraie dans les églises qui cherche à créer des sectes et des partis pris.} : Le Royaume des cieux est semblable à un homme qui a semé de la bonne semence dans son champ.
\VS{25}Mais, pendant que les hommes dormaient, son ennemi vint, sema de l'ivraie parmi le blé, puis s'en alla.
\VS{26}Lorsque l'herbe eut poussé et donné du fruit, l'ivraie parut aussi.
\VS{27}Et les serviteurs du maître de la maison vinrent à lui et lui dirent : Seigneur, n'as-tu pas semé de la bonne semence dans ton champ ? D'où vient donc qu'il y a de l'ivraie ?
\VS{28}Mais il leur répondit : C'est un ennemi qui a fait cela. Et les serviteurs lui dirent : Veux-tu donc que nous allions l'arracher ?
\VS{29}Et il leur dit : Non, de peur qu'en arrachant l'ivraie, vous ne déraciniez le blé en même temps.
\VS{30}Laissez-les croître tous deux ensemble, jusqu'à la moisson ; et au temps de la moisson, je dirai aux moissonneurs : Arrachez premièrement l'ivraie, et liez-la en gerbes pour la brûler, mais amassez le blé dans mon grenier.
\TextTitle{3. Parabole du grain de sénevé\FTNTT{Mc. 4:30-32 ; Lu. 13:18-19}}
\VS{31}Il leur proposa une autre parabole et il dit : Le Royaume des cieux est semblable au grain de sénevé qu'un homme a pris et semé dans son champ.
\VS{32}C'est la plus petite de toutes les semences ; mais, quand il a poussé, il est plus grand que les autres plantes et devient un arbre, de sorte que les oiseaux du ciel viennent habiter et font leurs nids dans ses branches.
\TextTitle{4. Parabole du levain\FTNTT{Lu. 13:20-21}}
\VS{33}Il leur dit une autre parabole : Le Royaume des cieux est semblable à du levain qu'une femme a pris et mis dans trois mesures de farine, jusqu'à ce que toute la pâte soit levée.
\VS{34}Jésus dit à la foule toutes ces choses en paraboles, et il ne lui parlait point sans paraboles,
\VS{35}afin que s'accomplisse ce qui avait été annoncé par le prophète : J'ouvrirai ma bouche en paraboles, je déclarerai les choses qui ont été cachées dès la fondation du monde\FTNT{Ps. 78:2.}.
\TextTitle{Explication de la parabole du blé et de l'ivraie}
\VS{36}Alors Jésus renvoya la foule et entra dans la maison, et ses disciples s'approchèrent de lui et lui dirent : Explique-nous la parabole de l'ivraie du champ.
\VS{37}Et il leur répondit et dit : Celui qui sème la bonne semence, c'est le Fils de l'homme ;
\VS{38}le champ, c'est le monde ; la bonne semence ce sont les fils du Royaume, et l'ivraie ce sont les fils du malin ;
\VS{39}et l'ennemi qui l'a semée, c'est le diable ; la moisson, c'est la fin du monde, et les moissonneurs sont les anges.
\VS{40}Or, comme on arrache l'ivraie et qu'on la brûle au feu, il en sera de même à la fin de ce monde.
\VS{41}Le Fils de l'homme enverra ses anges qui arracheront de son Royaume tous les scandales et ceux qui commettent l'iniquité,
\VS{42}et les jetteront dans la fournaise ardente, où il y aura des pleurs et des grincements de dents.
\VS{43}Alors les justes resplendiront comme le soleil dans le Royaume de leur Père. Que celui qui a des oreilles pour entendre, qu'il entende.
\TextTitle{5. Parabole du trésor caché}
\VS{44}Le Royaume des cieux est encore semblable à un trésor caché dans un champ. L'homme qui l'a trouvé, le cache ; puis dans sa joie, il va vendre tout ce qu'il a, et achète ce champ.
\TextTitle{6. Parabole de la perle}
\VS{45}Le Royaume des cieux est encore semblable à un marchand qui cherche de bonnes perles.
\VS{46}Il a trouvé une perle de grand prix et il est allé vendre tout ce qu'il avait, et l'a achetée.
\TextTitle{7. Parabole du filet}
\VS{47}Le Royaume des cieux est encore semblable à un filet jeté dans la mer et ramassant toutes sortes de choses.
\VS{48}Quand il est rempli, les pêcheurs le tirent en haut sur le rivage, puis s'étant assis, ils mettent ce qu'il y a de bon à part dans leurs vases et jettent dehors ce qui est mauvais.
\VS{49}Il en sera de même à la fin du monde, les anges viendront séparer les méchants d'avec les justes,
\VS{50}et les jetteront dans la fournaise ardente, où il y aura des pleurs et des grincements de dents.
\VS{51}Jésus leur dit : Avez-vous compris toutes ces choses ? Ils lui répondirent : Oui, Seigneur.
\TextTitle{8. Le maître de la maison}
\VS{52}Et il leur dit : C'est pourquoi, tout scribe instruit de ce qui regarde le Royaume des cieux, est semblable à un père de famille qui tire de son trésor des choses nouvelles et des choses anciennes.
\TextTitle{Jésus à Nazareth\FTNTT{Mc. 6:1-6}}
\VS{53}Et quand Jésus eut achevé ces paraboles, il partit de là.
\VS{54}Et s'étant rendu dans sa patrie, il enseignait dans la synagogue, de telle sorte que ceux qui l'entendirent étaient étonnés et disaient : D'où lui viennent cette sagesse et ces miracles ?
\VS{55}Celui-ci n'est-il pas le fils du charpentier ? Sa mère ne s'appelle-t-elle pas Marie ? Et ses frères ne s'appellent-ils pas Jacques, Joseph, Simon et Jude ?
\VS{56}Et ses sœurs ne sont-elles pas toutes parmi nous ? D'où lui viennent donc toutes ces choses ?
\VS{57}Et il était pour eux une occasion de chute. Mais Jésus leur dit : Un prophète n'est méprisé que dans sa patrie et dans sa maison.
\VS{58}Et il ne fit là que peu de miracles, à cause de leur incrédulité.
\Chap{14}
\TextTitle{Mort de Jean-Baptiste\FTNTT{Mc. 6:14-29; Lu. 9:7-9}}
\VerseOne{}En ce temps-là, Hérode le tétrarque entendit parler de la renommée de Jésus, et il dit à ses serviteurs : C'est Jean-Baptiste !
\VS{2}Il est ressuscité des morts, c'est pourquoi la puissance de faire des miracles agit puissamment en lui.
\VS{3}Car Hérode avait fait arrêter Jean, et l'avait fait lier et mettre en prison, à cause d'Hérodias, femme de Philippe son frère.
\VS{4}Parce que Jean lui disait : Il ne t'est pas permis de l'avoir pour femme.
\VS{5}Et il voulait le faire mourir, mais il craignait la foule, parce qu'elle regardait Jean comme un prophète.
\VS{6}Or, le jour où l'on célébra la naissance d'Hérode, la fille d'Hérodias dansa au milieu de l'assemblée et plut à Hérode.
\VS{7}C'est pourquoi il lui promit avec serment de lui donner tout ce qu'elle demanderait.
\VS{8}A l'instigation de sa mère, elle dit : Donne-moi ici, sur un plat, la tête de Jean-Baptiste.
\VS{9}Le roi fut attristé ; mais à cause de ses serments et de ceux qui étaient à table avec lui, il commanda qu'on la lui donne.
\VS{10}Et il envoya décapiter Jean dans la prison.
\VS{11}Et sa tête fut apportée sur un plat et donnée à la fille qui la présenta à sa mère.
\VS{12}Puis ses disciples vinrent, et emportèrent son corps, et l'ensevelirent. Et ils allèrent l'annoncer à Jésus.
\VS{13}Et Jésus, ayant appris ce qu'Hérode avait fait, partit de là dans une barque, pour se retirer à l'écart dans un lieu désert ; et la foule l'ayant appris, sortit des villes voisines et le suivit à pied.
\VS{14}Et Jésus étant sorti, vit une grande foule, et il fut ému de compassion pour elle, et guérit les malades.
\TextTitle{Multiplication des pains pour les cinq mille hommes\FTNTT{Mc. 6:32-44 ; Lu. 9:12-17 ; Jn. 6:1-14}}
\VS{15}Et comme il se faisait tard, ses disciples vinrent à lui et lui dirent : Ce lieu est désert et l'heure est déjà avancée. Renvoie la foule, afin qu'elle aille dans les villages, pour s'acheter des vivres.
\VS{16}Mais Jésus leur dit : Ils n'ont pas besoin de s'en aller ; donnez-leur vous-mêmes à manger.
\VS{17}Et ils lui dirent : Nous n'avons ici que cinq pains et deux poissons.
\VS{18}Et il leur dit : Apportez-les-moi ici.
\VS{19}Et après avoir ordonné à la foule de s'asseoir sur l'herbe, il prit les cinq pains et les deux poissons, et levant les yeux au ciel, il rendit grâces à Dieu. Puis ayant rompu les pains, il les donna aux disciples qui les distribuèrent à la foule.
\VS{20}Tous en mangèrent et furent rassasiés, et l'on emporta douze paniers pleins des morceaux qui restaient.
\VS{21}Ceux qui avaient mangé étaient environ cinq mille hommes, sans compter les femmes et les petits enfants.
\TextTitle{Jésus marche sur les eaux, incrédulité de Pierre\FTNTT{Mc. 6:45-56 ; Jn. 6:15-21}}
\VS{22}Aussitôt après, Jésus obligea ses disciples à monter dans la barque et à passer avant lui de l'autre côté, pendant qu'il renverrait la foule.
\VS{23}Et quand il l'eut renvoyée, il monta sur une montagne pour être à part, afin de prier ; et le soir étant venu, il était là seul.
\VS{24}La barque, déjà au milieu de la mer, était battue par les flots ; car le vent était contraire.
\VS{25}Et vers la quatrième veille de la nuit, Jésus alla vers eux, marchant sur la mer.
\VS{26}Et ses disciples le voyant marcher sur la mer, ils furent troublés et ils dirent : C'est un fantôme ! Et, dans leur frayeur, ils poussèrent des cris.
\VS{27}Jésus leur dit aussitôt : Rassurez-vous, c'est moi, n'ayez pas de peur !
\VS{28}Et Pierre lui répondit : Seigneur, si c'est toi, ordonne que j'aille vers toi sur les eaux.
\VS{29}Et il lui dit : Viens ! Pierre sortit de la barque, marcha sur les eaux pour aller vers Jésus.
\VS{30}Mais voyant que le vent était fort, il eut peur ; et comme il commençait à enfoncer, il s'écria : Seigneur ! Sauve-moi !
\VS{31}Et aussitôt Jésus étendit sa main et le prit en lui disant : Homme de peu de foi, pourquoi as-tu douté ?
\VS{32}Et quand ils furent montés dans la barque, le vent s'apaisa.
\VS{33}Alors ceux qui étaient dans la barque, vinrent adorer Jésus et dirent : Certes, tu es le Fils de Dieu.
\TextTitle{Jésus guérit des malades à Génésareth\FTNTT{Mc. 6:53-56}}
\VS{34}Après avoir traversé la mer, ils vinrent dans le pays de Génézareth.
\VS{35}Les gens de ce lieu ayant reconnu Jésus, envoyèrent des messagers dans tous les environs et on lui amena tous les malades.
\VS{36}Et ils le prièrent de leur permettre de toucher seulement le bord de son vêtement. Et tous ceux qui le touchèrent furent guéris.
\Chap{15}
\TextTitle{Jésus-Christ condamne les traditions\FTNTT{Mc. 7:1-13}}
\VerseOne{}Alors des scribes et des pharisiens vinrent de Jérusalem auprès de Jésus et lui dirent :
\VS{2}Pourquoi tes disciples transgressent-ils la tradition des anciens ? Car ils ne se lavent point les mains quand ils prennent leur repas.
\VS{3}Il leur répondit : Et vous, pourquoi transgressez-vous le commandement de Dieu par votre tradition ?
\VS{4}Car Dieu a dit : Honore ton père et ta mère. Et il a dit aussi : Celui qui maudira son père ou sa mère finira à la mort.
\VS{5}Mais vous, vous dites : Celui qui dira à son père ou à sa mère : Tout ce dont j'aurais pu t'assister est une offrande à Dieu, n'est pas coupable, quoiqu'il n'honore pas son père ou sa mère.
\VS{6}Vous annulez ainsi le commandement de Dieu par votre tradition.
\VS{7}Hypocrites, Esaïe a bien prophétisé de vous, en disant :
\VS{8}Ce peuple s'approche de moi de sa bouche et m'honore des lèvres ; mais son cœur est très éloigné de moi.
\VS{9} Mais ils m'honorent en vain, en enseignant des doctrines qui ne sont que des commandements d'hommes\FTNT{Es. 29:13.}.
\TextTitle{Verdict sur le coeur humain\FTNTT{Mc. 7:14-23}}
\VS{10}Puis ayant appelé à lui la foule, il lui dit : Ecoutez, et comprenez ceci :
\VS{11}Ce n'est pas ce qui entre dans la bouche qui souille l'homme ; mais ce qui sort de la bouche c'est ce qui souille l'homme.
\VS{12}Sur cela les disciples s'approchant, lui dirent : Sais-tu que les pharisiens ont été scandalisés quand ils ont entendus ce discours ?
\VS{13}Et il répondit et dit : Toute plante que mon Père céleste n'a pas plantée sera déracinée.
\VS{14}Laissez-les, ce sont des aveugles, conducteurs d'aveugles ; si un aveugle conduit un autre aveugle, ils tomberont tous deux dans la fosse.
\VS{15}Alors Pierre prenant la parole, lui dit : Explique-nous cette parabole.
\VS{16}Et Jésus dit : Vous aussi, êtes-vous encore sans intelligence ?
\VS{17}Ne comprenez-vous pas encore que tout ce qui entre dans la bouche va dans le ventre, puis est jeté dans les lieux secrets ?
\VS{18}Mais les choses qui sortent de la bouche partent du cœur, et ces choses-là souillent l'homme.
\VS{19}Car c'est du cœur que sortent les mauvaises pensées, les meurtres, les adultères, les fornications, les vols, les faux témoignages, les médisances.
\VS{20}Ce sont ces choses-là qui souillent l'homme ; mais de manger sans avoir les mains lavées, cela ne souille point l'homme.
\TextTitle{Jésus et la femme cananéenne\FTNTT{Mc. 7:24-30}}
\VS{21}Alors Jésus, partant de là se retira dans le territoire de Tyr et de Sidon.
\VS{22}Et voici, une femme cananéenne, qui venait de ces contrées, lui cria : Seigneur ! Fils de David, aie pitié de moi ! Ma fille est cruellement tourmentée par le démon.
\VS{23}Mais il ne lui répondit pas un mot. Et ses disciples s'approchèrent et lui dirent : Renvoie-la, car elle crie derrière nous.
\VS{24}Et il répondit : Je n'ai été envoyé qu'aux brebis perdues de la maison d'Israël.
\VS{25}Mais elle vint et l'adora, disant : Seigneur, assiste-moi !
\VS{26}Et il lui répondit en disant : Il ne convient pas de prendre le pain des enfants et de le jeter aux petits chiens.
\VS{27}Mais elle dit : Cela est vrai, Seigneur ! Cependant les petits chiens mangent des miettes qui tombent de la table de leurs maîtres.
\VS{28}Alors Jésus répondant, lui dit : Ô femme ! Ta foi est grande. Qu'il te soit fait comme tu le souhaites. Et, à l'heure même, sa fille fut guérie.
\TextTitle{Nouvelles guérisons}
\VS{29}Et Jésus quitta ces lieux et vint près de la mer de Galilée. Puis il monta sur une montagne et s'y assit là.
\VS{30}Et une grande foule vint à lui, ayant avec elle des boiteux, des aveugles, des muets, des estropiés et beaucoup d'autres malades. On les mit aux pieds de Jésus et il les guérit ;
\VS{31}de sorte que la foule était dans l'admiration de voir que les muets parlaient, que les estropiés étaient guéris, que les boiteux marchaient, que les aveugles voyaient ; et elle glorifiait le Dieu d'Israël.
\TextTitle{Seconde multiplication des pains\FTNTT{Mc. 8:1-9}}
\VS{32}Alors Jésus, ayant appelé ses disciples, dit : Je suis ému de compassion pour cette foule de gens ; car voilà trois jours qu'ils sont près de moi et ils n'ont rien à manger. Je ne veux pas les renvoyer à jeun, de peur que les forces ne leur manquent en chemin.
\VS{33}Et ses disciples lui dirent : D'où pourrions-nous tirer dans ce désert assez de pains pour rassasier une si grande multitude ?
\VS{34}Et Jésus leur dit : Combien avez-vous de pains ? Ils lui dirent : Sept, et quelque peu de petits poissons.
\VS{35}Alors il commanda aux foules de s'asseoir par terre.
\VS{36}Et ayant prit les sept pains et les poissons, et après avoir béni Dieu, il les rompit et les donna à ses disciples, qui les distribuèrent à la foule.
\VS{37}Et tous mangèrent et furent rassasiés, et l'on emporta sept corbeilles pleines des morceaux qui restaient.
\VS{38}Or, ceux qui avaient mangé étaient quatre mille hommes, sans compter les femmes et les petits enfants.
\VS{39}Et Jésus renvoya la foule, monta sur une barque, et se rendit dans le territoire de Magdala.
\Chap{16}
\TextTitle{La cécité d'une génération méchante et adultère\FTNTT{Mc. 8:10-14}}
\VerseOne{}Alors les pharisiens et les sadducéens vinrent à lui, et pour l'éprouver, lui demandèrent de leur faire voir un signe venant du ciel.
\VS{2}Mais il leur répondit : Quand le soir est venu, vous dites : Il fera beau temps, car le ciel est rouge.
\VS{3}Et le matin vous dites : Il y aura de l'orage aujourd'hui, car le ciel est d'un rouge sombre. Hypocrites, vous savez bien discerner l'aspect du ciel, et vous ne pouvez discerner les signes des temps !
\VS{4}Une génération méchante et adultère demande un miracle ; mais il ne lui sera point donné d'autre miracle que celui de Jonas le prophète. Puis il les quitta et s'en alla.
\VS{5}Et ses disciples, en passant sur l'autre bord, avaient oublié de prendre des pains.
\TextTitle{Le levain des pharisiens et des sadducéens, une doctrine corrompue\FTNTT{Mc. 8:15-21 ; Lu. 12:1-15}}
\VS{6}Et Jésus leur dit : Gardez-vous avec soin du levain des pharisiens et des sadducéens.
\VS{7}Ils résonnaient en eux-mêmes et disaient : C'est parce que nous n'avons pas pris de pains.
\VS{8}Et Jésus connaissant leurs pensées leur dit : Gens de peu de foi, pourquoi raisonnez-vous en vous-mêmes sur le fait que vous n'avez pas pris de pains ?
\VS{9}Ne comprenez-vous point encore, et ne vous rappelez-vous plus les cinq pains des cinq mille hommes et combien de paniers vous avez emportés,
\VS{10}ni des sept pains des quatre mille hommes et combien de corbeilles vous avez emportées ?
\VS{11}Comment ne comprenez-vous pas que ce n'est pas au sujet du pain que je vous ai dit, de vous garder du levain des pharisiens et des sadducéens ?
\VS{12}Alors ils comprirent que ce n'était pas du levain du pain qu'il leur avait dit de se garder, mais de la doctrine des pharisiens et des sadducéens.
\TextTitle{Pierre reconnaît Jésus comme le Messie\FTNTT{Mc. 8:27-30 ; Lu. 9:18-21 ; Jn. 6:66-71}}
\VS{13}Jésus, étant arrivé dans le territoire de Césarée de Philippe, demanda à ses disciples : Qui disent les hommes que je suis, moi le Fils de l'homme ?
\VS{14}Et ils lui répondirent : Les uns disent que tu es Jean-Baptiste ; les autres, Elie ; et les autres, Jérémie, ou l'un des prophètes.
\VS{15}Il leur dit : et vous, qui dites-vous que je suis ?
\VS{16}Simon Pierre répondit et dit : Tu es le Christ, le Fils du Dieu vivant.
\TextTitle{Jésus bâtit son Eglise}
\VS{17}Et Jésus lui répondit et dit : Tu es heureux, Simon, fils de Jonas, car ce ne sont pas la chair et le sang qui t'ont révélé cela, mais mon Père qui est dans les cieux.
\VS{18}Et moi je te dis, que tu es Pierre, et que sur ce Roc\FTNT{Le Roc : Ce passage a été mal traduit dans beaucoup de Bibles comme suit : « Et moi, je te dis que tu es Pierre, et que sur cette pierre je bâtirai mon Eglise… ». Or pour une bonne compréhension des propos de Jésus, il est important d'insister sur la distinction que le grec fait entre « Petros » (pierre, caillou), l'apôtre Pierre, et « Petra » (roc, rocher), qui n'est autre que Jésus-Christ, le rocher des siècles (Es. 17:10 ; Es. 26:4 ; 1 Co. 10:4). De là en découle un enseignement fondamental : l'Eglise n'est bâtie ni par un homme ni sur l'homme, en l'occurrence Pierre et ses supposés successeurs (papes), mais par Jésus-Christ lui-même qui en est la Pierre Angulaire et le fondement inébranlable (Ac. 4:11 ; Ep. 2:20).} je bâtirai mon Eglise ; et les portes de l'enfer\FTNT{Enfer : du grec « Hadès ». Hadès chez les Grecs ou Pluton chez les Romains, était considéré comme le dieu des profondeurs souterraines et le maître des enfers. Ce terme est parfois traduit par « séjour des morts », équivalent hébreu de « Scheol ». Les Grecs utilisaient l'euphémisme Pylartes, signifiant « aux portes solidement closes », pour parler du très craint Hadès. En effet, Juifs, Grecs et Romains avaient conscience que les portes closes de l'enfer ne laissaient personne sortir du royaume de la mort. Tous les impies, et même les croyants d'avant Jésus-Christ, étaient retenus par les portes de l'enfer. Toutefois, les croyants allaient dans une partie de l'enfer que les juifs appelaient « sein d'Abraham » (1 Sam. 28:7-19 ; Lu. 16:22-25 ; Lu. 23:43) où ils ne subissaient pas les tourments infligés aux impies. Lorsque le Seigneur est mort, il est descendu « dans les régions inférieures de la terre » pour prendre les clés du Hadès, clés du séjour des morts (Col. 2:15 ; Ap. 1:17-18) et libérer les captifs pieux. Jésus affirme que les portes de l'enfer ne prévaudront jamais contre son Eglise puisque c'est lui qui l'a bâtie. Malgré tout, Hadès, bien que vaincu par le Seigneur, essaie d'attirer l'Eglise que le Seigneur a établie dans les lieux célestes (Ep. 2:4-9 ; Col. 3:1) vers le royaume des ténèbres, au travers des fausses doctrines et du péché. Au jour du jugement dernier, Hadès et la mort, qui sont deux démons, seront jetés dans l'étang de feu et de soufre (Ap. 20:11-15).} ne prévaudront point contre elle.
\VS{19}Je te donnerai les clefs du Royaume des cieux ; et tout ce que tu lieras sur la terre, sera lié dans les cieux ; et tout ce que tu délieras sur la terre, sera délié dans les cieux\FTNT{Une mauvaise compréhension de ce verset a contribué à propager l'idée erronée selon laquelle Pierre serait le médiateur entre Dieu et les hommes, puisque c'est lui qui détiendrait les clés du Royaume des cieux. Toutefois, Es. 22:22 affirme que seul Jésus-Christ détient ces clés qui symbolisent l'autorité et la domination. Or dans le cadre de l'héritage que le Seigneur nous a laissé, cette autorité est désormais exercée en son Nom par tous les membres du corps de Christ (Mt. 18:18).}.
\VS{20}Alors il commanda expressément à ses disciples de ne dire à personne qu'il était Jésus le Christ.
\TextTitle{Jésus parle de sa mort et de sa résurrection\FTNTT{Mc. 8:31-33 ; Lu. 9:22}}
\VS{21}Dès lors Jésus commença à déclarer à ses disciples qu'il fallait qu'il aille à Jérusalem, qu'il souffre beaucoup de la part des anciens, des principaux sacrificateurs et des scribes, qu'il soit mis à mort, et qu'il ressuscite le troisième jour.
\VS{22}Mais Pierre l'ayant pris à part, se mit à le reprendre en lui disant : Seigneur, aie pitié de toi, cela ne t'arrivera point !
\VS{23}Mais lui, s'étant retourné, dit à Pierre : Arrière de moi, Satan ! Tu m'es en scandale, car tu ne comprends pas les choses qui sont de Dieu, mais celles qui sont des hommes.
\TextTitle{La consécration du disciple\FTNTT{Mc. 8:34-38 ; Lu. 9:23-26}}
\VS{24}Alors Jésus dit à ses disciples : Si quelqu'un veut venir après moi, qu'il renonce à lui-même, et qu'il se charge de sa croix, et qu'il me suive.
\VS{25}Car quiconque voudra sauver son âme, la perdra ; mais quiconque perdra son âme à cause de son amour pour moi, la trouvera.
\VS{26}Et que servirait-il à un homme de gagner tout le monde, s'il perdait son âme ? Ou, que donnerait un homme en échange de son âme ?
\VS{27}Car le Fils de l'homme doit venir dans la gloire de son Père avec ses anges ; et alors il rendra à chacun selon ses œuvres.
\VS{28}Je vous le dis en vérité, quelques-uns de ceux qui sont ici présents, ne mourront point, qu'ils n'aient vu le Fils de l'homme venir dans son règne\FTNT{Ce passage doit être lu de concert avec Mt. 24:32-34. Jésus utilise un langage prophétique pour expliquer deux réalités. La première réalité est spirituelle et concerne ses contemporains qui allaient vivre l'effusion de l'Esprit pour rétablir le Royaume de Dieu dans le cœur des gens. En effet, le Seigneur ne les a pas laissés orphelins, mais il est revenu sous la forme de l'Esprit (Jn. 14:17-18 ; Ac. 2 ; Ac. 16:7). Aussi, les apôtres ont pu proclamer ce Royaume partout où ils allaient (Ac. 20:25). La deuxième réalité est matérielle et concerne le fleurissement du figuier, c'est-à-dire Israël. L'histoire atteste le fleurissement de ce figuier tant sur le plan géographique que sur le plan numérique. Depuis le 14 mai 1948, date de la naissance officielle de l'état hébreu, Israël ne cesse de s'étendre. Cette nation est l'horloge des temps car le Messie gouvernera le monde entier depuis Jérusalem (Mi. 4 ; Za. 14).}.
\Chap{17}
\TextTitle{Transfiguration de Jésus-Christ\FTNTT{Mc. 9:1-8 ; Lu. 9:27-36}}
\VerseOne{}Six jours après, Jésus prit Pierre, Jacques et Jean son frère, et les conduisit à l'écart sur une haute montagne.
\VS{2}Et il fut transfiguré en leur présence et son visage resplendit comme le soleil ; et ses vêtements devinrent blancs comme la lumière.
\VS{3}Et voici, ils virent Moïse et Elie qui s'entretenaient avec lui.
\VS{4}Alors Pierre prenant la parole, dit à Jésus : Seigneur, il est bon que nous soyons ici. Faisons-y, si tu le veux, trois tentes, une pour toi, une pour Moïse, et une pour Elie.
\VS{5}Et comme il parlait encore, voici une nuée resplendissante les couvrit de son ombre. Et voici, une voix fit entendre de la nuée ces paroles : Celui-ci est mon Fils bien-aimé, en qui j'ai pris mon bon plaisir : Ecoutez-le !
\VS{6}Lorsque les disciples entendirent cette voix, ils tombèrent le visage contre terre et furent saisis d'une très grande frayeur.
\VS{7}Mais Jésus, s'approchant, les toucha et leur dit : Levez-vous et n'ayez pas peur.
\VS{8}Ils levèrent les yeux, et ne virent personne, excepté Jésus tout seul.
\VS{9}Et comme ils descendaient de la montagne, Jésus leur donna cet ordre, en disant : Ne parlez à personne de cette vision, jusqu'à ce que le Fils de l'homme soit ressuscité des morts.
\VS{10}Et ses disciples l'interrogèrent, en disant : Pourquoi donc les scribes disent-ils qu'il faut qu'Elie vienne premièrement ?
\VS{11}Et Jésus répondant, leur dit : Il est vrai qu'Elie viendra premièrement et rétablira toutes choses.
\VS{12}Mais je vous dis qu'Elie est déjà venu, et ils ne l'ont pas reconnu et ils lui ont fait tout ce qu'ils ont voulu. De même, le Fils de l'homme doit souffrir aussi de leur part.
\VS{13}Alors les disciples comprirent que c'était de Jean-Baptiste qu'il leur parlait.
\TextTitle{Le manque de foi des disciples\FTNTT{Mc. 9:14-29 ; Lu. 9:37-43}}
\VS{14}Et quand ils furent arrivés près de la foule, un homme s'approcha et se mit à genoux devant lui,
\VS{15}et lui dit : Seigneur ! Aie pitié de mon fils qui est lunatique et misérablement affligé ; car il tombe souvent dans le feu et souvent dans l'eau.
\VS{16}Et je l'ai présenté à tes disciples, mais ils n'ont pas pu le guérir.
\VS{17}Et Jésus répondit et dit : Ô race incrédule et perverse, jusqu'à quand serai-je avec vous ? Jusqu'à quand vous supporterai-je ? Amenez-le-moi ici.
\VS{18}Et Jésus parla sévèrement au démon, qui sortit de lui, et à l'heure même l'enfant fut guéri.
\VS{19}Alors les disciples s'approchèrent de Jésus et lui dirent en particulier : Pourquoi n'avons-nous pas pu le chasser ?
\VS{20}Et Jésus leur répondit : C'est à cause de votre incrédulité. Je vous le dis en vérité, si vous aviez de la foi, comme un grain de sénevé, vous diriez à cette montagne : Transporte-toi d'ici là, et elle se transporterait ; et rien ne vous serait impossible.
\VS{21}Mais cette sorte de démon ne sort que par la prière et par le jeûne.
\TextTitle{Jésus évoque à nouveau sa mort et sa résurrection\FTNTT{Mc. 9:30-32 ; Lu. 9:44-45}}
\VS{22}Et comme ils se trouvaient en Galilée, Jésus leur dit : Il arrivera que le Fils de l'homme sera livré entre les mains des hommes ;
\VS{23}Et qu'ils le feront mourir, mais le troisième jour il ressuscitera. Et les disciples en furent fort attristés.
\TextTitle{La pièce d'argent dans la bouche d'un poisson\FTNTT{Mc. 12:13-17}}
\VS{24}Et lorsqu'ils arrivèrent à Capernaüm, ceux qui percevaient les deux drachmes s'adressèrent à Pierre et lui dirent : Votre Maître ne paye-t-il pas les deux drachmes ?
\VS{25}Oui dit-il. Et quand il fut entré dans la maison, Jésus le prévint en lui disant : Qu'est-ce qu'il t'en semble, Simon ? Les rois de la terre, de qui perçoivent-ils des tributs ou des impôts ? Est-ce de leurs enfants ou des étrangers ?
\VS{26}Pierre dit : Des étrangers. Jésus lui répondit : Les enfants en sont donc exempts.
\VS{27}Mais afin que nous ne les scandalisions point, va-t'en à la mer et jette l'hameçon, et prends le premier poisson qui viendra ; ouvre-lui la bouche, tu trouveras un statère. Prends-le et donne-le-leur pour moi et pour toi.
\Chap{18}
\TextTitle{L'humilité, secret de la vraie grandeur\FTNTT{Mc. 9:33-37; Lu. 9:46-48}}
\VerseOne{}En cette même heure-là, les disciples s'approchèrent de Jésus, en lui disant : Qui est le plus grand dans le Royaume des cieux ?
\VS{2}Et Jésus ayant appelé un petit enfant, le mit au milieu d'eux,
\VS{3}et leur dit : Je vous le dis en vérité, que si vous ne vous convertissez pas et si vous ne devenez pas comme les petits enfants, vous n'entrerez pas dans le Royaume des cieux.
\VS{4}C'est pourquoi quiconque deviendra humble, comme ce petit enfant, celui-là est le plus grand dans le Royaume des cieux.
\VS{5}Et quiconque reçoit en mon Nom un petit enfant comme celui-ci, il me reçoit.
\VS{6}Mais, quiconque scandalise un de ces petits qui croient en moi, il vaudrait mieux pour lui qu'on mette à son cou une meule d'âne, et qu'on le jette au fond de la mer.
\TextTitle{Les scandales et les occasions de chute}
\VS{7}Malheur au monde à cause des scandales ! Car il est nécessaire qu'il arrive des scandales ; mais malheur à l'homme par qui le scandale arrive !
\VS{8}Si ta main ou ton pied est pour toi une occasion de chute, coupe-les et jette-les loin de toi ; car il vaut mieux que tu entres boiteux ou manchot dans la vie, que d'avoir deux pieds ou deux mains, et d'être jeté dans le feu éternel.
\VS{9}Et si ton œil est pour toi une occasion de chute, arrache-le et jette-le loin de toi ; car il vaut mieux que tu entres dans la vie n'ayant qu'un œil, que d'avoir deux yeux, et d'être jeté dans le feu de la géhenne.
\VS{10}Gardez-vous de mépriser un seul de ces petits ; car je vous dis que dans les cieux leurs anges voient continuellement la face de mon Père qui est aux cieux.
\VS{11}Car le Fils de l'homme est venu pour sauver ce qui était perdu.
\TextTitle{Parabole de la brebis égarée\FTNTT{Lu. 15:3-7}}
\VS{12}Que vous en semble ? Si un homme a cent brebis, et que l'une d'elles s'égare, ne laisse-t-il pas les quatre-vingt-dix-neuf autres, pour aller dans les montagnes chercher celle qui s'est égarée ?
\VS{13}Et, s'il arrive qu'il la trouve, je vous le dis en vérité, qu'il en a plus de joie, que les quatre-vingt-dix-neuf qui ne se sont pas égarées.
\VS{14}Ainsi la volonté de votre Père qui est aux cieux n'est pas qu'un seul de ces petits périsse.
\TextTitle{Discipline dans les assemblées}
\VS{15}Que si ton frère a péché contre toi, va, et reprends-le entre toi et lui seul. S'il t'écoute, tu as gagné ton frère.
\VS{16}Mais s'il ne t'écoute pas, prends encore avec toi une ou deux personnes, afin que par la bouche de deux ou trois témoins toute parole soit ferme\FTNT{De. 19:15.}.
\VS{17}S'il refuse de les écouter, dis-le à l'Eglise ; et s'il refuse aussi d'écouter l'Eglise, qu'il soit pour toi comme un païen et comme un publicain.
\VS{18}En vérité je vous dis que tout ce que vous lierez sur la terre, sera lié dans le ciel ; et tout ce que vous délierez sur la terre sera délié dans le ciel\FTNT{Voir commentaire en Mt. 16:19.}.
\VS{19}Je vous dis aussi que si deux d'entre vous s'accordent sur la terre, tout ce qu'ils demanderont leur sera donné par mon Père qui est aux cieux.
\VS{20}Car là où deux ou trois sont assemblés en mon Nom, je suis là au milieu d'eux.
\TextTitle{Ne jamais se lasser de pardonner}
\VS{21}Alors Pierre s'approchant, lui dit : Seigneur, combien de fois mon frère péchera-il contre moi et lui pardonnerai-je ? Sera-ce jusqu'à sept fois ?
\VS{22}Jésus lui répondit : Je ne te dis pas jusqu'à sept fois, mais jusqu'à soixante-dix fois sept fois.
\TextTitle{Parabole du roi et du méchant serviteur}
\VS{23}C'est pourquoi le Royaume des cieux est semblable à un roi qui voulut faire rendre compte à ses serviteurs.
\VS{24}Et quand il se mit à compter, on lui en présenta un qui lui devait dix mille talents.
\VS{25}Et parce qu'il n'avait pas de quoi payer, son maître ordonna qu'il soit vendu, lui, sa femme, ses enfants et tout ce qu'il avait, et que la dette soit payée.
\VS{26}Mais ce serviteur se jetant à ses pieds, le suppliait en disant : Seigneur, aie patience envers moi et je te rendrai le tout.
\VS{27}Alors le maître de ce serviteur, ému de compassion, le relâcha et lui remit la dette.
\VS{28}Mais ce serviteur étant sorti, rencontra un de ses compagnons de service, qui lui devait cent deniers ; et l'ayant pris, il l'étranglait, en lui disant : paye-moi ce que tu me dois.
\VS{29}Mais son compagnon de service se jetant à ses pieds, le suppliait en disant : Aie patience et je te rendrai le tout.
\VS{30}Mais l'autre ne voulut pas et il alla le jeter en prison, jusqu'à ce qu'il ait payé la dette.
\VS{31}Or ses autres compagnons de service voyant ce qui était arrivé, en furent extrêmement attristés et ils allèrent raconter à leur maître tout ce qui s'était passé.
\VS{32}Alors son maître le fit venir et lui dit : Méchant serviteur, je t'avais remis en entier ta dette, parce que tu m'en avais supplié ;
\VS{33}Ne te fallait-il pas aussi avoir pitié de ton compagnon de service, comme j'avais eu pitié de toi ?
\VS{34}Et son maître étant en colère le livra aux bourreaux, jusqu'à ce qu'il lui ait payé tout ce qu'il devait.
\VS{35}C'est ainsi que vous fera mon Père céleste, si vous ne pardonnez de tout votre cœur, chacun à son frère, ses fautes.
\Chap{19}
\TextTitle{Enseignement de Jésus sur le mariage et le divorce\FTNTT{Mt. 5:31-32 ; Mc. 10:2-12 ; Lu. 16:18 ; Ro. 7:1-3 ; 1 Co. 7:10-16}}
\VerseOne{}Et il arriva que quand Jésus eut achevé ces discours, il quitta la Galilée, et alla dans le territoire de la Judée, au-delà du Jourdain.
\VS{2}Et de grandes foules le suivirent, et là il guérit leurs malades.
\VS{3}Alors les pharisiens vinrent à lui pour l'éprouver, et ils lui dirent : Est-il permis à un homme de répudier sa femme pour quelque cause que ce soit ?
\VS{4}Et il répondit et leur dit : N'avez-vous pas lu que le Créateur, au commencement, fit l'homme et la femme ?
\VS{5}Et il dit : A cause de cela, l'homme quittera son père et sa mère, et s'attachera à sa femme, et les deux ne seront qu'une seule chair ?
\VS{6}Ainsi ils ne sont plus deux, mais une seule chair\FTNT{Ge. 2:24.}. Que l'homme donc ne sépare pas ce que Dieu a mis ensemble sous un joug\FTNT{La majorité des traducteurs traduisent ce verset par « Que l'homme donc ne sépare pas ce que Dieu a joint. ». Or le terme grecque « suzeugnumi » qu'ils ont traduit par « joint » signifie plutôt « attacher un joug à quelqu'un, mettre ensemble sous un joug ». Un joug est une pièce de bois servant à atteler une paire d'animaux. De ce fait, les animaux sont contraints d'avancer dans la même direction, côte à côte (Ge. 2:22). En De. 22:10, Dieu interdit d'atteler un âne avec un bœuf ensemble. L'adverbe « ensemble » vient de l'hébreu « Yachad » et signifie « union d'une façon unitaire ». Ce verset fait référence symboliquement aux paroles de l'apôtre Paul en 2 Co. 6:14-16 qui nous mettent en garde contre le mariage avec des infidèles. Le mariage est donc semblable à un joug qui nous contraint à marcher à l'unisson dans la même direction. Ainsi, si on se lie à un inconverti, ce dernier risque de nous entraîner sur la voie de la perdition. En Mt 11:29, Christ nous invite à nous mettre sous son joug, qui est doux et léger. Quelle belle demande en mariage !}.
\VS{7}Ils lui dirent : Pourquoi donc Moïse a-t-il commandé de donner la lettre de divorce, et de répudier sa femme\FTNT{De. 24:1.} ?
\VS{8}Il leur répondit : C'est à cause de la dureté de votre cœur que Moïse vous a permis de répudier vos femmes, mais au commencement il n'en était pas ainsi.
\VS{9}Et moi je vous dis, que quiconque répudiera sa femme, si ce n'est pour cause d'adultère\FTNT{Adultère : Du grec « porneia » c'est-à-dire relation sexuelle illicite, impudicité.}, et se mariera à une autre, commet un adultère ; et que celui qui se sera marié à celle qui est répudiée, commet un adultère.
\VS{10}Ses disciples lui dirent : Si telle est la condition de l'homme à l'égard de sa femme, il ne convient pas de se marier.
\VS{11}Mais il leur répondit : Tous ne sont pas capables de cela, mais seulement ceux à qui il est donné.
\VS{12}Car il y a des eunuques, qui sont ainsi nés dés le ventre de leur mère ; et il y a des eunuques, qui ont été faits eunuques par les hommes ; et il y a des eunuques qui se sont faits eux-mêmes eunuques pour le Royaume des cieux. Que celui qui peut comprendre ceci, le comprenne.
\TextTitle{Le Royaume des cieux pour ceux qui ressemblent aux petits enfants\FTNTT{Mc. 10:13-16 ; Lu. 18:15-17}}
\VS{13}Alors on lui présenta des petits enfants, afin qu'il leur impose les mains et qu'il prie pour eux. Mais les disciples les en reprenaient.
\VS{14}Et Jésus leur dit : Laissez venir à moi les petits enfants et ne les empêchez pas ; car le Royaume des cieux est pour ceux qui leur ressemblent.
\VS{15}Puis il leur imposa les mains et il partit de là.
\TextTitle{Le jeune homme riche\FTNTT{Mc. 10:17-31 ; Lu. 10:25-37 ; Lu. 18:18-27.}}
\VS{16}Et voici, quelqu'un s'approchant lui dit : Maître qui est bon, quel bien ferai-je pour avoir la vie éternelle ?
\VS{17}Il lui répondit : Pourquoi m'appelles-tu bon ? Dieu est le seul être qui soit bon. Que si tu veux entrer dans la vie, garde les commandements.
\VS{18}Il lui dit : Lesquels ? Et Jésus lui répondit : Tu ne tueras point. Tu ne commettras point d'adultère. Tu ne déroberas point. Tu ne diras point de faux témoignage.
\VS{19}Honore ton père et ta mère ; et tu aimeras ton prochain comme toi-même\FTNT{Ex. 20:12-16 ; Lé. 19:18.}.
\VS{20}Le jeune homme lui dit : J'ai gardé toutes ces choses dès ma jeunesse. Que me manque-t-il encore ?
\VS{21}Jésus lui dit : Si tu veux être parfait, va, vends ce que tu as, et donne-le aux pauvres, et tu auras un trésor dans le ciel ; puis viens et suis-moi.
\VS{22}Mais quand ce jeune homme eut entendu cette parole, il s'en alla tout triste, parce qu'il avait de grands biens.
\VS{23}Alors Jésus dit à ses disciples : Je vous le dis en vérité, un riche entrera difficilement dans le Royaume des cieux.
\VS{24}Je vous le dis encore : Il est plus aisé à un chameau de passer par le trou d'une aiguille\FTNT{Le trou d'une aiguille : Jésus fait référence à une porte de la ville de Jérusalem qui était trop basse pour que les chameaux puissent y passer avec leurs chargements.}, qu'il ne l'est qu'un riche entre dans le Royaume de Dieu.
\VS{25}Ses disciples ayant entendu ces choses furent très étonnés et dirent : Qui peut donc être sauvé ?
\VS{26}Et Jésus les regarda et leur dit : Quant aux hommes, cela est impossible, mais quant à Dieu toutes choses sont possibles.
\TextTitle{Récompenses actuelles et dans le Royaume à venir\FTNTT{Mc. 10:28-31 ; Lu. 18:28-30}}
\VS{27}Alors Pierre prenant la parole, lui dit : Voici, nous avons tout quitté et nous t'avons suivi ; que nous en arrivera-t-il donc ?
\VS{28}Et Jésus leur dit : Je vous le dis en vérité, quand le Fils de l'homme, au renouvellement de toutes choses, sera assis sur le trône de sa gloire, vous qui m'avez suivi, vous serez assis sur douze trônes et vous jugerez les douze tribus d'Israël.
\VS{29}Et quiconque aura quitté ou maisons, ou frères, ou sœurs, ou père, ou mère, ou femme, ou enfants, ou champs, à cause de mon Nom, il en recevra cent fois autant et héritera la vie éternelle.
\VS{30}Mais plusieurs qui sont les premiers seront les derniers et les derniers seront les premiers.
\Chap{20}
\TextTitle{Parabole des ouvriers}
\VerseOne{} Car le Royaume des cieux est semblable à un père de famille, qui sortit dès le point du jour afin de louer des ouvriers pour sa vigne.
\VS{2}Et quand il eut accordé avec les ouvriers à un denier par jour, il les envoya à sa vigne.
\VS{3}Puis étant sorti vers la troisième heure, il en vit d'autres qui étaient sur la place publique, sans rien faire.
\VS{4}Il leur dit : Allez aussi à ma vigne, et je vous donnerai ce qui sera raisonnable.
\VS{5}Et ils y allèrent. Puis il sortit de nouveau vers la sixième heure et vers la neuvième, et il fit de même.
\VS{6}Et étant sorti vers la onzième heure, il en trouva d'autres qui étaient sur la place publique sans rien faire, et il leur dit : Pourquoi vous tenez-vous ici toute la journée sans rien faire ?
\VS{7}Ils lui répondirent : Parce que personne ne nous a loués. Et il leur dit : Allez-vous aussi à ma vigne et vous recevrez ce qui sera raisonnable.
\VS{8}Et le soir étant venu, le maître de la vigne dit à son intendant : Appelle les ouvriers et paye-leur le salaire, en commençant depuis les derniers jusqu'aux premiers.
\VS{9}Alors ceux qui avaient été loués vers la onzième heure vinrent et reçurent chacun un denier.
\VS{10}Or quand les premiers furent venus, ils croyaient recevoir davantage, mais ils reçurent aussi chacun un denier.
\VS{11}Et l'ayant reçu, ils murmuraient contre le père de famille,
\VS{12}en disant : Ces derniers n'ont travaillé qu'une heure, et tu les as faits égaux à nous, qui avons supporté le poids du jour et la chaleur.
\VS{13}Et il répondit à l'un d'eux et lui dit : Mon ami, je ne te fais pas de tort, n'es-tu pas tombé d'accord avec moi pour un denier ?
\VS{14}Prends ce qui est à toi et va-t'en. Mais je veux donner à ce dernier autant qu'à toi,
\VS{15}ne m'est-il pas permis de faire ce que je veux de mes biens ? Ou vois-tu d'un mauvais œil que je sois bon ?
\VS{16}Ainsi les derniers seront les premiers et les premiers seront les derniers, car il y a beaucoup d'appelés, mais peu d'élus.
\TextTitle{Jésus annonce à nouveau sa mort et sa résurrection\FTNTT{Mt. 12:38-42 ; 16:21-28 ; 17:22-23 ; Mc. 10:32-34 ; Lu. 18:31-34.}}
\VS{17}Pendant que Jésus montait à Jérusalem, il prit à part ses douze disciples et il leur dit en chemin :
\VS{18}Voici, nous montons à Jérusalem, et le Fils de l'homme sera livré aux principaux sacrificateurs et aux scribes, et ils le condamneront à la mort.
\VS{19}Ils le livreront aux nations pour qu'elles se moquent de lui, le battent de verges et le crucifient ; et le troisième jour il ressuscitera.
\TextTitle{Réponse de Jésus à la requête de la mère de Jacques et Jean\FTNTT{Mc. 10:35-45}}
\VS{20}Alors la mère des fils de Zébédée s'approcha de lui avec ses fils et se prosterna pour lui demander quelque chose.
\VS{21}Et il lui dit : Que veux-tu ? Elle lui dit : Ordonne que mes deux fils, qui sont ici, soient assis l'un à ta droite, et l'autre à ta gauche dans ton Royaume.
\VS{22}Et Jésus répondit et dit : Vous ne savez pas ce que vous demandez. Pouvez-vous boire la coupe que je dois boire, et être baptisés du baptême dont je dois être baptisé ? Ils lui répondirent : Nous le pouvons.
\VS{23}Et il leur dit : Il est vrai que vous boirez ma coupe et que vous serez baptisés du baptême dont je serai baptisé ; mais pour ce qui est d'être assis à ma droite ou à ma gauche, cela ne dépend pas de moi, et ne sera donné qu'à ceux à qui mon Père l'a réservé.
\VS{24}Les dix autres disciples ayant entendu cela, furent indignés contre les deux frères.
\VS{25}Mais Jésus les appela et leur dit : Vous savez que les princes des nations les dominent, et que les grands les asservissent.
\VS{26}Mais il n'en sera pas ainsi entre vous. Au contraire, quiconque veut être grand entre vous, qu'il soit votre serviteur.
\VS{27}Et quiconque veut être le premier parmi vous, qu'il soit votre serviteur.
\VS{28}De même que le Fils de l'homme n'est pas venu pour être servi, mais pour servir, et afin de donner sa vie en rançon pour plusieurs.
\TextTitle{Jésus guérit deux aveugles\FTNTT{Mc. 10:46-53 ; Lu. 18:35-43}}
\VS{29}Et comme ils partaient de Jéricho, une grande foule le suivit.
\VS{30}Et voici, deux aveugles qui étaient assis au bord du chemin, entendirent que Jésus passait, et crièrent en disant : Seigneur, Fils de David ! Aie pitié de nous !
\VS{31}Et la foule les reprenait pour les faire taire ; mais ils criaient encore plus fort : Seigneur, Fils de David ! Aie pitié de nous !
\VS{32}Jésus s'arrêta les appela et leur dit : Que voulez-vous que je vous fasse ?
\VS{33}Ils lui dirent : Seigneur, que nos yeux soient ouverts.
\VS{34}Et Jésus étant ému de compassion, toucha leurs yeux, et aussitôt ils recouvrèrent la vue et ils le suivirent.
\Chap{21}
\TextTitle{Jésus-Christ se présente publiquement comme Roi\FTNTT{Za. 9:9 ; Mc. 11:1-11 ; Lu. 19:28-40 ; Jn. 12:12-19.}}
\VerseOne{}Et quand ils furent près de Jérusalem, et qu'ils furent arrivés à Bethphagé vers le Mont des Oliviers, Jésus envoya alors deux disciples,
\VS{2}en leur disant : Allez au village qui est devant vous. Vous trouverez une ânesse attachée, et son ânon avec elle. Détachez-les et amenez-les-moi.
\VS{3}Et si quelqu'un vous dit quelque chose, vous direz que le Seigneur en a besoin ; et aussitôt il les laissera aller.
\VS{4}Or, tout cela arriva afin que s'accomplisse ce qui avait été annoncé par le prophète, en disant :
\VS{5}Dites à la fille de Sion : Voici, ton Roi vient à toi, plein de douceur, et monté sur un âne, sur un ânon, le petit d'une ânesse\FTNT{Za. 9:9.}.
\VS{6}Les disciples donc s'en allèrent et firent ce que Jésus leur avait ordonné.
\VS{7}Et ils amenèrent l'ânesse et l'ânon, et mirent leurs vêtements sur eux, et le firent asseoir dessus.
\VS{8}Alors de grandes foules étendirent leurs vêtements sur le chemin, et les autres coupaient des rameaux des arbres, et les étendaient sur le chemin.
\VS{9}Et les foules qui allaient devant, et celles qui suivaient, criaient en disant : Hosanna au Fils de David ! Béni soit celui qui vient au Nom du Seigneur ! Hosanna dans les lieux très hauts !
\VS{10}Lorsqu'il entra dans Jérusalem, toute la ville fut émue et l'on disait : Qui est celui-ci ?
\VS{11}Et les foules disaient : C'est Jésus, le prophète de Nazareth en Galilée.
\TextTitle{Jésus chasse les marchands du temple\FTNTT{Mc. 11:15-18 ; Lu. 19:45-46 ; Jn. 2:13-16}}
\VS{12}Jésus entra dans le temple de Dieu. Il chassa dehors tous ceux qui vendaient et qui achetaient dans le temple ; il renversa les tables des changeurs et les sièges de ceux qui vendaient des pigeons ;
\VS{13}et il leur dit : Il est écrit : Ma maison sera appelée une maison de prière, mais vous en avez fait une caverne de voleurs\FTNT{Es. 56:7 ; Jé. 7:11.}.
\VS{14}Alors des aveugles et des boiteux s'approchèrent de lui dans le temple et il les guérit.
\VS{15}Mais les principaux sacrificateurs et les scribes furent indignés à la vue des choses merveilleuses qu'il avait faites, et des enfants qui criaient dans le temple : Hosanna au Fils de David !
\VS{16}Et ils lui dirent : Entends-tu ce qu'ils disent ? Oui, leur répondit Jésus. N'avez-vous jamais lu ces paroles : Tu as tiré des louanges de la bouche des enfants, et de ceux qui sont à la mamelle\FTNT{Ps. 8:3.} ?
\VS{17}Et, les ayant laissés, il sortit de la ville, pour aller à Béthanie, où il passa la nuit.
\TextTitle{Le figuier stérile\FTNTT{Mc. 11:12-14,20-26}}
\VS{18}Le matin, comme il retournait à la ville, il eut faim.
\VS{19}Et voyant un figuier qui était sur le chemin, il s'en approcha, mais il n'y trouva que des feuilles ; et il lui dit : Qu'aucun fruit ne naisse plus jamais de toi ! Et aussitôt le figuier sécha.
\VS{20}Les disciples qui virent cela furent étonnés et dirent : Comment ce figuier est-il devenu sec en un instant ?
\VS{21}Jésus leur répondit : Je vous le dis en vérité, si vous aviez la foi, et que vous ne doutiez point, non seulement vous ferez ce qui a été fait à ce figuier, mais quand vous diriez à cette montagne : Ôte-toi de là et jette-toi dans la mer, cela se ferait.
\VS{22}Et quoi que vous demandiez en priant Dieu si vous croyez, vous le recevrez.
\TextTitle{L'incrédulité des principaux sacrificateurs et des anciens\FTNTT{Mc. 11:27-33 ; Lu. 20:1-8}}
\VS{23}Puis, s'étant rendu dans le temple, les principaux sacrificateurs et les anciens du peuple vinrent auprès de lui, pendant qu'il enseignait, et lui dirent : Par quelle autorité fais-tu ces choses ; et qui t'a donné cette autorité ?
\VS{24}Jésus répondant leur dit : Je vous interrogerai aussi sur une chose, et si vous me répondez, je vous dirai par quelle autorité je fais ces choses.
\VS{25}Le baptême de Jean d'où venait-il ? Du ciel ou des hommes ? Mais ils raisonnèrent ainsi entre eux : Si nous disons : Du ciel, il nous dira : Pourquoi n'avez-vous pas cru en lui ?
\VS{26}Et si nous disons : Des hommes, nous craignons la foule, car tous tiennent Jean pour un prophète.
\VS{27}Alors ils répondirent à Jésus : Nous ne savons pas. Et il leur dit : Moi non plus, je ne vous dirai pas par quelle autorité je fais ces choses.
\TextTitle{Parabole des deux fils}
\VS{28}Mais que vous en semble ? Un homme avait deux fils ; et s'adressant au premier, il lui dit : Mon fils, va travailler aujourd'hui dans ma vigne.
\VS{29}Il répondit : Je ne veux pas y aller. Ensuite il se repentit et y alla.
\VS{30}S'adressant à l'autre, il lui dit la même chose. Et ce fils répondit : Je veux bien, seigneur. Et il n'alla pas.
\VS{31}Lequel des deux a fait la volonté du père ? Ils lui répondirent : Le premier. Et Jésus leur dit : Je vous le dis en vérité, les publicains et les prostituées vous devanceront dans le Royaume de Dieu.
\VS{32}Car Jean est venu à vous dans la voie de la justice et vous ne l'avez pas cru ; mais les publicains et les femmes débauchées ont cru en lui. Et vous, qui avez vu cela, vous ne vous êtes pas ensuite repentis pour croire en lui.
\TextTitle{Parabole des vignerons\FTNTT{Es. 5:1-7 ; Mc. 12:1-12 ; Lu. 20:9-18.}}
\VS{33}Ecoutez une autre parabole : Il y avait un père de famille qui planta une vigne, et l'entoura d'une haie, et y creusa un pressoir, et bâtit une tour ; puis il l'afferma à des vignerons, et quitta le pays.
\VS{34}Lorsque la saison de la récolte fut arrivé, il envoya ses serviteurs vers les vignerons pour recevoir les fruits.
\VS{35}Mais les vignerons s'étant saisis de ses serviteurs, fouettèrent l'un, tuèrent l'autre et lapidèrent le troisième.
\VS{36}Il envoya encore d'autres serviteurs en plus grand nombre que les premiers, et ils leur firent de même.
\VS{37}Enfin, il envoya vers eux son propre fils, en disant : Ils auront du respect pour mon fils.
\VS{38}Mais quand les vignerons virent le fils, ils dirent entre eux : Voici l'héritier. Venez, tuons-le et emparons-nous de son héritage.
\VS{39}Et s'étant saisis de lui, il le jetèrent hors de la vigne et le tuèrent.
\VS{40}Quand donc le maître de la vigne viendra, que fera-t-il à ces vignerons ?
\VS{41}Ils lui dirent : Il les fera périr malheureusement comme des méchants et louera sa vigne à d'autres vignerons, qui lui en rendront les fruits en leur saison.
\VS{42}Et Jésus leur dit : N'avez-vous jamais lu dans les Ecritures : La pierre qu'ont rejetée ceux qui bâtissaient, est devenue la principale de l'angle. C'est du Seigneur que cela est venu, et c'est un prodige à nos yeux\FTNT{Es. 8:13-17 ; Es. 28:16.} ?
\VS{43}C'est pourquoi je vous dis que le Royaume de Dieu vous sera enlevé et il sera donné à une nation qui en rendra les fruits.
\VS{44}Celui qui tombera sur cette pierre s'y brisera, et celui sur qui elle tombera sera écrasé.
\VS{45}Après avoir entendu ses paraboles, les principaux sacrificateurs et les pharisiens comprirent qu'il parlait d'eux.
\VS{46}Et ils cherchaient à se saisir de lui, mais ils craignaient la foule, parce qu'elle le tenait pour un prophète.
\Chap{22}
\TextTitle{Parabole des noces\FTNTT{Lu. 14:16-24}}
\VerseOne{}Alors Jésus, prenant la parole, leur parla de nouveau en paraboles et il dit :
\VS{2}Le Royaume des cieux est semblable à un roi qui fit des noces pour son fils.
\VS{3}Il envoya ses serviteurs pour appeler ceux qui avaient été conviés aux noces ; mais ils ne voulurent pas venir.
\VS{4}Il envoya encore d'autres serviteurs, disant : Dites aux conviés : Voici, j'ai préparé mon festin ; mes bœufs et mes bêtes grasses sont tués, et tout est prêt ; venez aux noces.
\VS{5}Mais, sans tenir compte de l'invitation, ils s'en allèrent l'un à son champ, et l'autre à son trafic.
\VS{6}Et les autres se saisirent de ses serviteurs les outragèrent, et les tuèrent.
\VS{7}Quand le roi l'entendit, il se mit en colère ; il envoya ses troupes, fit périr ces meurtriers et brûla leur ville.
\VS{8}Puis il dit à ses serviteurs : Les noces sont prêtes, mais les conviés n'en étaient pas dignes.
\VS{9}Allez donc dans les carrefours des chemins, et autant de gens que vous trouverez, appelez-les aux noces.
\VS{10}Alors ces serviteurs allèrent dans les chemins et rassemblèrent tous ceux qu'ils trouvèrent, méchants et bons, et la salle des noces fut remplie de conviés qui étaient à table.
\VS{11}Et le roi étant entré pour voir ceux qui étaient à table, il aperçut là un homme qui n'avait pas revêtu un habit de noces\FTNT{Ap. 19:7-8.}.
\VS{12}Et il lui dit : Mon ami, comment es-tu entré ici sans avoir un habit de noces ? Cet homme eut la bouche fermée.
\VS{13}Alors le roi dit aux serviteurs : Liez-lui les pieds et les mains, emportez-le et jetez-le dans les ténèbres de dehors, où il y aura des pleurs et des grincements de dents.
\VS{14}Car il y a beaucoup d'appelés, mais peu d'élus.
\TextTitle{Le tribut dû à César\FTNTT{Mc. 12:13-17 ; Lu. 20:19-26}}
\VS{15}Alors les pharisiens allèrent se consulter ensemble sur les moyens de le surprendre par ses propres paroles.
\VS{16}Ils envoyèrent auprès de lui leurs disciples, avec des hérodiens, qui dirent : Maître, nous savons que tu es véritable, que tu enseignes la voie de Dieu selon la vérité, sans t'inquiéter de personne ; car tu ne regardes point à l'apparence des hommes.
\VS{17}Dis-nous donc ce qu'il t'en semble : Est-il permis de payer le tribut à César, ou non ?
\VS{18}Et Jésus connaissant leur malice, dit : Hypocrites, pourquoi me tentez-vous ?
\VS{19}Montrez-moi la monnaie avec laquelle on paie le tribut ; et ils lui présentèrent un denier.
\VS{20}Il leur demanda : De qui porte-t-il l'image et l'inscription ?
\VS{21}De César, lui répondirent-ils. Alors il leur dit : Rendez donc à César ce qui est à César, et à Dieu, ce qui est à Dieu.
\VS{22}Et ayant entendu cela, ils furent étonnés, ils le quittèrent et s'en allèrent.
\TextTitle{Enseignement de Jésus sur la résurrection\FTNTT{Mc. 12:18-27 ; Lu. 20:27-38}}
\VS{23}Le même jour, les sadducéens, qui disent qu'il n'y a pas de résurrection, vinrent auprès de lui et lui posèrent cette question,
\VS{24}en disant : Maître, Moïse a dit : Si quelqu'un meurt sans enfants, son frère épousera sa femme et suscitera une postérité à son frère.
\VS{25}Or, il y avait parmi nous sept frères. Le premier se maria et mourut ; et, n'ayant pas eu d'enfants, il laissa sa femme à son frère.
\VS{26}Il en fut de même du deuxième, puis du troisième, jusqu'au septième.
\VS{27}Après eux tous, la femme mourut aussi.
\VS{28}A la résurrection, duquel des sept sera-t-elle la femme ? Car tous l'ont eue.
\VS{29}Mais Jésus répondant leur dit : Vous êtes dans l'erreur, parce que vous ne connaissez ni les Ecritures, ni la puissance de Dieu.
\VS{30}Car à la résurrection on ne prendra ni on ne donnera de femmes en mariage, mais on sera comme les anges de Dieu dans le ciel.
\VS{31}Et quant à la résurrection des morts, n'avez-vous point lu ce dont Dieu vous a parlé, disant :
\VS{32}Je suis le Dieu d'Abraham, le Dieu d'Isaac et le Dieu de Jacob\FTNT{Ge. 17:7 ; Ge. 26:24 ; Ge. 28:21.}. Or Dieu n'est pas le Dieu des morts, mais des vivants.
\VS{33}Ce que les foules ayant entendu, elles admirèrent sa doctrine.
\TextTitle{Le plus grand commandement de la loi\FTNTT{Mc. 12:28-34 ; Lu. 10:25-28}}
\VS{34}Quand les pharisiens apprirent qu'il avait fermé la bouche aux sadducéens, ils se rassemblèrent dans un même lieu,
\VS{35}et l'un d'eux, qui était docteur de la loi, l'interrogea pour l'éprouver, en disant :
\VS{36}Maître, quel est le plus grand commandement de la loi ?
\VS{37}Jésus lui dit : Tu aimeras le Seigneur ton Dieu de tout ton cœur, de toute ton âme et de toute ta pensée.
\VS{38}Celui-ci est le premier et le grand commandement.
\VS{39}Et voici le deuxième qui lui est semblable : Tu aimeras ton prochain comme toi-même.
\VS{40}De ces deux commandements dépendent toute la loi et les prophètes.
\TextTitle{Jésus interroge les pharisiens au sujet du Messie\FTNTT{Mc. 12:35-37 ; Lu. 20:39-44}}
\VS{41}Et les Pharisiens étant assemblés, Jésus les interrogea,
\VS{42}Disant : que pensez-vous du Christ ? De qui est-il Fils ? Ils lui répondirent : de David.
\VS{43}Et il leur dit : Comment donc David, parlant par l'Esprit, l'appelle-t-il son Seigneur ? Disant :
\VS{44}Le Seigneur a dit à mon Seigneur, assieds-toi à ma droite, jusqu'à ce que j'aie mis tes ennemis pour le marchepied de tes pieds\FTNT{Ps. 110:1.}.
\VS{45}Si donc David l'appelle son Seigneur, comment est-il son Fils ?
\VS{46}Et personne ne pouvait lui répondre un seul mot. Et depuis ce jour, personne n'osa plus lui poser des questions.
\Chap{23}
\TextTitle{Caractéristiques des scribes et des pharisiens\FTNTT{Mc. 12:38-40 ; Lu. 11:39-54 ; Lu. 20:45-47}}
\VerseOne{}Alors Jésus parla à la foule et à ses disciples,
\VS{2}disant : Les scribes et les pharisiens sont assis dans la chaire de Moïse.
\VS{3}Toutes les choses donc qu'ils vous diront d'observer, observez-les et faites-les, mais non point leurs œuvres : parce qu'ils disent et ne font pas.
\VS{4}Car ils lient ensemble des fardeaux pesants et insupportables et les mettent sur les épaules des hommes ; mais ils ne veulent point les remuer de leur doigt.
\VS{5}Et ils font toutes leurs œuvres pour être vus des hommes. Ainsi, ils portent de larges phylactères et de longues franges à leurs vêtements.
\VS{6}Ils aiment les premières places dans les festins, et les premiers sièges dans les synagogues.
\VS{7}Ils aiment les salutations dans les places publiques, et à être appelés par les hommes : Notre maître ! Notre maître !
\VS{8}Mais vous, ne vous faites pas appeler, Notre maître ; car Christ seul est votre Docteur ; et vous êtes tous frères.
\VS{9}Et n'appelez personne sur la terre votre père ; car un seul est votre Père, celui qui est dans les cieux.
\VS{10}Et ne soyez point appelés Docteurs : car Christ seul est votre Docteur.
\VS{11}Mais que celui qui est le plus grand entre vous, soit votre serviteur.
\VS{12}Car quiconque s'élèvera sera abaissé ; et quiconque s'abaissera, sera élevé.
\VS{13}Mais malheur à vous, scribes et pharisiens hypocrites, qui fermez le Royaume des cieux aux hommes : car vous-mêmes n'y entrez point, et vous n'y laissez pas entrer ceux qui veulent y entrer.
\VS{14}Malheur à vous, scribes et pharisiens hypocrites, car vous dévorez les maisons des veuves, même sous le prétexte de faire de longues prières, c'est pourquoi vous en recevrez une plus grande condamnation.
\VS{15}Malheur à vous, scribes et pharisiens hypocrites ! Parce que vous courez la mer et la terre pour faire un prosélyte, et quand il l'est devenu, vous le rendez fils de la géhenne, deux fois plus que vous.
\VS{16}Malheur à vous conducteurs aveugles, qui dites : Si quelqu'un jure par le temple, ce n'est rien ; mais si quelqu'un jure par l'or du temple, il est engagé.
\VS{17}Insensés et aveugles ! Car lequel est le plus grand, l'or, ou le temple qui sanctifie l'or ?
\VS{18}Si quelqu'un, dites-vous encore, jure par l'autel, ce n'est rien ; mais si quelqu'un jure par l'offrande qui est sur l'autel, il est engagé.
\VS{19}Insensés et aveugles ! Car lequel est le plus grand, l'offrande, ou l'autel qui sanctifie l'offrande ?
\VS{20}Celui donc qui jure par l'autel, jure par l'autel et par toutes les choses qui sont dessus.
\VS{21}Celui qui jure par le temple, jure par le temple et par celui qui y habite ;
\VS{22}et celui qui jure par le ciel, jure par le trône de Dieu et par celui qui y est assis.
\VS{23}Malheur à vous, scribes et pharisiens hypocrites ! Parce que vous payez la dîme\FTNT{Voir commentaire en Mal. 3:10.} de la menthe, de l'aneth et du cumin ; et vous laissez les choses les plus importantes de la loi, c'est-à-dire la justice, la miséricorde et la fidélité. Il fallait pratiquer ces choses-là, sans négliger les autres choses.
\VS{24}Conducteurs aveugles ! Vous coulez le moucheron et vous engloutissez le chameau\FTNT{Les pharisiens filtraient leur eau par crainte d'avaler un moucheron.}.
\VS{25}Malheur à vous, scribes et pharisiens hypocrites ! Parce que vous nettoyez le dehors de la coupe et du plat ; alors qu'au-dedans ils sont pleins de rapine et d'intempérance.
\VS{26}Pharisien aveugle, nettoie premièrement l'intérieur de la coupe et du plat, afin que l'extérieur aussi devienne net.
\VS{27}Malheur à vous, scribes et pharisiens hypocrites ! Parce que vous êtes semblables aux sépulcres blanchis, qui paraissent beaux au-dehors, et qui au-dedans sont pleins d'ossements de morts, et de toutes espèces d'impuretés.
\VS{28}Ainsi, au-dehors vous paraissez justes aux hommes, mais au-dedans vous êtes pleins d'hypocrisie et d'iniquité.
\VS{29}Malheur à vous, scribes et pharisiens hypocrites ! Parce que vous bâtissez les tombeaux des prophètes et vous ornez les sépulcres des justes ;
\VS{30}et vous dites : Si nous avions vécu du temps de nos pères, nous n'aurions pas participé avec eux au meurtre des prophètes.
\VS{31}Ainsi vous êtes témoins contre vous-mêmes, que vous êtes les enfants de ceux qui ont fait mourir les prophètes.
\VS{32}Et vous achevez de remplir la mesure de vos pères.
\VS{33}Serpents, race de vipères ! Comment éviterez-vous le supplice de la géhenne ?
\VS{34}Car voici, je vous envoie des prophètes, des sages et des scribes. Vous tuerez et crucifierez les uns, vous battrez de verges les autres dans vos synagogues, et vous les persécuterez de ville en ville,
\VS{35}afin que vienne sur vous tout le sang innocent qui a été répandu sur la terre, depuis le sang d'Abel le juste, jusqu'au sang de Zacharie, fils de Barachie, que vous avez tué entre le temple et l'autel.
\VS{36}Je vous le dis en vérité, que toutes ces choses viendront sur cette génération.
\TextTitle{Lamentations de Jésus sur Jérusalem\FTNTT{Jé. 22:5 ; Lu. 13:34-35 ; 19:41-44.}}
\VS{37}Jérusalem, Jérusalem, qui tues les prophètes, et qui lapides ceux qui te sont envoyés, combien de fois ai-je voulu rassembler tes enfants, comme la poule rassemble ses poussins sous ses ailes, et vous ne l'avez point voulu !
\VS{38}Voici, votre maison va devenir déserte.
\VS{39}Car je vous dis, que désormais vous ne me verrez plus, jusqu'à ce que vous disiez : Béni soit celui qui vient au Nom du Seigneur\FTNT{Ps. 118:26.}!
\Chap{24}
\TextTitle{Prophétie sur la destruction du temple de Jérusalem\FTNTT{Mc. 13:1-2 ; Lu. 21:5-6}}
\VerseOne{}Comme Jésus sortait et s'en allait du temple, ses disciples s'approchèrent de lui pour lui faire remarquer les bâtiments du temple.
\VS{2}Mais Jésus leur dit : Voyez-vous bien toutes ces choses ? Je vous le dis en vérité, il ne restera pas ici pierre sur pierre qui ne soit démolie.
\TextTitle{Le signe de l'accomplissement\FTNTT{Mc. 13:3-4 ; Lu. 21:7}}
\VS{3}Puis s'étant assis sur la Montagne des Oliviers, ses disciples vinrent à lui en particulier et lui dirent : Dis-nous quand ces choses arriveront, et quel sera le signe de ton avènement, et de la fin du monde ?
\TextTitle{Les temps de la fin\FTNTT{Da. 9:27 ; Mc. 13:5-13 ; Lu. 21:8-11}}
\VS{4}Et Jésus répondant leur dit : Prenez garde que personne ne vous séduise.
\VS{5}Car plusieurs viendront sous mon Nom, disant : Je suis le Christ. Et ils en séduiront plusieurs.
\VS{6}Vous entendrez parler de guerres et de bruits de guerres ; gardez-vous d'être troublés ; car il faut que toutes ces choses arrivent ; mais ce ne sera pas encore la fin.
\VS{7}Car une nation s'élèvera contre une autre nation, et un royaume contre un autre royaume ; et il y aura des famines, des pestes, et des tremblements de terre en divers lieux.
\VS{8}Mais toutes ces choses ne seront que le commencement des douleurs.
\VS{9}Alors ils vous livreront aux tourments, et vous tueront ; et vous serez haïs de toutes les nations, à cause de mon Nom.
\VS{10}Alors aussi plusieurs seront scandalisés, se trahiront et se haïront les uns les autres.
\VS{11}Et il s'élèvera plusieurs faux prophètes, qui en séduiront plusieurs.
\VS{12}Et parce que l'iniquité sera multipliée, la charité de plusieurs se refroidira.
\VS{13}Mais celui qui persévérera jusqu'à la fin, sera sauvé.
\VS{14}Cet Evangile du Royaume sera prêché dans toute la terre habitable, pour servir de témoignage à toutes les nations, et alors viendra la fin.
\TextTitle{L'abomination de la désolation\FTNTT{Da. 9:27 ; Da.11:32-35 ; Mc. 13:14-18 ; Lu. 21:20-23}}
\VS{15}Or quand vous verrez l'abomination qui causera la désolation, qui a été prédite par Daniel le Prophète\FTNT{Daniel fut le premier à parler de l'abomination de la désolation (Da. 9:24-27). « Et les forces se présenteront de sa part, elles profaneront le sanctuaire, la forteresse, elles feront cesser le sacrifice perpétuel, et dresseront l'abomination qui causera la désolation. » Da. 11:31. Cette prophétie s'est partiellement accomplie en 168 av. J.-C. lorsqu'Antiochus Epiphane (215 av. J.-C. - 163 av. J.-C.), roi de Syrie, défenseur zélé de la culture grecque, finança la construction du temple de Zeus à Athènes. Sa tentative d'hellénisation forcée de la Judée, soutenue par les grands prêtres Jason et Ménélas, provoqua la colère des Juifs traditionalistes. Antiochus avait interdit le culte mosaïque et consacra le temple de Jérusalem aux dieux grecs. En effet, il le pilla et y installa un autel du dieu Baal Shamen, puis il détruisit les murailles de la ville. Dans un édit de décembre 167 av. J.-C., il ordonna d'offrir des porcs en holocauste, interdit la circoncision, la lecture de la Torah, l'observance des fêtes de Yahweh et pourchassa les adversaires de l'hellénisation. En agissant de la sorte, il y avait deux choses principales que cet archétype de l'antichrist espérait changer en Israël : Les temps (le calendrier juif) et la Torah (la loi selon Da. 7:25). La deuxième partie de cette prophétie s'est accomplie en l'an 70 lors de la destruction du temple de Jérusalem par Titus (39-81 ap. J.-C.). La dernière partie de cette prophétie est en train de s'accomplir actuellement dans les assemblées où Satan distille des enseignements erronés au travers des faux prophètes. La prédication d'un évangile expurgé de son caractère christocentrique, de la nécessité de porter sa croix, axé sur les choses de ce monde, maintient les chrétiens dans une vie de péché. Ainsi, alors qu'ils sont censés être des temples vivants du Saint-Esprit (1 Co. 6:19), Satan s'est établi dans leurs cœurs. Enfin, la prophétie de Daniel trouvera son parfait accomplissement pendant le règne de la bête. L'homme impie s'introduira alors dans le temple de Jérusalem qui sera rebâti, se fera passer pour Dieu et se fera adorer à sa place (2 Th. 2:4).}, être établie dans le lieu saint, que celui qui lit ce prophète y fasse attention !
\VS{16}Alors, que ceux qui seront en Judée fuient dans les montagnes ;
\VS{17}et que celui qui sera sur le toit ne descende pas pour emporter quoi que ce soit de sa maison ;
\VS{18}que celui qui sera dans les champs ne retourne pas en arrière pour prendre ses habits.
\VS{19}Malheur aux femmes enceintes, et à celles qui allaiteront en ces jours-là.
\VS{20}Priez pour que votre fuite n'arrive pas en hiver, ni un jour de sabbat\FTNT{Sous la loi mosaïque, il était interdit aux juifs de parcourir plus de 2 000 coudées du lieu où ils se trouvaient pendant le sabbat (Ex. 16:29).}.
\TextTitle{La grande tribulation\FTNTT{Ps. 2:5 ; Jé. 30:5-8 ; Da.12:1 ; Mc. 13:19-23 ; Lu. 21:23-24}}
\VS{21}Car alors, la détresse sera si grande qu'il n'y en a point eu de semblable depuis le commencement du monde jusqu'à présent, et qu'il n'y en aura jamais.
\VS{22}Et si ces jours n'étaient abrégés, personne ne serait sauvé ; mais à cause des élus, ces jours seront abrégés.
\VS{23}Alors si quelqu'un vous dit : Voici, le Christ est ici ; ou, il est là ; ne le croyez point.
\VS{24}Car il s'élèvera de faux christs et de faux prophètes, ils feront de grands prodiges et des miracles, pour séduire même les élus, s'il était possible.
\VS{25}Voici, je vous l'ai prédit.
\VS{26}Si on vous dit : Voici, il est dans le désert, ne sortez point ; voici, il est dans les chambres, ne le croyez point.
\VS{27}Car, comme l'éclair part de l'orient et se montre jusqu'en occident, il en sera de même de l'avènement du Fils de l'homme.
\VS{28}Car là où est le cadavre, là s'assembleront les vautours.
\TextTitle{Retour du Roi sur la terre\FTNTT{Mc. 13:24-27 ; Lu. 21:25-28}}
\VS{29}Aussitôt après ces jours de détresse, le soleil s'obscurcira, la lune ne donnera plus sa lumière, et les étoiles tomberont du ciel, et les puissances des cieux seront ébranlées.
\VS{30}Alors le signe du Fils de l'homme paraîtra dans le ciel, toutes les tribus de la terre se lamenteront en se frappant la poitrine, et verront le Fils de l'homme venant sur les nuées du ciel, avec une grande puissance et une grande gloire.
\VS{31}Il enverra ses anges avec un grand son de trompette et ils rassembleront ses élus, des quatre vents, d'une extrémité des cieux à l'autre.
\TextTitle{Parabole du figuier\FTNTT{Mc. 13:28-31 ; Lu. 21:29-33}}
\VS{32}Mais apprenez la leçon tirée de la parabole du figuier. Dès que ses jeunes branches deviennent tendres et que ses feuilles poussent, vous savez que l'été est proche.
\VS{33}De même, quand vous verrez toutes ces choses, sachez que le Fils de l'homme est proche, à la porte.
\VS{34}Je vous le dis en vérité, cette génération ne passera point, jusqu'à ce que tout cela n'arrive.
\VS{35}Le ciel et la terre passeront, mais mes paroles ne passeront point.
\TextTitle{Exhortation à la vigilance\FTNTT{Mc. 13:32-37 ; Lu. 21:34-38}}
\VS{36}Pour ce qui est du jour et de l'heure, personne ne le sait, ni les anges des cieux, mais mon Père seul.
\VS{37}Mais comme il en était aux jours de Noé, il en sera de même de l'avènement du fils de l'homme.
\VS{38}Car, comme dans les jours avant le déluge, les hommes mangeaient et buvaient, se mariaient, et donnaient en mariage, jusqu'au jour où Noé entra dans l'arche ;
\VS{39}et ils ne connurent point que le déluge viendrait, jusqu'à ce qu'il vint et les emporta tous ; il en sera de même de l'avènement du Fils de l'homme.
\VS{40}Alors, de deux hommes qui seront dans un champ ; l'un sera pris, et l'autre laissé ;
\VS{41}de deux femmes qui moudront au moulin, l'une sera prise et l'autre laissée.
\VS{42}Veillez donc, car vous ne savez point à quelle heure votre Seigneur doit venir.
\VS{43}Mais sachez ceci, que si un père de famille savait à quelle veille de la nuit le voleur doit venir, il veillerait et ne laisserait pas percer sa maison.
\VS{44}C'est pourquoi, vous aussi tenez-vous prêts ; car le Fils de l'homme viendra à l'heure où vous n'y penserez pas.
\VS{45}Quel est donc le serviteur fidèle et prudent, que son maître a établi sur tous ses serviteurs, pour leur donner la nourriture au temps opportun ?
\VS{46}Heureux est ce serviteur que son maître en arrivant trouvera agir de cette manière.
\VS{47}Je vous le dis en vérité, il l'établira sur tous ses biens.
\VS{48}Mais si c'est un méchant serviteur, qui dit en lui-même : Mon maître tarde à venir ;
\VS{49}et s'il se met à battre ses compagnons de service, s'il mange et boit avec les ivrognes,
\VS{50}le maître de ce serviteur viendra le jour où il ne s'y attend pas et à l'heure qu'il ne connaît pas.
\VS{51}Et il le séparera, et le mettra au rang des hypocrites ; là il y aura des pleurs et des grincements de dents.
\Chap{25}
\TextTitle{Parabole des dix vierges}
\VerseOne{}Alors le Royaume des cieux sera semblable à dix vierges qui, ayant pris leurs lampes, allèrent à la rencontre de l'époux.
\VS{2}Or il y en avait cinq sages et cinq folles.
\VS{3}Les folles, en prenant leurs lampes, ne prirent pas d'huile avec elles ;
\VS{4}mais les sages prirent de l'huile dans leurs vases avec leurs lampes.
\VS{5}Et comme l'époux tardait à venir, elles s'assoupirent et s'endormirent toutes.
\VS{6}Or à minuit il se fit un cri disant : Voici, l'époux vient, allez à sa rencontre !
\VS{7}Alors toutes ces vierges se réveillèrent\FTNT{Réveiller : Du grec « egeiro ». Ce terme signifie également ressusciter. Les saints qui attendent le retour du Seigneur connaîtront un réveil après un temps de sommeil spirituel (Ro. 13:11).} et préparèrent leurs lampes.
\VS{8}Et les folles dirent aux sages : Donnez-nous de votre huile, car nos lampes s'éteignent.
\VS{9}Mais les sages répondirent en disant : Nous ne pouvons pas vous en donner, de peur que nous n'en ayons pas assez pour nous et pour vous ; mais allez plutôt chez ceux qui en vendent et achetez-en pour vous-mêmes.
\VS{10}Or pendant qu'elles allaient en acheter, l'époux arriva. Celles qui étaient prêtes entrèrent avec lui dans la salle des noces, puis la porte fut fermée.
\VS{11}Après cela, les autres vierges vinrent aussi, et dirent : Seigneur ! Seigneur ! Ouvre-nous !
\VS{12}Mais il leur répondit, et dit : Je vous le dis en vérité, je ne vous connais point.
\VS{13}Veillez donc ; car vous ne savez ni le jour ni l'heure en laquelle le Fils de l'homme viendra.
\TextTitle{Parabole des talents}
\VS{14}Car il en sera comme d'un homme qui, partant pour un voyage, appela ses serviteurs et leur remit ses biens.
\VS{15}Il donna à l'un cinq talents, à l'autre deux, et au troisième un ; à chacun selon sa capacité ; et aussitôt après il partit.
\VS{16}Celui qui avait reçu les cinq talents, s'en alla, et les fit valoir, et gagna cinq autres talents.
\VS{17}De même, celui qui avait reçu les deux talents, en gagna aussi deux autres.
\VS{18}Mais celui qui n'en avait reçu qu'un, alla et creusa dans la terre, et y cacha l'argent de son maître.
\VS{19}Longtemps après, le maître de ces serviteurs revint et leur fit rendre compte.
\VS{20}Alors celui qui avait reçu les cinq talents, vint et présenta cinq autres talents, en disant : Seigneur, tu m'as confié cinq talents, voici, j'en ai gagné cinq autres par-dessus.
\VS{21}Et son Seigneur lui dit : Cela est bien, bon et fidèle serviteur ; tu as été fidèle en peu de choses, je t'établirai sur beaucoup ; viens participer à la joie de ton Seigneur.
\VS{22}Ensuite, celui qui avait reçu les deux talents, vint et dit : Seigneur, tu m'as confié deux talents ; voici, j'en ai gagné deux autres par-dessus.
\VS{23}Et son Seigneur lui dit : Cela est bien, bon et fidèle serviteur, tu as été fidèle en peu de choses, je t'établirai sur beaucoup ; viens prendre part à la joie de ton Seigneur.
\VS{24}Mais celui qui n'avait reçu qu'un talent, vint et dit : Seigneur, je savais que tu es un homme dur, qui moissonnes où tu n'as point semé, et qui amasses où tu n'as point vanné,
\VS{25}c'est pourquoi craignant de perdre ton talent, je suis allé le cacher dans la terre. Voici, tu as ici ce qui t'appartient.
\VS{26}Et son Seigneur répondant, lui dit : Méchant et lâche serviteur, tu savais que je moissonnais où je n'ai point semé, et que j'amassais où je n'ai point vanné,
\VS{27}il te fallait donc remettre mon argent aux banquiers et à mon retour, je l'aurais retiré avec l'intérêt.
\VS{28}Ôtez-lui donc le talent et donnez-le à celui qui a les dix talents.
\VS{29}Car à celui qui a, il sera donné et il en aura encore plus, mais à celui qui n'a rien, cela même qu'il a, lui sera ôté.
\VS{30}Jetez donc le serviteur inutile dans les ténèbres de dehors ; où il y aura des pleurs et des grincements de dents.
\TextTitle{Séparation et jugement des brebis et des boucs\FTNTT{1 Co. 6:2}}
\VS{31}Or quand le Fils de l'homme viendra environné de sa gloire et accompagné de tous les saints anges, alors il s'assiéra sur le trône de sa gloire.
\VS{32}Et toutes les nations seront assemblées devant lui ; et il séparera les uns d'avec les autres, comme le berger sépare les brebis d'avec les boucs.
\VS{33}Et il mettra les brebis à sa droite et les boucs à sa gauche.
\VS{34}Alors le Roi dira à ceux qui seront à sa droite : Venez, vous qui êtes bénis de mon Père, possédez en héritage le Royaume qui vous a été préparé dès la fondation du monde.
\VS{35}Car j'ai eu faim et vous m'avez donné à manger ; j'ai eu soif et vous m'avez donné à boire ; j'étais étranger et vous m'avez recueilli ;
\VS{36}j'étais nu et vous m'avez vêtu ; j'étais malade et vous m'avez visité ; j'étais en prison et vous êtes venus vers moi.
\VS{37}Alors les justes lui répondront : Seigneur, quand t'avons-nous vu avoir faim et t'avons-nous donné à manger ; ou avoir soif et t'avons-nous donné à boire ?
\VS{38}Quand t'avons-nous vu étranger et t'avons-nous recueilli ; ou nu, et t'avons-nous vêtu ?
\VS{39}Ou quand t'avons-nous vu malade, ou en prison, et sommes-nous allés vers toi ?
\VS{40}Et le Roi répondant, leur dira : Je vous le dis en vérité, toutes les fois que vous avez fait ces choses à l'un de ces plus petits de mes frères, c'est à moi que vous les avez faites.
\VS{41}Alors il dira aussi à ceux qui seront à sa gauche : Maudits, retirez-vous de moi et allez dans le feu éternel, qui a été préparé pour le diable et pour ses anges.
\VS{42}Car j'ai eu faim et vous ne m'avez point donné à manger ; j'ai eu soif et vous ne m'avez point donné à boire ;
\VS{43}j'étais étranger et vous ne m'avez point recueilli ; j'ai été nu, et vous ne m'avez point vêtu ; j'ai été malade et en prison, et vous ne m'avez point visité.
\VS{44}Alors ils répondront aussi en disant : Seigneur, quand t'avons-nous vu avoir faim, ou avoir soif, ou être étranger, ou nu, ou malade, ou en prison, et ne t'avons-nous point secouru ?
\VS{45}Alors il leur répondra, en disant : Je vous le dis en vérité, toutes les fois que vous n'avez pas fait ces choses à l'un de ces plus petits, c'est à moi que vous ne les avez pas faites.
\VS{46}Et ceux-ci iront au châtiment éternel, mais les justes à la vie éternelle.
\Chap{26}
\TextTitle{Le complot\FTNTT{Mc. 14:1-2 ; Lu. 22:1-2}}
\VerseOne{}Et il arriva que quand Jésus eut achevé tous ces discours, il dit à ses disciples :
\VS{2}Vous savez que la fête de Pâque a lieu dans deux jours ; et le Fils de l'homme sera livré pour être crucifié.
\VS{3}Alors les principaux sacrificateurs, les scribes et les anciens du peuple, se réunirent dans la cour du souverain sacrificateur, appelé Caïphe ;
\VS{4}et tinrent conseil ensemble pour se saisir de Jésus par finesse, afin de le faire mourir.
\VS{5}Mais ils dirent : Que ce ne soit pas pendant la fête, de peur qu'il ne se fasse quelque tumulte parmi le peuple.
\TextTitle{Geste prophétique de Marie de Béthanie\FTNTT{Mc. 14:3-9 ; Jn. 12:1-8}}
\VS{6}Comme Jésus était à Béthanie, dans la maison de Simon le lépreux,
\VS{7}une femme s'approcha de lui tenant un vase d'albâtre, plein d'un parfum de grand prix, et pendant qu'il était à table, elle répandit le parfum sur sa tête.
\VS{8}Mais ses disciples voyant cela, en furent indignés et dirent : À quoi sert cette perte ?
\VS{9}Car ce parfum pouvait être vendu bien cher et être donné aux pauvres.
\VS{10}Mais Jésus connaissant cela, leur dit : Pourquoi faites-vous de la peine à cette femme ? Car elle a fait une bonne action à mon égard ;
\VS{11}car vous aurez toujours des pauvres avec vous ; mais vous ne m'aurez pas toujours.
\VS{12}En répandant ce parfum sur mon corps, elle l'a fait pour ma sépulture.
\VS{13}Je vous le dis en vérité, partout où cet Evangile sera prêché, dans le monde entier, on racontera aussi en mémoire de cette femme ce qu'elle a fait.
\TextTitle{La trahison de Judas\FTNTT{Mc. 14:10-11 ; Lu. 22:3-6}}
\VS{14}Alors l'un des douze, appelé Judas Iscariot, alla vers les principaux sacrificateurs,
\VS{15}et leur dit : Que voulez-vous me donner, et je vous le livrerai ? Et ils lui comptèrent trente pièces d'argent\FTNT{Za. 11:12-13.}.
\VS{16}Et dès lors, il cherchait une occasion favorable pour le livrer.
\TextTitle{La dernière Pâque\FTNTT{Mc. 14:12-21 ; Lu. 22:7-20 ; Jn. 13:1-12}}
\VS{17}Or le premier jour des pains sans levain, les disciples s'approchèrent de Jésus pour lui dire : Où veux-tu que nous te préparions le repas de la Pâque ?
\VS{18}Il répondit : Allez à la ville chez un tel et dites-lui : Le Maître dit : Mon temps est proche ; je ferai la Pâque chez toi avec mes disciples.
\VS{19}Les disciples firent comme Jésus leur avait ordonné et préparèrent la Pâque.
\VS{20}Et quand le soir fut venu, il se mit à table avec les douze.
\VS{21}Et comme ils mangeaient, il dit : Je vous le dis en vérité, l'un de vous me trahira.
\VS{22}Ils furent profondément attristés, et chacun d'eux commença à lui dire : Seigneur, est-ce moi ?
\VS{23}Mais il leur répondit : Celui qui a mis avec moi la main dans le plat pour tremper, c'est celui qui me trahira.
\VS{24}Le Fils de l'homme s'en va, selon qu'il est écrit de lui ; mais malheur à cet homme par qui le Fils de l'homme est trahi ! Mieux vaudrait pour cet homme qu'il ne soit pas né.
\VS{25}Judas qui le trahissait, prit la parole et dit : Maître, est-ce moi ? Jésus lui dit : Tu l'as dit.
\TextTitle{Le repas de la Pâque\FTNTT{Mc. 14:22-25 ; Lu. 22:17-20 ; Jn. 13:12-30 ; 1 Co. 11:23-26}}
\VS{26}Pendant qu'ils mangeaient, Jésus prit le pain, et après avoir rendu grâces à Dieu, il le rompit et le donna à ses disciples et leur dit : Prenez, mangez, ceci est mon corps.
\VS{27}Puis ayant pris la coupe, et béni Dieu, il la leur donna, en leur disant : buvez-en tous.
\VS{28}car ceci est mon sang, le sang de la Nouvelle Alliance, qui est répandu pour beaucoup, pour la rémission des péchés.
\VS{29}Or je vous dis : que depuis cette heure je ne boirai point de ce fruit de vigne, jusqu'au jour que je le boirai de nouveau avec vous, dans le Royaume de mon Père.
\TextTitle{Jésus informe Pierre de son triple reniement\FTNTT{Mc. 14:26-31 ; Lu. 22:31-34 ; Jn. 13:36-38}}
\VS{30}Quand ils eurent chanté le cantique\FTNT{Les cantiques : Du grec « humneo », chants d'hymnes pascals. Il s'agit plus précisément des psaumes 113 à 118 et du psaume 136, que les Juifs appellent le « grand Hallel ». Le Hallel consiste en six psaumes (113 à 118). Cet ensemble de textes est généralement entonné à haute voix par toute la communauté de prière lors de l'office religieux du matin, à l'issue de la « Amidah » (prière récitée debout), à l'occasion de la Pâque (le premier soir), de la Pentecôte et des Tabernacles, ainsi que pour Hanoucca et Rosh Hodesh. Voir Mc. 14:26.}, ils se rendirent à la Montagne des Oliviers.
\VS{31}Alors Jésus leur dit : Vous serez tous cette nuit scandalisés à cause de moi ; car il est écrit : Je frapperai le Berger, et les brebis du troupeau seront dispersées\FTNT{Za. 13:7.}.
\VS{32}Mais, après que je serai ressuscité, je vous précéderai en Galilée.
\VS{33}Pierre, prenant la parole, lui dit : Quand même tous seraient scandalisés à cause de toi, je ne le serai jamais.
\VS{34}Jésus lui dit : En vérité je te dis, qu'en cette même nuit, avant que le coq ait chanté, tu me renieras trois fois.
\VS{35}Pierre lui répondit : Même s'il me fallait mourir avec toi, je ne te renierai pas. Et tous les disciples dirent la même chose.
\TextTitle{Jésus dans le jardin de Gethsémané\FTNTT{Mc. 14:32-42 ; Lu. 22:39-46 ; Jn. 18:1}}
\VS{36}Alors Jésus alla avec eux dans un lieu appelé Gethsémané et il dit à ses disciples : Asseyez-vous ici, jusqu'à ce que j'aie prié dans le lieu où je vais.
\VS{37}Il prit avec lui Pierre et les deux fils de Zébédée, et il commença à être attristé et fort angoissé.
\VS{38}Alors il leur dit : Mon âme est de toutes parts saisie de tristesse jusqu'à la mort ; demeurez ici et veillez avec moi.
\TextTitle{Première prière de Jésus\FTNTT{Mc. 14:35-38 ; Lu. 22:41-42}}
\VS{39}Puis, ayant fait quelques pas en avant, il se prosterna le visage contre terre, priant et disant : Mon Père, s'il est possible, fais que cette coupe passe loin de moi ; toutefois non point comme je le veux, mais comme tu le veux.
\TextTitle{Jésus trouve les disciples endormis\FTNTT{Mc. 14:37-40 ; Lu. 22:45-46}}
\VS{40}Puis il vint vers ses disciples, qu'il trouva endormis, et il dit à Pierre : Vous n'avez pas pu veiller une heure avec moi ?
\VS{41}Veillez et priez, afin que vous ne tombiez pas en tentation : car l'esprit est prompt, mais la chair est faible.
\TextTitle{Deuxième prière\FTNTT{Mc. 14:39 ; Lu. 22:44}}
\VS{42}Il s'éloigna encore pour la seconde fois, et il pria, disant : Mon Père, s'il n'est pas possible que cette coupe s'éloigne sans que je la boive, que ta volonté soit faite.
\VS{43}Il revint ensuite et les trouva encore endormis ; car leurs yeux étaient appesantis.
\TextTitle{Troisième prière\FTNTT{Mc. 14:41}}
\VS{44}Et les ayant laissés, il s'en alla encore, et pria pour la troisième fois, disant les mêmes paroles.
\VS{45}Puis il alla vers ses disciples et leur dit : Dormez maintenant et reposez-vous ; voici, l'heure est proche, et le Fils de l'homme va être livré entre les mains des méchants.
\VS{46}Levez-vous, allons. Voici, celui qui me trahit s'approche.
\TextTitle{Jésus trahi et arrêté\FTNTT{Mc. 14:43-50 ; Lu. 22:47-53 ; Jn. 18:2-11}}
\VS{47}Comme il parlait encore, voici, Judas, l'un des douze, vint, et avec lui une grande foule, avec des épées et des bâtons, envoyée par les principaux sacrificateurs et par les anciens du peuple.
\VS{48}Celui qui le trahissait leur avait donné ce signe : Celui à qui je donnerai un baiser, c'est lui, saisissez-le.
\VS{49}Aussitôt, s'approchant de Jésus, il lui dit : Maître, je te salue ; et il le baisa.
\VS{50}Et Jésus lui dit : Mon ami, pour quel sujet es-tu ici ? Alors s'étant approchés, ils mirent les mains sur Jésus et le saisirent.
\VS{51}Et voici, l'un de ceux qui étaient avec Jésus, étendit la main et tira son épée ; il frappa le serviteur du souverain sacrificateur, et lui emporta l'oreille.
\VS{52}Alors Jésus lui dit : Remets ton épée à sa place ; car tous ceux qui prendront l'épée, périront par l'épée.
\VS{53}Crois-tu que je ne puisse pas maintenant prier mon Père, qui me donnerait à l'instant plus de douze légions d'anges ?
\VS{54}Mais comment donc s'accompliraient les Ecritures qui disent qu'il faut que cela arrive ainsi ?
\VS{55}En ce même instant Jésus dit à la foule : Vous êtes venus avec des épées et des bâtons, comme après un brigand, pour me prendre ; j'étais tous les jours assis parmi vous, enseignant dans le temple, et vous ne m'avez pas saisi.
\VS{56}Mais tout ceci est arrivé afin que les Ecritures des prophètes soient accomplies. Alors tous les disciples l'abandonnèrent et s'enfuirent.
\TextTitle{Jésus devant Caïphe et le sanhédrin\FTNTT{Mc. 14:53-65 ; Jn. 18:12-14, 19-24.}}
\VS{57}Ceux qui avaient saisi Jésus l'amenèrent chez Caïphe, le souverain sacrificateur, où les scribes et les anciens étaient assemblés.
\VS{58}Pierre le suivit de loin, jusqu'à la cour du souverain sacrificateur, y entra et s'assit avec les officiers pour voir comment cela finirait.
\VS{59}Les principaux sacrificateurs, les anciens et tout le sanhédrin cherchaient des faux témoignages contre Jésus pour le faire mourir.
\VS{60}Mais ils n'en trouvèrent point, et bien que plusieurs faux témoins se soient présentés, ils n'en trouvèrent point de propres ; mais à la fin, deux faux témoins s'approchèrent
\VS{61}et dirent : Celui-ci a dit : Je puis détruire le temple de Dieu et le rebâtir en trois jours.
\VS{62}Alors le souverain sacrificateur se leva et lui dit : Ne réponds-tu rien ? Qu'est-ce que ces hommes déposent contre toi ?
\VS{63}Jésus garda le silence. Et le souverain sacrificateur prenant la parole, lui dit : Je te somme par le Dieu vivant, de nous dire si tu es le Christ, le Fils de Dieu.
\VS{64}Jésus lui dit : Tu l'as dit. De plus, je vous dis que désormais vous verrez le Fils de l'homme assis à la droite de la puissance de Dieu et venant sur les nuées du ciel.
\VS{65}Alors le souverain sacrificateur déchira ses vêtements, en disant : Il a blasphémé ! Qu'avons-nous encore besoin de témoins ? Voici, vous avez entendu maintenant son blasphème. Que vous en semble ?
\VS{66}Ils répondirent : Il est digne de mort.
\VS{67}Alors ils lui crachèrent au visage, et lui donnèrent des coups de poing et des soufflets, et les autres le frappaient avec leurs bâtons ;
\VS{68}en disant : Christ, prophétise-nous qui est celui qui t'a frappé.
\TextTitle{Le triple reniement de Pierre\FTNTT{Mc. 14:66-72 ; Lu. 22:55-62 ; Jn. 18:15-18, 25-27.}}
\VS{69}Or Pierre était assis dehors dans la cour. Une servante s'approcha de lui et lui dit : Toi aussi, tu étais aussi avec Jésus le Galiléen.
\VS{70}Mais il le nia devant tous, en disant : Je ne sais pas ce que tu dis.
\VS{71}Et comme il était sorti dans le vestibule, une autre servante le vit et elle dit à ceux qui étaient là : Celui-ci aussi était avec Jésus de Nazareth.
\VS{72}Et il le nia encore avec serment, disant : Je ne connais pas cet homme.
\VS{73}Peu après, ceux qui se trouvaient là s'approchèrent et dirent à Pierre : Certainement tu es aussi de ces gens-là, car ton langage te fait connaître.
\VS{74}Alors il commença à faire des imprécations et à jurer, en disant : Je ne connais pas cet homme. Et aussitôt le coq chanta.
\VS{75}Et Pierre se souvint de la parole de Jésus, qui lui avait dit : Avant que le coq chante, tu me renieras trois fois. Et étant sorti dehors, il pleura amèrement.
\Chap{27}
\TextTitle{Jésus devant le gouverneur Pilate ; suicide de Judas\FTNTT{Ac. 1:16-19.}}
\VerseOne{}Puis quand le matin fut venu, tous les principaux sacrificateurs et les anciens du peuple tinrent conseil contre Jésus pour le faire mourir.
\VS{2}Après l'avoir lié, ils l'amenèrent et le livrèrent à Ponce Pilate, qui était le gouverneur.
\VS{3}Alors Judas qui l'avait trahi, voyant qu'il était condamné, se repentit et rapporta les trente pièces d'argent aux principaux sacrificateurs et aux anciens,
\VS{4}en leur disant : J'ai péché en trahissant le sang innocent ; mais ils lui dirent : Que nous importe ? Cela te regarde.
\VS{5}Et après avoir jeté les pièces d'argent dans le temple, il se retira et alla se pendre.
\VS{6}Mais les principaux sacrificateurs prirent les pièces d'argent et dirent : Il n'est pas permis de les mettre dans le trésor ; car c'est le prix du sang.
\VS{7}Et, après en avoir délibéré, ils achetèrent avec cet argent le champ d'un potier pour la sépulture des étrangers.
\VS{8}C'est pourquoi ce champ-là a été appelé jusqu'à aujourd'hui, le champ du sang.
\VS{9}Alors s'accomplit ce qui avait été annoncé par Jérémie le prophète, disant : Ils ont pris les trente pièces d'argent, le prix de celui qui a été estimé, qu'on a estimé de la part des enfants d'Israël ;
\VS{10}et ils les ont données pour acheter le champ d'un potier, selon ce que le Seigneur m'avait ordonné\FTNT{Ce verset se réfère certainement à Za. 11:12-13, avec une allusion à Jé. 18:1-4.}.
\VS{11}Jésus comparut devant le gouverneur. Le gouverneur l'interrogea : Es-tu le Roi des Juifs ? Jésus lui répondit : Tu le dis.
\VS{12}Mais il ne répondit rien aux accusations des principaux sacrificateurs et des anciens.
\VS{13}Alors Pilate lui dit : N'entends-tu pas de combien de choses ils t'accusent ?
\VS{14}Mais il ne lui donna de réponse sur aucune parole, ce qui étonna beaucoup le gouverneur.
\TextTitle{Jésus ou Barabbas ?\FTNTT{Mc. 15:6-15 ; Lu. 23:17-25 ; Jn. 18:39-40}}
\VS{15}Or le gouverneur avait coutume de relâcher un prisonnier à chaque fête, celui que demandait la foule.
\VS{16}Et il y avait alors un prisonnier fameux, nommé Barabbas.
\VS{17}Comme ils étaient assemblés, Pilate leur dit : Lequel voulez-vous que je vous relâche ? Barabbas ou Jésus qu'on appelle Christ ?
\VS{18}Car il savait bien qu'ils l'avaient livré par envie.
\VS{19}Et pendant qu'il siégeait au tribunal, sa femme envoya lui dire : Ne te mêle point de l'affaire de ce juste, car j'ai beaucoup souffert aujourd'hui en songe à cause de lui.
\VS{20}Les principaux sacrificateurs et les anciens persuadèrent la multitude du peuple de demander Barabbas et de faire périr Jésus.
\VS{21}Et le gouverneur prenant la parole leur dit : Lequel des deux voulez-vous que je vous relâche ? Ils dirent : Barabbas.
\VS{22}Pilate leur dit : Que ferai-je donc de Jésus qu'on appelle Christ ? Ils lui dirent tous : Qu'il soit crucifié !
\VS{23}Et le gouverneur leur dit : Mais quel mal a-t-il fait ? Et ils crièrent encore plus fort, en disant : Qu'il soit crucifié !
\VS{24}Alors Pilate voyant qu'il ne gagnait rien, mais que le tumulte s'augmentait, prit de l'eau et lava ses mains devant le peuple, en disant : Je suis innocent du sang de ce juste. Cela vous regarde.
\VS{25}Et tout le peuple répondit : Que son sang retombe sur nous et sur nos enfants !
\VS{26}Alors il leur relâcha Barabbas ; et après avoir fait fouetter Jésus, il le leur livra pour être crucifié.
\TextTitle{Le Roi couronné d'épines\FTNTT{Mc. 15:16-23 ; Lu. 23:26-32 ; Jn. 19:16-17}}
\VS{27}Les soldats du gouverneur amenèrent Jésus dans le prétoire et assemblèrent devant lui toute la cohorte.
\VS{28}Et après l'avoir dépouillé, ils le revêtirent d'un manteau d'écarlate.
\VS{29}Puis, ayant fait une couronne d'épines entrelacées, ils la mirent sur sa tête et ils lui mirent un roseau dans sa main droite ; puis s'agenouillant devant lui, ils se moquaient de lui, en disant : Nous te saluons, Roi des Juifs !
\VS{30}Et ils crachaient contre lui, prenaient le roseau et frappaient sur sa tête.
\VS{31}Après s'être ainsi moqués de lui, ils lui ôtèrent le manteau, et lui remirent ses vêtements, et l'amenèrent pour le crucifier.
\VS{32}Comme ils sortaient, ils rencontrèrent un homme de Cyrène, appelé Simon et ils le forcèrent à porter la croix de Jésus.
\TextTitle{La crucifixion de Jésus\FTNTT{Mc. 15:24-32 ; Lu. 23:33-43 ; Jn. 19:17-24}}
\VS{33}Et étant arrivés au lieu appelé Golgotha, c'est-à-dire le lieu du crâne,
\VS{34}ils lui donnèrent à boire du vinaigre mêlé avec du fiel\FTNT{Le vinaigre mêlé au fiel (Ps. 69:22) : Ce breuvage, appelé « posca », était un vin amer composé qui se transformait en vinaigre à cause des mauvaises conditions de conservation. Allongé avec de l'eau et parfois adoucie avec de l'œuf, cette boisson bon marché et très rafraîchissante était consommée principalement par les légionnaires et les esclaves. Connue pour ses vertus antiseptiques, les soldats de l'Antiquité avaient coutume d'y ajouter des drogues comme la myrrhe et le fiel (opium) pour atténuer les souffrances. En refusant de le boire, le Seigneur Jésus-Christ a réellement pris sur lui la plénitude du châtiment que nous méritons à cause de nos péchés.} ; mais quand il l'eut goûté, il ne voulut pas boire.
\VS{35}Et après l'avoir crucifié, ils partagèrent ses vêtements, en tirant au sort, afin que s'accomplît ce qui avait été annoncé par le prophète : Ils se sont partagés mes vêtements, et ont jeté ma tunique au sort\FTNT{Ps. 22:19.}.
\VS{36}Puis s'étant assis, ils le gardaient là.
\VS{37}Ils mirent aussi au-dessus de sa tête un écriteau, où la cause de sa condamnation était marquée en ces mots : CELUI-CI EST JESUS, LE ROI DES JUIFS.
\VS{38}Avec lui furent crucifiés deux brigands, l'un à sa droite et l'autre à sa gauche.
\VS{39}Et Ceux qui passaient par là, l'injuriaient et secouaient la tête
\VS{40}en disant : Toi qui détruis le temple et qui le rebâtis en trois jours, sauve-toi toi-même ! Si tu es le Fils de Dieu, descends de la croix !
\VS{41}Pareillement aussi, les principaux sacrificateurs avec les scribes et les anciens, se moquant, disaient :
\VS{42}Il a sauvé les autres et il ne peut pas se sauver lui-même ! S'il est le Roi d'Israël, qu'il descende maintenant de la croix et nous croirons en lui.
\VS{43}Il se confie en Dieu ; mais si Dieu l'aime, qu'il le délivre maintenant, car il a dit : Je suis le Fils de Dieu.
\VS{44}Les brigands aussi qui étaient crucifiés avec lui, lui reprochaient la même chose.
\TextTitle{Jésus accomplit la loi par sa mort\FTNTT{Mc. 15:33-41 ; Lu. 23:44-49 ; Jn. 19:30-37 ; Hé. 9:3-8 ; 10:19-20}}
\VS{45}Depuis la sixième heure jusqu'à la neuvième, il y eut des ténèbres sur toute la terre.
\VS{46}Et vers la neuvième heure, Jésus s'écria d'une voix forte : Eli, Eli, lama sabachthani ? C'est-à-dire : Mon Dieu ! Mon Dieu ! Pourquoi m'as-tu abandonné ?
\VS{47}Quelques-uns de ceux qui étaient là présents, ayant entendu cela, disaient : Il appelle Elie.
\VS{48}Et aussitôt l'un d'entre eux courut prendre une éponge, qu'il remplit de vinaigre, et l'ayant fixée au bout d'un roseau, lui donna à boire.
\VS{49}Mais les autres disaient : Laisse, voyons si Elie viendra le sauver.
\VS{50}Alors Jésus, poussa de nouveau un grand cri, et rendit l'esprit.
\VS{51}Et voici, le voile du temple se déchira en deux, depuis le haut jusqu'en bas\FTNT{C'est ici que s'achève la Première Alliance. Cette dernière était relative à la loi de Moïse, c'est-à-dire aux ordonnances liées au culte, qui reposait sur le sacerdoce lévitique et les sacrifices d'animaux, et au sanctuaire terrestre, à savoir le temple de Jérusalem (Hé. 9:1). Le Seigneur ayant offert une fois pour toutes le sacrifice parfait, les exigences de la justice divine ont été pleinement satisfaites (Hé. 9:11-12 ; 25-26). Désormais, la Première Alliance n'a plus de raison d'être et peut donc disparaître (Hé. 8:13). Non seulement la déchirure du voile séparant le lieu saint du Saint des saints atteste la fin de la Première Alliance, mais invite aussi tout homme à s'approcher de Dieu en esprit, sans intermédiaires (Lévites, sacrificateurs, pasteurs, prophètes…) ni nécessité de se rendre dans un temple (Jn. 4:23). La Nouvelle Alliance est aussi un testament puisque Jésus-Christ, notre légataire, est passé par la mort (Hé. 9:16-18). Voir aussi commentaire en Ex. 19:5.} ; et la terre trembla, et les pierres se fendirent.
\TextTitle{Le voile déchiré : Fin de la loi mosaïque ou de la Première Alliance}
\VS{52}Et les sépulcres s'ouvrirent et plusieurs corps des saints qui étaient morts ressuscitèrent.
\VS{53}Et étant sortis des sépulcres après la résurrection de Jésus, ils entrèrent dans la ville sainte et se montrèrent à plusieurs.
\VS{54}Le centenier et ceux qui étaient avec lui pour garder Jésus, ayant vu le tremblement de terre et tout ce qui venait d'arriver, furent saisis d'une grande frayeur et dirent : Certainement cet homme était le Fils de Dieu.
\VS{55}Il y avait là aussi plusieurs femmes qui regardaient de loin, et qui avaient suivi Jésus depuis la Galilée, pour le servir.
\VS{56}Entre lesquelles étaient Marie de Magdala, Marie mère de Jacques et de Joseph, et la mère des fils de Zébédée.
\TextTitle{Jésus enseveli\FTNTT{Mc. 15:42-47 ; Lu. 23:50-56 ; Jn. 19:38-42}}
\VS{57}Le soir étant venu, un homme riche d'Arimathée, appelé Joseph, qui était aussi disciple de Jésus,
\VS{58}se rendit vers Pilate et demanda le corps de Jésus. En même temps Pilate ordonna que le corps soit rendu.
\VS{59}Joseph prit le corps et l'enveloppa d'un linceul pur ;
\VS{60}et le mit dans un sépulcre neuf, qu'il s'était fait tailler dans le roc. Puis il roula une grande pierre à l'entrée du sépulcre et il s'en alla.
\VS{61}Marie de Magdala et l'autre Marie étaient là, assises vis-à-vis du sépulcre.
\TextTitle{Le sépulcre scellé et gardé}
\VS{62}Le lendemain, qui était le jour de la préparation du sabbat, les principaux sacrificateurs et les pharisiens allèrent ensemble auprès de Pilate,
\VS{63}et lui dirent : Seigneur ! Nous nous souvenons que ce séducteur disait, quand il était encore en vie : Après trois jours je ressusciterai.
\VS{64}Ordonne donc que le sépulcre soit gardé sûrement jusqu'au troisième jour ; de peur que ses disciples ne viennent de nuit, et ne dérobent son corps, et qu'ils ne disent au peuple : Il est ressuscité des morts. Cette dernière imposture serait pire que la première.
\VS{65}Pilate leur dit : Vous avez une garde ; allez et faites-le garder comme vous l'entendez.
\VS{66}Ils s'en allèrent donc, et s'assurèrent du sépulcre, au moyen d'une garde, après avoir scellé la pierre.
\Chap{28}
\TextTitle{Résurrection et apparition de Jésus-Christ\FTNTT{Mc. 16:1-14 ; Lu. 24:1-49 ; Jn. 20:1-23}}
\VerseOne{}Après le sabbat, à l'aube du premier jour de la semaine, Marie de Magdala et l'autre Marie allèrent voir le sépulcre.
\VS{2}Et voici, il eut un grand tremblement de terre ; car un ange du Seigneur descendit du ciel, vint rouler la pierre à côté de l'entrée du sépulcre et s'assit dessus.
\VS{3}Son visage était comme un éclair, et son vêtement blanc comme de la neige.
\VS{4}Les gardes furent tellement saisis de frayeur, qu'ils devinrent comme morts.
\VS{5}Mais l'ange prit la parole et dit aux femmes : Pour vous, ne craignez pas ; car je sais que vous cherchez Jésus, qui a été crucifié.
\VS{6}Il n'est point ici car il est ressuscité comme il l'avait dit. Venez et voyez le lieu où le Seigneur était couché,
\VS{7}et allez-vous-en promptement, et dites à ses disciples qu'il est ressuscité des morts. Et voici, il vous précède en Galilée ; c'est là que vous le verrez. Voici, je vous l'ai dit.
\VS{8}Alors elles sortirent promptement du sépulcre avec crainte et grande joie ; et coururent l'annoncer à ses disciples.
\VS{9}Mais comme elles allaient pour l'annoncer à ses disciples, voici, Jésus se présenta devant elles et leur dit : Je vous salue. Et elles s'approchèrent, embrassèrent ses pieds et l'adorèrent.
\VS{10}Alors Jésus leur dit : Ne craignez point. Allez et dites à mes frères d'aller en Galilée, c'est là qu'ils me verront.
\TextTitle{Les soldats soudoyés par les sacrificateurs}
\VS{11} Or quand elles furent parties, voici, quelques-uns de la garde vinrent dans la ville et ils rapportèrent aux principaux sacrificateurs toutes les choses qui étaient arrivées.
\VS{12}Sur quoi les sacrificateurs s'assemblèrent avec les anciens, et après avoir tenu conseil, donnèrent une forte somme d'argent aux soldats,
\VS{13}en leur disant : Dites : Ses disciples sont venus de nuit le dérober, pendant que nous dormions.
\VS{14}Et si le gouverneur l'apprend, nous l'apaiserons et nous vous tirerons de peine.
\VS{15}Les soldats prirent l'argent et suivirent les instructions qui leur furent données. Et ce bruit s'est répandu parmi les juifs, jusqu'à aujourd'hui.
\TextTitle{Mission des apôtres\FTNTT{Mc. 16:15-18 ; Lu. 24:46-48 ; Jn. 17:18 ; 20:21 ; Ac. 1:8 ; 1 Co. 15:6}}
\VS{16}Mais les onze disciples allèrent en Galilée, sur la montagne, où Jésus leur avait ordonné de se rendre.
\VS{17}Quand ils le virent, ils l'adorèrent, mais quelques-uns doutèrent.
\VS{18}Jésus s'étant approché, leur parla, en disant : Tout puissance m'a été donnée dans le ciel et sur la terre.
\VS{19}Allez donc et enseignez toutes les nations, les baptisant au Nom du Père, du Fils et du Saint-Esprit ;
\VS{20}les enseignant à garder toutes les choses que je vous ai commandées ; et voici, je suis avec vous toujours jusqu'à la fin du monde. Amen.
\PPE{}
\end{multicols}

%\clearpage\ShortTitle{Marc}\BookTitle{Marc}\BFont
\noindent\hrulefill
\textit{
\bigskip
{\centering{}
\\Signifie : Qui brille, luisant
\\Thème : Jésus le serviteur
\\Auteur : Marc
\\Date de rédaction : Env. 68 apr. J.-C.\\}
}
%\bigskip
\textit{
\\Originaire de Jérusalem, Marc, aussi appelé Jean, fut l’auteur de l’évangile du même nom. Cousin de Barnabas et collaborateur de Paul, ce dernier l’éconduit lors d’un voyage car Marc l’avait abandonné lors d’une précédente mission. Ce fut d’ailleurs la cause de la séparation entre Barnabas et Paul.  Par la suite, il renoua le contact avec Paul et devint un de ses fidèles compagnons de ministère. Lié à l’apôtre Pierre tel un fils, ce fut probablement sous son autorité qu’il écrivit. En effet, l’évangile de Marc expose le témoignage de Pierre sur Christ.
\bigskip
\\Adressé aux gentils, cet évangile contient peu de références à l’ancienne alliance ; on y découvre Jésus l’inlassable serviteur de Dieu et des hommes. Marc y exposa la richesse de ses bonnes œuvres, son incomparable dévouement et révéla les sentiments intimes du maître. Même si Marc présenta principalement Jésus en tant que serviteur, son récit des miracles met en exergue toute la puissance du Christ.\bigskip
}
\par\nobreak\noindent\hrulefill
\begin{multicols}{2}
\TextTitle{[Ministère de Jean-Baptiste]
\\(Mt. 3:1-12 ; Lu. 3:1-20 ; Jn. 1:6-8,15-37)}
\Chap{1}
\VerseOne{}Commencement de l'Evangile de Jésus-Christ, Fils de Dieu ;
\VS{2}selon qu'il est écrit dans les prophètes : Voici, j'envoie mon messager devant ta face, lequel préparera ta voie devant toi.
\VS{3}C’est la voix de celui qui crie dans le désert : Préparez le chemin du Seigneur, aplanissez ses sentiers{\FTNT{Es. 40:3 ; Mal. 3:1.}}.
\VS{4}Jean baptisait dans le désert, et prêchait le baptême de repentance, pour obtenir la rémission des péchés.
\VS{5}Et tout le pays de Judée, et les habitants de Jérusalem allaient vers lui, et confessant leurs péchés, ils se faisaient tous baptiser par lui dans le fleuve du Jourdain.
\VS{6}Jean était vêtu de poils de chameau, il avait une ceinture de cuir autour de ses reins, et mangeait des sauterelles et du miel sauvage.
\VS{7}Et il prêchait, en disant : Il vient après moi, celui qui est plus puissant que moi, et je ne suis pas digne de délier en me baissant la courroie de ses souliers.
\VS{8}Moi, je vous ai baptisés d'eau ; mais lui, il vous baptisera du Saint-Esprit.
\TextTitle{[Baptême de Jésus-Christ]
\\(Mt. 3:13-17 ; Lu. 3:21-22 ; Jn. 1:31-34)}
\VS{9}En ce temps-là, Jésus vint de Nazareth, ville de Galilée, et il fut baptisé par Jean dans le Jourdain.
\VS{10}Au moment où il sortait de l'eau, Jean vit les cieux s’ouvrir, et le Saint-Esprit descendre sur lui comme une colombe.
\VS{11}Et une voix fit entendre des cieux ces paroles : Tu es mon Fils bien-aimé, en qui j'ai mis toute mon affection.
\TextTitle{[La tentation]
\\(Mt. 4:1-11 ; Lu. 4:1-13)}
\VS{12}Aussitôt l'Esprit le poussa à se rendre dans un désert,
\VS{13}où il passa quarante jours, tenté par Satan. Il était avec les bêtes sauvages, et les anges le servaient.
\TextTitle{[Jésus en Galilée]
\\(Mt. 4:12-17 ; Lu. 4:14-15)}
\VS{14}Après que Jean eut été mis en prison, Jésus alla dans la Galilée, prêchant l'Evangile du Royaume de Dieu.
\VS{15}Il disait : Le temps est accompli, et le Royaume de Dieu est proche. Repentez-vous, et croyez à 1'Evangile.
\TextTitle{[Appel de Simon (Pierre), André, Jacques et Jean]
\\(Lu. 5:1-11 ; Jn. 1:35-51)}
\VS{16}Comme il marchait près de la mer de Galilée, il vit Simon et André son frère, qui jetaient leurs filets dans la mer, car ils étaient pêcheurs.
\VS{17}Jésus leur dit : Suivez-moi, et je vous ferai pêcheurs d'hommes.
\VS{18}Aussitôt ils laissèrent leurs filets et ils le suivirent.
\VS{19}Etant allé un peu plus loin, il vit Jacques fils de Zébédée, et Jean son frère, qui raccommodaient leurs filets dans la barque.
\VS{20}Aussitôt, il les appela ; et laissant leur père Zébédée dans la barque avec les ouvriers, ils le suivirent.
\TextTitle{[Jésus chasse un démon dans la synagogue]
\\(Lu. 4:31-37)}
\VS{21}Ils entrèrent dans Capernaüm. Et le jour du sabbat, Jésus entra d’abord dans la synagogue, et il enseigna.
\VS{22}Ils étaient étonnés de sa doctrine ; car il les enseignait comme ayant autorité, et non pas comme les scribes.
\VS{23}Il se trouvait dans leur synagogue un homme qui avait un esprit impur, et qui s'écria,
\VS{24}en disant : Ha ! Qu’y a-t-il entre toi et nous, Jésus de Nazareth ? Es-tu venu pour nous perdre ? Je sais qui tu es : Tu es le Saint de Dieu.
\VS{25}Mais Jésus le menaça, disant : Tais-toi, et sors de cet homme.
\VS{26}Alors l'esprit impur sortit de cet homme, en l’agitant avec violence, et en poussant un grand cri.
\VS{27}Tous furent étonnés, de sorte qu'ils se demandaient les uns aux autres, et disaient : Qu'est-ce que ceci ? Quelle est cette nouvelle doctrine ? Il commande avec autorité même aux esprits impurs, et ils lui obéissent.
\VS{28}Et sa renommée se répandit aussitôt dans tout le pays des environs de la Galilée.
\TextTitle{[Jésus guérit la belle-mère de Pierre]
\\(Mt. 8:14-15 ; Lu. 4:38-39)}
\VS{29}En sortant de la synagogue, ils se rendirent avec Jacques et Jean à la maison de Simon et d'André.
\VS{30}La belle-mère de Simon était couchée, ayant la fièvre ; et aussitôt on parla d’elle à Jésus.
\VS{31}S’étant approché, il la fit lever en la prenant par la main ; et à l'instant la fièvre la quitta ; et elle les servit.
\TextTitle{[Jésus guérit les malades et chasse des démons ; prédications en Galilée]
\\(Mt. 8:16-17 ; Lu. 4:40-44)}
\VS{32}Le soir étant venu, comme le soleil se couchait, on lui amena tous les malades, et les démoniaques.
\VS{33}Et toute la ville était assemblée devant sa porte.
\VS{34}Il guérit beaucoup de malades qui avaient différentes maladies et chassa beaucoup de démons, et il ne permettait pas aux démons de parler, parce qu’ils le connaissaient.
\VS{35}Vers le matin, pendant qu’il faisait encore très sombre, il se leva, et sortit pour aller dans un lieu désert, où il pria.
\VS{36}Simon et ceux qui étaient avec lui se mirent à sa recherche,
\VS{37}et quand ils l’eurent trouvé, ils lui dirent : Tous te cherchent.
\VS{38}Et il leur dit : Allons aux bourgades voisines, afin que j'y prêche aussi ; car c’est pour cela que je suis venu.
\VS{39}Il prêchait donc dans leurs synagogues, par toute la Galilée, et chassait les démons.
\TextTitle{[Jésus guérit un lépreux]
\\(Mt. 8:2-4 ; Lu. 5:12-14)}
\VS{40}Un lépreux vint à lui, le priant et se mettant à genoux devant lui, et lui dit : Si tu veux, tu peux me rendre pur.
\VS{41}Jésus, ému de compassion, étendit sa main et le toucha, en lui disant : Je le veux, sois pur.
\VS{42}La lèpre quitta aussitôt cet homme, et il fut purifié.
\VS{43}Jésus le renvoya sur-le-champ, avec de sévères recommandations,
\VS{44}et lui dit : Garde-toi de ne rien dire à personne ; mais va te montrer au sacrificateur, et présente pour ta purification les choses que Moïse a commandées, pour leur servir de témoignage{\FTNT{Loi sur la purification de la lèpre~: Lé. 14:1-32. Avant sa mort et sa résurrection, Jésus-Christ observait la loi de Moïse (Mt. 23:1-2).}}.
\VS{45}Mais cet homme, s’en étant allé, commença à publier ouvertement la chose et à divulguer ce qui s'était passé ; de sorte que Jésus ne pouvait plus entrer publiquement dans la ville, mais il se tenait dehors, dans des lieux déserts, et l’on venait à lui de toutes parts.
\TextTitle{[Jésus guérit un paralytique]
\\(Mt. 9:2-8 ; Lu. 5:18-26)}
\Chap{2}
\VerseOne{}Quelques jours après, Jésus revint à Capernaüm. On apprit qu'il était à la maison,
\VS{2}et aussitôt il s’assembla un si grand nombre de personnes, que l'espace même devant la porte ne pouvait plus les contenir. Il leur annonçait la parole.
\VS{3}Et quelques-uns vinrent à lui, amenant un paralytique qui était porté par quatre personnes.
\VS{4}Comme ils ne pouvaient pas s’approcher de lui à cause de la foule, ils découvrirent le toit du lieu où il était, et l'ayant percé, ils descendirent le lit dans lequel le paralytique était couché.
\VS{5}Jésus, voyant leur foi, dit au paralytique : Mon enfant, tes péchés te sont pardonnés.
\VS{6}Et quelques scribes qui étaient assis là, raisonnaient ainsi en eux-mêmes :
\VS{7}Comment cet homme parle-t-il ainsi ? Il blasphème. Qui peut pardonner les péchés, si ce n’est Dieu seul ?
\VS{8}Jésus, ayant aussitôt connu par son esprit qu'ils raisonnaient ainsi en eux-mêmes, leur dit : Pourquoi avez-vous de telles pensées dans vos cœurs ?
\VS{9}Lequel est le plus aisé de dire au paralytique : Tes péchés te sont pardonnés, ou de dire : Lève-toi, prends ton lit, et marche ?
\VS{10}Mais afin que vous sachiez que le Fils de l'homme a le pouvoir sur la terre de pardonner les péchés, il dit au paralytique :
\VS{11}Je te dis : Lève-toi, prends ton lit, et va dans ta maison.
\VS{12}Et il se leva aussitôt, et ayant pris son lit, il sortit en présence de tous ; de sorte qu'ils furent tous étonnés, et ils glorifièrent Dieu, en disant : Nous n’avons jamais rien vu de pareil.
\TextTitle{[Appel de Matthieu]
\\(Mt. 9:9 ; Lu. 5:27-28)}
\VS{13}Jésus sortit de nouveau du côté de la mer, toute la foule venait à lui, et il les enseignait.
\VS{14}En passant, il vit Lévi, fils d'Alphée, assis au bureau des péages, et il lui dit : Suis-moi. Et Lévi s'étant levé, le suivit.
\TextTitle{[Jésus appelle des pêcheurs à la repentance, non des justes]
\\(Mt. 9:10-15 ; Lu. 5:29-35)}
\VS{15}Comme Jésus était à table dans la maison de Lévi, plusieurs publicains et des gens de mauvaise vie se mirent aussi à table avec lui et avec ses disciples ; car ils étaient nombreux, et l'avaient suivi.
\VS{16}Mais les scribes et les pharisiens voyant qu'il mangeait avec les publicains et les gens de mauvaise vie, disaient à ses disciples : Pourquoi mange-t-il et boit-il avec les publicains et les gens de mauvaise vie ?
\VS{17}Jésus ayant entendu cela, leur dit : Ce ne sont pas ceux qui se portent bien qui ont besoin de médecin, mais les malades. Je ne suis pas venu appeler à la repentance les justes, mais les pécheurs.
\TextTitle{[Les pharisiens et les disciples de Jean interrogent Jésus sur le jeûne]}
\VS{18}Les disciples de Jean et ceux des pharisiens jeûnaient ; ils vinrent à Jésus et lui dirent : Pourquoi les disciples de Jean, et ceux des pharisiens, jeûnent-ils, tandis que tes disciples ne jeûnent point ?
\VS{19}Jésus leur répondit : Les amis de l'Epoux peuvent-ils jeûner pendant que l'Epoux est avec eux ? Aussi longtemps qu’ils ont avec eux l'Epoux, ils ne peuvent jeûner.
\VS{20}Mais les jours viendront où l'Epoux leur sera ôté, alors ils jeûneront en ce jour-là.
\TextTitle{[Parabole du drap neuf et des outres neuves]
\\(Mt. 9:16-17 ; Lu. 5:36-39)}
\VS{21}Personne ne coud une pièce de drap neuf à un vieil habit ; autrement, la pièce du drap neuf emporterait une partie du vieux, et la déchirure serait pire.
\VS{22}Et personne ne met du vin nouveau dans de vieilles outres ; autrement, le vin nouveau fait rompre les outres, et le vin se répand, et les outres sont perdues ; mais le vin nouveau doit être mis dans des outres neuves.
\TextTitle{[Jésus, le Maître du sabbat]
\\(Mt. 12:1-8 ; Lu. 6:1-5)}
\VS{23}Il arriva, un jour de sabbat, que Jésus traversa des champs de blé. Ses disciples en marchant se mirent à arracher des épis.
\VS{24}Les pharisiens lui dirent : Regarde, pourquoi font-ils ce qui n'est pas permis les jours de sabbat ?
\VS{25}Mais il leur dit : N'avez-vous jamais lu ce que fit David quand il fut dans la nécessité, et qu'il eut faim, lui et ceux qui étaient avec lui ?
\VS{26}Comment il entra dans la maison de Dieu, au temps du souverain sacrificateur Abiathar, et mangea les pains de proposition{\FTNT{1 S. 21:1-7.}} , qu’il n'est permis qu'aux sacrificateurs de manger ; et il en donna même à ceux qui étaient avec lui!
\VS{27}Puis il leur dit : Le sabbat a été fait pour l'homme, et non pas l'homme pour le sabbat ;
\VS{28}de sorte que le Fils de l'homme est Maître même du sabbat.
\TextTitle{[Jésus-Christ guérit un homme à la main sèche le jour du sabbat]
\\(Mt. 12:9-13 ; Lu. 6:6-11)}
\Chap{3}
\VerseOne{}Jésus entra de nouveau dans la synagogue, et il y avait là un homme qui avait une main sèche.
\VS{2}Ils l'observaient, pour voir s'il le guérirait le jour du sabbat, afin de l'accuser.
\VS{3}Et Jésus dit à l'homme qui avait la main sèche : Lève-toi, et tiens-toi là au milieu.
\VS{4}Puis il leur dit : Est-il permis de faire du bien les jours de sabbat, ou de faire du mal, de sauver une personne, ou de la tuer ? Mais ils gardèrent le silence.
\VS{5}Alors, les regardant tous avec indignation, et étant affligé de l'endurcissement de leur cœur, il dit à cet homme : Etends ta main. Il l'étendit, et sa main fut rendue saine comme l'autre.
\TextTitle{[Nombreuses guérisons de Jésus]
\\(Mt. 12:15-16 ; Lu. 6:17-19}
\VS{6}Alors les pharisiens sortirent, et aussitôt, ils se consultèrent avec les hérodiens, sur les moyens de le faire périr.
\VS{7}Mais Jésus se retira vers la mer avec ses disciples. Une grande multitude le suivit de la Galilée,
\VS{8}de Judée, de Jérusalem, de l’Idumée, d’au-delà du Jourdain, et des environs de Tyr et de Sidon, une grande multitude, ayant entendu les grandes choses qu'il faisait, vint vers lui en grand nombre.
\VS{9}Et il dit à ses disciples de tenir toujours à sa disposition une petite barque, afin de ne pas être pressé par la foule.
\VS{10}Car, comme il guérissait beaucoup de gens, tous ceux qui avaient des maladies se jetaient sur lui pour le toucher.
\VS{11}Et les esprits impurs, quand ils le voyaient, se prosternaient devant lui, et s'écriaient en disant : Tu es le Fils de Dieu.
\VS{12}Mais il leur défendait avec de grandes menaces de le faire connaître.
\TextTitle{[L'appel des douze apôtres]
\\(Mt. 10:1-4 ; Lu. 6:13-16)}
\VS{13}Puis il monta sur une montagne, appela ceux qu'il voulut, et ils vinrent auprès de lui.
\VS{14}Il en établit douze pour être avec lui,
\VS{15}et pour les envoyer prêcher, avec la puissance de guérir les maladies, et de chasser les démons.
\VS{16}Voici les douze qu’il établit : Simon qu'il nomma Pierre ;
\VS{17}Jacques fils de Zébédée, et Jean, frère de Jacques, auxquels il donna le nom de Boanergès, ce qui veut dire fils de tonnerre.
\VS{18}André ; Philippe ; Barthélemy ; Matthieu ; Thomas ; Jacques, fils d'Alphée ; Thaddée ; Simon le Cananite ;
\VS{19}et Judas Iscariot, celui qui livra Jésus.
\VS{20}Ils se rendirent à la maison, et une grande multitude s’assembla de nouveau, en sorte qu’ils ne pouvaient même pas prendre leur repas.
\VS{21}Quand les parents de Jésus apprirent cela, ils sortirent pour se saisir de lui. Car ils disaient : Il est hors de sens.
\TextTitle{[Le blasphème contre le Saint-Esprit]
\\(Mt. 12:24-32 ; Lu. 11:15-23)}
\VS{22}Et les scribes, qui étaient descendus de Jérusalem, disaient : Il est possédé par Béelzébul ; c’est par le prince des démons qu’il chasse les démons.
\VS{23}Mais Jésus les appela, et leur dit sous forme de paraboles : Comment Satan peut-il chasser Satan ?
\VS{24}Si un royaume est divisé contre lui-même, ce royaume ne peut subsister ;
\VS{25}et si une maison est divisée contre elle-même, cette maison ne peut subsister.
\VS{26}Si donc Satan s'élève contre lui-même, il est divisé, il ne peut subsister, mais il tend vers sa fin.
\VS{27}Personne ne peut entrer dans la maison d'un homme fort et piller ses biens, sans avoir auparavant lié cet homme fort ; alors il pillera sa maison.
\VS{28}Je vous le dis en vérité, que toutes sortes de péchés seront pardonnés aux enfants des hommes, et aussi toutes sortes de blasphèmes par lesquels ils auront blasphémé ;
\VS{29}mais quiconque blasphémera contre le Saint-Esprit n’obtiendra jamais de pardon : Il est coupable et subira une condamnation éternelle {\FTNT{Voir commentaire Mt. 12:32.}}.
\VS{30}Jésus parla ainsi parce qu'ils disaient : Il est possédé d'un esprit impur.
\TextTitle{[La famille spirituelle]
\\(Mt. 12:46-50 ; Lu. 8:19-21)}
\VS{31}Survinrent ses frères et sa mère qui, se tenant dehors, l'envoyèrent appeler. La multitude était assise autour de lui,
\VS{32}et on lui dit : Voici, ta mère et tes frères sont dehors et te demandent.
\VS{33}Mais il leur répondit : Qui est ma mère, et qui sont mes frères ?
\VS{34}Et, jetant les regards sur ceux qui étaient assis tout autour de lui, il dit : Voici ma mère et mes frères.
\VS{35}Car quiconque fera la volonté de Dieu, celui-là est mon frère, ma sœur, et ma mère.
\TextTitle{[Parabole du semeur et des quatre terrains]
\\(Mt. 13:1-17 ; Lu. 8:4-10)}
\Chap{4}
\VerseOne{}Jésus se mit de nouveau à enseigner près de la mer, et une grande foule s’étant assemblée auprès de lui, il monta dans une barque et s’assit dans la barque, sur la mer. Toute la foule était à terre sur le rivage de la mer.
\VS{2}Il leur enseignait beaucoup de choses en paraboles, et il leur dit dans son enseignement :
\VS{3}Ecoutez. Un semeur sortit pour semer.
\VS{4}Comme il semait, une partie de la semence tomba le long du chemin, et les oiseaux du ciel vinrent, et la mangèrent toute.
\VS{5}Une autre partie tomba dans les endroits pierreux, où elle n'avait pas beaucoup de terre ; elle leva aussitôt, parce qu'elle n'entrait pas profondément dans la terre ;
\VS{6}mais, quand le soleil parut, elle fut brûlée, et parce qu'elle n'avait pas de racine, elle se sécha.
\VS{7}Une autre partie tomba parmi les épines ; et les épines montèrent, et l'étouffèrent, et elle ne donna pas de fruit.
\VS{8}Une autre partie tomba dans la bonne terre, et donna du fruit qui montait et croissait en sorte qu'un grain en rapporta trente, un autre soixante, et un autre cent.
\VS{9}Et il leur dit : Que celui qui a des oreilles pour entendre, qu'il entende !
\VS{10}Lorsqu’il fut à l’écart, ceux qui étaient autour de lui avec les douze, l'interrogèrent touchant cette parabole.
\VS{11}Et il leur dit : Il vous est donné de connaître le mystère du Royaume de Dieu ; mais pour ceux qui sont dehors, tout se passe en paraboles,
\VS{12}afin qu'en voyant ils voient et n'aperçoivent point, et qu'en entendant ils entendent et ne comprennent point, de peur qu'ils ne se convertissent, et que leurs péchés ne leur soient pardonnés.
\TextTitle{[Explication de la parabole]
\\(Mt. 13:18-23 ; Lu. 8:11-15)}
\VS{13}Puis il leur dit : Ne comprenez-vous pas cette parabole ? Et comment donc comprendrez-vous toutes les paraboles ?
\VS{14}Le semeur c'est celui qui sème la parole.
\VS{15}Ceux qui sont le long du chemin, ce sont ceux en qui la parole est semée. Quand ils l’ont entendue, aussitôt Satan vient et enlève la parole qui a été semée dans leurs cœurs.
\VS{16}De même, ceux qui reçoivent la semence dans les endroits pierreux, ce sont ceux qui entendent la parole, ils la reçoivent aussitôt avec joie ;
\VS{17}mais ils n'ont pas de racine en eux-mêmes, ils croient pour un temps, et dès que survient une tribulation ou une persécution à cause de la parole, ils y trouvent une occasion de chute.
\VS{18}D’autres reçoivent la semence parmi les épines ; ce sont ceux qui entendent la parole,
\VS{19}mais en qui les soucis de ce monde, et la séduction des richesses, et les convoitises des autres choses étant entrées dans leurs esprits, étouffent la parole, et elle devient infructueuse.
\VS{20}Mais ceux qui ont reçu la semence dans la bonne terre, ce sont ceux qui entendent la parole, la reçoivent, et portent du fruit : L'un trente, et l'autre soixante, et l'autre cent{\FTNT{Voir commentaire Mt. 13:8.}}.
\TextTitle{[Parabole de la lampe]
\\(Mt. 5:15-16 ; Lu. 8:16-18 ; 11:33-36)}
\VS{21}Il leur dit encore : Apporte-t-on la lampe pour la mettre sous un boisseau, ou sous un lit ? N'est-ce pas pour la mettre sur un chandelier ?
\VS{22}Car il n'y a rien de secret qui ne doive être découvert, rien de caché qui ne doive être mis à jour.
\VS{23}Si quelqu'un a des oreilles pour entendre, qu'il entende.
\VS{24}Il leur dit encore : Prenez garde à ce que vous entendez. On vous mesurera avec la mesure dont vous vous serez servis, et on y ajoutera pour vous.
\VS{25}Car on donnera à celui qui a ; mais à celui qui n’a pas, on ôtera même ce qu’il a.
\TextTitle{[Parabole de la semence et de la croissance spirituelle]}
\VS{26}Il dit encore : Il en est du Royaume de Dieu comme quand un homme jette la semence en terre ;
\VS{27}qu’il dorme ou qu’il veille, nuit et jour, la semence germe et croit, sans qu'il sache comment.
\VS{28}Car la terre produit d'elle-même, premièrement l'herbe, ensuite l'épi, et puis le grain formé dans l'épi ;
\VS{29}et quand le fruit est mûr, on y met aussitôt la faucille, parce que la moisson est prête.
\TextTitle{[Parabole du grain de moutarde]
\\(Mt. 13:31-33 ; Lu. 13:18-19)}
\VS{30}Il dit encore : A quoi comparerons-nous le Royaume de Dieu, ou par quelle parabole le représenterons-nous ?
\VS{31}Il en est comme du grain de moutarde, qui, lorsqu'on le sème dans la terre, est la plus petite de toutes les semences qui sont jetées dans la terre.
\VS{32}Mais après qu'il a été semé, il monte et devient plus grand que toutes les autres plantes, et pousse de grandes branches, en sorte que les oiseaux du ciel peuvent faire leurs nids sous son ombre.
\VS{33}C’est par beaucoup de paraboles de cette sorte qu’il leur annonçait la parole de Dieu, selon qu'ils pouvaient l'entendre.
\VS{34}Et il ne leur parlait point sans paraboles ; mais en particulier, il expliquait tout à ses disciples.
\TextTitle{[Jésus apaise la tempête]
\\(Mt. 8:23-27 ; Lu. 8:22-25)}
\VS{35}Ce même jour sur le soir, Jésus leur dit : Passons sur l’autre bord.
\VS{36}Après avoir renvoyé la foule, ils l'emmenèrent avec eux, dans la barque ; et il y avait aussi d'autres petites barques avec lui.
\VS{37}Et il se leva un grand tourbillon, et les flots se jetaient dans la barque, de sorte qu'elle se remplissait déjà.
\VS{38}Et lui, il dormait à la poupe sur un oreiller. Ils le réveillèrent, et lui dirent : Maître, ne t’inquiètes-tu pas de ce que nous périssions ?
\VS{39}S’étant réveillé, il menaça le vent, et dit à la mer : Silence ! Tais-toi ! Et le vent cessa, et il eut un grand calme.
\VS{40}Puis il leur dit : Pourquoi avez-vous si peur ? Comment n'avez-vous point de foi ?
\VS{41}Et ils furent saisis d'une grande crainte, et ils se dirent les uns les autres : Quel est donc celui-ci, à qui obéissent le vent et la mer ?
\TextTitle{[Jésus-Christ délivre un possédé à Gadara]
\\(Mt. 8:28-34 ; Lu. 8:26-40)}
\Chap{5}
\VerseOne{}Ils arrivèrent sur l’autre bord de la mer, dans le pays des Gadaréniens.
\VS{2}Aussitôt que Jésus fut descendu de la barque, un homme possédé d’un esprit impur, sortit des sépulcres, et vint le rencontrer.
\VS{3}Cet homme avait sa demeure dans les sépulcres, et personne ne pouvait plus le lier, pas même avec des chaînes.
\VS{4}Car souvent, il avait eu les fers aux pieds et avait été lié de chaînes, mais il avait rompu les chaînes et brisé les fers, et personne ne pouvait le dompter.
\VS{5}Il était continuellement, nuit et jour sur les montagnes, et dans les sépulcres, criant et se meurtrissant avec des pierres.
\VS{6}Ayant vu Jésus de loin, il courut et se prosterna devant lui.
\VS{7}Et s’écria d’une voix forte : Qu'y a-t-il entre toi et moi, Jésus, Fils du Dieu Très-Haut ? Je te conjure au Nom de Dieu de ne pas me tourmenter.
\VS{8}Car Jésus lui disait : Sors de cet homme, esprit impur.
\VS{9}Alors il lui demanda : Quel est ton nom ? Légion{\FTNT{Une légion romaine contenait entre trois et six mille soldats. C’est autant de démons dont l’homme était possédé.}} est mon nom, lui répondit-il, car nous sommes plusieurs.
\VS{10}Et il le priait instamment de ne pas les envoyer hors de cette contrée.
\VS{11}Il y avait là, vers les montagnes, un grand troupeau de pourceaux qui paissaient.
\VS{12}Et tous ces démons le priaient en disant : Envoie-nous dans les pourceaux, afin que nous entrions en eux ; et aussitôt Jésus le leur permit.
\VS{13}Alors ces esprits impurs étant sortis, entrèrent dans les pourceaux, qui était environ deux mille, et le troupeau se précipita des pentes escarpées dans la mer ; et ils se noyèrent dans la mer.
\VS{14}Ceux qui paissaient les pourceaux s'enfuirent, et répandirent la nouvelle dans la ville et dans les campagnes.
\VS{15}Ceux de la ville sortirent pour voir ce qui était arrivé. Ils vinrent à Jésus et ils virent le démoniaque, celui qui avait eu la légion, assis et vêtu, et dans son bon sens ; et ils furent saisis de crainte.
\VS{16}Et ceux qui avaient vu le miracle leur racontèrent ce qui était arrivé au démoniaque et aux pourceaux.
\VS{17}Alors ils se mirent à supplier Jésus de quitter leur territoire.
\VS{18}Comme il montait dans la barque, celui qui avait été démoniaque le pria de lui permettre de rester avec lui.
\VS{19}Mais Jésus ne le lui permit pas, mais il lui dit : Va dans ta maison, vers les tiens, et raconte-leur les grandes choses que le Seigneur t'a faites, et comment il a eu pitié de toi.
\VS{20}Il s'en alla donc, et se mit à publier dans la Décapole les grandes choses que Jésus lui avait faites. Et tous furent dans l’étonnement.
\TextTitle{[La résurrection de la fille de Jaïrus et la guérison de la femme atteinte d'une perte de sang]
\\(Mt. 9:18-26 ; Lu. 8:41-56)}
\VS{21}Jésus dans la barque regagna l’autre rive, où une grande foule s’assembla près de lui. Il était près de la mer.
\VS{22}Alors vint un des chefs de la synagogue, nommé Jaïrus, qui l’ayant aperçu, se jeta à ses pieds,
\VS{23}et le pria instamment, en disant : Ma petite fille est à l'extrémité. Je te prie de venir et de lui imposer les mains, afin qu'elle soit guérie et qu'elle vive.
\VS{24}Jésus s'en alla donc avec lui. Et de grandes foules de gens le suivaient et le pressaient.
\VS{25}Or, il y avait une femme qui avait une perte de sang depuis douze ans,
\VS{26}et qui avait beaucoup souffert entre les mains de plusieurs médecins. Elle avait dépensé tout ce qu’elle possédait, sans avoir éprouvé aucun soulagement, mais était allée plutôt en empirant.
\VS{27}Ayant entendu parler de Jésus, elle vint dans la foule par derrière et toucha son vêtement.
\VS{28}Car elle disait : Si je puis seulement toucher ses vêtements, je serai guérie.
\VS{29}Au même instant, la perte de sang s'arrêta ; et elle sentit en son corps qu'elle était guérie de son fléau.
\VS{30}Et aussitôt Jésus connut en lui-même qu’une force était sortie de lui, et, se retournant vers la foule, il dit : Qui a touché mes vêtements ?
\VS{31}Et ses disciples lui dirent : Tu vois que la foule te presse, et tu dis : Qui m'a touché ?
\VS{32}Mais il regardait tout autour pour voir celle qui avait fait cela.
\VS{33}Alors la femme saisie de crainte et toute tremblante, sachant ce qui s’était passé en elle, vint et se jeta à ses pieds, et lui déclara toute la vérité.
\VS{34}Mais Jésus lui dit : Ma fille ! Ta foi t'a sauvée. Va en paix, et sois guérie de ton fléau.
\VS{35}Comme il parlait encore, il vint des gens de chez le chef de la synagogue, qui lui dirent : Ta fille est morte, pourquoi importuner davantage le Maître ?
\VS{36}Mais aussitôt que Jésus eut entendu cela, il dit au chef de la synagogue : Ne crains pas, crois seulement.
\VS{37}Et il ne permit à personne de le suivre, si ce n’est à Pierre, à Jacques, et à Jean, frère de Jacques.
\VS{38}Ils arrivèrent à la maison du chef de la synagogue, où Jésus vit le tumulte, c'est-à-dire ceux qui pleuraient et qui poussaient de grands cris.
\VS{39}Il entra, et leur dit : Pourquoi faites-vous tout ce bruit, et pourquoi pleurez-vous ? L’enfant n'est pas morte, mais elle dort.
\VS{40}Et ils se moquèrent de lui. Mais Jésus les ayant tous fait sortir, prit le père et la mère de la petite fille, et ceux qui étaient avec lui, et entra là où la petite fille était couchée.
\VS{41}Il la saisit par la main, et lui dit : Talitha koumi, ce qui signifie : Jeune fille, je te dis lève-toi.
\VS{42}Aussitôt la petite fille se leva, et se mit à marcher ; car elle était âgée de douze ans. Et ils furent dans un grand étonnement.
\VS{43}Jésus leur recommanda fort expressément que personne ne le sache ; et il dit qu'on donne à manger à la jeune fille.
\TextTitle{[Jésus à Nazareth]}
\Chap{6}
\VerseOne{}Jésus partit de là, et se rendit dans sa patrie. Ses disciples le suivirent.
\VS{2}Quand le jour du sabbat fut venu, il se mit à enseigner dans la synagogue. Et beaucoup de ceux qui l'entendaient étaient dans l'étonnement, et ils disaient : D'où lui viennent ces choses ? Et quelle est cette sagesse qui lui a été donnée, et comment de tels prodiges se font-ils par ses mains ?
\VS{3}N’est-ce pas le charpentier, le Fils de Marie, frère de Jacques, de Joses, de Jude, et de Simon ? Et ses sœurs ne sont-elles pas ici parmi nous ? Et ils étaient scandalisés à cause de lui.
\VS{4}Mais Jésus leur dit : Un prophète n'est méprisé que dans sa patrie, parmi ses parents et dans sa famille.
\VS{5}Et il ne put faire là aucun miracle, si ce n’est qu'il guérit quelques malades en leur imposant les mains.
\VS{6}Et il s'étonnait de leur incrédulité. Jésus parcourait les villages d'alentour, en enseignant.
\TextTitle{[Mission des apôtres]
\\(Mt. 10:1-42 ; Lu. 9:1-6)}
\VS{7}Alors il appela les douze, et commença à les envoyer deux à deux, en leur donnant pouvoir sur les esprits impurs.
\VS{8}Il leur prescrit de ne rien prendre pour le chemin, si ce n’est un bâton, et de ne porter ni sac, ni pain, ni monnaie dans leur ceinture ;
\VS{9}de chausser des sandales, et de ne pas porter deux tuniques.
\VS{10}Il leur disait aussi : Dans quelque maison que vous entriez, demeurez-y jusqu'à ce que vous partiez de là.
\VS{11}Et tous ceux qui ne vous recevront pas, et ne vous écouteront pas, en partant de là, secouez la poussière de vos pieds, en témoignage contre eux. Je vous le dis en vérité que ceux de Sodome et de Gomorrhe seront traités moins rigoureusement au jour du jugement que cette ville-là.
\VS{12}Ils partirent, et ils prêchèrent la repentance.
\VS{13}Ils chassèrent beaucoup de démons hors des possédés, et ils oignirent d'huile beaucoup de malades et les guérirent.
\TextTitle{[Jean-Baptiste décapité]
\\(Mt. 14:1-14 ; Lu. 9:7-9)}
\VS{14}Le roi Hérode entendit parler de Jésus, dont le nom était devenu fort célèbre, et il dit : C’est Jean-Baptiste qui est ressuscité des morts ; c'est pourquoi la puissance de faire des miracles agit puissamment en lui.
\VS{15}D’autres disaient : C'est Elie. Et les autres disaient : C'est un prophète, comme l’un des prophètes.
\VS{16}Mais Hérode en apprenant cela, disait : C'est Jean que j'ai fait décapiter, il est ressuscité des morts.
\VS{17}Car Hérode avait fait arrêter Jean, et l'avait fait lier en prison, à cause d'Hérodias, femme de Philippe son frère, parce qu'il l'avait prise en mariage.
\VS{18}Et que Jean lui disait : Il ne t'est pas permis d'avoir la femme de ton frère.
\VS{19}C'est pourquoi Hérodias était irritée contre Jean, et voulait le faire mourir, mais elle ne le pouvait pas ;
\VS{20}parce qu’Hérode craignait Jean, sachant que c'était un homme juste et saint ; il le protégeait, et, après l’avoir entendu, il faisait beaucoup selon ses avis, et l’écoutait avec plaisir.
\VS{21}Cependant, un jour propice arriva, lorsque Hérode à l’occasion du jour de sa naissance, donna un festin aux grands de sa cour, aux chefs militaires et aux principaux de la Galilée.
\VS{22}La fille d'Hérodias entra dans la salle ; elle dansa et plut à Hérode, et à ceux qui étaient à table avec lui. Le roi dit à la jeune fille : Demande-moi ce que tu voudras, et je te le donnerai.
\VS{23}Il ajouta avec serment : Tout ce que tu me demanderas, je te le donnerai, serait-ce la moitié de mon royaume.
\VS{24}Etant sortie, elle dit à sa mère : Que demanderai-je ? Et sa mère lui dit : La tête de Jean-Baptiste.
\VS{25}Et étant revenue en toute hâte vers le roi, et lui fit cette demande : Je veux que tu me donnes à l’instant sur un plat, la tête de Jean-Baptiste.
\VS{26}Le roi fut attristé, mais à cause de son serment et des convives, il ne voulut pas refuser.
\VS{27}Il envoya sur-le-champ l’un de ses gardes, avec ordre d'apporter la tête de Jean.
\VS{28}Le garde alla décapiter Jean dans la prison, et apporta sa tête sur un plat, et la donna à la jeune fille. Et la jeune fille la donna à sa mère.
\VS{29}Les disciples de Jean ayant appris cela, vinrent et emportèrent son corps, et le mirent dans un sépulcre.
\TextTitle{[Les apôtres rendent compte de leur mission à Jésus]
\\(Lu. 9:10)}
\VS{30}Les apôtres se rassemblèrent auprès de Jésus, et lui racontèrent tout ce qu'ils avaient fait et enseigné.
\VS{31}Jésus leur dit : Venez à l'écart dans un lieu désert, et reposez-vous un peu ; car il y avait beaucoup de gens qui allaient et qui venaient, de sorte qu'ils n'avaient même pas le temps de manger.
\TextTitle{[Multiplication des pains pour les cinq mille hommes]
\\(Mt. 14:12-21 ; Lu. 9:15-17 ; Jn. 6:1-14)}
\VS{32}Ils s'en allèrent donc dans une barque, à l’écart, dans un lieu désert.
\VS{33}Beaucoup de gens les virent s’en aller et les reconnurent, et de toutes les villes on accourut à pied et on les devança au lieu où ils se rendaient.
\VS{34}Quand il sortit, Jésus vit une grande foule, et fut ému de compassion pour elle, parce qu’ils étaient comme des brebis qui n'ont pas de pasteur ; et il se mit à leur enseigner plusieurs choses.
\VS{35}Comme il était déjà tard, ses disciples s'approchèrent de lui, en disant : Ce lieu est désert, et il est déjà tard,
\VS{36}renvoie-les, afin qu’ils s'en aillent dans les campagnes et dans les villages des environs pour s’acheter des pains ; car ils n'ont rien à manger.
\VS{37}Jésus leur répondit : Donnez-leur vous-mêmes à manger. Et ils lui dirent : Irions-nous acheter des pains pour deux cents deniers, et leur donnerions-nous à manger ?
\VS{38}Et il leur dit : Combien avez-vous de pains ? Allez voir. Et quand ils le surent, ils répondirent : Cinq, et deux poissons.
\VS{39}Alors il leur commanda de les faire tous asseoir par groupes sur l'herbe verte.
\VS{40}Et ils s'assirent par rangées de cent et de cinquante personnes.
\VS{41}Il prit les cinq pains et les deux poissons, et, levant les yeux vers le ciel, il bénit Dieu et rompit les pains, puis il les donna à ses disciples, afin qu'ils les distribuent à la foule. Il partagea aussi les deux poissons entre tous.
\VS{42}Tous mangèrent et furent rassasiés.
\VS{43}Et l’on emporta douze paniers pleins de morceaux de pains et de ce qui restait des poissons.
\VS{44}Ceux qui avaient mangé les pains étaient environ cinq mille hommes.
\TextTitle{[Jésus marche sur la mer]
\\(Mt. 14:22-33 ; Jn. 6:15-21)}
\VS{45}Et aussitôt après, il obligea ses disciples à monter dans la barque, et à le devancer sur l’autre bord, vers Bethsaïda, pendant que lui-même renverrait la foule.
\VS{46}Quand il l’eut renvoyée, il s'en alla sur la montagne pour prier.
\VS{47}Le soir étant venu, la barque était au milieu de la mer, et Jésus était seul à terre.
\VS{48}Il vit qu'ils avaient beaucoup de peine à ramer, parce que le vent leur était contraire. Vers la quatrième veille de la nuit, il alla vers eux marchant sur la mer, et il voulait les devancer.
\VS{49}Quand ils le virent marcher sur la mer, ils crurent que c’était un fantôme, et ils poussèrent des cris ;
\VS{50}car ils le voyaient tous, et ils furent troublés. Mais il leur parla aussitôt, et leur dit : Rassurez-vous, c'est moi. N’ayez pas peur.
\VS{51}Et il monta vers eux dans la barque, et le vent cessa. Et ils furent en eux-mêmes excessivement étonnés et remplis d’admiration.
\VS{52}Car ils n'avaient pas compris le miracle des pains, parce que leur cœur était endurci.
\TextTitle{[Jésus guérit les malades à Génésareth]
\\(Mt. 14:34-36)}
\VS{53}Après avoir traversé la mer, ils arrivèrent dans la contrée de Génésareth, où ils abordèrent.
\VS{54}Et dès qu’ils furent sortis de la barque, les gens, ayant aussitôt reconnu Jésus,
\VS{55}parcoururent tous les environs, et se mirent à lui apporter de tous côtés les malades sur de petits lits, partout où ils apprenaient qu'il était.
\VS{56}Et partout où il entrait, dans les villages, dans les villes, ou dans les campagnes, ils mettaient les malades dans les places publiques, et ils le priaient de leur permettre seulement de toucher le bord de son vêtement. Et tous ceux qui le touchaient étaient guéris.
\TextTitle{[Jésus condamne les traditions]
\\(Mt. 15:1-9)}
\Chap{7}
\VerseOne{}Alors les pharisiens, et quelques scribes qui étaient venus de Jérusalem, s'assemblèrent auprès de Jésus.
\VS{2}Ils virent quelques-uns de ses disciples mangeant du pain avec des mains impures, c'est-à-dire non lavées, et ils les blâmèrent.
\VS{3}Or, les pharisiens et tous les Juifs ne mangent pas sans s’être lavé leurs mains jusqu’au coude, conformément à la tradition des anciens.
\VS{4}Et quand ils reviennent de la place publique, ils ne mangent qu’après s’être lavés{\FTNT{Le verbe laver vient du grec «~baptizo~»~: «~Plonger, immerger, submerger, purifier en plongeant ou en submergeant, laver, rendre pur avec de l'eau, se baigner~» (Mt. 3:6-16~; Mt. 28:19~; Ac. 1:5~; Ac. 2:38~; 1 Co. 12:13, etc.) Jésus évoque ici les rites de purification chez les Juifs au premier siècle. A cette époque, le souci de purification avait conduit des groupes comme les pharisiens et les esséniens à multiplier les rites d’eau. Les découvertes de Qumran ont montré que les esséniens vivaient dans la hantise de ce qui aurait pu les rendre impurs. Ainsi, les rituels de purification avec de l’eau rythmaient la vie des juifs. A titre d’exemple, les jarres de Cana étaient utilisés à cet effet (Jn. 2:6).}}. Il y a plusieurs autres observances dont ils se sont chargés, comme le lavage des coupes, de cruches, des vases d'airain, et des lits.
\VS{5}Et les pharisiens et les scribes l'interrogèrent, en disant : Pourquoi tes disciples ne se conduisent-ils pas selon la tradition des anciens, mais prennent-ils leur repas sans se laver les mains ?
\VS{6}Jésus leur répondit : Hypocrites, Esaïe a bien prophétisé de vous, ainsi qu’il est écrit : Ce peuple m'honore des lèvres, mais leur cœur est éloigné de moi{\FTNT{Es. 29:13.}}.
\VS{7}C’est en vain qu’ils m'honorent, en enseignant des doctrines qui sont des commandements d'hommes.
\VS{8}Vous abandonnez le commandement de Dieu, et vous retenez la tradition des hommes, à savoir le lavage des cruches et des coupes, et vous faites beaucoup d'autres choses semblables.
\VS{9}Il leur dit aussi : Vous rejetez bien le commandement de Dieu, afin de garder votre tradition.
\VS{10}Car Moïse a dit : Honore ton père et ta mère ; et : celui qui maudira son père ou sa mère, sera puni de mort.
\VS{11}Mais vous, vous dites : Si quelqu'un dit à son père ou à sa mère : Tout ce dont je pourrais t’assister est corban, c’est-à-dire une offrande à Dieu, il ne sera point coupable.
\VS{12}Et vous ne lui permettez plus de rien faire pour son père ou pour sa mère,
\VS{13}anéantissant ainsi la parole de Dieu par votre tradition que vous avez établie. Et vous faites encore beaucoup d’autres choses semblables.
\TextTitle{[Le coeur humain]
\\(Mt. 15:10-20)}
\VS{14}Ensuite, ayant appelé la foule, il leur dit : Ecoutez-moi vous tous, et comprenez.
\VS{15}Il n’est hors de l’homme rien qui, entrant en lui, puisse le souiller ; mais ce qui sort de l’homme, c’est ce qui le souille.
\VS{16}Si quelqu'un a des oreilles pour entendre, qu'il entende.
\VS{17}Lorsqu’il fut entré dans la maison, loin de la foule, ses disciples l'interrogèrent sur cette parabole.
\VS{18}Et il leur dit : Vous aussi, êtes-vous sans intelligence ? Ne comprenez-vous pas que rien de ce qui du dehors entre dans l’homme ne peut le souiller ?
\VS{19}Car cela n'entre pas dans son cœur, mais dans son ventre, puis s’en va dans les lieux secrets, qui purifient le corps de tous les aliments.
\VS{20}Mais il leur dit : Ce qui sort de l'homme, c'est ce qui souille l'homme.
\VS{21}Car c’est du dedans, c'est-à-dire du cœur des hommes, que sortent les mauvaises pensées, les adultères, les fornications, les meurtres,
\VS{22}les vols, les cupidités, les méchancetés, la fraude, l'impudicité, le regard envieux, la calomnie, l’orgueil, la folie.
\VS{23}Tous ces maux sortent du dedans, et souillent l'homme.
\TextTitle{[Jésus et la femme syro-phénicienne]
\\(Mt. 15:21-28)}
\VS{24}Jésus, étant parti de là, s'en alla dans le territoire de Tyr et de Sidon. Il entra dans une maison, désirant que personne ne le sache ; mais il ne put rester caché.
\VS{25}Car une femme, dont la fille était possédée d'un esprit impur, ayant entendu parler de lui, vint et se jeta à ses pieds.
\VS{26}Cette femme était Grecque, Syro-Phénicienne d’origine. Elle le pria de chasser le démon hors de sa fille. Jésus lui dit :
\VS{27}Laisse premièrement les enfants se rassasier ; car il n'est pas raisonnable de prendre le pain des enfants, et de le jeter aux petits chiens.
\VS{28}Et elle lui répondit : Cela est vrai, Seigneur ! Cependant les petits chiens mangent sous la table les miettes que les enfants laissent tomber.
\VS{29}Alors il lui dit : A cause de cette parole va, le démon est sorti de ta fille.
\VS{30}Et quand elle rentra dans sa maison, elle trouva l’enfant couchée sur le lit, le démon étant sorti.
\TextTitle{[Jésus guérit un sourd-muet]
\\(Mt. 15:29-31)}
\VS{31}Jésus quitta le territoire de Tyr et de Sidon, et revint vers la mer de Galilée en traversant le pays de la Décapole.
\VS{32}On lui amena un sourd qui avait la parole empêchée, et on le pria de lui imposer les mains.
\VS{33}Jésus le prit à part, hors de la foule, lui mit les doigts dans les oreilles, et lui toucha la langue avec sa propre salive.
\VS{34}Puis, levant les yeux vers le ciel, il soupira, et lui dit : Ephphatha, c'est-à-dire : Ouvre-toi.
\VS{35}Aussitôt ses oreilles s'ouvrirent, et le lien de sa langue se délia, et il parla aisément.
\VS{36}Jésus leur recommanda de ne le dire à personne ; mais plus il le leur recommanda, plus ils le publièrent.
\VS{37}Et ils en étaient extrêmement étonnés, et disaient : Il fait tout à merveille ; même il fait entendre les sourds, et parler les muets.
\TextTitle{[Seconde multiplication des pains]
\\(Mt. 15:32-39)}
\Chap{8}
\VerseOne{}En ces jours-là, une grande foule s’était de nouveau réunie et n’avait rien à manger. Jésus appela ses disciples, et leur dit :
\VS{2}Je suis ému de compassion pour cette foule, car il y a déjà trois jours qu'ils sont près de moi, et ils n'ont rien à manger.
\VS{3}Si je les renvoie chez eux à jeun, ils tomberont en défaillance en chemin, car quelques-uns d'eux sont venus de loin.
\VS{4}Ses disciples lui répondirent : Comment pourrait-t-on les rassasier de pains, ici, dans un désert ?
\VS{5}Jésus leur demanda : Combien avez-vous de pains ? Sept lui répondirent-ils.
\VS{6}Alors il ordonna à la foule de s'asseoir par terre, et il prit les sept pains, et après avoir béni Dieu, il les rompit, et les donna à ses disciples pour les distribuer ; et ils les distribuèrent à la foule.
\VS{7}Ils avaient aussi quelques petits poissons ; et après avoir béni Dieu, il les fit aussi distribuer.
\VS{8}Ils mangèrent, et furent rassasiés ; et l’on remporta sept corbeilles pleines des morceaux qui restaient.
\VS{9}Ceux qui avaient mangé étaient environ quatre mille. Ensuite Jésus les renvoya.
\TextTitle{[L'enseignement corrompu des pharisiens]
\\(Mt. 16:1-12)}
\VS{10}Aussitôt après, il monta dans la barque avec ses disciples, et se rendit dans la contrée de Dalmanutha.
\VS{11}Les pharisiens survinrent, se mirent à discuter avec lui, et pour l'éprouver, lui demandèrent un signe venant du ciel.
\VS{12}Alors, Jésus soupirant profondément en son esprit, dit : Pourquoi cette génération demande-t-elle un signe ? Je vous le dis en vérité, il ne sera point donné de signe à cette génération.
\VS{13}Puis il les quitta, et remonta dans la barque, pour passer à l'autre rivage.
\VS{14}Les disciples avaient oublié de prendre des pains ; et ils n'en avaient qu'un seul avec eux dans la barque.
\VS{15}Jésus leur fit cette recommandation : Gardez-vous avec soin du levain des pharisiens et du levain d'Hérode.
\VS{16}Ils raisonnaient entre eux, disant : C'est parce que nous n'avons pas de pains.
\VS{17}Jésus, le sachant, leur dit : Pourquoi discourez-vous sur ce que vous n'avez pas de pains ? N’entendez-vous pas encore, et ne comprenez-vous pas ?
\VS{18}Avez-vous encore votre cœur endurci ? Ayant des yeux, ne voyez-vous point ? Ayant des oreilles, n'entendez-vous point ? Et n'avez-vous point de mémoire ?
\VS{19}Quand j’ai rompu les cinq pains pour les cinq mille hommes, combien de paniers pleins de morceaux avez-vous emportés ? Douze, lui répondirent-ils.
\VS{20}Et quand j’ai rompu les sept pains pour quatre mille hommes, combien de corbeilles pleines de morceaux avez-vous emportées ? Sept, répondirent-ils.
\VS{21}Et il leur dit : Comment n'avez-vous pas d'intelligence ?
\TextTitle{[Jésus guérit un aveugle]}
\VS{22}Ils se rendirent à Bethsaïda, et on lui présenta un aveugle, qu’on le pria de toucher.
\VS{23}Alors il prit la main de l'aveugle, et le conduisit hors du village ; puis il lui mit de la salive sur les yeux, lui imposa les mains, et lui demanda s'il voyait quelque chose.
\VS{24}Et cet homme ayant regardé, dit : Je vois des hommes qui marchent, et qui me paraissent comme des arbres.
\VS{25}Jésus lui mit de nouveau les mains sur les yeux, et lui dit de regarder ; et il fut rétabli, et les voyait tous distinctement.
\VS{26}Puis il le renvoya dans sa maison, en lui disant : N'entre pas dans le village, et ne le dis à personne du village.
\TextTitle{[Pierre reconnaît Jésus comme le Messie]
\\(Mt. 16:13-16 ; Lu. 9:18-21 ; Jn. 6:67-71)}
\VS{27}Jésus s’en alla, avec ses disciples, dans les villages de Césarée de Philippe, et sur le chemin il interrogea ses disciples, leur disant : Qui dit-on que je suis ?
\VS{28}Ils répondirent : Les uns disent que tu es Jean-Baptiste ; les autres, Elie ; et les autres, l'un des prophètes.
\VS{29}Alors il leur dit : Et vous, qui dites-vous que je suis ? Pierre lui répondit : Tu es le Christ.
\VS{30}Et il leur défendit très sévèrement de ne dire cela de lui à personne.
\VS{31}Alors il commença à leur enseigner qu'il fallait que le Fils de l'homme souffre beaucoup, qu'il soit rejeté par les anciens, par les principaux sacrificateurs et par les scribes, qu'il soit mis à mort, et qu'il ressuscite trois jours après.
\VS{32}Il leur tenait ces discours ouvertement. Et Pierre l’ayant pris à part, se mit à le reprendre.
\VS{33}Mais Jésus, se retournant et regardant ses disciples, réprimanda Pierre en lui disant : Va arrière de moi, Satan ! Car tu ne comprends pas les choses de Dieu, mais celles des hommes.
\TextTitle{[La consécration du disciple]
\\(Mt. 16:24-28 ; Lu. 9:23-26)}
\VS{34}Puis, ayant appelé la foule et ses disciples, il leur dit : Si quelqu’un veut venir après moi, qu'il renonce à lui-même, qu'il se charge de sa croix, et qu’il me suive.
\VS{35}Car quiconque voudra sauver son âme, la perdra ; mais quiconque perdra son âme pour l'amour de moi et de l'Evangile, celui-là la sauvera.
\VS{36}Car que sert-il à un homme de gagner tout le monde, s'il perd son âme ?
\VS{37}Que donnerait un homme en échange de son âme ?
\VS{38}Car quiconque aura honte de moi et de mes paroles au milieu de cette génération adultère et pécheresse, le Fils de l'homme aura aussi honte de lui, quand il viendra dans la gloire de son Père avec les saints anges.
\TextTitle{[La transfiguration]
\\(Mt. 17:1-8 ; Lu. 9:27-36)}
\Chap{9}
\VerseOne{}Il leur disait aussi : Je vous le dis en vérité, quelques-uns de ceux qui sont ici présents, ne mourront point qu’ils n’aient vu le Royaume de Dieu venir avec puissance{\FTNT{Voir commentaire Mt. 16:28.}}.
\VS{2}Six jours après, Jésus prit avec lui Pierre, Jacques et Jean, et les conduisit seuls à l'écart sur une haute montagne. Il fut transfiguré devant eux,
\VS{3}ses vêtements devinrent resplendissants, et blancs comme de la neige, tels qu'il n’est pas de foulon sur la terre qui puisse blanchir ainsi.
\VS{4}Et en même temps leur apparurent Moïse et Elie, qui s’entretenaient avec Jésus.
\VS{5}Alors Pierre prenant la parole, dit à Jésus : Rabbi, il est bon que nous soyons ici ; faisons donc trois tentes, une pour toi, une pour Moïse, et une pour Elie.
\VS{6}Car il ne savait pas quoi dire, ils étaient épouvantés.
\VS{7}Une nuée vint les couvrir de son ombre, et de la nuée sortit une voix : Celui-ci est mon Fils bien-aimé, écoutez-le.
\VS{8}Aussitôt les disciples regardèrent tout autour, et ils ne virent que Jésus seul avec eux.
\VS{9}Comme ils descendaient de la montagne, Jésus leur recommanda expressément de ne raconter à personne ce qu'ils avaient vu, jusqu’à ce que le Fils de l'homme soit ressuscité des morts.
\VS{10}Ils retinrent cette parole, se demandant entre eux ce que c'était que ressusciter des morts.
\VS{11}Les disciples l'interrogèrent, disant : Pourquoi les scribes disent-ils qu'il faut qu'Elie vienne premièrement ?
\VS{12}Il leur répondit : Il est vrai, Elie viendra premièrement, et rétablira toutes choses. Et pourquoi est-il écrit du Fils de l'homme qu’il doit beaucoup souffrir et être méprisé ?
\VS{13}Mais je vous dis qu’Elie est venu, et qu'ils lui ont fait tout ce qu’ils ont voulu, selon qu’il est écrit de lui.
\TextTitle{[Incapacité des disciples et la toute-puissance de Jésus-Christ]
\\(Mt. 17:14-21 ; Lu. 9:37-43)}
\VS{14}Lorsqu’il fut arrivé près des disciples, il vit une grande foule autour d’eux, et des scribes qui discutaient avec eux.
\VS{15}Dès que la foule vit Jésus, elle fut saisie d'étonnement, et accourut pour le saluer.
\VS{16}Alors il demanda aux scribes : De quoi discutez-vous avec eux ?
\VS{17}Et un homme de la foule prenant la parole, dit : Maître, je t'ai amené mon fils qui est possédé d’un esprit muet.
\VS{18}En quelque lieu qu’il le saisisse, il le jette par terre ; l’enfant écume, grince des dents, et devient tout raide. J’ai prié tes disciples de le chasser, mais ils n'ont pas pu.
\VS{19}Alors Jésus leur répondit : Ô génération incrédule ! Jusqu’à quand serai-je avec vous ? Jusqu’à quand vous supporterai-je ? Amenez-le-moi. Ils le lui amenèrent.
\VS{20}Et aussitôt que l’enfant vit Jésus, l'esprit l'agita sur-le-champ avec violence ; il tomba par terre, et se roulait en écumant.
\VS{21}Jésus demanda au père de l'enfant : Combien y a-t-il de temps que cela lui arrive ? Et il dit : Dès son enfance.
\VS{22}Et souvent l’esprit l’a jeté dans le feu et dans l'eau pour le faire périr. Mais si tu peux quelque chose, secours-nous, aie compassion de nous.
\VS{23}Alors Jésus lui dit : Si tu peux croire, tout est possible à celui qui croit.
\VS{24}Et aussitôt le père de l'enfant s'écriant avec larmes : Je crois, Seigneur ! Secours-moi dans mon incrédulité.
\VS{25}Jésus voyant accourir la foule, reprit sévèrement l’esprit impur, et lui dit : Esprit muet et sourd, je te l’ordonne, sors de cet enfant, et n'y rentre plus !
\VS{26}Et le démon sortit, en poussant des cris, et en l’agitant avec une grande violence. L’enfant devint comme mort, de sorte que plusieurs disaient qu’il était mort.
\VS{27}Mais Jésus, l'ayant pris par la main, le fit lever. Et il se tint debout.
\VS{28}Quand Jésus fut entré dans la maison, ses disciples lui demandèrent en particulier : Pourquoi n’avons-nous pas pu chasser cet esprit ?
\VS{29}Il leur répondit : Cette sorte de démons ne peut sortir que par la prière et par le jeûne.
\TextTitle{[Jésus annonce sa mort et sa résurrection]
\\(Mt. 17:22-23 ; Lu. 9:44-45)}
\VS{30}Puis étant partis de là, ils traversèrent la Galilée. Jésus ne voulait pas qu’on le sache.
\VS{31}Car il enseignait ses disciples, et il leur dit : Le Fils de l'homme va être livré entre les mains des hommes, et ils le feront mourir, mais après qu'il aura été mis à mort, il ressuscitera le troisième jour.
\VS{32}Mais ils ne comprenaient point ce discours, et ils craignaient de l'interroger.
\TextTitle{[L'humilité, secret de la vraie grandeur]
\\(Mt. 18:1-6 ; Lu. 9:46-48)}
\VS{33}Après ces choses il vint à Capernaüm, et quand il fut arrivé à la maison, il leur demanda : De quoi discutiez-vous ensemble en chemin ?
\VS{34}Mais ils gardèrent le silence, car ils avaient discuté entre eux en chemin sur celui qui serait le plus grand.
\VS{35}Alors il s’assit, appela les douze, et leur dit : Si quelqu'un veut être le premier parmi vous, il sera le dernier de tous, et le serviteur de tous.
\VS{36}Et ayant pris un petit enfant, il le mit au milieu d'eux, et après l'avoir pris entre ses bras, il leur dit :
\VS{37}Quiconque reçoit en mon Nom un de ces petits enfants, me reçoit ; et quiconque me reçoit, ce n'est pas moi qu’il reçoit, mais celui qui m'a envoyé.
\TextTitle{[Jésus condamne l'esprit sectaire]
\\(Lu. 9:49-50)}
\VS{38}Alors Jean prit la parole, et dit : Maître, nous avons vu quelqu'un qui chasse les démons en ton Nom et qui ne nous suit pas, et nous l'en avons empêché, parce qu'il ne nous suit pas.
\VS{39}Mais Jésus leur dit : Ne l'en empêchez pas ; car il n’est personne qui, faisant un miracle en mon Nom, puisse aussitôt après parler mal de moi.
\VS{40}Qui n'est pas contre nous est pour nous.
\VS{41}Et quiconque vous donnera à boire un verre d'eau en mon Nom, parce que vous êtes à Christ, je vous le dis en vérité, il ne perdra point sa récompense.
\TextTitle{[Avertissement de Jésus concernant les occasions de chute]}
\VS{42}Mais quiconque scandalisera un de ces petits qui croient en moi, il vaudrait mieux pour lui qu'on lui mette une pierre de moulin au cou, et qu'on le jette dans la mer.
\VS{43}Si ta main est pour toi une occasion de chute, coupe-la ; mieux vaut pour toi entrer manchot dans la vie, que d'avoir les deux mains, et d’aller dans la géhenne, dans le feu qui ne s'éteint point ;
\VS{44}là où leur ver ne meurt point, et le feu ne s'éteint point.
\VS{45}Si ton pied est pour toi une occasion de chute, coupe-le ; mieux vaut pour toi entrer boiteux dans la vie, que d'avoir les deux pieds, et d’être jeté dans la géhenne, dans le feu qui ne s'éteint point ;
\VS{46}là où leur ver ne meurt point, et où le feu ne s'éteint point.
\VS{47}Si ton œil est pour toi une occasion de chute, arrache-le ; mieux vaut pour toi entrer dans le Royaume de Dieu n'ayant qu'un œil, que d'avoir les deux yeux, et d’être jeté dans le feu de la géhenne,
\VS{48}où leur ver ne meurt point, et où le feu ne s'éteint point.
\VS{49}Car chacun sera salé de feu ; et toute offrande sera salée de sel.
\VS{50}Le sel est une bonne chose ; mais si le sel devient sans saveur, avec quoi lui rendra-t-on sa saveur ?
\VS{51}Ayez du sel en vous-mêmes, et soyez en paix les uns avec les autres.
\TextTitle{[Enseignement de Jésus sur le mariage et le divorce]
\\(Mt. 5:31-32 ; 19:1-9 ; Lu. 16:18 ; Ro. 7:1-3 ; 1 Co. 7:10-16)}
\Chap{10}
\VerseOne{}Jésus, étant parti de là, se rendit dans le territoire de la Judée, au-delà du Jourdain. La foule s’assembla de nouveau auprès de lui, et selon sa coutume, il se mit à l’enseigner.
\VS{2}Alors les pharisiens vinrent à lui, et, pour l'éprouver, ils lui demandèrent s’il est permis à un homme de répudier sa femme.
\VS{3}Il répondit et leur dit : Qu'est-ce que Moïse vous a prescrit ?
\VS{4}Moïse, dirent-ils, a permis d'écrire une lettre de divorce, et de répudier ainsi sa femme{\FTNT{De. 24:1.}}.
\VS{5}Et Jésus leur répondit : C’est à cause de la dureté de votre cœur que Moïse vous a donné ce commandement.
\VS{6}Mais au commencement de la création, Dieu fit l’homme et la femme.
\VS{7}C'est pourquoi l'homme quittera son père et sa mère, et s'attachera à sa femme,
\VS{8}et les deux deviendront une seule chair. Ainsi, ils ne sont plus deux, mais ils sont une seule chair.
\VS{9}Que l'homme donc ne sépare pas ce que Dieu a mis ensemble sous un joug{\FTNT{Voir commentaire Mt. 19:6.}}.
\VS{10}Lorsqu’ils furent dans la maison, ses disciples l'interrogèrent encore là-dessus.
\VS{11}Il leur dit : Celui qui répudie sa femme et qui en épouse une autre, commet un adultère à son égard.
\VS{12}Pareillement si la femme répudie son mari, et se marie à un autre, elle commet un adultère.
\TextTitle{[Jésus bénit les petits enfants]
\\(Mt. 19:13-15 ; Lu. 18:15-17)}
\VS{13}On lui amena de petits enfants afin qu'il les touche. Mais les disciples reprirent ceux qui les amenaient.
\VS{14}Jésus, voyant cela, fut indigné, et leur dit : Laissez venir à moi les petits enfants et ne les en empêchez point, car le Royaume de Dieu appartient à ceux qui leur ressemblent.
\VS{15}Je vous le dis en vérité, quiconque ne recevra pas comme un petit enfant le Royaume de Dieu, il n'y entrera point.
\VS{16}Après les avoir pris dans ses bras, il les bénit, en leur imposant les mains.
\TextTitle{[Le jeune homme riche]
\\(Mt. 19:16-30 ; Lu. 18:18-30 ; Lu. 10:25-37)}
\VS{17}Comme Jésus se mettait en chemin, un homme accourut, et se jetant à genoux devant lui : Bon Maître, lui demanda-t-il, que dois-je faire pour hériter la vie éternelle ?
\VS{18}Jésus lui répondit : Pourquoi m'appelles-tu bon ? Il n'y a de bon que Dieu seul{\FTNT{La même histoire est racontée en Lu. 18:18 qui précise que c’était un chef qui avait interrogé Jésus. La réponse du Seigneur est ironique. Jésus aurait aussi pu lui poser la question comme suit : «~Puisque tu penses que je ne suis qu’un simple homme, pourquoi m’appelles-tu bon ?~».}}.
\VS{19}Tu connais les commandements : Ne commets point d’adultère ; ne tue point ; ne dérobe point ; ne dis point de faux témoignage ; ne fais aucun tort à personne ; honore ton père et ta mère.
\VS{20}Il lui répondit : Maître, j'ai observé toutes ces choses dès ma jeunesse.
\VS{21}Jésus, l’ayant regardé, l'aima, et lui dit : Il te manque une chose : Va, et vends tout ce que tu as, et donne-le aux pauvres, et tu auras un trésor dans le ciel. Puis, viens, et suis-moi en te chargeant de ta croix.
\VS{22}Mais, affligé de cette parole, il s'en alla tout triste, parce qu'il avait de grands biens.
\TextTitle{[Tout est possible à Dieu]}
\VS{23}Alors Jésus, ayant regardé autour de lui, dit à ses disciples : Qu’il est difficile à ceux qui ont des richesses d’entrer dans le Royaume de Dieu.
\VS{24}Ses disciples furent étonnés de ces paroles ; mais Jésus reprenant la parole, leur dit : Mes enfants, qu'il est difficile à ceux qui se confient dans les richesses d'entrer dans le Royaume de Dieu !
\VS{25}Il est plus facile à un chameau de passer par le trou d'une aiguille{\FTNT{Voir commentaire Mt. 19:24.}}, qu’à un riche d’entrer dans le Royaume de Dieu.
\VS{26}Les disciples furent encore plus étonnés, et ils se dire les uns les autres : Et qui peut être sauvé ?
\VS{27}Mais Jésus les ayant regardés, leur dit : Cela est impossible aux hommes, mais non à Dieu ; car tout est possible à Dieu.
\TextTitle{[La fidélité à Jésus-Christ sera récompensée]}
\VS{28}Alors Pierre se mit à lui dire : Voici, nous avons tout quitté et nous t'avons suivi.
\VS{29}Et Jésus répondit, disant : Je vous le dis en vérité, il n’est personne qui, ayant quitté pour l'amour de moi et de l’Evangile, sa maison, ou ses frères, ou ses sœurs, ou son père, ou sa mère, ou sa femme, ou ses enfants, ou ses terres,
\VS{30}ne reçoive au centuple, présentement dans ce temps-ci, des maisons, des frères, des sœurs, des mères, des enfants, et des terres, avec des persécutions ; et dans le siècle à venir, la vie éternelle.
\VS{31}Plusieurs des premiers seront les derniers ; et plusieurs des derniers seront les premiers.
\TextTitle{[Jésus annonce sa mort et sa résurrection]
\\(Mt. 20:17-19 ; Lu. 18:31-34)}
\VS{32}Ils étaient en chemin, pour monter à Jérusalem, et Jésus allait devant eux. Les disciples étaient troublés, et le suivaient avec crainte. Et Jésus prit de nouveau à l'écart les douze, et commença à leur déclarer ce qui devait lui arriver,
\VS{33}disant : Voici, nous montons à Jérusalem, et le Fils de l'homme sera livré aux principaux sacrificateurs et aux scribes. Ils le condamneront à mort, et le livreront aux gentils
\VS{34}qui se moqueront de lui, le battront de verges, cracheront sur lui, et le feront mourir ; et il ressuscitera trois jours après.
\TextTitle{[Jésus répond à la question de Jacques et Jean]}
\VS{35}Alors Jacques et Jean, fils de Zébédée, s’approchèrent de Jésus et lui dirent : Maître, nous voudrions que tu fasses pour nous ce que nous te demanderons.
\VS{36}Il leur dit : Que voulez-vous que je fasse pour vous ?
\VS{37}Et ils lui dirent : Accorde-nous, lui dirent-ils, d’être assis l’un à ta droite et l’autre à ta gauche, quand tu seras dans ta gloire.
\VS{38}Jésus leur dit : Vous ne savez pas ce que vous demandez. Pouvez-vous boire la coupe que je dois boire, et être baptisés du baptême dont je dois être baptisé ?
\VS{39}Ils lui répondirent : Nous le pouvons. Et Jésus leur répondit : Il est vrai que vous boirez la coupe que je dois boire, et que vous serez baptisés du baptême dont je dois être baptisé ;
\VS{40}mais pour ce qui est d'être assis à ma droite et à ma gauche, ce n'est pas à moi de l’accorder ; mais cela ne sera donné qu’à ceux à qui cela est préparé.
\VS{41}Les dix autres, ayant entendu cela, commencèrent à s’indigner contre Jacques et Jean.
\VS{42}Jésus les appela et leur dit : Vous savez que ceux qu’on regarde comme les chefs des nations les dominent, et que les grands les asservissent.
\VS{43}Il n'en sera pas de même parmi vous. Mais quiconque veut être le plus grand parmi vous, qu’il soit votre serviteur,
\VS{44}et quiconque veut être le premier parmi vous, qu’il soit l’esclave de tous.
\VS{45}Car le Fils de l'homme est venu, non pour être servi, mais pour servir et donner sa vie en rançon pour plusieurs.
\TextTitle{[Jésus guérit l'aveugle Bartimée]
\\(Mt. 20:29-34 ; Lu. 18:35-43)}
\VS{46}Ils arrivèrent à Jéricho. Et lorsque Jésus en sortit, avec ses disciples et une grande foule, un aveugle, appelé Bartimée, c'est-à-dire le fils de Timée, était assis au bord du chemin et mendiait.
\VS{47}Il entendit que c'était Jésus de Nazareth, et il se mit à crier et à dire : Jésus, Fils de David, aie pitié de moi !
\VS{48}Plusieurs le reprenaient pour le faire taire ; mais il criait beaucoup plus fort : Fils de David, aie pitié de moi !
\VS{49}Jésus s’arrêta, et dit : Appelez-le. Ils appelèrent l’aveugle en lui disant : Prends courage, lève-toi, il t'appelle.
\VS{50}L’aveugle jeta son manteau, il se leva et vint vers Jésus.
\VS{51}Jésus, prenant la parole, lui dit : Que veux-tu que je te fasse ? Et l'aveugle lui dit : Maître, que je recouvre la vue.
\VS{52}Et Jésus lui dit : Va, ta foi t'a sauvé.
\VS{53}Et aussitôt il recouvra la vue, et suivit Jésus dans le chemin.
\TextTitle{[Entrée de Jésus à Jérusalem]
\\(Mt. 21:1-11 ; Lu. 19:28-40 ; Jn. 12:12-19 ; Za. 9:9)}
\Chap{11}
\VerseOne{}Lorsqu’ils approchaient de Jérusalem, et qu’ils furent près de Bethphagé et de Béthanie, vers le Mont des oliviers, Jésus envoya deux de ses disciples,
\VS{2}en leur disant : Allez au village qui est devant vous. Dès que vous y serez entrés, vous trouverez un ânon attaché, sur lequel aucun homme ne s’est encore assis. Détachez-le, et amenez-le.
\VS{3}Si quelqu'un vous dit : Pourquoi faites-vous cela ? Dites que le Seigneur en a besoin ; et à l’instant, il le laissera venir ici.
\VS{4}Ils partirent donc, et trouvèrent l'ânon qui était attaché dehors, près d’une porte, au contour du chemin, et ils le détachèrent.
\VS{5}Quelques-uns de ceux qui étaient là leur dirent : Pourquoi détachez-vous cet ânon ?
\VS{6}Ils leur répondirent comme Jésus l’avait ordonné ; et on les laissa faire.
\VS{7}Ils amenèrent donc l'ânon à Jésus, sur lequel ils jetèrent leurs vêtements, et Jésus s’assit dessus.
\VS{8}Beaucoup étendirent leurs vêtements sur le chemin, et d'autres des branches qu’ils coupèrent dans les champs.
\VS{9}Ceux qui allaient devant, et ceux qui suivaient, criaient en disant : Hosanna ! Béni soit celui qui vient au Nom du Seigneur !
\VS{10}Béni soit le règne de David notre père, le règne qui vient au Nom du Seigneur ! Hosanna dans les lieux très hauts !
\VS{11}Jésus entra ainsi à Jérusalem, dans le temple. Quand il eut tout considéré, il était déjà tard, il sortit pour aller à Béthanie avec les douze.
\TextTitle{[Le figuier sans fruit]
\\(Mt. 21:18-22)}
\VS{12}Le lendemain, après qu’ils furent sortis de Béthanie, Jésus eut faim.
\VS{13}Apercevant de loin un figuier qui avait des feuilles, il alla voir s'il y trouverait quelque chose ; et s’en étant approché, il ne trouva que des feuilles, car ce n'était pas la saison des figues.
\VS{14}Jésus prenant la parole dit au figuier : Que jamais personne ne mange de ton fruit ! Et ses disciples l'entendirent.
\TextTitle{[Jésus chasse les marchands du temple]
\\(Mt. 21:12-13 ; Lu. 19:45-46 ; Jn. 2:13-16)}
\VS{15}Ils arrivèrent donc à Jérusalem, et Jésus entra dans le temple. Il se mit à chasser dehors ceux qui vendaient, et ceux qui achetaient dans le temple, et il renversa les tables des changeurs, et les sièges de ceux qui vendaient des pigeons.
\VS{16}Il ne laissait personne porter aucun objet à travers le temple.
\VS{17}Et il les enseignait, en leur disant : N'est-il pas écrit : Ma maison sera appelée une maison de prière par toutes les nations ? Mais vous, vous en avez fait une caverne de voleurs{\FTNT{Jé. 7:11.}}.
\VS{18}Les scribes et les principaux sacrificateurs l’ayant entendu, cherchèrent les moyens de le faire périr ; car ils le craignaient, parce que toute la foule était frappée de sa doctrine.
\VS{19}Le soir étant venu, Jésus sortit de la ville.
\TextTitle{[La prière de la foi]
\\(1 Jn. 5:14-15)}
\VS{20}Le matin, en passant, les disciples virent le figuier séché jusqu’aux racines.
\VS{21}Pierre s'étant souvenu de ce qui s'était passé, dit à Jésus : Maître, voici, le figuier que tu as maudit a séché.
\VS{22}Jésus répondant, leur dit : Ayez foi en Dieu.
\VS{23}Je vous le dis en vérité, si quelqu’un dit à cette montagne : Ôte-toi de là et jette-toi dans la mer, et s’il ne doute point en son cœur, mais croit que ce qu’il a dit arrive, il le verra s’accomplir.
\VS{24}C'est pourquoi je vous dis : Tout ce que vous demanderez en priant, croyez que vous l’avez reçu, et vous le verrez s’accomplir.
\TextTitle{[Le pardon]}
\VS{25}Mais quand vous vous présenterez pour faire votre prière, si vous avez quelque chose contre quelqu'un, pardonnez-lui, afin que votre Père qui est dans les cieux vous pardonne aussi vos fautes.
\VS{26}Mais si vous ne pardonnez pas, votre Père qui est dans les cieux ne vous pardonnera point aussi vos fautes.
\TextTitle{[L'autorité de Jésus-Christ mise en doute]
\\(Mt. 21:23-27 ; Lu. 20:1-8)}
\VS{27}Ils se rendirent de nouveau à Jérusalem, et pendant que Jésus marchait dans le temple, les principaux sacrificateurs, les scribes et les anciens vinrent à lui,
\VS{28}et lui dirent : Par quelle autorité fais-tu ces choses, et qui t'a donné cette autorité pour faire les choses que tu fais ?
\VS{29}Jésus leur répondit : Je vous demanderai aussi une chose, et répondez-moi ; puis je vous dirai par quelle autorité je fais ces choses.
\VS{30}Le baptême de Jean venait-il du ciel, ou des hommes ? Répondez-moi.
\VS{31}Et ils raisonnaient entre eux, disant : Si nous disons, du ciel : Il nous dira : Pourquoi donc n’avez-vous pas cru en lui ?
\VS{32}Et si nous disons : Des hommes, nous avons à craindre le peuple, car tous croyaient que Jean était un vrai prophète.
\VS{33}Alors ils répondirent à Jésus : Nous ne savons pas. Et Jésus leur dit : Moi non plus je ne vous dirai pas par quelle autorité je fais ces choses.
\TextTitle{[Parabole des vignerons]
\\(Mt. 21:33-46 ; Lu. 20:9-18 ; Es. 5:1-7)}
\Chap{12}
\VerseOne{}Jésus se mit à leur parler en paraboles : Quelqu'un, dit-il, planta une vigne, et l'environna d'une haie, creusa un pressoir, et bâtit une tour ; puis il la loua à des vignerons, et quitta le pays.
\VS{2}Au temps de la récolte, il envoya un serviteur vers les vignerons, pour recevoir d'eux le fruit de la vigne.
\VS{3}S’étant saisis de lui, ils le battirent, et le renvoyèrent à vide.
\VS{4}Il envoya de nouveau un autre serviteur vers eux. Ils lui jetèrent des pierres, le frappèrent à la tête, et le renvoyèrent après l'avoir outragé.
\VS{5}Il en envoya de nouveau un troisième, qu’ils tuèrent ; et plusieurs autres, et ils battirent les uns, et tuèrent les autres.
\VS{6}Il avait encore un fils, son bien-aimé, il le leur envoya le dernier, disant : Ils auront du respect pour mon fils.
\VS{7}Mais ces vignerons dirent entre eux : Voici l'héritier, venez, tuons-le, et l'héritage sera à nous.
\VS{8}Ils se saisirent de lui, le tuèrent, et le jetèrent hors de la vigne.
\VS{9}Que fera donc le maître de la vigne ? Il viendra, et fera périr ces vignerons, et donnera la vigne à d'autres.
\VS{10}N'avez-vous pas lu cette parole de l’Ecriture ? La pierre qu’ont rejetée ceux qui bâtissaient est devenue la principale de l’angle{\FTNT{Jésus-Christ, la pierre angulaire : Ps. 118:22-23 ; Es. 8:13-17.}} ?
\VS{11}Cela a été fait par le Seigneur, et c'est une chose merveilleuse à nos yeux.
\VS{12}Alors ils cherchaient à se saisir de lui, mais ils craignirent la foule. Ils avaient compris que c’était pour eux que Jésus avait dit cette parabole. Et ils le quittèrent, et s’en allèrent.
\TextTitle{[Le tribut dû à César]
\\(Mt. 22:15-22 ; Lu. 20:19-26)}
\VS{13}Mais ils envoyèrent quelques-uns des pharisiens et des hérodiens auprès de Jésus afin de le surprendre par ses discours.
\VS{14}Et ils vinrent lui dire : Maître, nous savons que tu es vrai, et que tu ne t’inquiètes de personne ; car tu ne regardes pas à l'apparence des hommes, et tu enseignes la voie de Dieu selon la vérité. Est-il permis ou non de payer le tribut à César ? Devons-nous payer, ou ne pas payer ?
\VS{15}Mais Jésus, connaissant leur hypocrisie, leur dit : Pourquoi me tentez-vous ? Apportez-moi un denier, afin que je le voie.
\VS{16}Ils lui en apportèrent un. Alors il leur dit : De qui porte-t-il l’image et l’inscription ? De César, lui répondirent-ils.
\VS{17}Alors Jésus leur dit : Rendez à César ce qui est à César, et à Dieu ce qui est à Dieu. Et ils furent remplis d’admiration pour lui.
\TextTitle{[Jésus répond aux sadducéens sur la résurrection]
\\(Mt. 22:23-33 ; Lu. 20:27-38)}
\VS{18}Alors les sadducéens, qui disent qu'il n'y a point de résurrection, vinrent à lui, et l'interrogèrent, disant :
\VS{19}Maître, voici ce que Moïse nous a prescrit : Si le frère de quelqu'un meurt, et laisse sa femme sans avoir d'enfants, son frère épousera sa veuve et suscitera une postérité à son frère.
\VS{20}Or, il y avait sept frères. Le premier prit une femme et mourut sans laisser d'enfants.
\VS{21}Le deuxième prit la veuve pour femme, et mourut sans laisser de postérité. Il en fut de même du troisième,
\VS{22}et les sept l’épousèrent sans laisser de postérité. Après eux tous, la femme mourut aussi.
\VS{23}A la résurrection, quand ils seront ressuscités, duquel d’entre eux sera-t-elle la femme ? Car les sept l’ont eue pour femme.
\VS{24}Jésus leur répondit : La raison pour laquelle vous tombez dans l'erreur, c'est que vous ne connaissez ni les Ecritures ni la puissance de Dieu.
\VS{25}Car, à la résurrection des morts, les hommes ne prendront point de femmes, ni les femmes de maris, mais ils seront comme les anges dans les cieux.
\VS{26}Et quant aux morts, pour vous montrer qu'ils ressuscitent, n'avez-vous point lu dans le livre de Moïse, comment Dieu lui parla dans le buisson, en disant : Je suis le Dieu d'Abraham, et le Dieu d'Isaac, et le Dieu de Jacob ?
\VS{27}Or il n'est pas le Dieu des morts, mais le Dieu des vivants. Vous êtes donc dans une grande erreur.
\TextTitle{[Jésus répond aux pharisiens concernant le plus grand commandement de la loi]
\\(Mt. 22:34-40 ; Lu. 10:25-28)}
\VS{28}Un des scribes, qui les avait entendus discuter, voyant qu'il leur avait bien répondu, s'approcha de lui, et lui demanda : Quel est le premier de tous les commandements ?
\VS{29}Jésus lui répondit : Le premier de tous les commandements est : Ecoute Israël{\FTNT{Ecoute Israël~: Jésus se réfère ici à De. 6:4~: «~Ecoute, Israël ! Yahweh, notre Dieu Yahweh est Un~». Le Shema Israël est le noyau central de la prière que le Juif adulte doit lire matin et soir. C’est la confession de foi juive. Jacob est le premier à l’avoir enseignée à ses enfants dans Ge. 49:1-2.}}, le Seigneur notre Dieu, le Seigneur est Un{\FTNT{Jésus-Christ, notre Seigneur et notre modèle, a confirmé le Shema Israël qui déclare haut et fort que Dieu est Un et non trois en un. Le scribe, homme versé dans les Ecritures, était satisfait de la réponse de Jésus car il croyait aussi en un seul Dieu. Or le monothéisme est le fondement de la foi juive et des premiers chrétiens.}}.
\VS{30}Tu aimeras le Seigneur ton Dieu de tout ton cœur, de toute ton âme, de toute ta pensée, et de toute ta force. C'est là le premier commandement.
\VS{31}Voici le second, qui est semblable au premier : Tu aimeras ton prochain comme toi-même. Il n'y a pas d'autre commandement plus grand que ceux-là.
\VS{32}Et le scribe lui dit : Maître, tu as bien dit selon la vérité, qu'il y a un seul Dieu, et qu'il n'y en a point d'autre que lui ;
\VS{33}et que de l'aimer de tout son cœur, de toute son intelligence, de toute son âme, et de toute sa force ; et d'aimer son prochain comme soi-même, c'est plus que tous les holocaustes et les sacrifices.
\VS{34}Jésus voyant que ce scribe avait répondu prudemment, lui dit : Tu n'es pas loin du Royaume de Dieu. Et personne n'osait plus l'interroger.
\TextTitle{[Jésus dénonce les scribes]
\\(Mt. 22:41-46 ; Lu. 20:39-44)}
\VS{35}Comme Jésus enseignait dans le temple, il prit la parole et dit : Comment les scribes disent-ils que le Christ est le Fils de David ?
\VS{36}Car David lui-même a dit par le Saint-Esprit : Le Seigneur a dit à mon Seigneur : Assieds-toi à ma droite, jusqu'a ce que je fasse de tes ennemis ton marchepied{\FTNT{Ps. 110:1.}}.
\VS{37}David lui-même l'appelle son Seigneur, comment est-il son fils ? Et une grande foule l’écoutait avec plaisir.
\VS{38}Il leur disait dans son enseignement : Gardez-vous des scribes qui prennent plaisir à se promener en robes longues, et qui aiment les salutations dans les places publiques,
\VS{39}qui recherchent les premiers sièges dans les synagogues, et les premières places dans les festins ;
\VS{40}qui dévorent entièrement les maisons des veuves, et qui font pour l’apparence de longues prières. Ils seront jugés plus sévèrement.
\TextTitle{[L'offrande de la pauvre veuve]
\\(Lu. 21:1-4)}
\VS{41}Jésus, s’étant assis vis-à-vis du tronc, regardait comment la foule y mettait de l'argent. Plusieurs riches y mettaient beaucoup.
\VS{42}Et une pauvre veuve vint, elle y mit deux petites pièces, faisant le quart d’un sou.
\VS{43}Et Jésus, ayant appelé ses disciples, leur dit : Je vous le dis en vérité, cette pauvre veuve a plus mis dans le tronc que tous ceux qui y ont mis.
\VS{44}Car tous ont mis de leur superflu ; mais elle a mis de son nécessaire, tout ce qu'elle possédait, tout ce qu’elle avait pour vivre.
\TextTitle{[Les deux questions des disciples et la prophétie sur la destruction du temple de Jérusalem]
\\Mt. 24:3 ; Lu. 21:7)}
\Chap{13}
\VerseOne{}Lorsque Jésus sortit du temple, un de ses disciples lui dit : Maître, regarde quelles pierres et quelles constructions !
\VS{2}Jésus lui répondit : Vois-tu ces grands bâtiments ? Il ne restera pas pierre sur pierre qui ne soit pas démolie.
\VS{3}Il s’assit sur le Mont des oliviers, en face du temple. Et Pierre, Jacques, Jean et André, lui posèrent en particulier cette question :
\VS{4}Dis-nous quand cela arrivera-t-il, et à quel signe connaîtra-t-on que ces choses vont s'accomplir ?
\TextTitle{[Les temps de la fin]}
\VS{5}Jésus se mit à leur dire : Prenez garde que personne ne vous séduise.
\VS{6}Car plusieurs viendront en mon Nom, disant : C'est moi qui suis le Christ. Et ils séduiront beaucoup de gens.
\VS{7}Quand vous entendrez parler de guerres et des bruits de guerres, ne soyez point troublés ; parce qu'il faut que ces choses arrivent ; mais ce ne sera pas encore la fin.
\VS{8}Car une nation s'élèvera contre une autre nation, et un royaume contre un autre royaume ; et il y aura des tremblements de terre en divers lieux, et il y aura des famines et des troubles. Ces choses seront le commencement des douleurs.
\VS{9}Mais prenez garde à vous-mêmes. Car ils vous livreront aux tribunaux, et aux synagogues, vous serez battus de verges ; vous serez présentés devant les gouverneurs et devant les rois, à cause de moi, pour leur servir de témoignage.
\VS{10}Mais il faut premièrement que l'Evangile soit prêché à toutes les nations.
\VS{11}Et quand ils vous emmèneront pour vous livrer, ne vous inquiétez pas d’avance de ce que vous aurez à dire, mais dites ce qui vous sera donné à l’instant ; car ce n’est pas vous qui parlerez, mais le Saint-Esprit.
\VS{12}Le frère livrera son frère à la mort, et le père son enfant ; et les enfants se soulèveront contre leurs parents, et les feront mourir.
\VS{13}Vous serez haïs de tous à cause de mon Nom ; mais celui qui persévérera jusqu’à la fin, sera sauvé.
\TextTitle{[L'abomination de la désolation]
\\(Mt. 24:15-28 ; Ps. 2.5 ; Lu. 21:20-24 ; Ap. 7:14)}
\VS{14}Lorsque vous verrez l'abomination qui cause la désolation{\FTNT{Voir commentaire Mt. 24:15.}} qui a été prédite par Daniel, le prophète, établie là où elle ne doit pas être, que celui qui lit ce prophète fasse attention ! Alors que ceux qui seront en Judée fuient dans les montagnes.
\VS{15}Que celui qui sera sur le toit, ne descende pas dans la maison, et n’entre pas pour emporter quoi que ce soit de sa maison,
\VS{16}et que celui qui sera dans les champs, ne retourne pas en arrière pour emporter son manteau.
\VS{17}Malheur aux femmes qui seront enceintes, et à celles qui allaiteront en ces jours-là.
\VS{18}Priez Dieu que votre fuite n'arrive pas en hiver.
\VS{19}Car la détresse, en ces jours, sera telle qu’il n’y en a point eu de semblable depuis le commencement du monde que Dieu a créé jusqu’à présent, et qu’il n’y en aura jamais.
\VS{20}Et si le Seigneur n’avait abrégé ces jours, personne ne serait sauvé ; mais il les a abrégés, à cause des élus qu'il a choisis.
\VS{21}Si quelqu'un vous dit : Voici, le Christ est ici ; ou voici, il est là, ne le croyez point.
\VS{22}Car il s'élèvera des faux christs et des faux prophètes, qui feront des prodiges et des miracles, pour séduire même les élus s'il était possible.
\VS{23}Soyez sur vos gardes ; voici, je vous ai tout annoncé d’avance.
\TextTitle{[Retour du Messie sur la terre]
\\(Mt. 24:29-31 ; Lu. 21:25-28)}
\VS{24}Mais dans ces jours, après cette détresse, le soleil s’obscurcira, et la lune ne donnera plus sa clarté ;
\VS{25}les étoiles du ciel tomberont, et les puissances qui sont dans les cieux seront ébranlées.
\VS{26}Alors ils verront le Fils de l'homme venant sur les nuées, avec une grande puissance et une grande gloire.
\VS{27}Alors il enverra ses anges, et il rassemblera ses élus des quatre vents, de l’extrémité de la terre jusqu’à l’extrémité du ciel.
\TextTitle{[Parabole du figuier]
\\(Mt. 24:32-35 ; Lu. 21:29-33)}
\VS{28}Instruisez-vous par une comparaison tirée du figuier. Dès que ses branches deviennent tendres, et que les feuilles poussent, vous savez que l'été est proche.
\VS{29}Ainsi, quand vous verrez ces choses arriver, sachez que le Fils de l’homme est proche, à la porte.
\VS{30}Je vous le dis en vérité, cette génération ne passera point, que toutes ces choses ne soient arrivées.
\VS{31}Le ciel et la terre passeront, mais mes paroles ne passeront point.
\TextTitle{[Exhortation de Jésus sur la vigilance]
\\(Mt. 24:36-51 ; Lu. 21:34-38)}
\VS{32}Pour ce qui est du jour ou de l’heure, personne ne le sait, ni les anges dans le ciel, ni le Fils{\FTNT{Comment expliquer l’ignorance du Fils quant à l’heure de son retour ? En prenant la condition d’un homme, Jésus s’est dépouillé de ses prérogatives divines et a connu des limites propres au genre humain (Ph. 2:7)~: la fatigue (Jn. 4:6 ; Mc. 4:38), la faim (Mc. 11:12), l’angoisse et la peur (Mc. 14:33), la mortalité physique… Ce dépouillement incluait le renoncement à l’omniscience, d’où le fait que Jésus-Christ homme ne connaissait pas le jour et l’heure de son retour.}}, mais mon Père seul.
\VS{33}Prenez garde, veillez et priez ; car vous ne savez quand ce temps viendra.
\VS{34}Il en sera comme d’un homme qui, partant pour un voyage, laisse sa maison, remet l’autorité à ses serviteurs, marquant à chacun sa tâche, et ordonne au portier de veiller.
\VS{35}Veillez donc, car vous ne savez quand le Maître de la maison viendra, ou le soir, ou à minuit, ou à l'heure où le coq chante, ou le matin ;
\VS{36}craignez qu’il ne vous trouve endormis, à son arrivée soudaine.
\VS{37}Ce que je vous dis, je le dis à tous : Veillez.
\TextTitle{[Le complot]
\\(Mt. 26:1-5 ; Lu. 22:1-2)}
\Chap{14}
\VerseOne{}La fête de Pâque et des pains sans levain devait avoir lieu deux jours après. Les principaux sacrificateurs et les scribes cherchaient les moyens de se saisir de Jésus par ruse, et de le faire mourir.
\VS{2}Mais ils disaient : Que ce ne soit pas pendant la fête, afin qu'il n’y ait pas de tumulte parmi le peuple.
\TextTitle{[Marie de Béthanie oint Jésus pour sa sépulture]
\\(Mt. 26:6-13 ; Jn. 12:1-8)}
\VS{3}Comme Jésus était à Béthanie, dans la maison de Simon le lépreux, et pendant qu’il était à table, une femme vint à lui avec un vase d'albâtre, rempli d'un parfum de nard pur et de grand prix ; et ayant rompu le vase, elle répandit le parfum sur la tête de Jésus.
\VS{4}Quelques-uns en furent indignés en eux-mêmes, et ils disaient : A quoi sert la perte de ce parfum ?
\VS{5}On aurait pu le vendre plus de trois cents deniers, et les donner aux pauvres. Ainsi ils murmuraient contre elle.
\VS{6}Mais Jésus dit : Laissez-la. Pourquoi lui faites-vous de la peine ? Elle a fait une bonne action à mon égard.
\VS{7}Parce que vous aurez toujours des pauvres avec vous, et vous pouvez leur faire du bien quand vous voulez ; mais vous ne m'aurez pas toujours.
\VS{8}Elle a fait ce qu’elle a pu ; elle a d’avance embaumé mon corps pour la sépulture.
\VS{9}Je vous le dis en vérité, partout où cet Evangile sera prêché, dans le monde entier, on racontera aussi en mémoire de cette femme ce qu’elle a fait.
\TextTitle{[La trahison de Judas]
\\(Mt. 26:14-16 ; Lu. 22:3-6)}
\VS{10}Alors Judas Iscariot, l'un des douze, alla vers les principaux sacrificateurs pour le livrer.
\VS{11}Après l’avoir entendu, ils furent dans la joie, et promirent de lui donner de l'argent. Et Judas cherchait une occasion favorable pour le livrer.
\TextTitle{[La dernière Pâque]
\\(Mt. 26:17-25 ; Lu. 22:7-20 ; Jn. 13:1-12)}
\VS{12}Le premier jour des pains sans levain, où l’on sacrifiait l'agneau de Pâque, ses disciples lui dirent : Où veux-tu que nous allions te préparer l'agneau de Pâque afin que tu manges ?
\VS{13}Et il envoya deux de ses disciples, et leur dit : Allez dans la ville, vous rencontrerez un homme portant une cruche d'eau, suivez-le.
\VS{14}Où qu’il entre, dites au maître de la maison : Le Maître dit : Où est le lieu où je mangerai l'agneau de Pâque avec mes disciples ?
\VS{15}Et il vous montrera une grande chambre haute, meublée et toute prête : C’est là que vous nous préparerez l'agneau de Pâque.
\VS{16}Ses disciples partirent, arrivèrent dans la ville, ils trouvèrent les choses comme il l’avait dit ; et ils apprêtèrent l'agneau de Pâque.
\VS{17}Le soir étant venu, il arriva avec les douze.
\VS{18}Pendant qu’ils étaient à table, et qu'ils mangeaient, Jésus leur dit : Je vous le dis en vérité, l'un de vous, qui mange avec moi, me trahira.
\VS{19}Ils commencèrent à s'attrister, et ils lui dirent l'un après l'autre : Est-ce moi ?
\VS{20}Mais il leur répondit : C'est l'un des douze qui trempe avec moi dans le plat.
\VS{21}Certes le Fils de l'homme s'en va, selon qu'il est écrit de lui. Mais malheur à l'homme par qui le Fils de l'homme est trahi ! Mieux vaudrait pour cet homme qu’il ne soit pas né.
\TextTitle{[Le repas de la Pâque]
\\(Mt. 26:26-29 ; Lu. 22:17-20 ; Jn. 13:12-30 ; 1 Co. 11:23-26)}
\VS{22}Pendant qu’ils mangeaient, Jésus prit du pain, et après avoir béni Dieu, il le rompit et le leur donna, et leur dit : Prenez, mangez, ceci est mon corps.
\VS{23}Il prit ensuite une coupe, et après avoir rendu grâces, il la leur donna, et ils en burent tous.
\VS{24}Et il leur dit : Ceci est mon sang{\FTNT{Nouvelle Alliance : Voir Jn. 19:30.}}, le sang de la nouvelle alliance, qui est répandu pour plusieurs.
\VS{25}Je vous le dis en vérité, je ne boirai plus du fruit de la vigne jusqu'au jour où j’en boirai du nouveau dans le Royaume de Dieu.
\TextTitle{[Jésus avertit Pierre de son triple reniement]
\\(Mt. 26:30-35 ; Lu. 22:31-34 ; Jn. 13:36-38)}
\VS{26}Après avoir chanté les cantiques{\FTNT{Cantiques~: Voir Mt. 26:30.}}, ils se rendirent à la montagne des oliviers.
\VS{27}Jésus leur dit : Vous serez tous cette nuit scandalisés en moi ; car il est écrit : Je frapperai le Berger, et les brebis seront dispersées{\FTNT{Za. 13:7.}}.
\VS{28}Mais, après que je serai ressuscité, je vous précéderai en Galilée.
\VS{29}Pierre lui dit : Quand même tous seraient scandalisés, je ne le serai pourtant pas moi.
\VS{30}Et Jésus lui dit : Je te le dis en vérité, qu'aujourd'hui, cette nuit même, avant que le coq chante deux fois, tu me renieras trois fois.
\VS{31}Mais Pierre disait encore plus fortement : Quand même il me faudrait mourir avec toi, je ne te renierai pas. Et tous lui dirent la même chose.
\TextTitle{[Jésus dans le jardin de Gethsémané]
\\(Mt. 26:36-46 ; Lu. 22:39-46 ; Jn. 18:1)}
\VS{32}Ils allèrent dans un lieu appelé Gethsémané, et Jésus dit à ses disciples : Asseyez-vous ici jusqu'à ce que j’aie prié.
\VS{33}Il prit avec lui Pierre, Jacques et Jean, et il commença à être effrayé et fort angoissé.
\VS{34}Il leur dit : Mon âme est saisie de tristesse jusqu’à la mort, restez ici, et veillez.
\TextTitle{[Première prière de Jésus]
\\(Mt. 26:36 ; Lu. 22:41-42)}
\VS{35}Puis s'en allant un peu plus en avant, il se jeta contre terre, et pria que s'il était possible, cette heure s’éloigne de lui.
\VS{36}Il disait : Abba, Père, toutes choses te sont possibles, éloigne de moi cette coupe ! Toutefois, non pas ce que je veux, mais ce que tu veux.
\VS{37}Puis il vint vers les disciples qu’il trouva endormis, et il dit à Pierre : Simon, tu dors ! Tu n’as pas pu veiller une heure !
\VS{38}Veillez et priez afin que vous ne tombiez pas en tentation, l'esprit est bien disposé, mais la chair est faible.
\TextTitle{[Deuxième prière]
\\(Mt. 26:42 ; Lu. 22:44)}
\VS{39}Il s’éloigna de nouveau, et fit la même prière, disant les mêmes paroles.
\VS{40}Il revint, et les trouva encore endormis, car leurs yeux étaient appesantis. Ils ne surent que lui répondre.
\TextTitle{[Troisième prière]
\\(Mt. 26:44)}
\VS{41}Il revint encore, pour la troisième fois, et leur dit : Dormez maintenant, et reposez-vous ! C’est assez ! L’heure est venue ; voici, le Fils de l'homme est livré entre les mains des méchants.
\VS{42}Levez-vous, allons ; voici, celui qui me trahit s'approche.
\TextTitle{[Jésus trahi, abandonné et arrêté]
\\(Mt. 26:47-56 ; Lu. 22:47-53 ; Jn. 18:2-11)}
\VS{43}Et aussitôt, comme il parlait encore, Judas, l'un des douze, vint, et avec lui une grande foule ayant des épées et des bâtons, envoyée par les principaux sacrificateurs, par les scribes et par les anciens.
\VS{44}Celui qui le trahissait leur avait donné ce signe : Celui que j'embrasserai, c’est lui ; saisissez-le, et emmenez-le sûrement.
\VS{45}Dès qu’il fut arrivé, il s'approcha aussitôt de Jésus, et lui dit : Rabbi, Rabbi ! Et il l’embrassa.
\VS{46}Alors ils mirent la main sur Jésus, et le saisirent.
\VS{47}Un de ceux qui étaient là présents, tirant son épée, frappa le serviteur du souverain sacrificateur et lui emporta l'oreille.
\VS{48}Alors Jésus prit la parole, et leur dit : Vous êtes venus comme après un brigand, avec des épées et des bâtons, pour m’arrêter.
\VS{49}J’étais tous les jours parmi vous, enseignant dans le temple, et vous ne m'avez point saisi ; mais tout ceci est arrivé afin que les Ecritures soient accomplies.
\VS{50}Alors tous ses disciples l'abandonnèrent et s'enfuirent.
\VS{51}Un jeune homme le suivait, n’ayant sur le corps qu’un drap. Et quelques jeunes gens le saisirent,
\VS{52}mais il abandonna son linceul, et se sauva tout nu.
\TextTitle{[Jésus devant Caïphe et le sanhédrin]
\\(Mt. 26:57-68 ; Jn. 18:12-14,19-24)}
\VS{53}Ils emmenèrent Jésus chez le souverain sacrificateur, où s'assemblèrent tous les principaux sacrificateurs, les anciens et les scribes.
\VS{54}Pierre le suivait de loin jusque dans la cour du souverain sacrificateur ; et il était assis avec les serviteurs, et se chauffait près du feu.
\VS{55}Les principaux sacrificateurs et tout le sanhédrin cherchaient quelque témoignage contre Jésus pour le faire mourir, mais ils n'en trouvaient point.
\VS{56}Car plusieurs rendaient de faux témoignages contre lui, mais les témoignages ne s’accordaient pas.
\VS{57}Alors quelques-uns s'élevèrent, et portèrent de faux témoignages contre lui, disant :
\VS{58}Nous l’avons entendu dire : Je détruirai ce temple qui est fait de main d’homme, et en trois jours j'en rebâtirai un autre qui ne sera pas fait de main d’homme.
\VS{59}Même sur ce point-là leurs témoignages ne s’accordaient pas.
\VS{60}Alors le souverain sacrificateur se levant au milieu, interrogea Jésus, disant : Ne réponds-tu rien ? Qu’est-ce que ces gens déposent contre toi ?
\VS{61}Mais Jésus garda le silence, et ne répondit rien. Le souverain sacrificateur l'interrogea de nouveau, et lui dit : Es-tu le Christ, le Fils du Dieu béni ?
\VS{62}Jésus lui répondit : Je le suis. Et vous verrez le Fils de l'homme assis à la droite de la puissance de Dieu, et venant sur les nuées du ciel.
\VS{63}Alors le souverain sacrificateur déchira ses vêtements et dit : Qu'avons-nous encore besoin de témoins ?
\VS{64}Vous avez entendu le blasphème. Que vous en semble ? Alors tous le condamnèrent comme méritant la mort.
\VS{65}Et quelques-uns se mirent à cracher sur lui, à lui voiler le visage, et à lui donner des soufflets, en lui disant : Prophétise ! Et les serviteurs lui donnaient des coups avec leurs verges.
\TextTitle{[Triple reniement de Pierre]
\\(Mt. 26:69-75 ; Lu. 22:55-62 ; Jn. 18:15-18,25-27)}
\VS{66}Pendant que Pierre était en bas dans la cour, une des servantes du souverain sacrificateur vint.
\VS{67}Apercevant Pierre qui se chauffait, elle le regarda en face, et lui dit : Toi aussi, tu étais avec Jésus de Nazareth.
\VS{68}Mais il le nia, disant : Je ne le connais pas, et je ne sais pas ce que tu dis ; puis il sortit dehors pour aller dans le vestibule. Et le coq chanta.
\VS{69}La servante l'ayant vu de nouveau, elle se mit à dire à ceux qui étaient là présents : Celui-ci est de ces gens-là. Et il le nia de nouveau.
\VS{70}Peu après, ceux qui étaient là présents, dirent à Pierre : Certainement tu es de ces gens-là, car tu es Galiléen, et ton langage s'y rapporte.
\VS{71}Alors il commença à faire des imprécations et à jurer : Je ne connais pas cet homme dont vous parlez.
\VS{72}Et le coq chanta pour la seconde fois. Et Pierre se souvint de la parole que Jésus lui avait dite : Avant que le coq chante deux fois, tu me renieras trois fois. Et étant sorti promptement, il pleura.
\TextTitle{[Jésus livré à Pilate]
\\(Mt. 27:1-2,11-15 ; Lu. 23:1-7,13-18 ; Jn. 18:28-38 ; 19:1-15)}
\Chap{15}
\VerseOne{}Dès le matin, les principaux sacrificateurs tinrent conseil avec les anciens et les scribes, et tout le sanhédrin. Après avoir lié Jésus, ils l'emmenèrent, et le livrèrent à Pilate.
\VS{2}Pilate l'interrogea : Es-tu le Roi des Juifs ? Et Jésus répondit : Tu le dis.
\VS{3}Les principaux sacrificateurs l'accusaient de plusieurs choses, mais il ne répondit rien.
\VS{4}Pilate l'interrogea de nouveau : Ne réponds-tu rien ? Vois de combien de choses ils t’accusent.
\VS{5}Mais Jésus ne donna plus aucune réponse, ce qui étonna Pilate.
\TextTitle{[Jésus ou Barabbas ?]
\\(Mt. 27:15-26 ; Lu. 23:17-25 ; Jn. 18:39)}
\VS{6}A chaque fête, il relâchait un prisonnier, celui que demandait la foule.
\VS{7}Il y avait en prison un nommé Barabbas avec ses complices pour une sédition, dans laquelle ils avaient commis un meurtre.
\VS{8}La foule se mit à demander à Pilate, avec de grands cris, ce qu’il avait coutume de leur accorder.
\VS{9}Pilate leur répondit : Voulez-vous que je vous relâche le Roi des Juifs ?
\VS{10}Car il savait bien que les principaux sacrificateurs l'avaient livré par envie.
\VS{11}Mais les principaux sacrificateurs excitèrent la foule, afin que Pilate leur relâche plutôt Barabbas.
\VS{12}Pilate reprenant la parole, leur dit encore : Que voulez-vous donc que je fasse de celui que vous appelez Roi des Juifs ?
\VS{13}Ils crièrent de nouveau : Crucifie-le !
\VS{14}Alors Pilate leur dit : Mais quel mal a-t-il fait ? Et ils crièrent encore plus fort : Crucifie-le !
\VS{15}Pilate, voulant satisfaire la foule, leur relâcha Barabbas ; et après avoir fait battre de verges Jésus, il le livra pour être crucifié.
\TextTitle{[Jésus couronné d'épines]
\\(Mt. 27:27-31 ; Jn. 19:16-17)}
\VS{16}Alors les soldats emmenèrent Jésus dans l’intérieur de la cour, c’est-à-dire dans le prétoire, et ils assemblèrent toute la cohorte.
\VS{17}Ils le revêtirent d'une robe de pourpre, et posèrent sur sa tête une couronne d'épines qu’ils avaient tressée.
\VS{18}Puis ils commencèrent à le saluer, en lui disant : Nous te saluons, Roi des Juifs !
\VS{19}Et ils lui frappaient la tête avec un roseau, et crachaient sur lui, et fléchissant les genoux, ils se prosternaient devant lui.
\VS{20}Et après s'être ainsi moqués de lui, ils le dépouillèrent de la robe de pourpre, lui remirent ses habits, et l'emmenèrent dehors pour le crucifier.
\VS{21}Et un certain homme de Cyrène, nommé Simon, père d’Alexandre et de Rufus, passant par là en revenant des champs, fut forcé à porter la croix de Jésus.
\VS{22}Et ils conduisirent Jésus au lieu appelé Golgotha{\FTNT{Golgotha~: Le Golgotha (crâne) était une colline située à l'extérieur de Jérusalem, sur laquelle les Romains crucifiaient les condamnés.}}, c'est-à-dire, le lieu du Crâne.
\VS{23}Ils lui donnèrent à boire du vin mêlé de myrrhe, mais il ne le prit pas.
\TextTitle{[Jésus crucifié]
\\(Mt. 27:33-56 ; Lu. 23:33-49 ; Jn. 19:17-37)}
\VS{24}Ils le crucifièrent, et se partagèrent ses vêtements, en tirant au sort pour savoir ce que chacun aurait.
\VS{25}C’était la troisième heure, quand ils le crucifièrent.
\VS{26}L’écriteau indiquant la cause de sa condamnation portait ces mots : Le Roi des Juifs.
\VS{27}Ils crucifièrent aussi avec lui deux brigands, l'un à sa droite, et l'autre à sa gauche.
\VS{28}Et ainsi fut accomplie l'Ecriture, qui dit : Et il a été mis au rang des malfaiteurs{\FTNT{Es. 53:12.}}.
\VS{29}Les passants l’injuriaient, et secouaient la tête, en disant : Hé ! Toi qui détruis le temple et qui le rebâtis en trois jours,
\VS{30}sauve-toi toi-même, et descends de la croix !
\VS{31}Les principaux sacrificateurs aussi avec les scribes se moquaient entre eux, et disaient : Il a sauvé les autres, et il ne peut se sauver lui-même.
\VS{32}Que le Christ, le Roi d’Israël descende maintenant de la croix, afin que nous le voyions et que nous croyions ! Ceux qui étaient crucifiés avec lui l’insultaient aussi.
\VS{33}La sixième heure étant venue, il y eut des ténèbres sur toute la terre jusqu'à la neuvième heure.
\VS{34}Et à la neuvième heure, Jésus s’écria d’une voix forte : Eloï, Eloï, lama sabachthani ? C’est-à-dire : Mon Dieu ! Mon Dieu ! Pourquoi m'as-tu abandonné ?
\VS{35}Quelques-uns de ceux qui étaient là présents, l’ayant entendu, dirent : Voici, il appelle Elie.
\VS{36}Et l’un d’eux courut remplir une éponge de vinaigre{\FTNT{Le vinaigre~: Voir Mt. 27:34.}}, et l'ayant fixée au bout d'un roseau, il lui donna à boire, en disant : Laissez, voyons si Elie viendra le descendre de la croix.
\VS{37}Mais Jésus, ayant poussé un grand cri, expira.
\VS{38}Et le voile du temple se déchira en deux, depuis le haut jusqu'en bas{\FTNT{Hé 10:19-20.}}.
\TextTitle{[Fin de la Première Alliance]
\\(Hé. 9:16-18)}
\VS{39}Le centenier, qui était en face de Jésus, voyant qu'il avait expiré en criant de la sorte, dit : Certainement cet homme était Fils de Dieu.
\VS{40}Il y avait là aussi des femmes qui regardaient de loin. Parmi elles étaient Marie de Madgala, Marie mère de Jacques le mineur et de Joses, et Salomé,
\VS{41}qui le suivaient et le servaient lorsqu'il était en Galilée, et plusieurs autres qui étaient montées avec lui à Jérusalem.
\TextTitle{[Jésus enseveli]}
\VS{42}Le soir étant venu, comme c'était la préparation, c’est-à-dire le sabbat,
\VS{43}arriva Joseph d'Arimathée, conseiller de distinction, qui attendait aussi le Royaume de Dieu. Il osa se rendre vers Pilate pour demander le corps de Jésus.
\VS{44}Pilate s'étonna qu'il soit mort si tôt ; il fit venir le centenier, et lui demanda s'il était mort depuis longtemps.
\VS{45}S’en étant assuré par le centenier, il donna le corps à Joseph.
\VS{46}Et Joseph ayant acheté un linceul, descendit Jésus de la croix, et l'enveloppa du linceul, et le déposa dans un sépulcre taillé dans le roc. Puis il roula une pierre sur l'entrée du sépulcre.
\VS{47}Marie de Magdala, et Marie mère de Joses regardaient où on le mettait.
\TextTitle{[Jésus, ressuscité, apparaît à plusieurs disciples]
\\(Mt. 28:1-15 ; Lu. 24:1-49 ; Jn. 20:1-23)}
\Chap{16}
\VerseOne{}Lorsque le sabbat fut passé, Marie de Magdala, Marie mère de Jacques, et Salomé, achetèrent des aromates pour embaumer Jésus.
\VS{2}Le premier jour de la semaine, de grand matin, elles se rendirent au sépulcre, comme le soleil venait de se lever.
\VS{3}Elles disaient entre elles : Qui nous roulera la pierre de l'entrée du sépulcre ?
\VS{4}Et levant les yeux, elles virent que la pierre, qui était très grande, avait été roulée.
\VS{5}Elles entrèrent dans le sépulcre, virent un jeune homme assis à droite, vêtu d'une robe blanche, et elles furent épouvantées.
\VS{6}Mais il leur dit : Ne vous épouvantez pas. Vous cherchez Jésus de Nazareth qui a été crucifié. Il est ressuscité, il n'est point ici ; voici le lieu où on l'avait mis.
\VS{7}Mais allez, et dites à ses disciples, et à Pierre, qu'il vous précède en Galilée. C’est là que vous le verrez, comme il vous l'a dit.
\VS{8}Elles partirent aussitôt et s'enfuirent du sépulcre. La peur et le trouble les avaient saisies ; et elles ne dirent rien à personne, à cause de la peur.
\VS{9}Jésus étant ressuscité, le matin du premier jour de la semaine{\FTNT{Jésus a-t-il été crucifié un vendredi ? Si c’est le cas, comment a-t-il pu séjourner trois jours dans le tombeau s’il est ressuscité le dimanche matin comme l’enseigne la tradition catholique et la majorité des églises protestantes et évangéliques ? Tout d’abord il convient de signaler que selon Ge. 1, le jour commence au coucher du soleil, aux environs de dix-huit heures, et s’achève le lendemain au coucher du soleil. Chez les Romains, le jour commence à minuit et se termine le lendemain à minuit. C’est de cette manière que l’évangile de Jean compte les heures. Dans les autres évangiles, les journées commencent avec le lever du soleil. Jésus a été crucifié à «~la troisième heure~» (Mc. 15:25), ce qui correspond à neuf heures du matin. Ensuite, les évangiles nous apprennent qu’il y a eu des ténèbres sur la terre de la sixième à la neuvième heure, donc de midi à quinze heures (Mt. 27:45-46 ; Mc. 15:33-34 ; Lu. 23:44). Jésus est donc mort avant dix-huit heures. Ainsi, il est évident qu’il n’a pas pu passer toute la journée du vendredi au tombeau. Les Ecritures ne déclarent pas spécifiquement quel jour de la semaine Jésus a été crucifié. Les deux opinions dominantes sont vendredi et mercredi. D’autres font la synthèse des deux et acceptent le jeudi comme étant le jour de la crucifixion.  Jésus dit dans Mt. 12:40~: «~Car, de même que Jonas fut trois jours et trois nuits dans le ventre d’une baleine, de même le Fils de l'homme sera trois jours et trois nuits dans le sein de la terre~». Ceux qui défendent la crucifixion un vendredi disent qu’il est possible de compter de telle manière qu’on puisse effectivement considérer qu’il a été dans la tombe pendant trois jours. L’argument principal pour le vendredi se trouve dans Mc. 15:42 qui précise que Jésus a été crucifié la «~veille du sabbat~». S’il s’agit bien du sabbat hebdomadaire, c’est à dire le samedi, alors la crucifixion a bien eu lieu un vendredi. Un autre argument en faveur du vendredi se fonde sur des versets tels que Mt. 16:21 et Lu. 9:22 où Jésus enseigne qu’il ressuscitera le troisième jour, ce qui suppose qu’il ne restera pas trois jours et trois nuits entiers dans la tombe. Plusieurs traducteurs utilisent l’expression «~le troisième jour~», mais pas tous. Cependant, aucun d’eux ne conteste la manière de traduire ces versets. Dans Mc. 8:31, il est bien dit que Jésus sera ressuscité «~après~» trois jours.  Le débat sur le jeudi se construit sur celui du vendredi en concluant qu’il y a trop d’événements, selon ses défenseurs, qui se passent entre l’ensevelissement du Christ et dimanche matin, pour que tout se soit déroulé entre vendredi et dimanche matin. Il faut signaler qu’il est particulièrement problématique que le seul jour plein entre vendredi et dimanche soit le samedi. Un jour de plus ou deux résolvent ce problème. L’hypothèse du mercredi avance qu’il y avait deux sabbats cette semaine-là. Après le premier sabbat (celui qui débute le soir de la crucifixion, Mc. 15:42 ; Lu. 23:52-54), les femmes sont allées acheter les aromates. Notez bien qu’elles les ont achetées après le sabbat (Mc. 16:1). Dans cette hypothèse du mercredi, ce premier sabbat est la Pâque (cf. Lé. 16:29-31 ; Lé. 23:24-32,39) car les jours très saints sont aussi appelés sabbats. Le second sabbat de cette semaine était le sabbat hebdomadaire classique, le samedi. Notez que dans Lu. 23:56, les femmes qui avaient acheté les aromates après le premier sabbat s’en retournèrent, préparèrent les aromates puis «~se reposèrent durant le sabbat~» (Lu. 23:56). On ne peut pas imaginer qu’elles ont acheté les aromates après le sabbat et qu’elles les ont préparées avant le sabbat que s’il y a eu deux sabbats cette semaine-là. Avec l’hypothèse des deux sabbats, si le Messie a été crucifié un jeudi, alors le jour très saint (la Pâque) aurait débuté au coucher du soleil le jeudi et pris fin le vendredi au coucher du soleil – juste au début du sabbat hebdomadaire – le samedi. Acheter les aromates après le premier sabbat signifierait alors en faire l’acquisition le samedi, en violation des lois du sabbat.  L’hypothèse du mercredi est la seule qui corrobore les récits bibliques des femmes et des aromates et confirme la prophétie du Seigneur en Mt. 12:40. Le Messie a été arrêté à Gethsémané le mardi soir selon le calendrier romain et le mercredi selon le calendrier hébraïque. Le premier sabbat était un jour très saint, celui de Pâque (Mt. 26 ; Mc. 14 ; Lu. 22), un jeudi selon le calendrier hébraïque. Les femmes achetèrent les aromates le vendredi et s’en retournèrent les préparer le jour même. Elles se sont reposées le samedi, qui était le sabbat hebdomadaire, et ont enfin apporté les aromates au tombeau tôt le dimanche matin. Jésus a été enseveli au moment du coucher du soleil le mercredi, ce qui est le début du jeudi selon le calendrier juif. En usant de ce calendrier, nous avons~: - le jeudi nuit (première nuit), jeudi jour (premier jour) - le vendredi nuit (deuxième nuit), vendredi jour (deuxième jour) - le samedi nuit (troisième nuit), samedi jour (troisième jour). Nous ne savons pas exactement à quelle heure Jésus est ressuscité, mais sous savons que ce fut avant le lever du soleil du dimanche. En effet, Jn. 20:1 nous apprend que Marie de Magdala vint au tombeau «~alors qu’il faisait encore sombre~». Ainsi, Jésus serait ressuscité juste après le coucher du soleil du samedi soir, ce qui correspond au premier jour de la semaine pour les juifs.}} apparut d’abord à Marie de Madgala, de laquelle il avait chassé sept démons.
\VS{10}Elle alla l’annoncer à ceux qui avaient été avec lui, et qui étaient dans le deuil et pleuraient.
\VS{11}Mais quand ils entendirent qu'il était vivant, et qu'elle l'avait vu, ils ne la crurent point.
\VS{12}Après cela, il se montra sous une autre forme à deux d'entre eux, qui étaient en chemin pour aller à la campagne.
\VS{13}Ils revinrent l’annoncer aux autres, mais ils ne les crurent pas non plus.
\VS{14}Enfin, il se montra aux onze, qui étaient assis ensemble, et il leur reprocha leur incrédulité et leur dureté de cœur, parce ce qu'ils n'avaient pas cru ceux qui l'avaient vu ressuscité.
\TextTitle{[Nouvelle mission aux onze apôtres]
\\(Mt. 28:16-20 ; Lu. 24:46-48 ; Jn. 17:18 ; 20:21 ; Ac. 1:8)}
\VS{15}Puis il leur dit : Allez par tout le monde, et prêchez l'Evangile à toute créature.
\VS{16}Celui qui croira et qui sera baptisé, sera sauvé ; mais celui qui ne croira pas sera condamné.
\VS{17}Voici les miracles qui accompagneront ceux qui auront cru : Ils chasseront les démons en mon Nom ; ils parleront de nouvelles langues ;
\VS{18}ils saisiront les serpents avec la main, et s’ils boivent quelque breuvage mortel, il ne leur fera point de mal ; ils imposeront les mains aux malades, et les malades seront guéris.
\TextTitle{[Jésus enlevé au ciel]
\\(Lu. 24:49-53 ; Ac. 1:9-11)}
\VS{19}Le Seigneur, après leur avoir parlé de la sorte, fut enlevé au ciel, et il s'assit à la droite de Dieu.
\VS{20}Et ils s’en allèrent prêcher partout. Le Seigneur travaillait avec eux, et confirmait la parole par les miracles qui l'accompagnaient.
\PPE{}
\end{multicols}

%\clearpage\ShortTitle{Luc}\BookTitle{Luc}\BFont
\noindent\hrulefill
{\footnotesize
\textit{
\bigskip
{\centering{}
\\Auteur : Luc
\\(Gr. : Loukas)
\\Signifie : Qui donne la lumière
\\Thème : Jésus le Fils de l'homme
\\Date de rédaction : Env. 60 ap. J.-C.\\}
}
%\bigskip
\textit{
\\D'origine grecque, Luc fut l'auteur de l'évangile éponyme et du livre des « Actes des apôtres ». Celui que Paul appelait le « médecin bien-aimé », et qui fut son compagnon d'œuvre, avait entrepris des investigations visant à narrer avec exactitude la vie terrestre de Jésus-Christ dont il était devenu le disciple, probablement à la suite d'une prédication de Paul. Adressés initialement à Théophile, Luc était loin de penser que ses écrits constitueraient avec le temps une véritable richesse pour l'Eglise et pour le monde.
%\bigskip
\\L'évangile de Luc présente l'humanité parfaite de Jésus, sa compassion et sa miséricorde à l'égard des plus faibles. Rédigé avec rigueur et soin, il retrace le parcours du Fils de l'homme, de sa naissance à son adolescence, puis de sa mort à sa résurrection, et enfin son ascension. Il souligne aussi sa vie de prière et son fardeau pour le salut de l'homme. Par ailleurs, il fait ressortir la manière dont les femmes ont assisté Jésus par leurs biens durant son ministère.
%\bigskip
\\Fruit de recherches minutieuses, le récit de Luc présente certaines similitudes avec ceux de Matthieu et Marc, mais il est le seul à relater la célèbre parabole du fils prodigue, profonde représentation de l'amour du Père.\bigskip
}
}
\par\nobreak\noindent\hrulefill
\begin{multicols}{2}
\Chap{1}
\TextTitle{Introduction}
\VerseOne{}Parce que plusieurs se sont appliqués à mettre par ordre un récit des évènements qui ont été pleinement certifiés parmi nous,
\VS{2}suivant ce que nous ont transmis ceux qui ont été des témoins oculaires dès le commencement et sont devenus des ministres de la parole,
\VS{3}Il m'a aussi semblé bon, après avoir examiné exactement toutes choses depuis le commencement jusqu'à la fin, très excellent Théophile, de te les mettre en ordre par écrit,
\VS{4}afin que tu connaisses la certitude des choses dont tu as été informé.
\TextTitle{Annonce de la naissance de Jean-Baptiste}
\VS{5}Au temps d'Hérode, roi de Judée, il y avait un sacrificateur nommé Zacharie, de la classe d'Abia ; et sa femme était d'entre les filles d'Aaron, et s'appelait Elisabeth.
\VS{6}Et ils étaient tous deux justes devant Dieu, marchant dans tous les commandements, et dans {toutes } les ordonnances du Seigneur, sans reproche.
\VS{7}Et ils n'avaient point d'enfants, parce qu'Elisabeth était stérile, et qu'ils étaient fort avancés en âge.
\VS{8}Or il arriva que comme Zacharie exerçait la sacrificature devant Dieu, selon le tour de sa classe, il fut appelé par le sort,
\VS{9}selon la coutume d'exercer le sacerdoce, à entrer dans le temple du Seigneur pour offrir le parfum.
\VS{10}Toute la multitude du peuple était dehors en prière, à l'heure du parfum.
\VS{11}Et l'ange du Seigneur lui apparut, et se tint debout à droite de l'autel des parfums.
\VS{12}Zacharie fut troublé quand il le vit, et il fut saisi de crainte.
\VS{13}Mais l'ange lui dit : Zacharie, ne crains point ; car ta prière est exaucée. Et Elisabeth, ta femme, t'enfantera un fils, et tu lui donneras le nom de Jean.
\VS{14}Et il sera pour toi le sujet d'une grande joie et d'allégresse, et plusieurs se réjouiront de sa naissance.
\VS{15}Car il sera grand devant le Seigneur. Et il ne boira ni vin, ni boisson forte et il sera rempli du Saint-Esprit dès le ventre de sa mère.
\VS{16}Et il ramènera plusieurs des enfants d'Israël au Seigneur, leur Dieu.
\VS{17}Car il marchera devant lui animé de l'esprit et de la puissance d'Elie, pour ramener les cœurs des pères vers les enfants\FTNT{Mal. 4:6.}, et les rebelles à la sagesse des justes, pour préparer au Seigneur un peuple bien disposé.
\VS{18}Alors Zacharie dit à l'ange : A quoi reconnaîtrai-je cela ? Car je suis vieux, et ma femme est fort âgée.
\VS{19}L'ange répondant lui dit : Je suis Gabriel, je me tiens devant Dieu, et j'ai été envoyé pour te parler, et pour t'annoncer cette bonne nouvelle.
\VS{20}Et voici, tu seras muet, et tu ne pourras point parler jusqu'au jour où ces choses arriveront, parce que tu n'as pas cru à mes paroles qui s'accompliront en leur temps.
\VS{21}Or le peuple attendait Zacharie, et on s'étonnait de ce qu'il tardait tant dans le temple.
\VS{22}Mais quand il fut sorti, il ne pouvait pas leur parler, et ils comprirent qu'il avait eu une vision dans le temple ; car il leur faisait des signes et il resta muet.
\VS{23}Et il arriva que quand les jours de son ministère furent achevés, il retourna dans sa maison.
\VS{24}Et après ces jours-là, Elisabeth sa femme conçut, et elle se cacha l’espace de cinq mois, en disant :
\VS{25}Certes, le Seigneur en a agi avec moi ainsi aux jours qu’il m’a regardée pour ôter mon opprobre d’entre les hommes.
\TextTitle{Annonce de la naissance de Jésus-Christ}
\VS{26}Or au sixième mois, l'ange Gabriel fut envoyé par Dieu dans une ville de Galilée, appelée Nazareth,
\VS{27}vers une vierge fiancée à un homme nommé Joseph, qui était de la maison de David. Et le nom de la vierge était Marie.
\VS{28}Et l'ange étant entré dans le lieu où elle était, lui dit : Je te salue, toi à qui une grâce a été faite. Le Seigneur est avec toi ; tu es bénie parmi les femmes.
\VS{29}Troublée par cette parole, Marie se demandait ce que pouvait signifier une telle salutation.
\VS{30}L'ange lui dit : Marie, ne crains point ; car tu as trouvé grâce devant Dieu.
\VS{31}Et voici, tu concevras en ton ventre, et tu enfanteras un fils, et tu lui donneras le Nom de JESUS.
\VS{32}Il sera grand, et sera appelé le Fils du Très-Haut, et le Seigneur Dieu lui donnera le trône de David, son père.
\VS{33}Il régnera sur la maison de Jacob éternellement, et son règne n'aura pas de fin.
\TextTitle{Naissance miraculeuse de Jésus-Christ}
\VS{34}Alors Marie dit à l'ange : Comment cela se fera-t-il, puisque je ne connais point d'homme ?
\VS{35}L'ange lui répondit et dit : Le Saint-Esprit viendra sur toi, et la puissance du Très-Haut te couvrira de son ombre. C'est pourquoi, le Saint qui naîtra de toi sera appelé Fils de Dieu.
\VS{36}Voici, Elizabeth, ta cousine, a conçu elle aussi un fils en sa vieillesse, celle qui était appelée stérile est dans son sixième mois de grossesse.
\VS{37}Car rien n'est impossible à Dieu.
\VS{38}Et Marie dit : Voici la servante du Seigneur, qu'il me soit fait selon ta parole ! Et l'ange la quitta.
\TextTitle{Marie se rend chez Elisabeth}
\VS{39}Dans ce même temps, Marie se leva, et s'en alla en hâte au pays des montagnes dans une ville de Juda.
\VS{40}Elle entra dans la maison de Zacharie, et salua Elisabeth.
\VS{41}Et il arriva, comme Elisabeth entendait la salutation de Marie, que le petit enfant tressaillit dans son ventre ; et Elisabeth fut remplie de l’Esprit Saint,
\VS{42}Elle s'écria d'une voix forte et dit : Tu es bénie entre les femmes, et béni est le fruit de ton ventre.
\VS{43}Comment m'est-il accordé que la mère de mon Seigneur vienne vers moi ?
\VS{44}Car voici, dès que la voix de ta salutation est parvenue à mes oreilles, le petit enfant a tressailli de joie dans mon ventre.
\VS{45}Heureuse celle qui a cru, parce que les choses qui lui ont été dites par le Seigneur auront leur accomplissement.
\TextTitle{Cantique de Marie\FTNTT{Cp. 1 S. 2:1-10}}
\VS{46}Alors Marie dit : Mon âme magnifie le Seigneur,
\VS{47}et mon esprit se réjouit en Dieu, mon Sauveur.
\VS{48}Car il a jeté les yeux sur la bassesse de sa servante. Voici, certes désormais toutes les générations me diront bienheureuse,
\VS{49}parce que le Tout-Puissant a fait pour moi de grandes choses, et son Nom est Saint.
\VS{50}Et sa miséricorde s'étend de génération en génération en faveur de ceux qui le craignent.
\VS{51}Il a puissamment opéré par son bras. Il a dissipé les desseins que les orgueilleux formaient dans leurs cœurs.
\VS{52}Il a renversé de dessus leurs trônes les puissants, et il a élevé les petits.
\VS{53}Il a rassasié de biens les affamé, il a renvoyé les riches à vide.
\VS{54}Il a pris sous sa protection Israël, son serviteur, et il s'est souvenu de sa miséricorde,
\VS{55}comme il l'avait dit à nos pères, envers Abraham et sa postérité à jamais.
\VS{56}Marie demeura avec elle environ trois mois. Puis elle retourna dans sa maison.
\TextTitle{Naissance de Jean}
\VS{57}Le temps où Elisabeth devait accoucher arriva, et elle enfanta un fils.
\VS{58}Ses voisins et ses parents ayant appris que le Seigneur avait fait éclater sa miséricorde envers elle, s'en réjouissaient avec elle.
\VS{59}Et il arriva qu'au huitième jour, ils vinrent pour circoncire le petit enfant, et ils l'appelaient Zacharie, du nom de son père.
\VS{60}Mais sa mère prit la parole, et dit : Non, mais il sera appelé Jean.
\VS{61}Et ils lui dirent : Il n'y a personne dans ta parenté qui soit appelé de ce nom.
\VS{62}Alors ils firent signe à son père pour savoir comment il voulait qu'on l'appelle.
\VS{63}Et Zacharie ayant demandé des tablettes, écrivit : Jean est son nom. Et tous furent dans l'étonnement.
\VS{64}Au même instant, sa bouche s'ouvrit et sa langue se délia, et il parlait, bénissant Dieu.
\VS{65}Tous ses voisins furent saisis de crainte et toutes ces choses furent divulguées dans tout le pays des montagnes de Judée.
\VS{66}Tous ceux qui les apprirent les gardèrent dans leur cœur, disant : Que sera donc cet enfant ? Et la main du Seigneur était avec lui.
\TextTitle{Cantique de Zacharie}
\VS{67}Alors Zacharie, son père, fut rempli du Saint-Esprit, et il prophétisa en ces mots :
\VS{68}Béni soit le Seigneur, le Dieu d'Israël, de ce qu'il a visité et délivré son peuple
\VS{69}et de ce qu'il nous a suscité un puissant Sauveur dans la maison de David, son serviteur,
\VS{70}selon ce qu'il avait dit par la bouche de ses saints prophètes des temps anciens :
\VS{71}Un Sauveur qui nous délivre de nos ennemis et de la main de tous ceux qui nous haïssent !
\VS{72}C'est ainsi qu'il manifeste sa miséricorde envers nos pères, et se souvient de sa sainte alliance.
\VS{73}Selon le serment par lequel il avait juré à Abraham notre père,
\VS{74}de nous permettre, après que nous serions délivrés de la main de nos ennemis, de le servir sans crainte,
\VS{75}en marchant devant lui dans la sainteté et dans la justice tous les jours de notre vie.
\VS{76}Et toi, petit enfant, tu seras appelé prophète du Très-Haut ; car tu marcheras devant la face du Seigneur, pour préparer ses voies,
\VS{77}afin de donner à son peuple la connaissance du salut, par la rémission de leurs péchés,
\VS{78}grâce aux entrailles de la miséricorde de notre Dieu, en vertu de laquelle le Soleil Levant nous a visités d'en haut,
\VS{79}pour éclairer ceux qui sont assis dans les ténèbres et dans l'ombre de la mort, et pour conduire nos pas dans le chemin de la paix.
\VS{80}Or, le petit enfant croissait, et se fortifiait en esprit. Et il demeura dans les déserts jusqu'au jour où il se présenta à Israël.
\TextTitle{[Naissance de Jésus à Bethléhem]
\\(Mt. 1:18-25 ; 2:1 ; cp. Jn. 1:14}
\Chap{2}
\VerseOne{}En ces jours-là fut publié un édit par César Auguste, ordonnant un recensement de toute la terre.
\VS{2}Ce premier recensement eut lieu pendant que Quirinius était gouverneur de Syrie.
\VS{3}Ainsi, tous allaient pour s'inscrire, chacun dans sa ville.
\VS{4}Joseph aussi monta de Galilée en Judée, de la ville de Nazareth, la ville de David, appelée Bethléhem, parce qu'il était de la maison et de la famille de David ;
\VS{5}afin de se faire inscrire avec Marie, sa fiancée, qui était enceinte.
\VS{6}Pendant qu'ils étaient là, le temps où Marie devait accoucher arriva,
\VS{7}et elle enfanta son fils, premier-né et elle l'emmaillota et le coucha dans une crèche, parce qu'il n'y avait point de place pour eux dans l'hôtellerie.
\TextTitle{L'Ange du Seigneur annonce la naissance de Jésus}
\VS{8}Et il y avait dans cette même contrée des bergers qui couchaient dans les champs, et qui gardaient leur troupeau pendant les veilles de la nuit.
\VS{9}Et voici, l'Ange du Seigneur survint vers eux, et la gloire du Seigneur resplendit autour d'eux, et ils furent saisis d'une grande peur.
\VS{10}Mais l'Ange leur dit : Ne craignez point ; car, voici je vous annonce une bonne nouvelle qui sera un sujet de joie pour tout le peuple :
\VS{11}C'est qu'aujourd'hui, dans la ville de David, vous est né le Sauveur, qui est le Christ, le Seigneur.
\VS{12}Et voici à quel signe vous le reconnaîtrez : Vous trouverez le petit enfant emmailloté, et couché dans une crèche.
\VS{13}Et aussitôt il se joignit à l'Ange une multitude de l'armée céleste, louant Dieu et disant :
\VS{14}Gloire soit à Dieu dans les lieux très-hauts, que la paix soit sur la terre et la bonne volonté dans les hommes !
\TextTitle{Les bergers de Bethléhem}
\VS{15}Et il arriva qu'après que les anges s’en furent allés d’avec eux au ciel, les bergers se dirent les uns aux autres : Allons donc jusqu'à Bethléhem, et voyons cette chose qui est arrivée, ce que le Seigneur nous a fait connaître.
\VS{16}Ils y allèrent donc en hâte, et ils trouvèrent Marie et Joseph, et le petit enfant couché dans une crèche.
\VS{17}Après l'avoir vu, ils divulguèrent ce qui leur avait été dit au sujet de ce petit enfant.
\VS{18}Tous ceux qui les entendirent furent dans l'étonnement de ce que leur disaient les bergers.
\VS{19}Et Marie gardait soigneusement toutes ces choses, et les repassait dans son esprit.
\VS{20}Puis les bergers s'en retournèrent, glorifiant et louant Dieu pour tout ce qu'ils avaient entendu et vu, et qui était conforme à ce qui leur avait été annoncé.
\TextTitle{Jésus circoncis et présenté au temple de Jérusalem\FTNTT{Cp. Ex. 13:12,15}}
\VS{21}Et quand les huit jours furent accomplis pour circoncire l’enfant, on lui donna le Nom de Jésus, nom qu'avait indiqué l'Ange avant qu'il soit conçu dans le sein de sa mère.
\VS{22}Et quand les jours de la purification\FTNT{Lé : 12:2-6.} de Marie furent accomplis selon la loi de Moïse, Joseph et Marie le portèrent à Jérusalem, pour le présenter au Seigneur,
\VS{23}selon ce qui est écrit dans la loi du Seigneur : Tout mâle premier-né sera appelé Saint au Seigneur\FTNT{Ex. 13:2 ; Ex. 13:12 ; No. 3:13 ; No. 8:17.} ;
\VS{24}et pour offrir en sacrifice deux tourterelles ou deux jeunes pigeons comme cela est prescrit dans la loi du Seigneur\FTNT{Lé. 12:8.}.
\TextTitle{Adoration de Siméon et sa prophétie}
\VS{25}Et voici, il y avait à Jérusalem un homme appelé Siméon. Et cet homme était juste et pieux, il attendait la consolation d'Israël, et le Saint-Esprit était sur lui.
\VS{26}Il avait été averti divinement par le Saint-Esprit qu'il ne mourrait point avant d'avoir vu le Christ du Seigneur.
\VS{27}Il vint au temple, poussé par l'Esprit. Et comme les parents apportaient dans le temple l'enfant Jésus, pour accomplir à son égard ce qu'ordonnait la loi,
\VS{28}il le prit dans ses bras, bénit Dieu, et dit :
\VS{29}Seigneur, tu laisses maintenant ton serviteur s'en aller en paix selon ta parole.
\VS{30}Car mes yeux ont vu ton salut.
\VS{31}Lequel tu as préparé devant la face de tous les peuples.
\VS{32}La lumière pour éclairer les nations ; et pour être la gloire de ton peuple d'Israël.
\VS{33}Joseph et sa mère s'étonnaient des choses qui étaient dites de lui.
\VS{34}Siméon le bénit, et dit à Marie, sa mère : Voici, cet enfant est destiné à être une occasion de chute et de relèvement de beaucoup en Israël, et à devenir un signe qui provoquera la contradiction,
\VS{35}en sorte que les pensées de beaucoup de cœurs seront découvertes. Et pour toi, une épée te transpercera l'âme.
\TextTitle{Anne témoigne du Messie}
\VS{36}Il y avait aussi Anne, la prophétesse, fille de Phanuel de la tribu d'Aser, qui était déjà avancée en âge, et qui avait vécu avec son mari sept ans depuis sa virginité.
\VS{37}Restée veuve, et âgée d'environ quatre-vingt-quatre ans, elle ne quittait pas le temple, et elle servait Dieu nuit et jour dans le jeûne et dans les prières.
\VS{38}Etant arrivée à cette heure, elle louait aussi Dieu, et parlait de lui à tous ceux qui attendaient la délivrance de Jérusalem.
\TextTitle{Retour à Nazareth\FTNTT{Suite aux évènements de Mt. 2.}}
\VS{39}Et quand ils eurent accompli tout ce qui est ordonné par la loi du Seigneur, ils s'en retournèrent en Galilée, à Nazareth, leur ville.
\VS{40}Et le petit enfant croissait et se fortifiait en esprit. Il était rempli de sagesse, et la grâce de Dieu était sur lui.
\TextTitle{Jésus assis dans le temple de Jérusalem au milieu des docteurs}
\VS{41}Ses parents allaient tous les ans à Jérusalem, à la fête de Pâque.
\VS{42}Lorsqu'il fut âgé de douze ans, ses parents montèrent à Jérusalem selon la coutume de la fête.
\VS{43}Puis, quand les jours furent écoulés, et qu'ils s'en retournèrent, l'enfant Jésus resta à Jérusalem. Et son père et sa mère ne s'en aperçurent point.
\VS{44}Mais croyant qu'il était avec leurs compagnons de voyage, ils marchèrent une journée, puis ils le cherchèrent parmi leurs parents et parmis leur connaissance.
\VS{45}Et ne le trouvant point, ils retournèrent à Jérusalem, pour le chercher.
\VS{46}Or il arriva que trois jours après, ils le trouvèrent dans le temple, assis au milieu des docteurs, les écoutant, et les interrogeant.
\VS{47}Tous ceux qui l'entendaient s'étonnaient de sa sagesse et de ses réponses.
\VS{48}Quand ses parents le virent, ils furent saisis d'étonnement, et sa mère lui dit : Mon enfant, pourquoi nous as-tu fait ainsi ? Voici, ton père et moi te cherchions avec angoisse.
\VS{49}Et il leur dit : Pourquoi me cherchiez-vous ? Ne saviez-vous pas qu'il faut que je m'occupe des affaires de mon Père ?
\VS{50}Mais ils ne comprirent point ce qu'il leur disait.
\VS{51}Alors il descendit avec eux, et vint à Nazareth ; et il leur était soumis. Et sa mère gardait toutes ces paroles dans son cœur.
\TextTitle{Jésus grandit en sagesse, en stature et en grâce}
\VS{52}Et Jésus croissait en sagesse, en stature, et en grâce, au près de Dieu et devant les hommes.
\Chap{3}
\TextTitle{Ministère de Jean-Baptiste\FTNTT{Mt. 3:1-12 ; Mc. 1:1-8 ; Jn. 1:6-8,15-37}}
\VerseOne{}La quinzième année du règne de Tibère César, lorsque Ponce Pilate était gouverneur de la Judée, Hérode tétrarque de la Galilée, et son frère Philippe tétrarque de l'Iturée et du territoire de la Trachonite, et Lysanias tétrarque de l'Abilène,
\VS{2}et du temps des souverains sacrificateurs Anne et Caïphe, la parole de Dieu fut adressée à Jean, fils de Zacharie, dans le désert.
\VS{3}Et il alla dans tout le pays des environs du Jourdain, prêchant le baptême de repentance, pour la rémission des péchés,
\VS{4}comme il est écrit dans le livre des paroles d'Esaïe, le prophète disant : C'est la voix de celui qui crie dans le désert : Préparez le chemin du Seigneur, aplanissez ses sentiers.
\VS{5}Toute vallée sera comblée, et toute montagne et toute colline seront abaissées, et ce qui est tortueux sera redressé, et les chemins raboteux seront aplanis.
\VS{6}Et toute chair verra le salut de Dieu\FTNT{Es. 40:3-5.}.
\VS{7}Il disait donc à ceux qui venaient en foule pour être baptisés par lui : Races de vipères, qui vous a appris à fuir la colère à venir ?
\VS{8}Produisez donc des fruits dignes de la repentance, et ne vous mettez point à dire en vous-mêmes : Nous avons Abraham pour père. Car je vous dis que Dieu peut faire naître, même de ces pierres, des enfants à Abraham.
\VS{9}Or la cognée est déjà mise à la racine des arbres ; tout arbre donc qui ne produit pas de bon fruit, sera coupé, et jeté au feu.
\VS{10}Alors la foule l'interrogeait, disant : Que ferons-nous donc ?
\VS{11}Et il répondit, et leur dit : Que celui qui a deux tuniques partage avec celui qui n'en a point ; et que celui qui a de quoi manger en fasse de même.
\VS{12}Il vint aussi à lui des publicains pour être baptisés, et ils lui dirent : Maître, que ferons-nous ?
\VS{13}Et il leur dit : N'exigez rien au-delà de ce qui vous a été ordonné.
\VS{14}Des soldats l'interrogèrent aussi, disant : Et nous, que ferons-nous ? Et Il leur répondit : Ne commettez ni extorsion ni fraude envers personne, mais contentez-vous de votre solde.
\VS{15}Et comme le peuple était dans l'attente, et que tous se demandaient dans leurs cœurs si Jean n'était pas le Christ,
\VS{16}Jean prit la parole, et dit à tous : Moi, je vous baptise d'eau ; mais il vient, celui qui est plus puissant que moi, et je ne suis pas digne de délier la courroie de ses souliers. Lui, il vous baptisera du Saint-Esprit et de feu.
\VS{17}Il a son van à la main ; il nettoiera entièrement son aire, et amassera le froment dans son grenier, mais il brûlera la paille dans un feu qui ne s'éteint point.
\VS{18}Et faisant aussi plusieurs autres exhortations, il évangélisait le peuple.
\VS{19}Mais Hérode le tétrarque, étant repris par Jean au sujet d'Hérodias, femme de Philippe son frère, et à cause de toutes les choses méchantes qu'il faisait,
\VS{20}ajouta encore à toutes les autres celle de mettre Jean en prison.
\TextTitle{Baptême de Jésus-Christ\FTNTT{Mc. 1:9-11; cp. Jn. 1:31-34}}
\VS{21}Tout le peuple se faisait baptiser, Jésus aussi fut baptisé, et pendant qu'il priait, le ciel s'ouvrit,
\VS{22}et le Saint-Esprit descendit sur lui sous une forme corporelle, comme celle d'une colombe. Et une voix fit entendre du ciel ces paroles : Tu es mon Fils bien-aimé, en toi j'ai trouvé mon plaisir.
\TextTitle{Généalogie de Jésus-Christ\FTNTT{v. 31 ; Mt. 1:1-16}}
\VS{23}Jésus avait environ trente ans, lorsqu'il commença son ministère, étant comme on l'estimait, fils de Joseph, fils d'Héli,
\VS{24}fils de Matthat, fils de Lévi, fils de Melchi, fils de Jannaï, fils de Joseph,
\VS{25}fils de Mattathias, fils d'Amos, fils de Nahum, fils d'Esli, fils de Naggaï,
\VS{26}fils de Maath, fils de Mattathias, fils de Sémeï, fils de Josech, fils de Joda,
\VS{27}fils de Joanan, fils de Rhésa, fils de Zorobabel, fils de Salathiel, fils de Néri,
\VS{28}fils de Melchi, fils d'Addi, fils de Kosam, fils d'Elmadam, fils d'Er,
\VS{29}fils de Jésus, fils d'Eliézer, fils de Jorim, fils de Matthat, fils de Lévi,
\VS{30}fils de Siméon, fils de Juda, fils de Joseph, fils de Jonam, fils d'Eliakim,
\VS{31}fils de Méléa, fils de Menna, fils de Matthata, fils de Nathan, fils de David,
\VS{32}fils d'Isaï, fils d'Obed, fils de Booz, fils de Salmon, fils de Naasson,
\VS{33}fils d'Aminadab, fils d’Admin, fils d'Aram, fils d'Esrom, fils de Pérets, fils de Juda,
\VS{34}fils de Jacob, fils d'Isaac, fils d'Abraham, fils de Thara, fils de Nachor,
\VS{35}fils de Seruch, fils de Ragau, fils de Phalek, fils d'Eber, fils de Sala,
\VS{36}fils de Kaïnam, fils d’Arphaxad, fils de Sem, fils de Noé, fils de Lamech,
\VS{37}fils de Mathusala, fils d'Hénoc, fils de Jared, fils de Maléléel, fils de Kaïnan,
\VS{38}fils d'Enos, fils de Seth, fils d'Adam, fils de Dieu.
\Chap{4}
\TextTitle{Tentation de Jésus-Christ\FTNTT{Mt. 4:1 ; Mc. 1:12-13 ; cp. Ge. 3:6 ; 1 Jn. 2:16}}
\VerseOne{}Jésus, rempli du Saint-Esprit, revint du Jourdain, et il fut conduit par l'Esprit dans le désert,
\VS{2}où il fut tenté par le diable quarante jours. Et il ne mangea rien durant ces jours-là, et après qu'ils furent écoulés, il eut faim.
\VS{3}Le diable lui dit : Si tu es le Fils de Dieu, ordonne à cette pierre qu'elle devienne du pain.
\VS{4}Jésus lui répondit, en disant : Il est écrit que l'homme ne vivra pas seulement de pain, mais de toute parole de Dieu\FTNT{De. 8:3.}.
\VS{5}Alors le diable l'emmena sur une haute montagne, et lui montra en un instant tous les royaumes de la terre,
\VS{6}et le diable lui dit : Je te donnerai toute cette puissance et leur gloire ; car elle m'a été donnée, et je la donne à qui je veux.
\VS{7}Si donc tu m'adores, elle sera à toi.
\VS{8}Jésus lui répondit : Va arrière de moi, Satan ! Car il est écrit : Tu adoreras le Seigneur ton Dieu, et tu le serviras lui seul\FTNT{De. 6:13.}.
\VS{9}Le diable le conduisit encore à Jérusalem, et le plaça sur le haut du temple, et lui dit : Si tu es le Fils de Dieu, jette-toi d'ici en bas.
\VS{10}Car il est écrit : Il ordonnera à ses anges à ton sujet, afin qu'ils te gardent;
\VS{11}et ils te porteront dans leurs mains, de peur que ton pied ne heurte contre une pierre\FTNT{Ps. 91:11-12.}.
\VS{12}Mais Jésus répondant, lui dit : Il est écrit : Tu ne tenteras pas le Seigneur, ton Dieu\FTNT{De. 6:16.}.
\VS{13}Après l'avoir tenté de toutes ces manières, le diable s'éloigna de lui pour un temps.
\TextTitle{Jésus-Christ retourne en Galilée\FTNTT{Mt. 4:12-17 ; Mc. 1:14-15}}
\VS{14}Jésus retourna en Galilée dans la puissance de l'Esprit, et sa renommée se répandit dans tout le pays d'alentour.
\VS{15}Il enseignait dans leurs synagogues, et il était glorifié par tous.
\TextTitle{Jésus dans la synagogue de Nazareth le jour du sabbat\FTNTT{cp. Mt. 13:54-58 ; Mc. 6:1-6}}
\VS{16}Il se rendit à Nazareth, où il avait été élevé, et selon sa coutume, il entra dans la synagogue le jour du sabbat et il se leva pour faire la lecture,
\VS{17}et on lui donna le livre du prophète Esaïe et l'ayant déroulé, il trouva le passage où il est écrit :
\VS{18}L'Esprit du Seigneur est sur moi, parce qu'il m'a oint pour évangéliser les pauvres ; il m'a envoyé pour guérir ceux qui ont le cœur brisé,
\VS{19}pour proclamer aux captifs la délivrance, et aux aveugles le recouvrement de la vue ; pour mettre en liberté les opprimés ; pour publier une année de grâce du Seigneur\FTNT{Es. 61:1-2.}.
\VS{20}Ensuite, il roula le livre, le rendit au serviteur, et s'assit. Les yeux de tous ceux qui étaient dans la synagogue étaient fixés sur lui.
\VS{21}Alors il commença à leur dire : Aujourd'hui, cette parole de l'Ecriture que vous venez d'entendre, est accomplie.
\VS{22}Et tous lui rendaient témoignage, et s'étonnaient des paroles pleines de grâce qui sortaient de sa bouche ; et ils disaient : Celui-ci n'est-il pas le fils de Joseph ?
\VS{23}Et il leur dit : Assurément vous me direz ce proverbe : Médecin, guéris-toi toi-même. Et fais ici, dans ton pays, tout ce que nous avons appris que tu as fait à Capernaüm.
\VS{24}Mais il leur dit : En vérité je vous dis qu’aucun prophète n'est reçu dans son pays.
\VS{25}Je vous le dis en vérité : Il y avait plusieurs veuves en Israël, du temps d'Elie, lorsque le ciel fut fermé trois ans et six mois et qu'il y eut une grande famine dans tout le pays ;
\VS{26}toutefois Elie ne fut envoyé vers aucune d'elles, mais seulement vers une femme veuve à Sarepta, dans le pays de Sidon.
\VS{27}Il y avait aussi plusieurs lépreux en Israël du temps d'Elisée, le prophète, toutefois aucun d'eux ne fut purifié, si ce n'est Naaman, le Syrien.
\VS{28}Ils furent tous remplis de colère dans la synagogue lorsqu'ils entendirent ces choses.
\VS{29}Et s'étant levés, ils le chassèrent hors de la ville, et le menèrent jusqu'au bord de la montagne sur laquelle leur ville était bâtie, pour le jeter du haut en bas.
\VS{30}Mais il passa au milieu d'eux, et s'en alla.
\TextTitle{Jésus guérit un possédé\FTNTT{Mc. 1:21-28}}
\VS{31}Il descendit à Capernaüm, ville de Galilée, et il les enseignait les jours de sabbat.
\VS{32}Ils étaient frappés de sa doctrine ; car il parlait avec autorité.
\VS{33}Il y avait dans la synagogue un homme qui avait un esprit de démon impur, et qui s'écria d'une voix forte,
\VS{34}en disant : Ah ! Qu'y a-t-il entre nous et toi, Jésus de Nazareth ? Es-tu venu pour nous détruire ? Je sais qui tu es, le Saint de Dieu.
\VS{35}Jésus le menaça, en lui disant : Tais-toi, et sors de cet homme. Et le démon, l'ayant jeté avec impétuosité au milieu de l'assemblée, sortit de cet homme, sans lui faire aucun mal.
\VS{36}Et tous furent saisis de stupeur, et ils parlaient entre eux, et disaient : Quelle est cette parole ? Il commande avec autorité et puissance aux esprits impurs, et ils sortent ?
\VS{37}Et sa renommée se répandit dans tous les lieux d'alentour.
\TextTitle{Guérison de la belle-mère de Pierre et de plusieurs malades\FTNTT{Mt. 8:14-17 ; Mc. 1:29-34}}
\VS{38}Et quand Jésus se fut levé de la synagogue, et il se rendit à la maison de Simon, et la belle-mère de Simon avait une violente fièvre, et ils le prièrent en sa faveur.
\VS{39}Et s'étant penché sur elle, il menaça la fièvre, et la fièvre la quitta. A l'instant elle se leva, et les servit.
\VS{40}Et après le coucher du soleil, tous ceux qui avaient des malades atteints de diverses maladies, les lui amenèrent. Il imposa les mains à chacun d'eux, et il les guérit.
\VS{41}Les démons aussi sortirent de beaucoup de personnes, en criant et en disant : Tu es le Christ, le Fils de Dieu. Mais il les menaçait fortement, et ne leur permettait pas de dire qu'ils savaient qu'il était le Christ.
\VS{42}Dès que le jour parut, il sortit et alla dans un lieu désert et une foule de gens se mirent à sa recherche, et arrivèrent jusqu'à lui et ils voulaient le retenir, afin qu'il ne les quittât point.
\VS{43}Mais il leur dit : Il faut que j'annonce aux autres villes l'Evangile du Royaume de Dieu, car c'est pour cela que j'ai été envoyé.
\VS{44}Et il prêchait dans les synagogues de la Galilée.
\Chap{5}
\TextTitle{Appel des premiers disciples\FTNTT{Mt. 4:18-22 ; Mc. 1:16-20 ; cp. Jn. 1:35-51 ; 21:1-8}}
\VerseOne{}Or il arriva, comme la foule se jetait toute sur lui pour entendre la parole de Dieu, qu’il se tenait sur le bord du lac de Génézareth.
\VS{2}Et voyant deux barques qui étaient au bord du lac, et dont les pêcheurs étaient descendus, et lavaient leurs rets, il monta dans l’une de ces barques, qui était à Simon.
\VS{3}Il monta dans l'une de ces barques, qui était à Simon, et il le pria de s'éloigner un peu de terre. Puis il s'assit, et de la barque il enseignait la foule.
\VS{4}Et quand il eut cessé de parler, il dit à Simon : Avance en pleine eau, et jetez vos filets pour pêcher.
\VS{5}Et Simon répondant, lui dit : Maître, nous avons travaillé toute la nuit, et nous n’avons rien pris ; toutefois à ta parole je jetterai les filets.
\VS{6}Et ayant fait cela, ils prirent une si grande quantité de poissons que leur filet se rompait.
\VS{7}Et ils firent signe à leurs compagnons qui étaient dans l’autre barque, de venir les aider ; et étant venus, ils remplirent les deux barques, tellement qu’elles s’enfonçaient.
\VS{8}Et quand Simon Pierre vit cela, il se jeta aux genoux de Jésus, en lui disant : Seigneur, retire-toi de moi ; car je suis un homme pécheur.
\VS{9}Parce que la frayeur l’avait saisi, lui et tous ceux qui étaient avec lui, à cause de la prise de poissons qu’ils venaient de faire ; de même que Jacques et Jean, fils de Zébédée, qui étaient compagnons de Simon.
\VS{10}Alors Jésus dit à Simon : Ne crains point ; désormais tu seras un pêcheur d'hommes vivants.
\VS{11}Et quand ils eurent ramené les barques à terre, ils abandonnèrent tout et le suivirent.
\TextTitle{Guérison d'un lépreux\FTNTT{Mt. 8:2-4 ; Mc. 1:40-45}}
\VS{12}Et il arriva, comme il était dans une des villes, voici un homme plein de lèpre, voyant Jésus, se jeta sur sa face et le supplia, disant : Seigneur, si tu veux, tu peux me rendre pur.
\VS{13}Jésus étendit la main, et le toucha, en disant : Je le veux, sois pur. Aussitôt la lèpre le quitta.
\VS{14}Et il lui commanda de ne le dire à personne, mais va, lui dit-il, et montre-toi sacrificateur, et offre pour ta purification ce que Moïse a commandé\FTNT{Lé. 13 et 14.}, pour leur servir de témoignage.
\VS{15}Et sa renommée se répandait de plus en plus, tellement que de grandes foules s’assemblaient pour l’entendre, et pour être guéries par lui de leurs maladies.
\VS{16}Mais il se tenait retiré dans les déserts, et priait.
\TextTitle{Guérison d'un paralytique\FTNTT{Mt. 9:2-8 ; Mc. 2:3-12}}
\VS{17}Un jour Jésus enseignait. Et des pharisiens et des docteurs de la loi étaient là assis, venus de tous les villages de la Galilée, et de la Judée et de Jérusalem ; et la puissance du Seigneur se manifestait par des guérisons.
\VS{18}Et voici des hommes qui portaient sur un lit un homme qui était paralytique, et ils cherchaient le moyen de le porter dans la maison, et de le mettre devant lui.
\VS{19}Comme ils ne savaient pas par où l'introduire, à cause de la foule, ils montèrent sur le toit, et ils le descendirent par une ouverture, avec son lit, au milieu de la foule, devant Jésus.
\VS{20}Voyant leur foi, il dit au paralytique : Homme, tes péchés te sont pardonnés.
\VS{21}Alors les scribes et les pharisiens commencèrent à raisonner en eux-mêmes, disant : Qui est celui-ci qui profère des blasphèmes ? Qui est-ce qui peut pardonner les péchés, si ce n'est Dieu seul ?
\VS{22}Mais Jésus, connaissant leurs pensées, prit la parole et leur dit : Pourquoi raisonnez-vous ainsi en vous-mêmes ?
\VS{23}Lequel est le plus aisé de dire : Tes péchés te sont pardonnés ; ou de dire : Lève-toi et marche ?
\VS{24}Or afin que vous sachiez que le Fils de l'homme a le pouvoir sur la terre de pardonner les péchés, il dit au paralytique : Je te l'ordonne, lève-toi, prends ton lit, et va dans ta maison.
\VS{25}Et à l'instant, le paralytique s'étant levé devant eux, prit le lit sur lequel il était couché, et s'en alla dans sa maison, glorifiant Dieu.
\VS{26}Ils furent tous saisis d'étonnement, et ils glorifiaient Dieu ; et étant remplis de crainte, ils disaient : certainement nous avons vu aujourd'hui des choses étranges.
\TextTitle{Appel de Lévi\FTNTT{Mt. 9:9 ; Mc. 2:13-14}}
\VS{27}Après cela, Jésus sortit, et il vit un publicain nommé Lévi, assis au bureau des péages, et il lui dit : Suis-moi.
\VS{28}Et abandonnant tout, il se leva, et le suivit.
\TextTitle{Appelle des pécheurs à la repentance\FTNTT{Mt. 9:10-15 ; Mc. 2:13-14}}
\VS{29}Et Lévi lui fit un grand festin dans sa maison ; et il y avait une grande foule de publicains et d’autres gens qui étaient avec eux à table.
\VS{30}Les scribes de ce lieu-là et les pharisiens, murmuraient contre ses disciples en disant : Pourquoi mangez-vous et buvez-vous avec les publicains et les gens de mauvaise vie ?
\VS{31}Mais Jésus, prenant la parole, leur dit : ceux qui sont en santé n’ont pas besoin de médecin, mais ceux qui se portent mal.
\VS{32}Je ne suis point venu appeler à la repentance les justes, mais les pécheurs.
\VS{33}Ils lui dirent aussi : pourquoi est-ce que les disciples de Jean jeûnent souvent, et font des prières et également ceux des Pharisiens, mais les tiens mangent et boivent ?
\VS{34}Il leur répondit : Pouvez-vous faire jeûner les amis de l'Epoux pendant que l'Epoux est avec eux ?
\VS{35}Mais les jours viendront où l'Epoux leur sera enlevé alors ils jeûneront en ces jours-là.
\TextTitle{Parabole du drap neuf et des outres neuves\FTNTT{Mt. 9:16-17 ; Mc. 2:21-22}}
\VS{36}Puis il leur dit cette parabole : Personne ne met une pièce d'un habit neuf à un vieil habit ; autrement le neuf déchire le vieux, et la pièce du neuf ne s'accorde pas avec le vieux.
\VS{37}Et personne ne met du vin nouveau dans de vieilles outres ; autrement le vin nouveau fait rompre les outres, et il se répand, et les outres sont perdues.
\VS{38}Mais le vin nouveau doit être mis dans des outres neuves ; et ainsi ils se conservent l'un et l'autre.
\VS{39}Et personne, après avoir bu du vin vieux, ne veut du nouveau, car il dit : Le vieux est meilleur.
\Chap{6}
\TextTitle{Jésus, le Maître du sabbat\FTNTT{Mt. 12:1-8 ; Mc. 2:23-28}}
\VerseOne{}Or il arriva un jour de sabbat appelé second-premier, qu'il passait par des blés ; et ses disciples arrachaient des épis et les froissant dans leurs mains, ils les mangeaient.
\VS{2}Et quelques pharisiens leur dirent : Pourquoi faites-vous ce qu'il n'est pas permis de faire les jours du sabbat ?
\VS{3}Et Jésus prenant la parole, leur dit : N'avez-vous pas lu ce que fit David quand il eut faim, lui et ceux qui étaient avec lui ;
\VS{4}comment il entra dans la maison de Dieu, et prit les pains de proposition, et en mangea, et en donna aussi à ceux qui étaient avec lui, bien qu'il ne soit permis qu'aux sacrificateurs d'en manger ?\FTNT{2 S. 21:1-7.}
\VS{5}Puis il leur dit : Le Fils de l'homme est Maître même du sabbat.
\TextTitle{Guérison d'un homme à la main sèche\FTNTT{Mt. 12:9-13 ; Mc. 3:1-5}}
\VS{6}Et il arriva, un autre jour de sabbat, qu'il entra dans la synagogue, et qu'il enseignait et il s'y trouvait là un homme dont la main droite était sèche.
\VS{7}Or les scribes et les pharisiens l'observaient pour voir s'il ferait une guérison le jour du sabbat ; c'était afin d'avoir sujet de l'accuser.
\VS{8}Mais il connaissait leurs pensées et il dit à l'homme qui avait la main sèche : Lève-toi, et tiens-toi debout au milieu. Et il se leva et se tint debout.
\VS{9}Puis Jésus leur dit : Je vous demande une chose : Est-il permis de faire du bien les jours de sabbat, ou de faire du mal ? De sauver une personne, ou de la laisser mourir ?
\VS{10}Et ayant regardé tous ceux qui étaient autour de lui, il dit à l’homme : étends ta main ; ce qu’il fit, et sa main fut rendue saine comme l’autre.
\VS{11}Et ils furent remplis de fureur, et ils s’entretenaient ensemble touchant ce qu’ils pourraient faire à Jésus.
\VS{12}Or il arriva en ces jours-là, qu’il s’en alla sur une montagne pour prier, et qu’il passa toute la nuit à prier Dieu.
\TextTitle{Choix des douze apôtres\FTNTT{cp. Mt. 10:2-4 ; Mc. 3:13-19}}
\VS{13}Et quand le jour fut venu, il appela ses disciples. Et en ayant choisi douze d’entre eux, il les nomma apôtres :
\VS{14}Simon, qu'il nomma Pierre, et André son frère, Jacques et Jean, Philippe et Barthélemy ;
\VS{15}Matthieu et Thomas, Jacques fils d'Alphée, et Simon surnommé zélote\FTNT{Zélote : « Celui qui est zélé ». Les zélotes faisaient partie d'un mouvement politique Juif du premier siècle ap. J.-C., qui cherchait à inciter les gens de la province de Judée à se rebeller contre l'Empire Romain, et à le chasser du pays par les armes, pendant la Grande Révolte Juive (66-70 ap. J.-C.). Lorsque les Romains introduisirent le culte impérial, les Juifs se rebellèrent et furent réprimés. Les zélotes considéraient qu'Israël appartenait seulement à un roi Juif de la descendance de David. De plus, reconnaître l'empereur équivalait, à leurs yeux, à renier Dieu. Le mouvement zélote se réclamait intentionnellement de modèles bibliques tels que Phinées, le fils zélé d'Eléazar, fils d'Aaron (No. 25:11). Ce dernier s'était illustré par l'assassinat d'un prince de tribu d'Israël qui s'était fourvoyé dans la luxure aux yeux de tous.} ;
\VS{16}Jude, frère de Jacques, et Judas Iscariot, qui devint traître.
\TextTitle{Enseignement sur la montagne\FTNTT{Mt. 5-7}}
\VS{17}Puis descendant avec eux, il s'arrêta sur une plaine avec la foule de ses disciples et une grande multitude de peuple de toute la Judée, et de Jérusalem, et de la contrée maritime de Tyr et de Sidon, qui étaient venus pour l'entendre, et pour être guéris de leurs maladies.
\VS{18}Ceux aussi qui étaient tourmentés par des esprits impurs furent guéris.
\VS{19}Et toute la foule cherchait à le toucher, parce qu'une force sortait de lui et les guérissait tous.
\TextTitle{Enseignement de Jésus\FTNTT{Mt. 5:3-12}}
\VS{20}Alors Jésus, levant les yeux vers ses disciples, leur dit : Heureux vous qui êtes pauvres, car le Royaume de Dieu vous appartient.
\VS{21}Heureux vous qui avez faim maintenant, car vous serez rassasiés ! Heureux vous qui pleurez maintenant, car vous serez dans la joie !
\VS{22}Heureux serez-vous quand les hommes vous haïront, vous chasseront, vous outrageront, et rejetteront votre nom comme infâme, à cause du Fils de l'homme.
\VS{23}Réjouissez-vous en ce jour-là, et tressaillez d'allégresse, parce que votre récompense sera grande dans le ciel ; car leurs pères en faisaient de même aux prophètes.
\VS{24}Mais malheur à vous riches, car vous avez votre consolation.
\VS{25}Malheur à vous qui êtes rassasiés, car vous aurez faim. Malheur à vous qui riez maintenant, car vous serez dans le deuil et dans les larmes.
\VS{26}Malheur à vous quand tous les hommes diront du bien de vous ; car leurs pères en faisaient de même aux faux prophètes.
\VS{27}Mais à vous qui m’entendez, je vous dis : aimez vos ennemis ; faites du bien à ceux qui vous haïssent.
\VS{28}Bénissez ceux qui vous maudissent, et priez pour ceux qui vous maltraitent.
\VS{29}Si quelqu'un te frappe sur une joue, présente-lui aussi l'autre. Si quelqu'un prend ton manteau, ne l'empêche pas de prendre aussi ta tunique.
\VS{30}Donne à quiconque te demande, et ne réclame pas ton bien à celui qui s'en empare.
\VS{31}Ce que vous voulez que les hommes fassent pour vous, faites-le de même pour eux.
\VS{32}Mais si vous aimez seulement ceux qui vous aiment, quel gré vous en saura-t-on ? Les pécheurs aussi aiment ceux qui les aiment.
\VS{33}Et si vous faites du bien à ceux qui vous font du bien, quel gré vous en saura-t-on ? Les pécheurs aussi font de même.
\VS{34}Et si vous prêtez à ceux de qui vous espérez recevoir, quel gré vous en saura-t-on ? Les pécheurs aussi prêtent aux pécheurs, afin de recevoir la pareille.
\VS{35}C'est pourquoi aimez vos ennemis et faites-leur du bien, et prêtez sans rien espérer, et votre récompense sera grande, et vous serez les fils du Très-Haut, car il est bon envers les ingrats et les méchants.
\VS{36}Soyez donc miséricordieux comme votre Père est miséricordieux.
\VS{37}Ne jugez point, et vous ne serez point jugés ; ne condamnez point, et vous ne serez point condamnés ; absolvez, et vous serez absous.
\VS{38}Donnez, et il vous sera donné : On versera dans votre sein une bonne mesure, serrée, secouée et qui déborde ; car on vous mesurera avec la mesure dont vous vous serez servis.
\VS{39}Il leur disait aussi cette parabole : Un aveugle peut-il conduire un aveugle ? Ne tomberont-ils pas tous deux dans la fosse ?
\VS{40}Le disciple n'est pas au-dessus de son maître ; mais tout disciple accompli sera comme son maître.
\VS{41}Pourquoi regardes-tu la paille qui est dans l'œil de ton frère, et n'aperçois-tu pas la poutre qui est dans ton propre œil ?
\VS{42}Ou comment peux-tu dire à ton frère : Mon frère, laisse-moi enlever la paille qui est dans ton œil, toi qui ne vois pas la poutre qui est dans ton œil ? Hypocrite, ôte premièrement la poutre de ton œil, et après cela tu verras comment ôter la paille qui est dans l'œil de ton frère.
\VS{43}Ce n'est pas un bon arbre qui porte du mauvais fruit, ni un mauvais arbre qui porte du bon fruit.
\VS{44}Car chaque arbre se reconnaît à son fruit. On ne cueille pas des figues sur des épines, et l'on ne vendange pas des raisins sur des ronces.
\VS{45}L'homme de bien tire de bonnes choses du bon trésor de son cœur, et l'homme méchant tire de mauvaises choses du mauvais trésor de son cœur ; car c'est de l'abondance du cœur que la bouche parle.
\TextTitle{Parabole des deux bâtisseurs et des deux fondements\FTNTT{Mt. 7:24-27}}
\VS{46}Mais pourquoi m'appelez-vous Seigneur, Seigneur, et ne faites-vous pas ce que je dis ?
\VS{47}Je vous montrerai à qui est semblable celui qui vient à moi, entend mes paroles, et les met en pratique.
\VS{48}Il est semblable à un homme qui bâtissant une maison, a creusé, creusé profondément, et a mis le fondement sur le roc. Une inondation est venue, et le torrent s'est jeté contre cette maison, sans pouvoir l'ébranler, parce qu'elle était bâtie sur le roc.
\VS{49}Mais celui qui entend mes paroles, et ne les met pas en pratique, est semblable à un homme qui a bâti sa maison sur la terre, sans fondement. Le torrent s'est jeté contre elle ; aussitôt elle est tombée, et la ruine de cette maison a été grande.
\Chap{7}
\TextTitle{Guérison du serviteur d'un centenier\FTNTT{Mt. 8:5-13}}
\VerseOne{}Et quand il eut achevé tout ce discours devant le peuple qui l'écoutait, il entra dans Capernaüm.
\VS{2}Un centenier avait un serviteur, auquel il était très attaché, et qui était malade, sur le point de mourir.
\VS{3}Ayant entendu parler de Jésus, il envoya vers lui quelques anciens des Juifs, pour le prier de venir guérir son serviteur.
\VS{4}Et étant venu à Jésus, ils lui prièrent instamment, disant : Il mérite que tu lui accordes cela.
\VS{5}Car, disaient-ils, il aime notre nation, et c'est lui qui a bâti notre synagogue.
\VS{6}Jésus s'en alla donc avec eux. Il n'était guère éloigné de la maison, quand le centenier envoya ses amis au-devant de lui, pour lui dire : Seigneur, ne te fatigue point ; car je ne suis pas digne que tu entres sous mon toit.
\VS{7}C'est pourquoi aussi je ne me suis pas cru digne d'aller moi-même vers toi ; mais dis seulement une parole, et mon serviteur sera guéri.
\VS{8}Car, moi qui suis un homme soumis à des supérieurs, j'ai des soldats sous mes ordres ; et je dis à l'un : Va, et il va ; et à un autre : Viens, et il vient ; et à mon serviteur : Fais cela, et il le fait.
\VS{9}Lorsque Jésus entendit ces paroles, il admira le centenier ; et se tournant vers la foule qui le suivait, il dit : Je vous le dis, je n'ai pas trouvé, même en Israël, une si grande foi.
\VS{10}Et quand ceux qui avaient été envoyés furent de retour à la maison, ils trouvèrent le serviteur qui avait été malade, se portant bien.
\TextTitle{Le fils de la veuve de Naïn ressuscite}
\VS{11}Le jour suivant, Jésus alla dans une ville appelée Naïn ; plusieurs de ses disciples et une grande foule allaient avec lui.
\VS{12}Et comme il approchait de la porte de la ville, voici, on portait en terre un mort, fils unique de sa mère, qui était veuve ; et il y avait avec elle un grand nombre de gens de la ville.
\VS{13}Le Seigneur l'ayant vue, fut ému de compassion pour elle ; et il lui dit : Ne pleure pas !
\VS{14}Il s'approcha, et toucha le cercueil. Ceux qui le portaient s'arrêtèrent. Et il dit : Jeune homme, je te dis, lève-toi !
\VS{15}Et le mort s'assit, et se mit à parler. Et Jésus le rendit à sa mère.
\VS{16}Et ils furent tous saisis de crainte, et ils glorifiaient Dieu, disant : Certainement un grand prophète a paru parmi nous ; et Dieu a visité son peuple.
\VS{17}Cette parole sur ce miracle se répandit dans toute la Judée, et dans tout le pays d'alentour.
\VS{18}Jean fut informé de toutes ces choses par ses disciples.
\TextTitle{Jean-Baptiste, le plus grand des hommes\FTNTT{Mt. 11:1-19}}
\VS{19}Il en appela deux, et les envoya vers Jésus pour lui dire : Es-tu celui qui devait venir, ou devons-nous en attendre un autre ?
\VS{20}Et étant venus à lui, ils lui dirent : Jean-Baptiste nous a envoyés auprès de toi pour te dire : Es-tu celui qui devait venir, ou devons-nous en attendre un autre ?
\VS{21}A l'heure même, Jésus guérit plusieurs personnes de maladies,  d'infirmités, et d'esprits malins ; et il rendit la vue à plusieurs aveugles.
\VS{22}Ensuite Jésus leur répondit, et leur dit : Allez, et rapportez à Jean ce que vous avez vu et entendu : Les aveugles recouvrent la vue, les boiteux marchent, les lépreux sont purifiés, les sourds entendent, les morts ressuscitent, l'Evangile est annoncé aux pauvres.
\VS{23}Heureux celui qui n'aura point été scandalisé à cause de moi !
\VS{24}Lorsque les messagers de Jean furent partis, Jésus se mit à dire à la foule au sujet de Jean : Qu'êtes-vous allés voir au désert ? Un roseau agité par le vent ?
\VS{25}Mais qu'êtes-vous allés voir ? Un homme vêtu d'habits précieux ? Voici, ceux qui portent des habits magnifiques, et qui vivent dans les délices, sont dans les maisons des rois.
\VS{26}Mais qu'êtes-vous donc allés voir ? Un prophète ? Oui, vous dis-je, et plus qu'un prophète.
\VS{27}C'est de lui qu'il est écrit : Voici, j'envoie mon messager devant ta face, et il préparera ta voie devant toi\FTNT{Mal. 3:1.}.
\VS{28}Car je vous dis, parmi ceux qui sont nés de femmes, il n'y a aucun prophète plus grand que Jean-Baptiste. Cependant, le plus petit dans le Royaume de Dieu est plus grand que lui.
\VS{29}Et tout le peuple qui entendait cela, et les publicains, justifiaient Dieu, ayant été baptisés du baptême de Jean.
\VS{30}Mais les pharisiens, et les docteurs de la loi, qui n'avaient point été baptisés par lui, rendirent le dessein de Dieu inutile à leur égard.
\VS{31}Alors le Seigneur dit : A qui donc comparerai-je les hommes de cette génération ; et à quoi ressemblent-ils ?
\VS{32}Ils sont semblables aux enfants qui sont assis sur la place publique, et qui se parlant les uns aux autres, disent : Nous vous avons joué de la flûte, et vous n'avez pas dansé ; nous vous avons chanté des complaintes, et vous n'avez pas pleuré.
\VS{33}Car Jean-Baptiste est venu ne mangeant point de pain, et ne buvant point de vin ; et vous dites : Il a un démon.
\VS{34}Le Fils de l'homme est venu mangeant et buvant ; et vous dites : Voici un mangeur et un buveur, un ami des publicains et des pécheurs.
\VS{35}Mais la sagesse a été justifiée par tous ses enfants.
\TextTitle{Une pécheresse pardonnée par Jésus}
\VS{36}Un des pharisiens pria Jésus de manger chez lui ; et Jésus entra dans la maison de ce pharisien, et se mit à table.
\VS{37}Et voici, il y avait dans la ville une femme pécheresse, qui ayant su que Jésus était à table dans la maison du pharisien, apporta un vase d'albâtre plein de parfum,
\VS{38}et se tenant derrière à ses pieds, et pleurant, elle les mouilla de ses larmes, elle les essuya avec ses propres cheveux, et lui baisa les pieds, et les oignit de cette huile odoriférante.
\VS{39}Mais le pharisien qui l'avait invité, voyant cela, dit en lui-même : Si cet homme était prophète, certes il saurait qui et de quelle espèce est la femme qui le touche, il saurait que c'est une pécheresse.
\TextTitle{Parabole des deux débiteurs}
\VS{40}Et Jésus prenant la parole, lui dit : Simon, j'ai quelque chose à te dire. Maître, parle, répondit-il.
\VS{41}Un créancier avait deux débiteurs : L'un lui devait cinq cents deniers, et l'autre cinquante.
\VS{42}Et comme ils n'avaient pas de quoi payer, il leur remit à tous deux leur dette. Lequel l'aimera le plus ?
\VS{43}Et Simon répondant lui dit : Celui, je pense, à qui il a le plus remis. Jésus lui dit : Tu as droitement jugé.
\VS{44}Alors se tournant vers la femme, il dit à Simon : Vois-tu cette femme ? Je suis entré dans ta maison, et tu ne m'as point donné d'eau pour laver mes pieds ; mais elle, elle les a mouillés de ses larmes, elle les a essuyés avec ses propres cheveux.
\VS{45}Tu ne m'as point donné un baiser, mais elle, depuis que je suis entré, n'a cessé d'embrasser mes pieds.
\VS{46}Tu n'as pas oint ma tête d'huile ; mais elle, elle a oint mes pieds d'une huile odoriférante.
\VS{47}C'est pourquoi je te le dis, ses nombreux péchés ont été pardonnés, car elle a beaucoup aimé. Or celui à qui on pardonne peu, aime peu.
\VS{48}Puis il dit à la femme : Tes péchés sont pardonnés.
\VS{49}Ceux qui étaient avec lui à table, se mirent à dire en eux-mêmes : Qui est celui-ci qui pardonne même les péchés ?
\VS{50}Mais il dit à la femme : Ta foi t'a sauvée. Va en paix.
\Chap{8}
\TextTitle{Les femmes au service de Jésus durant son ministère}
\VerseOne{}Or il arriva après cela qu'il allait de ville en ville, et de villages en villages, prêchant et annonçant l'Evangile du Royaume de Dieu.
\VS{2}Les douze disciples étaient auprès de lui avec quelques femmes aussi qu'il avait délivrées d'esprits malins et de maladies : Marie de Magdala, de laquelle étaient sortis sept démons,
\VS{3}Et Jeanne, femme de Chuza, intendant d'Hérode, Susanne, et plusieurs autres qui l'assistaient de leurs biens.
\TextTitle{Parabole du semeur\FTNTT{Mt. 13:1-23 ; Mc. 4:1-20}}
\VS{4}Et comme une grande foule s'étant assemblée, et des gens étant venus de diverses villes auprès de lui, il leur dit cette parabole :
\VS{5}Un semeur sortit pour semer sa semence ; et en semant, une partie de la semence tomba le long du chemin ; elle fut foulée aux pieds, et les oiseaux du ciel la mangèrent toute.
\VS{6}Une autre partie tomba dans un endroit pierreux ; et quand elle fut levée, elle sécha, parce qu'elle n'avait point d'humidité.
\VS{7}Une autre partie tomba au milieu des épines ; les épines crurent avec elle, et l'étouffèrent.
\VS{8}Une autre partie tomba dans une bonne terre ; quand elle fut levée, elle donna du fruit au centuple. En disant ces choses, Jésus dit à haute voix : Que celui qui a des oreilles pour entendre, qu'il entende.
\VS{9}Et ses disciples l'interrogèrent pour savoir ce que signifiait cette parabole.
\VS{10}Il répondit : Il vous a été donné de connaître les mystères du Royaume de Dieu, mais pour les autres, cela leur est dit en paraboles, afin qu'en voyant ils ne voient point, et qu'en entendant ils ne comprennent point.
\VS{11}Voici donc ce que signifie cette parabole : La semence, c'est la parole de Dieu.
\VS{12}Ceux qui ont reçu la semence le long du chemin, ce sont ceux qui entendent la parole ; mais ensuite le diable vient et ôte la parole de leur cœur, de peur qu'ils ne croient et soient sauvés.
\VS{13}Et ceux qui ont reçu la semence dans un endroit pierreux, ce sont ceux qui, lorsqu'ils entendent la parole, la reçoivent avec joie ; mais ils n'ont point de racine ; ils croient pour un temps, mais au moment de la tentation ils se retirent.
\VS{14}Et ce qui est tombé parmi les épines, ce sont ceux qui ayant entendu la parole, s'en vont, et la laissent étouffer par les soucis, les richesses, et les plaisirs de la vie, et ils ne portent point de fruit qui vienne à maturité.
\VS{15}Mais ce qui est tombé dans une bonne terre, ce sont ceux qui ayant entendu la parole, la retiennent dans un cœur honnête et bon, et portent du fruit avec persévérance.
\TextTitle{Parabole du chandelier\FTNTT{Mt. 5:15-16 ; Mc. 4:21-23 ; Lu. 11:33-36}}
\VS{16}Personne, après avoir allumé la lampe, ne la couvre d'un vase ni ne la met sous un lit, mais il la met sur un chandelier, afin que ceux qui entrent voient la lumière.
\VS{17}Car il n'est rien de secret qui ne doive être découvert ; rien de caché qui ne doive être connu et qui ne vienne en évidence.
\VS{18}Prenez donc garde à la manière dont vous écoutez ; car on donnera à celui qui a, mais à celui qui n'a pas, on ôtera même ce qu'il croit avoir.
\TextTitle{La famille spirituelle\FTNTT{Mt. 12:46-50 ; Mc. 3:31-35}}
\VS{19}Alors sa mère et ses frères vinrent vers lui, mais ils ne pouvaient l'aborder à cause de la foule.
\VS{20}Et on vint lui dire : Ta mère et tes frères sont dehors, et ils désirent te voir.
\VS{21}Mais il répondit : Ma mère et mes frères sont ceux qui écoutent la parole de Dieu, et qui la mettent en pratique.
\TextTitle{Jésus calme la tempête\FTNTT{Mt. 8:23-27 ; Mc. 4:35-41}}
\VS{22}Or il arriva qu'un jour, Jésus monta dans une barque avec ses disciples, et il leur dit : Passons de l'autre côté du lac ; et ils partirent.
\VS{23}Pendant qu'ils naviguaient, il s'endormit. Un vent impétueux se leva sur le lac, la barque se remplissait d'eau, et ils étaient en danger.
\VS{24}Ils s'approchèrent et le réveillèrent, en disant : Maître ! Maître ! Nous périssons ! S'étant réveillé, il menaça le vent et les flots qui s'apaisèrent, et le calme revint.
\VS{25}Alors il leur dit : Où est votre foi ? Saisis de frayeur et d'étonnement, ils se dirent les uns aux autres : Quel est donc celui-ci qui commande même aux vents et à l'eau, et à qui ils obéissent ?
\TextTitle{Le démoniaque de Gérasa (Gadara) délivré\FTNTT{Mt. 8:28-34 ; Mc. 5:1-20}}
\VS{26}Puis ils abordèrent dans le pays des Géraséniens qui est vis-à-vis de la Galilée.
\VS{27}Et quand il fut descendu à terre, il vint à sa rencontre un homme de cette ville, qui depuis longtemps était possédé de plusieurs démons. Il ne portait point de vêtements, avait sa demeure, non dans une maison, mais dans les sépulcres.
\VS{28}Ayant vu Jésus, il s'écria et se prosterna devant lui, disant à haute voix : Qu'y a-t-il entre moi et toi, Jésus, Fils du Dieu Très-Haut ? Je te prie, ne me tourmente point.
\VS{29}Car Jésus commandait à l'esprit impur de sortir de cet homme, dont il s'était emparé depuis longtemps. On le gardait lié de chaînes et les fers aux pieds, mais il rompait les liens, et il était entrainé par le démon dans les déserts.
\VS{30}Jésus lui demanda : Quel est ton nom ? Légion\FTNT{Voir commentaire en Mc. 5:9.} répondit-il. Car plusieurs démons étaient entrés en lui.
\VS{31}Et ils priaient Jésus de ne pas leur ordonner d'aller dans l'abîme.
\VS{32}Or il y avait là, dans la montagne, un grand troupeau de pourceaux qui paissaient. Les démons supplièrent Jésus de leur permettre d'entrer dans ces pourceaux. Il le leur permit.
\VS{33}Les démons sortirent de cet homme et entrèrent dans les pourceaux ; et le troupeau se précipita des pentes escarpées dans le lac, et se noya.
\VS{34}Ceux qui les faisaient paître, voyant ce qui était arrivé, s'enfuirent et allèrent le raconter dans la ville et dans les campagnes.
\VS{35}Les gens sortirent pour voir ce qui était arrivé. Ils vinrent auprès de Jésus, et ils trouvèrent l'homme de qui étaient sortis les démons, assis aux pieds de Jésus, vêtu et dans son bon sens ; et ils furent saisis de frayeur.
\VS{36}Et ceux qui avaient vu ce qui s'était passé leur racontèrent comment le démoniaque avait été délivré.
\VS{37}Alors toute cette multitude venue de divers endroits voisins des Géraséniens, le prièrent de se retirer de chez eux ; car ils étaient saisis d'une grande crainte. Jésus monta donc dans la barque, et s'en retourna.
\VS{38}L'homme de qui étaient sortis les démons lui demanda la permission de rester avec lui ; mais Jésus le renvoya, en lui disant :
\VS{39}Retourne dans ta maison, et raconte tout ce que Dieu t'a fait. Il s'en alla donc, et publia par toute la ville tout ce que Jésus avait fait pour lui.
\VS{40}Quand Jésus fut de retour, la foule le reçut avec joie ; car tous l'attendaient.
\TextTitle{Les deux guérisons\FTNTT{Mt. 9:18-26 ; Mc. 5:21-43}}
\VS{41}Et voici, un homme appelé Jaïrus, qui était chef de la synagogue, vint et se jetant aux pieds de Jésus, le pria d'entrer dans sa maison
\VS{42}parce qu'il avait une fille unique, âgée d'environ douze ans, qui se mourait. Pendant que Jésus y allait, il était pressé par la foule.
\VS{43}Or, il y avait une femme atteinte d'une perte de sang depuis douze ans, et qui avait dépensé tout son bien pour les médecins, sans qu'aucun n'ait pu la guérir.
\VS{44}S'approchant de lui par derrière, elle toucha le bord de son vêtement. Au même instant la perte de sang s'arrêta.
\VS{45}Jésus dit : Qui m'a touché ? Comme tous le niaient, Pierre et ceux qui étaient avec lui, dirent : Maître, la foule qui t'entoure te presse et tu dis : Qui m'a touché ?
\VS{46}Mais Jésus dit : Quelqu'un m'a touché, car j'ai connu qu'une force est sortie de moi.
\VS{47}Alors la femme, se voyant découverte, vint toute tremblante se jeter à ses pieds, lui déclara devant tout le peuple pour quelle raison elle l'avait touché, et comment elle avait été guérie à l'instant.
\VS{48}Jésus lui dit : Ma fille, rassure-toi. Ta foi t'a guérie. Va en paix.
\VS{49}Et comme il parlait encore, quelqu'un vint de chez le chef de la synagogue, qui lui dit : Ta fille est morte, n'importune pas le Maître.
\VS{50}Mais Jésus ayant entendu cela, dit au père de la fille : Ne crains point ; crois seulement, et elle sera guérie.
\VS{51}Et quand il fut arrivé à la maison, il ne permit à personne d'entrer avec lui, si ce n'est à Pierre, à Jacques et à Jean, et au père et à la mère de la fille.
\VS{52}Or il la pleuraient tous et de douleur, ils se frappaient la poitrine ; mais il leur dit : Ne pleurez point, elle n'est pas morte, mais elle dort.
\VS{53}Et ils se moquaient de lui, sachant bien qu'elle était morte.
\VS{54}Mais les ayant tous fait sortir, il prit la main de la fille, et dit d'une voix forte : Enfant, lève-toi !
\VS{55}Et son esprit revint en elle, et à l'instant elle se leva ; et Jésus ordonna qu'on lui donne à manger.
\VS{56}Les parents de la fille furent dans l'étonnement, et il leur commanda de ne dire à personne ce qui était arrivé.
\Chap{9}
\TextTitle{Mission des douze apôtres\FTNTT{Mt. 10:1-42 ; cp. Mc. 6:7-13}}
\VerseOne{}Puis, Jésus ayant assemblé ses douze disciples, leur donna puissance et autorité sur tous les démons, avec le pouvoir de guérir les malades.
\VS{2}Il les envoya prêcher le Royaume de Dieu et guérir les malades.
\VS{3}Il leur dit : Ne prenez rien pour le voyage, ni bâton, ni sac, ni pain, ni argent ; et n'ayez pas chacun deux tuniques.
\VS{4}Dans quelque maison que vous entriez, demeurez-y jusqu'à ce que vous partiez de là.
\VS{5}Et partout où l'on ne vous recevra pas, en partant de cette ville secouez la poussière de vos pieds, en témoignage contre eux.
\VS{6}Ils partirent, et ils allèrent de village en village, évangélisant et opérant des guérisons partout.
\VS{7}Or Hérode le Tétrarque entendit parler de toutes les choses que Jésus faisait ; et il ne savait que penser. Car quelques-uns disaient que Jean était ressuscité des morts ;
\VS{8}d'autres, qu'Elie était apparu ; et d'autres, que quelqu'un des anciens prophètes était ressuscité.
\VS{9}Mais Hérode dit : J'ai fait décapiter Jean. Qui est donc celui-ci de qui j'entends dire de telles choses ? Et il cherchait à le voir.
\VS{10}Puis les apôtres étant de retour, lui racontèrent toutes les choses qu'ils avaient faites. Jésus les prit avec lui, et se retira dans un lieu désert, près de la ville appelée Bethsaïda.
\VS{11}Les foules l'ayant su, le suivirent. Jésus les accueillit, et il leur parlait du Royaume de Dieu ; il guérit aussi ceux qui avaient besoin d'être guéris.
\TextTitle{Multiplication des pains pour cinq mille hommes\FTNTT{Mt. 14:15-21 ; Mc. 6:32-44 ; Jn. 6:1-14}}
\VS{12}Comme le jour commençait à baisser, les douze disciples s'approchèrent, et lui dirent : Renvoie la foule, afin qu'elle aille dans les villages et dans les campagnes des environs, pour se loger et pour trouver à manger ; car nous sommes ici dans un pays désert.
\VS{13}Et il leur dit : Donnez-leur vous-mêmes à manger. Et ils dirent : Nous n'avons que cinq pains et deux poissons ; à moins que nous n'allions nous-mêmes acheter des vivres pour tout ce peuple.
\VS{14}Or, il y avait environ cinq mille hommes. Jésus dit aux disciples : Faites-les asseoir par rangées de cinquante chacune.
\VS{15}Ils le firent ainsi, et les firent tous asseoir.
\VS{16}Jésus prit les cinq pains et les deux poissons, et levant les yeux au ciel, il les bénit. Puis, il les rompit, et il les donna à ses disciples afin qu'ils les distribuent à la foule.
\VS{17}Tous mangèrent et furent rassasiés, et l'on remporta douze paniers pleins de morceaux de pain qui restaient.
\TextTitle{Pierre reconnaît Jésus comme le Messie\FTNTT{Mt. 16:13-16 ; Mc. 8:27-30 ; Jn. 6:66-71}}
\VS{18}Or il arriva que comme il était dans un lieu retiré pour prier, et que les disciples étaient avec lui, il les interrogea, disant : Qui disent les foules que je suis ?
\VS{19}Ils lui répondirent : Les uns disent que tu es Jean-Baptiste ; les autres, Elie ; et les autres, qu'un des anciens prophètes est ressuscité.
\VS{20}Il leur dit alors : Et vous, qui dites-vous que je suis ? Et Pierre répondit : Tu es le Christ de Dieu.
\VS{21}Jésus leur défendit sévèrement de ne le dire à personne.
\TextTitle{Jésus annonce sa mort et sa résurrection\FTNTT{Mt. 16:21-23 ; Mc. 8:31-33}}
\VS{22}Et il leur dit : Il faut que le Fils de l'homme souffre beaucoup, et qu'il soit rejeté par les anciens, par les principaux sacrificateurs et par les scribes, et qu'il soit mis à mort, et qu'il ressuscite le troisième jour.
\TextTitle{La consécration du disciple\FTNTT{Mt. 16:24-28 ; Mc. 8:34-38}}
\VS{23}Puis il dit à tous : Si quelqu'un veut venir après moi, qu'il renonce à lui-même, qu'il se charge chaque jour de sa croix, et qu'il me suive.
\VS{24}Car celui qui voudra sauver sa vie, la perdra ; mais celui qui perdra sa vie à cause de son amour pour moi, la sauvera.
\VS{25}Et que servirait-il à un homme de gagner tout le monde, s'il se détruisait ou se perdait lui-même ?
\VS{26}Car quiconque aura honte de moi et de mes paroles, le Fils de l'homme aura honte de lui quand il viendra dans sa gloire, et dans celle du Père et des saints anges.
\TextTitle{La transfiguration\FTNTT{Mt. 17:1-8 ; Mc. 9:1-8}}
\VS{27}Je vous le dis, en vérité, quelques-uns de ceux qui sont ici présents, ne mourront point qu'ils n'aient vu le Royaume de Dieu\FTNT{Voir commentaire en  Mt. 16:28.}.
\VS{28}Or il arriva environ huit jours après ces paroles, qu'il prit avec lui Pierre, et Jean, et Jacques, et qu'il monta sur une montagne pour prier.
\VS{29}Et comme il priait, l'aspect de son visage changea, et son vêtement devint blanc et resplendissant comme un éclair.
\VS{30}Et voici, deux hommes savoir Moïse et Elie, parlaient avec lui,
\VS{31}et ils apparurent environnés de gloire, et ils parlaient de sa mort\FTNT{Départ : Du grec « exodos », ce qui signifie « départ », « mort », « sortie », « hors de ». Jésus est le prophète de l'Exode dont Moïse a parlé dans De. 18:15.} qu'il allait accomplir à Jérusalem.
\VS{32}Or Pierre et ceux qui étaient avec lui étaient accablés de sommeil ; et quand ils furent réveillés, ils virent sa gloire, et les deux hommes qui étaient avec lui.
\VS{33}Et il arriva qu'au moment où ces hommes se séparaient de Jésus, Pierre dit : Maître, il est bon que nous soyons ici, dressons trois tentes, une pour toi, une pour Moïse, et une pour Elie. Il ne savait pas ce qu'il disait.
\VS{34}Et comme il parlait ainsi, une nuée vint les couvrir de son ombre ; et les disciples furent saisis de frayeur en les voyant entrer dans la nuée.
\VS{35}Et une voix vint de la nuée, disant : Celui-ci est mon Fils bien-aimé ; écoutez-le.
\VS{36}Quand la voix se fit entendre, Jésus se trouva seul. Les disciples gardèrent le silence, et ils ne rapportèrent rien à personne en ce temps-là de ce qu'ils avaient vu.
\TextTitle{Les disciples de Jésus montrent leur limite}
\VS{37}Or il arriva le jour suivant, lorsqu'ils furent descendus de la montagne, une grande foule vint à sa rencontre.
\VS{38}Et voici, du milieu de la foule un homme s'écria : Maître, je t'en prie, porte les regards sur mon fils, car c'est mon fils unique.
\VS{39}Et voici un esprit le saisit, et aussitôt le fait crier, et l'agite avec violence en le faisant écumer, et c'est à peine s'il se retire de lui après l'avoir broyé.
\VS{40}J'ai prié tes disciples de le chasser, mais ils n'ont pas pu.
\VS{41}Jésus répondit : Ô génération incrédule et perverse, jusqu'à quand serai-je avec vous, et vous supporterai-je ? Amène ici ton fils.
\VS{42}Comme il approchait, le démon l'agita violemment comme s'il voulait le déchirer ; mais Jésus menaça fortement l'esprit impur, guérit l'enfant, et le rendit à son père.
\VS{43}Et tous furent étonnés de la puissance magnifique de Dieu. Et comme ils étaient tous dans l'admiration de tout ce que Jésus faisait, il dit à ses disciples :
\TextTitle{Jésus annonce de nouveau sa mort et sa résurrection\FTNTT{Mt. 17:22-23 ; Mc. 9:30-32}}
\VS{44}Vous, écoutez bien ces discours : Le Fils de l'homme sera livré entre les mains des hommes.
\VS{45}Mais les disciples ne comprirent pas cette parole, elle était voilée pour eux, afin qu'ils n'en aient pas le sens ; et ils craignaient de l'interroger à ce sujet.
\TextTitle{L'humilité, le secret de la véritable grandeur\FTNTT{Mt. 18:1-6 ; Mc. 9:33-37}}
\VS{46}Or, une pensée leur vint à l'esprit, savoir lequel d'entre eux était le plus grand.
\VS{47}Mais Jésus voyant la pensée de leur cœur, prit un petit enfant et le mit auprès de lui.
\VS{48}Puis il leur dit : Quiconque reçoit ce petit enfant en mon Nom, me reçoit ; et quiconque me reçoit, reçoit celui qui m'a envoyé. Car celui qui est le plus petit d'entre vous tous, c'est celui-là qui est grand.
\TextTitle{Jésus condamne l'esprit sectaire de Jacques et Jean\FTNTT{Mc. 9:38-40}}
\VS{49}Et Jean prit la parole et dit : Maître, nous avons vu quelqu'un qui chassait les démons en ton Nom, et nous l'en avons empêché parce qu'il ne nous suit pas.
\VS{50}Mais Jésus lui dit : Ne l'en empêchez pas ; car celui qui n'est pas contre nous, est pour nous.
\TextTitle{Mission de Jésus : sauver les âmes}
\VS{51}Lorsque le temps où il devait être enlevé du monde approcha, Jésus prit la résolution d'aller à Jérusalem.
\VS{52}Il envoya devant lui des messagers, qui se mirent en route, et entrèrent dans un bourg des Samaritains, pour lui préparer un logement.
\VS{53}Mais les Samaritains ne le reçurent pas, parce qu'il se dirigeait sur Jérusalem.
\VS{54}Et quand Jacques et Jean, ses disciples virent cela, ils dirent : Seigneur ! Veux-tu que nous commandions que le feu descende du ciel, et les consume, comme fit Elie ?
\VS{55}Mais Jésus se tourna vers eux et les réprimanda fortement, en leur disant : Vous ne savez pas de quel esprit vous êtes animés.
\VS{56}Car le Fils de l'homme n'est pas venu pour perdre les âmes des hommes, mais pour les sauver. Ainsi, ils allèrent dans un autre bourg.
\TextTitle{Epreuves de l'engagement du disciple pour suivre Jésus\FTNTT{Mt. 8:19-22}}
\VS{57}Pendant qu'ils étaient en chemin, un homme lui dit : Seigneur, je te suivrai partout où tu iras.
\VS{58}Mais Jésus lui répondit : Les renards ont des tanières, et les oiseaux du ciel ont des nids, mais le Fils de l'homme n'a pas où reposer sa tête.
\VS{59}Puis il dit à un autre : Suis-moi. Et il répondit : Permets-moi d'aller d'abord ensevelir mon père.
\VS{60}Mais Jésus lui dit : Laisse les morts ensevelir leurs morts ; mais toi, va, et annonce le Royaume de Dieu.
\VS{61}Un autre aussi lui dit : Seigneur, je te suivrai ; mais permets-moi de prendre d'abord congé de ceux de ma maison.
\VS{62}Mais Jésus lui répondit : Quiconque met la main à la charrue, et regarde en arrière, n'est pas bien disposé pour le Royaume de Dieu.
\Chap{10}
\TextTitle{Soixante-dix disciples envoyés en mission}
\VerseOne{}Or après ces choses, le Seigneur désigna soixante-dix autres disciples, et il les envoya deux à deux devant lui, dans toutes les villes et dans tous les lieux où il devait aller.
\VS{2}Il leur dit : La moisson est grande, mais il y a peu d'ouvriers ; priez donc le seigneur de la moisson qu'il pousse des ouvriers dans sa moisson.
\VS{3}Allez, voici, je vous envoie comme des agneaux au milieu des loups.
\VS{4}Ne portez ni bourse, ni sac, ni souliers, et ne saluez personne en chemin.
\VS{5}En quelque maison que vous entriez, dites premièrement : Que la paix soit sur cette maison !
\VS{6}Et s'il y a là quelqu'un qui soit digne de paix, votre paix reposera sur lui ; sinon elle retournera à vous.
\VS{7}Et demeurez dans cette maison, mangeant et buvant de ce qui sera mis devant vous ; car l'ouvrier mérite son salaire. N'allez pas de maison en maison.
\VS{8}Dans quelque ville que vous entriez, et où l'on vous recevra, mangez ce qui sera mis devant vous,
\VS{9}guérissez les malades qui s'y trouveront, et dites-leur : Le Royaume de Dieu s'est approché de vous.
\VS{10}Mais dans quelque ville que vous entriez, et où l'on ne vous recevra pas, sortez dans ses rues et dites :
\VS{11}Nous secouons contre vous-mêmes la poussière de votre ville qui s'est attachée à nous ; toutefois sachez que le Royaume de Dieu s'est approché de vous.
\VS{12}Je vous le dis qu'en ce jour Sodome sera traitée moins rigoureusement que cette ville-là.
\TextTitle{Jésus dénonce les indifférents\FTNTT{Mt. 11:20-24}}
\VS{13}Malheur à toi Chorazin, malheur à toi Bethsaïda ! Car si les miracles qui ont été faits au milieu de vous avaient été faits dans Tyr et dans Sidon, il y a longtemps qu'elles se seraient repenties, couvertes d'un sac, et assises sur la cendre.
\VS{14}C'est pourquoi Tyr et Sidon seront traitées moins rigoureusement que vous au jour du jugement.
\VS{15}Et toi, Capernaüm, qui as été élevée jusqu'au ciel, tu seras précipitée jusque dans le Hadès\FTNT{Voir commentaire en Mt. 16:18}.
\VS{16}Celui qui vous écoute, m'écoute ; et celui qui vous rejette, me rejette. Or celui qui me rejette, rejette celui qui m'a envoyé.
\VS{17}Or les soixante-dix revinrent avec joie, disant : Seigneur, les démons mêmes nous sont soumis en ton Nom.
\VS{18}Jésus leur dit : Je voyais Satan tomber du ciel comme un éclair.
\VS{19}Voici, je vous ai donné le pouvoir de marcher sur les serpents et sur les scorpions, et sur toute la force de l'ennemi ; et rien ne pourra vous nuire.
\VS{20}Toutefois, ne vous réjouissez pas de ce que les esprits vous sont soumis, mais réjouissez-vous plutôt de ce que vos noms sont écrits dans les cieux.
\VS{21}En ce moment même, Jésus se réjouit en esprit, et dit : Je te loue, ô Père ! Seigneur du ciel et de la terre, de ce que tu as caché ces choses aux sages et aux intelligents, et que tu les as révélées aux petits enfants. Oui, Père, parce que telle a été ta bonne volonté.
\VS{22}Toutes choses m'ont été données en main par mon Père ; et personne ne connaît qui est le Fils, si ce n'est le Père ; ni qui est le Père, si ce n'est le Fils et celui à qui le Fils veut le révéler.
\VS{23}Puis, se tournant vers ses disciples, il leur dit en particulier : Heureux sont les yeux qui voient ce que vous voyez.
\VS{24}Car je vous dis que beaucoup de prophètes et de rois ont désiré voir ce que vous voyez, et ne l'ont pas vu, et entendre ce que vous entendez, et ne l'ont pas entendu.
\TextTitle{Un docteur de la loi tente d'éprouver Jésus\FTNTT{cp. Mt. 22:34-40 ; Mc. 12:28-34}}
\VS{25}Alors voici, un docteur de la loi s'étant levé pour l'éprouver lui dit : Maître, que dois-je faire pour avoir la vie éternelle ?
\VS{26}Et il lui dit : Qu'est-il écrit dans la loi ? Qu'y lis-tu ?
\VS{27}Il répondit : Tu aimeras le Seigneur ton Dieu de tout ton cœur, de toute ton âme, de toute ta force, et de toute ta pensée ; et ton prochain comme toi-même.
\VS{28}Jésus lui dit : Tu as bien répondu. Fais cela, et tu vivras.
\VS{29}Mais lui, voulant se justifier, dit à Jésus : Et qui est mon prochain ?
\TextTitle{Parabole du Samaritain}
\VS{30}Jésus reprit la parole et dit : Un homme descendait de Jérusalem à Jéricho. Il tomba entre les mains des brigands, qui le dépouillèrent, le chargèrent de plusieurs coups, et s'en allèrent, le laissant à demi mort.
\VS{31}Un sacrificateur, qui par hasard descendait par le même chemin, ayant vu cet homme, passa outre.
\VS{32}Un Lévite, qui arriva aussi dans ce lieu, l'ayant vu, passa outre.
\VS{33}Mais un Samaritain, qui voyageait, étant venu là, fut ému de compassion lorsqu'il le vit.
\VS{34}Il s'approcha, banda ses plaies, en y versant de l'huile et du vin ; puis le mit sur sa propre monture, et le conduisit à une hôtellerie, et prit soin de lui.
\VS{35}Et le lendemain, en partant il tira de sa bourse deux deniers, et les donna à l'hôte, en lui disant : Aie soin de lui ; et tout ce que tu dépenseras de plus, je te le rendrai à mon retour.
\VS{36}Lequel donc de ces trois te semble-t-il avoir été le prochain de celui qui était tombé entre les mains des brigands ?
\VS{37}Il répondit : C'est celui qui a usé de miséricorde envers lui. Jésus donc lui dit : Va, et toi aussi fais de même.
\TextTitle{Marthe et Marie}
\VS{38}Et il arriva comme ils s'en allaient, qu'il entra dans une bourgade ; et une femme nommée Marthe le reçut dans sa maison.
\VS{39}Elle avait une sœur nommée Marie, qui se tenant assise aux pieds de Jésus, écoutait sa parole.
\VS{40}Mais Marthe était distraite par divers soins domestiques ; et étant venue à Jésus, elle dit : Seigneur, ne te soucies-tu point que ma sœur me laisse servir toute seule, dis-lui donc de m'aider de son côté.
\VS{41}Jésus lui répondit : Marthe, Marthe, tu t'inquiètes et tu t'agites pour beaucoup de choses.
\VS{42}Mais une chose est nécessaire ; et Marie a choisi la bonne part, qui ne lui sera point ôtée.
\Chap{11}
\TextTitle{Enseignement de Jésus sur la prière\FTNTT{cp. Mt. 6:9-15}}
\VerseOne{}Et il arriva, comme il était en prière en un certain lieu, qu'après qu'il eut cessé de prier, un de ses disciples lui dit : Seigneur, enseigne-nous à prier, comme Jean l'a enseigné à ses disciples.
\VS{2}Il leur dit : Quand vous prierez, dites : Notre Père qui es aux cieux ! Que ton Nom soit sanctifié, que ton règne vienne ; que ta volonté soit faite sur la terre comme au ciel.
\VS{3}Donne-nous chaque jour notre pain quotidien.
\VS{4}Et pardonne-nous nos péchés ; car nous aussi, nous remettons les dettes à tous ceux qui nous doivent ; et ne nous induis point en tentation, mais délivre-nous du mal.
\TextTitle{Parabole des trois amis et de la prière importune}
\VS{5}Puis il leur dit : Si l'un de vous a un ami, et qu'il aille le trouver à minuit pour lui dire : Mon ami, prête-moi trois pains,
\VS{6}car un de mes amis est arrivé de voyage chez moi, et je n'ai rien à lui offrir,
\VS{7}et si, de l'intérieur de sa maison, cet ami lui répond : Ne m'importune pas ; car ma porte est déjà fermée, mes enfants et moi nous sommes au lit ; je ne puis me lever pour t'en donner.
\VS{8}Je vous le dis, même s'il ne se levait pas pour les lui donner parce que c'est son ami, il se lèverait à cause de son importunité, et lui donnerait tout ce dont il a besoin.
\VS{9}Ainsi je vous dis : Demandez, et il vous sera donné ; cherchez, et vous trouverez ; frappez, et l'on vous ouvrira.
\VS{10}Car quiconque demande, reçoit ; et celui qui cherche, trouve ; et l'on ouvre à celui qui frappe.
\TextTitle{Parabole du père}
\VS{11}Quel est parmi vous le père qui donnera une pierre à son fils, s'il lui demande du pain ? Ou, s'il lui demande un poisson, lui donnera-t-il un serpent au lieu d'un poisson ?
\VS{12}Ou, s'il demande un œuf, lui donnera-t-il un scorpion ?
\VS{13}Si donc vous qui êtes méchants, vous savez donner à vos enfants des choses bonnes, combien plus le Père qui est du ciel donnera-t-il l'Esprit Saint à ceux qui le lui demandent ,
\TextTitle{Jésus guérit un démoniaque}
\VS{14}Alors il chassa un démon qui était muet. Lorsque le démon fut sorti, le muet parla ; et la foule fut dans l'admiration.
\TextTitle{Le blasphème contre le Saint-Esprit\FTNTT{Mt. 12:24-32 ; Mc. 3:22-30}}
\VS{15}Mais quelques-uns d'entre eux dirent : C'est par Béelzebul\FTNT{Béelzebul : Dans 2 R. 1:2, il est fait mention de « Baal Zebub, dieu d'Eqrôn ». Littéralement, la formule signifie « maître (Baal) des mouches ». Ce mot a une autre signification que le grec de la Septante a adoptée en traduisant par Baal-myia, « Baal-mouche ». Le prince des démons.}, prince des démons, qu'il chasse les démons.
\VS{16}Mais les autres pour l'éprouver, lui demandaient un miracle venant du ciel.
\VS{17}Mais lui, connaissant leurs pensées, leur dit : Tout royaume divisé contre lui-même sera réduit en désert ; et toute maison divisée contre elle-même tombe en ruine.
\VS{18}Si donc Satan est divisé contre lui-même, comment son royaume subsistera-t-il ? Car vous dites que je chasse les démons par Béelzebul.
\VS{19}Et si moi, je chasse les démons par Béelzebul, vos fils par qui les chassent-ils ? C'est pourquoi ils seront eux-mêmes vos juges.
\VS{20}Mais si je chasse les démons par le doigt de Dieu, alors le royaume de Dieu est parvenu jusqu'à vous.
\VS{21}Lorsqu'un homme fort et bien armé garde sa bergerie\FTNT{Bergerie : Chez les Grecs, du temps d'Homère, c'était un espace découvert autour de la maison, fermé par un mur, tandis que chez les Orientaux, il s'agissait d'un espace dans la campagne, entouré d'un mur, où les troupeaux passaient la nuit. La bergerie désigne aussi la partie non couverte d'une maison. Dans la première alliance, il s'agit particulièrement du « parvis » du tabernacle et du temple à Jérusalem. Les demeures des gens de la haute société possédaient généralement deux de ces « cours » : une entre la porte et la rue, l'autre entourée par l'immeuble lui-même. C'est ce qui est mentionné en Mt. 26:69. Enfin, ce terme fait allusion à la maison elle-même, un palais.}, les biens qu'il a sont en sûreté.
\VS{22}Mais si un plus fort que lui survient et le vainque, il lui enlève toutes ses armes dans lesquelles il se confiait, et il partage ses dépouilles.
\VS{23}Celui qui n'est point avec moi est contre moi ; et celui qui n'assemble pas avec moi, il disperse.
\TextTitle{Le retour de l'esprit impur\FTNTT{Mt. 12:43-45}}
\VS{24}Quand l'esprit impur est sorti d'un homme, il va par des lieux secs, cherchant du repos. N'en trouvant point, il dit : Je retournerai dans ma maison, d'où je suis sorti,
\VS{25}et quand il arrive, il la trouve balayée et parée.
\VS{26}Alors il s'en va, et prend avec lui sept autres esprits plus méchants que lui, et ils entrent et demeurent là ; de sorte que la dernière condition de cet homme-là est pire que la première.
\VS{27}Or il arriva comme il disait ces choses, qu'une femme élevant sa voix du milieu de la foule, lui dit : Heureux est le ventre qui t'a porté, et les mamelles que tu as tétées !
\VS{28}Et il répondit : Heureux plutôt ceux qui écoutent la parole de Dieu, et qui la gardent !
\TextTitle{Le signe du prophète Jonas\FTNTT{Mt. 12:38-41}}
\VS{29}Et comme les foules s'amassaient ensemble, il se mit à dire : Cette génération est méchante ; elle demande un miracle, mais il ne lui sera donné d'autre miracle que celui de Jonas le prophète.
\VS{30}Car, de même que Jonas fut un miracle pour les Ninivites, de même le Fils de l'homme en sera un pour cette génération.
\VS{31}La reine du Midi se lèvera au jour du jugement contre les hommes de cette génération et les condamnera, parce qu'elle vint des extrémités de la terre pour entendre la sagesse de Salomon ; et voici, il y a ici plus que Salomon.
\VS{32}Les gens de Ninive se lèveront au jour du jugement contre cette génération et la condamneront, parce qu'ils se sont repentis à la prédication de Jonas ; et voici, il y a ici plus que Jonas.
\TextTitle{Parabole de la lampe\FTNTT{Mt. 5:14-16 ; Mc. 4:21-23 ; cp. Lu. 8:16-18}}
\VS{33}Or personne n'allume une lampe pour la mettre dans un lieu caché ou sous le boisseau, mais sur un chandelier, afin que ceux qui entrent voient la lumière.
\VS{34}La lumière du corps c'est l'œil. Si donc ton œil est sain, tout ton corps aussi sera éclairé ; mais s'il est mauvais, ton corps aussi sera ténébreux.
\VS{35}Prends donc garde que la lumière qui est en toi ne soit pas ténèbres.
\VS{36}Si donc ton corps est éclairé, n'ayant aucune partie dans les ténèbres, il sera entièrement éclairé, comme lorsque la lampe t'éclaire de sa lumière.
\VS{37}Comme il parlait, un pharisien le pria de dîner chez lui. Il entra, et se mit à table.
\VS{38}Mais le pharisien vit avec étonnement qu'il ne s'était pas premièrement lavé avant le dîner.
\TextTitle{Malheurs sur les pharisiens et les docteurs de la loi\FTNTT{cp. Mt. 12:38-41}}
\VS{39}Mais le Seigneur lui dit : Vous autres pharisiens, vous nettoyez le dehors de la coupe et du plat ; et à l'intérieur vous êtes pleins de rapine et de méchanceté.
\VS{40}Insensés, celui qui a fait le dehors, n'a-t-il pas fait aussi le dedans ?
\VS{41}Donnez plutôt en aumône ce qui est dedans, et voici, toutes choses seront pures pour vous.
\VS{42}Mais malheur à vous, pharisiens ! Car vous payez la dîme de la menthe, de la rue\FTNT{La rue : Il s'agit d'un arbuste ayant des propriétés médicinales. Les pharisiens poussaient leur zèle jusqu'à payer la dîme sur certaines herbes. Toutefois, en négligeant la justice et l'amour de Dieu, ils passaient à côté de l'essentiel. Toutes leurs œuvres étaient par conséquent vaines.}, et de toutes sortes d'herbes, et vous négligez la justice et l'amour de Dieu. C'est là ce qu'il fallait pratiquer, sans négliger les autres choses.
\VS{43}Malheur à vous, pharisiens, qui aimez les premières places dans les synagogues, et les salutations sur les places publiques.
\VS{44}Malheur à vous, scribes et pharisiens hypocrites, car vous êtes comme les sépulcres qui ne paraissent pas, et sur lesquels on marche sans le voir.
\VS{45}Alors un des docteurs de la loi prit la parole, et lui dit : Maître, en disant ces choses, tu nous outrages aussi.
\VS{46}Et il dit : À vous aussi, malheur, docteurs de la loi ! Car vous chargez les hommes de fardeaux difficiles à porter, et vous-mêmes vous ne touchez pas ces fardeaux d'un seul de vos doigts.
\VS{47}Malheur à vous, car vous bâtissez les sépulcres des prophètes, que vos pères ont tués.
\VS{48}Vous rendez donc témoignage aux œuvres de vos pères et vous y prenez plaisir ; car eux, ils les ont tués, et vous, vous bâtissez leurs sépulcres.
\VS{49}C'est pourquoi aussi la sagesse de Dieu a dit : Je leur enverrai des prophètes et des apôtres, et ils tueront les uns, et persécuteront les autres,
\VS{50}afin que le sang de tous les prophètes qui a été répandu dès la fondation du monde, soit redemandé à cette nation.
\VS{51}Depuis le sang d'Abel, jusqu'au sang de Zacharie, qui fut tué entre l'autel et le temple. Oui, je vous le dis qu'il sera redemandé à cette nation.
\VS{52}Malheur à vous, docteurs de la loi ! Parce que vous avez enlevé la clef de la science. Vous n'êtes pas entrés vous-mêmes, et vous avez empêché ceux qui entraient.
\VS{53}Et comme il leur disait ces choses, les scribes et les pharisiens commencèrent à le presser violemment, et à le faire parler sur beaucoup de choses,
\VS{54}lui dressant des pièges, et cherchant à tirer quelque chose de sa bouche, afin de l'accuser.
\TextTitle{[Enseignements divers de Jésus]
\\(cp. Mt. 16:6-12 ; Mc. 8:14-21}
\Chap{12}
\VerseOne{}Cependant les gens s'étaient rassemblés par milliers, au point de s'écraser les uns les autres. Jésus se mit à dire à ses disciples : Avant tout, gardez-vous surtout du levain des pharisiens qui est l'hypocrisie.
\VS{2}Car il n'y a rien de caché, qui ne doive être révélé, ni de secret, qui ne doive être connu.
\VS{3}C'est pourquoi tout ce que vous aurez dit dans les ténèbres, sera entendu dans la lumière ; et ce que vous aurez dit à l'oreille dans les chambres, sera prêché sur les toits.
\VS{4}Je vous dis à vous qui êtes mes amis : Ne craignez pas ceux qui tuent le corps, et qui après cela ne peuvent rien faire de plus.
\VS{5}Je vous montrerai qui vous devez craindre. Craignez celui qui, après avoir tué, a le pouvoir de jeter dans la géhenne ; oui, vous dis-je, craignez celui-là.
\VS{6}Ne vend-on pas cinq petits passereaux pour deux sous ? Cependant, aucun d'eux n'est oublié devant Dieu.
\VS{7}Et même les cheveux de votre tête sont tous comptés. Ne craignez donc point ; vous valez plus que beaucoup de passereaux.
\VS{8}Or, je vous dis, quiconque me confessera devant les hommes, le Fils de l'homme le confessera aussi devant les anges de Dieu.
\VS{9}Mais quiconque me reniera devant les hommes, il sera renié devant les anges de Dieu.
\VS{10}Et quiconque parlera contre le Fils de l'homme, il lui sera pardonné ; mais celui qui aura blasphémé contre le Saint-Esprit\FTNT{Voir commentaire en Mt. 12:32.}, il ne lui sera point pardonné.
\VS{11}Quand ils vous mèneront devant les synagogues, les magistrats et les autorités, ne vous inquiétez pas de la manière dont vous vous défendrez ni de ce que vous aurez à dire.
\VS{12}Car le Saint-Esprit vous enseignera à l'heure même ce qu'il faudra dire.
\TextTitle{Parabole du riche insensé}
\VS{13}Et quelqu'un de la foule lui dit : Maître, dis à mon frère qu'il partage avec moi notre héritage.
\VS{14}Mais il lui répondit : Ô homme ! Qui m'a établi sur vous pour être votre juge, et pour faire vos partages ?
\VS{15}Puis il leur dit : Gardez-vous avec soin de toute avarice ; car quoique les biens de quelqu'un abondent, il n'a pas la vie par ses biens.
\VS{16}Et il leur dit cette parabole : Les champs d'un homme riche avaient beaucoup rapporté.
\VS{17}Et il raisonnait en lui-même, disant : Que ferai-je, car je n'ai pas assez de place pour recueillir mes fruits ?
\VS{18}Puis il dit : Voici ce que je ferai : J'abattrai mes greniers et j'en bâtirai de plus grands, et j'y amasserai toute ma récolte et tous mes biens.
\VS{19}Puis je dirai à mon âme : Mon âme, tu as beaucoup de biens assemblés pour beaucoup d'années, repose-toi, mange, bois, et réjouis-toi.
\VS{20}Mais Dieu lui dit : Insensé ! Cette même nuit ton âme te sera redemandée ; et ces choses que tu as préparées, à qui seront-elles ?
\VS{21}Il en est ainsi de celui qui amasse des biens pour lui-même, et qui n'est pas riche en Dieu.
\TextTitle{Exhortation à se confier en Dieu}
\VS{22}Jésus dit à ses disciples : C'est pourquoi je vous dis : Ne vous inquiétez pas pour votre vie, de ce que vous mangerez, ni pour votre corps, de quoi vous serez vêtus.
\VS{23}La vie est plus que la nourriture, et le corps est plus que le vêtement.
\VS{24}Considérez les corbeaux, ils ne sèment, ni ne moissonnent, et ils n'ont point de cellier, ni de grenier, et cependant Dieu les nourrit. Combien ne valez-vous pas plus que les oiseaux ?
\VS{25}Qui de vous qui par ses inquiétudes peut ajouter une coudée à la durée de sa vie ?
\VS{26}Si donc vous ne pouvez pas même la moindre chose, pourquoi êtes-vous inquiets du reste ?
\VS{27}Considérez comment croissent les lis, ils ne travaillent, ni ne filent, et cependant je vous dis que Salomon même, dans toute sa gloire, n'a pas été vêtu comme l'un d'eux.
\VS{28}Si Dieu revêt ainsi l'herbe qui est aujourd'hui au champ, et qui demain sera jetée au four, à combien plus forte raison vous vêtira-t-il, ô gens de petite foi ?
\VS{29}Ne dites donc point : Que mangerons-nous, ou que boirons-nous ? Et ne soyez pas inquiets,
\VS{30}car toutes ces choses, ce sont les païens du monde qui les recherchent. Votre Père sait que vous en avez besoin.
\VS{31}Mais cherchez plutôt le Royaume de Dieu, et toutes ces choses vous seront données par-dessus.
\VS{32}Ne crains point petit troupeau, car il a plu à votre Père de vous donner le Royaume.
\VS{33}Vendez ce que vous avez, et donnez-le en aumône. Faites-vous des bourses qui ne s'usent point, un trésor dans les cieux qui ne défaille jamais, et où le voleur n'approche point, et où la teigne ne gâte rien.
\VS{34}Car là où est votre trésor, là sera aussi votre cœur.
\TextTitle{Importance de veiller en attendant le Maître\FTNTT{Mt. 24:36-25:30}}
\VS{35}Que vos reins soient ceints, et vos lampes allumées.
\VS{36}Et soyez semblables aux serviteurs qui attendent que leur maître revienne des noces, afin de lui ouvrir dès qu'il frappera.
\VS{37}Heureux ces serviteurs que le maître à son arrivée, trouvera veillant ! En vérité je vous le dis, il se ceindra, les fera mettre à table, et s'approchera pour les servir.
\VS{38}Qu'il arrive à la seconde veille ou à la troisième veille, heureux ces serviteurs, s'il les trouve veillant !
\VS{39}Or sachez ceci, si le père de famille savait à quelle heure le voleur doit venir, il veillerait, et ne laisserait pas percer sa maison.
\VS{40}Vous donc aussi tenez-vous prêts, car le Fils de l'homme viendra à l'heure où vous n'y penserez pas.
\TextTitle{Parabole des deux serviteurs}
\VS{41}Pierre lui dit : Seigneur, dis-tu cette parabole pour nous, ou aussi pour tous ?
\VS{42}Et le Seigneur dit : Quel est donc l'économe fidèle et prudent, que le maître établira sur les domestiques de sa maison pour leur donner la nourriture au temps convenable ?
\VS{43}Heureux ce serviteur, que son maître, à son arrivée, trouvera faisant ainsi.
\VS{44}Je vous le dis en vérité, il l'établira sur tous ses biens.
\VS{45}Mais si ce serviteur dit en son cœur : Mon maître tarde longtemps à venir, s'il se met à battre les serviteurs et les servantes, à manger, à boire et à s'enivrer,
\VS{46}le maître de cet esclave-là viendra en un jour qu'il n'attend pas, et à une heure qu'il ne sait pas, et il le coupera en deux\FTNT{Le mot grec « dichotomeo » signifie « couper en deux parts », « de couper quelqu'un en deux », « châtiant en coupant », « fléau sévère ». Certains peuples, dont les Hébreux, employaient cette méthode cruelle comme châtiment corporel.}, et lui donnera sa part avec les infidèles.
\VS{47}Or le serviteur qui a connu la volonté de son maître, et qui ne s'est pas tenu prêt, et n'a point fait selon sa volonté, sera battu de plusieurs coups.
\VS{48} Mais celui qui ne l'a point connue, et qui a fait des choses dignes de châtiment, sera battu de peu de coups. Et il sera beaucoup redemandé à quiconque il aura été beaucoup donné; et on exigera plus de celui à qui on aura beaucoup confié.
\TextTitle{Jésus objet de divisions}
\VS{49}Je suis venu jeter un feu sur la terre, et qu'ai-je à désirer, s'il est déjà allumé ?
\VS{50}Il est un baptême dont je dois être baptisé, et combien suis-je pressé jusqu'à ce qu'il soit accompli.
\VS{51}Pensez-vous que je sois venu apporter la paix sur la terre ? Non, vous dis-je ; mais plutôt la division.
\VS{52}Car désormais cinq dans une maison, seront divisés, trois contre deux, et deux contre trois.
\VS{53}Le père sera divisé contre le fils, et le fils contre le père ; la mère contre la fille, et la fille contre la mère ; la belle-mère contre sa belle-fille, et la belle-fille contre sa belle-mère.
\VS{54}Puis il dit encore aux foules : Quand vous voyez un nuage se lever à l'occident, vous dites aussitôt : La pluie vient, et cela arrive ainsi.
\VS{55}Et quand vous voyez souffler le vent du midi, vous dites qu'il fera chaud ; et cela arrive.
\VS{56}Hypocrites, vous savez bien discerner l'aspect du ciel et de la terre ; et comment ne discernez-vous point cette saison ?
\VS{57}Et pourquoi aussi ne reconnaissez-vous pas de vous-mêmes ce qui est juste ?
\VS{58}Or quand tu vas avec ton adversaire devant le magistrat, tâche en chemin de t'en délivrer, de peur qu'il ne te traîne devant le juge, et que le juge ne te livre à l'officier de justice et que celui-ci ne te mette en prison.
\VS{59}Je te le dis, tu ne sortiras pas de là que tu n'aies payé jusqu'au dernier pite\FTNT{Petite pièce de monnaie en laiton. Voir en annexe le « tableau des monnaies au temps de Jésus-Christ ».}.
\Chap{13}
\TextTitle{Exhortation à la repentance}
\VerseOne{}En ce même temps, quelques-uns qui se trouvaient là présents racontèrent à Jésus ce qui était arrivé à des Galiléens, dont Pilate avait mêlé le sang avec celui de leurs sacrifices.
\VS{2}Et Jésus répondant leur dit : Croyez-vous que ces Galiléens étaient de plus grands pécheurs que tous les autres Galiléens, parce qu'ils ont souffert de la sorte ?
\VS{3}Non, vous dis-je ; mais si vous ne vous repentez pas, vous périrez tous de la même manière.
\VS{4}Ou bien, ces dix-huit personnes sur qui est tombée la tour de Siloé et qu'elle a tuées, croyez-vous qu'elles étaient plus coupables que tous les habitants de Jérusalem ?
\VS{5}Non, vous dis-je ; mais si vous ne vous repentez pas, vous périrez tous de la même manière.
\TextTitle{Parabole du figuier stérile et le jugement différé d'Israël\FTNTT{cp. Mt. 21:18-21}}
\VS{6}Il disait aussi cette parabole : Un homme avait un figuier planté dans sa vigne. Il vint pour y chercher du fruit, mais il n'en trouva point.
\VS{7}Et il dit au vigneron : Voilà trois ans que je viens chercher du fruit à ce figuier, et je n'en trouve point. Coupe-le ; pourquoi occupe-t-il inutilement la terre ?
\VS{8}Et le vigneron lui répondit : Seigneur, laisse-le encore pour cette année, je creuserai tout autour, et j'y mettrai du fumier.
\VS{9}Peut-être portera-t-il du fruit ; sinon, tu le couperas après cela.
\TextTitle{Guérison de la femme courbée le jour du sabbat}
\VS{10}Or comme il enseignait dans une de leurs synagogues un jour de sabbat,
\VS{11}voici, il y avait là une femme qui était possédée d'un démon qui la rendait infirme depuis dix-huit ans, et elle était courbée, et ne pouvait nullement se redresser.
\VS{12}Et quand Jésus la vit, il l'appela, et lui dit : Femme, tu es délivrée de ton infirmité.
\VS{13}Et il lui imposa les mains ; et à l'instant elle se redressa, et glorifia Dieu.
\VS{14}Mais le chef de la synagogue, indigné de ce que Jésus avait opéré cette guérison un jour du sabbat, prenant la parole dit à l'assemblée : Il y a six jours pour travailler ; venez donc vous faire guérir ces jours-là, et non pas le jour du sabbat.
\VS{15}Hypocrites ! lui répondit le Seigneur, chacun de vous ne détache-t-il pas son bœuf ou son âne de la crèche le jour du sabbat, et ne les mène-t-il pas boire ?
\VS{16}Et ne fallait-il pas délier de ce lien le jour du sabbat cette femme qui est fille d'Abraham, et que Satan tenait liée depuis dix-huit ans ?
\VS{17}Comme il disait ces choses, tous ses adversaires étaient confus ; mais toutes les foules se réjouissaient de toutes les choses glorieuses qu'il opérait.
\TextTitle{Parabole du grain de moutarde et du levain\FTNTT{voir Mt. 13:31,33}}
\VS{18}Il disait aussi : A quoi est semblable le Royaume de Dieu, et à quoi le comparerai-je ?
\VS{19}Il est semblable au grain de semence de moutarde qu'un homme a pris et jeté dans son jardin ; il pousse, devient un grand arbre, et les oiseaux du ciel font leurs nids dans ses branches.
\VS{20}Il dit encore : A quoi comparerai-je le Royaume de Dieu ?
\VS{21}Il est semblable au levain qu'une femme a pris et mis dans trois mesures de farine, pour faire lever toute la pâte.
\TextTitle{Enseignements de Jésus sur le chemin de Jérusalem}
\VS{22}Puis il s'en allait par les villes et les villages, enseignant, et faisant route vers Jérusalem.
\VS{23}Quelqu'un lui dit : Seigneur, n'y a-t-il que peu de gens qui soient sauvés ? Il leur répondit :
\VS{24}Efforcez-vous d'entrer par la porte étroite. Car je vous le dis que beaucoup chercheront à entrer, et ne le pourront pas.
\VS{25}Quand le père de famille se sera levé, et aura fermé la porte, et que vous, étant dehors, vous vous mettrez à frapper à la porte, en disant : Seigneur ! Seigneur ! Ouvre-nous ! Il vous répondra : Je ne sais pas d'où vous êtes.
\VS{26}Alors vous vous mettrez à dire : Nous avons mangé et bu en ta présence, et tu as enseigné dans nos rues.
\VS{27}Mais il dira : Je vous le dis, je ne sais pas d'où vous êtes. Retirez-vous de moi, vous tous qui faites le métier d'iniquité.
\VS{28}C'est là qu'il y aura des pleurs et des grincements de dents, quand vous verrez Abraham, Isaac, et Jacob, et tous les prophètes dans le Royaume de Dieu, et que vous serez jetés dehors.
\VS{29}Il en viendra aussi d'orient et d'occident, du nord et du sud, et ils se mettront à table dans le Royaume de Dieu.
\VS{30}Et voici, ceux qui sont les derniers seront les premiers, et ceux qui sont les premiers seront les derniers.
\VS{31}En ce même jour, quelques pharisiens vinrent à lui et lui dirent : Retire-toi et va-t'en d'ici, car Hérode veut te tuer.
\VS{32}Il leur répondit : Allez, et dites à ce renard : Voici, je chasse les démons et j'achève de faire des guérisons aujourd'hui et demain, et le troisième jour je prends fin.
\VS{33}C'est pourquoi il me faut marcher aujourd'hui et demain, et le jour suivant ; car il ne convient pas qu'un prophète meure hors de Jérusalem.
\TextTitle{Lamentations de Jésus sur Jérusalem\FTNTT{Mt. 23:37-39 ; Lu. 19:41-44 ; cp. Jé. 22:5}}
\VS{34}Jérusalem, Jérusalem, qui tues les prophètes et qui lapides ceux qui te sont envoyés ; combien de fois ai-je voulu rassembler tes enfants, comme la poule rassemble ses poussins sous ses ailes, et vous ne l'avez pas voulu !
\VS{30}Voici, votre maison va être déserte ; et je vous le dis en vérité, que vous ne me verrez plus, jusqu'à ce que vous disiez : Béni soit celui qui vient au nom du Seigneur.
\Chap{14}
\TextTitle{Jésus guérit un hydropique le jour du sabbat\FTNTT{cp. Mt. 12:9-13}}
\VerseOne{}Jésus entra un jour de sabbat dans la maison d'un des chefs des pharisiens pour prendre un repas, les pharisiens l'observaient.
\VS{2}Et voici, un homme hydropique était là devant lui.
\VS{3}Jésus prit la parole, et dit aux docteurs de la loi et aux pharisiens : Est-il permis, ou non, de faire une guérison le jour du sabbat ?
\VS{4}Ils gardèrent le silence. Alors Jésus prit le malade, le guérit, et le renvoya.
\VS{5}Puis s'adressant à lui, il leur dit : Lequel de vous, si son fils ou son bœuf tombe dans un puits, ne l'en retirera pas aussitôt, le jour du sabbat ?
\VS{6}Et ils ne pouvaient répliquer à ces choses.
\TextTitle{Parabole de l'invité}
\VS{7}Il proposa aussi aux conviés une parabole, en voyant qu'ils choisissaient les premières places ; et il leur dit :
\VS{8}Quand tu seras convié par quelqu'un à des noces, ne te mets pas à la première place à table, de peur qu'il ne se trouve parmi les conviés une personne plus honorable que toi,
\VS{9}et que celui qui vous a conviés l'un et l'autre ne vienne te dire : Cède ta place à cette personne-là. Tu aurais alors honte d'aller occuper la dernière place.
\VS{10}Mais lorsque tu seras convié, va te mettre à la dernière place, afin que quand celui qui t'a convié viendra, il te dise : Mon ami, monte plus haut. Alors cela te fera honneur devant tous ceux qui seront à table avec toi.
\VS{11}Car quiconque s'élève, sera abaissé ; et quiconque s'abaisse, sera élevé.
\VS{12}Il dit aussi à celui qui l'avait convié : Lorsque tu fais un dîner ou un souper, n'invite pas tes amis, ni tes frères, ni tes parents, ni tes riches voisins ; de peur qu'ils ne te convient à leur tour, et qu'on ne te rende la pareille.
\VS{13}Mais, lorsque tu donneras un festin, convie les pauvres, les impotents, les boiteux et les aveugles.
\VS{14}Et tu seras heureux de ce qu'ils n'ont pas de quoi te rendre la pareille ; car elle te sera rendue à la résurrection des justes.
\TextTitle{Parabole du grand festin\FTNTT{Mt. 22:1-14}}
\VS{15}Un de ceux qui étaient à table, ayant entendu ces paroles, lui dit : Heureux celui qui mangera du pain dans le Royaume de Dieu.
\VS{16}Et Jésus lui répondit : Un homme fit un grand festin, et il convia beaucoup de gens.
\VS{17}Et à l'heure du souper, il envoya son serviteur pour dire aux conviés : Venez, car tout est déjà prêt.
\VS{18}Mais ils commencèrent tous unanimement à s'excuser. Le premier lui dit : J'ai acheté un champ, et il me faut nécessairement partir pour aller le voir ; je te prie, excuse-moi.
\VS{19}Un autre dit : J'ai acheté cinq paires de bœufs, et je vais les essayer ; je te prie, excuse-moi.
\VS{20}Et un autre dit : J'ai épousé une femme, c'est pourquoi je ne puis aller.
\VS{21}Le serviteur, de retour, rapporta ces choses à son maître. Alors le père de famille irrité, dit à son serviteur : Va promptement dans les places et dans les rues de la ville, et amène ici les pauvres, les impotents, les boiteux et les aveugles.
\VS{22}Puis le serviteur dit : Maître, ce que tu as commandé a été fait, et il y a encore de la place.
\VS{23}Et le maître dit au serviteur : Va dans les chemins et le long des haies, et ceux que tu trouveras, contrains-les d'entrer, afin que ma maison soit remplie.
\VS{24}Car je vous dis, qu'aucun de ces hommes qui avaient été conviés ne goûtera de mon souper.
\TextTitle{Test de la consécration du disciple}
\VS{25}Or de grandes foules faisaient route avec Jésus. Il se retourna et leur dit :
\VS{26}Si quelqu'un vient à moi, et ne hait pas son père et sa mère, sa femme et ses enfants, ses frères et ses sœurs, et même sa propre vie, il ne peut être mon disciple.
\VS{27}Et quiconque ne porte pas sa croix, et ne me suit pas, ne peut être mon disciple.
\TextTitle{Parabole de la tour}
\VS{28}Car lequel de vous, s'il veut bâtir une tour, ne s'assied pas premièrement pour calculer la dépense et voir s'il a de quoi l'achever ?
\VS{29}De peur qu'après avoir posé les fondements, il ne puisse pas l'achever, et que tous ceux qui le verront ne commencent à se moquer de lui,
\VS{30}en disant : Cet homme a commencé à bâtir, et il n'a pas pu achever.
\TextTitle{Parabole du roi qui se prépare à la guerre}
\VS{31}Ou, quel roi, s'il va faire la guerre à un autre roi, ne s'assied pas premièrement pour examiner s'il peut, avec dix mille hommes, aller à la rencontre de celui qui vient contre lui avec vingt mille ?
\VS{32}Autrement, pendant que cet autre roi est encore loin, il lui envoie une ambassade pour demander la paix.
\VS{33}Ainsi donc, quiconque d'entre vous ne renonce pas à tout ce qu'il possède ne peut être mon disciple.
\TextTitle{Parabole du sel}
\VS{34}Le sel est bon ; mais si le sel perd sa saveur, avec quoi l'assaisonnera-t-on ?
\VS{35}Il n'est bon ni pour la terre, ni pour le fumier ; mais on le jette dehors. Que celui qui a des oreilles pour entendre, qu'il entende !
\Chap{15}
\TextTitle{Trois paraboles sur la repentance}
\VerseOne{}Or tous les publicains et les pécheurs s'approchaient de Jésus pour l'entendre.
\VS{2}Mais les pharisiens et les scribes murmuraient, disant : Cet homme reçoit les pécheurs, et mange avec eux.
\TextTitle{Parabole de la brebis perdue\FTNTT{Mt. 18:12-14}}
\VS{3}Mais il leur proposa cette parabole, disant :
\VS{4}Lequel d'entre vous, s'il a cent brebis, et qu'il en perd une, ne laisse pas les quatre-vingt-dix-neuf dans le désert, pour aller à la recherche de celle qui est perdue, jusqu'à ce qu'il la trouve ?
\VS{5}Et l'ayant retrouvée, il la met avec joie sur ses épaules,
\VS{6}et, de retour à la maison, il appelle ses amis et ses voisins, et il leur dit : Réjouissez-vous avec moi ; car j'ai trouvé ma brebis qui était perdue.
\VS{7}De même, je vous le dis il y aura plus de joie dans le ciel pour un seul pécheur qui se repent, que pour les quatre-vingt-dix-neuf justes qui n'ont pas besoin de repentance.
\TextTitle{Parabole de la drachme perdue}
\VS{8}Ou quelle femme, si elle a dix drachmes, et qu'elle en perde une, n'allume pas une lampe, ne balaie la maison, et ne cherche avec soin, jusqu'à ce qu'elle la trouve ?
\VS{9}Lorsqu'elle l'a trouvée, elle appelle ses amies et ses voisines, en leur disant : Réjouissez-vous avec moi ; car j'ai trouvé la drachme que j'avais perdue.
\VS{10}Ainsi je vous le dis, il y a de la joie devant les anges de Dieu pour un seul pécheur qui vient à se repentir.
\TextTitle{Parabole du fils perdu}
\VS{11}Il leur dit aussi : Un homme avait deux fils ;
\VS{12}et le plus jeune dit à son père : Mon père, donne-moi la part de bien qui m'appartient ; et il leur partagea ses biens.
\VS{13}Et peu de jours après, le plus jeune fils, ayant tout ramassé, partit pour un pays éloigné, où il dissipa son bien en vivant dans la débauche.
\VS{14}Et après qu'il eut tout dépensé, une grande famine survint dans ce pays-là, et il commença à se trouver dans la disette.
\VS{15}Alors il alla se mettre au service d'un des habitants du pays, qui l'envoya dans ses possessions pour paître les pourceaux.
\VS{16}Il aurait bien voulu se rassasier des carouges que les pourceaux mangeaient ; mais personne ne lui en donnait.
\VS{17}Or étant revenu à lui-même, il dit : Combien d'ouvriers chez mon père ont du pain en abondance, et moi je meurs de faim !
\VS{18}Je me lèverai, j'irai vers mon père, et je lui dirai : Mon père, j'ai péché contre le ciel et devant toi ;
\VS{19}et je ne suis plus digne d'être appelé ton fils ; traite-moi comme l'un de tes ouvriers.
\VS{20}Il se leva donc, et alla vers son père. Et comme il était encore loin, son père le vit et fut ému de compassion, il courut se jeter à son cou et le baisa.
\VS{21}Mais le fils lui dit : Mon père, j'ai péché contre le ciel et devant toi ; et je ne suis plus digne d'être appelé ton fils.
\VS{22}Et le père dit à ses serviteurs : Apportez la plus belle robe et revêtez-le, mettez-lui un anneau au doigt, et des souliers aux pieds.
\VS{23}Amenez-moi le veau gras, et tuez-le. Mangeons et réjouissons-nous.
\VS{24}Car mon fils que voici, était mort, mais il est ressuscité ; il était perdu, mais il est retrouvé. Et ils commencèrent à se réjouir.
\VS{25}Or son fils aîné était dans les champs. Lorsqu'il revint et approcha de la maison, il entendit la musique et les danses.
\VS{26}Il appela un des serviteurs, et il lui demanda ce que c'était.
\VS{27}Ce serviteur lui dit : Ton frère est de retour, et ton père a tué le veau gras, parce qu'il l'a recouvré sain et sauf.
\VS{28}Mais il se mit en colère, et ne voulut pas entrer. Son père sortit et le pria d'entrer.
\VS{29}Mais il répondit, et dit à son père : voici, il y a tant d'années que je te sers, et jamais je n'ai transgressé ton commandement, et cependant tu ne m'as jamais donné un chevreau pour que je me réjouisse avec mes amis.
\VS{30}Mais quand ton fils est arrivé, celui qui a mangé ton bien avec des prostituées, c'est pour lui que tu as tué le veau gras.
\VS{31}Et le père lui dit : Mon enfant, tu es toujours avec moi, et tous mes biens sont à toi.
\VS{32}Or il fallait bien s'égayer et se réjouir, parce que ton frère que voici était mort et qu'il est ressuscité, parce qu'il était perdu et qu'il est retrouvé.
\Chap{16}
\TextTitle{Parabole de l'économe infidèle}
\VerseOne{}Il disait aussi à ses disciples : Il y avait un homme riche qui avait un économe, qui fut accusé devant lui comme dissipant ses biens.
\VS{2}Il l'appela et lui dit : Qu'est-ce que j'entends dire de toi ? Rends compte de ton administration ; car tu n'auras plus le pouvoir d'administrer mes biens.
\VS{3}Alors l'économe dit en lui-même : Que ferai-je, puisque mon maître m'ôte l'administration ? Travailler à la terre ? Je ne le puis. Mendier ? J'en ai honte.
\VS{4}Je sais ce que je ferai, afin que les gens me reçoivent dans leurs maisons quand mon administration me sera ôtée.
\VS{5}Alors il appela chacun des débiteurs de son maître, et il dit au premier : Combien dois-tu à mon maître ?
\VS{6}Il dit : Cent mesures d'huile. Et il lui dit : Prends ton billet, et assied-toi vite, et écris cinquante.
\VS{7}Puis il dit à un autre : Et toi, combien dois-tu ? Il dit : Cent mesures de froment. Et il lui dit : Prends ton billet, et écris quatre-vingts.
\VS{8}Et le maître loua l'économe infidèle de ce qu'il avait agi prudemment. Ainsi les enfants de ce siècle sont plus prudents dans leur génération, que les enfants de lumière.
\VS{9}Et moi aussi je vous dis : Faites-vous des amis avec les richesses injustes ; afin que quand vous viendrez à manquer, ils vous reçoivent dans les tabernacles éternels.
\VS{10}Celui qui est fidèle en très peu de chose, est fidèle aussi dans les grandes choses ; et celui qui est injuste en très peu de chose, est injuste aussi dans les grandes choses.
\VS{11}Si donc vous n'avez pas été fidèles dans les richesses injustes, qui vous confiera les véritables richesses ?
\VS{12}Et si en ce qui est à autrui vous n’avez pas été fidèles, qui vous donnera ce qui est vôtre ?
\VS{13}Nul serviteur ne peut servir deux maîtres. Car, ou il haïra l'un, et aimera l'autre ; ou il s'attachera à l'un, et méprisera l'autre. Vous ne pouvez pas servir Dieu et Mamon\FTNT{Voir commentaire en Mt. 6:24.}.
\TextTitle{L'avarice condamnée par Jésus}
\VS{14}Or les pharisiens aussi, qui étaient avares, entendaient toutes ces choses, et ils se moquaient de lui.
\VS{15}Et il leur dit : Vous, vous cherchez à paraître justes devant les hommes ; mais Dieu connaît vos cœurs ; c'est pourquoi ce qui est élevé parmi les hommes est une abomination devant Dieu.
\VS{16}La loi et les prophètes ont duré jusqu'à Jean ; depuis lors, le Royaume de Dieu est prêché, et chacun y fait violence.
\VS{17}Or il est plus aisé que le ciel et la terre passent, qu'il ne l'est qu'un trait de la lettre de la loi vienne à tomber.
\TextTitle{Enseignement de Jésus sur le divorce\FTNTT{Mt. 5:31-32 ; 19:1-9 ; Mc. 10:2-12}}
\VS{18}Quiconque répudie sa femme, et se marie à une autre, commet un adultère, et quiconque prend celle qui a été répudiée par son mari, commet un adultère.
\TextTitle{Histoire de l'homme riche et de Lazare}
\VS{19}Il y avait un homme riche, qui était vêtu de pourpre et de fin lin, et qui tous les jours se réjouissait d'une vie somptueuse.
\VS{20}Il y avait un pauvre, nommé Lazare, couché à la porte du riche, tout couvert d'ulcères,
\VS{21}et qui désirait se rassasier des miettes qui tombaient de la table du riche ; et même les chiens venaient encore lécher ses ulcères.
\VS{22}Et il arriva que le pauvre mourut, et il fut porté par les anges dans le sein d'Abraham\FTNT{Contrairement aux idées reçues, le sein d'Abraham ne se trouvait pas au ciel. En effet, le Seigneur a dit que personne n'était monté au ciel si ce n'est lui-même (Jn. 3:13). Avant l'ère de la grâce, tous les morts allaient dans le séjour des morts où ils étaient retenus prisonniers par le dieu Hadès (voir commentaire sur l'enfer en Mt. 16:18). Toutefois, ce lieu était séparé en deux parties distinctes, l'une réservée aux impies, où ils y subissaient des tourments, et l'autre réservée aux personnes pieuses qui se tenaient en repos, sans souffrir. En effet, lorsque Saül fit appel à une voyante pour faire remonter Samuel du séjour des morts afin de le consulter, Samuel lui annonça qu'il le rejoindrait dès le lendemain à l'endroit où il se trouvait (1 S. 28:19). De plus, dans le récit de la mort du pauvre Lazare et du riche, deux points importants sont à noter. D'une part, bien qu'étant séparés l'un de l'autre, ils pouvaient se voir et communiquer ensemble (Lu. 16:23-26). D'autre part, il est évident que le riche souffrait tandis que le pauvre Lazare était consolé (Lu. 16:25). Lorsque le Seigneur est mort, il est descendu dans « les régions inférieures de la terre » pour délivrer les captifs pieux qui avaient vécu avant Jésus-Christ (Ep. 4:8-9 ; 1 S. 2:6). Par la même occasion, il confirma la condamnation des impies (1 Pi. 3:19). Maintenant que Jésus-Christ est mort et ressuscité, tous ceux qui meurent dans le Seigneur vont au ciel (2 Co. 5:1-3. ; Ph. 1:22-23).}. Le riche mourut aussi, et il fut enseveli.
\VS{23}Etant en enfer\FTNT{Le mot traduit par « enfer » vient du grec « Hades ». Voir commentaire Mt. 16:18}, il leva ses yeux ; et, tandis qu'il était dans les tourments, il vit de loin Abraham et Lazare dans son sein.
\VS{24}Il s'écria : Père Abraham aie pitié de moi, et envoie Lazare, pour qu'il trempe le bout de son doigt dans l'eau et me rafraichisse la langue ; car je suis grièvement tourmenté dans cette flamme.
\VS{25}Abraham répondit : Mon enfant, souviens-toi que tu as reçu tes biens pendant ta vie, et que Lazare a eu ses maux pendant la sienne ; maintenant il est ici consolé, et toi, tu es grièvement tourmenté.
\VS{26}D'ailleurs, il y a entre nous et vous un grand abîme ; en sorte que ceux qui veulent passer d'ici vers vous ne le peuvent, et que ceux qui veulent passer de là ne traversent pas non plus vers nous.
\VS{27}Et il dit : Je te prie donc, père, de l'envoyer dans la maison de mon père ; car j'ai cinq frères.
\VS{28}Afin qu'il leur rende témoignage de l'état où je suis ; de peur qu'eux aussi ne viennent dans ce lieu de tourment.
\VS{29}Abraham lui répondit : Ils ont Moïse et les prophètes ; qu'ils les écoutent.
\VS{30}Mais il dit : Non, père Abraham, mais si quelqu'un des morts va vers eux, ils se repentiront.
\VS{31}Et Abraham lui dit : S'ils n'écoutent pas Moïse et les prophètes, ils ne seront pas non plus persuadés quand quelqu'un des morts ressusciterait.
\Chap{17}
\TextTitle{Instructions de Jésus au sujet des scandales, du pardon et de la foi\FTNTT{Mt. 5:31-32 ; 19:1-9 ; Mc. 10:2-12}}
\VerseOne{}Or il dit à ses disciples : Il est impossible qu'il n'arrive pas des scandales ; mais malheur à celui par qui ils arrivent.
\VS{2}Il vaudrait mieux pour lui qu'on lui mette une pierre de moulin autour de son cou, et qu'on le jette dans la mer, que de scandaliser un seul de ces petits.
\VS{3}Prenez garde à vous-mêmes. Si donc ton frère a péché contre toi, reprends-le ; et s'il se repent, pardonne-lui.
\VS{4}Et s'il a péché contre toi sept fois dans un jour et que sept fois il revienne à toi, disant : Je me repens, tu lui pardonneras.
\VS{5}Alors les apôtres dirent au Seigneur : Augmente-nous la foi.
\VS{6}Et le Seigneur dit : Si vous aviez de la foi aussi gros qu'un grain de semence de moutarde, vous diriez à ce sycomore : Déracine-toi, et plante-toi dans la mer ; et il vous obéirait.
\TextTitle{Les serviteurs inutiles}
\VS{7}Mais qui de vous, ayant un serviteur qui laboure ou paît les troupeaux, lui dira, quand il revient des champs : Approche-toi vite, et mets-toi à table.
\VS{8}Ne lui dira-t-il pas plutôt : Prépare-moi à souper, ceins-toi, et sers-moi jusqu'à ce que j'aie mangé et bu ; et après cela tu mangeras et tu boiras ?
\VS{9}Doit-il de la reconnaissance à ce serviteur parce qu'il a fait ce qui lui était ordonné ? Je ne le pense pas.
\VS{10}Vous de même, quand vous aurez fait tout ce qui vous a été ordonné, dites : Nous sommes des serviteurs inutiles ;  ce que nous étions obligés de faire, nous l'avons fait.
\TextTitle{Guérison de dix lépreux}
\VS{11}Et il arriva qu’en allant à Jérusalem, il passait par le milieu de la Samarie, et de la Galilée.
\VS{12}Et comme il entrait dans un village, dix hommes lépreux vinrent à sa rencontre. Se tenant à distance, ils élevèrent la voix, et dirent :
\VS{13}Jésus, Maître, aie pitié de nous !
\VS{14}Et quand il les eut vus, il leur dit : Allez, montrez-vous aux sacrificateurs\FTNT{Lé. 13.}. Et, pendant qu'ils y allaient, ils furent purifiés.
\VS{15}L'un d'eux se voyant guéri, revint sur ses pas, glorifiant Dieu à haute voix.
\VS{16}Et il se jeta en terre sur sa face aux pieds de Jésus, lui rendant grâces. Or c’était un Samaritain.
\VS{17}Alors Jésus prenant la parole, dit : Les dix n'ont-ils pas été rendus purs ? Et les neuf autres, où sont-ils ?
\VS{18}Il n'y a eu que cet étranger qui soit revenu pour rendre gloire à Dieu.
\VS{19}Alors il lui dit : Lève-toi. Va, ta foi t'a sauvé.
\TextTitle{Les pharisiens demandent à voir le Royaume de Dieu\FTNTT{cp. Lu. 19:11-27}}
\VS{20}Or les pharisiens demandèrent à Jésus quand viendrait le Royaume de Dieu. Il leur répondit, et leur dit : Le Royaume de Dieu ne vient pas de manière à attirer l'attention.
\VS{21}Et on ne dira point : Il est ici ; ou : Il est là. Car voici, le Royaume de Dieu est au milieu de vous.
\TextTitle{Jésus annonce sa seconde venue\FTNTT{voir De. 30:3}}
\VS{22}Il dit aussi à ses disciples : Des jours viendront où vous désirerez voir un des jours du Fils de l'homme, mais vous ne le verrez point. On vous dira :
\VS{23}Il est ici, ou : Il est là. N'allez pas, et ne les suivez point.
\VS{24}Car, comme l'éclair brille et resplendit d'une extrémité du ciel à l'autre, ainsi sera le Fils de l'homme en son jour.
\VS{25}Mais il faut premièrement qu'il souffre beaucoup, et qu'il soit rejeté par cette génération.
\VS{26}Ce qui arriva aux jours de Noé, arrivera de même aux jours du Fils de l'homme.
\VS{27}On mangeait et on buvait ; on prenait et on donnait des femmes en mariage jusqu'au jour où Noé entra dans l'arche ; le déluge vint, et les fit tous périr.
\VS{28}C’est encore ce qui arriva aux jours de Lot : On mangeait, on buvait, on achetait, on vendait, on plantait et on bâtissait.
\VS{29}Mais le jour où Lot sortit de Sodome, une pluie de feu et de soufre tomba du ciel, et les fit tous périr.
\VS{30}Il en sera de même au jour où le Fils de l'homme paraîtra.
\VS{31}En ce jour-là, que celui qui sera sur le toit, et qui aura ses effets dans la maison, ne descende point pour les prendre ; et que celui qui sera dans les champs, ne retourne pas non plus à ce qui est resté en arrière.
\VS{32}Souvenez-vous de la femme de Lot.
\VS{33}Quiconque cherchera à sauver sa vie, la perdra ; et quiconque la perdra, la retrouvera.
\VS{34}Je vous dis, qu’en cette nuit-là deux seront dans un même lit : l’un sera pris, et l’autre laissé ;
\VS{35}deux femmes moudront ensemble, l’une sera prise et l’autre laissée ;
\VS{36}deux seront aux champs, l’un sera pris et l’autre laissé.
\VS{37}Les disciples lui dirent : Où, Seigneur ? Et il leur dit, Là où est le corps, là aussi s’assembleront les aigles.
\Chap{18}
\TextTitle{Parabole du juge inique}
\VerseOne{}Et il leur proposa une parabole, pour montrer qu'il faut toujours prier, et ne point se relâcher,
\VS{2}disant : Il y avait dans une ville un juge qui ne craignait point Dieu et qui ne respectait personne.
\VS{3}Et dans la même ville, il y avait une veuve, qui venait souvent lui dire : Fais-moi justice de ma partie adverse.
\VS{4}Pendant longtemps il refusa. Mais après cela il dit en lui-même : Quoique je ne craigne point Dieu, et que je ne respecte personne,
\VS{5}néanmoins, parce que cette veuve me donne de la peine, je lui ferai justice, de peur qu'elle ne vienne sans cesse me casser la tête.
\VS{6}Et le Seigneur dit : Ecoutez ce que dit le juge inique.
\VS{7}Et Dieu ne ferait-il point justice à ses élus, qui crient à lui jour et nuit, quoiqu’il use de patience avant d’intervenir pour eux ?
\VS{8}Je vous le dis que bientôt il les vengera. Mais quand le Fils de l'homme viendra, pensez-vous qu'il trouvera la foi sur la terre ?
\TextTitle{Parabole du pharisien et du publicain}
\VS{9}Il dit aussi cette parabole au sujet de certaines personnes se persuadant qu'elles étaient justes, et ne faisant aucun cas des autres :
\VS{10}Deux hommes montèrent au temple pour prier, l'un était pharisien, et l'autre, publicain.
\VS{11}Le pharisien, se tenant debout, priait en lui-même en ces termes : Ô Dieu ! Je te rends grâces de ce que je ne suis pas comme le reste des hommes, qui sont ravisseurs, injustes, adultères, ni même comme ce publicain.
\VS{12}Je jeûne deux fois la semaine, et je donne la dîme de tout ce que je possède.
\VS{13}Mais le publicain se tenant loin, n'osait même pas lever les yeux vers le ciel, mais il se frappait la poitrine, en disant : Ô Dieu ! Sois apaisé envers moi qui suis pécheur !
\VS{14}Je vous dis que celui-ci descendit dans sa maison justifié, plutôt que l'autre ; car quiconque s'élève, sera abaissé, et quiconque s'abaisse, sera élevé.
\TextTitle{Le Royaume des cieux, pour ceux qui ressemblent aux petits enfants\FTNTT{Mt. 19:13-15 ; Mc. 10:13-16}}
\VS{15}Et quelques-uns lui présentèrent aussi de petits enfants, afin qu'il les touchât, mais les disciples voyant cela, reprenaient ceux qui les présentaient.
\VS{16}Mais Jésus les appela, et dit : Laissez venir à moi les petits enfants, et ne les en empêchez pas ; car le Royaume de Dieu est pour ceux qui leur ressemblent.
\VS{17}Je vous le dis en vérité, quiconque ne recevra point comme un enfant le Royaume de Dieu, n’y entrera point.
\TextTitle{Jésus dénonce l'attachement aux richesses\FTNTT{Mt. 19:16-30 ; Mc. 10:17-31 ; cp. Lu. 10:25-37}}
\VS{18}Un chef interrogea Jésus et dit : Bon Maître, que dois-je faire pour hériter la vie éternelle ?
\VS{19}Jésus lui dit : Pourquoi m'appelles-tu bon ? Il n'y a de bon que Dieu seul\FTNT{Voir commentaire Mc. 10:18.}.
\VS{20}Tu connais les commandements : Tu ne commettras point d'adultère. Tu ne tueras point. Tu ne déroberas point. Tu ne diras point de faux témoignage. Honore ton père et ta mère.
\VS{21}Et il lui dit : J'ai observé toutes ces choses dès ma jeunesse.
\VS{22}Et quand Jésus eut entendu cela, lui dit : Il te manque encore une chose : Vends tout ce que tu as, et distribue-le aux pauvres, et tu auras un trésor dans les cieux. Puis viens, et suis-moi.
\VS{23}Lorsqu'il entendit ces choses, il devint tout triste, car il était extrêmement riche.
\VS{24}Jésus voyant qu'il était devenu tout triste, dit : Qu'il est difficile à ceux qui ont des richesses d'entrer dans le Royaume de Dieu !
\VS{25}Car il est plus facile à un chameau de passer par le trou d'une aiguille, qu'à un riche d'entrer dans le Royaume de Dieu\FTNT{Voir commentaire Mt. 19:24.}.
\VS{26}Ceux qui entendirent cela, dirent : Et qui peut donc être sauvé ?
\VS{27}Jésus leur répondit : Ce qui est impossible aux hommes est possible à Dieu.
\TextTitle{Récompense pour un vrai disciple de Jésus}
\VS{28}Pierre dit : Voici, nous avons tout quitté, et nous t'avons suivi.
\VS{29}Et il leur dit : Je vous le dis en vérité, il n'est personne qui, ayant quitté pour l'amour du Royaume de Dieu, sa maison, ou ses parents, ou ses frères, ou sa femme, ou ses enfants,
\VS{30}ne reçoive beaucoup plus dans ce siècle-ci, et dans le siècle à venir la vie éternelle.
\TextTitle{Jésus annonce à nouveau sa mort et sa résurrection\FTNTT{Mt. 20:17-19 ; Mc. 10:32-34}}
\VS{31}Jésus prit à part les douze, et il leur dit : Voici, nous montons à Jérusalem, et tout ce qui est écrit par les prophètes au sujet du Fils de l'homme, s'accomplira.
\VS{32}Car il sera livré aux Gentils ; on se moquera de lui, on l'outragera, et on lui crachera au visage,
\VS{33}et après l'avoir battu de verges, on le fera mourir ; mais il ressuscitera le troisième jour.
\VS{34}Mais ils ne comprirent rien à cela, et ce discours était si obscur pour eux qu'ils ne comprirent point ce qu'il leur disait.
\TextTitle{Bartimée voit !\FTNTT{cp. Mt. 20:29-34 ; Mc. 10:46-53}}
\VS{35}Or comme il approchait de Jéricho, un aveugle était assis au bord du chemin, et mendiait.
\VS{36}Et entendant la foule qui passait, il demanda ce que c'était.
\VS{37}Et on lui dit : C'est Jésus de Nazareth qui passe.
\VS{38}Alors il cria, disant : Jésus, Fils de David, aie pitié de moi !
\VS{39}Ceux qui marchaient devant le reprenaient, pour le faire taire ; mais il criait beaucoup plus fort : Fils de David, aie pitié de moi !
\VS{40}Et Jésus s'étant arrêté ordonna qu'on le lui amène ; et, quand il se fut approché,
\VS{41}il lui demanda : Que veux-tu que je te fasse ? Il répondit : Seigneur, que je recouvre la vue.
\VS{42}Jésus lui dit : Recouvre la vue ; ta foi t'a sauvé.
\VS{43}Et à l'instant il recouvra la vue et suivit Jésus, glorifiant Dieu. Et tout le peuple voyant cela, loua Dieu.
\Chap{19}
\TextTitle{Conversion de Zachée}
\VerseOne{}Jésus, étant entré dans Jéricho, traversait la ville.
\VS{2}Et voici, un homme riche, appelé Zachée, chef des publicains, cherchait à voir qui était Jésus,
\VS{3}mais il ne le pouvait pas à cause de la foule, car il était de petite taille.
\VS{4}C'est pourquoi il accourut devant, et monta sur un sycomore pour le voir ; car il devait passer par là.
\VS{5}Et quand Jésus fut arrivé à cet endroit-là, il leva les yeux, le vit, et lui dit : Zachée, hâte-toi de descendre ; car il faut que je demeure aujourd'hui dans ta maison.
\VS{6}Zachée se hâta de descendre, et le reçut avec joie.
\VS{7}Et tous voyant cela murmuraient, et disaient : Il est entré chez un homme pécheur pour y loger.
\VS{8}Et Zachée, se présentant devant le Seigneur, lui dit : Voici, Seigneur, je donne la moitié de mes biens aux pauvres ; et si j'ai fait tort de quelque chose à quelqu'un, je lui rends le quadruple\FTNT{Lé. 5:20-24.}.
\VS{9}Et Jésus lui dit : Aujourd'hui le salut est entré dans cette maison ; parce que celui-ci aussi est fils d'Abraham.
\VS{10}Car le Fils de l'homme est venu chercher et sauver ce qui était perdu.
\TextTitle{Parabole des dix mines\FTNTT{Lu. 17:21}}
\VS{11}Et comme ils entendaient ces choses, Jésus poursuivit son discours, et proposa une parabole, parce qu'il était près de Jérusalem, et qu'ils pensaient que le royaume de Dieu allait immédiatement paraître.
\VS{12}Il dit donc : Un homme noble s'en alla dans un pays éloigné, pour prendre possession d'un Royaume, et revenir ensuite.
\VS{13}Il appela dix de ses serviteurs, il leur donna dix mines et leur dit : Faites-les valoir jusqu'à ce que je revienne.
\VS{14}Or ses concitoyens le haïssaient, c'est pourquoi ils envoyèrent après lui une ambassade, pour dire : Nous ne voulons pas que cet homme règne sur nous.
\VS{15} Il arriva donc après qu'il fut de retour, et après avoir pris possession du Royaume, qu'il fit appeler auprès de lui les serviteurs auxquels il avait confié son argent, afin de connaître comment chacun l'avait fait valoir.
\VS{16}Alors le premier vint, et dit : Seigneur, ta mine a produit dix autres mines.
\VS{17}Il lui dit : C'est bien, bon serviteur ; parce que tu as été fidèle en peu de choses, reçois le gouvernement de dix villes.
\VS{18}Et le second vint, et dit : Seigneur, ta mine a produit cinq autres mines.
\VS{19}Il dit aussi à celui-ci : Toi aussi, sois établi sur cinq villes.
\VS{20}Un autre vint, et dit : Seigneur, voici ta mine que j'ai gardée enveloppée dans un linge ;
\VS{21}car j'avais peur de toi, parce que tu es un homme sévère ; tu prends ce que tu n'as point déposé, et tu moissonnes ce que tu n'as pas semé.
\VS{22}Il lui dit : Méchant serviteur, je te jugerai sur tes propres paroles : Tu savais que je suis un homme sévère, prenant ce que je n'ai point déposé, et moissonnant ce que je n'ai point semé.
\VS{23}Pourquoi donc n'as-tu pas mis mon argent dans une banque, afin qu'à mon retour je le retire avec un intérêt ?
\VS{24}Alors il dit à ceux qui étaient présents : Ôtez-lui la mine, et donnez-la à celui qui a les dix.
\VS{25}Ils lui dirent : Seigneur, il a dix mines.
\VS{26}Ainsi je vous le dis, on donnera à celui qui a, mais à celui qui n'a pas, on ôtera ce qu'il a.
\VS{27}Au reste, amenez ici mes ennemis qui n'ont pas voulu que je règne sur eux, et tuez-les devant moi.
\TextTitle{Jésus fait son entrée à Jérusalem\FTNTT{Za. 9:9 ; Mt. 21:1-11 ; Mc. 11:1-11 ; Jn. 12:12-19}}
\VS{28}Après avoir ainsi parlé, Jésus marcha devant la foule, pour monter à Jérusalem.
\VS{29}Lorsqu'il approcha de Bethphagé et de Béthanie, vers la montagne appelée Montagne des Oliviers, Jésus envoya deux de ses disciples,
\VS{30}en leur disant : Allez au village qui est en face ; quand vous y serez entrés, vous trouverez un ânon attaché, sur lequel aucun homme n'est monté ; détachez-le, et amenez-le-moi.
\VS{31}Si quelqu'un vous demande pourquoi le détachez-vous, vous lui répondrez : Le Seigneur en a besoin.
\VS{32}Et ceux qui étaient envoyés s'en allèrent, et trouvèrent l'ânon comme il le leur avait dit.
\VS{33}Comme ils le détachaient, ses maîtres leur dirent : Pourquoi détachez-vous cet ânon ?
\VS{34}Ils répondirent : Le Seigneur en a besoin.
\VS{35}Ils emmenèrent à Jésus l'ânon, sur lequel ils jetèrent leurs vêtements, et firent monter Jésus dessus.
\VS{36}Quand il fut en marche, les gens étendirent leurs vêtements sur le chemin.
\VS{37}Et lorsque déjà il approchait de Jérusalem, vers la descente de la Montagne des Oliviers, toute la multitude des disciples saisie de joie, se mit à louer Dieu à haute voix, pour tous les miracles qu'ils avaient vus.
\VS{38}Ils disaient : Béni soit le Roi qui vient au Nom du Seigneur\FTNT{Ps. 118:26.} ! Paix dans le ciel, et gloire dans les lieux très hauts.
\VS{39}Quelques pharisiens, du milieu de la foule, lui dirent : Maître, reprends tes disciples.
\VS{40}Et Jésus répondit : Je vous le dis, s'ils se taisent, les pierres crieront.
\TextTitle{Nouvelles lamentations de Jésus sur Jérusalem\FTNTT{cp. Mt. 23:37-39 ; Lu. 13:34-35}}
\VS{41}Comme il approchait de la ville, Jésus, en la voyant, pleura sur elle, et dit :
\VS{42}Ô ! Si toi aussi, au moins en ce jour qui t'est donné, tu connaissais les choses qui appartiennent à ta paix ! Mais maintenant elles sont cachées à tes yeux.
\VS{43}Il viendra sur toi des jours où tes ennemis t'environneront de tranchées, t'enfermeront, et te serreront de tous côtés ;
\VS{44}ils te raseront, toi et tes enfants qui sont au milieu de toi, et ils ne laisseront pas en toi pierre sur pierre, parce que tu n'as pas connu le temps de ta visitation.
\TextTitle{Jésus chasse les marchands du temple}
\VS{45}Il entra dans le temple, et il se mit à chasser dehors ceux qui vendaient et qui achetaient.
\VS{46}Leur disant : Il est écrit : Ma maison sera appelée la maison de prière ; mais vous, vous en avez fait une caverne de voleurs\FTNT{Es. 56:7 ; Jé. 7:11.}.
\VS{47}Il enseignait tous les jours dans le temple. Et les principaux sacrificateurs et les scribes cherchaient à le faire mourir.
\VS{48}Mais ils ne savaient comment s'y prendre ; car tout le peuple s'attachait à ses paroles.
\Chap{20}
\TextTitle{L'autorité de Jésus et celle de Jean-Baptiste\FTNTT{Mt. 21:23-27 ; Mc. 11:27-33}}
\VerseOne{}Et il arriva un de ces jours-là, comme Jésus enseignait le peuple dans le temple, et qu'il évangélisait, les principaux sacrificateurs, les scribes et les anciens survinrent,
\VS{2}et lui parlèrent en disant : Dis-nous par quelle autorité fais-tu ces choses, ou qui est celui qui t'a donné cette autorité ?
\VS{3}Jésus leur répondit : Je vous adresserai aussi une question, et répondez-moi.
\VS{4}Le baptême de Jean venait-il du ciel ou des hommes ?
\VS{5}Ils raisonnaient entre eux, disant : Si nous répondons : Du ciel ; il dira : Pourquoi n'avez-vous pas cru en lui ?
\VS{6}Et si nous répondons : Des hommes, tout le peuple nous lapidera ; car il est persuadé que Jean était un prophète ;
\VS{7}Alors, ils répondirent qu'ils ne savaient d'où il était.
\VS{8}Et Jésus leur dit : Moi non plus, je ne vous dirai pas par quelle autorité je fais ces choses.
\TextTitle{Parabole des vignerons\FTNTT{Es. 5:1-7 ; Mt. 21:33-46 ; Mc. 12:1-12}}
\VS{9}Alors il se mit à dire au peuple cette parabole : Un homme planta une vigne, et la loua à des vignerons, et fut longtemps absent.
\VS{10}Et à la saison de la récolte, il envoya un serviteur vers les vignerons, afin qu'ils lui donnent du fruit de la vigne. Les vignerons le battirent, et le renvoyèrent à vide.
\VS{11}Il leur envoya encore un autre serviteur ; mais ils le battirent aussi, et après l'avoir traité indignement, ils le renvoyèrent à vide.
\VS{12}Il en envoya encore un troisième, mais ils le blessèrent aussi, et le jetèrent dehors.
\VS{13}Alors le maître de la vigne dit : Que ferai-je ? J'enverrai mon fils bien-aimé ; peut-être que quand ils le verront, ils le respecteront.
\VS{14}Mais quand les vignerons le virent, ils raisonnèrent entre eux, et dirent : Voici l'héritier ; venez, tuons-le, afin que l'héritage soit à nous.
\VS{15}Et ils le jetèrent hors de la vigne, et le tuèrent. Que leur fera donc le maître de la vigne ?
\VS{16}Il viendra, et fera périr ces vignerons-là, et il donnera la vigne à d'autres. Lorsqu'ils entendirent cela, ils dirent : A Dieu ne plaise !
\VS{17}Alors il les regarda, et dit : Que signifie donc ce qui est écrit : La pierre qu'on rejetée ceux qui bâtissaient est devenue la principale de l'angle\FTNT{Ps. 118:22.} ?
\VS{18}Quiconque tombera sur cette pierre, sera brisé ; et elle écrasera celui sur qui elle tombera.
\TextTitle{Le tribut à César\FTNTT{Mt. 22:15-22 ; Mc. 12:13-17}}
\VS{19}Les principaux sacrificateurs et les scribes cherchèrent à mettre la main sur lui à l'heure même, mais ils craignirent le peuple. Ils avaient compris que c'était pour eux que Jésus avait dit cette parabole.
\VS{20}Ils se mirent à observer Jésus ; et ils envoyèrent des agents secrets, qui feignaient d'être justes, pour lui tendre des pièges et saisir de lui quelque parole afin de le livrer au magistrat et à l'autorité du gouverneur.
\VS{21}Ils l'interrogèrent, en disant : Maître, nous savons que tu parles et enseignes conformément à la justice, et que tu ne regardes pas à l'apparence des personnes, mais que tu enseignes la voie de Dieu selon la vérité.
\VS{22}Nous est-il permis de payer le tribut à César, ou non ?
\VS{23}Jésus, apercevant leur ruse, leur dit : Pourquoi me tentez-vous ?
\VS{24}Montrez-moi un denier. De qui a-t-il l'image et l'inscription ? Ils lui répondirent : De César.
\VS{25}Alors il leur dit : Rendez donc à César ce qui est à César ; et à Dieu ce qui est à Dieu.
\VS{26}Ainsi ils ne purent le surprendre dans ses paroles devant le peuple ; mais, étonnés de sa réponse, ils gardèrent le silence.
\TextTitle{Les preuves de la résurrection\FTNTT{Mt. 22:23-33 ; Mc.12:18-27}}
\VS{27}Alors quelques-uns des sadducéens, qui nient formellement la résurrection, s'approchèrent et l'interrogèrent,
\VS{28}disant : Maître, voici ce que Moïse nous a prescrit : Si le frère de quelqu'un meurt, ayant une femme et pas d'enfants, son frère épousera la femme, et suscitera une postérité à son frère.
\VS{29}Or, il y avait sept frères. Le premier se maria, et mourut sans enfants.
\VS{30}Le deuxième épousa la femme et mourut sans enfants.
\VS{31}Puis le troisième l'épousa aussi, et tous les sept de même ; et ils moururent sans laisser d'enfants.
\VS{32}Enfin, la femme mourut aussi.
\VS{33}Duquel d'entre eux donc sera-t-elle la femme à la résurrection ? Car les sept l'ont eue pour femme.
\VS{34}Jésus leur répondit : Les enfants de ce siècle prennent des femmes et des maris ;
\VS{35}mais ceux qui seront trouvés dignes d'avoir part au siècle à venir et à la résurrection des morts, ne prendront ni femmes ni maris.
\VS{36}Car ils ne pourront plus mourir, parce qu'ils seront semblables aux anges, et qu'ils seront fils de Dieu, étant fils de la résurrection.
\VS{37}Que les morts ressuscitent, c'est ce que Moïse a fait connaître quand, à propos du buisson, il appelle le Seigneur le Dieu d'Abraham, le Dieu d'Isaac, et le Dieu de Jacob.
\VS{38}Or, Dieu n'est pas le Dieu des morts, mais des vivants ; car tous vivent en lui.
\TextTitle{Jésus dénonce l'attitude des scribes\FTNTT{cp. Mt. 22:41-23:36 ; Mc. 12:35-40}}
\VS{39}Quelques-uns des scribes prenant la parole, dirent : Maître, tu as bien parlé.
\VS{40}Et ils n'osaient plus lui poser aucune question.
\VS{41}Jésus leur dit : Comment dit-on que le Christ est Fils de David ?
\VS{42}Car David lui-même dit au livre des psaumes : Le Seigneur a dit à mon Seigneur : Assieds-toi à ma droite,
\VS{43}jusqu'à ce que j'aie mis tes ennemis pour le marchepied de tes pieds\FTNT{Ps. 110:1.}.
\VS{44}David donc l'appelle son Seigneur, comment est-il son Fils ?
\VS{45}Comme tout le peuple l'écoutait, il dit à ses disciples :
\VS{46}Gardez-vous des scribes, qui aiment à se promener en robes longues, et qui aiment les salutations sur les places publiques ; qui recherchent les premiers sièges dans les synagogues, et les premières places dans les festins ;
\VS{47}qui dévorent entièrement les maisons des veuves, et qui font pour l'apparence de longues prières. Ils seront jugés plus sévèrement.
\Chap{21}
\TextTitle{Offrande de la pauvre veuve\FTNTT{Mc. 12:41-44}}
\VerseOne{}Comme Jésus regardait, il vit des riches qui mettaient leurs offrandes dans le tronc.
\VS{2}Il vit aussi une pauvre veuve qui y mettait deux petites pièces de monnaie.
\VS{3}Et il dit : Je vous le dis en vérité, cette pauvre veuve a mis plus que tous les autres.
\VS{4}Car tous ceux-ci ont mis aux offrandes de Dieu, de leur superflu ; mais elle a mis de son nécessaire, tout ce qu'elle avait pour vivre.
\TextTitle{Enseignement sur le Mont des Oliviers\FTNTT{Mt. 24-25 ; Mc. 13}}
\VS{5}Comme quelques-uns disaient que le temple était orné de belles pierres et d'offrandes, il dit :
\VS{6}Vous contemplez ces choses ! Les jours viendront où, il ne restera pas pierre sur pierre qui ne soit démolie.
\TextTitle{Les disciples posent deux questions à Jésus\FTNTT{Mt. 24:3 ; Mc.13:3-4}}
\VS{7}Ils lui demandèrent : Maître, quand donc cela arrivera-t-il, et à quel signe connaîtra-t-on que ces choses vont arriver ?
\TextTitle{Les temps de la fin\FTNTT{Mt. 24:4-14 ; Mc. 13:5-13}}
\VS{8}Jésus répondit : Prenez garde que vous ne soyez point séduits. Car plusieurs viendront en mon Nom, disant : C'est moi qui suis le Christ et le temps approche. Ne les suivez pas.
\VS{9}Quand vous entendrez parler des guerres et des soulèvements, ne soyez pas effrayés ; car il faut que ces choses arrivent premièrement. Mais ce ne sera pas encore la fin.
\VS{10}Alors il leur dit : Une nation s'élèvera contre une autre nation, et un royaume contre un autre royaume.
\VS{11}Il y aura de grands tremblements de terre en divers lieux, des famines et des pestes ; il y aura des choses terribles, et de grands signes dans le ciel.
\TextTitle{Souffrance des croyants}
\VS{12}Mais, avant toutes ces choses, ils mettront la main sur vous, et l'on vous persécutera ; on vous livrera aux synagogues, on vous jettera en prison, on vous mènera devant des rois et devant des gouverneurs, à cause de mon Nom.
\VS{13}Cela vous arrivera pour que vous serviez de témoignage.
\VS{14}Mettez-vous donc dans vos cœurs de ne pas préméditer votre défense.
\VS{15}Car je vous donnerai une bouche et une sagesse à laquelle vos adversaires ne pourront résister ou contredire.
\VS{16}Vous serez livrés même par vos parents, par vos frères, par vos proches et par vos amis, et ils feront mourir plusieurs d'entre vous.
\VS{17}Vous serez haïs de tous à cause de mon Nom.
\VS{18}Mais il ne se perdra pas un cheveu de votre tête.
\VS{19}Vous sauverez vos âmes par votre persévérance.
\TextTitle{La destruction de Jérusalem prophétisée}
\VS{20}Lorsque vous verrez Jérusalem environnée par les armées, sachez alors que sa désolation est proche.
\VS{21}Alors, que ceux qui seront en Judée, fuient dans les montagnes ; et que ceux qui seront au milieu de Jérusalem, en sortent, et que ceux qui seront dans les champs, n'entrent pas dans la ville.
\VS{22}Car ce seront des jours de vengeance, afin que toutes les choses qui sont écrites soient accomplies.
\VS{23}Malheur aux femmes qui seront enceintes, et à celles qui allaiteront en ces jours-là ; car il y aura une grande calamité sur le pays, et une grande colère contre ce peuple.
\VS{24}Ils tomberont sous le tranchant de l'épée, ils seront emmenés captifs\FTNT{Les Juifs se révoltèrent plusieurs fois contre le joug des Romains installés en Palestine depuis l'an 65 av. J.-C. En 70, Titus s'empara de Jérusalem après une guerre de plusieurs années et un siège meurtrier de sept mois. Cette même année, le temple fut détruit. A la suite d'une dernière révolte, la ville fut prise de nouveau sous Hadrien. En l'an 135, les Juifs furent en grande partie exterminés, et les survivants furent à jamais chassés de Jérusalem. Ces événements marquèrent symboliquement les débuts de la dispersion des Juifs à travers le monde.} parmi toutes les nations ; et Jérusalem sera foulée par les nations, jusqu'à ce que les temps des nations soient accomplis.
\TextTitle{Retour du Messie sur la terre\FTNTT{Mt. 24:29-31 ; Mc. 13:24-27}}
\VS{25}Il y aura des signes dans le soleil, dans la lune, et dans les étoiles. Et sur terre, il y aura de la détresse chez les nations qui ne sauront que faire, au bruit de la mer et des flots,
\VS{26}les hommes seront comme rendant l'âme de frayeur, dans l'attente des choses qui surviendront dans le monde ; car les puissances des cieux seront ébranlées.
\VS{27}Alors on verra le Fils de l'homme venant sur une nuée avec puissance et grande gloire.
\VS{28}Quand ces choses commenceront à arriver, regardez en haut et levez vos têtes, parce que votre délivrance approche.
\TextTitle{Parabole du figuier\FTNTT{Mt. 24:29-31 ; Mc. 13:24-27}}
\VS{29}Et il leur proposa cette comparaison : Voyez le figuier, et tous les autres arbres.
\VS{30}Dès qu'ils ont poussé, vous savez de vous-mêmes, en regardant, que déjà l'été est proche.
\VS{31}Vous aussi de même, quand vous verrez arriver ces choses, sachez que le Royaume de Dieu est proche.
\VS{32}En vérité je vous le dis, que cette génération ne passera point, que toutes ces choses ne soient arrivées.
\VS{33}Le ciel et la terre passeront, mais mes paroles ne passeront point.
\TextTitle{Exhortation à veiller\FTNTT{Mt. 24:36-51 ; Mc. 13:32-37}}
\VS{34}Prenez donc garde à vous-mêmes, de peur que vos cœurs ne soient appesantis par la gourmandise et l'ivrognerie, et par les soucis de cette vie ; et que ce jour-là ne vous surprenne subitement.
\VS{35}Car il viendra comme un filet sur tous ceux qui habitent sur la surface de toute la terre.
\VS{36}Veillez donc, et priez en tout temps, afin que vous soyez trouvés dignes d'échapper à toutes ces choses qui arriveront, et de paraître devant le Fils de l'homme.
\VS{37}Pendant le jour, Jésus enseignait dans le temple, et il allait passer la nuit à la montagne appelée Montagne des Oliviers.
\VS{38}Et dès le point du jour, tout le peuple venait vers lui au temple pour l'entendre.
\Chap{22}
\TextTitle{Trahison de Judas\FTNTT{Mt. 26:14-16 ; Mc. 14:1-2,10-11}}
\VerseOne{}La fête des pains sans levain, qu'on appelle Pâque, approchait.
\VS{2}Les principaux sacrificateurs et les scribes cherchaient les moyens de faire mourir Jésus ; car ils craignaient le peuple.
\VS{3}Or, Satan entra dans Judas, surnommé Iscariot, qui était du nombre des douze.
\VS{4}Et Judas alla, et parla avec les principaux sacrificateurs et les chefs de gardes, sur la manière de le leur livrer.
\VS{5}Ils furent dans la joie, et convinrent de lui donner de l'argent.
\VS{6}Après s'être engagé, il cherchait une occasion favorable pour leur livrer Jésus à l'insu de la foule.
\TextTitle{La dernière Pâque\FTNTT{Mt. 26:17-25 ; Mc. 14:12-21 ; Jn. 13:1-12}}
\VS{7}Le jour des pains sans levain, où l'on devait immoler la Pâque, arriva.
\VS{8}Et Jésus envoya Pierre et Jean, en leur disant : Allez, et apprêtez-nous l'agneau de Pâque, afin que nous le mangions.
\VS{9}Et ils lui dirent : Où veux-tu que nous l'apprêtions ?
\VS{10}Il leur dit : Voici, quand vous serez entrés dans la ville vous rencontrerez un homme portant une cruche d'eau, suivez-le dans la maison où il entrera.
\VS{11}Et dites au maître de la maison : Le Maître te dit : Où est le lieu où je mangerai l'agneau de Pâque avec mes disciples ?
\VS{12}Et il vous montrera une grande chambre haute, meublée ; c'est là que vous apprêterez l'agneau de Pâque.
\VS{13}Ils partirent, et trouvèrent les choses comme il leur avait dit ; et ils apprêtèrent l'agneau de Pâque.
\VS{14}Et quand l'heure fut venue, il se mit à table, et les douze apôtres avec lui.
\VS{15}Il leur dit : J'ai désiré vivement manger cet agneau de Pâque avec vous avant de souffrir.
\VS{16}Car, je vous dis, que je ne le mangerai plus jusqu'à ce qu'il soit accompli dans le Royaume de Dieu.
\VS{17}Et, ayant pris la coupe, il rendit grâces, et il dit : Prenez cette coupe, et distribuez-la entre vous.
\VS{18}Car, je vous dis, que je ne boirai plus du fruit de la vigne, jusqu'à ce que le Royaume de Dieu soit venu.
\TextTitle{Institution du repas de la Pâque\FTNTT{Mt. 26:26-29 ; Mc. 14:22-25 ; cp. Jn. 13:12-30 ; 1 Co. 11:23-26}}
\VS{19}Ensuite il prit du pain, et après avoir rendu grâces, il le rompit et le leur donna, en disant : Ceci est mon corps, qui est donné pour vous ; faites ceci en mémoire de moi.
\VS{20}Il prit de même la coupe, après le souper, et la leur donna, en disant : Cette coupe est la Nouvelle Alliance en mon sang, qui est répandu pour vous.
\TextTitle{Jésus annonce qu'il sera livré\FTNTT{Mt. 26:21-25 ; Mc. 14:18-21 ; Jn. 13:18-30}}
\VS{21}Cependant voici, la main de celui qui me trahit est avec moi à table.
\VS{22}Le Fils de l'homme s'en va ; selon ce qui est déterminé. Mais malheur à cet homme par qui il est trahi.
\VS{23}Et ils commencèrent à se demander les uns aux autres, qui était celui d'entre eux qui ferait cela.
\TextTitle{Leçon d'humilité\FTNTT{Mt. 20:20-28 ; Mc. 9.33-37 ; 10:35-45 ; Jn. 13:1-17}}
\VS{24}Il s'éleva une contestation parmi les apôtres, pour savoir lequel d'entre eux devait être estimé le plus grand.
\VS{25}Jésus leur dit : Les rois des nations les maîtrisent ; et ceux qui les dominent sont appelés bienfaiteurs.
\VS{26}Mais il n'en sera pas ainsi de vous : Au contraire, que le plus grand parmi vous soit comme le plus petit ; et celui qui gouverne, comme celui qui sert.
\VS{27}Car lequel est le plus grand, celui qui est à table, ou celui qui sert ? N'est-ce pas celui qui est à table ? Or je suis au milieu de vous comme celui qui sert.
\TextTitle{Le Royaume, une récompense}
\VS{28}Vous, vous êtes ceux qui avez persévéré avec moi dans mes épreuves ;
\VS{29}c'est pourquoi je vous confie le Royaume comme mon Père me l'a confié,
\VS{30}afin que vous mangiez et buviez à ma table dans mon Royaume, et que vous soyez assis sur des trônes, pour juger les douze tribus d'Israël.
\TextTitle{Jésus prophétise le triple reniement de Pierre\FTNTT{Mt. 26:30-35 ; Mc. 14:26-31 ; Jn. 13:36-38}}
\VS{31}Le Seigneur dit aussi : Simon, Simon, voici, Satan vous a réclamés pour vous cribler comme le froment ;
\VS{32}mais j'ai prié pour toi afin que ta foi ne défaille point ; et toi donc, quand tu seras un jour converti, affermis tes frères.
\VS{33}Pierre lui dit : Seigneur, je suis prêt à aller avec toi en prison et à la mort.
\VS{34}Mais Jésus lui dit : Pierre, je te dis que le coq ne chantera pas aujourd'hui, que tu n'aies nié trois fois de me connaître.
\TextTitle{Recommandation aux disciples\FTNTT{cp. Jn. 14-16 ; contraste Mt. 10:9-13}}
\VS{35}Puis il leur dit : Quand je vous ai envoyés sans bourse, sans sac, et sans souliers, avez-vous manqué de quelque chose ? Ils répondirent : De rien.
\VS{36}Et il leur dit : Maintenant au contraire, que celui qui a une bourse la prenne, et de même celui qui a un sac ; et que celui qui n'a point d'épée vende son vêtement, et achète une épée.
\VS{37}Car je vous le dis, il faut que cette parole qui est écrite s'accomplisse en moi : Il a été mis au nombre des malfaiteurs\FTNT{Es. 53:12.}. Parce qu'en effet, ce qui me concerne est sur le point d'arriver.
\VS{38}Ils dirent : Seigneur, voici ici deux épées. Et il leur dit : Cela suffit.
\TextTitle{Gethsémané\FTNTT{Mt. 26:36-46 ; Mc. 14:32-42 ; Jn. 18:1 ; cp. Hé. 5:7-8}}
\VS{39}Après être sorti, il alla, selon sa coutume, au Mont des Oliviers ; et ses disciples le suivirent.
\VS{40}Lorsqu'ils arrivèrent dans ce lieu, il leur dit : Priez afin que vous ne tombiez pas en tentation.
\VS{41}Puis s'étant éloigné d'eux à la distance d'environ un jet de pierre, et s'étant mis à genoux, il pria,
\VS{42}disant : Père, si tu voulais éloigner cette coupe loin de moi ; toutefois que ma volonté ne soit point faite, mais la tienne.
\VS{43}Et un ange lui apparut du ciel, pour le fortifier.
\VS{44}Etant en agonie, il priait plus instamment, et sa sueur devint comme des grumeaux de sang qui tombaient à terre.
\VS{45}Après avoir prié, il revint vers ses disciples, qu'il trouva endormis de tristesse ;
\VS{46}et il leur dit : Pourquoi dormez-vous ? Levez-vous, et priez, afin que vous ne tombiez pas en tentation.
\TextTitle{Trahison de Judas\FTNTT{Mt. 26:47-54 ; Mc. 14:43-47 ; Jn. 18:2-11}}
\VS{47}Et comme il parlait encore, voici une foule arriva ; et celui qui s'appelait Judas, l'un des douze, marchait devant elle. Il s'approcha de Jésus pour l'embrasser.
\VS{48}Et Jésus lui dit : Judas, c'est par un baiser que tu trahis le Fils de l'homme ?
\VS{49}Alors ceux qui étaient autour de lui, voyant ce qui allait arriver, lui dirent : Seigneur, frapperons-nous de l'épée ?
\VS{50}Et l'un d'eux frappa le serviteur du souverain sacrificateur, et lui emporta l'oreille droite.
\VS{51}Mais Jésus prenant la parole dit : Laissez-les faire jusqu'ici. Et, ayant touché son oreille, il le guérit.
\VS{52}Puis Jésus dit aux principaux sacrificateurs, aux chefs des gardes du temple, et aux anciens qui étaient venus contre lui : Etes-vous venus comme après un brigand avec des épées et des bâtons ?
\VS{53}J'étais tous les jours avec vous dans le temple, et vous n'avez pas mis la main sur moi. Mais c'est ici votre heure, et la puissance des ténèbres.
\TextTitle{Triple reniement de Pierre\FTNTT{Mt. 26:55-58,69-75 ; Mc. 14:48-54,66-72 ; Jn. 18:15-18,25-27}}
\VS{54}Après avoir saisi Jésus, ils l'emmenèrent, et le conduisirent dans la maison du souverain sacrificateur. Pierre suivait de loin.
\VS{55}Ils allumèrent du feu au milieu de la cour, et ils s'assirent ensemble. Pierre s'assit aussi parmi eux.
\VS{56}Une servante le voyant assis auprès du feu, fixa sur lui les regards, et dit : Celui-ci aussi était avec lui.
\VS{57}Mais il le nia, disant : Femme, je ne le connais point.
\VS{58}Peu après, un autre le voyant, dit : Tu es aussi de ces gens-là, mais Pierre dit : Ô homme ! Je n'en suis point.
\VS{59}Environ une heure plus tard, un autre affirmait et disait : Certainement celui-ci aussi était avec lui car il est Galiléen.
\VS{60}Pierre dit : Ô homme ! Je ne sais pas ce que tu dis. Au même instant, comme il parlait encore, le coq chanta.
\VS{61}Et le Seigneur, s'étant retourné, regarda Pierre. Et Pierre se souvint de la parole que le Seigneur lui avait dite : Avant que le coq chante, tu me renieras trois fois.
\VS{62}Alors Pierre étant sorti dehors, pleura amèrement.
\TextTitle{Jésus est outragé\FTNTT{Mt. 26:67-68 ; Mc. 14:65 ; Jn. 18:22-23}}
\VS{63}Les hommes qui tenaient Jésus se moquaient de lui, et le frappaient.
\VS{64}Ils lui bandèrent les yeux, ils lui donnaient des coups sur le visage, et l'interrogeaient, disant : Devine qui est celui qui t'a frappé ?
\VS{65}Et ils proféraient contre lui beaucoup d'autres injures.
\TextTitle{Jésus déclare qu'il est fils de Dieu\FTNTT{Mt. 26:59-68 ; 27:1 ; Mc. 14:55-65 ; 15:1 ; Jn. 18:19-24}}
\VS{66}Quand le jour fut venu, les anciens du peuple, les principaux sacrificateurs, et les scribes, s'assemblèrent, et firent amener Jésus dans le sanhédrin.
\VS{67}Ils dirent : Si tu es le Christ, dis-le-nous. Et il leur répondit : Si je vous le dis, vous ne le croirez point ;
\VS{68}et si je vous interroge, vous ne me répondrez pas, et vous ne me laisserez pas aller.
\VS{69}Désormais le Fils de l'homme sera assis à la droite de la puissance de Dieu.
\VS{70}Alors ils dirent tous : Tu es donc le Fils de Dieu ? Et il leur répondit : Vous le dites vous-mêmes, je le suis.
\VS{71}Alors ils dirent : Qu'avons-nous besoin encore de témoignage ? Nous l'avons entendu nous-mêmes de sa bouche.
\Chap{23}
\TextTitle{Jésus devant Pilate\FTNTT{Mt. 27:2,11-14 ; Mc. 15:1-5 ; Jn. 18:28-38}}
\VerseOne{}Puis ils se levèrent tous, et ils conduisirent Jésus devant Pilate.
\VS{2}Et ils se mirent à l'accuser, disant : Nous avons trouvé cet homme excitant notre nation à la révolte, et empêchant de payer le tribut à César, et se disant lui-même Christ, Roi.
\VS{3}Pilate l'interrogea, disant : Es-tu le Roi des Juifs ? Et Jésus lui répondit : Tu le dis.
\VS{4}Alors Pilate dit aux principaux sacrificateurs et à la foule : Je ne trouve aucun crime en cet homme.
\VS{5}Mais ils insistèrent, et dirent : Il soulève le peuple, enseignant par toute la Judée, depuis la Galilée où il a commencé, jusqu'ici.
\TextTitle{Jésus envoyé devant Hérode par Pilate}
\VS{6}Quand Pilate entendit parler de la Galilée, il demanda si cet homme était Galiléen,
\VS{7}et, ayant appris qu'il était de la juridiction d'Hérode, il le renvoya à Hérode, qui se trouvait aussi à Jérusalem.
\VS{8}Lorsque Hérode vit Jésus, il en eut une grande joie ; car depuis longtemps il désirait le voir, à cause de ce qu'il avait entendu dire de lui, et il espérait qu'il le verrait faire quelque miracle.
\VS{9}Il lui adressa beaucoup de questions ; mais Jésus ne lui répondit rien.
\VS{10}Les principaux sacrificateurs et les scribes étaient là, et l'accusaient avec violence.
\VS{11}Mais Hérode, avec ses gardes, le traita avec mépris ; et, après s'être moqué de lui et l'avoir revêtu d'un vêtement éclatant, il le renvoya à Pilate.
\VS{12}Ce même jour, Pilate et Hérode devinrent amis ; car auparavant ils étaient ennemis.
\TextTitle{Hérode renvoie Jésus à Pilate\FTNTT{Mt. 27:15-26 ; Mc. 15:6-15 ; Jn. 18:39-19:15}}
\VS{13}Pilate, ayant assemblé les principaux sacrificateurs, les magistrats, et le peuple, leur dit :
\VS{14}Vous m'avez présenté cet homme comme soulevant le peuple. Et voici, je l'ai interrogé devant vous, et je ne l'ai trouvé coupable d'aucun des crimes dont vous l'accusez.
\VS{15}Hérode non plus ; car il nous l'a renvoyé, et voici, cet homme n'a rien fait qui soit digne de mort.
\VS{16}Je le relâcherai donc, après l'avoir châtié.
\VS{17}A chaque fête, il était obligé de leur relâcher un prisonnier.
\VS{18}Toutes les foules s'écrièrent ensemble, disant : Ôte celui-ci, et relâche-nous Barabbas.
\VS{19}Cet homme avait été mis en prison pour une sédition qui avait eu lieu dans la ville, et pour un meurtre.
\VS{20}Pilate leur parla de nouveau, ayant envie de relâcher Jésus.
\VS{21}Et ils crièrent : Crucifie, crucifie-le !
\VS{22}Pilate leur dit pour la troisième fois : Mais quel mal a fait cet homme ? Je ne trouve rien en lui qui soit digne de mort. Après l'avoir fait battre de verges, je le relâcherai.
\VS{23}Mais ils insistèrent à grands cris, demandant qu'il soit crucifié ; et leurs cris et ceux des principaux sacrificateurs l'emportèrent.
\VS{24}Alors Pilate prononça que ce qu'ils demandaient, serait fait.
\VS{25}Il leur relâcha celui qui avait été mis en prison pour sédition et pour meurtre, et qu'ils demandaient ; et il abandonna Jésus à leur volonté.
\TextTitle{Sur le chemin de Golgotha\FTNTT{Mt. 27:31-32 ; Mc. 15:20-21 ; Jn. 19:16-17}}
\VS{26}Comme ils l'emmenaient, ils prirent un certain Simon, de Cyrène, qui revenait des champs, et le chargèrent de la croix pour qu'il la porte derrière Jésus.
\VS{27}Il était suivi d'une grande multitude des gens du peuple et de femmes, qui se frappaient la poitrine, et se lamentaient sur lui.
\VS{28}Mais Jésus se tourna vers elles, leur dit : Filles de Jérusalem, ne pleurez point sur moi, mais pleurez sur vous-mêmes, et sur vos enfants.
\VS{29}Car voici, des jours viendront où l'on dira : Heureuses les stériles, les entrailles qui n'ont point enfanté, et les mamelles qui n'ont point allaité !
\VS{30}Alors ils se mettront à dire aux montagnes : Tombez sur nous ; et aux collines : Couvrez-nous !
\VS{31}Car s'ils font ces choses au bois vert, que sera-t-il fait au bois sec ?
\VS{32}On conduisait en même temps deux malfaiteurs, qui devaient être mis à mort avec Jésus.
\TextTitle{Crucifixion de Jésus\FTNTT{Mt. 27:33-43 ; Mc. 15:24-32 ; Jn. 19:17-37}}
\VS{33}Lorsqu'ils furent arrivés au lieu qui est appelé Calvaire (le Crâne), ils le crucifièrent là, et les malfaiteurs aussi, l'un à la droite, et l'autre à la gauche.
\VS{34}Jésus dit : Père, pardonne-leur, car ils ne savent pas ce qu'ils font. Ils se partagèrent ensuite ses vêtements, en tirant au sort.
\VS{35}Le peuple se tenait là, et regardait. Les magistrats se moquaient de Jésus disant : Il a sauvé les autres, qu'il se sauve lui-même, s'il est le Christ, l'élu de Dieu.
\VS{36}Les soldats aussi se moquaient de lui ; s'approchant et lui présentant du vinaigre,
\VS{37}ils disaient : Si tu es le Roi des Juifs, sauve-toi toi-même !
\VS{38}Or il y avait au-dessus de lui un écriteau en lettres Grecques, Romaines et Hébraïques, en ces mots : Celui-ci est le roi des juifs.
\TextTitle{Repentance du malfaiteur crucifié\FTNTT{cp. Mt. 27:44 ; Mc. 15:32}}
\VS{39}L'un des malfaiteurs qui étaient crucifiés, l'outrageait, disant : Si tu es le Christ, sauve-toi toi-même, et sauve-nous !
\VS{40}Mais l'autre le reprenait, et disait : Ne crains-tu pas Dieu, car tu es condamné au même supplice ?
\VS{41}Pour nous, c'est juste, car nous recevons ce qu'ont mérité nos crimes ; mais celui-ci n'a fait aucun mal.
\VS{42}Et il dit à Jésus : Seigneur ! Souviens-toi de moi quand tu viendras dans ton règne.
\VS{43}Jésus lui dit : Je te le dis en vérité, aujourd'hui tu seras avec moi dans le paradis.
\TextTitle{Jésus remet son esprit\FTNTT{Mt. 27:45-56 ; Mc. 15:33-41 ; Jn. 19:30-37}}
\VS{44}Il était déjà environ la sixième heure, et il eut des ténèbres sur toute la terre jusqu'à la neuvième heure.
\VS{45}Le soleil s'obscurcit, et le voile du temple se déchira par le milieu.
\VS{46}Et Jésus criant à haute voix, dit : Père, je remets mon esprit entre tes mains ! Et, en disant cela, il expira\FTNT{C'est la fin de la Première Alliance. Voir commentaire Jn. 19:30.}.
\TextTitle{Fin de la loi mosaïque ou de la Première Alliance}
\VS{47}Le centenier, voyant ce qui était arrivé, glorifia Dieu, et dit : Certes, cet homme était juste.
\VS{48}Et tous ceux qui assistaient en foule à ce spectacle, après avoir vu ce qui était arrivé, s'en retournèrent, se frappant la poitrine.
\VS{49}Tous ceux qui connaissaient Jésus, et les femmes qui l'avaient suivi de Galilée, se tenaient dans l'éloignement et regardaient ces choses.
\TextTitle{Sépulture de Jésus\FTNTT{Mt. 27:57-61 ; Mc. 15:42-47 ; Jn. 19:38-42}}
\VS{50}Il y avait un conseiller, nommé Joseph, homme bon et juste,
\VS{51}qui n'avait point participé au conseil et aux actes des autres ; il était d'Arimathée, ville des Juifs, et il attendait le Royaume de Dieu.
\VS{52}Cet homme se rendit vers Pilate et lui demanda le corps de Jésus.
\VS{53}Il le descendit de la croix, l'enveloppa d'un linceul, et le déposa dans un sépulcre taillé dans le roc, où personne n'avait encore été mis.
\VS{54}C'était le jour de la préparation, et le sabbat allait commencer.
\VS{55}Les femmes qui étaient venues de Galilée avec Jésus, accompagnèrent Joseph, virent le sépulcre, et la manière dont le corps de Jésus y fut déposé.
\VS{56}Et s'en étant retournées, elles préparèrent des aromates et des parfums ; et le jour du sabbat elles se reposèrent selon la loi.
\Chap{24}
\TextTitle{Résurrection du Messie\FTNTT{Mt. 28:1-15 ; Mc. 16:1-11 ; Jn. 20:1-18}}
\VerseOne{}Le premier jour de la semaine, elles se rendirent au sépulcre de grand matin, apportant les aromates qu'elles avaient préparés.
\VS{2}Elles trouvèrent la pierre roulée à côté du sépulcre.
\VS{3}Et, étant entrées, elles ne trouvèrent point le corps du Seigneur Jésus.
\VS{4}Comme elles ne savaient que penser de cela, voici, deux hommes leur apparurent en habits resplendissants.
\VS{5}Saisies de frayeur, elles baissèrent le visage contre terre, mais ils leur dirent : Pourquoi cherchez-vous parmi les morts celui qui est vivant ?
\VS{6}Il n'est point ici, mais il est ressuscité. Souvenez-vous comment il vous a parlé quand il était encore en Galilée,
\VS{7}et qu'il disait : Il faut que le Fils de l'homme soit livré entre les mains des pécheurs, et qu'il soit crucifié, et qu'il ressuscite le troisième jour.
\VS{8}Et elles se souvinrent de ses paroles.
\VS{9}A leur retour du sépulcre, elles annoncèrent toutes ces choses aux onze disciples, et à tous les autres.
\VS{10}Or c'étaient Marie de Magdala, Jeanne, Marie, mère de Jacques, et les autres qui étaient avec elles, qui dirent ces choses aux apôtres.
\VS{11}Mais les paroles de ces femmes leur semblèrent comme des paroles futiles, et ils ne les crurent point.
\VS{12}Mais Pierre s'étant levé, courut au sépulcre et s'étant courbé pour regarder, il ne vit que les linges là tout seuls, puis il s'en alla chez lui, dans l'étonnement de ce qui était arrivé.
\TextTitle{Jésus et les deux disciples sur le chemin d'Emmaüs\FTNTT{Mc. 16:12-13}}
\VS{13}Or voici, deux d'entre eux étaient ce jour-là en chemin, pour aller à un village nommée Emmaüs, éloigné de Jérusalem de soixante stades.
\VS{14}Et ils s'entretenaient ensemble de toutes ces choses qui étaient arrivées.
\VS{15}Et il arriva que, comme ils s'entretenaient et discutaient entre eux, Jésus lui-même s'approcha et se mit à marcher avec eux.
\VS{16}Mais leurs yeux étaient retenus de sorte qu'ils ne le reconnaissaient pas.
\VS{17}Et il leur dit : Quels sont ces discours que vous tenez ensemble en marchant ? Et pourquoi êtes-vous tout tristes ?
\VS{18}Et l'un d'eux, nommé Cléopas, lui répondit, et lui dit : Es-tu le seul étranger dans Jérusalem qui ne sache point les choses qui s'y sont passées ces jours-ci ?
\VS{19}Et il leur dit : Quelles ? Ils répondirent : Celles concernant Jésus de Nazareth, qui était un prophète puissant en œuvres et en paroles devant Dieu, et devant tout le peuple.
\VS{20}Et comment les principaux sacrificateurs et nos magistrats l'ont livré pour être condamné à mort, et l'ont crucifié.
\VS{21}Or nous espérions que ce serait lui qui délivrerait Israël ; mais avec tout cela, c'est aujourd'hui le troisième jour que ces choses sont arrivées.
\VS{22}Toutefois quelques femmes d'entre nous nous ont fort étonnés, car elles ont été de grand matin au sépulcre
\VS{23}et n'ayant point trouvé son corps, elles sont venues dire que même elles avaient vu une apparition d'anges, qui disaient qu'il est vivant.
\VS{24}Et quelques-uns des nôtres sont allés au sépulcre, et ont trouvé les choses comme les femmes l'avaient dit ; mais lui, ils ne l'ont point vu.
\VS{25}Alors Jésus leur dit : Ô gens sans intelligence, et dont le cœur est lent à croire tout ce que les prophètes ont annoncé !
\VS{26}Ne fallait-il pas que le Christ souffrît ces choses, et qu'il entra dans sa gloire ?
\VS{27}Puis commençant par Moïse, et continuant par tous les prophètes, il leur expliquait dans toutes les Ecritures ce qui le concernait.
\VS{28}Et comme ils furent près du village où ils allaient, il faisait comme s'il voulait aller plus loin.
\VS{29}Mais ils le forcèrent, en lui disant : Reste avec nous, car le soir approche et le jour commence à baisser. Et il entra donc pour rester avec eux.
\VS{30}Et il arriva comme il était à table avec eux, il prit le pain, et il le bénit ; et l'ayant rompu, il leur distribua.
\VS{31}Alors leurs yeux s'ouvrirent, et ils le reconnurent ; mais il disparut de devant eux.
\VS{32}Et ils se dirent l'un à l'autre : Notre cœur ne brûlait-il pas au-dedans de nous lorsqu'il nous parlait en chemin, et qu'il nous ouvrait les Ecritures ?
\TextTitle{Nouvelles apparitions du réssuscité\FTNTT{Mc. 16:14 ; Jn. 20:19-25 ; cp. Jn. 20:26-21:25}}
\VS{33}Et se levant à l'heure même, ils retournèrent à Jérusalem, et ils trouvèrent assemblés les onze et ceux qui étaient avec eux,
\VS{34}qui disaient : Le Seigneur est véritablement ressuscité, et il est apparu à Simon.
\VS{35}A leur tour, ils racontèrent ce qui leur était arrivé en chemin, et comment il avait été reconnu d'eux en rompant le pain.
\VS{36}Comme ils tenaient ces discours, Jésus se présenta lui-même au milieu d'eux, et leur dit : Que la paix soit avec vous !
\VS{37}Mais eux tout terrifiés et effrayés croyaient voir un esprit.
\VS{38}Et il leur dit : Pourquoi êtes-vous troublés, et pourquoi monte-t-il des pensées dans vos coeurs ?
\VS{39}Voyez mes mains et mes pieds, c'est bien moi. Touchez-moi, et voyez : Car un esprit n'a ni chair ni os, comme vous voyez que j'ai.
\VS{40}Et en disant cela, il leur montra ses mains et ses pieds.
\VS{41}Mais comme de joie, ils ne croyaient point encore, et qu'ils s'étonnaient, il leur dit : Avez-vous ici quelque chose à manger ?
\VS{42}Et ils lui présentèrent un morceau de poisson rôti, et un rayon de miel.
\VS{43}Et l'ayant pris, il mangea devant eux.
\TextTitle{La nouvelle mission des onze\FTNTT{Mt. 28:18-20 ; Mc. 16:15-18 ; Jn. 17:18 ; 20:21 ; Ac. 1:8}}
\VS{44}Puis il leur dit : Ce sont ici les paroles que je vous disais lorsque j'étais encore avec vous, qu'il fallait que s'accomplisse tout ce qui est écrit de moi dans la loi de Moïse, dans les prophètes, et dans les psaumes.
\VS{45}Alors il leur ouvrit l'esprit\FTNT{Pour comprendre les Ecritures, nous avons besoin de l'aide de l'Esprit de Dieu. La vraie connaissance ne vient pas des hommes, mais de Dieu (Da. 9:22).} afin qu'ils comprennent les Ecritures.
\VS{46}Et il leur dit : Il est ainsi écrit, et ainsi il fallait que le Christ souffre, et qu'il ressuscite des morts le troisième jour,
\VS{47}et que la repentance et le pardon des péchés seraient prêchés en son Nom à toutes les nations, à commencer par Jérusalem\FTNT{Es. 53.}.
\VS{48}Et vous êtes témoins de ces choses. 
\VS{49}Et voici, j'enverrai sur vous la promesse de mon Père, mais vous donc restez dans la ville de Jérusalem, jusqu'à ce que vous soyez revêtus de la puissance d'en haut.
\TextTitle{Jésus enlevé au ciel\FTNTT{Mc. 16:19-20 ; Ac. 1:9-11}}
\VS{50}Après quoi il les conduisit dehors jusqu'en Béthanie, et levant ses mains en haut, il les bénit.
\VS{51}Et pendant qu'il les bénissait, il se sépara d'eux, et fut élevé au ciel.
\VS{52}Pour eux, après l'avoir adoré, ils retournèrent à Jérusalem avec une grande joie.
\VS{53}Et ils étaient toujours dans le temple, louant et bénissant Dieu. Amen !
\PPE{}
\end{multicols}

%\clearpage\ShortTitle{Jean}\BookTitle{Jean}\BFont
\noindent\hrulefill
\textit{
\bigskip
{\centering{}
\\Signifie : Dieu pardonne, don de Dieu
\\Thème : Christ, Dieu
\\Auteur : Jean
\\Date de rédaction : Env. 85-90 apr. J.-C.\\}
}
%\bigskip
\textit{
\\Auteur d’un des quatre évangiles, des trois épîtres éponymes et de l’Apocalypse, Jean, fils de Zébédée, fut l’un des douze. Témoin oculaire du ministère terrestre de Jésus-Christ, il attesta par l’essence de ses écrits le caractère divin de ce dernier.
\bigskip
\\Fidèle au livre d’Exode où Yahweh se révéla comme étant « Je suis », Jean reprit les propos de Jésus et le présenta comme la Parole incarnée, le Pain de vie, la Lumière du monde, la Porte des brebis, le Bon berger, la Résurrection, la Vie… Proche du maître, Jean fut à même de relater les évènements marquants de sa vie comme la gloire de la Transfiguration, l’angoisse de la passion exprimée à Gethsémané, ou encore les déclarations solennelles précédées de l’expression «  En vérité, en vérité »… Il mit également en évidence la controverse suscitée par le Christ et l’opposition dont il fit l’objet de la part de certains pharisiens qui souhaitaient sa mort.
\bigskip
\\L’évangile de Jean exprime la nécessité de la nouvelle naissance et dévoile les attributs du Fils de Dieu, le Messie tant attendu.\bigskip
}
\par\nobreak\noindent\hrulefill
\begin{multicols}{2}
\TextTitle{[La divinité de Jésus-Christ]
\\(Jn. 10:30 ; Hé. 1:5-13)}
\Chap{1}
\VerseOne{}Au commencement était la Parole, et la Parole était avec Dieu, et la Parole était Dieu.
\VS{2}Elle était au commencement avec Dieu.
\TextTitle{[L'oeuvre de Jésus avant son incarnation]}
\VS{3}Toutes choses ont été faites par elle, et rien de ce qui a été fait, n'a été fait sans elle.
\VS{4}En elle était la vie, et la vie était la Lumière des hommes\FTNT{Jésus-Christ notre Lumière : Es. 60:19-20.}.
\VS{5}Et la Lumière luit dans les ténèbres, mais les ténèbres ne l'ont point reçue.
\TextTitle{[Ministère de Jean-Baptiste]}
\VS{6}Il y eut un homme appelé Jean, qui fut envoyé de Dieu.
\VS{7}Il vint pour servir de témoin, pour rendre témoignage à la Lumière, afin que tous croient par lui.
\VS{8}Il n'était pas la Lumière, mais il était envoyé pour rendre témoignage à la Lumière.
\TextTitle{[Jésus-Christ, la véritable lumière]
\\(Jn. 3:17-21 ; 8:12 ; 9:5 ; 12:46)}
\VS{9}Cette Lumière était la véritable Lumière, qui en venant dans le monde éclaire tout homme.
\VS{10}Elle était dans le monde, et le monde a été fait par elle ; mais le monde ne l'a point connue.
\VS{11}Elle est venue chez les siens ; et les siens ne l'ont point reçue.
\VS{12}Mais à tous ceux qui l'ont reçue, à ceux qui croient en son Nom, elle leur a donné le pouvoir de devenir enfants de Dieu.
\VS{13}Lesquels sont nés, non du sang, ni de la volonté de la chair, ni de la volonté de l'homme ; mais ils sont nés de Dieu.
\TextTitle{[La Parole faite chair]
\\(Jn. 14:9 ; Mt. 1:18-23 ; Lu. 1:30-35 ; 2:11 ; 1Tim. 3:16)}
\VS{14}Et la Parole a été faite chair, elle a habité parmi nous, pleine de grâce et de vérité, et nous avons contemplé sa gloire, une gloire, comme la gloire du Fils unique du Père.
\TextTitle{[Premier témoignage de Jean-Baptiste]
\\(Mt. 3:1-12 ; Mc. 1:1-11 ; Lu. 3:1-22)}
\VS{15}Jean a donc rendu témoignage de lui, et s’est écrié, disant : C'est celui dont j’ai dit : Celui qui vient après moi m’a précédé, car il était avant moi.
\VS{16}Et nous avons tous reçu de sa plénitude, et grâce pour grâce.
\VS{17}Car la loi\FTNT{La loi a été promulguée par Moïse.} a été donnée par Moïse, la grâce et la vérité sont venues par Jésus-Christ.
\VS{18}Personne n’a jamais vu Dieu, le Fils unique qui est dans le sein du Père, est celui qui nous l'a révélé.
\VS{19}Et c'est ici le témoignage de Jean, lorsque les Juifs envoyèrent de Jérusalem des sacrificateurs et des lévites pour l'interroger, et lui dire : Toi qui es-tu ?
\VS{20}Il confessa, et ne le nia point, il déclara, en disant : Ce n'est pas moi qui suis le Christ.
\VS{21}Et ils lui demandèrent : Quoi donc ? Es-tu Elie ? Et il dit : Je ne le suis point\FTNT{En Mt. 11:14 Jésus confirme pourtant que Jean-Baptiste est bien l’Elie qui devait venir. Comment expliquer qu’il nia l’être lorsqu’il fut interrogé par les pharisiens ? La seule explication plausible c’est qu’il l’ignorait. Toutefois, il avait conscience qu’il était ~la voix~ prophétisée par Esaïe. Remarquez que lorsqu’il fut emprisonné, il avait envoyé quelques-uns de ses disciples pour demander à Jésus s’il était bien le Messie (Mt. 11:13 ; Lu. 7:19-20) alors qu’il fut le premier à rendre témoignage du Seigneur. Ces éléments ne sont pas contradictoires, ils ne font que révéler les failles liées à la nature humaine de Jean.}. Es-tu le Prophète ? Et il répondit : Non.
\VS{22}Ils lui dirent donc : Qui es-tu, afin que nous donnions une réponse à ceux qui nous ont envoyés. Que dis-tu de toi-même ?
\VS{23}Il dit : Je suis la voix de celui qui crie dans le désert : Aplanissez le chemin du Seigneur, comme a dit Esaïe le prophète\FTNT{Es. 40:3.}.
\VS{24}Or ceux qui avaient été envoyés vers lui étaient des pharisiens.
\VS{25}Ils l'interrogèrent encore, et lui dirent : Pourquoi donc baptises-tu si tu n'es point le Christ, ni Elie, ni le Prophète ?
\VS{26}Jean leur répondit : Pour moi, je baptise d'eau ; mais il y a quelqu’un au milieu de vous que vous ne connaissez point.
\VS{27}C'est celui qui vient après moi, il m’a précédé, et je ne suis pas digne de délier la courroie de ses souliers.
\VS{28}Ces choses se passèrent à Béthanie, au-delà du Jourdain, où Jean baptisait.
\VS{29}Le lendemain Jean vit Jésus venir à lui, et il dit : Voici l'Agneau de Dieu, qui ôte le péché du monde.
\VS{30}C'est celui dont j’ai dit : Après moi vient un homme qui m’a précédé ; car il était avant moi.
\VS{31}Et pour moi, je ne le connaissais pas ; mais c'est afin qu'il soit manifesté à Israël que je suis venu baptiser d'eau.
\VS{32}Jean rendit aussi témoignage, en disant : J'ai vu l'Esprit descendre du ciel comme une colombe, et s'arrêter sur lui.
\VS{33}Et pour moi, je ne le connaissais point ; mais celui qui m'a envoyé baptiser d'eau, m'avait dit : Celui sur qui tu verras l'Esprit descendre et s’arrêter, c'est celui qui baptise du Saint-Esprit.
\VS{34}Et je l'ai vu, et j'ai rendu témoignage, que c'est lui qui est le Fils de Dieu.
\TextTitle{[Premiers disciples de Jésus-Christ]
\\(Mt. 4:18-22 ; Mc. 1:16-20 ; Lu. 5:1-11)}
\VS{35}Le lendemain Jean était encore là, avec deux de ses disciples ;
\VS{36}et regardant Jésus qui marchait, il dit : Voici l'Agneau de Dieu.
\VS{37}Les deux disciples l'entendirent prononcer ces paroles, et ils suivirent Jésus.
\VS{38}Et Jésus se retournant, et voyant qu'ils le suivaient, il leur dit : Que cherchez-vous ? Ils lui répondirent : Rabbi, c'est-à-dire Maître, où demeures-tu ?
\VS{39}Il leur dit : Venez, et voyez. Ils y allèrent, et ils virent où il demeurait ; et ils demeurèrent avec lui ce jour-là ; car il était environ dix heures.
\VS{40}André, frère de Simon Pierre, était l'un des deux qui avaient entendu les paroles de Jean et qui avaient suivi Jésus.
\VS{41}Ce fut lui qui rencontra le premier Simon son frère, et il lui dit : Nous avons trouvé le Messie, c'est-à-dire le Christ.
\VS{42}Et il le conduisit vers Jésus, et Jésus l’ayant regardé, dit : Tu es Simon, fils de Jonas, tu seras appelé Céphas ; c'est-à-dire, Pierre.
\VS{43}Le lendemain Jésus voulut se rendre en Galilée, et il trouva Philippe. Et il lui dit : Suis-moi.
\VS{44}Philippe était de Bethsaïda, la ville d'André et de Pierre.
\VS{45}Philippe rencontra Nathanaël, et lui dit : Nous avons trouvé celui de qui Moïse a écrit dans la loi, et dont les prophètes ont parlé, Jésus, qui est de Nazareth, fils de Joseph.
\VS{46}Et Nathanaël lui dit : Peut-il venir quelque chose de bon de Nazareth ? Philippe lui dit : Viens, et vois.
\VS{47}Jésus aperçut Nathanaël venir vers lui, et il dit de lui : Voici vraiment un Israëlite dans lequel il n'y a point de fraude.
\VS{48}Nathanaël lui dit : D'où me connais-tu ? Jésus répondit et lui dit : Avant que Philippe t’appelle, quand tu étais sous le figuier, je t’ai vu.
\VS{49}Nathanaël répondit et lui dit : Maître, tu es le Fils de Dieu. Tu es le Roi d'Israël.
\VS{50}Jésus lui répondit et dit : Parce que je t'ai dit que je t’ai vu sous le figuier, tu crois. Tu verras des choses plus grandes encore.
\VS{51}Il lui dit aussi : En vérité, en vérité je vous dis : Désormais vous verrez le ciel ouvert, et les anges de Dieu monter et descendre sur le Fils de l'homme.
\TextTitle{[Premier miracle, à Cana]}
\Chap{2}
\VerseOne{}Trois jours après, il y eut des noces à Cana en Galilée, et la mère de Jésus était là.
\VS{2}Et Jésus fut aussi convié aux noces avec ses disciples.
\VS{3}Et le vin ayant manqué, la mère de Jésus lui dit : Ils n'ont plus de vin.
\VS{4}Jésus lui répondit : Qu'y a-t-il entre moi et toi, femme ? Mon heure n'est point encore venue.
\VS{5}Sa mère dit aux serviteurs : Faites tout ce qu'il vous dira.
\VS{6}Or il y avait là six vases de pierre, destinés aux purifications des Juifs, et contenant chacun deux ou trois mesures.
\VS{7}Et Jésus leur dit : Remplissez d'eau ces vases. Et ils les remplirent jusqu’au bord.
\VS{8}Puis il leur dit : Puisez maintenant, et apportez-en au maître d'hôtel. Et ils lui en apportèrent.
\VS{9}Quand le maître d'hôtel eut goûté l'eau changée en vin, ne sachant d'où venait ce vin, tandis que les serviteurs qui avaient puisé l'eau le savaient bien, il s'adressa à l'époux
\VS{10}et lui dit : Tout homme sert d’abord le bon vin, et ensuite le moins bon, après qu’on s’est enivré ; mais toi, tu as gardé le bon vin jusqu'à maintenant.
\VS{11}Jésus fit ce premier miracle à Cana en Galilée, et il manifesta sa gloire, et ses disciples crurent en lui.
\VS{12}Après cela, il descendit à Capernaüm avec sa mère, et ses frères, et ses disciples ; mais ils y demeurèrent peu de jours.
\TextTitle{[La première Pâque]
\\(Jn. 6:4 ; 11:55)}
\VS{13}La Pâque des Juifs était proche ; c'est pourquoi Jésus monta à Jérusalem.
\VS{14}Et il trouva dans le temple des vendeurs de bœufs, de brebis, et de pigeons ; et les changeurs qui y étaient assis.
\VS{15}Et ayant fait un fouet avec des petites cordes, il les chassa tous du temple, avec les brebis, et les bœufs ; et il dispersa la monnaie des changeurs, et renversa les tables.
\VS{16}Et il dit aux vendeurs des pigeons : Otez ces choses d'ici, et ne faites pas de la maison de mon Père une maison de marché.
\VS{17}Alors ses disciples se souvinrent qu'il était écrit : Le zèle de ta maison me dévore\FTNT{Ps. 69:10.}.
\VS{18}Mais les Juifs prenant la parole, lui dirent : Quel signe nous montres-tu pour agir de la sorte ?
\VS{19}Jésus répondit et leur dit : Détruisez ce temple, et en trois jours je le relèverai.
\VS{20}Et les Juifs dirent : Il a fallu quarante-six ans pour bâtir ce temple, et toi, tu le relèveras en trois jours !
\VS{21}Mais il parlait du temple de son corps.
\VS{22}C'est pourquoi lorsqu'il fut ressuscité des morts, ses disciples se souvinrent qu'il leur avait dit cela, et ils crurent à l'Ecriture et à la parole que Jésus avait dite.
\VS{23}Et comme il était à Jérusalem le jour de la fête de Pâque, plusieurs crurent en son Nom, voyant les miracles qu'il faisait.
\VS{24}Mais Jésus ne se fiait point à eux, parce qu'il les connaissait tous ;
\VS{25}et parce qu'il n'avait pas besoin qu’on lui rende témoignage d'aucun homme ; car il savait lui-même ce qui était dans l'homme.
\TextTitle{[Jésus et Nicodème : la naissance d'en haut]}
\Chap{3}
\VerseOne{}Mais il y eut un homme d'entre les pharisiens, nommé Nicodème, qui était un des chefs des Juifs,
\VS{2}qui vint de nuit auprès de Jésus, et lui dit : Rabbi, nous savons que tu es un Docteur venu de Dieu, car personne ne peut faire les miracles que tu fais, si Dieu n'est avec lui.
\VS{3}Jésus lui répondit et dit : En vérité, en vérité je te le dis : Si quelqu'un ne naît d’en haut\FTNT{Naître d’en haut : Dans la plupart des Bibles modernes, on trouve l’expression naître « de nouveau », or cette traduction n’est pas correcte puisque le texte grec utilise l'expression naître « d’en haut ». L’adverbe « d’en haut » vient du mot grec « ahothen » qui signifie : depuis le haut, depuis un endroit plus élevé, ce qui vient des cieux ou de Dieu, depuis le début, l'origine. Ce mot se retrouve dans Mt. 27:51 ; Mc. 15:38 ; Lu. 1:3 ; Jn. 3:31 ; Jn. 19:11 ; Jn. 19:23 ; Ja. 1:17 ; Ja. 3:15 ; Ja. 3:17. ~Anothen~ vient de ~ano~ : choses d’en haut. En Ga. 4:26 ~ano~ peut se référer au lieu ou au temps. Le lieu : La Jérusalem qui est au-dessus, dans les cieux. Le temps : La Jérusalem éternelle qui a précédé la terrestre. Le mot ~ano~ a été traduit par ~en haut~ dans Jn. 8:23 ; Jn. 11:41 ; Ac. 2:19 ; Ga. 4:26 ; Col. 3:1-2 ; et par ~céleste~ dans Ph. 3:14. Jésus nous enseigne donc que la nouvelle naissance est en réalité la naissance d’en haut, une naissance qui a eu lieu dans la Nouvelle Jérusalem.}, il ne peut voir le Royaume de Dieu.
\VS{4}Nicodème lui dit : Comment un homme peut-il naître quand il est vieux ? Peut-il rentrer dans le sein de sa mère et naître une seconde fois ?
\VS{5}Jésus répondit : En vérité, en vérité je te dis : Si quelqu’un ne naît d'eau et d'Esprit, il ne peut entrer dans le Royaume de Dieu.
\VS{6}Ce qui est né de la chair est chair ; et ce qui est né de l'Esprit est esprit.
\VS{7}Ne t'étonne pas de ce que je t'ai dit : Il faut que vous naissiez d’en haut.
\VS{8}Le vent souffle où il veut, et tu en entends le bruit ; mais tu ne sais pas d'où il vient ni où il va : Il en est ainsi de tout homme qui est né de l'Esprit.
\VS{9}Nicodème lui dit : Comment cela peut-il se faire ?
\VS{10}Jésus répondit et lui dit : Tu es le docteur d'Israël, et tu ne connais point ces choses !
\VS{11}En vérité, en vérité je te le dis, nous disons ce que nous savons, et nous rendons témoignage de ce que nous avons vu ; et vous ne recevez pas notre témoignage.
\VS{12}Si vous ne croyez pas quand je vous ai parlé des choses terrestres, comment croirez-vous quand je vous parlerai des choses célestes ?
\VS{13}Personne n'est monté au ciel, si ce n’est celui qui est descendu du ciel, le Fils de l'homme qui est dans le ciel.
\VS{14}Et comme Moïse éleva le serpent\FTNT{Le serpent d’airain : No. 21:9}dans le désert, il faut de même que le Fils de l'homme soit élevé,
\VS{15}afin que quiconque croit en lui ne périsse point, mais qu'il ait la vie éternelle.
\VS{16}Car Dieu a tant aimé le monde, qu'il a donné son Fils unique, afin que quiconque croit en lui ne périsse point, mais qu'il ait la vie éternelle.
\VS{17}Car Dieu n'a point envoyé son Fils dans le monde pour condamner le monde, mais afin que le monde soit sauvé par lui.
\VS{18}Celui qui croit en lui ne sera point jugé ; mais celui qui ne croit point est déjà jugé ; parce qu'il n'a point cru au Nom du Fils unique de Dieu.
\VS{19}Et ce jugement c’est que la lumière est venue dans le monde et que les hommes ont préféré les ténèbres à la lumière, parce que leurs œuvres étaient mauvaises.
\VS{20}Car quiconque fait le mal, hait la lumière, et ne vient point à la lumière, de peur que ses œuvres ne soient condamnées.
\VS{21}Mais celui qui agit selon la vérité, vient à la lumière, afin que ses œuvres soient manifestées, parce qu'elles sont faites selon Dieu.
\TextTitle{[Nouveau témoignage de Jean-Baptiste]}
\VS{22}Après ces choses, Jésus s’en alla avec ses disciples dans la terre de Judée ; et là, il demeurait avec eux et il baptisait.
\VS{23}Jean aussi baptisait à Enon, près de Salim, parce qu'il y avait là beaucoup d'eau, et on y venait pour être baptisé.
\VS{24}Car Jean n'avait pas encore été mis en prison.
\VS{25}Or, il y eut une dispute entre les disciples de Jean et les Juifs touchant la purification.
\VS{26}Ils vinrent trouver Jean, et lui dirent : Maître, celui qui était avec toi au-delà du Jourdain, et à qui tu as rendu témoignage, voilà, il baptise, et tous vont à lui.
\VS{27}Jean répondit et dit : Un homme ne peut recevoir que ce qui lui a été donné du ciel.
\VS{28}Vous-mêmes m'êtes témoins que j'ai dit : Ce n'est pas moi qui suis le Christ, mais j’ai été envoyé devant lui.
\VS{29}Celui à qui appartient l'Epouse c’est l'Epoux ; mais l'ami de l'Epoux qui se tient là et qui l’entend, éprouve une grande joie à cause de la voix de l’Epoux ; c'est pourquoi cette joie qui est la mienne est parfaite.
\VS{30}Il faut qu'il croisse, et que je diminue.
\TextTitle{[Conclusion apportée par Jean]}
\VS{31}Celui qui vient d'en haut est au-dessus de tous ; celui qui est venu de la terre est de la terre, et il parle comme venant de la terre. Celui qui est venu du ciel est au-dessus de tous.
\VS{32}Et ce qu'il a vu et entendu, il le témoigne ; mais personne ne reçoit son témoignage.
\VS{33}Celui qui a reçu son témoignage a certifié que Dieu est véritable.
\VS{34}Car celui que Dieu a envoyé annonce les paroles de Dieu ; car Dieu ne lui donne point l'Esprit avec mesure.
\VS{35}Le Père aime le Fils, et il a remis toutes choses entre ses mains.
\VS{36}Celui qui croit au Fils a la vie éternelle, mais celui qui désobéit au Fils ne verra point la vie, mais la colère de Dieu demeure sur lui.
\TextTitle{[Jésus se rend en Galilée]}
\Chap{4}
\VerseOne{}Le Seigneur sut que les pharisiens avaient appris qu'il faisait et baptisait plus de disciples que Jean.
\VS{2}Toutefois Jésus ne baptisait point lui-même, mais c'étaient ses disciples.
\VS{3}Il quitta la Judée, et retourna encore en Galilée.
\TextTitle{[Jésus et la femme samaritaine]}
\VS{4}Comme il fallait qu'il passe par la Samarie,
\VS{5}il arriva dans une ville de Samarie nommée Sychar, près du champ que Jacob avait donné à Joseph son fils\FTNT{Ge. 48:22.}.
\VS{6}Or il y avait là le puits de Jacob ; et Jésus, fatigué du voyage, se tenait là, assis au bord du puits. C'était environ la sixième heure\FTNT{Sixième heure ou midi.}.
\VS{7}Une femme Samaritaine vint puiser de l'eau, Jésus lui dit : Donne-moi à boire.
\VS{8}Car ses disciples étaient allés à la ville pour acheter des vivres.
\VS{9}La femme Samaritaine lui dit : Comment toi qui es Juif, me demandes-tu à boire, à moi qui suis une femme Samaritaine ? Les Juifs, en effet, n’ont pas de relations avec les Samaritains.
\VS{10}Jésus lui répondit et dit : Si tu connaissais le don de Dieu, et qui est celui qui te dit : Donne-moi à boire, tu lui aurais toi-même demandé à boire, et il t’aurait donné de l'eau vive.
\VS{11}La femme lui dit : Seigneur, tu n'as rien pour puiser, et le puits est profond ; d'où aurais-tu donc cette eau vive ?
\VS{12}Es-tu plus grand que Jacob, notre père, qui nous a donné ce puits, et qui en a bu lui-même, ainsi que ses enfants et son bétail ?
\VS{13}Jésus répondit et lui dit : Quiconque boit de cette eau aura encore soif ;
\VS{14}mais celui qui boira de l'eau que je lui donnerai, n'aura jamais soif ; mais l'eau que je lui donnerai deviendra en lui une source d'eau qui jaillira jusque dans la vie éternelle.
\VS{15}La femme lui dit : Seigneur, donne-moi de cette eau, afin que je n'aie plus soif, et que je ne vienne plus ici puiser de l'eau.
\VS{16}Jésus lui dit : Va, appelle ton mari, et viens ici.
\VS{17}La femme lui répondit et dit : Je n'ai point de mari. Jésus lui dit : Tu as bien dit : Je n'ai point de mari.
\VS{18}Car tu as eu cinq maris, et celui que tu as maintenant n'est point ton mari ; en cela tu as dit la vérité.
\VS{19}La femme lui dit : Seigneur, je vois que tu es un prophète.
\VS{20}Nos pères ont adoré sur cette montagne\FTNT{Cette montagne, dont parle la Samaritaine, c'est le Mont Garizim (ou montagne de Sichem) sur lequel les samaritains construisirent leur temple et établirent leur culte, au temps de Néhémie.}, et vous, vous dites que le lieu où il faut adorer est à Jérusalem.
\VS{21}Jésus lui dit : Femme, crois-moi, l'heure vient où ce ne sera ni sur cette montagne ni à Jérusalem que vous adorerez le Père.
\VS{22}Vous adorez ce que vous ne connaissez pas ; nous, nous adorons ce que nous connaissons ; car le salut vient des Juifs.
\VS{23}Mais l'heure vient, et elle est déjà venue, où les vrais adorateurs adoreront le Père en esprit et en vérité ; car ce sont là les adorateurs que le Père demande.
\VS{24}Dieu est Esprit, et il faut que ceux qui l'adorent, l'adorent en esprit et en vérité.
\VS{25}La femme lui répondit : Je sais que le Messie, c'est-à-dire le Christ, doit venir ; quand il sera venu, il nous annoncera toutes choses.
\VS{26}Jésus lui dit : Je le suis, moi qui te parle.
\VS{27}Là-dessus arrivèrent ses disciples, et ils s'étonnèrent de ce qu'il parlait avec une femme. Toutefois aucun ne dit : Que demandes-tu ? Ou : Pourquoi parles-tu avec elle ?
\VS{28}La femme, ayant laissé sa cruche, s'en alla dans la ville, et elle dit aux habitants :
\VS{29}Venez voir un homme qui m'a dit tout ce que j'ai fait, ne serait-ce point le Christ ?
\VS{30}Ils sortirent donc de la ville, et vinrent vers lui.
\VS{31}Cependant les disciples le pressaient, disant : Maître, mange.
\VS{32}Mais il leur dit : J'ai à manger une nourriture que vous ne connaissez point.
\VS{33}Sur quoi les disciples se demandaient entre eux : Quelqu'un lui aurait-il apporté à manger ?
\VS{34}Jésus leur dit : Ma nourriture est de faire la volonté de celui qui m'a envoyé, et d’accomplir son œuvre.
\VS{35}Ne dites-vous pas qu'il y a encore quatre mois jusqu’à la moisson ? Voici, je vous dis, levez vos yeux, et regardez les champs qui déjà blanchissent pour la moisson.
\VS{36}Celui qui moissonne reçoit un salaire, et amasse des fruits pour la vie éternelle ; afin que celui qui sème et celui qui moissonne se réjouissent ensemble.
\VS{37}Car en ceci ce qu’on dit d’ordinaire est vrai : L’un sème et l'autre moissonne.
\VS{38}Je vous ai envoyés moissonner où vous n'avez point travaillé ; d'autres ont travaillé, et vous êtes entrés dans leur travail.
\TextTitle{[Jésus et les samaritains]}
\VS{39}Plusieurs Samaritains de cette ville crurent en lui, à cause de la parole de la femme qui avait rendu ce témoignage : Il m'a dit tout ce que j'ai fait.
\VS{40}Quand donc les Samaritains vinrent le trouver, ils le prièrent de demeurer avec eux ; et il demeura là deux jours.
\VS{41}Et beaucoup plus de gens crurent à cause de sa parole ;
\VS{42}et ils disaient à la femme : Ce n'est plus à cause de ta parole que nous croyons ; car nous l'avons entendu nous-mêmes, et nous savons qu’il est véritablement le Christ, le Sauveur du monde.
\VS{43}Après ces deux jours, Jésus partit de là, et s'en alla en Galilée.
\VS{44}Car il avait rendu témoignage qu'un prophète n'est pas honoré dans son pays.
\VS{45}Lorsqu’il arriva en Galilée, les Galiléens le reçurent, ayant vu toutes les choses qu'il avait faites à Jérusalem le jour de la Fête, car eux aussi étaient allés à la Fête.
\TextTitle{[Jésus guérit le fils d'un officier]}
\VS{46}Jésus retourna encore à Cana de Galilée, où il avait changé l'eau en vin. Or il y avait à Capernaüm un officier du roi, dont le fils était malade.
\VS{47}Ayant appris que Jésus était venu de Judée en Galilée, il alla vers lui, et le pria de descendre pour guérir son fils qui était près de mourir.
\VS{48}Mais Jésus lui dit : Si vous ne voyez pas des prodiges et des miracles, vous ne croyez point.
\VS{49}L’officier du roi lui dit : Seigneur, descends avant que mon fils meure.
\VS{50}Jésus lui dit : Va, ton fils vit. Cet homme crut à la parole que Jésus lui avait dite, et il s'en alla.
\VS{51}Et comme il descendait déjà, ses serviteurs vinrent au-devant de lui, et lui apportèrent des nouvelles, disant : Ton fils vit.
\VS{52}Et il leur demanda à quelle heure il s'était trouvé mieux ; et ils lui dirent : Hier, à la septième heure, la fièvre l’a quitté.
\VS{53}Le père reconnut que c'était à cette même heure-là que Jésus lui avait dit : Ton fils vit. Et il crut, avec toute sa maison.
\VS{54}Jésus fit encore ce second miracle quand il fut venu de Judée en Galilée.
\TextTitle{[Nouvelle fête des juifs et guérison d'un paralytique à la piscine de Béthesda]}
\Chap{5}
\VerseOne{}Après ces choses, il y eut une fête des Juifs, et Jésus monta à Jérusalem.
\VS{2}Or à Jérusalem, près de la porte des brebis, il y avait une piscine appelée en hébreu Béthesda, et qui avait cinq portiques.
\VS{3}Sous ces portiques étaient couchés un grand nombre de malades, des aveugles, des boiteux, des paralytiques, attendant le mouvement de l'eau.
\VS{4}Car un ange descendait de temps en temps dans la piscine, et agitait l'eau ; et alors le premier qui y descendait après que l'eau avait été agitée, était guéri, quelle que fût sa maladie.
\VS{5}Or il y avait là un homme malade depuis trente-huit ans.
\VS{6}Jésus, le voyant couché par terre, et sachant qu'il était déjà malade depuis longtemps, lui dit : Veux-tu être guéri ?
\VS{7}Le malade lui répondit : Seigneur, je n'ai personne pour me jeter dans la piscine quand l'eau est agitée, et pendant que j'y vais, un autre descend avant moi.
\VS{8}Jésus lui dit : Lève-toi, prends ton lit, et marche.
\VS{9}Et aussitôt cet homme fut guéri, il prit son lit, et marcha. Or c'était un jour de sabbat.
\VS{10}Les Juifs dirent donc à celui qui avait été guéri : C'est un jour de sabbat, il ne t'est pas permis de prendre ton lit.
\VS{11}Il leur répondit : Celui qui m'a guéri m'a dit : Prends ton lit et marche.
\VS{12}Alors ils lui demandèrent : Qui est celui qui t'a dit : Prends ton lit et marche ?
\VS{13}Mais celui qui avait été guéri ne savait pas qui c'était, car Jésus s'était éclipsé du milieu de la foule qui était en ce lieu-là.
\VS{14}Depuis, Jésus le trouva dans le temple, et lui dit : Voici, tu as été guéri ; ne pèche plus désormais, de peur qu’il ne t'arrive quelque chose de pire.
\VS{15}Cet homme s'en alla, et rapporta aux Juifs que c'était Jésus qui l'avait guéri.
\VS{16}C'est pourquoi les Juifs poursuivaient Jésus et cherchaient à le faire mourir, parce qu'il avait fait ces choses le jour du sabbat.
\TextTitle{[Jésus déclare son égalité avec le Père]}
\VS{17}Mais Jésus leur répondit : Mon Père agit jusqu'à présent ; moi aussi, j’agis.
\VS{18}A cause de cela, les Juifs cherchaient encore plus à le faire mourir, parce que non seulement il avait violé le sabbat, mais aussi parce qu'il disait que Dieu était son propre Père, se faisant égal à Dieu.
\VS{19}Mais Jésus répondit et leur dit : En vérité, en vérité je vous le dis, le Fils ne peut rien faire de lui-même, il ne fait que ce qu'il voit faire au Père ; et tout ce que le Père fait, le Fils le fait pareillement.
\VS{20}Car le Père aime le Fils, et lui montre toutes les choses qu'il fait ; et il lui montrera de plus grandes œuvres que celles-ci, afin que vous soyez dans l'admiration.
\VS{21}Car comme le Père ressuscite les morts et donne la vie, de même aussi le Fils donne la vie à ceux qu'il veut.
\VS{22}Car le Père ne juge personne ; mais il a donné tout jugement au Fils,
\VS{23}afin que tous honorent le Fils, comme ils honorent le Père ; celui qui n'honore point le Fils, n'honore point le Père qui l'a envoyé.
\VS{24}En vérité, en vérité je vous le dis, celui qui entend ma parole, et croit à celui qui m'a envoyé, a la vie éternelle et ne vient pas en jugement, mais il est passé de la mort à la vie.
\TextTitle{[Les deux résurrections]}
\VS{25}En vérité, en vérité je vous le dis, l'heure vient, et elle est déjà venue, où les morts entendront la voix du Fils de Dieu, et ceux qui l'auront entendue vivront.
\VS{26}Car comme le Père a la vie en lui-même, ainsi il a donné au Fils d'avoir la vie en lui-même.
\VS{27}Et il lui a donné le pouvoir de juger parce qu'il est le Fils de l'homme.
\VS{28}Ne soyez point étonnés de cela ; car l'heure vient où tous ceux qui sont dans les sépulcres entendront sa voix, et en sortiront.
\VS{29}Ceux qui auront fait le bien, ressusciteront pour la vie, mais ceux qui auront fait le mal, ressusciteront pour le jugement.
\TextTitle{[Témoignages confirmant celui de Jésus]}
\VS{30}Je ne puis rien faire de moi-même : Je juge conformément à ce que j'entends, et mon jugement est juste ; car je ne cherche point ma volonté, mais la volonté du Père qui m'a envoyé.
\VS{31}Si je rends témoignage de moi-même, mon témoignage n'est pas digne de foi.
\VS{32}C'est un autre qui rend témoignage de moi, et je sais que le témoignage qu'il rend de moi est digne de foi.
\TextTitle{[a. Le témoignage de Jean-Baptiste]}
\VS{33}Vous avez envoyé une délégation vers Jean, et il a rendu témoignage à la vérité.
\VS{34}Or je ne cherche point le témoignage des hommes ; mais je dis ces choses afin que vous soyez sauvés.
\VS{35}Jean était une lampe ardente et brillante ; et vous avez voulu vous réjouir pour un peu de temps à sa lumière.
\TextTitle{[b. Le témoignage des oeuvres de Jésus]}
\VS{36}Mais moi, j'ai un témoignage plus grand que celui de Jean ; car les œuvres que mon Père m'a donné d’accomplir, ces œuvres mêmes que je fais, témoignent de moi que c’est mon Père qui m'a envoyé.
\TextTitle{[c. Le témoignage du Père]
\\(Mt. 3:17)}
\VS{37}Et le Père qui m'a envoyé, a lui-même rendu témoignage de moi. Vous n’avez jamais entendu sa voix, vous n’avez jamais vu sa face.
\VS{38}Et sa parole ne demeure point en vous, puisque vous ne croyez pas à celui qu'il a envoyé.
\TextTitle{[d. Le témoignage de l'Ecriture]
\\(Lu. 24:27,44)}
\VS{39}Vous sondez les Ecritures, car vous pensez avoir en elles la vie éternelle, et ce sont elles qui rendent témoignage de moi.
\VS{40}Et vous ne voulez pas venir à moi, pour avoir la vie.
\VS{41}Je ne tire pas ma gloire des hommes.
\VS{42}Mais je sais que vous n'avez point l'amour de Dieu en vous.
\VS{43}Je suis venu au Nom de mon Père, et vous ne me recevez pas, si un autre vient en son propre nom, vous le recevrez.
\VS{44}Comment pouvez-vous croire, puisque vous recevez la gloire les uns des autres, et ne cherchez point la gloire qui vient de Dieu seul ?
\VS{45}Ne croyez point que je vous accuserai devant mon Père ; Moïse sur qui vous vous fondez, est celui qui vous accusera.
\VS{46}Car si vous croyiez Moïse, vous me croiriez aussi ; parce qu’il a écrit à mon sujet.
\VS{47}Mais si vous ne croyez pas à ses écrits, comment croirez-vous à mes paroles ?
\TextTitle{[Une autre Pâque et la multiplication des pains pour les cinq mille hommes]
\\(Mt. 14:15-21 ; Mc. 6:32-44 ; Lu. 9:12-17)}
\Chap{6}
\VerseOne{}Après ces choses, Jésus s'en alla au-delà de la mer de Galilée, qui est la mer de Tibériade.
\VS{2}Une grande foule le suivait, parce qu’elle voyait les miracles qu'il opérait sur les malades.
\VS{3}Jésus monta sur une montagne, et il s'assit là avec ses disciples.
\VS{4}Or, la Pâque, la fête des Juifs, était proche.
\VS{5}Et Jésus ayant levé ses yeux, et voyant qu’une grande foule venait à lui, dit à Philippe : Où achèterons-nous des pains, afin que ces gens aient à manger ?
\VS{6}Il disait cela pour l'éprouver, car il savait bien ce qu'il allait faire.
\VS{7}Philippe lui répondit : Les pains qu’on aurait pour deux cents deniers ne suffiraient pas pour que chacun en reçoive un peu.
\VS{8}Un de ses disciples, André, frère de Simon Pierre, lui dit :
\VS{9}Il y a ici un petit garçon qui a cinq pains d'orge et deux poissons ; mais qu'est-ce que cela pour tant de gens ?
\VS{10}Alors Jésus dit : Faites asseoir les gens. Il y avait beaucoup d'herbe dans ce lieu. Ils s'assirent au nombre d'environ cinq mille.
\VS{11}Et Jésus prit les pains ; et après avoir rendu grâces, il les distribua aux disciples, et les disciples à ceux qui étaient assis, et de même des poissons, autant qu'ils en voulaient.
\VS{12}Et après qu'ils furent rassasiés, il dit à ses disciples : Ramassez les morceaux qui restent, afin que rien ne soit perdu.
\VS{13}Ils les ramassèrent donc, et ils remplirent douze paniers avec les morceaux qui restèrent des cinq pains d'orge, après que tous eurent mangé.
\VS{14}Ces gens, ayant vu le miracle que Jésus avait fait, disaient : Celui-ci est véritablement le Prophète qui devait venir dans le monde.
\TextTitle{[Jésus marche sur les eaux]
\\(Mt. 14:22-33 ; Mc. 6:45-52)}
\VS{15}Mais Jésus, sachant qu'ils allaient venir l'enlever pour le faire Roi, se retira encore, lui seul, sur la montagne.
\VS{16}Et quand le soir fut venu, ses disciples descendirent à la mer.
\VS{17}Etant montés dans la barque, ils traversaient la mer pour se rendre à Capernaüm. Il faisait déjà nuit, et Jésus ne les avait pas encore rejoints.
\VS{18}Il soufflait un grand vent, et la mer était agitée.
\VS{19}Après avoir ramé environ vingt-cinq ou trente stades, ils virent Jésus marchant sur la mer, et s'approchant de la barque. Et ils eurent peur.
\VS{20}Mais il leur dit : C’est moi, ne craignez point.
\VS{21}Ils le reçurent donc avec plaisir dans la barque, et aussitôt la barque aborda au lieu où ils allaient.
\TextTitle{[Jésus, le pain de vie]}
\VS{22}Le lendemain, la foule qui était restée de l'autre côté de la mer, vit qu’il ne se trouvait là qu’une seule barque, et que Jésus n’était pas monté avec ses disciples dans la barque, mais qu’ils étaient partis seuls.
\VS{23}Cependant, d’autres barques étaient arrivées de Tibériade près du lieu où ils avaient mangé le pain, après que le Seigneur eut rendu grâces.
\VS{24}Quand la foule vit que ni Jésus ni ses disciples n’étaient là, les gens montèrent eux-mêmes dans ces barques, et allèrent à Capernaüm chercher Jésus.
\VS{25}Et l'ayant trouvé au-delà de la mer, ils lui dirent : Rabbi, quand es-tu arrivé ici ?
\VS{26}Jésus leur répondit et leur dit : En vérité, en vérité je vous le dis : Vous me cherchez, non parce que vous avez vu des miracles, mais parce que vous avez mangé des pains et que vous avez été rassasiés.
\VS{27}Travaillez, non pour la nourriture qui périt, mais pour celle qui est permanente jusqu’à la vie éternelle, et que le Fils de l'homme vous donnera ; car c’est lui que le Père, que Dieu, a marqué de son sceau.
\VS{28}Ils lui dirent donc : Que devons-nous faire pour accomplir les œuvres de Dieu ?
\VS{29}Jésus répondit et leur dit : C’est ici l’œuvre de Dieu, que vous croyiez en celui qu'il a envoyé.
\TextTitle{[Jésus envoyé du ciel]}
\VS{30}Alors ils lui dirent : Quel miracle fais-tu donc, afin que nous le voyions, et que nous croyions en toi ? Quelle œuvre fais-tu ?
\VS{31}Nos pères ont mangé la manne dans le désert ; selon ce qui est écrit : Il leur a donné à manger le pain du ciel\FTNT{} (1).
\VS{32}Mais Jésus leur dit : En vérité, en vérité je vous le dis : Moïse ne vous a pas donné le pain du ciel ; mais mon Père vous donne le vrai pain du ciel.
\VS{33}Car le pain de Dieu c'est celui qui est descendu du ciel et qui donne la vie au monde.
\VS{34}Ils lui dirent donc : Seigneur, donne-nous toujours ce pain-là.
\VS{35}Et Jésus leur dit : Je suis le pain de vie. Celui qui vient à moi, n'aura jamais faim ; et celui qui croit en moi, n'aura jamais soif.
\VS{36}Mais, je vous ai dit que vous m'avez vu, et cependant vous ne croyez point.
\VS{37}Tous ceux que mon Père me donne viendront à moi ; et je ne mettrai point dehors celui qui viendra à moi.
\VS{38}Car je suis descendu du ciel, non point pour faire ma volonté, mais la volonté de celui qui m'a envoyé.
\VS{39}Or, la volonté du Père qui m'a envoyé, c’est que je ne perde aucun de tous ceux qu'il m'a donnés, mais que je les ressuscite au dernier jour.
\VS{40}La volonté de celui qui m'a envoyé, c’est que quiconque contemple le Fils, et croit en lui, ait la vie éternelle ; et je le ressusciterai au dernier jour.
\VS{41}Les Juifs murmuraient contre lui de ce qu'il avait dit : Je suis le pain qui est descendu du ciel.
\VS{42}Et ils disaient : N’est-ce pas là Jésus, le fils de Joseph, celui dont nous connaissons le père et la mère ? Comment donc dit-il : Je suis descendu du ciel ?
\VS{43}Jésus leur répondit et leur dit : Ne murmurez pas entre vous.
\VS{44}Nul ne peut venir à moi, si le Père qui m'a envoyé ne l’attire ; et je le ressusciterai au dernier jour.
\VS{45}Il est écrit dans les prophètes : Ils seront tous enseignés de Dieu. Ainsi, quiconque a entendu le Père et a été instruit de ses intentions, vient à moi.
\VS{46}C’est que nul n’a vu le Père, sinon celui qui vient de Dieu, celui-là a vu le Père.
\VS{47}En vérité, en vérité je vous le dis : Celui qui croit en moi a la vie éternelle.
\VS{48}Je suis le pain de vie.
\VS{49}Vos pères ont mangé la manne dans le désert, et ils sont morts.
\VS{50}C'est ici le pain qui est descendu du ciel, afin que celui qui en mange, ne meure point.
\VS{51}Je suis le pain vivant qui est descendu du ciel. Si quelqu'un mange de ce pain, il vivra éternellement ; et le pain que je donnerai, c'est ma chair, que je donnerai pour la vie du monde.
\VS{52}Les Juifs donc discutaient entre eux, et disaient : Comment peut-il nous donner sa chair à manger ?
\VS{53}Et Jésus leur dit : En vérité, en vérité je vous le dis : Si vous ne mangez pas la chair du Fils de l'homme, et ne buvez pas son sang, vous n'aurez point la vie en vous-mêmes.
\VS{54}Celui qui mange ma chair, et qui boit mon sang, a la vie éternelle ; et je le ressusciterai au dernier jour.
\VS{55}Car ma chair est une véritable nourriture, et mon sang est un véritable breuvage.
\VS{56}Celui qui mange ma chair, et qui boit mon sang, demeure en moi, et moi en lui.
\VS{57}Comme le Père qui est vivant m'a envoyé, et que je suis vivant par le Père ; ainsi celui qui me mangera, vivra aussi par moi.
\VS{58}C'est ici le pain qui est descendu du ciel. Il n’en est pas comme de vos pères qui ont mangé la manne, et qui sont morts ; celui qui mangera ce pain, vivra éternellement.
\VS{59}Il dit ces choses dans la synagogue, enseignant à Capernaüm.
\TextTitle{[Epreuve de la consécration des disciples]
\\(Mt. 8:19-22 ; 10:36 ; Lu. 9:23-26)}
\VS{60}Plusieurs de ses disciples l'ayant entendu, dirent : Cette parole est dure, qui peut l’écouter ?
\VS{61}Mais Jésus sachant en lui-même que ses disciples murmuraient à ce sujet, leur dit : Cela vous scandalise-t-il ?
\VS{62}Que sera-ce donc si vous voyez le Fils de l'homme monter où il était auparavant ?
\VS{63}C'est l'Esprit qui vivifie ; la chair ne sert à rien. Les paroles que je vous ai dites, sont Esprit et vie.
\VS{64}Mais il en est parmi vous qui ne croient point. En effet, Jésus savait dès le commencement qui étaient ceux qui ne croiraient point, et qui était celui qui le trahirait.
\VS{65}Il leur dit donc : C’est pour cela que je vous ai dit, que nul ne peut venir à moi, si cela ne lui a pas été donné par mon Père.
\TextTitle{[Pierre reconnaît Jésus comme le Christ]
\\(Mt. 16:13-16 ; Mc. 8:27-30 ; Lu. 9:18-21}
\VS{66}Dès ce moment, plusieurs de ses disciples l'abandonnèrent, et ils ne marchèrent plus avec lui.
\VS{67}Et Jésus dit aux douze : Et vous, ne voulez-vous pas aussi vous en aller ?
\VS{68}Mais Simon Pierre lui répondit : Seigneur ! Auprès de qui irions-nous ? Tu as les paroles de la vie éternelle.
\VS{69}Et nous avons cru, et nous avons connu que tu es le Christ, le Fils du Dieu vivant.
\VS{70}Jésus leur répondit : Ne vous ai-je pas choisis, vous les douze ? Et toutefois l'un de vous est un démon.
\VS{71}Il parlait de Judas Iscariot, fils de Simon ; car c'était lui qui devait le trahir, quoiqu'il fût l'un des douze.
\TextTitle{[Jésus engagé par ses frères incrédules à se rendre à Jérusalem]}
\Chap{7}
\VerseOne{}Après ces choses, Jésus parcourait la Galilée, car il ne voulait pas parcourir la Judée, parce que les Juifs cherchaient à le faire mourir.
\VS{2}Or la fête des Juifs, appelée la fête des tabernacles, était proche.
\VS{3}Et ses frères lui dirent : Pars d'ici, et va en Judée, afin que tes disciples aussi contemplent les œuvres que tu fais.
\VS{4}Personne n’agit en secret, lorsqu'il cherche à être connu ; si tu fais ces choses, montre-toi toi-même au monde.
\VS{5}Car ses frères non plus ne croyaient pas en lui.
\VS{6}Et Jésus leur dit : Mon temps n'est pas encore venu, mais votre temps est toujours prêt.
\VS{7}Le monde ne peut pas vous haïr, mais il me hait parce que je rends témoignage contre lui que ses œuvres sont mauvaises.
\VS{8}Montez, vous, à cette fête ; pour moi, je n’y monte pas encore, parce que mon temps n'est pas encore accompli.
\VS{9}Après leur avoir dit ces choses, il resta en Galilée.
\TextTitle{[Jésus à la fête des tabernacles]}
\VS{10}Lorsque ses frères furent montés, alors il y monta aussi lui-même, non publiquement, mais comme en secret.
\VS{11}Les Juifs le cherchaient pendant la fête, et ils disaient : Où est-il ?
\VS{12}Et il y avait un grand murmure à son sujet parmi la foule. Les uns disaient : C’est un homme de bien ; et les autres disaient : Non, il séduit le peuple.
\VS{13}Toutefois personne ne parlait franchement de lui, à cause de la crainte qu'on avait des Juifs.
\VS{14}Vers le milieu de la fête, Jésus monta au temple. Et il enseignait.
\VS{15}Les Juifs s’étonnaient, disant : Comment connaît-il les Ecritures, lui qui n’a point étudié ?
\VS{16}Jésus leur répondit et dit : Ma doctrine n'est pas de moi, mais de celui qui m'a envoyé.
\VS{17}Si quelqu'un veut faire sa volonté, il connaîtra si ma doctrine est de Dieu, ou si je parle de moi-même.
\VS{18}Celui qui parle de son propre chef cherche sa propre gloire ; mais celui qui cherche la gloire de celui qui l'a envoyé, est véritable, et il n'y a point d'injustice en lui.
\VS{19}Moïse ne vous a-t-il pas donné la loi ? Cependant, nul de vous n'observe la loi. Pourquoi cherchez-vous à me faire mourir ?
\VS{20}La foule répondit : Tu as un démon ; qui est-ce qui cherche à te faire mourir ?
\VS{21}Jésus répondit et leur dit : J’ai fait une œuvre, et vous en êtes tous étonnés.
\VS{22}Moïse vous a donné la circoncision, non qu’elle vienne de Moïse, mais des pères, vous circoncisez bien un homme le jour du sabbat.
\VS{23}Si un homme reçoit la circoncision le jour du sabbat, afin que la loi de Moïse ne soit pas violée, pourquoi êtes-vous irrités contre moi de ce que j'ai guéri un homme tout entier le jour du sabbat ?
\VS{24}Ne jugez pas selon les apparences, mais jugez selon la justice.
\VS{25}Alors quelques-uns de ceux de Jérusalem disaient : N'est-ce pas celui qu'ils cherchent à faire mourir ?
\VS{26}Et cependant voici, il parle librement, et ils ne lui disent rien ! Est-ce que vraiment les chefs auraient reconnu qu’il est véritablement le Christ ?
\VS{27}Cependant celui-ci, nous savons d'où il est ; mais quand le Christ viendra, personne ne saura d'où il est.
\VS{28}Jésus, enseignant dans le temple, s’écria : Vous me connaissez, et vous savez d'où je suis ! Je ne suis pas venu de moi-même, mais celui qui m'a envoyé est véritable, et vous ne le connaissez pas.
\VS{29}Mais moi, je le connais ; car je viens de lui, et c'est lui qui m'a envoyé.
\VS{30}Ils cherchaient donc à se saisir de lui, mais personne ne mit la main sur lui, parce que son heure n'était pas encore venue.
\VS{31}Cependant, plusieurs parmi la foule crurent en lui, et ils disaient : Quand le Christ sera venu, fera-t-il plus de miracles que celui-ci n'a fait ?
\VS{32}Les pharisiens entendirent la foule murmurant ces choses de lui. Alors les principaux sacrificateurs et les pharisiens envoyèrent des huissiers pour le prendre.
\VS{33}Et Jésus leur dit : Je suis encore pour un peu de temps avec vous, puis je m'en vais vers celui qui m'a envoyé.
\VS{34}Vous me chercherez, mais vous ne me trouverez pas, et vous ne pouvez pas venir où je serai.
\VS{35}Les Juifs dirent donc entre eux : Où ira-t-il, pour que nous ne le trouvions pas ? Ira-t-il parmi ceux qui sont dispersés chez les Grecs, et enseignera-t-il les Grecs ?
\VS{36}Quel est ce discours qu'il a tenu : Vous me chercherez, mais vous ne me trouverez pas, vous ne pouvez pas venir où je serai ?
\TextTitle{[La grande prophétie sur le secret de la puissance du Saint-Esprit]
\\(Ac. 2:2-4 ; Jn. 4:14)}
\VS{37}Le dernier jour, le grand jour de la fête, Jésus, se tenant debout, s’écria : Si quelqu'un a soif, qu'il vienne à moi, et qu'il boive.
\VS{38}Celui qui croit en moi, des fleuves d'eau vive couleront de son sein, comme dit l'Ecriture.
\VS{39}Il dit cela de l'Esprit que devaient recevoir ceux qui croiraient en lui ; car le Saint-Esprit n'était pas encore donné, parce que Jésus n'était pas encore glorifié.
\TextTitle{[Diversité d'opinions au sujet de Jésus]}
\VS{40}Plusieurs de la foule ayant entendu ce discours, disaient : Celui-ci est véritablement le Prophète.
\VS{41}Les autres disaient : Celui-ci est le Christ. Et les autres disaient : Est-ce bien de la Galilée que doit venir le Christ ?
\VS{42}L'Ecriture ne dit-elle pas que le Christ doit venir de la postérité de David, et du village de Bethléem, où était David ?
\VS{43}Il y eut donc division parmi la foule à cause de lui.
\VS{44}Et quelques-uns d'entre eux voulaient le saisir, mais personne ne mit la main sur lui.
\VS{45}Ainsi les huissiers retournèrent vers les principaux sacrificateurs et les pharisiens, qui leur dirent : Pourquoi ne l'avez-vous pas amené ?
\VS{46}Les huissiers répondirent : Jamais homme n’a parlé comme cet homme.
\VS{47}Mais les pharisiens leur répondirent : Est-ce que vous aussi, vous avez été séduits ?
\VS{48}Y a-t-il quelqu’un des chefs ou des pharisiens qui ait cru en lui ?
\VS{49}Mais cette foule, qui ne connaît pas la loi, ce sont des maudits.
\VS{50}Nicodème, qui était venu vers Jésus de nuit, et qui était l'un d'entre eux, leur dit :
\VS{51}Notre loi condamne-t-elle un homme avant qu’on l’entende et qu’on ne sache ce qu’il a fait ?
\VS{52}Ils lui répondirent : Es-tu aussi Galiléen ? Examine, et tu verras qu'aucun prophète n’est sorti de la Galilée.
\VS{53}Et chacun s'en alla dans sa maison.
\TextTitle{[Les scribes et les pharisiens accusent une femme surprise en flagrant délit d'adultère]}
\Chap{8}
\VerseOne{}Jésus se rendit à la montagne des oliviers.
\VS{2}Et, dès le matin, il alla de nouveau dans le temple, et tout le peuple vint à lui ; et s'étant assis, il les enseignait.
\VS{3}Alors les scribes et les pharisiens lui amenèrent une femme surprise en adultère ;
\VS{4}et l'ayant placée au milieu du peuple, ils dirent à Jésus : Maître, cette femme a été surprise en flagrant délit d’adultère.
\VS{5}Moïse nous a ordonné dans la loi de lapider celles qui sont dans son cas ; toi donc qu'en dis-tu ?
\VS{6}Or ils disaient cela pour l'éprouver, afin de pouvoir l'accuser. Mais Jésus s'étant penché en bas, écrivait avec son doigt sur la terre.
\VS{7}Et comme ils continuaient à l'interroger, s'étant relevé, il leur dit : Que celui de vous qui est sans péché, jette le premier la pierre contre elle.
\VS{8}Et s'étant encore baissé, il écrivait sur la terre.
\VS{9}Quand ils entendirent cela, accusés par leur conscience, ils se retirèrent un à un, depuis les plus âgés jusqu’aux derniers ; et Jésus resta seul avec la femme qui était là au milieu.
\VS{10}Alors Jésus s'étant relevé, et ne voyant plus que la femme, il lui dit : Femme, où sont ceux qui t'accusaient ? Personne ne t’a-t-il condamnée ?
\VS{11}Elle dit : Non, Seigneur. Et Jésus lui dit : Je ne te condamne pas non plus ; va, et ne pèche plus.
\TextTitle{[Point crucial du conflit entre Jésus et les pharisiens : l'origine de Christ, Lumière du monde]
\\(Jn. 1:9)}
\VS{12}Et Jésus leur parla encore, en disant : Je suis la Lumière du monde ; celui qui me suit ne marchera pas dans les ténèbres, mais il aura la lumière de la vie.
\VS{13}Alors les pharisiens lui dirent : Tu rends témoignage de toi-même, ton témoignage n'est pas digne de foi.
\VS{14}Jésus répondit et leur dit : Quoique je rende témoignage de moi-même, mon témoignage est digne de foi ; car je sais d'où je suis venu et où je vais ; mais vous ne savez pas d'où je viens ni où je vais.
\VS{15}Vous jugez selon la chair, mais moi, je ne juge personne.
\VS{16}Et si je juge, mon jugement est digne de foi ; car je ne suis pas seul, mais le Père qui m'a envoyé est avec moi.
\VS{17}Il est même écrit dans votre loi que le témoignage de deux hommes est digne de foi\FTNT{De. 19:15.}.
\VS{18}Je rends témoignage de moi-même, et le Père qui m'a envoyé rend aussi témoignage de moi.
\VS{19}Alors ils lui dirent : Où est ton Père ? Jésus répondit : Vous ne connaissez ni moi ni mon Père. Si vous me connaissiez, vous connaîtriez aussi mon Père.
\VS{20}Jésus dit ces paroles au lieu où était le trésor, enseignant dans le temple ; mais personne ne le saisit, parce que son heure n'était pas encore venue.
\VS{21}Et Jésus leur dit encore : Je m'en vais, et vous me chercherez, et vous mourrez dans vos péchés ; vous ne pouvez pas venir où je vais.
\VS{22}Les Juifs disaient donc : Se tuera-t-il lui-même, puisqu’il dit : Vous ne pouvez pas venir où je vais ?
\VS{23}Alors il leur dit : Vous êtes d'en bas, mais moi, je suis d'en haut ; vous êtes de ce monde, mais moi, je ne suis pas de ce monde.
\VS{24}C'est pourquoi je vous ai dit que vous mourrez dans vos péchés ; car si vous ne croyez pas que je suis l'envoyé de Dieu, vous mourrez dans vos péchés.
\VS{25}Alors ils lui dirent : Toi, qui es-tu ? Et Jésus leur dit : Ce que je vous dis dès le commencement.
\VS{26}J'ai beaucoup de choses à dire de vous et à juger en vous, mais celui qui m'a envoyé est véritable, et les choses que j'ai entendues de lui, je les dis au monde.
\VS{27}Ils ne comprirent point qu'il leur parlait du Père.
\VS{28}Jésus leur dit donc : Quand vous aurez élevé le Fils de l'homme, vous connaîtrez alors que je suis l'envoyé de Dieu, et que je ne fais rien de moi-même, mais que je dis ces choses selon ce que mon Père m'a enseigné.
\VS{29}Celui qui m'a envoyé est avec moi ; le Père ne m'a pas laissé seul, parce que je fais toujours les choses qui lui plaisent.
\VS{30}Comme il disait ces choses, plusieurs crurent en lui.
\VS{31}Et Jésus disait aux Juifs qui avaient cru en lui : Si vous demeurez dans ma parole, vous serez vraiment mes disciples.
\VS{32}Vous connaîtrez la vérité, et la vérité vous rendra libres.
\VS{33}Ils lui répondirent : Nous sommes la postérité d'Abraham, et nous ne fûmes jamais esclaves de personne ; comment donc dis-tu : Vous deviendrez libres ?
\VS{34}Jésus leur répondit : En vérité, en vérité je vous le dis : Quiconque se livre au péché, est esclave du péché.
\VS{35}Or l'esclave ne demeure pas toujours dans la maison ; le fils y demeure toujours.
\VS{36}Si donc le Fils vous affranchit, vous serez véritablement libres.
\VS{37}Je sais que vous êtes la postérité d'Abraham, pourtant vous cherchez à me faire mourir, parce que ma parole n'est pas reçue dans vos cœurs.
\VS{38}Je vous dis ce que j'ai vu chez mon Père ; et vous aussi vous faites les choses que vous avez vues chez votre père.
\VS{39}Ils répondirent et lui dirent : Notre père c'est Abraham. Jésus leur dit : Si vous étiez enfants d'Abraham, vous feriez les œuvres d'Abraham.
\VS{40}Mais maintenant vous cherchez à me faire mourir, moi, un homme qui vous ai dit la vérité que j'ai entendue de Dieu. Cela, Abraham ne l’a point fait.
\VS{41}Vous faites les œuvres de votre père. Et ils lui dirent : Nous ne sommes pas des enfants illégitimes ; nous avons un seul père, Dieu.
\VS{42}Mais Jésus leur dit : Si Dieu était votre Père, certes vous m'aimeriez, car c’est de Dieu que je suis sorti et que je viens ; je ne suis pas venu de moi-même, mais c'est lui qui m'a envoyé.
\VS{43}Pourquoi ne comprenez-vous pas mon langage ? C’est parce que vous ne pouvez pas écouter ma parole.
\VS{44}Vous avez pour père le diable, et vous voulez accomplir les désirs de votre père. Il a été meurtrier dès le commencement, et il n'a pas persévéré dans la vérité, car la vérité n'est pas en lui. Toutes les fois qu'il profère le mensonge, il parle de son propre fond ; car il est menteur et le père du mensonge.
\VS{45}Et moi, parce que je dis la vérité, vous ne me croyez pas.
\VS{46}Qui de vous me convaincra de péché ? Si je dis la vérité, pourquoi ne me croyez-vous pas ?
\VS{47}Celui qui est de Dieu écoute les paroles de Dieu ; vous n’écoutez pas, parce que vous n'êtes pas de Dieu.
\VS{48}Alors les Juifs répondirent : N’avons-nous pas raison de dire que tu es un Samaritain, et que tu as un démon ?
\VS{49}Jésus répondit : Je n'ai point un démon, mais j'honore mon Père, et vous m’outragez.
\VS{50}Je ne cherche point ma gloire ; il y en a un qui la cherche, et qui juge.
\VS{51}En vérité, en vérité je vous le dis : Si quelqu'un garde ma parole, il ne verra jamais la mort.
\VS{52}Les Juifs lui dirent donc : Maintenant nous savons que tu as un démon. Abraham est mort, et les prophètes aussi, et tu dis : Si quelqu'un garde ma parole, il ne verra jamais la mort.
\VS{53}Es-tu plus grand que notre père Abraham qui est mort ? Les prophètes aussi sont morts. Qui prétends-tu être ?
\VS{54}Jésus répondit : Si je me glorifie moi-même, ma gloire n'est rien ; mon Père est celui qui me glorifie, celui que vous dites être votre Dieu.
\VS{55}Toutefois vous ne l'avez point connu, mais moi je le connais ; et si je disais que je ne le connais point, je serais un menteur, semblable à vous ; mais je le connais, et je garde sa parole.
\VS{56}Abraham votre père a tressailli de joie de ce qu’il verrait mon jour ; et il l'a vu, et il s’est réjoui.
\VS{57}Les Juifs lui dirent : Tu n'as pas encore cinquante ans, et tu as vu Abraham !
\VS{58}Jésus leur dit : En vérité, en vérité je vous le dis : Avant qu'Abraham fût, Je suis\FTNT{Je suis : L'évangile de Jean rapporte plusieurs déclarations incroyables que Jésus a faites à son sujet : Je suis le pain de vie (6:35), Je suis la Lumière du monde (8:12), Je suis le bon berger (10:11), Je suis la porte (10:7), Je suis la résurrection (11:25), Je suis le chemin, la vérité et la vie (14:6), Je suis la vraie vigne (15:1). Toutefois, dans ce verset, en déclarant être ~Je suis~, il s’identifie clairement au Nom que YHWH avait révélé à Moïse dans Ex. 3:14. C'est précisément pour cette raison que les juifs ont voulu le lapider.}.
\VS{59}Alors ils prirent des pierres pour les jeter contre lui, mais Jésus se cacha et sortit du temple, passant au milieu d'eux ; et ainsi il s'en alla.
\TextTitle{[Jésus guérit un aveugle-né]}
\Chap{9}
\VerseOne{}Comme Jésus passait, il vit un homme aveugle de naissance.
\VS{2}Ses disciples lui posèrent cette question : Rabbi, qui a péché ? Cet homme ou ses parents pour qu’il soit né aveugle ?
\VS{3}Jésus répondit : Ce n’est pas que lui ou ses parents aient péché ; mais c'est afin que les œuvres de Dieu soient manifestées en lui.
\VS{4}Il faut que je fasse, tandis qu’il est jour, les œuvres de celui qui m'a envoyé. La nuit vient, où personne ne peut travailler.
\VS{5}Pendant que je suis dans le monde, je suis la Lumière du monde.
\VS{6}Ayant dit ces paroles, il cracha à terre et fit de la boue avec sa salive, et mit de cette boue sur les yeux de l'aveugle.
\VS{7}Et il lui dit : Va, et lave-toi au réservoir de Siloé (nom qui veut dire envoyé). Il y alla donc, se lava, et s’en retourna voyant clair.
\VS{8}Ses voisins et ceux qui auparavant l’avaient connu comme mendiant disaient : N'est-ce pas celui qui était assis et qui mendiait ?
\VS{9}Les uns disaient : C’est lui. Et les autres disaient : Il lui ressemble. Mais lui-même disait : C'est moi.
\VS{10}Ils lui dirent donc : Comment tes yeux ont-ils été ouverts ?
\VS{11}Il répondit et dit : Cet homme, qu'on appelle Jésus, a fait de la boue et il l'a mise sur mes yeux, et m'a dit : Va au réservoir de Siloé et lave-toi. J’y suis allé, je me suis lavé, et j’ai recouvert la vue.
\VS{12}Alors ils lui dirent : Où est cet homme ? Il répondit : Je ne sais pas.
\VS{13}Ils amenèrent vers les pharisiens celui qui auparavant avait été aveugle.
\VS{14}Or c'était en un jour de sabbat que Jésus avait fait de la boue et lui avait ouvert les yeux.
\VS{15}C'est pourquoi les pharisiens l'interrogèrent encore, comment il avait pu voir ; et il leur dit : Il a mis de la boue sur mes yeux, et je me suis lavé, et je vois.
\VS{16}Sur quoi quelques-uns des pharisiens dirent : Cet homme n'est pas un envoyé de Dieu ; car il n’observe pas le sabbat ; mais d'autres disaient : Comment un homme pécheur peut-il faire de tels prodiges ? Et il y avait de la division entre eux.
\VS{17}Ils dirent encore à l'aveugle : Toi, que dis-tu de lui, sur ce qu'il t'a ouvert les yeux ? Il répondit : C’est un Prophète.
\VS{18}Mais les Juifs ne crurent point que cet homme avait été aveugle, et qu'il avait pu voir, jusqu'à ce qu'ils aient fait venir ses parents.
\VS{19}Et ils les interrogèrent, disant : Est-ce là votre fils, que vous dites être né aveugle ? Comment donc voit-il maintenant ?
\VS{20}Ses parents leur répondirent : Nous savons que c'est notre fils et qu'il est né aveugle.
\VS{21}Mais comment il voit maintenant, ou qui lui a ouvert les yeux, nous ne le savons pas ; il a de l'âge, interrogez-le, il parlera de ce qui le regarde.
\VS{22}Ses parents dirent ces choses parce qu'ils craignaient les Juifs ; car les Juifs avaient déjà convenu que si quelqu'un reconnaissait Jésus pour le Christ, il serait exclu de la synagogue.
\VS{23}C’est pourquoi ses parents dirent : Il a de l'âge, interrogez-le lui-même.
\VS{24}Ils appelèrent donc pour la seconde fois l'homme qui avait été aveugle et ils lui dirent : Donne gloire à Dieu ; nous savons que cet homme est un pécheur.
\VS{25}Il répondit : Je ne sais pas si c’est un pécheur ; je sais une chose, c’est que j’étais aveugle et que maintenant je vois.
\VS{26}Ils lui dirent donc encore : Que t'a-t-il fait ? Comment a-t-il ouvert tes yeux ?
\VS{27}Il leur répondit : Je vous l'ai déjà dit, et vous ne l'avez point écouté, pourquoi voulez-vous l’entendre encore ? Voulez-vous aussi devenir ses disciples ?
\VS{28}Alors ils l'injurièrent et lui dirent : C’est toi son disciple ; nous, nous sommes disciples de Moïse.
\VS{29}Nous savons que Dieu a parlé à Moïse ; mais celui-ci, nous ne savons pas d'où il est.
\VS{30}Cet homme répondit : Certes, c'est une chose étrange que vous ne sachiez point d'où il est ; et toutefois il a ouvert mes yeux.
\VS{31}Nous savons que Dieu n'exauce point les méchants, mais si quelqu'un est pieux envers Dieu, et fait sa volonté, il l'exauce.
\VS{32}Jamais on n’a entendu dire que quelqu’un ait ouvert les yeux d’un aveugle-né.
\VS{33}Si cet homme n'était pas un envoyé de Dieu, il ne pourrait rien faire de semblable.
\VS{34}Ils répondirent : Tu es entièrement né dans le péché, et tu nous enseignes ! Et ils le chassèrent dehors.
\TextTitle{[Jésus affirme sa divinité]}
\VS{35}Jésus apprit qu'ils l'avaient chassé dehors ; et l'ayant rencontré, il lui dit : Crois-tu au Fils de Dieu ?
\VS{36}Cet homme lui répondit : Qui est-il Seigneur, afin que je croie en lui ?
\VS{37}Jésus lui dit : Tu l'as vu, et c'est celui qui te parle.
\VS{38}Alors il dit : Je crois, Seigneur ; et il l'adora\FTNT{Au travers de la lecture de la Bible, on constate que les anges refusent l’adoration (Ap. 19:9-10) de même que les apôtres (Ac. 10:25-26 ; Ac. 14:5-18). Seul Dieu accepte l’adoration puisqu’il en est le seul digne. Jésus n’a jamais refusé l’adoration des hommes, car il est Dieu.}.
\VS{39}Et Jésus dit : Je suis venu dans ce monde pour exercer le jugement, afin que ceux qui ne voient point voient ; et que ceux qui voient deviennent aveugles.
\VS{40}Quelques pharisiens qui étaient avec lui, ayant entendu ces paroles, dirent : Et nous, sommes-nous aussi aveugles ?
\VS{41}Jésus leur répondit : Si vous étiez aveugles, vous n'auriez point de péché ; mais maintenant vous dites : Nous voyons. C’est à cause de cela que votre péché demeure.
\TextTitle{[Jésus, le Bon Berger]
\\(Ps. 23 ; Hé. 13:20 ; 1Pi. 5:4)}
\Chap{10}
\VerseOne{}En vérité, en vérité je vous le dis : Celui qui n'entre point par la porte dans la bergerie des brebis, mais y monte par ailleurs, est un voleur et un brigand.
\VS{2}Mais celui qui entre par la porte est le berger des brebis.
\VS{3}Le portier lui ouvre, et les brebis entendent sa voix, et il appelle les brebis qui lui appartiennent par leur nom, et il les conduit dehors.
\VS{4}Lorsqu’il a fait sortir toutes ses brebis dehors, il marche devant elles, et les brebis le suivent, parce qu'elles connaissent sa voix.
\VS{5}Mais elles ne suivront point un étranger, au contraire, elles fuiront loin de lui ; parce qu'elles ne connaissent point la voix des étrangers.
\VS{6}Jésus leur dit cette parabole, mais ils ne comprirent pas ce qu'il leur disait.
\VS{7}Jésus leur dit encore : En vérité, en vérité je vous le dis : Je suis la Porte par où entrent les brebis\FTNT{La porte des brebis était située près du temple et avait été bâtie du temps de Néhémie (Né. 3:1). Les animaux que l’on sacrifiait à Dieu franchissaient probablement cette porte.}.
\VS{8}Tout ceux qui sont venus avant moi sont des brigands et des voleurs ; mais les brebis ne les ont point écoutés.
\VS{9}Je suis la Porte : Si quelqu'un entre par moi, il sera sauvé ; il entrera et il sortira, et il trouvera des pâturages.
\VS{10}Le voleur ne vient que pour dérober, tuer et détruire ; moi, je suis venu afin que mes brebis aient la vie, et qu'elles l'aient même en abondance.
\VS{11}Je suis le bon berger : Le bon berger donne sa vie pour ses brebis.
\VS{12}Mais le mercenaire, qui n’est pas le berger, à qui n'appartiennent pas les brebis, voit venir le loup, abandonne les brebis, et s'enfuit ; et le loup ravit et disperse les brebis.
\VS{13}Ainsi le mercenaire s'enfuit, parce qu'il est mercenaire, et qu'il ne se soucie pas des brebis. Je suis le bon berger.
\VS{14}Je connais mes brebis, et mes brebis me connaissent.
\VS{15}Comme le Père me connaît, et comme je connais le Père ; et je donne ma vie pour mes brebis.
\VS{16}J'ai encore d'autres brebis qui ne sont pas de cette bergerie ; celles-là, il faut aussi que je les amène ; elles entendront ma voix, et il y aura un seul troupeau, et un seul berger.
\VS{17}Le Père m'aime, parce que je donne ma vie, afin de la reprendre.
\VS{18}Personne ne me l'ôte, mais je la donne de moi-même. J'ai le pouvoir de la donner, et j’ai le pouvoir de la reprendre ; j'ai reçu cet ordre de mon Père.
\VS{19}Il y eut de nouveau division parmi les Juifs à cause de ces discours.
\VS{20}Car plusieurs disaient : Il a un démon, il est fou ! Pourquoi l'écoutez-vous ?
\VS{21}Et les autres disaient : Ce ne sont pas les paroles d'un démoniaque ; un démon peut-il ouvrir les yeux des aveugles ?
\TextTitle{[Jésus réaffirme sa divinité]
\\(Jn. 5:26-27 ; 14:9 ; 20:28-29)}
\VS{22}On célébrait la fête de la dédicace\FTNT{Le terme ~dédicace~ est la traduction du mot hébreu ~Hanoukka~ qui sert à désigner la consécration ou l'inauguration de l'autel servant à offrir des sacrifices à Dieu (No. 7:10 ; 2 Ch. 7:9). La Bible l’utilise aussi pour parler de l'inauguration des murailles de Jérusalem après leur reconstruction au temps de Néhémie (Né. 12:27). La fête d’Hanoukka a été instituée par Judas Maccabé en 164 av. J.-C. en mémoire de la purification du temple qui avait été profané par Antiochus Epiphane. Elle débute le 25 du mois de chisleu (mi décembre) de chaque année et dure huit jours.} à Jérusalem. Et c'était l’hiver.
\VS{23}Et Jésus se promenait dans le temple, au portique de Salomon.
\VS{24}Et les Juifs l’entourèrent et lui dirent : Jusqu’à quand tiendras-tu notre âme en suspens ? Si tu es le Christ, dis-le-nous franchement.
\VS{25}Jésus leur répondit : Je vous l'ai dit, et vous ne le croyez point. Les œuvres que je fais au Nom de mon Père rendent témoignage de moi.
\VS{26}Mais vous ne croyez point, parce que vous n'êtes point de mes brebis, comme je vous l'ai dit.
\VS{27}Mes brebis entendent ma voix ; je les connais, et elles me suivent.
\VS{28}Et moi, je leur donne la vie éternelle, et elles ne périront jamais ; et personne ne les ravira de ma main.
\VS{29}Mon Père, qui me les a données, est plus grand que tous ; et personne ne peut les ravir des mains de mon Père.
\VS{30}Moi et le Père nous sommes un.
\VS{31}Alors les Juifs prirent de nouveau des pierres pour le lapider.
\VS{32}Jésus leur dit : Je vous ai fait voir plusieurs bonnes œuvres de la part de mon Père : Pour laquelle me lapidez-vous ?
\VS{33}Les Juifs répondirent : Ce n’est pas pour une bonne œuvre que nous te lapidons, mais pour un blasphème, parce que toi qui es un homme, tu te fais Dieu.
\VS{34}Jésus leur répondit : N’est-il pas écrit dans votre loi : J'ai dit : Vous êtes des dieux\FTNT{Ps. 82:6 : Le sens du mot ~dieu~ peut désigner des personnes ayant un certain pouvoir. D'ailleurs, le mot hébreu utilisé dans Ps. 82:6 est ~Elohim~, or ce mot signifie aussi ~juge~. De plus, dans le contexte du psaume, ~vous êtes des dieux~ ne s'applique pas à tous, mais seulement à une certaine catégorie de personnes qui exerçaient un pouvoir en Israël : rois, scribes, souverains sacrificateurs… Rappelons-nous aussi que Dieu a fait de Moïse un dieu pour Aaron (Ex. 7:1-2), mais cela n’a pas fait de lui le Dieu Créateur pour autant. En Jn. 17:3, Jésus atteste qu'il n’y a qu’un seul vrai Dieu. Satan veut nous faire croire que nous sommes des dieux et nous amener ainsi à pécher par l’orgueil (Ge. 3:5). Toutefois, comme le souligne si bien l’apôtre Paul, même s’il existe des créatures qu’on appelle dieux ou déesses, il ne reste pas moins vrai qu’il n’y a qu’un seul Dieu (1 Co. 8:5-7).} ?
\VS{35}Si elle a appelé dieux ceux à qui la parole de Dieu est adressée, et cependant l'Ecriture ne peut être anéantie,
\VS{36}celui que le Père a sanctifié et envoyé dans le monde, vous lui dites : Tu blasphèmes ! Et cela parce que j’ai dit : Je suis le Fils de Dieu ?
\VS{37}Si je ne faisais pas les œuvres de mon Père, ne me croyez pas.
\VS{38}Mais si je les fais, même si vous ne me croyez pas, croyez à ces œuvres, afin que vous sachiez que le Père est en moi et que je suis dans le Père.
\VS{39}Là-dessus, ils cherchaient encore à le saisir ; mais il s’échappa de leurs mains.
\TextTitle{[Jésus se retire de Jérusalem]}
\VS{40}Il s'en alla de nouveau au-delà du Jourdain, à l'endroit où Jean avait baptisé au commencement, et il demeura là.
\VS{41}Beaucoup de gens vinrent à lui, et ils disaient : Jean n’a fait aucun miracle ; mais tout ce que Jean a dit de cet homme, était vrai.
\VS{42}Et dans ce lieu-là, plusieurs crurent en lui.
\TextTitle{[Jésus ressuscite Lazare de Béthanie]}
\Chap{11}
\VerseOne{}Il y avait un homme malade, Lazare, de Béthanie, village de Marie et de Marthe, sa sœur.
\VS{2}C’était cette Marie qui oignit de parfum le Seigneur, et qui essuya ses pieds avec ses cheveux ; et c’était son frère Lazare qui était malade.
\VS{3}Ses sœurs envoyèrent donc dire à Jésus : Seigneur, voici, celui que tu aimes est malade.
\VS{4}Après avoir entendu cela, Jésus dit : Cette maladie n'est point à la mort, mais elle est pour la gloire de Dieu, afin que le Fils de Dieu soit glorifié par elle.
\VS{5}Or Jésus aimait Marthe, sa sœur, et Lazare.
\VS{6}Après qu'il eut appris que Lazare était malade, il resta deux jours encore dans le lieu où il était,
\VS{7}et il dit à ses disciples : Retournons en Judée.
\VS{8}Les disciples lui dirent : Rabbi, les Juifs tout récemment cherchaient à te lapider, et tu retournes en Judée !
\VS{9}Jésus répondit : N'y a-t-il pas douze heures au jour ? Si quelqu'un marche pendant le jour, il ne bronche point ; car il voit la lumière de ce monde.
\VS{10}Mais si quelqu'un marche pendant la nuit, il bronche ; car il n'y a point de lumière avec lui.
\VS{11}Après ces paroles, il leur dit : Notre ami Lazare dort, mais je vais le réveiller.
\VS{12}Ses disciples lui dirent : Seigneur, s'il dort, il sera guéri.
\VS{13}Jésus avait parlé de sa mort, mais ils pensaient qu'il parlait de l’assoupissement.
\VS{14}Alors Jésus leur dit ouvertement : Lazare est mort.
\VS{15}Et je me réjouis, à cause de vous, de ce que je n’étais pas là, afin que vous croyiez. Mais allons vers lui.
\VS{16}Alors Thomas, appelé Didyme, dit aux autres disciples : Allons-y aussi, afin que nous mourions avec lui.
\VS{17}Jésus, étant arrivé, trouva que Lazare était déjà depuis quatre jours dans le sépulcre.
\VS{18}Et comme Béthanie était près de Jérusalem à quinze stades environ,
\VS{19}beaucoup de Juifs étaient venus vers Marthe et Marie pour les consoler au sujet de leur frère.
\VS{20}Lorsque Marthe apprit que Jésus arrivait, elle alla au-devant de lui ; mais Marie se tenait assise à la maison.
\VS{21}Marthe dit à Jésus : Seigneur, si tu avais été ici mon frère ne serait pas mort.
\VS{22}Mais maintenant je sais que tout ce que tu demanderas à Dieu, Dieu te le donnera.
\VS{23}Jésus lui dit : Ton frère ressuscitera.
\VS{24}Marthe lui dit : Je sais qu'il ressuscitera à la résurrection, au dernier jour.
\VS{25}Jésus lui dit : Je suis la résurrection et la vie : Celui qui croit en moi vivra même s’il meurt.
\VS{26}Et quiconque vit et croit en moi ne mourra jamais ; crois-tu cela ?
\VS{27}Elle lui dit : Oui, Seigneur, je crois que tu es le Christ, le Fils de Dieu, qui devait venir dans le monde.
\VS{28}Ayant ainsi parlé, elle alla appeler secrètement Marie sa sœur, en lui disant : Le Maître est ici, et il t'appelle.
\VS{29}Aussitôt que Marie eut entendu, elle se leva rapidement, et alla vers lui.
\VS{30}Or Jésus n'était pas encore entré dans le village, mais il était au lieu où Marthe l'avait rencontré.
\VS{31}Alors les Juifs qui étaient avec Marie à la maison, et qui la consolaient, ayant vu qu'elle s'était levée si promptement, et qu'elle était sortie, la suivirent en disant : Elle va au sépulcre pour y pleurer.
\VS{32}Lorsque Marie fut arrivée où était Jésus, et qu’elle le vit, elle se jeta à ses pieds, en lui disant : Seigneur, si tu avais été ici, mon frère ne serait pas mort.
\VS{33}Jésus, la voyant pleurer, elle et les Juifs qui étaient venus avec elle, frémit en son esprit et fut tout ému.
\VS{34}Et il dit : Où l'avez-vous mis ? Ils lui répondirent : Seigneur, viens et vois.
\VS{35}Jésus pleura.
\VS{36}Sur quoi les Juifs dirent : Voyez comme il l'aimait.
\VS{37}Et quelques-uns d'entre eux disaient : Lui qui a ouvert les yeux de l'aveugle, ne pouvait-il pas faire aussi que cet homme ne meure point ?
\VS{38}Alors Jésus frémissant de nouveau en lui-même, se rendit au sépulcre. C'était une grotte, et il y avait une pierre placée devant.
\VS{39}Jésus dit : Ôtez la pierre. Mais Marthe, la sœur du mort, lui dit : Seigneur, il sent déjà, car il est là depuis quatre jours.
\VS{40}Jésus lui dit : Ne t'ai-je pas dit que si tu crois tu verras la gloire de Dieu ?
\VS{41}Ils ôtèrent donc la pierre de dessus le lieu où le mort était couché. Et Jésus levant ses yeux au ciel, dit : Père, je te rends grâces de ce que tu m'as exaucé.
\VS{42}Pour moi, je savais que tu m'exauces toujours ; mais j’ai parlé à cause de la foule qui m’entoure, afin qu’ils croient que c’est toi qui m'as envoyé.
\VS{43}Ayant dit ces choses, il cria à haute voix : Lazare sors dehors !
\VS{44}Alors le mort sortit, ayant les mains et les pieds liés de bandes ; et son visage était enveloppé d'un linge. Jésus leur dit : Déliez-le, et laissez-le aller.
\TextTitle{[Nombreuses conversions]
\\(Jn. 12:10-11)}
\TextTitle{[Conspiration des Pharisiens]}
\VS{45}Plusieurs des Juifs qui étaient venus vers Marie, et qui avaient vu ce que Jésus avait fait, crurent en lui.
\VS{46}Mais quelques-uns d'entre eux allèrent trouver les pharisiens et leur dirent les choses que Jésus avait faites.
\VS{47}Alors les principaux sacrificateurs et les pharisiens assemblèrent le sanhédrin, et ils dirent : Que ferons-nous ? Car cet homme fait beaucoup de miracles.
\VS{48}Si nous le laissons faire, tout le monde croira en lui, et les Romains viendront et ils détruiront et ce lieu et notre nation.
\VS{49}Alors l'un d'eux appelé Caïphe, qui était le souverain sacrificateur cette année-là, leur dit : Vous n’y comprenez rien.
\VS{50}Et vous ne réfléchissez pas qu'il est de notre intérêt qu'un homme meure pour le peuple, et que toute la nation ne périsse point.
\VS{51}Or il ne dit pas cela de lui-même, mais étant souverain sacrificateur de cette année-là, il prophétisa que Jésus devait mourir pour la nation.
\VS{52}Et non pas seulement pour la nation, mais aussi pour rassembler en un seul corps les enfants de Dieu dispersés.
\VS{53}Depuis ce jour, ils se concertèrent ensemble pour le faire mourir.
\VS{54}C'est pourquoi Jésus ne se montrait plus ouvertement parmi les Juifs, mais il se retira dans la contrée voisine du désert, dans une ville appelée Ephraïm, et il demeura là avec ses disciples.
\VS{55}La Pâque des Juifs était proche. Et beaucoup de gens du pays montèrent à Jérusalem avant Pâque, afin de se purifier.
\VS{56}Et ils cherchaient Jésus, et se disaient les uns les autres dans le temple : Que vous en semble ? Ne viendra-t-il pas à la Fête ?
\VS{57}Or, les principaux sacrificateurs et les pharisiens avaient donné l’ordre que si quelqu'un savait où il était, il le déclare, afin qu’on se saisisse de lui.
\TextTitle{[Marie de Béthanie oint les pieds de Jésus]
\\(Mt. 26:6-13 ; Mc. 14:3-9)}
\Chap{12}
\VerseOne{}Six jours avant la Pâque, Jésus arriva à Béthanie, où était Lazare qui avait été mort, et qu'il avait ressuscité des morts.
\VS{2}Là, on lui fit un souper ; Marthe servait, et Lazare était un de ceux qui étaient à table avec lui.
\VS{3}Alors Marie ayant pris une livre de nard pur de grand prix, oignit les pieds de Jésus, et les essuya avec ses cheveux ; et la maison fut remplie de l'odeur du parfum.
\VS{4}Alors Judas Iscariot, fils de Simon, l'un de ses disciples, celui qui devait le trahir, dit :
\VS{5}Pourquoi ce parfum n'a-t-il pas été vendu trois cents deniers, pour donner cet argent aux pauvres ?
\VS{6}Il dit cela, non parce qu’il se mettait en peine des pauvres, mais parce qu'il était voleur, et que tenant la bourse, il prenait ce qu’on y mettait.
\VS{7}Mais Jésus lui dit : Laisse-la faire ; elle l'a gardé pour le jour de ma sépulture.
\VS{8}Car vous aurez toujours des pauvres avec vous ; mais vous ne m'aurez pas toujours.
\VS{9}Une grande multitude des Juifs apprirent que Jésus était à Béthanie, et ils y vinrent, non seulement à cause de lui, mais aussi pour voir Lazare qu'il avait ressuscité des morts.
\VS{10}Sur quoi les principaux sacrificateurs résolurent de faire mourir aussi Lazare.
\VS{11}Car plusieurs des Juifs se retiraient d'avec eux à cause de lui, et croyaient en Jésus.
\TextTitle{[Entrée triomphante de Jésus à Jérusalem]
\\(Mt. 21:1-11 ; Mc. 11:1-11 ; Lu. 19:28-40 ; Za. 9:9 ; Ap. 19:11-16}
\VS{12}Le lendemain, une grande quantité de foules qui étaient venues à la fête, ayant entendu dire que Jésus se rendait à Jérusalem,
\VS{13}prit des branches de palmes, et sortit au-devant de lui en criant : Hosanna ! Béni soit le Roi d'Israël qui vient au Nom du Seigneur !
\VS{14}Jésus trouva un ânon, s'assit dessus, selon ce qui est écrit : 15 Ne crains point, fille de Sion ; voici, ton Roi vient, assis sur le petit d'une ânesse\FTNT{Za. 9:9.}.
\VS{16}Ses disciples ne comprirent pas d'abord ces choses ; mais quand Jésus eut été glorifié, ils se souvinrent alors qu’elles étaient écrites de lui, et qu’elles avaient été accomplies à son égard.
\VS{17}Tous ceux qui avaient été avec Jésus, quand il appela Lazare du sépulcre et le ressuscita des morts, lui rendaient témoignage ;
\VS{18}et la foule alla au-devant de lui, parce qu’elle avait appris qu'il avait fait ce miracle.
\VS{19}Sur quoi les pharisiens dirent entre eux : Vous ne voyez pas que vous ne gagnez rien ? Voici, le monde va après lui.
\TextTitle{[Quelques Grecs cherchent à voir Jésus]}
\VS{20}Quelques Grecs du nombre de ceux qui étaient montés pour adorer Dieu pendant la fête,
\VS{21}s’adressèrent à Philippe, qui était de Bethsaïda de Galilée, et lui dirent avec instances : Seigneur ! Nous voudrions voir Jésus.
\VS{22}Philippe alla le dire à André, et André et Philippe le dirent à Jésus.
\TextTitle{[Jésus annonce sa crucifixion]}
\VS{23}Jésus leur répondit, disant : L'heure est venue où le Fils de l'homme doit être glorifié.
\VS{24}En vérité, en vérité je vous le dis : Si le grain de blé qui est tombé en la terre ne meurt, il reste seul ; mais s'il meurt, il porte beaucoup de fruits.
\VS{25}Celui qui aime sa vie la perdra ; et celui qui hait sa vie dans ce monde, la conservera pour la vie éternelle.
\VS{26}Si quelqu'un me sert, qu'il me suive ; et là où je serai, là aussi sera celui qui me sert ; et si quelqu'un me sert, mon Père l'honorera.
\VS{27}Maintenant mon âme est troublée. Et que dirai-je ? Ô Père, délivre-moi de cette heure ? Mais c'est pour cela que je suis venu jusqu’à cette heure.
\VS{28}Père glorifie ton Nom ! Alors une voix vint du ciel et dit : Je l'ai glorifié, et je le glorifierai encore.
\VS{29}Et la foule qui était là, et qui avait entendu cette voix, disait que c'était un coup de tonnerre ; les autres disaient : Un ange lui a parlé.
\VS{30}Jésus prit la parole et dit : Ce n’est pas à cause de moi que cette voix s’est fait entendre ; c’est à cause de vous.
\VS{31}Maintenant est venu le jugement de ce monde ; maintenant le prince de ce monde sera jeté dehors.
\VS{32}Et moi, quand je serai élevé de la terre, j’attirerai tous les hommes à moi.
\VS{33}En parlant ainsi, il indiquait de quelle mort il devait mourir.
\VS{34}La foule lui répondit : Nous avons appris par la loi que le Christ demeure éternellement, comment donc dis-tu qu'il faut que le Fils de l'homme soit élevé ? Qui est ce Fils de l'homme ?
\VS{35}Alors Jésus leur dit : La Lumière est encore avec vous pour un peu de temps : Marchez pendant que vous avez la Lumière, de peur que les ténèbres ne vous surprennent ; car celui qui marche dans les ténèbres ne sait pas où il va.
\VS{36}Pendant que vous avez la Lumière, croyez en la Lumière, afin que vous soyez enfants de lumière. Jésus dit ces choses, puis il s'en alla, et se cacha de devant eux.
\VS{37}Malgré tant de miracles qu’il avait faits en leur présence, ils ne croyaient point en lui,
\VS{38}afin que s’accomplisse cette parole qui a été dite par Esaïe le prophète : Seigneur, qui a cru à notre parole, et à qui a été révélé le bras du Seigneur\FTNT{Es. 53:1.} ?
\VS{39}C'est pourquoi ils ne pouvaient pas croire, parce qu'Esaïe a dit encore :
\VS{40}Il a aveuglé leurs yeux, et il a endurci leur cœur, de peur qu'ils ne voient de leurs yeux, qu'ils ne comprennent du cœur, qu'ils ne se convertissent, et que je ne les guérisse\FTNT{Es. 6:9-10.}.
\VS{41}Esaïe dit ces choses quand il vit sa gloire, et qu'il parla de lui.
\VS{42}Cependant, même parmi les chefs, plusieurs crurent en lui ; mais ils ne le confessaient pas à cause des pharisiens, de peur d'être exclus de la synagogue.
\VS{43}Car ils aimèrent la gloire des hommes, plus que la gloire de Dieu.
\VS{44}Or Jésus s'écria et dit : Celui qui croit en moi, ne croit pas seulement en moi, mais en celui qui m'a envoyé.
\VS{45}Et celui qui me voit, voit celui qui m'a envoyé.
\VS{46}Je suis venu dans le monde pour en être la Lumière, afin que quiconque croit en moi ne demeure point dans les ténèbres.
\VS{47}Si quelqu'un entend mes paroles, et ne les garde point, ce n’est pas moi qui le juge ; car je ne suis point venu pour juger le monde, mais pour sauver le monde.
\VS{48}Celui qui me rejette et qui ne reçoit pas mes paroles, a son juge : La parole que j'ai annoncée sera celle qui le jugera au dernier jour.
\VS{49}Car je n'ai point parlé de moi-même, mais le Père qui m'a envoyé, m'a prescrit ce que je dois dire et annoncer.
\VS{50}Et je sais que son commandement est la vie éternelle ; les choses donc que je dis, je les dis comme mon Père me les a dites.
\TextTitle{[L'entretien de Jn. 13-14 eut lieu dans la chambre haute ; Mc. 14:14-16]}
\Chap{13}
\VerseOne{}Avant la fête de Pâque, Jésus sachant que son heure était venue de passer de ce monde au Père, et ayant aimé les siens, qui étaient dans le monde, il les aima jusqu'à la fin.
\TextTitle{[La dernière Pâque ; Jésus lave les pieds de ses disciples]
\\(Mt. 26:20-24 ; Mc. 14:17 ; Lu. 22:14,21-23)}
\VS{2}Pendant le souper, alors que le diable avait déjà mis dans le cœur de Judas Iscariot, fils de Simon, de le trahir,
\VS{3}Jésus sachant que le Père avait remis toutes choses entre ses mains, qu'il était venu de Dieu, et qu’il s'en allait à Dieu,
\VS{4}se leva de table, ôta ses vêtements, et prit un linge, dont il se ceignit.
\VS{5}Puis il mit de l'eau dans un bassin, et se mit à laver les pieds de ses disciples, et à les essuyer avec le linge dont il se ceignit.
\VS{6}Alors il vint à Simon Pierre, mais Pierre lui dit : Toi, Seigneur, tu me laves les pieds ?
\VS{7}Jésus répondit et lui dit : Tu ne comprends pas maintenant ce que je fais, mais tu le sauras dans la suite.
\VS{8}Pierre lui dit : Tu ne me laveras jamais les pieds ! Jésus lui répondit : Si je ne te lave pas, tu n'auras point de part avec moi.
\VS{9}Simon Pierre lui dit : Seigneur, non seulement mes pieds, mais aussi les mains et la tête.
\VS{10}Jésus lui dit : Celui qui est baigné n’a besoin que de se laver les pieds pour être entièrement pur ; vous êtes purs, mais non pas tous.
\VS{11}Car il savait qui était celui qui le trahirait ; c'est pourquoi il dit : Vous n'êtes pas tous purs.
\VS{12}Après qu'il leur eut lavé les pieds, il reprit ses vêtements, et s'étant remis à table, il leur dit : Comprenez-vous ce que je vous ai fait ?
\VS{13}Vous m'appelez Maître et Seigneur ; et vous dites bien, car je le suis.
\VS{14}Si donc moi, qui suis le Seigneur et le Maître, j'ai lavé vos pieds, vous devez aussi vous laver les pieds les uns des autres.
\VS{15}Car je vous ai donné un exemple, afin que vous fassiez comme je vous ai fait.
\VS{16}En vérité, en vérité je vous le dis : Le serviteur n'est pas plus grand que son maître ni l’apôtre plus grand que celui qui l'a envoyé.
\VS{17}Si vous savez ces choses, vous êtes heureux, pourvu que vous les pratiquiez.
\VS{18}Je ne parle pas de vous tous, je connais ceux que j'ai choisis. Mais il faut que l’Ecriture s’accomplisse : Celui qui mange le pain avec moi, a levé son talon contre moi\FTNT{Ps. 41:10.}.
\VS{19}Je vous dis ceci dès maintenant, avant que la chose arrive, afin que lorsqu’elle arrivera, vous croyiez que c'est moi que le Père a envoyé.
\VS{20}En vérité, en vérité je vous le dis : Celui qui reçoit celui que j’aurai envoyé, me reçoit ; et celui qui me reçoit, reçoit celui qui m'a envoyé.
\TextTitle{[Jésus annonce la trahison de Judas]
\\(Mt. 26:21-25 ; Mc. 14:18-21 ; Lu. 22:21-23)}
\VS{21}Ayant ainsi parlé, Jésus fut ému dans son esprit, et il déclara : En vérité, en vérité je vous le dis, l'un de vous me trahira.
\VS{22}Alors les disciples se regardaient les uns les autres, ne sachant de qui il parlait.
\VS{23}Un des disciples, celui que Jésus aimait, était à table couché sur le sein de Jésus.
\VS{24}Simon Pierre lui fit signe de demander qui était celui dont Jésus parlait.
\VS{25}Et ce disciple, s’étant penché sur la poitrine de Jésus, lui dit : Seigneur, qui est-ce ?
\VS{26}Jésus répondit : C'est celui à qui je donnerai le morceau trempé ; et ayant trempé le morceau, il le donna à Judas Iscariot, fils de Simon.
\VS{27}Après que Judas eut pris le morceau, Satan entra en lui. Jésus lui dit : Ce que tu fais, fais-le promptement.
\VS{28}Mais aucun de ceux qui étaient à table ne comprit pourquoi il lui avait dit cela.
\VS{29}Car quelques-uns pensaient que, comme Judas avait la bourse, Jésus voulait lui dire : Achète ce qui nous est nécessaire pour la Fête ; ou qu'il lui commandait de donner quelque chose aux pauvres.
\VS{30}Judas, ayant pris le morceau, sortit aussitôt. Il faisait nuit.
\VS{31}Lorsque Judas fut sorti, Jésus dit : Maintenant le Fils de l'homme est glorifié ; et Dieu est glorifié en lui.
\VS{32}Si Dieu est glorifié en lui, Dieu aussi le glorifiera en lui-même, et il le glorifiera bientôt.
\VS{33}Mes petits enfants, je suis encore pour un peu de temps avec vous ; vous me chercherez, mais comme j'ai dit aux Juifs : Vous ne pouvez pas venir où je vais, je vous le dis aussi maintenant.
\VS{34}Je vous donne un nouveau commandement : Aimez-vous les uns les autres. Comme je vous ai aimés, vous aussi, aimez-vous les uns les autres.
\VS{35}A ceci tous connaîtront que vous êtes mes disciples, si vous avez de l'amour les uns pour les autres.
\TextTitle{[Jésus annonce le reniement de Pierre]
\\(Mt. 26:30-35 ; Mc. 14:26-31 ; Lu. 22:31-34)}
\VS{36}Simon Pierre lui dit : Seigneur ! Où vas-tu ? Jésus lui répondit : Là où je vais, tu ne peux pas me suivre maintenant, mais tu me suivras plus tard.
\VS{37}Pierre lui dit : Seigneur ! Pourquoi ne puis-je pas te suivre maintenant ? J’exposerai ma vie pour toi.
\VS{38}Jésus lui répondit : Tu exposeras ta vie pour moi ? En vérité, en vérité je te le dis, le coq ne chantera pas, que tu ne m'aies renié trois fois.
\TextTitle{[Jésus réconforte les apôtres : Il reviendra vers eux]}
\Chap{14}
\VerseOne{}Que votre cœur ne se trouble point ; vous croyez en Dieu, croyez aussi en moi.
\VS{2}Il y a plusieurs demeures dans la maison de mon Père. Si cela n’était pas, je vous l’aurais dit ; je vais vous préparer une place.
\VS{3}Et quand je m'en serai allé, et que je vous aurai préparé une place, je reviendrai, et je vous prendrai avec moi ; afin que là où je suis, vous y soyez aussi.
\VS{4}Et vous savez où je vais, et vous en savez le chemin.
\VS{5}Thomas lui dit : Seigneur ! Nous ne savons point où tu vas, comment donc pouvons-nous en savoir le chemin ?
\VS{6}Jésus lui dit : Je suis le chemin, la vérité, et la vie ; nul ne vient au Père que par moi.
\TextTitle{[Le Père et le Fils sont un]}
\VS{7}Si vous me connaissiez, vous connaîtriez aussi mon Père ; mais dès maintenant vous le connaissez, et vous l'avez vu.
\VS{8}Philippe lui dit : Seigneur ! Montre-nous le Père, et cela nous suffit.
\VS{9}Jésus lui répondit : Je suis depuis si longtemps avec vous, et tu ne m'as pas connu ? Philippe ! Celui qui m'a vu a vu mon Père. Et comment dis-tu : Montre-nous le Père ?
\VS{10}Ne crois-tu pas que je suis dans le Père, et que le Père est en moi ? Les paroles que je vous dis, je ne les dis pas de moi-même ; mais le Père qui demeure en moi est celui qui fait les œuvres.
\VS{11}Croyez-moi, je suis dans le Père, et le Père est en moi, sinon croyez-moi à cause de ces œuvres.
\VS{12}En vérité, en vérité je vous le dis : Celui qui croit en moi fera les œuvres que je fais, et il en fera même de plus grandes que celles-ci, parce que je m'en vais vers mon Père.
\TextTitle{[Nouveau privilège par la prière]}
\VS{13}Et tout ce que vous demanderez en mon Nom, je le ferai ; afin que le Père soit glorifié dans le Fils.
\VS{14}Si vous demandez en mon Nom quelque chose, je le ferai.
\TextTitle{[Promesse quant à l'habitation de l'Esprit dans le coeur du croyant]}
\VS{15}Si vous m'aimez, gardez mes commandements.
\VS{16}Et moi, je prierai le Père, et il vous donnera un autre Consolateur, pour demeurer avec vous éternellement,
\VS{17}l'Esprit de vérité que le monde ne peut recevoir, parce qu'il ne le voit point, et qu'il ne le connaît point ; mais vous le connaissez, car il demeure avec vous, et il sera en vous.
\VS{18}Je ne vous laisserai pas orphelins, je viendrai vers vous.
\VS{19}Encore un peu de temps, et le monde ne me verra plus ; mais vous me verrez, parce que je vis, et vous aussi vous vivrez.
\VS{20}En ce jour-là vous connaîtrez que je suis en mon Père, que vous êtes en moi, et moi en vous.
\VS{21}Celui qui a mes commandements et qui les garde, c'est celui qui m'aime ; et celui qui m'aime sera aimé de mon Père ; je l'aimerai, et je me ferai connaître à lui.
\VS{22}Jude, non pas Iscariot, lui dit : Seigneur ! D’où vient que tu te feras connaître à nous, et non au monde ?
\VS{23}Jésus répondit et lui dit : Si quelqu'un m'aime, il gardera ma parole, et mon Père l'aimera, et nous viendrons à lui, et nous ferons notre demeure chez lui.
\VS{24}Celui qui ne m'aime point ne garde point mes paroles. Et la parole que vous entendez n'est point ma parole, mais c'est celle du Père qui m'a envoyé.
\VS{25}Je vous ai dit ces choses pendant que je demeure avec vous.
\VS{26}Mais le Consolateur, le Saint-Esprit, que le Père enverra en mon Nom, vous enseignera toutes choses, et il vous rappellera tout ce que je vous ai dit.
\TextTitle{[Christ donne sa paix]}
\VS{27}Je vous laisse la paix, je vous donne ma paix ; je ne vous la donne pas comme le monde la donne ; que votre cœur ne se trouble point, et ne s’alarme point.
\VS{28}Vous avez entendu que je vous ai dit : Je m'en vais, et je reviens à vous ; si vous m'aimiez, vous seriez certes joyeux de ce que j'ai dit : Je m'en vais au Père, car le Père est plus grand que moi.
\VS{29}Et maintenant je vous l'ai dit avant que cela soit arrivé, afin que quand il sera arrivé, vous croyiez.
\VS{30}Je ne parlerai plus guère avec vous ; car le prince de ce monde vient ; mais il n'a rien en moi.
\VS{31}Mais afin que le monde sache que j'aime le Père, et que je fais ce que le Père m'a commandé : Levez-vous, partons d'ici.
\TextTitle{[Le cep et les sarments]}
\Chap{15}
\VerseOne{}Je suis le vrai Cep\FTNT{Jésus est l’arbre de vie qui produit de bons fruits en nous, à condition que nous nous tenions loin de l’arbre de la connaissance du bien et du mal. Jésus, le vrai Cep, est la source de vie. La viabilité du sarment dépend de son attachement au Cep. Jésus a été pendu au bois (Ac 5:30), s’est chargé de nos malédictions (Ga. 3:13) et a été retranché à notre place.}, et mon Père est le Vigneron.
\VS{2}Il retranche tout sarment qui est en moi et qui ne porte pas de fruits ; et tout sarment qui porte du fruit, il l'émonde afin qu'il porte encore plus de fruits.
\VS{3}Vous êtes déjà purs à cause de la parole que je vous ai enseignée.
\VS{4}Demeurez en moi, et je demeurerai en vous ; comme le sarment ne peut de lui-même porter du fruit s'il ne demeure pas attaché au cep ; ainsi vous ne le pouvez pas non plus si vous ne demeurez pas en moi.
\VS{5}Je suis le Cep, et vous en êtes les sarments ; celui qui demeure en moi, et en qui je demeure porte beaucoup de fruits ; car hors de moi, vous ne pouvez rien produire.
\VS{6}Si quelqu'un ne demeure point en moi, il est jeté dehors comme le sarment, et il se sèche ; puis on l'amasse, on le met au feu, et il brûle.
\VS{7}Si vous demeurez en moi, et que mes paroles demeurent en vous, demandez tout ce que vous voudrez, et cela vous sera fait.
\VS{8}Si vous portez beaucoup de fruits, mon Père sera glorifié et vous serez alors mes disciples.
\VS{9}Comme le Père m'a aimé, ainsi je vous ai aimés, demeurez dans mon amour.
\VS{10}Si vous gardez mes commandements, vous demeurerez dans mon amour ; comme j'ai gardé les commandements de mon Père, et je demeure dans son amour.
\VS{11}Je vous ai dit ces choses afin que ma joie demeure en vous, et que votre joie soit parfaite.
\VS{12}C'est ici mon commandement : Aimez-vous les uns les autres comme je vous ai aimés.
\VS{13}Il n’y a pas de plus grand amour que de donner sa vie pour ses amis.
\VS{14}Vous serez mes amis, si vous faites tout ce que je vous commande.
\TextTitle{[Nouvelle intimité entre le Seigneur et les siens]}
\VS{15}Je ne vous appelle plus serviteurs, car le serviteur ne sait pas ce que fait son maître; mais je vous ai appelés mes amis, parce que je vous ai fait connaître tout ce que j'ai appris de mon Père.
\VS{16}Ce n'est pas vous qui m'avez choisi ; mais moi, je vous ai choisis, et je vous ai établis afin que vous alliez partout et que vous produisiez du fruit, et que votre fruit demeure ; afin que tout ce que vous demanderez au Père en mon Nom, il vous le donne.
\VS{17}Ce que je vous commande, c’est de vous aimer les uns les autres.
\TextTitle{[L'attitude du monde à l'égard des croyants en Christ]}
\VS{18}Si le monde vous hait, sachez qu’il m’a haï avant vous.
\VS{19}Si vous étiez du monde, le monde aimerait ce qui est à lui ; mais parce que vous n'êtes pas du monde, et que je vous ai choisis du milieu du monde, à cause de cela le monde vous hait.
\VS{20}Souvenez-vous de la parole que je vous ai dite : Le serviteur n'est pas plus grand que son maître ; s'ils m'ont persécuté, ils vous persécuteront aussi ; s'ils ont gardé ma parole, ils garderont aussi la vôtre.
\VS{21}Mais ils vous feront toutes ces choses à cause de mon Nom, parce qu'ils ne connaissent point celui qui m'a envoyé.
\VS{22}Si je n’étais pas venu, et que je ne leur avais point parlé, ils n'auraient point de péché, mais maintenant ils n'ont point d'excuse de leur péché.
\VS{23}Celui qui me hait, hait aussi mon Père.
\VS{24}Si je n’avais pas fait parmi eux les œuvres qu'aucun autre n'a faites, ils n'auraient point de péché ; mais maintenant ils les ont vues, et ils ont haï et moi et mon Père.
\VS{25}Mais cela est arrivé afin que s’accomplisse la parole qui est écrite dans leur loi : Ils m'ont haï sans cause\FTNT{Ps. 35:19 ; Ps. 69:5.}.
\VS{26}Mais quand le Consolateur sera venu, que je vous enverrai de la part de mon Père, l'Esprit de vérité qui procède de mon Père, il rendra témoignage de moi.
\VS{27}Et vous aussi, vous rendrez témoignage, car vous êtes dès le commencement avec moi.
\TextTitle{[Jésus avertit les siens de la persécution]
\\(Mt. 24:9-10 ; Lu. 21:16-19)}
\Chap{16}
\VerseOne{}Je vous ai dit ces choses, afin que vous ne soyez pas scandalisés.
\VS{2}Ils vous chasseront des synagogues ; et même l’heure vient où quiconque vous fera mourir, croira rendre un culte à Dieu.
\VS{3}Et ils vous feront ces choses, parce qu'ils n’ont connu ni le Père ni moi.
\VS{4}Je vous ai dit ces choses, afin que, lorsque l'heure sera venue, vous vous souveniez que je vous les ai dites ; je ne vous en ai pas parlé dès le commencement, parce que j'étais avec vous.
\VS{5}Mais maintenant je m'en vais vers celui qui m'a envoyé, et aucun de vous ne me demande : Où vas-tu ?
\VS{6}Mais parce que je vous ai dit ces choses, la tristesse a rempli votre cœur.
\TextTitle{[La triple activité de l'Esprit agit en faveur du monde]}
\VS{7}Toutefois je vous dis la vérité, il vous est avantageux que je m'en aille, car si je ne m'en vais pas, le Consolateur ne viendra pas vers vous ; mais si je m'en vais, je vous l'enverrai.
\VS{8}Et quand il sera venu, il convaincra le monde de péché, de justice, et de jugement :
\VS{9}du péché, parce qu'ils ne croient point en moi,
\VS{10}de justice, parce que je m'en vais à mon Père, et que vous ne me verrez plus ;
\VS{11}de jugement, parce que le prince de ce monde est déjà jugé.
\TextTitle{[Après son ascension, Christ continuera de révéler la verité par l'Esprit]}
\VS{12}J'ai encore beaucoup de choses à vous dire, mais vous ne pouvez pas les porter maintenant.
\VS{13}Mais quand le Consolateur sera venu, l'Esprit de vérité, il vous conduira dans toute la vérité ; car il ne parlera pas de lui-même, mais il dira tout ce qu'il aura entendu, et il vous annoncera les choses à venir.
\VS{14}Il me glorifiera, car il prendra ce qui est à moi, et vous l'annoncera.
\VS{15}Tout ce que mon Père a, est à moi ; c'est pourquoi j'ai dit qu'il prendra ce qui est à moi et qu’il vous l'annoncera.
\TextTitle{[Jésus parle de sa mort, de sa grandeur]}
\VS{16}Encore un peu de temps, et vous ne me verrez plus ; et après un peu de temps, vous me verrez, car je m'en vais à mon Père.
\VS{17}Quelques-uns de ses disciples dirent entre eux : Qu'est-ce qu'il nous dit : Encore un peu de temps, et vous ne me verrez plus ; et un peu de temps après, vous me verrez, car je m'en vais à mon Père ?
\VS{18}Ils disaient donc : Que signifient ces mots : Encore un peu de temps ? Nous ne comprenons pas ce qu'il dit.
\VS{19}Jésus sachant qu'ils voulaient l’interroger, leur dit : Vous vous demandez entre vous sur ce que j'ai dit : Encore un peu de temps, et vous ne me verrez plus, et un peu de temps après, vous me verrez.
\VS{20}En vérité, en vérité je vous le dis : Vous pleurerez et vous vous lamenterez, et le monde se réjouira ; vous serez, dis-je, attristés ; mais votre tristesse sera changée en joie.
\VS{21}La femme, lorsqu’elle enfante, éprouve de la tristesse, parce que son heure est venue ; mais, lorsqu’elle a donné le jour à l’enfant, elle ne se souvient plus de la souffrance, à cause de ce qu’un homme est né dans le monde.
\VS{22}Vous donc aussi, vous êtes maintenant dans la tristesse ; mais je vous reverrai encore, et votre cœur se réjouira, et personne ne vous ôtera votre joie.
\VS{23}En ce jour-là, vous ne m'interrogerez plus sur rien. En vérité, en vérité je vous le dis : Tout ce que vous demanderez au Père en mon Nom, il vous le donnera.
\VS{24}Jusqu'à présent vous n'avez rien demandé en mon Nom ; demandez, et vous recevrez, afin que votre joie soit parfaite.
\VS{25}Je vous ai dit ces choses en paraboles. Mais l'heure vient où je ne vous parlerai plus en paraboles ; mais je vous parlerai ouvertement de mon Père.
\VS{26}En ce jour-là, vous demanderez des grâces en mon Nom, et je ne vous dis pas que je prierai le Père pour vous ;
\VS{27}car le Père lui-même vous aime, parce que vous m'avez aimé, et que vous avez cru que je suis sorti de Dieu.
\VS{28}Je suis sorti du Père, et je suis venu dans le monde ; maintenant je quitte le monde, et je m'en vais au Père.
\VS{29}Ses disciples lui dirent : Voici, maintenant tu parles ouvertement, et tu n'uses plus de paraboles.
\VS{30}Maintenant nous savons que tu sais toutes choses\FTNT{Jésus est omniscient. Il s’est lui-même présenté à l’apôtre Jean comme celui qui est, qui était et qui sera (Ap. 1:7-8).}, et que tu n'as pas besoin que quelqu’un t'interroge ; à cause de cela nous croyons que tu es sorti de Dieu.
\VS{31}Jésus leur répondit : Croyez-vous maintenant ?
\VS{32}Voici, l'heure vient, et elle est déjà venue, où vous serez dispersés chacun de son côté, et vous me laisserez seul ; mais je ne suis pas seul, car le Père est avec moi.
\VS{33}Je vous ai dit ces choses afin que vous ayez la paix en moi. Vous aurez des tribulations dans le monde, mais prenez courage, j'ai vaincu le monde.
\TextTitle{[La prière d'intercession de Christ, le souverain sacrificateur]}
\Chap{17}
\VerseOne{}Après avoir ainsi parlé, Jésus leva ses yeux au ciel, et dit : Père, l'heure est venue, glorifie ton Fils, afin que ton Fils te glorifie ;
\VS{2}selon que tu lui as donné pouvoir sur tous les hommes ; afin qu'il donne la vie éternelle à tous ceux que tu lui as donnés.
\VS{3}Or, la vie éternelle, ce qu’ils te connaissent, toi, le seul vrai Dieu, et celui que tu as envoyé, Jésus-Christ.
\VS{4}Je t'ai glorifié sur la terre, j'ai achevé l’œuvre que tu m'avais donnée à faire.
\VS{5}Et maintenant glorifie-moi, toi Père, auprès de toi, de la gloire que j’avais auprès de toi avant que le monde soit.
\VS{6}J'ai fait connaître ton Nom aux hommes que tu m'as donnés du milieu du monde ; ils étaient à toi, et tu me les as donnés ; et ils ont gardé ta parole.
\VS{7}Maintenant ils ont connu que tout ce que tu m'as donné vient de toi.
\VS{8}Car je leur ai donné les paroles que tu m'as données, et ils les ont reçues, et ils ont vraiment connu que je suis sorti de toi, et ils ont cru que tu m'as envoyé.
\VS{9}C’est pour eux que je prie ; je ne prie pas pour le monde, mais pour ceux que tu m'as donnés, parce qu'ils sont à toi.
\VS{10}Et tout ce qui est à moi est à toi, et ce qui est à toi est à moi ; et je suis glorifié en eux.
\VS{11}Et maintenant je ne suis plus dans le monde, et ils sont dans le monde ; et moi je vais à toi. Père saint, garde en ton Nom ceux que tu m'as donnés, afin qu'ils soient un comme nous sommes un.
\VS{12}Quand j'étais avec eux dans le monde, je les gardais en ton Nom ; j'ai gardé ceux que tu m'as donnés, et aucun d'eux ne s’est perdu, sinon le fils de perdition, afin que l'Ecriture soit accomplie.
\VS{13}Et maintenant je vais à toi, et je dis ces choses étant encore dans le monde, afin qu'ils aient ma joie parfaite en eux-mêmes.
\VS{14}Je leur ai donné ta parole, et le monde les a haïs, parce qu'ils ne sont pas du monde, comme moi je ne suis pas du monde.
\VS{15}Je ne te prie pas de les ôter du monde, mais de les préserver du mal.
\VS{16}Ils ne sont pas du monde, comme moi je ne suis pas du monde.
\VS{17}Sanctifie-les par ta vérité ; ta parole est la vérité.
\VS{18}Comme tu m'as envoyé dans le monde, ainsi je les ai envoyés dans le monde.
\VS{19}Et je me sanctifie moi-même pour eux, afin qu'eux aussi soient sanctifiés par la vérité.
\VS{20}Je ne prie pas seulement pour eux, mais aussi pour ceux qui croiront en moi par leur parole.
\VS{21}Afin que tous soient un, ainsi que toi, Père, tu es en moi, et moi en toi ; afin qu'eux aussi soient un en nous ; et que le monde croie que c'est toi qui m'as envoyé.
\VS{22}Je leur ai donné la gloire que tu m'as donnée, afin qu'ils soient un comme nous sommes un.
\VS{23}Je suis en eux, et toi en moi, afin qu'ils soient parfaitement un, et que le monde connaisse que c'est toi qui m'as envoyé, et que tu les aimes, comme tu m'as aimé.
\VS{24}Père, mon désir est que ceux que tu m'as donnés soient avec moi là où je suis, afin qu'ils contemplent la gloire que tu m'as donnée ; parce que tu m'as aimé avant la fondation du monde.
\VS{25}Père juste, le monde ne t'a point connu ; mais moi je t'ai connu, et ceux-ci ont connu que c'est toi qui m'as envoyé.
\VS{26}Et je leur ai fait connaître ton Nom, et je le leur ferai connaître, afin que l'amour dont tu m'as aimé soit en eux, et que je sois en eux.
\TextTitle{[Jésus à Gethsémané]
\\(Mt. 26:36-46 ; Mc. 14:32-42 ; Lu. 22:39-46)}
\Chap{18}
\VerseOne{}Après que Jésus eut dit ces choses, il s'en alla avec ses disciples au-delà du torrent de Cédron, où il y avait un jardin dans lequel il entra avec ses disciples.
\TextTitle{[Jésus trahi et arrêté]
\\Mt. 26:47-56 ; Mc. 14:43-50 ; Lu. 22:47-54)}
\VS{2}Or Judas, qui le trahissait, connaissait aussi ce lieu-là, car Jésus s'y était souvent assemblé avec ses disciples.
\VS{3}Judas donc, ayant pris la cohorte, et des huissiers qu’envoyèrent les principaux sacrificateurs et les pharisiens, s'en vint là avec des lanternes, des flambeaux, et des armes.
\VS{4}Jésus, sachant tout ce qui devait lui arriver, s'avança et leur dit : Qui cherchez-vous ?
\VS{5}Ils lui répondirent : Jésus de Nazareth. Jésus leur dit : Moi, Je suis\FTNT{~Moi, Je suis~ (~ego eimi~), ce qui fait écho au nom sous lequel Dieu s’était révélé à Moïse en Ex. 3:14.}. Et Judas qui le trahissait était aussi avec eux.
\VS{6}Or après que Jésus leur eut dit : Moi Je suis, ils reculèrent, et tombèrent par terre.
\VS{7}Il leur demanda une seconde fois : Qui cherchez-vous ? Et ils répondirent : Jésus de Nazareth.
\VS{8}Jésus répondit : Je vous ai dit que moi, Je suis ; si donc vous me cherchez, laissez aller ceux-ci.
\VS{9}Il dit cela afin que s’accomplisse la parole qu'il avait dite : Je n'ai perdu aucun de ceux que tu m'as donnés.
\TextTitle{[Malchus frappé par Pierre]}
\VS{10}Simon Pierre, qui avait une épée, la tira, frappa le serviteur du souverain sacrificateur, et lui coupa l'oreille droite. Ce serviteur s’appelait Malchus.
\VS{11}Mais Jésus dit à Pierre : Remets ton épée au fourreau : Ne boirai-je pas la coupe que le Père m'a donnée ?
\TextTitle{[Jésus conduit auprès du souverain sacrificateur]
\\(v. 27 ; Mt. 26:57-68 ; Mc. 14:53-65 ; Lu. 22:63-71)}
\VS{12}La cohorte, le tribun, et les huissiers des Juifs se saisirent alors de Jésus et le lièrent.
\VS{13}Et ils l'emmenèrent premièrement chez Anne, car il était le beau-père de Caïphe, qui était le souverain sacrificateur de cette année-là.
\VS{14}Et Caïphe était celui qui avait donné ce conseil aux Juifs, qu'il était avantageux qu'un seul homme meure pour le peuple.
\TextTitle{[Le triple reniement de Pierre]
\\(v. 25-27 ; Mt. 26:69-75 ; Mc. 14:66-72 ; Lu. 22:54-62)}
\VS{15}Simon Pierre, avec un autre disciple, suivait Jésus ; et ce disciple était connu du souverain sacrificateur, et il entra avec Jésus dans la cour du souverain sacrificateur.
\VS{16}Mais Pierre était dehors à la porte, et l'autre disciple, qui était connu du souverain sacrificateur, sortit dehors et parla à la portière, et il fit entrer Pierre.
\VS{17}Et la servante qui était la portière dit à Pierre : N'es-tu pas aussi des disciples de cet homme ? Il dit : Je n'en suis point.
\VS{18}Les serviteurs et les huissiers qui étaient là avaient allumé un feu, parce qu'il faisait froid, et ils se chauffaient ; Pierre aussi était avec eux, et se chauffait.
\VS{19}Et le souverain sacrificateur interrogea Jésus sur ses disciples et sur sa doctrine.
\VS{20}Jésus lui répondit : J’ai ouvertement parlé au monde ; j'ai toujours enseigné dans la synagogue et dans le temple, où les Juifs s'assemblent toujours, et je n'ai rien dit en secret.
\VS{21}Pourquoi m'interroges-tu ? Interroge ceux qui ont entendu ce que je leur ai dit ; voici, ils savent ce que j'ai dit.
\VS{22}Quand il eut dit ces choses, un des huissiers qui se tenait là, donna un coup de sa verge à Jésus en lui disant : Est-ce ainsi que tu réponds au souverain sacrificateur ?
\VS{23}Jésus lui répondit : Si j'ai mal parlé, explique-moi ce que j’ai dit de mal ; et si j'ai bien parlé, pourquoi me frappes-tu ?
\VS{24}Anne l’envoya lié à Caïphe, le souverain sacrificateur.
\VS{25}Simon Pierre était là, et se chauffait. On lui dit : N’es-tu pas aussi de ses disciples ? Il le nia et dit : Je n'en suis point.
\VS{26}Un des serviteurs du souverain sacrificateur, parent de celui à qui Pierre avait coupé l'oreille, dit : Ne t'ai-je pas vu dans le jardin avec lui ?
\VS{27}Mais Pierre le nia de nouveau, et aussitôt le coq chanta.
\TextTitle{[Jésus devant Pilate]
\\(Mt. 27:2,11-14 ; Mc. 15:1-5 ; Lu. 23:1-7,13-15)}
\VS{28}Ils conduisirent Jésus de chez Caïphe au Prétoire\FTNT{Le Prétoire était à l'origine le nom du quartier général de la légion romaine. Il s’agissait plus particulièrement de la tente du général en chef d'une armée.} ; c'était le matin. Mais ils n'entrèrent point eux-mêmes dans le Prétoire, afin de ne pas se souiller, et de pouvoir manger l'agneau de Pâque.
\VS{29}C'est pourquoi Pilate\FTNT{Ponce Pilate était le préfet procurateur de la province romaine de Judée au Ier siècle (de 26 à 36).} sortit vers eux, et leur dit : Quelle accusation portez-vous contre cet homme ?
\VS{30}Ils lui répondirent : Si ce n'était pas un malfaiteur, nous ne te l’aurions pas livré.
\VS{31}Alors Pilate leur dit : Prenez-le vous-mêmes, et jugez-le selon votre loi. Mais les Juifs lui dirent : Il ne nous est pas permis de mettre quelqu’un à mort.
\VS{32}C’était afin que s’accomplisse la parole que Jésus avait dite, lorsqu’il indiquait de quelle mort il devait mourir.
\VS{33}Pilate entra de nouveau dans le Prétoire, et ayant appelé Jésus, il lui dit : Es-tu le Roi des Juifs ?
\VS{34}Jésus lui répondit : Est-ce de toi-même que tu dis cela, ou d’autres te l’ont dit de moi ?
\VS{35}Pilate répondit : Suis-je Juif ? Ta nation et les principaux sacrificateurs t'ont livré à moi ; qu'as-tu fait ?
\VS{36}Jésus répondit : Mon Royaume n'est pas de ce monde ; si mon Royaume était de ce monde, mes serviteurs auraient combattu pour moi afin que je ne sois pas livré aux Juifs ; mais maintenant mon règne n'est point d'ici-bas.
\VS{37}Alors Pilate lui dit : Es-tu donc Roi ? Jésus répondit : Tu le dis, que je suis Roi ; je suis né pour cela, et c'est pour cela que je suis venu dans le monde, pour rendre témoignage à la vérité. Quiconque est de la vérité entend ma voix.
\VS{38}Pilate lui dit : Qu'est-ce que la vérité ? Et quand il eut dit cela, il sortit de nouveau vers les Juifs, et il leur dit : Je ne trouve aucun crime en lui.
\TextTitle{[Barabbas libéré et Jésus condamné]
\\(Mt. 27:15-21 ; Mc. 15:6-11 ; Lu. 23:18-19)}
\VS{39}Or, comme c’est parmi vous une coutume que je vous relâche un prisonnier à la fête de Pâque ; voulez-vous donc que je vous relâche le Roi des Juifs ?
\VS{40}Et tous s'écrièrent, disant : Non pas celui-ci, mais Barrabas ; or Barrabas était un brigand.
\TextTitle{[Jésus couronné d'épines]
\\(Mt. 27:-30 ; Mc. 15:16-18)}
\Chap{19}
\VerseOne{}Alors Pilate prit Jésus, et le fit battre de verges.
\VS{2}Les soldats tressèrent une couronne d'épines qu'ils posèrent sur sa tête, et le vêtirent d'un vêtement de pourpre.
\VS{3}Puis ils lui disaient : Roi des Juifs, nous te saluons ; et ils lui donnaient des coups avec leurs verges.
\TextTitle{[Pilate fait un ultime effort pour relâcher Jésus]
\\(Mt. 27:22-26 ; Mc. 15:12-15 ; Lu. 23:20-25)}
\VS{4}Pilate sortit de nouveau dehors, et leur dit : Voici, je vous l'amène dehors, afin que vous sachiez que je ne trouve aucun crime en lui.
\VS{5}Jésus donc sortit portant la couronne d'épines et le manteau de pourpre ; et Pilate leur dit : Voici l'homme.
\VS{6}Mais quand les principaux sacrificateurs et leurs huissiers le virent, ils s'écrièrent, en disant : Crucifie ! Crucifie ! Pilate leur dit : Prenez-le vous-mêmes et crucifiez-le, car je ne trouve point de crime en lui.
\VS{7}Les Juifs lui répondirent : Nous avons une loi, et selon notre loi il doit mourir, car il s'est fait Fils de Dieu.
\VS{8}Quand Pilate entendit cette parole, sa frayeur augmenta.
\VS{9}Et il rentra dans le Prétoire et dit à Jésus : D'où es-tu ? Mais Jésus ne lui donna point de réponse.
\VS{10}Et Pilate lui dit : Est-ce à moi que tu ne parles pas ? Ne sais-tu pas que j'ai le pouvoir de te crucifier, et que j’ai le pouvoir de te délivrer ?
\VS{11}Jésus lui répondit : Tu n'aurais aucun pouvoir sur moi s'il ne t‘avait été donné d'en haut ; c'est pourquoi celui qui m'a livré à toi, commet un plus grand péché.
\VS{12}Dès ce moment, Pilate cherchait à le délivrer ; mais les Juifs criaient en disant : Si tu le délivres, tu n'es pas ami de César ; car quiconque se fait Roi est contre César.
\VS{13}Pilate, ayant entendu ces paroles, amena Jésus dehors, et il siégea au tribunal, au lieu appelé le Pavé, et en hébreu Gabbatha.
\VS{14}C’était la préparation de la Pâque, et environ la sixième heure ; et Pilate dit aux Juifs : Voici votre Roi.
\VS{15}Mais ils criaient : Ôte, ôte, crucifie-le ! Pilate leur dit : Crucifierai-je votre Roi ? Les principaux sacrificateurs répondirent : Nous n'avons pas d'autre roi que César.
\TextTitle{[Jésus crucifié]
\\(Mt. 27:31-50 ; Mc. 15:19-37 ; Lu. 23:26-46)}
\VS{16}Alors il le leur livra pour être crucifié. Ils prirent donc Jésus et l'emmenèrent.
\VS{17}Jésus, portant sa croix, arriva au lieu appelé le Crâne, et en hébreu Golgotha,
\VS{18}où ils le crucifièrent, et deux autres avec lui, un de chaque côté, et Jésus au milieu.
\VS{19}Pilate fit un écriteau, qu'il mit sur la croix, où étaient écrits ces mots : Jésus de Nazareth, le Roi des juifs.
\VS{20}Beaucoup des Juifs lurent cet écriteau, parce que le lieu où Jésus était crucifié, était près de la ville ; et cet écriteau était en hébreu, en grec et en latin.
\VS{21}C'est pourquoi les principaux sacrificateurs des Juifs dirent à Pilate : N'écris pas le Roi des Juifs, mais que celui-ci a dit : Je suis le Roi des Juifs.
\VS{22}Pilate répondit : Ce que j'ai écrit, je l'ai écrit.
\VS{23}Les soldats, après avoir crucifié Jésus, prirent ses vêtements, et ils en firent quatre parts, une part pour chaque soldat. Ils prirent aussi sa tunique, qui était sans couture, d’un seul tissu depuis le haut jusqu'en bas.
\VS{24}Ils se dirent entre eux : Ne la déchirons pas, mais tirons au sort, pour savoir à qui elle sera. Et cela arriva ainsi, afin que s’accomplisse cette parole de l’Ecriture : Ils ont partagé entre eux mes vêtements, et ils ont tiré au sort ma tunique\FTNT{Ps. 22:19.} ; ainsi firent les soldats.
\VS{25}Près de la croix de Jésus se tenaient sa mère, et la sœur de sa mère, Marie femme de Cléopas, et Marie de Magdala.
\VS{26}Jésus voyant sa mère, et auprès d'elle le disciple qu'il aimait, il dit à sa mère : Femme, voilà ton Fils.
\VS{27}Puis il dit au disciple : Voilà ta mère ; et dès ce moment, ce disciple la prit chez lui.
\VS{28}Après cela, Jésus sachant que toutes choses étaient déjà accomplies, il dit, afin que l'Ecriture soit accomplie : J'ai soif.
\VS{29}Et il y avait là un vase plein de vinaigre. Les soldats en remplirent une éponge et la mirent au bout d'une branche d'hysope, et la lui présentèrent à la bouche.
\VS{30}Quand Jésus eut pris le vinaigre, il dit : Tout est accompli\FTNT{La fin de la période de la première alliance n’a pas eu lieu à la naissance du Seigneur. En effet, Ga. 4:4 nous dit que Jésus est né sous la loi de Moïse et le récit des quatre évangiles atteste que depuis sa naissance jusqu’à sa mort, Jésus a scrupuleusement respecté et accompli toute la loi. En effet, il a lui-même dit : ~Ne croyez pas que je sois venu abolir la loi ou les prophètes ; je ne suis pas venu les abolir, mais les accomplir.~ (Mt. 5:17). Ainsi, durant son ministère terrestre, le Seigneur demandait à ce qu’on applique la loi (Mt. 8:4 ; Mt. 23:23 ; Lu. 17:11-14) tout en préparant ses disciples à la nouvelle alliance. L’évangile de Matthieu nous relate un événement capital qui a eu lieu juste après la mort du Seigneur : ~Alors Jésus, poussa de nouveau un grand cri, et rendit l'esprit. Et voici, le voile du temple se déchira en deux, depuis le haut jusqu'en bas ; et la terre trembla, et les pierres se fendirent.~ (Mt. 27:50-51). Il convient de rappeler que le temple était divisé en trois parties : le Parvis, le lieu Saint et le Saint des saints. Le Parvis était accessible à tout le monde, y compris aux non-Juifs. Le lieu Saint n’était accessible qu’aux lévites. La troisième partie, le Saint des saints, n’était accessible qu’au souverain sacrificateur. Le lieu Saint était séparé du Saint des saints par un voile qui symbolisait le mur d’inimitié (Es. 59:2 ; Ro. 3:23) qui sépare l’homme pécheur de la présence de Dieu, représentée dans le temple par l’arche de l’alliance. Ce voile n’avait rien d’un tissu léger et vaporeux, mais il ressemblait davantage à un épais tapis, opaque et surtout très résistant, et donc très difficile à déchirer. Le souverain sacrificateur rentrait seulement une fois par an dans le Saint des saints pour y offrir le sacrifice d’expiation pour le peuple ainsi que pour lui-même (Lé. 16 ; Hé. 9:7). Toutefois, la nécessité de répéter ce sacrifice chaque année prouvait que les exigences de la justice divine n’étaient pas pleinement satisfaites (Hé. 10:3-4). L’auteur de l’épître aux Hébreux nous apprend que le voile symbolisait également le corps physique de Christ (Hé. 10:19-20). Ainsi, lorsque le Seigneur a succombé à ses meurtrissures, le fameux voile s’est déchiré du haut jusqu’au bas. Or tant que le voile subsistait, l’accès à la présence de Dieu était fermé (Hé. 9:8). La déchirure atteste donc qu’en Christ, nous pouvons désormais nous approcher avec assurance du trône de Dieu, sans autre médiateur que le Seigneur lui-même (1 Ti. 2:5). ~Or là où les péchés sont pardonnés, il n'y a plus d’offrande pour le péché. Ainsi donc, mes frères, nous avons la liberté d'entrer dans le Saint des saints au moyen du sang de Jésus, qui est le chemin nouveau et vivant qu'il nous a frayé au travers du voile, c’est-à-dire de sa chair. Et ayant un Souverain Sacrificateur établi sur la maison de Dieu, approchons-nous de lui avec un cœur sincère et une foi inébranlable, ayant les cœurs purifiés d’une mauvaise conscience, et le corps lavé d’une eau pure. Retenons sans fléchir la profession de notre espérance, car celui qui nous a fait la promesse est fidèle.~ Hé. 10:18-23. Jésus-Christ est notre Pâque (1 Co. 5:5-8), il est le sacrifice parfait qui a expié nos péchés une fois pour toutes (Hé. 10:10). Par conséquent, il est celui à qui nous devons nous adresser pour recevoir pardon, miséricorde et compassion. ~Tout est accompli~. En s’écriant de la sorte, Jésus-Christ a proclamé la fin de l'ancienne alliance. En effet, la loi a été promulguée par Moïse, mais la grâce et la vérité sont venues par Jésus-Christ (Jn. 1:17). Toutefois, la nouvelle alliance n’a réellement débuté qu’à la Pentecôte avec l’effusion du Saint-Esprit. Voir commentaire en Actes 2.} ; et ayant baissé la tête, il rendit l'esprit.
\TextTitle{[Fin de la première Alliance]
\\(Mt.27:50-51 ; Mc. 15:37-38 ; Lu. 23:45-46)}
\VS{31}De peur que les corps ne restent sur la croix pendant le sabbat, car c’était la préparation, et ce jour de sabbat était un grand jour, les Juifs demandèrent à Pilate qu’on rompe les jambes aux crucifiés, et qu’on les enlève.
\VS{32}Les soldats vinrent donc, et ils rompirent les jambes au premier, et de même à l'autre qui était crucifié avec lui.
\VS{33}Puis étant venus à Jésus, et voyant qu'il était déjà mort, ils ne lui rompirent point les jambes ;
\VS{34}mais un des soldats lui perça le côté avec une lance, et aussitôt il sortit du sang et de l'eau.
\VS{35}Celui qui l'a vu en a rendu témoignage, et son témoignage est digne de foi ; et il sait qu'il dit vrai, afin que vous le croyiez.
\VS{36}Ces choses sont arrivées, afin que l’Ecriture soit accomplie : Aucun de ses os ne sera brisé\FTNT{Ps. 34:21 ; Ex. 12:46 ; No 9:12.}.
\VS{37}Et encore une autre Ecriture, qui dit : Ils verront celui qu'ils ont percé\FTNT{Za. 12:10.}.
\TextTitle{[Jésus enseveli]
\\(Mt. 27:57-66 ; Mc. 15:42-47 ; Lu. 23:50-56)}
\VS{38}Après ces choses, Joseph d'Arimathée, qui était disciple de Jésus, mais en secret parce qu'il craignait les Juifs, demanda à Pilate la permission d’enlever le corps de Jésus ; et Pilate le lui ayant permis, il vint et prit le corps de Jésus.
\VS{39}Nicodème, qui auparavant était allé de nuit vers Jésus, vint aussi, apportant un mélange de myrrhe et d'aloès d'environ cent livres.
\VS{40}Et ils prirent le corps de Jésus, et l'enveloppèrent de linges avec des aromates, comme les Juifs ont coutume d'ensevelir.
\VS{41}Or il y avait un jardin dans le lieu où Jésus fut crucifié, et dans le jardin un sépulcre neuf, où personne n'avait encore été mis.
\VS{42}Ce fut là qu’ils déposèrent Jésus, à cause de la préparation des Juifs, parce que le sépulcre était proche.
\TextTitle{[Déroulement des événements du jour de la résurection]
\\(Mt. 28:1-15 ; Mc. 16:1-14 ; Lu. 24:1-32)}
\Chap{20}
\VerseOne{}Le premier jour de la semaine, Marie de Magdala se rendit dès le matin au sépulcre, comme il faisait encore obscur ; et elle vit que la pierre était ôtée du sépulcre.
\VS{2}Elle courut vers Simon Pierre et vers l'autre disciple que Jésus aimait, et elle leur dit : Ils ont enlevé le Seigneur du sépulcre, et nous ne savons pas où ils l’ont mis.
\VS{3}Alors Pierre partit avec l'autre disciple, et ils s'en allèrent au sépulcre.
\VS{4}Ils couraient tous deux ensemble, mais l'autre disciple courait plus vite que Pierre, et il arriva le premier au sépulcre.
\VS{5}Et s'étant baissé, il vit les linges à terre ; mais il n'y entra point.
\VS{6}Alors Simon Pierre qui le suivait, arriva, et entra dans le sépulcre, et vit les linges à terre,
\VS{7}et le linge qu’on avait mis sur la tête de Jésus, non pas avec les bandes, mais plié dans un lieu à part.
\VS{8}Alors l'autre disciple, qui était arrivé le premier au sépulcre, entra aussi, il vit, et il crut.
\VS{9}Car ils ne comprenaient pas encore que, selon l'Ecriture, Jésus devait ressusciter des morts.
\VS{10}Et les disciples s'en retournèrent chez eux.
\TextTitle{[Jésus apparait aux disciples, Thomas étant absent]
\\(Mc. 16:14 ; Lu. 24:33-49)}
\VS{11}Mais Marie se tenait près du sépulcre dehors, et pleurait. Comme elle pleurait, elle se baissa dans le sépulcre,
\VS{12}et elle vit deux anges vêtus de blanc, assis à la place où avait été couché le corps de Jésus, l'un à la tête et l'autre aux pieds.
\VS{13}Ils lui dirent : Femme, pourquoi pleures-tu ? Elle leur dit : Parce qu'on a enlevé mon Seigneur, et je ne sais point où on l'a mis.
\VS{14}En disant cela, elle se retourna, et elle vit Jésus qui était là, mais elle ne savait pas que c’était Jésus.
\VS{15}Jésus lui dit : Femme, pourquoi pleures-tu ? Qui cherches-tu ? Elle, pensant que c’était le jardinier, lui dit : Seigneur, si c’est toi qui l'as emporté, dis-moi où tu l'as mis, et je le prendrai.
\VS{16}Jésus lui dit : Marie ! Et elle se retourna et lui dit : Rabbouni ! C’est-à-dire, mon Maître !
\VS{17}Jésus lui dit : Ne me touche pas ; car je ne suis point encore monté vers mon Père. Mais va trouver mes frères, et dis-leur que je monte vers mon Père et votre Père, vers mon Dieu et votre Dieu.
\VS{18}Marie de Magdala alla annoncer aux disciples qu'elle avait vu le Seigneur, et qu'il lui avait dit ces choses.
\VS{19}Le soir de ce jour, qui était le premier de la semaine, les portes du lieu où les disciples étaient assemblés, à cause de la crainte qu'ils avaient des Juifs, étaient fermées. Jésus vint, se présenta au milieu d'eux, et il leur dit : Que la paix soit avec vous !
\VS{20}Et quand il leur eut dit cela, il leur montra ses mains et son côté. Les disciples furent dans la joie en voyant le Seigneur.
\VS{21}Jésus leur dit de nouveau : Que la paix soit avec vous ! Comme mon Père m'a envoyé, ainsi je vous envoie.
\VS{22}Après ces paroles, il souffla sur eux, et leur dit : Recevez le Saint-Esprit.
\VS{23}Ceux à qui vous pardonnerez les péchés, ils leur seront pardonnés ; et ceux à qui vous les retiendrez, ils leur seront retenus.
\TextTitle{[Jésus apparait aux disciples, Thomas étant présent]}
\VS{24}Thomas, appelé Didyme, l'un des douze, n'était pas avec eux quand Jésus vint.
\VS{25}Les autres disciples lui dirent : Nous avons vu le Seigneur. Mais il leur dit : Si je ne vois pas les marques des clous dans ses mains, et si je ne mets pas mon doigt où étaient les clous, et si je ne mets pas ma main dans son côté, je ne le croirai point.
\VS{26}Huit jours après, les disciples étaient de nouveau dans la maison, et Thomas se trouvait avec eux. Jésus vint, les portes étant fermées, se présenta au milieu d'eux, et il leur dit : Que la paix soit avec vous !
\VS{27}Puis il dit à Thomas : Mets ton doigt ici, et regarde mes mains, avance aussi ta main, et mets-la dans mon côté ; et ne sois point incrédule, mais crois.
\VS{28}Et Thomas répondit et lui dit : Mon Seigneur, et mon Dieu !
\VS{29}Jésus lui dit : Parce que tu m'as vu, Thomas, tu as cru. Heureux sont ceux qui n'ont pas vu et qui ont cru.
\TextTitle{[But de l'Evangile selon Jean]}
\VS{30}Jésus fit encore, en présence de ses disciples, beaucoup d’autres miracles qui ne sont pas écrits dans ce livre.
\VS{31}Mais ces choses sont écrites afin que vous croyiez que Jésus est le Christ, le Fils de Dieu, et qu'en croyant vous ayez la vie par son Nom.
\TextTitle{[Jésus apparait à sept apôtres au bord de la mer de Galilée]}
\Chap{21}
\VerseOne{}Après cela, Jésus se montra de nouveau à ses disciples, près de la mer de Tibériade. Et voici de quelle manière il se montra.
\VS{2}Simon Pierre, Thomas, appelé Didyme, Nathanaël, de Cana en Galilée, les fils de Zébédée, et deux autres disciples de Jésus étaient ensemble.
\TextTitle{[Christ et notre service :
\\a. Le service de la volonté propre, sous des directives humaines]}
\VS{3}Simon Pierre leur dit : Je vais pêcher. Ils lui dirent : Nous allons aussi avec toi. Ils partirent et montèrent dans une barque ; mais ils ne prirent rien cette nuit-là.
\VS{4}Le matin étant venu, Jésus se trouva sur le rivage ; mais les disciples ne savaient pas que c’était Jésus.
\TextTitle{[b. Inutilité du service de la volonté propre]}
\VS{5}Jésus leur dit : Mes enfants, avez-vous quelque petit poisson à manger ? Ils lui répondirent : Non.
\TextTitle{[c. Résultat du service sous les directives de Christ]}
\VS{6}Et il leur dit : Jetez le filet du côté droit de la barque, et vous trouverez. Ils le jetèrent donc, et ils ne pouvaient plus le retirer à cause de la grande quantité de poissons.
\VS{7}Alors le disciple que Jésus aimait dit à Pierre : C’est le Seigneur. Et quand Simon Pierre eut entendu que c'était le Seigneur, il mit sa tunique et sa ceinture parce qu'il était nu, et il se jeta dans la mer.
\VS{8}Les autres disciples vinrent dans la barque, car ils n'étaient pas loin de terre, mais seulement à environ deux cents coudées, traînant le filet de poissons.
\VS{9}Lorsqu’ils furent descendus à terre, ils virent de la braise, et du poisson dessus, et du pain.
\VS{10}Jésus leur dit : Apportez des poissons que vous venez maintenant de prendre.
\VS{11}Simon Pierre monta et tira le filet à terre, plein de cent cinquante-trois grands poissons ; et quoiqu'il y en eût tant, le filet ne se rompit point.
\TextTitle{[d) Les ressources de Christ pour ses serviteurs]
\\(Lu. 22:35 ; Ph. 4:19)}
\VS{12}Jésus leur dit : Venez et mangez. Et aucun de ses disciples n'osait lui demander : Qui es-tu ? Sachant que c'était le Seigneur.
\VS{13}Jésus donc vint, et prit du pain, et leur en donna ; il fit de même du poisson aussi.
\VS{14}C’était déjà la troisième fois que Jésus se montrait à ses disciples depuis qu’il était ressuscité des morts.
\TextTitle{[e) La charité, seul motif valable pour le vrai service]
\\(1 Co. 13 ; 2 Co. 5:14 ; Ap. 2:4-5)}
\VS{15}Après qu'ils eurent mangé, Jésus dit à Simon Pierre : Simon fils de Jonas, m'aimes-tu plus que ne m’aiment ceux-ci ? Il lui répondit : Oui, Seigneur ! Tu sais que je t'aime. Il lui dit : Pais mes agneaux.
\VS{16}Il lui dit encore : Simon fils de Jonas, m'aimes-tu ? Il lui répondit : Oui, Seigneur ! Tu sais que je t'aime. Il lui dit : Pais mes brebis.
\VS{17}Il lui dit pour la troisième fois : Simon fils de Jonas, m'aimes-tu ? Pierre fut attristé de ce qu'il lui avait dit pour la troisième fois : M'aimes-tu ? Et il lui répondit : Seigneur, tu sais toutes choses, tu sais que je t'aime. Jésus lui dit : Pais mes brebis.
\TextTitle{[f) Le Maître révèle à Pierre qu'il Lui appartient de fixer le temps et la forme de sa mort]}
\VS{18}En vérité, en vérité je te le dis : Quand tu étais plus jeune, tu te ceignais toi-même, et tu allais où tu voulais ; mais quand tu seras vieux, tu étendras tes mains, et un autre te ceindra, et te mènera où tu ne voudras pas.
\VS{19}Il dit cela pour indiquer par quelle mort Pierre glorifierait Dieu. Et ayant ainsi parlé, il lui dit : Suis-moi.
\TextTitle{[g) Tous ses serviteurs ne mourront pas]
\\(1 Co. 15:51-52 ; 1 Th. 4:14-18)}
\VS{20}Pierre se retournant, vit venir après eux le disciple que Jésus aimait, celui qui pendant le souper s'était penché sur la poitrine de Jésus et avait dit : Seigneur, qui est celui qui te trahit ?
\VS{21}Quand donc Pierre le vit, il dit à Jésus : Seigneur, et celui-ci, que lui arrivera-t-il ?
\VS{22}Jésus lui dit : Si je veux qu'il demeure jusqu'à ce que je vienne, que t'importe ? Toi, suis-moi.
\VS{23}Là-dessus, le bruit courut parmi les frères que ce disciple ne mourrait point. Cependant Jésus ne lui avait pas dit : Il ne mourra point ; mais : Si je veux qu'il demeure jusqu'à ce que je vienne, que t'importe ?
\VS{24}C'est ce disciple qui rend témoignage de ces choses, et qui les a écrites. Et nous savons que son témoignage est digne de foi.
\VS{25}Jésus a fait encore beaucoup d’autres choses. Si on les écrivait en détail, je ne pense pas que le monde même pourrait contenir les livres qu’on écrirait. AMEN !
\PPE{}
\end{multicols}

%\addcontentsline{toc}{chapter}{Testament de Jésus}\clearpage
%\clearpage\ShortTitle{Ac.}\BookTitle{Actes}\BFont
\noindent\hrulefill
{\footnotesize
\textit{
\bigskip
{\centering{}
\\Auteur~: Luc
\\Thème~: Les missions du 1er siècle
\\Date de rédaction~: Env. 60 ap. J.-C.\\}
}
\textit{
\\D'origine grecque, Luc fut l'auteur du livre communément appelé «~actes des apôtres~» et de l'évangile éponyme tous deux adressés à Théophile. Ce livre retrace la genèse de l'Eglise, de l'ascension de Jésus à la Pentecôte, de la prédication vivante et fructueuse de Pierre à la conversion de Paul, jusqu'au voyage de celui-ci à Rome en tant que prisonnier. On y découvre des apôtres déterminés, des ouvriers de Christ qui acceptèrent de subir l'humiliation et la persécution par amour de la vérité. Sont également présentés des hommes et des femmes qui - touchés par la simplicité de l'Evangile du Royaume - se convertirent puis se firent baptiser.
\\Bien plus qu'un recueil relatant de banales manifestations, ce livre est avant tout celui des actes du Saint-Esprit. Il témoigne de la résurrection et de la puissance de Jésus-Christ manifestée au travers de son Corps. Il retrace l'origine et le développement du premier réveil après Jésus-Christ, qui fut un véritable bouleversement au sein d'un empire en proie à l'impiété et à l'idolâtrie.\bigskip
}
}
\par\nobreak\noindent\hrulefill
\begin{multicols}{2}
\Chap{1}
\TextTitle{Introduction~: le Messie ressuscité parle des choses qui concernent le Royaume de Dieu pendant quarante jours}
\VerseOne{}Nous avons rempli le premier traité, ô Théophile~! De toutes les choses que Jésus a faites et enseignées, 
\VS{2}jusqu'au jour où il fut élevé au ciel, après avoir donné par le Saint-Esprit, ses ordres aux apôtres qu'il avait élus.
\VS{3}A qui aussi, après avoir souffert, il se présenta lui-même vivant, avec plusieurs preuves assurées, étant vu par eux pendant quarante jours, et leur parlant des choses qui concernent le Royaume de Dieu.
\VS{4}Et les ayant assemblés, il leur ordonna de ne pas partir de Jérusalem, mais d'attendre la promesse du Père, ce que vous avez entendu de moi~;
\VS{5}car Jean a baptisé d'eau, mais vous serez baptisés du Saint-Esprit dans peu de jours.
\VS{6}Eux donc étant assemblés, l'interrogèrent, disant~: Seigneur, est-ce en ce temps-ci que tu rétabliras le royaume d'Israël~?
\VS{7}Mais il leur dit~: Ce n'est pas à vous de connaître les temps et les moments que le Père a fixés de sa propre autorité.
\TextTitle{La puissance du Saint-Esprit pour évangéliser les nations\FTNTT{\vref{Mt. 28:18-20}~; \vref{Mc. 16:15-18}~; \vref{Lu. 24:47-48}~; \vref{Jn. 20:21-22}.}}
\VS{8}Mais vous recevrez la puissance du Saint-Esprit qui viendra sur vous, et vous serez mes témoins, tant à Jérusalem que dans toute la Judée, et la Samarie, et jusqu'aux extrémités de la terre.
\VS{9}Et ayant dit ces choses, il fut élevé, comme ils le regardaient, une nuée le prit et l'emporta de devant leurs yeux.
\TextTitle{Promesse du retour de Jésus}
\VS{10}Et comme ils avaient les yeux fixés vers le ciel, à mesure qu'il s'en allait, voici, deux hommes en vêtements blancs se présentèrent devant eux,
\VS{11}et leur dirent~: Hommes Galiléens, pourquoi vous arrêtez-vous à regarder au ciel~? Ce Jésus qui a été élevé du milieu de vous au ciel, en descendra de la même manière que vous l'avez contemplé montant au ciel\FTNT{Jésus-Christ est monté au ciel depuis la Montagne des Oliviers et lors de son retour, ses pieds se poseront sur cette montagne. Voir \vref{Za. 14}.}.
\TextTitle{Attente du Saint-Esprit promis}
\VS{12}Alors ils s'en retournèrent à Jérusalem de la montagne appelée la Montagne des Oliviers, qui est près de Jérusalem, le chemin d'un sabbat\FTNT{Chemin de sabbat~: C'est la distance qu'il est permis à un juif de parcourir le jour de sabbat (\vref{Ex. 16:29}). Elle correspond à deux mille coudées ou 1100 m.}.
\VS{13}Et quand ils furent entrés dans la ville, ils montèrent dans une chambre haute où demeuraient Pierre et Jacques, Jean et André, Philippe et Thomas, Barthélemy et Matthieu, Jacques, fils d'Alphée, et Simon le zélote, et Jude, frère de Jacques.
\VS{14}Tous ceux-ci, d'un commun accord, persévéraient dans la prière et dans la supplication avec les femmes, avec Marie, mère de Jésus, et avec ses frères.
\TextTitle{Matthias désigné apôtre pour remplacer Judas}
\VS{15}Et en ces jours-là, Pierre se leva au milieu des disciples, qui étaient là assemblés au nombre d'environ cent vingt personnes, et il leur dit~:
\VS{16}Hommes frères, il fallait que s'accomplisse ce qui a été écrit, ce que le Saint-Esprit a annoncé d'avance par la bouche de David, au sujet de Judas, qui a été le guide de ceux qui ont saisi Jésus.
\VS{17}Car il était compté parmi nous, et il avait reçu en partage ce même service.
\VS{18}Mais après avoir acquis un champ avec le salaire du crime qui lui avait été donné, il est tombé, s'est rompu par le milieu, et toutes ses entrailles ont été répandues.
\VS{19}Et ceci a été connu de tous les habitants de Jérusalem, de sorte que ce champ a été appelé dans leur propre langue Hakeldama, c'est-à-dire le champ du sang.
\VS{20}Car il est écrit dans le livre des Psaumes~: Que sa demeure soit déserte, que personne ne l'habite\FTNT{\vref{Ps. 69:26}.}, et qu'un autre prenne sa charge\FTNT{Charge~: Du grec «~episkope~», il s'agit de la fonction d'un ancien. \vref{Ps. 109:8}.}.
\VS{21}Il faut donc que d'entre ces hommes qui se sont assemblés avec nous pendant tout le temps que le Seigneur Jésus a vécu entre nous,
\VS{22}en commençant depuis le baptême de Jean jusqu'au jour où il a été enlevé du milieu de nous, qu'il y en ait un qui soit témoin avec nous de sa résurrection.
\VS{23}Et ils en présentèrent deux~: Joseph, appelé Barsabbas, surnommé Justus, et Matthias.
\VS{24}Et en priant, ils dirent~: Toi, Seigneur, qui connais les cœurs de tous, désigne lequel de ces deux tu as choisi,
\VS{25}afin qu'il prenne part à ce service et à cet apostolat que Judas a abandonné pour aller en son lieu.
\VS{26}Puis ils les tirèrent au sort, et le sort tomba sur Matthias, qui, d'une commune voix, fut mis au rang des onze apôtres.
\Chap{2}
\TextTitle{Effusion de l'Esprit à la Pentecôte~; naissance de l'Eglise\FTNT{\vref{Joë. 2:32}.}}
\VerseOne{}Et comme le jour de la Pentecôte s'accomplissait, ils étaient tous ensemble dans un même lieu.
\VS{2}Et il se fit tout à coup un bruit du ciel, comme est le bruit d'un vent qui souffle avec véhémence, et il remplit toute la maison où ils étaient assis.
\VS{3}Et il leur apparut des langues divisées, comme de feu, qui se posèrent sur chacun d'eux.
\VS{4}Et ils furent tous remplis du Saint-Esprit, et commencèrent à parler des langues étrangères selon que l'Esprit leur donnait de parler.
\VS{5}Or il y avait à Jérusalem des Juifs qui y séjournaient, hommes pieux, de toute nation qui est sous le ciel.
\VS{6}Et ce bruit s'étant répandu, une multitude vint ensemble, et fut confondue de ce que chacun les entendait parler dans sa propre langue. 
\VS{7}Ils en étaient donc tout surpris, et s'en étonnaient, disant l'un à l'autre~: Voici, tous ceux-ci qui parlent ne sont-ils pas Galiléens~?
\VS{8}Comment donc chacun de nous les entendons-nous parler la propre langue du pays où nous sommes nés~? 
\VS{9}Parthes, Mèdes, Elamites, et ceux qui habitent la Mésopotamie, la Judée, la Cappadoce, le Pont, l'Asie,
\VS{10}la Phrygie, la Pamphylie, l'Egypte, le territoire de la Libye qui est près de Cyrène, et ceux qui sont venus de Rome~? Juifs et Prosélytes,
\VS{11} Crétois et Arabes, comment les entendons-nous parler chacun dans notre langue des merveilles de Dieu~?
\VS{12}Ils étaient donc tout étonnés, et ils ne savaient que penser, disant l'un à l'autre~: Que veut dire ceci~?
\VS{13}Mais les autres se moquaient, et disaient~: C'est qu'ils sont pleins de vin doux.
\TextTitle{Prédication de Pierre}
\VS{14}Alors Pierre, se présentant avec les onze, éleva sa voix, et leur dit~: Hommes Juifs, et vous tous qui habitez à Jérusalem, apprenez ceci, et faites attention à mes paroles~!
\VS{15}Ces gens ne sont pas ivres, comme vous le pensez, car c'est la troisième heure\FTNT{Neuf heures du matin.} du jour.
\VS{16}Mais c'est ici ce qui a été dit par le prophète Joël~:
\VS{17}Et il arrivera dans les derniers jours, dit Dieu, que je répandrai de mon Esprit sur toute chair~; vos fils et vos filles prophétiseront, vos jeunes gens auront des visions, et vos vieillards songeront des songes.
\VS{18}Et dans ces jours-là je répandrai de mon Esprit sur mes serviteurs et sur mes servantes, et ils prophétiseront.
\VS{19}Et je ferai des choses merveilleuses en haut dans le ciel, et des prodiges en bas sur la terre, du sang, du feu, et une vapeur de fumée.
\VS{20}Le soleil se changera en ténèbres, et la lune en sang, avant que ce grand et notable jour du Seigneur vienne.
\VS{21}Mais il arrivera que quiconque invoquera le Nom du Seigneur sera sauvé\FTNT{\vref{Joë. 2:28-32}.}.
\TextTitle{Proclamation de la résurrection du Messie}
\VS{22}Hommes Israélites, écoutez ces paroles~! Jésus de Nazareth, homme approuvé de Dieu parmi vous par les miracles et les prodiges et les signes que Dieu a faits par lui au milieu de vous, comme vous-mêmes vous le savez,
\VS{23}ayant été livré selon le dessein arrêté et selon la prescience de Dieu, vous l'avez pris et mis à la croix, vous l'avez fait mourir par les mains des impies.
\VS{24}Mais Dieu l'a ressuscité, ayant brisé les liens de la mort, parce qu'il n'était pas possible qu'il soit retenu par elle.
\VS{25}Car David dit de lui~: Je contemplais constamment le Seigneur devant moi, parce qu'il est à ma droite, afin que je ne sois point ébranlé\FTNT{\vref{Ps. 16:8-11}.}.
\VS{26}C'est pourquoi mon cœur est dans la joie, et ma langue dans l'allégresse~; et de plus, ma chair reposera avec espérance.
\VS{27}Car tu ne laisseras point mon âme en enfer\FTNT{Voir commentaire en \vref{Mt. 16:18}.} et tu ne permettras point que ton Saint voie la corruption.
\VS{28}Tu m'as fait connaître le chemin de la vie, tu me rempliras de joie dans ta présence\FTNT{\vref{Ps. 16:11}.}.
\VS{29}Hommes frères, qu'il me soit permis de vous dire librement, au sujet du patriarche David, qu'il est mort, qu'il a été enseveli, et que son sépulcre existe encore parmi nous jusqu'à ce jour.
\VS{30}Mais comme il était prophète, et qu'il savait que Dieu lui avait promis avec serment, que du fruit de ses reins il ferait naître selon la chair le Christ, pour le faire asseoir sur son trône~;
\VS{31}c'est la résurrection du Christ qu'il a prévue et annoncée, en disant qu'il ne serait pas abandonné en enfer et que sa chair ne verrait pas la corruption.
\VS{32}Dieu a ressuscité ce Jésus~; nous en sommes tous témoins.
\VS{33}Après donc qu'il a été élevé au ciel par la puissance de Dieu, et qu'il a reçu de son Père la promesse du Saint-Esprit, il a répandu ce que maintenant vous voyez et ce que vous entendez.
\VS{34}Car David n'est pas monté au ciel~; mais lui-même dit~: Le Seigneur a dit à mon Seigneur~: Assieds-toi à ma droite,
\VS{35}jusqu'à ce que j'aie mis tes ennemis pour le marchepied de tes pieds\FTNT{\vref{Ps. 110:1}.}. 
\VS{36}Que toute la maison d'Israël sache donc avec certitude que Dieu a fait Seigneur et Christ, ce Jésus, dis-je, que vous avez crucifié.
\TextTitle{Exhortation à la repentance}
\VS{37}Après avoir entendu ces choses, ils eurent le cœur touché de componction\FTNT{Componction~: Tristesse produite par les effets du repentir, le regret d'avoir offensé Dieu.}, et ils dirent à Pierre et aux autres apôtres~: Hommes frères, que ferons-nous~?
\VS{38}Et Pierre leur dit~: Repentez-vous, et que chacun de vous soit baptisé au Nom de Jésus-Christ, pour obtenir le pardon de vos péchés, et vous recevrez le don du Saint-Esprit.
\VS{39}Car à vous et à vos enfants est faite la promesse, et à tous ceux qui sont loin, autant que le Seigneur, notre Dieu en appellera à lui.
\VS{40}Et par plusieurs autres paroles, il les conjurait et les exhortait, en disant~: Sauvez-vous de cette génération perverse.
\TextTitle{Conversion et baptême de trois mille personnes~; les débuts de l'Eglise}
\VS{41}Ceux donc qui reçurent de bon cœur sa parole, furent baptisés~; et en ce jour-là furent ajoutées à l'Eglise environ trois mille âmes.
\VS{42}Et ils persévéraient tous dans la doctrine des apôtres, dans la communion fraternelle, dans la fraction du pain, et dans les prières.
\VS{43}Et tout le monde avait de la crainte, et beaucoup de miracles et de prodiges se faisaient par les apôtres.
\VS{44}Tous ceux qui croyaient étaient ensemble dans le même lieu, et ils avaient tout en commun~;
\VS{45}et ils vendaient leurs possessions et leurs biens, et les distribuaient à tous, selon les besoins de chacun.
\VS{46}Et tous les jours, ils persévéraient tous d'un commun accord dans le temple~; et rompant le pain de maison en maison, ils prenaient leur repas avec joie et simplicité de cœur~;
\VS{47}louant Dieu et se rendant agréables à tout le peuple. Et le Seigneur ajoutait tous les jours à l'Eglise des gens pour être sauvés.
\Chap{3}
\TextTitle{Guérison d'un homme boiteux de naissance}
\VerseOne{}Et comme Pierre et Jean montaient ensemble au temple à l'heure de la prière~; c'était la neuvième heure.
\VS{2}Et il y avait un homme boiteux de naissance, qu'on portait, et qu'on mettait tous les jours à la porte du temple, appelée la Belle, pour demander l'aumône à ceux qui entraient dans le temple. 
\VS{3}Cet homme voyant Pierre et Jean qui allaient entrer au temple, les pria de lui donner l'aumône.
\VS{4}Alors Pierre, de même que Jean, fixa les yeux sur lui, et lui dit~: Regarde-nous.
\VS{5}Et il les regardait attentivement, s'attendant de recevoir quelque chose d'eux.
\VS{6}Mais Pierre lui dit~: Je n'ai ni argent, ni or~; mais ce que j'ai, je te le donne~: Au Nom de Jésus-Christ de Nazareth, lève-toi et marche.
\VS{7}Et l'ayant pris par la main droite, il le fit lever~; et aussitôt les plantes et les chevilles de ses pieds devinrent fermes.
\VS{8}Et faisant un saut, il se tint debout, et marcha~; et il entra avec eux au temple, marchant, sautant, et louant Dieu.
\VS{9}Et tout le peuple le vit marchant et louant Dieu.
\VS{10}Et reconnaissant que c'était celui-là même qui était assis à la Belle, porte du temple, pour avoir l'aumône, ils furent remplis d'admiration et d'étonnement de ce qui lui était arrivé.
\VS{11}Et comme le boiteux, qui avait été guéri, tenait par la main Pierre et Jean, tout le peuple étonné accourut vers eux, au portique qu'on appelle de Salomon.
\TextTitle{Christ, le Messie annoncé par les prophètes}
\VS{12}Mais Pierre voyant cela, dit au peuple~: Hommes Israélites, pourquoi vous étonnez-vous de ceci~? Ou pourquoi avez-vous les regards fixés sur nous, comme si par notre puissance ou par notre piété, nous avions fait marcher cet homme~?
\VS{13}Le Dieu d'Abraham, d'Isaac, et de Jacob, le Dieu de nos pères, a glorifié son Fils Jésus, que vous avez livré et renié devant Pilate, quoiqu'il jugeât qu'il devait être relâché.
\VS{14}Mais vous avez renié le Saint et le Juste, et vous avez demandé qu'on vous relâche un meurtrier.
\VS{15}Vous avez fait mourir le Prince de la vie, que Dieu a ressuscité des morts~; nous en sommes témoins.
\VS{16}C'est par la foi en son Nom, que son Nom a raffermi les pieds de cet homme que vous voyez et connaissez. La foi, dis-je, que nous avons en lui, a donné à cet homme cette entière guérison de tous ses membres, en présence de vous tous.
\VS{17}Et maintenant, mes frères, je sais que vous avez agi par ignorance, de même que vos chefs.
\VS{18}Mais Dieu a ainsi accompli les choses qu'il avait prédites par la bouche de tous ses prophètes, que le Christ devait souffrir\FTNT{\vref{Es. 53}.}.
\VS{19}Repentez-vous donc, et convertissez-vous, afin que vos péchés soient effacés~;
\VS{20}afin que des temps de rafraîchissement viennent par la présence du Seigneur, et qu'il envoie celui qui vous a été auparavant annoncé, Jésus-Christ,
\VS{21}lequel il faut que le ciel reçoive, jusqu'au temps du rétablissement de toutes les choses que Dieu a prononcées par la bouche de tous ses saints prophètes, dès le commencement du monde.
\VS{22}Car Moïse lui-même a dit à nos pères~: Le Seigneur votre Dieu, vous suscitera d'entre vos frères un Prophète comme moi~; vous l'écouterez dans tout ce qu'il vous dira,
\VS{23}et il arrivera que toute personne qui n'aura pas écouté ce Prophète, sera exterminé du milieu du peuple\FTNT{\vref{De. 18:15-19}.}.
\VS{24}Et même tous les Prophètes depuis Samuel, et ceux qui l'ont suivi, tout autant qu'il y en a eu qui ont parlé, ont aussi prédit ces jours.
\VS{25}Vous êtes les enfants des prophètes et de l'Alliance que Dieu a traitée avec nos pères, en disant à Abraham~: Toutes les familles de la terre seront bénies en ta postérité\FTNT{\vref{Ge. 12:2}.}.
\VS{26}C'est à vous premièrement que Dieu, ayant suscité son Fils Jésus, l'a envoyé pour vous bénir, en détournant chacun de vous de vos iniquités.
\Chap{4}
\TextTitle{Première persécution de l'Eglise~: Pierre et Jean jetés en prison}
\VerseOne{}Mais comme ils parlaient au peuple, survinrent les prêtres, le commandant du temple et les sadducéens,
\VS{2}étant offensés de ce qu'ils enseignaient le peuple, et qu'ils annonçaient la résurrection des morts au Nom de Jésus.
\VS{3}Et les ayant fait arrêter, ils les mirent en prison jusqu'au lendemain, parce qu'il était déjà tard.
\VS{4}Et plusieurs de ceux qui avaient entendu la parole crurent~; et le nombre des personnes fut d'environ cinq mille.
\TextTitle{Pierre et Jean convoqués au sanhédrin}
\VS{5}Or il arriva que le lendemain, les chefs, les anciens et les scribes s'assemblèrent à Jérusalem~;
\VS{6}avec Anne, le grand-prêtre, Caïphe, Jean, Alexandre, et tous ceux qui étaient de la race des principaux prêtres.
\VS{7}Et ayant fait comparaître devant eux Pierre et Jean, ils leur demandèrent~: Par quelle puissance, ou au nom de qui avez-vous fait cette guérison~?
\VS{8}Alors Pierre étant rempli du Saint-Esprit, leur dit~: Chefs du peuple, et vous anciens d'Israël~:
\VS{9}Puisque nous sommes jugés aujourd'hui sur un bienfait accordé à un homme impotent, afin que nous disions comment il a été guéri,
\VS{10}sachez, vous tous et tout le peuple d'Israël, que c'est au Nom de Jésus-Christ de Nazareth, que vous avez crucifié, et que Dieu a ressuscité des morts~; c'est en son Nom, que cet homme qui parait ici devant vous, a été guéri.
\VS{11}C'est cette pierre rejetée, par vous qui bâtissez, qui est devenue la pierre principale de l'angle\FTNT{\vref{Ps. 118:22}.}.
\VS{12}Il n'y a de salut en aucun autre~: Car il n'y a sous le ciel aucun autre Nom qui ait été donné aux hommes par lequel nous devions être sauvés.
\TextTitle{Le sanhédrin interdit aux apôtres de prêcher au Nom de Jésus}
\VS{13}Eux, voyant la hardiesse de Pierre et de Jean, et sachant aussi qu'ils étaient des hommes sans instruction et du commun peuple~; s'en étonnaient, et ils reconnaissaient bien qu'ils avaient été avec Jésus.
\VS{14}Et voyant que l'homme qui avait été guéri, était présent avec eux, ils ne pouvaient contredire en rien.
\VS{15}Alors ils leur ordonnèrent de sortir hors du sanhédrin, et ils délibérèrent entre eux, disant~: Que ferons-nous à ces gens~?
\VS{16}Car il est manifeste pour tous les habitants de Jérusalem, qu'un miracle a été fait par eux, et cela est si évident que nous ne pouvons le nier.
\VS{17}Mais afin qu'il ne soit plus divulgué parmi le peuple, défendons-leur avec menaces expresses, qu'ils n'aient plus à parler à qui que ce soit en ce Nom.
\VS{18}Et les ayant donc appelés, ils leur ordonnèrent de ne plus parler ni d'enseigner en aucune manière au Nom de Jésus. 
\VS{19}Mais Pierre et Jean leur répondirent~: Jugez s'il est juste devant Dieu de vous obéir plutôt qu'à Dieu.
\VS{20}Car nous ne pouvons pas ne pas parler de ce que nous avons vu et entendu.
\VS{21}Alors ils les relâchèrent avec menaces, ne trouvant point comment ils pourraient les punir, à cause du peuple, parce que tous glorifiaient Dieu de ce qui avait été fait.
\VS{22}Car l'homme en qui cette miraculeuse guérison avait été faite, avait plus de quarante ans.
\TextTitle{L'Eglise demande l'assistance de Dieu}
\VS{23}Après avoir été relâchés, ils allèrent vers les leurs, et leur racontèrent tout ce que les principaux prêtres et les anciens leur avaient dit.
\VS{24}Eux l'ayant entendu, élevèrent tous ensemble la voix à Dieu, et dirent~: Seigneur, tu es le Dieu qui as fait le ciel et la terre, la mer, et toutes les choses qui y sont~;
\VS{25}et qui as dit par la bouche de David ton serviteur~: Pourquoi ce tumulte parmi les nations et ces vaines pensées parmi les peuples~?
\VS{26}Les rois de la terre se sont soulevés en personne, et les princes se sont ligués ensemble contre le Seigneur, et contre son Christ\FTNT{\vref{Ps. 2:1-2}.}.
\VS{27}En effet, contre ton Saint Fils Jésus, que tu as oint, se sont assemblés Hérode et Ponce Pilate, avec les Gentils, et le peuple d'Israël,
\VS{28}pour faire toutes les choses que ta main et ton conseil avaient auparavant déterminé qui seraient faites. 
\VS{29}Maintenant donc, Seigneur, regarde à leurs menaces, et donne à tes serviteurs d'annoncer ta parole avec toute hardiesse~;
\VS{30}en étendant ta main afin qu'il se fasse des guérisons, des prodiges, et des merveilles, par le Nom de ton Saint Fils Jésus.
\VS{31}Et quand ils eurent prié, le lieu où ils étaient assemblés trembla~; et ils furent tous remplis du Saint-Esprit, et ils annonçaient la parole de Dieu avec hardiesse.
\TextTitle{La multitude unie comme un seul corps\FTNTT{\vref{Ac. 2:42-47}.}}
\VS{32}Or la multitude de ceux qui croyaient n'était qu'un cœur et qu'une âme et nul ne disait d'aucune des choses qu'il possédait, qu'elle fût à lui, mais toutes choses étaient communes entre eux.
\VS{33}Aussi les apôtres rendaient témoignage avec une grande force à la résurrection du Seigneur Jésus~; et une grande grâce était sur eux tous.
\VS{34}Car il n'y avait parmi eux aucun indigent~; parce que tous ceux qui possédaient des champs ou des maisons, les vendaient, et ils apportaient le prix des choses vendues,
\VS{35}et le mettaient aux pieds des apôtres~; et il était distribué à chacun selon qu'il en avait besoin.
\VS{36}Or Joseph, surnommé par les apôtres Barnabas, c'est-à-dire, fils de consolation, Lévite, originaire de Chypre,
\VS{37}ayant une possession, la vendit, et en apporta le prix, et le mit aux pieds des apôtres.
\Chap{5}
\TextTitle{Mensonge d'Ananias et Saphira~: Leur mort}
\VerseOne{}Mais un homme appelé Ananias, et Saphira sa femme, vendit une possession,
\VS{2}et retint une partie du prix, sa femme le sachant~; puis il apporta le reste, et le déposa aux pieds des apôtres.
\VS{3}Mais Pierre lui dit~: Ananias comment Satan s'est-il emparé de ton cœur jusqu'à t'inciter à mentir au Saint-Esprit, et à soustraire une partie du prix de la possession~?
\VS{4}Si tu l'avais gardée, ne te restait-elle pas~? Et après qu'elle ait été vendue, le prix n'était-il pas à ta disposition~? Comment as-tu pu mettre en ton cœur un pareil dessein~? Tu n'as pas menti aux hommes mais à Dieu.
\VS{5}Et Ananias, entendant ces paroles, tomba et rendit l'âme~; ce qui causa une grande crainte à tous ceux qui en entendirent parler.
\VS{6}Et quelques jeunes hommes se levant le prirent, et l'emportèrent dehors, et l'ensevelirent.
\VS{7}Et il arriva environ trois heures après que sa femme entra, sans savoir ce qui était arrivé.
\VS{8}Et Pierre prenant la parole, lui dit~: Dis-moi, avez-vous autant vendu le champ~? Et elle dit~: Oui, autant.
\VS{9}Alors Pierre lui dit~: Pourquoi avez-vous fait un complot entre vous pour tenter l'Esprit du Seigneur~? Voici, à la porte, les pieds de ceux qui ont enterré ton mari, et ils t'emporteront.
\VS{10}Et au même instant, elle tomba à ses pieds et rendit l'esprit. Et quand les jeunes hommes furent entrés, ils la trouvèrent morte, et ils l'emportèrent dehors, et l'ensevelirent auprès de son mari.
\VS{11}Et cela donna une grande crainte à toute l'Eglise, et à tous ceux qui entendaient ces choses.
\TextTitle{Miracles à Jérusalem}
\VS{12}Beaucoup de prodiges et de miracles se faisaient parmi le peuple par les mains des apôtres~; et ils étaient tous d'un commun accord au portique de Salomon.
\VS{13}Cependant aucun des autres n'osait se joindre à eux, mais le peuple les louait hautement.
\VS{14}Et le nombre de ceux qui croyaient au Seigneur, tant d'hommes que de femmes, se multipliait de plus en plus.
\VS{15}Et on apportait les malades dans les rues, et on les mettait sur de petits lits et sur des couchettes, afin que quand Pierre viendrait, au moins son ombre passe sur quelqu'un d'eux.
\VS{16}La multitude accourait aussi des villes voisines à Jérusalem, amenant des malades, et ceux qui étaient tourmentés des esprits impurs~; et tous étaient guéris.
\TextTitle{Deuxième persécution de l'Eglise~: Les apôtres en prison puis devant le sanhédrin}
\VS{17}Alors le grand-prêtre se leva, lui et tous ceux qui étaient avec lui, à savoir la secte des sadducéens, et ils furent remplis de jalousie~;
\VS{18}et mettant la main sur les apôtres, ils les jetèrent dans la prison publique.
\VS{19}Mais l'Ange du Seigneur ouvrit pendant la nuit les portes de la prison, les fit sortir, et leur dit~:
\VS{20}Allez, et présentez-vous dans le temple, annoncez au peuple toutes les paroles de cette vie.
\VS{21}Ayant entendu cela, ils entrèrent dès le matin dans le temple, et se mirent à enseigner. Mais le grand-prêtre et ceux qui étaient avec lui étant arrivés, ils convoquèrent le sanhédrin et tous les anciens des fils d'Israël, et ils envoyèrent chercher les apôtres à la prison.
\VS{22}Mais, les huissiers à leur arrivée, ne les trouvèrent point dans la prison. Ils retournèrent, et firent leur rapport,
\VS{23}en disant~: Nous avons trouvé la prison fermée avec toute sûreté, et les gardes aussi qui étaient devant les portes~; mais après l'avoir ouverte, nous n'avons trouvé personne dedans.
\VS{24}Lorsque le grand-prêtre, le commandant du temple, et les principaux prêtres, eurent entendu ces paroles, ils ne savaient que penser au sujet des apôtres, ne sachant ce qui arriverait de tout cela.
\VS{25}Mais quelqu'un vint leur dire~: Voici, les hommes que vous avez mis en prison sont dans le temple, et ils enseignent le peuple.
\VS{26}Alors le commandant du temple partit avec les huissiers, et il les conduisit sans violence, car ils avaient peur d'être lapidés par le peuple.
\VS{27}Après qu'ils les eurent amenés, ils les présentèrent au sanhédrin. Et le grand-prêtre les interrogea, disant~: 
\VS{28}Ne vous avons-nous pas défendu expressément d'enseigner en ce Nom-là~? Et cependant voici, vous avez rempli Jérusalem de votre doctrine, et vous voulez faire retomber sur nous le sang de cet homme.
\VS{29}Alors Pierre et les autres apôtres répondant, dirent~: Il faut plutôt obéir à Dieu qu'aux hommes.
\VS{30}Le Dieu de nos pères a ressuscité Jésus, que vous avez fait mourir en le pendant au bois.
\VS{31}Dieu l'a élevé par sa puissance pour être Prince et Sauveur, afin de donner à Israël la repentance et la rémission des péchés.
\VS{32}Nous sommes témoins de ce que nous disons, de même que le Saint-Esprit que Dieu a donné à ceux qui lui obéissent, en est aussi témoin.
\TextTitle{Parole de sagesse de Gamaliel}
\VS{33}Mais eux, ayant entendu ces choses, grinçaient les dents, et consultaient pour les faire mourir.
\VS{34}Mais un pharisien nommé Gamaliel, docteur de la loi, honoré de tout le peuple, se leva dans le sanhédrin, et ordonna de faire sortir un instant les apôtres.
\VS{35}Puis il leur dit~: Hommes Israélites, prenez garde à ce que vous allez faire à l'égard de ces gens.
\VS{36}Car il n'y a pas longtemps que Theudas s'éleva, se disant être quelque chose, et auquel se joignit un nombre d'environ quatre cents hommes~; mais il fut tué, et tous ceux qui s'étaient joints à lui ont été dissipés et réduits à rien.
\VS{37}Après lui parut Judas le Galiléen au temps du recensement, et il attira à lui un grand peuple~; il périt aussi, et tous ceux qui s'étaient joints à lui ont été dispersés.
\VS{38}Maintenant donc je vous dis~: Ne continuez plus vos poursuites contre ces hommes, et laissez-les. Car si cette entreprise ou cette œuvre vient des hommes, elle sera détruite~;
\VS{39}mais si elle vient de Dieu, vous ne pourrez pas la détruire. Et prenez garde qu'il ne se trouve que vous combattiez contre Dieu.
\VS{40}Et ils furent de son avis. Et ayant appelé les apôtres, ils les firent battre de verges, ils leur défendirent de parler au Nom de Jésus, et ils les relâchèrent.
\TextTitle{Frappés, les apôtres continuent de prêcher le Nom de Jésus}
\VS{41}Et les apôtres se retirèrent de devant le sanhédrin, joyeux d'avoir été jugés dignes de subir des outrages pour le Nom de Jésus.
\VS{42}Et tous les jours, ils ne cessaient d'enseigner, et d'annoncer l'Evangile de Jésus-Christ dans le temple, et de maison en maison.
\Chap{6}
\TextTitle{Sept hommes choisis pour le service}
\VerseOne{}En ces jours-là, comme les disciples se multipliaient, il s'éleva un murmure des Hellénistes\FTNT{Les Hellénistes étaient des juifs issus de la diaspora ayant adopté la culture et la langue grecque.} contre les Hébreux, parce que leurs veuves étaient méprisées dans le service ordinaire.
\VS{2}C'est pourquoi les douze, ayant convoqué la multitude des disciples, leur dirent~: Il n'est pas raisonnable que nous laissions la parole de Dieu pour servir aux tables.
\VS{3}Regardez donc, mes frères, pour choisir sept hommes d'entre vous, de qui on ait bon témoignage, pleins du Saint-Esprit et de sagesse, auxquels nous confierons ce devoir.
\VS{4}Et nous, nous continuerons à vaquer à la prière et au service de la parole.
\VS{5}Et ce discours plut à toute l'assemblée qui était là présente~; et ils élurent Etienne, homme plein de foi et du Saint-Esprit, Philippe, Prochore, Nicanor, Timon, Parménas, et Nicolas, prosélyte d'Antioche.
\VS{6}Ils les présentèrent aux apôtres~; qui, après avoir prié, leur imposèrent les mains.
\VS{7}Et la parole de Dieu croissait, et le nombre des disciples se multipliait beaucoup dans Jérusalem~; un grand nombre aussi de prêtres obéissait à la foi.
\VS{8}Or Etienne, plein de foi et de puissance, faisait de grands miracles et de grands prodiges parmi le peuple.
\TextTitle{Troisième persécution de l'Eglise~; Etienne convoqué au sanhédrin}
\VS{9}Quelques-uns de la synagogue appelée la synagogue des affranchis\FTNT{Affranchis~: Du grec «~libertinos~», c'est-à-dire «~libertins~»~: Hommes libres. Fraction de la communauté Juive qui avait sa propre synagogue à Jérusalem. Probablement des Juifs qui avaient été faits prisonniers par Pompée et d'autres généraux romains, qui avaient été déportés à Rome, puis libérés.}, de celle des Cyrénéens et de celle des Alexandrins, avec ceux de Cilicie et d'Asie, se levèrent pour disputer contre Etienne.
\VS{10}Mais ils ne pouvaient pas résister à la sagesse et à l'Esprit par lequel il parlait.
\VS{11}Alors ils soudoyèrent des hommes qui dirent~: Nous l'avons entendu proférer des paroles blasphématoires contre Moïse et contre Dieu.
\VS{12}Et ils soulevèrent le peuple, les anciens, et les scribes, et se jetant sur lui, ils l'enlevèrent et l'amenèrent au sanhédrin.
\VS{13}Et ils présentèrent de faux témoins qui dirent~: Cet homme ne cesse de proférer des paroles blasphématoires contre ce saint lieu et contre la loi.
\VS{14}Car nous l'avons entendu dire que Jésus, ce Nazaréen, détruira ce lieu-ci, et changera les coutumes que Moïse nous a données.
\VS{15}Tous ceux qui siégeaient au sanhédrin avaient les yeux fixés sur lui, son visage leur parut comme celui d'un ange.
\Chap{7}
\TextTitle{Discours d'Etienne devant le sanhédrin}
\VerseOne{}Alors le grand-prêtre lui dit~: Ces choses sont-elles ainsi~?
\VS{2}Etienne répondit~: Hommes frères et pères, écoutez-moi~! Le Dieu de gloire apparut à notre père Abraham, lorsqu'il était en Mésopotamie, avant qu'il s'établisse à Charran, et lui dit~:
\VS{3}Sors de ton pays et de ta famille, et va dans le pays que je te montrerai.
\VS{4}Il sortit donc du pays des Chaldéens, et alla demeurer à Charran. De là, après la mort de son père, Dieu le fit passer dans ce pays que vous habitez maintenant.
\VS{5}Et il ne lui donna aucun héritage dans ce pays, non pas même d'un pied de terre, quoiqu'il lui ait promis de le lui donner en possession, et à sa postérité après lui, dans un temps où il n'avait point encore d'enfant.
\VS{6}Dieu lui parla ainsi~: Ta postérité séjournera dans une terre étrangère pendant quatre cents ans~; et on la réduira à la servitude et on la maltraitera.
\VS{7}Mais je jugerai la nation à laquelle ils auront été asservis, dit Dieu~; et après cela ils sortiront, et me serviront en ce lieu-ci\FTNT{\vref{Ge. 15:13-14}.}.
\VS{8}Puis il donna à Abraham l'alliance de la circoncision~; et après cela Abraham engendra Isaac qu'il circoncit le huitième jour. Isaac engendra Jacob, et Jacob les douze patriarches.
\VS{9}Les patriarches, jaloux de Joseph, le vendirent pour être emmené en Egypte.
\VS{10}Mais Dieu était avec lui, et le délivra de toutes ses afflictions~; et l'ayant rempli de sagesse il le rendit agréable à Pharaon, roi d'Egypte, qui l'établit gouverneur sur l'Egypte, et sur toute sa maison.
\VS{11}Or il survint dans tout le pays d'Egypte, et dans celui de Canaan, une famine et une grande détresse, en sorte que nos pères ne pouvaient trouver des vivres.
\VS{12}Mais Jacob apprit qu'il y avait du blé en Egypte, il y envoya une première fois nos pères.
\VS{13}Et la seconde fois, Joseph fut reconnu par ses frères, et la famille de Joseph fut déclarée à Pharaon.
\VS{14}Alors Joseph envoya chercher Jacob, son père, et toute sa famille, composée de soixante-quinze personnes.
\VS{15}Jacob descendit en Egypte, et il y mourut, lui et nos pères~;
\VS{16}qui furent transportés à Sichem, et mis dans le sépulcre qu'Abraham avait acheté à prix d'argent des fils d'Hamor, fils de Sichem.
\VS{17}Mais comme le temps de la promesse, pour laquelle Dieu avait juré à Abraham, s'approchait, le peuple s'augmenta et se multiplia en Egypte~;
\VS{18}jusqu'à ce que parut en Egypte un autre roi, qui n'avait pas connu Joseph.
\VS{19}Ce roi, usant d'artifice contre notre race, maltraita nos pères jusqu'à leur faire exposer leurs enfants à l'abandon, afin d'en faire périr la race.
\VS{20}En ce temps-là naquit Moïse, qui fut divinement beau. Et il fut nourri trois mois dans la maison de son père.
\VS{21}Mais ayant été exposé à l'abandon, la fille de Pharaon le recueillit et l'éleva comme son fils.
\VS{22}Moïse fut instruit dans toute la sagesse des Egyptiens~; et il était puissant en paroles et en œuvres.
\VS{23}Mais quand il fut parvenu à l'âge de quarante ans, il forma le dessein d'aller visiter ses frères, les enfants d'Israël.
\VS{24}Et voyant l'un d'eux à qui l'on faisait tort, il le défendit, et vengea celui qui était outragé en tuant l'Egyptien.
\VS{25}Il croyait que ses frères comprendraient par là que Dieu les délivrerait par son moyen~; mais ils ne le comprirent point.
\VS{26}Et le jour suivant, il parut au milieu d'eux comme ils se querellaient, et il tâcha de les mettre d'accord en leur disant~: Hommes, vous êtes frères, pourquoi vous faites-vous tort l'un à l'autre~?
\VS{27}Mais celui qui maltraitait son prochain le repoussa, en disant~: Qui t'a établi prince et juge sur nous~?
\VS{28}Veux-tu me tuer, comme tu as tué hier l'Egyptien~?
\VS{29}Alors Moïse s'enfuit sur un tel discours, et fut étranger dans le pays de Madian, où il eut deux fils.
\VS{30}Et quarante ans étant accomplis, l'Ange du Seigneur lui apparut au désert de la montagne de Sinaï, dans la flamme d'un buisson en feu.
\VS{31}Et quand Moïse le vit, il fut étonné de la vision, et comme il approchait pour considérer ce que c'était, la voix du Seigneur lui fut adressée, disant~:
\VS{32} Je suis le Dieu de tes pères, le Dieu d'Abraham, le Dieu d'Isaac, et le Dieu de Jacob. Et Moïse tout tremblant n'osait pas regarder.
\VS{33}Le Seigneur lui dit~: Ôte tes souliers de tes pieds, car le lieu sur lequel tu te tiens est une terre sainte.
\VS{34}J'ai vu, j'ai vu l'affliction de mon peuple qui est en Egypte, et j'ai entendu leur gémissement, et je suis descendu pour les délivrer. Maintenant donc, va, je t'enverrai en Egypte.
\VS{35}Ce Moïse, qu'ils avaient rejeté en disant~: Qui t'a établi prince et juge~? C'est lui que Dieu envoya comme prince et comme libérateur par le moyen de l'Ange qui lui était apparu dans le buisson.
\VS{36}C'est celui qui les tira dehors, en opérant des miracles et des prodiges au pays d'Egypte, au sein de la Mer Rouge, et au désert pendant quarante ans.
\VS{37}C'est ce Moïse qui a dit aux enfants d'Israël~: Le Seigneur votre Dieu vous suscitera d'entre vos frères un Prophète comme moi~; écoutez-le\FTNT{\vref{De. 18:15}.}~!
\VS{38}C'est lui, qui, lors de l'assemblée au désert, étant avec l'Ange qui lui parlait sur la montagne de Sinaï et avec nos pères, reçut les paroles de vie pour nous les donner.
\VS{39}Nos pères ne voulurent pas lui obéir, mais ils le rejetèrent, et ils tournèrent leur cœur vers l'Egypte,
\VS{40}en disant à Aaron~: Fais-nous des dieux qui marchent devant nous~; car nous ne savons point ce qui est arrivé à ce Moïse qui nous a amenés hors du pays d'Egypte.
\VS{41}Ils firent donc en ces jours-là un veau, et ils offrirent des sacrifices à l'idole, et se réjouirent de l'œuvre de leurs mains.
\VS{42}C'est pourquoi aussi Dieu se détourna d'eux, et les livra au culte de l'armée du ciel, ainsi qu'il est écrit dans le livre des prophètes~: Maison d'Israël, m'avez-vous offert des sacrifices et des victimes pendant quarante ans au désert~?
\VS{43}Mais vous avez porté la tente de Moloc\FTNT{\vref{Lé. 18:21}.}, et l'étoile de votre dieu Remphan~; qui sont des figures que vous avez faites pour les adorer. C'est pourquoi je vous transporterai au-delà de Babylone.
\VS{44}Nos pères avaient au désert le tabernacle du témoignage, comme l'avait ordonné celui qui avait dit à Moïse de le faire selon le modèle qu'il avait vu.
\VS{45}Et nos pères avaient reçu ce tabernacle, ils le portèrent sous la conduite de Josué dans le pays qui était possédé par les nations que Dieu chassa de devant eux, et il y resta jusqu'aux jours de David.
\VS{46}David trouva grâce devant Dieu, et demanda de pouvoir dresser une tente pour le Dieu de Jacob.
\VS{47}Et ce fut Salomon qui lui bâtit une maison.
\VS{48}Mais le Très-Haut n'habite pas dans des temples faits de main d'homme, selon ces paroles du prophète~:
\VS{49}Le ciel est mon trône, et la terre est le marchepied de mes pieds~: Quelle maison me bâtirez-vous, dit le Seigneur, ou quel pourrait être le lieu de mon repos~?
\VS{50}Ma main n'a-t-elle pas fait toutes ces choses\FTNT{\vref{Es. 66:1}.}~?
\VS{51}Hommes au cou raide, et incirconcis de cœur et d'oreilles, vous vous obstinez toujours contre le Saint-Esprit~; vous faites comme vos pères ont fait.
\VS{52}Lequel des prophètes vos pères n'ont-ils pas persécuté~? Ils ont même tué ceux qui annonçaient d'avance l'avènement du Juste, dont vous avez été les traîtres et les meurtriers,
\VS{53}vous qui avez reçu la loi par une ordonnance des anges, et qui ne l'avez point gardée.
\TextTitle{Etienne~: Premier martyr}
\VS{54}En entendant ces choses, leur cœur s'enflamma de colère, et ils grinçaient des dents contre lui.
\VS{55}Mais Etienne, rempli du Saint-Esprit, et fixant les yeux vers le ciel, vit la gloire de Dieu, et Jésus qui était à la droite de Dieu.
\VS{56}Et il dit~: Voici, je vois les cieux ouverts, et le Fils de l'homme étant à la droite de Dieu.
\VS{57}Alors ils s'écrièrent à haute voix, et bouchèrent leurs oreilles, et tous d'un commun accord se jetèrent sur lui.
\VS{58}Et l'ayant tiré hors de la ville, ils le lapidèrent~; et les témoins déposèrent leurs vêtements aux pieds d'un jeune homme nommé Saul.
\VS{59}Et ils lapidaient Etienne qui priait et disait~: Seigneur Jésus, reçois mon esprit\FTNT{Dans \vref{Ec. 12:9}, il est dit qu'à la mort, l'esprit retourne à Dieu qui l'a donné. Jésus est donc Dieu puisqu'il a reçu l'esprit d'Etienne.}~!
\VS{60}Et s'étant mis à genoux, il cria à haute voix~: Seigneur, ne leur impute point ce péché~! Et quand il eut dit cela, il s'endormit.
\Chap{8}
\TextTitle{Quatrième persécution de l'Eglise~: Saul opprime les saints}
\VerseOne{}Or Saul consentait à la mort d'Etienne, et en ce temps-là, il y eut une grande persécution contre l'Eglise de Jérusalem. Et tous, excepté les apôtres, se dispersèrent dans les contrées de la Judée et de la Samarie.
\VS{2}Et quelques hommes pieux emportèrent Etienne pour l'ensevelir, et le pleurèrent à grand bruit.
\VS{3}Mais Saul ravageait l'église, entrant dans toutes les maisons, et traînant par force hommes et femmes, il les mettait en prison.
\TextTitle{Le déploiement des chrétiens\FTNTT{\vref{Ac. 11:19-21}}}
\VS{4}Ceux qui avaient été dispersés allaient de lieu en lieu, annonçant la parole de Dieu.
\TextTitle{Philippe en Samarie~; Simon le magicien}
\VS{5}Philippe, étant descendu dans la ville de Samarie, leur prêcha Christ.
\VS{6}Et les foules tout entières étaient attentives à ce que Philippe disait, l'écoutant, lorsqu'elles virent les miracles qu'il faisait,
\VS{7}car les esprits impurs sortaient, en criant à haute voix, hors de plusieurs qui en étaient possédés, et beaucoup de paralytiques et de boiteux furent guéris.
\VS{8}Ce qui causa une grande joie dans cette ville-là.
\VS{9}Or il y avait auparavant dans la ville un homme nommé Simon qui exerçait l'art d'enchanteur, et ensorcelait le peuple de Samarie, se disant être quelque grand personnage.
\VS{10}Tous, depuis le plus petit jusqu'au plus grand étaient attachés à lui, et disaient~: Celui-ci est la grande puissance de Dieu.
\VS{11}Et ils étaient attachés à lui, parce que depuis longtemps il les avait éblouis par sa magie.
\VS{12}Mais quand ils eurent cru ce que Philippe leur annonçait, touchant l'Evangile du Royaume de Dieu, et le Nom de Jésus-Christ, tant les hommes que les femmes furent baptisés.
\VS{13}Et Simon crut aussi lui-même, et après avoir été baptisé, il ne quittait plus Philippe~; et voyant les prodiges et les grands miracles qui se faisaient, il était comme ravi hors de lui même.
\VS{14}Or quand les apôtres, qui étaient à Jérusalem eurent entendu que la Samarie avait reçu la parole de Dieu, ils leur envoyèrent Pierre et Jean~;
\VS{15}qui y étant descendus prièrent pour eux, afin qu'ils reçoivent le Saint-Esprit,
\VS{16}car il n'était pas encore descendu sur aucun d'eux, mais seulement ils étaient baptisés au Nom du Seigneur Jésus.
\VS{17}Puis ils leur imposèrent les mains, et ils reçurent le Saint-Esprit.
\VS{18}Lorsque Simon vit que le Saint-Esprit était donné par l'imposition des mains des apôtres, il leur présenta de l'argent,
\VS{19}en leur disant~: Donnez-moi aussi ce pouvoir, afin que tous ceux à qui j'imposerai les mains reçoivent le Saint-Esprit.
\VS{20}Mais Pierre lui dit~: Que ton argent périsse avec toi, puisque tu as estimé que le don de Dieu s'acquérait avec de l'argent.
\VS{21}Tu n'as point de part ni d'héritage en cette affaire~; car ton cœur n'est point droit devant Dieu.
\VS{22}Repens-toi donc de cette méchanceté, et prie Dieu, afin que, s'il est possible, la pensée de ton cœur te soit pardonnée.
\VS{23}Car je vois que tu es dans un fiel très amer et dans un lien d'iniquité.
\VS{24}Alors Simon répondit, et dit~: Vous, priez le Seigneur pour moi, afin qu'il ne m'arrive rien de ce que vous avez dit. 
\VS{25}Eux donc après avoir prêché et annoncé la parole du Seigneur, retournèrent à Jérusalem et annoncèrent l'Evangile dans plusieurs villages des Samaritains.
\TextTitle{Conversion et baptême de l'eunuque éthiopien}
\VS{26}Puis l'Ange du Seigneur parla à Philippe, en disant~: Lève-toi et va vers le Midi, sur le chemin qui descend de Jérusalem à Gaza, celui qui est désert.
\VS{27}Il se leva donc, et s'en alla. Et voici, un homme éthiopien, un eunuque, qui était un des principaux seigneurs de la cour de Candace, reine des Ethiopiens, et surintendant de toutes ses richesses, venu à Jérusalem pour adorer,
\VS{28}s'en retournait, assis dans son char, et lisait le prophète Esaïe.
\VS{29}L'Esprit dit à Philippe~: Avance, et approche-toi de ce char.
\VS{30}Philippe accourut et entendit l'Ethiopien qui lisait le prophète Esaïe~; et il lui dit~: Comprends-tu ce que tu lis~?
\VS{31}Et il lui dit~: Comment pourrais-je le comprendre, si quelqu'un ne me guide pas~? Et il pria Philippe de monter et s'asseoir avec lui.
\VS{32}Le passage de l'Ecriture qu'il lisait était celui-ci~: Il a été mené comme une brebis à la boucherie, et comme un agneau muet devant celui qui le tond~; en sorte qu'il n'a point ouvert sa bouche.
\VS{33}Dans son humiliation, son jugement a été levé~; mais qui racontera sa durée~? Car sa vie est retranchée de la terre\FTNT{\vref{Es. 53:7-8}.}.
\VS{34}Et l'eunuque prenant la parole, dit à Philippe~: Je te prie, de qui est-ce que le prophète dit cela~? Est-ce de lui-même, ou de quelque autre~?
\VS{35}Alors Philippe, ouvrant sa bouche, et commençant par cette Ecriture, lui annonça l'Evangile de Jésus.
\VS{36}Comme ils continuaient leur chemin, ils arrivèrent à un endroit où il y avait de l'eau. Et l'eunuque dit~: Voici de l'eau, qu'est-ce qui empêche que je ne sois baptisé~?
\VS{37}Philippe dit~: Si tu crois de tout ton cœur, cela t'est permis~; et l'eunuque répondit~: Je crois que Jésus-Christ est le Fils de Dieu.
\VS{38}Il fit arrêter le char~; Philippe et l'eunuque descendirent tous deux dans l'eau, et Philippe le baptisa.
\VS{39}Quand ils furent sortis de l'eau, l'Esprit du Seigneur enleva Philippe, et l'eunuque ne le vit plus. Tandis que tout joyeux il continua son chemin,
\VS{40}Philippe se trouva dans Azot, d'où il alla jusqu'à Césarée, en évangélisant toutes les villes par lesquelles il passait.
\Chap{9}
\TextTitle{Jésus se révèle à Saul\FTNTT{\vref{Ac. 22:1-16}~; \vref{26:9-18}}}
\VerseOne{}Or Saul, respirant encore la menace et le carnage contre les disciples du Seigneur, s'adressa au grand-prêtre,
\VS{2}et lui demanda des lettres de sa part pour les porter aux synagogues de Damas, afin que, s'il trouvait quelques-uns de cette secte, hommes ou femmes, il les amène liés à Jérusalem.
\VS{3}Or il arriva qu'en marchant, il approcha de Damas et tout à coup une lumière resplendit du ciel comme un éclair autour de lui.
\VS{4}Il tomba par terre et il entendit une voix qui lui disait~: Saul, Saul, pourquoi me persécutes-tu~?
\VS{5}Et il répondit~: Qui es-tu, Seigneur~? Et le Seigneur lui dit~: Je suis Jésus, que tu persécutes. Il te serait dur de regimber contre les aiguillons.
\VS{6}Alors, tout tremblant et tout effrayé, il dit~: Seigneur, que veux-tu que je fasse~? Et le Seigneur lui dit~: Lève-toi, et entre dans la ville, et on te dira ce que tu dois faire.
\VS{7}Les hommes qui l'accompagnaient s'arrêtèrent tout épouvantés, entendant bien la voix, mais ne voyant personne.
\VS{8}Et Saul se leva de terre, et ouvrant ses yeux, il ne voyait personne~; c'est pourquoi ils le conduisirent par la main, et le menèrent à Damas,
\VS{9}où il fut trois jours sans voir, sans manger ni boire.
\VS{10}Or il y avait à Damas un disciple, nommé Ananias, à qui le Seigneur dit en vision~: Ananias~! Et il répondit~: Me voici Seigneur~!
\VS{11}Et le Seigneur lui dit~: Lève-toi, va dans la rue appelée la droite, et cherche dans la maison de Judas un homme appelé Saul, de Tarse.
\VS{12}Car il prie. Or Saul avait vu en vision un homme appelé Ananias, entrant et lui imposant les mains, afin qu'il recouvre la vue. Et Ananias répondit~:
\VS{13} Seigneur, j'ai entendu parler plusieurs fois de cet homme-là~; et combien de maux il a faits à tes saints dans Jérusalem.
\VS{14}Il a même ici le pouvoir de la part des principaux prêtres, de lier tous ceux qui invoquent ton Nom.
\VS{15}Mais le Seigneur lui dit~: Va~; car il m'est un vase\FTNTT{Le mot «~vase~» vient du grec «~skeuos~». «~Vase~» était une métaphore grecque commune pour «~le corps~» car les Grecs pensaient que l'âme vivait temporairement dans les corps. \vref{2 Co. 4:7}~; \vref{Ro. 9:21-23}~; \vref{2 Ti. 2:20-21}.} que j'ai choisi, pour porter mon Nom devant les Gentils, et les rois, et les enfants d'Israël.
\VS{16}Car je lui montrerai combien il aura à souffrir pour mon Nom.
\TextTitle{Saul rempli du Saint-Esprit}
\VS{17}Ananias sortit~; et lorsqu'il fut arrivé dans la maison, il imposa les mains à Saul, et lui dit~: Saul mon frère, le Seigneur Jésus, qui t'est apparu sur le chemin par lequel tu venais, m'a envoyé, afin que tu recouvres la vue, et que tu sois rempli du Saint-Esprit.
\TextTitle{Saul est baptisé et évangélise Damas}
\VS{18}Et aussitôt il tomba de ses yeux comme des écailles~; et à l'instant il recouvra la vue. Puis il se leva, et fut baptisé.
\VS{19}Et ayant mangé, il reprit ses forces. Et Saul fut quelques jours avec les disciples qui étaient à Damas.
\VS{20}Et aussitôt il prêcha dans les synagogues que Jésus était le Fils de Dieu.
\VS{21}Et tous ceux qui l'entendaient, étaient comme ravis hors d'eux-mêmes, et ils disaient~: N'est-ce pas celui-là qui a détruit à Jérusalem ceux qui invoquaient ce Nom, et qui est venu ici exprès pour les amener liés aux principaux prêtres~?
\VS{22}Mais Saul se fortifiait de plus en plus, et confondait les Juifs qui habitaient à Damas, prouvant que Jésus était le Christ.
\TextTitle{Complot contre Saul}
\VS{23}Longtemps après, les Juifs conspirèrent ensemble pour le faire mourir~;
\VS{24}et leur complot parvint à la connaissance de Saul. Or ils gardaient les portes jour et nuit, afin de le faire mourir.
\VS{25}Mais pendant une nuit, les disciples le prirent, et le descendirent par la muraille dans une corbeille.
\TextTitle{Saul rencontre Barnaba et les apôtres à Jérusalem}
\VS{26}Lorsqu'il se rendit à Jérusalem, Saul tâcha de se joindre aux disciples~; mais tous le craignaient, ne croyant pas qu'il fût un disciple.
\VS{27}Alors Barnabas, l'ayant pris avec lui, le conduisit vers les apôtres, et leur raconta comment sur le chemin, Saul avait vu le Seigneur, qui lui avait parlé, et comment à Damas il parlait librement au Nom de Jésus.
\VS{28}Et il allait et venait avec eux dans Jérusalem, il parlait franchement au Nom du Seigneur, se montrant publiquement.
\VS{29}Et parlant sans déguisement au Nom du Seigneur Jésus, il disputait contre les Hellénistes, mais ils tentaient de le faire mourir.
\TextTitle{Retour à Tarse}
\VS{30}Les frères, l'ayant découvert, l'emmenèrent à Césarée, et le firent partir à Tarse.
\VS{31}Les églises étaient en paix dans toute la Judée, la Galilée, et la Samarie, étant édifiées et marchant dans la crainte du Seigneur~; et elles s'accroissaient par le rafraîchissement du Saint-Esprit.
\TextTitle{Guérison d'Enée, le paralytique}
\VS{32}Or il arriva que comme Pierre les visitait tous, il descendit aussi vers les saints qui demeuraient à Lydde.
\VS{33}Il y vint aussi un homme appelé Enée, qui était couché dans un petit lit depuis huit ans, car il était paralytique.
\VS{34}Et Pierre lui dit~: Enée, Jésus-Christ te guérit~! Lève-toi et arrange ton lit. Et aussitôt il se leva.
\VS{35}Tous ceux qui habitaient à Lydde et à Saron le virent, et ils se convertirent au Seigneur.
\TextTitle{Résurrection de Tabitha}
\VS{36}Il y avait à Joppé une femme disciple, appelée Tabitha, qui signifie en grec Dorcas~; elle faisait beaucoup de bonnes œuvres et d'aumônes.
\VS{37}Elle tomba malade en ce temps-là, et mourut. Après l'avoir lavée, on la déposa dans une chambre haute.
\VS{38}Comme Lydde était près de Joppé, les disciples ayant appris que Pierre était à Lydde, ils envoyèrent vers lui deux hommes, pour le prier de venir chez eux sans tarder.
\VS{39}Pierre se leva, et partit avec ces hommes. Lorsqu'il fut arrivé, on le conduisit dans la chambre haute. Toutes les veuves l'entourèrent en pleurant, et lui montrèrent les tuniques et les vêtements que faisait Dorcas quand elle était avec elles.
\VS{40}Pierre fit sortir tout le monde, se mit à genoux, et pria~; puis se tournant vers le corps, il dit~: Tabitha, lève-toi~! Et elle ouvrit ses yeux, et voyant Pierre, elle s'assit.
\VS{41}Il lui donna la main, et la fit lever. Puis ayant appelé les saints et les veuves, il la leur présenta vivante.
\VS{42}Cela fut connu dans tout Joppé~; et plusieurs crurent au Seigneur.
\VS{43}Et il arriva qu'il demeura plusieurs jours à Joppé, chez un corroyeur nommé Simon.
\Chap{10}
\TextTitle{Un ange de Dieu apparait à Corneille}
\VerseOne{}Il y avait à Césarée un homme nommé Corneille, centenier d'une cohorte de la légion appelée Italienne.
\VS{2}Cet homme était pieux et craignait Dieu avec toute sa famille. Il faisait aussi beaucoup d'aumônes au peuple, et priait Dieu continuellement.
\VS{3}Vers la neuvième heure du jour, il vit clairement dans une vision un ange de Dieu qui entra chez lui, et qui lui dit~: Corneille~!
\VS{4}Corneille ayant les yeux fixés sur lui, et tout effrayé, lui dit~: Qu'y a-t-il Seigneur~? Et il lui dit~: Tes prières et tes aumônes sont montées devant Dieu, et il s'en est souvenu.
\VS{5}Maintenant donc envoie des gens à Joppé, et fais venir Simon, surnommé Pierre.
\VS{6}Il est logé chez un certain Simon, corroyeur, qui a sa maison près de la mer~; c'est lui qui te dira ce qu'il faut que tu fasses.
\VS{7}Dès que l'ange qui lui parlait fut parti, Corneille appela deux de ses serviteurs, et un soldat craignant Dieu, d'entre ceux qui se tenaient près de lui.
\VS{8}Et après leur avoir tout raconté, il les envoya à Joppé.
\TextTitle{Vision de Pierre~: Une nappe descend du ciel}
\VS{9}Le lendemain, comme ils marchaient et qu'ils approchaient de la ville, Pierre monta sur le toit, vers la sixième heure, pour prier.
\VS{10}Et il arriva qu'ayant faim, il voulut prendre son repas. Pendant qu'on lui préparait à manger, il tomba en extase.
\VS{11}Il vit le ciel ouvert, et un vase descendant sur lui semblable à une grande nappe, attachée par les quatre coins, qui descendait vers la terre,
\VS{12}où se trouvaient tous les quadrupèdes, les bêtes sauvages, les reptiles et les oiseaux du ciel.
\VS{13}Et une voix lui dit~: Pierre, lève-toi, tue, et mange.
\VS{14}Mais Pierre répondit~: Non, Seigneur, car je n'ai jamais rien mangé de souillé ni d'impur.
\VS{15}Et la voix lui dit encore pour la seconde fois~: Les choses que Dieu a purifiées, ne les tiens point pour souillées.
\VS{16}Et cela arriva jusqu'à trois fois, et puis le vase fut retiré au ciel.
\VS{17}Comme Pierre ne savait pas en lui-même que penser du sens de la vision qu'il avait eue, voici, les hommes envoyés par Corneille s'étant mis en quête de la maison de Simon, se présentèrent à la porte,
\VS{18}et demandèrent à haute voix si c'était là que logeait Simon, surnommé Pierre.
\VS{19}Et comme Pierre pensait à la vision, l'Esprit lui dit~: Voici trois hommes qui te demandent.
\VS{20}Lève-toi donc et descends, et pars avec eux sans hésiter, car c'est moi qui les ai envoyés.
\VS{21}Pierre donc, descendit vers les gens qui lui avaient été envoyés par Corneille et leur dit~: Voici, je suis celui que vous cherchez~; pour quel sujet êtes-vous venus~?
\VS{22}Et ils dirent~: Corneille, centenier, homme juste et craignant Dieu, et à qui toute la nation des Juifs rend un bon témoignage, a été averti de Dieu par un saint ange de te faire venir dans sa maison et d'entendre tes paroles.
\TextTitle{Pierre chez Corneille}
\VS{23}Alors Pierre les fit entrer, et les logea. Le lendemain il s'en alla avec eux, et quelques-uns des frères de Joppé l'accompagnèrent.
\VS{24}Ils arrivèrent à Césarée le jour suivant. Corneille les attendait, et avait invité ses parents et ses amis.
\VS{25}Lorsque Pierre entra, Corneille qui était allé au-devant de lui, se jeta à ses pieds, et se prosterna.
\VS{26}Mais Pierre le releva en lui disant~: Lève-toi, moi aussi je suis un homme.
\VS{27}Et s'entretenant avec lui, il entra et trouva plusieurs personnes réunies.
\VS{28}Et il leur dit~: Vous savez qu'il n'est pas permis à un homme Juif de se lier avec un étranger, ou d'aller chez lui, mais Dieu m'a montré que je ne devais estimer aucun homme être impur ou souillé.
\VS{29}C'est pourquoi, ayant été appelé, je suis venu sans difficulté. Je vous demande donc pour quel sujet vous m'avez fait venir.
\VS{30}Corneille lui dit~: Il y a quatre jours, à cette heure-ci, j'étais en jeûne et en prière dans ma maison, et tout à coup, un homme, vêtu d'un habit resplendissant, se présenta devant moi et me dit~:
\VS{31}Corneille, ta prière est exaucée, et Dieu s'est souvenu de tes aumônes.
\VS{32}Envoie donc quelqu'un à Joppé, et fais venir Simon, surnommé Pierre, qui est logé dans la maison de Simon, le corroyeur, près de la mer. Quand il sera venu, il te parlera.
\VS{33}Aussitôt j'ai envoyé quelqu'un vers toi, et tu as bien fait de venir. Maintenant donc nous sommes tous présents devant Dieu pour entendre tout ce que Dieu t'a ordonné de nous dire.
\TextTitle{Pierre évangélise les Gentils\FTNTT{\vref{Ac. 2:14-41}}}
\VS{34}Alors Pierre prenant la parole, dit~: En vérité, je reconnais que Dieu n'a point égard à l'apparence des personnes,
\VS{35}mais qu'en toute nation celui qui le craint et qui pratique la justice, lui est agréable.
\VS{36}C'est ce qu'il a fait entendre aux enfants d'Israël, en leur annonçant la paix par Jésus-Christ, qui est le Seigneur de tous.
\VS{37}Vous savez ce qui est arrivé dans toute la Judée, après avoir commencé en Galilée, à la suite du baptême que Jean a prêché~;
\VS{38}vous savez comment Dieu a oint du Saint-Esprit et de force Jésus de Nazareth, qui allait de lieu en lieu, faisant du bien et guérissant tous ceux qui étaient sous l'empire du diable, car Dieu était avec lui.
\VS{39}Nous sommes témoins de toutes les choses qu'il a faites, dans le pays des Juifs et à Jérusalem. Cependant ils l'ont fait mourir en le pendant au bois.
\VS{40}Dieu l'a ressuscité le troisième jour, et il a permis qu'il apparaisse,
\VS{41}non à tout le peuple, mais aux témoins choisis d'avance par Dieu, à nous, qui avons mangé et bu avec lui après qu'il fut ressuscité des morts.
\VS{42}Et il nous a ordonné de prêcher au peuple, et d'attester que c'est lui qui a été établi par Dieu, juge des vivants et des morts.
\VS{43}Tous les prophètes rendent de lui le témoignage que quiconque croit en lui, reçoit la rémission de ses péchés par son Nom.
\TextTitle{Le Saint-Esprit descend sur les Gentils}
\VS{44}Comme Pierre prononçait encore ce discours, le Saint-Esprit descendit sur tous ceux qui écoutaient la parole.
\VS{45}Tous les fidèles circoncis qui étaient venus avec Pierre, furent étonnés de ce que le don du Saint-Esprit était aussi répandu sur les Gentils.
\VS{46}Car ils les entendaient parler diverses langues et glorifier Dieu.
\VS{47}Alors Pierre prenant la parole, dit~: Quelqu'un pourrait-il empêcher qu'on baptise dans l'eau ceux qui ont reçu le Saint-Esprit aussi bien que nous~?
\VS{48}Et il ordonna qu'ils soient baptisés au Nom du Seigneur. Après cela, ils le prièrent de rester quelques jours auprès d'eux.
\Chap{11}
\TextTitle{Dieu accorde la repentance aux Gentils}
\VerseOne{}Or les apôtres et les frères qui étaient en Judée apprirent que les Gentils aussi avaient reçu la parole de Dieu.
\VS{2}Et quand Pierre fut monté à Jérusalem, ceux de la circoncision disputaient contre lui,
\VS{3}disant~: Tu es entré chez des hommes incirconcis, et tu as mangé avec eux.
\VS{4}Alors Pierre commençant, leur exposa le tout par ordre, disant~:
\VS{5}J'étais dans la ville de Joppé, et pendant que je priais, je tombai en extase et j'eus une vision. Un vase semblable à une grande nappe, attachée par les quatre coins, descendit du ciel, et vint jusqu'à moi.
\VS{6}Les regards fixés sur cette nappe, j'examinai, et je vis les quadrupèdes, les bêtes sauvages, les reptiles, et les oiseaux du ciel.
\VS{7}Et j'entendis une voix qui me disait~: Pierre, lève-toi, tue, et mange.
\VS{8}Et je répondis~: Non Seigneur, car jamais rien de souillé ni d'impur n'est entré dans ma bouche.
\VS{9}La voix me parla du ciel une seconde fois~: Ce que Dieu a déclaré pur, ne le regarde pas comme souillé.
\VS{10}Cela arriva jusqu'à trois fois, puis toutes ces choses furent retirées dans le ciel.
\VS{11}Et voici, aussitôt trois hommes qui avaient été envoyés de Césarée vers moi, se présentèrent à la maison où j'étais.
\VS{12}L'Esprit me dit de partir avec eux sans hésiter. Les six frères que voici m'accompagnèrent, et nous entrâmes dans la maison de Corneille.
\VS{13}Cet homme nous raconta comment il avait vu dans sa maison un ange qui s'était présenté à lui, et lui avait dit~: Envoie des gens à Joppé, et fais venir Simon, surnommé Pierre,
\VS{14}qui te dira des choses par lesquelles tu seras sauvé, toi et toute ta maison.
\VS{15}Lorsque je me fus mis à parler, le Saint-Esprit descendit sur eux, comme il était descendu sur nous au commencement.
\VS{16}Et je me souvins de cette parole du Seigneur, et comment il avait dit~: Jean a baptisé d'eau, mais vous, vous serez baptisés du Saint-Esprit.
\VS{17}Or puisque Dieu leur a accordé le même don qu'à nous qui avons cru au Seigneur Jésus-Christ, pouvais-je, moi, m'opposer à Dieu~?
\VS{18}Après avoir entendu ces choses, ils s'apaisèrent, et ils glorifièrent Dieu en disant~: Dieu a donc accordé la repentance aussi aux Gentils, afin qu'ils aient la vie.
\TextTitle{Les disciples appelés «~chrétiens~» pour la première fois à Antioche}
\VS{19}Ceux qui avaient été dispersés par la persécution survenue à cause d'Etienne, allèrent jusqu'en Phénicie, dans l'île de Chypre, et à Antioche\FTNT{Antioche~: Capitale de la Syrie située sur le fleuve Oronte, fondée en 300 av. J.-C., et ainsi nommée en l'honneur de son fondateur Antiochus. De nombreux Juifs grecs y vivaient et c'est là que les disciples de Christ furent appelés pour la première fois, chrétiens.}, n'annonçant la parole à personne, seulement aux Juifs.
\VS{20}Mais il y eut parmi eux quelques hommes de Chypre et de Cyrène qui, étant venus à Antioche, parlèrent aussi aux Grecs, et leur annoncèrent l'Evangile du Seigneur Jésus.
\VS{21}La main du Seigneur était avec eux, et un grand nombre de personnes crurent et se convertirent au Seigneur.
\VS{22}Le bruit en parvint aux oreilles de l'Eglise qui était à Jérusalem, et ils envoyèrent Barnabas jusqu'à Antioche.
\VS{23}Lorsqu'il fut arrivé, et qu'il eut vu la grâce de Dieu, il s'en réjouit, et il les exhortait tous à demeurer attachés au Seigneur de tout leur cœur.
\VS{24}Car c'était un homme de bien, plein du Saint-Esprit et de foi. Et un grand nombre de personnes se joignirent au Seigneur.
\VS{25}Barnabas s'en alla à Tarse pour chercher Saul~;
\VS{26}et l'ayant trouvé, il l'amena à Antioche. Pendant toute une année, ils se réunirent aux assemblées de l'Eglise, et ils enseignèrent beaucoup de personnes. Ce fut à Antioche que, pour la première fois, les disciples furent appelés chrétiens.
\TextTitle{Prophétie d'Agabus}
\VS{27}En ce temps-là, quelques prophètes descendirent de Jérusalem à Antioche.
\VS{28}L'un d'eux, nommé Agabus, se leva et déclara par l'Esprit qu'une grande famine devait arriver sur toute la terre. Elle arriva, en effet, sous Claude César.
\VS{29}Les disciples résolurent d'envoyer, chacun selon ses moyens, quelque secours pour subvenir aux besoins des frères qui habitaient la Judée.
\VS{30}Ils le firent parvenir aux anciens par les mains de Barnabas et de Saul.
\Chap{12}
\TextTitle{Cinquième persécution de l'Eglise~: Meurtre de Jacques et arrestation de Pierre}
\VerseOne{}En ce même temps, le roi Hérode se mit à maltraiter quelques membres de l'Eglise~;
\VS{2}et il fit mourir par l'épée Jacques, frère de Jean.
\VS{3}Voyant que cela était agréable aux Juifs, il fit aussi arrêter Pierre. C'était pendant les jours des pains sans levain.
\VS{4}Après l'avoir saisi et jeté en prison, il le mit sous la garde de quatre bandes de quatre soldats chacune, avec l'intention de le faire comparaître devant le peuple après la fête de Pâque.
\TextTitle{L'Ange du Seigneur délivre Pierre de la prison}
\VS{5}Pierre était donc gardé dans la prison~; mais l'Eglise faisait sans cesse des prières à Dieu pour lui.
\VS{6}La nuit qui précéda le jour où Hérode devait l'envoyer au supplice, Pierre dormait entre deux soldats, lié de deux chaînes~; et les gardes qui étaient devant la porte gardaient la prison.
\VS{7}Et voici, l'Ange du Seigneur survint, et une lumière resplendit dans la prison. L'Ange réveilla Pierre en le frappant au côté, et en disant~: Lève-toi promptement~! Et les chaînes tombèrent de ses mains.
\VS{8}Et l'Ange lui dit~: Mets ta ceinture et tes sandales. Et il fit ainsi. L'Ange lui dit encore~: Enveloppe-toi de ton manteau et suis-moi.
\VS{9}Pierre sortit et le suivit, ne sachant pas que ce qui se faisait par l'Ange était réel, car il croyait qu'il avait une vision.
\VS{10}Lorsqu'ils eurent passé la première et la seconde garde, ils arrivèrent à la porte de fer qui mène à la ville, et qui s'ouvrit d'elle-même devant eux~; et ils sortirent et s'avancèrent dans une rue. Et subitement, l'Ange quitta Pierre.
\VS{11}Revenu à lui-même, Pierre dit~: Je vois à présent d'une manière certaine que le Seigneur a envoyé son Ange, et qu'il m'a délivré de la main d'Hérode, et de toute l'attente du peuple Juif.
\VS{12}Après avoir réfléchi, il alla à la maison de Marie, mère de Jean, surnommé Marc, où plusieurs personnes étaient assemblées et priaient.
\VS{13}Il frappa à la porte du vestibule, une servante, appelée Rhode, vint pour écouter.
\VS{14}Elle reconnut la voix de Pierre, et dans sa joie elle n'ouvrit pas la porte du vestibule, mais elle courut dans la maison et annonça que Pierre était devant la porte.
\VS{15}Ils lui dirent~: Tu es folle. Mais elle affirma que ce qu'elle disait était vrai.
\VS{16}Et ils dirent~: C'est son ange. Cependant Pierre continuait à frapper. Et quand ils eurent ouvert, ils le virent, et furent étonnés de le voir.
\VS{17}Mais leur ayant fait signe de la main de se taire, il leur raconta comment le Seigneur l'avait fait sortir de la prison, et il leur dit~: Annoncez ces choses à Jacques et aux frères. Puis sortant de là il s'en alla dans un autre lieu.
\VS{18}Quand il fit jour, les soldats furent dans une grande agitation, pour savoir ce que Pierre était devenu.
\VS{19}Et Hérode l'ayant cherché, et ne le trouvant point, après en avoir fait le procès aux gardes, il commanda qu'ils fussent menés au supplice.
\TextTitle{Mort d'Hérode}
\VS{20}Hérode avait le dessein de faire la guerre aux Tyriens et aux Sidoniens~; mais ils vinrent le trouver d'un commun accord~; et ayant gagné Blaste, son Chambellan, ils demandèrent la paix, parce que leur pays tirait sa subsistance de celui du roi.
\VS{21}A un jour marqué, Hérode, revêtu de ses habits royaux, s'assit sur son trône et les harangua publiquement.
\VS{22}Le peuple s'écria~: Voix d'un dieu et non point d'un homme~!
\VS{23}Et à l'instant l'Ange du Seigneur le frappa, parce qu'il n'avait pas donné gloire à Dieu. Et il expira, rongé des vers.
\VS{24}Cependant la parole de Dieu se répandait de plus en plus, et le nombre des disciples augmentait.
\VS{25}Barnabas et Saul, après s'être acquittés de leur service, s'en retournèrent de Jérusalem, ayant aussi pris avec eux Jean, surnommé Marc.
\Chap{13}
\TextTitle{Saul et Barnabas mis à part par le Saint-Esprit}
\VerseOne{}Or il y avait dans l'église qui était à Antioche des prophètes et des docteurs, Barnabas, Siméon, appelé Niger, Lucius, le Cyrénien, Manahen, qui avait été élevé avec Hérode, le tétrarque, et Saul.
\VS{2}Et tandis qu'ils servaient\FTNT{Certains traducteurs ont rajouté la phrase «~dans leur ministère~» alors que les textes originaux ne la mentionne pas.} le Seigneur et jeûnaient, le Saint-Esprit dit~: Séparez-moi maintenant Barnabas et Saul pour l'œuvre à laquelle je les ai appelés.
\VS{3}Alors après avoir jeûné et prié, ils leur imposèrent les mains, et les laissèrent partir\FTNT{Voir annexe «~Les voyages missionnaires de Paul~».}.
\TextTitle{Saul, Barnabas et Jean sur l'île de Chypre}
\VS{4}Barnabas et Saul, envoyés par le Saint-Esprit, descendirent à Séleucie, et de là ils s'embarquèrent pour l'île de Chypre.
\VS{5}Et lorsqu'ils furent à Salamine, ils annoncèrent la parole de Dieu dans les synagogues des Juifs~; ils avaient Jean avec eux pour les aider.
\TextTitle{Bar-Jésus aveuglé et conversion du proconsul Sergius Paulus}
\VS{6}Ayant ensuite traversé l'île jusqu'à Paphos, ils trouvèrent là un certain magicien, faux prophète Juif, nommé Bar-Jésus,
\VS{7}qui était avec le proconsul Sergius Paulus, homme intelligent qui fit appeler Barnabas et Saul, désirant entendre la parole de Dieu.
\VS{8}Mais Elymas, le magicien, car c'est ce que signifie ce nom, leur résistait, cherchant à détourner de la foi le proconsul.
\VS{9}Alors Saul, appelé aussi Paul, rempli du Saint-Esprit, fixa les yeux sur lui et dit~:
\VS{10}Ô homme plein de toute fraude et de toute ruse, fils du diable, ennemi de toute justice, ne cesseras-tu point de renverser les voies droites du Seigneur~?
\VS{11}C'est pourquoi, voici la main du Seigneur est sur toi, tu seras aveugle, et pour un temps tu ne verras pas le soleil. Aussitôt l'obscurité et les ténèbres tombèrent sur lui, et il cherchait, en tâtonnant, des personnes pour le guider.
\VS{12}Alors le proconsul voyant ce qui était arrivé, crut, étant rempli d'admiration pour la doctrine du Seigneur.
\VS{13}Et quand Paul et ceux qui étaient avec lui furent partis de Paphos, ils vinrent à Perge, ville de Pamphylie. Jean se sépara d'eux et retourna à Jérusalem.
\TextTitle{Paul à Antioche de Pisidie~: Le salut par la foi en Jésus}
\VS{14}De Perge, ils poursuivirent leur route, et arrivèrent à Antioche, ville de Pisidie\FTNT{Antioche de Pisidie~: Ville de Pisidie (en Turquie), à la frontière de Phrygie, fondée par Seleucus Nicanor. Elle devint une colonie romaine et fut aussi appelée Césarée.}, et étant entrés dans la synagogue le jour du sabbat, ils s'assirent.
\VS{15}Après la lecture de la loi et des prophètes, les chefs de la synagogue leur envoyèrent dire~: Hommes frères, si vous avez quelque parole d'exhortation pour le peuple, dites-la.
\VS{16}Alors Paul s'étant levé, et ayant fait signe de la main qu'on fasse silence, dit~: Hommes Israélites, et vous qui craignez Dieu, écoutez.
\VS{17}Le Dieu de ce peuple d'Israël a choisi nos pères. Il a distingué glorieusement ce peuple pendant son séjour au pays d'Egypte, et il l'en fit sortir par son bras élevé.
\VS{18}Il les supporta\FTNT{Le verbe supporter vient du grec «~tropophoreo~» qui signifie supporter les manières, endurer le caractère de quelqu'un.} au désert environ quarante ans.
\VS{19}Et ayant détruit sept nations au pays de Canaan, il leur distribua le pays par le sort.
\VS{20}Après cela, durant quatre cent cinquante ans, il leur donna des juges, jusqu'à Samuel le prophète.
\VS{21}Puis ils demandèrent un roi, et Dieu leur donna Saül fils de Kis, homme de la tribu de Benjamin~; et ainsi se passèrent quarante ans.
\VS{22}Et Dieu l'ayant rejeté, il leur suscita pour roi David, auquel il a rendu ce témoignage~: J'ai trouvé David, fils d'Isaï, homme selon mon cœur, qui exécutera toute ma volonté.
\VS{23}C'est de la postérité de David que Dieu, selon sa promesse, a suscité Jésus pour être le Sauveur d'Israël.
\VS{24}Avant la venue de Jésus, Jean avait prêché le baptême de repentance à tout le peuple d'Israël.
\VS{25}Et comme Jean achevait sa course, il disait~: Qui pensez-vous que je sois~? Je ne suis point le Christ~; mais voici, il en vient un après moi, dont je ne suis pas digne de délier le soulier de ses pieds.
\VS{26}Hommes frères, fils de la race d'Abraham, et vous qui craignez Dieu, c'est à vous que la parole de ce salut a été envoyée.
\VS{27}Car les habitants de Jérusalem et leurs chefs ont méconnu Jésus, et en le condamnant, ils ont accompli les paroles des prophètes qui se lisent chaque sabbat.
\VS{28}Quoiqu'ils n'aient rien trouvé en lui qui soit digne de mort, ils demandèrent à Pilate de le faire mourir.
\VS{29}Et après qu'ils eurent accompli toutes les choses qui avaient été écrites de lui, ils le descendirent du bois, et le déposèrent dans un sépulcre.
\VS{30}Mais Dieu l'a ressuscité des morts.
\VS{31}Il est apparu pendant plusieurs jours à ceux qui étaient montés avec lui de Galilée à Jérusalem, et qui sont ses témoins devant le peuple.
\VS{32}Et nous, nous vous annonçons cette bonne nouvelle que la promesse faite à nos pères,
\VS{33}Dieu l'a accomplie pour nous, leurs enfants, en ressuscitant Jésus, selon qu'il est écrit dans le deuxième psaume~: Tu es mon Fils, je t'ai aujourd'hui engendré\FTNT{\vref{Ps. 2:7}.}.
\VS{34}Et pour montrer qu'il l'a ressuscité des morts, pour ne plus devoir retourner au sépulcre, il a dit ainsi~: Je vous donnerai les grâces saintes promises à David, ces grâces qui sont assurées.
\VS{35}C'est pourquoi il a dit aussi dans un autre endroit~: Tu ne permettras point que ton Saint voie la corruption\FTNT{\vref{Ps. 16:10}.}.
\VS{36}Or David, après avoir servi en son temps au dessein de Dieu, est mort, a été réuni à ses pères, et a vu la corruption.
\VS{37}Mais celui que Dieu a ressuscité n'a pas vu la corruption.
\VS{38}Sachez donc, hommes frères, que c'est par lui que la rémission des péchés vous est annoncée,
\VS{39}et que quiconque croit est justifié par lui, de tout ce dont vous n'avez pas pu être justifiés par la loi de Moïse.
\VS{40}Prenez donc garde qu'il ne vous arrive ce qui est dit dans les prophètes~:
\VS{41}Voyez, vous mépriseurs, soyez étonnés et disparaissez~: Car je vais faire une œuvre en votre temps, une œuvre que vous ne croiriez pas si quelqu'un vous la racontait.
\VS{42}Lorsqu'ils sortirent de la synagogue des Juifs, les Gentils les prièrent de parler le sabbat suivant sur les mêmes choses.
\VS{43}Et quand l'assemblée fut séparée, beaucoup de Juifs et de prosélytes craignant Dieu, suivirent Paul et Barnabas qui les exhortèrent à persévérer dans la grâce de Dieu.
\TextTitle{Les juifs d'Antioche rejettent la Parole~; l'Évangile annoncé aux Gentils\FTNTT{\vref{Ac. 18:6}~; \vref{28:25-28}.}}
\VS{44}Le sabbat suivant, presque toute la ville s'assembla pour entendre la parole de Dieu.
\VS{45}Mais les Juifs voyant toute cette foule, furent remplis de jalousie, et ils s'opposaient à ce que Paul disait, en le contredisant et en blasphémant.
\VS{46}Alors Paul et Barnabas leur dirent avec assurance~: C'est à vous premièrement qu'il fallait annoncer la parole de Dieu, mais puisque vous la rejetez, et que vous vous jugez vous-mêmes indignes de la vie éternelle, voici, nous nous tournons vers les Gentils.
\VS{47}Car ainsi nous l'a ordonné le Seigneur~: Je t'ai établi pour être la lumière des Gentils, pour porter le salut jusqu'aux extrémités de la terre.
\VS{48}Les Gentils en entendant cela, se réjouissaient et ils glorifiaient la parole du Seigneur~; et tous ceux qui étaient destinés à la vie éternelle crurent.
\VS{49}Ainsi la parole du Seigneur se répandait dans tout le pays.
\VS{50}Mais les Juifs excitèrent quelques femmes dévotes et distinguées, et les principaux de la ville, et ils provoquèrent une persécution contre Paul et Barnabas, et les chassèrent de leur territoire.
\VS{51}Paul et Barnabas secouèrent contre eux la poussière de leurs pieds et allèrent à Icone,
\VS{52}tandis que les disciples étaient remplis de joie et du Saint-Esprit.
\Chap{14}
\TextTitle{Paul et Barnabas à Icone}
\VerseOne{}A Icone, Paul et Barnabas entrèrent ensemble dans la synagogue des Juifs, et ils parlèrent d'une telle manière qu'une grande multitude de Juifs et de Grecs crurent.
\VS{2}Mais ceux des Juifs qui furent rebelles, émurent et irritèrent les esprits des Gentils contre les frères.
\VS{3}Ils restèrent cependant assez longtemps à Icone, parlant avec assurance du Seigneur, qui rendait témoignage à la parole de sa grâce, en faisant par leurs mains des prodiges et des miracles.
\VS{4}La population de la ville fut partagée en deux, et les uns étaient du côté des Juifs, et les autres du côté des apôtres.
\TextTitle{Paul prêche à Derbe et à Lystre~; guérison d'un boiteux de naissance}
\VS{5}Et comme il se faisait une émeute des Gentils et des Juifs, avec leurs principaux chefs, pour outrager et lapider les apôtres,
\VS{6}Paul et Barnabas en ayant eu connaissance, se réfugièrent dans les villes de Lycaonie, à Lystre, à Derbe, et dans les contrées d'alentour.
\VS{7}Et ils y annoncèrent l'Evangile.
\VS{8}A Lystre, se tenait assis un homme impotent des pieds, boiteux dès sa naissance, et qui n'avait jamais marché.
\VS{9}Cet homme écoutait parler Paul. Et Paul fixant ses yeux sur lui, et voyant qu'il avait la foi pour être guéri,
\VS{10}lui dit à haute voix~: Lève-toi droit sur tes pieds. Et il se leva en sautant, et marcha.
\VS{11}Et les gens qui étaient là assemblés, ayant vu ce que Paul avait fait, élevèrent leur voix, disant en langue lycaonienne~: Les dieux sous une forme humaine, sont descendus vers nous.
\VS{12}Et ils appelaient Barnabas Jupiter, et Paul Mercure, parce que c'était lui qui portait la parole.
\VS{13}Le prêtre de Jupiter, qui était à l'entrée de leur ville, ayant amené des taureaux et des couronnes jusqu'à l'entrée de la porte voulait, de même que la foule, offrir un sacrifice.
\VS{14}Mais les apôtres Barnabas et Paul ayant appris cela, déchirèrent leurs vêtements et se précipitèrent au milieu de la foule,
\VS{15}et disant~: Ô hommes, pourquoi faites-vous cela~? Nous aussi, nous sommes des hommes, sujets aux mêmes passions que vous, et vous apportant l'Evangile, nous vous exhortons à renoncer à ces choses vaines, pour vous convertir au Dieu vivant, qui a fait le ciel et la terre, la mer, et tout ce qui s'y trouve.
\VS{16}Ce Dieu, dans les siècles passés, a laissé toutes les nations marcher dans leurs voies,
\VS{17}quoiqu'il n'ait cessé de rendre témoignage de ce qu'il est, en faisant du bien, en nous dispensant du ciel les pluies et les saisons fertiles, en nous donnant la nourriture avec abondance, et en remplissant nos cœurs de joie.
\VS{18}A peine purent-ils, par ces paroles, empêcher la foule de leur offrir un sacrifice.
\TextTitle{Paul lapidé à Lystre}
\VS{19}Alors survinrent quelques Juifs d'Antioche et d'Icone qui gagnèrent la foule, et qui après avoir lapidé Paul, le traînèrent hors de la ville, croyant qu'il était mort.
\VS{20}Mais les disciples s'étant assemblés autour de lui, il se leva et entra dans la ville~; et le lendemain il s'en alla avec Barnabas à Derbe.
\TextTitle{Vote et établissement des anciens dans les églises}
\VS{21}Quand ils eurent évangélisé cette ville, et fait un certain nombre de disciples, ils retournèrent à Lystre, à Icone, et à Antioche~;
\VS{22}fortifiant l'esprit des disciples, et les exhortant à persévérer dans la foi, disant que c'est par beaucoup de tribulations qu'il nous faut entrer dans le Royaume de Dieu.
\VS{23}Après le vote à main levée des assemblées, ils établirent des anciens dans chaque église, et après avoir prié et jeûné, ils les recommandèrent au Seigneur, en qui ils avaient cru.
\VS{24}Traversant ensuite la Pisidie, ils allèrent en Pamphylie,
\VS{25}annoncèrent la parole à Perge, et descendirent à Attalie.
\TextTitle{Retour à Antioche}
\VS{26}De là, ils s'embarquèrent pour Antioche, d'où ils avaient été recommandés à la grâce de Dieu, pour l'œuvre qu'ils venaient d'accomplir.
\VS{27}Et quand ils furent arrivés, ils convoquèrent l'église, et ils racontèrent toutes les choses que Dieu avait faites par eux, et comment il avait ouvert aux Gentils la porte de la foi.
\VS{28}Et ils demeurèrent assez longtemps avec les disciples.
\Chap{15}
\TextTitle{Des hommes venus de Judée veulent imposer la circoncision}
\VerseOne{}Quelques hommes qui étaient descendus de Judée, enseignaient les frères en disant~: Si vous n'êtes pas circoncis selon le rite de Moïse, vous ne pouvez pas être sauvés.
\VS{2}Paul et Barnabas eurent avec eux un débat et une vive discussion~; et les frères décidèrent que Paul et Barnabas, avec quelques-uns des leurs, monteraient à Jérusalem vers les apôtres et les anciens, pour traiter cette question.
\VS{3}Après avoir été accompagnés par l'assemblée, ils traversèrent la Phénicie et la Samarie, racontant la conversion des Gentils~; et ils causèrent une grande joie à tous les frères.
\VS{4}Arrivés à Jérusalem, ils furent reçus par l'église, les apôtres et les anciens, et ils racontèrent toutes les choses que Dieu avait faites par leur moyen.
\VS{5}Mais quelques-uns, de la secte des pharisiens qui avaient cru, se levèrent, en disant qu'il fallait circoncire les Gentils et leur ordonner de garder la loi de Moïse.
\TextTitle{Opposition de Pierre~: Les Gentils n'ont pas à être sous le joug de la loi}
\VS{6}Alors les apôtres et les anciens se réunirent pour examiner cette affaire.
\VS{7}Et après une grande discussion, Pierre se leva et leur dit~: Hommes frères, vous savez que depuis longtemps Dieu m'a choisi parmi nous, afin que par ma bouche, les Gentils entendent la parole de l'Evangile, et qu'ils croient.
\VS{8}Et Dieu, qui connaît les cœurs, leur a rendu témoignage en leur donnant le Saint-Esprit, de même qu'à nous.
\VS{9}Il n'a fait aucune différence entre nous et eux, ayant purifié leurs cœurs par la foi.
\VS{10}Maintenant donc pourquoi tentez-vous Dieu en voulant imposer aux disciples un joug que ni nos pères ni nous n'avons pu porter~?
\VS{11}Mais nous croyons que nous serons sauvés par la grâce du Seigneur Jésus-Christ, comme eux aussi.
\VS{12}Alors toute l'assemblée garda le silence, et l'on écouta Barnabas et Paul qui racontèrent tous les miracles et les prodiges que Dieu avait faits par leur moyen au milieu des Gentils.
\TextTitle{Discours de Jacques~: Les prophètes ont annoncé le salut pour les Gentils} 
\VS{13}Lorsqu'ils eurent cessé de parler, Jacques prit la parole et dit~: Hommes frères, écoutez-moi~!
\VS{14}Simon a raconté comment Dieu a premièrement jeté les regards sur les nations pour choisir du milieu d'elles un peuple consacré à son Nom. 
\VS{15}Et avec cela s'accordent les paroles des prophètes, selon qu'il est écrit~:
\VS{16}Après cela, je reviendrai, et je rebâtirai le tabernacle de David qui est tombé, je réparerai ses ruines et je le relèverai\FTNT{\vref{Am. 9:11}.}.
\VS{17}Afin que le reste des hommes recherche le Seigneur, et aussi toutes les nations sur lesquelles mon Nom est invoqué, dit le Seigneur, qui fait toutes ces choses.
\VS{18}Toutes les œuvres de Dieu lui sont connues de toute éternité.
\TextTitle{Les chrétiens issus des nations ne sont pas soumis à la loi mosaïque}
\VS{19}C'est pourquoi je suis d'avis qu'on ne crée pas des difficultés à ceux des Gentils qui se convertissent à Dieu~;
\VS{20}mais qu'on leur écrive de s'abstenir des souillures des idoles et de la fornication, des animaux étouffés et du sang.
\VS{21}Car depuis bien des générations, Moïse a, dans chaque ville, des gens qui le prêchent, puisqu'on le lit tous les jours de sabbat dans les synagogues.
\VS{22}Alors il parut bon aux apôtres et aux anciens avec toute l'Eglise, de choisir parmi eux et d'envoyer à Antioche avec Paul et Barnabas, Jude, appelé Barsabas, et Silas, hommes considérés entre les frères.
\VS{23}Ils écrivirent par eux en ces termes~: Les apôtres, les anciens, et les frères, aux frères d'entre les Gentils qui sont à Antioche, en Syrie, et en Cilicie, salut~!
\VS{24}Ayant appris que quelques hommes partis de chez nous, et auxquels nous n'avons donné aucun ordre, vous ont troublés par leurs discours et ont ébranlé vos âmes, en vous disant qu'il faut être circoncis et garder la loi,
\VS{25}nous avons été d'avis, étant assemblés tous d'un commun accord, d'envoyer vers vous, avec nos très chers Barnabas et Paul, des hommes que nous avons choisis. 
\VS{26}Ce sont des hommes qui ont abandonné leurs vies pour le Nom de notre Seigneur Jésus-Christ.
\VS{27}Nous avons donc envoyé Jude et Silas, qui vous feront entendre les mêmes choses de vive voix.
\VS{28}Car il a paru bon au Saint-Esprit et à nous, de ne vous imposer d'autre charge que ce qui est nécessaire,
\VS{29}savoir, de vous abstenir des viandes sacrifiées aux idoles, du sang, des animaux étouffés, et de la fornication~; choses contre lesquelles vous vous trouverez bien de vous tenir en garde. Adieu~!
\TextTitle{Mission de Jude et Silas à Antioche}
\VS{30}Après avoir donc pris congé de l'église, ils allèrent à Antioche, et ayant assemblé l'église, ils remirent la lettre.
\VS{31}Après l'avoir lue, les frères d'Antioche furent réjouis de la consolation qu'elle leur apportait.
\VS{32}Jude et Silas, qui étaient eux-mêmes prophètes, exhortèrent les frères par plusieurs discours, et les fortifièrent.
\VS{33}Au bout de quelque temps, ils furent renvoyés en paix par les frères vers les apôtres.
\VS{34}Toutefois Silas trouva bon de rester.
\VS{35}Et Paul et Barnabas demeurèrent aussi à Antioche, enseignant et annonçant, avec plusieurs autres, la parole du Seigneur.
\TextTitle{Paul et Barnabas se séparent}
\VS{36}Quelques jours après, Paul dit à Barnabas~: Retournons visiter nos frères dans toutes les villes où nous avons annoncé la parole du Seigneur, pour voir quel est leur état\FTNT{Voir annexe «~Les voyages missionaires de Paul~».}.
\VS{37}Barnabas voulait emmener avec eux Jean, surnommé Marc.
\VS{38}Mais Paul jugea plus convenable de ne pas prendre avec eux celui qui les avait quittés depuis la Pamphylie, et qui ne les avait point accompagnés dans leur œuvre.
\VS{39}Il y eut donc entre eux une contestation, en sorte qu'ils se séparèrent l'un de l'autre. Barnabas, prenant Marc avec lui, s'embarqua pour l'île de Chypre.
\VS{40}Mais Paul, ayant choisi Silas, pour l'accompagner, partit après avoir été recommandé à la grâce de Dieu par les frères.
\VS{41}Il traversa la Syrie et la Cilicie, fortifiant les églises.
\Chap{16}
\TextTitle{Circoncis, Timothée rejoint Paul dans la mission}
\VerseOne{}Il se rendit à Derbe et à Lystre, et voici, il y avait là un disciple, nommé Timothée, fils d'une femme juive fidèle et d'un père grec.
\VS{2}Les frères de Lystre et d'Icone rendaient de lui un bon témoignage.
\VS{3}C'est pourquoi Paul voulut l'emmener avec lui~; et l'ayant pris, il le circoncit, à cause des Juifs qui étaient dans ces lieux-là, car ils savaient tous que son père était grec.
\VS{4}En passant par les villes, ils recommandaient aux frères d'observer les ordonnances établies par les apôtres et les anciens de Jérusalem.
\VS{5}Ainsi les églises étaient affermies dans la foi et augmentaient en nombre chaque jour.
\TextTitle{Vision de Paul}
\VS{6}Ayant traversé la Phrygie et le pays de Galatie, le Saint-Esprit leur défendit d'annoncer la parole dans l'Asie.
\VS{7}Arrivés près de la Mysie, ils se disposaient à entrer en Bithynie~; mais l'Esprit de Jésus\FTNT{Notons que le Saint-Esprit est appelé Esprit de Jésus. Ainsi, de la même manière qu'on ne peut dissocier un homme de son esprit pour en faire deux entités distinctes, on ne peut dissocier Jésus de son Esprit. Dieu est un.} ne le leur permit pas.
\VS{8}Ils traversèrent ensuite la Mysie, et descendirent à Troas.
\VS{9}Pendant la nuit, Paul eut une vision d'un homme macédonien qui se présenta devant lui, et le pria, disant~: Passe en Macédoine et secours-nous~!
\VS{10}Après cette vision de Paul, nous cherchâmes aussitôt à nous rendre en Macédoine, concluant que le Seigneur nous appelait à les évangéliser.
\TextTitle{Paul à Philippes}
\VS{11}Ainsi étant partis de Troas, nous fîmes voile directement vers la Samothrace, et le lendemain à Néapolis.
\VS{12}De là nous allâmes à Philippes, qui est la première ville d'un district de Macédoine, et une colonie romaine. Nous séjournâmes quelque temps dans la ville.
\VS{13}Et le jour du sabbat nous sortîmes de la ville, et allâmes au lieu où on avait accoutumé de faire la prière, près du fleuve, et nous étant là assis nous parlâmes aux femmes qui y étaient assemblées.
\TextTitle{Conversion de Lydie}
\VS{14}L'une d'elles, appelée Lydie, marchande de pourpre, de la ville de Thyatire, était une femme craignant Dieu, et elle nous écoutait. Le Seigneur lui ouvrit le cœur, afin qu'elle soit attentive à ce que disait Paul.
\VS{15}Lorsqu'elle eut été baptisée, avec sa famille, elle nous fit cette demande~: Si vous me jugez fidèle au Seigneur, entrez dans ma maison, et demeurez-y. Et elle nous pressa par ses instances.
\TextTitle{Paul et Silas battus de verges et mis en prison}
\VS{16}Or il arriva que comme nous allions à la prière, une servante qui avait un esprit de python, et qui, en devinant, apportait un grand profit à ses maîtres, nous rencontra,
\VS{17}et elle se mit à nous suivre, Paul et nous, en criant et disant~: Ces hommes sont les serviteurs du Dieu Très-Haut, et ils vous annoncent la voie du salut~!
\VS{18}Elle fit cela pendant plusieurs jours. Mais Paul, fatigué, se retourna et dit à l'esprit~: Je t'ordonne au Nom de Jésus-Christ de sortir de cette fille. Et il sortit au même instant.
\VS{19}Mais les maîtres de la servante voyant disparaître l'espoir de leur gain, se saisirent de Paul et de Silas, et les traînèrent sur la place publique devant les magistrats.
\VS{20}Ils les présentèrent aux préteurs, en disant~: Ces hommes, qui sont Juifs, troublent notre ville.
\VS{21}Car ils annoncent des coutumes qu'il ne nous est pas permis de recevoir ni de suivre, à nous qui sommes Romains.
\VS{22}La foule se souleva aussi contre eux, et les préteurs, ayant fait déchirer leurs vêtements, ordonnèrent qu'ils soient battus de verges.
\VS{23}Après qu'on les eut chargés de coups de fouet, ils les mirent en prison, en recommandant au geôlier de les garder sûrement.
\VS{24}Le geôlier ayant reçu cet ordre, les mit au fond de la prison, et leur serra les pieds dans des ceps.
\TextTitle{Libération miraculeuse de Paul et Silas}
\VS{25}Vers minuit, Paul et Silas priaient et chantaient les louanges de Dieu, et les prisonniers les entendaient.
\VS{26}Tout à coup, il se fit un grand tremblement de terre, en sorte que les fondements de la prison furent ébranlés~; au même instant, toutes les portes s'ouvrirent et les liens de tous furent rompus.
\VS{27}Le geôlier se réveilla, et voyant les portes de la prison ouvertes, il tira son épée et allait se tuer, croyant que les prisonniers s'étaient enfuis.
\VS{28}Mais Paul cria d'une voix forte~: Ne te fais point de mal, nous sommes tous ici.
\TextTitle{Conversion et baptême du geôlier et de sa famille}
\VS{29}Alors le geôlier, ayant demandé de la lumière, entra précipitamment dans le cachot, et se jeta tout tremblant aux pieds de Paul et de Silas.
\VS{30}Il les fit sortir, et dit~: Seigneur, que faut-il que je fasse pour être sauvé~?
\VS{31}Paul et Silas répondirent~: Crois au Seigneur Jésus-Christ et tu seras sauvé, toi et ta famille.
\VS{32}Et ils lui annoncèrent la parole du Seigneur, et à tous ceux qui étaient dans sa maison.
\VS{33}Après cela, les prenant en cette même heure de la nuit, il lava leurs plaies, et aussitôt après il fut baptisé, avec tous ceux de sa maison.
\VS{34}Les ayant amenés dans sa maison, il leur servit à manger, et il se réjouit avec toute sa famille de ce qu'il avait cru en Dieu.
\TextTitle{Paul et Silas relâchés}
\VS{35}Quand il fit jour, les préteurs envoyèrent des huissiers pour dire au geôlier~: Relâche ces hommes.
\VS{36}Et le geôlier rapporta ces paroles à Paul, disant~: Les préteurs ont envoyé dire qu'on vous relâche~; maintenant donc sortez, et allez en paix.
\VS{37}Mais Paul dit aux huissiers~: Après nous avoir battus de verges publiquement et sans jugement, nous qui sommes Romains, ils nous ont jetés en prison, et maintenant ils nous font sortir secrètement~! Il n'en sera pas ainsi. Qu'ils viennent eux-mêmes nous mettre en liberté.
\VS{38}Les licteurs rapportèrent ces paroles aux préteurs qui furent effrayés en apprenant qu'ils étaient Romains.
\VS{39}Ils vinrent vers eux et leur firent des excuses, et ils les mirent en liberté en les priant de quitter la ville.
\VS{40}Quand ils furent sortis de la prison, ils entrèrent chez Lydie, et après avoir vu et consolé les frères, ils partirent.
\Chap{17}
\TextTitle{Paul et Silas à Thessalonique}
\VerseOne{}Paul et Silas passèrent par Amphipolis et par Apollonie, et ils arrivèrent à Thessalonique, où les Juifs avaient une synagogue.
\VS{2}Paul y entra, selon sa coutume. Pendant trois sabbats, il discuta avec eux d'après les Ecritures~;
\VS{3}expliquant et établissant que le Christ devait souffrir et ressusciter des morts. Et ce Jésus, que je vous annonce, disait-il, c'est lui qui est le Christ.
\VS{4}Quelques-uns d'entre eux crurent, et se joignirent à Paul et à Silas, ainsi qu'une grande multitude de Grecs craignant Dieu, et beaucoup de femmes de qualité.
\TextTitle{Emeute à Thessalonique}
\VS{5}Mais les Juifs rebelles et jaloux, prirent avec eux quelques hommes méchants et fainéants de la populace, provoquèrent des attroupements, et répandirent l'agitation dans la ville. Ils se rendirent à la maison de Jason, et ils cherchèrent Paul et Silas, pour les amener vers le peuple.
\VS{6}Ne les ayant pas trouvés, ils traînèrent Jason et quelques frères devant les magistrats de la ville, en criant~: Ces gens, qui ont bouleversé le monde, sont aussi venus ici, et Jason les a reçus chez lui.
\VS{7}Ils sont tous rebelles aux édits de César, disant qu'il y a un autre Roi, qu'ils nomment Jésus.
\VS{8}Ils soulevèrent donc le peuple et les magistrats de la ville, qui, entendant ces choses,
\VS{9}ne laissèrent aller Jason et les autres qu'après avoir obtenu d'eux une caution. 
\TextTitle{Paul et Silas fuient à Bérée}
\VS{10}Aussitôt les frères firent partir de nuit Paul et Silas pour Bérée. Lorsqu'ils furent arrivés, ils entrèrent dans la synagogue des Juifs.
\VS{11}Ces Juifs avaient des sentiments plus nobles que ceux de Thessalonique~; ils reçurent la parole avec beaucoup de promptitude, et ils examinaient tous les jours les Ecritures, pour voir si ce qu'on leur disait était exact.
\VS{12}Plusieurs d'entre eux crurent, ainsi que des femmes grecques de distinction, et des hommes en assez grand nombre.
\VS{13}Mais quand les Juifs de Thessalonique surent que Paul annonçait aussi à Bérée la parole de Dieu, ils vinrent y agiter la foule.
\VS{14}Alors les frères firent aussitôt partir Paul du côté de la mer~; Silas et Timothée restèrent à Bérée.
\VS{15}Ceux qui avaient pris la charge de mettre Paul en sûreté, le conduisirent jusqu'à Athènes. Puis ils s'en retournèrent, après avoir reçu l'ordre de Paul de dire à Silas et à Timothée de le rejoindre au plus tôt.
\TextTitle{Paul à Athènes}
\VS{16}Comme Paul les attendait à Athènes, il sentit au-dedans son esprit s'irriter à la vue de cette ville entièrement adonnée à l'idolâtrie.
\VS{17}Il s'entretenait donc dans la synagogue avec les Juifs et les hommes craignant Dieu, et tous les jours sur la place publique avec ceux qui s'y rencontraient.
\VS{18}Quelques philosophes épicuriens\FTNT{L'épicurisme a été fondé par Epicure (341 av. J.-C. - 270 av. J.-C.). Cette philosophie est axée sur la recherche du bonheur par l'évitement de la souffrance et des inquiétudes (ataraxie).} et stoïciens\FTNT{Les stoïciens étaient disciples de Zénon (336-264 av. J.-C.). Leur philosophie se fondait sur la conception d'un homme se suffisant à lui-même, sur une discipline rigoureuse, et sur la solidarité du genre humain.} se mirent à parler avec lui. Et les uns disaient~: Que veut dire ce discoureur~? Les autres disaient~: Il semble qu'il annonce des divinités étrangères. Parce qu'il leur annonçait Jésus et la résurrection.
\VS{19}Alors ils le prirent et le menèrent à l'Aréopage\FTNT{A l'origine, l'aréopage désignait le tribunal d'Athènes qui siégeait sur la colline d'Arès. Le sens figuré est le suivant~: Assemblée de juges, de savants, d'hommes de lettres très compétents.}, et lui dirent~: Pourrions-nous savoir quelle est cette nouvelle doctrine que tu enseignes~?
\VS{20}Car tu nous remplis les oreilles de certaines choses étranges~; nous voudrions donc savoir ce que veulent dire ces choses.
\VS{21}Or tous les Athéniens et les étrangers qui demeuraient à Athènes, ne passaient leur temps qu'à dire ou à écouter des nouvelles.
\TextTitle{Prédication de Paul à l'Aréopage}
\VS{22}Paul, debout au milieu de l'Aréopage, leur dit~: Hommes Athéniens, je vous trouve à tous égards extrêmement religieux.
\VS{23}Car en passant et en regardant vos divinités, j'ai même trouvé un autel sur lequel était écrit~: Au Dieu inconnu~! Celui que vous révérez sans le connaître, c'est celui que je vous annonce.
\VS{24}Le Dieu qui a fait le monde et tout ce qui s'y trouve, étant le Seigneur du ciel et de la terre, n'habite point dans des temples faits de main d'homme.
\VS{25}Il n'est point servi par les mains des hommes, comme s'il avait besoin de quoi que ce soit, lui qui donne à tous la vie, la respiration, et toutes choses.
\VS{26}Il a fait que tous les hommes, sortis d'un seul sang, habitent sur toute l'étendue de la terre, ayant déterminé la durée des temps et les bornes de leur habitation.
\VS{27}Il a voulu qu'ils cherchent le Seigneur, et qu'ils s'efforcent de le trouver en tâtonnant, quoiqu'il ne soit pas loin de chacun de nous,
\VS{28}car c'est par lui que nous avons la vie, le mouvement et l'être. C'est ce qu'ont dit quelques-uns même de vos poètes~: De lui nous sommes la race.
\VS{29}Ainsi donc, étant de la race de Dieu, nous ne devons pas croire que la divinité soit semblable à de l'or, ou à de l'argent, ou à de la pierre taillée par l'art et l'industrie des hommes.
\VS{30}Mais Dieu, sans tenir compte des temps d'ignorance, annonce maintenant à tous les hommes en tous lieux qu'ils se repentent,
\VS{31}parce qu'il a arrêté un jour où il jugera le monde selon la justice, par l'homme qu'il a établi pour cela, ce dont il a donné à tous une preuve certaine, en le ressuscitant des morts.
\VS{32}Lorsqu'ils entendirent parler de la résurrection des morts, les uns se moquèrent, et les autres dirent~: Nous t'entendrons là-dessus une autre fois.
\VS{33}Ainsi Paul se retira du milieu d'eux.
\VS{34}Quelques-uns néanmoins se joignirent à lui et crurent~: Denys, juge de l'Aéropage, une femme nommée Damaris, et d'autres avec eux.
\Chap{18}
\TextTitle{Paul enseigne à Corinthe pendant un an et demi}
\VerseOne{}Après cela, Paul partit d'Athènes, et se rendit à Corinthe.
\VS{2}Il y trouva un Juif, nommé Aquilas, originaire du Pont, récemment arrivé d'Italie, avec Priscille, sa femme, parce que Claude avait ordonné à tous les Juifs de sortir de Rome. Il s'approcha d'eux,
\VS{3}et comme il était du même métier qu'eux, il demeura chez eux et y travailla. Et leur métier était de faire des tentes.
\VS{4}Paul discourait dans la synagogue chaque sabbat, et il persuadait des Juifs et des Grecs.
\VS{5}Quand Silas et Timothée furent arrivés de Macédoine, Paul étant poussé par l'Esprit, rendait témoignage aux Juifs que Jésus était le Christ.
\VS{6}Mais comme ils s'opposaient à lui et qu'ils blasphémaient, il secoua ses vêtements, et leur dit~: Que votre sang retombe sur votre tête~! J'en suis pur~! Dès maintenant, j'irai vers les Gentils.
\VS{7}Et sortant de là, il entra dans la maison d'un homme appelé Justus, homme craignant Dieu, et dont la maison était contiguë à la synagogue.
\VS{8}Cependant Crispus, le chef de la synagogue, crut au Seigneur avec toute sa famille. Et plusieurs Corinthiens qui avaient entendu Paul, crurent aussi, et ils furent baptisés.
\VS{9}Le Seigneur dit à Paul dans une vision pendant la nuit~: Ne crains point, mais parle et ne te tais point,
\VS{10}parce que je suis avec toi, et personne ne mettra la main sur toi pour te faire du mal. Parle, car j'ai un peuple nombreux dans cette ville.
\VS{11}Il y demeura un an et six mois, enseignant parmi eux la parole de Dieu.
\TextTitle{Soulèvement des Juifs contre Paul}
\VS{12}Pendant que Gallion était proconsul de l'Achaïe, les Juifs se soulevèrent d'un commun accord contre Paul, et le menèrent devant le tribunal,
\VS{13}en disant~: Cet homme incite les gens à servir Dieu d'une manière contraire à la loi.
\VS{14}Et comme Paul voulait ouvrir la bouche pour parler, Gallion dit aux Juifs~: Ô Juifs~! S'il s'agissait de quelque injustice, ou de quelque crime, je vous écouterais patiemment, autant qu'il serait raisonnable.
\VS{15}Mais il s'agit de discussions sur une parole, sur des noms, et sur votre loi, vous y mettrez de l'ordre vous-mêmes, car je ne veux pas être juge de ces choses.
\VS{16}Et il les renvoya du tribunal.
\VS{17}Alors tous les Grecs se saisirent de Sosthène, le chef de la synagogue, le battirent devant le tribunal, sans que Gallion s'en mît en peine.
\TextTitle{Paul fait un vœu\FTNTT{\vref{Ga. 3:23-28}~; \vref{2 Co. 3:7-14}~; \vref{Ro. 6:14}}}
\VS{18}Paul resta encore assez longtemps à Corinthe. Ensuite il prit congé des frères et s'embarqua pour la Syrie, avec Priscille et Aquilas, après s'être fait raser la tête à Cenchrées, car il avait fait un vœu.
\VS{19}Ils arrivèrent à Ephèse, et Paul y laissa ses compagnons. Etant entré dans la synagogue, il s'entretint avec les Juifs,
\VS{20}qui le prièrent de rester encore plus longtemps avec eux.
\VS{21}Mais il n'y consentit point, et il prit congé d'eux en leur disant~: Il faut absolument que je célèbre la fête prochaine à Jérusalem. Je reviendrai vers vous, s'il plaît à Dieu. Ainsi il partit d'Ephèse.
\VS{22}Etant débarqué à Césarée, il monta à Jérusalem, et après avoir salué l'église, il descendit à Antioche.
\VS{23}Et ayant séjourné là quelque temps, il s'en alla, et traversa tout de suite la contrée de Galatie et de Phrygie, fortifiant tous les disciples\FTNT{Voir annexe «~Les voyages missionnaires de Paul~»}.
\TextTitle{Apollos annonce l'Evangile à Ephèse et à Corinthe}
\VS{24}En ce temps-là, un Juif, nommé Apollos, originaire d'Alexandrie, homme éloquent et puissant dans les Ecritures, vint à Ephèse.
\VS{25}Il était en quelque sorte instruit dans la voie du Seigneur, et fervent d'esprit~; il expliquait et enseignait avec exactitude ce qui concerne Jésus, bien qu'il ne connaisse que le baptême de Jean.
\VS{26}Il commança donc à parler avec hardiesse dans la synagogue~; et quand Aquilas et Priscille l'eurent entendu, ils le prirent avec eux, et lui exposèrent plus exactement la voie de Dieu.
\VS{27}Et comme il voulut passer en Achaïe, les frères, qui l'y encouragèrent écrivirent aux disciples de bien le recevoir. Quand il fut arrivé, il aida beaucoup ceux qui avaient cru par la grâce.
\VS{28}Car il réfutait publiquement les Juifs avec une grande véhémence, démontrant par les Ecritures que Jésus était le Christ.
\Chap{19}
\TextTitle{Paul enseigne à Ephèse\FTNTT{v. \vref{9-10}~; \vref{Ac. 20:31}}}
\VerseOne{}Pendant qu'Apollos était à Corinthe, Paul, après avoir parcouru toutes les hautes provinces de l'Asie, arriva à Ephèse. Ayant rencontré quelques disciples, il leur dit~:
\VS{2}Avez-vous reçu le Saint-Esprit quand vous avez cru~? Ils lui répondirent~: Nous n'avons même pas entendu dire qu'il y ait un Saint-Esprit.
\VS{3}Et il leur dit~: De quel baptême donc avez-vous été baptisés~? Ils répondirent~: Du baptême de Jean.
\VS{4}Alors Paul dit~: Il est vrai que Jean a baptisé du baptême de repentance, disant au peuple de croire en celui qui venait après lui, c'est-à-dire en Jésus-Christ.
\VS{5}Après avoir entendu ces choses, ils furent baptisés au Nom du Seigneur Jésus.
\VS{6}Lorsque Paul leur eut imposé les mains, le Saint-Esprit descendit sur eux, et ils parlaient diverses langues et prophétisaient.
\VS{7}Ils étaient en tout environ douze hommes.
\VS{8}Ensuite, Paul entra dans la synagogue où il parla librement. Pendant trois mois, il discourut sur les choses qui concernent le Royaume de Dieu avec persuasion.
\VS{9}Mais comme quelques-uns restaient endurcis et rebelles, décriant devant la multitude la voie du Seigneur, il se retira d'eux, sépara les disciples, et enseigna tous les jours dans l'école d'un nommé Tyrannus.
\VS{10}Cela dura deux ans, de sorte que tous ceux qui habitaient l'Asie, Juifs et Grecs, entendirent la parole du Seigneur Jésus.
\TextTitle{Réveil et prodiges à Ephèse}
\VS{11}Et Dieu faisait des prodiges extraordinaires par les mains de Paul,
\VS{12}au point qu'on appliquait sur les malades des mouchoirs ou des linges qui avaient touché son corps, et ils étaient guéris de leurs maladies, et les esprits malins sortaient.
\VS{13}Alors quelques exorcistes Juifs ambulants essayèrent d'invoquer le Nom du Seigneur Jésus sur ceux qui étaient possédés d'esprits malins, en disant~: Nous vous conjurons par ce Jésus que Paul prêche~!
\VS{14}Ceux qui faisaient cela étaient sept fils de Scéva, un homme Juif, l'un des principaux prêtres.
\VS{15}Mais l'esprit malin leur répondit~: Je connais Jésus, et je sais qui est Paul~; mais vous, qui êtes-vous~?
\VS{16}Et l'homme dans lequel était l'esprit malin se jeta sur eux, se rendit maître de deux d'entre eux, et les maltraita de telle sorte qu'ils s'enfuirent de cette maison nus et blessés.
\VS{17}Cela fut connu de tous les Juifs et de tous les Grecs qui demeuraient à Ephèse~; et ils furent tous saisis de crainte, et le Nom du Seigneur Jésus était glorifié.
\VS{18}Plusieurs de ceux qui avaient cru venaient, confessant et déclarant ce qu'ils avaient fait.
\VS{19}Et un grand nombre de ceux qui s'étaient adonnés à des pratiques magiques, apportèrent leurs livres et les brûlèrent devant tous. On en estima la valeur à cinquante mille pièces d'argent.
\VS{20}Ainsi la parole du Seigneur se répandait sensiblement, et produisait de grands effets.
\VS{21}Après que ces choses se furent passées, Paul se proposa par un mouvement de l'Esprit\FTNT{Paul fut conduit par le Saint-Esprit (\vref{Jn. 3:8}).} d'aller à Jérusalem, en traversant la Macédoine et l'Achaïe. Quand j'y serai allé, se disait-il, il faut aussi que je voie Rome.
\VS{22}Il envoya en Macédoine deux de ceux qui l'assistaient, Timothée et Eraste, et il resta lui-même quelque temps en Asie.
\TextTitle{Emeute suscitée par Démétrius}
\VS{23}Mais en ce temps-là il arriva un grand trouble, à cause de la doctrine.
\VS{24}Car un certain homme, nommé Démétrius, orfèvre, fabriquait de petits temples d'argent de Diane, et apportait beaucoup de profit aux ouvriers du métier.
\VS{25}Il les rassembla, avec ceux du même métier, et dit~: Ô hommes, vous savez que tout notre gain vient de cet ouvrage,
\VS{26}et vous voyez et entendez que, non seulement à Ephèse, mais dans presque toute l'Asie, ce Paul par ses persuasions a détourné beaucoup de monde, en disant que les dieux faits de main d'homme ne sont pas des dieux.
\VS{27}Et il n'y a pas seulement à craindre pour nous que notre métier ne soit décrié, mais même que le temple de la grande Diane ne tombe dans le mépris, et que sa majesté, que toute l'Asie et que le monde entier révère, ne soit anéantie.
\VS{28}Ayant entendu ces choses, ils furent tous remplis de colère, et s'écrièrent, disant~: Grande est la Diane des Ephésiens~!
\VS{29}Et toute la ville fut remplie de confusion~; et ils se jetèrent en foule dans le théâtre, et enlevèrent Gaïus et Aristarque Macédoniens, compagnons de voyage de Paul.
\VS{30}Et comme Paul voulait entrer vers le peuple, les disciples ne le lui permirent point.
\VS{31}Quelques-uns même des Asiarques, qui étaient ses amis, envoyèrent quelqu'un vers lui pour le prier de ne pas se présenter au théâtre.
\VS{32}Les uns criaient d'une manière, les autres d'une autre, car l'assemblée était confuse, et la plupart ne savaient pas pourquoi ils s'étaient assemblés.
\VS{33}Alors Alexandre fut contraint de sortir hors de la foule, les Juifs le poussant en avant~; et Alexandre, faisant signe de la main, voulait présenter quelque excuse au peuple.
\VS{34}Mais quand ils reconnurent qu'il était Juif, tous d'une seule voix crièrent pendant deux heures~: Grande est la Diane des Ephésiens~!
\VS{35}Cependant, le secrétaire de la ville, ayant apaisé la foule, dit~: Hommes éphésiens, quel est celui des hommes qui ignore que la ville d'Ephèse est la gardienne de la grande déesse Diane et de son image tombée de Jupiter\FTNT{Tombée de Jupiter~: C'est-à-dire du ciel.}~?
\VS{36}Cela étant donc incontestable, vous devez vous apaiser et ne rien faire avec précipitation.
\VS{37}Car ces gens que vous avez amenés ne sont ni sacrilèges ni blasphémateurs de votre déesse.
\VS{38}Mais si Démétrius et ses ouvriers ont à se plaindre de quelqu'un, il y a des jours d'audience et des proconsuls~; qu'ils s'appellent en justice les uns les autres~!
\VS{39}Et si vous avez quelque autre chose à réclamer, on pourra en décider dans une assemblée légale.
\VS{40}Car nous risquons d'être accusés de sédition pour ce qui s'est passé aujourd'hui, n'ayant aucune raison pour justifier ce rassemblement. Après ces paroles, il congédia l'assemblée.
\Chap{20}
\TextTitle{Paul annonce l'Evangile en Macédoine et en Grèce}
\VerseOne{}Lorsque le tumulte eut cessé, Paul fit venir les disciples, et après les avoir embrassés, il partit pour aller en Macédoine.
\VS{2}Il parcourut cette contrée, en adressant aux disciples de nombreuses exhortations.
\VS{3}Puis il se rendit en Grèce où il séjourna trois mois. Il était sur le point de s'embarquer pour la Syrie, quand les Juifs lui dressèrent des embûches. Alors il se décida à reprendre la route de la Macédoine.
\VS{4}Il avait pour l'accompagner jusqu'en Asie~: Sopater de Bérée, Aristarque et Second de Thessalonique, Gaïus de Derbe, Timothée, ainsi que Tychique et Trophime, originaires d'Asie.
\VS{5}Ceux-ci prirent les devants et nous attendirent à Troas.
\TextTitle{Paul ressuscite un jeune homme à Troas}
\VS{6}Et nous, ayant levé l'ancre à Philippes, après les jours des pains sans levain, nous arrivâmes au bout de cinq jours auprès d'eux à Troas, et nous y séjournâmes sept jours.
\VS{7}Le premier jour de la semaine, les disciples étant assemblés pour rompre le pain, Paul, qui devait partir le lendemain, leur fit un discours qu'il étendit jusqu'à minuit.
\VS{8}Or il y avait beaucoup de lampes dans la chambre haute où ils étaient assemblés.
\VS{9}Et un jeune homme nommé Eutychus, qui était assis sur une fenêtre, s'endormit profondément pendant le long discours de Paul~; entraîné par le sommeil, il tomba du troisième étage en bas, et quand on voulut le relever, il était mort.
\VS{10}Mais Paul, étant descendu, se pencha sur lui, le prit dans ses bras, et dit~: Ne vous troublez pas, car son âme est en lui.
\VS{11}Quand il fut remonté, il rompit le pain et mangea, et il parla longtemps encore jusqu'au jour. Après quoi il partit.
\VS{12}Ils ramenèrent le jeune homme vivant, et ce fut le sujet d'une grande consolation.
\TextTitle{Passage à Milet}
\VS{13}Pour nous, étant montés sur un navire, nous fîmes voile vers Assos, où nous avions convenu de reprendre Paul, parce qu'il devait faire la route à pied.
\VS{14}Lorsqu'il nous eut rejoints à Assos, nous le prîmes avec nous, et nous allâmes à Mytilène.
\VS{15}Puis étant partis de là, le jour suivant nous abordâmes vis-à-vis de Chios. Le lendemain, nous arrivâmes vers Samos, et nous nous arrêtâmes à Trogyle~; le jour d'après, nous vînmes à Milet.
\VS{16}Car Paul avait résolu de passer devant Ephèse sans s'y arrêter, afin de ne pas perdre de temps en Asie~; parce qu'il se hâtait pour être, si cela lui était possible, à Jérusalem le jour de la Pentecôte.
\TextTitle{Paul exhorte et prend congé des anciens d'Ephèse}
\VS{17}Cependant de Milet, il envoya chercher à Ephèse les anciens de l'Eglise.
\VS{18}Lorsqu'ils furent arrivés vers lui, il leur dit~: Vous savez de quelle manière je me suis toujours conduit avec vous dès le premier jour où je suis entré en Asie~;
\VS{19}servant le Seigneur en toute humilité, avec beaucoup de larmes, et au milieu des épreuves que me suscitaient les embûches des Juifs.
\VS{20}Vous savez que je n'ai rien caché de ce qui vous était utile, et que je n'ai pas craint de vous prêcher et de vous enseigner publiquement et dans les maisons,
\VS{21}prêchant tant aux Juifs qu'aux Grecs la repentance envers Dieu, et la foi en Jésus-Christ, notre Seigneur.
\VS{22}Et maintenant voici, étant lié par l'Esprit, je vais à Jérusalem, ignorant ce qui m'y arrivera~;
\VS{23}seulement, de ville en ville le Saint-Esprit m'avertit que des liens et des tribulations m'attendent.
\VS{24}Mais je ne fais pour moi-même aucun cas de ma vie, comme si elle m'était précieuse, pourvu que j'achève ma course avec joie, et le service que j'ai reçu du Seigneur Jésus, pour rendre témoignage à l'Evangile de la grâce de Dieu.
\VS{25}Et maintenant voici, je sais que vous ne verrez plus mon visage, vous tous au milieu desquels j'ai passé en prêchant le Royaume de Dieu.
\VS{26}C'est pourquoi je vous prends aujourd'hui à témoin que je suis net du sang de tous.
\VS{27}Car je vous ai annoncé tout le conseil de Dieu, sans en rien cacher.
\VS{28}Prenez donc garde à vous-mêmes, et à tout le troupeau sur lequel le Saint-Esprit vous a établis évêques\FTNT{Evêque, «~episcopos~» en grec~: surveillant, gardien. Ce terme désigne la fonction des anciens. Dans la Nouvelle Alliance, les évêques (ou anciens) sont des personnes dont la mission est de veiller au bon fonctionnement des assemblées locales. Jésus-Christ, notre Dieu, est l'Evêque par excellence (\vref{1 Pi. 2:25}).}, pour paître l'Eglise de Dieu, qu'il a acquise par son propre sang.
\VS{29}Car je sais qu'après mon départ, il s'introduira parmi vous des loups très dangereux, qui n'épargneront pas le troupeau,
\VS{30}et qu'il se lèvera du milieu de vous des hommes qui enseigneront des doctrines corrompues dans le but d'attirer les disciples après eux.
\VS{31}C'est pourquoi veillez, vous souvenant que durant l'espace de trois ans, je n'ai cessé nuit et jour d'avertir chacun de vous avec larmes.
\VS{32}Et maintenant, mes frères, je vous recommande à Dieu, et à la parole de sa grâce, à celui qui est puissant pour achever de vous édifier, et pour vous donner l'héritage avec tous les saints.
\VS{33}Je n'ai désiré ni l'argent, ni l'or, ni les vêtements de personne.
\VS{34}Et vous savez vous-mêmes que ces mains ont pourvu à mes besoins et à ceux des personnes qui étaient avec moi.
\VS{35}Je vous ai montré de toutes manières que c'est en travaillant ainsi qu'il faut soutenir les faibles, et se rappeler les paroles du Seigneur Jésus, qui a dit lui-même~: Il y a plus de bénédiction à donner qu'à recevoir\FTNT{\vref{Lu. 14:12}.}.
\VS{36}Après avoir ainsi parlé, il se mit à genoux et il pria avec eux tous.
\VS{37}Alors tous fondirent en larmes, et se jetant au cou de Paul,
\VS{38}ils l'embrassèrent, étant principalement affligés de ce qu'il avait dit qu'ils ne verraient plus son visage. Et ils l'accompagnèrent jusqu'au navire.
\Chap{21}
\TextTitle{L'équipe missionnaire à Tyr~; avertissement de l'Esprit}
\VerseOne{}Nous nous embarquâmes, après nous être séparés d'eux, et nous allâmes directement à Cos, et le jour suivant à Rhodes, et de là à Patara.
\VS{2}Et ayant trouvé un navire qui faisait la traversée vers la Phénicie, nous montâmes et partîmes.
\VS{3}Puis ayant découvert l'île de Chypre, nous la laissâmes à gauche, nous fîmes route vers la Syrie, nous arrivâmes à Tyr, car le navire devait y décharger sa cargaison.
\VS{4}Nous trouvâmes les disciples et nous restâmes là sept jours. Les disciples, poussés par l'Esprit, disaient à Paul de ne pas monter à Jérusalem.
\VS{5}Mais ces jours étant passés, nous partîmes et nous nous acheminâmes pour partir de Tyr, et tous nous accompagnèrent avec leurs femmes et leurs enfants, jusqu'à l'extérieur de la ville. Nous nous mîmes à genoux sur le rivage et nous fîmes la prière.
\VS{6}Et après nous être embrassés les uns les autres, nous montâmes sur le navire, et les autres retournèrent chez eux.
\TextTitle{Escales à Ptolémaïs puis à Césarée~; prophétie d'Agabus}
\VS{7}Et ainsi achevant notre navigation, nous allâmes de Tyr à Ptolémaïs~; et après avoir salué les frères, nous passâmes un jour avec eux.
\VS{8}Nous partîmes le lendemain, et nous arrivâmes à Césarée. Etant entrés dans la maison de Philippe, l'évangéliste, qui était l'un des sept, nous restâmes chez lui.
\VS{9}Il avait quatre filles vierges qui prophétisaient.
\VS{10}Comme nous étions là depuis plusieurs jours, un prophète, nommé Agabus, arriva de Judée
\VS{11}et vint nous trouver. Il prit la ceinture de Paul, se lia les mains et les pieds, et il dit~: Voici ce que déclare le Saint-Esprit~: L'homme à qui appartient cette ceinture, les Juifs le lieront de la même manière à Jérusalem, et le livreront entre les mains des Gentils.
\VS{12}Quand nous entendîmes ces choses, nous et ceux de l'endroit, nous priâmes Paul de ne pas monter à Jérusalem.
\VS{13}Mais Paul répondit~: Que faites-vous en pleurant et en affligeant mon cœur~? Je suis prêt, non seulement à être lié, mais aussi à mourir à Jérusalem pour le Nom du Seigneur Jésus.
\VS{14}Comme il ne se laissait pas persuader, nous n'insistâmes pas, et nous dîmes~: Que la volonté du Seigneur soit faite~!
\TextTitle{QUATRIEME VOYAGE~: DE JERUSALEM A ROME}
\TextTitle{Arrivée à Jérusalem~; accueil des anciens}
\VS{15}Quelques jours après, nous fîmes nos préparatifs et nous montâmes à Jérusalem \FTNT{Voir annexe «~Les voyages missionnaires de Paul~»}.
\VS{16}Quelques disciples de Césarée vinrent avec nous, amenant avec eux un homme appelé Mnason, de l'île de Chypre, disciple de longue date, chez qui nous devions loger.
\VS{17}Lorsque nous arrivâmes à Jérusalem, les frères nous reçurent avec joie.
\VS{18}Et le jour suivant, Paul se rendit avec nous chez Jacques, et tous les anciens s'y réunirent.
\VS{19}Après les avoir embrassés, il raconta en détail les choses que Dieu avait faites au milieu des Gentils par son service.
\VS{20}Quand ils l'eurent entendu, ils glorifièrent le Seigneur. Puis ils dirent à Paul~: Tu vois frère, combien de milliers de Juifs ont cru~; mais ils sont tous zélés pour la loi.
\VS{21}Or ils ont appris que tu enseignes à tous les Juifs qui sont parmi les Gentils, à renoncer à Moïse, en leur disant qu'ils ne doivent pas circoncire leurs enfants et de ne pas vivre selon les ordonnances de la loi.
\VS{22}Que faut-il donc faire~? Il faut absolument rassembler la multitude des fidèles, car ils apprendront que tu es venu.
\VS{23}C'est pourquoi fais ce que nous allons te dire~: Nous avons quatre hommes qui ont fait un vœu,
\VS{24}prends-les avec toi, purifie-toi avec eux, et pourvois à leurs besoins afin qu'ils se rasent la tête. Et ainsi tous sauront que ce qu'ils ont entendu sur ton compte est faux, mais que toi aussi tu te conduis en observateur de la loi.
\VS{25}A l'égard des Gentils qui ont cru, nous avons décidé et nous leur avons écrit qu'ils doivent s'abstenir des viandes sacrifiées aux idoles, du sang, des animaux étouffés, et de la débauche.
\VS{26}Alors Paul prit ces hommes, se purifia, et entra le lendemain dans le temple avec eux, pour annoncer quel jour leur purification devait s'achever, et quand l'offrande devait être présentée pour chacun d'eux.
\TextTitle{Paul chassé du temple et brutalisé par les Juifs}
\VS{27}A la fin des sept jours, les Juifs d'Asie ayant vu Paul dans le temple, soulevèrent tout le peuple, et mirent la main sur lui,
\VS{28}en criant~: Hommes israélites, au secours~! Voici l'homme qui prêche partout et à tout le monde contre le peuple, contre la loi, et contre ce lieu. Il a même introduit des Grecs dans le temple, et a profané ce saint lieu.
\VS{29}Car ils avaient vu auparavant Trophime d'Ephèse avec lui dans la ville, et ils croyaient que Paul l'avait fait entrer dans le temple.
\VS{30}Toute la ville fut émue, et le peuple accourut de toutes parts. Ils se saisirent de Paul et le traînèrent hors du temple, dont les portes furent aussitôt fermées.
\TextTitle{Intervention des soldats et des centeniers}
\VS{31}Comme ils cherchaient à le tuer, le bruit vint au tribun de la cohorte que tout Jérusalem était en trouble.
\VS{32}A l'instant, il prit des soldats et des centeniers, et courut vers eux. Voyant le tribun et les soldats, ils cessèrent de frapper Paul.
\VS{33}Alors le tribun s'approcha, se saisit de Paul, et le fit lier de deux chaînes. Puis il demanda qui il était et ce qu'il avait fait.
\VS{34}Les uns criaient d'une manière, et les autres d'une autre, dans la foule. Ne pouvant donc rien apprendre de certain à cause du tumulte, il ordonna de mener Paul dans la forteresse.
\VS{35}Lorsque Paul fut sur les degrés, il dut être porté par les soldats, à cause de la violence de la foule~;
\VS{36}car la multitude du peuple le suivait, en criant~: Fais-le mourir~!
\VS{37}Comme on allait faire entrer Paul dans la forteresse, il dit au tribun~: M'est-il permis de te dire quelque chose~? Et le tribun répondit~: Tu sais parler le grec~?
\VS{38}Tu n'es donc pas cet Egyptien qui a excité une sédition dernièrement, et qui a emmené dans le désert quatre mille brigands~?
\VS{39}Paul lui dit~: Je suis Juif de Tarse, citoyen de la ville renommée de la Cilicie. Permets-moi, je te prie, de parler au peuple.
\VS{40}Et quand il le lui permit, Paul se tenant sur les degrés fit signe de la main au peuple, et s'étant fait un grand silence, il leur parla en langue hébraïque, disant~:
\Chap{22}
\TextTitle{Paul raconte son témoignage de conversion\FTNTT{\vref{Ac. 9:1-18}~; \vref{26:9-18}.}}
\VerseOne{}Hommes frères et pères, écoutez ce que j'ai maintenant à vous dire pour ma défense~!
\VS{2}Lorsqu'ils entendirent qu'il leur parlait en langue hébraïque, ils redoublèrent de silence. Et Paul leur dit~:
\VS{3}Je suis Juif, né à Tarse en Cilicie~; mais j'ai été élevé dans cette ville-ci aux pieds de Gamaliel et instruit dans la connaissance exacte de la loi de nos pères, étant plein de zèle pour la loi de Dieu, comme vous l'êtes tous aujourd'hui.
\VS{4}J'ai persécuté à mort cette doctrine, liant et mettant en prison hommes et femmes.
\VS{5}Le grand-prêtre lui-même et toute l'assemblée des anciens m'en sont témoins. J'ai même reçu d'eux des lettres pour les frères de Damas, où je me rendis afin d'amener liés à Jérusalem ceux qui se trouvaient là et de les faire punir.
\VS{6}Or il arriva comme j'étais en chemin, que j'approchais de Damas, tout à coup, vers midi, une grande lumière venant du ciel resplendit comme un éclair autour de moi.
\VS{7}Je tombai par terre, et j'entendis une voix qui me dit~: Saul, Saul, pourquoi me persécutes-tu~?
\VS{8}Je répondis~: Qui es-tu Seigneur~? Et il me dit~: Je suis Jésus de Nazareth, que tu persécutes.
\VS{9}Ceux qui étaient avec moi furent tout effrayés, ils virent bien la lumière, mais ils ne comprirent pas la voix de celui qui me parlait. Alors je dis~: Que ferai-je Seigneur~?
\VS{10}Et le Seigneur me dit~: Lève-toi, va à Damas, et là on te dira tout ce que tu dois faire.
\VS{11}Comme je ne voyais rien, à cause de l'éclat de cette lumière, ceux qui étaient avec moi me prirent par la main, et j'arrivai à Damas.
\VS{12}Or un nommé Ananias, homme pieux selon la loi, et de qui tous les Juifs demeurant à Damas rendaient un bon témoignage, vint me trouver
\VS{13}et me dit~: Saul, mon frère, recouvre la vue. Au même instant, je recouvrai la vue et je le regardai.
\VS{14}Et il me dit~: Le Dieu de nos pères t'a destiné à connaître sa volonté, à voir le Juste, et à entendre les paroles de sa bouche.
\VS{15}Car tu lui serviras de témoin auprès de tous les hommes, des choses que tu as vues et entendues.
\VS{16}Et maintenant, pourquoi tardes-tu~? Lève-toi, et sois baptisé et purifié de tes péchés, en invoquant le Nom du Seigneur.
\TextTitle{Le Seigneur appelle Paul à quitter Jérusalem et l'envoie dans les nations}
\VS{17}Or il arriva qu'après que je sois retourné à Jérusalem, comme je priais dans le temple, je fus ravi en extase,
\VS{18}et je vis le Seigneur qui me disait~: Hâte-toi, et sors promptement de Jérusalem, parce qu'ils ne recevront pas le témoignage que tu leur rendras de moi.
\VS{19}Et je dis~: Seigneur, ils savent eux-mêmes que je faisais mettre en prison et battre de verges dans les synagogues ceux qui croyaient en toi~;
\VS{20}et que lorsque le sang d'Etienne, ton martyr, fut répandu, j'étais moi-même présent, je consentais à sa mort, et je gardais les vêtements de ceux qui le faisaient mourir.
\VS{21}Alors il me dit~: Va, car je t'enverrai au loin vers les Gentils.
\TextTitle{Les Juifs demandent la mort de Paul}
\VS{22}Et ils l'écoutèrent jusqu'à cette parole~; mais alors ils élevèrent leur voix, en disant~: Ote de la terre un tel homme~! Car il n'est pas concevable qu'il vive.
\VS{23}Et comme ils criaient à haute voix, secouaient leurs vêtements, et jetaient de la poussière en l'air,
\VS{24}le tribun commanda de faire entrer Paul dans la forteresse, et de lui donner la question par le fouet, afin de savoir pour quel sujet ils criaient ainsi contre lui.
\TextTitle{Paul revendique ses droits de citoyen romain}
\VS{25}Comme on l'attachait pour le frapper, Paul dit au centenier qui était près de lui~: Vous est-il permis de fouetter un homme romain, et qui n'est même pas condamné~?
\VS{26}A ces mots, le centenier alla vers le tribun pour l'avertir, disant~: Prends garde à ce que tu feras, car cet homme est Romain.
\VS{27}Et le tribun, étant venu, dit à Paul~: Dis-moi, es-tu Romain~? Et il répondit~: Oui, je le suis.
\VS{28}Le tribun lui dit~: J'ai acquis ce droit de citoyen pour une grande somme d'argent. Et moi, dit Paul, je l'ai par ma naissance.
\VS{29}Aussitôt, ceux qui devaient lui donner la question se retirèrent, et le tribun, voyant que Paul était Romain, fut dans la crainte parce qu'il l'avait fait lier.
\TextTitle{Paul devant le sanhédrin}
\VS{30}Le lendemain, voulant savoir avec certitude de quoi les Juifs l'accusaient, le tribun lui fit ôter ses liens, et donna l'ordre aux principaux prêtres et à tout le sanhédrin de se réunir~; puis, il fit descendre Paul, et il le présenta devant eux.
\Chap{23}
\VerseOne{}Paul regardant fixement le sanhédrin, dit~: Hommes frères~! Je me suis conduit en toute bonne conscience devant Dieu jusqu'à ce jour.
\VS{2}Le grand-prêtre Ananias ordonna à ceux qui étaient près de lui de le frapper sur la bouche.
\VS{3}Alors Paul lui dit~: Dieu te frappera, muraille blanchie~! Tu es assis pour me juger selon la loi, et tu violes la loi en ordonnant qu'on me frappe~!
\VS{4}Ceux qui étaient présents lui dirent~: Tu insultes le grand-prêtre de Dieu~?
\VS{5}Et Paul dit~: Je ne savais pas, mes frères, que c'était le grand-prêtre~; car il est écrit~: Tu ne parleras pas mal du chef de ton peuple.
\TextTitle{Dissensions entre pharisiens et sadducéens}
\VS{6}Paul, sachant qu'une partie de l'assemblée était composée de sadducéens et l'autre de pharisiens, s'écria dans le sanhédrin~: Hommes frères~! Je suis pharisien, fils de pharisien, c'est à cause de l'espérance et de la résurrection des morts que je suis mis en jugement.
\VS{7}Quand il eut dit cela, il s'éleva un débat entre les pharisiens et les sadducéens~; et l'assemblée se divisa.
\VS{8}Car les sadducéens disent qu'il n'y a point de résurrection, ni d'ange, ni d'esprit, mais les pharisiens soutiennent les deux choses.
\VS{9}Il y eut une grande clameur. Alors les scribes du parti des pharisiens se levèrent et contestèrent, disant~: Nous ne trouvons aucun mal en cet homme~; peut-être un esprit ou un ange lui a parlé, ne combattons point contre Dieu.
\VS{10}Et comme il se faisait une grande division, le tribun craignant que Paul ne soit mis en pièces par eux, ordonna que les soldats descendent, et qu'ils l'enlèvent du milieu d'eux, et l'amènent dans la forteresse.
\TextTitle{Le Seigneur fortifie Paul}
\VS{11}La nuit suivante, le Seigneur apparut à Paul et lui dit~: Prends courage~; car, de même que tu as rendu témoignage de moi dans Jérusalem, il faut aussi que tu rendes témoignage à Rome.
\TextTitle{Complot des Juifs pour tuer Paul}
\VS{12}Quand le jour fut venu, les Juifs formèrent un complot, et firent des imprécations contre eux-mêmes, en disant qu'ils ne mangeraient pas ni ne boiraient jusqu'à ce qu'ils aient tué Paul.
\VS{13}Ceux qui formèrent ce complot étaient plus de quarante,
\VS{14}et ils s'adressèrent aux principaux prêtres et aux anciens, et leur dirent~: Nous nous sommes engagés, avec des imprécations contre nous-mêmes, à ne rien manger jusqu'à ce que nous ayons tué Paul.
\VS{15}Vous donc, maintenant, adressez-vous, avec le sanhédrin, au tribun pour le faire descendre demain au milieu de vous, comme si vous vouliez examiner sa cause plus exactement~; et nous, avant qu'il approche, nous sommes tous prêts à le tuer.
\VS{16}Le fils de la sœur de Paul, ayant eu connaissance de ce complot, alla dans la forteresse et le rapporta à Paul.
\VS{17}Paul appela l'un des centeniers et lui dit~: Mène ce jeune homme au tribun, car il a quelque chose à lui rapporter.
\VS{18}Il le prit donc et le mena au tribun, et il lui dit~: Le prisonnier Paul m'a appelé et m'a prié de t'amener ce jeune homme qui a quelque chose à te dire.
\VS{19}Et le tribun le prenant par la main, se retira à part, et lui demanda~: Qu'est-ce que tu as à me rapporter~?
\VS{20}Et il lui dit~: Les Juifs ont conspiré de te prier que demain tu envoies Paul au sanhédrin, comme s'ils voulaient s'enquérir de lui plus exactement de quelque chose.
\VS{21}Mais n'y consens point, car plus de quarante hommes d'entre eux sont en embûches contre lui, qui ont fait un vœu avec exécration de serment, de ne manger ni boire jusqu'à ce qu'ils l'aient tué~; et ils sont maintenant tous prêts, attendant ce que tu leur permettras.
\VS{22}Le tribun donc renvoya le jeune homme, en lui recommandant de ne parler à personne de ce rapport qu'il lui avait fait.
\TextTitle{Paul transféré à Césarée}
\VS{23}Ensuite, il appela deux des centeniers, et il leur dit~: Tenez prêts, dès la troisième heure de la nuit, deux cents soldats, soixante-dix cavaliers, et deux cents archers, pour aller jusqu'à Césarée.
\VS{24}Et ayez soin qu'il y ait des montures prêtes, afin qu'ayant fait monter Paul, ils le mènent sûrement au gouverneur Félix. \FTNT{Marcus Antonuis Félix était procurateur de la province romaine de la Judée de 52 à 60 ap. J.-C.}.
\VS{25}Et il lui écrivit une lettre en ces termes~:
\VS{26}Claude Lysias au très excellent gouverneur Félix, salut~!
\VS{27}Les Juifs s'étaient saisis de cet homme et allaient le tuer, lorsque je survins avec des soldats et le leur enlevai, ayant appris qu'il était Romain.
\VS{28}Voulant connaître le motif pour lequel ils l'accusaient, je l'amenai devant leur sanhédrin.
\VS{29}J'ai trouvé qu'il était accusé au sujet de questions relatives à leur loi, mais qu'il n'avait commis aucun crime qui mérite la mort ou la prison.
\VS{30}Ayant été averti des embûches que les Juifs avaient dressées contre lui, je te l'ai aussitôt envoyé, en ordonnant à ses accusateurs de te dire eux-mêmes ce qu'ils ont contre lui. Adieu~!
\TextTitle{Paul arrive à Césarée}
\VS{31}Les soldats prirent Paul, selon l'ordre qu'ils avaient reçu, et le conduisirent pendant la nuit jusqu'à Antipatris.
\VS{32}Le lendemain, laissant les cavaliers poursuivre la route avec Paul, ils retournèrent à la forteresse.
\VS{33}Arrivés à Césarée, les cavaliers remirent la lettre au gouverneur, et lui présentèrent aussi Paul.
\VS{34}Le gouverneur, après avoir lu la lettre, demanda à Paul de quelle province il était. Ayant appris qu'il était de Cilicie~:
\VS{35}Je t'entendrai, lui dit-il, plus amplement quand tes accusateurs seront venus. Et il ordonna qu'il soit gardé dans le Prétoire d'Hérode.
\Chap{24}
\TextTitle{Paul devant le gouverneur Félix~; accusation des Juifs}
\VerseOne{}Or cinq jours après, Ananias le grand-prêtre descendit avec les anciens, et un certain orateur, nommé Tertulle, qui comparurent devant le gouverneur contre Paul. 
\VS{2}Et Paul étant appelé, Tertulle commença à l'accuser, en disant~:
\VS{3}Très excellent Félix, nous reconnaissons en toutes choses partout et avec une entière reconnaissance, que nous avons obtenu une grande tranquillité par ton moyen, et par les bons règlements que tu as faits pour ce peuple, selon ta prudence.
\VS{4}Mais afin de ne pas te retenir plus longtemps, je te prie de nous entendre, selon ton équité, dans ce que nous allons te dire en peu de paroles.
\VS{5}Nous avons trouvé cet homme, qui est une peste, qui sème des divisions parmi tous les Juifs du monde entier, et qui est le chef de la secte des Nazaréens.
\VS{6}Il a même tenté de profaner le temple~; et nous l'avons saisi, et avons voulu le juger selon notre loi.
\VS{7}Mais le tribun Lysias étant survenu, il nous l'a arraché de nos mains avec une grande violence,
\VS{8}en ordonnant à ses accusateurs de venir vers toi. Tu pourras toi-même, en l'interrogeant, apprendre de lui tout ce dont nous l'accusons.
\VS{9}Les Juifs consentirent à cela, en disant que les choses étaient ainsi.
\TextTitle{Paul défend sa cause devant Félix}
\VS{10}Et après que le gouverneur eut fait signe à Paul de parler, il répondit~: Sachant qu'il y a déjà plusieurs années que tu es le juge de cette nation je réponds pour moi avec plus de courage:
\VS{11}Puisque tu peux comprendre qu'il n'y a pas plus de douze jours que je suis monté à Jérusalem pour adorer Dieu.
\VS{12}Mais ils ne m'ont point trouvé dans le temple disputant avec personne, ni faisant un amas de peuple, soit dans les synagogues, soit dans la ville. 
\VS{13}Et ils ne sauraient soutenir les choses dont ils m'accusent présentement.
\VS{14}Or je te confesse bien ce point, que selon la voie qu'ils appellent secte, je sers ainsi le Dieu de mes pères, croyant toutes les choses qui sont écrites dans la loi et dans les prophètes,
\VS{15}et ayant en Dieu cette espérance, comme ils l'ont eux-mêmes, qu'il y aura une résurrection des justes et des injustes.
\VS{16}C'est pourquoi aussi je travaille pour avoir toujours une conscience pure devant Dieu, et devant les hommes.
\VS{17}Or après plusieurs années, je suis venu pour faire des aumônes et des offrandes dans ma nation.
\VS{18}Et comme je m'occupais de ces choses, quelques Juifs d'Asie m'ont trouvé purifié dans le temple, sans attroupement ni tumulte.
\VS{19}Ils auraient dû eux-mêmes comparaître devant toi et m'accuser, s'ils avaient eu quelque chose contre moi.
\VS{20}Ou bien, que ceux-ci eux-mêmes disent, s'ils ont trouvé en moi quelque injustice, quand j'ai été présenté au sanhédrin~;
\VS{21}à moins que ce ne soit uniquement cette parole que j'ai fait entendre au milieu d'eux~; c'est à cause de la résurrection des morts que je suis aujourd'hui mis en jugement devant vous.
\VS{22}Félix, qui était parfaitement au courant de ce qui concerne cette secte, les ajourna, en disant~: Quand le tribun Lysias sera venu, j'examinerai votre affaire.
\VS{23}Et il donna l'ordre au centenier de garder Paul, en lui laissant une certaine liberté, et n'empêchant aucun des siens de le servir, ou de venir vers lui.
\TextTitle{Paul prêche Christ au gouverneur et à sa femme}
\VS{24}Quelques jours après, Félix vint avec Drusille, sa femme, qui était Juive, et il envoya chercher Paul. Il l'entendit sur la foi en Christ.
\VS{25}Et comme il parlait de la justice, de la tempérance, et du jugement à venir, Félix tout effrayé répondit~: Pour le moment retire-toi~; et quand j'aurai la commodité, je te rappellerai.
\VS{26}Il espérait en même temps que Paul lui donnerait de l'argent afin de le délivrer, c'est pourquoi il l'envoyait chercher souvent, et s'entretenait avec lui.
\TextTitle{Paul emprisonné deux ans à Césarée}
\VS{27}Deux ans s'écoulèrent ainsi, et Félix eut pour successeur Porcius Festus\FTNT{Porcius Festus était procurateur de Judée d'environ 60 à 62, succédant à Antonius Félix.}, qui voulant faire plaisir aux Juifs, laissa Paul en prison.
\Chap{25}
\TextTitle{Paul devant le gouverneur Festus}
\VerseOne{}Festus, étant arrivé dans la province, monta trois jours après de Césarée à Jérusalem.
\VS{2}Le grand-prêtre, et les principaux d'entre les Juifs portèrent plainte contre Paul devant lui. Ils firent des instances auprès de Festus, et dans des vues hostiles,
\VS{3}lui demandèrent une faveur contre lui~: Qu'il le fasse venir à Jérusalem. Ils avaient dressé des embûches pour le tuer en chemin.
\VS{4}Mais Festus leur répondit que Paul était bien gardé à Césarée, et que lui-même devait partir sous peu.
\VS{5}Et il ajouta~: Que les principaux d'entre vous descendent avec moi, et s'il y a quelque chose de coupable contre cet homme, qu'ils l'accusent.
\VS{6}Festus ne passa que dix jours parmi eux, puis il descendit à Césarée. Le lendemain, siégeant au tribunal, il ordonna que Paul soit amené.
\VS{7}Quand il fut amené, les Juifs qui étaient descendus de Jérusalem l'entourèrent et portèrent contre lui de nombreuses et graves accusations, qu'ils ne pouvaient pas prouver.
\VS{8}Tandis que Paul parlait pour sa défense~: Je n'ai rien fait de coupable, ni contre la loi des Juifs, ni contre le temple, ni contre César.
\VS{9}Mais Festus voulant faire plaisir aux Juifs, répondit à Paul, et dit~: Veux-tu monter à Jérusalem et y être jugé sur ces choses devant moi~?
\TextTitle{Paul en appelle à César}
\VS{10}Paul dit~: Je comparais devant le tribunal de César, où il faut que je sois jugé. Je n'ai fait aucun tort aux Juifs, comme tu le sais très bien.
\VS{11}Si j'ai commis quelque injustice, ou un crime digne de mort, je ne refuse pas de mourir~; mais si les choses dont ils m'accusent sont fausses, personne n'a le droit de me livrer à eux. J'en appelle à César.
\VS{12}Alors Festus ayant conféré avec le conseil, lui répondit~: En as-tu appelé à César~? Tu iras à César.
\TextTitle{Le roi Agrippa informé du cas de Paul}
\VS{13}Quelques jours après, le roi Agrippa\FTNT{Agrippa II (27-28 ap. J.-C. – 93-101 ap. J.-C.) était le fils d'Agrippa I(10 av. J.-C. – 44 ap. J.-C.), qui était lui-même le petit-fils d'Hérode le Grand (73 av. J.-C. – 4 av. J.-C.).} et Bérénice\FTNT{Bérénice (née vers 28 ap. J.-C.) était la fille d'Agrippa I et donc la sœur d'Agrippa II. Pendant tout le règne de son frère, elle fut présentée comme reine à ses cotés, raison pour laquelle on soupçonna une liaison incestueuse entre eux.} arrivèrent à Césarée pour saluer Festus.
\VS{14}Comme ils passèrent là plusieurs jours, Festus fit mention au roi de l'affaire de Paul, en disant~: Félix a laissé prisonnier un homme
\VS{15}contre lequel, lorsque j'étais à Jérusalem, les principaux prêtres et les anciens des Juifs ont porté plainte, en demandant sa condamnation.
\VS{16}Mais je leur ai répondu que ce n'est pas la coutume des Romains de livrer quelqu'un à la mort, avant que l'inculpé ait été mis en présence de ses accusateurs, et qu'il ait eu la liberté de se défendre sur le crime dont on l'accuse.
\VS{17}Ils sont donc venus ici, et sans différer, je siégeai le lendemain, et je donnai l'ordre qu'on amène cet homme.
\VS{18}Ses accusateurs s'étant présentés, ne lui imputèrent aucun des crimes dont je pensais qu'ils l'accuseraient.
\VS{19}Mais ils avaient avec lui des discussions relatives à leurs superstitions, et à un certain Jésus qui est mort, que Paul affirmait être vivant.
\VS{20}Ne sachant quel parti prendre dans ce débat, je demandai à cet homme s'il voulait aller à Jérusalem et y être jugé sur ces choses.
\VS{21}Mais Paul en ayant appelé, pour que sa cause soit réservée à la connaissance de l'empereur, j'ai ordonné qu'on le garde jusqu'à ce que je l'envoie à César.
\VS{22}Alors Agrippa dit à Festus~: Je voudrais bien aussi entendre cet homme. Demain, dit-il, tu l'entendras.
\TextTitle{Paul est amené dans la salle d'audience}
\VS{23}Le lendemain donc, Agrippa et Bérénice étant venus en grande pompe, et étant entrés dans la salle d'audience avec les tribuns et les principaux de la ville, Paul fut amené sur l'ordre de Festus.
\VS{24}Et Festus dit~: Roi Agrippa, et vous tous qui êtes ici avec nous, vous voyez cet homme au sujet duquel toute la multitude des Juifs s'est adressée à moi, soit à Jérusalem soit ici, en s'écriant qu'il ne devait plus vivre.
\VS{25}Pour moi, ayant trouvé qu'il n'avait rien fait qui mérite la mort, et lui-même en ayant appelé à Auguste, j'ai résolu de le faire partir.
\VS{26}Comme je n'ai rien de certain à écrire à l'empereur sur son compte, je vous l'ai présenté, et principalement à toi, roi Agrippa, afin qu'après en avoir fait l'examen, j'aie de quoi écrire.
\VS{27}Car il me semble qu'il n'est pas raisonnable d'envoyer un prisonnier sans marquer les faits dont on l'accuse.
\Chap{26}
\TextTitle{Discours de Paul devant Agrippa\FTNTT{\vref{Ac. 9:1-18}~; \vref{22:1-16}}}
\VerseOne{}Agrippa dit à Paul~: Il t'est permis de parler pour toi-même. Alors Paul ayant étendu la main, parla ainsi pour sa défense~:
\VS{2}Roi Agrippa~! Je m'estime béni de ce que je dois me défendre aujourd'hui devant toi, de toutes les choses dont les Juifs m'accusent~;
\VS{3}car tu connais parfaitement leurs coutumes et leurs discussions. Je te prie donc de m'écouter avec patience.
\VS{4}Ma vie, dès les premiers temps de ma jeunesse, est connue de tous les Juifs, puisqu'elle s'est passée à Jérusalem, au milieu de ma nation.
\VS{5}Car ils savent depuis longtemps, s'ils veulent en rendre témoignage, que j'ai vécu en pharisien, selon la secte la plus rigide de notre religion.
\VS{6}Et maintenant, je suis mis en jugement pour l'espérance de la promesse que Dieu a faite à nos pères,
\VS{7}et à laquelle nos douze tribus, qui servent Dieu continuellement nuit et jour, espèrent parvenir~; et c'est pour cette espérance, ô roi Agrippa, que je suis accusé par les Juifs.
\VS{8}Quoi~? Jugez-vous incroyable que Dieu ressuscite les morts~?
\VS{9}Pour moi, j'avais cru devoir agir vigoureusement contre le Nom de Jésus de Nazareth.
\VS{10}C'est ce que j'ai fait à Jérusalem. J'ai mis en prison plusieurs des saints, après en avoir reçu le pouvoir des principaux prêtres, et quand on les faisait mourir, je joignais mon suffrage à celui des autres.
\VS{11}Je les ai souvent châtiés dans toutes les synagogues, et les forçais à blasphémer. Dans mes excès de fureur contre eux, je les persécutais même jusque dans les villes étrangères.
\VS{12}Comme j'allais aussi à Damas dans ce dessein, avec l'autorisation et la permission des principaux prêtres,
\VS{13}en plein midi, ô roi, je vis en chemin resplendir autour de moi et de mes compagnons, une lumière venant du ciel et dont l'éclat surpassait celui du soleil.
\VS{14}Nous tombâmes tous par terre, et j'entendis une voix qui me parlait en langue hébraïque~: Saul, Saul, pourquoi me persécutes-tu~? Il te serait dur de regimber contre les aiguillons.
\VS{15}Je répondis~: Qui es-tu Seigneur~? Et il répondit~: Je suis Jésus que tu persécutes.
\VS{16}Mais lève-toi, et tiens-toi sur tes pieds~; car je te suis apparu pour t'établir serviteur et témoin des choses que tu as vues et de celles pour lesquelles je t'apparaîtrai.
\VS{17}Je t'ai arraché du milieu du peuple et des Gentils, vers qui je t'envoie maintenant,
\VS{18}pour ouvrir leurs yeux afin qu'ils passent des ténèbres à la lumière, et de la puissance de Satan à Dieu~; afin que par la foi qu'ils auront en moi, ils reçoivent la rémission de leurs péchés et qu'ils aient part à l'héritage des saints.
\VS{19}Ainsi, ô roi Agrippa, je n'ai pas été désobéissant à la vision céleste.
\VS{20}A ceux de Damas d'abord, puis à Jérusalem, dans toute la Judée, et chez les Gentils, j'ai prêché la repentance et la conversion à Dieu, avec la pratique d'œuvres dignes de la repentance.
\VS{21}C'est pour cela que les Juifs se sont saisis de moi dans le temple, et ont tâché de me tuer.
\VS{22}Mais ayant été secouru par l'aide de Dieu, je suis vivant jusqu'à ce jour, rendant témoignage aux petits et aux grands, sans m'écarter en rien de ce que les prophètes et Moïse ont prédit devoir arriver,
\VS{23}à savoir que le Christ souffrirait, et que ressuscité le premier d'entre les morts, il annoncerait la lumière au peuple et aux nations.
\TextTitle{Paul exhorte Agrippa}
\VS{24}Comme il parlait ainsi pour sa défense, Festus dit à haute voix~: Tu es fou Paul~! Ton grand savoir dans les lettres te fait déraisonner.
\VS{25}Et Paul dit~: Je ne suis point fou, très excellent Festus, mais je dis des paroles de vérité et de bon sens.
\VS{26}Car le roi est bien informé de ces choses~; et je lui en parle librement, parce que je suis persuadé qu'il n'en ignore aucune, puisque ce n'est pas en cachette qu'elles se sont passées.
\VS{27}Ô Roi Agrippa~! Crois-tu aux prophètes~? Je sais que tu y crois.
\VS{28}Et Agrippa répondit à Paul~: Tu vas bientôt me persuader de devenir chrétien~!
\VS{29}Et Paul lui dit~: Je souhaiterais devant Dieu que non seulement toi, mais aussi tous ceux qui m'écoutent aujourd'hui, vous deveniez tels que je suis à l'exception de ces liens~!
\VS{30}Paul ayant dit ces choses, le roi se leva, avec le gouverneur et Bérénice, et ceux qui étaient assis avec eux.
\VS{31}Et s'étant retirés à part, ils se disaient les uns les autres~: Cet homme n'a rien fait qui mérite la mort ou la prison.
\VS{32}Et Agrippa dit à Festus~: Cet homme aurait pu être relâché s'il n'avait pas appelé à César.
\Chap{27}
\TextTitle{Toujours prisonnier, Paul embarque pour Rome}
\VerseOne{}Lorsqu'il fut décidé que nous embarquerions pour l'Italie, on remit Paul avec quelques autres prisonniers à un nommé Julius, centenier d'une cohorte de la légion appelée Auguste.
\VS{2}Nous montâmes sur un navire d'Adramytte, nous partîmes prenant notre route vers les côtes de l'Asie, ayant avec nous Aristarque, un Macédonien de la ville de Thessalonique.
\VS{3}Le jour suivant, nous arrivâmes à Sidon~; et Julius, qui traitait Paul avec bienveillance, lui permit d'aller vers ses amis afin de recevoir leurs soins.
\VS{4}Puis étant partis de là, nous longeâmes l'île de Chypre, parce que les vents étaient contraires.
\VS{5}Après avoir traversé la mer de Cilicie et de Pamphylie, nous arrivâmes à Myra, ville de Lycie.
\VS{6}Et là, le centenier trouva un navire d'Alexandrie qui allait en Italie, dans lequel il nous fit monter.
\VS{7}Et comme nous naviguions lentement pendant plusieurs jours, et que nous étions arrivés avec peine vis-à-vis de Cnide, parce que le vent ne nous permettait pas d'avancer, nous naviguâmes en dessous de la Crète, vers Salmone.
\VS{8}Nous la côtoyâmes avec peine, nous arrivâmes à un lieu qui est appelé Beaux-Ports, près duquel était la ville de Lasée.
\VS{9}Il s'était écoulé beaucoup de temps, et la navigation devenait dangereuse, car le temps du jeûne était déjà passé\FTNT{Ce jeûne correspondait au jour de l'expiation célébré le dixième jour du septième mois. \vref{Lé. 23:27}.}.
\VS{10}C'est pourquoi Paul les avertit, en disant~: Ô hommes, je vois que la navigation ne se fera pas sans péril et sans dommage, non seulement pour la cargaison et pour le navire, mais aussi pour nos propres vies.
\VS{11}Mais le centenier écouta plus le pilote et le maître du navire, plutôt que les paroles de Paul.
\VS{12}Et comme le port n'était pas bon pour y passer l'hiver, la plupart furent d'avis de partir de là, pour tâcher de gagner Phénix, qui est un port de Crète, qui regarde le vent d'Afrique et le couchant septentrional, afin d'y passer l'hiver.
\VS{13}Un vent du midi commença à souffler doucement, et se croyant maîtres de leur dessein, ils levèrent l'ancre et côtoyèrent de près l'île de Crète.
\TextTitle{Une tempête de plusieurs jours}
\VS{14}Mais bientôt un vent impétueux, du nord-est, qu'on appelle Euraquilon\FTNT{Euraquilon~: Vagues et vent d'Est}, se leva du côté de l'île.
\VS{15}Le navire fut emporté par la violence de la tempête, et ne pouvant résister, nous nous laissâmes aller au gré du vent.
\VS{16}Nous passâmes au-dessous d'une petite île nommée Clauda, et nous eûmes de la peine à nous rendre maîtres de la chaloupe~;
\VS{17}après l'avoir hissée, les matelots se servirent des moyens de secours pour ceindre le navire, et dans la crainte de tomber sur la Syrte\FTNT{Syrte~: Il s'agit de la Grande Syrte et de la Petite Syrte~: deux bancs de sables mouvants très redoutés.}, ils abaissèrent les voiles. C'est ainsi qu'on se laissa emporter par le vent.
\VS{18}Comme nous étions violemment battus par la tempête, le jour suivant, ils jetèrent la cargaison à la mer~;
\VS{19}et le troisième jour, nous jetâmes de nos propres mains les agrès du navire.
\VS{20}Le soleil et les étoiles ne parurent pas pendant plusieurs jours, et la tempête nous agitait si violemment que nous perdîmes enfin toute espérance de nous sauver.
\TextTitle{Paul rassure les membres du navire}
\VS{21}On n'avait pas mangé depuis longtemps. Paul se tenant alors debout au milieu d'eux, leur dit~: Ô hommes, il fallait m'écouter et ne pas partir de Crète, afin d'éviter cette tempête et ce dommage.
\VS{22}Maintenant je vous exhorte à prendre courage~; car aucun de vous ne perdra la vie, et il n'y aura de perte que celle du navire.
\VS{23}Car un ange du Dieu à qui j'appartiens et que je sers m'est apparu cette nuit,
\VS{24}et m'a dit~: Paul, ne crains point~; il faut que tu comparaisses devant César~; et voici, Dieu t'a donné tous ceux qui naviguent avec toi.
\VS{25}C'est pourquoi, ô hommes, prenez courage, car j'ai cette confiance en Dieu que la chose arrivera comme elle m'a été dite.
\VS{26}Mais nous devons échouer sur une île.
\VS{27}La quatorzième nuit, vers minuit, tandis que nous étions ballotés sur l'Adriatique, les matelots soupçonnèrent qu'on approchait de quelque terre.
\VS{28}Ayant jeté la sonde, ils trouvèrent vingt brasses~; puis étant passés un peu plus loin, et ayant encore jeté la sonde, ils trouvèrent quinze brasses.
\VS{29}Mais craignant de heurter contre des écueils, ils jetèrent quatre ancres de la poupe, et attendirent le jour avec impatience.
\VS{30}Mais comme les matelots cherchaient à s'échapper du navire, et mettaient la chaloupe à la mer, sous prétexte de jeter les ancres de la proue,
\VS{31}Paul dit au centenier et aux soldats~: Si ces hommes ne restent pas dans le navire, vous ne pouvez pas être sauvés.
\VS{32}Alors les soldats coupèrent les cordes de la chaloupe, et la laissèrent tomber.
\VS{33}Avant que le jour paraisse, Paul les exhorta tous à prendre de la nourriture, en leur disant~: C'est aujourd'hui le quatorzième jour que vous êtes en attente et que vous persistez à vous abstenir de manger.
\VS{34}Je vous exhorte donc à prendre quelque nourriture, vu que cela est nécessaire pour votre conservation, et aucun de vos cheveux ne se perdra.
\VS{35}Ayant ainsi parlé, il prit du pain, et rendit grâces à Dieu en présence de tous~; il le rompit et se mit à manger.
\VS{36}Et tous, reprenant courage, mangèrent aussi.
\VS{37}Nous étions dans le navire deux cent soixante-seize personnes.
\VS{38}Quand ils eurent mangé jusqu'à être rassasiés, ils allégèrent le navire en jetant le blé dans la mer.
\TextTitle{Naufrage du navire}
\VS{39}Lorsque le jour fut venu, ils ne reconnurent point la terre~; mais ayant aperçu un golfe avec un rivage, ils résolurent d'y faire échouer le navire, s'ils le pouvaient.
\VS{40}Ayant donc retiré les ancres, ils abandonnèrent le navire à la mer, lâchant en même temps les attaches des gouvernails~; et ayant tendu la voile de l'artimon, ils tâchaient de se diriger vers le rivage.
\VS{41}Mais ils rencontrèrent une langue de terre, où ils firent échouer le navire~; et la proue, s'étant engagée, resta immobile, tandis que la poupe se brisait par la violence des vagues.
\VS{42}Les soldats furent d'avis de tuer les prisonniers, de peur que quelqu'un d'eux ne s'échappe à la nage.
\VS{43}Mais le centenier, voulant sauver Paul, les empêcha d'exécuter ce conseil. Il ordonna à ceux qui savaient nager de se jeter les premiers dans l'eau pour gagner la terre,
\VS{44}et aux autres de se mettre sur des planches ou sur des débris du navire. Et ainsi tous parvinrent à terre sains et saufs.
\Chap{28}
\TextTitle{Paul mordu par une vipère sur l'île de Malte}
\VerseOne{}Une fois hors de danger, ils reconnurent alors que l'île s'appelait Malte.
\VS{2}Les barbares nous traitèrent avec beaucoup d'humanité~; ils nous recueillirent tous auprès d'un grand feu, qu'ils avaient allumé parce que la pluie tombait et qu'il faisait très froid.
\VS{3}Paul ayant ramassé un tas de broussailles et l'ayant mis au feu, une vipère en sortit à cause de la chaleur et s'attacha à sa main.
\VS{4}Et quand les barbares virent cette bête suspendue à sa main, ils se dirent les uns les autres~: Certainement cet homme est un meurtrier~; puisque après être échappé de la mer, la justice ne permet pas qu'il vive. 
\VS{5}Mais Paul ayant secoué la bête dans le feu, ne ressentit aucun mal.
\VS{6}Les barbares s'attendaient à le voir enfler ou tomber subitement mort~; mais après avoir longtemps attendu, voyant qu'il ne lui arrivait aucun mal, ils changèrent de langage et dirent que c'était un dieu.
\TextTitle{Guérison du père de Publius}
\VS{7}Or en cet endroit-là étaient des terres qui appartenaient au principal de l'île, nommé Publius, qui nous reçut et nous logea pendant trois jours avec beaucoup de bonté.
\VS{8}Et il arriva que le père de Publius était au lit, malade de la fièvre et de la dysenterie~; Paul s'étant rendu vers lui, pria, lui imposa les mains, et le guérit.
\VS{9}Là-dessus, vinrent tous les autres malades de l'île, et ils furent guéris.
\VS{10}Ils nous rendirent de grands honneurs, et à notre départ, on nous fournit ce qui nous était nécessaire.
\TextTitle{Paul arrive à Rome}
\VS{11}Trois mois après, nous partîmes sur un navire d'Alexandrie qui avait passé l'hiver dans l'île, et qui avait pour enseigne Castor et Pollux.
\VS{12}Ayant abordé à Syracuse, nous y restâmes trois jours.
\VS{13}De là, en suivant la côte, nous arrivâmes à Reggio~; et un jour après, le vent du Midi s'étant levé, nous fîmes en deux jours le trajet jusqu'à Pouzzoles,
\VS{14}où nous trouvâmes des frères qui nous prièrent de passer sept jours avec eux. Et ensuite, nous arrivâmes à Rome.
\VS{15}Et les frères qui y étaient, ayant appris de nos nouvelles, vinrent à notre rencontre jusqu'au Forum d'Appius et aux Trois-Tavernes~; en les voyant, Paul rendit grâces à Dieu et prit courage.
\TextTitle{Paul annonce Christ aux Juifs de Rome}
\VS{16}Lorsque nous fûmes arrivés à Rome, le centenier mit les prisonniers entre les mains du préfet du prétoire~; mais quant à Paul, il lui permit de demeurer dans un domicile particulier avec un soldat qui le gardait.
\VS{17}Or il arriva que trois jours après que Paul convoqua les principaux des Juifs~; et quand ils furent réunis, il leur dit~: Hommes frères~! Sans avoir rien fait contre le peuple ni contre les coutumes des pères, j'ai été mis en prison à Jérusalem, et livré entre les mains des Romains,
\VS{18}qui après m'avoir examiné, voulaient me relâcher parce qu'il n'y avait en moi aucun crime qui mérite la mort.
\VS{19}Mais les Juifs s'y opposèrent, j'ai été contraint d'en appeler à César~; n'ayant du reste aucun dessein d'accuser ma nation.
\VS{20}C'est pour ce sujet que je vous ai appelés, afin de vous voir et vous parler~; car c'est pour l'espérance d'Israël que je porte cette chaîne.
\VS{21}Mais ils lui répondirent~: Nous n'avons reçu de Judée aucune lettre à ton sujet, et il n'est venu aucun frère qui ait rapporté ou dit quelque mal de toi.
\VS{22}Cependant nous entendrons volontiers de toi quel est ton sentiment~; car quant à cette secte, il nous est connu qu'on la contredit partout.
\VS{23}Et après lui avoir assigné un jour, plusieurs vinrent auprès de lui dans son logis~; et il leur expliquait par plusieurs témoignages le Royaume de Dieu, et depuis le matin jusqu'au soir, il cherchait à les persuader de ce qui concerne Jésus, tant par la loi de Moïse que par les prophètes.
\VS{24}Et les uns furent persuadés par les choses qu'il disait~; et les autres n'y crurent point.
\TextTitle{Incrédulité des Juifs~: Paul se tourne vers les Gentils\FTNTT{\vref{Ap. 13:14}~; \vref{18:6}.}}
\VS{25}C'est pourquoi n'étant pas d'accord entre eux, ils se retirèrent après que Paul leur eut dit ces paroles~: Le Saint-Esprit a bien parlé à nos pères par le prophète Esaïe en disant~:
\VS{26}Va vers ce peuple et dis-lui~: Vous entendrez de vos oreilles, et vous ne comprendrez point~; vous regarderez de vos yeux, et vous ne verrez point.
\VS{27}Car le cœur de ce peuple est devenu insensible~; ils ont endurci leurs oreilles, et ils ont fermé leurs yeux~; de peur qu'ils ne voient des yeux, qu'ils n'entendent des oreilles, qu'ils ne comprennent de leur cœur, qu'ils ne se convertissent, et que je ne les guérisse\FTNT{\vref{Es. 6:10}.}.
\VS{28}Sachez donc que ce salut de Dieu est envoyé aux Gentils, et ils l'écouteront.
\VS{29}Lorsqu'il eut dit cela, les Juifs s'en allèrent, discutant vivement entre eux.
\VS{30}Paul demeura deux ans entiers dans une maison qu'il avait louée. Il recevait tous ceux qui venaient le voir,
\VS{31}prêchant le Royaume de Dieu, et enseignant les choses qui concernent le Seigneur Jésus-Christ en toute liberté dans les paroles et sans aucun empêchement.
\PPE{}
\end{multicols}

%\clearpage\ShortTitle{Jacques}\BookTitle{Jacques}\BFont
\noindent\hrulefill
{\footnotesize
\textit{
\bigskip
{\centering{}
\\Auteur : Jacques
\\Signification : Qui supplante
\\Thème : La vie chrétienne sous son aspect pratique
\\Date de rédaction : Env. 45-50 ap. J.-C.\\}
}
%\bigskip
\textit{
\\Jacques, frère de Jésus-Christ homme, et ancien au sein de la première église chrétienne située à Jérusalem, écrit aux chrétiens d'origine juive, dispersés dans l'empire romain. Il les console suite à l'adversité qu'ils rencontraient et les exhorte à tenir ferme, leur expliquant que la foi authentique doit être accompagnée d'œuvres. Il les met en garde contre la convoitise, source de toutes les tentations, et les prévient également quant à l'amour du monde et la confiance que certains peuvent mettre dans l'argent. Pour terminer, il leur recommande d'être patients dans l'épreuve et de prier sans cesse jusqu'au retour du Seigneur.\bigskip
}
}
\par\nobreak\noindent\hrulefill
\begin{multicols}{2}
\Chap{1}
\TextTitle{Introduction}
\VerseOne{}Jacques, serviteur de Dieu, et du Seigneur Jésus-Christ, aux douze tribus qui sont dispersées, salut !
\TextTitle{La nécessité de l'épreuve de la foi}
\VS{2}Mes frères, regardez comme un sujet d'une parfaite joie quand vous êtes exposés à diverses épreuves,
\VS{3}sachant que l'épreuve de votre foi produit la patience.
\VS{4}Mais il faut que la patience accomplisse parfaitement son œuvre, afin que vous soyez parfaits et accomplis, en sorte qu'il ne vous manque rien.
\VS{5}Et si quelqu'un de vous manque de sagesse, qu'il la demande à Dieu, qui la donne à tous libéralement, et sans reproche, et elle lui sera donnée.
\VS{6}Mais qu'il la demande avec foi, ne doutant nullement ; car celui qui doute est semblable au flot de la mer, agité et poussé çà et là par le vent.
\VS{7}Qu'un tel homme ne s'attende pas à recevoir quelque chose du Seigneur.
\VS{8}L'homme double de coeur est inconstant dans toutes ses voies.
\VS{9}Que le frère de basse condition se glorifie dans son élévation.
\VS{10}Que le riche, au contraire, se glorifie dans sa basse condition ; car il passera comme la fleur de l'herbe.
\VS{11}En effet, le soleil s'est levé avec sa chaleur ardente, et l'herbe a séché, et sa fleur est tombée, et son éclat a péri, ainsi le riche se flétrira dans ses entreprises.
\TextTitle{Dieu ne tente personne ; la justice de Dieu}
\VS{12}Heureux l'homme qui endure la tentation\FTNT{Le terme grec « peirasmos » utilisé dans ce verset veut aussi dire « épreuve ».} ; car après avoir été éprouvé, il recevra la couronne de vie, que le Seigneur a promise à ceux qui l'aiment.
\VS{13}Quand quelqu'un est tenté, qu'il ne dise pas : Je suis tenté par Dieu. Car Dieu ne peut être tenté par le mal, et aussi ne tente-t-il personne.
\VS{14}Mais chacun est tenté quand il est attiré et amorcé par sa propre convoitise.
\VS{15}Puis quand la convoitise a conçu, elle enfante le péché ; et le péché, étant consommé, produit la mort.
\VS{16}Mes frères bien-aimés, ne vous y trompez pas :
\VS{17}Tout ce qui nous est donné d'excellent et tout don parfait viennent d'en haut et descendent du Père des lumières, en qui il n'y a ni changement ni ombre de variation.
\VS{18}Il nous a engendrés de sa propre volonté, par la parole de la vérité, afin que nous soyons comme les prémices de ses créatures.
\VS{19}Ainsi, mes frères bien-aimés, que tout homme soit prompt à écouter, lent à parler et lent à la colère ;
\VS{20}car la colère de l'homme n'accomplit pas la justice de Dieu.
\TextTitle{Importance de la mise en pratique de la parole}
\VS{21}C'est pourquoi, rejetant toute souillure et tout résidu\FTNT{Le mot « résidu » vient du grec « perisseia », ce mot signifie « abondance », « surabondamment », « tout excès », « reste ». Les Grecs utilisaient ce terme pour décrire l'excès de cire dans leurs oreilles. Il est question de la méchanceté qui reste dans un Chrétien et qui provient de son état antérieur à sa conversion.} de méchanceté, recevez avec douceur la parole qui a été plantée en vous et qui peut sauver vos âmes.
\VS{22}Et mettez en pratique la parole, et ne l'écoutez pas seulement, en vous trompant vous-mêmes par de vains discours.
\VS{23}Car, si quelqu'un écoute la parole et ne la met pas en pratique, il est semblable à un homme qui regarde dans un miroir son visage naturel,
\VS{24}et qui, après s'être regardé, s'en va, et oublie aussitôt comment il était.
\VS{25}Mais celui qui aura plongé les regards dans la loi parfaite, la loi de la liberté, et qui aura persévéré, n'étant point un auditeur oublieux, mais pratiquant les œuvres qui lui sont prescrites, celui-là sera heureux dans son oeuvre.
\TextTitle{La religion pure et sans tache}
\VS{26}Si quelqu'un parmi vous croit être religieux alors qu'il ne tient pas sa langue en bride, mais séduit son cœur, la religion d'un tel homme est vaine.
\VS{27}La religion pure et sans tache envers notre Dieu et notre Père, c'est de visiter les orphelins et les veuves dans leurs afflictions, et de se conserver pur des souillures de ce monde.
\Chap{2}
\TextTitle{L'amour pour son prochain en pratique}
\VerseOne{}Mes frères, n'ayez point la foi en notre Seigneur Jésus-Christ glorieux, en ayant égard à l'apparence des personnes.
\VS{2}En effet, s'il entre dans votre assemblée un homme qui porte un anneau d'or et un habit magnifique, et qu'il y entre aussi un pauvre misérablement vêtu ;
\VS{3}et que vous ayez égard à celui qui porte l'habit magnifique et lui disiez : Toi, assieds-toi ici honorablement ! Et que vous disiez au pauvre : Toi, tiens-toi là debout ! Ou, assieds-toi ici sur mon marchepied !
\VS{4}n'avez-vous pas fait de différence en vous-mêmes, et n'êtes-vous pas des juges qui avez des pensées injustes ?
\VS{5}Ecoutez, mes frères bien-aimés : Dieu n'a-t-il pas choisi les pauvres de ce monde, qui sont riches en la foi, et héritiers du Royaume qu'il a promis à ceux qui l'aiment ?
\VS{6}Mais vous avez déshonoré le pauvre ! Et cependant les riches ne vous oppriment-ils pas, et ne vous traînent-ils pas devant les tribunaux ?
\VS{7}N'est-ce pas eux qui blasphèment le beau Nom qui a été invoqué sur vous ?
\VS{8}Si, en effet, vous accomplissez la loi royale, qui est selon l'Ecriture : Tu aimeras ton prochain comme toi-même\FTNT{Lé. 19:18.}, vous faites bien.
\VS{9}Mais si vous avez égard à l'apparence des personnes, vous commettez un péché, et vous êtes convaincus par la loi comme des transgresseurs.
\VS{10}Car quiconque observe toute la loi, mais pèche contre un seul commandement, devient coupable de tous.
\VS{11}En effet, celui qui a dit : Tu ne commettras point d'adultère, a dit aussi : Tu ne tueras point. Or, si donc tu ne commets point d'adultère\FTNT{Ex. 20:13-14.}, mais que tu tues, tu deviens transgresseur de la loi.
\VS{12}Ainsi parlez et ainsi agissez comme devant être jugés par la loi de la liberté,
\VS{13}car il y aura un jugement sans miséricorde sur celui qui n'aura point usé de miséricorde\FTNT{Mt. 7:2.} ; mais la miséricorde triomphe du jugement.
\TextTitle{Les œuvres de la foi}
\VS{14}Mes frères, que servira-t-il à quelqu'un de dire qu'il a la foi, s'il n'a pas les œuvres ? Cette foi peut-elle le sauver ?
\VS{15}Et si un frère ou une sœur sont nus et manquent de ce qui leur est nécessaire chaque jour pour vivre,
\VS{16}et que l'un d'entre vous leur dise : Allez en paix, chauffez-vous, et rassasiez-vous ! Et que vous ne leur donniez pas les choses nécessaires pour le corps, que leur servira cela ?
\VS{17}De même aussi la foi, si elle n'a pas les œuvres, elle est morte en elle-même.
\VS{18}Mais quelqu'un dira : Tu as la foi ; et moi, j'ai les œuvres. Montre-moi donc ta foi sans les œuvres, et moi, je te montrerai ma foi par mes œuvres.
\VS{19}Tu crois qu'il n'y a qu'un Dieu, tu fais bien ; les démons le croient aussi, et ils tremblent.
\VS{20}Mais, ô homme vain, veux-tu savoir que la foi qui est sans les œuvres est morte ?
\TextTitle{La foi d'Abraham et de Rahab manifestée dans leurs œuvres\FTNT{Ro. 4:1-25}}
\VS{21}Abraham, notre père, ne fut-il pas justifié par les œuvres, quand il offrit son fils Isaac sur l'autel ?
\VS{22}Ne vois-tu donc pas que sa foi agissait avec ses œuvres, et que ce fut par ses œuvres que sa foi fut rendue parfaite ?
\VS{23}Ainsi s'accomplit ce que dit l'Ecriture : Abraham crut en Dieu, et cela lui fut imputé à justice\FTNT{Ge. 15:6.}; et il fut appelé ami de Dieu.
\VS{24}Vous voyez donc que l'homme est justifié par les œuvres, et non par la foi seulement.
\VS{25}Pareillement, Rahab, la prostituée, ne fut-elle pas également justifiée par les œuvres, lorsqu'elle reçut les messagers, et qu'elle les fit partir par un autre chemin\FTNT{Jos. 2:1-21.} ?
\VS{26}Car, comme le corps sans esprit est mort, de même la foi sans les œuvres est morte.
\Chap{3}
\TextTitle{Les enseignants jugés plus sévèrement}
\VerseOne{}Ne soyez pas nombreux, mes frères, à devenir des enseignants\FTNT{Du grec « didaskalos » : « maître », « professeur », « docteur chargé d'instruire, d'enseigner la parole ». Mt. 8:19 ; Mt. 22:16 ; 1 Co. 12:28.}, sachant que nous en recevrons un plus grand jugement.
\TextTitle{Enseignements sur la langue}
\VS{2}Car nous péchons tous en plusieurs choses. Si quelqu'un ne pèche pas en paroles, c'est un homme parfait, et il peut même tenir en bride tout le corps.
\VS{3}Voici, nous mettons le mors dans la bouche des chevaux, afin qu'ils nous obéissent, et nous menons çà et là tout le corps.
\VS{4}Voici, aussi les navires, quoiqu'ils soient si grands et qu'ils soient agités par la tempête, ils sont dirigés partout çà et là par un petit gouvernail, selon qu'il plaît à celui qui les gouverne.
\VS{5}Il en est ainsi de la langue, c'est un petit membre, et cependant elle peut se vanter de grandes choses. Voici, un petit feu, combien de bois allume-t-il?
\VS{6}La langue aussi est un feu ; c'est le monde de l'iniquité. La langue est placée parmi nos membres, souillant tout le corps, et enflammant tout le cours de la vie, étant elle-même enflammée par le feu de la géhenne.
\VS{7}Car toutes les espèces d'animaux sauvages, d'oiseaux, de reptiles, et d'animaux marins, se domptent et ont été domptés par la nature humaine ;
\VS{8}mais nul homme ne peut dompter la langue ; c'est un mal qu'on ne peut réprimer ; elle est pleine d'un venin mortel.
\VS{9}Par elle nous bénissons Dieu notre Père, et par elle nous maudissons les hommes faits à la ressemblance de Dieu.
\VS{10}De la même bouche sortent la bénédiction et la malédiction. Il ne faut pas qu'il en soit ainsi, mes frères.
\VS{11}Une fontaine fait-elle jaillir par la même ouverture l'eau douce et l'eau amère ?
\VS{12}Mes frères, un figuier peut-il produire des olives, ou une vigne des figues ? De même, aucune fontaine ne peut produire de l'eau salée et de l'eau douce.
\TextTitle{La sagesse humaine et la sagesse d'en haut}
\VS{13}Y a-t-il parmi vous quelque homme sage et intelligent ? Qu'il fasse voir ses oeuvres par une bonne conduite avec douceur et sagesse.
\VS{14}Mais si vous avez une envie amère et un esprit de querelle dans vos cœurs, ne vous glorifiez pas, et ne mentez pas en déshonorant la vérité de l'Evangile.
\VS{15}Car ce n'est pas là la sagesse qui descend d'en haut ; mais c'est une sagesse terrestre, animale\FTNT{Animale ou charnelle.} et diabolique.
\VS{16}Car là où il y a de l'envie et un esprit de querelle, là est le désordre, et toute sorte de mal.
\VS{17}Mais la sagesse d'en haut est premièrement pure, ensuite pacifique, modérée, conciliante, pleine de miséricorde et de bons fruits, sans partialité, et sans hypocrisie.
\VS{18}Or le fruit de la justice est semé dans la paix pour ceux qui s'adonnent à la paix.
\Chap{4}
\TextTitle{Condamnation des mauvais désirs}
\VerseOne{}D'où viennent parmi vous les disputes et les querelles ? N'est-ce pas de vos voluptés qui combattent dans vos membres ?
\VS{2}Vous convoitez, et vous n'obtenez pas ce que vous désirez ; vous avez une envie mortelle, vous êtes jaloux, et vous ne pouvez obtenir ce que vous enviez ; vous vous querellez, vous vous disputez, et vous n'avez pas ce que vous désirez, parce que vous ne demandez pas.
\VS{3}Vous demandez, et vous ne recevez point, parce que vous demandez mal, dans le but de satisfaire vos voluptés.
\VS{4}Hommes et femmes adultères ! Ne savez-vous pas que l'amitié du monde est inimitié contre Dieu ? Celui donc qui veut être ami du monde, se rend ennemi de Dieu.
\VS{5}Pensez-vous que l'Ecriture parle en vain ? L'Esprit qui habite en nous, vous inspire-t-il l'envie ?
\TextTitle{S'humilier devant Yahweh, le juste Juge}
\VS{6}Il vous accorde, au contraire, une plus grande grâce ; c'est pourquoi l'Ecriture dit : Dieu résiste aux orgueilleux, mais il fait grâce aux humbles\FTNT{Pr. 3:34.}.
\VS{7}Soumettez-vous donc à Dieu ; résistez au diable, et il s'enfuira de vous.
\VS{8}Approchez-vous de Dieu, et il s'approchera de vous. Pécheurs, nettoyez vos mains ; et vous qui êtes doubles de cœur, purifiez vos cœurs.
\VS{9}Sentez vos misères ; et soyez dans le deuil et dans les larmes ; que votre rire se change en pleurs, et votre joie en tristesse.
\VS{10}Humiliez-vous dans la présence du Seigneur, et il vous élèvera.
\VS{11}Mes frères, ne médisez point les uns des autres. Celui qui médit de son frère, et qui condamne son frère, médit de la loi et juge la loi. Or, si tu juges la loi, tu n'es pas observateur de la loi, mais le juge.
\VS{12}Il n'y a qu'un seul Législateur, qui peut sauver et qui peut perdre ; mais toi, qui es-tu, qui juges les autres ?
\TextTitle{Abandonner ses désirs au profit de la volonté de Dieu}
\VS{13}A vous, maintenant, qui dites : Aujourd'hui ou demain nous irons dans telle ou telle ville, et nous y passerons une année, et nous trafiquerons et nous gagnerons !
\VS{14}Qui toutefois ne savez pas ce qui arrivera le lendemain ! Car qu'est-ce que votre vie ? Ce n'est certes qu'une vapeur qui parait pour un peu de temps, et qui ensuite s'évanouit.
\VS{15}Au lieu de dire : Si le Seigneur le veut, et si nous vivons, nous ferons aussi ceci ou cela.
\VS{16}Mais maintenant vous vous glorifiez dans vos pensées orgueilleuses. Une telle gloire est mauvaise.
\VS{17}Il y a donc du péché en celui qui sait faire le bien, et qui ne le fait pas.
\Chap{5}
\TextTitle{Avertissement aux riches}
\VerseOne{}A vous maintenant, riches ! Pleurez et gémissez à cause des malheurs qui vont tomber sur vous.
\VS{2}Vos richesses sont pourries et vos vêtements sont rongés par les vers.
\VS{3}Votre or et votre argent sont rouillés ; et leur rouille s'élèvera en témoignage contre vous et dévorera vos chairs comme un feu. Vous avez amassé des trésors pour les derniers jours.
\VS{4}Voici, le salaire des ouvriers qui ont moissonné vos champs, et dont vous les avez frustrés, crie ; et les cris des moissonneurs sont parvenus aux oreilles du Seigneur des armées.
\VS{5}Vous avez vécu dans les délices sur la terre, vous vous êtes livrés aux voluptés, et vous avez rassasié vos cœurs comme en un  jour de sacrifices.
\VS{6}Vous avez condamné et mis à mort le juste qui ne vous a pas résisté.
\TextTitle{Se préparer à l'avènement du Seigneur}
\VS{7}Mais vous, mes frères, attendez patiemment jusqu'à l'avènement du Seigneur. Voici, le laboureur attend le précieux fruit de la terre, prenant patience à son égard, jusqu'à ce qu'il ait reçu les pluies de la première et de la dernière saison.
\VS{8}Vous aussi, attendez patiemment, et affermissez vos cœurs, car l'avènement du Seigneur est proche.
\VS{9}Mes frères, ne vous plaignez pas les uns des autres, afin que vous ne soyez pas condamnés. Voici, le Juge se tient à la porte.
\VS{10}Mes frères, prenez pour exemple de patience dans les afflictions les prophètes qui ont parlé au Nom du Seigneur.
\VS{11}Voici, nous tenons pour bienheureux ceux qui ont enduré l'épreuve avec patience. Vous avez appris quelle a été la patience de Job, et vous avez vu la fin du Seigneur, car le Seigneur est plein de compassion et de miséricorde.
\VS{12}Avant toutes choses, mes frères, ne jurez ni par le ciel, ni par la terre, ni par aucun autre serment. Mais que votre oui soit oui, et que votre non soit non, afin que vous ne tombiez pas sous le jugement\FTNT{Mt. 5:37 ; Mt. 12:36.}.
\VS{13}Quelqu'un parmi vous est-il dans la souffrance ? Qu'il prie. Quelqu'un est-il dans la joie ? Qu'il chante.
\VS{14}Quelqu'un parmi vous est-il malade ? Qu'il appelle les anciens de l'église, et qu'ils prient pour lui en l'oignant d'huile au Nom du Seigneur.
\VS{15}Et la prière faite avec foi sauvera le malade, et le Seigneur le relèvera ; et s'il a commis des péchés, ils lui seront pardonnés.
\VS{16}Confessez donc vos péchés les uns les autres, et priez les uns pour les autres afin que vous soyez guéris. Car la prière du juste faite avec ferveur est de grande efficacité.
\VS{17}Elie était un homme sujet aux mêmes infirmités que nous, et cependant il pria avec instance pour qu'il ne pleuve point, et il ne tomba point de pluie sur la terre pendant trois ans et six mois\FTNT{1 R. 17:1.}.
\VS{18}Puis il pria de nouveau, et le ciel donna de la pluie, et la terre produisit son fruit.
\TextTitle{Conclusion}
\VS{19}Mes frères, si quelqu'un parmi vous s'est égaré loin de la vérité, et qu'un autre l'y ramène,
\VS{20}qu'il sache que celui qui ramènera un pécheur de son égarement, sauvera une âme de la mort et couvrira une multitude de péchés.
\PPE{}
\end{multicols}

%\clearpage\ShortTitle{Ga.}\BookTitle{Galates}\BFont
\noindent\hrulefill
{\footnotesize
\textit{
\bigskip
{\centering{}
\\Auteur~: Paul
\\Thème~: Le salut par la grâce
\\Date de rédaction~: Env. 50 ap. J.-C.\\}
}
\textit{
\\Province antique de l'Asie Mineure, la Galatie se situait en Anatolie. Elle devait son nom aux Galates, Celtes provenant des Balkans.
\\La lettre de Paul aux Galates est la seule épître dont le début ne contient pas de témoignage d'affection. Paul commence par justifier l'origine de son appel, en employant un ton sec et sévère. Les Galates, qu'il avait lui-même évangélisés lors de son premier voyage, s'étaient promptement détournés de l'Evangile qu'ils avaient reçu. Ils ne l'avaient pas totalement abandonné, mais y avaient ajouté ce qui ne leur avait point été prescrit. Troublés par les enseignements des judaïsants - des juifs ayant cru en Jésus-Christ, mais persistant toujours dans la pratique de la loi - les Galates avaient repris à leur compte leurs traditions, annihilant ainsi l'œuvre de la croix. Par cette lettre, Paul les exhorte d'une part à revenir à l'Evangile véritable et d'autre part à marcher par l'Esprit afin d'en porter le fruit.\bigskip
}
}
\par\nobreak\noindent\hrulefill
\begin{multicols}{2}
\Chap{1}
\TextTitle{Introduction}
\VerseOne{}Paul, apôtre, non de la part des hommes, ni de la part d'aucun homme, mais de la part de Jésus-Christ, et de la part de Dieu le Père, qui l'a ressuscité des morts,
\VS{2}et tous les frères qui sont avec moi, aux églises de Galatie\FTNT{La Galatie, ou Gallo-Grèce, était une province de l'Asie Mineure (région de la Turquie actuelle). Au nord, elle était délimitée par la Bithynie et la
Paphlagonie, à l'est par le Pont et la Cappadoce, au sud par la Cappadoce, la Lycaonie et la Phrygie, et à l'ouest par la Phrygie et la Bithynie. Son nom vient des Gaulois qui s'étaient installés dans la région en 279 av. J.-C. Conquise par les Romains en 189 av. J.-C., elle devint une province de l'Empire en 25 av. J.-C.} :
\VS{3}Que la grâce et la paix vous soient données de la part de Dieu le Père, et de la part de notre Seigneur Jésus-Christ,
\VS{4}qui s'est donné lui-même pour nos péchés, afin de nous arracher du présent siècle mauvais, selon la volonté de Dieu notre Père.
\VS{5}A lui soit la gloire aux siècles des siècles. Amen~!
\TextTitle{Les Galates se détournent de l'Evangile véritable}
\VS{6}Je m'étonne que vous abandonniez si promptement celui qui vous avait appelés à la grâce de Christ, pour passer à un autre évangile. 
\VS{7}Non qu'il y ait un autre évangile, mais il y a des gens qui vous troublent, et qui veulent renverser l'Evangile de Christ.
\VS{8}Mais quand nous-mêmes, ou quand un ange venu du ciel vous évangéliserait, outre\FTNT{Voir 1 R. 13:11-34.} ce que nous vous avons évangélisé, qu'il soit anathème~!
\VS{9}Comme nous l'avons déjà dit, je le dis encore maintenant~: Si quelqu'un vous évangélise outre ce que vous avez reçu, qu'il soit anathème~!
\TextTitle{Paul reçoit la révélation de l'Evangile}
\VS{10}Car est-ce les hommes que je prêche ou Dieu~? Ou est-ce que je cherche à plaire aux hommes~? Certes si je plaisais encore aux hommes, je ne serais pas le serviteur de Christ.
\VS{11}Je vous le déclare donc, mes frères, que l'Evangile que j'ai annoncé n'est pas selon l'homme,
\VS{12}parce que je ne l'ai ni reçu ni appris d'aucun homme, mais par la révélation de Jésus-Christ.
\VS{13}Car vous avez appris quelle a été autrefois ma conduite dans le judaïsme, et comment je persécutais à outrance l'Eglise de Dieu et la ravageais,
\VS{14}et comment j'étais plus avancé dans le judaïsme que beaucoup de ceux de mon âge et de ma nation, étant le plus ardent zélateur des traditions de mes pères.
\VS{15}Mais quand il a plu à Dieu, qui m'avait choisi dès le ventre de ma mère, et qui m'a appelé par sa grâce,
\VS{16}de révéler en moi son Fils, afin que je le prêche parmi les Gentils, aussitôt, je ne consultai ni la chair ni le sang,
\VS{17}et je ne montai point à Jérusalem vers ceux qui furent apôtres avant moi, mais je partis pour l'Arabie, puis je revins encore à Damas.
\VS{18}Ensuite, trois ans après, je montai à Jérusalem pour visiter Pierre, et je demeurai chez lui quinze jours.
\VS{19}Et je ne vis aucun des autres apôtres, sinon Jacques, le frère du Seigneur.
\VS{20}Or, dans les choses que je vous écris, voici, devant Dieu je vous dis que je ne mens point.
\VS{21}J'allai ensuite dans les pays de Syrie et de Cilicie.
\VS{22}Or j'étais inconnu de visage aux églises de Judée qui sont en Christ,
\VS{23}mais elles avaient seulement entendu dire~: Celui qui autrefois nous persécutait, annonce maintenant la foi qu'il détruisait autrefois.
\VS{24}Et elles glorifiaient Dieu à cause de moi.
\Chap{2}
\TextTitle{Paul et Barnabas se rendent à Jérusalem\FTNTT{Ac. 15.}}
\VerseOne{}Quatorze ans après, je montai de nouveau à Jérusalem\FTNT{La grande assemblée de Jérusalem. Voir Ac. 15.}, avec Barnabas, et je pris aussi avec moi Tite.
\VS{2}Et ce fut d'après une révélation que j'y montai. J'exposai l'Evangile que je prêche parmi les Gentils à ceux de Jérusalem, en particulier à ceux qui sont les plus considérés, afin de ne pas courir ou avoir couru en vain.
\VS{3}Et même on n'obligea pas Tite, qui était avec moi, de se faire circoncire quoiqu'il fût Grec.
\VS{4}Et cela à cause des faux frères qui s'étaient furtivement introduits et glissés dans l'église pour épier la liberté que nous avons en Jésus-Christ, afin de nous ramener dans la servitude.
\VS{5}Nous ne leur cédâmes pas un instant et nous résistâmes à leurs exigences, afin que la vérité de l'Evangile soit maintenue parmi vous.
\VS{6}Et je ne suis différent en rien de ceux qui sont les plus estimés, quels qu'ils aient été autrefois, Dieu n'ayant point d'égard à l'apparence extérieure de l'homme car ceux qui sont en estime ne m'ont rien communiqué de plus.
\VS{7}Au contraire, quand ils virent que la prédication de l'Evangile pour les incirconcis m'avait été confiée, comme à Pierre pour les circoncis,
\VS{8}car celui qui a opéré avec efficacité par Pierre dans la charge d'apôtre pour les circoncis, a aussi opéré avec efficacité par moi envers les Gentils.
\VS{9}Jacques, dis-je, Céphas, et Jean, qui sont estimés comme des colonnes, ayant reconnu la grâce que j'avais reçue, me donnèrent, à moi et à Barnabas, la main d'association, afin que nous allions, nous vers les Gentils, et qu'ils aillent eux vers les circoncis.
\VS{10}Ils nous recommandèrent seulement de nous souvenir des pauvres, ce que j'ai eu bien soin de faire.
\TextTitle{Paul reprend Pierre à Antioche}
\VS{11}Mais lorsque Pierre vint à Antioche, je lui résistai en face parce qu'il méritait d'être repris.
\VS{12}Car avant l'arrivée de quelques personnes envoyées par Jacques, il mangeait avec les Gentils, mais quand elles furent venues, il s'esquiva et se sépara des Gentils, craignant les circoncis.
\VS{13}Les autres Juifs aussi usèrent de dissimulation comme lui, de sorte que Barnabas même se laissait entraîner par leur hypocrisie.
\VS{14}Mais quand je vis qu'ils ne marchaient pas droit selon la vérité de l'Evangile, je dis à Pierre devant tous~: Si toi qui es Juif, tu vis comme les Gentils, et non pas comme les Juifs, pourquoi contrains-tu les Gentils à judaïser~?
\TextTitle{Le chrétien est mort à la loi mosaïque}
\VS{15}Nous qui sommes Juifs de naissance, et non point pécheurs d'entre les Gentils,
\VS{16}sachant que l'homme n'est pas justifié par les œuvres de la loi, mais seulement par la foi en Jésus-Christ\FTNT{La justification. Voir Ro. 5:1.}, nous, dis-je, nous avons cru en Jésus-Christ, afin que nous soyons justifiés par la foi en Christ, et non point par les œuvres de la loi~; parce que personne ne sera justifié par les œuvres de la loi.
\VS{17}Or si en cherchant à être justifiés par Christ, nous sommes aussi trouvés pécheurs, Christ est-il pourtant serviteur du péché~? A Dieu ne plaise~!
\VS{18}Car si je rebâtis les choses que j'ai renversées, je montre que je suis moi-même un transgresseur.
\VS{19}Car c'est par la loi que je suis mort à la loi, afin de vivre pour Dieu.
\TextTitle{La vie chrétienne doit refléter la vie de Jésus-Christ\FTNTT{Ga. 5:15-23.}}
\VS{20}Je suis crucifié avec Christ~; et si je vis, ce n'est plus moi qui vis, c'est Christ qui vit en moi~; si je vis maintenant dans la chair, je vis dans la foi au Fils de Dieu, qui m'a aimé et qui s'est livré lui-même pour moi.
\VS{21}Je n'anéantis point la grâce de Dieu, car si la justice vient de la loi, Christ est donc mort inutilement.
\Chap{3}
\TextTitle{L'Esprit s'acquiert par la foi}
\VerseOne{}Ô Galates insensés~! Qui vous a ensorcelés pour faire que vous n'obéissiez point à la vérité, vous, aux yeux de qui Jésus-Christ a été auparavant dépeint crucifié, au milieu de vous~?
\VS{2}Je voudrais seulement entendre ceci de vous~: Avez-vous reçu l'Esprit par les œuvres de la loi, ou par la prédication de la foi~?
\VS{3}Etes-vous si insensés, qu'après avoir commencé par l'Esprit, voulez-vous maintenant finir par la chair~?
\VS{4}Avez-vous tant souffert en vain~? Si toutefois c'est en vain.
\VS{5}Celui donc qui vous donne l'Esprit, et qui produit en vous les dons miraculeux, le fait-il par les œuvres de la loi ou par la prédication de la foi~?
\TextTitle{L'alliance avec Abraham, une promesse fondée sur la foi\FTNTT{Ro. 4.}}
\VS{6}Comme Abraham crut à Dieu, et cela lui fut imputé à justice,
\VS{7}sachez donc que ce sont ceux qui ont la foi qui sont fils d'Abraham.
\VS{8}Aussi, l'Ecriture prévoyant que Dieu justifierait les Gentils par la foi, a auparavant évangélisé à Abraham, en lui disant~: Toutes les nations seront bénies en toi\FTNT{Ge. 12:3.}.
\VS{9}C'est pourquoi ceux qui ont la foi sont bénis avec Abraham, le croyant.
\TextTitle{L'attachement aux œuvres de la loi produit la malédiction}
\VS{10}Car tous ceux qui s'attachent aux œuvres de la loi sont sous la malédiction~; car il est écrit~: Maudit est quiconque ne persévère pas dans toutes les choses qui sont écrites dans le livre de la loi et ne les met pas en pratique\FTNT{De. 27:26.}.
\VS{11}Et que nul ne soit justifié devant Dieu par la loi, cela est évident, puisqu'il est dit~: Le juste vivra de la foi\FTNT{Ha. 2:4.}.
\VS{12}Or la loi ne procède pas de la foi, mais elle dit~: L'homme qui mettra ces choses en pratique vivra par elles\FTNT{Lé. 18:5.}.
\TextTitle{Le Messie a racheté les chrétiens de la malédiction de la loi}
\VS{13}Christ nous a rachetés de la malédiction de la loi quand il a été fait malédiction pour nous~; car il est écrit~: Maudit est quiconque est pendu au bois\FTNT{De. 21:23.},
\VS{14}afin que la bénédiction d'Abraham ait son accomplissement pour les Gentils en Jésus-Christ, et que nous recevions par la foi l'Esprit qui avait été promis.
\VS{15}Mes frères, je parle à la manière des hommes, un testament en bonne forme, bien que fait par un homme, n'est annulé par personne, et personne n'y ajoute.
\VS{16}Or les promesses ont été faites à Abraham et à sa postérité. Il n'est pas dit~: Et aux postérités, comme s'il avait parlé de plusieurs, mais comme parlant d'une seule, et à sa postérité, c'est-à-dire Christ.
\VS{17}Voici ce que j'entends~: Une alliance, que Dieu a confirmée antérieurement, ne peut pas être annulée, et ainsi la promesse rendue vaine, par la loi survenue quatre cent trente ans plus tard.
\VS{18}Car si l'héritage venait de la loi, il ne viendrait plus de la promesse. Or c'est par la promesse que Dieu a fait à Abraham ce don de sa grâce.
\TextTitle{La loi~: Pédagogue révélant le péché et conduisant à Christ}
\VS{19}A quoi donc sert la loi~? Elle a été donnée ensuite à cause des transgressions, jusqu'à ce que vienne la postérité à qui la promesse avait été faite~; et elle a été promulguée par des anges, au moyen d'un médiateur.
\VS{20}Or le médiateur n'est pas médiateur d'un seul, mais Dieu est un seul.
\VS{21}La loi a-t-elle donc été ajoutée contre les promesses de Dieu~? Nullement ! Car s'il avait été donné une loi qui puisse procurer la vie, la justice viendrait réellement de la loi.
\VS{22}Mais l'Ecriture a renfermé tous les hommes sous le péché, afin que ce qui avait été promis soit donné par la foi en Jésus-Christ à ceux qui croient.
\VS{23}Or avant que la foi vienne, nous étions renfermés sous la garde de la loi, en vue de la foi qui devait être révélée.
\VS{24}Ainsi la loi a donc été notre pédagogue\FTNT{Le mot «~pédagogue~» du grec «~paidagogos~»~: «~celui qui dirige un garçon~». Un pédagogue était un tuteur, un gardien et un guide de garçons. Parmi les Grecs et les Romains, le mot était appliqué aux esclaves dignes de confiance qui étaient chargés de veiller à la vie et à la moralité des garçons appartenant aux classes supérieures. Les garçons ne pouvaient faire le moindre pas hors de la maison sans ces tuteurs tant qu'ils n'avaient pas atteint leur majorité.} pour nous amener à Christ, afin que nous soyons justifiés par la foi.
\VS{25}Mais la foi étant venue, nous ne sommes plus sous ce pédagogue.
\TextTitle{Ceux qui croient au Messie sont justifiés}
\VS{26}Parce que vous êtes tous fils de Dieu par la foi en Jésus-Christ,
\VS{27}car vous tous qui avez été baptisés en Christ, vous avez revêtu Christ.
\VS{28}Il n'y a plus ni Juif ni Grec, il n'y a plus ni esclave ni libre, il n'y a plus ni homme ni femme~; car vous êtes tous un en Jésus-Christ\FTNT{Ro. 10:12~; Col. 3:11.}.
\VS{29}Et si vous êtes de Christ, vous êtes donc la postérité d'Abraham, et héritiers selon la promesse.
\Chap{4}
\VerseOne{}Or aussi longtemps que l'héritier est enfant\FTNT{«~Enfant~», du grec «~nepios~», signifie aussi «~ignorant~».}, je dis qu'il ne diffère en rien d'un esclave, quoiqu'il soit le maître de tout.
\VS{2}Mais il est sous des tuteurs et des administrateurs jusqu'au temps déterminé par le Père.
\VS{3}Nous aussi, lorsque nous étions enfants, nous étions sous l'esclavage des rudiments du monde.
\VS{4}Mais lorsque les temps ont été accomplis, Dieu a envoyé son Fils, né d'une femme, né sous la loi,
\VS{5}afin qu'il rachète ceux qui étaient sous la loi, afin que nous recevions l'adoption.
\VS{6}Et parce que vous êtes fils, Dieu a envoyé l'Esprit de son Fils dans vos cœurs, lequel crie~: Abba ! C'est-à-dire Père.
\VS{7}Maintenant donc tu n'es plus esclave, mais fils~; or si tu es fils, tu es aussi héritier de Dieu par Christ.
\TextTitle{Le légalisme et la religiosité privent de la grâce}
\VS{8}Autrefois, ne connaissant pas Dieu, vous serviez des dieux qui ne le sont pas de leur nature.
\VS{9}Et maintenant que vous avez connu Dieu, ou plutôt que vous avez été connus de Dieu, comment retournez-vous encore à ces faibles et misérables éléments, auxquels vous voulez encore vous asservir comme auparavant~?
\VS{10}Vous observez les jours, les mois, les temps et les années.
\VS{11}Je crains d'avoir travaillé inutilement pour vous.
\VS{12}Soyez comme moi~; car je suis aussi comme vous~; je vous en prie mes frères.
\VS{13}Vous ne m'avez fait aucun tort. Et vous savez que ce fut à cause d'une infirmité de la chair\FTNT{Les Ecritures ne donnent pas de précisions au sujet de l'infirmité de la chair dont souffrait Paul. On suppose toutefois qu'il avait un handicap au niveau de ses yeux. Quatre arguments viennent renforcer cette hypothèse. Tout d'abord, l'allusion de Paul aux Galates qui étaient prêts à «~s'arracher les yeux~» pour les lui donner (Ga. 4:15) et le fait qu'il ait lui-même écrit cette épître avec de «~grandes lettres~» (Ga. 6:11). Ensuite, lors de sa comparution devant le sanhédrin à Jérusalem, Paul n'a pas reconnu le grand-prêtre pourtant facilement identifiable par sa tenue vestimentaire (Ac. 23:5). Enfin, l'apôtre avait l'habitude de dicter ses lettres, ce qui constitue un argument majeur. L'épître aux Galates était une exception parce qu'il n'avait sans doute pas de secrétaire à disposition.} que je vous ai pour la première fois évangélisés.
\VS{14}Et vous ne m'avez point méprisé ni rejeté à cause de ces épreuves que j'ai dans ma chair~; mais vous m'avez reçu comme un ange de Dieu, et comme Jésus-Christ.
\VS{15}Où donc est l'expression de votre bonheur~? Car je vous atteste que, si cela avait été possible, vous vous seriez arrachés les yeux pour me les donner.
\VS{16}Suis-je donc devenu votre ennemi en vous disant la vérité~?
\VS{17}Ils ont du zèle pour vous, mais non loyalement. Au contraire, ils veulent vous détacher de nous afin que vous soyez zélés pour eux.
\VS{18}Il est bon d'être zélé pour le bien en tout temps, et non pas seulement quand je suis présent parmi vous.
\TextTitle{La loi et la grâce ne peuvent cohabiter~: Agar et Sara représentent deux alliances}
\VS{19}Mes petits enfants, pour qui j'éprouve de nouveau les douleurs de l'enfantement, jusqu'à ce que Christ soit formé en vous,
\VS{20}je voudrais être maintenant avec vous, et changer de langage, car je suis dans une grande inquiétude à votre sujet.
\VS{21}Dites-moi, vous qui voulez être sous la loi, ne comprenez-vous point la loi~?
\VS{22}Car il est écrit qu'Abraham eut deux fils, un de l'esclave, et un de la femme libre.
\VS{23}Mais celui de l'esclave naquit selon la chair~; et celui de la femme libre naquit en vertu de la promesse.
\VS{24}Ces faits ont une valeur allégorique, car ces deux femmes sont deux alliances~: L'une du Mont Sinaï, qui n'enfante que des esclaves, et c'est Agar.
\VS{25}Car le nom d'Agar veut dire Sinaï, qui est une montagne en Arabie correspondant à la Jérusalem actuelle qui est dans la servitude avec ses enfants.
\VS{26}Mais la Jérusalem d'en haut est la femme libre, et c'est notre mère à nous tous.
\VS{27}Car il est écrit~: Réjouis-toi, stérile, toi qui n'enfantes point~! Eclate et pousse des cris, toi qui n'as pas éprouvé les douleurs de l'enfantement~! Car les enfants de la délaissée seront plus nombreux que les enfants de celle qui était mariée\FTNT{Es. 54:1.}.
\VS{28}Or pour nous, mes frères, nous sommes enfants de la promesse comme Isaac.
\VS{29}Et de même qu'alors, celui qui était né selon la chair persécutait celui qui était né selon l'Esprit, il en est de même maintenant.
\VS{30}Mais que dit l'Ecriture~? Chasse l'esclave et son fils, car le fils de l'esclave n'héritera pas avec le fils de la femme libre\FTNT{Ge. 21:10.}.
\VS{31}C'est pourquoi, mes frères, nous ne sommes pas enfants de l'esclave, mais de la femme libre.
\Chap{5}
\TextTitle{Le Messie nous a libérés de la servitude}
\VerseOne{}Demeurez donc fermes dans la liberté pour laquelle Christ nous a affranchis, et ne vous mettez plus sous le joug de la servitude.
\VS{2}Moi, Paul, je vous dis que si vous vous faites circoncire, Christ ne vous servira à rien.
\VS{3}Et j'affirme encore une fois à tout homme qui se fait circoncire qu'il est tenu de pratiquer la loi tout entière.
\VS{4}Vous êtes séparés de Christ, vous tous qui cherchez la justification dans la loi~; vous êtes déchus de la grâce.
\VS{5}Mais pour nous, nous attendons par l'Esprit l'espérance d'être justifiés par la foi.
\VS{6}Car en Jésus-Christ ni la circoncision ni le prépuce\FTNT{Voir le commentaire en 1 Co. 7:18.} n'ont de valeur, mais seulement la foi qui opère par la charité.
\VS{7}Vous couriez bien~: Qui vous a arrêtés pour vous empêcher d'obéir à la vérité~?
\VS{8}Cette influence ne vient pas de celui qui vous appelle.
\VS{9}Un peu de levain fait lever toute la pâte\FTNT{1 Co. 5:6.}.
\VS{10}J'ai cette confiance en vous dans le Seigneur que vous n'aurez pas d'autre sentiment~; mais celui qui vous trouble, quel qu'il soit, en portera la condamnation.
\VS{11}Quant à moi, mes frères, si je prêche encore la circoncision, pourquoi suis-je encore persécuté~? Le scandale de la croix est donc aboli.
\VS{12}Plaise à Dieu que ceux qui vous troublent soient retranchés~!
\VS{13}Car, mes frères, vous avez été appelés à la liberté, seulement ne faites pas de cette liberté une occasion de vivre selon la chair, mais servez-vous les uns les autres avec charité.
\VS{14}Car toute la loi est accomplie dans cette seule parole~: Tu aimeras ton prochain comme toi-même\FTNT{Lé. 19:18~; Mt. 22:39.}.
\VS{15}Mais si vous vous mordez et vous dévorez les uns les autres, prenez garde que vous ne soyez détruits les uns par les autres.
\VS{16}Je vous dis donc~: Marchez selon l'Esprit, et vous n'accomplirez point les désirs de la chair.
\TextTitle{La chair et ses œuvres s'opposent à l'Esprit de Dieu\FTNTT{Ro. 8:2.}}
\VS{17}Car la chair a des désirs contraires à ceux de l'Esprit, et l'Esprit en a des contraires à ceux de la chair~; et ils sont opposés entre eux afin que vous ne fassiez point ce que vous voudriez.
\VS{18}Or si vous êtes conduits par l'Esprit, vous n'êtes point sous la loi.
\VS{19}Car les œuvres de la chair sont évidentes~: Ce sont l'adultère, la fornication, l'impureté, l'impudicité,
\VS{20}l'idolâtrie, la sorcellerie\FTNT{La sorcellerie~: du grec «~pharmakeia~»~: «~usage~» ou «~administration~» de drogues, «~empoisonnement~», «~sorcellerie~», «~arts magiques~», souvent trouvés en liaison avec l'idolâtrie et nourrie par celle-ci.}, les inimitiés, les querelles, les jalousies, les animosités, les disputes, les divisions, les sectes,
\VS{21}les envies, les meurtres, l'ivrognerie, les excès de table, et les choses semblables à celles-là, au sujet desquelles je vous prédis, comme je vous l'ai déjà dit, que ceux qui commettent de telles choses n'hériteront point le Royaume de Dieu.
\TextTitle{Le fruit de l'Esprit\FTNTT{Jn. 15:1-5~; Ga. 2:20.}}
\VS{22}Mais le fruit de l'Esprit c'est la charité\FTNT{Il est question ici de l'amour «~agape~»~: l'amour fraternel, la charité désintéressée.}, la joie, la paix, la patience, la bonté, la bienveillance, la foi, la douceur, la tempérance.
\VS{23}La loi n'est pas contre ces choses.
\VS{24}Ceux qui sont à Christ ont crucifié la chair avec ses passions et ses désirs.
\VS{25}Si nous vivons par l'Esprit, marchons aussi par l'Esprit.
\VS{26}Ne cherchons pas une vaine gloire, en nous provoquant les uns les autres et en nous portant envie les uns aux autres.
\Chap{6}
\TextTitle{La mise en pratique de la vie nouvelle en Jésus-Christ}
\VerseOne{}Mes frères, lorsqu'un homme est surpris en quelque faute, vous qui êtes spirituels, redressez-le avec un esprit de douceur. Prends garde à toi-même, de peur que tu ne sois aussi tenté.
\VS{2}Portez les fardeaux les uns des autres, et vous accomplirez ainsi la loi de Christ.
\VS{3}Car si quelqu'un pense être quelque chose, quoiqu'il ne soit rien, il s'abuse lui-même.
\VS{4}Que chacun examine ses propres œuvres, et alors il aura de quoi se glorifier pour lui-même seulement, et non par rapport aux autres.
\VS{5}Car chacun portera son propre fardeau.
\VS{6}Que celui à qui l'on enseigne la parole fasse part en tous biens à celui qui l'enseigne\FTNT{Le mot «~bien~» vient du grec «~agathos~» qui donne en français~: «~de bonne constitution ou nature~», «~utile~», «~salutaire~», «~bon~», «~agréable~», «~plaisant~», «~joyeux~», «~heureux~», «~excellent~», «~distingué~», «~droit~», «~honorable~» et n'a rien à voir avec les biens matériels (Voir Ga. 6:10). Il ne doit en aucun cas servir de prétexte à ceux qui enseignent la Parole de Dieu pour exiger l'argent et les biens matériels des chrétiens. Ces derniers doivent donner sans contrainte, s'ils le veulent et comme ils le veulent (2 Co. 9:7). Le salaire de l'ouvrier du Seigneur c'est avant tout le gîte et le couvert (Mt. 10:10~; Lu. 10:8~; 1 Ti. 6:8). Ainsi, malgré le droit qu'il avait de moissonner les biens matériels pour avoir semé des biens spirituels (1 Co. 9:11-12), Paul «~n'a désiré ni l'or ni l'argent~» mais a travaillé de ses propres mains afin de pourvoir à ses besoins et de n'être à la charge de personne (Ac. 20:33-35~; 1 Th. 2:9~; 2 Th. 3:8~; 2 Co. 12:14 ).}.
\VS{7}Ne vous séduisez pas, on ne se moque pas de Dieu. Ce qu'un homme aura semé, il le moissonnera aussi.
\VS{8}C'est pourquoi celui qui sème pour sa chair moissonnera de la chair la corruption~; mais celui qui sème pour l'Esprit moissonnera de l'Esprit la vie éternelle.
\VS{9}Ne nous lassons pas de faire le bien~; car nous moissonnerons au temps convenable, si nous ne nous relâchons pas.
\VS{10}C'est pourquoi, pendant que nous en avons le temps, faisons du bien envers tous, mais principalement envers ceux qui sont de la famille de la foi.
\VS{11}Vous voyez avec quelles grandes lettres je vous ai écrit de ma propre main.
\VS{12}Tous ceux qui veulent se rendre agréables selon la chair vous contraignent à vous faire circoncire, uniquement afin de ne pas être persécutés pour la croix de Christ.
\VS{13}Car les circoncis eux-mêmes n'observent pas la loi~; mais ils veulent que vous soyez circoncis pour se glorifier dans votre chair.
\VS{14}Pour ce qui me concerne, loin de moi la pensée de me glorifier d'autre chose que de la croix de notre Seigneur Jésus-Christ, par qui le monde est crucifié pour moi, comme je le suis pour le monde~!
\VS{15}Car ce n'est rien que d'être circoncis ou incirconcis~; ce qui est quelque chose c'est d'être une nouvelle créature.
\VS{16}Que la paix et la miséricorde soient sur tous ceux qui suivront cette règle, et sur l'Israël de Dieu !
\TextTitle{Conclusion}
\VS{17}Au reste, que personne ne me fasse de la peine, car je porte sur mon corps les marques du Seigneur Jésus.
\VS{18}Mes frères, que la grâce de notre Seigneur Jésus-Christ soit avec votre esprit~! Amen~!
\PPE{}
\end{multicols}

%\clearpage\ShortTitle{1 Th.}\BookTitle{1 Thessaloniciens}\BFont
\noindent\hrulefill
{\footnotesize
\textit{
\bigskip
{\centering{}
\\Auteur~: Paul
\\Thème~: Le retour de Christ
\\Date de rédaction~: Env. 51 ap. J.-C.\\}
}
\textit{
\\Autrefois appelée Therme ou Therma, qui signifie «~source chaude~», Thessalonique reçut son nouveau nom de Cassandre, en l'honneur de sa femme Thessalonike, qui était aussi la sœur d'Alexandre le Grand (356 av. J.-C. - 323 av. J.-C.), à qui il succéda. Cette ville est située au nord de la Grèce actuelle, sur la côte de la mer Egée. Du temps de Paul, ce pays était divisé en deux parties. Dans la région du nord, la Macédoine, se trouvaient les villes de Philippes, Thessalonique et Bérée. Quant à la région du sud, l'Achaïe, elle comportait les villes d'Athènes et de Corinthe. Aujourd'hui, la ville s'appelle Salonique.
\\En ce temps-là, Thessalonique comptait environ 200 000 habitants (Grecs, Romains et Juifs) et jouissait d'une importante fréquentation puisqu'elle figurait parmi les trois ports principaux de la Méditerranée et se situait sur l'une des plus grandes routes commerciales de l'époque~: La Voie Egnatienne reliant Rome à Byzance.
\\Sur le plan religieux, les habitants étaient polythéistes et pratiquaient une variété de cultes, dont le culte impérial. Durant trois semaines, Paul enseigna dans une synagogue à Thessalonique et réussit à constituer un groupe de croyants composé de Juifs, de Gentils, de pauvres et de plusieurs femmes de la haute société. Toutefois, une violente persécution l'obligea à quitter promptement la ville, laissant la communauté nouvellement formée vulnérable et fragile.
\\La première épître adressée par Paul aux Thessaloniciens leur parvint quelques mois après le passage de l'équipe apostolique et après la visite de Timothée. Cette lettre avait pour but d'affermir les Thessaloniciens dans les vérités fondamentales qui leur avaient été enseignées, de les exhorter à vivre une vie de sainteté pour être agréables à Dieu, de les éclairer quant au devenir des défunts et de les assurer du retour certain du Seigneur.\bigskip
}
}
\par\nobreak\noindent\hrulefill
\begin{multicols}{2}
\Chap{1}
\TextTitle{Introduction}
\VerseOne{}Paul, et Silvain, et Timothée, à l'église des Thessaloniciens qui est en Dieu le Père, et en Jésus-Christ, notre Seigneur~: Que la grâce et la paix vous soient données de la part de Dieu notre Père, et du Seigneur Jésus-Christ~!
\VS{2}Nous rendons toujours grâces à Dieu pour vous tous, faisant mention de vous dans nos prières,
\VS{3}en nous rappelant sans cesse l'œuvre de votre foi, le travail de votre charité, et l'immuabilité de votre espérance en notre Seigneur Jésus-Christ devant notre Dieu et Père,
\VS{4}sachant, mes frères bien-aimés de Dieu, votre élection.
\TextTitle{Proclamation de l'Evangile avec puissance et avec l'Esprit Saint}
\VS{5}Car notre Evangile ne vous a pas été prêché en paroles seulement, mais aussi en puissance, avec l'Esprit Saint, et avec une pleine persuasion~; car vous n'ignorez pas que nous nous sommes montrés ainsi parmi vous, à cause de vous.
\VS{6}Aussi avez-vous été nos imitateurs et ceux du Seigneur, ayant reçu avec la joie du Saint-Esprit, la parole au milieu de grandes afflictions,
\VS{7}de sorte que vous avez été des modèles à tous les fidèles de la Macédoine\FTNT{La Macédoine était le pays natal d'Alexandre le Grand. Elle fut conquise par les Romains et devint une province romaine, dont la capitale était Thessalonique.} et de l'Achaïe\FTNT{L'Achaïe était une province romaine placée sous l'autorité d'un proconsul résidant dans la capitale qui était Corinthe (2 Co. 1:1).}.
\VS{8}Car la parole du Seigneur a retenti de chez vous, non seulement dans la Macédoine et dans l'Achaïe mais aussi en tous lieux, et votre foi envers Dieu est si répandue, que nous n'avons pas besoin d'en parler.
\VS{9}Car eux-mêmes racontent de nous quel accès nous avons eu auprès de vous, et comment vous vous êtes convertis à Dieu en vous séparant des idoles, pour servir le Dieu vivant et vrai,
\VS{10}et pour attendre des cieux son Fils Jésus, qu'il a ressuscité des morts, et qui nous délivre de la colère à venir\FTNT{La colère à venir. Voir les sept coupes de la colère de Dieu (Ap. 15:5-8~; 16:1-21).}.
\Chap{2}
\TextTitle{Annoncer l'Evangile en recherchant l'approbation de Dieu et non celle des hommes}
\VerseOne{}Car, mes frères, vous savez vous-mêmes que notre entrée au milieu de vous n'a point été vaine. 
\VS{2}Après avoir souffert et reçu des outrages à Philippes\FTNT{Philippes était une ville de Macédoine située en Thrace, près de la côte nord de la mer Egée. Voir Ac. 16:12-40 et l'épître de Paul aux Philippiens.}, comme vous le savez, nous avons pris de l'assurance en notre Dieu, pour vous annoncer l'Evangile de Dieu au milieu de beaucoup de combats.
\VS{3}Car il n'y a eu dans notre prédication ni séduction, ni motif impur, ni fraude.
\VS{4}Mais comme Dieu nous a considérés dignes de nous confier la prédication de l'Evangile, ainsi nous parlons non comme pour plaire aux hommes, mais à Dieu qui éprouve nos cœurs.
\VS{5}Car, en effet, nous n'avons jamais été surpris avec des paroles flatteuses, comme vous le savez~; jamais nous n'avons eu pour prétexte la cupidité, Dieu en est témoin.
\VS{6}Et nous n'avons point cherché la gloire qui vient des hommes, ni de vous, ni des autres~; nous aurions pu nous imposer comme apôtres de Christ,
\VS{7}mais nous avons été doux au milieu de vous, de même qu'une nourrice chérit ses enfants.
\VS{8}Nous aurions voulu dans notre affection envers vous, non seulement vous donner l'Evangile de Dieu, mais encore notre propre vie, tant vous nous étiez devenus chers.
\VS{9}Car, mes frères, vous vous souvenez de notre peine et de notre travail~; vu que nous vous avons prêché l'Evangile de Dieu, en travaillant nuit et jour, pour n'être point à charge à aucun de vous.
\VS{10}Vous êtes témoins et Dieu aussi, combien notre conduite envers vous qui croyez a été sainte, juste, et irréprochable.
\VS{11}Et vous savez que nous avons exhorté chacun de vous, comme un père exhorte ses enfants,
\VS{12}en vous exhortant, vous encourageant et vous conjurant de vous conduire d'une manière digne de Dieu, qui vous appelle à son Royaume et à sa gloire.
\VS{13}C'est pourquoi nous rendons sans cesse grâces à Dieu, de ce que, quand vous avez reçu de nous la parole de la prédication de Dieu, vous l'avez reçue non comme une parole des hommes, mais ainsi qu'elle est véritablement, comme la parole de Dieu, laquelle aussi agit avec efficacité en vous qui croyez.
\VS{14}En effet, mes frères, vous êtes devenus les imitateurs des églises de Dieu qui sont en Jésus-Christ dans la Judée, parce que vous aussi, vous avez souffert de la part de ceux de votre propre nation les mêmes choses qu'elles ont souffertes de la part des Juifs,
\VS{15}qui ont même mis à mort le Seigneur Jésus, et leurs propres prophètes, qui nous ont persécutés, qui ne plaisent point à Dieu, et qui sont ennemis de tous les hommes,
\VS{16}nous empêchant de parler aux Gentils afin qu'ils soient sauvés, comblant ainsi toujours plus la mesure de leurs péchés. Mais la colère de Dieu est venue sur eux jusqu'au plus haut degré.
\VS{17}Pour nous, mes frères, après avoir été quelque temps séparés de vous de corps et non de cœur, nous avons eu d'autant plus d'ardeur et d'empressement de vous revoir.
\VS{18}Nous avons donc voulu, une et même deux fois, aller chez vous, au moins, moi Paul~; mais Satan nous en a empêchés.
\VS{19}Car quelle est notre espérance, ou notre joie, ou notre couronne de gloire~? N'est-ce pas vous qui l'êtes, devant notre Seigneur Jésus-Christ lors de son avènement~?
\VS{20}Certes, vous êtes notre gloire et notre joie.
\Chap{3}
\TextTitle{La persévérance des Thessaloniciens dans l'affliction}
\VerseOne{}C'est pourquoi ne pouvant plus soutenir la privation de vos nouvelles, nous avons trouvé bon de demeurer seuls à Athènes.
\VS{2}Et nous avons envoyé Timothée, notre frère, serviteur de Dieu, et notre compagnon d'œuvre dans l'Evangile de Christ, pour vous affermir et vous exhorter au sujet de votre foi,
\VS{3}afin que nul ne soit troublé dans ces afflictions, puisque vous savez vous-mêmes que nous sommes destinés à cela.
\VS{4}Et lorsque nous étions avec vous, nous vous annoncions d'avance que nous aurions à souffrir des afflictions, comme cela est aussi arrivé, et vous le savez.
\VS{5}C'est pourquoi, dis-je, ne pouvant plus soutenir cette inquiétude, j'ai envoyé Timothée pour connaître l'état de votre foi, de peur que le tentateur ne vous ait tentés en quelque sorte, et que nous n'ayons travaillé en vain.
\VS{6}Mais Timothée étant revenu depuis peu de chez vous, nous a apporté d'agréables nouvelles de votre foi et de votre charité, et nous a dit que vous conservez toujours un bon souvenir de nous, désirant nous voir, comme nous désirons aussi vous voir.
\VS{7}C'est pourquoi, mes frères, nous avons été consolés par votre foi, dans toutes nos afflictions et dans toutes nos détresses.
\VS{8}Car maintenant nous vivons puisque vous demeurez fermes dans le Seigneur.
\VS{9}Et quelles actions de grâces ne pouvons-nous pas rendre à Dieu à votre sujet, pour toute la joie que nous éprouvons devant notre Dieu, à cause de vous.
\VS{10}Nuit et jour, nous le prions avec une extrême ardeur de nous permettre de vous voir, et de compléter\FTNT{Compléter~: du grec «~katartizo~» qui signifie «~redresser~», «~ajuster~», «~compléter~», «~raccommoder~» (ce qui a été abîmé), «~réparer~». Ce verbe est également utilisé dans Mt. 4:21 lorsque Jacques et Jean réparaient leurs filets. Le terme «~katartismos~» traduit par «~perfectionnement~» dans Ep. 4:11 vient de ce verbe. Ainsi, l'un des rôles de ces services est le perfectionnement des saints et non leur destruction.} ce qui manque à votre foi.
\VS{11}Que Dieu lui-même, notre Père, et notre Seigneur Jésus-Christ, aplanisse\FTNT{Le verbe «~aplanir~» vient du grec «~kateuthuno~». On constate que ce verbe est conjugué au singulier, y compris dans le texte original grec, ce qui atteste l'unité entre le Père et le Fils (voir 2 Th. 2:16-17).} notre chemin pour que nous allions vers vous.
\VS{12}Et que le Seigneur vous fasse croître et abonder de plus en plus en charité les uns envers les autres, et envers tous, comme nous abondons aussi en charité envers vous~;
\VS{13}qu'il affermisse vos cœurs pour qu'ils soient irréprochables dans la sainteté, devant Dieu qui est notre Père, lors de l'avènement de notre Seigneur Jésus-Christ, accompagné de tous ses saints.
\Chap{4}
\TextTitle{Appel à la sanctification et à l'amour fraternel}
\VerseOne{}Au reste, mes frères, nous vous prions donc, et nous vous conjurons par le Seigneur Jésus, que comme vous avez appris de nous de quelle manière on doit se conduire, et plaire à Dieu, vous y fassiez tous les jours de nouveaux progrès.
\VS{2}Car vous savez quels préceptes nous vous avons donnés de la part du Seigneur Jésus.
\VS{3}Parce que c'est ici la volonté de Dieu~; savoir votre sanctification\FTNT{La sanctification personnelle (1 Pi. 1:15-18~; Hé. 12:14~; Ap. 22:11). Chaque chrétien doit fournir un effort, en se servant quotidiennement de la Parole de Dieu et de la prière, pour se maintenir dans la sanctification. Cela implique la séparation d'avec le mal et des mauvaises compagnies (2 Co. 6:14-18). Elle se développe au prix de nombreuses souffrances et de multiples sacrifices (Ro. 12:1-3).}, et que vous vous absteniez de la fornication,
\VS{4}c'est que chacun de vous sache posséder son corps dans la sanctification et dans l'honneur,
\VS{5}et sans se laisser aller aux désirs de la convoitise, comme les Gentils qui ne connaissent point Dieu.
\VS{6}Que personne n'use de fraude envers son frère et de cupidité dans les affaires, parce que le Seigneur tire vengeance de toutes ces choses, comme nous vous l'avons dit et attesté.
\VS{7}Car Dieu ne nous a pas appelés à l'impureté, mais à la sanctification.
\VS{8}C'est pourquoi celui qui rejette ceci ne rejette pas un homme, mais Dieu qui a aussi donné son Saint-Esprit.
\VS{9}Quant à la charité fraternelle\FTNT{Le mot grec employé ici est «~philadelphia~». Ce terme désigne l'amour fraternel, l'amour que les chrétiens se portent entre eux.}, vous n'avez pas besoin que je vous en écrive~; car vous-mêmes vous êtes enseignés de Dieu à vous aimer les uns les autres,
\VS{10}et c'est aussi ce que vous faites à l'égard de tous les frères qui sont dans toute la Macédoine. Mais, mes frères, nous vous prions de vous perfectionner tous les jours davantage,
\VS{11}et de tâcher de vivre paisiblement~; de faire vos propres affaires, et de travailler de vos propres mains, ainsi que nous vous l'avons ordonné.
\VS{12}En sorte que vous vous conduisiez honnêtement envers ceux du dehors, et que vous n'ayez besoin de rien.
\TextTitle{L'enlèvement de l'Eglise}
\VS{13}Or, mes frères, je ne veux pas que vous soyez dans l'ignorance au sujet de ceux qui dorment, afin que vous ne soyez point attristés comme les autres qui n'ont point d'espérance. 
\VS{14}Car si nous croyons que Jésus est mort, et qu'il est ressuscité~; de même aussi ceux qui dorment en Jésus, Dieu les ramènera avec lui.
\VS{15}Car nous vous disons ceci par la parole du Seigneur, que nous qui vivrons et resterons pour l'avènement du Seigneur, ne précéderons point ceux qui dorment.
\VS{16}Car le Seigneur lui-même, avec un cri de commandement\FTNT{L'expression «~cri de commandement~» vient du grec «~keleuma~», ce mot signifie un ordre, et en particulier un cri stimulant, comme celui que reçoit un animal pressé par un homme, tels les chevaux par les conducteurs de chariots, les chiens de chasse par les chasseurs, etc.~; ou par lequel un ordre est donné par le capitaine d'un navire, aux soldats par un chef, un appel de trompette. La sagesse de Dieu crie (Pr. 8). Esaïe devait crier à plein gosier (Es. 58:1). Le cri du Seigneur ne sera entendu que par l'Eglise véritable qui est son épouse (Mt. 25:6).}, et une voix d'archange, et avec la trompette de Dieu, descendra du ciel, et les morts en Christ ressusciteront premièrement.
\VS{17}Puis nous qui vivrons et qui resterons, serons enlevés ensemble avec eux dans les nuées, à la rencontre du Seigneur, dans les airs et ainsi nous serons toujours avec le Seigneur. 
\VS{18}C'est pourquoi consolez-vous les uns les autres par ces paroles.
\Chap{5}
\TextTitle{Veiller en attendant le jour du Seigneur~; encouragements divers\FTNTT{Joë. 1:15.}}
\VerseOne{}Pour ce qui est des temps et des moments, mes frères, vous n'avez pas besoin qu'on vous en écrive,
\VS{2}puisque vous savez vous-mêmes très bien que le jour du Seigneur viendra comme un voleur dans la nuit\FTNT{Mt. 25:6~; 2 Pi. 3:10~; Ap. 3:3~; 16:15.}.
\VS{3}Quand ils diront~: Nous sommes en paix et en sûreté. Alors une destruction soudaine les surprendra, comme les douleurs de l'enfantement surprennent la femme enceinte, et ils n'échapperont point.
\VS{4}Mais quant à vous, mes frères, vous n'êtes pas dans les ténèbres pour que ce jour-là vous surprenne comme un voleur.
\VS{5}Vous êtes tous des enfants de la lumière, et des enfants du jour. Nous ne sommes point de la nuit ni des ténèbres.
\VS{6}Ne dormons donc point comme les autres, mais veillons et soyons sobres.
\VS{7}Car ceux qui dorment, dorment la nuit, et ceux qui s'enivrent, s'enivrent la nuit.
\VS{8}Mais nous qui sommes enfants du jour, soyons sobres, ayant revêtu la cuirasse de la foi et de la charité, et ayant pour casque l'espérance du salut\FTNT{Ro. 13:12~; Ep. 6:14~; 6:17.}.
\VS{9}Car Dieu ne nous a pas destinés à la colère\FTNT{La colère à venir. Voir 1 Th. 1:9-10.}, mais à l'acquisition du salut par notre Seigneur Jésus-Christ,
\VS{10}qui est mort pour nous, afin que soit que nous veillons, soit que nous dormions, nous vivions avec lui.
\VS{11}C'est pourquoi exhortez-vous réciproquement, et édifiez-vous tous, les uns les autres, comme aussi vous le faites.
\VS{12}Nous vous prions, mes frères, d'avoir de la considération pour ceux qui travaillent parmi vous, qui dirigent dans le Seigneur, et qui vous exhortent.
\VS{13}Ayez pour eux beaucoup d'affection\FTNT{Littéralement «~agape~»~: amour, charité, affection.} à cause de l'œuvre qu'ils font. Soyez en paix entre vous.
\VS{14}Nous vous en prions aussi, mes frères, avertissez ceux qui vivent dans le désordre\FTNT{Mt. 18:15~; Ga. 6:1.}, consolez ceux qui ont l'esprit abattu, supportez les faibles, et soyez patients envers tous.
\VS{15}Prenez garde que personne ne rende à autrui le mal pour le mal\FTNT{Mt. 5:44~; Ro. 12:21.}~; mais recherchez toujours ce qui est bon, soit entre vous, soit envers tous les hommes.
\VS{16}Soyez toujours joyeux.
\VS{17}Priez sans cesse.
\VS{18}Rendez grâces pour toutes choses, car c'est la volonté de Dieu par Jésus-Christ.
\VS{19}N'éteignez point l'Esprit.
\VS{20}Ne méprisez point les prophéties.
\VS{21}Eprouvez toutes choses~; retenez ce qui est bon.
\VS{22}Abstenez-vous de toute apparence de mal.
\VS{23}Que le Dieu de paix veuille vous sanctifier entièrement, et faire que votre être entier, l'esprit, l'âme et le corps soient conservés sans reproche lors de la venue de notre Seigneur Jésus-Christ\FTNT{L'avènement du Seigneur. Voir Mt. 24:1-3.}.
\VS{24}Celui qui vous appelle est fidèle, c'est pourquoi il fera ces choses en vous.
\TextTitle{Salutations}
\VS{25}Mes frères, priez pour nous.
\VS{26}Saluez tous les frères par un saint baiser.
\VS{27}Je vous en conjure par le Seigneur que cette épître soit lue à tous les saints frères.
\VS{28}Que la grâce de notre Seigneur Jésus-Christ soit avec vous~! Amen~!
\PPE{}
\end{multicols}

%\clearpage\ShortTitle{2 Th.}\BookTitle{2 Thessaloniciens}\BFont
\noindent\hrulefill
{\footnotesize
\textit{
\bigskip
{\centering{}
\\Auteur~: Paul
\\Thème~: Le jour de Christ
\\Date de rédaction~: Env. 51 ap. J.-C.\\}
}
\textit{
\\Autrefois appelée Therme ou Therma, qui signifie «~source chaude~», Thessalonique reçut son nouveau nom de Cassandre, en l'honneur de sa femme Thessalonike, qui était aussi la sœur d'Alexandre le Grand (356 av. J.-C. - 323 av. J.-C.), à qui il succéda.
\\Cette ville est située au nord de la Grèce actuelle, sur la côte de la mer Egée. Du temps de Paul, ce pays était divisé en deux parties. Dans la région du nord, la Macédoine, se trouvaient les villes de Philippes, Thessalonique et Bérée. Quant à la région du sud, l'Achaïe, comportait les villes d'Athènes et de Corinthe. Aujourd'hui, la ville s'appelle Salonique. La seconde épître de Paul aux Thessaloniciens fut rédigée peu de temps après la première. Elle fut motivée par des troubles survenus dans la communauté à la suite d'une annonce basée sur une lettre faussement attribuée à Paul prétendant que le «~jour du Seigneur~» était arrivé. Dans cette seconde épître, l'apôtre exhorte les chrétiens de Thessalonique à tenir ferme dans leur foi malgré la persécution, leur expliquant que le «~jour de Christ~» devait être précédé par l'apostasie et la venue de l'homme impie. Il conclut sa lettre en demandant aux chrétiens de s'éloigner de ceux qui vivent dans le désordre.\bigskip
}
}
\par\nobreak\noindent\hrulefill
\begin{multicols}{2}
\Chap{1}
\TextTitle{Introduction}
\VerseOne{}Paul, Silvain, et Timothée à l'église des Thessaloniciens\FTNT{Thessalonique. Voir Ac. 17:1-9.} qui est en Dieu notre Père, et en notre Seigneur Jésus-Christ~:
\VS{2}Que la grâce et la paix vous soient données de la part de Dieu notre Père, et de la part du Seigneur Jésus-Christ~!
\TextTitle{La persévérance dans l'affliction~; Dieu, le juste Juge}
\VS{3}Mes frères, nous devons toujours rendre grâces à Dieu à cause de vous, comme il est bien raisonnable, parce que votre foi augmente beaucoup, et que votre charité mutuelle fait des progrès.
\VS{4}De sorte que nous-mêmes nous nous glorifions de vous dans les églises de Dieu, à cause de votre persévérance et de votre foi au milieu de toutes vos persécutions, et des afflictions que vous avez à supporter,
\VS{5}qui sont une manifeste démonstration du juste jugement de Dieu, afin que vous soyez jugés dignes du Royaume de Dieu, pour lequel aussi vous souffrez.
\TextTitle{La fin de ceux qui ne connaissent pas Dieu et qui n'obéissent pas à l'Evangile}
\VS{6}Car il est juste devant Dieu qu'il rende l'affliction à ceux qui vous affligent,
\VS{7}et qu'il vous donne du repos à vous qui êtes affligés, de même qu'à nous, lorsque le Seigneur Jésus se révélera\FTNT{Révélation, du grec «~apokalupsis~», signifie «~mettre à nu, révélation d'une vérité~». Le fait de rendre visible ce qui était caché.} du ciel avec les anges de sa puissance,
\VS{8}avec des flammes de feu, pour exercer la vengeance contre ceux qui ne connaissent pas Dieu, contre ceux qui n'obéissent pas à l'Evangile de notre Seigneur Jésus-Christ.
\VS{9}Ils auront pour châtiment une ruine éternelle, loin de la face du Seigneur, et de la gloire de sa force,
\VS{10}quand il viendra pour être glorifié en ce jour-là dans ses saints, et pour être admiré dans tous ceux qui croient, parce le témoignage que avons rendu auprès de vous à été cru.
\VS{11}C'est pourquoi nous prions toujours pour vous, afin que notre Dieu vous juge dignes de la vocation, et qu'il accomplisse puissamment en vous tout le bon plaisir de sa bonté, et l'œuvre de la foi,
\VS{12}afin que le Nom de notre Seigneur Jésus-Christ soit glorifié en vous, et vous en lui, selon la grâce de notre Dieu et Seigneur Jésus-Christ.
\Chap{2}
\TextTitle{Le jour du Seigneur et l'apparition de l'homme impie}
\VerseOne{}Or pour ce qui concerne l'avènement\FTNT{L'avènement du Seigneur Jésus-Christ. Voir Mt. 24:1-3.} de notre Seigneur Jésus-Christ et notre réunion en lui, mes frères, nous vous prions
\VS{2}de ne pas vous laisser subitement ébranler dans votre entendement, ni troubler par une inspiration, ni par une parole, ou par quelque lettre qu'on dirait venir de nous, comme si le jour de Christ était déjà là.
\VS{3}Que personne donc ne vous séduise d'aucune manière~; car il faut que l'apostasie soit arrivée auparavant et que l'homme de péché, le fils de la perdition\FTNT{Il est question ici de l'homme impie, de l'antichrist, qui est la bête qui monte de la mer décrite par Jean (Ap. 13:11-18). Voir aussi Da. 11:36-38.}, soit révélé,
\VS{4}lequel s'oppose et s'élève contre tout ce qui est appelé Dieu, ou qu'on adore, jusqu'à être assis comme Dieu dans le temple de Dieu\FTNT{Selon les chapitres 40 à 42 d'Ezéchiel, le culte lévitique sera restauré à la fin des temps, ce qui suppose nécessairement la reconstruction du temple de Jérusalem. Cette prophétie est actuellement (2014-2015) en train de s'accomplir puisque des juifs religieux militent activement pour la réalisation de ce projet. L'organisation la plus connue œuvrant en ce sens est l'Institut du temple (fondé en 1987), qui a déjà restauré un grand nombre d'objets servant au culte. Toutefois, il ne faut pas sous-estimer la ruse de Satan, car au-delà du temple physique, il cherche prioritairement à s'asseoir dans les temples spirituels que sont les chrétiens (1 Co. 6:19). Pour parvenir à ses fins, Satan a envoyé plusieurs de ses émissaires pour prêcher un autre évangile et un autre christ. C'est ainsi que de nombreuses assemblées, séduites et captivées par de faux docteurs, n'ont plus Jésus-Christ comme Seigneur, mais Satan en personne. L'apostasie étant installée premièrement dans les cœurs, l'antichrist n'aura donc aucun mal à se faire passer pour le Christ et à s'asseoir dans le temple physique, où il usurpera l'adoration qui revient au Dieu véritable.} se proclamant lui-même être Dieu.
\VS{5}Ne vous souvenez-vous pas que je vous disais ces choses, lorsque j'étais encore chez vous~?
\VS{6}Et maintenant vous savez ce qui le retient, afin qu'il soit révélé en son temps.
\VS{7}Car le mystère de l'iniquité\FTNT{Le mystère de l'iniquité. Paul nous enseigne que ce mystère était déjà à l'œuvre au sein des églises primitives. Le prophète Zacharie, au chapitre 5 de son livre, l'avait personnifié en relatant une vision dans laquelle il avait vu «~deux femmes avec des ailes de cigogne~» emportant l'épha de l'iniquité des enfants d'Israël. Sur cet épha était assise une femme personnifiant l'iniquité, c'est-à-dire la femme de l'homme impie, la Babylone religieuse. Ces deux femmes aux ailes de cigogne allaient lui bâtir une maison au pays de Schinéar (Babylone selon Ge. 10:6-14).} opère déjà, seulement celui qui le retient en ce moment le fera jusqu’à ce qu’il soit hors du chemin.
\VS{8}Et alors sera révélé le méchant\FTNT{Es. 11:4.}, que le Seigneur détruira par le souffle de sa bouche et qu'il anéantira par l'éclat de son avènement.
\VS{9}L'avènement\FTNT{Il y aura un autre avènement, celui de l'homme impie.} de cet impie, se fera par la puissance de Satan, avec toutes sortes de miracles, de signes, et de prodiges mensongers,
\VS{10}et avec toutes les séductions de l'iniquité, pour ceux qui périssent parce qu'ils n'ont pas reçu l'amour de la vérité pour être sauvés.
\VS{11}C'est pourquoi Dieu leur envoie une puissance d'égarement\FTNT{L'esprit d'égarement. Voir Ro. 1:26,28~; 1 R. 22.}, pour qu'ils croient au mensonge,
\VS{12}afin que tous ceux qui n'ont pas cru à la vérité, mais qui ont pris plaisir à l'iniquité soient condamnés.
\TextTitle{Encouragements}
\VS{13}Mais nous, mes frères bien-aimés du Seigneur, nous devons toujours rendre grâces à Dieu pour vous, de ce que Dieu vous a élus dès le commencement pour le salut par la sanctification de l'Esprit, et par la foi en la vérité.
\VS{14}C'est à quoi il vous a appelés par notre Evangile, afin que vous possédiez la gloire qui nous a été acquise par notre Seigneur Jésus-Christ.
\VS{15}C'est pourquoi, mes frères, demeurez fermes, et retenez les enseignements que vous avez appris, soit par notre parole, soit par notre lettre.
\VS{16}Et que notre Seigneur Jésus-Christ lui-même, et notre Dieu et Père, qui nous a aimés, et qui nous a donné une consolation éternelle, et une bonne espérance par sa grâce,
\VS{17}console vos cœurs, et vous affermisse en toute bonne parole, et en toute bonne œuvre.
\Chap{3}
\VerseOne{}Au reste, mes frères, priez pour nous, afin que la parole du Seigneur poursuive sa course, et qu'elle soit glorifiée comme elle l'est parmi vous,
\VS{2}et que nous soyons délivrés des hommes méchants et pervers, car tous n'ont pas la foi.
\VS{3}Le Seigneur est fidèle, il vous affermira et vous gardera du mal.
\VS{4}Nous avons à votre égard cette confiance dans le Seigneur, que vous faites et que vous ferez les choses que nous recommandons.
\VS{5}Que le Seigneur veuille diriger vos cœurs vers l'amour de Dieu et vers l'attente de Christ~!
\TextTitle{Se séparer des mauvaises compagnies~; être un modèle~; subvenir à ses besoins}
\VS{6}Nous vous recommandons aussi, mes frères, au Nom de notre Seigneur Jésus-Christ, de vous éloigner\FTNT{La séparation d'avec la mauvaise compagnie. Voir 1 Co. 5:9-13, 15:33~; 2 Co. 6:14-18~; Ro. 16:17-18~; Tit. 3:10-11~; 2 Jn. 2-11.} de tout homme qui se dit frère, et qui vit d'une manière déréglée, et non selon les enseignements qu'il a reçus de nous.
\VS{7}Car vous savez vous-mêmes comment il faut nous imiter, puisque nous n'avons pas marché dans le désordre parmi vous,
\VS{8}et nous n'avons mangé gratuitement le pain de personne. Mais dans le labeur et dans la peine, nous avons travaillé nuit et jour, pour n'être à la charge\FTNT{Les véritables ouvriers de Dieu ne s'attendent pas aux hommes pour avoir leur salaire. Ils mettent leur confiance en Dieu qui est leur rémunérateur. Voir Ac. 20:33-35.} d'aucun de vous.
\VS{9}Ce n'est pas que nous n'en ayons pas le droit, mais c'est pour donner en nous-mêmes un modèle à imiter.
\VS{10}Car lorsque nous étions avec vous, nous vous déclarions expressément que si quelqu'un ne veut pas travailler, qu'il ne mange pas non plus.
\VS{11}Car nous apprenons qu'il y en a quelques-uns parmi vous qui marchent dans le désordre, qui ne travaillent pas, mais qui s'occupent de futilités.
\VS{12}C'est pourquoi nous recommandons donc à ces gens-là et nous les exhortons par notre Seigneur Jésus-Christ, à manger leur propre pain en travaillant paisiblement.
\VS{13}Mais pour vous, mes frères, ne vous lassez pas de faire le bien.
\VS{14}Et si quelqu'un n'obéit pas à ce que nous vous disons par cette épître, faites-le connaître, et n'ayez pas de relation avec lui, afin qu'il éprouve de la honte.
\VS{15}Toutefois, ne le regardez pas comme un ennemi, mais avertissez-le comme un frère.
\TextTitle{Conclusion}
\VS{16}Que le Seigneur de paix vous donne toujours la paix en tout temps~! Que le Seigneur soit avec vous tous~!
\VS{17}La salutation est de ma propre main, de moi Paul, c'est là ma signature dans toutes mes épîtres, c'est ainsi que j'écris.
\VS{18}Que la grâce de notre Seigneur Jésus-Christ soit avec vous tous~! Amen~!
\PPE{}
\end{multicols}

%\clearpage\ShortTitle{1 Corinthiens}\BookTitle{1 Corinthiens}\BFont
\noindent\hrulefill
{\footnotesize
\textit{
\bigskip
{\centering{}
\\Auteur : Paul
\\Thème : Le comportement du chrétien
\\Date de rédaction : Env. 56 ap. J.-C.\\}
}
%\bigskip
\textit{
\\Dans l’antiquité, Corinthe, capitale de l’Achaïe, était la ville la plus prospère et la plus puissante de Grèce. Située sur
un isthme séparant la mer Egée de la mer Ionienne, Corinthe était au carrefour de l’Asie et de l’Italie et constituait un  véritable centre commercial où les produits orientaux et occidentaux se croisaient.
%\bigskip
\\L’apôtre Paul arriva à Corinthe en 51, sous le règne de l’empereur romain Claude (10 av. J.-C. – 54 apr. J.-C.), et y demeura 18 mois. Il trouva une ville riche en pleine expansion, une population parlant diverses langues et rendant des cultes à une multitude de divinités. Rédigée au terme des trois ans passés à Ephèse, la première épître de Paul aux Corinthiens répond à une lettre dans laquelle ceux-ci s’interrogeaient sur le mariage et sur les aliments consacrés aux idoles. Ce fut aussi l’occasion pour lui de procéder à la correction de cette jeune église dont l’état charnel constituait un frein à l’avancée spirituelle. Les Corinthiens avaient en effet confondu le culte raisonnable et les pratiques liées aux cultes à mystères.\bigskip
}
}
\par\nobreak\noindent\hrulefill
\begin{multicols}{2}
\Chap{1}
\TextTitle{La grâce de Christ manifeste dans la vie des saints\FTNTT{Ro. 5:1-2 ; Ep. 1:3-14}}
\VerseOne{}Paul, appelé à être apôtre de Jésus-Christ, par la volonté de Dieu, et le frère Sosthène,
\VS{2}à l'église de Dieu qui est à Corinthe, aux sanctifiés en Jésus-Christ, appelés à être saints, et à tous ceux qui en quelque lieu que ce soit invoquent le Nom de notre Seigneur Jésus-Christ, leur Seigneur et le nôtre.
\VS{3}Que la grâce et la paix vous soient données de la part de Dieu notre Père et du Seigneur Jésus-Christ.
\VS{4}Je rends toujours grâces à mon Dieu à votre sujet, pour la grâce de Dieu qui vous a été donnée en Jésus-Christ.
\VS{5}Car en lui vous avez été enrichis de toutes les richesses qui concernent la parole et la connaissance,
\VS{6}selon que le témoignage de Jésus-Christ a été confirmé en vous,
\VS{7}de sorte qu'il ne vous manque aucun don, pendant que vous attendez la manifestation de notre Seigneur Jésus-Christ.
\VS{8}Qui vous affermira aussi jusqu’à la fin pour que vous soyez irrépréhensibles au jour de notre Seigneur Jésus-Christ.
\VS{9}Dieu qui vous a appelés à la communion de son Fils Jésus-Christ notre Seigneur est fidèle.
\TextTitle{Les rivalités, causes de divisions}
\VS{10}Je vous prie, mes frères, par le Nom de notre Seigneur Jésus-Christ, à tenir tous un même langage, et à ne point avoir de divisions parmi vous, mais à être parfaitement unis dans une même pensée et dans un même jugement.
\VS{11}Car mes frères, j’ai été informé par ceux de la maison de Chloé qu'il y a des dissensions parmi vous.
\VS{12}Je veux dire que chacun de vous parle ainsi : Moi je suis de Paul ! Et moi d'Apollos ! Et moi de Céphas ! Et moi de Christ !
\VS{13}Christ est-il divisé ? Paul a-t-il été crucifié pour vous ? Ou avez-vous été baptisés au nom de Paul ?
\VS{14}Je rends grâces à Dieu de ce que je n'ai baptisé aucun de vous, sinon Crispus et Gaïus,
\VS{15}afin que personne ne dise que j'ai baptisé en mon nom.
\VS{16}J'ai bien aussi baptisé la famille de Stéphanas ; du reste, je ne sais pas si j'ai baptisé quelque autre.
\VS{17}Car Christ ne m'a pas envoyé pour baptiser, mais pour évangéliser, non pas avec des discours de la sagesse humaine, afin que la croix de Christ ne soit pas anéantie.
\TextTitle{La sagesse de Dieu à la croix, dépasse l'entendement humain}
\VS{18}Car la prédication de la croix est une folie pour ceux qui périssent, mais pour nous qui sommes sauvés, elle est la puissance de Dieu.
\VS{19}Car il est écrit : Je détruirai la sagesse des sages et j'anéantirai l'intelligence des hommes intelligents\FTNT{Es. 29:14.}.
\VS{20}Où est le sage ? Où est le scribe ? Où est le disputeur de ce siècle ? Dieu n'a-t-il pas convaincu de folie la sagesse de ce monde ?
\VS{21}Puisque le monde, avec sa sagesse, n’a pas connu Dieu, dans la sagesse de Dieu, il a plu à Dieu de sauver les croyants par la folie de la prédication.
\VS{22}Les Juifs demandent des miracles et les Grecs cherchent la sagesse,
\VS{23}mais pour nous, nous prêchons Christ crucifié, scandale pour les Juifs, et folie pour les Grecs,
\VS{24}à ceux qui sont appelés, tant Juifs que Grecs, nous leur prêchons Christ, la puissance de Dieu et la sagesse de Dieu.
\VS{25}Parce que la folie de Dieu est plus sage que les hommes, et la faiblesse de Dieu est plus forte que les hommes.
\TextTitle{Dieu se sert des choses viles pour confondre le monde et sa sagesse}
\VS{26}Considérez, mes frères, que parmi vous qui avez été appelés, il n’y a pas beaucoup de sages selon la chair, ni beaucoup de puissants, ni beaucoup de nobles.
\VS{27}Mais Dieu a choisi les choses folles de ce monde pour confondre les sages ; et Dieu a choisi les choses faibles de ce monde pour confondre les fortes ;
\VS{28}et Dieu a choisi les choses viles de ce monde et les méprisées, même celles qui ne sont point, pour réduire à néant celles qui sont,
\VS{29}afin que nulle chair ne se glorifie devant lui.
\VS{30}Or c'est par lui que vous êtes en Jésus-Christ, lequel, de par Dieu, a été fait pour nous sagesse, justice, sanctification et rédemption ;
\VS{31}afin que comme il est écrit, celui qui se glorifie se glorifie dans le Seigneur\FTNT{Jé. 9:24.}.
\Chap{2}
\TextTitle{La foi en Dieu ne se base pas sur la sagesse humaine}
\VerseOne{}Pour moi donc, mes frères, lorsque je suis allé chez vous, ce n’est pas avec des discours pompeux, remplis de la sagesse humaine, que je suis allé vous annoncer le témoignage de Dieu.
\VS{2}Car je n’ai pas eu la pensée de savoir parmi vous autre chose que Jésus-Christ et Jésus-Christ crucifié.
\VS{3}Et j'ai même été parmi vous dans la faiblesse, dans la crainte, et dans un grand tremblement.
\VS{4}Et ma parole et ma prédication ne reposaient pas sur les discours persuasifs de la sagesse humaine, mais sur une démonstration d'Esprit et de puissance ;
\VS{5}afin que votre foi ne soit pas fondée sur la sagesse des hommes, mais sur la puissance de Dieu.
\VS{6}Cependant, nous prêchons une sagesse parmi les parfaits, une sagesse, dis-je, qui n'est pas de ce monde, ni des chefs de ce siècle, qui vont être anéantis.
\VS{7}Mais nous prêchons la sagesse de Dieu, qui est un mystère, c'est-à-dire cachée, que Dieu avant les siècles, avait prédestinée pour notre gloire,
\VS{8}sagesse qu’aucun des chefs de ce siècle n'a connue, car s'ils l’avaient connue, ils n’auraient pas crucifié le Seigneur de gloire.
\TextTitle{C'est l'Esprit de Dieu qui revèle les profondeurs de Dieu}
\VS{9}Mais comme il est écrit : Ce sont des choses que l’œil n'a point vues, que l'oreille n'a point entendues, et qui ne sont point montées au cœur de l'homme, des choses que Dieu a préparées pour ceux qui l’aiment\FTNT{Es. 64:4.}.
\VS{10}Mais Dieu nous les a révélées par son Esprit. Car l'Esprit sonde toutes choses, même les choses profondes de Dieu.
\VS{11}Qui donc, parmi les hommes, connaît les choses de l'homme, sinon l’esprit de l'homme qui est en lui ? De même aussi, personne ne connaît les choses de Dieu, si ce n’est l'Esprit de Dieu.
\VS{12}Or nous, nous n’avons pas reçu l'esprit de ce monde, mais l'Esprit qui vient de Dieu, afin que nous connaissions les choses qui nous ont été données de Dieu.
\TextTitle{La sagesse humaine n'accepte pas les choses de l'Esprit}
\VS{13}Et nous en parlons, non avec des discours que la sagesse humaine enseigne, mais avec celle qu'enseigne le Saint-Esprit, communiquant des choses spirituelles à ceux qui sont spirituels.
\VS{14}Mais l'homme animal\FTNT{L’homme animal (ou naturel) est un homme incrédule. C’est un homme  non-régénéré, ayant le principe de la vie animale, c’est-à-dire ce que les hommes ont en commun avec les brutes. Sa nature sensuelle est sujette aux appétits et aux passions (Jud. 1:19).} ne comprend pas les choses de l'Esprit de Dieu, car elles sont une folie pour lui ; et il ne peut même pas les entendre, parce c’est spirituellement qu’on en juge.
\VS{15}Mais l'homme spirituel\FTNT{L’homme spirituel est un homme dont l’esprit est régénéré et qui marche par l’Esprit. Il a la pensée de Christ et porte les fruits de l’Esprit.} juge de tout et il n'est jugé par personne.
\VS{16}Car qui a connu la pensée du Seigneur pour pouvoir l’instruire\FTNT{Es. 40:13.} ? Mais nous, nous avons la pensée de Christ.
\Chap{3}
\TextTitle{Les œuvres de la chair nuisent à la croissance chrétienne}
\VerseOne{}Pour moi, mes frères, je n'ai pas pu vous parler comme à des hommes spirituels, mais comme à des hommes charnels\FTNT{L’homme charnel est gouverné par la nature humaine et non par l'Esprit de Dieu (Ga. 5:16-21). L’homme charnel est un enfant en Christ, littéralement «~ignorant~» (Ga. 4:1). Il est comparé à un esclave.}, c'est-à-dire comme à des enfants en Christ.
\VS{2}Je vous ai donné du lait à boire, et non pas de la viande, parce que vous ne pouviez pas la supporter ; et même maintenant vous ne le pouvez pas encore, parce que vous êtes encore charnels.
\VS{3}Car puisqu'il y a parmi vous de la jalousie, des disputes, et des divisions, n'êtes-vous pas charnels, et ne vous conduisez-vous pas à la manière des hommes ?
\VS{4}Car quand l'un dit : Moi je suis de Paul ; et l'autre : Moi je suis d'Apollos, n'êtes-vous pas charnels ?
\TextTitle{Dieu est le maître de tout}
\VS{5}Qu’est-ce donc Paul, et qui est Apollos ? Des ministres, par le moyen desquels vous avez cru, selon que le Seigneur l’a donné à chacun.
\VS{6}J'ai planté, Apollos a arrosé, mais c'est Dieu qui a donné l'accroissement,
\VS{7}en sorte que ce n’est pas celui qui plante qui est quelque chose, ni celui qui arrose, mais Dieu qui donne l'accroissement.
\VS{8}Celui qui plante et celui qui arrose sont égaux, et chacun recevra sa récompense selon son propre travail.
\VS{9}Car nous sommes ouvriers avec Dieu. Vous êtes le champ de Dieu et l'édifice de Dieu.
\VS{10}Selon la grâce de Dieu qui m'a été donnée, j'ai posé le fondement comme un sage architecte, et un autre édifie dessus. Mais que chacun prenne garde comment il édifie dessus.
\TextTitle{Le seul fondement : Jésus-Christ}
\VS{11}Car personne ne peut poser un autre fondement que celui qui a été posé, à savoir Jésus-Christ.
\TextTitle{Deux types de construction}
\VS{12}Si quelqu'un édifie sur ce fondement avec de l'or, de l'argent, des pierres précieuses, du bois, du foin, du chaume, l’œuvre de chacun sera manifestée ;
\VS{13}car le jour la fera connaître, parce qu'elle sera manifestée par le feu ; et le feu éprouvera ce qu’est l’œuvre de chacun.
\VS{14}Si l’œuvre édifiée par quelqu’un sur le fondement subsiste, il recevra la récompense.
\VS{15}Si l’œuvre de quelqu'un est consumée, il perdra sa récompense ; mais pour lui, il sera sauvé, toutefois comme au travers du feu.
\VS{16}Ne savez-vous pas que vous êtes le temple\FTNT{Le temple de Dieu. Beaucoup veulent construire des bâtiments qu’ils appellent «~temples ou maisons de Dieu~» alors que chaque chrétien est le temple de Dieu. Voir Es. 66:1 ; Ac. 17:24 ; 1 Co. 6:19.} de Dieu et que l’Esprit de Dieu habite en vous ?
\VS{17}Si quelqu'un détruit le temple de Dieu, Dieu le détruira ; car le temple de Dieu est saint, et vous êtes ce temple.
\VS{18}Que personne ne s'abuse lui-même : Si quelqu'un d'entre vous croit être sage selon ce monde, qu'il devienne fou, afin de devenir sage.
\VS{19}Parce que la sagesse de ce monde est une folie devant Dieu ; car il est écrit : Il surprend les sages dans leur ruse\FTNT{Job 5:13.}.
\VS{20}Et encore : Le Seigneur connaît les pensées des sages, il sait qu’elles sont vaines\FTNT{Ps. 94:11.}.
\VS{21}Que personne donc ne mette sa gloire dans les hommes, car toutes choses sont à vous,
\VS{22}soit Paul, soit Apollos, soit Céphas, soit le monde, soit la vie, soit la mort, soit les choses présentes, soit les choses à venir, toutes choses sont à vous,
\VS{23}et vous à Christ, et Christ à Dieu.
\Chap{4}
\TextTitle{Le Seigneur est le seul véritable juge}
\VerseOne{}Que chacun nous regarde comme des serviteurs de Christ et des dispensateurs des mystères de Dieu.
\VS{2}Du reste, il est exigé des dispensateurs que chacun soit trouvé fidèle.
\VS{3}Pour moi, il m’importe fort peu d'être jugé par vous, ou par un jugement d'homme. Je ne me juge pas non plus moi-même, car je ne me sens coupable de rien,
\VS{4}mais ce n’est pas pour cela que je suis justifié. Celui qui me juge, c'est le Seigneur.
\VS{5}C'est pourquoi ne jugez de rien avant le temps, jusqu'à ce que le Seigneur vienne, alors il mettra en lumière les choses cachées dans les ténèbres et manifestera les desseins des cœurs. Alors chacun recevra de Dieu la louange qui lui sera due.
\VS{6}Or mes frères, j’ai fait de ces choses une application à ma personne et à celle d’Apollos, à cause de vous ; afin que vous appreniez de nous à ne point aller au-delà de ce qui est écrit, et que nul de vous ne conçoive de l’orgueil en faveur de l’un contre l’autre.
\VS{7}Car qui est-ce qui met de la différence entre toi et un autre ? Qu’as-tu que tu n’aies reçu ? Et si tu l'as reçu, pourquoi te glorifies-tu comme si tu ne l'avais pas reçu\FTNT{Les diverses grâces que Dieu accorde à ses enfants doivent les amener à l’humilité.} ?
\VS{8}Vous êtes déjà rassasiés, vous êtes déjà enrichis, vous êtes devenus rois sans nous. Plaise à Dieu que vous régniez en effet, afin que nous aussi nous régnions avec vous !
\TextTitle{L'humilité et la patience}
\VS{9}Car je pense que Dieu nous a exposés publiquement, nous qui sommes les derniers des apôtres, comme des gens condamnés à la mort, puisque nous avons été en spectacle au monde, aux anges et aux hommes.
\VS{10}Nous sommes fous pour l'amour de Christ, mais vous êtes sages en Christ ; nous sommes faibles, et vous êtes forts ; vous êtes dans l'estime, et nous sommes dans le mépris.
\VS{11}Jusqu'à cette heure, nous souffrons la faim, la soif, la nudité ; on nous frappe au visage, et nous sommes errants çà et là ;
\VS{12}nous nous fatiguons à travailler de nos propres mains ; on dit du mal de nous, et nous bénissons ; nous sommes persécutés, et nous le supportons.
\VS{13}Nous sommes calomniés, et nous prions ; nous sommes devenus comme les balayures du monde, comme le rebut de tous, jusqu'à maintenant.
\VS{14}Je n'écris pas ces choses pour vous faire honte, mais je vous avertis comme mes chers enfants.
\VS{15}Car même si vous aviez dix mille maîtres en Christ, vous n'avez pourtant pas plusieurs pères, car c'est moi qui vous ai engendrés en Jésus-Christ par l'Evangile.
\VS{16}Je vous prie donc d'être mes imitateurs.
\VS{17}C'est pour cela que je vous ai envoyé Timothée, qui est mon fils bien-aimé, et qui est fidèle dans le Seigneur, afin qu'il vous rappelle quelles sont mes voies en Christ et comment j'enseigne partout dans toutes les églises.
\TextTitle{L'autorité de Paul}
\VS{18}Quelques-uns se sont enflés d’orgueil comme si je ne devais pas aller chez vous.
\VS{19}Mais j'irai bientôt chez vous, si le Seigneur le veut ; et je connaîtrai non les paroles, mais la puissance de ceux qui se sont glorifiés.
\VS{20}Car le Royaume de Dieu ne consiste pas en paroles, mais en puissance.
\VS{21}Que voulez-vous ? Que j’aille chez vous avec la verge, ou avec charité et dans un esprit de douceur ?
\Chap{5}
\TextTitle{L'inceste à Corinthe}
\VerseOne{}On entend dire de toutes parts qu'il y a parmi vous de l’impudicité, et une impudicité telle qu’elle ne se rencontre même pas chez les gentils ; c'est au point où l’un de vous a la femme de son père\FTNT{L’inceste est interdit par la loi (Lé. 18:6-8).}.
\TextTitle{Oter le mal dans l'Eglise}
\VS{2}Et vous êtes enflés d'orgueil ! Et vous n'avez pas été plutôt dans le deuil, afin que celui qui a commis cette action soit retranché du milieu de vous.
\VS{3}Pour moi, étant absent de corps, mais présent en esprit, j'ai déjà jugé comme si j'étais présent, celui qui a commis une telle action.
\VS{4}Vous et mon esprit étant assemblés au nom de notre Seigneur Jésus-Christ, j'ai ordonné, avec la puissance de notre Seigneur Jésus-Christ,
\VS{5}qu'un tel homme soit livré à Satan\FTNT{Cette déclaration de Paul peut paraître choquante pour certains, mais elle nous rappelle l'histoire de Job, qui fut mis à l'épreuve par Yahweh qui l'avait livré à Satan (Job. 1:12). Paul espérait ainsi amener cet homme à la repentance en l’excluant de l’assemblée.} pour la destruction de la chair, afin que l'esprit soit sauvé au jour du Seigneur Jésus.
\VS{6}Votre vanité est mal fondée. Ne savez-vous pas qu'un peu de levain\FTNT{Le levain fait gonfler ou enfler. Il symbolise la cause principale de nombreux péchés : l'orgueil. Dans la Bible, le levain représente aussi des péchés spirituellement destructeurs comme la malice, la méchanceté, l'hypocrisie et les faux enseignements (Lu. 12:1, Mt. 16:11-12).} fait lever toute la pâte ?
\VS{7}Otez donc le vieux levain, afin que vous soyez une nouvelle pâte, puisque vous êtes sans levain ; car Christ, notre Pâque\FTNT{Ex. 12.}, a été sacrifié pour nous.
\VS{8}C'est pourquoi célébrons donc la fête, non avec du vieux levain, non avec un levain de méchanceté et de malice, mais avec les pains sans levain de la sincérité et de la vérité.
\TextTitle{Le disciple du Seigneur ne doit pas fréquenter les faux frères}
\VS{9}Je vous ai écrit dans ma lettre de ne pas vous mêlez\FTNT{Mêler vient du grec «~sunanamignumi~» qui signifie : «~mêler ensemble, se tenir en compagnie avec, être intime avec quelqu'un. Avoir des relations, être en communication~» (Ps. 1:1 ; Ro. 16:17-18 ; 1 Co. 15:33 ; Tit. 3:10).} avec les fornicateurs,
\VS{10}non pas d’une manière absolue avec les fornicateurs de ce monde, ou avec les cupides, ou les ravisseurs, ou les idolâtres ; autrement, il vous faudrait sortir du monde.
\VS{11}Maintenant, ce que je vous ai écrit, c’est de ne pas avoir de relations avec quelqu’un qui, se nommant frère, est fornicateur, ou cupide, ou idolâtre, ou médisant, ou ivrogne, ou ravisseur, de ne même pas manger avec un tel homme.
\VS{12}Car qu'ai-je à juger ceux qui sont dehors ? N’est-ce pas ceux du dedans que vous avez à juger ?
\VS{13}Mais Dieu juge ceux qui sont du dehors. Otez donc le méchant du milieu de vous.
\Chap{6}
\TextTitle{Procès entre chrétiens ou face aux non croyants}
\VerseOne{}Quand quelqu'un d'entre vous a une affaire contre un autre, ose-t-il bien aller en jugement devant les injustes, et il ne va pas devant les saints ?
\VS{2}Ne savez-vous pas que les saints jugeront le monde\FTNT{L’Eglise jugera les nations. Les douze apôtres jugeront Israël (Mt. 19:28 ; Lu. 22:30).} ? Or si le monde doit être jugé par vous, êtes-vous indignes de rendre les moindres jugements ?
\VS{3}Ne savez-vous pas que nous jugerons les anges\FTNT{Le mot ange vient du grec «~aggelos~» et veut dire «~messager, envoyé, ange~». Ce terme s’applique donc aussi bien aux hommes qu’aux créatures spirituelles.} ? Et à plus forte raison les choses de cette vie ?
\VS{4}Si donc vous avez des procès pour les affaires de cette vie, prenez pour juge ceux qui sont des moins estimés dans l'Eglise !
\VS{5}Je le dis à votre honte. Ainsi il n’y a parmi vous pas un seul homme sage qui puisse prononcer un jugement entre frères.
\VS{6}Mais un frère a des procès contre son frère, et cela devant les infidèles.
\VS{7}C'est déjà un grand défaut chez vous que vous ayez des procès entre vous. Pourquoi ne souffrez-vous pas plutôt quelque injustice ? Pourquoi ne vous laissez-vous pas plutôt dépouiller ?
\VS{8}Mais c’est vous qui commettez l’injustice et qui dépouillez, et c’est envers des frères que vous agissez de la sorte !
\TextTitle{Le chrétien est sanctifié, lavé et justifié}
\VS{9}Ne savez-vous pas que les injustes n'hériteront point le Royaume de Dieu ? Ne vous y trompez pas : Ni les fornicateurs, ni les idolâtres, ni les adultères,
\VS{10}ni les efféminés, ni les homosexuels, ni les voleurs, ni les avares, ni les ivrognes, ni les médisants, ni les ravisseurs, n'hériteront le Royaume de Dieu.
\VS{11}Et c’est là ce que vous étiez ; mais vous avez été lavés, mais vous avez été sanctifiés, mais vous avez été justifiés au nom du Seigneur Jésus, et par l'Esprit de notre Dieu.
\VS{12}Tout m’est permis, mais tout n’est pas utile ; tout m’est permis, mais je ne me rendrai esclave d’aucune chose.
\TextTitle{Le chrétien appartient au Seigneur}
\VS{13}Les aliments sont pour le ventre, et le ventre pour les aliments ; et Dieu détruira l'un comme les autres. Or le corps n'est point pour la fornication, mais pour le Seigneur, et le Seigneur pour le corps.
\VS{14}Et Dieu qui a ressuscité le Seigneur, nous ressuscitera aussi par sa puissance.
\VS{15}Ne savez-vous pas que vos corps sont les membres de Christ ? Prendrai-je donc les membres de Christ pour en faire les membres d'une prostituée ? Loin de là !
\VS{16}Ne savez-vous pas que celui qui s'unit à la prostituée devient un même corps avec elle ? Car il est dit : Les deux deviendront une même chair\FTNT{Ge. 2:24.}.
\VS{17}Mais celui qui s’unit au Seigneur est avec lui un seul esprit.
\VS{18}Fuyez la fornication. Quelque autre péché qu’un homme commette, ce péché est hors du corps ; mais le fornicateur pèche contre son propre corps.
\TextTitle{Le chrétien est le temple Saint-Esprit}
\VS{19}Ne savez-vous pas que votre corps est le temple du Saint-Esprit qui est en vous, et que vous avez reçu de Dieu, et que vous ne vous appartenez point à vous-mêmes ?
\VS{20}Car vous avez été achetés à un prix ; glorifiez donc Dieu dans votre corps et dans votre esprit, qui appartiennent à Dieu.
\Chap{7}
\TextTitle{La sainteté dans le mariage}
\VerseOne{}Pour ce qui concerne les choses au sujet desquelles vous m'avez écrit : Je vous dis qu'il est bon à l'homme de ne pas se marier.
\VS{2}Toutefois, pour éviter la fornication, que chacun ait sa femme, et que chaque femme ait son mari.
\VS{3}Que le mari rende à sa femme la bienveillance qui lui est due ; et que la femme de même la rende à son mari.
\VS{4}Car la femme n'a pas de pouvoir sur son propre corps, mais c’est son mari. De même, le mari n'a pas de pouvoir sur son propre corps, mais c’est sa femme.
\VS{5}Ne vous privez point l'un de l'autre, si ce n'est par un consentement mutuel, pour un temps, afin que vous vaquiez au jeûne et à la prière, mais après cela retournez ensemble, de peur que Satan ne vous tente par votre manque de contrôle.
\VS{6}Or je dis ceci par conseil, et non par commandement.
\VS{7}Car je voudrais que tous les hommes soient comme moi ; mais chacun a reçu de Dieu un don particulier, l'un d’une manière, l’autre d’une autre.
\VS{8}A ceux qui ne sont pas mariés, et aux veuves, je dis qu'il leur est bon de demeurer comme moi.
\VS{9}Mais s'ils manquent de maîtrise, qu'ils se marient ; car il vaut mieux se marier que de brûler.
\TextTitle{Recommandations à ceux qui sont mariés}
\VS{10}Et quant à ceux qui sont mariés, je leur ordonne, non pas moi, mais le Seigneur, que la femme ne se sépare point de son mari.
\VS{11}Et si elle s'en sépare, qu'elle demeure sans être mariée, ou qu'elle se réconcilie avec son mari ; que le mari aussi ne quitte point sa femme.
\VS{12}Mais aux autres je leur dis, et non pas le Seigneur : Si un frère a une femme incrédule et qu'elle consente d'habiter avec lui, qu'il ne la quitte point.
\VS{13}Et si une femme a un mari incrédule et qu'il consente d'habiter avec elle, qu'elle ne le quitte point.
\VS{14}Car le mari incrédule est sanctifié par la femme, et la femme incrédule est sanctifiée par le mari ; autrement vos enfants seraient impurs, or maintenant ils sont saints.
\VS{15}Que si l'incrédule se sépare, qu'il se sépare ; le frère ou la sœur ne sont point liés dans ce cas-là, car Dieu nous a appelés à la paix.
\VS{16}Car sais-tu, femme, si tu sauveras ton mari ? Ou que sais-tu, mari, si tu sauveras ta femme ?
\TextTitle{La circoncision et l'incirconcision ne sont rien, Dieu est tout}
\VS{17}Toutefois, que chacun marche selon le don qu'il a reçu de Dieu, chacun selon l’appel qu’il a reçu du Seigneur. C’est ainsi que je l’ordonne dans toutes les églises.
\VS{18}Quelqu'un a-t-il été appelé étant circoncis ? Qu’il ne redevienne pas incirconcis\FTNT{Vient du grec «~Epispaomai~» qui a pour définition : Ne pas devenir incirconcis. Aux jours  d'Antiochus IV, dit aussi Antioche Epiphane (voir commentaire en Da.8:9), certains Juifs, voulant échapper aux persécutions, cachaient le signe de leur nationalité, la circoncision, en se faisant reproduire artificiellement le prépuce par une opération chirurgicale qui étendait la peau restante.}. Quelqu'un a-t-il été appelé incirconcis ? Qu’il ne se fasse pas circoncire.
\VS{19}La circoncision n'est rien, et l’incirconcision aussi n'est rien, mais l'observation des commandements de Dieu est tout.
\VS{20}Que chacun demeure dans la condition où il était quand il a été appelé.
\VS{21}As-tu été appelé étant esclave ? Ne t'en inquiète pas ; mais si tu peux être mis en liberté, profites-en plutôt.
\VS{22}Car l’esclave qui a été appelé par notre Seigneur est un affranchi du Seigneur ; de même, celui qui est appelé étant libre, est un esclave de Christ.
\VS{23}Vous avez été rachetés à un prix, ne devenez pas les esclaves des hommes.
\VS{24}Mes frères, que chacun demeure devant Dieu dans l'état où il était quand il a été appelé.
\TextTitle{Conseils de Paul aux célibataires}
\VS{25}Pour ce qui concerne les vierges, je n'ai point de commandement du Seigneur, mais je donne un avis comme ayant obtenu miséricorde du Seigneur pour être fidèle.
\VS{26}Voici donc ce que j'estime bon, à cause des afflictions présentes : Il est avantageux à chacun de demeurer comme il est.
\VS{27}Es-tu lié à une femme ? Ne cherche pas à rompre ce lien. N’es-tu pas lié à une femme ? Ne cherche point de femme.
\VS{28}Si tu te maries, tu ne pèches point ; et si la vierge se marie, elle ne pèche point aussi ; mais ceux qui sont mariés auront des afflictions dans la chair ; or je voudrais vous les épargner.
\VS{29}Mais je vous dis ceci, mes frères : Le temps est court, que désormais ceux qui ont une femme soient comme n’en ayant pas ;
\VS{30}ceux qui pleurent comme ne pleurant pas, ceux qui se réjouissent comme ne se réjouissant pas, ceux qui achètent comme ne possédant pas,
\VS{31}et ceux qui usent de ce monde comme n'en usant pas, car la figure de ce monde passe.
\VS{32}Or je voudrais que vous soyez sans inquiétude. Celui qui n'est pas marié s’occupe des choses du Seigneur, cherchant à plaire au Seigneur.
\VS{33}Mais celui qui est marié s’occupe des choses de ce monde, cherchant à plaire à sa femme, et ainsi il est divisé.
\VS{34}Il y a de même une différence entre la femme mariée et la vierge : Celle qui n’est pas mariée s’occupe des choses du Seigneur, afin d’être sainte de corps et d'esprit ; mais celle qui est mariée s’occupe des choses du monde pour plaire à son mari.
\VS{35}Je dis cela dans votre intérêt, ce n’est pas pour vous tendre un piège, mais pour vous porter à ce qui est bienséant et propre à vous unir au Seigneur sans aucune distraction.
\VS{36}Mais si quelqu'un croit qu’il n’est pas honorable que sa fille dépasse la fleur de l’âge sans être mariée, et qu’il faille la marier, qu'il fasse ce qu'il veut, il ne pèche point ; qu'elle soit mariée.
\VS{37}Mais celui qui a pris une ferme résolution, sans contrainte, et avec l’exercice de sa propre volonté en son cœur, de garder sa fille vierge, celui-là fait bien.
\VS{38}Celui donc qui la marie fait bien, mais celui qui ne la marie pas fait mieux.
\VS{39}La femme est liée par la loi pendant tout le temps que son mari est en vie\FTNT{Dieu est contre le divorce. Pour le Seigneur, le mariage doit être un engagement à vie (Mal. 2:16 ; Ro. 7:1-3).}, mais si son mari meurt, elle est libre de se marier à qui elle veut ; seulement, que ce soit dans le Seigneur.
\VS{40}Elle est néanmoins plus heureuse si elle demeure ainsi, selon mon avis ; or j'estime que j'ai aussi l'Esprit de Dieu.
\Chap{8}
\TextTitle{Viandes sacrifiées aux idoles et les limites de la liberté chrétienne}
\VerseOne{}Pour ce qui concerne les choses qui sont sacrifiées aux idoles\FTNT{A Corinthe, on offrait rituellement des viandes sacrifiées aux idoles. A ces occasions, certaines parties des animaux sacrifiés étaient déposées sur l’autel de l’idole, d’autres étaient données aux prêtres et aux adorateurs, qui les mangeaient lors d’un repas ou d’un festin, soit dans le temple, soit dans une maison particulière. Certains morceaux de la chair offerte aux idoles étaient ensuite apportés au marché pour être vendus (Da. 1).}, nous savons que nous avons tous de la connaissance. La connaissance enfle, mais la charité édifie.
\VS{2}Et si quelqu'un croit savoir quelque chose, il n'a encore rien connu comme il faut connaître.
\VS{3}Mais si quelqu'un aime Dieu, il est connu de lui.
\VS{4}Pour ce qui est donc de manger des choses sacrifiées aux idoles, nous savons que l'idole n'est rien dans le monde et qu'il n'y a aucun autre Dieu qu’un seul\FTNT{Paul affirme avec force que le Dieu Créateur n’est pas mélangé avec d’autres divinités. Voir Dt. 6:4.}.
\VS{5}Car s’il est des êtres qui sont appelés dieux, soit dans le ciel, soit sur la terre, comme il existe réellement plusieurs dieux, et plusieurs seigneurs,
\VS{6}nous n’avons pourtant qu'un seul Dieu, qui est le Père, de qui viennent toutes choses, et pour qui nous sommes ; et un seul Seigneur : Jésus-Christ, par qui sont toutes choses, et par qui nous sommes.
\VS{7}Mais tous n’ont pas cette connaissance. Car quelques-uns, d’après la manière dont ils envisagent encore l'idole, mangent de ces choses comme étant sacrifiées aux idoles, et leur conscience qui est faible en est souillée.
\VS{8}Ce n’est pas une viande qui nous rend agréables à Dieu ; car si nous en mangeons, nous n'avons rien de plus ; si nous n’en mangeons pas, nous n’avons rien de moins.
\VS{9}Mais prenez garde que cette liberté que vous avez ne soit en quelque sorte un scandale pour les faibles.
\VS{10}Car si quelqu'un te voit, toi qui as de la connaissance, être à table dans le temple des idoles, sa conscience, à lui qui est faible, ne le portera-t-elle pas à manger des choses sacrifiées aux idoles ?
\VS{11}Et ainsi ton frère, qui est faible, et pour lequel Christ est mort, périra par ta connaissance.
\VS{12}Or quand vous péchez ainsi contre vos frères, et que vous blessez leur conscience qui est faible, vous péchez contre Christ.
\VS{13}C'est pourquoi, si la viande scandalise mon frère, je ne mangerai jamais de chair pour ne point scandaliser mon frère.
\Chap{9}
\TextTitle{Paul défend son apostolat\FTNTT{Ga. 1:11 ; 2:21}}
\VerseOne{}Ne suis-je pas apôtre ? Ne suis-je pas libre ? N’ai-je pas vu notre Seigneur Jésus-Christ ? N’êtes-vous pas mon ouvrage dans le Seigneur ?
\VS{2}Si je ne suis pas apôtre pour les autres, je le suis au moins pour vous, car vous êtes le sceau de mon apostolat dans le Seigneur.
\VS{3}C'est là ma défense contre ceux qui me condamnent.
\VS{4}N'avons-nous pas le droit de manger et de boire ?
\VS{5}N'avons-nous pas le droit de mener avec nous une sœur qui soit notre femme, comme font les autres apôtres, et les frères du Seigneur, et Céphas ?
\VS{6}N'y a-t-il que Barnabas et moi qui n'ayons pas le droit de ne pas travailler ?
\TextTitle{Dieu prend soin de ses serviteurs}
\VS{7}Qui est-ce qui va à la guerre à ses propres frais ? Qui est-ce qui plante une vigne et n’en mange pas le fruit ? Qui est-ce qui fait paître un troupeau et ne se nourrit pas du lait du troupeau ?
\VS{8}Ces choses que je dis n’existent-elles que dans la coutume des hommes ? La loi ne dit-elle pas aussi la même chose ?
\VS{9}Car il est écrit dans la Loi de Moïse : Tu ne muselleras pas le bœuf qui foule le grain\FTNT{De. 25:4.}. Dieu se met-il en peine des bœufs ?
\VS{10}Ou parle-t-il uniquement à cause de nous ? Oui, c’est à cause de nous qu’il a été écrit que celui qui laboure doit labourer avec espérance, et celui qui foule le blé, le foule avec l’espérance d’y avoir part.
\VS{11}Si nous avons semé parmi vous des biens spirituels, est-ce une grosse affaire si nous moissonnons vos biens temporels ?
\VS{12}Si d'autres usent de ce droit à votre égard, pourquoi n'en userions-nous pas plutôt qu'eux ? Cependant nous n'avons point usé de ce droit, mais au contraire, nous supportons toutes sortes d'incommodités, afin de ne pas créer d’obstacle à l'Evangile de Christ.
\VS{13}Ne savez-vous pas que ceux qui font le service sacré mangent des choses sacrées ; et que ceux qui servent à l'autel participent à l'autel\FTNT{No. 18:8-31.} ?
\VS{14}Le Seigneur a ordonné que ceux qui annoncent l'Evangile vivent de l'Evangile.
\VS{15}Pour moi, je n’ai usé d’aucun de ces droits, et ce n’est pas afin de les réclamer en ma faveur que j’écris ainsi ; car j’aimerais mieux mourir que de me laisser enlever cette gloire.
\VS{16}Car si j'évangélise, ce n’est pas pour moi un sujet de gloire, c’est parce que la nécessité m'en est imposée ; et malheur à moi si je n'évangélise pas !
\VS{17}Si je le fais de bon cœur, j’en aurai la récompense ; mais si je le fais malgré moi, c’est une charge qui m’est confiée.
\VS{18}Quelle récompense en ai-je donc ? C’est qu'en prêchant l'Evangile, je prêche l'Evangile de Christ sans qu'il en coûte rien\FTNT{Paul annonçait l’Evangile gratuitement. Donnez gratuitement : C’est la suite logique des choses, on reçoit gratuitement et on donne gratuitement. Si nous sommes comme Christ (car là est le sens du mot disciple), nous devons agir comme lui. Il a donné ses enseignements et nourrit les gens gratuitement. Dans Ap. 21:6 et 22:17, le Seigneur invite toutes les personnes qui ont soif à venir s’abreuver gratuitement. Alors pourquoi vendre la parole c’est-à-dire l’eau qu’on a reçue gratuitement ? Nous devons donner gratuitement.}, afin que je n'abuse pas de mon autorité dans l'Evangile.
\TextTitle{L'attitude d'un vrai serviteur de Dieu}
\VS{19}Car bien que je sois libre à l'égard de tous, je me suis pourtant rendu le serviteur de tous, afin de gagner plus de personnes.
\VS{20}Avec les Juifs, j’ai été comme Juif, afin de gagner les Juifs ; avec ceux qui sont sous la loi, comme si j'étais sous la loi, afin de gagner ceux qui sont sous la loi ;
\VS{21}avec ceux qui sont sans loi, comme si j'étais sans loi (quoique je ne sois point sans la Loi de Dieu, étant sous la Loi de Christ), afin de gagner ceux qui sont sans loi.
\VS{22}J’ai été faible avec les faibles, afin de gagner les faibles ; je me suis fait tout à tous, afin d’en sauver au moins quelques-uns.
\VS{23}Je fais cela à cause de l'Evangile, afin que j'en sois fait aussi participant avec les autres.
\VS{24}Ne savez-vous pas que ceux qui courent dans le stade, courent tous, mais qu’un seul remporte le prix ? Courez de manière à le remporter.
\VS{25}Tout homme qui combat, vit entièrement de régime ; et ces gens-là le font pour obtenir une couronne corruptible\FTNT{Couronne corruptible : Aux Jeux panhelléniques, il n’y avait qu’un seul vainqueur qui remportait pour prix une couronne de feuillage. Sur chacun des sites, les couronnes étaient fabriquées avec des feuillages différents : – A Olympie, c’était une couronne d’olivier sauvage – A Delphes, une couronne de laurier – A l’Isthme (Corinthe), une couronne de pin – A Némée, une couronne de céleri. En plus de sa couronne, l’athlète victorieux recevait un ruban de laine rouge. Des amphores remplies d’huile d’olive étaient également remises au vainqueur. A cette époque, l’huile d’olive était extrêmement précieuse et valait beaucoup d’argent. D’autres prix, comme des trépieds en bronze (grands vases munis de trois pieds), des boucliers en bronze ou des coupes en argent, pouvaient aussi faire partie des lots. La modeste couronne de feuillage était cependant la plus haute récompense attribuée alors dans le monde grec, car elle garantissait l’honneur et le respect de tous à celui qui la recevait.} ; mais nous, faisons-le pour une couronne incorruptible.
\VS{26}Moi donc je cours, non pas comme à l’aventure ; je combats, mais non pas comme battant l'air.
\VS{27}Mais je traite durement mon corps et je le tiens assujetti, de peur d’être moi-même désapprouvé après avoir prêché aux autres.
\Chap{10}
\TextTitle{Paul donne l'exemple d'Israël dans le désert}
\VerseOne{}Mes frères, je ne veux pas que vous ignoriez que nos pères ont tous été sous la nuée, et qu'ils ont tous passé au travers de la mer,
\VS{2}et qu'ils ont tous été baptisés en Moïse dans la nuée et dans la mer ;
\VS{3}et qu'ils ont tous mangé la même viande spirituelle ;
\VS{4}et qu'ils ont tous bu le même breuvage spirituel : Car ils buvaient de l'eau du rocher spirituel qui les suivait, et ce rocher\FTNT{Jésus-Christ, le Rocher des âges. Voir Es. 8:13-17.} était Christ.
\VS{5}Mais la plupart d’entre eux ne furent point agréables à Dieu puisqu’ils périrent dans le désert.
\VS{6}Or ces choses ont été des exemples pour nous, afin que nous ne convoitions point des choses mauvaises, comme eux-mêmes les ont convoitées.
\VS{7}Ne devenez point idolâtres, comme quelques-uns d’entre eux, selon qu'il est écrit : Le peuple s’assit pour manger et pour boire, puis ils se levèrent pour jouer\FTNT{Ex. 32:6.}.
\VS{8}Ne nous livrons pas à la fornication, comme quelques-uns d’entre eux s’y livrèrent, de sorte qu’il en tomba vingt-trois mille en un jour\FTNT{No. 25:9.}.
\VS{9}Ne tentons\FTNT{Tenter : Du grec «~ekpeirazo~» : mettre à l’épreuve ; éprouver le caractère de Dieu et son pouvoir.} point Christ, comme le tentèrent\FTNT{Tenter : Du grec «~peirazo~» : essayer si une chose peut être faite ; éprouver malicieusement, astucieusement, pour prouver ses sentiments et ses jugements ; essayer ou éprouver la foi, la vertu, le caractère par la séduction du péché ; solliciter à pécher ; infliger des maux dans le but d’éprouver. Ce terme est aussi utilisé lorsque les hommes veulent tenter Dieu en montrant leur méfiance, par une conduite impie ou méchante, pour éprouver la justice et la patience de Dieu, et le défier, pour le pousser à donner une preuve de ses perfections.} quelques-uns d’entre eux qui périrent par les serpents\FTNT{No. 21:6-9.}.
\VS{10}Ne murmurez point, comme quelques-uns d’entre eux qui périrent par le destructeur\FTNT{No. 14:2-29 ; No. 26:63-65.}.
\TextTitle{L'Eglise doit s'instruire par l'expérience d'Israël}
\VS{11}Or toutes ces choses leur sont arrivées pour servir d’exemples, et elles ont été écrites pour notre instruction, comme étant ceux auxquels les derniers temps sont parvenus.
\VS{12}Que celui donc qui pense demeurer debout prenne garde qu'il ne tombe.
\VS{13}Aucune tentation ne vous a éprouvés, qui n’ait été une tentation humaine, et Dieu qui est fidèle ne permettra pas que vous soyez tentés au-delà de vos forces, mais avec la tentation il préparera aussi le moyen d’en sortir, afin que vous puissiez la supporter.
\VS{14}C'est pourquoi, mes bien-aimés, fuyez l'idolâtrie.
\VS{15}Je vous parle comme à des personnes intelligentes, jugez vous-mêmes de ce que je dis.
\TextTitle{Distinction entre le repas et l'idolâtrie}
\VS{16}La coupe de bénédiction, que nous bénissons, n'est-elle pas la communion du sang de Christ ? Et le pain que nous rompons, n'est-il pas la communion au corps de Christ ?
\VS{17}Parce qu'il n'y a qu'un seul pain, nous qui sommes plusieurs sommes un seul corps ; car nous sommes tous participants du même pain.
\VS{18}Voyez l'Israël selon la chair, ceux qui mangent les sacrifices ne sont-ils pas en communion avec l'autel ?
\VS{19}Que dis-je donc ? Que l'idole soit quelque chose ? Ou que ce qui est sacrifié à l'idole soit quelque chose ? Nullement.
\VS{20}Mais je dis que les choses que les Gentils sacrifient, ils les sacrifient aux démons, et non à Dieu ; or je ne veux pas que vous soyez en communion avec des démons.
\VS{21}Vous ne pouvez pas boire la coupe du Seigneur et la coupe des démons ; vous ne pouvez pas participer à la table du Seigneur et à la table des démons\FTNT{L’apôtre Paul nous parle de deux sortes de tables : la table de Jézabel (ou des démons) et la table du Seigneur. La table du Seigneur à été révélée à Moïse (Ex. 25:23-30 ; Lé. 24:5-9). Il y avait dessus 12 pains destinés à la consommation des sacrificateurs. Ces pains étaient renouvelés chaque sabbat et représentaient Christ, le Pain de Dieu, qui est l’aliment du croyant-sacrificateur (Jn. 6:33-58). La table de Jézabel nous est présentée dans 1 R. 18:19 : «~Fais maintenant rassembler tout Israël auprès de moi, à la montagne du Carmel, et aussi les quatre cent cinquante prophètes de Baal et les quatre cents prophètes d’Astarté qui mangent à la table de Jézabel~». Jézabel avait à sa table 850 faux prophètes qui partageaient son repas. Voir Ap. 17. Satan est maître en matière de déguisement et d’imitation (2 Co. 11:13-15). Il a donc imité la table du Seigneur et propose aux hommes les mets du roi et le vin de la débauche (Da. 1). Il invite ceux qui cherchent Dieu à sa table afin de les détourner de la vision du ciel. Voir Mt. 6:24 ; Lu. 16:13.}.
\VS{22}Voulons-nous provoquer la jalousie du Seigneur ? Sommes-nous plus forts que lui ?
\TextTitle{La loi de l'amour s'applique dans le manger et le boire\FTNTT{Ro. 14:1-23}}
\VS{23}Toutes choses me sont permises, mais toutes ne sont pas utiles ; toutes choses me sont permises, mais toutes n'édifient pas.
\VS{24}Que personne ne cherche son propre intérêt, mais que chacun cherche celui d’autrui.
\VS{25}Mangez de tout ce qui se vend au marché, sans vous enquérir de rien par motif de conscience\FTNT{1 Ti. 4:3-5.}.
\VS{26}Car la terre avec tout ce qu'elle contient est au Seigneur.
\VS{27}Si un incrédule vous invite et que vous vouliez aller, mangez de tout ce qui sera mis devant vous, sans vous enquérir par motif de conscience.
\VS{28}Mais si quelqu'un vous dit : Ceci a été sacrifié aux idoles, n'en mangez pas, à cause de celui qui vous a avertis, et à cause de la conscience ; car la terre avec tout ce qu'elle contient est au Seigneur.
\VS{29}Je parle ici, non de votre conscience, mais de celle de l'autre. Pourquoi ma liberté serait-elle condamnée par la conscience d'un autre ?
\VS{30}Et si par la grâce j'en suis participant, pourquoi suis-je blâmé pour une chose dont je rends grâces ?
\VS{31}Soit donc que vous mangiez, soit que vous buviez, ou que vous fassiez quelque autre chose, faites tout à la gloire de Dieu.
\VS{32}Soyez tels que vous ne donniez aucun scandale ni aux Juifs, ni aux Grecs, ni à l'Eglise de Dieu,
\VS{33}de la même manière que moi aussi, je m’efforce en toutes choses de complaire à tous, cherchant, non pas mon avantage, mais celui du plus grand nombre, afin qu’ils soient sauvés.
\Chap{11}
\VerseOne{}Soyez mes imitateurs comme je le suis moi-même de Christ.
\TextTitle{Homme et femme devant Dieu}
\VS{2}Or mes frères, je vous loue de ce que vous vous souvenez de tout ce qui me concerne, et de ce que vous retenez mes instructions comme je vous les ai données.
\VS{3}Mais je veux que vous sachiez que Christ est le chef\FTNT{Le mot «~chef~» vient du grec «~kephal~» qui signifie tête. Jésus-Christ est la seule tête et l’unique chef de l’Eglise (Ep. 1:22-23 ; Col. 1:18). Toute personne qui se proclame la tête de l’église devient naturellement antéchrist.} de tout homme, que l’homme est le chef de la femme, et que Dieu est le chef de Christ.
\VS{4}Tout homme qui prie ou qui prophétise, ayant quelque chose sur la tête, déshonore son chef.
\VS{5}Toute femme au contraire qui prie, ou qui prophétise sans avoir la tête couverte, déshonore son chef, c'est comme si elle était rasée.
\VS{6}Car si une femme n'est pas couverte, qu’on lui coupe aussi les cheveux. Or, s'il est honteux pour une femme d'avoir les cheveux coupés, ou d'être rasée, qu'elle se voile.
\VS{7}Car pour ce qui est de l'homme, il ne doit point couvrir sa tête, vu qu'il est l'image et la gloire de Dieu ; mais la femme est la gloire de l'homme.
\VS{8}Parce que l'homme n'a point été tiré de la femme, mais la femme a été tirée de l'homme.
\VS{9}Et aussi l'homme n'a pas été créé pour la femme, mais la femme pour l'homme.
\VS{10}C'est pourquoi la femme à cause des anges doit avoir sur la tête une marque de l’autorité de son mari dont elle dépend.
\VS{11}Toutefois, dans le Seigneur, l'homme n'est point sans la femme ni la femme sans l'homme.
\VS{12}Car comme la femme est par l'homme, de même l'homme est par la femme, et tout cela procède de Dieu.
\VS{13}Jugez-en vous-mêmes : Est-il convenable que la femme prie Dieu sans être couverte ?
\VS{14}La nature elle-même ne vous enseigne-t-elle pas que c’est une honte pour l'homme d’avoir de longs cheveux,
\VS{15}mais que c’est une gloire pour la femme de porter des longs cheveux, parce que la chevelure lui a été donnée pour lui servir de voile ?
\VS{16}Si quelqu'un aime à contester, nous n'avons pas une telle coutume, ni les églises de Dieu.
\TextTitle{Le repas du Seigneur et les abus dénoncés par Paul}
\VS{17}Or en ce que je vais vous dire, je ne vous loue point : C’est que vous vous assemblez, non pour devenir meilleurs, mais pour empirer.
\VS{18}Car premièrement, lorsque vous vous réunissez en assemblée, j'apprends qu'il y a des divisions parmi vous et j'en crois une partie,
\VS{19}car il faut qu'il y ait même des hérésies parmi vous, afin que ceux qui sont dignes d’être approuvés soient reconnus parmi vous.
\VS{20}Quand donc vous vous assemblez ainsi tous ensemble, ce n'est pas pour manger le repas du Seigneur ;
\VS{21}car, quand on se met à table, chacun commence par prendre son souper particulier, et l'un a faim tandis que l'autre est ivre.
\VS{22}N'avez-vous donc pas de maisons pour manger et pour boire ? Ou méprisez-vous l'Eglise de Dieu et faites-vous honte à ceux qui n'ont rien ? Que vous dirai-je ? Vous louerai-je ? Je ne vous loue point en cela.
\TextTitle{Le repas du Seigneur}
\VS{23}Car j'ai reçu du Seigneur ce qu'aussi je vous ai donné ; c’est que le Seigneur Jésus, la nuit où il fut trahi, prit du pain,
\VS{24}et après avoir rendu grâces, le rompit et dit : Prenez, mangez : Ceci est mon corps qui est rompu pour vous ; faites ceci en mémoire de moi.
\VS{25}De même aussi après le souper, il prit la coupe, en disant : Cette coupe est la nouvelle alliance en mon sang ; faites ceci toutes les fois que vous en boirez, en mémoire de moi\FTNT{Mt. 26:26-28 ; Mc. 14:22-24 ; Lu. 22:19-20.}.
\VS{26}Car toutes les fois que vous mangerez de ce pain, et que vous boirez de cette coupe, vous annoncerez la mort du Seigneur, jusqu’à ce qu'il vienne.
\VS{27}C'est pourquoi quiconque mangera de ce pain ou boira de la coupe du Seigneur indignement, sera coupable envers le corps et le sang du Seigneur.
\VS{28}Que chacun donc s'éprouve soi-même, et ainsi qu'il mange de ce pain, et qu'il boive de cette coupe.
\VS{29} Car celui qui en mange et qui en boit indignement, mange et boit sa condamnation, ne distinguant point le corps du Seigneur.
\VS{30}C’est pour cela qu’il y a parmi vous beaucoup d’infirmes et de malades, et que plusieurs dorment.
\VS{31}Car si nous nous jugions nous-mêmes, nous ne serions point jugés.
\VS{32}Mais quand nous sommes jugés, nous sommes enseignés par le Seigneur, afin que nous ne soyons point condamnés avec le monde.
\VS{33}C'est pourquoi, mes frères, quand vous vous assemblez pour manger, attendez-vous les uns les autres.
\VS{34}Et si quelqu'un a faim, qu'il mange dans sa maison, afin que vous ne vous assembliez pas pour votre condamnation.  Touchant les autres points, je les réglerai quand je serai arrivé.
\Chap{12}
\TextTitle{L'Esprit révèle Christ}
\VerseOne{}Pour ce qui concerne les dons spirituels, je ne veux point, mes frères, que vous soyez ignorants.
\VS{2}Vous savez que lorsque vous étiez des gentils, vous vous laissiez entraîner vers les idoles muettes, selon que vous étiez conduits.
\VS{3}C'est pourquoi je vous fais savoir que personne, s’il parle par l'Esprit de Dieu, ne dit : Jésus est anathème ! Et personne ne peut dire : Jésus est le Seigneur ! Si ce n’est par le Saint-Esprit.
\TextTitle{La diversité des dons de l'Esprit\FTNTT{Ep. 4:7-16}}
\VS{4} Or il y a diversité de dons, mais il n'y a qu'un même Esprit.
\VS{5}Il y a aussi diversité de ministères, mais il n'y a qu'un même Seigneur.
\VS{6}Il y a aussi diversité d'opérations, mais il n'y a qu'un même Dieu qui opère toutes choses en tous.
\VS{7}Or à chacun est donnée la manifestation de l'Esprit pour l'utilité commune.
\VS{8}Car à l'un est donnée par l'Esprit, la parole de sagesse ; et à l'autre par le même Esprit, la parole de connaissance ;
\VS{9}et à un autre, la foi par ce même Esprit ; à un autre, les dons de guérison par ce même Esprit ;
\VS{10}et à un autre, les opérations des miracles ; à un autre, la prophétie ; à un autre, le don de discerner les esprits ; à un autre, la diversité de langues ; et à un autre, le don d'interpréter les langues.
\VS{11}Un seul et même Esprit opère toutes ces choses, distribuant à chacun ses dons en particulier comme il lui plaît.
\TextTitle{Chaque membre à son utilité dans le corps de Christ}
\VS{12}Car comme le corps est un, et cependant a plusieurs membres, et comme tous les membres du corps, malgré leur nombre, ne forment qu’un seul corps, il en est de même de Christ.
\VS{13}Nous avons tous, en effet, été baptisés d'un même Esprit\FTNT{Le baptême du Saint-Esprit : Les signes du baptême du Saint-Esprit (la conversion) sont les fruits de l’Esprit et sont abordés en Ga. 5:22. A aucun endroit, les écritures stipulent que le parler en langues, qui est un don gratuit (Mt. 7:16-20), est en soi le signe du baptême du Saint-Esprit. Ainsi, il nous est dit que chaque croyant en Christ a le Saint-Esprit (1 Co. 12:13 ; Ro. 8:9 ; Ep. 1:13-14) mais que tous les croyants ne parlent pas forcément en langues (1 Co. 12:29-31).}, pour être un même corps, soit Juifs, soit Grecs, soit esclaves, soit libres, nous avons tous, dis-je, été abreuvés d'un seul Esprit.
\VS{14}Ainsi, le corps n’est pas un seul membre, mais il est formé de plusieurs membres.
\VS{15}Si le pied dit : Parce que je ne suis pas la main, je ne suis point du corps ; ne serait-il pas pourtant du corps ?
\VS{16}Et si l'oreille dit : Parce que je ne suis pas l’œil, je ne suis point du corps ; ne serait-elle pas pourtant du corps ?
\VS{17}Si tout le corps est l’œil, où serait l'ouïe ? Si tout est l'ouïe, où serait l'odorat ?
\VS{18}Mais maintenant Dieu a placé chaque membre dans le corps comme il a voulu.
\VS{19}Et si tous étaient un seul membre, où serait le corps ?
\VS{20}Maintenant donc, il y a plusieurs membres et un seul corps.
\VS{21}L’œil ne peut pas dire à la main : Je n'ai pas besoin de toi ; ni la tête dire aux pieds : Je n'ai pas besoin de vous.
\VS{22}Et qui plus est, les membres du corps qui semblent être les plus faibles sont beaucoup plus nécessaires ;
\VS{23}et ceux que nous estimons être les moins honorables au corps, nous les entourons d’un plus grand honneur. Ainsi, nos membres les moins décents reçoivent le plus d’honneur,
\VS{24}Car les parties qui sont belles en nous, n’en ont pas besoin. Mais Dieu a disposé le corps de manière à donner plus d’honneur à ce qui en manquait,
\VS{25}afin qu'il n'y ait pas de division dans le corps, mais que les membres aient un soin mutuel les uns des autres.
\VS{26}Et si l'un des membres souffre quelque chose, tous les membres souffrent avec lui ; si l'un des membres est honoré, tous les membres ensemble se réjouissent avec lui.
\VS{27}Vous êtes le corps de Christ, et vous êtes chacun l’un de ses membres.
\VS{28}Et Dieu a établi dans l'Eglise premièrement des apôtres, deuxièmement des prophètes, troisièmement des docteurs, ensuite ceux qui ont le don des miracles, puis ceux qui ont les dons de guérir, de secourir, de gouverner, de parler diverses langues.
\VS{29}Tous sont-ils apôtres ? Tous sont-ils prophètes ? Tous sont-ils docteurs ? Tous ont-ils le don des miracles ?
\VS{30}Tous ont-ils les dons de guérisons ? Tous parlent-ils diverses langues ? Tous interprètent-ils ?
\VS{31}Désirez avec ardeur des dons plus excellents, et je vais vous montrer la voie la plus excellente.
\Chap{13}
\TextTitle{L'amour est la base de tout}
\VerseOne{}Quand je parlerais toutes les langues des hommes\FTNT{Les langues des hommes. Les 120 Galiléens ont été rendus capables de s’exprimer dans diverses langues afin de pouvoir annoncer la vérité aux personnes en voyage à Jérusalem dans leurs propres langues. Voir Es. 28:11-12 ; Ac. 2:1-13.}, et même des anges\FTNT{La langue des anges ou langue inconnue est incompréhensible à notre intelligence, elle est un des moyens par lequel nous disons des mystères à Dieu. Voir Ro. 8:25-26 ; 1 Co. 14:2 et 28. Il faut une interprétation si l’on veut parler cette langue dans l’assemblée à cause des non croyants qui nous visitent (1 Co. 14:23). Voir Mc. 16:17.}, si je n'ai pas la charité\FTNT{Il est question ici de l’amour «~agape~» : l’amour divin et désintéressé, l’amour fraternel.}, je suis un airain qui résonne ou une cymbale qui retentit.
\VS{2}Et quand j'aurais le don de prophétie et que je connaîtrais tous les mystères et la science de toutes choses ; et quand j'aurais même toute la foi qu'on puisse avoir, jusqu’à transporter les montagnes, si je n'ai pas la charité, je ne suis rien.
\VS{3}Et quand je distribuerais tous mes biens pour la nourriture des pauvres, quand je livrerais mon corps pour être brûlé, si je n'ai pas la charité, cela ne me sert à rien.
\VS{4}La charité est patiente, la charité est douce, la charité n'est point envieuse, la charité n'use point d'insolence, elle ne s’enfle point d’orgueil,
\VS{5}elle ne fait rien de malhonnête, elle ne cherche point son intérêt, elle ne s’irrite point, elle n’impute pas le mal,
\VS{6}elle ne se réjouit point de l'injustice, mais elle se réjouit de la vérité.
\VS{7}Elle couvre\FTNT{Dans ce passage, le grec utilisé, «~stego~», signifie «~toit, couverture, protéger ou garder en recouvrant, préserver~» (Pr. 10:12 ; Pr. 17:9). La charité ne rappelle pas sans cesses les erreurs des uns et des autres, mais sait préserver en gardant secret les fautes est expiées. Par contre, en aucun cas elle ne permet la compromission du péché en ne dénonçant pas les oeuvres des ténèbres (Mt. 18:15-18 ; Ja. 5:19-20).} tout, elle croit tout, elle espère tout, elle supporte tout.
\VS{8}La charité ne périt jamais. Les prophéties seront abolies et les langues cesseront, la connaissance sera abolie.
\VS{9}Car nous connaissons en partie et nous prophétisons en partie.
\VS{10}Mais quand la perfection sera venue, alors ce qui est en partie sera aboli.
\VS{11}Quand j'étais enfant, je parlais comme un enfant, je jugeais comme un enfant, je pensais comme un enfant ; mais quand je suis devenu homme, j'ai aboli ce qui était de l'enfance.
\VS{12}Car aujourd’hui nous voyons au moyen d’un miroir, de manière obscure, mais alors nous verrons face à face. Aujourd’hui je connais en partie, mais alors je connaîtrai comme j'ai été connu.
\VS{13}Maintenant ces trois choses demeurent : La foi, l'espérance et la charité ; mais la plus excellente de ces trois vertus c'est la charité.
\Chap{14}
\TextTitle{Importance du don de prophétie}
\VerseOne{}Recherchez la charité. Désirez avec ardeur les dons spirituels, mais surtout celui de prophétiser.
\VS{2}Parce que celui qui parle une langue inconnue ne parle point aux hommes, mais à Dieu, car personne ne le comprend, et c’est en esprit qu’il dit des mystères.
\VS{3}Mais celui qui prophétise, édifie, exhorte et console les hommes qui l'entendent.
\VS{4}Celui qui parle une langue inconnue s'édifie lui-même, mais celui qui prophétise édifie l'Eglise.
\VS{5}Je désire que vous parliez tous diverses langues, mais encore plus que vous prophétisiez. Celui qui prophétise est plus grand que celui qui parle diverses langues, à moins que ce dernier n’interprète, afin que l'Eglise en reçoive de l'édification.
\VS{6}Maintenant donc, mes frères, si je viens à vous et que je parle des langues inconnues, que vous servira cela si je ne vous parle pas par révélation, ou par science, ou par prophétie, ou par doctrine ?
\VS{7}De même, si les choses inanimées qui rendent un son, comme une flûte ou une harpe, ne rendent pas des sons distincts, comment reconnaîtra-t-on ce qui est joué sur la flûte ou sur la harpe ?
\VS{8}Et si la trompette rend un son confus, qui se préparera à la bataille ?
\VS{9}De même vous, si vous ne prononcez dans votre langue une parole distincte, comment saura-t-on ce que vous dites ? Car vous parlerez en l'air.
\VS{10}Et il y a, selon qu'il se rencontre, tant de divers sons dans le monde, et cependant aucun de ces sons n'est muet ;
\VS{11}mais si je ne sais point ce qu'on veut signifier par la parole, je serai un barbare pour celui qui parle, et celui qui parle sera un barbare pour moi.
\VS{12}Ainsi, puisque vous désirez avec ardeur les dons spirituels, que ce soit pour l’édification de l'Eglise que vous cherchiez à en posséder abondamment.
\VS{13}C'est pourquoi que celui qui parle une langue inconnue prie pour avoir le don d’interpréter.
\VS{14}Car si je prie dans une langue inconnue mon esprit est en prière, mais l'intelligence que j'en ai, est sans fruit.
\VS{15}Que faire donc ? Je prierai par l’esprit, mais je prierai aussi d'une manière à être entendu ; je chanterai par l’esprit, mais je chanterai aussi d'une manière à être entendu.
\VS{16}Autrement, si tu rends grâces par l’esprit, comment celui qui est du simple peuple dira-t-il Amen ! à ton action de grâces\FTNT{L’expression «~actions de grâces~» vient du grec «~eucharisteo~» ce qui signifie être reconnaissant, rendre grâces, remercier. Contrairement à ce que l’on enseigne dans beaucoup d’églises, il n’est pas question ici de faire une offrande d’argent mais de se montrer reconnaissant envers le Seigneur. Voir aussi commentaires en Lé. 3 et Lé. 7.}, puisqu'il ne sait pas ce que tu dis ?
\VS{17}Il est vrai que tu rends grâces, mais l’autre n’est pas édifié.
\VS{18}Je rends grâces à mon Dieu de ce que je parle plus de langues que vous tous.
\VS{19}Mais j'aime mieux prononcer dans l'Eglise cinq paroles d'une manière à être entendu, afin d’instruire aussi les autres, que dix mille paroles dans une langue inconnue.
\VS{20}Mes frères, ne soyez point des enfants sous le rapport du jugement, mais soyez des enfants à l’égard de la malice ; et à l'égard du jugement, soyez des hommes faits.
\VS{21}Il est écrit dans la loi : je parlerai à ce peuple par des gens d'une autre langue, et par des lèvres étrangères, et ils ne m’écouteront pas même ainsi, dit le Seigneur\FTNT{Es. 28:11.}.
\VS{22}C’est pourquoi les langues sont un signe, non pour les croyants, mais pour les non-croyants ; la prophétie, au contraire, est un signe, non pour les non-croyants, mais pour les croyants.
\TextTitle{L'exercice des dons spirituels dans les églises locales}
\VS{23}Si donc, l’Eglise entière s'assemble en un corps, et que tous parlent des langues étrangères et qu'il entre des gens du commun peuple ou des non-croyants, ne diront-ils pas que vous êtes hors de sens ?
\VS{24}Mais si tous prophétisent, et qu'il entre un non-croyant ou quelqu'un du commun peuple, il est convaincu par tous et il est jugé de tous,
\VS{25}ainsi les secrets de son cœur sont manifestés, de telle sorte qu'il tombera sur sa face, il adorera Dieu et publiera que Dieu est véritablement parmi vous.
\VS{26}Que faire donc mes frères ? Lorsque vous vous assemblez, les uns ou les autres parmi vous ont-ils un cantique, une instruction, une langue étrangère, une révélation, une interprétation, que tout se fasse pour l'édification.
\VS{27}Et si quelqu'un parle une langue inconnue, que cela se fasse par deux, ou tout au plus par trois, chacun à son tour, et que quelqu’un interprète ;
\VS{28}s'il n'y a point d'interprète, que cet homme se taise dans l'Eglise, et qu'il parle à lui-même et à Dieu.
\VS{29}Et que deux ou trois prophètes parlent, et que les autres en jugent ;
\VS{30}et si quelque chose est révélé à un autre qui est assis, que le premier se taise.
\VS{31}Car vous pouvez tous prophétiser l'un après l'autre, afin que tous soient instruits et que tous soient consolés.
\VS{32}Et les esprits des prophètes sont soumis aux prophètes.
\VS{33}Car Dieu n'est point un Dieu de confusion, mais de paix, comme on le voit dans toutes les églises des saints.
\VS{34}Que les femmes qui sont parmi vous se taisent dans les églises ; car il ne leur est point permis d’y parler, mais elles doivent être soumises, comme le dit aussi la loi.
\VS{35}Et si elles veulent s’instruire sur quelque chose, qu'elles interrogent leurs maris à la maison ; car il est honteux à une femme de parler dans l'église.
\VS{36}Est-ce de chez vous que la parole de Dieu est sortie ? Ou est-elle parvenue seulement à vous ?
\VS{37}Si quelqu'un croit être prophète, ou spirituel, qu'il reconnaisse que les choses que je vous écris sont des commandements du Seigneur.
\VS{38}Et si quelqu'un l’ignore, qu'il l’ignore.
\VS{39}C'est pourquoi, mes frères, désirez avec ardeur de prophétiser, et n'empêchez point de parler diverses langues.
\VS{40}Que toutes choses se fassent avec bienséance, et avec ordre.
\Chap{15}
\TextTitle{L'Evangile basé sur la résurrection de Christ}
\VerseOne{}Or, mes frères, je vous rappelle l'Evangile que je vous ai annoncé, que vous avez reçu, et auquel vous vous tenez ferme,
\VS{2}et par lequel vous êtes sauvés, si vous le retenez tel je vous l'ai annoncé ; à moins que vous n'ayez cru en vain. 
\VS{3}Car avant toutes choses, je vous ai donné ce que j'avais aussi reçu, à savoir que Christ est mort pour nos péchés, selon les Ecritures,
\VS{4}et qu'il a été enseveli, et qu'il est ressuscité\FTNT{La Résurrection du Messie. La résurrection de Jésus est un espoir pour tous les êtres humains. Elle est un principe fondamental de la foi chrétienne. Contrairement à toutes les autres religions, la foi chrétienne est la seule qui apporte l’espérance face à la mort. Toutes les autres religions ont été fondées par des hommes, leurs prophètes ou fondateurs sont morts et aucun n’est revenu à la vie. En tant que disciples de Jésus, nous sommes réconfortés par le fait que notre Dieu s'est fait homme, afin de mourir pour nos péchés, et est ressuscité le troisième jour. L’Enfer ne pouvait pas le retenir, et il tient les clés de la mort et de l’Enfer (Ap. 1:18). Voir Jn. 11:25-26. Jésus-Christ est la Résurrection.} le troisième jour, selon les Ecritures ;
\VS{5}et qu'il a été vu de Céphas, et ensuite des douze.
\VS{6}Depuis, il a été vu de plus de cinq cents frères à la fois, dont plusieurs sont encore vivants, et quelques-uns sont morts.
\VS{7}Depuis, il est apparu à Jacques, puis à tous les apôtres.
\VS{8}Après eux tous, il a été vu aussi de moi, comme d'un avorton.
\VS{9}Car je suis le moindre des apôtres, je ne suis pas digne d'être appelé apôtre, parce que j'ai persécuté l'Eglise de Dieu.
\VS{10}Mais par la grâce de Dieu, je suis ce que je suis ; et sa grâce envers moi n'a pas été vaine, mais j'ai travaillé plus qu'eux tous, toutefois non pas moi, mais la grâce de Dieu qui est avec moi.
\VS{11}Soit donc moi, soit eux, nous prêchons ainsi et vous l'avez cru ainsi.
\TextTitle{Importance de la résurrection de Christ}
\VS{12}Or si on prêche que Christ est ressuscité des morts, comment disent quelques-uns d'entre vous qu'il n'y a point de résurrection des morts ?
\VS{13}Car s'il n'y a point de résurrection des morts, Christ aussi n'est point ressuscité.
\VS{14}Et si Christ n'est pas ressuscité, notre prédication est donc vaine, et votre foi aussi est vaine.
\VS{15}Et même nous sommes de faux témoins de la part de Dieu, car nous avons rendu témoignage à l’égard de Dieu qu'il a ressuscité Christ, tandis qu’il ne l’aurait pas ressuscité, si les morts ne ressuscitent point.
\VS{16}Car si les morts ne ressuscitent point, Christ non plus n'est point ressuscité.
\VS{17}Et si Christ n'est pas ressuscité, votre foi est vaine, et vous êtes encore dans vos péchés,
\VS{18}et par conséquent aussi ceux qui dorment en Christ sont perdus.
\VS{19}Si nous n'avons d'espérance en Christ que pour cette vie seulement, nous sommes les plus misérables de tous les hommes.
\TextTitle{Détails sur les résurrections}
\VS{20}Mais maintenant Christ est ressuscité des morts, il est les prémices de ceux qui dorment.
\VS{21}Car puisque la mort est venue par un seul homme, c’est aussi par un homme qu’est venue la résurrection des morts.
\VS{22}Car comme tous meurent en Adam, de même aussi tous seront vivifiés en Christ.
\VS{23}Mais chacun en son rang, Christ comme prémices, puis ceux qui sont à Christ seront vivifiés lors de son avènement.
\VS{24}Ensuite viendra la fin, quand il aura remis le Royaume à Dieu le Père, après avoir aboli tout empire, toute puissance, et toute force.
\VS{25}Car il faut qu'il règne jusqu'à ce qu'il ait mis tous ses ennemis sous ses pieds\FTNT{Ps. 110:1.}.
\VS{26}L'ennemi qui sera détruit le dernier c'est la mort.
\VS{27}Car Dieu a tout mis sous ses pieds. Mais lorsqu’il dit que tout lui a été soumis, il est évident que celui qui lui a soumis toutes choses est excepté.
\VS{28}Et lorsque toutes choses lui auront été soumises, alors le Fils lui-même sera soumis à celui qui lui a soumis toutes choses, afin que Dieu soit tout en tous.
\VS{29}Autrement que feraient ceux qui se font baptiser pour les morts ? Si les morts ne ressuscitent absolument pas, pourquoi se font-ils baptiser pour les morts ?
\VS{30}Et nous, pourquoi sommes-nous en danger à toute heure ?
\VS{31}Tous les jours je suis exposé à la mort, je l’atteste, par la gloire de notre Seigneur Jésus-Christ.
\VS{32}Si j'ai combattu contre les bêtes à Ephèse dans des vues humaines, quel profit m’en revient-il ? Si les morts ne ressuscitent pas, mangeons et buvons, car demain nous mourrons.
\VS{33}Ne soyez point séduits : Les mauvaises compagnies corrompent les bonnes mœurs.
\VS{34}Réveillez-vous pour vivre justement, et ne péchez point ; car quelques-uns ne connaissent pas Dieu, je le dis à votre honte.
\TextTitle{Corps de résurrection}
\VS{35}Mais quelqu'un dira : Comment les morts ressuscitent-ils, et avec quel corps viennent-ils ?
\VS{36}Insensé ! Ce que tu sèmes ne reprend point vie s'il ne meurt pas\FTNT{Jn. 12:24.}.
\VS{37}Et ce que tu sèmes, tu ne sèmes point le corps qui naîtra, c’est un simple grain, de blé peut-être, ou d’une autre semence.
\VS{38}Mais Dieu lui donne le corps comme il veut, et à chacune des semences son propre corps.
\VS{39}Toute chair n'est pas de la même chair, mais autre est la chair des hommes, autre la chair des bêtes, autre celle des poissons, autre celle des oiseaux.
\VS{40}Il y a aussi des corps célestes, et des corps terrestres ; mais autre est l’éclat des corps célestes, et autre celui des corps terrestres.
\VS{41}Autre est l’éclat du soleil, autre l’éclat de la lune, autre l’éclat des étoiles ; même une étoile diffère d'une autre étoile en éclat.
\VS{42}Il en sera aussi de même à la résurrection des morts : Le corps est semé corruptible, il ressuscitera incorruptible.
\VS{43}Il est semé en déshonneur, il ressuscite glorieux ; il est semé en faiblesse, il ressuscite plein de force.
\VS{44}Il est semé corps animal, il ressuscitera corps spirituel. S’il y a un corps animal, il y a aussi un corps spirituel.
\VS{45}Comme aussi il est écrit : Le premier homme, Adam, devint une âme vivante\FTNT{Ge. 2:7.}. Le dernier Adam est devenu un Esprit vivifiant\FTNT{Jn. 5 : 21 ; Ro. 8 : 11. Jésus-Christ est le dernier Adam. Voir Ph. 2:7 ; 1 T. 3:16.}.
\VS{46}Or ce qui est spirituel n'est pas le premier, mais ce qui est animal ; et puis vient ce qui est spirituel.
\VS{47}Le premier homme, étant de la terre, est tiré de la poussière, mais le second homme, à savoir le Seigneur, est du ciel.
\VS{48}Tel qu'est celui qui est tiré de la poussière, tels aussi sont ceux qui sont tirés de la poussière ; et tel qu'est le céleste, tels aussi sont les célestes.
\VS{49}Et comme nous avons porté l'image de celui qui est tiré de la poussière, nous porterons aussi l'image du céleste.
\VS{50}Voici donc ce que je dis, mes frères, c'est que la chair et le sang ne peuvent hériter le Royaume de Dieu, et que la corruption n'hérite pas l'incorruptibilité.
\TextTitle{Mystère de la résurrection\FTNTT{1 Th. 4:14-17}}
\VS{51}Voici, je vous dis un mystère : Nous ne mourrons pas tous, mais tous nous serons changés,
\VS{52}en un instant, en un clin d’œil, à la dernière trompette\FTNT{Le mot dernier dans ce passage est «~eschatos~»  qui signifie «~dernier en temps ou en lieu, dernier dans des séries de lieux, dernier dans une succession dans le temps~». Paul associe le mystère de la résurrection à la dernière trompette. Or, dans le livre d'Apocalypse il n'y a que sept trompettes (Ap. 8 ; 9 et 11:15-19), et c'est à la dernière, c'est-à-dire la septième que le mystère de Dieu s'accomplit.}. La trompette sonnera, et les morts ressusciteront incorruptibles, et nous, nous serons changés.
\VS{53}Car il faut que ce corps corruptible revête l'incorruptibilité, et que ce corps mortel revête l'immortalité.
\TextTitle{La mort engloutie}
\VS{54}Lorsque ce corps corruptible aura revêtu l'incorruptibilité, et que ce corps mortel aura revêtu l'immortalité, alors cette parole de l'Ecriture sera accomplie : La mort a été engloutie dans la victoire\FTNT{Es. 25:8.}.
\VS{55}Ô mort, où est ta victoire ? Ô mort, où est ton aiguillon\FTNT{Os. 13:14.} ?
\VS{56}L'aiguillon de la mort c'est le péché ; et la puissance du péché c'est la loi.
\VS{57}Mais grâces soient rendues à Dieu qui nous a donné la victoire par notre Seigneur Jésus-Christ !
\VS{58}C'est pourquoi, mes frères bien-aimés, soyez fermes, inébranlables, vous appliquant toujours avec un nouveau zèle à l’œuvre du Seigneur, sachant que votre travail ne sera pas vain dans le Seigneur.
\Chap{16}
\TextTitle{Instructions et salutations de Paul}
\VerseOne{}Pour ce qui concerne la collecte en faveur des saints, faites comme je l’ai ordonné aux églises de Galatie.
\VS{2}C’est que chaque premier jour de la semaine, chacun de vous mette à part chez lui ce qu’il pourra assembler, selon la prospérité que Dieu lui accordera, afin qu’on n’attende pas mon arrivée pour recueillir les dons.
\VS{3}Puis quand je serai arrivé, j'enverrai les personnes que vous aurez approuvées avec des lettres pour porter votre libéralité à Jérusalem.
\VS{4}Et s’il convient que j’y aille moi-même, ils viendront aussi avec moi.
\VS{5}J'irai donc chez vous quand j’aurai traversé la Macédoine, car je traverserai par la Macédoine.
\VS{6}Et peut-être que je séjournerai parmi vous, ou même que j'y passerai l'hiver, afin que vous me conduisiez partout où j’irai.
\VS{7}Car je ne veux pas cette fois vous voir en passant, mais j’espère demeurer quelque temps auprès de vous, si le Seigneur le permet.
\VS{8}Toutefois, je resterai à Ephèse jusqu'à la Pentecôte.
\VS{9}Car une grande porte et un accès efficace m'y est ouverte, et les adversaires sont nombreux.
\VS{10}Si Timothée arrive, faites en sorte qu'il soit en sûreté parmi vous, car il travaille à l’œuvre du Seigneur comme moi-même.
\VS{11}Que personne donc ne le méprise. Accompagnez-le en paix, afin qu'il vienne vers moi, car je l'attends avec les frères.
\VS{12}Quant à Apollos, notre frère, je l'ai beaucoup exhorté à se rendre chez vous avec les frères, mais ce n’était décidément pas sa volonté de le faire maintenant ; il partira quand il en aura l’occasion.
\VS{13}Veillez, soyez fermes dans la foi, agissez courageusement, fortifiez-vous.
\VS{14}Que tout ce que vous faites se fasse avec charité.
\VS{15}Or, mes frères, vous connaissez la famille de Stéphanas, et vous savez qu'elle est les prémices de l'Achaïe, et qu’elle s’est entièrement appliquée au service des saints.
\VS{16}Je vous prie de vous soumettre à de tels hommes, et à tous de ceux qui s’emploient à l’œuvre du Seigneur, et qui travaillent avec nous.
\VS{17}Je me réjouis de l’arrivée de Stéphanas, de Fortunatus, et d’Achaïcus, parce qu'ils ont suppléé à votre absence.
\VS{18}Car ils ont tranquillisé mon esprit et le vôtre. Ayez donc de la considération pour de telles personnes.
\VS{19}Les églises d'Asie vous saluent. Aquilas et Priscille, avec l'église qui est dans leur maison, vous saluent affectueusement dans le Seigneur.
\VS{20}Tous les frères vous saluent. Saluez-vous les uns les autres par un saint baiser.
\VS{21}Je vous salue, moi Paul, de ma propre main.
\VS{22}Si quelqu'un n'aime pas le Seigneur Jésus-Christ, qu'il soit anathème ! Maranatha\FTNT{Maranatha signifie littéralement «~Le Seigneur vient !~}.
\VS{23}Que la grâce de notre Seigneur Jésus-Christ soit avec vous !
\VS{24}Mon amour est avec vous tous en Jésus-Christ. Amen.
\PPE{}
\end{multicols}

%\clearpage\ShortTitle{2 Corinthiens}\BookTitle{2 Corinthiens}\BFont
\noindent\hrulefill
{\footnotesize
\textit{
\bigskip
{\centering{}
\\Auteur : Paul avec Tite et Luc
\\Thème : L'autorité de Paul
\\Date de rédaction : Env. 57 ap. J.-C.\\}
}
%\bigskip
\textit{
\\Dans l’antiquité, Corinthe, capitale de l’Achaïe, était la ville la plus prospère et la plus puissante de Grèce. Située sur
un isthme séparant la mer Egée de la mer Ionienne, Corinthe était au carrefour de l’Asie et de l’Italie et constituait un
véritable centre commercial où les produits orientaux et occidentaux se croisaient.
%\bigskip
\\Rédigée quelques mois après la première, la seconde lettre de Paul aux Corinthiens fait état d’une vague de méfiance à l’égard de Paul et exprime les souffrances qui furent les siennes et qui somme toute authentifient son apostolat.\bigskip
}
}
\par\nobreak\noindent\hrulefill
\begin{multicols}{2}
\Chap{1}
\TextTitle{Introduction}
\VerseOne{}Paul, apôtre de Jésus-Christ par la volonté de Dieu, et le frère Timothée, à l'église de Dieu qui est à Corinthe, et à tous les saints qui sont dans toute l'Achaïe.
\VS{2}Que la grâce et la paix vous soient données de la part de Dieu notre Père et du Seigneur Jésus-Christ.
\TextTitle{Consolation de Paul dans ses afflictions}
\VS{3}Béni soit Dieu, le Père de notre Seigneur Jésus-Christ, le Père des miséricordes et le Dieu de toute consolation,
\VS{4}qui nous console dans toutes nos afflictions, afin que par la consolation dont nous sommes l’objet de la part de Dieu, nous puissions consoler ceux qui se trouvent dans l’affliction.
\VS{5}Car de même que les souffrances de Christ abondent en nous, de même notre consolation abonde par Christ.
\VS{6}Et si nous sommes affligés, c'est pour votre consolation et pour votre salut ; si nous sommes consolés, c'est pour votre consolation et pour votre salut qui se réalise par la patience à supporter les mêmes souffrances que nous endurons aussi.
\VS{7}Et l'espérance que nous avons de vous est ferme, sachant que comme vous êtes participants des souffrances, de même aussi vous le serez de la consolation.
\VS{8}Car mes frères, nous ne voulons pas que vous ignoriez l’affliction qui nous est survenue en Asie, que nous avons été excessivement accablés, au-delà de nos forces, de telle sorte que nous avions perdu l'espérance de conserver notre vie.
\VS{9}Et nous regardions comme certain notre arrêt de mort, afin de ne pas placer notre confiance en nous-mêmes, mais en Dieu qui ressuscite les morts.
\VS{10}C’est lui qui nous a délivrés et qui nous délivrera d'une si grande mort, et en qui nous espérons qu'il nous délivrera aussi à l'avenir.
\VS{11}Etant aussi aidés par la prière que vous faites pour nous, afin que la grâce obtenue pour nous par plusieurs soit pour plusieurs une occasion de rendre grâces à notre sujet.
\TextTitle{La sincérité de Paul dans son ministère}
\VS{12}Car ce qui fait notre gloire c’est le témoignage de notre conscience, que nous nous sommes conduits dans le monde, et surtout à votre égard, avec simplicité et sincérité de Dieu, non point avec une sagesse charnelle, mais avec la grâce de Dieu.
\VS{13}Nous ne vous écrivons pas autre chose que ce que vous lisez, et vous-mêmes le reconnaissez. Et j'espère que vous les reconnaîtrez aussi jusqu'à la fin,
\VS{14}de même que vous avez reconnu en partie que nous sommes votre gloire, comme vous serez aussi la nôtre au jour du Seigneur Jésus.
\TextTitle{Sa manière d'agir}
\VS{15}C’est dans une telle confiance que je voulais premièrement aller vers vous, afin que vous ayez une seconde grâce ;
\VS{16}et passer de chez vous en Macédoine, puis de Macédoine revenir vers vous, et être accompagné par vous en Judée.
\VS{17}Or quand je me proposais cela, ai-je usé de légèreté ? ou les choses  que je propose, sont-elles proposées selon la chair, de sorte qu'il y ait eu en moi le oui et le non ?
\VS{18}Mais Dieu est fidèle, la parole que nous vous avons adressée n’a pas été oui et non.
\VS{19}Car le Fils de Dieu, Jésus-Christ, qui a été prêché par nous au milieu de vous, par moi, par Silvain, et par Timothée, n'a pas été oui et non, mais il a été oui en lui.
\VS{20}Car autant qu’il y a de promesses de Dieu, elles sont oui en lui, et amen en lui, afin que Dieu soit glorifié par nous.
\VS{21}Or celui qui nous affermit avec vous en Christ, et qui nous a oints, c'est Dieu,
\VS{22}lequel nous a aussi marqués d’un sceau, et a mis dans nos cœurs les arrhes\FTNT{Du grec «~arrhabon~»~: arrhes~; monnaie donnée en gage d’un futur paiement, en attendant que le solde soit payé.} de l'Esprit.
\VS{23}Or j'appelle Dieu à témoin sur mon âme, que c’est pour vous épargner que je ne suis plus allé à Corinthe.
\VS{24}Non que nous dominions sur votre foi, mais nous contribuons à votre joie, puisque vous demeurez fermes dans la foi.
\Chap{2}
\TextTitle{Les fruits de la repentance}
\VerseOne{}Je résolus en moi-même de ne pas retourner chez vous avec tristesse.
\VS{2}Car si je vous attriste, qui peut me réjouir, sinon celui que j'aurai moi-même affligé ?
\VS{3}Je vous ai écrit ceci, pour ne pas éprouver à mon arrivée de la tristesse de la part de ceux de qui je devais recevoir de la joie, ayant en vous tous cette confiance que ma joie est la vôtre à tous.
\VS{4}Je vous ai écrit dans une grande affliction et angoisse de cœur, avec beaucoup de larmes, non pas afin que vous soyez attristés, mais afin que vous connaissiez la charité\FTNT{Littéralement «~agape~»~: amour fraternel.} toute particulière que j'ai pour vous.
\VS{5}Si quelqu'un a été la cause de cette tristesse, ce n'est pas moi seul qu'il a attristé, afin que je ne le surcharge point, mais en quelque sorte c'est vous tous.
\VS{6}C'est assez pour cet homme, de la correction qui lui a été faite par plusieurs,
\VS{7}en sorte que vous devez bien plutôt lui pardonner et le consoler, de peur qu’il ne soit accablé par une trop grande tristesse.
\VS{8}C'est pourquoi je vous prie de confirmer envers lui votre charité.
\VS{9}C’est aussi pour cela que je vous ai écrit, afin de vous éprouver, et de connaître si vous êtes obéissants en toutes choses.
\VS{10}Or à celui à qui vous pardonnez quelque chose, je pardonne aussi ; si j’ai pardonné à celui à qui j'ai pardonné, c’est à cause de vous, en présence de Christ,
\VS{11}afin que Satan n'ait pas le dessus sur nous, car nous n'ignorons pas ses machinations.
\VS{12}Au reste, lorsque je fus arrivé à Troas pour l'Evangile de Christ, quoique la porte m'y fût ouverte par le Seigneur, je n’eus point de repos en mon esprit, parce que je ne trouvai pas Tite, mon frère ;
\VS{13}mais ayant pris congé d'eux, je partis pour la Macédoine.
\VS{14}Grâces soient rendues à Dieu, qui nous fait toujours triompher en Christ, et qui manifeste par nous l'odeur de sa connaissance en tout lieu.
\VS{15}Nous sommes, en effet, pour Dieu le parfum de Christ, parmi ceux qui sont sauvés, et parmi ceux qui périssent :
\VS{16}Aux uns, une odeur mortelle, pour la mort ; aux autres, une odeur vivifiante, pour la vie. Mais qui est suffisant pour ces choses ?
\VS{17}Car nous ne falsifions pas la parole de Dieu, comme font plusieurs, mais nous parlons de Christ avec sincérité, comme de la part de Dieu, et devant Dieu.
\Chap{3}
\TextTitle{Les corinthiens : lettre de Christ écrite avec l'Esprit du Dieu vivant}
\VerseOne{}Commençons-nous de nouveau à nous recommander nous-mêmes ? Ou avons-nous besoin, comme quelques-uns, de lettres de recommandation auprès de vous, ou de lettres de recommandation de votre part ?
\VS{2}Vous êtes vous-mêmes notre lettre, écrite dans nos cœurs, connue et lue de tous les hommes.
\VS{3}Car il est manifeste que vous êtes la lettre de Christ, écrite par notre ministère, non avec de l'encre, mais avec l'Esprit du Dieu vivant, non sur des tables de pierre, mais sur les tables de chair, qui sont vos cœurs.
\VS{4}Or nous avons une telle confiance en Dieu par Christ.
\VS{5}Non que nous soyons capables de nous-mêmes de penser quelque chose, comme de nous-mêmes, mais notre capacité vient de Dieu.
\TextTitle{Paul ministre de la nouvelle alliance}
\VS{6}Lequel nous a rendus capables d'être ministres de la nouvelle alliance\FTNT{Dans la plupart des versions, le mot grec «~diatheke~» a été traduit par «~testament~» alors que ce mot signifie aussi «~alliance~». On le retrouve notamment dans les passages suivants~: Mt. 26:28~; Mc. 14:24~; Lu. 1:72~; 22:20~; Ac. 3:25~; 7:8~; Ro. 9:4~; 11:27~; 1 Co. 11:25~; Ga. 3:15,17~; 4:24~; Ep. 2:12~; Hé. 7:22~; 8:6,8-9~; 9:4,15-17,20~; 10:16,29~; 12:24~; 13:20~; Ap. 11:19. Le fait d’avoir regroupé les écrits de Genèse à Malachie sous l’appellatif «~Ancien Testament~» a induit beaucoup de chrétiens en erreur. L’Ancienne Alliance correspond uniquement à la loi cérémonielle de Moïse qui a été accomplie par Christ à la croix (Jn. 19:30). Ainsi, avant la mort du Seigneur, on ne peut pas parler de testament puisqu’il faut qu’il y ait au préalable la mort du testateur. Or il est évident que les animaux sacrifiés sous la loi ne nous ont rien légué (Hé. 9:1-16).}, non de la lettre, mais de l'Esprit ; car la lettre tue, mais l'Esprit vivifie.
\VS{7}Or si le ministère de la mort, écrit sur des lettres, et gravé avec des pierres, a été glorieux au point que les enfants d'Israël ne pouvaient regarder fixement le visage de Moïse, à cause de la gloire de son visage, bien que cette gloire devait disparaître,
\VS{8}comment le ministère de l'Esprit ne sera-t-il pas plus glorieux ?
\VS{9}Car si le ministère de la condamnation a été glorieux, le ministère de la justice le surpasse de beaucoup en gloire.
\VS{10}Et même ce premier ministère qui a été si glorieux, ne l'a pas été en comparaison du second qui le surpasse de beaucoup en gloire.
\VS{11}Car si ce qui devait disparaître a été glorieux, ce qui est permanent est beaucoup plus glorieux.
\VS{12}Ayant donc une telle espérance, nous usons d'une grande liberté,
\VS{13}et pas comme Moïse qui mettait un voile sur son visage, afin que les enfants d'Israël ne fixassent pas les yeux sur la fin de ce qui devait disparaître.
\VS{14}Mais ils sont devenus durs d’entendement. Car jusqu'à aujourd'hui, ce même voile qui n’est ôté que par Christ, demeure quand ils font la lecture de l'ancienne alliance.
\VS{15}Jusqu'à ce jour, quand on lit Moïse, un voile est jeté sur leur cœur.
\VS{16}Mais lorsque les cœurs se convertissent au Seigneur, le voile est ôté.
\VS{17}Or le Seigneur c'est l'Esprit ; et là où est l'Esprit du Seigneur, là est la liberté.
\VS{18}Nous tous qui contemplons comme dans un miroir la gloire du Seigneur à visage découvert, nous sommes transformés en la même image, de gloire en gloire, comme par l'Esprit du Seigneur.
\Chap{4}
\TextTitle{La vérité pratique de son ministère}
\VerseOne{}C'est pourquoi, ayant ce ministère selon la miséricorde que nous avons reçue, nous ne nous relâchons point.
\VS{2}Mais nous avons entièrement rejeté les choses honteuses que l'on cache, ne marchant point avec ruse, et ne falsifiant point la parole de Dieu, mais nous rendant approuvés à toute conscience des hommes devant Dieu, par la manifestation de la vérité.
\VS{3}Que si notre Evangile est encore voilé, il ne l'est que pour ceux qui périssent ;
\VS{4}pour les incrédules dont le dieu de ce siècle a aveuglé l’esprit, afin qu’ils ne soient pas éclairés par la lumière de l'Evangile de la gloire de Christ, lequel est l'image de Dieu.
\VS{5}Car nous ne nous prêchons pas nous-mêmes mais nous prêchons Jésus-Christ le Seigneur, et nous déclarons que nous sommes vos serviteurs pour l'amour de Jésus.
\VS{6}Car Dieu qui a dit que la lumière resplendisse des ténèbres\FTNT{Ge. 1:3.}, est celui qui a resplendi dans nos coeurs, pour manifester la connaissance de la gloire de Dieu en la présence de Jésus-Christ.
\VS{7}Mais nous avons ce trésor dans des vases de terre, afin que l'excellence de cette puissance soit de Dieu, et non pas de nous.
\TextTitle{Les souffrances de Paul}
\VS{8}Etant affligés à tous égards, mais non réduits entièrement à l’extrémité ; étant en perplexité, mais non sans secours ;
\VS{9}étant persécutés, mais non abandonnés ; étant abattus, mais non perdus ;
\VS{10}portant toujours partout dans notre corps la mort du Seigneur Jésus, afin que la vie de Jésus soit aussi manifestée dans notre corps.
\VS{11}Car nous qui vivons, nous sommes sans cesse livrés à la mort pour l'amour de Jésus, afin que la vie de Jésus soit aussi manifestée dans notre chair mortelle.
\VS{12}De sorte que la mort agit en nous, et la vie agit en vous.
\VS{13}Or ayant un même esprit de foi, selon qu'il est écrit : J'ai cru, c'est pourquoi j'ai parlé\FTNT{Ps. 116:10.} ! Nous croyons aussi , et c'est aussi pourquoi nous parlons,
\VS{14}sachant que celui qui a ressuscité le Seigneur Jésus nous ressuscitera aussi par Jésus, et nous fera comparaître en sa présence avec vous.
\VS{15}Car toutes ces choses sont pour vous, afin que cette grâce surabonde, à la gloire de Dieu, les actions de grâces d’un très grand nombre.
\VS{16}C'est pourquoi nous ne nous relâchons pas. Mais quoique notre homme extérieur se détruit, toutefois l'intérieur est renouvelé de jour en jour.
\VS{17}Car nos légères afflictions du moment, produisent pour nous, au-delà de toute mesure, un poids éternel d'une gloire souverainement excellente,
\VS{18}quand nous ne regardons point aux choses visibles, mais aux invisibles ; car les choses visibles ne sont que pour un temps, mais les invisibles sont éternelles.
\Chap{5}
\TextTitle{Ses ambitions}
\VerseOne{} Car nous savons que si notre habitation terrestre, qui n'est qu'une tente, est détruite, nous avons un édifice de Dieu qui n’a pas été fait de main d’homme, une maison éternelle dans les cieux.
\VS{2}Car c'est aussi pour cela que nous gémissons, désirant avec ardeur d'être revêtus de notre domicile qui est du ciel,
\VS{3}si toutefois nous sommes trouvés vêtus, et non pas nus.
\VS{4}Car nous qui sommes dans cette tente, nous gémissons, accablés, parce que nous désirons, non pas d'être dépouillés, mais d'être revêtus, afin que ce qui est mortel soit englouti par la vie.
\VS{5}Et celui qui nous a formés pour cela c'est Dieu qui nous a donné les arrhes de l'Esprit.
\VS{6}Nous avons donc toujours confiance ; et nous savons que logeant dans ce corps, nous demeurons loin du Seigneur,
\VS{7}car nous marchons par la foi, et non par la vue.
\VS{8}Nous avons, dis-je, de la confiance, et nous aimons mieux être absents de ce corps, et être avec le Seigneur.
\VS{9}C'est pourquoi aussi nous nous efforçons de lui être agréables, et présents, et absents.
\VS{10}Car il nous faut tous comparaître devant le tribunal\FTNT{Le Tribunal de Christ n’a pas vocation à déterminer le salut des enfants de Dieu. Les chrétiens y seront jugés en fonction des œuvres produites sur la terre. En effet, chacun devra rendre compte de ce qu’il aura fait et de la gestion des dons et ministères reçus. Voir Ro. 14:10~; 1 Co. 4:4~; 2 Co. 3:10-14~; 2 Tim. 4:8.} de Christ, afin que chacun reçoive en son corps selon ce qu'il aura fait, soit bien, soit mal.
\TextTitle{Ses motifs d'action}
\VS{11}Connaissant donc combien le Seigneur doit être craint, nous persuadons les hommes; et nous sommes connus de Dieu, et j’espère que dans vos consciences vous nous connaissez aussi.
\VS{12}Car nous ne nous recommandons pas de nouveau à vous, mais nous vous donnons l'occasion de vous glorifier à notre sujet, afin que vous ayez de quoi répondre à ceux qui se glorifient de l'apparence, et non pas de ce qui est dans le cœur.
\VS{13}Car soit que nous soyons hors de sens, c’est pour Dieu ; soit que nous soyons de bon sens c’est pour vous.
\VS{14}Car la charité de Christ nous lie, parce que nous jugeons que, si un est mort pour tous, tous aussi sont morts ;
\VS{15}et qu'il est mort pour tous, afin que ceux qui vivent, ne vivent plus pour eux-mêmes, mais pour celui qui est mort et ressuscité pour eux.
\VS{16}C'est pourquoi dès maintenant nous ne connaissons personne selon la chair ; et si nous avons connu Christ selon la chair, maintenant nous ne le connaissons plus ainsi.
\VS{17}Si donc quelqu'un est en Christ, il est une nouvelle créature ; les choses anciennes sont passées ; voici, toutes choses sont faites nouvelles.
\VS{18}Et tout cela vient de Dieu, qui nous a réconciliés avec lui par Jésus-Christ, et qui nous a donné le ministère de la réconciliation.
\VS{19}Car Dieu était en Christ, réconciliant le monde avec lui-même, en ne leur imputant point leurs péchés, et il a mis en nous la parole de la réconciliation.
\VS{20}Nous sommes donc ambassadeurs pour Christ, c'est comme si Dieu vous exhortait par notre ministère ; nous vous supplions donc pour l’amour de Christ : Réconciliez-vous avec Dieu !
\VS{21}Car il a fait celui qui n'a point connu de péché, être péché pour nous, afin que nous soyons en lui justice de Dieu.
\Chap{6}
\TextTitle{Son humilité}
\VerseOne{}Puisque nous travaillons avec le Seigneur, nous vous prions de ne pas recevoir la grâce de Dieu en vain.
\VS{2}Car il dit : Je t'ai exaucé au temps favorable et t'ai secouru au jour du salut\FTNT{Es. 49:8.} ; voici maintenant le temps favorable, voici maintenant le jour du salut.
\VS{3}Ne donnant aucun scandale en quoi que ce soit, afin que notre ministère ne soit point blâmé.
\VS{4}Mais nous rendant recommandables, en toutes choses, comme ministres de Dieu, en grande patience, en afflictions, en nécessités, en détresses,
\VS{5}sous les coups, dans les prisons, dans les troubles, dans les travaux, dans les veilles, dans les jeûnes,
\VS{6}par la pureté, par la connaissance, par la persévérance, par la douceur, par le Saint-Esprit, par une charité sincère,
\VS{7}par la parole de vérité, par la puissance de Dieu, par les armes de justice que l'on porte à la main droite et à la main gauche ;
\VS{8}au milieu de la gloire et de l'ignominie, au milieu de la mauvaise et de la bonne réputation ; étant regardés comme des séducteurs, quoique véridiques,
\VS{9}comme inconnus, quoique bien connus, comme mourants, et voici nous vivons, comme châtiés, quoique non mis à mort,
\VS{10}comme attristés, et nous sommes toujours joyeux, comme pauvres, et nous en enrichissons plusieurs, comme n'ayant rien, et nous possédons toutes choses.
\TextTitle{Appel à la séparation et à la purification}
\VS{11}Ô Corinthiens ! Notre bouche s’est ouverte pour vous, notre cœur s'est élargi.
\VS{12}Vous n’y êtes point à l'étroit, mais c’est votre cœur qui s’est rétréci pour nous.
\VS{13}Rendez-nous la pareille (je vous parle comme à mes enfants) élargissez aussi votre cœur !
\VS{14}Ne portez pas un même joug avec les infidèles ; car quelle communion y a-t-il entre justice et l'iniquité ? Ou qu’y a-t-il de commun entre la lumière et les ténèbres ?
\VS{15}Et quel accord y a-t-il entre Christ et Bélial\FTNT{Bélial~: De l’hébreu «~beliya’al~»~: méchants, pervers, pervertis, vil, destruction, dangereusement. L’un des noms de Satan qui signifie «~indignité, méchanceté, impiété~».} ? Ou quelle part a le fidèle avec l'infidèle ?
\VS{16}Et quel rapport y a-t-il entre le temple de Dieu et les idoles ? Car vous êtes le temple du Dieu vivant, selon ce que Dieu a dit : J'habiterai au milieu d'eux et j'y marcherai ; je serai leur Dieu, et ils seront mon peuple\FTNT{Lé. 26:12~; Ez. 37:26.}.
\VS{17}C'est pourquoi sortez du milieu d'eux, et séparez-vous, dit le Seigneur ; ne touchez pas à ce qui est impur, et je vous accueillerai\FTNT{Es. 52:11~; Ap. 18:4.}.
\VS{18}Je serai pour vous un Père, et vous serez pour moi des fils et des filles, dit le Seigneur Tout-Puissant\FTNT{Jn. 1:12~; Ap. 21:7.}.
\Chap{7}
\VerseOne{}Or donc mes bien-aimés, puisque nous avons de telles promesses, nettoyons-nous de toute souillure de la chair et de l'esprit, perfectionnant la sanctification dans la crainte de Dieu.
\TextTitle{Paul ouvre son coeur aux Corinthiens}
\VS{2}Recevez-nous, nous n'avons fait tort à personne, nous n'avons corrompu personne, nous n'avons pillé personne.
\VS{3}Je ne dis pas ceci pour vous condamner, car je vous ai déjà dit que vous êtes dans nos cœurs à la vie et à la mort.
\VS{4}J'ai une grande liberté envers vous, j'ai grand sujet de me glorifier de vous ; je suis rempli de consolation, je suis comblé de joie au milieu de toutes nos afflictions.
\VS{5}Car depuis notre arrivée en Macédoine, notre chair n’eut aucun repos, mais nous avons été affligés de toute manière, ayant eu des combats au dehors, et des craintes au dedans.
\VS{6}Mais Dieu qui console les abattus nous a consolés par l’arrivée de Tite,
\VS{7}et non seulement par son arrivée, mais aussi par la consolation qu'il a reçue de vous ; car il nous a raconté votre grand désir, vos larmes, votre affection ardente pour moi, en sorte que je m'en suis extrêmement réjoui.
\VS{8}Quoique je vous aie attristés par ma lettre, je ne m'en repens pas. Et si je m'en suis repenti, car je vois que cette lettre vous a affligés, bien que momentanément,
\VS{9}je me réjouis à présent, non de ce que vous avez été affligés, mais de ce que votre tristesse vous a portés à la repentance ; car vous avez été attristés selon Dieu, de sorte que vous n'avez reçu aucun dommage de notre part.
\VS{10}En effet, la tristesse selon Dieu produit une repentance à salut dont on ne se repent jamais, mais que la tristesse du monde produit la mort.
\VS{11}En effet, cette tristesse qui est selon Dieu, quel empressement n’a-t-elle pas produit en vous ! quelle justification, quelle indignation, quelle crainte, quel grand désir, quel zèle, quelle vengeance! Vous  vous êtes montrés de toutes manières purs dans cette affaire.
\VS{12}Quoi que je vous aie écrit, ce n'était ni à cause de celui qui a commis la faute, ni à cause de celui envers qui elle a été commise, mais pour faire voir parmi vous l'empressement que j'ai de vous devant Dieu.
\VS{13}C'est pourquoi nous avons été consolés de ce que vous avez fait pour notre consolation. Mais nous nous sommes encore plus réjouis de la joie qu'a eu Tite, en ce que son esprit a été tranquillisé par vous tous.
\VS{14}Et si en quelque chose je me suis glorifié de vous devant lui, je n’en ai point eu de confusion ; mais, comme nous avons toujours parlé selon la vérité, ce dont nous nous sommes glorifiés auprès de Tite s’est trouvé être aussi la vérité.
\VS{15}C'est pourquoi quand il se souvient de l'obéissance de vous tous, et comment vous l'avez reçu avec crainte et tremblement, son affection pour vous en est beaucoup plus grande.
\VS{16}Je me réjouis d'avoir confiance en vous en toutes choses.
\Chap{8}
\TextTitle{Exemple des Macédoniens concernant la collecte en faveur des pauvres de Jérusalem}
\VerseOne{}Au reste, mes frères, nous voulons vous faire connaître la grâce que Dieu a faite aux églises de la Macédoine.
\VS{2}Au travers de leur grande épreuve d'affliction, leur joie a été augmentée, et leur profonde pauvreté s'est répandue en richesses par leur prompte libéralité.
\VS{3}Car je suis témoin qu'ils ont donné volontairement selon leurs moyens, et même au-delà de leurs moyens,
\VS{4}Nous pressant avec de grandes prières de recevoir la grâce et de prendre part à cette contribution en faveur des Saints.
\VS{5}Et ils n'ont pas fait seulement comme nous l'espérions, mais ils se sont donnés premièrement eux-mêmes au Seigneur, et puis à nous, par la volonté de Dieu.
\VS{6}Nous avons exhorté Tite, comme il avait auparavant commencé, d'achever aussi cette grâce envers vous.
\TextTitle{Exemple du Messie}
\VS{7}C'est pourquoi, comme vous excellez en toutes choses, en foi, en parole, en connaissance, en toute diligence, et dans la charité que vous avez pour nous, faites en sorte d’exceller aussi dans cette œuvre de charité.
\VS{8}Je ne dis pas cela pour vous donner un ordre, mais pour éprouver par l’empressement des autres la sincérité de votre charité.
\VS{9}Car vous connaissez la grâce de notre Seigneur Jésus-Christ qui, étant riche, s'est fait pauvre pour vous, afin que par sa pauvreté vous soyez enrichis.
\VS{10}C’est un avis que je donne là-dessus, parce qu'il vous est convenable, à vous qui non seulement avez commencé à agir, mais en ayant même eu la volonté dès l'année passée.
\VS{11}Achevez donc maintenant d'agir, afin que comme vous avez été prompts à en avoir la volonté ; vous l'accomplissiez aussi selon vos moyens.
\VS{12}Car si la promptitude de la volonté existe,  on est agréable selon ce qu'on a, et non point selon ce qu'on n'a pas.
\VS{13}Je ne veux pas vous exposer à la détresse pour soulager les autres, mais suivre une règle d’égalité. Dans la circonstance présente, votre superflu pourvoira à leurs besoins,
\VS{14}afin que leur superflu pourvoie pareillement aux vôtres, en sorte qu’il y ait égalité,
\VS{15}selon ce qui est écrit : Celui qui avait beaucoup n'a rien eu de superflu, et celui qui avait peu n'en a pas eu moins.\FTNT{Ex. 16:18.}.
\TextTitle{Exemple des églises}
\VS{16}Grâces soient rendues à Dieu qui a mis dans le cœur de Tite le même empressement pour vous ;
\VS{17}lequel a bien reçu mon exhortation, et c'est avec un nouveau zèle et de son plein gré qu'il part pour aller chez vous.
\VS{18}Et nous avons aussi envoyé avec lui le frère dont la louange dans l'Evangile est répandue par toutes les églises ;
\VS{19}de plus, il a été choisi par élection des églises pour être notre compagnon de voyage et pour cette grâce\FTNT{ également traduit par «~les aumônes~»} qui est administrée par nous à la gloire du Seigneur même, et afin de répondre à l’ardeur de votre zèle. %et pour servir à la promptitude de votre zèle.
\VS{20}Evitant ainsi que personne ne nous blâme dans cette abondante collecte, qui est administrée par nous;
\VS{21}ayant soin de faire ce qui est bon, non seulement devant le Seigneur, mais aussi devant les hommes.
\VS{22}Nous avons envoyé aussi avec eux notre autre frère, dont nous avons souvent éprouvé le zèle à plusieurs occasions, qui est maintenant encore plus zélé, à cause de la grande confiance qu'il a en vous.
\VS{23}Ainsi donc, quant à Tite, il est mon associé et mon compagnon d’œuvre auprès de vous ; et quant à nos frères, ils sont les envoyés des églises, la gloire de Christ.
\VS{24}Montrez donc envers eux et devant les églises une preuve de votre charité, et du sujet que nous avons de nous glorifier de vous.
\Chap{9}
\TextTitle{Encouragement par rapport aux dons}
\VerseOne{}Il est superflu que je vous écrive touchant la collecte destinée aux saints.
\VS{2}Car je vois la promptitude de votre zèle, dont je me glorifie de vous devant ceux de Macédoine, leur disant que l’Achaïe est prête dès l'année dernière ; et votre zèle en a excité plusieurs.
\VS{3}J’ai envoyé ces frères, afin que ce en quoi je me suis glorifié de vous, ne soit pas vain en cette occasion, et que vous soyez prêts, comme j'ai dit.
\VS{4}De peur que ceux de Macédoine venant avec moi, et ne vous trouvant pas prêts, nous, pour ne pas dire vous, n'ayons de honte de l'assurance dont nous nous sommes glorifiés.
\VS{5}C'est pourquoi j'ai estimé nécessaire de prier les frères à se rendre premièrement vers vous, et d'achever de préparer votre bienfait déjà promis, afin qu'il soit prêt, comme un bienfait, et non comme de l'avarice.
\VS{6}Au reste, je vous avertis que celui qui sème peu moissonnera peu, et celui qui sème abondamment moissonnera abondamment.
\VS{7}Mais que chacun contribue selon qu'il se l'est proposé en son cœur, non à regret, ou par contrainte; car Dieu aime celui qui donne avec joie.
\VS{8}Et Dieu est Tout-Puissant pour vous combler de toutes sortes de grâces, afin qu'ayant toujours tout ce qui suffit en toute chose, vous soyez abondants en toute bonne œuvre,
\VS{9}selon ce qui est écrit : Il a fait des largesses, il a donné aux pauvres ; sa justice demeure éternellement\FTNT{Ps. 112:9.}.
\VS{10}Que celui qui fournit de la semence au semeur veuille aussi vous donner du pain à manger, et multiplier votre semence, et augmenter les revenus de votre justice ;
\VS{11}afin que vous soyez pleinement enrichis pour exercer une parfaite libéralité, laquelle fait que nous en rendons grâces à Dieu.
\VS{12}Car le service de cette assistance est non seulement suffisant pour subvenir aux nécessités des Saints, mais il abonde aussi de telle sorte, que plusieurs ont de quoi en rendre grâces à Dieu.
\VS{13}Glorifiant Dieu pour l’épreuve qu’ils font de cette assistance, en ce que vous vous soumettez à l’Évangile de Christ ; et de votre prompte et libérale communication envers eux, et envers tous.
\VS{14}ils prient Dieu pour vous, et ils vous aiment\FTNT{Aimer~: Du grec «~epipotheo~»~: désirer, chérir.} très affectueusement à cause de la grâce excellente que Dieu vous a accordée.
\VS{15}Grâces soient rendues à Dieu pour son don inexprimable.
\Chap{10}
\TextTitle{Paul défend son autorité apostolique}
\VerseOne{}Au reste, je vous prie, moi Paul, par la douceur et la bonté de Christ - moi qui parais méprisable lorsque je suis en votre présence, et plein de hardiesse quand je suis éloigné,
\VS{2}je vous prie, dis-je, que lorsque je serai présent, il ne faille point que j'use de hardiesse, laquelle je me propose d’user contre quelques-uns qui nous regardent comme marchant selon la chair.
\VS{3}Mais en marchant dans la chair, nous ne combattons pas selon la chair.
\VS{4}Car les armes de notre guerre ne sont pas charnelles, mais elles sont puissantes par la vertu de Dieu, pour la destruction des forteresses.
\VS{5}Détruisant les raisonnements et toute hauteur qui s'élèvent contre la connaissance de Dieu, et amenant toute pensée captive à l'obéissance de Christ.
\VS{6}Et étant prêts à tirer vengeance de toute désobéissance, lorsque votre obéissance sera complète.
\VS{7}Considérez-vous les choses selon l'apparence ? Si quelqu'un se persuade qu’il est de Christ, qu'il se dise bien en lui-même que, comme il est de Christ, nous aussi nous sommes de Christ.
\VS{8}Car si même je veux me glorifier davantage de l’autorité que le Seigneur nous a donnée pour votre édification et non votre destruction, je ne saurais en avoir honte,
\VS{9}afin que je ne paraisse pas vouloir vous effrayer par mes lettres.
\VS{10}Car mes lettres, disent-ils, sont graves et fortes, mais la présence de corps est faible, et la parole est méprisable.
\VS{11}Que celui qui est tel, considère que tels nous sommes en paroles dans nos lettres, étant absents, tels aussi nous sommes dans nos actes, étant présents.
\VS{12}Car nous n'osons pas nous joindre ni nous comparer à quelques-uns de ceux qui se recommandent eux-mêmes. Mais en se mesurant à leur propre mesure et en se comparant à eux-mêmes, ils manquent d’intelligence.
\VS{13}Mais pour nous, nous ne voulons pas nous glorifier outre mesure, mais seulement dans la limite du champ d’action que Dieu nous assigné en nous amenant jusqu’à vous.
\VS{14}Car nous ne nous étendons pas nous même au delà des limites prescrites, comme si nous n'étions pas parvenus jusqu'à vous ; vu que nous sommes parvenus même jusqu’à vous par la prédication de l'Evangile de Christ.
\VS{15}Nous ne nous glorifions pas des travaux d’autrui qui sont hors de nos limites. Mais nous avons l’espérance, si votre foi augmente, de devenir encore plus grands parmi vous, selon les limites qui nous sont assignées,
\VS{16}jusqu'à évangéliser dans les lieux qui sont au-delà de chez vous, sans nous glorifier de ce qui a déjà été fait dans le domaine des autres.
\VS{17}Que celui qui se glorifie se glorifie dans le Seigneur.
\VS{18}Car ce n'est pas celui qui se recommande lui-même qui est approuvé, c'est celui que le Seigneur recommande\FTNT{Le Seigneur recommande ses serviteurs, il témoigne d’eux auprès des autres (Ac. 10:1-48). Un véritable serviteur de Dieu laisse au Seigneur le soin de témoigner de lui auprès des autres alors que les faux ouvriers se recommandent eux-mêmes (2 Co. 3:1).}.
\Chap{11}
\VerseOne{}Oh ! Si vous pouviez supportez de ma part un peu de folie ! Mais vous me supportez !
\VS{2}Car je suis jaloux de vous d'une jalousie de Dieu, parce que je vous ai fiancés à un seul Epoux, pour vous présenter à Christ comme une vierge pure.
\TextTitle{Les faux docteurs}
\VS{3}Mais je crains que comme le serpent séduisit Eve\FTNT{Ge. 3:1-6.} par sa ruse, vos pensées aussi ne se corrompent en se détournant de la simplicité à l’égard de Christ.
\VS{4}Car si quelqu'un vient vous prêcher un autre Jésus que nous n'avons pas prêché, ou si vous recevez un autre esprit que celui que vous avez reçu, ou un autre évangile que celui que vous avez embrassé, vous le supportez fort bien.
\VS{5}Or j'estime que je n'ai été en rien moindre que les plus excellents apôtres.
\VS{6}Si je suis un ignorant sous le rapport du langage, je ne le suis pourtant point sous celui de la connaissance, et nous l’avons montré parmi vous à tous égards et en toutes choses.
\VS{7}Ai-je commis une faute, en m’abaissant moi-même afin que vous soyez élevés, quand je vous ai annoncé gratuitement l’Evangile de Dieu ?
\VS{8}J'ai dépouillé les autres églises prenant de quoi m'entretenir pour vous . Et lorsque j’étais chez vous et que je me suis trouvé dans le besoin, je n’ai été à la charge de personne,
\VS{9}car les frères venus de la Macédoine ont pourvu à ce qui me manquait. Et en toutes choses, je me suis gardé d’être à votre charge, et je m'en garderai encore.
\VS{10}Par la vérité de Christ qui est en moi, j’atteste que ce sujet de gloire ne me sera point ravi dans les contrées de l'Achaïe.
\VS{11}Pourquoi ? Est-ce parce que je ne vous aime point ? Dieu le sait !
\VS{12}Mais ce que je fais, je le ferai encore, pour ôter ce prétexte à ceux qui cherchent un prétexte, afin qu’ils soient trouvés tels que nous dans les choses dont ils se glorifient.
\VS{13}Car ces hommes-là sont de faux apôtres, des ouvriers trompeurs qui se déguisent en apôtres de Christ.
\VS{14}Et cela n'est pas étonnant puisque Satan lui-même se déguise en ange de lumière\FTNT{Satan est maître en matière de déguisement et d’imitation.}.
\VS{15}Ce n'est donc pas un grand sujet d'étonnement si ses ministres aussi se déguisent en ministres de justice ; mais leur fin sera conforme à leurs œuvres.
\TextTitle{Sujets de gloire de Paul\FTNTT{2 Co. 11:16-12:18}}
\VS{16}Je le dis encore, afin que personne ne me regarde comme un insensé ; sinon, supportez-moi comme un insensé, afin que je me glorifie aussi un peu.
\VS{17}Ce que je dis, avec l’assurance d’avoir sujet de me glorifier, je ne le dis pas selon le Seigneur, mais comme par folie.
\VS{18}Puisqu’il en est plusieurs qui se glorifient selon la chair, je me glorifierai aussi.
\VS{19}Car vous supportez bien volontiers les insensés, vous qui êtes sages.
\VS{20}Si quelqu'un vous asservit, si quelqu'un vous dévore, si quelqu'un prend votre bien, si quelqu'un est arrogant, si quelqu'un vous frappe au visage, vous le supportez.
\VS{21}Je le dis avec honte, nous avons montré de la faiblesse. Mais si en quelque chose quelqu'un ose se glorifier, je parle en insensé, j'ai la même hardiesse !
\VS{22}Sont-ils Hébreux ? Moi aussi. Sont-ils Israélites ? Moi aussi. Sont-ils de la postérité d'Abraham ? Moi aussi.
\VS{23}Sont-ils ministres de Christ ? - je parle comme un insensé - je le suis plus qu'eux ; par les travaux, bien plus ; par les blessures, bien plus ; par les emprisonnements, bien plus. Plusieurs fois en danger de mort,
\VS{24}cinq fois j’ai reçu des Juifs quarante coups moins un,
\VS{25}j'ai été battu de verges trois fois, j'ai été lapidé une fois, j'ai fait naufrage trois fois, j'ai passé un jour et une nuit dans l’abîme.
\VS{26}Fréquemment en voyage, j’ai été en péril sur les fleuves, en péril de la part des brigands, en péril de la part de ceux de ma nation, en péril de la part des gentils, en péril dans les villes, en péril dans les déserts, en péril sur la mer, en péril parmi de faux frères.
\VS{27}J’ai été dans le travail et dans la peine, exposé à de nombreuses veilles, à la faim et à la soif, à des jeûnes multipliés, au froid et à la nudité.
\VS{28}Outre les choses de dehors, ce qui me tient assiégé tous les jours, c'est le soucis que j'ai de toutes les églises.
\VS{29}Qui est affaibli, que je ne sois faible ? Qui est scandalisé, que je n'en sois aussi brûlé ?
\VS{30}S'il faut se glorifier, je me glorifierai des choses qui sont de mon infirmité.
\VS{31}Le Dieu et Père de notre Seigneur Jésus-Christ, lui qui est béni éternellement, sait que je ne mens point.
\VS{32}A Damas, le gouverneur du roi Arétas avait fait garder la ville des Damascéniens pour me prendre,
\VS{33}mais on me descendit par une fenêtre, dans une corbeille, le long de la muraille, et ainsi j'échappai de ses mains.
\Chap{12}
\VerseOne{}Certes, il ne me convient pas de me glorifier, car j’en viendrai jusqu’aux visions et aux révélations du Seigneur.
\VS{2}Je connais un homme en Christ, qui fut ravi jusqu’au troisième ciel, il y a quatorze ans passés, (si ce fut dans son corps, je ne sais pas ; si ce fut hors du corps, je ne sais pas ; Dieu le sait).
\VS{3}Et je sais que cet homme (si ce fut dans son corps, ou si ce fut hors du corps, je ne sais pas ; Dieu le sait),
\VS{4}fut ravi dans le paradis, et qu’il entendit des paroles inexprimables qu'il n'est pas permis à l'homme de révéler.
\TextTitle{Paul et son écharde}
\VS{5}Je me glorifierai d'un tel homme, mais je ne me glorifierai point de moi-même, sinon de mes infirmités.
\VS{6}Si je voulais me glorifier, je ne serais pas un insensé, car je dirais la vérité ; mais je m'en abstiens, afin que personne ne m'estime au-dessus de ce qu'il me voit être, ou de ce qu'il entend dire de moi.
\VS{7}Mais pour que je ne sois pas enflé d’orgueil, à cause de l'excellence de ces révélations, il m'a été mis une écharde\FTNT{La nature exacte de l'écharde de Paul ne nous est pas détaillée. Elle lui avait été infligée par un «~ange de Satan~», par la volonté de Dieu. Nous constatons une chose qui est commune à tous les enfants de Dieu~: Paul avait un adversaire constamment aux aguets pour essayer de le décourager, le détruire ou l’intimider en s'opposant par tous les moyens à la mission que le Seigneur lui avait confiée. Cette écharde était aussi un moyen utilisé par Dieu pour garder Paul dans l’humilité.} dans la chair, un ange de Satan pour me souffleter et m’empêcher de m’enorgueillir.
\VS{8}Trois fois j'ai prié le Seigneur de faire que cet ange de Satan se retire de moi.
\VS{9}Mais le Seigneur m'a dit : Ma grâce te suffit, car ma puissance s’accomplit dans la faiblesse. Je me glorifierai donc bien volontiers de mes faiblesses, afin que la puissance de Christ habite en moi.
\VS{10}C’est pourquoi je me plais dans les faiblesses, dans les outrages, dans les calamités, dans les persécutions, et dans les angoisses pour Christ ; car quand je suis faible, c'est alors que je suis fort.
\TextTitle{Avertissements}
\VS{11}J'ai été insensé en me glorifiant, mais vous m'y avez contraint ; c’est par vous que je devais être recommandé, car je n'ai été inférieur en rien aux apôtres par excellence, quoique je ne sois rien.
\VS{12}Certainement les preuves de mon apostolat ont éclaté au milieu de vous par une patience à toute épreuve, par des signes, des prodiges et des miracles.
\VS{13}Car en quoi avez-vous été inférieurs aux autres églises, sinon en ce que je n’ai point été à votre charge ? Pardonnez-moi ce tort.
\VS{14}Voici pour la troisième fois que je suis prêt à aller vers vous, et je ne serai point à votre charge ; car ce ne sont pas vos biens que je cherche, c’est vous-mêmes. Ce n’est pas, en effet, aux enfants d’amasser pour les parents, mais aux parents pour les enfants\FTNT{Un bon père amasse dans le but de préparer l’avenir de ses enfants et non l’inverse.}.
\VS{15}Pour moi, je dépenserai très volontiers pour vous tout ce que j’ai, et je me donnerai encore moi-même pour vos âmes. En vous aimant davantage, serais-je moins aimé de vous ?
\VS{16}Soit ! Dira-t-on, que je ne vous ai point été à charge, c'est qu'étant un homme intelligent, je vous ai pris par ruse !
\VS{17}Ai-je donc tiré profit de vous par quelqu’un de ceux que je vous ai envoyés ?
\VS{18}J'ai engagé Tite à aller chez vous, et avec lui j’ai envoyé le frère. Tite a-t-il tiré profit de vous ? Et n'avons-nous pas lui et moi marché dans le même esprit ? N'avons-nous pas marché sur les mêmes traces ?
\VS{19}Pensez-vous encore que nous voulions nous justifier auprès de vous ? Nous parlons devant Dieu en Christ, et tout cela, mes très chers frères, pour votre édification.
\VS{20}Car je crains de ne pas vous trouver, à mon arrivée, tels que je voudrais, et d’être moi-même trouvé par vous tel que vous ne voudriez pas. Je crains de trouver des querelles, de la jalousie, des animosités, des rivalités, des médisances, des calomnies, de l’orgueil, des troubles.
\VS{21}Je crains qu’à mon arrivée, mon Dieu ne m’humilie de nouveau à votre sujet, et que je n’aie à pleurer sur plusieurs de ceux qui ont péché précédemment, et qui ne se sont pas repentis de l’impureté, de la débauche et des dérèglements dont ils se sont rendus coupables.
\Chap{13}
\TextTitle{S'examiner}
\VerseOne{}Je vais chez vous pour la troisième fois. Toute affaire se réglera sur la déclaration de deux ou de trois témoins\FTNT{De. 19:15.}.
\VS{2}Lorsque j’étais présent pour la deuxième fois, j’ai déjà dit, et aujourd’hui que je suis absent je dis encore d’avance à ceux qui ont péché précédemment et à tous les autres, que si je retourne chez vous, je n'épargnerai personne,
\VS{3}puisque vous cherchez la preuve que Christ parle par moi, lui qui n'est point faible envers vous, mais qui est puissant parmi vous.
\VS{4}Car il a été crucifié à cause de sa faiblesse, mais il vit par la puissance de Dieu ; et nous de même, nous sommes aussi faibles comme lui, mais nous vivrons avec lui par la puissance que Dieu a déployée envers vous.
\VS{5}Examinez-vous vous-mêmes pour savoir si vous êtes dans la foi ; éprouvez-vous vous-mêmes. Ne reconnaissez-vous pas que Jésus-Christ est en vous ? A moins peut-être que vous ne soyez désapprouvés.
\VS{6}Mais j'espère que vous reconnaîtrez que nous, nous ne sommes pas désapprouvés.
\VS{7}Et je prie Dieu que vous ne fassiez rien de mal, non pour paraître nous-mêmes approuvés, mais afin que vous pratiquiez ce qui est bien et que nous, nous soyons comme désapprouvés.
\VS{8}Car nous n’avons pas de pouvoir contre la vérité, nous n’en avons que pour la vérité.
\VS{9}Nous nous réjouissons lorsque nous sommes faibles, tandis que vous êtes forts ; et ce que nous demandons à Dieu, c’est votre perfectionnement.
\VS{10}C'est pourquoi j'écris ces choses étant absent, afin que présent, je n’aie pas à user de rigueur, selon l’autorité que le Seigneur m'a donnée pour l'édification et non point pour la destruction.
\TextTitle{Conclusion}
\VS{11}Au reste, mes frères, réjouissez-vous, perfectionnez-vous, consolez-vous, ayez un même sentiment, vivez en paix ; et le Dieu de charité et de paix sera avec vous.
\VS{12}Saluez-vous les uns les autres par un saint baiser. Tous les saints vous saluent.
\VS{13}Que la grâce du Seigneur Jésus-Christ, la charité de Dieu, et la communion du Saint-Esprit soient avec vous tous. Amen !
\PPE{}
\end{multicols}

%\clearpage\ShortTitle{Romains}\BookTitle{Romains}\BFont
\noindent\hrulefill
{\footnotesize
\textit{
\bigskip
{\centering{}
\\Thème : L'Evangile de Dieu
\\Auteur : Paul
\\Date de rédaction : Env. 56\\}
}
%\bigskip
\textit{
\\Rome est une ville située dans la région du Latium, au centre de l’Italie, à la confluence de l’Aniene et du Tibre. Centre de l’Empire romain,  elle domina  l’Europe, l’Afrique du Nord et le Moyen-Orient du 1er siècle avant J.-C. au 5ème siècle après J.-C.
%\bigskip
\\La lettre était destinée à l’Eglise de Rome, fondée sans doute par des chrétiens convertis au  travers du ministère de Paul et d’autres apôtres itinérants. Cette Eglise comptait quelques juifs mais surtout des membres d’origine païenne. Cette épître fut rédigée au cours du 3ème voyage missionnaire de Paul,  pendant les trois mois que l’apôtre passa à Corinthe. En attendant de leur rendre visite physiquement, Paul avait le désir de communiquer aux chrétiens de Rome les grandes lignes du principe de la grâce, dont il avait eu la révélation. Il y aborda plusieurs doctrines majeures comme le salut par la foi et la grâce ainsi que des enseignements pratiques sur l’amour, le devoir du chrétien et la sainteté.\bigskip
}
}
\par\nobreak\noindent\hrulefill
\begin{multicols}{2}
\Chap{1}
\TextTitle{Introduction : L’Evangile de Christ, puissance de Dieu pour le salut de tous}
\VerseOne{}Paul, serviteur de Jésus-Christ, appelé à être apôtre, mis à part pour annoncer l'Evangile de Dieu,
\VS{2}qu’il avait auparavant promis par ses prophètes dans les saintes Ecritures,
\VS{3}et qui concerne son Fils, qui est né de la postérité de David, selon la chair,
\VS{4}et qui a été pleinement déclaré Fils de Dieu avec puissance, selon l'Esprit de sainteté, par sa résurrection d'entre les morts, c'est-à-dire, Jésus-Christ notre Seigneur,
\VS{5}par qui nous avons reçu la grâce et l’apostolat, pour amener en son Nom tous les gentils à l’obéissance de la foi,
\VS{6}parmi lesquels aussi vous êtes, vous qui êtes appelés par Jésus-Christ.
\VS{7}A vous tous qui êtes à Rome, bien-aimés de Dieu, appelés à être saints\FTNT{Le terme «~saint~» est tiré du grec «~hagios~» qui signifie «~consacré à Dieu~», «~saint~», «~sacré~», «~pieux~». Ce mot est souvent utilisé au pluriel dans le testament de Jésus (Ac. 9:13 ;  Ac. 26:10). Il n’y a aucun rapport entre la compréhension catholique romaine du terme «~saint~» et l’enseignement biblique. Dans l’enseignement catholique romain, une personne ne devient pas sainte tant qu’elle n’a pas été béatifiée ou canonisée par le pape ou l’éminent évêque. Dans la Bible, tous ceux qui reçoivent Jésus-Christ par la foi sont appelés saints car ils sont mis à part pour Dieu. Dans le culte de l’église catholique romaine, les saints sont vénérés, priés, et parfois adorés. Dans la Bible, les saints sont appelés à adorer et à prier Dieu seul par Jésus-Christ homme, le seul Médiateur entre Dieu et les hommes (1 Ti. 2:5). Dans le Tanakh, les mots «~sanctifié~», «~saint~» et leurs dérivés viennent du mot hébreu «~Quodesch~» dont le sens général est : «~Mis à part pour Dieu~». Dans la Bible, ces mots sont appliqués à des objets et à des personnes. Le terme «~sanctification~» appliqué à des objets sous-entend l'idée qu'ils sont réservés uniquement pour le service de Dieu ; ils sont sanctifiés, mis à part pour Dieu. Dans le Testament de Jésus, il est appliqué aux personnes et comprend plusieurs sens : - Les croyants, par leur position, sont éternellement mis à part pour Dieu par la rédemption (Hé. 10:10-14). Ils sont donc considérés comme «~saints~», «~sanctifiés~» dès leur conversion (Ph. 1:1 ; Hé. 3:1). - Les croyants sont amenés à la sanctification par l'action du Saint-Esprit au moyen des Ecritures (Jn. 15:3 ; Jn. 17:17 ; 2 Co. 3:18 ; Ep. 5:25-26 ; 1Th. 5:23-24). - Les croyants attendent la venue du Seigneur pour la réalisation complète de leur sanctification (1 Co. 15:29-50 ; Ph. 3:20-21 ; Ep. 5:27 ; 1 Jn. 3:2 ; Ap. 22:12).} ; que la grâce et la paix vous soient données de la part de Dieu notre Père et du Seigneur Jésus-Christ.
\VS{8}Je rends premièrement grâces à mon Dieu par Jésus-Christ, au sujet de vous tous, de ce que votre foi est renommée dans le monde entier.
\VS{9}Car Dieu, que je sers en mon esprit dans l'Evangile de son Fils, m'est témoin que je fais sans cesse mention de vous,
\VS{10}demandant continuellement dans mes prières que je puisse enfin trouver, par la volonté de Dieu, quelque moyen favorable pour aller vers vous.
\VS{11}Car je désire extrêmement vous voir, pour vous communiquer quelque don spirituel, afin que vous soyez affermis ;
\VS{12}et aussi, afin qu'étant parmi vous, nous nous consolions ensemble par la foi qui nous est commune.
\VS{13}Or mes frères, je ne veux pas que vous ignoriez que j’ai souvent formé le dessein d'aller vers vous, afin de recueillir quelque fruit parmi vous, comme parmi les autres nations ; mais j'en ai été empêché jusqu'à présent.
\VS{14}Je me dois aux Grecs et aux barbares, aux sages et aux ignorants.
\VS{15}Ainsi, autant qu’il dépend de moi, je suis prêt à vous annoncer aussi l'Evangile à vous qui êtes à Rome.
\VS{16}Car je n'ai point honte de l'Evangile de Christ, vu qu'il est la puissance de Dieu pour le salut de tous ceux qui croient : Du Juif premièrement, puis du Grec,
\VS{17}parce qu’en lui est révélée la justice de Dieu pleinement de foi en foi, selon qu'il est écrit : Le juste vivra par la foi\FTNT{Ha. 2:4.}.
\TextTitle{Jugement sur ceux qui retiennent la vérité captive}
\VS{18}Car la colère de Dieu se révèle pleinement du ciel contre toute impiété et injustice des hommes qui retiennent injustement la vérité captive.
\VS{19}Car ce qu’on peut connaître de Dieu est manifesté parmi eux ; car Dieu le leur a fait connaître.
\VS{20}En effet, les perfections invisibles de Dieu, à savoir sa puissance éternelle et sa divinité, se voient comme à l’œil nu, depuis la création du monde, quand on les considère dans ses ouvrages, de sorte qu'ils sont inexcusables.
\VS{21}Parce qu'ayant connu Dieu, ils ne l'ont point glorifié comme Dieu, et ils ne lui ont point rendu grâces, mais ils se sont égarés dans leurs pensées, et leur cœur sans intelligence a été plongé dans les ténèbres.
\VS{22}Se vantant d’être sages, ils sont devenus fous.
\VS{23}Et ils ont changé la gloire du Dieu incorruptible en images\FTNT{Ex. 20:4-5 ; Mt. 22:20 ; Mc. 12:16 ; Lu. 20:24. Ap.13:14-15 ; Ap. 14:9-11 ; Ap. 15:2 ; Ap. 16:2 ; 19:20 ;  Ap. 20:4.} représentant l'homme corruptible, des oiseaux, des quadrupèdes, et des reptiles.
\TextTitle{Les conséquences de l'endurcissement des hommes}
\VS{24}C'est pourquoi aussi Dieu les a livrés aux convoitises de leurs cœurs et à l’impureté\FTNT{Dieu les a livrés à l’esprit d’égarement (1 R. 22 ; 2 Th. 2:10-13).}, ainsi ils déshonorent eux-mêmes leurs propres corps ;
\VS{25}eux qui ont changé la vérité de Dieu en mensonge, et qui ont adoré et servi la créature, au lieu du Créateur, qui est béni éternellement. Amen !
\VS{26}C'est pourquoi Dieu les a livrés à des passions infâmes, car leurs femmes ont changé l'usage naturel en celui qui est contre la nature.
\VS{27}Et de même les hommes, abandonnant l'usage naturel de la femme, se sont enflammés dans leurs désirs les uns envers les autres, commettant homme avec homme des choses infâmes, et recevant en eux-mêmes le salaire que méritait leur égarement.
\VS{28}Car comme ils ne se sont pas souciés de connaître Dieu, aussi Dieu les a livrés à leur sens réprouvé, pour commettre des choses indignes.
\VS{29}Etant remplis de toute espèce d’injustice, d'impureté, de méchanceté, d'avarice, de malignité, pleins d'envie, de meurtre, de querelles, de fraude, de mauvaises mœurs,
\VS{30}rapporteurs, médisants, haïssant Dieu, outrageux, orgueilleux, vains, ingénieux au mal, rebelles à leurs parents,
\VS{31}dépourvus d’intelligence, de loyauté, d’affection naturelle, de miséricorde.
\VS{32}Et bien qu'ils connaissent le jugement de Dieu, déclarant dignes de mort ceux qui commettent de telles choses, non seulement ils les font, mais encore ils approuvent ceux qui les font.
\Chap{2}
\TextTitle{Condamnation du moralisme}
\VerseOne{}C'est pourquoi, ô homme, qui que tu sois, toi qui juges les autres, tu es donc inexcusable ; car en jugeant les autres, tu te condamnes toi-même, puisque toi qui juges, tu commets les mêmes choses.
\VS{2}Or nous savons que le jugement de Dieu est selon la vérité pour ceux qui commettent de telles choses.
\VS{3}Et penses-tu, ô homme, qui juges ceux qui commettent de telles choses, et qui les commets, que tu échapperas au jugement de Dieu ?
\VS{4}Ou méprises-tu les richesses de sa douceur, et de sa patience, et de sa bonté ; ne reconnaissant pas que la bonté de Dieu te convie à la repentance ?
\VS{5}Mais par ta dureté, et par ton cœur qui est sans repentance, tu t'amasses la colère pour le jour de la colère, et de la manifestation du juste jugement de Dieu,
\VS{6}qui rendra à chacun selon ses œuvres ;
\VS{7}à savoir la vie éternelle à ceux qui, en persévérant dans les bonnes œuvres, cherchent la gloire, l'honneur et l'immortalité.
\VS{8}Mais il y aura de l'indignation et de la colère contre ceux qui ont un esprit de dispute, et qui se rebellent contre la vérité, et obéissent à l'injustice.
\VS{9}Il y aura tribulation et angoisse sur toute âme d'homme qui fait le mal, pour le Juif premièrement, puis pour le Grec.
\VS{10}Mais gloire, honneur, et paix pour quiconque fait le bien ; pour le Juif premièrement, puis pour le Grec.
\VS{11}Car Dieu n'a point d'égard à l'apparence des personnes.
\VS{12}Tous ceux qui auront péché sans la loi, périront aussi sans la loi ; et tous ceux qui auront péché ayant la loi, seront jugés par la loi.
\VS{13}Car ce ne sont pas, en effet, ceux qui écoutent la loi qui sont justes devant Dieu, mais ce sont ceux qui la mettent en pratique qui seront justifiés.
\VS{14}Or quand les gentils, qui n'ont point la loi, font naturellement ce que prescrit la loi, n'ayant point la loi, ils sont une loi pour eux-mêmes.
\VS{15}Et ils montrent par-là que l’œuvre de la loi est écrite dans leurs cœurs, puisque leur conscience leur rend témoignage, et que leurs pensées les accusent ou les défendent.
\VS{16}Tous, dis-je, donc seront jugés le jour où Dieu jugera les secrets des hommes par Jésus-Christ, selon mon Evangile.
\TextTitle{Les Juifs, connaissant la loi, sont condamnés par leur transgression de la loi}
\VS{17}Voici, tu portes le nom de Juif, tu te reposes entièrement sur la loi, et tu te glorifies de Dieu ;
\VS{18}tu connais sa volonté, et tu sais discerner ce qui est contraire, étant instruit par la loi ; 
\VS{19}et tu te crois être le conducteur des aveugles, la lumière de ceux qui sont dans les ténèbres,
\VS{20}le docteur des insensés, le maître des ignorants, ayant le modèle de la science et de la vérité dans la loi.
\VS{21}Toi donc qui enseignes les autres, tu ne t’enseignes pas toi-même ! Toi qui prêches de ne pas dérober, tu dérobes !
\VS{22}Toi qui dis de ne pas commettre d’adultère, tu commets l’adultère ! Toi qui as en abomination les idoles, tu commets des sacrilèges !
\VS{23}Toi qui te glorifies de la loi, tu déshonores Dieu par la transgression de la loi.
\VS{24}Car le nom de Dieu est blasphémé parmi les gentils à cause de vous comme cela est écrit.
\VS{25}Il est vrai que la circoncision est profitable, si tu gardes la loi ; mais si tu es transgresseur de la loi, ta circoncision devient incirconcision.
\VS{26}Si donc l’incirconcis observe les ordonnances de la loi, son incirconcision ne sera-t-elle pas tenue pour circoncision ?
\VS{27}L’incirconcis de nature, qui accomplit la loi, ne te condamnera-t-il pas, toi qui la transgresses, tout en ayant la lettre de la loi et la circoncision ?
\VS{28}Le Juif, ce n’est pas celui qui en a les apparences\FTNT{Le formalisme (2 Ti. 3:5). L’apparence de la piété correspond aux vêtements des brebis : «~Gardez-vous des faux prophètes, ils viennent à vous en habits de brebis, mais au-dedans ce sont des loups ravisseurs.~» (Mt 7:15). «~Puis je vis une autre bête qui montait de la terre, et qui avait deux cornes semblables à celles de l'Agneau ; mais elle parlait comme le dragon.~» Ap. 13:11. Il a l’apparence d’un agneau, mais sa voix est celle du dragon, c’est-à-dire Satan.} ; et la circoncision, ce n’est pas celle qui est visible dans la chair.
\VS{29}Mais le Juif, c’est celui qui l’est intérieurement ; et la circoncision, c’est celle du cœur, selon l’Esprit et non selon la lettre. La louange de ce Juif ne vient pas des hommes, mais de Dieu.
\Chap{3}
\TextTitle{L'avantage du Juif peut devenir une condamnation}
\VerseOne{}Quel est donc l'avantage du Juif, ou quelle est l’utilité de la circoncision ?
\VS{2}Cet avantage est grand de toute manière, et tout d’abord en ce que les oracles de Dieu leur ont été confiés.
\VS{3}Eh quoi ! Si quelques-uns n'ont pas cru, leur incrédulité anéantira-t-elle la fidélité de Dieu ?
\VS{4}Nullement ! Que Dieu au contraire soit reconnu pour vrai, et tout homme pour menteur ; selon ce qui est écrit : Afin que tu sois trouvé juste dans tes paroles, et que tu triomphes lorsqu’on te juge\FTNT{Ps. 51:6.}.
\VS{5}Mais si notre injustice établit la justice de Dieu, que dirons-nous ? Dieu est-il injuste quand il déchaine sa colère ? (Je parle à la manière des hommes.)
\VS{6}Nullement ! Autrement, comment Dieu jugera-t-il le monde ?
\VS{7}Et si par mon mensonge la vérité de Dieu est plus abondante pour sa gloire, pourquoi suis-je encore condamné comme pécheur ?
\VS{8}Et pourquoi ne ferions-nous pas le mal, afin qu'il en arrive du bien, comme quelques-uns, qui nous calomnient, prétendent que nous le disons ? La condamnation de ces gens est juste.
\TextTitle{Juifs et Grecs coupables devant Dieu}
\VS{9}Quoi donc ! Sommes-nous plus excellents ? Nullement. Car nous avons déjà prouvé que tous, tant Juifs que Grecs, sont assujettis au péché.
\VS{10}Selon qu'il est écrit : Il n'y a point de juste, pas même un seul\FTNT{Ps. 14:3.}.
\VS{11}Il n'y a personne qui ait de l'intelligence, il n'y a personne qui recherche Dieu.
\VS{12}Ils se sont tous égarés, ils se sont tous corrompus : Il n'y en a aucun qui fasse le bien, pas même un seul.
\VS{13}Leur gosier est un sépulcre ouvert ; ils se servent de leur langue pour tromper ; il y a du venin d'aspic sous leurs lèvres.
\VS{14}Leur bouche est pleine de malédictions et d'amertume.
\VS{15}Leurs pieds sont légers pour répandre le sang.
\VS{16}La destruction et la misère sont sur leurs voies.
\VS{17}Et ils n'ont point connu la voie de la paix.
\VS{18}La crainte de Dieu n'est pas devant leurs yeux\FTNT{Ps. 14.}.
\VS{19}Or nous savons que tout ce que la loi dit, elle le dit à ceux qui sont sous la loi, afin que toute bouche soit fermée, et que tout le monde soit reconnu coupable devant Dieu.
\VS{20}C'est pourquoi personne ne sera justifié devant lui par les œuvres de la loi, puisque c’est par la loi que vient la connaissance du péché.
\TextTitle{La justification par la foi}
\VS{21}Mais maintenant, sans la loi, la justice de Dieu est manifestée, à laquelle rendent témoignage la loi et les prophètes.
\VS{22}La justice, dis-je, de Dieu par la foi en Jésus-Christ, envers tous et sur tous ceux qui croient. Car il n'y a point de distinction.
\VS{23}Car tous ont péché\FTNT{Le mot péché vient du terme grec «~hamartano~» : «~manquer la marque, manquer le chemin de la droiture et de l’honneur, s’éloigner de la loi de Dieu~». Le péché est la violation délibérée de la loi divine et l’absence de la droiture.} et sont entièrement privés de la gloire de Dieu.
\VS{24}Et ils sont gratuitement justifiés par sa grâce, par la rédemption\FTNT{La rédemption est la délivrance par le paiement d’un prix. Trois termes grecs sont utilisés pour parler de la rédemption : - Agorazo : acheter un objet au marché (agora signifiant marché). Les pécheurs sont considérés comme des esclaves vendus au marché (Ro. 7:14). - Exagorazo : acheter et amener un objet hors du marché (Ga. 3:13 ; Ga. 4:5). L’esclave acheté et amené hors du marché est définitivement délivré. - Lutroo : détacher, rendre libre (Lu. 24:21 ; Tit. 2:14 ; 1 P. 1:18.) Jésus-Christ nous a délivrés du péché, de la puissance de Satan et de la loi mosaïque. (Col. 1:12-14 ; Col. 2:14-17 ; 1 Jn. 3:5).} qui est en Jésus-Christ.
\VS{25}C’est lui que Dieu a destiné à être, par son sang, la victime propitiatoire\FTNT{Le terme propitiation vient du grec «~hilastérion~» qui signifie «~ce qui est expié, ce qui rend propice ou le don qui assure la propitiation~». C’est aussi le lieu où s’accomplit la propitiation (Hé. 9:5), c’est-à-dire le couvercle de l’arche. Lors du grand jour des expiations (Yom Kippour en hébreu), l’aspersion du sang était faite sur le propitiatoire (Lé. 16:14). Le Seigneur Jésus-Christ est notre victime expiatoire (1 Jn. 2:2 ; 1 Jn. 4:10).} pour ceux qui croiraient, afin de montrer sa justice, parce qu’il avait laissé impunis les péchés commis auparavant, au temps de sa patience.
\VS{26}Il montre, dis-je, sa justice dans le temps présent, de manière à être trouvé juste tout en justifiant celui qui a la foi en Jésus.
\VS{27}Où est donc le sujet de se glorifier ? Il est exclu. Par quelle loi ? Est-ce par la loi des œuvres ? Non, mais par la loi de la foi.
\VS{28}Nous concluons donc que l'homme est justifié par la foi, sans les œuvres de la loi.
\TextTitle{Circoncis et incirconcis, justifiés par la foi}
\VS{29}Dieu est-il seulement le Dieu des Juifs ? Ne l'est-il pas aussi des gentils ? Certes, il l'est aussi des gentils,
\VS{30}puisqu’il y a un seul Dieu qui justifiera par la foi les circoncis, et aussi les incirconcis par la foi.
\VS{31}Anéantissons-nous donc la loi par la foi ? Nullement ! Mais au contraire, nous affermissons la loi.
\Chap{4}
\TextTitle{Abraham et David justifiés par la foi\FTNTT{cp. v. 18-25}}
\VerseOne{}Que dirons-nous donc, qu'Abraham, notre père, a obtenu selon la chair ?
\VS{2}Certes, si Abraham a été justifié par les œuvres, il a de quoi se glorifier, mais non pas envers Dieu.
\VS{3}Car que dit l'Ecriture ? Qu’Abraham a cru en Dieu, et que cela lui a été imputé à justice\FTNT{Ge. 15:6.}.
\VS{4}Or à celui qui fait les œuvres, le salaire ne lui est pas imputé comme une grâce, mais comme une chose due.
\VS{5}Mais à celui qui ne fait pas les œuvres, mais qui croit en celui qui justifie le méchant, sa foi lui est imputée à justice.
\VS{6}De même, David exprime le bonheur de l'homme à qui Dieu impute la justice sans les œuvres, en disant :
\VS{7}Heureux sont ceux à qui les iniquités sont pardonnées, et dont les péchés sont couverts.
\VS{8}Heureux l'homme à qui le Seigneur n’impute pas son péché\FTNT{Ps. 32:1-2.}.
\TextTitle{Abraham obtient la justification par la foi avant sa circoncision}
\VS{9}Cette déclaration de bénédiction, est-elle seulement pour les circoncis, ou également pour les incirconcis ? Car nous disons que la foi a été imputée à Abraham à justice.
\VS{10}Comment donc lui a-t-elle été imputée ? Etait-ce après, ou avant sa circoncision ? Il n’était pas encore circoncis, il était incirconcis.
\VS{11}Et il reçut le signe de la circoncision comme sceau de la justice, qu’il avait obtenue par la foi, quand il était incirconcis, afin d’être le père de tous les incirconcis qui croient, pour que la justice leur soit aussi imputée ;
\VS{12}et le père des circoncis, qui ne sont pas seulement circoncis, mais encore qui marchent sur les traces de la foi de notre père Abraham, quand il était incirconcis.
\TextTitle{La justification s'accomplit sans la loi}
\VS{13}En effet, ce n’est pas par loi que la promesse d'être héritier du monde a été faite à Abraham, ou à sa postérité, mais par la justice de la foi.
\VS{14}Car, si les héritiers le sont par la loi, la promesse est annulée, et la foi est vaine
\VS{15}car la loi produit la colère ; car là où il n'y a point de loi, il n'y a point non plus de transgression.
\VS{16}C'est pourquoi les héritiers le sont par la foi, pour que ce soit par la grâce, et afin que la promesse soit assurée à toute la postérité ; non seulement à celle qui est de la loi, mais aussi à celle qui est de la foi d'Abraham, qui est le père de nous tous,
\VS{17}selon qu'il est écrit : Je t'ai établi père de plusieurs nations\FTNT{Ge 17:4-5.}. Il est notre père devant celui auquel il a cru, Dieu qui donne la vie aux morts, et qui appelle les choses qui ne sont point, comme si elles étaient.
\VS{18}Et Abraham ayant espéré contre toute espérance, crut qu'il deviendrait le père de plusieurs nations, selon ce qui lui avait été dit : Ainsi sera ta postérité.
\VS{19}Et sans faiblir dans la foi, il ne considéra point que son corps était déjà usé ; puisqu’il avait environ cent ans, et que Sara n’était plus en âge d'avoir des enfants.
\VS{20}Et il ne douta point de la promesse de Dieu par incrédulité, mais il fut fortifié par la foi, donnant gloire à Dieu,
\VS{21}étant pleinement persuadé que celui qui lui avait fait la promesse était aussi puissant pour l'accomplir.
\VS{22}C'est pourquoi cela lui fut imputé à justice.
\VS{23}Mais ce n’est pas à cause de lui seul qu’il est écrit que cela lui fut imputé à justice ;
\VS{24}c’est encore à cause de nous, à qui cela sera imputé, à nous, dis-je, qui croyons en celui qui a ressuscité des morts, Jésus notre Seigneur,
\VS{25}qui a été livré pour nos offenses, et est ressuscité pour notre justification.
\Chap{5}
\TextTitle{La justification permet la réconciliation avec Dieu}
\VerseOne{}Etant donc justifiés\FTNT{La justification est l’œuvre de Dieu par laquelle la justice de Jésus est comptée en faveur du pécheur, de sorte que le pécheur est déclaré juste par Dieu (Ro. 4 : 3 ; Ro. 5:1-9 ; Ga. 2:16 ; Ga. 3:11). Cette justice n’est pas obtenue par les efforts de la personne sauvée. La justification est une action instantanée qui a pour résultat la vie éternelle. Elle repose totalement et exclusivement sur le sacrifice de Jésus à la croix (1 Pi. 2:24). Elle ne peut être reçue que par la foi en Jésus-Christ (Ep. 2:8-9). La justification est un acte d’imputation divine et non une reconnaissance personnelle de l’homme. Elle provient de la grâce (Ro. 3:24 ; Tit. 3:7).} par la foi, nous avons la paix avec Dieu, par notre Seigneur Jésus-Christ.
\VS{2}Par lequel aussi nous avons été amenés par la foi à cette grâce, dans laquelle nous tenons ferme ; et nous nous glorifions dans l'espérance de la gloire de Dieu.
\VS{3}Bien plus, nous nous glorifions même dans les afflictions ; sachant que l'affliction produit la persévérance ;
\VS{4}et la persévérance l'épreuve ; et l'épreuve l'espérance.
\VS{5}Or l'espérance ne trompe point, parce que l'amour de Dieu est répandu dans nos cœurs par le Saint-Esprit qui nous a été donné.
\VS{6}Car lorsque nous étions encore sans force, Christ est mort en son temps pour nous qui étions des impies.
\VS{7}A peine mourrait-on pour un juste ; quelqu’un peut-être mourrait pour un homme de bien.
\VS{8}Mais Dieu prouve son amour envers nous, en ce que lorsque nous étions encore pécheurs, Christ est mort pour nous.
\VS{9}Etant donc maintenant justifiés par son sang, à plus forte raison serons-nous sauvés par lui de la colère.
\VS{10}Car si, lorsque nous étions ennemis, nous avons été réconciliés avec Dieu par la mort de son Fils, à plus forte raison, étant réconciliés, serons-nous sauvés par sa vie.
\VS{11}Et non seulement cela, mais encore nous nous glorifions même en Dieu par notre Seigneur Jésus-Christ, par qui nous avons maintenant obtenu la réconciliation.
\TextTitle{Parallèle entre l'œuvre de Jésus-Christ et celle d'Adam}
\VS{12}C'est pourquoi comme par un seul homme le péché est entré dans le monde, et par le péché la mort, et qu’ainsi la mort s’est étendue sur tous les hommes, parce que tous ont péché…
\VS{13}Car jusqu'à la loi le péché était dans le monde ; or le péché n'est point imputé quand il n'y a point de loi.
\VS{14}Mais la mort a régné depuis Adam jusqu'à Moïse, même sur ceux qui n'avaient pas péché par une transgression semblable à celle d’Adam, lequel est la figure de celui qui devait venir.
\VS{15}Mais il n'en est pas du don gratuit comme de l'offense ; car si par l'offense d'un seul il en est beaucoup qui sont morts, à plus forte raison la grâce de Dieu, et le don de la grâce, venant d'un seul homme, à savoir de Jésus-Christ, ont-ils été abondamment répandus sur plusieurs.
\VS{16}Et il n'en est pas du don comme de ce qui est arrivé par un seul qui a péché ; car c’est après une seule offense que le jugement est devenu condamnation, mais le don gratuit devient justification après plusieurs offenses.
\VS{17}Si par l'offense d'un seul la mort a régné par lui seul, à plus forte raison ceux qui reçoivent l'abondance de la grâce, et du don de la justice, régneront-ils dans la vie par Jésus-Christ lui seul.
\VS{18}Ainsi donc, comme par une seule offense la condamnation est venue sur tous les hommes, de même par un acte de justice la justification qui donne la vie s’étend à tous les hommes.
\VS{19}Car, comme par la désobéissance d'un seul homme plusieurs ont été rendus pécheurs, de même par l'obéissance d'un seul plusieurs seront rendus justes.
\VS{20}Or la loi est intervenue afin que l'offense abonde, mais là où le péché a abondé, la grâce a surabondé,
\VS{21}afin que, comme le péché a régné par la mort, ainsi la grâce règne par la justice pour donner la vie éternelle, par Jésus-Christ notre Seigneur.
\Chap{6}
\TextTitle{Délivré de la puissance du péché lié au cœur de l’homme }
\VerseOne{}Que dirons-nous donc ? Demeurerions-nous dans le péché, afin que la grâce abonde ?
\VS{2}A Dieu ne plaise ! Car nous qui sommes morts au péché, comment vivrions-nous encore dans le péché ?
\VS{3}Ignoriez-vous que nous tous qui avons été baptisés en Jésus-Christ, c’est en sa mort que nous avons été baptisés ?
\VS{4}Nous avons donc été ensevelis avec lui par le baptême en sa mort ; afin que comme Christ est ressuscité des morts par la gloire du Père, de même nous aussi nous marchions en nouveauté de vie.
\VS{5}Car, si nous sommes devenus une même plante avec lui par la conformité à sa mort, nous le serons aussi par la conformité à sa résurrection.
\VS{6}Sachant que notre vieil homme a été crucifié avec lui, afin que le corps du péché soit détruit, pour que nous ne soyons plus esclaves du péché.
\VS{7}Car celui qui est mort est libre du péché.
\VS{8}Or si nous sommes morts avec Christ, nous croyons que nous vivrons aussi avec lui,
\VS{9}sachant que Christ ressuscité des morts ne meurt plus, et que la mort n'a plus de pouvoir sur lui.
\VS{10}Car il est mort, et c’est pour le péché qu’il est mort une fois pour toutes ; il est revenu à la vie, et c’est pour Dieu qu’il est vivant.
\TextTitle{Mort au péché pour une vie nouvelle en Dieu}
\VS{11}Ainsi vous-mêmes, considérez-vous comme morts au péché, et comme vivants pour Dieu en Jésus-Christ notre Seigneur.
\VS{12}Que le péché ne règne donc point dans votre corps mortel, et n’obéissez pas à ses convoitises.
\VS{13}Et ne livrez pas vos membres au péché comme des instruments d'iniquité ; mais donnez-vous vous-mêmes à Dieu comme de morts étant devenus vivants, et offrez vos membres à Dieu pour être des instruments de justice.
\VS{14}Car le péché n'aura pas de domination sur vous, parce que vous n'êtes point sous la loi, mais sous la grâce.
\VS{15}Quoi donc ? Pécherions-nous parce que nous ne sommes point sous la loi, mais sous la grâce ? A Dieu ne plaise !
\VS{16}Ne savez-vous pas qu’en vous livrant à quelqu’un comme esclaves pour lui obéir, vous êtes esclaves de celui à qui vous obéissez, soit du péché qui conduit à la mort, soit de l'obéissance qui conduit à la justice ?
\VS{17}Mais grâces à Dieu de ce qu'ayant été les esclaves du péché, vous avez obéi de cœur à la forme expresse de la doctrine dans laquelle vous avez été élevés.
\VS{18}Ayant donc été affranchis du péché, vous avez été asservis à la justice.
\VS{19}Je parle à la façon des hommes, à cause de l'infirmité de votre chair. Comme donc vous avez appliqué vos membres pour servir à la souillure et à l’iniquité, ainsi appliquez vos membres pour servir à la justice en sainteté.
\VS{20}Car lorsque vous étiez esclaves du péché, vous étiez libres à l'égard de la justice.
\VS{21}Quel fruit portiez-vous alors ? Des fruits dont vous avez honte maintenant. Car la fin de ces choses c’est la mort.
\VS{22}Mais maintenant que vous êtes affranchis du péché, et asservis à Dieu, vous avez pour fruit la sanctification, et pour fin la vie éternelle.
\VS{23}Car le salaire du péché, c'est la mort ; mais le don gratuit de Dieu, c'est la vie éternelle par Jésus-Christ notre Seigneur.
\Chap{7}
\TextTitle{Le chrétien lié à Christ comme à un époux}
\VerseOne{}Ignorez-vous, frères, car je parle à des gens qui connaissent la loi, que la loi exerce son pouvoir sur l’homme aussi longtemps qu’il vit ?
\VS{2}Car la femme qui est sous la puissance d'un mari, est liée à son mari par la loi tandis qu'il est en vie ; mais si son mari meurt, elle est délivrée de la loi du mari.
\VS{3}Si donc, du vivant de son mari, elle épouse un autre homme, elle sera appelée adultère ; mais si son mari meurt, elle est délivrée de la loi, de sorte qu'elle ne sera point adultère si elle épouse un autre homme.
\VS{4}Ainsi donc, vous aussi, mes frères, vous avez été, par le corps de Christ, mis à mort en ce qui concerne la loi, pour que vous apparteniez à un autre, à savoir, à celui qui est ressuscité des morts, afin que nous portions des fruits pour Dieu.
\VS{5}Car lorsque nous étions dans la chair, les passions des péchés excitées par la loi, agissaient dans nos membres de manière à produire des fruits pour la mort.
\VS{6}Mais maintenant nous sommes délivrés de la loi, étant morts à cette loi sous laquelle nous étions retenus ; afin que nous servions Dieu dans un esprit nouveau, et non selon la lettre qui a vieilli.
\TextTitle{La loi a révélé le péché mais la délivrance vient par Jésus-Christ}
\VS{7}Que dirons-nous donc ? La loi est-elle péché ? Nullement ! Au contraire, je n'ai connu le péché que par la loi ; car je n’aurais pas connu la convoitise, si la loi n’avait pas dit : Tu ne convoiteras point\FTNT{Ex. 20:17.}.
\VS{8}Et le péché, saisissant l’occasion, produisit en moi, par le commandement, toutes sortes de convoitises ; parce que sans la loi le péché est mort.
\VS{9}Pour moi, étant autrefois sans loi, je vivais. Mais quand le commandement vint, le péché reprit vie, et moi je mourus.
\VS{10}Ainsi, le commandement qui conduit à la vie se trouva pour moi conduire à la mort.
\VS{11}Car le péché, saisissant l’occasion, me séduisit par le commandement, et par lui me fit mourir.
\VS{12}La loi donc est sainte, et le commandement est saint, juste, et bon.
\VS{13}Ce qui est bon a-t-il donc été pour moi une cause de mort ? Nullement ! Mais c’est le péché, afin qu'il se manifeste comme péché, en me donnant la mort par ce qui est bon, et que par le commandement, il devienne condamnable au plus haut point.
\VS{14}Car nous savons, en effet, que la loi est spirituelle ; mais moi, je suis charnel, vendu au péché.
\TextTitle{[la connaissance du bien incapable de délivrer l'homme du péché}
\VS{15}Car je n'approuve pas ce que je fais, puisque je ne fais point ce que je veux, mais je fais ce que je hais.
\VS{16}Or si je fais ce que je ne veux pas, je reconnais par cela même que la loi est bonne.
\VS{17}Et maintenant donc ce n'est plus moi qui fais cela, mais c'est le péché qui habite en moi.
\VS{18}Ce qui est bon, je le sais, n’habite pas en moi, c’est-à-dire dans ma chair. J’ai la volonté, mais non le pouvoir de faire le bien.
\VS{19}Car je ne fais pas le bien que je veux, mais je fais le mal que je ne veux point.
\VS{20}Or si je fais ce que je ne veux point, ce n'est plus moi qui le fais, mais c'est le péché qui habite en moi.
\VS{21}Je trouve donc cette loi au-dedans de moi : Quand je veux faire le bien, le mal est attaché à moi.
\VS{22}Car je prends bien plaisir à la loi de Dieu quant à l'homme intérieur,
\VS{23}mais je vois dans mes membres une autre loi, qui combat contre la loi de mon entendement\FTNT{Entendement : Du grec «~nous~», c’est-à-dire l’esprit, l’intelligence, le bon sens, la raison.}, et qui me rend prisonnier à la loi du péché qui est dans mes membres.
\VS{24}Ah, misérable que je suis ! Qui me délivrera du corps de cette mort ?
\TextTitle{Seul l'Esprit de Christ libère de la loi du péché}
\VS{25}Je rends grâces à Dieu par Jésus-Christ notre Seigneur !… Ainsi donc, moi-même, je suis par l’entendement esclave de la loi de Dieu, et je suis par la chair esclave de la loi du péché.
\Chap{8}
\VerseOne{}Il n'y a donc maintenant aucune condamnation pour ceux qui sont en Jésus-Christ, qui marchent, non selon la chair, mais selon l'Esprit.
\VS{2}Parce que la loi de l'Esprit de vie qui est en Jésus-Christ m'a affranchi de la loi du péché et de la mort.
\VS{3}Car chose impossible à la loi, parce que la chair la rendait impuissante, Dieu a condamné le péché dans la chair, en envoyant, à cause du péché, son propre Fils dans une chair semblable à celle du péché.
\VS{4}Afin que la justice de la loi soit accomplie en nous, qui ne marchons point selon la chair, mais selon l'Esprit.
\TextTitle{L'affection de l'Esprit opposée à celle de la chair\FTNTT{cp. Ga. 5:15-18}}
\VS{5}Car ceux, en effet, qui vivent selon la chair, s’affectionnent aux choses de la chair, tandis que ceux qui vivent selon l'Esprit, s’affectionnent aux choses de l'Esprit.
\VS{6}Or l'affection de la chair c’est la mort, tandis que l'affection de l'Esprit c’est la vie et la paix.
\VS{7}Car l'affection de la chair est inimitié contre Dieu, parce qu’elle ne se soumet pas à la loi de Dieu, et qu’elle ne le peut même pas.
\VS{8}C'est pourquoi ceux qui vivent selon la chair ne sauraient plaire à Dieu.
\VS{9}Pour vous, vous ne vivez pas selon la chair, mais selon l'Esprit, si du moins l'Esprit de Dieu habite en vous. Si quelqu'un n'a pas l'Esprit de Christ\FTNT{Notez que le Saint-Esprit est aussi appelé l’Esprit de Jésus (Ac. 16:7).}, il ne lui appartient pas.
\VS{10}Et si Christ est en vous, le corps est bien mort à cause du péché, mais l'Esprit est vie à cause de la justice.
\VS{11}Et si l'Esprit de celui qui a ressuscité Jésus d’entre les morts habite en vous, celui qui a ressuscité Christ d’entre les morts rendra aussi la vie à vos corps mortels par son Esprit qui habite en vous.
\VS{12}Ainsi donc, mes frères, nous ne sommes point redevables à la chair, pour vivre selon la chair.
\VS{13}Car si vous vivez selon la chair, vous mourrez ; mais si par l'Esprit vous faites mourir les actions du corps, vous vivrez.
\TextTitle{L'Esprit d'adoption\FTNTT{Ga. 4:7}}
\VS{14}Car tous ceux qui sont conduits par l'Esprit de Dieu sont enfants de Dieu.
\VS{15}Et vous n'avez point reçu un esprit de servitude pour être encore dans la crainte ; mais vous avez reçu l'Esprit d'adoption, par lequel nous crions Abba, c'est-à-dire Père.
\VS{16}L’Esprit lui-même rend témoignage à notre esprit que nous sommes enfants de Dieu.
\VS{17}Et si nous sommes enfants, nous sommes aussi héritiers : Héritiers, dis-je, de Dieu, et cohéritiers de Christ ; si toutefois nous souffrons avec lui, afin d’être glorifiés avec lui.
\TextTitle{La gloire à venir\FTNTT{cp. Ge. 3:18-19}}
\VS{18}Car tout bien compté, j'estime que les souffrances du temps présent ne sauraient être comparables à la gloire à venir qui doit être révélée pour nous.
\VS{19}Aussi, la création attend-elle avec un ardent désir la révélation des fils de Dieu.
\VS{20}Car la création a été soumise à la vanité, non de son gré, mais à cause de celui qui l’y a soumise ;
\VS{21}avec l’espérance qu’elle aussi sera affranchie de la servitude de la corruption, pour avoir part à la liberté de la gloire des enfants de Dieu.
\VS{22}Or, nous savons que jusqu’à ce jour, toute la création soupire et souffre les douleurs de l’enfantement.
\VS{23}Et non seulement elle, mais nous aussi, qui avons les prémices de l'Esprit ; nous-mêmes, dis-je, soupirons en nous-mêmes, en attendant l'adoption, c'est-à-dire la rédemption de notre corps\FTNT{1 Co. 15:35-43 ; 1 Co. 15:51-54.}.
\VS{24}Car c’est en espérance que nous sommes sauvés. Or l’espérance qu’on voit n’est plus espérance : Ce qu’on voit, peut-on l’espérer encore ?
\VS{25}Mais si nous espérons ce que nous ne voyons pas, c'est que nous l'attendons avec patience.
\TextTitle{L'Esprit intercède pour les saints\FTNTT{Hé. 7:25}}
\VS{26}De même aussi l’Esprit nous aide dans notre faiblesse, car nous ne savons pas ce qu’il nous convient de demander dans nos prières. Mais l’Esprit lui-même intercède par des soupirs inexprimables.
\VS{27}Et celui qui sonde les cœurs connaît quelle est la pensée de l'Esprit, car il intercède en faveur des saints, selon Dieu.
\TextTitle{Le plan de Dieu s'accomplit par l'Evangile}
\VS{28}Or nous savons aussi que toutes choses concourent au bien de ceux qui aiment Dieu, c'est-à-dire de ceux qui sont appelés selon son dessein.
\VS{29}Car ceux qu'il a connus d’avance, il les a aussi prédestinés à être semblables à l'image de son Fils, afin qu'il soit le premier-né de beaucoup de frères.
\VS{30}Et ceux qu'il a prédestinés, il les a aussi appelés ; et ceux qu'il a appelés, il les a aussi justifiés ; et ceux qu'il a justifiés, il les a aussi glorifiés.
\VS{31}Que dirons-nous donc à l’égard de ces choses ? Si Dieu est pour nous, qui sera contre nous ?
\VS{32}Lui qui n'a point épargné son propre Fils, mais qui l'a livré pour nous tous, comment ne nous donnera-t-il point aussi toutes choses avec lui ?
\VS{33}Qui accusera les élus de Dieu ? Dieu est celui qui justifie.
\VS{34}Qui les condamnera ? Christ est mort ; et bien plus, il est ressuscité, il est à la droite de Dieu, et il intercède pour nous.
\TextTitle{L'amour de Christ résiste contre tout}
\VS{35}Qui nous séparera de l'amour de Christ ? Sera-ce l'oppression, ou l'angoisse, ou la persécution, ou la famine, ou la nudité, ou le péril, ou l'épée ?
\VS{36}Ainsi qu'il est écrit : C’est à cause de toi que nous sommes livrés à la mort tous les jours, et qu’on nous regarde comme des brebis destinées à la boucherie\FTNT{Ps. 44:23.}.
\VS{37}Mais dans toutes ces choses nous sommes plus que vainqueurs par celui qui nous a aimés.
\VS{38}Car j’ai l’assurance que ni la mort, ni la vie, ni les anges, ni les principautés, ni les puissances, ni les choses présentes, ni les choses à venir,
\VS{39}ni la hauteur, ni la profondeur, ni aucune autre créature, ne pourra nous séparer de l'amour de Dieu manifesté en Jésus-Christ notre Seigneur.
\Chap{9}
\TextTitle{Le chagrin de Paul pour Israël son peuple}
\VerseOne{}Je dis la vérité en Christ, je ne mens point, ma conscience m’en rend témoignage par le Saint-Esprit :
\VS{2}J’éprouve une grande tristesse et un chagrin continuel dans mon cœur.
\VS{3}Car moi-même je souhaiterais être anathème et séparé de Christ pour mes frères, mes parents selon la chair,
\TextTitle{Les enfants de la chair et ceux de la promesse}
\VS{4}qui sont Israélites, à qui appartiennent l'adoption, la gloire, les alliances, l'ordonnance de la loi, le culte,
\VS{5}les promesses, les patriarches, et de qui est issu selon la chair Christ, qui est Dieu au-dessus de toutes choses, béni éternellement, Amen !
\VS{6}Toutefois il ne peut pas se faire que la parole de Dieu soit anéantie. Car tous ceux qui descendent d’Israël ne sont pas Israël.
\VS{7}Et bien qu’ils soient de la postérité d'Abraham, ils ne sont pas tous ses enfants, car il est dit : C'est en Isaac que tu auras une postérité appelée de ton nom ;
\VS{8}c'est-à-dire que ce ne sont pas ceux qui sont enfants de la chair qui sont enfants de Dieu, mais que ce sont les enfants de la promesse qui sont regardés comme la postérité.
\VS{9}Car voici la parole de la promesse : Je viendrai à cette même époque, et Sara aura un fils\FTNT{Ge. 18:10.}.
\VS{10}Et de plus, il en fut ainsi de Rébecca, qui conçut du seul Isaac notre père ;
\VS{11}car les enfants n’étaient pas encore nés et ils n’avaient fait ni bien ni mal, afin que le dessein arrêté selon l'élection de Dieu subsiste, sans dépendre des œuvres, mais par la volonté de celui qui appelle,
\VS{12}il lui fut dit : L’aîné sera assujetti au plus petit\FTNT{Ge. 25:23.}, selon qu’il est écrit :
\VS{13}J'ai aimé Jacob, et j'ai haï Esaü\FTNT{Mal. 1:2-3.}.
\TextTitle{La volonté souveraine de Dieu}
\VS{14}Que dirons-nous donc : Y a-t-il de l’injustice en Dieu ? A Dieu ne plaise !
\VS{15}Car il dit à Moïse : J'aurai compassion de celui de qui j’aurai compassion et je ferai miséricorde à celui à qui je ferai miséricorde\FTNT{Ex. 33:19.}.
\VS{16}Ainsi donc, cela ne vient pas de celui qui veut, ni de celui qui court, mais de Dieu qui fait miséricorde.
\VS{17}Car l'Ecriture dit à Pharaon : Je t'ai suscité dans le but de démontrer en toi ma puissance, et afin que mon Nom soit publié par toute la terre\FTNT{Ex. 9:16.}.
\VS{18}Ainsi, il fait miséricorde à qui il veut, et il endurcit qui il veut.
\VS{19}Tu me diras : Pourquoi se plaint-il encore ? Car qui est celui qui peut résister à sa volonté ?
\VS{20}Mais plutôt, ô homme, qui es-tu, toi qui contestes contre Dieu ? Le vase d’argile dira-t-il à celui qui l'a formé : Pourquoi m'as-tu ainsi fait ?
\VS{21}Le potier n'a-t-il pas le pouvoir de faire avec la même masse de terre un vase d’honneur et un vase d’un usage vil ?
\VS{22}Et que dire, si Dieu, en voulant montrer sa colère, et faire connaître sa puissance, a supporté avec une grande patience les vases de colère, préparés pour la perdition ?
\VS{23}Et s’il a voulu faire connaître les richesses de sa gloire envers les vases de miséricorde, qu'il a préparés d’avance pour la gloire ?
\VS{24}Ainsi il nous a appelés, non seulement d'entre les juifs, mais aussi d'entre les gentils,
\TextTitle{Les prophéties concernant l'aveuglement d'Israël et la grâce sur les gentils}
\VS{25}selon ce qu'il dit dans Osée : J'appellerai mon peuple celui qui n'était point mon peuple ; et la bien-aimée, celle qui n'était point la bien-aimée ;
\VS{26}et il arrivera qu'au lieu où il leur a été dit : Vous n’êtes pas mon peuple, là ils seront appelés les fils du Dieu vivant\FTNT{Os. 2:1.}.
\VS{27}Aussi Esaïe s'écrie au sujet d'Israël : Quand le nombre des enfants d'Israël serait comme le sable de la mer, un petit reste seulement sera sauvé.
\VS{28}Car le Seigneur exécutera pleinement et promptement sa parole sur la terre ce qu’il a résolu\FTNT{Es. 10:22-23.}.
\VS{29}Et comme Esaïe avait dit auparavant : Si le Seigneur des armées ne nous avait laissé une postérité, nous serions devenus comme Sodome, et nous aurions été semblables à Gomorrhe\FTNT{Es. 1:9.}.
\VS{30}Que dirons-nous donc ? Que les gentils, qui ne cherchaient pas la justice, ont obtenu la justice, la justice qui vient de la foi,
\VS{31}tandis qu’Israël qui cherchait la loi de la justice, n'est pas parvenu à cette loi.
\VS{32}Pourquoi ? Parce qu’Israël l’a cherchée non par la foi, mais comme provenant des œuvres de la loi. Ils se sont heurtés contre la pierre d'achoppement,
\VS{33}selon qu’il est écrit : Voici, je mets en Sion la pierre d'achoppement ; et un rocher de scandale, et quiconque croit en lui ne sera point confus\FTNT{Es. 28:16.}.
\Chap{10}
\TextTitle{La foi, seule condition du salut}
\VerseOne{}Mes frères, le souhait de mon cœur, et la prière que je fais à Dieu pour les Israélites, c'est qu'ils soient sauvés.
\VS{2}Car je leur rends témoignage qu'ils ont du zèle pour Dieu, mais sans connaissance.
\VS{3}Parce que ne connaissant point la justice de Dieu, et cherchant à établir leur propre justice, ils ne se sont point soumis à la justice de Dieu.
\VS{4}Car Christ est la fin de la loi\FTNT{Il est question de la loi cérémonielle relative au culte mosaïque. Avant sa mort, Jésus qui était né sous la loi (Ga. 4:4), demandait aux gens de l’appliquer. Ainsi, il demanda au lépreux qu’il avait guéri de présenter une offrande pour sa purification au temple (Mt. 8:1-4) et à ses disciples d'observer l'enseignement des scribes (Mt. 23:1-2). En effet, il fallait que les lois cérémonielles soient respectées jusqu’à sa résurrection. Une fois que Jésus eut dit «~tout est accompli~» (Jn. 19:30), toutes ces lois n’avaient plus aucune raison d’être (Col. 2:14-17 ; Hé. 7:11-22 ; Hé. 10:1-2).} pour la justification de tous ceux qui croient.
\VS{5}En effet, Moïse décrit ainsi la justice qui vient de la loi : L'homme qui fera ces choses vivra par elles\FTNT{Lé. 18:5.}.
\VS{6}Mais voici comment s'exprime la justice qui vient de la foi : Ne dis pas en ton cœur : Qui montera au ciel ? C’est en faire descendre Christ.
\VS{7}Ou : Qui descendra dans l'abîme ? C’est faire remonter Christ d’entre les morts.
\VS{8}Mais que dit-elle ? La parole est près de toi, dans ta bouche, et dans ton cœur. Or voilà la parole de foi que nous prêchons.
\VS{9}C'est pourquoi, si tu confesses de ta bouche le Seigneur Jésus, et si tu crois dans ton cœur que Dieu l'a ressuscité des morts, tu seras sauvé.
\VS{10}Car c’est en croyant du cœur qu’on parvient à la justice, et c’est en confessant de la bouche qu’on parvient au salut, selon ce que dit l’Ecriture :
\VS{11}Quiconque croit en lui ne sera point confus\FTNT{Es. 49:23.}.
\VS{12}Parce qu'il n'y a point de différence, en effet, entre le Juif et le Grec, puisqu’ils ont un même Seigneur, qui est riche pour tous ceux qui l'invoquent.
\VS{13}Car quiconque invoquera le nom du Seigneur sera sauvé\FTNT{Joë. 2:32.}.
\TextTitle{La proclamation de l'Evangile dans les nations}
\VS{14}Mais comment invoqueront-ils celui en qui ils n'ont point cru ? Et comment croiront-ils en celui dont ils n'ont point entendu parler ? Et comment en entendront-ils parler s'il n'y a personne qui leur prêche ?
\VS{15}Et comment y aura-t-il des prédicateurs, s’ils ne sont pas envoyés ? Selon qu'il est écrit : Qu’ils sont beaux les pieds de ceux qui annoncent la paix, de ceux qui annoncent de bonnes nouvelles\FTNT{Es. 52:7.} !
\VS{16}Mais tous n'ont pas obéi à l'Evangile ; car Esaïe dit : Seigneur, qui a cru à notre prédication\FTNT{Es. 53:1.} ?
\VS{17}Ainsi la foi vient de ce qu’on entend, et ce qu’on entend vient de la parole de Christ.
\VS{18}Mais je dis : Ne l'ont-ils point entendue ? Au contraire, leur voix est allée par toute la terre, et leur parole jusqu’aux extrémités du monde.
\VS{19}Mais je dis : Israël ne l'a-t-il point su ? Moïse le premier dit : J’exciterai votre jalousie par ce qui n'est point une nation, je provoquerai votre colère par une nation sans intelligence\FTNT{De. 32:21.}.
\VS{20}Et Esaïe pousse la hardiesse jusqu’à dire : J'ai été trouvé par ceux qui ne me cherchaient point, et je me suis clairement manifesté à ceux qui ne me demandaient pas\FTNT{Es. 65:1.}.
\VS{21}Mais au sujet d’Israël, il dit : J'ai tout le jour tendu mes mains vers un peuple rebelle et contredisant\FTNT{Es. 65:2.}.
\Chap{11}
\TextTitle{Un reste d'Israël participe à la grâce}
\VerseOne{}Je dis donc : Dieu a-t-il rejeté son peuple ? A Dieu ne plaise ! Car je suis aussi Israélite, de la postérité d'Abraham, de la tribu de Benjamin.
\VS{2}Dieu n'a point rejeté son peuple, qu’il a connu d’avance. Et ne savez-vous pas ce que l'Ecriture dit d'Elie, comment il a fait requête à Dieu contre Israël, disant :
\VS{3}Seigneur, ils ont tué tes prophètes, et ils ont démoli tes autels, et je suis resté moi seul ; et ils cherchent à m'ôter la vie\FTNT{1 R. 19:10.}.
\VS{4}Mais quelle réponse Dieu lui donna-t-il ? Je me suis réservé sept mille hommes, qui n'ont point fléchi le genou devant Baal\FTNT{1 R.19:18.}.
\VS{5}De même aussi dans le temps présent, il y a un reste selon l'élection de la grâce.
\VS{6}Or si c'est par la grâce, ce n'est plus par les œuvres ; autrement la grâce n'est plus la grâce. Mais si c'est par les œuvres, ce n'est plus par une grâce ; autrement l’œuvre n'est plus une œuvre.
\TextTitle{La nation d'Israël est temporairement mise à l'écart mais non rejetée}
\VS{7}Quoi donc ? Ce qu'Israël cherche, il ne l'a point obtenu ; mais les élus l’ont obtenu, tandis que les autres ont été endurcis,
\VS{8}selon qu'il est écrit : Dieu leur a donné un esprit d’assoupissement, des yeux pour ne point voir, et des oreilles pour ne point entendre\FTNT{Es. 29:10.}, jusqu’à ce jour. Et David dit :
\VS{9}Que leur table soit pour eux un filet, un piège, une occasion de chute, et cela pour leur récompense.
\VS{10}Que leurs yeux soient obscurcis pour ne point voir\FTNT{Ps. 69:23-24.} ; et tiens continuellement leur dos courbé !
\VS{11}Mais je dis : Est-ce pour tomber qu’ils ont bronché ? Nullement ! Mais par leur chute, le salut est accordé aux Gentils, afin qu’ils soient excités à la jalousie.
\VS{12}Or si leur chute est la richesse du monde, et leur amoindrissement la richesse des Gentils, combien plus en sera-t-il quand ils se convertiront tous ?
\TextTitle{Avertissement aux gentils}
\VS{13}Car je vous parle à vous, Gentils, en tant qu’apôtre des Gentils, je glorifie mon ministère,
\VS{14}afin, s’il est possible, d’exciter la jalousie de ceux de ma race et d’en sauver quelques-uns.
\VS{15}Car si leur mise à l’écart a été la réconciliation du monde, quelle sera leur réintégration, sinon le passage de la mort à la vie ?
\VS{16}Or si les prémices sont saintes, la masse l'est aussi ; et si la racine est sainte, les branches le sont aussi.
\VS{17}Mais si quelques-unes des branches ont été retranchées, et si toi qui étais un olivier sauvage, tu as été greffé à leur place et rendu participant de la racine et de la graisse de l'olivier,
\VS{18}ne te glorifie pas contre ces branches ; car si tu te glorifies, ce n'est pas toi qui portes la racine, mais c'est la racine qui te porte.
\VS{19}Mais tu diras : Les branches ont été retranchées, afin que moi je sois greffé.
\VS{20}Cela est vrai, elles ont été retranchées à cause de leur incrédulité, et tu es debout par la foi ; ne t'élève donc point par orgueil, mais crains.
\VS{21}Car si Dieu n'a point épargné les branches naturelles, prends garde qu'il ne t'épargne pas non plus.
\VS{22}Considère donc la bonté et la sévérité de Dieu ; la sévérité envers ceux qui sont tombés ; et la bonté envers toi, si tu persévères dans cette bonté : Car autrement tu seras aussi retranché.
\VS{23}Eux de même, s'ils ne persistent pas dans leur incrédulité, ils seront greffés ; car Dieu est puissant pour les greffer de nouveau.
\VS{24}Car si toi tu as été coupé de l'olivier sauvage selon sa nature, et greffé contrairement à ta nature sur l'olivier franc, à plus forte raison eux seront-ils greffés selon leur nature sur leur propre olivier.
\VS{25}Car mes frères, je ne veux pas que vous ignoriez ce mystère, afin que vous ne vous regardiez point comme sages : Une partie d’Israël est tombée dans l’endurcissement, jusqu’à ce que la totalité des gentils soit entrée.
\TextTitle{Yahweh prédit le salut futur d'Israël\FTNTT{Es. 66:8}}
\VS{26}Et ainsi tout Israël sera sauvé, selon qu’il est écrit : Le Libérateur viendra de Sion, et il détournera de Jacob les infidélités ;
\VS{27}et c'est là l'alliance que je ferai avec eux, lorsque j'ôterai leurs péchés\FTNT{Es. 59:20-21.}.
\VS{28}Ils sont certes ennemis par rapport à l'Evangile, à cause de vous ; mais en ce qui concerne l’élection, ils sont aimés à cause de leurs pères.
\VS{29}Car Dieu ne se repent pas de ses dons et de sa vocation.
\VS{30}De même que vous avez autrefois désobéi à Dieu et que par leur désobéissance vous avez maintenant obtenu miséricorde,
\VS{31}de même ils ont maintenant désobéi, afin que par la miséricorde qui vous a été faite, ils obtiennent aussi miséricorde.
\VS{32}Car Dieu les a tous renfermés sous la rébellion afin de faire miséricorde à tous.
\TextTitle{Les voies incompréhensibles de Dieu}
\VS{33}Ô profondeur de la richesse, de la sagesse et de la connaissance de Dieu ! Que ses jugements sont insondables et ses voies incompréhensibles !
\VS{34}Car qui a connu la pensée du Seigneur ? Ou qui a été son conseiller ?
\VS{35}Qui lui a donné le premier, pour qu’il ait à recevoir en retour ?
\VS{36}Car c’est de lui, par lui, et pour lui que sont toutes choses. A lui soit la gloire éternellement. Amen !
\Chap{12}
\TextTitle{Le culte raisonnable}
\VerseOne{}Je vous exhorte donc, mes frères, par les compassions de Dieu, à offrir vos corps comme un sacrifice vivant, saint, agréable à Dieu, ce qui est votre culte raisonnable.
\VS{2}Et ne vous conformez pas au siècle présent, mais soyez transformés\FTNT{Le verbe «~transformer~» est la traduction du terme grec «~metamorphoo~» qui a donné en français «~transfigurer~». C’est le même terme qui a été utilisé en Mt. 17:2 pour parler de la transfiguration du Seigneur. Si Paul recommandait cela à des personnes déjà converties, c’est parce que Dieu les appelait à aller plus loin. La transformation d’une chenille en papillon est un très bel exemple pour illustrer le changement radical qui doit s’opérer en nous. Pour atteindre ce stade, cet insecte passe par plusieurs étapes. La transformation nous permet de croître spirituellement. En effet, tout enfant de Dieu est appelé à devenir mature, à passer du stade de petit enfant à celui de jeune homme, et de celui de jeune homme à celui de père (1 Jn. 2:12-14).} par le renouvellement de votre entendement, afin que vous discerniez quelle est la volonté de Dieu, ce qui est bon, agréable et parfait.
\TextTitle{Exhortation à l'humilité et au service selon les dons de l'Esprit}
\VS{3}Par la grâce qui m’a été donnée, je dis à chacun de vous que nul ne présume d'être plus sage qu'il ne faut, mais d’avoir des sentiments modestes, selon la mesure de foi que Dieu a départie à chacun.
\VS{4}Car comme nous avons plusieurs membres dans un seul corps, et que tous les membres n'ont pas la même fonction,
\VS{5}ainsi, nous qui sommes plusieurs, nous formons un seul corps en Christ, et nous sommes tous membres les uns des autres.
\VS{6}Puisque nous avons des dons différents, selon la grâce qui nous est donnée, que celui qui a le don de prophétie l’exerce en analogie de la foi ;
\VS{7}que celui qui est appelé au ministère, s’attache à son ministère ; que celui qui enseigne s’attache à son enseignement,
\VS{8}et celui qui exhorte, à l’exhortation ; que celui qui donne, le fasse avec simplicité ; que celui qui préside, le fasse avec zèle ; que celui qui exerce la miséricorde, le fasse avec joie.
\TextTitle{Les relations mutuelles entre chrétien}
\VS{9}Que la charité soit sincère. Ayez en horreur le mal, attachez-vous fortement au bien.
\VS{10}Par charité fraternelle, soyez pleins d’affection les uns pour les autres ; par honneur, usez de prévenances réciproques.
\VS{11}Ne soyez point paresseux à vous employer pour autrui. Soyez fervents d'esprit. Servez le Seigneur.
\VS{12}Soyez joyeux dans l'espérance. Soyez patients dans la tribulation. Persévérez dans la prière.
\VS{13}Pourvoyez aux besoins des saints. Exercez l'hospitalité.
\VS{14}Bénissez ceux qui vous persécutent ; bénissez-les, et ne les maudissez point.
\VS{15}Réjouissez-vous avec ceux qui se réjouissent. Pleurez avec ceux qui pleurent.
\VS{16}Ayez les mêmes sentiments les uns envers les autres. N’aspirez pas à ce qui est élevé, mais laissez-vous attirer par ce qui est humble. Ne soyez point sages à votre propre jugement.
\TextTitle{Les relations du chrétiens avec ceux du dehors}
\VS{17}Ne rendez à personne le mal pour le mal. Recherchez les choses honnêtes devant tous les hommes.
\VS{18}S’il est possible, autant que cela dépend de vous, soyez en paix avec tous les hommes.
\VS{19}Ne vous vengez point vous-mêmes, mes bien-aimés, mais laissez agir la colère de Dieu, car il est écrit : A moi appartient la vengeance, à moi la rétribution, dit le Seigneur\FTNT{De. 32:35.}.
\VS{20}Si donc ton ennemi a faim, donne-lui à manger ; s'il a soif, donne-lui à boire, car en faisant cela, tu amasseras des charbons ardents sur sa tête.
\VS{21}Ne te laisse pas vaincre par le mal, mais surmonte le mal par le bien.
\Chap{13}
\TextTitle{Le chrétien et les autorités}
\VerseOne{}Que toute personne soit soumise aux autorités supérieures, car il n'y a point d’autorité qui ne vienne pas de Dieu, et les autorités qui existent ont été instituées de Dieu.
\VS{2}C'est pourquoi celui qui s’oppose à l’autorité résiste à l’ordre de Dieu ; et ceux qui y résistent attireront la condamnation sur eux-mêmes.
\VS{3}Car ce n’est pas pour une bonne action, c’est pour une mauvaise que les magistrats sont à craindre. Veux-tu ne pas craindre l’autorité ? Fais le bien, et tu auras sa louange.
\VS{4}Car le magistrat est un serviteur de Dieu pour ton bien. Mais si tu fais le mal, crains, car ce n’est pas en vain qu’il porte l'épée, étant serviteur de Dieu, ordonné pour faire justice en punissant celui qui fait le mal.
\VS{5}C'est pourquoi il faut être soumis, non seulement à cause de la punition, mais aussi à cause de la conscience.
\VS{6}Car c'est aussi pour cela que vous payez les impôts, parce que les magistrats sont les ministres de Dieu, s'employant à rendre la justice.
\VS{7}Rendez donc à tous ce qui leur est dû : L’impôt à qui vous devez l’impôt, le tribut à qui vous devez le tribut, le péage à qui vous devez le péage, la crainte à qui vous devez la crainte, l’honneur à qui vous devez l'honneur.
\TextTitle{L'amour de son prochain : accomplissement de la loi\FTNTT{cp. Lu. 10:29-37}}
\VS{8}Ne devez rien à personne, si ce n’est de vous aimer les uns les autres ; car celui qui aime les autres a accompli la loi.
\VS{9}En effet, les commandements : Tu ne commettras point d’adultère, tu ne tueras point, tu ne déroberas point, tu ne convoiteras point, et ceux qu’il peut encore y avoir, se résument dans cette parole : Tu aimeras ton prochain comme toi-même\FTNT{Ex. 20:12-17 ; Mt. 22:39.}.
\VS{10}La charité ne fait point de mal au prochain ; la charité est donc l'accomplissement de la loi.
\VS{11}Cela importe d’autant plus que vous savez en quelle saison nous sommes ; parce qu'il est déjà l’heure de nous réveiller du sommeil ; car maintenant le salut est plus près de nous que lorsque nous avons cru.
\VS{12}La nuit est avancée\FTNT{Mt. 25:1-13.} et le jour approche. Rejetons donc les œuvres des ténèbres, et soyons revêtus des armes de lumière.
\VS{13}Marchons honnêtement, comme en plein jour, loin des orgies\FTNT{Orgies : Du grec «~komos~». Ce terme désigne la procession nocturne et rituelle, qui avait lieu après un souper, de gens à moitié ivres, à l'esprit folâtre, qui défilaient à travers les rues avec torches et musique en l'honneur de Bacchus ou quelque autre divinité, et chantaient et jouaient devant les maisons de leurs amis, hommes ou femmes. Ce mot est aussi utilisé pour les fêtes et beuveries de nuit qui se terminaient en orgies.} et de l’ivrognerie, de la luxure et de la débauche, des querelles et des jalousies.
\VS{14}Mais revêtez-vous du Seigneur Jésus-Christ, et n'ayez point soin de la chair pour en satisfaire les convoitises.
\TextTitle{[L'attitude du chrétien face aux opinions différentes]
\\(cp. 1 Co. 8:1-10:33}
\Chap{14}
\VerseOne{}Or quant à celui qui est faible dans la foi, recevez-le, et n'ayez point avec lui des discussions sur les opinions.
\VS{2}L'un croit qu'on peut manger de tout, et l'autre, qui est faible, mange des légumes.
\VS{3}Que celui qui mange de tout ne méprise pas celui qui n'en mange point ; et que celui qui n'en mange point, ne juge point celui qui en mange, car Dieu l'a accueilli.
\VS{4}Qui es-tu, toi qui juges le serviteur d'autrui ? S’il se tient ferme ou s'il tombe, c’est à son maître de le juger ; mais il sera affermi, car Dieu est Puissant pour l'affermir.
\VS{5}Tel fait une distinction entre les jours, tel autre les estime tous égaux. Que chacun ait en son esprit une pleine conviction.
\VS{6}Celui qui distingue entre les jours agit ainsi pour le Seigneur. Celui qui mange, c’est pour le Seigneur qu’il mange, car il rend grâces à Dieu ; celui qui ne mange pas, c’est pour le Seigneur qu’il ne mange pas, et il rend grâces à Dieu.
\VS{7}Car nul de nous ne vit pour lui-même, et nul ne meurt pour lui-même.
\VS{8}Car si nous vivons, nous vivons pour le Seigneur ; et si nous mourons, nous mourons pour le Seigneur. Soit donc que nous vivions, soit que nous mourions, nous sommes au Seigneur.
\VS{9}Car c'est pour cela que Christ est mort, qu'il est ressuscité, et qu'il a repris la vie, afin de dominer sur les morts et sur les vivants.
\VS{10}Mais toi, pourquoi juges-tu ton frère ? Ou toi, pourquoi méprises-tu ton frère ? Puisque nous comparaîtrons tous devant le tribunal de Christ.
\VS{11}Car il est écrit : Je suis vivant, dit le Seigneur, tout genou fléchira devant moi, et toute langue donnera gloire à Dieu\FTNT{Es. 45:23 ; Ph. 2:10-11.}.
\VS{12} Ainsi, chacun de nous rendra compte à Dieu pour lui-même.
\TextTitle{Se garder d'être une occasion de chute}
\VS{13}Ne nous jugeons donc plus les uns les autres ; mais pensez plutôt à ne rien faire qui soit pour votre frère une pierre d’achoppement ou une occasion de chute.
\VS{14}Je sais, et je suis persuadé par le Seigneur Jésus, que rien n'est souillé en soi, et qu’une chose n’est souillée que par celui qui la croit souillée.
\VS{15}Mais si ton frère est attristé au sujet d’un aliment, tu ne marches plus selon la charité ; ne détruis point, par ton aliment, celui pour qui Christ est mort.
\VS{16}Que votre privilège ne soit pas un sujet de calomnie.
\VS{17}Car le Royaume de Dieu ne consiste ni dans le manger ni dans le boire, mais dans la justice, la paix et la joie par le Saint-Esprit.
\VS{18}Celui qui sert Christ de cette manière est agréable à Dieu et approuvé des hommes.
\VS{19}Recherchons donc ce qui contribue à la paix et à l’édification mutuelle.
\VS{20}Ne détruis pas l’œuvre de Dieu pour un aliment. Il est vrai que toutes choses sont pures, mais il est mal à l’homme, quand il mange, de devenir une pierre d’achoppement.
\VS{21}Il est bien de ne pas manger de viande, de ne pas boire de vin, et de s’abstenir de ce qui peut être pour ton frère une occasion de chute, de scandale ou de faiblesse.
\VS{22}As-tu la foi ? Garde-la devant Dieu. Heureux est celui qui ne se condamne pas lui-même dans ce qu'il approuve.
\VS{23}Mais celui qui a des doutes au sujet de ce qu’il mange est condamné, parce qu’il n’agit pas avec foi. Tout ce que l’on ne fait pas avec foi est un péché.
\Chap{15}
\VerseOne{}Nous devons, nous qui sommes forts, supporter les infirmités des faibles, et ne pas nous complaire en nous-mêmes.
\VS{2}Que chacun de nous plaise au prochain pour ce qui est bien, en vue de l’édification.
\VS{3}Car même Jésus-Christ n'a pas cherché ce qui lui plaisait, mais, selon qu’il est écrit : Les outrages de ceux qui t’insultent sont tombés sur moi\FTNT{Ps. 69:10.}.
\TextTitle{Les Juifs et gentils rachetés par un même salut}
\VS{4}Or tout ce qui a été écrit autrefois, a été écrit pour notre instruction, afin que par la patience et la consolation que donnent les Ecritures, nous possédions l’espérance.
\VS{5}Que le Dieu de patience et de consolation vous donne d’avoir les mêmes sentiments les uns envers les autres, selon Jésus-Christ,
\VS{6}afin que tous d'un même cœur et d'une même bouche, vous glorifiiez Dieu, qui est le Père de notre Seigneur Jésus-Christ.
\VS{7}C'est pourquoi, accueillez-vous les uns les autres, comme Christ nous a accueillis, pour la gloire de Dieu.
\VS{8}Je dis donc que Jésus-Christ a été Ministre des circoncis, pour prouver la vérité de Dieu, afin de confirmer les promesses faites aux pères,
\VS{9}afin que les gentils glorifient Dieu pour sa miséricorde, selon ce qui est écrit : C’est pourquoi je te louerai parmi les nations, et je chanterai à la gloire de ton Nom\FTNT{Ps. 18:50.}. Et il est dit encore :
\VS{10}Nations, réjouissez-vous avec son peuple\FTNT{De. 32:43.} !
\VS{11}Et encore : Louez le Seigneur, vous toutes les nations, et célébrez-le, vous tous les peuples\FTNT{Ps. 117:1.}. Esaïe dit aussi :
\VS{12}Il sortira d’Isaï un rejeton, qui se lèvera pour régner sur les nations ; les nations espéreront en lui\FTNT{Es. 11:1 ; Es. 11:10.}.
\VS{13}Que le Dieu de l’espérance vous remplisse de toute joie et de toute paix, dans la foi, afin que vous abondiez en espérance par la puissance du Saint-Esprit.
\TextTitle{Paul envisage d'aller à Jérusalem, à Rome et en Espagne}
\VS{14}Pour moi, mes frères, je suis persuadé que vous êtes pleins de bonté, remplis de toute connaissance, et capables de vous exhorter les uns les autres.
\VS{15}Cependant, mes frères, je vous ai écrit en quelque sorte plus librement, comme pour réveiller vos souvenirs, à cause de la grâce que Dieu m’a faite,
\VS{16}d’être ministre de Jésus Christ parmi les gentils ; je m’acquitte du divin service de l'Evangile de Dieu, afin que les gentils lui soient une offrande agréable, étant sanctifiée par le Saint-Esprit.
\VS{17}J'ai donc sujet de me glorifier en Jésus-Christ pour ce qui regarde les choses de Dieu.
\VS{18}Car je n’oserais parler de quoi que ce soit que Christ n’ait opéré par moi, pour amener les gentils à son obéissance, par la parole et par les œuvres,
\VS{19}par la puissance des prodiges et des miracles, par la puissance de l'Esprit de Dieu. Ainsi, depuis Jérusalem et les pays voisins jusqu’en Illyrie, j’ai abondamment répandu l’Evangile de Christ.
\VS{20}M'attachant ainsi avec affection à annoncer l’Evangile là où Christ n’avait point encore été prêché, afin que je ne bâtisse pas sur le fondement qu'un autre a déjà posé. 
\VS{21}Mais selon qu'il est écrit : Ceux à qui il n'a point été annoncé le verront ; et ceux qui n'en avaient point entendu parler l’entendront\FTNT{Es. 52:15.}.
\VS{22}Et c’est ce qui m'a souvent empêché d’aller vous voir.
\VS{23}Mais maintenant, n’ayant plus rien qui me retienne dans ces contrées, et ayant depuis plusieurs années le désir d’aller vers vous,
\VS{24}j’espère vous voir en passant, quand je me rendrai en Espagne, et y être accompagné par vous, après que j’aurai satisfait en partie mon désir de me trouver chez vous.
\VS{25}Maintenant je vais à Jérusalem pour assister les saints.
\VS{26}Car il a semblé bon à ceux de Macédoine et d’Achaïe de s’imposer une contribution pour les pauvres parmi les saints de Jérusalem.
\VS{27}Ils l’ont bien voulu, et ils le leur devaient, car si les gentils ont eu part à leurs avantages spirituels, ils doivent aussi les assister dans les choses temporelles.
\VS{28}Dès que j'aurai achevé cette affaire, et que je leur aurai remis ce fruit, j'irai en Espagne en passant par vos quartiers.
\VS{29}Et je sais qu’en allant vers vous, j’irai avec une pleine bénédiction de l'Evangile de Christ.
\VS{30}Je vous exhorte, mes frères, par notre Seigneur Jésus-Christ, et par la charité de l'Esprit, à combattre avec moi en adressant des prières à Dieu en ma faveur,
\VS{31}afin que je sois délivré des incrédules de Judée, et que mon ministère\FTNT{Ministère : Du grec «~diakonia~», terme qui désigne le service ou le ministère de ceux qui répondent aux besoins des autres. Ce vocable fait aussi allusion à l’office des diacres.} à Jérusalem soit agréable aux saints,
\VS{32}en sorte que, par la volonté de Dieu, j’arrive chez vous avec joie, et que je me repose avec vous.
\VS{33}Que le Dieu de paix soit avec vous tous. Amen !
\Chap{16}
\TextTitle{Salutations personnelles de Paul}
\VerseOne{}Je vous recommande notre sœur Phœbé, qui est diaconesse de l'église de Cenchrées,
\VS{2}afin que vous la receviez selon le Seigneur, comme il faut recevoir les saints, et que vous l'assistiez dans tout ce dont elle aura besoin ; car elle a exercé l'hospitalité à l'égard de plusieurs, et même à mon égard.
\VS{3}Saluez Priscille et Aquilas, mes compagnons d’œuvre en Jésus-Christ,
\VS{4}qui ont exposé leur cou pour ma vie ; ce n’est pas moi seul qui leur rends grâces, mais aussi toutes les églises des gentils.
\VS{5}Saluez aussi l'église qui est dans leur maison. Saluez Epaïnète, mon bien-aimé, qui a été pour Christ les prémices d'Achaïe.
\VS{6}Saluez Marie, qui a beaucoup travaillé pour nous.
\VS{7}Saluez Andronicus et Junias, mes parents, qui ont été prisonniers avec moi, et qui sont distingués parmi les apôtres, et qui ont même été en Christ avant moi.
\VS{8}Saluez Amplias, mon bien-aimé dans le Seigneur.
\VS{9}Saluez Urbain, notre compagnon d’œuvre en Christ, et Stachys, mon bien-aimé.
\VS{10}Saluez Apellès, qui est éprouvé en Christ. Saluez ceux de chez Aristobule.
\VS{11}Saluez Hérodion, mon parent. Saluez ceux de chez Narcisse qui sont dans le Seigneur.
\VS{12}Saluez Tryphène et Tryphose, qui travaillent pour le Seigneur. Saluez Perside, la bien-aimée qui a beaucoup travaillé pour le Seigneur.
\VS{13}Saluez Rufus, l’élu du Seigneur, et sa mère, qui est aussi la mienne.
\VS{14}Saluez Asyncrite, Phlégon, Hermas, Patrobas, Hermès, et les frères qui sont avec eux.
\VS{15}Saluez Philologue et Julie, Nérée et sa sœur, et Olympe, et tous les saints qui sont avec eux.
\VS{16}Saluez-vous les uns les autres par un saint baiser. Les églises de Christ vous saluent.
\TextTitle{Se garder de ceux qui causent des divisions et des scandales}
\VS{17}Je vous exhorte, mes frères, à prendre garde à ceux qui causent des divisions et des scandales contre la doctrine que vous avez apprise. Eloignez-vous d'eux.
\VS{18}Car de tels hommes ne servent point notre Seigneur Jésus-Christ, mais leur propre ventre, et par des paroles douces et flatteuses, ils séduisent les cœurs des simples.
\VS{19}Pour vous, votre obéissance est connue de tous ; je me réjouis donc à votre sujet, et je désire que vous soyez sages à l’égard du bien, et purs à l’égard du mal.
\VS{20}Le Dieu de paix brisera bientôt Satan sous vos pieds. Que la grâce de notre Seigneur Jésus-Christ soit avec vous. Amen !
\VS{21}Timothée, mon compagnon d’œuvre, vous salue, ainsi que Lucius, et Jason et Sosipater, mes parents.
\VS{22}Je vous salue dans le Seigneur, moi Tertius, qui ai écrit cette lettre.
\VS{23}Gaïus, mon hôte, et celui de toute l'église, vous salue. Eraste, l’économe de la ville, vous salue, et Quartus, notre frère.
\TextTitle{Bénédiction}
\VS{24}Que la grâce de notre Seigneur Jésus-Christ soit avec vous tous. Amen !
\VS{25}or à celui qui est puissant pour vous affermir selon mon Evangile, et selon la prédication de Jésus-Christ, conformément à la révélation du mystère qui a été caché dans les temps passés.
\VS{26}mais manifesté maintenant par les écrits des prophètes, d’après l’ordre du Dieu éternel, et porté à la connaissance de toutes les nations, afin qu’elles obéissent à la foi.
\VS{27}A Dieu, seul sage, soit la gloire éternellement, par Jésus-Christ. Amen !
\PPE{}
\end{multicols}

%\clearpage\ShortTitle{Ep.}\BookTitle{Ephésiens}\BFont
\noindent\hrulefill
{\footnotesize
\textit{
\bigskip
{\centering{}
\\Auteur~: Paul
\\Thème~: L'Eglise, corps de Christ
\\Date de rédaction~: Env. 60 ap. J.-C.\\}
}
\textit{
\\Ephèse figurait parmi les principales villes de l'Empire romain sous le règne de l'empereur Claude Ier (10 av. J.-C. – 54 ap. J.-C.). Bien que Pergame était considérée comme la capitale de l'Asie Mineure, en raison de sa position géographique et grâce à ses affluents, Ephèse possédait le plus grand port de la région, ce qui lui a valu le contrôle du trafic commercial. Richissime et prospère, elle était renommée pour son faste et sa liberté de parole, et constituait donc un endroit privilégié pour les philosophes. C'était une ville où l'activité culturelle tenait une grande place (Jeux olympiques, théâtres, cirques, etc.) et où chacun pouvait y pratiquer la religion de son choix (croyances gréco-romaines, égyptiennes, judaïque etc.).
\\Ephèse, dont le nom signifie «~désirable~», était la gardienne de l'Artémision, temple dédié à la déesse grecque Artémis, la Diane des Ephésiens.
\\L'église d'Ephèse vit le jour lors du second voyage missionnaire de Paul (50-52). Quand il repartit, il laissa à Aquilas et Priscille la charge de la toute jeune assemblée. Paul s'installa à Ephèse lors de son troisième voyage (53-57) et y demeura presque trois ans. Il discourut pendant trois mois dans la synagogue sur le Royaume de Dieu, mais se retrouva confronté à l'endurcissement de certains. C'est alors qu'il se retira pour enseigner dans l'école d'un certain Tyrannus durant deux ans, de sorte que tous ceux qui habitaient l'Asie, Juifs et Grecs, entendirent parler de Jésus-Christ.
\\Plusieurs de ceux qui avaient cru confessèrent leurs péchés et un certain nombre de ceux qui avaient pratiqué la magie allèrent même jusqu'à brûler leurs livres publiquement. C'est ainsi que l'église d'Ephèse croissait en puissance et en force. La prédication de Paul vint troubler le marché fructueux des fabricants d'idoles au point que Démétrius (orfèvre tirant un grand profit de cette industrie) entraina une émeute contre lui. Paul était cependant soutenu par des amis influents~: Les asiarques.
\\Rédigée en prison, cette épître a pour vocation d'enseigner les chrétiens d'Ephèse sur la manière dont il convient de vivre les uns avec les autres au sein de l'Eglise, corps du Christ.\bigskip
}
}
\par\nobreak\noindent\hrulefill
\begin{multicols}{2}
\Chap{1}
\TextTitle{Introduction}
\VerseOne{}Paul, apôtre de Jésus-Christ par la volonté de Dieu, aux saints et fidèles en Jésus-Christ qui sont à Ephèse~:
\VS{2}Que la grâce et la paix vous soient données par Dieu notre Père, et par le Seigneur Jésus-Christ~!
\TextTitle{La position des élus dans le Royaume de Dieu}
\VS{3}Béni soit Dieu, qui est le Père de notre Seigneur Jésus-Christ, qui nous a bénis de toutes bénédictions spirituelles dans les lieux célestes en Christ~!
\VS{4}Selon qu'il nous a élus en lui avant la fondation du monde, afin que nous soyons saints et irrépréhensibles devant lui dans la charité, 
\VS{5}nous ayant prédestinés pour nous adopter pour lui par Jésus-Christ, selon le bon plaisir de sa volonté,
\VS{6}à la louange de la gloire de sa grâce, par laquelle il nous a rendus agréables en son bien-aimé.
\VS{7}En lui nous avons la rédemption par son sang, à savoir la rémission des offenses, selon les richesses de sa grâce,
\VS{8}qu'il a fait abonder sur nous en toute sagesse et intelligence,
\VS{9}nous ayant donné à connaître le mystère de sa volonté, qu'il avait premièrement arrêté en lui-même,
\VS{10}afin que dans l'accomplissement des temps qu'il avait réglés, il réunit tout en Christ, tant ce qui est dans les cieux, que ce qui est sur la terre, en lui-même. 
\VS{11}En qui nous sommes aussi devenus héritiers, ayant été prédestinés, suivant la résolution de celui qui accomplit toutes choses avec efficacité selon le conseil de sa volonté,
\VS{12}afin que nous soyons à la louange de sa gloire, nous qui avons les premiers espéré en Christ.
\VS{13}En qui vous êtes aussi, ayant entendu la parole de la vérité, qui est l'Evangile de votre salut, et auquel ayant cru, vous avez été scellés du Saint-Esprit qui avait été promis,
\VS{14}lequel est le gage de notre héritage jusqu'à la rédemption de ceux qu'il s'est acquis à la louange de sa gloire.
\VS{15}C'est pourquoi, ayant aussi entendu parler de la foi que vous avez en notre Seigneur Jésus, et de la charité que vous avez envers tous les saints,
\VS{16}je ne cesse de rendre grâces pour vous dans mes prières,
\VS{17}afin que le Dieu de notre Seigneur Jésus-Christ, le Père de gloire, vous donne l'Esprit de sagesse et de révélation, dans ce qui regarde sa connaissance.
\VS{18}Qu'il illumine les yeux de votre esprit, afin que vous sachiez quelle est l'espérance de sa vocation, et quelles sont les richesses de la gloire de son héritage qu'il réserve aux saints,
\VS{19}et quelle est l'excellente grandeur de sa puissance envers nous qui croyons selon l'efficacité de la puissance de sa force, 
\VS{20}qu'il a déployée avec efficacité en Christ, quand il l'a ressuscité des morts et qu'il l'a fait asseoir à sa droite dans les lieux célestes,
\VS{21}au-dessus de toute principauté, de toute puissance, de toute dignité et de toute domination, et au-dessus de tout nom qui se nomme, non seulement dans le siècle présent, mais aussi dans celui qui est à venir.
\TextTitle{Le Messie est le Chef suprême de l'Eglise}
\VS{22}Et il a assujetti toutes choses sous ses pieds, et l'a établi sur toutes choses pour être le Chef de l'Eglise,
\VS{23}qui est son corps, et la plénitude de celui qui remplit tout en tous.
\Chap{2}
\TextTitle{Le salut par la grâce}
\VerseOne{}Et vous étiez morts par vos offenses et par vos péchés,
\VS{2}dans lesquels vous marchiez autrefois, suivant le train de ce monde, selon le prince de la puissance de l'air, qui est l'esprit qui agit maintenant avec efficacité dans les fils rebelles à Dieu,
\VS{3}parmi lesquels nous vivions tous autrefois, selon les convoitises de notre chair, accomplissant les désirs de la chair et de nos pensées. Et nous étions par nature des enfants de colère comme les autres.
\VS{4}Mais Dieu, qui est riche en miséricorde, à cause de sa grande charité dont il nous a aimés,
\VS{5}lorsque nous étions morts dans nos offenses, il nous a vivifiés ensemble avec Christ~; c'est par grâce que vous êtes sauvés.
\VS{6}Et il nous a ressuscités ensemble, et nous a fait asseoir ensemble dans les lieux célestes en Jésus-Christ,
\VS{7}afin qu'il montre dans les siècles à venir les immenses richesses de sa grâce par sa bonté envers nous, en Jésus-Christ.
\VS{8}Car vous êtes sauvés par la grâce, par la foi~; et cela ne vient pas de vous, c'est le don de Dieu~;
\VS{9}non pas par les œuvres, afin que personne ne se glorifie.
\VS{10}Car nous sommes son ouvrage, ayant été créés en Jésus-Christ pour les bonnes œuvres que Dieu a préparées d'avance, afin que nous marchions en elles.
\VS{11}C'est pourquoi, souvenez-vous que vous qui étiez autrefois Gentils dans la chair, et qui étiez appelés incirconcis par ceux qu'on appelle circoncis, et qui le sont dans la chair par la main des hommes, 
\VS{12}vous étiez en ce temps-là sans Christ, privés du droit de cité en Israël, étant étrangers des alliances de la promesse, n'ayant pas d'espérance, et étant sans Dieu dans le monde.
\VS{13}Mais maintenant, par Jésus-Christ, vous qui étiez autrefois éloignés, vous avez été rapprochés par le sang de Christ.
\TextTitle{Juifs et Gentils forment un seul corps}
\VS{14}Car il est notre paix, lui qui des deux n'en a fait qu'un en détruisant le mur de séparation,
\VS{15}ayant aboli dans sa chair l'inimitié, à savoir la loi des commandements qui consiste en ordonnances, afin de créer les deux en lui-même pour être un homme nouveau, en faisant la paix~;
\VS{16}et de réconcilier les uns et les autres avec Dieu pour former un seul corps par sa croix, ayant détruit par elle l'inimitié.
\VS{17}Et il est venu prêcher la paix à vous qui étiez loin, et à ceux qui étaient près,
\VS{18}car nous avons par lui les uns et les autres accès auprès du Père dans un même Esprit.
\TextTitle{L'Eglise véritable}
\VS{19}C'est pourquoi vous n'êtes plus des étrangers ni des gens de dehors, mais concitoyens des saints et gens de la maison de Dieu~;
\VS{20}étant édifiés sur le fondement\FTNT{Le fondement a été posé une fois pour toutes par les apôtres et les prophètes. Et ce fondement est notre Seigneur Jésus-Christ (1 Co. 3:11).} des apôtres et des prophètes, et Jésus-Christ lui-même étant la pierre angulaire~;
\VS{21}en qui tout l'édifice, bien ajusté ensemble, s'élève pour être un temple saint dans le Seigneur,
\VS{22}en qui vous êtes édifiés ensemble, pour être une habitation de Dieu en Esprit.
\Chap{3}
\TextTitle{Le mystère caché de tout temps\FTNTT{Col. 1:24-27.}}
\VerseOne{}C'est pour cela que moi, Paul, je suis prisonnier de Jésus-Christ pour vous Gentils.
\VS{2}Si toutefois vous avez entendu quelle est la gestion de la grâce de Dieu qui m'a été donnée pour vous,
\VS{3}comment par révélation ce mystère m'a été manifesté, ainsi que je l'ai écrit ci-dessus en peu de mots~;
\VS{4}d'où vous pouvez voir en le lisant, quelle est l'intelligence que j'ai du mystère de Christ,
\VS{5}lequel n'a pas été manifesté aux fils des hommes dans les autres générations, comme il a été révélé maintenant par l'Esprit à ses saints apôtres et à ses prophètes,
\VS{6}à savoir que les Gentils sont cohéritiers et d'un même corps, et qu'ils participent ensemble à sa promesse en Christ par l'Evangile,
\VS{7}dont j'ai été fait serviteur, selon le don de la grâce de Dieu qui m'a été donnée selon l'efficacité de sa puissance.
\VS{8}Cette grâce, dis-je, m'a été donnée à moi, qui suis le moindre de tous les saints, pour annoncer parmi les Gentils les richesses incompréhensibles de Christ,
\VS{9}et pour mettre en évidence devant tous quelle est la communion qui nous a été accordée du mystère qui était caché de tout temps en Dieu, lequel a créé toutes choses par Jésus-Christ,
\VS{10}afin que les principautés et les puissances dans les lieux célestes connaissent aujourd'hui par l'Eglise la sagesse infiniment variée de Dieu,
\VS{11}suivant le dessein arrêté dès les siècles, qu'il a établi en Jésus-Christ, notre Seigneur,
\VS{12}par lequel nous avons hardiesse et accès avec confiance, par la foi que nous avons en lui.
\VS{13}C'est pourquoi, je vous prie de ne pas vous relâcher à cause de mes afflictions que je souffre pour l'amour de vous, ce qui est votre gloire.
\VS{14}A cause de cela, je fléchis mes genoux devant le Père de notre Seigneur Jésus-Christ,
\VS{15}duquel toute parenté est nommée dans les cieux et sur la terre,
\VS{16}afin que selon les richesses de sa gloire, il vous donne d'être puissamment fortifiés par son Esprit dans l'homme intérieur,
\VS{17}en sorte que Christ habite dans vos cœurs par la foi~; afin qu'étant enracinés et fondés dans la charité,
\VS{18}vous puissiez comprendre avec tous les saints quelle est la largeur et la longueur, la profondeur et la hauteur,
\VS{19}et connaître la charité de Christ qui surpasse toute connaissance, afin que vous soyez remplis de toute la plénitude de Dieu.
\VS{20}Or à celui qui par la puissance qui agit en nous avec efficacité, peut faire infiniment au-delà de tout ce que nous demandons et pensons,
\VS{21}à lui soit la gloire dans l'Eglise, en Jésus-Christ, dans toutes les générations, aux siècles des siècles~! Amen~!
\Chap{4}
\TextTitle{L'unité}
\VerseOne{}Je vous prie donc, moi, le prisonnier dans le Seigneur, à marcher d'une manière digne de la vocation à laquelle vous êtes appelés,
\VS{2}avec toute humilité et douceur, avec patience, vous supportant les uns les autres dans la charité,
\VS{3}vous efforçant de garder l'unité de l'Esprit par le lien de la paix.
\VS{4}Il y a un seul corps, un seul Esprit, comme aussi vous êtes appelés à une seule espérance par votre vocation~;
\VS{5}il y a un seul Seigneur, une seule foi, un seul baptême,
\VS{6}un seul Dieu et Père de tous, qui est au-dessus de tous, parmi tous, et en vous tous.
\TextTitle{Les dons de Christ pour le perfectionnement et l'édification de son Corps\FTNTT{1 Co. 12:4-11.}}
\VS{7}Mais la grâce est donnée à chacun de nous selon la mesure du don de Christ.
\VS{8}C'est pourquoi il est dit~: Etant monté en haut, il a emmené captive une grande multitude de captifs, et il a donné des dons aux hommes\FTNT{Ps. 68:19.}.
\VS{9}Or que signifie~: Il est monté, sinon qu'il est premièrement descendu dans les parties les plus basses de la terre~?
\VS{10}Celui qui est descendu, c'est le même qui est monté au-dessus de tous les cieux, afin de remplir toutes choses.
\VS{11}Lui-même donc a donné les uns pour être apôtres, les autres pour être prophètes, les autres pour être évangélistes, les autres pour être pasteurs et docteurs,
\VS{12}pour travailler au perfectionnement\FTNT{Le mot «~perfectionnement~» vient du grec «~katartismos~», qui tire son origine du terme «~katartizo~»~: «~redresser, ajuster, compléter, raccommoder (ce qui a été abîmé), réparer~». Ainsi, les divers services ont vocation, d'une part, à réparer les dégâts causés par le péché dans les âmes, et d'autre part, à préparer les disciples à rentrer à leur tour dans leur propre service.} des saints, pour l'œuvre du service\FTNT{Le mot «~service~» vient du grec «~diakonos~», il signifie «~service, ministère de ceux qui répondent aux besoins des autres~». Jésus-Christ lui-même a pris la forme d'un serviteur pour nous servir et non pour être servi (Mt. 20:28).}, pour l'édification du corps de Christ,
\VS{13}jusqu'à ce que nous soyons tous parvenus à l'unité de la foi et de la connaissance du Fils de Dieu, à l'état d'homme parfait, à la mesure de la parfaite stature de Christ,
\VS{14}afin que nous ne soyons plus des enfants flottants et emportés çà et là à tous vents de doctrine, par la tromperie des hommes et par leur ruse à séduire artificieusement.
\VS{15}Mais afin que, suivant la vérité avec la charité, nous croissions en toutes choses en celui qui est le Chef, c'est-à-dire Christ,
\VS{16}dont tout le corps bien ajusté et lié ensemble par toutes les jointures de son assistance, tire son accroissement selon la force qu'il distribue à chaque membre, afin qu'il soit édifié dans la charité.
\TextTitle{Se dépouiller du vieil homme}
\VS{17}Je vous dis donc, et je vous conjure de la part du Seigneur, de ne plus vous conduire comme le reste des Gentils qui suivent la vanité de leurs pensées.
\VS{18}Ils ont l'intelligence obscurcie par les ténèbres et sont étrangers à la vie de Dieu, à cause de l'ignorance qui est en eux, par l'endurcissement de leur cœur.
\VS{19}Ils ont perdu tout sentiment, et se sont abandonnés à la dissolution pour commettre toute sorte d'impureté avec cupidité.
\VS{20}Mais vous n'avez pas ainsi appris Christ,
\VS{21}si toutefois vous l'avez entendu, et si vous avez été enseignés par lui~, selon que la vérité est en Jésus~; 
\VS{22}à savoir que vous dépouilliez le vieil homme, pour ce qui est de votre conduite précédente, qui se corrompt par les convoitises qui séduisent~; 
\VS{23}et que vous soyez renouvelés dans l'esprit de votre entendement,
\VS{24}et que vous soyez revêtus du nouvel homme, créé selon Dieu dans une justice et une sainteté véritables.
\VS{25}C'est pourquoi, ayant dépouillé le mensonge, parlez en vérité chacun avec son prochain~; car nous sommes membres les uns des autres.
\VS{26}Si vous vous mettez en colère, ne péchez pas, que le soleil ne se couche pas sur votre colère.
\VS{27}Ne donnez pas lieu au diable de vous perdre.
\VS{28}Que celui qui dérobait ne dérobe plus~; mais plutôt qu'il travaille en faisant de ses mains ce qui est bon, pour avoir de quoi donner à celui qui est dans le besoin.
\VS{29}Qu'aucun discours malhonnête ne sorte de votre bouche, mais seulement celui qui est propre à édifier, afin qu'il soit agréable à ceux qui l'écoutent.
\VS{30}Et n'attristez pas le Saint-Esprit de Dieu, par lequel vous avez été scellés pour le jour de la rédemption.
\VS{31}Que toute amertume, toute colère, toute irritation, toute clameur, toute médisance, et toute malice soient bannies du milieu de vous.
\VS{32}Mais soyez doux les uns envers les autres, pleins de compassion, et vous pardonnant les uns aux autres, ainsi que Dieu vous a pardonné par Christ.
\Chap{5}
\VerseOne{}Soyez donc les imitateurs de Dieu, comme ses enfants bien-aimés~;
\VS{2}et marchez dans la charité, ainsi que Christ nous a aimés et s'est livré lui-même pour nous comme une offrande et un sacrifice de bonne odeur à Dieu.
\VS{3}Que la fornication, ni aucune impureté, ni la cupidité, ne soient pas même nommées parmi vous, ainsi qu'il est convenable à des saints.
\VS{4}Qu'on n'entende ni parole grossière, ni propos insensés, ni plaisanterie, choses qui sont contraires à la bienséance, mais plutôt des actions de grâces.
\VS{5}Car sachez-le bien qu'aucun fornicateur, ni impur, ni cupide, qui est un idolâtre, n'a d'héritage dans le Royaume de Christ et de Dieu.
\VS{6}Que personne ne vous séduise par de vains discours~; car à cause de ces choses la colère de Dieu vient sur les fils de la rébellion.
\VS{7}Ne soyez donc pas leurs associés.
\VS{8}Car vous étiez autrefois ténèbres, mais maintenant vous êtes lumière dans le Seigneur. Conduisez-vous donc comme des enfants de la lumière~!
\VS{9}car le fruit de l'Esprit consiste en toute bonté, justice et vérité,
\VS{10}éprouvant ce qui est agréable au Seigneur~;
\VS{11}et ne participez pas aux œuvres infructueuses des ténèbres, mais au contraire condamnez-les~!
\VS{12}Car il est honteux de dire les choses qu'ils font en secret~;
\VS{13}mais toutes choses, étant mises en évidence par la lumière, sont rendues manifestes, car la lumière est celle qui manifeste tout.
\VS{14}C'est pourquoi il est dit~: Réveille-toi, toi qui dors, et relève-toi d'entre les morts, et Christ t'éclairera\FTNT{Es. 60:1.}.
\VS{15}Prenez donc garde de vous conduire soigneusement, non pas comme étant dépourvus de sagesse, mais comme étant sages,
\VS{16}rachetant le temps, car les jours sont mauvais.
\VS{17}C'est pourquoi ne soyez pas sans intelligence, mais comprenez bien quelle est la volonté du Seigneur.
\VS{18}Et ne vous enivrez pas du vin dans lequel il y a de la dissolution, mais soyez remplis de l'Esprit.
\VS{19}Entretenez-vous par des psaumes, des hymnes et des cantiques spirituels, chantant et psalmodiant de votre cœur au Seigneur~;
\VS{20}rendez toujours grâces pour toutes choses à Dieu notre Père, au Nom de notre Seigneur Jésus-Christ~;
\VS{21}soumettez-vous les uns aux autres dans la crainte de Christ.
\TextTitle{Le mariage selon Dieu}
\VS{22}Femmes, soyez soumises à vos maris comme au Seigneur~;
\VS{23}car le mari est le chef de la femme, comme Christ est le Chef de l'Eglise, qui est son corps, et dont il est le Sauveur.
\VS{24}Or de même que l'Eglise est soumise à Christ, les femmes aussi doivent l'être à leurs maris en toutes choses.
\VS{25}Et vous maris, aimez vos femmes, comme Christ a aimé l'Eglise, et s'est livré lui-même pour elle,
\VS{26}afin de la sanctifier en la purifiant et en la lavant par l'eau de la parole~;
\VS{27}afin de faire paraître devant lui cette Eglise glorieuse, sans tache, ni ride, ni rien de semblable, mais sainte et irréprochable.
\VS{28}C'est ainsi que les maris doivent aimer leurs femmes comme leurs propres corps. Celui qui aime sa femme s'aime lui-même,
\VS{29}car personne n'a jamais eu en haine sa propre chair, mais il la nourrit et l'entretient, comme le Seigneur entretient l'Eglise,
\VS{30}car nous sommes membres de son corps étant de sa chair et de ses os.
\VS{31}C'est pourquoi l'homme quittera son père et sa mère et s'attachera à sa femme, et les deux deviendront une seule chair.
\VS{32}Ce mystère est grand, or je parle de Christ et de l'Eglise.
\VS{33}Que chacun de vous donc aime sa femme comme lui-même, et que la femme respecte son mari.
\Chap{6}
\TextTitle{La famille selon Dieu}
\VerseOne{}Enfants, obéissez à vos pères et à vos mères, dans ce qui est selon le Seigneur, car cela est juste.
\VS{2}Honore ton père et ta mère, c'est le premier commandement avec une promesse,
\VS{3}afin que tout aille bien pour toi et que tu vives longtemps sur la terre.
\VS{4}Et vous, pères, n'irritez pas vos enfants, mais élevez-les\FTNT{Le verbe «~élever~» vient du grec «~ektrepho~» qui signifie nourrir jusqu'à maturité.} en les instruisant et les avertissant selon le Seigneur.
\TextTitle{Les rapports entre les maîtres et les serviteurs selon Dieu}
\VS{5}Serviteurs, obéissez à vos maîtres selon la chair avec crainte et tremblement, dans la simplicité de votre cœur, comme à Christ,
\VS{6}ne les servant pas seulement sous leurs yeux, comme cherchant à plaire aux hommes, mais comme serviteurs de Christ, faisant de bon cœur la volonté de Dieu,
\VS{7}servant avec bienveillance, comme servant le Seigneur et non pas les hommes~;
\VS{8}sachant que chacun, soit esclave, soit libre, recevra du Seigneur le bien qu'il aura fait.
\VS{9}Et vous maîtres, faites envers eux la même chose et renoncez aux menaces, sachant que leur Seigneur et le vôtre est dans les cieux, et qu'il n'y a pas en lui acception de personnes.
\TextTitle{Le combat spirituel}
\VS{10}Au reste, mes frères, fortifiez-vous dans le Seigneur, et dans la puissance de sa force.
\VS{11}Revêtez-vous de toutes les armes de Dieu, afin de pouvoir résister aux embûches du diable.
\VS{12}Car nous n'avons pas à lutter\FTNT{Le mot grec utilisé ici est «~pale~», il était employé pour parler de la lutte entre deux combattants où chacun essaye de renverser l’autre~; la victoire étant acquise par le maintien de l’adversaire au sol, et en lui mettant la main sur la nuque.} contre la chair et le sang, mais contre les principautés, contre les puissances, contre les seigneurs du monde des ténèbres de ce siècle, contre les méchancetés spirituelles qui sont dans les lieux célestes.
\VS{13}C'est pourquoi prenez toutes les armes de Dieu\FTNT{Voir en annexe «~Les armes du chrétien~».}, afin de pouvoir résister dans le mauvais jour, et tenir ferme après avoir tout surmonté.
\VS{14}Soyez donc fermes, ayant à vos reins la vérité pour ceinture, ayant revêtu la cuirasse de la justice~;
\VS{15}et ayant vos pieds chaussés, prêts pour l'Evangile de paix~;
\VS{16}par-dessus tout, prenez le bouclier de la foi, avec lequel vous pourrez éteindre tous les dards enflammés du malin~;
\VS{17}prenez aussi le casque du salut, et l'épée de l'Esprit, qui est la parole de Dieu~;
\VS{18}priant en votre esprit par toutes sortes de prières et de supplications en tout temps, veillant à cela avec une entière persévérance, et priant pour tous les saints,
\VS{19}et pour moi aussi, afin qu'il me soit donné de parler en toute liberté et avec hardiesse, pour faire connaître le mystère de l'Evangile,
\VS{20}pour lequel je suis ambassadeur quoique chargé de chaînes, afin, dis-je, que je parle librement, ainsi qu'il faut que je parle.
\TextTitle{Salutations}
\VS{21}Or afin que vous aussi vous sachiez ce qui me concerne et ce que je fais, Tychique, notre frère bien-aimé et fidèle serviteur du Seigneur, vous fera tout savoir.
\VS{22}Je l'envoie exprès vers vous, afin que vous connaissiez notre situation, et pour qu'il console vos cœurs.
\VS{23}Que la paix soit avec les frères, et la charité avec la foi, de la part de Dieu le Père, et du Seigneur Jésus-Christ~!
\VS{24}Que la grâce soit avec tous ceux qui aiment notre Seigneur Jésus-Christ dans l'incorruptibilité~! Amen~!
\PPE{}
\end{multicols}

%\clearpage\ShortTitle{Ph.}\BookTitle{Philippiens}\BFont
\noindent\hrulefill
{\footnotesize
\textit{
\bigskip
{\centering{}
\\Auteur~: Paul
\\Thème~: Expérience chrétienne
\\Date de rédaction~: Env. 60 ap. J.-C.\\}
}
\textit{
\\Fondée par Philippe II (382 av. J.-C. – 336 av. J.-C.) en 356 av. J.-C., Philippes est une ville grecque de Macédoine orientale. Située sur une voie romaine qui traversait les Balkans (Via Egnatia), elle est restée de taille modeste en dépit de son fort taux de fréquentation.
\\La première mention de l'assemblée de Philippes se trouve dans Actes 16, lors de la rencontre de Paul avec des femmes réunies à l'extérieur de la ville pour la prière. Au travers des paroles de Paul, le Seigneur toucha particulièrement Lydie qui, après avoir été baptisée avec sa famille, reçut Paul et ses compagnons dans sa maison.
\\C'est à Rome, sous le règne de Néron (37-68), que Paul, alors captif, rédigea cette lettre. Cet écrit de l'apôtre accusait réception d'un don monétaire que l'église de Philippes lui avait fait parvenir par le biais d'Epaphrodite. Paul y exprimait sa joie en dépit des souffrances et invitait les Philippiens à faire de même. Loin des erreurs doctrinales reprochées à d'autres, ces chrétiens recevaient ainsi l'expression de l'affection de Paul et ses encouragements à persévérer dans la foi en Christ en toutes circonstances.\bigskip
}
}
\par\nobreak\noindent\hrulefill
\begin{multicols}{2}
\Chap{1}
\TextTitle{Introduction}
\VerseOne{}Paul et Timothée, serviteurs de Jésus-Christ, à tous les saints en Jésus-Christ qui sont à Philippes, avec les évêques et les diacres~:
\VS{2}Que la grâce et la paix vous soient données de la part de Dieu, notre Père et du Seigneur Jésus-Christ~!
\VS{3}Je rends grâces à mon Dieu toutes les fois que je fais mention de vous,
\VS{4}en priant toujours pour vous tous avec joie dans toutes mes prières,
\VS{5}à cause de votre attachement à l'Evangile, depuis le premier jour jusqu'à maintenant.
\VS{6}Etant persuadé de cela même, que celui qui a commencé cette bonne œuvre en vous, l'achèvera jusqu'au jour de Jésus-Christ.
\VS{7}Comme il est juste que je pense ainsi de vous tous, parce que je retiens dans mon cœur, que vous avez tous été participants de la grâce avec moi dans mes liens, et dans la défense et la confirmation de l'Evangile.
\VS{8}Car Dieu m'est témoin que je vous aime tous tendrement, conformément à la charité de Jésus-Christ.
\VS{9}Et je lui demande cette grâce~: Que votre charité abonde encore de plus en plus avec connaissance et toute intelligence,
\VS{10}pour le discernement des choses contraires, afin que vous soyez purs et irréprochables pour le jour de Christ,
\VS{11}étant remplis de fruits de justice, qui sont par Jésus-Christ, à la gloire et à la louange de Dieu.
\TextTitle{Les chrétiens encouragés par la souffrance de Paul}
\VS{12}Or, mes frères, je veux bien que vous sachiez que les choses qui me sont arrivées, sont arrivées pour un plus grand avancement de l'Evangile.
\VS{13}De sorte que mes liens en Christ ont été rendus célèbres dans tout le Prétoire, et partout ailleurs. 
\VS{14}Et que plusieurs de nos frères en notre Seigneur, étant rassurés par mes liens, osent annoncer la parole plus hardiment, et sans crainte. 
\VS{15}Il est vrai que quelques-uns prêchent Christ par envie et par un esprit de dispute~; et que les autres le font, au contraire, par une bonne volonté. 
\VS{16}Les uns, dis-je, annoncent Christ par un esprit de dispute, et non pas purement, croyant ajouter de l'affliction à mes liens.
\VS{17}Mais les autres le font par charité, sachant que je suis établi pour la défense de l'Evangile\FTNT{Les versets 16 et 17 sont inversés dans les versions Segond, Darby et TOB notamment. Ces bibles sont basées sur les textes minoritaires, moins précis. Les versions Martin, Ostervald et King James, basées sur le texte majoritaire (byzantin), utilisent bien cet ordre des versets que l'on retrouvent ainsi dans les écrits grecs.}.
\VS{18}Quoi donc~? Toutefois, de toute manière, que ce soit par ostentation, ou par amour de la vérité, Christ n'est pas moins annoncé. Je m'en réjouis, et je m'en réjouirai encore.
\VS{19}Car je sais que cela tournera à mon salut par vos prières et par le secours de l'Esprit de Jésus-Christ,
\VS{20}selon ma ferme attente et mon espérance, je ne serai confus en rien, mais qu'en toute assurance, Christ sera maintenant, comme il l'a toujours été glorifié dans mon corps, soit par ma vie, soit par ma mort.
\VS{21}Car Christ est ma vie, et la mort m'est un gain.
\VS{22}Mais s'il est utile pour mon œuvre de vivre dans la chair, ce que je dois choisir, je n'en sais rien.
\VS{23}Car je suis pressé des deux côtés~: Mon désir tendant bien à déloger, et à être avec Christ, ce qui me serait beaucoup meilleur. 
\VS{24}Mais il est plus nécessaire pour vous que je demeure dans la chair.
\VS{25}Et je suis persuadé, je sais que je demeurerai et que je resterai avec vous tous, pour votre avancement et pour votre joie dans la foi.
\VS{26}Afin que vous ayez en moi un sujet de vous glorifier de plus en plus en Jésus-Christ, par mon retour au milieu de vous. 
\VS{27}Seulement, conduisez-vous dignement comme il est séant selon l'Evangile de Christ~; afin que, soit que je vienne, et que je vous voie~; soit que je sois absent, j'entende quant à votre état, que vous persistez dans un même esprit, combattant ensemble d'un même courage par la foi de l'Evangile, et n'étant en rien épouvantés par les adversaires.
\VS{28}Ce qui est pour eux une preuve de perdition, mais pour vous de salut~; et cela de la part de Dieu.
\TextTitle{Souffrir pour Christ: Une grâce}
\VS{29}Parce qu'il vous a été gratuitement donné dans ce qui a du rapport à Christ, non seulement de croire en lui, mais aussi de souffrir pour lui,
\VS{30}en soutenant le même combat que vous m'avez vu soutenir, et que vous apprenez maintenant que je soutiens encore.
\Chap{2}
\TextTitle{Exhortation à l'unité}
\VerseOne{}Si donc il y a quelque consolation en Christ, s'il y a quelque soulagement dans la charité, s'il y a quelque communion d'esprit, s'il y a quelques cordiales affections et quelques compassions,
\VS{2}rendez ma joie parfaite, ayant un même sentiment, un même amour, une même âme, et consentant tous à une même chose.
\VS{3}Ne faites rien par esprit de parti\FTNT{Parti~: «~eritheia~» en grec. Avant la Nouvelle Alliance, ce mot ne se trouve que dans les écrits d'Aristote (philosophe grec, disciple de Platon, né en 384 et mort en 322 av. J.-C) où il dénote une «~recherche personnelle, la poursuite d'une fonction politique par des moyens injustes~». Ce mot signifie aussi «~faire une campagne électorale~» ou «~intriguer pour une fonction dans un esprit partisan, querelleur~». On retrouve le mot grec dans Ja. 3:14,16~; Ga. 5:20~; Ro. 2:8~; Ph. 1:16}, ou par vaine gloire~; mais que l'humilité de cœur vous fasse regarder les autres comme étant au-dessus de vous-mêmes.
\VS{4}Ne regardez point chacun, à votre intérêt particulier, mais que chacun ait égard aussi à ce qui concerne les autres.
\TextTitle{L'humilité de Christ}
\VS{5}Qu'il y ait donc en vous un même sentiment qui a été en Jésus-Christ. 
\VS{6}Lequel étant en forme de Dieu, n'a point regardé son égalité avec Dieu comme une usurpation.
\VS{7}Cependant il s'est vidé lui-même, ayant pris la forme de serviteur, fait à la ressemblance des hommes.
\VS{8}Et, étant trouvé en apparence comme un homme, il s'est abaissé lui-même, en se rendant obéissant jusqu'à la mort, même jusqu'à la mort de la croix. 
\VS{9}C'est pourquoi aussi Dieu l'a souverainement élevé, et lui a donné le Nom qui est au-dessus de tout nom~;
\VS{10}afin qu'au Nom de Jésus, tout genou fléchisse, tant de ceux qui sont dans les cieux, que de ceux qui sont sur la terre, et sous la terre,
\VS{11}et que toute langue confesse que Jésus-Christ est le Seigneur, à la gloire de Dieu le Père.
\VS{12}C'est pourquoi, mes bien-aimés, comme vous avez toujours obéi, mettez en œuvre votre propre salut avec crainte et tremblement, non seulement comme en ma présence, mais beaucoup plus maintenant que je suis absent.
\VS{13}Car c'est Dieu qui produit en vous avec efficacité le vouloir et le faire, selon son bon plaisir.
\VS{14}Faites toutes choses sans murmures et sans disputes,
\VS{15}afin que vous soyez sans reproche, et purs, des enfants de Dieu, irrépréhensibles au milieu de la génération corrompue et perverse, parmi lesquels vous brillez comme des flambeaux dans le monde, qui portent au devant d'eux la parole de la vie. 
\VS{16}Pour me glorifier au jour de Christ de n'avoir point couru en vain, ni travaillé en vain. 
\VS{17}Et même si je sers de libation sur le sacrifice et sur le service de votre foi, je m'en réjouis, et je m'en réjouis avec vous tous.
\VS{18}Vous aussi pareillement, réjouissez-vous, et réjouissez-vous avec moi.
\TextTitle{Paul témoigne de Timothée et d'Epaphrodite}
\VS{19}Or j'espère avec la grâce du Seigneur Jésus, vous envoyer bientôt Timothée afin que j'aie aussi plus de courage quand j'aurai connu votre état. 
\VS{20}Car je n'ai personne d'un pareil courage, et qui soit vraiment soigneux de ce qui vous concerne,
\VS{21}parce que tous cherchent leur intérêt particulier, et non les intérêts de Jésus-Christ. 
\VS{22}Mais vous savez l'épreuve que j'ai faite de lui, puisqu'il a servi avec moi en l'Evangile, comme l'enfant sert son père.
\VS{23}J'espère donc vous l'envoyer dès que j'aurai pourvu à mes affaires.
\VS{24}Et j'ai cette confiance en notre Seigneur que moi-même aussi j'irai bientôt.
\VS{25}Mais j'ai cru nécessaire de vous envoyer Epaphrodite, mon frère, mon compagnon d'œuvre et mon compagnon d'armes, par qui vous m'aviez envoyé de quoi pourvoir à mes besoins.
\VS{26}Car aussi il désirait ardemment vous voir tous, et il était fort affligé de ce que vous aviez appris qu'il avait été malade.
\VS{27}En effet, il a été malade et tout près de la mort~; mais Dieu a eu pitié de lui, et non seulement de lui, mais aussi de moi, afin que je n'aie pas tristesse sur tristesse.
\VS{28}Je l'ai donc envoyé à cause de cela avec plus de soin, afin qu'en le revoyant vous ayez de la joie, et que j'aie moins de tristesse.
\VS{29}Recevez-le donc en notre Seigneur, avec toute sorte de joie~; et ayez de l'estime pour ceux qui sont tels que lui. 
\VS{30}Car il a été proche de la mort pour l'œuvre de Christ, n'ayant eu aucun égard à sa propre vie, afin de suppléer au défaut de votre service envers moi.
\Chap{3}
\TextTitle{Le légalisme et la justice de la loi mosaïque}
\VerseOne{}Au reste, mes frères, réjouissez-vous dans le Seigneur. Je ne me lasse point de vous écrire les mêmes choses, mais pour vous c'est une sécurité.
\VS{2}Prenez garde aux chiens~; prenez garde aux mauvais ouvriers~; prenez garde aux faux circoncis.
\VS{3}Car c'est nous qui sommes les circoncis, qui rendons à Dieu notre culte en Esprit, et qui nous glorifions en Jésus-Christ, et qui n'avons point de confiance en la chair.
\VS{4}Moi aussi, cependant, j'aurais sujet de mettre ma confiance en la chair. Si quelqu'un estime qu'il a de quoi se confier en la chair, je le puis bien davantage~:
\VS{5}Moi, circoncis le huitième jour, de la race d'Israël, de la tribu de Benjamin, Hébreu né d'Hébreux, pharisien en ce qui concerne la loi~;
\VS{6}quant au zèle, persécutant l'Eglise~; et quant à la justice à l'égard de la loi, étant sans reproche.
\TextTitle{Le Messie, objet de notre foi} 
\VS{7}Mais ces choses qui étaient pour moi un gain, je les ai regardées comme une perte à cause de l'amour de Christ.
\VS{8}Et certes, je regarde toutes les autres choses comme m'étant nuisibles en comparaison de l'excellence de la connaissance de Jésus-Christ, mon Seigneur, pour l'amour duquel je me suis privé de toutes ces choses, et je les estime comme du fumier, afin de gagner Christ,
\VS{9}et que je sois trouvé en lui, ayant non pas ma justice qui est de la loi, mais celle qui est par la foi en Christ, c'est-à-dire, la justice qui est de Dieu par la foi.
\VS{10}Ainsi, je connaîtrai Jésus-Christ et la puissance de sa résurrection, et la communion de ses souffrances, en devenant conforme à lui dans sa mort, pour parvenir,
\VS{11}si je puis, à la résurrection d'entre les morts.
\VS{12}Non que j'aie déjà atteint le but, ou que je sois déjà rendu parfait, mais je poursuis ce but pour tâcher d'y parvenir~; c'est pourquoi aussi j'ai été pris par Jésus-Christ.
\VS{13}Mes frères, pour moi, je ne me persuade pas d'avoir atteint le but~;
\VS{14}mais je fais une chose~: Oubliant les choses qui sont en arrière, et me portant vers celles qui sont en avant, je cours vers le but, pour remporter le prix de la vocation céleste de Dieu en Jésus-Christ.
\TextTitle{Paul exhorte les croyants à l'unité}
\VS{15}C'est pourquoi, nous tous qui sommes parfaits, ayons ce même sentiment~; et si vous êtes en quelque point d'un autre avis, Dieu vous le révélera aussi.
\VS{16}Cependant, marchons suivant une même règle pour les choses auxquelles nous sommes parvenus, et ayons un même sentiment.
\VS{17}Soyez tous ensemble mes imitateurs, mes frères, et portez les regards sur ceux qui marchent selon le modèle que vous avez en nous.
\VS{18}Car il y en a plusieurs qui marchent d'une telle manière, que je vous ai souvent dit, et maintenant je vous le dis encore en pleurant, qu'ils sont ennemis de la croix de Christ. 
\VS{19}Eux dont la fin est la perdition, qui ont pour dieu leur ventre, et dont la gloire est dans leur confusion, n'ayant d'affection que pour les choses de la terre.
\TextTitle{Le Messie: Notre espérance}
\VS{20}Mais pour nous, notre cité est dans les cieux, d'où nous aussi nous attendons le Sauveur, le Seigneur Jésus-Christ,
\VS{21}qui transformera notre corps vil, afin qu'il soit rendu conforme à son corps glorieux, selon cette efficacité\FTNT{Ep. 3:7} par laquelle il peut même s'assujettir toutes choses. 
\Chap{4}
\TextTitle{Avoir le même sentiment}
\VerseOne{}C'est pourquoi, mes très chers frères bien-aimés, vous qui êtes ma joie et ma couronne, demeurez ainsi fermes dans le Seigneur, mes bien-aimés.
\VS{2}J'exhorte Evodie, et j'exhorte aussi Syntyche, à être d'un même sentiment dans le Seigneur.
\VS{3}Et toi aussi, mon vrai compagnon\FTNT{Le mot compagnon est la traduction du grec «~suzugos~» qui signifie littéralement: Ensemble sous le joug, compagnon de peine, de joug. Cette expression renvoie à 2 Co. 6:14.}, oui je te prie de les aider, elles qui ont combattu avec moi pour l'Evangile, avec Clément, et mes autres compagnons d'œuvre, dont les noms sont écrits dans le livre de vie.
\VS{4}Réjouissez-vous toujours dans le Seigneur~; je vous le répète, réjouissez-vous~!
\VS{5}Que votre douceur soit connue de tous les hommes. Le Seigneur est proche.
\VS{6}Ne vous inquiétez de rien, mais en toutes choses présentez vos demandes à Dieu par des prières et des supplications, avec des actions de grâces.
\VS{7}Et la paix de Dieu, qui surpasse toute intelligence, gardera vos cœurs et vos sentiments en Jésus-Christ.
\TextTitle{L'objet de nos pensées}
\VS{8} Au reste, mes frères, que toutes les choses qui sont véritables, toutes les choses qui sont vénérables, toutes les choses qui sont justes, toutes les choses qui sont pures, toutes les choses qui sont aimables, toutes les choses qui sont de bonne renommée, toutes celles où il y a quelque vertu et quelque louange~; pensez à ces choses.
\VS{9}Car vous les avez aussi apprises, reçues, entendues et vues en moi. Faites ces choses, et le Dieu de paix sera avec vous. 
\TextTitle{Dieu soutient ses serviteurs}
\VS{10}Or je me suis fort réjoui en notre Seigneur, de ce qu'à la fin vous avez fait revivre le soin que vous aviez pour moi~; à quoi aussi vous pensiez, mais vous n'en aviez pas l'occasion. 
\VS{11}Je ne dis pas ceci à cause de mes besoins, car, moi, j'ai appris à être content en moi-même dans les circonstances où je me trouve.
\VS{12}Je sais être abaissé, je sais aussi être dans l'abondance~; partout et en toutes choses je suis instruit tant à être rassasié, qu'à avoir faim~; tant à être dans l'abondance, que dans la disette.
\VS{13}Je puis toutes choses en Christ qui me fortifie.
\VS{14}Néanmoins, vous avez bien fait de prendre part à mon affliction.
\VS{15}Vous savez aussi, vous Philippiens, qu'au commencement de la prédication de l'Evangile, quand je partis de Macédoine, aucune église ne me communiqua rien en matière de donner et de recevoir, excepté vous seuls. 
\VS{16}Et même lorsque j'étais à Thessalonique, vous m'avez envoyé une fois, et même deux fois, ce dont j'avais besoin. 
\VS{17}Ce n'est pas que je recherche des présents, mais je cherche le fruit qui abonde pour votre compte.
\VS{18}J'ai tout reçu, et je suis dans l'abondance, et j'ai été comblé de biens en recevant d'Epaphrodite ce qui vient de vous, comme un parfum de bonne odeur, comme un sacrifice que Dieu accepte et qui lui est agréable.
\VS{19}Aussi mon Dieu pourvoira à tout ce dont vous aurez besoin selon ses richesses, avec gloire en Jésus-Christ. 
\TextTitle{Salutations}
\VS{20}Or à notre Dieu et Père soit la gloire aux siècles des siècles~! Amen~!
\VS{21}Saluez tous les saints en Jésus-Christ. Les frères qui sont avec moi vous saluent.
\VS{22}Tous les saints vous saluent, et principalement ceux qui sont de la maison de César.
\VS{23}Que la grâce de notre Seigneur Jésus-Christ soit avec vous tous~! Amen~!
\PPE{}
\end{multicols}

%\clearpage\ShortTitle{Col.}\BookTitle{Colossiens}\BFont
\noindent\hrulefill
{\footnotesize
\textit{
\bigskip
{\centering{}
\\Auteur~: Paul
\\Thème~: La prééminence de Christ
\\Date de rédaction~: Env. 60 ap. J.-C.\\}
}
\textit{
\\Située en Asie Mineure, Colosses était une ville de Phrygie qui se trouvait à environ deux cents kilomètres d'Ephèse.
\\Rédigée lors de la première captivité romaine de Paul, la lettre aux Colossiens a pour but de rétablir la suprématie de Christ. En effet, cette église - dont Epaphras, le probable fondateur, s'était converti à Ephèse au cours des trois années que Paul y passa - était sous l'influence d'enseignements séducteurs basés sur le gnosticisme. Cette philosophie à la fois attrayante et très dangereuse prônait entre autres le salut par la connaissance et le dualisme.\bigskip
}
}
\par\nobreak\noindent\hrulefill
\begin{multicols}{2}
\Chap{1}
\TextTitle{Introduction}
\VerseOne{}Paul, apôtre de Jésus-Christ, par la volonté de Dieu, et le frère Timothée~:
\VS{2}Aux saints et frères, fidèles en Christ, qui sont à Colosses, que la grâce et la paix vous soient données de la part de Dieu notre Père et de la part du Seigneur Jésus-Christ~!
\VS{3}Nous rendons grâces à Dieu, qui est le Père de notre Seigneur Jésus-Christ, et nous prions toujours pour vous,
\VS{4}ayant entendu parler de votre foi en Jésus-Christ, et de votre charité envers tous les saints,
\VS{5}à cause de l'espérance des biens qui vous sont réservés dans les cieux, et dont vous avez eu précédemment connaissance par la parole de la vérité, c'est-à-dire par l'Evangile,
\VS{6}qui est parvenu jusqu'à vous, comme il l'est aussi dans le monde entier. Et il porte des fruits, comme aussi parmi vous, depuis le jour où vous avez entendu et connu la grâce de Dieu dans la vérité,
\VS{7}ainsi que vous en avez aussi été instruits par Epaphras, notre cher compagnon de service, qui est pour vous un fidèle serviteur de Christ,
\VS{8}et qui nous a fait connaître votre charité par le Saint-Esprit.
\TextTitle{Prière de Paul pour les Colossiens}
\VS{9}C'est pourquoi depuis le jour où nous l'avons appris, nous ne cessons point de prier pour vous, et de demander à Dieu que vous soyez remplis de la connaissance de sa volonté, en toute sagesse et intelligence spirituelle,
\VS{10}afin que vous vous conduisiez d'une manière digne du Seigneur, pour lui plaire en toutes choses, portant des fruits en toutes sortes de bonnes œuvres, et croissant dans la connaissance de Dieu,
\VS{11}étant fortifiés en toute force, selon la puissance de sa gloire, pour toute patience, et constance, avec joie.
\TextTitle{Le salut de Dieu}
\VS{12}Rendant grâces au Père, qui nous a rendus capables d'avoir part à l'héritage des saints dans la lumière,
\VS{13}qui nous a délivrés de la puissance des ténèbres, et nous a transportés dans le Royaume du Fils de son amour,
\VS{14}en qui nous avons la rédemption par son sang, à savoir la rémission des péchés.
\VS{15}Lequel est l'image de Dieu invisible, le premier-né\FTNT{Dans les Ecritures, l'expression «~premier-né~» est appliquée au Seigneur pour exprimer trois réalités. Tout d'abord, on parle de Jésus en tant que premier-né de Marie, c'est-à-dire son fils aîné (Lu. 2:6-7). Ensuite, on trouve cette expression au sens figuré, pour marquer une distinction (par exemple concernant Israël~; Ex. 4:2) ou désigner la particularité et la suprématie d'une personne. Ainsi, bien que David était le dernier-né de son père Isaï (Ps. 89:28), Dieu en fit un premier-né, «~le plus élevé des rois de la terre~» (Ps. 89:28). Il en va de même pour Jésus-Christ. Il n'est pas le premier-né de la création dans le sens de rang de naissance ou de création, autrement Paul aurait employé le terme grec «~prôtoktisis~» qui signifie «~premier-créé~», au lieu de «~prôtotokos~», c'est-à-dire «~premier-né~». Il faut donc voir dans cette expression un titre de supériorité et d'hiérarchie, pour marquer sa prééminence. En effet, la Parole de Dieu déclare clairement que le Seigneur Jésus-Christ est l'Alpha et le Commencement de toutes choses (Ap. 1:8~; Ap. 21:6~; Ap. 22:13), le Créateur suprême (Ge. 1:1~; Ge. 2:7~; Es. 45:11-18~; Ps. 104:30~; Job. 33:4~; Jn. 1:3~; 1 Co. 8:6~; Col. 1:12-16~; Ap. 22:3~; Ap. 14:6). D'ailleurs il l'a lui-même affirmé sans ambigüité~: «~Avant qu'Abraham fût, Je suis~» (Jn. 8:58). Enfin, Jésus-Christ est aussi appelé le premier-né d'entre les morts (Col. 1:18). Cela ne signifie pas qu'il a été le premier à ressusciter, car il y a eu plusieurs résurrections avant la sienne, mais il fut le premier à ressusciter avec un corps glorieux. Sa résurrection est donc le gage de la promesse de la résurrection de tous ceux qui ont foi en lui (Jn. 3:16).} de toute la création.
\VS{16}Car par lui ont été créées toutes les choses qui sont dans les cieux et sur la terre, les visibles et les invisibles, soit les trônes, ou les dominations, ou les principautés, ou les puissances, toutes choses ont été créées par lui, et pour lui.
\VS{17}Et il est avant toutes choses, et toutes choses subsistent par lui.
\VS{18}Et c'est lui qui est le Chef du corps de l'Eglise, et qui est le commencement et le premier-né d'entre les morts, afin qu'il tienne le premier rang en toutes choses,
\VS{19}car le bon plaisir du Père a été que toute plénitude habitât en lui.
\VS{20}Et de réconcilier par lui toutes choses avec lui même, ayant fait la paix par le sang de sa croix, à savoir, tant les choses qui sont dans les cieux que celles qui sont sur la terre.
\VS{21}Et vous, qui étiez autrefois étrangers, et qui étiez ses ennemis dans votre entendement, et dans les mauvaises œuvres, il vous a maintenant réconciliés 
\VS{22}par le corps de sa chair, par sa mort, pour vous présenter saints, et sans tache, et irrépréhensibles devant lui.
\VS{23}Si toutefois vous demeurez dans la foi, étant fondés et fermes, et n'étant point transportés hors de l'espérance de l'Evangile que vous avez entendu, lequel est prêché à toute créature qui est sous le ciel, dont moi Paul, j'ai été fait le serviteur.
\VS{24}Je me réjouis donc maintenant dans mes souffrances pour vous~; et j'accomplis le reste des afflictions de Christ dans ma chair, pour son corps, qui est l'Eglise.
\VS{25}C'est d'elle que j'ai été fait le serviteur, selon la gestion que Dieu m'a donnée auprès de vous, afin que j'exécute pleinement la parole de Dieu,
\VS{26}à savoir le mystère qui avait été caché dans tous les siècles et dans tous les âges, mais qui est maintenant manifesté à ses saints~;
\VS{27}auxquels Dieu a voulu donner à connaître quelles sont les richesses de la gloire de ce mystère parmi les Gentils, c'est à savoir Christ, qui a été prêché parmi vous, et qui est l'espérance de la gloire~; 
\VS{28}lequel nous annonçons, en exhortant tout homme, et en enseignant tout homme en toute sagesse, afin que nous présentions tout homme parfait en Jésus-Christ.
\TextTitle{Le combat de Paul}
\VS{29}A quoi aussi je travaille, en combattant selon son efficacité\FTNTT{le terme «~efficacité~» vient du grec «~energeia~» qui signifie «~action~», «~fonctionnement~», «~compétence~», ou encore «~force à l'œuvre dans~». Ce mot est utilisé seulement pour parler du pouvoir surhumain que ce soit celui de Dieu ou celui du diable. (Ep. 1:19~; Ph. 3:21~; 2 Ti. 2:9).}, qui agit puissamment en moi.
\Chap{2}
\VerseOne{}Or je veux, en effet, que vous sachiez combien est grand le combat que j'ai pour vous, et pour ceux qui sont à Laodicée, et pour tous ceux qui n'ont pas vu mon visage dans la chair,
\VS{2}afin que leurs cœurs soient consolés, étant unis ensemble dans la charité, et enrichis d'une pleine intelligence, pour la connaissance du mystère de notre Dieu et Père, et de Christ,
\VS{3}en qui sont cachés tous les trésors de la sagesse et de la connaissance.
\TextTitle{Mise en garde contre les discours séduisants et la philosophie\FTNTT{1 Co. 2:4~; Ro. 16:17-18~; 2 Pi. 2:3.}}
\VS{4}Or je dis ceci afin que personne ne vous trompe par des discours séduisants.
\VS{5}Car, quoique je sois absent de corps, toutefois je suis avec vous en esprit, me réjouissant, et voyant votre ordre et la fermeté de votre foi, que vous avez en Christ.
\VS{6}Ainsi, comme vous avez reçu le Seigneur Jésus-Christ, marchez en lui,
\VS{7}étant enracinés et édifiés en lui, et fortifiés en la foi, selon que vous avez été enseignés, abondant en elle avec action de grâces.
\VS{8}Prenez garde que personne ne fasse de vous sa proie par la philosophie, et par de vaines tromperies conformes à la tradition des hommes et aux rudiments du monde, et non point à la doctrine de Christ.
\TextTitle{La divinité du Christ}
\VS{9}Car en lui habite corporellement toute la plénitude de la divinité\FTNT{En Jésus-Christ habite toute la plénitude de la divinité. Il est le Dieu Tout-Puissant.}.
\VS{10}Et vous êtes rendus accomplis en lui, qui est le Chef de toute principauté et puissance.
\TextTitle{L'oeuvre de la croix}
\VS{11}En qui aussi vous êtes circoncis d'une circoncision faite sans main, qui consiste à dépouiller le corps des péchés de la chair, ce qui est la circoncision de Christ.
\VS{12}Etant ensevelis avec lui par le baptême, en qui aussi vous êtes ensemble ressuscités par la foi de l'efficacité de Dieu, qui l'a ressuscité des morts.
\VS{13}Et lorsque vous étiez morts dans vos offenses, et dans l'incirconcision de votre chair, il vous a vivifiés ensemble avec lui, vous ayant gratuitement pardonné toutes vos offenses.
\VS{14}Il a effacé l'acte qui était contre nous, qui consistait en des ordonnances, et qui nous était contraire, et il l'a entièrement aboli en le clouant à la croix.
\VS{15}Il a dépouillé les principautés et les puissances, et les a exposées publiquement en spectacle, en triomphant d'elles par la croix.
\TextTitle{Mise en garde contre les commandements et les doctrines des hommes}
\VS{16}Que personne donc ne vous juge au sujet du manger ou du boire, ou au sujet d'un jour de fête, ou d'un jour de nouvelle lune, ou de sabbat,
\VS{17}qui sont l'ombre des choses qui devaient venir, mais le corps est en Christ.
\VS{18}Que personne ne vous enlève à son gré le prix de la course, sous l'apparence d'humilité d'esprit et par un culte des anges, s'ingérant dans les choses qu'il n'a pas vues, étant témérairement enflé par ses pensées charnelles,
\VS{19}sans s'attacher au Chef, dont tout le corps étant joint et ajusté ensemble par des jointures et des liens, s'accroît d'un accroissement de Dieu.
\VS{20}Si donc vous êtes morts avec Christ quant aux rudiments du monde, pourquoi vous impose-t-on ces ordonnances, comme si vous viviez dans le monde~?
\VS{21}A savoir: Ne prends pas~! Ne goûte pas~! Ne touche pas~!
\VS{22}Lesquelles sont toutes périssables par l'usage, et établies suivant les commandements et les doctrines des hommes~; 
\VS{23}et qui ont pourtant quelque apparence de sagesse en dévotion volontaire, et en humilité d'esprit, et en ce qu'elles n'épargnent pas le corps, et n'ont aucun égard à la satisfaction de la chair.
\Chap{3}
\TextTitle{Rechercher les choses d'en haut}
\VerseOne{}Si donc vous êtes ressuscités avec Christ, cherchez les choses qui sont en haut, où Christ est assis à la droite de Dieu.
\VS{2}Pensez aux choses d'en haut, et non à celles qui sont sur la terre.
\VS{3}Car vous êtes morts, et votre vie est cachée avec Christ en Dieu.
\VS{4}Quand Christ, qui est votre vie, apparaîtra, vous paraîtrez aussi alors avec lui dans la gloire.
\TextTitle{La mort à soi en pratique}
\VS{5}Mortifiez donc vos membres qui sont sur la terre~: La fornication, l'impureté, les passions, les mauvais désirs, et la cupidité, qui est une idolâtrie.
\VS{6}C'est à cause de ces choses que la colère de Dieu vient sur les fils de la rébellion,
\VS{7}parmi lesquels vous marchiez autrefois, quand vous viviez dans ces choses.
\VS{8}Mais maintenant, vous aussi, rejetez toutes ces choses~: La colère, l'animosité, la médisance, et les paroles déshonnêtes qui pourraient sortir de votre bouche.
\VS{9}Ne mentez point les uns aux autres, vous étant dépouillés du vieil homme et de ses œuvres,
\VS{10}et ayant revêtu le nouvel homme, qui se renouvelle dans la connaissance, selon l'image de celui qui l'a créé.
\VS{11}En qui il n'y a ni Grec ni Juif, ni circoncis ni incirconcis, ni barbare ni Scythe, ni esclave ni libre~; mais Christ y est tout et en tous.
\VS{12}Ainsi donc, comme des élus de Dieu, saints et bien-aimés, revêtez-vous des entrailles de miséricorde, de bonté, d'humilité, de douceur, de patience.
\VS{13}Vous supportant les uns les autres, et vous pardonnant les uns aux autres~; et si l'un a querelle contre l'autre, comme Christ vous a pardonné, vous aussi faites-en de même.
\VS{14}Mais par-dessus toutes ces choses, revêtez-vous de la charité, qui est le lien de la perfection.
\VS{15}Et que la paix de Dieu, à laquelle aussi vous êtes appelés pour être un seul corps, tienne le principal lieu dans vos cœurs. Et soyez reconnaissants.
\VS{16}Que la parole de Christ habite en vous abondamment en toute sagesse~; vous enseignant et vous exhortant l'un l'autre par des psaumes, et des hymnes et des cantiques spirituels, avec grâce, chantant de votre cœur au Seigneur.
\VS{17}Et quoi que vous fassiez, en parole ou en œuvre, faites tout au Nom du Seigneur Jésus, rendant grâces par lui à notre Dieu et Père.
\TextTitle{La famille selon Dieu}
\VS{18}Femmes, soyez soumises à vos maris, comme il convient dans le Seigneur\FTNT{Ep. 5:22.}.
\VS{19}Maris, aimez vos femmes, et ne vous aigrissez pas contre elles\FTNT{Ep. 5:25.}.
\VS{20}Enfants, obéissez à vos pères et à vos mères en toutes choses, car cela est agréable au Seigneur\FTNT{Ep. 6:1-2.}.
\VS{21}Pères, n'irritez pas vos enfants\FTNT{Ep. 6:4.}, afin qu'ils ne se découragent pas.
\TextTitle{Les rapports entre serviteurs et maîtres selon Dieu}
\VS{22}Serviteurs, obéissez en toutes choses à ceux qui sont vos maîtres selon la chair, ne servant point seulement sous leurs yeux, comme voulant complaire aux hommes, mais en simplicité de cœur, craignant Dieu\FTNT{Ep. 6:5-6.}.
\VS{23}Et quoi que vous fassiez, faites tout de bon cœur, comme le faisant pour le Seigneur, et non pas pour les hommes,
\VS{24}sachant que vous recevrez du Seigneur l'héritage pour récompense. Car vous servez Christ, le Seigneur.
\VS{25}Mais celui qui agit injustement recevra ce qu'il aura fait injustement, car en Dieu il n'y a point d'égard à l'apparence des personnes.
\Chap{4}
\VerseOne{}Maîtres, accordez à vos serviteurs ce qui est juste et équitable, sachant que vous avez, vous aussi, un Maître dans les cieux.
\TextTitle{La persévérance dans la prière}
\VS{2}Persévérez dans la prière, veillant dans cet exercice avec des actions de grâces.
\VS{3}Priez aussi tous ensemble pour nous, afin que Dieu nous ouvre la porte de la parole, pour annoncer le mystère de Christ pour lequel aussi je suis prisonnier, 
\VS{4}afin que je le fasse connaître comme je dois en parler.
\VS{5}Conduisez-vous sagement envers ceux du dehors, et rachetez le temps.
\VS{6}Que votre parole soit toujours assaisonnée de sel, avec grâce, afin que vous sachiez comment vous avez à répondre à chacun.
\TextTitle{Salutations}
\VS{7}Tychique, notre frère bien-aimé, et fidèle serviteur, et mon compagnon de service en notre Seigneur, vous fera savoir tout mon état.
\VS{8}Je l'envoie vers vous expressément, afin qu'il connaisse quel est votre état, et qu'il console vos cœurs~;
\VS{9}avec Onésime, notre fidèle et bien-aimé frère, qui est des vôtres. Ils vous feront connaitre toutes les choses d'ici.
\VS{10}Aristarque, qui est prisonnier avec moi, vous salue aussi, et Marc qui est le cousin de Barnabas, au sujet duquel vous avez reçu un ordre, s'il vient à vous, recevez-le.
\VS{11}Et Jésus, appelé Justus, vous salue aussi. Ils sont du nombre des circoncis, et les seuls qui travaillent avec moi pour le Royaume de Dieu, et qui ont été pour moi une consolation.
\VS{12}Epaphras, qui est des vôtres, et serviteur de Jésus-Christ, vous salue~; il ne cesse de combattre pour vous dans ses prières afin que vous demeuriez parfaits et accomplis en toute la volonté de Dieu.
\VS{13}Car je lui rends témoignage qu'il a un grand zèle pour vous, et pour ceux de Laodicée, et pour ceux d'Hiérapolis.
\VS{14}Luc, le médecin bien-aimé, vous salue, et Démas aussi.
\VS{15}Saluez les frères qui sont à Laodicée, et Nymphas, avec l'église qui est dans sa maison.
\VS{16}Et quand cette lettre aura été lue entre vous, faites en sorte qu'elle soit aussi lue dans l'église des Laodicéens, et que vous lisiez aussi celle qui viendra de Laodicée.
\VS{17}Et dites à Archippe~: Prends garde au service que tu as reçu dans le Seigneur afin de bien le remplir.
\VS{18}Je vous salue, moi Paul, de ma propre main. Souvenez-vous de mes liens. Que la grâce soit avec vous~! Amen~!
\PPE{}
\end{multicols}

%\clearpage\ShortTitle{Philémon}\BookTitle{Philémon}\BFont
\noindent\hrulefill
{\footnotesize
\textit{
\bigskip
{\centering{}
\\Auteur : Paul
\\(Gr. : Philemon)
\\Signification : Attentionné, qui embrasse
\\Thème : Un exemple d'amour
\\Date de rédaction : Env. 60 ap. J.-C.\\}
}
%\bigskip
\textit{
\\Paul écrivit cette lettre en prison, lors de sa deuxième captivité à Rome vers l'été 62, en même temps que l'épître aux Colossiens. Il s'adresse à Philémon, chrétien fortuné de Colosses ainsi qu'à sa femme Apphia, son fils Archippe et à l'église qui se réunissait dans leur maison. Paul demande à Philémon de pardonner à Onésime, son esclave, de s'être échappé d'auprès de lui. Il assure à Philémon que désormais une nouvelle relation le lierait à Onésime qui avait accepté Jésus-Christ dans sa vie. Il va même jusqu'à proposer de payer personnellement ce qu'Onésime lui devait tout en exprimant l'espoir que Philémon ferait plus que ce qu'il lui demande. Ainsi, Paul plaide pour Onésime comme Christ le fit en notre faveur.\bigskip
}
}
\par\nobreak\noindent\hrulefill
\begin{multicols}{2}
\Chap{1}
\TextTitle{Introduction}
\VerseOne{}Paul, prisonnier de Jésus-Christ, et le frère Timothée, à Philémon notre bien-aimé et compagnon d'œuvre ;
\VS{2}à Apphia, notre bien-aimée, à Archippe, notre compagnon de combat, et à l'église qui est dans ta maison.
\VS{3}Que la grâce et la paix vous soient données de la part de Dieu notre Père, et de la part du Seigneur Jésus-Christ.
\VS{4}Je rends grâces à mon Dieu, faisant toujours mention de toi dans mes prières ;
\VS{5}apprenant la foi que tu as au Seigneur Jésus, et ta charité envers tous les saints.
\VS{6}Afin que la communication de ta foi devienne efficace, en se faisant connaître par tout le bien qui est en vous, par Jésus-Christ.
\VS{7}Car, mon frère, nous avons une grande joie et une grande consolation de ta charité, en ce que tu as réjoui les entrailles des saints.
\TextTitle{Paul plaide en faveur d'Onésime}
\VS{8}C'est pourquoi, bien que j'aie une grande liberté en Christ de t'ordonner ce qui est convenable,
\VS{9}cependant je te prie plutôt par la charité, bien que je suis ce que je suis, à savoir Paul, un vieillard, et même maintenant prisonnier de Jésus-Christ ;
\VS{10}je te prie donc pour mon fils Onésime, que j'ai engendré dans mes liens ;
\VS{11}qui t'a été autrefois inutile, mais qui maintenant est bien utile à toi et à moi, et que je te renvoie.
\VS{12}Reçois-le donc comme mes propres entrailles.
\VS{13}Je voulais le retenir auprès de moi, afin qu'il me serve à ta place, dans les liens de l'Evangile.
\VS{14}Mais je n'ai rien voulu faire sans ton avis, afin que ce ne soit point comme par contrainte, mais volontairement, que tu me laisses un bien qui est à toi.
\VS{15}Car peut-être n'a-t-il été séparé de toi que pour un temps, afin que tu le recouvres\FTNT{Synonymes : retrouver, reconquérir, regagner.} pour toujours ;
\VS{16}non plus comme un esclave, mais comme étant au-dessus d'un esclave, à savoir comme un frère bien-aimé, principalement de moi ; et combien plus de toi, soit selon la chair, soit selon le Seigneur ?
\VS{17}Si donc tu me tiens pour ton compagnon, reçois-le comme moi-même.
\VS{18}Que s'il t'a fait quelque tort, ou s'il te doit quelque chose, mets-le sur mon compte.
\VS{19}Moi Paul, j'ai écrit ceci de ma propre main, je te le payerai ; pour ne pas te dire que tu te dois toi-même à moi.
\VS{20}Oui, mon frère, que je reçoive ce plaisir de toi en notre Seigneur ; réjouis mes entrailles en notre Seigneur.
\VS{21}Je t'ai écrit m'assurant de ton obéissance, et sachant que tu feras même plus que ce que je te dis.
\TextTitle{Conclusion}
\VS{22}Mais aussi, en même temps, prépare-moi un logement ; car j'espère que je vous serai rendu par vos prières.
\VS{23}Epaphras, qui est prisonnier avec moi en Jésus-Christ, te salue ;
\VS{24}Marc aussi, Aristarque, Démas, et Luc, mes compagnons d'œuvre.
\VS{25}Que la grâce de notre Seigneur Jésus-Christ soit avec votre esprit, Amen !
\PPE{}
\end{multicols}

%\clearpage\ShortTitle{1 Ti.}\BookTitle{1 Timothée}\BFont
\noindent\hrulefill
{\footnotesize
\textit{
\bigskip
{\centering{}
\\Auteur~: Paul
\\(Gr.~: Timotheos)
\\Signification~: Qui adore ou honore Dieu
\\Thème~: Comment se conduire dans l'église
\\Date de rédaction~: Env. 64 ap. J.-C.\\}
}
\textit{
\\Cette lettre s'adresse à Timothée dont le père était grec et la mère juive. Le jeune homme se convertit à Christ avec sa mère et sa grand-mère dès le premier voyage missionnaire de Paul au cours duquel il passa à Lystre.
\\Cette épître fut rédigée après la première captivité de Paul à Rome. Alors que les Eglises connaissaient une certaine expansion, Paul s'adresse à Timothée, jeune et fidèle compagnon d'œuvre qu'il a lui-même formé, sur des questions d'ordre disciplinaire et sur la pureté de la foi. Dans cette épître, dite pastorale, Paul donne des instructions précises à Timothée pour enseigner, exhorter, diriger le culte public et choisir ses collaborateurs.\bigskip
}
}
\par\nobreak\noindent\hrulefill
\begin{multicols}{2}
\Chap{1}
\TextTitle{Introduction}
\VerseOne{}Paul, apôtre de Jésus-Christ par l'ordre de Dieu, notre Sauveur, et du Seigneur Jésus-Christ, notre espérance,
\VS{2}à Timothée mon véritable fils dans la foi~: Que la grâce, la miséricorde et la paix te soient données de la part de Dieu notre Père, et de Jésus-Christ, notre Seigneur.
\TextTitle{Mise en garde contre les erreurs doctrinales~; le but de la loi} 
\VS{3}Suivant la prière que je te fis de demeurer à Ephèse, lorsque j'allais en Macédoine, je te prie encore d'ordonner à certaines personnes de ne pas enseigner une autre doctrine,
\VS{4}et de ne pas s'adonner aux fables et aux généalogies sans fin, qui produisent des disputes plutôt que l'édification en Dieu qui consiste dans la foi.
\VS{5}Or le but du commandement c'est la charité qui procède d'un cœur pur, d'une bonne conscience, et d'une foi sincère.
\VS{6}Quelques-uns, s'étant détournés de ces choses, se sont écartés dans de vains discours,
\VS{7}voulant être docteurs de la loi~; mais ils ne comprennent ni ce qu'ils disent ni ce qu'ils affirment.
\VS{8}Or nous savons que la loi est bonne pour celui qui en fait un usage légitime,
\VS{9}sachant ceci, que ce n'est pas pour le juste que la loi a été établie, mais pour les méchants et les rebelles, pour les impies et les pécheurs, pour les irréligieux et les profanes, pour les parricides, les meurtriers,
\VS{10}pour les fornicateurs, pour les homosexuels, pour les voleurs d'hommes, pour les menteurs, pour les parjures, et contre telle autre chose qui est contraire à la saine doctrine,
\VS{11}selon l'Evangile de la gloire du Dieu béni, Evangile qui m'a été confié.
\TextTitle{Témoignage de Paul}
\VS{12}Je rends grâces à celui qui m'a fortifié, c'est-à-dire à Jésus-Christ, notre Seigneur, de ce qu'il m'a estimé fidèle en m'établissant dans le service,
\VS{13}moi qui auparavant étais un blasphémateur, un persécuteur, et un homme violent~; mais j'ai obtenu miséricorde parce que j'agissais par ignorance, étant dans l'incrédulité.
\VS{14}Or la grâce de notre Seigneur a surabondé en moi, avec la foi et l'amour qui est en Jésus-Christ.
\VS{15}Cette parole est certaine et entièrement digne d'être reçue, que Jésus-Christ est venu dans le monde pour sauver les pécheurs, dont je suis le premier.
\VS{16}Mais j'ai obtenu miséricorde, afin que Jésus-Christ fasse voir en moi le premier, toute sa clémence, pour que je serve d'exemple à ceux qui croiraient en lui pour la vie éternelle.
\VS{17}Or au Roi des siècles, immortel, invisible, à Dieu seul sage, soient honneur et gloire aux siècles des siècles~! Amen~!
\TextTitle{Recommandations à Timothée}
\VS{18}Mon fils Timothée, je te recommande ce commandement que conformément aux prophéties qui auparavant ont été faites sur toi, tu t'acquittes, selon elles, du devoir de combattre dans cette bonne guerre,
\VS{19}en gardant la foi et une bonne conscience, laquelle quelques-uns ayant rejetée, ont fait naufrage quant à la foi.
\VS{20}De ce nombre sont Hyménée et Alexandre, que j'ai livrés à Satan, afin qu'ils apprennent par ce châtiment à ne plus blasphémer.
\Chap{2}
\TextTitle{Instructions sur la prière}
\VerseOne{}J'exhorte donc, avant toutes choses, à faire des requêtes, des prières, des supplications, et des actions de grâces pour tous les hommes,
\VS{2}pour les rois et pour tous ceux qui sont constitués en dignité, afin que nous menions une vie paisible et tranquille, en toute piété et honnêteté.
\VS{3}Car cela est bon et agréable devant Dieu, notre Sauveur,
\VS{4}qui veut que tous les hommes soient sauvés et qu'ils viennent à la connaissance de la vérité.
\VS{5}Car il y a un seul Dieu, et aussi un seul Médiateur entre Dieu et les hommes, à savoir Jésus-Christ homme,
\VS{6}qui s'est donné lui-même en rançon pour tous. C'est le témoignage qui a été rendu en son propre temps.
\VS{7}C'est dans cette vue que j'ai été établi prédicateur, apôtre (je dis la vérité en Christ, je ne mens point) et docteur des Gentils dans la foi et dans la vérité.
\VS{8}Je veux donc que les hommes prient en tout lieu, levant leurs mains pures, sans colère, et sans dispute.
\TextTitle{Le tenue de la femme}
\VS{9}Et de même, que les femmes, vêtues d'une manière décente, avec pudeur et modestie, ne se parent ni de tresses, ni d'or, ni de perles, ni d'habits somptueux,
\VS{10}mais qu'elles se parent de bonnes œuvres, comme il convient à des femmes qui font profession de servir Dieu.
\TextTitle{Le comportement de la femme envers son mari}
\VS{11}Que la femme apprenne dans le silence, en toute soumission.
\VS{12}Car je ne permets pas à la femme d'enseigner ni d'user d'autorité\FTNT{Le mot «~autorité~» vient du grec «~authenteo~» et signifie «~celui qui tue de ses propres mains un autre ou lui-même~; celui qui agit de sa propre autorité, autocrate~; un maître absolu~; gouverneur, exercer une domination~».} sur le mari~; mais elle doit demeurer dans le silence\FTNT{Le mot grec traduit par «~silence~» est «~hesuchia~» qui signifie «~en silence~; paisiblement~». La racine de ce terme est «~hesuchios~»~: tranquille, paisible.}.
\VS{13}Car Adam a été formé le premier, Eve ensuite.
\VS{14}Et ce n'est pas Adam qui a été séduit, mais la femme, ayant été séduite, a été la cause de la transgression.
\VS{15}Elle sera néanmoins sauvée en mettant des enfants au monde\FTNT{Il est évident que le salut ne dépend pas du fait d'enfanter puisque nous sommes sauvés par grâce et non par les œuvres. Ce verset fait référence à Eve, la mère de tous les vivants. Par elle le péché et la mort sont entrés dans le monde (Ro. 5:12) mais c'est aussi par sa postérité, à savoir Christ (Ge. 3:15), qu'elle, ainsi que tout le genre humain (hommes et femmes), sera sauvé.}, pourvu qu'elle persévère dans la foi, dans la charité, et dans la sanctification, avec modestie.
\Chap{3}
\TextTitle{Les évêques et les diacres doivent manifester le caractère de Christ}
\VerseOne{}Cette parole est certaine, si quelqu'un désire la charge d'évêque\FTNT{Evêque, du grec «~episkope~», signifie «~investigation, inspection, visite d'inspection~». C'est un acte par lequel Dieu visite les hommes, observe leurs voies, leurs caractères, pour leur accorder en partage joie ou tristesse. Ce terme signifie également surveillance, charge, contrôle, fonction, la fonction d'un ancien. Voir Ac. 1:20.}, il désire une œuvre excellente.
\VS{2}Mais il faut que l'évêque soit irrépréhensible, mari\FTNT{Paul ne dit pas que les évêques ne peuvent pas être célibataires. Il y a en effet une différence entre mari et marié. L'apôtre met l'accent sur la monogamie. Un homme célibataire peut en effet être évêque s'il remplit les caractéristiques décrites dans ce passage.} d'une seule femme, vigilant, modéré, honorable, hospitalier, propre à enseigner.
\VS{3}Il faut qu'il ne soit ni adonné au vin, ni violent, ni porté au gain déshonnête, mais modéré, éloigné des querelles, exempt d'avarice.
\VS{4}Il faut qu'il dirige honnêtement sa propre maison, et qu'il tienne ses enfants dans la soumission et dans une parfaite honnêteté~;
\VS{5}car si quelqu'un ne sait pas diriger sa propre maison, comment pourra-t-il gouverner l'église de Dieu~?
\VS{6}Il ne faut pas qu'il soit un nouveau converti, de peur qu'enflé d'orgueil, il ne tombe sous le jugement du diable.
\VS{7}Il faut aussi qu'il reçoive un bon témoignage de ceux du dehors, afin de ne pas tomber dans l'opprobre et dans les pièges du diable.
\VS{8}Que les diacres aussi soient honnêtes, éloignés de la duplicité, des excès du vin, d'un gain sordide,
\VS{9}conservant le mystère de la foi dans une conscience pure.
\VS{10}Que ceux-ci aussi soient premièrement éprouvés, et qu'ensuite ils servent, après avoir été trouvés sans reproche.
\VS{11}Leurs femmes, de même, doivent être honnêtes, non médisantes, sobres, fidèles en toutes choses.
\VS{12}Les diacres doivent être maris d'une seule femme, dirigeant honnêtement leurs enfants, et leurs propres maisons.
\VS{13}Car ceux qui auront bien servi s'acquièrent un rang honorable, et une grande liberté dans la foi qui est en Jésus-Christ.
\VS{14}Je t'écris ces choses espérant que j'irai bientôt vers toi~;
\VS{15}mais si je tarde, je t'écris ces choses afin que tu saches comment il faut se conduire dans la maison de Dieu, qui est l'Eglise du Dieu vivant, la colonne et l'appui de la vérité.
\VS{16}Et sans contredit, le mystère de la piété\FTNT{Le mystère de la piété. Il s'agit de la connaissance de Dieu manifestée en chair dans la personne de Jésus-Christ, 100\% homme et 100\% Dieu. C'est l'incarnation du Dieu Tout-Puissant dans le seul but de sauver les hommes et de produire dans leurs cœurs la véritable piété.} est grand~: Dieu a été manifesté en chair, justifié par l'Esprit, vu des anges, prêché aux Gentils, cru dans le monde, et élevé dans la gloire.
\Chap{4}
\TextTitle{L'apostasie et la séduction: Signes des derniers temps}
\VerseOne{}Mais l'Esprit dit expressément que dans les derniers temps, quelques-uns se détourneront de la foi pour s'attacher à des esprits séducteurs et à des doctrines de démons\FTNT{Il est indéniable que nous vivons les dernières minutes avant le retour glorieux de Jésus-Christ. Toutes les conditions sont pratiquement réunies pour que le Seigneur revienne, c'est pourquoi chaque enfant de Dieu doit se préparer à la rencontre avec l'Epoux. Les prophètes, notamment Paul, ont annoncé que la fin des temps serait caractérisée par la séduction et l'abandon de la foi de beaucoup de chrétiens.},
\VS{2}par l'hypocrisie de faux docteurs, ayant leur propre conscience marquée au fer rouge\FTNT{L'expression «~marqué au fer~» ou «~marque de la flétrissure~» se dit «~kauteriazo~» en grec et veut dire «~ceux dont l'âme est stigmatisée par les marques du péché~». Dans un sens médical, ce mot signifie «~cautériser~». Ce passage fait allusion à la marque de la bête qui sera imprimée dans la conscience des hommes~; voilà pourquoi Dieu nous demande de garder sa parole dans nos cœurs (Ps. 119:11). Les Juifs devaient avoir sur leurs mains et sur leurs fronts la marque de Dieu qui est sa parole (De. 6:6-8). La main se dit «~yad~» en hébreu, ce qui signifie «~pouvoir~», «~force~» ou encore «~autorité~»~; elle symbolise donc l'action. Le front se dit «~towphaphah~» en hébreu, ce qui signifie «~marque~»~; il s'agit de la pensée.}~;
\VS{3}défendant de se marier et commandant de s'abstenir des viandes que Dieu a créées afin que les fidèles, et ceux qui ont connu la vérité, en usent avec actions de grâces.
\VS{4}Car tout ce que Dieu a créé est bon, et rien ne doit être rejeté, pourvu qu'on le prenne avec actions de grâces,
\VS{5}parce que tout est sanctifié par la parole de Dieu et par la prière.
\TextTitle{S'exercer à la piété}
\VS{6}En exposant ces choses aux frères, tu seras un bon serviteur de Jésus-Christ, nourri des paroles de la foi et de la bonne doctrine que tu as exactement suivie.
\VS{7}Mais rejette les fables profanes, et semblables aux récits de vieilles femmes.
\VS{8}Exerce-toi à la piété~; car l'exercice corporel est utile à peu de chose, tandis que la piété est utile à toutes choses, ayant les promesses de la vie présente et de celle qui est à venir.
\VS{9}C'est là une parole certaine et digne d'être entièrement reçue.
\VS{10}Car c'est aussi à cause de cela que nous endurons des travaux et des opprobres, parce que nous espérons dans le Dieu vivant, qui est le Sauveur de tous les hommes, mais principalement des fidèles.
\VS{11}Déclare ces choses et enseigne-les.
\VS{12}Que personne ne méprise ta jeunesse~; mais sois le modèle pour les fidèles en paroles, en conduite, en charité, en esprit, en foi, en pureté.
\VS{13}Applique-toi à la lecture, à l'exhortation et à l'enseignement, jusqu'à ce que je vienne.
\VS{14}Ne néglige pas le don qui est en toi, et qui t'a été donné par prophétie, par l'imposition des mains de l'assemblée des anciens.
\VS{15}Pratique ces choses et donne-toi tout entier à elles, afin que tes progrès soient évidents pour tous.
\VS{16}Veille sur toi-même et sur la doctrine~; persévère dans ces choses, car en agissant ainsi, tu te sauveras toi-même et tu sauveras ceux qui t'écoutent.
\Chap{5}
\TextTitle{Recommandations concernant les veuves}
\VerseOne{}Ne reprends pas rudement le vieillard, mais exhorte-le comme un père~; les jeunes gens comme des frères,
\VS{2}les femmes âgées comme des mères, celles qui sont jeunes comme des sœurs, en toute pureté.
\VS{3}Honore les veuves qui sont véritablement veuves.
\VS{4}Mais si une veuve a des enfants, ou des petits enfants, qu'ils apprennent avant tout à exercer la piété envers leur propre famille, et à rendre à leurs parents ce qu'ils ont reçu d'eux~; car cela est bon et agréable à Dieu.
\VS{5}Or celle qui est véritablement veuve, et qui est laissée seule, espère en Dieu, et persévère nuit et jour dans les supplications et les prières.
\VS{6}Mais celle qui vit dans les plaisirs est morte quoique vivante.
\VS{7}Avertis-les donc de ces choses, afin qu'elles soient irrépréhensibles.
\VS{8}Que si quelqu'un n'a pas soin des siens, et principalement de ceux de sa famille, il a renié la foi, et il est pire qu'un infidèle.
\VS{9}Qu'une veuve, pour être enregistrée sur le rôle\FTNT{Inscription sur le rôle~: Expression qui s'apparente à l'enrôlement des soldats. Il est question des veuves ayant une place importante dans l'église, du fait qu'elles exercent une certaine responsabilité sur le reste des femmes, et ayant en charge les veuves et les orphelins pris en compte pour la dépense publique.}, n'ait pas moins de soixante ans, qu'elle ait été la femme d'un seul mari,
\VS{10}ayant le témoignage d'avoir fait de bonnes œuvres, comme d'avoir bien élevé ses propres enfants, d'avoir exercé l'hospitalité envers les étrangers, d'avoir lavé les pieds des saints, d'avoir secouru les affligés, et de s'être ainsi constamment appliquée à toutes sortes de bonnes œuvres.
\VS{11}Mais refuse les veuves qui sont plus jeunes~; car quand elles sont devenues lascives\FTNT{Ce mot vient du grec «~katastreniao~»~: «~ressentir les pulsions du désir sexuel~».} contre Christ, elles veulent se marier,
\VS{12}ayant leur condamnation, en ce qu'elles ont violé leur première foi.
\VS{13}Et avec cela aussi, étant oisives, elles apprennent à aller de maison en maison~; et non seulement elles sont oisives, mais encore causeuses, et curieuses, et parlant de choses qui ne sont pas bienséantes.
\VS{14}Je veux donc que les jeunes veuves se marient, qu'elles aient des enfants, qu'elles gouvernent leur ménage, et qu'elles ne donnent à l'adversaire aucune occasion de médire.
\VS{15}Car quelques-unes se sont déjà détournées pour suivre Satan.
\VS{16}Si quelque fidèle, homme ou quelque femme, a des veuves, qu'ils les assistent, et que l'église n'en soit point chargée, afin qu'elle puisse assister celles qui sont véritablement veuves.
\TextTitle{Recommandations concernant les anciens}
\VS{17}Que les anciens qui dirigent\FTNT{Du grec «~proistemi~»~: «~disposer~» ou «~placer devant~», «~diriger~», «~présider~» (1 Th. 5:12~; Ro. 12:8~; \vref{1 Ti. 3:4-5,12}.} convenablement soient jugés dignes d'un double honneur, spécialement ceux qui travaillent à la prédication et à l'enseignement.
\VS{18}Car l'Ecriture dit~: Tu n'emmuselleras point le bœuf quand il foule le grain\FTNT{De. 25:4.}. Et l'ouvrier mérite son salaire\FTNT{Lu. 10:7.}.
\VS{19}Ne reçois point d'accusation contre un ancien, si ce n'est sur la déposition de deux ou de trois témoins\FTNT{De. 19:15~; Mt. 18:16~; 2 Co. 13:1}.
\VS{20}Reprends publiquement ceux qui pèchent, afin que les autres aussi en aient de la crainte.
\VS{21}Je te conjure devant Dieu, et devant le Seigneur Jésus-Christ, et devant les anges élus, d'observer ces choses sans préférer l'un à l'autre, et de ne rien faire avec partialité.
\VS{22}N'impose les mains à personne avec précipitation, et ne participe pas aux péchés d'autrui~; toi-même, conserve-toi pur.
\VS{23}Ne bois plus uniquement de l'eau~; mais use d'un peu de vin, à cause de ton estomac et de tes fréquentes maladies.
\VS{24}Les péchés de certains hommes sont manifestes, même avant tout jugement, alors que chez d'autres, ils ne se découvrent qu'après.
\VS{25}De même, les bonnes œuvres sont manifestes, et celles qui ne les sont pas ne peuvent pas rester cachées\FTNT{Mt. 10:26~; Mc. 4:22~; Lu. 8:17~; Lu. 12:2.}.
\Chap{6}
\TextTitle{L'attitude du serviteur envers son maître}
\VerseOne{}Que tous les esclaves qui sont sous le joug sachent qu'ils doivent à leurs maîtres toute sorte d'honneur, afin qu'on ne blasphème pas le Nom de Dieu et sa doctrine.
\VS{2}Et que ceux qui ont des fidèles pour maîtres ne les méprisent point sous prétexte qu'ils sont leurs frères, mais qu'ils les servent d'autant mieux que ce sont des fidèles et des bien-aimés de Dieu, étant participants de la grâce. Enseigne ces choses et recommande-les.
\VS{3}Si quelqu'un enseigne des fausses doctrines, et ne se soumet pas aux saines paroles de notre Seigneur Jésus-Christ, et à la doctrine qui est selon la piété,
\VS{4}il est enflé d'orgueil, il ne sait rien~; mais il a la maladie des questions et des disputes de mots, d'où naissent l'envie, les querelles, les médisances et les mauvais soupçons,
\VS{5}les vaines disputes d'hommes corrompus d'entendement et privés de la vérité, qui estiment que la piété est un moyen de gagner. Sépare-toi de ces sortes de gens.
\TextTitle{L'amour de l'argent~: La racine de tous les maux}
\VS{6}Or la piété avec le contentement d'esprit est un grand gain.
\VS{7}Car nous n'avons rien apporté dans le monde, et aussi il est évident que nous n'en pouvons rien emporter.
\VS{8}Si nous avons la nourriture et le vêtement, cela nous suffira.
\VS{9}Mais ceux qui veulent devenir riches tombent dans la tentation\FTNT{La tentation se rapporte à l'envie de toujours posséder, de s'enrichir et de gagner plus d'argent. Cela finit par faire tomber les gens dans l'orgueil, le mensonge, la duplicité, dans la fornication, etc.}, dans le piège\FTNT{Le mot «~piège~» vient du grec «~pagis~» qui donne en français «~trappe~», «~filet~». «~Car il surprendra comme un filet tous ceux qui habitent sur la surface de toute la terre.~» Lu. 21:35. Ce mot suggère l'inattendu, l'improviste, la surprise, car les oiseaux et autres animaux pris dans le filet sont attrapés par surprise. Les conséquences de la cupidité sont nombreuses, notamment le mensonge et l'adultère. En effet, une personne cupide finit en général par tromper son conjoint.}, et dans beaucoup de désirs insensés et pernicieux\FTNT{Les désirs insensés et pernicieux sont multiples~: l'envie de toujours posséder plus que les autres, la convoitise, les rivalités, la concurrence, la folie des grandeurs. Ces choses sortent les gens de la vision du Seigneur (Mc. 4:19).} qui plongent les hommes dans la ruine et la perdition\FTNT{La ruine et la perdition. Une personne cupide se perd en s'éloignant du Seigneur (2 Pi. 2). Selon Salomon, l'argent ne rassasie personne. «~Celui qui aime l'argent n'est point rassasié par l'argent, et celui qui aime un grand train, n'en est pas nourri~; cela aussi est une vanité.~» (Ec. 5:9). Selon les Ecritures, le système bancaire mondial s'écroulera dans les prochaines années (Ap. 18).}.
\VS{10}Car l'amour de l'argent est la racine de tous les maux\FTNT{L'amour de l'argent est la racine de tous les maux. Ceux qui espèrent en une sécurité divine doivent renoncer à la sécurité matérielle et financière que la chair désire.}~; et quelques-uns en étant possédés, se sont détournés de la foi et se sont jetés eux-mêmes dans bien des tourments.
\VS{11}Mais toi, homme de Dieu~! Fuis ces choses, et recherche la justice, la piété, la foi, la charité, la patience, la douceur.
\VS{12}Combats le bon combat de la foi, saisis la vie éternelle, à laquelle aussi tu as été appelé, et pour laquelle tu as fait une belle confession en présence de plusieurs témoins.
\VS{13}Je t'ordonne, devant Dieu qui donne la vie à toutes choses, et devant Jésus-Christ qui a fait cette belle confession devant Ponce Pilate,
\VS{14}de garder ce commandement, en te conservant sans tache et irrépréhensible, jusqu'à l'apparition de notre Seigneur Jésus-Christ,
\VS{15}qui sera manifesté en son temps, qui est le Béni et seul Prince, le Roi des rois, et le Seigneur des seigneurs,
\VS{16}qui seul possède l'immortalité, et qui habite une lumière inaccessible, que nul homme n'a vu ni ne peut voir, à qui appartiennent l'honneur et la puissance éternelle. Amen~!
\VS{17}Ordonne à ceux qui sont riches dans ce monde, qu'ils ne soient pas hautains, et qu'ils ne mettent pas leur confiance dans l'incertitude des richesses, mais dans le Dieu vivant, qui nous donne toutes choses abondamment pour en jouir.
\VS{18}Qu'ils fassent du bien, qu'ils soient riches en bonnes œuvres, qu'ils soient prompts à donner, avec libéralité,
\VS{19}s'amassant ainsi pour l'avenir un trésor placé sur un fondement solide, afin qu'ils obtiennent la vie éternelle.
\TextTitle{Conclusion}
\VS{20}Timothée, garde le dépôt, en fuyant les discours vains et profanes, et les contradictions d'une science faussement ainsi nommée,
\VS{21}dont font profession quelques-uns qui se sont détournés de la foi. Que la grâce soit avec toi~! Amen~!
\PPE{}
\end{multicols}

%\clearpage\ShortTitle{Tit.}\BookTitle{Tite}\BFont
\noindent\hrulefill
{\footnotesize
\textit{
\bigskip
{\centering{}
\\Auteur : Paul
\\(Gr. : Titos)
\\Signifie : Nourrice, honorable
\\Thème : L'ordre dans les églises
\\Date de rédaction : Env. 65 ap. J.-C.\\}
}
%\bigskip
\textit{
\\Cette épître pastorale fut écrite après la libération de Paul de sa première captivité romaine, peut-être dans la ville de Philippes. Tite, disciple d'origine païenne et collaborateur de Paul, se trouvait alors en Crète où Paul l'avait laissé afin qu'il organise les églises. Dans cette lettre, l'apôtre traite des conditions requises pour assumer la charge d'ancien en mettant l'accent sur la saine doctrine. Mentionnant également les obligations morales des jeunes, des personnes âgées, ainsi que des serviteurs, il exhorte Tite à veiller et à s'éloigner des apostats.\bigskip
}
}
\par\nobreak\noindent\hrulefill
\begin{multicols}{2}
\Chap{1}
\TextTitle{Introduction}
\VerseOne{}Paul, serviteur de Dieu, et apôtre de Jésus-Christ, selon la foi des élus de Dieu et la connaissance de la vérité qui est selon la piété,
\VS{2}dans l'espérance de la vie éternelle, que Dieu, qui ne peut mentir, avait promise avant les temps éternels,
\VS{3}mais qu'il a manifestée en son propre temps par sa parole, dans la prédication qui m'a été confiée, par le commandement de Dieu notre Sauveur,
\VS{4}à Tite mon vrai fils, selon la foi qui nous est commune: Que la grâce, la miséricorde, et la paix te soient données de la part de Dieu notre Père, et de la part du Seigneur Jésus-Christ, notre Sauveur !
\TextTitle{Les caractéristiques d'un ancien}
\VS{5}La raison pour laquelle je t'ai laissé en Crète, c'est afin que tu achèves de mettre en bon ordre les choses qui restent à régler, et que tu établisses des anciens de ville en ville, suivant ce que je t'ai ordonné,
\VS{6}s'il s'y trouve un homme qui soit irrépréhensible, mari d'une seule femme, ayant des enfants fidèles, qui ne soient ni accusés de dissolution, ni rebelles.
\VS{7}Car il faut que l'évêque soit irrépréhensible, comme étant économe dans la maison de Dieu ; qu'il ne soit ni arrogant, ni coléreux, ni adonné au vin, ni violent, non convoiteux d'un gain déshonnête ;
\VS{8}mais hospitalier, aimant les gens de bien, sage, juste, saint, tempérant,
\VS{9}attaché à la parole de la vérité comme elle lui a été enseignée, afin qu'il soit capable tant d'exhorter par la saine doctrine, que de réfuter les contredisants\FTNT{Lu. 2:34 ; Jn. 19:12 ; Ac. 13:45 ; 28:19 ; 28:22 ; Ro. 10:21.}.
\VS{10}Car il y en a plusieurs qui ne veulent pas se soumettre, vains discoureurs, et séducteurs d'esprits, principalement ceux qui sont de la circoncision,
\VS{11}auxquels il faut fermer la bouche, et qui renversent les maisons tout entières enseignant pour un gain déshonnête des choses qu'on ne doit point enseigner.
\VS{12}Quelqu'un d'entre eux, qui était leur propre prophète, a dit : Les Crétois sont toujours menteurs, de mauvaises bêtes, des ventres paresseux.
\VS{13}Ce témoignage est véritable. C'est pourquoi reprends-les vivement, afin qu'ils soient sains dans la foi,
\VS{14}et qu'ils ne s'attachent point aux fables judaïques et aux commandements d'hommes qui se détournent de la vérité.
\VS{15}Toutes choses sont bien pures pour ceux qui sont purs, mais rien n'est pur pour les impurs et les infidèles ; mais leur entendement et leur conscience sont souillés.
\VS{16}Ils font profession de connaître Dieu, mais ils le renient par leurs œuvres, car ils sont abominables, et rebelles, et réprouvés pour toute bonne œuvre.
\Chap{2}
\TextTitle{Recommandations de Paul à Tite}
\VerseOne{}Mais toi, annonce les choses qui conviennent à la saine doctrine.
\VS{2}Que les vieillards soient sobres, honnêtes, prudents, sains dans la foi, dans la charité, et dans la patience.
\VS{3}De même, que les femmes âgées règlent leur extérieur d'une manière convenable à la sainteté ; qu'elles ne soient ni médisantes, ni sujettes à beaucoup de vin, mais qu'elles enseignent de bonnes choses,
\VS{4}afin qu'elles instruisent les jeunes femmes à être modestes, à aimer leurs maris, à aimer leurs enfants,
\VS{5}à être modérées, pures, occupées aux soins domestiques, bonnes, soumises à leurs maris, afin que la Parole de Dieu ne soit point blasphémée.
\VS{6}Exhorte aussi les jeunes hommes à être modérés,
\VS{7}te montrant toi-même un modèle de bonnes œuvres en toutes choses, en une doctrine exempte de toute altération, en pureté, en intégrité,
\VS{8}en paroles saines, que l'on ne puisse point condamner, afin que celui qui vous est contraire, soit rendu confus, n'ayant aucun mal à dire de vous.
\VS{9}Que les serviteurs soient soumis à leurs maîtres, leur complaisant en toutes choses, n'étant point contredisants,
\VS{10}ne dérobant rien de ce qui appartient à leurs maîtres, mais faisant toujours paraître une grande fidélité, afin de rendre honorable en toutes choses la doctrine de Dieu, notre Sauveur.
\VS{11}Car la grâce de Dieu, salutaire à tous les hommes, a été manifestée.
\VS{12}Et elle nous enseigne à renoncer à l'impiété et aux passions mondaines, et à vivre dans le présent siècle, selon la sagesse, la justice et la piété,
\VS{13}en attendant la bienheureuse espérance, et l'apparition de la gloire du grand Dieu et notre Sauveur Jésus-Christ,
\VS{14}qui s'est donné lui-même pour nous, afin de nous racheter de toute iniquité, et de nous purifier, pour lui être un peuple qui lui appartienne en propre, et qui soit zélé pour les bonnes œuvres.
\VS{15}Enseigne ces choses, exhorte, et reprends avec une pleine autorité. Et que personne ne te méprise.
\Chap{3}
\TextTitle{Conseils pratiques de Paul}
\VerseOne{}Rappelle-leur d'être soumis aux magistrats et aux autorités, d'obéir aux gouverneurs, d'être prêts à faire toutes sortes de bonnes actions,
\VS{2}de ne médire de personne, de n'être point querelleurs, mais doux, et montrant une parfaite douceur envers tous les hommes.
\VS{3}Car nous aussi, nous étions autrefois insensés, désobéissants, égarés, asservis à toute espèce de convoitises et de voluptés, vivant dans la méchanceté et dans l'envie, dignes d'être haïs, et nous haïssant les uns les autres.
\VS{4}Mais, quand la bonté de Dieu notre Sauveur et son amour envers les hommes ont été manifestés, il nous a sauvés,
\VS{5}non par des œuvres de justice que nous aurions faites, mais selon la miséricorde, par le bain de la régénération et le renouvellement du Saint-Esprit,
\VS{6}qu'il a répandu abondamment sur nous par Jésus-Christ notre Sauveur,
\VS{7}afin qu'ayant été justifiés par sa grâce, nous soyons les héritiers de la vie éternelle selon notre espérance.
\VS{8}Cette parole est certaine, et je veux que tu affirmes ces choses, afin que ceux qui ont cru en Dieu aient soin principalement de s'appliquer à pratiquer les bonnes œuvres. Voilà les choses qui sont bonnes et utiles aux hommes.
\VS{9}Mais évite les discussions folles, les généalogies, les querelles et les disputes de la loi ; car elles sont inutiles et vaines.
\VS{10}Rejette l'homme hérétique, après le premier et le second avertissement,
\VS{11}sachant qu'un tel homme est perverti, et qu'il pèche en se condamnant lui-même.
\TextTitle{Salutations}
\VS{12}Quand je t'enverrai Artémas ou Tychique, hâte-toi de venir vers moi à Nicopolis ; car j'ai résolu d'y passer l'hiver.
\VS{13}Accompagne soigneusement Zénas, docteur de la loi, et Apollos, afin que rien ne leur manque.
\VS{14}Que les nôtres aussi apprennent à être les premiers à s'appliquer aux bonnes œuvres, pour les usages nécessaires, afin qu'ils ne soient point sans fruits.
\VS{15}Tous ceux qui sont avec moi te saluent. Salue ceux qui nous aiment dans la foi. Grâce soit avec vous tous ! Amen !
\PPE{}
\end{multicols}

%\clearpage\ShortTitle{1 Pierre}\BookTitle{1 Pierre}\BFont
\noindent\hrulefill
{\footnotesize
\textit{
\bigskip
{\centering{}
\\Auteur : Pierre
\\(Gr. : Petro)
\\Signification : Roc, pierre
\\Thème : La victoire sur la souffrance
\\Date de rédaction : Env. 65 ap. J.-C.\\}
}
%\bigskip
\textit{
\\Cette lettre semble avoir été écrite à Rome même si Pierre y parlait de « Babylone ». En ces temps de persécutions, les chrétiens devaient être prudents quant à la manière dont ils parlaient du pouvoir en place, c'est pourquoi ils utilisaient souvent des codes. C'est donc durant une période difficile que fut rédigée cette épître qui s'adressait à des églises d'Asie Mineure dont la plupart furent fondées par Paul. Au travers de ces quelques lignes, Pierre exhorte les frères et sœurs à tenir ferme dans la foi malgré les souffrances liées aux épreuves, et les encourage à espérer en Jésus-Christ, leur salut. Il finit cette épître en donnant des conseils quant à l'attitude à avoir au sein de l'église.\bigskip
}
}
\par\nobreak\noindent\hrulefill
\begin{multicols}{2}
\Chap{1}
\TextTitle{Introduction}
\VerseOne{}Pierre, apôtre de Jésus-Christ, à ceux qui sont étrangers et dispersés dans le Pont\FTNT{Le Pont : province formant presque la totalité de l'Asie Mineure.}, la Galatie, la Cappadoce, l'Asie et la Bithynie,
\VS{2}élus selon la prescience de Dieu le Père, par la sanctification de l'Esprit, afin d'obéir à Jésus-Christ, et qu'ils participent à l'aspersion de son sang : Que la grâce et la paix vous soient multipliées !
\TextTitle{Les souffrances du chrétien et sa conduite à la lumière d'un salut parfait}
\VS{3}Béni soit Dieu, et le Père de notre Seigneur Jésus-Christ, qui, par sa grande miséricorde, nous a régénérés, pour une espérance vivante, par la résurrection de Jésus-Christ d'entre les morts,
\VS{4}pour un héritage incorruptible, et qui ne peut ni se souiller, ni se flétrir, qui est conservé dans les cieux pour nous,
\VS{5}qui sommes gardés par la puissance de Dieu, par la foi, afin que nous obtenions le salut, qui est prêt à être révélé dans les derniers temps !
\VS{6}En quoi vous vous réjouissez, quoique vous soyez maintenant affligés pour un peu de temps par diverses épreuves, vu que cela est convenable,
\VS{7}afin que l'épreuve de votre foi, beaucoup plus précieuse que l'or périssable, et qui toutefois est éprouvé par le feu, ait pour résultat la louange, l'honneur et la gloire, lorsque Jésus-Christ sera révélé.
\VS{8}Lequel vous aimez quoique vous ne l'ayez point vu, en qui vous croyez, quoique maintenant vous ne le voyiez pas, et vous vous réjouissez d'une joie ineffable et glorieuse,
\VS{9}remportant la fin de votre foi, savoir le salut de vos âmes.
\VS{10}C'est au sujet de ce salut que les prophètes, qui ont prophétisé touchant la grâce qui vous était destinée, ont fait leurs recherches et leurs investigations.
\VS{11}Ils voulaient sonder l'époque et les circonstances marquées par l'Esprit prophétique de Christ qui était en eux, et qui rendait à l'avance témoignage, leur faisant connaître les souffrances de Christ et la gloire dont elles seraient suivies.
\VS{12}Mais il leur fut révélé que ce n'était pas pour eux-mêmes, mais pour nous, qu'ils administraient ces choses, que vous ont annoncées maintenant ceux qui vous ont prêché l'Evangile par le Saint-Esprit envoyé du ciel, et dans lesquelles les anges désirent plonger leurs regards.
\VS{13}C'est pourquoi, ceignez les reins de votre entendement, soyez sobres, et ayez une entière espérance dans la grâce qui vous est présentée, jusqu'à ce que Jésus-Christ soit révélé\FTNT{Révélé, voir commentaire en 2 Th. 1 :7.}.
\VS{14}Comme des enfants obéissants, ne vous conformez pas à vos convoitises d'autrefois, pendant votre ignorance. 
\VS{15}Mais, comme celui qui vous a appelés est saint, vous aussi de même soyez saints dans toute votre conduite,
\VS{16}selon ce qu'il est écrit : Soyez saints, car je suis saint\FTNT{Lé. 11:44.}.
\VS{17}Et si vous invoquez comme votre Père celui qui juge selon l'œuvre de chacun, sans favoritisme, conduisez-vous avec crainte pendant le temps de votre séjour sur la terre,
\VS{18}sachant que vous avez été rachetés de votre vaine conduite, qui vous avait été enseignée par vos pères, non point par des choses corruptibles, comme par argent, ou par or,
\VS{19}mais par le sang précieux de Christ, comme d'un agneau sans défaut et sans tache,
\VS{20}prédestiné avant la fondation du monde, et manifesté dans les derniers temps, pour vous.
\VS{21}Par lui, vous croyez en Dieu, qui l'a ressuscité des morts et lui a donné la gloire, afin que votre foi et votre espérance reposent sur Dieu.
\VS{22}Ayant donc purifié vos âmes en obéissant à la vérité par le Saint-Esprit, afin que vous ayez un amour fraternel et sans hypocrisie, aimez-vous ardemment les uns les autres d'un cœur pur,
\VS{23}puisque vous avez été régénérés, non par une semence corruptible, mais par une semence incorruptible, par la parole de Dieu qui vit et demeure éternellement.
\VS{24}Car toute chair est comme l'herbe, et toute la gloire de l'homme comme la fleur de l'herbe. L'herbe sèche, et sa fleur tombe ;
\VS{25}mais la parole du Seigneur demeure éternellement\FTNT{Es. 40:6-8.}. Et cette parole est celle qui vous a été annoncée par l'Evangile.
\Chap{2}
\VerseOne{}Ayant donc renoncé à toute sorte de malice, et de toute fraude, et de dissimulation, et d'envie et de toutes médisances,
\VS{2}désirez ardemment, comme des enfants nouveau-nés, le lait spirituel et pur, afin que vous croissiez par lui,
\VS{3}si toutefois vous avez goûté combien le Seigneur est bon.
\VS{4}Et vous approchant de lui, pierre vivante, rejetée par les hommes, mais choisie et précieuse devant Dieu ;
\VS{5}et vous aussi, comme des pierres vivantes, vous êtes édifiés pour être une maison spirituelle, et une sainte sacrificature, afin d'offrir des sacrifices spirituels, agréables à Dieu par Jésus-Christ. 
\VS{6}C'est pourquoi aussi, il est dit dans l'Ecriture : Voici, je mets en Sion la principale pierre\FTNT{Jésus-Christ est la Pierre rejetée par les bâtisseurs. Voir Es. 28:16 ; Ps. 118:22.} de l'angle, choisie et précieuse ; et celui qui croit en elle ne sera point confus.
\VS{7}Elle est donc précieuse pour vous qui croyez. Mais, par rapport aux rebelles, il est dit : La pierre que ceux qui bâtissaient ont rejetée, est devenue la principale de l'angle, 
\VS{8}et une pierre d'achoppement, et un rocher de scandale ; ils se heurtent contre la parole, et sont rebelles et c'est à cela qu'ils sont destinés.
\TextTitle{La position du croyant}
\VS{9}Mais vous, vous êtes la race élue, vous êtes la sacrificature royale, la nation sainte, le peuple acquis, afin que vous annonciez les vertus de celui qui vous a appelés des ténèbres à sa merveilleuse lumière,
\VS{10}vous qui autrefois n'étiez pas son peuple, mais qui maintenant êtes le peuple de Dieu, vous qui n'aviez point obtenu miséricorde, mais qui maintenant avez obtenu miséricorde.
\VS{11}Mes bien-aimés, je vous exhorte, comme étrangers et voyageurs, à vous abstenir des convoitises charnelles qui font la guerre à l'âme.
\VS{12}Ayant une conduite honnête avec les Gentils, afin que, là même où ils vous calomnient comme si vous étiez des malfaiteurs, ils remarquent vos bonnes œuvres, et glorifient Dieu, au jour où il les visitera.
\VS{13}Soyez donc soumis à tout établissement humain, pour l'amour de Dieu : Soit au roi, comme à celui qui est au-dessus des autres,
\VS{14}soit aux gouverneurs, comme à ceux qui sont envoyés de sa part pour punir les méchants et pour honorer les gens de bien.
\VS{15}Car c'est là la volonté de Dieu, qu'en faisant bien vous fermiez la bouche à l'ignorance des hommes insensés.
\VS{16}Comme libres, et non pas comme ayant la liberté pour servir de voile à la méchanceté, mais agissant comme des serviteurs de Dieu.
\VS{17}Honorez tout le monde ; aimez tous vos frères ; craignez Dieu ; honorez le roi.
\VS{18}Serviteurs, soyez soumis en toute crainte à vos maîtres, non seulement à ceux qui sont bons et équitables, mais aussi à ceux qui sont méchants.
\VS{19}Car c'est une chose agréable à Dieu si quelqu'un à cause de la conscience qu'il a envers Dieu, endure des afflictions en souffrant injustement. 
\VS{20}Autrement, quelle gloire en aurez-vous, si lorsque vous péchez et qu’on vous frappe, vous le supportez  patiemment ? Mais si quand vous faites le bien et que vous souffrez, vous le supportez patiemment, voilà où Dieu prend plaisir. 
\TextTitle{Les souffrances de Christ, le Substitut des hommes}
\VS{21}Car vous êtes appelés à cela, vu même que Christ a souffert pour nous, nous laissant un modèle, afin que vous suiviez ses traces, 
\VS{22}lui qui n'a point commis de péché, et dans la bouche duquel il ne s'est point trouvé de fraude ;
\VS{23}qui, lorsqu'on lui disait des outrages, n'en rendait point, et quand on lui faisait du mal, n'usait point de menaces, mais il se remettait à celui qui juge justement ; 
\VS{24}lui qui a porté lui-même nos péchés en son corps sur le bois, afin qu'étant morts aux péchés nous vivions pour la justice ; lui par la meurtrissure\FTNT{Es. 53:5.} duquel même vous avez été guéris.
\VS{25}Car vous étiez comme des brebis errantes, mais maintenant vous êtes convertis au Pasteur et à l'Evêque de vos âmes. 
\Chap{3}
\TextTitle{La conduite chrétienne à la maison et à l'église}
\VerseOne{}Femmes, soyez de même soumises à vos maris, afin que, si quelques-uns n'obéissent point à la parole, ils soient gagnés sans paroles par la conduite de leurs femmes,
\VS{2}lorsqu'ils verront la pureté de votre conduite, accompagnée de crainte.
\VS{3}Et que votre ornement ne soit point celui de dehors, qui consiste dans la frisure des cheveux, et dans une parure d'or, et dans la magnificence des habits,
\VS{4}mais que votre parure consiste dans l’homme caché dans le cœur, c’est-à-dire dans l’incorruptibilité d’un esprit doux et paisible, qui est d’un grand prix devant Dieu.
\VS{5}Car c'est ainsi que se paraient aussi autrefois les saintes femmes qui espéraient en Dieu, étant soumises à leurs maris,
\VS{6}comme Sara, qui obéissait à Abraham et l'appelait son seigneur. C'est d'elle que vous êtes devenues les filles, en faisant ce qui est bien et sans vous laisser troubler par aucune crainte.
\VS{7}Et vous, maris, de même comportez-vous selon la sagesse avec vos femmes, comme un vase\FTNT{Paul utilise une métaphore connu des grecs pour parler du corps : le vase.} plus fragile, c'est-à-dire, féminin ; leur portant honneur comme étant aussi ensemble héritiers de la grâce de la vie, afin que vos prières ne soient pas interrompues. 
\VS{8}Enfin, soyez tous d'un même sentiment, remplis de compassion les uns envers les autres, d'amour fraternel, miséricordieux et doux.
\VS{9}Ne rendez point mal pour mal, ou injure pour injure\FTNT{Mt. 5:44.} ; mais, au contraire, bénissez ; sachant que c'est à cela que vous êtes appelés, afin d'hériter la bénédiction.
\VS{10}Car celui qui veut aimer sa vie et voir des jours heureux, qu'il préserve sa langue du mal, et ses lèvres de prononcer aucune fraude,
\VS{11}qu'il se détourne du mal, et fasse le bien, qu'il recherche la paix, et qu'il tâche de se la procurer ;
\VS{12}car les yeux du Seigneur sont sur les justes, et ses oreilles sont attentives à leurs prières, mais la face du Seigneur est contre ceux qui se conduisent mal.
\TextTitle{La conduite chrétienne aux yeux du monde}
\VS{13}Et qui vous maltraitera, si vous êtes les imitateurs de celui qui est bon ?
\VS{14}Que si toutefois vous souffrez quelque chose pour la justice, vous êtes bienheureux. Mais ne craignez point les maux dont ils veulent vous faire peur, et n'en soyez point troublés ;
\VS{15}mais sanctifiez le Seigneur dans vos coeurs, et soyez toujours prêts à répondre, avec douceur et avec respect, à chacun qui vous demande raison de l'espérance qui est en vous, 
\VS{16}et ayant une bonne conscience, afin que, ceux qui blâment votre bonne conduite en Christ, soient confus de ce qu'ils médisent de vous, comme si vous étiez des malfaiteurs.
\VS{17}Car il vaut mieux, si telle est la volonté de Dieu, que vous souffriez en faisant le bien qu’en faisant le mal.
\TextTitle{Les souffrances de Christ}
\VS{18}Car aussi Christ a souffert une fois pour les péchés, lui juste pour les injustes, afin de nous amener à Dieu, étant mort en la chair, mais vivifié par l'Esprit,
\VS{19}par lequel aussi étant allé, il a  prêché aux esprits qui sont en prison\FTNT{La possibilité du salut après la mort n’a aucun fondement biblique (Hé. 9 :27). Dans ce passage, il est fait mention des pécheurs qui ont vécu du temps de Noé et auxquels le Seigneur Jésus a confirmé la condamnation lorsqu’il est descendu dans l’Hadès ( l'enfer; Ep. 4 :9). Voir aussi commentaire en Mt 16 :18.},
\VS{20}et qui avaient été autrefois incrédules, quand la patience de Dieu les attendait, durant les jours de Noé, tandis que l'arche se préparait dans laquelle un petit nombre, à savoir huit personnes furent sauvées par l'eau.
\VS{21}A quoi aussi maintenant répond la figure qui nous sauve, c'est-à-dire, le baptême ; non point celui par lequel les ordures de la chair sont nettoyées, mais la promesse faite à Dieu d'une conscience pure, par la résurrection de Jésus-Christ,
\VS{22}qui est à la droite de Dieu, étant allé au Ciel, et auquel sont assujettis les anges, et les dominations et les puissances.
\Chap{4}
\TextTitle{Souffrir dans la chair}
\VerseOne{}Puisque Christ a souffert pour nous dans la chair, vous aussi armez-vous de la même pensée. Car celui qui a souffert dans la chair a cessé de pécher,
\VS{2}afin de vivre, non plus selon les convoitises des hommes, mais selon la volonté de Dieu, pendant le temps qui lui reste à vivre dans la chair.
\VS{3}Car il nous suffit d'avoir accompli la volonté des Gentils, pendant le temps de notre vie passée, quand nous nous abandonnions aux impudicités, aux convoitises, à l'ivrognerie, aux excès dans le manger et dans le boire, et aux idolâtries abominables.
\VS{4}Ce que ces Gentils trouvent fort étrange, ils vous calomnient de ce que vous ne courez pas avec eux dans un même débordement de dissolution. 
\VS{5}Mais ils rendront compte à celui qui est prêt à juger les vivants et les morts.
\VS{6}Car c'est aussi pour cela que les morts ont été évangélisés, afin qu'ils soient jugés selon les hommes dans la chair, et qu'ils vivent selon Dieu dans l'esprit. 
\TextTitle{La conduite chrétienne dans le temps présent}
\VS{7}Or la fin de toutes choses est proche : Soyez donc sobres et vigilants pour prier. 
\VS{8}Mais surtout, ayez les uns pour les autres une ardente charité, car la charité couvre une multitude de péchés.
\VS{9}Soyez hospitaliers les uns envers les autres, sans murmures. 
\VS{10}Que chacun selon le don qu'il a reçu, l'emploie pour le service des autres, comme de bons gestionnaires des diverses grâces de Dieu. 
\VS{11}Si quelqu'un parle, qu'il parle comme annonçant les paroles de Dieu ; si quelqu'un administre, qu'il administre comme par la puissance que Dieu lui en a fournie, afin qu'en toutes choses Dieu soit glorifié par Jésus-Christ, auquel appartient la gloire et la force, aux siècles des siècles. Amen ! 
\VS{12}Mes bien-aimés, ne trouvez point étrange quand vous êtes comme dans une fournaise pour votre épreuve, comme s'il vous arrivait quelque chose d'extraordinaire. 
\VS{13}Mais réjouissez-vous, de ce que vous participez aux souffrances de Christ, afin qu'aussi à la révélation de sa gloire, vous vous réjouissiez avec allégresse.
\VS{14}Si on vous dit des injures pour le Nom de Christ, vous êtes heureux, car l'Esprit de gloire et de Dieu repose sur vous, lequel est blasphémé par ceux qui vous noircissent mais pour vous, vous le glorifiez.
\VS{15}Que nul de vous, ne souffre comme meurtrier, ou voleur, ou malfaiteur, ou curieux des affaires d'autrui. 
\VS{16}Mais si quelqu'un souffre comme chrétien, qu'il n'en ait point de honte, mais qu'il glorifie Dieu en cela.
\VS{17}Car il est temps que le jugement commence par la maison de Dieu\FTNT{Le jugement commence par la maison de Dieu. Ez. 9:1-11.}. Or s'il commence premièrement par nous, quelle sera la fin de ceux qui n'obéissent pas à l'Evangile de Dieu ?
\VS{18}Et si le juste est difficilement sauvé, où comparaîtra le méchant et le pécheur ? 
\VS{19}Que ceux-là donc aussi, qui souffrent par la volonté de Dieu, puisqu'ils font ce qui est bon lui recommandent leurs âmes, comme au fidèle Créateur. 
\Chap{5}
\TextTitle{Servir sans rien attendre en retour}
\VerseOne{}Je prie les anciens qui sont parmi vous, moi qui suis ancien avec eux, et témoin des souffrances de Christ, et participant de la gloire qui doit être révélée et je leur dis : 
\VS{2}Paissez le troupeau de Dieu qui vous est commis, en prenant garde sur lui, non point par contrainte, mais volontairement ; non point pour un gain déshonnête, mais par un principe d'affection. 
\VS{3}Et non pas comme ayant domination sur les héritages du Seigneur, mais de telle manière que vous soyez les modèles du troupeau. 
\VS{4}Et quand le souverain Pasteur\FTNT{Jésus est notre Souverain Pasteur. Voir Ps. 23 ; Jn. 10.} apparaîtra, vous obtiendrez la couronne incorruptible de la gloire.
\VS{5}De même, vous jeunes gens, soyez soumis aux anciens. Et ayant tous de la soumission les uns pour les autres, soyez parés par-dedans d'humilité; parce que Dieu résiste aux orgueilleux, mais il fait grâce aux humbles. 
\VS{6}Humiliez-vous donc sous la puissante main de Dieu, afin qu'il vous élève quand le temps sera venu ;
\VS{7}remettez-lui tout ce qui peut vous inquiéter, car il prend soin de vous.
\VS{8}Soyez sobres et veillez : Car le diable, votre adversaire, tourne autour de vous comme un lion rugissant, cherchant qui il pourra dévorer. 
\VS{9}Résistez-lui donc en demeurant fermes dans la foi, sachant que les mêmes souffrances s'accomplissent dans la compagnie de vos frères qui sont dans le monde. 
\TextTitle{Salutations}
\VS{10}Or que le Dieu de toute grâce, qui nous a appelés à sa gloire éternelle en Jésus-Christ, après que vous aurez souffert un peu de temps, vous rende parfaits, vous affermisse, vous fortifie et vous établisse. 
\VS{11}A lui soient la gloire et la force, aux siècles des siècles ! Amen !
\VS{12}Je vous ai écrit brièvement par Silvain, notre frère, que je crois vous être fidèle, vous déclarant et vous protestant que la grâce de Dieu dans laquelle vous êtes est la véritable. 
\VS{13}L'Eglise qui est à Babylone, élue avec vous, et Marc, mon fils, vous saluent. 
\VS{14}Saluez-vous les uns les autres par un baiser de charité. Que la paix soit avec vous tous qui êtes en Jésus-Christ ! Amen !
\PPE{}
\end{multicols}

%\clearpage\ShortTitle{2 Pierre}\BookTitle{2 Pierre}\BFont
\begin{multicols}{2}
\TextTitle{[Introduction]}
\Chap{1}
\VerseOne{}Simon Pierre, serviteur et apôtre de Jésus-Christ, à vous qui avez reçu en partage une foi du même prix que la nôtre, par la justice de notre Dieu et Sauveur Jésus-Christ (1).
\VS{2}Que la grâce et la paix vous soient multipliées, par la connaissance de Dieu, et de notre Seigneur Jésus.
\TextTitle{[Les grandes vertus chrétiennes]}
\VS{3}Puisque sa divine puissance nous a donné tout ce qui appartient à la vie et à la piété, par la connaissance de celui qui nous a appelés par sa gloire et par sa vertu,
\VS{4}par lesquelles nous sont données les grandes et précieuses promesses, afin que par elles vous soyez faits participants de la nature divine, en fuyant la corruption qui règne dans le monde par la convoitise.
\VS{5}A cause de cela même, faites tous vos efforts pour ajouter la vertu à votre foi ; à la vertu, la connaissance,
\VS{6}à la connaissance, la tempérance, à la tempérance, la patience, à la patience, la piété,
\VS{7}à la piété, l'amour fraternel, et à l'amour fraternel, la charité.
\VS{8}Car si ces choses sont en vous, et y abondent, elles ne vous laisseront point oisifs ni stériles pour la connaissance de notre Seigneur Jésus-Christ.
\VS{9}Mais celui en qui ces choses ne se trouvent point est aveugle, et ne voit point de loin, ayant oublié la purification de ses anciens péchés.
\VS{10}C'est pourquoi, mes frères, efforcez-vous plutôt à affermir votre vocation, et votre élection ; car en faisant cela vous ne broncherez jamais.
\VS{11}Car par ce moyen, l'entrée au Royaume éternel de notre Seigneur et Sauveur Jésus-Christ vous sera abondamment accordée.
\TextTitle{[Sollicitude de l'Apôtre pour ses lecteurs ; autorité de son témoignage et de la parole des prophètes]}
\VS{12}C'est pourquoi je ne négligerai pas de vous rappeler sans cesse ces choses, quoique vous ayez de la connaissance, et que vous soyez fondés dans la vérité présente.
\VS{13}Car je crois qu'il est juste que je vous réveille par des avertissements, pendant que je suis dans cette tente (2),
\VS{14}sachant que dans peu de temps je dois la quitter, comme notre Seigneur Jésus-Christ lui-même me l'a déclaré.
\TextTitle{[Souvenir de la transfiguration]}
\VS{15}Mais j'aurai soin qu’après mon départ vous puissiez toujours vous souvenir de ces choses.
\VS{16}Car ce n’est pas en suivant des fables composées avec artifice, que nous vous avons fait connaître la puissance et l’avènement (3) de notre Seigneur Jésus-Christ, mais comme ayant vu sa majesté de nos propres yeux.
\VS{17}Car il reçut de Dieu le Père, honneur et gloire, lorsque cette voix lui fut adressée du milieu de la gloire magnifique : Celui-ci est mon Fils bien-aimé, en qui j'ai mis toute mon affection (4).
\VS{18}Et nous entendîmes cette voix envoyée du ciel, lorsque nous étions avec lui sur la sainte montagne.
\TextTitle{[Témoignage à la véracité des Ecritures prophétiques]}
\VS{19}Nous avons aussi la parole des prophètes qui est très ferme, à laquelle vous faites bien d'être attentifs, comme à une lampe qui brille dans un lieu obscur, jusqu'à ce que le jour vienne à paraître et que l'étoile du matin (5) se lève dans vos cœurs.
\VS{20}Sachant premièrement ceci, qu'aucune prophétie de l'Ecriture ne procède d’une interprétation particulière.
\VS{21}Car la prophétie n'a jamais été autrefois apportée par la volonté humaine, mais les saints hommes de Dieu étant poussés par le Saint-Esprit, ont parlé.
\TextTitle{[Avertissement contre les faux docteurs]}
\Chap{2}
\VerseOne{}Mais comme il y a eu de faux prophètes parmi le peuple, il y aura aussi parmi vous de faux docteurs, qui introduiront secrètement des sectes pernicieuses, et qui reniant le Seigneur qui les a rachetés, attireront sur eux-mêmes une ruine soudaine.
\VS{2}Et plusieurs suivront leurs sectes de perdition, et à cause d'eux, la voie de la vérité sera blasphémée.
\VS{3}Par cupidité, ils trafiqueront (1) de vous au moyen de paroles déguisées, mais la condamnation qui leur est destinée depuis longtemps ne tarde point, et leur perdition ne sommeille point.
\VS{4}Car si Dieu n'a pas épargné les anges qui ont péché, s’il les a précipités dans l'abîme (2), les a liés avec des chaînes d'obscurité, les a livrés pour y être gardés jusqu’au jugement ;
\VS{5}et s'il n'a point épargné l’ancien monde, mais a gardé Noé (3), lui huitième, qui était le prédicateur de la justice ; et a fait venir le déluge sur le monde des impies ;
\VS{6}et s'il a condamné à la destruction totale les villes de Sodome et de Gomorrhe, les réduisant en cendres, et les mettant pour être un exemple à ceux qui vivraient dans l'impiété ;
\VS{7}et s'il a délivré le juste Lot (4), qui cruellement affligé de la conduite de ces hommes sans frein, eut beaucoup à souffrir de ces abominables par leur infâme conduite ;
\VS{8}car cet homme juste, qui habitait au milieu d’eux, affligeait chaque jour son âme juste, à cause de ce qu’il voyait et entendait dire de leurs méchantes actions.
\VS{9}Le Seigneur sait ainsi délivrer de l’épreuve les hommes pieux, et réserver les injustes pour être punis au jour du jugement ;
\VS{10}principalement ceux qui vont après la chair, dans la passion de l'impureté, et qui méprisent l’autorité. Gens audacieux et arrogants, ils ne craignent point d’injurier les gloires ;
\VS{11}alors que les anges qui sont supérieurs en force et en puissance, ne prononcent point contre elles de jugement blasphématoire devant le Seigneur ;
\VS{12}Mais eux, semblables à des bêtes brutes, qui s’abandonnent à leurs penchants naturels, et qui sont nées pour être prises et détruites, ils parlent d’une manière blasphématoire de ce qu’ils ignorent, et ils périront par leur propre corruption.
\VS{13}Et ils recevront la récompense de leur iniquité. Ils aiment à être tous les jours dans les délices. Ce sont des taches et des souillures, et ils font leurs délices de leurs tromperies dans les repas qu'ils font avec vous.
\VS{14}Ils ont les yeux pleins d'adultère, ils ne cessent jamais de pécher, ils attirent les âmes mal affermies ; ils ont le cœur exercé à la cupidité, ce sont des enfants de malédiction,
\TextTitle{[Caractéristiques des faux docteurs
\\a. ils ressemblent à Balaam]}
\VS{15}qui ayant laissé le droit chemin, se sont égarés, et ont suivi la voie de Balaam (5), fils de Bosor, qui aima le salaire de l'iniquité ; mais il fut repris pour sa transgression.
\VS{16}Car une ânesse muette parlant d'une voix humaine, arrêta la folie du prophète.
\TextTitle{[b. ils sont dépourvus dIntroduction]}
\VS{17}Ce sont des fontaines sans eau, des nuées agitées par le tourbillon, et des gens à qui l'obscurité des ténèbres est réservée éternellement.
\TextTitle{[c. leurs discours sont savants et prétentieux]
\\cp. 1 Co. 2:1-5)}
\VS{18}Car en prononçant des discours fort enflés de vanité, ils amorcent par les convoitises de la chair, et par leurs impudicités, ceux qui s'étaient véritablement retirés de ceux qui vivent dans l’égarement ;
\TextTitle{[d. ils corrompent la liberté chrétienne]}
\VS{19}ils leur promettent la liberté, quand ils sont eux-mêmes esclaves de la corruption ; car chacun est esclave de ce qui a triomphé de lui.
\VS{20}En effet, si après s’être retirés des souillures du monde, par la connaissance du Seigneur et Sauveur Jésus-Christ, ils s’y engagent de nouveau et sont vaincus, leur dernière condition est pire que la première (6).
\VS{21}Car mieux valait pour eux n'avoir pas connu la voie de la justice, que de l'avoir connue et se détourner du saint commandement qui leur avait été donné.
\TextTitle{[e. ils retournent à leurs premiers péchés]}
\VS{22}Mais ce qu'on dit par un proverbe véritable leur est arrivé : Le chien est retourné à ce qu'il avait vomi ; et la truie lavée est retournée se vautrer dans le bourbier.
\TextTitle{[Le but de l'Epitre]}
\Chap{3}
\VerseOne{}Mes bien-aimés, c'est ici la seconde lettre que je vous écris, afin de réveiller, dans l'une et dans l'autre, par mes avertissements, les sentiments purs que vous avez.
\VS{2}Et afin que vous vous souveniez des paroles qui ont été dites auparavant par les saints prophètes, et du commandement que vous avez reçu de nous, qui sommes apôtres du Seigneur et Sauveur.
\VS{3}Sur toutes choses, sachez qu'aux derniers jours (1) il viendra des moqueurs, se conduisant selon leurs propres convoitises,
\TextTitle{[La seconde venue de Christ et le jour du Seigneur
\\a. l'incrédulité sera générale quant au retour de Christ]}
\VS{4}et disant : Où est la promesse de son avènement ? Car depuis que les pères sont morts, toutes choses demeurent comme elles ont été dès le commencement de la création.
\VS{5}Car ils ignorent volontairement ceci, c’est que les cieux furent autrefois créés par la parole de Dieu, et que la terre est sortie de l'eau, et qu'elle subsiste parmi l'eau ;
\VS{6}et que par ces choses-là, le monde d'alors périt, étant submergé par les eaux du déluge (2).
\VS{7}Mais les cieux et la terre d’à présent sont gardés par la même parole, étant réservés pour le feu au jour du jugement, et de la destruction des hommes impies.
\VS{8}Mais vous mes bien-aimés, n'ignorez pas ceci, qu'un jour est devant le Seigneur comme mille ans, et mille ans comme un jour (3).
\VS{9}Le Seigneur ne retarde point l'exécution de sa promesse, comme quelques-uns croient qu'il y ait du retard, mais il est patient envers nous, ne voulant qu'aucun ne périsse, mais que tous se repentent.
\TextTitle{[b) la purification des cieux et de la terre]}
\VS{10}Or le jour (4) du Seigneur viendra comme un voleur dans la nuit, et en ce jour-là, les cieux passeront avec le bruit d’une effroyable tempête, et les éléments seront dissous par l'ardeur du feu ; et la terre avec toutes les œuvres qu’elle renferme sera brûlée entièrement.
\VS{11}Puisque toutes ces choses doivent se dissoudre, quelles ne doivent pas être la sainteté de votre conduite et votre piété.
\VS{12}En attendant, et en hâtant par vos désirs la venue du jour de Dieu, par lequel les cieux étant enflammés seront dissous, et les éléments se fondront par l'ardeur du feu.
\VS{13}Mais nous attendons, selon sa promesse, de nouveaux cieux et une nouvelle terre (5), où la justice habitera.
\VS{14}C'est pourquoi, mes bien-aimés, en attendant ces choses, appliquez-vous à être trouvés par lui sans tache et sans reproche dans la paix.
\VS{15}Et croyez que la longue patience de notre Seigneur est la preuve qu'il veut votre salut ; comme Paul, notre frère bien-aimé, vous l’a aussi écrit selon la sagesse qui lui a été donnée ;
\VS{16}comme il le fait aussi dans toutes ses lettres, où il parle de ces points, dans lesquels il y a des choses difficiles à comprendre, dont les personnes ignorantes et mal affermies tordent le sens (6), comme celui des autres Ecritures, pour leur propre perdition.
\TextTitle{[Conclusion]}
\VS{17}Vous donc mes bien-aimés, puisque vous êtes déjà avertis, prenez garde qu'étant emportés avec les autres par la séduction des abominables, vous ne veniez à déchoir de votre fermeté.
\VS{18}Mais croissez dans la grâce et dans la connaissance de notre Seigneur et Sauveur Jésus-Christ. A lui soit la gloire maintenant, et jusqu'au jour d'éternité ! Amen !
\PPE{}
\end{multicols}

%\clearpage\ShortTitle{2 Timothée}\BookTitle{2 Timothée}\BFont
\noindent\hrulefill
{\footnotesize
\textit{
\bigskip
{\centering{}
\\Signifie : Qui adore ou honore Dieu
\\Thème : Le maintien de la vérité
\\Auteur : Paul
\\Date de rédaction : Env. 67\\}
}
%\bigskip
\textit{
\\Cette lettre s’adresse à Timothée dont le père était grec et la mère juive. Le jeune homme se convertit à Christ avec sa mère et sa grand-mère dès le premier voyage missionnaire de Paul au cours duquel il passa à Lystre.
%\bigskip
\\Paul écrit cette épître pastorale en prison à Rome, après avoir été arrêté dans une province orientale à Ephèse ou Troas.  Ses conditions de détentions étant plus rudes que la première fois, Paul restait dubitatif quant à sa mise en liberté. Il demanda donc à Timothée, son fils dans la foi et fidèle compagnon d’œuvre, de le rejoindre à Rome afin semble-t-il de recevoir ses dernières volontés. Après avoir exposé à Timothée les qualités et les devoirs d’un bon serviteur de l’évangile, il l’encouragea à lutter contre les faux docteurs et l’apostasie en prêchant la Parole en toutes circonstances.\bigskip
}
}
\par\nobreak\noindent\hrulefill
\begin{multicols}{2}
\TextTitle{[Introduction]}
\Chap{1}
\VerseOne{}Paul, apôtre de Jésus-Christ, par la volonté de Dieu, selon la promesse de la vie qui est en Jésus-Christ.
\VS{2}A Timothée, mon fils bien-aimé, que la grâce, la miséricorde et la paix te soient données de la part de Dieu le Père, et de la part de Jésus-Christ notre Seigneur.
\TextTitle{[Paul encourage Timothée]}
\VS{3}Je rends grâces à Dieu, que mes ancêtres ont servi et que je sers avec une conscience pure, faisant sans cesse mention de toi dans mes prières nuit et jour,
\VS{4}me souvenant de tes larmes, je désire fort te voir afin que je sois rempli de joie.
\VS{5}Et me souvenant de la foi sincère qui est en toi, et qui a premièrement habité en Loïs, ta grand-mère, et en Eunice, ta mère, et qui, je suis persuadé qu'elle habite aussi en toi.
\VS{6}C'est pourquoi je t'exhorte de ranimer le don de Dieu qui est en toi par l'imposition de mes mains.
\VS{7}Car Dieu ne nous a pas donné un esprit de timidité, mais de force, de charité\FTNT{Il est question ici de l’amour «~agape~», c’est-à-dire divin.} et de sagesse.
\VS{8}N’aie donc point honte du témoignage à rendre à notre Seigneur ni de moi, qui suis son prisonnier ; mais souffre avec moi les afflictions de l'Evangile, selon la puissance de Dieu,
\VS{9}qui nous a sauvés et qui nous a appelés par une sainte vocation, non selon nos œuvres, mais selon son propre dessein, et selon la grâce qui nous a été donnée en Jésus-Christ avant les temps éternels,
\VS{10}et qui maintenant a été manifestée par l'apparition de notre Sauveur Jésus-Christ, qui a détruit la mort et qui a mis en lumière la vie et l'immortalité par l'Evangile,
\VS{11}pour lequel j'ai été établi prédicateur, apôtre et docteur des Gentils.
\VS{12}C'est pourquoi aussi je souffre ces choses, mais je n'en ai point de honte ; car je connais celui en qui j'ai cru, et je suis persuadé qu'il est Puissant pour garder mon dépôt\FTNT{Dépôt : Il est question ici de la connaissance correcte et de la pure doctrine de l’Evangile qui doit être fermement et fidèlement gardée, et qui doit être consciencieusement délivrée aux autres.} jusqu'à ce jour-là.
\VS{13}Retiens dans la foi et dans la charité qui est en Jésus-Christ le modèle des saines paroles que tu as apprises de moi.
\VS{14}Garde le bon dépôt par le Saint-Esprit qui habite en nous.
\VS{15}Tu sais que tous ceux qui sont en Asie se sont éloignés de moi ; entre lesquels sont Phygelle et Hermogène.
\VS{16}Que le Seigneur accorde sa miséricorde à la maison d'Onésiphore, car souvent il m'a consolé, et il n'a point eu honte de mes chaînes.
\VS{17}Au contraire, quand il a été à Rome, il m'a cherché avec beaucoup d’empressement, et il m'a trouvé.
\VS{18}Que le Seigneur lui fasse trouver miséricorde envers le Seigneur en ce jour-là ; et tu sais mieux que personne combien il m'a rendu de services à Ephèse.
\TextTitle{[La conduite d'un disciple de Christ dans les jours d'apostasie]}
\Chap{2}
\VerseOne{}Toi donc, mon fils, sois fortifié dans la grâce qui est en Jésus-Christ.
\VS{2}Et les choses que tu as entendues de moi devant plusieurs témoins, confie-les à des personnes fidèles qui soient capables de les enseigner aussi à d'autres.
\VS{3}Toi donc, souffre avec moi comme un bon soldat de Jésus-Christ.
\VS{4}Il n’est pas de soldat qui s'embarrasse des affaires de cette vie s’il veut plaire à celui qui l'a enrôlé pour la guerre.
\VS{5}De même, l’athlète qui combat n'est point couronné s'il n'a pas combattu selon les règles.
\VS{6}Il faut aussi que le laboureur travaille premièrement, et ensuite il recueille les fruits.
\VS{7}Considère ce que je dis, car le Seigneur te donne de l’intelligence en toutes choses.
\VS{8}Souviens-toi que Jésus-Christ, qui est de la semence de David, est ressuscité des morts, selon mon Evangile,
\VS{9}pour lequel je souffre beaucoup de maux, jusqu'à être mis dans les chaînes comme un malfaiteur, mais cependant, la parole de Dieu n'est point liée.
\VS{10}C'est pourquoi je souffre tout pour l'amour des élus, afin qu'eux aussi obtiennent le salut qui est en Jésus-Christ, avec la gloire éternelle.
\VS{11}Cette parole est certaine, que si nous mourons avec lui, nous vivrons aussi avec lui.
\VS{12}Si nous souffrons avec lui, nous régnerons aussi avec lui. Si nous le renions, il nous reniera aussi\FTNT{Lu. 9:26.}.
\VS{13}Si nous sommes infidèles, il demeure fidèle, car il ne peut pas se renier lui-même.
\VS{14}Remets ces choses en mémoire, protestant devant Dieu qu'on ait pas de disputes de mots, qui est une chose dont il ne revient aucun profit, mais elle est la ruine des auditeurs.
\VS{15}Efforce-toi de te rendre approuvé\FTNT{Approuvé vient du grec ~dokimos~. Du temps de l’apôtre Paul, les systèmes bancaires actuels n’existaient pas, toute la monnaie était en métal. Pour obtenir les pièces de monnaie, le métal était fondu et versé dans des moules et après le démoulage, il était nécessaire d’enlever les bavures. Or de nombreuses personnes les grattaient pour récupérer le surplus de métal et même davantage, ce qui faussait le poids de la monnaie. Face à ce problème, de nombreuses lois furent promulguées à Athènes pour éradiquer la pratique du rognage des pièces en circulation. Il existait toutefois quelques changeurs intègres qui ne mettaient en circulation que des pièces au bon poids. On appelait ces personnes des ~dokimos~, ce qui signifie ~éprouvés~ ou ~approuvés~.} devant Dieu, comme un ouvrier sans reproche, enseignant purement la parole de la vérité.
\VS{16}Mais évite les discours vains et profanes ; car ceux qui les tiennent avanceront toujours plus dans l'impiété,
\VS{17}et leur parole rongera comme une gangrène. Et parmi ceux-là sont Hyménée et Philète,
\VS{18}qui se sont écartés\FTNT{Ecarter, dévier, s'écarter de, manquer le but. A l’époque des apôtres, il y avait plusieurs faux frères qui semaient la zizanie au milieu des enfants de Dieu. Parmi eux étaient Alexandre le forgeron (1 Ti. 1:18-20), Hyménée (1 Ti. 1:18-20), Philète (2 Ti. 2:16-18), les judaïsants (Ac. 15 ; Ga. 2), Diotrèphe (3 Jn.). Les faux frères sont des séducteurs.} de la vérité, en disant que la résurrection est déjà arrivée, et qui renversent la foi de quelques-uns.
\VS{19}Toutefois, le fondement de Dieu demeure ferme, ayant ce sceau : Le Seigneur connaît ceux qui lui appartiennent\FTNT{Le Seigneur connaît ses brebis. Voir No. 16:5 ; Jn. 10:14.} ; et : Quiconque invoque le nom du Seigneur, qu'il s’éloigne de l'iniquité.
\VS{20}Or dans une grande maison, il n'y a pas seulement des vases d'or et d'argent, mais il y en a aussi de bois et de terre. Les uns sont des vases d’honneur et les autres sont d’un usage vil.
\VS{21}Si quelqu'un donc se purifie de ces choses, il sera un vase d’honneur, sanctifié et utile au Seigneur, et préparé pour toute bonne œuvre.
\VS{22}Fuis aussi les désirs de la jeunesse, et recherche la justice, la foi, la charité, et la paix avec ceux qui invoquent le Seigneur d'un cœur pur.
\VS{23}Et rejette les questions\FTNT{Questions folles. Il est question ici de disputes, débats, discussions ou questions oiseuses.} folles, et qui sont sans instruction, sachant qu'elles ne font que produire des querelles.
\VS{24}Or, il ne faut pas que le serviteur du Seigneur soit querelleur, il doit au contraire avoir de la douceur envers tout le monde, propre à enseigner, supportant patiemment les mauvais,
\VS{25}enseignant avec douceur ceux qui ont un sentiment contraire, dans l'espérance qu'un jour Dieu leur donnera la repentance pour reconnaître la vérité,
\VS{26}et afin qu'ils se réveillent pour sortir des pièges du diable, par lesquels ils ont été pris pour faire sa volonté.
\TextTitle{[L'Ecriture, l'arme du chrétien face à l'apostasie]}
\Chap{3}
\VerseOne{}Or sache ceci, que dans les derniers jours\FTNT{Les derniers jours. Voir Ge. 49:1-2.} il surviendra des temps difficiles.
\VS{2}Car les hommes seront idolâtres d’eux-mêmes, amis de l’argent, fanfarons, orgueilleux, blasphémateurs, rebelles à leurs parents, ingrats, irréligieux,
\VS{3}sans affection naturelle, sans fidélité, calomniateurs, intempérants, cruels, haïssant les gens de bien,
\VS{4}traîtres, emportés, enflés d'orgueil, amis des voluptés plûtot qu'amis de Dieu\FTNT{(Le mot grec «~philotheos~» (amour de Dieu) du préfixe «~philos~» qui signifie «~amis, être lié d'amitié avec quelqu'un~» (Matt. 11:19 ; Lu. 7:6 ;Jn. 15:13-15 etc...) et de «~theos~» qui signifie «~Dieu~».}
\VS{5}ayant l'apparence\FTNT{L’apparence de la piété. Le mot ~apparnec~ vient du grec ~morphosis~ et du latin ~forma~ qui donnent ~forme~ en français. Il est question du formalisme, de l’attachement excessif aux règles, aux rites, aux coutumes et aux traditions. Dans l’église de Laodicée, l’accent est plutôt mis sur les règles à observer et les apparences que sur la vie spirituelle et intérieure. Les manifestations extérieures du formalisme sont : les lieux «sacrés» pour adorer (temples, cathédrales, pèlerinages, etc.) ; l’observation des jours sacrés (dimanche et sabbat) ; les rituels censés permettre au croyant d’expérimenter Dieu et de rentrer dans une vie bénie (circoncision, ordination, bénédiction nuptiale, paiement de la dîme, présentation des enfants à Dieu par le pasteur...) ; une manière spéciale de s’habiller (toge, soutane, collet clérical, kippa, voile, costume/cravate, un régime alimentaire spécial, etc.). Voir Mt. 6:1-8.} de la piété, mais en ayant renié la force. Eloigne-toi donc de telles gens.
\VS{6}Il en est parmi eux qui se glissent dans les maisons et qui tiennent captives les femmes chargées de péchés et agitées de diverses convoitises,
\VS{7}qui apprennent toujours, mais qui ne peuvent jamais parvenir à la pleine connaissance de la vérité.
\VS{8}Et comme Jannès et Jambrès ont résisté à Moïse, ceux-ci de même résistent à la vérité, étant des gens qui ont l'esprit corrompu, et qui sont réprouvés quant à la foi.
\VS{9}Mais ils ne feront pas de plus grands progrès, car leur folie sera manifestée à tous, comme le fut celle de ceux-là.
\VS{10}Mais pour toi, tu as pleinement compris ma doctrine, ma conduite, mon intention, ma foi, ma douceur, ma charité, ma persévérance.
\VS{11}Et tu sais les persécutions et les afflictions qui me sont arrivées à Antioche, à Iconie, et à Lystre. Quelles persécutions n’ai-je pas supportées ? Et comment le Seigneur m'a délivré de toutes.
\VS{12}Or tous ceux aussi qui veulent vivre pieusement en Jésus-Christ seront persécutés.
\VS{13}Mais les hommes méchants et imposteurs iront en empirant, séduisant les autres, et étant séduits.
\VS{14}Mais toi, demeure ferme dans les choses que tu as apprises et qui t'ont été confiées, sachant de qui tu les as apprises,
\VS{15}vu même que dès ton enfance tu as la connaissance des saintes lettres, qui peuvent te rendre sage pour le salut par la foi en Jésus-Christ.
\VS{16}Toute l'Ecriture est inspirée de Dieu, et utile pour enseigner, pour convaincre, pour corriger, et pour instruire selon la justice,
\VS{17}afin que l'homme de Dieu soit accompli et parfaitement instruit pour toute bonne œuvre.
\TextTitle{[Paul encourage solennellement Timothée à prêcher la parole]}
\Chap{4}
\VerseOne{}Je te somme devant Dieu, et devant le Seigneur Jésus-Christ, qui doit juger les vivants et les morts, lors de son apparition et de son règne.
\VS{2}Prêche la parole, insiste en toute occasion, favorable ou non. Reprends, censure, exhorte avec toute douceur d'esprit, et avec doctrine.
\VS{3}Car il viendra un temps où les hommes ne supporteront pas la saine doctrine, mais aimant qu'on leur chatouille les oreilles par des discours agréables, ils chercheront des docteurs qui répondent à leurs désirs\FTNT{Beaucoup refusent la saine doctrine et acceptent un évangile basé sur les biens matériels.}.
\VS{4}Et ils détourneront leurs oreilles de la vérité, et se tourneront vers les fables.
\VS{5}Mais toi, veille en toutes choses, souffre les afflictions, fais l'œuvre d'un évangéliste, rends ton ministère pleinement approuvé.
\VS{6}Car pour moi, je m'en vais maintenant servir de libation, et le temps de mon départ est proche.
\VS{7}J'ai combattu le bon combat, j'ai achevé la course, j'ai gardé la foi.
\VS{8}Au reste, la couronne de justice m'est réservée, et le Seigneur, juste Juge, me la rendra en ce jour-là, et non seulement à moi, mais aussi à tous ceux qui auront aimé son apparition.
\VS{9}Hâte-toi de venir bientôt vers moi.
\VS{10}Car Démas m'a abandonné, ayant aimé le présent siècle, et il s'en est allé à Thessalonique ; Crescens est allé en Galatie ; et Tite en Dalmatie.
\VS{11}Luc est seul avec moi ; prends Marc, et amène-le avec toi, car il m'est fort utile pour le ministère.
\VS{12}J'ai aussi envoyé Tychique à Ephèse.
\VS{13}Quand tu viendras, apporte avec toi le manteau que j'ai laissé à Troas, chez Carpus, et les livres aussi ; mais principalement mes parchemins.
\VS{14}Alexandre le forgeron m'a fait beaucoup de mal. Le Seigneur lui rendra selon ses œuvres.
\VS{15}Garde-toi donc de lui, car il s'est fortement opposé à nos paroles.
\VS{16}Personne ne m'a assisté dans ma première défense, mais tous m'ont abandonné ; toutefois que cela ne leur soit point imputé !
\VS{17}Mais le Seigneur m'a assisté et fortifié, afin que ma prédication soit pleinement approuvée, et que tous les Gentils l’entendent ; et j'ai été délivré de la gueule du lion.
\VS{18}Le Seigneur aussi me délivrera de toute mauvaise œuvre, et me sauvera dans son Royaume céleste. A lui soit la gloire aux siècles des siècles. Amen !
\TextTitle{[Conclusion]}
\VS{19}Salue Priscille et Aquilas, et la famille d'Onésiphore.
\VS{20}Eraste est resté à Corinthe, et j'ai laissé Trophime malade à Milet.
\VS{21}Hâte-toi de venir avant l'hiver. Eubulus et Pudens, et Linus, et Claudia, et tous les frères te saluent.
\VS{22}Que le Seigneur Jésus-Christ soit avec ton esprit. Que la grâce soit avec vous. Amen !
\PPE{}
\end{multicols}

%\clearpage\ShortTitle{Jude}\BookTitle{Jude}\BFont
\begin{multicols}{2}
\TextTitle{[Introduction]}
\Chap{1}
\VerseOne{}Jude serviteur de Jésus-Christ, et frère de Jacques, à ceux qui ont été appelés par l'Evangile, que Dieu a sanctifiés et gardés pour Jésus-Christ :
\VS{2}Que la miséricorde, la paix et l'amour vous soient multipliés.
\TextTitle{[Mise en garde contre l'apostasie]}
\VS{3}Mes bien-aimés, comme je désirais vous écrire avec empressement au sujet de notre salut commun, j’ai jugé nécessaire de le faire pour vous exhorter à combattre pour la foi qui a été transmise aux saints une fois pour toutes.
\VS{4}Car il s’est glissé parmi vous, certains hommes dont la condamnation est écrite depuis longtemps, des impies qui changent la grâce de notre Dieu en dissolution, et qui renient le seul Dominateur Jésus-Christ, notre Dieu et Seigneur.
\TextTitle{[Exemples historiques d'incrédulité et de révolte]}
\VS{5}Je veux vous rappeler une chose que vous savez déjà : C'est que le Seigneur après avoir délivré le peuple du pays d'Egypte, fit ensuite périr les incrédules,
\VS{6}qu’il a réservés pour le jugement du grand jour, enchainés éternellement par les ténèbres, les anges qui n'ont pas gardé leur origine, mais qui ont abandonné leur propre demeure ;
\VS{7}que Sodome et Gomorrhe, et les villes voisines qui s'étaient abandonnées comme eux à l'impureté et à des vices contre nature, sont données en exemples, subissant la peine d’un feu éternel.
\TextTitle{[Description des faux docteurs]}
\VS{8}Malgré cela, ces hommes aussi, plongés dans leurs rêveries, souillent leur chair, méprisent l’autorité, et blasphèment contre les dignités.
\VS{9}Or, l'archange Michel, lorsqu’il contestait avec le diable et lui disputait le corps de Moïse, n'osa pas prononcer contre lui un jugement blasphématoire, mais il dit seulement : Que le Seigneur te réprime !
\VS{10}Eux, au contraire, ils blasphèment contre tout ce qu'ils ignorent, et ils se corrompent dans tout ce qu'ils savent naturellement, comme font les bêtes brutes.
\VS{11}Malheur à eux ! Car ils ont suivi la voie de Caïn, et ils se sont jetés dans l’égarement de Balaam, pour l’amour du gain, ils se sont perdus par la rébellion de Koré (1).
\VS{12}Ce sont des écueils dans vos agapes, lorsqu’ils prennent leurs repas avec vous sans aucune retenue, et se repaissant eux-mêmes ; ce sont des nuées sans eau, emportées par des vents çà et là ; des arbres d’automne dont le fruit se pourrit, et sans fruits, deux fois morts, et déracinés ;
\VS{13}des vagues impétueuses de la mer, jetant l'écume de leurs impuretés ; des étoiles errantes, à qui l'obscurité des ténèbres est réservée éternellement.
\VS{14}C’est aussi pour eux qu’Hénoc, le septième homme après Adam, a prophétisé en disant :
\VS{15}Voici, le Seigneur est venu avec ses saintes myriades, pour exercer un jugement contre tous les hommes, et pour convaincre tous les impies parmi eux de tous les actes d'impiété qu’ils ont commis et de toutes les paroles blasphématoires qu’ont proférées contre lui des pécheurs impies.
\VS{16}Ce sont des gens qui murmurent, qui se plaignent toujours, qui marchent selon leurs convoitises, qui ont à la bouche des discours hautains, qui admirent les personnes pour le profit qui leur en revient.
\VS{17}Mais vous, mes bien-aimés, souvenez-vous des choses qui ont été prédites par les apôtres de notre Seigneur Jésus-Christ.
\VS{18}Ils vous disaient que dans les derniers temps il y aurait des moqueurs, qui marcheraient selon leurs convoitises impies.
\VS{19}Ce sont ceux qui provoquent des divisions, des gens sensuels, n'ayant pas l'Esprit.
\TextTitle{[Exhortation aux chrétiens]}
\VS{20}Mais vous, mes bien-aimés, vous édifiant vous-mêmes sur votre très sainte foi, et priant par le Saint-Esprit,
\VS{21}maintenez-vous les uns les autres dans l'amour de Dieu, en attendant la miséricorde de notre Seigneur Jésus-Christ, pour obtenir la vie éternelle.
\VS{22}Et ayez pitié des uns en usant de discernement ;
\VS{23}sauvez-en d’autres avec crainte, en les arrachant hors du feu, haïssant jusqu’à la tunique souillée par la chair.
\TextTitle{[Conclusion]}
\VS{24}Or, à celui qui est puissant pour vous préserver de toute chute et vous faire paraître devant sa gloire irréprochables et dans l’allégresse,
\VS{25}à Dieu, seul sage, notre Sauveur, par Jésus-Christ notre Seigneur, soient gloire et magnificence, force et puissance, dès maintenant et dans tous les siècles, Amen !
\PPE{}
\end{multicols}

%\clearpage\ShortTitle{Hé.}\BookTitle{Hébreux}\BFont
\noindent\hrulefill
{\footnotesize
\textit{
\bigskip
{\centering{}
\\Auteur~: Inconnu
\\Thème~: La prêtrise du Messie
\\Date de rédaction~: Env. 68 ap. J.-C.\\}
}
\textit{
\\Cette épître fut rédigée avant la destruction de Jérusalem, car le temple y subsistait encore. Elle s'adressait à des juifs convertis connaissant bien l'auteur. Parmi eux, certains étaient tentés de retourner au judaïsme à cause des persécutions. L'auteur désire affermir ces chrétiens en leur montrant que l'objectif de la loi avait été réalisé par Christ qui est supérieur aux anges, aux prophètes et à Moïse. Il leur montre combien son œuvre rédemptrice est parfaite et les invite à suivre le Seigneur avec une foi indéfectible en persévérant dans l'amour fraternel.\bigskip
}
}
\par\nobreak\noindent\hrulefill
\begin{multicols}{2}
\Chap{1}
\TextTitle{Dieu parle par le Fils}
\VerseOne{}Dieu ayant anciennement parlé à nos pères par les prophètes, à plusieurs reprises et de plusieurs manières,
\VS{2}nous a parlé dans ces derniers jours\FTNT{Les derniers jours ont commencé avec la naissance de l'Eglise. Voir Joë. 2:28~; Ac. 2:14-17.} par son Fils, qu'il a établi héritier de toutes choses, et par lequel il a aussi créé l'univers~;
\VS{3}et qui étant la splendeur de sa gloire, et l'empreinte de sa substance, et soutenant toutes choses par sa parole puissante, ayant fait par lui-même la purification de nos péchés, s'est assis à la droite de la Majesté divine dans les lieux très hauts.
\TextTitle{Le Fils, supérieur aux anges}
\VS{4}Etant devenu d'autant supérieur aux anges, il a hérité d'un nom plus excellent que le leur.
\VS{5}Car auquel des anges a-t-il jamais dit~: Tu es mon Fils, je t'ai engendré aujourd'hui\FTNT{Ps. 2:7.}~? Et encore~: Je serai pour lui un Père, et il sera pour moi un Fils\FTNT{2 S. 7:14.}~?
\VS{6}Et quand il introduit de nouveau dans le monde son Fils premier-né\FTNT{Voir commentaire en Col. 1:15.}, il est dit~: Et que tous les anges de Dieu l'adorent\FTNT{Ps. 97:7.}~!
\VS{7}Car quant aux anges, il est dit~: Il fait de ses anges des vents, et de ses serviteurs des flammes de feu\FTNT{Ps. 104:4.}.
\VS{8}Mais à l'égard du Fils, il dit~: Ô Dieu, ton trône demeure aux siècles des siècles~; et le sceptre de ton Royaume est un sceptre d'équité~;
\VS{9}tu as aimé la justice, et tu as haï l'iniquité~; c'est pourquoi, ô Dieu, ton Dieu t'a oint d'une huile de joie par-dessus tous tes semblables\FTNT{Ps. 45:7-8.}~!
\VS{10}Et dans un autre endroit~: Toi, Seigneur, tu as fondé la terre dès le commencement, et les cieux sont les ouvrages de tes mains~;
\VS{11}ils périront, mais tu es permanent~; et ils vieilliront tous comme un vêtement,
\VS{12}et tu les rouleras comme un manteau et ils seront changés~; mais toi, tu restes le même, et tes années ne finiront point\FTNT{Es. 50:9~; Es. 51:6~; Ps. 102:27-28.}.
\VS{13}Et auquel des anges a-t-il jamais dit~: Assieds-toi à ma droite, jusqu'à ce que j'aie mis tes ennemis pour le marchepied de tes pieds\FTNT{Ps. 110:1.}~?
\VS{14}Ne sont-ils pas tous des esprits administrateurs, envoyés pour servir en faveur de ceux qui doivent recevoir l'héritage du salut~?
\Chap{2}
\TextTitle{Ne pas négliger le salut}
\VerseOne{}C'est pourquoi il nous faut prendre garde de plus près aux choses que nous avons entendues, de peur que nous les laissions s'échapper.
\VS{2}Car, si la parole prononcée par les anges a été ferme, et si toute transgression et toute désobéissance a reçu une juste rétribution,
\VS{3}comment échapperons-nous, si nous négligeons un si grand salut, qui, ayant été premièrement annoncé par le Seigneur, nous a été confirmé par ceux qui l'avaient entendu~?
\VS{4}Dieu confirmant aussi leur témoignage par des prodiges, et des miracles, et par plusieurs autres différents effets de sa puissance, et par les dons du Saint-Esprit, selon sa volonté.
\TextTitle{Toutes choses doivent être soumises à Christ}
\VS{5}Car, ce n'est pas aux anges qu'il a soumis le monde à venir dont nous parlons.
\VS{6}Et quelqu'un a rendu ce témoignage en quelque autre endroit, disant~: Qu'est-ce que l'homme, pour que tu te souviennes de lui, ou le fils de l'homme, pour que tu le visites~?
\VS{7}Tu l'as fait un peu moindre que les anges, tu l'as couronné de gloire et d'honneur, et l'as établi sur les œuvres de tes mains.
\VS{8}Tu as assujetti toutes choses sous ses pieds\FTNT{Ps. 8:5-7.}. En effet, en lui assujettissant toutes choses, il n'a rien laissé qui ne lui soit assujetti. Mais, nous ne voyons pourtant pas encore que toutes choses lui soient assujetties.
\TextTitle{Jésus abaissé un peu de temps pour sauver l'homme}
\VS{9}Mais celui qui a été fait un peu moindre que les anges, Jésus, nous le voyons couronné de gloire et d'honneur par la passion de sa mort, afin que par la grâce de Dieu, il souffrît la mort pour tous.
\VS{10}Car il était convenable, que celui pour qui sont toutes choses et par qui sont toutes choses, puisqu'il a amené plusieurs enfants à la gloire, consacre le Prince de leur salut par les afflictions.
\VS{11}Car, et celui qui sanctifie et ceux qui sont sanctifiés descendent tous d'un même père. C'est pourquoi il n'a pas honte de les appeler ses frères,
\VS{12}disant~: J'annoncerai ton Nom à mes frères, et je te louerai au milieu de l'assemblée\FTNT{Ps. 22:23.}.
\VS{13}Et encore~: Je me confierai en lui. Et encore~: Me voici, moi et les enfants que Dieu m'a donnés\FTNT{Es. 8:17-18.}.
\VS{14}Ainsi donc, puisque les enfants participent à la chair et au sang, lui aussi de même a participé aux mêmes choses, afin que, par la mort, il rende impuissant celui qui avait le pouvoir de la mort, c'est-à-dire le diable,
\VS{15}et qu'il délivre tous ceux qui, par crainte de la mort, étaient assujettis toute leur vie à la servitude.
\VS{16}Car, certes, il n'a nullement secouru les anges, mais il a secouru la postérité d'Abraham.
\VS{17}C'est pourquoi il a fallu qu'il soit semblable en toutes choses à ses frères, afin qu'il soit un Grand-Prêtre miséricordieux et fidèle dans les choses qui doivent être faites envers Dieu, pour faire la propitiation pour les péchés du peuple~;
\VS{18}car, parce qu'il a souffert lui-même, étant tenté, il est puissant pour secourir ceux qui sont tentés.
\Chap{3}
\TextTitle{Christ, supérieur à Moïse}
\VerseOne{}C'est pourquoi, mes frères saints, qui avez part à la vocation céleste, considérez attentivement Jésus-Christ, l'Apôtre et le Grand-Prêtre de notre profession,
\VS{2}qui a été fidèle à celui qui l'a établi, comme le fut Moïse dans toute sa maison.
\VS{3}Car Jésus-Christ a été jugé digne d'une gloire d'autant supérieure à celle de Moïse, que celui qui a construit une maison, a plus d'honneur que la maison même.
\VS{4}Car chaque maison est construite par quelqu'un, mais celui qui a construit toutes choses, c'est Dieu.
\VS{5}Et quant à Moïse, il a été fidèle dans toute sa maison, comme serviteur, pour témoigner des choses qui devaient être dites~;
\VS{6}mais Christ l'est comme Fils sur sa maison~; et nous sommes sa maison\FTNT{L'Eglise véritable est la maison de Dieu. Voir Es. 66:1~; 1 Co. 3:16~; 1 Co. 6:19~; Ep. 2:21-22. Les bâtiments ne sont pas la maison de Dieu. Le premier bâtiment d'église avait été édifié par des fidèles sous le règne d'Alexandre Sévère en 222-235. L'Eglise véritable est composée de pierres vivantes qui ont pour fondement le Roc (Jésus), parce qu'elle est bâtie par Jésus-Christ lui-même et qu'elle est sa propriété~; les démons ne peuvent pas la détruire. L'Eglise véritable ne peut donc être confondue avec un bâtiment ou une maison physique.}, pourvu que nous retenions fermement jusqu'à la fin l'assurance et la gloire de l'espérance.
\TextTitle{Résultat de l'incrédulité de la génération qui sortit d'Egypte}
\VS{7}C'est pourquoi, comme dit le Saint-Esprit~: Aujourd'hui, si vous entendez sa voix,
\VS{8}n'endurcissez point vos cœurs, comme il arriva dans le lieu de la rébellion, au jour de la tentation dans le désert,
\VS{9}où vos pères me tentèrent et m'éprouvèrent, et ils virent mes œuvres pendant quarante ans\FTNT{Ps. 95:8-11.}.
\VS{10}C'est pourquoi je fus irrité contre cette génération, et je dis~: Leur cœur s'égare toujours. Et ils n'ont pas connu mes voies.
\VS{11}Aussi, je jurai dans ma colère~: Ils n'entreront pas dans mon repos~!
\VS{12}Mes frères, prenez garde que quelqu'un de vous n'ait un cœur mauvais et incrédule, au point de se révolter contre le Dieu vivant,
\VS{13}mais exhortez-vous les uns les autres chaque jour, aussi longtemps qu'on peut dire~: Aujoud'hui~! De peur que quelqu'un d'entre vous ne s'endurcisse par la séduction du péché.
\VS{14}Car nous sommes devenus participants de Christ, pourvu que nous gardions ferme jusqu'à la fin notre première assurance,
\VS{15}pendant qu'il est dit~: Aujourd'hui, si vous entendez sa voix, n'endurcissez pas vos cœurs, comme il arriva dans le lieu de la rébellion.
\VS{16}Car, quelques-uns l'ayant entendue, le provoquèrent à la colère~; mais ce ne furent pas tous ceux qui étaient sortis d'Egypte par Moïse. 
\VS{17}Et contre qui Dieu fut-il irrité pendant quarante ans~? Ne fut-ce pas contre ceux qui péchèrent, et dont les cadavres tombèrent dans le désert~?
\VS{18}Et à qui jura-t-il qu'ils n'entreraient point dans son repos, sinon à ceux qui furent rebelles~?
\VS{19}Aussi, nous voyons qu'ils ne purent y entrer à cause de leur incrédulité.
\Chap{4}
\TextTitle{Le repos}
\VerseOne{}Craignons donc, que quelqu'un d'entre vous, venant à négliger la promesse d'entrer dans son repos, ne s'en trouve privé.
\VS{2}Car il nous a été évangélisé, aussi bien qu'à eux~; mais la parole qu'ils entendirent ne leur servit de rien, parce qu'elle n'était pas mêlée avec la foi dans ceux qui l'entendirent.
\VS{3}Pour nous qui avons cru, nous entrons dans le repos, suivant ce qui a été dit~: C'est pourquoi je jurai dans ma colère, ils n'entreront pas dans mon repos\FTNT{Hé. 3:11.}~! Il dit cela, quoique ses œuvres aient été achevées depuis la fondation du monde.
\VS{4}Car il a parlé quelque part ainsi du septième jour~: Et Dieu se reposa de toutes ses œuvres le septième jour\FTNT{Ge. 2:2.}.
\VS{5}Et encore dans ce passage~: Ils n'entreront pas dans mon repos~!
\VS{6}Puisqu'il reste donc à quelques-uns d'y entrer, et que ceux à qui d'abord il a été évangélisé n'y sont pas entrés à cause de leur désobéissance,
\VS{7}Dieu détermine de nouveau un certain jour, qu'il appelle aujourd'hui, en disant par David si longtemps après, selon ce qui a été dit~: Aujourd'hui, si vous entendez sa voix, n'endurcissez point vos cœurs\FTNT{Ps. 95:8-11.}.
\VS{8}Car, si Josué les avait introduits dans le repos, jamais après cela il n'aurait parlé d'un autre jour.
\TextTitle{Entrer dans le repos de Dieu}
\VS{9}Il reste donc encore un repos réservé au peuple de Dieu.
\VS{10}Car celui qui est entré dans son repos, se repose aussi de ses œuvres, comme Dieu s'est reposé des siennes.
\VS{11}Efforçons-nous donc d'entrer dans ce repos-là, de peur que quelqu'un ne tombe en imitant une semblable désobéissance.
\VS{12}Car la Parole de Dieu est vivante et efficace, et plus pénétrante qu'une épée quelconque à deux tranchants, et atteignant jusqu'à la division de l'âme et de l'esprit, et des jointures et des mœlles~; et elle juge les pensées et les intentions du cœur.
\VS{13}Et il n'y a aucune créature qui soit cachée devant lui, mais toutes choses sont nues et entièrement découvertes aux yeux de celui devant lequel nous devons rendre compte.
\VS{14}Ainsi, puisque nous avons un Souverain Grand-Prêtre, Jésus, le Fils de Dieu, qui a traversé les cieux, tenons ferme notre profession.
\VS{15}Car nous n'avons pas un Grand-Prêtre qui ne puisse avoir compassion de nos infirmités~; mais, nous avons celui qui a été tenté comme nous en toutes choses, mais sans pécher.
\VS{16}Approchons donc avec assurance du trône de la grâce, afin d'obtenir miséricorde et de trouver grâce, pour être secourus dans le temps convenable.
\Chap{5}
\TextTitle{Le service du grand-prêtre}
\VerseOne{}Or tout grand-prêtre pris d'entre les hommes est établi pour les hommes dans les choses qui concernent Dieu, afin qu'il offre des dons et des sacrifices pour les péchés.
\VS{2}Etant capable d'avoir de l'indulgence pour les ignorants et les égarés, puisqu'il est aussi lui-même enveloppé d'infirmité.
\VS{3}Et à cause de cette infirmité, il doit offrir pour les péchés, non seulement pour le peuple, mais aussi pour lui-même.
\VS{4}Et nul ne s'attribue cet honneur, si ce n'est celui qui est appelé de Dieu, comme Aaron.
\TextTitle{Christ, Grand-Prêtre selon l'ordre de Melchisédek}
\VS{5}De même, aussi Christ ne s'est point glorifié lui-même d'être fait Grand-Prêtre, mais celui qui lui a dit~: C'est toi qui es mon Fils, je t'ai engendré aujourd'hui\FTNT{Ps. 2:7.}~!
\VS{6}Comme il dit encore ailleurs~: Tu es prêtre éternellement, selon l'ordre de Melchisédek\FTNT{Ps. 110:4.}.
\VS{7}C'est lui qui, pendant les jours de sa chair, a offert avec de grands cris et avec larmes des prières et des supplications à celui qui pouvait le sauver de la mort, et il a été exaucé à cause de sa piété.
\VS{8}Quoiqu'il soit le Fils de Dieu, il a pourtant appris l'obéissance par les choses qu'il a souffertes.
\VS{9}Après avoir été consacré, il est devenu l'auteur du salut éternel pour tous ceux qui lui obéissent,
\VS{10}étant appelé de Dieu à être Grand-Prêtre selon l'ordre de Melchisédek~;
\VS{11}de qui nous avons beaucoup de choses à dire, mais elles sont difficiles à expliquer, parce que vous êtes devenus lents à comprendre.
\TextTitle{Du lait à la nourriture solide\FTNTT{jusqu'à Hé. 6:12}}
\VS{12}En effet, tandis que vous devriez être maîtres depuis longtemps, vous avez encore besoin qu'on vous enseigne quels sont les premiers rudiments des oracles de Dieu, et vous êtes devenus tels, que vous avez encore besoin de lait et non d'une nourriture solide.
\VS{13}Or quiconque use de lait, ne sait point ce que c'est que la parole de la justice, parce qu'il est un enfant\FTNT{Le mot enfant dans ce passage vient du grec «~nepios~» qui signifie «~ignorant~».}.
\VS{14}Mais la viande solide est pour ceux qui sont déjà hommes faits, {c'est-à-dire}, pour ceux qui, pour y être habitués, ont les sens exercés à discerner le bien et le mal.
\Chap{6}
\TextTitle{Tendre à la perfection}
\VerseOne{}C'est pourquoi, laissant la parole qui n'enseigne que les premiers principes de Christ, tendons à la perfection, ne posant pas de nouveau le fondement de la repentance des œuvres mortes, et de la foi en Dieu,
\VS{2}de la doctrine des baptêmes, et de l'imposition des mains, et de la résurrection des morts, et du jugement éternel.
\VS{3}Et c'est ce que nous ferons, si Dieu le permet.
\VS{4}Or il est impossible que ceux qui ont été une fois illuminés, et qui ont goûté le don céleste, et qui ont été fait participants au Saint-Esprit,
\VS{5}qui ont goûté la bonne parole de Dieu, et les puissances du siècle à venir,
\VS{6}s'ils retombent, soient changés de nouveau par la repentance, vu que, quant à eux, ils crucifient de nouveau le Fils de Dieu, et l'exposent à l'opprobre.
\VS{7}Car la terre qui est abreuvée par la pluie qui tombe souvent sur elle, et qui produit des herbes propres à ceux par qui elle est labourée, reçoit la bénédiction de Dieu~;
\VS{8}mais, celle qui produit des épines et des chardons, est rejetée et proche de malédiction, et sa fin est d'être brûlée.
\VS{9}Mais nous sommes persuadés, quoique nous parlions ainsi, en ce qui vous concerne, mes bien-aimés, des choses meilleures et qui tiennent au salut.
\VS{10}Car Dieu n'est pas injuste, pour oublier votre œuvre, et le travail de la charité que vous avez témoigné pour son Nom, en ce que vous avez secouru les saints, et que vous les secourez encore.
\VS{11}Or nous souhaitons que chacun de vous montre jusqu'à la fin le même empressement pour la pleine certitude de l'espérance,
\VS{12}afin que vous ne vous relâchiez point, mais que vous imitiez ceux qui, par la foi et par la patience, héritent ce qui leur a été promis.
\TextTitle{Christ entré au-delà du voile}
\VS{13}Car, lorsque Dieu fit la promesse à Abraham, ne pouvant jurer par un plus grand, il jura par lui-même,
\VS{14}en disant~: Certainement, je te bénirai abondement et je te multiplierai merveilleusement\FTNT{Ge. 22:16-17.}.
\VS{15}Et ainsi, Abraham ayant attendu patiemment, obtint ce qui lui avait été promis.
\VS{16}Or les hommes jurent par celui qui est plus grand qu'eux, et le serment qu'ils font pour confirmer leur parole met fin à tous leurs différends.
\VS{17}C'est pourquoi Dieu, voulant faire mieux connaître aux héritiers de la promesse la fermeté immuable de sa résolution, il y a fait intervenir le serment,
\VS{18}afin que, par deux choses immuables, dans lesquelles il est impossible que Dieu mente, nous ayons une ferme consolation, nous qui avons notre refuge à obtenir l'espérance qui nous est proposée.
\VS{19}Laquelle nous tenons comme une ancre sûre et ferme de l'âme, et qui pénètre jusqu'au-delà du voile,
\VS{20}où Jésus est entré comme notre précurseur, ayant été fait Grand-Prêtre éternellement, selon l'ordre de Melchisédek\FTNT{Voir Ge. 14.}.
\Chap{7}
\TextTitle{Melchisédek, type de Christ\FTNTT{Ge. 14}}
\VerseOne{}En effet, ce Melchisédek était Roi de Salem et Prêtre du Dieu Très-Haut\FTNT{Ge. 14:18.}. Il alla au-devant d'Abraham lorsqu'il revenait de la défaite des rois, et il le bénit,
\VS{2}et auquel Abraham donna pour sa part la dîme de tout\FTNT{Ge. 14:20. Pour en savoir plus sur la dîme, voir les commentaires en De. 14:22, No. 18:21 et Mal. 3:10.}. Son nom signifie premièrement Roi de justice, et puis il a été Roi de Salem, c'est-à-dire, Roi de paix.
\VS{3}Il est sans père, sans mère, sans généalogie, n'ayant ni commencement de jours ni fin de vie, mais il est rendu semblable au Fils de Dieu. Il demeure Prêtre continuellement.
\TextTitle{La prêtrise de Melchisédek, supérieure à celle d'Aaron}
\VS{4}Considérez donc combien est grand celui à qui même Abraham, le patriarche, donna la dîme du butin.
\VS{5}Car, quant à ceux d'entre les fils de Lévi qui reçoivent la prêtrise, ils ont bien une ordonnance de dîmer le peuple selon la loi, c'est-à-dire, de dîmer leurs frères, bien qu'ils soient sortis des reins d'Abraham.
\VS{6}Mais celui qui n'était pas de la même famille qu'eux reçut d'Abraham la dîme, et bénit celui qui avait les promesses.
\VS{7}Or sans contredit, celui qui est le moindre est béni par celui qui est le plus grand.
\VS{8}Et ici, ce sont les hommes mortels qui prennent les dîmes~; mais là, c'est celui de qui il est rendu témoignage qu'il est vivant.
\VS{9}Et pour ainsi dire, Lévi même qui prend des dîmes, les a payées en Abraham~;
\VS{10}car il était encore dans les reins de son père, quand Melchisédek alla au-devant de lui.
\TextTitle{La prêtrise selon l'ordre d'Aaron n'a rien amené à la perfection}
\VS{11}Si donc la perfection s'était trouvée dans la prêtrise lévitique, (car c'est sous elle que le peuple a reçu la loi) quel besoin était-il après cela qu'un autre prêtre se lève selon l'ordre de Melchisédek, et qui ne soit point nommé selon l'ordre d'Aaron~?
\VS{12}Or la prêtrise étant changée, il est nécessaire qu'il y ait aussi un changement de loi.
\VS{13}Car, celui à l'égard duquel ces choses sont dites, appartient à une autre tribu, de laquelle nul n'a assisté à l'autel~;
\VS{14}car il est évident que notre Seigneur est descendu de la tribu de Juda\FTNT{Mt. 1:2.}, à l'égard de laquelle Moïse n'a rien dit de la prêtrise.
\VS{15}Et cela est encore plus incontestable, en ce qu'un autre prêtre, à la ressemblance de Melchisédek, est suscité~;
\VS{16}qui n'a point été fait prêtre selon la loi du commandement charnel, mais selon la puissance de la vie impérissable.
\VS{17}Car Dieu lui rend ce témoignage~: Tu es prêtre éternellement, selon l'ordre de Melchisédek.
\VS{18}Or il se fait une abolition du commandement qui a précédé, à cause de sa faiblesse, et parce qu'il ne pouvait point profiter.
\VS{19}Car la loi n'a rien amené à la perfection, mais ce qui a amené à la perfection, c'est ce qui a été introduit par-dessus, à savoir une meilleure espérance, par laquelle nous approchons de Dieu.
\VS{20}D'autant plus, même que cela n'a pas été sans serment,
\VS{21}car les Lévites sont devenus prêtres sans serment, mais celui-ci l'est devenu avec serment par celui qui lui a dit~: Le Seigneur l'a juré, et il ne s'en repentira pas\FTNT{Voir Ps. 110:4}~: Tu es prêtre éternellement, selon l'ordre de Melchisédek.
\VS{22}C'est donc d'une alliance d'autant plus excellente que Jésus a été fait le garant.
\TextTitle{Les prêtres sont mortels, seul Christ est éternel}
\VS{23}Et quant aux prêtres, il y en a eu plusieurs qui se sont succédés parce que la mort les empêchait d'être perpétuels.
\VS{24}Mais lui, parce qu'il demeure éternellement, possède une prêtrise qui n'est pas transmissible.
\VS{25}C'est pourquoi aussi il peut sauver parfaitement ceux qui s'approchent de Dieu par lui, étant toujours vivant pour intercéder\FTNT{Le Seigneur Jésus-Christ est le modèle parfait en ce qui concerne la prière d'intercession. Il se tient devant le Père pour nous. En tant qu'homme (1 Ti. 2:5) et Grand-Prêtre, il se tient entre le Père et l'homme pécheur, comme le faisaient les prêtres sous la loi mosaïque. Voir Lu. 22:31-32~; Ro. 8:34~; 1 Jn. 2:1-2.} pour eux.
\VS{26}Or il nous était convenable d'avoir un tel Grand-Prêtre, saint, innocent, sans tache, séparé des pécheurs, et élevé au-dessus des cieux,
\VS{27}qui n'avait pas besoin, comme les grands-prêtres, d'offrir tous les jours des sacrifices, premièrement pour ses péchés, et ensuite pour ceux du peuple, vu qu'il a fait cela une fois, s'étant offert lui-même.
\VS{28}Car, la loi établit grands-prêtres des hommes faibles~; mais la parole du serment qui a été fait après la loi, établit le Fils, qui est parfait pour toujours.
\Chap{8}
\TextTitle{L'ancienne prêtrise~: L'ombre des choses célestes}
\VerseOne{}La chose principale de notre discours, c'est que nous avons un tel Grand-Prêtre, qui est assis à la droite du trône de la majesté de Dieu dans les cieux,
\VS{2}serviteur du sanctuaire, et du véritable tabernacle, que le Seigneur a dressé et non pas les hommes.
\VS{3}Car tout grand-prêtre est établi pour offrir des offrandes et des sacrifices~; c'est pourquoi il est nécessaire que celui-ci ait aussi quelque chose à offrir.
\VS{4}Vu même que s'il était sur la terre, il ne serait pas prêtre, pendant qu'il y aurait encore des prêtres qui offrent les offrandes selon la loi~;
\VS{5}lesquels font le service dans le lieu qui n'est que l'image et l'ombre des choses célestes, selon que Dieu le dit à Moïse, quand il devait achever le tabernacle~: Or prends garde, lui dit-il, de faire toutes choses selon le modèle qui t'a été montré sur la montagne\FTNT{Ex. 25:40.}.
\TextTitle{Christ, le Médiateur d'une alliance plus excellente}
\VS{6}Mais maintenant, notre Grand-Prêtre a obtenu un service d'autant supérieur qu'il est le Médiateur d'une alliance plus excellente, qui a été établie sur de meilleures promesses.
\TextTitle{Les prophètes ont annoncé la Première Alliance}
\VS{7}En effet, si la Première Alliance avait été irréprochable, il n'y aurait pas eu lieu d'en chercher une seconde.
\VS{8}Car en censurant les Juifs, Dieu leur dit~: Voici, les jours viendront, dit le Seigneur, où je traiterai avec la maison d'Israël et avec la maison de Juda une Alliance Nouvelle,
\VS{9}non selon l'alliance que je traitai avec leurs pères, le jour où je les saisis par la main pour les tirer du pays d'Egypte~; car ils n'ont pas persévéré dans mon alliance, c'est pourquoi je les ai méprisés, dit le Seigneur.
\VS{10}Mais voici l'alliance que je traiterai, après ces jours-là, avec la maison d'Israël, dit le Seigneur~: Je mettrai mes lois dans leur esprit, et je les écrirai dans leur cœur, je serai leur Dieu, et ils seront mon peuple.
\VS{11}Personne n'enseignera plus son prochain, ni personne son frère, en disant~: Connais le Seigneur~! Parce que tous me connaîtront, depuis le plus petit jusqu'au plus grand d'entre eux~;
\VS{12}car je serai miséricordieux par rapport à leurs injustices, et je ne me souviendrai plus de leurs péchés, ni de leurs iniquités\FTNT{Jé. 31:31-34.}.
\VS{13}En disant une Nouvelle Alliance, il a déclaré vieille la première~; or, ce qui devient vieux et ancien, est près d'être aboli.
\Chap{9}
\TextTitle{Les ordonnances et le sanctuaire de la Première Alliance~: Des symboles}
\VerseOne{}En vérité, la Première Alliance avait aussi des ordonnances touchant le service divin, et un sanctuaire terrestre.\FTNT{Ex. 25:1-9.}.
\VS{2}Car il fut construit un premier tabernacle, appelé le lieu saint, dans lequel étaient le chandelier, et la table, et les pains de proposition\FTNT{Ex. 25:30.}.
\VS{3}Et après le second voile\FTNT{Ex. 26:31-35.} était le tabernacle, qui était appelé le Saint des saints,
\VS{4}ayant un encensoir d'or\FTNT{Encensoir ou autel d'or pour les parfums~: Lé. 16:12.}, et l'arche de l'alliance\FTNT{Ex. 25:10.}, entièrement couverte d'or tout autour, dans laquelle était le vase d'or\FTNT{Ex. 16:33.} où était la manne, et la verge d'Aaron\FTNT{No. 17:1-10.} qui avait fleuri, et les tables de l'alliance\FTNT{Les tables de l'alliance ou tables du témoignage~: Ex. 34:29~; De. 10:2-5.}.
\VS{5}Et au-dessus de l'arche étaient les chérubins de la gloire, couvrant de leur ombre le propitiatoire\FTNT{Propitiatoire ou couvercle de l'arche de l'alliance~: Lé. 9:7~; Lé. 16:15-17.}. Ce n'est pas le moment de parler en détail là-dessus.
\VS{6}Or ces choses étant ainsi disposées, les prêtres qui font le service entrent en tout temps dans le premier tabernacle\FTNT{No. 28:3.}~;
\VS{7}mais seul le grand-prêtre entre dans le second une fois par an, non sans y porter du sang, qu'il offre pour lui-même et pour les péchés du peuple\FTNT{Lé. 16:34.}.
\VS{8}Le Saint-Esprit faisant connaitre par là que le chemin du Saint des saints n'était pas encore manifesté, tandis que le premier tabernacle était encore debout,
\VS{9}lequel était une figure destinée pour le temps présent, durant lequel étaient offerts des offrandes et des sacrifices qui ne pouvaient point sanctifier la conscience de celui qui faisait le service,
\VS{10}ordonnés seulement en aliments, et en breuvages, en diverses ablutions, et en des cérémonies charnelles, jusqu'au temps de la réforme.
\TextTitle{La réalité du sacrifice s'accomplit en Christ}
\VS{11}Mais Christ est venu comme Grand-Prêtre des biens à venir~; il a traversé un tabernacle plus excellent et plus parfait, qui n'est pas un tabernacle construit de main d'homme, c'est-à-dire, qui n'est pas de cette création~;
\VS{12}et il est entré une fois pour toutes dans le Saint des saints, non avec le sang des veaux ou des boucs, mais avec son propre sang, après avoir obtenu une rédemption éternelle.
\VS{13}Car si le sang des taureaux et des boucs, et la cendre de la génisse\FTNT{No. 19:1-12.}, répandue sur ceux qui sont souillés, sanctifient et procurent la pureté de la chair,
\VS{14}combien plus le sang de Christ, qui, par l'Esprit éternel, s'est offert lui-même à Dieu sans nulle tache, purifiera-t-il votre conscience des œuvres mortes, pour servir le Dieu vivant~?
\VS{15}C'est pourquoi il est le Médiateur de la Nouvelle Alliance, afin que, la mort étant intervenue pour la rançon des transgressions commises sous la Première Alliance, ceux qui ont été appelés reçoivent l'héritage éternel qui leur a été promis.
\TextTitle{Les clauses du testament du Messie}
\VS{16}Car là où il y a un testament, il est nécessaire que la mort du testateur intervienne,
\VS{17}parce que c'est par la mort du testateur qu'un testament est rendu ferme, puisqu'il n'a aucune force tant que le testateur est en vie.
\VS{18}C'est pourquoi la Première Alliance elle-même n'a point été confirmée sans le sang.
\VS{19}Car Moïse, après avoir prononcé devant tout le peuple tous les commandements de la loi, prit le sang des veaux et des boucs, avec de l'eau, et de la laine écarlate, et de l'hysope~; et il en fit l'aspersion sur le livre et sur tout le peuple, en disant~:
\VS{20}Ceci est le sang de l'Alliance que Dieu vous a ordonné d'observer\FTNT{Ex. 24:3-8.}.
\VS{21}Puis il fit aussi aspersion avec du sang sur le tabernacle et sur tous les ustensiles du service\FTNT{Ex. 29:12~; Ex. 29:36.}.
\VS{22}Et presque toutes choses, selon la loi, sont purifiées par le sang, et sans effusion de sang il n'y a pas de rémission des péchés.
\TextTitle{Un sacrifice plus excellent\FTNTT{Lé. 16:33}}
\VS{23}Il a donc fallu que les choses qui représentaient celles qui sont aux cieux, soient purifiées par de telles choses, mais que les célestes le soient par des sacrifices plus excellents que ceux-là.
\VS{24}Car Christ n'est pas entré dans un sanctuaire fait de main d'homme, et qui n'était que la figure du véritable, mais il est entré dans le ciel même, afin de comparaître maintenant pour nous devant la face de Dieu.
\VS{25}Et ce n'est pas pour s'offrir lui-même plusieurs fois qu'il y est entré, ainsi que le grand-prêtre entre dans le Saint des saints, chaque année, avec un autre sang~;
\VS{26}autrement, il aurait fallu qu'il ait souffert plusieurs fois depuis la création du monde~; mais maintenant, à la fin des siècles, il a paru une seule fois pour l'abolition du péché par son sacrifice.
\VS{27}Et comme il est réservé aux hommes de mourir une seule fois\FTNT{Ce passage réfute la doctrine de la réincarnation.}, et après cela suit le jugement,
\VS{28}de même aussi Christ, qui s'est offert une seule fois pour ôter les péchés de plusieurs, apparaîtra sans péché une seconde fois à ceux qui l'attendent pour le salut.
\Chap{10}
\TextTitle{Le sacrifice unique de Christ est supérieur à tous les sacrifices}
\VerseOne{}Car la loi qui possède l'ombre des biens à venir, et non l'image exacte des choses, ne peut jamais, par les mêmes sacrifices que l'on offre continuellement chaque année, sanctifier ceux qui s'y attachent.
\VS{2}Autrement, n'auraient-ils pas cessé d'être offerts~? Parce que les adorateurs, une fois expurgés, n'auraient plus eu conscience des péchés.
\VS{3}Or le souvenir des péchés est réitéré dans ces sacrifices chaque année~;
\VS{4}car il est impossible que le sang des taureaux et des boucs ôte les péchés.
\VS{5}C'est pourquoi Jésus-Christ, en entrant dans le monde, a dit~: Tu n'as pas voulu de sacrifice, ni d'offrande, mais tu m'as formé un corps~;
\VS{6}tu n'as pas pris plaisir aux holocaustes, ni aux sacrifices pour le péché\FTNT{Ps. 40:7-9.}.
\VS{7}Alors j'ai dit~: Me voici, je viens, il est écrit de moi au commencement du livre~: Que je fasse, ô Dieu, ta volonté~!
\VS{8}Après avoir dit d'abord~: Tu n'as pas voulu de sacrifice, ni d'offrande, ni d'holocauste, ni d'offrande pour le péché et tu n'y as point pris plaisir, lesquelles choses sont pourtant offertes selon la loi, alors il dit~: Me voici, je viens afin de faire, ô Dieu, ta volonté~!
\VS{9}Il abolit ainsi le premier afin d'établir le second.
\VS{10}Or c'est par cette volonté que nous sommes sanctifiés, à savoir par l'offrande du corps de Jésus-Christ qui a été faite une fois pour toutes.
\VS{11}De plus, tout prêtre fait chaque jour le service et offre souvent les mêmes sacrifices, qui ne peuvent jamais ôter les péchés,
\VS{12}mais lui, après avoir offert un seul sacrifice pour les péchés, s'est assis pour toujours à la droite de Dieu,
\VS{13}attendant désormais que ses ennemis soient mis pour le marchepied de ses pieds.
\VS{14}Car, par une seule offrande, il a rendu parfaits pour toujours ceux qui sont sanctifiés.
\VS{15}Et c'est aussi ce que le Saint-Esprit nous témoigne~; car, après avoir dit premièrement~:
\VS{16}Voici l'alliance que je ferai avec eux, après ces jours-là, dit le Seigneur\FTNT{Voir Jé. 31:31-34.}~: C'est que je mettrai mes lois dans leur cœur, et je les écrirai dans leur esprit~;
\VS{17}et je ne me souviendrai plus de leurs péchés, ni de leurs iniquités.
\VS{18}Or, là où les péchés sont pardonnés, il n'y a plus d'offrande pour le péché.
\TextTitle{Exhortation à s'approcher de Dieu avec foi}
\VS{19}Ainsi donc, mes frères, nous avons la liberté d'entrer dans le Saint des saints au moyen du sang de Jésus,
\VS{20}qui est le chemin\FTNT{Jésus est le chemin qui conduit au Saint des saints, à la vie (Voir Jn. 14:6), et ce chemin n'était pas encore manifesté avant sa naissance. Hé. 9:8.} nouveau et vivant qu'il a inauguré pour nous à travers le voile, c'est-à-dire sa propre chair,
\VS{21}et ayant un Grand-Prêtre établi sur la maison de Dieu,
\VS{22}approchons-nous de lui avec un cœur sincère, et une foi inébranlable, ayant les cœurs purifiés d'une mauvaise conscience, et le corps lavé d'une eau pure.
\VS{23}Retenons fermement la profession de notre espérance, car celui qui nous a fait la promesse est fidèle.
\VS{24}Veillons les uns sur les autres pour nous exciter à la charité et aux bonnes œuvres.
\VS{25}N'abandonnons pas notre assemblée\FTNT{Assemblée~: Du grec «~episunagoge~» qui veut dire «~être assemblé en un lieu, assemblée religieuse des chrétiens~». Il est question de ne pas abandonner la communion fraternelle et non une église locale. En effet, il est du devoir du chrétien de se séparer des faux frères de peur d'être entraîné dans leur égarement (Mt. 18:15-17~; 1 Co. 5:11~; 1 Co. 15:33).}, comme c'est la coutume de quelques-uns~; mais exhortons-nous les uns les autres, et cela d'autant plus que vous voyez approcher le jour.
\TextTitle{Ne pas mépriser le sacrifice de Christ}
\VS{26}Car, si nous péchons volontairement après avoir reçu la connaissance de la vérité, il ne reste plus de sacrifice pour les péchés,
\VS{27}mais une attente terrible du jugement et l'ardeur d'un feu qui doit dévorer les adversaires.
\VS{28}Si quelqu'un avait méprisé la loi de Moïse, il mourait sans miséricorde, sur la déposition de deux ou de trois témoins\FTNT{De. 17:6.}~;
\VS{29}de combien pires tourments pensez-vous donc que sera jugé digne celui qui aura foulé aux pieds le Fils de Dieu, et qui aura tenu pour une chose profane le sang de l'Alliance, par lequel il avait été sanctifié, et qui aura outragé l'Esprit de grâce~?
\VS{30}Car nous connaissons celui qui a dit~: C'est à moi que la vengeance appartient, et je le rendrai~! Dit le Seigneur. Et encore~: Le Seigneur jugera son peuple.\FTNT{De. 32:35-36.}
\VS{31}C'est une chose terrible que de tomber entre les mains du Dieu vivant.
\VS{32}Or rappelez-vous des premiers jours, où, après avoir été éclairés, vous avez soutenu un grand combat de souffrances,
\VS{33}ayant été, d'une part, exposés à la vue de tout le monde par des opprobres et des afflictions, et de l'autre, ayant participé aux maux de ceux qui ont souffert de semblables indignités.
\VS{34}Car vous avez aussi été participants de l'affliction de mes liens, et vous avez reçu avec joie l'enlèvement de vos biens, sachant en vous-mêmes que vous avez dans les cieux des biens meilleurs et permanents.
\VS{35}N'abandonnez donc pas cette fermeté que vous avez fait paraître, et qui sera bien récompensée.
\VS{36}Parce que vous avez besoin de patience, afin qu'après avoir fait la volonté de Dieu, vous receviez l'effet de sa promesse.
\TextTitle{La marche par la foi~: Exemples d'hommes et de femmes de foi}
\VS{37}Car, encore un peu de temps, et celui qui doit venir, viendra, et il ne tardera point.
\VS{38}Or le juste vivra de la foi~; mais si quelqu'un se retire, mon âme ne prend point de plaisir en lui\FTNT{Ha. 2:4.}.
\VS{39}Mais pour nous, nous ne sommes pas de ceux qui se retirent~; ce serait notre perdition~; mais nous persévérons dans la foi, pour le salut de l'âme.
\Chap{11}
\VerseOne{}Or la foi rend présentes les choses qu'on espère, et elle est une démonstration de celles qu'on ne voit point.
\VS{2}Car c'est par elle que les anciens ont obtenu un bon témoignage.
\VS{3}Par la foi, nous comprenons que l'univers a été fait par la parole de Dieu, de sorte que les choses qui se voient, n'ont pas été faites des choses visibles.
\VS{4}Par la foi, Abel\FTNT{Ge. 4:3-5.} offrit à Dieu un sacrifice plus excellent que Caïn~; et par elle il obtînt le témoignage d'être juste, parce que Dieu rendait témoignage de ses offrandes~; et c'est par elle qu'il parle encore, quoique mort.
\VS{5}Par la foi, Hénoc\FTNT{Ge. 5:22-24.} fut enlevé pour ne pas voir la mort, et il ne parut plus parce que Dieu l'avait enlevé~; car, avant qu'il soit enlevé, il avait obtenu le témoignage d'avoir été agréable à Dieu.
\VS{6}Or il est impossible de lui être agréable sans la foi~; car il faut que celui qui vient à Dieu, croie que Dieu est, et qu'il est le rémunérateur de ceux qui le cherchent.
\VS{7}Par la foi, Noé\FTNT{Ge. 6:14-22.}, ayant été divinement averti des choses qui ne se voyaient point encore, craignit, et bâtit l'arche pour la conservation de sa famille~; et par cette arche, il condamna le monde, et devint héritier de la justice qui est selon la foi.
\VS{8}Par la foi, Abraham\FTNT{Ge 12:1-4.}, étant appelé, obéit, pour aller sur la terre, qu'il devait recevoir en héritage, et il partit sans savoir où il allait.
\VS{9}Par la foi, il demeura comme étranger sur la terre, qui lui avait été promise, comme si elle ne lui avait point appartenu, demeurant sous des tentes avec Isaac et Jacob, qui étaient héritiers avec lui de la même promesse.
\VS{10}Car il attendait la cité qui a des fondements, celle dont Dieu est l'architecte et le constructeur.
\VS{11}Par la foi, aussi Sara\FTNT{Ge. 21:1-2.} reçut la force de concevoir un enfant, et elle enfanta hors d'âge, parce qu'elle fut persuadée que celui qui le lui avait promis, était fidèle.
\VS{12}C'est pourquoi d'un seul homme, et qui était déjà affaibli, il est né une multitude aussi nombreuse que les étoiles du ciel, et que le sable du bord de la mer, qui ne peut se compter\FTNT{Ge. 22:17.}.
\VS{13}Tous ceux-ci sont morts dans la foi, sans avoir reçu les choses dont ils avaient eu les promesses, mais ils les ont vues de loin, crues, et saluées, et ils ont fait profession qu'ils étaient étrangers et voyageurs sur la terre\FTNT{1 Pi. 2:11.}.
\VS{14}Car ceux qui tiennent ces discours montrent clairement qu'ils cherchent encore leur patrie.
\VS{15}Et certes, s'ils avaient eu en vue celle d'où ils étaient sortis, ils auraient eu le temps d'y retourner.
\VS{16}Mais maintenant, ils en désirent une meilleure, c'est-à-dire une céleste. C'est pourquoi Dieu n'a pas honte d'être appelé leur Dieu, parce qu'il leur a préparé une cité\FTNT{Jn. 14:2~; Ap. 21:2.}.
\VS{17}Par la foi, Abraham étant éprouvé, offrit Isaac~; celui qui avait reçu les promesses offrit même son fils unique\FTNT{Ge. 22:1.},
\VS{18}à l'égard duquel il lui avait été dit~: Les descendants d'Isaac seront ta véritable postérité\FTNT{Ge. 21:12.}.
\VS{19}Ayant estimé que Dieu pouvait même le ressusciter d'entre les morts~; c'est pourquoi aussi il le recouvra par une espèce de résurrection.
\VS{20}Par la foi, Isaac bénit Jacob et Esaü, en vue des choses à venir\FTNT{Ge. 27:26-40.}.
\VS{21}Par la foi, Jacob, mourant, bénit chacun des fils de Joseph\FTNT{Ge. 48:1-22.}, et adora Dieu, appuyé sur l'extrémité de son bâton\FTNT{Ge. 47:31.}.
\VS{22}Par la foi, Joseph mourant fit mention de la sortie des enfants d'Israël, et il donna des ordres au sujet de ses os\FTNT{Ge. 50:24-25.}.
\VS{23}Par la foi, Moïse\FTNT{Ex. 2:1-3.}, à sa naissance, fut caché pendant trois mois par son père et sa mère, parce qu'ils virent que l'enfant était beau, et ils ne craignirent pas l'ordre du roi.
\VS{24}Par la foi, Moïse, devenu grand, refusa d'être nommé fils de la fille de Pharaon,
\VS{25}choisissant plutôt d'être affligé avec le peuple de Dieu, que de jouir pour un peu de temps des délices du péché.
\VS{26}Et ayant estimé que l'opprobre de Christ était un plus grand trésor que les richesses de l'Egypte, parce qu'il avait égard à la rémunération.
\VS{27}Par la foi, il quitta l'Egypte, sans craindre la fureur du roi~; car il demeura ferme, comme voyant celui qui est invisible.
\VS{28}Par la foi, il fit la Pâque et l'aspersion du sang, afin que le destructeur qui tuait les premiers-nés, ne touche pas aux premiers-nés des Israélites\FTNT{Ex. 12:1-51.}.
\VS{29}Par la foi, ils traversèrent la Mer Rouge, comme un lieu sec, ce que les Egyptiens essayèrent de tenter, ils furent engloutis dans les eaux\FTNT{Ex. 14:13-31.}.
\VS{30}Par la foi, les murs de Jéricho tombèrent, après qu'on en eut fait le tour pendant sept jours\FTNT{Jos. 6:1-20.}.
\VS{31}Par la foi, Rahab, la prostituée, ne périt pas avec les incrédules, parce qu'elle avait reçu les espions et les avait renvoyés en paix\FTNT{Jos. 2:1-21~; Jos. 6:23.}.
\VS{32}Et que dirai-je encore~? Car le temps me manquerait si je voulais parler de Gédéon\FTNT{Jg. 6:11.}, et de Barak\FTNT{Jg. 4:6.}, et de Samson\FTNT{Jg. 13:24.}, et de Jephté\FTNT{Jg. 11:1.}, et de David\FTNT{1 S. 16-17.}, et de Samuel\FTNT{1 S. et 2 S.}, et des prophètes,
\VS{33}qui par la foi combattirent des royaumes, exercèrent la justice, obtinrent des promesses, fermèrent la gueule des lions,
\VS{34}éteignirent la force du feu, échappèrent au tranchant des épées, des malades devinrent vigoureux, se montrèrent fort dans la bataille, et mirent en fuite des armées étrangères.
\VS{35}Des femmes recouvrèrent leurs morts par le moyen de la résurrection~; et d'autres furent livrés aux tourments et n'acceptèrent point d'être délivrés, afin d'obtenir une meilleure résurrection.
\VS{36}Et d'autres subirent les moqueries et le fouet, les chaînes et la prison~;
\VS{37}ils furent lapidés, sciés, subirent de rudes épreuves, ils furent mis à mort par le tranchant de l'épée, ils errèrent çà et là, vêtus de peaux de brebis et de chèvres, réduits à la misère, affligés, tourmentés,
\VS{38}eux dont le monde n'était pas digne, errant dans les déserts et dans les montagnes, et dans les cavernes et dans les trous de la terre.
\VS{39} Et quoiqu'ils aient tous été recommandables par leur foi, ils n'ont pourtant point reçu l'effet de la promesse,
\VS{40}Dieu ayant pourvu quelque chose de meilleur pour nous, en sorte qu'ils ne parviennent pas à la perfection sans nous.
\Chap{12}
\TextTitle{Fixer les regards sur Jésus}
\VerseOne{}Nous donc aussi, puisque nous sommes environnés d'une si grande nuée de témoins\FTNT{Témoin~: du grec «~martus~», terme qui dans un sens légal et historique signifie «~celui qui est spectateur d'une chose~». Dans un sens éthique, il est question de «~ceux qui ont prouvé la force et l'authenticité de leur foi en Christ en supportant une mort violente~». «~Martus~» a donné le mot «~martyr~» en français.}, rejetons tout fardeau, et le péché qui nous enveloppe si aisément, et poursuivons constamment la course qui nous est proposée,
\VS{2}portant les yeux sur Jésus, le chef et le consommateur de la foi qui en échange de la joie qui lui était réservée, il a souffert la croix, ayant méprisé la honte, et s'est assis à la droite du trône de Dieu.
\VS{3}C'est pourquoi, considérez soigneusement celui qui a supporté contre sa personne une telle opposition de la part des pécheurs, afin que vous ne succombiez point, en perdant courage.
\VS{4}Vous n'avez pas encore résisté jusqu'à répandre votre sang en combattant contre le péché.
\TextTitle{La correction du Père}
\VS{5}Et cependant vous avez oublié l'exhortation qui vous est adressée comme à ses fils, disant~: Mon fils, ne méprise pas le châtiment du Seigneur, et ne perds point courage lorsqu'il te reprend~;
\VS{6}car le Seigneur châtie celui qu'il aime, et il frappe de la verge tous ceux qu'il reconnaît pour ses fils\FTNT{Pr. 3:11-12.}.
\VS{7}Si vous endurez le châtiment, Dieu se présente à vous comme à ses fils~; car qui est le fils que le père ne châtie point~?
\VS{8}Mais si vous êtes sans châtiment auquel tous participent, vous êtes donc des enfants illégitimes, et non pas des fils.
\VS{9}Et puisque nos pères selon la chair nous ont châtiés, et que malgré cela nous les avons respectés, ne serons-nous pas beaucoup plus soumis au Père des esprits, pour avoir la vie~?
\VS{10}Car par rapport à ceux-là, ils nous châtiaient pour un peu de temps, suivant leur volonté, mais celui-ci nous châtie pour notre profit, afin que nous soyons participants de sa sainteté.
\VS{11}Or tout châtiment ne semble pas sur l'heure être un sujet de joie, mais de tristesse~; mais ensuite il produit un fruit paisible de justice à ceux qui sont exercés par ce moyen.
\VS{12}Fortifiez donc vos mains languissantes et vos genoux affaiblis~;
\VS{13}et suivez avec vos pieds des chemins droits, afin que ce qui est boiteux ne dévie pas, mais plutôt se consolide.
\VS{14}Recherchez la paix avec tous, et la sanctification, sans laquelle nul ne verra le Seigneur.
\TextTitle{Que nul ne se prive de la grâce de Dieu !}
\VS{15}Veillez à ce que personne ne se prive de la grâce de Dieu~; à ce qu'aucune racine d'amertume, poussant des rejetons, ne vous trouble, et que plusieurs n'en soient souillés par elles~;
\VS{16}que nul de vous ne soit fornicateur, ou profane comme Esaü, qui pour un aliment vendit son droit d'aînesse\FTNT{Ge. 25:33}.
\VS{17}Car vous savez que plus tard, désirant hériter la bénédiction, il fut rejeté, car il ne trouva point de lieu à la repentance, quoiqu'il l'ait demandée avec larmes.
\TextTitle{L'Eglise véritable s'est approchée de Sion}
\VS{18}Vous ne vous êtes pas approchés d'une montagne qu'on pouvait toucher avec la main\FTNT{Ex. 19:12.}, ni du feu brûlant, ni de la nuée épaisse, ni des ténèbres, ni de la tempête,
\VS{19}ni du retentissement de la trompette, ni du son des paroles, au sujet duquel ceux qui l'entendirent prièrent que la parole ne leur soit plus adressée\FTNT{Ex. 20:18-26.},
\VS{20}car ils ne pouvaient pas supporter ce qui était ordonné, que si même une bête touche la montagne, elle sera lapidée ou percée d'un dard\FTNT{Ex. 19:13.}.
\VS{21}Et ce spectacle était si terrible que Moïse dit~: Je suis épouvanté et tout tremblant~!
\VS{22}Mais vous vous êtes approchés de la montagne de Sion, de la Cité du Dieu vivant, la Jérusalem céleste, d'une multitude innombrable d'anges,
\VS{23}et de l'assemblée et de l'Eglise des premiers-nés qui sont inscrits dans les cieux, du Dieu qui est le juge de tous, et des esprits des justes qui ont été rendus parfaits,
\VS{24}de Jésus, qui est le Médiateur de la Nouvelle Alliance, et du sang de l'aspersion, qui prononce des meilleurs choses que celui d'Abel.
\TextTitle{Exhortation à la crainte de Dieu}
\VS{25}Prenez garde de ne pas mépriser celui qui vous parle~; car si ceux qui méprisèrent celui qui leur parlait sur la terre, n'ont pas échappé, nous serons punis beaucoup plus, si nous nous détournons de celui qui parle des cieux,
\VS{26}lui, dont la voix ébranla alors la terre, mais à l'égard du temps présent, il a fait cette promesse, disant~: J'ébranlerai encore une fois non seulement la terre, mais aussi le ciel\FTNT{Ag. 2:6.}.
\VS{27}Or ces mots~: Une fois encore, marquent le changement des choses ébranlées, comme étant faites pour un temps, afin que celles qui sont inébranlables demeurent.
\VS{28}C'est pourquoi, saisissant le Royaume qui ne peut point être ébranlé, retenons la grâce par laquelle nous servions Dieu, en sorte que nous lui soyons agréables avec respect et avec crainte,
\VS{29}car notre Dieu est aussi un feu dévorant\FTNT{De. 4:24.}.
\Chap{13}
\TextTitle{Exhortations~; invariabilité de Christ}
\VerseOne{}Que la charité fraternelle demeure dans vos cœurs.
\VS{2}N'oubliez pas l'hospitalité~; car, par elle, quelques-uns ont logé des anges sans le savoir.
\VS{3}Souvenez-vous des prisonniers, comme si vous étiez emprisonnés avec eux~; et de ceux qui sont maltraités, comme étant aussi vous-mêmes du même corps.
\VS{4}Le mariage est honorable entre tous, et le lit sans souillure~; mais Dieu jugera les fornicateurs et les adultères.
\VS{5}Que votre conduite soit sans avarice, étant contents de ce que vous avez présentement~; car lui-même a dit~: Je ne te délaisserai point, et je ne t'abandonnerai point\FTNT{De. 31:6.}.
\VS{6}De sorte que nous pouvons dire avec assurance~: Le Seigneur est mon aide, et je ne craindrai point ce que l'homme pourrait me faire\FTNT{Ps. 118:6.}.
\VS{7}Souvenez-vous de vos conducteurs qui vous ont annoncé la parole de Dieu~; considérez quelle a été la fin de leur vie, et imitez leur foi.
\VS{8}Jésus-Christ est le même hier, aujourd'hui, et il l'est aussi éternellement.
\VS{9}Ne soyez point emportés çà et là par des doctrines diverses et étrangères~; car il est bon que le cœur soit affermi par la grâce, et non point par les aliments, lesquelles n'ont en rien profité à ceux qui s'y sont attachés.
\TextTitle{Porter ses regards sur la cité céleste}
\VS{10}Nous avons un autel dont ceux qui servent dans le tabernacle n'ont pas le droit de manger.
\VS{11}Car les corps des animaux, dont le sang est porté dans le sanctuaire par le grand-prêtre pour le péché, sont brûlés hors du camp.
\VS{12}C'est pourquoi aussi Jésus, afin de sanctifier le peuple par son propre sang, a souffert hors de la porte\FTNT{Ex. 29:14. Jésus a souffert hors de Jérusalem (Jn. 19:17-18).}.
\VS{13}Sortons donc vers lui, hors du camp\FTNT{Le mot «~camp~» dans ce passage vient du grec «~parambole~», terme faisant référence au judaïsme antique dans lequel s'étaient embourbés les chrétiens d'origine hébraïque. Aujourd'hui, il représente plutôt le christianisme paganisé, essentiellement basé sur la loi de Moïse et constituant une prison qui empêche certains enfants de Dieu de vivre pleinement leur liberté en Christ.}, en portant son opprobre.
\VS{14}Car nous n'avons point ici-bas de cité permanente, mais nous recherchons celle qui est à venir.
\TextTitle{Le sacrifice de louange et du serviteur de Dieu}
\VS{15}Offrons donc par lui sans cesse à Dieu un sacrifice de louange, c'est-à-dire, le fruit des lèvres, en confessant son Nom.
\VS{16}Or n'oubliez pas la bienfaisance et de faire part de vos biens, car Dieu prend plaisir à de tels sacrifices.
\TextTitle{L'obéissance aux conducteurs}
\VS{17}Obéissez\FTNT{Le terme «~obéissez~», en grec «~peitho~», veut dire «~se laisser persuader par des mots~». Il signifie aussi «~donner avec persuasion l'envie à quelqu'un de faire quelque chose en le rassurant~». Par conséquent, les conducteurs doivent comprendre que la soumission et l'obéissance des chrétiens n'a rien à voir avec la dictature et l'autoritarisme. Ils doivent les rassurer et les convaincre - car tout ce qui n'est pas fait avec foi est péché (Ro. 14:23) - et ne pas tyranniser leurs frères en les obligeant à leur obéir (Mt. 20:25~; 1 Pi. 5:2-3).} à vos conducteurs, et soyez-leur soumis, car ils veillent pour vos âmes, comme devant en rendre compte~; afin que ce qu'ils en font, ils le fassent avec joie, et non en gémissant, car cela ne vous serait pas profitable.
\VS{18} Priez pour nous, car nous nous assurons que nous avons une bonne conscience, désirant nous conduire honnêtement parmi tous.
\VS{19}C'est avec instance que je vous demande de le faire, afin que je vous sois rendu plus tôt.
\TextTitle{Bénédictions et salutations}
\VS{20}Que le Dieu de paix, qui a ramené d'entre les morts le grand Pasteur des brebis, par le sang de l'Alliance éternelle, notre Seigneur Jésus-Christ,
\VS{21}vous rende capables de toute bonne œuvre pour faire sa volonté~; qu'il fasse en vous ce qui lui est agréable par Jésus-Christ~; auquel soit la gloire aux siècles des siècles~! Amen~!
\VS{22}Aussi, mes frères, je vous prie de supporter la parole d'exhortation, car je vous ai écrit en peu de mots.
\VS{23}Sachez que notre frère Timothée a été relâché~; s'il vient bientôt, je vous verrai avec lui.
\VS{24}Saluez tous vos conducteurs, et tous les saints. Ceux d'Italie vous saluent.
\VS{25}Que la grâce soit avec vous tous~! Amen~!
\PPE{}
\end{multicols}

%\clearpage\ShortTitle{1 Jean}\BookTitle{1 Jean}\BFont
\noindent\hrulefill
{\footnotesize
\textit{
\bigskip
{\centering{}
\\Auteur : Jean
\\Signification : Yahweh a fait grâce
\\Thème : La communion fraternelle, la connaissance et l'amour
\\Date de rédaction : Env. 85 ap. J.-C.\\}
}
%\bigskip
\textit{
\\Cette épître, écrite par Jean à Ephèse, était destinée aux églises de la province d’Asie qu’il connaissait bien. Il souhaite rendre leur joie parfaite en fortifiant leur foi en Christ et en leur donnant l’assurance de la vie éternelle ;  tout en les mettant en garde contre les faux docteurs.\bigskip
}
}
\par\nobreak\noindent\hrulefill
\begin{multicols}{2}
\Chap{1}
\TextTitle{La Parole incarnée}
\VerseOne{}Ce qui était dès le commencement, ce que nous avons entendu, ce que nous avons vu de nos propres yeux, ce que nous avons contemplé, et que nos propres mains ont touché concernant la Parole de vie,
\VS{2}car la vie a été manifestée, et nous l'avons vue et nous lui rendons témoignage, et nous vous annonçons la vie éternelle, qui était avec le Père, et qui nous a été manifestée.
\TextTitle{Communion avec le Père et le Fils}
\VS{3}Ce que nous avons vu dis-je, et ce que nous avons entendu, nous vous l'annonçons, afin que vous soyez en communion avec nous, et que notre communion soit avec le Père et avec son Fils Jésus-Christ.
\VS{4}Et nous vous écrivons ces choses, afin que votre joie soit parfaite.
\TextTitle{De la communion avec Dieu, qui est Lumière, et de la confession des péchés}
\TextTitle{[Conditions de la communion avec Dieu
\\a. Position de l'enfant de Dieu dans la Lumière]}
\VS{5}Or c'est ici la déclaration que nous avons entendue de lui et que nous vous annonçons, à savoir que Dieu est Lumière et qu'il n'y a point en lui de ténèbres.
\VS{6}Si nous disons que nous sommes en communion avec lui, et que nous marchions dans les ténèbres, nous mentons, et nous n'agissons pas selon la vérité.
\VS{7}Mais si nous marchons dans la Lumière, comme Dieu est dans la Lumière, nous sommes en communion les uns avec les autres, et le sang de son Fils Jésus-Christ nous purifie de tout péché.
\TextTitle{b. Reconnaissance de la présence du péché en nous}
\VS{8}Si nous disons que nous n'avons point de péché, nous nous séduisons nous-mêmes, et la vérité n'est point en nous.
\TextTitle{c. la confession des péchés, le pardon et la purification}
\VS{9}Si nous confessons nos péchés, il est fidèle et juste pour nous les pardonner, et pour nous purifier de toute iniquité.
\VS{10}Si nous disons que nous n'avons point de péché, nous le faisons menteur, et sa Parole n'est point en nous.
\TextTitle{Celui qui connaît Jésus-Christ garde ses commandements}
\Chap{2}
\TextTitle{d. Christ, notre avocat pour nos péchés}
\VerseOne{}Mes petits-enfants, je vous écris ces choses afin que vous ne péchiez point. Et si quelqu'un a péché, nous avons un avocat\FTNT{Jésus, notre Avocat. Le mot grec «~parakletos~», traduit ici par «~avocat~», se trouve également en Jean 14 et 16, où il est traduit par «~Consolateur~» et s’applique au Saint-Esprit. Le Seigneur exerce la fonction d'avocat actuellement pour nous dans le ciel. Voir Ro. 8:33 ; Hé. 7:25.} auprès du Père, Jésus-Christ, le Juste.
\VS{2}Car c'est lui qui est la victime de propitiation pour nos péchés, et non seulement pour les nôtres, mais aussi pour ceux de tout le monde.
\TextTitle{e. reconnaissance de la sainteté de Dieu}
\VS{3}Et nous savons que nous l'avons connu, si nous gardons ses commandements.
\VS{4}Celui qui dit : Je l'ai connu, et qui ne garde point ses commandements, est un menteur, et il n'y a point de vérité en lui.
\VS{5}Mais celui qui garde sa Parole, l'amour de Dieu est véritablement parfait en lui : et c'est par cela que nous savons que nous sommes en lui.
\VS{6}Celui qui dit qu'il demeure en lui doit aussi vivre comme Jésus-Christ lui-même a vécu.
\VS{7}Mes frères, je ne vous écris point un commandement nouveau, mais un commandement ancien, que vous avez eu dès le commencement ; et ce commandement ancien c'est la Parole que vous avez entendue dès le commencement.
\VS{8}Cependant, le commandement que je vous écris est un commandement nouveau, c’est une chose véritable en lui et en vous, parce que les ténèbres sont passées, et que la véritable Lumière paraît déjà.
\VS{9}Celui qui dit qu'il est dans la Lumière, et qui hait son frère, est dans les ténèbres jusqu'à présent.
\VS{10}Celui qui aime son frère demeure dans la Lumière, et il n'y a rien en lui qui puisse le faire tomber.
\VS{11}Mais celui qui hait son frère est dans les ténèbres, et il marche dans les ténèbres, et il ne sait pas où il va car les ténèbres ont aveuglé ses yeux.
\TextTitle{Exhortations à la famille spirituelle}
\VS{12}Mes petits enfants, je vous écris parce que vos péchés vous sont pardonnés à cause de son Nom.
\VS{13}Pères, je vous écris parce que vous avez connu celui qui est dès le commencement. Jeunes gens, je vous écris parce que vous avez vaincu l’esprit du malin.
\VS{14}Jeunes enfants, je vous écris parce que vous avez connu le Père. Pères, je vous ai écrit parce que vous avez connu celui qui est dès le commencement. Jeunes gens, je vous ai écrit parce que vous êtes forts et que la Parole de Dieu demeure en vous, et que vous avez vaincu l’esprit du malin.
\TextTitle{Les enfants de Dieu ne doivent pas aimer le monde}
\VS{15}N'aimez point le monde ni les choses qui sont dans le monde ; si quelqu'un aime le monde, l'amour du Père n'est point en lui.
\VS{16}Car tout ce qui est dans le monde, c'est-à-dire la convoitise de la chair, la convoitise des yeux et l'orgueil de la vie, ne vient point du Père, mais vient du monde.
\VS{17}Et le monde passe, avec sa convoitise ; mais celui qui fait la volonté de Dieu demeure éternellement.
\TextTitle{Les enfants de Dieu mis en garde contre les apostats}
\VS{18}Petits enfants, c'est ici la dernière\FTNT{Dernière, du grec «~eschatos~», signifie «~dernier dans une succession dans le temps~». Voir Ge. 49:1-2.} heure ; et comme vous avez entendu que l'Antéchrist viendra, il y a maintenant plusieurs antéchrists ; et par là nous connaissons que c'est la dernière heure.
\VS{19}Ils sont sortis du milieu de nous, mais ils n'étaient pas des nôtres ; car s'ils avaient été des nôtres, ils seraient demeurés avec nous, mais c'est afin qu'il soit manifeste que tous ne sont point des nôtres.
\VS{20}Mais vous avez été oints par le Saint-Esprit, et vous connaissez toutes choses.
\VS{21}Je ne vous ai pas écrit comme si vous ne connaissiez point la vérité, mais parce que vous la connaissez, et qu'aucun mensonge ne vient de la vérité.
\VS{22}Qui est le menteur, sinon celui qui nie que Jésus est le Christ ? Celui-là est l'Antéchrist qui nie le Père et le Fils.
\VS{23}Quiconque nie le Fils, n'a point non plus le Père ; quiconque confesse le Fils, a aussi le Père.
\VS{24}Que ce que vous avez entendu dès le commencement demeure en vous, car si ce que vous avez entendu dès le commencement demeure en vous, vous demeurerez aussi dans le Fils et dans le Père.
\VS{25}Et c'est ici la promesse qu'il nous a faite, à savoir la vie éternelle.
\VS{26}Je vous ai écrit ces choses au sujet de ceux qui vous séduisent.
\VS{27}Mais l'onction que vous avez reçue de lui demeure en vous, et vous n'avez pas besoin qu'on vous enseigne ; mais comme la même onction vous enseigne toutes choses, qu'elle est véritable et n'est pas un mensonge, demeurez en lui selon les enseignements qu’elle vous a donnés.
\TextTitle{Exhortations à deumeurer en Christ}
\VS{28}Maintenant donc, mes petits enfants, demeurez en lui ; afin que quand il apparaîtra, nous ayons de l'assurance, et que nous ne soyons point confus devant lui lors de son avènement\FTNT{Avènement, du grec «~parousia~», veut dire «~l’arrivée~» ou «~la présence~». Lors de cette seconde venue, le Messie prendra son Epouse pour les noces, ensuite il posera ses pieds sur le Mont des Oliviers, détruira les armées de l'Antichrist, puis commencera son règne de mille ans. Voir Za. 14.}.
\VS{29}Si vous savez qu'il est juste, sachez que quiconque fait ce qui est juste est né de lui.
\Chap{3}
\VerseOne{}Voyez quelle charité le Père nous a témoignée, pour que nous soyons appelés enfants de Dieu ! Mais le monde ne nous connaît point, parce qu'il ne l'a point connu.
\VS{2}Mes bien-aimés, nous sommes maintenant enfants de Dieu, et ce que nous serons n'est pas encore manifesté ; or nous savons que lorsque le Fils de Dieu apparaîtra, nous serons semblables à lui, car nous le verrons tel qu'il est.
\VS{3}Et quiconque a cette espérance en lui se purifie, comme lui aussi est pur.
\TextTitle{Caractéristiques des enfants de Dieu et des enfants du diable}
\VS{4}Quiconque pèche, transgresse la loi, car le péché est la transgression de la loi.
\VS{5}Or vous savez qu'il est apparu pour ôter nos péchés ; et il n'y a point de péché en lui.
\VS{6}Quiconque demeure en lui ne pèche point ; quiconque pèche, ne l'a pas vu, et ne l'a pas connu.
\VS{7}Mes petits-enfants, que personne ne vous séduise. Celui qui fait ce qui est juste est une personne juste, comme Jésus-Christ est juste.
\VS{8}Celui qui vit dans le péché est du diable, car le diable pèche dès le commencement. Or le Fils de Dieu est apparu afin de détruire les œuvres du diable.
\VS{9}Quiconque est né de Dieu ne vit pas dans le péché, car la semence de Dieu demeure en lui ; et il ne peut pécher, parce qu'il est né de Dieu.
\VS{10}Et c'est par là que nous connaissons les enfants de Dieu et les enfants du diable. Quiconque ne fait pas ce qui est juste et qui n'aime pas son frère n'est point de Dieu.
\VS{11}Car ce qui vous a été annoncé et ce que vous avez entendu dès le commencement c’est que nous nous aimions les uns les autres.
\VS{12}Et que nous ne soyons pas comme Caïn\FTNT{La doctrine de la semence du serpent est présentée par certains comme une explication du sens caché de la chute de l’homme dans le jardin en Eden et du péché originel. Selon cette doctrine, l’acte sexuel serait le fruit de l’arbre de la connaissance du bien et du mal. Cependant, cette doctrine n’est pas biblique, Eve n’a jamais eu de relations sexuelles avec le serpent. Dans Jean 8:44, lorsque Jésus dit aux pharisiens «~vous avez pour père le diable…~» suppose-t-il que le diable engendre des enfants physiquement ? Bien sûr que non !}, qui était de l'esprit malin et qui tua son frère. Et pourquoi le tua-t-il ? C’est parce que ses œuvres étaient mauvaises, et que celles de son frère étaient justes.
\VS{13}Mes frères, ne vous étonnez point si le monde vous hait.
\VS{14}Nous savons que nous sommes passés de la mort à la vie parce que nous aimons nos frères. Celui qui n'aime pas son frère demeure dans la mort.
\VS{15}Quiconque hait son frère est un meurtrier, et vous savez qu'aucun meurtrier ne possède la vie éternelle.
\VS{16}Nous avons connu la charité en ce qu'il a donné sa vie pour nous ; nous aussi, nous devons donner nos vies pour nos frères\FTNT{Jn. 15:13.}.
\VS{17}Si quelqu’un possède les biens du monde, et que voyant son frère dans la nécessité, il lui ferme ses entrailles, comment la charité de Dieu demeure-t-elle en lui ?
\VS{18}Mes petits-enfants, n'aimons pas en paroles et avec la langue, mais par des œuvres et en vérité.
\VS{19}Car c'est par là que nous connaissons que nous sommes de la vérité ; et nous rassurerons ainsi nos cœurs devant lui.
\VS{20}Si notre cœur nous condamne, certes Dieu est plus grand que notre cœur, et il connaît toutes choses.
\VS{21}Mes bien-aimés, si notre cœur ne nous condamne point, nous avons de l’assurance devant Dieu.
\VS{22}Et quoi que nous demandions, nous le recevons de lui, parce que nous gardons ses commandements, et que nous faisons les choses qui lui sont agréables.
\VS{23}Et c'est ici son commandement, que nous croyions au Nom de son Fils Jésus-Christ, et que nous nous aimions les uns les autres, selon le commandement qu’il nous a donné.
\VS{24}Celui qui garde ses commandements demeure en Jésus-Christ, et Jésus-Christ demeure en lui ; et par là nous connaissons qu'il demeure en nous, par l'Esprit qu'il nous a donné.
\Chap{4}
\TextTitle{Il faut éprouver les esprits}
\VerseOne{}Mes bien-aimés, ne croyez pas à tout esprit, mais éprouvez les esprits pour savoir s'ils sont de Dieu, car plusieurs faux prophètes sont venus dans le monde.
\TextTitle{[Caractéristiques des faux prophètes
\\a. leur confession sur Jésus-Christ]}
\VS{2}Reconnaissez à cette marque l'Esprit de Dieu : Tout esprit qui confesse que Jésus-Christ est venu en chair est de Dieu.
\VS{3}Et tout esprit qui ne confesse point que Jésus-Christ est venu en chair n'est point de Dieu ; c’est l'esprit de l'Antéchrist, dont vous avez appris la venue, et qui maintenant est déjà dans le monde.
\VS{4}Mes petits-enfants, vous êtes de Dieu, et vous les avez vaincus, parce que celui qui est en vous est plus grand que celui qui est dans le monde.
\TextTitle{b. leur appartenance au monde}
\VS{5}Eux, ils sont du monde, c'est pourquoi ils parlent comme étant du monde, et le monde les écoute.
\VS{6}Nous sommes de Dieu ; celui qui connaît Dieu nous écoute ; mais celui qui n'est pas de Dieu ne nous écoute point ; c’est par là que nous connaissons l'esprit de vérité et l'esprit de l’erreur.
\TextTitle{La charité de Dieu}
\VS{7}Mes bien-aimés, aimons-nous les uns les autres, car la charité est de Dieu ; et quiconque aime son prochain est né de Dieu et connaît Dieu.
\VS{8}Celui qui n'aime point son prochain n'a pas connu Dieu, car Dieu est Charité\FTNT{« Agapé » en grec.}.
\VS{9}La charité de Dieu a été manifestée envers nous en ce que Dieu a envoyé son Fils unique dans le monde, afin que nous vivions par lui.
\VS{10}Et cette charité consiste, non point en ce que nous avons aimé Dieu, mais en ce qu'il nous a aimés, et qu'il a envoyé son Fils pour être la propitiation\FTNT{Du grec «~hilasmos~» qui signifie «~apaisement~». Les écritures nous parlent aussi du «~propitiatoire~», c'est-à-dire « le siège de la misericorde » ou « Lieu de l'expiation ». Le propitiatoire était une plaque en or du sommet de l'Arche de l'Alliance. Le souverain sacrificateur l'aspergeait sept fois, le jour de l'expiation afin de reconcilier symboliquement Yahweh et son peuple. Voir Ex. 25:17-22} pour nos péchés.
\VS{11}Mes bien-aimés, si Dieu nous a ainsi aimés, nous devons aussi nous aimer les uns les autres.
\VS{12}Personne n'a jamais vu Dieu ; si nous nous aimons les uns les autres, Dieu demeure en nous et sa charité est parfaite en nous.
\VS{13}A ceci nous connaissons que nous demeurons en lui, et lui en nous, c'est qu'il nous a donné de son Esprit.
\VS{14}Et nous l'avons vu, et nous témoignons que le Père a envoyé le Fils pour être le sauveur du monde.
\VS{15}Quiconque confessera que Jésus est le Fils de Dieu, Dieu demeure en lui, et lui en Dieu.
\VS{16}Et nous, nous avons connu et cru en la charité que Dieu a pour nous. Dieu est charité ; et celui qui demeure dans la charité, demeure en Dieu, et Dieu en lui.
\VS{17}Tel il est, tels aussi nous sommes dans ce monde : C’est en cela que la charité est parfaite en nous, afin que nous ayons de l’assurance au jour du jugement.
\VS{18}Il n'y a point de crainte dans la charité, mais la parfaite charité bannit la crainte, car la crainte suppose un châtiment ; or celui qui craint n'est pas accompli dans la charité.
\VS{19}Nous l'aimons, parce qu'il nous a aimés le premier.
\VS{20}Si quelqu'un dit : J'aime Dieu, et qu’il haïsse son frère, c’est un menteur ; car comment celui qui n'aime point son frère, qu'il voit, peut-il aimer Dieu, qu’il ne voit pas ?
\VS{21}Et nous avons ce commandement de sa part, que celui qui aime Dieu, aime aussi son frère.
\Chap{5}
\TextTitle{La foi, principe qui triomphe des conflits avec le monde}
\VerseOne{}Quiconque croit que Jésus est le Christ, est né de Dieu, et quiconque aime celui qui l'a engendré, aime aussi celui qui est né de lui.
\VS{2}Nous connaissons à ceci que nous aimons les enfants de Dieu, lorsque nous aimons Dieu et que nous gardons ses commandements.
\VS{3}Car c'est en ceci que consiste notre amour pour Dieu : Que nous gardions ses commandements. Et ses commandements ne sont point pénibles.
\VS{4}Parce que tout ce qui est né de Dieu est victorieux du monde ; et ce qui nous fait remporter la victoire sur le monde, c'est notre foi.
\VS{5}Qui est celui qui a remporté la victoire sur le monde, sinon celui qui croit que Jésus est le Fils de Dieu ?
\VS{6}C'est ce Jésus, le Christ, qui est venu avec l’eau et le sang, et pas seulement avec l'eau, mais avec l'eau et le sang ; et c'est l'Esprit qui rend témoignage, or l'Esprit est la vérité.
\VS{7}Car il y en a trois dans le ciel qui rendent témoignage, le Père, la Parole, et le Saint-Esprit ; et ces trois-là ne sont qu'un\FTNT{Dieu est UN. Voir De. 6:4.}.
\VS{8}Il y en a aussi trois qui rendent témoignage sur la terre, à savoir l'Esprit, l'eau, et le sang, et ces trois-là se rapportent à un.
\TextTitle{Une assurance bénie}
\VS{9}Si nous recevons le témoignage des hommes, le témoignage de Dieu est plus grand, car le témoignage de Dieu consiste en ce qu’il a rendu témoignage à son Fils.
\VS{10}Celui qui croit au Fils de Dieu a le témoignage de Dieu en lui-même ; mais celui qui ne croit pas Dieu, le fait menteur, car il ne croit pas au témoignage que Dieu a rendu de son Fils.
\VS{11}Et c'est ici le témoignage, à savoir que Dieu nous a donné la vie éternelle, et cette vie est dans son Fils.
\VS{12}Celui qui a le Fils a la vie, celui qui n'a pas le Fils de Dieu n'a pas la vie.
\VS{13}Je vous ai écrit ces choses, à vous qui croyez au Nom du Fils de Dieu, afin que vous sachiez que vous avez la vie éternelle, et afin que vous croyiez au Nom du Fils de Dieu.
\VS{14}Et c'est ici l’assurance que nous avons en Dieu, que si nous demandons quelque chose selon sa volonté, il nous exauce.
\VS{15}Et si nous savons qu'il nous exauce, quelque chose que nous demandions, nous savons que nous possédons la chose que nous lui avons demandée.
\VS{16}Si quelqu'un voit son frère commettre un péché qui ne mène point à la mort\FTNT{Le péché qui mène à la mort c’est le blasphème contre le Saint-Esprit. Voir commentaire en Mt. 12:32.}, qu’il prie pour lui, et Dieu donnera la vie à ce frère. Il la donnera à ceux qui commettent un péché qui ne mène point à la mort. Il y a un péché qui mène à la mort ; je ne te dis point de prier pour ce péché-là.
\VS{17}Toute iniquité est un péché, mais il y a quelque péché qui ne mène pas à la mort.
\VS{18}Nous savons que quiconque est né de Dieu ne pèche point ; mais celui qui est engendré de Dieu se garde lui-même, et le malin ne le touche point.
\VS{19}Nous savons que nous sommes nés de Dieu, mais le monde entier est plongé dans le mal.
\TextTitle{Conclusion}
\VS{20}Or nous savons que le Fils de Dieu est venu, et il nous a donné l'intelligence pour connaître le Véritable ; et nous sommes dans le Véritable, en son Fils Jésus-Christ. Il est le vrai\FTNT{Dans Jean 17:3, le Père est présenté comme le Vrai Dieu ; le terme grec traduit par «~vrai~» dans Jean est aussi appliqué à Jésus dans ce passage. Jésus est donc le Vrai Dieu.} Dieu, et la vie éternelle.
\VS{21}Mes petits enfants, gardez-vous des idoles. Amen.
\PPE{}
\end{multicols}

%\clearpage\ShortTitle{2 Jean}\BookTitle{2 Jean}\BFont
\noindent\hrulefill
{\footnotesize
\textit{
\bigskip
{\centering{}
\\Auteur : Jean
\\(Gr. : Ioannes)
\\Signification : Yahweh a fait grâce
\\Thème : Amour et vérité
\\Date de rédaction : Env. 85 ap. J.-C.\\}
}
%\bigskip
\textit{
\\Il semblerait que cette épître était adressée à une église se réunissant chez une personne du nom de Kyria. Jean les invite à demeurer dans la communion avec Dieu et les met en garde contre les hérésies et la fréquentation des faux docteurs.\bigskip
}
}
\par\nobreak\noindent\hrulefill
\begin{multicols}{2}
\Chap{1}
\TextTitle{Introduction}
\VerseOne{}L'ancien à Kyria l’élue, et à ses enfants, que j'aime dans la vérité, et ce n’est pas moi seul qui les aime, mais aussi tous ceux qui ont connu la vérité.
\VS{2}A cause de la vérité qui demeure en nous, et qui sera avec nous éternellement,
\VS{3}que la grâce, la miséricorde, et la paix de la part de Dieu le Père, et de la part du Seigneur Jésus-Christ, le Fils du Père, soient avec vous dans la vérité et dans la charité.
\TextTitle{La marche dans la vérité et dans la charité}
\VS{4}Je me suis fort réjoui d'avoir trouvé quelques-uns de tes enfants qui marchent dans la vérité selon le commandement que nous avons reçu du Père.
\VS{5}Et maintenant, ô Kyria ! Je te prie, non comme t'écrivant un nouveau commandement, mais celui que nous avons eu dès le commencement, que nous ayons de la charité les uns pour les autres.
\VS{6}Et c'est ici la charité, que nous marchions selon ses commandements. Et c'est là son commandement, comme vous l'avez entendu dès le commencement, afin que vous l'observiez.
\TextTitle{Le signe du séducteur et de l'antéchrist}
\VS{7}Car plusieurs séducteurs sont venus dans le monde, qui ne confessent point que Jésus-Christ est venu en chair ; un tel homme est un séducteur et un Antéchrist.
\VS{8}Prenez garde à vous-mêmes, afin que vous ne perdiez point le fruit du travail que vous avez fait, mais que vous en receviez une pleine récompense.
\VS{9}Quiconque transgresse la doctrine de Jésus-Christ et ne lui demeure point fidèle n'a point Dieu ; celui qui demeure dans la doctrine de Christ a le Père et le Fils.
\VS{10}Si quelqu'un vient à vous, et qu'il n'apporte point cette doctrine, ne le recevez point dans votre maison, et ne le saluez pas ;
\VS{11}car celui qui le salue participe à ses mauvaises œuvres.
\TextTitle{Conclusion}
\VS{12}Quoique j’aie plusieurs choses à vous écrire, je n’ai pas voulu les écrire avec du papier et de l'encre, mais j'espère aller vers vous, et vous parler bouche à bouche, afin que notre joie soit parfaite.
\VS{13}Les enfants de ta sœur élue te saluent. Amen !
\PPE{}
\end{multicols}

%\clearpage\ShortTitle{3 Jean}\BookTitle{3 Jean}\BFont
\noindent\hrulefill
{\footnotesize
\textit{
\bigskip
{\centering{}
\\Signifie : Yahweh a fait grâce
\\Thème : Sincérité, hospitalité et caractère chrétien
\\Auteur : Jean
\\Date de rédaction : Env. 85\\}
}
%\bigskip
\textit{
\\Cette épître fut destinée à Gaïus, l’un des responsables d’une église d’Asie Mineure dont Jean loua la piété et la générosité. Il l’avertit de l’orgueil et des agissements de Diotrèphe qui étaient contraires à la Parole mais souligna le bon témoignage de Démétrius.\bigskip
}
}
\par\nobreak\noindent\hrulefill
\begin{multicols}{2}
\TextTitle{[Introduction]}
\Chap{1}
\VerseOne{}L'ancien à Gaïus, le bien-aimé, que j'aime dans la vérité.
\VS{2}Bien-aimé, je souhaite que tu prospères{\FTNT{La prospérité dont il est question dans ce passage n’a rien à voir avec l’Evangile de prospérité qui met l’accent sur la richesse matérielle. Le mot grec «~euodoo~» signifie «~concevoir~» un voyage prospère et diligent~», «~mener par une voie directe et facile~», «~prospérer~», «~être heureux~».}} en toutes choses, et que tu sois en bonne santé, comme ton âme est en prospérité.
\VS{3}Car j’ai été fort réjoui quand les frères sont venus et ont rendu témoignage de ta sincérité, et comment tu marches dans la vérité.
\VS{4}Je n'ai pas de plus grande joie que d’apprendre que mes enfants marchent dans la vérité.
\TextTitle{[L'hospitalité]}
\VS{5}Bien-aimé, tu agis fidèlement dans tout ce que tu fais envers les frères et envers les étrangers,
\VS{6}qui en présence de l'église ont rendu témoignage de ta charité. Et tu feras bien de les accompagner dignement, comme il est séant selon Dieu.
\VS{7}Car ils sont partis pour son Nom, ne prenant rien des gentils.
\VS{8}Nous devons donc recevoir de tels hommes, afin d’être ouvriers avec eux pour la vérité.
\TextTitle{[Les mauvais actes de Diotrèphe et son caractère dominateur]}
\VS{9}J'ai écrit à l'Eglise, mais Diotrèphe, qui aime être le premier parmi eux, ne nous reçoit point.
\VS{10}C'est pourquoi, si je viens, je rappellerai les actions qu'il commet, en tenant contre nous de mauvais discours ; et n'étant pas content de cela, non seulement il ne reçoit pas les frères, mais il empêche même ceux qui veulent les recevoir et les chasse de l'église.
\VS{11}Bien-aimé, n'imite point le mal, mais le bien. Celui qui fait le bien est de Dieu ; mais celui qui fait le mal n'a point vu Dieu.
\TextTitle{[Témoignage de Démétrius]}
\VS{12}Tous rendent témoignage à Démétrius, et la vérité même le lui rend, et nous aussi nous lui rendons témoignage, et vous savez que notre témoignage est véritable.
\TextTitle{[Conclusion]}
\VS{13}J'avais plusieurs choses à écrire, mais je ne veux pas t'écrire avec l'encre et avec la plume.
\VS{14}Mais j'espère te voir bientôt, et nous parlerons de bouche à bouche.
\VS{15}Que la paix soit avec toi ! Les amis te saluent. Salue les amis, chacun par son nom.
\PPE{}
\end{multicols}

%\clearpage\ShortTitle{Ap.}\BookTitle{Apocalypse}\BFont
\noindent\hrulefill
{\footnotesize
\textit{
\bigskip
{\centering{}
\\Auteur~: Jean
\\Thème~: L'aboutissement de toutes choses
\\(Gr.~: Apokalupsis)
\\Signification~: Mettre à nu, révélation d'une vérité, action de révéler
\\Date de rédaction~: Env. 95 ap. J.-C.\\}
}
\textit{
\\Le terme apocalypse, du grec «~apokalupsis~», évoque «~l'action de révéler ce qui était caché ou inconnu~». Ce mot a pour racine «~apokalupto~» qui signifie aussi «~découvrir, dévoiler ce qui est voilé ou recouvert~».
\\C'est à Patmos, île grecque de la mer Egée - où il s'exila en raison de la persécution de l'empereur Domitien (51 - 96 ap. J.C.) - que Jean reçut une révélation de Jésus-Christ ainsi qu'un message s'adressant aux «~sept églises~» qui constituaient certainement les villes de l'Asie Mineure où se trouvaient les principales concentrations de chrétiens. Si Ephèse figure dans les écrits de la Nouvelle Alliance et que Thyatire et Laodicée y sont brièvement mentionnées, les quatre autres églises - qu'on ne retrouve nulle part ailleurs dans les Ecritures - étaient sans doute le fruit du travail missionnaire de Paul. Les sept lettres s'adressent à l'ange de chacune de ces assemblées locales, autrement dit aux messagers de celles-ci (probablement un ancien ou un responsable).
\\Ce livre, qui arrive en conclusion des Ecritures, annonce les événements qui doivent précéder la fin de l'histoire de l'humanité.\bigskip
}
} 
\par\nobreak\noindent\hrulefill
\begin{multicols}{2}
\Chap{1}
\TextTitle{Introduction}
\VerseOne{}La révélation\FTNT{«~Apokalupsis~» en grec. Voir l'introduction du livre.} de Jésus-Christ, que Dieu lui a donnée pour montrer à ses serviteurs les choses qui doivent arriver bientôt, et qui les a fait connaître en les envoyant par son ange à Jean, son serviteur,
\VS{2}qui a annoncé la parole de Dieu, et le témoignage de Jésus-Christ, et toutes les choses qu'il a vues.
\VS{3}Béni est celui qui lit et ceux qui écoutent les paroles de cette prophétie, et qui gardent les choses qui y sont écrites~! Car le temps est proche.
\TextTitle{Jésus-Christ}
\VS{4}Jean aux sept églises qui sont en Asie~: Que la grâce et la paix vous soient données de la part de celui QUI EST, QUI ETAIT, et QUI VIENT\FTNT{Les prophètes ont prophétisé la venue de Yahweh en personne~: Es. 35:4~; Es. 40:10-11~; Es. 60:1-2~; Za. 14:1-21~; Jn. 14:1-3. Jésus-Christ est bien Yahweh qui vient.}, et de la part des sept Esprits qui sont devant son trône,
\VS{5}et de la part de Jésus-Christ, qui est le témoin fidèle, le premier-né d'entre les morts\FTNT{Voir commentaire Col. 1:15.}, et le Prince des rois de la terre.
\VS{6}A lui, dis-je, qui nous a aimés, et qui nous a lavés de nos péchés dans son sang, et qui a fait de nous des rois et des prêtres pour Dieu, son Père, à lui soient la gloire et la force aux siècles des siècles. Amen~!
\TextTitle{La venue du Christ}
\VS{7}Voici, il vient avec les nuées, et tout œil le verra, et même ceux qui l'ont percé~; et toutes les tribus de la terre se lamenteront devant lui. Oui, amen~!
\VS{8}Je suis l'Alpha et l'Oméga, le commencement et la fin, dit le Seigneur, QUI EST, QUI ETAIT, et QUI VIENT, le Tout-Puissant.
\TextTitle{La vision doit être écrite}
\VS{9}Moi Jean, qui suis aussi votre frère et qui participe à la tribulation, au règne, et à la patience de Jésus-Christ, j'étais sur l'île appelée Patmos à cause de la parole de Dieu, et du témoignage de Jésus-Christ.
\VS{10}Je fus ravi en esprit au jour du Seigneur, et j'entendis derrière moi une voix forte, comme le son d'une trompette,
\VS{11}qui disait~: Je suis l'Alpha et l'Oméga, le premier et le dernier. Ecris dans un livre ce que tu vois, et envoie-le aux sept églises qui sont en Asie, à savoir à Ephèse, à Smyrne, à Pergame, à Thyatire, à Sardes, à Philadelphie, et à Laodicée.
\VS{12}Alors je me retournai pour voir celui dont la voix m'avait parlé, et après m'être retourné, je vis sept chandeliers d'or,
\VS{13}et au milieu des sept chandeliers d'or, quelqu'un qui ressemblait à un fils d'homme, vêtu d'une longue robe, et ayant une ceinture d'or sur la poitrine.
\VS{14}Sa tête et ses cheveux étaient blancs comme de la laine blanche, et comme de la neige, et ses yeux étaient comme une flamme de feu.
\VS{15}Ses pieds étaient semblables à de l'airain ardent, comme s'ils avaient été embrasés dans une fournaise~; et sa voix était comme le bruit des grandes eaux.
\VS{16}Et il avait dans sa main droite sept étoiles, et de sa bouche sortait une épée aiguë à deux tranchants, et son visage était semblable au soleil lorsqu'il brille dans sa force.
\VS{17}Quand je le vis, je tombai à ses pieds comme mort, et il mit sa main droite sur moi, en me disant~: Ne crains pas~!
\VS{18}Je suis le premier et le dernier, et je vis~; j'étais mort, et voici, je suis vivant aux siècles des siècles. Amen~! Et je tiens les clefs de Hadès\FTNT{Voir commentaire Mt. 16:18.} et de la mort.
\VS{19}Ecris les choses que tu as vues, celles qui sont présentement, et celles qui doivent arriver ensuite.
\VS{20}Le mystère des sept étoiles que tu as vues dans ma main droite, et les sept chandeliers d'or. Les sept étoiles sont les anges des sept églises~; et les sept chandeliers que tu as vus sont les sept églises.
\Chap{2}
\TextTitle{Ephèse~: L'église qui a perdu le premier amour}
\VerseOne{}Ecris à l'ange\FTNT{Ange, du grec «~aggelos~»~: Envoyé, messager, un ange. Un messager de Dieu. Ce terme sert à désigner aussi bien les créatures spirituelles que les êtres humains.} de l'église d'Ephèse~: Voici ce que dit celui qui tient les sept étoiles dans sa main droite, et qui marche au milieu des sept chandeliers d'or~:
\VS{2}Je connais tes œuvres, et ton travail, et ta patience, et je sais que tu ne peux pas supporter les méchants, et que tu as éprouvé ceux qui se disent être apôtres et qui ne le sont pas, et que tu les as trouvés menteurs~;
\VS{3}et que tu as souffert, et que tu as eu de la patience, et que tu as travaillé pour mon Nom, et que tu ne t'es pas lassé.
\VS{4}Mais j'ai quelque chose contre toi, c'est que tu as abandonné ta première charité\FTNT{Il est question ici de l'amour «~agape~»~: L'amour fraternel, l'amour divin.}.
\VS{5}C'est pourquoi souviens-toi donc d'où tu es tombé, repens-toi, et fais tes premières œuvres. Autrement, je viendrai à toi à toute vitesse, et j'ôterai ton chandelier de sa place si tu ne te repens pas.
\VS{6}Mais pourtant tu as ceci de bon, c'est que tu hais les œuvres des Nicolaïtes\FTNT{Nicolaïtes~: Tiré du nom Nicolas, qui signifie littéralement «~victorieux du peuple~». Il s'agit d'une secte dont les membres furent peut-être des disciples d'un certain Nicolas, l'un des diacres de l'église d'Antioche qui aurait dévié (Ac. 6:5). Ces derniers suivaient la doctrine de Balaam, enseignant aux chrétiens qu'à cause du principe de liberté, ils pouvaient manger des viandes sacrifiées aux idoles et commettre des actes immoraux comme les Gentils.}, œuvres que je hais moi aussi.
\VS{7}Que celui qui a des oreilles entende ce que l'Esprit dit aux églises~! A celui qui vaincra, je lui donnerai à manger de l'arbre de vie, qui est au milieu du paradis de Dieu.
\TextTitle{Smyrne~: L'église sous la persécution}
\VS{8}Ecris aussi à l'ange de l'église de Smyrne~: Voici ce que dit celui qui est le premier et le dernier, qui a été mort, et qui est revenu à la vie~:
\VS{9}Je connais tes œuvres, ton affliction et ta pauvreté, quoique tu sois riche, et le blasphème de ceux qui se disent être Juifs et qui ne le sont pas, mais qui sont la synagogue de Satan.
\VS{10}Ne crains rien des choses que tu as à souffrir. Voici, il arrivera que le diable mettra quelques-uns d'entre vous en prison, afin que vous soyez éprouvés~; et vous aurez une affliction de dix jours. Sois fidèle jusqu'à la mort, et je te donnerai la couronne de vie.
\VS{11}Que celui qui a des oreilles, entende ce que l'Esprit dit aux églises~! Celui qui vaincra n'aura pas à souffrir la seconde mort.
\TextTitle{Pergame~: L'église établie dans le monde}
\VS{12}Ecris aussi à l'ange de l'église de Pergame~: Voici ce que dit celui qui a l'épée aiguë à deux tranchants\FTNT{Hé. 4:12.}~: 
\VS{13}Je connais tes œuvres, et le lieu où tu habites, à savoir là où est le trône de Satan. Et que cependant tu retiens mon Nom, et tu n'as pas renié ma foi, même aux jours d'Antipas, mon fidèle martyr\FTNT{Du grec «~martus~» qui signifie «~témoin~».}, qui a été mis à mort chez vous, là où Satan habite.
\VS{14}Mais j'ai quelque chose contre toi, c'est que tu as là des gens attachés à la doctrine de Balaam, qui enseignait à Balak à mettre un scandale devant les enfants d'Israël, afin qu'ils mangent des viandes sacrifiées aux idoles, et qu'ils se livrent à la fornication\FTNT{No. 25:1-2~; No. 31:16.}.
\VS{15}De même, toi aussi tu as des gens attachés à la doctrine des Nicolaïtes~; ce que je hais~!
\VS{16}Repens-toi donc, autrement je viendrai à toi à toute vitesse, et je les combattrai avec l'épée de ma bouche.
\VS{17}Que celui qui a des oreilles, entende ce que l'Esprit dit aux églises~! A celui qui vaincra, je lui donnerai à manger de la manne qui est cachée, et je lui donnerai un caillou blanc, et sur ce caillou sera écrit un nouveau nom, que nul ne connaît, sinon celui qui le reçoit.
\TextTitle{Thyatire~: L'église en temps d'idolâtrie}
\VS{18}Ecris aussi à l'ange de l'église de Thyatire~: Voici ce que dit le Fils de Dieu, qui a ses yeux comme une flamme de feu, et dont les pieds sont semblables à de l'airain ardent.
\VS{19}Je connais tes œuvres, ta charité, ton service, ta foi, ta patience, et que tes dernières œuvres surpassent les premières.
\VS{20}Mais j'ai quelque peu de chose contre toi, c'est que tu laisses cette femme Jézabel\FTNT{1 R. 16:31~; 1 R. 21:25~; 2 R. 9:7~; 2 R. 9:22.}, qui se dit prophétesse, enseigner et séduire mes serviteurs pour les porter à la fornication, et leur faire manger des choses sacrifiées aux idoles.
\VS{21}Et je lui ai donné du temps, afin qu'elle se repente de sa prostitution, mais elle ne s'est pas repentie.
\VS{22}Voici, je vais la jeter sur un lit, et mettre dans une grande affliction ceux qui commettent l'adultère avec elle, s'ils ne se repentent pas de leurs œuvres.
\VS{23}Et je ferai mourir de mort ses enfants~; et toutes les églises connaîtront que je suis celui qui sonde les reins et les cœurs, et je rendrai à chacun de vous selon ses œuvres.
\VS{24}Mais je vous dis à vous et aux autres qui sont à Thyatire, à tous ceux qui n'ont pas cette doctrine, et qui n'ont pas connu les profondeurs de Satan, comme ils disent, je vous dis~: Je ne mettrai pas sur vous d'autre charge.
\VS{25}Mais retenez ce que vous avez, jusqu'à ce que je vienne.
\VS{26}Car à celui qui aura vaincu, et qui aura gardé mes œuvres jusqu'à la fin, je lui donnerai autorité sur les nations.
\VS{27}Et il les gouvernera avec un sceptre de fer, et elles seront brisées comme les vases d'un potier, ainsi que j'en ai moi-même reçu le pouvoir de mon Père.
\VS{28}Et je lui donnerai l'étoile du matin.
\VS{29}Que celui qui a des oreilles entende ce que l'Esprit dit aux églises~!
\Chap{3}
\TextTitle{Sardes~: L'église morte}
\VerseOne{}Ecris aussi à l'ange de l'église de Sardes~: Voici ce que dit celui qui a les sept Esprits de Dieu, et les sept étoiles~: Je connais tes œuvres. Tu as la réputation d'être vivant, mais tu es mort.
\VS{2}Sois vigilant, et affermis le reste qui va mourir~; car je n'ai pas trouvé tes œuvres parfaites devant Dieu.
\VS{3}Souviens-toi donc des choses que tu as reçues et entendues, garde-les, et repens-toi. Si tu ne veilles pas, je viendrai contre toi comme un voleur, et tu ne sauras pas à quelle heure je viendrai contre toi\FTNT{Mt. 24:43~; Lu. 12:39~; 1 Th. 5:2~; 2 Pi. 3:10.}.
\VS{4}Toutefois, tu as quelque peu de personnes à Sardes qui n'ont pas souillé leurs vêtements, et qui marcheront avec moi en vêtements blancs, car ils en sont dignes.
\VS{5}Celui qui vaincra sera vêtu de vêtements blancs, et je n'effacerai pas son nom du Livre de vie, mais je confesserai son nom devant mon Père, et devant ses anges.
\VS{6}Que celui qui a des oreilles, entende ce que l'Esprit dit aux églises~!
\TextTitle{Philadelphie~: L'église réveillée et fidèle}
\VS{7}Ecris aussi à l'ange de l'église de Philadelphie~: Voici ce que dit le Saint et le Véritable, qui a la clef de David, qui ouvre et nul ne ferme, qui ferme et nul n'ouvre.
\VS{8}Je connais tes œuvres. Voici, j'ai ouvert une porte devant toi, et personne ne peut la fermer~; parce que tu as peu de puissance, que tu as gardé ma parole, et que tu n'as pas renié mon Nom.
\VS{9}Voici, je ferai venir ceux de la synagogue de Satan qui se disent Juifs, et ne le sont pas, mais qui mentent~; voici, dis-je, je les ferai venir et se prosterner à tes pieds, et ils connaîtront que je t'aime.
\VS{10}Parce que tu as gardé la parole de ma persévérance, je te garderai aussi de l'heure de la tentation qui doit arriver dans le monde entier, pour éprouver les habitants de la terre.
\VS{11}Voici, je viens à toute vitesse. Tiens ferme ce que tu as, afin que personne ne t'enlève ta couronne.
\VS{12}Celui qui vaincra, je ferai de lui une colonne dans le temple de mon Dieu, et il n'en sortira plus~; et j'écrirai sur lui le Nom de mon Dieu, et le nom de la cité de mon Dieu, qui est la nouvelle Jérusalem qui descend du ciel d'auprès de mon Dieu, et mon nouveau Nom.
\VS{13}Que celui qui a des oreilles entende ce que l'Esprit dit aux églises~!
\TextTitle{Laodicée~: L'église apostate}
\VS{14}Ecris aussi à l'ange de l'église de Laodicée~: Voici ce que dit l'Amen, le témoin fidèle et véritable, le commencement de la création de Dieu~:
\VS{15}Je connais tes œuvres. Je sais que tu n'es ni froid ni bouillant~; puisses-tu être ou froid ou bouillant~!
\VS{16}Parce que tu es tiède, et que tu n'es ni froid ni bouillant, je te vomirai de ma bouche.
\VS{17}Car tu dis~: Je suis riche, je suis dans l'abondance, et je n'ai besoin de rien~; mais tu ne sais pas que tu es malheureux, misérable, pauvre, aveugle et nu.
\VS{18}Je te conseille d'acheter de moi de l'or éprouvé par le feu, afin que tu deviennes riche; et des vêtements blancs, afin que tu sois vêtu et que la honte de ta nudité ne paraisse pas~; et d'oindre tes yeux de collyre, afin que tu voies.
\VS{19}Moi, je reprends et je châtie\FTNT{De. 8:5~; 2 S. 7:14~; Pr. 13:24~; Hé. 12:7.} tous ceux que j'aime. Aie donc du zèle et repens-toi.
\TextTitle{Le Messie se retrouve hors des églises apostates}
\VS{20}Voici, je me tiens à la porte, et je frappe. Si quelqu'un entend ma voix et m'ouvre la porte, j'entrerai chez lui, et je souperai avec lui, et lui avec moi.
\VS{21}Celui qui vaincra, je le ferai asseoir avec moi sur mon trône, ainsi que j'ai vaincu et me suis assis avec mon Père sur son trône.
\VS{22}Que celui qui a des oreilles entende ce que l'Esprit dit aux églises~!
\Chap{4}
\TextTitle{Vision avant l'ouverture des sceaux}
\VerseOne{}Après ces choses, je regardai, et voici une porte était ouverte dans le ciel. Et la première voix que j'avais entendue, comme le son d'une trompette, et qui parlait avec moi, me dit~: Monte ici, et je te montrerai les choses qui doivent arriver à l'avenir.
\VS{2}Aussitôt, je fus ravi en esprit. Et voici, un trône était dressé dans le ciel, et sur ce trône, quelqu'un était assis.
\VS{3}Et celui qui y était assis était semblable à une pierre de jaspe et de sardoine~; et le trône était environné d'un arc-en-ciel semblable à de l'émeraude.
\TextTitle{Les trônes des vingt-quatre anciens}
\VS{4}Et il y avait autour du trône vingt-quatre trônes et je vis sur ces trônes vingt-quatre anciens assis, vêtus de vêtements blancs, et ayant sur leurs têtes des couronnes d'or.
\VS{5}Et du trône sortaient des éclairs, des tonnerres, et des voix~; et il y avait devant le trône sept lampes de feu ardentes, qui sont les sept Esprits de Dieu.
\TextTitle{Le Messie est digne de recevoir la louange et la gloire}
\VS{6}Et devant le trône, il y avait une mer de verre semblable à du cristal~; et au milieu du trône et autour du trône quatre animaux, pleins d'yeux devant et derrière.
\VS{7}Et le premier animal était semblable à un lion~; le second animal était semblable à un veau~; le troisième animal avait la face comme un homme~; et le quatrième animal était semblable à un aigle qui vole.
\VS{8}Et les quatre animaux avaient chacun six ailes, et tout autour et au-dedans ils étaient pleins d'yeux~; et ils ne cessent pas de dire jour et nuit~: Saint~! Saint~! Saint est le Seigneur Dieu Tout-puissant, QUI ETAIT, QUI EST, et QUI VIENT.
\VS{9}Et quand ces animaux rendaient gloire et honneur et des actions de grâces à celui qui était assis sur le trône, à celui qui est vivant aux siècles des siècles,
\VS{10}les vingt-quatre anciens se prosternaient devant celui qui était assis sur le trône, et adoraient celui qui est vivant aux siècles des siècles, et ils jetaient leurs couronnes devant le trône, en disant~:
\VS{11}Seigneur, tu es digne de recevoir gloire, honneur et puissance~; car tu as créé toutes choses, et c'est par ta volonté qu'elles existent et qu'elles ont été créées.
\Chap{5}
\TextTitle{Le Messie est le seul digne d'ouvrir le livre}
\VerseOne{}Puis je vis dans la main droite de celui qui était assis sur le trône, un livre écrit en dedans et en dehors, scellé de sept sceaux.
\VS{2}Et je vis aussi un ange remarquable par sa force, qui proclamait d'une voix forte~: Qui est digne d'ouvrir le livre, et d'en rompre les sceaux~?
\VS{3}Et il n'y avait personne, ni dans le ciel, ni sur la terre, ni sous la terre qui pouvait ouvrir le livre, ni le regarder.
\VS{4}Et je pleurais beaucoup parce que personne n'était trouvé digne d'ouvrir le livre, ni de le lire, ni de le regarder.
\VS{5}Et l'un des anciens me dit~: Ne pleure pas, voici le Lion qui vient de la tribu de Juda, de la racine de David, a vaincu pour ouvrir le livre et pour en rompre les sept sceaux.
\VS{6}Et je regardai, et voici il y avait au milieu du trône et des quatre animaux, et au milieu des anciens, un Agneau qui se tenait là comme immolé, ayant sept cornes, et sept yeux, qui sont les sept Esprits de Dieu envoyés par toute la terre.
\VS{7}Et il vint et prit le livre de la main droite de celui qui était assis sur le trône.
\TextTitle{L'Agneau est adoré\FTNTT{Ph. 2:9-11.}}
\VS{8}Et quand il eut pris le livre, les quatre animaux et les vingt-quatre anciens se prosternèrent devant l'Agneau, ayant chacun des harpes et des coupes d'or pleines de parfums, qui sont les prières des saints.
\VS{9}Et ils chantaient un cantique nouveau, en disant~: Tu es digne de prendre le livre, et d'en ouvrir les sceaux~; car tu as été mis à mort, et tu nous as rachetés pour Dieu par ton sang, de toute tribu, de toute langue, de tout peuple, et de toute nation~;
\VS{10}et tu as fait de nous des rois et des prêtres pour notre Dieu~; et nous régnerons sur la terre.
\VS{11}Puis je regardai, et j'entendis la voix de plusieurs anges autour du trône, et des anciens; et leur nombre était de plusieurs millions.
\VS{12}Et ils disaient à haute voix~: L'Agneau qui a été mis à mort est digne de recevoir puissance, richesses, sagesse, force, honneur, gloire et louange.
\VS{13}J'entendis aussi toutes les créatures qui sont dans le ciel, sur la terre, et sous la terre, et dans la mer, et toutes les choses qui y sont, disant~: A celui qui est assis sur le trône et à l'Agneau, soient louange, honneur, gloire, et force, aux siècles des siècles~!
\VS{14}Et les quatre animaux disaient~: Amen~! Et les vingt-quatre anciens se prosternèrent et adorèrent celui qui est vivant aux siècles des siècles.
\Chap{6}
\TextTitle{Premier sceau~: Le cavalier qui part pour vaincre}
\VerseOne{}Et quand l'Agneau eut ouvert l'un des sceaux, je regardai, et j'entendis l'un des quatre animaux qui disait comme avec une voix de tonnerre~: Viens, et vois.
\VS{2}Je regardai, et je vis un cheval blanc~; celui qui était monté dessus avait un arc, et il lui fut donné une couronne~; et il est sortit en vainqueur pour vaincre\FTNT{Contrairement aux apparences, ce cavalier couronné d'un diadème et qui monte un cheval blanc n'est pas Jésus-Christ, mais l'Antichrist qui singe le retour glorieux du Seigneur~: Da. 7:21~; Mt. 24:4-5~; 2 Th. 2:9-12~; Ap. 13:7. Le vrai Christ revenant triomphalement avec son Eglise est décrit en Ap. 19:11-16.}.
\TextTitle{Deuxième sceau~: La guerre}
\VS{3}Et quand il eut ouvert le second sceau, j'entendis le second animal qui disait~: Viens, et vois.
\VS{4}Et il sortit un autre cheval qui était roux~; il fut donné à celui qui était monté dessus de pouvoir ôter la paix de la terre, afin que les hommes se tuent les uns les autres~; et il lui fut donné une grande épée.
\TextTitle{Troisième sceau~: La famine}
\VS{5}Et quand il eut ouvert le troisième sceau, j'entendis le troisième animal qui disait~: Viens, et vois. Je regardai, et je vis un cheval noir, et celui qui était monté dessus avait une balance dans sa main.
\VS{6}Et j'entendis au milieu des quatre animaux une voix qui disait~: Une mesure de blé pour un denier, et les trois mesures d'orge pour un denier~; mais ne fais pas de mal au vin et à l'huile.
\TextTitle{Quatrième sceau~: La mort}
\VS{7}Et quand il eut ouvert le quatrième sceau, j'entendis la voix du quatrième animal qui disait~: Viens, et vois.
\VS{8}Je regardai, et je vis un cheval verdâtre~; et celui qui était monté dessus se nommait la Mort, et le Hadès l'accompagnait. Il leur fut donné le pouvoir sur le quart de la terre pour tuer par l'épée, par la famine, par la mortalité, et par les bêtes sauvages de la terre.
\TextTitle{Cinquième sceau~: Les martyrs}
\VS{9}Et quand il eut ouvert le cinquième sceau, je vis sous l'autel les âmes de ceux qui avaient été tués pour la parole de Dieu, et pour le témoignage qu'ils avaient gardé.
\VS{10}Et elles criaient à haute voix, disant~: Jusqu'à quand, Seigneur qui es saint et véritable, ne jugeras-tu pas et ne vengeras-tu pas notre sang de ceux qui habitent sur la terre~?
\VS{11}Et il leur fut donné à chacun des robes blanches, et il leur fut dit de se tenir en repos encore un peu de temps, jusqu'à ce que le nombre de leurs compagnons de service, et de leurs frères qui doivent être mis à mort comme eux, soit complet.
\TextTitle{Sixième sceau~: L'anarchie}
\VS{12}Et je regardai quand il eut ouvert le sixième sceau, et voici, il se fit un grand tremblement de terre, et le soleil devint noir comme un sac de crin, et la lune entière devint comme du sang.
\VS{13}Et les étoiles du ciel tombèrent sur la terre\FTNT{Mt. 24:29~; Mc. 13:25.}, comme lorsque le figuier est agité par un grand vent et laisse tomber ses figues encore vertes.
\VS{14}Et le ciel se retira comme un livre qu'on roule~; et toutes les montagnes et les îles furent remuées de leurs places.
\VS{15}Et les rois de la terre, et les princes, et les riches, et les capitaines, et les puissants, et tout esclave, et tout homme libre se cachèrent dans les cavernes et entre les rochers des montagnes.
\VS{16}Et ils disaient aux montagnes et aux rochers~: Tombez sur nous\FTNT{Lu. 23:30.}, et cachez-nous devant la face de celui qui est assis sur le trône, et devant la colère de l'Agneau~;
\VS{17}car le grand jour de sa colère est venu, et qui peut subsister~?
\Chap{7}
\TextTitle{Les 144 000 marqués du sceau de Dieu}
\VerseOne{}Après cela, je vis quatre anges qui se tenaient aux quatre coins de la terre, et qui retenaient les quatre vents de la terre, afin qu'ils ne soufflent pas sur la terre, ni sur la mer, ni sur aucun arbre.
\VS{2}Puis je vis un autre ange qui montait du côté de l'orient, tenant le sceau du Dieu vivant, et il cria d'une voix forte aux quatre anges à qui il avait été donné de faire du mal à la terre et à la mer,
\VS{3}et leur dit~: Ne faites pas de mal à la terre, ni à la mer, ni aux arbres, jusqu'à ce que nous ayons marqué du sceau les serviteurs de notre Dieu sur leurs fronts.
\VS{4}Et j'entendis que le nombre de ceux qui avaient été marqués du sceau était de cent quarante-quatre mille, de toutes les tribus des enfants d'Israël.
\VS{5}de la tribu de Juda, douze mille marqués du sceau~; de la tribu de Ruben, douze mille marqués du sceau~; de la tribu de Gad, douze mille marqués du sceau~;
\VS{6}de la tribu d'Aser, douze mille marqués du sceau~; de la tribu de Nephthali, douze mille marqués du sceau~; de la tribu de Manassé, douze mille marqués du sceau~;
\VS{7}de la tribu de Siméon, douze mille marqués du sceau~; de la tribu de Lévi, douze mille marqués du sceau~; de la tribu d'Issacar, douze mille marqués du sceau~;
\VS{8}de la tribu de Zabulon, douze mille marqués du sceau~; de la tribu de Joseph, douze mille marqués du sceau~; de la tribu de Benjamin, douze mille marqués du sceau.
\TextTitle{Multitude de sauvés pendant la grande tribulation}
\VS{9}Après cela, je regardai, et voici une grande multitude de gens, que personne ne pouvait compter, de toute nation, de toute tribu, de tout peuple et de toute langue, se tenaient devant le trône, et devant l'Agneau, vêtus de longues robes blanches, et ils avaient des palmes dans leurs mains.
\VS{10}Et ils criaient d'une voix forte, en disant~: Le salut est à notre Dieu, qui est assis sur le trône, et à l'Agneau.
\VS{11}Et tous les anges se tenaient autour du trône, et des anciens, et des quatre animaux, et ils se prosternèrent devant le trône sur leurs faces et adorèrent Dieu,
\VS{12}en disant~: Amen~! La louange, la gloire, la sagesse, les actions de grâces, l'honneur, la puissance et la force soient à notre Dieu, aux siècles des siècles. Amen~!
\VS{13}Et l'un des anciens prit la parole et me dit~: Ceux qui sont revêtus de longues robes blanches, qui sont-ils et d'où sont-ils venus~?
\VS{14}Et je lui dis~: Seigneur, tu le sais. Et il me dit~: Ce sont ceux qui sont venus de la grande tribulation\FTNT{Les saints ont toujours été persécutés. Cela a débuté dès la Genèse avec Caïn qui tua son frère Abel (Ge. 4:5-10). La grande tribulation correspond néanmoins à une période de persécutions particulièrement cruelles qui seront orchestrées par l'homme impie (à la tête de plusieurs nations) principalement contre les Juifs (Jé. 30:7~; Da. 9:24~; Lu. 21:20-24) et sans doute contre les personnes converties à Christ issues des nations (Ap. 7:9-17~; Ap. 12:17). Le Seigneur Jésus a prédit la grande tribulation à ses disciples (Mt. 24:15-29~; Mc. 13:14-19) en précisant qu'en ce temps là on verrait «~l'abomination de la désolation~» établie en lieu saint prophétisée par Daniel (Da. 11:31). La grande tribulation durera trois ans et demi, c'est ce que Daniel appelle «~un temps, des temps et la moitié d'un temps~» (Da. 7:25~; Ap. 11:3.) L'ère de paix factice instaurée par l'impie cédera alors soudainement la place à un temps d'angoisse sans précédent (1 Th. 5:3).}, et qui ont lavé et blanchi leurs longues robes dans le sang de l'Agneau.
\VS{15}C'est pourquoi ils sont devant le trône de Dieu, et ils le servent jour et nuit dans son temple~; et celui qui est assis sur le trône habitera avec eux.
\VS{16}Ils n'auront plus faim ni soif, et le soleil ne les frappera plus, ni aucune chaleur.
\VS{17}Car l'Agneau qui est au milieu du trône les paîtra, et les conduira aux sources des eaux de la vie, et Dieu essuiera toutes les larmes de leurs yeux.
\Chap{8}
\TextTitle{Septième sceau~: Annonce des sept trompettes\FTNTT{Ap. 4:1.}}
\VerseOne{}Et quand il eut ouvert le septième sceau, il y eut un silence dans le ciel d'environ une demi-heure.
\VS{2}Et je vis les sept anges qui se tiennent devant Dieu, et sept trompettes leur furent données.
\VS{3}Et un autre ange vint et se tint devant l'autel, ayant un encensoir d'or, et plusieurs parfums lui furent donnés pour les offrir, avec les prières de tous les saints, sur l'autel d'or qui est devant le trône.
\VS{4}Et la fumée des parfums monta avec les prières des saints de la main de l'ange devant Dieu.
\VS{5}Puis l'ange prit l'encensoir, et l'ayant rempli du feu de l'autel, il le jeta sur la terre~; et il y eut des tonnerres, des voix, des éclairs, et un tremblement de terre.
\VS{6}Alors les sept anges qui avaient les sept trompettes se préparèrent à en sonner.
\TextTitle{Première trompette~: Grêle et feu mêlés de sang}
\VS{7}Et le premier ange sonna de la trompette. Et il y eut de la grêle et du feu mêlés de sang, qui furent jetés sur la terre~; et le tiers des arbres fut brûlé, et toute herbe verte aussi fut brûlée.
\TextTitle{Deuxième trompette~: La montagne embrasée}
\VS{8}Et le second ange sonna de la trompette, et je vis comme une grande montagne embrasée de feu, qui fut jetée dans la mer~; et le tiers de la mer devint du sang,
\VS{9}et le tiers des créatures vivantes qui étaient dans la mer mourut, et le tiers des navires périt.
\TextTitle{Troisième trompette~: Absinthe, l'étoile tombée du ciel}
\VS{10}Et le troisième ange sonna de la trompette, et il tomba du ciel une grande étoile ardente comme un flambeau, et elle tomba sur le tiers des fleuves et sur les sources des eaux.
\VS{11}Le nom de l'étoile est Absinthe~; et le tiers des eaux fut changé en absinthe, et beaucoup d'hommes moururent par les eaux, parce qu'elles étaient devenues amères.
\TextTitle{Quatrième trompette~: Des signes dans le ciel}
\VS{12}Puis le quatrième ange sonna de la trompette, et le tiers du soleil fut frappé, ainsi que le tiers de la lune, et le tiers des étoiles, afin que le tiers en soit obscurci~; le jour fut privé d'un tiers de sa clarté, et la nuit de même.
\VS{13}Je regardai, et j'entendis un ange qui volait au milieu du ciel et qui disait à haute voix~: Malheur~! Malheur~! Malheur aux habitants de la terre à cause des autres sons de trompettes que les trois autres anges vont faire retentir.
\Chap{9}
\TextTitle{Cinquième trompette~: Ouverture du puits de l'abîme}
\VerseOne{}Le cinquième ange sonna de la trompette, et je vis une étoile qui tomba du ciel sur la terre, et la clef du puits de l'abîme fut donnée à cet ange.
\VS{2}Et il ouvrit le puits de l'abîme, et une fumée monta du puits comme la fumée d'une grande fournaise~; et le soleil et l'air furent obscurcis par la fumée du puits.
\VS{3}Des sauterelles sortirent de la fumée du puits et se répandirent sur la terre, et il leur fut donné un pouvoir comme le pouvoir qu'ont les scorpions de la terre.
\VS{4}Et il leur fut dit de ne pas faire de mal à l'herbe de la terre, ni à aucune verdure, ni à aucun arbre, mais seulement aux hommes qui n'avaient pas la marque de Dieu sur leurs fronts.
\VS{5}Et il leur fut donné, non de les tuer, mais de les tourmenter pendant cinq mois~; et le tourment qu'elles causaient était comme le tourment que cause le scorpion quand il pique un homme.
\VS{6}Et en ces jours-là, les hommes chercheront la mort, mais ils ne la trouveront pas~; et ils désireront mourir, mais la mort fuira loin d'eux.
\VS{7}Ces sauterelles ressemblaient à des chevaux préparés pour la guerre, et sur leurs têtes il y avait comme des couronnes semblables à de l'or, et leurs faces étaient comme des faces d'hommes.
\VS{8}Elles avaient les cheveux comme des cheveux de femmes~; et leurs dents étaient comme des dents de lions.
\VS{9}Elles avaient des cuirasses comme des cuirasses de fer~; et le bruit de leurs ailes était comme le bruit des chars à plusieurs chevaux qui courent à la guerre.
\VS{10}Elles avaient des queues armées d'aiguillons, comme les scorpions, et c'est dans leurs queues qu'était le pouvoir de faire du mal aux hommes pendant cinq mois.
\VS{11}Elles avaient sur elles comme roi l'ange de l'abîme, dont le nom en hébreu est Abaddon, mais en grec son nom est Apollyon\FTNT{Abaddon ou Apollyon~: Le nom de ce démon signifie «~Le destructeur~».}.
\VS{12}Le premier malheur est passé, et voici venir encore deux malheurs après celui-ci.
\TextTitle{Sixième trompette~: Les quatre anges de l'Euphrate déliés\FTNTT{Ap. 16:12.}}
\VS{13}Alors le sixième ange sonna de sa trompette, et j'entendis une voix sortant des quatre cornes de l'autel d'or qui est devant Dieu,
\VS{14}et disant au sixième ange qui avait la trompette~: Délie les quatre anges qui sont liés sur le grand fleuve, l'Euphrate.
\VS{15}On délia donc les quatre anges qui étaient prêts pour l'heure, le jour, le mois et l'année, afin de tuer le tiers des hommes.
\VS{16}Le nombre des cavaliers de l'armée était de deux cents millions, car j'en entendis le nombre.
\VS{17}Et je vis aussi dans la vision les chevaux et ceux qui étaient montés dessus, ayant des cuirasses de feu, d'hyacinthe et de soufre~; et les têtes des chevaux étaient comme des têtes de lions~; et de leurs bouches sortaient du feu, de la fumée et du soufre.
\VS{18}Le tiers des hommes fut tué par ces trois fléaux, par le feu, et par la fumée et par le soufre qui sortaient de leur bouche.
\VS{19}Car le pouvoir des chevaux était dans leurs bouches et dans leurs queues~; et leurs queues étaient semblables à des serpents ayant des têtes, et c'est avec elles qu'ils faisaient du mal.
\VS{20}Mais les autres hommes qui ne furent pas tués par ces fléaux, ne se repentirent pas des œuvres de leurs mains, ils ne cessèrent pas d'adorer les démons, les idoles d'or, d'argent, de cuivre, de pierre, et de bois, qui ne peuvent ni voir, ni entendre, ni marcher.
\VS{21}Et ils ne se repentirent pas aussi de leurs meurtres, ni de leurs enchantements, ni de leur impudicité, ni de leurs vols.
\Chap{10}
\TextTitle{Un ange puissant descend du ciel}
\VerseOne{}Je vis un autre ange puissant qui descendait du ciel, environné d'une nuée, au-dessus de sa tête était l'arc-en-ciel, son visage était comme le soleil, et ses pieds comme des colonnes de feu.
\VS{2}Et il avait dans sa main un petit livre ouvert, et il posa son pied droit sur la mer, et le pied gauche sur la terre~;
\VS{3}et il cria d'une voix forte, comme lorsqu'un lion rugit. Et quand il eut crié, les sept tonnerres firent entendre leurs voix.
\VS{4}Et après que les sept tonnerres eurent fait entendre leurs voix, j'allais écrire, mais j'entendis une voix du ciel qui me disait~: Scelle les choses que les sept tonnerres ont fait entendre, et ne les écris pas.
\VS{5}Et l'ange que j'avais vu se tenant sur la mer et sur la terre, leva sa main vers le ciel,
\VS{6}et jura par celui qui est vivant aux siècles des siècles, qui a créé le ciel avec les choses qui y sont, et la terre avec les choses qui y sont, et la mer avec les choses qui y sont, qu'il n'y aurait plus de temps~;
\VS{7}mais qu'aux jours de la voix du septième ange, quand il commencera à sonner de la trompette, le mystère de Dieu sera accompli, comme il l'a déclaré à ses serviteurs les prophètes.
\TextTitle{Nouvelle mission de Jean}
\VS{8}Et la voix que j'avais entendue du ciel me parla encore et me dit~: Va, et prends le petit livre ouvert qui est dans la main de l'ange qui se tient sur la mer et sur la terre.
\VS{9}Et j'allai vers l'ange, en lui disant~: Donne-moi le petit livre~; et il me dit~: Prends-le et mange-le~; il remplira tes entrailles d'amertume, mais il sera doux dans ta bouche comme du miel\FTNT{Ez. 3:1-3.}.
\VS{10}Je pris donc le petit livre de la main de l'ange, et je le mangeai~; il fut doux dans ma bouche comme du miel, mais quand je l'eus mangé, mes entrailles furent remplies d'amertume.
\VS{11}Alors il me dit~: Il faut que tu prophétises de nouveau sur beaucoup de peuples, et sur plusieurs nations, sur plusieurs langues et plusieurs rois.
\Chap{11}
\TextTitle{Le temps des nations}
\VerseOne{}On me donna un roseau semblable à une verge, et l'ange se présenta et me dit~: Lève-toi et mesure le temple de Dieu et l'autel, et ceux qui y adorent.
\VS{2}Mais laisse de côté le parvis extérieur du temple, et ne le mesure pas~; car il est donné aux Gentils, et ils fouleront aux pieds la ville sainte pendant quarante-deux mois\FTNT{C'est le temps que durera la grande tribulation, soit trois ans et demi. Daniel parle d'une semaine, un jour comptant pour une année (Da. 9:27). La grande tribulation débutera à la moitié de cette semaine, ce qui correspond bien à quarante-deux mois (Ap. 13:5) et à mille deux cent soixante jours (Ap. 11:3~; Ap. 12:6).}.
\TextTitle{Les deux témoins ressuscitent}
\VS{3}Mais je donnerai à mes deux témoins de prophétiser pendant mille deux cent soixante jours, revêtus de sacs.
\VS{4}Ce sont les deux oliviers\FTNT{Za. 4:14.} et les deux chandeliers qui se tiennent devant le Dieu de la terre. 
\VS{5}Et si quelqu'un veut leur faire du mal, du feu sort de leurs bouches et dévore leurs ennemis~; car si quelqu'un veut leur faire du mal, il faut qu'il soit tué de cette manière.
\VS{6}Ils ont le pouvoir de fermer le ciel, afin qu'il ne pleuve pas pendant les jours de leur prophétie~; ils ont aussi le pouvoir de changer les eaux en sang, et de frapper la terre de toutes sortes de plaies, toutes les fois qu'ils le voudront.
\VS{7}Et quand ils auront achevé de rendre leur témoignage, la bête qui monte de l'abîme\FTNT{L'homme impie, l'Antichrist, ou encore le fils de la perdition dont il est question dans 2 Th. 2:3~; 2 Th. 2:8-9.} leur fera la guerre, les vaincra, et les tuera.
\VS{8}Et leurs cadavres seront étendus sur les places de la grande ville, qui est appelée spirituellement Sodome et Egypte, où aussi notre Seigneur a été crucifié.
\VS{9}Et ceux des tribus, des peuples, des langues, et des nations verront leurs cadavres pendant trois jours et demi, et ils ne permettront pas que leurs cadavres soient mis dans des sépulcres.
\VS{10}Et les habitants de la terre se réjouiront, ils seront dans l'allégresse, ils s'enverront des présents les uns aux autres, parce que ces deux prophètes ont tourmenté les habitants de la terre.
\VS{11}Mais après ces trois jours et demi, l'Esprit de vie venant de Dieu entra en eux, et ils se tinrent sur leurs pieds, et une grande crainte saisit ceux qui les virent.
\VS{12}Après cela, ils entendirent une forte voix du ciel, leur disant~: Montez ici~! Et ils montèrent au ciel sur une nuée, et leurs ennemis les virent.
\VS{13}Et à cette même heure-là, il eut un grand tremblement de terre, et la dixième partie de la ville tomba, et sept mille hommes furent tués par ce tremblement de terre~; et les autres furent épouvantés et donnèrent gloire au Dieu du ciel.
\VS{14}Le second malheur est passé. Voici, le troisième malheur vient bientôt.
\TextTitle{Septième trompette~: Le règne du Messie annoncé, cantique des vingt-quatre vieillards\FTNTT{Ap. 8:2.}}
\VS{15}Le septième ange sonna de la trompette, et il se fit entendre au ciel de grandes voix qui disaient~: Les royaumes du monde sont soumis à notre Seigneur et à son Christ, et il régnera aux siècles des siècles.
\VS{16}Alors les vingt-quatre anciens qui étaient assis devant Dieu sur leurs trônes, se prosternèrent sur leurs faces et adorèrent Dieu,
\VS{17}en disant~: Nous te rendons grâces, Seigneur Dieu Tout-Puissant, QUI ES, QUI ETAIS, et QUI VIENS, de ce que tu as fait éclater ta grande puissance, et de ce que tu as agi en Roi.
\VS{18}Les nations se sont irritées, mais ta colère est venue, et le temps est venu de juger les morts, et de donner la récompense à tes serviteurs les prophètes et aux saints, et à ceux qui craignent ton Nom, petits et grands, et de détruire ceux qui corrompent la terre.
\VS{19}Et le temple de Dieu fut ouvert dans le ciel, et l'arche de son alliance apparut dans son temple. Et il y eut des éclairs, des voix, des tonnerres, un tremblement de terre, et une grosse grêle.
\Chap{12}
\TextTitle{Vision de la femme et du dragon}
\VerseOne{}Et un grand signe parut dans le ciel~: Une femme revêtue du soleil, la lune sous ses pieds, et sur sa tête une couronne de douze étoiles\FTNT{En Ge. 37:9-10, Joseph raconte à ses parents et à ses frères un songe particulier où il voyait le soleil, la lune et onze étoiles se prosterner devant lui. Jacob comprit que les onze étoiles représentaient ses enfants, la lune sa femme Rachel, qui était la mère de Joseph, et que le soleil c'était lui-même. Il est donc question ici d'Israël, qui a toujours été identifié à une femme (Ez. 16) de qui est issu le Messie selon la chair (Ro. 9:5).}.
\VS{2}Elle était enceinte, et elle criait, étant en travail d'enfant, souffrant les grandes douleurs de l'enfantement.
\VS{3}Il parut aussi un autre signe dans le ciel, et voici un grand dragon rouge feu ayant sept têtes et dix cornes, et sur ses têtes sept diadèmes.
\VS{4}Sa queue entraînait le tiers des étoiles du ciel et les jeta sur la terre\FTNT{Da. 8:10.}. Puis le dragon s'arrêta devant la femme qui devait accoucher, afin de dévorer son enfant\FTNT{Cet enfant est évidemment Jésus-Christ (Mt. 2:16.)}, dès qu'elle l'aurait mis au monde.
\TextTitle{La naissance du Messie}
\VS{5}Et elle accoucha d'un fils, qui doit gouverner toutes les nations avec un sceptre de fer\FTNT{Ps. 2:8-9.}. Et son enfant fut enlevé vers Dieu et vers son trône\FTNT{Lu. 24:51~; Ac. 1:9-11.}.
\VS{6}Et la femme s'enfuit dans un désert, où elle avait un lieu préparé par Dieu, afin d'y être nourrie pendant mille deux cent soixante jours.
\TextTitle{Guerre entre l'archange Michel et le dragon}
\VS{7}Et il y eut une guerre dans le ciel. Michel et ses anges combattirent contre le dragon. Et le dragon et ses anges combattirent contre Michel,
\VS{8}mais ils ne furent pas les plus forts, et leur place ne fut plus trouvée dans le ciel.
\VS{9}Et il fut précipité le grand dragon, le serpent ancien, appelé le diable et Satan, celui qui séduit toute la terre, il fut précipité sur la terre, et ses anges furent précipités avec lui\FTNT{Es. 14:12-15~; Ez. 28~; Lu. 10:18.}.
\VS{10}Et j'entendis une voix forte dans le ciel qui disait~: Maintenant le salut est arrivé, ainsi que la force, le règne de notre Dieu, et la puissance de son Christ~; car l'accusateur de nos frères, qui les accusait devant notre Dieu jour et nuit, a été précipité.
\VS{11}Et ils l'ont vaincu à cause du sang de l'Agneau, et à cause de la parole de leur témoignage, et ils n'ont pas aimé leurs vies, mais les ont exposées à la mort.
\VS{12}C'est pourquoi réjouissez-vous cieux, et vous qui y habitez. Mais malheur à vous habitants de la terre et de la mer~! Car le diable est descendu vers vous animé d'une grande fureur, sachant qu'il a peu de temps.
\TextTitle{Le dragon persécute la femme, sa postérité et les témoins du Messie}
\VS{13}Quand le dragon vit qu'il avait été précipité sur la terre, il persécuta la femme qui avait enfanté le fils.
\VS{14}Mais deux ailes d'un grand aigle furent données à la femme, afin qu'elle s'envole de devant le serpent au désert, où elle est nourrie un temps, des temps, et la moitié d'un temps.
\VS{15}Et de sa gueule, le serpent lança de l'eau comme un fleuve derrière la femme, afin de l'entraîner par le fleuve.
\VS{16}Mais la terre secourut la femme, elle ouvrit sa bouche, et elle engloutit le fleuve que le dragon avait lancé de sa gueule.
\VS{17}Alors le dragon fut irrité contre la femme, et s'en alla faire la guerre contre les autres qui sont de la semence de la femme, qui gardent les commandements de Dieu, et qui ont le témoignage de Jésus-Christ.
\VS{18}Et je me tins sur le sable qui borde la mer.
\Chap{13}
\TextTitle{La bête qui monte de la mer, l'antichrist}
\VerseOne{}Et je vis monter de la mer une bête\FTNT{Cette bête représente deux entités. Tout d'abord l'homme impie, l'Antichrist, et ensuite un système politique. Les dix cornes sur sa tête symbolisent les dix nations les plus puissantes de la terre avec lesquelles il imposera sa dictature mondiale (Da. 7:16-25). L'alliage des quatre métaux dans la statue de Nébucadnetsar en Da. 2 et la vision des quatre animaux en Da. 7, annoncent l'instauration d'un quatrième empire ou encore le système politique à la tête duquel sera la bête.} qui avait sept têtes et dix cornes, et sur ses cornes dix diadèmes, et sur ses têtes des noms de blasphème\FTNT{Voir annexe «~La bête d'apocalypse~».}.
\VS{2}Et la bête que je vis était semblable à un léopard, ses pieds étaient comme ceux d'un ours~; sa gueule était comme la gueule d'un lion\FTNT{Da. 7:7.}. Et le dragon lui donna sa puissance, son trône, et une grande autorité.
\VS{3}Et je vis l'une de ses têtes comme blessée à mort, mais sa blessure mortelle fut guérie. Remplie d'admiration, la terre entière suivit la bête.
\VS{4}Et ils adorèrent le dragon, parce qu'il avait donné l'autorité à la bête, et ils adorèrent aussi la bête, en disant~: Qui est semblable à la bête, et qui peut combattre contre elle~?
\VS{5}Et il lui fut donné une bouche qui proférait des discours pleins d'orgueil, et des blasphèmes~; et il lui fut aussi donné le pouvoir d'agir pendant quarante-deux mois.
\VS{6}Elle ouvrit sa bouche pour blasphémer contre Dieu, pour blasphémer son Nom et son tabernacle, et ceux qui habitent dans le ciel.
\VS{7}Et il lui fut donné de faire la guerre aux saints et de les vaincre. Il lui fut aussi donné autorité sur toute tribu, toute langue et toute nation.
\VS{8}Et tous les habitants de la terre l'adoreront, ceux dont les noms n'ont pas été écrits dans le livre de vie de l'Agneau immolé dès la fondation du monde.
\VS{9}Si quelqu'un a des oreilles qu'il entende.
\VS{10}Si quelqu'un est destiné à la captivité, il ira en captivité~; si quelqu'un tue avec l'épée, il faut qu'il soit lui-même tué avec l'épée. C'est ici la persévérance et la foi des saints.
\TextTitle{La bête qui monte de la terre, le faux prophète}
\VS{11}Puis je vis une autre bête qui montait de la terre\FTNT{Cette bête est identifiée au faux-prophète car son rôle consiste à amener les habitants de la terre à adorer la première bête, tout comme les vrais prophètes invitent les gens à l'adoration du Dieu véritable (Mt. 7:15).}, et qui avait deux cornes semblables à celles de l'Agneau~; mais elle parlait comme le dragon.
\VS{12}Et elle exerçait toute l'autorité de la première bête en sa présence, et elle obligeait la terre et ses habitants à adorer la première bête, dont la blessure mortelle avait été guérie\FTNT{Cette bête a existé par le passé sous la forme de l'empire romain qui s'est écroulé le 4 septembre 476. Ce régime a marqué l'histoire par son caractère universel et brutal. Le fait que cette bête blessée à mort reprenne vie, annonce l'instauration d'un empire universel qui aura les caractéristiques combinées de l'empire babylonien, médo-perse, gréco-macédonien et romain, ceux-ci correspondant aux quatre animaux de la vision de Da. 7:1-8~: Le lion, l'ours, le léopard et le quatrième animal.}.
\VS{13}Elle opérait de grands prodiges, même jusqu'à faire descendre le feu du ciel sur la terre devant les hommes.
\VS{14}Et elle séduisait les habitants de la terre, à cause des prodiges qu'il lui était donné d'opérer en présence de la bête, disant aux habitants de la terre de faire une image\FTNT{Dieu interdit la vénération des images (Ex. 20:4-5) La particularité de l'image de la bête est qu'elle possède un esprit (démon).} de la bête qui avait reçu le coup mortel de l'épée, et qui était bien vivante.
\VS{15}Et il lui fut donné de mettre un esprit à l'image de la bête, afin que même l'image de la bête parle, et qu'elle fasse que tous ceux qui n'adoreraient pas l'image de la bête soient mis à mort.
\VS{16}Elle fit que tous, petits et grands, riches et pauvres, libres et esclaves, reçoivent une marque sur leur main droite, ou sur leur front\FTNT{Il s'agit d'une marque qui est avant tout spirituelle. Car de la même façon que nous sommes scellés et marqués par l'Esprit de Dieu qui produit en nous la sainteté (Ga. 5:22~; Ro. 6:20-22~; Ep. 1:13~; Ep. 4:30) Satan marque les siens par le péché (1 Ti. 4:1-2~; 2 Ti. 3:1-5).}~;
\VS{17}et que personne ne puisse acheter ni vendre, sans avoir la marque ou le nom de la bête, ou le nombre de son nom.
\VS{18}Ici est la sagesse~: Que celui qui a de l'intelligence compte le nombre de la bête, car c'est un nombre d'homme, et son nombre est six cent soixante-six.
\Chap{14}
\TextTitle{L'Agneau et les 144 000}
\VerseOne{}Puis je regardai, et voici, l'Agneau se tenait sur la montagne de Sion, et il y avait avec lui cent quarante-quatre mille personnes qui avaient le Nom de son Père écrit sur leurs fronts.
\VS{2}Et j'entendis une voix du ciel comme le bruit des grandes eaux, et comme le bruit d'un grand tonnerre~; et j'entendis une voix de joueurs de harpe jouant de leurs harpes.
\VS{3}Et ils chantaient comme un cantique nouveau devant le trône, et devant les quatre animaux, et devant les anciens. Et personne ne pouvait apprendre le cantique, si ce n'est les cent quarante-quatre mille qui avaient été rachetés de la terre.
\VS{4}Ce sont ceux qui ne se sont pas souillés avec les femmes, car ils sont vierges~; ce sont ceux qui suivent l'Agneau partout où il va. Ils ont été rachetés d'entre les hommes pour être des prémices pour Dieu et pour l'Agneau.
\VS{5}Et dans leur bouche il ne s'est pas trouvé de fraude, car ils sont sans tache devant le trône de Dieu\FTNT{Ps. 32:2.}.
\TextTitle{l'Evangile éternel et la chute de Babylone}
\VS{6}Puis je vis un autre ange qui volait au milieu du ciel, il avait l'Evangile éternel pour évangéliser les habitants de la terre, de toute nation, de toute tribu, de toute langue et de tout peuple.
\VS{7}Il disait d'une voix forte~: Craignez Dieu, et donnez-lui gloire, car l'heure de son jugement est venue~; et adorez celui qui a fait le ciel et la terre, la mer et les sources des eaux.
\VS{8}Et un autre ange le suivit, disant~: Elle est tombée, elle est tombée Babylone, la grande ville, parce qu'elle a abreuvé toutes les nations du vin de la fureur de son impudicité~!
\TextTitle{Le jugement des adorateurs de la bête}
\VS{9}Et un troisième ange les suivit, disant d'une voix forte~: Si quelqu'un adore la bête et son image, et reçoit la marque sur son front ou sur sa main,
\VS{10}il boira, lui aussi, du vin de la colère de Dieu, du vin pur versé dans la coupe de sa colère, et il sera tourmenté dans le feu et le soufre devant les saints anges et devant l'Agneau.
\VS{11}Et la fumée de leur tourment montera aux siècles des siècles, et ils n'auront de repos ni jour ni nuit, ceux qui adorent la bête et son image, et quiconque reçoit la marque de son nom.
\VS{12}Ici est la persévérance des saints~; ici sont ceux qui gardent les commandements de Dieu, et la foi de Jésus.
\TextTitle{Bénédiction de ceux qui meurent en Christ}
\VS{13}Alors j'entendis une voix du ciel qui me disait~: Ecris~: Bénis sont dès à présent les morts qui meurent dans le Seigneur~! Oui, c'est vrai~! dit l'Esprit, afin qu'ils se reposent de leurs travaux, car leurs œuvres les suivent.
\TextTitle{Prophétie sur Harmaguédon}
\VS{14}Et je regardai, et voici, il y avait une nuée blanche, et sur la nuée était assis quelqu'un qui ressemblait à un homme\FTNT{Ez. 1:26~; Da. 7:13~; Mt. 24:30~; Mt. 26:64~; Ap. 1:13.}, ayant sur sa tête une couronne d'or, et dans sa main une faucille tranchante.
\VS{15}Et un autre ange sortit du temple, criant à haute voix à celui qui était assis sur la nuée~: Jette ta faucille, et moissonne~; car c'est ton heure de moissonner, parce que la moisson de la terre est mûre\FTNT{Jé. 51:33~; Mt. 13:30-39.}.
\VS{16}Alors celui qui était assis sur la nuée jeta sa faucille sur la terre, et la terre fut moissonnée.
\VS{17}Et un autre ange sortit du temple qui est dans le ciel, ayant lui aussi une faucille tranchante.
\VS{18}Et un autre ange, qui avait autorité sur le feu, sortit de l'autel, et s'adressant d'une voix forte à celui qui avait la faucille tranchante, dit~: Jette ta faucille tranchante, et vendange les grappes de la vigne de la terre, car ses raisins sont mûrs.
\VS{19}Et l'ange jeta sa faucille tranchante sur la terre et vendangea la vigne de la terre, et il jeta la vendange dans la grande cuve de la colère de Dieu.
\VS{20}Et la cuve fut foulée hors de la ville~; et du sang sortit de la cuve, jusqu'aux mors des chevaux, sur une étendue de mille six cents stades\FTNT{Es. 63:1-6.}.
\Chap{15}
\TextTitle{Une scène glorieuse au ciel}
\VerseOne{}Puis je vis dans le ciel un autre signe, grand et admirable~: Sept anges qui tenaient les sept derniers fléaux, car c'est par eux que s'accomplit la colère de Dieu.
\VS{2}Et je vis aussi comme une mer de verre mêlée de feu, et ceux qui avaient vaincu la bête et son image, et sa marque, et le nombre de son nom, étaient debout sur la mer qui était comme de verre, et ayant les harpes de Dieu.
\VS{3}Ils chantaient le cantique de Moïse, serviteur de Dieu, et le cantique de l'Agneau, en disant~: Tes œuvres sont grandes et merveilleuses, ô Seigneur Dieu Tout-Puissant~! Tes voies sont justes et véritables, ô Roi des saints~!
\VS{4}Seigneur, qui ne te craindrait, et qui ne glorifierait ton Nom~? Car toi seul tu es Saint, c'est pourquoi toutes les nations viendront et se prosterneront devant toi~; car tes jugements sont pleinement manifestés.
\VS{5}Et après ces choses, je regardai, et voici le temple du tabernacle du témoignage fut ouvert dans le ciel.
\VS{6}Et les sept anges qui avaient les sept fléaux sortirent du temple, revêtus d'un lin pur et blanc, et ayant des ceintures d'or autour de leurs poitrines.
\VS{7}Et l'un des quatre animaux donna aux sept anges sept coupes d'or, pleines de la colère du Dieu qui vit aux siècles des siècles.
\VS{8}Et le temple fut rempli de la fumée à cause de la gloire de Dieu et de sa puissance~; et personne ne pouvait entrer dans le temple jusqu'à ce que les sept fléaux des sept anges soient accomplis.
\Chap{16}
\TextTitle{Première coupe~: Les ulcères}
\VerseOne{}Et j'entendis du temple une voix éclatante qui disait aux sept anges~: Allez, et versez sur la terre les coupes de la colère de Dieu.
\VS{2}Et le premier ange s'en alla, et versa sa coupe sur la terre. Et un ulcère malin et dangereux frappa les hommes qui avaient la marque de la bête, et ceux qui adoraient son image.
\TextTitle{Deuxième coupe~: La mer changée en sang}
\VS{3}Et le second ange versa sa coupe sur la mer, et elle devint comme le sang d'un corps mort, et tout être qui vivait dans la mer mourut.
\TextTitle{Troisième coupe~: Les sources changées en sang}
\VS{4}Et le troisième ange versa sa coupe sur les fleuves et sur les sources des eaux, et elles devinrent du sang.
\VS{5}Et j'entendis l'ange des eaux qui disait~: Seigneur, QUI ES, QUI ETAIS, et QUI VIENS, tu es juste, parce que tu as exercé ce jugement.
\VS{6}Parce qu'ils ont répandu le sang des saints et des prophètes, tu leur as aussi donné du sang à boire, car ils le méritent.
\VS{7}Et j'entendis un autre de l'autel, qui disait~: Certainement, Seigneur Dieu Tout-Puissant, tes jugements sont véritables et justes.
\TextTitle{Quatrième coupe~: Une chaleur extrême}
\VS{8}Ensuite, le quatrième ange versa sa coupe sur le soleil, et le pouvoir lui fut donné de brûler les hommes par le feu,
\VS{9}de sorte que les hommes furent brûlés par de grandes chaleurs, et ils blasphémèrent le Nom de Dieu qui a puissance sur ces fléaux~; et ils ne se repentirent pas pour lui donner gloire.
\TextTitle{Cinquième coupe~: Les ténèbres sur le trône de la bête}
\VS{10}Après cela, le cinquième ange versa sa coupe sur le trône de la bête. Et son royaume fut couvert de ténèbres, et les hommes se mordaient la langue à cause de la douleur qu'ils ressentaient.
\VS{11}Et ils blasphémèrent le Dieu du ciel à cause de leurs douleurs et de leurs ulcères~; et ils ne se repentirent pas de leurs œuvres.
\TextTitle{Sixième coupe~: L'Euphrate asséché}
\VS{12}Puis le sixième ange versa sa coupe sur le grand fleuve, l'Euphrate. Et son eau tarit, afin de préparer la voie des rois venant du côté où le soleil se lève.
\VS{13}Et je vis sortir de la gueule du dragon, et de la gueule de la bête, et de la bouche du faux prophète, trois esprits impurs semblables à des grenouilles.
\VS{14}Car ce sont des esprits de démons, qui font des prodiges, et qui vont vers les rois de la terre et du monde entier, afin de les assembler pour le combat de ce grand jour du Dieu Tout-Puissant.
\VS{15}Voici, je viens comme un voleur. Béni est celui qui veille et qui garde ses vêtements, afin de ne pas marcher nu, et qu'on ne voie pas sa honte~!
\VS{16}Et ils les assemblèrent dans le lieu qui est appelé en hébreu Harmaguédon\FTNT{Le terme «~Harmaguédon~», mentionné uniquement dans ce passage, vient du mot hébreu «~Har-Magidown~», ce qui signifie «~Montagne de Megiddo~». Bien qu'il n'existe pas de montagne portant spécifiquement ce nom, l'emplacement probable de cet endroit est la plaine de Meggido se trouvant à proximité de Jérusalem. Par le passé, elle fut le théâtre de la victoire de Barak sur les Cananéens (Jg. 4:15) et de celle de Gédéon sur les Madianites (Jg. 7). C'est aussi à cet endroit que Saül et ses fils (1 Sa. 31~:8) ainsi que le roi Josias (2 R. 23:29-30~; 2 Ch. 35:22) trouvèrent la mort. Pour toutes ces raisons, elle devint au fil du temps le symbole de l'affrontement entre Dieu et la puissance des ténèbres. Selon les prophéties bibliques, la plaine de Meggido et la vallée de Jizréel constitueront le site de l'ultime guerre mondiale, celle opposant l'Antichrist et ses alliés (dirigeants des nations) contre Israël. Le Seigneur interviendra alors ouvertement dans les affaires humaines pour déverser la coupe de sa colère (Ap. 16:1) et anéantir l'homme impie et toute son armée (Ez. 38-39~; Joë. 3~; Mi. 4:11~; So. 1~; Za. 14~; Mt. 24:29-30~; Ap. 20:1-3~; Ap. 20:7-10).}.
\TextTitle{Septième coupe~: Une grosse grêle tombe du ciel}
\VS{17}Puis le septième ange versa sa coupe dans l'air~; et il sortit du temple du ciel une voix forte qui venait du trône, disant~: C'en est fait.
\VS{18}Et il y eut des éclairs, et des voix, et des tonnerres, et il se fit un grand tremblement de terre, dis-je, tel qu'il n'y en avait jamais eu depuis que les hommes sont sur la terre.
\VS{19}La grande ville fut divisée en trois parties, et les villes des nations tombèrent, et Dieu se souvint de Babylone la grande, pour lui donner la coupe du vin de son ardente colère.
\VS{20}Toutes les îles s'enfuirent et les montagnes ne furent plus retrouvées.
\VS{21}Une grosse grêle, dont les grêlons pesaient un talent\FTNT{Un talent d'argent pesait 45 kg, un talent d'or pesait 90 kg.}, tomba du ciel sur les hommes~; et les hommes blasphémèrent Dieu, à cause du fléau de la grêle, car le fléau qu'elle causa fut très grand.
\Chap{17}
\TextTitle{La prostituée}
\VerseOne{}Puis l'un des sept anges qui tenaient les sept coupes vint, et il m'adressa la parole, en disant~: Viens, je te montrerai le jugement de la grande prostituée, qui est assise sur les grandes eaux.
\VS{2}Avec elle, les rois de la terre ont commis la fornication, et les habitants de la terre ont été enivrés du vin de sa prostitution.
\VS{3}Il me transporta en esprit dans un désert~; et je vis une femme assise sur une bête écarlate, pleine de noms de blasphème, ayant sept têtes et dix cornes.
\VS{4}Et la femme était vêtue de pourpre et d'écarlate, et parée d'or, de pierres précieuses, et de perles~; et elle tenait à la main une coupe d'or, pleine des abominations de l'impureté de sa prostitution.
\VS{5}Et il y avait sur son front un nom écrit, un mystère~: Babylone la grande, la mère des impudicités et des abominations de la terre\FTNT{Symboliquement, Babylone la grande incarne l'Eglise apostate. Elle est soutenue par la bête qu'elle chevauche, c'est-à-dire l'homme impie. Ces deux entités forment un système impie où la politique et la religion se mélangent (Da. 2:43).}.
\VS{6}Et je vis cette femme ivre du sang des saints, et du sang des martyrs de Jésus. Et quand je la vis, je fus saisi d'un grand étonnement.
\TextTitle{Alliance entre la prostituée et la bête}
\VS{7}Et l'ange me dit~: Pourquoi t'étonnes-tu~? Je te dirai le mystère de la femme et de la bête qui la porte, qui a les sept têtes et les dix cornes.
\VS{8}La bête que tu as vue, était, et elle n'est plus. Elle doit monter de l'abîme, et aller à la perdition. Et les habitants de la terre, ceux dont les noms ne sont pas écrits dans le Livre de vie dès la fondation du monde, s'étonneront en voyant la bête parce qu'elle était, et qu'elle n'est plus, et qui toutefois est.
\VS{9}C'est ici qu'il faut un esprit intelligent et qui ait de la sagesse. Les sept têtes sont sept montagnes sur lesquelles la femme est assise.
\VS{10}Ce sont aussi sept rois, les cinq sont tombés~; l'un est, et l'autre n'est pas encore venu~; et quand il sera venu, il faut qu'il demeure pour un peu de temps.
\VS{11}Et la bête qui était, et qui n'est plus, est elle-même un huitième roi, et elle est du nombre des sept, mais elle tend à sa ruine.
\VS{12}Et les dix cornes que tu as vues sont dix rois, qui n'ont pas encore commencé à régner, mais ils recevront autorité comme rois en même temps avec la bête, pour une heure.
\VS{13}Ils ont un même dessein, et ils donneront leur puissance et leur autorité à la bête.
\TextTitle{Victoire de l'Agneau sur la prostituée}
\VS{14}Ils combattront contre l'Agneau et l'Agneau les vaincra, parce qu'il est le Seigneur des seigneurs, et le Roi des rois~; et les appelés, les élus et les fidèles qui sont avec lui, les vaincront aussi.
\VS{15}Puis il me dit~: Les eaux que tu as vues, et sur lesquelles la prostituée est assise, sont des peuples, des nations et des langues.
\VS{16}Les dix cornes que tu as vues sur la bête haïront la prostituée, la rendront désolée et nue, la dépouilleront, et mangeront sa chair, et la brûleront au feu.
\VS{17}Car Dieu a mis dans leurs cœurs de faire ce qu'il lui plaît, et de former un même dessein, et de donner leur royaume à la bête, jusqu'à ce que les paroles de Dieu soient accomplies.
\VS{18}Et la femme que tu as vue, c'est la grande ville, qui règne sur les rois de la terre.
\Chap{18}
\TextTitle{Babylone détruite}
\VerseOne{}Après ces choses, je vis descendre du ciel un autre ange, qui avait une grande autorité, et la terre fut illuminée de sa gloire.
\VS{2}Il cria avec force à haute voix, et il dit~: Elle est tombée, elle est tombée Babylone la grande, et elle est devenue la demeure de démons, et la retraite de tout esprit impur, et le repaire de tout oiseau impur et exécrable.
\VS{3}Car toutes les nations ont bu du vin de sa prostitution effrénée, et les rois de la terre ont commis la fornication avec elle, et les marchands de la terre se sont enrichis par l'excès de son luxe.
\VS{4}Puis j'entendis une autre voix du ciel, qui disait~: Sortez de Babylone, mon peuple, afin que vous ne participiez pas à ses péchés, et que vous n'ayez pas de part à ses fléaux.
\VS{5}Car ses péchés sont montés jusqu'au ciel, et Dieu s'est souvenu de ses iniquités.
\VS{6}Rendez-lui selon ce qu'elle vous a fait, et payez-lui au double selon ses œuvres~; et dans la même coupe où elle vous a versé à boire versez-lui au double.
\VS{7}Autant elle s'est glorifiée et plongée dans le luxe, autant donnez-lui de tourment et de deuil~; car elle dit en son cœur~: Je siège en reine, je ne suis pas veuve, et je ne verrai pas de deuil.
\VS{8}C'est pourquoi ses plaies, qui sont la mort, le deuil, et la famine, viendront en un même jour, et elle sera entièrement brûlée au feu~; car le Seigneur Dieu qui la jugera est puissant.
\TextTitle{Conséquence de la chute de Babylone~: Gémissements des habitants de la terre}
\VS{9}Et les rois de la terre, qui ont commis la fornication avec elle, et qui ont vécu dans le luxe, la pleureront, et mèneront deuil sur elle en se frappant la poitrine, quand ils verront la fumée de son embrasement~;
\VS{10}et ils se tiendront éloignés dans la crainte de son tourment, et diront~: Malheur~! Malheur~! Babylone la grande, cette ville si puissante, comment ta condamnation est-elle venue en une seule heure~?
\VS{11}Les marchands de la terre aussi pleureront, et seront dans le deuil à cause d'elle, parce que personne n'achète plus de leurs marchandises,
\VS{12}qui sont des marchandises d'or, d'argent, de pierres précieuses, de perles, de fin lin, de pourpre, de soie, d'écarlate, de toute sorte de bois odoriférant, de toute espèce de bois de senteur, d'ivoire, et de toute espèce de vaisseaux de bois très précieux, d'airain, de fer, et de marbre,
\VS{13}du cinnamome, des parfums, des essences, de l'encens, du vin, de l'huile, de la fine fleur de farine, du blé, des bœufs, des brebis, des chevaux, des chars, des esclaves, et des âmes d'hommes.
\VS{14}Car les fruits du désir de ton âme se sont éloignés de toi, et toutes les choses délicates et excellentes sont perdues pour toi, et dorénavant tu ne les trouveras plus.
\VS{15}Les marchands, dis-je, de ces choses, qui se sont enrichis par elle, se tiendront éloignés, dans la crainte de son tourment~; ils pleureront et seront dans le deuil,
\VS{16}et diront~: Malheur~! Malheur~! La grande ville qui était vêtue de fin lin, de pourpre, d'écarlate, qui était parée d'or, ornée de pierres précieuses, et de perles, comment en une seule heure tant de richesses ont été détruites~?
\VS{17}Et tous les pilotes aussi, tous ceux qui naviguent vers ce lieu, tous les marins, et tous ceux qui exploitent la mer, se tiendront éloignés,
\VS{18}et, en voyant la fumée de son embrasement, ils s'écrieront, en disant~: Quelle ville était semblable à cette grande ville~?
\VS{19}Ils jetteront de la poussière sur leurs têtes, pleurant et menant deuil, ils crieront, en disant~: Malheur~! Malheur~! La grande ville, où se sont enrichis par son opulence tous ceux qui ont des navires sur la mer, comment a-t-elle été réduite en désert en une seule heure~?
\TextTitle{Réjouissance des anges suite à la chute de Babylone}
\VS{20}Ô ciel~! Réjouis-toi à cause d'elle~; et vous aussi saints apôtres et prophètes, réjouissez-vous~! Car Dieu l'a punie à cause de vous.
\VS{21}Alors un ange d'une grande force prit une pierre semblable à une grande meule, et la jeta dans la mer, en disant~: Ainsi sera précipitée avec impétuosité Babylone, cette grande ville, et elle ne sera plus retrouvée\FTNT{Jé. 51:63-64.}.
\VS{22}Et l'on entendra plus chez toi les sons des joueurs de harpe, des musiciens, des joueurs de flûte, et de ceux qui sonnent de la trompette~; et on ne trouvera plus chez toi aucun artisan d'un métier quelconque, on n'entendra plus chez toi le bruit de la meule,
\VS{23}et la lumière de la lampe ne brillera plus chez toi, et la voix de l'époux et de l'épouse ne sera plus entendue chez toi~; car tes marchands étaient des princes de la terre, et parce que par tes enchantements toutes les nations ont été séduites,
\VS{24}et l'on a trouvé chez elle le sang des prophètes et des saints, et de tous ceux qui ont été mis à mort sur la terre.
\Chap{19}
\TextTitle{Allégresse dans les cieux suite au jugement de la grande prostituée\FTNTT{Ap. 17:16-17~; 18:8.}}
\VerseOne{}Après cela, j'entendis dans le ciel une voix forte d'une foule nombreuse, disant~: Alléluia~! Le salut, la gloire, l'honneur et la puissance appartiennent au Seigneur, notre Dieu,
\VS{2}car ses jugements sont véritables et justes~; car il a jugé la grande prostituée qui a corrompu la terre par son impudicité, et parce qu'il a vengé le sang de ses serviteurs versé de la main de la prostituée.
\VS{3}Et ils dirent encore~: Alléluia~! Et sa fumée monte aux siècles des siècles.
\VS{4}Et les vingt-quatre anciens et les quatre animaux se prosternèrent sur leurs faces, et adorèrent Dieu, qui était assis sur le trône, en disant~: Amen~! Alléluia~!
\VS{5}Et il sortit du trône une voix qui disait~: Louez notre Dieu, vous tous ses serviteurs, et vous qui le craignez, tant les petits que les grands\FTNT{Ps. 134.}.
\VS{6}J'entendis ensuite comme la voix d'une grande assemblée, et comme le bruit de grandes eaux, et comme l'éclat de grands tonnerres, disant~: Alléluia~! Car le Seigneur notre Dieu Tout-Puissant a pris possession de son Royaume.
\TextTitle{Festin des noces de l'Agneau}
\VS{7}Réjouissons-nous et tressaillons de joie, et donnons-lui gloire, car les noces de l'Agneau sont venues, et son Epouse s'est préparée.
\VS{8}Et il lui a été donné de se revêtir d'un fin lin pur et éclatant. Car le fin lin désigne la justice des saints.
\VS{9}Alors il me dit~: Ecris~: Bénis sont ceux qui sont appelés au festin des noces de l'Agneau\FTNT{Mt. 22:1-13~; Lu. 14:15-24.}~! Il me dit aussi~: Ces paroles de Dieu sont véritables.
\VS{10}Alors je tombai à ses pieds pour l'adorer, mais il me dit~: Garde-toi de le faire~! Je suis ton compagnon de service, et celui de tes frères qui ont le témoignage de Jésus. Adore Dieu~! Car le témoignage de Jésus est l'Esprit de la prophétie.
\TextTitle{Seconde venue du Messie dans la gloire\FTNTT{Mt. 24:16-30.}}
\VS{11}Puis je vis le ciel ouvert, et voici parut un cheval blanc. Et celui qui était monté dessus s'appelle FIDELE et VERITABLE, et il juge et combat avec justice.
\VS{12}Et ses yeux étaient comme une flamme de feu~; il y avait sur sa tête plusieurs diadèmes, et il avait un nom écrit que personne ne connaît, si ce n'est lui-même.
\VS{13}Il était revêtu d'un vêtement teint de sang, et son Nom s'appelle LA PAROLE DE DIEU.
\VS{14}Les armées qui sont dans le ciel le suivaient sur des chevaux blancs, revêtues de fin lin blanc et pur.
\VS{15}De sa bouche sortait une épée tranchante\FTNT{Es. 11:4~; 2 Th. 2:8~; Hé. 4:12.}, pour frapper les nations~; il les gouvernera avec un sceptre de fer\FTNT{Ps. 2:8-9.}, et il foulera la cuve du vin de l'indignation et de la colère du Dieu Tout-Puissant.
\VS{16}Et sur son vêtement et sur sa cuisse étaient écrits ces mots~: LE ROI DES ROIS ET LE SEIGNEUR DES SEIGNEURS.
\TextTitle{Bataille d'Harmaguédon\FTNTT{Ap. 16:16.}}
\VS{17}Puis je vis un ange qui se tenait dans le soleil. Il cria d'une voix forte, et dit à tous les oiseaux qui volaient au milieu du ciel~: Venez et rassemblez-vous pour le grand festin de Dieu,
\VS{18}afin de manger la chair des rois, la chair des chefs militaires, la chair des puissants, la chair des chevaux et de ceux qui les montent, et la chair de toute sorte de personnes libres, esclaves, petits et grands.
\VS{19}Alors je vis la bête et les rois de la terre, et leurs armées rassemblées pour faire la guerre\FTNT{Guerre d' Harmaguédon~: Voir commentaire Ap. 16:16.} contre celui qui était monté sur le cheval et contre son armée.
\TextTitle{Condamnation de la bête et du faux prophète}
\VS{20}Et la bête fut prise, et avec elle le faux prophète qui avait fait devant elle les prodiges par lesquels il avait séduit ceux qui avaient pris la marque de la bête, et adoré son image. Et ils furent tous deux jetés vivants dans l'étang ardent de feu et de soufre.
\TextTitle{Condamnation des rois et des armées}
\VS{21}Et le reste fut tué par l'épée qui sortait de la bouche de celui qui était monté sur le cheval, et tous les oiseaux furent rassasiés de leur chair.
\Chap{20}
\TextTitle{Satan lié pour mille ans et règne du Messie}
\VerseOne{}Après cela, je vis descendre du ciel un ange, qui avait la clef de l'abîme et une grande chaîne dans sa main.
\VS{2}Il saisit le dragon, le serpent ancien, qui est le diable et Satan, et le lia pour mille ans.
\VS{3}Il le jeta dans l'abîme, et il l'enferma et mit le sceau sur lui, afin qu'il ne séduise plus les nations, jusqu'à ce que les mille ans soient accomplis. Après quoi, il faut qu'il soit délié pour un peu de temps.
\TextTitle{Dernière phase de la première résurrection}
\VS{4}Je vis des trônes, sur lesquels des gens s'assirent, à qui l'autorité de juger fut donnée\FTNT{1 Co. 6:2.}. Et je vis les âmes de ceux qui avaient été décapités pour le témoignage de Jésus, et pour la parole de Dieu, et de ceux qui n'avaient pas adoré la bête ni son image, et qui n'avaient pas pris sa marque sur leurs fronts, ou sur leurs mains. Et ils vécurent\FTNT{Jn. 14:19.} et régnèrent avec Christ mille ans.
\VS{5}Les autres morts ne revinrent pas à la vie jusqu'à ce que les mille ans soient accomplis. C'est la première résurrection.
\VS{6}Bénis et saints sont ceux qui ont part à la première résurrection~! La seconde mort n'a pas de puissance sur eux, mais ils seront prêtres de Dieu, et de Christ, et ils régneront avec lui mille ans.
\TextTitle{Satan délié~; sa chute finale}
\VS{7}Et quand les mille ans seront accomplis, Satan sera délié de sa prison.
\VS{8}Et il sortira pour séduire les nations qui sont aux quatre coins de la terre, Gog et Magog, afin de les rassembler pour la guerre, et leur nombre est comme le sable de la mer.
\VS{9}Ils montèrent et se répandirent à la surface de la terre, et ils environnèrent le camp des saints, et la ville bien-aimée. Mais Dieu fit descendre un feu du ciel qui les dévora.
\TextTitle{Satan jeté dans l'étang de feu}
\VS{10}Et le diable qui les séduisait fut jeté dans l'étang de feu et de soufre, où sont la bête et le faux prophète. Et ils seront tourmentés jour et nuit, aux siècles des siècles.
\TextTitle{Résurrection des impies et jugement dernier~; l'Hadès (ou enfer) et la mort jetés dans l'étang de feu}
\VS{11}Puis je vis un grand trône blanc, et celui qui était assis dessus. La terre et le ciel s'enfuirent devant sa face, et il ne fut plus trouvé de place pour eux.
\VS{12}Et je vis les morts, les grands et les petits, qui se tenaient devant Dieu. Des livres furent ouverts. Et un autre livre fut ouvert, celui qui est le Livre de vie. Et les morts furent jugés selon les choses qui étaient écrites dans les livres, c'est-à-dire selon leurs œuvres.
\VS{13}Et la mer rendit les morts qui étaient en elle, et la mort et l'enfer\FTNT{le mot «~enfer~» vient de l'hébreu «~Hadès~». Voir commentaire dans Mt. 16:18.} rendirent les morts qui étaient en eux~; et ils furent jugés chacun selon ses œuvres.
\VS{14}Et la mort et l'enfer furent jetés dans l'étang de feu\FTNT{L'étang de feu est aussi appelé «~seconde mort~», c'est la destination finale de tous les impies, des démons et de Satan. On l'appelle «~la seconde mort~» parce qu'elle a été précédée de la mort physique. Cette mort n'est pas un anéantissement, mais une condition de souffrances éternelles. C'est la séparation définitive d'avec Dieu. A l'issue du jugement dernier, le séjour des morts (le dieu Hadès ou l'enfer) sera jeté dans le lac de feu (voir commentaire en Mt. 16:18). La Bible utilise également le mot «~géhenne~» pour décrire l'endroit où les impies passeront l'éternité. Ce terme vient de l'hébreu «~ge-hinnom~», autrement dit vallée de Ben Hinnom (littéralement «~le lieu du feu~») qui se trouve en Israël, en contrebas du mont Sion sur lequel est bâtie la ville de Jérusalem (Mt. 5:22~; Mt. 5:29-30~; Mt. 10:28~; Mt. 18:9~; Mt. 23:15~; Mt. 23:33~; Mc. 9:47~; Lu. 12:5~; Ja. 3:6). Autrefois, on y brûlait des enfants en l'honneur de Moloc, une divinité ammonite (2 R. 23:10~; Jé. 32:35), puis des immondices. Ce lieu est devenu avec le temps le symbole du péché et de l'affliction et c'est ainsi qu'il finit par désigner le lieu du châtiment éternel.}. C'est la seconde mort.
\VS{15}Et quiconque ne fut pas trouvé écrit dans le Livre de vie fut jeté dans l'étang de feu.
\Chap{21}
\TextTitle{Nouveaux cieux et une nouvelle terre~; la nouvelle Jérusalem}
\VerseOne{}Puis je vis un nouveau ciel et une nouvelle terre~; car le premier ciel et la première terre avaient disparu, et la mer n'était plus.
\VS{2}Et moi, Jean, je vis la ville sainte, la nouvelle Jérusalem, qui descendait du ciel, d'auprès de Dieu, parée comme une épouse qui s'est ornée pour son mari.
\VS{3}Et j'entendis du trône une voix forte qui disait~: Voici le tabernacle de Dieu avec les hommes~! Il habitera avec eux, et ils seront son peuple, et Dieu lui-même sera leur Dieu, et il sera avec eux.
\VS{4}Et Dieu essuiera toute larme de leurs yeux, et la mort ne sera plus~; et il n'y aura plus ni deuil, ni cri, ni douleur, car les premières choses sont passées.
\VS{5}Et celui qui était assis sur le trône dit~: Voici, je fais toutes choses nouvelles. Puis il me dit~: Ecris, car ces paroles sont véritables et certaines.
\VS{6}Il me dit aussi~: Tout est accompli. Je suis l'Alpha et l'Oméga, le commencement et la fin. A celui qui a soif, je lui donnerai de la source d'eau vive gratuitement\FTNT{Es. 55:1-2~; Mt. 10:8~; Ap. 22:17. Voir commentaire Mt. 10:8.}.
\VS{7}Celui qui vaincra héritera toutes choses~; je serai son Dieu, et il sera mon fils.
\VS{8}Mais pour les timides, les incrédules, les abominables, les meurtriers, les fornicateurs, les sorciers, les idolâtres et tous les menteurs, leur part sera dans l'étang ardent de feu et de soufre, qui est la seconde mort.
\TextTitle{L'Epouse de l'Agneau et la nouvelle Jérusalem}
\VS{9}Puis l'un des sept anges qui tenaient les sept coupes pleines des sept derniers fléaux s'approcha de moi et me parla, en disant~: Viens, et je te montrerai l'Epouse, la femme de l'Agneau.
\VS{10}Et il me transporta en esprit sur une grande et haute montagne, et il me montra la grande ville, la sainte Jérusalem, qui descendait du ciel d'auprès de Dieu,
\VS{11}ayant la gloire de Dieu. Son éclat était semblable à une pierre très précieuse, comme à une pierre de jaspe transparente comme du cristal.
\VS{12}Et elle avait une grande et haute muraille, avec douze portes, et aux portes douze anges, et des noms écrits sur elles, qui sont les noms des douze tribus des fils d'Israël\FTNT{Ez. 48:31-34.}.
\VS{13}A l'orient, trois portes, au nord, trois portes, du côté du sud, trois portes et du côté de l'occident, trois portes.
\VS{14}Et la muraille de la ville avait douze fondements, et les noms des douze apôtres de l'Agneau étaient écrits dessus\FTNT{Lu. 22:29-30~; Ep. 2:20.}.
\VS{15}Et celui qui parlait avec moi avait un roseau d'or pour mesurer la ville, ses portes et sa muraille.
\VS{16}Et la ville était bâtie en carré, et sa longueur était aussi grande que sa largeur. Il mesura donc la ville avec le roseau d'or, jusqu'à douze mille stades~; la longueur, la largeur et la hauteur étaient égales.
\VS{17}Puis il mesura la muraille qui fut de cent quarante-quatre coudées, de la mesure du personnage, c'est-à-dire de l'ange.
\VS{18}Et le bâtiment de la muraille était de jaspe, mais la ville était d'or pur, semblable à du verre fort transparent.
\VS{19}Et les fondements de la muraille de la ville étaient ornés de toutes sortes de pierres précieuses\FTNT{Es. 54:11-12.}~: Le premier fondement était de jaspe, le second de saphir, le troisième de calcédoine, le quatrième d'émeraude,
\VS{20}le cinquième de sardonyx, le sixième de sardoine, le septième de chrysolithe, le huitième de béryl, le neuvième de topaze, le dixième de chrysoprase, le onzième d'hyacinthe, le douzième d'améthyste.
\VS{21}Et les douze portes étaient douze perles~; chacune des portes était d'une seule perle. Et la place de la ville était d'or pur, comme du verre transparent.
\VS{22}Et je ne vis pas de temple dans la ville, parce que le Seigneur Dieu Tout-Puissant et l'Agneau en sont le Temple.
\VS{23}Et la ville n'a pas besoin du soleil ni de la lune pour l'éclairer, car la gloire de Dieu l'éclaire, et l'Agneau est son flambeau\FTNT{Es. 60:19.}.
\VS{24}Et les nations qui auront été sauvées marcheront à la faveur de sa lumière, et les rois de la terre y apporteront ce qu'ils ont de plus magnifique et de plus précieux.
\VS{25}Et ses portes ne se fermeront pas le jour, car il n'y aura pas de nuit\FTNT{Es. 60:11.}.
\VS{26}Et on y apportera la gloire et l'honneur des nations.
\VS{27}Il n'entrera chez elle rien de souillé, ni personne qui s'abandonne à l'abomination et au mensonge~; mais seulement ceux qui sont écrits dans le Livre de vie de l'Agneau.
\Chap{22}
\TextTitle{Règne éternel des saints avec l'Agneau}
\VerseOne{}Puis il me montra un fleuve d'eau de la vie\FTNT{Ce fleuve représente le Saint-Esprit~: Ez. 47:1-12~; Ps. 46:5~; Da. 7:9-10~; Jn. 7:38-39.}, transparent comme du cristal, qui sortait du trône de Dieu et de l'Agneau.
\VS{2}Et au milieu de la place de la ville, et des deux côtés du fleuve, était l'arbre de vie, portant douze fruits, et rendant son fruit chaque mois et les feuilles de l'arbre servaient à la guérison des nations\FTNT{Ge. 2:9~; Ge. 3:22~; Ez. 47:12.}.
\VS{3}Et il n'y aura plus d'anathème. Le trône de Dieu et de l'Agneau sera dans la ville, et ses serviteurs le serviront,
\VS{4}et ils verront sa face, et son Nom sera sur leurs fronts.
\VS{5}Et il n'y aura plus de nuit~; et ils n'auront besoin ni de lumière, ni de lampe, ni du soleil, parce que le Seigneur Dieu les éclairera, et ils régneront aux siècles des siècles.
\TextTitle{Certitude des prophéties de ce livre}
\VS{6}Puis il me dit~: Ces paroles sont certaines et véritables~; et le Seigneur, le Dieu des saints prophètes, a envoyé son ange pour manifester à ses serviteurs les choses qui doivent arriver bientôt.
\VS{7}Voici, je viens à toute vitesse\FTNT{Dans la plupart des traductions, ce passage a été traduit par «~Je viens bientôt~». Or le texte grec utilise le mot «~tachu~» qui signifie «~rapidement, à toute vitesse (sans tarder)~». Beaucoup doutent de cette promesse du Seigneur en faisant la même réflexion évoquée par Pierre~: «~Où est la promesse de son avènement~? Car depuis que les pères sont morts, toutes choses demeurent comme elles ont été dès le commencement de la création.~» (2 Pi. 3:4). Or le Seigneur ne tarde pas dans l'accomplissement de sa promesse, car il a fixé de sa propre autorité une date pour son retour, que lui seul connaît (Za. 14:7~; Mt. 24:36~; Mc. 13:32~; Ac. 1:6-7). Il sera donc fidèle à son calendrier et ne tardera pas (2 Pi. 3:9.~; Hé. 10:37).}. Béni est celui qui garde les paroles de la prophétie de ce livre~!
\VS{8}C'est moi, Jean, qui ai entendu et vu ces choses. Et après les avoir entendues et vues, je tombai à terre aux pieds de l'ange qui me les montrait pour l'adorer.
\VS{9}Mais il me dit~: Garde-toi de le faire~! Car je suis ton compagnon de service\FTNT{Hé. 1:14.} et celui de tes frères les prophètes, et de ceux qui gardent les paroles de ce livre. Adore Dieu~!
\VS{10}Il me dit aussi~: Ne scelle pas les paroles de la prophétie de ce livre. Car le temps est proche.
\VS{11}Que celui qui est injuste soit encore injuste, et que celui qui est souillé se souille encore~; et que celui qui est juste pratique encore la justice~; et que celui qui est saint se sanctifie encore~!
\VS{12}Voici, je viens à toute vitesse, et ma rétribution est avec moi\FTNT{Jésus affirme de nouveau ici sa divinité et confirme les prophéties d'Es. 35:4~; Es. 40:10~; Es. 62:11, où il est dit que Yahweh lui-même viendra avec ses rétributions.} pour rendre à chacun selon son œuvre.
\VS{13}Je suis l'Alpha et l'Oméga, le premier et le dernier, le commencement et la fin.
\VS{14}Bénis sont ceux qui lavent leurs robes afin d'avoir droit à l'arbre de vie, et d'entrer par les portes dans la ville.
\VS{15}Mais seront laissés dehors les chiens, les empoisonneurs, les fornicateurs, les meurtriers, les idolâtres et quiconque aime et pratique le mensonge.
\VS{16}Moi, Jésus, j'ai envoyé mon ange\FTNT{Cette déclaration de Jésus fait écho au verset 6 où il est dit que le Seigneur, le Dieu des esprits des prophètes, a envoyé son ange. Jésus confirme donc qu'il est Seigneur et Dieu.} pour vous confirmer ces choses dans les églises. Je suis le rejeton et la postérité de David, l'étoile brillante du matin.
\VS{17}Et l'Esprit et l'Epouse disent~: Viens~! Et que celui qui entend dise~: Viens~! Et que celui qui a soif vienne~; que celui qui veut, prenne gratuitement de l'eau de la vie.
\TextTitle{Nul ne doit y ajouter ou y retrancher}
\VS{18}Je le déclare à quiconque entend les paroles de la prophétie de ce livre~: Si quelqu'un y ajoute quelque chose, Dieu le frappera des fléaux décrits dans ce livre,
\VS{19}et si quelqu'un retranche quelque chose des paroles du livre de cette prophétie, Dieu retranchera la part qu'il a dans le livre de vie, dans la ville sainte et dans les choses qui sont écrites dans ce livre.
\VS{20}Celui qui rend témoignage de ces choses, dit~: Certainement, je viens à toute vitesse. Amen~! Oui, Seigneur Jésus, viens~!
\VS{21}Que la grâce de notre Seigneur Jésus-Christ soit avec vous tous~! Amen~!
\PPE{}
\end{multicols}

% inclusion des annexes
%\addcontentsline{toc}{chapter}{Aide}\clearpage
%\addcontentsline{toc}{section}{Dictionnaire}\clearpage
%% annexe dictionnaire
%\makeatletter
%    % mise en forme dictionnaire
%    \def\@oddhead{{\small{Dictionnaire\hfil\thepage\hfil\rightmark---\leftmark}}}
%    \def\@evenhead{{\small{\rightmark---\leftmark\hfil\thepage\hfil Dictionnaire}}}\clearpage
%    \makeatother
%        % inclusion dictionnaire
%        \small{\parindent=0mm{\begin{center}\Large\bfseries{Dictionnaire}\end{center}}\par\begin{multicols}{2}

\DicoEntry{AARON}\textit{, de l'hébreu «~Aharown~»~: «~haut placé~» ou «~éclairé~»}\newline
Issu de la tribu de Lévi, frère aîné de Moïse dont il fut le porte-parole. Premier grand prêtre* en Israël. Voir \vref{Ex. 4:14}~; \vref{Ex. 6:16-20} et \vref{Ex. 28}.

\DicoEntry{ABSALOM}\textit{, de l'hébreu «~'Abiyshalowm~»~: «~père de la paix~»}\newline
Fils du roi David et de Maaca, né à Hébron. Il tua Amnon son demi-frère aîné, car ce dernier avait déshonoré sa sœur Tamar. Quelques années plus tard, il conspira contre son père et se fit proclamer roi à Hébron. Il fut finalement tué par Joab, chef de l'armée de David. Voir \vref{2 S. 3:3}~; \vref{2 S. 13}~; \vref{2 S. 15-19}.

\DicoEntry{ABDIAS}\textit{, de l'hébreu «~Obadyah~»~: «~adorateur~» ou «~serviteur de Yahweh~»}\newline
Prophète de Yahweh dont le livre éponyme figure dans le Tanakh.

\DicoEntry{ABEL}\textit{, de l'hébreu «~Hebel~»~: «~souffle, vapeur~»}\newline
Deuxième fils d'Adam et Eve et première victime d'homicide de l'histoire, il fut assassiné par son frère Caïn et déclaré juste par Yahweh. Voir \vref{Ge. 4:2,8} et \vref{Mt. 23:35}.

\DicoEntry{ABIRAM}\textit{, de l'hébreu «~'Abyiram~»~: «~mon père est exalté~»}\newline
Issu de la tribu de Ruben, fils d'Eliab et frère de Dathan, il conspira avec Koré contre Moïse et Aaron. Voir \vref{No. 16:1-35}.

\DicoEntry{ABLUTION}\textit{, de l'hébreu «~rachats~»~: «~laver, baigner, nettoyer~»}\newline
Lavage de purification prescrit par la loi mosaïque et effectué avec de l'eau. Voir \vref{Ex. 29:4} et \vref{Hé. 9:10}.

\DicoEntry{ABOMINATION}\textit{, de l'hébreu «~tow'ebah~»~: «~une chose dégoûtante, abominable~» et du grec «~bdelugma~»~: «~chose folle, détestable~»}\newline
Pratique violant la loi de Yahweh et manifestant l'infidélité à Dieu comme l'idolâtrie* sous toutes ses formes, la magie ou l'homosexualité*. Voir \vref{Lé. 18:6-29}~; \vref{De. 29:17-18} et \vref{Ap. 21:27}.

\DicoEntry{ABRAM}\textit{, de l'hébreu «~Abryram~»~: «~père élevé~»}\newline
Voir ABRAHAM.

\DicoEntry{ABRAHAM}\textit{, de l'hébreu «~'Abraham~»~: «~père d'une multitude~»}\newline
Hébreu, fils de Térach, et originaire d'Ur en Chaldée. Dieu lui demanda de quitter sa terre et sa famille pour Canaan, lui promettant que sa postérité hériterait de cette terre. De sa servante Agar, lui naquit un premier fils, Ismaël, ancêtre du peuple arabe. De sa femme Sara, lui naquit Isaac qui hérita des promesses. Il mourut à cent soixante-quinze ans. Voir \vref{Ge. 12:1-7}~; \vref{Ge. 17:4-13}~; \vref{Ge. 16}~; \vref{Ge. 21:1-8} et \vref{Ge. 25:7}.

\DicoEntry{ACACIA}\textit{, de l'hébreu «~shittah~»~: «~acacia, bois d'acacia~»}\newline
Arbre épineux poussant en abondance dans la péninsule du Sinaï et dans la vallée du Jourdain, il est aussi appelé bois de Sittim. Il fut l'un des matériaux utilisés pour la fabrication des objets du culte lévitique, dont l'arche*. Voir \vref{Ex. 25:10,13,23,28}.

\DicoEntry{ACHAB}\textit{, de l'hébreu «~Ach'ab~»~: «~un frère du père~»}\newline
Fils d'Omri, il fut roi d'Israël pendant vingt-deux ans. Marié à Jézabel*, fille du roi des Sidoniens, Achab et sa femme commirent de grandes abominations* et s'opposèrent au prophète Elie*. Voir \vref{1 R. 16:29-31}~; \vref{1 R. 18:1-40} et \vref{1 R. 22:29-40}.

\DicoEntry{ADAM}\textit{, de l'hébreu «~'Adam~»~: «~être humain~» ou «~de la terre~»}\newline
Premier homme, il vécut la première partie de sa vie dans le jardin d'Eden* avec sa femme Eve. Après avoir désobéi à Dieu en goûtant le fruit de l'arbre de la connaissance du bien et du mal, ils furent chassés du jardin. Adam fut le père de Caïn, Abel et Seth. Il mourut à neuf cent trente ans. Voir \vref{Ge. 2:7-8}~; \vref{Ge. 3}~; \vref{Ge. 4:1-2}, \vref{25-26} et \vref{Ge. 5:5}.

\DicoEntry{ADONIJA}\textit{, de l'hébreu «~'Adoniyah~»~: «~Yahweh est Seigneur~»}\newline
Quatrième fils de David et de Haggith. Peu avant la mort de son père, il s'autoproclama roi tentant en vain de prendre la place qui devait revenir à Salomon. Ce dernier lui laissa la vie sauve, mais le fit tuer plus tard alors qu'il semblait encore convoiter le trône d'Israël. Voir \vref{2 S. 3:4} et \vref{1 R. 1,2}.

\DicoEntry{ADOPTION}\textit{, du grec «~huiothesia~»~: «~adoption, adoption comme fils~»}\newline
Manifestation de l'amour éternel de Dieu, l'adoption permet à tout homme de devenir par la foi enfant de Dieu. Ce privilège, autrefois réservé au peuple d'Israël, fut étendu à toutes les nations par le sacrifice de Jésus. Cette adoption est manifestée par l'Esprit de Dieu qui témoigne à l'esprit du chrétien son appartenance à Dieu~; elle inclut les avantages du fils, dont l'héritage. Voir \vref{Jn. 1:12}~; \vref{Ga. 4:7}~; \vref{Ro. 8:15-17}~; \vref{Ro. 9:4}~; \vref{Ep. 1:5,11} et \vref{1 Jn. 3:1}.

\DicoEntry{AGAR}\textit{, de l'hébreu «~Hagar~»~: «~fuite~»}\newline
Servante égyptienne de Sara que cette dernière donna à Abraham comme concubine. Elle enfanta Ismaël, fils premier-né d'Abraham. Après la naissance d'Isaac, Abraham la chassa avec son fils. Voir \vref{Ge. 16} et \vref{Ge. 21:1-18}.

\DicoEntry{AGABUS}\textit{, de l'hébreu «~Chagab~» et du grec «~Agabos~»~: «~sauterelle~»}\newline
Prophète de Yahweh suscité au temps de l'Eglise primitive. Il prophétisa une famine qui se réalisa sous le règne de l'empereur Claude. Il annonça aussi l'arrestation de Paul à Jérusalem. Voir \vref{Ac. 11:27-28} et \vref{Ac. 21:10-33}.

\DicoEntry{AGGÉE}\textit{, de l'hébreu «~Chaggay~»~: «~en fête~» ou «~né un jour de fête~»}\newline
Prophète de Yahweh d'après la captivité, dont le livre éponyme figure dans le Tanakh.

\DicoEntry{AGNEAU}\textit{, de l'hébreu «~kebes~»~: «~agneau, brebis, jeune bélier~»}\newline
Animal sacrifié et consommé lors de la Pâque des juifs. Il préfigurait Christ, l'Agneau de Dieu qui ôte le péché du monde. Voir \vref{Ex. 12:1-28} et \vref{Jn. 1:29}.

\DicoEntry{AÏ}\textit{, de l'hébreu «~'Ay~»~: «~tas de ruines~»}\newline
Ville située au sud-est de Béthel, à proximité de laquelle Abraham dressa sa tente à deux reprises. Il s'agit également de la deuxième ville que Dieu livra entre les mains de Josué après la prise de Jéricho. Voir \vref{Ge. 12:8}~; \vref{Ge. 13:3} et \vref{Jos. 8}.

\DicoEntry{ALLÉLUIA}\textit{, de l'hébreu «~allelouia~»~: «~Louez Yahweh~»}\newline
Retrouvé à maintes reprises dans les Psaumes sous la forme «~Louez Yahweh~», cette exclamation encourage à célébrer Dieu et à se réjouir en lui. Voir \vref{Ap. 19:1-6}.

\DicoEntry{ALLIANCE}\textit{, de l'hébreu «~beriyth~»~: «~pacte, alliance, engagement~»}\newline
Dieu a conclu plusieurs alliances avec les hommes (ex~: Noé, Abraham, David). On distingue communément deux alliances majeures dans les Ecritures~: l'Ancienne Alliance - conclue avec Israël au travers de Moïse - et la Nouvelle Alliance inaugurée par Jésus-Christ. Voir \vref{Ge. 9:8-17}~; \vref{Ge. 17}~; \vref{Ex. 19-34}~; \vref{2 S. 7:12-16} et \vref{Hé. 9-13}.

\DicoEntry{ALPHA ET OMEGA}\textit{}\newline
Première et dernière lettre de l'alphabet grec, la combinaison de ces deux lettres mentionnées ensemble se rapporte à l'idée que Dieu est le premier et le dernier. Jésus fut présenté plusieurs fois comme étant «~l'alpha et l'oméga~» soulignant ainsi son caractère éternel. Voir \vref{Ap. 1:8}~; \vref{Ap. 21:6} et \vref{Ap. 22:13}.

\DicoEntry{ÂME}\textit{, de l'hébreu «~nephesh~»~: «~âme, une personne, la vie, être vivant~», «~ce qui respire~», «~ce qui a une vie par le sang~» et du grec «~psuche~»~: «~le souffle, la vie, l'âme~»}\newline
L'âme correspond au sang~; elle est le siège des émotions, de la volonté humaine et de l'intelligence. Avec l'esprit et le corps, l'âme constitue l'être humain. Voir \vref{Ge. 19:20}~; \vref{Ge. 44:30}~; \vref{Lé. 17:11}~; \vref{Mt. 10:28}~; \vref{Ac. 20:10} et \vref{1 Th. 5:23}.

\DicoEntry{AMEN}\textit{, de l'hébreu «~'amen~»~: «~assuré, établi~» ou «~ainsi soit-il~!~»}\newline
Se rapportant exclusivement à ce qui est sûr, avéré et certain, ce terme est souvent utilisé comme interjection. Christ est appelé «~l'Amen~», faisant référence à la vérité qu'il incarne. Voir \vref{Jé. 28:6}~; \vref{1 Ch. 16:36}~; \vref{2 Co. 1:20} et \vref{Ap. 3:14}.

\DicoEntry{AMOUR}\textit{}\newline
Il existe plusieurs traductions et définitions du mot «~amour~» en hébreu et en grec, elles varient selon le contexte.
\\- Les termes hébreux désignant l'amour~:
\\1. «~'Ahab~»~: «~amours~»
\\Amours, amis. Voir \vref{Os. 8:9} et \vref{Pr. 5:19}.
\\2. «~'Ahabah~»~: «~amour humain, amour de Dieu pour son peuple~»
\\Amour, affection, aimer. Voir \vref{De. 7:8}~; \vref{1 S. 20:17} et \vref{Pr. 10:12}.
\\3. «~Checed~»~: «~bonté, miséricorde, fidélité~»
\\Grâce, miséricorde, compassion, affection. Voir \vref{Ge. 40:14}~; \vref{Ex. 34:7} et Nb. \vref{14:19}.
\\4. «~Yediyd~»~: «~bien-aimé~»
\\Bien-aimé, amour. Voir \vref{De. 33:12} et \vref{Es. 5:1}.
\\- Les termes grecs désignant l'amour~:
\\1. «~Agape~»~: «~amour, charité, affection, bienveillance~»
\\Amour de Dieu, amour désintéressé que doit manifester l'homme né d'en haut. Voir \vref{Jn. 15:13}~; \vref{Jn. 17:26}~; \vref{1 Co. 8:1}~; \vref{1 Co. 13:3}~; \vref{Ro. 5:5} et \vref{1 Jn. 4:8}.
\\2. «~Eros~»~: «~l'amour qui prend~»
\\Amour dans la dimension sexuelle.
\\3. «~Phileo~»~: «~aimer, montrer des signes d'amour~»
\\Amour filial. Voir \vref{Jn. 21:17}~; \vref{1 Co. 16:22}.
\\4. «~Philadelphia~»~: «~amour fraternel~»
\\Amour des frères et sœurs d'une même famille, amour des chrétiens les uns pour les autres. Voir \vref{1 Th. 4:9}~; \vref{Ro. 12:10} et \vref{Hé. 13:1}.
\\5. «~Storge~»~: «~amour filial~»
\\Amour familial, affection naturelle. Voir \vref{Ro. 1:31}.

\DicoEntry{AMOS}\textit{, de l'hébreu «~'Amowc~»~: «~fardeau, porteur de fardeaux~»}\newline
Originaire de Tekoa en Juda, prophète de Yahweh dont le livre éponyme figure dans le Tanakh.

\DicoEntry{AMMONITES}\textit{, de l'hébreu «~'Ammown~»~: «~appartenant à la nation~»}\newline
Peuple issu de Ben-Ammi, né de l'inceste entre Lot et sa fille cadette. Ils furent ennemis d'Israël. Voir \vref{Ge. 19:30-38} et \vref{Ez. 25:1-7}.

\DicoEntry{ANAKIM}\textit{, de l'hébreu «~'Anaqiy~»~: «~au long cou~»}\newline
Descendants d'Anak, race de géants habitant Canaan avant sa conquête par le peuple d'Israël. Ils furent vaincus par Josué et Caleb qui hérita d'une partie de leur territoire. Voir \vref{No. 13:28-33}~; \vref{De. 9:1-3}~; \vref{Jos. 11:21-22} et \vref{Jos. 14:6-15}.

\DicoEntry{ANANIAS}\textit{, de l'hébreu «~Chananyah~»~: «~Dieu a été miséricordieux~»}\newline
1. Chrétien ayant vendu un champ avec sa femme Saphira et ayant fait croire qu'ils avaient donné la totalité du prix rapporté pour l'Eglise alors qu'ils en avaient secrètement gardé une partie. Ce mensonge les conduisit tous deux à la mort. Voir \vref{Ac. 5:1-10}.
\\2. Homme pieux vivant à Damas que le Seigneur envoya imposer les mains à Saul qui venait de se convertir afin qu'il recouvre la vue. C'est également lui qui le baptisa. Voir \vref{Ac. 9:10-18} et \vref{Ac. 22:12-16}.

\DicoEntry{ANATHÈME}\textit{, du grec «~anathema~»~: «~tout ce qui est livré au malheur~»}\newline
Terme désignant une personne ou une chose maudite, vouée à la destruction. Voir \vref{Ga. 1:8} et \vref{1 Co. 12:3}.

\DicoEntry{ANCIENS}\textit{, de l'hébreu «~zaqen~»~: «~vieux, aîné, de ceux qui ont de l'autorité~» et du grec «~presbuteros~»~: «~ayant de l'âge~»}\newline
Chez les juifs, il s'agissait des chefs de famille ou de clan qui représentaient le peuple dans les affaires religieuses et civiles. Voir \vref{Ex. 3:16}~; \vref{Lé. 4:15} et \vref{De. 31:28}. Sous la Nouvelle Alliance, les églises de la Galatie avaient élu des anciens («~presbuteros~») pour prendre soin des frères et sœurs. Il s'agit d'un terme relatif aux personnes ayant de l'âge et non à la fonction d'évêque*. Voir \vref{Ac. 14:23}~; \vref{1 Ti. 5:17}~; \vref{Tit. 1:5-9} et \vref{1 Pi. 5:1-5}.

\DicoEntry{ANDRÉ}\textit{, du grec «~Andreas~»~: «~virilité~»}\newline
Frère de Simon Pierre, originaire de Bethsaïda en Galilée, et pêcheur de métier. Il devint l'un des douze apôtres de Jésus-Christ. Voir \vref{Mt. 10:2}~; \vref{Mc. 1:16-17} et \vref{Jn. 1:40}.

\DicoEntry{ANGE}\textit{, de l'hébreu «~mal'ak~» et du grec «~aggelos~»~: «~messager, envoyé~»}\newline
Etre spirituel au service de Dieu pouvant prendre une forme humaine. Les anges sont au service de Yahweh pour des missions spécifiques au ciel ou sur la terre. Ils peuvent avoir une fonction de messager, protecteur ou combattant. Voir \vref{Da. 10:10-13}~; \vref{Lu. 1:26-38} et \vref{Ap. 12:7}.

\DicoEntry{ANNE}\textit{, de l'hébreu «~Channah~»~: «~grâce, faveur~»}\newline
1. Une des deux femmes d'Elkana. Stérile, elle pria Yahweh de lui accorder un fils qu'elle lui consacrerait. Elle enfanta ainsi Samuel qui entra au service de Yahweh dès son plus jeune âge. Voir \vref{1 S. 1,2}.
\\2. Fille de Phanuel de la tribu d'Aser, prophétesse. Veuve, elle servait le Seigneur nuit et jour dans le temple. Elle rencontra Jésus nourrisson, lorsqu'il fut amené au temple pour y être présenté à Dieu. Voir \vref{Lu. 2:36-38}.
\\3. Grand prêtre, beau-père de Caïphe*. Il participa à la conspiration qui mena Jésus à la croix. Voir \vref{Lu. 3:2} et \vref{Jn. 18:13}.

\DicoEntry{ANTICHRIST}\textit{, du grec «~antichristos~»~: «~l'adversaire du Messie~»}\newline
Aussi appelé «~homme impie~» et «~fils de la perdition~», personnage dont l'apparition se fera avant le retour glorieux du Seigneur. Il dominera le monde avant d'être vaincu par Christ. Voir \vref{2 Th. 2:1-4}~; \vref{2 Jn. 1:7} et \vref{Ap. 19:19-21}.

\DicoEntry{ANTIOCHE}\textit{, du grec «~Antiocheia~»~: «~rapide comme un char~»}\newline
Capitale de la Syrie, elle fut fondée en 300 av. J.-C. par Séleucus Nicator (358-281 av. J.-C.) qui la baptisa du nom de son père Antiochus. Cette ville accueillit des chrétiens en exil~; l'évangile y fut ainsi annoncé aux juifs puis aux Grecs et un grand nombre de personnes se convertirent. Barnabas et Paul y demeurèrent une année durant laquelle ils enseignèrent la parole. C'est à Antioche que les disciples furent appelés chrétiens pour la première fois. Voir \vref{Ac. 11:19-26}.

\DicoEntry{APIS}\textit{}\newline
Divinité égyptienne symbolisant la force et la fertilité. Il est représenté sous la forme d'un veau d'or ou d'un homme à tête de taureau dont les cornes entourent un disque solaire. Les Hébreux se corrompirent plusieurs fois en le vénérant. Voir \vref{Ex. 32:1-6} et \vref{1 R. 12:28-30}.

\DicoEntry{APOCALYPSE}\textit{, du grec «~apokalupsis~»~: «~mettre à nu, révélation d'une vérité, action de révéler~»}\newline
Dernier livre de la Bible écrit par Jean, ce récit comporte une révélation de la gloire de Jésus-Christ et raconte les derniers événements de l'histoire de l'humanité jusqu'à l'avènement de la Nouvelle Jérusalem.

\DicoEntry{APOLLOS}\textit{, du grec «~Apollos~»~: «~donné par Apollon~»}\newline
Juif érudit d'Alexandrie ayant une très bonne connaissance des Ecritures et enseignant avec exactitude au sujet de Jésus. Sa rencontre avec Aquilas et Priscille lui permit d'aller plus en profondeur dans la Parole et d'annoncer avec plus de force, notamment aux Juifs, que Jésus est le Messie en se basant sur les écrits du Tanakh. Il réalisa plusieurs voyages missionnaires notamment à Corinthe. Voir \vref{Ac. 18:24-28}, \vref{1 Co. 3:5-6} et \vref{1 Co. 16:12}.

\DicoEntry{APOSTASIE}\textit{, du grec «~apostasia~»~: «~action de s'éloigner de, désertion, défection~»}\newline
Abandon de la foi en Jésus-Christ et de la saine doctrine* se manifestant sous deux formes principales. Certaines personnes abandonnent ouvertement la foi, la communion avec Dieu et l'assemblée des saints. D'autres continuent de fréquenter les assemblées chrétiennes, mais ont laissé la saine doctrine pour s'attacher à des doctrines séductrices. Voir \vref{Mt. 24:11-12}~; \vref{2 Th. 2:3}~; \vref{1 Ti. 4:1-3}~; \vref{2 Pi. 2:1-3}~; \vref{2 Ti. 3:1-8}~; \vref{Jud. 1:17-19} et \vref{1 Jn. 4:1}.

\DicoEntry{APÔTRE}\textit{, du grec «~apostolos~»~: «~envoyé en avant, messager, ambassadeur~»}\newline
Lors de son service terrestre, Jésus choisit douze apôtres qu'il forma pour continuer l'œuvre après lui. Plusieurs autres apôtres furent suscités au temps de l'Eglise primitive, notamment Paul et Jacques, frère du Seigneur, qui avec Jean et Pierre furent les principaux instruments utilisés pour poser les fondements de la doctrine de l'Eglise. Le service apostolique existe encore aujourd'hui, mais la mission des apôtres actuels n'est pas d'écrire des épîtres, car la fondation a déjà été posée. Leur travail aujourd'hui consiste davantage à enseigner et veiller à ce que le fondement demeure. Voir \vref{Mc. 3:14}~; \vref{Ac. 15}~; \vref{Ga. 2:19}~; \vref{Ro. 1:1}~; \vref{Ep. 2:20} et \vref{Ep. 4:11}.

\DicoEntry{AQUILAS}\textit{, du latin «~Aquilas~»~: «~un aigle~» et PRISCILLE, du latin «~Priscilla~»~: «~petite vieille~»}\newline
Couple de Juifs ayant accepté l'évangile. Après avoir été chassés de Rome, ils s'installèrent à Corinthe où ils hébergèrent Paul à son arrivée et devinrent par la suite compagnons d'œuvre de ce dernier. Ils participèrent à plusieurs voyages missionnaires, notamment à Ephèse où ils enseignèrent Apollos*. Voir \vref{Ac. 18:1-3,18,24-26} et \vref{Ro. 16:3-5}.

\DicoEntry{ARBRE}\textit{, de l'hébreu «~'ets~»~: «~arbre, bois~»}\newline
Organisme vivant porteur de semence produisant des feuilles et des fruits selon les espèces. Lors de la création, Dieu créa différents arbres dont les fruits furent donnés pour nourrir l'homme et également deux arbres spécifiques placés au milieu du jardin d'Eden.

\DicoEntry{ARBRE DE LA CONNAISSANCE DU BIEN ET DU MAL}\textit{}\newline
Arbre dont le fruit contenait la connaissance du bien et du mal. Dieu interdit la consommation de ce dernier à l'homme sous peine de mort, mais Adam et Eve transgressèrent le commandement. C'est ainsi que le péché et la mort régnèrent sur l'humanité. Voir \vref{Ge. 2:17}~; \vref{Ge. 3:1-6} et \vref{Ro. 5:12}.

\DicoEntry{ARBRE DE VIE}\textit{}\newline
Arbre dont la consommation donne la vie éternelle. Après la chute d'Adam et Eve, Dieu les chassa du jardin pour les empêcher d'y accéder. L'arbre de vie se trouve dans la ville sainte, la Nouvelle Jérusalem~; ses feuilles servent à la guérison des nations. Voir \vref{Ge. 3:22-24} et \vref{Ap. 22:2,14,19}.

\DicoEntry{ARC-EN-CIEL}\textit{, de l'hébreu «~qesheth~»~: «~arc~»}\newline
Signe de l'alliance* que Dieu conclut avec Noé et les générations qui le suivraient suite au déluge*. Cette alliance stipulait que Yahweh ne détruirait plus les hommes par les eaux. Voir \vref{Ge. 9:12-17}.

\DicoEntry{ARCHANGES}\textit{, du grec «~archaggelos~»~: «~chef des anges~»}\newline
Catégorie d'anges* ayant un rang et une dignité plus élevés que les autres. Voir \vref{1 Th. 4:16} et \vref{Jud. 1:9}.

\DicoEntry{ARCHE DE NOÉ}\textit{, de l'hébreu «~tebah~»~: «~arche, vaisseau, coffre~»}\newline
Embarcation construite par Noé pour le sauver lui, sa famille ainsi que les animaux, du déluge* qui allait s'abattre sur la terre. Voir \vref{Ge. 6:8-16}~; \vref{Mt. 24:37-39} et \vref{Lu. 17:26-27}.

\DicoEntry{ARCHE DU TEMOIGNAGE ou DE L'ALLIANCE}\textit{, de l'hébreu «~'arown~»~: «~arche, coffre, cercueil~»}\newline
Coffre rectangulaire en bois d'acacia* recouvert d'or pur, contenant les tables de l'alliance, la verge d'Aaron et une urne contenant un échantillon de la manne. Construite selon le modèle que Moïse avait reçu au mont Sinaï, elle était couverte par le propitiatoire*. L'arche fut placée dans le Saint des saints du tabernacle*, puis du temple*. Voir \vref{Ex. 25:10-22}~; \vref{1 R. 8:6}~; \vref{2 R. 25:8-9} et \vref{Hé. 9:4}.

\DicoEntry{ARTAXERXÈS}\textit{, (règne~: 465 av. J.-C.- 424 av. J.-C.), du persan «~Artachshashta~»~: «~celui qui fait régner la loi sacrée~»}\newline
Fils d'Assuérus*, roi de Perse. Il autorisa Esdras à retourner à Jérusalem avec des prêtres et des Lévites pour effectuer le sacerdoce dans le temple et faire respecter la loi de Yahweh. Voir \vref{Esd. 7:11-28}.

\DicoEntry{ASAPH}\textit{, de l'hébreu «~'Acaph~»~: «~celui qui rassemble, collecteur~»}\newline
Lévite et chef des chantres sous David, il participa au transfert de l'arche* à Jérusalem et écrivit certains psaumes. Voir \vref{1 Ch. 15:16-19} et \vref{1 Ch. 16:4-7}.

\DicoEntry{ASER}\textit{, de l'hébreu «~'Asher~»~: «~heureux~»}\newline
Fils de Jacob et de Zilpa, servante de Léa, il est le père de la tribu d'Aser. Voir \vref{Ge. 30:13}.

\DicoEntry{ASHERAH}\textit{, de l'hébreu «~'Asherah~»~: «~pieu sacré~».}\newline
Voir commentaire en \vref{Ex. 34:13}.

\DicoEntry{ASSUÉRUS ou XERXÈS Ier}\textit{, (485 av. J.-C. – 465 av. J.-C.), du persan «~'Achashverowsh~»~: «~je serai silencieux et pauvre~».}\newline
Père d'Artaxerxès, roi de Perse et époux d'Esther*. Voir le livre d'Esther.

\DicoEntry{ASTARTE}\\textit{, de l'hébreu «~Ashtoreth~»~: «~étoile~».}\newline
Voir commentaire en \vref{Jg. 2:13}.

\DicoEntry{AUTEL}\textit{, de l'hébreu «~mizbeach~»~: «~autel~»}\newline
Table généralement façonnée avec des monticules de pierres ou en terre et élevée spécialement pour offrir des holocaustes* et des sacrifices en l'honneur de Dieu. Voir \vref{Ge. 12:7}~; \vref{Ge. 35:7}~; \vref{Ex. 20:24-26} et \vref{Ex. 30:1-8}.

\DicoEntry{BAAL}\textit{, de l'hébreu «~Ba'al~»~: «~maître, possesseur, seigneur~»}\newline
Dieu primaire des Phéniciens et des Cananéens auquel les Israélites s'attachèrent à plusieurs reprises pour l'adorer. Voir \vref{No. 25:3}~; \vref{Jg. 2:11} et \vref{1 R. 18:21}. Voir aussi commentaire en \vref{Jg. 2:11}.

\DicoEntry{BABEL ou BABYLONE}\textit{, de l'hébreu «~Babel~»~: «~confusion (par le mélange)~»}\newline
Ville de Mésopotamie* située sur l'Euphrate, capitale de la Babylonie. Les hommes y entreprirent la construction de la tour de Babel. Cependant, Yahweh confondit leur langage et les dispersa sur toute la terre. Voir \vref{Ge. 10:8-10} et \vref{Ge. 11:1-9}.

\DicoEntry{BALAAM}\textit{, de l'hébreu «~Bil'am~»~: «~sans peuple~», «~dévorant~»}\newline
Prophète de Yahweh ayant vécu pendant la marche d'Israël dans le désert, il fut séduit par Balak, roi de Moab, qui lui proposa de maudire Israël contre de généreux présents. Son témoignage a été utilisé plusieurs fois pour avertir les enfants de Dieu des scandales* dont ils pourraient être la cause en suivant la voie de la cupidité. Voir \vref{No. 22-24}~; \vref{No. 31:8}~; \vref{Jud. 1:11} et \vref{Ap. 2:14}.

\DicoEntry{BALAK}\textit{, de l'hébreu «~Balaq~»~: «~gaspilleur, dévastateur~»}\newline
Roi de Moab, il essaya de convaincre Balaam* de maudire Israël qu'il redoutait. Voir \vref{No. 22-24}.

\DicoEntry{BANNIÈRE}\textit{, de l'hébreu «~nec~»~: «~quelque chose de levé, étendard, signal, enseigne~»}\newline
Drapeau, étendard élevé en signe d'appartenance à ce qu'il représente. Moïse bâtit un autel du nom de Yahweh-Nissi~: «~Yahweh ma bannière~». Voir \vref{Ex. 17:15}~; \vref{Es. 11:10,12} et \vref{Ps. 60:4}.

\DicoEntry{BAPTÊME}\textit{, du grec «~baptizo~»~: «~plonger, immerger, purifier en plongeant~»}\newline
On distingue trois types de baptêmes dans les Ecritures (\vref{Mt. 3:11})~:
\\1. le baptême d'eau~: acte suivant la conversion par lequel une personne est immergée dans l'eau - symbolisant la mort et la résurrection en Jésus-Christ. Il s'agit selon Pierre de l'«~engagement d'une bonne conscience envers Dieu~». Voir \vref{Ac. 2:38}~; \vref{Ac. 16:30-33}~; \vref{Col. 2:12-13} et \vref{1 Pi. 3:21}.
\\2. le baptême du Saint-Esprit~: lors de la naissance d'en haut, gage que le Seigneur donne au nouveau converti par l'envoi du Saint-Esprit. Voir \vref{Jn. 3:5-6}~; \vref{Tit. 3:4-7} et \vref{Ep. 1:13}.
\\3. le baptême de feu~: symbole des souffrances que Christ a endurées à la croix et par lesquelles tous les chrétiens sont appelés à passer pour être purifiés. Voir \vref{Mc. 10:35-39}~; \vref{Lu. 12:50}~; \vref{1 Pi. 1:6-9} et \vref{1 Pi. 4:12-13}.

\DicoEntry{BARAK}\textit{, de l'hébreu «~Baraq~»~: «~éclairs, foudre~»}\newline
Fils d'Abinoam, issu de la tribu de Nephtali, il vécut en Israël au temps des juges. Encouragé et accompagné par Débora, il battit l'armée de Jabin, roi de Canaan. Voir \vref{Jg. 4}.

\DicoEntry{BARTHÉLÉMY}\textit{, du grec «~Bartholomaios~»~: «~fils de Tolmaï~»}\newline
Un des douze apôtres de Jésus. Voir \vref{Mt. 10:3}.

\DicoEntry{BARTIMÉE}\textit{, du grec «~Bartimaios~»~: «~fils de Timée~»}\newline
Fils de Timée, mendiant aveugle que Jésus guérit suite à ses cris de supplications sur la route de Jéricho. Voir \vref{Mc. 10:46-52}.

\DicoEntry{BATH-SCHEBA}\textit{, de l'hébreu «~Bath-Sheba`~»~: «~fille d'un serment~»}\newline
Fille d'Eliam, femme d'Urie* le Héthien que David fit mourir après l'avoir mise enceinte. Elle devint la femme de David et fut la mère de Salomon. Voir \vref{2 S. 11:3-5,26-27} et \vref{2 S. 12:24-25}.

\DicoEntry{BEELZÉBUL}\textit{, de l'hébreu «~Ba'al-Zebuwb~»~: «~seigneur des mouches~»}\newline
Divinité adorée par les Philistins et considérée comme le prince des démons. Voir \vref{2 R. 1:2-6} et \vref{Mc. 3:22-26}.

\DicoEntry{BÉLIAL}\textit{, de l'hébreu «~Beliya'al~»~: «~indignité~»}\newline
Symbolisant l'infidélité, la méchanceté et la perversité, il s'agit d'un autre nom de Satan. Voir \vref{De. 15:9}~; \vref{1 S. 1:16}~; \vref{2 Co. 6:15}.

\DicoEntry{BÉNÉDICTION}\textit{, de l'hébreu «~barak~», «~berakah~», et du grec «~eulogia~»~: «~louange~»}\newline
Parole au travers de laquelle le Seigneur annonce sa grâce sur la vie d'une personne ou d'un peuple~; les bontés liées à la bénédiction sont cependant conditionnées par l'obéissance du bénéficiaire. Sous l'Ancienne Alliance, les pères avaient coutume de bénir leurs enfants~; la bénédiction se manifestait souvent par la prospérité matérielle, la fécondité et la santé. La bénédiction est la marque du chrétien qui voit avec un œil spirituel la faveur de Dieu dans sa vie et qui bénit Dieu dans toutes les circonstances. Voir \vref{Ge. 49:1-28}~; \vref{De. 28:1-14}~; \vref{Ps. 103:1-2} et \vref{Ep. 1:3}.

\DicoEntry{BENJAMIN}\textit{, de l'hébreu «~Binyamiyn~»~: «~fils de ma main droite~»}\newline
Dernier fils de Jacob et Rachel~; sa mère mourut en lui donnant naissance. Il est l'ancêtre de la tribu de Benjamin. Voir \vref{Ge. 35:16-18} et \vref{Ge. 49:27}.

\DicoEntry{BÊTE}\textit{, de l'araméen «~cheyva´~» et du grec «~therion~»~: «~bête, animal~»}\newline
Dans les récits à caractère apocalyptique, les bêtes sont des animaux symbolisant les puissances politiques. Voir \vref{Da. 7} et \vref{Ap. 13,17}.

\DicoEntry{BÉTHANIE}\textit{, du grec «~Bethania~»~: «~maison des dattes non mûres~», «~maison de l'affligé~»}\newline
Village proche de Jérusalem, près de la Montagne des Oliviers, où vivaient Simon le lépreux, Marthe*, Marie* et Lazare* que Jésus ressuscita des morts. Voir \vref{Mc. 11:1}~; \vref{Mc. 14:3} et \vref{Jn. 11:1}.

\DicoEntry{BÉTHEL}\textit{, de l'hébreu «~Beyth-'El~»~: «~maison de Dieu~»}\newline
Ville cananéenne située à l'occident de Aï. Autrefois appelé Luz - mais renommé par Jacob quand il y eut la visitation de Yahweh - Béthel devient la possession de la tribu d'Ephraïm lors de la conquête de Canaan conduite par Josué. Elle était connue pour être un lieu d'adoration où on y rendait un culte à Yahweh. Malheureusement suite au schisme d'Israël - et notamment sous le règne de Jéroboam, roi de Juda - elle devient un lieu d'abomination. C'est Josias son successeur qui, désirant marcher avec Yahweh, y ôta les faux dieux, rétablissant ainsi le culte en l'honneur du Dieu d'Israël. (\vref{Ge. 28:10-22}~; \vref{Ge. 31:13}~; \vref{1 R. 12:26-32}~; \vref{2 R. 23:1-15})

\DicoEntry{BETHLEHEM}\textit{, de l'hébreu «~Beyth Lechem~»~: «~maison du pain~»}\newline
Ville de Juda, lieu de naissance de David et de Jésus-Christ. Voir \vref{1 S. 16}~; \vref{Mt. 2:16} et \vref{Lu. 2:4-7}.

\DicoEntry{BIBLE}\textit{, du grec «~biblia~»~: «~livres~»}\newline
Aussi appelée «~Parole de Dieu~», recueil de livres inspirés de Dieu et utiles pour enseigner, convaincre, corriger et instruire dans la justice. Voir \vref{2 Ti. 3:16}.

\DicoEntry{BLASPHÈME}\textit{, de l'hébreu «~na'ats~»~: «~repousser, mépriser, rejeter~» et du grec «~blasphemia~»~: «~discours impie et injurieux envers Dieu~»}\newline
Parole outrageante ou insultante envers Dieu. Voir \vref{2 S. 12:14} et \vref{Ap. 16:9}.

\DicoEntry{BLASPHÈME CONTRE LE SAINT-ESPRIT}\textit{}\newline
Voir commentaire \vref{Mt. 12:22-32}.

\DicoEntry{BOAZ}\textit{, de l'hébreu «~Bo`az~»~: «~en lui est la force~»}\newline
Fils de Salmon et arrière-grand-père du roi David, il épousa Ruth la Moabite. Voir \vref{Ru. 4:13} et \vref{Mt. 1:1-6}.

\DicoEntry{BREBIS}\textit{}\newline
Femelle du bélier, c'est l'animal pour qui le berger donne sa vie. Elle est le symbole du véritable disciple qui n'obéit qu'à la voix de son Maître et qui se laisse conduire et choyer par Jésus, le bon berger. Voir \vref{Jn. 10:1-16}.

\DicoEntry{CAIN}\textit{, de l'hébreu «~Qayin~»~: «~possession~», «~artisan, forgeron~»}\newline
Fils aîné d'Adam et Eve, il fut l'auteur du premier homicide en tuant son frère Abel. Il engendra Lémec, premier polygame de l'histoire. Voir \vref{Ge. 4:1-8,16-19}.

\DicoEntry{CAÏPHE}\textit{, du grec «~Kaiaphas~»~: «~avenant, pierre~»}\newline
Grand prêtre nommé par Valerius Gratus, gouverneur de Judée de 15 à 26 ap. J.-C. Caïphe exerça sa fonction de 18 à 36. N'ayant pas reconnu en Christ le Messie, il déclara néanmoins qu'il était avantageux qu'un seul homme meure pour le peuple et participa à la condamnation à mort de Jésus. Voir \vref{Mt. 26:3,57-66}~; \vref{Jn. 11:47-53} et \vref{Jn. 18:12-14}.

\DicoEntry{CALEB}\textit{, de l'hébreu «~Kaleb~»~: «~chien~»}\newline
Fils de Jephunné, issu de la tribu de Juda, il fut l'un des espions envoyés pour explorer le pays de Canaan. Avec Josué, il fut le seul, parmi la génération sortie d'Egypte, à entrer dans la terre promise. Voir \vref{No. 13:1-6} et \vref{No. 14:22-30}.

\DicoEntry{CALENDRIER HEBRAÏQUE}\textit{}\newline
Nisan (ou Abib) = Mars~; Iyyar (ou Ziv) = Avril~; Sivan = Mai~; Thammuz = Juin~; Ab = Juillet~; Elul = Août~; Tisri (ou Ethanim) = Septembre~; Marchesvan (ou Bul) = Octobre~; Chislev (ou Kisleu) = Novembre~; Tébeth = Décembre~; Schebat = Janvier~; Adar = Février

\DicoEntry{CAMP}\textit{, de l'hébreu «~machaneh~»~: «~campement, camp~»}\newline
Lieu de stationnement temporaire d'un groupement civil ou militaire. Voir \vref{Ge. 32:2} et \vref{Ex. 14:19}.

\DicoEntry{CANAAN}\textit{, de l'hébreu «~Kena'an~»~: «~terre basse~», «~marchand~»}\newline
Fils de Cham. Ses descendants occupèrent la région éponyme qui correspond plus ou moins aujourd'hui aux territoires réunissant la Palestine, l'État d'Israël, l'ouest de la Jordanie, le sud du Liban et l'ouest de la Syrie. Ce territoire correspondait également à la terre promise par Dieu aux Israélites dont ils prirent possession sous la conduite de Josué. Voir \vref{Ge. 9:18}~; \vref{Jos. 6-21} et \vref{Ac. 13:19}.

\DicoEntry{CÉSAR, Jules}\textit{, (100 av. J.-C - 44 av. J.-C.) du latin «~kaisar~»~: «~séparé~», «~chef~»}\newline
Général romain. Son nom devint par la suite celui de certains empereurs romains. Dans les Ecritures, César symbolise également les autorités séculières. Voir \vref{Mt. 22:21}.

\DicoEntry{CESARÉE de Philippes}\textit{, du grec «~Kaisereia~»~: «~appartenant à César~»}\newline
Située près des sources du Jourdain, territoire qui doit son nom à l'empereur Tibère. C'est dans cette contrée que Pierre reconnut en Jésus le Messie, le Fils du Dieu vivant. Voir \vref{Mt. 16:13-17}.

\DicoEntry{CHAIR}\textit{, du grec «~sarx~»~: «~la chair, le corps, la nature sensuelle de l'homme, la nature animale~»}\newline
Selon le contexte, désigne le corps humain, l'être humain ou la nature humaine conduite par le péché*. Voir \vref{Lu. 3:6}~; \vref{Lu. 24:39}~; \vref{Jn. 17:2}~; \vref{Ga. 5:16-21}~; \vref{Ro. 8:5-9} et \vref{Ep. 2:3}.

\DicoEntry{CHALDÉE}\textit{, de l'hébreu «~Kasdiy~»~: «~briseurs de mottes~», «~comme des démons~»}\newline
Région située au sud de la Mésopotamie dont Abraham est originaire. Voir \vref{Ge. 11:28}.

\DicoEntry{CHAM}\textit{, de l'hébreu «~Cham~»~: «~chaud, bouillant~»}\newline
Fils de Noé et père de Canaan qui fut maudit par Noé. Voir \vref{Ge. 9:18-27}.

\DicoEntry{CHARAN}\textit{, de l'hébreu «~Charan~»~: «~montagnard~», «~route, caravane~»}\newline
Région proche d'Ur en Chaldée où Abraham séjourna jusqu'à la mort de son père Térach. Voir \vref{Ge. 11:31} et \vref{Ge. 12:4}.

\DicoEntry{CHEMIN DE SABBAT}\textit{}\newline
Selon la loi de Moïse, distance maximum que les juifs peuvent parcourir de leur demeure le jour du sabbat* (cf. tableau des mesures et.distances). Voir \vref{Ac. 1:12}.

\DicoEntry{CHÉRUBINS}\textit{, de l'hébreu «~keruwb~»~: «~être angélique, chérubin~»}\newline
Catégorie d'anges portant et.ou gardant la gloire de Dieu. Yahweh en avait placé à l'entrée du jardin d'Eden pour empêcher l'homme d'y accéder. Deux chérubins sur lesquels Dieu siégeait étaient représentés sur le propitiatoire*. Avant sa chute, Satan était un chérubin protecteur. Voir \vref{Ge. 3:24}~; \vref{Ex. 25:17-20}~; \vref{Es. 37:16} et \vref{Ez. 28:14}.

\DicoEntry{CHRÉTIEN}\textit{, du grec «~christanos~»~: «~de Christ~», «~petit christ~», «~comme Christ~»}\newline
Comme son étymologie le suggère, le chrétien appartient à Christ, dont il a la nature et à qui il ressemble. Il est donc un disciple* de Jésus-Christ qui suit son enseignement et le met en pratique. Ce terme fut employé pour la première fois à Antioche. Voir \vref{Ac. 11:26}.

\DicoEntry{CHRIST}\textit{, du grec «~christos~» et de l'hébreu «~mashiyach~»~: «~oint~»}\newline
Souvent accolé au nom de Jésus*, ce terme suggère que ce dernier est l'oint* de Dieu, le Messie tant attendu. Jésus annonça l'émergence de faux christs (= faux ouvriers de Christ) à la fin des temps. Voir \vref{Ro. 1:1}~; \vref{Mt. 16:15-16}~; \vref{Mt. 24:24} et \vref{Mc. 13:22-23} et \vref{Hé. 1:9}.

\DicoEntry{CIRCONCISION}\textit{, de l'hébreu «~muwlah~»~: «~circoncision~: couper autour~»}\newline
Section et ablation du prépuce. En signe d'alliance, Dieu ordonna à Abraham de circoncire tous les mâles de sa maison~; les enfants d'Israël ont perpétré cette pratique. Sous la Nouvelle Alliance, la circoncision requise est celle du cœur. Voir \vref{Ge. 17:9-14}~; \vref{Lu. 1:59}~; \vref{1 Co. 7:19} et \vref{Ro. 2:25-29}.

\DicoEntry{CLAUDE}\textit{, (10 av. J.-C. – 54 ap. J.-C.), du grec «~Klaudios~»~: «~boiteux~»}\newline
Fils de Nero Claudius Drusus (38 av. J.-C. – 9 av. J.-C.). Empereur romain qui régna de 41 à 54 ap. J.-C.~; il chassa les Juifs de Rome, parmi lesquels Aquilas et Priscille. Voir \vref{Ac. 18:2}.

\DicoEntry{CLERGÉ}\textit{, du grec «~klêrikos~»~: «~homme d'église~»}\newline
Au sein de l'Eglise catholique, corps séparé des fidèles ayant une fonction gouvernante~; ses membres sont appelés les clercs ou les ecclésiastiques. Ils accèdent à leur position par le sacrement de l'ordre (ou ordination*) qui comporte trois classes~: les diacres, les prêtres et les évêques.

\DicoEntry{CLÉRICALISME}\textit{, dérivé de clérical~: «~dévoué aux intérêts du clergé~»}\newline
Tendance en vertu de laquelle le clergé sort du domaine religieux pour se mêler des affaires publiques et politiques afin d'y exercer une influence et faire prédominer ses idées.

\DicoEntry{CŒUR}\textit{, de l'hébreu «~lebab~»~: «~homme intérieur, volonté, cœur, partie interne, pensée~»}\newline
Organe permettant la circulation du sang, les Ecritures définissent le cœur comme un grand abîme. Siège des émotions et des pensées intimes, il peut être une bonne ou une mauvaise source. Voir \vref{Ge. 20:6}~; \vref{Lé. 19:17}~; \vref{De. 4:29}~; \vref{1 S. 12:24} et \vref{Mc. 7:21}.

\DicoEntry{COLOSSES}\textit{, du grec «~Kolossai~»~: «~monstruosités~»}\newline
Située en Asie Mineure, ville de Phrygie se trouvant à environ deux cents kilomètres d'Ephèse. Il s'y trouvait une église à qui Paul écrivit une lettre qui figure dans le canon biblique.

\DicoEntry{COMMUNION}\textit{, du grec «~koinonia~»~: «~ce qui est commun à plusieurs personnes, association, union~»}\newline
Le disciple* de Christ est appelé à vivre deux types de communion. Il doit tout d'abord être en communion intime avec Dieu puis avec d'autres membres du corps de Christ pour vivre la communion fraternelle. Voir \vref{Ps. 133}~; \vref{Ac. 2:42}~; \vref{2 Co. 13:11-13} et \vref{1 Jn. 1:3}.

\DicoEntry{CONCILE}\textit{, du latin «~concilium~»~: «~assemblée~»}\newline
Assemblée d'évêques de l'Eglise catholique (également connue sous l'appellation «~pères de l'Eglise catholique~») réunis dans le but de définir les règles de la foi chrétienne. Cette pratique va à l'encontre du message de Christ puisqu'il a strictement condamné la modification du message qu'il a lui-même prêché et confié aux apôtres*. Voir \vref{Mt. 5:18} et \vref{Ga. 1:8-9}.

\DicoEntry{CONFESSION}\textit{, du grec «~exomologeo~»~: «~confesser, professer, reconnaître ouvertement~»}\newline
On peut confesser des péchés pour exposer les ténèbres ou le nom du Seigneur pour le louer et annoncer la vérité. Voir \vref{Mc. 1:5}~; \vref{Ac. 19:18} et \vref{Ph. 2:11}.

\DicoEntry{CONVERSION}\textit{, du grec «~epistrepho~»~: «~action de se retourner, de se tourner vers~»}\newline
Fruit d'une sincère repentance, la conversion est la décision de se tourner vers Christ et de se détourner des œuvres des ténèbres. Voir \vref{Ac. 26:20}~; \vref{Ga. 4:9}~; \vref{2 Co. 3:16} et \vref{1 Pi. 2:25}.

\DicoEntry{CONVOITISE}\textit{, du grec «~epithumia~»~: «~désir, convoitise, luxure~»}\newline
Précédant l'acte du péché, désir amorcé par les sens humains et lié à la soif de posséder ce qui est défendu et ce que le monde offre. Voir \vref{Ja. 1:14-15} et \vref{1 Jn. 2:15-17}.

\DicoEntry{CORINTHE}\textit{, du grec «~Korinthos~»~: «~rassasié~»}\newline
Dans l'Antiquité, Corinthe, capitale de l'Achaïe, était la ville la plus prospère et la plus puissante de Grèce. Située sur un isthme séparant la mer Egée de la mer Ionienne, Corinthe était au carrefour de l'Asie et de l'Italie et constituait un véritable centre commercial où les produits orientaux et occidentaux se croisaient. Paul demeura au moins un an et six mois à Corinthe, durée pendant laquelle il enseigna la parole de Dieu. Il écrivit par la suite deux lettres aux saints de cette ville qu'on retrouve dans le canon biblique.

\DicoEntry{CORNEILLE}\textit{, du grec «~Kornelios~»~: «~d'une corne~»}\newline
Centenier romain juste et craignant Dieu. Il vivait à Césarée où Simon Pierre fut envoyé pour lui annoncer la Parole. Au travers de l'expérience de Corneille, Dieu confirma que le salut était pour toutes les nations. Voir \vref{Ac. 10}.

\DicoEntry{COURONNE}\textit{, du grec «~stephanos~»~: «~couronne, une marque de rang royal, récompense de la justice, ornement~»}\newline
Jésus-Christ reçut une couronne d'épines lors de la crucifixion pour rappeler ironiquement son titre de «~roi des Juifs~». Après la résurrection, les chrétiens recevront une couronne en récompense de leur intégrité. Devant le trône de Dieu, les vingt-quatre vieillards jettent leurs couronnes pour rendre gloire à Dieu. Voir \vref{Mt. 27:29}~; \vref{Ja. 1:12}~; \vref{1 Co. 9:25}~; \vref{1 Pi. 5:4}~; \vref{2 Ti. 4:8}~; \vref{Ap. 2:10} et \vref{Ap. 4:4,10}.

\DicoEntry{CROIX}\textit{, du grec «~stauros~»~: «~pieu, croix~»}\newline
Châtiment romain consistant à clouer les mains et les pieds des condamnés sur des poteaux en bois en forme de croix. Symbole du sacrifice de Jésus pour le pardon des péchés, la croix est aussi l'image de la vie de souffrance et de consécration totale à laquelle est appelé tout disciple du Seigneur. Voir \vref{Es. 53}~; \vref{Mt. 16:24} et \vref{Lu. 9:23}.

\DicoEntry{CUPIDITÉ}\textit{, du grec «~pleonexia~»~: «~désir avide d'avoir plus, avarice~»}\newline
Forme d'idolâtrie*, péché consistant à désirer de manière excessive les biens de ce monde (argent, richesses, etc.) et menant à la perdition. Voir \vref{Ep. 5:3}~; \vref{Col. 3:5} et \vref{2 Pi. 2:14}.

\DicoEntry{CYRÈNE}\textit{, du grec «~Kurene~»~: «~suprématie de la bride~», «~qui gouverne, froid~»}\newline
Ville prospère située dans la région fertile d'Afrique du Nord (actuelle Libye) où vivait une importante communauté juive et de laquelle était originaire Simon à qui l'on demanda de porter la croix de Jésus. Voir \vref{Mc. 15:20-22} et \vref{Ac. 2:10}.

\DicoEntry{CYRUS II LE GRAND}\textit{(règne~: 559-530 av. J.-C.), du persan «~Kowresh~»~: «~possède la puissance, puissance suprême~»}\newline
Fils de Cambyse, il régna sur l'Empire perse. Réveillé par Yahweh, il publia un édit en faveur du retour des Juifs à Jérusalem pour la reconstruction du temple. Voir \vref{Esd. 1:1-2} et \vref{2 Ch. 36:22-23}.

\DicoEntry{DAGON}\textit{, «~Dagown~»~: «~un poisson~»}\newline
Divinité païenne adorée par les Philistins, il était représenté par un personnage avec des mains et une face humaine et le corps d'un poisson. Voir \vref{1 S. 5:1-5}.

\DicoEntry{DAN}\textit{, de l'hébreu «~dan~»~: «~un juge~»}\newline
Fils de Jacob et de Bilha, servante de Rachel, il est le père de la tribu des Danites. Voir \vref{Ge. 30:1-6} et \vref{Ge. 49:16-18}.

\DicoEntry{DANIEL}\textit{, de l'hébreu «~Daniye'l~»~: «~Dieu est mon juge~»}\newline
Issu d'une famille princière de Juda, il fut déporté pendant sa jeunesse de Jérusalem à Babylone où il reçut le nom de Beltshatsar. Son histoire est racontée dans le livre éponyme.

\DicoEntry{DARIQUE}\textit{, de l'hébreu «~darkemown~»~: «~darique, drachme, unité de mesure~»}\newline
Utilisée après le retour de l'exil babylonien, monnaie d'or mise en place par le roi Darius et circulant dans l'Empire perse. Voir \vref{Esd. 8:26-27} et \vref{Né. 7:71-72}.

\DicoEntry{DARIUS Ier}\textit{, (règne~: 522 av. J.-C. – 486 av. J.-C.), de l'hébreu «~Dar`yavesh~»~: «~seigneur~» (origine~: perse)}\newline
Fils d'Assuérus, d'origine mède, roi des Chaldéens. Il encouragea la reconstruction du temple de Jérusalem après la découverte des instructions laissées par Cyrus sur un rouleau retrouvé dans la province de Médie. Voir \vref{Esd. 6}.

\DicoEntry{DATHAN}\textit{, de l'hébreu «~Dathan~»~: «~appartenant à une fontaine~»}\newline
Issu de la tribu de Ruben, fils d'Eliab et frère d'Abiram, il participa avec Koré à la révolte contre Moïse et Aaron. Voir \vref{No. 16:1-35}.

\DicoEntry{DAVID}\textit{, de l'hébreu «~David~»~: «~bien aimé~»}\newline
Issu de la tribu de Juda et dernier fils d'Isaï, il entra dès son plus jeune âge au service du roi Saül avant de devenir roi d'Israël. Homme selon le cœur de Dieu, il connut de grands succès sur les champs de bataille et fut l'auteur de nombreux psaumes. Il régna quarante-quatre ans sur Israël puis son fils Salomon* lui succéda. Voir \vref{1 S. 13:14}~; \vref{1 S. 16:14-23}~; \vref{1 S. 17}~; \vref{1 R. 2:10-11} et \vref{Ac. 13:22}.

\DicoEntry{DÉBORA}\textit{, de l'hébreu «~Debowrah~»~: «~abeille~»}\newline
Femme de Lapiddoth, elle exerça les fonctions de prophétesse et juge en Israël. Elle fut utilisée par Dieu pour prophétiser la victoire d'Israël sur Canaan par Barak qu'elle accompagna sur le champ de bataille. Voir \vref{Jg. 4-5}.

\DicoEntry{DÉLUGE}\textit{, de l'hébreu «~mabbuwl~»~: «~inondation, déluge~»}\newline
Pluie torrentielle s'étant abattue sur la terre pendant quarante jours et quarante nuits au temps de Noé. Le déluge symbolisait le jugement de Dieu sur une génération dont la méchanceté avait atteint un niveau sans précédent. Tous les habitants et les animaux de la terre furent emportés par les eaux du déluge hormis Noé, sa famille et les animaux qui étaient avec eux dans l'arche*. Voir \vref{Ge. 6-8}.

\DicoEntry{DEMAS}\textit{, du grec «~Demas~»~: «~gouverneur du peuple~»}\newline
Compagnon d'œuvre de Paul qui le délaissa «~par amour pour le siècle présent~». Voir \vref{Col. 4:14} et \vref{2 Ti. 4:10}.

\DicoEntry{DEMETRIUS}\textit{, du grec «~Demetrios~»~: «~qui appartient à Déméter (déesse grecque de l'agriculture)~»}\newline
Orfèvre qui fabriquait des statues de la déesse Diane à Ephèse. Voyant son commerce mis en danger par les prédications de Paul, il déclencha une émeute contre ce dernier. Voir \vref{Ac. 19:23-41}.

\DicoEntry{DÉMONS}\textit{, du grec «~daimonion~»~: «~divinité inférieure, mauvais esprit, ministres du diable~»}\newline
Egalement appelés «~esprits impurs~», anges* déchus ayant pris part à la révolte et à la chute de Satan*. Ils peuvent posséder le corps d'une personne, mais sont soumis à la puissance de Jésus, au nom duquel les chrétiens peuvent les chasser. Voir \vref{Mt. 10:8}~; \vref{Mc. 7:26}~; \vref{Mc. 16:17}~; \vref{Lu. 4:33}~; \vref{Lu. 10:17}~; \vref{Jud. 1:6} et \vref{Ap. 12:4}.

\DicoEntry{DIABLE}\textit{}\newline
Voir SATAN.

\DicoEntry{DIACRE}\textit{, du grec «~diakonos~»~: «~domestique, subordonné, messager~»}\newline
Les premiers diacres étaient des hommes remplis de l'Esprit Saint et de sagesse~; ils furent nommés pour faire un travail complémentaire aux ministres de la Parole au sein de l'église de Jérusalem. Etienne* était l'un d'eux. Il existait aussi des femmes diaconesses comme Phœbe, de l'église de Cenchrées. Voir \vref{Ac. 6:1-8}~; \vref{Ro. 16:1-2} et \vref{1 Ti. 3:8-13}.

\DicoEntry{DIANE}\textit{, du grec «~Artemis~»~: «~de la lumière~»}\newline
Aussi appelée «~Artemis d'Ephèse~», divinité révérée dans toute l'Asie. Il existait un temple en son honneur à Ephèse. Voir \vref{Ac. 19:24-37}.

\DicoEntry{DIEU}\textit{}\newline
Dieu des dieux et Seigneur des seigneurs, il est le Créateur de l'univers, du ciel, de la terre et de tout ce qui s'y trouve. Architecte d'excellence, il forma l'homme à son image et lui manifesta un amour inconditionnel par son incarnation en Jésus-Christ*. Dieu se présenta à Moïse sous le nom YHWH* (=Je suis celui qui suis) montrant son caractère éternel. Il s'est révélé à différentes personnes sous divers noms et aspects, en fonction des situations traversées montrant qu'il est celui qui remplit tout en tous et qu'il est et a tout ce dont l'homme a besoin. Ainsi, on le découvre dans les Ecritures comme étant grand, unique et indivisible, omniprésent, omniscient, souverain, incorruptible, sage, patient, saint, parfait, merveilleux, tout-puissant, fidèle, juste et bon. Bien évidemment, Dieu ne peut en aucun cas être défini dans tout ce qu'il est, dans la mesure où sa nature même échappe à toute possibilité de frontière ou de limite. Toutefois, les saints auront l'éternité pour découvrir ce Père incomparable. Voir \vref{Ge. 1,2}~; \vref{Ge. 17:1}~; \vref{Ex. 3:14}~; \vref{De. 6:14}~; \vref{De. 10:17}~; \vref{Es. 6:3}~; \vref{Mal. 3:6}~; \vref{Ps. 11:7}~; \vref{Ps. 139:7-10}~; \vref{La. 3:22-23}~; \vref{Lu. 1:49}~; \vref{Ja. 1:17}~; \vref{1 Th. 4:17}~; \vref{1 Co. 1:9}~; \vref{Ro. 1:23}~; \vref{Ro. 2:4}~; \vref{Ro. 11:33-36}~; \vref{2 Ti. 4:8}~; \vref{Hé. 4:13} et \vref{1 Jn. 4:8}.

\DicoEntry{DÎME}\textit{, de l'hébreu «~ma'aser~»~: «~dîme, dixième partie~»}\newline
Abraham donna à Melchisédek la dîme du butin d'une bataille remportée (\vref{Ge. 14:17-20} et \vref{Hé. 7:1-2}). Yahweh instaura, au travers de Moïse, la dîme comme une loi à respecter par les enfants d'Israël. Il en existait quatre sortes~:
\\1. la dîme que les Lévites prélevaient sur le peuple (\vref{No. 18:21-24})
\\2. la dîme de la dîme, que les prêtres prélevaient sur les Lévites (\vref{No. 18:25-31}~; \vref{Né. 10:38})
\\3. la dîme consommée par les Juifs eux-mêmes lors des fêtes de Yahweh (\vref{De. 14:22-26})
\\4. la dîme pour l'étranger, la veuve, l'orphelin et le Lévite, donnée tous les trois ans (\vref{De. 14:28-29}).
\\Cette loi concernait exclusivement Israël et non l'Eglise – Jésus-Christ ayant accompli la loi (\vref{Mt. 5:17}). Sous la grâce, les chrétiens sont invités à faire des offrandes * librement et sans contrainte.

\DicoEntry{DINA}\textit{, de l'hébreu «~Diynah~»~: «~jugement, justice~»}\newline
Fille de Jacob et Léa. Elle fut enlevée et déshonorée par Sichem, fils de Hamor, prince du pays de Canaan. Sichem et tous les hommes de la ville furent ensuite tués par les frères de la jeune fille, Siméon et Lévi. Voir \vref{Ge. 34}.

\DicoEntry{DIOTRÈPHE}\textit{, du grec «~Diotrephes~»~: «~nourri par Zeus~»}\newline
Chrétien dont Jean dénonça l'arrogance et les mauvais agissements. Voir \vref{3 Jn. 1:9-11}.

\DicoEntry{DISCIPLE}\textit{, du grec «~mathetes~»~: «~un étudiant, un élève, un disciple~»}\newline
Personne qui écoute les enseignements de son maître et les met en pratique en vue de devenir comme lui. Jésus en choisit douze qu'il forma pendant son service. Le disciple de Christ doit manifester le caractère de son maître, lui être pleinement consacré et être prêt à souffrir en son nom. Voir \vref{Mt. 10}~; \vref{Lu. 6:12-16}~; \vref{Lu. 14:26-33}.

\DicoEntry{DIVORCE}\textit{, du grec «~apostasion~»~: «~divorce, répudiation, lettre de divorce~»}\newline
Brisement des liens du mariage*. Il fut autorisé sous la loi de Moïse à cause de la dureté des cœurs, mais Christ rappela l'indissolubilité du mariage au commencement. Voir \vref{De. 24:1-3} et \vref{Mt. 19:3-8}.

\DicoEntry{DOCTEUR}\textit{, du grec «~didaskalos~»~: «~professeur~», «~maître~».}\newline
Sous la loi de Moïse, les docteurs de la loi étaient chargés d'expliquer la Torah. Certains d'entre eux s'opposèrent à Jésus. Sous la Nouvelle Alliance, le docteur est un des cinq services liés à la Parole évoqués en \vref{Ep. 4:11}. Il enseigne la Parole de Dieu qui guérit les blessures de l'âme. Selon \vref{Ja. 3:1}, nous ne sommes pas tous appelés à être des docteurs. Voir \vref{Lu. 2:46}~; \vref{Lu. 5:17}~; \vref{1 Co. 12:28} et \vref{Ep. 4:11}.

\DicoEntry{DONS SPIRITUELS}\textit{, du grec «~charisma~»~: «~faveur que reçoit quelqu'un sans aucun mérite de sa part~», «~dons provenant du pouvoir de la grâce divine~»}\newline
Capacités distribuées par le Saint-Esprit aux chrétiens en vue de la formation et de l'édification des saints. Voir \vref{1 Co. 12:1-11}~; \vref{1 Co. 14:12}~; \vref{Ro. 12:6} et \vref{1 Pi. 4:10}.

\DicoEntry{EDEN}\textit{, de l'hébreu «~'Eden~»~: «~plaisir, délices~»}\newline
Appelé aussi jardin de Dieu, premier lieu de résidence d'Adam et Eve. Yahweh y avait fait pousser des arbres de toutes espèces~; il y avait également placé au milieu l'arbre de vie* ainsi que l'arbre de la connaissance du bien et du mal* dont la consommation des fruits conduirait à la mort. L'homme fut établi en tant que gardien et cultivateur de ce jardin. Cependant, il pêcha avec la femme et ils furent chassés de ce lieu des délices. Voir \vref{Ge. 2}~; \vref{Ge. 3:23-24} et \vref{Ez. 28:13}.

\DicoEntry{ÉGLISE}\textit{, du grec «~ekklesia~»~: «~appel hors de~»}\newline
Peuple mis à part dont Christ est le chef. L'Eglise est la sainte habitation de Dieu en esprit, le corps de Christ, l'épouse de l'Agneau. On distingue l'Eglise universelle - qui regroupe tous les saints du monde entier - de l'église locale - qui est composée de tous les chrétiens d'une ville. Voir \vref{Ac. 2:47}~; \vref{1 Th. 1:1}~; \vref{1 Co. 1:2}~; \vref{1 Co. 3:16}~; \vref{1 Co. 12:27}~; \vref{Ep. 2:20-22}~; \vref{Ep. 5:22-32}~; \vref{Ph. 1:1} et \vref{1 Ti. 2:4}.

\DicoEntry{ÉLÉAZAR}\textit{, de l'hébreu «~'El'azar~»~: «~Dieu a secouru~»}\newline
Fils d'Aaron, il était chef des chefs des Lévites avant de devenir le second grand prêtre d'Israël. Voir \vref{No. 3:32} et \vref{No. 20:25-28}.

\DicoEntry{ÉLECTION, ELU}\textit{, de l'hébreu «~bachiyr~» et du grec «~eklektos~»~: «~choisi, élu de Dieu~»}\newline
Dans le Tanakh, Israël fut présenté comme le peuple élu de Yahweh, appelé à être un exemple pour toutes les nations de la terre. Cette élection n'est pas synonyme de préférence, car la volonté de Dieu est de sauver tous les hommes. Au travers de l'œuvre de la croix, Dieu a effectivement montré que son choix se porte vers l'humanité tout entière en payant le prix des péchés de tous. Dans son omniscience, il sait toutefois d'avance qui croira en lui ou pas. Selon la parole, même après la conversion, le chrétien doit travailler son élection, c'est-à-dire se sanctifier et obéir aux commandements de Yahweh pour entrer dans son royaume. Voir \vref{Es. 45:4}~; \vref{Es. 49:6}~; \vref{Mt. 22:14}~; \vref{Ep. 1:4-6} et \vref{2 Pi. 1:10-11}.

\DicoEntry{ÉLIE}\textit{, de l'hébreu «~Eliyah~»~: «~Yahweh est mon Dieu~»}\newline
Prophète d'origine tschibite que Dieu suscita en Israël au temps du roi Achab*. Il ne connut point la mort, mais fut enlevé par le Seigneur. Son histoire, ses combats et ses exploits sont racontés dans les livres des Rois.

\DicoEntry{ÉLISÉE}\textit{, de l'hébreu «~'Eliysha'~»~: «~Dieu est sauveur~»}\newline
Prophète du royaume d'Israël, il succéda à Elie après avoir reçu la double portion de l'esprit qui était sur ce dernier. Après sa mort, ses os rendirent la vie à un défunt. Voir \vref{1 R. 19:16-21}~; \vref{2 R. 2:9-11} et \vref{2 R. 13:20-21}.

\DicoEntry{ENFER}\textit{}\newline
Voir SEJOUR DES MORTS.

\DicoEntry{ENLÈVEMENT}\textit{, du grec «~metathesis~»~: «~transfert d'un lieu à un autre, changement~»}\newline
Ravissement d'hommes au ciel sans que ces derniers ne connaissent la mort*. Dans le Tanakh, se trouvent deux cas d'enlèvement~: Hénoc (\vref{Ge. 5:24}~; \vref{Hé. 11:5}) et Elie (\vref{2 R. 2:11}). L'Eglise sera de même enlevée par le Seigneur au son de la dernière trompette. Voir \vref{1 Th. 4:17} et \vref{1 Co. 15:51-57}.

\DicoEntry{ÉPHÈSE}\textit{, du grec «~Ephesos~»~: «~permis~»}\newline
Une des principales villes de l'empire romain sous le règne de l'empereur Claude 1er (10 av. J.-C. – 54 ap. J.-C.) Ephèse possédait le plus grand port de l'Asie Mineure, ce qui lui attribuait le contrôle du trafic commercial. Richissime et prospère, elle était renommée pour son faste, sa liberté de parole et constituait donc un endroit privilégié pour les philosophes. L'église d'Ephèse naquit du ministère de Paul, qui y enseigna pendant au moins deux ans lors de son troisième voyage missionnaire. Cette église - figurant parmi les sept du livre d'Apocalypse - fit preuve de discernement et pratiquait de bonnes œuvres, mais le Seigneur avait néanmoins un reproche à lui adresser. Elle représente l'église apostate. Voir Epître aux Ephésiens et \vref{Ap. 2:1-7}.

\DicoEntry{ÉPHOD}\textit{, de l'hébreu «~ephowd~»~: «~couverture~»}\newline
Vêtement que les prêtres portaient par-dessus leur tunique lorsqu'ils étaient en service. L'éphod du grand prêtre était de broderie~; le pectoral était posé sur son devant. Voir \vref{Ex. 28} et \vref{Lé. 8:7}.

\DicoEntry{ÉPHRAIM}\textit{, de l'hébreu «~'Ephrayim~»: «~double fertilité~»}\newline
Second fils de Joseph né en Egypte, il fut adopté par Jacob avant sa mort et devint ainsi l'ancêtre d'une des douze tribus d'Israël. Voir \vref{Ge. 41:52}~; \vref{Ge. 48:5} et \vref{Jos. 14:4}.

\DicoEntry{ÉPICURIENS}\textit{, du grec «~epikoureios~»~: «~celui qui aide, le défenseur~»}\newline
Fondé à Athènes en 306 av. J.-C., groupe de philosophes se réclamant de la doctrine d'Epicure (341 av. J.-C. à 270 av. J.-C.). Ce dernier fonda une des plus importantes écoles philosophiques de l'Antiquité. Il développa une théorie athée selon laquelle l'homme est encouragé à rechercher les plaisirs matériels et sensuels. Rejetant la pensée d'une vie après la mort, les épicuriens renient l'existence d'un créateur qui se préoccuperait des hommes. Des adeptes de cette philosophie se confrontèrent à la doctrine de Christ annoncée par Paul et cherchèrent à l'entendre. Voir \vref{Ac. 17:18-20}.

\DicoEntry{ÉSAÏE}\textit{, de l'hébreu «~Yesha'yah~»~: «~Yahweh a sauvé~»}\newline
Fils d'Amotz, un prophète de Yahweh contemporain des rois Ozias, Jotham, Achaz et Ezéchias, il annonça la venue du Messie. L'ensemble de ses prophéties est contenu dans le livre portant son nom.

\DicoEntry{ÉSAÜ}\textit{, de l'hébreu «~'Esav~»~: «~velu, poilu, chevelu~»}\newline
Fils d'Isaac et Rebecca et frère jumeau de Jacob, qui lui soutira son droit d'aînesse et sa bénédiction. Il prit pour femmes Judith et Basmath, toutes deux originaires de Canaan. Egalement connu sous le nom d'Edom, il devint l'ancêtre des Edomites. Voir \vref{Ge. 25:25-34}~; \vref{Ge. 27} et \vref{Ge. 36}.

\DicoEntry{ESDRAS}\textit{, de l'hébreu «~`Ezra'~»~: «~secours~»}\newline
Fils de Sereja et descendant du grand prêtre Aaron, Esdras était scribe et prêtre. Il enseigna le peuple de Dieu dans la loi et mit en place des réformes après la reconstruction du temple. Son histoire se trouve dans le livre éponyme.

\DicoEntry{ESPRIT}\textit{, de l'hébreu «~ruwach~»~: «~vent, souffle, esprit~» et du grec «~pneuma~»~: «~vérité, inspiration, souffle, vent~»}\newline
L'esprit humain est aussi appelé homme intérieur, il constitue la partie spirituelle de l'homme lui permettant d'agir, de prendre des décisions et d'être en contact avec Dieu ou tout autre esprit. Principe vital, il amène l'âme à la vie. Avec l'âme* et le corps, l'esprit constitue l'être humain. Voir \vref{Ge. 6:3}~; \vref{Ex. 31:3}~; \vref{Job 27:3}~; \vref{Job 32:8}~; \vref{Mt. 12:28} et \vref{1 Th. 5:23}.

\DicoEntry{ESPRIT IMPUR}\textit{}\newline
Voir DEMONS.

\DicoEntry{ESTHER}\textit{, dérivation du perse «~'Ecter~»~: «~étoile~»}\newline
Cousine de Mardochée, Juif d'origine benjaminite, reine de Perse, épouse du roi Assuérus. Son nom juif était Hadassa~: «~myrte~». Son histoire, qui se déroula à Suse, est racontée dans le livre portant son nom.

\DicoEntry{ÉTANG DE FEU}\textit{}\newline
Lieu de douleur et de damnation éternelle créé initialement pour le diable et ses anges. Y seront jetés la bête et le faux prophète, le diable, la mort et le séjour des morts* ainsi que tous ceux dont le nom ne sera pas trouvé dans le livre de vie. Voir \vref{Mt. 25:41}~; \vref{Ap. 19:20} et \vref{Ap. 20:7-15}.

\DicoEntry{ÉTIENNE}\textit{, du grec «~stephanos~»~: «~couronne~»}\newline
Diacre* de l'église de Jérusalem rempli de sagesse et d'Esprit Saint. Premier martyr chrétien, sa mort marqua le début d'une grande persécution contre l'Eglise. Voir \vref{Ac. 6:1-6}~; \vref{Ac. 7} et \vref{Ac. 8:1-3}.

\DicoEntry{EUNUQUE}\textit{, du grec «~cariyc~»~: «~eunuque, chambellan castré~»}\newline
Homme dans l'incapacité de procréer ou émasculé. Dans l'Antiquité, les rois se choisissaient des eunuques pour les servir. En les castrant, ils s'assuraient de la fidélité et l'intégrité de ces derniers. En outre, Jésus distingua trois types d'eunuques. Voir \vref{2 R. 20:18}~; \vref{Da. 1:7}~; \vref{1 Ch. 28:1} et \vref{Mt. 19:12}).

\DicoEntry{ÉVANGELISTE}\textit{}\newline
Un des cinq ministères d'\vref{Ep. 4:11} dont la mission est de prêcher la repentance et la conversion à Jésus-Christ. Comme les ministères évoqués en \vref{Ep. 4:11}, il travaille également à la perfection des saints. Philippe exerça ce ministère, Timothée fut de même encouragé à faire l'œuvre d'un évangéliste. Tous les chrétiens doivent également évangéliser. Voir \vref{Ac. 21:8}~; \vref{Ep. 4:11} et \vref{2 Ti. 4:5}.

\DicoEntry{ÉVANGILE}\textit{}\newline
Enseignement donné par Jésus-Christ, la prédication de la croix (la mort et la résurrection de Jésus-Christ) et du Royaume de Dieu qui s'est approché des hommes (voir ROYAUME DE DIEU). Ce message annonce le salut, la guérison du cœur, la joie en Jésus-Christ, la justice, la paix, la grâce et la vie éternelle accordée à l'homme repentant, mais aussi le jugement à venir. Les apôtres propagèrent l'Evangile~; de même, tous les chrétiens sont appelés à le faire. Voir \vref{Es. 61}~; \vref{Mt. 10:7}~; \vref{Mt. 28:19-20}~; \vref{1 Co. 15:1-4}~; \vref{Ro. 1:16} et \vref{2 Ti. 4:1}.

\DicoEntry{EVE}\textit{, de l'hébreu «~Chavvah~»~: «~vie~»}\newline
Première femme et épouse d'Adam, elle fut formée à partir de la côte de son mari dans le but d'être l'aide de ce dernier. Séduite par Satan déguisé en serpent, elle mangea le fruit de la connaissance du bien et du mal et fut avec Adam chassée du jardin. Elle donna naissance à Caïn, Abel et Seth. Voir \vref{Ge. 2:18-24}~; \vref{Ge. 3:1-13} et \vref{Ge. 4:1-2,25}.

\DicoEntry{ÉVÊQUE}\textit{, du grec «~episcopos~»~: «~investigation, inspection, visite d'inspection~», «~acte par lequel Dieu visite les hommes, observe leurs voies, leurs caractères, pour leur accorder en partage joie ou tristesse~», «~surveillance, contrôle, fonction d'un ancien~», «~la charge d'une église chrétienne~».}\newline
Il est question ici d'une fonction consistant à visiter les assemblées, les inspecter afin de s'assurer du bon ordre. Voir \vref{Lu. 19:44}~; \vref{Ac. 1:20}~; \vref{1 Ti. 3:1}~; \vref{1 Pi. 2:12}.

\DicoEntry{EXPIATION}\textit{, de l'hébreu «~kaphar~»~: «~couvrir, purger, faire une expiation~»}\newline
Action de couvrir les fautes et les souillures de l'homme afin qu'il soit réconcilié avec Dieu. Sous l'Ancienne Alliance, le grand prêtre faisait tous les ans un sacrifice d'expiation en entrant dans le Saint des saints pour ses péchés et les péchés du peuple. Par son sacrifice, Christ est devenu la victime expiatoire pour les péchés de tous les hommes en les prenant sur lui à la croix~; il est l'Agneau de Dieu qui ôte les péchés du monde. Voir \vref{Lé. 16}~; \vref{Jn. 1:29}~; \vref{1 Jn. 2:2} et \vref{1 Jn. 4:10}.

\DicoEntry{ÉZÉCHIAS}\textit{, de l'hébreu «~Yechizqiyah~»~: «~Yahweh est ma force~»}\newline
Fils d'Osée, roi de Juda sur qui il régna vingt-neuf ans. Figurant parmi les rois les plus intègres, son règne fut caractérisé par la droiture et la fidélité à Yahweh. Voir \vref{2 R. 18-19}.

\DicoEntry{ÉZÉCHIEL}\textit{, de l'hébreu «~Yechezqe'l~»~: «~Dieu fortifie~»}\newline
Fils de Buri, prêtre et prophète de Yahweh ayant été déporté à Babylone. Il reçut de nombreuses visions - sur son temps et les temps de la fin - racontées dans le livre qui porte son nom.

\DicoEntry{FÉLIX}\textit{, du grec «~Phestos~»~: «~joyeux, en fête~»}\newline
Gouverneur de Judée de 52 à 60 ap. J.-C., il emprisonna Paul à la suite des plaintes des Juifs. S'entretenant avec lui de temps en temps et lui octroyant certaines libertés, Felix garda Paul en prison deux ans pour plaire aux Juifs. Voir \vref{Ac. 24}.

\DicoEntry{FESTUS}\textit{, du grec «~Phestos~»~: «~en fête, joyeux~»}\newline
Gouverneur de Judée qui succéda à Félix* de 60 à 62 ap. J.-C. Il poursuivit l'instruction du procès de Paul que les Juifs accusaient. Il permit à Paul de s'exprimer devant le roi Agrippa* et l'envoya à Rome afin qu'il comparaisse devant César*. Voir \vref{Ac. 24:27} et \vref{Ac. 25,26}.

\DicoEntry{FÊTES DE YAHWEH}\textit{}\newline
Selon la loi juive, sept fêtes étaient célébrées en l'honneur de Yahweh~: la Pâque de Yahweh, la fête des pains sans levain~; la fête des prémices~; la Pentecôte~; la fête des trompettes~; le jour des expiations et la fête des tabernacles. Voir \vref{Lé. 23:6-43}.

\DicoEntry{FIGUIER}\textit{}\newline
Arbre fruitier sous lequel il était coutume d'étudier la Torah en Israël. Ses fruits excellents et doux servaient en médecine. Le figuier est retrouvé dans de nombreuses histoires et paraboles des Ecritures. Il symbolise la douceur et l'humilité. Voir \vref{Jg. 9:11}~; \vref{2 R. 20:1-7}~; \vref{Lu. 13:6-9} et \vref{Jn. 1:43-51}.

\DicoEntry{FILS DE DIEU}\textit{}\newline
Expression désignant selon le contexte~:
\\1. les anges. Voir \vref{Ge. 6:2-4}~; \vref{Job 38:7} et \vref{Da. 3:25}.
\\2. Adam. Voir \vref{Lu. 3:38}.
\\3. les chrétiens. Voir \vref{Ga. 3:26} et \vref{Ro. 8:14}.
\\4. Jésus-Christ, le Fils unique de Dieu, en qui habite la plénitude de la divinité. Voir \vref{Mc. 15:39}~; \vref{Lu. 22:70}~; \vref{Jn. 1:14,34,49}~; \vref{Ro. 1:4}~; \vref{Col. 2:9} et \vref{1 Jn. 4:9,15}.

\DicoEntry{FILS DE L'HOMME}\textit{}\newline
Expression désignant un être humain, elle fut attribuée au prophète Ezéchiel près de cent fois. A de nombreuses reprises, Jésus-Christ se nomma lui-même «~Fils de l'homme~» afin de souligner sa nature humaine. Voir \vref{Ez. 2:1}~; \vref{Ez. 3:10}~; \vref{Ez. 4:1}~; \vref{Mc. 14:62}~; \vref{Jn. 5:27}~; \vref{Ro. 8:3} et \vref{Ph. 2:5-7}.

\DicoEntry{FIN DES TEMPS}\textit{, du grec «~eschatos~»~: «~extrême, dernier, fin~» et «~chronos~»~: «~temps, date, siècles~»}\newline
Appelée aussi derniers jours, période précédant la fin du monde*. Elle a commencé à l'effusion du Saint-Esprit selon la prophétie de Joël. La fin des temps est caractérisée d'un côté par des manifestations extraordinaires de l'Esprit de Dieu et l'annonce de l'Evangile à tous les peuples~; de l'autre par la séduction, l'apostasie* et le péché dans des dimensions jamais atteintes auparavant. Voir \vref{Joë. 2:28-29}~; \vref{Mt. 24:3-14}~; \vref{Ac. 2:16-18}~; \vref{1 Ti. 4:1} et \vref{2 Ti. 3:1-5}.

\DicoEntry{FIN DU MONDE}\textit{, du grec «~eschatos~»~: «~extrême, dernier, fin~» et «~aion~»~: «~monde, univers, période de temps~»}\newline
Cet événement correspond à la fin de notre ère. Après le jugement dernier, les impies iront dans l'étang de feu*, tandis que la Nouvelle Jérusalem accueillera les saints~; la terre sera détruite. Voir \vref{Mt. 13:36-43}~; \vref{2 Pi. 3:10-13}~; \vref{Ap. 20:11-15} et \vref{Ap. 21}.

\DicoEntry{FOI}\textit{, du grec «~pistis~»~: «~conviction de la vérité~»}\newline
Confiance en la véracité de Dieu, ses paroles et l'accomplissement de ses promesses. Bien qu'il n'existe qu'une seule foi, elle est présentée sous trois formes principales sous la Nouvelle Alliance~:
\\1. en tant que fruit de l'esprit*, c'est la foi qui sauve (\vref{Ga. 5:22} et \vref{Ro. 10:9})
\\2. en tant que don de l'Esprit,* c'est la foi accordée pour accomplir une tâche particulière (\vref{1 Co. 12:9})
\\3. en tant que Parole, c'est la foi liée à la saine doctrine, la vérité (\vref{Ro. 10:17} et \vref{2 Ti. 4:7})
\\Condition essentielle pour être agréable à Dieu~; la foi est éprouvée tout au long de la vie du croyant. Voir \vref{Lu. 7:50}~; \vref{Hé. 11} et \vref{1 Pi. 1:7}.

\DicoEntry{FORNICATION ou IMPUDICITÉ}\textit{, du grec «~pœrneia~»~: «~relation sexuelle illicite~»}\newline
Tous les rapports sexuels condamnés par la Parole, voir \vref{Lé. 18}~; \vref{1 Co. 6:13,16-18} et \vref{1 Co. 7:2}.

\DicoEntry{FRUIT DE L'ESPRIT}\textit{}\newline
Résultat de l'action de l'Esprit Saint dans l'homme intérieur dans le but de communiquer le caractère de Yahweh au chrétien né d'en haut. Voir \vref{Ga. 5:22}.

\DicoEntry{GABRIEL}\textit{, de l'hébreu «~Gabriy'el~»~: «~héros de Dieu~» ou «~homme de Dieu~»}\newline
Archange* que Dieu envoya pour délivrer des messages, notamment à Daniel, Zacharie et Marie. Voir \vref{Da. 9:21-27}~; \vref{Lu. 1:11-20} et \vref{Lu. 1:26-38}.

\DicoEntry{GAD}\textit{, de l'hébreu «~Gad~»~: «~bonheur~», «~heureux~», «~troupe~»}\newline
Fils de Jacob et Zilpa, servante de Léa, il devint l'ancêtre de la tribu de Gad. Voir \vref{Ge. 30:11} et \vref{Ge. 49:16}.

\DicoEntry{GALATIE}\textit{, du grec «~Galatia~»~: «~territoire des Gaulois, Gaule~»}\newline
Province antique de l'Asie Mineure, la Galatie se situait en Anatolie, dans l'actuelle Turquie autour d'Ankara. Elle devait son nom aux Galates, Celtes provenant des Balkans. Lors de son premier voyage missionnaire, Paul avait traversé cette région où plusieurs assemblées émergèrent. Il y revint plus tard pour fortifier les disciples et leur écrivit une lettre suite au trouble apporté par les judaïsants. Voir \vref{Ac. 16:6}~; \vref{Ac. 18:23} et \vref{Ga. 1-5}.

\DicoEntry{GALILÉE}\textit{, de l'hébreu «~Galiyl~»~: «~cercle, région, district~»}\newline
Région située au nord de la Palestine dans laquelle se trouve la localité de Nazareth où Jésus grandit. Il y commença son ministère, c'est aussi là qu'il se montra vivant à ses disciples après sa résurrection. Les disciples de Jésus étaient originaires de Galilée. Voir \vref{Mt. 2:19-23}~; \vref{Mc. 16:7}~; \vref{Jn. 2}~; \vref{Ac. 1:11} et \vref{Ac. 2:7}.

\DicoEntry{GARIZIM}\textit{, de l'hébreu «~Geriziym~»~: «~lieux arides~»}\newline
Montagne située au sud de Sichem, en face du mont Ebal, de laquelle les enfants d'Israël devaient prononcer la bénédiction* une fois entrés en Canaan. Voir \vref{De. 11:29}~; \vref{Jg. 9:7} et \vref{Jos. 8:33}.

\DicoEntry{GÉDÉON}\textit{, de l'hébreu «~Gid'own~»~: «~coupant, abattant~»}\newline
Issu de la tribu de Manassé et fils de Joas. Il fut mandaté pour délivrer Israël de la main des Madianites et fut juge en Israël pendant quarante ans. Voir \vref{Jg. 6-8}.

\DicoEntry{GÉHENNE}\textit{, du grec «~geena~»~: «~vallée de Hinnom~»}\newline
Initialement, vallée située au sud de Jérusalem où des enfants étaient jetés dans le feu en sacrifice à Moloc. Le terme «~géhenne~» représente la destruction future des méchants et se rapporte à l'étang de feu*. Voir \vref{2 R. 23:10} et \vref{Mt. 10:28}.

\DicoEntry{GENTILS}\textit{, du grec «~ethnos~»~: «~nations~», «~peuples~»}\newline
Dans les Ecritures, ce terme se rapportait initialement à tous ceux n'appartenant pas au peuple juif. Paul fut mandaté pour évangéliser les Gentils. A partir du IIIème siècle, le terme «~païen~» fut introduit dans le jargon chrétien pour désigner le «~non-chrétien~». Voir \vref{Mt. 18:17} et \vref{Ac. 26:17}.

\DicoEntry{GERME}\textit{, de l'hébreu «~tsemach~»~: «~pousse, croissance, branche~»}\newline
Terme désignant le Messie dans certains écrits prophétiques. Voir \vref{Es. 4:2}~; \vref{Jé. 23:5} et \vref{Za. 3:8}.

\DicoEntry{GLOIRE}\textit{, de l'hébreu «~kabhod~»~: «~poids~» ou «~kabowd~»~: «~gloire, honneur, richesse~»}\newline
La gloire se rapporte à ce qui a du poids, ce qui est lourd et écrasant – il est en effet difficile pour l'homme de supporter la splendeur et la magnificence de Yahweh. Image de sa sainteté, elle s'est manifestée dans un feu dévorant sur le mont Sinaï et fut révélée à Moïse au travers de la bonté et du nom de Dieu. Cette gloire sanctifie et génère de grands miracles~; elle est racontée par les cieux et toute la création. La gloire de Yahweh sera le luminaire de la Nouvelle Jérusalem. Elle invite à la crainte, la révérence, l'humilité, la louange~; lui seul mérite la gloire. Voir \vref{Ex. 16:10}~; \vref{Ex. 24:17}~; \vref{Ex. 29:43}~; \vref{Ex. 33:18-23}~; \vref{Es. 42:8}~; \vref{Es. 48:11}~; \vref{Ez. 44:4}~; \vref{Ps. 19:1}~; \vref{Pr. 15:33}~; \vref{2 Ch. 5:14}~; \vref{1 Th. 2:12} et \vref{Ap. 21:23}.

\DicoEntry{GOG}\textit{, de l'hébreu «~Gowg~»~: «~montagne~»}\newline
Très certainement le chef du pays de Magog. Voir \vref{Ez. 38} et \vref{Ap. 20:8}.

\DicoEntry{GOLGOTHA}\textit{, de l'araméen «~gulgoleth~»~: «~tête, crâne~»}\newline
Lieu de la crucifixion de Jésus-Christ, situé non loin de Jérusalem. Voir \vref{Jn. 19:17-20}.

\DicoEntry{GOMORRHE}\textit{, de l'hébreu «~Amorah~»~: «~submersion~»}\newline
Ville située dans la plaine du Jourdain. Après avoir atteint un haut degré de perversion et de débauche, elle fut détruite par Yahweh avec sa ville voisine, Sodome. Voir \vref{Ge. 13:10}~; \vref{Ge. 18:20-21} et \vref{Ge. 19:24}.

\DicoEntry{GRÂCE}\textit{, du grec «~charis~»~: «~bonne volonté~», «~bonté~», «~faveur~»}\newline
Don immérité de Dieu, elle est la source du salut* de tous les hommes et invite à la crainte de Dieu. La grâce est venue par Jésus-Christ et fut révélée au travers de l'œuvre parfaite de la croix*. Voir \vref{Jn. 1:17}~; \vref{Ro. 3:23-24}~; \vref{Tit. 2:11-12}.

\DicoEntry{GRAND PRÊTRE}\textit{}\newline
Voir PREMIER PRÊTRE.

\DicoEntry{GRANDE TRIBULATION}\textit{}\newline
Voir commentaire \vref{Ap. 7:14}.

\DicoEntry{GUILGAL}\textit{, de l'hébreu «~Gilgal~»~: «~action de rouler~»}\newline
Territoire situé à l'ouest du Jourdain et à l'est de Jéricho~; il fut le lieu de campement des Israélites après avoir passé le Jourdain à sec. Voir \vref{Jos. 4-5}.

\DicoEntry{HABAKUK}\textit{, de l'hébreu «~Chabaqquwq~»~: «~embrasser~», «~amour~»}\newline
Prophète de Yahweh qui exerça son ministère dans le royaume de Juda. L'ensemble de ses prophéties se trouve dans le livre éponyme.

\DicoEntry{HARMAGUEDON}\textit{, de l'hébreu «~Armageddon~»~: «~montagne de Méguiddo~»}\newline
Lieu situé au nord d'Israël dans la tribu de Zabulon. A la fin des temps, les rois et puissants de la terre s'y rassembleront pour combattre Yahweh et son armée. Voir \vref{2 R. 23:29} et \vref{Ap. 16:13-16}.

\DicoEntry{HÉBREU}\textit{, de l'hébreu «~`Ibriy~»~: «~qui provient de l'autre côté, qui traverse~»}\newline
Terme désignant les descendants d'Héber, fils de Schélach, de la postérité de Sem, dont est issu Abraham. Voir \vref{Ge. 11:10-32}~; \vref{Ge. 14:13-14} et \vref{Ex. 1:15-22}.

\DicoEntry{HELLÉNISTE}\textit{, du grec «~hellenistes~»~: «~celui qui adopte les manières et coutumes des Grecs~»}\newline
Israélites nés hors de la terre promise ayant adopté le mode de vie grec et parlant la langue grecque. Voir \vref{Ac. 6:1}.

\DicoEntry{HÉNOC}\textit{, de l'hébreu «~Chanowk~»~: «~consacré, dédié~»}\newline
Fils de Jéred et père de Metuschéla. Homme pieux ayant vécu trois cent soixante-cinq ans avant d'être enlevé au ciel sans connaître la mort. Voir \vref{Ge. 5:21-24} et \vref{Hé. 11:5}.

\DicoEntry{HÉRODE LE GRAND}\textit{, (73 av. J.-C.à 4 av. J.-C), du grec «~Herodes~»: «~héroïque~»}\newline
Roi de Judée, il fut l'instigateur du massacre des enfants de la région de Bethléhem au moment de la naissance de Jésus. Il mourut quand Jésus était encore enfant. Voir \vref{Mt. 2}.

\DicoEntry{HÉRODE ANTIPAS}\textit{, (ou le Tétrarque) (\ 21 av. J.-C. à 39 ap. J.-C.)}\newline
Fils d'Hérode le Grand*, il exerça la fonction de tétrarque* de Galilée et fut contemporain à Jésus-Christ pendant presque toute la vie de ce dernier. Hérode épousa sa belle-sœur Hérodias* et fit décapiter Jean-Baptiste. Il fut qualifié de «~renard~» par Jésus et s'accorda avec son ennemi Pilate lors de la crucifixion du Seigneur. Voir \vref{Mc. 6:14-28}~; \vref{Lu. 3:1}~; \vref{Lu. 13:31-32} et \vref{Lu. 23:8-12}.

\DicoEntry{HÉRODE AGRIPPA Ier}\textit{, (\ 10 av. J.-C. à 44 ap. J.-C)}\newline
Roi et tétrarque de Judée et petit fils du roi Hérode le Grand, il accéda au pouvoir à la genèse de l'Eglise primitive. Pour plaire aux Juifs, il fit mourir Jacques, fils de Zébédée, et emprisonna Pierre. Il mourut brusquement après avoir reçu du peuple la gloire qui devait revenir à Dieu. Voir \vref{Ac. 12}.

\DicoEntry{HÉRODE AGRIPPA II}\textit{, (\ 27 ap. J.-C. à 93 ap. J.-C.)}\newline
Fils d'Agrippa Ier, il est appelé «~roi Agrippa~» dans les Ecritures. Il fut inspecteur du temple de Jérusalem et avait le pouvoir de choisir les grands prêtres. Il rencontra Paul à Césarée lors d'une visite au gouverneur Festus*. Voir \vref{Ac. 25-26}.

\DicoEntry{HÉRODIAS}\textit{, du grec «~Herodias~»~: «~héroïque~»}\newline
Femme de Philippe I puis de son frère, Hérode le tétrarque. Elle commanda la décapitation de Jean-Baptiste. Voir \vref{Mc. 6:17-28}.

\DicoEntry{HOLOCAUSTE}\textit{, de l'hébreu «~'olah~»~: «~offrande entièrement consumée~»}\newline
Prescrit par la loi de Moïse, sacrifice consumé par le feu d'une agréable odeur à Yahweh. Il préfigurait le sacrifice à la croix de Jésus-Christ, l'Agneau de Dieu. Voir \vref{Lé. 1:1-17}~; \vref{Hé. 9:11-22} et \vref{Hé. 10:1-19}.

\DicoEntry{HOMOSEXUALITÉ}\textit{}\newline
Pratique abominable et fermement réprouvée par Dieu consistant en l'union de deux personnes du même sexe. Voir \vref{Lé. 18}~; \vref{1 Co. 6:9-10} et \vref{Ro. 1:24-32}.

\DicoEntry{HOSANNA}\textit{, de l'hébreu «~yasha'~»~: «~sauve~» et «~na´~»~: «~je te prie, maintenant~» et du grec «~hosanna~»~: «~sauve maintenant~!~»}\newline
Cri par lequel Jésus fut accueilli par la foule quand il entra à Jérusalem. Voir \vref{Mt. 21:9,15}~; \vref{Mc. 11:9-10} et \vref{Jn. 12:13}.

\DicoEntry{HULDA}\textit{, de l'hébreu «~chuldah~»~: «~belette, taupe~»}\newline
Femme de Schallum, prophétesse habitant à Jérusalem du temps de Josias, roi de Juda. Le roi chercha à consulter Yahweh au travers d'elle quand il découvrit le livre de la loi et les malheurs qui devaient suivre la désobéissance d'Israël. Voir \vref{2 R. 22:14-20} et \vref{2 Ch. 34:21-33}.

\DicoEntry{HYSOPE}\textit{, du grec «~hussopos~»~: «~hysope, branche d'hysope~»}\newline
Plante aromatique utilisée pour faire l'aspersion du sang ou d'eau sous l'Ancienne Alliance. C'est à l'aide d'une branche d'hysope qu'on présenta à Jésus une éponge trempée de vinaigre lors de sa crucifixion. Voir \vref{Ex. 12:22}~; \vref{Lé. 14:1-7}~; \vref{No. 19:18-19}~; \vref{Jn. 19:29} et \vref{Hé. 9:19}.

\DicoEntry{IDOLE, IDOLÂTRIE}\textit{, de l'hébreu «~gilluwl~»~: «~image~» et du grec «~eidolon~»~: «~image pour adorer~»}\newline
Une idole peut être l'image d'un faux dieu, l'image faussée de Yawheh ou encore une personne, un objet, une activité à qui l'on donne le rang de Dieu. L'idolâtrie - culte rendu à ces idoles – est fermement réprouvée dans la Parole. Voir \vref{Ex. 20:3-5}~; \vref{Ex. 32}~; \vref{1 R. 15:11-13}~; \vref{1 Co. 6:9}~; \vref{Ep. 5:5} et \vref{Col. 3:5}.

\DicoEntry{IMPOSITION DES MAINS}\textit{}\newline
Avant leur mort, les patriarches imposaient les mains à leurs enfants pour les bénir (\vref{Ge. 48:14}). Moïse imposa également les mains à Josué qui devait lui succéder (\vref{De. 34:9}). Sous la Nouvelle Alliance, on peut imposer les mains à quelqu'un en vue de lui transmettre la guérison divine, l'autorité liée à une fonction particulière, les dons spirituels et même le Saint-Esprit dans certains cas. Ce geste ne doit cependant pas être fait dans la précipitation. Voir \vref{Lu. 4:40}~; \vref{Mc. 16:18}~; \vref{Ac. 6:6}~; \vref{Ac. 8:17}~; \vref{1 Ti. 4:14} et \vref{1 Ti. 5:22}.

\DicoEntry{INCORRUPTIBILITÉ}\textit{, du grec «~aphtharsia~»~: «~perpétuité, pureté, sincérité~»}\newline
Terme désignant ce qui ne peut ni se corrompre, ni se flétrir, ni se détruire. A l'enlèvement de l'Eglise, les morts en Christ ressusciteront incorruptibles et les chrétiens revêtiront de même des corps incorruptibles. Voir \vref{Mt. 24:35} et \vref{1 Co. 15:40-57}.

\DicoEntry{INCREDULITÉ}\textit{, du grec «~apistia~»~: «~infidélité, sans foi, faiblesse dans la foi~»}\newline
Rejet, doute par rapport à la véracité de Dieu et de sa parole. Thomas fit preuve d'incrédulité quant à la résurrection de Christ avant de le voir vivant. Les incrédules ne peuvent pas hériter le Royaume de Dieu. Voir \vref{Jn. 1:1-14}~; \vref{Jn. 14:6}~; \vref{Jn. 20:24-29} et \vref{Ap. 21:8}.

\DicoEntry{INIQUITÉ}\textit{, du grec «~adikia~»~: «~injustice, tortuosité d'un cœur, violation volontaire de la loi~»}\newline
Tout ce qui constitue une violation de la loi* et la justice de Dieu. Voir \vref{Ro. 6:13}~; \vref{2 Pi. 2:13} et \vref{1 Jn. 5:17}.

\DicoEntry{MYSTEÈRE DE L'INIQUITÉ}\textit{}\newline
Voir commentaire \vref{2 Th. 2:7}.

\DicoEntry{INTERCESSION}\textit{, de l'hébreu «~palal~»~: «~intervenir, s'interposer, prier, agir en médiateur~»}\newline
Sous l'Ancienne Alliance, le grand prêtre avait la mission d'intercéder pour les péchés du peuple en offrant des sacrifices. A présent, Jésus-Christ le grand prêtre à perpétuité et l'avocat intercède pour ses enfants après s'être offert en sacrifice pour les péchés de l'humanité. Les hommes peuvent aussi faire des prières d'intercession comme Abraham pour Lot, Moïse pour Marie et l'Eglise pour tous les hommes. Voir \vref{Ge. 18:16-33}~; \vref{Lé. 16}~; \vref{No. 12:10-15}~; \vref{1 Ti. 2:1}~; \vref{Hé. 9:11-15} et \vref{1 Jn. 2:1}.

\DicoEntry{ISAAC}\textit{, de l'hébreu «~Yitschaq~»~: «~il rit~»}\newline
Fils de la promesse qui naquit à Abraham et Sara dans leur vieillesse. Il fut épargné quand Yahweh demanda à Abraham de lui offrir son fils en sacrifice. Isaac épousa Rébecca avec qui il eut deux fils~: Esaü et Jacob. Voir \vref{Ge. 17:17-21}~; \vref{Ge. 22:1-13} et \vref{Ge. 25:19-26}.

\DicoEntry{ISAÏ}\textit{, de l'hébreu «~Yishay~»~: «~je possède~»}\newline
Bethléhémite, petit-fils de Boaz et de Ruth, fils d'Obed et père de David. Voir \vref{Ru. 4:13-22}.

\DicoEntry{ISMAËL}\textit{, de l'hébreu «~Yishma'e'l~»~: «~Dieu entend~»}\newline
Fils d'Abraham et d'Agar, servante de Sara. Béni par Yahweh même après avoir été chassé avec sa mère par Sara, il devint le père des douze tribus ismaélites. Voir \vref{Ge. 16} et \vref{Ge. 25:12-16}.

\DicoEntry{ISSACAR}\textit{, de l'hébreu «~Yissaskar~»~: «~il donnera un salaire~»}\newline
Fils de Jacob et Léa, il devint l'ancêtre de la tribu d'Isaacar. Voir \vref{Ge. 30:18} et \vref{Ge. 49:14}.

\DicoEntry{ISRAËL}\textit{, de l'hébreu~: «~Yisra'el~»~: «~Dieu prévaut~»}\newline
Nom que Dieu donna à Jacob* après avoir lutté avec lui. Il s'agit également du nom désignant le peuple issu des douze fils de Jacob et le territoire que Dieu leur donna en héritage dont Jérusalem était la capitale. Après le schisme*, Israël se rapportait au royaume du nord composé de dix tribus. Voir \vref{Ge. 32:28}~; \vref{De. 33:5} et \vref{1 R. 12:1-24}.

\DicoEntry{IVRAIE}\textit{, du grec «~zizanion~»~: «~ivraie, ressemblant au blé, mais avec des grains noirs~»}\newline
Comme le blé, l'ivraie est une plante de la famille des graminées, mais c'est une mauvaise semence qui étouffe le blé. Elle représente les enfants du diable qui s'introduisent discrètement parmi les enfants de Dieu et qui en seront séparés uniquement à la fin du monde* pour aller vers la damnation éternelle. Voir \vref{Mt. 13:24-30,36-42}.

\DicoEntry{JACOB}\textit{, de l'hébreu «~Ya`aqob~»~: «~celui qui prend par le talon~» ou «~qui supplante~»}\newline
Fils d'Isaac et de Rebecca et frère jumeau d'Esaü. Il usa de stratagèmes pour ravir le droit d'aînesse ainsi que la bénédiction qui devaient revenir à son frère Esaü. Après avoir fui ce dernier, il se réfugia chez son oncle Laban dont il épousa les deux filles~: Léa et Rachel. De retour en Canaan après plusieurs années, Yahweh le rencontra en chemin et changea son nom en Israël. Jacob eut douze fils qui formèrent par la suite la nation d'Israël. Voir \vref{Ge. 25:21-34}~; \vref{Ge. 27-28}~; \vref{Ge. 29:1-30} et \vref{Ge. 49:1-28}.

\DicoEntry{JACQUES}\textit{, de l'hébreu~: «~Iakob~»~: «~qui supplante~» (variante de Jacob)}\newline
1. Fils de Zébédée et frère de Jean. Un des douze apôtres. Le roi Hérode Agrippa Ier* le fit mourir par l'épée. Voir \vref{Mt. 4:21-22}~; \vref{Lu. 6:12-16}~; \vref{Mc. 9:2-8}~; \vref{Mc. 14:32-33} et \vref{Ac. 12:1-2}.
\\2. Fils d'Alphée, un des douze apôtres~; il était aussi appelé Jacques le mineur. Voir \vref{Mt. 10:1-4}~; \vref{Mc. 15:40} et \vref{Lu. 6:12-16}.
\\3. Frère du Seigneur et apôtre, auteur de l'épître de Jacques. Voir \vref{Ac. 15:13-21}~; \vref{Ga. 1:19} et \vref{Mc. 6:3}.
\\4. Père de Jude, l'apôtre. Voir \vref{Lu. 6:16} et \vref{Ac. 1:13}.

\DicoEntry{JAPHET}\textit{, de l'hébreu «~Yepheth~»~: «~ouvert~», «~qui s'étend~»}\newline
Dernier des trois fils de Noé. Voir \vref{Ge. 10:1}.

\DicoEntry{JEAN}\textit{, de l'hébreu «~Yowchanan~» et du grec «~Ioannes~»~: «~Yahweh a fait grace~»}\newline
1. Fils de Zébédée, frère de Jacques et disciple aimé du Seigneur. Jean fut l'auteur de l'évangile éponyme, des trois épîtres qui portent son nom et de l'Apocalypse. Voir \vref{Mt. 10:2} et \vref{Jn. 13:23}.
\\2. Fils de Zacharie et Elisabeth, cousin de Jésus. Plus connu sous le nom de Jean-Baptiste, il fut envoyé pour préparer le chemin du Seigneur. Il fut décapité par Hérode Antipas*. Voir \vref{Lu. 1}~; \vref{Mal. 3:1-6}~; \vref{Mt. 1:12}~; \vref{Lu. 7:28} et \vref{Mt. 14:1-12}.

\DicoEntry{JELEK}\textit{, de l'hébreu «~yekeq~»~: «~jeune sauterelle~»}\newline
Désignant les sauterelles, il est souvent employé dans les Ecritures pour symboliser un grand nombre ou le dévoreur que Dieu envoie. Voir \vref{Joë. 1:4} et \vref{Na. 3:15-16}.

\DicoEntry{JÉRÉMIE}\textit{, de l'hébreu «~Yirmeyah~»~: «~celui que Yahweh a désigné~»}\newline
Fils de Hilkija, issu d'une famille de prêtres. Prophète de Yahweh, Jérémie fut appelé dès son plus jeune âge et exerça un ministère prophétique avant et pendant les premières années de déportation. Appelé à être eunuque*, il ne se maria jamais et n'eut point d'enfant. Il fut l'auteur des livres Jérémie et Lamentations de Jérémie.

\DicoEntry{JÉRICHO}\textit{, de l'hébreu «~Yeriychow~»~: «~ville de la lune~» ou «~ville des palmiers~»}\newline
Ville située à l'est de la tribu de Benjamin, près des rives du Jourdain. A la sortie du désert, les espions hébreux y furent cachés par Rahab la prostituée~; Jéricho fut ensuite détruite et livrée miraculeusement entre les mains d'Israël. C'est à Jéricho que Jésus guérit l'aveugle Bartimée et fut reçu par Zachée. Voir \vref{Jos. 2,6}~; \vref{Mc. 10:46-53} et \vref{Lu. 19:1-10}.

\DicoEntry{JÉROBOAM}\textit{, de l'hébreu «~Yarob'am~»~: «~le peuple devient nombreux~»}\newline
Fils de Nebath et serviteur de Salomon, il devint plus tard son ennemi. Après le schisme, il fut le premier roi du royaume du nord sur lequel il régna vingt-deux ans. Il fut une occasion de chute pour le peuple qu'il plongea dans l'idolâtrie*. Voir \vref{1 R. 11:26-40}~; \vref{1 R. 12-13}.

\DicoEntry{JÉRUSALEM}\textit{, de l'hébreu «~Yeruwshalem~»~: «~fondement de la paix~»}\newline
Ville située en Palestine, au nord de la Judée. Lors de la conquête de Canaan, la ville fut sous le contrôle des Jébusiens. Aux environs du Xe siècle av. J.-C., David reprit la ville alors devenue forteresse jébusienne. Il en fit la capitale politique et religieuse du royaume en y faisant établir l'arche de l'alliance. Salomon y construisit le temple* sur le mont Morija. En 586 av. J.-C., bien après le schisme*, les Babyloniens la détruisirent. Elle fut rebâtie par Néhémie après le retour de la captivité babylonienne. Jésus-Christ se lamenta sur la ville à cause de son incrédulité* et y annonça sa future destruction. Jérusalem fut en effet détruite par le général romain Titus en 70 ap. J.-C puis de nouveau rebâtie. Lors de son retour glorieux, le Seigneur Jésus posera ses pieds sur le Mont des Oliviers* qui est situé à Jérusalem. Le livre d'Apocalypse annonce après la fin du monde l'apparition de la Nouvelle Jérusalem, cité céleste. Voir \vref{2 S. 5:6-9}~; \vref{2 S. 6}~; \vref{Za. 14:1-4}~; \vref{2 Ch. 3:1}~; \vref{Lu. 19:41-44}~; et \vref{Ap. 21:2}.

\DicoEntry{JÉSUS}\textit{, de l'hébreu «~Yehowshuwa~»~: «~Yahweh est salut~»}\newline
Fils de l'homme* et fils de Dieu*, Jésus est le Dieu vivant manifesté en chair. Il fut conçu dans le ventre de Marie par la puissance du Saint-Esprit alors que cette dernière n'avait point connu d'homme. Selon les recherches de l'historien Flavius Josèphe, sa date de naissance se situerait autour de l'an 6 av. J.-C. - l'an zéro n'étant qu'une indication approximative. Fils adoptif de Joseph le charpentier et cousin de Jean Baptiste, il vécut la plus grande partie de sa vie en Galilée, dans la ville de Nazareth. Vers l'âge de 30 ans, il se fit baptiser dans le Jourdain et commença par la suite son ministère public. Grâce à sa vie exemplaire sans péché, il put se présenter comme une offrande agréable à Dieu répondant aux exigences de la justice divine pour sauver le monde. Remplissant toutes les prophéties relatives au Messie*, il fut trahi par un de ses disciples, Judas Iscariot*. Arrêté, maltraité puis crucifié, il mourut portant le poids des péchés de l'humanité, mais il ressuscita le troisième jour. Le salut réside dans la foi en son nom. Vivant de toute éternité, Jésus est le Dieu véritable et la vie éternelle. Sa justice ne tardera pas à se manifester~: il revient à toute vitesse. Voir \vref{Es. 53}~; \vref{Mt. 1:18-25}~; \vref{Mt. 2:23}~; \vref{Mc. 2:28}~; \vref{Lu. 1:36}~; \vref{Lu. 6:16}~; \vref{Lu. 24:46}~; \vref{Jn. 1:34}~; \vref{Ac. 4:12}~; \vref{2 Co. 5:21}~; \vref{1 Ti. 3:16}~; \vref{1 Pi. 2:21-25}~; \vref{1 Jn. 5:20} et \vref{Ap. 22:20}.

\DicoEntry{JETHRO}\textit{, de l'hébreu «~Yithrow~»~: «~son abondance, excellence~»}\newline
Prêtre de Madian chez qui Moise se réfugia après avoir fui l'Egypte. Il donna sa fille Séphora* pour femme à Moïse*. Voir \vref{Ex. 2:15-21}.

\DicoEntry{JEÛNE}\textit{, de l'hébreu «~tsuwn~»~: «~s'abstenir de nourriture~» et du grec «~nesteia~»~: «~le jeûne, un exercice volontaire et religieux~»}\newline
Privation totale ou partielle de nourriture dans le but d'humilier sa chair et d'adresser à Dieu des prières spécifiques. Le jeûne doit être exempt de toute hypocrisie et accompagné d'actes de justice pour être agréé de Dieu. Voir \vref{Es. 58}~; \vref{Est. 4:16}~; \vref{Da. 10}~; \vref{Mt. 6:16-18}~; \vref{Lu. 2:37}.

\DicoEntry{JÉZABEL}\textit{, de l'hébreu «~'Iyzebel~»~: «~Baal est l'époux~» ou «~l'impudique~»}\newline
Fille d'Ethbaal, roi de Sidon, et femme d'Achab roi d'Israël, elle extermina les prophètes de Yahweh et accueillait huit cent cinquante faux prophètes à sa table. Elle conduisit le peuple d'Israël dans l'idolâtrie au temps d'Elie*. Jézabel est associée à l'esprit du même nom qui prolifère de faux enseignements et entraîne le peuple de Dieu dans l'impudicité. Voir \vref{1 R. 16:31}, \vref{1 R. 18:4,19} et \vref{Ap. 2:20}.

\DicoEntry{JOB}\textit{, de l'hébreu «~'Iyowb~»~: «~haï, ennemi~» ou «~Je m'exclamerai~»}\newline
Originaire du pays d'Uts, homme prospère dont Yahweh témoigna l'intégrité et la droiture. Il subit en très peu de temps une succession de malheurs que Dieu permit pour se révéler à lui. Son histoire est racontée dans le livre portant son nom.

\DicoEntry{JOËL}\textit{, de l'hébreu «~Yow'el~»~: «~Yahweh est Dieu~»}\newline
Fils de Pethuel, il exerça la fonction de prophète dans le royaume de Juda. Il annonça la venue du Saint-Esprit sur toute chair à la fin des temps. Le contenu de son message se trouve dans le livre éponyme.

\DicoEntry{JONAS}\textit{, de l'hébreu «~Yonah~»~: «~colombe~»}\newline
Prophète de Yahweh envoyé à Ninive pour leur annoncer la destruction de leur ville. Son refus d'obéir à Dieu le conduisit dans le ventre d'un grand poisson. Son histoire est racontée dans le livre portant son nom.

\DicoEntry{JONATHAN}\textit{, de l'hébreu «~Yehownathan~»~: «~Yahweh a donné~»}\newline
Fils du roi Saül, homme de guerre reconnu. Lié à David par une très forte amitié, il protégea plusieurs fois ce dernier des relents meurtriers de son père. Il mourut à la bataille de Guilboa avec son père et ses frères. Voir \vref{1 S. 14:1-15}, \vref{1 S. 18:1-4}~; \vref{1 S. 19:1-8}~; \vref{1 S. 20} et \vref{1 S. 31:1-2}.

\DicoEntry{JOSAPHAT}\textit{, de l'hébreu «~Yehowshaphat~»~: «~Yahweh a jugé~»}\newline
Fils d'Asa et d'Azuba, il fut roi de Juda pendant vingt-cinq ans. Il eut un règne prospère et fit ce qui est droit aux yeux de Yahweh. Voir \vref{1 R. 15:24}~; \vref{1 R. 22:41-46} et \vref{2 Ch. 17}.

\DicoEntry{JOSEPH}\textit{, de l'hébreu «~Yowceph~»~: «~que Yahweh ajoute~» ou «~il enlève~»}\newline
1. Fils de Jacob et Rachel. Vendu comme esclave par ses frères, il devint, après plusieurs années de prison, gouverneur d'Egypte. Ses fils, Ephraïm et Manasée, furent adoptés par son père Jacob et furent les pères de deux des douze tribus d'Israël. Voir \vref{Ge. 30:22-24}~; \vref{Ge. 37,39,40,45,46}~; \vref{Ge. 48:5} et \vref{Jos. 14:4}.
\\2. Fils d'Héli, charpentier originaire de la tribu de Juda. Epoux de Marie, la mère de Jésus. Voir \vref{Mt. 1:18-25} et \vref{Mt. 13:55}.

\DicoEntry{JOSIAS}\textit{, de l'hébreu «~Yo'shiyah~»~: «~Yahweh guérit~»}\newline
Fils d'Amon, il devint roi de Juda à huit ans et y régna durant trente et un ans. Grand réformateur, à l'origine d'un grand réveil spirituel, il répara le temple, purifia le royaume des idoles et conclut une alliance de fidélité envers Yahweh. Voir \vref{2 R. 22-23}.

\DicoEntry{JOSUÉ}\textit{, de l'hébreu «~Yehowshuwa`~»~: «~Yahweh est salut~»}\newline
Fils de Nun de la tribu d'Ephraïm, choisi par Dieu pour succéder à Moïse. Accompagné de la puissante main de Dieu, il conduisit Israël à entrer en possession de Canaan. Son histoire se trouve dans le livre portant son nom.

\DicoEntry{JOURDAIN}\textit{, de l'hébreu «~Yarden~»~: «~celui qui descend~»}\newline
Très certainement le fleuve le plus connu des Ecritures, il est situé aux limites est de l'actuel territoire d'Israël. Josué et le peuple d'Israël passèrent le fleuve à sec. De même, Elie, puis Elisée, partagèrent les eaux du fleuve en deux. Après s'y être baigné sept fois sur les conseils d'Elisée, Naaman fut guéri de la lèpre. Jésus se fit baptiser par Jean dans le Jourdain. Voir \vref{Jos. 3}~; \vref{2 R. 2:8,12-14}~; \vref{2 R. 5:10-14} et \vref{Mt. 3:13-17}.

\DicoEntry{JOUR DU SEIGNEUR}\textit{}\newline
Jour où Yahweh manifestera sa justice et frappera les nations à cause de leurs péchés. Ce jour arrivera comme un voleur et surprendra beaucoup. Voir \vref{Es. 13:6-16}~; \vref{So. 1}~; \vref{2 Pi. 3:10}.

\DicoEntry{JUDA}\textit{, de l'hébreu «~Yehuwdah~»~: «~qu'il (Dieu) soit loué~»}\newline
Fils de Jacob et Léa, il est le père de la tribu du même nom installée au sud de Canaan. Sa descendance reçut la prédominance et la royauté~; David et Jésus-Christ étaient issus de cette tribu. Après le schisme*, Juda désigna aussi le nom du royaume du sud composé des tribus de Juda et Benjamin. Voir \vref{Ge. 29:35}~; \vref{Ge. 49:8-12}~; \vref{Jos. 15:1-12}~; \vref{1 R. 12:16-24}~; \vref{Mt. 1:1-16}.

\DicoEntry{JUDAS ISCARIOT}\textit{, de l'hébreu «~Yehuwdah~»~: «~qu'il (Dieu) soit loué~»}\newline
Fils de Simon Iscariot, il fut un des douze disciples de Jésus-Christ et était chargé de la trésorerie. Il trahit le Seigneur, ce qu'il regretta amèrement et le poussa à se suicider. Voir \vref{Mt. 26:14-16}~; \vref{Mt. 27:3-5}~; \vref{Lu. 6:16} et \vref{Jn. 12:4-6}.

\DicoEntry{JUDE}\textit{, de l'hébreu «~Yehuwdah~»~: «~qu'il (Dieu) soit loué~»}\newline
1. Fils de Jacques, un des douze apôtres, connu également sous le nom de Thadée. Voir \vref{Mc. 3:18} et \vref{Lu. 6:16}.
\\2. Prophète également appelé Barsabas, compagnon d'œuvre de Silas. Voir \vref{Ac. 15:22,32}.
\\3. Frère du Seigneur, auteur d'une épître qui porte son nom. Voir \vref{Mt. 13:55}~; \vref{Mc. 6:3} et \vref{Jud. 1:1}.

\DicoEntry{JUDÉE}\textit{, de l'hébreu «~Yehuwdah~»~: «~qu'il (Dieu) soit loué~»}\newline
Région située au sud de la Palestine où se trouvent notamment Jérusalem et Bethléhem. Elle correspondrait approximativement au territoire de l'ancien royaume de Juda. Ce terme n'est pas utilisé dans le Tanakh. Voir \vref{Mt. 2:1}~; \vref{Mc. 1:5} et \vref{Ga. 1:22}.

\DicoEntry{JUGE, JUGEMENT}\textit{, de l'hébreu «~shaphat~»~: «~juger, gouverner, défendre, punir~», «~agir comme un législateur, juge ou gouverneur~», «~exécuter un jugement~»}\newline
Dans toute la Parole, Yahweh est présenté comme le juge droit et incorruptible. Après la sortie d'Egypte, des juges ont été suscités par Dieu au milieu d'Israël pour délivrer le peuple de ses ennemis et le ramener vers lui (voir livre des Juges). Le Seigneur a toujours envoyé des prophètes pour annoncer ses jugements et ses décisions ainsi que des juges pour faire respecter sa loi. Sous la Nouvelle Alliance, l'homme spirituel est appelé à juger (discerner selon la Parole), mais condamner et décider du sort final d'une personne demeure la prérogative de Dieu. Yahweh est en effet le juste juge qui siège et tranche non seulement au tribunal de Christ, mais également au jugement dernier. Voir \vref{Ge. 18:25}~; \vref{Jé. 11:20}~; \vref{2 Co. 5:10} et \vref{Ap. 20:11-15}.

\DicoEntry{JUPITER}\textit{, du grec «~Zeus~»~: «~un père des secours~»}\newline
Divinité romaine assimilée à Zeus chez les Grecs. Lors d'une guérison miraculeuse à Lystres, la foule pensa voir en Paul la réincarnation de Mercure et en Barnabas celle de Jupiter. Pour cela, on voulut les adorer, ce qu'ils refusèrent avec véhémence. Voir \vref{Ac. 14:8-15}.

\DicoEntry{JUSTE, JUSTICE}\textit{, de l'hébreu «~tsedeq~»~: «~droiture, exactitude, conforme~» ou encore «~tsadiq~»~: «~juste, exact, innocent~» et du grec «~dikaiosune~»~: «~la condition acceptable par Dieu~» ou «~intégrité, vertu, pureté de vie, droiture~»}\newline
En qualité de juste juge, Yahweh a toujours recherché cette qualité chez l'homme, mais il ne l'a pas trouvé déclarant que nul n'est juste. Au travers de l'œuvre de la croix et avec l'aide du Saint-Esprit, le chrétien peut à présent marcher dans la justice de Dieu. Il est appelé à la rechercher plus que tout et à devenir esclave de la justice. Voir \vref{Mt. 5-7}~; \vref{Lu. 1:75}~; \vref{Ro. 3:10}~; \vref{Ro. 6:18} et \vref{2 Ti. 4:8}.

\DicoEntry{JUSTIFICATION}\textit{, du grec «~dikaiosis~»~: «~état du juste~»}\newline
Au travers de l'œuvre de la croix, Jésus-Christ est devenu la justification de tous ceux qui croient en lui, les rendant acceptables et libres de toute culpabilité. Voir \vref{Ro. 3:23-28}~; \vref{Ro. 4:25} et \vref{Ro. 5:18}.

\DicoEntry{KORÉ}\textit{, de l'hébreu «~Qorach~»~: «~chauve~»}\newline
Fils de Jitsehar, originaire de la tribu de Lévi, il se révolta avec Dathan* et Abiram* contre Moïse* et Aaron*. Suite à sa rébellion, il périt avec les gens de sa maison. Voir \vref{No. 16:1-35}.

\DicoEntry{LAÏC}\textit{, du grec «~laos~»~: «~peuple~»}\newline
Notion propre à l'Eglise catholique romaine. Opposé au clergé*, les laïcs sont les autres membres de l'église, ceux qui n'ont pas de fonction dirigeante, mais qui sont tout de même appelés à honorer Dieu dans leur vie et faire connaître leur foi au milieu du monde.

\DicoEntry{LANGUES}\textit{, de l'hébreu «~lashown~»~: «~langue, langage~» et du grec «~glossa~»~: «~la langue~» ou «~le langage d'un peuple particulier~»}\newline
Selon les Ecritures, les langues sont nées à Babylone* lorsque les hommes se sont rebellés contre la volonté de Yahweh et que ce dernier a confondu leur langage dans le but de les disperser. Dans la Parole sont cités différents types de langues, chacune liée à un don ou une manifestation particulière de l'Esprit de Dieu. Lors de l'effusion du Saint-Esprit à la Pentecôte, les disciples reçurent la capacité de parler des merveilles de Dieu dans des langues étrangères. Il s'agit du don spirituel* appelé la diversité des langues et concerne uniquement les langues usuelles. Il existe également des langues angéliques ou dites inconnues que le croyant peut utiliser pour s'adresser à Dieu. Les langues étrangères tout comme les langues des anges peuvent donner lieu à une interprétation, c'est ce qu'on appelle le don d'interpréter les langues. Voir \vref{Ge. 11}~; \vref{Ac. 2:1-11}~; \vref{1 Co. 12:10}~; \vref{1 Co. 13:1} et \vref{1 Co. 14:1-14,26-27}.

\DicoEntry{LAODICÉE}\textit{, du grec «~Laodikeia~»~: «~justice du peuple~»}\newline
Capitale de la Phrygie, l'une des provinces de l'Asie Mineure, réputée dans le domaine du commerce, notamment dans l'industrie textile. Ses vêtements et sa tapisserie principalement de couleur noire, firent sa renommée. Elle possédait une grande école de médecine qui fabriquait des remèdes réputés pour les yeux, notamment le fameux collyre. L'église de Laodicée est la dernière à qui fut adressée une lettre dans l'Apocalypse. Caractérisée par la tiédeur, l'affection aux choses terrestres et l'aveuglement spirituel, le Seigneur l'appela à la repentance*. Elle est l'image de l'église matérialiste. Voir \vref{Ap. 3:14-22}.

\DicoEntry{LAZARE}\textit{, du grec «~Lazaros~»~: «~Yahweh a secouru~»}\newline
1. Homme pauvre qui fut recueilli dans le sein d'Abraham après sa mort. Voir \vref{Lu. 16:19-21}.
\\2. Frère de Marthe et de Marie de Béthanie, et ami de Jésus-Christ qui le ressuscita des morts. Voir \vref{Jn. 11}.

\DicoEntry{LÉA}\textit{, de l'hébreu «~Le'ah~»~: «~lasse~»}\newline
Fille aînée de Laban et première femme de Jacob. Elle enfanta six fils, pères de six des douze tribus d'Israël (Ruben, Siméon, Lévi, Juda, Issacar et Zabulon) ainsi qu'une fille nommée Dina. Voir \vref{Ge. 29:16-23}~; \vref{Ge. 30:21} et \vref{Ge. 35:23}.

\DicoEntry{LÉMEC}\textit{, de l'hébreu «~Lemek~»~: «~puissant~»}\newline
Fils de Metuschaël et descendant de Caïn, il fut le premier polygame de l'histoire en prenant deux femmes~: Ada et Tsilla. Voir \vref{Ge. 4:16-24}.

\DicoEntry{LÈPRE}\textit{, de l'hébreu «~tsara'~»~: «~être morbide de peau~»}\newline
Commune en Egypte et en orient, maladie de la peau dont le virus peut se développer dans tout le corps. Contagieuse, elle peut même souiller les vêtements et les habitations. Sous la loi mosaïque, les personnes atteintes de cette maladie étaient considérées comme impures et devaient se tenir à l'écart. Durant son ministère, Jésus guérit plusieurs lépreux. Voir \vref{Lé. 13-14} et \vref{Lu. 17:11-14}.

\DicoEntry{LEVAIN}\textit{, de l'hébreu «~chametz~»~: «~ce qui est levé~»}\newline
Symbole du mal et de la corruption, le levain était interdit dans la quasi-totalité des offrandes. Jésus a assimilé le levain des pharisiens à l'hypocrisie, à la doctrine erronée. Les chrétiens sont appelés à faire disparaître le vieux levain et à devenir le levain du monde en y faisant progresser l'évangile du royaume. Voir \vref{Lé. 2:11}~; \vref{Mt. 16:6-12}~; \vref{Mt. 13:33} et \vref{1 Co. 5:6-7}.

\DicoEntry{LÉVI, LÉVITES}\textit{, de l'hébreu «~Leviy~»~: «~attachement~»}\newline
Fils de Jacob et Léa, Lévi participa avec son frère Siméon au massacre des hommes de la ville de Sichem après le viol de leur sœur Dina. Consacrés au service de Yahweh, ses descendants, les Lévites, n'eurent point d'héritage en Canaan, mais habitèrent différentes villes qui leur furent spécifiquement attribuées en Israël. Voir \vref{Ge. 29:34}~; \vref{Ge. 34}~; \vref{No. 18:20-24} et \vref{Jos. 13:14}.

\DicoEntry{LOI}\textit{, de l'hébreu «~towrah~»~: «~loi, direction, commandement~», «~loi mosaïque~»}\newline
L'ensemble des préceptes et ordonnances relatifs à l'alliance conclue entre Yahweh et le peuple hébreu, par l'intermédiaire de Moïse, est contenu dans les cinq premiers livres de la Bible appelée aussi «~le Pentateuque~». Selon la tradition juive, il existerait 613 commandements relatifs à la moralité, la vie en société et le culte rendu à Yahweh. L'homme en étant incapable, Jésus-Christ a accompli les exigences de la loi. Il est donc possible aux hommes d'obtenir le salut par la foi et non plus par les œuvres. La loi est maintenant gravée dans les cœurs des enfants de Dieu à qui le Saint-Esprit rappelle les paroles de Jésus. Voir \vref{Ex. 18:20}~; \vref{Ex. 24:12}~; \vref{Jn. 14:26} et \vref{Ro. 3:19-31}.

\DicoEntry{LOI DU PÉCHÉ}\textit{, du grec «~nomos~»~: «~toute chose établie, une coutume, un commandement~»}\newline
Loi spirituelle inscrite dans la chair qui pousse l'homme charnel à se révolter contre Dieu en commettant le péché. Voir \vref{Ro. 7:13-25}.

\DicoEntry{LOT}\textit{, de l'hébreu~: «~Lowt~»~: «~voile, couverture~»}\newline
Fils de Haran et neveu d'Abraham, Lot quitta Ur avec ce dernier avant de s'en séparer. Grâce à l'intercession d'Abraham, il fut sauvé de la destruction de Sodome avec ses deux filles. Ces dernières enivrèrent leur père et eurent des relations incestueuses avec lui de qui naquirent Moab, père des Moabites, et Amon, père des Ammonites. Voir \vref{Ge. 11:31}~; \vref{Ge. 13:1-13}~; et \vref{Ge. 19}.

\DicoEntry{LUC}\textit{, du grec «~Loukas~»~: «~qui donne la lumière~»}\newline
Médecin de métier, il fut un des compagnons d'œuvre de Paul et l'auteur de l'évangile qui porte son nom et du livre Actes des Apôtres. Voir \vref{Col. 4:14} et \vref{Phm. 1:24}.

\DicoEntry{MACÉDOINE}\textit{, du grec «~Makedonia~»~: «~terre étendue~»}\newline
Province romaine située au nord de la Grèce. Paul y effectua quelques voyages missionnaires et y implanta plusieurs assemblées. Voir \vref{Ac. 16:9-12}~; \vref{Ac. 20:1-3}~; \vref{1 Co. 8:1} et \vref{2 Co. 11:9} et \vref{Ro. 15:23}.

\DicoEntry{MADIAN}\textit{, de l'hébreu «~Midyan~»~: «~lutte, dispute~»}\newline
Un des fils issu de l'union d'Abraham* et Ketura, il devint l'ancêtre des Madianites, peuple qui habita à l'est de Canaan et au nord du désert d'Arabie. Voir \vref{Ge. 25:1-2}~; \vref{No. 31:1-12} et \vref{Jg. 6:2}.

\DicoEntry{MAGOG}\textit{, de l'hébreu «~Magowg~»~: «~territoire de montagne, qui domine~»}\newline
Fils de Japhet. Associé à Gog, il correspond aussi à la nation d'où vient le roi Gog qui fera la guerre à Dieu et à son peuple juste avant le jugement dernier. Voir \vref{Ge. 10:2,9} et \vref{Ap. 20:8}.

\DicoEntry{MAIN}\textit{, de l'hébreu «~yad~»~: «~main, force, pouvoir~»}\newline
Partie du corps permettant de toucher, saisir ou posséder, elle représente aussi l'action, la provision, la protection ou le joug. Tout au long des Ecritures, la main de Yahweh révèle sa puissance et sa bienveillance. Voir \vref{Es. 40:2}~; \vref{Jé. 18:6}~; \vref{Ps. 71:4}~; \vref{Pr. 10:4}~; \vref{Mc. 14:58}~; \vref{Lu. 11:20} et \vref{Ac. 11:21}.

\DicoEntry{MALACHIE}\textit{, de l'hébreu «~Mal`akiy~»~: «~mon messager~»}\newline
Dernier prophète du Tanakh, il condamna les péchés et l'hypocrisie des enfants d'Israël et annonça la venue de Jean-Baptiste. L'ensemble de ses prophéties est contenu dans le livre portant son nom.

\DicoEntry{MALÉDICTION}\textit{, de l'hébreu «~arar, meerah, qelalah~» et du grec «~ara, katara~»}\newline
Parole attirant le malheur sur un bien, une personne ou un peuple. Dieu a le pouvoir de maudire et aussi d'écarter toute malédiction. La malédiction de Dieu, contraire de la bénédiction*, fait suite à la désobéissance. A la nouvelle naissance, toutes les chaînes de malédiction qui liaient le chrétien sont brisées. Le chrétien ne doit pas maudire, mais bénir en tout temps, même ses ennemis. Voir \vref{De. 28:15-68}~; \vref{Mt. 5:44}~; \vref{2 Co. 5:17} et \vref{Ro. 8:1}.

\DicoEntry{MALFAITEUR REPENTANT}\textit{}\newline
Un des hommes coupables qui fut crucifié à côté de Jésus. Son humilité, sa sincérité et sa repentance lui permirent d'accéder au salut, Jésus-Christ lui ayant garanti l'accès au paradis. Voir \vref{Lu. 23:33-43}.

\DicoEntry{MAMON}\textit{, du grec «~Mammonas~»~: «~richesses~»}\newline
Dieu de l'argent. Jésus utilisa ce terme pour personnifier la richesse que beaucoup idolâtrent et qui est par conséquent en concurrence avec Yahweh dans le cœur de certains. Voir \vref{Mt. 6:24}.

\DicoEntry{MANASSÉ}\textit{, de l'hébreu «~Menashsheh~»~: «~oublieux~»}\newline
1. Fils aîné de Joseph* et d'Asnath, adopté par Jacob avant sa mort, ancêtre de la tribu de Manassé. Voir \vref{Ge. 41:51}~; \vref{Ge. 48:5} et \vref{Jos. 14:4}.
\\2. Fils d'Ezéchias et de Hephtsiba, il fut l'un des pires rois du royaume de Juda qui régna 55 ans. Malgré le réveil impulsé par son père, il se détourna entièrement de Yahweh et servit des dieux étrangers. Voir \vref{2 R. 21:1-18}.

\DicoEntry{MANNE}\textit{, de l'hébreu «~man~»~: «~qu'est-ce que cela~?~»}\newline
Nourriture céleste - à l'aspect de la graine de coriandre et au goût de gâteau de miel - que Dieu donna quotidiennement aux Israélites durant toute leur marche dans le désert. \vref{Ex. 16:15,31-35}.

\DicoEntry{MARANATHA}\textit{, de l'araméen «~maran atha~»~: «~le Seigneur vient~» ou «~Seigneur, viens~»}\newline
Expression prononcée par Paul quand il s'adressa aux Corinthiens et qui doit également être le cri du cœur de tout enfant de Dieu. Voir \vref{1 Co. 16:22} et \vref{Ap. 22:17,20}.

\DicoEntry{MARC}\textit{, du grec «~Markos~»~: «~une défense~» ou «~grand marteau~»}\newline
Appelé aussi Jean, cousin de Barnabas, il fut la cause de la séparation de Paul et Barnabas. Il partit avec ce dernier à Chypre et devint par la suite un fidèle compagnon d'œuvre de Paul. Il écrivit l'évangile portant son nom. Voir \vref{Ac. 12:12}~; \vref{Ac. 15:36-39}~; \vref{Col. 4:10} et \vref{Phm. 24}.

\DicoEntry{MARDOCHÉE}\textit{, de l'hébreu «~Mordekay~»~: «~petit homme~»}\newline
Fils de Jaïr de la tribu de Benjamin*, il adopta Esther*, fille de son oncle. Il sauva la vie du roi Assuérus* en déjouant les plans de Bigthan et Théresch et préserva le peuple juif des desseins meurtriers d'Haman. Il devint puissant dans la maison du roi et instaura la fête du Purim. Voir le livre d'Esther.

\DicoEntry{MARIAGE}\textit{, de l'hébreu «~chathan~»~: «~devenir un gendre, s'allier~»}\newline
Bénédiction de Dieu, le mariage est une alliance en principe indissoluble entre un homme et une femme dans le but d'accomplir le plan de Dieu. Il doit être célébré dans le respect des autorités du pays dans lequel le couple se trouve et honoré de tous, particulièrement des parents dont la bénédiction est essentielle. Voir \vref{Ge. 2:22-24}~; \vref{Ge. 24:60}~; \vref{Pr. 18:22}~; \vref{1 Co. 7} et \vref{Hé. 13:4}. Voir commentaire en \vref{Mt. 19:6}.

\DicoEntry{MARIE}\textit{, de l'hébreu «~Miryam~»~: «~rébellion, obstination~»}\newline
1. Sœur de Moïse et d'Aaron, prophètesse. Elle se rebella contre Moïse et fut frappée par la lèpre, mais en guérit grâce à l'intercession de Moïse. Voir \vref{Ex. 15:20} et \vref{No. 12}.
\\2. Mère de Jésus~: Elle conçut, par la vertu du Saint-Esprit, Jésus homme. Elle devint une de ses disciples et se trouvait parmi ceux qui persévéraient dans la prière dans la chambre haute lors de l'effusion du Saint-Esprit promis. Voir \vref{Es. 7:14}~; \vref{Mt. 1:18-25}~; \vref{Mc. 15:40-41}~; \vref{Lu. 1:26-38} et \vref{Ac. 1:13-14}.
\\3. Marie de Magdala~: Elle fut délivrée de sept démons par Jésus qu'elle suivit pendant son ministère terrestre, et ce jusqu'à la croix. Elle fut mandatée par le Seigneur pour annoncer sa résurrection aux apôtres. Voir \vref{Mt. 27:55-56}~; \vref{Mc. 16:1-11}~; \vref{Lu. 8:2} et \vref{Jn. 20:1-18}.
\\4. Marie de Béthanie~: Sœur de Marthe* et de Lazare*, que Jésus ressuscita des morts. Contrairement à sa sœur, elle choisit la bonne part en restant aux pieds du Maître. Elle toucha le cœur de ce dernier en l'oignant d'un parfum de grand prix. Voir \vref{Lu. 10:38-42}~; \vref{Jn. 11:1-44} et \vref{Jn. 12:1-7}.

\DicoEntry{MARTHE}\textit{, du grec «~Martha~»~: «~maîtresse, dame~»}\newline
Sœur de Lazare* - dont elle fut témoin de la résurrection - et de Marie* de Béthanie, elle reçut Christ dans sa maison, mais ce dernier lui reprocha son activisme au détriment de l'écoute de sa Parole. Voir \vref{Lu. 10:38-42} et \vref{Jn. 11:1-44}.

\DicoEntry{MATTHIAS}\textit{, de l'hébreu «~Mattithyah~»~: «~don de Yahweh~»}\newline
Disciple de Jésus et témoin oculaire de son ministère, il fut désigné pour devenir l'un des douze apôtres en remplacement de Judas Iscariot* qui avait trahi le Seigneur pour ensuite se suicider. Voir \vref{Ac. 1:15-26}.

\DicoEntry{MATTHIEU}\textit{, du grec «~Matthaios~»~: «~don de Yahweh~»}\newline
Collecteur d'impôts, il fut l'un des douze apôtres de Jésus et l'auteur de l'évangile qui porte son nom. Voir \vref{Mt. 9:9} et \vref{Mt. 10:3}.

\DicoEntry{MEÉDIATEUR}\textit{, du grec «~mesites~»~: «~celui qui intervient entre deux parties~», «~intermédiaire de communication~»}\newline
Moïse a exercé cette fonction auprès du peuple d'Israël qui avait expressément demandé que Dieu ne leur parle pas directement. Christ, garant d'une Nouvelle Alliance, est à présent l'unique intermédiaire et médiateur entre Dieu et les hommes. Voir \vref{Ex. 20:19}~; \vref{1 Ti. 2:5}~; \vref{Hé. 8:6} et \vref{Hé. 9:15}.

\DicoEntry{MELCHISÉDEK}\textit{, de l'hébreu «~Malkiy-Tsedeq~»~: «~roi de justice~»}\newline
Roi de Salem et prêtre du Dieu Très-Haut, il était une apparition de Jésus-Christ avant son apparition. Il bénit Abraham après sa victoire contre Kedorlaomer. Jésus-Christ est grand prêtre selon l'ordre de Melchisédek. Voir \vref{Ge. 14:14-20}~; \vref{Hé. 5:5-10} et \vref{Hé. 6:20}.

\DicoEntry{MENSONGE}\textit{, de l'hébreu «~sheqer~»~: «~mensonge, déception, fausseté, tromperie, fraude~» et du grec «~pseudos~»~: «~fausseté consciente et intentionnelle~»}\newline
Modification de la vérité. Satan est appelé père du mensonge et les menteurs auront droit à la même sentence que lui. Voir \vref{Ex. 20:16}~; \vref{Jn. 8:44} et \vref{Ap. 21:8}.

\DicoEntry{MÉSOPOTAMIE}\textit{, de l'hébreu «~'Aram Naharayim~»~: «~pays entre deux fleuves~»}\newline
Située entre le Tigre et l'Euphrate, région correspondant à l'actuel Irak. Avant son appel, Abraham vivait à Ur en Chaldée qui se trouvait au sud de la Mésopotamie. Voir \vref{Ge. 11:31}.

\DicoEntry{MESSIE}\textit{, de l'hébreu «~mashiyach~»~: «~oint, celui qui est l'oint~»}\newline
Voir CHRIST.

\DicoEntry{MICHÉE}\textit{, de l'hébreu «~Miykayehuw~»~: «~qui est comme Dieu~?~»}\newline
Originaire de Moréscheth, Michée exerça la fonction de prophète dans le royaume du sud au temps d'Ezéchias, roi de Juda. L'ensemble de ses prophéties se trouve dans le livre éponyme.

\DicoEntry{MICHEL ou MICHAËL}\textit{, de l'hébreu «~Miyka'el~»~: «~qui est semblable à Dieu~?~»}\newline
Archange* de Dieu, il est un des principaux chefs des anges*. Souvent présent dans les grandes batailles, il lutta notamment contre le roi de Perse et contre le diable. Voir \vref{Da. 10:13-21}~; \vref{Jud. 1:9} et \vref{Ap. 12:7}.

\DicoEntry{MILLE}\textit{, du grec «~million~»~: «~distance de mille pas~»}\newline
Unité de mesure romaine correspondant à 1480m environ. Voir \vref{Mt. 5:41}.

\DicoEntry{MILLÉNIUM}\textit{}\newline
Période de paix de mille ans durant laquelle le Seigneur régnera sur la terre. Voir \vref{Es. 11,12} et \vref{Ap. 20:2-7}.

\DicoEntry{MINISTÈRE}\textit{, du grec «~diakonia~»~: «~service~», dérivé du mot grec «~diakonos~»~: «~domestique~»}\newline
voir SERVICE. 

\DicoEntry{MISÉRICORDE}\textit{, de l'hébreu «~checed~»~: «~bonté, miséricorde, fidélité~» et du grec «~eleos~»~: «~bonne volonté envers le misérable associée à un désir de l'aider~»}\newline
Comme en témoigne le plan du salut* qu'il a déployé, Dieu est riche en miséricorde. Le disciple de Christ doit comme son maître se revêtir d'entrailles de miséricorde afin de représenter le Royaume de Dieu. \vref{Ge. 24:7}~; \vref{No. 24:18}~; \vref{Mt. 9:13}~; \vref{Lu. 1:78}~; \vref{Ro. 11:31} et \vref{2 Jn. 1:3}.

\DicoEntry{MOAB}\textit{, de l'hébreu «~Mow'ab~»~: «~issu d'un père~»}\newline
Fils de Lot*, né de sa relation incestueuse avec sa fille aînée, il donna naissance au peuple des moabites. Ils s'établirent au sud-est de la mer morte et s'opposèrent plusieurs fois aux enfants d'Israël. Voir \vref{Ge. 19:37}~; \vref{Jg. 3:12}~; \vref{2 S. 8:2}~; \vref{Ez. 25:8-11}.

\DicoEntry{MODALISME}\textit{}\newline
Doctrine enseignée à Rome au début du troisième siècle par Sabellius selon laquelle le Père, le Fils et le Saint-Esprit sont différents aspects au travers desquels Dieu se révèle et non trois personnes distinctes. Réfutant ainsi la doctrine de la trinité* largement acceptée par les catholiques, Sabellius fut condamné par le pape Callixte à cause de son enseignement pourtant biblique. Voir \vref{1 Th. 3:11}~; \vref{2 Th. 2:16-17} et \vref{1 Jn. 5:20}.

\DicoEntry{MOÏSE}\textit{, de l'hébreu «~Mosheh~»~: «~tiré de~»}\newline
Issu de la tribu de Lévi, il fut miraculeusement sauvé du massacre des enfants de sa génération pendant la servitude d'Israël en Egypte. Il vécut les quarante premières années de sa vie dans la maison de Pharaon puis les quarante suivantes dans le désert auprès de Madian*. A l'issue de cette deuxième période, Yahweh se révéla à lui et le mandata pour délivrer le peuple d'Israël de la captivité égyptienne afin de le faire entrer dans la terre promise. Après l'avoir fait sortir au milieu des miracles et des prodiges, Moïse conduisit le peuple dans le désert pendant quarante années au cours desquelles il leur communiqua l'intégralité de la loi*. Il mourut à la porte de la terre promise à l'âge de cent vingt ans. On lui attribue l'écriture des cinq premiers livres du Tanakh. Voir \vref{Ex. 1-2}~; \vref{Ex. 12:40-41}~; \vref{Ex. 14:21-31}~; \vref{Ex. 24:12}~; \vref{De. 8:2}~; \vref{De. 34:5-7}~; \vref{Ac. 7:20-43} et \vref{Hé. 11:23-29}.

\DicoEntry{MOISSON}\textit{, de l'hébreu «~qatsiyr~»~: «~moisson, travail de la moisson, récolte~»}\newline
Sous la loi, la fête des prémices avait lieu lors de la moisson. Jésus utilise ce terme pour parler du champ missionnaire, les personnes à qui l'évangile doit être annoncé. Dans le cadre de la fin du monde, la moisson se rapporte au jugement de Dieu qui va apporter la séparation entre ses fils et les fils du diable. Voir \vref{Lé. 23:10-14}~; \vref{Mt. 9:37-38}~; \vref{Mt. 13:33-43}.

\DicoEntry{MOLOC}\textit{, de l'hébreu «~Molek~»~: «~roi, conseiller~»}\newline
Divinité vénérée par les Ammonites à qui il était coutume de sacrifier des enfants brûlés vifs. Les Israélites se prostituèrent plusieurs fois à Moloc. Voir \vref{1 R. 11:5-7} et \vref{2 R. 23:10}.

\DicoEntry{MONT DES OLIVIERS}\textit{}\newline
Colline située à l'est de Jérusalem près de la vallée du Cédron. C'est du Mont des Oliviers que Jésus fut enlevé au ciel après avoir donné ses dernières recommandations aux apôtres~; c'est à ce même endroit qu'il posera les pieds lors de son glorieux retour. Voir \vref{Za. 14:1-4} et \vref{Ac. 1:4-12}.

\DicoEntry{MORT}\textit{, de l'hébreu «~muwth~»~: «~mourir, tuer, être exécuté~» et du grec «~thanatos~»~: «~mort du corps~»}\newline
La Bible distingue deux morts. La première entra dans le monde suite à la désobéissance de l'homme et correspond à la séparation d'avec Dieu et à la mort physique. La deuxième mort concerne uniquement ceux dont le nom n'est pas écrit dans le livre de vie et correspond à la souffrance éternelle dans l'étang de feu. Voir \vref{Ge. 3}~; \vref{Ro. 5:12}~; \vref{Ro. 6:23} et \vref{Ap. 20:11-15}.

\DicoEntry{MYRRHE}\textit{, de l'hébreu «~more~»~: «~myrrhe~»}\newline
Résine provenant de certains arbres d'Asie et d'Afrique, réputée pour son arôme de grand prix. Elle était utilisée sous forme d'huile pour l'onction sainte et pouvait atténuer les douleurs quand elle était mélangée au vin. Les mages offrirent de la myrrhe à Jésus lors de sa naissance. Voir \vref{Ex. 30:22-30}~; \vref{Mt. 2:11}~; \vref{Mt. 27:34} et \vref{Mc. 15:23}.

\DicoEntry{NAHUM}\textit{, de l'hébreu «~Nachuwm~»~: «~consolation, qui a compassion~»}\newline
Prophète de Yahweh né à Elkosch, il annonça la destruction de Ninive. L'ensemble de ses prophéties se trouve dans le livre portant son nom.

\DicoEntry{NAISSANCE D'EN HAUT}\textit{, du grec «~anothen~»~: «~depuis le haut, depuis un endroit plus élevé~»}\newline
Naissance d'eau et d'esprit symbolisant respectivement la Parole qui purifie et le Saint-Esprit* qui est le gage de l'appartenance à Dieu. La naissance d'en haut est l'œuvre du Saint-Esprit qui délivre une personne du royaume des ténèbres et la transporte dans le Royaume de Dieu. L'homme charnel devient alors spirituel, le cœur de pierre est ôté pour accueillir un cœur de chair, le citoyen terrestre se transforme en citoyen céleste et le vieil homme laisse place à une nouvelle créature. Voir \vref{Ez. 36:25-27}~; \vref{Jn. 3:1-8}~; \vref{Ja. 1:18}~; \vref{1 Co. 12:13} et \vref{2 Co. 5:17}~; \vref{Ep. 2:6}~; \vref{Ep. 5:26}~; \vref{1 Jn. 3:9}.

\DicoEntry{NATHAN}\textit{, de l'hébreu «~Nathan~»~: «~il (Yahweh) a donné~»}\newline
Prophète de Yahweh au temps du roi David. Il prophétisa le règne éternel de la postérité de David et la construction du temple par son fils. Il reprit David lorsque ce dernier fit assassiner Urie* pour prendre sa femme. Voir \vref{2 S. 7,12}.

\DicoEntry{NAZARÉEN, NAZIRÉEN}\textit{, de l'hébreu «~naziyr~»~: «~consacré ou voué~»}\newline
Terme pouvant désigner soit un habitant de la ville de Nazareth, soit une personne qui s'est consacrée à Yahweh dans le cadre d'un vœu de naziréat. Voir \vref{No. 6}.

\DicoEntry{NAZARETH}\textit{, du grec «~Nazareth~»~: «~verdoyant, germe, rejeton~»}\newline
Ville située dans la région de Galilée où Jésus passa la majeure partie de sa vie. Voir \vref{Mt. 2:22-23}.

\DicoEntry{NEBUCADNETSAR}\textit{, (règne~: 605 av. J.-C. – 562 av. J.-C.), «~Nebuwkadne'tstsar~»~: «~que Nebo protège la couronne, les frontières~» (origine inconnue)}\newline
Roi de Babylone, il mit fin au royaume de Juda en emmenant le peuple en captivité~; il détruisit le temple de Jérusalem. Il reçut l'interprétation de plusieurs songes au travers de Daniel* et reconnut le règne dominant et éternel de Yahweh. Voir \vref{2 R. 25}, \vref{Da. 1:1} et \vref{Da. 2,4}.

\DicoEntry{NÉHÉMIE}\textit{, de l'hébreu «~Nechemyah~»~: «~Yahweh a consolé~»}\newline
Fils d'Hacalia, il fut échanson du roi Artaxerxès* à Suse, pendant la captivité de Juda. Il entreprit la réparation des murailles de Jérusalem et initia une réforme en son temps. Il devint ensuite gouverneur de Juda. Son histoire est racontée dans le livre éponyme.

\DicoEntry{NÉPHILIM}\textit{, de l'hébreu «~nephiyl~»~: «~géant~», racine~: «~naphal~»~: «~tomber, chuter~»}\newline
Etres de grande taille nés de l'union des fils de Dieu et des filles des hommes avant le déluge. On en retrouve aussi en Canaan lorsque les douze espions hébreux étaient allés observer la terre promise. Voir \vref{Ge. 6:4} et \vref{No. 13:32-33}.

\DicoEntry{NEPHTHALI}\textit{, de l'hébreu «~Naphtaliy~»~: «~lutte, mon combat~»}\newline
Fils de Jacob* et de Bilha, servante de Rachel*, il est l'ancêtre de la tribu de Nephthali. Voir \vref{Ge. 30:8} et \vref{Ge. 49:21}.

\DicoEntry{NICODÈME}\textit{, du grec «~Nikodemos~»~: «~victorieux du peuple~»}\newline
Docteur de la loi, il s'approcha de Jésus de nuit, qui l'enseigna sur la naissance d'en haut. Après la crucifixion, il aida Joseph d'Arimathée pour embaumer le corps du Seigneur et pour le mettre dans un sépulcre. Voir \vref{Jn. 3:1-21} et \vref{Jn. 19:38-42}.

\DicoEntry{NICOLAÏTES}\textit{, du grec «~Nikolaites~»~: «~destruction du peuple~»}\newline
Secte suivant la doctrine de Nicolas, liée à la doctrine de Balaam, qui poussait à la consommation de viandes sacrifiées aux idoles et à l'impudicité. Voir \vref{Ap. 2:6,14-15}.

\DicoEntry{NIL}\textit{, de l'hébreu «~Shiychowr~»~: «~sombre, noir, boueux~»}\newline
Principal fleuve d'Egypte situé à l'est du pays. Voir \vref{Es. 23:3}~; \vref{Jé. 2:18}.

\DicoEntry{NIMROD}\textit{, de l'hébreu «~Nimrowd~»~: «~rebelle~»}\newline
Fils de Cush et descendant de Noé. Chasseur, il fut le premier homme puissant sur la terre et régna sur plusieurs villes dont Babel*. Voir \vref{Ge. 10:8-11} et \vref{Ge. 11:1-9}.

\DicoEntry{NINIVE}\textit{, de l'hébreu «~Niyneveh~»~: «~habitation de Ninus~»}\newline
Grande ville située sur la rive est du Tigre. Ses habitants se repentirent de leurs mauvaises voies suite à la prédication de Jonas, mais ils retombèrent dans le péché quelques années plus tard. Ninive fut finalement détruite sous le jugement de Dieu. Voir livres de Jonas et de Nahum.

\DicoEntry{NOCES}\textit{, du grec «~gamos~»~: «~fête du mariage~»}\newline
Festivités célébrant le mariage. Dans la tradition juive, les noces duraient sept jours même si une longue période pouvait parfois s'écouler entre la conclusion du mariage (accord des familles) et la consommation du mariage (nuit de noces). Ainsi, la fiancée devait se tenir prête pour les noces à tout moment. De même, l'Eglise se prépare à être enlevée par Jésus à tout moment pour les noces de l'Agneau qui seront célébrées au ciel pendant sept ans. Voir \vref{Ge. 29:27}~; \vref{Jg. 14:12}~; \vref{1 Th. 4:16-17} et \vref{Ap. 19:7}.

\DicoEntry{NOÉ}\textit{, de l'hébreu «~Noach~»~: «~repos, tranquillité~»}\newline
Fils de Lamech, il fut le père de trois fils~: Sem, Cham et Japhet. Qualifié d'homme juste et intègre en son temps, il trouva grâce devant Yahweh qui lui ordonna de construire une arche* pour le sauver lui, sa famille et une partie des animaux de la terre du déluge qui arrivait. Son obéissance sauva la race humaine. Il vécut 950 ans. Voir \vref{Ge. 6-9}.

\DicoEntry{NOUVELLE NAISSANCE}\textit{}\newline
Voir NAISSANCE D'EN HAUT.

\DicoEntry{OFFRANDE}\textit{, de l'hébreu «~minchah~»~: «~don, tribut, présent, oblation, sacrifice~»}\newline
Sous la loi, le peuple d'Israël avait reçu des prescriptions relatives aux offrandes agréables à Yahweh~; elles consistaient essentiellement en bétail et produits naturels et étaient offertes dans le cadre de cérémonies spécifiques. Des offrandes en argent pouvaient aussi être données, notamment pour soutenir l'entretien du temple. Sous la Nouvelle Alliance, les offrandes monétaires doivent être libres et volontaires~; l'offrande la plus importante aux yeux de Dieu reste la vie consacrée de ses enfants. Voir \vref{Lé. 1-7}~; \vref{Mc. 12:41-42}~; \vref{2 Co. 8:10-12}~; \vref{2 Co. 9:7}~; \vref{Ro. 12:1} et \vref{Ro. 15:15-16}.

\DicoEntry{OLIVIER}\textit{}\newline
Arbre fruitier donnant des olives avec lesquelles on produit de l'huile. Sous la loi de Moïse, elle était notamment utilisée pour alimenter les lampes qui devaient brûler continuellement dans le temple et pour oindre les personnes désignées par Dieu pour une tâche spécifique. L'olivier symbolise en outre le témoignage et la paix. Voir \vref{Ex. 27:20-21}~; \vref{Ex. 30:22-25}~; \vref{Jg. 9:8-9}~; \vref{1 S. 16:3} et \vref{1 R. 19:16}.

\DicoEntry{OMEGA}\textit{}\newline
Dernière lettre de l'alphabet grec désignant aussi la fin d'une chose (voir ALPHA et OMEGA).

\DicoEntry{ONCTION}\textit{, de l'hébreu «~mishchah~»~: «~portion consacrée, huile d'onction, oindre~» et du grec «~chrisma~»~: «~toute chose qui sert à enduire~» de la racine «~chrio~»~: «~oindre, imprégner les chrétiens des dons du Saint-Esprit~»}\newline
Sous l'Ancienne Alliance, l'onction était souvent accordée par l'action de verser de l'huile sur la tête de la personne ou de l'objet à consacrer. On oignait ainsi les prêtres, les rois et les prophètes selon leur mandat. Sous la Nouvelle Alliance, l'onction demeure en celui qui a reçu en lui le Seigneur Jésus. Toutefois, l'onction d'huile peut être pratiquée dans le cadre de la prière pour les malades. Voir \vref{Ex. 30:22-31}~; \vref{1 S. 16:3} et \vref{1 R. 19:16}~; \vref{Ac. 1:8}~; \vref{Ja. 5:14} et \vref{1 Jn. 2:20-27}.

\DicoEntry{ORDINATION}\textit{, du latin «~ordinatio~»~: «~action de disposer, de mettre en œuvre~»}\newline
Rite initiatique mis en place par l'Eglise catholique qu'on ne retrouve pas dans les Ecritures. Elle confère, par l'imposition des mains accompagnée d'une prière, la capacité d'exercer une fonction dirigeante au sein de l'église locale.

\DicoEntry{OTHNIEL}\textit{, de l'hébreu «~`Othniy'el~»~: «~Dieu est puissant~»}\newline
Fils de Kenaz et frère cadet de Caleb, il fut le premier juge en Israël, fonction qu'il exerça pendant 40 ans. Il délivra les enfants d'Israël du joug du roi de Mésopotamie, Cuschan-Rischeathaïm. Voir \vref{Jg. 3:8-11}.

\DicoEntry{OSÉE}\textit{, de l'hébreu «~Howshea`~»~: «~salut, sauve~»}\newline
Fils de Beéri, prophète qui, sous les ordres de Yahweh, épousa une prostituée pour illustrer l'infidélité des enfants d'Israël envers leur Dieu. L'histoire d'Osée et l'ensemble de ses prophéties se trouvent dans le livre portant son nom.

\DicoEntry{PAÏEN}\textit{, du latin «~paganus~» qui signifie «~paysan~», qui provient lui-même du mot «~pagus~» qui signifie «~campagne~».}\newline
Personne qui pratiquait une des religions polythéistes de l'Antiquité.

\DicoEntry{PAIX}\textit{, de l'hébreu «~shalowm~»~: «~état complet, perfection, bien-être, paix~» et du grec «~eirene~»~: «~état de tranquillité, paix entre les individus, harmonie, sécurité~»}\newline
Sous l'Ancienne Alliance, la paix était matérialisée par la prospérité, l'absence de guerre et de toutes sortes de malheurs. Sous la Nouvelle Alliance, la paix est un fruit de l'Esprit*, une promesse acquise en Jésus qui est lui-même le Prince de Paix. Différente de celle que le monde offre, la paix de Christ permet de rester confiant en toutes circonstances. Voir \vref{Lé. 26:6}~; \vref{Es. 26:12}~; \vref{Jn. 14:27}~; \vref{Jn. 16:33} et \vref{Ga. 5:22}.

\DicoEntry{PALMIER}\textit{, de l'hébreu «~tamar~»~: «~palmier, dattier~»}\newline
Arbre à tronc peu ou pas ramifié, on le retrouve essentiellement dans le désert. L'image du palmier fut utilisée en décoration dans le temple. Ses branches étaient utilisées pendant la fête des tentes. Symbole de la justice et de la victoire, on le retrouve lors de l'entrée royale de Jésus à Jérusalem et devant le trône de Dieu. Voir \vref{Lé. 23:40}~; \vref{1 R. 6:29}~; \vref{Jn. 12:12-13} et \vref{Ap. 7:9}.

\DicoEntry{PÂQUE}\textit{, de l'hébreu «~pecach~»~: «~passer outre, épargner~», «~sacrifice de la Pâque~» ou «~fête de la Pâque~»}\newline
Première fête du calendrier hébraïque, elle fut instituée par ordonnance perpétuelle dès la sortie d'Egypte. Cette fête commémore le salut de Yahweh accordé par le sacrifice de l'agneau~; elle préfigurait Christ, l'Agneau de Dieu qui est «~notre Pâque~». Voir \vref{Ex. 12}~; \vref{Lé. 23:5}~; \vref{Jn. 1:29} et \vref{1 Co. 5:7-8}.

\DicoEntry{PARADIS}\textit{, du grec «~paradeisos~»~: «~jardin~»}\newline
Lieu de repos et de félicité, le paradis fut ouvert par Jésus lors de sa résurrection. Il y emmena les justes décédés qui étaient jusque-là captifs dans le séjour des morts*. Les chrétiens rejoignent ce lieu céleste à leur décès, en attendant la résurrection*. A la croix, Christ garantit l'accès à ce lieu au malfaiteur repentant. Paul fut ravi à cet endroit où il entendit des paroles merveilleuses. Voir \vref{Lu. 23:43}~; \vref{2 Co. 12:2-4}~; \vref{Ep. 4:8-10} et \vref{Hé. 10:19-20}.

\DicoEntry{PARDON}\textit{, de l'hébreu «~nas a´~»~: «~action de lever, supporter, prendre~» et du grec «~aphesis~»~: «~libérer de l'esclavage~» ou «~oubli des péchés, rémission des peines~»}\newline
Sous l'Ancienne Alliance, le pardon était conditionné par les sacrifices d'animaux, mais Jésus-Christ a accompli cette prérogative en devenant la victime expiatoire pour nos péchés. En lui, l'homme repentant est pardonné de ses fautes et trouve également la force de pardonner à ceux qui l'offensent. Voir \vref{Lé. 4-6}~; \vref{Mt. 6:12,14-15}~; \vref{Jn. 1:29}~; \vref{Ac. 10:43} et \vref{1 Jn. 1:9}.

\DicoEntry{PARVIS}\textit{, de l'hébreu «~chatser~»~: «~cour, enclos, colonie, ville, village~» (voir illustration du temple)}\newline
Première des trois parties du tabernacle et du temple~; il s'agissait d'une cour dans laquelle se trouvait l'autel d'airain où se faisaient des sacrifices et la cuve d'airain contenant de l'eau pour la purification. Voir \vref{Ex. 27:9-19}.

\DicoEntry{PASTEUR}\textit{, de l'hébreu «~ra`ah~»~: «~berger~»}\newline
Un des cinq services d'\vref{Ep. 4:11} travaillant en collège, établi pour veiller pour le troupeau, le nourrir de la Parole et encourager les chrétiens à exercer pleinement et librement leur ministère. Toutefois, Jésus-Christ demeure le pasteur par excellence, le bon berger qui donne sa vie pour ses brebis et le gardien des âmes qui ne sommeille ni ne dort. Voir \vref{Ep. 4:11}~; \vref{Jn. 10:11-16}~; \vref{Ps. 23} et \vref{1 Pi. 2:25}.

\DicoEntry{PATMOS}\textit{, du grec «~Patmos~»~: «~mortel, fascinant~»}\newline
Petite île grecque de la mer Egée sur laquelle Jean fut exilé à la fin de sa vie. Il y reçut la révélation de l'Apocalypse. Voir \vref{Ap. 1:9}.

\DicoEntry{PAUL}\textit{, du grec «~Paulos~»~: «~petit~»}\newline
Issu de la tribu de Benjamin et né dans la ville de Tarse, son nom était initialement Saul*. Pharisien, son zèle excessif le poussa à persécuter violemment les chrétiens à la naissance de l'Eglise. Il rencontra Christ sur la route de Damas et devint par la suite l'apôtre des Gentils annonçant l'Evangile de villes en villes et de pays en pays au cours de nombreux voyages. Même en prison, il continua l'œuvre de Dieu en écrivant plusieurs lettres riches en enseignements que l'on peut retrouver dans le canon biblique. Voir \vref{Ac. 9-28} et les épîtres de Paul.

\DicoEntry{PÉAGER ou PUBLICAIN}\textit{, du grec «~telones~»~: «~un loueur, un collecteur de taxes~»}\newline
Les péagers d'origine juive étaient dépréciés de leurs compatriotes et assimilés à des pécheurs, car on les considérait comme des collaborateurs au service des romains. De plus, certains profitaient de leur fonction pour s'enrichir. Voir \vref{Mt. 9:10}~; \vref{Mt. 21:31}~; \vref{Lu. 3:12-13} et \vref{Lu. 19:2-8}.

\DicoEntry{PÉCHÉ}\textit{, de l'hébreu «~chatta'ah~»~: «~ce qui manque le but~» et du grec «~hamartano~»~: «~erreur, faux état d'esprit~»}\newline
Le péché entra dans le monde par la transgression d'Adam et Eve et tous les hommes en furent infectés. Origine de la séparation entre Dieu et les hommes, le péché conduit à la mort*. Voir \vref{Ge. 3}~; \vref{1 Co. 15:3}~; \vref{Ro. 5:12}~; \vref{Ro. 6:23}~; \vref{Ro. 8:1-4} et \vref{1 Pi. 2:21-24}.

\DicoEntry{PENTECÔTE}\textit{, du grec «~pentekoste~»~: «~le cinquantième jour~»}\newline
Fête annuelle juive célébrant la moisson des blés. La venue du Saint-Esprit* promis par Jésus eut lieu pendant la célébration de la Pentecôte. Voir \vref{Lé. 23:15-22}~; \vref{Jn. 16:7-11}~; \vref{Ac. 1:5} et \vref{Ac. 2:1-21}.

\DicoEntry{PHARAON}\textit{, de l'hébreu «~Par`oh~»~: «~grand palais~»}\newline
Titre donné aux rois égyptiens durant l'Antiquité. Voir \vref{Ge. 37:36} et \vref{Ge. 41}.

\DicoEntry{PHARISIEN}\textit{, du grec «~Pharisaios~»~: «~séparé~»}\newline
Secte juive dont les membres manifestaient un attachement excessif aux coutumes et traditions religieuses. Certains d'entre eux combattirent Jésus qui dénonça ouvertement leur fausse piété et leur dévouement hypocrite envers Dieu. Désirant la mort du Seigneur, ils participèrent à la conspiration qui précéda sa crucifixion. Voir \vref{Mt. 23:23-39}~; \vref{Mc. 7:1-13} et \vref{Jn. 18:2-3}.

\DicoEntry{PHILADELPHIE}\textit{, du grec «~Philadelpheia~»~: «~amour fraternel~»}\newline
Ville de Lydie en Asie Mineure. Irriguée par le fleuve Hermus, Philadelphie était une contrée très fertile, propice à l'agriculture et surtout à la culture de la vigne. Elle fut construite par le roi de Pergame, et plusieurs fois sujette à des tremblements de terre. Une des sept lettres d'Apocalypse s'adressait à l'église de Philadelphie. Cette dernière - contrairement aux autres qui cumulèrent des reproches – fut très encouragée par le Seigneur. Bien que située à 45 km de Sardes à laquelle elle était rattachée, l'Eglise de Philadelphie resta ferme en retenant la Parole de Dieu et ne se laissa pas influencer par les séductions du péché. Elle incarne ainsi l'Eglise que Jésus revient chercher, l'Eglise réveillée.

\DicoEntry{PHILÉMON}\textit{, du grec «~Philemon~»~: «~attentionné, qui embrasse~»}\newline
Disciple de Colosses qui recevait une église dans sa maison. Il avait un esclave nommé Onésime au sujet duquel Paul lui écrivit une lettre. Voir épître de Paul à Philémon.

\DicoEntry{PHILIPPE}\textit{, du grec «~Philippos~»~: «~aimant les chevaux~»}\newline
1. Homme de Bethsaïda, il fut l'un des douze apôtres* choisis par Jésus. Voir \vref{Mt. 10:3}~; \vref{Mc. 3:18} et \vref{Lu. 6:14}.
\\2. Un des sept diacres élus au sein de l'église de Jérusalem. Evangéliste*, il prêcha le Christ dans la ville de Samarie, à l'eunuque éthiopien qu'il baptisa et dans différentes villes. Voir \vref{Ac. 6:5}~; \vref{Ac. 8:4-8,26-40} et \vref{Ac. 21:8}.

\DicoEntry{PHILIPPES}\textit{, du grec «~Philippoi~»~: «~appartenant à Philippe~»}\newline
Fondée par Philippe II (382 av. J.-C. – 336 av. J.-C.) en 356 av. J.-C., ville grecque de Macédoine orientale. Située sur une voie romaine qui traversait les Balkans, elle est restée de taille modeste en dépit de son fort taux de fréquentation. Une église y naquit après la rencontre de Paul avec des femmes qui priaient à l'extérieur de la ville. L'apôtre leur écrivit une lettre qui figure dans le canon biblique. Voir \vref{Ac. 16:9} et l'épître aux Philippiens.

\DicoEntry{PHILISTINS}\textit{, de l'hébreu~: «~Pelesheth~»~: «~immigrants~»}\newline
Peuple qui habitait à l'extrême ouest de Canaan, le long de la mer Méditerranée. Ils furent plusieurs fois en conflit avec les Israélites~; Goliath était philistin. Voir \vref{Jg. 13-16} et \vref{1 S. 17}.

\DicoEntry{PHILOSOPHIE}\textit{, du grec «~philosophia~»~: «~amour de la sagesse~»}\newline
Discipline existant depuis l'Antiquité et ayant plusieurs courants de pensée en son sein comme les épicuriens* et les stoïciens*. Elle pousse ses adeptes à rechercher la sagesse par l'intelligence humaine. Paul invita les chrétiens à se garder de ces doctrines. Voir \vref{Ac. 17:16-20} et \vref{Col. 2:8}.

\DicoEntry{PHINÉES}\textit{, de l'hébreu «~Piynechac~»~: «~bouche de cuivre~»}\newline
Fils d'Eléazar, petit-fils d'Aaron et prêtre. Il se démarqua par son zèle pour Dieu pour arrêter un fléau sur Israël. A cette occasion, Yahweh fit alliance perpétuelle avec Phinées et sa descendance. Voir \vref{No. 25}.

\DicoEntry{PIERRE}\textit{, de l'hébreu «~Cephas~» et du grec «~Petros~»~: «~un roc ou une pierre~»}\newline
Fils de Jonas et frère d'André*, son nom était initialement Simon*. Pêcheur de métier originaire de la ville de Bethsaïda, il fut choisi comme apôtre pour les circoncis. Il écrivit deux épîtres portant son nom. Il aurait été crucifié à Rome. Voir \vref{Mt. 10:2}~; \vref{Jn. 1:42-44}~; \vref{Ga. 2:7-8}~; 1 Pi. et 2 Pi.

\DicoEntry{PILATE}\textit{, du grec «~Pilatos~»~: «~armé d'une lance~»}\newline
Gouverneur romain de la Judée en fonction pendant le ministère de Jésus. Il s'accorda avec son ennemi Hérode lorsqu'il fallut crucifier le Seigneur. N'ayant pas trouvé de crime en Jésus, il permit finalement sa crucifixion et fit mettre l'inscription suivante sur sa croix~: Jésus de Nazareth, roi des Juifs. Voir \vref{Lu. 3:1}~; \vref{Lu. 23:11} et \vref{Jn. 19:1-19}.

\DicoEntry{PRÉDESTINATION}\textit{, du grec «~proginosko~»~: «~avoir la connaissance avant~»}\newline
Révélant l'omniscience de Dieu qui connaît toutes choses à l'avance, la prédestination concerne l'œuvre de la croix prévue de toute éternité – l'Agneau ayant été immolé avant la fondation du monde. La prédestination est non pas la décision de Dieu d'envoyer certaines personnes en enfer, mais plutôt la capacité de Yahweh à connaître à l'avance ceux qui allaient devenir ses enfants d'adoption, transformés à l'image du Fils, en acceptant sa parole. Voir \vref{Jn. 1:12}~; \vref{Ro. 8:29-30}~; \vref{Ep. 1:5} et \vref{1 Pi. 1:19-20}.

\DicoEntry{PREMIER PRÊTRE}\textit{, de l'hébreu «~rosh~»~: «~tête, dessus, sommet, partie supérieure, chef, principal, premier, total, somme, hauteur, front, le devant, commencement~» et de «~kohen~»~: «~prêtre, intendant principal, ministre d'état~»}\newline
Sous la loi, le premier prêtre descendait d'Aaron. Il servait le Seigneur dans le sanctuaire* et enseignait la loi*. Tel un médiateur* entre Yahwhe et le peuple, il portait constamment le jugement de ce dernier pour qui il consultait Dieu au moyen de l'urim et du thummim. Il devait, une fois par an, entrer dans le Saint des saints et offrir des sacrifices d'animaux pour ses propres péchés et pour ceux du peuple. Par la suite, Jésus-Christ est devenu premier prêtre à perpétuité en s'offrant comme victime expiatoire et en présentant son sang une fois pour toutes dans le Saint des saints du temple céleste. Voir \vref{Ex. 28:30}~; \vref{Ex. 29:9}~; \vref{Esd. 2:63}~; \vref{No. 35:25}~; \vref{Hé. 4:14-16}~; \vref{Hé. 7:25-28} et \vref{Hé. 9:6-12,24-28}.

\DicoEntry{PRÉTOIRE}\textit{, du grec «~praitorion~»~: «~quartier général dans un camp romain, la tente du commandant en chef~»}\newline
Dans les évangiles, lieu de résidence des gouverneurs dans lequel se trouvaient notamment un tribunal et une prison. Voir \vref{Mt. 27:27}~; \vref{Jn. 18:28-29} et \vref{Ac. 23:35}.

\DicoEntry{PRIÈRE}\textit{, de l'hébreu «~palal~»~: «~intervenir, s'interposer, prier~» ou «~'athar~»~: «~prier, supplier, implorer~» et du grec «~proseuche~»~: «~prière adressée à Dieu~» ou «~parakaleo~»~: «~appeler à, convoquer, supplier, exhorter~»}\newline
Acte par lequel on s'approche de Dieu et on instaure un dialogue avec lui, en ayant foi dans sa présence et son action. Invité à prier constamment, le chrétien peut le faire pour se repentir, intercéder en faveur d'une situation particulière, demander quelque chose à Dieu, le remercier, le louer ou tout simplement lui exprimer son amour. La prière garde les enfants de Dieu dans la paix. Dieu connaissant toutes les pensées de l'homme, le plus important dans la prière reste l'écoute de la voix de Yahweh. Voir \vref{Ge. 20:17}~; \vref{1 S. 2:1}~; \vref{Job 22:27}~; \vref{Mt. 14:36}~; \vref{Ac. 16:9}~; \vref{1 Th. 5:17}~; \vref{Ph. 4:6-7} et \vref{1 Pi. 4:7}.

\DicoEntry{PROPHÈTE}\textit{, de l'hébreu «~nabiy'~»~: «~l'homme qui parle, celui qui est appelé, qui a reçu une inspiration~» et du grec «~prophetes~»~: «~celui qui interprète des oracles~», «~quelqu'un qui déclare ce qu'il a reçu par inspiration~»}\newline
Sous l'Ancienne Alliance, Dieu suscita de nombreux prophètes oints de l'Esprit afin qu'ils annoncent des messages particuliers et conduisent le peuple à l'obéissance et à la crainte de Yahweh. Sous la Nouvelle Alliance, il existe au moins trois types de prophètes. Le premier concerne ceux et celles qui prophétisent au sein des assemblées locales (\vref{Ac. 21:8-9}~; \vref{1 Co. 14:29-32}), ils exhortent, édifient et consolent le peuple (\vref{1 Co. 14:1-3}). Le deuxième concerne les personnes qui ont reçu la charge d'enseigner, poser les fondements, implanter des assemblées selon \vref{Ep. 4:11}. Parmi ces prophètes, on compte Barnabas, Siméon, Lucius de Cyrène, Manahen, Saul (\vref{Ac. 13:1-5}), Jude et Silas (\vref{Ac. 15:32-33}). Le troisième concerne tous les chrétiens qui sont des potentiels prophètes puisqu'ils ont l'Esprit de Christ en eux (\vref{1 Co. 14:23-25}~; \vref{1 Co. 14:31}). Dieu peut se servir n'importe quel chrétien pour prophétiser, c'est-à-dire communiquer une parole inspirée. Voir \vref{Ep. 2:20} et \vref{Ep. 4:11}.

\DicoEntry{PROPHÉTIE}\textit{, du grec «~propheteia~»~: «~discours émanant de l'inspiration divine et déclarant les desseins de Dieu~»}\newline
Depuis l'effusion du Saint-Esprit, tous les chrétiens nés d'en haut peuvent prophétiser sans pour autant avoir le ministère de prophète. La prophétie est en effet un don spirituel* auquel il faut aspirer et qui est attribué par le Saint-Esprit selon la volonté de Dieu. Voir \vref{Ac. 2:16-18}~; \vref{1 Co. 12:4-10} et \vref{1 Co. 14:1}.

\DicoEntry{PROPITIATOIRE}\textit{, de l'hébreu «~kapporeth~»~: «~siège de miséricorde, lieu d'expiation~»}\newline
Couvercle de l'arche* composé d'or pur, il était surmonté de deux chérubins* d'or se faisant face au milieu desquels Yahweh siégeait et se manifestait pour donner des instructions à Israël. Une fois par an, le grand prêtre entrait dans le Saint des saints et aspergeait le propitiatoire du sang des animaux sacrifiés pour la purification des péchés d'Israël. Voir \vref{Ex. 25:17-22} et \vref{Lé. 16}.

\DicoEntry{PROSÉLYTE}\textit{, du grec «~proselutos~»~: «~un nouveau venu, un étranger~»}\newline
Personne issue d'une nation païenne s'étant agrégée au peuple d'Israël par le rite de la circoncision* et la pratique de la loi mosaïque. Voir \vref{Mt. 23:15}~; \vref{Ac. 2:10}~; \vref{Ac. 6:5} et \vref{Ac. 13:43}.

\DicoEntry{PYTHON}\textit{, du grec «~Puthon~»~: «~un serpent ou un dragon}\newline
Esprit de divination auquel Paul fut confronté en Macédoine. Voir \vref{Ac. 16:16-18}.

\DicoEntry{RABBI}\textit{, de l'hébreu «~rab~»~: «~capitaine, chef~» et du grec «~rhabbi~»~: «~maître~», «~un grand monsieur, honorable~» ou «~un enseignant~»}\newline
Les disciples appelaient Jésus «~Rabbi~». Cependant, il a exhorté la foule et les conducteurs religieux à ne pas attribuer une telle marque de distinction aux hommes rappelant que seul Yahweh est maître. Voir \vref{Mt. 23:8}~; \vref{Mc. 11:21}~; \vref{Jn. 9:2}.

\DicoEntry{RACHEL}\textit{, de l'hébreu «~Rachel~»~: «~agnelle, brebis~»}\newline
Fille de Laban, deuxième femme de Jacob pour laquelle il travailla quatorze ans. Longtemps stérile, Yahweh lui donna finalement deux garçons~: Joseph et Benjamin. Elle mourut à l'accouchement du deuxième. Voir \vref{Ge. 29:10-31}~; \vref{Ge. 30:22-24} et \vref{Ge. 35:16-19}.

\DicoEntry{RAHAB}\textit{, de l'hébreu «~Rachab~»~: «~large, spacieux, tumultueux~»}\newline
Prostituée habitant Jéricho, elle cacha les deux espions juifs chez elle. Grâce à son acte, Josué* lui laissa la vie sauve ainsi qu'à sa famille lorsqu'il détruisit la ville et tous ceux qui s'y trouvaient. Rahab habita ensuite au milieu d'Israël~; elle figure non seulement parmi les héros de la foi, mais aussi dans la généalogie de Jésus-Christ. Voir \vref{Jos. 2:1}~; \vref{Jos. 6:17-25}~; \vref{Mt. 1:5-16} et \vref{Hé. 11:31}.

\DicoEntry{REBECCA}\textit{, de l'hébreu «~Ribqah~»~: «~ensorcelante, qui prend au piège~»}\newline
Fille de Bethuel et sœur de Laban, elle fut l'épouse d'Isaac*. Yahweh mit fin à sa stérilité et elle donna naissance à des jumeaux, Esaü et Jacob, qui devinrent deux grandes nations. Voir \vref{Ge. 24} et \vref{Ge. 25:21-26}.

\DicoEntry{RÉCONCILIATION}\textit{, du grec «~katallage~»~: «~échange, change, ajustement d'une différence~»}\newline
Jésus-Christ mourut à la croix pour réconcilier l'homme avec Dieu, c'est-à-dire le faire passer de l'état de séparation (causée par le péché) à l'état d'intimité avec Dieu. L'Eglise a le ministère de réconciliation et doit en ce sens présenter à l'homme pécheur la voie de la réconciliation avec Dieu au travers de la prédication de l'Evangile*. Voir \vref{Ro. 5:11}~; \vref{Hé. 10:18-20} et \vref{2 Co. 5:18-20}.

\DicoEntry{RÉDEMPTION}\textit{, de l'hébreu~: «~peduwth~»~: «~rachat~» et du grec: «~apolutrosis~»~: «~libération effectuée suite au paiement d'une rançon~»}\newline
Jésus-Christ a payé le prix nécessaire au rachat des péchés de tous les hommes par son sacrifice à la croix, leur permettant d'échapper à la mort éternelle au moyen de la foi*. Voir \vref{Ro. 3:23-24}~; \vref{Col. 1:14}~; \vref{Ep. 1:7} et \vref{Hé. 9:12}.

\DicoEntry{RÉFORME}\textit{, de l'hébreu «~yatab~»~: «~agir bien~» et du grec «~diorthosis~»~: «~remettre droit~»}\newline
La plupart des prophètes du Tanakh sont des réformateurs dans la mesure où ils prônent un retour à Dieu~; le roi Josias a institué une profonde réforme pendant son règne en déployant des efforts pour revenir à l'obéissance de la Parole. Jésus-Christ est le plus grand réformateur en ce qu'il marchait à contre-courant et vint restaurer l'homme à sa condition originelle, celle d'avant la chute. Ainsi, l'homme qui reçoit Jésus entre dans un processus où il est continuellement réformé par le Saint-Esprit au travers de la Parole. Voir \vref{2 R. 22}~; \vref{Jé. 7:5}~; \vref{Jé. 26:13}~; \vref{Os. 6:1}~; \vref{Mt. 19:8} et \vref{Jn. 16:7-15}.

\DicoEntry{REPENTANCE}\textit{, du grec «~metanoia~»~: «~changement de mentalité, d'intention~», «~tristesse qu'on éprouve de ses péchés~»}\newline
Un des points majeurs de la prédication de Jean-Baptiste* puis des apôtres*. La repentance est essentielle pour obtenir la rémission des péchés et doit être accompagnée de fruits. La repentance ne concerne pas uniquement le nouveau converti, mais tout disciple de Christ qui, jusqu'à la fin de sa vie, est dans un processus de perfectionnement. Voir \vref{Mc. 1:4}~; \vref{Lu. 3:8}~; \vref{2 Co. 7:9-10}~; \vref{Ro. 2:4}~; \vref{Ac. 2:38}~; \vref{Ac. 13:24}~; \vref{Ac. 17:30} et \vref{Ac. 26:20}.

\DicoEntry{RÉSURRECTION}\textit{, du grec «~anastasis~»~: «~se lever, ressusciter de la mort~».}\newline
Christ fut le premier à expérimenter la résurrection d'entre les morts. Au son de la dernière trompette*, les chrétiens décédés ressusciteront de même avec des corps incorruptibles pour les noces* de l'Agneau. Voir \vref{Mt. 28:6}~; \vref{1 Pi. 1:3}~; \vref{Ap. 1:5}~; \vref{1 Co. 15:52} et \vref{1 Th. 4:16}.

\DicoEntry{RÉVEIL}\textit{, du grec «~egeiro~»~: «~réveiller du sommeil, revenir à la vie, se lever~»}\newline
Prise de conscience personnelle ou collective sur sa condition de péché et.ou l'imminence du jugement de Dieu. Il en résulte la repentance*, la véritable conversion*, la crainte de Dieu, la préparation à la rencontre de Yahweh. Une personne réveillée a les yeux focalisés sur Christ et peut accomplir la volonté du Seigneur. Voir Jon.~; \vref{Ep. 5:14} et \vref{Ro. 13:11-14}.

\DicoEntry{ROBOAM}\textit{, de l'hébreu «~Rhoboam~»~: «~qui affranchit le peuple~»}\newline
Fils et successeur du roi Salomon. C'est sous son règne que se produit le schisme* entre les royaumes du nord et celui du sud. Il régna sur Juda dix-sept années pendant lesquelles il fut en guerre avec le royaume du nord et fit ce qui est mal aux yeux de Yahweh. Voir \vref{1 R. 12:1-24} et \vref{1 R. 14:21-31}.

\DicoEntry{ROMAIN}\textit{, du grec «~rhome~»~: «~force~»}\newline
Pendant la vie de Jésus et pendant l'époque de l'Eglise primitive, Israël était sous la domination de l'Empire romain qui l'oppressait et lui soutirait des impôts. Paul, né à Tarse - ville romaine - put bénéficier des privilèges liés à la nationalité romaine quand il fut livré aux tribunaux. Ce dernier écrivit une lettre aux chrétiens romains - figurant dans le canon biblique - avant de les rencontrer physiquement. Voir \vref{Mt. 22:17}~; \vref{Jn. 11:48}~; \vref{Ac. 16:35-39}~; \vref{Ac. 22:25-29}~; \vref{Ac. 23:27} et \vref{Ac. 25:16}.

\DicoEntry{ROME}\textit{, du grec «~rhome~»~: «~force~»}\newline
Capitale de l'Empire romain située en Italie, Rome jouissait d'une grande notoriété à l'époque de l'Eglise primitive. Bien que l'empereur Claude* ait ordonné aux Juifs de quitter la ville, Paul manifesta le désir de s'y rendre pour y annoncer l'Evangile. Il y arriva après bien des difficultés quelques années plus tard en tant que prisonnier. Voir \vref{Ac. 18:1-2}~; \vref{Ac. 19:21}~; \vref{Ac. 23:11} et \vref{Ac. 28:14-31}.

\DicoEntry{ROYAUME DE DIEU}\textit{, du grec «~basileia~»~: «~pouvoir royal, royauté, domination, autorité~»}\newline
Lors de son service terrestre, Jésus a annoncé que le Royaume de Dieu était proche. Il parlait de son autorité sur toutes choses et de son règne. Ne consistant pas dans les choses terrestres, ce royaume se manifeste par la puissance de Dieu, la justice, la paix et la joie par le Saint-Esprit. Voir \vref{Lu. 9:1-2}~; \vref{Lu. 11:17-20}~; \vref{Lu. 17:20-21} et \vref{Ro. 14:17}.

\DicoEntry{RUBEN}\textit{, de l'hébreu «~Re'uwben~»~: «~voici un fils~»}\newline
Premier fils de Jacob et Léa, il devint le père de la tribu des Rubénites qui s'installa à l'est de la terre promise. Il perdit son droit d'aînesse après avoir eu des rapports intimes avec Bilha, concubine de son père. Voir \vref{Ge. 29:32}~; \vref{Ge. 35:22} et \vref{Ge. 49:3-4}.

\DicoEntry{RUTH}\textit{, de l'hébreu «~Ruwth~»~: «~amitié, une amie~»}\newline
Originaire de Moab, belle-fille de Naomi avec qui elle s'installa à Bethléhem. Elle y épousa Boaz avec qui elle eut un fils, Obed, grand-père du roi David*. Son histoire est racontée dans le livre portant son nom.

\DicoEntry{SABBAT}\textit{, de l'hébreu «~shabbath~»~: «~repos, cessation d'activité~»}\newline
Septième et dernier jour de la semaine consacré à Yahweh pendant lequel aucune activité ne devait être pratiquée selon la loi. Le sabbat figure dans les dix commandements, son infraction devait être punie de mort. Suscitant de vives critiques de la part des religieux, Jésus a plusieurs fois enfreint le sabbat dont il s'est déclaré le maître. Sous la Nouvelle Alliance, le sabbat se trouve en Jésus-Christ, le chrétien n'est donc pas tenu de le respecter comme ce fut le cas sous la loi de Moïse. \vref{Ex. 20:8-11}~; \vref{Ex. 31:14-15}~; \vref{De. 5:12-15}~; \vref{Mt. 11:28-30}~; \vref{Mc. 2:23-28} et \vref{Mc. 3:1-6}.

\DicoEntry{SACERDOTALISME}\textit{}\newline
Doctrine d'origine catholique reconnaissant le prêtre ou le pasteur comme l'intermédiaire entre Dieu et les hommes. Voir \vref{1 Ti. 2:5}.

\DicoEntry{SACRIFICATURE, SACERDOCE}\textit{, de l'hébreu «~kahan~»~: «~service~»}\newline
Sous la loi mosaïque, il était exercé par les Lévites descendants d'Aaron dans le tabernacle puis le temple et consistait notamment à accomplir les différents rituels relatifs aux sacrifices d'animaux et aux offrandes de toutes sortes. Depuis le sacrifice de Jésus à la croix, le sacerdoce concerne tous les enfants de Dieu qui sont non seulement les prêtres, mais aussi les sacrifices auxquels le Seigneur prend plaisir. Voir Lé.~; \vref{Ro. 12:1}~; \vref{1 Pi. 2:9} et \vref{Ap. 1:6}.

\DicoEntry{SADDUCÉENS}\textit{, du grec «~saddoukaios~»~: «~les justes~»}\newline
Parti religieux juif attaché au Pentateuque de manière stricte, ils ne croyaient ni en la résurrection des morts ni aux anges. Ils s'opposèrent au service de Jésus qui les reprit sévèrement et échappa à leurs pièges. Ils combattirent de même les apôtres qu'ils jetèrent en prison. Voir \vref{Mt. 16:6-12}~; \vref{Mt. 22:23-33}~; \vref{Ac. 5:17-19} et \vref{Ac. 23:1-10}.

\DicoEntry{SAINT}\textit{, de l'hébreu «~qodesh~»~: «~consacré, mis à part~» et du grec «~hagios~»~: «~chose très sainte, consacré, un saint~»}\newline
Dieu appela Israël à la sainteté, c'est-à-dire à ne pas se mélanger avec les autres peuples de peur d'être contaminés par leurs pratiques méchantes et idolâtres. Yahweh est le Saint d'Israël. Sous la Nouvelle Alliance, les chrétiens sont appelés saints, car le Saint-Esprit qui est en eux leur communique sa nature, les purifie et leur enseigne la haine du péché. Voir \vref{De. 7:6}~; \vref{Es. 49:7}~; \vref{1 Co. 6:11,19}~; \vref{1 Th. 4:1-8} et \vref{Hé. 12:14}.

\DicoEntry{SAINT-ESPRIT}\textit{, (voir étymologie des mots «~saint~» et «~esprit~»)}\newline
Le Saint-Esprit est l'esprit de Dieu, l'Esprit de Jésus~; il est Dieu. Lors de son service terrestre, le Seigneur déclara qu'un consolateur viendrait habiter dans les corps des croyants. Cette parole s'accomplit lors de la Pentecôte. Le Saint-Esprit a pour mission de convaincre le monde en ce qui concerne le péché, la justice et le jugement. A la naissance d'en haut, il régénère l'esprit du chrétien sur qui il dépose son sceau, gage de l'adoption. Il enseigne et guide le chrétien tout au long de sa marche avec Dieu. Il transforme son caractère et distribue les dons spirituels pour l'édification de l'Eglise. Voir \vref{1 S. 10:10}~; \vref{2 Ch. 15:1}~; \vref{Jn. 14:16-17,26}~; \vref{Jn. 16:7-15}~; \vref{Ac. 2}~; \vref{1 Co. 6:11}~; \vref{Ro. 8:9}~; \vref{1 Co. 3:16}~; \vref{1 Co. 12:4-13}~; \vref{Ep. 1:13} et \vref{Ga. 5:16,22}.

\DicoEntry{SALOMON}\textit{, de l'hébreu «~Shelomoh~»~: «~paix, pacifique~»}\newline
Fils de David, il succéda à son père et fut roi d'Israël pendant quarante ans. Il construisit le premier temple de Yahweh sur le mont Morija à Jérusalem puis un palais royal. Outre ses importantes richesses, c'est la grande sagesse que Dieu lui donna qui fit sa renommée parmi tous les peuples. Il eut sept cents femmes et trois cent concubines - dont un grand nombre de femmes étrangères - ce qui détourna son cœur de son Dieu. On lui attribue la rédaction des livres Cantique des cantiques et Ecclésiaste~; il a aussi écrit certains psaumes et plusieurs proverbes. Voir \vref{1 R. 4:29-34}~; \vref{1 R. 5-7}~; \vref{1 R. 9:15-28}~; \vref{1 R. 11:1-10,42}~; \vref{Ps. 72,127} et \vref{Pr. 25-29}.

\DicoEntry{SALUT}\textit{, de l'hébreu «~yesha'~» et du grec «~soteria~»~: «~délivrance, sûreté, sécurité~»}\newline
Libération des chaînes du péché, de la condamnation et de tout type d'asservissement spirituel, le salut est un don gratuit de Dieu qui s'obtient par la grâce, au moyen de la foi. C'est la manifestation de l'amour éternel de Dieu qui - ne voulant pas que l'homme périsse dans le feu de la géhenne - a payé le prix pour lui offrir la vie éternelle. Le salut réside dans le seul nom de Jésus-Christ. Voir \vref{Jn. 3:16}~; \vref{Ac. 4:12}~; \vref{Ro. 8:1}~; \vref{1 Th. 5:9}~; \vref{Tit. 3:4-6} et \vref{Ep. 2:4-8}.

\DicoEntry{SAMARIE}\textit{, de l'hébreu «~Shomerown~» et du grec «~Samareia~»~: «~montagne de guet~»}\newline
Située dans l'actuelle Cisjordanie, ville fondée par Omri, roi d'Israël, et qui devint la capitale du royaume du nord. La ville fut prise par Salmanasar, roi d'Assyrie, sous le règne d'Osée, roi d'Israël. Au temps de Jésus, la Samarie n'était qu'une simple circonscription romaine dont la population était issue du métissage entre Israélites et des colons assyriens. Suite aux persécutions subies par l'Eglise primitive à Jérusalem, des chrétiens s'y réfugièrent et l'Evangile s'y propagea. Voir \vref{1 R. 16:23-24}~; \vref{2 R. 3:1}~; \vref{2 R. 18:9}~; \vref{Os. 7} et \vref{Ac. 8:1-17}.

\DicoEntry{SAMARITAINS}\textit{, du grec «~samareites~»~: «~un habitant de Samarie~»}\newline
Après l'assujettissement de la Samarie par Salmanasar, roi d'Assyrie, des peuples étrangers s'y établirent et s'assemblèrent avec les Israélites. Au IVe siècle av J.-C., les samaritains construisirent un temple sur le mont Garizim, qui devint le centre religieux de Samarie, entraînant une séparation avec le reste des Juifs qui adoraient à Jérusalem. Les samaritains étaient considérés comme des étrangers et non comme de véritables juifs du fait de la mixité de leur religion. Jésus-Christ ouvrit la voie de la réconciliation avec ce peuple en racontant la parabole du bon samaritain et en annonçant la bonne nouvelle à la femme samaritaine. Voir \vref{2 R. 17:3,24-29} et \vref{2 R. 18:9}~; \vref{Jn. 4:4-26} et \vref{Lu. 10:30-37}.

\DicoEntry{SAMSON}\textit{, de l'hébreu «~Shimshown~»~: «~petit soleil~»}\newline
Fils de Manoach, de la tribu de Dan, il fut juge en Israël pendant vingt-ans. Consacré à Dieu dès le sein maternel et doté d'une force extraordinaire, il réalisa des prouesses qui suscitèrent la crainte de ses ennemis. Choisi pour être le libérateur d'Israël, il fut incompris par les siens qui ne le soutinrent pas. Il mourut suite à la trahison de Delila, une femme d'origine philistine. Voir \vref{Jg. 13-16}.

\DicoEntry{SAMUEL}\textit{, de l'hébreu «~Shemuw'el~»~: «~entendu ou exaucé de Dieu~»}\newline
Fils d'Elkana, de la tribu d'Ephraïm, et d'Anne, il fut consacré au service de Yahweh dès son plus jeune âge. Il exerça les fonctions de juge, prêtre et prophète sur Israël. Il oignit les deux premiers rois d'Israël~: Saül et David. Son histoire est racontée dans les deux livres du Tanakh portant son nom.

\DicoEntry{SANCTIFICATION}\textit{, de l'hébreu «~Qadash~» et du grec «~hagiasmos~»~: «~consécration, purification, sainteté~» ou «~l'effet de la purification~»}\newline
Fruit de l'action conjointe de la Parole et l'Esprit de Dieu dans la vie du croyant, la sanctification doit être recherchée par le chrétien tout au long de sa vie. Sans elle, nul ne verra Dieu. \vref{Jn. 17:17}~; \vref{1 Th. 4:3-8}~; \vref{Hé. 12:14} et \vref{Ap. 22:11}.

\DicoEntry{SANCTUAIRE}\textit{, de l'hébreu «~miqdash~»~: «~lieu sacré, lieu saint, sanctuaire de Yahweh~»}\newline
Le sanctuaire terrestre, dont Moïse avait reçu le modèle, était une représentation de celui qui se trouve au ciel et où Jésus alla présenter son sang. Voir \vref{Ex. 25:8-9} et \vref{Hé. 9:1-24}.

\DicoEntry{SANG}\textit{, de l'hébreu «~dam~» et du grec «~haima~»~: «~sang~»}\newline
Déterminant le lien de famille et la lignée, le sang est, selon les Ecritures, l'âme*, la vie. Ainsi, l'effusion de sang fut nécessaire pour le pardon des péchés et le sang de Christ, qui ôte définitivement le péché, donne la vie. Voir \vref{Lé. 17:11}~; \vref{Ac. 17:26}~; \vref{Ro. 5:9}~; \vref{Hé. 9:22-28} et \vref{Ap. 5:9}.

\DicoEntry{SANHÉDRIN}\textit{, du grec «~sunedrion~»~: «~conseil, tribunal~»}\newline
Désignait d'une part, les petits tribunaux se tenant dans chaque ville pour régler les affaires locales et d'autre part, le grand conseil de Jérusalem où étaient traitées les affaires plus importantes. Ce dernier était composé de soixante et onze membres sélectionnés parmi l'élite religieuse et les anciens d'Israël~; le grand prêtre en était le président. Sous la domination romaine, ce tribunal fonctionnait de manière quasi autonome~; la sentence de la peine de mort devait néanmoins être validée par le gouverneur romain. Jésus fut jugé coupable de blasphème par le sanhédrin qui le condamna à mort. Voir \vref{No. 11:16-17}~; \vref{Mt. 5:22}~; \vref{Mt. 26:59-66}~; \vref{Jn. 11:47}~; \vref{Ac. 5:21-41} et \vref{Ac. 6:12-15}.

\DicoEntry{SARA}\textit{, de l'hébreu «~Sarah~»~: «~princesse, femme noble~»}\newline
Femme d'Abraham, elle enfanta Isaac à l'âge de 90 ans selon la promesse de Yahweh. Sara figure parmi les héros de la foi~; elle mourut à cent vingt-sept ans. Voir \vref{Ge. 12:5}~; 17~; \vref{Ge. 15-16}~; \vref{Ge. 21:1-7}~; \vref{Ge. 23:1} et \vref{Hé. 11:11}.

\DicoEntry{SARDES}\textit{, du grec «~sardeis~»~: «~les rouges~», «~prince de joie~»}\newline
Capitale antique de la Lydie, Sardes se situait sur la rivière Pactole, à environ 50 km au sud de Thyatire et 75 km à l'est de Smyrne. Réputée riche et puissante en raison de ses ressources en or, ses épithètes étaient sournois car sa forteresse reposait sur un sol boueux. En effet, au VIème siècle av. J.-C., Cyrus Le Grand - vainqueur de Crésus alors roi de Lydie - s'empara de Sardes par une attaque nocturne. Par la suite, la ville subit plusieurs invasions puis un tremblement de terre en 17 ap. J.-C. L'église de Sardes fut probablement fondée par Paul au cours d'un voyage à Ephèse. Au moment où ils reçurent le message de l'ange de l'Apocalypse, il semblerait que certains chrétiens de Sardes étaient retournés au culte licencieux de Cybèle, déesse-mère et gardienne des savoirs. Ceux qui s'étaient gardés purs devaient ainsi revivifier les autres membres. Cette église symbolise l'église morte. Voir \vref{Ap. 3:1-6}.

\DicoEntry{SATAN}\textit{, de l'hébreu «~Satan~»~: «~adversaire, ennemi~»}\newline
Autrefois chérubin protecteur, il a péché en voulant s'approprier la gloire qui ne revient qu'à Dieu. Dans sa rébellion, il entraîna un tiers des anges qui furent précipités avec lui sur la terre. Connu également sous les noms «~Prince de ce monde~», «~Prince des ténèbres~», «~Belzébul~», «~le malin~», «~l'accusateur~» ou «~le diable~», il est l'adversaire des enfants de Dieu à qui il fait la guerre. Il a cependant été vaincu à la croix par Jésus-Christ, au nom duquel les chrétiens peuvent le chasser. Satan sera enchaîné pendant le millenium puis libéré pour un peu de temps. Il sera finalement jeté dans l'étang de feu pour l'éternité. Voir \vref{Ez. 28:14-19}~; \vref{Es. 14:12-17}~; \vref{Ap. 12:4}~; \vref{Lu. 10:18-19}~; \vref{Ja. 4:7}~; \vref{Jn. 16:11}~; \vref{Ap. 12:4} et \vref{Ap. 20:1-15}.

\DicoEntry{SAÜL}\textit{, de l'hébreu «~Sha'uwl~»~: «~désiré, demandé (à Dieu)~»}\newline
1. Fils de Kis, israélite de la tribu de Benjamin, il fut choisi par Dieu pour être le premier roi d'Israël sur qui il régna pendant quarante ans. Il désobéit à la loi de Yahweh et tenta plusieurs fois d'assassiner David, choisi par Dieu pour lui succéder sur le trône. Saül mourut avec ses trois fils pendant la bataille de Guilboa. Voir \vref{1 S. 10}~; \vref{1 S. 13:1-14}~; \vref{1 S. 15:10-11}~; \vref{1 S. 18:8-16}~; \vref{1 S. 19:8-17} et \vref{1 S. 31}.
\\2. Nom initial de Paul*.

\DicoEntry{SCANDALE}\textit{, de l'hébreu «~mikshowl~»~: «~trébucher~» et du grec «~skandalon~»~: «~obstacle, piège~»}\newline
Pierre qu'on rencontre et qui peut faire glisser sur le chemin ou encore situation ou comportement qui provoque un trouble emmenant quelqu'un à fauter. Le scandale n'est pas forcément une mauvaise action en soi~; Christ lui-même fut un scandale pour les Juifs. Toutefois, il reste souvent lié aux œuvres de la chair et peut être provoqué par un manque de discernement. Le chrétien doit veiller par rapport aux scandales. Voir \vref{Ps. 106:36}~; \vref{Mt. 13:41}~; \vref{Mt. 18:7}~; \vref{1 Co. 1:23} et \vref{1 Pi. 2:7-8}.

\DicoEntry{SCEAU}\textit{, du grec «~sphragizo~»~: «~mettre un sceau dessus, poser une marque par l'impression d'un sceau~»}\newline
Sous l'Ancienne Alliance, la circoncision était une marque de l'alliance établie entre Yahweh et son peuple. A la naissance d'en haut, le chrétien est scellé du Saint-Esprit, témoignant de son appartenance à Christ. Voir \vref{Ge. 17:10-11}~; \vref{Ep. 1:13}~; \vref{Ap. 7:3} et \vref{Ap. 9:4}.

\DicoEntry{SCHEOL}\textit{}\newline
Voir SÉJOUR DES MORTS.

\DicoEntry{SCHISME D'ISRAËL}\textit{}\newline
Le schisme est la séparation d'Israël en deux royaumes suite à la dérive de Salomon. En 931 av. J.-C., Roboam succéda à son père Salomon sur le trône royal et n'accepta pas d'alléger le joug que son père avait mis sur eux, cela entraîna la séparation du royaume en deux. On retrouva d'une part, le royaume d'Israël dirigé par Jéroboam - appelé aussi royaume du nord -, composé des dix tribus du nord et d'autre part, le royaume de Juda gouverné par le roi Roboam composé des deux tribus du sud (Benjamin et Juda). Voir \vref{1 R. 12:1-24}.

\DicoEntry{SCRIBE}\textit{, de l'hébreu «~caphar~»~: «~secrétaire, scribe~», «~homme instruit, qui a le savoir~»}\newline
Les scribes occupaient une position importante auprès du peuple juif, ayant non seulement une mission d'enseignement de la loi, mais également une fonction au sein de la justice juive en prenant part au sanhédrin*. Voir \vref{Esd. 7:6-10} et \vref{Mt. 16:21}.

\DicoEntry{SECTE}\textit{, du grec «~hairesis~»~: «~action de prendre, capturer~»}\newline
Groupement de personnes adhérant à une doctrine particulière et vivant marginalement, comme les sadducéens ou les pharisiens. Les premiers disciples furent qualifiés de «~secte des Nazaréens~». Pierre met en garde contre les faux prophètes qui introduisent des sectes pernicieuses pour ravir la foi des chrétiens afin de les entraîner dans la dissolution. Voir \vref{Ac. 5:17}~; \vref{Ac. 26:5}~; \vref{Ac. 24:5} et \vref{2 Pi. 2:1}.

\DicoEntry{SÉDÉCIAS}\textit{, de l'hébreu «~Tsidqiyah~»~: «~Yahweh est justice~»}\newline
Fils d'Hamoutal et oncle de Jojakin, il fut le dernier roi de Juda sur qui il régna onze ans. Son nom initial, Matthania, fut changé en Sédécias par Nebucadnetsar, roi de Babylone. Il fit ce qui est mal aux yeux de Yahweh et connut un destin tragique~: ses fils furent égorgés devant lui, Nebucadnestar lui creva ensuite les yeux, Jérusalem et le temple* furent détruits et il fut emmené captif avec le peuple à Babylone. Voir \vref{2 R. 24:17-19}~; \vref{2 R. 25:1-21}~; \vref{Jé. 21}~; \vref{Jé. 22:1-9}~; \vref{Jé. 37,38,39:6-7}.

\DicoEntry{SÉJOUR DES MORTS}\textit{de l'hébreu «~she'owl~»~: «~monde souterrain, tombe, enfer, fosse~» et du grec «~hades~»~: «~dieu des profondeurs de la terre~»}\newline
Lieu de captivité où allaient les âmes de tous les défunts avant le sacrifice de Christ. Il était scindé en deux parties séparées par un grand abîme. D'un côté se trouvait un lieu de tourments et de souffrances extrêmes accueillant tous les méchants qui ont vécu dans le péché durant leur vie terrestre et qui n'y ont pas renoncé. D'un autre côté, il y avait le sein d'Abraham où reposaient et séjournaient les âmes des justes qui avaient foi en Yahweh. Après la résurrection de Jésus, ces derniers ont été arrachés du séjour des morts par le Seigneur qui les a emmenés au paradis*. Le ciel, en tant que destination des personnes décédées, fut en effet ouvert par Christ après sa résurrection. Par conséquent, le sein d'Abraham n'a jamais accueilli de chrétiens. Le séjour des morts est à présent composé uniquement d'impies~; à la fin du monde, il sera jeté avec tous ses habitants dans l'étang de feu*. Voir \vref{Lu. 16:19-31}~; \vref{1 S. 28:6-20}~; \vref{Mt. 11:23}~; \vref{Ac. 2:27}~; \vref{Jn. 3:13}~; \vref{Ep. 4:8} et \vref{Ap. 20:14}.
\\Note~: L'histoire de Lazare et de l'homme riche racontée dans Luc \vref{16:19-31} n'est pas une parabole. A la différence de tous les récits à caractère parabolique contés dans les Ecritures, cette histoire mentionne un nom.

\DicoEntry{SEM}\textit{, de l'hébreu «~Shem~»~: «~nom, renommée~»}\newline
Fils aîné de Noé et ancêtre d'Abraham. Voir \vref{Ge. 10:1} et \vref{Ge. 11:10-27}.

\DicoEntry{SÉNEVÉ}\textit{, du grec «~sinapi~»~: «~graine de moutarde~»}\newline
Plante des régions orientales ayant la forme d'une petite semence pouvant grandir de manière exponentielle et atteignant jusqu'à trois mètres. Elle symbolise spirituellement la puissance de la foi* capable de déplacer les montagnes. Voir \vref{Mt. 13:32-33} et \vref{Mt. 17:20}.

\DicoEntry{SEPHORA}\textit{, de l'hébreu «~Tsipporah~»~: «~petit oiseau, moineau~»}\newline
Fille de Jéthro, femme de Moïse et mère d'Eliézer et de Guerschom. Elle partit dans le pays d'Egypte avec Moïse quand il répondit à l'appel de Yahweh pour aller libérer Israël. \vref{Ex. 2:15-21} et \vref{Ex. 4:18-20}.

\DicoEntry{SÉRAPHINS}\textit{, de l'hébreu «~saraph~»~: «~être majestueux avec six ailes au service de Dieu~»}\newline
Catégorie d'anges* proclamant la sainteté de Dieu. Voir \vref{Es. 6:1-7}.

\DicoEntry{SERPENT}\textit{, de l'hébreu «~nachash~»~: «~serpent, reptile~»}\newline
C'est sous la forme du serpent que Satan vint séduire Eve dans le jardin d'Eden. Le serpent fut maudit d'entre tous les animaux pour son action. Le serpent ancien ou rusé désigne le diable*~; il s'oppose au serpent d'airain, Jésus, qui a donné à ses enfants le pouvoir de marcher sur les serpents. Voir \vref{Ge. 3:1-14}~; \vref{No. 21:4-9}~; \vref{2 Co. 11:3}~; \vref{Jn. 3:14-15}~; \vref{Lu. 10:19} et \vref{Ap. 12:9,14-15}.

\DicoEntry{SERVICE}\textit{, du grec «~diakonia~»~: «~service~», dérivé du mot grec «~diakonos~»~: «~domestique~»}\newline
Tâche que le chrétien exerce au service de Dieu et des hommes selon l'onction* et le mandat que Dieu lui donne. Le serviteur de Dieu est donc un serviteur inutile, un simple instrument utilisé pour la gloire de Yahweh. Voir \vref{Lu. 17:10}~; \vref{1 Co. 12}~; \vref{2 Co. 3:5}~; \vref{Ro. 12} et \vref{1 Pi. 4:10-11}.

\DicoEntry{SERVITEUR}\textit{, du grec «~diakonos~»~: «~domestique, subordonné, messager~» ou «~doulos~»~: «~esclave~»}\newline
Christ a renoncé à sa gloire et a pris la forme d'un simple serviteur. De même, le chrétien n'est pas uniquement serviteur de Dieu, il doit comme le maître servir son prochain. Voir \vref{Mc. 10:45}~; \vref{Ph. 1:1}~; \vref{Ph. 2:5-8} et \vref{2 Co. 6:4}.

\DicoEntry{SETH}\textit{, de l'hébreu «~Sheth~»~: «~compensation, mis à la place~»}\newline
Troisième fils d'Adam et Eve~; il naquit après le meurtre de son frère Abel que Caïn avait tué. Seth fut l'ancêtre de Noé et de Jésus-Christ. Voir \vref{Ge. 4:25}~; \vref{Ge. 5:6-29} et \vref{Lu. 3:38}.

\DicoEntry{SHOFAR}\textit{, de l'hébreu «~showphar~»~: «~corne, corne de bélier~».}\newline
Instrument de musique à vent fait à partir de la corne de bélier. Voir TROMPETTE.

\DicoEntry{SIDON}\textit{, de l'hébreu «~Tsiydown~»~: «~abondance de poisson, pêche~»}\newline
Ville de l'antique Phénicie (actuel Liban) située non loin de Tyr~; on y vénérait les Baals et les Astartés. La reine Jézabel était originaire de Sidon. Voir \vref{Jg. 10:6} et \vref{1 R. 16:31}.

\DicoEntry{SILAS}\textit{, du grec «~Silas~»~: «~de la forêt, demandé~»}\newline
Prophète, compagnon d'œuvre de Paul avec qui il effectua plusieurs voyages missionnaires. Voir \vref{Ac. 15-18}.

\DicoEntry{SILO}\textit{, de l'hébreu «~Shiyloh~»~: «~lieu de repos~»}\newline
Ville située au nord-est de la tribu d'Ephraïm où les enfants d'Israël se répartirent les territoires avant la conquête de Canaan. Avant d'être placée à Jérusalem, l'arche de l'alliance se trouvait à Silo. Voir \vref{Jos. 18:10}~; \vref{Jos. 19:51}~; \vref{1 S. 3:19-21} et \vref{1 S. 4:3}.

\DicoEntry{SILOÉ}\textit{, de l'hébreu «~Shiloach~»~: «~envoyé~»}\newline
Source d'eau se trouvant au sud-est de Jérusalem. Voir \vref{Né. 3:15} et \vref{Jn. 9:6-7}.

\DicoEntry{SIMÉON}\textit{, de l'hébreu «~Shim`own~»~: «~qui écoute, qui a été entendu~»}\newline
1. Fils de Jacob et Léa. Avec Lévi, son frère, il vengea le déshonneur de sa sœur Dina, en tuant Sichem, prince de Canaan, son père Hamor, et tous leurs hommes. Il fut gardé comme otage en Egypte, lorsque Joseph voulut éprouver la sincérité de ses frères. Il devint le père de la tribu des Siméonites qui s'installèrent au sud de Canaan. Voir \vref{Ge. 29:33}~; \vref{Ge. 34}~; \vref{Ge. 42:21-38} et \vref{Jos. 19:1-9}.
\\2. Homme de foi à qui le Saint-Esprit avait promis qu'il ne mourrait pas sans avoir vu le Messie. Il rencontra Jésus lorsqu'il était enfant, à Jérusalem. Voir \vref{Lu. 2:25-35}.

\DicoEntry{SIMON}\textit{, de l'hébreu «~Shiymown~»~: «~désert~» ou «~qui entend~»}\newline
1. Simon Pierre, le nom originel de Pierre* était Simon. Voir \vref{Jn. 1:40-42}.
\\2. Simon le zélote, il faisait partie du groupuscule des zélotes avant de devenir apôtre de Christ. Voir \vref{Lu. 6:13-16}.
\\3. Simon de Cyrène, il fut contraint d'aider Jésus à porter la croix jusqu'à Golgotha. Voir \vref{Mt. 27:32}.
\\4. Simon le magicien, originaire de la ville de Samarie, il fut baptisé par Philippe et crut pouvoir acheter à prix d'argent la puissance du Saint-Esprit. Voir \vref{Ac. 8:9-24}.

\DicoEntry{SION}\textit{, de l'hébreu «~Tsiyown~»~: «~lieu desséché~»}\newline
Autre nom pour parler de Jérusalem. Sous la Nouvelle Alliance, la montagne de Sion est l'image de la Jérusalem céleste. Voir \vref{De. 4:48}~; \vref{1 R. 8:1}~; \vref{Es. 2:3}~; \vref{2 S. 5:6-7} et \vref{Hé. 12:22}.

\DicoEntry{SISERA}\textit{, de l'hébreu «~Ciycera'~»~: «~déploiement, champ de bataille~»}\newline
Chef de l'armée du roi cananéen Jabin, son armée fut vaincue par Barak et Sisera fut tué par Jaël, femme de Héber, le Kénien. Voir \vref{Jg. 4}.

\DicoEntry{SMYRNE}\textit{, du grec «~Smurna~»~: «~myrrhe~»}\newline
Cité de la côte occidentale de l'Asie Mineure, Smyrne (aujourd'hui Izmir) était située au nord d'Ephèse et réputée pour sa splendeur et ses richesses. Ses forteresses et ses tours de l'acropole évoquaient une couronne. Très unie à Rome, des cultes en l'honneur du dieu Zeus, de la déesse Cybèle, ou encore de l'empereur Tibère et sa mère Julie y étaient célébrés. Proche d'Ephèse, l'église de Smyrne fut probablement le fruit du travail apostolique de Paul. En proie à ces doctrines impies, l'église de Smyrne était fortement persécutée aussi bien par les Romains que par «~les faux Juifs~» membres «~d'une synagogue de Satan~». Sa persévérance face aux afflictions lui permit de recevoir un bon témoignage du Seigneur. Elle incarne l'église persécutée. Voir \vref{Ap. 2:8-11}.

\DicoEntry{SODOME}\textit{, de l'hébreu «~Cedom~»~: «~qui brûle~»}\newline
Ville cananéenne située dans la plaine du Jourdain à proximité de laquelle Lot s'installa après s'être séparé d'Abraham. Ses habitants étaient de grands pêcheurs devant Yahweh à un tel point qu'il détruisit la ville - avec Gomorrhe* - en faisant tomber du ciel une pluie de feu et de soufre. Lot et ses deux filles furent épargnés grâce à l'intercession* d'Abraham. Voir \vref{Ge. 13:10-13} et \vref{Ge. 19:1-29}.

\DicoEntry{SOPHONIE}\textit{, de l'hébreu «~Tsephanyah~»~: «~Yahweh a caché, protégé~»}\newline
Fils de Cuschi, descendant du roi Ezéchias, prophète de Yahweh ayant vécu au temps du roi Josias. L'ensemble de ses prophéties se trouve dans le livre portant son nom.

\DicoEntry{STOÏCIENS}\textit{, du grec «~stoikos~»~: «~appartenant au portique~»}\newline
Adeptes de la doctrine de Zénon de Kition (336 av. J.-C. – 264 av. J.-C.) qui fonda le stoïcisme à Chypre en 301 av. J.-C. Le stoïcisme était l'une des principales doctrines philosophiques de la Grèce antique avec l'épicurisme. Elle reposait sur la morale et la maîtrise de ses sentiments par une vie en conformité avec la nature. A Athènes, quelques stoïciens, accompagnés d'épicuriens, se confrontèrent à Paul, le menant à l'aréopage afin de l'interroger. Voir \vref{Ac. 17:18-20}.

\DicoEntry{SYNAGOGUE}\textit{, du grec «~sunagogue~»~: «~assemblée, lieu de réunion~»}\newline
Assemblée de Juifs réunis pour prier et écouter la lecture des Ecritures. Jésus y enseigna régulièrement pendant son service. Les apôtres annoncèrent également l'Evangile dans des synagogues. Voir \vref{Mt. 4:23}~; \vref{Mt. 9:35}~; \vref{Mc. 6:2} et \vref{Ac. 14:1}.

\DicoEntry{TABERNACLE}\textit{, de l'hébreu «~mishkan~»~: «~sanctuaire, demeure, lieu d'habitation~»}\newline
Appelée aussi tente d'assignation, habitation mobile de Yahweh construite selon le modèle que Dieu donna à Moïse dans le désert. Les Lévites en assuraient le service avec tous les ustensiles qui lui étaient dédiés. Une nuée s'élevait au-dessus du tabernacle pour signifier aux Israélites qu'ils devaient lever le camp* et poursuivre leur marche. Voir \vref{Ex. 25:8-9}~; \vref{Ex. 39:32}~; \vref{No. 1:50-51}~; \vref{Ex. 40:36-38} et \vref{1 Ch. 6:48}.

\DicoEntry{TANAKH}\textit{}\newline
Voir Introduction.

\DicoEntry{TEMPLE}\textit{, de l'hébreu «~heykal~»~: «~palais, temple, sanctuaire~» (voir illustration)}\newline
David projeta de construire un temple pour Yahweh~; son fils Salomon fut mandaté pour l'ériger en remplacement du tabernacle. Il fut détruit une première fois par les Babyloniens au VIème siècle av. J.-C. Reconstruit lors du retour d'exil des Juifs, il fut de nouveau détruit en 70 par les Romains~; il n'en reste qu'un mur aujourd'hui appelé «~mur des lamentations~». Sous la Nouvelle Alliance, Yahweh a choisi pour temple l'Eglise*, le corps de chaque chrétien en qui il vient résider à la naissance d'en haut. Voir \vref{2 S. 7}~; \vref{1 R. 6}~; \vref{2 R. 25:8-9}~; \vref{Esd. 6:15}~; \vref{Ep. 2:21-22} et \vref{1 Co. 6:19}.

\DicoEntry{TÉNÈBRES}\textit{, de l'hébreu «~chosnek~»~: «~obscurité, ténèbres, nuit, lieu caché~»}\newline
Dès la Genèse, la lumière est séparée des ténèbres qui peuvent symboliser le péché, l'ignorance et l'absence de la vie de Dieu. Véritable prison rendant les hommes captifs, les ténèbres éternelles du séjour des morts* seront pour les anges déchus, le diable et tous les méchants. Voir \vref{Ge. 1:2-5}~; \vref{2 S. 22:29}~; \vref{Ps. 107:10}~; \vref{Job 17:13}~; \vref{Ro. 13:12}~; \vref{1 Th. 5:5}~; \vref{2 Pi. 2:4}~; \vref{1 Jn. 2:11} et \vref{Jud. 1:6-13}.

\DicoEntry{TÉRÉBINTHE}\textit{, de l'hébreu «~'elah~»~: «~térébinthe ou chêne~»}\newline
Grand arbre robuste dont l'ombrage est agréable, il est répandu en Israël. Jacob enterra les dieux étrangers de sa maison sous un térébinthe. L'ange de Yahweh apparut sous un térébinthe à Gédéon. Des cultes idolâtres étaient célébrés à l'ombre de ces arbres. C'est à la vallée des térébinthes, située au sud-ouest de Jérusalem, que David tua Goliath. Voir \vref{Ge. 35:4}~; \vref{Jg. 6:11,19}~; \vref{2 S. 18:9}~; \vref{1 Ch. 10:12}~; \vref{Os. 4:13}~; \vref{Es. 57:5} et \vref{1 S. 17:1-50}.

\DicoEntry{TÉTRARQUE}\textit{, du grec «~tetrarches~»~: «~tétrarque~»}\newline
Titre donné au gouverneur d'un territoire sous domination romaine. Hérode Antipas* était le tétrarque de Galilée. Voir \vref{Mt. 14:1} et \vref{Lu. 3:1}.

\DicoEntry{THADÉE}\textit{}\newline
Voir JUDE.

\DicoEntry{THÉRAPHIM}\textit{, de l'hébreu «~teraphiym~»~: «~idolâtries, idoles~»}\newline
Amulette utilisée dans les cultes idolâtres. Rachel déroba les théraphim de son père Laban avant de quitter sa maison. Voir \vref{Ge. 31:19,34-35}.

\DicoEntry{THESSALONIQUE}\textit{, du grec «~Thessalonike~»~: «~victoire de ce qui est faux~»}\newline
Ville située au Nord de la Grèce actuelle, sur la côte de la mer Egée, elle jouissait d'une importante fréquentation puisqu'elle figurait parmi les trois ports principaux de la Méditerranée et se situait sur l'une des plus grandes routes commerciales de l'époque~: la Voie Egnatienne reliant Rome à Byzance. Sur le plan religieux, les habitants étaient polythéistes et pratiquaient une variété de cultes, dont le culte impérial. Durant trois semaines, Paul enseigna dans une synagogue* à Thessalonique~; de là, il réussit à constituer un groupe de croyants. Toutefois, une violente persécution l'obligea à quitter promptement la ville, laissant la communauté nouvellement formée vulnérable et fragile. Il écrivit deux lettres aux saints de Thessalonique qui figurent dans le canon biblique.

\DicoEntry{THOMAS}\textit{, de l'hébreu «~Ta'own~»~: «~jumeau~»}\newline
Surnommé Didyme, il était l'un des douze apôtres*. Dans un premier temps incrédule quant à la résurrection de Jésus, il confessa la Seigneurie de ce dernier lorsqu'il le vit ressuscité. Voir \vref{Lu. 6:12}~; \vref{Jn. 11:16} et \vref{Jn. 20:24-29}.

\DicoEntry{TIMOTHÉE}\textit{, du grec «~Timotheos~»~: «~qui adore, ou honore Dieu~»}\newline
Fils d'une femme juive croyante et d'un père grec. Lié à Paul comme un fils à son père, il devint l'un de ses plus fidèles collaborateurs et l'accompagna à plusieurs reprises dans ses voyages missionnaires. Malgré sa jeunesse, il lui fut confié des tâches liées à la direction des églises, notamment à Ephèse. Timothée reçut de Paul deux lettres regorgeant de conseils et d'instructions pour être un bon serviteur de l'Evangile*. Voir \vref{Ac. 16:1-3}~; \vref{Ac. 18:5}~; \vref{1 Co. 16:10} et les deux épîtres de Paul à Timothée.

\DicoEntry{TITE}\textit{, du grec «~Titos~»~: «~nourrice, honorable~»}\newline
D'origine grecque, Tite fut un fidèle compagnon d'œuvre de l'apôtre Paul. Il l'accompagna à Jérusalem, œuvra à Corinthe et en Dalmatie et s'occupa plus particulièrement de l'église de Crète. Il reçut une lettre de Paul qui figure dans le canon biblique. Voir \vref{2 Co. 8:6,23}~; \vref{Ga. 2:1}~; \vref{2 Ti. 4:10} et l'épître de Paul à Tite.

\DicoEntry{TRIBULATION}\textit{, du grec «~thlipsis~»~: «~une pression, une oppression~»}\newline
Persécution, tourment provoqué par l'annonce de l'Evangile. Inévitables pour entrer dans le royaume de Dieu*, les tribulations ont pour but de rendre le chrétien patient, joyeux et persévérant en toutes circonstances. Voir \vref{Mc. 4:17}~; \vref{Jn. 16:33}~; \vref{Ac. 14:22}~; \vref{2 Co. 6:4}~; \vref{2 Co. 8:2}~; \vref{Ph. 1:29}~; \vref{1 Th. 3:3} et \vref{2 Th. 1:4}.

\DicoEntry{TRIBUNAL}\textit{, de l'hébreu «~qahal~»~: «~assembler, convoquer~» et du grec «~bema~»~: «~tribune~»}\newline
Lieu où les hommes sont jugés afin de recevoir une sentence en fonction des actes qu'ils ont posés. Chaque être humain comparaîtra devant le tribunal de Christ afin de rendre compte pour lui-même. Voir \vref{Ro. 14:10-12} et \vref{2 Co. 5:10}.

\DicoEntry{TRINITÉ}\textit{}\newline
Doctrine selon laquelle le Dieu unique se manifesterait en trois personnes distinctes~: Père, Fils et Saint-Esprit. Inspirée des triades païennes (babylonienne, égyptienne…), cette fausse doctrine d'origine catholique apparut au IIème siècle et fut fixée aux Conciles de Nicée en 325 et de Constantinople I en 381. Elle fut largement reprise par les protestants et la plupart des mouvements chrétiens alors que ni le mot ni le concept de trinité n'apparaissent dans les Ecritures. Voir \vref{De. 6:4}~; \vref{Es. 9:5}~; \vref{Jn. 4:23-24}~; \vref{Col. 2:8-10}~; \vref{2 Th. 1:12} et \vref{1 Jn. 5:20}.

\DicoEntry{TROMPETTE}\textit{, plusieurs mots hébreux ont été traduits par trompette, les plus utilisés sont~: «~chatsotserah~»~: «~trompette, clairon~»~; «~yobel~»~: «~bélier, corne de bélier~», «~retentissant~», «~jubilé~», et «~showphar~»~: «~corne de bélier~». Plusieurs mots grecs ont aussi été utilisés, notamment «~salpigx~»~: «~une trompette~» et «~salpizo~»~: «~sonner de la trompette~».}\newline
Sous l'Ancienne Alliance, on l'utilisait pour donner un signal, publier une sainte convocation, fêter des moments de joie, signifier une victoire, chanter des cantiques en l'honneur de Yahweh, avertir et rassembler le peuple. Le son de la trompette est aussi l'image des voix prophétiques qui crient et appellent le peuple à revenir totalement à Dieu. Selon les Ecritures, lorsque la dernière trompette retentira, l'Eglise sera enlevée pour les noces. Dans le livre d'Apocalypse, la voix du Seigneur est comparée au son d'une trompette. Voir \vref{Ex. 19:13}~; \vref{Lé. 23:24}~; \vref{No. 10:9-10}~; \vref{1 Ch. 16:42}~; \vref{Ez. 33:3}~; \vref{Mt. 24:31}~; \vref{1 Co. 15:52}~; \vref{1 Th. 4:16} et \vref{Ap. 1:10}.

\DicoEntry{TYR}\textit{, de l'hébreu «~Tsor~»~: «~un rocher~»}\newline
Ville de l'antique Phénicie (actuel Liban). Hiram, roi de Tyr, donna- en échange de vivres - du bois de cèdre et du bois de cyprès à Salomon pour la construction du temple. Le roi de Tyr est une image de Satan* dans une prophétie d'Ezéchiel. Voir \vref{1 R. 5:1-12} et \vref{Ez. 28}.

\DicoEntry{UR}\textit{, de l'hébreu «~'Uwr~»~: «~flamme, éclat, feu~»}\newline
Ville de Chaldée située au sud de la Babylonie et d'où Abraham était originaire. Voir \vref{Ge. 11:27-31}.

\DicoEntry{URIE}\textit{, de l'hébreu «~Uwriyah~»~: «~Yahweh est ma lumière~»}\newline
Héthien, mari de Bath-Schéba. Il mourut sur le champ de bataille suite à une conspiration de David qui avait connu sa femme et l'avait mise enceinte. Voir \vref{2 S. 11}.

\DicoEntry{VIE ÉTERNELLE}\textit{, de l'hébreu «~aionios~»~: «~sans commencement ni fin~»}\newline
La vie éternelle est un don gratuit de Dieu, un héritage, une promesse qui commence dès la conversion au travers de la connaissance de Dieu. La vie éternelle est Christ lui-même. Voir \vref{Jn. 3:16,36}~; \vref{Ro. 2:7}~; \vref{Ro. 6:23}~; \vref{Tit. 3:7}~; \vref{1 Jn. 2:25} et \vref{1 Jn. 5:20}.

\DicoEntry{VIGNE}\textit{}\newline
Arbre cultivé pour son fruit, la vigne est assimilée à la joie à cause du vin produit par le raisin et consommé dans le cadre de festivités. Le peuple d'Israël était la première vigne de Yahweh, mais elle ne porta pas de fruits. Le royaume de Dieu est aussi associé à la vigne~: Dieu est le vigneron, Jésus-Christ est le cep et tous les enfants de Dieu sont les sarments. Tout sarment qui ne porte pas de fruits est jeté au feu, c'est-à-dire en enfer. Voir \vref{Es. 5:1-7}~; \vref{Mt. 21:33-43} et \vref{Jn. 15:1-8}.

\DicoEntry{VOILE}\textit{, de l'hébreu «~porokhet~»~: «~rideau, voile~» et du grec «~peribolaion~»~: «~une couverture, une enveloppe~»}\newline
1. Etoffe de fin lin retors qui servait de séparation entre le lieu saint et le Saint des saints. Lorsque Jésus-Christ fut crucifié, ce voile se déchira en deux, de haut en bas, ouvrant ainsi l'accès au Saint des saints. Cet événement symbolisait que, par Jésus, tout homme pouvait accéder librement à la présence du Père. Voir \vref{Ex. 26:31-33}~; \vref{Lé. 16:11-19}~; \vref{Mt. 27:50-51}~; \vref{Hé. 9:7-8} et \vref{Hé. 10:19-20}.
\\2. Pan de tissu utilisé pour se couvrir la tête dans certaines cultures. Paul expliqua que les longs cheveux étaient une gloire pour la femme et qu'ils faisaient office de voile naturel. Voir \vref{Ge. 24:65} et \vref{1 Co. 11:15}.
\\3. Au sens figuré, le voile symbolise l'intelligence obscurcie, le cœur non converti et le manque de révélation de la parole qui sont des barrières à la compréhension de la loi. Voir \vref{2 Co. 3:14-16}.

\DicoEntry{YHWH}\textit{}\newline
Aussi appelé tétragramme (mot de quatre lettres), nom avec lequel Dieu se révéla à Moïse lorsque ce dernier le rencontra pour la première fois à Horeb. Ce nom, prononcé Yahweh, signifie «~Je suis celui qui suis~» et souligne le caractère éternel de Dieu. Voir \vref{Ex. 3:1-14}.

\DicoEntry{ZABULON}\textit{, de l'hébreu «~Zebuwluwn~»~: «~habitation~»}\newline
Fils de Jacob et Léa, il devint l'ancêtre de la tribu de Zabulon. Voir \vref{Ge. 30:19-20} et \vref{No. 2:7}.

\DicoEntry{ZACHARIE}\textit{, de l'hébreu «~Zekaryah~»~: «~Yahweh se souvient~»}\newline
1. Fils de Jéroboam, roi d'Israël sur qui il régna uniquement six mois. Il fit ce qui est mal devant Yahweh et fut tué suite à une conspiration contre lui. Voir \vref{2 R. 15:8-11}.
\\2. Prophète et prêtre, fils de Bérékia et petit-fils d'Iddo. Avec le prophète Aggée, il assista Zorobabel, gouverneur de Juda, et Josué, grand prêtre, dans la restauration du temple de Yahweh au retour de la captivité des Juifs. L'ensemble de ses prophéties se trouve dans le livre portant son nom. Voir \vref{Esd. 5:1-2} et \vref{Esd. 6:14-5}.
\\3. Prêtre et père de Jean-Baptiste qu'il eut avec sa femme Elisabeth à un âge avancé. Voir \vref{Lu. 1:5}.

\DicoEntry{ZÉLOTE}\textit{, du grec «~zelotes~»~: «~celui qui est zélé~»}\newline
Patriotes juifs fervents défenseurs de la loi et des traditions ayant pour objectif de résister à l'invasion romaine. Simon, l'un des douze apôtres, en faisait partie. Voir \vref{Lu. 6:15} et \vref{Ac. 1:13}.

\DicoEntry{ZOROBABEL}\textit{, de l'hébreu «~Zerubbabel~»~: «~rejeton de Babylone~»}\newline
Fils de Schealthiel, gouverneur de Juda, il participa à la restauration du temple* de Yahweh après le retour de la captivité du peuple juif. Il figure dans la généalogie de Jésus. Voir \vref{Esd. 3:2}~; \vref{Esd. 5:2}~; \vref{Ag. 1:14}~; \vref{Mt. 1:13} et \vref{Lu. 3:27}.

\end{multicols}
\clearpage}\markboth{}{}
%    \makeatletter
%    % réinitialiser mise en forme
%    \def\@oddhead{\hfil\thepage\hfil}
%    \def\@evenhead{\hfil}
%\makeatother
%% tableaux
%\addcontentsline{toc}{section}{Histoire de la Bible}\clearpage
%\begin{center}Histoire de la Bible 1\end{center}\clearpage
%\begin{center}Histoire de la Bible 2\end{center}\clearpage
%\addcontentsline{toc}{section}{Dénominations}\clearpage
%\begin{center}Dénominations 1\end{center}\clearpage
%\begin{center}Dénominations 2\end{center}\clearpage
%\addcontentsline{toc}{section}{Doctrines}\clearpage
%\begin{center}Doctrines 1\end{center}\clearpage
%\begin{center}Doctrines 2\end{center}\clearpage
%\begin{center}Doctrines 3\end{center}\clearpage
%\begin{center}Doctrines 4\end{center}\clearpage
%\begin{center}Doctrines 5\end{center}\clearpage
%\begin{center}Doctrines 6\end{center}\clearpage
%\begin{center}Doctrines 7\end{center}\clearpage
%\begin{center}Doctrines 8\end{center}\clearpage
%\begin{center}Doctrines 9\end{center}\clearpage
%\begin{center}Doctrines 10\end{center}\clearpage
%\begin{center}Doctrines 11\end{center}\clearpage
%\addcontentsline{toc}{section}{Monnaies}\clearpage
%\begin{center}Monnaies\end{center}\clearpage
%\addcontentsline{toc}{section}{Longueurs / Liquides}\clearpage
%\begin{center}Longueurs / Liquides\end{center}\clearpage
%\addcontentsline{toc}{section}{Poids}\clearpage
%\begin{center}Poids\end{center}\clearpage
%\addcontentsline{toc}{section}{Fêtes de Yahweh}\clearpage
%\begin{center}Fêtes de Yahweh\end{center}\clearpage
%\addcontentsline{toc}{section}{Alphabet hébreu}\clearpage
%\begin{center}Alphabet hébreu\end{center}\clearpage
\end{document}

\clearpage\ShortTitle{Esther}\BookTitle{Esther}\BFont
\noindent\hrulefill
{\footnotesize
\textit{
\bigskip
{\centering{}
\\(Ecter)
\\Signifie : Etoile (persan) ; Myrthe (hébreu)
\\Thème : Délivrance des juifs de l’extermination
\\Auteur : Inconnu
\\Date de rédaction : 5ème siècle av. J.C.\\}
}
%\bigskip
\textit{
\\Dernier livre à caractère historique du Tanahk, l’histoire d’Esther se déroula à Suse, capitale du royaume de Perse. En ce temps, le peuple d’Israël était dispersé et le roi Assuérus régnait sur un large territoire allant de l’Inde à l'Ethiopie.
%\bigskip
\\Ce livre raconte la vie d’Esther, son ascension au trône royal où elle succéda à la reine Vasthi et la manière dont elle fut utilisée pour éviter le génocide du peuple juif.  
%\bigskip
\\Bien que ne comportant pas le nom de Dieu ni d’allusion à une œuvre spirituelle, hormis le jeûne, ce récit met en évidence le secours divin.\bigskip
}
}
\par\nobreak\noindent\hrulefill
\begin{multicols}{2}
\TextTitle{[Un festin de sept jour au palais de Suse]}
\Chap{1}
\VerseOne{}Or il arriva qu’au temps d’Assuérus, de cet Assuérus qui régnait depuis les Indes jusqu'en Ethiopie, sur cent vingt-sept provinces ;
\VS{2}[Il arriva, dis-je], en ce temps-là, que le roi Assuérus était assis sur le trône royal à Suse, dans la capitale.
\VS{3}La troisième année de son règne, il fit un festin à tous les principaux princes de ses pays ; à ses serviteurs, à l’armée des Perses et de Mèdes, aux nobles et aux chefs des provinces qui furent réunis devant lui,
\VS{4}pour leur montrer la gloire de la richesse de son royaume et la splendeur de sa grandeur, durant plusieurs jours, pendant cent quatre-vingts jours.
\VS{5}Lorsque ces jours furent achevés, le roi fit pour tout le peuple qui se trouvait à Suse, la capitale, depuis le plus grand jusqu'au plus petit, un festin pendant sept jours, dans la cour du jardin du palais royal.
\VS{6}Des étoffes blanches, vertes et violettes, étaient attachées par des cordons de byssus et de pourpre à des anneaux d'argent et à des colonnes de marbre. Les lits étaient d'or et d'argent sur un pavé de porphyre, de marbre, de nacre, et de pierres noires.
\VS{7}On servait à boire dans des vases d'or, de différentes espèces, et il y avait du vin royal en abondance, selon la libéralité du roi.
\VS{8}On ne forçait personne à boire, car le roi avait ordonné à tous les chefs de sa maison de se conformer à la volonté de chacun.
\VS{9}La reine Vasthi fit aussi un festin aux femmes dans la maison royale du roi Assuérus.
\TextTitle{[Destitution de la reine Vasthi]}
\VS{10}Or le septième jour, le cœur du roi était réjoui par le vin, il ordonna à Mehuman, Biztha, Harbona, Bigtha, Abagtha, Zéthar, et Carcas, les sept eunuques qui servaient devant le roi Assuérus,
\VS{11}d’amener en sa présence la reine Vasthi, portant la couronne royale, afin de montrer sa beauté aux peuples et aux princes, car elle était belle de figure.
\VS{12}Les eunuques transmirent l’ordre du roi à la reine Vasthi, mais elle refusa de venir. Et le roi fut très irrité, et il s’enflamma de colère.
\VS{13}Alors le roi dit aux sages qui avaient la connaissance des temps. Car le roi traitait ainsi les affaires en présence de tous ceux qui connaissaient les lois et le droit.
\VS{14}Il avait auprès de lui, Carschena, Schéthar, Admatha, Tarsis, Mérès, Marsena, Memucan, sept princes de Perse et de Médie, qui voyaient la face du roi et qui occupaient le premier rang dans le royaume.
\VS{15}Que faut-il faire dit-il, selon les lois, à la reine Vasthi, pour n'avoir pas observé l’ordre que le roi Assuérus lui a ordonné par les eunuques ?
\VS{16}Alors Memucan répondit en présence du roi et des princes : La reine Vasthi n'a pas seulement mal agi contre le roi, mais aussi contre tous les princes et tous les peuples qui sont dans toutes les provinces du roi Assuérus.
\VS{17}Car l'action de la reine parviendra à la connaissance de toutes les femmes, et les portera à mépriser leurs maris ; elles diront : Le roi Assuérus avait ordonné qu'on fasse venir en sa présence la reine, et elle n'y est pas allée.
\VS{18}Dès ce jour, les princesses de Perse et de Médie qui auront appris l’action de la reine répondront de même à tous les princes du roi ; ce sera une marque de mépris et un sujet de colère.
\VS{19}Si le roi le trouve bon, qu'un édit royal soit publié de sa part, et qu'il soit écrit parmi les lois de Perse et de Médie, avec défense de la transgresser, que Vasthi ne vienne plus devant le roi Assuérus et le roi donnera sa royauté à une compagne, qui sera meilleure qu'elle.
\VS{20}L’édit du roi sera présenté et connu dans tout son royaume, quelque grand qu'il soit, toutes les femmes honoreront leurs maris\FTNT{Respect ou soumission de la femme à l’égard de son mari : Ep. 5 : 22 ; Col. 3 : 18 ; Ti. 2 : 5 ; 1 Pi. 3 : 1-5.}, depuis le plus grand jusqu'au plus petit.
\VS{21}Cette parole plut au roi et aux princes, et le roi fit selon la parole de Memucan.
\VS{22}Il envoya des lettres à toutes les provinces du royaume, à chaque province selon son écriture et à chaque peuple selon sa langue ; elles portaient que tout homme devait être le maître de sa maison\FTNT{L’homme, chef de la femme et maître de la maison : 1 Co. 11 : 3 ; Ep. 5 : 23.}, et qu’il parlerait la langue de son peuple.
\TextTitle{[Le roi choisit une autre reine]}
\Chap{2}
\VerseOne{}Après ces choses, quand la colère du roi Assuérus fut calmée, il se souvint de Vasthi, de ce qu'elle avait fait, et de ce qui avait été décrété à son sujet.
\VS{2}Les serviteurs qui servaient le roi dirent : Qu'on cherche pour le roi des jeunes filles, vierges, et belles de figure.
\VS{3}Que le roi désigne des commissaires dans toutes les provinces de son royaume chargés de rassembler toutes les jeunes filles, vierges et belles de figure, dans Suse, la capitale, dans la maison des femmes sous la charge d'Hégué, eunuque du roi et gardien des femmes, qu'on leur donne les parfums nécessaires pour leur toilette ;
\VS{4}et la jeune fille qui plaira au roi régnera à la place de Vasthi. Ce discours plût au roi, et il fit ainsi.
\VS{5}Or, il y avait à Suse, la capitale, un juif nommé Mardochée, fils de Jaïr, fils de Schimeï, fils de Kis, Benjamite,
\VS{6}qui avait été emmené de Jérusalem\FTNT{La captivité babylonienne : Voir 2 R. 24.}, parmi les captifs déportés avec Jeconia, roi de Juda, par Nebucadnetsar, roi de Babylone.
\VS{7}Il élevait Hadassa, qui est Esther, fille de son oncle ; car elle n'avait ni père ni mère. La jeune fille était belle de taille et très belle de figure. Après la mort de son père et de sa mère, Mardochée l'avait prise pour fille.
\VS{8}Lorsqu’on eut publié l’ordre du roi et son édit, un grand nombre de jeunes filles furent rassemblées à Suse, la capitale, sous la charge d'Hégaï. Esther fut aussi amenée dans la maison du roi, sous la charge d'Hégaï, gardien des femmes.
\VS{9}La jeune fille lui plut, et trouva grâce à ses yeux, il s’empressa de lui fournir les parfums nécessaires pour sa toilette, et pour sa subsistance, lui donna sept jeunes filles choisies, et établies dans la maison du roi, il lui fit changer d'appartement, et la logea, elle et ses servantes, dans le meilleur des appartements de la maison des femmes.
\VS{10}Esther ne fit connaître ni son peuple ni sa parenté, car Mardochée lui avait ordonné de ne rien raconter.
\VS{11}Tous les jours Mardochée allait et venait devant la cour de la maison des femmes, pour savoir comment se portait Esther, et comment on s’occupait d'elle.
\VS{12}Chaque jeune fille allait à son tour vers le roi Assuérus, après s’être conformée au décret concernant les femmes pendant douze mois\FTNT{Esther se soumit à une toilette particulière avant de rencontrer le roi. Le mot « toilette » vient de l’hébreu « tam-rook », qui signifie « grattement ». La racine de ce mot signifie « nettoyer », « purifier », « polir » (voir Lé. 6 : 28 ; Jé. 46 : 4). Ce grattage symbolise le dépouillement du vieil homme et le renoncement aux œuvres de la chair (Ep. 4 : 22).
Douze mois étaient nécessaires pour préparer Esther aux noces : Six mois avec de l’huile de myrrhe et six mois avec des aromates et des parfums. La myrrhe était l’une des composantes de l’onction sainte dont on s’est servi pour oindre notamment la tente d’assignation, l’arche du témoignage ainsi qu’Aaron et ses fils (Ex. 30 : 23-30). Cet aspect de la toilette d’Esther nous parle de la sanctification sans laquelle nul ne peut voir le Seigneur (Hé. 12 : 14). La myrrhe est par ailleurs citée à sept reprises dans le livre du Cantique des cantiques, véritable hymne de l’amour parfait qui lie Christ à son Eglise. Le parfum quant à lui symbolise les prières que nous devons faire en tout temps afin de maintenir notre communion avec Jésus, notre époux (Ap. 5 : 8 ; Ap. 8 : 4 ; Ep. 6 : 18 ; 1 Th. 5 : 17).
Ainsi, à l’instar d’Esther qui se préparait à rencontrer le roi, l’Eglise se prépare depuis deux mille ans pour les noces de l’agneau (Ap. 19 : 7-9).}. C'est ainsi que s'accomplissaient les jours de leurs préparatifs, six mois avec de l'huile de myrrhe, et six autres mois avec des aromates et des parfums en usage parmi les femmes.
\VS{13}C'est ainsi que la jeune fille entrait vers le roi ; et, quand elle passait de la maison des femmes à la maison du roi, on lui laissait prendre ce qu’elle voulait.
\VS{14}Elle y entrait le soir, et le matin elle retournait dans la seconde maison des femmes sous la charge de Schaaschgaz, eunuque du roi et gardien des concubines. Elle ne retournait plus vers le roi, à moins que le roi n’en ait le désir et qu'elle soit appelée par son nom.
\TextTitle{[Esther, reine de Suse]}
\VS{15}Quand son tour d’aller vers le roi fut arrivé, Esther, fille d'Abichaïl, oncle de Mardochée qui l’avait prise pour sa fille, ne demanda rien sinon ce qui fut ordonné par Hégaï, eunuque du roi et gardien des femmes. Esther trouva grâce aux yeux de tous ceux qui la voyaient.
\VS{16}Ainsi Esther fut amenée auprès du roi Assuérus, dans sa maison royale, le dixième mois, qui est le mois de Tébeth, la septième année de son règne.
\VS{17}Le roi aima Esther plus que toutes les autres femmes, elle obtint sa grâce et sa bienveillance plus que toutes les vierges. Il mit la couronne royale sur sa tête, et l'établit reine à la place de Vasthi.
\VS{18}Le roi fit alors un grand festin à tous les princes de ses pays, et à ses serviteurs, un festin en l’honneur d'Esther ; il donna du repos aux provinces, et fit des présents selon la puissance du roi.
\VS{19}Or pendant qu'on assemblait les vierges pour la seconde fois, Mardochée s’assit à la porte du roi.
\VS{20}Esther n’avait fait connaître ni sa parenté ni son peuple, car Mardochée le lui avait défendu. Elle faisait tout ce que lui disait Mardochée, comme à l’époque où elle était élevée par lui.
\TextTitle{[Mardochée sauve la vie du roi]}
\VS{21}En ces jours-là, Mardochée s’assit à la porte du roi, Bigthan et Théresch, deux eunuques du roi, gardes du seuil, s’irritèrent et cherchèrent à mettre la main sur le roi Assuérus.
\VS{22}Mardochée ayant eu connaissance de l’affaire, informa la reine Esther, qui le redit au roi de la part de Mardochée.
\VS{23}On vérifia l’affaire et on trouva que cela était exact, les deux eunuques furent pendus à un bois, et cela fut écrit dans le livre des chroniques en présence du roi.
\TextTitle{[Conspiration de Haman contre les Juifs]}
\Chap{3}
\VerseOne{}Après ces choses, le roi Assuérus fit de grands honneurs à Haman, fils d'Hammedatha, l’Agaguite ; il l'éleva en dignité et plaça son siège au-dessus de tous les princes qui étaient auprès de lui.
\VS{2}Tous les serviteurs du roi qui étaient à la porte du roi s'inclinaient et se prosternaient devant Haman, car le roi l’avait ainsi ordonné. Mais Mardochée ne s'inclinait pas et ne se prosternait pas devant lui.
\VS{3}Les serviteurs du roi, qui étaient à la porte du roi, disaient à Mardochée : Pourquoi transgresses-tu l’ordre du roi ?
\VS{4}Comme ils le lui répétaient chaque jour et qu'il ne les écoutait pas, ils le rapportèrent à Haman, pour voir si Mardochée tiendrait ferme dans sa résolution ; car il leur avait déclaré qu'il était juif.
\VS{5}Haman vit que Mardochée ne s'inclinait pas et ne se prosternait pas devant lui et il fut rempli de colère.
\VS{6}Mais il dédaigna de porter la main sur Mardochée seul, car on lui avait rapporté de quel peuple était Mardochée. Haman chercha à exterminer tous les juifs, le peuple de Mardochée qui se trouvait dans tout le royaume d'Assuérus.
\VS{7}Au premier mois, qui est le mois de Nissan, la douzième année du roi Assuérus, on jeta le pur, c'est-à-dire le sort, devant Haman, pour chaque jour et pour chaque mois, jusqu’au douzième mois, qui est le mois d'Adar.
\VS{8}Haman dit au roi Assuérus : Il y a un peuple dispersé dans toutes les provinces de ton royaume, qui se tient à part parmi les peuples. Leurs lois sont différentes de celles de tous les autres peuples, ils n’observent pas les lois du roi. Il n'est pas dans l’intérêt du roi de le laisser en repos.
\VS{9}S'il plaît au roi, qu'on écrive l’ordre de les faire périr, et je pèserai dix mille talents d'argent entre les mains de ceux qui s’occupent des affaires, pour les porter dans le trésor du roi.
\VS{10}Le roi ôta son anneau de sa main, et le donna à Haman fils de Hammedatha, l’Agaguite, l’adversaire des Juifs.
\VS{11}Outre cela, le roi dit à Haman : Cet argent t'est donné avec ce peuple ; fais-en ce que tu voudras.
\VS{12}Le treizième jour du premier mois, les secrétaires du roi furent appelés, et on écrivit selon l’ordre d'Haman, aux satrapes du roi, aux gouverneurs de chaque province et aux princes de chaque peuple, à chaque province selon son écriture et à chaque peuple selon sa langue. Ce fut au nom du roi Assuérus que l’on écrivit, et on scella avec l'anneau du roi.
\VS{13}Les lettres furent envoyées par des coureurs dans toutes les provinces du roi, afin qu'on extermine, qu’on tue et qu’on fasse périr tous les juifs, jeunes et vieux, petits enfants et femmes, en un seul jour, le treizième du douzième mois, qui est le mois d'Adar, et pour que leurs biens soient livrés au pillage.
\VS{14}Ces lettres qui furent écrites portaient une copie de l’édit, qui devait être publié dans chaque province, et invitaient publiquement tous les peuples, à se tenir prêts pour ce jour-là.
\VS{15}Ainsi les coureurs partirent en toute hâte d’après l’ordre du roi. L'édit fut aussi publié dans Suse, la capitale. Or le roi et Haman étaient assis pour boire, pendant que la ville de Suse était dans la confusion.
\TextTitle{[Esther avertie du complot d'Haman]}
\Chap{4}
\VerseOne{}Mardochée, ayant appris ce qui se passait, déchira ses vêtements et se couvrit d'un sac et de la cendre. Puis il alla au milieu de la ville en poussant avec force des cris amers,
\VS{2}et se rendit jusqu'à la porte du roi, or il était interdit d'entrer dans le palais du roi revêtu d'un sac.
\VS{3}Dans chaque province, partout où arrivait l’ordre du roi et son édit, il y eut une grande désolation parmi les juifs ; ils jeûnaient, pleuraient, gémissaient, et beaucoup se couchaient sur le sac et la cendre.
\VS{4}Les servantes d'Esther et ses eunuques vinrent lui raconter ces choses, et la reine fut très effrayée. Elle envoya des vêtements à Mardochée pour le couvrir et lui faire ôter son sac, mais il ne les prit pas.
\VS{5}Alors Esther appela Hathac, l'un des eunuques que le roi avait établi pour la servir, et elle le chargea de demander à Mardochée ce qui s’était passé et pourquoi il agissait ainsi.
\VS{6}Hathac sortit donc vers Mardochée sur la place de la ville, devant la porte du roi.
\VS{7}Mardochée lui raconta tout ce qui lui était arrivé, et la somme d'argent qu'Haman avait promis de payer comptant au trésor du roi, pour la destruction des juifs.
\VS{8}Il lui donna aussi une copie de l'édit publié dans Suse en vue de leur extermination, afin qu’il le montre à Esther et lui fasse tout connaître ; et il ordonna qu’Esther se rende chez le roi pour implorer sa miséricorde, et faire une requête en faveur de son peuple.
\VS{9}Hathac vint rapporter à Esther les paroles de Mardochée.
\TextTitle{[Mardochée incite Esther à risquer sa vie pour ses frères]}
\VS{10}Esther chargea Hathac de dire à Mardochée :
\VS{11}Tous les serviteurs du roi et le peuple des provinces du roi savent qu'il existe une loi prescrivant la peine de mort contre quiconque, homme ou femme, entre chez le roi, dans la cour intérieure sans avoir été appelé ; à moins que le roi ne lui tende le sceptre d'or, celui-là a la vie sauve. Or il y a déjà trente jours que je n'ai pas été appelée pour entrer chez le roi.
\VS{12}On rapporta les paroles d'Esther à Mardochée.
\VS{13}Mardochée fit cette réponse à Esther : Ne t’imagine pas que tu échapperas seule d'entre tous les juifs parce que tu es dans la maison du roi.
\VS{14}Mais si tu te tais et gardes le silence en ce temps-ci, les juifs seront secourus et délivrés par un autre moyen, mais toi et la maison de ton père vous périrez. Et qui sait si tu n'es pas arrivée à la royauté pour un temps comme celui-ci ?
\TextTitle{[Esther demande un jeûne]}
\VS{15}Esther fit cette réponse à Mardochée :
\VS{16}Va, rassemble tous les juifs qui se trouvent à Suse, et jeûnez pour moi, sans manger ni boire pendant trois jours, ni la nuit ni le jour. Moi aussi et mes servantes nous jeûnerons de même, puis j'entrerai chez le roi, malgré la loi ; et si je dois périr, je périrai.
\VS{17}Mardochée s'en alla, et fit comme Esther lui avait ordonné.
\TextTitle{[Esther se présente devant le roi]}
\Chap{5}
\VerseOne{}Le troisième jour, Esther mit des vêtements royaux et se présenta dans la cour intérieure de la maison du roi, devant la maison du roi. Le roi était assis sur le trône dans la maison royale, en face de l’entrée de la maison.
\VS{2}Dès que le roi vit la reine Esther debout dans la cour, elle trouva grâce à ses yeux ; le roi tendit à Esther le sceptre d'or qui était dans sa main. Esther s'approcha, et toucha le bout du sceptre.
\VS{3}Le roi lui dit : Qu'as-tu, reine Esther, et que demandes-tu ? Quand ce serait la moitié du royaume, elle te serait donnée.
\VS{4}Esther répondit : Si le roi le trouve bon, que le roi vienne aujourd'hui avec Haman au festin que je lui ai préparé.
\VS{5}Alors le roi dit : Qu'on fasse venir en toute hâte, Haman, pour accomplir la parole d'Esther. Le roi vint donc avec Haman au festin qu'Esther avait préparé.
\VS{6}Le roi dit à Esther, pendant qu’on buvait le vin : Quelle est ta demande ? Elle te sera accordée. Quelle est ta requête ? Quand ce serait la moitié du royaume, tu l’obtiendras.
\VS{7}Esther répondit et dit : Voici ce que je demande et ce que je désire.
\VS{8}Si j'ai trouvé grâce aux yeux du roi, et si le roi trouve bon d'accorder ma requête, que le roi et Haman viennent au festin que je leur préparerai, et je donnerai demain une réponse au roi selon sa parole.
\VS{9}Haman sortit ce jour-là, joyeux et le cœur content. Mais aussitôt qu'il vit, à la porte du roi, Mardochée, qui ne se levait ni ne tremblait devant lui, il fut rempli de colère contre Mardochée.
\VS{10}Il sut toutefois se contenir, et il alla dans sa maison. Puis il envoya chercher ses amis et Zéresch, sa femme.
\VS{11}Haman leur parla de la magnificence de ses richesses, du nombre de ses fils, et tout ce qu’avait fait le roi pour le rendre puissant, et comment il l'avait élevé au-dessus des princes et des serviteurs du roi.
\VS{12}Puis Haman ajouta : Même la reine Esther n'a fait venir que moi et le roi au festin qu'elle a fait, et je suis encore invité demain chez elle avec le roi.
\VS{13}Mais tout cela n’est d’aucun intérêt, aussi longtemps que je verrai Mardochée, le juif, assis à la porte du roi.
\VS{14}Zéresch sa femme, et tous ses amis lui répondirent : Qu'on prépare un bois haut de cinquante coudées, et demain matin dis au roi qu'on y pende Mardochée ; et tu iras joyeux au festin avec le roi. Cette parole plut à Haman, et il fit préparer le bois.
\TextTitle{[Le roi Assuérus se souvient de Mardochée]}
\Chap{6}
\VerseOne{}Cette nuit-là, le roi ne put dormir, il fit apporter le livre des annales, les chroniques. On les lut devant le roi,
\VS{2}et l’on trouva écrit ce que Mardochée avait rapporté au sujet de la conspiration de Bigthan et de Théresch, les deux eunuques du roi, gardes du seuil, qui avaient cherché à mettre la main sur le roi Assuérus.
\VS{3}Le roi dit : Quel honneur et quelle distinction a-t-on accordé à Mardochée pour cela ? Il n’a rien reçu répondirent les serviteurs du roi.
\VS{4}Le roi dit : Qui est dans la cour ? Haman était venu dans la cour extérieure de la maison du roi, pour demander au roi de pendre Mardochée au bois qu'il avait préparé.
\VS{5}Les serviteurs du roi répondirent : C’est Haman qui se tient dans la cour. Et le roi dit : Qu'il entre.
\VS{6}Haman entra, et le roi lui dit : Que faudrait-il faire à un homme que le roi désire honorer ? Haman se dit en lui-même : A qui le roi voudrait-il faire plus honneur qu'à moi ?
\VS{7}Haman répondit au roi : Pour un homme que le roi désire honorer,
\VS{8}qu’on lui apporte le vêtement royal, dont le roi se revêt, et qu'on lui amène le cheval que le roi monte, et qu'on lui mette la couronne royale sur la tête.
\VS{9}Et qu'ensuite on donne ce vêtement et ce cheval à quelqu'un des principaux et des plus grands chefs qui sont auprès du roi, et qu'on revête l'homme que le roi prend plaisir d'honorer, et qu'on le fasse aller à cheval par les rues de la ville ; et qu'on crie devant lui : C'est ainsi qu'on doit faire à l'homme que le roi prend plaisir d'honorer.
\VS{10}Alors le roi dit à Haman : Prends tout de suite le vêtement, et le cheval, comme tu l'as dit, et fais ainsi à Mardochée, le juif qui est assis à la porte du roi ; ne néglige rien de tout ce que tu as déclaré.
\VS{11}Et Haman prit le vêtement et le cheval, il revêtit Mardochée, il le promena à cheval à travers les rues de la ville, et il criait devant lui : C'est ainsi que l’on fait à l'homme que le roi désire honorer.
\VS{12}Mardochée retourna à la porte du roi, et Haman se retira en hâte dans sa maison, pleurant et ayant la tête voilée.
\VS{13}Haman raconta à Zéresch, sa femme, et à tous ses amis, tout ce qui lui était arrivé. Ses sages, et Zéresch, sa femme, lui répondirent : Si Mardochée devant lequel tu as commencé à tomber, est de la race des juifs, tu n'auras pas le dessus sur lui, mais tu tomberas certainement devant lui.
\VS{14}Comme ils parlaient encore avec lui, les eunuques du roi vinrent, et se hâtèrent d'amener Haman au festin qu'Esther avait préparé.
\TextTitle{[Esther plaide sa cause et celle de son peuple]}
\Chap{7}
\VerseOne{}Le roi et Haman allèrent au festin chez la reine Esther.
\VS{2}Le roi dit encore à Esther, ce second jour, pendant qu’on buvait le vin : Quelle est ta demande, reine Esther ? Elle te sera donnée. Que désires-tu ? Quand ce serait la moitié du royaume, cela te sera accordé.
\VS{3}Alors la reine Esther répondit, et dit : Si j'ai trouvé grâce à tes yeux, ô roi ! et si le roi le trouve bon, que ma vie me soit donnée à ma demande, et que mon peuple me soit donné à ma prière.
\VS{4}Car nous avons été vendus, mon peuple et moi, pour être détruits, tués, exterminés. Si nous avions été vendus pour être esclaves et serviteurs, j’aurais gardé le silence, bien que l'oppresseur ne saurait compenser le dommage fait au roi.
\VS{5}Le roi Assuérus parla et dit à la reine Esther : Qui est-il et où est l’homme dont le cœur est consacré à faire cela ?
\VS{6}Esther répondit : L'oppresseur, l'ennemi, c’est Haman, ce méchant ! Alors Haman fut terrifié en présence du roi et de la reine.
\TextTitle{[Haman pendu au gibet qu'il avait dressé]}
\VS{7}Le roi, dans sa colère, se leva et quitta le festin, il entra dans le jardin du palais. Haman resta pour demander grâce pour sa vie à la reine Esther, car il voyait bien que sa perte était résolue par le roi.
\VS{8}Puis le roi revint du jardin du palais dans la salle du festin, il vit Haman qui s’était précipité sur le lit où était Esther, et il dit : Serait-ce encore pour faire violence sous mes yeux à la reine dans cette maison ? Dès que la parole fut sortie de la bouche du roi, on voila le visage d'Haman.
\VS{9}Et Harbona, l'un des eunuques, dit en présence du roi : Voici, le bois préparé par Haman pour Mardochée, qui a parlé pour le bien du roi, est dressé dans la maison d'Haman, à une hauteur de cinquante coudées. Le roi dit : Qu’on y pende Haman !
\VS{10}On pendit Haman au bois qu'il avait préparé pour Mardochée. Et la colère du roi fut apaisée.
\TextTitle{[Un décret royal fait échouer le complot d'Haman]}
\Chap{8}
\VerseOne{}Ce même jour, le roi Assuérus donna à la reine Esther la maison d'Haman, l'oppresseur des juifs ; et Mardochée fut introduit devant le roi, car Esther avait déclaré quel était son lien de parenté avec elle.
\VS{2}Le roi ôta son anneau, qu'il avait repris à Haman, et le donna à Mardochée ; Esther établit Mardochée sur la maison d'Haman.
\VS{3}Esther parla encore en présence du roi. Elle se jeta à ses pieds, elle pleura, elle l’implora d’empêcher les effets de la méchanceté d'Haman, l’Agaguite, et la réussite de ses projets contre les juifs.
\VS{4}Le roi tendit le sceptre d'or à Esther qui se releva et resta debout devant le roi.
\VS{5}Elle dit : Si le roi le trouve bon, et si j'ai trouvé grâce devant lui, si mes paroles semblent convenables au roi et si je suis agréable à ses yeux, qu'on écrive pour révoquer les lettres conçues par Haman, fils d'Hammedatha, l’Agaguite, qu'il écrivit afin de détruire les juifs qui sont dans toutes les provinces du roi.
\VS{6}Car comment pourrais-je voir le mal qui atteindrait mon peuple, et comment pourrais-je voir la destruction de ma race ?
\VS{7}Le roi Assuérus dit à la reine Esther et au juif Mardochée : Voici, j'ai donné la maison d'Haman à Esther, et il a été pendu au bois pour avoir étendu sa main contre les juifs.
\VS{8}Ecrivez donc, au nom du roi, en faveur des juifs comme il vous plaira, et scellez l'écrit de l'anneau du roi ; car un édit écrit au nom du roi et scellé de l'anneau du roi ne peut être révoqué.
\VS{9}En ce temps, le vingt-troisième jour du troisième mois, qui est le mois de Sivan, les secrétaires du roi furent appelés, et on écrivit, comme Mardochée l’ordonna, aux juifs, aux satrapes, aux gouverneurs, et aux princes des cent vingt-sept provinces, de l’Inde jusqu'en Ethiopie, à chaque province selon son écriture, à chaque peuple selon sa langue, et aux juifs selon leur écriture et selon leur langue.
\VS{10}On écrivit les lettres au nom du roi Assuérus, et on les scella de l'anneau du roi. On les envoya par des coureurs, ayant pour montures des chevaux et des mulets nés de juments.
\VS{11}Par ces lettres, le roi accordait aux juifs, qui étaient dans chaque ville la permission de se rassembler et de défendre leur vie, de détruire, de tuer, et d’exterminer toute force armée du peuple et de quelque province que ce soit, qui prendraient les armes pour les attaquer, ainsi que leurs petits enfants et leurs femmes, et de piller leurs biens ;
\VS{12}et cela en un seul jour, dans toutes les provinces du roi Assuérus, le treizième jour du douzième mois, qui est le mois d'Adar.
\VS{13}Ces lettres écrites portaient une copie de l’édit qui devait être publié dans chaque province, et informaient tous les peuples que les juifs seraient prêts en ce jour à se venger de leurs ennemis.
\VS{14}Les coureurs, montés sur des chevaux et des mulets, partirent aussitôt et en toute hâte, d’après l’ordre du roi. L'édit fut aussi publié dans Suse, la capitale.
\TextTitle{[Mardochée honoré]}
\VS{15}Mardochée sortit de chez le roi, en vêtement royal violet et blanc, avec une grande couronne d'or, et une robe de byssus et de pourpre. La ville de Suse poussait des cris, et elle fut dans la joie.
\VS{16}Il eut pour les juifs du bonheur et de la joie, des réjouissances et des honneurs.
\VS{17}Dans chaque province et dans chaque ville, partout où arrivaient l’ordre du roi et son décret, il y eut pour les juifs de la joie, des réjouissances, des festins, et des fêtes. Et beaucoup de gens d'entre les peuples du pays se faisaient juifs, parce que la crainte des juifs les avait saisis.
\TextTitle{[Les juifs triomphent de leurs ennemis]}
\Chap{9}
\VerseOne{}Le douzième mois, qui est le mois d'Adar, le treizième jour du mois, où l’ordre du roi et son décret devaient être exécutés, au jour où les ennemis des juifs espéraient dominer, ce fut le contraire qui arriva, les juifs dominèrent sur leurs ennemis.
\VS{2}Les juifs se rassemblèrent dans leurs villes, dans toutes les provinces du roi Assuérus, pour mettre la main sur ceux qui cherchaient leur perte ; et personne ne put leur résister, car la crainte qu'on avait d'eux avait saisi tous les peuples.
\VS{3}Et tous les princes des provinces, les satrapes, les gouverneurs, et ceux qui s’occupaient des affaires du roi, soutenaient les juifs, à cause de la terreur que leur inspirait Mardochée.
\VS{4}Car Mardochée était puissant dans la maison du roi, et sa renommée se répandait dans toutes les provinces, parce qu’il devenait de plus en plus puissant.
\VS{5}Les juifs frappèrent tous leurs ennemis à coups d'épée, ils les tuèrent et les détruisirent ; ils traitèrent selon leurs désirs ceux qui les haïssaient.
\VS{6}Dans Suse, la capitale, les juifs tuèrent et firent périr cinq cents hommes.
\VS{7}Ils tuèrent aussi Parschandatha, Dalphon, Aspatha,
\VS{8}Poratha, Adalia, Aridatha,
\VS{9}Parmaschtha, Arizaï, Aridaï, et Vajezatha,
\VS{10}les dix fils d'Haman, fils d'Hammedatha, l'oppresseur des juifs. Mais ils ne mirent pas leurs mains au pillage.
\VS{11}Ce jour-là, on rapporta au roi le nombre de ceux qui avaient été tués dans Suse, la capitale.
\VS{12}Le roi dit à la reine Esther : Dans Suse, la capitale, les juifs ont tué et détruit cinq cents hommes, et les dix fils d'Haman, qu'auront-ils fait dans le reste des provinces du roi ? Quelle est ta demande ? Et elle te sera accordée. Que désires-tu encore ? Tu l’obtiendras.
\VS{13}Esther répondit : Si le roi le trouve bon qu'il soit permis aux juifs, qui sont à Suse, d’agir encore demain selon le décret d’aujourd'hui, et que l'on pende au bois les dix fils d'Haman.
\VS{14}Et le roi ordonna de faire ainsi. L'édit fut publié dans Suse. On pendit les dix fils d'Haman ;
\VS{15}et les juifs qui étaient dans Suse se rassemblèrent encore le quatorzième jour du mois d'Adar et tuèrent dans Suse trois cents hommes. Mais ils ne mirent pas la main au pillage.
\VS{16}Les autres juifs qui étaient dans les provinces du roi se rassemblèrent, et défendirent leur vie ; ils eurent du repos et furent délivrés de leurs ennemis, et ils tuèrent soixante-quinze mille hommes de ceux qui les haïssaient. Mais ils ne mirent pas la main au pillage.
\VS{17}Ces choses arrivèrent le treizième jour du mois d'Adar, et le quatorzième du même mois ils se reposèrent, et ils en firent un jour de festin et de joie.
\VS{18}Les juifs qui étaient dans Suse, s'assemblèrent le treizième et le quatorzième jour du même mois, et ils se reposèrent le quinzième jour, et ils en firent un jour de festin et de joie.
\VS{19}C'est pourquoi les juifs des campagnes qui habitent dans des villes sans murailles, font le quatorzième jour du mois d'Adar, un jour de réjouissance, de festin et de fête, où l’on s’envoie des portions les uns aux autres.
\TextTitle{[Esther confirme l'instauration la fête des Purim]}
\VS{20}Mardochée écrivit ces choses, et il envoya les lettres à tous les juifs qui étaient dans toutes les provinces du roi Assuérus, auprès et au loin.
\VS{21}Il leur prescrivait de célébrer chaque année le quatorzième jour et le quinzième jour du mois d'Adar.
\VS{22}Comme les jours où les juifs avaient obtenu du repos en se délivrant de leurs ennemis, de célébrer le mois où leur angoisse fut changée en joie, et leur deuil en jour heureux, et de faire de ces jours des jours de festin et de joie, où l’on s’envoie des portions les uns aux autres, et des dons aux pauvres.
\VS{23}Les juifs s’engagèrent à faire ce qu’ils avaient déjà commencé et ce que Mardochée leur prescrivit.
\VS{24}Car Haman, fils d'Hammedatha, l’Agaguite, l'oppresseur de tous les juifs, avait projeté de détruire les juifs, et il avait jeté le pur, c'est-à-dire le sort, afin de les détruire et de les tuer ;
\VS{25}mais Esther s’étant présentée devant le roi, le roi ordonna par écrit que le méchant projet qu'Haman avait imaginé contre les juifs, retombe sur sa tête, et qu'on le pende au bois, lui et ses fils.
\VS{26}C'est pourquoi on appelle ces jours-là purim, du nom de pur\FTNT{Pur ou purim : Ce terme signifie sort (Est. 3 : 7). La fête de purim a été instituée pour célébrer leur délivrance de l’extermination planifiée par Haman, à la suite de l’intervention héroïque d’Esther. Les juifs l’observent désormais chaque année le 14 du mois d’Adar (février ou mars) depuis le temps d’Esther jusqu’à ce jour.}. D’après tout le contenu de cette lettre, et selon ce qu’ils avaient eux-mêmes vu et ce qui leur était arrivé,
\VS{27}Les juifs établirent et adoptèrent pour eux, pour leur postérité, et pour tous ceux qui s’attacheraient à eux, l’engagement de ne pas manquer de célébrer chaque année ces deux jours, selon le mode prescrit et au temps fixé.
\VS{28}Ces jours devaient être rappelés et observés de génération en génération, dans chaque famille, dans chaque province et dans chaque ville ; et ces jours de Purim ne devaient jamais être abolis au milieu des juifs, ni le souvenir s’en effacer parmi leurs descendants.
\VS{29}La reine Esther, fille d'Abichaïl, écrivit aussi avec le juif Mardochée, de manière pressante pour la seconde fois, pour confirmer la lettre sur les Purim.
\VS{30}On envoya des lettres à tous les juifs, dans les cent vingt-sept provinces du royaume d'Assuérus. Elles contenaient des paroles de paix et de vérité,
\VS{31}pour établir ces jours de Purim au temps fixé, comme Mardochée le juif et la reine Esther les avaient établis pour eux, et comme ils les avaient établis pour eux-mêmes et pour leur postérité, à l’occasion de leur jeûne et de leurs cris.
\VS{32}Ainsi l'édit d'Esther confirma l’institution des Purim, et cela fut écrit dans le livre.
\TextTitle{[Mardochée établi dans la cours du roi]}
\Chap{10}
\VerseOne{}Le roi Assuérus imposa un tribut au pays, et aux îles de la mer.
\VS{2}Tous les faits concernant ses exploits, et les détails sur la grandeur à laquelle le roi éleva Mardochée, ne sont-ils pas écrits dans le livre des chroniques des rois de Médie et de Perse ?
\VS{3}Car Mardochée le juif était le premier après le roi Assuérus ; grand parmi les juifs et agréable à la multitude de ses frères, il chercha le bien-être de son peuple, et parla pour la paix de toute sa race.
\PPE{}
\end{multicols}

\clearpage\ShortTitle{Daniel}\BookTitle{Daniel}\BFont
\noindent\hrulefill
{\footnotesize
\textit{
\bigskip
{\centering{}
\\Auteur : Daniel
\\(Heb. : Daniye'l)
\\Signification : Dieu est mon juge
\\Thème : Ascension et chute des royaumes
\\Date de rédaction : 6\up{ème} siècle av. J.-C.\\}
}
%\bigskip
\textit{
\\Issu d'une famille princière de Juda, Daniel fut déporté de Jérusalem à Babylone pendant sa jeunesse, sous le règne de Nebucadnestar. Lui et trois de ses amis – eux aussi de noble lignée - furent choisis pour être instruits selon la sagesse babylonienne en vue de servir le roi. Fervent dans sa foi en Yahweh, Daniel - imité ensuite par ses compagnons – résolut de ne point se souiller et obtint ainsi la faveur de son Dieu. Son intégrité et sa crainte de Dieu lui valurent de miraculeuses victoires, de nombreuses distinctions et une grande sagesse. Daniel avait reçu du discernement pour expliquer songes et visions et délivra plusieurs prophéties dont certaines se sont déjà accomplies, d'autres se réaliseront à la fin du temps des nations, au moment du retour de Christ.
%\bigskip
\\Dieu témoigna de la justice de Daniel au prophète Ezéchiel dont il fut contemporain.\bigskip
}
}
\par\nobreak\noindent\hrulefill
\begin{multicols}{2}
\Chap{1}
\TextTitle{Juda livré à la captivité babylonienne}
\VerseOne{}La troisième année du règne de Jojakim, roi de Juda, Nebucadnetsar, roi de Babylone, vint contre Jérusalem et l'assiégea.
\VS{2}Le Seigneur livra entre ses mains Jojakim\FTNT{En 597 av. J-C., la ville de Jérusalem tomba entre les mains des Babyloniens qui déportèrent le roi Jojakim et nommèrent comme roi à sa place son oncle Sédécias. Une petite partie de la population fut déportée à cette occasion. Cette première déportation ne concernait que l'élite administrative et sacerdotale : prêtres, scribes, hauts fonctionnaires, membres de la famille royale et artisans métallurgistes. Pour de nombreux historiens, il s'agissait moins d'une déportation que d'une constitution d'un groupe d'otages. Le roi, quelques membres de sa famille, et de diverses familles de notables, furent tenus en résidence surveillée à la cour babylonienne pour s'assurer que le royaume de Juda resterait pacifié.}, roi de Juda et une partie des vases de la maison de Dieu. Nebucadnetsar emporta les vases au pays de Schinear\FTNT{Schinear : « le pays des deux fleuves ». C'est l'ancien nom du territoire qui est devenu Babylonie ou Chaldée. C'est le pays de Nimrod (Ge. 10:6-12). C'est à Schinear qu'on tenta de construire la tour de Babel et de mettre en place le premier gouvernement mondial.}, dans la maison de son dieu, il les mit dans la maison du trésor de son dieu.
\VS{3}Le roi dit à Aschpenaz, capitaine de ses eunuques, d'amener quelques-uns des enfants d'Israël de race royale\FTNT{Daniel était de la race royale (2 R. 20:16-19 ; Es. 39:1-8).} et des principaux seigneurs,
\VS{4}quelques jeunes enfants en qui il n'y avait aucun défaut corporel, beaux de figure, instruits en toute sagesse, connaissant les sciences, pleins d'intelligence, et capables de se tenir dans le palais du roi ; et à qui l'on enseignerait les lettres et la langue des Chaldéens.
\VS{5}Le roi leur assigna pour provision chaque jour une portion de la viande royale et du vin dont il buvait, afin qu'on les nourrisse ainsi pendant trois ans au bout desquels ils se tiendraient devant le roi.
\VS{6}Il y avait parmi eux, d'entre les fils de Juda, Daniel, Hanania, Mischaël et Azaria.
\VS{7}Mais le capitaine des eunuques leur donna d'autres noms, il donna à Daniel le nom de Beltschatsar, à Hanania celui de Schadrac, à Mischaël celui de Méschac et à Azaria celui d'Abed-Nego.
\TextTitle{La fermeté de Daniel à Babylone}
\VS{8}Daniel résolut dans son cœur de ne pas se souiller par la portion de la viande du roi et par le vin dont le roi buvait; c'est pourquoi il supplia le chef des eunuques afin qu'il ne l'oblige pas à se souiller.
\VS{9}Et Dieu fit trouver à Daniel faveur et grâce auprès du chef des eunuques. 
\VS{10}Et le chef des eunuques dit à Daniel : Je crains le roi, mon seigneur, qui a fixé ce que vous devez manger et boire ; car pourquoi verrait-il vos visages plus défaits que ceux des jeunes gens de votre âge ? Vous exposeriez ma tête auprès du roi.
\VS{11}Mais Daniel dit à Meltsar, l'intendant à qui le chef des eunuques avait remis la surveillance de Daniel, Hanania, Mischaël et Azaria :
\VS{12}Eprouve, je te prie, tes serviteurs pendant dix jours, et qu'on nous donne des légumes à manger et de l'eau à boire.
\VS{13}Après cela, tu regarderas nos visages et ceux des jeunes enfants qui mangent la portion de la viande royale; puis tu feras à tes serviteurs selon ce que tu auras vu.
\VS{14}Et il les écouta dans cette affaire et les éprouva pendant dix jours.
\VS{15}Au bout des dix jours, leurs visages parurent en meilleur état et plus d'embonpoint que tous les jeunes gens qui mangeaient la portion de la viande royale.
\VS{16} Ainsi Meltsar prenait la portion de leur viande et le vin qu'ils devaient boire, et leur donnait des légumes.
\VS{17}Et Dieu donna à ces quatre jeunes gens de la science et de l'intelligence dans toutes les lettres, et de la sagesse ; et Daniel comprenait toutes les visions et tous les songes.
\VS{18}Et à la fin des jours fixés par le roi pour qu'on les lui amène, le chef des eunuques les présenta à Nebucadnetsar.
\VS{19}Le roi s'entretint avec eux ; mais entre eux tous il ne s'en trouva pas de tels que Daniel, Hanania, Mischaël et Azaria ; et ils entrèrent au service du roi.
\VS{20}Sur toutes les questions savantes qui réclamaient de la sagesse et de l'intelligence, et sur lesquelles le roi les interrogeait, il les trouva dix fois supérieurs à tous les magiciens et les astrologues qui étaient dans tout son royaume.
\VS{21}Et Daniel fut là jusqu'à la première année du roi Cyrus.
\Chap{2}
\TextTitle{Les sages de Babylone tous condamnés à mort}
\VerseOne{}La deuxième année du règne de Nebucadnetsar, Nebucadnetsar eut des songes, et son esprit fut agité, et son sommeil fut interrompu.
\VS{2}Alors le roi fit appeler les magiciens, les astrologues, les enchanteurs et les Chaldéens, pour qu'ils lui expliquent ses songes ; ils vinrent donc et se présentèrent devant le roi.
\VS{3}Le roi leur dit : J'ai eu un songe, mon esprit est agité, tâchant de connaître ce songe\FTNT{Ce songe annonce la future mise en place d'un gouvernement mondial. Voir commentaire en Da. 7:3.}.
\VS{4}Et les Chaldéens répondirent au roi en langue araméenne\FTNT{L'araméen : De Daniel 2:5 à 7:28, le livre est écrit en araméen.} : Ô roi, vis éternellement ! Dis le songe à tes serviteurs et nous en donnerons l'interprétation.
\VS{5}Mais le roi répondit et dit aux Chaldéens : La chose m'a échappé ; si vous ne me faites connaître le songe et son interprétation, vous serez mis en pièces et vos maisons seront réduites en un tas d'immondices.
\VS{6}Mais si vous me faites connaître le songe et son interprétation, vous recevrez de moi, des dons, des présents et un grand honneur. Quoi qu'il en soit faites-moi connaître le songe et son interprétation.
\VS{7}Ils répondirent pour la seconde fois et dirent : Que le roi dise le songe à ses serviteurs et nous en donnerons l'interprétation.
\VS{8}Le roi répondit et dit : Je m'aperçois en vérité, que vous ne cherchez qu'à gagner du temps, parce que vous voyez que la chose m'a échappée.
\VS{9}Mais si vous ne me faites pas connaître le songe, il y a une même sentence contre vous tous ; car vous vous êtes préparés à dire devant moi des mensonges et des faussetés en attendant que le temps soit changé. Quoi qu'il en soit, dites-moi le songe et je saurai que vous pouvez m'en donner l'interprétation.
\VS{10}Les Chaldéens répondirent au roi et dirent : Il n'y a aucun homme sur la terre qui puisse exécuter ce que le roi demande. Et aussi il n'y a ni roi, ni seigneur, ni gouverneur qui ait jamais demandé une telle chose à quelque magicien, astrologue ou Chaldéen que ce soit.
\VS{11}Car la chose que le roi demande est extrêmement difficile et il n'y a personne qui puisse le faire connaître au roi, excepté les dieux dont la demeure n'est pas parmi les hommes.
\VS{12}A cause de cela, le roi s'irrita et se mit dans une très grande colère, et ordonna qu'on fasse périr tous les sages de Babylone.
\VS{13}La sentence fut donc publiée; on mettait à mort les sages et l'on cherchait Daniel et ses compagnons pour les faire périr.
\TextTitle{Daniel implore la miséricorde de Dieu}
\VS{14}Alors Daniel détourna l'exécution du conseil et l'arrêt donné à Arjoc, chef des gardes du roi, qui était sorti pour tuer les sages de Babylone.
\VS{15}Et il demanda et dit à Arjoc, commandant du roi : Pourquoi la sentence du roi est-elle si sévère ? Arjoc exposa la chose à Daniel.
\VS{16}Et Daniel entra et pria le roi de lui accorder du temps pour donner l'interprétation au roi.
\VS{17}Alors Daniel alla dans sa maison et informa de cette affaire Hanania, Mischaël et Azaria, ses compagnons,
\VS{18}pour implorer la miséricorde du Dieu des cieux sur ce secret, afin qu'on ne mette pas à mort Daniel et ses compagnons avec le reste des sages de Babylone. 
\TextTitle{Le songe de la grande statue révélé à Daniel}
\VS{19}Et le secret fut révélé à Daniel dans une vision pendant la nuit. Et Daniel bénit le Dieu des cieux.
\VS{20}Daniel prit donc la parole et dit : Béni soit le nom de Dieu, d'éternité en éternité ! A lui appartiennent la sagesse et la force\FTNT{Job. 12:13 ; Ap. 5:12 ; Ap. 7:12.}.
\VS{21}C'est lui qui change les temps et les saisons, qui ôte et qui établit les rois, qui donne la sagesse aux sages et la connaissance à ceux qui ont de l'intelligence.
\VS{22}C'est lui qui révèle les choses profondes et cachées, il connaît les choses qui sont dans les ténèbres et la lumière demeure avec lui\FTNT{De. 29:29 ; Es. 48:6 ; Jé. 33:3 ; Lu. 12:2-3.}.
\VS{23}Ô Dieu de nos pères ! Je te glorifie et te loue de ce que tu m'as donné de la sagesse et de la force, et de ce que tu m'as maintenant fait connaître ce que nous t'avons demandé, en nous ayant fait connaître le secret du roi.
\VS{24}Après cela, Daniel alla auprès d'Arjoc, à qui le roi avait ordonné de faire périr les sages de Babylone. Il alla et lui parla ainsi : Ne fais pas périr les sages de Babylone, mais fais-moi entrer devant le roi et je donnerai au roi l'interprétation qu'il souhaite.
\VS{25}Alors Arjoc conduisit promptement Daniel devant le roi et lui parla ainsi : J'ai trouvé parmi les captifs de Juda un homme qui donnera au roi l'interprétation de son songe.
\VS{26}Le roi prit la parole et dit à Daniel, qu'on nommait Beltschatsar : Es-tu capable de me faire connaître le songe que j'ai eu et son interprétation ?
\VS{27}Daniel répondit en présence du roi et dit : Ce que le roi demande est un secret que les sages, les astrologues, les magiciens et les devins ne sont pas capables de révéler au roi.
\VS{28}Mais il y a dans les cieux un Dieu qui révèle les secrets et qui a fait connaître au roi Nebucadnetsar ce qui doit arriver dans les derniers jours\FTNT{« Les derniers jours » voir commentaires dans Ge. 49:1-2.}. Voici ton songe et les visions de ta tête que tu as eues sur ta couche.
\VS{29}Sur ta couche, ô roi, il t'est monté des pensées touchant ce qui arriverait après ce temps-ci ; et celui qui révèle les secrets t'a fait connaître ce qui doit arriver.
\VS{30}Si ce secret m'a été révélé, ce n'est point qu'il y ait en moi une sagesse supérieure à celle de tous les vivants, mais c'est afin de donner au roi l'interprétation de son songe et afin que tu connaisses les pensées de ton cœur.
\VS{31}Ô roi, tu regardais et tu voyais une grande statue\FTNT{Voir annexe « La statue de Nebucadnetsar ».} ; cette grande statue, dont la splendeur était extraordinaire, était debout devant toi et son apparence était terrible.
\VS{32}La tête de cette statue était d'un or très fin, sa poitrine et ses bras étaient d'argent ; son ventre et ses cuisses étaient d'airain ;
\VS{33}ses jambes étaient de fer et ses pieds étaient en partie de fer et en partie de terre.
\VS{34}Tu regardais cela, jusqu'à ce qu'une pierre se détacha sans main, frappa les pieds de fer et d'argile de la statue et les brisa.
\VS{35}Alors le fer, l'argile, l'airain, l'argent et l'or furent brisés ensemble et devinrent comme la paille de l'aire en été que le vent transporte çà et là ; et nulle trace n'en fut retrouvée. Mais la pierre qui avait frappé la statue devint une grande montagne et remplit toute la terre.
\TextTitle{Premier empire universel : Babylone\FTNT{Cp. Da. 7:4.}}
\VS{36}C'est là le songe. Nous en donnerons maintenant l'interprétation devant le roi.
\VS{37}Ô roi, tu es le roi des rois, parce que le Dieu des cieux t'a donné le royaume, la puissance, la force et la gloire.
\VS{38}Il a remis entre tes mains, en quelque lieu qu'ils habitent, les enfants des hommes, les bêtes des champs et les oiseaux du ciel, et il t'a fait dominer sur eux tous : C'est toi qui es la tête d'or\FTNT{Jé. 27:6-7.}.
\TextTitle{Deuxième et troisième empires : les Mèdes et les Perses\FTNTT{cp. Da. 7:5 ; 8:20} et la Grèce\FTNTT{cp. Da. 7:6 ; 8:21}}
\VS{39}Mais après toi, il s'élèvera un autre royaume, moindre que le tien ; et ensuite un troisième royaume qui sera d'airain et qui dominera sur toute la terre.
\TextTitle{Quatrième empire : Rome\FTNTT{Cp. Da. 7:7 ; 9:26.}}
\VS{40}Puis il y aura un quatrième royaume, fort comme du fer ; de même que le fer brise et rompt tout ainsi il brisera et rompra tout, comme le fer qui met tout en pièces.
\VS{41}Et quant à ce que tu as vu, que les pieds et les orteils étaient en partie d'argile de potier et en partie de fer, c'est que ce royaume sera divisé, mais il y aura en lui de la force du fer, parce que tu as vu le fer mêlé avec l'argile de potier.
\VS{42}Et comme les doigts des pieds étaient en partie de fer et en partie d'argile, ce royaume sera en partie fort et en partie fragile.
\VS{43} Quant à ce que tu as vu, le fer mêlé avec l'argile de potier, c'est qu'ils se mêleront par des alliances humaines\FTNT{Le mot « alliances » vient de l'araméen « zera » qui signifie « semence » et « descendant ».} ; mais ils ne seront point unis l'un à l'autre de même que le fer ne s'allie point avec l'argile.
\TextTitle{Le royaume du Messie}
\VS{44}Dans le temps de ces rois, le Dieu des cieux suscitera un Royaume qui ne sera jamais détruit, et ce Royaume ne passera point à un autre peuple ; il brisera et anéantira tous ces royaumes-là, et lui-même sera établi éternellement.
\VS{45}Selon que tu as vu que de la montagne une pierre a été coupée sans main et qu'elle a brisé le fer, l'airain, la terre, l'argent et l'or. Le grand Dieu a fait connaître au Roi ce qui arrivera ci-après ; or le songe est véritable et son interprétation est certaine.
\TextTitle{Yahweh, le Dieu qui revèle les secrets}
\VS{46}Alors le roi Nebucadnetsar tomba sur sa face et se prosterna devant Daniel et il ordonna qu'on lui offre des offrandes de bonne odeur et des parfums.
\VS{47}Le roi parla à Daniel et lui dit : Certainement, votre Dieu est le Dieu des dieux, et le Seigneur des rois, et il révèle les secrets, puisque tu as pu découvrir ce secret.
\VS{48}Alors le roi éleva Daniel en dignité et lui fit de nombreux et riches présents ; il l'établit gouverneur sur toute la province de Babylone et chef suprême de tous les sages de Babylone.
\VS{49}Daniel pria le roi de remettre l'intendance de la province de Babylone à Schadrac, Méschac et Abed-Négo. Et Daniel se tenait à la porte du roi.
\Chap{3}
\TextTitle{La statue d'or de Nebucadnetsar}
\VerseOne{}Le roi Nebucadnetsar fit une statue d'or\FTNT{Nebucadnetsar est un type de l'antéchrist qui s'oppose aux plans de Dieu. Contrairement à la statue composée de plusieurs métaux qu'il avait vue en songe, et où il est représenté par la tête en or (Da. 2:38), il s'est fait construire une statue entièrement en or, se déclarant ainsi symboliquement invincible et immortel. En agissant de la sorte, Nebucadnetsar se fait Dieu et exige d'être adoré (2 Th. 2:3-4). Cette statue annonçait prophétiquement la mise en place d'une religion mondiale, fruit d'un mélange entre la politique et la religion. Ces choses sont déjà bien installées, il ne manque plus que la révélation de l'impie. Nous vivons dans une époque où on oblige les chrétiens à adhérer à des organisations politiques, religieuses, sous contrôle de l'état, et cela dans le but de contrôler les individus et le message qu'ils entendent et diffusent. Ainsi, les dirigeants actuels doivent passer au préalable par des études théologiques, dont l'enseignement contredit de plus en plus la vérité biblique pour se conformer aux préceptes de ce monde. Une fois ordonnés, ils doivent affilier leurs églises à des fédérations qui sont sous contrôle de l'état. En échange des subventions, beaucoup accepteront de diluer l'évangile, privant ainsi les âmes de la vérité.}, dont la hauteur était de soixante coudées, et la largeur de six coudées. Il la dressa dans la vallée de Dura, dans la province de Babylone.
\VS{2}Puis le roi Nebucadnetsar envoya pour rassembler les satrapes, les intendants, les gouverneurs, les conseillers, les trésoriers, les jurisconsultes, les juges, et tous les magistrats des provinces, afin qu'ils se rendent à la dédicace de la statue que le roi Nebucadnetsar avait dressée.
\VS{3}Ainsi furent assemblés les satrapes, les intendants, les gouverneurs, les conseillers, les trésoriers, les jurisconsultes, les juges, et tous les magistrats des provinces, pour la dédicace de la statue que le roi Nebucadnetsar avait dressée. Ils s'assemblèrent devant la statue que le roi Nebucadnetsar avait dressée.
\VS{4}Alors un héraut cria à haute voix, en disant : On vous fait savoir, ô peuples, nations, et langues !
\VS{5}Au moment où vous entendrez le son du cor, du chalumeau, de la guitare, de la sambuque, du psaltérion, de la cornemuse, et de toutes sortes d'instruments de musique, vous vous jetterez à terre et vous adorerez la statue d'or que le roi Nebucadnetsar a dressée.
\VS{6}Quiconque ne se jettera pas à terre et n'adorera pas sera jeté à l'instant même au milieu de la fournaise de feu ardent.
\VS{7}C'est pourquoi, au moment où tous les peuples entendirent le son du cor, du chalumeau, de la guitare, de la sambuque, du psaltérion, et de toutes sortes d'instruments de musique, tous les peuples, les nations, et les hommes de toutes les langues, se prosternèrent et adorèrent la statue d'or que le roi avait dressée.
\TextTitle{Le refus de l'idolâtrie}
\VS{8}Alors à ce même moment, certains chaldéens s'approchèrent et accusèrent les Juifs.
\VS{9}Et ils parlèrent et dirent au roi Nebucadnetsar : Roi, vis éternellement !
\VS{10}Toi, ô roi, tu as donné un ordre d'après lequel tout homme qui entendrait le son du cor, du chalumeau, de la guitare, de la sambuque, du psaltérion, de la cornemuse, et de toutes sortes d'instruments de musique, devrait se prosterner et adorer la statue d'or,
\VS{11}et que quiconque ne se prosternerait pas et ne l'adorerait pas, serait jeté au milieu d'une fournaise ardente.
\VS{12}Or, il y a certains Juifs que tu as établis sur les affaires de la province de Babylone, Schadrac, Méschac, et Abed-Négo ; ces hommes-là, ô roi, ne tiennent aucun compte de toi ; ils ne servent pas tes dieux, et ils n'adorent pas la statue d'or que tu as dressée.
\VS{13}Alors le roi Nebucadnetsar, saisi de colère et de fureur, ordonna qu'on amène Schadrac, Méschac, et Abed-Négo. Et ces hommes furent amenés devant le roi.
\VS{14}Et le roi Nebucadnetsar prit la parole et leur dit : Est-il vrai, Schadrac, Méschac, et Abed-Négo, que vous ne servez pas mes dieux, et que vous ne vous prosternez pas devant la statue d'or que j'ai dressée ?
\VS{15}Maintenant si vous êtes prêts, au moment où vous entendrez le son du cor, du chalumeau, de la guitare, de la sambuque, du psaltérion, de la cornemuse, et de toutes sortes d'instruments de musique, vous vous prosternerez, et vous adorerez la statue que j'ai faite ; si vous ne l'adorez pas, vous serez jetés à l'instant au milieu de la fournaise de feu ardent. Et qui est le dieu qui vous délivrera de mes mains ?
\VS{16}Schadrac, Méschac et Abed-Négo répondirent et dirent au roi Nebucadnetsar : Nous n'avons pas besoin de te répondre sur ce sujet.
\VS{17}Voici, notre Dieu, que nous servons, peut nous délivrer de la fournaise de feu ardent, et il nous délivrera de ta main, ô roi !
\VS{18}Sinon, sache, ô roi, que nous ne servirons pas tes dieux, et que nous n'adorerons pas la statue d'or que tu as dressée.
\TextTitle{L'épreuve de la fournaise de feu ardent}
\VS{19}Alors Nebucadnetsar fut rempli de fureur, et il changea de visage en tournant ses regards contre Schadrac, Méschac, et Abed-Négo. Il prit la parole et ordonna de chauffer la fournaise sept fois plus qu'on avait coutume de la chauffer.
\VS{20}Puis il commanda aux hommes les plus forts et les plus vaillants qui étaient dans son armée de lier Schadrac, Méschac, et Abed-Négo, et de les jeter dans la fournaise de feu ardent.
\VS{21}Et en même temps ces hommes furent liés avec leurs caleçons, leurs chaussures, leurs tiares, et leurs vêtements, et furent jetés au milieu de la fournaise de feu ardent.
\VS{22}Et parce que l'ordre du roi était sévère, et que la fournaise était extraordinairement chauffée, la flamme tua les hommes qui y avaient jetés, Schadrac, Méschac, et Abed-Négo.
\VS{23}Et ces trois hommes, Schadrac, Méschac, et Abed-Négo, tombèrent tous liés au milieu de la fournaise ardente.
\TextTitle{La grandeur de Yahweh, le Dieu qui délivre}
\VS{24}Alors le roi Nebucadnetsar fut effrayé, et se leva précipitamment. Il prit la parole et il dit à ses conseillers : N'avons-nous pas jeté trois hommes liés au milieu du feu ? Ils répondirent et dirent au roi : Certainement, ô roi !
\VS{25}Il reprit et dit : Voici, je vois quatre hommes sans liens qui marchent au milieu du feu, et qui n'ont point de mal ; et la figure du quatrième est semblable à celle d'un fils de Dieu.
\VS{26}Alors Nebucadnetsar s'approcha vers la porte de la fournaise de feu ardent ; et prenant la parole, il dit : Schadrac, Méschac, et Abed-Négo, serviteurs du Dieu Très-Haut, sortez et venez ! Alors Schadrac, Méschac, et Abed-Négo sortirent du milieu du feu.
\VS{27}Puis les satrapes, les intendants, les gouverneurs, et les conseillers du roi s'assemblèrent pour contempler ces hommes-là, et le feu n'avait eu aucun pouvoir sur leurs corps, et aucun cheveu de leur tête n'était brûlé, et leurs caleçons n'étaient point endommagés, et l'odeur du feu n'avait pas passé sur eux.
\VS{28}Alors Nebucadnetsar prit la parole et dit : Béni soit le Dieu de Schadrac, Méschac, et Abed-Négo, lequel a envoyé son ange et délivré ses serviteurs qui ont eu confiance en lui, et qui ont violé l'ordre du roi et livré leur corps plutôt que de servir et d'adorer aucun autre dieu que leur Dieu\FTNT{Mt. 4:10 ; Ac. 4:19 ; Ac. 5:29.}.
\TextTitle{Schadrac, Méschac et Abed-Nego élevé par le roi}
\VS{29}Voici maintenant l'ordre que je donne : Tout homme, à quelque nation ou langue qu'il appartienne, qui parlera mal du Dieu de Schadrac, Méschac, et Abed-Négo, sera mis en pièces, et sa maison sera réduite en un tas d'immondices, parce qu'il n'y a aucun autre dieu qui puisse délivrer comme lui.
\VS{30}Alors le roi fit réussir Schadrac, Méschac, et Abed-Négo dans la province de Babylone.
\Chap{4}
\TextTitle{La suprématie de Yahweh déclarée aux nations}
\VerseOne{}Le roi Nebucadnetsar, à tous les peuples, aux nations, aux hommes de toutes langues qui habitent sur toute la terre : Que votre paix soit multipliée !
\VS{2}Il m'a semblé bon de vous déclarer les signes et les merveilles que le Dieu Très-Haut a opérés à mon égard.
\VS{3}Ô que ses signes sont grands, et ses merveilles pleines de force ! Son règne est un règne éternel, et sa domination subsiste de génération en génération\FTNT{Ps 102:12 ; La. 5:19 ; Lu. 1:33.}.
\TextTitle{La vision du grand arbre}
\VS{4}Moi, Nebucadnetsar, j'étais tranquille dans ma maison, et heureux dans mon palais.
\VS{5}J'ai eu un songe qui m'épouvanta ; et les pensées sur ma couche et les visions de ma tête me troublèrent.
\VS{6}J'ordonnai qu'on fasse venir devant moi tous les sages de Babylone, afin qu'ils me donnent l'interprétation du songe.
\VS{7}Alors vinrent les magiciens, les astrologues, les Chaldéens et les devins. Je leur dis le songe, mais ils ne purent m'en donner l'interprétation.
\VS{8}En dernier lieu, se présenta devant moi Daniel, nommé Beltschatsar, selon le nom de mon Dieu, et qui a en lui l'Esprit des dieux saints. Je lui dis le songe :
\VS{9}Beltschatsar, chef des magiciens, comme je sais que l'Esprit des dieux saints est en toi, et que nul secret ne t'est difficile, écoute les visions que j'ai eues en songe, et donne-moi son interprétation.
\VS{10}Voici les visions de ma tête, pendant que j'étais sur ma couche. Je regardais, et voici, il y avait un arbre au milieu de la terre d'une grande hauteur.
\VS{11}Cet arbre était devenu grand et fort, sa cime s'élevait jusqu'aux cieux, et on le voyait des extrémités de toute la terre.
\VS{12}Son feuillage était beau, et son fruit abondant, et il portait de la nourriture pour tous ; les bêtes des champs s'abritaient sous son ombre, les oiseaux du ciel habitaient dans ses branches, et tout être vivant tirait de lui sa nourriture.
\VS{13}Dans les visions de ma tête que j'avais sur ma couche, je regardais, et voici, un de ceux qui veillent et qui sont saints descendit des cieux.
\VS{14}Il cria à haute voix et parla ainsi : Abattez l'arbre, et coupez ses branches ! Secouez son feuillage, et dispersez son fruit ; que les bêtes s'enfuient de dessous, et les oiseaux du milieu de ses branches !
\VS{15}Mais laissez en terre le tronc où se trouvent ses racines, et liez-le avec des chaînes de fer et d'airain, qu'il soit parmi l'herbe des champs. Qu'il soit trempé de la rosée des cieux, et qu'il ait, comme les bêtes, l'herbe de la terre pour partage.
\VS{16}Que son cœur d'homme soit changé, et qu'un cœur de bête lui soit donné ; et que sept temps passent sur lui.
\VS{17}Cette sentence est le décret de ceux qui veillent, cette résolution est un ordre des saints, afin que les vivants sachent que le Très-Haut domine sur le royaume des hommes, qu'il le donne à qui il lui plaît, et qu'il y élève le plus vil des hommes\FTNT{Cette vérité est confirmée par l'apôtre Paul dans Ro. 13:1. C'est Dieu qui choisit souverainement qui il établit à la tête d'un pays. Selon les Ecritures, toute autorité vient de Dieu.}.
\VS{18}Voilà le songe que j'ai eu, moi, le roi Nebucadnetsar. Toi donc Beltschatsar, donnes-en l'interprétation, puisque tous les sages de mon royaume ne peuvent me la donner ; mais toi, tu le peux, parce l'Esprit des dieux saints est en toi.
\TextTitle{Interprétation de la vision}
\VS{19}Alors Daniel, nommé Beltschatsar, fut stupéfait environ une heure, et ses pensées le troublaient. Le roi reprit et dit : Beltschatsar, que le songe et son interprétation ne te troublent pas ! Et Beltschatsar répondit : Mon seigneur, que le songe soit pour ceux qui te haïssent, et son interprétation pour tes ennemis !
\VS{20}L'arbre que tu as vu, qui était devenu grand et fort, dont la cime s'élevait jusqu'aux cieux, et qu'on voyait de tous les points de la terre ;
\VS{21}cet arbre, dont le feuillage était beau, et les fruits abondants, qui portait de la nourriture pour tous, sous lequel s'abritaient les bêtes des champs, et parmi les branches duquel les oiseaux du ciel faisaient leur demeure,
\VS{22}c'est toi, ô roi, qui es devenu grand et fort, dont la grandeur s'est accrue et s'est élevée jusqu'aux cieux, et dont la domination s'étend jusqu'aux extrémités de la terre.
\VS{23}Le roi a vu un de ceux qui veillent et qui sont saints descendre des cieux et dire : Abattez l'arbre, et détruisez-le ! Toutefois, laissez en terre le tronc où se trouvent ses racines, et liez-le avec des chaînes de fer et d'airain, parmi l'herbe des champs, qu'il soit trempé de la rosée du ciel, et que son partage soit avec les bêtes des champs, jusqu'à ce que sept temps soient passés sur lui.
\VS{24}Voici l'interprétation, ô roi, voici le décret du Très-Haut, qui s'accomplira sur mon seigneur, le roi :
\VS{25}On te chassera du milieu des hommes, tu auras ta demeure avec les bêtes des champs, et l'on te donnera de l'herbe comme aux bœufs, et tu seras trempé de la rosée du ciel ; et sept temps passeront sur toi, jusqu'à ce que tu reconnaisses que le Très-Haut domine sur le royaume des hommes, et qu'il le donne à qui il lui plaît.
\VS{26}L'ordre de laisser le tronc où se trouvent les racines de cet arbre signifie que ton royaume te sera rendu, dès que tu auras reconnu que les cieux dominent.
\VS{27}C'est pourquoi, ô roi, que mon conseil te soit agréable : Rachète tes péchés par la justice, et tes iniquités en faisant miséricorde aux pauvres, et ta paix pourra se prolonger.
\TextTitle{Le roi déchu à cause de son orgueil}
\VS{28}Toutes ces choses se sont accomplies sur le roi Nebucadnetsar.
\VS{29}Au bout de douze mois, comme il se promenait dans le palais royal de Babylone,
\VS{30}le roi prit la parole et dit : N'est-ce pas ici Babylone la grande, que j'ai bâtie pour être la maison royale, par la puissance de ma force et pour la gloire de ma magnificence ?
\VS{31}La parole était encore dans la bouche du roi, qu'une voix descendit du ciel, disant : Roi Nebucadnetsar, on t'annonce que ton royaume va t'être ôté.
\VS{32}On te chassera du milieu des hommes, tu auras ta demeure avec les bêtes des champs ; on te donnera de l'herbe à manger comme aux bœufs ; et sept temps passeront sur toi, jusqu'à ce que tu reconnaisses que le Très-Haut domine sur le royaume des hommes, et qu'il le donne à qui il lui plaît.
\VS{33}Au même instant, la parole s'accomplit sur Nebucadnetsar. Il fut chassé du milieu des hommes, il mangea de l'herbe comme les bœufs, et son corps fut trempé de la rosée du ciel jusqu'à ce que ses cheveux croissent comme les plumes des aigles, et ses ongles comme ceux des oiseaux.
\TextTitle{Le roi est rétabli ; il loue le Dieu Très-Haut}
\VS{34}Mais à la fin de ces jours-là, moi Nebucadnetsar, je levai mes yeux vers le ciel, et la raison me revint. J'ai béni le Très-Haut, j'ai loué et glorifié celui qui vit éternellement, celui dont la domination est une domination éternelle, et dont le règne subsiste de génération en génération.
\VS{35}Tous les habitants de la terre ne sont à ses yeux que néant ; il agit comme il lui plaît avec l'armée des cieux et avec les habitants de la terre, et il n'y a personne qui empêche sa main, et qui lui dise : Que fais-tu\FTNT{Es. 45:9 ; Jé. 23:18-22 ; ps.115:3 ; Job. 9:12.} ?
\VS{36}En ce temps, la raison me revint, et je retournai à la gloire de mon royaume, ma magnificence et ma splendeur me furent rendues ; mes conseillers et mes grands me redemandèrent ; je fus rétabli dans mon royaume, et ma gloire fut augmentée.
\VS{37}Maintenant, moi, Nebucadnetsar, je loue, j'exalte, et je glorifie le Roi des cieux, dont toutes les œuvres sont véritables et ses voies justes, et qui peut abaisser ceux qui marchent avec orgueil\FTNT{De. 32:4 ; Es. 13:11 ; Ez. 17:24 ; Ps. 145:17.}.
\Chap{5}
\TextTitle{Les vases du temple souillés}
\VerseOne{}Le roi Belschatsar donna un grand festin à ses grands au nombre de mille, et il buvait le vin devant ces mille courtisans.
\VS{2}Et ayant goûté au vin, Belschatsar ordonna qu'on apporte les vases d'or et d'argent que Nebucadnetsar, son père, avait enlevés du temple de Jérusalem\FTNT{Ex. 27 ; Ex. 30 ; 2 Ch. 36:10.}, afin que le roi et ses grands, ses femmes et ses concubines, s'en servent pour boire.
\VS{3}Alors furent apportés les vases d'or qui avaient été enlevés du temple, de la maison de Dieu qui était à Jérusalem ; et le roi, et ses grands, ses femmes et ses concubines, s'en servirent pour boire.
\VS{4}Ils burent du vin, et ils louèrent leurs dieux d'or, d'argent, d'airain, de fer, de bois et de pierre.
\TextTitle{L'écriture sur la muraille}
\VS{5}Et à cette même heure-là sortirent de la muraille des doigts d'une main d'homme, qui écrivaient à l'endroit du chandelier, sur l'enduit de la muraille du palais royal ; et le Roi voyait cette partie de main qui écrivait.
\VS{6}Alors l'aspect du roi changea, et ses pensées l'effrayèrent, si bien que les jointures de ses reins se desserrèrent, et ses genoux se cognèrent l'un contre l'autre.
\VS{7}Puis le roi cria avec force qu'on fasse venir les astrologues, les Chaldéens et les devins ; et le roi prit la parole et dit aux sages de Babylone : Quiconque lira cette écriture, et m'en donnera l'interprétation, sera revêtu de pourpre, il aura un collier d'or à son cou, et sera le troisième dans le gouvernement du royaume.
\VS{8}Alors tous les sages du roi entrèrent, mais ils ne purent pas lire l'écriture et en donner au roi l'interprétation.
\VS{9}Sur quoi le roi Belschatsar fut très effrayé, il changea de couleur, et ses grands furent consternés.
\TextTitle{Interprétation de l'écriture: Division de l'empire babylonien}
\VS{10}La reine entra dans la maison du festin, à cause de ce qui était arrivé au roi et à ses grands. La reine prit la parole et dit : Ô roi, vis éternellement ! Que tes pensées ne te troublent pas, et que ton visage ne change pas de couleur !
\VS{11}Il y a dans ton royaume un homme qui a en lui l'Esprit des dieux saints ; et du temps de ton père, on trouva en lui une lumière, une intelligence, et une sagesse semblable à la sagesse des dieux. Aussi, le roi Nebucadnetsar, ton père, et le roi, ton père\FTNT{Belschatsar était le petit-fils de Nebucadnetsar qui avait régné conjointement avec son père, Nabonide, à partir de 552 av. J.-C.}, ô roi, l'établirent chef des magiciens, des astrologues, des Chaldéens et des devins,
\VS{12}parce qu'on trouva chez lui, chez Daniel, que le roi avait nommé Beltschatsar, un esprit supérieur, de la connaissance et de l'intelligence, pour interpréter les songes, pour expliquer les énigmes et résoudre les questions difficiles. Que Daniel soit donc appelé et il donnera l'interprétation que tu souhaites.
\VS{13}Alors Daniel fut introduit devant le roi. Le roi prit la parole et dit à Daniel : Es-tu ce Daniel, l'un des captifs de Juda, que le roi, mon père, a amenés de Juda ?
\VS{14}J'ai appris sur ton compte que tu as en toi l'Esprit des dieux, et qu'on trouve en toi une lumière, une intelligence et une sagesse extraordinaires.
\VS{15}On vient d'amener devant moi les sages et les astrologues, afin qu'ils lisent cette écriture et m'en donnent l'interprétation, mais ils n'ont pas pu donner l'interprétation de la chose.
\VS{16}J'ai appris que tu peux interpréter et résoudre les choses difficiles ; maintenant donc si tu peux lire cette écriture, et m'en donner l'interprétation, tu seras revêtu de pourpre, tu porteras à ton cou un collier d'or, et tu seras le troisième dans le gouvernement du royaume.
\VS{17}Alors Daniel répondit et dit en présence du roi : Que tes dons restent à toi, et donne tes présents à un autre ; toute fois je lirai l'écriture au roi, et je lui en donnerai l'interprétation.
\VS{18}Ô roi ! Le Dieu Très-Haut avait donné à Nebucadnetsar, ton père, le royaume, la magnificence, la gloire et l'honneur.
\VS{19}Et à cause de la grandeur qu'il lui avait donnée, tous les peuples, les nations, et les hommes de toutes langues tremblaient devant lui et le redoutaient. Il faisait mourir ceux qu'il voulait, et il laissait la vie à ceux qu'il voulait ; il élevait ceux qu'il voulait, et il abaissait ceux qu'il voulait.
\VS{20}Mais lorsque son cœur s'éleva et que son esprit s'endurcit jusqu'à l'arrogance, il fut renversé de son trône royal et dépouillé de sa gloire ;
\VS{21}il fut chassé du milieu des fils des hommes, son cœur fut rendu semblable à celui des bêtes, et sa demeure fut avec les ânes sauvages ; on lui donna comme aux bœufs de l'herbe à manger, et son corps fut trempé de la rosée du ciel, jusqu'à ce qu'il reconnaisse que le Dieu Très-Haut domine sur les royaumes des hommes, et qu'il y établit ceux qu'il lui plaît.
\VS{22}Et toi aussi, Belschatsar, son fils, tu n'as pas humilié ton cœur, quoique tu saches toutes ces choses.
\VS{23}Mais tu t'es élevé contre le Seigneur des cieux ; les vases de sa maison ont été apportés devant toi, et vous vous en êtes servis pour boire du vin, toi et tes grands, tes femmes et tes concubines ; tu as loué les dieux d'argent, d'or, d'airain, de fer, de bois et de pierre, qui ne voient point, qui n'entendent point, et qui ne savent rien, et tu n'as pas glorifié le Dieu dans la main duquel est ton souffle, et toutes tes voies\FTNT{Job. 12:10 et 33:4.}.
\VS{24}Alors de sa part a été envoyée cette partie de main, et cette écriture a été gravée.
\VS{25}Voici l'écriture qui a été gravée : Compté, compté, pesé et divisé.
\VS{26}Et voici l'interprétation de ces paroles. Compté : Dieu a compté ton règne, et y a mis la fin.
\VS{27}Pesé : Tu as été pesé dans la balance, et tu as été trouvé léger.
\VS{28}Mesuré : Ton royaume a été divisé, et donné aux Mèdes et aux Perses.
\VS{29}Aussitôt, Belschatsar donna des ordres, et l'on revêtit Daniel de pourpre, on lui mit un collier d'or au cou, et on publia qu'il serait le troisième dans le gouvernement du royaume.
\VS{30}Cette même nuit, Belschatsar, roi des Chaldéens, fut tué.
\VS{31}Et Darius, le Mède, reçut le royaume, étant âgé d'environ soixante-deux ans\FTNT{Es. 13:17 ; Es. 21:2 ; Jé. 51:11.}.
\Chap{6}
\TextTitle{Règne de Darius, le Mède}
\VerseOne{}Darius trouva bon d'établir sur le royaume cent vingt satrapes, qui devaient être répartis dans tout le royaume.
\VS{2}Il mit à leur tête trois chefs, au nombre desquels était Daniel, afin que ces satrapes leur rendent compte, et que le roi ne souffre aucun préjudice.
\VS{3}Daniel surpassait les autres chefs et satrapes, parce qu'il y avait en lui un Esprit supérieur ; et le roi pensait à l'établir sur tout le royaume.
\TextTitle{Daniel refuse l'idolâtrie et persévère dans la prière}
\VS{4}Alors les chefs et les satrapes cherchèrent une occasion d'accuser Daniel en ce qui concerne les affaires du royaume. Mais ils ne purent trouver en lui aucune occasion, ni aucune fausseté, parce qu'il était fidèle, et il ne se trouvait en lui ni faute ni vice.
\VS{5}Et ces hommes dirent : Nous ne trouverons aucune occasion d'accuser ce Daniel, à moins que nous n'en trouvions une dans la loi de son Dieu.
\VS{6}Alors ces chefs et ces satrapes se rendirent tumultueusement auprès du roi, et lui parlèrent ainsi : Roi Darius, vis éternellement !
\VS{7}Tous les chefs de ton royaume, les intendants, les satrapes, les conseillers, et les gouverneurs, sont d'avis d'établir un édit royal et une défense sévère, portant que quiconque, dans l'espace de trente jours, adressera des prières à quelque dieu ou à quelque homme, excepté à toi, ô roi, sera jeté dans la fosse aux lions.
\VS{8}Maintenant donc, ô roi, établis cette défense, et écris le décret afin qu'il soit irrévocable, selon la loi des Mèdes et des Perses, qui est immuable.
\VS{9}Là-dessus, le roi Darius écrivit le décret et la défense.
\VS{10}Lorsque Daniel sut que le décret était écrit, il entra dans sa maison, où les fenêtres de sa chambre étaient ouvertes dans la direction de Jérusalem ; et trois fois par jour, il se mettait à genoux, il priait, et il louait son Dieu, comme il le faisait auparavant\FTNT{1 R. 8:44 ; Ps. 55:17-18.}.
\VS{11}Alors ces hommes entrèrent tumultueusement, et ils trouvèrent Daniel qui priait et invoquait son Dieu.
\VS{12}Puis ils s'approchèrent du roi, et lui dirent au sujet de la défense royale : N'as-tu pas écrit une défense portant que tout homme dans l'espace de trente jours qui adresserait des prières à quelque dieu ou à quelque homme, excepté à toi, ô roi, serait jeté dans la fosse aux lions ? Le roi répondit : La chose est certaine, selon la loi des Mèdes et des Perses, qui est irrévocable.
\VS{13}Ils prirent de nouveau la parole et dirent au roi : Daniel, l'un des captifs de Juda, n'a tenu aucun compte de toi, ô roi, ni de la défense que tu as écrite, et il fait sa prière trois fois par jour.
\VS{14}Le roi fut très affligé quand il entendit cela ; il prit à cœur de délivrer Daniel, et jusqu'au coucher du soleil il s'efforça de le sauver.
\VS{15}Mais ces hommes se rendirent tumultueusement auprès du roi, et lui dirent : Sache, ô roi, que la loi des Mèdes et des Perses exige que toute défense ou tout décret établi par le roi soit irrévocable.
\TextTitle{Daniel demeure fidèle à Dieu face à la mort}
\VS{16}Alors le roi commanda qu'on amène Daniel, et qu'on le jette dans la fosse aux lions. Et le roi prenant la parole et dit à Daniel : Ton Dieu, lequel tu sers constamment, sera celui qui te délivrera.
\VS{17}On apporta une pierre, et on la mit sur l'ouverture de la fosse ; le roi la scella de son anneau, et de l'anneau de ses grands, afin que rien ne soit changé à l'égard de Daniel.
\TextTitle{Yahweh fait justice à Daniel}
\VS{18}Le roi se rendit ensuite dans son palais ; il passa la nuit à jeun, il ne fit point venir des danseuses\FTNT{Le mot « danseuse » vient de l'araméen « dachavah » qui signifie « divertissement », « instrument de musique », « danseuse », « concubine », « musique ».} auprès de lui, et il ne put se livrer au sommeil.
\VS{19}Puis le roi se leva au point du jour, avec l'aurore, et il alla précipitamment à la fosse aux lions.
\VS{20}En s'approchant de la fosse, il cria d'une voix triste : Daniel ! Le roi prit la parole et dit à Daniel : Daniel, serviteur du Dieu vivant, ton Dieu, que tu sers avec persévérance, a-t-il pu te délivrer des lions ?
\VS{21}Alors Daniel dit au roi : Ô roi, vis éternellement !
\VS{22}Mon Dieu a envoyé son ange, et a tellement fermé la gueule des lions, qu'ils ne m'ont fait aucun mal, parce que j'ai été trouvé innocent devant lui ; et même à ton égard, ô Roi ! je n'ai commis aucune faute. 
\VS{23}Alors le roi fut extrêmement heureux pour lui et il ordonna qu'on fasse retirer Daniel de la fosse. Ainsi Daniel fut retiré de la fosse, et on ne trouva sur lui aucune blessure, parce qu'il avait cru en son Dieu.
\VS{24}Le roi ordonna que ces hommes qui avaient accusé Daniel, soient amenés et jetés dans la fosse aux lions, eux, leurs enfants et leurs femmes, et avant qu'ils soient parvenus au fond de la fosse, les lions se saisirent d'eux, et leur brisèrent tous les os.
\TextTitle{Les merveilles de Yahweh proclamées aux nations}
\VS{25}Après cela, le roi Darius écrivit à tous les peuples, à toutes les nations, aux hommes de toutes les langues, qui habitent sur toute la terre : Que votre paix soit multipliée !
\VS{26}J'ordonne que dans toute l'étendue de mon royaume on ait de la crainte et de la frayeur pour le Dieu de Daniel, car c'est le Dieu vivant, et il subsiste éternellement ; son Royaume ne sera jamais détruit, et sa domination durera jusqu'à la fin\FTNT{Lu. 1:33 ; Es. 11.}.
\VS{27}Il sauve et délivre, il fait des prodiges et des merveilles dans les cieux et sur la terre, et il a délivré Daniel de la puissance des lions.
\VS{28}Ainsi Daniel prospéra sous le règne de Darius, et sous le règne de Cyrus, roi de Perse.
\Chap{7}
\TextTitle{Songe des quatre animaux ; Explication des visions de Daniel}
\VerseOne{}La première année de Belschatsar, roi de Babylone, Daniel eut un songe et des visions de sa tête, étant dans sa couche. Ensuite il écrivit le songe, et il relata les principales choses.
\VS{2}Daniel donc parla et dit : Je regardais dans ma vision nocturne, et voici, les quatre vents des cieux se levèrent avec impétuosité sur la grande mer.
\VS{3}Puis quatre grandes bêtes\FTNT{Les quatre bêtes représentent les quatre empires historiques. 
Le lion :
V. 4 : Le premier animal est un lion, il représente l'Empire néo-babylonien (625 – 539 av. J.-C.). Les ailes suggèrent la rapidité de la conquête babylonienne (Ha.1:6-8 ; Jé. 4:13). En 30 ans, l'Arabie, la Judée, la Syrie et la Phénicie furent conquises.
Les ailes arrachées annonçant l'arrêt des grandes conquêtes avec la mort de Nebucadnetsar. 
Le cœur d'homme donné au lion symbolise la conversion de Nebucadnetsar et le changement dans l'attitude des rois babyloniens (Da.4:30-31 ; 2 R. 25:27-30).
L'ours :
V. 5 : Le deuxième animal est un ours. Il représente l'empire médo-perse (539–331 av. J.-C.) qui succéda à l'Empire Babylonien.
Le fait que l'ours se tienne sur un côté indique que les Mèdes sont soumis aux Perses qui sont les véritables maîtres de l'empire. Les trois côtes dans la gueule de l'ours symbolisent trois grandes conquêtes médo-perses : la Lydie (546 av. J.-C.), la Babylonie (539 av. J.-C.) et l'Egypte (524 av. J.-C.).
Le léopard :
V. 6 : Le troisième animal est un léopard, qui représente l'Empire gréco-macédonien (331 – 146 av. J.-C.). En 331 av. J.-C., le coup de grâce est donné aux Médo-Perses à la bataille d'Arbèles.
Les quatre ailes symbolisent la grande rapidité des conquêtes. Quand Alexandre le Grand mourut à l'âge de 33 ans, il avait le plus grand Empire jamais vu jusqu'à l'époque. Ses conquêtes s'étendaient jusqu'en Inde !
Les quatre têtes symbolisent quatre de ses généraux qui, à la mort d'Alexandre, se partagèrent l'immense empire : Cassandre en Grèce et en Macédoine, Lysimaque en Thrace et en Asie Mineure, Séleucus en Syrie et en orient, Ptolémée en Égypte.
Très rapidement, la Palestine, qui se trouvait au croisement des routes, fut l'objet de rivalités entre les généraux et leurs successeurs. Après quelques années de stabilité, les généraux luttèrent entre eux jusqu'au maintien de deux dynasties : les Séleucides, au nord, et les Lagides, au sud, en Égypte. Cela dura jusqu'à l'apparition de l'Empire romain.
La quatrième bête, est différente des autres
V. 7, 19, 24 : La quatrième bête est extraordinaire, terrible, effrayante, elle ne porte même pas de nom ! Elle représente l'Empire romain qui succéda à l'Empire gréco-macédonien (146 av. J.-C. – 476 ap. J.-C.). En 168 av. J.-C., la Macédoine passa sous le contrôle de Rome, puis, en 146 av. J.-C., c'est au tour de la Grèce de devenir une province romaine.
 Le quatrième empire ne peut être que celui de Rome comme l'enseigne l'histoire de l'antiquité. 
Dès le quatrième siècle, l'Empire romain fut assailli par les tribus barbares venues du nord (Alamans, Wisigoths, Goths, Vandales, Burgondes, Ostrogoths, etc.) et, en 476, le dernier empereur romain d'occident, Romulus Augustule, fut chassé par le roi barbare Odoacre (Goth). L'Empire romain n'est plus.
Les orteils en partie de fer et en partie d'argile représentent les nations européennes issues de la fragmentation de l'Empire romain qui a eu lieu le 4 septembre 476.} montèrent de la mer, différentes les unes des autres.
\TextTitle{Premier empire universel: Babylone\FTNTT{Cp. Da. 2:37-38.}}
\VS{4}La première était semblable à un lion, et avait des ailes d'aigle ; je la regardai jusqu'à ce que les plumes de ses ailes furent arrachées ; elle fut enlevée de terre et dressée sur ses pieds comme un homme, et un cœur d'homme lui fut donné.
\TextTitle{Deuxième empire: Les Mèdes et les Perses\FTNTT{Cp. Da. 2:39 ; 8:20.}}
\VS{5}Et voici, une deuxième bête était semblable à un ours, et se tenait sur un côté ; il avait trois côtes dans la gueule entre ses dents ; et on lui disait ainsi : Lève-toi, mange beaucoup de chair.
\TextTitle{Troisième empire: La Grèce\FTNTT{Cp. Da. 2:39 ; 8:21-22 ; 11:2-4.}}
\VS{6}Après cela je regardai, et voici une autre bête, semblable à un léopard, qui avait sur son dos quatre ailes d'oiseau, et cette bête avait quatre têtes, et la domination lui fut donnée.
\TextTitle{Quatrième empire: Rome\FTNTT{Cp. Da. 2:40-43 ; 7:23-24 ; 9:26.}}
\VS{7}Après cela, je regardai dans mes visions nocturnes, et voici, il y avait une quatrième bête, terrible, épouvantable et extraordinairement forte ; elle avait de grandes dents de fer, elle mangeait, brisait, et elle foulait à ses pieds ce qui restait ; elle était différente de toutes les bêtes qui avaient été avant elle, et elle avait dix cornes.
\TextTitle{Les dix cornes et la petite corne\FTNTT{Da. 7:24-27.}}
\VS{8}Je considérai ses cornes, et voici, une autre petite corne sortit du milieu d'elles, et trois des premières cornes furent arrachées par elle ; et voici, elle avait des yeux comme des yeux d'homme, et une bouche qui proférait de grandes choses.
\TextTitle{Le règne de Yahweh, l'Ancien des jours\FTNTT{Cp. Mt. 24:27-30 ; 25:31-34 ; Ap. 19:11-21.}}
\VS{9}Je regardai jusqu'à ce que les trônes soient placés. Et l'Ancien des jours s'assit. Son vêtement était blanc comme la neige, et les cheveux de sa tête étaient comme de la laine pure ; son trône était des flammes de feu, et ses roues un feu ardent.
\VS{10}Un fleuve de feu coulait et sortait de devant lui. Mille milliers le servaient, et dix mille millions se tenaient en sa présence. Le jugement se tint, et les livres furent ouverts\FTNT{Ap. 5:11 ; Ps. 68:18 ; 1 R. 22:19.}.
\VS{11}Je regardai alors, à cause du bruit des paroles arrogantes que proférait la corne ; et tandis que je regardais, la bête fut tuée, et son corps fut détruit et livré pour être brûlé au feu.
\VS{12}Les autres bêtes furent dépouillées de leur domination, mais une prolongation de vie leur fut accordée jusqu'à un temps déterminé.
\TextTitle{La domination du Fils de l'homme est éternelle\FTNTT{Cp. Ap. 5:1-14.}}
\VS{13}Je regardai encore dans les visions nocturnes, et je vis, comme le Fils de l'homme, qui venait avec les nuées des cieux, et il vint jusqu'à l'Ancien des jours, et se tint devant lui\FTNT{Ap. 19:14 ; Jud. 1:14.}.
\VS{14}Et il lui donna la domination, la gloire et le règne ; et tous les peuples, les nations et les langues le serviront. Sa domination est une domination éternelle qui ne passera point, et son règne ne sera jamais détruit.
\TextTitle{Interprétation de la vision du quatrième animal}
\VS{15}Moi, Daniel, j'eus l'esprit troublé au-dedans de moi, et les visions de ma tête m'effrayèrent.
\VS{16}Je m'approchai de l'un des assistants, et lui demandai ce qu'il y avait de vrai dans toutes ces choses. Il me parla, et me donna l'interprétation de ces choses, en disant :
\VS{17}Ces quatre grandes bêtes sont quatre rois, qui s'élèveront de la terre.
\VS{18}Mais les saints du Très-Haut recevront le Royaume, et ils posséderont le Royaume éternellement, d'éternité en éternité.
\VS{19}Alors, je désirai savoir la vérité sur la quatrième bête, qui était différente de toutes les autres, extraordinairement terrible, qui avait des dents de fer et des ongles d'airain, qui mangeait, brisait, et foulait à ses pieds ce qui restait ;
\VS{20}et sur les dix cornes qu'elle avait à la tête, et sur l'autre corne qui était sortie et devant laquelle trois étaient tombées, sur cette corne qui avait une bouche parlant avec arrogance, et une plus grande apparence que celle de ses associées.
\VS{21}Je regardai comment cette corne faisait la guerre aux saints et l'emportait sur eux\FTNT{Ap. 13:2-7.},
\VS{22}jusqu'au moment où l'Ancien des jours vint donner droit aux saints du Très-Haut, et que le temps arriva où les saints furent en possession du Royaume.
\VS{23}Il me parla donc ainsi : La quatrième bête est un quatrième royaume qui sera sur la terre, différent de tous les royaumes, et qui dévorera toute la terre, la foulera, et la brisera.
\TextTitle{Règne de l'homme impie et jugement de Dieu}
\VS{24}Mais les dix cornes sont dix rois qui s'élèveront de ce royaume. Un autre s'élèvera après eux, il sera différent des premiers, et il abattra trois rois.
\VS{25}Il proférera des paroles contre le Très-Haut, il harcelera les saints du Très-Haut, et il aura l'intention de changer les temps et la loi ; et les saints seront livrés entre ses mains pendant un temps, des temps, et la moitié d'un temps.
\VS{26}Mais le jugement se tiendra, et on lui ôtera sa domination, en la détruisant et la faisant périr, jusqu'à en voir la fin..
\VS{27}Afin que le règne, la domination, et la grandeur de tous les royaumes qui sont sous les cieux, soient donnés au peuple des saints du Très-Haut. Son royaume est un royaume éternel, et tous les royaumes lui seront assujettis et lui obéiront.
\VS{28}Jusqu'ici est la fin de cette affaire. Quant à moi, Daniel, mes pensées m'effrayèrent beaucoup, et ma splendeur changea en moi, toutefois je gardais cette affaire dans mon cœur.
\Chap{8}
\TextTitle{Vision du bélier et du bouc}
\VerseOne{}La troisième année du règne du roi Belschatsar, moi, Daniel, j'eus cette vision, en plus de celle que j'avais eue auparavant.
\VS{2}Je vis cette vision, et il arriva, comme je regardais, que j'étais à Suse, la capitale, dans la province d'Élam, et dans ma vision, je me trouvais près du fleuve d'Ulaï.
\VS{3}Et je levai mes yeux, je regardai, et voici, un bélier se tenait devant le fleuve, et il avait deux cornes ; et les deux cornes étaient hautes, mais l'une était plus haute que l'autre, et la plus haute s'éleva sur la dernière.
\VS{4}Je vis ce bélier qui frappait de ses cornes à l'occident, au nord, et au midi ; aucune bête ne pouvait subsister devant lui,et il n'y avait personne qui puisse délivrer de sa puissance ; et il agissait selon sa volonté et devenait grand.
\VS{5}Comme je regardais attentivement, voici, un bouc d'entre les chèvres venait de l'occident, et parcourait toute la terre à sa surface, sans la toucher ; ce bouc avait entre les yeux une corne considérable.
\VS{6}Il arriva jusqu'au bélier qui avait deux cornes et que j'avais vu se tenant devant le fleuve, et il courut sur lui dans la fureur de sa force.
\VS{7}Je le vis qui s'approchait du bélier et s'irritait contre lui ; il frappa le bélier et lui brisa les deux cornes, et le bélier n'avait aucune force pour tenir ferme contre lui ; et quand il l'eut jeté par terre, il le foula; et il n'y eut personne pour délivrer le bélier de sa puissance.
\VS{8}Alors le bouc d'entre les chèvres grandit extrêmement ; mais lorsqu'il fut puissant, sa grande corne se brisa. Quatre grandes cornes s'élevèrent pour la remplacer, aux quatre vents des cieux.
\TextTitle{La petite corne renverse la vérité}
\VS{9}De l'une d'elles sortit une petite corne\FTNT{Antiochus IV Épiphane est le fils d'Antiochos III le Grand, né vers 215 av. J.-C. Il gouverna le royaume séleucide de 175 av. J.-C. à 164 av. J.-C., date de sa mort. Ce dernier avait profané le temple de Jérusalem en sacrifiant des porcs sur l'autel (Voir commentaire en Mt. 24:15). Cette petite corne, qui fait tomber par terre une partie de l'armée des étoiles, agit de même que Satan au ciel qui avait fait chuter un tiers des étoiles, soit des anges (Ap. 12:3-4).}, qui s'agrandit beaucoup vers le midi, et vers l'orient, et vers le pays de noblesse.
\VS{10}Elle s'éleva même jusqu'à l'armée des cieux, elle fit tomber à terre une partie de l'armée et des étoiles, et elle les foula\FTNT{Es. 14:12-15 ; Ez. 28:12-19.}.
\VS{11}Et elle s'éleva même jusqu'au chef de l'armée, lui enleva le sacrifice perpétuel, et renversa la demeure de son sanctuaire.
\VS{12}L'armée fut livrée avec le sacrifice perpétuel, à cause du péché ; la corne jeta la vérité par terre, et fit de grands exploits, et prospéra.
\VS{13}Alors j'entendis un saint qui parlait ; et un autre saint disait à celui qui parlait : Jusqu'à quand durera cette vision sur le sacrifice perpétuel et sur le péché qui cause la désolation ? Jusqu'à quand le sanctuaire et l'armée seront-ils foulés ?
\VS{14}Et il me dit : Deux mille trois cents soirs et matins ; puis le sanctuaire sera purifié.
\TextTitle{La vision du bélier et du bouc interprétée}
\VS{15}Et quand à moi, Daniel, j'avais cette vision et que je désirais la comprendre, voici, quelqu'un qui avait l'apparence d'un homme se tenait devant moi.
\VS{16}Et j'entendis la voix d'un homme au milieu du fleuve Ulaï ; il cria et dit : Gabriel, explique-lui la vision.
\VS{17}Puis Gabriel vint alors près du lieu où je me tenais ; et à son approche, je fus effrayé, et je tombai sur ma face. Il me dit : Comprends, fils de l'homme, car la vision est pour le temps de la fin.
\VS{18}Comme il me parlait, je restai frappé d'étourdissement, la face contre terre. Il me toucha, et me fit tenir debout à la place où je me trouvais.
\VS{19}Et il dit : Voici, je vais t'apprendre ce qui arrivera à la fin de la colère, car il y a un temps marqué pour la fin.
\VS{20}Le bélier que tu as vu qui avait deux cornes, ce sont les rois des Mèdes et des Perses ;
\VS{21}et le bouc velu, c'est le roi de Javan\FTNT{Javan ou Grèce.} ; et la grande corne entre ses yeux, c'est le premier roi.
\VS{22}Les quatre cornes qui se sont élevées pour remplacer cette corne brisée, ce sont quatre royaumes qui s'élèveront de cette nation, mais qui n'auront pas autant de force.
\TextTitle{Le roi impie, adversaire de Dieu ; la vision scellée}
\VS{23}A la fin de leur règne, lorsque les pécheurs seront consumés, il se lèvera un roi cruel et artificieux.
\VS{24}Sa puissance s'accroîtra, mais non par sa propre force ; il fera d'incroyables ravages, il réussira dans ses entreprises, il détruira les puissants et le peuple des saints.
\VS{25}Et par la subtilité de son esprit, il fera prospérer la fraude dans sa main. Il aura de l'arrogance dans le cœur, et fera périr beaucoup d'hommes qui vivaient dans la paix, et il s'élèvera contre le Prince des princes ; mais il sera brisé, sans l'effort d'aucune main.
\VS{26}Et la vision du soir et du matin, dont il s'agit, est véritable. Mais toi, scelle la vision, car elle se rapporte à un temps éloigné.
\VS{27}Moi Daniel, je fus tout défait et malade pendant quelques jours ; puis je me levai, et je m'occupai des affaires du roi. J'étais étonné de la vision, et personne n'en eut connaissance.
\Chap{9}
\TextTitle{Supplications de Daniel à Yahweh}
\VerseOne{}La première année de Darius, fils d'Assuérus, de la race des Mèdes, lequel était établi roi sur le royaume des Chaldéens.
\VS{2}La première année, dis-je, de son règne, moi Daniel, je discernai par les livres, que le nombre des années dont Yahweh avait parlé au prophète Jérémie\FTNT{Jé. 25:11.} pour finir les désolations de Jérusalem, était de soixante et dix ans. 
\VS{3}Et je tournai ma face vers le Seigneur Dieu, pour le chercher par la prière et des supplications, avec jeûne, et le sac et la cendre.
\VS{4}Je priai Yahweh, mon Dieu, et je lui fis ma confession : Ah ! Seigneur, Dieu grand et redoutable, toi qui gardes ton alliance et qui fais miséricorde à ceux qui t'aiment et qui gardent tes commandements !
\VS{5}Nous avons péché, nous avons commis l'iniquité, nous avons agi méchamment, nous avons été rebelles, et nous nous sommes détournés de tes commandements et de tes ordonnances.
\VS{6}Nous n'avons pas écouté tes serviteurs, les prophètes, qui ont parlé en ton Nom à nos rois, à nos chefs, à nos pères, et à tout le peuple du pays.
\VS{7}Ô Seigneur ! A toi est la justice, et à nous la confusion de face, en ce jour, aux hommes de Juda, aux habitants de Jérusalem, et à tout Israël, à ceux qui sont près et à ceux qui sont loin, dans tous les pays où tu les as dispersés, à cause des infidélités dont ils se sont rendus coupables envers toi\FTNT{Né. 9:30 ; Ps. 106:6 ; La. 3:42.}.
\VS{8}Seigneur, à nous est la confusion de face, à nos rois, à nos chefs, et à nos pères, parce que nous avons péché contre toi.
\VS{9}Auprès du Seigneur, notre Dieu, la miséricorde et le pardon, car nous avons été rebelles envers lui.
\VS{10}Nous n'avons pas écouté la voix de Yahweh, notre Dieu, pour marcher dans ses lois, qu'il a mises devant nous par le moyen de ses serviteurs, les prophètes.
\VS{11}Tout Israël a transgressé ta loi, et s'est détourné pour ne pas écouter ta voix. Alors se sont répandues sur nous les malédictions et les imprécations qui sont écrites dans la loi de Moïse, serviteur de Dieu, parce que nous avons péché contre Dieu\FTNT{Lé. 26:14-33 ; De. 27:15-33.}.
\VS{12}Il a accompli les paroles qu'il avait prononcées contre nous, et contre nos chefs qui nous ont gouvernés, et il a fait venir sur nous un grand mal, et il n'en est jamais arrivé sous le ciel entier un semblable à celui qui est arrivé à Jérusalem.
\VS{13}Comme cela est écrit dans la loi de Moïse, ce mal est venu sur nous ; et nous n'avons pas imploré Yahweh, notre Dieu, pour nous détourner de nos iniquités, et pour nous rendre attentifs à ta vérité.
\VS{14}Yahweh a veillé sur le mal que nous avons fait et il l'a fait venir sur nous ; car Yahweh, notre Dieu, est juste dans toutes les œuvres qu'il a faites, vu que nous n'avons point obéi à sa voix.
\VS{15}Or maintenant, Seigneur, notre Dieu ! Toi qui as tiré ton peuple du pays d'Egypte par ta main puissante, et qui t'es acquis un Nom comme il l'est aujourd'hui, nous avons péché, nous avons été méchants.
\VS{16}Seigneur, je te prie que selon ta justice, que ta colère et ton indignation se détournent de ta ville de Jérusalem, de la montagne de ta sainteté ; car à cause de nos péchés et des iniquités de nos pères, Jérusalem et ton peuple sont en opprobre à tous ceux qui nous entourent.
\VS{17}Maintenant donc, ô notre Dieu, écoute la prière et les supplications de ton serviteur, et pour l'amour du Seigneur, fais briller ta face sur ton sanctuaire dévasté.
\VS{18}Mon Dieu ! Prête l'oreille, et écoute ; ouvre tes yeux, et regarde nos ruines, et la ville sur laquelle ton Nom a été invoqué ; car ce n'est pas à cause de notre justice que nous te présentons nos supplications, c'est à cause de tes grandes compassions.
\VS{19}Seigneur, exauce, Seigneur pardonne, Seigneur sois attentif, et opère ; ne tarde pas, par amour pour toi, ô mon Dieu ! Car ton Nom a été invoqué sur ta ville, et sur ton peuple.
\VS{20}Je parlais encore, je priais, je confessais mon péché, et le péché de mon peuple d'Israël, et je présentais ma supplication à Yahweh, mon Dieu, en faveur de la sainte montagne de mon Dieu.
\TextTitle{Les soixante-dix semaines}
\VS{21}Je parlais encore dans ma prière, quand l'homme Gabriel, que j'avais vu précédemment dans une vision, s'approcha de moi d'un vol rapide au moment de l'offrande du soir.
\VS{22}Il m'instruisit, et s'entretint avec moi. Il me dit : Daniel, je suis venu maintenant pour ouvrir ton intelligence.
\VS{23}La parole est sortie dès le commencement de tes supplications, et je suis venu pour te la déclarer, car tu es un bien-aimé. Sois attentif à la parole, et comprends la vision.
\VS{24}Il y a soixante-dix semaines\FTNT{Le verset 24 concerne la chronologie de l'accomplissement de la prophétie de Jérémie (25 : 11). Les soixante-dix semaines auxquelles elle fait allusion représentent une période de 490 ans, conformément au principe biblique prophétique selon lequel un jour prophétique équivaut à une année (No. 14:33-34 ; Ez. 4:4-6). Dans les versets 25 et 27, les soixante-dix semaines sont divisées en trois périodes : 7 semaines (49 ans), 62 semaines (434 ans), et une semaine (7 ans). Les soixante-dix semaines devaient débuter au moment où la parole a annoncé que Jérusalem serai rebâtie (v. 25). En 445 avant notre ère, dans la vingtième année de son règne, le roi Artaxerxès publia un décret permettant à Esdras de retourner à Jérusalem pour achever la reconstruction de la ville (Esd. 7:6-10 ; Esd. 9:9 ; Né 2:5). Il est attesté par l'histoire profane que cette date est le point de départ de la soixante-dixième semaine de Daniel. Les 69 premières semaines vont jusqu'au Messie conducteur. La semaine qui reste (7 ans) concerne la période du règne de l'Antéchrist, celle-ci est divisée en deux période : trois ans et demi de fausse paix (1 Th. 5:3), et trois ans et demi concernent la Grande Tribulation (Ap. 7:9-17 ; Ap. 11:1-3 ; Ap. 12:6 ; Ap. 13:5).} fixées sur ton peuple et sur ta ville sainte, pour abolir la transgression et mettre fin aux péchés, faire la propitiation pour l'iniquité, pour amener la justice éternelle, pour mettre le sceau à la vision et à la prophétie et pour oindre le Saint des saints.
\VS{25}Tu sauras donc et tu comprendras, que depuis le moment où la parole a annoncé que Jérusalem sera rebâtie jusqu'au Messie, le Conducteur, il y a sept semaines et soixante-deux semaines ; et les places et les brèches seront rebâties, mais en des temps d'angoisse.
\VS{26}Et après ces soixante-deux semaines, le Messie sera retranché, mais non pas pour lui. Le peuple du chef qui viendra, détruira la ville et le sanctuaire, et sa fin arrivera comme par une inondation ; il est déterminé que les dévastations dureront jusqu'à la fin de la guerre.
\VS{27}Et il confirmera l'alliance à plusieurs pour une semaine, et à la moitié de cette semaine il fera cesser le sacrifice, et l'offrande ; puis par le moyen des ailes abominables, qui causeront la désolation, même jusqu'à une consomption déterminée, la désolation fondra sur le désolé.
\Chap{10}
\TextTitle{Daniel voit la gloire du Messie}
\VerseOne{}La troisième année de Cyrus, roi de Perse, une parole fut révélée à Daniel, qu'on nommait Beltschatsar. Cette parole est véritable et annonce une grande guerre. Il fut attentif à cette parole, et il eut l'intelligence de la vision.
\VS{2}En ce temps-là, moi Daniel, je fus dans le deuil pendant trois semaines entières.
\VS{3}Je ne mangeai aucun mets délicat, il n'entra ni viande ni vin dans ma bouche, et je ne m'oignis point, jusqu'à ce que ces trois semaines entières soient accomplies.
\VS{4}Le vingt-quatrième jour du premier mois, j'étais au bord du grand fleuve qui est Hiddékel.
\VS{5}Je levai les yeux, et je regardai, et voici, il y avait un homme vêtu de lin, et ayant sur les reins une ceinture d'or fin d'Uphaz.
\VS{6}Son corps était comme de chrysolithe, et son visage brillait comme l'éclair, ses yeux étaient comme des flammes de feu, ses bras et ses pieds ressemblaient à de l'airain poli, et le son de sa voix était comme le bruit d'une multitude de gens\FTNT{Ap. 1:13-15.}.
\VS{7}Moi, Daniel, je vis seul la vision, et les hommes qui étaient avec moi ne la virent point ; mais ils furent saisis d'une grande frayeur, et ils s'enfuirent pour se cacher.
\VS{8}Je restai seul, et je vis cette grande vision ; les forces me manquèrent, mon visage changea de couleur et fut tout défait, et je ne conservai aucune vigueur.
\VS{9}J'entendis le son de ses paroles ; et comme j'entendais le son de ses paroles, je tombai frappé d'étourdissement, la face contre terre.
\TextTitle{Le combat du monde spirituel}
\VS{10}Et voici, une main me toucha et me fit mettre sur mes genoux, et sur les paumes de mes mains.
\VS{11}Puis il me dit : Daniel, homme aimé de Dieu, sois attentif aux paroles que je vais te dire, et tiens-toi debout à la place où tu es ; car je suis maintenant envoyé vers toi. Lorsqu'il m'eut ainsi parlé, je me tins debout en tremblant.
\VS{12}Il me dit : Ne crains rien, Daniel, car dès le premier jour où tu as appliqué ton cœur à comprendre, et à t'humilier devant ton Dieu, tes paroles ont été exaucées, et c'est à cause de tes paroles que je viens.
\VS{13}Mais le chef du royaume de Perse m'a résisté vingt et un jours ; mais voici, Micaël, l'un des principaux chefs, est venu à mon secours, et je suis demeuré là auprès des rois de Perse.
\VS{14}Je viens maintenant pour te faire connaître ce qui doit arriver à ton peuple dans les derniers jours, car la vision s'étend jusqu'à ces temps-là.
\VS{15}Pendant qu'il m'adressait ces paroles, je mis mon visage contre terre, et je gardai le silence.
\VS{16}Et voici, quelqu'un qui avait l'apparence des fils de l'homme toucha mes lèvres. J'ouvris la bouche, je parlai, et je dis à celui qui se tenait devant moi : Mon seigneur ! La vision m'a rempli d'effroi, et j'ai perdu toute vigueur.
\VS{17}Comment le serviteur de mon seigneur pourrait-il parler avec mon seigneur ? Maintenant les forces me manquent, et je n'ai plus de souffle.
\VS{18}Alors celui qui avait l'apparence d'un homme me toucha encore, et me fortifia.
\VS{19}Puis il me dit : Ne crains rien, homme bien-aimé, que la paix soit avec toi ! Fortifie-toi, fortifie-toi ! Et comme il me parlait, je repris des forces, et je dis : Que mon seigneur parle, car tu m'as fortifié.
\VS{20}Il me dit : Ne sais-tu pas pourquoi je suis venu vers toi ? Maintenant je m'en retournerai pour combattre le chef de Perse ; et quand je partirai, voici, le chef de Javan viendra.
\VS{21}Mais je veux te faire connaître ce qui est écrit dans le livre de vérité. Et il n'y a personne qui me soutienne contre ceux-là, excepté Micaël, votre chef.
\Chap{11}
\TextTitle{Succession des monarques jusqu'à l'homme impie\FTNTT{Da. 11:1 - 12:13.}}
\VerseOne{}Et moi, dans la première année de Darius, le Mède, je me tenais auprès de lui pour l'aider et le fortifier.
\VS{2}et maintenant, je vais te faire connaître la vérité : Voici, il y aura encore trois rois en Perse. Le quatrième amassera plus de richesses que les autres ; et quand il sera puissant par ses richesses, il soulèvera tout le monde contre le royaume de Javan.
\VS{3}Mais il s'élèvera un vaillant roi\FTNT{Ce vaillant roi est Alexandre le Grand qui règna de 336 à 323 av. J.-C.}, qui dominera avec une grande puissance, et fera ce qu'il voudra.
\VS{4}Et sitôt qu'il sera élevé, son royaume sera brisé et sera divisé\FTNT{A la mort d'Alexandre le Grand, ses quatre principaux généraux se partagèrent l'empire : 
-Lysimaque régna sur l'Asie mineure. 
-Cassandre régna sur la Grèce et la Macédoine.
-Seleucos régna en Syrie, en Babylonie et sur toutes les régions à l'est jusqu'aux Indes. 
-Ptolémée régna sur l'Égypte, la Judée et une partie de la Syrie.} vers les quatre vents des cieux ; il ne passera point à ses descendants, et n'aura pas la même puissance qu'il a exercée, car son royaume sera déchiré, et il passera à d'autres qu'à eux.
\VS{5}Le roi du midi\FTNT{Le roi du midi est Ptolémée 1er Soter (règne : 323-285 av. J.-C.), le chef plus fort que lui est Séleucus 1er Nicator (règne : 305-281 av. J.-C.).} deviendra fort et puissant. Mais un de ses chefs [roi de Javan] sera plus puissant que lui et dominera ; sa domination sera puissante.
\VS{6}Au bout de quelques années, ils s'allieront, et la fille du roi du midi viendra vers le roi du nord pour redresser les affaires. Mais elle ne conservera pas la force de son bras, et il ne résistera pas, ni lui ni son bras ; elle sera livrée avec ceux qui l'auront amenée, avec son père et avec celui qui aura été son soutien dans ce temps-là.
\VS{7}Mais un rejeton de ses racines s'élèvera pour le remplacer\FTNT{Ptolémée III Evergète (règne : 246-222 av. J.-C.)} ; il viendra à l'armée, il entrera dans les forteresses du roi du nord, il en disposera à son gré, et il se rendra puissant.
\VS{8}Et même il emmènera captifs en Egypte leurs dieux, avec leurs images de fonte et avec leurs vases précieux d'argent et d'or. Puis il restera quelques années éloigné du roi du nord.
\VS{9}Et celui-ci marchera contre le royaume du roi du midi, et retournera dans son pays.
\VS{10}Ses fils\FTNT{Ses fils : Ce sont les deux rois de Syrie, Seuleucus III Ceraunus (règne : 225-223 av. J.-C.), et Antiochus III le Grand (223-187 av. J.-C.).} entreront en guerre et rassembleront une multitude nombreuse de troupes ; l'un d'eux s'avancera et se répandra comme un torrent, débordera, puis reviendra ; et il poussera la guerre jusqu'à la forteresse du roi du midi.
\VS{11}Et le roi du midi sera irrité, il sortira et combattra contre lui, savoir contre le roi du nord ; il soulèvera une grande multitude, et les troupes du roi du nord seront livrées entre les mains du roi du midi.
\VS{12}Et après avoir défait cette multitude, le cœur du roi s'élèvera ; il fera tomber des milliers, mais il ne triomphera pas.
\VS{13}Car le roi du nord reviendra et rassemblera une plus grande multitude que la première ; au bout de quelque temps, de quelques années, il viendra avec une grande armée et de grandes richesses.
\VS{14}Et en ce temps-là, plusieurs s'élèveront contre le roi du midi ; et des hommes violents parmi ton peuple se révolteront pour accomplir la vision, mais ils succomberont.
\VS{15}Le\FTNT{Le roi du nord est Séleucos IV Philopator (règne :187-175 av. J.-C.)} roi du nord viendra, il élèvera des terrasses, et prendra les villes fortes. Les bras du midi et l'élite du roi ne résisteront pas, ils manqueront de force pour résister.
\VS{16}Celui qui marchera contre lui fera ce qu'il voudra, et personne ne lui résistera ; il s'arrêtera dans le pays de noblesse, exterminant ce qui tombera sous sa main.
\VS{17}Puis il tournera sa face pour entrer avec la force de tout son royaume, et fera un accord avec le roi du midi, et il lui donnera sa fille pour femme, pour ruiner le royaume ; mais cela ne tiendra pas, et elle ne sera pas pour lui. 
\VS{18}Puis il tournera ses vues vers les îles, et il en prendra plusieurs ; mais un chef mettra fin à l'opprobre qu'il voulait lui attirer, et le fera retomber sur lui.
\VS{19}Il se dirigera ensuite vers les forteresses de son pays ; et il chancellera, il tombera, et on ne le trouvera plus.
\VS{20}Et un autre sera établi à sa place, qui fera passer un exacteur dans l'ornement du royaume, et en peu de jours il sera brisé, et ce ne sera ni par la colère ni par la guerre.
\TextTitle{Usage de la tromperie pour régner}
\VS{21}Et sa place il en sera établi un autre qui sera méprisé, auquel on ne donnera pas l'honneur royal ; mais il viendra en paix, et il s'emparera du royaume par des flatteries.
\VS{22}Les troupes qui se répandront comme un torrent seront submergées devant lui, et brisées, de même qu'un chef de l'alliance.
\VS{23}Mais après les accords faits avec lui, il usera de tromperie, et il montera, et il aura le dessus avec peu de gens.
\VS{24} Il entrera tranquillement dans les lieux les plus riches de la province, et il fera ce que n'avaient pas fait ses pères, ni les pères de ses pères ; il distribuera le butin, le pillage et les richesses ; et il formera des desseins contre les places fortes, et cela jusqu'à un certain temps.
\VS{25}Puis il réveillera sa force et son coeur contre le roi du midi avec une grande armée. Et le roi du midi s'avancera en bataille avec une très grande et très forte armée ; mais il ne résistera pas, car on formera des complots contre lui.
\VS{26}Ceux qui mangent les mets de sa table le mettront en pièces ; son armée se répandra comme un torrent, et beaucoup de gens tomberont blessés à mort.
\VS{27}Et les deux rois chercheront en leur cœur à se nuire, et à la même table ils parleront avec fausseté. Mais cela ne réussira pas ; car la fin ne viendra qu'au temps marqué.
\VS{28}Après quoi il retournera dans son pays avec de grandes richesses ; et son cœur sera contre la sainte alliance, il agira contre elle, puis retournera dans son pays.
\VS{29}Ensuite il retournera au temps fixé, et il viendra contre le midi ; mais cette dernière expédition ne sera pas comme la précédente.
\VS{30}Car les navires de Kittim viendront contre lui ; affligé, il rebroussera chemin. Puis irrité contre la sainte alliance, il agira contre elle, il retournera et s'entendra avec les apostats de la sainte alliance.
\VS{31}Et les forces seront de son côté, et on profanera le sanctuaire qui est la forteresse, et on fera cesser le sacrifice perpétuel, et on y dressera l'abomination qui causera la désolation.
\VS{32}Et il corrompra par des flatteries ceux qui agissent méchamment à l'égard de l'alliance. Mais ceux du peuple qui connaîtront leur Dieu agiront avec courage.
\VS{33}Et les plus intelligents parmi le peuple donneront instruction à plusieurs. Il en est qui succomberont pour un temps à l'épée et à la flamme, à la captivité et au pillage.
\VS{34}Dans le temps où ils succomberont, ils seront un peu secourus, et plusieurs se joindront à eux par hypocrisie.
\VS{35}Et quelques-uns des hommes intelligents succomberont, afin qu'ils soient épurés, purifiés et blanchis, jusqu'au temps de la fin, car elle n'arrivera qu'au temps marqué.
\TextTitle{Blasphème du roi contre Yahweh, le Dieu des dieux}
\VS{36}Le roi fera ce qu'il voudra, il s'élèvera, il se glorifiera au-dessus de tous les dieux ; il proférera des choses étranges contre le Dieu des dieux, il prospérera jusqu'à ce que la colère soit consommée, car ce qui est décrété sera exécuté.
\VS{37}Il n'aura égard ni aux dieux de ses pères, ni à l'objet du désir des femmes ; il n'aura égard à aucun dieu ; car il s'élèvera au-dessus de tout.
\VS{38}Mais, à la place, il honorera le dieu Mahuzzim ; ce dieu que ses pères n'ont pas connu, il rendra des hommages avec de l'or et de l'argent, et des pierres précieuses, et des objets de prix.
\VS{39}C'est avec le dieu étranger qu'il agira contre les lieux les plus fortifiés ; et il comblera d'honneurs ceux qui le reconnaîtront, il les fera dominer sur plusieurs, il leur partagera des terres à prix d'argent.
\VS{40}Au temps de la fin, le roi du midi se heurtera contre lui de ses cornes. Et le roi du nord fondra sur lui comme une tempête, avec des chars et des cavaliers, et avec de nombreux navires ; il s'avancera dans les terres, se répandra comme un torrent et débordera.
\VS{41}Il entrera dans le pays de noblesse, et plusieurs pays succomberont ; mais Edom, Moab et les principaux des enfants d'Ammon seront délivrés de sa main.
\VS{42}Il étendra sa main sur ces pays-là, et le pays d'Egypte n'échappera point.
\VS{43}Il se rendra maître des trésors d'or et d'argent, et de toutes les choses précieuses de l'Egypte ; les Libyens et les Ethiopiens seront à sa suite.
\VS{44}Mais des nouvelles de l'orient et du nord viendront le troubler, et il partira avec une grande fureur, pour détruire et exterminer beaucoup de gens.
\VS{45}Il dressera les tentes de son palais entre les mers, vers la glorieuse et sainte montagne. Puis il arrivera à la fin, et personne ne lui donnera du secours.
\Chap{12}
\TextTitle{La résurrection pour le jugement éternel}
\VerseOne{}Or, en ce temps-là Michaël, ce grand Chef qui tient ferme pour les enfants de ton peuple, tiendra ferme ; et ce sera un temps de détresse, tel qu'il n'y en a point eu de semblable depuis que les nations existent jusqu'à ce temps-là. En ce temps-là, ceux de ton peuple qui seront trouvés inscrits dans le livre seront sauvés.
\TextTitle{Les deux résurrections}
\VS{2}Plusieurs de ceux qui dorment dans la poussière de la terre se réveilleront\FTNT{Il est question ici de la résurrection. Tout d'abord, il y aura la résurrection des morts en Christ, lors du retour de Jésus-Christ (1 Th. 4:12-17). Ensuite, il y aura celle de tous les saints lors du retour de Christ avec l'Eglise (Ap. 19 et 20). Enfin, la dernière résurrection interviendra à l'issue du millénium. Il s'agit de la résurrection des impies (Ap. 20:11-15). Voir également Jn. 5 : 24-29 ; Jn. 11:25.}, les uns pour la vie éternelle, et les autres pour l'opprobre, pour l'infamie éternelle.
\VS{3}Ceux qui auront été intelligents, brilleront comme la splendeur du ciel, et ceux qui auront amené plusieurs à la justice brilleront comme les étoiles, à toujours et à perpétuité\FTNT{Mt. 13:43.}.
\TextTitle{Dernières paroles de Yahweh à Daniel ; le livre scellé jusqu'au temps de la fin}
\VS{4}Mais toi, Daniel, tiens secrètes ces paroles, et scelle le livre jusqu'au temps de la fin. Plusieurs le liront et la connaissance augmentera\FTNT{Ap. 10:4 ; Ap. 5:2.}.
\VS{5}Et moi, Daniel, je regardai, et voici, deux autres hommes se tenaient debout, l'un en deçà du bord du fleuve, et l'autre au-delà du bord du fleuve.
\VS{6}L'un d'eux dit à l'homme vêtu de lin, qui se tenait au-dessus des eaux du fleuve : Quand sera la fin de ces merveilles ?
\VS{7}Et j'entendis l'homme vêtu de lin, qui se tenait au-dessus des eaux du fleuve ; il leva sa main droite et sa main gauche vers les cieux, et il jura par celui qui vit éternellement que ce sera dans un temps, des temps, et la moitié d'un temps, et que toutes ces choses finiront quand la force du peuple saint sera entièrement brisée.
\VS{8}J'entendis, mais je ne compris pas ; et je dis : Mon seigneur, quelle sera l'issue de ces choses ?
\VS{9}Il répondit : Va, Daniel, car ces paroles sont tenues secrètes et scellées jusqu'au temps de la fin.
\VS{10}Plusieurs seront purifiés, blanchis et éprouvés ; mais les méchants agiront avec méchanceté, et aucun des méchants ne comprendra, mais les sages comprendront.
\VS{11}Depuis le temps où cessera le sacrifice perpétuel et où sera dressée l'abomination de la désolation, il y aura mille deux cent quatre-vingt-dix jours\FTNT{Mt. 24:15 ; Mc. 13:14 ; Lu. 21:20.}.
\VS{12}Heureux celui qui attendra et qui parviendra jusqu'à mille trois cent trente-cinq jours.
\VS{13}Mais toi, marche vers ta fin ; néanmoins tu te reposeras, et tu seras debout pour ton héritage à la fin des jours.
\PPE{}
\end{multicols}

\clearpage\ShortTitle{Esdras}\BookTitle{Esdras}\BFont
\noindent\hrulefill
{\footnotesize
\textit{
\bigskip
{\centering{}
\\Auteur : Esdras
\\(Heb : Ezrah)
\\Signification : Secours
\\Thème : Edit de Cyrus et reconstruction du temple 
\\Date de rédaction : 5ème siècle av. J.-C.\\}
}
%\bigskip
\textit{
\\Conformément aux prophéties reçues par Esaïe et Jérémie, Yahweh toucha le cœur du roi Cyrus afin de renvoyer les fils d’Israël sur leur terre avec la mission de reconstruire le temple détruit quelques décennies auparavant. Ce livre montre comment Dieu ramena glorieusement son peuple à Jérusalem et retrace la reconstruction du temple ainsi que les épreuves ayant accompagné ce projet. Il traite des réformes sociales et religieuses mises en place dans le cadre d’un retour total à Yahweh.\bigskip
}
}
\par\nobreak\noindent\hrulefill
\begin{multicols}{2}
\Chap{1}
\TextTitle{Publication de Cyrus}
\VerseOne{}La première année de Cyrus\FTNT{538 av. J.-C.}, roi de Perse, afin que la parole de Yahweh prononcée par la bouche de Jérémie\FTNT{Jé. 25:12 ; 29:10 ; 33:7-10.} soit accomplie,  Yahweh réveilla l'esprit de Cyrus, roi de Perse, qui fit publier par écrit et de vive voix dans tout son royaume, en disant :
\VS{2}Ainsi parle Cyrus, roi de Perse : Yahweh, le Dieu des cieux, m'a donné tous les royaumes de la terre, et il m'a ordonné de lui bâtir une maison à Jérusalem, en Juda.
\VS{3}Qui d'entre vous est de son peuple, qui veut s'y employer ? Que son Dieu soit avec lui, qu'il monte à Jérusalem, en Juda, et qu'il rebâtisse la maison de Yahweh, le Dieu d'Israël ! C'est le Dieu qui est à Jérusalem.
\VS{4}Dans tout lieu où séjournent des restes du peuple,  les gens du lieu leur donneront de l’argent, de l’or, des biens, et du bétail, avec des offrandes volontaires pour la maison du Dieu qui est à Jérusalem.
\TextTitle{Cyrus rend les ustensiles}
\VS{5}Alors les chefs des familles de Juda et de Benjamin, les sacrificateurs et les Lévites, tous ceux dont Dieu réveilla l'esprit, se levèrent afin de monter pour rebâtir la maison de Yahweh à Jérusalem.
\VS{6}Tous ceux qui étaient autour d'eux les encouragèrent, leur fournissant des objets d'argent, d'or, des biens, du bétail, et des choses précieuses, outre toutes les offrandes volontaires.
\VS{7}Le roi Cyrus prit les ustensiles de la maison de Yahweh, que Nebucadnetsar avait emportés\FTNT{2 R. 24:13 ; 2 Ch 36:7.} de Jérusalem et mis dans la maison de son dieu.
\VS{8}Cyrus, roi de Perse, les fit sortir par Mithredath, le trésorier, qui les remit à Scheschbatsar, prince de Juda.
\VS{9}Et voici leur nombre : Trente bassins d'or, mille bassins d'argent, vingt-neuf couteaux,
\VS{10}trente coupes d'or, quatre cent dix coupes d'argent de second ordre, et d'autres ustensiles par milliers.
\VS{11}Tous les ustensiles d'or et d'argent étaient de cinq mille quatre cents. Scheschbatsar emporta le tout, lorsqu’on fit remonter de Babylone à Jérusalem ceux de la captivité.
\Chap{2}
\TextTitle{Dénombrement des Israélites revenus de captivité}
\VerseOne{}Voici ceux de la province qui revinrent de la captivité, d'entre ceux que Nebucadnetsar, roi de Babylone, avait transportés en exil à Babylone, et qui retournèrent à Jérusalem, et en Juda ; chacun dans sa ville\FTNT{Esd. 5:8 ; Né. 1:3 ; Né. 7:6.}.
\VS{2}Ils vinrent avec Zorobabel, Josué, Néhémie, Seraja, Reélaja, Mardochée, Bilschan, Mispar, Bigvaï, Rehum, Baana. Nombre des hommes du peuple d'Israël :
\VS{3}Les fils de Pareosch, deux mille cent soixante-douze\FTNT{Né. 7:8.} ;
\VS{4}les fils de Schephathia, trois cent soixante-douze ;
\VS{5}les fils d'Arach, sept cent soixante-quinze ;
\VS{6}les fils de Pachath-Moab, des fils de Josué, de Joab, deux mille huit cent douze ;
\VS{7}les fils d'Elam, mille deux cent cinquante-quatre ;
\VS{8}les fils de Zatthu, neuf cent quarante-cinq ;
\VS{9}les fils de Zaccaï, sept cent soixante ;
\VS{10}les fils de Bani, six cent quarante-deux\FTNT{Né. 7:15.} ;
\VS{11}les fils de Bébaï, six cent vingt-trois ;
\VS{12}les fils d'Hazgad, mille deux cent vingt-deux ;
\VS{13}les fils d'Adonikam, six cent soixante-six ;
\VS{14}les fils de Bigvaï, deux mille cinquante-six ;
\VS{15}les fils d’Adin, quatre cent cinquante-quatre ;
\VS{16}les fils d'Ather, de la famille d'Ezéchias, quatre-vingt-dix-huit ;
\VS{17}les fils de Betsaï, trois cent vingt-trois ;
\VS{18}les fils de Jora, cent douze ;
\VS{19}les fils de Haschum, deux cent vingt-trois ;
\VS{20}les fils de Guibbar, quatre-vingt-quinze ;
\VS{21}les fils de Bethléhem, cent vingt-trois ;
\VS{22}les gens de Netopha, cinquante-six ;
\VS{23}les gens d'Anathoth, cent vingt-huit ;
\VS{24}les fils d'Azmaveth, quarante-deux ;
\VS{25}les fils de Kirjath-Arim, de Kephira, et de Beéroth, sept cent quarante-trois ;
\VS{26}les fils de Rama et de Guéba, six cent vingt et un ;
\VS{27}les gens de Micmas, cent vingt-deux ;
\VS{28}les gens de Béthel et d’Aï, deux cent vingt-trois ;
\VS{29}les fils de Nebo, cinquante-deux ;
\VS{30}les fils de Magbisch, cent cinquante-six.
\VS{31}les fils d'un autre Elam, mille deux cent cinquante-quatre ;
\VS{32}les fils de Harim, trois cent vingt ;
\VS{33}les fils de Lod, de Hadid, et d'Ono, sept cent vingt-cinq ;
\VS{34}les fils de Jéricho, trois cent quarante-cinq ;
\VS{35}les fils de Senaa, trois mille six cent trente.
\TextTitle{Dénombrement des sacrificateurs revenus de captivité}
\VS{36}Des sacrificateurs : Les fils de Jedaeja, de la maison de Josué, neuf cent soixante-treize ;
\VS{37}les fils d'Immer, mille cinquante-deux ;
\VS{38}les fils de Paschhur, mille deux cent quarante-sept ;
\VS{39}les fils de Harim, mille dix-sept.
\TextTitle{Dénombrement des Lévites revenus de captivité}
\VS{40}Des Lévites : Les fils de Josué et de Kadmiel, d'entre les fils d’Hodavia, soixante-quatorze.
\VS{41}Des chantres : Les fils d'Asaph, cent vingt-huit.
\VS{42}Des fils des portiers : Les fils de Schallum, les fils d'Ather, les fils de Thalmon, les fils d’Akkub, les fils de Hathitha, les fils de Schobaï, en tout cent trente-neuf.
\VS{43}Des Néthiniens : Les fils de Tsicha, les fils de Hasupha, les fils de Tabbahoth\FTNT{Esd.8:17 ; Jos. 9:23.},
\VS{44}les fils de Kéros, les fils de Siaha, les fils de Padon,
\VS{45}les fils de Lebana, les fils de Hagaba, les fils d'Akkub,
\VS{46}les fils de Hagab, les fils de Schamlaï, les fils de Hanan,
\VS{47}les fils de Guiddel, les fils de Gachar, les fils de Reaja,
\VS{48}les fils de Retsin, les fils de Nekoda, les fils de Gazzam,
\VS{49}les fils d'Uzza, les fils de Paséach, les fils de Bésaï,
\VS{50}les fils d'Asna, les fils de Mehunim, les fils de Nephusim,
\VS{51}les fils de Bakbuk, les fils de Hakupha, les fils de Harhur,
\VS{52}les fils de Batsluth, les fils de Mehida, les fils de Harscha,
\VS{53}les fils de Barkos, les fils de Sisera, les fils de Thamach,
\VS{54}les fils de Netsiach, les fils de Hathipha.
\TextTitle{Dénombrement des serviteurs de Salomon revenus de captivité}
\VS{55}Des fils des serviteurs de Salomon : Les fils de Sothaï, les fils de Sophéreth, les fils de Peruda,
\VS{56}les fils de Jaala, les fils de Darkon, les fils de Guiddel,
\VS{57}les fils de Schephathia, les fils de Hatthil, les fils de Pokéreth-Hatsebaïm, les fils d'Ami.
\VS{58}Total des Néthiniens et des fils des serviteurs de Salomon : Trois cent quatre-vingt-douze.
\VS{59}Voici ceux qui montèrent de Thel-Mélach, de Thel-Harscha, de Kerub-Addan et qui ne purent pas faire connaître leur maison paternelle et leur race, pour prouver qu’ils étaient d'Israël :
\VS{60}Les fils de Delaja, les fils de Tobija, les fils de Nekoda, six cent cinquante-deux.
\TextTitle{Certains sacrificateurs rejetés de la sacrificature}
\VS{61}Des fils des sacrificateurs : Les fils de Habaja, les fils d'Hakkots, les fils de Barzillaï, qui avait pris pour femme une des filles de Barzillaï, le Galaadite, fut appelé de leur nom.
\VS{62}Ils cherchèrent leurs registres généalogiques, mais ils ne les trouvèrent point. C'est pourquoi ils furent rejetés pour ne pas souiller le sacerdoce,
\VS{63}et le gouverneur leur dit de ne pas manger des choses très saintes, en attendant qu'un sacrificateur ait consulté l'urim et le thummim.
\TextTitle{Nombre total des Israélites revenus de captivité}
\VS{64}L’assemblée tout entière était de quarante-deux mille trois cent soixante,
\VS{65}sans leurs serviteurs et leurs servantes, qui étaient sept mille trois cent trente-sept. Ils avaient deux cents chantres ou chanteuses.
\VS{66}Ils avaient sept cent trente-six chevaux, et deux cent quarante-cinq mulets,
\VS{67}quatre cent trente-cinq chameaux, et six mille sept cent vingt ânes.
\VS{68}Quelques-uns d'entre les chefs des pères, quand ils vinrent à la maison de Yahweh à Jérusalem, firent des offrandes volontaires pour la maison de Dieu, afin qu'on la rétablît sur son emplacement.
\VS{69}Ils donnèrent au trésor de l'ouvrage, selon leurs moyens, soixante et un mille drachmes d'or, et cinq mille mines d'argent, et cent tuniques de sacrificateurs.
\VS{70}Ainsi, les sacrificateurs, les Lévites, quelques-uns du peuple, les chantres, les portiers et les Néthiniens habitèrent dans leurs villes. Et tous ceux d'Israël dans leurs villes aussi.
\Chap{3}
\TextTitle{Rétablissement de l'autel et des sacrifices}
\VerseOne{}Le septième mois approcha, et les fils d'Israël étaient dans leurs villes. Le peuple s'assembla alors comme un seul homme à Jérusalem.
\VS{2}Alors\FTNT{Ag. 1:1 ; De.12:5-6.} Josué, fils de Jotsadak, avec ses frères les sacrificateurs, et Zorobabel, fils de Schealthiel, avec ses frères, se levèrent et bâtirent l'autel du Dieu d'Israël, pour y offrir des holocaustes, comme il est écrit dans la loi de Moïse, homme de Dieu.
\VS{3}Ils rétablirent l'autel de Dieu sur ses fondements, parce qu'ils avaient peur en eux-mêmes des peuples du pays, et ils y offrirent des holocaustes à Yahweh, les holocaustes du matin et du soir\FTNT{No. 28:3.}.
\VS{4}Ils célébrèrent aussi la fête des tabernacles, comme il est écrit, et ils offrirent des holocaustes, autant qu'il en fallait  chaque jour\FTNT{Lé. 23:34 ; No. 29:12.}.
\VS{5}Après cela, ils offrirent l'holocauste perpétuel, ceux des nouvelles lunes, de toutes les fêtes solennelles consacrées à Yahweh, et ceux de quiconque faisait des offrandes volontaires à Yahweh\FTNT{No. 28:11 ; Né.10:33.}.
\VS{6}Dès le premier jour du septième mois, ils commencèrent à offrir des holocaustes à Yahweh. Cependant, les fondements du temple de Yahweh n'étaient pas encore posés.
\VS{7}Ils donnèrent de l'argent aux tailleurs de pierres et aux charpentiers, et aussi de la nourriture, des boissons, de l'huile aux Sidoniens et aux Tyriens, afin qu'ils amènent du bois de cèdre du Liban par la mer de Japho, selon la permission que Cyrus, roi de Perse, leur en avait donnée.
\TextTitle{Les fondements du temple posés}
\VS{8}Et la deuxième année depuis leur arrivée à la maison de Dieu à Jérusalem, au deuxième mois, Zorobabel, fils de Schealthiel,  Josué, fils de Jotsadak, et le reste de leurs frères les sacrificateurs et les Lévites, et tous ceux qui étaient revenus de la captivité à Jérusalem, débutèrent l’œuvre et désignèrent des Lévites, depuis l'âge de vingt ans et au-dessus pour surveiller l'ouvrage de la maison de Yahweh.
\VS{9}Et Josué, avec ses fils et ses frères, Kadmiel, avec ses fils, fils de Juda, les fils de Hénadad, avec leurs fils et leurs frères les Lévites, se tenaient debout pour surveiller ceux qui faisaient l'ouvrage de la maison de Dieu.
\VS{10}Et lorsque ceux qui bâtissaient posèrent les fondements du temple de Yahweh, on fit assister les sacrificateurs revêtus de leurs habits, avec leurs trompettes, et les Lévites, fils d'Asaph, avec les cymbales, pour qu’ils célèbrent Yahweh, selon l’institution de David, roi d'Israël.
\VS{11}Et en louant et célébrant Yahweh, ils s'entre-répondaient : Il est bon, parce que sa miséricorde demeure à toujours sur Israël ! Et tout le peuple poussait de grands cris de joie en louant Yahweh, parce qu'on posait les fondements de la maison de Yahweh.
\VS{12}Mais plusieurs des sacrificateurs et des Lévites, et des chefs de familles âgés, qui avaient vu la première maison, pleuraient à grand bruit pendant qu'on posait sous leurs yeux les fondements de cette maison. Et beaucoup élevaient leur voix avec des cris de joie,
\VS{13}et le peuple ne pouvait distinguer le bruit des cris de joie  d'avec le bruit des pleurs du peuple, car le peuple poussait de grands cris de joie dont le son s’entendait de très loin.
\Chap{4}
\TextTitle{Les ennemis de Juda et de Benjamin découragent le peuple de Juda}
\VerseOne{}Les ennemis de Juda et de Benjamin entendirent que les fils de la captivité rebâtissaient un temple à Yahweh, le Dieu d'Israël.
\VS{2}Ils vinrent vers Zorobabel et vers les chefs des familles, et leur dirent : Nous bâtirons\FTNT{On ne doit jamais s’associer avec les impies pour bâtir l’œuvre du Seigneur. Satan essaie toujours de s’infiltrer dans les assemblées afin de nous éloigner de la vérité, c’est pour cela que nous devons faire preuve de discernement (2 Co. 6:14-16).} avec vous ; car nous invoquons votre Dieu comme vous ; et nous lui avons sacrifié depuis le temps d'Esar-Haddon, roi d'Assyrie, qui nous a fait monter ici.
\VS{3}Mais Zorobabel, Josué, et les autres chefs des familles d'Israël, leur répondirent : Il ne convient pas à vous de bâtir la maison de notre Dieu ; mais nous, qui sommes ici ensemble, nous la bâtirons à Yahweh, le Dieu d'Israël, comme nous l'a ordonné le roi Cyrus, roi de Perse\FTNT{Esd. 1:1,2,5.}.
\VS{4}Alors les gens du pays rendirent paresseuses les mains du peuple de Juda ; ils l’intimidèrent pour l'empêcher de bâtir,
\VS{5}ils avaient même engagé à prix d’argent des conseillers pour faire échouer leur projet, pendant toute la durée de vie de Cyrus, roi de Perse, jusqu'au règne de Darius, roi de Perse.
\VS{6}Et sous le règne d'Assuérus, au commencement de son règne, ils écrivirent une accusation contre les habitants de Juda et de Jérusalem.
\TextTitle{Lettre envoyé à Artaxerxès}
\VS{7}Et du temps d'Artaxerxès, Bischlam, Mithredath, Thabeel, et le reste de leurs collègues, écrivirent à Artaxerxès, roi de Perse. La lettre était écrite en caractères araméens, et elle était traduite en araméen.
\VS{8}Rehum, le gouverneur, et Schimschaï, le secrétaire, écrivirent au roi Artaxerxès la lettre suivante concernant Jérusalem :
\VS{9}Rehum, gouverneur, Schimschaï, secrétaire, et le reste de leurs collègues, ceux de Din, d'Arpharsathac, de Tharpel, d'Apharas, d'Erec, de Babylone, de Suse, de Déha, d'Elam,
\VS{10}et les autres peuples que le grand et illustre Osnappar a transportés et fait habiter dans la ville de Samarie et les autres régions au-delà du fleuve, à cette date.
\VS{11}Voici donc ici la copie de la lettre qu'ils envoyèrent au roi Artaxerxès : Tes serviteurs, les gens de ce côté du fleuve, à cette date.
\VS{12}Que le roi sache que les Juifs qui sont montés de chez lui et arrivés vers nous à Jérusalem rebâtissent la ville rebelle et méchante, et achèvent en finissant de poser et de réparer les fondements des murs.
\VS{13}Que le roi sache donc que si cette ville est rebâtie et si ses murs sont réparés, ils ne paieront plus de tribut, ni d’impôt, ni de droit de passage, et elle causera une grande nuisance aux revenus du roi.
\VS{14}Et parce que nous mangeons le sel du palais, il ne nous parait pas convenable de voir le roi déshonoré ; c'est pourquoi nous envoyons au roi ces informations.
\VS{15}Qu'on recherche dans le livre des mémoires de tes pères, et tu trouveras et tu apprendras dans ce livre des mémoires que cette ville est une ville rebelle, nuisible aux rois et aux provinces ; et qu’on s’y est livré à la révolte depuis toujours. Donc cette ville a été détruite à cause de cela.
\VS{16}Nous faisons donc savoir au roi que si cette ville est rebâtie et si ses murs sont relevés, il n'aura plus de possession de ce côté du fleuve.
\TextTitle{Réponse du roi Artaxerxès}
\VS{17}Et c'est ici le décret envoyé par le roi à Rehum, le gouverneur, à Schimschaï, le secrétaire, et au reste de leurs collègues demeurant à Samarie, et aux autres de l'autre côté du fleuve : Paix sur vous, à cette date.
\VS{18}La lettre que vous nous avez envoyée a été lue exactement en ma présence.
\VS{19}J'ai donné ordre de faire des recherches et l’on a trouvé  que depuis toujours cette ville s'est soulevée contre les rois, et qu'on s’y est livré à la sédition et à la révolte.
\VS{20}Il y eut aussi à Jérusalem des rois puissants, maîtres de tout le pays de l'autre côté du fleuve, et auxquels on payait  tribut, impôt et droit de passage\FTNT{2 S. 8:2,6 ; 1 R 4:21 ; 2 Ch. 17:11 ; 32:23.}.
\VS{21}A présent, donnez l’ordre de ne pas laisser continuer ces gens-là, afin que cette ville ne se rebâtisse point, jusqu’à ce que je l'ordonne par décret.
\VS{22}Gardez-vous de mettre en cela de la négligence, de peur que le mal  n’augmente au préjudice des rois.
\VS{23}Aussitôt que la copie de la lettre du roi Artaxerxès eut été lue en présence de Rehum, de Schimschaï,  le secrétaire, et de leurs collègues, ils allèrent en hâte à Jérusalem vers les Juifs, et ils les firent cesser leurs travaux avec violence et force.
\VS{24}Alors l’ouvrage de la maison de Dieu, à Jérusalem, cessa, et elle demeura dans cet état, jusqu'à la deuxième année du règne de Darius, roi de Perse\FTNT{Esd. 5:2}.
\Chap{5}
\TextTitle{Aggée et Zacharie prophétisent}
\VerseOne{}Aggée, le prophète, et Zacharie, fils d'Iddo, le prophète, prophétisèrent aux Juifs qui étaient en Juda et à Jérusalem, au nom du Dieu d'Israël, qui s’adressait à eux\FTNT{Ag. 1:4 ; Za. 1:1.}.
\VS{2}Alors Zorobabel, fils de Schealthiel, et Josué, fils de Jotsadak, se levèrent et commencèrent à rebâtir la maison de Dieu à Jérusalem. Et ils avaient avec eux les prophètes de Dieu qui les soutenaient\FTNT{Ag. 1:14 ; Esd. 6:14.}.
\VS{3}En ce temps-là, Thathnaï, gouverneur de ce côté du fleuve, et Schethar-Boznaï, et leurs collègues, vinrent à eux et leur parlèrent ainsi : Qui vous a donné l’ordre de rebâtir cette maison et de relever ces murs ?\FTNT{Esd. 5:9.}
\VS{4}Ils leur dirent alors : Quels sont les noms des hommes qui construisent cet édifice ?
\VS{5}Mais l’œil de Dieu était sur les anciens des Juifs. Et on ne les laissa continuer les travaux, pendant l’envoi d’un rapport à Darius, et jusqu'à la réception d’une lettre sur cet objet.
\TextTitle{Thathnaï, Schethar-Boznaï et leurs collègues d'Apharsac écrivent à Darius}
\VS{6}Copie de la lettre envoyée au roi Darius par Thathnaï, gouverneur de ce côté du fleuve, Schethar-Boznaï, et leurs collègues d'Apharsac, de l’autre côté du fleuve.
\VS{7}Ils lui envoyèrent un rapport ainsi écrit : Paix parfaite soit au roi Darius !
\VS{8}Que le roi sache que nous sommes allés dans la province de Juda, vers la maison du grand Dieu. Elle se bâtit avec des pierres de taille, et le bois se pose dans les murs ; ce travail se réalise complètement et prospère entre leurs mains\FTNT{Esd. 2:1.}.
\VS{9}Nous avons interrogé ces anciens, et nous leur avons parlé ainsi : Qui vous a donné l'autorisation de rebâtir cette maison et de finir ces murs ?\FTNT{Esd. 5:3.}
\VS{10}Nous leur avons aussi demandé leurs noms pour te les faire connaître, et nous avons mis par écrit les noms des hommes à leur tête.
\VS{11}Et ils nous ont répondu de cette manière, disant : Nous sommes les serviteurs du Dieu des cieux et de la terre, et nous rebâtissons la maison qui avait été bâtie il y a de nombreuses années ; un grand roi d'Israël l’avait bâtie et finie.
\VS{12}Mais après que nos pères eurent provoqué la colère du Dieu des cieux, il les livra entre les mains de Nebucadnetsar\FTNT{Voir 2 R. 24 et 25.}, roi de Babylone, Chaldéen, qui détruisit cette maison et qui emmena le peuple en exil à Babylone\FTNT{2 Ch. 36:7}.
\VS{13}Mais la première année de Cyrus, roi de Babylone, le roi Cyrus prit un décret pour rebâtir cette maison de Dieu\FTNT{Esd. 1:1-2.}.
\VS{14}Et même le roi Cyrus ôta du temple de Babylone les ustensiles d'or et d'argent de la maison de Dieu, que Nebucadnetsar avait sortis du temple qui était à Jérusalem et transportés dans le temple de Babylone, et il les fit remettre au nommé Scheschbatsar, qu’il établit gouverneur\FTNT{Esd. 1:8.},
\VS{15}et il lui dit : Prends ces ustensiles, et va les déposer dans le temple de Jérusalem ; et que la maison de Dieu soit rebâtie sur sa place.
\VS{16}Alors ce Scheschbatsar est venu, et il a posé les fondements de la maison de Dieu à Jérusalem ; et depuis ce temps-là jusqu'à présent, on la bâtit, et elle n'est point encore achevée.
\VS{17}Maintenant, s'il semble bon au roi, que l’on fasse des recherches dans la maison des trésors du roi à Babylone, pour voir s'il est vrai qu'il y a eu un ordre donné par Cyrus de rebâtir cette  maison de Dieu à Jérusalem. Puis, que le roi nous transmette sa volonté sur cet objet.
\Chap{6}
\TextTitle{Darius confirme l'édit de Cyrus}
\VerseOne{}Alors le roi Darius donna un ordre de faire des recherches dans la maison des livres où l'on déposait les  trésors à Babylone.
\VS{2}Et l’on trouva à Achmetha, dans un coffre, capitale de la province de Médie, un rouleau à l’intérieur duquel était écrit  le mémoire suivant :
\VS{3}La première année du roi Cyrus, le roi Cyrus prit un décret quant à la maison de Dieu à Jérusalem : Que cette maison soit rebâtie, afin d’être un lieu où l'on offre des sacrifices, et que ses fondements soient solides pour porter sa charge. La hauteur sera de soixante coudées, et la longueur de soixante coudées,
\VS{4}trois rangées de pierres de taille et une rangée de bois neuf.  La dépense sera payée par la maison du roi.
\VS{5}Aussi, les ustensiles d'or et d'argent de la maison de Dieu, que Nebucadnetsar avait enlevés du temple de Jérusalem et apportés à Babylone, seront remis et apportés dans le temple de Jérusalem, à leur place, et déposés dans la maison de Dieu.
\VS{6}Maintenant, Thathnaï, gouverneur de l'autre côté du fleuve, Schethar-Boznaï, et vos collègues d'Apharsac de l'autre côté du fleuve, tenez-vous loin de ce lieu.
\VS{7}Laissez le travail de cette maison de Dieu ; que le gouverneur des Juifs et les anciens des Juifs rebâtissent cette maison de Dieu à sa place.
\VS{8}En raison de ce décret pris, ce que vous aurez à exécuter, avec les anciens de ces Juifs pour rebâtir cette maison de Dieu : Sur les finances du roi provenant du tribut de l’autre côté du fleuve, les frais seront complètement payés à ces hommes, afin qu'il n'y ait pas d'interruption.
\VS{9}Et ce qui sera nécessaire pour les holocaustes du Dieu des cieux, veaux, béliers et agneaux, blé, sel, vin et huile, seront livrés, sur leur demande, aux sacrificateurs de Jérusalem, jour après jour, sans négligence,
\VS{10}afin qu'ils offrent des sacrifices de bonne odeur au Dieu des cieux et qu'ils prient pour la vie du roi et de ses fils.
\VS{11}Et voici l’ordre que je donne touchant quiconque changera cette parole : On arrachera de sa maison une pièce de bois, on la dressera, afin qu'il y soit exterminé, et l’on fera de sa maison un tas de déchets\FTNT{2 R. 10:27 ; Ez. 6:11 ; Da. 3:29.}.
\VS{12}Et que Dieu, qui fait résider en ce lieu son nom, renverse tout roi et tout peuple qui étendrait sa main pour changer et détruire cette maison de Dieu à Jérusalem ! Moi, Darius, j’ai donné cet ordre. Qu'il soit donc exécuté complètement.
\TextTitle{Achèvement et dédicace de la maison de Dieu}
\VS{13}Alors Thathnaï, gouverneur de l'autre côté du fleuve,  Schethar-Boznaï, et leurs collègues, firent exécuter ainsi complètement ce que le roi Darius leur envoya.
\VS{14}Et les anciens des Juifs bâtirent avec succès, selon les prophéties d'Aggée, le prophète, et de Zacharie, fils d’Iddo ; ils bâtirent et finirent, d'après l'ordre du Dieu d'Israël, et d'après l'ordre de Cyrus, de Darius, et d'Artaxerxès, roi de Perse.
\VS{15}Cette maison fut achevée le troisième jour du mois d'Adar, dans la sixième année du règne du roi Darius.
\VS{16}Les fils d'Israël, les sacrificateurs, les Lévites, et le reste des fils de la captivité, célébrèrent la dédicace de cette maison de Dieu avec joie.
\VS{17}Ils offrirent pour la dédicace de cette maison de Dieu, cent taureaux, deux cents béliers, quatre cents agneaux, et douze boucs comme victimes expiatoires pour tout Israël, selon le nombre des tribus d'Israël.
\VS{18}Ils établirent les sacrificateurs selon leurs classes et les Lévites selon leurs divisions, pour le service de Dieu à Jérusalem,  selon ce qui est écrit dans le livre de Moïse\FTNT{No. 3:6,32 ; No. 8:11}.
\TextTitle{Rétablissement de la Pâque}
\VS{19}Puis les fils de la captivité célébrèrent la Pâque le quatorzième jour du premier mois\FTNT{Lé. 23:5 ; No. 28:16 ; De. 16:2.}.
\VS{20}Les sacrificateurs et les Lévites s'étaient purifiés comme un seul homme, tous étaient purs ; c'est pourquoi ils immolèrent la Pâque pour tous les fils de la captivité, pour leurs frères les sacrificateurs, et pour eux-mêmes\FTNT{2 Ch. 30:15,17,21}.
\VS{21}Les fils d'Israël revenus de la captivité mangèrent la Pâque, avec tous ceux qui s'étaient séparés de l’impureté des nations du pays pour chercher Yahweh, le Dieu d'Israël.
\VS{22}Ils célébrèrent avec joie la fête des pains sans levain pendant sept jours, car Yahweh les avait réjouis en disposant  le cœur du roi d'Assyrie à fortifier leurs mains dans l’œuvre de la maison de Dieu, du Dieu d'Israël.
\Chap{7}
\TextTitle{Voyage d'Esdras jusqu'à Jérusalem}
\VerseOne{}Après ces choses, sous le règne d'Artaxerxès, roi de Perse, Esdras, fils de Seraja, fils d'Azaria, fils de Hilkija\FTNT{Esd. 6:14.},
\VS{2}fils de Schallum, fils de Tsadok, fils d'Achithub,
\VS{3}fils d'Amaria, fils d'Azaria, fils de Merajoth,
\VS{4}fils de Zerachja, fils d'Uzzi, fils de Bukki,
\VS{5}fils d'Abischua, fils de Phinées, fils d'Eléazar, fils d'Aaron, souverain sacrificateur.
\VS{6}Esdras monta de Babylone : C’était un scribe bien exercé dans la loi de Moïse, donnée par Yahweh, le Dieu d'Israël. Et comme la main de Yahweh, son Dieu, était sur lui, le roi lui accorda toute sa requête\FTNT{Vers. 9,28.}.
\VS{7}Des fils d'Israël, des sacrificateurs, des Lévites, des chantres, des portiers, et des Néthiniens, montèrent à Jérusalem, la septième année du roi Artaxerxès.
\VS{8}Il entra à Jérusalem le cinquième mois de la septième année du roi ;
\VS{9}il était parti de Babylone au premier jour du premier mois, et il entra à Jérusalem au premier jour du cinquième mois, selon que la main de son Dieu était bonne sur lui.
\VS{10}Car Esdras avait disposé son cœur à étudier la loi de Yahweh, à l’observer et à enseigner les lois et les ordonnances parmi le peuple d'Israël.
\TextTitle{Lettre d'Artaxerxès à Esdras}
\VS{11}Voici la copie de la lettre que le roi Artaxerxès donna à Esdras, sacrificateur et scribe, enseignant les paroles des commandements de Yahweh et ses ordonnances concernant Israël :
\VS{12}Artaxerxès, roi des rois, à Esdras, sacrificateur et scribe de la loi du Dieu des cieux, à cette date.
\VS{13}J’ai donné ordre de laisser aller tous ceux de mon royaume qui sont du peuple d'Israël, de ses sacrificateurs et Lévites, qui se présenteront volontairement pour aller avec toi à Jérusalem.
\VS{14}Tu es envoyé de la part du roi, et de ses sept conseillers, pour inspecter Juda et Jérusalem touchant la loi de ton Dieu, laquelle est entre tes mains,
\VS{15}et pour porter l'argent et l'or que le roi et ses conseillers ont offert volontairement au Dieu d'Israël, dont la demeure est à Jérusalem\FTNT{Esd. 8:24.},
\VS{16}tout l'argent et l'or que tu trouveras dans toute la province de Babylone, avec les offrandes volontaires du peuple et des sacrificateurs, qu'ils feront volontairement à la maison de leur Dieu à Jérusalem.
\VS{17} C'est pourquoi tu achèteras avec cet argent des taureaux, des béliers, des agneaux, avec leurs offrandes et leurs libations, et tu les offriras sur l'autel de la maison de votre Dieu à Jérusalem.
\VS{18}Vous ferez, selon la volonté de votre Dieu, ce qu'il te semblera bon à toi et à tes frères de faire du reste de l'argent et de l'or.
\VS{19}Et pour ce qui est des ustensiles qui te sont remis pour le service de la maison de ton Dieu, déposes-les en présence du Dieu de Jérusalem.
\VS{20}Quand au reste de ce qui sera nécessaire pour la maison de ton Dieu, autant qu'il t'en faudra employer, tu le prendras de la maison des trésors du roi.
\VS{21}Moi, le roi Artaxerxès, je donne l’ordre à tous les trésoriers qui sont de l'autre côté du fleuve de livrer exactement à Esdras, sacrificateur et scribe de la loi du Dieu des cieux, tout ce qu’il vous demandera,
\VS{22}jusqu'à cent talents d'argent, cent cors de froment, cent baths de vin, cent baths d'huile, et du sel sans nombre.
\VS{23}Que tout ce qui est ordonné par le Dieu des cieux se fasse exactement pour la maison du Dieu des cieux, afin que sa colère ne soit pas sur le royaume, sur le roi et sur ses fils.
\VS{24}Nous vous faisons savoir qu'on ne pourra imposer ni tribut, ni impôt, ni droit de passage sur aucun des sacrificateurs, des Lévites, des chantres, des portiers, des  Néthiniens, et des serviteurs de cette maison de Dieu.
\VS{25}Et toi, Esdras, établis des magistrats et des juges selon la sagesse de ton Dieu que tu possèdes, afin qu'ils rendent justice à tout ce peuple de l'autre côté du fleuve, à tous ceux qui connaissent les lois de ton Dieu ; afin que vous enseigniez celui qui ne les connaît point.
\VS{26}Et tous ceux qui n'observeront point la loi de ton Dieu et la loi du roi seront aussitôt jugés, soit à la mort, soit au bannissement, soit à une amende pécuniaire, ou à l'emprisonnement.
\VS{27}Béni soit Yahweh, le Dieu de nos pères, qui a mis cela au cœur du roi, pour honorer la maison de Yahweh, qui est à Jérusalem ;
\VS{28}et qui a fait que j'ai trouvé grâce devant le  roi, devant ses conseillers, et devant tous les puissants chefs ! Fortifié par la main de Yahweh, mon Dieu, qui était sur moi, j'ai rassemblé les chefs d'Israël, afin qu'ils montent avec moi.
\Chap{8}
\TextTitle{Dénombrement de ceux qui montèrent avec Esdras}
\VerseOne{}Voici les chefs des pères, avec le dénombrement fait selon les généalogies de ceux qui montèrent avec moi de Babylone, pendant le règne du roi Artaxerxès\FTNT{1 Ch. 4:33.}.
\VS{2}Des fils de Phinées, Guerschom ; des fils d'Ithamar, Daniel ; des fils de David, Hattusch ;
\VS{3}des fils de Schecania ; des fils de Pareosch, Zacharie, et avec lui, en faisant le dénombrement par leur généalogie selon les hommes, cent cinquante hommes ;
\VS{4}des fils de Pachat Moab, Eljoénaï, fils de Zerachja, et avec lui deux cents hommes;
\VS{5}des fils de Schecania, le fils de Jachaziel, et avec lui trois cents hommes;
\VS{6}des fils d'Adin, Ebed, fils de Jonathan, et avec lui cinquante hommes ;
\VS{7}des fils d'Elam, Esaïe, fils d'Athalia, et avec lui soixante-dix hommes;
\VS{8}des fils de Schephathia, Zebadia, fils de Micaël, et avec lui quatre-vingts hommes ;
\VS{9}des fils de Joab, Abdias, fils de Jehiel, et avec lui deux cent dix-huit hommes ;
\VS{10}des fils de Schelomith, le fils de Josiphia, et avec lui cent soixante hommes ;
\VS{11}des fils de Bébaï, Zacharie, fils de Bébaï, et avec lui vingt-huit hommes ;
\VS{12}des fils d'Azgad, Jochanan, fils d'Hakkathan, et avec lui cent-dix hommes ;
\VS{13}des fils d'Adonikam, les derniers, dont voici les noms: Eliphélet, Jeïel, et Schemaeja, et avec eux soixante hommes ;
\VS{14}des fils de Bigvaï, Uthaï, Zabbud, et avec eux soixante-dix hommes.
\VS{15}Je les rassemblai près du fleuve qui coule vers Ahava, et nous campâmes là trois jours. Puis je portai mon attention sur  le peuple et les sacrificateurs, et je n'y trouvai aucun des fils de Lévi.
\VS{16}Alors j'envoyai d'entre les chefs Eliézer, Ariel, Schemaeja, Elnathan, Jarib, Elnathan, Nathan, Zacharie et Meschullam, avec les docteurs Jojarib et Elnathan.
\VS{17}Je leur donnai des ordres pour le chef Iddo, demeurant à Casiphia, et je mis dans leur bouche les paroles qu'ils devaient dire à Iddo et à ses frères les Néthiniens, qui étaient à Casiphia, afin qu'ils nous amènent des serviteurs pour la maison de notre Dieu\FTNT{Esd. 2:43.}.
\VS{18}Et comme la bonne main de notre Dieu était sur nous, ils nous amenèrent Schérébia, un homme intelligent, d'entre les fils de Machli, fils de Lévi, fils d'Israël, et avec ses fils et ses frères, au nombre dix-huit\FTNT{Esd. 7:6,9,28.} ;
\VS{19}Haschabia, et avec lui Esaïe, d'entre les fils de Merari, ses frères, et leurs fils, au nombre vingt ;
\VS{20}et des Néthiniens, que David et les chefs du peuple avaient assignés pour le service des Lévites, deux cent vingt Néthiniens, tous désignés par leurs noms\FTNT{Esd. 2:43,58.}.
\TextTitle{Esdras publie un jeûne pour obtenir la protection de Dieu}
\VS{21}Et je publiai là un jeûne près de la rivière d'Ahava, afin de nous humilier devant notre Dieu, le priant de nous donner un heureux voyage, pour nos enfants, et pour tous nos biens.
\VS{22}Car j'aurais eu honte de demander au roi une armée et des cavaliers pour nous soutenir contre des ennemis pendant le chemin ; car nous avions dit au roi : La main de notre Dieu est favorable sur tous ceux qui le cherchent ; mais sa force et sa colère sont contre ceux qui l'abandonnent.
\VS{23}Nous jeûnâmes donc, et nous cherchâmes notre Dieu à cause de cela. Et il se laissa fléchir par nos prières.
\TextTitle{Trésors remis par Esdras entre les mains de douze sacrificateurs}
\VS{24}Alors je mis à part douze chefs des sacrificateurs, Schérébia, Haschabia, et dix de leurs frères.
\VS{25}Je pesai l'argent, l'or et les ustensiles donnés en offrandes pour la maison de notre Dieu par le roi, ses conseillers, ses chefs, et tous ceux d'Israël qu'on avait trouvés\FTNT{Esd. 7:14,15.}.
\VS{26}Je pesai donc, et je remis entre leurs mains six cent cinquante talents d'argent, des ustensiles d'argent pesant cent talents, cent talents d'or,
\VS{27}vingt coupes d'or valant mille drachmes, et deux ustensiles d’un bel airain poli, aussi précieux que de l'or.
\VS{28}Et je leur dis : Vous êtes consacrés à Yahweh ; et les ustensiles sont sanctifiés, et cet argent et cet or sont une offrande volontaire faite à Yahweh, le Dieu de vos pères.
\VS{29}Soyez vigilants et gardez-les, jusqu'à ce que vous les pesiez devant les chefs des sacrificateurs et les Lévites, et devant les chefs des pères d'Israël, à Jérusalem, dans les chambres de la maison de Yahweh.
\VS{30}Les sacrificateurs et les Lévites reçurent le poids de l'argent, de l'or, et des ustensiles, pour les apporter à Jérusalem, dans la maison de notre Dieu.
\TextTitle{Esdras arrive à Jérusalem}
\VS{31}Nous partîmes du fleuve d'Ahava pour aller à Jérusalem, le douzième jour du premier mois. La main de notre Dieu fut sur nous et nous délivra de la main des ennemis et des  embûches sur le chemin.
\VS{32}Puis nous arrivâmes à Jérusalem, et nous nous y reposâmes trois jours.
\VS{33}Le quatrième jour, nous pesâmes l'argent, l'or, et les ustensiles dans la maison de notre Dieu, et nous les remîmes à Merémoth, fils d'Urie, le sacrificateur - il était avec Eléazar, fils de Phinées, et avec eux les Lévites Jozabad, fils de Josué, et Noadia, fils de Binnuï-
\VS{34} selon tout le nombre et le poids de toutes ces choses, et tout le poids fut mis alors par écrit.
\VS{35}Et les fils de la captivité revenus de l’exil offrirent en holocauste au Dieu d'Israël douze taureaux, quatre-vingt-seize béliers, soixante-dix-sept agneaux, et douze boucs comme victimes expiatoires pour tout Israël, le tout en holocauste à Yahweh.
\VS{36}Ils transmirent les ordres du roi entre les mains des satrapes du roi et des gouverneurs qui étaient de ce côté du fleuve, lesquels favorisèrent le peuple et la maison de Dieu.
\Chap{9}
\TextTitle{La désobéissance}
\VerseOne{}Après que ces choses furent terminées, les chefs du peuple s'approchèrent de moi, en disant : Le peuple d'Israël,  les sacrificateurs et les Lévites ne se sont point séparés des peuples de ces pays, quant à leurs abominations, celles des Cananéens, des Héthiens, des Phéréziens, des Jébusiens, des Ammonites, des Moabites, des Egyptiens, et des Amoréens.
\VS{2}Car ils ont pris de leurs filles pour eux et pour leurs fils, et ont mêlé la semence sainte avec les peuples de ces pays ; et des chefs et des magistrats ont été les premiers à commettre ce péché\FTNT{Né. 13:3.}.
\VS{3}Lorsque j'entendis cela, je déchirai mes vêtements et mon manteau, j'arrachai les cheveux de ma tête et ma barbe, et je m'assis tout épouvanté.
\VS{4}Et tous ceux qui tremblaient aux paroles du Dieu d'Israël, s'assemblèrent auprès de moi, à cause de l’infidélité de ceux de la captivité ; et je demeurai assis tout épouvanté jusqu'à l'offrande du soir.
\TextTitle{Prière et confession d'Esdras}
\VS{5}Et au temps de l'offrande du soir, je me levai du sein de mon affliction, et ayant mes vêtements et mon manteau déchirés, je me mis à genoux, et j'étendis mes mains vers Yahweh, mon Dieu,
\VS{6}et je dis : Mon Dieu ! J’ai honte, et je suis trop confus, ô mon Dieu, pour lever ma face vers toi ; car nos iniquités se sont multipliées au-dessus de nos têtes, et notre péché s'est élevé jusqu’aux cieux.
\VS{7}Depuis les jours de nos pères jusqu'à ce jour, nous sommes grandement coupables, et c’est à cause de nos iniquités que nous avons été livrés, nous, nos rois et nos sacrificateurs entre les mains des rois des pays, à l'épée, à la captivité, au pillage, et à la honte, comme il paraît aujourd'hui.
\VS{8}Et cependant Yahweh, notre Dieu, nous a maintenant fait grâce, en épargnant un reste, et il nous a donné un clou dans son saint lieu, afin d'éclaircir nos yeux et nous donner un peu de répit dans notre servitude\FTNT{Es. 22:23.}.
\VS{9}Car nous sommes esclaves, mais notre Dieu ne nous a point abandonnés dans notre servitude. Il a incliné la bienveillance des rois de Perse pour nous accorder de préserver nos vies afin que nous puissions relever la maison de notre Dieu, et rétablir ces lieux en ruines, et pour nous donner une clôture en Juda et à Jérusalem.
\VS{10}Mais maintenant, ô notre Dieu ! Que dirons-nous après ces choses ? Car nous avons abandonné tes commandements,
\VS{11}que tu as ordonnés par tes serviteurs les prophètes, en disant : Le pays dans lequel vous entrez pour le posséder est un pays souillé par les impuretés des peuples de ces pays, à cause des abominations dont ils l'ont rempli d’un bout à l'autre par leurs impuretés\FTNT{Lé. 18:25-27.};
\VS{12}maintenant donc, ne donnez point vos filles à leurs fils, et ne prenez point leurs filles pour vos fils, ne cherchez jamais ni leur bonheur, ni leur paix, ainsi vous deviendrez forts,  vous mangerez les meilleurs productions du pays, et vous le laisserez hériter à vos fils pour toujours\FTNT{De. 7:3.}.
\VS{13}Après toutes les choses qui nous sont arrivées à cause de nos mauvaises actions et des grandes offenses que nous avons commises - quoi que tu ne nous aies pas, ô notre Dieu, punis en proportion de nos péchés et maintenant que  tu nous as conservé ces réchappés ; 
\VS{14}retournerions-nous à violer tes commandements, et à faire alliance avec ces peuples abominables ? Ne serais-tu pas en colère contre nous, jusqu'à nous exterminer, sans aucun reste ni aucun réchappé ?
\VS{15}Yahweh, Dieu d'Israël ! Tu es juste, car nous sommes aujourd'hui un reste de réchappés. Voici, nous sommes devant toi avec nos fautes, ne pouvant subsister à cause d’elles devant ta face.
\Chap{10}
\TextTitle{Confession et séparation}
\VerseOne{}Pendant qu’Esdras priait et faisait cette confession, pleurant et étant prosterné à terre devant la maison de Dieu, une grande multitude d'hommes, de femmes, et d’enfants d'Israël, s'assembla auprès de lui ; et le peuple se lamenta abondamment par des pleurs.
\VS{2}Alors Schecania, fils de Jehiel, d'entre les fils d’Elam, prit la parole, et dit à Esdras : Nous avons péché contre notre Dieu, en nous mariant avec des femmes étrangères d'entre les peuples de ce pays. Mais Israël ne reste pas pour cela sans espérance\FTNT{De. 7:22,23.}.
\VS{3}Faisons maintenant une alliance avec notre Dieu pour le renvoi de toutes ces femmes et de leurs enfants, selon le conseil de mon seigneur et de ceux qui tremblent devant les commandements de notre Dieu. Et qu'il en soit fait selon la loi\FTNT{Esd. 9:4 ; Mal. 3:16.}.
\VS{4}Lève-toi, car cette affaire te regarde. Nous serons avec toi. Prends donc courage et agis.
\VS{5}Esdras se leva, et il fit jurer aux chefs des sacrificateurs, des Lévites, et de tout Israël, de faire selon cette parole. Et ils le jurèrent.
\VS{6}Puis Esdras se retira de devant la maison de Dieu, et s'en alla dans la chambre de Jochanan, fils d'Eliaschib ; et quand il y fut entré, il ne mangea point de pain, ne but point d'eau, parce qu'il se lamentait à cause du péché de ceux de la captivité.
\VS{7}Alors on publia dans le pays de Juda et à Jérusalem que tous ceux qui étaient retournés de la captivité aient à s'assembler à Jérusalem,
\VS{8}et que quiconque ne s'y rendrait pas dans trois jours, selon l'avis des chefs et des anciens, aurait tous ses biens complètement détruits, et que lui-même serait séparé de l'assemblée de ceux de la captivité.
\VS{9}Ainsi tous ceux de Juda et de Benjamin s'assemblèrent à Jérusalem dans les trois jours. C’était le vingtième jour du neuvième mois. Tout le peuple se tenait sur la place de la maison de Dieu, tremblant au sujet de cette affaire et à cause des pluies\FTNT{1 S. 12:18.}.
\VS{10}Esdras, le sacrificateur, se leva et leur dit : Vous avez péché en vous mariant avec des femmes étrangères, de sorte que vous avez augmenté la culpabilité d'Israël\FTNT{De. 7:3.}.
\VS{11}Prononcez maintenant votre confession à Yahweh, le Dieu de vos pères, et faites sa volonté ! Séparez-vous des peuples du pays et des femmes étrangères.
\VS{12}Et toute l'assemblée répondit à haute voix : A nous de faire ce que tu as dit !
\VS{13}Mais le peuple est nombreux, le temps est pluvieux, et il n'y a pas moyen de se tenir dehors ; d’ailleurs, ce n’est pas l’affaire d’un jour ou de deux, car il y en a beaucoup parmi nous qui ont péché dans cette affaire.
\VS{14}Que tous nos chefs se présentent donc devant toute l'assemblée, et que tous ceux qui sont dans nos villes, et qui se sont mariés avec des femmes étrangères, viennent à un temps fixé, et que les anciens de chaque ville et ses juges soient avec eux, jusqu'à ce que nous détournions de nous l'ardente colère de notre Dieu à ce sujet.
\VS{15}Il n'y eut que Jonathan, fils d'Asaël, et Jachzia, fils de Thikva, qui s'opposèrent a cet avis ; et Meschullam et Schabthaï, Lévites, les appuyèrent ;
\VS{16}mais ceux qui étaient retournés de la captivité s’y conformèrent. On choisit Esdras, le sacrificateur, et des chefs de famille selon leurs maisons paternelles, tous désignés par leurs noms ; ils siégèrent le premier jour du dixième mois, pour suivre cette affaire.
\VS{17}Le premier jour du premier mois, ils en finirent avec tous les hommes qui s’étaient mariés à des femmes étrangères.
\VS{18}Parmi les fils des sacrificateurs qui s’étaient mariés à des femmes étrangères, il se trouva d'entre les fils de Josué, fils de Jotsadak et de ses frères, Maaséja, Eliézer, Jarib et Guedalia,
\VS{19}qui, en donnant leurs mains, renvoyèrent leurs femmes ; et offrirent un bélier comme sacrifice de culpabilité ;
\VS{20}des fils d'Immer, Hanani et Zebadia ;
\VS{21}des fils de Harim, Maaséja, Elie, Schemaeja, Jehiel et Ozias ;
\VS{22}des fils de Paschhur, Eljoénaï, Maaséja, Ismaël, Nethaneel, Jozabad et Eleasa.
\VS{23}Parmi les Lévites : Jozabad, Schimeï, Kélaja (ou Kelitha) Pethachja, Juda et Eliézer.
\VS{24}Parmi les chantres : Eliaschib. Et des portiers : Schallum, Thélem et Uri.
\VS{25}Parmi ceux d'Israël : Des fils de Pareosch, Ramia, Jizzija, Malkija, Mijamin, Eléazar, Malkija et Benaja ;
\VS{26}des fils d’Elam, Matthania, Zacharie, Jehiel, Abdi, Jérémoth et Elie ;
\VS{27}des fils de Zatthu, Eljoénaï, Eliaschib, Matthania, Jérémoth, Zabad et Aziza ;
\VS{28}des fils de Bébaï, Jochanan, Hanania, Zabbaï et Athlaï ;
\VS{29}des fils de Bani, Meschullam, Malluc, Adaja, Jaschub, Scheal et Ramoth ;
\VS{30}des fils de Pachath-Moab, Adna, Kelal, Benaja, Maaséja, Matthania, Betsaleel, Binnuï et Manassé ;
\VS{31}des fils de Harim, Eliézer, Jischija, Malkija, Schemaeja, Siméon,
\VS{32}Benjamin, Malluc et Schemaria ;
\VS{33}des fils de Haschum, Matthnaï, Matthattha, Zabad, Eliphéleth, Jerémaï, Manassé et Schimeï ;
\VS{34}des fils de Bani, Maadaï, Amram, Uel,
\VS{35}Benaja, Bédia, Keluhu,
\VS{36}Vania, Merémoth, Eliaschib,
\VS{37}Matthania, Matthnaï, Jaasaï,
\VS{38}Bani, Binnuï, Schimeï,
\VS{39}Schélémia, Nathan, Adaja,
\VS{40}Macnadbaï, Schaschaï, Scharaï,
\VS{41}Azareel, Schélémia, Schemaria,
\VS{42}Schallum, Amaria et Joseph ;
\VS{43}des fils de Nebo, Jeïel, Matthithia, Zabad, Zebina, Jaddaï, Joël et Benaja.
\VS{44}Tous ceux-là avaient pris des femmes étrangères ; et  quelques-uns avaient eu des fils avec ces femmes-là.
\PPE{}
\end{multicols}

\clearpage\ShortTitle{Néhémie}\BookTitle{Néhémie}\BFont
\noindent\hrulefill
{\footnotesize
\textit{
\bigskip
{\centering{}
\\Auteur : Néhémie
\\(Heb. : Nechemyah)
\\Signification : Yahweh a consolé
\\Thème : Reconstruction des murailles de Jérusalem
\\Date de rédaction : 5\up{ème} siècle av. J.-C.\\}
}
%\bigskip
\textit{
\\En apprenant l'état de ruine dans lequel se trouvait Jérusalem, Néhémie, échanson du roi perse Artaxerxés Ier, fut profondément affecté. Après plusieurs jours dans la désolation et l'humiliation, le Seigneur toucha le cœur du roi qui lui donna l'autorisation et le matériel nécessaire pour rebâtir la muraille de Jérusalem. Malgré les nombreuses oppositions dont il fit l'objet au cours de son entreprise, Néhémie acheva l'œuvre qui lui avait été confiée. Dans le même temps, il mit en place de profondes réformes dans le cadre du retour à la loi de Yahweh.
%\bigskip
\\Complément du livre d'Esdras avec lequel il ne formait initialement qu'un ouvrage, le livre de Néhémie présente un homme de prière, un serviteur œuvrant pour, avec, et au Nom de Yahweh.\bigskip
}
}
\par\nobreak\noindent\hrulefill
\begin{multicols}{2}
\Chap{1}
\TextTitle{La détresse du peuple resté à Jérusalem est racontée à Néhémie}
\VerseOne{}Paroles de Néhémie, fils de Hacalia. Il arriva au mois de Kisleu, la vingtième année, comme j'étais à Suse, la capitale,
\VS{2}Hanani, l'un de mes frères et quelques hommes arrivèrent de Juda. Je les questionnai au sujet des Juifs réchappés qui étaient restés de la captivité et au sujet de Jérusalem.
\VS{3}Et ils me dirent : Ceux qui sont restés de la captivité sont là dans la province, dans une grande misère et dans l'opprobre ; et la muraille de Jérusalem demeure renversée et ses portes ont été consumées par le feu.
\TextTitle{Néhémie prie Yahweh et implore sa grâce}
\VS{4}Or il arriva que, dès que j'entendis ces paroles, je m'assis, je pleurai et je fus dans le deuil plusieurs jours. Je jeûnai et je priai devant le Dieu des cieux,
\VS{5} et je dis : Je te prie, ô Yahweh ! Dieu des cieux, Dieu grand et redoutable, qui garde l'alliance et la miséricorde de ceux qui t'aiment et qui observent tes commandements !
\VS{6}Je te prie que ton oreille soit attentive et que tes yeux soient ouverts pour entendre la prière que ton serviteur te présente en ce temps-ci, jour et nuit, pour tes serviteurs les enfants d'Israël, en confessant les péchés des enfants d'Israël, que nous avons commis contre toi ; même moi et la maison de mon père, nous avons péché.
\VS{7}Certainement nous sommes coupables devant toi, nous n'avons pas gardé les commandements, les lois et les ordonnances que tu prescrivis à Moïse, ton serviteur.
\VS{8}Mais, je te prie, souviens-toi de la parole que tu chargeas Moïse, ton serviteur, de dire : Vous pécherez et je vous disperserai parmi les peuples\FTNT{De. 28:63-67.} ;
\VS{9}mais si vous revenez à moi, et si vous gardez mes commandements et les observez ; et s'il y en a d'entre vous qui ont été chassés jusqu'à l'extrémité du ciel, je vous rassemblerai de là, et je vous ramènerai au lieu que j'aurai choisi pour y faire habiter mon Nom\FTNT{De. 30:1-10.}.
\VS{10}Ils sont tes serviteurs et ton peuple, que tu as rachetés par ta grande puissance et par ta main forte.
\VS{11}Je te prie donc, Seigneur, que ton oreille soit maintenant attentive à la prière de ton serviteur, et à la prière de tes serviteurs qui prennent plaisir à craindre ton Nom ! Je te prie, donne aujourd'hui du succès à ton serviteur, et fais-lui trouver grâce devant cet homme ! J'étais alors échanson du roi.
\Chap{2}
\TextTitle{Yahweh exauce Néhémie et lui donne la faveur du roi}
\VerseOne{}Et il arriva, au mois de Nisan, la vingtième année du roi Artaxerxès, comme le vin était devant lui, je pris le vin et le présentai au roi. Je n'avais jamais été triste devant lui\FTNT{Pr. 15:13.}.
\VS{2}Et le roi me dit : Pourquoi as-tu mauvais visage, puisque tu n'es point malade ? Cela ne peut être qu'une tristesse de cœur. Je fus alors saisi d'une grande crainte,
\VS{3}et je répondis au roi : Que le roi vive éternellement ! Comment n'aurais-je pas mauvais visage, puisque la ville où sont les sépulcres de mes pères demeure désolée et que ses portes ont été consumées par le feu ?
\VS{4}Et le roi dit : Que me demandes-tu ? Alors je priai le Dieu des cieux,
\VS{5}et je dis au roi : Si le roi le trouve bon, et si ton serviteur lui est agréable, envoie-moi en Juda, vers la ville des sépulcres de mes pères, pour la rebâtir\FTNT{La reconstruction de la ville de Jérusalem sous Néhémie date, selon certains, de l'an 445 av. J.-C., suite au décret d'Artaxerxès. Cette date marquerait le point de départ des soixante-dix semaines d'années annoncées par Daniel (Da. 9:24-27).}.
\VS{6}Le roi me dit, et sa femme aussi qui était assise auprès de lui : Combien ton voyage durera-t-il, et quand seras-tu de retour ? Je lui précisai le temps, et le roi trouva bon de m'envoyer.
\VS{7}Puis je dis au roi : Si le roi le trouve bon, qu'on me donne des lettres pour les gouverneurs de l'autre côté du fleuve, afin qu'ils me laissent passer, jusqu'à ce que j'arrive en Juda ;
\VS{8}et des lettres pour Asaph, le garde de la forêt du roi, afin qu'il me donne du bois pour la charpente des portes de la forteresse près de la maison, pour les murailles de la ville, et pour la maison dans laquelle j'entrerai. Et le roi me l'accorda, car la main de mon Dieu était bonne sur moi.
\TextTitle{Arrivée à Jérusalem, constat des murailles en ruines}
\VS{9}J'allai donc vers les gouverneurs qui sont de l'autre côté du fleuve et je leur donnai les lettres du roi. Le roi avait aussi envoyé avec moi des chefs de l'armée et des cavaliers.
\VS{10}Quand Sanballat, le Horonite, et Tobija, le serviteur Ammonite, l'ayant appris, ils eurent un très grand déplaisir de ce qu'il venait un homme pour procurer du bien aux enfants d'Israël.
\VS{11}Ainsi j'arrivai à Jérusalem et j'y passai trois jours.
\VS{12}Puis je me levai de nuit, avec quelques hommes ; mais je ne dis à personne ce que Dieu avait mis dans mon cœur de faire pour Jérusalem. Il n'y avait point d'autre bête avec moi que celle sur laquelle j'étais monté.
\VS{13}Je sortis donc de nuit par la porte de la vallée et me dirigeai vers la source du dragon, vers la porte du fumier ; et je considérai les murailles de Jérusalem qui étaient en ruines\FTNT{Jé. 39:8.}, et ses portes consumées par le feu.
\VS{14}Je passai près de la porte de la source et vers l'étang du roi ; et il n'y avait point de place par où je puisse passer avec ma monture.
\VS{15}Je montai de nuit par le torrent et je considérai la muraille. Puis en revenant, je rentrai par la porte de la vallée ; et ainsi je fus de retour.
\VS{16}Or les magistrats ne savaient pas où j'étais allé, ni ce que je faisais ; car je n'avais rien dit jusqu'à ce moment, ni aux Juifs, ni aux sacrificateurs, ni aux chefs, ni aux magistrats, ni au reste de ceux qui s'occupaient des affaires.
\TextTitle{Néhémie partage sa vision de rebâtir la muraille}
\VS{17}Alors je leur dis : Vous voyez la misère dans laquelle nous sommes ! Comment Jérusalem demeure désolée et ses portes brûlées par le feu ! Venez et rebâtissons les murailles de Jérusalem et nous ne serons plus dans l'opprobre.
\VS{18}Et je leur déclarai comment la main de mon Dieu avait été bonne sur moi, et quelles paroles le roi m'avait dites. Alors ils dirent : Levons-nous et bâtissons ! Ils fortifièrent leurs mains pour bien faire.
\TextTitle{Premières oppositions}
\VS{19}Mais Sanballat, le Horonite, Tobija, le serviteur Ammonite, et Guéschem, l'Arabe, l'ayant appris, se moquèrent de nous et nous méprisèrent. Ils dirent : Qu'est-ce que vous faites ? Ne vous rebellez-vous pas contre le roi ?
\VS{20}Et je leur répondis cette parole : Le Dieu des cieux lui-même nous donnera le succès ! Nous donc, qui sommes ses serviteurs, nous nous lèverons et nous bâtirons ; mais vous, vous n'avez aucune part, ni droit, ni souvenir, à Jérusalem.
\Chap{3}
\TextTitle{Les participants à la reconstruction de la muraille}
\VerseOne{}Eliaschib, le souverain sacrificateur, se leva donc avec ses frères, les sacrificateurs et ils rebâtirent la porte des brebis\FTNT{La première porte qui fut reconstruite fut la porte des brebis. Cette porte est très proche du temple, c'est par elle que l'on faisait entrer les brebis destinées aux sacrifices dans la cour du temple. Cette porte est la préfiguration de Jésus-Christ qui s'est lui-même présenté comme étant la « porte des brebis » (Jn. 10:7).}. Ils la sanctifièrent, ils y posèrent ses battants. Ils la sanctifièrent depuis la tour de Méa jusqu'à la tour de Hananeel.
\VS{2}Et les gens de Jéricho rebâtirent à son côté ; et à côté d'eux Zaccur, fils d'Imri, rebâtit aussi.
\VS{3}Les fils de Senaa rebâtirent la porte des poissons. Ils en firent la charpente et y mirent ses portes, ses serrures et ses barres.
\VS{4}Et à leur côté travailla aux réparations Merémoth, fils d'Urie, fils d'Hakkots ; et à leur côté travailla Meschullam, fils de Bérékia, fils de Meschézabeel, et à leur côté travailla Tsadok, fils de Baana.
\VS{5}A leur côté travaillèrent les Tekoïtes ; mais les chefs d'entre eux ne vinrent point au service de leur Seigneur.
\VS{6}Et Jojada, fils de Paséach, et Meschullam, fils de Besodia, réparèrent la vieille porte. Ils en firent la charpente, y mirent ses battants, ses serrures et ses barres.
\VS{7}A leur côté travaillèrent Melatia, le Gabaonite, Jadon, le Méronothite, et les hommes de Gabaon et de Mitspa, vers le siège du gouverneur de ce côté du fleuve.
\VS{8}A côté d'eux travailla Uzziel, fils de Harhaja, d'entre les orfèvres, et à côté de lui travailla Hanania, d'entre les parfumeurs. Et ainsi ils relevèrent Jérusalem jusqu'à la muraille large.
\VS{9}Et à leur côté travailla Rephaja, fils de Hur, chef d'un demi-quartier de Jérusalem.
\VS{10}Puis à leur côté travailla Jedaja, fils de Harumaph, devant sa maison ; et à son côté travailla Hattusch, fils de Haschabnia.
\VS{11}Et Malkija, fils de Harim, et Haschub, fils de Pachath-Moab, en réparèrent une seconde section, et la tour des fours.
\VS{12}Et à leur côté travailla, avec ses filles, Schallum, fils de d'Hallochesch, chef de la moitié du quartier de Jérusalem.
\VS{13}Hanun et les habitants de Zanoach réparèrent la porte de la vallée. Ils la rebâtirent et mirent ses battants, ses serrures, et ses barres, et ils bâtirent mille coudées de muraille, jusqu'à la porte du fumier.
\VS{14}Et Malkija, fils de Récab, chef du quartier de Beth-Hakkérem, répara la porte du fumier. Il la rebâtit et mit ses battants, ses serrures et ses barres.
\VS{15}Schallum, fils de Col-Hozé, chef du quartier de Mitspa, répara la porte de la source. Il la rebâtit et la couvrit, et mit ses portes, ses serrures, et ses barres. Il répara aussi la muraille de l'étang de Siloé, vers le jardin du roi, et jusqu'aux marches qui descendent de la cité de David.
\VS{16}Après lui travailla Néhémie, fils d'Azbuk, chef de la moitié du quartier de Beth-Tsur, jusqu'à l'endroit des sépulcres de David, et jusqu'à l'étang qui avait été refait, et jusqu'à la maison des hommes vaillants.
\VS{17}Après lui travaillèrent les Lévites, Rehum, fils de Bani ; et à son côté travailla Haschabia, chef de la moitié du quartier de Keïla, pour ceux de son quartier.
\VS{18}Après lui travaillèrent leurs frères, Bavvaï, fils de Hénadad, chef de la moitié du quartier de Keïla.
\VS{19}A son côté, Ezer, fils de Josué, chef de Mitspa, en répara autant, à l'endroit où l'on monte à l'arsenal, à l'angle.
\VS{20}Après lui Baruc, fils de Zabbaï, répara avec ardeur une seconde section, depuis l'angle jusqu'à la porte de la maison d'Eliaschib, le souverain sacrificateur.
\VS{21}Après lui Merémoth, fils d'Urie, fils d'Hakkots, répara une seconde section, depuis l'entrée de la maison d'Eliaschib, jusqu'à l'extrémité de la maison d'Eliaschib.
\VS{22}Et après lui travaillèrent les sacrificateurs, habitants des environs.
\VS{23}Après eux, Benjamin et Haschub travaillèrent devant leur maison. Après eux, Azaria, fils de Maaséja, fils d'Anania, travailla auprès de sa maison.
\VS{24}Après lui, Binnuï, fils de Hénadad, répara une seconde section, depuis la maison d'Azaria jusqu'à l'angle et jusqu'au coin.
\VS{25}Palal, fils d'Uzaï, travailla vis-à-vis de l'angle, et de la tour qui sort de la tour supérieure du roi, qui est auprès de la cour de la prison. Après lui travailla Pedaja, fils de Pareosch.
\VS{26}Les Néthiniens, qui demeuraient sur la colline, réparèrent vers l'orient, jusqu'à l'endroit de la porte des eaux, et vers la tour qui sort.
\VS{27}Après eux, les Tekoïtes réparèrent une seconde section, depuis l'endroit de la grande tour qui sort en dehors, jusqu'à la muraille de la colline.
\VS{28}Au-dessus de la porte des chevaux, les sacrificateurs travaillèrent, chacun devant de sa maison.
\VS{29}Après eux, Tsadok, fils d'Immer, travailla devant sa maison. Après lui répara Schemaeja, fils de Schecania, gardien de la porte orientale.
\VS{30}Après lui, Hanania, fils de Schélémia et Hanun le sixième fils de Tsalaph, en réparèrent une seconde section. Après eux, Meschullam, fils de Bérékia, travailla vis-à-vis de sa chambre.
\VS{31}Après lui, Malkija, fils de l'orfèvre, répara jusqu'à la maison des Néthiniens et des marchands, vis-à-vis de la porte de Miphkad, et jusqu'à la chambre haute du coin.
\VS{32}Et les orfèvres et les marchands travaillèrent entre la chambre haute du coin et la porte des brebis.
\Chap{4}
\TextTitle{La prière, solution pour faire face aux attaques et moqueries}
\VerseOne{}Or il arriva que Sanballat apprit que nous rebâtissions la muraille, il devint furieux et très fâché. Il se moqua des Juifs.
\VS{2}Et il dit en présence de ses frères, et des gens de guerre de Samarie : Que font ces faibles Juifs ? Les laissera-t-on faire ? Sacrifieront-ils ? Et achèveront-ils tout en un jour ? Pourront-ils faire revenir à la vie les pierres des monceaux de poussière, puisqu'elles sont brûlées ?
\VS{3}Et Tobija, l'Ammonite, qui était auprès de lui, dit : Qu'ils bâtissent encore ! Si un renard monte, il rompra leur muraille de pierre !
\VS{4}Ô notre Dieu, écoute comment nous sommes méprisés ! Fais retourner leurs insultes sur leur tête, et donne-les en pillage dans un pays de captivité.
\VS{5}Ne couvre point leur iniquité, et que leur péché ne soit point effacé de devant ta face ; car ils ont irrité les bâtisseurs.
\VS{6}Nous rebâtîmes donc la muraille, et tout le mur fut achevé jusqu'à sa moitié ; et le peuple avait le cœur au travail.
\VS{7}Mais quand Sanballat et Tobija, les Arabes, les Ammonites et les Asdodiens eurent appris que la muraille de Jérusalem avait été refaite, et qu'on avait commencé à fermer les brèches, ils s'enflammèrent de colère.
\VS{8}Et ils se liguèrent tous ensemble pour venir faire la guerre contre Jérusalem, et pour les faire échouer.
\VS{9}Alors nous priâmes notre Dieu, et ayant peur d'eux, nous établîmes une garde jour et nuit pour nous défendre contre leurs attaques.
\TextTitle{Persévérance du peuple prêt à se battre à tout moment}
\VS{10}Et Juda disait : La force des ouvriers est affaiblie, et il y a beaucoup de débris, en sorte que nous ne pourrons pas bâtir la muraille.
\VS{11}Et nos ennemis disaient : Qu'ils n'en sachent rien et qu'ils ne voient rien, jusqu'à ce que nous entrions au milieu d'eux ; nous les tuerons et ferons ainsi cesser l'ouvrage.
\VS{12}Mais il arriva que les Juifs, qui habitaient près d'eux, vinrent dix fois nous avertir, de tous les lieux d'où ils se rendaient vers nous.
\VS{13}C'est pourquoi je plaçai le peuple depuis le bas, derrière la muraille, et sur des lieux élevés, secs et lumineux, selon leurs familles, avec leurs épées, leurs lances et leurs arcs.
\VS{14}Puis je regardai et m'étant levé, je dis aux chefs, aux magistrats et au reste du peuple : N'ayez point peur d'eux ! Souvenez-vous du Seigneur, qui est grand et terrible, et combattez pour vos frères, pour vos fils et pour vos filles, pour vos femmes et pour vos maisons !
\VS{15}Et quand nos ennemis entendirent que nous étions avertis, Dieu fit échouer leur projet, et nous retournâmes tous aux murailles, chacun à son travail.
\VS{16}Depuis ce jour-là, la moitié de mes serviteurs travaillait, et l'autre moitié avait des lances, des boucliers, des arcs et des cuirasses. Les gouverneurs suivaient chaque maison de Juda.
\VS{17}Ceux qui bâtissaient la muraille, et ceux qui portaient ou chargeaient les fardeaux, travaillaient chacun d'une main, et de l'autre ils tenaient une arme.
\VS{18}Car chacun de ceux qui bâtissaient avait son épée ceinte autour des reins. Et celui qui sonnait du shofar se tenait près de moi.
\VS{19}Et je dis aux chefs, aux magistrats et au reste du peuple : L'ouvrage est grand et étendu, et nous sommes séparés sur la muraille, éloignés les uns des autres.
\VS{20}En quelque lieu donc d'où vous entendrez le son du shofar, courez-y vers nous ; notre Dieu combattra pour nous\FTNT{Ex. 14:14 ; De. 1:30 ; 2 Ch. 20:29.}.
\VS{21}C'est donc ainsi que nous accomplissions le travail ; la moitié tenait des lances, depuis le lever du jour jusqu'à l'apparition des étoiles.
\VS{22}En ce temps-là, je dis aussi au peuple : Que chacun passe la nuit dans Jérusalem avec son serviteur, afin de faire la garde la nuit et de travailler le jour.
\VS{23}Et nous ne quittions point nos vêtements, ni moi, ni mes frères, ni mes serviteurs, ni les hommes de garde qui me suivaient; chacun n'avait que ses armes et de l'eau.
\Chap{5}
\TextTitle{Cupidité des chefs dévoilée ; rétablissement de la justice}
\VerseOne{}Or il y eut un grand cri du peuple et de leurs femmes, contre les Juifs, leurs frères.
\VS{2}Les uns disaient : Nous, nos fils et nos filles, nous sommes nombreux; qu'on nous donne du blé, afin que nous mangions et que nous vivions.
\VS{3}Et d'autres disaient : Nous engageons nos champs, nos vignes et nos maisons, pour avoir du blé pendant la famine.
\VS{4}D'autres disaient : Nous avons emprunté de l'argent sur nos champs et sur nos vignes pour le tribut du roi.
\VS{5}Toutefois notre chair est comme la chair de nos frères, et nos fils sont comme leurs fils ; et voici, nous soumettons à la servitude nos fils et nos filles ; et quelques-unes de nos filles sont déjà esclaves et ne sont plus en notre pouvoir ; et nos champs et nos vignes sont à d'autres.
\VS{6}Je fus très en colère quand j'entendis leur cri et ces paroles-là.
\VS{7}Je résolus dans mon cœur de réprimander les chefs et les magistrats, et je leur dis : Vous prêtez avec intérêt à vos frères\FTNT{Ex.22:25 ; Lé. 25:36.} ! Et je fis convoquer autour d'eux une grande foule.
\VS{8}Et je leur dis : Nous avons racheté selon notre pouvoir nos frères Juifs vendus aux nations, et vous vendriez vous-mêmes vos frères, ou c'est à nous qu'ils seraient vendus ? Ils se turent, ne trouvant rien à dire.
\VS{9}Et je dis : Ce que vous faites n'est pas bien. Ne voulez-vous pas marcher dans la crainte de notre Dieu, plutôt que d'être insultés par les nations qui sont nos ennemies ?
\VS{10}Moi aussi, mes frères et mes serviteurs, nous leur avons prêté de l'argent et du blé. Abandonnons je vous prie, cette dette !
\VS{11}Rendez-leur, je vous prie, aujourd'hui leurs champs, leurs vignes, leurs oliviers et leurs maisons ; et outre cela, le centième de l'argent, du blé, du vin, et de l'huile que vous exigez d'eux.
\VS{12}Et ils répondirent : Nous les rendrons et nous ne leur demanderons rien ; nous ferons ce que tu dis. Alors j'appelai les sacrificateurs et je les fis jurer de tenir parole.
\VS{13}Et je secouai mon bras et je dis : Que Dieu secoue ainsi de sa maison et de son travail tout homme qui n'aura pas tenu parole, et qu'il soit ainsi secoué et vidé ! Et toute l'assemblée répondit : Amen ! Et ils louèrent Yahweh. Et le peuple fit selon cette parole.
\TextTitle{Néhémie, modèle de dévouement}
\VS{14}Et même, depuis le jour où le roi m'établit comme gouverneur au pays de Juda, depuis la vingtième année jusqu'à la trente-deuxième année du roi Artaxerxès, pendant douze ans, moi et mes frères, nous n'avons pas pris ce qui était assigné au gouverneur comme revenu.
\VS{15}Quoique, les premiers gouverneurs qui avaient été avant moi, chargeaient le peuple, et prenaient de lui du pain et du vin, outre quarante sicles d'argent, et leurs serviteurs tyrannisaient le peuple. Mais je n'ai point fait ainsi, à cause de la crainte de mon Dieu.
\VS{16}Et même, j'ai travaillé à la réparation d'une partie de cette muraille, et nous n'avons acheté aucun champ, et tous mes serviteurs étaient tous ensemble à l'ouvrage.
\VS{17}Et outre cela, j'avais aussi à ma table les Juifs et les magistrats, au nombre de cent cinquante hommes, et ceux qui venaient vers nous des nations d'alentour.
\VS{18}On m'apprêtait chaque jour un bœuf, six moutons choisis et aussi des volailles ; et tous les dix jours on me présentait toutes sortes de vins en abondance. Malgré cela, je n'ai point demandé le revenu qui était assigné au gouverneur ; parce que les travaux étaient à la charge de ce peuple.
\VS{19}Ô mon Dieu ! Souviens-toi de moi en bien, à cause de tout ce que j'ai fait pour ce peuple.
\Chap{6}
\TextTitle{Complot et mensonge contre Néhémie ; fermeté et confiance en Dieu}
\VerseOne{}Or il arriva que quand Sanballat, Tobija, et Guéschem l'Arabe, et le reste de nos ennemis apprirent que j'avais rebâti la muraille, et qu'il n'y restait aucune brèche. (bien que jusqu'à ce temps-là, je n'avais pas encore mis les battants aux portes.)
\VS{2}Alors Sanballat et Guéschem envoyèrent vers moi, pour dire : Viens, et ayons ensemble une rencontre dans les villages qui sont dans la vallée d'Ono. Or ils avaient comploté de me faire du mal.
\VS{3}Mais j'envoyai des messagers vers eux pour leur dire : J'ai un grand ouvrage à faire, et je ne puis descendre. Le travail serait interrompu pendant que je le quitterais pour aller vers vous.
\VS{4}Ils m'adressèrent la même chose quatre fois ; et je leur répondis la même réponse.
\VS{5}Alors Sanballat m'envoya son serviteur pour me tenir le même discours une cinquième fois ; et il avait dans sa main une lettre ouverte.
\VS{6}Il y était écrit : On entend dire parmi les nations, et Gaschmu le dit, que vous pensez, toi et les Juifs, à vous révolter, et que c'est pour cela que tu rebâtis la muraille. Et tu vas, dit-on, devenir leur roi ;
\VS{7}Même que tu as ordonné des prophètes pour te louer dans Jérusalem, et pour dire : Il est roi de Juda. Et maintenant, on fera entendre au roi ces mêmes choses. Viens donc afin que nous consultions ensemble.
\VS{8}Et je renvoyai vers lui pour lui dire : Ce que tu dis là n'est point, mais c'est toi qui l'inventes dans ton propre coeur !
\VS{9}Car tous ces gens voulaient nous effrayer, en disant : Leurs mains relâcheront le travail, de sorte qu'il ne se fera point. Maintenant donc, ô Dieu, fortifie-moi !
\VS{10}Je me rendis à la maison de Schemaeja, fils de Delaja, fils de Mehétabeel. Il s'était enfermé et il me dit : Assemblons-nous dans la maison de Dieu, au milieu du temple et fermons les portes du temple ; car ils doivent venir pour te tuer, et ils viendront pendant la nuit pour te tuer.
\VS{11}Mais je répondis : Un homme tel que moi s'enfuirait-il ? Et quel homme tel que moi pourrait entrer dans le temple pour sauver sa vie ? Je n'y entrerai point.
\VS{12}Et voilà, je reconnus bien que Dieu ne l'avait point envoyé, mais qu'il avait prononcé cette prophétie contre moi parce que Sanballat et Tobija lui avaient donné de l'argent.
\VS{13}Car il était leur pensionnaire pour m'épouvanter, et pour m'obliger à agir de la sorte, et à commettre cette faute, afin qu'ils aient quelque mauvaise chose à me reprocher.
\VS{14}Ô mon Dieu ! Souviens-toi de Tobija et de Sanballat, et de leurs actions et aussi de Noadia, la prophétesse, et du reste des prophètes qui cherchaient à m'effrayer !
\TextTitle{Achèvement de la muraille}
\VS{15}Néanmoins, la muraille fut achevée le vingt-cinquième jour du mois d'Elul, en cinquante-deux jours.
\VS{16}Quand donc tous nos ennemis l'apprirent et qu'ils la virent, toutes les nations qui étaient autour de nous furent dans la crainte ; elles éprouvèrent une grande humiliation, et ils reconnurent que cet ouvrage s'était accompli par le secours de notre Dieu.
\VS{17}Mais aussi en ce temps-là, les chefs de Juda adressaient fréquemment des lettres à Tobija, et celles de Tobija venaient à eux.
\VS{18}Car il y en avait plusieurs en Juda qui s'étaient liés à lui par serment, parce qu'il était gendre de Schecania, fils d'Arach, et que son fils Jochanan avait pris la fille de Meschullam, fils de Bérékia.
\VS{19}Ils racontaient même du bien de lui en ma présence, et lui rapportaient mes paroles. Et Tobija envoyait des lettres pour m'effrayer.
\Chap{7}
\TextTitle{Instructions spécifiques à Hanani et Hanania}
\VerseOne{}Or après que la muraille fut rebâtie, et que j'aie mis les portes, et qu'on ait fait la revue des portiers, des chantres et des Lévites ; 
\VS{2}je donnai cet ordre à Hanani, mon frère, et à Hanania, chef de la forteresse de Jérusalem ; car il était tel qu'un homme fidèle doit être, et il craignait Dieu plus que plusieurs autres ;
\VS{3}et je leur dis : Que les portes de Jérusalem ne s'ouvrent point avant la chaleur du soleil ; et pendant que les gardes seront encore là, que l'on ferme les portes, et qu'on y mette les barres ; que l'on place comme gardes les habitants de Jérusalem, chacun à son poste, et chacun devant de sa maison.
\VS{4}Or la ville était spacieuse et grande, mais il y avait peu de gens, et ses maisons n'étaient point bâties\FTNT{De. 4:27.}.
\TextTitle{Liste des familles revenues de captivité avec Zorobabel}
\VS{5}Et mon Dieu me mit à coeur d'assembler les chefs, les magistrats et le peuple, pour en faire le dénombrement selon leurs généalogies. Je trouvai le registre du dénombrement selon les généalogies de ceux qui étaient montés la première fois. Et j'y trouvai ainsi écrit :
\VS{6}Ce sont ici ceux de la province qui remontèrent de la captivité, d'entre ceux que Nebucadnetsar, roi de Babylone, avait transportés en exil, et qui retournèrent à Jérusalem et en Juda, chacun dans sa ville.
\VS{7}Ils vinrent avec Zorobabel\FTNT{Esd. 5:2.}, Josué, Néhémie, Azaria, Raamia, Nachamani, Mardochée, Bilschan, Mispéreth, Bigvaï, Nehum, et Baana. Nombre des hommes du peuple d'Israël :
\VS{8}Les fils de Pareosch, deux mille cent soixante-douze.
\VS{9}Les fils de Schephathia, trois cent soixante-douze.
\VS{10}Les fils d'Arach, six cent cinquante-deux.
\VS{11}Les fils de Pachath-Moab, des fils de Josué et de Joab, deux mille huit cent dix-huit.
\VS{12}Les fils d'Elam, mille deux cent cinquante-quatre.
\VS{13}Les fils de Zatthu, huit cent quarante-cinq.
\VS{14}Les fils de Zaccaï, sept cent soixante.
\VS{15}Les fils de Binnuï, six cent quarante-huit.
\VS{16}Les fils de Bébaï, six cent vingt-huit.
\VS{17}Les fils d'Azgad, deux mille trois cent vingt-deux.
\VS{18}Les fils d'Adonikam, six cent soixante-sept.
\VS{19}Les fils de Bigvaï, deux mille soixante-sept.
\VS{20}Les fils d'Adin, six cent cinquante-cinq.
\VS{21}Les fils d'Ather, issu d'Ezéchias, quatre-vingt-dix-huit.
\VS{22}Les fils de Haschum, trois cent vingt-huit.
\VS{23}Les fils de Betsaï, trois cent vingt-quatre.
\VS{24}Les fils de Hariph, cent douze.
\VS{25}Les fils de Gabaon, quatre-vingt-quinze.
\VS{26}Les gens de Bethléhem et de Netopha, cent quatre-vingt-huit.
\VS{27}Les gens d'Anathoth, cent vingt-huit.
\VS{28}Les gens de Beth-Azmaveth, quarante-deux.
\VS{29}Les gens de Kirjath-Jearim, de Kephira et de Beéroth, sept cent quarante-trois.
\VS{30}Les gens de Rama et de Guéba, six cent vingt et un.
\VS{31}Les gens de Micmas, cent vingt-deux.
\VS{32}Les gens de Béthel et d'Aï, cent vingt-trois.
\VS{33}Les gens de l'autre Nebo, cinquante-deux.
\VS{34}Les fils d'un autre Elam, mille deux cent cinquante-quatre.
\VS{35}Les fils de Harim, trois cent vingt.
\VS{36}Les fils de Jéricho, trois cent quarante-cinq.
\VS{37}Les fils de Lod, de Hadid et d'Ono, sept cent vingt et un.
\VS{38}Les fils de Senaa, trois mille neuf cent trente.
\TextTitle{Liste des sacrificateurs revenus de captivité}
\VS{39}Sacrificateurs : Les fils de Jedaeja, de la maison de Josué, neuf cent soixante-treize.
\VS{40}Les fils d'Immer, mille cinquante-deux.
\VS{41}Les fils de Paschhur, mille deux cent quarante-sept.
\VS{42}Les fils de Harim, mille dix-sept.
\TextTitle{Liste des Lévites revenus de captivité}
\VS{43}Lévites : Les fils de Josué et de Kadmiel, d'entre les fils de Hodva, soixante quatorze.
\VS{44}Chantres : Les fils d'Asaph, cent quarante-huit.
\VS{45}Portiers : Les fils de Schallum, les fils d'Ather, les fils de Thalmon, les fils d'Akkub, les fils de Hathitha, les fils de Schobaï, cent trente-huit.
\TextTitle{Liste des Néthiniens revenus de captivité}
\VS{46}Néthiniens : Les fils de Tsicha, les fils de Hasupha, les fils de Thabbaoth,
\VS{47}les fils de Kéros, les fils de Sia, les fils de Padon,
\VS{48}les fils de Lebana, les fils de Hagaba, les fils de Salmaï,
\VS{49}les fils de Hanan, les fils de Guiddel, les fils de Gachar,
\VS{50}les fils de Reaja, les fils de Retsin, les fils de Nekoda,
\VS{51}les fils de Gazzam, les fils d'Uzza, les fils de Paséach,
\VS{52}les fils de Bésaï, les fils de Mehunim, les fils de Nephischsim,
\VS{53}les fils de Bakbuk, les fils de Hakupha, les fils de Harhur,
\VS{54}les fils de Batslith, les fils de Mehida, les fils de Harscha,
\VS{55}les fils de Barkos, les fils de Sisera, les fils de Thamach,
\VS{56}les fils de Netsiach, les fils de Hathipha.
\TextTitle{Liste des fils des serviteurs de Salomon revenus de captivité}
\VS{57}Fils des serviteurs de Salomon : Les fils de Sothaï, les fils de Sophéreth, les fils de Perida,
\VS{58}les fils de Jaala, les fils de Darkon, les fils de Guiddel,
\VS{59}les fils de Schephathia, les fils de Hatthil, les fils de Pokéreth-Hatsebaïm, les fils d'Amon.
\VS{60}Tous les Néthiniens, et les fils des serviteurs de Salomon, étaient trois cent quatre-vingt-douze.
\VS{61}Voici ceux qui montèrent de Thel-Mélach, de Thel-Harscha, de Kerub-Addon et d'Immer, lesquels ne purent montrer la maison de leurs pères, ni leur race, pour prouver qu'ils étaient d'Israël.
\VS{62}Les fils de Delaja, les fils de Tobija, les fils de Nekoda, six cent quarante-deux.
\TextTitle{Liste des sacrificateurs exclus de la sacrificature}
\VS{63}Et les sacrificateurs : Les fils de Hobaja, les fils d'Hakkots, les fils de Barzillaï, qui prit pour femme une des filles de Barzillaï, le Galaadite, et qui fut appelé de leur nom.
\VS{64}Ils cherchèrent leur registre généalogique, mais ils n'y furent point trouvés ; c'est pourquoi ils furent exclus de la sacrificature.
\VS{65}Et le gouverneur leur dit de ne pas manger des choses très saintes, jusqu'à ce que le sacrificateur eût consulté l'urim et le thummim\FTNT{Ex. 28:30.}.
\TextTitle{Somme des Israélites revenus de captivité}
\VS{66}Toute l'assemblée réunie était de quarante-deux mille trois cent soixante ;
\VS{67}sans leurs serviteurs et leurs servantes, qui étaient sept mille trois cent trente-sept ; et ils avaient deux cent quarante-cinq chantres ou chanteuses.
\TextTitle{Dons des fils d'Israël pour le trésor}
\VS{68}Ils avaient sept cent trente-six chevaux, deux cent quarante-cinq mulets ;
\VS{69}quatre cent trente-cinq chameaux et six mille sept cent vingt ânes.
\VS{70}Or quelques-uns des chefs des pères firent des dons pour l'ouvrage. Le gouverneur donna au trésor mille drachmes d'or, cinquante coupes, cinq cent trente tuniques de sacrificateurs.
\VS{71}Quelques autres d'entre les chefs des pères donnèrent pour le trésor de l'ouvrage vingt mille drachmes d'or et deux mille deux cent mines d'argent.
\VS{72}Le reste du peuple donna vingt mille drachmes d'or, deux mille mines d'argent et soixante-sept tuniques de sacrificateurs.
\VS{73}Et ainsi les sacrificateurs, les Lévites, les portiers, les chantres, quelques-uns du peuple, les Néthiniens, et tous ceux d'Israël habitèrent dans leurs villes. Ainsi, quand le septième mois approcha, les enfants d'Israël étaient dans leurs villes.
\Chap{8}
\TextTitle{Lecture du livre de la loi, conviction de péché du peuple}
\VerseOne{}Or tout le peuple s'assembla, comme un seul homme, sur la place qui est devant la porte des eaux. Et ils dirent à Esdras, le scribe, d'apporter le livre de la loi de Moïse, que Yahweh avait ordonnée à Israël.
\VS{2}Et ainsi le premier jour du septième mois, Esdras, le sacrificateur, apporta la loi devant l'assemblée, composée d'hommes et de femmes, et de tous ceux qui étaient capables de l'entendre.
\VS{3}Et il lut dans le livre, sur la place qui est devant la porte des eaux, depuis le matin jusqu'au milieu du jour, en présence des hommes et des femmes, et de ceux qui étaient capables d'entendre. Et les oreilles de tout le peuple étaient attentives à la lecture du livre de la loi.
\VS{4}Ainsi Esdras, le scribe, était debout sur une tour bâtie de bois, qu'on avait dressée pour cela. Il avait auprès de lui, à sa droite, Matthithia, Schéma, Anaja, Urie, Hilkija et Maaséja ; et à sa gauche étaient Pedaja, Mischaël, Malkija, Haschum, Haschbaddana, Zacharie, et Meschullam.
\VS{5}Esdras ouvrit le livre devant les yeux de tout le peuple ; car il était au-dessus de tout le peuple ; et sitôt qu'il l'eut ouvert, tout le peuple se tint debout.
\VS{6}Puis Esdras bénit Yahweh, le grand Dieu ; et tout le peuple répondit en élevant leurs mains: Amen ! Amen ! Et ils s'inclinèrent et se prosternèrent devant Yahweh, le visage contre terre.
\VS{7}Aussi Josué, Bani, Schérébia, Jamin, Akkub, Schabbethaï, Hodija, Maaséja, Kelitha, Azaria, Jozabad, Hanan, Pelaja, et les Lévites, faisaient comprendre la loi au peuple, et le peuple se tenait à sa place.
\VS{8}Et ils lisaient dans le livre de la loi de Dieu, ils l'expliquaient et en donnaient l'intelligence, la faisant comprendre par l'Ecriture elle-même.
\VS{9}Or Néhémie, qui est le gouverneur, Esdras, le sacrificateur et le scribe, et les Lévites qui instruisaient le peuple dirent à tout le peuple : Ce jour est consacré à Yahweh, notre Dieu ; ne soyez pas dans les lamentations, et ne pleurez point ! Car tout le peuple pleurait en entendant les paroles de la loi.
\VS{10}Puis on leur dit : Allez, mangez des viandes grasses, et buvez du vin doux ; et envoyez-en des portions à ceux qui n'ont rien de prêt ; car ce jour est consacré à notre Seigneur. Ne soyez donc point tristes, car la joie de Yahweh est votre force.
\VS{11}Et les Lévites faisaient faire silence parmi tout le peuple, en disant : Taisez-vous, car ce jour est saint, et ne vous affligez point.
\VS{12}Ainsi tout le peuple s'en alla pour manger et pour boire, pour envoyer des portions, et pour faire une grande réjouissance, parce qu'ils avaient bien compris les paroles qu'on leur avait fait connaître.
\TextTitle{Célébration de la fête des tabernacles}
\VS{13}Et le second jour, les chefs des pères de tout le peuple, les sacrificateurs et les Lévites, s'assemblèrent auprès d'Esdras, le scribe, pour sagement comprendre les paroles de la loi.
\VS{14}Et ils trouvèrent écrit dans la loi que Yahweh avait ordonnée par Moïse, que les enfants d'Israël devaient habiter sous des tentes\FTNT{Voir les sept fêtes de Yahweh en Lé. 23.} pendant la fête solennelle au septième mois.
\VS{15}Ce qu'ils firent savoir et qu'ils publièrent dans toutes leurs villes et à Jérusalem, en disant : Allez sur la montagne, et apportez des rameaux d'oliviers, et des rameaux d'autres arbres huileux, des rameaux de myrte, des rameaux de palmier, et des rameaux d'arbres touffus, afin de faire des tentes, selon ce qui est écrit.
\VS{16}Alors le peuple alla et apporta des rameaux. Ils se firent des tentes, chacun sur son toit, dans les cours de leurs maisons, et dans les parvis de la maison de Dieu, sur la place de la porte des eaux, et sur la place de la porte d'Ephraïm.
\VS{17}Ainsi toute l'assemblée de ceux qui étaient revenus de la captivité fit des tentes, et ils habitèrent sous ces tentes. Or les enfants d'Israël n'en avaient point fait de telles depuis les jours de Josué, fils de Nun, jusqu'à ce jour ; et il y eut une très grande joie.
\VS{18}On lut dans le livre de la loi de Dieu chaque jour, depuis le premier jour jusqu'au dernier. On célébra la fête pendant sept jours, et il y eut une assemblée solennelle au huitième jour, comme cela est ordonné.
\Chap{9}
\TextTitle{Confession, jeune et prière du peuple}
\VerseOne{}Et le vingt-quatrième jour du même mois, les enfants d'Israël s'assemblèrent, jeûnant, revêtus de sacs, et ayant de la terre sur eux.
\VS{2}Et la race d'Israël se sépara de tous les étrangers, et ils se présentèrent confessant leurs péchés et les iniquités de leurs pères.
\VS{3}Ils se levèrent donc à leur place, et on lut dans le livre de la loi de Yahweh, leur Dieu, pendant un quart de la journée, et pendant un autre quart, ils faisaient confession, et se prosternaient devant Yahweh, leur Dieu.
\TextTitle{Prière des Lévites, alliance avec Yahweh}
\VS{4}Josué, Bani, Kadmiel, Schebania, Bunni, Schérébia, Bani et Kenani se levèrent sur le lieu qu'on avait élevé pour les Lévites, et crièrent à haute voix à Yahweh, leur Dieu.
\VS{5}Et les Lévites Josué, Kadmiel, Bani, Haschabnia, Schérébia, Hodija, Schebania et Pethachja, dirent : Levez-vous, bénissez Yahweh, votre Dieu, d'éternité en éternité ! Que l'on bénisse ton Nom glorieux, qui est au-dessus de toute bénédiction et de toute louange !
\VS{6}Toi seul, Yahweh, tu as fait les cieux, les cieux des cieux, et toute leur armée ; la terre, et tout ce qui y est ; les mers, et toutes les choses qui y vivent. Tu donnes la vie à toutes ces choses, et l'armée des cieux se prosterne devant toi.
\VS{7}Tu es Yahweh, notre Dieu, qui as choisi Abram, et qui l'as fait sortir d'Ur en Chaldée, et qui lui as donné le nom d'Abraham\FTNT{Ge. 11:31 ; Ge. 17:5.}.
\VS{8}Tu trouvas son coeur fidèle devant toi, et tu traitas avec lui cette alliance que tu donneras à sa postérité le pays des Cananéens, des Héthiens, des Amoréens, des Phéréziens, des Jébusiens, et des Guirgasiens. Et tu as accompli ce que tu as promis, parce que tu es juste.
\VS{9}Car tu vis l'affliction de nos pères en Egypte et tu entendis leurs cris près de la Mer Rouge\FTNT{Ex. 2:23-25.}.
\VS{10}Tu fis des miracles et des prodiges sur Pharaon et sur tous ses serviteurs, et sur tout le peuple de son pays ; parce que tu connus qu'ils s'étaient orgueilleusement élevés contre eux, et tu t'es acquis un renom, tel qu'il paraît aujourd'hui.
\VS{11}Tu fendis aussi la mer devant eux, et ils passèrent à sec au milieu de la mer ; et tu jetas dans l'abîme ceux qui les poursuivaient, comme une pierre dans les eaux violentes.
\VS{12}Tu les fis marcher de jour par la colonne de nuée, et de nuit par la colonne de feu, pour les éclairer dans le chemin par où ils devaient aller\FTNT{Ex. 13:21.}.
\VS{13}Tu descendis sur la montagne de Sinaï, tu parlas avec eux du haut des cieux, tu leur donnas des ordonnances justes et des lois de vérité, des statuts et des commandements bons.
\VS{14}Tu leur fis connaître ton saint sabbat\FTNT{Ge 2:1-3 ; Ex. 20:8-11.} ; et tu leur donnas les commandements, les statuts, et la loi par Moïse, ton serviteur.
\VS{15}Tu leur donnas aussi, du haut des cieux, du pain quand ils avaient faim, et tu fis sortir de l'eau du rocher quand ils avaient soif\FTNT{Ex. 16:13-36 ; No. 20 : 8.}. Et tu leur dis d'entrer et de posséder le pays que tu avais juré de leur donner.
\VS{16}Mais nos pères s'élevèrent orgueilleusement et raidirent leur cou. Ils n'écoutèrent point tes commandements.
\VS{17}Ils refusèrent d'écouter et ne se souvinrent point des merveilles que tu avais faites en leur faveur. Mais ils raidirent leur cou, et par leur rébellion, ils s'attribuèrent un chef pour retourner à leur servitude. Mais toi, tu es un Dieu qui pardonne, miséricordieux, compatissant, lent à la colère et abondant en bonté, et tu ne les abandonnas pas.
\VS{18}Et quand ils se firent un veau en métal fondu et qu'ils dirent : Voici ton Dieu qui t'a fait sortir hors d'Egypte, et qu'ils te firent de grands outrages\FTNT{Ex. 32:1-14.} ;
\VS{19}dans ton immense miséricorde, tu ne les abandonnas pourtant pas dans le désert ; et la colonne de nuée ne se retira point pour les conduire le jour par le chemin, ni la colonne de feu la nuit, pour les éclairer dans le chemin par lequel ils devaient aller.
\VS{20}Tu leur donnas ton bon Esprit pour les rendre sages ; tu ne retiras point ta manne de leur bouche, et tu leur donnas de l'eau pour leur soif.
\VS{21}Tu les nourris ainsi quarante ans au désert, en sorte que rien ne leur manqua. Leurs vêtements ne s'usèrent point, et leurs pieds ne s'enflèrent point.
\VS{22}Tu leur donnas les royaumes et les peuples, dont tu partageas entre eux les contrées ; et ils possédèrent le pays de Sihon, le pays du roi de Hesbon, et le pays d'Og, roi de Basan.
\VS{23}Et tu multiplias leurs fils comme les étoiles des cieux, et les fis entrer au pays dont tu avais dit à leurs pères qu'ils y entreraient pour le posséder.
\VS{24}Ainsi leurs fils y entrèrent et possédèrent le pays ; tu humilias devant eux les habitants du pays, les Cananéens, et les livras entre leurs mains, eux et leurs rois, et les peuples du pays, afin qu'ils en fissent selon leur volonté.
\VS{25}Ils prirent les villes fortifiées et la terre grasse, ils possédèrent les maisons remplies de toutes sortes de biens, les puits qu'on avait creusés, les vignes, les oliviers, et les arbres fruitiers en abondance ; ils mangèrent, ils se rassasièrent ; ils s'engraissèrent et ils vécurent dans les délices de ta grande bonté.
\VS{26}Mais ils se rebellèrent et se révoltèrent contre toi. Ils jetèrent ta loi derrière leur dos, ils tuèrent tes prophètes qui les avertissaient pour les ramener à toi, et ils te firent de grands outrages.
\VS{27}C'est pourquoi tu les donnas aux mains de leurs ennemis, qui les opprimèrent. Mais au temps de leur détresse, ils crièrent à toi, et tu les entendis des cieux ; et selon ta grande miséricorde, tu leur donnas des libérateurs qui les délivrèrent de la main de leurs ennemis.
\VS{28}Mais dès qu'ils eurent du repos, ils recommencèrent à faire le mal devant toi. Alors tu les abandonnas entre les mains de leurs ennemis, qui dominèrent sur eux. Puis ils revinrent et crièrent vers toi, et tu les entendis des cieux. Ainsi tu les délivras selon tes miséricordes, plusieurs fois, et en divers temps.
\VS{29}Et tu les exhortas à revenir à ta loi, mais ils s'élevèrent orgueilleusement et n'écoutèrent pas tes commandements ; ils péchèrent contre tes ordonnances, qui font vivre l'homme qui les observe. Ils tirèrent l'épaule en arrière, raidirent leur cou et n'écoutèrent pas.
\VS{30}Tu les supportas patiemment plusieurs années, et tu les avertissais par ton Esprit, par la main de tes prophètes ; mais ils ne prêtèrent point l'oreille. C'est pourquoi tu les livras entre les mains des peuples des pays étrangers.
\VS{31}Néanmoins, dans ta grande miséricorde, tu ne les anéantis pas et tu ne les abandonnas pas ; car tu es un Dieu compatissant et miséricordieux.
\VS{32}Et maintenant donc, ô notre Dieu ! Grand, puissant et terrifiant, qui garde ton alliance et la miséricorde ; ne regarde pas comme peu de chose cette affliction qui nous est arrivée, à nous, à nos rois, à nos chefs, à nos sacrificateurs, à nos prophètes, à nos pères et à tout ton peuple, depuis le temps des rois d'Assyrie jusqu'à aujourd'hui.
\VS{33}Tu as été juste dans toutes les choses qui nous sont arrivées ; car tu as agi avec fidélité, mais nous, nous avons agi méchamment.
\VS{34}Nos rois, nos chefs, nos sacrificateurs et nos pères n'ont point pratiqué ta loi et n'ont point été attentifs à tes commandements ni à tes témoignages par lesquels tu les as avertis.
\VS{35}Ils ne t'ont point servi durant leur règne ni durant les grands biens que tu leur as faits, même dans le pays vaste et riche que tu leur avais donné pour être à leur disposition, et ils ne se sont point détournés de leurs mauvaises oeuvres.
\VS{36}Voici, nous sommes aujourd'hui esclaves ! Sur la terre que tu as donnée à nos pères pour en manger le fruit et les biens ; voici, nous y sommes esclaves !
\VS{37}Elle rapporte ses produits en abondance pour les rois que tu as établis sur nous à cause de nos péchés, et qui dominent sur nos corps et sur nos bêtes, à leur volonté, de sorte que nous sommes dans une grande angoisse !
\VS{38}C'est pourquoi, à cause de toutes ces choses, nous contractâmes une alliance et nous l'écrivîmes ; et les chefs d'entre nous, nos Lévites et nos sacrificateurs y apposèrent leur sceau.
\Chap{10}
\TextTitle{Liste des contractants et termes de l'alliance}
\VerseOne{}Voici ceux qui apposèrent leur sceau. Néhémie, qui est le gouverneur, fils de Hacalia, et Sédécias.
\VS{2}Seraja, Azaria, Jérémie,
\VS{3}Paschhur, Amaria, Malkija,
\VS{4}Hattusch, Schebania, Malluc,
\VS{5}Harim, Merémoth, Abdias,
\VS{6}Daniel, Guinnethon, Baruc,
\VS{7}Meschullam, Abija, Mijamin,
\VS{8}Maazia, Bilgaï et Schemaeja. Ce sont les sacrificateurs.
\VS{9}Des Lévites : Josué, fils d'Azania, Binnuï d'entre les fils de Hénadad, et Kadmiel.
\VS{10}Et leurs frères, Schebania, Hodija, Kelitha, Pelaja, Hanan,
\VS{11}Michée, Rehob, Haschabia.
\VS{12}Zaccur, Schérébia, Schebania,
\VS{13}Hodija, Bani et Beninu.
\VS{14}Des chefs du peuple : Pareosch, Pachath-Moab, Elam, Zatthu, Bani,
\VS{15}Bunni, Azgad, Bébaï,
\VS{16}Adonija, Bigvaï, Adin,
\VS{17}Ather, Ezéchias, Azzur,
\VS{18}Hodija, Haschum, Betsaï,
\VS{19}Hariph, Anathoth, Nébaï,
\VS{20}Magpiasch, Meschullam, Hézir,
\VS{21}Meschézabeel, Tsadok, Jaddua,
\VS{22}Pelathia, Hanan, Anaja,
\VS{23}Hosée, Hanania, Haschub,
\VS{24}Hallochesch, Pilcha, Schobek,
\VS{25}Rehum, Haschabna, Maaséja,
\VS{26}Achija, Hanan, Anan,
\VS{27}Malluc, Harim et Baana.
\VS{28}Quant au reste du peuple, les sacrificateurs, les Lévites, les portiers, les chantres, les Néthiniens et tous ceux qui s'étaient séparés des peuples de ces pays pour suivre la loi de Dieu, leurs femmes, leurs fils et leurs filles, tous ceux qui étaient capables de connaissance et d'intelligence,
\VS{29}se joignirent à leurs frères les plus considérables d'entre eux. Ils s'engagèrent par serment et jurèrent de marcher dans la loi de Dieu, qui avait été donnée par Moïse, serviteur de Dieu ; de garder et faire tous les commandements de Yahweh, notre Seigneur, ses jugements et ses ordonnances ;
\VS{30}de ne pas donner nos filles aux peuples du pays, et de ne pas prendre leurs filles pour nos fils ;
\VS{31}de ne rien prendre le jour du sabbat, ou tel autre jour consacré, des peuples du pays qui apporteraient des marchandises et toutes sortes de denrées, le jour du sabbat, pour les vendre, d'abandonner la septième année et de faire remise de toute dette.
\VS{32}Nous fîmes aussi des ordonnances, nous chargeant de donner chaque année le tiers d'un sicle, pour le service de la maison de notre Dieu,
\VS{33}pour les pains de proposition, pour l'offrande perpétuelle et pour l'holocauste perpétuel ; pour ceux des sabbats, des nouvelles lunes et des fêtes ; pour les choses consacrées, pour les sacrifices d'expiation afin de faire propitiation pour Israël ; et pour toute l'oeuvre de la maison de notre Dieu.
\VS{34}Nous tirâmes au sort, pour l'offrande du bois, tant les sacrificateurs et les Lévites, que le peuple, afin de l'amener dans la maison de notre Dieu, selon les maisons de nos pères, et dans les temps fixés, d'année en année, pour le brûler sur l'autel de Yahweh, notre Dieu, ainsi qu'il est écrit dans la loi.
\VS{35}Nous décidâmes aussi d'apporter dans la maison de Yahweh, d'année en année, les premiers fruits de notre terre, et les prémices de tous les fruits de tous les arbres ;
\VS{36}d'amener les premiers-nés de nos fils, et de nos bêtes, comme il est écrit dans la loi ; et d'amener dans la maison de notre Dieu, aux sacrificateurs qui font le service dans la maison de notre Dieu, les premiers-nés de nos boeufs et de notre menu bétail;
\VS{37}d'apporter les prémices de notre pâte, nos offrandes, les fruits de tous les arbres, le vin, et l'huile aux sacrificateurs, dans les chambres de la maison de notre Dieu, et la dîme de notre terre aux Lévites, et que les Lévites prendraient les dîmes dans toutes les villes agricoles.
\VS{38}Le sacrificateur, fils d'Aaron, sera avec les Lévites, lorsque les Lévites paieront la dîme\FTNT{Il est question de la dîme des Lévites (No. 18:24 ; De. 14:28-29).}; et les Lévites apporteront la dîme de la dîme\FTNT{Il s'agit ici de la dîme de la dîme que les Lévites donnaient aux sacificateurs. Elle était apportée aux magasins du temple. Voir commentaires en No. 18:21 et Mal. 3:10.} à la maison de notre Dieu, dans les chambres de la maison où sont les magasins\FTNT{Le mot hébreu « owtsar » (trésor) signifie aussi magasin (Né. 12:44 ; Né. 13:12. Né. 13:13).}.
\VS{39}Car les enfants d'Israël et les fils de Lévi apporteront dans ces chambres les offrandes du blé, du vin et de l'huile ; là sont les ustensiles du sanctuaire, et les sacrificateurs qui font le service, les portiers, et les chantres. Et nous n'abandonnâmes point la maison de notre Dieu.
\Chap{11}
\TextTitle{Les habitants de Jérusalem}
\VerseOne{}Les chefs du peuple demeurèrent à Jérusalem. Mais tout le reste du peuple tira au sort, afin qu'un sur dix vînt habiter à Jérusalem, la ville sainte, et que les neuf autres parties demeurassent dans les autres villes.
\VS{2}Et le peuple bénit tous ceux qui se présentèrent volontairement pour habiter à Jérusalem.
\VS{3}Voici les chefs de la province qui habitèrent à Jérusalem ; les autres s'étant établis dans les villes de Juda, chacun dans sa propriété, selon sa ville, Israélites, sacrificateurs, Lévites, Néthiniens, et les fils des serviteurs de Salomon.
\VS{4}A Jérusalem habitèrent donc des fils de Juda et des fils de Benjamin. Des fils de Juda : Athaja, fils d'Ozias, fils de Zacharie, fils d'Amaria, fils de Schephathia, fils de Mahalaleel, d'entre les fils de Pérets,
\VS{5}et Maaséja, fils de Baruc, fils de Col-Hozé, fils de Hazaja, fils d'Adaja, fils de Jojarib, fils de Zacharie, fils de Schiloni.
\VS{6}Total des fils de Pérets, qui s'établirent à Jérusalem : Quatre cent soixante-huit vaillants hommes.
\VS{7}Voici les fils de Benjamin : Sallu, fils de Meschullam, fils de Joëd, fils de Pedaja, fils de Kolaja, fils de Maaséja, fils d'Ithiel, fils d'Esaïe,
\VS{8}et après lui, Gabbaï et Sallaï : Neuf cent vingt-huit.
\VS{9}Joël, fils de Zicri, était leur chef ; et Juda, fils de Senua, était le second chef de la ville.
\VS{10}Des sacrificateurs : Jedaeja, fils de Jojarib, Jakin,
\VS{11}Seraja, fils de Hilkija, fils de Meschullam, fils de Tsadok, fils de Merajoth, fils d'Achithub, prince de la maison de Dieu,
\VS{12}et leurs frères, faisant le service de la maison : Huit cent vingt-deux. Adaja, fils de Jerocham, fils de Pelalia, fils d'Amtsi, fils de Zacharie, fils de Paschhur, fils de Malkija,
\VS{13}et ses frères, chefs des pères : Deux cent quarante-deux ; et Amaschsaï, fils d'Azareel, fils d'Achzaï, fils de Meschillémoth, fils d'Immer,
\VS{14}et leurs frères, forts et vaillants: Cent vingt-huit. Zabdiel, fils de Guedolim, était leur chef.
\VS{15}Des Lévites : Schemaeja, fils de Haschub, fils d'Azrikam, fils de Haschabia, fils de Bunni,
\VS{16}Schabbethaï et Jozabad chargés des travaux extérieurs pour la maison de Dieu, étant d'entre les chefs des Lévites ;
\VS{17}Matthania, fils de Michée, fils de Zabdi, fils d'Asaph, était le chef qui commençait le premier à chanter les louanges dans la prière, et Bakbukia, le second parmi ses frères, puis Abda, fils de Schammua, fils de Galal, fils de Jeduthun.
\VS{18}Total des Lévites dans la ville sainte : Deux cent quatre-vingt-quatre.
\VS{19}Et les portiers : Akkub, Thalmon, et leurs frères qui gardaient les portes : Cent soixante-douze.
\TextTitle{Les habitants des autres villes}
\VS{20}Le reste d'Israël, des sacrificateurs et des Lévites, fut dans toutes les villes de Juda, chacun dans son héritage.
\VS{21}Mais les Néthiniens habitèrent sur la colline ; et Tsicha et Guischpa étaient leurs chefs.
\VS{22}Celui qui avait la charge des Lévites à Jérusalem était Uzzi, fils de Bani, fils de Haschabia, fils de Matthania, fils de Michée, d'entre les fils d'Asaph, chantres, pour l'ouvrage de la maison de Dieu ;
\VS{23}car il y avait un commandement du roi à leur égard, et il y avait chaque jour un salaire assuré pour les chantres.
\VS{24}Pethachja, fils de Meschézabeel, d'entre les fils de Zérach, fils de Juda, était commissaire du roi pour toutes les affaires du peuple.
\VS{25}Dans les villages et leurs territoires, quelques-uns des fils de Juda habitèrent à Kirjath-Arba, et dans les lieux de son ressort ; à Dibon, et dans les lieux de son ressort ; à Jekabtseel, et dans les villages de son ressort,
\VS{26}à Jéschua, à Molada, à Beth-Paleth,
\VS{27}à Hatsar-Schual, à Beer-Schéba, et dans les lieux de son ressort,
\VS{28}à Tsiklag, à Mecona, et dans les lieux de son ressort,
\VS{29}à En-Rimmon, à Tsorea, à Jarmuth,
\VS{30}à Zanoach, à Adullam, et dans leurs villages, à Lakis et dans ses territoires, à Azéka et dans les lieux de son ressort. Ils habitèrent depuis Beer-Schéba jusqu'à la vallée de Hinnom.
\VS{31}Et les fils de Benjamin habitèrent depuis Guéba à Micmasch, à Ajja, à Béthel, et dans les lieux de son ressort,
\VS{32}à Anathoth, à Nob, à Hanania,
\VS{33}à Hatsor, à Rama, à Guitthaïm,
\VS{34}à Hadid, à Tseboïm, à Neballath,
\VS{35}à Lod, et à Ono, la vallée des ouvriers.
\VS{36}D'entre les Lévites, des classes de Juda se rattachèrent à Benjamin.
\Chap{12}
\TextTitle{Les sacrificateurs et les Lévites montés avec Zorobabel}
\VerseOne{}Voici les sacrificateurs et les Lévites qui montèrent avec Zorobabel, fils de Schealthiel, et avec Josué : Seraja, Jérémie, Esdras,
\VS{2}Amaria, Malluc, Hattusch,
\VS{3}Schecania, Rehum, Merémoth,
\VS{4}Iddo, Guinnethoï, Abija,
\VS{5}Mijamin, Maadia, Bilga,
\VS{6}Schemaeja, Jojarib, Jedaeja,
\VS{7}Sallu, Amok, Hilkija, Jedaeja. Ce furent là les chefs des sacrificateurs, et de leurs frères, du temps de Josué.
\VS{8}Lévites : Josué, Binnuï, Kadmiel, Schérébia, Juda, Matthania, qui dirigeait les louanges, lui et ses frères.
\VS{9}Bakbukia et Unni, leurs frères, étaient avec eux pour la surveillance.
\TextTitle{Les fils des sacrificateurs}
\VS{10}Josué engendra Jojakim, Jojakim engendra Eliaschib, Eliaschib engendra Jojada,
\VS{11}Jojada engendra Jonathan, et Jonathan engendra Jaddua.
\VS{12}Au temps de Jojakim, étaient sacrificateurs, chefs des pères : Pour Seraja, Meraja ; pour Jérémie, Hanania ;
\VS{13}pour Esdras, Meschullam ; pour Amaria, Jochanan ;
\VS{14}pour Meluki, Jonathan ; pour Schebania, Joseph ;
\VS{15}pour Harim, Adna ; pour Merajoth, Helkaï ;
\VS{16}pour Iddo, Zacharie ; pour Guinnethon, Meschullam ;
\VS{17}pour Abija, Zicri ; pour Minjamin et Moadia, Pilthaï ;
\VS{18}pour Bilga, Schammua ; pour Schemaeja, Jonathan ;
\VS{19}pour Jojarib, Matthnaï ; pour Jedaeja, Uzzi ;
\VS{20}pour Sallaï, Kallaï ; pour Amok, Eber ;
\VS{21}pour Hilkija, Haschabia ; pour Jedaeja, Nethaneel.
\TextTitle{Les chefs des fils de Lévi}
\VS{22}Au temps d'Eliaschib, de Jojada, de Jochanan et de Jaddua, les Lévites, chefs de famille, et les sacrificateurs, furent inscrits sous le règne de Darius, le Perse.
\VS{23}Les fils de Lévi, chefs des pères, furent enregistrés dans le livre des Chroniques jusqu'au temps de Jochanan, fils d'Eliaschib.
\VS{24}Les chefs des Lévites, Haschabia, Schérébia, et Josué, fils de Kadmiel, et leurs frères, étaient vis-à-vis d'eux, pour louer et célébrer, selon l'ordre de David, homme de Dieu.
\VS{25}Matthania, Bakbukia, Abdias, Meschullam, Thalmon, et Akkub, les portiers, faisaient la garde au seuil des portes.
\VS{26}Ce fut du temps de Jojakim, fils de Josué, fils de Jotsadak, et du temps de Néhémie, le gouverneur, et d'Esdras, sacrificateur et scribe.
\TextTitle{La dédicace de la muraille de Jérusalem}
\VS{27}Lors de la dédicace de la muraille de Jérusalem, on envoya chercher les Lévites de tous les lieux où ils étaient, pour les faire venir à Jérusalem, afin de célébrer la dédicace avec joie, par des louanges, et par des chants sur des cymbales, des luths et des harpes.
\VS{28}Les fils des chantres se rassemblèrent des plaines aux alentours de Jérusalem, des villages des Nethophatiens,
\VS{29}de Beth-Guilgal, et des territoires de Guéba et d'Azmaveth ; car les chantres s'étaient bâtis des villages aux alentours de Jérusalem.
\VS{30}Les sacrificateurs et les Lévites se purifièrent, et ils purifièrent le peuple, les portes et la muraille.
\VS{31}Puis je fis monter sur la muraille les chefs de Juda, et j'établis deux grands chœurs. Le premier se mit en marche du côté droit sur la muraille, vers la porte du fumier.
\VS{32}Et après eux marchait Hosée, avec la moitié des chefs de Juda,
\VS{33}Azaria, Esdras, Meschullam,
\VS{34}Juda, Benjamin, Schemaeja et Jérémie,
\VS{35}des fils des sacrificateurs avec les trompettes, Zacharie, fils de Jonathan, fils de Schemaeja, fils de Matthania, fils de Michée, fils de Zaccur, fils d'Asaph,
\VS{36}et ses frères, Schemaeja, Azareel, Milalaï, Guilalaï, Maaï, Nethaneel, Juda, et Hanani, avec les instruments des cantiques de David, homme de Dieu. Esdras, le scribe, marchait devant eux.
\VS{37}A la porte de la source, qui était vis-à-vis d'eux, ils montèrent aux marches de la cité de David, par la montée de la muraille, depuis la maison de David, jusqu'à la porte des eaux, vers l'orient.
\VS{38}Le second choeur de ceux qui chantaient les louanges allait à l'opposé. J'allais après lui, avec l'autre moitié du peuple, allant sur la muraille. Passant par-dessus la tour des fours, jusqu'à la muraille large ;
\VS{39}puis vers la porte d'Ephraïm, vers la vieille porte, vers la porte des poissons, la tour de Hananeel, et la tour de Méa, jusqu'à la porte des brebis. Et l'on s'arrêta à la porte de la prison.
\VS{40}Les deux choeurs s'arrêtèrent dans la maison de Dieu ; moi aussi, avec les magistrats qui étaient avec moi,
\VS{41}et les sacrificateurs Eliakim, Maaséja, Minjamin, Michée, Eljoénaï, Zacharie, Hanania, avec les trompettes,
\VS{42}et Maaséja, Schemaeja, Eléazar, Uzzi, Jochanan, Malkija, Elam et Ezer. Puis les chantres, desquels Jizrachja avait la charge, se firent entendre.
\VS{43}On offrit ce jour-là de nombreux sacrifices, et on se réjouit, parce que Dieu leur avait donné un grand sujet de joie. Les femmes et les enfants se réjouirent aussi ; et la joie de Jérusalem fut entendue au loin.
\TextTitle{Les sacrificateurs et les Lévites à leur poste}
\VS{44}En ce jour-là, on établit des hommes sur les chambres des trésors, des offrandes, des prémices et des dîmes ; pour rassembler du territoire des villes les portions ordonnées par la loi aux sacrificateurs et aux Lévites. Car Juda se réjouissait de ce que les sacrificateurs et de ce que les Lévites étaient à leur poste,
\VS{45}et parce qu'ils avaient gardé la charge qui leur avait été donnée de la part de leur Dieu, et la charge de la purification. Les chantres et les portiers remplissaient aussi leurs fonctions, selon le commandement de David, et de Salomon, son fils.
\VS{46}Car autrefois, du temps de David et d'Asaph, on avait établi des chefs de chantres et des cantiques de louange et de reconnaissance à Dieu.
\VS{47}Tout Israël, du temps de Zorobabel et de Néhémie, donna les portions des chantres et des portiers, jour par jour, et les consacraient aux Lévites, et les Lévites les consacraient aux fils d'Aaron.
\Chap{13}
\TextTitle{Lecture du livre de Moïse, séparation d'avec les étrangers}
\VerseOne{}Dans ce temps-là, on lut en présence du peuple dans le livre de Moïse, et l'on y trouva écrit que les Ammonites et les Moabites ne devaient jamais entrer dans l'assemblée de Dieu,
\VS{2}parce qu'ils n'étaient pas venus au-devant des enfants d'Israël avec du pain et de l'eau ; et qu'ils avaient engagé à prix d'argent Balaam\FTNT{Balaam : voir No. 22,23 et 24.} contre eux pour qu'il les maudisse ; mais notre Dieu changea la malédiction en bénédiction.
\VS{3}Dès qu'on eut entendu la loi, on sépara d'Israël tous les étrangers.
\TextTitle{Purification des chambres du temple}
\VS{4}Avant cela, le sacrificateur Eliaschib, établi sur les chambres de la maison de notre Dieu, et parent de Tobija,
\VS{5}avait disposé pour lui une grande chambre, où on mettait auparavant les offrandes, l'encens, les ustensiles, les dîmes du blé, du vin et de l'huile, qui étaient ordonnées pour les Lévites, pour les chantres et pour les portiers, avec les contributions pour les sacrificateurs.
\VS{6}Je n'étais point à Jérusalem pendant tout cela, car j'étais retourné vers le roi la trente-deuxième année d'Artaxerxès, roi de Babylone. Et à la fin de l'année, j'obtins du roi la permission
\VS{7}de revenir à Jérusalem, et je m'aperçus du mal qu'Eliaschib avait fait, en disposant une chambre pour Tobija dans le parvis de la maison de Dieu.
\VS{8}J'en éprouvai un vif déplaisir, et je jetai tous les objets de Tobija hors de la chambre ;
\VS{9}j'ordonnai qu'on purifie les chambres, et j'y ramenai les ustensiles de la maison de Dieu, les offrandes et l'encens.
\TextTitle{Sur les portions des Lévites}
\VS{10}J'appris aussi que les portions des Lévites ne leur avaient point été données ; et que les Lévites et les chantres qui faisaient le service s'étaient enfuis chacun sur sa terre.
\VS{11}Je fis des réprimandes aux magistrats, leur disant : Pourquoi a-t-on abandonné la maison de Dieu ? Je rassemblai les Lévites et les chantres, et les rétablis à leur place.
\VS{12}Alors tous ceux de Juda apportèrent dans le trésor les dîmes du blé, du vin et de l'huile.
\VS{13}Je confiai la surveillance du trésor à Schélémia, le sacrificateur, et Tsadok, le scribe, et Pedaja, l'un des Lévites ; et pour les aider, Hanan, fils de Zaccur, fils de Matthania, parce qu'ils étaient considérés comme très fidèles. Ils furent chargés de faire les distributions à leurs frères.
\VS{14}Souviens-toi de moi, ô mon Dieu, à cause de cela et n'efface point ce que j'ai fait avec fidélité pour la maison de mon Dieu, et pour ce qu'il est ordonné d'y faire !
\TextTitle{Avertissement pour le respect du sabbat}
\VS{15}En ces jours-là, je vis quelques-uns de Juda fouler aux pressoirs le jour du sabbat, et d'autres apporter des gerbes, et charger sur des ânes du vin, des raisins, des figues, et toutes autres sortes de fardeaux, et les apporter à Jérusalem le jour du sabbat ; et je les avertis le jour où ils vendaient leurs denrées.
\VS{16}Les Tyriens, qui demeuraient aussi à Jérusalem, apportaient du poisson, et plusieurs autres marchandises, et les vendaient aux fils de Juda dans Jérusalem le jour du sabbat.
\VS{17}Je fis des réprimandes aux chefs de Juda, et leur dis : Quel mal ne faites-vous pas, en violant le jour du sabbat ?
\VS{18}Vos pères n'ont-ils pas fait la même chose, et n'est-ce pas pour cela que notre Dieu a fait venir tout ce mal sur nous et sur cette ville ? Et vous amenez de nouveau son ardente colère contre Israël, en violant le sabbat !
\VS{19}C'est pourquoi, dès que le soleil s'était retiré des portes de Jérusalem, avant le sabbat, par mon commandement, on ferma les portes ; j'ordonnai aussi qu'on ne les ouvre point jusqu'après le sabbat. Et je plaçai quelques-uns de mes serviteurs aux portes, afin d'empêcher l'entrée des fardeaux le jour du sabbat.
\VS{20}Alors les marchands et les vendeurs de toutes sortes de denrées passèrent une ou deux fois la nuit hors de Jérusalem.
\VS{21}Je les avertis et je leur dis : Pourquoi passez-vous la nuit devant la muraille ? Si vous le faites encore, je mettrai la main sur vous. Ainsi, depuis ce temps-là, ils ne vinrent plus le jour du sabbat.
\VS{22}J'ordonnai aussi aux Lévites de se purifier, et de venir garder les portes pour sanctifier le jour du sabbat. Souviens-toi de moi, ô mon Dieu, à cause de cela, et ai compassion de moi selon la grandeur de ta miséricorde !
\TextTitle{Condamnation des unions mixtes ; rétablissement des fonctions des sacrificateurs et des Lévites}
\VS{23}En ces jours-là, je vis des Juifs qui avaient pris des femmes Asdodiennes, Ammonites et Moabites.
\VS{24}La moitié de leurs fils parlaient en partie asdodien et ne savaient point parler l'hébreu ; mais ils parlaient la langue de divers peuples.
\VS{25}Je leur fis des réprimandes et les maudis ; j'en frappai même quelques-uns, leur arrachai les cheveux et les fis jurer par Dieu, qu'ils ne donneraient point leurs filles à leurs fils, et qu'ils ne prendraient point leurs filles pour leurs fils, ou pour eux.
\VS{26}Salomon, le roi d'Israël, n'avait-il point péché par ce moyen ? Il n'y avait point de roi semblable à lui parmi un grand nombre de nations, il était aimé de son Dieu, et Dieu l'avait établi pour roi sur tout Israël ; toutefois, les femmes étrangères l'amenèrent à pécher.
\VS{27}Faut-il donc apprendre que vous fassiez tout ce grand mal, de commettre ce péché contre notre Dieu, en prenant des femmes étrangères ?
\VS{28}Or un des fils de Jojada, fils d'Eliaschib, grand sacrificateur, était gendre de Sanballat, le Horonite. Je le chassai loin de moi.
\VS{29}Souviens-toi d'eux, ô mon Dieu, car ils ont souillé le sacerdoce et l'alliance contractée par les sacrificateurs et les Lévites.
\VS{30}Ainsi je les nettoyai de tous les étrangers et je rétablis les fonctions des sacrificateurs et des Lévites, chacun selon ce qu'il avait à faire,
\VS{31}et ce qui concernait l'offrande du bois aux temps fixés, de même que les prémices. Souviens-toi de moi en bien, ô mon Dieu !
\PPE{}
\end{multicols}

\clearpage\ShortTitle{1 Chroniques}\BookTitle{1 Chroniques}\BFont
\noindent\hrulefill
{\footnotesize
\textit{
\bigskip
{\centering{}
\\Auteur : Probablement Esdras
\\(Heb. : Hayyamim dibre)
\\Signification : Actes des journées
\\Thème : Généalogies et Histoire
\\Date de rédaction : 5ème siècle av. J.-C.\\}
}
%\bigskip
\textit{
\\Les deux livres des Chroniques constituent des compléments aux livres des Rois dans la mesure où ils confirment les récits de ceux-ci.
%\bigskip
\\Après avoir établi la généalogie d’Adam à Jacob, puis une généalogie plus détaillée de la descendance de Jacob jusqu’au retour de la captivité babylonienne, le premier livre des Chroniques reprend l’histoire du roi David et met un accent particulier sur certains combats qu’il eut à mener, les rapports avec ses serviteurs, ainsi que les préparatifs de la construction du temple. Il présente aussi l’organisation du travail des sacrificateurs et des Lévites au service de
Dieu et du peuple.\bigskip
}
}
\par\nobreak\noindent\hrulefill
\begin{multicols}{2}
\Chap{1}
\TextTitle{Généalogie d'Adam à Noé\FTNTT{Ge. 5:1-32}}
\VerseOne{}Adam, Seth, Enosch\FTNT{Les généalogies se faisaient par les premiers-nés de chaque famille.}.
\VS{2}Kénan, Mahalaleel, Jéred ;
\VS{3}Hénoc, Metuschélah, Lémec.
\VS{4}Noé, Sem, Cham et Japhet\FTNT{Ge. 5:1-32.}.
\TextTitle{Les fils de Japhet\FTNTT{Ge. 10:2-5.}}
\VS{5}Les fils de Japhet furent : Gomer, Magog, Madaï, Javan, Tubal, Méschec et Tiras.
\VS{6}Les fils de Gomer furent : Aschkenaz, Diphat et Togarma.
\VS{7}Les fils de Javan furent : Elischa, Tarsisa, Kittim et Rodanim.
\TextTitle{Les fils de Cham\FTNTT{Ge. 10:6-20.}}
\VS{8}Les fils de Cham furent : Cusch, Mitsraïm, Puth et Canaan.
\VS{9}Les fils de Cusch furent  : Saba, Havila, Sabta, Raema et Sabteca. Les fils de Raema furent : Séba et Dedan.
\VS{10}Cusch engendra aussi Nimrod qui commença à être puissant sur la terre.
\VS{11}Mitsraïm engendra les Ludim, les Anamim, les Lehabim, les Naphtuhim,
\VS{12}les Patrusim, les Casluhim, desquels sont issus les Philistins et les Caphtorim.
\VS{13}Canaan engendra Sidon, son fils aîné, et Heth;
\VS{14}les Jébusiens, les Amoréens, les Guirgasiens,
\VS{15}les Héviens, les Arkiens, les Siniens,
\VS{16}les Arvadiens, les Tsemariens et les Hamathiens.
\TextTitle{Les fils de Sem\FTNTT{Ge. 10:21-31.}}
\VS{17}Les fils de Sem furent : Elam, Assur, Arpacschad, Lud, Aram, Uts, Hul, Guéter et Méschec.
\VS{18}Arpacschad engendra Schélach, et Schélach engendra Héber.
\VS{19}A Héber naquirent deux fils : L'un s'appelait Péleg, car en son temps la terre fut partagée; et son frère s’appelait Jokthan.
\VS{20}Jokthan engendra Almodad, Schéleph, Hatsarmaveth, Jérach,
\VS{21}Hadoram, Uzal, Dikla,
\VS{22}Ebal, Abimaël, Séba,
\VS{23}Ophir, Havila et Jobab; tous ceux-là furent des fils de Jokthan\FTNT{Ge. 10:2-31.}.
\TextTitle{De Sem aux fils d'Abraham\FTNTT{Ge. 11:10-26.}}
\VS{24}Sem, Arpacschad, Schélach\FTNT{Ge. 11:10-26},
\VS{25}Héber, Péleg, Rehu,
\VS{26}Serug, Nachor, Térach,
\VS{27}et Abram, qui est Abraham.
\VS{28}Les fils d’Abraham furent  Isaac et Ismaël.
\TextTitle{Les fils d’Ismaël\FTNTT{Ge. 25:12-18.}}
\VS{29}Voici leur postérité\FTNT{Ge 25:12-18} : Le premier-né d'Ismaël fut Nebajoth, puis Kédar, Adbeel, Mibsam,
\VS{30}Mischma, Duma, Massa, Hadad, Téma,
\VS{31}Jethur, Naphisch et Kedma; ce sont là les fils d'Ismaël.
\TextTitle{Les fils de Ketura\FTNTT{Ge. 25:1-4.}}
\VS{32}Quant aux fils de Ketura, concubine d'Abraham, elle enfanta Zimran, Jokschan, Medan, Madian, Jischbak et Schuach; et les fils de Jokschan furent Séba et Dedan.
\VS{33}Les fils de Madian furent  Epha, Epher, Hénoc, Abida et Eldaa. Tous ceux-là furent les fils de Ketura.
\TextTitle{Les fils d’Isaac\FTNTT{Ge. 25:19-26.}}
\VS{34}Or Abraham engendra Isaac; et les fils d'Isaac furent  Esaü et Israël.
\TextTitle{Les descendants d’Esaü \FTNTT{Ge. 36:1-14.}}
\VS{35}Les fils d’Esaü furent  Eliphaz, Reuel, Jeusch, Jaelam et Koré\FTNT{Ge. 36:1-14.}.
\VS{36}Les fils d’Eliphaz furent Théman, Omar, Tsephi, Gaetham et Kenaz; Thimna lui enfanta Amalek.
\VS{37}Les fils de Reuel furent Nahath, Zérach, Schamma et Mizza.
\VS{38}Les fils de Séir furent Lothan, Schobal, Tsibeon, Ana, Dischon, Etser et Dischan.
\VS{39}Les fils de Lothan furent  Hori et Homam; et Thimna fut la soeur de Lothan.
\VS{40}Les fils de Schobal furent Aljan, Manahath, Ebal, Schephi et Onam. Les fils de Tsibeon furent Ajja et Ana.
\VS{41}Ana eut un fils : Dischon.  Les fils de Dischon furent Hamran, Eschban, Jithran et Keran.
\VS{42}Les fils d’Etser furent Bilhan, Zaavan et Jaakan. Les fils de Dischon furent Uts et Aran.
\TextTitle{Les rois et les chefs d’Edom\FTNTT{Ge. 36:15-19, 25-43}}
\VS{43}Voici les rois qui ont régné au pays d'Edom, avant qu’un roi ne règne sur les fils d’Israël : Béla, fils de Beor, et le nom de sa ville était Dinhaba.
\VS{44}Béla mourut, et Jobab, fils de Zérach de Botsra, régna à sa place.
\VS{45}Jobab mourut, et Huscham, du pays des Thémanites, régna à sa place.
\VS{46}Huscham mourut, et Hadad, fils de Bedad, régna à sa place. C’est lui qui frappa Madian dans les champs de Moab. Le nom de sa ville était Avith.
\VS{47}Hadad mourut, et Samla de Masréka, régna à sa place.
\VS{48}Samla mourut, et Saül de Rehoboth, sur le fleuve, régna à sa place.
\VS{49}Saül mourut, et Baal-Hanan, fils de Acbor, régna à sa place.
\VS{50}Baal-Hanan mourut, et Hadad régna à sa place. Le nom de sa ville était Pahi, et le nom de sa femme Mehéthabeel, qui était fille de Mathred, et petite- fille de Mézahab.
\VS{51}Enfin Hadad mourut. Ensuite vinrent les chefs d'Edom, le chef Thimna, le chef Alja, le chef Jetheth.
\VS{52}Le chef Oholibama, le chef Ela, le chef Pinon.
\VS{53}Le chef Kenaz, le chef Théman, le chef Mibtsar.
\VS{54}Le chef Magdiel, et le chef Iram. Ce sont là les chefs d'Edom.
\Chap{2}
\TextTitle{Les douze fils de Jacob (Israël)\FTNTT{Ge. 29:31-35 ; 30:6-24 ; 35:16-18}}
\VerseOne{}Voici les fils d'Israël : Ruben, Siméon, Lévi, Juda, Issacar, Zabulon,
\VS{2}Dan, Joseph, Benjamin, Nephthali, Gad et Aser.
\TextTitle{Les descendants de Juda jusqu’aux fils d’Hetsron\FTNTT{Ge. 46:12 ; No 26:19-22}}
\VS{3}Les fils de Juda furent  Er, Onan, et Schéla. Ces trois lui naquirent de la fille de Schua, la Cananéenne. Mais Er, premier-né de Juda, fut méchant aux yeux de Yahweh, qui le fit mourir.
\VS{4}Et Tamar, belle-fille de Juda, lui enfanta Pérets et Zérach. Tous les fils de Juda furent cinq.
\VS{5}Les fils de Pérets furent Hetsron et Hamul.
\VS{6}Et les fils de Zérach furent Zimri, Ethan, Héman, Calcol et Dara, cinq en tout.
\VS{7}Carmi n'eut point d’autre fils qu'Acar qui troubla Israël et qui pécha en prenant de l'interdit.
\VS{8}Ethan eut un seul fils : Azaria.
\VS{9}Les fils qui naquirent à Hetsron furent Jerachmeel, Ram et Kelubaï.
\TextTitle{Les descendants de Ram jusqu’à David\FTNTT{Ru. 4:17-22}}
\VS{10}Ram engendra Amminadab et Amminadab engendra Nachschon, chef des fils de Juda.
\VS{11}Nachschon engendra Salma et Salma engendra Boaz.
\VS{12}Boaz engendra Obed et Obed engendra Isaï.
\VS{13}Isaï engendra son premier-né Eliab, le second Abinadab, le troisième Schimea,
\VS{14}le quatrième Nethaneel, le cinquième Raddaï,
\VS{15}le sixième Otsem, et le septième, David.
\VS{16}Tseruja et Abigaïl furent leurs soeurs. Tseruja eut trois fils : Abischaï, Joab, et Asaël.
\VS{17}Abigaïl enfanta Amasa, dont le père fut Jéther l’Ismaélite.
\TextTitle{Les descendants de Caleb}
\VS{18}Or Caleb, fils de Hetsron, eut des enfants d’Azuba sa femme, et aussi de Jerioth; et ses fils furent Jéscher, Schobab et Ardon.
\VS{19}Azuba mourut, et Caleb prit pour femme Ephrath, qui lui enfanta Hur.
\VS{20}Hur engendra Uri, et Uri engendra Betsaleel.
\VS{21}Après cela, Hetsron vint vers la fille de Makir, père de Galaad, et la prit pour sa femme ; il était âgé de soixante ans, et elle lui enfanta Segub.
\VS{22}Segub engendra Jaïr, qui eut vingt-trois villes au pays de Galaad.
\VS{23}Il prit sur Gueschur et sur la Syrie les bourgades de Jaïr, et Kenath, avec les villes de son ressort, au nombre de soixante. Tous ceux-là furent fils de Makir, père de Galaad.
\VS{24}Après la mort de Hetsron, à Caleb-Ephratha, la femme de Hetsron, Abija, lui enfanta Aschchur, père de Tekoa.
\VS{25}Les fils de Jerachmeel, premier-né de Hetsron furent : Ram, son fils aîné, puis Buna, Oren et Otsem, nés d'Achija.
\VS{26}Jerachmeel eut aussi une autre femme, dont le nom était Athara, qui fut mère d'Onam.
\VS{27}Les fils de Ram, premier-né de Jerachmeel, furent Maats, Jamin et Eker.
\VS{28}Les fils  d'Onam furent  Schammaï et Jada; et les fils de Schammaï furent  Nadab et Abischur.
\VS{29}Le nom de la femme d'Abischur fut Abichaïl, qui lui enfanta Achban et Molid.
\VS{30}Les fils de Nadab furent Séled et Appaïm; mais Séled mourut sans fils.
\VS{31}Appaïm eut un seul fils : Jischeï. Jischeï eut un seul fils : Schéschan. Schéschan n'eut qu' Achlaï.
\VS{32}Les fils de Jada, frère de Schammaï, furent Jéther et Jonathan ; mais Jéther mourut sans fils.
\VS{33}Les fils de Jonathan furent Péleth et Zara ; ce furent là les fils de Jerachmeel.
\VS{34}Schéschan n'eut point de fils, mais des filles ; or il avait un serviteur Egyptien, dont le nom était Jarcha ;
\VS{35}Schéschan donna sa fille pour femme à Jarcha, son serviteur, et elle lui enfanta Attaï.
\VS{36}Attaï engendra Nathan, et Nathan engendra Zabad ;
\VS{37}Zabad engendra Ephlal; et Ephlal engendra Obed ;
\VS{38} Obed engendra Jéhu; Jéhu engendra Azaria;
\VS{39}Azaria engendra Halets;  Halets engendra Elasa;
\VS{40}Elasa engendra Sismaï; Sismaï engendra Schallum;
\VS{41}Schallum engendra Jekamja; Jekamja engendra Elischama.
\TextTitle{Les autres fils de Caleb}
\VS{42}Les fils de Caleb, frère de Jerachmeel, furent Méscha, son premier-né, qui fut le père de Ziph, et les fils de Maréscha, père d'Hébron.
\VS{43}Les fils d'Hébron furent  Koré, Thappuach, Rékem et Schéma.
\VS{44}Schéma engendra Racham, père de Jorkeam, et Rékem engendra Schammaï.
\VS{45}Le fils de Schammaï fut Maon. Maon fut père de Beth-Tsur.
\VS{46}Et Epha, concubine de Caleb, enfanta Haran, Motsa et Gazez; Haran aussi engendra Gazez.
\VS{47}Les fils de Jahdaï furent Réguem, Jotham, Guéschan, Péleth, Epha et Schaaph.
\VS{48}Maaca, la concubine de Caleb, enfanta Schéber et Tirchana.
\VS{49}La femme de Schaaph, père de Madmanna, enfanta Scheva, père de Macbéna, et le père de Guibea, et la fille de Caleb fut Acsa.
\TextTitle{Les descendants de Hur, fils de Caleb\FTNTT{v. 19 ; cp. 1 Ch. 4:1}}
\VS{50}Ceux-ci furent les fils de Caleb, fils de Hur, premier-né d'Ephrata : Schobal, père de Kirjath-Jearim.
\VS{51}Salma, père de Bethléhem, Hareph, père de Beth-Gader.
\VS{52}Schobal, père de Kirjath-Jearim, eut des fils : Haroé et Hatsi-Hammenuhoth.
\VS{53}Les familles de Kirjath-Jearim furent les Jéthriens, les Puthiens, les Schumathiens et les Mischraïens, desquels sont sortis les Tsoreathiens et les Eschthaoliens.
\VS{54}Les fils de Salma : Bethléhem et les Nethophatiens, Athroth-Beth-Joab, Hatsi-Hammanachthi et les Tsoreïens.
\VS{55}Et les familles des scribes, qui habitaient à Jaebets : Les Thireathiens, les Schimeathiens, les Sucathiens ; ce sont les Kéniens, qui sont sortis de Hamath père de Récab.
\Chap{3}
\TextTitle{Les fils de David\FTNTT{2 S. 3:2-5 ; 5:13-16}}
\VerseOne{}Voici les fils de David, qui lui naquirent à Hébron\FTNT{2 S. 3:2-5.}. Le premier-né fut Amnon, fils d' Achinoam de Jizreel; le second Daniel, d'Abigaïl de Carmel.
\VS{2}Le troisième, Absalom, fils de Maaca, fille de Talmaï, roi de Gueschur ; le quatrième, Adonija, fils de Haggith ;
\VS{3}le cinquième, Schephatia, d'Abithal; le sixième, Jithream, d'Egla sa femme.
\VS{4}Ces six lui naquirent à Hébron, où il régna sept ans et six mois ; puis il régna trente-trois ans à Jérusalem.
\VS{5}Ceux-ci lui naquirent à Jérusalem : Schimea, Schobab, Nathan et Salomon, tous quatre de Bath-Schua, fille d'Ammiel ;
\VS{6}et Jibhar, Elischama, Eliphéleth,
\VS{7}Noga, Népheg, Japhia,
\VS{8}Elischama, Eliada et Eliphéleth, qui sont neuf.
\VS{9}Ce sont tous des fils de David, outre les fils de ses concubines. Et Tamar était leur sœur.
\TextTitle{De Salomon à Sédécias}
\VS{10}Le fils de Salomon fut  Roboam. Abija, son fils ; Asa, son fils ; Josaphat, son fils ;
\VS{11}Joram, son fils ; Achazia, son fils ; Joas, son fils ;
\VS{12}Amatsia, son fils ; Azaria, son fils ; Jotham, son fils ;
\VS{13}Achaz, son fils ; Ezéchias, son fils ; Manassé, son fils ;
\VS{14}Amon, son fils ; Josias, son fils.
\VS{15}Les fils de Josias furent Jochanan, son premier-né ; le deuxième, Jojakim ; le troisième Sédécias ; le quatrième, Schallum.
\VS{16}Les fils de Jojakim furent  Jéconias, son fils, qui eut pour fils Sédécias.
\TextTitle{Les fils de Jéconias}
\VS{17}Quant aux fils de Jéconias, Assir qui fut emmené en captivité, Schealthiel fut son fils ;
\VS{18}dont les fils furent Malkiram, Pedaja, Schénatsar, Jekamia, Hoschama et Nedabia.
\VS{19}Les fils de Pedaja furent Zorobabel et Schimeï ; et les fils de Zorobabel furent Meschullam et Hanania ; et Schelomith était leur soeur.
\VS{20}De Meschullam, Haschuba, Ohel, Bérékia, Hasadia et Juschab-Hésed, en tout cinq.
\VS{21}Les fils de Hanania furent  Pelathia et Esaïe ; les fils de Rephaja, les fils d'Arnan, les fils d’Abdias et les fils de Schecania.
\VS{22}De Schecania naquit Schemaeja ; et les fils de Schemaeja, Hattusch, Jigueal, Bariach, Nearia, Schaphath, en tout six.
\VS{23}Les fils de Nearia furent trois : Eljoénaï, Ezéchias et Azrikam.
\VS{24}Et les fils d' Eljoénaï furent  sept : Hodavia, Eliaschib, Pelaja, Akkub, Jochanan, Delaja, et Anani.
\Chap{4}
\TextTitle{Les autres fils de Hur\FTNTT{1 Ch. 2:50}}
\VerseOne{}Les fils de Juda furent Pérets, Hetsron, Carmi, Hur et Schobal.
\VS{2}Reaja, fils de Schobal, engendra Jachath ; et Jachath engendra Achumaï et Lahad. Ce sont les familles des Tsoreathiens.
\VS{3}Voici les descendants du père d’Etham : Jizreel, Jischma, et Jidbasch ; le nom de leur soeur était Hatselelponi.
\VS{4}Penuel, père de Guedor, et Ezer, père de Huscha, sont les fils de Hur, premier-né d'Ephrata, père de Bethléhem.
\TextTitle{Les descendants d’Aschchur\FTNTT{1 Ch. 2:24}}
\VS{5}Aschchur, père de Tekoa, eut deux femmes : Hélea et Naara.
\VS{6}Naara lui enfanta Achuzzam, Hépher, Thémeni et Achaschthari. Ce sont là les fils de Naara.
\VS{7}Les fils de Hélea furent Tséreth, Tsochar et Ethnan.
\VS{8}Kots engendra Anub, Hatsobéba et les familles Acharchel, fils de Harum.
\TextTitle{Jaebets invoque Dieu}
\VS{9}Jaebets était plus honoré que ses frères ; sa mère lui avait donné le nom de Jaebets, parce que, dit-elle, je l'ai enfanté avec douleur.
\VS{10}Jaebets invoqua le Dieu d'Israël, en disant : Ô, si tu me bénis abondamment et que tu étends mes limites, si ta main est avec moi, et si tu me mets à l'abri du mal, en sorte que je ne sois pas dans l’affliction !... Et Dieu lui accorda ce qu'il avait demandé.
\TextTitle{Les fils de Juda et de Caleb}
\VS{11}Kelub, frère de Schucha, engendra Mechir, qui fut père d'Eschthon.
\VS{12}Et Eschthon engendra la maison de Rapha, Paséach et Thechinna, père de la ville de Nachasch ; ce sont là les gens de Réca.
\VS{13}Les fils de Kenaz furent Othniel et Seraja. Et le fils d’Othniel, Hathath.
\VS{14}Meonothaï engendra Ophra ; et Seraja engendra Joab, père de la vallée des ouvriers ; car ils étaient ouvriers.
\VS{15}Les fils de Caleb, fils de Jephunné, furent Iru, Ela et Naam, et les fils d'Ela, Kenaz.
\VS{16}Les fils de Jehalléleel furent Ziph, Zipha, Thirja, et Asareel.
\VS{17}Les fils d'Esdras furent Jéther, Méred, Epher, et Jalon ; et la femme de Méred enfanta Miriam, Schammaï, et Jischbach, père d' Eschthemoa.
\VS{18}Sa femme, la Juive, enfanta Jéred, père de Guedor ;  Héber, père de Soco ;  Jekuthiel, père de Zanoach. Ceux-là sont les fils de Bithja, fille de Pharaon, que Méred prit pour femme.
\VS{19}Les fils de la femme de Hodija, soeur de Nacham : Le père de Kehila, le Garmien, et Eschthemoa, le Maacathien.
\VS{20}Et les fils de Simon furent Amnon, Rinna, Ben-Hanan et Thilon. Les fils de Jischeï furent Zocheth et Ben-Zocheth.
\TextTitle{Les fils de Juda par Schéla\FTNTT{1 Ch. 2:3}}
\VS{21}Les fils de Schéla, fils de Juda, furent Er, père de Léca ; Laeda, père de Maréscha ; et les familles de la maison où l'on travaille le byssus, qui sont de la maison d'Aschbéa.
\VS{22}Jokim, et les gens de Cozéba, Joas et Saraph dominèrent sur Moab, avec Jaschubi-Léchem. Mais ce sont là des choses anciennes.
\VS{23}C’étaient les potiers et les habitants des plantations  et des parcs. Ils cohabitaient là chez le roi et oeuvraient pour lui.
\TextTitle{Les descendants de Siméon ; leurs terres et leurs conquêtes}
\VS{24}Les fils de Siméon furent Nemuel, Jamin, Jarib, Zérach et Saül.
\VS{25}Schallum son fils, Mibsam son fils, et Mischma son fils.
\VS{26}Les fils de Mischma furent Hammuel son fils, Zaccur son fils, et Schimeï son fils.
\VS{27}Schimeï eut seize fils et six filles ; mais ses frères n'eurent pas beaucoup de fils, et toute leur famille ne put être aussi nombreuse que celle des fils de Juda.
\VS{28}Ils habitèrent à Beer-Schéba, à Molada, à Hatsar-Schual,
\VS{29}à Bilha, à Etsem, à Tholad,
\VS{30}à Bethuel, à Horma, à Tsiklag,
\VS{31}à Beth-Marcaboth, à Hatsar-Susim, à Beth-Bireï, et à Schaaraïm. Ce furent là leurs villes jusqu'au temps où David devint roi.
\VS{32}Leurs villages furent Etham, Aïn, Rimmon, Thoken, et Aschan, cinq villes ;
\VS{33}et tous leurs villages, qui étaient autour de ces villes-là, jusqu'à Baal. Ce sont là leurs habitations et leur généalogie :
\VS{34}Meschobab, Jamlec, Joscha fils d'Amatsia ;
\VS{35}Joël, Jéhu fils de Joschibia, fils de Seraja, fils d'Asiel ;
\VS{36}Eljoénaï, Jaakoba, Jeschochaja, Asaja, Adiel, Jesimiel, Benaja,
\VS{37}Ziza, fils de Schipheï, fils d'Allon, fils de Jedaja, fils de Schimri, fils de Schemaeja.
\VS{38}Ceux-là furent désignés pour être des chefs dans leurs familles, et les maisons de leurs pères s’étendirent abondamment.
\VS{39}Et ils allèrent pour entrer dans Guedor, jusqu'à l'orient de la vallée, cherchant des pâturages pour leurs troupeaux.
\VS{40}Ils trouvèrent des pâturages gras et bons, et un pays spacieux, paisible et fertile ; car ceux qui habitaient là auparavant étaient descendus de Cham.
\VS{41}Ceux-ci, dont les noms sont inscrits, vinrent du temps d'Ezéchias, roi de Juda, et abattirent leurs tentes ; et quant aux Maonites qui s’y trouvaient, et les détruisirent à la façon de l'interdit jusqu'à ce jour, et y habitèrent à leur place, car il y avait là des pâturages pour leurs troupeaux.
\VS{42}Cinq cents hommes d'entre eux, c'est-à-dire des fils de Siméon, s'en allèrent à la montagne de Séir, et ils avaient à leur tête Pelathia, Nearia, Rephaja, et Uziel, fils de Jischeï ;
\VS{43}ils frappèrent le reste des réchappés d'Amalek, et ils demeurèrent là jusqu'à ce jour.
\Chap{5}
\TextTitle{Les descendants de Ruben jusqu’au temps des captivités}
\VerseOne{}Les fils de Ruben, le premier-né d'Israël. - Car il était le premier-né ; mais après qu'il eut souillé le lit de son père, son droit d'aînesse fut donné aux fils de Joseph fils d'Israël ; cependant, Joseph ne fut pas enregistré  dans la généalogie selon le droit d'aînesse.
\VS{2}Car Juda fut le plus puissant parmi ses frères, et de lui est issu un chef ; mais le droit d'aînesse est à Joseph.
\VS{3}Les fils de Ruben, premier-né d'Israël, furent donc Hénoc, Pallu, Hetsron, et Carmi.-
\VS{4}Les fils de Joël furent  Schemaeja, son fils ; Gog, son fils ; Schimeï, son fils ;
\VS{5}Michée, son fils ; Reaja, son fils ; Baal,  son fils ;
\VS{6}Beéra, son fils, qui fut emmené captif par Tilgath- Pilnéser, roi d’Assyrie ; c'est lui qui était le principal chef des Rubénites.
\VS{7}Ses frères, selon leurs familles, d’après le registre généalogique et selon leurs générations, avaient pour chefs Jeïel et Zacharie.
\VS{8}Béla, fils d’Azaz, fils de Schéma, fils de Joël, habitait depuis Aroër jusqu'à Nebo et Baal-Meon.
\VS{9}Ensuite, il habita du côté de l’orient jusqu'à l'entrée du désert, depuis le fleuve d'Euphrate; car son bétail s'était multiplié dans le pays de Galaad.
\VS{10}Du temps de Saül, ils firent la guerre contre les Hagaréniens, qui tombèrent par leurs mains, et ils habitèrent dans leurs tentes, dans toute la partie orientale de Galaad.
\TextTitle{Les descendants de Gad et leurs villes}
\VS{11}Les fils de Gad habitaient près d'eux, au pays de Basan, jusqu'à Salca.
\VS{12}Joël fut le premier chef, et Schapham le deuxième après lui, puis Jaenaï, puis Schaphath en Basan.
\VS{13}Et leurs frères, selon la maison de leurs pères, furent sept : Micaël, Meschullam, Schéba, Joraï, Jaecan, Zia, et Eber.
\VS{14}Ceux-ci furent les fils d'Abichaïl, fils de Huri, fils de Jaroach, fils de Galaad, fils de Micaël, fils de Jeschischaï, fils de Jachdo, fils de Buz.
\VS{15}Achi, fils d'Abdiel, fils de Guni, fut le chef de la maison de leurs pères.
\VS{16}Ils habitèrent en Galaad, et en Basan, dans les villes de son ressort, et dans toutes les banlieues de Saron, jusqu’à leurs limites.
\VS{17}Tous ceux-ci furent inscrits dans la généalogie du temps de Jotham, roi de Juda, et du temps de Jéroboam, roi d'Israël.
\TextTitle{Captivité de Ruben, Gad et la demi-tribu de Manassé}
\VS{18}Il y eut des fils de Ruben, et de ceux de Gad, et de la demi-tribu de Manassé, d'entre les vaillants hommes, portant le bouclier et l'épée, tirant de l'arc, et exercés à la guerre, quarante-quatre mille sept cent soixante, en état d’aller à l’armée.
\VS{19}Ils firent la guerre contre les Hagaréniens, contre Jethur, Naphisch, et Nodab.
\VS{20}Et ils reçurent du secours contre eux, de sorte que les Hagaréniens, et tous ceux qui étaient avec eux furent livrés entre leurs mains, parce qu'ils crièrent à Dieu dans la bataille, et il les exauça parce qu'ils avaient mis leur confiance en lui.
\VS{21}Ainsi ils prirent leurs troupeaux, consistant en cinquante mille chameaux, deux cent cinquante mille brebis, deux mille ânes, avec cent mille personnes ;
\VS{22}car il y eut beaucoup de morts, parce que la bataille venait de Dieu.  Ils habitèrent là, à leur place, jusqu'au temps de la déportation.
\VS{23}Les fils de la demi-tribu de Manassé habitèrent aussi dans ce pays-là, et s'étendirent depuis Basan jusqu'à Baal-Hermon et à Sénir, à la montagne d’Hermon ; ils étaient nombreux.
\VS{24}Et voici les chefs de la maison de leurs pères : Epher, Jischeï, Eliel, Azriel, Jérémie, Hodavia, et Jachdiel, hommes forts et vaillants, gens de réputation, et chefs des maisons de leurs pères.
\VS{25}Mais ils péchèrent contre le Dieu de leurs pères, et se prostituèrent après les dieux des peuples du pays, que Dieu avait détruits devant eux.
\VS{26}Le Dieu d'Israël excita l'esprit de Pul, roi d’Assyrie, et l'esprit de Thilgath-Pilnéser, roi d’Assyrie,  qui emmena en captivité les Rubénites, les Gadites et la demi-tribu de Manassé, et les emmena à Chalach, à Chabor, à Hara, et au fleuve de Gozan, où ils sont restés jusqu'à ce jour.
\Chap{6}
\TextTitle{Les fils de Kehath le Lévite, jusqu'à la captivité}
\VerseOne{}Les fils de Lévi furent Guerschon, Kehath et Merari.
\VS{2}Les fils de Kehath furent  Amram, Jitsehar, Hébron, et Uziel.
\VS{3}Et les fils d'Amram furent Aaron, Moïse et Marie. Les fils d'Aaron furent  Nadab, Abihu, Eléazar et Ithamar.
\VS{4}Eléazar engendra Phinées, et Phinées engendra Abischua.
\VS{5}Abischua engendra Bukki, et Bukki engendra Uzzi.
\VS{6}Uzzi engendra Zerachja, et Zerachja engendra Merajoth.
\VS{7}Merajoth engendra Amaria, et Amaria engendra Achithub.
\VS{8}Achithub engendra Tsadok, et Tsadok engendra Achimaats.
\VS{9}Achimaats engendra Azaria, et Azaria engendra Jochanan.
\VS{10}Jochanan engendra Azaria, qui exerça la sacrificature au temple que Salomon bâtit à Jérusalem.
\VS{11}Azaria engendra Amaria, et Amaria engendra Achithub.
\VS{12}Achithub engendra Tsadok, et Tsadok engendra Schallum.
\VS{13}Schallum engendra Hilkija, et Hilkija engendra Azaria.
\VS{14}Azaria engendra Seraja, et Seraja engendra Jehotsadak,
\VS{15}Jehotsadak s'en alla, quand Yahweh emmena en exil Juda et Jérusalem par le moyen de Nebucadnetsar.
\TextTitle{Les fils de Guerschon, Kehath et Mérari}
\VS{16}Les fils de Lévi  furent donc  Guerschon, Kehath et Merari.
\VS{17}Voici les noms des fils de Guerschon : Libni et Schimeï.
\VS{18}Les fils de Kehath furent Amram, Jitsehar, Hébron et Uziel.
\VS{19}Les fils de Merari furent  Machli et Muschi. Ce sont là les familles des Lévites, selon les maisons de leurs pères.
\VS{20}De Guerschon, Libni son fils, Jachath son fils, Zimma son fils,
\VS{21}Joach son fils, Iddo son fils, Zérach son fils, Jeathraï son fils.
\VS{22}Des fils de Kehath, Amminadab son fils, Koré son fils, Assir son fils,
\VS{23}Elkana son fils, Ebjasaph son fils, Assir son fils,
\VS{24}Thachath son fils, Uriel son fils, Ozias son fils, et Saül son fils.
\VS{25}Les fils d’Elkana furent  Amasaï, Achimoth ;
\VS{26}Elkana, son fils ; les fils d’Elkana furent Elkana-Tsophaï, son fils, Nachath son fils,
\VS{27}Eliab son fils, Jerocham son fils, Elkana son fils.
\VS{28}Quant aux fils de Samuel, fils d'Elkana, son fils aîné fut Vaschni, puis Abija.
\VS{29}Les fils de Merari furent Machli, Libni son fils, Schimeï son fils, Uzza son fils,
\VS{30}Schimea son fils, Hagguija son fils, Asaja son fils.
\TextTitle{Les chefs des chantres}
\VS{31}Or voici ceux que David établit pour la direction de la musique dans la maison de Yahweh, depuis que l’arche fut en lieu de repos.
\VS{32}Ils faisaient le service comme chantres devant le tabernacle, devant la tente d'assignation, jusqu'à ce que Salomon eût bâti la maison de Yahweh à Jérusalem ; ils continuèrent dans leur service selon l'ordonnance qui était prescrite. Voici ceux qui firent le service avec leurs fils : D'entre les fils des Kehathites, Héman le chantre, fils de Joël, fils de Samuel,
\VS{34}fils d'Elkana, fils de Jerocham, fils d’Eliel, fils de Thoach,
\VS{35}fils de Tsuph, fils d'Elkana, fils de Machath, fils de Amasaï,
\VS{36}fils d'Elkana, fils de Joël, fils d’Azaria, fils de Sophonie,
\VS{37}fils de Thachath, fils d'Assir, fils de Ebjasaph, fils de Koré,
\VS{38}fils de Jitsehar, fils de Kehath, fils de Lévi, fils d'Israël.
\VS{39}Son frère Asaph, qui se tenait à sa droite. Asaph était fils de Bérékia, fils de Schimea,
\VS{40}fils de Micaël, fils de Baaséja, fils de Malkija,
\VS{41}fils d’Ethni, fils de Zérach, fils d’Adaja,
\VS{42}fils d'Ethan, fils de Zimma, fils de Schimeï,
\VS{43}fils de Jachath, fils de Guerschon, fils de Lévi.
\VS{44}Les fils de Merari, leurs frères étaient à la gauche ; à savoir Ethan, fils de Kischi, fils d’Abdi, fils de Malluc,
\VS{45}fils de Haschabia, fils d'Amatsia, fils de Hilkija,
\VS{46}fils d'Amtsi, fils de Bani, fils de Schémer,
\VS{47}fils de Machli, fils de Muschi, fils de Merari, fils de Lévi.
\VS{48}Et leurs autres frères Lévites furent ordonnés pour tout le service du tabernacle de la maison de Dieu.
\VS{49}Mais Aaron et ses fils offraient les parfums sur l'autel de l'holocauste et sur l'autel des parfums ; pour tout ce qu'il fallait faire dans le Saint des saints, et pour faire propitiation pour Israël ; comme Moïse, serviteur de Dieu, l'avait commandé.
\TextTitle{Les sacrificateurs d’Aaron à Achimaats}
\VS{50}Voici les fils d'Aaron : Eléazar son fils, Phinées son fils, Abischua son fils,
\VS{51}Bukki son fils, Uzzi son fils, Zerachja son fils,
\VS{52}Merajoth son fils, Amaria son fils, Achithub son fils,
\VS{53}Tsadok son fils, Achimaats son fils.
\TextTitle{Villes des fils d’Aaron et des Lévites}
\VS{54}Voici leurs lieux d’habitation, selon leurs demeures et leurs limites. Aux fils d'Aaron, qui appartiennent à la famille des Kehathites, désignés par le sort,
\VS{55}on leur donna Hébron dans le pays de Juda, et sa banlieue tout autour.
\VS{56}Mais on donna à Caleb, fils de Jephunné, le territoire de la ville et ses villages.
\VS{57}On donna donc aux fils d'Aaron, d'entre les villes de refuge, Hébron, Libna et sa banlieue, Jatthir et Eschthemoa, avec leurs banlieues,
\VS{58}Hilen, avec sa banlieue, Debir avec sa banlieue,
\VS{59}Aschan avec sa banlieue, et Beth-Schémesch avec sa banlieue.
\VS{60}De la tribu de Benjamin, Guéba, avec sa banlieue, Allémeth avec sa banlieue, et Anathoth avec sa banlieue. Toutes leurs villes, selon leurs familles, étaient treize en nombre.
\VS{61}On donna au reste des fils de Kehath, par le sort, dix villes des familles de la demi-tribu, c'est-à-dire de la demi-tribu de Manassé.
\VS{62}Et aux fils de Guerschon, selon leurs familles, de la tribu d'Issacar, de la tribu d'Aser, de la tribu de Nephthali, et de la tribu de Manassé en Basan, treize villes.
\VS{63}Aux fils de Merari, selon leurs familles, par le sort, douze villes, de la tribu de Ruben, de la tribu de Gad, et de la tribu de Zabulon.
\VS{64}Ainsi, les fils d’Israël donnèrent aux Lévites ces villes-là, avec leurs banlieues.
\VS{65}Et ils donnèrent, par le sort, de la tribu des fils de Juda, de la tribu des fils de Siméon, et de la tribu des fils de Benjamin, ces villes qu’ils désignèrent par leurs noms.
\VS{66}Et pour les autres familles des fils de Kehath, ils eurent pour territoire des villes de la tribu d'Ephraïm.
\VS{67}Car on leur donna entre les villes de refuge, Sichem avec sa banlieue, dans la montagne d'Ephraïm, Guézer avec sa banlieue,
\VS{68}Jokmeam avec sa banlieue, Beth-Horon avec sa banlieue,
\VS{69}Ajalon avec sa banlieue, et Gath-Rimmon avec sa banlieue.
\VS{70}De la demi-tribu de Manassé, Aner avec sa banlieue, et Bileam avec sa banlieue, on donna ces villes-là aux familles qui restaient des fils de Kehath.
\VS{71}Aux fils de Guerschon, on donna, des familles de la demi-tribu de Manassé, Golan en Basan avec sa banlieue, et Aschtaroth, avec sa banlieue.
\VS{72}De la tribu d'Issacar, Kédesch avec sa banlieue, Dobrath avec sa banlieue,
\VS{73}Ramoth avec sa banlieue, et Anem avec sa banlieue.
\VS{74}Et de la tribu d'Aser, Maschal, avec sa banlieue, Abdon, avec sa banlieue,
\VS{75}Hukok avec sa banlieue, et Rehob avec sa banlieue.
\VS{76}De la tribu de Nephthali, Kédesch en Galilée avec sa banlieue, Hammon avec sa banlieue, et Kirjathaïm avec sa banlieue.
\VS{77}Aux fils de Merari, qui étaient le reste d'entre les Lévites, on donna, de la tribu de Zabulon, Rimmono avec sa banlieue, et Thabor avec sa banlieue.
\VS{78}Au-delà du Jourdain, vis-à-vis de Jéricho, vers l’orient du Jourdain, de la tribu de Ruben, Betser au désert avec sa banlieue, Jahtsa avec sa banlieue,
\VS{79}Kedémoth avec sa banlieue, et Méphaath avec sa banlieue.
\VS{80}De la tribu de Gad, Ramoth en Galaad avec sa banlieue, Mahanaïm avec sa banlieue,
\VS{81}Hesbon avec sa banlieue, et Jaezer avec sa banlieue.
\Chap{7}
\TextTitle{Les descendants d'Issacar}
\VerseOne{}Les fils d'Issacar furent  Thola, Pua, Jaschub et Schimron, quatre.
\VS{2}Les fils de Thola furent Uzzi, Rephaja, Jeriel, Jachmaï, Jibsam et Samuel, chefs des maisons de leurs pères qui étaient de Thola, gens forts et vaillants dans leurs générations ; leur nombre, aux jours de David, était de vingt-deux mille six cents.
\VS{3}Le fils d’Uzzi : Jizrachja. Et les fils de Jizrachja : Micaël, Abdias, Joël, et Jischija, en tout cinq chefs.
\VS{4}Ils avaient avec eux, selon leurs générations, et selon les familles de leurs pères, trente-six mille hommes de troupe, armés pour la guerre, car ils eurent plusieurs femmes et plusieurs fils.
\VS{5}Leurs frères selon toutes les familles d'Issacar, hommes forts et vaillants, étant comptés tous selon leur généalogie, furent quatre-vingt-sept mille.
\TextTitle{Les descendants de Benjamin}
\VS{6}Les fils de Benjamin furent Béla, Béker et Jediaël, trois\FTNT{Benjamin avait encore d’autres fils (Ge. 46:21 ; No. 26:38-41 ; 1 Ch. 8:1-2).}.
\VS{7}Les fils de Béla furent  Etsbon, Uzzi, Uziel, Jerimoth et Iri, cinq chefs des familles de leurs pères, hommes forts et vaillants, et enregistrés dans la généalogie au nombre de vingt-deux mille trente-quatre.
\VS{8}Les fils de Béker furent  Zemira, Joasch, Eliézer, Eljoénaï, Omri, Jerémoth, Abija, Anathoth, et Alameth, tous ceux-là furent fils de Béker,
\VS{9}et enregistrés dans les généalogies, selon leurs générations, comme chefs des familles de leurs pères, hommes forts et vaillants au nombre de vingt mille deux cents.
\VS{10}Jediaël eut pour fils Bilhan. Et les fils de Bilhan furent Jeusch, Benjamin, Ehud, Kenaana, Zéthan, Tarsis, et Achischachar.
\VS{11}Tous ceux-là furent fils de Jediaël, comme chefs des familles de leurs pères, dix-sept mille deux cents hommes forts et vaillants, en état de porter les armes et d’aller à la guerre.
\VS{12}Schuppim et Huppim furent  des fils d’Ir ; et Huschim fut fils d'Acher.
\TextTitle{Les descendants de Nephtali}
\VS{13}Les fils de Nephthali furent Jahtsiel, Guni, Jetser, et Schallum, fils de Bilha.
\TextTitle{Les descendants de Manassé}
\VS{14}Les fils de Manassé : Asriel, qu’enfanta sa concubine Araméenne. Elle enfanta Makir, père de Galaad.
\VS{15}Makir prit une femme de la parenté de Huppim et de Schuppim ; car ils avaient une sœur dont le nom était Maaca. Et le nom d'un des petits-fils de Galaad fut Tselophchad ; et Tselophchad eut des filles.
\VS{16}Maaca, femme de Makir, enfanta un fils et l'appela Péresch, et le nom de son frère Schéresch, dont les fils furent Ulam et Rékem.
\VS{17}Le fils d'Ulam fut Bedan. Ce sont là les fils de Galaad, fils de Makir, fils de Manassé.
\VS{18}Mais sa soeur Hammoléketh enfanta Ischhod, Abiézer et Machla.
\VS{19}Les fils de Schemida furent Achjan, Sichem, Likchi et Aniam.
\TextTitle{Les descendants d'Ephraïm et leurs villes}
\VS{20}Or les fils d'Ephraïm furent  Schutélach ; Béred son fils, Tachath son fils, Eleada son fils, Tachath son fils.
\VS{21}Zabad son fils, Schutélach son fils, Ezer, et Elead. Mais ceux de Gath, nés dans le pays, les mirent à mort, parce qu'ils étaient descendus pour prendre leur bétail.
\VS{22}Ephraïm, leur père, fut dans le deuil plusieurs jours, et ses frères vinrent pour le consoler.
\VS{23}Puis il alla vers sa femme, qui conçut et enfanta un fils ; et elle l'appela du nom de Beria, parce que le malheur était dans sa maison.
\VS{24}Il eut pour fille Schééra, qui bâtit la basse et la haute Beth-Horon, et Uzzen-Schééra.
\VS{25}Son fils fut  Réphach, puis Réscheph, et Thélach son fils, Thachan son fils,
\VS{26}Laedan son fils, Ammihud son fils, Elischama son fils,
\VS{27}Nun son fils, Josué son fils.
\VS{28}Ils possédaient et habitaient Béthel ainsi que les villes de son ressort ; à l’orient Naaran, à l’occident Guézer, avec les villes de son ressort, et Sichem avec les villes de son ressort, jusqu'à Gaza avec les villes de son ressort.
\VS{29}Les lieux qui étaient aux fils de Manassé furent  Beth-Schean avec les villes de son ressort, Thaanac avec les villes de son ressort, Meguiddo avec les villes de son ressort, et Dor avec les villes de son ressort. Les fils de Joseph, fils d'Israël, habitèrent dans ces villes.
\TextTitle{Les descendants d'Aser}
\VS{30}Les fils d’Aser furent Jimna, Jischva, Jischvi, Beria, et Sérach leur soeur.
\VS{31}Les fils de Beria furent Héber et Malkiel, qui fut père de Birzavith.
\VS{32}Héber engendra Japhleth, Schomer, Hotham, et Schua leur soeur.
\VS{33}Les fils de Japhleth furent Pasac, Bimhal, et Aschvath. Ce sont là les fils de Japhlet.
\VS{34}Et les fils de Schamer furent Achi, Rohega, Hubba et Aram.
\VS{35}Les fils d'Hélem, son frère, furent  Tsophach, Jimna, Schélesch et Amal.
\VS{36}Les fils de Tsophach furent Suach, Harnépher, Schual, Béri, Jimra,
\VS{37}Betser, Hod, Schamma, Schilscha, Jithran, et Beéra.
\VS{38}Les fils de Jéther furent Jephunné, Pispa et Ara.
\VS{39}Les fils d'Ulla furent Arach, Hanniel et Ritsja.
\VS{40}Tous ceux-là furent fils d'Aser, chefs des maisons de leurs pères, gens d'élite, forts et vaillants, chefs des princes, enregistrés au nombre de vingt-six mille hommes, en état de porter les armes et d’aller en guerre.
\Chap{8}
\TextTitle{Les descendants de Benjamin}
\VerseOne{}Benjamin engendra Béla, qui fut son premier-né, Aschbel le deuxième, Achrach le troisième,
\VS{2}Nocha le quatrième, et Rapha le cinquième.
\VS{3}Les fils de Béla furent Addar, Guéra, Abihud,
\VS{4}Abischua, Naaman, Achoach,
\VS{5}Guéra, Schephuphan et Huram.
\VS{6}Voici les fils d'Echud, qui étaient chefs des maisons des pères des habitants de Guéba, et qui les transportèrent à Manachath :
\VS{7}Naaman, Achija, et Guéra. Guéra, qui les transporta et qui après engendra Uzza et Achichud.
\VS{8}Or Schacharaïm eut des enfants au pays de Moab, après avoir renvoyé Huschim et Baara, ses femmes.
\VS{9}Il engendra, de Hodesch sa femme, Jobab, Tsibja, Méscha, Malcam,
\VS{10}Jeuts, Schocja et Mirma. Ce sont là ses fils, chefs des pères.
\VS{11}Mais de Huschim, il engendra Abithub, Elpaal.
\VS{12}Les fils d'Elpaal furent Eber, Mischeam, et Schémer, qui bâtit Ono, Lod et les villes de son ressort.
\VS{13}Et Beria et Schéma furent chefs des pères des habitants d'Ajalon ; ils mirent en fuite les habitants de Gath.
\VS{14}Achjo, Schaschak, Jerémoth,
\VS{15}Zebadja, Arad, Eder,
\VS{16}Micaël, Jischpha, et Jocha, fils de Beria.
\VS{17}Zebadja, Meschullam, Hizki, Héber,
\VS{18}Jischmeraï, Jizlia, et Jobab, fils d' Elpaal.
\VS{19}Jakim, Zicri, Zabdi,
\VS{20}Eliénaï, Tsilthaï, Eliel,
\VS{21}Adaja, Beraja, et Schimrath, fils de Schimeï.
\VS{22}Jischpan, Eber, Eliel,
\VS{23}Abdon, Zicri, Hanan,
\VS{24}Hanania, Elam, Anthothija,
\VS{25}Jiphdeja et Penuel, fils de Schaschak.
\VS{26}Schamscheraï, Schecharia, Athalia,
\VS{27}Jaaréschia, Elija, et Zicri, fils de Jerocham.
\VS{28}Ce sont là les chefs des pères, selon leurs générations ; et ils habitèrent à Jérusalem.
\TextTitle{Les fils du père de Gabaon, ascendant de Saül}
\VS{29}Le père de Gabaon habita à Gabaon, sa femme avait pour nom Maaca.
\VS{30}Son fils premier-né fut Abdon, puis Tsur, Kis, Baal, Nadab,
\VS{31}Guedor, Achjo, et Zéker.
\VS{32}Mikloth engendra Schimea. Ils habitèrent aussi vis-à-vis de leurs frères à Jérusalem, avec leurs frères.
\VS{33}Ner engendra Kis, et Kis engendra Saül, et Saül engendra Jonathan, Malki-Schua, Abinadab, et Eschbaal.
\VS{34}Le fils de Jonathan fut  Merib-Baal ; et Merib-Baal engendra Michée.
\VS{35}Les fils de Michée furent Pithon, Mélec, Thaeréa, et Achaz.
\VS{36}Achaz engendra Jehoadda ; et Jehoadda engendra Alémeth, Azmaveth et Zimri ; Zimri engendra Motsa.
\VS{37}Motsa engendra Binea, qui eut pour fils Rapha, qui eut pour fils Eleasa, qui eut pour fils Atsel.
\VS{38}Atsel eut six fils, dont les noms sont : Azrikam, Bocru, Ismaël, Schearia, Abdias, et Hanan ; tous ceux-là furent fils d'Atsel.
\VS{39}Les fils d'Eschek, son frère, furent  Ulam son premier-né, Jéusch le second, Eliphéleth le troisième.
\VS{40}Et les fils d'Ulam furent des hommes forts et vaillants, tirant bien de l'arc, et ils eurent beaucoup de fils et de petits-fils, jusqu'à cent cinquante ; tous des fils de Benjamin.
\Chap{9}
\TextTitle{Les habitants de Jérusalem}
\VerseOne{}Ainsi, tous ceux d'Israël furent enregistrés par généalogie et inscrits dans le livre des rois d'Israël. Et ceux de Juda furent emmenés en captivité à Babylone à cause de leurs péchés\FTNT{La captivité babylonienne voir 2 R. 24 et 25.}.
\VS{2}Mais ce sont ici les premiers qui habitèrent dans leurs possessions, et dans leurs villes, tant d'Israël que des sacrificateurs, des Lévites, et des Néthiniens.
\VS{3}A Jérusalem habitaient les fils de Juda, les fils de Benjamin, et les fils d'Ephraïm et de Manassé.
\VS{4}Uthaï, fils d'Ammihud, fils d'Omri, fils d'Imri, fils de Bani, des fils de Pérets, fils de Juda.
\VS{5}Des Schilonites, Asaja le premier-né, et ses fils.
\VS{6}Des fils de Zérach, Jeuel, et ses frères, six cent quatre-vingt-dix.
\VS{7}Des fils de Benjamin, Sallu fils de Meschullam, fils de Hodavia, fils d'Assenua.
\VS{8}Jibneja, fils de Jerocham, et Ela fils d’Uzzi, fils de Micri ; et Meschullam fils de Schephathia, fils de Reuel, fils de Jibnija.
\VS{9}Leurs frères, selon leurs générations, furent neuf cent cinquante-six. Tous ces hommes-là furent chefs des pères dans les maisons de leurs  pères.
\VS{10}Des sacrificateurs : Jedaeja, Jehojarib, et Jakin.
\VS{11}Azaria fils de Hilkija, fils de Meschullam, fils de Tsadok, fils de Merajoth, fils d' Achithub, intendant de la maison de Dieu.
\VS{12}Adaja, fils de Jerocham, fils de Paschhur, fils de Malkija; et Maesaï, fils d'Adiel, fils de Jachzéra, fils de Meschullam, fils de Meschillémith, fils d'Immer.
\VS{13}Leurs frères, chefs de la maison de leurs pères, mille sept cent soixante hommes, forts et vaillants, occupés au service de la maison de Dieu.
\VS{14}Des Lévites : Schemaeja, fils de Haschub, fils d'Azrikam, fils de Haschabia, des fils de Merari,
\VS{15}Bakbakkar, Héresch, et Galal ; et Matthania, fils de Michée, fils de Zicri, fils d'Asaph,
\VS{16}Abdias fils de Schemaeja, fils de Galal, fils de Jeduthun ; et Bérékia, fils d'Asa, fils d'Elkana, qui habita dans les villages des Nethophathiens.
\VS{17}Et les portiers : Schallum, Akkub, Thalmon, et Achiman, et leurs frères ; mais Schallum était le chef.
\VS{18}Il l'a été jusqu'à maintenant, ayant la charge de la porte du roi vers l’orient. Ceux-là furent portiers pour le camp des fils de Lévi.
\VS{19}Schallum, fils de Koré, fils d'Ebiasaph, fils de Koré, et ses frères Koréites, de la maison de son père, remplissaient les fonctions de gardiens, gardant les seuils de la tente, comme leurs pères en avaient gardé l'entrée au camp de Yahweh ;
\VS{20}Phinées, fils d'Eléazar, fut établi chef sur eux en présence de Yahweh qui était avec lui.
\VS{21}Zacharie, fils de Meschélémia, était le portier de l'entrée de la tente d'assignation.
\VS{22}Ils étaient en tout deux cent douze, choisis pour être les portiers des seuils, et enregistrés selon les familles dans la généalogie, selon leurs villages ; David et Samuel, le voyant, les avaient établis dans leurs fonctions.
\VS{23}Eux, dis-je, et leurs fils furent établis sur les portes de la maison de Yahweh, qui est la maison du tabernacle, pour y faire la garde.
\VS{24}Il y avait des portiers aux quatre vents, à l'orient, à l'occident, au nord et au midi.
\VS{25}Et leurs frères, qui étaient dans leurs villages, devaient de temps à autre venir auprès d’eux pendant sept jours.
\VS{26}Car selon cette fonction, il y avait toujours quatre chefs des portiers, des Lévites, qui avaient la surveillance des chambres et des trésors de la maison de Dieu.
\VS{27}Ils se tenaient la nuit tout autour de la maison de Dieu, dont ils avaient la garde, et qu’ils devaient ouvrir tous les matins.
\VS{28}Certains d’entre eux prenaient soin des ustensiles du service; car on en faisait le compte lorsqu'on les rentrait et qu'on les sortait.
\VS{29}D’autres veillaient sur les ustensiles, sur tous les ustensiles du sanctuaire, sur la fleur de farine, sur le vin, sur l'huile, sur l'encens et sur les aromates.
\VS{30}Mais ceux qui composaient les parfums aromatiques étaient des fils de sacrificateurs.
\VS{31}Matthithia, d'entre les Lévites, premier-né de Schallum, Koréite, s’occupait des gâteaux cuits sur les plaques.
\VS{32}Et quelques-uns de leurs frères, parmi les fils des Kehathites, avaient la charge du pain de proposition\FTNT{Il y avait douze gâteaux de pain qu’on plaçait sur une table dans le tabernacle ou dans le temple et qu’on remplaçait chaque sabbat (Ex. 35:13 ;  Ex. 39:36 ; 1 R. 7:48 ; 2 Ch. 13:11 ; Né. 10:32-33). En hébreu, le pain de proposition signifie littéralement le « pain de la face ». Le mot rendu par « face » se rapporte parfois à la « présence » (2 R. 13:23). Le pain de proposition est en réalité l’image du Seigneur Jésus-Christ, notre pain de vie (Jn. 6:48-59).}  pour l'apprêter chaque sabbat.
\VS{33}Certains étaient des chantres, chefs des pères des Lévites, qui demeuraient dans les chambres, sans avoir d’autres charges, parce qu'ils devaient être en fonction le jour et la nuit.
\VS{34}Ce sont là les chefs des pères des Lévites, selon leurs familles ; ils furent chefs, et ils habitèrent à Jérusalem.
\TextTitle{De Jeïel au roi Saül, de Jonathan à Arsel\FTNTT{1 Ch. 10 ; 1 S. 1 ; 30}}
\VS{35}Or Jeïel, le père de Gabaon, habita à Gabaon ; et le nom de sa femme était Maaca.
\VS{36}Son fils premier-né, Abdon, puis Tsur, Kis, Baal, Ner, Nadab,
\VS{37}Guedor, Achjo, Zacharie, et Mikloth.
\VS{38}Mikloth engendra Schimeam ; et ils habitèrent vis-à-vis de leurs frères à Jérusalem, avec leurs frères.
\VS{39}Ner engendra Kis, et Kis engendra Saül, et Saül engendra Jonathan, Malki-Schua, Abinadab et Eschbaal.
\VS{40}Le fils de Jonathan fut  Merib-Baal; et Merib-Baal engendra Michée.
\VS{41}Et les fils de Michée furent Pithon, Mélec, Thachréa et Achaz.
\VS{42}Achaz engendra Jaera ; et Jaera engendra Alémeth, Azmaveth, et Zimri ; et Zimri engendra Motsa.
\VS{43}Motsa engendra Binea, qui eut pour fils Rephaja, qui eut pour fils Eleasa, qui eut pour fils Atsel.
\VS{44}Atsel eut six fils, dont les noms sont Azrikam, Bocru, Ismaël, Schearia, Abdias et Hanan. Ce furent là les fils d'Atsel.
\Chap{10}
\TextTitle{Mort de Saül\FTNTT{1 S. 31:1-10 ; 2 S. 1}}
\VerseOne{}Les Philistins combattirent contre Israël, et les hommes d'Israël s'enfuirent devant les Philistins, et tombèrent blessés à mort sur la montagne de Guilboa\FTNT{1 S. 31:1-10}.
\VS{2}Les Philistins poursuivirent et atteignirent Saül et ses fils, et tuèrent Jonathan, Abinadab et Malki-Schua, les fils de Saül.
\VS{3}L’effort du combat se porta sur Saül ; de sorte que les archers l'atteignirent, et il eut peur de ces archers.
\VS{4}Alors Saül dit à celui qui portait ses armes : Tire ton épée, et transperce-moi, de peur que ces incirconcis ne viennent et ne fassent de moi selon leur volonté ; mais celui qui portait ses armes ne voulut pas, parce qu’il avait très peur. Saül prit donc son épée, et se jeta dessus.
\VS{5}Alors celui qui portait les armes de Saül, ayant vu que Saül était mort, se jeta aussi sur son épée, et il mourut.
\VS{6}Ainsi mourut Saül, et ses trois fils, et toute sa maison périt avec lui.
\VS{7}Tous ceux d'Israël, qui étaient dans la vallée, ayant vu qu'on avait fui, et que Saül et ses fils étaient morts, abandonnèrent leurs villes et s'enfuirent, de sorte que les Philistins y entrèrent et y habitèrent.
\VS{8}Or il arriva que dès le lendemain, les Philistins vinrent pour dépouiller les morts, et ils trouvèrent Saül et ses fils étendus sur la montagne de Guilboa.
\VS{9}Ils le dépouillèrent et emportèrent sa tête et ses armes. Puis ils firent annoncer ces bonnes nouvelles par tout le pays des Philistins, et aux environs, pour en faire savoir les nouvelles à leurs dieux et au peuple.
\VS{10}Ils mirent ses armes dans la maison de leur dieu, et ils attachèrent sa tête dans la maison de Dagon\FTNT{1 S. 5:1-11.}.
\VS{11}Tous ceux de Jabès de Galaad, ayant appris tout ce que les Philistins avaient fait à Saül,
\VS{12}tous les vaillants hommes d’entre eux se levèrent et enlevèrent le corps de Saül et les corps de ses fils ; ils les apportèrent à Jabès, et ils ensevelirent leurs os sous un chêne à Jabès, et jeûnèrent pendant sept jours.
\VS{13}Saül mourut pour le crime qu'il avait commis contre Yahweh, en ce qu'il n'avait point gardé la parole de Yahweh, et qu'il avait même consulté ceux qui évoquent les morts\FTNT{1 S. 28:7-20.} pour savoir ce qui devait lui arriver.
\VS{14}Il ne consulta point Yahweh ; c'est pourquoi Yahweh le fit mourir, et transféra la royauté à David, fils d'Isaï.
\Chap{11}
\TextTitle{David règne sur Israël\FTNTT{2 S. 5:1-3 ; 2 S. 2-4}}
\VerseOne{}Tous ceux d'Israël s'assemblèrent auprès de David à Hébron, et lui dirent : Voici, nous sommes tes os et ta chair.
\VS{2}Autrefois déjà, quand Saül était roi, tu étais celui qui faisais sortir et qui ramenais Israël. Yahweh, ton Dieu, t'a dit : Tu paîtras mon peuple d'Israël, et tu seras le chef de mon peuple d'Israël.
\VS{3}Ainsi, tous les anciens d'Israël vinrent auprès du roi à Hébron ; et David traita alliance avec eux à Hébron, devant Yahweh. Ils oignirent David pour roi sur Israël, selon la parole de Yahweh, prononcée par Samuel\FTNT{2 S. 2, 3, 4 ; 2 S. 5:1-3.}.
\TextTitle{Jérusalem devient la cité de David\FTNTT{2 S. 5:6-10}}
\VS{4}David et tous ceux d'Israël s'en allèrent à Jérusalem, qui est Jébus. Là étaient  les Jébusiens qui habitaient le pays.
\VS{5}Ceux qui habitaient à Jébus dirent à David : Tu n'entreras point ici. Mais David prit la forteresse de Sion, qui est la cité de David.
\VS{6}Car David avait dit: Quiconque battra le premier les Jébusiens sera chef et prince. Joab, fils de Tseruja, monta le premier, et fut fait chef.
\VS{7}David s’établit dans la forteresse ; c'est pourquoi on l'appela la cité de David\FTNT{2 S. 5:6-10.}.
\VS{8}Il bâtit aussi la ville tout autour, depuis Millo et ses environs ; et Joab répara le reste de la ville.
\VS{9}David devenait de plus en plus grand, car Yahweh des armées était avec lui.
\TextTitle{Les vaillants hommes de David\FTNTT{2 S. 23:8-39}}
\VS{10}Voici les chefs des hommes vaillants qui étaient au service de David, qui l’aidèrent avec tout Israël à assurer sa royauté, afin de le faire régner selon la parole de Yahweh au sujet d’Israël.
\VS{11}Ceux-ci sont du nombre des vaillants hommes que David avait. Jaschobeam, fils de Hacmoni, chef entre les trois principaux. Il brandit sa lance contre trois cents hommes et les blessa à mort en une seule fois\FTNT{2 S. 23:8-39. }.
\VS{12}Après lui était Eléazar, fils de Dodo, l'Achochite, qui fut l’un des trois vaillants hommes.
\VS{13}Il se trouvait avec David à Pas-Dammim, lorsque les Philistins s'étaient assemblés pour combattre. Il y avait là une parcelle de terre remplie d'orge ; et le peuple fuyait devant les Philistins.
\VS{14}Ils s'arrêtèrent au milieu de cette parcelle de champ, la défendirent, et battirent les Philistins. Ainsi, Yahweh accorda une grande délivrance.
\VS{15}Il en descendit encore trois des trente chefs près du rocher, auprès de David, dans la caverne d'Adullam, lorsque l'armée des Philistins campait dans la vallée des Rephaïm.
\VS{16}David était alors dans la forteresse, et la garnison des Philistins était en ce même temps-là à Bethléhem.
\VS{17}David eut un désir, et dit : Qui est-ce qui me fera boire de l'eau du puits qui est à la porte de Bethléhem ?
\VS{18}Alors ces trois hommes passèrent au travers du camp des Philistins, et puisèrent de l'eau du puits qui était à la porte de Bethléhem ; et l'ayant apportée, la présentèrent à David, qui ne voulut point la boire, mais la répandit en l'honneur de Yahweh.
\VS{19}Car il dit : Que mon Dieu me garde de faire une telle chose ! Boirais-je le sang de ces hommes qui ont fait un tel voyage au péril de leur vie ? Car ils m'ont apporté cette eau au péril de leur vie. Ainsi, il ne voulut point la boire. Voilà ce que firent ces trois vaillants hommes.
\VS{20}Abischaï, frère de Joab, était chef des trois. Il sortit sa lance sur trois cents hommes, les blessa à mort ; et il eut du renom entre les trois.
\VS{21}Entre les trois, il fut plus honoré que les deux autres, et il fut leur chef ; cependant, il n'égala point ces trois premiers.
\VS{22}Benaja aussi, fils de Jehojada, fils d'un vaillant homme de Kabtseel, avait fait de grands exploits. Il tua deux des plus puissants hommes de Moab. Il descendit et frappa un lion au milieu d'une fosse en un jour de neige.
\VS{23}Il tua aussi un homme Egyptien qui était haut de cinq coudées. Cet Egyptien avait à la main une lance grosse comme une ensouple de tisserand ; mais il descendit contre lui avec un bâton, et arracha la lance de la main de l'Egyptien, et le tua avec sa propre lance.
\VS{24}Benaja, fils de Jehojada, fit ces choses-là, et fut célèbre entre ces trois vaillants hommes.
\VS{25}Voilà, il était le plus honoré des trente ; cependant, il n'égala point les trois premiers. David l'établit dans son conseil privé.
\VS{26}Et les plus vaillants d'entre les gens de guerre furent Asaël, frère de Joab ; et Elchanan fils de Dodo, de Bethléhem,
\VS{27}Schammoth d'Haror, Hélets de Palon,
\VS{28}Ira, fils d'Ikkesch, de Tekoa, Abiézer d'Anathoth,
\VS{29}Sibbecaï le Huschatite, Ilaï d'Achoach,
\VS{30}Maharaï de Nethopha, Héled fils de Baana de Nethopha,
\VS{31}Ithaï fils de Ribaï, de Guibea des fils de Benjamin, Benaja de Pirathon,
\VS{32}Huraï de Nachalé-Gaasch, Abiel d'Araba,
\VS{33}Azmaveth de Bacharum, Eliachba de Schaalbon,
\VS{34}Bené-Haschem de Guizon, Jonathan fils de Schagué d'Harar,
\VS{35}Achiam fils de Sacar d'Harar, Eliphal fils d'Ur,
\VS{36}Hépher de Mekéra, Achija de Palon,
\VS{37}Hetsro de Carmel, Naaraï fils d'Ezbaï,
\VS{38}Joël frère de Nathan, Mibchar fils d'Hagri,
\VS{39}Tsélek l'Ammonite, Nachraï de Béroth, qui portait les armes de Joab fils de Tseruja,
\VS{40}Ira de Jéther, Gareb de Jéther,
\VS{41}Urie le Héthien, Zabad fils d' Achlaï,
\VS{42}Adina fils de Schiza le Rubénite, chef des Rubénites, et trente avec lui.
\VS{43}Hanan fils de Maaca, et Josaphat de Mithni,
\VS{44}Ozias d'Aschtharoth, Schama et Jehiel fils de Hotham d'Aroër,
\VS{45}Jediaël fils de Schimri, et Jocha son frère, le Thitsite,
\VS{46}Eliel de Machavim, Jeribaï, et Joschavia fils d'Elnaam, et Jithma le Moabite,
\VS{47}Eliel, et Obed, et Jaasie-Metsobaja.
\Chap{12}
\TextTitle{Les guerriers venus chez David à Tsiklag\FTNTT{2 S. 5:17 ; 1 Ch. 12:8-15 ; 1 Ch. 14:8}}
\VerseOne{}Voici ceux qui allèrent trouver David à Tsiklag, lorsqu'il était encore éloigné de la présence de Saül, fils de Kis. Ils étaient parmi les vaillants hommes qui lui prêtèrent leur secours pendant la guerre.
\VS{2}Ils étaient équipés d'arcs, se servant de la main droite et de la gauche pour jeter des pierres, et pour tirer des flèches avec l'arc. Ils étaient frères de Saül, de Benjamin,
\VS{3}Achiézer, le chef, et Joas, fils de Schemaa, qui était de Guibea, Jeziel, Péleth, fils d'Azmaveth, Beraca et Jéhu d'Anathoth ;
\VS{4}Jischmaeja de Gabaon, vaillant entre les trente, et même au-dessus des trente, et Jérémie, Jachaziel, Jochanan et Jozabad de Guedéra ;
\VS{5}Eluzaï, Jerimoth, Bealia, Schemaria et Schephathia de Haroph ;
\VS{6}Elkana, Jischija, Azareel, Joézer et Jaschobeam Koréites ;
\VS{7}Joéla et Zebadia, fils de Jerocham de Guedor.
\TextTitle{Les guerriers venus chez David dans la forteresse de Moab\FTNTT{1 S. 22:2-4}}
\VS{8}Quelques-uns aussi des Gadites se retirèrent auprès de David, dans la forteresse, au désert, hommes forts et vaillants, experts à la guerre et maniant le bouclier et la lance. Leurs visages étaient comme des faces de lion, et aussi prompts que des gazelles sur les montagnes.
\VS{9}Ezer le premier, Abdias le second, Eliab le troisième ;
\VS{10}Mischmanna le quatrième, Jérémie le cinquième ;
\VS{11}Attaï le sixième ; Eliel le septième ;
\VS{12}Jochanan le huitième, Elzabad le neuvième ;
\VS{13}Jérémie le dixième, Macbannaï le onzième.
\VS{14}C’étaient des fils de Gath, qui furent chefs de l'armée ; le plus petit avait la charge de cent hommes, et le plus grand de mille.
\VS{15}Ce sont ceux qui passèrent le Jourdain au premier mois, quand il déborde sur tous ses rivages ; et ils chassèrent ceux qui demeuraient dans les vallées, vers l'orient et l'occident.
\VS{16}Il vint aussi des fils de Benjamin et de Juda vers David à la forteresse.
\VS{17}David sortit au-devant d'eux, et prenant la parole, il leur dit : Si vous êtes venus en paix vers moi pour m'aider, mon cœur s’unira à vous ; mais si c'est pour me trahir et me livrer à mes ennemis, quoique je ne sois coupable d'aucune violence, que le Dieu de nos pères le voie, et qu'il fasse justice !
\VS{18}Alors Amasaï, l’un des principaux officiers, fut revêtu de l’Esprit, et dit : Que la paix soit avec toi, ô David ! Qu'elle soit avec toi, fils d'Isaï ! Que la paix soit à ceux qui t'aident, puisque ton Dieu t'aide ! Et David les reçut, et les établit parmi les chefs de ses troupes.
\VS{19}Des hommes de Manassé se joignirent à David, lorsqu’il alla combattre Saül avec les Philistins. Mais David et ses gens ne les aidèrent pas, parce que les princes des Philistins, après en avoir délibéré entre eux, le renvoyèrent, en disant : Il se tournera vers son maître Saül, au péril de nos têtes.
\VS{20}Quand donc il retournait à Tsiklag, Adnach, Jozabad, Jediaël, Micaël, Jozabad, Elihu et Tsilthaï, chefs des milliers qui étaient en Manassé, se tournèrent vers lui.
\VS{21}Et ils aidèrent David contre la troupe des Amalécites, car ils étaient tous forts et vaillants, et ils furent faits chefs dans l'armée.
\VS{22}De jour en jour, il venait des gens auprès de David pour l'aider, de sorte qu'il eut une grande armée, comme une armée de Dieu\FTNT{1 S. 22:2-4}.
\TextTitle{Les guerriers venus chez David à Hébron\FTNTT{2 S. 5:1-3}}
\VS{23}Voici le nombre des hommes équipés pour la guerre, qui vinrent auprès de David à Hébron, afin de lui transférer la royauté de Saül, selon le commandement de Yahweh\FTNT{2 S. 5:1-3}.
\VS{24}Des fils de Juda, qui portaient le bouclier et la lance, six mille huit cents, équipés pour la guerre.
\VS{25}Des fils de Siméon, forts et vaillants pour la guerre, sept mille cent.
\VS{26}Des fils de Lévi, quatre mille six cents.
\VS{27}Et Jehojada, prince de ceux d'Aaron, et avec lui trois mille sept cents ;
\VS{28}et Tsadok, jeune homme fort et vaillant, et vingt-deux chefs de la maison de son père.
\VS{29}Des fils de Benjamin, parents de Saül, trois mille ; car jusqu'alors la plus grande partie d’entre eux soutenaient la maison de Saül.
\VS{30}Des fils d'Ephraïm, vingt mille huit cents, forts et vaillants, et hommes de renom dans la maison de leurs pères.
\VS{31}De la demi-tribu de Manassé, dix-huit mille, qui furent désignés par leur nom pour aller établir David roi.
\VS{32}Des fils d'Issacar\FTNT{Les fils d’Issacar avaient la connaissance des temps. Discerner les temps dans lesquels nous sommes n’a rien à voir avec le fait de chercher à connaître la date du retour du Seigneur. Seul le Père connaît la date du retour du Messie (Za. 14:7 ; Mt. 24:36). Comprendre les caractéristiques de notre époque nous aide à nous réveiller afin d’accomplir les oeuvres que le Seigneur nous confie. Cette prise de conscience nous aidera à éviter les pièges de Satan et à mieux nous préparer aux noces de l’Agneau. Voir Mt. 16:3 ;  Ro. 13:11-14 ;  2 Pi. 1:19.}, fort intelligents dans la connaissance des temps, pour savoir ce que devait faire Israël, deux cents de leurs chefs, et tous leurs frères sous leurs ordres.
\VS{33}De Zabulon, cinquante mille combattants, rangés en bataille avec toutes sortes d'armes, et prêts à livrer bataille d’un cœur assuré.
\VS{34}De Nephthali, mille capitaines, et avec eux trente-sept mille, portant le bouclier et la lance.
\VS{35}Des Danites, vingt-huit mille six cents, équipés pour la guerre.
\VS{36}D'Aser, quarante mille combattants, et prêts à combattre.
\VS{37}De l’autre côté du Jourdain, à savoir des Rubénites, des Gadites, et de la demi-tribu de Manassé, cent vingt mille, avec tous les instruments de guerre pour combattre.
\VS{38}Tous ces hommes, gens de guerre, prêts a combattre, vinrent tous de bon cœur à Hébron, pour établir David roi sur tout Israël. Et tout le reste d'Israël était aussi d'un même sentiment pour établir David roi.
\VS{39}Et ils furent là avec David, mangeant et buvant pendant trois jours ; car leurs frères leur avaient préparé des vivres.
\VS{40}Et même ceux qui étaient les plus proches d'eux, jusqu'à Issacar, Zabulon et Nephthali, apportaient du pain sur des ânes, sur des chameaux, sur des mulets et sur des bœufs, de la farine, des figues sèches, des raisins secs, du vin, et de l'huile ; et ils amenaient des bœufs et des brebis en abondance, car il y avait une joie en Israël.
\Chap{13}
\TextTitle{Retour de l'arche, Uzza frappé par Yahweh\FTNTT{2 S. 6:1-11}}
\VerseOne{}Or David tint conseil avec les chefs de milliers et de centaines, avec tous les princes du peuple.
\VS{2}Et il dit à toute l'assemblée d'Israël : Si vous l'approuvez, et que cela vient de Yahweh, notre Dieu, envoyons partout vers nos autres frères, qui sont dans toutes les contrées d'Israël, et avec lesquels sont les sacrificateurs et les Lévites, dans leurs villes et dans leurs banlieues, afin qu'ils se réunissent à nous,
\VS{3}et que nous ramenions auprès de nous l’arche de notre Dieu ; car nous ne nous en sommes pas occupés du temps de Saül.
\VS{4}Et toute l'assemblée répondit qu'on le fasse ainsi ; car la chose fut approuvée par tout le peuple.
\VS{5}David donc assembla tout Israël, depuis Schichor, le torrent d'Egypte, jusqu'à l'entrée du pays de Hamath, pour ramener de Kirjath-Jearim l’arche de Dieu.
\VS{6}Et David monta avec tout Israël vers Baala à Kirjath-Jearim, qui appartient à Juda, pour faire amener de là l’arche de Dieu, devant laquelle est invoqué le Nom de Yahweh, qui habite entre les chérubins.
\VS{7}Ils mirent l’arche de Dieu sur un char neuf, et l'emmenèrent de la maison d'Abinadab ; et Uzza et Achjo conduisaient le char.
\VS{8}Et David et tout Israël dansaient en présence de Dieu de toute leur force, en chantant des cantiques et en jouant sur des violons, des luths, des tambourins, des cymbales, et des trompettes.
\VS{9}Quand ils furent arrivés à l'aire de Kidon, Uzza\FTNT{L’arche devait être transportée grâce à  des barres faites spécialement à cet effet, qui ne devaient pas être enlevées (Ex 27:6-7 ;  No. 1:51). Selon la Loi, seuls les Lévites devaient préparer et déplacer tout ce qui concernait le tabernacle.  Et même parmi les Lévites, chaque famille avait une fonction spécifique (No. 3 ; No. 4). Les Kehathites n’étaient pas autorisés à toucher l’arche, leur rôle se limitait seulement à la transporter à l’aide des barres (No. 4:15). Uzza a étendu sa main sur l’arche, alors qu’il n’était certainement pas Lévite. Il était devenu trop familier avec les choses saintes et avait pris à la légère les principes de Dieu. Il a voulu aider le Seigneur. Or, il ne faut jamais chercher à servir Dieu sans être appelé par lui.}  étendit sa main pour retenir l’arche, parce que les boeufs avaient glissé.
\VS{10}Et la colère de Yahweh s'enflamma contre Uzza, et il le frappa, parce qu'il avait étendu sa main sur l’arche. Uzza mourut en présence de Dieu.
\VS{11}David fut irrité de ce que Yahweh avait fait une brèche en la personne de Uzza. On a appelé jusqu'à ce jour ce lieu-là Pérets-Uzza, brèche d'Uzza.
\VS{12}David eut peur de Dieu en ce jour-là, et il dit: Comment ferais-je entrer chez moi l’arche de Dieu ?
\VS{13}C'est pourquoi David ne la retira point chez lui, dans la cité de David, mais il la fit conduire dans la maison d'Obed- Edom de Gath.
\VS{14}Et l’arche de Dieu demeura trois mois avec la famille d'Obed-Edom, dans sa maison. Yahweh bénit la maison d'Obed-Edom, et tout ce qui lui appartenait.
\Chap{14}
\TextTitle{Rayonnement du règne de David\FTNTT{2 S. 5:11-25 ; 23:13-17 ; 1 Ch. 3:5-9 ; 11:15-19 ; 12:8-15}}
\VerseOne{}Hiram, roi de Tyr, envoya des messagers à David, et du bois de cèdre, des tailleurs de pierres et des charpentiers, pour lui bâtir une maison.
\VS{2}Alors David reconnut que Yahweh l'affermissait comme roi sur Israël, et que son règne était fort élevé, à cause de son peuple d'Israël.
\VS{3}David prit encore des femmes à Jérusalem, et il engendra encore des fils et des filles.
\VS{4}Voici les noms des fils qu'il eut à Jérusalem : Schammua, Schobab, Nathan, Salomon,
\VS{5}Jibhar, Elischua, Elphéleth,
\VS{6}Noga, Népheg, Japhia,
\VS{7}Elischama, Beéliada et Eliphéleth.
\VS{8}Or quand les Philistins apprirent que David avait été oint pour roi sur tout Israël, ils montèrent tous à sa recherche. David l'ayant appris, sortit au-devant d'eux.
\VS{9}Les Philistins vinrent et se répandirent dans la vallée des Rephaïm.
\VS{10}David consulta Dieu, en disant : Monterai-je contre les Philistins, et les livreras-tu entre mes mains? Yahweh lui répondit : Monte, et je les livrerai entre tes mains.
\VS{11}Alors ils montèrent à Baal-Peratsim\FTNT{Baal-Peratsim signifie « seigneur des brèches »}, où David les battit. Puis il dit : Dieu a fait une brèche au milieu de mes ennemis par ma main, comme une brèche faite par les eaux. C'est pourquoi on donna à ce lieu-là le nom de Baal- Peratsim.
\VS{12}Et ils laissèrent là leurs dieux, et David ordonna qu'on les brûle au feu.
\VS{13}Les Philistins se répandirent encore une autre fois dans cette même vallée.
\VS{14}David consulta encore Dieu ; et Dieu lui répondit : Tu ne monteras point vers eux, mais tu te détourneras d'eux, et tu iras contre eux vis-à-vis des mûriers.
\VS{15}Dès que tu auras entendu au sommet des mûriers un bruit comme des gens qui marchent, tu sortiras alors pour combattre, car c’est Dieu qui marche devant toi pour frapper le camp des Philistins.
\VS{16}David fit selon ce que Dieu lui avait ordonné, et on frappa le camp des Philistins, depuis Gabaon jusqu'à Guézer.
\VS{17}Ainsi, la renommée de David se répandit par tous ces pays-là, et Yahweh remplit de frayeur toutes ces nations-là, au seul nom de David.
\Chap{15}
\TextTitle{David coordonne avec minutie l’arrivé de l’arche à Jérusalem\FTNTT{2 S. 6:12}}
\VerseOne{}David se bâtit des maisons dans la cité de David ; il prépara un lieu pour l’arche de Dieu, et dressa pour elle une tente.
\VS{2}Et David dit : L’arche de Dieu ne doit être portée que par les Lévites, car Yahweh les a choisis pour porter l’arche de Dieu, et pour faire le service à toujours\FTNT{No. 4:15.}.
\VS{3}David donc assembla tous ceux d'Israël à Jérusalem, pour faire monter l’arche de Yahweh dans le lieu qu'il lui avait préparé.
\VS{4}David assembla aussi les fils d'Aaron, et les Lévites.
\VS{5}Des fils de Kehath : Uriel, le chef, et ses frères, cent vingt.
\VS{6}Des fils de Merari : Asaja, le chef, et ses frères, deux cent vingt.
\VS{7}Des fils de Guerschon : Joël, le chef, et ses frères, cent trente.
\VS{8}Des fils d'Elitsaphan : Schemaeja, le chef, et ses frères, deux cents.
\VS{9}Des fils de Hébron : Eliel, le chef, et ses frères, quatre-vingts.
\VS{10}Des fils de Uziel : Amminadab, le chef, et ses frères, cent douze.
\VS{11}David appela les sacrificateurs Tsadok et Abiathar, et les Lévites, à savoir Uriel, Asaja, Joël, Schemaeja, Eliel, et Amminadab ;
\VS{12}et il leur dit: Vous qui êtes les chefs des familles des Lévites, sanctifiez-vous, vous et vos frères ; et transportez l’arche de Yahweh, le Dieu d'Israël, au lieu que je lui ai préparé.
\VS{13}Parce que vous n'y étiez pas la première fois, Yahweh, notre Dieu, a fait une brèche parmi nous ; car nous ne l'avons pas cherché selon la loi.
\VS{14}Les sacrificateurs donc et les Lévites se sanctifièrent pour faire monter l’arche de Yahweh, le Dieu d'Israël.
\VS{15}Et les fils des Lévites portèrent l’arche de Dieu sur leurs épaules, avec les barres qu'ils avaient sur eux, comme Moïse l'avait ordonné selon la parole de Yahweh.
\VS{16}David dit aux chefs des Lévites d'établir quelques-uns de leurs frères chantres, avec des instruments de musique, des luths, des violons, et des cymbales qui feraient retentir des sons éclatants, en signe de réjouissance.
\VS{17}Les Lévites donc établirent Héman, fils de Joël, et parmi ses frères, Asaph, fils de Bérékia; et des fils de Merari, qui étaient leurs frères, Ethan, fils de Kuschaja ;
\VS{18}avec eux leurs frères pour être du second ordre : Zacharie, Ben, Jaaziel, Schemiramoth, Jehiel, Unni, Eliab, Benaja, Maaséja, Matthithia, Eliphelé, Miknéja, Obed-Edom, et Jeïel, les portiers.
\VS{19}Quant aux chantres : Héman, Asaph et Ethan, ils avaient des cymbales d'airain pour les faire retentir.
\VS{20}Zacharie, Aziel, Schemiramoth, Jehiel, Unni, Eliab, Maaséja, et Benaja jouaient des luths sur alamoth ;
\VS{21}et Matthithia, Eliphelé, Miknéja, Obed-Edom, Jeïel et Azazia jouaient des harpes à huit cordes, pour conduire le chant.
\VS{22}Mais Kenania, le chef des Lévites, avait la charge de faire porter l’arche, enseignant comment il fallait la porter, car il était un homme très intelligent.
\VS{23}Bérékia et Elkana étaient portiers de l’arche.
\VS{24}Schebania, Josaphat, Nethaneel, Amasaï, Zacharie, Benaja, Eliézer, les sacrificateurs, sonnaient des trompettes devant l’arche de Dieu, et Obed-Edom et Jechija étaient portiers de l’arche.
\TextTitle{L'arche transportée au milieu des réjouissances\FTNTT{2 S. 6:12}}
\VS{25}David et les anciens d'Israël, avec les gouverneurs de milliers, marchaient, amenant avec joie l’arche de l'alliance de Yahweh, de la maison d' Obed-Edom.
\VS{26}Dieu aidait les Lévites qui portaient l’arche de l'alliance de Yahweh, et l’on sacrifia sept veaux et sept béliers.
\VS{27}David était vêtu d'un manteau de fin lin ; et tous les Lévites aussi qui portaient l’arche, les chantres ;  et Kenania, qui avait la principale charge de faire porter l’arche, était avec les chantres ; et David avait un éphod de lin.
\VS{28}Ainsi tout Israël amena l’arche de l'alliance de Yahweh, avec de grands cris de joie, et au son du cor, des shofars et des cymbales, faisant retentir leur voix avec des luths et des harpes.
\VS{29}Mais il arriva, comme l’arche de l'alliance de Yahweh entrait dans la cité de David, que Mical, fille de Saül, regardant par la fenêtre, vit le roi David sautant et dansant, et elle le méprisa dans son coeur.
\Chap{16}
\TextTitle{L’arche placée dans une tente à Jérusalem ; sacrifices et cantiques pour Yahweh\FTNTT{2 S. 6:17-19}}
\VerseOne{}Ils amenèrent donc l’arche de Dieu et la posèrent au milieu de la tente que David avait dressée pour elle ; et on offrit devant Dieu des holocaustes et des sacrifices d’offrande de paix.
\VS{2}Quand David eut achevé d'offrir les holocaustes et les sacrifices d’offrande de paix, il bénit le peuple au Nom de Yahweh.
\VS{3}Et il distribua à chacun, tant aux hommes qu'aux femmes, un pain,  un morceau de viande et un gâteau de raisin.
\VS{4}Et il établit quelques-uns des Lévites pour faire le service devant l’arche de Yahweh, pour célébrer, remercier, et louer le Dieu d'Israël.
\VS{5}Asaph était le premier et Zacharie le second ; Jeïel, Schemiramoth, Jehiel, Matthithia, Eliab, Benaja, Obed-Edom, et Jeïel, qui avaient des instruments de musique, à savoir des luths et des harpes ; et Asaph faisait retentir sa voix avec des cymbales.
\VS{6}Benaja et Jachaziel, les sacrificateurs, étaient continuellement avec des trompettes devant l’arche de l'alliance de Dieu.
\VS{7}Et en ce même jour, David remit entre les mains d'Asaph et de ses frères, les Psaumes suivants, pour commencer à célébrer Yahweh :
\VS{8}Célébrez Yahweh, invoquez son Nom ! Faites connaître parmi les peuples ses exploits !
\VS{9}Chantez-le,  célébrez-le !  Parlez de toutes ses merveilles !
\VS{10}Glorifiez-vous de son saint Nom !  Que le cœur de ceux qui cherchent Yahweh se réjouisse !
\VS{11}Recherchez Yahweh et sa force, cherchez continuellement sa face !
\VS{12}Souvenez-vous des merveilles qu'il a faites, de ses miracles et des jugements de sa bouche.
\VS{13}Postérité d'Israël, son serviteur, fils de Jacob, ses élus !
\VS{14}Yahweh est notre Dieu ; ses jugements s’exercent sur toute la terre.
\VS{15}Souvenez-vous toujours de son alliance, de ses promesses établies pour mille générations ;
\VS{16}du traité qu'il a fait avec Abraham et du serment qu’il a fait à Isaac,
\VS{17}et qu’il a confirmé à Jacob et à Israël, pour être une loi et une alliance éternelle,
\VS{18}en disant : Je te donnerai le pays de Canaan, comme l’héritage qui vous est échu.
\VS{19}Ils étaient alors une  poignée de gens, peu nombreux, et étrangers dans le pays,
\VS{20}car ils étaient errants de nation en nation, et d'un royaume vers un autre peuple.
\VS{21}Il ne permit à personne de les opprimer ; il a même châtié des rois à cause d'eux.
\VS{22}Et il a dit : Ne touchez point à mes oints, et ne faites point de mal à mes prophètes\FTNT{L’expression « ne touchez pas à mes oints » signifie qu'il ne faut pas leur porter physiquement atteinte. C’est une expression associée à des mauvais traitements physiques. Il est donc clair que ce verset, qu’on trouve également dans le  Ps. 105 : 15, ne peut absolument pas concerner la remise en question des enseignements d'un quelconque pasteur, prophète ou apôtre. Dans le contexte de ce passage, il est question des rois, des prophètes et des sacrificateurs, car c’est sur eux que reposait l’onction. Aujourd’hui, tous les chrétiens sont oints de Dieu (Ep. 1:13 ; Ep. 4:30).} !
\VS{23}Habitants de la terre, chantez à Yahweh ! Racontez chaque jour sa délivrance.
\VS{24}Racontez sa gloire parmi les nations, et ses merveilles parmi tous les peuples !
\VS{25}Car Yahweh est grand et très digne de louanges, il est plus redoutable que tous les dieux.
\VS{26}Car tous les dieux des peuples sont des idoles\FTNT{Jésus-Christ est le seul et le véritable Dieu (1 Co. 8:6 ; 1 Jn. 5:20).}, mais Yahweh a fait les cieux.
\VS{27}La majesté et la magnificence marchent devant lui; la force et la joie sont dans le lieu où il habite.
\VS{28}Familles des peuples, donnez à Yahweh, donnez à Yahweh gloire et force !
\VS{29}Donnez à Yahweh la gloire due à son Nom ! Apportez des offrandes, et présentez-vous devant lui. Prosternez-vous devant Yahweh avec des ornements saints !
\VS{30}Tremblez, vous tous habitants de la terre tout étonnés devant sa face ! Car la terre habitable est affermie par lui, et elle ne chancelle point.
\VS{31}Que les cieux se réjouissent, que la terre soit dans l’allégresse ! Et que l’on dise parmi les nations : Yahweh règne !
\VS{32}Que la mer retentisse avec tout ce qu'elle contient ! Que la campagne se réjouisse avec tout ce qu’elle renferme !
\VS{33}Que les arbres de la forêt poussent des cris de joie au devant de Yahweh, parce qu'il vient juger la terre\FTNT{Yahweh vient juger la terre. Cette prophétie confirme de façon incontestable la divinité de Jésus-Christ. Voir Za. 14:1-7.}.
\VS{34}Célébrez Yahweh, car il est bon, car sa miséricorde demeure à jamais !
\VS{35}Et dites : Ô Dieu de notre salut, sauve-nous, et rassemble-nous, et retire-nous d'entre les nations, pour célébrer ton saint Nom, et que nous nous glorifions de ta louange !
\VS{36}Béni soit Yahweh, le Dieu d'Israël, de siècle en siècle ! Et tout le peuple dit : Amen ! Louez Yahweh !
\VS{37}On laissa donc là, devant l’arche de l'alliance de Yahweh, Asaph et ses frères, pour faire le service continuellement, remplissant leur tâche jour par jour devant l’arche.
\VS{38}On laissa Obed-Edom, et ses frères, au nombre de soixante-huit, Obed-Edom, dis-je, fils de Jeduthun, et Hosa comme portiers.
\VS{39}On établit le sacrificateur Tsadok, et les sacrificateurs ses frères, devant le tabernacle de Yahweh, dans le haut lieu qui était à Gabaon,
\VS{40}pour offrir des holocaustes à Yahweh continuellement sur l'autel de l'holocauste, matin et soir, selon tout ce qui est écrit dans la loi de Yahweh, qu’il ordonna à Israël.
\VS{41}Auprès d’eux étaient Héman et Jeduthun, et les autres qui furent choisis et désignés par leur nom, pour célébrer Yahweh, parce que sa miséricorde demeure éternellement.
\VS{42}Et Héman et Jeduthun étaient avec ceux-là; il y avait aussi des trompettes et des cymbales pour ceux qui les faisaient retentir, et des instruments pour chanter les cantiques de Dieu. Les fils de Jeduthun étaient portiers.
\VS{43}Puis tout le peuple s'en alla chacun dans sa maison, et David aussi s'en retourna pour bénir sa maison.
\Chap{17}
\TextTitle{David veut construire un temple à Yahweh\FTNTT{2 S. 7:1-3}}
\VerseOne{}Or il arriva après que David fut établi dans sa maison, qu'il dit à Nathan, le prophète : Voici,  j’habite dans une maison de cèdres, et l’arche de l'alliance de Yahweh est sous une tente.
\VS{2}Nathan dit à David : Fais tout ce que tu as dans le cœur, car Dieu est avec toi.
\TextTitle{Réponse de Yahweh à David\FTNTT{2 S. 7:4-17}}
\VS{3}Mais il arriva cette nuit-là que la parole de Dieu fut adressée à Nathan, en disant :
\VS{4}Va, et dis à David, mon serviteur : Ainsi parle Yahweh :  Tu ne me bâtiras point de maison pour y habiter.
\VS{5}Puisque je n'ai point habité dans une maison depuis le jour où j'ai fait monter les fils d’Israël hors d'Egypte jusqu'à ce jour ; mais j'ai été de tente en tente, et de tabernacle en tabernacle.
\VS{6}Partout où j’ai marché avec tout Israël, ai-je dit un mot à un seul des juges d'Israël, auxquels j'ai ordonné de paître mon peuple, ai-je dit : Pourquoi ne m'avez-vous point bâti une maison de cèdres ?
\VS{7}Maintenant donc tu diras ainsi à David, mon serviteur : Ainsi parle Yahweh des armées : Je t'ai pris d'une cabane, d'auprès des brebis, afin que tu sois le conducteur de mon peuple d'Israël ;
\VS{8}j'ai été avec toi partout où tu as marché, j'ai exterminé devant toi tous tes ennemis, et j’ai rendu ton nom semblable au nom des grands qui sont sur la terre.
\VS{9}J’ai établi un lieu pour mon peuple d'Israël, et je l’ai planté afin qu’il habite chez lui et ne soit plus agité.  Les fils d'iniquité ne le détruiront plus comme ils l’ont fait auparavant,
\VS{10}et comme à l’époque où j'ai établi des juges sur mon peuple d'Israël. J’ai humilié tous tes ennemis. Je t’informe que Yahweh te bâtira une maison.
\VS{11}Quand tes jours seront accomplis pour t'en aller avec tes pères, je ferai lever ta postérité après toi, l’un de tes fils, et j'affermirai son règne\FTNT{Cette prophétie est relative au Messie. Voir 2 S. 7:12-17.}.
\VS{12}Il me bâtira une maison, et j'affermirai son trône éternellement.
\VS{13}Je serai pour lui un père, et il sera pour moi un fils ; et je ne retirerai point de lui ma grâce, comme je l'ai retirée de celui qui a été avant toi.
\VS{14}Mais je l'établirai dans ma maison et dans mon royaume éternellement, et son trône sera affermi pour toujours.
\VS{15}Nathan récita à David toutes ces paroles, et toute cette vision.
\TextTitle{Adoration et reconnaissance de David à Yahweh\FTNTT{2 S. 7:18-29}}
\VS{16}Alors le roi David entra, et se tint devant Yahweh, et dit : Ô Yahweh Dieu ! Qui suis-je, et quelle est ma maison, que tu m'aies fait parvenir au point où je suis ?
\VS{17}Mais cela t'a semblé être peu de chose, ô Dieu ! Et tu as parlé de la maison de ton serviteur pour le temps à venir, et tu as porté les regards sur moi à la manière de l'homme, toi qui es élevé, ô Yahweh Dieu !
\VS{18}Que pourrait te dire encore David de l'honneur que tu fais à ton serviteur ? Car tu connais ton serviteur.
\VS{19}Ô Yahweh ! Pour l'amour de ton serviteur, et selon ton cœur, tu as fait toutes ces grandes choses, pour lui révéler toutes ces grandeurs.
\VS{20}Ô Yahweh ! Nul n’est semblable à toi, et il n'y a point d'autre Dieu que toi selon tout ce que nous avons entendu de nos oreilles.
\VS{21}Et qui est comme ton peuple d'Israël, la seule nation sur la terre que Dieu lui-même est venu racheter pour lui, afin qu'elle soit son peuple, et pour te faire un Nom et pour accomplir des miracles et des prodiges, en chassant les nations devant ton peuple que tu as racheté d'Egypte ?
\VS{22}Et tu as établi ton peuple d'Israël afin qu’il soit ton peuple à toujours ; et toi, ô Yahweh ! Tu as été son Dieu.
\VS{23}Maintenant donc, ô Yahweh ! Que la parole que tu as prononcée sur ton serviteur et sur sa maison, soit ferme à jamais, et agis selon ta parole !
\VS{24}Et que ton Nom subsiste et soit magnifié éternellement, de sorte qu'on dise : Yahweh des armées, le Dieu d'Israël, est Dieu pour Israël ; et que la maison de David, ton serviteur, soit affermie devant toi.
\VS{25}Car, ô mon Dieu ! Tu as révélé à ton serviteur que tu lui bâtirais une maison. C'est pourquoi ton serviteur a pris la hardiesse de te faire cette prière.
\VS{26}Maintenant, ô Yahweh ! Tu es Dieu, et tu as parlé de ce bien à ton serviteur.
\VS{27}Veuille donc maintenant bénir la maison de ton serviteur, afin qu'elle soit éternellement devant toi ; car tu l'as bénie, ô Yahweh ! Et elle sera bénie à jamais !
\Chap{18}
\TextTitle{Le règne de David affermi\FTNTT{2 S. 8:1-18}}
\VerseOne{}Et il arriva que David battit les Philistins, et les humilia, et il enleva de la main des Philistins Gath et les villes de son ressort\FTNT{2 S. 8.  }.
\VS{2}Il battit aussi les Moabites, et les Moabites furent asservis à David et lui payèrent un tribut.
\VS{3}David battit aussi Hadarézer, roi de Tsoba, vers Hamath, lorsqu’il alla établir sa domination sur le fleuve de l'Euphrate.
\VS{4}David lui prit mille chars, sept mille cavaliers, et vingt mille hommes de pied ; et il coupa les jarrets des chevaux de tous les chars, mais il réserva cent chars.
\VS{5}Les Syriens de Damas vinrent au secours d’Hadarézer, roi de Tsoba, et David battit vingt-deux mille Syriens.
\VS{6}Puis David mit des garnisons dans la Syrie de Damas. Et les Syriens furent assujettis à David et lui payèrent un tribut. Yahweh sauvait David partout où il allait.
\VS{7}Et David prit les boucliers d'or qui étaient aux serviteurs de Hadarézer, et les apporta à Jérusalem.
\VS{8}Il emporta aussi de Thibchath, et de Cun, villes de Hadarézer, une grande quantité d'airain, dont Salomon fit la mer d'airain, les colonnes et les ustensiles d'airain.
\VS{9}Thohu, roi de Hamath, apprit que David avait défait toute l'armée de Hadarézer, roi de Tsoba.
\VS{10}Et il envoya Hadoram, son fils, vers le roi David pour le saluer et le féliciter de ce qu'il avait combattu Hadarézer, et qu'il l'avait défait. Car Hadarézer était dans une guerre continuelle contre Thohu. Quant à tous les vases d'or, d'argent, et d'airain,
\VS{11}le roi David les consacra aussi à Yahweh, avec l'argent et l'or qu'il avait emporté de toutes les nations, à savoir d'Edom, de Moab, des fils d'Ammon, des Philistins, et d’Amalek.
\VS{12}Et Abischaï, fils de Tseruja battit dix-huit mille Edomites dans la vallée du sel.
\VS{13}Il mit une garnison dans Edom, et tous les Edomites furent asservis à David ; et Yahweh gardait David partout où il allait.
\VS{14}Ainsi, David régna sur tout Israël, rendant jugement et justice à tout son peuple.
\VS{15}Joab, fils de Tseruja, avait la charge de l'armée, et Josaphat, fils d'Achilud, était archiviste.
\VS{16}Tsadok, fils d'Achithub, et Abimélec, fils d'Abiathar, étaient les sacrificateurs; et Schavscha était le secrétaire.
\VS{17}Benaja, fils de Jehojada, était sur les Kéréthiens et les Péléthiens ; mais les fils de David étaient les premiers auprès du roi.
\Chap{19}
\TextTitle{David monte contre les Ammonites et les Syriens\FTNTT{2 S. 10}}
\VerseOne{}Or il arriva après cela que Nachasch, roi des fils d’Ammon, mourut ; et son fils régna à sa place.
\VS{2}David dit : J'userai de bonté envers Hanun, fils de Nachasch, car son père a usé de bonté envers moi. Ainsi, David envoya des messagers pour le consoler de la mort de son père ; et les serviteurs de David vinrent au pays des fils d’Ammon vers Hanun pour le consoler.
\VS{3}Mais les chefs d'entre les fils d’Ammon dirent à Hanun : Penses-tu que ce soit pour honorer ton père que David t'a envoyé des consolateurs ? N'est-ce pas pour examiner et épier le pays, afin de le détruire, que ses serviteurs sont venus vers toi ?
\VS{4}Alors Hanun prit les serviteurs de David, les fit raser, et les fit couper leurs habits par le milieu jusqu'aux hanches. Puis il les renvoya.
\VS{5}Ils s'en allèrent, et le firent savoir par le moyen de quelques personnes à David, qui envoya des gens à leur rencontre ; car ces hommes-là étaient fort confus. Et le roi leur fit dire : Restez à Jéricho jusqu'à ce que votre barbe ait repoussé, et revenez ensuite.
\VS{6}Or les fils d’Ammon voyant qu'ils s'étaient rendus odieux à David, Hanun et les fils d'Ammon envoyèrent mille talents d'argent pour prendre à leur solde des chars et des cavaliers de Mésopotamie, de Syrie, de Maaca et de Tsoba.
\VS{7}Ils prirent à leur solde trente-deux mille hommes et des chars, et le roi de Maaca avec son peuple, lesquels vinrent camper devant Médeba. Les fils d'Ammon aussi s'assemblèrent de leurs villes et vinrent pour combattre.
\VS{8}David l’ayant appris, envoya Joab et ceux de toute l'armée qui étaient les plus vaillants.
\VS{9}Les fils d’Ammon sortirent et rangèrent leur armée en bataille à l'entrée de la ville ; et les rois qui étaient venus étaient à part dans la campagne.
\VS{10}Joab, voyant que l'armée était tournée contre lui devant et derrière, prit de tous les gens d'élite d'Israël, et les rangea contre les Syriens.
\VS{11}Et il donna la conduite du reste du peuple à Abischaï, son frère; et on les rangea contre les fils d’Ammon.
\VS{12}Et Joab lui dit: Si les Syriens sont plus forts que moi, tu viendras me délivrer ; et si les fils d'Ammon sont plus forts que toi, je te délivrerai.
\VS{13}Sois ferme, et montrons-nous vaillants pour notre peuple, et pour les villes de notre Dieu ; et que Yahweh fasse ce qui lui semblera bon.
\VS{14}Alors Joab et le peuple qui était avec lui s'approchèrent pour livrer bataille aux Syriens qui s'enfuirent de devant lui.
\VS{15}Et les fils d'Ammon voyant que les Syriens s'étaient enfuis, eux aussi s'enfuirent devant Abischaï, frère de Joab, et rentrèrent dans la ville, et Joab revint à Jérusalem.
\VS{16}Mais les Syriens, qui avaient été battus par ceux d'Israël, envoyèrent des messagers et firent venir les Syriens qui étaient au-delà du fleuve ; et Schophach, chef de l'armée d'Hadarézer, les conduisait.
\VS{17}On le rapporta à David, qui assembla tout Israël, passa le Jourdain, alla au-devant d'eux, et se rangea en bataille contre eux. David donc rangea la bataille contre les Syriens, et ils combattirent contre lui.
\VS{18}Mais les Syriens s'enfuirent de devant Israël ; et David défit sept mille chars des Syriens et quarante mille hommes de pied ; et il tua Schophach, le chef de l'armée.
\VS{19}Alors les serviteurs d'Hadarézer, voyant qu'ils avaient été battus par ceux d'Israël, firent la paix avec David, et lui furent asservis ; et les Syriens ne voulurent plus secourir les fils d’Ammon.
\Chap{20}
\TextTitle{Conquête de Rabba\FTNTT{2 S. 11:1-12:25 ; 2 S. 12:26-31}}
\VerseOne{}L’année suivante, au temps où les rois se mettaient en campagne, Joab conduisit une forte armée et ravagea le pays des fils d’Ammon ; puis il alla assiéger Rabba, tandis que David resta à Jérusalem. Joab battit Rabba, et la détruisit\FTNT{2 S 12:26-31.}.
\VS{2}David enleva la couronne de dessus la tête de son roi, et il trouva qu'elle pesait un talent d'or : Elle était garnie de pierres précieuses. On la mit sur la tête de David, qui emmena un très grand butin de la ville.
\VS{3}Il emmena aussi le peuple qui y était, et les mit aux scies, aux pics de fer et aux haches de fer ; David traita de la sorte toutes les villes des fils d’Ammon ; puis il s'en retourna avec tout le peuple à Jérusalem.
\TextTitle{Guerre contre les Philistins\FTNTT{2 S. 21:15-22}}
\VS{4}Il arriva après cela que la guerre continua à Guézer contre les Philistins. Alors Sibbecaï, le Huschatite, frappa Sippaï, qui était des fils de Rapha, et ils furent humiliés\FTNT{2 S 21:15-22.}.
\VS{5}Il y eut encore une autre guerre contre les Philistins. Et Elchanan, fils de Jaïr, frappa Lachmi, frère de Goliath de Gath, qui avait une lance dont le bois était comme une ensouple de tisserand.
\VS{6}Il y eut encore une autre guerre à Gath, où se trouva un homme de grande stature, qui avait six doigts à chaque main, et six orteils à chaque pied, de sorte qu'il en avait en tout vingt-quatre ; et il était aussi issu de Rapha.
\VS{7}Et il défia Israël ; mais Jonathan, fils de Schimea, frère de David, le tua.
\VS{8}Ceux-là naquirent à Gath ; ils étaient des enfants de Rapha, et ils moururent par les mains de David, et par les mains de ses serviteurs.
\Chap{21}
\TextTitle{David fait le dénombrement contre la volonté de Yahweh\FTNTT{2 S. 24:1-17}}
\VerseOne{}Mais Satan s'éleva contre Israël, et il incita David à faire le dénombrement d'Israël.
\VS{2}Et David dit à Joab et aux chefs du peuple : Allez et faites le dénombrement d'Israël, depuis Beer-Schéba jusqu'à Dan, et rapportez-le-moi, afin que j’en connaisse le nombre.
\VS{3}Mais Joab répondit : Que Yahweh veuille augmenter son peuple cent fois encore plus qu'il ne l’est, ô roi, mon seigneur. Tous ne sont-ils pas serviteurs de mon seigneur ? Pourquoi mon seigneur cherche-t-il cela ? Et pourquoi cela serait-il imputé comme un crime à Israël ?
\VS{4}Mais la parole du roi l'emporta sur Joab. Et Joab partit et parcourut tout Israël ; puis il revint à Jérusalem.
\VS{5}Et Joab donna à David le rôle du dénombrement du peuple, et il se trouva dans tout Israël onze cent mille hommes tirant l'épée ; et dans Juda quatre cent soixante-dix mille hommes tirant l'épée.
\VS{6}Bien qu'il n'eût pas compté entre eux ni Lévi ni Benjamin, parce que Joab exécutait la parole du roi en l’ayant en abomination,
\VS{7}cette chose déplut à Dieu, c'est pourquoi il frappa Israël.
\VS{8}Et David dit à Dieu : J'ai commis un très grand péché d'avoir fait une telle chose ; je te prie, pardonne maintenant l'iniquité de ton serviteur, car j'ai agi en insensé.
\VS{9}Et Yahweh parla à Gad, le voyant de David, en disant :
\VS{10}Va, parle à David, et dis-lui : Ainsi parle Yahweh, je te propose trois choses ; choisis l'une d'elles, afin que je te la fasse.
\VS{11}Et Gad vint à David, et lui dit : Ainsi parle Yahweh :
\VS{12}Choisis soit la famine durant l'espace de trois ans ; soit d'être consumé durant trois mois, étant poursuivi par tes ennemis, en sorte que l'épée de tes ennemis t'atteigne ; ou que l'épée de Yahweh, la peste, soit durant trois jours sur le pays, et que l'Ange de Yahweh porte la destruction dans toutes les contrées d'Israël. Vois maintenant ce que j'aurai à répondre à celui qui m'a envoyé.
\VS{13}Alors David répondit à Gad : Je suis dans une très grande angoisse ! Que je tombe, je te prie, entre les mains de Yahweh, parce que ses compassions sont immenses ; mais que je ne tombe point entre les mains des hommes !
\VS{14}Yahweh envoya donc la peste sur Israël, et il tomba soixante-dix mille hommes d'Israël.
\VS{15}Dieu envoya aussi un ange à Jérusalem pour la détruire ; et comme il la détruisait, Yahweh regarda et se repentit de ce mal. Et il dit à l'Ange qui détruisait : C'est assez ! Retire à présent ta main. Et l'Ange de Yahweh se tenait près de l'aire d'Ornan, le Jébusien.
\VS{16}Or David leva les yeux,  et vit l'Ange de Yahweh\FTNT{Ge. 16:7.} qui était entre la terre et le ciel, ayant dans sa main son épée nue, tournée contre Jérusalem. Et David et les anciens, couverts de sacs, tombèrent sur leurs faces.
\VS{17}Et David dit à Dieu : N'est-ce pas moi qui ai ordonné qu'on fasse le dénombrement du peuple ? C'est donc moi qui ai péché et qui ai très mal agi ; mais ces brebis qu'ont-elles fait ? Yahweh, mon Dieu ! Je te prie que ta main soit contre moi, et contre la maison de mon père, mais qu'elle ne soit pas contre ton peuple, pour le détruire.
\TextTitle{Fin de la plaie après l’offrande de David\FTNTT{2 S. 24:18-25}}
\VS{18}Alors l'Ange de Yahweh ordonna à Gad de dire à David, qu'il monte pour dresser un autel à Yahweh, dans l'aire d'Ornan, le Jébusien.
\VS{19}David donc monta selon la parole que Gad lui avait dite au Nom de Yahweh.
\VS{20}Ornan s'étant retourné, et ayant vu l'Ange,  ses quatre fils se cachèrent avec lui. Or Ornan foulait du blé.
\VS{21}David vint jusqu'à Ornan, et Ornan regarda, et ayant vu David, il sortit de l'aire et se prosterna devant lui, le visage à terre.
\VS{22}Et David dit à Ornan : Donne-moi la place de cette aire, et j'y bâtirai un autel à Yahweh ; donne-la-moi pour le prix qu'elle vaut, afin que cette plaie soit arrêtée de dessus le peuple.
\VS{23}Et Ornan dit à David : Prends-la, et que le roi, mon seigneur, fasse tout ce qui lui semblera bon. Voici, je donne ces bœufs pour les holocaustes, et ces instruments à fouler du blé pour le bois, et ce blé pour l'offrande ; je donne toutes ces choses.
\VS{24}Mais le roi David lui répondit : Non, mais certainement j'achèterai tout cela au prix qu'il vaut ; car je ne présenterai point à Yahweh ce qui est à toi, et je n'offrirai point un holocauste qui ne me coûte rien.
\VS{25}David donna donc à Ornan pour cette place, six cents sicles d'or de poids.
\VS{26}Puis il bâtit là un autel à Yahweh, et il offrit des holocaustes et des sacrifices d’offrande de paix, et il invoqua Yahweh, qui l'exauça par le feu envoyé des cieux sur l'autel de l'holocauste.
\VS{27}Alors Yahweh parla à l'ange, et l'ange remit son épée dans son fourreau.
\VS{28}En ce temps-là, David, voyant que Yahweh l'avait exaucé dans l'aire d'Ornan, le Jébusien, y offrait des sacrifices.
\VS{29}Or le tabernacle de Yahweh, que Moïse avait construit au désert, et l'autel des holocaustes, étaient en ce temps-là dans le haut lieu de Gabaon.
\VS{30}Mais David ne pouvait pas aller devant cet autel pour invoquer Dieu, parce qu'il avait été épouvanté à cause de l'épée de l'Ange de Yahweh.
\Chap{22}
\TextTitle{Préparatifs de David pour la construction du temple}
\VerseOne{}Et David dit : C'est ici la maison de Yahweh Dieu, et c'est ici l'autel pour les holocaustes d'Israël.
\VS{2}David ordonna de rassembler les étrangers qui étaient dans le pays d'Israël, et il établit des tailleurs de pierres pour tailler des pierres de taille, pour la construction de la maison de Dieu.
\VS{3}David prépara aussi du fer en abondance, afin d'en faire des clous pour les battants des portes et pour les crampons,  de l’airain en quantité telle qu’il n’était pas possible de le peser,
\VS{4}et du bois de cèdre sans nombre, parce que les Sidoniens et les Tyriens amenaient à David du bois de cèdre en abondance.
\VS{5}David dit : Salomon, mon fils, est jeune et délicat, et la maison qu'il faut bâtir à Yahweh doit être magnifique en excellence, en réputation, et en gloire, dans tous les pays. Je lui préparerai donc maintenant de quoi la bâtir. Ainsi, David prépara, avant sa mort, ces choses en abondance.
\TextTitle{Recommandation de David à Salomon }
\VS{6}Puis il appela Salomon, son fils, et lui ordonna de bâtir une maison à Yahweh, le Dieu d'Israël.
\VS{7}David donc dit à Salomon : Mon fils, j’avais à cœur de bâtir une maison au Nom de Yahweh, mon Dieu.
\VS{8}Mais la parole de Yahweh m'a été adressée, en disant : Tu as répandu beaucoup de sang, et tu as fait de grandes guerres ; tu ne bâtiras point de maison à mon Nom, parce que tu as répandu beaucoup de sang sur la terre devant moi.
\VS{9}Voici, il te naîtra un fils, qui sera un homme de repos,  et à qui je donnerai du repos par rapport à tous ses ennemis tout autour, c'est pourquoi son nom sera Salomon. Et en son temps, je donnerai la paix et le repos à Israël.
\VS{10}Ce sera lui qui bâtira une maison à mon Nom ; et il sera un fils pour moi, et je serai un père pour lui ; et j'affermirai le trône de son règne sur Israël à jamais.
\VS{11}Maintenant donc, mon fils, Yahweh sera avec toi, et tu prospéreras, et tu bâtiras la maison de Yahweh, ton Dieu, ainsi qu'il l’a déclaré à ton égard.
\VS{12}Seulement, que Yahweh te donne de la sagesse et de l'intelligence, et qu'il t'instruise touchant le gouvernement d'Israël, et comment tu dois garder la loi de Yahweh, ton Dieu.
\VS{13}Tu prospéreras si tu as soin de mettre en pratique les lois et les ordonnances que Yahweh a prescrites à Moïse pour Israël. Fortifie-toi et prends courage ; ne crains point et ne t'effraie de rien.
\VS{14}Voici, selon ma petitesse, j'ai préparé pour la maison de Yahweh cent mille talents d'or et un million de talents d'argent. Quant à l'airain et au fer, il est d'un poids incalculable, car il est en abondance. J'ai aussi préparé le bois et les pierres ; et tu y ajouteras ce qu'il faudra .
\VS{15}Tu as avec toi beaucoup d'ouvriers, de maçons, de tailleurs de pierres, de charpentiers, et toutes sortes de gens experts dans toute espèce d’ouvrage.
\VS{16}Il y a de l'or et de l'argent, de l'airain et du fer sans nombre. Lève-toi et agis, et Yahweh sera avec toi.
\VS{17}David ordonna aussi à tous les chefs d'Israël d'aider Salomon, son fils ; et il leur dit :
\VS{18}Yahweh,votre Dieu, n'est-il pas avec vous, et ne vous a-t-il pas donné du repos de tous côtés ? Car il a livré entre mes mains les habitants du pays, et le pays a été soumis devant Yahweh, et devant son peuple.
\VS{19}Maintenant donc, appliquez vos cœurs et vos âmes à rechercher Yahweh, votre Dieu ; levez-vous et bâtissez le sanctuaire de Yahweh Dieu, afin d’amener l’arche de l'alliance de Yahweh, et les ustensiles consacrés à Dieu dans la maison qui doit être bâtie au Nom de Yahweh.
\Chap{23}
\TextTitle{David désigne Salomon comme son successeur\FTNTT{1 Ch. 28:1}}
\VerseOne{}David étant vieux et rassasié de jours, établit Salomon, son fils, pour roi sur Israël.
\VS{2}Et il assembla tous les principaux d'Israël, les sacrificateurs et les Lévites.
\VS{3}On fit le dénombrement des Lévites, depuis l'âge de trente ans et au-dessus ; et les mâles d' entre-eux étant comptés, chacun par tête, il y eut trente-huit mille hommes\FTNT{No. 3:25-37}.
\VS{4}Et David dit : Qu’il y en ait  parmi eux vingt-quatre mille pour vaquer ordinairement à l'œuvre de la maison de Yahweh, et six mille comme  magistrats et juges,
\VS{5}quatre mille portiers, et quatre autres mille pour louer Yahweh avec des instruments que j'ai faits pour le louer.
\TextTitle{Dénombrement des Lévites\FTNTT{No. 3:25-37}}
\VS{6}David les divisa en classes d'après les fils de Lévi, à savoir Guerschon, Kehath et Merari.
\VS{7}Des Guerschonites il y eut Laedan et Schimeï.
\VS{8}Les fils de Laedan furent ces trois : Jehiel le premier, puis Zétham, puis Joël.
\VS{9}Les fils de Schimeï furent ces trois : Schelomith, Haziel et Haran. Ce sont là les chefs des maisons paternelles de la famille de Laedan.
\VS{10}Et les fils de Schimeï furent Jachath, Zina, Jeusch et Beria. Ce sont là les quatre fils de Schimeï.
\VS{11}Jachath était le premier et Zina le second; mais Jeusch et Beria n'eurent pas beaucoup de fils, c'est pourquoi ils furent comptés pour une seule maison paternelle dans le dénombrement.
\VS{12}Des fils de Kehath il y eut Amram, Jitsehar, Hébron et Uziel, en tout quatre.
\VS{13}Les fils d’Amram furent Aaron et Moïse. Aaron fut séparé lui et ses fils à toujours, pour sanctifier le Saint des saints, pour faire brûler des parfums en présence de Yahweh, pour le servir, et pour bénir en son Nom à toujours.
\VS{14}Et quant à Moïse, homme de Dieu, ses fils devaient être comptés de la tribu de Lévi.
\VS{15}Les fils de Moïse furent Guerschom et Eliézer.
\VS{16}Des fils de Guerschom, Schebuel le premier.
\VS{17}Quant aux fils d' Eliézer, Rechabia fut le premier ; et Eliézer n'eut point d'autres fils, mais les fils de Rechabia furent très nombreux.
\VS{18}Des fils de Jitsehar, Schelomith était le premier.
\VS{19}Les fils de Hébron furent Jerija le premier, Amaria le second, Jachaziel le troisième, Jekameam le quatrième.
\VS{20}Les fils d’Uziel furent Michée le premier, Jischija le second.
\VS{21}Des fils de Merari il y eut Machli et Muschi. Les fils de Machli furent Eléazar et Kis.
\VS{22}Eléazar mourut, et n'eut point de fils, mais des filles ; et les fils de Kis, leurs frères les prirent pour femmes.
\VS{23}Les fils de Muschi furent Machli, Eder et Jerémoth, eux trois.
\TextTitle{Fonctions des Lévites\FTNTT{No. 3:5-12}}
\VS{24}Ce sont là les fils de Lévi, selon les maisons de leurs pères, chefs des maisons paternelles, selon leurs dénombrements qui furent faits en comptant leurs noms, étant comptés chacun par tête ; et ils faisaient l’œuvre du service de la maison de Yahweh, depuis l'âge de vingt ans et au-dessus.
\VS{25}Car David dit : Yahweh, le Dieu d'Israël, a donné du repos à son peuple, et il établira sa demeure dans Jérusalem  à toujours.
\VS{26}Quant aux Lévites, ils n'auront plus à porter le tabernacle ni tous les ustensiles pour son service.
\VS{27}C'est pourquoi, dans les derniers registres de David, les fils de Lévi furent dénombrés depuis l'âge de vingt ans et au-dessus.
\VS{28}Car leur charge était d'assister les fils d'Aaron pour le service de la maison de Yahweh, étant établis sur le parvis et sur les chambres, pour la purification de toutes les choses saintes, pour l'œuvre du service de la maison de Dieu,
\VS{29}pour les pains de proposition, de la fleur de farine pour l'offrande, des galettes sans levain, pour tout ce qui se cuit sur la plaque, pour tout ce qui est rissolé, et pour la petite et grande mesure,
\VS{30}pour se présenter tous les matins et tous les soirs, afin de célébrer et louer Yahweh,
\VS{31}et offrir tous les holocaustes qu'il fallait offrir à Yahweh les jours de sabbat, aux nouvelles lunes, et aux fêtes solennelles, continuellement devant Yahweh, selon le nombre et les usages prescrits.
\VS{32}Ils donnaient leurs soins à la tente d’assignation, au lieu saint, et aux fils d’Aaron, leurs frères, pour le service de la maison de Yahweh.
\Chap{24}
\TextTitle{Vingt-quatre classes de sacrificateurs}
\VerseOne{}Quant aux fils d'Aaron, voici leurs classes\FTNT{Les vingt-quatre classes de sacrificateurs qui se tenaient devant Yahweh dans le temple de Jérusalem étaient une représentation des vingt-quatre vieillards qui se tiennent devant le trône de Dieu (Ap. 4:4).}. Les fils d’Aaron furent Nadab, Abihu, Eléazar et Ithamar.
\VS{2}Mais Nadab et Abihu\FTNT{Lé. 10:1-4.} moururent en présence de leur père, et n'eurent point de fils ; et Eléazar et Ithamar exercèrent la sacrificature.
\VS{3}Or David les sépara, à savoir Tsadok, qui était des fils d'Eléazar, et Achimélec, qui était des fils d'Ithamar, en fonction de leurs charges dans le service qu'ils avaient à faire.
\VS{4}Il se trouva parmi les fils d'Eléazar plus de chefs que parmi les fils d'Ithamar, et on en fit la division ; les fils d'Eléazar avaient seize chefs, selon leurs maisons paternelles, et les fils d'Ithamar huit chefs de maisons paternelles.
\VS{5}Et on les classa par le sort, les entremêlant les uns avec les autres, car les chefs du sanctuaire et les chefs de la maison de Dieu furent tirés tant des fils d'Eléazar que des fils d'Ithamar.
\VS{6}Schemaeja, fils de Nethaneel, le scribe, qui était de la tribu de Lévi, les mit par écrit devant le roi, les princes du peuple, devant Tsadok, le sacrificateur, et Achimélec, fils d'Abiathar, et devant les chefs de maisons paternelles des sacrificateurs et des Lévites. On tira au sort une maison paternelle pour Eléazar, et une autre fut tirée pour Ithamar.
\VS{7}Le premier sort échut à Jehojarib, le second à Jedaeja,
\VS{8}le troisième à Harim, le quatrième à Seorim,
\VS{9}le cinquième à Malkija, le sixième à Mijamin,
\VS{10}le septième à Hakkots, le huitième à Abija,
\VS{11}le neuvième à Josué, le dixième à Schecania,
\VS{12}le onzième à Eliaschib, le douzième à Jakim,
\VS{13}le treizième à Huppa, le quatorzième à Jeschébeab,
\VS{14}le quinzième à Bilga, le seizième à Immer,
\VS{15}le dix-septième à Hézir, le dix-huitième à Happitsets,
\VS{16}le dix-neuvième à Pethachja, le vingtième à Ezéchiel,
\VS{17}le vingt et unième à Jakim, et le vingt-deuxième à Gamul,
\VS{18}le vingt-troisième à Delaja, le vingt-quatrième à Maazia.
\VS{19}Tel fut leur classement pour le service qu'ils avaient à faire, lorsqu'ils entraient dans la maison de Yahweh, selon qu'il leur avait été ordonné par Aaron, leur père, comme Yahweh, le Dieu d'Israël, le lui avait ordonné.
\TextTitle{Les chefs des lévites ; les fils de Kehath et de Merari}
\VS{20}Voici les chefs du reste des Lévites. Des fils de Amram : Schubaël ; et des fils de Shubaël, Jechdia.
\VS{21}De Rechabia, des fils de Rechabia, Jischija était le premier.
\VS{22}Des Jitseharites, Schelomoth ; des fils de Schelomoth, Jachath.
\VS{23}Des fils d’Hébron, Jerija, Amaria le second ; Jachaziel le troisième, Jekameam le quatrième.
\VS{24}Des fils d'Uziel, Michée ; des fils de Michée, Schamir.
\VS{25}Le frère de Michée était Jischija ; des fils de Jischija, Zacharie.
\VS{26}Des fils de Merari, Machli et Muschi. Des fils de Jaazija, son fils.
\VS{27}Des fils de Merari, de Jaazija, son fils : Schoham, Zaccur et Ibri.
\VS{28}De Machli, Eléazar, qui n'eut point de fils.
\VS{29}De Kis, les fils de Kis, Jerachmeel.
\VS{30}Et des fils de Muschi, Machli, Eder et Jerimoth. Ce sont là les fils des Lévites, selon les maisons de leurs pères.
\VS{31}Eux aussi, comme leurs frères, les fils d'Aaron, ils tirèrent au sort, devant le  roi David, Tsadok et Ahimélec, et les chefs des pères des sacrificateurs et des Lévites. Il en fut ainsi pour chaque chef de maison comme pour le moindre de ses frères.
\Chap{25}
\TextTitle{Dénombrement des musiciens et des chantres}
\VerseOne{}David et les chefs de l'armée mirent à part pour le service ceux des fils d'Asaph, d'Héman et de Jeduthun qui prophétisaient avec des harpes, des luths et des cymbales. Et voici le nombre des hommes employés pour le service qu’ils avaient à faire.
\VS{2}Des fils d'Asaph : Zaccur, Joseph, Nethania et Aschareéla, fils d'Asaph, sous la conduite d'Asaph, qui prophétisait selon les ordres du roi.
\VS{3}De Jeduthun, les six fils de Jeduthun : Guedalia, Tseri, Esaïe, Haschabia, Matthithia et Schimeï, jouaient de la harpe, sous la conduite de leur père Jeduthun, qui prophétisait en célébrant et louant Yahweh.
\VS{4}D'Héman, les fils d'Héman : Bukkija, Matthania, Uziel, Schebuel, Jerimoth, Hanania, Hanani, Eliatha, Guiddalthi, Romamthi-Ezer, Joschbekascha, Mallothi, Hothir, Machazioth.
\VS{5}Tous ceux-là étaient fils d'Héman, le voyant du roi, qui révélait les paroles de Dieu pour en exalter la puissance. Dieu donna à Héman quatorze fils et trois filles.
\VS{6}Tous ceux-là étaient employés, sous la conduite de leurs pères, aux cantiques de la maison de Yahweh, avec des cymbales, des luths, et des harpes, dans le service de la maison de Dieu, selon les ordres du roi donnés à Asaph, à Jeduthun et à Héman.
\VS{7}Et leur nombre avec leurs frères, auxquels on avait enseigné les cantiques de Yahweh, était de deux cent quatre-vingt-huit, tous très habiles.
\TextTitle{Les musiciens et chantres répartis en vingt-quatre classes}
\VS{8}Et ils tirèrent au sort pour leurs fonctions, petits et grands, maîtres et disciples.
\VS{9}Et le premier sort échut à Asaph, à savoir à Joseph. Le second à Guedalia, lui, ses frères et ses fils étaient douze.
\VS{10}Le troisième à Zaccur, lui, ses fils et ses frères étaient douze.
\VS{11}Le quatrième à Jitseri, lui, ses fils et ses frères étaient douze.
\VS{12}Le cinquième à Nethania, lui, ses fils et ses frères étaient douze.
\VS{13}Le sixième à Bukkija, lui, ses fils et ses frères étaient douze.
\VS{14}Le septième à Jesareéla, lui, ses fils et ses frères étaient douze.
\VS{15}Le huitième à Esaïe, lui, ses fils et ses frères étaient douze.
\VS{16}Le neuvième à Matthania, lui, ses fils et ses frères étaient douze.
\VS{17}Le dixième à Schimeï, lui, ses fils et ses frères étaient douze.
\VS{18}L'onzième à Azareel, lui, ses fils et ses frères étaient douze.
\VS{19}Le douzième à Haschabia, lui, ses fils, et ses frères étaient douze.
\VS{20}Le treizième à Schubaël, lui, ses fils et ses frères étaient douze.
\VS{21}Le quatorzième à Matthithia, lui, ses fils et ses frères étaient douze.
\VS{22}Le quinzième à Jerémoth, lui, ses fils et ses frères étaient douze.
\VS{23}Le seizième à Hanania, lui, ses fils et ses frères étaient douze.
\VS{24}Le dix-septième à Joschbekascha, lui, ses fils et ses frères étaient douze.
\VS{25}Le dix-huitième à Hanani, lui, ses fils et ses frères étaient douze.
\VS{26}Le dix-neuvième à Mallothi, lui, ses fils et ses frères étaient douze.
\VS{27}Le vingtième à Elijatha, lui, ses fils et ses frères étaient douze.
\VS{28}Le vingt et unième à Hothir, lui, ses fils et ses frères étaient douze.
\VS{29}Le vingt-deuxième à Guiddalthi, lui, ses fils et ses frères étaient douze.
\VS{30}Le vingt-troisième à Machazioth, lui, ses fils et ses frères étaient douze.
\VS{31}Le vingt-quatrième à Romamthi-Ezer, lui, ses fils et ses frères étaient douze.
\Chap{26}
\TextTitle{Les classes des portiers}
\VerseOne{}Et quant aux classes des portiers, il y eut pour les Koréites : Meschélémia, fils de Koré, d'entre les fils d'Asaph.
\VS{2}Les fils de Meschélémia furent Zacharie, le premier-né, Jediaël le second, Zebadia le troisième, Jathniel le quatrième,
\VS{3}Elam le cinquième, Jochanan le sixième et Eljoénaï le septième.
\VS{4}Les fils d’Obed-Edom furent Schemaeja le premier-né, Jozabad le second, Joach le troisième, Sacar le quatrième, Nethaneel le cinquième,
\VS{5}Ammiel le sixième, Issacar le septième, Peulthaï le huitième ; car Dieu l'avait béni.
\VS{6}A Schemaeja, son fils, naquirent des fils qui eurent le commandement sur la maison de leur père, parce qu'ils étaient des hommes forts et vaillants.
\VS{7}Les fils donc de Schemaeja furent Othni, Rephaël, Obed, Elzabad et ses frères, hommes vaillants, Elihu et Semaeja.
\VS{8}Tous ceux-là étaient des fils d' Obed-Edom, eux, leurs fils et leurs frères, étaient des hommes pleins de vigueur et de force pour le service ; ils étaient soixante-deux d'Obed-Edom.
\VS{9}Les fils de Meschélémia avec ses frères, vaillants hommes étaient au nombre de dix-huit.
\VS{10}Les fils de Hosa, d'entre les fils de Merari, furent  Schimri le chef, quoiqu'il ne fût pas l'aîné, néanmoins son père l'établit pour chef ;
\VS{11}Hilkija était le second, Thebalia le troisième, Zacharie le quatrième; tous les fils et frères de Hosa furent treize.
\VS{12}A ces classes de portiers, aux chefs de ces hommes et à leurs frères, fut remise la garde pour le service de la maison de Yahweh.
\VS{13}Ils tirèrent au sort pour chaque porte, autant pour le plus petit que pour le plus grand, selon leurs familles.
\VS{14}Et ainsi, le sort pour la porte vers l'orient échut à Schélémia. Puis on tira au sort pour Zacharie, son fils, qui était un sage conseiller, et la porte du côté du nord lui fut échue par le sort.
\VS{15}Le sort d'Obed-Edom lui échut pour la porte du côté du sud, et la maison des magasins échut à ses fils.
\VS{16}A Schuppim et à Hosa pour la porte vers l'occident, auprès de la porte de Schalléketh, au chemin montant ; une garde étant vis-à-vis de l'autre.
\VS{17}Il y avait vers l'orient six Lévites ; vers le nord, quatre par jour  ; vers le sud, quatre aussi par jour ; et vers la maison des magasins, deux de chaque côté ;
\VS{18}du côté de la banlieue vers l'occident, il y en avait quatre au chemin, et deux vers la banlieue.
\VS{19}Ce sont là les classes des portiers pour les fils des Koréites, et pour les fils de Merari.
\TextTitle{Les Lévites commis sur les trésors du temple}
\VS{20}Ceux-ci aussi étaient Lévites : Achija commis sur les trésors de la maison de Dieu et les trésors des choses consacrées.
\VS{21}Des fils de Laedan, qui étaient d'entre les fils des Guerschonites, du côté de Laedan, d'entre les chefs des maisons paternelles appartenant à Laedan le Guerschonite, Jehiéli.
\VS{22}D'entre les fils de Jehiéli : Zétham et Joël, son frère, commis sur les trésors de la maison de Yahweh.
\VS{23}Pour les Amramites, les Jitseharites, les Hébronites et les  Uziélites,
\VS{24}Schebuel, fils de Guerschom, fils de Moïse, était commis sur les autres trésors.
\VS{25}Et quant à ses frères issus d'Eliézer, dont Rechabia fut fils, dont le fils fut Esaïe, dont le fils fut Joram, dont le fils fut  Zicri, dont le fils fut Schelomith,
\VS{26}c’étaient Schelomith et ses frères qui gardaient tous les trésors des choses saintes que le roi David, les chefs des familles paternelles, les chefs de milliers et de centaines, et les chefs de l'armée avaient consacrées.
\VS{27}C'était le butin de guerre qu'ils avaient consacré, pour l’entretien de la maison de Yahweh.
\VS{28}Tout ce qu'avait consacré Samuel, le voyant, Saül, fils de Kis, Abner, fils de Ner et Joab, fils de Tseruja, toutes les choses consacrées étaient mises sous la main de Schelomith et de ses frères.
\TextTitle{Les magistrats et juges en Israël}
\VS{29}Parmi les Jitseharites, Kenania et ses fils étaient employés aux affaires extérieures sur Israël pour être magistrats et juges.
\VS{30}Quant aux Hébronites, Haschabia et ses frères, hommes vaillants, au nombre de mille sept cents, avaient la surveillance d'Israël de l’autre côté du Jourdain, vers l'occident, pour toute œuvre qui concernait Yahweh, et pour le service du roi.
\VS{31}Quant aux Hébronites, selon leurs générations dans les maisons paternelles, Jerija fut le chef des Hébronites. On fit une recherche au sujet des Hébronites à la quarantième année du règne de David, et on trouva parmi eux à Jaezer de Galaad, des hommes forts et vaillants.
\VS{32}Les frères de Jerija, hommes vaillants, furent deux mille sept cents, issus des maisons paternelles ; et le roi David les établit sur les Rubénites, sur les Gadites, et sur la demi-tribu de Manassé, pour toute œuvre qui concernait Dieu, et pour les affaires du roi.
\Chap{27}
\TextTitle{Les douze chefs de guerre de David}
\VerseOne{}Quant aux fils d'Israël, selon leur dénombrement, il y avait des chefs de maisons paternelles, des chefs de milliers et de centaines, et leurs officiers, qui servaient le roi pour tout ce qui concernait les divisions, leur arrivée et leur départ, mois par mois, pendant tous les mois de l'année, et chaque division était de vingt-quatre mille hommes.
\VS{2}Et Jaschobeam, fils de Zabdiel, présidait sur la première division, pour le premier mois ; et dans sa division il y avait vingt-quatre mille hommes.
\VS{3}Il était des fils de Pérets, chef de tous les capitaines de l'armée du premier mois.
\VS{4}Dodaï, l'Achochite, présidait sur la division du deuxième mois, Mikloth, l’un des chefs de sa division ; et il avait une division de vingt-quatre mille hommes.
\VS{5}Le chef de la troisième armée pour le troisième mois était Benaja, fils de Jehojada, le sacrificateur et le capitaine en chef ; et dans sa division il y avait vingt-quatre mille hommes.
\VS{6}C'est ce Benaja qui était fort entre les trente, et par dessus les trente ; et Ammizadab, son fils, était dans sa division.
\VS{7}Le quatrième pour le quatrième mois était Asaël, frère de Joab, et Zébadia son fils, après lui ; et il y avait dans sa division vingt-quatre mille hommes.
\VS{8}Le cinquième pour le cinquième mois était le capitaine Schamehuth, le Jizrachite ; et dans sa division il y avait vingt-quatre mille hommes.
\VS{9}Le sixième pour le sixième mois était Ira, fils d' Ikkesch le Tekoïte ; et dans sa division il y avait vingt-quatre mille hommes.
\VS{10}Le septième pour le septième mois était Hélets le Pelonite, des fils d'Ephraïm ; et il y avait dans sa division vingt-quatre mille hommes.
\VS{11}Le huitième pour le huitième mois était Sibbecaï le Huschatite, de la famille des Zérachites ; et il y avait dans sa division vingt-quatre mille hommes.
\VS{12}Le neuvième pour le neuvième mois était Abiézer d'Anathoth, des Benjamites ; et il y avait dans sa division vingt-quatre mille hommes.
\VS{13}Le dixième pour le dixième mois était Maharaï de Nethopha, de la famille des Zérachites ; et il y avait dans sa division vingt-quatre mille hommes.
\VS{14}Le onzième pour le onzième mois était Benaja de Pirathon, des fils d'Ephraïm; et il y avait dans sa division vingt-quatre mille hommes.
\VS{15}Le douzième pour le douzième mois était Heldaï de Nethopha, appartenant à Othniel; et il y avait dans sa division vingt-quatre mille hommes.
\TextTitle{Les douze chefs des tribus d’Israël}
\VS{16}Et ceux-ci présidaient sur les tribus d'Israël : Eliézer, fils de Zicri, était le conducteur des Rubénites. Des Siméonites : Schephathia, fils de Maaca.
\VS{17}Des Lévites, Haschabia, fils de Kemuel. De ceux d'Aaron : Tsadok.
\VS{18}De Juda : Elihu, qui était des frères de David. De ceux d'Issacar : Omri, fils de Micaël.
\VS{19}De ceux de Zabulon : Jischemaeja, fils d'Abdias. De ceux de Nephthali : Jerimoth, fils d'Azriel.
\VS{20}Des fils d'Ephraïm : Hosée, fils d'Azazia. De la demi-tribu de Manassé : Joël, fils de Pedaja.
\VS{21}De l'autre demi-tribu de Manassé en Galaad : Jiddo, fils de Zacharie. De ceux de Benjamin : Jaasiel, fils d'Abner.
\VS{22}De ceux de Dan : Azareel, fils de Jerocham. Ce sont là les chefs des tribus d'Israël.
\TextTitle{Dénombrement arrêté par Yahweh}
\VS{23}Mais David ne fit point le dénombrement des Israélites, depuis l'âge de vingt ans et au-dessous ; parce que Yahweh avait dit qu'il multiplierait Israël comme les étoiles du ciel.
\VS{24}Joab, fils de Tseruja, avait bien commencé à en faire le dénombrement, mais il n'acheva pas parce que la colère de Dieu s'était répandue à cause de cela sur Israël ; c'est pourquoi ce dénombrement ne fut point mis parmi les dénombrements enregistrés dans les Chroniques du roi David.
\TextTitle{Les gestionnaires de David}
\VS{25}Or Azmaveth, fils d'Adiel, était commis sur les finances du roi ; mais Jonathan, fils d'Ozias, était commis sur les provisions dans les champs, dans les villes, les villages et les châteaux.
\VS{26}Et Ezri, fils de Kelub, était commis sur ceux qui travaillaient dans la campagne et cultivaient la terre.
\VS{27}Et Schimeï de Rama sur les vignes, et Zabdi de Schepham sur ce qui provenait des vignes, et sur les celliers du vin.
\VS{28}Et Baal-Hanan de Guéder sur les oliviers et sur les figuiers qui étaient à la campagne ; et Joasch sur les celliers à huile.
\VS{29}Schithraï de Saron était commis sur le gros bétail qui paissait en Saron ; Schaphath, fils d'Adlaï, sur le gros bétail qui paissait dans les vallées.
\VS{30}Obil, l'Ismaélite, sur les chameaux; Jechdia de Méronoth, sur les ânesses.
\VS{31}Jaziz, l'Hagarénien, sur les troupeaux du menu bétail. Tous ceux-là avaient la charge des biens qui appartenaient au roi David.
\TextTitle{Les conseillers de David}
\VS{32}Mais Jonathan, oncle de David, était conseiller, homme très intelligent et scribe ; et Jehiel, fils de Hacmoni, était avec les fils du roi.
\VS{33}Achitophel était le conseiller du roi ; et Huschaï, l'Arkien, était l'intime ami du roi.
\VS{34}Après Achitophel était Jehojada, fils de Benaja et Abiathar; et Joab était le chef de l'armée du roi.
\Chap{28}
\TextTitle{Dernières paroles de David, la royauté remise à Salomon\FTNTT{1 Ch. 23:2}}
\VerseOne{}David convoqua à Jérusalem tous les chefs d'Israël, les chefs des tribus, et les chefs des divisions qui servaient le roi ; et les chefs de milliers et de centaines, et ceux qui avaient la charge de tous les biens du roi, et de tout ce qu’il possédait, ses fils avec ses eunuques, et les hommes puissants, et tous les  héros et tous les hommes vaillants.
\VS{2}Puis le roi David se leva sur ses pieds, et dit : Mes frères et mon peuple, écoutez-moi ! J’avais à cœur de bâtir une maison de repos pour l’arche de l'alliance de Yahweh, et pour le marchepied de notre Dieu, et j'ai fait les préparatifs pour la bâtir.
\VS{3}Mais Dieu m'a dit : Tu ne bâtiras point de maison à mon Nom, parce que tu es un homme de guerre, et que tu as répandu beaucoup de sang.
\VS{4}Or comme Yahweh, le Dieu d'Israël, m'a choisi dans toute la maison de mon père pour être roi sur Israël à toujours ; car il a choisi Juda pour conducteur, et de la maison de Juda la maison de mon père, et d'entre les fils de mon père il a pris son plaisir en moi, pour me faire régner sur tout Israël.
\VS{5}Aussi, entre tous mes fils, car Yahweh m'a donné beaucoup de fils, il a choisi Salomon, mon fils, pour le faire asseoir sur le trône du royaume de Yahweh, sur Israël.
\VS{6}Et il m'a dit : Salomon, ton fils, est celui qui bâtira ma maison et mes parvis ; car je me le suis choisi pour fils et je serai pour lui un père.
\VS{7}Et j'affermirai son règne à toujours s'il s'applique à pratiquer mes commandements et à observer mes ordonnances, comme il le fait aujourd'hui.
\VS{8}Maintenant donc, je vous somme en présence de tout Israël, qui est l'assemblée de Yahweh, et devant notre Dieu qui l'entend, que vous ayez à garder et à rechercher diligemment tous les commandements de Yahweh, votre Dieu, afin que vous possédiez ce bon pays, et que vous le fassiez hériter à vos fils après vous, à jamais.
\VS{9}Et toi, Salomon, mon fils, connais le Dieu de ton père, et sers-le avec un cœur droit et une bonne volonté ; car Yahweh sonde tous les cœurs, et connaît toutes les dispositions des pensées. Si tu le cherches, il se laissera trouver par toi ; mais si tu l'abandonnes, il te rejettera pour toujours.
\VS{10}Considère maintenant que Yahweh t'a choisi pour bâtir une maison pour son sanctuaire. Fortifie-toi donc et applique-toi à y travailler.
\VS{11}David donna à Salomon, son fils, le modèle\FTNT{David donna le modèle du temple qu’il  avait reçu de Dieu à Salomon. Beaucoup veulent servir Dieu sans modèle, tandis que d’autres vont chercher des modèles dans le monde (2 R. 16:10-18). Nous devons faire l’œuvre de Dieu uniquement selon le modèle biblique.} du portique, de ses maisons, des chambres du trésor, des chambres hautes, des chambres intérieures et du lieu du propitiatoire.
\VS{12}Il lui donna le modèle de toutes les choses qui lui avaient été inspirées par l'Esprit qui était avec lui, pour les parvis de la maison de Yahweh, pour les chambres d'alentour, pour les trésors de la maison de Yahweh et pour les trésors des choses saintes ;
\VS{13}pour les divisions des sacrificateurs et des Lévites, pour toute l'œuvre du service de la maison de Yahweh, et pour tous les ustensiles du service de la maison de Yahweh.
\VS{14}Il lui donna aussi de l'or, un certain poids, pour les choses qui devaient être d'or, à savoir pour tous les ustensiles de chaque service ; et de l'argent, un certain poids, pour tous les ustensiles d'argent, à savoir pour tous les ustensiles de chaque service.
\VS{15}Le poids des chandeliers d'or, et de leurs lampes d'or, selon le poids de chaque chandelier et de ses lampes ; et le poids des chandeliers d'argent, selon le poids de chaque chandelier et de ses lampes, selon l’usage de chaque chandelier.
\VS{16}Et le poids de l'or pesant ce qu'il fallait pour chaque table des pains de proposition; et de l'argent pour les tables d'argent.
\VS{17}Il lui donna le modèle pour les fourchettes, pour les bassins et pour les calices d’or pur ; le modèle pour les coupes d'or, selon le poids de chaque coupe,  et de l'argent pour les coupes d'argent, selon le poids de chaque coupe ;
\VS{18}et le modèle pour l'autel des parfums en or épuré, avec le poids. Il lui donna encore le modèle du char, des chérubins d’or qui étendent les ailes et qui couvrent l’arche de l'alliance de Yahweh.
\VS{19}C’est par un écrit de sa main, dit-il, que Yahweh m’a donné l'intelligence de tout cela, de tous les ouvrages de ce modèle.
\TextTitle{David demande à Salomon de  bâtir le temple}
\VS{20}C'est pourquoi David dit à Salomon, son fils : Fortifie-toi, prends courage et travaille ; ne crains point et ne t'effraie point. Car Yahweh Dieu, mon Dieu, sera avec toi, il ne te délaissera point, et il ne t'abandonnera point, jusqu'à ce que tu aies achevé tout l'ouvrage du service de la maison de Yahweh.
\VS{21}Et voici, j'ai fait les divisions des sacrificateurs et des Lévites pour tout le service de la maison de Dieu ; et il y a avec toi pour tout cet ouvrage toutes sortes de gens prompts et experts, pour toutes sortes de services ; et les chefs avec tout le peuple seront prêts pour exécuter tout ce que tu diras.
\Chap{29}
\TextTitle{Offrandes volontaires de David et de tout le peuple}
\VerseOne{}Puis le roi David dit à toute l'assemblée : Dieu a choisi un seul de mes fils, à savoir Salomon, qui est encore jeune et délicat, et l'ouvrage est considérable, car ce palais n'est point pour un homme, mais pour Yahweh Dieu.
\VS{2}Et moi, j'ai préparé de toutes mes forces pour la maison de mon Dieu, de l'or pour les choses qui doivent être d'or, de l'argent pour celles qui doivent être d'argent, de l'airain pour celles d'airain, du fer pour celles de fer, du bois pour celles de bois, des pierres d'onyx, et des pierres pour être enchâssées, des pierres d'escarboucle, et des pierres de diverses couleurs, des pierres précieuses de toutes sortes, et du marbre en abondance.
\VS{3}Et outre cela, parce que j'ai une grande affection pour la maison de mon Dieu, je donne pour la maison de mon Dieu, outre toutes les choses que j'ai préparées pour la maison du sanctuaire, l'or et l'argent que j'ai entre mes plus précieux joyaux :
\VS{4}Trois mille talents d'or, de l'or d'Ophir, et sept mille talents d'argent affiné, pour revêtir les murailles de la maison ;
\VS{5}afin qu'il y ait de l'or partout où il faut de l'or, et de l'argent partout où il faut de l'argent ; et pour tout l'ouvrage qui se fera par la main des ouvriers. Or qui est celui d'entre vous qui se disposera volontairement à offrir aujourd'hui libéralement à Yahweh ?
\VS{6}Alors les chefs des maisons paternelles, les chefs des tribus d'Israël, les chefs de milliers et de centaines et les intendants du roi offrirent volontairement.
\VS{7}Ils donnèrent pour le service de la maison de Dieu cinq mille talents et dix mille drachmes d'or, dix mille talents d'argent, dix-huit mille talents d'airain, et cent mille talents de fer.
\VS{8}Ils mirent aussi les pierres que chacun avait, pour le trésor de la maison de Yahweh, entre les mains de Jehiel, le Guerschonite.
\VS{9}Et le peuple offrait avec joie volontairement, car ils offraient de tout leur cœur leurs offrandes volontaires à Yahweh; et David en eut une très grande joie.
\TextTitle{Prières de David}
\VS{10}Puis David bénit Yahweh en présence de toute l'assemblée, et dit : Ô Yahweh, Dieu d'Israël, notre père ! Tu es béni de tout temps et à toujours.
\VS{11}Ô Yahweh ! C’est à toi qu'appartient la magnificence, la puissance, la gloire, l'éternité, et la majesté; car tout ce qui est aux cieux et sur la terre est à toi, ô Yahweh ! Le règne est à toi, et tu t'élèves en souverain au-dessus de toutes choses !
\VS{12}Les richesses et les honneurs viennent de toi, et tu as la domination sur toutes choses ; la force et la puissance sont dans ta main, et il est aussi du pouvoir de ta main d'agrandir et de fortifier toutes choses.
\VS{13}Maintenant donc, ô notre Dieu ! Nous te célébrons et nous louons ton Nom glorieux.
\VS{14}Mais qui suis-je, et qui est mon peuple, que nous ayons assez pour pouvoir t’offrir ces choses volontairement ? Car toutes choses viennent de toi, et les ayant reçues de ta main, nous te les présentons.
\VS{15}Et même nous sommes devant toi des étrangers et des habitants, comme ont été tous nos pères ; et nos jours sont comme l'ombre sur la terre, et il n'y a point d’espérance.
\VS{16}Yahweh, notre Dieu,  toute cette abondance que nous avons préparée pour bâtir une maison à ton saint Nom, est de ta main, et toutes ces choses sont à toi.
\VS{17}Et je sais, ô mon Dieu, que c'est toi qui sondes les cœurs, et que tu prends plaisir à la droiture. C'est pourquoi j'ai volontairement offert d'un cœur droit toutes ces choses, et j'ai vu maintenant avec joie que ton peuple, qui se trouve ici, t'a fait son offrande volontairement.
\VS{18}Ô Yahweh ! Dieu d'Abraham, d'Isaac et d'Israël, nos pères, conserve à toujours dans le cœur de ton peuple, ces dispositions et ces pensées, et affermis leurs cœurs en toi.
\VS{19}Donne aussi un cœur droit à Salomon, mon fils, afin qu'il garde tes commandements, tes préceptes et tes lois, et qu'il fasse tout ce qui est nécessaire et qu'il bâtisse le palais que j'ai préparé.
\TextTitle{Sacrifices en l’honneur de Yahweh ; Salomon oint roi\FTNTT{1 Ch. 23:1 ; 1 R. 2:12 ; 1 R. 1:32-37}}
\VS{20}Après cela, David dit à toute l'assemblée : Bénissez maintenant Yahweh, votre Dieu ! Et toute l'assemblée bénit Yahweh, le Dieu de leurs pères. Ils s'inclinèrent et se prosternèrent devant Yahweh et devant le roi.
\VS{21}Et le lendemain, ils offrirent des sacrifices à Yahweh, et des holocaustes ; à savoir mille veaux, mille moutons, et mille agneaux, avec leurs libations ; et des sacrifices en grand nombre pour tous ceux d'Israël.
\VS{22}Et ils mangèrent et burent ce jour-là devant Yahweh avec une grande joie ; et ils établirent roi pour la seconde fois Salomon, fils de David, et l'oignirent en l'honneur de Yahweh pour être leur conducteur, et Tsadok pour sacrificateur.
\VS{23}Salomon s'assit donc sur le trône de Yahweh pour être roi à la place de David, son père. Il prospéra, car tout Israël lui obéit.
\VS{24}Et tous les chefs et les héros, et même tous les fils du roi David consentirent d'être les sujets du roi Salomon.
\VS{25}Ainsi, Yahweh éleva souverainement Salomon, à la vue de tout Israël, et lui donna une majesté royale telle qu'aucun roi avant lui n'en avait eue en Israël.
\TextTitle{Fin du règne de David ; sa mort\FTNTT{2 S. 5:4-5 ; 1 R. 2:10-12 ; 1 Ch. 3:4}}
\VS{26}David donc, fils d'Isaï, régna sur tout Israël.
\VS{27}Et les jours qu'il régna sur Israël furent quarante ans ; il régna sept ans à Hébron et trente-trois ans à Jérusalem.
\VS{28}Puis il mourut dans une heureuse vieillesse, rassasié de jours, de richesses, et de gloire. Et Salomon, son fils, régna à sa place.
\VS{29}Les actions du roi David, tant les premières que les dernières, sont écrites dans le livre de Samuel le voyant, dans le livre de Nathan le prophète, et dans le livre de Gad le prophète,
\VS{30}avec tout son règne, ses exploits et ce qui se passa de son temps, tant sur Israël que sur tous les royaumes du territoire.
\PPE{}
\end{multicols}

\clearpage\ShortTitle{2 Chroniques}\BookTitle{2 Chroniques}\BFont
\noindent\hrulefill
{\footnotesize
\textit{
\bigskip
{\centering{}
\\Auteur : Inconnu
\\(Heb. : Hayyamim dibre)
\\Signification : Actes des journées
\\Thème : La grandeur de Juda
\\Date de rédaction : 5\up{ème} siècle av. J.-C.\\}
}
%\bigskip
\textit{
\\Initialement, 1 et 2 Chroniques ne constituaient qu'un seul ouvrage. Ce livre raconte le règne de Salomon, la construction de la maison de Dieu et du palais. Il reprend ensuite l'histoire des royaumes d'Israël et de Juda, du schisme à la captivité babylonienne, mettant en exergue l'instabilité du peuple dont le cœur balançait entre Yahweh et les idoles.\bigskip
}
}
\par\nobreak\noindent\hrulefill
\begin{multicols}{2}
\Chap{1}
\TextTitle{Yahweh élève Salomon qui demande la sagesse\FTNTT{1 R. 2:12 ; 3:4-9 ; 1 Ch. 29:23-25}}
\VerseOne{}Or Salomon, fils de David, se fortifia dans son royaume ; Yahweh, son Dieu, fut avec lui, et l'éleva au plus haut.
\VS{2}Salomon parla à tout Israël, aux chefs de milliers et de centaines, aux juges et à tous les principaux de tout Israël, chefs des pères.
\VS{3}Salomon et toute l'assemblée avec lui allèrent au haut lieu qui était à Gabaon ; car là était la tente d'assignation de Dieu, que Moïse, serviteur de Yahweh, avait faite dans le désert.
\VS{4}Mais David avait fait monter l'arche de Dieu de Kirjath-Jearim au lieu qu'il avait préparé ; car il lui avait dressé une tente à Jérusalem.
\VS{5}L'autel d'airain que Betsaleel, fils d'Uri, fils de Hur, avait fait, était là devant le tabernacle de Yahweh. Et Salomon et l'assemblée y cherchèrent Yahweh\FTNT{Ex. 27:1-8 ; Ex. 36:1-2.}.
\VS{6}Salomon offrit là, devant Yahweh, mille holocaustes, sur l'autel d'airain qui était devant la tente d'assignation.
\VS{7}En cette nuit-là, Dieu apparut à Salomon, et lui dit : Demande ce que tu veux que je te donne.
\VS{8}Et Salomon répondit à Dieu : Tu as usé d'une grande bienveillance envers David, mon père, et tu m'as établi roi à sa place.
\VS{9}Maintenant, ô Yahweh Dieu ! Que ta parole à David, mon père, se confirme ; car tu m'as établi roi sur un peuple nombreux comme la poussière de la terre.
\VS{10}Donne-moi donc maintenant de la sagesse et de l'intelligence, afin que je sache me conduire devant ce peuple ; car qui pourrait juger ton peuple, ce peuple si grand ?
\TextTitle{Yahweh agrée la prière de Salomon et l'exauce\FTNTT{1 R. 3:10-28}}
\VS{11}Et Dieu dit à Salomon : Puisque c'est là ce qui est dans ton cœur, et que tu n'as demandé ni des richesses, ni des biens, ni de la gloire, ni la mort de ceux qui te haïssent, ni même des jours nombreux, mais que tu as demandé pour toi de la sagesse et de l'intelligence, afin de pouvoir juger mon peuple, sur lequel je t'ai établi roi,
\VS{12}la sagesse et l'intelligence te sont données. Je te donnerai aussi des richesses, des biens et de la gloire, comme n'en ont pas eu les rois qui ont été avant toi, et comme il n'en aura aucun après toi.
\VS{13}Puis Salomon s'en retourna à Jérusalem, du haut lieu qui était à Gabaon devant la tente d'assignation ; et il régna sur Israël.
\VS{14}Salomon rassembla des chars et des cavaliers ; il avait quatorze cents chars et douze mille cavaliers ; et il les plaça dans les villes où il tenait ses chars, et auprès du roi, à Jérusalem.
\VS{15}Et le roi fit que l'argent et l'or étaient aussi communs à Jérusalem que les pierres, et les cèdres que les sycomores de la plaine.
\VS{16}Le lieu d'où étaient issus les chevaux de Salomon était l'Egypte ; une caravane de marchands du roi allait les prendre par troupe à un prix convenu.
\VS{17}On faisait monter et sortir d'Egypte un char pour six cents sicles d'argent, et un cheval pour cent cinquante. On en amenait de même par eux pour tous les rois des Héthiens, et pour les rois de Syrie.
\Chap{2}
\TextTitle{La prière de Salomon exaucée\FTNTT{1 R. 5:1-18 ; 7:13,14}}
\VerseOne{}Or Salomon ordonna de bâtir une maison au Nom de Yahweh, ainsi qu'une maison royale.
\VS{2}Et il fit un dénombrement de soixante et dix mille hommes qui portaient les fardeaux, et de quatre vingt mille qui coupaient le bois sur la montagne, et de trois mille six cents qui étaient commis sur eux.
\VS{3}Puis Salomon envoya vers Huram, roi de Tyr, pour lui dire : Fais pour moi comme tu as fait pour David, mon père, à qui tu as envoyé des cèdres, pour se bâtir une maison afin d'y habiter.
\VS{4}Voici, je vais bâtir une maison au Nom de Yahweh, mon Dieu, pour la lui consacrer, pour faire brûler devant lui le parfum des aromates, pour présenter continuellement devant lui les pains de proposition, et pour offrir les holocaustes du matin et du soir, des sabbats, des nouvelles lunes, et des fêtes de Yahweh, notre Dieu, ce qui est perpétuel en Israël.
\VS{5}La maison que je vais bâtir sera grande ; car notre Dieu est plus grand que tous les dieux.
\VS{6}Mais qui aurait le pouvoir de lui bâtir une maison, puisque les cieux et les cieux des cieux ne sauraient le contenir ? Et qui suis-je pour lui bâtir une maison, si ce n'est pour faire brûler des parfums devant sa face ?
\VS{7}Maintenant, envoie-moi un homme habile pour travailler l'or, l'argent, l'airain et le fer, en écarlate, en cramoisi et en pourpre, sachant faire des sculptures, pour travailler avec les hommes habiles que j'ai avec moi en Juda et à Jérusalem, et que David, mon père, a préparés.
\VS{8}Envoie-moi aussi du Liban du bois de cèdre, de cyprès et de santal ; car je sais que tes serviteurs savent couper les bois du Liban. Voici, mes serviteurs seront avec les tiens.
\VS{9}Qu'on me prépare du bois en grande quantité ; car la maison que je vais bâtir sera grande et magnifique.
\VS{10}Et je donnerai à tes serviteurs qui couperont, qui abattront les bois, vingt mille cors de froment foulé, vingt mille cors d'orge, vingt mille baths de vin, et vingt mille baths d'huile.
\VS{11}Huram, roi de Tyr, répondit dans un écrit qu'il envoya à Salomon : C'est parce que Yahweh aime son peuple qu'il t'a établi roi sur eux.
\VS{12}Et Huram dit : Béni soit Yahweh, le Dieu d'Israël, qui a fait les cieux et la terre, de ce qu'il a donné au roi David un fils sage, prudent et intelligent, qui va bâtir une maison à Yahweh, et une maison royale !
\VS{13}Je t'envoie donc un homme habile et intelligent, Huram-Abi,
\VS{14}fils d'une femme d'entre les filles de Dan, et d'un père tyrien. Il sait travailler l'or, l'argent, l'airain et le fer, les pierres et le bois, en écarlate, en pourpre, en fin lin et en cramoisi ; il sait faire toutes sortes de sculptures et imaginer toutes sortes d'objets d'art qu'on lui donne à faire. Il travaillera avec tes hommes habiles et avec les hommes habiles de mon seigneur David, ton père.
\VS{15}Et maintenant, que mon seigneur envoie à ses serviteurs le froment, l'orge, l'huile et le vin comme il l'a dit.
\VS{16}Et nous couperons des bois du Liban autant que tu en auras besoin, et nous te les amènerons en radeaux, par la mer, jusqu'à Japho, et tu les feras monter à Jérusalem.
\VS{17}Alors Salomon compta tous les hommes étrangers qui étaient au pays d'Israël, d'après le dénombrement que David, son père, en avait fait. On en trouva cent cinquante-trois mille six cents.
\VS{18}Et il en établit soixante-dix mille qui portaient des fardeaux, quatre-vingt mille qui taillaient les pierres dans la montagne, et trois mille six cents surveillants pour faire travailler le peuple.
\Chap{3}
\TextTitle{Salomon commence la construction du temple\FTNTT{1 R. 6:1}}
\VerseOne{}Salomon commença donc à bâtir la maison de Yahweh à Jérusalem, sur la montagne de Morija, qui avait été indiquée à David, son père, au lieu même que David avait préparé dans l'aire d'Ornan, le Jébusien.
\VS{2}Il commença à bâtir, le second jour du second mois, la quatrième année de son règne.
\TextTitle{Les matériaux du temple et les dimensions\FTNTT{1 R. 6:2-38 ; 7:13-22}}
\VS{3}Or voici les fondements fixés par Salomon pour bâtir la maison de Dieu : La longueur, en coudées de l'ancienne mesure, était de soixante coudées, et la largeur de vingt coudées.
\VS{4}Le portique qui était sur le devant, et dont la longueur répondait à la largeur de la maison, avait vingt coudées, et cent vingt de hauteur. Il le revêtit intérieurement d'or pur.
\VS{5}Et il recouvrit la grande maison de bois de cyprès ; il la revêtit d'or fin, et y fit mettre des palmes et des chaînettes.
\VS{6}Il revêtit la maison de pierres précieuses, pour l'ornement ; et l'or était de l'or de Parvaïm.
\VS{7}Il revêtit d'or la maison, les poutres, les seuils, les parois et les portes ; et il fit sculpter des chérubins sur les parois.
\VS{8}Il fit le Saint des saints, dont la longueur était de vingt coudées, selon la largeur de la maison, et la largeur de vingt coudées ; et il le couvrit d'or fin, pour une valeur de six cents talents.
\VS{9}Et le poids des clous montait à cinquante sicles d'or. Il revêtit aussi d'or les chambres hautes.
\VS{10}Il fit dans le Saint des saints deux chérubins sculptés, et on les couvrit d'or ;
\VS{11}La longueur des ailes des chérubins était de vingt coudées. L'aile du premier, longue de cinq coudées, touchait la paroi de la maison, et l'autre aile, longue de cinq coudées, touchait une aile de l'autre chérubin.
\VS{12}Et une aile de l'autre chérubin, longue de cinq coudées, touchait la paroi de la maison ; et l'autre aile longue de cinq coudées, joignait l'aile de l'autre chérubin.
\VS{13}Les ailes étendues de ces chérubins faisaient vingt coudées. Ils se tenaient debout sur leurs pieds, leurs faces tournées vers la maison.
\VS{14}Il fit le voile de pourpre, d'écarlate, de cramoisi et de fin lin: Et il y représenta par dessus des chérubins.
\VS{15}Devant la maison, il fit deux colonnes de trente-cinq coudées de hauteur, et le chapiteau sur leur sommet était de cinq coudées.
\VS{16}Il fit des chaînes dans le sanctuaire ; et il en mit sur le sommet des colonnes ; et il fit cent grenades qu'il mit aux chaînes.
\VS{17}Il dressa les colonnes sur le devant du temple, l'une à droite, et l'autre à gauche ; il appela celle de droite Jakin, et celle de gauche Boaz.
\Chap{4}
\TextTitle{L'autel d'airain, la mer de fonte et les ustensiles du temple\FTNTT{1 R. 7:23-50}}
\VerseOne{}Il fit aussi un autel d'airain\FTNT{Voir l'annexe « Le temple de Salomon - extérieur »} long de vingt coudées, large de vingt coudées, et haut de dix coudées.
\VS{2}Il fit la mer de fonte de dix coudées d'un bord à l'autre, ronde tout autour, et haute de cinq coudées, et une circonférence que mesurait un cordon de trente coudées.
\VS{3}Des figures de bœufs l'entouraient en dessous, dix par coudée, faisant tout le tour de la mer ; il y avait deux rangées de bœufs fondus avec elle en une seule pièce.
\VS{4}Elle était posée sur douze bœufs, dont trois tournés vers le nord, trois tournés vers l'occident, trois tournés vers le sud, et trois tournés vers l'orient. La mer était sur eux, et toute la partie postérieure de leur corps était en dedans.
\VS{5}Son épaisseur était d'une paume ; et son bord était comme le bord d'une coupe en fleur de lis. Elle avait une contenance de trois mille baths\FTNT{Ex 25 ; Ex 27.}.
\VS{6}Il fit aussi dix cuves, et en mit cinq à droite et cinq à gauche, pour servir à la purification. On y lavait ce qui appartenait aux holocaustes, et la mer servait aux sacrificateurs pour s'y laver.
\VS{7}Il fit dix chandeliers d'or, d'après l'ordonnance, et les mit dans le temple, cinq à droite et cinq à gauche.
\VS{8}Il fit aussi dix tables, et il les mit dans le temple, cinq à droite et cinq à gauche. Il fit cent coupes d'or.
\VS{9}Il fit encore le parvis des sacrificateurs, le grand parvis et des portes pour ce parvis, et couvrit d'airain ces portes.
\VS{10}Il mit la mer du côté droit, vers l'orient, face au sud-est.
\VS{11}Et Huram fit les cuves, les pelles et les bassins. Huram acheva de faire l'ouvrage qu'il faisait pour le roi Salomon dans la maison de Dieu :
\VS{12}Deux colonnes, les bourrelets et les deux chapiteaux sur le sommet des colonnes ; les deux maillages pour couvrir les deux bourrelets des chapiteaux sur le sommet des colonnes ;
\VS{13}et les quatre cents grenades pour les deux maillages, deux rangs de grenades à chaque maille, pour couvrir les deux bourrelets des chapiteaux sur le sommet des colonnes.
\VS{14}Il fit aussi les bases, et il fit les cuves sur les bases ;
\VS{15}la mer et les douze bœufs sous elle ;
\VS{16}les pots, les pelles et les fourchettes et tous leurs ustensiles ; Huram-Abi les fit au roi Salomon, pour la maison de Yahweh, en airain poli.
\VS{17}Le roi les fit fondre dans la plaine du Jourdain, dans une terre grasse, entre Succoth et Tseréda.
\VS{18}Et Salomon fit tous ces ustensiles en si grand nombre qu'on ne rechercha point le poids de l'airain.
\VS{19}Salomon fit encore tous les ustensiles\FTNT{Voir l'annexe « Le temple de Salomon - intérieur »} qui étaient dans la maison de Yahweh : L'autel d'or, et les tables sur lesquelles on mettait le pain de proposition ;
\VS{20}les chandeliers et leurs lampes d'or fin, qu'on devait allumer devant le sanctuaire, selon l'ordonnance ;
\VS{21}les fleurs, les lampes, et les mouchettes d'or, d'un or parfaitement pur ;
\VS{22}et les mouchettes, les bassins, les tasses et les encensoirs d'or fin. Quant à l'entrée de la maison, les portes intérieures conduisant dans le Saint des saints, et les portes de la maison pour entrer au temple étaient d'or.
\Chap{5}
\TextTitle{L'arche dans le sanctuaire, Yahweh manifeste sa gloire\FTNTT{1 R. 7:51-8:11}}
\VerseOne{}Ainsi fut achevé tout l'ouvrage que Salomon fit pour la maison de Yahweh. Puis Salomon fit apporter ce que David, son père, avait consacré : L'argent, l'or et tous les ustensiles ; et il les mit dans les trésors de la maison de Dieu.
\VS{2}Alors Salomon assembla à Jérusalem les anciens d'Israël, et tous les chefs des tribus, les chefs des pères des fils d'Israël, pour transporter de la ville de David, qui est Sion, l'arche de l'alliance de Yahweh.
\VS{3}Et tous les hommes d'Israël s'assemblèrent auprès du roi pour la fête ; c'était le septième mois.
\VS{4}Tous les anciens d'Israël vinrent, et les Lévites portèrent l'arche.
\VS{5}Ils transportèrent l'arche, la tente d'assignation, et tous les ustensiles sacrés qui étaient dans la tente ; les sacrificateurs et les Lévites les emportèrent.
\VS{6}Or le roi Salomon et toute l'assemblée d'Israël réunie auprès de lui étaient devant l'arche, sacrifiant du menu et du gros bétail en si grand nombre qu'on ne pouvait ni dénombrer ni compter.
\VS{7}Les sacrificateurs portèrent l'arche de l'alliance de Yahweh à sa place, dans le sanctuaire de la maison, dans le Saint des saints, sous les ailes des chérubins.
\VS{8}Les chérubins étendaient les ailes sur l'endroit où devait être l'arche, et les chérubins couvraient l'arche et ses barres par-dessus.
\VS{9}Les barres avaient une longueur telle que leurs extrémités se voyaient en avant de l'arche, devant le sanctuaire ; mais elles ne se voyaient point du dehors. Et l'arche a été là jusqu'à ce jour.
\VS{10}Il n'y avait dans l'arche que les deux tables que Moïse y avait mises en Horeb, quand Yahweh traita alliance avec les enfants d'Israël à leur sortie d'Egypte.
\VS{11}Or il arriva que comme les sacrificateurs sortaient du lieu saint (car tous les sacrificateurs présents s'étaient sanctifiés, sans observer l'ordre des classes),
\VS{12}etque tous les Lévites qui étaient chantres, Asaph, Héman, Jeduthun, leurs fils et leurs frères, vêtus de fin lin, avec des cymbales, des luths et des harpes, se tenaient à l'orient de l'autel ; et il y avait avec eux cent vingt sacrificateurs sonnant des trompettes.
\VS{13}Il arriva, dis-je, que comme un seul homme, ceux qui sonnaient des trompettes et ceux qui chantaient firent entendre leur voix d'un même accord, pour célébrer et pour louer Yahweh, et firent retentir le son des trompettes, des cymbales et d'autres instruments de musique, et ils célébrèrent Yahweh, en disant : Car il est bon, car sa miséricorde demeure à toujours\FTNT{Jé. 33:11 ; Ps. 118:29 ;  Ps. 136} ! Il arriva que la maison de Yahweh fut remplie d'une nuée.
\VS{14}Les sacrificateurs ne purent s'y tenir pour faire le service, à cause de la nuée ; car la gloire de Yahweh remplissait la maison de Dieu.
\Chap{6}
\TextTitle{Salomon s'adresse à l'assemblée d'Israël\FTNTT{1 R. 8:12-21}}
\VerseOne{}Alors Salomon dit : Yahweh a dit qu'il habiterait dans l'obscurité\FTNT{Nous avons ici une prophétie concernant la venue du Messie. Dieu, qui est lumière, a accepté d'habiter dans les ténèbres afin de nous sauver (Mt. 4:16 ; Jn. 1:5).}.
\VS{2}Et moi, j'ai bâti une maison qui sera ta demeure, et un domicile afin que tu y résides à toujours !
\VS{3}Puis le roi tourna son visage, et bénit toute l'assemblée d'Israël ; et toute l'assemblée d'Israël était debout.
\VS{4}Et il dit : Béni soit Yahweh, le Dieu d'Israël, qui de sa bouche a parlé à David, mon père, et qui par sa main puissante accomplit ce qu'il avait déclaré en disant :
\VS{5}Depuis le jour où j'ai fait sortir mon peuple du pays d'Egypte, je n'ai point choisi de ville entre toutes les tribus d'Israël pour y bâtir une maison afin que mon Nom y réside, et je n'ai point choisi d'homme pour être chef de mon peuple d'Israël.
\VS{6}Mais j'ai choisi Jérusalem pour que mon Nom y réside, et j'ai choisi David pour qu'il règne sur mon peuple d'Israël.
\VS{7}Or David, mon père, avait à cœur de bâtir une maison au Nom de Yahweh, le Dieu d'Israël.
\VS{8}Mais Yahweh parla à David, mon père : Puisque tu as eu à cœur de bâtir une maison à mon Nom, tu as bien fait d'avoir eu cette intention.
\VS{9}Seulement, ce n'est pas toi qui bâtiras cette maison ; mais ce sera ton fils, qui sortira de tes entrailles, qui bâtira cette maison à mon Nom.
\VS{10}Yahweh a accompli la parole qu'il avait déclarée ; j'ai succédé à David, mon père, et je me suis assis sur le trône d'Israël, comme Yahweh l'avait dit, et j'ai bâti cette maison au Nom de Yahweh, le Dieu d'Israël.
\VS{11}J'y ai mis l'arche où est l'alliance de Yahweh, qu'il traita avec les enfants d'Israël.
\TextTitle{Prière de Salomon\FTNTT{1 R. 8:22-61}}
\VS{12}Puis il se plaça devant l'autel de Yahweh, en face de toute l'assemblée d'Israël, et il étendit ses mains.
\VS{13}Car Salomon avait fait une tribune d'airain, et il l'avait mise au milieu du grand parvis ; elle était longue de cinq coudées, large de cinq coudées, et haute de trois coudées. Il s'y plaça, se mit à genoux en face de toute l'assemblée d'Israël, et étendant ses mains vers les cieux, il dit :
\VS{14}Ô Yahweh, Dieu d'Israël ! Il n'y a ni dans les cieux ni sur la terre de Dieu semblable à toi, qui gardes l'alliance et la miséricorde envers tes serviteurs qui marchent de tout leur cœur devant ta face.
\VS{15}Toi qui as tenu parole à ton serviteur David, mon père. Ce que tu lui avais promis, et ce que tu as déclaré de ta bouche, tu l'as accompli de ta main puissante, comme il paraît aujourd'hui.
\VS{16}Maintenant, ô Yahweh, Dieu d'Israël ! Tiens la parole que tu as faite à ton serviteur David, mon père, en disant : Tu ne manqueras jamais devant moi d'un successeur assis sur le trône d'Israël, pourvu que tes fils prennent garde à leur voie pour marcher dans ma loi, comme tu as marché devant ma face.
\VS{17}Et maintenant, ô Yahweh, Dieu d'Israël ! Que ta parole, que tu as déclarée à David, ton serviteur, soit confirmée !
\VS{18}Mais Dieu habiterait-il véritablement sur la terre avec les hommes ? Voici, les cieux, même les cieux des cieux, ne peuvent te contenir, combien moins cette maison que j'ai bâtie !
\VS{19}Toutefois, ô Yahweh, mon Dieu, aie égard à la prière de ton serviteur et à sa supplication, pour écouter le cri et la prière que ton serviteur t'adresse.
\VS{20}Que tes yeux soient ouverts jour et nuit sur cette maison, sur le lieu où tu as promis de mettre ton Nom ! Ecoute la prière que ton serviteur te fait en ce lieu.
\VS{21}Exauce les supplications de ton serviteur et de ton peuple d'Israël, quand ils prieront en ce lieu. Exauce des cieux, du lieu de ta demeure ; exauce et pardonne !
\VS{22}Si quelqu'un pèche contre son prochain, et qu'on lui impose un serment pour le faire jurer, et qu'il vient prêter serment devant ton autel, dans cette maison ;
\VS{23}écoute-le des cieux, agis et juge tes serviteurs, en donnant au méchant son salaire, et fais retomber sa conduite sur sa tête, en justifiant le juste, et lui rendant selon sa justice.
\VS{24}Quand ton peuple d'Israël sera battu par l'ennemi, pour avoir péché contre toi ; s'ils retournent à toi, s'ils donnent gloire à ton Nom, s'ils t'adressent dans cette maison des prières et des supplications ;
\VS{25}toi, exauce-les des cieux, et pardonne le péché de ton peuple d'Israël, et ramène-les dans la terre que tu leur as donnée à eux et à leurs pères.
\VS{26}Quand les cieux seront fermés, et qu'il n'y aura point de pluie, parce qu'ils auront péché contre toi ; s'ils prient en ce lieu, s'ils donnent gloire à ton Nom, et s'ils se détournent de leurs péchés, parce que tu les auras affligés ;
\VS{27}toi, exauce-les des cieux, et pardonne le péché de tes serviteurs et de ton peuple d'Israël, après que tu leur auras enseigné le bon chemin, par lequel ils doivent marcher ; et envoie de la pluie sur la terre que tu as donnée en héritage à ton peuple.
\VS{28}Quand il y aura dans le pays la famine ou la peste, quand il y aura la rouille, la nielle, les sauterelles d'une espèce ou d'une autre, quand les ennemis les assiégeront dans leur pays, dans leurs portes, ou qu'il y aura un fléau, une maladie quelconque ;
\VS{29}si un homme, si tout ton peuple d'Israël fait entendre des prières et des supplications, et que chacun reconnaît sa plaie et sa douleur, et étend ses mains vers cette maison ;
\VS{30}exauce-le des cieux, du lieu de ta demeure, et pardonne. Rends à chacun selon toutes ses voies, toi qui connais leur cœur ; car seul tu connais le cœur des fils des hommes ;
\VS{31}afin qu'ils te craignent, pour marcher dans tes voies, tout le temps qu'ils vivront sur la terre que tu as donnée à nos pères.
\VS{32}Et l'étranger, qui ne sera pas de ton peuple d'Israël, mais qui viendra d'un pays éloigné, à cause de ton grand Nom, de ta main puissante, et de ton bras étendu ; quand il viendra prier dans cette maison,
\VS{33}exauce-le des cieux, du lieu de ta demeure, et accorde tout ce que cet étranger réclamera de toi ; afin que tous les peuples de la terre connaissent ton Nom pour te craindre comme ton peuple d'Israël, et sachent que ton Nom est invoqué sur cette maison que j'ai bâtie.
\VS{34}Quand ton peuple sortira en guerre contre ses ennemis, par la voie par laquelle tu l'auras envoyé ; s'ils te prient, en regardant vers cette ville que tu as choisie, et vers cette maison que j'ai bâtie à ton Nom,
\VS{35}exauce des cieux leur prière et leur supplication, et fais-leur droit.
\VS{36}Quand ils pécheront contre toi, car il n'y a point d'homme qui ne pèche, et qu'irrité contre eux, tu les auras livrés à leurs ennemis, et que ceux qui les auront pris les auront emmenés captifs en quelque pays, soit éloigné soit proche ;
\VS{37}si dans le pays où ils seront captifs, ils rentrent en eux-mêmes et s'ils se repentent, s'ils t'adressent des supplications dans le pays de leur captivité, en disant : Nous avons péché, nous avons commis l'iniquité, nous avons agi méchamment !
\VS{38}S'ils retournent à toi de tout leur cœur et de toute leur âme, dans le pays de leur captivité où ils ont été emmenés captifs, et s'ils t'adressent des prières, les regards tournés vers leur pays que tu as donné à leurs pères, vers cette ville que tu as choisie, et vers cette maison que j'ai bâtie à ton Nom ;
\VS{39}exauce des cieux, du lieu de ta demeure, leurs prières et leurs supplications, et fais-leur droit ; pardonne à ton peuple qui aura péché contre toi !
\VS{40}Maintenant, ô mon Dieu, que tes yeux soient ouverts et que tes oreilles soient attentives à la prière qu'on te fera en ce lieu !
\VS{41}Et maintenant, Yahweh Dieu ! Lève-toi, viens au lieu de ton repos, toi et l'arche de ta puissance. Yahweh Dieu, que tes sacrificateurs soient revêtus du salut, et que tes bien-aimés se réjouissent du bien que tu leur fais !
\VS{42}Yahweh Dieu, ne repousse la face pas ton oint ; souviens-toi des grâces accordées à David, ton serviteur.
\Chap{7}
\TextTitle{Yahweh répond par le feu : Sa gloire remplit la maison}
\VerseOne{}Lorsque Salomon eut achevé de prier, le feu descendit du ciel et consuma l'holocauste et les sacrifices\FTNT{Lé 9:24 ; 1 R 18:38.} ; et la gloire de Yahweh remplit la maison.
\VS{2}Les sacrificateurs ne pouvaient entrer dans la maison de Yahweh, parce que la gloire de Yahweh avait rempli la maison de Yahweh.
\VS{3}Tous les enfants d'Israël virent descendre le feu et la gloire de Yahweh sur la maison ; et ils se courbèrent, le visage contre terre, sur le pavé, se prosternèrent et louèrent Yahweh, en disant : Car il est bon, car sa miséricorde demeure éternellement !
\TextTitle{Salomon et le peuple offrent des sacrifices à Yahweh\FTNTT{1 R. 8:62-66}}
\VS{4}Or le roi et tout le peuple offraient des sacrifices devant Yahweh.
\VS{5}Le roi Salomon offrit un sacrifice de vingt-deux mille bœufs, et cent vingt mille brebis. Ainsi, le roi et tout le peuple firent la dédicace de la maison de Dieu.
\VS{6}Les sacrificateurs se tenaient à leurs fonctions, ainsi que les Lévites, avec les instruments de musique de Yahweh, que le roi David avait faits pour louer Yahweh en disant : Car sa miséricorde demeure éternellement ; ayant les Psaumes de David entre leurs mains. Et les sacrificateurs sonnaient des trompettes vis-à-vis d'eux, et tout Israël se tenait debout.
\VS{7}Salomon consacra le milieu du parvis, qui est devant la maison de Yahweh ; car il offrit là les holocaustes et les graisses des sacrifices d'offrande de paix\FTNT{Voir commentaire en Lé. 3:1.}, parce que l'autel d'airain que Salomon avait fait ne pouvait contenir les holocaustes, les offrandes et les graisses.
\VS{8}Ainsi Salomon célébra, en ce temps-là, la fête pendant sept jours, avec tout Israël. Il y avait une grande multitude, venue depuis l'entrée d'Hamath jusqu'au torrent d'Egypte.
\VS{9}Le huitième jour, ils firent une assemblée solennelle ; car ils firent la dédicace de l'autel pendant sept jours, et la fête pendant sept jours.
\VS{10}Le vingt-troisième jour du septième mois, il laissa aller le peuple dans ses tentes, se réjouissant et ayant le cœur plein de joie, à cause du bien que Yahweh avait fait à David, à Salomon, et à Israël, son peuple.
\TextTitle{Yahweh apparaît à Salomon\FTNTT{1 R. 9:1-9}}
\VS{11}Salomon acheva donc la maison de Yahweh et la maison du roi ; et Salomon réussit dans tout ce qui lui vint à cœur de faire dans la maison de Yahweh et dans sa maison.
\VS{12}Yahweh apparut à Salomon pendant la nuit, et lui dit : J'exauce ta prière, et je choisis ce lieu comme une maison de sacrifices.
\VS{13}Quand je fermerai les cieux, et qu'il n'y aura point de pluie, et quand j'ordonnerai aux sauterelles de consumer le pays, et quand j'enverrai la peste parmi mon peuple ;
\VS{14}si mon peuple, sur lequel mon Nom est invoqué, s'humilie, prie, et cherche ma face, et s'il se détourne de ses mauvaises voies, alors je l'exaucerai des cieux, je pardonnerai ses péchés, et je guérirai son pays.
\VS{15}Mes yeux seront désormais ouverts, et mes oreilles seront attentives à la prière faite en ce lieu.
\VS{16}Maintenant je choisis et je sanctifie cette maison, afin que mon Nom y soit à toujours ; mes yeux et mon cœur seront toujours là.
\VS{17}Et toi, si tu marches devant moi comme David, ton père, a marché, faisant tout ce que je t'ai ordonné, et si tu gardes mes lois et mes ordonnances,
\VS{18}j'affermirai le trône de ton royaume, comme je l'ai déclaré à David, ton père, en disant : Il ne te manquera point de successeur qui règne en Israël.
\VS{19}Mais si vous vous détournez, et si vous abandonnez mes lois et mes commandements que je vous ai prescrits, et si vous allez servir d'autres dieux et vous prosterner devant eux,
\VS{20}je vous arracherai de mon pays que je vous ai donné, je rejetterai loin de moi cette maison que j'ai consacrée à mon Nom, et j'en ferai un sujet de sarcasmes et de moqueries parmi tous les peuples.
\VS{21}Et quiconque passera près de cette maison qui aura été élevée, sera dans l'étonnement et dira : Pourquoi Yahweh a-t-il ainsi traité ce pays et cette maison?
\VS{22}Et on répondra : Parce qu'ils ont abandonné Yahweh, le Dieu de leurs pères, qui les a fait sortir du pays d'Egypte, et qu'ils se sont attachés à d'autres dieux, et qu'ils se sont prosternés devant eux, et les ont servis ; à cause de cela, il a fait venir sur eux tous ces maux.
\Chap{8}
\TextTitle{Les réalisations de Salomon\FTNTT{1 R. 9:15-28 ; 10:26-29}}
\VerseOne{}Au bout de vingt ans, pendant lesquels Salomon bâtit la maison de Yahweh et sa propre maison,
\VS{2}il bâtit les villes que Huram lui avait données et y fit habiter les enfants d'Israël.
\VS{3}Puis Salomon marcha contre Hamath de Tsoba, et la conquit.
\VS{4}Il bâtit Thadmor au désert, et toutes les villes servant de magasins qu'il bâtit dans le pays de Hamath.
\VS{5}Il bâtit Beth-Horon la haute, et Beth-Horon la basse, villes fortes de murailles, de portes et de barres ;
\VS{6}Baalath, et toutes les villes servant de magasins qu'avait Salomon, toutes les villes pour les chars, les villes pour la cavalerie, et tout ce que Salomon prit plaisir à bâtir à Jérusalem, au Liban, et dans tout le pays de sa domination.
\VS{7}Tout le peuple qui était resté des Héthiens, des Amoréens, des Phéréziens, des Héviens et des Jébusiens, qui n'étaient point d'Israël ;
\VS{8}leurs descendants, qui étaient restés après eux dans le pays, et que les enfants d'Israël n'avaient pas détruits, Salomon les leva comme des gens de corvée jusqu'à ce jour.
\VS{9}Salomon n'employa comme esclave pour ses travaux aucun des fils d'Israël ; car ils étaient des hommes de guerre, les chefs de ses officiers, les chefs de ses chars et de ses hommes d'armes.
\VS{10}Voici le nombre des chefs de ceux qui étaient préposés aux travaux du roi Salomon : Ils étaient deux cent cinquante, ayant autorité sur le peuple.
\VS{11}Salomon fit monter la fille de Pharaon de la cité de David dans la maison qu'il lui avait bâtie ; car il dit : Ma femme n'habitera point dans la maison de David, roi d'Israël, parce que les lieux où l'arche de Yahweh est entrée sont saints.
\VS{12}Alors Salomon offrit des holocaustes à Yahweh, sur l'autel de Yahweh qu'il avait bâti devant le portique.
\VS{13}Il offrait chaque jour ce qui était prescrit par Moïse pour les sabbats, pour les nouvelles lunes, et pour les fêtes, trois fois l'année, à la fête des pains sans levain, à la fête des semaines, et à la fête des tabernacles\FTNT{Ex. 14:17 ; Lé. 23:1-44.}.
\VS{14}Il établit, selon l'ordonnance de David, son père, les classes des sacrificateurs selon leur fonction, et les Lévites selon leurs charges, pour célébrer Yahweh et pour faire, jour par jour, le service en présence des sacrificateurs ; et les portiers, selon leurs classes, à chaque porte ; car tel était le commandement de David, homme de Dieu.
\VS{15}Et on ne s'écarta pas du commandement du roi à l'égard des Sacrificateurs et des Lévites, en aucune chose, ni à l'égard les trésors.
\VS{16}Ainsi fut préparé tout l'ouvrage de Salomon, jusqu'au jour de la fondation de la maison de Yahweh et jusqu'à ce qu'elle fut terminée. La maison de Yahweh fut donc achevée.
\VS{17}Alors Salomon alla à Etsjon-Guéber et à Eloth, sur le rivage de la mer, dans le pays d'Edom.
\VS{18}Et Huram lui envoya, sous la conduite de ses serviteurs, des navires et des serviteurs connaissant la mer. Ils allèrent avec les serviteurs de Salomon à Ophir, et ils y prirent quatre cent cinquante talents d'or, qu'ils apportèrent au roi Salomon.
\Chap{9}
\TextTitle{La reine de Séba chez Salomon\FTNTT{1 R. 10:1-13}}
\VerseOne{}Or la reine de Séba, ayant appris la renommée de Salomon, vint à Jérusalem pour éprouver Salomon par des énigmes. Elle avait une suite très nombreuse, et des chameaux portant des aromates, de l'or en grande quantité et des pierres précieuses. Elle vint auprès de Salomon, et elle lui parla de tout ce qu'elle avait dans le cœur.
\VS{2}Salomon lui expliqua tout ce qu'elle lui proposa ; il n'y eut rien que Salomon n'entendît et qu'il ne sût lui expliquer.
\VS{3}Alors, la reine de Séba vit toute la sagesse de Salomon, et la maison qu'il avait bâtie,
\VS{4}les mets de sa table, la demeure de ses serviteurs, l'ordre de service et les vêtements de ceux qui le servaient, ses échansons et leurs vêtements, et les marches par où l'on montait à la maison de Yahweh, et elle fut toute ravie hors d'elle-même.
\VS{5}Elle parla ainsi au roi : Ce que j'ai entendu dire dans mon pays de tes actions et de ta sagesse était donc vrai !
\VS{6}Je ne croyais pas ce qu'on en disait avant d'être venue et que mes yeux ne l'aient vu ; et voici, on ne m'avait pas rapporté la moitié de la grandeur de ta sagesse ; tu surpasses la rumeur que j'avais entendue.
\VS{7}Heureux tes gens ! Heureux tes serviteurs qui se tiennent continuellement devant toi, et qui entendent ta sagesse !
\VS{8}Béni soit Yahweh, ton Dieu, qui a pris plaisir en toi pour te placer sur son trône comme roi pour Yahweh, ton Dieu ! C'est parce que ton Dieu aime Israël et veut le faire subsister à jamais, qu'il t'a établi roi sur eux pour faire droit et justice.
\VS{9}Puis elle donna au roi cent vingt talents d'or, une très grande quantité d'aromates, et des pierres précieuses ; et il n'y eut plus d'aromates tels que ceux que la reine de Séba donna au roi Salomon.
\VS{10}Les serviteurs de Huram et les serviteurs de Salomon, qui amenèrent de l'or d'Ophir, amenèrent aussi du bois de santal et des pierres précieuses.
\VS{11}Le roi fit de ce bois de santal les chemins qui allaient à la maison de Yahweh et à la maison du roi, et des harpes et des luths pour les chantres. On n'en avait point vu auparavant de semblable dans le pays de Juda.
\VS{12}Le roi Salomon donna à la reine de Séba tout ce qu'elle désira, ce qu'elle demanda, plus qu'elle n'avait apporté au roi ; et elle s'en retourna, revint dans son pays, elle et ses serviteurs.
\TextTitle{Les richesses de Salomon\FTNTT{cp. 1 R. 4:1-34}}
\VS{13}Le poids de l'or qui arrivait à Salomon chaque année était de six cent soixante-six talents d'or,
\VS{14}outre ce qu'il retirait des négociants et des marchands qui en apportaient, et de tous les rois d'Arabie et des gouverneurs de ces pays-là, qui apportaient de l'or et de l'argent à Salomon.
\VS{15}Le roi Salomon fit deux cents grands boucliers d'or battu, employant six cents sicles d'or battu pour chaque bouclier ;
\VS{16}et trois cents autres boucliers plus petits d'or battu, employant trois cents sicles d'or pour chaque bouclier ; et le roi les mit dans la maison de la forêt du Liban.
\VS{17}Le roi fit aussi un grand trône d'ivoire, qu'il couvrit d'or pur.
\VS{18}Ce trône avait six marches et un marchepied d'or qui était accolé au trône ; et il avait des accoudoirs de l'un et de l'autre côté du siège ; et deux lions se tenaient auprès des accoudoirs.
\VS{19}Douze lions se tenaient là sur les six marches de part et d'autre. Rien de pareil n'avait été fait pour aucun royaume.
\VS{20}Et toutes les coupes à boire du roi Salomon étaient d'or, et toute la vaisselle de la maison de la forêt du Liban était d'or pur ; rien n'était d'argent ; on n'en faisait aucun cas du temps de Salomon.
\VS{21}Car les navires du roi allaient à Tarsis avec les serviteurs de Huram ; et une fois tous les trois ans arrivaient les navires de Tarsis, apportant de l'or, de l'argent, des dents d'éléphants, des singes et des paons.
\VS{22}Le roi Salomon fut plus grand que tous les rois de la terre, tant en richesses qu'en sagesse.
\VS{23}Tous les rois de la terre cherchaient à voir la face de Salomon, pour écouter la sagesse que Dieu avait mise dans son cœur.
\VS{24}Et chacun d'eux apportait son présent : Des ustensiles d'argent, des ustensiles d'or, des vêtements, des armes, des aromates, des chevaux et des mulets, et il en était ainsi année après année.
\VS{25}Salomon avait quatre mille écuries pour ses chevaux, avec des chars ; et douze mille cavaliers qu'il plaça dans les villes où il avait des chars et auprès du roi à Jérusalem.
\VS{26}Il dominait sur tous les rois depuis le fleuve jusqu'au pays des Philistins, et jusqu'à la frontière d'Egypte.
\VS{27}Et le roi fit que l'argent était aussi commun à Jérusalem que les pierres, et les cèdres aussi nombreux que les sycomores qui sont dans les plaines.
\VS{28}On tirait des chevaux pour Salomon de l'Egypte et de tous les pays.
\TextTitle{Mort de Salomon\FTNTT{1 R. 11:1-40}}
\VS{29}Le reste des actions de Salomon, les premières et les dernières, cela n'est-il pas écrit dans le livre de Nathan le prophète, dans la prophétie d'Achija de Silo, et dans la vision de Jéedo le voyant, touchant Jéroboam, fils de Nebath ?
\VS{30}Salomon régna quarante ans à Jérusalem sur tout Israël.
\VS{31}Puis Salomon s'endormit avec ses pères, et on l'ensevelit dans la cité de David, son père ; et Roboam, son fils, régna à sa place.
\Chap{10}
\TextTitle{Roboam règne sur Israël\FTNTT{1 R. 12:1-15}}
\VerseOne{}Roboam se rendit à Sichem, car tout Israël était venu à Sichem pour l'établir roi.
\VS{2}Quand Jéroboam, fils de Nebath, qui était en Egypte, où il s'était enfui de devant le roi Salomon, l'eut appris, il revint d'Egypte.
\VS{3}Or on l'envoya appeler. Ainsi Jéroboam et tout Israël vinrent et parlèrent à Roboam, en disant :
\VS{4}Ton père a mis sur nous un joug pesant. Allège maintenant cette rude servitude de ton père, et ce joug pesant qu'il a mis sur nous, et nous te servirons.
\VS{5}Alors il leur dit : Revenez vers moi dans trois jours. Et le peuple s'en alla.
\VS{6}Le roi Roboam demanda conseil aux vieillards qui avaient été auprès de Salomon, son père, pendant sa vie, et il leur parla ainsi : Comment, et quelle chose me conseillez-vous de répondre à ce peuple ?
\VS{7} Et ils lui répondirent en ces termes : Si tu es bon envers ce peuple, si tu es bienveillant envers eux, et que tu leur dises de bonnes paroles, ils seront tes serviteurs à toujours.
\VS{8}Mais il laissa le conseil que les vieillards lui avaient donné, et il demanda conseil aux jeunes gens qui avaient grandi avec lui, et qui se tenaient auprès de lui.
\VS{9}Et il leur dit : Que me conseillez-vous de répondre à ce peuple qui m'a parlé en disant : Allège le joug que ton père a mis sur nous ?
\VS{10}Et les jeunes gens qui avaient grandi avec lui, lui parlèrent en disant : Tu répondras en disant à ce peuple qui t'a parlé et t'a dit : Ton père a mis sur nous un joug pesant, mais toi, allège-le ; tu leur répondras donc : Mon petit doigt est plus gros que les reins de mon père.
\VS{11}Or mon père a mis sur vous un joug pesant, mais moi, je rendrai votre joug encore plus pesant. Mon père vous a châtiés avec des fouets, mais moi, je vous châtierai avec des scorpions.
\TextTitle{Roboam délaisse le conseil des anciens}
\VS{12}Trois jours après, Jéroboam, avec tout le peuple, vint vers Roboam, suivant ce qu'avait dit le roi : Revenez vers moi dans trois jours.
\VS{13}Mais le roi leur répondit durement. Le roi Roboam délaissa le conseil des anciens,
\VS{14}et leur parla suivant le conseil des jeunes gens, en disant : Mon père a mis sur vous un joug pesant ; mais moi, j'y ajouterai encore. Mon père vous a châtiés avec des fouets ; mais moi, je vous châtierai avec des scorpions.
\VS{15}Le roi n'écouta donc point le peuple ; cela était conduit par Dieu, afin que Yahweh accomplisse la parole qu'il avait déclarée par Achija de Silo, à Jéroboam, fils de Nebath.
\TextTitle{Israël se détache de la maison de David\FTNTT{1 R. 12:16-19}}
\VS{16}Quand tout Israël vit que le roi ne les écoutait pas, le peuple répondit au roi, en disant : Quelle part avons-nous avec David ? Nous n'avons point d'héritage avec le fils d'Isaï. Israël, chacun à ses tentes ! Et toi David, pourvois maintenant à ta maison. Ainsi, tout Israël s'en alla dans ses tentes.
\VS{17}Mais quant aux enfants d'Israël qui habitaient les villes de Juda, Roboam régna sur eux.
\VS{18}Alors le roi Roboam envoya Hadoram, qui était préposé aux impôts ; mais les enfants d'Israël le lapidèrent à coups de pierres et il mourut. Et le roi Roboam se hâta de monter sur un char pour s'enfuir à Jérusalem.
\VS{19}C'est ainsi qu'Israël s'est rebellé contre la maison de David, jusqu'à ce jour\FTNT{1 R. 12:16-19.}.
\Chap{11} 
\TextTitle{Yahweh interdit la guerre entre Juda et Israël\FTNTT{1 R. 12:21-24}}
\VerseOne{}Roboam, étant arrivé à Jérusalem, assembla la maison de Juda et de Benjamin, cent quatre-vingt mille hommes d'élite et de guerre, afin de combattre contre Israël, pour le ramener sous le règne de Roboam.
\VS{2}Mais la parole de Yahweh fut adressée à Schemaeja, homme de Dieu, en ces termes :
\VS{3}Parle à Roboam, fils de Salomon, roi de Juda, et à ceux d'Israël qui sont en Juda et en Benjamin, et dis-leur :
\VS{4}Ainsi parle Yahweh : Ne montez point, et ne combattez point contre vos frères. Retournez chacun dans sa maison ; c'est par moi que cette chose est arrivée. Et ils obéirent aux paroles de Yahweh, et ils s'en retournèrent sans aller contre Jéroboam\FTNT{1 R. 12:21-24.}.
\VS{5}Roboam demeura donc à Jérusalem, et il bâtit des villes fortes en Juda.
\VS{6}Il bâtit Bethléhem, Etham, Tekoa,
\VS{7}Beth-Tsur, Soco, Adullam,
\VS{8}Gath, Maréscha, Ziph,
\VS{9}Adoraïm, Lakis, Azéka,
\VS{10}Tsorea, Ajalon et Hébron, qui étaient en Juda et en Benjamin, et en fit des villes fortes.
\VS{11}Il les fortifia et y mit des gouverneurs, des provisions de vivres, d'huile et de vin.
\VS{12}Dans chacune de ces villes, il mit des boucliers et des lances, et il les rendit puissantes. Ainsi Juda et Benjamin lui furent soumis.
\TextTitle{Les sacrificateurs et les Lévites soutiennent Roboam}
\VS{13}Les sacrificateurs et les Lévites, qui étaient dans tout Israël, vinrent de toutes leurs contrées se joindre à lui.
\TextTitle{Jéroboam abandonne Yahweh\FTNTT{1 R. 12:26-30 ; 14:7-8}}
\VS{14}Car les Lévites abandonnèrent leurs faubourgs et leurs possessions et vinrent en Juda et à Jérusalem, parce que Jéroboam et ses fils les avaient rejetés des fonctions de sacrificateurs pour Yahweh.
\VS{15}Car il s'était établi des sacrificateurs pour les hauts lieux, pour les boucs, et pour les veaux qu'il avait faits.
\VS{16}Et à leur suite, ceux d'entre toutes les tribus d'Israël qui avaient appliqué leur cœur à chercher Yahweh, le Dieu d'Israël, vinrent à Jérusalem pour sacrifier à Yahweh, le Dieu de leurs pères.
\VS{17}Ils fortifièrent le royaume de Juda et affermirent Roboam, fils de Salomon, pendant trois ans ; car on suivit les voies de David et de Salomon pendant trois ans.
\TextTitle{Les femmes et les enfants de Roboam}
\VS{18}Or Roboam prit pour femme : Mahalath, fille de Jerimoth, fils de David et d'Abichaïl, fille d'Eliab, fils d'Isaï.
\VS{19}Elle lui enfanta des fils : Jeusch, Schemaria et Zaham.
\VS{20}Après elle, il prit Maaca, fille d'Absalom, qui lui enfanta Abija, Attaï, Ziza et Schelomith.
\VS{21}Roboam aima Maaca, fille d'Absalom, plus que toutes ses femmes et ses concubines. Car il prit dix-huit femmes et soixante concubines, et il engendra vingt-huit fils et soixante filles.
\VS{22}Roboam établit pour chef Abija, fils de Maaca, comme prince entre ses frères ; car il voulait le faire roi.
\VS{23}Il agit prudemment et dispersa tous ses fils dans toutes les contrées de Juda et de Benjamin, dans toutes les villes fortes ; il leur donna de quoi vivre en abondance, et demanda pour eux une multitude de femmes.
\Chap{12}
\TextTitle{Roboam affermi, il abandonne Yahweh\FTNTT{1 R. 14:21-24}}
\VerseOne{}Lorsque la royauté de Roboam fut affermie et qu'il eut acquis de la force, il abandonna la loi de Yahweh, et tout Israël avec lui\FTNT{1 R. 14:21-29.}.
\TextTitle{Yahweh veut livrer Juda à Schischak\FTNTT{1 R. 14:25-28}}
\VS{2}C'est pourquoi il arriva que la cinquième année du Roi Roboam, Schischak, roi d'Egypte, monta contre Jérusalem, parce qu'ils avaient péché contre Yahweh.
\VS{3}Il avait mille deux cents chars et soixante mille cavaliers, et le peuple qui vint avec lui d'Egypte, des Libyens, des Sukkiens et des Ethiopiens, était innombrable.
\VS{4}Il prit les villes fortes qui appartenaient à Juda, et vint jusqu'à Jérusalem.
\VS{5}Alors Schemaeja, le prophète, vint vers Roboam et les chefs de Juda, qui s'étaient assemblés à Jérusalem à cause de Schischak, et leur dit : Ainsi parle Yahweh : Vous m'avez abandonné ; moi aussi je vous abandonne aux mains de Schischak.
\VS{6}Alors les chefs d'Israël et le roi s'humilièrent, et dirent : Yahweh est juste !
\VS{7}Et quand Yahweh vit qu'ils s'humiliaient, la parole de Yahweh fut adressée à Schemaeja, et il lui dit : Ils se sont humiliés ; je ne les détruirai pas, mais je leur donnerai dans peu de temps un moyen d'échapper, et ma fureur ne se répandra point sur Jérusalem par la main de Schischak.
\VS{8}Toutefois, ils lui seront asservis, afin qu'ils sachent ce que c'est que de me servir ou de servir les royaumes de la terre.
\VS{9}Schischak, roi d'Egypte, monta donc contre Jérusalem, et prit les trésors de la maison de Yahweh et les trésors de la maison du roi ; il prit tout. Il prit les boucliers d'or que Salomon avait faits.
\VS{10}Le roi Roboam fit des boucliers d'airain à leur place, et il les mit entre les mains des chefs des coureurs qui gardaient la porte de la maison du roi.
\VS{11}Et toutes les fois que le roi entrait dans la maison de Yahweh, les coureurs venaient et les portaient ; puis ils les rapportaient dans la chambre des coureurs.
\VS{12}Ainsi comme il s'était humilié, la colère de Yahweh se détourna de lui, et ne le détruisit pas entièrement ; car il y avait encore de bonnes choses en Juda.
\TextTitle{Mort de Roboam\FTNTT{1 R. 14:21,29,31}}
\VS{13}Le roi Roboam se fortifia donc dans Jérusalem, et régna. Il avait quarante et un ans quand il devint roi, et il régna dix-sept ans à Jérusalem, la ville que Yahweh avait choisie de toutes les tribus d'Israël, pour y mettre son Nom. Sa mère s'appelait Naama, l'Ammonite.
\VS{14}Il fit le mal, car il ne disposa point son cœur pour chercher Yahweh.
\VS{15}Or les actions de Roboam, les premières et les dernières, ne sont-elles pas écrites dans les livres de Schemaeja le prophète, et d'Iddo le voyant, parmi les registres généalogiques ? Les guerres entre Roboam et Jéroboam furent continuelles.
\VS{16}Roboam s'endormit avec ses pères, et il fut enseveli dans la cité de David ; et Abija, son fils, régna à sa place.
\Chap{13}
\TextTitle{Abija règne sur Juda ; guerre entre Israël et Juda\FTNTT{1 R. 15:1-8}}
\VerseOne{}La dix-huitième année du roi Jéroboam, Abija commença à régner sur Juda.
\VS{2}Il régna trois ans à Jérusalem. Sa mère s'appelait Micaja, fille d'Uriel, de Guibea. Or il y eut guerre entre Abija et Jéroboam.
\VS{3}Abija engagea la guerre avec une armée de vaillants guerriers, quatre cent mille hommes d'élite ; et Jéroboam se rangea en bataille contre lui avec huit cent mille hommes d'élite, forts et vaillants.
\VS{4}Et Abija se leva du haut de la montagne de Tsemaraïm, parmi les montagnes d'Ephraïm, et dit : Jéroboam et tout Israël, écoutez-moi!
\VS{5}Ne savez-vous pas que Yahweh, le Dieu d'Israël, a donné pour toujours la royauté sur Israël à David, à lui et à ses fils, par une alliance de sel\FTNT{Sel : Voir commentaire en Lé. 2:13} !
\VS{6}Mais Jéroboam, fils de Nebath, serviteur de Salomon, fils de David, s'est élevé et s'est rebellé contre son seigneur.
\VS{7}Et des gens sans valeur, des fils de Belial, se sont assemblés avec lui et se sont fortifiés contre Roboam, fils de Salomon. Or Roboam était un jeune homme craintif et sans force devant eux.
\VS{8}Et maintenant, vous vous dites être forts devant la royauté de Yahweh, qui est aux mains des fils de David ; vous êtes une multitude, et vous avez avec vous les veaux d'or que Jéroboam vous a faits pour dieux.
\VS{9}N'avez-vous pas rejeté les sacrificateurs de Yahweh, les fils d'Aaron, et les Lévites ? Et ne vous êtes-vous pas faits des sacrificateurs comme les peuples des autres pays ? Quiconque venait, avec un jeune taureau et sept béliers pour être consacré, devenait sacrificateur de ce qui n'est pas Dieu.
\VS{10}Mais quant à nous, Yahweh est notre Dieu, et nous ne l'avons pas abandonné ; les sacrificateurs qui font le service de Yahweh sont fils d'Aaron, et ce sont les Lévites qui tiennent cette fonction.
\VS{11}Nous faisons brûler pour Yahweh, chaque matin et chaque soir, les holocaustes et le parfum d'aromates. Les pains de proposition sont rangés sur la table pure, et on allume le chandelier d'or avec ses lampes, chaque soir. Car nous gardons ce que Yahweh, notre Dieu,veut qu'on garde ; mais vous, vous l'avez abandonné.
\VS{12}Voici, Dieu est avec nous pour être notre chef, avec ses sacrificateurs, et les trompettes retentissantes, pour les faire sonner contre vous. Fils d'Israël, ne combattez pas contre Yahweh, le Dieu de vos pères ; car cela ne vous réussira pas.
\VS{13}Mais Jéroboam fit une embuscade par un détour, et arriva derrière eux. De sorte que les Israélites étaient en face de Juda, qui avait l'embuscade par-derrière.
\VS{14}Ceux de Juda se retournèrent et voici ils avaient la bataille par-devant et par-derrière. Alors ils crièrent à Yahweh, et les sacrificateurs sonnèrent des trompettes.
\TextTitle{Victoire de Juda sur Israël}
\VS{15}Les hommes de Juda poussèrent un cri, et au cri de guerre des hommes de Juda, Yahweh frappa Jéroboam et tout Israël devant Abija et Juda.
\VS{16}Les fils d'Israël s'enfuirent devant ceux de Juda, parce que Dieu les livra entre leurs mains.
\VS{17}Abija et son peuple leur firent un grand un carnage, et il tomba d'Israël cinq cent mille hommes d'élite blessés à mort.
\VS{18}Ainsi, les enfants d'Israël furent humiliés en ce temps-là ; et les enfants de Juda devinrent plus forts, parce qu'ils s'étaient appuyés sur Yahweh, le Dieu de leurs pères.
\VS{19} Abija poursuivit Jéroboam, et lui prit ces villes : Béthel et les villes de son ressort, Jeschana et les villes de son ressort, Ephron et les villes de son ressort.
\TextTitle{Mort de Jéroboam\FTNTT{1 R. 14:19,20}}
\VS{20}Et Jéroboam n'eut plus de force durant le temps d'Abija ; et Yahweh le frappa, et il mourut.
\TextTitle{Les femmes et les fils d'Abija\FTNTT{1 R. 15:7-8}}
\VS{21}Mais Abija se fortifia ; il prit quatorze femmes, et engendra vingt-deux fils et seize filles.
\VS{22}Le reste des actions d'Abija, sa conduite et ses paroles sont écrites dans les mémoires du prophète Iddo.
\VS{23}Abija s'endormit avec ses pères, et on l'ensevelit dans la cité de David ; et Asa, son fils, régna à sa place. De son temps, le pays fut en repos pendant dix ans.
\Chap{14}
\TextTitle{Asa règne sur Juda, il rétablit l'ordre de Yahweh\FTNTT{1 R. 15:11}}
\VerseOne{}Asa fit ce qui est bon et droit aux yeux de Yahweh, son Dieu.
\VS{2}Il ôta les autels étrangers et les hauts lieux ; il brisa les statues et mit en pièces les idoles d'Asherah.
\VS{3}Et il recommanda à Juda de rechercher Yahweh, le Dieu de leurs pères, et de pratiquer la loi et les commandements.
\VS{4}Il ôta de toutes les villes de Juda les hauts lieux et les colonnes consacrées au soleil. Et le royaume fut en repos devant lui.
\VS{5}Il bâtit des villes fortes en Juda, car le pays fut en repos. Et pendant ces années-là, il n'y eut point de guerre contre lui, parce que Yahweh lui donna du repos.
\VS{6}Et il dit à Juda : Bâtissons ces villes, et entourons-les de murailles, de tours, de portes et de barres ; le pays est encore devant nous, parce que nous avons recherché Yahweh, notre Dieu. Nous l'avons recherché, et il nous a donné du repos de toutes parts. Ainsi, ils bâtirent et prospérèrent.
\TextTitle{Asa s'appuie sur Yahweh et triomphe de Zérach\FTNTT{2 Ch. 16:1-10}}
\VS{7}Or Asa avait dans son armée trois cent mille hommes de Juda, portant le grand bouclier et la lance, et deux cent quatre-vingt mille de Benjamin, portant le bouclier et tirant de l'arc, tous vaillants guerriers.
\VS{8}Mais Zérach, l'Ethiopien, sortit contre eux avec une armée d'un million d'hommes, et de trois cents chars ; et il vint jusqu'à Maréscha.
\VS{9}Asa alla au-devant de lui, et ils se rangèrent en bataille dans la vallée de Tsephata, près de Maréscha.
\VS{10}Alors Asa cria à Yahweh, son Dieu, et dit : Yahweh ! Toi seul peux nous secourir, que l'on soit nombreux ou sans force ! Aide-nous, Yahweh, notre Dieu ! Car nous nous appuyons sur toi, et nous sommes venus en ton Nom contre cette multitude. Tu es Yahweh, notre Dieu : Que l'homme ne prévale pas contre toi !
\VS{11}Et Yahweh frappa les Ethiopiens devant Asa et devant Juda ; et les Ethiopiens s'enfuirent.
\VS{12}Asa et le peuple qui était avec lui les poursuivirent jusqu'à Guérar, et tant d'Ethiopiens tombèrent sans pouvoir sauver leur vie ; car ils furent brisés devant Yahweh et son armée, et on emporta un très grand butin.
\VS{13}Ils frappèrent aussi toutes les villes autour de Guérar, car la terreur de Yahweh était sur eux ; et ils pillèrent toutes ces villes, car il s'y trouvait un grand butin.
\VS{14}Ils frappèrent aussi les tentes des troupeaux, et emmenèrent des brebis et des chameaux en abondance ; puis ils retournèrent à Jérusalem.
\Chap{15}
\TextTitle{Azaria le prophète avertit Asa}
\VerseOne{}Alors l'Esprit de Dieu fut sur Azaria, fils d'Oded.
\VS{2}Et il sortit au-devant d'Asa, et lui dit : Asa, et tout Juda et Benjamin, écoutez-moi ! Yahweh est avec vous quand vous êtes avec lui. Si vous le cherchez, vous le trouverez ; mais si vous l'abandonnez, il vous abandonnera.
\VS{3}Pendant longtemps Israël a été sans vrai Dieu, sans sacrificateur qui l'enseignait, et sans loi.
\VS{4}Mais dans leur détresse, ils sont revenus vers Yahweh, le Dieu d'Israël ; ils l'ont cherché, et ils l'ont trouvé\FTNT{Ps. 107:19-20.}.
\VS{5}Dans ces temps-là, il n'y avait point de sûreté pour ceux qui allaient et venaient, car il y avait de grands troubles parmi tous les habitants du pays.
\VS{6}Une nation était écrasée par une autre nation, et une ville par une autre ville ; car Dieu les agitait par toutes sortes d'angoisses.
\VS{7}Mais vous, fortifiez-vous, et que vos mains ne se relâchent pas ; car il y a une récompense pour vos œuvres.
\TextTitle{Asa écoute les paroles d'Azaria\FTNTT{1 R. 15:12-15}}
\VS{8}Or dès qu'Asa eut entendu ces paroles et la prophétie d'Oded le prophète, il se fortifia ; et fit disparaître les abominations de tout le pays de Juda et de Benjamin, et des villes qu'il avait prises dans les montagnes d'Ephraïm ; et il rétablit l'autel de Yahweh, qui était devant le portique de Yahweh.
\VS{9}Puis il assembla tout Juda et Benjamin, et ceux d'Ephraïm, de Manassé et de Siméon, qui habitaient avec eux ; car un grand nombre de gens d'Israël passaient à lui, voyant que Yahweh, son Dieu, était avec lui.
\VS{10}Ils s'assemblèrent donc à Jérusalem, le troisième mois de la quinzième année du règne d'Asa ;
\VS{11}et ils sacrifièrent ce jour-là à Yahweh sept cents bœufs et sept mille brebis, du butin qu'ils avaient amené.
\VS{12}Et ils rentrèrent dans l'alliance pour chercher Yahweh, le Dieu de leurs pères, de tout leur cœur et de toute leur âme ;
\VS{13}de sorte qu'on devait faire mourir quiconque ne rechercherait pas Yahweh, le Dieu d'Israël, petit ou grand, homme ou femme.
\VS{14}Et ils jurèrent à Yahweh, à haute voix, avec des cris de joie, et au son des shofars et des cors.
\VS{15}Tout Juda se réjouit de ce serment, parce qu'ils avaient juré de tout leur cœur et qu'ils avaient recherché Yahweh de leur plein gré, et qu'ils l'avaient trouvé. Et Yahweh leur donna du repos de toutes parts.
\VS{16}Le roi Asa destitua même sa mère, Maaca, de son rang de reine, parce qu'elle avait fait une idole pour Astarté. Asa abattit l'idole, l'écrasa et la brûla près du torrent de Cédron.
\VS{17}Mais les hauts lieux ne furent point ôtés du milieu d'Israël. Néanmoins, le cœur d'Asa fut intègre tout le long de ses jours.
\VS{18}Il remit dans la maison de Dieu les choses que son père avait consacrées, avec ce qu'il avait lui-même consacré, l'argent, l'or et les ustensiles.
\VS{19}Et il n'y eut point de guerre jusqu'à la trente-cinquième année du règne d'Asa.
\Chap{16}
\TextTitle{Alliance d'Asa et du roi de Syrie contre le roi d'Israël\FTNTT{ 1 R. 15:16-22 ; cp. 1 R. 15:27 ; 16:7}}
\VerseOne{}La trente-sixième année du règne d'Asa, Baescha, roi d'Israël, monta contre Juda, et il bâtit Rama, pour empêcher quiconque de sortir et d'entrer vers Asa, roi de Juda.
\VS{2}Alors Asa sortit de l'argent et de l'or des trésors de la maison de Yahweh et de la maison royale, et il envoya dire à Ben-Hadad, roi de Syrie, qui habitait à Damas :
\VS{3}Il y a alliance entre nous, et entre mon père et ton père ; voici, je t'envoie de l'argent et de l'or ; va, romps l'alliance que tu as avec Baescha, roi d'Israël, afin qu'il s'éloigne de moi.
\VS{4}Ben-Hadad écouta le roi Asa, et il envoya les chefs de son armée contre les villes d'Israël, et ils frappèrent Ijjon, Dan, Abel-Maïm, et tous les magasins des villes de Nephthali.
\VS{5}Et aussitôt que Baescha l'apprit, il cessa de bâtir Rama et suspendit ses travaux.
\VS{6}Alors le roi Asa prit avec lui tout Juda, et ils emportèrent les pierres et le bois de Rama, que Baescha faisait bâtir ; et il en bâtit Guéba et Mitspa.
\TextTitle{Hanani condamne l'alliance d'Asa}
\VS{7}En ce temps-là, Hanani le voyant, vint vers Asa, roi de Juda, et lui dit : Parce que tu t'es appuyé sur le roi de Syrie, et que tu ne t'es point appuyé sur Yahweh, ton Dieu, l'armée du roi de Syrie a échappé de ta main.
\VS{8}Les Ethiopiens et les Libyens n'étaient-ils pas une grande armée, ayant des chars et une multitude de cavaliers ? Mais parce que tu t'étais appuyé sur Yahweh, il les livra entre tes mains.
\VS{9}Car les yeux de Yahweh parcourent toute la terre, pour soutenir ceux dont le cœur est tout entier à lui. Tu as agi follement en cela ; car désormais tu auras des guerres.
\VS{10}Asa fut irrité contre le voyant, et le mit en prison, car il était indigné contre lui à ce sujet. Asa opprima aussi, en ce temps-là, quelques-uns du peuple.
\TextTitle{Mort d'Asa\FTNTT{1 R. 15:23-24}}
\VS{11}Or voici, les actions d'Asa, les premières et les dernières, sont écrites dans le livre des rois de Juda et d'Israël.
\VS{12}Asa fut malade des pieds la trente-neuvième année de son règne, et sa maladie fut très grave. Toutefois, il ne chercha point Yahweh dans sa maladie, mais les médecins.
\VS{13}Puis Asa s'endormit avec ses pères, et il mourut la quarante et unième année de son règne.
\VS{14}On l'ensevelit dans le sépulcre qu'il s'était creusé dans la cité de David. On le coucha dans un lit qui était rempli de parfums et d'aromates, composés par le travail d'un parfumeur ; et on lui en brûla une quantité considérable.
\Chap{17}
\TextTitle{Josaphat règne sur Juda, il recherche Yahweh\FTNTT{1 R. 15:24}}
\VerseOne{}Josaphat son fils régna à sa place et se fortifia contre Israël.
\VS{2}Il mit des troupes dans toutes les villes fortes de Juda, et des garnisons dans le pays de Juda, et dans les villes d'Ephraïm qu'Asa, son père, avait prises.
\VS{3}Yahweh fut avec Josaphat, parce qu'il suivit les premières voies de David, son père, et qu'il ne rechercha point les Baals ;
\VS{4}car il rechercha le Dieu de son père, et il marcha dans ses commandements, et non pas selon ce que faisait Israël.
\VS{5}Yahweh affermit donc le royaume entre ses mains ; et tout Juda apportait des présents à Josaphat, et il eut en abondance des richesses et de la gloire.
\VS{6}Son cœur grandit dans les voies de Yahweh, et il ôta encore de Juda les hauts lieux et les idoles d'Astarté.
\VS{7}Puis, la troisième année de son règne, il envoya ses chefs Ben-Haïl, Abdias, Zacharie, Nethaneel et Michée, pour enseigner dans les villes de Juda ;
\VS{8}et avec eux les Lévites Schemaeja, Nethania, Zebadia, Asaël, Schemiramoth, Jonathan, Adonija, Tobija et Tob-Adonija, Lévites, et avec eux Elischama et Joram, les sacrificateurs.
\VS{9}Ils enseignèrent dans Juda, ayant avec eux le livre de la loi de Yahweh. Ils firent le tour de toutes les villes de Juda, et enseignèrent parmi le peuple.
\TextTitle{Affermissement du règne de Josaphat}
\VS{10}La terreur de Yahweh fut sur tous les royaumes des pays qui entouraient Juda, et ils ne firent point la guerre à Josaphat.
\VS{11}On apporta aussi à Josaphat des présents de la part des Philistins, et un impôt en argent ; et les Arabes lui amenèrent aussi du bétail, sept mille sept cents béliers et sept mille sept cents boucs.
\VS{12}Ainsi Josaphat s'élevait jusqu'au plus haut degré de gloire. Et il bâtit en Juda des châteaux et des villes pour servir de magasins.
\VS{13}Il fit de grands travaux dans les villes de Juda ; et il avait à Jérusalem des gens de guerre puissants et vaillants.
\VS{14}Voici leur dénombrement, selon les maisons de leurs pères. Les chefs de milliers de Juda furent Adna le chef, avec trois cent mille vaillants guerriers.
\VS{15}Et après lui, Jochanan le chef, avec deux cent quatre-vingt mille hommes.
\VS{16}A ses côtés, Amasia, fils de Zicri, qui s'était volontairement offert à Yahweh, avec deux cent mille vaillants guerriers.
\VS{17}De Benjamin, Eliada, vaillant guerrier, avec deux cent mille hommes, armés d'arcs et de boucliers,
\VS{18}à côté de lui Zozabad, avec cent quatre-vingt mille hommes équipés pour le combat.
\VS{19}Tels sont ceux qui étaient au service du roi, outre ceux que le roi avait placés dans toutes les villes fortes de Juda.
\Chap{18}
\TextTitle{Josaphat s'allie à Achab contre les Syriens\FTNTT{1 R. 22:2-4}}
\VerseOne{}Or Josaphat, ayant beaucoup de richesses et de gloire, s'allia par mariage avec Achab.
\VS{2}Et au bout de quelques années, il descendit vers Achab, à Samarie. Achab tua pour lui, et pour le peuple qui était avec lui, un grand nombre de brebis et de bœufs, et l'incita à monter contre Ramoth de Galaad\FTNT{1 R. 22:2-40.}.
\VS{3}Achab, roi d'Israël, dit à Josaphat, roi de Juda : Viendras-tu avec moi contre Ramoth de Galaad ? Et il lui répondit : Compte sur moi comme sur toi, et sur mon peuple comme sur ton peuple, nous irons avec toi à la guerre.
\TextTitle{Les prophètes de mensonge encouragent Achab\FTNTT{1 R. 22:5-12}}
\VS{4}Puis Josaphat dit au roi d'Israël : Consulte aujourd'hui, je te prie, la parole de Yahweh.
\VS{5}Le roi d'Israël assembla les prophètes, au nombre de quatre cents, et leur dit : Irons-nous à la guerre contre Ramoth de Galaad, ou dois-je y renoncer ? Ils répondirent : Monte, et Dieu la livrera entre les mains du roi.
\VS{6}Mais Josaphat dit : N'y a-t-il point encore ici quelque prophète de Yahweh, afin que nous l'interrogions ?
\VS{7}Le roi d'Israël dit à Josaphat : Il y a encore un homme par qui on peut consulter Yahweh ; mais je le hais parce qu'il ne me prophétise rien de bon, mais du mal ; c'est Michée, fils de Jimla. Josaphat dit : Que le roi ne parle pas ainsi !
\VS{8}Alors le roi d'Israël appela un eunuque, et dit : Fais promptement venir Michée, fils de Jimla.
\VS{9}Or le roi d'Israël et Josaphat, roi de Juda, étaient assis, chacun sur son trône, revêtus de leurs habits, et ils étaient assis dans la place, à l'entrée de la porte de Samarie ; et tous les prophètes prophétisaient en leur présence.
\VS{10}Alors Sédécias, fils de Kenaana, s'étant fait des cornes de fer, dit : Ainsi parle Yahweh : Avec ces cornes tu heurteras les Syriens jusqu'à les détruire.
\VS{11}Tous les prophètes prophétisaient de même, en disant : Monte à Ramoth de Galaad, et tu prospèreras ; Yahweh la livrera entre les mains du roi.
\TextTitle{Michée annonce la défaite et la mort d'Achab\FTNTT{1 R. 22:13-28 ; 1 R. 22:29-40}}
\VS{12}Or le messager qui était allé appeler Michée, lui parla et lui dit : Voici, tous les prophètes disent d'une même bouche du bien au roi ; je te prie que ta parole soit semblable à celle de chacun d'eux ! Annonce du bien !
\VS{13}Mais Michée répondit : Yahweh est vivant ! Je dirai ce que mon Dieu dira.
\VS{14}Il vint donc vers le roi, et le roi lui dit : Michée, irons-nous à la guerre contre Ramoth de Galaad, devons-nous y renoncer ? Et il répondit : Montez, vous prospérerez, et ils seront livrés entre vos mains.
\VS{15}Et le roi lui dit : Combien de fois devrais-je te faire jurer de ne me dire que la vérité au Nom de Yahweh ?
\VS{16}Et il répondit : J'ai vu tout Israël dispersé par les montagnes, comme un troupeau de brebis qui n'a point de berger ; et Yahweh a dit : Ces gens n'ont point de seigneur ; que chacun retourne en paix dans sa maison !
\VS{17}Alors le roi d'Israël dit à Josaphat : Ne t'ai-je pas dit qu'il ne prophétise rien de bon quand il s'agit de moi, mais seulement du mal ?
\VS{18}Et Michée dit : Ecoute la parole de Yahweh ! J'ai vu Yahweh assis sur son trône, et toute l'armée des cieux se tenant à sa droite et à sa gauche.
\VS{19}Et Yahweh dit : Qui est-ce qui séduira Achab, roi d'Israël, afin qu'il monte et qu'il tombe à Ramoth de Galaad ? Et l'un répondait d'une façon et l'autre d'une autre.
\VS{20}Alors un esprit s'avança et se tint devant Yahweh, et dit : Moi, je le séduirai. Yahweh lui dit : Comment ?
\VS{21}Il répondit : Je sortirai, dit-il, et je serai un esprit de mensonge\FTNT{Achab a été frappé de l'esprit d'égarement (2 Th. 2:9-11). Voir commentaires en Ge. 6:3 ; Mt. 12:31.} dans la bouche de tous ses prophètes. Et Yahweh dit : Tu le séduiras, et même tu en viendras à bout. Sors, et fais ainsi.
\VS{22}Maintenant voici, Yahweh a mis un esprit de mensonge dans la bouche de tes prophètes que voilà ; et Yahweh a prononcé du mal contre toi.
\VS{23}Alors Sédécias, fils de Kenaana, s'étant approché, frappa Michée sur la joue, et dit : Par quel chemin l'Esprit de Yahweh s'est-il retiré de moi pour te parler ?
\VS{24}Et Michée répondit : Voici, tu le verras au jour où tu iras de chambre en chambre pour te cacher !
\VS{25}Alors le roi d'Israël dit : Prenez Michée, et emmenez-le vers Amon, chef de la ville, et vers Joas, fils du roi.
\VS{26}Et vous direz : Ainsi parle le roi : Mettez cet homme en prison, et nourrissez-le du pain et de l'eau de l'affliction, jusqu'à ce que je revienne en paix.
\VS{27}Et Michée dit : Si jamais tu retournes et reviens en paix, Yahweh n'aura point parlé par moi. Et il dit : Entendez cela peuples, vous tous qui êtes ici !
\VS{28}Le roi d'Israël monta donc avec Josaphat, roi de Juda, à Ramoth de Galaad.
\VS{29}Le roi d'Israël dit à Josaphat : Je vais me déguiser pour aller au combat ; mais toi, revêts-toi de tes habits. Ainsi le roi d'Israël se déguisa ; et ils allèrent au combat.
\VS{30}Or le roi des Syriens avait donné cet ordre aux chefs de ses chars, disant : Vous ne combattrez ni petit ni grand, mais seulement le roi d'Israël.
\VS{31}Les chefs des chars aperçurent Josaphat, et dirent : C'est le roi d'Israël ! Et ils se tournèrent vers lui pour le combattre ; mais Josaphat poussa un cri, et Yahweh le secourut, et Dieu les éloigna de lui.
\VS{32}Quand les chefs des chars virent que ce n'était pas le roi d'Israël, ils se détournèrent de lui.
\VS{33}Alors quelqu'un tira de son arc au hasard, et frappa le roi d'Israël entre les jointures de la cuirasse ; et le roi dit à son conducteur de char : Tourne-toi, et sors-moi du camp ; car je suis blessé.
\VS{34}Or en ce jour-là, le combat fut très rude. Le roi d'Israël se posa dans son char, en face des Syriens, jusqu'au soir ; et il mourut vers le coucher du soleil.
\Chap{19}
\TextTitle{Jéhu dénonce l'alliance de Josaphat avec Achab}
\VerseOne{}Josaphat roi de Juda, revint en paix dans sa maison, à Jérusalem.
\VS{2}Mais Jéhu, fils de Hanani, le voyant, sortit au-devant du roi Josaphat, et lui dit : Faut-il donner du secours au méchant, ou aimer ceux qui haïssent Yahweh ? A cause de cela, Yahweh est irrité contre toi.
\VS{3}Mais il s'est trouvé de bonnes choses en toi, puisque tu as ôté du pays les idoles d'Astarté, et tu as appliqué ton cœur à rechercher Dieu.
\VS{4}Josaphat demeura à Jérusalem. Puis, il ressortit de nouveau parmi le peuple, depuis Beer-Schéba jusqu'à la montagne d'Ephraïm, et il les ramena à Yahweh, le Dieu de leurs pères.
\TextTitle{Josaphat organise la justice}
\VS{5}Il établit aussi des juges dans le pays, dans toutes les villes fortes de Juda, de ville en ville.
\VS{6}Et il dit aux juges : Veillez sur ce que vous ferez ; car vous n'exercez pas la justice de la part d'un homme, mais de la part de Yahweh, qui sera avec vous quand vous prononcerez les jugements.
\VS{7}Maintenant, que la crainte de Yahweh soit sur vous ; prenez garde à ce que vous ferez ; car il n'y a point d'iniquité chez Yahweh, notre Dieu, ni d'acception de personnes, ni d'acceptation de présents. 
\VS{8}Josaphat établit aussi à Jérusalem des Lévites, des sacrificateurs, et des chefs des pères d'Israël, pour le jugement de Yahweh, et pour les contestations ; car on revenait à Jérusalem.
\VS{9}Il leur donna des ordres, en disant : Vous agirez ainsi dans la crainte de Yahweh, avec fidélité et avec intégrité de cœur.
\VS{10}Dans toute contestation qui viendra devant vous, de la part de vos frères qui habitent dans leurs villes, soit d'un meurtre, d'une loi, d'un commandement, d'un statut ou d'une ordonnance, vous les instruirez, afin qu'ils ne se rendent pas coupables envers Yahweh, et que sa colère ne vienne pas sur vous et sur vos frères. Vous agirez ainsi afin de ne pas être coupables.
\VS{11}Et voici, Amaria, le souverain sacrificateur, sera au-dessus de vous pour toutes les affaires de Yahweh ; et Zebadia, fils d'Ismaël, prince de la maison de Juda, pour toutes les affaires du roi ; et pour secrétaires, vous avez devant vous les Lévites. Fortifiez-vous et faites ainsi ; et que Yahweh soit avec l'homme de bien !
\Chap{20}
\TextTitle{Menaces des ennemis de Juda, prière de Josaphat}
\VerseOne{}Après ces choses, les fils de Moab et les fils d'Ammon, et avec eux les Maonites, vinrent contre Josaphat pour lui faire la guerre.
\VS{2}On vint le rapporter à Josaphat, en disant : Il vient contre toi une grande multitude depuis l'autre bord de la mer, de Syrie ; et les voici à Hatsatson-Thamar, qui est En-Guédi.
\VS{3}Alors Josaphat craignit ; mais il se disposa à rechercher Yahweh, et publia un jeûne pour tout Juda.
\VS{4}Juda s'assembla donc pour rechercher Yahweh ; on vint même de toutes les villes de Juda pour chercher Yahweh.
\VS{5}Et Josaphat se tint au milieu de l'assemblée de Juda et de Jérusalem, dans la maison de Yahweh, devant le nouveau parvis.
\VS{6}Il dit : Yahweh, Dieu de nos pères ! N'es-tu pas Dieu dans les cieux, toi qui domines sur tous les royaumes des nations ? Ne tiens-tu pas dans ta main la force et la puissance, et à qui nul ne peut résister ?
\VS{7}N'est-ce pas toi, ô notre Dieu, qui as dépossédé les habitants de ce pays devant ton peuple d'Israël, et qui l'as donné pour toujours à la postérité d'Abraham, qui t'aimait ?
\VS{8}Ils y ont habité et t'y ont bâti un sanctuaire pour ton Nom, en disant :
\VS{9}S'il nous arrive quelque malheur, l'épée, le jugement, la peste, ou la famine, nous nous tiendrons devant cette maison, et en ta présence ; car ton Nom est en cette maison ; et nous crierons à toi dans notre détresse, et tu exauceras et tu délivreras !
\VS{10}Maintenant, voici les enfants d'Ammon et de Moab, et ceux de la montagne de Séir, chez lesquels tu ne permis pas à Israël d'entrer quand il venait du pays d'Egypte, car il se détourna d'eux, et ne les détruisit pas.
\VS{11}Voici, pour nous récompenser, ils viennent nous chasser de ton héritage, que tu nous as fait posséder.
\VS{12}Ô notre Dieu ! Ne seras-tu pas juge contre eux ? Car nous sommes sans force devant cette grande multitude qui vient contre nous, et nous ne savons que faire ; mais nos yeux sont sur toi.
\VS{13}Or tout Juda se tenait devant Yahweh, même avec leurs petits enfants, leurs femmes et leurs fils.
\TextTitle{Yahweh répond à Josaphat}
\VS{14}Alors l'Esprit de Yahweh saisit au milieu de l'assemblée Jachaziel, fils de Zacharie, fils de Benaja, fils de Jeïel, fils de Matthania, Lévite, d'entre les fils d'Asaph,
\VS{15}et il dit : Soyez attentifs, tout Juda et habitants de Jérusalem, et toi, roi Josaphat ! Ainsi parle Yahweh : Ne craignez point, et ne soyez point effrayés en face de cette grande multitude ; car ce ne sera pas à vous de combattre, mais à Dieu.
\VS{16}Descendez demain vers eux ; les voici qui montent par la montée de Tsits, et vous les trouverez à l'extrémité de la vallée, en face du désert de Jeruel.
\VS{17}Ce ne sera point à vous de combattre en cette bataille ; présentez-vous, tenez-vous là, et voyez la délivrance que Yahweh va vous donner. Juda et Jérusalem, ne craignez point, et ne soyez point effrayés ! Demain, sortez au-devant d'eux, et Yahweh sera avec vous.
\VS{18}Alors Josaphat s'inclina le visage contre terre, et tout Juda et les habitants de Jérusalem se jetèrent devant Yahweh, se prosternant devant Yahweh.
\VS{19}Et les Lévites, d'entre les fils des Kehathites et d'entre les fils des Koréites, se levèrent pour célébrer Yahweh, le Dieu d'Israël, d'une voix haute et forte.
\TextTitle{Yahweh délivre Juda des armées ennemies}
\VS{20}Puis, le matin, ils se levèrent de bonne heure et sortirent vers le désert de Tekoa. Et comme ils sortaient, Josaphat se tint debout et dit : Ecoutez-moi Juda et vous, habitants de Jérusalem ! Croyez en Yahweh, votre Dieu, et vous serez en sûreté ; croyez en ses prophètes, et vous réussirez.
\VS{21}Puis, ayant consulté le peuple, il établit des chantres de Yahweh, qui célébraient sa sainte magnificence ; et marchant devant l'armée, ils disaient : Louez Yahweh, car sa miséricorde dure à toujours\FTNT{Ps. 136} !
\VS{22}Et au moment où ils commencèrent le chant et la louange, Yahweh mit des embuscades contre les fils d'Ammon, de Moab, et ceux de la montagne de Séir, qui venaient contre Juda. Et ils furent battus.
\VS{23}Les fils d'Ammon et de Moab se levèrent contre les habitants de la montagne de Séir pour les dévouer par interdit et les exterminer ; et quand ils en eurent fini avec les habitants de Séir, ils s'aidèrent l'un l'autre à se détruire mutuellement.
\VS{24}Et quand Juda fut arrivé sur la hauteur d'où l'on voit le désert, ils regardèrent vers cette multitude, et voici, c'étaient des cadavres gisant à terre, et personne n'avait échappé.
\VS{25}Ainsi Josaphat et son peuple vinrent pour piller leurs dépouilles, et ils trouvèrent parmi les cadavres des biens en abondance, et des objets précieux ; et ils en saisirent tant qu'ils ne pouvaient tout porter ; et ils pillèrent le butin pendant trois jours, car il était considérable.
\VS{26}Le quatrième jour, ils s'assemblèrent dans la vallée de Beraca ; car ils bénirent là Yahweh ; c'est pourquoi on a appelé ce lieu, jusqu'à ce jour, la vallée de Beraca.
\VS{27}Et tous les hommes de Juda et de Jérusalem, et Josaphat à leur tête, s'en retournèrent, revenant à Jérusalem avec joie ; car Yahweh les avait réjouis au sujet de leurs ennemis. 
\VS{28}Ils entrèrent donc à Jérusalem, dans la maison de Yahweh, avec des luths, des harpes et des trompettes.
\VS{29}Et la crainte de Dieu fut sur tous les royaumes des autres pays, lorsqu'ils apprirent que Yahweh avait combattu contre les ennemis d'Israël.
\VS{30}Ainsi le royaume de Josaphat fut tranquille, et son Dieu lui donna du repos de toutes parts.
\TextTitle{Règne de Josaphat, son alliance coupable\FTNTT{1 R. 22:41-49}}
\VS{31}Josaphat régna donc sur Juda. Il était âgé de trente-cinq ans quand il devint roi, et il régna vingt-cinq ans à Jérusalem. Sa mère s'appelait Azuba, fille de Schilchi.
\VS{32}Il suivit les traces d'Asa, son père, et il ne s'en détourna point, faisant ce qui est droit aux yeux de Yahweh.
\VS{33}Seulement les hauts lieux ne furent pas ôtés, et le peuple n'avait pas encore le cœur fermement attaché au Dieu de ses pères.
\VS{34}Or le reste des actions de Josaphat, les premières et les dernières, voici, elles sont écrites dans les mémoires de Jéhu, fils de Hanani, insérées dans le livre des rois d'Israël.
\VS{35}Après cela, Josaphat, roi de Juda, s'associa avec Achazia, roi d'Israël, dont la conduite était impie.
\VS{36}Il s'associa avec lui pour faire des navires, afin d'aller à Tarsis ; et ils firent des navires à Etsjon-Guéber.
\VS{37}Alors Eliézer, fils de Dodava, de Maréscha, prophétisa contre Josaphat, en disant : Parce que tu t'es associé avec Achazia, Yahweh a détruit ton œuvre. Et les navires furent brisés, et ne purent aller à Tarsis.
\Chap{21}
\TextTitle{Joram règne sur Juda\FTNTT{1 R. 22:50 ; 2 R. 8:16-19}}
\VerseOne{}Puis Josaphat s'endormit avec ses pères, et il fut enseveli avec eux dans la cité de David. Et Joram, son fils, régna à sa place\FTNT{1 R. 22:51.}.
\VS{2}Il avait des frères, fils de Josaphat : Azaria, Jehiel, Zacharie, Azaria, Micaël et Schephathia. Tous ceux-là étaient fils de Josaphat, roi d'Israël.
\VS{3}Leur père leur avait fait de grands dons d'argent, d'or et de choses précieuses, avec des villes fortes en Juda ; mais il avait donné le royaume à Joram, parce qu'il était le premier-né.
\VS{4}Quand Joram fut élevé sur le royaume de son père, et s'y fut fortifié, il tua avec l'épée tous ses frères, et quelques-uns aussi des chefs d'Israël.
\VS{5}Joram était âgé de trente-deux ans quand il devint roi, et il régna huit ans à Jérusalem.
\VS{6}Il marcha dans la voie des rois d'Israël, comme avait fait la maison d'Achab ; car la fille d'Achab était sa femme, et il fit ce qui est mal aux yeux de Yahweh.
\VS{7}Toutefois, Yahweh, à cause de l'alliance qu'il avait traitée avec David, ne voulut pas détruire la maison de David, selon qu'il avait dit qu'il lui donnerait une lampe, à lui et à ses fils, pour toujours.
\TextTitle{Rébellion d'Edom et de Libna\FTNTT{1 R. 8:20-23}}
\VS{8}De son temps, Edom se rebella de l'autorité de Juda, et établit un roi sur lui\FTNT{2 R. 8:20-23}.
\VS{9}Joram se mit donc en marche avec ses chefs et tous ses chars ; et s'étant levé de nuit, il battit les Edomites qui l'entouraient, et tous les chefs des chars.
\VS{10}Néanmoins, Edom se rebella contre l'autorité de Juda jusqu'à ce jour. En ce même temps, Libna se rebella aussi contre son autorité, parce qu'il avait abandonné Yahweh, le Dieu de ses pères.
\VS{11}Il fit aussi des hauts lieux dans les montagnes de Juda ; il fit que les habitants de Jérusalem se prostituèrent, et il y entraîna ceux de Juda.
\TextTitle{Elie prononce un jugement sur Joram}
\VS{12}Alors il lui vint un écrit de la part d'Elie, le prophète, disant : Ainsi parle Yahweh, le Dieu de David, ton père : Parce que tu n'as point suivi le chemin de Josaphat, ton père, ni celui d'Asa, roi de Juda,
\VS{13}mais que tu as suivi les voies des rois d'Israël, et que tu as poussé à la prostitution Juda et les habitants de Jérusalem, comme s'est prostituée la maison d'Achab, et que tu as tué tes frères, meilleurs que toi, la maison même de ton père ;
\VS{14}voici, Yahweh frappera d'une grande plaie ton peuple, tes fils, tes femmes et tous tes biens.
\VS{15}Et toi, tu auras une grosse maladie, une maladie d'entrailles ; jusqu'à ce que, de jour en jour, tes entrailles sortent par la force de la maladie.
\TextTitle{Yahweh excite les Philistins et les Arabes contre Joram}
\VS{16}Yahweh souleva contre Joram l'esprit des Philistins et des Arabes, qui habitent près des Ethiopiens.
\VS{17}Ils montèrent donc contre Juda, et firent une brèche pour piller toutes les richesses qui furent trouvées dans la maison du roi ; et même, ils emmenèrent captifs ses fils et ses femmes, de sorte qu'il ne lui resta d'autre fils que Joachaz, le plus jeune de ses fils.
\TextTitle{Mort de Joram}
\VS{18}Après tout cela, Yahweh frappa ses entrailles d'une maladie sans remède.
\VS{19}Elle s'avança chaque jour, et vers la fin de la seconde année, ses entrailles sortirent par la force de son mal, et il mourut dans de grandes souffrances. Son peuple ne brûla pas sur lui de parfums, comme il l'avait fait pour ses pères.
\VS{20}Il était âgé de trente-deux ans quand il devint roi, et il régna huit ans à Jérusalem. Il s'en alla sans être regretté, et on l'ensevelit dans la cité de David, mais non dans les sépulcres des rois.
\Chap{22}
\TextTitle{Achazia règne sur Juda\FTNTT{2 R. 8:24-29}}
\VerseOne{}Les habitants de Jérusalem firent régner à sa place Achazia, le plus jeune de ses fils, parce que les troupes qui étaient venues au camp avec les Arabes avaient tué tous les plus âgés ; et Achazia, fils de Joram, roi de Juda, régna\FTNT{2 R. 8:24-29 ; 2 R. 9:16.}.
\VS{2}Achazia était âgé de quarante-deux ans quand il devint roi, et il régna un an à Jérusalem. Sa mère avait pour nom Athalie, fille d'Omri.
\VS{3}Il suivit aussi les voies de la maison d'Achab, car sa mère lui donnait des conseils impies.
\VS{4}Il fit donc ce qui est mal aux yeux de Yahweh, comme la maison d'Achab ; parce qu'ils furent ses conseillers après la mort de son père, pour sa ruine.
\TextTitle{Achazia livré aux mains de Jéhu\FTNTT{2 R. 8:28-29 ; 2 R. 9:1-30}}
\VS{5}Conduit par leurs conseils, il alla avec Joram, fils d'Achab, roi d'Israël, à la guerre à Ramoth de Galaad, contre Hazaël, roi de Syrie. Et les Syriens frappèrent Joram,
\VS{6}qui s'en retourna à Jizreel, pour guérir des blessures que les Syriens lui avaient faites à Rama, lorsqu'il faisait la guerre contre Hazaël, roi de Syrie. Azaria, fils de Joram, roi de Juda, descendit pour voir Joram, le fils d'Achab, à Jizreel, parce qu'il était malade.
\VS{7}Dieu fit pour sa ruine qu'Achazia vint auprès de Joram. En effet, quand il fut arrivé, il sortit avec Joram pour aller au-devant de Jéhu, fils de Nimschi, que Yahweh avait oint pour retrancher la maison d'Achab.
\VS{8}Et comme Jéhu faisait justice de la maison d'Achab\FTNT{2 R. 10:12-30.}, il trouva les chefs de Juda et les fils des frères d'Achazia, qui servaient Achazia, et il les tua.
\VS{9}Il chercha ensuite Achazia, qui s'était caché en Samarie. On le prit, et on l'amena vers Jéhu qui le fit mourir. Puis on l'ensevelit, car on dit : C'est le fils de Josaphat, qui cherchait Yahweh de tout son cœur. Et il n'y eut plus personne dans la maison d'Achazia qui fut capable de régner.
\TextTitle{Joas échappe au massacre de sa famille\FTNTT{2 R. 11:1-3}}
\VS{10}Or Athalie, mère d'Achazia, voyant que son fils était mort, se leva et fit périr toute la race royale de la maison de Juda\FTNT{1 R. 11:1-3.}.
\VS{11}Mais Joschabeath, fille du roi Joram, prit Joas, fils d'Achazia, en le dérobant d'entre les fils du roi qu'on faisait mourir. Elle le mit avec sa nourrice dans la salle des lits. Ainsi Joschabeath, fille du roi Joram et femme de Jehojada le sacrificateur, étant la sœur d'Achazia, le cacha de la vue d'Athalie, qui ne put le faire mourir.
\VS{12}Il fut ainsi caché avec eux dans la maison de Dieu six ans ; et c'est Athalie qui régnait sur le pays.
\Chap{23}
\TextTitle{Joas devient roi grâce à Jehojada\FTNTT{2 R. 11:4-12}}
\VerseOne{}Mais la septième année, Jehojada prit courage et traita alliance avec les chefs de centaines, Azaria, fils de Jerocham, Ismaël, fils de Jochanan, Azaria, fils d'Obed, Maaséja, fils d'Adaja, et Elischaphath, fils de Zicri.
\VS{2}Ils firent le tour de Juda, pour rassembler de toutes les villes de Juda les Lévites et les chefs des pères d'Israël ; puis ils vinrent à Jérusalem.
\VS{3}Et toute cette assemblée traita alliance avec le roi dans la maison de Dieu. Jehojada leur dit : Voici, c'est le fils du roi qui régnera, selon la parole de Yahweh au sujet des fils de David.
\VS{4}Vous ferez donc ceci : Le tiers qui parmi vous entre en service au sabbat, sacrificateurs et Lévites, fera la garde des seuils.
\VS{5}Un autre tiers se tiendra dans la maison du roi, et un tiers à la porte de Jesod ; et tout le peuple sera dans les parvis de la maison de Yahweh.
\VS{6}Que personne n'entre dans la maison de Yahweh, sauf les sacrificateurs et les Lévites de service : Ils entreront car ils sont sanctifiés ; et tout le reste du peuple gardera les ordres de Yahweh.
\VS{7}Les Lévites environneront le roi de toutes parts, tenant chacun leurs armes à la main, et donneront la mort à quiconque voudra entrer dans la maison ; vous serez avec le roi quand il entrera et quand il sortira.
\VS{8}Les Lévites et tout Juda firent tout ce que Jehojada le sacrificateur, avait ordonné. Ils prirent chacun leurs gens, tant ceux qui entraient en service que ceux qui en sortaient au sabbat ; car Jehojada, le sacrificateur, n'avait exempté aucune classe.
\VS{9}Et Jehojada le sacrificateur, donna aux chefs de centaines les lances, les grands et les petits boucliers qui provenaient du roi David, et qui étaient dans la maison de Dieu.
\VS{10}Puis il rangea tout le peuple autour du roi, chacun tenant ses armes à la main, du côté droit du temple jusqu'au côté gauche de la maison, près de l'autel et de la maison.
\VS{11}Alors ils firent sortir le fils du roi, et mirent sur lui la couronne et le témoignage. Ils l'établirent roi, et Jehojada et ses fils l'oignirent et dirent : Vive le roi !
\TextTitle{Mort d'Athalie\FTNTT{2 R. 11:13-16}}
\VS{12}Mais Athalie, entendant le bruit du peuple qui courait et célébrait le roi, vint vers le peuple, dans la maison de Yahweh.
\VS{13}Elle regarda, et voici, le roi se tenait près de la colonne, à l'entrée ; les chefs et les trompettes étaient près du roi, et tout le peuple du pays était dans la joie, et l'on sonnait des trompettes ; les chantres, avec des instruments de musique, dirigeaient les chants de louanges. Alors Athalie déchira ses vêtements et dit : Conspiration ! Conspiration !
\VS{14}Le sacrificateur Jehojada fit sortir les chefs de centaines qui étaient à la tête de l'armée, et leur dit : Faites-la sortir hors des rangs, et que celui qui la suivra soit mis à mort par l'épée ! Car le sacrificateur avait dit : Ne la mettez pas à mort dans la maison de Yahweh.
\VS{15}Ils mirent donc la main sur elle pour la faire entrer dans la maison du roi, par l'entrée de la porte des chevaux ; et là ils la firent mourir.
\TextTitle{Jehojada fait asseoir Joas sur le trône de Juda\FTNTT{2 R. 11:17-20}}
\VS{16}Puis Jehojada traita, avec tout le peuple et le roi, une alliance pour être le peuple de Yahweh.
\VS{17}Et tout le peuple entra dans la maison de Baal pour la détruire. Ils brisèrent ses autels et ses images et ils tuèrent devant les autels Matthan, sacrificateur de Baal.
\VS{18}Jehojada remit aussi les fonctions de la maison de Yahweh entre les mains des sacrificateurs, des Lévites, comme David les avait répartis dans la maison de Yahweh, afin qu'ils élèvent des holocaustes à Yahweh, comme cela est écrit dans la loi de Moïse, avec joie et avec des chants, selon les ordonnances de David.
\VS{19}Il établit aussi les portiers aux portes de la maison de Yahweh, afin qu'il n'y entrât aucune personne souillée de quelque manière que ce fût.
\VS{20}Il prit les chefs de centaines, hommes considérés, qui avaient de l'autorité parmi le peuple, et tout le peuple du pays. Il fit descendre le roi, de la maison de Yahweh à la maison du roi, en entrant par la porte supérieure ; et ils firent asseoir le roi sur le trône royal.
\VS{21}Alors tout le peuple du pays se réjouit, et la ville fut tranquille, bien qu'on eût mis à mort Athalie par l'épée.
\Chap{24}
\TextTitle{Joas règne sur Juda ; ses travaux sur le temple\FTNTT{2 R. 11:21-12:8}}
\VerseOne{}Joas était âgé de sept ans quand il devint roi, et il régna quarante ans à Jérusalem. Sa mère avait pour nom Tsibja, de Beer-Schéba\FTNT{2 R. 11:21 ; 2 R. 12:1-3.}.
\VS{2}Joas fit ce qui est droit aux yeux de Yahweh, pendant toute la vie de Jehojada, le sacrificateur.
\VS{3}Et Jehojada prit pour lui deux femmes, dont il eut des fils et des filles.
\VS{4}Après cela Joas eut la pensée de renouveler la maison de Yahweh\FTNT{2 R. 12:4-16.}.
\VS{5}Il assembla donc les sacrificateurs et les Lévites, et leur dit : Allez vers les villes de Juda, et recueillez de l'argent par tout Israël, suffisamment pour réparer la maison de votre Dieu d'année en année, et hâtez-vous pour cette affaire. Mais les Lévites ne se hâtèrent point.
\VS{6}Alors le roi appela Jehojada, leur chef, et lui dit : Pourquoi n'as-tu pas veillé à ce que les Lévites aient apporté de Juda et de Jérusalem, l'impôt sur l'assemblée d'Israël, selon Moïse, serviteur de Yahweh, pour la tente du témoignage ?
\VS{7}Car l'impie Athalie et ses fils ont ravagé la maison de Dieu ; et même ils ont employé pour les Baals toutes les choses consacrées à la maison de Yahweh.
\TextTitle{Offrandes volontaires pour la réparation du temple\FTNTT{2 R. 12:9-16}}
\VS{8}Et le roi ordonna qu'on fasse un seul coffre, et qu'on le mette à la porte de la maison de Yahweh, à l'extérieur.
\VS{9}Puis on publia dans Juda et dans Jérusalem, pour qu'on qu'on apportât à Yahweh l'impôt mis par Moïse, serviteur de Dieu, sur Israël dans le désert.
\VS{10}Tous les chefs et tout le peuple s'en réjouirent, et l'on apporta et jeta le tribut dans le coffre, jusqu'à ce qu'il fût plein.
\VS{11}Au moment venu, les Lévites apportaient le coffre aux inspecteurs du roi, car ceux-ci voyaient qu'il y avait beaucoup d'argent. Un secrétaire du roi et un commissaire du souverain sacrificateur venaient et vidaient le coffre ; puis ils le rapportaient et le remettaient à sa place. Ils faisaient ainsi jour après jour, et ils recueillaient de l'argent en abondance.
\VS{12}Le roi et Jehojada le donnaient à ceux qui étaient chargés de l'ouvrage pour le service de la maison de Yahweh, et ceux-ci engageaient des tailleurs de pierres et des charpentiers pour réparer la maison de Yahweh, et aussi des ouvriers pour le fer et l'airain, afin de réparer la maison de Yahweh.
\VS{13}Ceux qui étaient chargés de l'ouvrage travaillèrent donc ; et par leurs mains, les travaux s'exécutèrent, de sorte qu'ils rétablirent la maison de Dieu en son état, et l'affermirent.
\VS{14}Lorsqu'ils eurent achevé, ils apportèrent devant le roi et devant Jehojada le reste de l'argent ; et on fit faire des ustensiles pour la maison de Yahweh, des ustensiles pour le service et pour les holocaustes, des coupes et d'autres ustensiles d'or et d'argent. Et on offrit continuellement des holocaustes dans la maison de Yahweh, tant que vécut Jehojada.
\TextTitle{Mort de Jehojada, Joas abandonne Yahweh\FTNTT{2 R. 12:9-16}}
\VS{15}Or Jehojada devint vieux et rassasié de jours, et il mourut ; il était âgé de cent trente ans quand il mourut.
\VS{16}On l'ensevelit dans la cité de David avec les rois ; car il avait fait du bien à Israël, et à l'égard de Dieu et de sa maison.
\VS{17}Mais, après la mort de Jehojada, les chefs de Juda vinrent et se prosternèrent devant le roi ; et le roi les écouta.
\VS{18}Ils abandonnèrent la maison de Yahweh, le Dieu de leurs pères, et ils servirent les idoles d'Astarté et les faux dieux ; et la colère de Yahweh fut sur Juda et sur Jérusalem, parce qu'ils s'étaient ainsi rendus coupables.
\VS{19}Yahweh envoya parmi eux des prophètes, pour les faire retourner à lui par leurs avertissements ; mais ils ne voulurent point les écouter.
\VS{20}Alors l'Esprit de Dieu revêtit Zacharie, fils de Jehojada, le sacrificateur, et se tenant devant le peuple, il leur dit : Dieu m'a parlé ainsi : Pourquoi transgressez-vous les commandements de Yahweh ? Vous ne prospérerez point ; car vous avez abandonné Yahweh, et il vous abandonnera aussi.
\VS{21}Mais ils se liguèrent contre lui et le lapidèrent, par ordre du roi, dans le parvis de la maison de Yahweh.
\VS{22}Ainsi le roi Joas ne se souvint point de la bonté dont Jehojada, père de Zacharie, avait usé envers lui ; et il tua son fils, qui dit en mourant : Yahweh le voit, et il en demandera compte !
\TextTitle{Invasion des Syriens, conspiration et mort de Joas\FTNTT{2 R. 12:17-21 ; cp. 2 R. 13:7}}
\VS{23}A la fin de cette année-là, l'armée de Syrie monta contre Joas, et vint en Juda et à Jérusalem. Ils détruisirent, d'entre le peuple, tous les chefs du peuple, et ils envoyèrent au roi de Damas tout leur butin.
\VS{24}Et quoique l'armée venue de Syrie fût peu nombreuse, Yahweh livra entre leurs mains une armée très nombreuse, parce qu'ils avaient abandonné Yahweh, le Dieu de leurs pères. Ainsi les Syriens furent le châtiment de Joas.
\VS{25}Quand ils s'éloignèrent de lui, après l'avoir laissé dans de grandes souffrances, ses serviteurs conspirèrent contre lui, à cause du sang des fils de Jehojada, le sacrificateur ; ils le tuèrent sur son lit, et il mourut. On l'ensevelit dans la cité de David, mais on ne l'ensevelit pas dans les sépulcres des rois.
\VS{26}Ce sont ici ceux qui conspirèrent contre lui : Zabad, fils de Schimeath, femme ammonite, et Jozabad, fils de Schimrith, femme Moabite.
\VS{27}Quant à ses fils et à la grande charge qui reposa sur lui, et à la réparation de la maison de Dieu, voici, ces choses sont écrites dans les mémoires du livre des rois. Amatsia, son fils, régna à sa place.
\Chap{25}
\TextTitle{Amatsia règne sur Juda\FTNTT{2 R. 12:21 ; 2 R. 14:1-6}}
\VerseOne{}Amatsia devint roi à l'âge de vingt-cinq ans, et il régna vingt-neuf ans à Jérusalem. Sa mère avait pour nom Joaddan, de Jérusalem\FTNT{2 R. 12:21 ; 2 R. 14:1-20}.
\VS{2}Il fit ce qui est droit aux yeux de Yahweh, mais non d'un cœur entier.
\VS{3}Après qu'il fut affermi dans son règne, il fit mourir ses serviteurs qui avaient tué le roi, son père.
\VS{4}Mais il ne fit pas mourir leurs fils ; car il fit selon ce qui est écrit dans la loi, dans le livre de Moïse, où Yahweh a donné ce commandement : Les pères ne mourront point pour les fils, et les fils ne mourront point pour les pères ; mais chacun mourra pour son péché.
\TextTitle{Amatsia en guerre contre les Edomites, sa victoire\FTNTT{2 R. 14:7}}
\VS{5}Puis Amatsia rassembla ceux de Juda, et il les rangea selon les maisons paternelles, par chefs de milliers et par chefs de centaines, pour tout Juda et Benjamin ; il en fit le dénombrement depuis l'âge de vingt ans et au-dessus ; et il trouva trois cent mille hommes d'élite, propres à l'armée, maniant la lance et le bouclier.
\VS{6}Il prit aussi à sa solde, pour cent talents d'argent, cent mille vaillants hommes de guerre d'Israël.
\VS{7}Mais un homme de Dieu vint à lui, et lui dit : Ô roi ! Que l'armée d'Israël ne marche point avec toi ; car Yahweh n'est point avec Israël ni avec tous ces fils d'Ephraïm.
\VS{8}Si tu vas avec eux, quand bien même tu ferais de vaillants combats, Dieu te fera tomber devant l'ennemi ; car Dieu a la puissance d'aider et de faire tomber.
\VS{9}Amatsia dit à l'homme de Dieu : Mais que faire des cent talents que j'ai donnés à la troupe d'Israël ? L'homme de Dieu dit : Yahweh peut t'en donner beaucoup plus.
\VS{10}Ainsi Amatsia sépara les troupes qui lui étaient venues d'Ephraïm, et les fit retourner chez elles ; mais leur colère s'enflamma très ardemment contre Juda, et ces gens retournèrent chez eux dans une grande colère.
\VS{11}Alors Amatsia prit courage, conduisit son peuple et s'en alla dans la vallée du sel, où il battit dix mille hommes des fils de Séir.
\VS{12}Les fils de Juda prirent dix mille hommes vivants, et les ayant amenés sur le sommet d'une roche, ils les jetèrent du haut de la roche, de sorte qu'ils furent tous brisés.
\VS{13}Mais les gens de la troupe qu'Amatsia avait renvoyée, afin qu'ils n'aillent pas avec lui à la guerre, firent une incursion dans les villes de Juda, depuis Samarie jusqu'à Beth-Horon. Ils y tuèrent trois mille personnes et emportèrent un gros butin.
\TextTitle{Idolâtrie d'Amatsia\FTNTT{2 R. 14:7}}
\VS{14}Lorsqu'Amatsia fut de retour de la défaite des Edomites, et ayant apporté les dieux des fils de Séir, il se les établit pour dieux ; il se prosterna devant eux et leur brûla de l'encens.
\VS{15}Et la colère de Yahweh s'enflamma contre Amatsia, et il envoya vers lui un prophète qui lui dit : Pourquoi as-tu recherché les dieux d'un peuple qui n'ont point délivré leur peuple de ta main ?
\VS{16}Et comme il parlait au roi, il lui répondit : T'a-t-on établi conseiller du roi ? Cesse maintenant ! Pourquoi veux-tu qu'on te tue ? Et le prophète se retira, mais en disant : Je sais que Dieu a résolu de te détruire, parce que tu as fait cela, et que tu n'as point écouté mon conseil.
\TextTitle{Défaite d'Amatsia contre Israël\FTNTT{2 R. 14:8-14}}
\VS{17}Puis Amatsia, roi de Juda, ayant tenu conseil, envoya vers Joas, fils de Joachaz, fils de Jéhu, roi d'Israël, pour lui dire : Viens, voyons-nous en face !
\VS{18}Mais Joas, roi d'Israël, envoya dire à Amatsia, roi de Juda : L'épine du Liban envoya dire au cèdre du Liban : Donne ta fille pour femme à mon fils ! Et les bêtes sauvages qui sont au Liban passèrent et foulèrent l'épine.
\VS{19}Voici, tu dis que tu as frappé les Edomites, et ton cœur s'est élevé pour te glorifier. Maintenant, reste dans ta maison ! Pourquoi t'engagerais-tu dans un combat où tu tomberais, et Juda avec toi ?
\VS{20}Mais Amatsia ne l'écouta point ; Dieu avait résolu de le livrer aux mains de Joas parce qu'il eût recours aux dieux d'Edom.
\VS{21}Joas, roi d'Israël, monta ; et ils se virent en face, lui et Amatsia, roi de Juda, à Beth-Schémesch, qui est de Juda.
\VS{22}Juda fut battu en face d'Israël, et chacun s'enfuit dans sa tente.
\VS{23}Joas, roi d'Israël, prit Amatsia, roi de Juda, fils de Joas, fils de Joachaz, à Beth-Schémesch. Il l'emmena à Jérusalem et fit une brèche de quatre cents coudées dans la muraille de Jérusalem, depuis la porte d'Ephraïm jusqu'à la porte de l'angle.
\VS{24}Il prit l'or, l'argent, tous les vases qui se trouvaient dans la maison de Dieu sous la garde d'Obed-Edom, les trésors de la maison du roi ; il fit des otages et il retourna à Samarie.
\TextTitle{Assassinat d'Amatsia\FTNTT{2 R. 14:17-20}}
\VS{25}Amatsia, fils de Joas, roi de Juda, vécut quinze ans, après que Joas, fils de Joachaz, roi d'Israël, mourut.
\VS{26}Le reste des actions d'Amatsia, les premières et les dernières, voici cela n'est-il pas écrit dans le livre des rois de Juda et d'Israël ?
\VS{27}Or depuis le moment où Amatsia se détourna de Yahweh, on fit une conspiration contre lui à Jérusalem, et il s'enfuit à Lakis ; mais on le poursuivit à Lakis, et on le fit mourir.
\VS{28}Puis on le transporta sur des chevaux, et on l'ensevelit avec ses pères dans la ville de Juda.
\Chap{26}
\TextTitle{Ozias règne sur Juda ; il est fidèle à Yahweh\FTNTT{2 R. 14:21-15:4}}
\VerseOne{}Alors, tout le peuple de Juda prit Ozias, âgé de seize ans, et l'établit roi à la place de son père Amatsia\FTNT{2 R. 14:21 ; 2 R. 15:1-4.}.
\VS{2}Ce fut lui qui bâtit Eloth, et la ramena sous la puissance de Juda, après que le roi se fut endormi avec ses pères.
\VS{3}Ozias était âgé de seize ans quand il devint roi, et il régna cinquante-deux ans à Jérusalem. Sa mère avait pour nom Jecolia, de Jérusalem.
\VS{4}Il fit ce qui est droit aux yeux de Yahweh, comme avait fait Amatsia, son père.
\VS{5}Il s'appliqua à rechercher Dieu pendant les jours de Zacharie, qui avait une intelligence dans les visions de Dieu et pendant les jours où il rechercha Yahweh, Dieu le fit prospérer.
\VS{6}Il sortit et fit la guerre contre les Philistins. Il brisa la muraille de Gath, la muraille de Jabné, et la muraille d'Asdod ; et il bâtit des villes dans le pays d'Asdod et chez les Philistins.
\VS{7}Dieu le secourut contre les Philistins et contre les Arabes qui habitaient à Gur-Baal, et contre les Maonites.
\VS{8}Même les Ammonites faisaient des présents à Ozias, et sa renommée parvint jusqu'à l'entrée de l'Egypte ; car il était devenu très puissant.
\VS{9}Ozias bâtit des tours à Jérusalem, sur la porte de l'angle, sur la porte de la vallée, sur l'angle, et il les fortifia.
\VS{10}Il bâtit des tours dans le désert, et il creusa de nombreux puits, parce qu'il avait de nombreux troupeaux dans la plaine et dans la campagne, des laboureurs et des vignerons sur les montagnes, et au Carmel ; car il aimait l'agriculture.
\VS{11}Ozias avait une armée de gens de guerre, allant à la guerre par bandes, selon le compte de leur dénombrement fait par Jeïel le scribe, et Maaséja le commissaire, et sous la conduite de Hanania l'un des chefs du roi.
\VS{12}Le nombre total des chefs des maisons paternelles, des vaillants guerriers, était de deux mille six cents.
\VS{13}Il y avait sous leur conduite une armée de trois cent sept mille cinq cents combattants, tous gens de guerre, puissants et vaillants, capables de soutenir le roi contre l'ennemi.
\VS{14}Ozias leur procura, pour toute l'armée, des boucliers, des lances, des casques, des cuirasses, des arcs et des pierres de fronde.
\VS{15}Il fit faire à Jérusalem des machines inventées par un ingénieur, pour être placées sur les tours et sur les angles, pour lancer des flèches et de grosses pierres. Et sa renommée se répandit au loin ; car il fut extraordinairement soutenu, jusqu'à ce qu'il devienne fort puissant.
\TextTitle{Ozias pèche et est frappé de lèpre\FTNTT{2 R. 15:5-7, 32}}
\VS{16}Mais dès qu'il fut puissant, son cœur s'éleva pour le corrompre. Et il pécha contre Yahweh, son Dieu : Il entra dans le temple de Yahweh pour brûler des parfums sur l'autel des parfums\FTNT{2 R. 15:5-7.}.
\VS{17}Mais Azaria le sacrificateur, entra après lui, et avec lui quatre-vingts sacrificateurs de Yahweh, hommes vaillants,
\VS{18}qui s'opposèrent au roi Ozias, et lui dirent : Ce n'est pas à toi, Ozias, d'offrir le parfum à Yahweh, mais aux sacrificateurs, fils d'Aaron, qui sont consacrés pour cela. Sors du sanctuaire, car tu as péché ! Et cela ne sera pas à ta gloire devant Yahweh Dieu.
\VS{19}Alors Ozias, qui avait à la main un encensoir pour faire brûler le parfum, se mit en colère. Et comme il s'irritait contre les sacrificateurs, la lèpre parut sur son front, en présence des sacrificateurs, dans la maison de Yahweh, près de l'autel des parfums.
\VS{20}Azaria, le principal sacrificateur, le regarda ainsi que tous les sacrificateurs. Et voici, il avait de la lèpre sur le front. Ils le pressèrent et lui-même se hâta de sortir, parce que Yahweh l'avait frappé.
\VS{21}Le roi Ozias fut ainsi lépreux jusqu'au jour de sa mort ; il habita seul comme lépreux dans une maison écartée, car il était exclu de la maison de Yahweh. Et Jotham, son fils, avait la charge de la maison du roi, jugeant le peuple du pays.
\VS{22}Esaïe, fils d'Amots, le prophète, a écrit le reste des actions d'Ozias, les premières et les dernières.
\VS{23}Ozias s'endormit avec ses pères, et on l'ensevelit avec ses pères dans le champ de la sépulture des rois ; car on dit : Il est lépreux. Et Jotham, son fils, régna à sa place.
\Chap{27}
\TextTitle{Jotham règne sur Juda ; sa mort\FTNTT{2 R. 15:7, 32-38}}
\VerseOne{}Jotham était âgé de vingt-cinq ans quand il devint roi, et il régna seize ans à Jérusalem\FTNT{1 R. 15:7.}. Sa mère avait le nom de Jeruscha, fille de Tsadok.
\VS{2}Il fit ce qui est droit aux yeux de Yahweh, tout comme Ozias, son père, avait fait ; mais il n'entra pas dans le temple de Yahweh. Néanmoins, le peuple se corrompait encore.
\VS{3}Ce fut lui qui bâtit la porte supérieure de la maison de Yahweh, et il fit beaucoup de constructions sur les murs de la colline.
\VS{4}Il bâtit des villes sur les montagnes de Juda, des châteaux et des tours dans les forêts.
\VS{5}Il fut en guerre avec le roi des fils d'Ammon, et fut le plus fort. Cette année-là, les fils d'Ammon lui donnèrent cent talents d'argent, dix mille cors de froment, et dix mille d'orge. Les fils d'Ammon lui en donnèrent autant la seconde et la troisième année.
\VS{6}Jotham devint donc très puissant, parce qu'il avait affermi ses voies devant Yahweh, son Dieu.
\VS{7}Le reste des actions de Jotham, tous ses combats et sa conduite, voici, toutes ces choses sont écrites dans le livre des rois d'Israël et de Juda.
\VS{8}Il était âgé de vingt-cinq ans quand il devint roi, et il régna seize ans à Jérusalem.
\VS{9}Puis Jotham s'endormit avec ses pères, et on l'ensevelit dans la cité de David. Et Achaz, son fils, régna à sa place.
\Chap{28}
\TextTitle{Achaz règne sur Juda\FTNTT{2 R. 15:38-16:4}}
\VerseOne{}Achaz était âgé de vingt ans quand il devint roi, et il régna seize ans à Jérusalem\FTNT{2 R. 15:38 ; 2 R. 16:1-4.}. Il ne fit point ce qui est droit aux yeux de Yahweh, comme David, son père.
\VS{2}Il suivit la voie des rois d'Israël ; et il fit même des images de fonte pour les Baals.
\VS{3}Il brûla des parfums dans la vallée du fils de Hinnom, et il brûla ses fils au feu, suivant les abominations des nations que Yahweh avait chassées devant les enfants d'Israël.
\VS{4}Il offrait aussi des sacrifices et brûlait des parfums dans les hauts lieux, sur les collines, et sous tout arbre vert.
\TextTitle{La Syrie et Israël envahissent Juda\FTNTT{2 R. 16:5-6}}
\VS{5}C'est pourquoi Yahweh, son Dieu, le livra entre les mains du roi de Syrie. Les Syriens le battirent et lui prirent un grand nombre de prisonniers, qu'ils emmenèrent à Damas. Il fut livré aussi entre les mains du roi d'Israël, qui lui fit endurer une grande défaite\FTNT{2 R . 16:5-20.}.
\VS{6}Car Pékach, fils de Remalia, tua en un seul jour en Juda cent vingt mille hommes, tous vaillants, parce qu'ils avaient abandonné Yahweh, le Dieu de leurs pères.
\VS{7}Zicri, homme vaillant d'Ephraïm, tua Maaséja, fils du roi, et Azrikam, chef de la maison, et Elkana, le second après le roi.
\VS{8}Les fils d'Israël emmenèrent prisonniers deux cent mille de leurs frères, tant femmes que fils et filles ; ils firent aussi sur eux un gros butin. Ils emmenèrent le butin à Samarie.
\TextTitle{Les captifs de Juda libérés grâce à Obed}
\VS{9}Or il y avait un prophète de Yahweh nommé Oded. Il sortit au-devant de cette armée qui revenait à Samarie, et leur dit : Voici, Yahweh, le Dieu de vos pères, étant indigné contre Juda, les a livrés entre vos mains, et vous les avez tués avec une colère telle qu'elle est parvenue aux cieux.
\VS{10}Et maintenant, vous pensez assujettir les fils de Juda et de Jérusalem pour serviteurs et pour servantes ! Mais n'êtes-vous pas également coupables envers Yahweh, votre Dieu ?
\VS{11}Maintenant écoutez-moi, et ramenez les prisonniers que vous vous êtes faits parmi vos frères ; car la colère ardente de Yahweh est sur vous.
\VS{12}Alors quelques-uns des chefs des fils d'Ephraïm, Azaria, fils de Jochanan, Bérékia, fils de Meschillémoth, Ezéchias, fils de Schallum, et Amasa, fils de Hadlaï, s'élevèrent contre ceux qui retournaient de la guerre,
\VS{13}et leur dirent : Vous ne ferez point entrer ici ces captifs. C'est pour nous rendre coupables devant Yahweh, voulez-vous en rajouter à nos péchés et à notre culpabilité ; car nous sommes déjà grandement coupables, et une colère ardente est sur Israël.
\VS{14}Alors les soldats abandonnèrent les captifs et le butin devant les chefs et toute l'assemblée.
\VS{15}Et des hommes, désignés par leurs noms, se levèrent, prirent les captifs, utilisèrent le butin pour revêtir tous ceux d'entre eux qui étaient nus avec des vêtements et des chaussures. Ils leur donnèrent à manger et à boire, les oignirent et ils conduisirent sur des ânes tous ceux qui étaient affaiblis pour les emmener à Jéricho, la ville des palmiers, auprès de leurs frères ; puis ils s'en retournèrent à Samarie.
\TextTitle{Achaz fait appel aux Assyriens\FTNTT{2 R. 15:29 ; 16:7-18}}
\VS{16}En ce temps-là, le roi Achaz envoya demander du secours aux rois d'Assyrie.
\VS{17}Les Edomites étaient revenus, avaient battu Juda et avaient emmené des prisonniers.
\VS{18}Les Philistins s'étaient aussi jetés sur les villes de la plaine et du sud de Juda ; et ils avaient pris Beth-Schémesch, Ajalon, Guedéroth, Soco et les villes de son ressort, Thimna et les villes de son ressort, Guimzo et les villes de son ressort, et ils y demeurèrent.
\VS{19}Car Yahweh humilia Juda, à cause d'Achaz, roi d'Israël, parce qu'il avait mis le désordre en Juda, et qu'il avait commis des transgressions contre Yahweh.
\VS{20}Tilgath-Pilnéser, roi d'Assyrie, vint vers lui ; mais il l'assiégea, et ne le fortifia pas.
\VS{21}Or Achaz dépouilla la maison de Yahweh, la maison du roi et celle des chefs, pour faire des dons au roi d'Assyrie, mais sans avoir du secours.
\TextTitle{Achaz irrite Yahweh par ses péchés}
\VS{22}Dans le temps de sa détresse, il continua à pécher contre Yahweh, lui, le roi Achaz.
\VS{23}Il sacrifia aux dieux de Damas qui l'avaient battu, et il dit : Puisque les dieux des rois de Syrie leur viennent en aide, je leur sacrifierai, afin qu'ils me viennent en aide. Mais ils furent la cause de sa chute et de celle de tout Israël.
\VS{24}Or Achaz rassembla les ustensiles de la maison de Dieu, et il mit en pièces les ustensiles de la maison de Dieu. Il ferma les portes de la maison de Yahweh, et se fit des autels dans tous les coins de Jérusalem.
\VS{25}Il fit des hauts lieux dans chaque ville de Juda, pour offrir des parfums à d'autres dieux ; et il irrita Yahweh, le Dieu de ses pères.
\TextTitle{Mort d'Achaz\FTNTT{2 R. 16:19-20}}
\VS{26}Quant au reste de ses actions et de toutes ses voies, les premières et les dernières, voici, elles sont écrites dans le livre des rois de Juda et d'Israël.
\VS{27}Puis Achaz s'endormit avec ses pères, et on l'ensevelit dans la ville de Jérusalem ; car on ne le mit point dans les sépulcres des rois d'Israël. Et Ezéchias, son fils, régna à sa place.
\Chap{29}
\TextTitle{Ezéchias règne sur Juda ; le réveil du peuple\FTNTT{2 R. 18:1-7 ; cp. Es. 36-39}}
\VerseOne{}Ezéchias devint roi à l'âge de vingt-cinq ans, et il régna vingt-neuf ans à Jérusalem\FTNT{ Es. 36 ; Es. 37 ; Es. 38, Es. 39 ; 2 R. 18:1-7 ;.}. Sa mère avait pour nom Abija, fille de Zacharie.
\VS{2}Il fit ce qui est droit aux yeux de Yahweh, tout comme avait fait David, son père.
\VS{3}La première année de son règne, au premier mois, il ouvrit les portes de la maison de Yahweh, et il les répara.
\VS{4}Il fit venir les sacrificateurs et les Lévites, et les rassembla dans la place orientale.
\VS{5}Et il leur dit : Ecoutez-moi, Lévites ! Sanctifiez-vous et sanctifiez la maison de Yahweh, le Dieu de vos pères, et ôtez du sanctuaire tout ce qui est impur.
\VS{6}Car nos pères ont péché, ils ont fait ce qui est mal aux yeux de Yahweh, notre Dieu. Ils l'ont abandonné, ils ont détourné leurs faces du tabernacle de Yahweh et lui ont tourné le dos.
\VS{7}Ils ont même fermé les portes du portique et ont éteint les lampes, ils n'ont fait ni monter d'offrande, ni brûler du parfum et des holocaustes au Dieu d'Israël dans le sanctuaire.
\VS{8}C'est pourquoi la colère de Yahweh a été sur Juda et sur Jérusalem ; et il les a livrés à de grands troubles, à la ruine et à la moquerie, comme vous le voyez de vos yeux.
\VS{9}Car voici, nos pères sont tombés par l'épée, et nos fils, nos filles et nos femmes sont en captivité.
\VS{10}Maintenant donc j'ai à cœur de traiter alliance avec Yahweh, le Dieu d'Israël, pour que son ardente colère se détourne de nous.
\VS{11}Or mes fils, cessez d'être négligents ; car Yahweh vous a choisis, afin que vous vous teniez devant lui à son service, comme ses serviteurs, pour lui brûler des parfums.
\VS{12}Les Lévites se levèrent : Machath, fils d'Amasaï, Joël, fils d'Azaria, des fils des Kehathites ; et des fils des Merarites, Kis, fils d'Abdi, Azaria, fils de Jehalléleel ; et des Guerschonites, Joach, fils de Zimma, et Eden, fils de Joach ;
\VS{13}et des fils d'Elitsaphan, Schimri et Jeïel ; et des fils d'Asaph, Zacharie et Matthania ;
\VS{14}et des fils d'Héman, Jehiel et Schimeï, et des fils de Jeduthun, Schemaeja et Uzziel.
\VS{15}Ils assemblèrent leurs frères, et ils se sanctifièrent ; puis ils entrèrent selon l'ordre du roi, et d'après la parole de Yahweh, pour purifier la maison de Yahweh.
\VS{16}Ainsi les sacrificateurs entrèrent à l'intérieur de la maison de Yahweh pour la purifier. Ils firent sortir dans le parvis de la maison de Yahweh toutes les impuretés qu'ils trouvèrent dans le temple de Yahweh. Les Lévites les prirent pour les emporter dehors, au torrent de Cédron.
\VS{17}Ils commencèrent à sanctifier le temple le premier jour du premier mois. Le huitième jour du mois, ils entrèrent au portique de Yahweh, et ils sanctifièrent la maison de Yahweh pendant huit jours. Le seizième jour du premier mois, ils avaient achevé.
\VS{18}Puis ils se rendirent chez le roi Ezéchias, et dirent : Nous avons purifié toute la maison de Yahweh, l'autel des holocaustes et ses ustensiles, la table des pains de proposition et ses ustensiles\FTNT{Ex. 29.}.
\VS{19}Nous avons remis en état et sanctifié tous les ustensiles que le roi Achaz avait rendus odieux pendant son règne, par ses transgressions ; ils sont maintenant devant l'autel de Yahweh.
\TextTitle{Nouvelle consécration du temple}
\VS{20}Alors le roi Ezéchias se leva de bonne heure, rassembla les chefs de la ville, et monta à la maison de Yahweh.
\VS{21}Ils amenèrent sept taureaux, sept béliers, sept agneaux et sept boucs sans défaut, en sacrifice pour le péché, pour le royaume, pour le sanctuaire et pour Juda\FTNT{Lé 4:3-26}. Puis le roi dit aux sacrificateurs, fils d'Aaron, de les faire monter en offrande sur l'autel de Yahweh.
\VS{22}Ils égorgèrent donc les bœufs, et les sacrificateurs recueillirent le sang et en aspergèrent l'autel ; ils égorgèrent les béliers et aspergèrent le sang sur l'autel ; ils égorgèrent les agneaux et aspergèrent le sang sur l'autel.
\VS{23}Puis on fit approcher les boucs pour le sacrifice du péché, devant le roi et devant l'assemblée, et ils posèrent leurs mains sur eux\FTNT{Lé 8:14.}.
\VS{24}Alors les sacrificateurs les égorgèrent, et offrirent en expiation leur sang vers l'autel, afin de faire propitiation pour tout Israël ; car le roi avait ordonné cet holocauste et ce sacrifice d'expiation pour tout Israël.
\VS{25}Il plaça aussi les Lévites dans la maison de Yahweh, avec des cymbales, des luths et des harpes, comme l'avait ordonné David, Gad, le voyant du roi, et Nathan le prophète ; car c'était un commandement de Yahweh, par ses prophètes.
\VS{26}Les Lévites se tinrent donc là avec les instruments de David, et les sacrificateurs avec les trompettes.
\VS{27}Alors Ezéchias ordonna de faire monter en offrande l'holocauste sur l'autel ; et au moment où commença l'holocauste, le cantique de Yahweh commença aussi, avec les trompettes et les instruments de David, roi d'Israël.
\VS{28}Toute l'assemblée se prosterna en chantant le cantique, et les trompettes sonnèrent ; et cela continua jusqu'à ce que l'holocauste fût achevé.
\VS{29}Et quand on eut achevé de faire monter l'holocauste, le roi et tous ceux qui se trouvaient avec lui fléchirent les genoux et se prosternèrent.
\VS{30}Puis le roi Ezéchias et les chefs dirent aux Lévites de célébrer Yahweh par les paroles de David et d'Asaph le voyant ; et ils le célébrèrent dans des réjouissances, et s'inclinèrent pour se prosterner.
\VS{31}Alors Ezéchias prit la parole, et dit : Vous avez maintenant consacré vos mains à Yahweh. Approchez-vous, amenez des sacrifices et faites des sacrifices de reconnaissances dans la maison de Yahweh. Et l'assemblée amena des sacrifices et firent des sacrifices de reconnaissances, et tous ceux qui étaient d'un cœur volontaire offrirent des holocaustes.
\VS{32}Le nombre des holocaustes que l'assemblée offrit fut de soixante-dix taureaux, cent béliers, deux cents agneaux, le tout en holocauste à Yahweh.
\VS{33}Et les autres choses consacrées furent, six cents bœufs, et trois brebis moutons.
\VS{34}Mais ils étaient peu de sacrificateurs et ne purent pas dépouiller tous les holocaustes ; les Lévites, leurs frères, les aidèrent jusqu'à ce que cette œuvre fut achevée, et jusqu'à ce que les autres sacrificateurs se fussent sanctifiés ; car les Lévites avaient eu plus à cœur de se sanctifier que les sacrificateurs.
\VS{35}Il y eut aussi un grand nombre d'holocaustes, avec les graisses des offrande de paix et avec les libations des holocaustes. Ainsi, le service de la maison de Yahweh fut rétabli.
\VS{36}Ezéchias et tout le peuple se réjouirent de ce que Dieu avait ainsi disposé le peuple ; car les choses se firent instantanément.
\Chap{30}
\TextTitle{Rétablissement de la Pâque}
\VerseOne{}Puis Ezéchias envoya dire à tout Israël et à Juda ; et il écrivit aussi des lettres à Ephraïm et à Manassé, pour les faire venir à la maison de Yahweh à Jérusalem, pour célébrer la Pâque en l'honneur de Yahweh, le Dieu d'Israël.
\VS{2}Le roi, ses chefs et toute l'assemblée avaient tenu un conseil à Jérusalem afin de célébrer la Pâque au second mois\FTNT{No. 9:10-11.} ;
\VS{3}car on ne pouvait la célébrer au temps ordinaire, parce qu'il n'y avait pas un nombre suffisant de sacrificateurs sanctifiés, et que le peuple n'était pas rassemblé à Jérusalem.
\VS{4}Le roi vit cela d'un bon œil ainsi que toute l'assemblée.
\VS{5}Ils décidèrent de faire une publication dans tout Israël, depuis Beer-Schéba jusqu'à Dan, pour que l'on vienne à Jérusalem célébrer la Pâque à Yahweh, le Dieu d'Israël. Car elle n'était pas célébrée par la multitude depuis longtemps conformément à ce qui était écrit.
\VS{6}Les coureurs allèrent donc avec des lettres de la part du roi et de ses chefs, partout en Israël et en Juda. Selon que le roi l'avait ordonné, ils disaient : Enfants d'Israël, retournez à Yahweh, le Dieu d'Abraham, d'Isaac et d'Israël, et afin qu'il revienne vers vous, qui êtes le reste échappé de la main des rois d'Assyrie.
\VS{7}Ne soyez pas comme vos pères ni comme vos frères, qui ont péché contre Yahweh, le Dieu de leurs pères, c'est pourquoi il les a livrés à la désolation, comme vous le voyez.
\VS{8}Maintenant, ne raidissez pas votre cou comme vos pères. Tendez les mains vers Yahweh, venez à son sanctuaire consacré pour toujours, servez Yahweh, votre Dieu, et son ardente colère se détournera de vous.
\VS{9}Car si vous revenez à Yahweh, vos frères et vos fils trouveront grâce auprès de ceux qui les ont emmenés captifs, et ils reviendront en ce pays parce que Yahweh, votre Dieu, est compatissant et miséricordieux ; et il ne détournera point sa face de vous, si vous revenez à lui.
\VS{10}Les coureurs passaient ainsi de ville en ville, par le pays d'Ephraïm et de Manassé jusqu'à Zabulon ; mais on riait et on se moquait d'eux.
\VS{11}Toutefois, quelques-uns d'Aser, de Manassé et de Zabulon s'humilièrent, et vinrent à Jérusalem.
\VS{12}La main de Dieu fut aussi sur Juda, pour leur donner un même cœur, afin d'exécuter l'ordre du roi et des chefs, selon la parole de Yahweh.
\VS{13}C'est pourquoi il s'assembla un grand peuple à Jérusalem pour célébrer la fête des pains sans levain\FTNT{Ex. 12:15 ; Lé. 23:6.}, au second mois. Ce fut une très grande assemblée.
\VS{14}Ils se levèrent et ôtèrent les autels qui étaient à Jérusalem ; ils ôtèrent aussi tous ceux où l'on brûlait de l'encens, et ils les jetèrent dans le torrent de Cédron.
\VS{15}Puis on immola la Pâque, au quatorzième jour du second mois ; car les sacrificateurs et les Lévites avaient eu honte et s'étaient sanctifiés, et ils amenèrent les holocaustes dans la maison de Yahweh.
\VS{16}Ils se tinrent à leur poste, selon leur charge, d'après la loi de Moïse, homme de Dieu. Et les sacrificateurs répandaient le sang qu'ils recevaient des mains des Lévites.
\VS{17}Car il y en avait un grand nombre dans cette assemblée qui ne s'étaient pas sanctifiés ; c'est pourquoi les Lévites eurent la charge d'immoler la Pâque pour tous ceux qui n'étaient pas purs, afin de les consacrer à Yahweh.
\VS{18}Car une grande partie du peuple, à savoir la plupart de ceux d'Ephraïm, de Manassé, d'Issacar et de Zabulon, ne s'étaient pas purifiés et mangèrent la Pâque contrairement à ce qui est écrit. Mais Ezéchias pria pour eux, en disant : Que Yahweh, qui est bon, tienne la propitiation pour faite,
\VS{19}pour quiconque a disposé son cœur à rechercher Dieu Yahweh, le Dieu de leurs pères, bien qu'il ne soit pas purifié conformément au sanctuaire !
\VS{20}Yahweh exauça Ezéchias, et guérit le peuple.
\VS{21}Les enfants d'Israël qui se trouvèrent à Jérusalem célébrèrent donc la fête des pains sans levain, pendant sept jours, dans une grande réjouissance ; les Lévites et les sacrificateurs célébraient Yahweh chaque jour, avec les instruments qui retentissaient à la louange de Yahweh.
\VS{22}Ezéchias parla au cœur de tous les Lévites, qui prêtaient une grande attention et de l'intelligence au service de Yahweh. Ils mangèrent pendant la fête, sept jours durant, offrant des sacrifices d'offrande de paix, et louant Yahweh, le Dieu de leurs pères.
\TextTitle{Sept jours supplémentaires pour la Pâque}
\VS{23}Puis toute l'assemblée fut d'avis de célébrer sept autres jours. Et ils célébrèrent ces sept jours dans la joie.
\VS{24}Car Ezéchias, roi de Juda, offrit à l'assemblée mille taureaux et sept mille brebis ; et les chefs donnèrent à l'assemblée mille taureaux et dix mille brebis ; et des sacrificateurs en grand nombre s'étaient sanctifiés.
\VS{25}Toute l'assemblée de Juda, avec les sacrificateurs et les Lévites, et toute l'assemblée venue d'Israël, ainsi que les étrangers venus du pays d'Israël, et ceux qui habitaient en Juda, se réjouirent.
\VS{26}Il y eut une grande joie à Jérusalem ; car depuis le temps de Salomon, fils de David, roi d'Israël, il ne s'était pas fait une telle chose dans Jérusalem.
\VS{27}Puis les sacrificateurs et les Lévites se levèrent et bénirent le peuple, et leur voix fut entendue, leur prière parvint jusqu'aux cieux, jusqu'à la sainte demeure de Yahweh.
\Chap{31}
\TextTitle{Destruction des idoles et organisation des services du temple}
\VerseOne{}Lorsque tout cela fut achevé, tous ceux d'Israël qui s'étaient retrouvés là, allèrent dans les villes de Juda, et brisèrent les statues, abattirent les idoles d'Astarté et renversèrent les hauts lieux et les autels, dans tout Juda et Benjamin, dans Ephraïm et Manassé, jusqu'à détruire tout\FTNT{2 R. 18:4.}. Puis tous les enfants d'Israël retournèrent dans leurs villes, chacun dans sa possession.
\VS{2}Et Ezéchias rétablit les classes des sacrificateurs et des Lévites, selon leur partage, chacun suivant sa charge, tant les sacrificateurs que les Lévites, pour les holocaustes et les offrandes de paix, pour faire le service, célébrer et chanter les louanges aux portes du camp de Yahweh.
\VS{3}Le roi donna une portion de ses biens pour les holocaustes, pour les holocaustes du matin et du soir, pour les holocaustes des sabbats, des nouvelles lunes et des fêtes, comme cela est écrit dans la loi de Yahweh.
\VS{4}Il dit au peuple, aux habitants de Jérusalem, de donner la portion des sacrificateurs et des Lévites, afin de s'appliquer à la loi de Yahweh.
\VS{5}Dès que la chose fut publiée, les enfants d'Israël amenèrent en abondance les prémices du blé, du moût, de l'huile, du miel et de tous les produits des champs ; ils apportèrent les dîmes de tout, en abondance.
\VS{6}Les enfants d'Israël et de Juda, qui demeuraient dans les villes de Juda, apportèrent aussi les dîmes du gros et du menu bétail, et les dîmes des choses saintes, qui étaient consacrées à Yahweh, leur Dieu ; et ils les mirent par tas.
\VS{7}Ils commencèrent à former les tas au troisième mois, et ils les achevèrent au septième mois.
\VS{8}Alors Ezéchias et les chefs vinrent voir les tas, et ils bénirent Yahweh et son peuple d'Israël.
\VS{9}Ezéchias interrogea les sacrificateurs et les Lévites au sujet de ces tas.
\VS{10}Le souverain sacrificateur Azaria, de la maison de Tsadok, lui répondit, et parla ainsi : Depuis qu'on a commencé à apporter des offrandes à la maison de Yahweh, nous avons mangé et avons été rassasiés, et il est resté cette grande quantité ; car Yahweh a béni son peuple, et cette grande quantité est le reste.
\VS{11}Alors Ezéchias leur dit de préparer des chambres dans la maison de Yahweh ; et ils les préparèrent.
\VS{12}On y apporta fidèlement les offrandes et les dîmes, les choses consacrées. Conania, le Lévite, en eut l'intendance, et Schimeï, son frère, était son second.
\VS{13}Jehiel, Azazia, Nachath, Asaël, Jerimoth, Jozabad, Eliel, Jismakia, Machath, et Benaja, étaient commis sous l'autorité de Conania et de Schimeï, son frère, d'après l'indication du roi Ezéchias, et d'Azaria, chef de la maison de Dieu.
\VS{14}Koré, le Lévite, fils de Jimna, portier de l'orient, avait la charge des offrandes volontaires offertes à Dieu, pour distribuer l'offrande élevée à Yahweh, et les choses consacrées et saintes.
\VS{15}Il avait sous sa direction Eden, Minjamin, Josué, Schemaeja, Amaria, et Schecania, dans les villes des sacrificateurs, pour distribuer fidèlement les portions à leurs frères, grands et petits, suivant leurs divisions,
\VS{16}à ceux qui étaient enregistrés comme mâles, depuis l'âge de trois ans et au-delà ; à tous ceux qui entraient dans la maison de Yahweh, pour le service quotidien, pour servir dans leurs charges et suivant leurs divisions ;
\VS{17}aux sacrificateurs et aux Lévites enregistrés selon la maison de leurs pères, depuis ceux de vingt ans et au-delà, selon leurs charges et selon leurs divisions ;
\VS{18}à ceux de toute l'assemblée enregistrés avec leurs petits enfants, leurs femmes, leurs fils et leurs filles ; car ils se consacraient avec fidélité aux choses saintes ;
\VS{19}et pour les enfants d'Aaron, les sacrificateurs, qui étaient à la campagne et dans les faubourgs de leurs villes, dans chaque ville, il y avait des gens désignés par leur nom, pour distribuer les portions à tous les mâles des sacrificateurs, et à tous les Lévites enregistrés.
\VS{20}Ezéchias en fit ainsi dans tout Juda ; et il fit ce qui est bon, droit et véritable, devant Yahweh, son Dieu.
\VS{21}Il travailla de tout son cœur et il réussit dans tout l'ouvrage qu'il entreprit pour le service de la maison de Dieu, et pour la loi, et pour les commandements, en recherchant son Dieu.
\Chap{32}
\TextTitle{Menaces de Sanchérib, roi d'Assyrie\FTNTT{2 R. 19:17-37 ; 19:8-13 ; Es. 36:2-20}}
\VerseOne{}Après que ces choses furent bien établies, Sanchérib, roi d'Assyrie, vint et entra en Juda, et campa contre les villes fortes, dans l'intention de faire une brèche\FTNT{Es. 36:2-21 ; 2 R. 18:13-37.}.
\VS{2}Ezéchias, voyant que Sanchérib était venu, et qu'il se tournait vers Jérusalem pour lui faire la guerre,
\VS{3}tint conseil avec ses chefs et ses vaillants hommes pour boucher les sources d'eau qui étaient hors de la ville, et ils l'aidèrent.
\VS{4}Un peuple nombreux s'assembla, et ils bouchèrent toutes les sources et le torrent qui coule par le milieu de la contrée, en disant : Pourquoi les rois d'Assyrie trouveraient-ils à leur venue de l'eau en abondance ?
\VS{5}Il se fortifia et rebâtit toute la muraille où il y avait une brèche, et l'éleva jusqu'aux tours ; il bâtit une autre muraille en dehors ; il répara Millo, dans la cité de David, et il fit faire beaucoup d'armes et de boucliers.
\VS{6}Il donna des chefs de guerre au peuple, les assembla auprès de lui sur la place de la porte de la ville, et parla à leur cœur, en disant :
\VS{7}Fortifiez-vous, soyez forts ! Ne craignez point et ne soyez pas effrayés devant le roi d'Assyrie et toute la multitude qui est avec lui ; car avec nous il y a quelqu'un de plus puissant.
\VS{8}Avec lui est le bras de la chair, mais avec nous est Yahweh, notre Dieu, pour nous aider et pour combattre dans nos combats. Et le peuple s'appuya sur les paroles d'Ezéchias, roi de Juda.
\VS{9}Après cela, Sanchérib, roi d'Assyrie, pendant qu'il était devant Lakis, ayant avec lui toutes les forces de son royaume, envoya ses serviteurs à Jérusalem vers Ezéchias, roi de Juda, et vers tous ceux de Juda qui étaient à Jérusalem, pour leur dire :
\VS{10}Ainsi parle Sanchérib, roi d'Assyrie : Sur qui vous confiez-vous pour que vous restiez à Jérusalem pour y être assiégés ?
\VS{11}Ezéchias ne vous incite-t-il pas pour vous livrer à la mort, par la famine et par la soif, en vous disant : Yahweh, notre Dieu, nous délivrera de la main du roi d'Assyrie ?
\VS{12}Cet Ezéchias n'a-t-il pas ôté les hauts lieux et les autels, et n'a-t-il pas ordonné à Juda et à Jérusalem : Vous vous prosternerez devant un seul autel pour y brûler le parfum ?
\VS{13}Ne savez-vous pas ce que nous avons fait, moi et mes pères, à tous les peuples des autres pays ? Les dieux des nations de ces pays ont-ils pu de quelque manière que ce soit délivrer leur pays de ma main ?
\VS{14}Quel est celui de tous les dieux de ces nations, que mes pères ont entièrement détruites, qui ait pu délivrer son peuple de ma main, pour que votre Dieu puisse vous délivrer de ma main ?
\VS{15}Maintenant donc, qu'Ezéchias ne vous abuse point, et qu'il ne vous incite plus de cette manière, et ne le croyez pas ; car aucun dieu d'aucune nation ni d'aucun royaume n'a pu délivrer son peuple de ma main ni de la main de mes pères ; combien moins votre Dieu vous délivrerait-il de ma main ?
\VS{16}Ses serviteurs parlèrent encore contre Yahweh Dieu, et contre Ezéchias, son serviteur.
\VS{17}Il écrivit aussi une lettre pour blasphémer contre Yahweh, le Dieu d'Israël, en parlant ainsi : Comme les dieux des nations des autres pays n'ont pu délivrer leur peuple de ma main, ainsi le Dieu d'Ezéchias ne pourra délivrer son peuple de ma main\FTNT{2 R. 19:14-37.}.
\VS{18}Et ses serviteurs crièrent à haute voix en langue judaïque, au peuple de Jérusalem qui était sur la muraille, pour les effrayer et les épouvanter, afin de prendre la ville.
\VS{19}Ils parlèrent du Dieu de Jérusalem comme des dieux des peuples de la terre, qui ne sont qu'un ouvrage de mains d'homme.
\TextTitle{Prière d'Ezéchias et exaucement de Yahweh\FTNTT{2 R. 19:14-37 ; Es. 36:21-37:35}}
\VS{20}Alors le roi Ezéchias, et Esaïe, le prophète, fils d'Amots, prièrent à ce sujet et crièrent vers les cieux.
\VS{21}Et Yahweh envoya un ange, dans le camp du roi d'Assyrie, qui extermina tous les vaillants hommes, les princes et les chefs, en sorte qu'il retourna dans son pays, dans la honte. Il entra dans la maison de son dieu ; et là, ceux qui étaient sortis de ses entrailles le firent tomber par l'épée.
\VS{22}C'est ainsi que Yahweh sauva Ezéchias et les habitants de Jérusalem de la main de Sanchérib, roi d'Assyrie, et de la main de tout homme, et il les protégea de toutes parts.
\VS{23}Plusieurs apportèrent des offrandes à Yahweh, à Jérusalem, et des choses précieuses à Ezéchias, roi de Juda, qui après cela fut élevé aux yeux de toutes les nations.
\TextTitle{Maladie et guérison d'Ezéchias\FTNTT{2 R. 20:1-11}}
\VS{24}En ces jours-là, Ezéchias fut malade à en mourir, et il pria Yahweh, qui l'exauça et lui accorda un prodige\FTNT{2 R. 20:1-11}.
\VS{25}Mais Ezéchias ne fut pas reconnaissant du bienfait qu'il avait reçu ; car son cœur s'éleva, et il y eut des maux contre lui, et contre Juda et Jérusalem.
\VS{26}Mais Ezéchias s'humilia de l'élévation de son cœur, lui et les habitants de Jérusalem, et la colère de Yahweh ne vint plus sur eux durant les jours d'Ezéchias.
\TextTitle{Fin du règne d'Ezéchias, sa mort\FTNTT{2 R. 20:12-21 ; cp. Es. 39}}
\VS{27}Ezéchias eut de très grandes richesses et de la gloire, et il se fit des trésors d'argent, d'or, de pierres précieuses, d'aromates, de boucliers, et de toutes sortes d'objets précieux ;
\VS{28}des magasins pour les récoltes de blé, de moût et d'huile, des étables pour toutes sortes de bétail, avec des rangées dans les étables.
\VS{29}Il se fit aussi des villes, et il acquit des troupeaux du gros et du menu bétail en abondance ; car Dieu lui avait donné de très grandes richesses.
\VS{30}Ce fut Ezéchias, qui boucha le canal du haut des eaux de Guihon, et les conduisit directement en bas, vers l'occident de la cité de David. Ainsi Ezéchias réussit dans tout ce qu'il fit.
\VS{31}Toutefois, lorsque les princes de Babylone envoyèrent des messagers vers lui pour s'informer du prodige qui s'était produit dans le pays, Dieu l'abandonna pour le mettre à l'épreuve, afin de connaître tout ce qui était dans son cœur\FTNT{Es. 29.}.
\VS{32}Le reste des actions d'Ezéchias, ses bonnes œuvres, voici, elles sont écrites dans la vision d'Esaïe, le prophète, fils d'Amots, dans le livre des rois de Juda et d'Israël.
\VS{33}Puis Ezéchias s'endormit avec ses pères, et on l'ensevelit au plus haut des sépulcres des fils de David ; et tout Juda, et Jérusalem lui firent honneur à sa mort, et Manassé, son fils régna à sa place.
\Chap{33}
\TextTitle{Manassé, roi impie de Juda\FTNTT{2 R. 21:1-9}}
\VerseOne{}Manassé était âgé de douze ans quand il devint roi, et il régna cinquante-cinq ans à Jérusalem.
\VS{2}Il fit ce qui est mal aux yeux de Yahweh, suivant les abominations des nations que Yahweh avait chassées devant les enfants d'Israël.
\VS{3}Il rebâtit les hauts lieux qu'Ezéchias, son père, avait démolis, il redressa les autels aux Baals, il fit des idoles d'Astarté, et se prosterna devant toute l'armée des cieux et la servit.
\VS{4}Il bâtit aussi des autels dans la maison de Yahweh, de laquelle Yahweh avait parlé ainsi : Mon Nom sera dans Jérusalem à jamais.
\VS{5}Il bâtit des autels à toute l'armée des cieux, dans les deux parvis de la maison de Yahweh.
\VS{6}Il fit passer ses fils par le feu dans la vallée du fils de Hinnom ; il pratiquait la magie, les sorcelleries et la voyance ; il établit des gens qui évoquaient les esprits et des devins. Il s'adonna à faire à l'extrême ce qui est mal aux yeux de Yahweh, pour l'irriter.
\VS{7}Il posa aussi une image taillée, une idole qu'il avait faite, dans la maison de Dieu, de laquelle Dieu avait dit à David, et à Salomon, son fils : Je mettrai à perpétuité mon Nom dans cette maison et dans Jérusalem, que j'ai choisie entre toutes les tribus d'Israël ;
\VS{8}et je ne ferai plus sortir Israël de la terre que j'ai assignée à leurs pères, pourvu seulement qu'ils prennent garde à faire tout ce que je leur ai ordonné, selon toute la loi, les préceptes et les ordonnances prescrites par Moïse.
\VS{9}Manassé donc fit s'égarer Juda et les habitants de Jérusalem, jusqu'à faire pire que les nations que Yahweh avait exterminées de devant les enfants d'Israël.
\TextTitle{Yahweh avertit Manassé\FTNTT{2 R. 21:10-16}}
\VS{10}Yahweh parla à Manassé et à son peuple ; mais ils ne furent pas attentifs.
\TextTitle{Manassé emmené captif se repent\FTNTT{2 R. 21:17-18}}
\VS{11}Alors Yahweh fit venir contre eux les chefs de l'armée du roi d'Assyrie, qui mirent Manassé dans les fers ; ils le lièrent d'une double chaîne d'airain, et l'emmenèrent à Babylone.
\VS{12}Et dès qu'il fut dans l'angoisse, il supplia Yahweh, son Dieu, et il s'humilia profondément devant le Dieu de ses pères.
\VS{13}Il lui adressa ses supplications, et Dieu se laissa fléchir par sa prière, et exauça sa supplication. Il le fit retourner à Jérusalem, dans son royaume. Manassé reconnut alors que c'est Yahweh qui est Dieu.
\VS{14}Après cela, il bâtit une muraille extérieure à la cité de David, vers l'occident de Guihon, dans la vallée, jusqu'à l'entrée de la porte des poissons ; il environna la colline et l'éleva à une grande hauteur ; il établit aussi des chefs d'armée dans toutes les villes fortes de Juda.
\VS{15}Il ôta de la maison de Yahweh l'idole, et les dieux étrangers, et tous les autels qu'il avait bâtis sur la montagne de la maison de Yahweh et à Jérusalem, et les jeta hors de la ville.
\VS{16}Puis il rebâtit l'autel de Yahweh et y offrit des sacrifices d'offrande de paix et de reconnaissance ; et il ordonna à Juda de servir Yahweh, le Dieu d'Israël.
\VS{17}Toutefois, le peuple sacrifiait encore dans les hauts lieux, mais seulement à Yahweh, son Dieu.
\VS{18}Le reste des actions de Manassé, et la prière qu'il fit à son Dieu, et les paroles des voyants qui lui parlaient, au Nom de Yahweh, le Dieu d'Israël, voilà, toutes ces choses sont écrites dans les actes des rois d'Israël.
\VS{19}Sa prière, et comment Dieu se laissa fléchir par sa prière, ses péchés et ses infidélités, les lieux sur lesquels il bâtit des hauts lieux, et dressa des idoles d'Astarté et des images taillées, avant de s'être humilié, voici cela est écrit dans le livre de Hozaï.
\VS{20}Puis Manassé s'endormit avec ses pères, et on l'ensevelit dans sa maison. Et Amon, son fils, régna à sa place.
\TextTitle{Amon règne brièvement sur Juda\FTNTT{2 R. 21:18-26}}
\VS{21}Amon était âgé de vingt-deux ans quand il devint roi, et il régna deux ans à Jérusalem.
\VS{22}Il fit ce qui est mal aux yeux de Yahweh, comme avait fait Manassé, son père. Il sacrifia à toutes les images taillées que Manassé, son père, avait faites, et il les servit.
\VS{23}Mais il ne s'humilia point devant Yahweh, comme s'était humilié Manassé, son père, mais se rendit de plus en plus coupable.
\VS{24}Et ses serviteurs ayant fait une conspiration contre lui, le firent mourir dans sa maison.
\VS{25}Mais le peuple du pays frappa tous ceux qui avaient conspiré contre le roi Amon. Et le peuple du pays établit pour roi, à sa place, Josias, son fils.
\Chap{34}
\TextTitle{Josias règne sur Juda ; ses réformes\FTNTT{2 R. 22:1-2}}
\VerseOne{}Josias était âgé de huit ans quand il devint roi, et il régna trente et un ans à Jérusalem.
\VS{2}Il fit ce qui est droit aux yeux de Yahweh. Il marcha dans les voies de David, son père ; et ne s'en détourna ni à droite ni à gauche.
\VS{3}La huitième année de son règne, lorsqu'il était jeune, il commença à rechercher le Dieu de David, son père ; et à la douzième année, il commença à purifier Juda et Jérusalem des hauts lieux, des idoles d'Astarté, et des images taillées, et des images de fonte.
\VS{4}On démolit dans sa présence les autels des Baals, et il abattit les tentes solaires\FTNT{Tentes solaires : lieux d’idolâtrie} qui étaient par-dessus. Il brisa les idoles d'Astarté, les images taillées et les images de fonte ; et les ayant réduites en poudre, il la répandit sur les sépulcres de ceux qui leur avaient sacrifié.
\VS{5}Puis il brûla les os des prêtres sur leurs autels, et il purifia ainsi Juda et Jérusalem.
\VS{6}Il fit la même chose dans les villes de Manassé, d'Ephraïm et de Siméon, et jusqu'à Nephthali, dans leurs ruines et tout autour.
\VS{7}Il démolit les autels et mit en pièces les idoles d'Astarté et les images taillées, et il les réduisit en poussière ; il abattit toutes les tentes solaires dans tout le pays d'Israël. Puis il revint à Jérusalem.
\TextTitle{Restauration du temple\FTNTT{2 R. 22:3-7}}
\VS{8}La dix-huitième année de son règne, après avoir purifié le pays et le temple, il envoya Schaphan, fils d'Atsalia, et Maaséja, chefs de la ville, et Joach, fils de Joachaz, commis sur les registres, pour réparer la maison de Yahweh, son Dieu.
\VS{9}Ils vinrent vers Hilkija, le souverain sacrificateur ; et on livra l'argent qui avait été apporté dans la maison de Dieu et que les Lévites, gardes du seuil, avaient amassé des mains de Manassé, d'Ephraïm et de tout le reste d'Israël, et aussi de tout Juda et Benjamin ; puis ils s'en retournèrent à Jérusalem.
\VS{10}On le remit entre les mains de ceux qui avaient la charge de l'ouvrage, qui étaient préposés sur la maison de Yahweh. Et ceux qui avaient la charge de l'ouvrage et qui travaillaient dans la maison de Yahweh le distribuèrent pour restaurer et réparer la maison de Yahweh.
\VS{11}Ils le donnèrent aux charpentiers et aux maçons, pour acheter des pierres de taille et du bois pour les poutres et pour la charpente des maisons que les rois de Juda avaient détruites.
\VS{12}Ces hommes s'employaient fidèlement à cet ouvrage. Jachath et Abdias, Lévites d'entre les fils de Merari, étaient préposés sur eux, et Zacharie et Meschullam, d'entre les fils des Kehathites, pour les diriger. Ces Lévites avaient tous de l'intelligence pour les instruments de musique.
\VS{13}Ils surveillaient ceux qui portaient les fardeaux, et dirigeaient tous ceux qui faisaient l'ouvrage, dans quelque service que ce soit ; les scribes, les administrateurs et les portiers, d'entre les Lévites.
\TextTitle{Le livre de la loi redécouvert\FTNTT{2 R. 22:8-10}}
\VS{14}Au moment où l'on sortit l'argent qui avait été apporté dans la maison de Yahweh, Hilkija, le sacrificateur, trouva le livre de la loi de Yahweh, donné par Moïse.
\VS{15}Alors Hilkija, prenant la parole, dit à Schaphan, le secrétaire : J'ai trouvé le livre de la loi dans la maison de Yahweh. Et Hilkija donna le livre à Schaphan.
\VS{16}Schaphan apporta le livre au roi, et rapporta tout au roi, en disant : Les mains de tes serviteurs ont fait tout ce qui leur a été donné à faire.
\VS{17}Ils ont amassé l'argent qui se trouvait dans la maison de Yahweh, et l'ont livré entre les mains des administrateurs, et entre les mains de ceux qui ont la charge de l'ouvrage.
\VS{18}Schaphan, le secrétaire, raconta en disant au roi : Hilkija, le sacrificateur, m'a donné un livre ; et Schaphan le lut devant le roi.
\TextTitle{Lecture du livre de la loi\FTNTT{2 R. 22:11-13}}
\VS{19}Lorsque le roi entendit les paroles de la loi, il déchira ses vêtements.
\VS{20}Il ordonna à Hilkija, à Achikam, fils de Schaphan, à Abdon, fils de Michée, à Schaphan, le secrétaire, et à Asaja, serviteur du roi, en disant :
\VS{21}Allez, consultez Yahweh pour moi et pour ce qui reste en Israël et en Juda, touchant les paroles du livre qui a été trouvé ; car la colère de Yahweh est grande, et elle s'est déversée sur nous, parce que nos pères n'ont point gardé la parole de Yahweh, pour faire selon tout ce qui est écrit dans ce livre.
\TextTitle{Instruction de la prophétesse Hulda\FTNTT{2 R. 22:14-20}}
\VS{22}Hilkija et les gens du roi allèrent vers Hulda, la prophétesse, femme de Schallum, fils de Thokehath, fils de Hasra, garde des vêtements, laquelle demeurait à Jérusalem, dans un autre quartier, et lui en parlèrent.
\VS{23}Alors elle leur répondit : Ainsi parle Yahweh, le Dieu d'Israël : Dites à l'homme qui vous a envoyés vers moi :
\VS{24}Ainsi parle Yahweh : Voici, je vais faire venir le malheur sur ce lieu et sur ses habitants, à savoir toutes les malédictions du serment qui sont écrites dans le livre qu'on a lu devant le roi de Juda.
\VS{25}Parce qu'ils m'ont abandonné, et qu'ils ont fait brûler des parfums aux autres dieux, pour m'irriter par toutes les œuvres de leurs mains, ma colère s'est déversée sur ce lieu, et elle ne sera point éteinte.
\VS{26}Mais quant au roi de Juda, qui vous a envoyés pour consulter Yahweh, vous lui direz : Ainsi parle Yahweh, le Dieu d'Israël, au sujet des paroles que tu as entendues :
\VS{27}Parce que ton cœur a été touché, et que tu t'es humilié devant Dieu, quand tu as entendu ses paroles contre ce lieu et contre ses habitants, et que t'étant humilié devant moi, tu as déchiré tes vêtements et pleuré devant moi, je t'ai aussi entendu, dit Yahweh.
\VS{28}Voici, je vais te recueillir avec tes pères, et tu seras recueilli dans tes sépulcres en paix, et tes yeux ne verront point tout ce mal que je vais faire venir sur ce lieu et sur ses habitants. Et ils rapportèrent cette parole au roi.
\TextTitle{Renouvellement de l'alliance avec Yahweh\FTNTT{2 R. 23:1-3}}
\VS{29}Alors le roi envoya assembler tous les anciens de Juda et de Jérusalem.
\VS{30}Le roi monta à la maison de Yahweh avec tous les hommes de Juda et les habitants de Jérusalem, les sacrificateurs et les Lévites, et tout le peuple, depuis le plus grand jusqu'au plus petit ; et on lut devant eux toutes les paroles du livre de l'alliance, qui avait été trouvé dans la maison de Yahweh.
\VS{31}Et le roi se tint debout à sa place ; et traita devant Yahweh cette alliance qu'ils suivraient Yahweh, et qu'ils garderaient ses commandements, ses témoignages et ses lois, chacun de tout son cœur et de toute son âme, en pratiquant les paroles de l'alliance écrites dans ce livre.
\VS{32}Et il fit tenir debout tous ceux qui se trouvèrent à Jérusalem et en Benjamin ; et les habitants de Jérusalem firent selon l'alliance de Dieu, le Dieu de leurs pères.
\VS{33}Josias ôta de tous les pays qui appartenaient aux enfants d'Israël, toutes les abominations ; et il obligea tous ceux qui se trouvaient en Israël à servir Yahweh, leur Dieu. Pendant toute sa vie, ils ne se détournèrent point de Yahweh, le Dieu de leurs pères.
\Chap{35}
\TextTitle{Josias rétablit la Pâque\FTNTT{2 R. 23:21-27}}
\VerseOne{}Or Josias célébra la Pâque à Yahweh à Jérusalem, et on immola la Pâque le quatorzième jour du premier mois.
\VS{2}Il rétablit les sacrificateurs dans leurs charges, et les encouragea au service de la maison de Yahweh.
\VS{3}Il dit aussi aux Lévites qui enseignaient tout Israël et qui étaient consacrés à Yahweh : Mettez l'arche sainte dans la maison que Salomon, fils de David, roi d'Israël, a bâtie. Qu'elle ne soit plus une charge sur vos épaules. Maintenant, servez Yahweh, votre Dieu, et son peuple d'Israël.
\VS{4}Préparez-vous, selon les maisons de vos pères, selon vos divisions suivant l'écrit de David, roi d'Israël, et suivant l'écrit de Salomon, son fils.
\VS{5}Tenez-vous dans le sanctuaire pour vos frères, les fils du peuple, selon les classes des maisons des pères, et selon que chaque famille des Lévites est partagée.
\VS{6}Immolez la Pâque, sanctifiez-vous, et préparez-la pour vos frères, afin qu'ils puissent la faire selon la parole que Yahweh a donnée par Moïse.
\VS{7}Josias éleva une offrande pour les gens du peuple et pour tous ceux qui se trouvaient là, des troupeaux d'agneaux et de chevreaux, au nombre de trente mille, et trois mille bœufs, le tout pour la Pâque ; cela fut pris sur les biens du roi.
\VS{8}Ses chefs élevèrent une offrande de bon gré pour le peuple, aux sacrificateurs et aux Lévites. Hilkija, Zacharie et Jehiel, princes de la maison de Dieu, donnèrent aux sacrificateurs, pour la Pâque, deux mille six cents agneaux, et trois cents bœufs.
\VS{9}Conania, Schemaeja et Nethaneel, ses frères, et Haschabia, Jeïel et Jozabad, qui étaient les princes des Lévites, élevèrent une offrande de cinq mille agneaux aux Lévites pour faire la Pâque, et cinq cents bœufs.
\VS{10}Le service étant préparé, les sacrificateurs se tinrent à leurs postes, et les Lévites suivant leurs divisions, selon l'ordre du roi.
\VS{11}Puis on immola la Pâque ; et les sacrificateurs répandaient le sang reçu de leurs mains, et les Lévites les dépouillaient.
\VS{12}Ils mirent à part les holocaustes, pour les donner aux gens du peuple, suivant les divisions des maisons de leurs pères, afin de les offrir à Yahweh, selon ce qui est écrit au livre de Moïse ; ils firent de même pour les bœufs.
\VS{13}Ils firent cuire la Pâque au feu, selon l'ordonnance ; et ils firent cuire dans des chaudières, des chaudrons et des poêles, les choses consacrées ; et ils les apportèrent rapidement à tous les gens du peuple.
\VS{14}Puis ils apprêtèrent ce qui était pour eux et pour les sacrificateurs, car les sacrificateurs, fils d'Aaron, furent occupés jusqu'à la nuit à élever en offrande les holocaustes et les graisses ; c'est pourquoi les Lévites apprêtèrent ce qui était pour eux et pour les sacrificateurs, fils d'Aaron.
\VS{15}Les chantres, fils d'Asaph, étaient à leur place, selon l'ordre de David, d'Asaph, d'Héman et de Jeduthun, le voyant du roi. Les portiers étaient à chaque porte, ils n'eurent pas à se détourner de leur service, car leurs frères les Lévites apprêtaient ce qui était pour eux.
\VS{16}Ainsi, tout le service de Yahweh, en ce jour-là, fut réglé pour faire la Pâque et pour élever en offrande les holocaustes sur l'autel de Yahweh, selon l'ordre du roi Josias.
\VS{17}Les fils d'Israël qui s'y trouvèrent célébrèrent donc la Pâque en ce temps-là, et la fête des pains sans levain pendant sept jours.
\VS{18}Or on n'avait point célébré en Israël de Pâque semblable à celle-là depuis les jours de Samuel le prophète ; et aucun des rois d'Israël n'avait célébré une Pâque pareille comme le fit Josias, avec les sacrificateurs et les Lévites, et tout Juda et Israël, qui s'y étaient trouvés avec les habitants de Jérusalem.
\VS{19}Cette Pâque fut célébrée la dix-huitième année du règne de Josias.
\TextTitle{Blessure et mort de Josias\FTNTT{2 R. 23:28-30}}
\VS{20}Après tout cela, quand Josias eut réparé la maison de Yahweh, Néco, roi d'Egypte, monta pour faire la guerre à Carkemisch, sur l'Euphrate. Josias sortit à sa rencontre.
\VS{21}Mais Néco envoya vers lui des messagers pour lui dire : Qu'y a-t-il entre nous, roi de Juda ? Ce n'est pas à toi que j'en veux aujourd'hui, mais à une maison qui me fait la guerre ; et Dieu m'a dit de me hâter. Désiste-toi donc de venir contre Dieu, qui est avec moi, de peur qu'il ne te détruise.
\VS{22}Cependant Josias ne se détourna point de lui, mais se déguisa pour combattre contre lui et il n'écouta pas les paroles de Néco, qui venaient de la bouche de Dieu. Il vint donc pour combattre dans la vallée de Meguiddo.
\VS{23}Les archers tirèrent sur le roi Josias ; et le roi dit à ses serviteurs : Emportez-moi, car je suis très blessé.
\VS{24}Ses serviteurs l'ôtèrent du char, le mirent sur un second char qu'il avait, et le menèrent à Jérusalem. Il mourut et il fut enseveli dans les sépulcres de ses pères, et tous ceux de Juda et de Jérusalem menèrent le deuil de Josias.
\VS{25}Jérémie fit aussi des lamentations sur Josias ; et tous les chanteurs et toutes les chanteuses parlèrent dans leurs complaintes sur Josias jusqu'à ce jour ; et on en a fait une coutume en Israël. Voici, ces choses sont écrites dans les lamentations.
\VS{26}Le reste des actions de Josias, et ses œuvres de piété, selon ce qui est écrit dans la loi de Yahweh,
\VS{27}ses premières et ses dernières actions, sont écrites dans le livre des rois d'Israël et de Juda.
\Chap{36}
\TextTitle{Joachaz règne brièvement sur Juda\FTNTT{2 R. 23:31-33}}
\VerseOne{}Alors le peuple du pays prit Joachaz, fils de Josias, et on l'établit roi à Jérusalem, à la place de son père.
\VS{2}Joachaz était âgé de vingt-trois ans quand il devint roi, et il régna trois mois à Jérusalem.
\VS{3}Le roi d'Egypte le destitua à Jérusalem, et condamna le pays à une amende de cent talents d'argent et d'un talent d'or.
\TextTitle{Règne de Jojakim, déportation à Babylone\FTNTT{2 R. 23:34-24:4-9}}
\VS{4}Le roi d'Egypte établit pour roi sur Juda et Jérusalem Eliakim, frère de Joachaz ; et changea son nom en celui de Jojakim. Puis Néco prit Joachaz, son frère, et l'emmena en Egypte.
\VS{5}Jojakim était âgé de vingt-cinq ans quand il devint roi, et il régna onze ans à Jérusalem. Il fit ce qui est mal aux yeux de Yahweh, son Dieu.
\VS{6}Nebucadnetsar, roi de Babylone, monta contre lui et le lia de doubles chaînes d'airain pour le mener à Babylone.
\VS{7}Nebucadnetsar emporta aussi à Babylone des ustensiles de la maison de Yahweh, et il les mit dans son temple à Babylone.
\VS{8}Le reste des actions de Jojakim, et les abominations qu'il commit, et ce qui fut trouvé en lui, cela est écrit dans le livre des rois d'Israël et de Juda. Et Jojakin, son fils, régna à sa place.
\VS{9}Jojakin était âgé de huit ans quand il devint roi, et il régna trois mois et dix jours à Jérusalem. Il fit ce qui est mal aux yeux de Yahweh.
\TextTitle{Sédécias le dernier roi de Juda, autres déportations à Babylone\FTNTT{2 R.24:10-20 ; cp. 2 R. 25:1-21 ; Jé. 39:8-10}}
\VS{10}Et l'année suivante, le roi Nebucadnetsar envoya, et le fit emmener à Babylone avec les ustensiles précieux de la maison de Yahweh ; et il établit roi sur Juda et Jérusalem, Sédécias, son frère.
\VS{11}Sédécias était âgé de vingt et un ans quand il devint roi, et il régna onze ans à Jérusalem.
\VS{12}Il fit ce qui est mal aux yeux de Yahweh, son Dieu ; et il ne s'humilia point devant Jérémie le prophète, qui lui parlait de la part de Yahweh.
\VS{13}Et même il se rebella contre le roi Nebucadnetsar, qui l'avait fait prêter serment par le Nom de Dieu. Il raidit son cou, et il obstina son cœur pour ne point retourner à Yahweh, le Dieu d'Israël.
\VS{14}Pareillement, tous les chefs des sacrificateurs et le peuple furent infidèles et continuèrent de plus en plus à pécher, selon toutes les abominations des nations ; et ils souillèrent la maison que Yahweh avait sanctifiée dans Jérusalem.
\VS{15}Or Yahweh, le Dieu de leurs pères, les avait sommés par ses messagers qu'il envoya de bonne heure, car il voulait épargner son peuple et sa propre demeure.
\VS{16}Mais ils se moquèrent des messagers de Dieu, ils méprisèrent ses paroles et traitèrent ses prophètes de séducteurs, jusqu'à ce que la fureur de Yahweh montat contre son peuple au point qu'il n'y eut plus de remède.
\VS{17}C'est pourquoi il fit monter contre eux le roi des Chaldéens, qui tua par l'épée leurs jeunes gens dans la maison de leur sanctuaire ; il n'épargna ni le jeune homme, ni la vierge, ni le vieillard, ni l'homme à cheveux blancs ; il les livra tous entre ses mains.
\VS{18}Il fit apporter à Babylone tous les ustensiles de la maison de Dieu, grands et petits, les trésors de la maison de Yahweh, et les trésors du roi et ceux de ses chefs.
\VS{19}Ils brûlèrent la maison de Dieu, ils démolirent les murailles de Jérusalem, ils livrèrent au feu tous ses palais et détruisirent tout ce qu'il y avait comme objets précieux.
\VS{20}Puis le roi de Babylone transporta à Babylone le reste qui échappa à l'épée, et ils furent ses esclaves et ceux de ses fils, jusqu'à la domination du royaume de Perse,
\VS{21}afin que la parole de Yahweh, prononcée par la bouche de Jérémie, fût accomplie ; jusqu'à ce que la terre eût pris plaisir à ses sabbats et durant tous les jours qu'elle demeura dévastée ; elle se reposa pour accomplir les soixante-dix années.
\TextTitle{L'édit de Cyrus autorise les juifs à retourner dans leurs villes}
\VS{22}Mais la première année de Cyrus, roi de Perse, afin que la parole de Yahweh prononcée par Jérémie fût accomplie, Yahweh réveilla l'esprit de Cyrus, roi de Perse, qui fit publier dans tout son royaume, et même par écrit, en disant :
\VS{23}Ainsi parle Cyrus, roi de Perse : Yahweh, le Dieu des cieux, m'a donné tous les royaumes de la terre, et lui-même m'a ordonné de lui bâtir une maison à Jérusalem, qui est en Juda. Qui d'entre vous est de son peuple ? Que Yahweh, son Dieu, soit avec lui, et qu'il monte !
\PPE{}
\end{multicols}
\clearpage
\addcontentsline{toc}{chapter}{Evangiles}\clearpage
\clearpage\makeatletter\def\@evenhead{}\def\@oddhead{}\makeatother

\vspace*{\fill}
\begin{center}
{\Huge Les 4 Evangiles}
\end{center}
\vspace*{\fill}

\clearpage

\makeatletter\def\@evenhead{{\NoAutoSpaceBeforeFDP{\small{\rightmark\hfil\thepage\hfil\leftmark}}}}\def\@oddhead{{\NoAutoSpaceBeforeFDP{\small{\rightmark\hfil\thepage\hfil\leftmark}}}}\makeatother

\clearpage\ShortTitle{Matthieu}\BookTitle{Matthieu}\BFont
\noindent\hrulefill
{\footnotesize
\textit{
\bigskip
{\centering{}
\\Auteur : Matthieu
\\(Gr. : Matthaios)
\\Signifie : Don de Yahweh
\\Thème : Jésus le Roi
\\Date de rédaction : Env. 50 ap. J.-C.\\}
}
%\bigskip
\textit{
\\Matthieu, également connu sous le nom de Lévi, était un juif percepteur d'impôts au service des Romains. Appelé par
Jésus-Christ à Capernaüm et choisi pour être l'un des douze disciples, il offrit un banquet en l'honneur de Jésus dans
sa maison, ce qui lui valut l'hostilité des pharisiens. Rédigé à Antioche de Syrie, son évangile était destiné à des juifs
convertis comme en témoignent ses nombreuses allusions à l'Ancienne Alliance.
%\bigskip
\\Matthieu, démontre l'hégémonie de Jésus, fils de David, fils d'Abraham, roi d'Israël. Il évoque son règne qui se manifestera un jour physiquement, lorsque le Roi jugera tous les hommes du « trône de sa gloire ». Il fait concorder les paroles et les événements de la vie de Jésus avec les prophéties de l'Ancienne Alliance.
%\bigskip
\\Son récit exalte la royauté de Jésus et expose l'Evangile du Royaume.\bigskip
}
}
\par\nobreak\noindent\hrulefill
\begin{multicols}{2}
\Chap{1}
\TextTitle{Présentation du Roi : La généalogie de Jésus-Christ}
\VerseOne{}Livre de la généalogie de Jésus-Christ, fils de David, fils d'Abraham.
\VS{2}Abraham engendra Isaac ; Isaac engendra Jacob ; Jacob engendra Juda et ses frères ;
\VS{3}Juda engendra Pérets et Zara, de Thamar ; Pérets engendra Esrom ; Esrom engendra Aram ;
\VS{4}Aram engendra Aminadab ; Aminadab engendra Naasson ; et Naasson engendra Salmon ;
\VS{5}Salmon engendra Boaz, de Rahab\FTNT{Rahab était une prostituée cananéenne qui est devenue l'ancêtre du Messie (Jos. 6).} ; Boaz engendra Obed, de Ruth\FTNT{Ruth était une Moabite, son peuple était issu de la relation incestueuse de Lot et sa fille aînée (Ge. 19:36-37). Elle est devenue l'ancêtre du Messie (Ru. 4:17).} ; Obed engendra Isaï ;
\VS{6}Isaï engendra le roi David ; le roi David engendra Salomon, de la femme d'Urie ;
\VS{7}Salomon engendra Roboam ; Roboam engendra Abia ; Abia engendra Asa ;
\VS{8}Asa engendra Josaphat ; Josaphat engendra Joram ; Joram engendra Ozias ;
\VS{9}Ozias engendra Joatham ; Joatham engendra Achaz ; Achaz engendra Ezéchias ;
\VS{10}Ezéchias engendra Manassé ; Manassé engendra Amon ; Amon engendra Josias ;
\VS{11}Josias engendra Jéchonias et ses frères, au temps de la déportation à Babylone.
\VS{12}Après la déportation à Babylone, Jéchonias engendra Salathiel ; Salathiel engendra Zorobabel ;
\VS{13}Zorobabel engendra Abiud ; Abiud engendra Eliakim ; Eliakim engendra Azor ;
\VS{14}Azor engendra Sadok ; Sadok engendra Achim ; Achim engendra Eliud ;
\VS{15}Eliud engendra Eléazar ; Eléazar engendra Matthan ; Matthan engendra Jacob ;
\VS{16}Jacob engendra Joseph, l'époux de Marie, de laquelle est né Jésus, qui est appelé Christ\FTNT{Christ : Du grec « christos », ce qui signifie « oint », est l'équivalent grec du mot hébreu « mashiyach », traduit par « messie » en français (Da. 9:25-26). C'est le titre officiel du Seigneur Jésus.}.
\VS{17}Ainsi, il y a en tout quatorze générations depuis Abraham jusqu'à David, quatorze générations depuis David jusqu'à la déportation à Babylone, et quatorze générations depuis la déportation à Babylone jusqu'au Christ.
\TextTitle{Naissance miraculeuse de Jésus-Christ\FTNTT{Lu. 1:26-38 ; 2:1-7 ; Jn. 1:1-2,14}}
\VS{18}Voici de quelle manière arriva la naissance de Jésus-Christ. Marie, sa mère, ayant été fiancée à Joseph, se trouva enceinte par l'opération du Saint-Esprit, avant qu'ils aient habité ensemble.
\VS{19}Joseph, son époux, qui était un homme juste et qui ne voulait pas la diffamer, se proposa de la répudier secrètement.
\VS{20}Mais comme il y pensait, voici, l'Ange du Seigneur lui apparut en songe et lui dit : Joseph, fils de David, ne crains point de prendre avec toi Marie, ta femme, car l'enfant qu'elle a conçu est du Saint-Esprit.
\VS{21}Elle enfantera un fils, et tu lui donneras le nom de Jésus. C'est lui qui sauvera son peuple de ses péchés.
\VS{22}Tout cela arriva afin que s'accomplisse ce que le Seigneur avait annoncé par le prophète :
\VS{23}Voici, la vierge deviendra enceinte, elle enfantera un fils ; et on lui donnera le nom d'Emmanuel\FTNT{Es. 7:14.}, ce qui signifie, Dieu est avec nous.
\VS{24}Joseph s'étant donc réveillé de son sommeil, fit ce que l'Ange du Seigneur lui avait ordonné, et il prit sa femme.
\VS{25}Mais il ne la connut point jusqu'à ce qu'elle ait enfanté son fils premier-né, auquel il donna le nom de Jésus.
\Chap{2}
\TextTitle{Les mages adorent Jésus}
\VerseOne{}Jésus étant né à Bethléhem, ville de Juda, au temps du roi Hérode, voici des mages d'orient arrivèrent à Jérusalem.
\VS{2}En disant : Où est le Roi des Juifs qui vient de naître ? Car nous avons vu son étoile en orient, et nous sommes venus l'adorer.
\VS{3}Le roi Hérode ayant entendu, fut troublé et tout Jérusalem avec lui.
\VS{4}Et ayant assemblé tous les principaux sacrificateurs et les scribes du peuple, il s'informa auprès d'eux où le Christ devait naître.
\VS{5}Et ils lui dirent : A Bethléhem, ville de Judée ; car voici ce qui a été écrit par le prophète :
\VS{6}Et toi, Bethléhem, terre de Juda, tu n'es nullement la plus petite parmi les gouverneurs de Juda, car de toi sortira le Chef qui paîtra mon peuple d'Israël\FTNT{Mi. 5:1.}.
\VS{7}Alors Hérode ayant appelé en secret les mages, s'informa soigneusement auprès d'eux depuis combien de temps brillait l'étoile.
\VS{8}Puis il les envoya à Bethléhem, en leur disant : Allez, et prenez des informations exactes sur le petit enfant ; et quand vous l'aurez trouvé, faites-le-moi savoir, afin que j'aille aussi moi-même l'adorer.
\VS{9}Après avoir entendu le roi, ils partirent. Et voici, l'étoile\FTNT{Le Seigneur Jésus-Christ s'est révélé à Jean comme l'étoile brillante du matin (Ap. 22:16).} qu'ils avaient vue en orient allait devant eux, jusqu'au moment où, arrivée au-dessus du lieu où était le petit enfant, elle s'arrêta.
\VS{10}Quand ils virent l'étoile, ils furent saisis d'une très grande joie.
\VS{11}Ils entrèrent dans la maison, virent le petit enfant avec Marie sa mère, se prosternèrent et l'adorèrent. Ils ouvrirent ensuite leurs trésors et lui offrirent des présents : De l'or, de l'encens et de la myrrhe.
\VS{12}Puis, divinement avertis en songe de ne pas retourner vers Hérode, ils regagnèrent leur pays par un autre chemin.
\TextTitle{Fuite en Egypte}
\VS{13}Lorsqu'ils furent partis, voici, l'Ange du Seigneur apparut dans un songe à Joseph et lui dit : Lève-toi et prends le petit enfant et sa mère, fuis en Egypte, et demeure là, jusqu'à ce que je te le dise ; car Hérode cherchera le petit enfant pour le faire mourir.
\VS{14}Joseph donc étant réveillé, prit de nuit le petit enfant et sa mère, et se retira en Egypte.
\VS{15}Il y resta là jusqu'à la mort d'Hérode ; afin que s'accomplisse ce que le Seigneur avait annoncé par le prophète : J'ai appelé mon Fils hors d'Egypte\FTNT{Os. 11:1}.
\TextTitle{Hérode envoie tuer des enfants innocents}
\VS{16}Alors Hérode voyant que les mages s'étaient moqués de lui, se mit dans une grande colère, et il envoya tuer tous les enfants qui étaient à Bethléhem, et dans tout son territoire ; depuis l'âge de deux ans, et au-dessous, selon la date dont il s'était exactement enquis auprès des mages.
\VS{17}Alors s'accomplit ce qui avait été annoncé par Jérémie le prophète :
\VS{18}On a entendu à Rama des cris, des lamentations, des plaintes, et des grands gémissements : Rachel pleure ses enfants et n'a pas voulu être consolée, parce qu'ils ne sont plus\FTNT{Jé. 31:15}.
\TextTitle{Joseph revient en Israël et s'installe à Nazareth\FTNTT{Lu. 2:39-52}}
\VS{19}Mais après qu'Hérode fut mort, voici, l'Ange du Seigneur apparut dans un songe à Joseph en Egypte,
\VS{20}et lui dit : Lève-toi, et prends le petit enfant et sa mère, et va dans le pays d'Israël ; car ceux qui cherchaient à ôter la vie au petit enfant sont morts.
\VS{21}Joseph donc s'étant réveillé, prit le petit enfant et sa mère, et alla dans le pays d'Israël.
\VS{22}Mais quand il eut appris qu'Archélaüs régnait en Judée, à la place d'Hérode son père, il craignit d'y aller ; et étant divinement averti dans un songe, il se retira dans le territoire de la Galilée,
\VS{23}et vint habiter dans la ville appelée Nazareth ; afin que s'accomplisse ce qui avait été dit par les prophètes : Il sera appelé Nazaréen.
\Chap{3}
\TextTitle{Ministère de Jean-Baptiste\FTNTT{Mc. 1:1-8 ; Lu. 3:1-20 ; Jn. 1:6-8,15-37}}
\VerseOne{}Or, en ce temps-là arriva Jean-Baptiste, prêchant dans le désert de la Judée.
\VS{2}Il disait : Repentez-vous, car le Royaume des cieux est proche.
\VS{3}Car c'est celui dont Esaïe le prophète a parlé, en disant : C'est ici la voix de celui qui crie dans le désert : Préparez le chemin du Seigneur, aplanissez ses sentiers.
\VS{4}Jean avait un vêtement de poils de chameau et une ceinture de cuir autour de ses reins. Et il se nourrissait de sauterelles et de miel sauvage.
\VS{5}Alors les habitants de Jérusalem, et de toute la Judée, et de tout le pays des environs du Jourdain, vinrent à lui ;
\VS{6}et confessant leurs péchés, ils se faisaient baptiser par lui dans le Jourdain.
\VS{7}Mais, voyant venir à son baptême beaucoup de pharisiens et de sadducéens, il leur dit : Race de vipères, qui vous a appris à fuir la colère à venir ?
\VS{8}Produisez donc des fruits convenables à la repentance
\VS{9}et ne prétendez pas dire en vous-mêmes : Nous avons Abraham pour père ! Car je vous dis que Dieu peut faire naître de ces pierres mêmes des enfants à Abraham.
\VS{10}Et déjà la cognée est mise à la racine des arbres ; c'est pourquoi tout arbre qui ne produit pas de bons fruits sera coupé et jeté au feu.
\VS{11}Pour moi, je vous baptise d'eau en signe de repentance ; mais celui qui vient après moi est plus puissant que moi, et je ne suis pas digne de porter ses souliers ; celui-là vous baptisera du Saint-Esprit et de feu\FTNT{Le baptême du Saint-Esprit ne doit pas être confondu avec la plénitude du Saint-Esprit. Le baptême est un acte définitif qui nous greffe au corps du Christ lors de la conversion (1 Co. 12:13). La plénitude consiste quant à elle en un constant renouvellement que nous devons impérativement rechercher (Ep. 5:18). Certains courants chrétiens charismatiques enseignent que le parler en langues est le signe distinctif du baptême du Saint-Esprit. Cette doctrine est basée sur au moins trois passages : Ac. 2:4 ; Ac. 10:44-46 et Ac. 19:1-7. Si cela était vraiment le cas, plusieurs chrétiens seraient encore dans leurs péchés et n'appartiendraient pas au Seigneur Jésus-Christ. En effet, Ro. 8:9 déclare ceci : « Si quelqu'un n'a pas l'Esprit de Christ, il ne lui appartient pas ». Or il est manifeste que bon nombre de chrétiens nés d'en haut ne parlent pas en langues, ce qui est d'ailleurs attesté par l'apôtre Paul (1 Co. 12:30). Il n'y a aucun verset dans les Ecritures qui nous ordonne de chercher le baptême du Saint-Esprit pour la bonne et simple raison que nous le recevons à la conversion.}.
\VS{12}Il a son van à la main, et il nettoiera entièrement son aire, et il assemblera son froment dans le grenier ; mais il brûlera la paille dans un feu qui ne s'éteint point.
\TextTitle{Jean baptise Jésus-Christ\FTNTT{Mc. 1:9-11 ; Lu. 3:21-22 ; Jn. 1:31-34}}
\VS{13}Alors Jésus vint de Galilée au Jourdain vers Jean pour être baptisé par lui.
\VS{14}Mais Jean l'en empêchait avec force en lui disant : J'ai besoin d'être baptisé par toi, et tu viens vers moi ?
\VS{15}Et Jésus répondit en disant : Laisse-moi faire pour le moment, car il nous est ainsi convenable d'accomplir tout ce qui est juste. Et alors il le laissa faire.
\VS{16}Dès que Jésus eut été baptisé, il sortit aussitôt hors de l'eau. Et voici, les cieux lui furent ouverts, et Jean vit l'Esprit de Dieu descendant comme une colombe et venant sur lui.
\VS{17}Et voici une voix du ciel déclara : Celui-ci est mon Fils bien-aimé en qui j'ai mis toute mon affection.
\Chap{4}
\TextTitle{La tentation\FTNTT{Ge. 3:6 ; Mc. 1:12-13 ; Lu. 4:1-13 ; 1 Jn. 2:16.}}
\VerseOne{}Alors Jésus fut emmené par l'Esprit dans le désert, pour être tenté par le diable.
\VS{2}Après avoir jeûné quarante jours et quarante nuits, finalement il eut faim.
\VS{3}Et le tentateur s'étant approché, lui dit : Si tu es le Fils de Dieu, ordonne que ces pierres deviennent des pains.
\VS{4}Mais Jésus répondit et dit : Il est écrit : L'homme ne vivra point de pain seulement, mais de toute parole qui sort de la bouche de Dieu\FTNT{De. 8:3.}.
\VS{5}Alors le diable le transporta dans la sainte ville et le mit sur le haut du temple ;
\VS{6}et il lui dit : Si tu es le Fils de Dieu, jette-toi en bas ; car il est écrit : Il ordonnera à ses anges de te porter sur leurs mains de peur que ton pied ne heurte contre une pierre\FTNT{Ps. 91:12-13.}.
\VS{7}Jésus lui dit : Il est aussi écrit : Tu ne tenteras point le Seigneur ton Dieu\FTNT{De. 6:16.}.
\VS{8}Le diable le transporta encore sur une forte haute montagne, et lui montra tous les royaumes du monde et leur gloire ;
\VS{9}et il lui dit : Je te donnerai toutes ces choses, si tu te prosternes et m'adores.
\VS{10}Mais Jésus lui dit : Retire-toi Satan ! Car il est écrit : Tu adoreras le Seigneur ton Dieu, et tu le serviras lui seul\FTNT{De. 6:13 ; De. 10:20.}.
\VS{11}Alors le diable le laissa. Et voici, des anges s'approchèrent et le servirent.
\TextTitle{Etablissement de Jésus à Capernaüm\FTNTT{Mc. 1:14-15 ; Lu. 4:14-15}}
\VS{12}Jésus, ayant appris que Jean avait été mis en prison, se retira dans la Galilée.
\VS{13}Et ayant quitté Nazareth, il alla demeurer à Capernaüm, ville maritime, sur les confins de Zabulon et de Nephthali ;
\VS{14}afin que s'accomplisse ce qui avait été annoncé par Esaïe, le prophète, en disant :
\VS{15}Le pays de Zabulon et le pays de Nephthali, de la contrée voisine de la mer, au-delà du Jourdain, et la Galilée des Gentils ;
\VS{16}Ce peuple, assis dans les ténèbres, a vu une grande lumière ; et à ceux qui étaient assis dans la région et l'ombre de la mort, la lumière elle-même s'est levée\FTNT{Es. 9:1.}.
\VS{17}Dès lors, Jésus commença à prêcher et à dire : Repentez-vous, car le Royaume des cieux est proche.
\TextTitle{Appel de Pierre, André, Jacques et Jean\FTNTT{Mc. 1:16-20 ; Lu. 5:1-11 ; Jn. 1:35-51}}
\VS{18}Comme Jésus marchait le long de la mer de Galilée, il vit deux frères, Simon, appelé Pierre, et André, son frère, qui jetaient leurs filets dans la mer ; car ils étaient pêcheurs.
\VS{19}Et il leur dit : Suivez-moi et je vous ferai pêcheurs d'hommes.
\VS{20}Et ayant aussitôt quitté leurs filets, ils le suivirent.
\VS{21}Et de là étant allé plus en avant, il vit deux autres frères, Jacques, fils de Zébédée, et Jean, son frère, dans une barque, avec Zébédée leur père, qui réparaient leurs filets, et il les appela.
\VS{22}Et ayant aussitôt quitté leur barque et leur père, ils le suivirent.
\TextTitle{Ministère de Jésus-Christ en Galilée}
\VS{23}Jésus allait par toute la Galilée, enseignant dans leurs synagogues, prêchant l'Evangile du Royaume, et guérissant toutes sortes de maladies, et toutes sortes d'infirmités parmi le peuple.
\VS{24}Et sa renommée se répandit par toute la Syrie ; et on lui présentait tous ceux qui se portaient mal, tourmentés de diverses maladies, des démoniaques, des lunatiques, des paralytiques ; et il les guérissait.
\VS{25}Une grande foule le suivit, de Galilée, de la Décapole, de Jérusalem, de Judée et au-delà du Jourdain.
\Chap{5}
\TextTitle{L'enseignement de Jésus sur la montagne\FTNTT{Lu. 6:20-49 ; Mt. 5:1-48}}
\VerseOne{}Voyant la foule, Jésus monta sur la montagne ; puis s'étant assis, ses disciples s'approchèrent de lui.
\VS{2}Puis, ayant ouvert la bouche, il les enseigna de la sorte :
\VS{3}Heureux les pauvres en esprit, car le Royaume des cieux est à eux.
\VS{4}Heureux ceux qui pleurent, car ils seront consolés.
\VS{5}Heureux les humbles, car ils hériteront la terre.
\VS{6}Heureux ceux qui ont faim et soif de la justice, car ils seront rassasiés.
\VS{7}Heureux les miséricordieux, car ils obtiendront miséricorde.
\VS{8}Heureux sont ceux qui sont purs de cœur, car ils verront Dieu.
\VS{9}Heureux ceux qui procurent la paix, car ils seront appelés enfants de Dieu.
\VS{10}Heureux ceux qui sont persécutés pour la justice, car le Royaume des cieux est à eux.
\VS{11}Heureux serez-vous lorsqu'on vous outragera, qu'on vous persécutera et qu'on dira faussement de vous toute sorte de mal à cause de moi.
\VS{12}Réjouissez-vous et soyez dans l'allégresse, parce que votre récompense sera grande dans les cieux ; car c'est ainsi qu'on a persécuté les prophètes qui ont été avant vous.
\TextTitle{Le sel de la terre et la lumière du monde\FTNTT{Mc. 4:21-23 ; Lu. 8:16-18 ; 11:33-36}}
\VS{13}Vous êtes le sel de la terre mais si le sel perd sa saveur, avec quoi le salera-t-on ? Il ne sert plus qu'à être jeté dehors, et foulé aux pieds par les hommes.
\VS{14}Vous êtes la lumière du monde. Une ville située sur une montagne ne peut être cachée,
\VS{15}et on n'allume point la lampe pour la mettre sous un boisseau, mais sur un chandelier et elle éclaire tous ceux qui sont dans la maison.
\VS{16}Ainsi, que votre lumière luise devant les hommes, afin qu'ils voient vos bonnes œuvres et qu'ils glorifient votre Père qui est dans les cieux.
\TextTitle{Le Messie et la loi}
\VS{17}Ne croyez pas que je sois venu abolir la loi ou les prophètes ; je ne suis pas venu les abolir, mais les accomplir.
\VS{18}Car, je vous le dis en vérité, tant que le ciel et la terre ne passeront point, il ne disparaîtra pas de la loi un seul iota ou un seul trait de lettre jusqu'à ce que tout soit arrivé.
\VS{19}Celui donc qui aura violé l'un de ces petits commandements, et qui aura enseigné les hommes à faire de même, sera appelé le plus petit au Royaume des cieux ; mais celui qui les observera et qui enseignera à les observer, celui-là sera appelé grand au Royaume des cieux.
\VS{20}Car je vous dis que si votre justice ne surpasse celle des scribes et des pharisiens, vous n'entrerez point dans le Royaume des cieux.
\VS{21}Vous avez entendu qu'il a été dit aux anciens : Tu ne tueras point, et celui qui tuera, sera puni par les juges.
\VS{22}Mais moi je vous dis que quiconque se met en colère sans cause contre son frère sera puni par les juges ; et celui qui dira à son frère : Raca\FTNT{Raca : Expression de mépris utilisée parmis les Juifs au temps de Jésus signifiant vide, indigne ou encore vaurien.} ! sera puni par le conseil ; et celui qui lui dira : Insensé ! sera puni par le feu de la géhenne\FTNT{La géhenne ou le lac de feu : Voir commentaire en Ap. 20:14.}.
\VS{23}Si donc tu apportes ton offrande à l'autel, et que là tu te souviennes que ton frère a quelque chose contre toi,
\VS{24}laisse là ton offrande devant l'autel et va te réconcilier d'abord avec ton frère, puis viens et offre ton offrande.
\VS{25}Accorde-toi rapidement avec ta partie adverse, tandis que tu es en chemin avec elle ; de peur que ta partie adverse ne te livre au juge, et que le juge ne te livre à l'officier de justice, et que tu ne sois mis en prison.
\VS{26}En vérité, je te dis, tu ne sortiras point de là, jusqu'à ce que tu n'aies payé le dernier quart de sou.
\TextTitle{Convoitise, adultère et divorce\FTNTT{Mt. 19:3-11 ; Mc. 10:2-12 ; 1 Co. 7:1-16}}
\VS{27}Vous avez entendu qu'il a été dit aux anciens : Tu ne commettras point d'adultère.
\VS{28}Mais moi, je vous dis que quiconque regarde une femme pour la convoiter a déjà commis dans son cœur un adultère avec elle.
\VS{29}Si ton œil droit est pour toi une occasion de chute, arrache-le et jette-le loin de toi ; car il est avantageux pour toi qu'un seul de tes membres périsse et que ton corps entier ne soit pas jeté dans la géhenne.
\VS{30}Si ta main droite est pour toi une occasion de chute, coupe-la et jette-la loin de toi ; car il est avantageux pour toi qu'un seul de tes membres périsse et que ton corps entier ne soit pas jeté dans la géhenne.
\VS{31}Il a été dit encore : Si quelqu'un répudie sa femme, qu'il lui donne une lettre de divorce.
\VS{32}Mais moi, je vous dis que celui qui répudie sa femme, si ce n'est pour cause d'adultère, l'expose à devenir adultère ; et que celui qui épouse une femme répudiée commet un adultère.
\TextTitle{Acquittement des promesses faites au Seigneur ; attitude face à son prochain}
\VS{33}Vous avez aussi appris qu'il a été dit aux anciens : Tu ne te parjureras point, mais tu rendras au Seigneur ce que tu auras promis par serment.
\VS{34}Mais moi, je vous dis de ne jurer aucunement, ni par le ciel, parce que c'est le trône de Dieu ;
\VS{35}ni par la terre, parce que c'est le marchepied de ses pieds, ni par Jérusalem, parce que c'est la ville du grand Roi.
\VS{36}Ne jure pas non plus par ta tête, car tu ne peux pas rendre blanc ou noir un seul cheveu.
\VS{37}Mais que votre parole soit : Oui, oui ; non, non ; car ce qui est de plus, vient du malin.
\VS{38}Vous avez appris qu'il a été dit : Œil pour œil et dent pour dent.
\VS{39}Mais moi, je vous dis : Ne résistez point au méchant. Si quelqu'un te frappe sur ta joue droite, présente-lui aussi l'autre.
\VS{40}Si quelqu'un veut plaider contre toi, et prendre ta tunique, laisse-lui encore ton manteau.
\VS{41}Si quelqu'un te force à faire un mille, fais-en deux avec lui.
\VS{42}Donne à celui qui te demande et ne te détourne point de celui qui veut emprunter de toi.
\TextTitle{Le standard de l'Amour\FTNTT{Lu. 6:27-36}}
\VS{43}Vous avez appris qu'il a été dit : Tu aimeras ton prochain et tu haïras ton ennemi.
\VS{44}Mais moi, je vous dis : Aimez vos ennemis, bénissez ceux qui vous maudissent, faites du bien à ceux qui vous haïssent, et priez pour ceux qui vous maltraitent et vous persécutent,
\VS{45}afin que vous soyez les fils de votre Père qui est dans les cieux ; car il fait lever son soleil sur les méchants et sur les gens de bien, et il envoie sa pluie sur les justes et sur les injustes.
\VS{46}Car si vous aimez seulement ceux qui vous aiment, quelle récompense en aurez-vous ? Les publicains aussi n'en font-ils pas tout autant ?
\VS{47}Et si vous faites accueil seulement à vos frères, que faites-vous de plus que les autres ? Les publicains aussi ne le font-ils pas de même ?
\VS{48}Soyez donc parfaits, comme votre Père qui est dans les cieux est parfait.
\Chap{6}
\TextTitle{Jésus condamne l'hypocrisie}
\VerseOne{}Gardez-vous de pratiquer votre justice devant les hommes, pour en être vus ; autrement, vous ne recevrez point la récompense de votre Père qui est dans les cieux.
\VS{2}Donc, lorsque tu fais ton aumône, ne fais point sonner la trompette devant toi, comme font les hypocrites dans les synagogues et dans les rues, afin d'être glorifiés par les hommes. Je vous le dis en vérité, ils reçoivent leur récompense.
\VS{3}Mais quand tu fais ton aumône, que ta main gauche ne sache pas ce que fait ta droite ;
\VS{4}afin que ton aumône se fasse en secret, et ton Père, qui voit ce qui se fait dans le secret, te récompensera publiquement.
\VS{5}Et quand tu pries, ne sois point comme les hypocrites ; car ils aiment à prier en se tenant debout dans les synagogues et aux coins des rues, pour être vus des hommes. Je vous le dis en vérité, ils reçoivent leur récompense.
\VS{6}Mais toi, quand tu pries, entre dans ta chambre, et ayant fermé ta porte, prie ton Père, qui est là dans ce lieu secret ; et ton Père qui te voit dans ce lieu secret, te récompensera publiquement.
\VS{7}Quand vous priez, ne multipliez pas de vaines paroles, comme font les Gentils ; car ils s'imaginent qu'à force de paroles ils seront exaucés.
\VS{8}Ne leur ressemblez donc point ; car votre Père sait de quoi vous avez besoin, avant que vous le lui demandiez.
\TextTitle{Instructions de Jésus sur la prière}
\VS{9}Voici donc comment vous devez prier : Notre Père qui es aux cieux, que ton Nom soit sanctifié.
\VS{10}Que ton règne vienne. Que ta volonté soit faite sur la terre comme au ciel.
\VS{11}Donne-nous aujourd'hui notre pain quotidien.
\VS{12}Et remets nous nos dettes\FTNT{Du grec « opheilema » : « ce qui est légalement dû, une dette ». Les Ecritures considèrent le péché comme une dette. Voir Mat. 18:21-35.}, comme nous aussi nous remettons les dettes à nos débiteurs.
\VS{13}Ne nous induis pas en tentation ; mais délivre-nous du mal. Car c'est à toi qu'appartiennent, dans tous les siècles, le règne, et la puissance et la gloire. Amen !
\VS{14}Car si vous pardonnez aux hommes leurs offenses, votre Père céleste vous pardonnera aussi les vôtres.
\VS{15}Mais si vous ne pardonnez point aux hommes leurs offenses, votre Père ne vous pardonnera point non plus vos offenses.
\TextTitle{Attitude pendant le jeûne}
\VS{16}Et quand vous jeûnerez, ne prenez pas un air triste, comme font les hypocrites ; car ils se rendent le visage tout défait, afin de montrer aux hommes qu'ils jeûnent. Je vous le dis en vérité, ils reçoivent leur récompense.
\VS{17}Mais toi, quand tu jeûnes, oint ta tête et lave ton visage,
\VS{18}afin qu'il ne paraisse pas aux hommes que tu jeûnes, mais à ton Père qui est présent dans ton lieu secret ; et ton Père qui te voit dans ton lieu secret, te récompensera publiquement.
\TextTitle{Le trésor selon Dieu}
\VS{19}Ne vous amassez point des trésors sur la terre, où les vers et la rouille détruisent, et où les voleurs percent et dérobent.
\VS{20}Mais amassez-vous des trésors dans le ciel, où les vers et la rouille ne détruisent point, et où les voleurs ne percent ni ne dérobent.
\VS{21}Car là où est ton trésor, là aussi sera ton cœur.
\VS{22}L'œil est la lampe du corps. Si donc ton œil est en bon état, tout ton corps sera éclairé.
\VS{23}Mais si ton œil est mal disposé, tout ton corps sera ténébreux. Si donc la lumière qui est en toi n'est que ténèbres, combien seront grandes les ténèbres même ?
\VS{24}Nul ne peut servir deux maîtres. Car, ou il haïra l'un et aimera l'autre ; ou il s'attachera à l'un et méprisera l'autre. Vous ne pouvez pas servir Dieu et Mammon\FTNT{Mammon : Mot d'origine araméenne signifiant « riche ». Certains le rapprochent de l'hébreu « matmon » signifiant « trésor, argent ». D'autres le rapprochent du phénicien « mommon » signifiant « bénéfice ». Dans les évangiles, il signifie « possession » (matérielle), mais il est parfois personnifié.}.
\TextTitle{Rechercher le Royaume}
\VS{25}C'est pourquoi je vous dis : Ne vous inquiétez pas pour votre vie, de ce que vous mangerez, et de ce que vous boirez ; ni pour votre corps, de quoi vous serez vêtus. La vie n'est-elle pas plus que la nourriture et le corps plus que le vêtement ?
\VS{26}Considérez les oiseaux du ciel ; car ils ne sèment, ni ne moissonnent, ni n'assemblent dans des greniers, et cependant votre Père céleste les nourrit. N'êtes vous pas beaucoup plus excellents qu'eux ?
\VS{27}Et qui est celui d'entre vous qui puisse, par ses inquiétudes, ajouter une coudée à sa taille ?
\VS{28}Et pourquoi vous inquiéter au sujet du vêtement ? Apprenez comment croissent les lis des champs : Ils ne travaillent ni ne filent ;
\VS{29}cependant je vous dis que Salomon même, dans toute sa gloire, n'a pas été vêtu comme l'un d'eux.
\VS{30}Si donc Dieu revêt ainsi l'herbe des champs, qui est aujourd'hui sur pied, et qui demain sera jetée au four, ne vous vêtira-t-il pas à plus forte raison, ô gens de petite foi ?
\VS{31}Ne vous inquiétez donc point en disant : Que mangerons-nous ? Ou, que boirons-nous ? Ou de quoi serons-nous vêtus ?
\VS{32}Vu que les païens recherchent toutes ces choses; car votre Père céleste sait que vous avez besoin de toutes ces choses.
\VS{33}Mais cherchez premièrement le Royaume de Dieu et sa justice, et toutes ces choses vous seront données par-dessus.
\VS{34}Ne vous inquiétez donc pas pour le lendemain ; car le lendemain prendra soin de lui-même. A chaque jour suffit sa peine.
\Chap{7}
\TextTitle{Le jugement hypocrite\FTNTT{Lu. 6:37-42}}
\VerseOne{}Ne jugez point afin que vous ne soyez point jugés.
\VS{2}Car de tel jugement que vous jugez, vous serez jugés ; et de telle mesure que vous mesurerez, on vous mesurera réciproquement.
\VS{3}Et pourquoi vois-tu la paille qui est dans l'œil de ton frère, et n'aperçois-tu pas la poutre qui est dans ton œil ?
\VS{4}Ou comment peux-tu dire à ton frère : Permets que j'ôte de ton œil cette paille et n'aperçois-tu pas la poutre dans ton œil ?
\VS{5}Hypocrite, ôte premièrement de ton œil la poutre, et après cela tu verras comment tu ôteras la paille de l'œil de ton frère.
\VS{6}Ne donnez point les choses saintes aux chiens et ne jetez point vos perles devant les pourceaux, de peur qu'ils ne les foulent aux pieds, ne se retournent et ne vous déchirent.
\TextTitle{Exhortation à la prière}
\VS{7}Demandez et il vous sera donné. Cherchez et vous trouverez. Frappez et l'on vous ouvrira.
\VS{8}Car quiconque demande, reçoit ; et celui qui cherche trouve ; et l'on ouvre à celui qui frappe.
\VS{9}Lequel de vous donnera une pierre à son fils, s'il lui demande du pain ?
\VS{10}Ou, s'il lui demande un poisson, lui donnera-t-il un serpent ?
\VS{11}Si donc vous, méchants comme vous l'êtes, savez donner à vos enfants de bonnes choses, à combien plus forte raison votre Père qui est dans les cieux, donnera-t-il des bonnes choses à ceux qui les lui demandent ?
\TextTitle{La règle d'or de la loi et des prophètes\FTNTT{Lu. 6:31; Ep.4:32}}
\VS{12}Tout ce que vous voulez que les hommes fassent pour vous, faites-le de même pour eux, car c'est la loi et les prophètes.
\TextTitle{Les deux chemins\FTNTT{Ps. 1}}
\VS{13}Entrez par la porte étroite, car c'est la porte large et le chemin spacieux qui mènent à la perdition, et il y en a beaucoup qui entrent par elle.
\VS{14}Mais étroite est la porte, resserré le chemin qui mènent à la vie, et il y en a peu qui les trouvent.
\TextTitle{Les faux prophètes, reconnaissables à leurs fruits\FTNTT{Lu. 6:43-45}}
\VS{15}Gardez-vous des faux prophètes, ils viennent à vous en habits de brebis, mais au-dedans ce sont des loups ravisseurs.
\VS{16}Vous les reconnaîtrez à leurs fruits. Cueille-t-on des raisins sur des épines, ou des figues sur des chardons ?
\VS{17}Ainsi tout bon arbre porte de bons fruits ; mais le mauvais arbre porte de mauvais fruits.
\VS{18}Un bon arbre ne peut porter de mauvais fruits ni le mauvais arbre porter de bons fruits.
\VS{19}Tout arbre qui ne porte pas de bons fruits est coupé et jeté au feu.
\VS{20}Vous les reconnaîtrez donc à leurs fruits.
\TextTitle{Fausse confession\FTNTT{Lu. 6:46}}
\VS{21}Ceux qui me disent : Seigneur ! Seigneur ! N'entreront pas tous dans le Royaume des cieux ; mais celui qui fait la volonté de mon Père qui est dans les cieux.
\VS{22}Plusieurs me diront en ce jour-là : Seigneur ! Seigneur ! N'avons-nous pas prophétisé en ton Nom ? N'avons-nous pas chassé les démons en ton Nom ? N'avons-nous pas fait beaucoup de miracles en ton Nom ?
\VS{23}Alors je leur dirai ouvertement : Je ne vous ai jamais connus. Retirez-vous de moi, vous qui commettez l'iniquité.
\TextTitle{Parabole des deux bâtisseurs et des deux fondements\FTNTT{Lu. 6:47-49}}
\VS{24}Quiconque entend ces paroles que je dis, et les met en pratique, je le comparerai à un homme prudent qui a bâti sa maison sur le roc.
\VS{25}La pluie est tombée, les torrents sont venus, les vents ont soufflé contre cette maison : Elle n'est point tombée parce qu'elle était fondée sur le roc\FTNT{Jésus-Christ le Rocher : voir Es. 8:13-17.}.
\VS{26}Mais quiconque entend ces paroles que je dis et ne les met point en pratique, sera semblable à un homme insensé qui a bâti sa maison sur le sable.
\VS{27}La pluie est tombée, les torrents sont venus, les vents ont soufflé contre cette maison : Elle est tombée et sa ruine a été grande.
\TextTitle{Effet de l'enseignement}
\VS{28}Or il arriva que quand Jésus eut achevé ce discours, la foule fut frappée de sa doctrine ;
\VS{29}car il les enseignait comme ayant de l'autorité et non comme les scribes.
\Chap{8}
\TextTitle{Le lépreux guérit\FTNTT{Mc. 1:40-45}}
\VerseOne{}Et quand il fut descendu de la montagne, de grandes foules le suivirent.
\VS{2}Et voici, un lépreux vint et se prosterna devant lui, en lui disant : Seigneur\FTNT{Seigneur : Du grec « kurios ». C'est la première fois que ce terme est appliqué à Jésus. Notez que c'est un lépreux qui a eu la révélation que Jésus-Christ est YHWH.}, si tu veux, tu peux me rendre pur.
\VS{3}Et Jésus étendit la main, le toucha, en disant : Je le veux, sois pur. A l'instant même il fut purifié de sa lèpre.
\VS{4}Puis Jésus lui dit : Prends garde de ne le dire à personne ; mais va te montrer au sacrificateur et offre l'offrande que Moïse a prescrite afin que cela leur serve de témoignage.
\TextTitle{Guérison du serviteur d'un centenier\FTNTT{Lu. 7:1-10}}
\VS{5}Et quand Jésus fut entré dans Capernaüm, un centenier vint à lui, le priant
\VS{6}et disant : Seigneur, mon serviteur qui est paralytique est couché à la maison et il souffre extrêmement.
\VS{7}Jésus lui dit : J'irai et je le guérirai.
\VS{8}Mais le centenier lui répondit : Seigneur, je ne suis pas digne que tu entres sous mon toit ; mais dis seulement une parole et mon serviteur sera guéri.
\VS{9}Car moi-même qui suis un homme soumis à l'autorité d'un autre, j'ai des soldats sous mes ordres, et je dis à l'un : Va ! et il va ; et à un autre : Viens ! et il vient ; et à mon serviteur : Fais cela ! et il le fait.
\VS{10}Après l'avoir entendu, Jésus fut étonné et dit à ceux qui le suivaient : Je vous le dis en vérité, même en Israël je n'ai pas trouvé une aussi grande foi.
\VS{11}Or, je vous dis que plusieurs viendront de l'orient et de l'occident, et seront à table dans le Royaume des cieux, avec Abraham, Isaac et Jacob.
\VS{12}Et les enfants du Royaume seront jetés dans les ténèbres du dehors, où il y aura des pleurs et des grincements de dents.
\VS{13}Alors Jésus dit au centenier : Va, et qu'il te soit fait selon ta foi. Et à l'heure même, son serviteur fut guéri.
\TextTitle{Guérison de la belle-mère de Pierre\FTNTT{Mc. 1:29-34 ; Lu. 4:38-41}}
\VS{14}Puis Jésus alla à la maison de Pierre, dont il vit la belle-mère couchée et ayant la fièvre.
\VS{15}Il toucha sa main, et la fièvre la quitta ; puis elle se leva, et les servit.
\VS{16}Et le soir étant venu, on lui amena plusieurs démoniaques. Et il chassa par sa parole les esprits malins, et guérit tous ceux qui étaient malades,
\VS{17}afin que s'accomplisse ce qui avait été annoncé par Esaïe le prophète, en disant : Il a pris nos faiblesses et a porté nos maladies\FTNT{Es. 53:4.}.
\TextTitle{Les disciples éprouvés dans leur consécration\FTNTT{Lu. 9:57-62}}
\VS{18}Or Jésus voyant autour de lui de grandes foules, donna l'ordre de passer à l'autre rive.
\VS{19}Et un scribe s'approchant, lui dit : Maître, je te suivrai partout où tu iras.
\VS{20}Jésus lui dit : Les renards ont des tanières, et les oiseaux du ciel ont des nids ; mais le Fils de l'homme n'a pas de place pour reposer sa tête.
\VS{21}Puis un autre de ses disciples lui dit : Seigneur, permets-moi d'aller d'abord ensevelir mon père.
\VS{22}Et Jésus lui dit : Suis-moi et laisse les morts ensevelir leurs morts.
\TextTitle{Autorité de Jésus face à la tempête\FTNTT{Mc. 4:35-41 ; Lu. 8:22-25}}
\VS{23}Il monta dans la barque et ses disciples le suivirent.
\VS{24}Et voici, il s'éleva sur la mer une si grande tempête que la barque était couverte de flots ; et Jésus dormait.
\VS{25}Et ses disciples vinrent le réveiller en lui disant : Seigneur, sauve-nous, nous périssons !
\VS{26}Et il leur dit : Pourquoi avez-vous peur, gens de peu de foi ? Alors s'étant levé, il menaça les vents et la mer, et il se fit un grand calme.
\VS{27}Et les gens qui étaient là furent étonnés, et dirent : Qui est celui-ci à qui obéissent même les vents et la mer ?
\TextTitle{Deux aveugles et un démoniaque guéris\FTNTT{Mc. 5:1-20 ; Lu. 8:26-40}}
\VS{28}Et quand il fut passé de l'autre côté, dans le pays des Gadaréniens, deux démoniaques sortant des sépulcres, vinrent le rencontrer. Ils étaient si dangereux que personne ne pouvait passer par ce chemin-là.
\VS{29}Et voici, ils s'écrièrent : Qu'y a-t-il entre nous et toi, Jésus Fils de Dieu ? Es-tu venu ici nous tourmenter avant le temps ?
\VS{30}Et il y avait loin d'eux un grand troupeau de pourceaux qui paissaient.
\VS{31}Et les démons le priaient en disant : Si tu nous chasses dehors, permets-nous d'entrer dans ce troupeau de pourceaux.
\VS{32}Et il leur dit : Allez ! Et ils sortirent et entrèrent dans le troupeau de pourceaux. Et voici, tout le troupeau de pourceaux se précipita des pentes escarpées dans la mer et ils périrent dans les eaux.
\VS{33}Ceux qui les gardaient s'enfuirent et allèrent dans la ville, ils racontèrent toutes ces choses et ce qui était arrivé aux démoniaques.
\VS{34}Et voici, toute la ville alla à la rencontre de Jésus, et l'ayant vu, ils le prièrent de se retirer de leur pays.
\Chap{9}
\TextTitle{Un paralytique guéri\FTNTT{Mc. 2:3-12 ; Lu. 5:18-26}}
\VerseOne{}Alors, étant monté dans une barque, il traversa la mer et vint dans sa ville.
\VS{2}Et voici, on lui présenta un paralytique couché sur un lit. Et Jésus voyant leur foi, dit au paralytique : Prends courage, mon enfant ! Tes péchés te sont pardonnés.
\VS{3}Et voici, quelques-uns des scribes disaient au dedans d'eux : Cet homme blasphème.
\VS{4}Mais Jésus, connaissant leurs pensées, leur dit : Pourquoi avez-vous de mauvaises pensées dans vos cœurs ?
\VS{5}Car lequel est le plus aisé de dire : Tes péchés te sont pardonnés ; ou de dire : Lève-toi et marche ?
\VS{6}Or afin que vous sachiez que le Fils de l'homme a le pouvoir sur la terre de pardonner les péchés : Lève-toi, dit-il au paralytique, prends ton lit et va dans ta maison.
\VS{7}Et il se leva et s'en alla dans sa maison.
\VS{8}Quand la foule vit cela, elle fut saisie d'étonnement, et elle glorifiait Dieu qui a donné aux hommes un tel pouvoir.
\TextTitle{L'appel de Matthieu\FTNTT{Mc. 2:14 ; Lu. 5:27-28}}
\VS{9}De là, étant allé plus loin, Jésus vit un homme nommé Matthieu, assis au bureau du péage et il lui dit : Suis-moi ; et il se leva et le suivit.
\TextTitle{L'appel des pécheurs\FTNTT{Mc. 2:15-20 ; Lu. 5:29-35}}
\VS{10}Comme Jésus était à table dans la maison de Matthieu, beaucoup de publicains et des gens de mauvaise vie, qui étaient venus là, se mirent à table avec Jésus et avec ses disciples.
\VS{11}Les pharisiens virent cela et ils dirent à ses disciples : Pourquoi votre Maître mange-t-il avec les publicains et les gens de mauvaise vie ?
\VS{12}Jésus l'ayant entendu, leur dit : Ce ne sont pas ceux qui sont en bonne santé qui ont besoin de médecin, mais les malades.
\VS{13}Mais allez et apprenez ce que veulent dire ces paroles : Je prends plaisir à la miséricorde et non aux sacrifices\FTNT{Os. 6:6.}. Car je ne suis pas venu appeler à la repentance les justes, mais les pécheurs.
\VS{14}Alors les disciples de Jean vinrent auprès de lui et lui dirent : Pourquoi nous et les pharisiens jeûnons-nous souvent, tandis que tes disciples ne jeûnent point ?
\VS{15}Et Jésus leur répondit : Les amis de l'époux peuvent-ils s'affliger pendant que l'époux est avec eux ? Mais les jours viendront où l'époux leur sera enlevé, alors ils jeûneront.
\TextTitle{Parabole du drap neuf et des outres neuves\FTNTT{Mc. 2:21-22 ; Lu. 5:36-39}}
\VS{16}Aussi personne ne met une pièce de drap neuf à un vieil habit car la pièce emporterait une partie de l'habit et la déchirure serait pire.
\VS{17}On ne met pas non plus du vin nouveau dans de vieilles outres ; autrement les outres se rompent, et le vin se répand, et les outres sont perdues ; mais on met le vin nouveau dans des outres neuves, et l'un et l'autre se conservent.
\TextTitle{Résurrection de la fille de Jaïrus et guérison de la femme à la perte de sang\FTNTT{Mc. 5:21-43 ; Lu. 8:41-56}}
\VS{18}Tandis qu'il leur disait ces choses, voici, arriva un chef qui se prosterna devant lui, en lui disant : Ma fille est morte il y a un instant, mais viens, et impose-lui ta main et elle vivra.
\VS{19}Et Jésus s'étant levé le suivit avec ses disciples.
\VS{20}Et voici, une femme atteinte d'une perte de sang depuis douze ans s'approcha par-derrière et toucha le bord de son vêtement.
\VS{21}Car elle disait en elle-même : Si je puis seulement toucher son vêtement, je serai guérie.
\VS{22}Et Jésus se retourna, et dit en la voyant : Prends courage, ma fille ! Ta foi t'a sauvée. Et cette femme fut guérie à l'heure même.
\VS{23}Lorsque Jésus fut arrivé à la maison du chef et qu'il vit les joueurs de flûte et une foule bruyante,
\VS{24}il leur dit : Retirez-vous car la jeune fille n'est pas morte, mais elle dort ; et ils se moquaient de lui.
\VS{25}Quand la foule eut été renvoyée, il entra, prit la main de la jeune fille et elle se leva.
\VS{26}Et le bruit s'en répandit dans toute la contrée.
\TextTitle{Deux aveugles et un démoniaque guéris}
\VS{27}Etant parti de là, Jésus fut suivi par deux aveugles qui criaient : Fils de David, aie pitié de nous !
\VS{28}Et quand il fut arrivé dans la maison, les aveugles s'approchèrent de lui et Jésus leur dit : Croyez-vous que je puisse faire ce que vous me demandez ? Ils lui répondirent : Oui, Seigneur !
\VS{29}Alors il toucha leurs yeux en disant : Qu'il vous soit fait selon votre foi.
\VS{30}Et leurs yeux s'ouvrirent. Alors Jésus leur dit sévèrement : Prenez garde que personne ne le sache.
\VS{31}Mais, dès qu'ils furent sortis, ils répandirent sa renommée dans tout le pays.
\VS{32}Comme ils s'en allaient, voici, on présenta à Jésus un homme muet et démoniaque.
\VS{33}Et le démon ayant été chassé, le muet parla ; et les foules étonnées disaient : Jamais pareille chose ne s'est vue en Israël.
\VS{34}Mais les pharisiens disaient : Il chasse les démons par le prince des démons.
\VS{35}Jésus allait dans toutes les villes et les villages, enseignant dans leurs synagogues, et prêchant l'Evangile du Royaume, et guérissant toutes sortes de maladies et toutes sortes d'infirmités parmi le peuple.
\TextTitle{Jésus ému de compassion pour la foule\FTNTT{Mc. 6:34}}
\VS{36}Et voyant les foules, il fut ému de compassion, parce qu'elles étaient dispersées et errantes comme des brebis qui n'ont point de pasteur.
\VS{37}Et il dit à ses disciples : La moisson est grande, mais il y a peu d'ouvriers.
\VS{38}Priez donc le Maître de la moisson d'envoyer des ouvriers dans sa moisson.
\Chap{10}
\TextTitle{Appel et mission des douze apôtres\FTNTT{Mc. 6:7-13 ; Lu. 9:1-6}}
\VerseOne{}Alors Jésus ayant appelé ses douze disciples, leur donna le pouvoir de chasser les esprits impurs et de guérir toutes sortes de maladies et toutes sortes d'infirmités.
\VS{2}Et voici les noms des douze apôtres : Le premier est Simon, nommé Pierre, et André son frère ; Jacques, fils de Zébédée, et Jean, son frère ;
\VS{3}Philippe et Barthélemy ; Thomas, et Matthieu le péager ; Jacques, fils d'Alphée, et Lebbée, surnommé Thaddée.
\VS{4}Simon le Cananite, et Judas Iscariot, celui qui le livra.
\VS{5}Tels sont les douze que Jésus envoya, et leur donna ses ordres en disant : N'allez point vers les Gentils et n'entrez point dans aucune ville des Samaritains ;
\VS{6}Mais allez plutôt vers les brebis perdues de la maison d'Israël.
\VS{7}Et quand vous serez partis, prêchez, en disant : Le Royaume des cieux est proche.
\VS{8}Guérissez les malades, rendez purs les lépreux, ressuscitez les morts, chassez les démons hors des possédés. Vous l'avez reçu gratuitement, donnez-le gratuitement\FTNT{Vous avez reçu gratuitement : Aucun chrétien, quel que soit son appel ou son don ne peut prétendre qu'il a payé pour avoir les talents qu'il a reçus du Seigneur. Dans 1 Co. 4:7 Paul nous pose une question : « Qu'as-tu que tu n'aies reçu et si tu l'as reçu pourquoi te glorifies-tu ? » Dieu interroge également Job : « De qui suis-je le débiteur ? » (Job 41:2). Vendre quelque chose qu'on a reçu gratuitement n'est rien d'autre que du vol. Donnez gratuitement : C'est la suite logique des choses, on reçoit gratuitement et on donne gratuitement. Si nous aimons Dieu, nous devons garder sa Parole et marcher comme lui a marché (Jn. 14:15 ; 1 Jn. 2:6). Il a donné ses enseignements et nourri les gens gratuitement. Dans Ap. 21:6 et 22:17, le Seigneur invite toutes les personnes qui ont soif à venir s'abreuver gratuitement. Alors pourquoi vendre la Parole qu'on a reçue gratuitement ? Le Seigneur a envoyé les douze en mission et leur a demandé d'apporter l'évangile du Royaume, de guérir les malades et de délivrer les possédés gratuitement (Ac. 8:18-24 ; Ac. 20:33-35 ; Ap. 21:6 ; Ap. 22:17).}.
\VS{9}Ne prenez ni or, ni argent, ni monnaie dans vos ceintures ;
\VS{10}ni de sac pour le voyage, ni deux tuniques, ni souliers, ni bâton ; car l'ouvrier mérite sa nourriture.
\VS{11}Et dans quelque ville ou village que vous entriez, informez-vous qui y est digne de vous loger ; et demeurez chez lui jusqu'à ce que vous partiez de là.
\VS{12}Et quand vous entrerez dans quelque maison, saluez-la.
\VS{13}Et si cette maison en est digne, que votre paix vienne sur elle ; mais si elle n'en est pas digne, que votre paix retourne à vous.
\VS{14}Mais lorsque quelqu'un ne vous recevra point et n'écoutera point vos paroles, secouez, en partant de cette maison ou de cette ville, la poussière de vos pieds.
\VS{15}Je vous dis en vérité que ceux du pays de Sodome et de Gomorrhe seront traités moins rigoureusement au jour du jugement que cette ville-là.
\TextTitle{La proclamation du Royaume avant le retour du Messie}
\VS{16}Voici, je vous envoie comme des brebis au milieu des loups ; soyez donc prudents comme des serpents et simples comme des colombes.
\VS{17}Et mettez-vous en garde contre les hommes ; car ils vous livreront aux tribunaux et vous battront de verges dans leurs synagogues.
\VS{18}Et vous serez menés devant des gouverneurs et même devant des rois, à cause de moi, pour rendre témoignage de moi devant eux et aux nations.
\VS{19}Mais, quand ils vous livreront, ne vous inquiétez pas de ce que vous aurez à dire, ni comment vous parlerez. Ce que vous aurez à dire vous sera donné à l'heure même.
\VS{20}Car ce n'est pas vous qui parlez, mais c'est l'Esprit de votre Père qui parlera en vous.
\VS{21}Le frère livrera son frère à la mort, et le père son enfant ; et les enfants s'élèveront contre leurs pères et leurs mères, et les feront mourir.
\VS{22}Et vous serez haïs de tous à cause de mon Nom ; mais celui qui persévérera jusqu'à la fin sera sauvé.
\VS{23}Quand ils vous persécuteront dans une ville, fuyez dans une autre. Je vous le dis en vérité, vous n'aurez pas achevé de parcourir toutes les villes d'Israël, que le Fils de l'homme sera venu.
\TextTitle{La consécration du disciple et sa récompense}
\VS{24}Le disciple n'est point au-dessus du maître, ni le serviteur au-dessus de son seigneur.
\VS{25}Il suffit au disciple d'être traité comme son maître, et au serviteur comme son seigneur. S'ils ont appelé le père de famille Béelzébul, à combien plus forte raison appelleront-ils ainsi ses domestiques ?
\VS{26}Ne les craignez donc point. Car il n'y a rien de caché qui ne doive être découvert, ni rien de secret qui ne doive être connu.
\VS{27}Ce que je vous dis dans les ténèbres, dites-le dans la lumière ; et ce que je vous dis à l'oreille, prêchez-le sur les toits.
\VS{28}Et ne craignez point ceux qui tuent le corps et qui ne peuvent tuer l'âme ; mais craignez plutôt celui qui peut faire périr et l'âme et le corps en les jetant dans la géhenne.
\VS{29}Ne vend-on pas deux passereaux pour un sou ? Cependant, il n'en tombe pas un à terre sans la volonté de votre Père.
\VS{30}Et même les cheveux de votre tête sont tous comptés.
\VS{31}Ne craignez donc point : Vous valez plus que beaucoup de passereaux.
\VS{32}Quiconque donc me confessera devant les hommes, je le confesserai aussi devant mon Père qui est aux cieux.
\VS{33}Mais quiconque me reniera devant les hommes, je le renierai aussi devant mon Père qui est dans les cieux.
\VS{34}Ne croyez pas que je sois venu apporter la paix sur la terre. Je ne suis pas venu apporter la paix, mais l'épée.
\VS{35}Car je suis venu mettre en division le fils contre son père, et la fille contre sa mère, et la belle-fille contre sa belle-mère.
\VS{36}Et les propres domestiques d'un homme seront ses ennemis.
\VS{37}Celui qui aime son père ou sa mère plus que moi, n'est pas digne de moi ; et celui qui aime son fils ou sa fille plus que moi, n'est pas digne de moi.
\VS{38}Et quiconque ne prend pas sa croix et ne vient pas après moi, n'est pas digne de moi.
\VS{39}Celui qui aura conservé sa vie la perdra ; mais celui qui aura perdu sa vie pour l'amour de moi la retrouvera.
\VS{40}Celui qui vous reçoit me reçoit, et celui qui me reçoit, reçoit celui qui m'a envoyé.
\VS{41}Celui qui reçoit un prophète en qualité de prophète, recevra la récompense d'un prophète ; et celui qui reçoit un juste en qualité de juste recevra la récompense d'un juste.
\VS{42}Et quiconque aura donné à boire seulement un verre d'eau froide à l'un de ces petits parce qu'il est mon disciple, je vous le dis en vérité qu'il ne perdra point sa récompense.
\Chap{11}
\TextTitle{Jean-Baptiste le plus grand des hommes\FTNTT{Lu. 7:19-35}}
\VerseOne{}Et il arriva que quand Jésus eut achevé de donner ses ordres à ses douze disciples, il partit de là pour aller enseigner et prêcher dans leurs villes.
\VS{2}Jean, ayant entendu parler dans sa prison des œuvres du Christ, envoya deux de ses disciples pour lui dire :
\VS{3}Es-tu celui qui devait venir, ou devons-nous en attendre un autre ?
\VS{4}Et Jésus leur répondit : Allez, et rapportez à Jean les choses que vous entendez et que vous voyez.
\VS{5}Les aveugles recouvrent la vue, les boiteux marchent, les lépreux sont purifiés, les sourds entendent, les morts sont ressuscités, et l'Evangile est annoncé aux pauvres\FTNT{Jésus-Christ est le Dieu véritable dont la venue était annoncée par Esaïe (Es. 35:4-6).}.
\VS{6}Mais, heureux est celui qui n'aura point été scandalisé en moi ;
celui pour qui je ne serai pas une occasion de chute !
\VS{7}Et comme ils s'en allaient, Jésus se mit à dire à la foule au sujet de Jean : Mais qu'êtes-vous allés voir dans le désert ? Un roseau agité par le vent ?
\VS{8}Mais qu'êtes-vous allés voir ? Un homme vêtu de précieux vêtements ? Voici, ceux qui portent des habits précieux sont dans les maisons des rois.
\VS{9}Mais qu'êtes-vous allés voir ? Un prophète ? Oui, vous dis-je, et plus qu'un prophète.
\VS{10}Car c'est celui dont il est écrit : Voici, j'envoie mon messager\FTNT{Mal. 3:1.} devant ta face, pour préparer ton chemin devant toi.
\VS{11}En vérité, je vous le dis, parmi ceux qui sont nés de femmes, il n'en a point paru de plus grand que Jean-Baptiste. Toutefois, le plus petit dans le Royaume des cieux, est plus grand que lui.
\VS{12}Or depuis le temps de Jean-Baptiste jusqu'à maintenant, le Royaume des cieux est forcé et ce sont les violents qui s'en emparent.
\VS{13}Car tous les prophètes et la loi ont prophétisé jusqu'à Jean.
\VS{14}Et si vous voulez recevoir mes paroles, c'est lui qui est l'Elie\FTNT{Mal. 4:5-6.} qui devait venir.
\VS{15}Que celui qui a des oreilles pour entendre, entende.
\VS{16}Mais à qui comparerai-je cette génération ? Elle est semblable aux petits-enfants qui sont assis sur les places publiques, et qui crient à leurs compagnons
\VS{17}et leur disent : Nous vous avons joué de la flûte et vous n'avez point dansé ; nous vous avons chanté des complaintes et vous ne vous êtes point lamentés.
\VS{18}Car Jean est venu ne mangeant ni ne buvant et ils disent : Il a un démon.
\VS{19}Le Fils de l'homme est venu mangeant et buvant et ils disent : C'est un mangeur et un buveur, un ami des publicains et des gens de mauvaise vie. Mais la sagesse a été justifiée par ses enfants.
\TextTitle{Jésus dénonce les indifférents}
\VS{20}Alors il se mit à faire des reproches aux villes où il avait fait beaucoup de miracles, parce qu'elles ne s'étaient point repenties.
\VS{21}Malheur à toi, Chorazin ! Malheur à toi, Bethsaïda ! Car si les miracles qui ont été faits au milieu de vous, avaient été faits dans Tyr et dans Sidon, il y a longtemps qu'elles se seraient repenties, en prenant le sac et la cendre.
\VS{22}C'est pourquoi je vous dis que Tyr et Sidon seront traitées moins rigoureusement que vous, au jour du jugement.
\VS{23}Et toi Capernaüm, qui as été élevée jusqu'au ciel, tu seras précipitée jusqu'à Hadès\FTNT{Voir commentaire Mt. 16:18} ; car si les miracles qui ont été faits au milieu de toi, avaient été faits dans Sodome, elle subsisterait encore.
\VS{24}C'est pourquoi je vous dis que ceux de Sodome seront traités moins rigoureusement que toi, au jour du jugement.
\TextTitle{La relation personnelle du disciple avec son Seigneur}
\VS{25}En ce temps-là, Jésus prenant la parole dit : Je te loue, ô mon Père ! Seigneur du ciel et de la terre, de ce que tu as caché ces choses aux sages et aux intelligents, et que tu les as révélées aux petits enfants.
\VS{26}Oui, Père, je te loue parce que telle a été ta bonne volonté.
\VS{27}Toutes choses m'ont été données par mon Père ! Et personne ne connaît le Fils si ce n'est le Père ; et personne ne connaît le Père si ce n'est le Fils, et celui à qui le Fils veut le révéler.
\VS{28}Venez à moi vous tous qui êtes fatigués et chargés, et je vous donnerai du repos.
\VS{29}Prenez mon joug sur vous et recevez mes instructions, car je suis doux et humble de cœur ; et vous trouverez le repos pour vos âmes.
\VS{30}Car mon joug est doux et mon fardeau est léger.
\Chap{12}
\TextTitle{Jésus, le Maître du sabbat\FTNTT{Mc. 2:23-28 ; Lu. 6:1-5}}
\VerseOne{}En ce temps-là, Jésus traversa des champs de blé un jour de sabbat. Et ses disciples qui avaient faim se mirent à arracher des épis et à les manger.
\VS{2}Les pharisiens voyant cela, lui dirent : Voici, tes disciples font ce qu'il n'est pas permis de faire le jour du sabbat.
\VS{3}Mais il leur dit : N'avez-vous pas lu ce que fit David quand il eut faim, lui et ceux qui étaient avec lui ?
\VS{4}Comment il entra dans la maison de Dieu, et mangea les pains de proposition, qu'il ne lui était pas permis de manger ni à lui, ni à ceux qui étaient avec lui, mais aux sacrificateurs seulement ?
\VS{5}Ou n'avez-vous pas lu dans la loi, qu'aux jours du sabbat, les sacrificateurs violent le sabbat dans le temple, sans se rendre coupables ?
\VS{6}Or, je vous le dis, qu'il y a ici quelqu'un de plus grand que le temple.
\VS{7}Si vous saviez ce que signifient ces paroles : Je veux la miséricorde, et non pas le sacrifice, vous n'auriez pas condamné ceux qui ne sont pas coupables\FTNT{1 S. 15:22 ; Os. 6:6.}.
\VS{8}Car le Fils de l'homme est Maître même du sabbat.
\TextTitle{Jésus guérit l'homme à la main sèche le jour du sabbat\FTNTT{Mc. 3:1-5 ; Lu. 6:6-11}}
\VS{9}Puis étant parti de là, il entra dans leur synagogue.
\VS{10}Et voici, il s'y trouvait un homme qui avait la main sèche. Et pour avoir sujet de l'accuser, ils l'interrogèrent en disant : Est-il permis de guérir les jours du sabbat ?
\VS{11}Et il répondit : Lequel d'entre vous s'il n'a qu'une brebis, et qu'elle vienne à tomber dans une fosse le jour du sabbat, ne la saisira-t-il pas pour l'en retirer ?
\VS{12}Combien un homme ne vaut-il pas plus qu'une brebis ! Il est donc permis de faire du bien les jours du sabbat.
\VS{13}Alors il dit à cet homme : Etends ta main. Il l'étendit et elle devint saine comme l'autre.
\TextTitle{Jésus accomplit de nombreuses guérisons}
\VS{14}Les pharisiens sortirent et ils se consultèrent sur les moyens de le faire périr.
\VS{15}Mais Jésus, l'ayant su, partit de là, et de grandes foules le suivirent. Il les guérit tous.
\VS{16}Et il leur défendit avec menaces de le faire connaître,
\VS{17}afin que s'accomplît ce qui avait été annoncé par Esaïe le prophète, en disant :
\VS{18}Voici mon serviteur que j'ai élu, mon bien-aimé, qui est l'objet de mon amour, je mettrai mon Esprit en lui et il annoncera le jugement aux nations.
\VS{19}Il ne contestera point, il ne criera point et personne n'entendra sa voix dans les rues.
\VS{20}Il ne brisera point le roseau cassé et n'éteindra point le lumignon qui fume, jusqu'à ce qu'il ait fait triompher la justice.
\VS{21}Et les nations espéreront en son nom\FTNT{Es. 42:1-4.}.
\VS{22}Alors on lui amena un homme tourmenté d'un démon, aveugle et muet, et il le guérit ; de sorte que celui qui avait été aveugle et muet, parlait et voyait.
\VS{23}Et toutes les foules en furent étonnées, et elles disaient : Celui-ci n'est-il pas le Fils de David ?
\TextTitle{Le blasphème contre le Saint-Esprit\FTNTT{Mc. 3:22-30 ; Lu. 11:15-23}}
\VS{24}Mais les pharisiens ayant entendu cela, disaient : Celui-ci ne chasse les démons que par Béelzébul, prince des démons.
\VS{25}Mais Jésus connaissant leurs pensées, leur dit : Tout royaume divisé contre lui-même sera réduit en désert ; et toute ville, ou maison, divisée contre elle-même ne subsistera point.
\VS{26}Or si Satan chasse Satan, il est divisé contre lui-même ; comment donc son royaume subsistera-t-il ?
\VS{27}Et si je chasse les démons par Béelzébul, par qui vos fils les chassent-ils ? C'est pourquoi ils seront eux-mêmes vos juges.
\VS{28}Mais si je chasse les démons par l'Esprit de Dieu, certes le Royaume de Dieu est donc venu jusqu'à vous.
\VS{29}Ou, comment quelqu'un peut-il entrer dans la maison d'un homme fort et piller ses biens, sans avoir auparavant lié cet homme fort ? Alors il pillera sa maison.
\VS{30}Celui qui n'est pas avec moi, est contre moi, et celui qui n'assemble pas avec moi disperse.
\VS{31}C'est pourquoi je vous dis, que tout péché et tout blasphème sera pardonné aux hommes ; mais le blasphème contre l'Esprit ne leur sera point pardonné.
\VS{32}Quiconque parlera contre le Fils de l'homme, il lui sera pardonné ; mais quiconque parlera contre le Saint-Esprit, il ne lui sera pardonné ni dans ce siècle, ni dans le siècle à venir\FTNT{Le blasphème contre le Saint-Esprit : Le blasphème est un outrage, une calomnie à l'encontre de Dieu. En attribuant l'œuvre de Dieu à Satan, les pharisiens ont commis l'impardonnable. Beaucoup de personnes craignent d'avoir commis ce péché par inadvertance, en ayant par exemple un doute sur l'origine d'un miracle. La Parole nous recommande de ne pas ajouter foi à tout esprit, mais d'éprouver les esprits pour savoir s'ils sont de Dieu (1 Jn. 4:1). On ne pèche donc pas lorsqu'on exerce son discernement. De plus, si l'on a commis une erreur de jugement par ignorance, le Seigneur ne nous en tiendra pas rigueur (Ac. 17:30). Le blasphème contre le Saint-Esprit est commis par des personnes qui, bien qu'ayant la connaissance et la capacité de différencier le bien du mal, font preuve de mauvaise foi. Ainsi, les pharisiens avaient constaté les bons fruits portés par Jésus, mais ils ont hypocritement qualifié de mal le bien qu'il faisait (Es. 5:20). Ceux qui blasphèment contre le Saint-Esprit sont loin d'être ignorants. Comme nous l'atteste Hé. 6:4-6, parmi ces gens, certains « ont goûté le don céleste », « ont eu part au Saint-Esprit, et ont goûté la bonne parole de Dieu, et les puissances du siècle à venir ». En choisissant sciemment de pécher, alors qu'ils ont expérimenté la grâce de Dieu, ils retournent à ce qu'ils ont vomi et outragent ainsi le Seigneur (2 Pi. 2:18-22). Leur cœur endurci à l'extrême rejette volontairement la vérité pour s'attacher au mensonge. Constatant leur refus définitif de se repentir, le Saint-Esprit finit par se retirer pour laisser la place à l'esprit d'égarement qui les maintiendra dans l'erreur (2 Th. 2:9-12). Enfin, il est à noter qu'en Ap. 14:9-11, ceux qui ont reçu la marque de la bête sont condamnés d'office. Il ne faut nullement conclure que le Seigneur leur a refusé son pardon, mais plutôt que les personnes ayant reçu cette marque ont aussi blasphémé contre le Saint-Esprit. Voir commentaire en Ap. 13:16.}.
\TextTitle{Toute parole proclamée appelle un jugement}
\VS{33}Ou dites que l'arbre est bon et son fruit est bon ; ou dites que l'arbre est mauvais et son fruit est mauvais ; car on connaît l'arbre par le fruit.
\VS{34}Race de vipères, comment pourriez-vous dire de bonnes choses, méchants comme vous l'êtes ? Car c'est de l'abondance du cœur que la bouche parle.
\VS{35}L'homme de bien\FTNT{Le mot « bien » dans ce passage vient du grec « Agathos » qui signifie : « de bonne constitution ou nature », « utile », « salutaire », « bon », « agréable », plaisant », « joyeux », « heureux », « excellent », « distingué », « droit », « honorable ».} tire de bonnes choses du bon trésor de son cœur ; et l'homme méchant tire de mauvaises choses du mauvais trésor de son cœur.
\VS{36}Je vous le dis : Les hommes rendront compte au jour du jugement, de toute parole vaine qu'ils auront proférée.
\VS{37}Car tu seras justifié par tes paroles et tu seras condamné par tes paroles.
\TextTitle{Le miracle du prophète Jonas\FTNTT{Jon. 2:1 ; Lu. 11:29-32.}}
\VS{38}Alors quelques-uns des scribes et des pharisiens lui dirent : Maître, nous voudrions bien te voir faire quelque miracle.
\VS{39}Mais il leur répondit et dit : Une génération méchante et adultère demande un miracle, mais il ne lui sera point donné d'autre miracle que celui de Jonas le prophète.
\VS{40}Car, de même que Jonas fut trois jours et trois nuits dans le ventre d'un grand poisson, de même le Fils de l'homme sera trois jours et trois nuits dans le sein de la terre.
\VS{41}Les Ninivites se lèveront au jour du jugement contre cette génération et la condamneront, parce qu'ils se repentirent à la prédication de Jonas ; et voici, il y a ici plus que Jonas.
\TextTitle{La condamnation de cette génération par la reine de Séba\FTNTT{2 Ch. 9:1-12}}
\VS{42}La reine du Midi se lèvera au jour du jugement contre cette nation et la condamnera, parce qu'elle vint des extrémités de la terre pour entendre la sagesse de Salomon ; et voici, il y a ici plus que Salomon.
\TextTitle{Le retour de l'esprit impur\FTNTT{Lu. 11:24-26}}
\VS{43}Lorsque l'esprit impur est sorti d'un homme, il va par des lieux arides, cherchant du repos, mais il n'en trouve point.
\VS{44}Alors il dit : Je retournerai dans ma maison, d'où je suis sorti ; et quand il arrive, il la trouve vide, balayée et ornée.
\VS{45}Puis il s'en va et prend avec lui sept autres esprits plus méchants que lui ; qui y étant entrés, habitent là ; et ainsi la dernière condition de cet homme est pire que la première. Il en sera de même pour cette génération perverse.
\TextTitle{La famille spirituelle\FTNTT{Mc. 3:31-35 ; Lu. 8:19-21}}
\VS{46}Et comme il parlait encore aux foules, voici, sa mère et ses frères se tenaient dehors, cherchant à lui parler.
\VS{47}Et quelqu'un lui dit : Voici, ta mère et tes frères sont là dehors, qui cherchent à te parler.
\VS{48}Mais il répondit à celui qui lui avait dit cela : Qui est ma mère et qui sont mes frères ?
\VS{49}Et étendant sa main sur ses disciples, il dit : Voici ma mère et mes frères.
\VS{50}Car, quiconque fera la volonté de mon Père qui est dans les cieux, celui-là est mon frère, et ma sœur, et ma mère.
\Chap{13}
\TextTitle{1. Parabole des quatre terrains\FTNTT{Mc. 4:1-20 ; Lu. 8:4-15}}
\VerseOne{}Ce même jour, Jésus sortit de la maison et s'assit au bord de la mer.
\VS{2}Une grande foule s'assembla auprès de lui, c'est pourquoi il monta dans une barque et il s'assit. Aussi, toute la foule se tenait sur le rivage.
\VS{3}Et il leur parla en paraboles sur beaucoup de choses et il dit : Un semeur sortit pour semer.
\VS{4}Et comme il semait, une partie de la semence tomba le long du chemin, et les oiseaux vinrent, et la mangèrent toute.
\VS{5}Et une autre partie tomba dans les endroits pierreux où elle n'avait pas beaucoup de terre : Elle leva aussitôt parce qu'elle n'entrait pas profondément dans la terre ;
\VS{6}mais, quand le soleil parut, elle fut brûlée et sécha parce qu'elle n'avait point de racines.
\VS{7}Une autre partie tomba parmi les épines ; et les épines montèrent et l'étouffèrent.
\VS{8}Une autre partie tomba dans la bonne terre : Et elle donna du fruit, un grain en donna cent, un autre, soixante, et un autre, trente.
\VS{9}Que celui qui a des oreilles pour entendre, qu'il entende.
\TextTitle{Explication aux disciples}
\VS{10}Alors les disciples s'approchèrent et lui dirent : Pourquoi leur parles-tu en paraboles ?
\VS{11}Il leur répondit et dit : Parce qu'il vous a été donné de connaître les mystères du Royaume des cieux, et que cela ne leur a pas été donné de les connaître.
\VS{12}Car on donnera à celui qui a, et il sera dans l'abondance, mais à celui qui n'a pas, on ôtera même ce qu'il a.
\VS{13}C'est pourquoi je leur parle en paraboles, parce qu'en voyant, ils ne voient point, et qu'en entendant, ils n'entendent point et ne comprennent point.
\VS{14}Et ainsi s'accomplit pour eux la prophétie d'Esaïe qui dit : Vous entendrez de vos oreilles et vous ne comprendrez point ; et vous regarderez des yeux, et vous ne verrez point.
\VS{15}Car le cœur de ce peuple est engraissé, et ils ont endurci leurs oreilles, et ils ont fermé leurs yeux de peur qu'ils ne voient de leurs yeux, qu'ils n'entendent de leurs oreilles, qu'ils ne comprennent de leur cœur, qu'ils ne se convertissent et que je ne les guérisse\FTNT{Es. 6:9-10.}.
\VS{16}Mais heureux sont vos yeux, car ils voient ; et vos oreilles, parce qu'elles entendent.
\VS{17}Je vous le dis en vérité, beaucoup de prophètes et de justes ont désiré voir les choses que vous voyez, et ils ne les ont point vues, entendre les choses que vous entendez, et ils ne les ont point entendues.
\VS{18}Vous donc, écoutez la signification de la parabole du semeur.
\VS{19}Lorsqu'un homme écoute la parole du Royaume et ne la comprend pas, le malin vient et ravit ce qui est semé dans son cœur ; cet homme est celui qui a reçu la semence le long du chemin.
\VS{20}Et celui qui a reçu la semence dans les endroits pierreux, c'est celui qui entend la parole et la reçoit aussitôt avec joie ;
\VS{21}mais il n'a point de racine en lui-même, il croit pour un temps, et dès que survient une tribulation ou une persécution à cause de la parole, il y trouve une occasion de chute.
\VS{22}Et celui qui a reçu la semence parmi les épines, c'est celui qui entend la parole de Dieu, mais en qui les soucis du siècle et la séduction des richesses étouffent la parole et la rendent infructueuse.
\VS{23}Mais celui qui a reçu la semence dans la bonne terre, c'est celui qui entend la parole et la comprend. Il porte du fruit, et un grain donne cent, un autre soixante, et un autre trente.
\TextTitle{2. Parabole du blé et de l'ivraie}
\VS{24}Il leur proposa une autre parabole et il dit\FTNT{La parabole du blé et de l'ivraie. En méditant cette parabole, nous remarquons que lorsque le blé eut poussé et donné du fruit, l'ivraie parut aussi. Il est vrai que lorsqu'il y a un réveil spirituel divin dans une assemblée ou dans un pays, l'ennemi suscite aussi un faux réveil avec des faux ouvriers et des fausses manifestations spirituelles. Voilà pourquoi l'ivraie côtoiera le blé jusqu'à la fin du monde. Le mot « ivraie » se dit « ebriacus » en latin, ce qui donne « ébriété » en français. Nous comprenons donc que l'un des rôles de l'ivraie est d'enivrer le blé (les enfants de Dieu). Dans les Ecritures, l'ivresse est synonyme de la débauche spirituelle ou physique. En grec l'ivraie se dit « zizanion » qui donne en français « zizanie ». Voir Mt. 12:25. La division est l'œuvre de l'ivraie dans les églises qui cherche à créer des sectes et des partis pris.} : Le Royaume des cieux est semblable à un homme qui a semé de la bonne semence dans son champ.
\VS{25}Mais, pendant que les hommes dormaient, son ennemi vint, sema de l'ivraie parmi le blé, puis s'en alla.
\VS{26}Lorsque l'herbe eut poussé et donné du fruit, l'ivraie parut aussi.
\VS{27}Et les serviteurs du maître de la maison vinrent à lui et lui dirent : Seigneur, n'as-tu pas semé de la bonne semence dans ton champ ? D'où vient donc qu'il y a de l'ivraie ?
\VS{28}Mais il leur répondit : C'est un ennemi qui a fait cela. Et les serviteurs lui dirent : Veux-tu donc que nous allions l'arracher ?
\VS{29}Et il leur dit : Non, de peur qu'en arrachant l'ivraie, vous ne déraciniez le blé en même temps.
\VS{30}Laissez-les croître tous deux ensemble, jusqu'à la moisson ; et au temps de la moisson, je dirai aux moissonneurs : Arrachez premièrement l'ivraie, et liez-la en gerbes pour la brûler, mais amassez le blé dans mon grenier.
\TextTitle{3. Parabole du grain de sénevé\FTNTT{Mc. 4:30-32 ; Lu. 13:18-19}}
\VS{31}Il leur proposa une autre parabole et il dit : Le Royaume des cieux est semblable au grain de sénevé qu'un homme a pris et semé dans son champ.
\VS{32}C'est la plus petite de toutes les semences ; mais, quand il a poussé, il est plus grand que les autres plantes et devient un arbre, de sorte que les oiseaux du ciel viennent habiter et font leurs nids dans ses branches.
\TextTitle{4. Parabole du levain\FTNTT{Lu. 13:20-21}}
\VS{33}Il leur dit une autre parabole : Le Royaume des cieux est semblable à du levain qu'une femme a pris et mis dans trois mesures de farine, jusqu'à ce que toute la pâte soit levée.
\VS{34}Jésus dit à la foule toutes ces choses en paraboles, et il ne lui parlait point sans paraboles,
\VS{35}afin que s'accomplisse ce qui avait été annoncé par le prophète : J'ouvrirai ma bouche en paraboles, je déclarerai les choses qui ont été cachées dès la fondation du monde\FTNT{Ps. 78:2.}.
\TextTitle{Explication de la parabole du blé et de l'ivraie}
\VS{36}Alors Jésus renvoya la foule et entra dans la maison, et ses disciples s'approchèrent de lui et lui dirent : Explique-nous la parabole de l'ivraie du champ.
\VS{37}Et il leur répondit et dit : Celui qui sème la bonne semence, c'est le Fils de l'homme ;
\VS{38}le champ, c'est le monde ; la bonne semence ce sont les fils du Royaume, et l'ivraie ce sont les fils du malin ;
\VS{39}et l'ennemi qui l'a semée, c'est le diable ; la moisson, c'est la fin du monde, et les moissonneurs sont les anges.
\VS{40}Or, comme on arrache l'ivraie et qu'on la brûle au feu, il en sera de même à la fin de ce monde.
\VS{41}Le Fils de l'homme enverra ses anges qui arracheront de son Royaume tous les scandales et ceux qui commettent l'iniquité,
\VS{42}et les jetteront dans la fournaise ardente, où il y aura des pleurs et des grincements de dents.
\VS{43}Alors les justes resplendiront comme le soleil dans le Royaume de leur Père. Que celui qui a des oreilles pour entendre, qu'il entende.
\TextTitle{5. Parabole du trésor caché}
\VS{44}Le Royaume des cieux est encore semblable à un trésor caché dans un champ. L'homme qui l'a trouvé, le cache ; puis dans sa joie, il va vendre tout ce qu'il a, et achète ce champ.
\TextTitle{6. Parabole de la perle}
\VS{45}Le Royaume des cieux est encore semblable à un marchand qui cherche de bonnes perles.
\VS{46}Il a trouvé une perle de grand prix et il est allé vendre tout ce qu'il avait, et l'a achetée.
\TextTitle{7. Parabole du filet}
\VS{47}Le Royaume des cieux est encore semblable à un filet jeté dans la mer et ramassant toutes sortes de choses.
\VS{48}Quand il est rempli, les pêcheurs le tirent en haut sur le rivage, puis s'étant assis, ils mettent ce qu'il y a de bon à part dans leurs vases et jettent dehors ce qui est mauvais.
\VS{49}Il en sera de même à la fin du monde, les anges viendront séparer les méchants d'avec les justes,
\VS{50}et les jetteront dans la fournaise ardente, où il y aura des pleurs et des grincements de dents.
\VS{51}Jésus leur dit : Avez-vous compris toutes ces choses ? Ils lui répondirent : Oui, Seigneur.
\TextTitle{8. Le maître de la maison}
\VS{52}Et il leur dit : C'est pourquoi, tout scribe instruit de ce qui regarde le Royaume des cieux, est semblable à un père de famille qui tire de son trésor des choses nouvelles et des choses anciennes.
\TextTitle{Jésus à Nazareth\FTNTT{Mc. 6:1-6}}
\VS{53}Et quand Jésus eut achevé ces paraboles, il partit de là.
\VS{54}Et s'étant rendu dans sa patrie, il enseignait dans la synagogue, de telle sorte que ceux qui l'entendirent étaient étonnés et disaient : D'où lui viennent cette sagesse et ces miracles ?
\VS{55}Celui-ci n'est-il pas le fils du charpentier ? Sa mère ne s'appelle-t-elle pas Marie ? Et ses frères ne s'appellent-ils pas Jacques, Joseph, Simon et Jude ?
\VS{56}Et ses sœurs ne sont-elles pas toutes parmi nous ? D'où lui viennent donc toutes ces choses ?
\VS{57}Et il était pour eux une occasion de chute. Mais Jésus leur dit : Un prophète n'est méprisé que dans sa patrie et dans sa maison.
\VS{58}Et il ne fit là que peu de miracles, à cause de leur incrédulité.
\Chap{14}
\TextTitle{Mort de Jean-Baptiste\FTNTT{Mc. 6:14-29; Lu. 9:7-9}}
\VerseOne{}En ce temps-là, Hérode le tétrarque entendit parler de la renommée de Jésus, et il dit à ses serviteurs : C'est Jean-Baptiste !
\VS{2}Il est ressuscité des morts, c'est pourquoi la puissance de faire des miracles agit puissamment en lui.
\VS{3}Car Hérode avait fait arrêter Jean, et l'avait fait lier et mettre en prison, à cause d'Hérodias, femme de Philippe son frère.
\VS{4}Parce que Jean lui disait : Il ne t'est pas permis de l'avoir pour femme.
\VS{5}Et il voulait le faire mourir, mais il craignait la foule, parce qu'elle regardait Jean comme un prophète.
\VS{6}Or, le jour où l'on célébra la naissance d'Hérode, la fille d'Hérodias dansa au milieu de l'assemblée et plut à Hérode.
\VS{7}C'est pourquoi il lui promit avec serment de lui donner tout ce qu'elle demanderait.
\VS{8}A l'instigation de sa mère, elle dit : Donne-moi ici, sur un plat, la tête de Jean-Baptiste.
\VS{9}Le roi fut attristé ; mais à cause de ses serments et de ceux qui étaient à table avec lui, il commanda qu'on la lui donne.
\VS{10}Et il envoya décapiter Jean dans la prison.
\VS{11}Et sa tête fut apportée sur un plat et donnée à la fille qui la présenta à sa mère.
\VS{12}Puis ses disciples vinrent, et emportèrent son corps, et l'ensevelirent. Et ils allèrent l'annoncer à Jésus.
\VS{13}Et Jésus, ayant appris ce qu'Hérode avait fait, partit de là dans une barque, pour se retirer à l'écart dans un lieu désert ; et la foule l'ayant appris, sortit des villes voisines et le suivit à pied.
\VS{14}Et Jésus étant sorti, vit une grande foule, et il fut ému de compassion pour elle, et guérit les malades.
\TextTitle{Multiplication des pains pour les cinq mille hommes\FTNTT{Mc. 6:32-44 ; Lu. 9:12-17 ; Jn. 6:1-14}}
\VS{15}Et comme il se faisait tard, ses disciples vinrent à lui et lui dirent : Ce lieu est désert et l'heure est déjà avancée. Renvoie la foule, afin qu'elle aille dans les villages, pour s'acheter des vivres.
\VS{16}Mais Jésus leur dit : Ils n'ont pas besoin de s'en aller ; donnez-leur vous-mêmes à manger.
\VS{17}Et ils lui dirent : Nous n'avons ici que cinq pains et deux poissons.
\VS{18}Et il leur dit : Apportez-les-moi ici.
\VS{19}Et après avoir ordonné à la foule de s'asseoir sur l'herbe, il prit les cinq pains et les deux poissons, et levant les yeux au ciel, il rendit grâces à Dieu. Puis ayant rompu les pains, il les donna aux disciples qui les distribuèrent à la foule.
\VS{20}Tous en mangèrent et furent rassasiés, et l'on emporta douze paniers pleins des morceaux qui restaient.
\VS{21}Ceux qui avaient mangé étaient environ cinq mille hommes, sans compter les femmes et les petits enfants.
\TextTitle{Jésus marche sur les eaux, incrédulité de Pierre\FTNTT{Mc. 6:45-56 ; Jn. 6:15-21}}
\VS{22}Aussitôt après, Jésus obligea ses disciples à monter dans la barque et à passer avant lui de l'autre côté, pendant qu'il renverrait la foule.
\VS{23}Et quand il l'eut renvoyée, il monta sur une montagne pour être à part, afin de prier ; et le soir étant venu, il était là seul.
\VS{24}La barque, déjà au milieu de la mer, était battue par les flots ; car le vent était contraire.
\VS{25}Et vers la quatrième veille de la nuit, Jésus alla vers eux, marchant sur la mer.
\VS{26}Et ses disciples le voyant marcher sur la mer, ils furent troublés et ils dirent : C'est un fantôme ! Et, dans leur frayeur, ils poussèrent des cris.
\VS{27}Jésus leur dit aussitôt : Rassurez-vous, c'est moi, n'ayez pas de peur !
\VS{28}Et Pierre lui répondit : Seigneur, si c'est toi, ordonne que j'aille vers toi sur les eaux.
\VS{29}Et il lui dit : Viens ! Pierre sortit de la barque, marcha sur les eaux pour aller vers Jésus.
\VS{30}Mais voyant que le vent était fort, il eut peur ; et comme il commençait à enfoncer, il s'écria : Seigneur ! Sauve-moi !
\VS{31}Et aussitôt Jésus étendit sa main et le prit en lui disant : Homme de peu de foi, pourquoi as-tu douté ?
\VS{32}Et quand ils furent montés dans la barque, le vent s'apaisa.
\VS{33}Alors ceux qui étaient dans la barque, vinrent adorer Jésus et dirent : Certes, tu es le Fils de Dieu.
\TextTitle{Jésus guérit des malades à Génésareth\FTNTT{Mc. 6:53-56}}
\VS{34}Après avoir traversé la mer, ils vinrent dans le pays de Génézareth.
\VS{35}Les gens de ce lieu ayant reconnu Jésus, envoyèrent des messagers dans tous les environs et on lui amena tous les malades.
\VS{36}Et ils le prièrent de leur permettre de toucher seulement le bord de son vêtement. Et tous ceux qui le touchèrent furent guéris.
\Chap{15}
\TextTitle{Jésus-Christ condamne les traditions\FTNTT{Mc. 7:1-13}}
\VerseOne{}Alors des scribes et des pharisiens vinrent de Jérusalem auprès de Jésus et lui dirent :
\VS{2}Pourquoi tes disciples transgressent-ils la tradition des anciens ? Car ils ne se lavent point les mains quand ils prennent leur repas.
\VS{3}Il leur répondit : Et vous, pourquoi transgressez-vous le commandement de Dieu par votre tradition ?
\VS{4}Car Dieu a dit : Honore ton père et ta mère. Et il a dit aussi : Celui qui maudira son père ou sa mère finira à la mort.
\VS{5}Mais vous, vous dites : Celui qui dira à son père ou à sa mère : Tout ce dont j'aurais pu t'assister est une offrande à Dieu, n'est pas coupable, quoiqu'il n'honore pas son père ou sa mère.
\VS{6}Vous annulez ainsi le commandement de Dieu par votre tradition.
\VS{7}Hypocrites, Esaïe a bien prophétisé de vous, en disant :
\VS{8}Ce peuple s'approche de moi de sa bouche et m'honore des lèvres ; mais son cœur est très éloigné de moi.
\VS{9} Mais ils m'honorent en vain, en enseignant des doctrines qui ne sont que des commandements d'hommes\FTNT{Es. 29:13.}.
\TextTitle{Verdict sur le coeur humain\FTNTT{Mc. 7:14-23}}
\VS{10}Puis ayant appelé à lui la foule, il lui dit : Ecoutez, et comprenez ceci :
\VS{11}Ce n'est pas ce qui entre dans la bouche qui souille l'homme ; mais ce qui sort de la bouche c'est ce qui souille l'homme.
\VS{12}Sur cela les disciples s'approchant, lui dirent : Sais-tu que les pharisiens ont été scandalisés quand ils ont entendus ce discours ?
\VS{13}Et il répondit et dit : Toute plante que mon Père céleste n'a pas plantée sera déracinée.
\VS{14}Laissez-les, ce sont des aveugles, conducteurs d'aveugles ; si un aveugle conduit un autre aveugle, ils tomberont tous deux dans la fosse.
\VS{15}Alors Pierre prenant la parole, lui dit : Explique-nous cette parabole.
\VS{16}Et Jésus dit : Vous aussi, êtes-vous encore sans intelligence ?
\VS{17}Ne comprenez-vous pas encore que tout ce qui entre dans la bouche va dans le ventre, puis est jeté dans les lieux secrets ?
\VS{18}Mais les choses qui sortent de la bouche partent du cœur, et ces choses-là souillent l'homme.
\VS{19}Car c'est du cœur que sortent les mauvaises pensées, les meurtres, les adultères, les fornications, les vols, les faux témoignages, les médisances.
\VS{20}Ce sont ces choses-là qui souillent l'homme ; mais de manger sans avoir les mains lavées, cela ne souille point l'homme.
\TextTitle{Jésus et la femme cananéenne\FTNTT{Mc. 7:24-30}}
\VS{21}Alors Jésus, partant de là se retira dans le territoire de Tyr et de Sidon.
\VS{22}Et voici, une femme cananéenne, qui venait de ces contrées, lui cria : Seigneur ! Fils de David, aie pitié de moi ! Ma fille est cruellement tourmentée par le démon.
\VS{23}Mais il ne lui répondit pas un mot. Et ses disciples s'approchèrent et lui dirent : Renvoie-la, car elle crie derrière nous.
\VS{24}Et il répondit : Je n'ai été envoyé qu'aux brebis perdues de la maison d'Israël.
\VS{25}Mais elle vint et l'adora, disant : Seigneur, assiste-moi !
\VS{26}Et il lui répondit en disant : Il ne convient pas de prendre le pain des enfants et de le jeter aux petits chiens.
\VS{27}Mais elle dit : Cela est vrai, Seigneur ! Cependant les petits chiens mangent des miettes qui tombent de la table de leurs maîtres.
\VS{28}Alors Jésus répondant, lui dit : Ô femme ! Ta foi est grande. Qu'il te soit fait comme tu le souhaites. Et, à l'heure même, sa fille fut guérie.
\TextTitle{Nouvelles guérisons}
\VS{29}Et Jésus quitta ces lieux et vint près de la mer de Galilée. Puis il monta sur une montagne et s'y assit là.
\VS{30}Et une grande foule vint à lui, ayant avec elle des boiteux, des aveugles, des muets, des estropiés et beaucoup d'autres malades. On les mit aux pieds de Jésus et il les guérit ;
\VS{31}de sorte que la foule était dans l'admiration de voir que les muets parlaient, que les estropiés étaient guéris, que les boiteux marchaient, que les aveugles voyaient ; et elle glorifiait le Dieu d'Israël.
\TextTitle{Seconde multiplication des pains\FTNTT{Mc. 8:1-9}}
\VS{32}Alors Jésus, ayant appelé ses disciples, dit : Je suis ému de compassion pour cette foule de gens ; car voilà trois jours qu'ils sont près de moi et ils n'ont rien à manger. Je ne veux pas les renvoyer à jeun, de peur que les forces ne leur manquent en chemin.
\VS{33}Et ses disciples lui dirent : D'où pourrions-nous tirer dans ce désert assez de pains pour rassasier une si grande multitude ?
\VS{34}Et Jésus leur dit : Combien avez-vous de pains ? Ils lui dirent : Sept, et quelque peu de petits poissons.
\VS{35}Alors il commanda aux foules de s'asseoir par terre.
\VS{36}Et ayant prit les sept pains et les poissons, et après avoir béni Dieu, il les rompit et les donna à ses disciples, qui les distribuèrent à la foule.
\VS{37}Et tous mangèrent et furent rassasiés, et l'on emporta sept corbeilles pleines des morceaux qui restaient.
\VS{38}Or, ceux qui avaient mangé étaient quatre mille hommes, sans compter les femmes et les petits enfants.
\VS{39}Et Jésus renvoya la foule, monta sur une barque, et se rendit dans le territoire de Magdala.
\Chap{16}
\TextTitle{La cécité d'une génération méchante et adultère\FTNTT{Mc. 8:10-14}}
\VerseOne{}Alors les pharisiens et les sadducéens vinrent à lui, et pour l'éprouver, lui demandèrent de leur faire voir un signe venant du ciel.
\VS{2}Mais il leur répondit : Quand le soir est venu, vous dites : Il fera beau temps, car le ciel est rouge.
\VS{3}Et le matin vous dites : Il y aura de l'orage aujourd'hui, car le ciel est d'un rouge sombre. Hypocrites, vous savez bien discerner l'aspect du ciel, et vous ne pouvez discerner les signes des temps !
\VS{4}Une génération méchante et adultère demande un miracle ; mais il ne lui sera point donné d'autre miracle que celui de Jonas le prophète. Puis il les quitta et s'en alla.
\VS{5}Et ses disciples, en passant sur l'autre bord, avaient oublié de prendre des pains.
\TextTitle{Le levain des pharisiens et des sadducéens, une doctrine corrompue\FTNTT{Mc. 8:15-21 ; Lu. 12:1-15}}
\VS{6}Et Jésus leur dit : Gardez-vous avec soin du levain des pharisiens et des sadducéens.
\VS{7}Ils résonnaient en eux-mêmes et disaient : C'est parce que nous n'avons pas pris de pains.
\VS{8}Et Jésus connaissant leurs pensées leur dit : Gens de peu de foi, pourquoi raisonnez-vous en vous-mêmes sur le fait que vous n'avez pas pris de pains ?
\VS{9}Ne comprenez-vous point encore, et ne vous rappelez-vous plus les cinq pains des cinq mille hommes et combien de paniers vous avez emportés,
\VS{10}ni des sept pains des quatre mille hommes et combien de corbeilles vous avez emportées ?
\VS{11}Comment ne comprenez-vous pas que ce n'est pas au sujet du pain que je vous ai dit, de vous garder du levain des pharisiens et des sadducéens ?
\VS{12}Alors ils comprirent que ce n'était pas du levain du pain qu'il leur avait dit de se garder, mais de la doctrine des pharisiens et des sadducéens.
\TextTitle{Pierre reconnaît Jésus comme le Messie\FTNTT{Mc. 8:27-30 ; Lu. 9:18-21 ; Jn. 6:66-71}}
\VS{13}Jésus, étant arrivé dans le territoire de Césarée de Philippe, demanda à ses disciples : Qui disent les hommes que je suis, moi le Fils de l'homme ?
\VS{14}Et ils lui répondirent : Les uns disent que tu es Jean-Baptiste ; les autres, Elie ; et les autres, Jérémie, ou l'un des prophètes.
\VS{15}Il leur dit : et vous, qui dites-vous que je suis ?
\VS{16}Simon Pierre répondit et dit : Tu es le Christ, le Fils du Dieu vivant.
\TextTitle{Jésus bâtit son Eglise}
\VS{17}Et Jésus lui répondit et dit : Tu es heureux, Simon, fils de Jonas, car ce ne sont pas la chair et le sang qui t'ont révélé cela, mais mon Père qui est dans les cieux.
\VS{18}Et moi je te dis, que tu es Pierre, et que sur ce Roc\FTNT{Le Roc : Ce passage a été mal traduit dans beaucoup de Bibles comme suit : « Et moi, je te dis que tu es Pierre, et que sur cette pierre je bâtirai mon Eglise… ». Or pour une bonne compréhension des propos de Jésus, il est important d'insister sur la distinction que le grec fait entre « Petros » (pierre, caillou), l'apôtre Pierre, et « Petra » (roc, rocher), qui n'est autre que Jésus-Christ, le rocher des siècles (Es. 17:10 ; Es. 26:4 ; 1 Co. 10:4). De là en découle un enseignement fondamental : l'Eglise n'est bâtie ni par un homme ni sur l'homme, en l'occurrence Pierre et ses supposés successeurs (papes), mais par Jésus-Christ lui-même qui en est la Pierre Angulaire et le fondement inébranlable (Ac. 4:11 ; Ep. 2:20).} je bâtirai mon Eglise ; et les portes de l'enfer\FTNT{Enfer : du grec « Hadès ». Hadès chez les Grecs ou Pluton chez les Romains, était considéré comme le dieu des profondeurs souterraines et le maître des enfers. Ce terme est parfois traduit par « séjour des morts », équivalent hébreu de « Scheol ». Les Grecs utilisaient l'euphémisme Pylartes, signifiant « aux portes solidement closes », pour parler du très craint Hadès. En effet, Juifs, Grecs et Romains avaient conscience que les portes closes de l'enfer ne laissaient personne sortir du royaume de la mort. Tous les impies, et même les croyants d'avant Jésus-Christ, étaient retenus par les portes de l'enfer. Toutefois, les croyants allaient dans une partie de l'enfer que les juifs appelaient « sein d'Abraham » (1 Sam. 28:7-19 ; Lu. 16:22-25 ; Lu. 23:43) où ils ne subissaient pas les tourments infligés aux impies. Lorsque le Seigneur est mort, il est descendu « dans les régions inférieures de la terre » pour prendre les clés du Hadès, clés du séjour des morts (Col. 2:15 ; Ap. 1:17-18) et libérer les captifs pieux. Jésus affirme que les portes de l'enfer ne prévaudront jamais contre son Eglise puisque c'est lui qui l'a bâtie. Malgré tout, Hadès, bien que vaincu par le Seigneur, essaie d'attirer l'Eglise que le Seigneur a établie dans les lieux célestes (Ep. 2:4-9 ; Col. 3:1) vers le royaume des ténèbres, au travers des fausses doctrines et du péché. Au jour du jugement dernier, Hadès et la mort, qui sont deux démons, seront jetés dans l'étang de feu et de soufre (Ap. 20:11-15).} ne prévaudront point contre elle.
\VS{19}Je te donnerai les clefs du Royaume des cieux ; et tout ce que tu lieras sur la terre, sera lié dans les cieux ; et tout ce que tu délieras sur la terre, sera délié dans les cieux\FTNT{Une mauvaise compréhension de ce verset a contribué à propager l'idée erronée selon laquelle Pierre serait le médiateur entre Dieu et les hommes, puisque c'est lui qui détiendrait les clés du Royaume des cieux. Toutefois, Es. 22:22 affirme que seul Jésus-Christ détient ces clés qui symbolisent l'autorité et la domination. Or dans le cadre de l'héritage que le Seigneur nous a laissé, cette autorité est désormais exercée en son Nom par tous les membres du corps de Christ (Mt. 18:18).}.
\VS{20}Alors il commanda expressément à ses disciples de ne dire à personne qu'il était Jésus le Christ.
\TextTitle{Jésus parle de sa mort et de sa résurrection\FTNTT{Mc. 8:31-33 ; Lu. 9:22}}
\VS{21}Dès lors Jésus commença à déclarer à ses disciples qu'il fallait qu'il aille à Jérusalem, qu'il souffre beaucoup de la part des anciens, des principaux sacrificateurs et des scribes, qu'il soit mis à mort, et qu'il ressuscite le troisième jour.
\VS{22}Mais Pierre l'ayant pris à part, se mit à le reprendre en lui disant : Seigneur, aie pitié de toi, cela ne t'arrivera point !
\VS{23}Mais lui, s'étant retourné, dit à Pierre : Arrière de moi, Satan ! Tu m'es en scandale, car tu ne comprends pas les choses qui sont de Dieu, mais celles qui sont des hommes.
\TextTitle{La consécration du disciple\FTNTT{Mc. 8:34-38 ; Lu. 9:23-26}}
\VS{24}Alors Jésus dit à ses disciples : Si quelqu'un veut venir après moi, qu'il renonce à lui-même, et qu'il se charge de sa croix, et qu'il me suive.
\VS{25}Car quiconque voudra sauver son âme, la perdra ; mais quiconque perdra son âme à cause de son amour pour moi, la trouvera.
\VS{26}Et que servirait-il à un homme de gagner tout le monde, s'il perdait son âme ? Ou, que donnerait un homme en échange de son âme ?
\VS{27}Car le Fils de l'homme doit venir dans la gloire de son Père avec ses anges ; et alors il rendra à chacun selon ses œuvres.
\VS{28}Je vous le dis en vérité, quelques-uns de ceux qui sont ici présents, ne mourront point, qu'ils n'aient vu le Fils de l'homme venir dans son règne\FTNT{Ce passage doit être lu de concert avec Mt. 24:32-34. Jésus utilise un langage prophétique pour expliquer deux réalités. La première réalité est spirituelle et concerne ses contemporains qui allaient vivre l'effusion de l'Esprit pour rétablir le Royaume de Dieu dans le cœur des gens. En effet, le Seigneur ne les a pas laissés orphelins, mais il est revenu sous la forme de l'Esprit (Jn. 14:17-18 ; Ac. 2 ; Ac. 16:7). Aussi, les apôtres ont pu proclamer ce Royaume partout où ils allaient (Ac. 20:25). La deuxième réalité est matérielle et concerne le fleurissement du figuier, c'est-à-dire Israël. L'histoire atteste le fleurissement de ce figuier tant sur le plan géographique que sur le plan numérique. Depuis le 14 mai 1948, date de la naissance officielle de l'état hébreu, Israël ne cesse de s'étendre. Cette nation est l'horloge des temps car le Messie gouvernera le monde entier depuis Jérusalem (Mi. 4 ; Za. 14).}.
\Chap{17}
\TextTitle{Transfiguration de Jésus-Christ\FTNTT{Mc. 9:1-8 ; Lu. 9:27-36}}
\VerseOne{}Six jours après, Jésus prit Pierre, Jacques et Jean son frère, et les conduisit à l'écart sur une haute montagne.
\VS{2}Et il fut transfiguré en leur présence et son visage resplendit comme le soleil ; et ses vêtements devinrent blancs comme la lumière.
\VS{3}Et voici, ils virent Moïse et Elie qui s'entretenaient avec lui.
\VS{4}Alors Pierre prenant la parole, dit à Jésus : Seigneur, il est bon que nous soyons ici. Faisons-y, si tu le veux, trois tentes, une pour toi, une pour Moïse, et une pour Elie.
\VS{5}Et comme il parlait encore, voici une nuée resplendissante les couvrit de son ombre. Et voici, une voix fit entendre de la nuée ces paroles : Celui-ci est mon Fils bien-aimé, en qui j'ai pris mon bon plaisir : Ecoutez-le !
\VS{6}Lorsque les disciples entendirent cette voix, ils tombèrent le visage contre terre et furent saisis d'une très grande frayeur.
\VS{7}Mais Jésus, s'approchant, les toucha et leur dit : Levez-vous et n'ayez pas peur.
\VS{8}Ils levèrent les yeux, et ne virent personne, excepté Jésus tout seul.
\VS{9}Et comme ils descendaient de la montagne, Jésus leur donna cet ordre, en disant : Ne parlez à personne de cette vision, jusqu'à ce que le Fils de l'homme soit ressuscité des morts.
\VS{10}Et ses disciples l'interrogèrent, en disant : Pourquoi donc les scribes disent-ils qu'il faut qu'Elie vienne premièrement ?
\VS{11}Et Jésus répondant, leur dit : Il est vrai qu'Elie viendra premièrement et rétablira toutes choses.
\VS{12}Mais je vous dis qu'Elie est déjà venu, et ils ne l'ont pas reconnu et ils lui ont fait tout ce qu'ils ont voulu. De même, le Fils de l'homme doit souffrir aussi de leur part.
\VS{13}Alors les disciples comprirent que c'était de Jean-Baptiste qu'il leur parlait.
\TextTitle{Le manque de foi des disciples\FTNTT{Mc. 9:14-29 ; Lu. 9:37-43}}
\VS{14}Et quand ils furent arrivés près de la foule, un homme s'approcha et se mit à genoux devant lui,
\VS{15}et lui dit : Seigneur ! Aie pitié de mon fils qui est lunatique et misérablement affligé ; car il tombe souvent dans le feu et souvent dans l'eau.
\VS{16}Et je l'ai présenté à tes disciples, mais ils n'ont pas pu le guérir.
\VS{17}Et Jésus répondit et dit : Ô race incrédule et perverse, jusqu'à quand serai-je avec vous ? Jusqu'à quand vous supporterai-je ? Amenez-le-moi ici.
\VS{18}Et Jésus parla sévèrement au démon, qui sortit de lui, et à l'heure même l'enfant fut guéri.
\VS{19}Alors les disciples s'approchèrent de Jésus et lui dirent en particulier : Pourquoi n'avons-nous pas pu le chasser ?
\VS{20}Et Jésus leur répondit : C'est à cause de votre incrédulité. Je vous le dis en vérité, si vous aviez de la foi, comme un grain de sénevé, vous diriez à cette montagne : Transporte-toi d'ici là, et elle se transporterait ; et rien ne vous serait impossible.
\VS{21}Mais cette sorte de démon ne sort que par la prière et par le jeûne.
\TextTitle{Jésus évoque à nouveau sa mort et sa résurrection\FTNTT{Mc. 9:30-32 ; Lu. 9:44-45}}
\VS{22}Et comme ils se trouvaient en Galilée, Jésus leur dit : Il arrivera que le Fils de l'homme sera livré entre les mains des hommes ;
\VS{23}Et qu'ils le feront mourir, mais le troisième jour il ressuscitera. Et les disciples en furent fort attristés.
\TextTitle{La pièce d'argent dans la bouche d'un poisson\FTNTT{Mc. 12:13-17}}
\VS{24}Et lorsqu'ils arrivèrent à Capernaüm, ceux qui percevaient les deux drachmes s'adressèrent à Pierre et lui dirent : Votre Maître ne paye-t-il pas les deux drachmes ?
\VS{25}Oui dit-il. Et quand il fut entré dans la maison, Jésus le prévint en lui disant : Qu'est-ce qu'il t'en semble, Simon ? Les rois de la terre, de qui perçoivent-ils des tributs ou des impôts ? Est-ce de leurs enfants ou des étrangers ?
\VS{26}Pierre dit : Des étrangers. Jésus lui répondit : Les enfants en sont donc exempts.
\VS{27}Mais afin que nous ne les scandalisions point, va-t'en à la mer et jette l'hameçon, et prends le premier poisson qui viendra ; ouvre-lui la bouche, tu trouveras un statère. Prends-le et donne-le-leur pour moi et pour toi.
\Chap{18}
\TextTitle{L'humilité, secret de la vraie grandeur\FTNTT{Mc. 9:33-37; Lu. 9:46-48}}
\VerseOne{}En cette même heure-là, les disciples s'approchèrent de Jésus, en lui disant : Qui est le plus grand dans le Royaume des cieux ?
\VS{2}Et Jésus ayant appelé un petit enfant, le mit au milieu d'eux,
\VS{3}et leur dit : Je vous le dis en vérité, que si vous ne vous convertissez pas et si vous ne devenez pas comme les petits enfants, vous n'entrerez pas dans le Royaume des cieux.
\VS{4}C'est pourquoi quiconque deviendra humble, comme ce petit enfant, celui-là est le plus grand dans le Royaume des cieux.
\VS{5}Et quiconque reçoit en mon Nom un petit enfant comme celui-ci, il me reçoit.
\VS{6}Mais, quiconque scandalise un de ces petits qui croient en moi, il vaudrait mieux pour lui qu'on mette à son cou une meule d'âne, et qu'on le jette au fond de la mer.
\TextTitle{Les scandales et les occasions de chute}
\VS{7}Malheur au monde à cause des scandales ! Car il est nécessaire qu'il arrive des scandales ; mais malheur à l'homme par qui le scandale arrive !
\VS{8}Si ta main ou ton pied est pour toi une occasion de chute, coupe-les et jette-les loin de toi ; car il vaut mieux que tu entres boiteux ou manchot dans la vie, que d'avoir deux pieds ou deux mains, et d'être jeté dans le feu éternel.
\VS{9}Et si ton œil est pour toi une occasion de chute, arrache-le et jette-le loin de toi ; car il vaut mieux que tu entres dans la vie n'ayant qu'un œil, que d'avoir deux yeux, et d'être jeté dans le feu de la géhenne.
\VS{10}Gardez-vous de mépriser un seul de ces petits ; car je vous dis que dans les cieux leurs anges voient continuellement la face de mon Père qui est aux cieux.
\VS{11}Car le Fils de l'homme est venu pour sauver ce qui était perdu.
\TextTitle{Parabole de la brebis égarée\FTNTT{Lu. 15:3-7}}
\VS{12}Que vous en semble ? Si un homme a cent brebis, et que l'une d'elles s'égare, ne laisse-t-il pas les quatre-vingt-dix-neuf autres, pour aller dans les montagnes chercher celle qui s'est égarée ?
\VS{13}Et, s'il arrive qu'il la trouve, je vous le dis en vérité, qu'il en a plus de joie, que les quatre-vingt-dix-neuf qui ne se sont pas égarées.
\VS{14}Ainsi la volonté de votre Père qui est aux cieux n'est pas qu'un seul de ces petits périsse.
\TextTitle{Discipline dans les assemblées}
\VS{15}Que si ton frère a péché contre toi, va, et reprends-le entre toi et lui seul. S'il t'écoute, tu as gagné ton frère.
\VS{16}Mais s'il ne t'écoute pas, prends encore avec toi une ou deux personnes, afin que par la bouche de deux ou trois témoins toute parole soit ferme\FTNT{De. 19:15.}.
\VS{17}S'il refuse de les écouter, dis-le à l'Eglise ; et s'il refuse aussi d'écouter l'Eglise, qu'il soit pour toi comme un païen et comme un publicain.
\VS{18}En vérité je vous dis que tout ce que vous lierez sur la terre, sera lié dans le ciel ; et tout ce que vous délierez sur la terre sera délié dans le ciel\FTNT{Voir commentaire en Mt. 16:19.}.
\VS{19}Je vous dis aussi que si deux d'entre vous s'accordent sur la terre, tout ce qu'ils demanderont leur sera donné par mon Père qui est aux cieux.
\VS{20}Car là où deux ou trois sont assemblés en mon Nom, je suis là au milieu d'eux.
\TextTitle{Ne jamais se lasser de pardonner}
\VS{21}Alors Pierre s'approchant, lui dit : Seigneur, combien de fois mon frère péchera-il contre moi et lui pardonnerai-je ? Sera-ce jusqu'à sept fois ?
\VS{22}Jésus lui répondit : Je ne te dis pas jusqu'à sept fois, mais jusqu'à soixante-dix fois sept fois.
\TextTitle{Parabole du roi et du méchant serviteur}
\VS{23}C'est pourquoi le Royaume des cieux est semblable à un roi qui voulut faire rendre compte à ses serviteurs.
\VS{24}Et quand il se mit à compter, on lui en présenta un qui lui devait dix mille talents.
\VS{25}Et parce qu'il n'avait pas de quoi payer, son maître ordonna qu'il soit vendu, lui, sa femme, ses enfants et tout ce qu'il avait, et que la dette soit payée.
\VS{26}Mais ce serviteur se jetant à ses pieds, le suppliait en disant : Seigneur, aie patience envers moi et je te rendrai le tout.
\VS{27}Alors le maître de ce serviteur, ému de compassion, le relâcha et lui remit la dette.
\VS{28}Mais ce serviteur étant sorti, rencontra un de ses compagnons de service, qui lui devait cent deniers ; et l'ayant pris, il l'étranglait, en lui disant : paye-moi ce que tu me dois.
\VS{29}Mais son compagnon de service se jetant à ses pieds, le suppliait en disant : Aie patience et je te rendrai le tout.
\VS{30}Mais l'autre ne voulut pas et il alla le jeter en prison, jusqu'à ce qu'il ait payé la dette.
\VS{31}Or ses autres compagnons de service voyant ce qui était arrivé, en furent extrêmement attristés et ils allèrent raconter à leur maître tout ce qui s'était passé.
\VS{32}Alors son maître le fit venir et lui dit : Méchant serviteur, je t'avais remis en entier ta dette, parce que tu m'en avais supplié ;
\VS{33}Ne te fallait-il pas aussi avoir pitié de ton compagnon de service, comme j'avais eu pitié de toi ?
\VS{34}Et son maître étant en colère le livra aux bourreaux, jusqu'à ce qu'il lui ait payé tout ce qu'il devait.
\VS{35}C'est ainsi que vous fera mon Père céleste, si vous ne pardonnez de tout votre cœur, chacun à son frère, ses fautes.
\Chap{19}
\TextTitle{Enseignement de Jésus sur le mariage et le divorce\FTNTT{Mt. 5:31-32 ; Mc. 10:2-12 ; Lu. 16:18 ; Ro. 7:1-3 ; 1 Co. 7:10-16}}
\VerseOne{}Et il arriva que quand Jésus eut achevé ces discours, il quitta la Galilée, et alla dans le territoire de la Judée, au-delà du Jourdain.
\VS{2}Et de grandes foules le suivirent, et là il guérit leurs malades.
\VS{3}Alors les pharisiens vinrent à lui pour l'éprouver, et ils lui dirent : Est-il permis à un homme de répudier sa femme pour quelque cause que ce soit ?
\VS{4}Et il répondit et leur dit : N'avez-vous pas lu que le Créateur, au commencement, fit l'homme et la femme ?
\VS{5}Et il dit : A cause de cela, l'homme quittera son père et sa mère, et s'attachera à sa femme, et les deux ne seront qu'une seule chair ?
\VS{6}Ainsi ils ne sont plus deux, mais une seule chair\FTNT{Ge. 2:24.}. Que l'homme donc ne sépare pas ce que Dieu a mis ensemble sous un joug\FTNT{La majorité des traducteurs traduisent ce verset par « Que l'homme donc ne sépare pas ce que Dieu a joint. ». Or le terme grecque « suzeugnumi » qu'ils ont traduit par « joint » signifie plutôt « attacher un joug à quelqu'un, mettre ensemble sous un joug ». Un joug est une pièce de bois servant à atteler une paire d'animaux. De ce fait, les animaux sont contraints d'avancer dans la même direction, côte à côte (Ge. 2:22). En De. 22:10, Dieu interdit d'atteler un âne avec un bœuf ensemble. L'adverbe « ensemble » vient de l'hébreu « Yachad » et signifie « union d'une façon unitaire ». Ce verset fait référence symboliquement aux paroles de l'apôtre Paul en 2 Co. 6:14-16 qui nous mettent en garde contre le mariage avec des infidèles. Le mariage est donc semblable à un joug qui nous contraint à marcher à l'unisson dans la même direction. Ainsi, si on se lie à un inconverti, ce dernier risque de nous entraîner sur la voie de la perdition. En Mt 11:29, Christ nous invite à nous mettre sous son joug, qui est doux et léger. Quelle belle demande en mariage !}.
\VS{7}Ils lui dirent : Pourquoi donc Moïse a-t-il commandé de donner la lettre de divorce, et de répudier sa femme\FTNT{De. 24:1.} ?
\VS{8}Il leur répondit : C'est à cause de la dureté de votre cœur que Moïse vous a permis de répudier vos femmes, mais au commencement il n'en était pas ainsi.
\VS{9}Et moi je vous dis, que quiconque répudiera sa femme, si ce n'est pour cause d'adultère\FTNT{Adultère : Du grec « porneia » c'est-à-dire relation sexuelle illicite, impudicité.}, et se mariera à une autre, commet un adultère ; et que celui qui se sera marié à celle qui est répudiée, commet un adultère.
\VS{10}Ses disciples lui dirent : Si telle est la condition de l'homme à l'égard de sa femme, il ne convient pas de se marier.
\VS{11}Mais il leur répondit : Tous ne sont pas capables de cela, mais seulement ceux à qui il est donné.
\VS{12}Car il y a des eunuques, qui sont ainsi nés dés le ventre de leur mère ; et il y a des eunuques, qui ont été faits eunuques par les hommes ; et il y a des eunuques qui se sont faits eux-mêmes eunuques pour le Royaume des cieux. Que celui qui peut comprendre ceci, le comprenne.
\TextTitle{Le Royaume des cieux pour ceux qui ressemblent aux petits enfants\FTNTT{Mc. 10:13-16 ; Lu. 18:15-17}}
\VS{13}Alors on lui présenta des petits enfants, afin qu'il leur impose les mains et qu'il prie pour eux. Mais les disciples les en reprenaient.
\VS{14}Et Jésus leur dit : Laissez venir à moi les petits enfants et ne les empêchez pas ; car le Royaume des cieux est pour ceux qui leur ressemblent.
\VS{15}Puis il leur imposa les mains et il partit de là.
\TextTitle{Le jeune homme riche\FTNTT{Mc. 10:17-31 ; Lu. 10:25-37 ; Lu. 18:18-27.}}
\VS{16}Et voici, quelqu'un s'approchant lui dit : Maître qui est bon, quel bien ferai-je pour avoir la vie éternelle ?
\VS{17}Il lui répondit : Pourquoi m'appelles-tu bon ? Dieu est le seul être qui soit bon. Que si tu veux entrer dans la vie, garde les commandements.
\VS{18}Il lui dit : Lesquels ? Et Jésus lui répondit : Tu ne tueras point. Tu ne commettras point d'adultère. Tu ne déroberas point. Tu ne diras point de faux témoignage.
\VS{19}Honore ton père et ta mère ; et tu aimeras ton prochain comme toi-même\FTNT{Ex. 20:12-16 ; Lé. 19:18.}.
\VS{20}Le jeune homme lui dit : J'ai gardé toutes ces choses dès ma jeunesse. Que me manque-t-il encore ?
\VS{21}Jésus lui dit : Si tu veux être parfait, va, vends ce que tu as, et donne-le aux pauvres, et tu auras un trésor dans le ciel ; puis viens et suis-moi.
\VS{22}Mais quand ce jeune homme eut entendu cette parole, il s'en alla tout triste, parce qu'il avait de grands biens.
\VS{23}Alors Jésus dit à ses disciples : Je vous le dis en vérité, un riche entrera difficilement dans le Royaume des cieux.
\VS{24}Je vous le dis encore : Il est plus aisé à un chameau de passer par le trou d'une aiguille\FTNT{Le trou d'une aiguille : Jésus fait référence à une porte de la ville de Jérusalem qui était trop basse pour que les chameaux puissent y passer avec leurs chargements.}, qu'il ne l'est qu'un riche entre dans le Royaume de Dieu.
\VS{25}Ses disciples ayant entendu ces choses furent très étonnés et dirent : Qui peut donc être sauvé ?
\VS{26}Et Jésus les regarda et leur dit : Quant aux hommes, cela est impossible, mais quant à Dieu toutes choses sont possibles.
\TextTitle{Récompenses actuelles et dans le Royaume à venir\FTNTT{Mc. 10:28-31 ; Lu. 18:28-30}}
\VS{27}Alors Pierre prenant la parole, lui dit : Voici, nous avons tout quitté et nous t'avons suivi ; que nous en arrivera-t-il donc ?
\VS{28}Et Jésus leur dit : Je vous le dis en vérité, quand le Fils de l'homme, au renouvellement de toutes choses, sera assis sur le trône de sa gloire, vous qui m'avez suivi, vous serez assis sur douze trônes et vous jugerez les douze tribus d'Israël.
\VS{29}Et quiconque aura quitté ou maisons, ou frères, ou sœurs, ou père, ou mère, ou femme, ou enfants, ou champs, à cause de mon Nom, il en recevra cent fois autant et héritera la vie éternelle.
\VS{30}Mais plusieurs qui sont les premiers seront les derniers et les derniers seront les premiers.
\Chap{20}
\TextTitle{Parabole des ouvriers}
\VerseOne{} Car le Royaume des cieux est semblable à un père de famille, qui sortit dès le point du jour afin de louer des ouvriers pour sa vigne.
\VS{2}Et quand il eut accordé avec les ouvriers à un denier par jour, il les envoya à sa vigne.
\VS{3}Puis étant sorti vers la troisième heure, il en vit d'autres qui étaient sur la place publique, sans rien faire.
\VS{4}Il leur dit : Allez aussi à ma vigne, et je vous donnerai ce qui sera raisonnable.
\VS{5}Et ils y allèrent. Puis il sortit de nouveau vers la sixième heure et vers la neuvième, et il fit de même.
\VS{6}Et étant sorti vers la onzième heure, il en trouva d'autres qui étaient sur la place publique sans rien faire, et il leur dit : Pourquoi vous tenez-vous ici toute la journée sans rien faire ?
\VS{7}Ils lui répondirent : Parce que personne ne nous a loués. Et il leur dit : Allez-vous aussi à ma vigne et vous recevrez ce qui sera raisonnable.
\VS{8}Et le soir étant venu, le maître de la vigne dit à son intendant : Appelle les ouvriers et paye-leur le salaire, en commençant depuis les derniers jusqu'aux premiers.
\VS{9}Alors ceux qui avaient été loués vers la onzième heure vinrent et reçurent chacun un denier.
\VS{10}Or quand les premiers furent venus, ils croyaient recevoir davantage, mais ils reçurent aussi chacun un denier.
\VS{11}Et l'ayant reçu, ils murmuraient contre le père de famille,
\VS{12}en disant : Ces derniers n'ont travaillé qu'une heure, et tu les as faits égaux à nous, qui avons supporté le poids du jour et la chaleur.
\VS{13}Et il répondit à l'un d'eux et lui dit : Mon ami, je ne te fais pas de tort, n'es-tu pas tombé d'accord avec moi pour un denier ?
\VS{14}Prends ce qui est à toi et va-t'en. Mais je veux donner à ce dernier autant qu'à toi,
\VS{15}ne m'est-il pas permis de faire ce que je veux de mes biens ? Ou vois-tu d'un mauvais œil que je sois bon ?
\VS{16}Ainsi les derniers seront les premiers et les premiers seront les derniers, car il y a beaucoup d'appelés, mais peu d'élus.
\TextTitle{Jésus annonce à nouveau sa mort et sa résurrection\FTNTT{Mt. 12:38-42 ; 16:21-28 ; 17:22-23 ; Mc. 10:32-34 ; Lu. 18:31-34.}}
\VS{17}Pendant que Jésus montait à Jérusalem, il prit à part ses douze disciples et il leur dit en chemin :
\VS{18}Voici, nous montons à Jérusalem, et le Fils de l'homme sera livré aux principaux sacrificateurs et aux scribes, et ils le condamneront à la mort.
\VS{19}Ils le livreront aux nations pour qu'elles se moquent de lui, le battent de verges et le crucifient ; et le troisième jour il ressuscitera.
\TextTitle{Réponse de Jésus à la requête de la mère de Jacques et Jean\FTNTT{Mc. 10:35-45}}
\VS{20}Alors la mère des fils de Zébédée s'approcha de lui avec ses fils et se prosterna pour lui demander quelque chose.
\VS{21}Et il lui dit : Que veux-tu ? Elle lui dit : Ordonne que mes deux fils, qui sont ici, soient assis l'un à ta droite, et l'autre à ta gauche dans ton Royaume.
\VS{22}Et Jésus répondit et dit : Vous ne savez pas ce que vous demandez. Pouvez-vous boire la coupe que je dois boire, et être baptisés du baptême dont je dois être baptisé ? Ils lui répondirent : Nous le pouvons.
\VS{23}Et il leur dit : Il est vrai que vous boirez ma coupe et que vous serez baptisés du baptême dont je serai baptisé ; mais pour ce qui est d'être assis à ma droite ou à ma gauche, cela ne dépend pas de moi, et ne sera donné qu'à ceux à qui mon Père l'a réservé.
\VS{24}Les dix autres disciples ayant entendu cela, furent indignés contre les deux frères.
\VS{25}Mais Jésus les appela et leur dit : Vous savez que les princes des nations les dominent, et que les grands les asservissent.
\VS{26}Mais il n'en sera pas ainsi entre vous. Au contraire, quiconque veut être grand entre vous, qu'il soit votre serviteur.
\VS{27}Et quiconque veut être le premier parmi vous, qu'il soit votre serviteur.
\VS{28}De même que le Fils de l'homme n'est pas venu pour être servi, mais pour servir, et afin de donner sa vie en rançon pour plusieurs.
\TextTitle{Jésus guérit deux aveugles\FTNTT{Mc. 10:46-53 ; Lu. 18:35-43}}
\VS{29}Et comme ils partaient de Jéricho, une grande foule le suivit.
\VS{30}Et voici, deux aveugles qui étaient assis au bord du chemin, entendirent que Jésus passait, et crièrent en disant : Seigneur, Fils de David ! Aie pitié de nous !
\VS{31}Et la foule les reprenait pour les faire taire ; mais ils criaient encore plus fort : Seigneur, Fils de David ! Aie pitié de nous !
\VS{32}Jésus s'arrêta les appela et leur dit : Que voulez-vous que je vous fasse ?
\VS{33}Ils lui dirent : Seigneur, que nos yeux soient ouverts.
\VS{34}Et Jésus étant ému de compassion, toucha leurs yeux, et aussitôt ils recouvrèrent la vue et ils le suivirent.
\Chap{21}
\TextTitle{Jésus-Christ se présente publiquement comme Roi\FTNTT{Za. 9:9 ; Mc. 11:1-11 ; Lu. 19:28-40 ; Jn. 12:12-19.}}
\VerseOne{}Et quand ils furent près de Jérusalem, et qu'ils furent arrivés à Bethphagé vers le Mont des Oliviers, Jésus envoya alors deux disciples,
\VS{2}en leur disant : Allez au village qui est devant vous. Vous trouverez une ânesse attachée, et son ânon avec elle. Détachez-les et amenez-les-moi.
\VS{3}Et si quelqu'un vous dit quelque chose, vous direz que le Seigneur en a besoin ; et aussitôt il les laissera aller.
\VS{4}Or, tout cela arriva afin que s'accomplisse ce qui avait été annoncé par le prophète, en disant :
\VS{5}Dites à la fille de Sion : Voici, ton Roi vient à toi, plein de douceur, et monté sur un âne, sur un ânon, le petit d'une ânesse\FTNT{Za. 9:9.}.
\VS{6}Les disciples donc s'en allèrent et firent ce que Jésus leur avait ordonné.
\VS{7}Et ils amenèrent l'ânesse et l'ânon, et mirent leurs vêtements sur eux, et le firent asseoir dessus.
\VS{8}Alors de grandes foules étendirent leurs vêtements sur le chemin, et les autres coupaient des rameaux des arbres, et les étendaient sur le chemin.
\VS{9}Et les foules qui allaient devant, et celles qui suivaient, criaient en disant : Hosanna au Fils de David ! Béni soit celui qui vient au Nom du Seigneur ! Hosanna dans les lieux très hauts !
\VS{10}Lorsqu'il entra dans Jérusalem, toute la ville fut émue et l'on disait : Qui est celui-ci ?
\VS{11}Et les foules disaient : C'est Jésus, le prophète de Nazareth en Galilée.
\TextTitle{Jésus chasse les marchands du temple\FTNTT{Mc. 11:15-18 ; Lu. 19:45-46 ; Jn. 2:13-16}}
\VS{12}Jésus entra dans le temple de Dieu. Il chassa dehors tous ceux qui vendaient et qui achetaient dans le temple ; il renversa les tables des changeurs et les sièges de ceux qui vendaient des pigeons ;
\VS{13}et il leur dit : Il est écrit : Ma maison sera appelée une maison de prière, mais vous en avez fait une caverne de voleurs\FTNT{Es. 56:7 ; Jé. 7:11.}.
\VS{14}Alors des aveugles et des boiteux s'approchèrent de lui dans le temple et il les guérit.
\VS{15}Mais les principaux sacrificateurs et les scribes furent indignés à la vue des choses merveilleuses qu'il avait faites, et des enfants qui criaient dans le temple : Hosanna au Fils de David !
\VS{16}Et ils lui dirent : Entends-tu ce qu'ils disent ? Oui, leur répondit Jésus. N'avez-vous jamais lu ces paroles : Tu as tiré des louanges de la bouche des enfants, et de ceux qui sont à la mamelle\FTNT{Ps. 8:3.} ?
\VS{17}Et, les ayant laissés, il sortit de la ville, pour aller à Béthanie, où il passa la nuit.
\TextTitle{Le figuier stérile\FTNTT{Mc. 11:12-14,20-26}}
\VS{18}Le matin, comme il retournait à la ville, il eut faim.
\VS{19}Et voyant un figuier qui était sur le chemin, il s'en approcha, mais il n'y trouva que des feuilles ; et il lui dit : Qu'aucun fruit ne naisse plus jamais de toi ! Et aussitôt le figuier sécha.
\VS{20}Les disciples qui virent cela furent étonnés et dirent : Comment ce figuier est-il devenu sec en un instant ?
\VS{21}Jésus leur répondit : Je vous le dis en vérité, si vous aviez la foi, et que vous ne doutiez point, non seulement vous ferez ce qui a été fait à ce figuier, mais quand vous diriez à cette montagne : Ôte-toi de là et jette-toi dans la mer, cela se ferait.
\VS{22}Et quoi que vous demandiez en priant Dieu si vous croyez, vous le recevrez.
\TextTitle{L'incrédulité des principaux sacrificateurs et des anciens\FTNTT{Mc. 11:27-33 ; Lu. 20:1-8}}
\VS{23}Puis, s'étant rendu dans le temple, les principaux sacrificateurs et les anciens du peuple vinrent auprès de lui, pendant qu'il enseignait, et lui dirent : Par quelle autorité fais-tu ces choses ; et qui t'a donné cette autorité ?
\VS{24}Jésus répondant leur dit : Je vous interrogerai aussi sur une chose, et si vous me répondez, je vous dirai par quelle autorité je fais ces choses.
\VS{25}Le baptême de Jean d'où venait-il ? Du ciel ou des hommes ? Mais ils raisonnèrent ainsi entre eux : Si nous disons : Du ciel, il nous dira : Pourquoi n'avez-vous pas cru en lui ?
\VS{26}Et si nous disons : Des hommes, nous craignons la foule, car tous tiennent Jean pour un prophète.
\VS{27}Alors ils répondirent à Jésus : Nous ne savons pas. Et il leur dit : Moi non plus, je ne vous dirai pas par quelle autorité je fais ces choses.
\TextTitle{Parabole des deux fils}
\VS{28}Mais que vous en semble ? Un homme avait deux fils ; et s'adressant au premier, il lui dit : Mon fils, va travailler aujourd'hui dans ma vigne.
\VS{29}Il répondit : Je ne veux pas y aller. Ensuite il se repentit et y alla.
\VS{30}S'adressant à l'autre, il lui dit la même chose. Et ce fils répondit : Je veux bien, seigneur. Et il n'alla pas.
\VS{31}Lequel des deux a fait la volonté du père ? Ils lui répondirent : Le premier. Et Jésus leur dit : Je vous le dis en vérité, les publicains et les prostituées vous devanceront dans le Royaume de Dieu.
\VS{32}Car Jean est venu à vous dans la voie de la justice et vous ne l'avez pas cru ; mais les publicains et les femmes débauchées ont cru en lui. Et vous, qui avez vu cela, vous ne vous êtes pas ensuite repentis pour croire en lui.
\TextTitle{Parabole des vignerons\FTNTT{Es. 5:1-7 ; Mc. 12:1-12 ; Lu. 20:9-18.}}
\VS{33}Ecoutez une autre parabole : Il y avait un père de famille qui planta une vigne, et l'entoura d'une haie, et y creusa un pressoir, et bâtit une tour ; puis il l'afferma à des vignerons, et quitta le pays.
\VS{34}Lorsque la saison de la récolte fut arrivé, il envoya ses serviteurs vers les vignerons pour recevoir les fruits.
\VS{35}Mais les vignerons s'étant saisis de ses serviteurs, fouettèrent l'un, tuèrent l'autre et lapidèrent le troisième.
\VS{36}Il envoya encore d'autres serviteurs en plus grand nombre que les premiers, et ils leur firent de même.
\VS{37}Enfin, il envoya vers eux son propre fils, en disant : Ils auront du respect pour mon fils.
\VS{38}Mais quand les vignerons virent le fils, ils dirent entre eux : Voici l'héritier. Venez, tuons-le et emparons-nous de son héritage.
\VS{39}Et s'étant saisis de lui, il le jetèrent hors de la vigne et le tuèrent.
\VS{40}Quand donc le maître de la vigne viendra, que fera-t-il à ces vignerons ?
\VS{41}Ils lui dirent : Il les fera périr malheureusement comme des méchants et louera sa vigne à d'autres vignerons, qui lui en rendront les fruits en leur saison.
\VS{42}Et Jésus leur dit : N'avez-vous jamais lu dans les Ecritures : La pierre qu'ont rejetée ceux qui bâtissaient, est devenue la principale de l'angle. C'est du Seigneur que cela est venu, et c'est un prodige à nos yeux\FTNT{Es. 8:13-17 ; Es. 28:16.} ?
\VS{43}C'est pourquoi je vous dis que le Royaume de Dieu vous sera enlevé et il sera donné à une nation qui en rendra les fruits.
\VS{44}Celui qui tombera sur cette pierre s'y brisera, et celui sur qui elle tombera sera écrasé.
\VS{45}Après avoir entendu ses paraboles, les principaux sacrificateurs et les pharisiens comprirent qu'il parlait d'eux.
\VS{46}Et ils cherchaient à se saisir de lui, mais ils craignaient la foule, parce qu'elle le tenait pour un prophète.
\Chap{22}
\TextTitle{Parabole des noces\FTNTT{Lu. 14:16-24}}
\VerseOne{}Alors Jésus, prenant la parole, leur parla de nouveau en paraboles et il dit :
\VS{2}Le Royaume des cieux est semblable à un roi qui fit des noces pour son fils.
\VS{3}Il envoya ses serviteurs pour appeler ceux qui avaient été conviés aux noces ; mais ils ne voulurent pas venir.
\VS{4}Il envoya encore d'autres serviteurs, disant : Dites aux conviés : Voici, j'ai préparé mon festin ; mes bœufs et mes bêtes grasses sont tués, et tout est prêt ; venez aux noces.
\VS{5}Mais, sans tenir compte de l'invitation, ils s'en allèrent l'un à son champ, et l'autre à son trafic.
\VS{6}Et les autres se saisirent de ses serviteurs les outragèrent, et les tuèrent.
\VS{7}Quand le roi l'entendit, il se mit en colère ; il envoya ses troupes, fit périr ces meurtriers et brûla leur ville.
\VS{8}Puis il dit à ses serviteurs : Les noces sont prêtes, mais les conviés n'en étaient pas dignes.
\VS{9}Allez donc dans les carrefours des chemins, et autant de gens que vous trouverez, appelez-les aux noces.
\VS{10}Alors ces serviteurs allèrent dans les chemins et rassemblèrent tous ceux qu'ils trouvèrent, méchants et bons, et la salle des noces fut remplie de conviés qui étaient à table.
\VS{11}Et le roi étant entré pour voir ceux qui étaient à table, il aperçut là un homme qui n'avait pas revêtu un habit de noces\FTNT{Ap. 19:7-8.}.
\VS{12}Et il lui dit : Mon ami, comment es-tu entré ici sans avoir un habit de noces ? Cet homme eut la bouche fermée.
\VS{13}Alors le roi dit aux serviteurs : Liez-lui les pieds et les mains, emportez-le et jetez-le dans les ténèbres de dehors, où il y aura des pleurs et des grincements de dents.
\VS{14}Car il y a beaucoup d'appelés, mais peu d'élus.
\TextTitle{Le tribut dû à César\FTNTT{Mc. 12:13-17 ; Lu. 20:19-26}}
\VS{15}Alors les pharisiens allèrent se consulter ensemble sur les moyens de le surprendre par ses propres paroles.
\VS{16}Ils envoyèrent auprès de lui leurs disciples, avec des hérodiens, qui dirent : Maître, nous savons que tu es véritable, que tu enseignes la voie de Dieu selon la vérité, sans t'inquiéter de personne ; car tu ne regardes point à l'apparence des hommes.
\VS{17}Dis-nous donc ce qu'il t'en semble : Est-il permis de payer le tribut à César, ou non ?
\VS{18}Et Jésus connaissant leur malice, dit : Hypocrites, pourquoi me tentez-vous ?
\VS{19}Montrez-moi la monnaie avec laquelle on paie le tribut ; et ils lui présentèrent un denier.
\VS{20}Il leur demanda : De qui porte-t-il l'image et l'inscription ?
\VS{21}De César, lui répondirent-ils. Alors il leur dit : Rendez donc à César ce qui est à César, et à Dieu, ce qui est à Dieu.
\VS{22}Et ayant entendu cela, ils furent étonnés, ils le quittèrent et s'en allèrent.
\TextTitle{Enseignement de Jésus sur la résurrection\FTNTT{Mc. 12:18-27 ; Lu. 20:27-38}}
\VS{23}Le même jour, les sadducéens, qui disent qu'il n'y a pas de résurrection, vinrent auprès de lui et lui posèrent cette question,
\VS{24}en disant : Maître, Moïse a dit : Si quelqu'un meurt sans enfants, son frère épousera sa femme et suscitera une postérité à son frère.
\VS{25}Or, il y avait parmi nous sept frères. Le premier se maria et mourut ; et, n'ayant pas eu d'enfants, il laissa sa femme à son frère.
\VS{26}Il en fut de même du deuxième, puis du troisième, jusqu'au septième.
\VS{27}Après eux tous, la femme mourut aussi.
\VS{28}A la résurrection, duquel des sept sera-t-elle la femme ? Car tous l'ont eue.
\VS{29}Mais Jésus répondant leur dit : Vous êtes dans l'erreur, parce que vous ne connaissez ni les Ecritures, ni la puissance de Dieu.
\VS{30}Car à la résurrection on ne prendra ni on ne donnera de femmes en mariage, mais on sera comme les anges de Dieu dans le ciel.
\VS{31}Et quant à la résurrection des morts, n'avez-vous point lu ce dont Dieu vous a parlé, disant :
\VS{32}Je suis le Dieu d'Abraham, le Dieu d'Isaac et le Dieu de Jacob\FTNT{Ge. 17:7 ; Ge. 26:24 ; Ge. 28:21.}. Or Dieu n'est pas le Dieu des morts, mais des vivants.
\VS{33}Ce que les foules ayant entendu, elles admirèrent sa doctrine.
\TextTitle{Le plus grand commandement de la loi\FTNTT{Mc. 12:28-34 ; Lu. 10:25-28}}
\VS{34}Quand les pharisiens apprirent qu'il avait fermé la bouche aux sadducéens, ils se rassemblèrent dans un même lieu,
\VS{35}et l'un d'eux, qui était docteur de la loi, l'interrogea pour l'éprouver, en disant :
\VS{36}Maître, quel est le plus grand commandement de la loi ?
\VS{37}Jésus lui dit : Tu aimeras le Seigneur ton Dieu de tout ton cœur, de toute ton âme et de toute ta pensée.
\VS{38}Celui-ci est le premier et le grand commandement.
\VS{39}Et voici le deuxième qui lui est semblable : Tu aimeras ton prochain comme toi-même.
\VS{40}De ces deux commandements dépendent toute la loi et les prophètes.
\TextTitle{Jésus interroge les pharisiens au sujet du Messie\FTNTT{Mc. 12:35-37 ; Lu. 20:39-44}}
\VS{41}Et les Pharisiens étant assemblés, Jésus les interrogea,
\VS{42}Disant : que pensez-vous du Christ ? De qui est-il Fils ? Ils lui répondirent : de David.
\VS{43}Et il leur dit : Comment donc David, parlant par l'Esprit, l'appelle-t-il son Seigneur ? Disant :
\VS{44}Le Seigneur a dit à mon Seigneur, assieds-toi à ma droite, jusqu'à ce que j'aie mis tes ennemis pour le marchepied de tes pieds\FTNT{Ps. 110:1.}.
\VS{45}Si donc David l'appelle son Seigneur, comment est-il son Fils ?
\VS{46}Et personne ne pouvait lui répondre un seul mot. Et depuis ce jour, personne n'osa plus lui poser des questions.
\Chap{23}
\TextTitle{Caractéristiques des scribes et des pharisiens\FTNTT{Mc. 12:38-40 ; Lu. 11:39-54 ; Lu. 20:45-47}}
\VerseOne{}Alors Jésus parla à la foule et à ses disciples,
\VS{2}disant : Les scribes et les pharisiens sont assis dans la chaire de Moïse.
\VS{3}Toutes les choses donc qu'ils vous diront d'observer, observez-les et faites-les, mais non point leurs œuvres : parce qu'ils disent et ne font pas.
\VS{4}Car ils lient ensemble des fardeaux pesants et insupportables et les mettent sur les épaules des hommes ; mais ils ne veulent point les remuer de leur doigt.
\VS{5}Et ils font toutes leurs œuvres pour être vus des hommes. Ainsi, ils portent de larges phylactères et de longues franges à leurs vêtements.
\VS{6}Ils aiment les premières places dans les festins, et les premiers sièges dans les synagogues.
\VS{7}Ils aiment les salutations dans les places publiques, et à être appelés par les hommes : Notre maître ! Notre maître !
\VS{8}Mais vous, ne vous faites pas appeler, Notre maître ; car Christ seul est votre Docteur ; et vous êtes tous frères.
\VS{9}Et n'appelez personne sur la terre votre père ; car un seul est votre Père, celui qui est dans les cieux.
\VS{10}Et ne soyez point appelés Docteurs : car Christ seul est votre Docteur.
\VS{11}Mais que celui qui est le plus grand entre vous, soit votre serviteur.
\VS{12}Car quiconque s'élèvera sera abaissé ; et quiconque s'abaissera, sera élevé.
\VS{13}Mais malheur à vous, scribes et pharisiens hypocrites, qui fermez le Royaume des cieux aux hommes : car vous-mêmes n'y entrez point, et vous n'y laissez pas entrer ceux qui veulent y entrer.
\VS{14}Malheur à vous, scribes et pharisiens hypocrites, car vous dévorez les maisons des veuves, même sous le prétexte de faire de longues prières, c'est pourquoi vous en recevrez une plus grande condamnation.
\VS{15}Malheur à vous, scribes et pharisiens hypocrites ! Parce que vous courez la mer et la terre pour faire un prosélyte, et quand il l'est devenu, vous le rendez fils de la géhenne, deux fois plus que vous.
\VS{16}Malheur à vous conducteurs aveugles, qui dites : Si quelqu'un jure par le temple, ce n'est rien ; mais si quelqu'un jure par l'or du temple, il est engagé.
\VS{17}Insensés et aveugles ! Car lequel est le plus grand, l'or, ou le temple qui sanctifie l'or ?
\VS{18}Si quelqu'un, dites-vous encore, jure par l'autel, ce n'est rien ; mais si quelqu'un jure par l'offrande qui est sur l'autel, il est engagé.
\VS{19}Insensés et aveugles ! Car lequel est le plus grand, l'offrande, ou l'autel qui sanctifie l'offrande ?
\VS{20}Celui donc qui jure par l'autel, jure par l'autel et par toutes les choses qui sont dessus.
\VS{21}Celui qui jure par le temple, jure par le temple et par celui qui y habite ;
\VS{22}et celui qui jure par le ciel, jure par le trône de Dieu et par celui qui y est assis.
\VS{23}Malheur à vous, scribes et pharisiens hypocrites ! Parce que vous payez la dîme\FTNT{Voir commentaire en Mal. 3:10.} de la menthe, de l'aneth et du cumin ; et vous laissez les choses les plus importantes de la loi, c'est-à-dire la justice, la miséricorde et la fidélité. Il fallait pratiquer ces choses-là, sans négliger les autres choses.
\VS{24}Conducteurs aveugles ! Vous coulez le moucheron et vous engloutissez le chameau\FTNT{Les pharisiens filtraient leur eau par crainte d'avaler un moucheron.}.
\VS{25}Malheur à vous, scribes et pharisiens hypocrites ! Parce que vous nettoyez le dehors de la coupe et du plat ; alors qu'au-dedans ils sont pleins de rapine et d'intempérance.
\VS{26}Pharisien aveugle, nettoie premièrement l'intérieur de la coupe et du plat, afin que l'extérieur aussi devienne net.
\VS{27}Malheur à vous, scribes et pharisiens hypocrites ! Parce que vous êtes semblables aux sépulcres blanchis, qui paraissent beaux au-dehors, et qui au-dedans sont pleins d'ossements de morts, et de toutes espèces d'impuretés.
\VS{28}Ainsi, au-dehors vous paraissez justes aux hommes, mais au-dedans vous êtes pleins d'hypocrisie et d'iniquité.
\VS{29}Malheur à vous, scribes et pharisiens hypocrites ! Parce que vous bâtissez les tombeaux des prophètes et vous ornez les sépulcres des justes ;
\VS{30}et vous dites : Si nous avions vécu du temps de nos pères, nous n'aurions pas participé avec eux au meurtre des prophètes.
\VS{31}Ainsi vous êtes témoins contre vous-mêmes, que vous êtes les enfants de ceux qui ont fait mourir les prophètes.
\VS{32}Et vous achevez de remplir la mesure de vos pères.
\VS{33}Serpents, race de vipères ! Comment éviterez-vous le supplice de la géhenne ?
\VS{34}Car voici, je vous envoie des prophètes, des sages et des scribes. Vous tuerez et crucifierez les uns, vous battrez de verges les autres dans vos synagogues, et vous les persécuterez de ville en ville,
\VS{35}afin que vienne sur vous tout le sang innocent qui a été répandu sur la terre, depuis le sang d'Abel le juste, jusqu'au sang de Zacharie, fils de Barachie, que vous avez tué entre le temple et l'autel.
\VS{36}Je vous le dis en vérité, que toutes ces choses viendront sur cette génération.
\TextTitle{Lamentations de Jésus sur Jérusalem\FTNTT{Jé. 22:5 ; Lu. 13:34-35 ; 19:41-44.}}
\VS{37}Jérusalem, Jérusalem, qui tues les prophètes, et qui lapides ceux qui te sont envoyés, combien de fois ai-je voulu rassembler tes enfants, comme la poule rassemble ses poussins sous ses ailes, et vous ne l'avez point voulu !
\VS{38}Voici, votre maison va devenir déserte.
\VS{39}Car je vous dis, que désormais vous ne me verrez plus, jusqu'à ce que vous disiez : Béni soit celui qui vient au Nom du Seigneur\FTNT{Ps. 118:26.}!
\Chap{24}
\TextTitle{Prophétie sur la destruction du temple de Jérusalem\FTNTT{Mc. 13:1-2 ; Lu. 21:5-6}}
\VerseOne{}Comme Jésus sortait et s'en allait du temple, ses disciples s'approchèrent de lui pour lui faire remarquer les bâtiments du temple.
\VS{2}Mais Jésus leur dit : Voyez-vous bien toutes ces choses ? Je vous le dis en vérité, il ne restera pas ici pierre sur pierre qui ne soit démolie.
\TextTitle{Le signe de l'accomplissement\FTNTT{Mc. 13:3-4 ; Lu. 21:7}}
\VS{3}Puis s'étant assis sur la Montagne des Oliviers, ses disciples vinrent à lui en particulier et lui dirent : Dis-nous quand ces choses arriveront, et quel sera le signe de ton avènement, et de la fin du monde ?
\TextTitle{Les temps de la fin\FTNTT{Da. 9:27 ; Mc. 13:5-13 ; Lu. 21:8-11}}
\VS{4}Et Jésus répondant leur dit : Prenez garde que personne ne vous séduise.
\VS{5}Car plusieurs viendront sous mon Nom, disant : Je suis le Christ. Et ils en séduiront plusieurs.
\VS{6}Vous entendrez parler de guerres et de bruits de guerres ; gardez-vous d'être troublés ; car il faut que toutes ces choses arrivent ; mais ce ne sera pas encore la fin.
\VS{7}Car une nation s'élèvera contre une autre nation, et un royaume contre un autre royaume ; et il y aura des famines, des pestes, et des tremblements de terre en divers lieux.
\VS{8}Mais toutes ces choses ne seront que le commencement des douleurs.
\VS{9}Alors ils vous livreront aux tourments, et vous tueront ; et vous serez haïs de toutes les nations, à cause de mon Nom.
\VS{10}Alors aussi plusieurs seront scandalisés, se trahiront et se haïront les uns les autres.
\VS{11}Et il s'élèvera plusieurs faux prophètes, qui en séduiront plusieurs.
\VS{12}Et parce que l'iniquité sera multipliée, la charité de plusieurs se refroidira.
\VS{13}Mais celui qui persévérera jusqu'à la fin, sera sauvé.
\VS{14}Cet Evangile du Royaume sera prêché dans toute la terre habitable, pour servir de témoignage à toutes les nations, et alors viendra la fin.
\TextTitle{L'abomination de la désolation\FTNTT{Da. 9:27 ; Da.11:32-35 ; Mc. 13:14-18 ; Lu. 21:20-23}}
\VS{15}Or quand vous verrez l'abomination qui causera la désolation, qui a été prédite par Daniel le Prophète\FTNT{Daniel fut le premier à parler de l'abomination de la désolation (Da. 9:24-27). « Et les forces se présenteront de sa part, elles profaneront le sanctuaire, la forteresse, elles feront cesser le sacrifice perpétuel, et dresseront l'abomination qui causera la désolation. » Da. 11:31. Cette prophétie s'est partiellement accomplie en 168 av. J.-C. lorsqu'Antiochus Epiphane (215 av. J.-C. - 163 av. J.-C.), roi de Syrie, défenseur zélé de la culture grecque, finança la construction du temple de Zeus à Athènes. Sa tentative d'hellénisation forcée de la Judée, soutenue par les grands prêtres Jason et Ménélas, provoqua la colère des Juifs traditionalistes. Antiochus avait interdit le culte mosaïque et consacra le temple de Jérusalem aux dieux grecs. En effet, il le pilla et y installa un autel du dieu Baal Shamen, puis il détruisit les murailles de la ville. Dans un édit de décembre 167 av. J.-C., il ordonna d'offrir des porcs en holocauste, interdit la circoncision, la lecture de la Torah, l'observance des fêtes de Yahweh et pourchassa les adversaires de l'hellénisation. En agissant de la sorte, il y avait deux choses principales que cet archétype de l'antichrist espérait changer en Israël : Les temps (le calendrier juif) et la Torah (la loi selon Da. 7:25). La deuxième partie de cette prophétie s'est accomplie en l'an 70 lors de la destruction du temple de Jérusalem par Titus (39-81 ap. J.-C.). La dernière partie de cette prophétie est en train de s'accomplir actuellement dans les assemblées où Satan distille des enseignements erronés au travers des faux prophètes. La prédication d'un évangile expurgé de son caractère christocentrique, de la nécessité de porter sa croix, axé sur les choses de ce monde, maintient les chrétiens dans une vie de péché. Ainsi, alors qu'ils sont censés être des temples vivants du Saint-Esprit (1 Co. 6:19), Satan s'est établi dans leurs cœurs. Enfin, la prophétie de Daniel trouvera son parfait accomplissement pendant le règne de la bête. L'homme impie s'introduira alors dans le temple de Jérusalem qui sera rebâti, se fera passer pour Dieu et se fera adorer à sa place (2 Th. 2:4).}, être établie dans le lieu saint, que celui qui lit ce prophète y fasse attention !
\VS{16}Alors, que ceux qui seront en Judée fuient dans les montagnes ;
\VS{17}et que celui qui sera sur le toit ne descende pas pour emporter quoi que ce soit de sa maison ;
\VS{18}que celui qui sera dans les champs ne retourne pas en arrière pour prendre ses habits.
\VS{19}Malheur aux femmes enceintes, et à celles qui allaiteront en ces jours-là.
\VS{20}Priez pour que votre fuite n'arrive pas en hiver, ni un jour de sabbat\FTNT{Sous la loi mosaïque, il était interdit aux juifs de parcourir plus de 2 000 coudées du lieu où ils se trouvaient pendant le sabbat (Ex. 16:29).}.
\TextTitle{La grande tribulation\FTNTT{Ps. 2:5 ; Jé. 30:5-8 ; Da.12:1 ; Mc. 13:19-23 ; Lu. 21:23-24}}
\VS{21}Car alors, la détresse sera si grande qu'il n'y en a point eu de semblable depuis le commencement du monde jusqu'à présent, et qu'il n'y en aura jamais.
\VS{22}Et si ces jours n'étaient abrégés, personne ne serait sauvé ; mais à cause des élus, ces jours seront abrégés.
\VS{23}Alors si quelqu'un vous dit : Voici, le Christ est ici ; ou, il est là ; ne le croyez point.
\VS{24}Car il s'élèvera de faux christs et de faux prophètes, ils feront de grands prodiges et des miracles, pour séduire même les élus, s'il était possible.
\VS{25}Voici, je vous l'ai prédit.
\VS{26}Si on vous dit : Voici, il est dans le désert, ne sortez point ; voici, il est dans les chambres, ne le croyez point.
\VS{27}Car, comme l'éclair part de l'orient et se montre jusqu'en occident, il en sera de même de l'avènement du Fils de l'homme.
\VS{28}Car là où est le cadavre, là s'assembleront les vautours.
\TextTitle{Retour du Roi sur la terre\FTNTT{Mc. 13:24-27 ; Lu. 21:25-28}}
\VS{29}Aussitôt après ces jours de détresse, le soleil s'obscurcira, la lune ne donnera plus sa lumière, et les étoiles tomberont du ciel, et les puissances des cieux seront ébranlées.
\VS{30}Alors le signe du Fils de l'homme paraîtra dans le ciel, toutes les tribus de la terre se lamenteront en se frappant la poitrine, et verront le Fils de l'homme venant sur les nuées du ciel, avec une grande puissance et une grande gloire.
\VS{31}Il enverra ses anges avec un grand son de trompette et ils rassembleront ses élus, des quatre vents, d'une extrémité des cieux à l'autre.
\TextTitle{Parabole du figuier\FTNTT{Mc. 13:28-31 ; Lu. 21:29-33}}
\VS{32}Mais apprenez la leçon tirée de la parabole du figuier. Dès que ses jeunes branches deviennent tendres et que ses feuilles poussent, vous savez que l'été est proche.
\VS{33}De même, quand vous verrez toutes ces choses, sachez que le Fils de l'homme est proche, à la porte.
\VS{34}Je vous le dis en vérité, cette génération ne passera point, jusqu'à ce que tout cela n'arrive.
\VS{35}Le ciel et la terre passeront, mais mes paroles ne passeront point.
\TextTitle{Exhortation à la vigilance\FTNTT{Mc. 13:32-37 ; Lu. 21:34-38}}
\VS{36}Pour ce qui est du jour et de l'heure, personne ne le sait, ni les anges des cieux, mais mon Père seul.
\VS{37}Mais comme il en était aux jours de Noé, il en sera de même de l'avènement du fils de l'homme.
\VS{38}Car, comme dans les jours avant le déluge, les hommes mangeaient et buvaient, se mariaient, et donnaient en mariage, jusqu'au jour où Noé entra dans l'arche ;
\VS{39}et ils ne connurent point que le déluge viendrait, jusqu'à ce qu'il vint et les emporta tous ; il en sera de même de l'avènement du Fils de l'homme.
\VS{40}Alors, de deux hommes qui seront dans un champ ; l'un sera pris, et l'autre laissé ;
\VS{41}de deux femmes qui moudront au moulin, l'une sera prise et l'autre laissée.
\VS{42}Veillez donc, car vous ne savez point à quelle heure votre Seigneur doit venir.
\VS{43}Mais sachez ceci, que si un père de famille savait à quelle veille de la nuit le voleur doit venir, il veillerait et ne laisserait pas percer sa maison.
\VS{44}C'est pourquoi, vous aussi tenez-vous prêts ; car le Fils de l'homme viendra à l'heure où vous n'y penserez pas.
\VS{45}Quel est donc le serviteur fidèle et prudent, que son maître a établi sur tous ses serviteurs, pour leur donner la nourriture au temps opportun ?
\VS{46}Heureux est ce serviteur que son maître en arrivant trouvera agir de cette manière.
\VS{47}Je vous le dis en vérité, il l'établira sur tous ses biens.
\VS{48}Mais si c'est un méchant serviteur, qui dit en lui-même : Mon maître tarde à venir ;
\VS{49}et s'il se met à battre ses compagnons de service, s'il mange et boit avec les ivrognes,
\VS{50}le maître de ce serviteur viendra le jour où il ne s'y attend pas et à l'heure qu'il ne connaît pas.
\VS{51}Et il le séparera, et le mettra au rang des hypocrites ; là il y aura des pleurs et des grincements de dents.
\Chap{25}
\TextTitle{Parabole des dix vierges}
\VerseOne{}Alors le Royaume des cieux sera semblable à dix vierges qui, ayant pris leurs lampes, allèrent à la rencontre de l'époux.
\VS{2}Or il y en avait cinq sages et cinq folles.
\VS{3}Les folles, en prenant leurs lampes, ne prirent pas d'huile avec elles ;
\VS{4}mais les sages prirent de l'huile dans leurs vases avec leurs lampes.
\VS{5}Et comme l'époux tardait à venir, elles s'assoupirent et s'endormirent toutes.
\VS{6}Or à minuit il se fit un cri disant : Voici, l'époux vient, allez à sa rencontre !
\VS{7}Alors toutes ces vierges se réveillèrent\FTNT{Réveiller : Du grec « egeiro ». Ce terme signifie également ressusciter. Les saints qui attendent le retour du Seigneur connaîtront un réveil après un temps de sommeil spirituel (Ro. 13:11).} et préparèrent leurs lampes.
\VS{8}Et les folles dirent aux sages : Donnez-nous de votre huile, car nos lampes s'éteignent.
\VS{9}Mais les sages répondirent en disant : Nous ne pouvons pas vous en donner, de peur que nous n'en ayons pas assez pour nous et pour vous ; mais allez plutôt chez ceux qui en vendent et achetez-en pour vous-mêmes.
\VS{10}Or pendant qu'elles allaient en acheter, l'époux arriva. Celles qui étaient prêtes entrèrent avec lui dans la salle des noces, puis la porte fut fermée.
\VS{11}Après cela, les autres vierges vinrent aussi, et dirent : Seigneur ! Seigneur ! Ouvre-nous !
\VS{12}Mais il leur répondit, et dit : Je vous le dis en vérité, je ne vous connais point.
\VS{13}Veillez donc ; car vous ne savez ni le jour ni l'heure en laquelle le Fils de l'homme viendra.
\TextTitle{Parabole des talents}
\VS{14}Car il en sera comme d'un homme qui, partant pour un voyage, appela ses serviteurs et leur remit ses biens.
\VS{15}Il donna à l'un cinq talents, à l'autre deux, et au troisième un ; à chacun selon sa capacité ; et aussitôt après il partit.
\VS{16}Celui qui avait reçu les cinq talents, s'en alla, et les fit valoir, et gagna cinq autres talents.
\VS{17}De même, celui qui avait reçu les deux talents, en gagna aussi deux autres.
\VS{18}Mais celui qui n'en avait reçu qu'un, alla et creusa dans la terre, et y cacha l'argent de son maître.
\VS{19}Longtemps après, le maître de ces serviteurs revint et leur fit rendre compte.
\VS{20}Alors celui qui avait reçu les cinq talents, vint et présenta cinq autres talents, en disant : Seigneur, tu m'as confié cinq talents, voici, j'en ai gagné cinq autres par-dessus.
\VS{21}Et son Seigneur lui dit : Cela est bien, bon et fidèle serviteur ; tu as été fidèle en peu de choses, je t'établirai sur beaucoup ; viens participer à la joie de ton Seigneur.
\VS{22}Ensuite, celui qui avait reçu les deux talents, vint et dit : Seigneur, tu m'as confié deux talents ; voici, j'en ai gagné deux autres par-dessus.
\VS{23}Et son Seigneur lui dit : Cela est bien, bon et fidèle serviteur, tu as été fidèle en peu de choses, je t'établirai sur beaucoup ; viens prendre part à la joie de ton Seigneur.
\VS{24}Mais celui qui n'avait reçu qu'un talent, vint et dit : Seigneur, je savais que tu es un homme dur, qui moissonnes où tu n'as point semé, et qui amasses où tu n'as point vanné,
\VS{25}c'est pourquoi craignant de perdre ton talent, je suis allé le cacher dans la terre. Voici, tu as ici ce qui t'appartient.
\VS{26}Et son Seigneur répondant, lui dit : Méchant et lâche serviteur, tu savais que je moissonnais où je n'ai point semé, et que j'amassais où je n'ai point vanné,
\VS{27}il te fallait donc remettre mon argent aux banquiers et à mon retour, je l'aurais retiré avec l'intérêt.
\VS{28}Ôtez-lui donc le talent et donnez-le à celui qui a les dix talents.
\VS{29}Car à celui qui a, il sera donné et il en aura encore plus, mais à celui qui n'a rien, cela même qu'il a, lui sera ôté.
\VS{30}Jetez donc le serviteur inutile dans les ténèbres de dehors ; où il y aura des pleurs et des grincements de dents.
\TextTitle{Séparation et jugement des brebis et des boucs\FTNTT{1 Co. 6:2}}
\VS{31}Or quand le Fils de l'homme viendra environné de sa gloire et accompagné de tous les saints anges, alors il s'assiéra sur le trône de sa gloire.
\VS{32}Et toutes les nations seront assemblées devant lui ; et il séparera les uns d'avec les autres, comme le berger sépare les brebis d'avec les boucs.
\VS{33}Et il mettra les brebis à sa droite et les boucs à sa gauche.
\VS{34}Alors le Roi dira à ceux qui seront à sa droite : Venez, vous qui êtes bénis de mon Père, possédez en héritage le Royaume qui vous a été préparé dès la fondation du monde.
\VS{35}Car j'ai eu faim et vous m'avez donné à manger ; j'ai eu soif et vous m'avez donné à boire ; j'étais étranger et vous m'avez recueilli ;
\VS{36}j'étais nu et vous m'avez vêtu ; j'étais malade et vous m'avez visité ; j'étais en prison et vous êtes venus vers moi.
\VS{37}Alors les justes lui répondront : Seigneur, quand t'avons-nous vu avoir faim et t'avons-nous donné à manger ; ou avoir soif et t'avons-nous donné à boire ?
\VS{38}Quand t'avons-nous vu étranger et t'avons-nous recueilli ; ou nu, et t'avons-nous vêtu ?
\VS{39}Ou quand t'avons-nous vu malade, ou en prison, et sommes-nous allés vers toi ?
\VS{40}Et le Roi répondant, leur dira : Je vous le dis en vérité, toutes les fois que vous avez fait ces choses à l'un de ces plus petits de mes frères, c'est à moi que vous les avez faites.
\VS{41}Alors il dira aussi à ceux qui seront à sa gauche : Maudits, retirez-vous de moi et allez dans le feu éternel, qui a été préparé pour le diable et pour ses anges.
\VS{42}Car j'ai eu faim et vous ne m'avez point donné à manger ; j'ai eu soif et vous ne m'avez point donné à boire ;
\VS{43}j'étais étranger et vous ne m'avez point recueilli ; j'ai été nu, et vous ne m'avez point vêtu ; j'ai été malade et en prison, et vous ne m'avez point visité.
\VS{44}Alors ils répondront aussi en disant : Seigneur, quand t'avons-nous vu avoir faim, ou avoir soif, ou être étranger, ou nu, ou malade, ou en prison, et ne t'avons-nous point secouru ?
\VS{45}Alors il leur répondra, en disant : Je vous le dis en vérité, toutes les fois que vous n'avez pas fait ces choses à l'un de ces plus petits, c'est à moi que vous ne les avez pas faites.
\VS{46}Et ceux-ci iront au châtiment éternel, mais les justes à la vie éternelle.
\Chap{26}
\TextTitle{Le complot\FTNTT{Mc. 14:1-2 ; Lu. 22:1-2}}
\VerseOne{}Et il arriva que quand Jésus eut achevé tous ces discours, il dit à ses disciples :
\VS{2}Vous savez que la fête de Pâque a lieu dans deux jours ; et le Fils de l'homme sera livré pour être crucifié.
\VS{3}Alors les principaux sacrificateurs, les scribes et les anciens du peuple, se réunirent dans la cour du souverain sacrificateur, appelé Caïphe ;
\VS{4}et tinrent conseil ensemble pour se saisir de Jésus par finesse, afin de le faire mourir.
\VS{5}Mais ils dirent : Que ce ne soit pas pendant la fête, de peur qu'il ne se fasse quelque tumulte parmi le peuple.
\TextTitle{Geste prophétique de Marie de Béthanie\FTNTT{Mc. 14:3-9 ; Jn. 12:1-8}}
\VS{6}Comme Jésus était à Béthanie, dans la maison de Simon le lépreux,
\VS{7}une femme s'approcha de lui tenant un vase d'albâtre, plein d'un parfum de grand prix, et pendant qu'il était à table, elle répandit le parfum sur sa tête.
\VS{8}Mais ses disciples voyant cela, en furent indignés et dirent : À quoi sert cette perte ?
\VS{9}Car ce parfum pouvait être vendu bien cher et être donné aux pauvres.
\VS{10}Mais Jésus connaissant cela, leur dit : Pourquoi faites-vous de la peine à cette femme ? Car elle a fait une bonne action à mon égard ;
\VS{11}car vous aurez toujours des pauvres avec vous ; mais vous ne m'aurez pas toujours.
\VS{12}En répandant ce parfum sur mon corps, elle l'a fait pour ma sépulture.
\VS{13}Je vous le dis en vérité, partout où cet Evangile sera prêché, dans le monde entier, on racontera aussi en mémoire de cette femme ce qu'elle a fait.
\TextTitle{La trahison de Judas\FTNTT{Mc. 14:10-11 ; Lu. 22:3-6}}
\VS{14}Alors l'un des douze, appelé Judas Iscariot, alla vers les principaux sacrificateurs,
\VS{15}et leur dit : Que voulez-vous me donner, et je vous le livrerai ? Et ils lui comptèrent trente pièces d'argent\FTNT{Za. 11:12-13.}.
\VS{16}Et dès lors, il cherchait une occasion favorable pour le livrer.
\TextTitle{La dernière Pâque\FTNTT{Mc. 14:12-21 ; Lu. 22:7-20 ; Jn. 13:1-12}}
\VS{17}Or le premier jour des pains sans levain, les disciples s'approchèrent de Jésus pour lui dire : Où veux-tu que nous te préparions le repas de la Pâque ?
\VS{18}Il répondit : Allez à la ville chez un tel et dites-lui : Le Maître dit : Mon temps est proche ; je ferai la Pâque chez toi avec mes disciples.
\VS{19}Les disciples firent comme Jésus leur avait ordonné et préparèrent la Pâque.
\VS{20}Et quand le soir fut venu, il se mit à table avec les douze.
\VS{21}Et comme ils mangeaient, il dit : Je vous le dis en vérité, l'un de vous me trahira.
\VS{22}Ils furent profondément attristés, et chacun d'eux commença à lui dire : Seigneur, est-ce moi ?
\VS{23}Mais il leur répondit : Celui qui a mis avec moi la main dans le plat pour tremper, c'est celui qui me trahira.
\VS{24}Le Fils de l'homme s'en va, selon qu'il est écrit de lui ; mais malheur à cet homme par qui le Fils de l'homme est trahi ! Mieux vaudrait pour cet homme qu'il ne soit pas né.
\VS{25}Judas qui le trahissait, prit la parole et dit : Maître, est-ce moi ? Jésus lui dit : Tu l'as dit.
\TextTitle{Le repas de la Pâque\FTNTT{Mc. 14:22-25 ; Lu. 22:17-20 ; Jn. 13:12-30 ; 1 Co. 11:23-26}}
\VS{26}Pendant qu'ils mangeaient, Jésus prit le pain, et après avoir rendu grâces à Dieu, il le rompit et le donna à ses disciples et leur dit : Prenez, mangez, ceci est mon corps.
\VS{27}Puis ayant pris la coupe, et béni Dieu, il la leur donna, en leur disant : buvez-en tous.
\VS{28}car ceci est mon sang, le sang de la Nouvelle Alliance, qui est répandu pour beaucoup, pour la rémission des péchés.
\VS{29}Or je vous dis : que depuis cette heure je ne boirai point de ce fruit de vigne, jusqu'au jour que je le boirai de nouveau avec vous, dans le Royaume de mon Père.
\TextTitle{Jésus informe Pierre de son triple reniement\FTNTT{Mc. 14:26-31 ; Lu. 22:31-34 ; Jn. 13:36-38}}
\VS{30}Quand ils eurent chanté le cantique\FTNT{Les cantiques : Du grec « humneo », chants d'hymnes pascals. Il s'agit plus précisément des psaumes 113 à 118 et du psaume 136, que les Juifs appellent le « grand Hallel ». Le Hallel consiste en six psaumes (113 à 118). Cet ensemble de textes est généralement entonné à haute voix par toute la communauté de prière lors de l'office religieux du matin, à l'issue de la « Amidah » (prière récitée debout), à l'occasion de la Pâque (le premier soir), de la Pentecôte et des Tabernacles, ainsi que pour Hanoucca et Rosh Hodesh. Voir Mc. 14:26.}, ils se rendirent à la Montagne des Oliviers.
\VS{31}Alors Jésus leur dit : Vous serez tous cette nuit scandalisés à cause de moi ; car il est écrit : Je frapperai le Berger, et les brebis du troupeau seront dispersées\FTNT{Za. 13:7.}.
\VS{32}Mais, après que je serai ressuscité, je vous précéderai en Galilée.
\VS{33}Pierre, prenant la parole, lui dit : Quand même tous seraient scandalisés à cause de toi, je ne le serai jamais.
\VS{34}Jésus lui dit : En vérité je te dis, qu'en cette même nuit, avant que le coq ait chanté, tu me renieras trois fois.
\VS{35}Pierre lui répondit : Même s'il me fallait mourir avec toi, je ne te renierai pas. Et tous les disciples dirent la même chose.
\TextTitle{Jésus dans le jardin de Gethsémané\FTNTT{Mc. 14:32-42 ; Lu. 22:39-46 ; Jn. 18:1}}
\VS{36}Alors Jésus alla avec eux dans un lieu appelé Gethsémané et il dit à ses disciples : Asseyez-vous ici, jusqu'à ce que j'aie prié dans le lieu où je vais.
\VS{37}Il prit avec lui Pierre et les deux fils de Zébédée, et il commença à être attristé et fort angoissé.
\VS{38}Alors il leur dit : Mon âme est de toutes parts saisie de tristesse jusqu'à la mort ; demeurez ici et veillez avec moi.
\TextTitle{Première prière de Jésus\FTNTT{Mc. 14:35-38 ; Lu. 22:41-42}}
\VS{39}Puis, ayant fait quelques pas en avant, il se prosterna le visage contre terre, priant et disant : Mon Père, s'il est possible, fais que cette coupe passe loin de moi ; toutefois non point comme je le veux, mais comme tu le veux.
\TextTitle{Jésus trouve les disciples endormis\FTNTT{Mc. 14:37-40 ; Lu. 22:45-46}}
\VS{40}Puis il vint vers ses disciples, qu'il trouva endormis, et il dit à Pierre : Vous n'avez pas pu veiller une heure avec moi ?
\VS{41}Veillez et priez, afin que vous ne tombiez pas en tentation : car l'esprit est prompt, mais la chair est faible.
\TextTitle{Deuxième prière\FTNTT{Mc. 14:39 ; Lu. 22:44}}
\VS{42}Il s'éloigna encore pour la seconde fois, et il pria, disant : Mon Père, s'il n'est pas possible que cette coupe s'éloigne sans que je la boive, que ta volonté soit faite.
\VS{43}Il revint ensuite et les trouva encore endormis ; car leurs yeux étaient appesantis.
\TextTitle{Troisième prière\FTNTT{Mc. 14:41}}
\VS{44}Et les ayant laissés, il s'en alla encore, et pria pour la troisième fois, disant les mêmes paroles.
\VS{45}Puis il alla vers ses disciples et leur dit : Dormez maintenant et reposez-vous ; voici, l'heure est proche, et le Fils de l'homme va être livré entre les mains des méchants.
\VS{46}Levez-vous, allons. Voici, celui qui me trahit s'approche.
\TextTitle{Jésus trahi et arrêté\FTNTT{Mc. 14:43-50 ; Lu. 22:47-53 ; Jn. 18:2-11}}
\VS{47}Comme il parlait encore, voici, Judas, l'un des douze, vint, et avec lui une grande foule, avec des épées et des bâtons, envoyée par les principaux sacrificateurs et par les anciens du peuple.
\VS{48}Celui qui le trahissait leur avait donné ce signe : Celui à qui je donnerai un baiser, c'est lui, saisissez-le.
\VS{49}Aussitôt, s'approchant de Jésus, il lui dit : Maître, je te salue ; et il le baisa.
\VS{50}Et Jésus lui dit : Mon ami, pour quel sujet es-tu ici ? Alors s'étant approchés, ils mirent les mains sur Jésus et le saisirent.
\VS{51}Et voici, l'un de ceux qui étaient avec Jésus, étendit la main et tira son épée ; il frappa le serviteur du souverain sacrificateur, et lui emporta l'oreille.
\VS{52}Alors Jésus lui dit : Remets ton épée à sa place ; car tous ceux qui prendront l'épée, périront par l'épée.
\VS{53}Crois-tu que je ne puisse pas maintenant prier mon Père, qui me donnerait à l'instant plus de douze légions d'anges ?
\VS{54}Mais comment donc s'accompliraient les Ecritures qui disent qu'il faut que cela arrive ainsi ?
\VS{55}En ce même instant Jésus dit à la foule : Vous êtes venus avec des épées et des bâtons, comme après un brigand, pour me prendre ; j'étais tous les jours assis parmi vous, enseignant dans le temple, et vous ne m'avez pas saisi.
\VS{56}Mais tout ceci est arrivé afin que les Ecritures des prophètes soient accomplies. Alors tous les disciples l'abandonnèrent et s'enfuirent.
\TextTitle{Jésus devant Caïphe et le sanhédrin\FTNTT{Mc. 14:53-65 ; Jn. 18:12-14, 19-24.}}
\VS{57}Ceux qui avaient saisi Jésus l'amenèrent chez Caïphe, le souverain sacrificateur, où les scribes et les anciens étaient assemblés.
\VS{58}Pierre le suivit de loin, jusqu'à la cour du souverain sacrificateur, y entra et s'assit avec les officiers pour voir comment cela finirait.
\VS{59}Les principaux sacrificateurs, les anciens et tout le sanhédrin cherchaient des faux témoignages contre Jésus pour le faire mourir.
\VS{60}Mais ils n'en trouvèrent point, et bien que plusieurs faux témoins se soient présentés, ils n'en trouvèrent point de propres ; mais à la fin, deux faux témoins s'approchèrent
\VS{61}et dirent : Celui-ci a dit : Je puis détruire le temple de Dieu et le rebâtir en trois jours.
\VS{62}Alors le souverain sacrificateur se leva et lui dit : Ne réponds-tu rien ? Qu'est-ce que ces hommes déposent contre toi ?
\VS{63}Jésus garda le silence. Et le souverain sacrificateur prenant la parole, lui dit : Je te somme par le Dieu vivant, de nous dire si tu es le Christ, le Fils de Dieu.
\VS{64}Jésus lui dit : Tu l'as dit. De plus, je vous dis que désormais vous verrez le Fils de l'homme assis à la droite de la puissance de Dieu et venant sur les nuées du ciel.
\VS{65}Alors le souverain sacrificateur déchira ses vêtements, en disant : Il a blasphémé ! Qu'avons-nous encore besoin de témoins ? Voici, vous avez entendu maintenant son blasphème. Que vous en semble ?
\VS{66}Ils répondirent : Il est digne de mort.
\VS{67}Alors ils lui crachèrent au visage, et lui donnèrent des coups de poing et des soufflets, et les autres le frappaient avec leurs bâtons ;
\VS{68}en disant : Christ, prophétise-nous qui est celui qui t'a frappé.
\TextTitle{Le triple reniement de Pierre\FTNTT{Mc. 14:66-72 ; Lu. 22:55-62 ; Jn. 18:15-18, 25-27.}}
\VS{69}Or Pierre était assis dehors dans la cour. Une servante s'approcha de lui et lui dit : Toi aussi, tu étais aussi avec Jésus le Galiléen.
\VS{70}Mais il le nia devant tous, en disant : Je ne sais pas ce que tu dis.
\VS{71}Et comme il était sorti dans le vestibule, une autre servante le vit et elle dit à ceux qui étaient là : Celui-ci aussi était avec Jésus de Nazareth.
\VS{72}Et il le nia encore avec serment, disant : Je ne connais pas cet homme.
\VS{73}Peu après, ceux qui se trouvaient là s'approchèrent et dirent à Pierre : Certainement tu es aussi de ces gens-là, car ton langage te fait connaître.
\VS{74}Alors il commença à faire des imprécations et à jurer, en disant : Je ne connais pas cet homme. Et aussitôt le coq chanta.
\VS{75}Et Pierre se souvint de la parole de Jésus, qui lui avait dit : Avant que le coq chante, tu me renieras trois fois. Et étant sorti dehors, il pleura amèrement.
\Chap{27}
\TextTitle{Jésus devant le gouverneur Pilate ; suicide de Judas\FTNTT{Ac. 1:16-19.}}
\VerseOne{}Puis quand le matin fut venu, tous les principaux sacrificateurs et les anciens du peuple tinrent conseil contre Jésus pour le faire mourir.
\VS{2}Après l'avoir lié, ils l'amenèrent et le livrèrent à Ponce Pilate, qui était le gouverneur.
\VS{3}Alors Judas qui l'avait trahi, voyant qu'il était condamné, se repentit et rapporta les trente pièces d'argent aux principaux sacrificateurs et aux anciens,
\VS{4}en leur disant : J'ai péché en trahissant le sang innocent ; mais ils lui dirent : Que nous importe ? Cela te regarde.
\VS{5}Et après avoir jeté les pièces d'argent dans le temple, il se retira et alla se pendre.
\VS{6}Mais les principaux sacrificateurs prirent les pièces d'argent et dirent : Il n'est pas permis de les mettre dans le trésor ; car c'est le prix du sang.
\VS{7}Et, après en avoir délibéré, ils achetèrent avec cet argent le champ d'un potier pour la sépulture des étrangers.
\VS{8}C'est pourquoi ce champ-là a été appelé jusqu'à aujourd'hui, le champ du sang.
\VS{9}Alors s'accomplit ce qui avait été annoncé par Jérémie le prophète, disant : Ils ont pris les trente pièces d'argent, le prix de celui qui a été estimé, qu'on a estimé de la part des enfants d'Israël ;
\VS{10}et ils les ont données pour acheter le champ d'un potier, selon ce que le Seigneur m'avait ordonné\FTNT{Ce verset se réfère certainement à Za. 11:12-13, avec une allusion à Jé. 18:1-4.}.
\VS{11}Jésus comparut devant le gouverneur. Le gouverneur l'interrogea : Es-tu le Roi des Juifs ? Jésus lui répondit : Tu le dis.
\VS{12}Mais il ne répondit rien aux accusations des principaux sacrificateurs et des anciens.
\VS{13}Alors Pilate lui dit : N'entends-tu pas de combien de choses ils t'accusent ?
\VS{14}Mais il ne lui donna de réponse sur aucune parole, ce qui étonna beaucoup le gouverneur.
\TextTitle{Jésus ou Barabbas ?\FTNTT{Mc. 15:6-15 ; Lu. 23:17-25 ; Jn. 18:39-40}}
\VS{15}Or le gouverneur avait coutume de relâcher un prisonnier à chaque fête, celui que demandait la foule.
\VS{16}Et il y avait alors un prisonnier fameux, nommé Barabbas.
\VS{17}Comme ils étaient assemblés, Pilate leur dit : Lequel voulez-vous que je vous relâche ? Barabbas ou Jésus qu'on appelle Christ ?
\VS{18}Car il savait bien qu'ils l'avaient livré par envie.
\VS{19}Et pendant qu'il siégeait au tribunal, sa femme envoya lui dire : Ne te mêle point de l'affaire de ce juste, car j'ai beaucoup souffert aujourd'hui en songe à cause de lui.
\VS{20}Les principaux sacrificateurs et les anciens persuadèrent la multitude du peuple de demander Barabbas et de faire périr Jésus.
\VS{21}Et le gouverneur prenant la parole leur dit : Lequel des deux voulez-vous que je vous relâche ? Ils dirent : Barabbas.
\VS{22}Pilate leur dit : Que ferai-je donc de Jésus qu'on appelle Christ ? Ils lui dirent tous : Qu'il soit crucifié !
\VS{23}Et le gouverneur leur dit : Mais quel mal a-t-il fait ? Et ils crièrent encore plus fort, en disant : Qu'il soit crucifié !
\VS{24}Alors Pilate voyant qu'il ne gagnait rien, mais que le tumulte s'augmentait, prit de l'eau et lava ses mains devant le peuple, en disant : Je suis innocent du sang de ce juste. Cela vous regarde.
\VS{25}Et tout le peuple répondit : Que son sang retombe sur nous et sur nos enfants !
\VS{26}Alors il leur relâcha Barabbas ; et après avoir fait fouetter Jésus, il le leur livra pour être crucifié.
\TextTitle{Le Roi couronné d'épines\FTNTT{Mc. 15:16-23 ; Lu. 23:26-32 ; Jn. 19:16-17}}
\VS{27}Les soldats du gouverneur amenèrent Jésus dans le prétoire et assemblèrent devant lui toute la cohorte.
\VS{28}Et après l'avoir dépouillé, ils le revêtirent d'un manteau d'écarlate.
\VS{29}Puis, ayant fait une couronne d'épines entrelacées, ils la mirent sur sa tête et ils lui mirent un roseau dans sa main droite ; puis s'agenouillant devant lui, ils se moquaient de lui, en disant : Nous te saluons, Roi des Juifs !
\VS{30}Et ils crachaient contre lui, prenaient le roseau et frappaient sur sa tête.
\VS{31}Après s'être ainsi moqués de lui, ils lui ôtèrent le manteau, et lui remirent ses vêtements, et l'amenèrent pour le crucifier.
\VS{32}Comme ils sortaient, ils rencontrèrent un homme de Cyrène, appelé Simon et ils le forcèrent à porter la croix de Jésus.
\TextTitle{La crucifixion de Jésus\FTNTT{Mc. 15:24-32 ; Lu. 23:33-43 ; Jn. 19:17-24}}
\VS{33}Et étant arrivés au lieu appelé Golgotha, c'est-à-dire le lieu du crâne,
\VS{34}ils lui donnèrent à boire du vinaigre mêlé avec du fiel\FTNT{Le vinaigre mêlé au fiel (Ps. 69:22) : Ce breuvage, appelé « posca », était un vin amer composé qui se transformait en vinaigre à cause des mauvaises conditions de conservation. Allongé avec de l'eau et parfois adoucie avec de l'œuf, cette boisson bon marché et très rafraîchissante était consommée principalement par les légionnaires et les esclaves. Connue pour ses vertus antiseptiques, les soldats de l'Antiquité avaient coutume d'y ajouter des drogues comme la myrrhe et le fiel (opium) pour atténuer les souffrances. En refusant de le boire, le Seigneur Jésus-Christ a réellement pris sur lui la plénitude du châtiment que nous méritons à cause de nos péchés.} ; mais quand il l'eut goûté, il ne voulut pas boire.
\VS{35}Et après l'avoir crucifié, ils partagèrent ses vêtements, en tirant au sort, afin que s'accomplît ce qui avait été annoncé par le prophète : Ils se sont partagés mes vêtements, et ont jeté ma tunique au sort\FTNT{Ps. 22:19.}.
\VS{36}Puis s'étant assis, ils le gardaient là.
\VS{37}Ils mirent aussi au-dessus de sa tête un écriteau, où la cause de sa condamnation était marquée en ces mots : CELUI-CI EST JESUS, LE ROI DES JUIFS.
\VS{38}Avec lui furent crucifiés deux brigands, l'un à sa droite et l'autre à sa gauche.
\VS{39}Et Ceux qui passaient par là, l'injuriaient et secouaient la tête
\VS{40}en disant : Toi qui détruis le temple et qui le rebâtis en trois jours, sauve-toi toi-même ! Si tu es le Fils de Dieu, descends de la croix !
\VS{41}Pareillement aussi, les principaux sacrificateurs avec les scribes et les anciens, se moquant, disaient :
\VS{42}Il a sauvé les autres et il ne peut pas se sauver lui-même ! S'il est le Roi d'Israël, qu'il descende maintenant de la croix et nous croirons en lui.
\VS{43}Il se confie en Dieu ; mais si Dieu l'aime, qu'il le délivre maintenant, car il a dit : Je suis le Fils de Dieu.
\VS{44}Les brigands aussi qui étaient crucifiés avec lui, lui reprochaient la même chose.
\TextTitle{Jésus accomplit la loi par sa mort\FTNTT{Mc. 15:33-41 ; Lu. 23:44-49 ; Jn. 19:30-37 ; Hé. 9:3-8 ; 10:19-20}}
\VS{45}Depuis la sixième heure jusqu'à la neuvième, il y eut des ténèbres sur toute la terre.
\VS{46}Et vers la neuvième heure, Jésus s'écria d'une voix forte : Eli, Eli, lama sabachthani ? C'est-à-dire : Mon Dieu ! Mon Dieu ! Pourquoi m'as-tu abandonné ?
\VS{47}Quelques-uns de ceux qui étaient là présents, ayant entendu cela, disaient : Il appelle Elie.
\VS{48}Et aussitôt l'un d'entre eux courut prendre une éponge, qu'il remplit de vinaigre, et l'ayant fixée au bout d'un roseau, lui donna à boire.
\VS{49}Mais les autres disaient : Laisse, voyons si Elie viendra le sauver.
\VS{50}Alors Jésus, poussa de nouveau un grand cri, et rendit l'esprit.
\VS{51}Et voici, le voile du temple se déchira en deux, depuis le haut jusqu'en bas\FTNT{C'est ici que s'achève la Première Alliance. Cette dernière était relative à la loi de Moïse, c'est-à-dire aux ordonnances liées au culte, qui reposait sur le sacerdoce lévitique et les sacrifices d'animaux, et au sanctuaire terrestre, à savoir le temple de Jérusalem (Hé. 9:1). Le Seigneur ayant offert une fois pour toutes le sacrifice parfait, les exigences de la justice divine ont été pleinement satisfaites (Hé. 9:11-12 ; 25-26). Désormais, la Première Alliance n'a plus de raison d'être et peut donc disparaître (Hé. 8:13). Non seulement la déchirure du voile séparant le lieu saint du Saint des saints atteste la fin de la Première Alliance, mais invite aussi tout homme à s'approcher de Dieu en esprit, sans intermédiaires (Lévites, sacrificateurs, pasteurs, prophètes…) ni nécessité de se rendre dans un temple (Jn. 4:23). La Nouvelle Alliance est aussi un testament puisque Jésus-Christ, notre légataire, est passé par la mort (Hé. 9:16-18). Voir aussi commentaire en Ex. 19:5.} ; et la terre trembla, et les pierres se fendirent.
\TextTitle{Le voile déchiré : Fin de la loi mosaïque ou de la Première Alliance}
\VS{52}Et les sépulcres s'ouvrirent et plusieurs corps des saints qui étaient morts ressuscitèrent.
\VS{53}Et étant sortis des sépulcres après la résurrection de Jésus, ils entrèrent dans la ville sainte et se montrèrent à plusieurs.
\VS{54}Le centenier et ceux qui étaient avec lui pour garder Jésus, ayant vu le tremblement de terre et tout ce qui venait d'arriver, furent saisis d'une grande frayeur et dirent : Certainement cet homme était le Fils de Dieu.
\VS{55}Il y avait là aussi plusieurs femmes qui regardaient de loin, et qui avaient suivi Jésus depuis la Galilée, pour le servir.
\VS{56}Entre lesquelles étaient Marie de Magdala, Marie mère de Jacques et de Joseph, et la mère des fils de Zébédée.
\TextTitle{Jésus enseveli\FTNTT{Mc. 15:42-47 ; Lu. 23:50-56 ; Jn. 19:38-42}}
\VS{57}Le soir étant venu, un homme riche d'Arimathée, appelé Joseph, qui était aussi disciple de Jésus,
\VS{58}se rendit vers Pilate et demanda le corps de Jésus. En même temps Pilate ordonna que le corps soit rendu.
\VS{59}Joseph prit le corps et l'enveloppa d'un linceul pur ;
\VS{60}et le mit dans un sépulcre neuf, qu'il s'était fait tailler dans le roc. Puis il roula une grande pierre à l'entrée du sépulcre et il s'en alla.
\VS{61}Marie de Magdala et l'autre Marie étaient là, assises vis-à-vis du sépulcre.
\TextTitle{Le sépulcre scellé et gardé}
\VS{62}Le lendemain, qui était le jour de la préparation du sabbat, les principaux sacrificateurs et les pharisiens allèrent ensemble auprès de Pilate,
\VS{63}et lui dirent : Seigneur ! Nous nous souvenons que ce séducteur disait, quand il était encore en vie : Après trois jours je ressusciterai.
\VS{64}Ordonne donc que le sépulcre soit gardé sûrement jusqu'au troisième jour ; de peur que ses disciples ne viennent de nuit, et ne dérobent son corps, et qu'ils ne disent au peuple : Il est ressuscité des morts. Cette dernière imposture serait pire que la première.
\VS{65}Pilate leur dit : Vous avez une garde ; allez et faites-le garder comme vous l'entendez.
\VS{66}Ils s'en allèrent donc, et s'assurèrent du sépulcre, au moyen d'une garde, après avoir scellé la pierre.
\Chap{28}
\TextTitle{Résurrection et apparition de Jésus-Christ\FTNTT{Mc. 16:1-14 ; Lu. 24:1-49 ; Jn. 20:1-23}}
\VerseOne{}Après le sabbat, à l'aube du premier jour de la semaine, Marie de Magdala et l'autre Marie allèrent voir le sépulcre.
\VS{2}Et voici, il eut un grand tremblement de terre ; car un ange du Seigneur descendit du ciel, vint rouler la pierre à côté de l'entrée du sépulcre et s'assit dessus.
\VS{3}Son visage était comme un éclair, et son vêtement blanc comme de la neige.
\VS{4}Les gardes furent tellement saisis de frayeur, qu'ils devinrent comme morts.
\VS{5}Mais l'ange prit la parole et dit aux femmes : Pour vous, ne craignez pas ; car je sais que vous cherchez Jésus, qui a été crucifié.
\VS{6}Il n'est point ici car il est ressuscité comme il l'avait dit. Venez et voyez le lieu où le Seigneur était couché,
\VS{7}et allez-vous-en promptement, et dites à ses disciples qu'il est ressuscité des morts. Et voici, il vous précède en Galilée ; c'est là que vous le verrez. Voici, je vous l'ai dit.
\VS{8}Alors elles sortirent promptement du sépulcre avec crainte et grande joie ; et coururent l'annoncer à ses disciples.
\VS{9}Mais comme elles allaient pour l'annoncer à ses disciples, voici, Jésus se présenta devant elles et leur dit : Je vous salue. Et elles s'approchèrent, embrassèrent ses pieds et l'adorèrent.
\VS{10}Alors Jésus leur dit : Ne craignez point. Allez et dites à mes frères d'aller en Galilée, c'est là qu'ils me verront.
\TextTitle{Les soldats soudoyés par les sacrificateurs}
\VS{11} Or quand elles furent parties, voici, quelques-uns de la garde vinrent dans la ville et ils rapportèrent aux principaux sacrificateurs toutes les choses qui étaient arrivées.
\VS{12}Sur quoi les sacrificateurs s'assemblèrent avec les anciens, et après avoir tenu conseil, donnèrent une forte somme d'argent aux soldats,
\VS{13}en leur disant : Dites : Ses disciples sont venus de nuit le dérober, pendant que nous dormions.
\VS{14}Et si le gouverneur l'apprend, nous l'apaiserons et nous vous tirerons de peine.
\VS{15}Les soldats prirent l'argent et suivirent les instructions qui leur furent données. Et ce bruit s'est répandu parmi les juifs, jusqu'à aujourd'hui.
\TextTitle{Mission des apôtres\FTNTT{Mc. 16:15-18 ; Lu. 24:46-48 ; Jn. 17:18 ; 20:21 ; Ac. 1:8 ; 1 Co. 15:6}}
\VS{16}Mais les onze disciples allèrent en Galilée, sur la montagne, où Jésus leur avait ordonné de se rendre.
\VS{17}Quand ils le virent, ils l'adorèrent, mais quelques-uns doutèrent.
\VS{18}Jésus s'étant approché, leur parla, en disant : Tout puissance m'a été donnée dans le ciel et sur la terre.
\VS{19}Allez donc et enseignez toutes les nations, les baptisant au Nom du Père, du Fils et du Saint-Esprit ;
\VS{20}les enseignant à garder toutes les choses que je vous ai commandées ; et voici, je suis avec vous toujours jusqu'à la fin du monde. Amen.
\PPE{}
\end{multicols}

\clearpage\ShortTitle{Marc}\BookTitle{Marc}\BFont
\noindent\hrulefill
\textit{
\bigskip
{\centering{}
\\Signifie : Qui brille, luisant
\\Thème : Jésus le serviteur
\\Auteur : Marc
\\Date de rédaction : Env. 68 apr. J.-C.\\}
}
%\bigskip
\textit{
\\Originaire de Jérusalem, Marc, aussi appelé Jean, fut l’auteur de l’évangile du même nom. Cousin de Barnabas et collaborateur de Paul, ce dernier l’éconduit lors d’un voyage car Marc l’avait abandonné lors d’une précédente mission. Ce fut d’ailleurs la cause de la séparation entre Barnabas et Paul.  Par la suite, il renoua le contact avec Paul et devint un de ses fidèles compagnons de ministère. Lié à l’apôtre Pierre tel un fils, ce fut probablement sous son autorité qu’il écrivit. En effet, l’évangile de Marc expose le témoignage de Pierre sur Christ.
\bigskip
\\Adressé aux gentils, cet évangile contient peu de références à l’ancienne alliance ; on y découvre Jésus l’inlassable serviteur de Dieu et des hommes. Marc y exposa la richesse de ses bonnes œuvres, son incomparable dévouement et révéla les sentiments intimes du maître. Même si Marc présenta principalement Jésus en tant que serviteur, son récit des miracles met en exergue toute la puissance du Christ.\bigskip
}
\par\nobreak\noindent\hrulefill
\begin{multicols}{2}
\TextTitle{[Ministère de Jean-Baptiste]
\\(Mt. 3:1-12 ; Lu. 3:1-20 ; Jn. 1:6-8,15-37)}
\Chap{1}
\VerseOne{}Commencement de l'Evangile de Jésus-Christ, Fils de Dieu ;
\VS{2}selon qu'il est écrit dans les prophètes : Voici, j'envoie mon messager devant ta face, lequel préparera ta voie devant toi.
\VS{3}C’est la voix de celui qui crie dans le désert : Préparez le chemin du Seigneur, aplanissez ses sentiers{\FTNT{Es. 40:3 ; Mal. 3:1.}}.
\VS{4}Jean baptisait dans le désert, et prêchait le baptême de repentance, pour obtenir la rémission des péchés.
\VS{5}Et tout le pays de Judée, et les habitants de Jérusalem allaient vers lui, et confessant leurs péchés, ils se faisaient tous baptiser par lui dans le fleuve du Jourdain.
\VS{6}Jean était vêtu de poils de chameau, il avait une ceinture de cuir autour de ses reins, et mangeait des sauterelles et du miel sauvage.
\VS{7}Et il prêchait, en disant : Il vient après moi, celui qui est plus puissant que moi, et je ne suis pas digne de délier en me baissant la courroie de ses souliers.
\VS{8}Moi, je vous ai baptisés d'eau ; mais lui, il vous baptisera du Saint-Esprit.
\TextTitle{[Baptême de Jésus-Christ]
\\(Mt. 3:13-17 ; Lu. 3:21-22 ; Jn. 1:31-34)}
\VS{9}En ce temps-là, Jésus vint de Nazareth, ville de Galilée, et il fut baptisé par Jean dans le Jourdain.
\VS{10}Au moment où il sortait de l'eau, Jean vit les cieux s’ouvrir, et le Saint-Esprit descendre sur lui comme une colombe.
\VS{11}Et une voix fit entendre des cieux ces paroles : Tu es mon Fils bien-aimé, en qui j'ai mis toute mon affection.
\TextTitle{[La tentation]
\\(Mt. 4:1-11 ; Lu. 4:1-13)}
\VS{12}Aussitôt l'Esprit le poussa à se rendre dans un désert,
\VS{13}où il passa quarante jours, tenté par Satan. Il était avec les bêtes sauvages, et les anges le servaient.
\TextTitle{[Jésus en Galilée]
\\(Mt. 4:12-17 ; Lu. 4:14-15)}
\VS{14}Après que Jean eut été mis en prison, Jésus alla dans la Galilée, prêchant l'Evangile du Royaume de Dieu.
\VS{15}Il disait : Le temps est accompli, et le Royaume de Dieu est proche. Repentez-vous, et croyez à 1'Evangile.
\TextTitle{[Appel de Simon (Pierre), André, Jacques et Jean]
\\(Lu. 5:1-11 ; Jn. 1:35-51)}
\VS{16}Comme il marchait près de la mer de Galilée, il vit Simon et André son frère, qui jetaient leurs filets dans la mer, car ils étaient pêcheurs.
\VS{17}Jésus leur dit : Suivez-moi, et je vous ferai pêcheurs d'hommes.
\VS{18}Aussitôt ils laissèrent leurs filets et ils le suivirent.
\VS{19}Etant allé un peu plus loin, il vit Jacques fils de Zébédée, et Jean son frère, qui raccommodaient leurs filets dans la barque.
\VS{20}Aussitôt, il les appela ; et laissant leur père Zébédée dans la barque avec les ouvriers, ils le suivirent.
\TextTitle{[Jésus chasse un démon dans la synagogue]
\\(Lu. 4:31-37)}
\VS{21}Ils entrèrent dans Capernaüm. Et le jour du sabbat, Jésus entra d’abord dans la synagogue, et il enseigna.
\VS{22}Ils étaient étonnés de sa doctrine ; car il les enseignait comme ayant autorité, et non pas comme les scribes.
\VS{23}Il se trouvait dans leur synagogue un homme qui avait un esprit impur, et qui s'écria,
\VS{24}en disant : Ha ! Qu’y a-t-il entre toi et nous, Jésus de Nazareth ? Es-tu venu pour nous perdre ? Je sais qui tu es : Tu es le Saint de Dieu.
\VS{25}Mais Jésus le menaça, disant : Tais-toi, et sors de cet homme.
\VS{26}Alors l'esprit impur sortit de cet homme, en l’agitant avec violence, et en poussant un grand cri.
\VS{27}Tous furent étonnés, de sorte qu'ils se demandaient les uns aux autres, et disaient : Qu'est-ce que ceci ? Quelle est cette nouvelle doctrine ? Il commande avec autorité même aux esprits impurs, et ils lui obéissent.
\VS{28}Et sa renommée se répandit aussitôt dans tout le pays des environs de la Galilée.
\TextTitle{[Jésus guérit la belle-mère de Pierre]
\\(Mt. 8:14-15 ; Lu. 4:38-39)}
\VS{29}En sortant de la synagogue, ils se rendirent avec Jacques et Jean à la maison de Simon et d'André.
\VS{30}La belle-mère de Simon était couchée, ayant la fièvre ; et aussitôt on parla d’elle à Jésus.
\VS{31}S’étant approché, il la fit lever en la prenant par la main ; et à l'instant la fièvre la quitta ; et elle les servit.
\TextTitle{[Jésus guérit les malades et chasse des démons ; prédications en Galilée]
\\(Mt. 8:16-17 ; Lu. 4:40-44)}
\VS{32}Le soir étant venu, comme le soleil se couchait, on lui amena tous les malades, et les démoniaques.
\VS{33}Et toute la ville était assemblée devant sa porte.
\VS{34}Il guérit beaucoup de malades qui avaient différentes maladies et chassa beaucoup de démons, et il ne permettait pas aux démons de parler, parce qu’ils le connaissaient.
\VS{35}Vers le matin, pendant qu’il faisait encore très sombre, il se leva, et sortit pour aller dans un lieu désert, où il pria.
\VS{36}Simon et ceux qui étaient avec lui se mirent à sa recherche,
\VS{37}et quand ils l’eurent trouvé, ils lui dirent : Tous te cherchent.
\VS{38}Et il leur dit : Allons aux bourgades voisines, afin que j'y prêche aussi ; car c’est pour cela que je suis venu.
\VS{39}Il prêchait donc dans leurs synagogues, par toute la Galilée, et chassait les démons.
\TextTitle{[Jésus guérit un lépreux]
\\(Mt. 8:2-4 ; Lu. 5:12-14)}
\VS{40}Un lépreux vint à lui, le priant et se mettant à genoux devant lui, et lui dit : Si tu veux, tu peux me rendre pur.
\VS{41}Jésus, ému de compassion, étendit sa main et le toucha, en lui disant : Je le veux, sois pur.
\VS{42}La lèpre quitta aussitôt cet homme, et il fut purifié.
\VS{43}Jésus le renvoya sur-le-champ, avec de sévères recommandations,
\VS{44}et lui dit : Garde-toi de ne rien dire à personne ; mais va te montrer au sacrificateur, et présente pour ta purification les choses que Moïse a commandées, pour leur servir de témoignage{\FTNT{Loi sur la purification de la lèpre~: Lé. 14:1-32. Avant sa mort et sa résurrection, Jésus-Christ observait la loi de Moïse (Mt. 23:1-2).}}.
\VS{45}Mais cet homme, s’en étant allé, commença à publier ouvertement la chose et à divulguer ce qui s'était passé ; de sorte que Jésus ne pouvait plus entrer publiquement dans la ville, mais il se tenait dehors, dans des lieux déserts, et l’on venait à lui de toutes parts.
\TextTitle{[Jésus guérit un paralytique]
\\(Mt. 9:2-8 ; Lu. 5:18-26)}
\Chap{2}
\VerseOne{}Quelques jours après, Jésus revint à Capernaüm. On apprit qu'il était à la maison,
\VS{2}et aussitôt il s’assembla un si grand nombre de personnes, que l'espace même devant la porte ne pouvait plus les contenir. Il leur annonçait la parole.
\VS{3}Et quelques-uns vinrent à lui, amenant un paralytique qui était porté par quatre personnes.
\VS{4}Comme ils ne pouvaient pas s’approcher de lui à cause de la foule, ils découvrirent le toit du lieu où il était, et l'ayant percé, ils descendirent le lit dans lequel le paralytique était couché.
\VS{5}Jésus, voyant leur foi, dit au paralytique : Mon enfant, tes péchés te sont pardonnés.
\VS{6}Et quelques scribes qui étaient assis là, raisonnaient ainsi en eux-mêmes :
\VS{7}Comment cet homme parle-t-il ainsi ? Il blasphème. Qui peut pardonner les péchés, si ce n’est Dieu seul ?
\VS{8}Jésus, ayant aussitôt connu par son esprit qu'ils raisonnaient ainsi en eux-mêmes, leur dit : Pourquoi avez-vous de telles pensées dans vos cœurs ?
\VS{9}Lequel est le plus aisé de dire au paralytique : Tes péchés te sont pardonnés, ou de dire : Lève-toi, prends ton lit, et marche ?
\VS{10}Mais afin que vous sachiez que le Fils de l'homme a le pouvoir sur la terre de pardonner les péchés, il dit au paralytique :
\VS{11}Je te dis : Lève-toi, prends ton lit, et va dans ta maison.
\VS{12}Et il se leva aussitôt, et ayant pris son lit, il sortit en présence de tous ; de sorte qu'ils furent tous étonnés, et ils glorifièrent Dieu, en disant : Nous n’avons jamais rien vu de pareil.
\TextTitle{[Appel de Matthieu]
\\(Mt. 9:9 ; Lu. 5:27-28)}
\VS{13}Jésus sortit de nouveau du côté de la mer, toute la foule venait à lui, et il les enseignait.
\VS{14}En passant, il vit Lévi, fils d'Alphée, assis au bureau des péages, et il lui dit : Suis-moi. Et Lévi s'étant levé, le suivit.
\TextTitle{[Jésus appelle des pêcheurs à la repentance, non des justes]
\\(Mt. 9:10-15 ; Lu. 5:29-35)}
\VS{15}Comme Jésus était à table dans la maison de Lévi, plusieurs publicains et des gens de mauvaise vie se mirent aussi à table avec lui et avec ses disciples ; car ils étaient nombreux, et l'avaient suivi.
\VS{16}Mais les scribes et les pharisiens voyant qu'il mangeait avec les publicains et les gens de mauvaise vie, disaient à ses disciples : Pourquoi mange-t-il et boit-il avec les publicains et les gens de mauvaise vie ?
\VS{17}Jésus ayant entendu cela, leur dit : Ce ne sont pas ceux qui se portent bien qui ont besoin de médecin, mais les malades. Je ne suis pas venu appeler à la repentance les justes, mais les pécheurs.
\TextTitle{[Les pharisiens et les disciples de Jean interrogent Jésus sur le jeûne]}
\VS{18}Les disciples de Jean et ceux des pharisiens jeûnaient ; ils vinrent à Jésus et lui dirent : Pourquoi les disciples de Jean, et ceux des pharisiens, jeûnent-ils, tandis que tes disciples ne jeûnent point ?
\VS{19}Jésus leur répondit : Les amis de l'Epoux peuvent-ils jeûner pendant que l'Epoux est avec eux ? Aussi longtemps qu’ils ont avec eux l'Epoux, ils ne peuvent jeûner.
\VS{20}Mais les jours viendront où l'Epoux leur sera ôté, alors ils jeûneront en ce jour-là.
\TextTitle{[Parabole du drap neuf et des outres neuves]
\\(Mt. 9:16-17 ; Lu. 5:36-39)}
\VS{21}Personne ne coud une pièce de drap neuf à un vieil habit ; autrement, la pièce du drap neuf emporterait une partie du vieux, et la déchirure serait pire.
\VS{22}Et personne ne met du vin nouveau dans de vieilles outres ; autrement, le vin nouveau fait rompre les outres, et le vin se répand, et les outres sont perdues ; mais le vin nouveau doit être mis dans des outres neuves.
\TextTitle{[Jésus, le Maître du sabbat]
\\(Mt. 12:1-8 ; Lu. 6:1-5)}
\VS{23}Il arriva, un jour de sabbat, que Jésus traversa des champs de blé. Ses disciples en marchant se mirent à arracher des épis.
\VS{24}Les pharisiens lui dirent : Regarde, pourquoi font-ils ce qui n'est pas permis les jours de sabbat ?
\VS{25}Mais il leur dit : N'avez-vous jamais lu ce que fit David quand il fut dans la nécessité, et qu'il eut faim, lui et ceux qui étaient avec lui ?
\VS{26}Comment il entra dans la maison de Dieu, au temps du souverain sacrificateur Abiathar, et mangea les pains de proposition{\FTNT{1 S. 21:1-7.}} , qu’il n'est permis qu'aux sacrificateurs de manger ; et il en donna même à ceux qui étaient avec lui!
\VS{27}Puis il leur dit : Le sabbat a été fait pour l'homme, et non pas l'homme pour le sabbat ;
\VS{28}de sorte que le Fils de l'homme est Maître même du sabbat.
\TextTitle{[Jésus-Christ guérit un homme à la main sèche le jour du sabbat]
\\(Mt. 12:9-13 ; Lu. 6:6-11)}
\Chap{3}
\VerseOne{}Jésus entra de nouveau dans la synagogue, et il y avait là un homme qui avait une main sèche.
\VS{2}Ils l'observaient, pour voir s'il le guérirait le jour du sabbat, afin de l'accuser.
\VS{3}Et Jésus dit à l'homme qui avait la main sèche : Lève-toi, et tiens-toi là au milieu.
\VS{4}Puis il leur dit : Est-il permis de faire du bien les jours de sabbat, ou de faire du mal, de sauver une personne, ou de la tuer ? Mais ils gardèrent le silence.
\VS{5}Alors, les regardant tous avec indignation, et étant affligé de l'endurcissement de leur cœur, il dit à cet homme : Etends ta main. Il l'étendit, et sa main fut rendue saine comme l'autre.
\TextTitle{[Nombreuses guérisons de Jésus]
\\(Mt. 12:15-16 ; Lu. 6:17-19}
\VS{6}Alors les pharisiens sortirent, et aussitôt, ils se consultèrent avec les hérodiens, sur les moyens de le faire périr.
\VS{7}Mais Jésus se retira vers la mer avec ses disciples. Une grande multitude le suivit de la Galilée,
\VS{8}de Judée, de Jérusalem, de l’Idumée, d’au-delà du Jourdain, et des environs de Tyr et de Sidon, une grande multitude, ayant entendu les grandes choses qu'il faisait, vint vers lui en grand nombre.
\VS{9}Et il dit à ses disciples de tenir toujours à sa disposition une petite barque, afin de ne pas être pressé par la foule.
\VS{10}Car, comme il guérissait beaucoup de gens, tous ceux qui avaient des maladies se jetaient sur lui pour le toucher.
\VS{11}Et les esprits impurs, quand ils le voyaient, se prosternaient devant lui, et s'écriaient en disant : Tu es le Fils de Dieu.
\VS{12}Mais il leur défendait avec de grandes menaces de le faire connaître.
\TextTitle{[L'appel des douze apôtres]
\\(Mt. 10:1-4 ; Lu. 6:13-16)}
\VS{13}Puis il monta sur une montagne, appela ceux qu'il voulut, et ils vinrent auprès de lui.
\VS{14}Il en établit douze pour être avec lui,
\VS{15}et pour les envoyer prêcher, avec la puissance de guérir les maladies, et de chasser les démons.
\VS{16}Voici les douze qu’il établit : Simon qu'il nomma Pierre ;
\VS{17}Jacques fils de Zébédée, et Jean, frère de Jacques, auxquels il donna le nom de Boanergès, ce qui veut dire fils de tonnerre.
\VS{18}André ; Philippe ; Barthélemy ; Matthieu ; Thomas ; Jacques, fils d'Alphée ; Thaddée ; Simon le Cananite ;
\VS{19}et Judas Iscariot, celui qui livra Jésus.
\VS{20}Ils se rendirent à la maison, et une grande multitude s’assembla de nouveau, en sorte qu’ils ne pouvaient même pas prendre leur repas.
\VS{21}Quand les parents de Jésus apprirent cela, ils sortirent pour se saisir de lui. Car ils disaient : Il est hors de sens.
\TextTitle{[Le blasphème contre le Saint-Esprit]
\\(Mt. 12:24-32 ; Lu. 11:15-23)}
\VS{22}Et les scribes, qui étaient descendus de Jérusalem, disaient : Il est possédé par Béelzébul ; c’est par le prince des démons qu’il chasse les démons.
\VS{23}Mais Jésus les appela, et leur dit sous forme de paraboles : Comment Satan peut-il chasser Satan ?
\VS{24}Si un royaume est divisé contre lui-même, ce royaume ne peut subsister ;
\VS{25}et si une maison est divisée contre elle-même, cette maison ne peut subsister.
\VS{26}Si donc Satan s'élève contre lui-même, il est divisé, il ne peut subsister, mais il tend vers sa fin.
\VS{27}Personne ne peut entrer dans la maison d'un homme fort et piller ses biens, sans avoir auparavant lié cet homme fort ; alors il pillera sa maison.
\VS{28}Je vous le dis en vérité, que toutes sortes de péchés seront pardonnés aux enfants des hommes, et aussi toutes sortes de blasphèmes par lesquels ils auront blasphémé ;
\VS{29}mais quiconque blasphémera contre le Saint-Esprit n’obtiendra jamais de pardon : Il est coupable et subira une condamnation éternelle {\FTNT{Voir commentaire Mt. 12:32.}}.
\VS{30}Jésus parla ainsi parce qu'ils disaient : Il est possédé d'un esprit impur.
\TextTitle{[La famille spirituelle]
\\(Mt. 12:46-50 ; Lu. 8:19-21)}
\VS{31}Survinrent ses frères et sa mère qui, se tenant dehors, l'envoyèrent appeler. La multitude était assise autour de lui,
\VS{32}et on lui dit : Voici, ta mère et tes frères sont dehors et te demandent.
\VS{33}Mais il leur répondit : Qui est ma mère, et qui sont mes frères ?
\VS{34}Et, jetant les regards sur ceux qui étaient assis tout autour de lui, il dit : Voici ma mère et mes frères.
\VS{35}Car quiconque fera la volonté de Dieu, celui-là est mon frère, ma sœur, et ma mère.
\TextTitle{[Parabole du semeur et des quatre terrains]
\\(Mt. 13:1-17 ; Lu. 8:4-10)}
\Chap{4}
\VerseOne{}Jésus se mit de nouveau à enseigner près de la mer, et une grande foule s’étant assemblée auprès de lui, il monta dans une barque et s’assit dans la barque, sur la mer. Toute la foule était à terre sur le rivage de la mer.
\VS{2}Il leur enseignait beaucoup de choses en paraboles, et il leur dit dans son enseignement :
\VS{3}Ecoutez. Un semeur sortit pour semer.
\VS{4}Comme il semait, une partie de la semence tomba le long du chemin, et les oiseaux du ciel vinrent, et la mangèrent toute.
\VS{5}Une autre partie tomba dans les endroits pierreux, où elle n'avait pas beaucoup de terre ; elle leva aussitôt, parce qu'elle n'entrait pas profondément dans la terre ;
\VS{6}mais, quand le soleil parut, elle fut brûlée, et parce qu'elle n'avait pas de racine, elle se sécha.
\VS{7}Une autre partie tomba parmi les épines ; et les épines montèrent, et l'étouffèrent, et elle ne donna pas de fruit.
\VS{8}Une autre partie tomba dans la bonne terre, et donna du fruit qui montait et croissait en sorte qu'un grain en rapporta trente, un autre soixante, et un autre cent.
\VS{9}Et il leur dit : Que celui qui a des oreilles pour entendre, qu'il entende !
\VS{10}Lorsqu’il fut à l’écart, ceux qui étaient autour de lui avec les douze, l'interrogèrent touchant cette parabole.
\VS{11}Et il leur dit : Il vous est donné de connaître le mystère du Royaume de Dieu ; mais pour ceux qui sont dehors, tout se passe en paraboles,
\VS{12}afin qu'en voyant ils voient et n'aperçoivent point, et qu'en entendant ils entendent et ne comprennent point, de peur qu'ils ne se convertissent, et que leurs péchés ne leur soient pardonnés.
\TextTitle{[Explication de la parabole]
\\(Mt. 13:18-23 ; Lu. 8:11-15)}
\VS{13}Puis il leur dit : Ne comprenez-vous pas cette parabole ? Et comment donc comprendrez-vous toutes les paraboles ?
\VS{14}Le semeur c'est celui qui sème la parole.
\VS{15}Ceux qui sont le long du chemin, ce sont ceux en qui la parole est semée. Quand ils l’ont entendue, aussitôt Satan vient et enlève la parole qui a été semée dans leurs cœurs.
\VS{16}De même, ceux qui reçoivent la semence dans les endroits pierreux, ce sont ceux qui entendent la parole, ils la reçoivent aussitôt avec joie ;
\VS{17}mais ils n'ont pas de racine en eux-mêmes, ils croient pour un temps, et dès que survient une tribulation ou une persécution à cause de la parole, ils y trouvent une occasion de chute.
\VS{18}D’autres reçoivent la semence parmi les épines ; ce sont ceux qui entendent la parole,
\VS{19}mais en qui les soucis de ce monde, et la séduction des richesses, et les convoitises des autres choses étant entrées dans leurs esprits, étouffent la parole, et elle devient infructueuse.
\VS{20}Mais ceux qui ont reçu la semence dans la bonne terre, ce sont ceux qui entendent la parole, la reçoivent, et portent du fruit : L'un trente, et l'autre soixante, et l'autre cent{\FTNT{Voir commentaire Mt. 13:8.}}.
\TextTitle{[Parabole de la lampe]
\\(Mt. 5:15-16 ; Lu. 8:16-18 ; 11:33-36)}
\VS{21}Il leur dit encore : Apporte-t-on la lampe pour la mettre sous un boisseau, ou sous un lit ? N'est-ce pas pour la mettre sur un chandelier ?
\VS{22}Car il n'y a rien de secret qui ne doive être découvert, rien de caché qui ne doive être mis à jour.
\VS{23}Si quelqu'un a des oreilles pour entendre, qu'il entende.
\VS{24}Il leur dit encore : Prenez garde à ce que vous entendez. On vous mesurera avec la mesure dont vous vous serez servis, et on y ajoutera pour vous.
\VS{25}Car on donnera à celui qui a ; mais à celui qui n’a pas, on ôtera même ce qu’il a.
\TextTitle{[Parabole de la semence et de la croissance spirituelle]}
\VS{26}Il dit encore : Il en est du Royaume de Dieu comme quand un homme jette la semence en terre ;
\VS{27}qu’il dorme ou qu’il veille, nuit et jour, la semence germe et croit, sans qu'il sache comment.
\VS{28}Car la terre produit d'elle-même, premièrement l'herbe, ensuite l'épi, et puis le grain formé dans l'épi ;
\VS{29}et quand le fruit est mûr, on y met aussitôt la faucille, parce que la moisson est prête.
\TextTitle{[Parabole du grain de moutarde]
\\(Mt. 13:31-33 ; Lu. 13:18-19)}
\VS{30}Il dit encore : A quoi comparerons-nous le Royaume de Dieu, ou par quelle parabole le représenterons-nous ?
\VS{31}Il en est comme du grain de moutarde, qui, lorsqu'on le sème dans la terre, est la plus petite de toutes les semences qui sont jetées dans la terre.
\VS{32}Mais après qu'il a été semé, il monte et devient plus grand que toutes les autres plantes, et pousse de grandes branches, en sorte que les oiseaux du ciel peuvent faire leurs nids sous son ombre.
\VS{33}C’est par beaucoup de paraboles de cette sorte qu’il leur annonçait la parole de Dieu, selon qu'ils pouvaient l'entendre.
\VS{34}Et il ne leur parlait point sans paraboles ; mais en particulier, il expliquait tout à ses disciples.
\TextTitle{[Jésus apaise la tempête]
\\(Mt. 8:23-27 ; Lu. 8:22-25)}
\VS{35}Ce même jour sur le soir, Jésus leur dit : Passons sur l’autre bord.
\VS{36}Après avoir renvoyé la foule, ils l'emmenèrent avec eux, dans la barque ; et il y avait aussi d'autres petites barques avec lui.
\VS{37}Et il se leva un grand tourbillon, et les flots se jetaient dans la barque, de sorte qu'elle se remplissait déjà.
\VS{38}Et lui, il dormait à la poupe sur un oreiller. Ils le réveillèrent, et lui dirent : Maître, ne t’inquiètes-tu pas de ce que nous périssions ?
\VS{39}S’étant réveillé, il menaça le vent, et dit à la mer : Silence ! Tais-toi ! Et le vent cessa, et il eut un grand calme.
\VS{40}Puis il leur dit : Pourquoi avez-vous si peur ? Comment n'avez-vous point de foi ?
\VS{41}Et ils furent saisis d'une grande crainte, et ils se dirent les uns les autres : Quel est donc celui-ci, à qui obéissent le vent et la mer ?
\TextTitle{[Jésus-Christ délivre un possédé à Gadara]
\\(Mt. 8:28-34 ; Lu. 8:26-40)}
\Chap{5}
\VerseOne{}Ils arrivèrent sur l’autre bord de la mer, dans le pays des Gadaréniens.
\VS{2}Aussitôt que Jésus fut descendu de la barque, un homme possédé d’un esprit impur, sortit des sépulcres, et vint le rencontrer.
\VS{3}Cet homme avait sa demeure dans les sépulcres, et personne ne pouvait plus le lier, pas même avec des chaînes.
\VS{4}Car souvent, il avait eu les fers aux pieds et avait été lié de chaînes, mais il avait rompu les chaînes et brisé les fers, et personne ne pouvait le dompter.
\VS{5}Il était continuellement, nuit et jour sur les montagnes, et dans les sépulcres, criant et se meurtrissant avec des pierres.
\VS{6}Ayant vu Jésus de loin, il courut et se prosterna devant lui.
\VS{7}Et s’écria d’une voix forte : Qu'y a-t-il entre toi et moi, Jésus, Fils du Dieu Très-Haut ? Je te conjure au Nom de Dieu de ne pas me tourmenter.
\VS{8}Car Jésus lui disait : Sors de cet homme, esprit impur.
\VS{9}Alors il lui demanda : Quel est ton nom ? Légion{\FTNT{Une légion romaine contenait entre trois et six mille soldats. C’est autant de démons dont l’homme était possédé.}} est mon nom, lui répondit-il, car nous sommes plusieurs.
\VS{10}Et il le priait instamment de ne pas les envoyer hors de cette contrée.
\VS{11}Il y avait là, vers les montagnes, un grand troupeau de pourceaux qui paissaient.
\VS{12}Et tous ces démons le priaient en disant : Envoie-nous dans les pourceaux, afin que nous entrions en eux ; et aussitôt Jésus le leur permit.
\VS{13}Alors ces esprits impurs étant sortis, entrèrent dans les pourceaux, qui était environ deux mille, et le troupeau se précipita des pentes escarpées dans la mer ; et ils se noyèrent dans la mer.
\VS{14}Ceux qui paissaient les pourceaux s'enfuirent, et répandirent la nouvelle dans la ville et dans les campagnes.
\VS{15}Ceux de la ville sortirent pour voir ce qui était arrivé. Ils vinrent à Jésus et ils virent le démoniaque, celui qui avait eu la légion, assis et vêtu, et dans son bon sens ; et ils furent saisis de crainte.
\VS{16}Et ceux qui avaient vu le miracle leur racontèrent ce qui était arrivé au démoniaque et aux pourceaux.
\VS{17}Alors ils se mirent à supplier Jésus de quitter leur territoire.
\VS{18}Comme il montait dans la barque, celui qui avait été démoniaque le pria de lui permettre de rester avec lui.
\VS{19}Mais Jésus ne le lui permit pas, mais il lui dit : Va dans ta maison, vers les tiens, et raconte-leur les grandes choses que le Seigneur t'a faites, et comment il a eu pitié de toi.
\VS{20}Il s'en alla donc, et se mit à publier dans la Décapole les grandes choses que Jésus lui avait faites. Et tous furent dans l’étonnement.
\TextTitle{[La résurrection de la fille de Jaïrus et la guérison de la femme atteinte d'une perte de sang]
\\(Mt. 9:18-26 ; Lu. 8:41-56)}
\VS{21}Jésus dans la barque regagna l’autre rive, où une grande foule s’assembla près de lui. Il était près de la mer.
\VS{22}Alors vint un des chefs de la synagogue, nommé Jaïrus, qui l’ayant aperçu, se jeta à ses pieds,
\VS{23}et le pria instamment, en disant : Ma petite fille est à l'extrémité. Je te prie de venir et de lui imposer les mains, afin qu'elle soit guérie et qu'elle vive.
\VS{24}Jésus s'en alla donc avec lui. Et de grandes foules de gens le suivaient et le pressaient.
\VS{25}Or, il y avait une femme qui avait une perte de sang depuis douze ans,
\VS{26}et qui avait beaucoup souffert entre les mains de plusieurs médecins. Elle avait dépensé tout ce qu’elle possédait, sans avoir éprouvé aucun soulagement, mais était allée plutôt en empirant.
\VS{27}Ayant entendu parler de Jésus, elle vint dans la foule par derrière et toucha son vêtement.
\VS{28}Car elle disait : Si je puis seulement toucher ses vêtements, je serai guérie.
\VS{29}Au même instant, la perte de sang s'arrêta ; et elle sentit en son corps qu'elle était guérie de son fléau.
\VS{30}Et aussitôt Jésus connut en lui-même qu’une force était sortie de lui, et, se retournant vers la foule, il dit : Qui a touché mes vêtements ?
\VS{31}Et ses disciples lui dirent : Tu vois que la foule te presse, et tu dis : Qui m'a touché ?
\VS{32}Mais il regardait tout autour pour voir celle qui avait fait cela.
\VS{33}Alors la femme saisie de crainte et toute tremblante, sachant ce qui s’était passé en elle, vint et se jeta à ses pieds, et lui déclara toute la vérité.
\VS{34}Mais Jésus lui dit : Ma fille ! Ta foi t'a sauvée. Va en paix, et sois guérie de ton fléau.
\VS{35}Comme il parlait encore, il vint des gens de chez le chef de la synagogue, qui lui dirent : Ta fille est morte, pourquoi importuner davantage le Maître ?
\VS{36}Mais aussitôt que Jésus eut entendu cela, il dit au chef de la synagogue : Ne crains pas, crois seulement.
\VS{37}Et il ne permit à personne de le suivre, si ce n’est à Pierre, à Jacques, et à Jean, frère de Jacques.
\VS{38}Ils arrivèrent à la maison du chef de la synagogue, où Jésus vit le tumulte, c'est-à-dire ceux qui pleuraient et qui poussaient de grands cris.
\VS{39}Il entra, et leur dit : Pourquoi faites-vous tout ce bruit, et pourquoi pleurez-vous ? L’enfant n'est pas morte, mais elle dort.
\VS{40}Et ils se moquèrent de lui. Mais Jésus les ayant tous fait sortir, prit le père et la mère de la petite fille, et ceux qui étaient avec lui, et entra là où la petite fille était couchée.
\VS{41}Il la saisit par la main, et lui dit : Talitha koumi, ce qui signifie : Jeune fille, je te dis lève-toi.
\VS{42}Aussitôt la petite fille se leva, et se mit à marcher ; car elle était âgée de douze ans. Et ils furent dans un grand étonnement.
\VS{43}Jésus leur recommanda fort expressément que personne ne le sache ; et il dit qu'on donne à manger à la jeune fille.
\TextTitle{[Jésus à Nazareth]}
\Chap{6}
\VerseOne{}Jésus partit de là, et se rendit dans sa patrie. Ses disciples le suivirent.
\VS{2}Quand le jour du sabbat fut venu, il se mit à enseigner dans la synagogue. Et beaucoup de ceux qui l'entendaient étaient dans l'étonnement, et ils disaient : D'où lui viennent ces choses ? Et quelle est cette sagesse qui lui a été donnée, et comment de tels prodiges se font-ils par ses mains ?
\VS{3}N’est-ce pas le charpentier, le Fils de Marie, frère de Jacques, de Joses, de Jude, et de Simon ? Et ses sœurs ne sont-elles pas ici parmi nous ? Et ils étaient scandalisés à cause de lui.
\VS{4}Mais Jésus leur dit : Un prophète n'est méprisé que dans sa patrie, parmi ses parents et dans sa famille.
\VS{5}Et il ne put faire là aucun miracle, si ce n’est qu'il guérit quelques malades en leur imposant les mains.
\VS{6}Et il s'étonnait de leur incrédulité. Jésus parcourait les villages d'alentour, en enseignant.
\TextTitle{[Mission des apôtres]
\\(Mt. 10:1-42 ; Lu. 9:1-6)}
\VS{7}Alors il appela les douze, et commença à les envoyer deux à deux, en leur donnant pouvoir sur les esprits impurs.
\VS{8}Il leur prescrit de ne rien prendre pour le chemin, si ce n’est un bâton, et de ne porter ni sac, ni pain, ni monnaie dans leur ceinture ;
\VS{9}de chausser des sandales, et de ne pas porter deux tuniques.
\VS{10}Il leur disait aussi : Dans quelque maison que vous entriez, demeurez-y jusqu'à ce que vous partiez de là.
\VS{11}Et tous ceux qui ne vous recevront pas, et ne vous écouteront pas, en partant de là, secouez la poussière de vos pieds, en témoignage contre eux. Je vous le dis en vérité que ceux de Sodome et de Gomorrhe seront traités moins rigoureusement au jour du jugement que cette ville-là.
\VS{12}Ils partirent, et ils prêchèrent la repentance.
\VS{13}Ils chassèrent beaucoup de démons hors des possédés, et ils oignirent d'huile beaucoup de malades et les guérirent.
\TextTitle{[Jean-Baptiste décapité]
\\(Mt. 14:1-14 ; Lu. 9:7-9)}
\VS{14}Le roi Hérode entendit parler de Jésus, dont le nom était devenu fort célèbre, et il dit : C’est Jean-Baptiste qui est ressuscité des morts ; c'est pourquoi la puissance de faire des miracles agit puissamment en lui.
\VS{15}D’autres disaient : C'est Elie. Et les autres disaient : C'est un prophète, comme l’un des prophètes.
\VS{16}Mais Hérode en apprenant cela, disait : C'est Jean que j'ai fait décapiter, il est ressuscité des morts.
\VS{17}Car Hérode avait fait arrêter Jean, et l'avait fait lier en prison, à cause d'Hérodias, femme de Philippe son frère, parce qu'il l'avait prise en mariage.
\VS{18}Et que Jean lui disait : Il ne t'est pas permis d'avoir la femme de ton frère.
\VS{19}C'est pourquoi Hérodias était irritée contre Jean, et voulait le faire mourir, mais elle ne le pouvait pas ;
\VS{20}parce qu’Hérode craignait Jean, sachant que c'était un homme juste et saint ; il le protégeait, et, après l’avoir entendu, il faisait beaucoup selon ses avis, et l’écoutait avec plaisir.
\VS{21}Cependant, un jour propice arriva, lorsque Hérode à l’occasion du jour de sa naissance, donna un festin aux grands de sa cour, aux chefs militaires et aux principaux de la Galilée.
\VS{22}La fille d'Hérodias entra dans la salle ; elle dansa et plut à Hérode, et à ceux qui étaient à table avec lui. Le roi dit à la jeune fille : Demande-moi ce que tu voudras, et je te le donnerai.
\VS{23}Il ajouta avec serment : Tout ce que tu me demanderas, je te le donnerai, serait-ce la moitié de mon royaume.
\VS{24}Etant sortie, elle dit à sa mère : Que demanderai-je ? Et sa mère lui dit : La tête de Jean-Baptiste.
\VS{25}Et étant revenue en toute hâte vers le roi, et lui fit cette demande : Je veux que tu me donnes à l’instant sur un plat, la tête de Jean-Baptiste.
\VS{26}Le roi fut attristé, mais à cause de son serment et des convives, il ne voulut pas refuser.
\VS{27}Il envoya sur-le-champ l’un de ses gardes, avec ordre d'apporter la tête de Jean.
\VS{28}Le garde alla décapiter Jean dans la prison, et apporta sa tête sur un plat, et la donna à la jeune fille. Et la jeune fille la donna à sa mère.
\VS{29}Les disciples de Jean ayant appris cela, vinrent et emportèrent son corps, et le mirent dans un sépulcre.
\TextTitle{[Les apôtres rendent compte de leur mission à Jésus]
\\(Lu. 9:10)}
\VS{30}Les apôtres se rassemblèrent auprès de Jésus, et lui racontèrent tout ce qu'ils avaient fait et enseigné.
\VS{31}Jésus leur dit : Venez à l'écart dans un lieu désert, et reposez-vous un peu ; car il y avait beaucoup de gens qui allaient et qui venaient, de sorte qu'ils n'avaient même pas le temps de manger.
\TextTitle{[Multiplication des pains pour les cinq mille hommes]
\\(Mt. 14:12-21 ; Lu. 9:15-17 ; Jn. 6:1-14)}
\VS{32}Ils s'en allèrent donc dans une barque, à l’écart, dans un lieu désert.
\VS{33}Beaucoup de gens les virent s’en aller et les reconnurent, et de toutes les villes on accourut à pied et on les devança au lieu où ils se rendaient.
\VS{34}Quand il sortit, Jésus vit une grande foule, et fut ému de compassion pour elle, parce qu’ils étaient comme des brebis qui n'ont pas de pasteur ; et il se mit à leur enseigner plusieurs choses.
\VS{35}Comme il était déjà tard, ses disciples s'approchèrent de lui, en disant : Ce lieu est désert, et il est déjà tard,
\VS{36}renvoie-les, afin qu’ils s'en aillent dans les campagnes et dans les villages des environs pour s’acheter des pains ; car ils n'ont rien à manger.
\VS{37}Jésus leur répondit : Donnez-leur vous-mêmes à manger. Et ils lui dirent : Irions-nous acheter des pains pour deux cents deniers, et leur donnerions-nous à manger ?
\VS{38}Et il leur dit : Combien avez-vous de pains ? Allez voir. Et quand ils le surent, ils répondirent : Cinq, et deux poissons.
\VS{39}Alors il leur commanda de les faire tous asseoir par groupes sur l'herbe verte.
\VS{40}Et ils s'assirent par rangées de cent et de cinquante personnes.
\VS{41}Il prit les cinq pains et les deux poissons, et, levant les yeux vers le ciel, il bénit Dieu et rompit les pains, puis il les donna à ses disciples, afin qu'ils les distribuent à la foule. Il partagea aussi les deux poissons entre tous.
\VS{42}Tous mangèrent et furent rassasiés.
\VS{43}Et l’on emporta douze paniers pleins de morceaux de pains et de ce qui restait des poissons.
\VS{44}Ceux qui avaient mangé les pains étaient environ cinq mille hommes.
\TextTitle{[Jésus marche sur la mer]
\\(Mt. 14:22-33 ; Jn. 6:15-21)}
\VS{45}Et aussitôt après, il obligea ses disciples à monter dans la barque, et à le devancer sur l’autre bord, vers Bethsaïda, pendant que lui-même renverrait la foule.
\VS{46}Quand il l’eut renvoyée, il s'en alla sur la montagne pour prier.
\VS{47}Le soir étant venu, la barque était au milieu de la mer, et Jésus était seul à terre.
\VS{48}Il vit qu'ils avaient beaucoup de peine à ramer, parce que le vent leur était contraire. Vers la quatrième veille de la nuit, il alla vers eux marchant sur la mer, et il voulait les devancer.
\VS{49}Quand ils le virent marcher sur la mer, ils crurent que c’était un fantôme, et ils poussèrent des cris ;
\VS{50}car ils le voyaient tous, et ils furent troublés. Mais il leur parla aussitôt, et leur dit : Rassurez-vous, c'est moi. N’ayez pas peur.
\VS{51}Et il monta vers eux dans la barque, et le vent cessa. Et ils furent en eux-mêmes excessivement étonnés et remplis d’admiration.
\VS{52}Car ils n'avaient pas compris le miracle des pains, parce que leur cœur était endurci.
\TextTitle{[Jésus guérit les malades à Génésareth]
\\(Mt. 14:34-36)}
\VS{53}Après avoir traversé la mer, ils arrivèrent dans la contrée de Génésareth, où ils abordèrent.
\VS{54}Et dès qu’ils furent sortis de la barque, les gens, ayant aussitôt reconnu Jésus,
\VS{55}parcoururent tous les environs, et se mirent à lui apporter de tous côtés les malades sur de petits lits, partout où ils apprenaient qu'il était.
\VS{56}Et partout où il entrait, dans les villages, dans les villes, ou dans les campagnes, ils mettaient les malades dans les places publiques, et ils le priaient de leur permettre seulement de toucher le bord de son vêtement. Et tous ceux qui le touchaient étaient guéris.
\TextTitle{[Jésus condamne les traditions]
\\(Mt. 15:1-9)}
\Chap{7}
\VerseOne{}Alors les pharisiens, et quelques scribes qui étaient venus de Jérusalem, s'assemblèrent auprès de Jésus.
\VS{2}Ils virent quelques-uns de ses disciples mangeant du pain avec des mains impures, c'est-à-dire non lavées, et ils les blâmèrent.
\VS{3}Or, les pharisiens et tous les Juifs ne mangent pas sans s’être lavé leurs mains jusqu’au coude, conformément à la tradition des anciens.
\VS{4}Et quand ils reviennent de la place publique, ils ne mangent qu’après s’être lavés{\FTNT{Le verbe laver vient du grec «~baptizo~»~: «~Plonger, immerger, submerger, purifier en plongeant ou en submergeant, laver, rendre pur avec de l'eau, se baigner~» (Mt. 3:6-16~; Mt. 28:19~; Ac. 1:5~; Ac. 2:38~; 1 Co. 12:13, etc.) Jésus évoque ici les rites de purification chez les Juifs au premier siècle. A cette époque, le souci de purification avait conduit des groupes comme les pharisiens et les esséniens à multiplier les rites d’eau. Les découvertes de Qumran ont montré que les esséniens vivaient dans la hantise de ce qui aurait pu les rendre impurs. Ainsi, les rituels de purification avec de l’eau rythmaient la vie des juifs. A titre d’exemple, les jarres de Cana étaient utilisés à cet effet (Jn. 2:6).}}. Il y a plusieurs autres observances dont ils se sont chargés, comme le lavage des coupes, de cruches, des vases d'airain, et des lits.
\VS{5}Et les pharisiens et les scribes l'interrogèrent, en disant : Pourquoi tes disciples ne se conduisent-ils pas selon la tradition des anciens, mais prennent-ils leur repas sans se laver les mains ?
\VS{6}Jésus leur répondit : Hypocrites, Esaïe a bien prophétisé de vous, ainsi qu’il est écrit : Ce peuple m'honore des lèvres, mais leur cœur est éloigné de moi{\FTNT{Es. 29:13.}}.
\VS{7}C’est en vain qu’ils m'honorent, en enseignant des doctrines qui sont des commandements d'hommes.
\VS{8}Vous abandonnez le commandement de Dieu, et vous retenez la tradition des hommes, à savoir le lavage des cruches et des coupes, et vous faites beaucoup d'autres choses semblables.
\VS{9}Il leur dit aussi : Vous rejetez bien le commandement de Dieu, afin de garder votre tradition.
\VS{10}Car Moïse a dit : Honore ton père et ta mère ; et : celui qui maudira son père ou sa mère, sera puni de mort.
\VS{11}Mais vous, vous dites : Si quelqu'un dit à son père ou à sa mère : Tout ce dont je pourrais t’assister est corban, c’est-à-dire une offrande à Dieu, il ne sera point coupable.
\VS{12}Et vous ne lui permettez plus de rien faire pour son père ou pour sa mère,
\VS{13}anéantissant ainsi la parole de Dieu par votre tradition que vous avez établie. Et vous faites encore beaucoup d’autres choses semblables.
\TextTitle{[Le coeur humain]
\\(Mt. 15:10-20)}
\VS{14}Ensuite, ayant appelé la foule, il leur dit : Ecoutez-moi vous tous, et comprenez.
\VS{15}Il n’est hors de l’homme rien qui, entrant en lui, puisse le souiller ; mais ce qui sort de l’homme, c’est ce qui le souille.
\VS{16}Si quelqu'un a des oreilles pour entendre, qu'il entende.
\VS{17}Lorsqu’il fut entré dans la maison, loin de la foule, ses disciples l'interrogèrent sur cette parabole.
\VS{18}Et il leur dit : Vous aussi, êtes-vous sans intelligence ? Ne comprenez-vous pas que rien de ce qui du dehors entre dans l’homme ne peut le souiller ?
\VS{19}Car cela n'entre pas dans son cœur, mais dans son ventre, puis s’en va dans les lieux secrets, qui purifient le corps de tous les aliments.
\VS{20}Mais il leur dit : Ce qui sort de l'homme, c'est ce qui souille l'homme.
\VS{21}Car c’est du dedans, c'est-à-dire du cœur des hommes, que sortent les mauvaises pensées, les adultères, les fornications, les meurtres,
\VS{22}les vols, les cupidités, les méchancetés, la fraude, l'impudicité, le regard envieux, la calomnie, l’orgueil, la folie.
\VS{23}Tous ces maux sortent du dedans, et souillent l'homme.
\TextTitle{[Jésus et la femme syro-phénicienne]
\\(Mt. 15:21-28)}
\VS{24}Jésus, étant parti de là, s'en alla dans le territoire de Tyr et de Sidon. Il entra dans une maison, désirant que personne ne le sache ; mais il ne put rester caché.
\VS{25}Car une femme, dont la fille était possédée d'un esprit impur, ayant entendu parler de lui, vint et se jeta à ses pieds.
\VS{26}Cette femme était Grecque, Syro-Phénicienne d’origine. Elle le pria de chasser le démon hors de sa fille. Jésus lui dit :
\VS{27}Laisse premièrement les enfants se rassasier ; car il n'est pas raisonnable de prendre le pain des enfants, et de le jeter aux petits chiens.
\VS{28}Et elle lui répondit : Cela est vrai, Seigneur ! Cependant les petits chiens mangent sous la table les miettes que les enfants laissent tomber.
\VS{29}Alors il lui dit : A cause de cette parole va, le démon est sorti de ta fille.
\VS{30}Et quand elle rentra dans sa maison, elle trouva l’enfant couchée sur le lit, le démon étant sorti.
\TextTitle{[Jésus guérit un sourd-muet]
\\(Mt. 15:29-31)}
\VS{31}Jésus quitta le territoire de Tyr et de Sidon, et revint vers la mer de Galilée en traversant le pays de la Décapole.
\VS{32}On lui amena un sourd qui avait la parole empêchée, et on le pria de lui imposer les mains.
\VS{33}Jésus le prit à part, hors de la foule, lui mit les doigts dans les oreilles, et lui toucha la langue avec sa propre salive.
\VS{34}Puis, levant les yeux vers le ciel, il soupira, et lui dit : Ephphatha, c'est-à-dire : Ouvre-toi.
\VS{35}Aussitôt ses oreilles s'ouvrirent, et le lien de sa langue se délia, et il parla aisément.
\VS{36}Jésus leur recommanda de ne le dire à personne ; mais plus il le leur recommanda, plus ils le publièrent.
\VS{37}Et ils en étaient extrêmement étonnés, et disaient : Il fait tout à merveille ; même il fait entendre les sourds, et parler les muets.
\TextTitle{[Seconde multiplication des pains]
\\(Mt. 15:32-39)}
\Chap{8}
\VerseOne{}En ces jours-là, une grande foule s’était de nouveau réunie et n’avait rien à manger. Jésus appela ses disciples, et leur dit :
\VS{2}Je suis ému de compassion pour cette foule, car il y a déjà trois jours qu'ils sont près de moi, et ils n'ont rien à manger.
\VS{3}Si je les renvoie chez eux à jeun, ils tomberont en défaillance en chemin, car quelques-uns d'eux sont venus de loin.
\VS{4}Ses disciples lui répondirent : Comment pourrait-t-on les rassasier de pains, ici, dans un désert ?
\VS{5}Jésus leur demanda : Combien avez-vous de pains ? Sept lui répondirent-ils.
\VS{6}Alors il ordonna à la foule de s'asseoir par terre, et il prit les sept pains, et après avoir béni Dieu, il les rompit, et les donna à ses disciples pour les distribuer ; et ils les distribuèrent à la foule.
\VS{7}Ils avaient aussi quelques petits poissons ; et après avoir béni Dieu, il les fit aussi distribuer.
\VS{8}Ils mangèrent, et furent rassasiés ; et l’on remporta sept corbeilles pleines des morceaux qui restaient.
\VS{9}Ceux qui avaient mangé étaient environ quatre mille. Ensuite Jésus les renvoya.
\TextTitle{[L'enseignement corrompu des pharisiens]
\\(Mt. 16:1-12)}
\VS{10}Aussitôt après, il monta dans la barque avec ses disciples, et se rendit dans la contrée de Dalmanutha.
\VS{11}Les pharisiens survinrent, se mirent à discuter avec lui, et pour l'éprouver, lui demandèrent un signe venant du ciel.
\VS{12}Alors, Jésus soupirant profondément en son esprit, dit : Pourquoi cette génération demande-t-elle un signe ? Je vous le dis en vérité, il ne sera point donné de signe à cette génération.
\VS{13}Puis il les quitta, et remonta dans la barque, pour passer à l'autre rivage.
\VS{14}Les disciples avaient oublié de prendre des pains ; et ils n'en avaient qu'un seul avec eux dans la barque.
\VS{15}Jésus leur fit cette recommandation : Gardez-vous avec soin du levain des pharisiens et du levain d'Hérode.
\VS{16}Ils raisonnaient entre eux, disant : C'est parce que nous n'avons pas de pains.
\VS{17}Jésus, le sachant, leur dit : Pourquoi discourez-vous sur ce que vous n'avez pas de pains ? N’entendez-vous pas encore, et ne comprenez-vous pas ?
\VS{18}Avez-vous encore votre cœur endurci ? Ayant des yeux, ne voyez-vous point ? Ayant des oreilles, n'entendez-vous point ? Et n'avez-vous point de mémoire ?
\VS{19}Quand j’ai rompu les cinq pains pour les cinq mille hommes, combien de paniers pleins de morceaux avez-vous emportés ? Douze, lui répondirent-ils.
\VS{20}Et quand j’ai rompu les sept pains pour quatre mille hommes, combien de corbeilles pleines de morceaux avez-vous emportées ? Sept, répondirent-ils.
\VS{21}Et il leur dit : Comment n'avez-vous pas d'intelligence ?
\TextTitle{[Jésus guérit un aveugle]}
\VS{22}Ils se rendirent à Bethsaïda, et on lui présenta un aveugle, qu’on le pria de toucher.
\VS{23}Alors il prit la main de l'aveugle, et le conduisit hors du village ; puis il lui mit de la salive sur les yeux, lui imposa les mains, et lui demanda s'il voyait quelque chose.
\VS{24}Et cet homme ayant regardé, dit : Je vois des hommes qui marchent, et qui me paraissent comme des arbres.
\VS{25}Jésus lui mit de nouveau les mains sur les yeux, et lui dit de regarder ; et il fut rétabli, et les voyait tous distinctement.
\VS{26}Puis il le renvoya dans sa maison, en lui disant : N'entre pas dans le village, et ne le dis à personne du village.
\TextTitle{[Pierre reconnaît Jésus comme le Messie]
\\(Mt. 16:13-16 ; Lu. 9:18-21 ; Jn. 6:67-71)}
\VS{27}Jésus s’en alla, avec ses disciples, dans les villages de Césarée de Philippe, et sur le chemin il interrogea ses disciples, leur disant : Qui dit-on que je suis ?
\VS{28}Ils répondirent : Les uns disent que tu es Jean-Baptiste ; les autres, Elie ; et les autres, l'un des prophètes.
\VS{29}Alors il leur dit : Et vous, qui dites-vous que je suis ? Pierre lui répondit : Tu es le Christ.
\VS{30}Et il leur défendit très sévèrement de ne dire cela de lui à personne.
\VS{31}Alors il commença à leur enseigner qu'il fallait que le Fils de l'homme souffre beaucoup, qu'il soit rejeté par les anciens, par les principaux sacrificateurs et par les scribes, qu'il soit mis à mort, et qu'il ressuscite trois jours après.
\VS{32}Il leur tenait ces discours ouvertement. Et Pierre l’ayant pris à part, se mit à le reprendre.
\VS{33}Mais Jésus, se retournant et regardant ses disciples, réprimanda Pierre en lui disant : Va arrière de moi, Satan ! Car tu ne comprends pas les choses de Dieu, mais celles des hommes.
\TextTitle{[La consécration du disciple]
\\(Mt. 16:24-28 ; Lu. 9:23-26)}
\VS{34}Puis, ayant appelé la foule et ses disciples, il leur dit : Si quelqu’un veut venir après moi, qu'il renonce à lui-même, qu'il se charge de sa croix, et qu’il me suive.
\VS{35}Car quiconque voudra sauver son âme, la perdra ; mais quiconque perdra son âme pour l'amour de moi et de l'Evangile, celui-là la sauvera.
\VS{36}Car que sert-il à un homme de gagner tout le monde, s'il perd son âme ?
\VS{37}Que donnerait un homme en échange de son âme ?
\VS{38}Car quiconque aura honte de moi et de mes paroles au milieu de cette génération adultère et pécheresse, le Fils de l'homme aura aussi honte de lui, quand il viendra dans la gloire de son Père avec les saints anges.
\TextTitle{[La transfiguration]
\\(Mt. 17:1-8 ; Lu. 9:27-36)}
\Chap{9}
\VerseOne{}Il leur disait aussi : Je vous le dis en vérité, quelques-uns de ceux qui sont ici présents, ne mourront point qu’ils n’aient vu le Royaume de Dieu venir avec puissance{\FTNT{Voir commentaire Mt. 16:28.}}.
\VS{2}Six jours après, Jésus prit avec lui Pierre, Jacques et Jean, et les conduisit seuls à l'écart sur une haute montagne. Il fut transfiguré devant eux,
\VS{3}ses vêtements devinrent resplendissants, et blancs comme de la neige, tels qu'il n’est pas de foulon sur la terre qui puisse blanchir ainsi.
\VS{4}Et en même temps leur apparurent Moïse et Elie, qui s’entretenaient avec Jésus.
\VS{5}Alors Pierre prenant la parole, dit à Jésus : Rabbi, il est bon que nous soyons ici ; faisons donc trois tentes, une pour toi, une pour Moïse, et une pour Elie.
\VS{6}Car il ne savait pas quoi dire, ils étaient épouvantés.
\VS{7}Une nuée vint les couvrir de son ombre, et de la nuée sortit une voix : Celui-ci est mon Fils bien-aimé, écoutez-le.
\VS{8}Aussitôt les disciples regardèrent tout autour, et ils ne virent que Jésus seul avec eux.
\VS{9}Comme ils descendaient de la montagne, Jésus leur recommanda expressément de ne raconter à personne ce qu'ils avaient vu, jusqu’à ce que le Fils de l'homme soit ressuscité des morts.
\VS{10}Ils retinrent cette parole, se demandant entre eux ce que c'était que ressusciter des morts.
\VS{11}Les disciples l'interrogèrent, disant : Pourquoi les scribes disent-ils qu'il faut qu'Elie vienne premièrement ?
\VS{12}Il leur répondit : Il est vrai, Elie viendra premièrement, et rétablira toutes choses. Et pourquoi est-il écrit du Fils de l'homme qu’il doit beaucoup souffrir et être méprisé ?
\VS{13}Mais je vous dis qu’Elie est venu, et qu'ils lui ont fait tout ce qu’ils ont voulu, selon qu’il est écrit de lui.
\TextTitle{[Incapacité des disciples et la toute-puissance de Jésus-Christ]
\\(Mt. 17:14-21 ; Lu. 9:37-43)}
\VS{14}Lorsqu’il fut arrivé près des disciples, il vit une grande foule autour d’eux, et des scribes qui discutaient avec eux.
\VS{15}Dès que la foule vit Jésus, elle fut saisie d'étonnement, et accourut pour le saluer.
\VS{16}Alors il demanda aux scribes : De quoi discutez-vous avec eux ?
\VS{17}Et un homme de la foule prenant la parole, dit : Maître, je t'ai amené mon fils qui est possédé d’un esprit muet.
\VS{18}En quelque lieu qu’il le saisisse, il le jette par terre ; l’enfant écume, grince des dents, et devient tout raide. J’ai prié tes disciples de le chasser, mais ils n'ont pas pu.
\VS{19}Alors Jésus leur répondit : Ô génération incrédule ! Jusqu’à quand serai-je avec vous ? Jusqu’à quand vous supporterai-je ? Amenez-le-moi. Ils le lui amenèrent.
\VS{20}Et aussitôt que l’enfant vit Jésus, l'esprit l'agita sur-le-champ avec violence ; il tomba par terre, et se roulait en écumant.
\VS{21}Jésus demanda au père de l'enfant : Combien y a-t-il de temps que cela lui arrive ? Et il dit : Dès son enfance.
\VS{22}Et souvent l’esprit l’a jeté dans le feu et dans l'eau pour le faire périr. Mais si tu peux quelque chose, secours-nous, aie compassion de nous.
\VS{23}Alors Jésus lui dit : Si tu peux croire, tout est possible à celui qui croit.
\VS{24}Et aussitôt le père de l'enfant s'écriant avec larmes : Je crois, Seigneur ! Secours-moi dans mon incrédulité.
\VS{25}Jésus voyant accourir la foule, reprit sévèrement l’esprit impur, et lui dit : Esprit muet et sourd, je te l’ordonne, sors de cet enfant, et n'y rentre plus !
\VS{26}Et le démon sortit, en poussant des cris, et en l’agitant avec une grande violence. L’enfant devint comme mort, de sorte que plusieurs disaient qu’il était mort.
\VS{27}Mais Jésus, l'ayant pris par la main, le fit lever. Et il se tint debout.
\VS{28}Quand Jésus fut entré dans la maison, ses disciples lui demandèrent en particulier : Pourquoi n’avons-nous pas pu chasser cet esprit ?
\VS{29}Il leur répondit : Cette sorte de démons ne peut sortir que par la prière et par le jeûne.
\TextTitle{[Jésus annonce sa mort et sa résurrection]
\\(Mt. 17:22-23 ; Lu. 9:44-45)}
\VS{30}Puis étant partis de là, ils traversèrent la Galilée. Jésus ne voulait pas qu’on le sache.
\VS{31}Car il enseignait ses disciples, et il leur dit : Le Fils de l'homme va être livré entre les mains des hommes, et ils le feront mourir, mais après qu'il aura été mis à mort, il ressuscitera le troisième jour.
\VS{32}Mais ils ne comprenaient point ce discours, et ils craignaient de l'interroger.
\TextTitle{[L'humilité, secret de la vraie grandeur]
\\(Mt. 18:1-6 ; Lu. 9:46-48)}
\VS{33}Après ces choses il vint à Capernaüm, et quand il fut arrivé à la maison, il leur demanda : De quoi discutiez-vous ensemble en chemin ?
\VS{34}Mais ils gardèrent le silence, car ils avaient discuté entre eux en chemin sur celui qui serait le plus grand.
\VS{35}Alors il s’assit, appela les douze, et leur dit : Si quelqu'un veut être le premier parmi vous, il sera le dernier de tous, et le serviteur de tous.
\VS{36}Et ayant pris un petit enfant, il le mit au milieu d'eux, et après l'avoir pris entre ses bras, il leur dit :
\VS{37}Quiconque reçoit en mon Nom un de ces petits enfants, me reçoit ; et quiconque me reçoit, ce n'est pas moi qu’il reçoit, mais celui qui m'a envoyé.
\TextTitle{[Jésus condamne l'esprit sectaire]
\\(Lu. 9:49-50)}
\VS{38}Alors Jean prit la parole, et dit : Maître, nous avons vu quelqu'un qui chasse les démons en ton Nom et qui ne nous suit pas, et nous l'en avons empêché, parce qu'il ne nous suit pas.
\VS{39}Mais Jésus leur dit : Ne l'en empêchez pas ; car il n’est personne qui, faisant un miracle en mon Nom, puisse aussitôt après parler mal de moi.
\VS{40}Qui n'est pas contre nous est pour nous.
\VS{41}Et quiconque vous donnera à boire un verre d'eau en mon Nom, parce que vous êtes à Christ, je vous le dis en vérité, il ne perdra point sa récompense.
\TextTitle{[Avertissement de Jésus concernant les occasions de chute]}
\VS{42}Mais quiconque scandalisera un de ces petits qui croient en moi, il vaudrait mieux pour lui qu'on lui mette une pierre de moulin au cou, et qu'on le jette dans la mer.
\VS{43}Si ta main est pour toi une occasion de chute, coupe-la ; mieux vaut pour toi entrer manchot dans la vie, que d'avoir les deux mains, et d’aller dans la géhenne, dans le feu qui ne s'éteint point ;
\VS{44}là où leur ver ne meurt point, et le feu ne s'éteint point.
\VS{45}Si ton pied est pour toi une occasion de chute, coupe-le ; mieux vaut pour toi entrer boiteux dans la vie, que d'avoir les deux pieds, et d’être jeté dans la géhenne, dans le feu qui ne s'éteint point ;
\VS{46}là où leur ver ne meurt point, et où le feu ne s'éteint point.
\VS{47}Si ton œil est pour toi une occasion de chute, arrache-le ; mieux vaut pour toi entrer dans le Royaume de Dieu n'ayant qu'un œil, que d'avoir les deux yeux, et d’être jeté dans le feu de la géhenne,
\VS{48}où leur ver ne meurt point, et où le feu ne s'éteint point.
\VS{49}Car chacun sera salé de feu ; et toute offrande sera salée de sel.
\VS{50}Le sel est une bonne chose ; mais si le sel devient sans saveur, avec quoi lui rendra-t-on sa saveur ?
\VS{51}Ayez du sel en vous-mêmes, et soyez en paix les uns avec les autres.
\TextTitle{[Enseignement de Jésus sur le mariage et le divorce]
\\(Mt. 5:31-32 ; 19:1-9 ; Lu. 16:18 ; Ro. 7:1-3 ; 1 Co. 7:10-16)}
\Chap{10}
\VerseOne{}Jésus, étant parti de là, se rendit dans le territoire de la Judée, au-delà du Jourdain. La foule s’assembla de nouveau auprès de lui, et selon sa coutume, il se mit à l’enseigner.
\VS{2}Alors les pharisiens vinrent à lui, et, pour l'éprouver, ils lui demandèrent s’il est permis à un homme de répudier sa femme.
\VS{3}Il répondit et leur dit : Qu'est-ce que Moïse vous a prescrit ?
\VS{4}Moïse, dirent-ils, a permis d'écrire une lettre de divorce, et de répudier ainsi sa femme{\FTNT{De. 24:1.}}.
\VS{5}Et Jésus leur répondit : C’est à cause de la dureté de votre cœur que Moïse vous a donné ce commandement.
\VS{6}Mais au commencement de la création, Dieu fit l’homme et la femme.
\VS{7}C'est pourquoi l'homme quittera son père et sa mère, et s'attachera à sa femme,
\VS{8}et les deux deviendront une seule chair. Ainsi, ils ne sont plus deux, mais ils sont une seule chair.
\VS{9}Que l'homme donc ne sépare pas ce que Dieu a mis ensemble sous un joug{\FTNT{Voir commentaire Mt. 19:6.}}.
\VS{10}Lorsqu’ils furent dans la maison, ses disciples l'interrogèrent encore là-dessus.
\VS{11}Il leur dit : Celui qui répudie sa femme et qui en épouse une autre, commet un adultère à son égard.
\VS{12}Pareillement si la femme répudie son mari, et se marie à un autre, elle commet un adultère.
\TextTitle{[Jésus bénit les petits enfants]
\\(Mt. 19:13-15 ; Lu. 18:15-17)}
\VS{13}On lui amena de petits enfants afin qu'il les touche. Mais les disciples reprirent ceux qui les amenaient.
\VS{14}Jésus, voyant cela, fut indigné, et leur dit : Laissez venir à moi les petits enfants et ne les en empêchez point, car le Royaume de Dieu appartient à ceux qui leur ressemblent.
\VS{15}Je vous le dis en vérité, quiconque ne recevra pas comme un petit enfant le Royaume de Dieu, il n'y entrera point.
\VS{16}Après les avoir pris dans ses bras, il les bénit, en leur imposant les mains.
\TextTitle{[Le jeune homme riche]
\\(Mt. 19:16-30 ; Lu. 18:18-30 ; Lu. 10:25-37)}
\VS{17}Comme Jésus se mettait en chemin, un homme accourut, et se jetant à genoux devant lui : Bon Maître, lui demanda-t-il, que dois-je faire pour hériter la vie éternelle ?
\VS{18}Jésus lui répondit : Pourquoi m'appelles-tu bon ? Il n'y a de bon que Dieu seul{\FTNT{La même histoire est racontée en Lu. 18:18 qui précise que c’était un chef qui avait interrogé Jésus. La réponse du Seigneur est ironique. Jésus aurait aussi pu lui poser la question comme suit : «~Puisque tu penses que je ne suis qu’un simple homme, pourquoi m’appelles-tu bon ?~».}}.
\VS{19}Tu connais les commandements : Ne commets point d’adultère ; ne tue point ; ne dérobe point ; ne dis point de faux témoignage ; ne fais aucun tort à personne ; honore ton père et ta mère.
\VS{20}Il lui répondit : Maître, j'ai observé toutes ces choses dès ma jeunesse.
\VS{21}Jésus, l’ayant regardé, l'aima, et lui dit : Il te manque une chose : Va, et vends tout ce que tu as, et donne-le aux pauvres, et tu auras un trésor dans le ciel. Puis, viens, et suis-moi en te chargeant de ta croix.
\VS{22}Mais, affligé de cette parole, il s'en alla tout triste, parce qu'il avait de grands biens.
\TextTitle{[Tout est possible à Dieu]}
\VS{23}Alors Jésus, ayant regardé autour de lui, dit à ses disciples : Qu’il est difficile à ceux qui ont des richesses d’entrer dans le Royaume de Dieu.
\VS{24}Ses disciples furent étonnés de ces paroles ; mais Jésus reprenant la parole, leur dit : Mes enfants, qu'il est difficile à ceux qui se confient dans les richesses d'entrer dans le Royaume de Dieu !
\VS{25}Il est plus facile à un chameau de passer par le trou d'une aiguille{\FTNT{Voir commentaire Mt. 19:24.}}, qu’à un riche d’entrer dans le Royaume de Dieu.
\VS{26}Les disciples furent encore plus étonnés, et ils se dire les uns les autres : Et qui peut être sauvé ?
\VS{27}Mais Jésus les ayant regardés, leur dit : Cela est impossible aux hommes, mais non à Dieu ; car tout est possible à Dieu.
\TextTitle{[La fidélité à Jésus-Christ sera récompensée]}
\VS{28}Alors Pierre se mit à lui dire : Voici, nous avons tout quitté et nous t'avons suivi.
\VS{29}Et Jésus répondit, disant : Je vous le dis en vérité, il n’est personne qui, ayant quitté pour l'amour de moi et de l’Evangile, sa maison, ou ses frères, ou ses sœurs, ou son père, ou sa mère, ou sa femme, ou ses enfants, ou ses terres,
\VS{30}ne reçoive au centuple, présentement dans ce temps-ci, des maisons, des frères, des sœurs, des mères, des enfants, et des terres, avec des persécutions ; et dans le siècle à venir, la vie éternelle.
\VS{31}Plusieurs des premiers seront les derniers ; et plusieurs des derniers seront les premiers.
\TextTitle{[Jésus annonce sa mort et sa résurrection]
\\(Mt. 20:17-19 ; Lu. 18:31-34)}
\VS{32}Ils étaient en chemin, pour monter à Jérusalem, et Jésus allait devant eux. Les disciples étaient troublés, et le suivaient avec crainte. Et Jésus prit de nouveau à l'écart les douze, et commença à leur déclarer ce qui devait lui arriver,
\VS{33}disant : Voici, nous montons à Jérusalem, et le Fils de l'homme sera livré aux principaux sacrificateurs et aux scribes. Ils le condamneront à mort, et le livreront aux gentils
\VS{34}qui se moqueront de lui, le battront de verges, cracheront sur lui, et le feront mourir ; et il ressuscitera trois jours après.
\TextTitle{[Jésus répond à la question de Jacques et Jean]}
\VS{35}Alors Jacques et Jean, fils de Zébédée, s’approchèrent de Jésus et lui dirent : Maître, nous voudrions que tu fasses pour nous ce que nous te demanderons.
\VS{36}Il leur dit : Que voulez-vous que je fasse pour vous ?
\VS{37}Et ils lui dirent : Accorde-nous, lui dirent-ils, d’être assis l’un à ta droite et l’autre à ta gauche, quand tu seras dans ta gloire.
\VS{38}Jésus leur dit : Vous ne savez pas ce que vous demandez. Pouvez-vous boire la coupe que je dois boire, et être baptisés du baptême dont je dois être baptisé ?
\VS{39}Ils lui répondirent : Nous le pouvons. Et Jésus leur répondit : Il est vrai que vous boirez la coupe que je dois boire, et que vous serez baptisés du baptême dont je dois être baptisé ;
\VS{40}mais pour ce qui est d'être assis à ma droite et à ma gauche, ce n'est pas à moi de l’accorder ; mais cela ne sera donné qu’à ceux à qui cela est préparé.
\VS{41}Les dix autres, ayant entendu cela, commencèrent à s’indigner contre Jacques et Jean.
\VS{42}Jésus les appela et leur dit : Vous savez que ceux qu’on regarde comme les chefs des nations les dominent, et que les grands les asservissent.
\VS{43}Il n'en sera pas de même parmi vous. Mais quiconque veut être le plus grand parmi vous, qu’il soit votre serviteur,
\VS{44}et quiconque veut être le premier parmi vous, qu’il soit l’esclave de tous.
\VS{45}Car le Fils de l'homme est venu, non pour être servi, mais pour servir et donner sa vie en rançon pour plusieurs.
\TextTitle{[Jésus guérit l'aveugle Bartimée]
\\(Mt. 20:29-34 ; Lu. 18:35-43)}
\VS{46}Ils arrivèrent à Jéricho. Et lorsque Jésus en sortit, avec ses disciples et une grande foule, un aveugle, appelé Bartimée, c'est-à-dire le fils de Timée, était assis au bord du chemin et mendiait.
\VS{47}Il entendit que c'était Jésus de Nazareth, et il se mit à crier et à dire : Jésus, Fils de David, aie pitié de moi !
\VS{48}Plusieurs le reprenaient pour le faire taire ; mais il criait beaucoup plus fort : Fils de David, aie pitié de moi !
\VS{49}Jésus s’arrêta, et dit : Appelez-le. Ils appelèrent l’aveugle en lui disant : Prends courage, lève-toi, il t'appelle.
\VS{50}L’aveugle jeta son manteau, il se leva et vint vers Jésus.
\VS{51}Jésus, prenant la parole, lui dit : Que veux-tu que je te fasse ? Et l'aveugle lui dit : Maître, que je recouvre la vue.
\VS{52}Et Jésus lui dit : Va, ta foi t'a sauvé.
\VS{53}Et aussitôt il recouvra la vue, et suivit Jésus dans le chemin.
\TextTitle{[Entrée de Jésus à Jérusalem]
\\(Mt. 21:1-11 ; Lu. 19:28-40 ; Jn. 12:12-19 ; Za. 9:9)}
\Chap{11}
\VerseOne{}Lorsqu’ils approchaient de Jérusalem, et qu’ils furent près de Bethphagé et de Béthanie, vers le Mont des oliviers, Jésus envoya deux de ses disciples,
\VS{2}en leur disant : Allez au village qui est devant vous. Dès que vous y serez entrés, vous trouverez un ânon attaché, sur lequel aucun homme ne s’est encore assis. Détachez-le, et amenez-le.
\VS{3}Si quelqu'un vous dit : Pourquoi faites-vous cela ? Dites que le Seigneur en a besoin ; et à l’instant, il le laissera venir ici.
\VS{4}Ils partirent donc, et trouvèrent l'ânon qui était attaché dehors, près d’une porte, au contour du chemin, et ils le détachèrent.
\VS{5}Quelques-uns de ceux qui étaient là leur dirent : Pourquoi détachez-vous cet ânon ?
\VS{6}Ils leur répondirent comme Jésus l’avait ordonné ; et on les laissa faire.
\VS{7}Ils amenèrent donc l'ânon à Jésus, sur lequel ils jetèrent leurs vêtements, et Jésus s’assit dessus.
\VS{8}Beaucoup étendirent leurs vêtements sur le chemin, et d'autres des branches qu’ils coupèrent dans les champs.
\VS{9}Ceux qui allaient devant, et ceux qui suivaient, criaient en disant : Hosanna ! Béni soit celui qui vient au Nom du Seigneur !
\VS{10}Béni soit le règne de David notre père, le règne qui vient au Nom du Seigneur ! Hosanna dans les lieux très hauts !
\VS{11}Jésus entra ainsi à Jérusalem, dans le temple. Quand il eut tout considéré, il était déjà tard, il sortit pour aller à Béthanie avec les douze.
\TextTitle{[Le figuier sans fruit]
\\(Mt. 21:18-22)}
\VS{12}Le lendemain, après qu’ils furent sortis de Béthanie, Jésus eut faim.
\VS{13}Apercevant de loin un figuier qui avait des feuilles, il alla voir s'il y trouverait quelque chose ; et s’en étant approché, il ne trouva que des feuilles, car ce n'était pas la saison des figues.
\VS{14}Jésus prenant la parole dit au figuier : Que jamais personne ne mange de ton fruit ! Et ses disciples l'entendirent.
\TextTitle{[Jésus chasse les marchands du temple]
\\(Mt. 21:12-13 ; Lu. 19:45-46 ; Jn. 2:13-16)}
\VS{15}Ils arrivèrent donc à Jérusalem, et Jésus entra dans le temple. Il se mit à chasser dehors ceux qui vendaient, et ceux qui achetaient dans le temple, et il renversa les tables des changeurs, et les sièges de ceux qui vendaient des pigeons.
\VS{16}Il ne laissait personne porter aucun objet à travers le temple.
\VS{17}Et il les enseignait, en leur disant : N'est-il pas écrit : Ma maison sera appelée une maison de prière par toutes les nations ? Mais vous, vous en avez fait une caverne de voleurs{\FTNT{Jé. 7:11.}}.
\VS{18}Les scribes et les principaux sacrificateurs l’ayant entendu, cherchèrent les moyens de le faire périr ; car ils le craignaient, parce que toute la foule était frappée de sa doctrine.
\VS{19}Le soir étant venu, Jésus sortit de la ville.
\TextTitle{[La prière de la foi]
\\(1 Jn. 5:14-15)}
\VS{20}Le matin, en passant, les disciples virent le figuier séché jusqu’aux racines.
\VS{21}Pierre s'étant souvenu de ce qui s'était passé, dit à Jésus : Maître, voici, le figuier que tu as maudit a séché.
\VS{22}Jésus répondant, leur dit : Ayez foi en Dieu.
\VS{23}Je vous le dis en vérité, si quelqu’un dit à cette montagne : Ôte-toi de là et jette-toi dans la mer, et s’il ne doute point en son cœur, mais croit que ce qu’il a dit arrive, il le verra s’accomplir.
\VS{24}C'est pourquoi je vous dis : Tout ce que vous demanderez en priant, croyez que vous l’avez reçu, et vous le verrez s’accomplir.
\TextTitle{[Le pardon]}
\VS{25}Mais quand vous vous présenterez pour faire votre prière, si vous avez quelque chose contre quelqu'un, pardonnez-lui, afin que votre Père qui est dans les cieux vous pardonne aussi vos fautes.
\VS{26}Mais si vous ne pardonnez pas, votre Père qui est dans les cieux ne vous pardonnera point aussi vos fautes.
\TextTitle{[L'autorité de Jésus-Christ mise en doute]
\\(Mt. 21:23-27 ; Lu. 20:1-8)}
\VS{27}Ils se rendirent de nouveau à Jérusalem, et pendant que Jésus marchait dans le temple, les principaux sacrificateurs, les scribes et les anciens vinrent à lui,
\VS{28}et lui dirent : Par quelle autorité fais-tu ces choses, et qui t'a donné cette autorité pour faire les choses que tu fais ?
\VS{29}Jésus leur répondit : Je vous demanderai aussi une chose, et répondez-moi ; puis je vous dirai par quelle autorité je fais ces choses.
\VS{30}Le baptême de Jean venait-il du ciel, ou des hommes ? Répondez-moi.
\VS{31}Et ils raisonnaient entre eux, disant : Si nous disons, du ciel : Il nous dira : Pourquoi donc n’avez-vous pas cru en lui ?
\VS{32}Et si nous disons : Des hommes, nous avons à craindre le peuple, car tous croyaient que Jean était un vrai prophète.
\VS{33}Alors ils répondirent à Jésus : Nous ne savons pas. Et Jésus leur dit : Moi non plus je ne vous dirai pas par quelle autorité je fais ces choses.
\TextTitle{[Parabole des vignerons]
\\(Mt. 21:33-46 ; Lu. 20:9-18 ; Es. 5:1-7)}
\Chap{12}
\VerseOne{}Jésus se mit à leur parler en paraboles : Quelqu'un, dit-il, planta une vigne, et l'environna d'une haie, creusa un pressoir, et bâtit une tour ; puis il la loua à des vignerons, et quitta le pays.
\VS{2}Au temps de la récolte, il envoya un serviteur vers les vignerons, pour recevoir d'eux le fruit de la vigne.
\VS{3}S’étant saisis de lui, ils le battirent, et le renvoyèrent à vide.
\VS{4}Il envoya de nouveau un autre serviteur vers eux. Ils lui jetèrent des pierres, le frappèrent à la tête, et le renvoyèrent après l'avoir outragé.
\VS{5}Il en envoya de nouveau un troisième, qu’ils tuèrent ; et plusieurs autres, et ils battirent les uns, et tuèrent les autres.
\VS{6}Il avait encore un fils, son bien-aimé, il le leur envoya le dernier, disant : Ils auront du respect pour mon fils.
\VS{7}Mais ces vignerons dirent entre eux : Voici l'héritier, venez, tuons-le, et l'héritage sera à nous.
\VS{8}Ils se saisirent de lui, le tuèrent, et le jetèrent hors de la vigne.
\VS{9}Que fera donc le maître de la vigne ? Il viendra, et fera périr ces vignerons, et donnera la vigne à d'autres.
\VS{10}N'avez-vous pas lu cette parole de l’Ecriture ? La pierre qu’ont rejetée ceux qui bâtissaient est devenue la principale de l’angle{\FTNT{Jésus-Christ, la pierre angulaire : Ps. 118:22-23 ; Es. 8:13-17.}} ?
\VS{11}Cela a été fait par le Seigneur, et c'est une chose merveilleuse à nos yeux.
\VS{12}Alors ils cherchaient à se saisir de lui, mais ils craignirent la foule. Ils avaient compris que c’était pour eux que Jésus avait dit cette parabole. Et ils le quittèrent, et s’en allèrent.
\TextTitle{[Le tribut dû à César]
\\(Mt. 22:15-22 ; Lu. 20:19-26)}
\VS{13}Mais ils envoyèrent quelques-uns des pharisiens et des hérodiens auprès de Jésus afin de le surprendre par ses discours.
\VS{14}Et ils vinrent lui dire : Maître, nous savons que tu es vrai, et que tu ne t’inquiètes de personne ; car tu ne regardes pas à l'apparence des hommes, et tu enseignes la voie de Dieu selon la vérité. Est-il permis ou non de payer le tribut à César ? Devons-nous payer, ou ne pas payer ?
\VS{15}Mais Jésus, connaissant leur hypocrisie, leur dit : Pourquoi me tentez-vous ? Apportez-moi un denier, afin que je le voie.
\VS{16}Ils lui en apportèrent un. Alors il leur dit : De qui porte-t-il l’image et l’inscription ? De César, lui répondirent-ils.
\VS{17}Alors Jésus leur dit : Rendez à César ce qui est à César, et à Dieu ce qui est à Dieu. Et ils furent remplis d’admiration pour lui.
\TextTitle{[Jésus répond aux sadducéens sur la résurrection]
\\(Mt. 22:23-33 ; Lu. 20:27-38)}
\VS{18}Alors les sadducéens, qui disent qu'il n'y a point de résurrection, vinrent à lui, et l'interrogèrent, disant :
\VS{19}Maître, voici ce que Moïse nous a prescrit : Si le frère de quelqu'un meurt, et laisse sa femme sans avoir d'enfants, son frère épousera sa veuve et suscitera une postérité à son frère.
\VS{20}Or, il y avait sept frères. Le premier prit une femme et mourut sans laisser d'enfants.
\VS{21}Le deuxième prit la veuve pour femme, et mourut sans laisser de postérité. Il en fut de même du troisième,
\VS{22}et les sept l’épousèrent sans laisser de postérité. Après eux tous, la femme mourut aussi.
\VS{23}A la résurrection, quand ils seront ressuscités, duquel d’entre eux sera-t-elle la femme ? Car les sept l’ont eue pour femme.
\VS{24}Jésus leur répondit : La raison pour laquelle vous tombez dans l'erreur, c'est que vous ne connaissez ni les Ecritures ni la puissance de Dieu.
\VS{25}Car, à la résurrection des morts, les hommes ne prendront point de femmes, ni les femmes de maris, mais ils seront comme les anges dans les cieux.
\VS{26}Et quant aux morts, pour vous montrer qu'ils ressuscitent, n'avez-vous point lu dans le livre de Moïse, comment Dieu lui parla dans le buisson, en disant : Je suis le Dieu d'Abraham, et le Dieu d'Isaac, et le Dieu de Jacob ?
\VS{27}Or il n'est pas le Dieu des morts, mais le Dieu des vivants. Vous êtes donc dans une grande erreur.
\TextTitle{[Jésus répond aux pharisiens concernant le plus grand commandement de la loi]
\\(Mt. 22:34-40 ; Lu. 10:25-28)}
\VS{28}Un des scribes, qui les avait entendus discuter, voyant qu'il leur avait bien répondu, s'approcha de lui, et lui demanda : Quel est le premier de tous les commandements ?
\VS{29}Jésus lui répondit : Le premier de tous les commandements est : Ecoute Israël{\FTNT{Ecoute Israël~: Jésus se réfère ici à De. 6:4~: «~Ecoute, Israël ! Yahweh, notre Dieu Yahweh est Un~». Le Shema Israël est le noyau central de la prière que le Juif adulte doit lire matin et soir. C’est la confession de foi juive. Jacob est le premier à l’avoir enseignée à ses enfants dans Ge. 49:1-2.}}, le Seigneur notre Dieu, le Seigneur est Un{\FTNT{Jésus-Christ, notre Seigneur et notre modèle, a confirmé le Shema Israël qui déclare haut et fort que Dieu est Un et non trois en un. Le scribe, homme versé dans les Ecritures, était satisfait de la réponse de Jésus car il croyait aussi en un seul Dieu. Or le monothéisme est le fondement de la foi juive et des premiers chrétiens.}}.
\VS{30}Tu aimeras le Seigneur ton Dieu de tout ton cœur, de toute ton âme, de toute ta pensée, et de toute ta force. C'est là le premier commandement.
\VS{31}Voici le second, qui est semblable au premier : Tu aimeras ton prochain comme toi-même. Il n'y a pas d'autre commandement plus grand que ceux-là.
\VS{32}Et le scribe lui dit : Maître, tu as bien dit selon la vérité, qu'il y a un seul Dieu, et qu'il n'y en a point d'autre que lui ;
\VS{33}et que de l'aimer de tout son cœur, de toute son intelligence, de toute son âme, et de toute sa force ; et d'aimer son prochain comme soi-même, c'est plus que tous les holocaustes et les sacrifices.
\VS{34}Jésus voyant que ce scribe avait répondu prudemment, lui dit : Tu n'es pas loin du Royaume de Dieu. Et personne n'osait plus l'interroger.
\TextTitle{[Jésus dénonce les scribes]
\\(Mt. 22:41-46 ; Lu. 20:39-44)}
\VS{35}Comme Jésus enseignait dans le temple, il prit la parole et dit : Comment les scribes disent-ils que le Christ est le Fils de David ?
\VS{36}Car David lui-même a dit par le Saint-Esprit : Le Seigneur a dit à mon Seigneur : Assieds-toi à ma droite, jusqu'a ce que je fasse de tes ennemis ton marchepied{\FTNT{Ps. 110:1.}}.
\VS{37}David lui-même l'appelle son Seigneur, comment est-il son fils ? Et une grande foule l’écoutait avec plaisir.
\VS{38}Il leur disait dans son enseignement : Gardez-vous des scribes qui prennent plaisir à se promener en robes longues, et qui aiment les salutations dans les places publiques,
\VS{39}qui recherchent les premiers sièges dans les synagogues, et les premières places dans les festins ;
\VS{40}qui dévorent entièrement les maisons des veuves, et qui font pour l’apparence de longues prières. Ils seront jugés plus sévèrement.
\TextTitle{[L'offrande de la pauvre veuve]
\\(Lu. 21:1-4)}
\VS{41}Jésus, s’étant assis vis-à-vis du tronc, regardait comment la foule y mettait de l'argent. Plusieurs riches y mettaient beaucoup.
\VS{42}Et une pauvre veuve vint, elle y mit deux petites pièces, faisant le quart d’un sou.
\VS{43}Et Jésus, ayant appelé ses disciples, leur dit : Je vous le dis en vérité, cette pauvre veuve a plus mis dans le tronc que tous ceux qui y ont mis.
\VS{44}Car tous ont mis de leur superflu ; mais elle a mis de son nécessaire, tout ce qu'elle possédait, tout ce qu’elle avait pour vivre.
\TextTitle{[Les deux questions des disciples et la prophétie sur la destruction du temple de Jérusalem]
\\Mt. 24:3 ; Lu. 21:7)}
\Chap{13}
\VerseOne{}Lorsque Jésus sortit du temple, un de ses disciples lui dit : Maître, regarde quelles pierres et quelles constructions !
\VS{2}Jésus lui répondit : Vois-tu ces grands bâtiments ? Il ne restera pas pierre sur pierre qui ne soit pas démolie.
\VS{3}Il s’assit sur le Mont des oliviers, en face du temple. Et Pierre, Jacques, Jean et André, lui posèrent en particulier cette question :
\VS{4}Dis-nous quand cela arrivera-t-il, et à quel signe connaîtra-t-on que ces choses vont s'accomplir ?
\TextTitle{[Les temps de la fin]}
\VS{5}Jésus se mit à leur dire : Prenez garde que personne ne vous séduise.
\VS{6}Car plusieurs viendront en mon Nom, disant : C'est moi qui suis le Christ. Et ils séduiront beaucoup de gens.
\VS{7}Quand vous entendrez parler de guerres et des bruits de guerres, ne soyez point troublés ; parce qu'il faut que ces choses arrivent ; mais ce ne sera pas encore la fin.
\VS{8}Car une nation s'élèvera contre une autre nation, et un royaume contre un autre royaume ; et il y aura des tremblements de terre en divers lieux, et il y aura des famines et des troubles. Ces choses seront le commencement des douleurs.
\VS{9}Mais prenez garde à vous-mêmes. Car ils vous livreront aux tribunaux, et aux synagogues, vous serez battus de verges ; vous serez présentés devant les gouverneurs et devant les rois, à cause de moi, pour leur servir de témoignage.
\VS{10}Mais il faut premièrement que l'Evangile soit prêché à toutes les nations.
\VS{11}Et quand ils vous emmèneront pour vous livrer, ne vous inquiétez pas d’avance de ce que vous aurez à dire, mais dites ce qui vous sera donné à l’instant ; car ce n’est pas vous qui parlerez, mais le Saint-Esprit.
\VS{12}Le frère livrera son frère à la mort, et le père son enfant ; et les enfants se soulèveront contre leurs parents, et les feront mourir.
\VS{13}Vous serez haïs de tous à cause de mon Nom ; mais celui qui persévérera jusqu’à la fin, sera sauvé.
\TextTitle{[L'abomination de la désolation]
\\(Mt. 24:15-28 ; Ps. 2.5 ; Lu. 21:20-24 ; Ap. 7:14)}
\VS{14}Lorsque vous verrez l'abomination qui cause la désolation{\FTNT{Voir commentaire Mt. 24:15.}} qui a été prédite par Daniel, le prophète, établie là où elle ne doit pas être, que celui qui lit ce prophète fasse attention ! Alors que ceux qui seront en Judée fuient dans les montagnes.
\VS{15}Que celui qui sera sur le toit, ne descende pas dans la maison, et n’entre pas pour emporter quoi que ce soit de sa maison,
\VS{16}et que celui qui sera dans les champs, ne retourne pas en arrière pour emporter son manteau.
\VS{17}Malheur aux femmes qui seront enceintes, et à celles qui allaiteront en ces jours-là.
\VS{18}Priez Dieu que votre fuite n'arrive pas en hiver.
\VS{19}Car la détresse, en ces jours, sera telle qu’il n’y en a point eu de semblable depuis le commencement du monde que Dieu a créé jusqu’à présent, et qu’il n’y en aura jamais.
\VS{20}Et si le Seigneur n’avait abrégé ces jours, personne ne serait sauvé ; mais il les a abrégés, à cause des élus qu'il a choisis.
\VS{21}Si quelqu'un vous dit : Voici, le Christ est ici ; ou voici, il est là, ne le croyez point.
\VS{22}Car il s'élèvera des faux christs et des faux prophètes, qui feront des prodiges et des miracles, pour séduire même les élus s'il était possible.
\VS{23}Soyez sur vos gardes ; voici, je vous ai tout annoncé d’avance.
\TextTitle{[Retour du Messie sur la terre]
\\(Mt. 24:29-31 ; Lu. 21:25-28)}
\VS{24}Mais dans ces jours, après cette détresse, le soleil s’obscurcira, et la lune ne donnera plus sa clarté ;
\VS{25}les étoiles du ciel tomberont, et les puissances qui sont dans les cieux seront ébranlées.
\VS{26}Alors ils verront le Fils de l'homme venant sur les nuées, avec une grande puissance et une grande gloire.
\VS{27}Alors il enverra ses anges, et il rassemblera ses élus des quatre vents, de l’extrémité de la terre jusqu’à l’extrémité du ciel.
\TextTitle{[Parabole du figuier]
\\(Mt. 24:32-35 ; Lu. 21:29-33)}
\VS{28}Instruisez-vous par une comparaison tirée du figuier. Dès que ses branches deviennent tendres, et que les feuilles poussent, vous savez que l'été est proche.
\VS{29}Ainsi, quand vous verrez ces choses arriver, sachez que le Fils de l’homme est proche, à la porte.
\VS{30}Je vous le dis en vérité, cette génération ne passera point, que toutes ces choses ne soient arrivées.
\VS{31}Le ciel et la terre passeront, mais mes paroles ne passeront point.
\TextTitle{[Exhortation de Jésus sur la vigilance]
\\(Mt. 24:36-51 ; Lu. 21:34-38)}
\VS{32}Pour ce qui est du jour ou de l’heure, personne ne le sait, ni les anges dans le ciel, ni le Fils{\FTNT{Comment expliquer l’ignorance du Fils quant à l’heure de son retour ? En prenant la condition d’un homme, Jésus s’est dépouillé de ses prérogatives divines et a connu des limites propres au genre humain (Ph. 2:7)~: la fatigue (Jn. 4:6 ; Mc. 4:38), la faim (Mc. 11:12), l’angoisse et la peur (Mc. 14:33), la mortalité physique… Ce dépouillement incluait le renoncement à l’omniscience, d’où le fait que Jésus-Christ homme ne connaissait pas le jour et l’heure de son retour.}}, mais mon Père seul.
\VS{33}Prenez garde, veillez et priez ; car vous ne savez quand ce temps viendra.
\VS{34}Il en sera comme d’un homme qui, partant pour un voyage, laisse sa maison, remet l’autorité à ses serviteurs, marquant à chacun sa tâche, et ordonne au portier de veiller.
\VS{35}Veillez donc, car vous ne savez quand le Maître de la maison viendra, ou le soir, ou à minuit, ou à l'heure où le coq chante, ou le matin ;
\VS{36}craignez qu’il ne vous trouve endormis, à son arrivée soudaine.
\VS{37}Ce que je vous dis, je le dis à tous : Veillez.
\TextTitle{[Le complot]
\\(Mt. 26:1-5 ; Lu. 22:1-2)}
\Chap{14}
\VerseOne{}La fête de Pâque et des pains sans levain devait avoir lieu deux jours après. Les principaux sacrificateurs et les scribes cherchaient les moyens de se saisir de Jésus par ruse, et de le faire mourir.
\VS{2}Mais ils disaient : Que ce ne soit pas pendant la fête, afin qu'il n’y ait pas de tumulte parmi le peuple.
\TextTitle{[Marie de Béthanie oint Jésus pour sa sépulture]
\\(Mt. 26:6-13 ; Jn. 12:1-8)}
\VS{3}Comme Jésus était à Béthanie, dans la maison de Simon le lépreux, et pendant qu’il était à table, une femme vint à lui avec un vase d'albâtre, rempli d'un parfum de nard pur et de grand prix ; et ayant rompu le vase, elle répandit le parfum sur la tête de Jésus.
\VS{4}Quelques-uns en furent indignés en eux-mêmes, et ils disaient : A quoi sert la perte de ce parfum ?
\VS{5}On aurait pu le vendre plus de trois cents deniers, et les donner aux pauvres. Ainsi ils murmuraient contre elle.
\VS{6}Mais Jésus dit : Laissez-la. Pourquoi lui faites-vous de la peine ? Elle a fait une bonne action à mon égard.
\VS{7}Parce que vous aurez toujours des pauvres avec vous, et vous pouvez leur faire du bien quand vous voulez ; mais vous ne m'aurez pas toujours.
\VS{8}Elle a fait ce qu’elle a pu ; elle a d’avance embaumé mon corps pour la sépulture.
\VS{9}Je vous le dis en vérité, partout où cet Evangile sera prêché, dans le monde entier, on racontera aussi en mémoire de cette femme ce qu’elle a fait.
\TextTitle{[La trahison de Judas]
\\(Mt. 26:14-16 ; Lu. 22:3-6)}
\VS{10}Alors Judas Iscariot, l'un des douze, alla vers les principaux sacrificateurs pour le livrer.
\VS{11}Après l’avoir entendu, ils furent dans la joie, et promirent de lui donner de l'argent. Et Judas cherchait une occasion favorable pour le livrer.
\TextTitle{[La dernière Pâque]
\\(Mt. 26:17-25 ; Lu. 22:7-20 ; Jn. 13:1-12)}
\VS{12}Le premier jour des pains sans levain, où l’on sacrifiait l'agneau de Pâque, ses disciples lui dirent : Où veux-tu que nous allions te préparer l'agneau de Pâque afin que tu manges ?
\VS{13}Et il envoya deux de ses disciples, et leur dit : Allez dans la ville, vous rencontrerez un homme portant une cruche d'eau, suivez-le.
\VS{14}Où qu’il entre, dites au maître de la maison : Le Maître dit : Où est le lieu où je mangerai l'agneau de Pâque avec mes disciples ?
\VS{15}Et il vous montrera une grande chambre haute, meublée et toute prête : C’est là que vous nous préparerez l'agneau de Pâque.
\VS{16}Ses disciples partirent, arrivèrent dans la ville, ils trouvèrent les choses comme il l’avait dit ; et ils apprêtèrent l'agneau de Pâque.
\VS{17}Le soir étant venu, il arriva avec les douze.
\VS{18}Pendant qu’ils étaient à table, et qu'ils mangeaient, Jésus leur dit : Je vous le dis en vérité, l'un de vous, qui mange avec moi, me trahira.
\VS{19}Ils commencèrent à s'attrister, et ils lui dirent l'un après l'autre : Est-ce moi ?
\VS{20}Mais il leur répondit : C'est l'un des douze qui trempe avec moi dans le plat.
\VS{21}Certes le Fils de l'homme s'en va, selon qu'il est écrit de lui. Mais malheur à l'homme par qui le Fils de l'homme est trahi ! Mieux vaudrait pour cet homme qu’il ne soit pas né.
\TextTitle{[Le repas de la Pâque]
\\(Mt. 26:26-29 ; Lu. 22:17-20 ; Jn. 13:12-30 ; 1 Co. 11:23-26)}
\VS{22}Pendant qu’ils mangeaient, Jésus prit du pain, et après avoir béni Dieu, il le rompit et le leur donna, et leur dit : Prenez, mangez, ceci est mon corps.
\VS{23}Il prit ensuite une coupe, et après avoir rendu grâces, il la leur donna, et ils en burent tous.
\VS{24}Et il leur dit : Ceci est mon sang{\FTNT{Nouvelle Alliance : Voir Jn. 19:30.}}, le sang de la nouvelle alliance, qui est répandu pour plusieurs.
\VS{25}Je vous le dis en vérité, je ne boirai plus du fruit de la vigne jusqu'au jour où j’en boirai du nouveau dans le Royaume de Dieu.
\TextTitle{[Jésus avertit Pierre de son triple reniement]
\\(Mt. 26:30-35 ; Lu. 22:31-34 ; Jn. 13:36-38)}
\VS{26}Après avoir chanté les cantiques{\FTNT{Cantiques~: Voir Mt. 26:30.}}, ils se rendirent à la montagne des oliviers.
\VS{27}Jésus leur dit : Vous serez tous cette nuit scandalisés en moi ; car il est écrit : Je frapperai le Berger, et les brebis seront dispersées{\FTNT{Za. 13:7.}}.
\VS{28}Mais, après que je serai ressuscité, je vous précéderai en Galilée.
\VS{29}Pierre lui dit : Quand même tous seraient scandalisés, je ne le serai pourtant pas moi.
\VS{30}Et Jésus lui dit : Je te le dis en vérité, qu'aujourd'hui, cette nuit même, avant que le coq chante deux fois, tu me renieras trois fois.
\VS{31}Mais Pierre disait encore plus fortement : Quand même il me faudrait mourir avec toi, je ne te renierai pas. Et tous lui dirent la même chose.
\TextTitle{[Jésus dans le jardin de Gethsémané]
\\(Mt. 26:36-46 ; Lu. 22:39-46 ; Jn. 18:1)}
\VS{32}Ils allèrent dans un lieu appelé Gethsémané, et Jésus dit à ses disciples : Asseyez-vous ici jusqu'à ce que j’aie prié.
\VS{33}Il prit avec lui Pierre, Jacques et Jean, et il commença à être effrayé et fort angoissé.
\VS{34}Il leur dit : Mon âme est saisie de tristesse jusqu’à la mort, restez ici, et veillez.
\TextTitle{[Première prière de Jésus]
\\(Mt. 26:36 ; Lu. 22:41-42)}
\VS{35}Puis s'en allant un peu plus en avant, il se jeta contre terre, et pria que s'il était possible, cette heure s’éloigne de lui.
\VS{36}Il disait : Abba, Père, toutes choses te sont possibles, éloigne de moi cette coupe ! Toutefois, non pas ce que je veux, mais ce que tu veux.
\VS{37}Puis il vint vers les disciples qu’il trouva endormis, et il dit à Pierre : Simon, tu dors ! Tu n’as pas pu veiller une heure !
\VS{38}Veillez et priez afin que vous ne tombiez pas en tentation, l'esprit est bien disposé, mais la chair est faible.
\TextTitle{[Deuxième prière]
\\(Mt. 26:42 ; Lu. 22:44)}
\VS{39}Il s’éloigna de nouveau, et fit la même prière, disant les mêmes paroles.
\VS{40}Il revint, et les trouva encore endormis, car leurs yeux étaient appesantis. Ils ne surent que lui répondre.
\TextTitle{[Troisième prière]
\\(Mt. 26:44)}
\VS{41}Il revint encore, pour la troisième fois, et leur dit : Dormez maintenant, et reposez-vous ! C’est assez ! L’heure est venue ; voici, le Fils de l'homme est livré entre les mains des méchants.
\VS{42}Levez-vous, allons ; voici, celui qui me trahit s'approche.
\TextTitle{[Jésus trahi, abandonné et arrêté]
\\(Mt. 26:47-56 ; Lu. 22:47-53 ; Jn. 18:2-11)}
\VS{43}Et aussitôt, comme il parlait encore, Judas, l'un des douze, vint, et avec lui une grande foule ayant des épées et des bâtons, envoyée par les principaux sacrificateurs, par les scribes et par les anciens.
\VS{44}Celui qui le trahissait leur avait donné ce signe : Celui que j'embrasserai, c’est lui ; saisissez-le, et emmenez-le sûrement.
\VS{45}Dès qu’il fut arrivé, il s'approcha aussitôt de Jésus, et lui dit : Rabbi, Rabbi ! Et il l’embrassa.
\VS{46}Alors ils mirent la main sur Jésus, et le saisirent.
\VS{47}Un de ceux qui étaient là présents, tirant son épée, frappa le serviteur du souverain sacrificateur et lui emporta l'oreille.
\VS{48}Alors Jésus prit la parole, et leur dit : Vous êtes venus comme après un brigand, avec des épées et des bâtons, pour m’arrêter.
\VS{49}J’étais tous les jours parmi vous, enseignant dans le temple, et vous ne m'avez point saisi ; mais tout ceci est arrivé afin que les Ecritures soient accomplies.
\VS{50}Alors tous ses disciples l'abandonnèrent et s'enfuirent.
\VS{51}Un jeune homme le suivait, n’ayant sur le corps qu’un drap. Et quelques jeunes gens le saisirent,
\VS{52}mais il abandonna son linceul, et se sauva tout nu.
\TextTitle{[Jésus devant Caïphe et le sanhédrin]
\\(Mt. 26:57-68 ; Jn. 18:12-14,19-24)}
\VS{53}Ils emmenèrent Jésus chez le souverain sacrificateur, où s'assemblèrent tous les principaux sacrificateurs, les anciens et les scribes.
\VS{54}Pierre le suivait de loin jusque dans la cour du souverain sacrificateur ; et il était assis avec les serviteurs, et se chauffait près du feu.
\VS{55}Les principaux sacrificateurs et tout le sanhédrin cherchaient quelque témoignage contre Jésus pour le faire mourir, mais ils n'en trouvaient point.
\VS{56}Car plusieurs rendaient de faux témoignages contre lui, mais les témoignages ne s’accordaient pas.
\VS{57}Alors quelques-uns s'élevèrent, et portèrent de faux témoignages contre lui, disant :
\VS{58}Nous l’avons entendu dire : Je détruirai ce temple qui est fait de main d’homme, et en trois jours j'en rebâtirai un autre qui ne sera pas fait de main d’homme.
\VS{59}Même sur ce point-là leurs témoignages ne s’accordaient pas.
\VS{60}Alors le souverain sacrificateur se levant au milieu, interrogea Jésus, disant : Ne réponds-tu rien ? Qu’est-ce que ces gens déposent contre toi ?
\VS{61}Mais Jésus garda le silence, et ne répondit rien. Le souverain sacrificateur l'interrogea de nouveau, et lui dit : Es-tu le Christ, le Fils du Dieu béni ?
\VS{62}Jésus lui répondit : Je le suis. Et vous verrez le Fils de l'homme assis à la droite de la puissance de Dieu, et venant sur les nuées du ciel.
\VS{63}Alors le souverain sacrificateur déchira ses vêtements et dit : Qu'avons-nous encore besoin de témoins ?
\VS{64}Vous avez entendu le blasphème. Que vous en semble ? Alors tous le condamnèrent comme méritant la mort.
\VS{65}Et quelques-uns se mirent à cracher sur lui, à lui voiler le visage, et à lui donner des soufflets, en lui disant : Prophétise ! Et les serviteurs lui donnaient des coups avec leurs verges.
\TextTitle{[Triple reniement de Pierre]
\\(Mt. 26:69-75 ; Lu. 22:55-62 ; Jn. 18:15-18,25-27)}
\VS{66}Pendant que Pierre était en bas dans la cour, une des servantes du souverain sacrificateur vint.
\VS{67}Apercevant Pierre qui se chauffait, elle le regarda en face, et lui dit : Toi aussi, tu étais avec Jésus de Nazareth.
\VS{68}Mais il le nia, disant : Je ne le connais pas, et je ne sais pas ce que tu dis ; puis il sortit dehors pour aller dans le vestibule. Et le coq chanta.
\VS{69}La servante l'ayant vu de nouveau, elle se mit à dire à ceux qui étaient là présents : Celui-ci est de ces gens-là. Et il le nia de nouveau.
\VS{70}Peu après, ceux qui étaient là présents, dirent à Pierre : Certainement tu es de ces gens-là, car tu es Galiléen, et ton langage s'y rapporte.
\VS{71}Alors il commença à faire des imprécations et à jurer : Je ne connais pas cet homme dont vous parlez.
\VS{72}Et le coq chanta pour la seconde fois. Et Pierre se souvint de la parole que Jésus lui avait dite : Avant que le coq chante deux fois, tu me renieras trois fois. Et étant sorti promptement, il pleura.
\TextTitle{[Jésus livré à Pilate]
\\(Mt. 27:1-2,11-15 ; Lu. 23:1-7,13-18 ; Jn. 18:28-38 ; 19:1-15)}
\Chap{15}
\VerseOne{}Dès le matin, les principaux sacrificateurs tinrent conseil avec les anciens et les scribes, et tout le sanhédrin. Après avoir lié Jésus, ils l'emmenèrent, et le livrèrent à Pilate.
\VS{2}Pilate l'interrogea : Es-tu le Roi des Juifs ? Et Jésus répondit : Tu le dis.
\VS{3}Les principaux sacrificateurs l'accusaient de plusieurs choses, mais il ne répondit rien.
\VS{4}Pilate l'interrogea de nouveau : Ne réponds-tu rien ? Vois de combien de choses ils t’accusent.
\VS{5}Mais Jésus ne donna plus aucune réponse, ce qui étonna Pilate.
\TextTitle{[Jésus ou Barabbas ?]
\\(Mt. 27:15-26 ; Lu. 23:17-25 ; Jn. 18:39)}
\VS{6}A chaque fête, il relâchait un prisonnier, celui que demandait la foule.
\VS{7}Il y avait en prison un nommé Barabbas avec ses complices pour une sédition, dans laquelle ils avaient commis un meurtre.
\VS{8}La foule se mit à demander à Pilate, avec de grands cris, ce qu’il avait coutume de leur accorder.
\VS{9}Pilate leur répondit : Voulez-vous que je vous relâche le Roi des Juifs ?
\VS{10}Car il savait bien que les principaux sacrificateurs l'avaient livré par envie.
\VS{11}Mais les principaux sacrificateurs excitèrent la foule, afin que Pilate leur relâche plutôt Barabbas.
\VS{12}Pilate reprenant la parole, leur dit encore : Que voulez-vous donc que je fasse de celui que vous appelez Roi des Juifs ?
\VS{13}Ils crièrent de nouveau : Crucifie-le !
\VS{14}Alors Pilate leur dit : Mais quel mal a-t-il fait ? Et ils crièrent encore plus fort : Crucifie-le !
\VS{15}Pilate, voulant satisfaire la foule, leur relâcha Barabbas ; et après avoir fait battre de verges Jésus, il le livra pour être crucifié.
\TextTitle{[Jésus couronné d'épines]
\\(Mt. 27:27-31 ; Jn. 19:16-17)}
\VS{16}Alors les soldats emmenèrent Jésus dans l’intérieur de la cour, c’est-à-dire dans le prétoire, et ils assemblèrent toute la cohorte.
\VS{17}Ils le revêtirent d'une robe de pourpre, et posèrent sur sa tête une couronne d'épines qu’ils avaient tressée.
\VS{18}Puis ils commencèrent à le saluer, en lui disant : Nous te saluons, Roi des Juifs !
\VS{19}Et ils lui frappaient la tête avec un roseau, et crachaient sur lui, et fléchissant les genoux, ils se prosternaient devant lui.
\VS{20}Et après s'être ainsi moqués de lui, ils le dépouillèrent de la robe de pourpre, lui remirent ses habits, et l'emmenèrent dehors pour le crucifier.
\VS{21}Et un certain homme de Cyrène, nommé Simon, père d’Alexandre et de Rufus, passant par là en revenant des champs, fut forcé à porter la croix de Jésus.
\VS{22}Et ils conduisirent Jésus au lieu appelé Golgotha{\FTNT{Golgotha~: Le Golgotha (crâne) était une colline située à l'extérieur de Jérusalem, sur laquelle les Romains crucifiaient les condamnés.}}, c'est-à-dire, le lieu du Crâne.
\VS{23}Ils lui donnèrent à boire du vin mêlé de myrrhe, mais il ne le prit pas.
\TextTitle{[Jésus crucifié]
\\(Mt. 27:33-56 ; Lu. 23:33-49 ; Jn. 19:17-37)}
\VS{24}Ils le crucifièrent, et se partagèrent ses vêtements, en tirant au sort pour savoir ce que chacun aurait.
\VS{25}C’était la troisième heure, quand ils le crucifièrent.
\VS{26}L’écriteau indiquant la cause de sa condamnation portait ces mots : Le Roi des Juifs.
\VS{27}Ils crucifièrent aussi avec lui deux brigands, l'un à sa droite, et l'autre à sa gauche.
\VS{28}Et ainsi fut accomplie l'Ecriture, qui dit : Et il a été mis au rang des malfaiteurs{\FTNT{Es. 53:12.}}.
\VS{29}Les passants l’injuriaient, et secouaient la tête, en disant : Hé ! Toi qui détruis le temple et qui le rebâtis en trois jours,
\VS{30}sauve-toi toi-même, et descends de la croix !
\VS{31}Les principaux sacrificateurs aussi avec les scribes se moquaient entre eux, et disaient : Il a sauvé les autres, et il ne peut se sauver lui-même.
\VS{32}Que le Christ, le Roi d’Israël descende maintenant de la croix, afin que nous le voyions et que nous croyions ! Ceux qui étaient crucifiés avec lui l’insultaient aussi.
\VS{33}La sixième heure étant venue, il y eut des ténèbres sur toute la terre jusqu'à la neuvième heure.
\VS{34}Et à la neuvième heure, Jésus s’écria d’une voix forte : Eloï, Eloï, lama sabachthani ? C’est-à-dire : Mon Dieu ! Mon Dieu ! Pourquoi m'as-tu abandonné ?
\VS{35}Quelques-uns de ceux qui étaient là présents, l’ayant entendu, dirent : Voici, il appelle Elie.
\VS{36}Et l’un d’eux courut remplir une éponge de vinaigre{\FTNT{Le vinaigre~: Voir Mt. 27:34.}}, et l'ayant fixée au bout d'un roseau, il lui donna à boire, en disant : Laissez, voyons si Elie viendra le descendre de la croix.
\VS{37}Mais Jésus, ayant poussé un grand cri, expira.
\VS{38}Et le voile du temple se déchira en deux, depuis le haut jusqu'en bas{\FTNT{Hé 10:19-20.}}.
\TextTitle{[Fin de la Première Alliance]
\\(Hé. 9:16-18)}
\VS{39}Le centenier, qui était en face de Jésus, voyant qu'il avait expiré en criant de la sorte, dit : Certainement cet homme était Fils de Dieu.
\VS{40}Il y avait là aussi des femmes qui regardaient de loin. Parmi elles étaient Marie de Madgala, Marie mère de Jacques le mineur et de Joses, et Salomé,
\VS{41}qui le suivaient et le servaient lorsqu'il était en Galilée, et plusieurs autres qui étaient montées avec lui à Jérusalem.
\TextTitle{[Jésus enseveli]}
\VS{42}Le soir étant venu, comme c'était la préparation, c’est-à-dire le sabbat,
\VS{43}arriva Joseph d'Arimathée, conseiller de distinction, qui attendait aussi le Royaume de Dieu. Il osa se rendre vers Pilate pour demander le corps de Jésus.
\VS{44}Pilate s'étonna qu'il soit mort si tôt ; il fit venir le centenier, et lui demanda s'il était mort depuis longtemps.
\VS{45}S’en étant assuré par le centenier, il donna le corps à Joseph.
\VS{46}Et Joseph ayant acheté un linceul, descendit Jésus de la croix, et l'enveloppa du linceul, et le déposa dans un sépulcre taillé dans le roc. Puis il roula une pierre sur l'entrée du sépulcre.
\VS{47}Marie de Magdala, et Marie mère de Joses regardaient où on le mettait.
\TextTitle{[Jésus, ressuscité, apparaît à plusieurs disciples]
\\(Mt. 28:1-15 ; Lu. 24:1-49 ; Jn. 20:1-23)}
\Chap{16}
\VerseOne{}Lorsque le sabbat fut passé, Marie de Magdala, Marie mère de Jacques, et Salomé, achetèrent des aromates pour embaumer Jésus.
\VS{2}Le premier jour de la semaine, de grand matin, elles se rendirent au sépulcre, comme le soleil venait de se lever.
\VS{3}Elles disaient entre elles : Qui nous roulera la pierre de l'entrée du sépulcre ?
\VS{4}Et levant les yeux, elles virent que la pierre, qui était très grande, avait été roulée.
\VS{5}Elles entrèrent dans le sépulcre, virent un jeune homme assis à droite, vêtu d'une robe blanche, et elles furent épouvantées.
\VS{6}Mais il leur dit : Ne vous épouvantez pas. Vous cherchez Jésus de Nazareth qui a été crucifié. Il est ressuscité, il n'est point ici ; voici le lieu où on l'avait mis.
\VS{7}Mais allez, et dites à ses disciples, et à Pierre, qu'il vous précède en Galilée. C’est là que vous le verrez, comme il vous l'a dit.
\VS{8}Elles partirent aussitôt et s'enfuirent du sépulcre. La peur et le trouble les avaient saisies ; et elles ne dirent rien à personne, à cause de la peur.
\VS{9}Jésus étant ressuscité, le matin du premier jour de la semaine{\FTNT{Jésus a-t-il été crucifié un vendredi ? Si c’est le cas, comment a-t-il pu séjourner trois jours dans le tombeau s’il est ressuscité le dimanche matin comme l’enseigne la tradition catholique et la majorité des églises protestantes et évangéliques ? Tout d’abord il convient de signaler que selon Ge. 1, le jour commence au coucher du soleil, aux environs de dix-huit heures, et s’achève le lendemain au coucher du soleil. Chez les Romains, le jour commence à minuit et se termine le lendemain à minuit. C’est de cette manière que l’évangile de Jean compte les heures. Dans les autres évangiles, les journées commencent avec le lever du soleil. Jésus a été crucifié à «~la troisième heure~» (Mc. 15:25), ce qui correspond à neuf heures du matin. Ensuite, les évangiles nous apprennent qu’il y a eu des ténèbres sur la terre de la sixième à la neuvième heure, donc de midi à quinze heures (Mt. 27:45-46 ; Mc. 15:33-34 ; Lu. 23:44). Jésus est donc mort avant dix-huit heures. Ainsi, il est évident qu’il n’a pas pu passer toute la journée du vendredi au tombeau. Les Ecritures ne déclarent pas spécifiquement quel jour de la semaine Jésus a été crucifié. Les deux opinions dominantes sont vendredi et mercredi. D’autres font la synthèse des deux et acceptent le jeudi comme étant le jour de la crucifixion.  Jésus dit dans Mt. 12:40~: «~Car, de même que Jonas fut trois jours et trois nuits dans le ventre d’une baleine, de même le Fils de l'homme sera trois jours et trois nuits dans le sein de la terre~». Ceux qui défendent la crucifixion un vendredi disent qu’il est possible de compter de telle manière qu’on puisse effectivement considérer qu’il a été dans la tombe pendant trois jours. L’argument principal pour le vendredi se trouve dans Mc. 15:42 qui précise que Jésus a été crucifié la «~veille du sabbat~». S’il s’agit bien du sabbat hebdomadaire, c’est à dire le samedi, alors la crucifixion a bien eu lieu un vendredi. Un autre argument en faveur du vendredi se fonde sur des versets tels que Mt. 16:21 et Lu. 9:22 où Jésus enseigne qu’il ressuscitera le troisième jour, ce qui suppose qu’il ne restera pas trois jours et trois nuits entiers dans la tombe. Plusieurs traducteurs utilisent l’expression «~le troisième jour~», mais pas tous. Cependant, aucun d’eux ne conteste la manière de traduire ces versets. Dans Mc. 8:31, il est bien dit que Jésus sera ressuscité «~après~» trois jours.  Le débat sur le jeudi se construit sur celui du vendredi en concluant qu’il y a trop d’événements, selon ses défenseurs, qui se passent entre l’ensevelissement du Christ et dimanche matin, pour que tout se soit déroulé entre vendredi et dimanche matin. Il faut signaler qu’il est particulièrement problématique que le seul jour plein entre vendredi et dimanche soit le samedi. Un jour de plus ou deux résolvent ce problème. L’hypothèse du mercredi avance qu’il y avait deux sabbats cette semaine-là. Après le premier sabbat (celui qui débute le soir de la crucifixion, Mc. 15:42 ; Lu. 23:52-54), les femmes sont allées acheter les aromates. Notez bien qu’elles les ont achetées après le sabbat (Mc. 16:1). Dans cette hypothèse du mercredi, ce premier sabbat est la Pâque (cf. Lé. 16:29-31 ; Lé. 23:24-32,39) car les jours très saints sont aussi appelés sabbats. Le second sabbat de cette semaine était le sabbat hebdomadaire classique, le samedi. Notez que dans Lu. 23:56, les femmes qui avaient acheté les aromates après le premier sabbat s’en retournèrent, préparèrent les aromates puis «~se reposèrent durant le sabbat~» (Lu. 23:56). On ne peut pas imaginer qu’elles ont acheté les aromates après le sabbat et qu’elles les ont préparées avant le sabbat que s’il y a eu deux sabbats cette semaine-là. Avec l’hypothèse des deux sabbats, si le Messie a été crucifié un jeudi, alors le jour très saint (la Pâque) aurait débuté au coucher du soleil le jeudi et pris fin le vendredi au coucher du soleil – juste au début du sabbat hebdomadaire – le samedi. Acheter les aromates après le premier sabbat signifierait alors en faire l’acquisition le samedi, en violation des lois du sabbat.  L’hypothèse du mercredi est la seule qui corrobore les récits bibliques des femmes et des aromates et confirme la prophétie du Seigneur en Mt. 12:40. Le Messie a été arrêté à Gethsémané le mardi soir selon le calendrier romain et le mercredi selon le calendrier hébraïque. Le premier sabbat était un jour très saint, celui de Pâque (Mt. 26 ; Mc. 14 ; Lu. 22), un jeudi selon le calendrier hébraïque. Les femmes achetèrent les aromates le vendredi et s’en retournèrent les préparer le jour même. Elles se sont reposées le samedi, qui était le sabbat hebdomadaire, et ont enfin apporté les aromates au tombeau tôt le dimanche matin. Jésus a été enseveli au moment du coucher du soleil le mercredi, ce qui est le début du jeudi selon le calendrier juif. En usant de ce calendrier, nous avons~: - le jeudi nuit (première nuit), jeudi jour (premier jour) - le vendredi nuit (deuxième nuit), vendredi jour (deuxième jour) - le samedi nuit (troisième nuit), samedi jour (troisième jour). Nous ne savons pas exactement à quelle heure Jésus est ressuscité, mais sous savons que ce fut avant le lever du soleil du dimanche. En effet, Jn. 20:1 nous apprend que Marie de Magdala vint au tombeau «~alors qu’il faisait encore sombre~». Ainsi, Jésus serait ressuscité juste après le coucher du soleil du samedi soir, ce qui correspond au premier jour de la semaine pour les juifs.}} apparut d’abord à Marie de Madgala, de laquelle il avait chassé sept démons.
\VS{10}Elle alla l’annoncer à ceux qui avaient été avec lui, et qui étaient dans le deuil et pleuraient.
\VS{11}Mais quand ils entendirent qu'il était vivant, et qu'elle l'avait vu, ils ne la crurent point.
\VS{12}Après cela, il se montra sous une autre forme à deux d'entre eux, qui étaient en chemin pour aller à la campagne.
\VS{13}Ils revinrent l’annoncer aux autres, mais ils ne les crurent pas non plus.
\VS{14}Enfin, il se montra aux onze, qui étaient assis ensemble, et il leur reprocha leur incrédulité et leur dureté de cœur, parce ce qu'ils n'avaient pas cru ceux qui l'avaient vu ressuscité.
\TextTitle{[Nouvelle mission aux onze apôtres]
\\(Mt. 28:16-20 ; Lu. 24:46-48 ; Jn. 17:18 ; 20:21 ; Ac. 1:8)}
\VS{15}Puis il leur dit : Allez par tout le monde, et prêchez l'Evangile à toute créature.
\VS{16}Celui qui croira et qui sera baptisé, sera sauvé ; mais celui qui ne croira pas sera condamné.
\VS{17}Voici les miracles qui accompagneront ceux qui auront cru : Ils chasseront les démons en mon Nom ; ils parleront de nouvelles langues ;
\VS{18}ils saisiront les serpents avec la main, et s’ils boivent quelque breuvage mortel, il ne leur fera point de mal ; ils imposeront les mains aux malades, et les malades seront guéris.
\TextTitle{[Jésus enlevé au ciel]
\\(Lu. 24:49-53 ; Ac. 1:9-11)}
\VS{19}Le Seigneur, après leur avoir parlé de la sorte, fut enlevé au ciel, et il s'assit à la droite de Dieu.
\VS{20}Et ils s’en allèrent prêcher partout. Le Seigneur travaillait avec eux, et confirmait la parole par les miracles qui l'accompagnaient.
\PPE{}
\end{multicols}

\clearpage\ShortTitle{Luc}\BookTitle{Luc}\BFont
\noindent\hrulefill
{\footnotesize
\textit{
\bigskip
{\centering{}
\\Auteur : Luc
\\(Gr. : Loukas)
\\Signifie : Qui donne la lumière
\\Thème : Jésus le Fils de l'homme
\\Date de rédaction : Env. 60 ap. J.-C.\\}
}
%\bigskip
\textit{
\\D'origine grecque, Luc fut l'auteur de l'évangile éponyme et du livre des « Actes des apôtres ». Celui que Paul appelait le « médecin bien-aimé », et qui fut son compagnon d'œuvre, avait entrepris des investigations visant à narrer avec exactitude la vie terrestre de Jésus-Christ dont il était devenu le disciple, probablement à la suite d'une prédication de Paul. Adressés initialement à Théophile, Luc était loin de penser que ses écrits constitueraient avec le temps une véritable richesse pour l'Eglise et pour le monde.
%\bigskip
\\L'évangile de Luc présente l'humanité parfaite de Jésus, sa compassion et sa miséricorde à l'égard des plus faibles. Rédigé avec rigueur et soin, il retrace le parcours du Fils de l'homme, de sa naissance à son adolescence, puis de sa mort à sa résurrection, et enfin son ascension. Il souligne aussi sa vie de prière et son fardeau pour le salut de l'homme. Par ailleurs, il fait ressortir la manière dont les femmes ont assisté Jésus par leurs biens durant son ministère.
%\bigskip
\\Fruit de recherches minutieuses, le récit de Luc présente certaines similitudes avec ceux de Matthieu et Marc, mais il est le seul à relater la célèbre parabole du fils prodigue, profonde représentation de l'amour du Père.\bigskip
}
}
\par\nobreak\noindent\hrulefill
\begin{multicols}{2}
\Chap{1}
\TextTitle{Introduction}
\VerseOne{}Parce que plusieurs se sont appliqués à mettre par ordre un récit des évènements qui ont été pleinement certifiés parmi nous,
\VS{2}suivant ce que nous ont transmis ceux qui ont été des témoins oculaires dès le commencement et sont devenus des ministres de la parole,
\VS{3}Il m'a aussi semblé bon, après avoir examiné exactement toutes choses depuis le commencement jusqu'à la fin, très excellent Théophile, de te les mettre en ordre par écrit,
\VS{4}afin que tu connaisses la certitude des choses dont tu as été informé.
\TextTitle{Annonce de la naissance de Jean-Baptiste}
\VS{5}Au temps d'Hérode, roi de Judée, il y avait un sacrificateur nommé Zacharie, de la classe d'Abia ; et sa femme était d'entre les filles d'Aaron, et s'appelait Elisabeth.
\VS{6}Et ils étaient tous deux justes devant Dieu, marchant dans tous les commandements, et dans {toutes } les ordonnances du Seigneur, sans reproche.
\VS{7}Et ils n'avaient point d'enfants, parce qu'Elisabeth était stérile, et qu'ils étaient fort avancés en âge.
\VS{8}Or il arriva que comme Zacharie exerçait la sacrificature devant Dieu, selon le tour de sa classe, il fut appelé par le sort,
\VS{9}selon la coutume d'exercer le sacerdoce, à entrer dans le temple du Seigneur pour offrir le parfum.
\VS{10}Toute la multitude du peuple était dehors en prière, à l'heure du parfum.
\VS{11}Et l'ange du Seigneur lui apparut, et se tint debout à droite de l'autel des parfums.
\VS{12}Zacharie fut troublé quand il le vit, et il fut saisi de crainte.
\VS{13}Mais l'ange lui dit : Zacharie, ne crains point ; car ta prière est exaucée. Et Elisabeth, ta femme, t'enfantera un fils, et tu lui donneras le nom de Jean.
\VS{14}Et il sera pour toi le sujet d'une grande joie et d'allégresse, et plusieurs se réjouiront de sa naissance.
\VS{15}Car il sera grand devant le Seigneur. Et il ne boira ni vin, ni boisson forte et il sera rempli du Saint-Esprit dès le ventre de sa mère.
\VS{16}Et il ramènera plusieurs des enfants d'Israël au Seigneur, leur Dieu.
\VS{17}Car il marchera devant lui animé de l'esprit et de la puissance d'Elie, pour ramener les cœurs des pères vers les enfants\FTNT{Mal. 4:6.}, et les rebelles à la sagesse des justes, pour préparer au Seigneur un peuple bien disposé.
\VS{18}Alors Zacharie dit à l'ange : A quoi reconnaîtrai-je cela ? Car je suis vieux, et ma femme est fort âgée.
\VS{19}L'ange répondant lui dit : Je suis Gabriel, je me tiens devant Dieu, et j'ai été envoyé pour te parler, et pour t'annoncer cette bonne nouvelle.
\VS{20}Et voici, tu seras muet, et tu ne pourras point parler jusqu'au jour où ces choses arriveront, parce que tu n'as pas cru à mes paroles qui s'accompliront en leur temps.
\VS{21}Or le peuple attendait Zacharie, et on s'étonnait de ce qu'il tardait tant dans le temple.
\VS{22}Mais quand il fut sorti, il ne pouvait pas leur parler, et ils comprirent qu'il avait eu une vision dans le temple ; car il leur faisait des signes et il resta muet.
\VS{23}Et il arriva que quand les jours de son ministère furent achevés, il retourna dans sa maison.
\VS{24}Et après ces jours-là, Elisabeth sa femme conçut, et elle se cacha l’espace de cinq mois, en disant :
\VS{25}Certes, le Seigneur en a agi avec moi ainsi aux jours qu’il m’a regardée pour ôter mon opprobre d’entre les hommes.
\TextTitle{Annonce de la naissance de Jésus-Christ}
\VS{26}Or au sixième mois, l'ange Gabriel fut envoyé par Dieu dans une ville de Galilée, appelée Nazareth,
\VS{27}vers une vierge fiancée à un homme nommé Joseph, qui était de la maison de David. Et le nom de la vierge était Marie.
\VS{28}Et l'ange étant entré dans le lieu où elle était, lui dit : Je te salue, toi à qui une grâce a été faite. Le Seigneur est avec toi ; tu es bénie parmi les femmes.
\VS{29}Troublée par cette parole, Marie se demandait ce que pouvait signifier une telle salutation.
\VS{30}L'ange lui dit : Marie, ne crains point ; car tu as trouvé grâce devant Dieu.
\VS{31}Et voici, tu concevras en ton ventre, et tu enfanteras un fils, et tu lui donneras le Nom de JESUS.
\VS{32}Il sera grand, et sera appelé le Fils du Très-Haut, et le Seigneur Dieu lui donnera le trône de David, son père.
\VS{33}Il régnera sur la maison de Jacob éternellement, et son règne n'aura pas de fin.
\TextTitle{Naissance miraculeuse de Jésus-Christ}
\VS{34}Alors Marie dit à l'ange : Comment cela se fera-t-il, puisque je ne connais point d'homme ?
\VS{35}L'ange lui répondit et dit : Le Saint-Esprit viendra sur toi, et la puissance du Très-Haut te couvrira de son ombre. C'est pourquoi, le Saint qui naîtra de toi sera appelé Fils de Dieu.
\VS{36}Voici, Elizabeth, ta cousine, a conçu elle aussi un fils en sa vieillesse, celle qui était appelée stérile est dans son sixième mois de grossesse.
\VS{37}Car rien n'est impossible à Dieu.
\VS{38}Et Marie dit : Voici la servante du Seigneur, qu'il me soit fait selon ta parole ! Et l'ange la quitta.
\TextTitle{Marie se rend chez Elisabeth}
\VS{39}Dans ce même temps, Marie se leva, et s'en alla en hâte au pays des montagnes dans une ville de Juda.
\VS{40}Elle entra dans la maison de Zacharie, et salua Elisabeth.
\VS{41}Et il arriva, comme Elisabeth entendait la salutation de Marie, que le petit enfant tressaillit dans son ventre ; et Elisabeth fut remplie de l’Esprit Saint,
\VS{42}Elle s'écria d'une voix forte et dit : Tu es bénie entre les femmes, et béni est le fruit de ton ventre.
\VS{43}Comment m'est-il accordé que la mère de mon Seigneur vienne vers moi ?
\VS{44}Car voici, dès que la voix de ta salutation est parvenue à mes oreilles, le petit enfant a tressailli de joie dans mon ventre.
\VS{45}Heureuse celle qui a cru, parce que les choses qui lui ont été dites par le Seigneur auront leur accomplissement.
\TextTitle{Cantique de Marie\FTNTT{Cp. 1 S. 2:1-10}}
\VS{46}Alors Marie dit : Mon âme magnifie le Seigneur,
\VS{47}et mon esprit se réjouit en Dieu, mon Sauveur.
\VS{48}Car il a jeté les yeux sur la bassesse de sa servante. Voici, certes désormais toutes les générations me diront bienheureuse,
\VS{49}parce que le Tout-Puissant a fait pour moi de grandes choses, et son Nom est Saint.
\VS{50}Et sa miséricorde s'étend de génération en génération en faveur de ceux qui le craignent.
\VS{51}Il a puissamment opéré par son bras. Il a dissipé les desseins que les orgueilleux formaient dans leurs cœurs.
\VS{52}Il a renversé de dessus leurs trônes les puissants, et il a élevé les petits.
\VS{53}Il a rassasié de biens les affamé, il a renvoyé les riches à vide.
\VS{54}Il a pris sous sa protection Israël, son serviteur, et il s'est souvenu de sa miséricorde,
\VS{55}comme il l'avait dit à nos pères, envers Abraham et sa postérité à jamais.
\VS{56}Marie demeura avec elle environ trois mois. Puis elle retourna dans sa maison.
\TextTitle{Naissance de Jean}
\VS{57}Le temps où Elisabeth devait accoucher arriva, et elle enfanta un fils.
\VS{58}Ses voisins et ses parents ayant appris que le Seigneur avait fait éclater sa miséricorde envers elle, s'en réjouissaient avec elle.
\VS{59}Et il arriva qu'au huitième jour, ils vinrent pour circoncire le petit enfant, et ils l'appelaient Zacharie, du nom de son père.
\VS{60}Mais sa mère prit la parole, et dit : Non, mais il sera appelé Jean.
\VS{61}Et ils lui dirent : Il n'y a personne dans ta parenté qui soit appelé de ce nom.
\VS{62}Alors ils firent signe à son père pour savoir comment il voulait qu'on l'appelle.
\VS{63}Et Zacharie ayant demandé des tablettes, écrivit : Jean est son nom. Et tous furent dans l'étonnement.
\VS{64}Au même instant, sa bouche s'ouvrit et sa langue se délia, et il parlait, bénissant Dieu.
\VS{65}Tous ses voisins furent saisis de crainte et toutes ces choses furent divulguées dans tout le pays des montagnes de Judée.
\VS{66}Tous ceux qui les apprirent les gardèrent dans leur cœur, disant : Que sera donc cet enfant ? Et la main du Seigneur était avec lui.
\TextTitle{Cantique de Zacharie}
\VS{67}Alors Zacharie, son père, fut rempli du Saint-Esprit, et il prophétisa en ces mots :
\VS{68}Béni soit le Seigneur, le Dieu d'Israël, de ce qu'il a visité et délivré son peuple
\VS{69}et de ce qu'il nous a suscité un puissant Sauveur dans la maison de David, son serviteur,
\VS{70}selon ce qu'il avait dit par la bouche de ses saints prophètes des temps anciens :
\VS{71}Un Sauveur qui nous délivre de nos ennemis et de la main de tous ceux qui nous haïssent !
\VS{72}C'est ainsi qu'il manifeste sa miséricorde envers nos pères, et se souvient de sa sainte alliance.
\VS{73}Selon le serment par lequel il avait juré à Abraham notre père,
\VS{74}de nous permettre, après que nous serions délivrés de la main de nos ennemis, de le servir sans crainte,
\VS{75}en marchant devant lui dans la sainteté et dans la justice tous les jours de notre vie.
\VS{76}Et toi, petit enfant, tu seras appelé prophète du Très-Haut ; car tu marcheras devant la face du Seigneur, pour préparer ses voies,
\VS{77}afin de donner à son peuple la connaissance du salut, par la rémission de leurs péchés,
\VS{78}grâce aux entrailles de la miséricorde de notre Dieu, en vertu de laquelle le Soleil Levant nous a visités d'en haut,
\VS{79}pour éclairer ceux qui sont assis dans les ténèbres et dans l'ombre de la mort, et pour conduire nos pas dans le chemin de la paix.
\VS{80}Or, le petit enfant croissait, et se fortifiait en esprit. Et il demeura dans les déserts jusqu'au jour où il se présenta à Israël.
\TextTitle{[Naissance de Jésus à Bethléhem]
\\(Mt. 1:18-25 ; 2:1 ; cp. Jn. 1:14}
\Chap{2}
\VerseOne{}En ces jours-là fut publié un édit par César Auguste, ordonnant un recensement de toute la terre.
\VS{2}Ce premier recensement eut lieu pendant que Quirinius était gouverneur de Syrie.
\VS{3}Ainsi, tous allaient pour s'inscrire, chacun dans sa ville.
\VS{4}Joseph aussi monta de Galilée en Judée, de la ville de Nazareth, la ville de David, appelée Bethléhem, parce qu'il était de la maison et de la famille de David ;
\VS{5}afin de se faire inscrire avec Marie, sa fiancée, qui était enceinte.
\VS{6}Pendant qu'ils étaient là, le temps où Marie devait accoucher arriva,
\VS{7}et elle enfanta son fils, premier-né et elle l'emmaillota et le coucha dans une crèche, parce qu'il n'y avait point de place pour eux dans l'hôtellerie.
\TextTitle{L'Ange du Seigneur annonce la naissance de Jésus}
\VS{8}Et il y avait dans cette même contrée des bergers qui couchaient dans les champs, et qui gardaient leur troupeau pendant les veilles de la nuit.
\VS{9}Et voici, l'Ange du Seigneur survint vers eux, et la gloire du Seigneur resplendit autour d'eux, et ils furent saisis d'une grande peur.
\VS{10}Mais l'Ange leur dit : Ne craignez point ; car, voici je vous annonce une bonne nouvelle qui sera un sujet de joie pour tout le peuple :
\VS{11}C'est qu'aujourd'hui, dans la ville de David, vous est né le Sauveur, qui est le Christ, le Seigneur.
\VS{12}Et voici à quel signe vous le reconnaîtrez : Vous trouverez le petit enfant emmailloté, et couché dans une crèche.
\VS{13}Et aussitôt il se joignit à l'Ange une multitude de l'armée céleste, louant Dieu et disant :
\VS{14}Gloire soit à Dieu dans les lieux très-hauts, que la paix soit sur la terre et la bonne volonté dans les hommes !
\TextTitle{Les bergers de Bethléhem}
\VS{15}Et il arriva qu'après que les anges s’en furent allés d’avec eux au ciel, les bergers se dirent les uns aux autres : Allons donc jusqu'à Bethléhem, et voyons cette chose qui est arrivée, ce que le Seigneur nous a fait connaître.
\VS{16}Ils y allèrent donc en hâte, et ils trouvèrent Marie et Joseph, et le petit enfant couché dans une crèche.
\VS{17}Après l'avoir vu, ils divulguèrent ce qui leur avait été dit au sujet de ce petit enfant.
\VS{18}Tous ceux qui les entendirent furent dans l'étonnement de ce que leur disaient les bergers.
\VS{19}Et Marie gardait soigneusement toutes ces choses, et les repassait dans son esprit.
\VS{20}Puis les bergers s'en retournèrent, glorifiant et louant Dieu pour tout ce qu'ils avaient entendu et vu, et qui était conforme à ce qui leur avait été annoncé.
\TextTitle{Jésus circoncis et présenté au temple de Jérusalem\FTNTT{Cp. Ex. 13:12,15}}
\VS{21}Et quand les huit jours furent accomplis pour circoncire l’enfant, on lui donna le Nom de Jésus, nom qu'avait indiqué l'Ange avant qu'il soit conçu dans le sein de sa mère.
\VS{22}Et quand les jours de la purification\FTNT{Lé : 12:2-6.} de Marie furent accomplis selon la loi de Moïse, Joseph et Marie le portèrent à Jérusalem, pour le présenter au Seigneur,
\VS{23}selon ce qui est écrit dans la loi du Seigneur : Tout mâle premier-né sera appelé Saint au Seigneur\FTNT{Ex. 13:2 ; Ex. 13:12 ; No. 3:13 ; No. 8:17.} ;
\VS{24}et pour offrir en sacrifice deux tourterelles ou deux jeunes pigeons comme cela est prescrit dans la loi du Seigneur\FTNT{Lé. 12:8.}.
\TextTitle{Adoration de Siméon et sa prophétie}
\VS{25}Et voici, il y avait à Jérusalem un homme appelé Siméon. Et cet homme était juste et pieux, il attendait la consolation d'Israël, et le Saint-Esprit était sur lui.
\VS{26}Il avait été averti divinement par le Saint-Esprit qu'il ne mourrait point avant d'avoir vu le Christ du Seigneur.
\VS{27}Il vint au temple, poussé par l'Esprit. Et comme les parents apportaient dans le temple l'enfant Jésus, pour accomplir à son égard ce qu'ordonnait la loi,
\VS{28}il le prit dans ses bras, bénit Dieu, et dit :
\VS{29}Seigneur, tu laisses maintenant ton serviteur s'en aller en paix selon ta parole.
\VS{30}Car mes yeux ont vu ton salut.
\VS{31}Lequel tu as préparé devant la face de tous les peuples.
\VS{32}La lumière pour éclairer les nations ; et pour être la gloire de ton peuple d'Israël.
\VS{33}Joseph et sa mère s'étonnaient des choses qui étaient dites de lui.
\VS{34}Siméon le bénit, et dit à Marie, sa mère : Voici, cet enfant est destiné à être une occasion de chute et de relèvement de beaucoup en Israël, et à devenir un signe qui provoquera la contradiction,
\VS{35}en sorte que les pensées de beaucoup de cœurs seront découvertes. Et pour toi, une épée te transpercera l'âme.
\TextTitle{Anne témoigne du Messie}
\VS{36}Il y avait aussi Anne, la prophétesse, fille de Phanuel de la tribu d'Aser, qui était déjà avancée en âge, et qui avait vécu avec son mari sept ans depuis sa virginité.
\VS{37}Restée veuve, et âgée d'environ quatre-vingt-quatre ans, elle ne quittait pas le temple, et elle servait Dieu nuit et jour dans le jeûne et dans les prières.
\VS{38}Etant arrivée à cette heure, elle louait aussi Dieu, et parlait de lui à tous ceux qui attendaient la délivrance de Jérusalem.
\TextTitle{Retour à Nazareth\FTNTT{Suite aux évènements de Mt. 2.}}
\VS{39}Et quand ils eurent accompli tout ce qui est ordonné par la loi du Seigneur, ils s'en retournèrent en Galilée, à Nazareth, leur ville.
\VS{40}Et le petit enfant croissait et se fortifiait en esprit. Il était rempli de sagesse, et la grâce de Dieu était sur lui.
\TextTitle{Jésus assis dans le temple de Jérusalem au milieu des docteurs}
\VS{41}Ses parents allaient tous les ans à Jérusalem, à la fête de Pâque.
\VS{42}Lorsqu'il fut âgé de douze ans, ses parents montèrent à Jérusalem selon la coutume de la fête.
\VS{43}Puis, quand les jours furent écoulés, et qu'ils s'en retournèrent, l'enfant Jésus resta à Jérusalem. Et son père et sa mère ne s'en aperçurent point.
\VS{44}Mais croyant qu'il était avec leurs compagnons de voyage, ils marchèrent une journée, puis ils le cherchèrent parmi leurs parents et parmis leur connaissance.
\VS{45}Et ne le trouvant point, ils retournèrent à Jérusalem, pour le chercher.
\VS{46}Or il arriva que trois jours après, ils le trouvèrent dans le temple, assis au milieu des docteurs, les écoutant, et les interrogeant.
\VS{47}Tous ceux qui l'entendaient s'étonnaient de sa sagesse et de ses réponses.
\VS{48}Quand ses parents le virent, ils furent saisis d'étonnement, et sa mère lui dit : Mon enfant, pourquoi nous as-tu fait ainsi ? Voici, ton père et moi te cherchions avec angoisse.
\VS{49}Et il leur dit : Pourquoi me cherchiez-vous ? Ne saviez-vous pas qu'il faut que je m'occupe des affaires de mon Père ?
\VS{50}Mais ils ne comprirent point ce qu'il leur disait.
\VS{51}Alors il descendit avec eux, et vint à Nazareth ; et il leur était soumis. Et sa mère gardait toutes ces paroles dans son cœur.
\TextTitle{Jésus grandit en sagesse, en stature et en grâce}
\VS{52}Et Jésus croissait en sagesse, en stature, et en grâce, au près de Dieu et devant les hommes.
\Chap{3}
\TextTitle{Ministère de Jean-Baptiste\FTNTT{Mt. 3:1-12 ; Mc. 1:1-8 ; Jn. 1:6-8,15-37}}
\VerseOne{}La quinzième année du règne de Tibère César, lorsque Ponce Pilate était gouverneur de la Judée, Hérode tétrarque de la Galilée, et son frère Philippe tétrarque de l'Iturée et du territoire de la Trachonite, et Lysanias tétrarque de l'Abilène,
\VS{2}et du temps des souverains sacrificateurs Anne et Caïphe, la parole de Dieu fut adressée à Jean, fils de Zacharie, dans le désert.
\VS{3}Et il alla dans tout le pays des environs du Jourdain, prêchant le baptême de repentance, pour la rémission des péchés,
\VS{4}comme il est écrit dans le livre des paroles d'Esaïe, le prophète disant : C'est la voix de celui qui crie dans le désert : Préparez le chemin du Seigneur, aplanissez ses sentiers.
\VS{5}Toute vallée sera comblée, et toute montagne et toute colline seront abaissées, et ce qui est tortueux sera redressé, et les chemins raboteux seront aplanis.
\VS{6}Et toute chair verra le salut de Dieu\FTNT{Es. 40:3-5.}.
\VS{7}Il disait donc à ceux qui venaient en foule pour être baptisés par lui : Races de vipères, qui vous a appris à fuir la colère à venir ?
\VS{8}Produisez donc des fruits dignes de la repentance, et ne vous mettez point à dire en vous-mêmes : Nous avons Abraham pour père. Car je vous dis que Dieu peut faire naître, même de ces pierres, des enfants à Abraham.
\VS{9}Or la cognée est déjà mise à la racine des arbres ; tout arbre donc qui ne produit pas de bon fruit, sera coupé, et jeté au feu.
\VS{10}Alors la foule l'interrogeait, disant : Que ferons-nous donc ?
\VS{11}Et il répondit, et leur dit : Que celui qui a deux tuniques partage avec celui qui n'en a point ; et que celui qui a de quoi manger en fasse de même.
\VS{12}Il vint aussi à lui des publicains pour être baptisés, et ils lui dirent : Maître, que ferons-nous ?
\VS{13}Et il leur dit : N'exigez rien au-delà de ce qui vous a été ordonné.
\VS{14}Des soldats l'interrogèrent aussi, disant : Et nous, que ferons-nous ? Et Il leur répondit : Ne commettez ni extorsion ni fraude envers personne, mais contentez-vous de votre solde.
\VS{15}Et comme le peuple était dans l'attente, et que tous se demandaient dans leurs cœurs si Jean n'était pas le Christ,
\VS{16}Jean prit la parole, et dit à tous : Moi, je vous baptise d'eau ; mais il vient, celui qui est plus puissant que moi, et je ne suis pas digne de délier la courroie de ses souliers. Lui, il vous baptisera du Saint-Esprit et de feu.
\VS{17}Il a son van à la main ; il nettoiera entièrement son aire, et amassera le froment dans son grenier, mais il brûlera la paille dans un feu qui ne s'éteint point.
\VS{18}Et faisant aussi plusieurs autres exhortations, il évangélisait le peuple.
\VS{19}Mais Hérode le tétrarque, étant repris par Jean au sujet d'Hérodias, femme de Philippe son frère, et à cause de toutes les choses méchantes qu'il faisait,
\VS{20}ajouta encore à toutes les autres celle de mettre Jean en prison.
\TextTitle{Baptême de Jésus-Christ\FTNTT{Mc. 1:9-11; cp. Jn. 1:31-34}}
\VS{21}Tout le peuple se faisait baptiser, Jésus aussi fut baptisé, et pendant qu'il priait, le ciel s'ouvrit,
\VS{22}et le Saint-Esprit descendit sur lui sous une forme corporelle, comme celle d'une colombe. Et une voix fit entendre du ciel ces paroles : Tu es mon Fils bien-aimé, en toi j'ai trouvé mon plaisir.
\TextTitle{Généalogie de Jésus-Christ\FTNTT{v. 31 ; Mt. 1:1-16}}
\VS{23}Jésus avait environ trente ans, lorsqu'il commença son ministère, étant comme on l'estimait, fils de Joseph, fils d'Héli,
\VS{24}fils de Matthat, fils de Lévi, fils de Melchi, fils de Jannaï, fils de Joseph,
\VS{25}fils de Mattathias, fils d'Amos, fils de Nahum, fils d'Esli, fils de Naggaï,
\VS{26}fils de Maath, fils de Mattathias, fils de Sémeï, fils de Josech, fils de Joda,
\VS{27}fils de Joanan, fils de Rhésa, fils de Zorobabel, fils de Salathiel, fils de Néri,
\VS{28}fils de Melchi, fils d'Addi, fils de Kosam, fils d'Elmadam, fils d'Er,
\VS{29}fils de Jésus, fils d'Eliézer, fils de Jorim, fils de Matthat, fils de Lévi,
\VS{30}fils de Siméon, fils de Juda, fils de Joseph, fils de Jonam, fils d'Eliakim,
\VS{31}fils de Méléa, fils de Menna, fils de Matthata, fils de Nathan, fils de David,
\VS{32}fils d'Isaï, fils d'Obed, fils de Booz, fils de Salmon, fils de Naasson,
\VS{33}fils d'Aminadab, fils d’Admin, fils d'Aram, fils d'Esrom, fils de Pérets, fils de Juda,
\VS{34}fils de Jacob, fils d'Isaac, fils d'Abraham, fils de Thara, fils de Nachor,
\VS{35}fils de Seruch, fils de Ragau, fils de Phalek, fils d'Eber, fils de Sala,
\VS{36}fils de Kaïnam, fils d’Arphaxad, fils de Sem, fils de Noé, fils de Lamech,
\VS{37}fils de Mathusala, fils d'Hénoc, fils de Jared, fils de Maléléel, fils de Kaïnan,
\VS{38}fils d'Enos, fils de Seth, fils d'Adam, fils de Dieu.
\Chap{4}
\TextTitle{Tentation de Jésus-Christ\FTNTT{Mt. 4:1 ; Mc. 1:12-13 ; cp. Ge. 3:6 ; 1 Jn. 2:16}}
\VerseOne{}Jésus, rempli du Saint-Esprit, revint du Jourdain, et il fut conduit par l'Esprit dans le désert,
\VS{2}où il fut tenté par le diable quarante jours. Et il ne mangea rien durant ces jours-là, et après qu'ils furent écoulés, il eut faim.
\VS{3}Le diable lui dit : Si tu es le Fils de Dieu, ordonne à cette pierre qu'elle devienne du pain.
\VS{4}Jésus lui répondit, en disant : Il est écrit que l'homme ne vivra pas seulement de pain, mais de toute parole de Dieu\FTNT{De. 8:3.}.
\VS{5}Alors le diable l'emmena sur une haute montagne, et lui montra en un instant tous les royaumes de la terre,
\VS{6}et le diable lui dit : Je te donnerai toute cette puissance et leur gloire ; car elle m'a été donnée, et je la donne à qui je veux.
\VS{7}Si donc tu m'adores, elle sera à toi.
\VS{8}Jésus lui répondit : Va arrière de moi, Satan ! Car il est écrit : Tu adoreras le Seigneur ton Dieu, et tu le serviras lui seul\FTNT{De. 6:13.}.
\VS{9}Le diable le conduisit encore à Jérusalem, et le plaça sur le haut du temple, et lui dit : Si tu es le Fils de Dieu, jette-toi d'ici en bas.
\VS{10}Car il est écrit : Il ordonnera à ses anges à ton sujet, afin qu'ils te gardent;
\VS{11}et ils te porteront dans leurs mains, de peur que ton pied ne heurte contre une pierre\FTNT{Ps. 91:11-12.}.
\VS{12}Mais Jésus répondant, lui dit : Il est écrit : Tu ne tenteras pas le Seigneur, ton Dieu\FTNT{De. 6:16.}.
\VS{13}Après l'avoir tenté de toutes ces manières, le diable s'éloigna de lui pour un temps.
\TextTitle{Jésus-Christ retourne en Galilée\FTNTT{Mt. 4:12-17 ; Mc. 1:14-15}}
\VS{14}Jésus retourna en Galilée dans la puissance de l'Esprit, et sa renommée se répandit dans tout le pays d'alentour.
\VS{15}Il enseignait dans leurs synagogues, et il était glorifié par tous.
\TextTitle{Jésus dans la synagogue de Nazareth le jour du sabbat\FTNTT{cp. Mt. 13:54-58 ; Mc. 6:1-6}}
\VS{16}Il se rendit à Nazareth, où il avait été élevé, et selon sa coutume, il entra dans la synagogue le jour du sabbat et il se leva pour faire la lecture,
\VS{17}et on lui donna le livre du prophète Esaïe et l'ayant déroulé, il trouva le passage où il est écrit :
\VS{18}L'Esprit du Seigneur est sur moi, parce qu'il m'a oint pour évangéliser les pauvres ; il m'a envoyé pour guérir ceux qui ont le cœur brisé,
\VS{19}pour proclamer aux captifs la délivrance, et aux aveugles le recouvrement de la vue ; pour mettre en liberté les opprimés ; pour publier une année de grâce du Seigneur\FTNT{Es. 61:1-2.}.
\VS{20}Ensuite, il roula le livre, le rendit au serviteur, et s'assit. Les yeux de tous ceux qui étaient dans la synagogue étaient fixés sur lui.
\VS{21}Alors il commença à leur dire : Aujourd'hui, cette parole de l'Ecriture que vous venez d'entendre, est accomplie.
\VS{22}Et tous lui rendaient témoignage, et s'étonnaient des paroles pleines de grâce qui sortaient de sa bouche ; et ils disaient : Celui-ci n'est-il pas le fils de Joseph ?
\VS{23}Et il leur dit : Assurément vous me direz ce proverbe : Médecin, guéris-toi toi-même. Et fais ici, dans ton pays, tout ce que nous avons appris que tu as fait à Capernaüm.
\VS{24}Mais il leur dit : En vérité je vous dis qu’aucun prophète n'est reçu dans son pays.
\VS{25}Je vous le dis en vérité : Il y avait plusieurs veuves en Israël, du temps d'Elie, lorsque le ciel fut fermé trois ans et six mois et qu'il y eut une grande famine dans tout le pays ;
\VS{26}toutefois Elie ne fut envoyé vers aucune d'elles, mais seulement vers une femme veuve à Sarepta, dans le pays de Sidon.
\VS{27}Il y avait aussi plusieurs lépreux en Israël du temps d'Elisée, le prophète, toutefois aucun d'eux ne fut purifié, si ce n'est Naaman, le Syrien.
\VS{28}Ils furent tous remplis de colère dans la synagogue lorsqu'ils entendirent ces choses.
\VS{29}Et s'étant levés, ils le chassèrent hors de la ville, et le menèrent jusqu'au bord de la montagne sur laquelle leur ville était bâtie, pour le jeter du haut en bas.
\VS{30}Mais il passa au milieu d'eux, et s'en alla.
\TextTitle{Jésus guérit un possédé\FTNTT{Mc. 1:21-28}}
\VS{31}Il descendit à Capernaüm, ville de Galilée, et il les enseignait les jours de sabbat.
\VS{32}Ils étaient frappés de sa doctrine ; car il parlait avec autorité.
\VS{33}Il y avait dans la synagogue un homme qui avait un esprit de démon impur, et qui s'écria d'une voix forte,
\VS{34}en disant : Ah ! Qu'y a-t-il entre nous et toi, Jésus de Nazareth ? Es-tu venu pour nous détruire ? Je sais qui tu es, le Saint de Dieu.
\VS{35}Jésus le menaça, en lui disant : Tais-toi, et sors de cet homme. Et le démon, l'ayant jeté avec impétuosité au milieu de l'assemblée, sortit de cet homme, sans lui faire aucun mal.
\VS{36}Et tous furent saisis de stupeur, et ils parlaient entre eux, et disaient : Quelle est cette parole ? Il commande avec autorité et puissance aux esprits impurs, et ils sortent ?
\VS{37}Et sa renommée se répandit dans tous les lieux d'alentour.
\TextTitle{Guérison de la belle-mère de Pierre et de plusieurs malades\FTNTT{Mt. 8:14-17 ; Mc. 1:29-34}}
\VS{38}Et quand Jésus se fut levé de la synagogue, et il se rendit à la maison de Simon, et la belle-mère de Simon avait une violente fièvre, et ils le prièrent en sa faveur.
\VS{39}Et s'étant penché sur elle, il menaça la fièvre, et la fièvre la quitta. A l'instant elle se leva, et les servit.
\VS{40}Et après le coucher du soleil, tous ceux qui avaient des malades atteints de diverses maladies, les lui amenèrent. Il imposa les mains à chacun d'eux, et il les guérit.
\VS{41}Les démons aussi sortirent de beaucoup de personnes, en criant et en disant : Tu es le Christ, le Fils de Dieu. Mais il les menaçait fortement, et ne leur permettait pas de dire qu'ils savaient qu'il était le Christ.
\VS{42}Dès que le jour parut, il sortit et alla dans un lieu désert et une foule de gens se mirent à sa recherche, et arrivèrent jusqu'à lui et ils voulaient le retenir, afin qu'il ne les quittât point.
\VS{43}Mais il leur dit : Il faut que j'annonce aux autres villes l'Evangile du Royaume de Dieu, car c'est pour cela que j'ai été envoyé.
\VS{44}Et il prêchait dans les synagogues de la Galilée.
\Chap{5}
\TextTitle{Appel des premiers disciples\FTNTT{Mt. 4:18-22 ; Mc. 1:16-20 ; cp. Jn. 1:35-51 ; 21:1-8}}
\VerseOne{}Or il arriva, comme la foule se jetait toute sur lui pour entendre la parole de Dieu, qu’il se tenait sur le bord du lac de Génézareth.
\VS{2}Et voyant deux barques qui étaient au bord du lac, et dont les pêcheurs étaient descendus, et lavaient leurs rets, il monta dans l’une de ces barques, qui était à Simon.
\VS{3}Il monta dans l'une de ces barques, qui était à Simon, et il le pria de s'éloigner un peu de terre. Puis il s'assit, et de la barque il enseignait la foule.
\VS{4}Et quand il eut cessé de parler, il dit à Simon : Avance en pleine eau, et jetez vos filets pour pêcher.
\VS{5}Et Simon répondant, lui dit : Maître, nous avons travaillé toute la nuit, et nous n’avons rien pris ; toutefois à ta parole je jetterai les filets.
\VS{6}Et ayant fait cela, ils prirent une si grande quantité de poissons que leur filet se rompait.
\VS{7}Et ils firent signe à leurs compagnons qui étaient dans l’autre barque, de venir les aider ; et étant venus, ils remplirent les deux barques, tellement qu’elles s’enfonçaient.
\VS{8}Et quand Simon Pierre vit cela, il se jeta aux genoux de Jésus, en lui disant : Seigneur, retire-toi de moi ; car je suis un homme pécheur.
\VS{9}Parce que la frayeur l’avait saisi, lui et tous ceux qui étaient avec lui, à cause de la prise de poissons qu’ils venaient de faire ; de même que Jacques et Jean, fils de Zébédée, qui étaient compagnons de Simon.
\VS{10}Alors Jésus dit à Simon : Ne crains point ; désormais tu seras un pêcheur d'hommes vivants.
\VS{11}Et quand ils eurent ramené les barques à terre, ils abandonnèrent tout et le suivirent.
\TextTitle{Guérison d'un lépreux\FTNTT{Mt. 8:2-4 ; Mc. 1:40-45}}
\VS{12}Et il arriva, comme il était dans une des villes, voici un homme plein de lèpre, voyant Jésus, se jeta sur sa face et le supplia, disant : Seigneur, si tu veux, tu peux me rendre pur.
\VS{13}Jésus étendit la main, et le toucha, en disant : Je le veux, sois pur. Aussitôt la lèpre le quitta.
\VS{14}Et il lui commanda de ne le dire à personne, mais va, lui dit-il, et montre-toi sacrificateur, et offre pour ta purification ce que Moïse a commandé\FTNT{Lé. 13 et 14.}, pour leur servir de témoignage.
\VS{15}Et sa renommée se répandait de plus en plus, tellement que de grandes foules s’assemblaient pour l’entendre, et pour être guéries par lui de leurs maladies.
\VS{16}Mais il se tenait retiré dans les déserts, et priait.
\TextTitle{Guérison d'un paralytique\FTNTT{Mt. 9:2-8 ; Mc. 2:3-12}}
\VS{17}Un jour Jésus enseignait. Et des pharisiens et des docteurs de la loi étaient là assis, venus de tous les villages de la Galilée, et de la Judée et de Jérusalem ; et la puissance du Seigneur se manifestait par des guérisons.
\VS{18}Et voici des hommes qui portaient sur un lit un homme qui était paralytique, et ils cherchaient le moyen de le porter dans la maison, et de le mettre devant lui.
\VS{19}Comme ils ne savaient pas par où l'introduire, à cause de la foule, ils montèrent sur le toit, et ils le descendirent par une ouverture, avec son lit, au milieu de la foule, devant Jésus.
\VS{20}Voyant leur foi, il dit au paralytique : Homme, tes péchés te sont pardonnés.
\VS{21}Alors les scribes et les pharisiens commencèrent à raisonner en eux-mêmes, disant : Qui est celui-ci qui profère des blasphèmes ? Qui est-ce qui peut pardonner les péchés, si ce n'est Dieu seul ?
\VS{22}Mais Jésus, connaissant leurs pensées, prit la parole et leur dit : Pourquoi raisonnez-vous ainsi en vous-mêmes ?
\VS{23}Lequel est le plus aisé de dire : Tes péchés te sont pardonnés ; ou de dire : Lève-toi et marche ?
\VS{24}Or afin que vous sachiez que le Fils de l'homme a le pouvoir sur la terre de pardonner les péchés, il dit au paralytique : Je te l'ordonne, lève-toi, prends ton lit, et va dans ta maison.
\VS{25}Et à l'instant, le paralytique s'étant levé devant eux, prit le lit sur lequel il était couché, et s'en alla dans sa maison, glorifiant Dieu.
\VS{26}Ils furent tous saisis d'étonnement, et ils glorifiaient Dieu ; et étant remplis de crainte, ils disaient : certainement nous avons vu aujourd'hui des choses étranges.
\TextTitle{Appel de Lévi\FTNTT{Mt. 9:9 ; Mc. 2:13-14}}
\VS{27}Après cela, Jésus sortit, et il vit un publicain nommé Lévi, assis au bureau des péages, et il lui dit : Suis-moi.
\VS{28}Et abandonnant tout, il se leva, et le suivit.
\TextTitle{Appelle des pécheurs à la repentance\FTNTT{Mt. 9:10-15 ; Mc. 2:13-14}}
\VS{29}Et Lévi lui fit un grand festin dans sa maison ; et il y avait une grande foule de publicains et d’autres gens qui étaient avec eux à table.
\VS{30}Les scribes de ce lieu-là et les pharisiens, murmuraient contre ses disciples en disant : Pourquoi mangez-vous et buvez-vous avec les publicains et les gens de mauvaise vie ?
\VS{31}Mais Jésus, prenant la parole, leur dit : ceux qui sont en santé n’ont pas besoin de médecin, mais ceux qui se portent mal.
\VS{32}Je ne suis point venu appeler à la repentance les justes, mais les pécheurs.
\VS{33}Ils lui dirent aussi : pourquoi est-ce que les disciples de Jean jeûnent souvent, et font des prières et également ceux des Pharisiens, mais les tiens mangent et boivent ?
\VS{34}Il leur répondit : Pouvez-vous faire jeûner les amis de l'Epoux pendant que l'Epoux est avec eux ?
\VS{35}Mais les jours viendront où l'Epoux leur sera enlevé alors ils jeûneront en ces jours-là.
\TextTitle{Parabole du drap neuf et des outres neuves\FTNTT{Mt. 9:16-17 ; Mc. 2:21-22}}
\VS{36}Puis il leur dit cette parabole : Personne ne met une pièce d'un habit neuf à un vieil habit ; autrement le neuf déchire le vieux, et la pièce du neuf ne s'accorde pas avec le vieux.
\VS{37}Et personne ne met du vin nouveau dans de vieilles outres ; autrement le vin nouveau fait rompre les outres, et il se répand, et les outres sont perdues.
\VS{38}Mais le vin nouveau doit être mis dans des outres neuves ; et ainsi ils se conservent l'un et l'autre.
\VS{39}Et personne, après avoir bu du vin vieux, ne veut du nouveau, car il dit : Le vieux est meilleur.
\Chap{6}
\TextTitle{Jésus, le Maître du sabbat\FTNTT{Mt. 12:1-8 ; Mc. 2:23-28}}
\VerseOne{}Or il arriva un jour de sabbat appelé second-premier, qu'il passait par des blés ; et ses disciples arrachaient des épis et les froissant dans leurs mains, ils les mangeaient.
\VS{2}Et quelques pharisiens leur dirent : Pourquoi faites-vous ce qu'il n'est pas permis de faire les jours du sabbat ?
\VS{3}Et Jésus prenant la parole, leur dit : N'avez-vous pas lu ce que fit David quand il eut faim, lui et ceux qui étaient avec lui ;
\VS{4}comment il entra dans la maison de Dieu, et prit les pains de proposition, et en mangea, et en donna aussi à ceux qui étaient avec lui, bien qu'il ne soit permis qu'aux sacrificateurs d'en manger ?\FTNT{2 S. 21:1-7.}
\VS{5}Puis il leur dit : Le Fils de l'homme est Maître même du sabbat.
\TextTitle{Guérison d'un homme à la main sèche\FTNTT{Mt. 12:9-13 ; Mc. 3:1-5}}
\VS{6}Et il arriva, un autre jour de sabbat, qu'il entra dans la synagogue, et qu'il enseignait et il s'y trouvait là un homme dont la main droite était sèche.
\VS{7}Or les scribes et les pharisiens l'observaient pour voir s'il ferait une guérison le jour du sabbat ; c'était afin d'avoir sujet de l'accuser.
\VS{8}Mais il connaissait leurs pensées et il dit à l'homme qui avait la main sèche : Lève-toi, et tiens-toi debout au milieu. Et il se leva et se tint debout.
\VS{9}Puis Jésus leur dit : Je vous demande une chose : Est-il permis de faire du bien les jours de sabbat, ou de faire du mal ? De sauver une personne, ou de la laisser mourir ?
\VS{10}Et ayant regardé tous ceux qui étaient autour de lui, il dit à l’homme : étends ta main ; ce qu’il fit, et sa main fut rendue saine comme l’autre.
\VS{11}Et ils furent remplis de fureur, et ils s’entretenaient ensemble touchant ce qu’ils pourraient faire à Jésus.
\VS{12}Or il arriva en ces jours-là, qu’il s’en alla sur une montagne pour prier, et qu’il passa toute la nuit à prier Dieu.
\TextTitle{Choix des douze apôtres\FTNTT{cp. Mt. 10:2-4 ; Mc. 3:13-19}}
\VS{13}Et quand le jour fut venu, il appela ses disciples. Et en ayant choisi douze d’entre eux, il les nomma apôtres :
\VS{14}Simon, qu'il nomma Pierre, et André son frère, Jacques et Jean, Philippe et Barthélemy ;
\VS{15}Matthieu et Thomas, Jacques fils d'Alphée, et Simon surnommé zélote\FTNT{Zélote : « Celui qui est zélé ». Les zélotes faisaient partie d'un mouvement politique Juif du premier siècle ap. J.-C., qui cherchait à inciter les gens de la province de Judée à se rebeller contre l'Empire Romain, et à le chasser du pays par les armes, pendant la Grande Révolte Juive (66-70 ap. J.-C.). Lorsque les Romains introduisirent le culte impérial, les Juifs se rebellèrent et furent réprimés. Les zélotes considéraient qu'Israël appartenait seulement à un roi Juif de la descendance de David. De plus, reconnaître l'empereur équivalait, à leurs yeux, à renier Dieu. Le mouvement zélote se réclamait intentionnellement de modèles bibliques tels que Phinées, le fils zélé d'Eléazar, fils d'Aaron (No. 25:11). Ce dernier s'était illustré par l'assassinat d'un prince de tribu d'Israël qui s'était fourvoyé dans la luxure aux yeux de tous.} ;
\VS{16}Jude, frère de Jacques, et Judas Iscariot, qui devint traître.
\TextTitle{Enseignement sur la montagne\FTNTT{Mt. 5-7}}
\VS{17}Puis descendant avec eux, il s'arrêta sur une plaine avec la foule de ses disciples et une grande multitude de peuple de toute la Judée, et de Jérusalem, et de la contrée maritime de Tyr et de Sidon, qui étaient venus pour l'entendre, et pour être guéris de leurs maladies.
\VS{18}Ceux aussi qui étaient tourmentés par des esprits impurs furent guéris.
\VS{19}Et toute la foule cherchait à le toucher, parce qu'une force sortait de lui et les guérissait tous.
\TextTitle{Enseignement de Jésus\FTNTT{Mt. 5:3-12}}
\VS{20}Alors Jésus, levant les yeux vers ses disciples, leur dit : Heureux vous qui êtes pauvres, car le Royaume de Dieu vous appartient.
\VS{21}Heureux vous qui avez faim maintenant, car vous serez rassasiés ! Heureux vous qui pleurez maintenant, car vous serez dans la joie !
\VS{22}Heureux serez-vous quand les hommes vous haïront, vous chasseront, vous outrageront, et rejetteront votre nom comme infâme, à cause du Fils de l'homme.
\VS{23}Réjouissez-vous en ce jour-là, et tressaillez d'allégresse, parce que votre récompense sera grande dans le ciel ; car leurs pères en faisaient de même aux prophètes.
\VS{24}Mais malheur à vous riches, car vous avez votre consolation.
\VS{25}Malheur à vous qui êtes rassasiés, car vous aurez faim. Malheur à vous qui riez maintenant, car vous serez dans le deuil et dans les larmes.
\VS{26}Malheur à vous quand tous les hommes diront du bien de vous ; car leurs pères en faisaient de même aux faux prophètes.
\VS{27}Mais à vous qui m’entendez, je vous dis : aimez vos ennemis ; faites du bien à ceux qui vous haïssent.
\VS{28}Bénissez ceux qui vous maudissent, et priez pour ceux qui vous maltraitent.
\VS{29}Si quelqu'un te frappe sur une joue, présente-lui aussi l'autre. Si quelqu'un prend ton manteau, ne l'empêche pas de prendre aussi ta tunique.
\VS{30}Donne à quiconque te demande, et ne réclame pas ton bien à celui qui s'en empare.
\VS{31}Ce que vous voulez que les hommes fassent pour vous, faites-le de même pour eux.
\VS{32}Mais si vous aimez seulement ceux qui vous aiment, quel gré vous en saura-t-on ? Les pécheurs aussi aiment ceux qui les aiment.
\VS{33}Et si vous faites du bien à ceux qui vous font du bien, quel gré vous en saura-t-on ? Les pécheurs aussi font de même.
\VS{34}Et si vous prêtez à ceux de qui vous espérez recevoir, quel gré vous en saura-t-on ? Les pécheurs aussi prêtent aux pécheurs, afin de recevoir la pareille.
\VS{35}C'est pourquoi aimez vos ennemis et faites-leur du bien, et prêtez sans rien espérer, et votre récompense sera grande, et vous serez les fils du Très-Haut, car il est bon envers les ingrats et les méchants.
\VS{36}Soyez donc miséricordieux comme votre Père est miséricordieux.
\VS{37}Ne jugez point, et vous ne serez point jugés ; ne condamnez point, et vous ne serez point condamnés ; absolvez, et vous serez absous.
\VS{38}Donnez, et il vous sera donné : On versera dans votre sein une bonne mesure, serrée, secouée et qui déborde ; car on vous mesurera avec la mesure dont vous vous serez servis.
\VS{39}Il leur disait aussi cette parabole : Un aveugle peut-il conduire un aveugle ? Ne tomberont-ils pas tous deux dans la fosse ?
\VS{40}Le disciple n'est pas au-dessus de son maître ; mais tout disciple accompli sera comme son maître.
\VS{41}Pourquoi regardes-tu la paille qui est dans l'œil de ton frère, et n'aperçois-tu pas la poutre qui est dans ton propre œil ?
\VS{42}Ou comment peux-tu dire à ton frère : Mon frère, laisse-moi enlever la paille qui est dans ton œil, toi qui ne vois pas la poutre qui est dans ton œil ? Hypocrite, ôte premièrement la poutre de ton œil, et après cela tu verras comment ôter la paille qui est dans l'œil de ton frère.
\VS{43}Ce n'est pas un bon arbre qui porte du mauvais fruit, ni un mauvais arbre qui porte du bon fruit.
\VS{44}Car chaque arbre se reconnaît à son fruit. On ne cueille pas des figues sur des épines, et l'on ne vendange pas des raisins sur des ronces.
\VS{45}L'homme de bien tire de bonnes choses du bon trésor de son cœur, et l'homme méchant tire de mauvaises choses du mauvais trésor de son cœur ; car c'est de l'abondance du cœur que la bouche parle.
\TextTitle{Parabole des deux bâtisseurs et des deux fondements\FTNTT{Mt. 7:24-27}}
\VS{46}Mais pourquoi m'appelez-vous Seigneur, Seigneur, et ne faites-vous pas ce que je dis ?
\VS{47}Je vous montrerai à qui est semblable celui qui vient à moi, entend mes paroles, et les met en pratique.
\VS{48}Il est semblable à un homme qui bâtissant une maison, a creusé, creusé profondément, et a mis le fondement sur le roc. Une inondation est venue, et le torrent s'est jeté contre cette maison, sans pouvoir l'ébranler, parce qu'elle était bâtie sur le roc.
\VS{49}Mais celui qui entend mes paroles, et ne les met pas en pratique, est semblable à un homme qui a bâti sa maison sur la terre, sans fondement. Le torrent s'est jeté contre elle ; aussitôt elle est tombée, et la ruine de cette maison a été grande.
\Chap{7}
\TextTitle{Guérison du serviteur d'un centenier\FTNTT{Mt. 8:5-13}}
\VerseOne{}Et quand il eut achevé tout ce discours devant le peuple qui l'écoutait, il entra dans Capernaüm.
\VS{2}Un centenier avait un serviteur, auquel il était très attaché, et qui était malade, sur le point de mourir.
\VS{3}Ayant entendu parler de Jésus, il envoya vers lui quelques anciens des Juifs, pour le prier de venir guérir son serviteur.
\VS{4}Et étant venu à Jésus, ils lui prièrent instamment, disant : Il mérite que tu lui accordes cela.
\VS{5}Car, disaient-ils, il aime notre nation, et c'est lui qui a bâti notre synagogue.
\VS{6}Jésus s'en alla donc avec eux. Il n'était guère éloigné de la maison, quand le centenier envoya ses amis au-devant de lui, pour lui dire : Seigneur, ne te fatigue point ; car je ne suis pas digne que tu entres sous mon toit.
\VS{7}C'est pourquoi aussi je ne me suis pas cru digne d'aller moi-même vers toi ; mais dis seulement une parole, et mon serviteur sera guéri.
\VS{8}Car, moi qui suis un homme soumis à des supérieurs, j'ai des soldats sous mes ordres ; et je dis à l'un : Va, et il va ; et à un autre : Viens, et il vient ; et à mon serviteur : Fais cela, et il le fait.
\VS{9}Lorsque Jésus entendit ces paroles, il admira le centenier ; et se tournant vers la foule qui le suivait, il dit : Je vous le dis, je n'ai pas trouvé, même en Israël, une si grande foi.
\VS{10}Et quand ceux qui avaient été envoyés furent de retour à la maison, ils trouvèrent le serviteur qui avait été malade, se portant bien.
\TextTitle{Le fils de la veuve de Naïn ressuscite}
\VS{11}Le jour suivant, Jésus alla dans une ville appelée Naïn ; plusieurs de ses disciples et une grande foule allaient avec lui.
\VS{12}Et comme il approchait de la porte de la ville, voici, on portait en terre un mort, fils unique de sa mère, qui était veuve ; et il y avait avec elle un grand nombre de gens de la ville.
\VS{13}Le Seigneur l'ayant vue, fut ému de compassion pour elle ; et il lui dit : Ne pleure pas !
\VS{14}Il s'approcha, et toucha le cercueil. Ceux qui le portaient s'arrêtèrent. Et il dit : Jeune homme, je te dis, lève-toi !
\VS{15}Et le mort s'assit, et se mit à parler. Et Jésus le rendit à sa mère.
\VS{16}Et ils furent tous saisis de crainte, et ils glorifiaient Dieu, disant : Certainement un grand prophète a paru parmi nous ; et Dieu a visité son peuple.
\VS{17}Cette parole sur ce miracle se répandit dans toute la Judée, et dans tout le pays d'alentour.
\VS{18}Jean fut informé de toutes ces choses par ses disciples.
\TextTitle{Jean-Baptiste, le plus grand des hommes\FTNTT{Mt. 11:1-19}}
\VS{19}Il en appela deux, et les envoya vers Jésus pour lui dire : Es-tu celui qui devait venir, ou devons-nous en attendre un autre ?
\VS{20}Et étant venus à lui, ils lui dirent : Jean-Baptiste nous a envoyés auprès de toi pour te dire : Es-tu celui qui devait venir, ou devons-nous en attendre un autre ?
\VS{21}A l'heure même, Jésus guérit plusieurs personnes de maladies,  d'infirmités, et d'esprits malins ; et il rendit la vue à plusieurs aveugles.
\VS{22}Ensuite Jésus leur répondit, et leur dit : Allez, et rapportez à Jean ce que vous avez vu et entendu : Les aveugles recouvrent la vue, les boiteux marchent, les lépreux sont purifiés, les sourds entendent, les morts ressuscitent, l'Evangile est annoncé aux pauvres.
\VS{23}Heureux celui qui n'aura point été scandalisé à cause de moi !
\VS{24}Lorsque les messagers de Jean furent partis, Jésus se mit à dire à la foule au sujet de Jean : Qu'êtes-vous allés voir au désert ? Un roseau agité par le vent ?
\VS{25}Mais qu'êtes-vous allés voir ? Un homme vêtu d'habits précieux ? Voici, ceux qui portent des habits magnifiques, et qui vivent dans les délices, sont dans les maisons des rois.
\VS{26}Mais qu'êtes-vous donc allés voir ? Un prophète ? Oui, vous dis-je, et plus qu'un prophète.
\VS{27}C'est de lui qu'il est écrit : Voici, j'envoie mon messager devant ta face, et il préparera ta voie devant toi\FTNT{Mal. 3:1.}.
\VS{28}Car je vous dis, parmi ceux qui sont nés de femmes, il n'y a aucun prophète plus grand que Jean-Baptiste. Cependant, le plus petit dans le Royaume de Dieu est plus grand que lui.
\VS{29}Et tout le peuple qui entendait cela, et les publicains, justifiaient Dieu, ayant été baptisés du baptême de Jean.
\VS{30}Mais les pharisiens, et les docteurs de la loi, qui n'avaient point été baptisés par lui, rendirent le dessein de Dieu inutile à leur égard.
\VS{31}Alors le Seigneur dit : A qui donc comparerai-je les hommes de cette génération ; et à quoi ressemblent-ils ?
\VS{32}Ils sont semblables aux enfants qui sont assis sur la place publique, et qui se parlant les uns aux autres, disent : Nous vous avons joué de la flûte, et vous n'avez pas dansé ; nous vous avons chanté des complaintes, et vous n'avez pas pleuré.
\VS{33}Car Jean-Baptiste est venu ne mangeant point de pain, et ne buvant point de vin ; et vous dites : Il a un démon.
\VS{34}Le Fils de l'homme est venu mangeant et buvant ; et vous dites : Voici un mangeur et un buveur, un ami des publicains et des pécheurs.
\VS{35}Mais la sagesse a été justifiée par tous ses enfants.
\TextTitle{Une pécheresse pardonnée par Jésus}
\VS{36}Un des pharisiens pria Jésus de manger chez lui ; et Jésus entra dans la maison de ce pharisien, et se mit à table.
\VS{37}Et voici, il y avait dans la ville une femme pécheresse, qui ayant su que Jésus était à table dans la maison du pharisien, apporta un vase d'albâtre plein de parfum,
\VS{38}et se tenant derrière à ses pieds, et pleurant, elle les mouilla de ses larmes, elle les essuya avec ses propres cheveux, et lui baisa les pieds, et les oignit de cette huile odoriférante.
\VS{39}Mais le pharisien qui l'avait invité, voyant cela, dit en lui-même : Si cet homme était prophète, certes il saurait qui et de quelle espèce est la femme qui le touche, il saurait que c'est une pécheresse.
\TextTitle{Parabole des deux débiteurs}
\VS{40}Et Jésus prenant la parole, lui dit : Simon, j'ai quelque chose à te dire. Maître, parle, répondit-il.
\VS{41}Un créancier avait deux débiteurs : L'un lui devait cinq cents deniers, et l'autre cinquante.
\VS{42}Et comme ils n'avaient pas de quoi payer, il leur remit à tous deux leur dette. Lequel l'aimera le plus ?
\VS{43}Et Simon répondant lui dit : Celui, je pense, à qui il a le plus remis. Jésus lui dit : Tu as droitement jugé.
\VS{44}Alors se tournant vers la femme, il dit à Simon : Vois-tu cette femme ? Je suis entré dans ta maison, et tu ne m'as point donné d'eau pour laver mes pieds ; mais elle, elle les a mouillés de ses larmes, elle les a essuyés avec ses propres cheveux.
\VS{45}Tu ne m'as point donné un baiser, mais elle, depuis que je suis entré, n'a cessé d'embrasser mes pieds.
\VS{46}Tu n'as pas oint ma tête d'huile ; mais elle, elle a oint mes pieds d'une huile odoriférante.
\VS{47}C'est pourquoi je te le dis, ses nombreux péchés ont été pardonnés, car elle a beaucoup aimé. Or celui à qui on pardonne peu, aime peu.
\VS{48}Puis il dit à la femme : Tes péchés sont pardonnés.
\VS{49}Ceux qui étaient avec lui à table, se mirent à dire en eux-mêmes : Qui est celui-ci qui pardonne même les péchés ?
\VS{50}Mais il dit à la femme : Ta foi t'a sauvée. Va en paix.
\Chap{8}
\TextTitle{Les femmes au service de Jésus durant son ministère}
\VerseOne{}Or il arriva après cela qu'il allait de ville en ville, et de villages en villages, prêchant et annonçant l'Evangile du Royaume de Dieu.
\VS{2}Les douze disciples étaient auprès de lui avec quelques femmes aussi qu'il avait délivrées d'esprits malins et de maladies : Marie de Magdala, de laquelle étaient sortis sept démons,
\VS{3}Et Jeanne, femme de Chuza, intendant d'Hérode, Susanne, et plusieurs autres qui l'assistaient de leurs biens.
\TextTitle{Parabole du semeur\FTNTT{Mt. 13:1-23 ; Mc. 4:1-20}}
\VS{4}Et comme une grande foule s'étant assemblée, et des gens étant venus de diverses villes auprès de lui, il leur dit cette parabole :
\VS{5}Un semeur sortit pour semer sa semence ; et en semant, une partie de la semence tomba le long du chemin ; elle fut foulée aux pieds, et les oiseaux du ciel la mangèrent toute.
\VS{6}Une autre partie tomba dans un endroit pierreux ; et quand elle fut levée, elle sécha, parce qu'elle n'avait point d'humidité.
\VS{7}Une autre partie tomba au milieu des épines ; les épines crurent avec elle, et l'étouffèrent.
\VS{8}Une autre partie tomba dans une bonne terre ; quand elle fut levée, elle donna du fruit au centuple. En disant ces choses, Jésus dit à haute voix : Que celui qui a des oreilles pour entendre, qu'il entende.
\VS{9}Et ses disciples l'interrogèrent pour savoir ce que signifiait cette parabole.
\VS{10}Il répondit : Il vous a été donné de connaître les mystères du Royaume de Dieu, mais pour les autres, cela leur est dit en paraboles, afin qu'en voyant ils ne voient point, et qu'en entendant ils ne comprennent point.
\VS{11}Voici donc ce que signifie cette parabole : La semence, c'est la parole de Dieu.
\VS{12}Ceux qui ont reçu la semence le long du chemin, ce sont ceux qui entendent la parole ; mais ensuite le diable vient et ôte la parole de leur cœur, de peur qu'ils ne croient et soient sauvés.
\VS{13}Et ceux qui ont reçu la semence dans un endroit pierreux, ce sont ceux qui, lorsqu'ils entendent la parole, la reçoivent avec joie ; mais ils n'ont point de racine ; ils croient pour un temps, mais au moment de la tentation ils se retirent.
\VS{14}Et ce qui est tombé parmi les épines, ce sont ceux qui ayant entendu la parole, s'en vont, et la laissent étouffer par les soucis, les richesses, et les plaisirs de la vie, et ils ne portent point de fruit qui vienne à maturité.
\VS{15}Mais ce qui est tombé dans une bonne terre, ce sont ceux qui ayant entendu la parole, la retiennent dans un cœur honnête et bon, et portent du fruit avec persévérance.
\TextTitle{Parabole du chandelier\FTNTT{Mt. 5:15-16 ; Mc. 4:21-23 ; Lu. 11:33-36}}
\VS{16}Personne, après avoir allumé la lampe, ne la couvre d'un vase ni ne la met sous un lit, mais il la met sur un chandelier, afin que ceux qui entrent voient la lumière.
\VS{17}Car il n'est rien de secret qui ne doive être découvert ; rien de caché qui ne doive être connu et qui ne vienne en évidence.
\VS{18}Prenez donc garde à la manière dont vous écoutez ; car on donnera à celui qui a, mais à celui qui n'a pas, on ôtera même ce qu'il croit avoir.
\TextTitle{La famille spirituelle\FTNTT{Mt. 12:46-50 ; Mc. 3:31-35}}
\VS{19}Alors sa mère et ses frères vinrent vers lui, mais ils ne pouvaient l'aborder à cause de la foule.
\VS{20}Et on vint lui dire : Ta mère et tes frères sont dehors, et ils désirent te voir.
\VS{21}Mais il répondit : Ma mère et mes frères sont ceux qui écoutent la parole de Dieu, et qui la mettent en pratique.
\TextTitle{Jésus calme la tempête\FTNTT{Mt. 8:23-27 ; Mc. 4:35-41}}
\VS{22}Or il arriva qu'un jour, Jésus monta dans une barque avec ses disciples, et il leur dit : Passons de l'autre côté du lac ; et ils partirent.
\VS{23}Pendant qu'ils naviguaient, il s'endormit. Un vent impétueux se leva sur le lac, la barque se remplissait d'eau, et ils étaient en danger.
\VS{24}Ils s'approchèrent et le réveillèrent, en disant : Maître ! Maître ! Nous périssons ! S'étant réveillé, il menaça le vent et les flots qui s'apaisèrent, et le calme revint.
\VS{25}Alors il leur dit : Où est votre foi ? Saisis de frayeur et d'étonnement, ils se dirent les uns aux autres : Quel est donc celui-ci qui commande même aux vents et à l'eau, et à qui ils obéissent ?
\TextTitle{Le démoniaque de Gérasa (Gadara) délivré\FTNTT{Mt. 8:28-34 ; Mc. 5:1-20}}
\VS{26}Puis ils abordèrent dans le pays des Géraséniens qui est vis-à-vis de la Galilée.
\VS{27}Et quand il fut descendu à terre, il vint à sa rencontre un homme de cette ville, qui depuis longtemps était possédé de plusieurs démons. Il ne portait point de vêtements, avait sa demeure, non dans une maison, mais dans les sépulcres.
\VS{28}Ayant vu Jésus, il s'écria et se prosterna devant lui, disant à haute voix : Qu'y a-t-il entre moi et toi, Jésus, Fils du Dieu Très-Haut ? Je te prie, ne me tourmente point.
\VS{29}Car Jésus commandait à l'esprit impur de sortir de cet homme, dont il s'était emparé depuis longtemps. On le gardait lié de chaînes et les fers aux pieds, mais il rompait les liens, et il était entrainé par le démon dans les déserts.
\VS{30}Jésus lui demanda : Quel est ton nom ? Légion\FTNT{Voir commentaire en Mc. 5:9.} répondit-il. Car plusieurs démons étaient entrés en lui.
\VS{31}Et ils priaient Jésus de ne pas leur ordonner d'aller dans l'abîme.
\VS{32}Or il y avait là, dans la montagne, un grand troupeau de pourceaux qui paissaient. Les démons supplièrent Jésus de leur permettre d'entrer dans ces pourceaux. Il le leur permit.
\VS{33}Les démons sortirent de cet homme et entrèrent dans les pourceaux ; et le troupeau se précipita des pentes escarpées dans le lac, et se noya.
\VS{34}Ceux qui les faisaient paître, voyant ce qui était arrivé, s'enfuirent et allèrent le raconter dans la ville et dans les campagnes.
\VS{35}Les gens sortirent pour voir ce qui était arrivé. Ils vinrent auprès de Jésus, et ils trouvèrent l'homme de qui étaient sortis les démons, assis aux pieds de Jésus, vêtu et dans son bon sens ; et ils furent saisis de frayeur.
\VS{36}Et ceux qui avaient vu ce qui s'était passé leur racontèrent comment le démoniaque avait été délivré.
\VS{37}Alors toute cette multitude venue de divers endroits voisins des Géraséniens, le prièrent de se retirer de chez eux ; car ils étaient saisis d'une grande crainte. Jésus monta donc dans la barque, et s'en retourna.
\VS{38}L'homme de qui étaient sortis les démons lui demanda la permission de rester avec lui ; mais Jésus le renvoya, en lui disant :
\VS{39}Retourne dans ta maison, et raconte tout ce que Dieu t'a fait. Il s'en alla donc, et publia par toute la ville tout ce que Jésus avait fait pour lui.
\VS{40}Quand Jésus fut de retour, la foule le reçut avec joie ; car tous l'attendaient.
\TextTitle{Les deux guérisons\FTNTT{Mt. 9:18-26 ; Mc. 5:21-43}}
\VS{41}Et voici, un homme appelé Jaïrus, qui était chef de la synagogue, vint et se jetant aux pieds de Jésus, le pria d'entrer dans sa maison
\VS{42}parce qu'il avait une fille unique, âgée d'environ douze ans, qui se mourait. Pendant que Jésus y allait, il était pressé par la foule.
\VS{43}Or, il y avait une femme atteinte d'une perte de sang depuis douze ans, et qui avait dépensé tout son bien pour les médecins, sans qu'aucun n'ait pu la guérir.
\VS{44}S'approchant de lui par derrière, elle toucha le bord de son vêtement. Au même instant la perte de sang s'arrêta.
\VS{45}Jésus dit : Qui m'a touché ? Comme tous le niaient, Pierre et ceux qui étaient avec lui, dirent : Maître, la foule qui t'entoure te presse et tu dis : Qui m'a touché ?
\VS{46}Mais Jésus dit : Quelqu'un m'a touché, car j'ai connu qu'une force est sortie de moi.
\VS{47}Alors la femme, se voyant découverte, vint toute tremblante se jeter à ses pieds, lui déclara devant tout le peuple pour quelle raison elle l'avait touché, et comment elle avait été guérie à l'instant.
\VS{48}Jésus lui dit : Ma fille, rassure-toi. Ta foi t'a guérie. Va en paix.
\VS{49}Et comme il parlait encore, quelqu'un vint de chez le chef de la synagogue, qui lui dit : Ta fille est morte, n'importune pas le Maître.
\VS{50}Mais Jésus ayant entendu cela, dit au père de la fille : Ne crains point ; crois seulement, et elle sera guérie.
\VS{51}Et quand il fut arrivé à la maison, il ne permit à personne d'entrer avec lui, si ce n'est à Pierre, à Jacques et à Jean, et au père et à la mère de la fille.
\VS{52}Or il la pleuraient tous et de douleur, ils se frappaient la poitrine ; mais il leur dit : Ne pleurez point, elle n'est pas morte, mais elle dort.
\VS{53}Et ils se moquaient de lui, sachant bien qu'elle était morte.
\VS{54}Mais les ayant tous fait sortir, il prit la main de la fille, et dit d'une voix forte : Enfant, lève-toi !
\VS{55}Et son esprit revint en elle, et à l'instant elle se leva ; et Jésus ordonna qu'on lui donne à manger.
\VS{56}Les parents de la fille furent dans l'étonnement, et il leur commanda de ne dire à personne ce qui était arrivé.
\Chap{9}
\TextTitle{Mission des douze apôtres\FTNTT{Mt. 10:1-42 ; cp. Mc. 6:7-13}}
\VerseOne{}Puis, Jésus ayant assemblé ses douze disciples, leur donna puissance et autorité sur tous les démons, avec le pouvoir de guérir les malades.
\VS{2}Il les envoya prêcher le Royaume de Dieu et guérir les malades.
\VS{3}Il leur dit : Ne prenez rien pour le voyage, ni bâton, ni sac, ni pain, ni argent ; et n'ayez pas chacun deux tuniques.
\VS{4}Dans quelque maison que vous entriez, demeurez-y jusqu'à ce que vous partiez de là.
\VS{5}Et partout où l'on ne vous recevra pas, en partant de cette ville secouez la poussière de vos pieds, en témoignage contre eux.
\VS{6}Ils partirent, et ils allèrent de village en village, évangélisant et opérant des guérisons partout.
\VS{7}Or Hérode le Tétrarque entendit parler de toutes les choses que Jésus faisait ; et il ne savait que penser. Car quelques-uns disaient que Jean était ressuscité des morts ;
\VS{8}d'autres, qu'Elie était apparu ; et d'autres, que quelqu'un des anciens prophètes était ressuscité.
\VS{9}Mais Hérode dit : J'ai fait décapiter Jean. Qui est donc celui-ci de qui j'entends dire de telles choses ? Et il cherchait à le voir.
\VS{10}Puis les apôtres étant de retour, lui racontèrent toutes les choses qu'ils avaient faites. Jésus les prit avec lui, et se retira dans un lieu désert, près de la ville appelée Bethsaïda.
\VS{11}Les foules l'ayant su, le suivirent. Jésus les accueillit, et il leur parlait du Royaume de Dieu ; il guérit aussi ceux qui avaient besoin d'être guéris.
\TextTitle{Multiplication des pains pour cinq mille hommes\FTNTT{Mt. 14:15-21 ; Mc. 6:32-44 ; Jn. 6:1-14}}
\VS{12}Comme le jour commençait à baisser, les douze disciples s'approchèrent, et lui dirent : Renvoie la foule, afin qu'elle aille dans les villages et dans les campagnes des environs, pour se loger et pour trouver à manger ; car nous sommes ici dans un pays désert.
\VS{13}Et il leur dit : Donnez-leur vous-mêmes à manger. Et ils dirent : Nous n'avons que cinq pains et deux poissons ; à moins que nous n'allions nous-mêmes acheter des vivres pour tout ce peuple.
\VS{14}Or, il y avait environ cinq mille hommes. Jésus dit aux disciples : Faites-les asseoir par rangées de cinquante chacune.
\VS{15}Ils le firent ainsi, et les firent tous asseoir.
\VS{16}Jésus prit les cinq pains et les deux poissons, et levant les yeux au ciel, il les bénit. Puis, il les rompit, et il les donna à ses disciples afin qu'ils les distribuent à la foule.
\VS{17}Tous mangèrent et furent rassasiés, et l'on remporta douze paniers pleins de morceaux de pain qui restaient.
\TextTitle{Pierre reconnaît Jésus comme le Messie\FTNTT{Mt. 16:13-16 ; Mc. 8:27-30 ; Jn. 6:66-71}}
\VS{18}Or il arriva que comme il était dans un lieu retiré pour prier, et que les disciples étaient avec lui, il les interrogea, disant : Qui disent les foules que je suis ?
\VS{19}Ils lui répondirent : Les uns disent que tu es Jean-Baptiste ; les autres, Elie ; et les autres, qu'un des anciens prophètes est ressuscité.
\VS{20}Il leur dit alors : Et vous, qui dites-vous que je suis ? Et Pierre répondit : Tu es le Christ de Dieu.
\VS{21}Jésus leur défendit sévèrement de ne le dire à personne.
\TextTitle{Jésus annonce sa mort et sa résurrection\FTNTT{Mt. 16:21-23 ; Mc. 8:31-33}}
\VS{22}Et il leur dit : Il faut que le Fils de l'homme souffre beaucoup, et qu'il soit rejeté par les anciens, par les principaux sacrificateurs et par les scribes, et qu'il soit mis à mort, et qu'il ressuscite le troisième jour.
\TextTitle{La consécration du disciple\FTNTT{Mt. 16:24-28 ; Mc. 8:34-38}}
\VS{23}Puis il dit à tous : Si quelqu'un veut venir après moi, qu'il renonce à lui-même, qu'il se charge chaque jour de sa croix, et qu'il me suive.
\VS{24}Car celui qui voudra sauver sa vie, la perdra ; mais celui qui perdra sa vie à cause de son amour pour moi, la sauvera.
\VS{25}Et que servirait-il à un homme de gagner tout le monde, s'il se détruisait ou se perdait lui-même ?
\VS{26}Car quiconque aura honte de moi et de mes paroles, le Fils de l'homme aura honte de lui quand il viendra dans sa gloire, et dans celle du Père et des saints anges.
\TextTitle{La transfiguration\FTNTT{Mt. 17:1-8 ; Mc. 9:1-8}}
\VS{27}Je vous le dis, en vérité, quelques-uns de ceux qui sont ici présents, ne mourront point qu'ils n'aient vu le Royaume de Dieu\FTNT{Voir commentaire en  Mt. 16:28.}.
\VS{28}Or il arriva environ huit jours après ces paroles, qu'il prit avec lui Pierre, et Jean, et Jacques, et qu'il monta sur une montagne pour prier.
\VS{29}Et comme il priait, l'aspect de son visage changea, et son vêtement devint blanc et resplendissant comme un éclair.
\VS{30}Et voici, deux hommes savoir Moïse et Elie, parlaient avec lui,
\VS{31}et ils apparurent environnés de gloire, et ils parlaient de sa mort\FTNT{Départ : Du grec « exodos », ce qui signifie « départ », « mort », « sortie », « hors de ». Jésus est le prophète de l'Exode dont Moïse a parlé dans De. 18:15.} qu'il allait accomplir à Jérusalem.
\VS{32}Or Pierre et ceux qui étaient avec lui étaient accablés de sommeil ; et quand ils furent réveillés, ils virent sa gloire, et les deux hommes qui étaient avec lui.
\VS{33}Et il arriva qu'au moment où ces hommes se séparaient de Jésus, Pierre dit : Maître, il est bon que nous soyons ici, dressons trois tentes, une pour toi, une pour Moïse, et une pour Elie. Il ne savait pas ce qu'il disait.
\VS{34}Et comme il parlait ainsi, une nuée vint les couvrir de son ombre ; et les disciples furent saisis de frayeur en les voyant entrer dans la nuée.
\VS{35}Et une voix vint de la nuée, disant : Celui-ci est mon Fils bien-aimé ; écoutez-le.
\VS{36}Quand la voix se fit entendre, Jésus se trouva seul. Les disciples gardèrent le silence, et ils ne rapportèrent rien à personne en ce temps-là de ce qu'ils avaient vu.
\TextTitle{Les disciples de Jésus montrent leur limite}
\VS{37}Or il arriva le jour suivant, lorsqu'ils furent descendus de la montagne, une grande foule vint à sa rencontre.
\VS{38}Et voici, du milieu de la foule un homme s'écria : Maître, je t'en prie, porte les regards sur mon fils, car c'est mon fils unique.
\VS{39}Et voici un esprit le saisit, et aussitôt le fait crier, et l'agite avec violence en le faisant écumer, et c'est à peine s'il se retire de lui après l'avoir broyé.
\VS{40}J'ai prié tes disciples de le chasser, mais ils n'ont pas pu.
\VS{41}Jésus répondit : Ô génération incrédule et perverse, jusqu'à quand serai-je avec vous, et vous supporterai-je ? Amène ici ton fils.
\VS{42}Comme il approchait, le démon l'agita violemment comme s'il voulait le déchirer ; mais Jésus menaça fortement l'esprit impur, guérit l'enfant, et le rendit à son père.
\VS{43}Et tous furent étonnés de la puissance magnifique de Dieu. Et comme ils étaient tous dans l'admiration de tout ce que Jésus faisait, il dit à ses disciples :
\TextTitle{Jésus annonce de nouveau sa mort et sa résurrection\FTNTT{Mt. 17:22-23 ; Mc. 9:30-32}}
\VS{44}Vous, écoutez bien ces discours : Le Fils de l'homme sera livré entre les mains des hommes.
\VS{45}Mais les disciples ne comprirent pas cette parole, elle était voilée pour eux, afin qu'ils n'en aient pas le sens ; et ils craignaient de l'interroger à ce sujet.
\TextTitle{L'humilité, le secret de la véritable grandeur\FTNTT{Mt. 18:1-6 ; Mc. 9:33-37}}
\VS{46}Or, une pensée leur vint à l'esprit, savoir lequel d'entre eux était le plus grand.
\VS{47}Mais Jésus voyant la pensée de leur cœur, prit un petit enfant et le mit auprès de lui.
\VS{48}Puis il leur dit : Quiconque reçoit ce petit enfant en mon Nom, me reçoit ; et quiconque me reçoit, reçoit celui qui m'a envoyé. Car celui qui est le plus petit d'entre vous tous, c'est celui-là qui est grand.
\TextTitle{Jésus condamne l'esprit sectaire de Jacques et Jean\FTNTT{Mc. 9:38-40}}
\VS{49}Et Jean prit la parole et dit : Maître, nous avons vu quelqu'un qui chassait les démons en ton Nom, et nous l'en avons empêché parce qu'il ne nous suit pas.
\VS{50}Mais Jésus lui dit : Ne l'en empêchez pas ; car celui qui n'est pas contre nous, est pour nous.
\TextTitle{Mission de Jésus : sauver les âmes}
\VS{51}Lorsque le temps où il devait être enlevé du monde approcha, Jésus prit la résolution d'aller à Jérusalem.
\VS{52}Il envoya devant lui des messagers, qui se mirent en route, et entrèrent dans un bourg des Samaritains, pour lui préparer un logement.
\VS{53}Mais les Samaritains ne le reçurent pas, parce qu'il se dirigeait sur Jérusalem.
\VS{54}Et quand Jacques et Jean, ses disciples virent cela, ils dirent : Seigneur ! Veux-tu que nous commandions que le feu descende du ciel, et les consume, comme fit Elie ?
\VS{55}Mais Jésus se tourna vers eux et les réprimanda fortement, en leur disant : Vous ne savez pas de quel esprit vous êtes animés.
\VS{56}Car le Fils de l'homme n'est pas venu pour perdre les âmes des hommes, mais pour les sauver. Ainsi, ils allèrent dans un autre bourg.
\TextTitle{Epreuves de l'engagement du disciple pour suivre Jésus\FTNTT{Mt. 8:19-22}}
\VS{57}Pendant qu'ils étaient en chemin, un homme lui dit : Seigneur, je te suivrai partout où tu iras.
\VS{58}Mais Jésus lui répondit : Les renards ont des tanières, et les oiseaux du ciel ont des nids, mais le Fils de l'homme n'a pas où reposer sa tête.
\VS{59}Puis il dit à un autre : Suis-moi. Et il répondit : Permets-moi d'aller d'abord ensevelir mon père.
\VS{60}Mais Jésus lui dit : Laisse les morts ensevelir leurs morts ; mais toi, va, et annonce le Royaume de Dieu.
\VS{61}Un autre aussi lui dit : Seigneur, je te suivrai ; mais permets-moi de prendre d'abord congé de ceux de ma maison.
\VS{62}Mais Jésus lui répondit : Quiconque met la main à la charrue, et regarde en arrière, n'est pas bien disposé pour le Royaume de Dieu.
\Chap{10}
\TextTitle{Soixante-dix disciples envoyés en mission}
\VerseOne{}Or après ces choses, le Seigneur désigna soixante-dix autres disciples, et il les envoya deux à deux devant lui, dans toutes les villes et dans tous les lieux où il devait aller.
\VS{2}Il leur dit : La moisson est grande, mais il y a peu d'ouvriers ; priez donc le seigneur de la moisson qu'il pousse des ouvriers dans sa moisson.
\VS{3}Allez, voici, je vous envoie comme des agneaux au milieu des loups.
\VS{4}Ne portez ni bourse, ni sac, ni souliers, et ne saluez personne en chemin.
\VS{5}En quelque maison que vous entriez, dites premièrement : Que la paix soit sur cette maison !
\VS{6}Et s'il y a là quelqu'un qui soit digne de paix, votre paix reposera sur lui ; sinon elle retournera à vous.
\VS{7}Et demeurez dans cette maison, mangeant et buvant de ce qui sera mis devant vous ; car l'ouvrier mérite son salaire. N'allez pas de maison en maison.
\VS{8}Dans quelque ville que vous entriez, et où l'on vous recevra, mangez ce qui sera mis devant vous,
\VS{9}guérissez les malades qui s'y trouveront, et dites-leur : Le Royaume de Dieu s'est approché de vous.
\VS{10}Mais dans quelque ville que vous entriez, et où l'on ne vous recevra pas, sortez dans ses rues et dites :
\VS{11}Nous secouons contre vous-mêmes la poussière de votre ville qui s'est attachée à nous ; toutefois sachez que le Royaume de Dieu s'est approché de vous.
\VS{12}Je vous le dis qu'en ce jour Sodome sera traitée moins rigoureusement que cette ville-là.
\TextTitle{Jésus dénonce les indifférents\FTNTT{Mt. 11:20-24}}
\VS{13}Malheur à toi Chorazin, malheur à toi Bethsaïda ! Car si les miracles qui ont été faits au milieu de vous avaient été faits dans Tyr et dans Sidon, il y a longtemps qu'elles se seraient repenties, couvertes d'un sac, et assises sur la cendre.
\VS{14}C'est pourquoi Tyr et Sidon seront traitées moins rigoureusement que vous au jour du jugement.
\VS{15}Et toi, Capernaüm, qui as été élevée jusqu'au ciel, tu seras précipitée jusque dans le Hadès\FTNT{Voir commentaire en Mt. 16:18}.
\VS{16}Celui qui vous écoute, m'écoute ; et celui qui vous rejette, me rejette. Or celui qui me rejette, rejette celui qui m'a envoyé.
\VS{17}Or les soixante-dix revinrent avec joie, disant : Seigneur, les démons mêmes nous sont soumis en ton Nom.
\VS{18}Jésus leur dit : Je voyais Satan tomber du ciel comme un éclair.
\VS{19}Voici, je vous ai donné le pouvoir de marcher sur les serpents et sur les scorpions, et sur toute la force de l'ennemi ; et rien ne pourra vous nuire.
\VS{20}Toutefois, ne vous réjouissez pas de ce que les esprits vous sont soumis, mais réjouissez-vous plutôt de ce que vos noms sont écrits dans les cieux.
\VS{21}En ce moment même, Jésus se réjouit en esprit, et dit : Je te loue, ô Père ! Seigneur du ciel et de la terre, de ce que tu as caché ces choses aux sages et aux intelligents, et que tu les as révélées aux petits enfants. Oui, Père, parce que telle a été ta bonne volonté.
\VS{22}Toutes choses m'ont été données en main par mon Père ; et personne ne connaît qui est le Fils, si ce n'est le Père ; ni qui est le Père, si ce n'est le Fils et celui à qui le Fils veut le révéler.
\VS{23}Puis, se tournant vers ses disciples, il leur dit en particulier : Heureux sont les yeux qui voient ce que vous voyez.
\VS{24}Car je vous dis que beaucoup de prophètes et de rois ont désiré voir ce que vous voyez, et ne l'ont pas vu, et entendre ce que vous entendez, et ne l'ont pas entendu.
\TextTitle{Un docteur de la loi tente d'éprouver Jésus\FTNTT{cp. Mt. 22:34-40 ; Mc. 12:28-34}}
\VS{25}Alors voici, un docteur de la loi s'étant levé pour l'éprouver lui dit : Maître, que dois-je faire pour avoir la vie éternelle ?
\VS{26}Et il lui dit : Qu'est-il écrit dans la loi ? Qu'y lis-tu ?
\VS{27}Il répondit : Tu aimeras le Seigneur ton Dieu de tout ton cœur, de toute ton âme, de toute ta force, et de toute ta pensée ; et ton prochain comme toi-même.
\VS{28}Jésus lui dit : Tu as bien répondu. Fais cela, et tu vivras.
\VS{29}Mais lui, voulant se justifier, dit à Jésus : Et qui est mon prochain ?
\TextTitle{Parabole du Samaritain}
\VS{30}Jésus reprit la parole et dit : Un homme descendait de Jérusalem à Jéricho. Il tomba entre les mains des brigands, qui le dépouillèrent, le chargèrent de plusieurs coups, et s'en allèrent, le laissant à demi mort.
\VS{31}Un sacrificateur, qui par hasard descendait par le même chemin, ayant vu cet homme, passa outre.
\VS{32}Un Lévite, qui arriva aussi dans ce lieu, l'ayant vu, passa outre.
\VS{33}Mais un Samaritain, qui voyageait, étant venu là, fut ému de compassion lorsqu'il le vit.
\VS{34}Il s'approcha, banda ses plaies, en y versant de l'huile et du vin ; puis le mit sur sa propre monture, et le conduisit à une hôtellerie, et prit soin de lui.
\VS{35}Et le lendemain, en partant il tira de sa bourse deux deniers, et les donna à l'hôte, en lui disant : Aie soin de lui ; et tout ce que tu dépenseras de plus, je te le rendrai à mon retour.
\VS{36}Lequel donc de ces trois te semble-t-il avoir été le prochain de celui qui était tombé entre les mains des brigands ?
\VS{37}Il répondit : C'est celui qui a usé de miséricorde envers lui. Jésus donc lui dit : Va, et toi aussi fais de même.
\TextTitle{Marthe et Marie}
\VS{38}Et il arriva comme ils s'en allaient, qu'il entra dans une bourgade ; et une femme nommée Marthe le reçut dans sa maison.
\VS{39}Elle avait une sœur nommée Marie, qui se tenant assise aux pieds de Jésus, écoutait sa parole.
\VS{40}Mais Marthe était distraite par divers soins domestiques ; et étant venue à Jésus, elle dit : Seigneur, ne te soucies-tu point que ma sœur me laisse servir toute seule, dis-lui donc de m'aider de son côté.
\VS{41}Jésus lui répondit : Marthe, Marthe, tu t'inquiètes et tu t'agites pour beaucoup de choses.
\VS{42}Mais une chose est nécessaire ; et Marie a choisi la bonne part, qui ne lui sera point ôtée.
\Chap{11}
\TextTitle{Enseignement de Jésus sur la prière\FTNTT{cp. Mt. 6:9-15}}
\VerseOne{}Et il arriva, comme il était en prière en un certain lieu, qu'après qu'il eut cessé de prier, un de ses disciples lui dit : Seigneur, enseigne-nous à prier, comme Jean l'a enseigné à ses disciples.
\VS{2}Il leur dit : Quand vous prierez, dites : Notre Père qui es aux cieux ! Que ton Nom soit sanctifié, que ton règne vienne ; que ta volonté soit faite sur la terre comme au ciel.
\VS{3}Donne-nous chaque jour notre pain quotidien.
\VS{4}Et pardonne-nous nos péchés ; car nous aussi, nous remettons les dettes à tous ceux qui nous doivent ; et ne nous induis point en tentation, mais délivre-nous du mal.
\TextTitle{Parabole des trois amis et de la prière importune}
\VS{5}Puis il leur dit : Si l'un de vous a un ami, et qu'il aille le trouver à minuit pour lui dire : Mon ami, prête-moi trois pains,
\VS{6}car un de mes amis est arrivé de voyage chez moi, et je n'ai rien à lui offrir,
\VS{7}et si, de l'intérieur de sa maison, cet ami lui répond : Ne m'importune pas ; car ma porte est déjà fermée, mes enfants et moi nous sommes au lit ; je ne puis me lever pour t'en donner.
\VS{8}Je vous le dis, même s'il ne se levait pas pour les lui donner parce que c'est son ami, il se lèverait à cause de son importunité, et lui donnerait tout ce dont il a besoin.
\VS{9}Ainsi je vous dis : Demandez, et il vous sera donné ; cherchez, et vous trouverez ; frappez, et l'on vous ouvrira.
\VS{10}Car quiconque demande, reçoit ; et celui qui cherche, trouve ; et l'on ouvre à celui qui frappe.
\TextTitle{Parabole du père}
\VS{11}Quel est parmi vous le père qui donnera une pierre à son fils, s'il lui demande du pain ? Ou, s'il lui demande un poisson, lui donnera-t-il un serpent au lieu d'un poisson ?
\VS{12}Ou, s'il demande un œuf, lui donnera-t-il un scorpion ?
\VS{13}Si donc vous qui êtes méchants, vous savez donner à vos enfants des choses bonnes, combien plus le Père qui est du ciel donnera-t-il l'Esprit Saint à ceux qui le lui demandent ,
\TextTitle{Jésus guérit un démoniaque}
\VS{14}Alors il chassa un démon qui était muet. Lorsque le démon fut sorti, le muet parla ; et la foule fut dans l'admiration.
\TextTitle{Le blasphème contre le Saint-Esprit\FTNTT{Mt. 12:24-32 ; Mc. 3:22-30}}
\VS{15}Mais quelques-uns d'entre eux dirent : C'est par Béelzebul\FTNT{Béelzebul : Dans 2 R. 1:2, il est fait mention de « Baal Zebub, dieu d'Eqrôn ». Littéralement, la formule signifie « maître (Baal) des mouches ». Ce mot a une autre signification que le grec de la Septante a adoptée en traduisant par Baal-myia, « Baal-mouche ». Le prince des démons.}, prince des démons, qu'il chasse les démons.
\VS{16}Mais les autres pour l'éprouver, lui demandaient un miracle venant du ciel.
\VS{17}Mais lui, connaissant leurs pensées, leur dit : Tout royaume divisé contre lui-même sera réduit en désert ; et toute maison divisée contre elle-même tombe en ruine.
\VS{18}Si donc Satan est divisé contre lui-même, comment son royaume subsistera-t-il ? Car vous dites que je chasse les démons par Béelzebul.
\VS{19}Et si moi, je chasse les démons par Béelzebul, vos fils par qui les chassent-ils ? C'est pourquoi ils seront eux-mêmes vos juges.
\VS{20}Mais si je chasse les démons par le doigt de Dieu, alors le royaume de Dieu est parvenu jusqu'à vous.
\VS{21}Lorsqu'un homme fort et bien armé garde sa bergerie\FTNT{Bergerie : Chez les Grecs, du temps d'Homère, c'était un espace découvert autour de la maison, fermé par un mur, tandis que chez les Orientaux, il s'agissait d'un espace dans la campagne, entouré d'un mur, où les troupeaux passaient la nuit. La bergerie désigne aussi la partie non couverte d'une maison. Dans la première alliance, il s'agit particulièrement du « parvis » du tabernacle et du temple à Jérusalem. Les demeures des gens de la haute société possédaient généralement deux de ces « cours » : une entre la porte et la rue, l'autre entourée par l'immeuble lui-même. C'est ce qui est mentionné en Mt. 26:69. Enfin, ce terme fait allusion à la maison elle-même, un palais.}, les biens qu'il a sont en sûreté.
\VS{22}Mais si un plus fort que lui survient et le vainque, il lui enlève toutes ses armes dans lesquelles il se confiait, et il partage ses dépouilles.
\VS{23}Celui qui n'est point avec moi est contre moi ; et celui qui n'assemble pas avec moi, il disperse.
\TextTitle{Le retour de l'esprit impur\FTNTT{Mt. 12:43-45}}
\VS{24}Quand l'esprit impur est sorti d'un homme, il va par des lieux secs, cherchant du repos. N'en trouvant point, il dit : Je retournerai dans ma maison, d'où je suis sorti,
\VS{25}et quand il arrive, il la trouve balayée et parée.
\VS{26}Alors il s'en va, et prend avec lui sept autres esprits plus méchants que lui, et ils entrent et demeurent là ; de sorte que la dernière condition de cet homme-là est pire que la première.
\VS{27}Or il arriva comme il disait ces choses, qu'une femme élevant sa voix du milieu de la foule, lui dit : Heureux est le ventre qui t'a porté, et les mamelles que tu as tétées !
\VS{28}Et il répondit : Heureux plutôt ceux qui écoutent la parole de Dieu, et qui la gardent !
\TextTitle{Le signe du prophète Jonas\FTNTT{Mt. 12:38-41}}
\VS{29}Et comme les foules s'amassaient ensemble, il se mit à dire : Cette génération est méchante ; elle demande un miracle, mais il ne lui sera donné d'autre miracle que celui de Jonas le prophète.
\VS{30}Car, de même que Jonas fut un miracle pour les Ninivites, de même le Fils de l'homme en sera un pour cette génération.
\VS{31}La reine du Midi se lèvera au jour du jugement contre les hommes de cette génération et les condamnera, parce qu'elle vint des extrémités de la terre pour entendre la sagesse de Salomon ; et voici, il y a ici plus que Salomon.
\VS{32}Les gens de Ninive se lèveront au jour du jugement contre cette génération et la condamneront, parce qu'ils se sont repentis à la prédication de Jonas ; et voici, il y a ici plus que Jonas.
\TextTitle{Parabole de la lampe\FTNTT{Mt. 5:14-16 ; Mc. 4:21-23 ; cp. Lu. 8:16-18}}
\VS{33}Or personne n'allume une lampe pour la mettre dans un lieu caché ou sous le boisseau, mais sur un chandelier, afin que ceux qui entrent voient la lumière.
\VS{34}La lumière du corps c'est l'œil. Si donc ton œil est sain, tout ton corps aussi sera éclairé ; mais s'il est mauvais, ton corps aussi sera ténébreux.
\VS{35}Prends donc garde que la lumière qui est en toi ne soit pas ténèbres.
\VS{36}Si donc ton corps est éclairé, n'ayant aucune partie dans les ténèbres, il sera entièrement éclairé, comme lorsque la lampe t'éclaire de sa lumière.
\VS{37}Comme il parlait, un pharisien le pria de dîner chez lui. Il entra, et se mit à table.
\VS{38}Mais le pharisien vit avec étonnement qu'il ne s'était pas premièrement lavé avant le dîner.
\TextTitle{Malheurs sur les pharisiens et les docteurs de la loi\FTNTT{cp. Mt. 12:38-41}}
\VS{39}Mais le Seigneur lui dit : Vous autres pharisiens, vous nettoyez le dehors de la coupe et du plat ; et à l'intérieur vous êtes pleins de rapine et de méchanceté.
\VS{40}Insensés, celui qui a fait le dehors, n'a-t-il pas fait aussi le dedans ?
\VS{41}Donnez plutôt en aumône ce qui est dedans, et voici, toutes choses seront pures pour vous.
\VS{42}Mais malheur à vous, pharisiens ! Car vous payez la dîme de la menthe, de la rue\FTNT{La rue : Il s'agit d'un arbuste ayant des propriétés médicinales. Les pharisiens poussaient leur zèle jusqu'à payer la dîme sur certaines herbes. Toutefois, en négligeant la justice et l'amour de Dieu, ils passaient à côté de l'essentiel. Toutes leurs œuvres étaient par conséquent vaines.}, et de toutes sortes d'herbes, et vous négligez la justice et l'amour de Dieu. C'est là ce qu'il fallait pratiquer, sans négliger les autres choses.
\VS{43}Malheur à vous, pharisiens, qui aimez les premières places dans les synagogues, et les salutations sur les places publiques.
\VS{44}Malheur à vous, scribes et pharisiens hypocrites, car vous êtes comme les sépulcres qui ne paraissent pas, et sur lesquels on marche sans le voir.
\VS{45}Alors un des docteurs de la loi prit la parole, et lui dit : Maître, en disant ces choses, tu nous outrages aussi.
\VS{46}Et il dit : À vous aussi, malheur, docteurs de la loi ! Car vous chargez les hommes de fardeaux difficiles à porter, et vous-mêmes vous ne touchez pas ces fardeaux d'un seul de vos doigts.
\VS{47}Malheur à vous, car vous bâtissez les sépulcres des prophètes, que vos pères ont tués.
\VS{48}Vous rendez donc témoignage aux œuvres de vos pères et vous y prenez plaisir ; car eux, ils les ont tués, et vous, vous bâtissez leurs sépulcres.
\VS{49}C'est pourquoi aussi la sagesse de Dieu a dit : Je leur enverrai des prophètes et des apôtres, et ils tueront les uns, et persécuteront les autres,
\VS{50}afin que le sang de tous les prophètes qui a été répandu dès la fondation du monde, soit redemandé à cette nation.
\VS{51}Depuis le sang d'Abel, jusqu'au sang de Zacharie, qui fut tué entre l'autel et le temple. Oui, je vous le dis qu'il sera redemandé à cette nation.
\VS{52}Malheur à vous, docteurs de la loi ! Parce que vous avez enlevé la clef de la science. Vous n'êtes pas entrés vous-mêmes, et vous avez empêché ceux qui entraient.
\VS{53}Et comme il leur disait ces choses, les scribes et les pharisiens commencèrent à le presser violemment, et à le faire parler sur beaucoup de choses,
\VS{54}lui dressant des pièges, et cherchant à tirer quelque chose de sa bouche, afin de l'accuser.
\TextTitle{[Enseignements divers de Jésus]
\\(cp. Mt. 16:6-12 ; Mc. 8:14-21}
\Chap{12}
\VerseOne{}Cependant les gens s'étaient rassemblés par milliers, au point de s'écraser les uns les autres. Jésus se mit à dire à ses disciples : Avant tout, gardez-vous surtout du levain des pharisiens qui est l'hypocrisie.
\VS{2}Car il n'y a rien de caché, qui ne doive être révélé, ni de secret, qui ne doive être connu.
\VS{3}C'est pourquoi tout ce que vous aurez dit dans les ténèbres, sera entendu dans la lumière ; et ce que vous aurez dit à l'oreille dans les chambres, sera prêché sur les toits.
\VS{4}Je vous dis à vous qui êtes mes amis : Ne craignez pas ceux qui tuent le corps, et qui après cela ne peuvent rien faire de plus.
\VS{5}Je vous montrerai qui vous devez craindre. Craignez celui qui, après avoir tué, a le pouvoir de jeter dans la géhenne ; oui, vous dis-je, craignez celui-là.
\VS{6}Ne vend-on pas cinq petits passereaux pour deux sous ? Cependant, aucun d'eux n'est oublié devant Dieu.
\VS{7}Et même les cheveux de votre tête sont tous comptés. Ne craignez donc point ; vous valez plus que beaucoup de passereaux.
\VS{8}Or, je vous dis, quiconque me confessera devant les hommes, le Fils de l'homme le confessera aussi devant les anges de Dieu.
\VS{9}Mais quiconque me reniera devant les hommes, il sera renié devant les anges de Dieu.
\VS{10}Et quiconque parlera contre le Fils de l'homme, il lui sera pardonné ; mais celui qui aura blasphémé contre le Saint-Esprit\FTNT{Voir commentaire en Mt. 12:32.}, il ne lui sera point pardonné.
\VS{11}Quand ils vous mèneront devant les synagogues, les magistrats et les autorités, ne vous inquiétez pas de la manière dont vous vous défendrez ni de ce que vous aurez à dire.
\VS{12}Car le Saint-Esprit vous enseignera à l'heure même ce qu'il faudra dire.
\TextTitle{Parabole du riche insensé}
\VS{13}Et quelqu'un de la foule lui dit : Maître, dis à mon frère qu'il partage avec moi notre héritage.
\VS{14}Mais il lui répondit : Ô homme ! Qui m'a établi sur vous pour être votre juge, et pour faire vos partages ?
\VS{15}Puis il leur dit : Gardez-vous avec soin de toute avarice ; car quoique les biens de quelqu'un abondent, il n'a pas la vie par ses biens.
\VS{16}Et il leur dit cette parabole : Les champs d'un homme riche avaient beaucoup rapporté.
\VS{17}Et il raisonnait en lui-même, disant : Que ferai-je, car je n'ai pas assez de place pour recueillir mes fruits ?
\VS{18}Puis il dit : Voici ce que je ferai : J'abattrai mes greniers et j'en bâtirai de plus grands, et j'y amasserai toute ma récolte et tous mes biens.
\VS{19}Puis je dirai à mon âme : Mon âme, tu as beaucoup de biens assemblés pour beaucoup d'années, repose-toi, mange, bois, et réjouis-toi.
\VS{20}Mais Dieu lui dit : Insensé ! Cette même nuit ton âme te sera redemandée ; et ces choses que tu as préparées, à qui seront-elles ?
\VS{21}Il en est ainsi de celui qui amasse des biens pour lui-même, et qui n'est pas riche en Dieu.
\TextTitle{Exhortation à se confier en Dieu}
\VS{22}Jésus dit à ses disciples : C'est pourquoi je vous dis : Ne vous inquiétez pas pour votre vie, de ce que vous mangerez, ni pour votre corps, de quoi vous serez vêtus.
\VS{23}La vie est plus que la nourriture, et le corps est plus que le vêtement.
\VS{24}Considérez les corbeaux, ils ne sèment, ni ne moissonnent, et ils n'ont point de cellier, ni de grenier, et cependant Dieu les nourrit. Combien ne valez-vous pas plus que les oiseaux ?
\VS{25}Qui de vous qui par ses inquiétudes peut ajouter une coudée à la durée de sa vie ?
\VS{26}Si donc vous ne pouvez pas même la moindre chose, pourquoi êtes-vous inquiets du reste ?
\VS{27}Considérez comment croissent les lis, ils ne travaillent, ni ne filent, et cependant je vous dis que Salomon même, dans toute sa gloire, n'a pas été vêtu comme l'un d'eux.
\VS{28}Si Dieu revêt ainsi l'herbe qui est aujourd'hui au champ, et qui demain sera jetée au four, à combien plus forte raison vous vêtira-t-il, ô gens de petite foi ?
\VS{29}Ne dites donc point : Que mangerons-nous, ou que boirons-nous ? Et ne soyez pas inquiets,
\VS{30}car toutes ces choses, ce sont les païens du monde qui les recherchent. Votre Père sait que vous en avez besoin.
\VS{31}Mais cherchez plutôt le Royaume de Dieu, et toutes ces choses vous seront données par-dessus.
\VS{32}Ne crains point petit troupeau, car il a plu à votre Père de vous donner le Royaume.
\VS{33}Vendez ce que vous avez, et donnez-le en aumône. Faites-vous des bourses qui ne s'usent point, un trésor dans les cieux qui ne défaille jamais, et où le voleur n'approche point, et où la teigne ne gâte rien.
\VS{34}Car là où est votre trésor, là sera aussi votre cœur.
\TextTitle{Importance de veiller en attendant le Maître\FTNTT{Mt. 24:36-25:30}}
\VS{35}Que vos reins soient ceints, et vos lampes allumées.
\VS{36}Et soyez semblables aux serviteurs qui attendent que leur maître revienne des noces, afin de lui ouvrir dès qu'il frappera.
\VS{37}Heureux ces serviteurs que le maître à son arrivée, trouvera veillant ! En vérité je vous le dis, il se ceindra, les fera mettre à table, et s'approchera pour les servir.
\VS{38}Qu'il arrive à la seconde veille ou à la troisième veille, heureux ces serviteurs, s'il les trouve veillant !
\VS{39}Or sachez ceci, si le père de famille savait à quelle heure le voleur doit venir, il veillerait, et ne laisserait pas percer sa maison.
\VS{40}Vous donc aussi tenez-vous prêts, car le Fils de l'homme viendra à l'heure où vous n'y penserez pas.
\TextTitle{Parabole des deux serviteurs}
\VS{41}Pierre lui dit : Seigneur, dis-tu cette parabole pour nous, ou aussi pour tous ?
\VS{42}Et le Seigneur dit : Quel est donc l'économe fidèle et prudent, que le maître établira sur les domestiques de sa maison pour leur donner la nourriture au temps convenable ?
\VS{43}Heureux ce serviteur, que son maître, à son arrivée, trouvera faisant ainsi.
\VS{44}Je vous le dis en vérité, il l'établira sur tous ses biens.
\VS{45}Mais si ce serviteur dit en son cœur : Mon maître tarde longtemps à venir, s'il se met à battre les serviteurs et les servantes, à manger, à boire et à s'enivrer,
\VS{46}le maître de cet esclave-là viendra en un jour qu'il n'attend pas, et à une heure qu'il ne sait pas, et il le coupera en deux\FTNT{Le mot grec « dichotomeo » signifie « couper en deux parts », « de couper quelqu'un en deux », « châtiant en coupant », « fléau sévère ». Certains peuples, dont les Hébreux, employaient cette méthode cruelle comme châtiment corporel.}, et lui donnera sa part avec les infidèles.
\VS{47}Or le serviteur qui a connu la volonté de son maître, et qui ne s'est pas tenu prêt, et n'a point fait selon sa volonté, sera battu de plusieurs coups.
\VS{48} Mais celui qui ne l'a point connue, et qui a fait des choses dignes de châtiment, sera battu de peu de coups. Et il sera beaucoup redemandé à quiconque il aura été beaucoup donné; et on exigera plus de celui à qui on aura beaucoup confié.
\TextTitle{Jésus objet de divisions}
\VS{49}Je suis venu jeter un feu sur la terre, et qu'ai-je à désirer, s'il est déjà allumé ?
\VS{50}Il est un baptême dont je dois être baptisé, et combien suis-je pressé jusqu'à ce qu'il soit accompli.
\VS{51}Pensez-vous que je sois venu apporter la paix sur la terre ? Non, vous dis-je ; mais plutôt la division.
\VS{52}Car désormais cinq dans une maison, seront divisés, trois contre deux, et deux contre trois.
\VS{53}Le père sera divisé contre le fils, et le fils contre le père ; la mère contre la fille, et la fille contre la mère ; la belle-mère contre sa belle-fille, et la belle-fille contre sa belle-mère.
\VS{54}Puis il dit encore aux foules : Quand vous voyez un nuage se lever à l'occident, vous dites aussitôt : La pluie vient, et cela arrive ainsi.
\VS{55}Et quand vous voyez souffler le vent du midi, vous dites qu'il fera chaud ; et cela arrive.
\VS{56}Hypocrites, vous savez bien discerner l'aspect du ciel et de la terre ; et comment ne discernez-vous point cette saison ?
\VS{57}Et pourquoi aussi ne reconnaissez-vous pas de vous-mêmes ce qui est juste ?
\VS{58}Or quand tu vas avec ton adversaire devant le magistrat, tâche en chemin de t'en délivrer, de peur qu'il ne te traîne devant le juge, et que le juge ne te livre à l'officier de justice et que celui-ci ne te mette en prison.
\VS{59}Je te le dis, tu ne sortiras pas de là que tu n'aies payé jusqu'au dernier pite\FTNT{Petite pièce de monnaie en laiton. Voir en annexe le « tableau des monnaies au temps de Jésus-Christ ».}.
\Chap{13}
\TextTitle{Exhortation à la repentance}
\VerseOne{}En ce même temps, quelques-uns qui se trouvaient là présents racontèrent à Jésus ce qui était arrivé à des Galiléens, dont Pilate avait mêlé le sang avec celui de leurs sacrifices.
\VS{2}Et Jésus répondant leur dit : Croyez-vous que ces Galiléens étaient de plus grands pécheurs que tous les autres Galiléens, parce qu'ils ont souffert de la sorte ?
\VS{3}Non, vous dis-je ; mais si vous ne vous repentez pas, vous périrez tous de la même manière.
\VS{4}Ou bien, ces dix-huit personnes sur qui est tombée la tour de Siloé et qu'elle a tuées, croyez-vous qu'elles étaient plus coupables que tous les habitants de Jérusalem ?
\VS{5}Non, vous dis-je ; mais si vous ne vous repentez pas, vous périrez tous de la même manière.
\TextTitle{Parabole du figuier stérile et le jugement différé d'Israël\FTNTT{cp. Mt. 21:18-21}}
\VS{6}Il disait aussi cette parabole : Un homme avait un figuier planté dans sa vigne. Il vint pour y chercher du fruit, mais il n'en trouva point.
\VS{7}Et il dit au vigneron : Voilà trois ans que je viens chercher du fruit à ce figuier, et je n'en trouve point. Coupe-le ; pourquoi occupe-t-il inutilement la terre ?
\VS{8}Et le vigneron lui répondit : Seigneur, laisse-le encore pour cette année, je creuserai tout autour, et j'y mettrai du fumier.
\VS{9}Peut-être portera-t-il du fruit ; sinon, tu le couperas après cela.
\TextTitle{Guérison de la femme courbée le jour du sabbat}
\VS{10}Or comme il enseignait dans une de leurs synagogues un jour de sabbat,
\VS{11}voici, il y avait là une femme qui était possédée d'un démon qui la rendait infirme depuis dix-huit ans, et elle était courbée, et ne pouvait nullement se redresser.
\VS{12}Et quand Jésus la vit, il l'appela, et lui dit : Femme, tu es délivrée de ton infirmité.
\VS{13}Et il lui imposa les mains ; et à l'instant elle se redressa, et glorifia Dieu.
\VS{14}Mais le chef de la synagogue, indigné de ce que Jésus avait opéré cette guérison un jour du sabbat, prenant la parole dit à l'assemblée : Il y a six jours pour travailler ; venez donc vous faire guérir ces jours-là, et non pas le jour du sabbat.
\VS{15}Hypocrites ! lui répondit le Seigneur, chacun de vous ne détache-t-il pas son bœuf ou son âne de la crèche le jour du sabbat, et ne les mène-t-il pas boire ?
\VS{16}Et ne fallait-il pas délier de ce lien le jour du sabbat cette femme qui est fille d'Abraham, et que Satan tenait liée depuis dix-huit ans ?
\VS{17}Comme il disait ces choses, tous ses adversaires étaient confus ; mais toutes les foules se réjouissaient de toutes les choses glorieuses qu'il opérait.
\TextTitle{Parabole du grain de moutarde et du levain\FTNTT{voir Mt. 13:31,33}}
\VS{18}Il disait aussi : A quoi est semblable le Royaume de Dieu, et à quoi le comparerai-je ?
\VS{19}Il est semblable au grain de semence de moutarde qu'un homme a pris et jeté dans son jardin ; il pousse, devient un grand arbre, et les oiseaux du ciel font leurs nids dans ses branches.
\VS{20}Il dit encore : A quoi comparerai-je le Royaume de Dieu ?
\VS{21}Il est semblable au levain qu'une femme a pris et mis dans trois mesures de farine, pour faire lever toute la pâte.
\TextTitle{Enseignements de Jésus sur le chemin de Jérusalem}
\VS{22}Puis il s'en allait par les villes et les villages, enseignant, et faisant route vers Jérusalem.
\VS{23}Quelqu'un lui dit : Seigneur, n'y a-t-il que peu de gens qui soient sauvés ? Il leur répondit :
\VS{24}Efforcez-vous d'entrer par la porte étroite. Car je vous le dis que beaucoup chercheront à entrer, et ne le pourront pas.
\VS{25}Quand le père de famille se sera levé, et aura fermé la porte, et que vous, étant dehors, vous vous mettrez à frapper à la porte, en disant : Seigneur ! Seigneur ! Ouvre-nous ! Il vous répondra : Je ne sais pas d'où vous êtes.
\VS{26}Alors vous vous mettrez à dire : Nous avons mangé et bu en ta présence, et tu as enseigné dans nos rues.
\VS{27}Mais il dira : Je vous le dis, je ne sais pas d'où vous êtes. Retirez-vous de moi, vous tous qui faites le métier d'iniquité.
\VS{28}C'est là qu'il y aura des pleurs et des grincements de dents, quand vous verrez Abraham, Isaac, et Jacob, et tous les prophètes dans le Royaume de Dieu, et que vous serez jetés dehors.
\VS{29}Il en viendra aussi d'orient et d'occident, du nord et du sud, et ils se mettront à table dans le Royaume de Dieu.
\VS{30}Et voici, ceux qui sont les derniers seront les premiers, et ceux qui sont les premiers seront les derniers.
\VS{31}En ce même jour, quelques pharisiens vinrent à lui et lui dirent : Retire-toi et va-t'en d'ici, car Hérode veut te tuer.
\VS{32}Il leur répondit : Allez, et dites à ce renard : Voici, je chasse les démons et j'achève de faire des guérisons aujourd'hui et demain, et le troisième jour je prends fin.
\VS{33}C'est pourquoi il me faut marcher aujourd'hui et demain, et le jour suivant ; car il ne convient pas qu'un prophète meure hors de Jérusalem.
\TextTitle{Lamentations de Jésus sur Jérusalem\FTNTT{Mt. 23:37-39 ; Lu. 19:41-44 ; cp. Jé. 22:5}}
\VS{34}Jérusalem, Jérusalem, qui tues les prophètes et qui lapides ceux qui te sont envoyés ; combien de fois ai-je voulu rassembler tes enfants, comme la poule rassemble ses poussins sous ses ailes, et vous ne l'avez pas voulu !
\VS{30}Voici, votre maison va être déserte ; et je vous le dis en vérité, que vous ne me verrez plus, jusqu'à ce que vous disiez : Béni soit celui qui vient au nom du Seigneur.
\Chap{14}
\TextTitle{Jésus guérit un hydropique le jour du sabbat\FTNTT{cp. Mt. 12:9-13}}
\VerseOne{}Jésus entra un jour de sabbat dans la maison d'un des chefs des pharisiens pour prendre un repas, les pharisiens l'observaient.
\VS{2}Et voici, un homme hydropique était là devant lui.
\VS{3}Jésus prit la parole, et dit aux docteurs de la loi et aux pharisiens : Est-il permis, ou non, de faire une guérison le jour du sabbat ?
\VS{4}Ils gardèrent le silence. Alors Jésus prit le malade, le guérit, et le renvoya.
\VS{5}Puis s'adressant à lui, il leur dit : Lequel de vous, si son fils ou son bœuf tombe dans un puits, ne l'en retirera pas aussitôt, le jour du sabbat ?
\VS{6}Et ils ne pouvaient répliquer à ces choses.
\TextTitle{Parabole de l'invité}
\VS{7}Il proposa aussi aux conviés une parabole, en voyant qu'ils choisissaient les premières places ; et il leur dit :
\VS{8}Quand tu seras convié par quelqu'un à des noces, ne te mets pas à la première place à table, de peur qu'il ne se trouve parmi les conviés une personne plus honorable que toi,
\VS{9}et que celui qui vous a conviés l'un et l'autre ne vienne te dire : Cède ta place à cette personne-là. Tu aurais alors honte d'aller occuper la dernière place.
\VS{10}Mais lorsque tu seras convié, va te mettre à la dernière place, afin que quand celui qui t'a convié viendra, il te dise : Mon ami, monte plus haut. Alors cela te fera honneur devant tous ceux qui seront à table avec toi.
\VS{11}Car quiconque s'élève, sera abaissé ; et quiconque s'abaisse, sera élevé.
\VS{12}Il dit aussi à celui qui l'avait convié : Lorsque tu fais un dîner ou un souper, n'invite pas tes amis, ni tes frères, ni tes parents, ni tes riches voisins ; de peur qu'ils ne te convient à leur tour, et qu'on ne te rende la pareille.
\VS{13}Mais, lorsque tu donneras un festin, convie les pauvres, les impotents, les boiteux et les aveugles.
\VS{14}Et tu seras heureux de ce qu'ils n'ont pas de quoi te rendre la pareille ; car elle te sera rendue à la résurrection des justes.
\TextTitle{Parabole du grand festin\FTNTT{Mt. 22:1-14}}
\VS{15}Un de ceux qui étaient à table, ayant entendu ces paroles, lui dit : Heureux celui qui mangera du pain dans le Royaume de Dieu.
\VS{16}Et Jésus lui répondit : Un homme fit un grand festin, et il convia beaucoup de gens.
\VS{17}Et à l'heure du souper, il envoya son serviteur pour dire aux conviés : Venez, car tout est déjà prêt.
\VS{18}Mais ils commencèrent tous unanimement à s'excuser. Le premier lui dit : J'ai acheté un champ, et il me faut nécessairement partir pour aller le voir ; je te prie, excuse-moi.
\VS{19}Un autre dit : J'ai acheté cinq paires de bœufs, et je vais les essayer ; je te prie, excuse-moi.
\VS{20}Et un autre dit : J'ai épousé une femme, c'est pourquoi je ne puis aller.
\VS{21}Le serviteur, de retour, rapporta ces choses à son maître. Alors le père de famille irrité, dit à son serviteur : Va promptement dans les places et dans les rues de la ville, et amène ici les pauvres, les impotents, les boiteux et les aveugles.
\VS{22}Puis le serviteur dit : Maître, ce que tu as commandé a été fait, et il y a encore de la place.
\VS{23}Et le maître dit au serviteur : Va dans les chemins et le long des haies, et ceux que tu trouveras, contrains-les d'entrer, afin que ma maison soit remplie.
\VS{24}Car je vous dis, qu'aucun de ces hommes qui avaient été conviés ne goûtera de mon souper.
\TextTitle{Test de la consécration du disciple}
\VS{25}Or de grandes foules faisaient route avec Jésus. Il se retourna et leur dit :
\VS{26}Si quelqu'un vient à moi, et ne hait pas son père et sa mère, sa femme et ses enfants, ses frères et ses sœurs, et même sa propre vie, il ne peut être mon disciple.
\VS{27}Et quiconque ne porte pas sa croix, et ne me suit pas, ne peut être mon disciple.
\TextTitle{Parabole de la tour}
\VS{28}Car lequel de vous, s'il veut bâtir une tour, ne s'assied pas premièrement pour calculer la dépense et voir s'il a de quoi l'achever ?
\VS{29}De peur qu'après avoir posé les fondements, il ne puisse pas l'achever, et que tous ceux qui le verront ne commencent à se moquer de lui,
\VS{30}en disant : Cet homme a commencé à bâtir, et il n'a pas pu achever.
\TextTitle{Parabole du roi qui se prépare à la guerre}
\VS{31}Ou, quel roi, s'il va faire la guerre à un autre roi, ne s'assied pas premièrement pour examiner s'il peut, avec dix mille hommes, aller à la rencontre de celui qui vient contre lui avec vingt mille ?
\VS{32}Autrement, pendant que cet autre roi est encore loin, il lui envoie une ambassade pour demander la paix.
\VS{33}Ainsi donc, quiconque d'entre vous ne renonce pas à tout ce qu'il possède ne peut être mon disciple.
\TextTitle{Parabole du sel}
\VS{34}Le sel est bon ; mais si le sel perd sa saveur, avec quoi l'assaisonnera-t-on ?
\VS{35}Il n'est bon ni pour la terre, ni pour le fumier ; mais on le jette dehors. Que celui qui a des oreilles pour entendre, qu'il entende !
\Chap{15}
\TextTitle{Trois paraboles sur la repentance}
\VerseOne{}Or tous les publicains et les pécheurs s'approchaient de Jésus pour l'entendre.
\VS{2}Mais les pharisiens et les scribes murmuraient, disant : Cet homme reçoit les pécheurs, et mange avec eux.
\TextTitle{Parabole de la brebis perdue\FTNTT{Mt. 18:12-14}}
\VS{3}Mais il leur proposa cette parabole, disant :
\VS{4}Lequel d'entre vous, s'il a cent brebis, et qu'il en perd une, ne laisse pas les quatre-vingt-dix-neuf dans le désert, pour aller à la recherche de celle qui est perdue, jusqu'à ce qu'il la trouve ?
\VS{5}Et l'ayant retrouvée, il la met avec joie sur ses épaules,
\VS{6}et, de retour à la maison, il appelle ses amis et ses voisins, et il leur dit : Réjouissez-vous avec moi ; car j'ai trouvé ma brebis qui était perdue.
\VS{7}De même, je vous le dis il y aura plus de joie dans le ciel pour un seul pécheur qui se repent, que pour les quatre-vingt-dix-neuf justes qui n'ont pas besoin de repentance.
\TextTitle{Parabole de la drachme perdue}
\VS{8}Ou quelle femme, si elle a dix drachmes, et qu'elle en perde une, n'allume pas une lampe, ne balaie la maison, et ne cherche avec soin, jusqu'à ce qu'elle la trouve ?
\VS{9}Lorsqu'elle l'a trouvée, elle appelle ses amies et ses voisines, en leur disant : Réjouissez-vous avec moi ; car j'ai trouvé la drachme que j'avais perdue.
\VS{10}Ainsi je vous le dis, il y a de la joie devant les anges de Dieu pour un seul pécheur qui vient à se repentir.
\TextTitle{Parabole du fils perdu}
\VS{11}Il leur dit aussi : Un homme avait deux fils ;
\VS{12}et le plus jeune dit à son père : Mon père, donne-moi la part de bien qui m'appartient ; et il leur partagea ses biens.
\VS{13}Et peu de jours après, le plus jeune fils, ayant tout ramassé, partit pour un pays éloigné, où il dissipa son bien en vivant dans la débauche.
\VS{14}Et après qu'il eut tout dépensé, une grande famine survint dans ce pays-là, et il commença à se trouver dans la disette.
\VS{15}Alors il alla se mettre au service d'un des habitants du pays, qui l'envoya dans ses possessions pour paître les pourceaux.
\VS{16}Il aurait bien voulu se rassasier des carouges que les pourceaux mangeaient ; mais personne ne lui en donnait.
\VS{17}Or étant revenu à lui-même, il dit : Combien d'ouvriers chez mon père ont du pain en abondance, et moi je meurs de faim !
\VS{18}Je me lèverai, j'irai vers mon père, et je lui dirai : Mon père, j'ai péché contre le ciel et devant toi ;
\VS{19}et je ne suis plus digne d'être appelé ton fils ; traite-moi comme l'un de tes ouvriers.
\VS{20}Il se leva donc, et alla vers son père. Et comme il était encore loin, son père le vit et fut ému de compassion, il courut se jeter à son cou et le baisa.
\VS{21}Mais le fils lui dit : Mon père, j'ai péché contre le ciel et devant toi ; et je ne suis plus digne d'être appelé ton fils.
\VS{22}Et le père dit à ses serviteurs : Apportez la plus belle robe et revêtez-le, mettez-lui un anneau au doigt, et des souliers aux pieds.
\VS{23}Amenez-moi le veau gras, et tuez-le. Mangeons et réjouissons-nous.
\VS{24}Car mon fils que voici, était mort, mais il est ressuscité ; il était perdu, mais il est retrouvé. Et ils commencèrent à se réjouir.
\VS{25}Or son fils aîné était dans les champs. Lorsqu'il revint et approcha de la maison, il entendit la musique et les danses.
\VS{26}Il appela un des serviteurs, et il lui demanda ce que c'était.
\VS{27}Ce serviteur lui dit : Ton frère est de retour, et ton père a tué le veau gras, parce qu'il l'a recouvré sain et sauf.
\VS{28}Mais il se mit en colère, et ne voulut pas entrer. Son père sortit et le pria d'entrer.
\VS{29}Mais il répondit, et dit à son père : voici, il y a tant d'années que je te sers, et jamais je n'ai transgressé ton commandement, et cependant tu ne m'as jamais donné un chevreau pour que je me réjouisse avec mes amis.
\VS{30}Mais quand ton fils est arrivé, celui qui a mangé ton bien avec des prostituées, c'est pour lui que tu as tué le veau gras.
\VS{31}Et le père lui dit : Mon enfant, tu es toujours avec moi, et tous mes biens sont à toi.
\VS{32}Or il fallait bien s'égayer et se réjouir, parce que ton frère que voici était mort et qu'il est ressuscité, parce qu'il était perdu et qu'il est retrouvé.
\Chap{16}
\TextTitle{Parabole de l'économe infidèle}
\VerseOne{}Il disait aussi à ses disciples : Il y avait un homme riche qui avait un économe, qui fut accusé devant lui comme dissipant ses biens.
\VS{2}Il l'appela et lui dit : Qu'est-ce que j'entends dire de toi ? Rends compte de ton administration ; car tu n'auras plus le pouvoir d'administrer mes biens.
\VS{3}Alors l'économe dit en lui-même : Que ferai-je, puisque mon maître m'ôte l'administration ? Travailler à la terre ? Je ne le puis. Mendier ? J'en ai honte.
\VS{4}Je sais ce que je ferai, afin que les gens me reçoivent dans leurs maisons quand mon administration me sera ôtée.
\VS{5}Alors il appela chacun des débiteurs de son maître, et il dit au premier : Combien dois-tu à mon maître ?
\VS{6}Il dit : Cent mesures d'huile. Et il lui dit : Prends ton billet, et assied-toi vite, et écris cinquante.
\VS{7}Puis il dit à un autre : Et toi, combien dois-tu ? Il dit : Cent mesures de froment. Et il lui dit : Prends ton billet, et écris quatre-vingts.
\VS{8}Et le maître loua l'économe infidèle de ce qu'il avait agi prudemment. Ainsi les enfants de ce siècle sont plus prudents dans leur génération, que les enfants de lumière.
\VS{9}Et moi aussi je vous dis : Faites-vous des amis avec les richesses injustes ; afin que quand vous viendrez à manquer, ils vous reçoivent dans les tabernacles éternels.
\VS{10}Celui qui est fidèle en très peu de chose, est fidèle aussi dans les grandes choses ; et celui qui est injuste en très peu de chose, est injuste aussi dans les grandes choses.
\VS{11}Si donc vous n'avez pas été fidèles dans les richesses injustes, qui vous confiera les véritables richesses ?
\VS{12}Et si en ce qui est à autrui vous n’avez pas été fidèles, qui vous donnera ce qui est vôtre ?
\VS{13}Nul serviteur ne peut servir deux maîtres. Car, ou il haïra l'un, et aimera l'autre ; ou il s'attachera à l'un, et méprisera l'autre. Vous ne pouvez pas servir Dieu et Mamon\FTNT{Voir commentaire en Mt. 6:24.}.
\TextTitle{L'avarice condamnée par Jésus}
\VS{14}Or les pharisiens aussi, qui étaient avares, entendaient toutes ces choses, et ils se moquaient de lui.
\VS{15}Et il leur dit : Vous, vous cherchez à paraître justes devant les hommes ; mais Dieu connaît vos cœurs ; c'est pourquoi ce qui est élevé parmi les hommes est une abomination devant Dieu.
\VS{16}La loi et les prophètes ont duré jusqu'à Jean ; depuis lors, le Royaume de Dieu est prêché, et chacun y fait violence.
\VS{17}Or il est plus aisé que le ciel et la terre passent, qu'il ne l'est qu'un trait de la lettre de la loi vienne à tomber.
\TextTitle{Enseignement de Jésus sur le divorce\FTNTT{Mt. 5:31-32 ; 19:1-9 ; Mc. 10:2-12}}
\VS{18}Quiconque répudie sa femme, et se marie à une autre, commet un adultère, et quiconque prend celle qui a été répudiée par son mari, commet un adultère.
\TextTitle{Histoire de l'homme riche et de Lazare}
\VS{19}Il y avait un homme riche, qui était vêtu de pourpre et de fin lin, et qui tous les jours se réjouissait d'une vie somptueuse.
\VS{20}Il y avait un pauvre, nommé Lazare, couché à la porte du riche, tout couvert d'ulcères,
\VS{21}et qui désirait se rassasier des miettes qui tombaient de la table du riche ; et même les chiens venaient encore lécher ses ulcères.
\VS{22}Et il arriva que le pauvre mourut, et il fut porté par les anges dans le sein d'Abraham\FTNT{Contrairement aux idées reçues, le sein d'Abraham ne se trouvait pas au ciel. En effet, le Seigneur a dit que personne n'était monté au ciel si ce n'est lui-même (Jn. 3:13). Avant l'ère de la grâce, tous les morts allaient dans le séjour des morts où ils étaient retenus prisonniers par le dieu Hadès (voir commentaire sur l'enfer en Mt. 16:18). Toutefois, ce lieu était séparé en deux parties distinctes, l'une réservée aux impies, où ils y subissaient des tourments, et l'autre réservée aux personnes pieuses qui se tenaient en repos, sans souffrir. En effet, lorsque Saül fit appel à une voyante pour faire remonter Samuel du séjour des morts afin de le consulter, Samuel lui annonça qu'il le rejoindrait dès le lendemain à l'endroit où il se trouvait (1 S. 28:19). De plus, dans le récit de la mort du pauvre Lazare et du riche, deux points importants sont à noter. D'une part, bien qu'étant séparés l'un de l'autre, ils pouvaient se voir et communiquer ensemble (Lu. 16:23-26). D'autre part, il est évident que le riche souffrait tandis que le pauvre Lazare était consolé (Lu. 16:25). Lorsque le Seigneur est mort, il est descendu dans « les régions inférieures de la terre » pour délivrer les captifs pieux qui avaient vécu avant Jésus-Christ (Ep. 4:8-9 ; 1 S. 2:6). Par la même occasion, il confirma la condamnation des impies (1 Pi. 3:19). Maintenant que Jésus-Christ est mort et ressuscité, tous ceux qui meurent dans le Seigneur vont au ciel (2 Co. 5:1-3. ; Ph. 1:22-23).}. Le riche mourut aussi, et il fut enseveli.
\VS{23}Etant en enfer\FTNT{Le mot traduit par « enfer » vient du grec « Hades ». Voir commentaire Mt. 16:18}, il leva ses yeux ; et, tandis qu'il était dans les tourments, il vit de loin Abraham et Lazare dans son sein.
\VS{24}Il s'écria : Père Abraham aie pitié de moi, et envoie Lazare, pour qu'il trempe le bout de son doigt dans l'eau et me rafraichisse la langue ; car je suis grièvement tourmenté dans cette flamme.
\VS{25}Abraham répondit : Mon enfant, souviens-toi que tu as reçu tes biens pendant ta vie, et que Lazare a eu ses maux pendant la sienne ; maintenant il est ici consolé, et toi, tu es grièvement tourmenté.
\VS{26}D'ailleurs, il y a entre nous et vous un grand abîme ; en sorte que ceux qui veulent passer d'ici vers vous ne le peuvent, et que ceux qui veulent passer de là ne traversent pas non plus vers nous.
\VS{27}Et il dit : Je te prie donc, père, de l'envoyer dans la maison de mon père ; car j'ai cinq frères.
\VS{28}Afin qu'il leur rende témoignage de l'état où je suis ; de peur qu'eux aussi ne viennent dans ce lieu de tourment.
\VS{29}Abraham lui répondit : Ils ont Moïse et les prophètes ; qu'ils les écoutent.
\VS{30}Mais il dit : Non, père Abraham, mais si quelqu'un des morts va vers eux, ils se repentiront.
\VS{31}Et Abraham lui dit : S'ils n'écoutent pas Moïse et les prophètes, ils ne seront pas non plus persuadés quand quelqu'un des morts ressusciterait.
\Chap{17}
\TextTitle{Instructions de Jésus au sujet des scandales, du pardon et de la foi\FTNTT{Mt. 5:31-32 ; 19:1-9 ; Mc. 10:2-12}}
\VerseOne{}Or il dit à ses disciples : Il est impossible qu'il n'arrive pas des scandales ; mais malheur à celui par qui ils arrivent.
\VS{2}Il vaudrait mieux pour lui qu'on lui mette une pierre de moulin autour de son cou, et qu'on le jette dans la mer, que de scandaliser un seul de ces petits.
\VS{3}Prenez garde à vous-mêmes. Si donc ton frère a péché contre toi, reprends-le ; et s'il se repent, pardonne-lui.
\VS{4}Et s'il a péché contre toi sept fois dans un jour et que sept fois il revienne à toi, disant : Je me repens, tu lui pardonneras.
\VS{5}Alors les apôtres dirent au Seigneur : Augmente-nous la foi.
\VS{6}Et le Seigneur dit : Si vous aviez de la foi aussi gros qu'un grain de semence de moutarde, vous diriez à ce sycomore : Déracine-toi, et plante-toi dans la mer ; et il vous obéirait.
\TextTitle{Les serviteurs inutiles}
\VS{7}Mais qui de vous, ayant un serviteur qui laboure ou paît les troupeaux, lui dira, quand il revient des champs : Approche-toi vite, et mets-toi à table.
\VS{8}Ne lui dira-t-il pas plutôt : Prépare-moi à souper, ceins-toi, et sers-moi jusqu'à ce que j'aie mangé et bu ; et après cela tu mangeras et tu boiras ?
\VS{9}Doit-il de la reconnaissance à ce serviteur parce qu'il a fait ce qui lui était ordonné ? Je ne le pense pas.
\VS{10}Vous de même, quand vous aurez fait tout ce qui vous a été ordonné, dites : Nous sommes des serviteurs inutiles ;  ce que nous étions obligés de faire, nous l'avons fait.
\TextTitle{Guérison de dix lépreux}
\VS{11}Et il arriva qu’en allant à Jérusalem, il passait par le milieu de la Samarie, et de la Galilée.
\VS{12}Et comme il entrait dans un village, dix hommes lépreux vinrent à sa rencontre. Se tenant à distance, ils élevèrent la voix, et dirent :
\VS{13}Jésus, Maître, aie pitié de nous !
\VS{14}Et quand il les eut vus, il leur dit : Allez, montrez-vous aux sacrificateurs\FTNT{Lé. 13.}. Et, pendant qu'ils y allaient, ils furent purifiés.
\VS{15}L'un d'eux se voyant guéri, revint sur ses pas, glorifiant Dieu à haute voix.
\VS{16}Et il se jeta en terre sur sa face aux pieds de Jésus, lui rendant grâces. Or c’était un Samaritain.
\VS{17}Alors Jésus prenant la parole, dit : Les dix n'ont-ils pas été rendus purs ? Et les neuf autres, où sont-ils ?
\VS{18}Il n'y a eu que cet étranger qui soit revenu pour rendre gloire à Dieu.
\VS{19}Alors il lui dit : Lève-toi. Va, ta foi t'a sauvé.
\TextTitle{Les pharisiens demandent à voir le Royaume de Dieu\FTNTT{cp. Lu. 19:11-27}}
\VS{20}Or les pharisiens demandèrent à Jésus quand viendrait le Royaume de Dieu. Il leur répondit, et leur dit : Le Royaume de Dieu ne vient pas de manière à attirer l'attention.
\VS{21}Et on ne dira point : Il est ici ; ou : Il est là. Car voici, le Royaume de Dieu est au milieu de vous.
\TextTitle{Jésus annonce sa seconde venue\FTNTT{voir De. 30:3}}
\VS{22}Il dit aussi à ses disciples : Des jours viendront où vous désirerez voir un des jours du Fils de l'homme, mais vous ne le verrez point. On vous dira :
\VS{23}Il est ici, ou : Il est là. N'allez pas, et ne les suivez point.
\VS{24}Car, comme l'éclair brille et resplendit d'une extrémité du ciel à l'autre, ainsi sera le Fils de l'homme en son jour.
\VS{25}Mais il faut premièrement qu'il souffre beaucoup, et qu'il soit rejeté par cette génération.
\VS{26}Ce qui arriva aux jours de Noé, arrivera de même aux jours du Fils de l'homme.
\VS{27}On mangeait et on buvait ; on prenait et on donnait des femmes en mariage jusqu'au jour où Noé entra dans l'arche ; le déluge vint, et les fit tous périr.
\VS{28}C’est encore ce qui arriva aux jours de Lot : On mangeait, on buvait, on achetait, on vendait, on plantait et on bâtissait.
\VS{29}Mais le jour où Lot sortit de Sodome, une pluie de feu et de soufre tomba du ciel, et les fit tous périr.
\VS{30}Il en sera de même au jour où le Fils de l'homme paraîtra.
\VS{31}En ce jour-là, que celui qui sera sur le toit, et qui aura ses effets dans la maison, ne descende point pour les prendre ; et que celui qui sera dans les champs, ne retourne pas non plus à ce qui est resté en arrière.
\VS{32}Souvenez-vous de la femme de Lot.
\VS{33}Quiconque cherchera à sauver sa vie, la perdra ; et quiconque la perdra, la retrouvera.
\VS{34}Je vous dis, qu’en cette nuit-là deux seront dans un même lit : l’un sera pris, et l’autre laissé ;
\VS{35}deux femmes moudront ensemble, l’une sera prise et l’autre laissée ;
\VS{36}deux seront aux champs, l’un sera pris et l’autre laissé.
\VS{37}Les disciples lui dirent : Où, Seigneur ? Et il leur dit, Là où est le corps, là aussi s’assembleront les aigles.
\Chap{18}
\TextTitle{Parabole du juge inique}
\VerseOne{}Et il leur proposa une parabole, pour montrer qu'il faut toujours prier, et ne point se relâcher,
\VS{2}disant : Il y avait dans une ville un juge qui ne craignait point Dieu et qui ne respectait personne.
\VS{3}Et dans la même ville, il y avait une veuve, qui venait souvent lui dire : Fais-moi justice de ma partie adverse.
\VS{4}Pendant longtemps il refusa. Mais après cela il dit en lui-même : Quoique je ne craigne point Dieu, et que je ne respecte personne,
\VS{5}néanmoins, parce que cette veuve me donne de la peine, je lui ferai justice, de peur qu'elle ne vienne sans cesse me casser la tête.
\VS{6}Et le Seigneur dit : Ecoutez ce que dit le juge inique.
\VS{7}Et Dieu ne ferait-il point justice à ses élus, qui crient à lui jour et nuit, quoiqu’il use de patience avant d’intervenir pour eux ?
\VS{8}Je vous le dis que bientôt il les vengera. Mais quand le Fils de l'homme viendra, pensez-vous qu'il trouvera la foi sur la terre ?
\TextTitle{Parabole du pharisien et du publicain}
\VS{9}Il dit aussi cette parabole au sujet de certaines personnes se persuadant qu'elles étaient justes, et ne faisant aucun cas des autres :
\VS{10}Deux hommes montèrent au temple pour prier, l'un était pharisien, et l'autre, publicain.
\VS{11}Le pharisien, se tenant debout, priait en lui-même en ces termes : Ô Dieu ! Je te rends grâces de ce que je ne suis pas comme le reste des hommes, qui sont ravisseurs, injustes, adultères, ni même comme ce publicain.
\VS{12}Je jeûne deux fois la semaine, et je donne la dîme de tout ce que je possède.
\VS{13}Mais le publicain se tenant loin, n'osait même pas lever les yeux vers le ciel, mais il se frappait la poitrine, en disant : Ô Dieu ! Sois apaisé envers moi qui suis pécheur !
\VS{14}Je vous dis que celui-ci descendit dans sa maison justifié, plutôt que l'autre ; car quiconque s'élève, sera abaissé, et quiconque s'abaisse, sera élevé.
\TextTitle{Le Royaume des cieux, pour ceux qui ressemblent aux petits enfants\FTNTT{Mt. 19:13-15 ; Mc. 10:13-16}}
\VS{15}Et quelques-uns lui présentèrent aussi de petits enfants, afin qu'il les touchât, mais les disciples voyant cela, reprenaient ceux qui les présentaient.
\VS{16}Mais Jésus les appela, et dit : Laissez venir à moi les petits enfants, et ne les en empêchez pas ; car le Royaume de Dieu est pour ceux qui leur ressemblent.
\VS{17}Je vous le dis en vérité, quiconque ne recevra point comme un enfant le Royaume de Dieu, n’y entrera point.
\TextTitle{Jésus dénonce l'attachement aux richesses\FTNTT{Mt. 19:16-30 ; Mc. 10:17-31 ; cp. Lu. 10:25-37}}
\VS{18}Un chef interrogea Jésus et dit : Bon Maître, que dois-je faire pour hériter la vie éternelle ?
\VS{19}Jésus lui dit : Pourquoi m'appelles-tu bon ? Il n'y a de bon que Dieu seul\FTNT{Voir commentaire Mc. 10:18.}.
\VS{20}Tu connais les commandements : Tu ne commettras point d'adultère. Tu ne tueras point. Tu ne déroberas point. Tu ne diras point de faux témoignage. Honore ton père et ta mère.
\VS{21}Et il lui dit : J'ai observé toutes ces choses dès ma jeunesse.
\VS{22}Et quand Jésus eut entendu cela, lui dit : Il te manque encore une chose : Vends tout ce que tu as, et distribue-le aux pauvres, et tu auras un trésor dans les cieux. Puis viens, et suis-moi.
\VS{23}Lorsqu'il entendit ces choses, il devint tout triste, car il était extrêmement riche.
\VS{24}Jésus voyant qu'il était devenu tout triste, dit : Qu'il est difficile à ceux qui ont des richesses d'entrer dans le Royaume de Dieu !
\VS{25}Car il est plus facile à un chameau de passer par le trou d'une aiguille, qu'à un riche d'entrer dans le Royaume de Dieu\FTNT{Voir commentaire Mt. 19:24.}.
\VS{26}Ceux qui entendirent cela, dirent : Et qui peut donc être sauvé ?
\VS{27}Jésus leur répondit : Ce qui est impossible aux hommes est possible à Dieu.
\TextTitle{Récompense pour un vrai disciple de Jésus}
\VS{28}Pierre dit : Voici, nous avons tout quitté, et nous t'avons suivi.
\VS{29}Et il leur dit : Je vous le dis en vérité, il n'est personne qui, ayant quitté pour l'amour du Royaume de Dieu, sa maison, ou ses parents, ou ses frères, ou sa femme, ou ses enfants,
\VS{30}ne reçoive beaucoup plus dans ce siècle-ci, et dans le siècle à venir la vie éternelle.
\TextTitle{Jésus annonce à nouveau sa mort et sa résurrection\FTNTT{Mt. 20:17-19 ; Mc. 10:32-34}}
\VS{31}Jésus prit à part les douze, et il leur dit : Voici, nous montons à Jérusalem, et tout ce qui est écrit par les prophètes au sujet du Fils de l'homme, s'accomplira.
\VS{32}Car il sera livré aux Gentils ; on se moquera de lui, on l'outragera, et on lui crachera au visage,
\VS{33}et après l'avoir battu de verges, on le fera mourir ; mais il ressuscitera le troisième jour.
\VS{34}Mais ils ne comprirent rien à cela, et ce discours était si obscur pour eux qu'ils ne comprirent point ce qu'il leur disait.
\TextTitle{Bartimée voit !\FTNTT{cp. Mt. 20:29-34 ; Mc. 10:46-53}}
\VS{35}Or comme il approchait de Jéricho, un aveugle était assis au bord du chemin, et mendiait.
\VS{36}Et entendant la foule qui passait, il demanda ce que c'était.
\VS{37}Et on lui dit : C'est Jésus de Nazareth qui passe.
\VS{38}Alors il cria, disant : Jésus, Fils de David, aie pitié de moi !
\VS{39}Ceux qui marchaient devant le reprenaient, pour le faire taire ; mais il criait beaucoup plus fort : Fils de David, aie pitié de moi !
\VS{40}Et Jésus s'étant arrêté ordonna qu'on le lui amène ; et, quand il se fut approché,
\VS{41}il lui demanda : Que veux-tu que je te fasse ? Il répondit : Seigneur, que je recouvre la vue.
\VS{42}Jésus lui dit : Recouvre la vue ; ta foi t'a sauvé.
\VS{43}Et à l'instant il recouvra la vue et suivit Jésus, glorifiant Dieu. Et tout le peuple voyant cela, loua Dieu.
\Chap{19}
\TextTitle{Conversion de Zachée}
\VerseOne{}Jésus, étant entré dans Jéricho, traversait la ville.
\VS{2}Et voici, un homme riche, appelé Zachée, chef des publicains, cherchait à voir qui était Jésus,
\VS{3}mais il ne le pouvait pas à cause de la foule, car il était de petite taille.
\VS{4}C'est pourquoi il accourut devant, et monta sur un sycomore pour le voir ; car il devait passer par là.
\VS{5}Et quand Jésus fut arrivé à cet endroit-là, il leva les yeux, le vit, et lui dit : Zachée, hâte-toi de descendre ; car il faut que je demeure aujourd'hui dans ta maison.
\VS{6}Zachée se hâta de descendre, et le reçut avec joie.
\VS{7}Et tous voyant cela murmuraient, et disaient : Il est entré chez un homme pécheur pour y loger.
\VS{8}Et Zachée, se présentant devant le Seigneur, lui dit : Voici, Seigneur, je donne la moitié de mes biens aux pauvres ; et si j'ai fait tort de quelque chose à quelqu'un, je lui rends le quadruple\FTNT{Lé. 5:20-24.}.
\VS{9}Et Jésus lui dit : Aujourd'hui le salut est entré dans cette maison ; parce que celui-ci aussi est fils d'Abraham.
\VS{10}Car le Fils de l'homme est venu chercher et sauver ce qui était perdu.
\TextTitle{Parabole des dix mines\FTNTT{Lu. 17:21}}
\VS{11}Et comme ils entendaient ces choses, Jésus poursuivit son discours, et proposa une parabole, parce qu'il était près de Jérusalem, et qu'ils pensaient que le royaume de Dieu allait immédiatement paraître.
\VS{12}Il dit donc : Un homme noble s'en alla dans un pays éloigné, pour prendre possession d'un Royaume, et revenir ensuite.
\VS{13}Il appela dix de ses serviteurs, il leur donna dix mines et leur dit : Faites-les valoir jusqu'à ce que je revienne.
\VS{14}Or ses concitoyens le haïssaient, c'est pourquoi ils envoyèrent après lui une ambassade, pour dire : Nous ne voulons pas que cet homme règne sur nous.
\VS{15} Il arriva donc après qu'il fut de retour, et après avoir pris possession du Royaume, qu'il fit appeler auprès de lui les serviteurs auxquels il avait confié son argent, afin de connaître comment chacun l'avait fait valoir.
\VS{16}Alors le premier vint, et dit : Seigneur, ta mine a produit dix autres mines.
\VS{17}Il lui dit : C'est bien, bon serviteur ; parce que tu as été fidèle en peu de choses, reçois le gouvernement de dix villes.
\VS{18}Et le second vint, et dit : Seigneur, ta mine a produit cinq autres mines.
\VS{19}Il dit aussi à celui-ci : Toi aussi, sois établi sur cinq villes.
\VS{20}Un autre vint, et dit : Seigneur, voici ta mine que j'ai gardée enveloppée dans un linge ;
\VS{21}car j'avais peur de toi, parce que tu es un homme sévère ; tu prends ce que tu n'as point déposé, et tu moissonnes ce que tu n'as pas semé.
\VS{22}Il lui dit : Méchant serviteur, je te jugerai sur tes propres paroles : Tu savais que je suis un homme sévère, prenant ce que je n'ai point déposé, et moissonnant ce que je n'ai point semé.
\VS{23}Pourquoi donc n'as-tu pas mis mon argent dans une banque, afin qu'à mon retour je le retire avec un intérêt ?
\VS{24}Alors il dit à ceux qui étaient présents : Ôtez-lui la mine, et donnez-la à celui qui a les dix.
\VS{25}Ils lui dirent : Seigneur, il a dix mines.
\VS{26}Ainsi je vous le dis, on donnera à celui qui a, mais à celui qui n'a pas, on ôtera ce qu'il a.
\VS{27}Au reste, amenez ici mes ennemis qui n'ont pas voulu que je règne sur eux, et tuez-les devant moi.
\TextTitle{Jésus fait son entrée à Jérusalem\FTNTT{Za. 9:9 ; Mt. 21:1-11 ; Mc. 11:1-11 ; Jn. 12:12-19}}
\VS{28}Après avoir ainsi parlé, Jésus marcha devant la foule, pour monter à Jérusalem.
\VS{29}Lorsqu'il approcha de Bethphagé et de Béthanie, vers la montagne appelée Montagne des Oliviers, Jésus envoya deux de ses disciples,
\VS{30}en leur disant : Allez au village qui est en face ; quand vous y serez entrés, vous trouverez un ânon attaché, sur lequel aucun homme n'est monté ; détachez-le, et amenez-le-moi.
\VS{31}Si quelqu'un vous demande pourquoi le détachez-vous, vous lui répondrez : Le Seigneur en a besoin.
\VS{32}Et ceux qui étaient envoyés s'en allèrent, et trouvèrent l'ânon comme il le leur avait dit.
\VS{33}Comme ils le détachaient, ses maîtres leur dirent : Pourquoi détachez-vous cet ânon ?
\VS{34}Ils répondirent : Le Seigneur en a besoin.
\VS{35}Ils emmenèrent à Jésus l'ânon, sur lequel ils jetèrent leurs vêtements, et firent monter Jésus dessus.
\VS{36}Quand il fut en marche, les gens étendirent leurs vêtements sur le chemin.
\VS{37}Et lorsque déjà il approchait de Jérusalem, vers la descente de la Montagne des Oliviers, toute la multitude des disciples saisie de joie, se mit à louer Dieu à haute voix, pour tous les miracles qu'ils avaient vus.
\VS{38}Ils disaient : Béni soit le Roi qui vient au Nom du Seigneur\FTNT{Ps. 118:26.} ! Paix dans le ciel, et gloire dans les lieux très hauts.
\VS{39}Quelques pharisiens, du milieu de la foule, lui dirent : Maître, reprends tes disciples.
\VS{40}Et Jésus répondit : Je vous le dis, s'ils se taisent, les pierres crieront.
\TextTitle{Nouvelles lamentations de Jésus sur Jérusalem\FTNTT{cp. Mt. 23:37-39 ; Lu. 13:34-35}}
\VS{41}Comme il approchait de la ville, Jésus, en la voyant, pleura sur elle, et dit :
\VS{42}Ô ! Si toi aussi, au moins en ce jour qui t'est donné, tu connaissais les choses qui appartiennent à ta paix ! Mais maintenant elles sont cachées à tes yeux.
\VS{43}Il viendra sur toi des jours où tes ennemis t'environneront de tranchées, t'enfermeront, et te serreront de tous côtés ;
\VS{44}ils te raseront, toi et tes enfants qui sont au milieu de toi, et ils ne laisseront pas en toi pierre sur pierre, parce que tu n'as pas connu le temps de ta visitation.
\TextTitle{Jésus chasse les marchands du temple}
\VS{45}Il entra dans le temple, et il se mit à chasser dehors ceux qui vendaient et qui achetaient.
\VS{46}Leur disant : Il est écrit : Ma maison sera appelée la maison de prière ; mais vous, vous en avez fait une caverne de voleurs\FTNT{Es. 56:7 ; Jé. 7:11.}.
\VS{47}Il enseignait tous les jours dans le temple. Et les principaux sacrificateurs et les scribes cherchaient à le faire mourir.
\VS{48}Mais ils ne savaient comment s'y prendre ; car tout le peuple s'attachait à ses paroles.
\Chap{20}
\TextTitle{L'autorité de Jésus et celle de Jean-Baptiste\FTNTT{Mt. 21:23-27 ; Mc. 11:27-33}}
\VerseOne{}Et il arriva un de ces jours-là, comme Jésus enseignait le peuple dans le temple, et qu'il évangélisait, les principaux sacrificateurs, les scribes et les anciens survinrent,
\VS{2}et lui parlèrent en disant : Dis-nous par quelle autorité fais-tu ces choses, ou qui est celui qui t'a donné cette autorité ?
\VS{3}Jésus leur répondit : Je vous adresserai aussi une question, et répondez-moi.
\VS{4}Le baptême de Jean venait-il du ciel ou des hommes ?
\VS{5}Ils raisonnaient entre eux, disant : Si nous répondons : Du ciel ; il dira : Pourquoi n'avez-vous pas cru en lui ?
\VS{6}Et si nous répondons : Des hommes, tout le peuple nous lapidera ; car il est persuadé que Jean était un prophète ;
\VS{7}Alors, ils répondirent qu'ils ne savaient d'où il était.
\VS{8}Et Jésus leur dit : Moi non plus, je ne vous dirai pas par quelle autorité je fais ces choses.
\TextTitle{Parabole des vignerons\FTNTT{Es. 5:1-7 ; Mt. 21:33-46 ; Mc. 12:1-12}}
\VS{9}Alors il se mit à dire au peuple cette parabole : Un homme planta une vigne, et la loua à des vignerons, et fut longtemps absent.
\VS{10}Et à la saison de la récolte, il envoya un serviteur vers les vignerons, afin qu'ils lui donnent du fruit de la vigne. Les vignerons le battirent, et le renvoyèrent à vide.
\VS{11}Il leur envoya encore un autre serviteur ; mais ils le battirent aussi, et après l'avoir traité indignement, ils le renvoyèrent à vide.
\VS{12}Il en envoya encore un troisième, mais ils le blessèrent aussi, et le jetèrent dehors.
\VS{13}Alors le maître de la vigne dit : Que ferai-je ? J'enverrai mon fils bien-aimé ; peut-être que quand ils le verront, ils le respecteront.
\VS{14}Mais quand les vignerons le virent, ils raisonnèrent entre eux, et dirent : Voici l'héritier ; venez, tuons-le, afin que l'héritage soit à nous.
\VS{15}Et ils le jetèrent hors de la vigne, et le tuèrent. Que leur fera donc le maître de la vigne ?
\VS{16}Il viendra, et fera périr ces vignerons-là, et il donnera la vigne à d'autres. Lorsqu'ils entendirent cela, ils dirent : A Dieu ne plaise !
\VS{17}Alors il les regarda, et dit : Que signifie donc ce qui est écrit : La pierre qu'on rejetée ceux qui bâtissaient est devenue la principale de l'angle\FTNT{Ps. 118:22.} ?
\VS{18}Quiconque tombera sur cette pierre, sera brisé ; et elle écrasera celui sur qui elle tombera.
\TextTitle{Le tribut à César\FTNTT{Mt. 22:15-22 ; Mc. 12:13-17}}
\VS{19}Les principaux sacrificateurs et les scribes cherchèrent à mettre la main sur lui à l'heure même, mais ils craignirent le peuple. Ils avaient compris que c'était pour eux que Jésus avait dit cette parabole.
\VS{20}Ils se mirent à observer Jésus ; et ils envoyèrent des agents secrets, qui feignaient d'être justes, pour lui tendre des pièges et saisir de lui quelque parole afin de le livrer au magistrat et à l'autorité du gouverneur.
\VS{21}Ils l'interrogèrent, en disant : Maître, nous savons que tu parles et enseignes conformément à la justice, et que tu ne regardes pas à l'apparence des personnes, mais que tu enseignes la voie de Dieu selon la vérité.
\VS{22}Nous est-il permis de payer le tribut à César, ou non ?
\VS{23}Jésus, apercevant leur ruse, leur dit : Pourquoi me tentez-vous ?
\VS{24}Montrez-moi un denier. De qui a-t-il l'image et l'inscription ? Ils lui répondirent : De César.
\VS{25}Alors il leur dit : Rendez donc à César ce qui est à César ; et à Dieu ce qui est à Dieu.
\VS{26}Ainsi ils ne purent le surprendre dans ses paroles devant le peuple ; mais, étonnés de sa réponse, ils gardèrent le silence.
\TextTitle{Les preuves de la résurrection\FTNTT{Mt. 22:23-33 ; Mc.12:18-27}}
\VS{27}Alors quelques-uns des sadducéens, qui nient formellement la résurrection, s'approchèrent et l'interrogèrent,
\VS{28}disant : Maître, voici ce que Moïse nous a prescrit : Si le frère de quelqu'un meurt, ayant une femme et pas d'enfants, son frère épousera la femme, et suscitera une postérité à son frère.
\VS{29}Or, il y avait sept frères. Le premier se maria, et mourut sans enfants.
\VS{30}Le deuxième épousa la femme et mourut sans enfants.
\VS{31}Puis le troisième l'épousa aussi, et tous les sept de même ; et ils moururent sans laisser d'enfants.
\VS{32}Enfin, la femme mourut aussi.
\VS{33}Duquel d'entre eux donc sera-t-elle la femme à la résurrection ? Car les sept l'ont eue pour femme.
\VS{34}Jésus leur répondit : Les enfants de ce siècle prennent des femmes et des maris ;
\VS{35}mais ceux qui seront trouvés dignes d'avoir part au siècle à venir et à la résurrection des morts, ne prendront ni femmes ni maris.
\VS{36}Car ils ne pourront plus mourir, parce qu'ils seront semblables aux anges, et qu'ils seront fils de Dieu, étant fils de la résurrection.
\VS{37}Que les morts ressuscitent, c'est ce que Moïse a fait connaître quand, à propos du buisson, il appelle le Seigneur le Dieu d'Abraham, le Dieu d'Isaac, et le Dieu de Jacob.
\VS{38}Or, Dieu n'est pas le Dieu des morts, mais des vivants ; car tous vivent en lui.
\TextTitle{Jésus dénonce l'attitude des scribes\FTNTT{cp. Mt. 22:41-23:36 ; Mc. 12:35-40}}
\VS{39}Quelques-uns des scribes prenant la parole, dirent : Maître, tu as bien parlé.
\VS{40}Et ils n'osaient plus lui poser aucune question.
\VS{41}Jésus leur dit : Comment dit-on que le Christ est Fils de David ?
\VS{42}Car David lui-même dit au livre des psaumes : Le Seigneur a dit à mon Seigneur : Assieds-toi à ma droite,
\VS{43}jusqu'à ce que j'aie mis tes ennemis pour le marchepied de tes pieds\FTNT{Ps. 110:1.}.
\VS{44}David donc l'appelle son Seigneur, comment est-il son Fils ?
\VS{45}Comme tout le peuple l'écoutait, il dit à ses disciples :
\VS{46}Gardez-vous des scribes, qui aiment à se promener en robes longues, et qui aiment les salutations sur les places publiques ; qui recherchent les premiers sièges dans les synagogues, et les premières places dans les festins ;
\VS{47}qui dévorent entièrement les maisons des veuves, et qui font pour l'apparence de longues prières. Ils seront jugés plus sévèrement.
\Chap{21}
\TextTitle{Offrande de la pauvre veuve\FTNTT{Mc. 12:41-44}}
\VerseOne{}Comme Jésus regardait, il vit des riches qui mettaient leurs offrandes dans le tronc.
\VS{2}Il vit aussi une pauvre veuve qui y mettait deux petites pièces de monnaie.
\VS{3}Et il dit : Je vous le dis en vérité, cette pauvre veuve a mis plus que tous les autres.
\VS{4}Car tous ceux-ci ont mis aux offrandes de Dieu, de leur superflu ; mais elle a mis de son nécessaire, tout ce qu'elle avait pour vivre.
\TextTitle{Enseignement sur le Mont des Oliviers\FTNTT{Mt. 24-25 ; Mc. 13}}
\VS{5}Comme quelques-uns disaient que le temple était orné de belles pierres et d'offrandes, il dit :
\VS{6}Vous contemplez ces choses ! Les jours viendront où, il ne restera pas pierre sur pierre qui ne soit démolie.
\TextTitle{Les disciples posent deux questions à Jésus\FTNTT{Mt. 24:3 ; Mc.13:3-4}}
\VS{7}Ils lui demandèrent : Maître, quand donc cela arrivera-t-il, et à quel signe connaîtra-t-on que ces choses vont arriver ?
\TextTitle{Les temps de la fin\FTNTT{Mt. 24:4-14 ; Mc. 13:5-13}}
\VS{8}Jésus répondit : Prenez garde que vous ne soyez point séduits. Car plusieurs viendront en mon Nom, disant : C'est moi qui suis le Christ et le temps approche. Ne les suivez pas.
\VS{9}Quand vous entendrez parler des guerres et des soulèvements, ne soyez pas effrayés ; car il faut que ces choses arrivent premièrement. Mais ce ne sera pas encore la fin.
\VS{10}Alors il leur dit : Une nation s'élèvera contre une autre nation, et un royaume contre un autre royaume.
\VS{11}Il y aura de grands tremblements de terre en divers lieux, des famines et des pestes ; il y aura des choses terribles, et de grands signes dans le ciel.
\TextTitle{Souffrance des croyants}
\VS{12}Mais, avant toutes ces choses, ils mettront la main sur vous, et l'on vous persécutera ; on vous livrera aux synagogues, on vous jettera en prison, on vous mènera devant des rois et devant des gouverneurs, à cause de mon Nom.
\VS{13}Cela vous arrivera pour que vous serviez de témoignage.
\VS{14}Mettez-vous donc dans vos cœurs de ne pas préméditer votre défense.
\VS{15}Car je vous donnerai une bouche et une sagesse à laquelle vos adversaires ne pourront résister ou contredire.
\VS{16}Vous serez livrés même par vos parents, par vos frères, par vos proches et par vos amis, et ils feront mourir plusieurs d'entre vous.
\VS{17}Vous serez haïs de tous à cause de mon Nom.
\VS{18}Mais il ne se perdra pas un cheveu de votre tête.
\VS{19}Vous sauverez vos âmes par votre persévérance.
\TextTitle{La destruction de Jérusalem prophétisée}
\VS{20}Lorsque vous verrez Jérusalem environnée par les armées, sachez alors que sa désolation est proche.
\VS{21}Alors, que ceux qui seront en Judée, fuient dans les montagnes ; et que ceux qui seront au milieu de Jérusalem, en sortent, et que ceux qui seront dans les champs, n'entrent pas dans la ville.
\VS{22}Car ce seront des jours de vengeance, afin que toutes les choses qui sont écrites soient accomplies.
\VS{23}Malheur aux femmes qui seront enceintes, et à celles qui allaiteront en ces jours-là ; car il y aura une grande calamité sur le pays, et une grande colère contre ce peuple.
\VS{24}Ils tomberont sous le tranchant de l'épée, ils seront emmenés captifs\FTNT{Les Juifs se révoltèrent plusieurs fois contre le joug des Romains installés en Palestine depuis l'an 65 av. J.-C. En 70, Titus s'empara de Jérusalem après une guerre de plusieurs années et un siège meurtrier de sept mois. Cette même année, le temple fut détruit. A la suite d'une dernière révolte, la ville fut prise de nouveau sous Hadrien. En l'an 135, les Juifs furent en grande partie exterminés, et les survivants furent à jamais chassés de Jérusalem. Ces événements marquèrent symboliquement les débuts de la dispersion des Juifs à travers le monde.} parmi toutes les nations ; et Jérusalem sera foulée par les nations, jusqu'à ce que les temps des nations soient accomplis.
\TextTitle{Retour du Messie sur la terre\FTNTT{Mt. 24:29-31 ; Mc. 13:24-27}}
\VS{25}Il y aura des signes dans le soleil, dans la lune, et dans les étoiles. Et sur terre, il y aura de la détresse chez les nations qui ne sauront que faire, au bruit de la mer et des flots,
\VS{26}les hommes seront comme rendant l'âme de frayeur, dans l'attente des choses qui surviendront dans le monde ; car les puissances des cieux seront ébranlées.
\VS{27}Alors on verra le Fils de l'homme venant sur une nuée avec puissance et grande gloire.
\VS{28}Quand ces choses commenceront à arriver, regardez en haut et levez vos têtes, parce que votre délivrance approche.
\TextTitle{Parabole du figuier\FTNTT{Mt. 24:29-31 ; Mc. 13:24-27}}
\VS{29}Et il leur proposa cette comparaison : Voyez le figuier, et tous les autres arbres.
\VS{30}Dès qu'ils ont poussé, vous savez de vous-mêmes, en regardant, que déjà l'été est proche.
\VS{31}Vous aussi de même, quand vous verrez arriver ces choses, sachez que le Royaume de Dieu est proche.
\VS{32}En vérité je vous le dis, que cette génération ne passera point, que toutes ces choses ne soient arrivées.
\VS{33}Le ciel et la terre passeront, mais mes paroles ne passeront point.
\TextTitle{Exhortation à veiller\FTNTT{Mt. 24:36-51 ; Mc. 13:32-37}}
\VS{34}Prenez donc garde à vous-mêmes, de peur que vos cœurs ne soient appesantis par la gourmandise et l'ivrognerie, et par les soucis de cette vie ; et que ce jour-là ne vous surprenne subitement.
\VS{35}Car il viendra comme un filet sur tous ceux qui habitent sur la surface de toute la terre.
\VS{36}Veillez donc, et priez en tout temps, afin que vous soyez trouvés dignes d'échapper à toutes ces choses qui arriveront, et de paraître devant le Fils de l'homme.
\VS{37}Pendant le jour, Jésus enseignait dans le temple, et il allait passer la nuit à la montagne appelée Montagne des Oliviers.
\VS{38}Et dès le point du jour, tout le peuple venait vers lui au temple pour l'entendre.
\Chap{22}
\TextTitle{Trahison de Judas\FTNTT{Mt. 26:14-16 ; Mc. 14:1-2,10-11}}
\VerseOne{}La fête des pains sans levain, qu'on appelle Pâque, approchait.
\VS{2}Les principaux sacrificateurs et les scribes cherchaient les moyens de faire mourir Jésus ; car ils craignaient le peuple.
\VS{3}Or, Satan entra dans Judas, surnommé Iscariot, qui était du nombre des douze.
\VS{4}Et Judas alla, et parla avec les principaux sacrificateurs et les chefs de gardes, sur la manière de le leur livrer.
\VS{5}Ils furent dans la joie, et convinrent de lui donner de l'argent.
\VS{6}Après s'être engagé, il cherchait une occasion favorable pour leur livrer Jésus à l'insu de la foule.
\TextTitle{La dernière Pâque\FTNTT{Mt. 26:17-25 ; Mc. 14:12-21 ; Jn. 13:1-12}}
\VS{7}Le jour des pains sans levain, où l'on devait immoler la Pâque, arriva.
\VS{8}Et Jésus envoya Pierre et Jean, en leur disant : Allez, et apprêtez-nous l'agneau de Pâque, afin que nous le mangions.
\VS{9}Et ils lui dirent : Où veux-tu que nous l'apprêtions ?
\VS{10}Il leur dit : Voici, quand vous serez entrés dans la ville vous rencontrerez un homme portant une cruche d'eau, suivez-le dans la maison où il entrera.
\VS{11}Et dites au maître de la maison : Le Maître te dit : Où est le lieu où je mangerai l'agneau de Pâque avec mes disciples ?
\VS{12}Et il vous montrera une grande chambre haute, meublée ; c'est là que vous apprêterez l'agneau de Pâque.
\VS{13}Ils partirent, et trouvèrent les choses comme il leur avait dit ; et ils apprêtèrent l'agneau de Pâque.
\VS{14}Et quand l'heure fut venue, il se mit à table, et les douze apôtres avec lui.
\VS{15}Il leur dit : J'ai désiré vivement manger cet agneau de Pâque avec vous avant de souffrir.
\VS{16}Car, je vous dis, que je ne le mangerai plus jusqu'à ce qu'il soit accompli dans le Royaume de Dieu.
\VS{17}Et, ayant pris la coupe, il rendit grâces, et il dit : Prenez cette coupe, et distribuez-la entre vous.
\VS{18}Car, je vous dis, que je ne boirai plus du fruit de la vigne, jusqu'à ce que le Royaume de Dieu soit venu.
\TextTitle{Institution du repas de la Pâque\FTNTT{Mt. 26:26-29 ; Mc. 14:22-25 ; cp. Jn. 13:12-30 ; 1 Co. 11:23-26}}
\VS{19}Ensuite il prit du pain, et après avoir rendu grâces, il le rompit et le leur donna, en disant : Ceci est mon corps, qui est donné pour vous ; faites ceci en mémoire de moi.
\VS{20}Il prit de même la coupe, après le souper, et la leur donna, en disant : Cette coupe est la Nouvelle Alliance en mon sang, qui est répandu pour vous.
\TextTitle{Jésus annonce qu'il sera livré\FTNTT{Mt. 26:21-25 ; Mc. 14:18-21 ; Jn. 13:18-30}}
\VS{21}Cependant voici, la main de celui qui me trahit est avec moi à table.
\VS{22}Le Fils de l'homme s'en va ; selon ce qui est déterminé. Mais malheur à cet homme par qui il est trahi.
\VS{23}Et ils commencèrent à se demander les uns aux autres, qui était celui d'entre eux qui ferait cela.
\TextTitle{Leçon d'humilité\FTNTT{Mt. 20:20-28 ; Mc. 9.33-37 ; 10:35-45 ; Jn. 13:1-17}}
\VS{24}Il s'éleva une contestation parmi les apôtres, pour savoir lequel d'entre eux devait être estimé le plus grand.
\VS{25}Jésus leur dit : Les rois des nations les maîtrisent ; et ceux qui les dominent sont appelés bienfaiteurs.
\VS{26}Mais il n'en sera pas ainsi de vous : Au contraire, que le plus grand parmi vous soit comme le plus petit ; et celui qui gouverne, comme celui qui sert.
\VS{27}Car lequel est le plus grand, celui qui est à table, ou celui qui sert ? N'est-ce pas celui qui est à table ? Or je suis au milieu de vous comme celui qui sert.
\TextTitle{Le Royaume, une récompense}
\VS{28}Vous, vous êtes ceux qui avez persévéré avec moi dans mes épreuves ;
\VS{29}c'est pourquoi je vous confie le Royaume comme mon Père me l'a confié,
\VS{30}afin que vous mangiez et buviez à ma table dans mon Royaume, et que vous soyez assis sur des trônes, pour juger les douze tribus d'Israël.
\TextTitle{Jésus prophétise le triple reniement de Pierre\FTNTT{Mt. 26:30-35 ; Mc. 14:26-31 ; Jn. 13:36-38}}
\VS{31}Le Seigneur dit aussi : Simon, Simon, voici, Satan vous a réclamés pour vous cribler comme le froment ;
\VS{32}mais j'ai prié pour toi afin que ta foi ne défaille point ; et toi donc, quand tu seras un jour converti, affermis tes frères.
\VS{33}Pierre lui dit : Seigneur, je suis prêt à aller avec toi en prison et à la mort.
\VS{34}Mais Jésus lui dit : Pierre, je te dis que le coq ne chantera pas aujourd'hui, que tu n'aies nié trois fois de me connaître.
\TextTitle{Recommandation aux disciples\FTNTT{cp. Jn. 14-16 ; contraste Mt. 10:9-13}}
\VS{35}Puis il leur dit : Quand je vous ai envoyés sans bourse, sans sac, et sans souliers, avez-vous manqué de quelque chose ? Ils répondirent : De rien.
\VS{36}Et il leur dit : Maintenant au contraire, que celui qui a une bourse la prenne, et de même celui qui a un sac ; et que celui qui n'a point d'épée vende son vêtement, et achète une épée.
\VS{37}Car je vous le dis, il faut que cette parole qui est écrite s'accomplisse en moi : Il a été mis au nombre des malfaiteurs\FTNT{Es. 53:12.}. Parce qu'en effet, ce qui me concerne est sur le point d'arriver.
\VS{38}Ils dirent : Seigneur, voici ici deux épées. Et il leur dit : Cela suffit.
\TextTitle{Gethsémané\FTNTT{Mt. 26:36-46 ; Mc. 14:32-42 ; Jn. 18:1 ; cp. Hé. 5:7-8}}
\VS{39}Après être sorti, il alla, selon sa coutume, au Mont des Oliviers ; et ses disciples le suivirent.
\VS{40}Lorsqu'ils arrivèrent dans ce lieu, il leur dit : Priez afin que vous ne tombiez pas en tentation.
\VS{41}Puis s'étant éloigné d'eux à la distance d'environ un jet de pierre, et s'étant mis à genoux, il pria,
\VS{42}disant : Père, si tu voulais éloigner cette coupe loin de moi ; toutefois que ma volonté ne soit point faite, mais la tienne.
\VS{43}Et un ange lui apparut du ciel, pour le fortifier.
\VS{44}Etant en agonie, il priait plus instamment, et sa sueur devint comme des grumeaux de sang qui tombaient à terre.
\VS{45}Après avoir prié, il revint vers ses disciples, qu'il trouva endormis de tristesse ;
\VS{46}et il leur dit : Pourquoi dormez-vous ? Levez-vous, et priez, afin que vous ne tombiez pas en tentation.
\TextTitle{Trahison de Judas\FTNTT{Mt. 26:47-54 ; Mc. 14:43-47 ; Jn. 18:2-11}}
\VS{47}Et comme il parlait encore, voici une foule arriva ; et celui qui s'appelait Judas, l'un des douze, marchait devant elle. Il s'approcha de Jésus pour l'embrasser.
\VS{48}Et Jésus lui dit : Judas, c'est par un baiser que tu trahis le Fils de l'homme ?
\VS{49}Alors ceux qui étaient autour de lui, voyant ce qui allait arriver, lui dirent : Seigneur, frapperons-nous de l'épée ?
\VS{50}Et l'un d'eux frappa le serviteur du souverain sacrificateur, et lui emporta l'oreille droite.
\VS{51}Mais Jésus prenant la parole dit : Laissez-les faire jusqu'ici. Et, ayant touché son oreille, il le guérit.
\VS{52}Puis Jésus dit aux principaux sacrificateurs, aux chefs des gardes du temple, et aux anciens qui étaient venus contre lui : Etes-vous venus comme après un brigand avec des épées et des bâtons ?
\VS{53}J'étais tous les jours avec vous dans le temple, et vous n'avez pas mis la main sur moi. Mais c'est ici votre heure, et la puissance des ténèbres.
\TextTitle{Triple reniement de Pierre\FTNTT{Mt. 26:55-58,69-75 ; Mc. 14:48-54,66-72 ; Jn. 18:15-18,25-27}}
\VS{54}Après avoir saisi Jésus, ils l'emmenèrent, et le conduisirent dans la maison du souverain sacrificateur. Pierre suivait de loin.
\VS{55}Ils allumèrent du feu au milieu de la cour, et ils s'assirent ensemble. Pierre s'assit aussi parmi eux.
\VS{56}Une servante le voyant assis auprès du feu, fixa sur lui les regards, et dit : Celui-ci aussi était avec lui.
\VS{57}Mais il le nia, disant : Femme, je ne le connais point.
\VS{58}Peu après, un autre le voyant, dit : Tu es aussi de ces gens-là, mais Pierre dit : Ô homme ! Je n'en suis point.
\VS{59}Environ une heure plus tard, un autre affirmait et disait : Certainement celui-ci aussi était avec lui car il est Galiléen.
\VS{60}Pierre dit : Ô homme ! Je ne sais pas ce que tu dis. Au même instant, comme il parlait encore, le coq chanta.
\VS{61}Et le Seigneur, s'étant retourné, regarda Pierre. Et Pierre se souvint de la parole que le Seigneur lui avait dite : Avant que le coq chante, tu me renieras trois fois.
\VS{62}Alors Pierre étant sorti dehors, pleura amèrement.
\TextTitle{Jésus est outragé\FTNTT{Mt. 26:67-68 ; Mc. 14:65 ; Jn. 18:22-23}}
\VS{63}Les hommes qui tenaient Jésus se moquaient de lui, et le frappaient.
\VS{64}Ils lui bandèrent les yeux, ils lui donnaient des coups sur le visage, et l'interrogeaient, disant : Devine qui est celui qui t'a frappé ?
\VS{65}Et ils proféraient contre lui beaucoup d'autres injures.
\TextTitle{Jésus déclare qu'il est fils de Dieu\FTNTT{Mt. 26:59-68 ; 27:1 ; Mc. 14:55-65 ; 15:1 ; Jn. 18:19-24}}
\VS{66}Quand le jour fut venu, les anciens du peuple, les principaux sacrificateurs, et les scribes, s'assemblèrent, et firent amener Jésus dans le sanhédrin.
\VS{67}Ils dirent : Si tu es le Christ, dis-le-nous. Et il leur répondit : Si je vous le dis, vous ne le croirez point ;
\VS{68}et si je vous interroge, vous ne me répondrez pas, et vous ne me laisserez pas aller.
\VS{69}Désormais le Fils de l'homme sera assis à la droite de la puissance de Dieu.
\VS{70}Alors ils dirent tous : Tu es donc le Fils de Dieu ? Et il leur répondit : Vous le dites vous-mêmes, je le suis.
\VS{71}Alors ils dirent : Qu'avons-nous besoin encore de témoignage ? Nous l'avons entendu nous-mêmes de sa bouche.
\Chap{23}
\TextTitle{Jésus devant Pilate\FTNTT{Mt. 27:2,11-14 ; Mc. 15:1-5 ; Jn. 18:28-38}}
\VerseOne{}Puis ils se levèrent tous, et ils conduisirent Jésus devant Pilate.
\VS{2}Et ils se mirent à l'accuser, disant : Nous avons trouvé cet homme excitant notre nation à la révolte, et empêchant de payer le tribut à César, et se disant lui-même Christ, Roi.
\VS{3}Pilate l'interrogea, disant : Es-tu le Roi des Juifs ? Et Jésus lui répondit : Tu le dis.
\VS{4}Alors Pilate dit aux principaux sacrificateurs et à la foule : Je ne trouve aucun crime en cet homme.
\VS{5}Mais ils insistèrent, et dirent : Il soulève le peuple, enseignant par toute la Judée, depuis la Galilée où il a commencé, jusqu'ici.
\TextTitle{Jésus envoyé devant Hérode par Pilate}
\VS{6}Quand Pilate entendit parler de la Galilée, il demanda si cet homme était Galiléen,
\VS{7}et, ayant appris qu'il était de la juridiction d'Hérode, il le renvoya à Hérode, qui se trouvait aussi à Jérusalem.
\VS{8}Lorsque Hérode vit Jésus, il en eut une grande joie ; car depuis longtemps il désirait le voir, à cause de ce qu'il avait entendu dire de lui, et il espérait qu'il le verrait faire quelque miracle.
\VS{9}Il lui adressa beaucoup de questions ; mais Jésus ne lui répondit rien.
\VS{10}Les principaux sacrificateurs et les scribes étaient là, et l'accusaient avec violence.
\VS{11}Mais Hérode, avec ses gardes, le traita avec mépris ; et, après s'être moqué de lui et l'avoir revêtu d'un vêtement éclatant, il le renvoya à Pilate.
\VS{12}Ce même jour, Pilate et Hérode devinrent amis ; car auparavant ils étaient ennemis.
\TextTitle{Hérode renvoie Jésus à Pilate\FTNTT{Mt. 27:15-26 ; Mc. 15:6-15 ; Jn. 18:39-19:15}}
\VS{13}Pilate, ayant assemblé les principaux sacrificateurs, les magistrats, et le peuple, leur dit :
\VS{14}Vous m'avez présenté cet homme comme soulevant le peuple. Et voici, je l'ai interrogé devant vous, et je ne l'ai trouvé coupable d'aucun des crimes dont vous l'accusez.
\VS{15}Hérode non plus ; car il nous l'a renvoyé, et voici, cet homme n'a rien fait qui soit digne de mort.
\VS{16}Je le relâcherai donc, après l'avoir châtié.
\VS{17}A chaque fête, il était obligé de leur relâcher un prisonnier.
\VS{18}Toutes les foules s'écrièrent ensemble, disant : Ôte celui-ci, et relâche-nous Barabbas.
\VS{19}Cet homme avait été mis en prison pour une sédition qui avait eu lieu dans la ville, et pour un meurtre.
\VS{20}Pilate leur parla de nouveau, ayant envie de relâcher Jésus.
\VS{21}Et ils crièrent : Crucifie, crucifie-le !
\VS{22}Pilate leur dit pour la troisième fois : Mais quel mal a fait cet homme ? Je ne trouve rien en lui qui soit digne de mort. Après l'avoir fait battre de verges, je le relâcherai.
\VS{23}Mais ils insistèrent à grands cris, demandant qu'il soit crucifié ; et leurs cris et ceux des principaux sacrificateurs l'emportèrent.
\VS{24}Alors Pilate prononça que ce qu'ils demandaient, serait fait.
\VS{25}Il leur relâcha celui qui avait été mis en prison pour sédition et pour meurtre, et qu'ils demandaient ; et il abandonna Jésus à leur volonté.
\TextTitle{Sur le chemin de Golgotha\FTNTT{Mt. 27:31-32 ; Mc. 15:20-21 ; Jn. 19:16-17}}
\VS{26}Comme ils l'emmenaient, ils prirent un certain Simon, de Cyrène, qui revenait des champs, et le chargèrent de la croix pour qu'il la porte derrière Jésus.
\VS{27}Il était suivi d'une grande multitude des gens du peuple et de femmes, qui se frappaient la poitrine, et se lamentaient sur lui.
\VS{28}Mais Jésus se tourna vers elles, leur dit : Filles de Jérusalem, ne pleurez point sur moi, mais pleurez sur vous-mêmes, et sur vos enfants.
\VS{29}Car voici, des jours viendront où l'on dira : Heureuses les stériles, les entrailles qui n'ont point enfanté, et les mamelles qui n'ont point allaité !
\VS{30}Alors ils se mettront à dire aux montagnes : Tombez sur nous ; et aux collines : Couvrez-nous !
\VS{31}Car s'ils font ces choses au bois vert, que sera-t-il fait au bois sec ?
\VS{32}On conduisait en même temps deux malfaiteurs, qui devaient être mis à mort avec Jésus.
\TextTitle{Crucifixion de Jésus\FTNTT{Mt. 27:33-43 ; Mc. 15:24-32 ; Jn. 19:17-37}}
\VS{33}Lorsqu'ils furent arrivés au lieu qui est appelé Calvaire (le Crâne), ils le crucifièrent là, et les malfaiteurs aussi, l'un à la droite, et l'autre à la gauche.
\VS{34}Jésus dit : Père, pardonne-leur, car ils ne savent pas ce qu'ils font. Ils se partagèrent ensuite ses vêtements, en tirant au sort.
\VS{35}Le peuple se tenait là, et regardait. Les magistrats se moquaient de Jésus disant : Il a sauvé les autres, qu'il se sauve lui-même, s'il est le Christ, l'élu de Dieu.
\VS{36}Les soldats aussi se moquaient de lui ; s'approchant et lui présentant du vinaigre,
\VS{37}ils disaient : Si tu es le Roi des Juifs, sauve-toi toi-même !
\VS{38}Or il y avait au-dessus de lui un écriteau en lettres Grecques, Romaines et Hébraïques, en ces mots : Celui-ci est le roi des juifs.
\TextTitle{Repentance du malfaiteur crucifié\FTNTT{cp. Mt. 27:44 ; Mc. 15:32}}
\VS{39}L'un des malfaiteurs qui étaient crucifiés, l'outrageait, disant : Si tu es le Christ, sauve-toi toi-même, et sauve-nous !
\VS{40}Mais l'autre le reprenait, et disait : Ne crains-tu pas Dieu, car tu es condamné au même supplice ?
\VS{41}Pour nous, c'est juste, car nous recevons ce qu'ont mérité nos crimes ; mais celui-ci n'a fait aucun mal.
\VS{42}Et il dit à Jésus : Seigneur ! Souviens-toi de moi quand tu viendras dans ton règne.
\VS{43}Jésus lui dit : Je te le dis en vérité, aujourd'hui tu seras avec moi dans le paradis.
\TextTitle{Jésus remet son esprit\FTNTT{Mt. 27:45-56 ; Mc. 15:33-41 ; Jn. 19:30-37}}
\VS{44}Il était déjà environ la sixième heure, et il eut des ténèbres sur toute la terre jusqu'à la neuvième heure.
\VS{45}Le soleil s'obscurcit, et le voile du temple se déchira par le milieu.
\VS{46}Et Jésus criant à haute voix, dit : Père, je remets mon esprit entre tes mains ! Et, en disant cela, il expira\FTNT{C'est la fin de la Première Alliance. Voir commentaire Jn. 19:30.}.
\TextTitle{Fin de la loi mosaïque ou de la Première Alliance}
\VS{47}Le centenier, voyant ce qui était arrivé, glorifia Dieu, et dit : Certes, cet homme était juste.
\VS{48}Et tous ceux qui assistaient en foule à ce spectacle, après avoir vu ce qui était arrivé, s'en retournèrent, se frappant la poitrine.
\VS{49}Tous ceux qui connaissaient Jésus, et les femmes qui l'avaient suivi de Galilée, se tenaient dans l'éloignement et regardaient ces choses.
\TextTitle{Sépulture de Jésus\FTNTT{Mt. 27:57-61 ; Mc. 15:42-47 ; Jn. 19:38-42}}
\VS{50}Il y avait un conseiller, nommé Joseph, homme bon et juste,
\VS{51}qui n'avait point participé au conseil et aux actes des autres ; il était d'Arimathée, ville des Juifs, et il attendait le Royaume de Dieu.
\VS{52}Cet homme se rendit vers Pilate et lui demanda le corps de Jésus.
\VS{53}Il le descendit de la croix, l'enveloppa d'un linceul, et le déposa dans un sépulcre taillé dans le roc, où personne n'avait encore été mis.
\VS{54}C'était le jour de la préparation, et le sabbat allait commencer.
\VS{55}Les femmes qui étaient venues de Galilée avec Jésus, accompagnèrent Joseph, virent le sépulcre, et la manière dont le corps de Jésus y fut déposé.
\VS{56}Et s'en étant retournées, elles préparèrent des aromates et des parfums ; et le jour du sabbat elles se reposèrent selon la loi.
\Chap{24}
\TextTitle{Résurrection du Messie\FTNTT{Mt. 28:1-15 ; Mc. 16:1-11 ; Jn. 20:1-18}}
\VerseOne{}Le premier jour de la semaine, elles se rendirent au sépulcre de grand matin, apportant les aromates qu'elles avaient préparés.
\VS{2}Elles trouvèrent la pierre roulée à côté du sépulcre.
\VS{3}Et, étant entrées, elles ne trouvèrent point le corps du Seigneur Jésus.
\VS{4}Comme elles ne savaient que penser de cela, voici, deux hommes leur apparurent en habits resplendissants.
\VS{5}Saisies de frayeur, elles baissèrent le visage contre terre, mais ils leur dirent : Pourquoi cherchez-vous parmi les morts celui qui est vivant ?
\VS{6}Il n'est point ici, mais il est ressuscité. Souvenez-vous comment il vous a parlé quand il était encore en Galilée,
\VS{7}et qu'il disait : Il faut que le Fils de l'homme soit livré entre les mains des pécheurs, et qu'il soit crucifié, et qu'il ressuscite le troisième jour.
\VS{8}Et elles se souvinrent de ses paroles.
\VS{9}A leur retour du sépulcre, elles annoncèrent toutes ces choses aux onze disciples, et à tous les autres.
\VS{10}Or c'étaient Marie de Magdala, Jeanne, Marie, mère de Jacques, et les autres qui étaient avec elles, qui dirent ces choses aux apôtres.
\VS{11}Mais les paroles de ces femmes leur semblèrent comme des paroles futiles, et ils ne les crurent point.
\VS{12}Mais Pierre s'étant levé, courut au sépulcre et s'étant courbé pour regarder, il ne vit que les linges là tout seuls, puis il s'en alla chez lui, dans l'étonnement de ce qui était arrivé.
\TextTitle{Jésus et les deux disciples sur le chemin d'Emmaüs\FTNTT{Mc. 16:12-13}}
\VS{13}Or voici, deux d'entre eux étaient ce jour-là en chemin, pour aller à un village nommée Emmaüs, éloigné de Jérusalem de soixante stades.
\VS{14}Et ils s'entretenaient ensemble de toutes ces choses qui étaient arrivées.
\VS{15}Et il arriva que, comme ils s'entretenaient et discutaient entre eux, Jésus lui-même s'approcha et se mit à marcher avec eux.
\VS{16}Mais leurs yeux étaient retenus de sorte qu'ils ne le reconnaissaient pas.
\VS{17}Et il leur dit : Quels sont ces discours que vous tenez ensemble en marchant ? Et pourquoi êtes-vous tout tristes ?
\VS{18}Et l'un d'eux, nommé Cléopas, lui répondit, et lui dit : Es-tu le seul étranger dans Jérusalem qui ne sache point les choses qui s'y sont passées ces jours-ci ?
\VS{19}Et il leur dit : Quelles ? Ils répondirent : Celles concernant Jésus de Nazareth, qui était un prophète puissant en œuvres et en paroles devant Dieu, et devant tout le peuple.
\VS{20}Et comment les principaux sacrificateurs et nos magistrats l'ont livré pour être condamné à mort, et l'ont crucifié.
\VS{21}Or nous espérions que ce serait lui qui délivrerait Israël ; mais avec tout cela, c'est aujourd'hui le troisième jour que ces choses sont arrivées.
\VS{22}Toutefois quelques femmes d'entre nous nous ont fort étonnés, car elles ont été de grand matin au sépulcre
\VS{23}et n'ayant point trouvé son corps, elles sont venues dire que même elles avaient vu une apparition d'anges, qui disaient qu'il est vivant.
\VS{24}Et quelques-uns des nôtres sont allés au sépulcre, et ont trouvé les choses comme les femmes l'avaient dit ; mais lui, ils ne l'ont point vu.
\VS{25}Alors Jésus leur dit : Ô gens sans intelligence, et dont le cœur est lent à croire tout ce que les prophètes ont annoncé !
\VS{26}Ne fallait-il pas que le Christ souffrît ces choses, et qu'il entra dans sa gloire ?
\VS{27}Puis commençant par Moïse, et continuant par tous les prophètes, il leur expliquait dans toutes les Ecritures ce qui le concernait.
\VS{28}Et comme ils furent près du village où ils allaient, il faisait comme s'il voulait aller plus loin.
\VS{29}Mais ils le forcèrent, en lui disant : Reste avec nous, car le soir approche et le jour commence à baisser. Et il entra donc pour rester avec eux.
\VS{30}Et il arriva comme il était à table avec eux, il prit le pain, et il le bénit ; et l'ayant rompu, il leur distribua.
\VS{31}Alors leurs yeux s'ouvrirent, et ils le reconnurent ; mais il disparut de devant eux.
\VS{32}Et ils se dirent l'un à l'autre : Notre cœur ne brûlait-il pas au-dedans de nous lorsqu'il nous parlait en chemin, et qu'il nous ouvrait les Ecritures ?
\TextTitle{Nouvelles apparitions du réssuscité\FTNTT{Mc. 16:14 ; Jn. 20:19-25 ; cp. Jn. 20:26-21:25}}
\VS{33}Et se levant à l'heure même, ils retournèrent à Jérusalem, et ils trouvèrent assemblés les onze et ceux qui étaient avec eux,
\VS{34}qui disaient : Le Seigneur est véritablement ressuscité, et il est apparu à Simon.
\VS{35}A leur tour, ils racontèrent ce qui leur était arrivé en chemin, et comment il avait été reconnu d'eux en rompant le pain.
\VS{36}Comme ils tenaient ces discours, Jésus se présenta lui-même au milieu d'eux, et leur dit : Que la paix soit avec vous !
\VS{37}Mais eux tout terrifiés et effrayés croyaient voir un esprit.
\VS{38}Et il leur dit : Pourquoi êtes-vous troublés, et pourquoi monte-t-il des pensées dans vos coeurs ?
\VS{39}Voyez mes mains et mes pieds, c'est bien moi. Touchez-moi, et voyez : Car un esprit n'a ni chair ni os, comme vous voyez que j'ai.
\VS{40}Et en disant cela, il leur montra ses mains et ses pieds.
\VS{41}Mais comme de joie, ils ne croyaient point encore, et qu'ils s'étonnaient, il leur dit : Avez-vous ici quelque chose à manger ?
\VS{42}Et ils lui présentèrent un morceau de poisson rôti, et un rayon de miel.
\VS{43}Et l'ayant pris, il mangea devant eux.
\TextTitle{La nouvelle mission des onze\FTNTT{Mt. 28:18-20 ; Mc. 16:15-18 ; Jn. 17:18 ; 20:21 ; Ac. 1:8}}
\VS{44}Puis il leur dit : Ce sont ici les paroles que je vous disais lorsque j'étais encore avec vous, qu'il fallait que s'accomplisse tout ce qui est écrit de moi dans la loi de Moïse, dans les prophètes, et dans les psaumes.
\VS{45}Alors il leur ouvrit l'esprit\FTNT{Pour comprendre les Ecritures, nous avons besoin de l'aide de l'Esprit de Dieu. La vraie connaissance ne vient pas des hommes, mais de Dieu (Da. 9:22).} afin qu'ils comprennent les Ecritures.
\VS{46}Et il leur dit : Il est ainsi écrit, et ainsi il fallait que le Christ souffre, et qu'il ressuscite des morts le troisième jour,
\VS{47}et que la repentance et le pardon des péchés seraient prêchés en son Nom à toutes les nations, à commencer par Jérusalem\FTNT{Es. 53.}.
\VS{48}Et vous êtes témoins de ces choses. 
\VS{49}Et voici, j'enverrai sur vous la promesse de mon Père, mais vous donc restez dans la ville de Jérusalem, jusqu'à ce que vous soyez revêtus de la puissance d'en haut.
\TextTitle{Jésus enlevé au ciel\FTNTT{Mc. 16:19-20 ; Ac. 1:9-11}}
\VS{50}Après quoi il les conduisit dehors jusqu'en Béthanie, et levant ses mains en haut, il les bénit.
\VS{51}Et pendant qu'il les bénissait, il se sépara d'eux, et fut élevé au ciel.
\VS{52}Pour eux, après l'avoir adoré, ils retournèrent à Jérusalem avec une grande joie.
\VS{53}Et ils étaient toujours dans le temple, louant et bénissant Dieu. Amen !
\PPE{}
\end{multicols}

\clearpage\ShortTitle{Jean}\BookTitle{Jean}\BFont
\noindent\hrulefill
\textit{
\bigskip
{\centering{}
\\Signifie : Dieu pardonne, don de Dieu
\\Thème : Christ, Dieu
\\Auteur : Jean
\\Date de rédaction : Env. 85-90 apr. J.-C.\\}
}
%\bigskip
\textit{
\\Auteur d’un des quatre évangiles, des trois épîtres éponymes et de l’Apocalypse, Jean, fils de Zébédée, fut l’un des douze. Témoin oculaire du ministère terrestre de Jésus-Christ, il attesta par l’essence de ses écrits le caractère divin de ce dernier.
\bigskip
\\Fidèle au livre d’Exode où Yahweh se révéla comme étant « Je suis », Jean reprit les propos de Jésus et le présenta comme la Parole incarnée, le Pain de vie, la Lumière du monde, la Porte des brebis, le Bon berger, la Résurrection, la Vie… Proche du maître, Jean fut à même de relater les évènements marquants de sa vie comme la gloire de la Transfiguration, l’angoisse de la passion exprimée à Gethsémané, ou encore les déclarations solennelles précédées de l’expression «  En vérité, en vérité »… Il mit également en évidence la controverse suscitée par le Christ et l’opposition dont il fit l’objet de la part de certains pharisiens qui souhaitaient sa mort.
\bigskip
\\L’évangile de Jean exprime la nécessité de la nouvelle naissance et dévoile les attributs du Fils de Dieu, le Messie tant attendu.\bigskip
}
\par\nobreak\noindent\hrulefill
\begin{multicols}{2}
\TextTitle{[La divinité de Jésus-Christ]
\\(Jn. 10:30 ; Hé. 1:5-13)}
\Chap{1}
\VerseOne{}Au commencement était la Parole, et la Parole était avec Dieu, et la Parole était Dieu.
\VS{2}Elle était au commencement avec Dieu.
\TextTitle{[L'oeuvre de Jésus avant son incarnation]}
\VS{3}Toutes choses ont été faites par elle, et rien de ce qui a été fait, n'a été fait sans elle.
\VS{4}En elle était la vie, et la vie était la Lumière des hommes\FTNT{Jésus-Christ notre Lumière : Es. 60:19-20.}.
\VS{5}Et la Lumière luit dans les ténèbres, mais les ténèbres ne l'ont point reçue.
\TextTitle{[Ministère de Jean-Baptiste]}
\VS{6}Il y eut un homme appelé Jean, qui fut envoyé de Dieu.
\VS{7}Il vint pour servir de témoin, pour rendre témoignage à la Lumière, afin que tous croient par lui.
\VS{8}Il n'était pas la Lumière, mais il était envoyé pour rendre témoignage à la Lumière.
\TextTitle{[Jésus-Christ, la véritable lumière]
\\(Jn. 3:17-21 ; 8:12 ; 9:5 ; 12:46)}
\VS{9}Cette Lumière était la véritable Lumière, qui en venant dans le monde éclaire tout homme.
\VS{10}Elle était dans le monde, et le monde a été fait par elle ; mais le monde ne l'a point connue.
\VS{11}Elle est venue chez les siens ; et les siens ne l'ont point reçue.
\VS{12}Mais à tous ceux qui l'ont reçue, à ceux qui croient en son Nom, elle leur a donné le pouvoir de devenir enfants de Dieu.
\VS{13}Lesquels sont nés, non du sang, ni de la volonté de la chair, ni de la volonté de l'homme ; mais ils sont nés de Dieu.
\TextTitle{[La Parole faite chair]
\\(Jn. 14:9 ; Mt. 1:18-23 ; Lu. 1:30-35 ; 2:11 ; 1Tim. 3:16)}
\VS{14}Et la Parole a été faite chair, elle a habité parmi nous, pleine de grâce et de vérité, et nous avons contemplé sa gloire, une gloire, comme la gloire du Fils unique du Père.
\TextTitle{[Premier témoignage de Jean-Baptiste]
\\(Mt. 3:1-12 ; Mc. 1:1-11 ; Lu. 3:1-22)}
\VS{15}Jean a donc rendu témoignage de lui, et s’est écrié, disant : C'est celui dont j’ai dit : Celui qui vient après moi m’a précédé, car il était avant moi.
\VS{16}Et nous avons tous reçu de sa plénitude, et grâce pour grâce.
\VS{17}Car la loi\FTNT{La loi a été promulguée par Moïse.} a été donnée par Moïse, la grâce et la vérité sont venues par Jésus-Christ.
\VS{18}Personne n’a jamais vu Dieu, le Fils unique qui est dans le sein du Père, est celui qui nous l'a révélé.
\VS{19}Et c'est ici le témoignage de Jean, lorsque les Juifs envoyèrent de Jérusalem des sacrificateurs et des lévites pour l'interroger, et lui dire : Toi qui es-tu ?
\VS{20}Il confessa, et ne le nia point, il déclara, en disant : Ce n'est pas moi qui suis le Christ.
\VS{21}Et ils lui demandèrent : Quoi donc ? Es-tu Elie ? Et il dit : Je ne le suis point\FTNT{En Mt. 11:14 Jésus confirme pourtant que Jean-Baptiste est bien l’Elie qui devait venir. Comment expliquer qu’il nia l’être lorsqu’il fut interrogé par les pharisiens ? La seule explication plausible c’est qu’il l’ignorait. Toutefois, il avait conscience qu’il était ~la voix~ prophétisée par Esaïe. Remarquez que lorsqu’il fut emprisonné, il avait envoyé quelques-uns de ses disciples pour demander à Jésus s’il était bien le Messie (Mt. 11:13 ; Lu. 7:19-20) alors qu’il fut le premier à rendre témoignage du Seigneur. Ces éléments ne sont pas contradictoires, ils ne font que révéler les failles liées à la nature humaine de Jean.}. Es-tu le Prophète ? Et il répondit : Non.
\VS{22}Ils lui dirent donc : Qui es-tu, afin que nous donnions une réponse à ceux qui nous ont envoyés. Que dis-tu de toi-même ?
\VS{23}Il dit : Je suis la voix de celui qui crie dans le désert : Aplanissez le chemin du Seigneur, comme a dit Esaïe le prophète\FTNT{Es. 40:3.}.
\VS{24}Or ceux qui avaient été envoyés vers lui étaient des pharisiens.
\VS{25}Ils l'interrogèrent encore, et lui dirent : Pourquoi donc baptises-tu si tu n'es point le Christ, ni Elie, ni le Prophète ?
\VS{26}Jean leur répondit : Pour moi, je baptise d'eau ; mais il y a quelqu’un au milieu de vous que vous ne connaissez point.
\VS{27}C'est celui qui vient après moi, il m’a précédé, et je ne suis pas digne de délier la courroie de ses souliers.
\VS{28}Ces choses se passèrent à Béthanie, au-delà du Jourdain, où Jean baptisait.
\VS{29}Le lendemain Jean vit Jésus venir à lui, et il dit : Voici l'Agneau de Dieu, qui ôte le péché du monde.
\VS{30}C'est celui dont j’ai dit : Après moi vient un homme qui m’a précédé ; car il était avant moi.
\VS{31}Et pour moi, je ne le connaissais pas ; mais c'est afin qu'il soit manifesté à Israël que je suis venu baptiser d'eau.
\VS{32}Jean rendit aussi témoignage, en disant : J'ai vu l'Esprit descendre du ciel comme une colombe, et s'arrêter sur lui.
\VS{33}Et pour moi, je ne le connaissais point ; mais celui qui m'a envoyé baptiser d'eau, m'avait dit : Celui sur qui tu verras l'Esprit descendre et s’arrêter, c'est celui qui baptise du Saint-Esprit.
\VS{34}Et je l'ai vu, et j'ai rendu témoignage, que c'est lui qui est le Fils de Dieu.
\TextTitle{[Premiers disciples de Jésus-Christ]
\\(Mt. 4:18-22 ; Mc. 1:16-20 ; Lu. 5:1-11)}
\VS{35}Le lendemain Jean était encore là, avec deux de ses disciples ;
\VS{36}et regardant Jésus qui marchait, il dit : Voici l'Agneau de Dieu.
\VS{37}Les deux disciples l'entendirent prononcer ces paroles, et ils suivirent Jésus.
\VS{38}Et Jésus se retournant, et voyant qu'ils le suivaient, il leur dit : Que cherchez-vous ? Ils lui répondirent : Rabbi, c'est-à-dire Maître, où demeures-tu ?
\VS{39}Il leur dit : Venez, et voyez. Ils y allèrent, et ils virent où il demeurait ; et ils demeurèrent avec lui ce jour-là ; car il était environ dix heures.
\VS{40}André, frère de Simon Pierre, était l'un des deux qui avaient entendu les paroles de Jean et qui avaient suivi Jésus.
\VS{41}Ce fut lui qui rencontra le premier Simon son frère, et il lui dit : Nous avons trouvé le Messie, c'est-à-dire le Christ.
\VS{42}Et il le conduisit vers Jésus, et Jésus l’ayant regardé, dit : Tu es Simon, fils de Jonas, tu seras appelé Céphas ; c'est-à-dire, Pierre.
\VS{43}Le lendemain Jésus voulut se rendre en Galilée, et il trouva Philippe. Et il lui dit : Suis-moi.
\VS{44}Philippe était de Bethsaïda, la ville d'André et de Pierre.
\VS{45}Philippe rencontra Nathanaël, et lui dit : Nous avons trouvé celui de qui Moïse a écrit dans la loi, et dont les prophètes ont parlé, Jésus, qui est de Nazareth, fils de Joseph.
\VS{46}Et Nathanaël lui dit : Peut-il venir quelque chose de bon de Nazareth ? Philippe lui dit : Viens, et vois.
\VS{47}Jésus aperçut Nathanaël venir vers lui, et il dit de lui : Voici vraiment un Israëlite dans lequel il n'y a point de fraude.
\VS{48}Nathanaël lui dit : D'où me connais-tu ? Jésus répondit et lui dit : Avant que Philippe t’appelle, quand tu étais sous le figuier, je t’ai vu.
\VS{49}Nathanaël répondit et lui dit : Maître, tu es le Fils de Dieu. Tu es le Roi d'Israël.
\VS{50}Jésus lui répondit et dit : Parce que je t'ai dit que je t’ai vu sous le figuier, tu crois. Tu verras des choses plus grandes encore.
\VS{51}Il lui dit aussi : En vérité, en vérité je vous dis : Désormais vous verrez le ciel ouvert, et les anges de Dieu monter et descendre sur le Fils de l'homme.
\TextTitle{[Premier miracle, à Cana]}
\Chap{2}
\VerseOne{}Trois jours après, il y eut des noces à Cana en Galilée, et la mère de Jésus était là.
\VS{2}Et Jésus fut aussi convié aux noces avec ses disciples.
\VS{3}Et le vin ayant manqué, la mère de Jésus lui dit : Ils n'ont plus de vin.
\VS{4}Jésus lui répondit : Qu'y a-t-il entre moi et toi, femme ? Mon heure n'est point encore venue.
\VS{5}Sa mère dit aux serviteurs : Faites tout ce qu'il vous dira.
\VS{6}Or il y avait là six vases de pierre, destinés aux purifications des Juifs, et contenant chacun deux ou trois mesures.
\VS{7}Et Jésus leur dit : Remplissez d'eau ces vases. Et ils les remplirent jusqu’au bord.
\VS{8}Puis il leur dit : Puisez maintenant, et apportez-en au maître d'hôtel. Et ils lui en apportèrent.
\VS{9}Quand le maître d'hôtel eut goûté l'eau changée en vin, ne sachant d'où venait ce vin, tandis que les serviteurs qui avaient puisé l'eau le savaient bien, il s'adressa à l'époux
\VS{10}et lui dit : Tout homme sert d’abord le bon vin, et ensuite le moins bon, après qu’on s’est enivré ; mais toi, tu as gardé le bon vin jusqu'à maintenant.
\VS{11}Jésus fit ce premier miracle à Cana en Galilée, et il manifesta sa gloire, et ses disciples crurent en lui.
\VS{12}Après cela, il descendit à Capernaüm avec sa mère, et ses frères, et ses disciples ; mais ils y demeurèrent peu de jours.
\TextTitle{[La première Pâque]
\\(Jn. 6:4 ; 11:55)}
\VS{13}La Pâque des Juifs était proche ; c'est pourquoi Jésus monta à Jérusalem.
\VS{14}Et il trouva dans le temple des vendeurs de bœufs, de brebis, et de pigeons ; et les changeurs qui y étaient assis.
\VS{15}Et ayant fait un fouet avec des petites cordes, il les chassa tous du temple, avec les brebis, et les bœufs ; et il dispersa la monnaie des changeurs, et renversa les tables.
\VS{16}Et il dit aux vendeurs des pigeons : Otez ces choses d'ici, et ne faites pas de la maison de mon Père une maison de marché.
\VS{17}Alors ses disciples se souvinrent qu'il était écrit : Le zèle de ta maison me dévore\FTNT{Ps. 69:10.}.
\VS{18}Mais les Juifs prenant la parole, lui dirent : Quel signe nous montres-tu pour agir de la sorte ?
\VS{19}Jésus répondit et leur dit : Détruisez ce temple, et en trois jours je le relèverai.
\VS{20}Et les Juifs dirent : Il a fallu quarante-six ans pour bâtir ce temple, et toi, tu le relèveras en trois jours !
\VS{21}Mais il parlait du temple de son corps.
\VS{22}C'est pourquoi lorsqu'il fut ressuscité des morts, ses disciples se souvinrent qu'il leur avait dit cela, et ils crurent à l'Ecriture et à la parole que Jésus avait dite.
\VS{23}Et comme il était à Jérusalem le jour de la fête de Pâque, plusieurs crurent en son Nom, voyant les miracles qu'il faisait.
\VS{24}Mais Jésus ne se fiait point à eux, parce qu'il les connaissait tous ;
\VS{25}et parce qu'il n'avait pas besoin qu’on lui rende témoignage d'aucun homme ; car il savait lui-même ce qui était dans l'homme.
\TextTitle{[Jésus et Nicodème : la naissance d'en haut]}
\Chap{3}
\VerseOne{}Mais il y eut un homme d'entre les pharisiens, nommé Nicodème, qui était un des chefs des Juifs,
\VS{2}qui vint de nuit auprès de Jésus, et lui dit : Rabbi, nous savons que tu es un Docteur venu de Dieu, car personne ne peut faire les miracles que tu fais, si Dieu n'est avec lui.
\VS{3}Jésus lui répondit et dit : En vérité, en vérité je te le dis : Si quelqu'un ne naît d’en haut\FTNT{Naître d’en haut : Dans la plupart des Bibles modernes, on trouve l’expression naître « de nouveau », or cette traduction n’est pas correcte puisque le texte grec utilise l'expression naître « d’en haut ». L’adverbe « d’en haut » vient du mot grec « ahothen » qui signifie : depuis le haut, depuis un endroit plus élevé, ce qui vient des cieux ou de Dieu, depuis le début, l'origine. Ce mot se retrouve dans Mt. 27:51 ; Mc. 15:38 ; Lu. 1:3 ; Jn. 3:31 ; Jn. 19:11 ; Jn. 19:23 ; Ja. 1:17 ; Ja. 3:15 ; Ja. 3:17. ~Anothen~ vient de ~ano~ : choses d’en haut. En Ga. 4:26 ~ano~ peut se référer au lieu ou au temps. Le lieu : La Jérusalem qui est au-dessus, dans les cieux. Le temps : La Jérusalem éternelle qui a précédé la terrestre. Le mot ~ano~ a été traduit par ~en haut~ dans Jn. 8:23 ; Jn. 11:41 ; Ac. 2:19 ; Ga. 4:26 ; Col. 3:1-2 ; et par ~céleste~ dans Ph. 3:14. Jésus nous enseigne donc que la nouvelle naissance est en réalité la naissance d’en haut, une naissance qui a eu lieu dans la Nouvelle Jérusalem.}, il ne peut voir le Royaume de Dieu.
\VS{4}Nicodème lui dit : Comment un homme peut-il naître quand il est vieux ? Peut-il rentrer dans le sein de sa mère et naître une seconde fois ?
\VS{5}Jésus répondit : En vérité, en vérité je te dis : Si quelqu’un ne naît d'eau et d'Esprit, il ne peut entrer dans le Royaume de Dieu.
\VS{6}Ce qui est né de la chair est chair ; et ce qui est né de l'Esprit est esprit.
\VS{7}Ne t'étonne pas de ce que je t'ai dit : Il faut que vous naissiez d’en haut.
\VS{8}Le vent souffle où il veut, et tu en entends le bruit ; mais tu ne sais pas d'où il vient ni où il va : Il en est ainsi de tout homme qui est né de l'Esprit.
\VS{9}Nicodème lui dit : Comment cela peut-il se faire ?
\VS{10}Jésus répondit et lui dit : Tu es le docteur d'Israël, et tu ne connais point ces choses !
\VS{11}En vérité, en vérité je te le dis, nous disons ce que nous savons, et nous rendons témoignage de ce que nous avons vu ; et vous ne recevez pas notre témoignage.
\VS{12}Si vous ne croyez pas quand je vous ai parlé des choses terrestres, comment croirez-vous quand je vous parlerai des choses célestes ?
\VS{13}Personne n'est monté au ciel, si ce n’est celui qui est descendu du ciel, le Fils de l'homme qui est dans le ciel.
\VS{14}Et comme Moïse éleva le serpent\FTNT{Le serpent d’airain : No. 21:9}dans le désert, il faut de même que le Fils de l'homme soit élevé,
\VS{15}afin que quiconque croit en lui ne périsse point, mais qu'il ait la vie éternelle.
\VS{16}Car Dieu a tant aimé le monde, qu'il a donné son Fils unique, afin que quiconque croit en lui ne périsse point, mais qu'il ait la vie éternelle.
\VS{17}Car Dieu n'a point envoyé son Fils dans le monde pour condamner le monde, mais afin que le monde soit sauvé par lui.
\VS{18}Celui qui croit en lui ne sera point jugé ; mais celui qui ne croit point est déjà jugé ; parce qu'il n'a point cru au Nom du Fils unique de Dieu.
\VS{19}Et ce jugement c’est que la lumière est venue dans le monde et que les hommes ont préféré les ténèbres à la lumière, parce que leurs œuvres étaient mauvaises.
\VS{20}Car quiconque fait le mal, hait la lumière, et ne vient point à la lumière, de peur que ses œuvres ne soient condamnées.
\VS{21}Mais celui qui agit selon la vérité, vient à la lumière, afin que ses œuvres soient manifestées, parce qu'elles sont faites selon Dieu.
\TextTitle{[Nouveau témoignage de Jean-Baptiste]}
\VS{22}Après ces choses, Jésus s’en alla avec ses disciples dans la terre de Judée ; et là, il demeurait avec eux et il baptisait.
\VS{23}Jean aussi baptisait à Enon, près de Salim, parce qu'il y avait là beaucoup d'eau, et on y venait pour être baptisé.
\VS{24}Car Jean n'avait pas encore été mis en prison.
\VS{25}Or, il y eut une dispute entre les disciples de Jean et les Juifs touchant la purification.
\VS{26}Ils vinrent trouver Jean, et lui dirent : Maître, celui qui était avec toi au-delà du Jourdain, et à qui tu as rendu témoignage, voilà, il baptise, et tous vont à lui.
\VS{27}Jean répondit et dit : Un homme ne peut recevoir que ce qui lui a été donné du ciel.
\VS{28}Vous-mêmes m'êtes témoins que j'ai dit : Ce n'est pas moi qui suis le Christ, mais j’ai été envoyé devant lui.
\VS{29}Celui à qui appartient l'Epouse c’est l'Epoux ; mais l'ami de l'Epoux qui se tient là et qui l’entend, éprouve une grande joie à cause de la voix de l’Epoux ; c'est pourquoi cette joie qui est la mienne est parfaite.
\VS{30}Il faut qu'il croisse, et que je diminue.
\TextTitle{[Conclusion apportée par Jean]}
\VS{31}Celui qui vient d'en haut est au-dessus de tous ; celui qui est venu de la terre est de la terre, et il parle comme venant de la terre. Celui qui est venu du ciel est au-dessus de tous.
\VS{32}Et ce qu'il a vu et entendu, il le témoigne ; mais personne ne reçoit son témoignage.
\VS{33}Celui qui a reçu son témoignage a certifié que Dieu est véritable.
\VS{34}Car celui que Dieu a envoyé annonce les paroles de Dieu ; car Dieu ne lui donne point l'Esprit avec mesure.
\VS{35}Le Père aime le Fils, et il a remis toutes choses entre ses mains.
\VS{36}Celui qui croit au Fils a la vie éternelle, mais celui qui désobéit au Fils ne verra point la vie, mais la colère de Dieu demeure sur lui.
\TextTitle{[Jésus se rend en Galilée]}
\Chap{4}
\VerseOne{}Le Seigneur sut que les pharisiens avaient appris qu'il faisait et baptisait plus de disciples que Jean.
\VS{2}Toutefois Jésus ne baptisait point lui-même, mais c'étaient ses disciples.
\VS{3}Il quitta la Judée, et retourna encore en Galilée.
\TextTitle{[Jésus et la femme samaritaine]}
\VS{4}Comme il fallait qu'il passe par la Samarie,
\VS{5}il arriva dans une ville de Samarie nommée Sychar, près du champ que Jacob avait donné à Joseph son fils\FTNT{Ge. 48:22.}.
\VS{6}Or il y avait là le puits de Jacob ; et Jésus, fatigué du voyage, se tenait là, assis au bord du puits. C'était environ la sixième heure\FTNT{Sixième heure ou midi.}.
\VS{7}Une femme Samaritaine vint puiser de l'eau, Jésus lui dit : Donne-moi à boire.
\VS{8}Car ses disciples étaient allés à la ville pour acheter des vivres.
\VS{9}La femme Samaritaine lui dit : Comment toi qui es Juif, me demandes-tu à boire, à moi qui suis une femme Samaritaine ? Les Juifs, en effet, n’ont pas de relations avec les Samaritains.
\VS{10}Jésus lui répondit et dit : Si tu connaissais le don de Dieu, et qui est celui qui te dit : Donne-moi à boire, tu lui aurais toi-même demandé à boire, et il t’aurait donné de l'eau vive.
\VS{11}La femme lui dit : Seigneur, tu n'as rien pour puiser, et le puits est profond ; d'où aurais-tu donc cette eau vive ?
\VS{12}Es-tu plus grand que Jacob, notre père, qui nous a donné ce puits, et qui en a bu lui-même, ainsi que ses enfants et son bétail ?
\VS{13}Jésus répondit et lui dit : Quiconque boit de cette eau aura encore soif ;
\VS{14}mais celui qui boira de l'eau que je lui donnerai, n'aura jamais soif ; mais l'eau que je lui donnerai deviendra en lui une source d'eau qui jaillira jusque dans la vie éternelle.
\VS{15}La femme lui dit : Seigneur, donne-moi de cette eau, afin que je n'aie plus soif, et que je ne vienne plus ici puiser de l'eau.
\VS{16}Jésus lui dit : Va, appelle ton mari, et viens ici.
\VS{17}La femme lui répondit et dit : Je n'ai point de mari. Jésus lui dit : Tu as bien dit : Je n'ai point de mari.
\VS{18}Car tu as eu cinq maris, et celui que tu as maintenant n'est point ton mari ; en cela tu as dit la vérité.
\VS{19}La femme lui dit : Seigneur, je vois que tu es un prophète.
\VS{20}Nos pères ont adoré sur cette montagne\FTNT{Cette montagne, dont parle la Samaritaine, c'est le Mont Garizim (ou montagne de Sichem) sur lequel les samaritains construisirent leur temple et établirent leur culte, au temps de Néhémie.}, et vous, vous dites que le lieu où il faut adorer est à Jérusalem.
\VS{21}Jésus lui dit : Femme, crois-moi, l'heure vient où ce ne sera ni sur cette montagne ni à Jérusalem que vous adorerez le Père.
\VS{22}Vous adorez ce que vous ne connaissez pas ; nous, nous adorons ce que nous connaissons ; car le salut vient des Juifs.
\VS{23}Mais l'heure vient, et elle est déjà venue, où les vrais adorateurs adoreront le Père en esprit et en vérité ; car ce sont là les adorateurs que le Père demande.
\VS{24}Dieu est Esprit, et il faut que ceux qui l'adorent, l'adorent en esprit et en vérité.
\VS{25}La femme lui répondit : Je sais que le Messie, c'est-à-dire le Christ, doit venir ; quand il sera venu, il nous annoncera toutes choses.
\VS{26}Jésus lui dit : Je le suis, moi qui te parle.
\VS{27}Là-dessus arrivèrent ses disciples, et ils s'étonnèrent de ce qu'il parlait avec une femme. Toutefois aucun ne dit : Que demandes-tu ? Ou : Pourquoi parles-tu avec elle ?
\VS{28}La femme, ayant laissé sa cruche, s'en alla dans la ville, et elle dit aux habitants :
\VS{29}Venez voir un homme qui m'a dit tout ce que j'ai fait, ne serait-ce point le Christ ?
\VS{30}Ils sortirent donc de la ville, et vinrent vers lui.
\VS{31}Cependant les disciples le pressaient, disant : Maître, mange.
\VS{32}Mais il leur dit : J'ai à manger une nourriture que vous ne connaissez point.
\VS{33}Sur quoi les disciples se demandaient entre eux : Quelqu'un lui aurait-il apporté à manger ?
\VS{34}Jésus leur dit : Ma nourriture est de faire la volonté de celui qui m'a envoyé, et d’accomplir son œuvre.
\VS{35}Ne dites-vous pas qu'il y a encore quatre mois jusqu’à la moisson ? Voici, je vous dis, levez vos yeux, et regardez les champs qui déjà blanchissent pour la moisson.
\VS{36}Celui qui moissonne reçoit un salaire, et amasse des fruits pour la vie éternelle ; afin que celui qui sème et celui qui moissonne se réjouissent ensemble.
\VS{37}Car en ceci ce qu’on dit d’ordinaire est vrai : L’un sème et l'autre moissonne.
\VS{38}Je vous ai envoyés moissonner où vous n'avez point travaillé ; d'autres ont travaillé, et vous êtes entrés dans leur travail.
\TextTitle{[Jésus et les samaritains]}
\VS{39}Plusieurs Samaritains de cette ville crurent en lui, à cause de la parole de la femme qui avait rendu ce témoignage : Il m'a dit tout ce que j'ai fait.
\VS{40}Quand donc les Samaritains vinrent le trouver, ils le prièrent de demeurer avec eux ; et il demeura là deux jours.
\VS{41}Et beaucoup plus de gens crurent à cause de sa parole ;
\VS{42}et ils disaient à la femme : Ce n'est plus à cause de ta parole que nous croyons ; car nous l'avons entendu nous-mêmes, et nous savons qu’il est véritablement le Christ, le Sauveur du monde.
\VS{43}Après ces deux jours, Jésus partit de là, et s'en alla en Galilée.
\VS{44}Car il avait rendu témoignage qu'un prophète n'est pas honoré dans son pays.
\VS{45}Lorsqu’il arriva en Galilée, les Galiléens le reçurent, ayant vu toutes les choses qu'il avait faites à Jérusalem le jour de la Fête, car eux aussi étaient allés à la Fête.
\TextTitle{[Jésus guérit le fils d'un officier]}
\VS{46}Jésus retourna encore à Cana de Galilée, où il avait changé l'eau en vin. Or il y avait à Capernaüm un officier du roi, dont le fils était malade.
\VS{47}Ayant appris que Jésus était venu de Judée en Galilée, il alla vers lui, et le pria de descendre pour guérir son fils qui était près de mourir.
\VS{48}Mais Jésus lui dit : Si vous ne voyez pas des prodiges et des miracles, vous ne croyez point.
\VS{49}L’officier du roi lui dit : Seigneur, descends avant que mon fils meure.
\VS{50}Jésus lui dit : Va, ton fils vit. Cet homme crut à la parole que Jésus lui avait dite, et il s'en alla.
\VS{51}Et comme il descendait déjà, ses serviteurs vinrent au-devant de lui, et lui apportèrent des nouvelles, disant : Ton fils vit.
\VS{52}Et il leur demanda à quelle heure il s'était trouvé mieux ; et ils lui dirent : Hier, à la septième heure, la fièvre l’a quitté.
\VS{53}Le père reconnut que c'était à cette même heure-là que Jésus lui avait dit : Ton fils vit. Et il crut, avec toute sa maison.
\VS{54}Jésus fit encore ce second miracle quand il fut venu de Judée en Galilée.
\TextTitle{[Nouvelle fête des juifs et guérison d'un paralytique à la piscine de Béthesda]}
\Chap{5}
\VerseOne{}Après ces choses, il y eut une fête des Juifs, et Jésus monta à Jérusalem.
\VS{2}Or à Jérusalem, près de la porte des brebis, il y avait une piscine appelée en hébreu Béthesda, et qui avait cinq portiques.
\VS{3}Sous ces portiques étaient couchés un grand nombre de malades, des aveugles, des boiteux, des paralytiques, attendant le mouvement de l'eau.
\VS{4}Car un ange descendait de temps en temps dans la piscine, et agitait l'eau ; et alors le premier qui y descendait après que l'eau avait été agitée, était guéri, quelle que fût sa maladie.
\VS{5}Or il y avait là un homme malade depuis trente-huit ans.
\VS{6}Jésus, le voyant couché par terre, et sachant qu'il était déjà malade depuis longtemps, lui dit : Veux-tu être guéri ?
\VS{7}Le malade lui répondit : Seigneur, je n'ai personne pour me jeter dans la piscine quand l'eau est agitée, et pendant que j'y vais, un autre descend avant moi.
\VS{8}Jésus lui dit : Lève-toi, prends ton lit, et marche.
\VS{9}Et aussitôt cet homme fut guéri, il prit son lit, et marcha. Or c'était un jour de sabbat.
\VS{10}Les Juifs dirent donc à celui qui avait été guéri : C'est un jour de sabbat, il ne t'est pas permis de prendre ton lit.
\VS{11}Il leur répondit : Celui qui m'a guéri m'a dit : Prends ton lit et marche.
\VS{12}Alors ils lui demandèrent : Qui est celui qui t'a dit : Prends ton lit et marche ?
\VS{13}Mais celui qui avait été guéri ne savait pas qui c'était, car Jésus s'était éclipsé du milieu de la foule qui était en ce lieu-là.
\VS{14}Depuis, Jésus le trouva dans le temple, et lui dit : Voici, tu as été guéri ; ne pèche plus désormais, de peur qu’il ne t'arrive quelque chose de pire.
\VS{15}Cet homme s'en alla, et rapporta aux Juifs que c'était Jésus qui l'avait guéri.
\VS{16}C'est pourquoi les Juifs poursuivaient Jésus et cherchaient à le faire mourir, parce qu'il avait fait ces choses le jour du sabbat.
\TextTitle{[Jésus déclare son égalité avec le Père]}
\VS{17}Mais Jésus leur répondit : Mon Père agit jusqu'à présent ; moi aussi, j’agis.
\VS{18}A cause de cela, les Juifs cherchaient encore plus à le faire mourir, parce que non seulement il avait violé le sabbat, mais aussi parce qu'il disait que Dieu était son propre Père, se faisant égal à Dieu.
\VS{19}Mais Jésus répondit et leur dit : En vérité, en vérité je vous le dis, le Fils ne peut rien faire de lui-même, il ne fait que ce qu'il voit faire au Père ; et tout ce que le Père fait, le Fils le fait pareillement.
\VS{20}Car le Père aime le Fils, et lui montre toutes les choses qu'il fait ; et il lui montrera de plus grandes œuvres que celles-ci, afin que vous soyez dans l'admiration.
\VS{21}Car comme le Père ressuscite les morts et donne la vie, de même aussi le Fils donne la vie à ceux qu'il veut.
\VS{22}Car le Père ne juge personne ; mais il a donné tout jugement au Fils,
\VS{23}afin que tous honorent le Fils, comme ils honorent le Père ; celui qui n'honore point le Fils, n'honore point le Père qui l'a envoyé.
\VS{24}En vérité, en vérité je vous le dis, celui qui entend ma parole, et croit à celui qui m'a envoyé, a la vie éternelle et ne vient pas en jugement, mais il est passé de la mort à la vie.
\TextTitle{[Les deux résurrections]}
\VS{25}En vérité, en vérité je vous le dis, l'heure vient, et elle est déjà venue, où les morts entendront la voix du Fils de Dieu, et ceux qui l'auront entendue vivront.
\VS{26}Car comme le Père a la vie en lui-même, ainsi il a donné au Fils d'avoir la vie en lui-même.
\VS{27}Et il lui a donné le pouvoir de juger parce qu'il est le Fils de l'homme.
\VS{28}Ne soyez point étonnés de cela ; car l'heure vient où tous ceux qui sont dans les sépulcres entendront sa voix, et en sortiront.
\VS{29}Ceux qui auront fait le bien, ressusciteront pour la vie, mais ceux qui auront fait le mal, ressusciteront pour le jugement.
\TextTitle{[Témoignages confirmant celui de Jésus]}
\VS{30}Je ne puis rien faire de moi-même : Je juge conformément à ce que j'entends, et mon jugement est juste ; car je ne cherche point ma volonté, mais la volonté du Père qui m'a envoyé.
\VS{31}Si je rends témoignage de moi-même, mon témoignage n'est pas digne de foi.
\VS{32}C'est un autre qui rend témoignage de moi, et je sais que le témoignage qu'il rend de moi est digne de foi.
\TextTitle{[a. Le témoignage de Jean-Baptiste]}
\VS{33}Vous avez envoyé une délégation vers Jean, et il a rendu témoignage à la vérité.
\VS{34}Or je ne cherche point le témoignage des hommes ; mais je dis ces choses afin que vous soyez sauvés.
\VS{35}Jean était une lampe ardente et brillante ; et vous avez voulu vous réjouir pour un peu de temps à sa lumière.
\TextTitle{[b. Le témoignage des oeuvres de Jésus]}
\VS{36}Mais moi, j'ai un témoignage plus grand que celui de Jean ; car les œuvres que mon Père m'a donné d’accomplir, ces œuvres mêmes que je fais, témoignent de moi que c’est mon Père qui m'a envoyé.
\TextTitle{[c. Le témoignage du Père]
\\(Mt. 3:17)}
\VS{37}Et le Père qui m'a envoyé, a lui-même rendu témoignage de moi. Vous n’avez jamais entendu sa voix, vous n’avez jamais vu sa face.
\VS{38}Et sa parole ne demeure point en vous, puisque vous ne croyez pas à celui qu'il a envoyé.
\TextTitle{[d. Le témoignage de l'Ecriture]
\\(Lu. 24:27,44)}
\VS{39}Vous sondez les Ecritures, car vous pensez avoir en elles la vie éternelle, et ce sont elles qui rendent témoignage de moi.
\VS{40}Et vous ne voulez pas venir à moi, pour avoir la vie.
\VS{41}Je ne tire pas ma gloire des hommes.
\VS{42}Mais je sais que vous n'avez point l'amour de Dieu en vous.
\VS{43}Je suis venu au Nom de mon Père, et vous ne me recevez pas, si un autre vient en son propre nom, vous le recevrez.
\VS{44}Comment pouvez-vous croire, puisque vous recevez la gloire les uns des autres, et ne cherchez point la gloire qui vient de Dieu seul ?
\VS{45}Ne croyez point que je vous accuserai devant mon Père ; Moïse sur qui vous vous fondez, est celui qui vous accusera.
\VS{46}Car si vous croyiez Moïse, vous me croiriez aussi ; parce qu’il a écrit à mon sujet.
\VS{47}Mais si vous ne croyez pas à ses écrits, comment croirez-vous à mes paroles ?
\TextTitle{[Une autre Pâque et la multiplication des pains pour les cinq mille hommes]
\\(Mt. 14:15-21 ; Mc. 6:32-44 ; Lu. 9:12-17)}
\Chap{6}
\VerseOne{}Après ces choses, Jésus s'en alla au-delà de la mer de Galilée, qui est la mer de Tibériade.
\VS{2}Une grande foule le suivait, parce qu’elle voyait les miracles qu'il opérait sur les malades.
\VS{3}Jésus monta sur une montagne, et il s'assit là avec ses disciples.
\VS{4}Or, la Pâque, la fête des Juifs, était proche.
\VS{5}Et Jésus ayant levé ses yeux, et voyant qu’une grande foule venait à lui, dit à Philippe : Où achèterons-nous des pains, afin que ces gens aient à manger ?
\VS{6}Il disait cela pour l'éprouver, car il savait bien ce qu'il allait faire.
\VS{7}Philippe lui répondit : Les pains qu’on aurait pour deux cents deniers ne suffiraient pas pour que chacun en reçoive un peu.
\VS{8}Un de ses disciples, André, frère de Simon Pierre, lui dit :
\VS{9}Il y a ici un petit garçon qui a cinq pains d'orge et deux poissons ; mais qu'est-ce que cela pour tant de gens ?
\VS{10}Alors Jésus dit : Faites asseoir les gens. Il y avait beaucoup d'herbe dans ce lieu. Ils s'assirent au nombre d'environ cinq mille.
\VS{11}Et Jésus prit les pains ; et après avoir rendu grâces, il les distribua aux disciples, et les disciples à ceux qui étaient assis, et de même des poissons, autant qu'ils en voulaient.
\VS{12}Et après qu'ils furent rassasiés, il dit à ses disciples : Ramassez les morceaux qui restent, afin que rien ne soit perdu.
\VS{13}Ils les ramassèrent donc, et ils remplirent douze paniers avec les morceaux qui restèrent des cinq pains d'orge, après que tous eurent mangé.
\VS{14}Ces gens, ayant vu le miracle que Jésus avait fait, disaient : Celui-ci est véritablement le Prophète qui devait venir dans le monde.
\TextTitle{[Jésus marche sur les eaux]
\\(Mt. 14:22-33 ; Mc. 6:45-52)}
\VS{15}Mais Jésus, sachant qu'ils allaient venir l'enlever pour le faire Roi, se retira encore, lui seul, sur la montagne.
\VS{16}Et quand le soir fut venu, ses disciples descendirent à la mer.
\VS{17}Etant montés dans la barque, ils traversaient la mer pour se rendre à Capernaüm. Il faisait déjà nuit, et Jésus ne les avait pas encore rejoints.
\VS{18}Il soufflait un grand vent, et la mer était agitée.
\VS{19}Après avoir ramé environ vingt-cinq ou trente stades, ils virent Jésus marchant sur la mer, et s'approchant de la barque. Et ils eurent peur.
\VS{20}Mais il leur dit : C’est moi, ne craignez point.
\VS{21}Ils le reçurent donc avec plaisir dans la barque, et aussitôt la barque aborda au lieu où ils allaient.
\TextTitle{[Jésus, le pain de vie]}
\VS{22}Le lendemain, la foule qui était restée de l'autre côté de la mer, vit qu’il ne se trouvait là qu’une seule barque, et que Jésus n’était pas monté avec ses disciples dans la barque, mais qu’ils étaient partis seuls.
\VS{23}Cependant, d’autres barques étaient arrivées de Tibériade près du lieu où ils avaient mangé le pain, après que le Seigneur eut rendu grâces.
\VS{24}Quand la foule vit que ni Jésus ni ses disciples n’étaient là, les gens montèrent eux-mêmes dans ces barques, et allèrent à Capernaüm chercher Jésus.
\VS{25}Et l'ayant trouvé au-delà de la mer, ils lui dirent : Rabbi, quand es-tu arrivé ici ?
\VS{26}Jésus leur répondit et leur dit : En vérité, en vérité je vous le dis : Vous me cherchez, non parce que vous avez vu des miracles, mais parce que vous avez mangé des pains et que vous avez été rassasiés.
\VS{27}Travaillez, non pour la nourriture qui périt, mais pour celle qui est permanente jusqu’à la vie éternelle, et que le Fils de l'homme vous donnera ; car c’est lui que le Père, que Dieu, a marqué de son sceau.
\VS{28}Ils lui dirent donc : Que devons-nous faire pour accomplir les œuvres de Dieu ?
\VS{29}Jésus répondit et leur dit : C’est ici l’œuvre de Dieu, que vous croyiez en celui qu'il a envoyé.
\TextTitle{[Jésus envoyé du ciel]}
\VS{30}Alors ils lui dirent : Quel miracle fais-tu donc, afin que nous le voyions, et que nous croyions en toi ? Quelle œuvre fais-tu ?
\VS{31}Nos pères ont mangé la manne dans le désert ; selon ce qui est écrit : Il leur a donné à manger le pain du ciel\FTNT{} (1).
\VS{32}Mais Jésus leur dit : En vérité, en vérité je vous le dis : Moïse ne vous a pas donné le pain du ciel ; mais mon Père vous donne le vrai pain du ciel.
\VS{33}Car le pain de Dieu c'est celui qui est descendu du ciel et qui donne la vie au monde.
\VS{34}Ils lui dirent donc : Seigneur, donne-nous toujours ce pain-là.
\VS{35}Et Jésus leur dit : Je suis le pain de vie. Celui qui vient à moi, n'aura jamais faim ; et celui qui croit en moi, n'aura jamais soif.
\VS{36}Mais, je vous ai dit que vous m'avez vu, et cependant vous ne croyez point.
\VS{37}Tous ceux que mon Père me donne viendront à moi ; et je ne mettrai point dehors celui qui viendra à moi.
\VS{38}Car je suis descendu du ciel, non point pour faire ma volonté, mais la volonté de celui qui m'a envoyé.
\VS{39}Or, la volonté du Père qui m'a envoyé, c’est que je ne perde aucun de tous ceux qu'il m'a donnés, mais que je les ressuscite au dernier jour.
\VS{40}La volonté de celui qui m'a envoyé, c’est que quiconque contemple le Fils, et croit en lui, ait la vie éternelle ; et je le ressusciterai au dernier jour.
\VS{41}Les Juifs murmuraient contre lui de ce qu'il avait dit : Je suis le pain qui est descendu du ciel.
\VS{42}Et ils disaient : N’est-ce pas là Jésus, le fils de Joseph, celui dont nous connaissons le père et la mère ? Comment donc dit-il : Je suis descendu du ciel ?
\VS{43}Jésus leur répondit et leur dit : Ne murmurez pas entre vous.
\VS{44}Nul ne peut venir à moi, si le Père qui m'a envoyé ne l’attire ; et je le ressusciterai au dernier jour.
\VS{45}Il est écrit dans les prophètes : Ils seront tous enseignés de Dieu. Ainsi, quiconque a entendu le Père et a été instruit de ses intentions, vient à moi.
\VS{46}C’est que nul n’a vu le Père, sinon celui qui vient de Dieu, celui-là a vu le Père.
\VS{47}En vérité, en vérité je vous le dis : Celui qui croit en moi a la vie éternelle.
\VS{48}Je suis le pain de vie.
\VS{49}Vos pères ont mangé la manne dans le désert, et ils sont morts.
\VS{50}C'est ici le pain qui est descendu du ciel, afin que celui qui en mange, ne meure point.
\VS{51}Je suis le pain vivant qui est descendu du ciel. Si quelqu'un mange de ce pain, il vivra éternellement ; et le pain que je donnerai, c'est ma chair, que je donnerai pour la vie du monde.
\VS{52}Les Juifs donc discutaient entre eux, et disaient : Comment peut-il nous donner sa chair à manger ?
\VS{53}Et Jésus leur dit : En vérité, en vérité je vous le dis : Si vous ne mangez pas la chair du Fils de l'homme, et ne buvez pas son sang, vous n'aurez point la vie en vous-mêmes.
\VS{54}Celui qui mange ma chair, et qui boit mon sang, a la vie éternelle ; et je le ressusciterai au dernier jour.
\VS{55}Car ma chair est une véritable nourriture, et mon sang est un véritable breuvage.
\VS{56}Celui qui mange ma chair, et qui boit mon sang, demeure en moi, et moi en lui.
\VS{57}Comme le Père qui est vivant m'a envoyé, et que je suis vivant par le Père ; ainsi celui qui me mangera, vivra aussi par moi.
\VS{58}C'est ici le pain qui est descendu du ciel. Il n’en est pas comme de vos pères qui ont mangé la manne, et qui sont morts ; celui qui mangera ce pain, vivra éternellement.
\VS{59}Il dit ces choses dans la synagogue, enseignant à Capernaüm.
\TextTitle{[Epreuve de la consécration des disciples]
\\(Mt. 8:19-22 ; 10:36 ; Lu. 9:23-26)}
\VS{60}Plusieurs de ses disciples l'ayant entendu, dirent : Cette parole est dure, qui peut l’écouter ?
\VS{61}Mais Jésus sachant en lui-même que ses disciples murmuraient à ce sujet, leur dit : Cela vous scandalise-t-il ?
\VS{62}Que sera-ce donc si vous voyez le Fils de l'homme monter où il était auparavant ?
\VS{63}C'est l'Esprit qui vivifie ; la chair ne sert à rien. Les paroles que je vous ai dites, sont Esprit et vie.
\VS{64}Mais il en est parmi vous qui ne croient point. En effet, Jésus savait dès le commencement qui étaient ceux qui ne croiraient point, et qui était celui qui le trahirait.
\VS{65}Il leur dit donc : C’est pour cela que je vous ai dit, que nul ne peut venir à moi, si cela ne lui a pas été donné par mon Père.
\TextTitle{[Pierre reconnaît Jésus comme le Christ]
\\(Mt. 16:13-16 ; Mc. 8:27-30 ; Lu. 9:18-21}
\VS{66}Dès ce moment, plusieurs de ses disciples l'abandonnèrent, et ils ne marchèrent plus avec lui.
\VS{67}Et Jésus dit aux douze : Et vous, ne voulez-vous pas aussi vous en aller ?
\VS{68}Mais Simon Pierre lui répondit : Seigneur ! Auprès de qui irions-nous ? Tu as les paroles de la vie éternelle.
\VS{69}Et nous avons cru, et nous avons connu que tu es le Christ, le Fils du Dieu vivant.
\VS{70}Jésus leur répondit : Ne vous ai-je pas choisis, vous les douze ? Et toutefois l'un de vous est un démon.
\VS{71}Il parlait de Judas Iscariot, fils de Simon ; car c'était lui qui devait le trahir, quoiqu'il fût l'un des douze.
\TextTitle{[Jésus engagé par ses frères incrédules à se rendre à Jérusalem]}
\Chap{7}
\VerseOne{}Après ces choses, Jésus parcourait la Galilée, car il ne voulait pas parcourir la Judée, parce que les Juifs cherchaient à le faire mourir.
\VS{2}Or la fête des Juifs, appelée la fête des tabernacles, était proche.
\VS{3}Et ses frères lui dirent : Pars d'ici, et va en Judée, afin que tes disciples aussi contemplent les œuvres que tu fais.
\VS{4}Personne n’agit en secret, lorsqu'il cherche à être connu ; si tu fais ces choses, montre-toi toi-même au monde.
\VS{5}Car ses frères non plus ne croyaient pas en lui.
\VS{6}Et Jésus leur dit : Mon temps n'est pas encore venu, mais votre temps est toujours prêt.
\VS{7}Le monde ne peut pas vous haïr, mais il me hait parce que je rends témoignage contre lui que ses œuvres sont mauvaises.
\VS{8}Montez, vous, à cette fête ; pour moi, je n’y monte pas encore, parce que mon temps n'est pas encore accompli.
\VS{9}Après leur avoir dit ces choses, il resta en Galilée.
\TextTitle{[Jésus à la fête des tabernacles]}
\VS{10}Lorsque ses frères furent montés, alors il y monta aussi lui-même, non publiquement, mais comme en secret.
\VS{11}Les Juifs le cherchaient pendant la fête, et ils disaient : Où est-il ?
\VS{12}Et il y avait un grand murmure à son sujet parmi la foule. Les uns disaient : C’est un homme de bien ; et les autres disaient : Non, il séduit le peuple.
\VS{13}Toutefois personne ne parlait franchement de lui, à cause de la crainte qu'on avait des Juifs.
\VS{14}Vers le milieu de la fête, Jésus monta au temple. Et il enseignait.
\VS{15}Les Juifs s’étonnaient, disant : Comment connaît-il les Ecritures, lui qui n’a point étudié ?
\VS{16}Jésus leur répondit et dit : Ma doctrine n'est pas de moi, mais de celui qui m'a envoyé.
\VS{17}Si quelqu'un veut faire sa volonté, il connaîtra si ma doctrine est de Dieu, ou si je parle de moi-même.
\VS{18}Celui qui parle de son propre chef cherche sa propre gloire ; mais celui qui cherche la gloire de celui qui l'a envoyé, est véritable, et il n'y a point d'injustice en lui.
\VS{19}Moïse ne vous a-t-il pas donné la loi ? Cependant, nul de vous n'observe la loi. Pourquoi cherchez-vous à me faire mourir ?
\VS{20}La foule répondit : Tu as un démon ; qui est-ce qui cherche à te faire mourir ?
\VS{21}Jésus répondit et leur dit : J’ai fait une œuvre, et vous en êtes tous étonnés.
\VS{22}Moïse vous a donné la circoncision, non qu’elle vienne de Moïse, mais des pères, vous circoncisez bien un homme le jour du sabbat.
\VS{23}Si un homme reçoit la circoncision le jour du sabbat, afin que la loi de Moïse ne soit pas violée, pourquoi êtes-vous irrités contre moi de ce que j'ai guéri un homme tout entier le jour du sabbat ?
\VS{24}Ne jugez pas selon les apparences, mais jugez selon la justice.
\VS{25}Alors quelques-uns de ceux de Jérusalem disaient : N'est-ce pas celui qu'ils cherchent à faire mourir ?
\VS{26}Et cependant voici, il parle librement, et ils ne lui disent rien ! Est-ce que vraiment les chefs auraient reconnu qu’il est véritablement le Christ ?
\VS{27}Cependant celui-ci, nous savons d'où il est ; mais quand le Christ viendra, personne ne saura d'où il est.
\VS{28}Jésus, enseignant dans le temple, s’écria : Vous me connaissez, et vous savez d'où je suis ! Je ne suis pas venu de moi-même, mais celui qui m'a envoyé est véritable, et vous ne le connaissez pas.
\VS{29}Mais moi, je le connais ; car je viens de lui, et c'est lui qui m'a envoyé.
\VS{30}Ils cherchaient donc à se saisir de lui, mais personne ne mit la main sur lui, parce que son heure n'était pas encore venue.
\VS{31}Cependant, plusieurs parmi la foule crurent en lui, et ils disaient : Quand le Christ sera venu, fera-t-il plus de miracles que celui-ci n'a fait ?
\VS{32}Les pharisiens entendirent la foule murmurant ces choses de lui. Alors les principaux sacrificateurs et les pharisiens envoyèrent des huissiers pour le prendre.
\VS{33}Et Jésus leur dit : Je suis encore pour un peu de temps avec vous, puis je m'en vais vers celui qui m'a envoyé.
\VS{34}Vous me chercherez, mais vous ne me trouverez pas, et vous ne pouvez pas venir où je serai.
\VS{35}Les Juifs dirent donc entre eux : Où ira-t-il, pour que nous ne le trouvions pas ? Ira-t-il parmi ceux qui sont dispersés chez les Grecs, et enseignera-t-il les Grecs ?
\VS{36}Quel est ce discours qu'il a tenu : Vous me chercherez, mais vous ne me trouverez pas, vous ne pouvez pas venir où je serai ?
\TextTitle{[La grande prophétie sur le secret de la puissance du Saint-Esprit]
\\(Ac. 2:2-4 ; Jn. 4:14)}
\VS{37}Le dernier jour, le grand jour de la fête, Jésus, se tenant debout, s’écria : Si quelqu'un a soif, qu'il vienne à moi, et qu'il boive.
\VS{38}Celui qui croit en moi, des fleuves d'eau vive couleront de son sein, comme dit l'Ecriture.
\VS{39}Il dit cela de l'Esprit que devaient recevoir ceux qui croiraient en lui ; car le Saint-Esprit n'était pas encore donné, parce que Jésus n'était pas encore glorifié.
\TextTitle{[Diversité d'opinions au sujet de Jésus]}
\VS{40}Plusieurs de la foule ayant entendu ce discours, disaient : Celui-ci est véritablement le Prophète.
\VS{41}Les autres disaient : Celui-ci est le Christ. Et les autres disaient : Est-ce bien de la Galilée que doit venir le Christ ?
\VS{42}L'Ecriture ne dit-elle pas que le Christ doit venir de la postérité de David, et du village de Bethléem, où était David ?
\VS{43}Il y eut donc division parmi la foule à cause de lui.
\VS{44}Et quelques-uns d'entre eux voulaient le saisir, mais personne ne mit la main sur lui.
\VS{45}Ainsi les huissiers retournèrent vers les principaux sacrificateurs et les pharisiens, qui leur dirent : Pourquoi ne l'avez-vous pas amené ?
\VS{46}Les huissiers répondirent : Jamais homme n’a parlé comme cet homme.
\VS{47}Mais les pharisiens leur répondirent : Est-ce que vous aussi, vous avez été séduits ?
\VS{48}Y a-t-il quelqu’un des chefs ou des pharisiens qui ait cru en lui ?
\VS{49}Mais cette foule, qui ne connaît pas la loi, ce sont des maudits.
\VS{50}Nicodème, qui était venu vers Jésus de nuit, et qui était l'un d'entre eux, leur dit :
\VS{51}Notre loi condamne-t-elle un homme avant qu’on l’entende et qu’on ne sache ce qu’il a fait ?
\VS{52}Ils lui répondirent : Es-tu aussi Galiléen ? Examine, et tu verras qu'aucun prophète n’est sorti de la Galilée.
\VS{53}Et chacun s'en alla dans sa maison.
\TextTitle{[Les scribes et les pharisiens accusent une femme surprise en flagrant délit d'adultère]}
\Chap{8}
\VerseOne{}Jésus se rendit à la montagne des oliviers.
\VS{2}Et, dès le matin, il alla de nouveau dans le temple, et tout le peuple vint à lui ; et s'étant assis, il les enseignait.
\VS{3}Alors les scribes et les pharisiens lui amenèrent une femme surprise en adultère ;
\VS{4}et l'ayant placée au milieu du peuple, ils dirent à Jésus : Maître, cette femme a été surprise en flagrant délit d’adultère.
\VS{5}Moïse nous a ordonné dans la loi de lapider celles qui sont dans son cas ; toi donc qu'en dis-tu ?
\VS{6}Or ils disaient cela pour l'éprouver, afin de pouvoir l'accuser. Mais Jésus s'étant penché en bas, écrivait avec son doigt sur la terre.
\VS{7}Et comme ils continuaient à l'interroger, s'étant relevé, il leur dit : Que celui de vous qui est sans péché, jette le premier la pierre contre elle.
\VS{8}Et s'étant encore baissé, il écrivait sur la terre.
\VS{9}Quand ils entendirent cela, accusés par leur conscience, ils se retirèrent un à un, depuis les plus âgés jusqu’aux derniers ; et Jésus resta seul avec la femme qui était là au milieu.
\VS{10}Alors Jésus s'étant relevé, et ne voyant plus que la femme, il lui dit : Femme, où sont ceux qui t'accusaient ? Personne ne t’a-t-il condamnée ?
\VS{11}Elle dit : Non, Seigneur. Et Jésus lui dit : Je ne te condamne pas non plus ; va, et ne pèche plus.
\TextTitle{[Point crucial du conflit entre Jésus et les pharisiens : l'origine de Christ, Lumière du monde]
\\(Jn. 1:9)}
\VS{12}Et Jésus leur parla encore, en disant : Je suis la Lumière du monde ; celui qui me suit ne marchera pas dans les ténèbres, mais il aura la lumière de la vie.
\VS{13}Alors les pharisiens lui dirent : Tu rends témoignage de toi-même, ton témoignage n'est pas digne de foi.
\VS{14}Jésus répondit et leur dit : Quoique je rende témoignage de moi-même, mon témoignage est digne de foi ; car je sais d'où je suis venu et où je vais ; mais vous ne savez pas d'où je viens ni où je vais.
\VS{15}Vous jugez selon la chair, mais moi, je ne juge personne.
\VS{16}Et si je juge, mon jugement est digne de foi ; car je ne suis pas seul, mais le Père qui m'a envoyé est avec moi.
\VS{17}Il est même écrit dans votre loi que le témoignage de deux hommes est digne de foi\FTNT{De. 19:15.}.
\VS{18}Je rends témoignage de moi-même, et le Père qui m'a envoyé rend aussi témoignage de moi.
\VS{19}Alors ils lui dirent : Où est ton Père ? Jésus répondit : Vous ne connaissez ni moi ni mon Père. Si vous me connaissiez, vous connaîtriez aussi mon Père.
\VS{20}Jésus dit ces paroles au lieu où était le trésor, enseignant dans le temple ; mais personne ne le saisit, parce que son heure n'était pas encore venue.
\VS{21}Et Jésus leur dit encore : Je m'en vais, et vous me chercherez, et vous mourrez dans vos péchés ; vous ne pouvez pas venir où je vais.
\VS{22}Les Juifs disaient donc : Se tuera-t-il lui-même, puisqu’il dit : Vous ne pouvez pas venir où je vais ?
\VS{23}Alors il leur dit : Vous êtes d'en bas, mais moi, je suis d'en haut ; vous êtes de ce monde, mais moi, je ne suis pas de ce monde.
\VS{24}C'est pourquoi je vous ai dit que vous mourrez dans vos péchés ; car si vous ne croyez pas que je suis l'envoyé de Dieu, vous mourrez dans vos péchés.
\VS{25}Alors ils lui dirent : Toi, qui es-tu ? Et Jésus leur dit : Ce que je vous dis dès le commencement.
\VS{26}J'ai beaucoup de choses à dire de vous et à juger en vous, mais celui qui m'a envoyé est véritable, et les choses que j'ai entendues de lui, je les dis au monde.
\VS{27}Ils ne comprirent point qu'il leur parlait du Père.
\VS{28}Jésus leur dit donc : Quand vous aurez élevé le Fils de l'homme, vous connaîtrez alors que je suis l'envoyé de Dieu, et que je ne fais rien de moi-même, mais que je dis ces choses selon ce que mon Père m'a enseigné.
\VS{29}Celui qui m'a envoyé est avec moi ; le Père ne m'a pas laissé seul, parce que je fais toujours les choses qui lui plaisent.
\VS{30}Comme il disait ces choses, plusieurs crurent en lui.
\VS{31}Et Jésus disait aux Juifs qui avaient cru en lui : Si vous demeurez dans ma parole, vous serez vraiment mes disciples.
\VS{32}Vous connaîtrez la vérité, et la vérité vous rendra libres.
\VS{33}Ils lui répondirent : Nous sommes la postérité d'Abraham, et nous ne fûmes jamais esclaves de personne ; comment donc dis-tu : Vous deviendrez libres ?
\VS{34}Jésus leur répondit : En vérité, en vérité je vous le dis : Quiconque se livre au péché, est esclave du péché.
\VS{35}Or l'esclave ne demeure pas toujours dans la maison ; le fils y demeure toujours.
\VS{36}Si donc le Fils vous affranchit, vous serez véritablement libres.
\VS{37}Je sais que vous êtes la postérité d'Abraham, pourtant vous cherchez à me faire mourir, parce que ma parole n'est pas reçue dans vos cœurs.
\VS{38}Je vous dis ce que j'ai vu chez mon Père ; et vous aussi vous faites les choses que vous avez vues chez votre père.
\VS{39}Ils répondirent et lui dirent : Notre père c'est Abraham. Jésus leur dit : Si vous étiez enfants d'Abraham, vous feriez les œuvres d'Abraham.
\VS{40}Mais maintenant vous cherchez à me faire mourir, moi, un homme qui vous ai dit la vérité que j'ai entendue de Dieu. Cela, Abraham ne l’a point fait.
\VS{41}Vous faites les œuvres de votre père. Et ils lui dirent : Nous ne sommes pas des enfants illégitimes ; nous avons un seul père, Dieu.
\VS{42}Mais Jésus leur dit : Si Dieu était votre Père, certes vous m'aimeriez, car c’est de Dieu que je suis sorti et que je viens ; je ne suis pas venu de moi-même, mais c'est lui qui m'a envoyé.
\VS{43}Pourquoi ne comprenez-vous pas mon langage ? C’est parce que vous ne pouvez pas écouter ma parole.
\VS{44}Vous avez pour père le diable, et vous voulez accomplir les désirs de votre père. Il a été meurtrier dès le commencement, et il n'a pas persévéré dans la vérité, car la vérité n'est pas en lui. Toutes les fois qu'il profère le mensonge, il parle de son propre fond ; car il est menteur et le père du mensonge.
\VS{45}Et moi, parce que je dis la vérité, vous ne me croyez pas.
\VS{46}Qui de vous me convaincra de péché ? Si je dis la vérité, pourquoi ne me croyez-vous pas ?
\VS{47}Celui qui est de Dieu écoute les paroles de Dieu ; vous n’écoutez pas, parce que vous n'êtes pas de Dieu.
\VS{48}Alors les Juifs répondirent : N’avons-nous pas raison de dire que tu es un Samaritain, et que tu as un démon ?
\VS{49}Jésus répondit : Je n'ai point un démon, mais j'honore mon Père, et vous m’outragez.
\VS{50}Je ne cherche point ma gloire ; il y en a un qui la cherche, et qui juge.
\VS{51}En vérité, en vérité je vous le dis : Si quelqu'un garde ma parole, il ne verra jamais la mort.
\VS{52}Les Juifs lui dirent donc : Maintenant nous savons que tu as un démon. Abraham est mort, et les prophètes aussi, et tu dis : Si quelqu'un garde ma parole, il ne verra jamais la mort.
\VS{53}Es-tu plus grand que notre père Abraham qui est mort ? Les prophètes aussi sont morts. Qui prétends-tu être ?
\VS{54}Jésus répondit : Si je me glorifie moi-même, ma gloire n'est rien ; mon Père est celui qui me glorifie, celui que vous dites être votre Dieu.
\VS{55}Toutefois vous ne l'avez point connu, mais moi je le connais ; et si je disais que je ne le connais point, je serais un menteur, semblable à vous ; mais je le connais, et je garde sa parole.
\VS{56}Abraham votre père a tressailli de joie de ce qu’il verrait mon jour ; et il l'a vu, et il s’est réjoui.
\VS{57}Les Juifs lui dirent : Tu n'as pas encore cinquante ans, et tu as vu Abraham !
\VS{58}Jésus leur dit : En vérité, en vérité je vous le dis : Avant qu'Abraham fût, Je suis\FTNT{Je suis : L'évangile de Jean rapporte plusieurs déclarations incroyables que Jésus a faites à son sujet : Je suis le pain de vie (6:35), Je suis la Lumière du monde (8:12), Je suis le bon berger (10:11), Je suis la porte (10:7), Je suis la résurrection (11:25), Je suis le chemin, la vérité et la vie (14:6), Je suis la vraie vigne (15:1). Toutefois, dans ce verset, en déclarant être ~Je suis~, il s’identifie clairement au Nom que YHWH avait révélé à Moïse dans Ex. 3:14. C'est précisément pour cette raison que les juifs ont voulu le lapider.}.
\VS{59}Alors ils prirent des pierres pour les jeter contre lui, mais Jésus se cacha et sortit du temple, passant au milieu d'eux ; et ainsi il s'en alla.
\TextTitle{[Jésus guérit un aveugle-né]}
\Chap{9}
\VerseOne{}Comme Jésus passait, il vit un homme aveugle de naissance.
\VS{2}Ses disciples lui posèrent cette question : Rabbi, qui a péché ? Cet homme ou ses parents pour qu’il soit né aveugle ?
\VS{3}Jésus répondit : Ce n’est pas que lui ou ses parents aient péché ; mais c'est afin que les œuvres de Dieu soient manifestées en lui.
\VS{4}Il faut que je fasse, tandis qu’il est jour, les œuvres de celui qui m'a envoyé. La nuit vient, où personne ne peut travailler.
\VS{5}Pendant que je suis dans le monde, je suis la Lumière du monde.
\VS{6}Ayant dit ces paroles, il cracha à terre et fit de la boue avec sa salive, et mit de cette boue sur les yeux de l'aveugle.
\VS{7}Et il lui dit : Va, et lave-toi au réservoir de Siloé (nom qui veut dire envoyé). Il y alla donc, se lava, et s’en retourna voyant clair.
\VS{8}Ses voisins et ceux qui auparavant l’avaient connu comme mendiant disaient : N'est-ce pas celui qui était assis et qui mendiait ?
\VS{9}Les uns disaient : C’est lui. Et les autres disaient : Il lui ressemble. Mais lui-même disait : C'est moi.
\VS{10}Ils lui dirent donc : Comment tes yeux ont-ils été ouverts ?
\VS{11}Il répondit et dit : Cet homme, qu'on appelle Jésus, a fait de la boue et il l'a mise sur mes yeux, et m'a dit : Va au réservoir de Siloé et lave-toi. J’y suis allé, je me suis lavé, et j’ai recouvert la vue.
\VS{12}Alors ils lui dirent : Où est cet homme ? Il répondit : Je ne sais pas.
\VS{13}Ils amenèrent vers les pharisiens celui qui auparavant avait été aveugle.
\VS{14}Or c'était en un jour de sabbat que Jésus avait fait de la boue et lui avait ouvert les yeux.
\VS{15}C'est pourquoi les pharisiens l'interrogèrent encore, comment il avait pu voir ; et il leur dit : Il a mis de la boue sur mes yeux, et je me suis lavé, et je vois.
\VS{16}Sur quoi quelques-uns des pharisiens dirent : Cet homme n'est pas un envoyé de Dieu ; car il n’observe pas le sabbat ; mais d'autres disaient : Comment un homme pécheur peut-il faire de tels prodiges ? Et il y avait de la division entre eux.
\VS{17}Ils dirent encore à l'aveugle : Toi, que dis-tu de lui, sur ce qu'il t'a ouvert les yeux ? Il répondit : C’est un Prophète.
\VS{18}Mais les Juifs ne crurent point que cet homme avait été aveugle, et qu'il avait pu voir, jusqu'à ce qu'ils aient fait venir ses parents.
\VS{19}Et ils les interrogèrent, disant : Est-ce là votre fils, que vous dites être né aveugle ? Comment donc voit-il maintenant ?
\VS{20}Ses parents leur répondirent : Nous savons que c'est notre fils et qu'il est né aveugle.
\VS{21}Mais comment il voit maintenant, ou qui lui a ouvert les yeux, nous ne le savons pas ; il a de l'âge, interrogez-le, il parlera de ce qui le regarde.
\VS{22}Ses parents dirent ces choses parce qu'ils craignaient les Juifs ; car les Juifs avaient déjà convenu que si quelqu'un reconnaissait Jésus pour le Christ, il serait exclu de la synagogue.
\VS{23}C’est pourquoi ses parents dirent : Il a de l'âge, interrogez-le lui-même.
\VS{24}Ils appelèrent donc pour la seconde fois l'homme qui avait été aveugle et ils lui dirent : Donne gloire à Dieu ; nous savons que cet homme est un pécheur.
\VS{25}Il répondit : Je ne sais pas si c’est un pécheur ; je sais une chose, c’est que j’étais aveugle et que maintenant je vois.
\VS{26}Ils lui dirent donc encore : Que t'a-t-il fait ? Comment a-t-il ouvert tes yeux ?
\VS{27}Il leur répondit : Je vous l'ai déjà dit, et vous ne l'avez point écouté, pourquoi voulez-vous l’entendre encore ? Voulez-vous aussi devenir ses disciples ?
\VS{28}Alors ils l'injurièrent et lui dirent : C’est toi son disciple ; nous, nous sommes disciples de Moïse.
\VS{29}Nous savons que Dieu a parlé à Moïse ; mais celui-ci, nous ne savons pas d'où il est.
\VS{30}Cet homme répondit : Certes, c'est une chose étrange que vous ne sachiez point d'où il est ; et toutefois il a ouvert mes yeux.
\VS{31}Nous savons que Dieu n'exauce point les méchants, mais si quelqu'un est pieux envers Dieu, et fait sa volonté, il l'exauce.
\VS{32}Jamais on n’a entendu dire que quelqu’un ait ouvert les yeux d’un aveugle-né.
\VS{33}Si cet homme n'était pas un envoyé de Dieu, il ne pourrait rien faire de semblable.
\VS{34}Ils répondirent : Tu es entièrement né dans le péché, et tu nous enseignes ! Et ils le chassèrent dehors.
\TextTitle{[Jésus affirme sa divinité]}
\VS{35}Jésus apprit qu'ils l'avaient chassé dehors ; et l'ayant rencontré, il lui dit : Crois-tu au Fils de Dieu ?
\VS{36}Cet homme lui répondit : Qui est-il Seigneur, afin que je croie en lui ?
\VS{37}Jésus lui dit : Tu l'as vu, et c'est celui qui te parle.
\VS{38}Alors il dit : Je crois, Seigneur ; et il l'adora\FTNT{Au travers de la lecture de la Bible, on constate que les anges refusent l’adoration (Ap. 19:9-10) de même que les apôtres (Ac. 10:25-26 ; Ac. 14:5-18). Seul Dieu accepte l’adoration puisqu’il en est le seul digne. Jésus n’a jamais refusé l’adoration des hommes, car il est Dieu.}.
\VS{39}Et Jésus dit : Je suis venu dans ce monde pour exercer le jugement, afin que ceux qui ne voient point voient ; et que ceux qui voient deviennent aveugles.
\VS{40}Quelques pharisiens qui étaient avec lui, ayant entendu ces paroles, dirent : Et nous, sommes-nous aussi aveugles ?
\VS{41}Jésus leur répondit : Si vous étiez aveugles, vous n'auriez point de péché ; mais maintenant vous dites : Nous voyons. C’est à cause de cela que votre péché demeure.
\TextTitle{[Jésus, le Bon Berger]
\\(Ps. 23 ; Hé. 13:20 ; 1Pi. 5:4)}
\Chap{10}
\VerseOne{}En vérité, en vérité je vous le dis : Celui qui n'entre point par la porte dans la bergerie des brebis, mais y monte par ailleurs, est un voleur et un brigand.
\VS{2}Mais celui qui entre par la porte est le berger des brebis.
\VS{3}Le portier lui ouvre, et les brebis entendent sa voix, et il appelle les brebis qui lui appartiennent par leur nom, et il les conduit dehors.
\VS{4}Lorsqu’il a fait sortir toutes ses brebis dehors, il marche devant elles, et les brebis le suivent, parce qu'elles connaissent sa voix.
\VS{5}Mais elles ne suivront point un étranger, au contraire, elles fuiront loin de lui ; parce qu'elles ne connaissent point la voix des étrangers.
\VS{6}Jésus leur dit cette parabole, mais ils ne comprirent pas ce qu'il leur disait.
\VS{7}Jésus leur dit encore : En vérité, en vérité je vous le dis : Je suis la Porte par où entrent les brebis\FTNT{La porte des brebis était située près du temple et avait été bâtie du temps de Néhémie (Né. 3:1). Les animaux que l’on sacrifiait à Dieu franchissaient probablement cette porte.}.
\VS{8}Tout ceux qui sont venus avant moi sont des brigands et des voleurs ; mais les brebis ne les ont point écoutés.
\VS{9}Je suis la Porte : Si quelqu'un entre par moi, il sera sauvé ; il entrera et il sortira, et il trouvera des pâturages.
\VS{10}Le voleur ne vient que pour dérober, tuer et détruire ; moi, je suis venu afin que mes brebis aient la vie, et qu'elles l'aient même en abondance.
\VS{11}Je suis le bon berger : Le bon berger donne sa vie pour ses brebis.
\VS{12}Mais le mercenaire, qui n’est pas le berger, à qui n'appartiennent pas les brebis, voit venir le loup, abandonne les brebis, et s'enfuit ; et le loup ravit et disperse les brebis.
\VS{13}Ainsi le mercenaire s'enfuit, parce qu'il est mercenaire, et qu'il ne se soucie pas des brebis. Je suis le bon berger.
\VS{14}Je connais mes brebis, et mes brebis me connaissent.
\VS{15}Comme le Père me connaît, et comme je connais le Père ; et je donne ma vie pour mes brebis.
\VS{16}J'ai encore d'autres brebis qui ne sont pas de cette bergerie ; celles-là, il faut aussi que je les amène ; elles entendront ma voix, et il y aura un seul troupeau, et un seul berger.
\VS{17}Le Père m'aime, parce que je donne ma vie, afin de la reprendre.
\VS{18}Personne ne me l'ôte, mais je la donne de moi-même. J'ai le pouvoir de la donner, et j’ai le pouvoir de la reprendre ; j'ai reçu cet ordre de mon Père.
\VS{19}Il y eut de nouveau division parmi les Juifs à cause de ces discours.
\VS{20}Car plusieurs disaient : Il a un démon, il est fou ! Pourquoi l'écoutez-vous ?
\VS{21}Et les autres disaient : Ce ne sont pas les paroles d'un démoniaque ; un démon peut-il ouvrir les yeux des aveugles ?
\TextTitle{[Jésus réaffirme sa divinité]
\\(Jn. 5:26-27 ; 14:9 ; 20:28-29)}
\VS{22}On célébrait la fête de la dédicace\FTNT{Le terme ~dédicace~ est la traduction du mot hébreu ~Hanoukka~ qui sert à désigner la consécration ou l'inauguration de l'autel servant à offrir des sacrifices à Dieu (No. 7:10 ; 2 Ch. 7:9). La Bible l’utilise aussi pour parler de l'inauguration des murailles de Jérusalem après leur reconstruction au temps de Néhémie (Né. 12:27). La fête d’Hanoukka a été instituée par Judas Maccabé en 164 av. J.-C. en mémoire de la purification du temple qui avait été profané par Antiochus Epiphane. Elle débute le 25 du mois de chisleu (mi décembre) de chaque année et dure huit jours.} à Jérusalem. Et c'était l’hiver.
\VS{23}Et Jésus se promenait dans le temple, au portique de Salomon.
\VS{24}Et les Juifs l’entourèrent et lui dirent : Jusqu’à quand tiendras-tu notre âme en suspens ? Si tu es le Christ, dis-le-nous franchement.
\VS{25}Jésus leur répondit : Je vous l'ai dit, et vous ne le croyez point. Les œuvres que je fais au Nom de mon Père rendent témoignage de moi.
\VS{26}Mais vous ne croyez point, parce que vous n'êtes point de mes brebis, comme je vous l'ai dit.
\VS{27}Mes brebis entendent ma voix ; je les connais, et elles me suivent.
\VS{28}Et moi, je leur donne la vie éternelle, et elles ne périront jamais ; et personne ne les ravira de ma main.
\VS{29}Mon Père, qui me les a données, est plus grand que tous ; et personne ne peut les ravir des mains de mon Père.
\VS{30}Moi et le Père nous sommes un.
\VS{31}Alors les Juifs prirent de nouveau des pierres pour le lapider.
\VS{32}Jésus leur dit : Je vous ai fait voir plusieurs bonnes œuvres de la part de mon Père : Pour laquelle me lapidez-vous ?
\VS{33}Les Juifs répondirent : Ce n’est pas pour une bonne œuvre que nous te lapidons, mais pour un blasphème, parce que toi qui es un homme, tu te fais Dieu.
\VS{34}Jésus leur répondit : N’est-il pas écrit dans votre loi : J'ai dit : Vous êtes des dieux\FTNT{Ps. 82:6 : Le sens du mot ~dieu~ peut désigner des personnes ayant un certain pouvoir. D'ailleurs, le mot hébreu utilisé dans Ps. 82:6 est ~Elohim~, or ce mot signifie aussi ~juge~. De plus, dans le contexte du psaume, ~vous êtes des dieux~ ne s'applique pas à tous, mais seulement à une certaine catégorie de personnes qui exerçaient un pouvoir en Israël : rois, scribes, souverains sacrificateurs… Rappelons-nous aussi que Dieu a fait de Moïse un dieu pour Aaron (Ex. 7:1-2), mais cela n’a pas fait de lui le Dieu Créateur pour autant. En Jn. 17:3, Jésus atteste qu'il n’y a qu’un seul vrai Dieu. Satan veut nous faire croire que nous sommes des dieux et nous amener ainsi à pécher par l’orgueil (Ge. 3:5). Toutefois, comme le souligne si bien l’apôtre Paul, même s’il existe des créatures qu’on appelle dieux ou déesses, il ne reste pas moins vrai qu’il n’y a qu’un seul Dieu (1 Co. 8:5-7).} ?
\VS{35}Si elle a appelé dieux ceux à qui la parole de Dieu est adressée, et cependant l'Ecriture ne peut être anéantie,
\VS{36}celui que le Père a sanctifié et envoyé dans le monde, vous lui dites : Tu blasphèmes ! Et cela parce que j’ai dit : Je suis le Fils de Dieu ?
\VS{37}Si je ne faisais pas les œuvres de mon Père, ne me croyez pas.
\VS{38}Mais si je les fais, même si vous ne me croyez pas, croyez à ces œuvres, afin que vous sachiez que le Père est en moi et que je suis dans le Père.
\VS{39}Là-dessus, ils cherchaient encore à le saisir ; mais il s’échappa de leurs mains.
\TextTitle{[Jésus se retire de Jérusalem]}
\VS{40}Il s'en alla de nouveau au-delà du Jourdain, à l'endroit où Jean avait baptisé au commencement, et il demeura là.
\VS{41}Beaucoup de gens vinrent à lui, et ils disaient : Jean n’a fait aucun miracle ; mais tout ce que Jean a dit de cet homme, était vrai.
\VS{42}Et dans ce lieu-là, plusieurs crurent en lui.
\TextTitle{[Jésus ressuscite Lazare de Béthanie]}
\Chap{11}
\VerseOne{}Il y avait un homme malade, Lazare, de Béthanie, village de Marie et de Marthe, sa sœur.
\VS{2}C’était cette Marie qui oignit de parfum le Seigneur, et qui essuya ses pieds avec ses cheveux ; et c’était son frère Lazare qui était malade.
\VS{3}Ses sœurs envoyèrent donc dire à Jésus : Seigneur, voici, celui que tu aimes est malade.
\VS{4}Après avoir entendu cela, Jésus dit : Cette maladie n'est point à la mort, mais elle est pour la gloire de Dieu, afin que le Fils de Dieu soit glorifié par elle.
\VS{5}Or Jésus aimait Marthe, sa sœur, et Lazare.
\VS{6}Après qu'il eut appris que Lazare était malade, il resta deux jours encore dans le lieu où il était,
\VS{7}et il dit à ses disciples : Retournons en Judée.
\VS{8}Les disciples lui dirent : Rabbi, les Juifs tout récemment cherchaient à te lapider, et tu retournes en Judée !
\VS{9}Jésus répondit : N'y a-t-il pas douze heures au jour ? Si quelqu'un marche pendant le jour, il ne bronche point ; car il voit la lumière de ce monde.
\VS{10}Mais si quelqu'un marche pendant la nuit, il bronche ; car il n'y a point de lumière avec lui.
\VS{11}Après ces paroles, il leur dit : Notre ami Lazare dort, mais je vais le réveiller.
\VS{12}Ses disciples lui dirent : Seigneur, s'il dort, il sera guéri.
\VS{13}Jésus avait parlé de sa mort, mais ils pensaient qu'il parlait de l’assoupissement.
\VS{14}Alors Jésus leur dit ouvertement : Lazare est mort.
\VS{15}Et je me réjouis, à cause de vous, de ce que je n’étais pas là, afin que vous croyiez. Mais allons vers lui.
\VS{16}Alors Thomas, appelé Didyme, dit aux autres disciples : Allons-y aussi, afin que nous mourions avec lui.
\VS{17}Jésus, étant arrivé, trouva que Lazare était déjà depuis quatre jours dans le sépulcre.
\VS{18}Et comme Béthanie était près de Jérusalem à quinze stades environ,
\VS{19}beaucoup de Juifs étaient venus vers Marthe et Marie pour les consoler au sujet de leur frère.
\VS{20}Lorsque Marthe apprit que Jésus arrivait, elle alla au-devant de lui ; mais Marie se tenait assise à la maison.
\VS{21}Marthe dit à Jésus : Seigneur, si tu avais été ici mon frère ne serait pas mort.
\VS{22}Mais maintenant je sais que tout ce que tu demanderas à Dieu, Dieu te le donnera.
\VS{23}Jésus lui dit : Ton frère ressuscitera.
\VS{24}Marthe lui dit : Je sais qu'il ressuscitera à la résurrection, au dernier jour.
\VS{25}Jésus lui dit : Je suis la résurrection et la vie : Celui qui croit en moi vivra même s’il meurt.
\VS{26}Et quiconque vit et croit en moi ne mourra jamais ; crois-tu cela ?
\VS{27}Elle lui dit : Oui, Seigneur, je crois que tu es le Christ, le Fils de Dieu, qui devait venir dans le monde.
\VS{28}Ayant ainsi parlé, elle alla appeler secrètement Marie sa sœur, en lui disant : Le Maître est ici, et il t'appelle.
\VS{29}Aussitôt que Marie eut entendu, elle se leva rapidement, et alla vers lui.
\VS{30}Or Jésus n'était pas encore entré dans le village, mais il était au lieu où Marthe l'avait rencontré.
\VS{31}Alors les Juifs qui étaient avec Marie à la maison, et qui la consolaient, ayant vu qu'elle s'était levée si promptement, et qu'elle était sortie, la suivirent en disant : Elle va au sépulcre pour y pleurer.
\VS{32}Lorsque Marie fut arrivée où était Jésus, et qu’elle le vit, elle se jeta à ses pieds, en lui disant : Seigneur, si tu avais été ici, mon frère ne serait pas mort.
\VS{33}Jésus, la voyant pleurer, elle et les Juifs qui étaient venus avec elle, frémit en son esprit et fut tout ému.
\VS{34}Et il dit : Où l'avez-vous mis ? Ils lui répondirent : Seigneur, viens et vois.
\VS{35}Jésus pleura.
\VS{36}Sur quoi les Juifs dirent : Voyez comme il l'aimait.
\VS{37}Et quelques-uns d'entre eux disaient : Lui qui a ouvert les yeux de l'aveugle, ne pouvait-il pas faire aussi que cet homme ne meure point ?
\VS{38}Alors Jésus frémissant de nouveau en lui-même, se rendit au sépulcre. C'était une grotte, et il y avait une pierre placée devant.
\VS{39}Jésus dit : Ôtez la pierre. Mais Marthe, la sœur du mort, lui dit : Seigneur, il sent déjà, car il est là depuis quatre jours.
\VS{40}Jésus lui dit : Ne t'ai-je pas dit que si tu crois tu verras la gloire de Dieu ?
\VS{41}Ils ôtèrent donc la pierre de dessus le lieu où le mort était couché. Et Jésus levant ses yeux au ciel, dit : Père, je te rends grâces de ce que tu m'as exaucé.
\VS{42}Pour moi, je savais que tu m'exauces toujours ; mais j’ai parlé à cause de la foule qui m’entoure, afin qu’ils croient que c’est toi qui m'as envoyé.
\VS{43}Ayant dit ces choses, il cria à haute voix : Lazare sors dehors !
\VS{44}Alors le mort sortit, ayant les mains et les pieds liés de bandes ; et son visage était enveloppé d'un linge. Jésus leur dit : Déliez-le, et laissez-le aller.
\TextTitle{[Nombreuses conversions]
\\(Jn. 12:10-11)}
\TextTitle{[Conspiration des Pharisiens]}
\VS{45}Plusieurs des Juifs qui étaient venus vers Marie, et qui avaient vu ce que Jésus avait fait, crurent en lui.
\VS{46}Mais quelques-uns d'entre eux allèrent trouver les pharisiens et leur dirent les choses que Jésus avait faites.
\VS{47}Alors les principaux sacrificateurs et les pharisiens assemblèrent le sanhédrin, et ils dirent : Que ferons-nous ? Car cet homme fait beaucoup de miracles.
\VS{48}Si nous le laissons faire, tout le monde croira en lui, et les Romains viendront et ils détruiront et ce lieu et notre nation.
\VS{49}Alors l'un d'eux appelé Caïphe, qui était le souverain sacrificateur cette année-là, leur dit : Vous n’y comprenez rien.
\VS{50}Et vous ne réfléchissez pas qu'il est de notre intérêt qu'un homme meure pour le peuple, et que toute la nation ne périsse point.
\VS{51}Or il ne dit pas cela de lui-même, mais étant souverain sacrificateur de cette année-là, il prophétisa que Jésus devait mourir pour la nation.
\VS{52}Et non pas seulement pour la nation, mais aussi pour rassembler en un seul corps les enfants de Dieu dispersés.
\VS{53}Depuis ce jour, ils se concertèrent ensemble pour le faire mourir.
\VS{54}C'est pourquoi Jésus ne se montrait plus ouvertement parmi les Juifs, mais il se retira dans la contrée voisine du désert, dans une ville appelée Ephraïm, et il demeura là avec ses disciples.
\VS{55}La Pâque des Juifs était proche. Et beaucoup de gens du pays montèrent à Jérusalem avant Pâque, afin de se purifier.
\VS{56}Et ils cherchaient Jésus, et se disaient les uns les autres dans le temple : Que vous en semble ? Ne viendra-t-il pas à la Fête ?
\VS{57}Or, les principaux sacrificateurs et les pharisiens avaient donné l’ordre que si quelqu'un savait où il était, il le déclare, afin qu’on se saisisse de lui.
\TextTitle{[Marie de Béthanie oint les pieds de Jésus]
\\(Mt. 26:6-13 ; Mc. 14:3-9)}
\Chap{12}
\VerseOne{}Six jours avant la Pâque, Jésus arriva à Béthanie, où était Lazare qui avait été mort, et qu'il avait ressuscité des morts.
\VS{2}Là, on lui fit un souper ; Marthe servait, et Lazare était un de ceux qui étaient à table avec lui.
\VS{3}Alors Marie ayant pris une livre de nard pur de grand prix, oignit les pieds de Jésus, et les essuya avec ses cheveux ; et la maison fut remplie de l'odeur du parfum.
\VS{4}Alors Judas Iscariot, fils de Simon, l'un de ses disciples, celui qui devait le trahir, dit :
\VS{5}Pourquoi ce parfum n'a-t-il pas été vendu trois cents deniers, pour donner cet argent aux pauvres ?
\VS{6}Il dit cela, non parce qu’il se mettait en peine des pauvres, mais parce qu'il était voleur, et que tenant la bourse, il prenait ce qu’on y mettait.
\VS{7}Mais Jésus lui dit : Laisse-la faire ; elle l'a gardé pour le jour de ma sépulture.
\VS{8}Car vous aurez toujours des pauvres avec vous ; mais vous ne m'aurez pas toujours.
\VS{9}Une grande multitude des Juifs apprirent que Jésus était à Béthanie, et ils y vinrent, non seulement à cause de lui, mais aussi pour voir Lazare qu'il avait ressuscité des morts.
\VS{10}Sur quoi les principaux sacrificateurs résolurent de faire mourir aussi Lazare.
\VS{11}Car plusieurs des Juifs se retiraient d'avec eux à cause de lui, et croyaient en Jésus.
\TextTitle{[Entrée triomphante de Jésus à Jérusalem]
\\(Mt. 21:1-11 ; Mc. 11:1-11 ; Lu. 19:28-40 ; Za. 9:9 ; Ap. 19:11-16}
\VS{12}Le lendemain, une grande quantité de foules qui étaient venues à la fête, ayant entendu dire que Jésus se rendait à Jérusalem,
\VS{13}prit des branches de palmes, et sortit au-devant de lui en criant : Hosanna ! Béni soit le Roi d'Israël qui vient au Nom du Seigneur !
\VS{14}Jésus trouva un ânon, s'assit dessus, selon ce qui est écrit : 15 Ne crains point, fille de Sion ; voici, ton Roi vient, assis sur le petit d'une ânesse\FTNT{Za. 9:9.}.
\VS{16}Ses disciples ne comprirent pas d'abord ces choses ; mais quand Jésus eut été glorifié, ils se souvinrent alors qu’elles étaient écrites de lui, et qu’elles avaient été accomplies à son égard.
\VS{17}Tous ceux qui avaient été avec Jésus, quand il appela Lazare du sépulcre et le ressuscita des morts, lui rendaient témoignage ;
\VS{18}et la foule alla au-devant de lui, parce qu’elle avait appris qu'il avait fait ce miracle.
\VS{19}Sur quoi les pharisiens dirent entre eux : Vous ne voyez pas que vous ne gagnez rien ? Voici, le monde va après lui.
\TextTitle{[Quelques Grecs cherchent à voir Jésus]}
\VS{20}Quelques Grecs du nombre de ceux qui étaient montés pour adorer Dieu pendant la fête,
\VS{21}s’adressèrent à Philippe, qui était de Bethsaïda de Galilée, et lui dirent avec instances : Seigneur ! Nous voudrions voir Jésus.
\VS{22}Philippe alla le dire à André, et André et Philippe le dirent à Jésus.
\TextTitle{[Jésus annonce sa crucifixion]}
\VS{23}Jésus leur répondit, disant : L'heure est venue où le Fils de l'homme doit être glorifié.
\VS{24}En vérité, en vérité je vous le dis : Si le grain de blé qui est tombé en la terre ne meurt, il reste seul ; mais s'il meurt, il porte beaucoup de fruits.
\VS{25}Celui qui aime sa vie la perdra ; et celui qui hait sa vie dans ce monde, la conservera pour la vie éternelle.
\VS{26}Si quelqu'un me sert, qu'il me suive ; et là où je serai, là aussi sera celui qui me sert ; et si quelqu'un me sert, mon Père l'honorera.
\VS{27}Maintenant mon âme est troublée. Et que dirai-je ? Ô Père, délivre-moi de cette heure ? Mais c'est pour cela que je suis venu jusqu’à cette heure.
\VS{28}Père glorifie ton Nom ! Alors une voix vint du ciel et dit : Je l'ai glorifié, et je le glorifierai encore.
\VS{29}Et la foule qui était là, et qui avait entendu cette voix, disait que c'était un coup de tonnerre ; les autres disaient : Un ange lui a parlé.
\VS{30}Jésus prit la parole et dit : Ce n’est pas à cause de moi que cette voix s’est fait entendre ; c’est à cause de vous.
\VS{31}Maintenant est venu le jugement de ce monde ; maintenant le prince de ce monde sera jeté dehors.
\VS{32}Et moi, quand je serai élevé de la terre, j’attirerai tous les hommes à moi.
\VS{33}En parlant ainsi, il indiquait de quelle mort il devait mourir.
\VS{34}La foule lui répondit : Nous avons appris par la loi que le Christ demeure éternellement, comment donc dis-tu qu'il faut que le Fils de l'homme soit élevé ? Qui est ce Fils de l'homme ?
\VS{35}Alors Jésus leur dit : La Lumière est encore avec vous pour un peu de temps : Marchez pendant que vous avez la Lumière, de peur que les ténèbres ne vous surprennent ; car celui qui marche dans les ténèbres ne sait pas où il va.
\VS{36}Pendant que vous avez la Lumière, croyez en la Lumière, afin que vous soyez enfants de lumière. Jésus dit ces choses, puis il s'en alla, et se cacha de devant eux.
\VS{37}Malgré tant de miracles qu’il avait faits en leur présence, ils ne croyaient point en lui,
\VS{38}afin que s’accomplisse cette parole qui a été dite par Esaïe le prophète : Seigneur, qui a cru à notre parole, et à qui a été révélé le bras du Seigneur\FTNT{Es. 53:1.} ?
\VS{39}C'est pourquoi ils ne pouvaient pas croire, parce qu'Esaïe a dit encore :
\VS{40}Il a aveuglé leurs yeux, et il a endurci leur cœur, de peur qu'ils ne voient de leurs yeux, qu'ils ne comprennent du cœur, qu'ils ne se convertissent, et que je ne les guérisse\FTNT{Es. 6:9-10.}.
\VS{41}Esaïe dit ces choses quand il vit sa gloire, et qu'il parla de lui.
\VS{42}Cependant, même parmi les chefs, plusieurs crurent en lui ; mais ils ne le confessaient pas à cause des pharisiens, de peur d'être exclus de la synagogue.
\VS{43}Car ils aimèrent la gloire des hommes, plus que la gloire de Dieu.
\VS{44}Or Jésus s'écria et dit : Celui qui croit en moi, ne croit pas seulement en moi, mais en celui qui m'a envoyé.
\VS{45}Et celui qui me voit, voit celui qui m'a envoyé.
\VS{46}Je suis venu dans le monde pour en être la Lumière, afin que quiconque croit en moi ne demeure point dans les ténèbres.
\VS{47}Si quelqu'un entend mes paroles, et ne les garde point, ce n’est pas moi qui le juge ; car je ne suis point venu pour juger le monde, mais pour sauver le monde.
\VS{48}Celui qui me rejette et qui ne reçoit pas mes paroles, a son juge : La parole que j'ai annoncée sera celle qui le jugera au dernier jour.
\VS{49}Car je n'ai point parlé de moi-même, mais le Père qui m'a envoyé, m'a prescrit ce que je dois dire et annoncer.
\VS{50}Et je sais que son commandement est la vie éternelle ; les choses donc que je dis, je les dis comme mon Père me les a dites.
\TextTitle{[L'entretien de Jn. 13-14 eut lieu dans la chambre haute ; Mc. 14:14-16]}
\Chap{13}
\VerseOne{}Avant la fête de Pâque, Jésus sachant que son heure était venue de passer de ce monde au Père, et ayant aimé les siens, qui étaient dans le monde, il les aima jusqu'à la fin.
\TextTitle{[La dernière Pâque ; Jésus lave les pieds de ses disciples]
\\(Mt. 26:20-24 ; Mc. 14:17 ; Lu. 22:14,21-23)}
\VS{2}Pendant le souper, alors que le diable avait déjà mis dans le cœur de Judas Iscariot, fils de Simon, de le trahir,
\VS{3}Jésus sachant que le Père avait remis toutes choses entre ses mains, qu'il était venu de Dieu, et qu’il s'en allait à Dieu,
\VS{4}se leva de table, ôta ses vêtements, et prit un linge, dont il se ceignit.
\VS{5}Puis il mit de l'eau dans un bassin, et se mit à laver les pieds de ses disciples, et à les essuyer avec le linge dont il se ceignit.
\VS{6}Alors il vint à Simon Pierre, mais Pierre lui dit : Toi, Seigneur, tu me laves les pieds ?
\VS{7}Jésus répondit et lui dit : Tu ne comprends pas maintenant ce que je fais, mais tu le sauras dans la suite.
\VS{8}Pierre lui dit : Tu ne me laveras jamais les pieds ! Jésus lui répondit : Si je ne te lave pas, tu n'auras point de part avec moi.
\VS{9}Simon Pierre lui dit : Seigneur, non seulement mes pieds, mais aussi les mains et la tête.
\VS{10}Jésus lui dit : Celui qui est baigné n’a besoin que de se laver les pieds pour être entièrement pur ; vous êtes purs, mais non pas tous.
\VS{11}Car il savait qui était celui qui le trahirait ; c'est pourquoi il dit : Vous n'êtes pas tous purs.
\VS{12}Après qu'il leur eut lavé les pieds, il reprit ses vêtements, et s'étant remis à table, il leur dit : Comprenez-vous ce que je vous ai fait ?
\VS{13}Vous m'appelez Maître et Seigneur ; et vous dites bien, car je le suis.
\VS{14}Si donc moi, qui suis le Seigneur et le Maître, j'ai lavé vos pieds, vous devez aussi vous laver les pieds les uns des autres.
\VS{15}Car je vous ai donné un exemple, afin que vous fassiez comme je vous ai fait.
\VS{16}En vérité, en vérité je vous le dis : Le serviteur n'est pas plus grand que son maître ni l’apôtre plus grand que celui qui l'a envoyé.
\VS{17}Si vous savez ces choses, vous êtes heureux, pourvu que vous les pratiquiez.
\VS{18}Je ne parle pas de vous tous, je connais ceux que j'ai choisis. Mais il faut que l’Ecriture s’accomplisse : Celui qui mange le pain avec moi, a levé son talon contre moi\FTNT{Ps. 41:10.}.
\VS{19}Je vous dis ceci dès maintenant, avant que la chose arrive, afin que lorsqu’elle arrivera, vous croyiez que c'est moi que le Père a envoyé.
\VS{20}En vérité, en vérité je vous le dis : Celui qui reçoit celui que j’aurai envoyé, me reçoit ; et celui qui me reçoit, reçoit celui qui m'a envoyé.
\TextTitle{[Jésus annonce la trahison de Judas]
\\(Mt. 26:21-25 ; Mc. 14:18-21 ; Lu. 22:21-23)}
\VS{21}Ayant ainsi parlé, Jésus fut ému dans son esprit, et il déclara : En vérité, en vérité je vous le dis, l'un de vous me trahira.
\VS{22}Alors les disciples se regardaient les uns les autres, ne sachant de qui il parlait.
\VS{23}Un des disciples, celui que Jésus aimait, était à table couché sur le sein de Jésus.
\VS{24}Simon Pierre lui fit signe de demander qui était celui dont Jésus parlait.
\VS{25}Et ce disciple, s’étant penché sur la poitrine de Jésus, lui dit : Seigneur, qui est-ce ?
\VS{26}Jésus répondit : C'est celui à qui je donnerai le morceau trempé ; et ayant trempé le morceau, il le donna à Judas Iscariot, fils de Simon.
\VS{27}Après que Judas eut pris le morceau, Satan entra en lui. Jésus lui dit : Ce que tu fais, fais-le promptement.
\VS{28}Mais aucun de ceux qui étaient à table ne comprit pourquoi il lui avait dit cela.
\VS{29}Car quelques-uns pensaient que, comme Judas avait la bourse, Jésus voulait lui dire : Achète ce qui nous est nécessaire pour la Fête ; ou qu'il lui commandait de donner quelque chose aux pauvres.
\VS{30}Judas, ayant pris le morceau, sortit aussitôt. Il faisait nuit.
\VS{31}Lorsque Judas fut sorti, Jésus dit : Maintenant le Fils de l'homme est glorifié ; et Dieu est glorifié en lui.
\VS{32}Si Dieu est glorifié en lui, Dieu aussi le glorifiera en lui-même, et il le glorifiera bientôt.
\VS{33}Mes petits enfants, je suis encore pour un peu de temps avec vous ; vous me chercherez, mais comme j'ai dit aux Juifs : Vous ne pouvez pas venir où je vais, je vous le dis aussi maintenant.
\VS{34}Je vous donne un nouveau commandement : Aimez-vous les uns les autres. Comme je vous ai aimés, vous aussi, aimez-vous les uns les autres.
\VS{35}A ceci tous connaîtront que vous êtes mes disciples, si vous avez de l'amour les uns pour les autres.
\TextTitle{[Jésus annonce le reniement de Pierre]
\\(Mt. 26:30-35 ; Mc. 14:26-31 ; Lu. 22:31-34)}
\VS{36}Simon Pierre lui dit : Seigneur ! Où vas-tu ? Jésus lui répondit : Là où je vais, tu ne peux pas me suivre maintenant, mais tu me suivras plus tard.
\VS{37}Pierre lui dit : Seigneur ! Pourquoi ne puis-je pas te suivre maintenant ? J’exposerai ma vie pour toi.
\VS{38}Jésus lui répondit : Tu exposeras ta vie pour moi ? En vérité, en vérité je te le dis, le coq ne chantera pas, que tu ne m'aies renié trois fois.
\TextTitle{[Jésus réconforte les apôtres : Il reviendra vers eux]}
\Chap{14}
\VerseOne{}Que votre cœur ne se trouble point ; vous croyez en Dieu, croyez aussi en moi.
\VS{2}Il y a plusieurs demeures dans la maison de mon Père. Si cela n’était pas, je vous l’aurais dit ; je vais vous préparer une place.
\VS{3}Et quand je m'en serai allé, et que je vous aurai préparé une place, je reviendrai, et je vous prendrai avec moi ; afin que là où je suis, vous y soyez aussi.
\VS{4}Et vous savez où je vais, et vous en savez le chemin.
\VS{5}Thomas lui dit : Seigneur ! Nous ne savons point où tu vas, comment donc pouvons-nous en savoir le chemin ?
\VS{6}Jésus lui dit : Je suis le chemin, la vérité, et la vie ; nul ne vient au Père que par moi.
\TextTitle{[Le Père et le Fils sont un]}
\VS{7}Si vous me connaissiez, vous connaîtriez aussi mon Père ; mais dès maintenant vous le connaissez, et vous l'avez vu.
\VS{8}Philippe lui dit : Seigneur ! Montre-nous le Père, et cela nous suffit.
\VS{9}Jésus lui répondit : Je suis depuis si longtemps avec vous, et tu ne m'as pas connu ? Philippe ! Celui qui m'a vu a vu mon Père. Et comment dis-tu : Montre-nous le Père ?
\VS{10}Ne crois-tu pas que je suis dans le Père, et que le Père est en moi ? Les paroles que je vous dis, je ne les dis pas de moi-même ; mais le Père qui demeure en moi est celui qui fait les œuvres.
\VS{11}Croyez-moi, je suis dans le Père, et le Père est en moi, sinon croyez-moi à cause de ces œuvres.
\VS{12}En vérité, en vérité je vous le dis : Celui qui croit en moi fera les œuvres que je fais, et il en fera même de plus grandes que celles-ci, parce que je m'en vais vers mon Père.
\TextTitle{[Nouveau privilège par la prière]}
\VS{13}Et tout ce que vous demanderez en mon Nom, je le ferai ; afin que le Père soit glorifié dans le Fils.
\VS{14}Si vous demandez en mon Nom quelque chose, je le ferai.
\TextTitle{[Promesse quant à l'habitation de l'Esprit dans le coeur du croyant]}
\VS{15}Si vous m'aimez, gardez mes commandements.
\VS{16}Et moi, je prierai le Père, et il vous donnera un autre Consolateur, pour demeurer avec vous éternellement,
\VS{17}l'Esprit de vérité que le monde ne peut recevoir, parce qu'il ne le voit point, et qu'il ne le connaît point ; mais vous le connaissez, car il demeure avec vous, et il sera en vous.
\VS{18}Je ne vous laisserai pas orphelins, je viendrai vers vous.
\VS{19}Encore un peu de temps, et le monde ne me verra plus ; mais vous me verrez, parce que je vis, et vous aussi vous vivrez.
\VS{20}En ce jour-là vous connaîtrez que je suis en mon Père, que vous êtes en moi, et moi en vous.
\VS{21}Celui qui a mes commandements et qui les garde, c'est celui qui m'aime ; et celui qui m'aime sera aimé de mon Père ; je l'aimerai, et je me ferai connaître à lui.
\VS{22}Jude, non pas Iscariot, lui dit : Seigneur ! D’où vient que tu te feras connaître à nous, et non au monde ?
\VS{23}Jésus répondit et lui dit : Si quelqu'un m'aime, il gardera ma parole, et mon Père l'aimera, et nous viendrons à lui, et nous ferons notre demeure chez lui.
\VS{24}Celui qui ne m'aime point ne garde point mes paroles. Et la parole que vous entendez n'est point ma parole, mais c'est celle du Père qui m'a envoyé.
\VS{25}Je vous ai dit ces choses pendant que je demeure avec vous.
\VS{26}Mais le Consolateur, le Saint-Esprit, que le Père enverra en mon Nom, vous enseignera toutes choses, et il vous rappellera tout ce que je vous ai dit.
\TextTitle{[Christ donne sa paix]}
\VS{27}Je vous laisse la paix, je vous donne ma paix ; je ne vous la donne pas comme le monde la donne ; que votre cœur ne se trouble point, et ne s’alarme point.
\VS{28}Vous avez entendu que je vous ai dit : Je m'en vais, et je reviens à vous ; si vous m'aimiez, vous seriez certes joyeux de ce que j'ai dit : Je m'en vais au Père, car le Père est plus grand que moi.
\VS{29}Et maintenant je vous l'ai dit avant que cela soit arrivé, afin que quand il sera arrivé, vous croyiez.
\VS{30}Je ne parlerai plus guère avec vous ; car le prince de ce monde vient ; mais il n'a rien en moi.
\VS{31}Mais afin que le monde sache que j'aime le Père, et que je fais ce que le Père m'a commandé : Levez-vous, partons d'ici.
\TextTitle{[Le cep et les sarments]}
\Chap{15}
\VerseOne{}Je suis le vrai Cep\FTNT{Jésus est l’arbre de vie qui produit de bons fruits en nous, à condition que nous nous tenions loin de l’arbre de la connaissance du bien et du mal. Jésus, le vrai Cep, est la source de vie. La viabilité du sarment dépend de son attachement au Cep. Jésus a été pendu au bois (Ac 5:30), s’est chargé de nos malédictions (Ga. 3:13) et a été retranché à notre place.}, et mon Père est le Vigneron.
\VS{2}Il retranche tout sarment qui est en moi et qui ne porte pas de fruits ; et tout sarment qui porte du fruit, il l'émonde afin qu'il porte encore plus de fruits.
\VS{3}Vous êtes déjà purs à cause de la parole que je vous ai enseignée.
\VS{4}Demeurez en moi, et je demeurerai en vous ; comme le sarment ne peut de lui-même porter du fruit s'il ne demeure pas attaché au cep ; ainsi vous ne le pouvez pas non plus si vous ne demeurez pas en moi.
\VS{5}Je suis le Cep, et vous en êtes les sarments ; celui qui demeure en moi, et en qui je demeure porte beaucoup de fruits ; car hors de moi, vous ne pouvez rien produire.
\VS{6}Si quelqu'un ne demeure point en moi, il est jeté dehors comme le sarment, et il se sèche ; puis on l'amasse, on le met au feu, et il brûle.
\VS{7}Si vous demeurez en moi, et que mes paroles demeurent en vous, demandez tout ce que vous voudrez, et cela vous sera fait.
\VS{8}Si vous portez beaucoup de fruits, mon Père sera glorifié et vous serez alors mes disciples.
\VS{9}Comme le Père m'a aimé, ainsi je vous ai aimés, demeurez dans mon amour.
\VS{10}Si vous gardez mes commandements, vous demeurerez dans mon amour ; comme j'ai gardé les commandements de mon Père, et je demeure dans son amour.
\VS{11}Je vous ai dit ces choses afin que ma joie demeure en vous, et que votre joie soit parfaite.
\VS{12}C'est ici mon commandement : Aimez-vous les uns les autres comme je vous ai aimés.
\VS{13}Il n’y a pas de plus grand amour que de donner sa vie pour ses amis.
\VS{14}Vous serez mes amis, si vous faites tout ce que je vous commande.
\TextTitle{[Nouvelle intimité entre le Seigneur et les siens]}
\VS{15}Je ne vous appelle plus serviteurs, car le serviteur ne sait pas ce que fait son maître; mais je vous ai appelés mes amis, parce que je vous ai fait connaître tout ce que j'ai appris de mon Père.
\VS{16}Ce n'est pas vous qui m'avez choisi ; mais moi, je vous ai choisis, et je vous ai établis afin que vous alliez partout et que vous produisiez du fruit, et que votre fruit demeure ; afin que tout ce que vous demanderez au Père en mon Nom, il vous le donne.
\VS{17}Ce que je vous commande, c’est de vous aimer les uns les autres.
\TextTitle{[L'attitude du monde à l'égard des croyants en Christ]}
\VS{18}Si le monde vous hait, sachez qu’il m’a haï avant vous.
\VS{19}Si vous étiez du monde, le monde aimerait ce qui est à lui ; mais parce que vous n'êtes pas du monde, et que je vous ai choisis du milieu du monde, à cause de cela le monde vous hait.
\VS{20}Souvenez-vous de la parole que je vous ai dite : Le serviteur n'est pas plus grand que son maître ; s'ils m'ont persécuté, ils vous persécuteront aussi ; s'ils ont gardé ma parole, ils garderont aussi la vôtre.
\VS{21}Mais ils vous feront toutes ces choses à cause de mon Nom, parce qu'ils ne connaissent point celui qui m'a envoyé.
\VS{22}Si je n’étais pas venu, et que je ne leur avais point parlé, ils n'auraient point de péché, mais maintenant ils n'ont point d'excuse de leur péché.
\VS{23}Celui qui me hait, hait aussi mon Père.
\VS{24}Si je n’avais pas fait parmi eux les œuvres qu'aucun autre n'a faites, ils n'auraient point de péché ; mais maintenant ils les ont vues, et ils ont haï et moi et mon Père.
\VS{25}Mais cela est arrivé afin que s’accomplisse la parole qui est écrite dans leur loi : Ils m'ont haï sans cause\FTNT{Ps. 35:19 ; Ps. 69:5.}.
\VS{26}Mais quand le Consolateur sera venu, que je vous enverrai de la part de mon Père, l'Esprit de vérité qui procède de mon Père, il rendra témoignage de moi.
\VS{27}Et vous aussi, vous rendrez témoignage, car vous êtes dès le commencement avec moi.
\TextTitle{[Jésus avertit les siens de la persécution]
\\(Mt. 24:9-10 ; Lu. 21:16-19)}
\Chap{16}
\VerseOne{}Je vous ai dit ces choses, afin que vous ne soyez pas scandalisés.
\VS{2}Ils vous chasseront des synagogues ; et même l’heure vient où quiconque vous fera mourir, croira rendre un culte à Dieu.
\VS{3}Et ils vous feront ces choses, parce qu'ils n’ont connu ni le Père ni moi.
\VS{4}Je vous ai dit ces choses, afin que, lorsque l'heure sera venue, vous vous souveniez que je vous les ai dites ; je ne vous en ai pas parlé dès le commencement, parce que j'étais avec vous.
\VS{5}Mais maintenant je m'en vais vers celui qui m'a envoyé, et aucun de vous ne me demande : Où vas-tu ?
\VS{6}Mais parce que je vous ai dit ces choses, la tristesse a rempli votre cœur.
\TextTitle{[La triple activité de l'Esprit agit en faveur du monde]}
\VS{7}Toutefois je vous dis la vérité, il vous est avantageux que je m'en aille, car si je ne m'en vais pas, le Consolateur ne viendra pas vers vous ; mais si je m'en vais, je vous l'enverrai.
\VS{8}Et quand il sera venu, il convaincra le monde de péché, de justice, et de jugement :
\VS{9}du péché, parce qu'ils ne croient point en moi,
\VS{10}de justice, parce que je m'en vais à mon Père, et que vous ne me verrez plus ;
\VS{11}de jugement, parce que le prince de ce monde est déjà jugé.
\TextTitle{[Après son ascension, Christ continuera de révéler la verité par l'Esprit]}
\VS{12}J'ai encore beaucoup de choses à vous dire, mais vous ne pouvez pas les porter maintenant.
\VS{13}Mais quand le Consolateur sera venu, l'Esprit de vérité, il vous conduira dans toute la vérité ; car il ne parlera pas de lui-même, mais il dira tout ce qu'il aura entendu, et il vous annoncera les choses à venir.
\VS{14}Il me glorifiera, car il prendra ce qui est à moi, et vous l'annoncera.
\VS{15}Tout ce que mon Père a, est à moi ; c'est pourquoi j'ai dit qu'il prendra ce qui est à moi et qu’il vous l'annoncera.
\TextTitle{[Jésus parle de sa mort, de sa grandeur]}
\VS{16}Encore un peu de temps, et vous ne me verrez plus ; et après un peu de temps, vous me verrez, car je m'en vais à mon Père.
\VS{17}Quelques-uns de ses disciples dirent entre eux : Qu'est-ce qu'il nous dit : Encore un peu de temps, et vous ne me verrez plus ; et un peu de temps après, vous me verrez, car je m'en vais à mon Père ?
\VS{18}Ils disaient donc : Que signifient ces mots : Encore un peu de temps ? Nous ne comprenons pas ce qu'il dit.
\VS{19}Jésus sachant qu'ils voulaient l’interroger, leur dit : Vous vous demandez entre vous sur ce que j'ai dit : Encore un peu de temps, et vous ne me verrez plus, et un peu de temps après, vous me verrez.
\VS{20}En vérité, en vérité je vous le dis : Vous pleurerez et vous vous lamenterez, et le monde se réjouira ; vous serez, dis-je, attristés ; mais votre tristesse sera changée en joie.
\VS{21}La femme, lorsqu’elle enfante, éprouve de la tristesse, parce que son heure est venue ; mais, lorsqu’elle a donné le jour à l’enfant, elle ne se souvient plus de la souffrance, à cause de ce qu’un homme est né dans le monde.
\VS{22}Vous donc aussi, vous êtes maintenant dans la tristesse ; mais je vous reverrai encore, et votre cœur se réjouira, et personne ne vous ôtera votre joie.
\VS{23}En ce jour-là, vous ne m'interrogerez plus sur rien. En vérité, en vérité je vous le dis : Tout ce que vous demanderez au Père en mon Nom, il vous le donnera.
\VS{24}Jusqu'à présent vous n'avez rien demandé en mon Nom ; demandez, et vous recevrez, afin que votre joie soit parfaite.
\VS{25}Je vous ai dit ces choses en paraboles. Mais l'heure vient où je ne vous parlerai plus en paraboles ; mais je vous parlerai ouvertement de mon Père.
\VS{26}En ce jour-là, vous demanderez des grâces en mon Nom, et je ne vous dis pas que je prierai le Père pour vous ;
\VS{27}car le Père lui-même vous aime, parce que vous m'avez aimé, et que vous avez cru que je suis sorti de Dieu.
\VS{28}Je suis sorti du Père, et je suis venu dans le monde ; maintenant je quitte le monde, et je m'en vais au Père.
\VS{29}Ses disciples lui dirent : Voici, maintenant tu parles ouvertement, et tu n'uses plus de paraboles.
\VS{30}Maintenant nous savons que tu sais toutes choses\FTNT{Jésus est omniscient. Il s’est lui-même présenté à l’apôtre Jean comme celui qui est, qui était et qui sera (Ap. 1:7-8).}, et que tu n'as pas besoin que quelqu’un t'interroge ; à cause de cela nous croyons que tu es sorti de Dieu.
\VS{31}Jésus leur répondit : Croyez-vous maintenant ?
\VS{32}Voici, l'heure vient, et elle est déjà venue, où vous serez dispersés chacun de son côté, et vous me laisserez seul ; mais je ne suis pas seul, car le Père est avec moi.
\VS{33}Je vous ai dit ces choses afin que vous ayez la paix en moi. Vous aurez des tribulations dans le monde, mais prenez courage, j'ai vaincu le monde.
\TextTitle{[La prière d'intercession de Christ, le souverain sacrificateur]}
\Chap{17}
\VerseOne{}Après avoir ainsi parlé, Jésus leva ses yeux au ciel, et dit : Père, l'heure est venue, glorifie ton Fils, afin que ton Fils te glorifie ;
\VS{2}selon que tu lui as donné pouvoir sur tous les hommes ; afin qu'il donne la vie éternelle à tous ceux que tu lui as donnés.
\VS{3}Or, la vie éternelle, ce qu’ils te connaissent, toi, le seul vrai Dieu, et celui que tu as envoyé, Jésus-Christ.
\VS{4}Je t'ai glorifié sur la terre, j'ai achevé l’œuvre que tu m'avais donnée à faire.
\VS{5}Et maintenant glorifie-moi, toi Père, auprès de toi, de la gloire que j’avais auprès de toi avant que le monde soit.
\VS{6}J'ai fait connaître ton Nom aux hommes que tu m'as donnés du milieu du monde ; ils étaient à toi, et tu me les as donnés ; et ils ont gardé ta parole.
\VS{7}Maintenant ils ont connu que tout ce que tu m'as donné vient de toi.
\VS{8}Car je leur ai donné les paroles que tu m'as données, et ils les ont reçues, et ils ont vraiment connu que je suis sorti de toi, et ils ont cru que tu m'as envoyé.
\VS{9}C’est pour eux que je prie ; je ne prie pas pour le monde, mais pour ceux que tu m'as donnés, parce qu'ils sont à toi.
\VS{10}Et tout ce qui est à moi est à toi, et ce qui est à toi est à moi ; et je suis glorifié en eux.
\VS{11}Et maintenant je ne suis plus dans le monde, et ils sont dans le monde ; et moi je vais à toi. Père saint, garde en ton Nom ceux que tu m'as donnés, afin qu'ils soient un comme nous sommes un.
\VS{12}Quand j'étais avec eux dans le monde, je les gardais en ton Nom ; j'ai gardé ceux que tu m'as donnés, et aucun d'eux ne s’est perdu, sinon le fils de perdition, afin que l'Ecriture soit accomplie.
\VS{13}Et maintenant je vais à toi, et je dis ces choses étant encore dans le monde, afin qu'ils aient ma joie parfaite en eux-mêmes.
\VS{14}Je leur ai donné ta parole, et le monde les a haïs, parce qu'ils ne sont pas du monde, comme moi je ne suis pas du monde.
\VS{15}Je ne te prie pas de les ôter du monde, mais de les préserver du mal.
\VS{16}Ils ne sont pas du monde, comme moi je ne suis pas du monde.
\VS{17}Sanctifie-les par ta vérité ; ta parole est la vérité.
\VS{18}Comme tu m'as envoyé dans le monde, ainsi je les ai envoyés dans le monde.
\VS{19}Et je me sanctifie moi-même pour eux, afin qu'eux aussi soient sanctifiés par la vérité.
\VS{20}Je ne prie pas seulement pour eux, mais aussi pour ceux qui croiront en moi par leur parole.
\VS{21}Afin que tous soient un, ainsi que toi, Père, tu es en moi, et moi en toi ; afin qu'eux aussi soient un en nous ; et que le monde croie que c'est toi qui m'as envoyé.
\VS{22}Je leur ai donné la gloire que tu m'as donnée, afin qu'ils soient un comme nous sommes un.
\VS{23}Je suis en eux, et toi en moi, afin qu'ils soient parfaitement un, et que le monde connaisse que c'est toi qui m'as envoyé, et que tu les aimes, comme tu m'as aimé.
\VS{24}Père, mon désir est que ceux que tu m'as donnés soient avec moi là où je suis, afin qu'ils contemplent la gloire que tu m'as donnée ; parce que tu m'as aimé avant la fondation du monde.
\VS{25}Père juste, le monde ne t'a point connu ; mais moi je t'ai connu, et ceux-ci ont connu que c'est toi qui m'as envoyé.
\VS{26}Et je leur ai fait connaître ton Nom, et je le leur ferai connaître, afin que l'amour dont tu m'as aimé soit en eux, et que je sois en eux.
\TextTitle{[Jésus à Gethsémané]
\\(Mt. 26:36-46 ; Mc. 14:32-42 ; Lu. 22:39-46)}
\Chap{18}
\VerseOne{}Après que Jésus eut dit ces choses, il s'en alla avec ses disciples au-delà du torrent de Cédron, où il y avait un jardin dans lequel il entra avec ses disciples.
\TextTitle{[Jésus trahi et arrêté]
\\Mt. 26:47-56 ; Mc. 14:43-50 ; Lu. 22:47-54)}
\VS{2}Or Judas, qui le trahissait, connaissait aussi ce lieu-là, car Jésus s'y était souvent assemblé avec ses disciples.
\VS{3}Judas donc, ayant pris la cohorte, et des huissiers qu’envoyèrent les principaux sacrificateurs et les pharisiens, s'en vint là avec des lanternes, des flambeaux, et des armes.
\VS{4}Jésus, sachant tout ce qui devait lui arriver, s'avança et leur dit : Qui cherchez-vous ?
\VS{5}Ils lui répondirent : Jésus de Nazareth. Jésus leur dit : Moi, Je suis\FTNT{~Moi, Je suis~ (~ego eimi~), ce qui fait écho au nom sous lequel Dieu s’était révélé à Moïse en Ex. 3:14.}. Et Judas qui le trahissait était aussi avec eux.
\VS{6}Or après que Jésus leur eut dit : Moi Je suis, ils reculèrent, et tombèrent par terre.
\VS{7}Il leur demanda une seconde fois : Qui cherchez-vous ? Et ils répondirent : Jésus de Nazareth.
\VS{8}Jésus répondit : Je vous ai dit que moi, Je suis ; si donc vous me cherchez, laissez aller ceux-ci.
\VS{9}Il dit cela afin que s’accomplisse la parole qu'il avait dite : Je n'ai perdu aucun de ceux que tu m'as donnés.
\TextTitle{[Malchus frappé par Pierre]}
\VS{10}Simon Pierre, qui avait une épée, la tira, frappa le serviteur du souverain sacrificateur, et lui coupa l'oreille droite. Ce serviteur s’appelait Malchus.
\VS{11}Mais Jésus dit à Pierre : Remets ton épée au fourreau : Ne boirai-je pas la coupe que le Père m'a donnée ?
\TextTitle{[Jésus conduit auprès du souverain sacrificateur]
\\(v. 27 ; Mt. 26:57-68 ; Mc. 14:53-65 ; Lu. 22:63-71)}
\VS{12}La cohorte, le tribun, et les huissiers des Juifs se saisirent alors de Jésus et le lièrent.
\VS{13}Et ils l'emmenèrent premièrement chez Anne, car il était le beau-père de Caïphe, qui était le souverain sacrificateur de cette année-là.
\VS{14}Et Caïphe était celui qui avait donné ce conseil aux Juifs, qu'il était avantageux qu'un seul homme meure pour le peuple.
\TextTitle{[Le triple reniement de Pierre]
\\(v. 25-27 ; Mt. 26:69-75 ; Mc. 14:66-72 ; Lu. 22:54-62)}
\VS{15}Simon Pierre, avec un autre disciple, suivait Jésus ; et ce disciple était connu du souverain sacrificateur, et il entra avec Jésus dans la cour du souverain sacrificateur.
\VS{16}Mais Pierre était dehors à la porte, et l'autre disciple, qui était connu du souverain sacrificateur, sortit dehors et parla à la portière, et il fit entrer Pierre.
\VS{17}Et la servante qui était la portière dit à Pierre : N'es-tu pas aussi des disciples de cet homme ? Il dit : Je n'en suis point.
\VS{18}Les serviteurs et les huissiers qui étaient là avaient allumé un feu, parce qu'il faisait froid, et ils se chauffaient ; Pierre aussi était avec eux, et se chauffait.
\VS{19}Et le souverain sacrificateur interrogea Jésus sur ses disciples et sur sa doctrine.
\VS{20}Jésus lui répondit : J’ai ouvertement parlé au monde ; j'ai toujours enseigné dans la synagogue et dans le temple, où les Juifs s'assemblent toujours, et je n'ai rien dit en secret.
\VS{21}Pourquoi m'interroges-tu ? Interroge ceux qui ont entendu ce que je leur ai dit ; voici, ils savent ce que j'ai dit.
\VS{22}Quand il eut dit ces choses, un des huissiers qui se tenait là, donna un coup de sa verge à Jésus en lui disant : Est-ce ainsi que tu réponds au souverain sacrificateur ?
\VS{23}Jésus lui répondit : Si j'ai mal parlé, explique-moi ce que j’ai dit de mal ; et si j'ai bien parlé, pourquoi me frappes-tu ?
\VS{24}Anne l’envoya lié à Caïphe, le souverain sacrificateur.
\VS{25}Simon Pierre était là, et se chauffait. On lui dit : N’es-tu pas aussi de ses disciples ? Il le nia et dit : Je n'en suis point.
\VS{26}Un des serviteurs du souverain sacrificateur, parent de celui à qui Pierre avait coupé l'oreille, dit : Ne t'ai-je pas vu dans le jardin avec lui ?
\VS{27}Mais Pierre le nia de nouveau, et aussitôt le coq chanta.
\TextTitle{[Jésus devant Pilate]
\\(Mt. 27:2,11-14 ; Mc. 15:1-5 ; Lu. 23:1-7,13-15)}
\VS{28}Ils conduisirent Jésus de chez Caïphe au Prétoire\FTNT{Le Prétoire était à l'origine le nom du quartier général de la légion romaine. Il s’agissait plus particulièrement de la tente du général en chef d'une armée.} ; c'était le matin. Mais ils n'entrèrent point eux-mêmes dans le Prétoire, afin de ne pas se souiller, et de pouvoir manger l'agneau de Pâque.
\VS{29}C'est pourquoi Pilate\FTNT{Ponce Pilate était le préfet procurateur de la province romaine de Judée au Ier siècle (de 26 à 36).} sortit vers eux, et leur dit : Quelle accusation portez-vous contre cet homme ?
\VS{30}Ils lui répondirent : Si ce n'était pas un malfaiteur, nous ne te l’aurions pas livré.
\VS{31}Alors Pilate leur dit : Prenez-le vous-mêmes, et jugez-le selon votre loi. Mais les Juifs lui dirent : Il ne nous est pas permis de mettre quelqu’un à mort.
\VS{32}C’était afin que s’accomplisse la parole que Jésus avait dite, lorsqu’il indiquait de quelle mort il devait mourir.
\VS{33}Pilate entra de nouveau dans le Prétoire, et ayant appelé Jésus, il lui dit : Es-tu le Roi des Juifs ?
\VS{34}Jésus lui répondit : Est-ce de toi-même que tu dis cela, ou d’autres te l’ont dit de moi ?
\VS{35}Pilate répondit : Suis-je Juif ? Ta nation et les principaux sacrificateurs t'ont livré à moi ; qu'as-tu fait ?
\VS{36}Jésus répondit : Mon Royaume n'est pas de ce monde ; si mon Royaume était de ce monde, mes serviteurs auraient combattu pour moi afin que je ne sois pas livré aux Juifs ; mais maintenant mon règne n'est point d'ici-bas.
\VS{37}Alors Pilate lui dit : Es-tu donc Roi ? Jésus répondit : Tu le dis, que je suis Roi ; je suis né pour cela, et c'est pour cela que je suis venu dans le monde, pour rendre témoignage à la vérité. Quiconque est de la vérité entend ma voix.
\VS{38}Pilate lui dit : Qu'est-ce que la vérité ? Et quand il eut dit cela, il sortit de nouveau vers les Juifs, et il leur dit : Je ne trouve aucun crime en lui.
\TextTitle{[Barabbas libéré et Jésus condamné]
\\(Mt. 27:15-21 ; Mc. 15:6-11 ; Lu. 23:18-19)}
\VS{39}Or, comme c’est parmi vous une coutume que je vous relâche un prisonnier à la fête de Pâque ; voulez-vous donc que je vous relâche le Roi des Juifs ?
\VS{40}Et tous s'écrièrent, disant : Non pas celui-ci, mais Barrabas ; or Barrabas était un brigand.
\TextTitle{[Jésus couronné d'épines]
\\(Mt. 27:-30 ; Mc. 15:16-18)}
\Chap{19}
\VerseOne{}Alors Pilate prit Jésus, et le fit battre de verges.
\VS{2}Les soldats tressèrent une couronne d'épines qu'ils posèrent sur sa tête, et le vêtirent d'un vêtement de pourpre.
\VS{3}Puis ils lui disaient : Roi des Juifs, nous te saluons ; et ils lui donnaient des coups avec leurs verges.
\TextTitle{[Pilate fait un ultime effort pour relâcher Jésus]
\\(Mt. 27:22-26 ; Mc. 15:12-15 ; Lu. 23:20-25)}
\VS{4}Pilate sortit de nouveau dehors, et leur dit : Voici, je vous l'amène dehors, afin que vous sachiez que je ne trouve aucun crime en lui.
\VS{5}Jésus donc sortit portant la couronne d'épines et le manteau de pourpre ; et Pilate leur dit : Voici l'homme.
\VS{6}Mais quand les principaux sacrificateurs et leurs huissiers le virent, ils s'écrièrent, en disant : Crucifie ! Crucifie ! Pilate leur dit : Prenez-le vous-mêmes et crucifiez-le, car je ne trouve point de crime en lui.
\VS{7}Les Juifs lui répondirent : Nous avons une loi, et selon notre loi il doit mourir, car il s'est fait Fils de Dieu.
\VS{8}Quand Pilate entendit cette parole, sa frayeur augmenta.
\VS{9}Et il rentra dans le Prétoire et dit à Jésus : D'où es-tu ? Mais Jésus ne lui donna point de réponse.
\VS{10}Et Pilate lui dit : Est-ce à moi que tu ne parles pas ? Ne sais-tu pas que j'ai le pouvoir de te crucifier, et que j’ai le pouvoir de te délivrer ?
\VS{11}Jésus lui répondit : Tu n'aurais aucun pouvoir sur moi s'il ne t‘avait été donné d'en haut ; c'est pourquoi celui qui m'a livré à toi, commet un plus grand péché.
\VS{12}Dès ce moment, Pilate cherchait à le délivrer ; mais les Juifs criaient en disant : Si tu le délivres, tu n'es pas ami de César ; car quiconque se fait Roi est contre César.
\VS{13}Pilate, ayant entendu ces paroles, amena Jésus dehors, et il siégea au tribunal, au lieu appelé le Pavé, et en hébreu Gabbatha.
\VS{14}C’était la préparation de la Pâque, et environ la sixième heure ; et Pilate dit aux Juifs : Voici votre Roi.
\VS{15}Mais ils criaient : Ôte, ôte, crucifie-le ! Pilate leur dit : Crucifierai-je votre Roi ? Les principaux sacrificateurs répondirent : Nous n'avons pas d'autre roi que César.
\TextTitle{[Jésus crucifié]
\\(Mt. 27:31-50 ; Mc. 15:19-37 ; Lu. 23:26-46)}
\VS{16}Alors il le leur livra pour être crucifié. Ils prirent donc Jésus et l'emmenèrent.
\VS{17}Jésus, portant sa croix, arriva au lieu appelé le Crâne, et en hébreu Golgotha,
\VS{18}où ils le crucifièrent, et deux autres avec lui, un de chaque côté, et Jésus au milieu.
\VS{19}Pilate fit un écriteau, qu'il mit sur la croix, où étaient écrits ces mots : Jésus de Nazareth, le Roi des juifs.
\VS{20}Beaucoup des Juifs lurent cet écriteau, parce que le lieu où Jésus était crucifié, était près de la ville ; et cet écriteau était en hébreu, en grec et en latin.
\VS{21}C'est pourquoi les principaux sacrificateurs des Juifs dirent à Pilate : N'écris pas le Roi des Juifs, mais que celui-ci a dit : Je suis le Roi des Juifs.
\VS{22}Pilate répondit : Ce que j'ai écrit, je l'ai écrit.
\VS{23}Les soldats, après avoir crucifié Jésus, prirent ses vêtements, et ils en firent quatre parts, une part pour chaque soldat. Ils prirent aussi sa tunique, qui était sans couture, d’un seul tissu depuis le haut jusqu'en bas.
\VS{24}Ils se dirent entre eux : Ne la déchirons pas, mais tirons au sort, pour savoir à qui elle sera. Et cela arriva ainsi, afin que s’accomplisse cette parole de l’Ecriture : Ils ont partagé entre eux mes vêtements, et ils ont tiré au sort ma tunique\FTNT{Ps. 22:19.} ; ainsi firent les soldats.
\VS{25}Près de la croix de Jésus se tenaient sa mère, et la sœur de sa mère, Marie femme de Cléopas, et Marie de Magdala.
\VS{26}Jésus voyant sa mère, et auprès d'elle le disciple qu'il aimait, il dit à sa mère : Femme, voilà ton Fils.
\VS{27}Puis il dit au disciple : Voilà ta mère ; et dès ce moment, ce disciple la prit chez lui.
\VS{28}Après cela, Jésus sachant que toutes choses étaient déjà accomplies, il dit, afin que l'Ecriture soit accomplie : J'ai soif.
\VS{29}Et il y avait là un vase plein de vinaigre. Les soldats en remplirent une éponge et la mirent au bout d'une branche d'hysope, et la lui présentèrent à la bouche.
\VS{30}Quand Jésus eut pris le vinaigre, il dit : Tout est accompli\FTNT{La fin de la période de la première alliance n’a pas eu lieu à la naissance du Seigneur. En effet, Ga. 4:4 nous dit que Jésus est né sous la loi de Moïse et le récit des quatre évangiles atteste que depuis sa naissance jusqu’à sa mort, Jésus a scrupuleusement respecté et accompli toute la loi. En effet, il a lui-même dit : ~Ne croyez pas que je sois venu abolir la loi ou les prophètes ; je ne suis pas venu les abolir, mais les accomplir.~ (Mt. 5:17). Ainsi, durant son ministère terrestre, le Seigneur demandait à ce qu’on applique la loi (Mt. 8:4 ; Mt. 23:23 ; Lu. 17:11-14) tout en préparant ses disciples à la nouvelle alliance. L’évangile de Matthieu nous relate un événement capital qui a eu lieu juste après la mort du Seigneur : ~Alors Jésus, poussa de nouveau un grand cri, et rendit l'esprit. Et voici, le voile du temple se déchira en deux, depuis le haut jusqu'en bas ; et la terre trembla, et les pierres se fendirent.~ (Mt. 27:50-51). Il convient de rappeler que le temple était divisé en trois parties : le Parvis, le lieu Saint et le Saint des saints. Le Parvis était accessible à tout le monde, y compris aux non-Juifs. Le lieu Saint n’était accessible qu’aux lévites. La troisième partie, le Saint des saints, n’était accessible qu’au souverain sacrificateur. Le lieu Saint était séparé du Saint des saints par un voile qui symbolisait le mur d’inimitié (Es. 59:2 ; Ro. 3:23) qui sépare l’homme pécheur de la présence de Dieu, représentée dans le temple par l’arche de l’alliance. Ce voile n’avait rien d’un tissu léger et vaporeux, mais il ressemblait davantage à un épais tapis, opaque et surtout très résistant, et donc très difficile à déchirer. Le souverain sacrificateur rentrait seulement une fois par an dans le Saint des saints pour y offrir le sacrifice d’expiation pour le peuple ainsi que pour lui-même (Lé. 16 ; Hé. 9:7). Toutefois, la nécessité de répéter ce sacrifice chaque année prouvait que les exigences de la justice divine n’étaient pas pleinement satisfaites (Hé. 10:3-4). L’auteur de l’épître aux Hébreux nous apprend que le voile symbolisait également le corps physique de Christ (Hé. 10:19-20). Ainsi, lorsque le Seigneur a succombé à ses meurtrissures, le fameux voile s’est déchiré du haut jusqu’au bas. Or tant que le voile subsistait, l’accès à la présence de Dieu était fermé (Hé. 9:8). La déchirure atteste donc qu’en Christ, nous pouvons désormais nous approcher avec assurance du trône de Dieu, sans autre médiateur que le Seigneur lui-même (1 Ti. 2:5). ~Or là où les péchés sont pardonnés, il n'y a plus d’offrande pour le péché. Ainsi donc, mes frères, nous avons la liberté d'entrer dans le Saint des saints au moyen du sang de Jésus, qui est le chemin nouveau et vivant qu'il nous a frayé au travers du voile, c’est-à-dire de sa chair. Et ayant un Souverain Sacrificateur établi sur la maison de Dieu, approchons-nous de lui avec un cœur sincère et une foi inébranlable, ayant les cœurs purifiés d’une mauvaise conscience, et le corps lavé d’une eau pure. Retenons sans fléchir la profession de notre espérance, car celui qui nous a fait la promesse est fidèle.~ Hé. 10:18-23. Jésus-Christ est notre Pâque (1 Co. 5:5-8), il est le sacrifice parfait qui a expié nos péchés une fois pour toutes (Hé. 10:10). Par conséquent, il est celui à qui nous devons nous adresser pour recevoir pardon, miséricorde et compassion. ~Tout est accompli~. En s’écriant de la sorte, Jésus-Christ a proclamé la fin de l'ancienne alliance. En effet, la loi a été promulguée par Moïse, mais la grâce et la vérité sont venues par Jésus-Christ (Jn. 1:17). Toutefois, la nouvelle alliance n’a réellement débuté qu’à la Pentecôte avec l’effusion du Saint-Esprit. Voir commentaire en Actes 2.} ; et ayant baissé la tête, il rendit l'esprit.
\TextTitle{[Fin de la première Alliance]
\\(Mt.27:50-51 ; Mc. 15:37-38 ; Lu. 23:45-46)}
\VS{31}De peur que les corps ne restent sur la croix pendant le sabbat, car c’était la préparation, et ce jour de sabbat était un grand jour, les Juifs demandèrent à Pilate qu’on rompe les jambes aux crucifiés, et qu’on les enlève.
\VS{32}Les soldats vinrent donc, et ils rompirent les jambes au premier, et de même à l'autre qui était crucifié avec lui.
\VS{33}Puis étant venus à Jésus, et voyant qu'il était déjà mort, ils ne lui rompirent point les jambes ;
\VS{34}mais un des soldats lui perça le côté avec une lance, et aussitôt il sortit du sang et de l'eau.
\VS{35}Celui qui l'a vu en a rendu témoignage, et son témoignage est digne de foi ; et il sait qu'il dit vrai, afin que vous le croyiez.
\VS{36}Ces choses sont arrivées, afin que l’Ecriture soit accomplie : Aucun de ses os ne sera brisé\FTNT{Ps. 34:21 ; Ex. 12:46 ; No 9:12.}.
\VS{37}Et encore une autre Ecriture, qui dit : Ils verront celui qu'ils ont percé\FTNT{Za. 12:10.}.
\TextTitle{[Jésus enseveli]
\\(Mt. 27:57-66 ; Mc. 15:42-47 ; Lu. 23:50-56)}
\VS{38}Après ces choses, Joseph d'Arimathée, qui était disciple de Jésus, mais en secret parce qu'il craignait les Juifs, demanda à Pilate la permission d’enlever le corps de Jésus ; et Pilate le lui ayant permis, il vint et prit le corps de Jésus.
\VS{39}Nicodème, qui auparavant était allé de nuit vers Jésus, vint aussi, apportant un mélange de myrrhe et d'aloès d'environ cent livres.
\VS{40}Et ils prirent le corps de Jésus, et l'enveloppèrent de linges avec des aromates, comme les Juifs ont coutume d'ensevelir.
\VS{41}Or il y avait un jardin dans le lieu où Jésus fut crucifié, et dans le jardin un sépulcre neuf, où personne n'avait encore été mis.
\VS{42}Ce fut là qu’ils déposèrent Jésus, à cause de la préparation des Juifs, parce que le sépulcre était proche.
\TextTitle{[Déroulement des événements du jour de la résurection]
\\(Mt. 28:1-15 ; Mc. 16:1-14 ; Lu. 24:1-32)}
\Chap{20}
\VerseOne{}Le premier jour de la semaine, Marie de Magdala se rendit dès le matin au sépulcre, comme il faisait encore obscur ; et elle vit que la pierre était ôtée du sépulcre.
\VS{2}Elle courut vers Simon Pierre et vers l'autre disciple que Jésus aimait, et elle leur dit : Ils ont enlevé le Seigneur du sépulcre, et nous ne savons pas où ils l’ont mis.
\VS{3}Alors Pierre partit avec l'autre disciple, et ils s'en allèrent au sépulcre.
\VS{4}Ils couraient tous deux ensemble, mais l'autre disciple courait plus vite que Pierre, et il arriva le premier au sépulcre.
\VS{5}Et s'étant baissé, il vit les linges à terre ; mais il n'y entra point.
\VS{6}Alors Simon Pierre qui le suivait, arriva, et entra dans le sépulcre, et vit les linges à terre,
\VS{7}et le linge qu’on avait mis sur la tête de Jésus, non pas avec les bandes, mais plié dans un lieu à part.
\VS{8}Alors l'autre disciple, qui était arrivé le premier au sépulcre, entra aussi, il vit, et il crut.
\VS{9}Car ils ne comprenaient pas encore que, selon l'Ecriture, Jésus devait ressusciter des morts.
\VS{10}Et les disciples s'en retournèrent chez eux.
\TextTitle{[Jésus apparait aux disciples, Thomas étant absent]
\\(Mc. 16:14 ; Lu. 24:33-49)}
\VS{11}Mais Marie se tenait près du sépulcre dehors, et pleurait. Comme elle pleurait, elle se baissa dans le sépulcre,
\VS{12}et elle vit deux anges vêtus de blanc, assis à la place où avait été couché le corps de Jésus, l'un à la tête et l'autre aux pieds.
\VS{13}Ils lui dirent : Femme, pourquoi pleures-tu ? Elle leur dit : Parce qu'on a enlevé mon Seigneur, et je ne sais point où on l'a mis.
\VS{14}En disant cela, elle se retourna, et elle vit Jésus qui était là, mais elle ne savait pas que c’était Jésus.
\VS{15}Jésus lui dit : Femme, pourquoi pleures-tu ? Qui cherches-tu ? Elle, pensant que c’était le jardinier, lui dit : Seigneur, si c’est toi qui l'as emporté, dis-moi où tu l'as mis, et je le prendrai.
\VS{16}Jésus lui dit : Marie ! Et elle se retourna et lui dit : Rabbouni ! C’est-à-dire, mon Maître !
\VS{17}Jésus lui dit : Ne me touche pas ; car je ne suis point encore monté vers mon Père. Mais va trouver mes frères, et dis-leur que je monte vers mon Père et votre Père, vers mon Dieu et votre Dieu.
\VS{18}Marie de Magdala alla annoncer aux disciples qu'elle avait vu le Seigneur, et qu'il lui avait dit ces choses.
\VS{19}Le soir de ce jour, qui était le premier de la semaine, les portes du lieu où les disciples étaient assemblés, à cause de la crainte qu'ils avaient des Juifs, étaient fermées. Jésus vint, se présenta au milieu d'eux, et il leur dit : Que la paix soit avec vous !
\VS{20}Et quand il leur eut dit cela, il leur montra ses mains et son côté. Les disciples furent dans la joie en voyant le Seigneur.
\VS{21}Jésus leur dit de nouveau : Que la paix soit avec vous ! Comme mon Père m'a envoyé, ainsi je vous envoie.
\VS{22}Après ces paroles, il souffla sur eux, et leur dit : Recevez le Saint-Esprit.
\VS{23}Ceux à qui vous pardonnerez les péchés, ils leur seront pardonnés ; et ceux à qui vous les retiendrez, ils leur seront retenus.
\TextTitle{[Jésus apparait aux disciples, Thomas étant présent]}
\VS{24}Thomas, appelé Didyme, l'un des douze, n'était pas avec eux quand Jésus vint.
\VS{25}Les autres disciples lui dirent : Nous avons vu le Seigneur. Mais il leur dit : Si je ne vois pas les marques des clous dans ses mains, et si je ne mets pas mon doigt où étaient les clous, et si je ne mets pas ma main dans son côté, je ne le croirai point.
\VS{26}Huit jours après, les disciples étaient de nouveau dans la maison, et Thomas se trouvait avec eux. Jésus vint, les portes étant fermées, se présenta au milieu d'eux, et il leur dit : Que la paix soit avec vous !
\VS{27}Puis il dit à Thomas : Mets ton doigt ici, et regarde mes mains, avance aussi ta main, et mets-la dans mon côté ; et ne sois point incrédule, mais crois.
\VS{28}Et Thomas répondit et lui dit : Mon Seigneur, et mon Dieu !
\VS{29}Jésus lui dit : Parce que tu m'as vu, Thomas, tu as cru. Heureux sont ceux qui n'ont pas vu et qui ont cru.
\TextTitle{[But de l'Evangile selon Jean]}
\VS{30}Jésus fit encore, en présence de ses disciples, beaucoup d’autres miracles qui ne sont pas écrits dans ce livre.
\VS{31}Mais ces choses sont écrites afin que vous croyiez que Jésus est le Christ, le Fils de Dieu, et qu'en croyant vous ayez la vie par son Nom.
\TextTitle{[Jésus apparait à sept apôtres au bord de la mer de Galilée]}
\Chap{21}
\VerseOne{}Après cela, Jésus se montra de nouveau à ses disciples, près de la mer de Tibériade. Et voici de quelle manière il se montra.
\VS{2}Simon Pierre, Thomas, appelé Didyme, Nathanaël, de Cana en Galilée, les fils de Zébédée, et deux autres disciples de Jésus étaient ensemble.
\TextTitle{[Christ et notre service :
\\a. Le service de la volonté propre, sous des directives humaines]}
\VS{3}Simon Pierre leur dit : Je vais pêcher. Ils lui dirent : Nous allons aussi avec toi. Ils partirent et montèrent dans une barque ; mais ils ne prirent rien cette nuit-là.
\VS{4}Le matin étant venu, Jésus se trouva sur le rivage ; mais les disciples ne savaient pas que c’était Jésus.
\TextTitle{[b. Inutilité du service de la volonté propre]}
\VS{5}Jésus leur dit : Mes enfants, avez-vous quelque petit poisson à manger ? Ils lui répondirent : Non.
\TextTitle{[c. Résultat du service sous les directives de Christ]}
\VS{6}Et il leur dit : Jetez le filet du côté droit de la barque, et vous trouverez. Ils le jetèrent donc, et ils ne pouvaient plus le retirer à cause de la grande quantité de poissons.
\VS{7}Alors le disciple que Jésus aimait dit à Pierre : C’est le Seigneur. Et quand Simon Pierre eut entendu que c'était le Seigneur, il mit sa tunique et sa ceinture parce qu'il était nu, et il se jeta dans la mer.
\VS{8}Les autres disciples vinrent dans la barque, car ils n'étaient pas loin de terre, mais seulement à environ deux cents coudées, traînant le filet de poissons.
\VS{9}Lorsqu’ils furent descendus à terre, ils virent de la braise, et du poisson dessus, et du pain.
\VS{10}Jésus leur dit : Apportez des poissons que vous venez maintenant de prendre.
\VS{11}Simon Pierre monta et tira le filet à terre, plein de cent cinquante-trois grands poissons ; et quoiqu'il y en eût tant, le filet ne se rompit point.
\TextTitle{[d) Les ressources de Christ pour ses serviteurs]
\\(Lu. 22:35 ; Ph. 4:19)}
\VS{12}Jésus leur dit : Venez et mangez. Et aucun de ses disciples n'osait lui demander : Qui es-tu ? Sachant que c'était le Seigneur.
\VS{13}Jésus donc vint, et prit du pain, et leur en donna ; il fit de même du poisson aussi.
\VS{14}C’était déjà la troisième fois que Jésus se montrait à ses disciples depuis qu’il était ressuscité des morts.
\TextTitle{[e) La charité, seul motif valable pour le vrai service]
\\(1 Co. 13 ; 2 Co. 5:14 ; Ap. 2:4-5)}
\VS{15}Après qu'ils eurent mangé, Jésus dit à Simon Pierre : Simon fils de Jonas, m'aimes-tu plus que ne m’aiment ceux-ci ? Il lui répondit : Oui, Seigneur ! Tu sais que je t'aime. Il lui dit : Pais mes agneaux.
\VS{16}Il lui dit encore : Simon fils de Jonas, m'aimes-tu ? Il lui répondit : Oui, Seigneur ! Tu sais que je t'aime. Il lui dit : Pais mes brebis.
\VS{17}Il lui dit pour la troisième fois : Simon fils de Jonas, m'aimes-tu ? Pierre fut attristé de ce qu'il lui avait dit pour la troisième fois : M'aimes-tu ? Et il lui répondit : Seigneur, tu sais toutes choses, tu sais que je t'aime. Jésus lui dit : Pais mes brebis.
\TextTitle{[f) Le Maître révèle à Pierre qu'il Lui appartient de fixer le temps et la forme de sa mort]}
\VS{18}En vérité, en vérité je te le dis : Quand tu étais plus jeune, tu te ceignais toi-même, et tu allais où tu voulais ; mais quand tu seras vieux, tu étendras tes mains, et un autre te ceindra, et te mènera où tu ne voudras pas.
\VS{19}Il dit cela pour indiquer par quelle mort Pierre glorifierait Dieu. Et ayant ainsi parlé, il lui dit : Suis-moi.
\TextTitle{[g) Tous ses serviteurs ne mourront pas]
\\(1 Co. 15:51-52 ; 1 Th. 4:14-18)}
\VS{20}Pierre se retournant, vit venir après eux le disciple que Jésus aimait, celui qui pendant le souper s'était penché sur la poitrine de Jésus et avait dit : Seigneur, qui est celui qui te trahit ?
\VS{21}Quand donc Pierre le vit, il dit à Jésus : Seigneur, et celui-ci, que lui arrivera-t-il ?
\VS{22}Jésus lui dit : Si je veux qu'il demeure jusqu'à ce que je vienne, que t'importe ? Toi, suis-moi.
\VS{23}Là-dessus, le bruit courut parmi les frères que ce disciple ne mourrait point. Cependant Jésus ne lui avait pas dit : Il ne mourra point ; mais : Si je veux qu'il demeure jusqu'à ce que je vienne, que t'importe ?
\VS{24}C'est ce disciple qui rend témoignage de ces choses, et qui les a écrites. Et nous savons que son témoignage est digne de foi.
\VS{25}Jésus a fait encore beaucoup d’autres choses. Si on les écrivait en détail, je ne pense pas que le monde même pourrait contenir les livres qu’on écrirait. AMEN !
\PPE{}
\end{multicols}
\clearpage
\addcontentsline{toc}{chapter}{Testament de Jésus}\clearpage
\clearpage\begin{center}
{\LARGE Testament de Jésus-Christ}
\end{center}

\clearpage\ShortTitle{Ac.}\BookTitle{Actes}\BFont
\noindent\hrulefill
{\footnotesize
\textit{
\bigskip
{\centering{}
\\Auteur~: Luc
\\Thème~: Les missions du 1er siècle
\\Date de rédaction~: Env. 60 ap. J.-C.\\}
}
\textit{
\\D'origine grecque, Luc fut l'auteur du livre communément appelé «~actes des apôtres~» et de l'évangile éponyme tous deux adressés à Théophile. Ce livre retrace la genèse de l'Eglise, de l'ascension de Jésus à la Pentecôte, de la prédication vivante et fructueuse de Pierre à la conversion de Paul, jusqu'au voyage de celui-ci à Rome en tant que prisonnier. On y découvre des apôtres déterminés, des ouvriers de Christ qui acceptèrent de subir l'humiliation et la persécution par amour de la vérité. Sont également présentés des hommes et des femmes qui - touchés par la simplicité de l'Evangile du Royaume - se convertirent puis se firent baptiser.
\\Bien plus qu'un recueil relatant de banales manifestations, ce livre est avant tout celui des actes du Saint-Esprit. Il témoigne de la résurrection et de la puissance de Jésus-Christ manifestée au travers de son Corps. Il retrace l'origine et le développement du premier réveil après Jésus-Christ, qui fut un véritable bouleversement au sein d'un empire en proie à l'impiété et à l'idolâtrie.\bigskip
}
}
\par\nobreak\noindent\hrulefill
\begin{multicols}{2}
\Chap{1}
\TextTitle{Introduction~: le Messie ressuscité parle des choses qui concernent le Royaume de Dieu pendant quarante jours}
\VerseOne{}Nous avons rempli le premier traité, ô Théophile~! De toutes les choses que Jésus a faites et enseignées, 
\VS{2}jusqu'au jour où il fut élevé au ciel, après avoir donné par le Saint-Esprit, ses ordres aux apôtres qu'il avait élus.
\VS{3}A qui aussi, après avoir souffert, il se présenta lui-même vivant, avec plusieurs preuves assurées, étant vu par eux pendant quarante jours, et leur parlant des choses qui concernent le Royaume de Dieu.
\VS{4}Et les ayant assemblés, il leur ordonna de ne pas partir de Jérusalem, mais d'attendre la promesse du Père, ce que vous avez entendu de moi~;
\VS{5}car Jean a baptisé d'eau, mais vous serez baptisés du Saint-Esprit dans peu de jours.
\VS{6}Eux donc étant assemblés, l'interrogèrent, disant~: Seigneur, est-ce en ce temps-ci que tu rétabliras le royaume d'Israël~?
\VS{7}Mais il leur dit~: Ce n'est pas à vous de connaître les temps et les moments que le Père a fixés de sa propre autorité.
\TextTitle{La puissance du Saint-Esprit pour évangéliser les nations\FTNTT{\vref{Mt. 28:18-20}~; \vref{Mc. 16:15-18}~; \vref{Lu. 24:47-48}~; \vref{Jn. 20:21-22}.}}
\VS{8}Mais vous recevrez la puissance du Saint-Esprit qui viendra sur vous, et vous serez mes témoins, tant à Jérusalem que dans toute la Judée, et la Samarie, et jusqu'aux extrémités de la terre.
\VS{9}Et ayant dit ces choses, il fut élevé, comme ils le regardaient, une nuée le prit et l'emporta de devant leurs yeux.
\TextTitle{Promesse du retour de Jésus}
\VS{10}Et comme ils avaient les yeux fixés vers le ciel, à mesure qu'il s'en allait, voici, deux hommes en vêtements blancs se présentèrent devant eux,
\VS{11}et leur dirent~: Hommes Galiléens, pourquoi vous arrêtez-vous à regarder au ciel~? Ce Jésus qui a été élevé du milieu de vous au ciel, en descendra de la même manière que vous l'avez contemplé montant au ciel\FTNT{Jésus-Christ est monté au ciel depuis la Montagne des Oliviers et lors de son retour, ses pieds se poseront sur cette montagne. Voir \vref{Za. 14}.}.
\TextTitle{Attente du Saint-Esprit promis}
\VS{12}Alors ils s'en retournèrent à Jérusalem de la montagne appelée la Montagne des Oliviers, qui est près de Jérusalem, le chemin d'un sabbat\FTNT{Chemin de sabbat~: C'est la distance qu'il est permis à un juif de parcourir le jour de sabbat (\vref{Ex. 16:29}). Elle correspond à deux mille coudées ou 1100 m.}.
\VS{13}Et quand ils furent entrés dans la ville, ils montèrent dans une chambre haute où demeuraient Pierre et Jacques, Jean et André, Philippe et Thomas, Barthélemy et Matthieu, Jacques, fils d'Alphée, et Simon le zélote, et Jude, frère de Jacques.
\VS{14}Tous ceux-ci, d'un commun accord, persévéraient dans la prière et dans la supplication avec les femmes, avec Marie, mère de Jésus, et avec ses frères.
\TextTitle{Matthias désigné apôtre pour remplacer Judas}
\VS{15}Et en ces jours-là, Pierre se leva au milieu des disciples, qui étaient là assemblés au nombre d'environ cent vingt personnes, et il leur dit~:
\VS{16}Hommes frères, il fallait que s'accomplisse ce qui a été écrit, ce que le Saint-Esprit a annoncé d'avance par la bouche de David, au sujet de Judas, qui a été le guide de ceux qui ont saisi Jésus.
\VS{17}Car il était compté parmi nous, et il avait reçu en partage ce même service.
\VS{18}Mais après avoir acquis un champ avec le salaire du crime qui lui avait été donné, il est tombé, s'est rompu par le milieu, et toutes ses entrailles ont été répandues.
\VS{19}Et ceci a été connu de tous les habitants de Jérusalem, de sorte que ce champ a été appelé dans leur propre langue Hakeldama, c'est-à-dire le champ du sang.
\VS{20}Car il est écrit dans le livre des Psaumes~: Que sa demeure soit déserte, que personne ne l'habite\FTNT{\vref{Ps. 69:26}.}, et qu'un autre prenne sa charge\FTNT{Charge~: Du grec «~episkope~», il s'agit de la fonction d'un ancien. \vref{Ps. 109:8}.}.
\VS{21}Il faut donc que d'entre ces hommes qui se sont assemblés avec nous pendant tout le temps que le Seigneur Jésus a vécu entre nous,
\VS{22}en commençant depuis le baptême de Jean jusqu'au jour où il a été enlevé du milieu de nous, qu'il y en ait un qui soit témoin avec nous de sa résurrection.
\VS{23}Et ils en présentèrent deux~: Joseph, appelé Barsabbas, surnommé Justus, et Matthias.
\VS{24}Et en priant, ils dirent~: Toi, Seigneur, qui connais les cœurs de tous, désigne lequel de ces deux tu as choisi,
\VS{25}afin qu'il prenne part à ce service et à cet apostolat que Judas a abandonné pour aller en son lieu.
\VS{26}Puis ils les tirèrent au sort, et le sort tomba sur Matthias, qui, d'une commune voix, fut mis au rang des onze apôtres.
\Chap{2}
\TextTitle{Effusion de l'Esprit à la Pentecôte~; naissance de l'Eglise\FTNT{\vref{Joë. 2:32}.}}
\VerseOne{}Et comme le jour de la Pentecôte s'accomplissait, ils étaient tous ensemble dans un même lieu.
\VS{2}Et il se fit tout à coup un bruit du ciel, comme est le bruit d'un vent qui souffle avec véhémence, et il remplit toute la maison où ils étaient assis.
\VS{3}Et il leur apparut des langues divisées, comme de feu, qui se posèrent sur chacun d'eux.
\VS{4}Et ils furent tous remplis du Saint-Esprit, et commencèrent à parler des langues étrangères selon que l'Esprit leur donnait de parler.
\VS{5}Or il y avait à Jérusalem des Juifs qui y séjournaient, hommes pieux, de toute nation qui est sous le ciel.
\VS{6}Et ce bruit s'étant répandu, une multitude vint ensemble, et fut confondue de ce que chacun les entendait parler dans sa propre langue. 
\VS{7}Ils en étaient donc tout surpris, et s'en étonnaient, disant l'un à l'autre~: Voici, tous ceux-ci qui parlent ne sont-ils pas Galiléens~?
\VS{8}Comment donc chacun de nous les entendons-nous parler la propre langue du pays où nous sommes nés~? 
\VS{9}Parthes, Mèdes, Elamites, et ceux qui habitent la Mésopotamie, la Judée, la Cappadoce, le Pont, l'Asie,
\VS{10}la Phrygie, la Pamphylie, l'Egypte, le territoire de la Libye qui est près de Cyrène, et ceux qui sont venus de Rome~? Juifs et Prosélytes,
\VS{11} Crétois et Arabes, comment les entendons-nous parler chacun dans notre langue des merveilles de Dieu~?
\VS{12}Ils étaient donc tout étonnés, et ils ne savaient que penser, disant l'un à l'autre~: Que veut dire ceci~?
\VS{13}Mais les autres se moquaient, et disaient~: C'est qu'ils sont pleins de vin doux.
\TextTitle{Prédication de Pierre}
\VS{14}Alors Pierre, se présentant avec les onze, éleva sa voix, et leur dit~: Hommes Juifs, et vous tous qui habitez à Jérusalem, apprenez ceci, et faites attention à mes paroles~!
\VS{15}Ces gens ne sont pas ivres, comme vous le pensez, car c'est la troisième heure\FTNT{Neuf heures du matin.} du jour.
\VS{16}Mais c'est ici ce qui a été dit par le prophète Joël~:
\VS{17}Et il arrivera dans les derniers jours, dit Dieu, que je répandrai de mon Esprit sur toute chair~; vos fils et vos filles prophétiseront, vos jeunes gens auront des visions, et vos vieillards songeront des songes.
\VS{18}Et dans ces jours-là je répandrai de mon Esprit sur mes serviteurs et sur mes servantes, et ils prophétiseront.
\VS{19}Et je ferai des choses merveilleuses en haut dans le ciel, et des prodiges en bas sur la terre, du sang, du feu, et une vapeur de fumée.
\VS{20}Le soleil se changera en ténèbres, et la lune en sang, avant que ce grand et notable jour du Seigneur vienne.
\VS{21}Mais il arrivera que quiconque invoquera le Nom du Seigneur sera sauvé\FTNT{\vref{Joë. 2:28-32}.}.
\TextTitle{Proclamation de la résurrection du Messie}
\VS{22}Hommes Israélites, écoutez ces paroles~! Jésus de Nazareth, homme approuvé de Dieu parmi vous par les miracles et les prodiges et les signes que Dieu a faits par lui au milieu de vous, comme vous-mêmes vous le savez,
\VS{23}ayant été livré selon le dessein arrêté et selon la prescience de Dieu, vous l'avez pris et mis à la croix, vous l'avez fait mourir par les mains des impies.
\VS{24}Mais Dieu l'a ressuscité, ayant brisé les liens de la mort, parce qu'il n'était pas possible qu'il soit retenu par elle.
\VS{25}Car David dit de lui~: Je contemplais constamment le Seigneur devant moi, parce qu'il est à ma droite, afin que je ne sois point ébranlé\FTNT{\vref{Ps. 16:8-11}.}.
\VS{26}C'est pourquoi mon cœur est dans la joie, et ma langue dans l'allégresse~; et de plus, ma chair reposera avec espérance.
\VS{27}Car tu ne laisseras point mon âme en enfer\FTNT{Voir commentaire en \vref{Mt. 16:18}.} et tu ne permettras point que ton Saint voie la corruption.
\VS{28}Tu m'as fait connaître le chemin de la vie, tu me rempliras de joie dans ta présence\FTNT{\vref{Ps. 16:11}.}.
\VS{29}Hommes frères, qu'il me soit permis de vous dire librement, au sujet du patriarche David, qu'il est mort, qu'il a été enseveli, et que son sépulcre existe encore parmi nous jusqu'à ce jour.
\VS{30}Mais comme il était prophète, et qu'il savait que Dieu lui avait promis avec serment, que du fruit de ses reins il ferait naître selon la chair le Christ, pour le faire asseoir sur son trône~;
\VS{31}c'est la résurrection du Christ qu'il a prévue et annoncée, en disant qu'il ne serait pas abandonné en enfer et que sa chair ne verrait pas la corruption.
\VS{32}Dieu a ressuscité ce Jésus~; nous en sommes tous témoins.
\VS{33}Après donc qu'il a été élevé au ciel par la puissance de Dieu, et qu'il a reçu de son Père la promesse du Saint-Esprit, il a répandu ce que maintenant vous voyez et ce que vous entendez.
\VS{34}Car David n'est pas monté au ciel~; mais lui-même dit~: Le Seigneur a dit à mon Seigneur~: Assieds-toi à ma droite,
\VS{35}jusqu'à ce que j'aie mis tes ennemis pour le marchepied de tes pieds\FTNT{\vref{Ps. 110:1}.}. 
\VS{36}Que toute la maison d'Israël sache donc avec certitude que Dieu a fait Seigneur et Christ, ce Jésus, dis-je, que vous avez crucifié.
\TextTitle{Exhortation à la repentance}
\VS{37}Après avoir entendu ces choses, ils eurent le cœur touché de componction\FTNT{Componction~: Tristesse produite par les effets du repentir, le regret d'avoir offensé Dieu.}, et ils dirent à Pierre et aux autres apôtres~: Hommes frères, que ferons-nous~?
\VS{38}Et Pierre leur dit~: Repentez-vous, et que chacun de vous soit baptisé au Nom de Jésus-Christ, pour obtenir le pardon de vos péchés, et vous recevrez le don du Saint-Esprit.
\VS{39}Car à vous et à vos enfants est faite la promesse, et à tous ceux qui sont loin, autant que le Seigneur, notre Dieu en appellera à lui.
\VS{40}Et par plusieurs autres paroles, il les conjurait et les exhortait, en disant~: Sauvez-vous de cette génération perverse.
\TextTitle{Conversion et baptême de trois mille personnes~; les débuts de l'Eglise}
\VS{41}Ceux donc qui reçurent de bon cœur sa parole, furent baptisés~; et en ce jour-là furent ajoutées à l'Eglise environ trois mille âmes.
\VS{42}Et ils persévéraient tous dans la doctrine des apôtres, dans la communion fraternelle, dans la fraction du pain, et dans les prières.
\VS{43}Et tout le monde avait de la crainte, et beaucoup de miracles et de prodiges se faisaient par les apôtres.
\VS{44}Tous ceux qui croyaient étaient ensemble dans le même lieu, et ils avaient tout en commun~;
\VS{45}et ils vendaient leurs possessions et leurs biens, et les distribuaient à tous, selon les besoins de chacun.
\VS{46}Et tous les jours, ils persévéraient tous d'un commun accord dans le temple~; et rompant le pain de maison en maison, ils prenaient leur repas avec joie et simplicité de cœur~;
\VS{47}louant Dieu et se rendant agréables à tout le peuple. Et le Seigneur ajoutait tous les jours à l'Eglise des gens pour être sauvés.
\Chap{3}
\TextTitle{Guérison d'un homme boiteux de naissance}
\VerseOne{}Et comme Pierre et Jean montaient ensemble au temple à l'heure de la prière~; c'était la neuvième heure.
\VS{2}Et il y avait un homme boiteux de naissance, qu'on portait, et qu'on mettait tous les jours à la porte du temple, appelée la Belle, pour demander l'aumône à ceux qui entraient dans le temple. 
\VS{3}Cet homme voyant Pierre et Jean qui allaient entrer au temple, les pria de lui donner l'aumône.
\VS{4}Alors Pierre, de même que Jean, fixa les yeux sur lui, et lui dit~: Regarde-nous.
\VS{5}Et il les regardait attentivement, s'attendant de recevoir quelque chose d'eux.
\VS{6}Mais Pierre lui dit~: Je n'ai ni argent, ni or~; mais ce que j'ai, je te le donne~: Au Nom de Jésus-Christ de Nazareth, lève-toi et marche.
\VS{7}Et l'ayant pris par la main droite, il le fit lever~; et aussitôt les plantes et les chevilles de ses pieds devinrent fermes.
\VS{8}Et faisant un saut, il se tint debout, et marcha~; et il entra avec eux au temple, marchant, sautant, et louant Dieu.
\VS{9}Et tout le peuple le vit marchant et louant Dieu.
\VS{10}Et reconnaissant que c'était celui-là même qui était assis à la Belle, porte du temple, pour avoir l'aumône, ils furent remplis d'admiration et d'étonnement de ce qui lui était arrivé.
\VS{11}Et comme le boiteux, qui avait été guéri, tenait par la main Pierre et Jean, tout le peuple étonné accourut vers eux, au portique qu'on appelle de Salomon.
\TextTitle{Christ, le Messie annoncé par les prophètes}
\VS{12}Mais Pierre voyant cela, dit au peuple~: Hommes Israélites, pourquoi vous étonnez-vous de ceci~? Ou pourquoi avez-vous les regards fixés sur nous, comme si par notre puissance ou par notre piété, nous avions fait marcher cet homme~?
\VS{13}Le Dieu d'Abraham, d'Isaac, et de Jacob, le Dieu de nos pères, a glorifié son Fils Jésus, que vous avez livré et renié devant Pilate, quoiqu'il jugeât qu'il devait être relâché.
\VS{14}Mais vous avez renié le Saint et le Juste, et vous avez demandé qu'on vous relâche un meurtrier.
\VS{15}Vous avez fait mourir le Prince de la vie, que Dieu a ressuscité des morts~; nous en sommes témoins.
\VS{16}C'est par la foi en son Nom, que son Nom a raffermi les pieds de cet homme que vous voyez et connaissez. La foi, dis-je, que nous avons en lui, a donné à cet homme cette entière guérison de tous ses membres, en présence de vous tous.
\VS{17}Et maintenant, mes frères, je sais que vous avez agi par ignorance, de même que vos chefs.
\VS{18}Mais Dieu a ainsi accompli les choses qu'il avait prédites par la bouche de tous ses prophètes, que le Christ devait souffrir\FTNT{\vref{Es. 53}.}.
\VS{19}Repentez-vous donc, et convertissez-vous, afin que vos péchés soient effacés~;
\VS{20}afin que des temps de rafraîchissement viennent par la présence du Seigneur, et qu'il envoie celui qui vous a été auparavant annoncé, Jésus-Christ,
\VS{21}lequel il faut que le ciel reçoive, jusqu'au temps du rétablissement de toutes les choses que Dieu a prononcées par la bouche de tous ses saints prophètes, dès le commencement du monde.
\VS{22}Car Moïse lui-même a dit à nos pères~: Le Seigneur votre Dieu, vous suscitera d'entre vos frères un Prophète comme moi~; vous l'écouterez dans tout ce qu'il vous dira,
\VS{23}et il arrivera que toute personne qui n'aura pas écouté ce Prophète, sera exterminé du milieu du peuple\FTNT{\vref{De. 18:15-19}.}.
\VS{24}Et même tous les Prophètes depuis Samuel, et ceux qui l'ont suivi, tout autant qu'il y en a eu qui ont parlé, ont aussi prédit ces jours.
\VS{25}Vous êtes les enfants des prophètes et de l'Alliance que Dieu a traitée avec nos pères, en disant à Abraham~: Toutes les familles de la terre seront bénies en ta postérité\FTNT{\vref{Ge. 12:2}.}.
\VS{26}C'est à vous premièrement que Dieu, ayant suscité son Fils Jésus, l'a envoyé pour vous bénir, en détournant chacun de vous de vos iniquités.
\Chap{4}
\TextTitle{Première persécution de l'Eglise~: Pierre et Jean jetés en prison}
\VerseOne{}Mais comme ils parlaient au peuple, survinrent les prêtres, le commandant du temple et les sadducéens,
\VS{2}étant offensés de ce qu'ils enseignaient le peuple, et qu'ils annonçaient la résurrection des morts au Nom de Jésus.
\VS{3}Et les ayant fait arrêter, ils les mirent en prison jusqu'au lendemain, parce qu'il était déjà tard.
\VS{4}Et plusieurs de ceux qui avaient entendu la parole crurent~; et le nombre des personnes fut d'environ cinq mille.
\TextTitle{Pierre et Jean convoqués au sanhédrin}
\VS{5}Or il arriva que le lendemain, les chefs, les anciens et les scribes s'assemblèrent à Jérusalem~;
\VS{6}avec Anne, le grand-prêtre, Caïphe, Jean, Alexandre, et tous ceux qui étaient de la race des principaux prêtres.
\VS{7}Et ayant fait comparaître devant eux Pierre et Jean, ils leur demandèrent~: Par quelle puissance, ou au nom de qui avez-vous fait cette guérison~?
\VS{8}Alors Pierre étant rempli du Saint-Esprit, leur dit~: Chefs du peuple, et vous anciens d'Israël~:
\VS{9}Puisque nous sommes jugés aujourd'hui sur un bienfait accordé à un homme impotent, afin que nous disions comment il a été guéri,
\VS{10}sachez, vous tous et tout le peuple d'Israël, que c'est au Nom de Jésus-Christ de Nazareth, que vous avez crucifié, et que Dieu a ressuscité des morts~; c'est en son Nom, que cet homme qui parait ici devant vous, a été guéri.
\VS{11}C'est cette pierre rejetée, par vous qui bâtissez, qui est devenue la pierre principale de l'angle\FTNT{\vref{Ps. 118:22}.}.
\VS{12}Il n'y a de salut en aucun autre~: Car il n'y a sous le ciel aucun autre Nom qui ait été donné aux hommes par lequel nous devions être sauvés.
\TextTitle{Le sanhédrin interdit aux apôtres de prêcher au Nom de Jésus}
\VS{13}Eux, voyant la hardiesse de Pierre et de Jean, et sachant aussi qu'ils étaient des hommes sans instruction et du commun peuple~; s'en étonnaient, et ils reconnaissaient bien qu'ils avaient été avec Jésus.
\VS{14}Et voyant que l'homme qui avait été guéri, était présent avec eux, ils ne pouvaient contredire en rien.
\VS{15}Alors ils leur ordonnèrent de sortir hors du sanhédrin, et ils délibérèrent entre eux, disant~: Que ferons-nous à ces gens~?
\VS{16}Car il est manifeste pour tous les habitants de Jérusalem, qu'un miracle a été fait par eux, et cela est si évident que nous ne pouvons le nier.
\VS{17}Mais afin qu'il ne soit plus divulgué parmi le peuple, défendons-leur avec menaces expresses, qu'ils n'aient plus à parler à qui que ce soit en ce Nom.
\VS{18}Et les ayant donc appelés, ils leur ordonnèrent de ne plus parler ni d'enseigner en aucune manière au Nom de Jésus. 
\VS{19}Mais Pierre et Jean leur répondirent~: Jugez s'il est juste devant Dieu de vous obéir plutôt qu'à Dieu.
\VS{20}Car nous ne pouvons pas ne pas parler de ce que nous avons vu et entendu.
\VS{21}Alors ils les relâchèrent avec menaces, ne trouvant point comment ils pourraient les punir, à cause du peuple, parce que tous glorifiaient Dieu de ce qui avait été fait.
\VS{22}Car l'homme en qui cette miraculeuse guérison avait été faite, avait plus de quarante ans.
\TextTitle{L'Eglise demande l'assistance de Dieu}
\VS{23}Après avoir été relâchés, ils allèrent vers les leurs, et leur racontèrent tout ce que les principaux prêtres et les anciens leur avaient dit.
\VS{24}Eux l'ayant entendu, élevèrent tous ensemble la voix à Dieu, et dirent~: Seigneur, tu es le Dieu qui as fait le ciel et la terre, la mer, et toutes les choses qui y sont~;
\VS{25}et qui as dit par la bouche de David ton serviteur~: Pourquoi ce tumulte parmi les nations et ces vaines pensées parmi les peuples~?
\VS{26}Les rois de la terre se sont soulevés en personne, et les princes se sont ligués ensemble contre le Seigneur, et contre son Christ\FTNT{\vref{Ps. 2:1-2}.}.
\VS{27}En effet, contre ton Saint Fils Jésus, que tu as oint, se sont assemblés Hérode et Ponce Pilate, avec les Gentils, et le peuple d'Israël,
\VS{28}pour faire toutes les choses que ta main et ton conseil avaient auparavant déterminé qui seraient faites. 
\VS{29}Maintenant donc, Seigneur, regarde à leurs menaces, et donne à tes serviteurs d'annoncer ta parole avec toute hardiesse~;
\VS{30}en étendant ta main afin qu'il se fasse des guérisons, des prodiges, et des merveilles, par le Nom de ton Saint Fils Jésus.
\VS{31}Et quand ils eurent prié, le lieu où ils étaient assemblés trembla~; et ils furent tous remplis du Saint-Esprit, et ils annonçaient la parole de Dieu avec hardiesse.
\TextTitle{La multitude unie comme un seul corps\FTNTT{\vref{Ac. 2:42-47}.}}
\VS{32}Or la multitude de ceux qui croyaient n'était qu'un cœur et qu'une âme et nul ne disait d'aucune des choses qu'il possédait, qu'elle fût à lui, mais toutes choses étaient communes entre eux.
\VS{33}Aussi les apôtres rendaient témoignage avec une grande force à la résurrection du Seigneur Jésus~; et une grande grâce était sur eux tous.
\VS{34}Car il n'y avait parmi eux aucun indigent~; parce que tous ceux qui possédaient des champs ou des maisons, les vendaient, et ils apportaient le prix des choses vendues,
\VS{35}et le mettaient aux pieds des apôtres~; et il était distribué à chacun selon qu'il en avait besoin.
\VS{36}Or Joseph, surnommé par les apôtres Barnabas, c'est-à-dire, fils de consolation, Lévite, originaire de Chypre,
\VS{37}ayant une possession, la vendit, et en apporta le prix, et le mit aux pieds des apôtres.
\Chap{5}
\TextTitle{Mensonge d'Ananias et Saphira~: Leur mort}
\VerseOne{}Mais un homme appelé Ananias, et Saphira sa femme, vendit une possession,
\VS{2}et retint une partie du prix, sa femme le sachant~; puis il apporta le reste, et le déposa aux pieds des apôtres.
\VS{3}Mais Pierre lui dit~: Ananias comment Satan s'est-il emparé de ton cœur jusqu'à t'inciter à mentir au Saint-Esprit, et à soustraire une partie du prix de la possession~?
\VS{4}Si tu l'avais gardée, ne te restait-elle pas~? Et après qu'elle ait été vendue, le prix n'était-il pas à ta disposition~? Comment as-tu pu mettre en ton cœur un pareil dessein~? Tu n'as pas menti aux hommes mais à Dieu.
\VS{5}Et Ananias, entendant ces paroles, tomba et rendit l'âme~; ce qui causa une grande crainte à tous ceux qui en entendirent parler.
\VS{6}Et quelques jeunes hommes se levant le prirent, et l'emportèrent dehors, et l'ensevelirent.
\VS{7}Et il arriva environ trois heures après que sa femme entra, sans savoir ce qui était arrivé.
\VS{8}Et Pierre prenant la parole, lui dit~: Dis-moi, avez-vous autant vendu le champ~? Et elle dit~: Oui, autant.
\VS{9}Alors Pierre lui dit~: Pourquoi avez-vous fait un complot entre vous pour tenter l'Esprit du Seigneur~? Voici, à la porte, les pieds de ceux qui ont enterré ton mari, et ils t'emporteront.
\VS{10}Et au même instant, elle tomba à ses pieds et rendit l'esprit. Et quand les jeunes hommes furent entrés, ils la trouvèrent morte, et ils l'emportèrent dehors, et l'ensevelirent auprès de son mari.
\VS{11}Et cela donna une grande crainte à toute l'Eglise, et à tous ceux qui entendaient ces choses.
\TextTitle{Miracles à Jérusalem}
\VS{12}Beaucoup de prodiges et de miracles se faisaient parmi le peuple par les mains des apôtres~; et ils étaient tous d'un commun accord au portique de Salomon.
\VS{13}Cependant aucun des autres n'osait se joindre à eux, mais le peuple les louait hautement.
\VS{14}Et le nombre de ceux qui croyaient au Seigneur, tant d'hommes que de femmes, se multipliait de plus en plus.
\VS{15}Et on apportait les malades dans les rues, et on les mettait sur de petits lits et sur des couchettes, afin que quand Pierre viendrait, au moins son ombre passe sur quelqu'un d'eux.
\VS{16}La multitude accourait aussi des villes voisines à Jérusalem, amenant des malades, et ceux qui étaient tourmentés des esprits impurs~; et tous étaient guéris.
\TextTitle{Deuxième persécution de l'Eglise~: Les apôtres en prison puis devant le sanhédrin}
\VS{17}Alors le grand-prêtre se leva, lui et tous ceux qui étaient avec lui, à savoir la secte des sadducéens, et ils furent remplis de jalousie~;
\VS{18}et mettant la main sur les apôtres, ils les jetèrent dans la prison publique.
\VS{19}Mais l'Ange du Seigneur ouvrit pendant la nuit les portes de la prison, les fit sortir, et leur dit~:
\VS{20}Allez, et présentez-vous dans le temple, annoncez au peuple toutes les paroles de cette vie.
\VS{21}Ayant entendu cela, ils entrèrent dès le matin dans le temple, et se mirent à enseigner. Mais le grand-prêtre et ceux qui étaient avec lui étant arrivés, ils convoquèrent le sanhédrin et tous les anciens des fils d'Israël, et ils envoyèrent chercher les apôtres à la prison.
\VS{22}Mais, les huissiers à leur arrivée, ne les trouvèrent point dans la prison. Ils retournèrent, et firent leur rapport,
\VS{23}en disant~: Nous avons trouvé la prison fermée avec toute sûreté, et les gardes aussi qui étaient devant les portes~; mais après l'avoir ouverte, nous n'avons trouvé personne dedans.
\VS{24}Lorsque le grand-prêtre, le commandant du temple, et les principaux prêtres, eurent entendu ces paroles, ils ne savaient que penser au sujet des apôtres, ne sachant ce qui arriverait de tout cela.
\VS{25}Mais quelqu'un vint leur dire~: Voici, les hommes que vous avez mis en prison sont dans le temple, et ils enseignent le peuple.
\VS{26}Alors le commandant du temple partit avec les huissiers, et il les conduisit sans violence, car ils avaient peur d'être lapidés par le peuple.
\VS{27}Après qu'ils les eurent amenés, ils les présentèrent au sanhédrin. Et le grand-prêtre les interrogea, disant~: 
\VS{28}Ne vous avons-nous pas défendu expressément d'enseigner en ce Nom-là~? Et cependant voici, vous avez rempli Jérusalem de votre doctrine, et vous voulez faire retomber sur nous le sang de cet homme.
\VS{29}Alors Pierre et les autres apôtres répondant, dirent~: Il faut plutôt obéir à Dieu qu'aux hommes.
\VS{30}Le Dieu de nos pères a ressuscité Jésus, que vous avez fait mourir en le pendant au bois.
\VS{31}Dieu l'a élevé par sa puissance pour être Prince et Sauveur, afin de donner à Israël la repentance et la rémission des péchés.
\VS{32}Nous sommes témoins de ce que nous disons, de même que le Saint-Esprit que Dieu a donné à ceux qui lui obéissent, en est aussi témoin.
\TextTitle{Parole de sagesse de Gamaliel}
\VS{33}Mais eux, ayant entendu ces choses, grinçaient les dents, et consultaient pour les faire mourir.
\VS{34}Mais un pharisien nommé Gamaliel, docteur de la loi, honoré de tout le peuple, se leva dans le sanhédrin, et ordonna de faire sortir un instant les apôtres.
\VS{35}Puis il leur dit~: Hommes Israélites, prenez garde à ce que vous allez faire à l'égard de ces gens.
\VS{36}Car il n'y a pas longtemps que Theudas s'éleva, se disant être quelque chose, et auquel se joignit un nombre d'environ quatre cents hommes~; mais il fut tué, et tous ceux qui s'étaient joints à lui ont été dissipés et réduits à rien.
\VS{37}Après lui parut Judas le Galiléen au temps du recensement, et il attira à lui un grand peuple~; il périt aussi, et tous ceux qui s'étaient joints à lui ont été dispersés.
\VS{38}Maintenant donc je vous dis~: Ne continuez plus vos poursuites contre ces hommes, et laissez-les. Car si cette entreprise ou cette œuvre vient des hommes, elle sera détruite~;
\VS{39}mais si elle vient de Dieu, vous ne pourrez pas la détruire. Et prenez garde qu'il ne se trouve que vous combattiez contre Dieu.
\VS{40}Et ils furent de son avis. Et ayant appelé les apôtres, ils les firent battre de verges, ils leur défendirent de parler au Nom de Jésus, et ils les relâchèrent.
\TextTitle{Frappés, les apôtres continuent de prêcher le Nom de Jésus}
\VS{41}Et les apôtres se retirèrent de devant le sanhédrin, joyeux d'avoir été jugés dignes de subir des outrages pour le Nom de Jésus.
\VS{42}Et tous les jours, ils ne cessaient d'enseigner, et d'annoncer l'Evangile de Jésus-Christ dans le temple, et de maison en maison.
\Chap{6}
\TextTitle{Sept hommes choisis pour le service}
\VerseOne{}En ces jours-là, comme les disciples se multipliaient, il s'éleva un murmure des Hellénistes\FTNT{Les Hellénistes étaient des juifs issus de la diaspora ayant adopté la culture et la langue grecque.} contre les Hébreux, parce que leurs veuves étaient méprisées dans le service ordinaire.
\VS{2}C'est pourquoi les douze, ayant convoqué la multitude des disciples, leur dirent~: Il n'est pas raisonnable que nous laissions la parole de Dieu pour servir aux tables.
\VS{3}Regardez donc, mes frères, pour choisir sept hommes d'entre vous, de qui on ait bon témoignage, pleins du Saint-Esprit et de sagesse, auxquels nous confierons ce devoir.
\VS{4}Et nous, nous continuerons à vaquer à la prière et au service de la parole.
\VS{5}Et ce discours plut à toute l'assemblée qui était là présente~; et ils élurent Etienne, homme plein de foi et du Saint-Esprit, Philippe, Prochore, Nicanor, Timon, Parménas, et Nicolas, prosélyte d'Antioche.
\VS{6}Ils les présentèrent aux apôtres~; qui, après avoir prié, leur imposèrent les mains.
\VS{7}Et la parole de Dieu croissait, et le nombre des disciples se multipliait beaucoup dans Jérusalem~; un grand nombre aussi de prêtres obéissait à la foi.
\VS{8}Or Etienne, plein de foi et de puissance, faisait de grands miracles et de grands prodiges parmi le peuple.
\TextTitle{Troisième persécution de l'Eglise~; Etienne convoqué au sanhédrin}
\VS{9}Quelques-uns de la synagogue appelée la synagogue des affranchis\FTNT{Affranchis~: Du grec «~libertinos~», c'est-à-dire «~libertins~»~: Hommes libres. Fraction de la communauté Juive qui avait sa propre synagogue à Jérusalem. Probablement des Juifs qui avaient été faits prisonniers par Pompée et d'autres généraux romains, qui avaient été déportés à Rome, puis libérés.}, de celle des Cyrénéens et de celle des Alexandrins, avec ceux de Cilicie et d'Asie, se levèrent pour disputer contre Etienne.
\VS{10}Mais ils ne pouvaient pas résister à la sagesse et à l'Esprit par lequel il parlait.
\VS{11}Alors ils soudoyèrent des hommes qui dirent~: Nous l'avons entendu proférer des paroles blasphématoires contre Moïse et contre Dieu.
\VS{12}Et ils soulevèrent le peuple, les anciens, et les scribes, et se jetant sur lui, ils l'enlevèrent et l'amenèrent au sanhédrin.
\VS{13}Et ils présentèrent de faux témoins qui dirent~: Cet homme ne cesse de proférer des paroles blasphématoires contre ce saint lieu et contre la loi.
\VS{14}Car nous l'avons entendu dire que Jésus, ce Nazaréen, détruira ce lieu-ci, et changera les coutumes que Moïse nous a données.
\VS{15}Tous ceux qui siégeaient au sanhédrin avaient les yeux fixés sur lui, son visage leur parut comme celui d'un ange.
\Chap{7}
\TextTitle{Discours d'Etienne devant le sanhédrin}
\VerseOne{}Alors le grand-prêtre lui dit~: Ces choses sont-elles ainsi~?
\VS{2}Etienne répondit~: Hommes frères et pères, écoutez-moi~! Le Dieu de gloire apparut à notre père Abraham, lorsqu'il était en Mésopotamie, avant qu'il s'établisse à Charran, et lui dit~:
\VS{3}Sors de ton pays et de ta famille, et va dans le pays que je te montrerai.
\VS{4}Il sortit donc du pays des Chaldéens, et alla demeurer à Charran. De là, après la mort de son père, Dieu le fit passer dans ce pays que vous habitez maintenant.
\VS{5}Et il ne lui donna aucun héritage dans ce pays, non pas même d'un pied de terre, quoiqu'il lui ait promis de le lui donner en possession, et à sa postérité après lui, dans un temps où il n'avait point encore d'enfant.
\VS{6}Dieu lui parla ainsi~: Ta postérité séjournera dans une terre étrangère pendant quatre cents ans~; et on la réduira à la servitude et on la maltraitera.
\VS{7}Mais je jugerai la nation à laquelle ils auront été asservis, dit Dieu~; et après cela ils sortiront, et me serviront en ce lieu-ci\FTNT{\vref{Ge. 15:13-14}.}.
\VS{8}Puis il donna à Abraham l'alliance de la circoncision~; et après cela Abraham engendra Isaac qu'il circoncit le huitième jour. Isaac engendra Jacob, et Jacob les douze patriarches.
\VS{9}Les patriarches, jaloux de Joseph, le vendirent pour être emmené en Egypte.
\VS{10}Mais Dieu était avec lui, et le délivra de toutes ses afflictions~; et l'ayant rempli de sagesse il le rendit agréable à Pharaon, roi d'Egypte, qui l'établit gouverneur sur l'Egypte, et sur toute sa maison.
\VS{11}Or il survint dans tout le pays d'Egypte, et dans celui de Canaan, une famine et une grande détresse, en sorte que nos pères ne pouvaient trouver des vivres.
\VS{12}Mais Jacob apprit qu'il y avait du blé en Egypte, il y envoya une première fois nos pères.
\VS{13}Et la seconde fois, Joseph fut reconnu par ses frères, et la famille de Joseph fut déclarée à Pharaon.
\VS{14}Alors Joseph envoya chercher Jacob, son père, et toute sa famille, composée de soixante-quinze personnes.
\VS{15}Jacob descendit en Egypte, et il y mourut, lui et nos pères~;
\VS{16}qui furent transportés à Sichem, et mis dans le sépulcre qu'Abraham avait acheté à prix d'argent des fils d'Hamor, fils de Sichem.
\VS{17}Mais comme le temps de la promesse, pour laquelle Dieu avait juré à Abraham, s'approchait, le peuple s'augmenta et se multiplia en Egypte~;
\VS{18}jusqu'à ce que parut en Egypte un autre roi, qui n'avait pas connu Joseph.
\VS{19}Ce roi, usant d'artifice contre notre race, maltraita nos pères jusqu'à leur faire exposer leurs enfants à l'abandon, afin d'en faire périr la race.
\VS{20}En ce temps-là naquit Moïse, qui fut divinement beau. Et il fut nourri trois mois dans la maison de son père.
\VS{21}Mais ayant été exposé à l'abandon, la fille de Pharaon le recueillit et l'éleva comme son fils.
\VS{22}Moïse fut instruit dans toute la sagesse des Egyptiens~; et il était puissant en paroles et en œuvres.
\VS{23}Mais quand il fut parvenu à l'âge de quarante ans, il forma le dessein d'aller visiter ses frères, les enfants d'Israël.
\VS{24}Et voyant l'un d'eux à qui l'on faisait tort, il le défendit, et vengea celui qui était outragé en tuant l'Egyptien.
\VS{25}Il croyait que ses frères comprendraient par là que Dieu les délivrerait par son moyen~; mais ils ne le comprirent point.
\VS{26}Et le jour suivant, il parut au milieu d'eux comme ils se querellaient, et il tâcha de les mettre d'accord en leur disant~: Hommes, vous êtes frères, pourquoi vous faites-vous tort l'un à l'autre~?
\VS{27}Mais celui qui maltraitait son prochain le repoussa, en disant~: Qui t'a établi prince et juge sur nous~?
\VS{28}Veux-tu me tuer, comme tu as tué hier l'Egyptien~?
\VS{29}Alors Moïse s'enfuit sur un tel discours, et fut étranger dans le pays de Madian, où il eut deux fils.
\VS{30}Et quarante ans étant accomplis, l'Ange du Seigneur lui apparut au désert de la montagne de Sinaï, dans la flamme d'un buisson en feu.
\VS{31}Et quand Moïse le vit, il fut étonné de la vision, et comme il approchait pour considérer ce que c'était, la voix du Seigneur lui fut adressée, disant~:
\VS{32} Je suis le Dieu de tes pères, le Dieu d'Abraham, le Dieu d'Isaac, et le Dieu de Jacob. Et Moïse tout tremblant n'osait pas regarder.
\VS{33}Le Seigneur lui dit~: Ôte tes souliers de tes pieds, car le lieu sur lequel tu te tiens est une terre sainte.
\VS{34}J'ai vu, j'ai vu l'affliction de mon peuple qui est en Egypte, et j'ai entendu leur gémissement, et je suis descendu pour les délivrer. Maintenant donc, va, je t'enverrai en Egypte.
\VS{35}Ce Moïse, qu'ils avaient rejeté en disant~: Qui t'a établi prince et juge~? C'est lui que Dieu envoya comme prince et comme libérateur par le moyen de l'Ange qui lui était apparu dans le buisson.
\VS{36}C'est celui qui les tira dehors, en opérant des miracles et des prodiges au pays d'Egypte, au sein de la Mer Rouge, et au désert pendant quarante ans.
\VS{37}C'est ce Moïse qui a dit aux enfants d'Israël~: Le Seigneur votre Dieu vous suscitera d'entre vos frères un Prophète comme moi~; écoutez-le\FTNT{\vref{De. 18:15}.}~!
\VS{38}C'est lui, qui, lors de l'assemblée au désert, étant avec l'Ange qui lui parlait sur la montagne de Sinaï et avec nos pères, reçut les paroles de vie pour nous les donner.
\VS{39}Nos pères ne voulurent pas lui obéir, mais ils le rejetèrent, et ils tournèrent leur cœur vers l'Egypte,
\VS{40}en disant à Aaron~: Fais-nous des dieux qui marchent devant nous~; car nous ne savons point ce qui est arrivé à ce Moïse qui nous a amenés hors du pays d'Egypte.
\VS{41}Ils firent donc en ces jours-là un veau, et ils offrirent des sacrifices à l'idole, et se réjouirent de l'œuvre de leurs mains.
\VS{42}C'est pourquoi aussi Dieu se détourna d'eux, et les livra au culte de l'armée du ciel, ainsi qu'il est écrit dans le livre des prophètes~: Maison d'Israël, m'avez-vous offert des sacrifices et des victimes pendant quarante ans au désert~?
\VS{43}Mais vous avez porté la tente de Moloc\FTNT{\vref{Lé. 18:21}.}, et l'étoile de votre dieu Remphan~; qui sont des figures que vous avez faites pour les adorer. C'est pourquoi je vous transporterai au-delà de Babylone.
\VS{44}Nos pères avaient au désert le tabernacle du témoignage, comme l'avait ordonné celui qui avait dit à Moïse de le faire selon le modèle qu'il avait vu.
\VS{45}Et nos pères avaient reçu ce tabernacle, ils le portèrent sous la conduite de Josué dans le pays qui était possédé par les nations que Dieu chassa de devant eux, et il y resta jusqu'aux jours de David.
\VS{46}David trouva grâce devant Dieu, et demanda de pouvoir dresser une tente pour le Dieu de Jacob.
\VS{47}Et ce fut Salomon qui lui bâtit une maison.
\VS{48}Mais le Très-Haut n'habite pas dans des temples faits de main d'homme, selon ces paroles du prophète~:
\VS{49}Le ciel est mon trône, et la terre est le marchepied de mes pieds~: Quelle maison me bâtirez-vous, dit le Seigneur, ou quel pourrait être le lieu de mon repos~?
\VS{50}Ma main n'a-t-elle pas fait toutes ces choses\FTNT{\vref{Es. 66:1}.}~?
\VS{51}Hommes au cou raide, et incirconcis de cœur et d'oreilles, vous vous obstinez toujours contre le Saint-Esprit~; vous faites comme vos pères ont fait.
\VS{52}Lequel des prophètes vos pères n'ont-ils pas persécuté~? Ils ont même tué ceux qui annonçaient d'avance l'avènement du Juste, dont vous avez été les traîtres et les meurtriers,
\VS{53}vous qui avez reçu la loi par une ordonnance des anges, et qui ne l'avez point gardée.
\TextTitle{Etienne~: Premier martyr}
\VS{54}En entendant ces choses, leur cœur s'enflamma de colère, et ils grinçaient des dents contre lui.
\VS{55}Mais Etienne, rempli du Saint-Esprit, et fixant les yeux vers le ciel, vit la gloire de Dieu, et Jésus qui était à la droite de Dieu.
\VS{56}Et il dit~: Voici, je vois les cieux ouverts, et le Fils de l'homme étant à la droite de Dieu.
\VS{57}Alors ils s'écrièrent à haute voix, et bouchèrent leurs oreilles, et tous d'un commun accord se jetèrent sur lui.
\VS{58}Et l'ayant tiré hors de la ville, ils le lapidèrent~; et les témoins déposèrent leurs vêtements aux pieds d'un jeune homme nommé Saul.
\VS{59}Et ils lapidaient Etienne qui priait et disait~: Seigneur Jésus, reçois mon esprit\FTNT{Dans \vref{Ec. 12:9}, il est dit qu'à la mort, l'esprit retourne à Dieu qui l'a donné. Jésus est donc Dieu puisqu'il a reçu l'esprit d'Etienne.}~!
\VS{60}Et s'étant mis à genoux, il cria à haute voix~: Seigneur, ne leur impute point ce péché~! Et quand il eut dit cela, il s'endormit.
\Chap{8}
\TextTitle{Quatrième persécution de l'Eglise~: Saul opprime les saints}
\VerseOne{}Or Saul consentait à la mort d'Etienne, et en ce temps-là, il y eut une grande persécution contre l'Eglise de Jérusalem. Et tous, excepté les apôtres, se dispersèrent dans les contrées de la Judée et de la Samarie.
\VS{2}Et quelques hommes pieux emportèrent Etienne pour l'ensevelir, et le pleurèrent à grand bruit.
\VS{3}Mais Saul ravageait l'église, entrant dans toutes les maisons, et traînant par force hommes et femmes, il les mettait en prison.
\TextTitle{Le déploiement des chrétiens\FTNTT{\vref{Ac. 11:19-21}}}
\VS{4}Ceux qui avaient été dispersés allaient de lieu en lieu, annonçant la parole de Dieu.
\TextTitle{Philippe en Samarie~; Simon le magicien}
\VS{5}Philippe, étant descendu dans la ville de Samarie, leur prêcha Christ.
\VS{6}Et les foules tout entières étaient attentives à ce que Philippe disait, l'écoutant, lorsqu'elles virent les miracles qu'il faisait,
\VS{7}car les esprits impurs sortaient, en criant à haute voix, hors de plusieurs qui en étaient possédés, et beaucoup de paralytiques et de boiteux furent guéris.
\VS{8}Ce qui causa une grande joie dans cette ville-là.
\VS{9}Or il y avait auparavant dans la ville un homme nommé Simon qui exerçait l'art d'enchanteur, et ensorcelait le peuple de Samarie, se disant être quelque grand personnage.
\VS{10}Tous, depuis le plus petit jusqu'au plus grand étaient attachés à lui, et disaient~: Celui-ci est la grande puissance de Dieu.
\VS{11}Et ils étaient attachés à lui, parce que depuis longtemps il les avait éblouis par sa magie.
\VS{12}Mais quand ils eurent cru ce que Philippe leur annonçait, touchant l'Evangile du Royaume de Dieu, et le Nom de Jésus-Christ, tant les hommes que les femmes furent baptisés.
\VS{13}Et Simon crut aussi lui-même, et après avoir été baptisé, il ne quittait plus Philippe~; et voyant les prodiges et les grands miracles qui se faisaient, il était comme ravi hors de lui même.
\VS{14}Or quand les apôtres, qui étaient à Jérusalem eurent entendu que la Samarie avait reçu la parole de Dieu, ils leur envoyèrent Pierre et Jean~;
\VS{15}qui y étant descendus prièrent pour eux, afin qu'ils reçoivent le Saint-Esprit,
\VS{16}car il n'était pas encore descendu sur aucun d'eux, mais seulement ils étaient baptisés au Nom du Seigneur Jésus.
\VS{17}Puis ils leur imposèrent les mains, et ils reçurent le Saint-Esprit.
\VS{18}Lorsque Simon vit que le Saint-Esprit était donné par l'imposition des mains des apôtres, il leur présenta de l'argent,
\VS{19}en leur disant~: Donnez-moi aussi ce pouvoir, afin que tous ceux à qui j'imposerai les mains reçoivent le Saint-Esprit.
\VS{20}Mais Pierre lui dit~: Que ton argent périsse avec toi, puisque tu as estimé que le don de Dieu s'acquérait avec de l'argent.
\VS{21}Tu n'as point de part ni d'héritage en cette affaire~; car ton cœur n'est point droit devant Dieu.
\VS{22}Repens-toi donc de cette méchanceté, et prie Dieu, afin que, s'il est possible, la pensée de ton cœur te soit pardonnée.
\VS{23}Car je vois que tu es dans un fiel très amer et dans un lien d'iniquité.
\VS{24}Alors Simon répondit, et dit~: Vous, priez le Seigneur pour moi, afin qu'il ne m'arrive rien de ce que vous avez dit. 
\VS{25}Eux donc après avoir prêché et annoncé la parole du Seigneur, retournèrent à Jérusalem et annoncèrent l'Evangile dans plusieurs villages des Samaritains.
\TextTitle{Conversion et baptême de l'eunuque éthiopien}
\VS{26}Puis l'Ange du Seigneur parla à Philippe, en disant~: Lève-toi et va vers le Midi, sur le chemin qui descend de Jérusalem à Gaza, celui qui est désert.
\VS{27}Il se leva donc, et s'en alla. Et voici, un homme éthiopien, un eunuque, qui était un des principaux seigneurs de la cour de Candace, reine des Ethiopiens, et surintendant de toutes ses richesses, venu à Jérusalem pour adorer,
\VS{28}s'en retournait, assis dans son char, et lisait le prophète Esaïe.
\VS{29}L'Esprit dit à Philippe~: Avance, et approche-toi de ce char.
\VS{30}Philippe accourut et entendit l'Ethiopien qui lisait le prophète Esaïe~; et il lui dit~: Comprends-tu ce que tu lis~?
\VS{31}Et il lui dit~: Comment pourrais-je le comprendre, si quelqu'un ne me guide pas~? Et il pria Philippe de monter et s'asseoir avec lui.
\VS{32}Le passage de l'Ecriture qu'il lisait était celui-ci~: Il a été mené comme une brebis à la boucherie, et comme un agneau muet devant celui qui le tond~; en sorte qu'il n'a point ouvert sa bouche.
\VS{33}Dans son humiliation, son jugement a été levé~; mais qui racontera sa durée~? Car sa vie est retranchée de la terre\FTNT{\vref{Es. 53:7-8}.}.
\VS{34}Et l'eunuque prenant la parole, dit à Philippe~: Je te prie, de qui est-ce que le prophète dit cela~? Est-ce de lui-même, ou de quelque autre~?
\VS{35}Alors Philippe, ouvrant sa bouche, et commençant par cette Ecriture, lui annonça l'Evangile de Jésus.
\VS{36}Comme ils continuaient leur chemin, ils arrivèrent à un endroit où il y avait de l'eau. Et l'eunuque dit~: Voici de l'eau, qu'est-ce qui empêche que je ne sois baptisé~?
\VS{37}Philippe dit~: Si tu crois de tout ton cœur, cela t'est permis~; et l'eunuque répondit~: Je crois que Jésus-Christ est le Fils de Dieu.
\VS{38}Il fit arrêter le char~; Philippe et l'eunuque descendirent tous deux dans l'eau, et Philippe le baptisa.
\VS{39}Quand ils furent sortis de l'eau, l'Esprit du Seigneur enleva Philippe, et l'eunuque ne le vit plus. Tandis que tout joyeux il continua son chemin,
\VS{40}Philippe se trouva dans Azot, d'où il alla jusqu'à Césarée, en évangélisant toutes les villes par lesquelles il passait.
\Chap{9}
\TextTitle{Jésus se révèle à Saul\FTNTT{\vref{Ac. 22:1-16}~; \vref{26:9-18}}}
\VerseOne{}Or Saul, respirant encore la menace et le carnage contre les disciples du Seigneur, s'adressa au grand-prêtre,
\VS{2}et lui demanda des lettres de sa part pour les porter aux synagogues de Damas, afin que, s'il trouvait quelques-uns de cette secte, hommes ou femmes, il les amène liés à Jérusalem.
\VS{3}Or il arriva qu'en marchant, il approcha de Damas et tout à coup une lumière resplendit du ciel comme un éclair autour de lui.
\VS{4}Il tomba par terre et il entendit une voix qui lui disait~: Saul, Saul, pourquoi me persécutes-tu~?
\VS{5}Et il répondit~: Qui es-tu, Seigneur~? Et le Seigneur lui dit~: Je suis Jésus, que tu persécutes. Il te serait dur de regimber contre les aiguillons.
\VS{6}Alors, tout tremblant et tout effrayé, il dit~: Seigneur, que veux-tu que je fasse~? Et le Seigneur lui dit~: Lève-toi, et entre dans la ville, et on te dira ce que tu dois faire.
\VS{7}Les hommes qui l'accompagnaient s'arrêtèrent tout épouvantés, entendant bien la voix, mais ne voyant personne.
\VS{8}Et Saul se leva de terre, et ouvrant ses yeux, il ne voyait personne~; c'est pourquoi ils le conduisirent par la main, et le menèrent à Damas,
\VS{9}où il fut trois jours sans voir, sans manger ni boire.
\VS{10}Or il y avait à Damas un disciple, nommé Ananias, à qui le Seigneur dit en vision~: Ananias~! Et il répondit~: Me voici Seigneur~!
\VS{11}Et le Seigneur lui dit~: Lève-toi, va dans la rue appelée la droite, et cherche dans la maison de Judas un homme appelé Saul, de Tarse.
\VS{12}Car il prie. Or Saul avait vu en vision un homme appelé Ananias, entrant et lui imposant les mains, afin qu'il recouvre la vue. Et Ananias répondit~:
\VS{13} Seigneur, j'ai entendu parler plusieurs fois de cet homme-là~; et combien de maux il a faits à tes saints dans Jérusalem.
\VS{14}Il a même ici le pouvoir de la part des principaux prêtres, de lier tous ceux qui invoquent ton Nom.
\VS{15}Mais le Seigneur lui dit~: Va~; car il m'est un vase\FTNTT{Le mot «~vase~» vient du grec «~skeuos~». «~Vase~» était une métaphore grecque commune pour «~le corps~» car les Grecs pensaient que l'âme vivait temporairement dans les corps. \vref{2 Co. 4:7}~; \vref{Ro. 9:21-23}~; \vref{2 Ti. 2:20-21}.} que j'ai choisi, pour porter mon Nom devant les Gentils, et les rois, et les enfants d'Israël.
\VS{16}Car je lui montrerai combien il aura à souffrir pour mon Nom.
\TextTitle{Saul rempli du Saint-Esprit}
\VS{17}Ananias sortit~; et lorsqu'il fut arrivé dans la maison, il imposa les mains à Saul, et lui dit~: Saul mon frère, le Seigneur Jésus, qui t'est apparu sur le chemin par lequel tu venais, m'a envoyé, afin que tu recouvres la vue, et que tu sois rempli du Saint-Esprit.
\TextTitle{Saul est baptisé et évangélise Damas}
\VS{18}Et aussitôt il tomba de ses yeux comme des écailles~; et à l'instant il recouvra la vue. Puis il se leva, et fut baptisé.
\VS{19}Et ayant mangé, il reprit ses forces. Et Saul fut quelques jours avec les disciples qui étaient à Damas.
\VS{20}Et aussitôt il prêcha dans les synagogues que Jésus était le Fils de Dieu.
\VS{21}Et tous ceux qui l'entendaient, étaient comme ravis hors d'eux-mêmes, et ils disaient~: N'est-ce pas celui-là qui a détruit à Jérusalem ceux qui invoquaient ce Nom, et qui est venu ici exprès pour les amener liés aux principaux prêtres~?
\VS{22}Mais Saul se fortifiait de plus en plus, et confondait les Juifs qui habitaient à Damas, prouvant que Jésus était le Christ.
\TextTitle{Complot contre Saul}
\VS{23}Longtemps après, les Juifs conspirèrent ensemble pour le faire mourir~;
\VS{24}et leur complot parvint à la connaissance de Saul. Or ils gardaient les portes jour et nuit, afin de le faire mourir.
\VS{25}Mais pendant une nuit, les disciples le prirent, et le descendirent par la muraille dans une corbeille.
\TextTitle{Saul rencontre Barnaba et les apôtres à Jérusalem}
\VS{26}Lorsqu'il se rendit à Jérusalem, Saul tâcha de se joindre aux disciples~; mais tous le craignaient, ne croyant pas qu'il fût un disciple.
\VS{27}Alors Barnabas, l'ayant pris avec lui, le conduisit vers les apôtres, et leur raconta comment sur le chemin, Saul avait vu le Seigneur, qui lui avait parlé, et comment à Damas il parlait librement au Nom de Jésus.
\VS{28}Et il allait et venait avec eux dans Jérusalem, il parlait franchement au Nom du Seigneur, se montrant publiquement.
\VS{29}Et parlant sans déguisement au Nom du Seigneur Jésus, il disputait contre les Hellénistes, mais ils tentaient de le faire mourir.
\TextTitle{Retour à Tarse}
\VS{30}Les frères, l'ayant découvert, l'emmenèrent à Césarée, et le firent partir à Tarse.
\VS{31}Les églises étaient en paix dans toute la Judée, la Galilée, et la Samarie, étant édifiées et marchant dans la crainte du Seigneur~; et elles s'accroissaient par le rafraîchissement du Saint-Esprit.
\TextTitle{Guérison d'Enée, le paralytique}
\VS{32}Or il arriva que comme Pierre les visitait tous, il descendit aussi vers les saints qui demeuraient à Lydde.
\VS{33}Il y vint aussi un homme appelé Enée, qui était couché dans un petit lit depuis huit ans, car il était paralytique.
\VS{34}Et Pierre lui dit~: Enée, Jésus-Christ te guérit~! Lève-toi et arrange ton lit. Et aussitôt il se leva.
\VS{35}Tous ceux qui habitaient à Lydde et à Saron le virent, et ils se convertirent au Seigneur.
\TextTitle{Résurrection de Tabitha}
\VS{36}Il y avait à Joppé une femme disciple, appelée Tabitha, qui signifie en grec Dorcas~; elle faisait beaucoup de bonnes œuvres et d'aumônes.
\VS{37}Elle tomba malade en ce temps-là, et mourut. Après l'avoir lavée, on la déposa dans une chambre haute.
\VS{38}Comme Lydde était près de Joppé, les disciples ayant appris que Pierre était à Lydde, ils envoyèrent vers lui deux hommes, pour le prier de venir chez eux sans tarder.
\VS{39}Pierre se leva, et partit avec ces hommes. Lorsqu'il fut arrivé, on le conduisit dans la chambre haute. Toutes les veuves l'entourèrent en pleurant, et lui montrèrent les tuniques et les vêtements que faisait Dorcas quand elle était avec elles.
\VS{40}Pierre fit sortir tout le monde, se mit à genoux, et pria~; puis se tournant vers le corps, il dit~: Tabitha, lève-toi~! Et elle ouvrit ses yeux, et voyant Pierre, elle s'assit.
\VS{41}Il lui donna la main, et la fit lever. Puis ayant appelé les saints et les veuves, il la leur présenta vivante.
\VS{42}Cela fut connu dans tout Joppé~; et plusieurs crurent au Seigneur.
\VS{43}Et il arriva qu'il demeura plusieurs jours à Joppé, chez un corroyeur nommé Simon.
\Chap{10}
\TextTitle{Un ange de Dieu apparait à Corneille}
\VerseOne{}Il y avait à Césarée un homme nommé Corneille, centenier d'une cohorte de la légion appelée Italienne.
\VS{2}Cet homme était pieux et craignait Dieu avec toute sa famille. Il faisait aussi beaucoup d'aumônes au peuple, et priait Dieu continuellement.
\VS{3}Vers la neuvième heure du jour, il vit clairement dans une vision un ange de Dieu qui entra chez lui, et qui lui dit~: Corneille~!
\VS{4}Corneille ayant les yeux fixés sur lui, et tout effrayé, lui dit~: Qu'y a-t-il Seigneur~? Et il lui dit~: Tes prières et tes aumônes sont montées devant Dieu, et il s'en est souvenu.
\VS{5}Maintenant donc envoie des gens à Joppé, et fais venir Simon, surnommé Pierre.
\VS{6}Il est logé chez un certain Simon, corroyeur, qui a sa maison près de la mer~; c'est lui qui te dira ce qu'il faut que tu fasses.
\VS{7}Dès que l'ange qui lui parlait fut parti, Corneille appela deux de ses serviteurs, et un soldat craignant Dieu, d'entre ceux qui se tenaient près de lui.
\VS{8}Et après leur avoir tout raconté, il les envoya à Joppé.
\TextTitle{Vision de Pierre~: Une nappe descend du ciel}
\VS{9}Le lendemain, comme ils marchaient et qu'ils approchaient de la ville, Pierre monta sur le toit, vers la sixième heure, pour prier.
\VS{10}Et il arriva qu'ayant faim, il voulut prendre son repas. Pendant qu'on lui préparait à manger, il tomba en extase.
\VS{11}Il vit le ciel ouvert, et un vase descendant sur lui semblable à une grande nappe, attachée par les quatre coins, qui descendait vers la terre,
\VS{12}où se trouvaient tous les quadrupèdes, les bêtes sauvages, les reptiles et les oiseaux du ciel.
\VS{13}Et une voix lui dit~: Pierre, lève-toi, tue, et mange.
\VS{14}Mais Pierre répondit~: Non, Seigneur, car je n'ai jamais rien mangé de souillé ni d'impur.
\VS{15}Et la voix lui dit encore pour la seconde fois~: Les choses que Dieu a purifiées, ne les tiens point pour souillées.
\VS{16}Et cela arriva jusqu'à trois fois, et puis le vase fut retiré au ciel.
\VS{17}Comme Pierre ne savait pas en lui-même que penser du sens de la vision qu'il avait eue, voici, les hommes envoyés par Corneille s'étant mis en quête de la maison de Simon, se présentèrent à la porte,
\VS{18}et demandèrent à haute voix si c'était là que logeait Simon, surnommé Pierre.
\VS{19}Et comme Pierre pensait à la vision, l'Esprit lui dit~: Voici trois hommes qui te demandent.
\VS{20}Lève-toi donc et descends, et pars avec eux sans hésiter, car c'est moi qui les ai envoyés.
\VS{21}Pierre donc, descendit vers les gens qui lui avaient été envoyés par Corneille et leur dit~: Voici, je suis celui que vous cherchez~; pour quel sujet êtes-vous venus~?
\VS{22}Et ils dirent~: Corneille, centenier, homme juste et craignant Dieu, et à qui toute la nation des Juifs rend un bon témoignage, a été averti de Dieu par un saint ange de te faire venir dans sa maison et d'entendre tes paroles.
\TextTitle{Pierre chez Corneille}
\VS{23}Alors Pierre les fit entrer, et les logea. Le lendemain il s'en alla avec eux, et quelques-uns des frères de Joppé l'accompagnèrent.
\VS{24}Ils arrivèrent à Césarée le jour suivant. Corneille les attendait, et avait invité ses parents et ses amis.
\VS{25}Lorsque Pierre entra, Corneille qui était allé au-devant de lui, se jeta à ses pieds, et se prosterna.
\VS{26}Mais Pierre le releva en lui disant~: Lève-toi, moi aussi je suis un homme.
\VS{27}Et s'entretenant avec lui, il entra et trouva plusieurs personnes réunies.
\VS{28}Et il leur dit~: Vous savez qu'il n'est pas permis à un homme Juif de se lier avec un étranger, ou d'aller chez lui, mais Dieu m'a montré que je ne devais estimer aucun homme être impur ou souillé.
\VS{29}C'est pourquoi, ayant été appelé, je suis venu sans difficulté. Je vous demande donc pour quel sujet vous m'avez fait venir.
\VS{30}Corneille lui dit~: Il y a quatre jours, à cette heure-ci, j'étais en jeûne et en prière dans ma maison, et tout à coup, un homme, vêtu d'un habit resplendissant, se présenta devant moi et me dit~:
\VS{31}Corneille, ta prière est exaucée, et Dieu s'est souvenu de tes aumônes.
\VS{32}Envoie donc quelqu'un à Joppé, et fais venir Simon, surnommé Pierre, qui est logé dans la maison de Simon, le corroyeur, près de la mer. Quand il sera venu, il te parlera.
\VS{33}Aussitôt j'ai envoyé quelqu'un vers toi, et tu as bien fait de venir. Maintenant donc nous sommes tous présents devant Dieu pour entendre tout ce que Dieu t'a ordonné de nous dire.
\TextTitle{Pierre évangélise les Gentils\FTNTT{\vref{Ac. 2:14-41}}}
\VS{34}Alors Pierre prenant la parole, dit~: En vérité, je reconnais que Dieu n'a point égard à l'apparence des personnes,
\VS{35}mais qu'en toute nation celui qui le craint et qui pratique la justice, lui est agréable.
\VS{36}C'est ce qu'il a fait entendre aux enfants d'Israël, en leur annonçant la paix par Jésus-Christ, qui est le Seigneur de tous.
\VS{37}Vous savez ce qui est arrivé dans toute la Judée, après avoir commencé en Galilée, à la suite du baptême que Jean a prêché~;
\VS{38}vous savez comment Dieu a oint du Saint-Esprit et de force Jésus de Nazareth, qui allait de lieu en lieu, faisant du bien et guérissant tous ceux qui étaient sous l'empire du diable, car Dieu était avec lui.
\VS{39}Nous sommes témoins de toutes les choses qu'il a faites, dans le pays des Juifs et à Jérusalem. Cependant ils l'ont fait mourir en le pendant au bois.
\VS{40}Dieu l'a ressuscité le troisième jour, et il a permis qu'il apparaisse,
\VS{41}non à tout le peuple, mais aux témoins choisis d'avance par Dieu, à nous, qui avons mangé et bu avec lui après qu'il fut ressuscité des morts.
\VS{42}Et il nous a ordonné de prêcher au peuple, et d'attester que c'est lui qui a été établi par Dieu, juge des vivants et des morts.
\VS{43}Tous les prophètes rendent de lui le témoignage que quiconque croit en lui, reçoit la rémission de ses péchés par son Nom.
\TextTitle{Le Saint-Esprit descend sur les Gentils}
\VS{44}Comme Pierre prononçait encore ce discours, le Saint-Esprit descendit sur tous ceux qui écoutaient la parole.
\VS{45}Tous les fidèles circoncis qui étaient venus avec Pierre, furent étonnés de ce que le don du Saint-Esprit était aussi répandu sur les Gentils.
\VS{46}Car ils les entendaient parler diverses langues et glorifier Dieu.
\VS{47}Alors Pierre prenant la parole, dit~: Quelqu'un pourrait-il empêcher qu'on baptise dans l'eau ceux qui ont reçu le Saint-Esprit aussi bien que nous~?
\VS{48}Et il ordonna qu'ils soient baptisés au Nom du Seigneur. Après cela, ils le prièrent de rester quelques jours auprès d'eux.
\Chap{11}
\TextTitle{Dieu accorde la repentance aux Gentils}
\VerseOne{}Or les apôtres et les frères qui étaient en Judée apprirent que les Gentils aussi avaient reçu la parole de Dieu.
\VS{2}Et quand Pierre fut monté à Jérusalem, ceux de la circoncision disputaient contre lui,
\VS{3}disant~: Tu es entré chez des hommes incirconcis, et tu as mangé avec eux.
\VS{4}Alors Pierre commençant, leur exposa le tout par ordre, disant~:
\VS{5}J'étais dans la ville de Joppé, et pendant que je priais, je tombai en extase et j'eus une vision. Un vase semblable à une grande nappe, attachée par les quatre coins, descendit du ciel, et vint jusqu'à moi.
\VS{6}Les regards fixés sur cette nappe, j'examinai, et je vis les quadrupèdes, les bêtes sauvages, les reptiles, et les oiseaux du ciel.
\VS{7}Et j'entendis une voix qui me disait~: Pierre, lève-toi, tue, et mange.
\VS{8}Et je répondis~: Non Seigneur, car jamais rien de souillé ni d'impur n'est entré dans ma bouche.
\VS{9}La voix me parla du ciel une seconde fois~: Ce que Dieu a déclaré pur, ne le regarde pas comme souillé.
\VS{10}Cela arriva jusqu'à trois fois, puis toutes ces choses furent retirées dans le ciel.
\VS{11}Et voici, aussitôt trois hommes qui avaient été envoyés de Césarée vers moi, se présentèrent à la maison où j'étais.
\VS{12}L'Esprit me dit de partir avec eux sans hésiter. Les six frères que voici m'accompagnèrent, et nous entrâmes dans la maison de Corneille.
\VS{13}Cet homme nous raconta comment il avait vu dans sa maison un ange qui s'était présenté à lui, et lui avait dit~: Envoie des gens à Joppé, et fais venir Simon, surnommé Pierre,
\VS{14}qui te dira des choses par lesquelles tu seras sauvé, toi et toute ta maison.
\VS{15}Lorsque je me fus mis à parler, le Saint-Esprit descendit sur eux, comme il était descendu sur nous au commencement.
\VS{16}Et je me souvins de cette parole du Seigneur, et comment il avait dit~: Jean a baptisé d'eau, mais vous, vous serez baptisés du Saint-Esprit.
\VS{17}Or puisque Dieu leur a accordé le même don qu'à nous qui avons cru au Seigneur Jésus-Christ, pouvais-je, moi, m'opposer à Dieu~?
\VS{18}Après avoir entendu ces choses, ils s'apaisèrent, et ils glorifièrent Dieu en disant~: Dieu a donc accordé la repentance aussi aux Gentils, afin qu'ils aient la vie.
\TextTitle{Les disciples appelés «~chrétiens~» pour la première fois à Antioche}
\VS{19}Ceux qui avaient été dispersés par la persécution survenue à cause d'Etienne, allèrent jusqu'en Phénicie, dans l'île de Chypre, et à Antioche\FTNT{Antioche~: Capitale de la Syrie située sur le fleuve Oronte, fondée en 300 av. J.-C., et ainsi nommée en l'honneur de son fondateur Antiochus. De nombreux Juifs grecs y vivaient et c'est là que les disciples de Christ furent appelés pour la première fois, chrétiens.}, n'annonçant la parole à personne, seulement aux Juifs.
\VS{20}Mais il y eut parmi eux quelques hommes de Chypre et de Cyrène qui, étant venus à Antioche, parlèrent aussi aux Grecs, et leur annoncèrent l'Evangile du Seigneur Jésus.
\VS{21}La main du Seigneur était avec eux, et un grand nombre de personnes crurent et se convertirent au Seigneur.
\VS{22}Le bruit en parvint aux oreilles de l'Eglise qui était à Jérusalem, et ils envoyèrent Barnabas jusqu'à Antioche.
\VS{23}Lorsqu'il fut arrivé, et qu'il eut vu la grâce de Dieu, il s'en réjouit, et il les exhortait tous à demeurer attachés au Seigneur de tout leur cœur.
\VS{24}Car c'était un homme de bien, plein du Saint-Esprit et de foi. Et un grand nombre de personnes se joignirent au Seigneur.
\VS{25}Barnabas s'en alla à Tarse pour chercher Saul~;
\VS{26}et l'ayant trouvé, il l'amena à Antioche. Pendant toute une année, ils se réunirent aux assemblées de l'Eglise, et ils enseignèrent beaucoup de personnes. Ce fut à Antioche que, pour la première fois, les disciples furent appelés chrétiens.
\TextTitle{Prophétie d'Agabus}
\VS{27}En ce temps-là, quelques prophètes descendirent de Jérusalem à Antioche.
\VS{28}L'un d'eux, nommé Agabus, se leva et déclara par l'Esprit qu'une grande famine devait arriver sur toute la terre. Elle arriva, en effet, sous Claude César.
\VS{29}Les disciples résolurent d'envoyer, chacun selon ses moyens, quelque secours pour subvenir aux besoins des frères qui habitaient la Judée.
\VS{30}Ils le firent parvenir aux anciens par les mains de Barnabas et de Saul.
\Chap{12}
\TextTitle{Cinquième persécution de l'Eglise~: Meurtre de Jacques et arrestation de Pierre}
\VerseOne{}En ce même temps, le roi Hérode se mit à maltraiter quelques membres de l'Eglise~;
\VS{2}et il fit mourir par l'épée Jacques, frère de Jean.
\VS{3}Voyant que cela était agréable aux Juifs, il fit aussi arrêter Pierre. C'était pendant les jours des pains sans levain.
\VS{4}Après l'avoir saisi et jeté en prison, il le mit sous la garde de quatre bandes de quatre soldats chacune, avec l'intention de le faire comparaître devant le peuple après la fête de Pâque.
\TextTitle{L'Ange du Seigneur délivre Pierre de la prison}
\VS{5}Pierre était donc gardé dans la prison~; mais l'Eglise faisait sans cesse des prières à Dieu pour lui.
\VS{6}La nuit qui précéda le jour où Hérode devait l'envoyer au supplice, Pierre dormait entre deux soldats, lié de deux chaînes~; et les gardes qui étaient devant la porte gardaient la prison.
\VS{7}Et voici, l'Ange du Seigneur survint, et une lumière resplendit dans la prison. L'Ange réveilla Pierre en le frappant au côté, et en disant~: Lève-toi promptement~! Et les chaînes tombèrent de ses mains.
\VS{8}Et l'Ange lui dit~: Mets ta ceinture et tes sandales. Et il fit ainsi. L'Ange lui dit encore~: Enveloppe-toi de ton manteau et suis-moi.
\VS{9}Pierre sortit et le suivit, ne sachant pas que ce qui se faisait par l'Ange était réel, car il croyait qu'il avait une vision.
\VS{10}Lorsqu'ils eurent passé la première et la seconde garde, ils arrivèrent à la porte de fer qui mène à la ville, et qui s'ouvrit d'elle-même devant eux~; et ils sortirent et s'avancèrent dans une rue. Et subitement, l'Ange quitta Pierre.
\VS{11}Revenu à lui-même, Pierre dit~: Je vois à présent d'une manière certaine que le Seigneur a envoyé son Ange, et qu'il m'a délivré de la main d'Hérode, et de toute l'attente du peuple Juif.
\VS{12}Après avoir réfléchi, il alla à la maison de Marie, mère de Jean, surnommé Marc, où plusieurs personnes étaient assemblées et priaient.
\VS{13}Il frappa à la porte du vestibule, une servante, appelée Rhode, vint pour écouter.
\VS{14}Elle reconnut la voix de Pierre, et dans sa joie elle n'ouvrit pas la porte du vestibule, mais elle courut dans la maison et annonça que Pierre était devant la porte.
\VS{15}Ils lui dirent~: Tu es folle. Mais elle affirma que ce qu'elle disait était vrai.
\VS{16}Et ils dirent~: C'est son ange. Cependant Pierre continuait à frapper. Et quand ils eurent ouvert, ils le virent, et furent étonnés de le voir.
\VS{17}Mais leur ayant fait signe de la main de se taire, il leur raconta comment le Seigneur l'avait fait sortir de la prison, et il leur dit~: Annoncez ces choses à Jacques et aux frères. Puis sortant de là il s'en alla dans un autre lieu.
\VS{18}Quand il fit jour, les soldats furent dans une grande agitation, pour savoir ce que Pierre était devenu.
\VS{19}Et Hérode l'ayant cherché, et ne le trouvant point, après en avoir fait le procès aux gardes, il commanda qu'ils fussent menés au supplice.
\TextTitle{Mort d'Hérode}
\VS{20}Hérode avait le dessein de faire la guerre aux Tyriens et aux Sidoniens~; mais ils vinrent le trouver d'un commun accord~; et ayant gagné Blaste, son Chambellan, ils demandèrent la paix, parce que leur pays tirait sa subsistance de celui du roi.
\VS{21}A un jour marqué, Hérode, revêtu de ses habits royaux, s'assit sur son trône et les harangua publiquement.
\VS{22}Le peuple s'écria~: Voix d'un dieu et non point d'un homme~!
\VS{23}Et à l'instant l'Ange du Seigneur le frappa, parce qu'il n'avait pas donné gloire à Dieu. Et il expira, rongé des vers.
\VS{24}Cependant la parole de Dieu se répandait de plus en plus, et le nombre des disciples augmentait.
\VS{25}Barnabas et Saul, après s'être acquittés de leur service, s'en retournèrent de Jérusalem, ayant aussi pris avec eux Jean, surnommé Marc.
\Chap{13}
\TextTitle{Saul et Barnabas mis à part par le Saint-Esprit}
\VerseOne{}Or il y avait dans l'église qui était à Antioche des prophètes et des docteurs, Barnabas, Siméon, appelé Niger, Lucius, le Cyrénien, Manahen, qui avait été élevé avec Hérode, le tétrarque, et Saul.
\VS{2}Et tandis qu'ils servaient\FTNT{Certains traducteurs ont rajouté la phrase «~dans leur ministère~» alors que les textes originaux ne la mentionne pas.} le Seigneur et jeûnaient, le Saint-Esprit dit~: Séparez-moi maintenant Barnabas et Saul pour l'œuvre à laquelle je les ai appelés.
\VS{3}Alors après avoir jeûné et prié, ils leur imposèrent les mains, et les laissèrent partir\FTNT{Voir annexe «~Les voyages missionnaires de Paul~».}.
\TextTitle{Saul, Barnabas et Jean sur l'île de Chypre}
\VS{4}Barnabas et Saul, envoyés par le Saint-Esprit, descendirent à Séleucie, et de là ils s'embarquèrent pour l'île de Chypre.
\VS{5}Et lorsqu'ils furent à Salamine, ils annoncèrent la parole de Dieu dans les synagogues des Juifs~; ils avaient Jean avec eux pour les aider.
\TextTitle{Bar-Jésus aveuglé et conversion du proconsul Sergius Paulus}
\VS{6}Ayant ensuite traversé l'île jusqu'à Paphos, ils trouvèrent là un certain magicien, faux prophète Juif, nommé Bar-Jésus,
\VS{7}qui était avec le proconsul Sergius Paulus, homme intelligent qui fit appeler Barnabas et Saul, désirant entendre la parole de Dieu.
\VS{8}Mais Elymas, le magicien, car c'est ce que signifie ce nom, leur résistait, cherchant à détourner de la foi le proconsul.
\VS{9}Alors Saul, appelé aussi Paul, rempli du Saint-Esprit, fixa les yeux sur lui et dit~:
\VS{10}Ô homme plein de toute fraude et de toute ruse, fils du diable, ennemi de toute justice, ne cesseras-tu point de renverser les voies droites du Seigneur~?
\VS{11}C'est pourquoi, voici la main du Seigneur est sur toi, tu seras aveugle, et pour un temps tu ne verras pas le soleil. Aussitôt l'obscurité et les ténèbres tombèrent sur lui, et il cherchait, en tâtonnant, des personnes pour le guider.
\VS{12}Alors le proconsul voyant ce qui était arrivé, crut, étant rempli d'admiration pour la doctrine du Seigneur.
\VS{13}Et quand Paul et ceux qui étaient avec lui furent partis de Paphos, ils vinrent à Perge, ville de Pamphylie. Jean se sépara d'eux et retourna à Jérusalem.
\TextTitle{Paul à Antioche de Pisidie~: Le salut par la foi en Jésus}
\VS{14}De Perge, ils poursuivirent leur route, et arrivèrent à Antioche, ville de Pisidie\FTNT{Antioche de Pisidie~: Ville de Pisidie (en Turquie), à la frontière de Phrygie, fondée par Seleucus Nicanor. Elle devint une colonie romaine et fut aussi appelée Césarée.}, et étant entrés dans la synagogue le jour du sabbat, ils s'assirent.
\VS{15}Après la lecture de la loi et des prophètes, les chefs de la synagogue leur envoyèrent dire~: Hommes frères, si vous avez quelque parole d'exhortation pour le peuple, dites-la.
\VS{16}Alors Paul s'étant levé, et ayant fait signe de la main qu'on fasse silence, dit~: Hommes Israélites, et vous qui craignez Dieu, écoutez.
\VS{17}Le Dieu de ce peuple d'Israël a choisi nos pères. Il a distingué glorieusement ce peuple pendant son séjour au pays d'Egypte, et il l'en fit sortir par son bras élevé.
\VS{18}Il les supporta\FTNT{Le verbe supporter vient du grec «~tropophoreo~» qui signifie supporter les manières, endurer le caractère de quelqu'un.} au désert environ quarante ans.
\VS{19}Et ayant détruit sept nations au pays de Canaan, il leur distribua le pays par le sort.
\VS{20}Après cela, durant quatre cent cinquante ans, il leur donna des juges, jusqu'à Samuel le prophète.
\VS{21}Puis ils demandèrent un roi, et Dieu leur donna Saül fils de Kis, homme de la tribu de Benjamin~; et ainsi se passèrent quarante ans.
\VS{22}Et Dieu l'ayant rejeté, il leur suscita pour roi David, auquel il a rendu ce témoignage~: J'ai trouvé David, fils d'Isaï, homme selon mon cœur, qui exécutera toute ma volonté.
\VS{23}C'est de la postérité de David que Dieu, selon sa promesse, a suscité Jésus pour être le Sauveur d'Israël.
\VS{24}Avant la venue de Jésus, Jean avait prêché le baptême de repentance à tout le peuple d'Israël.
\VS{25}Et comme Jean achevait sa course, il disait~: Qui pensez-vous que je sois~? Je ne suis point le Christ~; mais voici, il en vient un après moi, dont je ne suis pas digne de délier le soulier de ses pieds.
\VS{26}Hommes frères, fils de la race d'Abraham, et vous qui craignez Dieu, c'est à vous que la parole de ce salut a été envoyée.
\VS{27}Car les habitants de Jérusalem et leurs chefs ont méconnu Jésus, et en le condamnant, ils ont accompli les paroles des prophètes qui se lisent chaque sabbat.
\VS{28}Quoiqu'ils n'aient rien trouvé en lui qui soit digne de mort, ils demandèrent à Pilate de le faire mourir.
\VS{29}Et après qu'ils eurent accompli toutes les choses qui avaient été écrites de lui, ils le descendirent du bois, et le déposèrent dans un sépulcre.
\VS{30}Mais Dieu l'a ressuscité des morts.
\VS{31}Il est apparu pendant plusieurs jours à ceux qui étaient montés avec lui de Galilée à Jérusalem, et qui sont ses témoins devant le peuple.
\VS{32}Et nous, nous vous annonçons cette bonne nouvelle que la promesse faite à nos pères,
\VS{33}Dieu l'a accomplie pour nous, leurs enfants, en ressuscitant Jésus, selon qu'il est écrit dans le deuxième psaume~: Tu es mon Fils, je t'ai aujourd'hui engendré\FTNT{\vref{Ps. 2:7}.}.
\VS{34}Et pour montrer qu'il l'a ressuscité des morts, pour ne plus devoir retourner au sépulcre, il a dit ainsi~: Je vous donnerai les grâces saintes promises à David, ces grâces qui sont assurées.
\VS{35}C'est pourquoi il a dit aussi dans un autre endroit~: Tu ne permettras point que ton Saint voie la corruption\FTNT{\vref{Ps. 16:10}.}.
\VS{36}Or David, après avoir servi en son temps au dessein de Dieu, est mort, a été réuni à ses pères, et a vu la corruption.
\VS{37}Mais celui que Dieu a ressuscité n'a pas vu la corruption.
\VS{38}Sachez donc, hommes frères, que c'est par lui que la rémission des péchés vous est annoncée,
\VS{39}et que quiconque croit est justifié par lui, de tout ce dont vous n'avez pas pu être justifiés par la loi de Moïse.
\VS{40}Prenez donc garde qu'il ne vous arrive ce qui est dit dans les prophètes~:
\VS{41}Voyez, vous mépriseurs, soyez étonnés et disparaissez~: Car je vais faire une œuvre en votre temps, une œuvre que vous ne croiriez pas si quelqu'un vous la racontait.
\VS{42}Lorsqu'ils sortirent de la synagogue des Juifs, les Gentils les prièrent de parler le sabbat suivant sur les mêmes choses.
\VS{43}Et quand l'assemblée fut séparée, beaucoup de Juifs et de prosélytes craignant Dieu, suivirent Paul et Barnabas qui les exhortèrent à persévérer dans la grâce de Dieu.
\TextTitle{Les juifs d'Antioche rejettent la Parole~; l'Évangile annoncé aux Gentils\FTNTT{\vref{Ac. 18:6}~; \vref{28:25-28}.}}
\VS{44}Le sabbat suivant, presque toute la ville s'assembla pour entendre la parole de Dieu.
\VS{45}Mais les Juifs voyant toute cette foule, furent remplis de jalousie, et ils s'opposaient à ce que Paul disait, en le contredisant et en blasphémant.
\VS{46}Alors Paul et Barnabas leur dirent avec assurance~: C'est à vous premièrement qu'il fallait annoncer la parole de Dieu, mais puisque vous la rejetez, et que vous vous jugez vous-mêmes indignes de la vie éternelle, voici, nous nous tournons vers les Gentils.
\VS{47}Car ainsi nous l'a ordonné le Seigneur~: Je t'ai établi pour être la lumière des Gentils, pour porter le salut jusqu'aux extrémités de la terre.
\VS{48}Les Gentils en entendant cela, se réjouissaient et ils glorifiaient la parole du Seigneur~; et tous ceux qui étaient destinés à la vie éternelle crurent.
\VS{49}Ainsi la parole du Seigneur se répandait dans tout le pays.
\VS{50}Mais les Juifs excitèrent quelques femmes dévotes et distinguées, et les principaux de la ville, et ils provoquèrent une persécution contre Paul et Barnabas, et les chassèrent de leur territoire.
\VS{51}Paul et Barnabas secouèrent contre eux la poussière de leurs pieds et allèrent à Icone,
\VS{52}tandis que les disciples étaient remplis de joie et du Saint-Esprit.
\Chap{14}
\TextTitle{Paul et Barnabas à Icone}
\VerseOne{}A Icone, Paul et Barnabas entrèrent ensemble dans la synagogue des Juifs, et ils parlèrent d'une telle manière qu'une grande multitude de Juifs et de Grecs crurent.
\VS{2}Mais ceux des Juifs qui furent rebelles, émurent et irritèrent les esprits des Gentils contre les frères.
\VS{3}Ils restèrent cependant assez longtemps à Icone, parlant avec assurance du Seigneur, qui rendait témoignage à la parole de sa grâce, en faisant par leurs mains des prodiges et des miracles.
\VS{4}La population de la ville fut partagée en deux, et les uns étaient du côté des Juifs, et les autres du côté des apôtres.
\TextTitle{Paul prêche à Derbe et à Lystre~; guérison d'un boiteux de naissance}
\VS{5}Et comme il se faisait une émeute des Gentils et des Juifs, avec leurs principaux chefs, pour outrager et lapider les apôtres,
\VS{6}Paul et Barnabas en ayant eu connaissance, se réfugièrent dans les villes de Lycaonie, à Lystre, à Derbe, et dans les contrées d'alentour.
\VS{7}Et ils y annoncèrent l'Evangile.
\VS{8}A Lystre, se tenait assis un homme impotent des pieds, boiteux dès sa naissance, et qui n'avait jamais marché.
\VS{9}Cet homme écoutait parler Paul. Et Paul fixant ses yeux sur lui, et voyant qu'il avait la foi pour être guéri,
\VS{10}lui dit à haute voix~: Lève-toi droit sur tes pieds. Et il se leva en sautant, et marcha.
\VS{11}Et les gens qui étaient là assemblés, ayant vu ce que Paul avait fait, élevèrent leur voix, disant en langue lycaonienne~: Les dieux sous une forme humaine, sont descendus vers nous.
\VS{12}Et ils appelaient Barnabas Jupiter, et Paul Mercure, parce que c'était lui qui portait la parole.
\VS{13}Le prêtre de Jupiter, qui était à l'entrée de leur ville, ayant amené des taureaux et des couronnes jusqu'à l'entrée de la porte voulait, de même que la foule, offrir un sacrifice.
\VS{14}Mais les apôtres Barnabas et Paul ayant appris cela, déchirèrent leurs vêtements et se précipitèrent au milieu de la foule,
\VS{15}et disant~: Ô hommes, pourquoi faites-vous cela~? Nous aussi, nous sommes des hommes, sujets aux mêmes passions que vous, et vous apportant l'Evangile, nous vous exhortons à renoncer à ces choses vaines, pour vous convertir au Dieu vivant, qui a fait le ciel et la terre, la mer, et tout ce qui s'y trouve.
\VS{16}Ce Dieu, dans les siècles passés, a laissé toutes les nations marcher dans leurs voies,
\VS{17}quoiqu'il n'ait cessé de rendre témoignage de ce qu'il est, en faisant du bien, en nous dispensant du ciel les pluies et les saisons fertiles, en nous donnant la nourriture avec abondance, et en remplissant nos cœurs de joie.
\VS{18}A peine purent-ils, par ces paroles, empêcher la foule de leur offrir un sacrifice.
\TextTitle{Paul lapidé à Lystre}
\VS{19}Alors survinrent quelques Juifs d'Antioche et d'Icone qui gagnèrent la foule, et qui après avoir lapidé Paul, le traînèrent hors de la ville, croyant qu'il était mort.
\VS{20}Mais les disciples s'étant assemblés autour de lui, il se leva et entra dans la ville~; et le lendemain il s'en alla avec Barnabas à Derbe.
\TextTitle{Vote et établissement des anciens dans les églises}
\VS{21}Quand ils eurent évangélisé cette ville, et fait un certain nombre de disciples, ils retournèrent à Lystre, à Icone, et à Antioche~;
\VS{22}fortifiant l'esprit des disciples, et les exhortant à persévérer dans la foi, disant que c'est par beaucoup de tribulations qu'il nous faut entrer dans le Royaume de Dieu.
\VS{23}Après le vote à main levée des assemblées, ils établirent des anciens dans chaque église, et après avoir prié et jeûné, ils les recommandèrent au Seigneur, en qui ils avaient cru.
\VS{24}Traversant ensuite la Pisidie, ils allèrent en Pamphylie,
\VS{25}annoncèrent la parole à Perge, et descendirent à Attalie.
\TextTitle{Retour à Antioche}
\VS{26}De là, ils s'embarquèrent pour Antioche, d'où ils avaient été recommandés à la grâce de Dieu, pour l'œuvre qu'ils venaient d'accomplir.
\VS{27}Et quand ils furent arrivés, ils convoquèrent l'église, et ils racontèrent toutes les choses que Dieu avait faites par eux, et comment il avait ouvert aux Gentils la porte de la foi.
\VS{28}Et ils demeurèrent assez longtemps avec les disciples.
\Chap{15}
\TextTitle{Des hommes venus de Judée veulent imposer la circoncision}
\VerseOne{}Quelques hommes qui étaient descendus de Judée, enseignaient les frères en disant~: Si vous n'êtes pas circoncis selon le rite de Moïse, vous ne pouvez pas être sauvés.
\VS{2}Paul et Barnabas eurent avec eux un débat et une vive discussion~; et les frères décidèrent que Paul et Barnabas, avec quelques-uns des leurs, monteraient à Jérusalem vers les apôtres et les anciens, pour traiter cette question.
\VS{3}Après avoir été accompagnés par l'assemblée, ils traversèrent la Phénicie et la Samarie, racontant la conversion des Gentils~; et ils causèrent une grande joie à tous les frères.
\VS{4}Arrivés à Jérusalem, ils furent reçus par l'église, les apôtres et les anciens, et ils racontèrent toutes les choses que Dieu avait faites par leur moyen.
\VS{5}Mais quelques-uns, de la secte des pharisiens qui avaient cru, se levèrent, en disant qu'il fallait circoncire les Gentils et leur ordonner de garder la loi de Moïse.
\TextTitle{Opposition de Pierre~: Les Gentils n'ont pas à être sous le joug de la loi}
\VS{6}Alors les apôtres et les anciens se réunirent pour examiner cette affaire.
\VS{7}Et après une grande discussion, Pierre se leva et leur dit~: Hommes frères, vous savez que depuis longtemps Dieu m'a choisi parmi nous, afin que par ma bouche, les Gentils entendent la parole de l'Evangile, et qu'ils croient.
\VS{8}Et Dieu, qui connaît les cœurs, leur a rendu témoignage en leur donnant le Saint-Esprit, de même qu'à nous.
\VS{9}Il n'a fait aucune différence entre nous et eux, ayant purifié leurs cœurs par la foi.
\VS{10}Maintenant donc pourquoi tentez-vous Dieu en voulant imposer aux disciples un joug que ni nos pères ni nous n'avons pu porter~?
\VS{11}Mais nous croyons que nous serons sauvés par la grâce du Seigneur Jésus-Christ, comme eux aussi.
\VS{12}Alors toute l'assemblée garda le silence, et l'on écouta Barnabas et Paul qui racontèrent tous les miracles et les prodiges que Dieu avait faits par leur moyen au milieu des Gentils.
\TextTitle{Discours de Jacques~: Les prophètes ont annoncé le salut pour les Gentils} 
\VS{13}Lorsqu'ils eurent cessé de parler, Jacques prit la parole et dit~: Hommes frères, écoutez-moi~!
\VS{14}Simon a raconté comment Dieu a premièrement jeté les regards sur les nations pour choisir du milieu d'elles un peuple consacré à son Nom. 
\VS{15}Et avec cela s'accordent les paroles des prophètes, selon qu'il est écrit~:
\VS{16}Après cela, je reviendrai, et je rebâtirai le tabernacle de David qui est tombé, je réparerai ses ruines et je le relèverai\FTNT{\vref{Am. 9:11}.}.
\VS{17}Afin que le reste des hommes recherche le Seigneur, et aussi toutes les nations sur lesquelles mon Nom est invoqué, dit le Seigneur, qui fait toutes ces choses.
\VS{18}Toutes les œuvres de Dieu lui sont connues de toute éternité.
\TextTitle{Les chrétiens issus des nations ne sont pas soumis à la loi mosaïque}
\VS{19}C'est pourquoi je suis d'avis qu'on ne crée pas des difficultés à ceux des Gentils qui se convertissent à Dieu~;
\VS{20}mais qu'on leur écrive de s'abstenir des souillures des idoles et de la fornication, des animaux étouffés et du sang.
\VS{21}Car depuis bien des générations, Moïse a, dans chaque ville, des gens qui le prêchent, puisqu'on le lit tous les jours de sabbat dans les synagogues.
\VS{22}Alors il parut bon aux apôtres et aux anciens avec toute l'Eglise, de choisir parmi eux et d'envoyer à Antioche avec Paul et Barnabas, Jude, appelé Barsabas, et Silas, hommes considérés entre les frères.
\VS{23}Ils écrivirent par eux en ces termes~: Les apôtres, les anciens, et les frères, aux frères d'entre les Gentils qui sont à Antioche, en Syrie, et en Cilicie, salut~!
\VS{24}Ayant appris que quelques hommes partis de chez nous, et auxquels nous n'avons donné aucun ordre, vous ont troublés par leurs discours et ont ébranlé vos âmes, en vous disant qu'il faut être circoncis et garder la loi,
\VS{25}nous avons été d'avis, étant assemblés tous d'un commun accord, d'envoyer vers vous, avec nos très chers Barnabas et Paul, des hommes que nous avons choisis. 
\VS{26}Ce sont des hommes qui ont abandonné leurs vies pour le Nom de notre Seigneur Jésus-Christ.
\VS{27}Nous avons donc envoyé Jude et Silas, qui vous feront entendre les mêmes choses de vive voix.
\VS{28}Car il a paru bon au Saint-Esprit et à nous, de ne vous imposer d'autre charge que ce qui est nécessaire,
\VS{29}savoir, de vous abstenir des viandes sacrifiées aux idoles, du sang, des animaux étouffés, et de la fornication~; choses contre lesquelles vous vous trouverez bien de vous tenir en garde. Adieu~!
\TextTitle{Mission de Jude et Silas à Antioche}
\VS{30}Après avoir donc pris congé de l'église, ils allèrent à Antioche, et ayant assemblé l'église, ils remirent la lettre.
\VS{31}Après l'avoir lue, les frères d'Antioche furent réjouis de la consolation qu'elle leur apportait.
\VS{32}Jude et Silas, qui étaient eux-mêmes prophètes, exhortèrent les frères par plusieurs discours, et les fortifièrent.
\VS{33}Au bout de quelque temps, ils furent renvoyés en paix par les frères vers les apôtres.
\VS{34}Toutefois Silas trouva bon de rester.
\VS{35}Et Paul et Barnabas demeurèrent aussi à Antioche, enseignant et annonçant, avec plusieurs autres, la parole du Seigneur.
\TextTitle{Paul et Barnabas se séparent}
\VS{36}Quelques jours après, Paul dit à Barnabas~: Retournons visiter nos frères dans toutes les villes où nous avons annoncé la parole du Seigneur, pour voir quel est leur état\FTNT{Voir annexe «~Les voyages missionaires de Paul~».}.
\VS{37}Barnabas voulait emmener avec eux Jean, surnommé Marc.
\VS{38}Mais Paul jugea plus convenable de ne pas prendre avec eux celui qui les avait quittés depuis la Pamphylie, et qui ne les avait point accompagnés dans leur œuvre.
\VS{39}Il y eut donc entre eux une contestation, en sorte qu'ils se séparèrent l'un de l'autre. Barnabas, prenant Marc avec lui, s'embarqua pour l'île de Chypre.
\VS{40}Mais Paul, ayant choisi Silas, pour l'accompagner, partit après avoir été recommandé à la grâce de Dieu par les frères.
\VS{41}Il traversa la Syrie et la Cilicie, fortifiant les églises.
\Chap{16}
\TextTitle{Circoncis, Timothée rejoint Paul dans la mission}
\VerseOne{}Il se rendit à Derbe et à Lystre, et voici, il y avait là un disciple, nommé Timothée, fils d'une femme juive fidèle et d'un père grec.
\VS{2}Les frères de Lystre et d'Icone rendaient de lui un bon témoignage.
\VS{3}C'est pourquoi Paul voulut l'emmener avec lui~; et l'ayant pris, il le circoncit, à cause des Juifs qui étaient dans ces lieux-là, car ils savaient tous que son père était grec.
\VS{4}En passant par les villes, ils recommandaient aux frères d'observer les ordonnances établies par les apôtres et les anciens de Jérusalem.
\VS{5}Ainsi les églises étaient affermies dans la foi et augmentaient en nombre chaque jour.
\TextTitle{Vision de Paul}
\VS{6}Ayant traversé la Phrygie et le pays de Galatie, le Saint-Esprit leur défendit d'annoncer la parole dans l'Asie.
\VS{7}Arrivés près de la Mysie, ils se disposaient à entrer en Bithynie~; mais l'Esprit de Jésus\FTNT{Notons que le Saint-Esprit est appelé Esprit de Jésus. Ainsi, de la même manière qu'on ne peut dissocier un homme de son esprit pour en faire deux entités distinctes, on ne peut dissocier Jésus de son Esprit. Dieu est un.} ne le leur permit pas.
\VS{8}Ils traversèrent ensuite la Mysie, et descendirent à Troas.
\VS{9}Pendant la nuit, Paul eut une vision d'un homme macédonien qui se présenta devant lui, et le pria, disant~: Passe en Macédoine et secours-nous~!
\VS{10}Après cette vision de Paul, nous cherchâmes aussitôt à nous rendre en Macédoine, concluant que le Seigneur nous appelait à les évangéliser.
\TextTitle{Paul à Philippes}
\VS{11}Ainsi étant partis de Troas, nous fîmes voile directement vers la Samothrace, et le lendemain à Néapolis.
\VS{12}De là nous allâmes à Philippes, qui est la première ville d'un district de Macédoine, et une colonie romaine. Nous séjournâmes quelque temps dans la ville.
\VS{13}Et le jour du sabbat nous sortîmes de la ville, et allâmes au lieu où on avait accoutumé de faire la prière, près du fleuve, et nous étant là assis nous parlâmes aux femmes qui y étaient assemblées.
\TextTitle{Conversion de Lydie}
\VS{14}L'une d'elles, appelée Lydie, marchande de pourpre, de la ville de Thyatire, était une femme craignant Dieu, et elle nous écoutait. Le Seigneur lui ouvrit le cœur, afin qu'elle soit attentive à ce que disait Paul.
\VS{15}Lorsqu'elle eut été baptisée, avec sa famille, elle nous fit cette demande~: Si vous me jugez fidèle au Seigneur, entrez dans ma maison, et demeurez-y. Et elle nous pressa par ses instances.
\TextTitle{Paul et Silas battus de verges et mis en prison}
\VS{16}Or il arriva que comme nous allions à la prière, une servante qui avait un esprit de python, et qui, en devinant, apportait un grand profit à ses maîtres, nous rencontra,
\VS{17}et elle se mit à nous suivre, Paul et nous, en criant et disant~: Ces hommes sont les serviteurs du Dieu Très-Haut, et ils vous annoncent la voie du salut~!
\VS{18}Elle fit cela pendant plusieurs jours. Mais Paul, fatigué, se retourna et dit à l'esprit~: Je t'ordonne au Nom de Jésus-Christ de sortir de cette fille. Et il sortit au même instant.
\VS{19}Mais les maîtres de la servante voyant disparaître l'espoir de leur gain, se saisirent de Paul et de Silas, et les traînèrent sur la place publique devant les magistrats.
\VS{20}Ils les présentèrent aux préteurs, en disant~: Ces hommes, qui sont Juifs, troublent notre ville.
\VS{21}Car ils annoncent des coutumes qu'il ne nous est pas permis de recevoir ni de suivre, à nous qui sommes Romains.
\VS{22}La foule se souleva aussi contre eux, et les préteurs, ayant fait déchirer leurs vêtements, ordonnèrent qu'ils soient battus de verges.
\VS{23}Après qu'on les eut chargés de coups de fouet, ils les mirent en prison, en recommandant au geôlier de les garder sûrement.
\VS{24}Le geôlier ayant reçu cet ordre, les mit au fond de la prison, et leur serra les pieds dans des ceps.
\TextTitle{Libération miraculeuse de Paul et Silas}
\VS{25}Vers minuit, Paul et Silas priaient et chantaient les louanges de Dieu, et les prisonniers les entendaient.
\VS{26}Tout à coup, il se fit un grand tremblement de terre, en sorte que les fondements de la prison furent ébranlés~; au même instant, toutes les portes s'ouvrirent et les liens de tous furent rompus.
\VS{27}Le geôlier se réveilla, et voyant les portes de la prison ouvertes, il tira son épée et allait se tuer, croyant que les prisonniers s'étaient enfuis.
\VS{28}Mais Paul cria d'une voix forte~: Ne te fais point de mal, nous sommes tous ici.
\TextTitle{Conversion et baptême du geôlier et de sa famille}
\VS{29}Alors le geôlier, ayant demandé de la lumière, entra précipitamment dans le cachot, et se jeta tout tremblant aux pieds de Paul et de Silas.
\VS{30}Il les fit sortir, et dit~: Seigneur, que faut-il que je fasse pour être sauvé~?
\VS{31}Paul et Silas répondirent~: Crois au Seigneur Jésus-Christ et tu seras sauvé, toi et ta famille.
\VS{32}Et ils lui annoncèrent la parole du Seigneur, et à tous ceux qui étaient dans sa maison.
\VS{33}Après cela, les prenant en cette même heure de la nuit, il lava leurs plaies, et aussitôt après il fut baptisé, avec tous ceux de sa maison.
\VS{34}Les ayant amenés dans sa maison, il leur servit à manger, et il se réjouit avec toute sa famille de ce qu'il avait cru en Dieu.
\TextTitle{Paul et Silas relâchés}
\VS{35}Quand il fit jour, les préteurs envoyèrent des huissiers pour dire au geôlier~: Relâche ces hommes.
\VS{36}Et le geôlier rapporta ces paroles à Paul, disant~: Les préteurs ont envoyé dire qu'on vous relâche~; maintenant donc sortez, et allez en paix.
\VS{37}Mais Paul dit aux huissiers~: Après nous avoir battus de verges publiquement et sans jugement, nous qui sommes Romains, ils nous ont jetés en prison, et maintenant ils nous font sortir secrètement~! Il n'en sera pas ainsi. Qu'ils viennent eux-mêmes nous mettre en liberté.
\VS{38}Les licteurs rapportèrent ces paroles aux préteurs qui furent effrayés en apprenant qu'ils étaient Romains.
\VS{39}Ils vinrent vers eux et leur firent des excuses, et ils les mirent en liberté en les priant de quitter la ville.
\VS{40}Quand ils furent sortis de la prison, ils entrèrent chez Lydie, et après avoir vu et consolé les frères, ils partirent.
\Chap{17}
\TextTitle{Paul et Silas à Thessalonique}
\VerseOne{}Paul et Silas passèrent par Amphipolis et par Apollonie, et ils arrivèrent à Thessalonique, où les Juifs avaient une synagogue.
\VS{2}Paul y entra, selon sa coutume. Pendant trois sabbats, il discuta avec eux d'après les Ecritures~;
\VS{3}expliquant et établissant que le Christ devait souffrir et ressusciter des morts. Et ce Jésus, que je vous annonce, disait-il, c'est lui qui est le Christ.
\VS{4}Quelques-uns d'entre eux crurent, et se joignirent à Paul et à Silas, ainsi qu'une grande multitude de Grecs craignant Dieu, et beaucoup de femmes de qualité.
\TextTitle{Emeute à Thessalonique}
\VS{5}Mais les Juifs rebelles et jaloux, prirent avec eux quelques hommes méchants et fainéants de la populace, provoquèrent des attroupements, et répandirent l'agitation dans la ville. Ils se rendirent à la maison de Jason, et ils cherchèrent Paul et Silas, pour les amener vers le peuple.
\VS{6}Ne les ayant pas trouvés, ils traînèrent Jason et quelques frères devant les magistrats de la ville, en criant~: Ces gens, qui ont bouleversé le monde, sont aussi venus ici, et Jason les a reçus chez lui.
\VS{7}Ils sont tous rebelles aux édits de César, disant qu'il y a un autre Roi, qu'ils nomment Jésus.
\VS{8}Ils soulevèrent donc le peuple et les magistrats de la ville, qui, entendant ces choses,
\VS{9}ne laissèrent aller Jason et les autres qu'après avoir obtenu d'eux une caution. 
\TextTitle{Paul et Silas fuient à Bérée}
\VS{10}Aussitôt les frères firent partir de nuit Paul et Silas pour Bérée. Lorsqu'ils furent arrivés, ils entrèrent dans la synagogue des Juifs.
\VS{11}Ces Juifs avaient des sentiments plus nobles que ceux de Thessalonique~; ils reçurent la parole avec beaucoup de promptitude, et ils examinaient tous les jours les Ecritures, pour voir si ce qu'on leur disait était exact.
\VS{12}Plusieurs d'entre eux crurent, ainsi que des femmes grecques de distinction, et des hommes en assez grand nombre.
\VS{13}Mais quand les Juifs de Thessalonique surent que Paul annonçait aussi à Bérée la parole de Dieu, ils vinrent y agiter la foule.
\VS{14}Alors les frères firent aussitôt partir Paul du côté de la mer~; Silas et Timothée restèrent à Bérée.
\VS{15}Ceux qui avaient pris la charge de mettre Paul en sûreté, le conduisirent jusqu'à Athènes. Puis ils s'en retournèrent, après avoir reçu l'ordre de Paul de dire à Silas et à Timothée de le rejoindre au plus tôt.
\TextTitle{Paul à Athènes}
\VS{16}Comme Paul les attendait à Athènes, il sentit au-dedans son esprit s'irriter à la vue de cette ville entièrement adonnée à l'idolâtrie.
\VS{17}Il s'entretenait donc dans la synagogue avec les Juifs et les hommes craignant Dieu, et tous les jours sur la place publique avec ceux qui s'y rencontraient.
\VS{18}Quelques philosophes épicuriens\FTNT{L'épicurisme a été fondé par Epicure (341 av. J.-C. - 270 av. J.-C.). Cette philosophie est axée sur la recherche du bonheur par l'évitement de la souffrance et des inquiétudes (ataraxie).} et stoïciens\FTNT{Les stoïciens étaient disciples de Zénon (336-264 av. J.-C.). Leur philosophie se fondait sur la conception d'un homme se suffisant à lui-même, sur une discipline rigoureuse, et sur la solidarité du genre humain.} se mirent à parler avec lui. Et les uns disaient~: Que veut dire ce discoureur~? Les autres disaient~: Il semble qu'il annonce des divinités étrangères. Parce qu'il leur annonçait Jésus et la résurrection.
\VS{19}Alors ils le prirent et le menèrent à l'Aréopage\FTNT{A l'origine, l'aréopage désignait le tribunal d'Athènes qui siégeait sur la colline d'Arès. Le sens figuré est le suivant~: Assemblée de juges, de savants, d'hommes de lettres très compétents.}, et lui dirent~: Pourrions-nous savoir quelle est cette nouvelle doctrine que tu enseignes~?
\VS{20}Car tu nous remplis les oreilles de certaines choses étranges~; nous voudrions donc savoir ce que veulent dire ces choses.
\VS{21}Or tous les Athéniens et les étrangers qui demeuraient à Athènes, ne passaient leur temps qu'à dire ou à écouter des nouvelles.
\TextTitle{Prédication de Paul à l'Aréopage}
\VS{22}Paul, debout au milieu de l'Aréopage, leur dit~: Hommes Athéniens, je vous trouve à tous égards extrêmement religieux.
\VS{23}Car en passant et en regardant vos divinités, j'ai même trouvé un autel sur lequel était écrit~: Au Dieu inconnu~! Celui que vous révérez sans le connaître, c'est celui que je vous annonce.
\VS{24}Le Dieu qui a fait le monde et tout ce qui s'y trouve, étant le Seigneur du ciel et de la terre, n'habite point dans des temples faits de main d'homme.
\VS{25}Il n'est point servi par les mains des hommes, comme s'il avait besoin de quoi que ce soit, lui qui donne à tous la vie, la respiration, et toutes choses.
\VS{26}Il a fait que tous les hommes, sortis d'un seul sang, habitent sur toute l'étendue de la terre, ayant déterminé la durée des temps et les bornes de leur habitation.
\VS{27}Il a voulu qu'ils cherchent le Seigneur, et qu'ils s'efforcent de le trouver en tâtonnant, quoiqu'il ne soit pas loin de chacun de nous,
\VS{28}car c'est par lui que nous avons la vie, le mouvement et l'être. C'est ce qu'ont dit quelques-uns même de vos poètes~: De lui nous sommes la race.
\VS{29}Ainsi donc, étant de la race de Dieu, nous ne devons pas croire que la divinité soit semblable à de l'or, ou à de l'argent, ou à de la pierre taillée par l'art et l'industrie des hommes.
\VS{30}Mais Dieu, sans tenir compte des temps d'ignorance, annonce maintenant à tous les hommes en tous lieux qu'ils se repentent,
\VS{31}parce qu'il a arrêté un jour où il jugera le monde selon la justice, par l'homme qu'il a établi pour cela, ce dont il a donné à tous une preuve certaine, en le ressuscitant des morts.
\VS{32}Lorsqu'ils entendirent parler de la résurrection des morts, les uns se moquèrent, et les autres dirent~: Nous t'entendrons là-dessus une autre fois.
\VS{33}Ainsi Paul se retira du milieu d'eux.
\VS{34}Quelques-uns néanmoins se joignirent à lui et crurent~: Denys, juge de l'Aéropage, une femme nommée Damaris, et d'autres avec eux.
\Chap{18}
\TextTitle{Paul enseigne à Corinthe pendant un an et demi}
\VerseOne{}Après cela, Paul partit d'Athènes, et se rendit à Corinthe.
\VS{2}Il y trouva un Juif, nommé Aquilas, originaire du Pont, récemment arrivé d'Italie, avec Priscille, sa femme, parce que Claude avait ordonné à tous les Juifs de sortir de Rome. Il s'approcha d'eux,
\VS{3}et comme il était du même métier qu'eux, il demeura chez eux et y travailla. Et leur métier était de faire des tentes.
\VS{4}Paul discourait dans la synagogue chaque sabbat, et il persuadait des Juifs et des Grecs.
\VS{5}Quand Silas et Timothée furent arrivés de Macédoine, Paul étant poussé par l'Esprit, rendait témoignage aux Juifs que Jésus était le Christ.
\VS{6}Mais comme ils s'opposaient à lui et qu'ils blasphémaient, il secoua ses vêtements, et leur dit~: Que votre sang retombe sur votre tête~! J'en suis pur~! Dès maintenant, j'irai vers les Gentils.
\VS{7}Et sortant de là, il entra dans la maison d'un homme appelé Justus, homme craignant Dieu, et dont la maison était contiguë à la synagogue.
\VS{8}Cependant Crispus, le chef de la synagogue, crut au Seigneur avec toute sa famille. Et plusieurs Corinthiens qui avaient entendu Paul, crurent aussi, et ils furent baptisés.
\VS{9}Le Seigneur dit à Paul dans une vision pendant la nuit~: Ne crains point, mais parle et ne te tais point,
\VS{10}parce que je suis avec toi, et personne ne mettra la main sur toi pour te faire du mal. Parle, car j'ai un peuple nombreux dans cette ville.
\VS{11}Il y demeura un an et six mois, enseignant parmi eux la parole de Dieu.
\TextTitle{Soulèvement des Juifs contre Paul}
\VS{12}Pendant que Gallion était proconsul de l'Achaïe, les Juifs se soulevèrent d'un commun accord contre Paul, et le menèrent devant le tribunal,
\VS{13}en disant~: Cet homme incite les gens à servir Dieu d'une manière contraire à la loi.
\VS{14}Et comme Paul voulait ouvrir la bouche pour parler, Gallion dit aux Juifs~: Ô Juifs~! S'il s'agissait de quelque injustice, ou de quelque crime, je vous écouterais patiemment, autant qu'il serait raisonnable.
\VS{15}Mais il s'agit de discussions sur une parole, sur des noms, et sur votre loi, vous y mettrez de l'ordre vous-mêmes, car je ne veux pas être juge de ces choses.
\VS{16}Et il les renvoya du tribunal.
\VS{17}Alors tous les Grecs se saisirent de Sosthène, le chef de la synagogue, le battirent devant le tribunal, sans que Gallion s'en mît en peine.
\TextTitle{Paul fait un vœu\FTNTT{\vref{Ga. 3:23-28}~; \vref{2 Co. 3:7-14}~; \vref{Ro. 6:14}}}
\VS{18}Paul resta encore assez longtemps à Corinthe. Ensuite il prit congé des frères et s'embarqua pour la Syrie, avec Priscille et Aquilas, après s'être fait raser la tête à Cenchrées, car il avait fait un vœu.
\VS{19}Ils arrivèrent à Ephèse, et Paul y laissa ses compagnons. Etant entré dans la synagogue, il s'entretint avec les Juifs,
\VS{20}qui le prièrent de rester encore plus longtemps avec eux.
\VS{21}Mais il n'y consentit point, et il prit congé d'eux en leur disant~: Il faut absolument que je célèbre la fête prochaine à Jérusalem. Je reviendrai vers vous, s'il plaît à Dieu. Ainsi il partit d'Ephèse.
\VS{22}Etant débarqué à Césarée, il monta à Jérusalem, et après avoir salué l'église, il descendit à Antioche.
\VS{23}Et ayant séjourné là quelque temps, il s'en alla, et traversa tout de suite la contrée de Galatie et de Phrygie, fortifiant tous les disciples\FTNT{Voir annexe «~Les voyages missionnaires de Paul~»}.
\TextTitle{Apollos annonce l'Evangile à Ephèse et à Corinthe}
\VS{24}En ce temps-là, un Juif, nommé Apollos, originaire d'Alexandrie, homme éloquent et puissant dans les Ecritures, vint à Ephèse.
\VS{25}Il était en quelque sorte instruit dans la voie du Seigneur, et fervent d'esprit~; il expliquait et enseignait avec exactitude ce qui concerne Jésus, bien qu'il ne connaisse que le baptême de Jean.
\VS{26}Il commança donc à parler avec hardiesse dans la synagogue~; et quand Aquilas et Priscille l'eurent entendu, ils le prirent avec eux, et lui exposèrent plus exactement la voie de Dieu.
\VS{27}Et comme il voulut passer en Achaïe, les frères, qui l'y encouragèrent écrivirent aux disciples de bien le recevoir. Quand il fut arrivé, il aida beaucoup ceux qui avaient cru par la grâce.
\VS{28}Car il réfutait publiquement les Juifs avec une grande véhémence, démontrant par les Ecritures que Jésus était le Christ.
\Chap{19}
\TextTitle{Paul enseigne à Ephèse\FTNTT{v. \vref{9-10}~; \vref{Ac. 20:31}}}
\VerseOne{}Pendant qu'Apollos était à Corinthe, Paul, après avoir parcouru toutes les hautes provinces de l'Asie, arriva à Ephèse. Ayant rencontré quelques disciples, il leur dit~:
\VS{2}Avez-vous reçu le Saint-Esprit quand vous avez cru~? Ils lui répondirent~: Nous n'avons même pas entendu dire qu'il y ait un Saint-Esprit.
\VS{3}Et il leur dit~: De quel baptême donc avez-vous été baptisés~? Ils répondirent~: Du baptême de Jean.
\VS{4}Alors Paul dit~: Il est vrai que Jean a baptisé du baptême de repentance, disant au peuple de croire en celui qui venait après lui, c'est-à-dire en Jésus-Christ.
\VS{5}Après avoir entendu ces choses, ils furent baptisés au Nom du Seigneur Jésus.
\VS{6}Lorsque Paul leur eut imposé les mains, le Saint-Esprit descendit sur eux, et ils parlaient diverses langues et prophétisaient.
\VS{7}Ils étaient en tout environ douze hommes.
\VS{8}Ensuite, Paul entra dans la synagogue où il parla librement. Pendant trois mois, il discourut sur les choses qui concernent le Royaume de Dieu avec persuasion.
\VS{9}Mais comme quelques-uns restaient endurcis et rebelles, décriant devant la multitude la voie du Seigneur, il se retira d'eux, sépara les disciples, et enseigna tous les jours dans l'école d'un nommé Tyrannus.
\VS{10}Cela dura deux ans, de sorte que tous ceux qui habitaient l'Asie, Juifs et Grecs, entendirent la parole du Seigneur Jésus.
\TextTitle{Réveil et prodiges à Ephèse}
\VS{11}Et Dieu faisait des prodiges extraordinaires par les mains de Paul,
\VS{12}au point qu'on appliquait sur les malades des mouchoirs ou des linges qui avaient touché son corps, et ils étaient guéris de leurs maladies, et les esprits malins sortaient.
\VS{13}Alors quelques exorcistes Juifs ambulants essayèrent d'invoquer le Nom du Seigneur Jésus sur ceux qui étaient possédés d'esprits malins, en disant~: Nous vous conjurons par ce Jésus que Paul prêche~!
\VS{14}Ceux qui faisaient cela étaient sept fils de Scéva, un homme Juif, l'un des principaux prêtres.
\VS{15}Mais l'esprit malin leur répondit~: Je connais Jésus, et je sais qui est Paul~; mais vous, qui êtes-vous~?
\VS{16}Et l'homme dans lequel était l'esprit malin se jeta sur eux, se rendit maître de deux d'entre eux, et les maltraita de telle sorte qu'ils s'enfuirent de cette maison nus et blessés.
\VS{17}Cela fut connu de tous les Juifs et de tous les Grecs qui demeuraient à Ephèse~; et ils furent tous saisis de crainte, et le Nom du Seigneur Jésus était glorifié.
\VS{18}Plusieurs de ceux qui avaient cru venaient, confessant et déclarant ce qu'ils avaient fait.
\VS{19}Et un grand nombre de ceux qui s'étaient adonnés à des pratiques magiques, apportèrent leurs livres et les brûlèrent devant tous. On en estima la valeur à cinquante mille pièces d'argent.
\VS{20}Ainsi la parole du Seigneur se répandait sensiblement, et produisait de grands effets.
\VS{21}Après que ces choses se furent passées, Paul se proposa par un mouvement de l'Esprit\FTNT{Paul fut conduit par le Saint-Esprit (\vref{Jn. 3:8}).} d'aller à Jérusalem, en traversant la Macédoine et l'Achaïe. Quand j'y serai allé, se disait-il, il faut aussi que je voie Rome.
\VS{22}Il envoya en Macédoine deux de ceux qui l'assistaient, Timothée et Eraste, et il resta lui-même quelque temps en Asie.
\TextTitle{Emeute suscitée par Démétrius}
\VS{23}Mais en ce temps-là il arriva un grand trouble, à cause de la doctrine.
\VS{24}Car un certain homme, nommé Démétrius, orfèvre, fabriquait de petits temples d'argent de Diane, et apportait beaucoup de profit aux ouvriers du métier.
\VS{25}Il les rassembla, avec ceux du même métier, et dit~: Ô hommes, vous savez que tout notre gain vient de cet ouvrage,
\VS{26}et vous voyez et entendez que, non seulement à Ephèse, mais dans presque toute l'Asie, ce Paul par ses persuasions a détourné beaucoup de monde, en disant que les dieux faits de main d'homme ne sont pas des dieux.
\VS{27}Et il n'y a pas seulement à craindre pour nous que notre métier ne soit décrié, mais même que le temple de la grande Diane ne tombe dans le mépris, et que sa majesté, que toute l'Asie et que le monde entier révère, ne soit anéantie.
\VS{28}Ayant entendu ces choses, ils furent tous remplis de colère, et s'écrièrent, disant~: Grande est la Diane des Ephésiens~!
\VS{29}Et toute la ville fut remplie de confusion~; et ils se jetèrent en foule dans le théâtre, et enlevèrent Gaïus et Aristarque Macédoniens, compagnons de voyage de Paul.
\VS{30}Et comme Paul voulait entrer vers le peuple, les disciples ne le lui permirent point.
\VS{31}Quelques-uns même des Asiarques, qui étaient ses amis, envoyèrent quelqu'un vers lui pour le prier de ne pas se présenter au théâtre.
\VS{32}Les uns criaient d'une manière, les autres d'une autre, car l'assemblée était confuse, et la plupart ne savaient pas pourquoi ils s'étaient assemblés.
\VS{33}Alors Alexandre fut contraint de sortir hors de la foule, les Juifs le poussant en avant~; et Alexandre, faisant signe de la main, voulait présenter quelque excuse au peuple.
\VS{34}Mais quand ils reconnurent qu'il était Juif, tous d'une seule voix crièrent pendant deux heures~: Grande est la Diane des Ephésiens~!
\VS{35}Cependant, le secrétaire de la ville, ayant apaisé la foule, dit~: Hommes éphésiens, quel est celui des hommes qui ignore que la ville d'Ephèse est la gardienne de la grande déesse Diane et de son image tombée de Jupiter\FTNT{Tombée de Jupiter~: C'est-à-dire du ciel.}~?
\VS{36}Cela étant donc incontestable, vous devez vous apaiser et ne rien faire avec précipitation.
\VS{37}Car ces gens que vous avez amenés ne sont ni sacrilèges ni blasphémateurs de votre déesse.
\VS{38}Mais si Démétrius et ses ouvriers ont à se plaindre de quelqu'un, il y a des jours d'audience et des proconsuls~; qu'ils s'appellent en justice les uns les autres~!
\VS{39}Et si vous avez quelque autre chose à réclamer, on pourra en décider dans une assemblée légale.
\VS{40}Car nous risquons d'être accusés de sédition pour ce qui s'est passé aujourd'hui, n'ayant aucune raison pour justifier ce rassemblement. Après ces paroles, il congédia l'assemblée.
\Chap{20}
\TextTitle{Paul annonce l'Evangile en Macédoine et en Grèce}
\VerseOne{}Lorsque le tumulte eut cessé, Paul fit venir les disciples, et après les avoir embrassés, il partit pour aller en Macédoine.
\VS{2}Il parcourut cette contrée, en adressant aux disciples de nombreuses exhortations.
\VS{3}Puis il se rendit en Grèce où il séjourna trois mois. Il était sur le point de s'embarquer pour la Syrie, quand les Juifs lui dressèrent des embûches. Alors il se décida à reprendre la route de la Macédoine.
\VS{4}Il avait pour l'accompagner jusqu'en Asie~: Sopater de Bérée, Aristarque et Second de Thessalonique, Gaïus de Derbe, Timothée, ainsi que Tychique et Trophime, originaires d'Asie.
\VS{5}Ceux-ci prirent les devants et nous attendirent à Troas.
\TextTitle{Paul ressuscite un jeune homme à Troas}
\VS{6}Et nous, ayant levé l'ancre à Philippes, après les jours des pains sans levain, nous arrivâmes au bout de cinq jours auprès d'eux à Troas, et nous y séjournâmes sept jours.
\VS{7}Le premier jour de la semaine, les disciples étant assemblés pour rompre le pain, Paul, qui devait partir le lendemain, leur fit un discours qu'il étendit jusqu'à minuit.
\VS{8}Or il y avait beaucoup de lampes dans la chambre haute où ils étaient assemblés.
\VS{9}Et un jeune homme nommé Eutychus, qui était assis sur une fenêtre, s'endormit profondément pendant le long discours de Paul~; entraîné par le sommeil, il tomba du troisième étage en bas, et quand on voulut le relever, il était mort.
\VS{10}Mais Paul, étant descendu, se pencha sur lui, le prit dans ses bras, et dit~: Ne vous troublez pas, car son âme est en lui.
\VS{11}Quand il fut remonté, il rompit le pain et mangea, et il parla longtemps encore jusqu'au jour. Après quoi il partit.
\VS{12}Ils ramenèrent le jeune homme vivant, et ce fut le sujet d'une grande consolation.
\TextTitle{Passage à Milet}
\VS{13}Pour nous, étant montés sur un navire, nous fîmes voile vers Assos, où nous avions convenu de reprendre Paul, parce qu'il devait faire la route à pied.
\VS{14}Lorsqu'il nous eut rejoints à Assos, nous le prîmes avec nous, et nous allâmes à Mytilène.
\VS{15}Puis étant partis de là, le jour suivant nous abordâmes vis-à-vis de Chios. Le lendemain, nous arrivâmes vers Samos, et nous nous arrêtâmes à Trogyle~; le jour d'après, nous vînmes à Milet.
\VS{16}Car Paul avait résolu de passer devant Ephèse sans s'y arrêter, afin de ne pas perdre de temps en Asie~; parce qu'il se hâtait pour être, si cela lui était possible, à Jérusalem le jour de la Pentecôte.
\TextTitle{Paul exhorte et prend congé des anciens d'Ephèse}
\VS{17}Cependant de Milet, il envoya chercher à Ephèse les anciens de l'Eglise.
\VS{18}Lorsqu'ils furent arrivés vers lui, il leur dit~: Vous savez de quelle manière je me suis toujours conduit avec vous dès le premier jour où je suis entré en Asie~;
\VS{19}servant le Seigneur en toute humilité, avec beaucoup de larmes, et au milieu des épreuves que me suscitaient les embûches des Juifs.
\VS{20}Vous savez que je n'ai rien caché de ce qui vous était utile, et que je n'ai pas craint de vous prêcher et de vous enseigner publiquement et dans les maisons,
\VS{21}prêchant tant aux Juifs qu'aux Grecs la repentance envers Dieu, et la foi en Jésus-Christ, notre Seigneur.
\VS{22}Et maintenant voici, étant lié par l'Esprit, je vais à Jérusalem, ignorant ce qui m'y arrivera~;
\VS{23}seulement, de ville en ville le Saint-Esprit m'avertit que des liens et des tribulations m'attendent.
\VS{24}Mais je ne fais pour moi-même aucun cas de ma vie, comme si elle m'était précieuse, pourvu que j'achève ma course avec joie, et le service que j'ai reçu du Seigneur Jésus, pour rendre témoignage à l'Evangile de la grâce de Dieu.
\VS{25}Et maintenant voici, je sais que vous ne verrez plus mon visage, vous tous au milieu desquels j'ai passé en prêchant le Royaume de Dieu.
\VS{26}C'est pourquoi je vous prends aujourd'hui à témoin que je suis net du sang de tous.
\VS{27}Car je vous ai annoncé tout le conseil de Dieu, sans en rien cacher.
\VS{28}Prenez donc garde à vous-mêmes, et à tout le troupeau sur lequel le Saint-Esprit vous a établis évêques\FTNT{Evêque, «~episcopos~» en grec~: surveillant, gardien. Ce terme désigne la fonction des anciens. Dans la Nouvelle Alliance, les évêques (ou anciens) sont des personnes dont la mission est de veiller au bon fonctionnement des assemblées locales. Jésus-Christ, notre Dieu, est l'Evêque par excellence (\vref{1 Pi. 2:25}).}, pour paître l'Eglise de Dieu, qu'il a acquise par son propre sang.
\VS{29}Car je sais qu'après mon départ, il s'introduira parmi vous des loups très dangereux, qui n'épargneront pas le troupeau,
\VS{30}et qu'il se lèvera du milieu de vous des hommes qui enseigneront des doctrines corrompues dans le but d'attirer les disciples après eux.
\VS{31}C'est pourquoi veillez, vous souvenant que durant l'espace de trois ans, je n'ai cessé nuit et jour d'avertir chacun de vous avec larmes.
\VS{32}Et maintenant, mes frères, je vous recommande à Dieu, et à la parole de sa grâce, à celui qui est puissant pour achever de vous édifier, et pour vous donner l'héritage avec tous les saints.
\VS{33}Je n'ai désiré ni l'argent, ni l'or, ni les vêtements de personne.
\VS{34}Et vous savez vous-mêmes que ces mains ont pourvu à mes besoins et à ceux des personnes qui étaient avec moi.
\VS{35}Je vous ai montré de toutes manières que c'est en travaillant ainsi qu'il faut soutenir les faibles, et se rappeler les paroles du Seigneur Jésus, qui a dit lui-même~: Il y a plus de bénédiction à donner qu'à recevoir\FTNT{\vref{Lu. 14:12}.}.
\VS{36}Après avoir ainsi parlé, il se mit à genoux et il pria avec eux tous.
\VS{37}Alors tous fondirent en larmes, et se jetant au cou de Paul,
\VS{38}ils l'embrassèrent, étant principalement affligés de ce qu'il avait dit qu'ils ne verraient plus son visage. Et ils l'accompagnèrent jusqu'au navire.
\Chap{21}
\TextTitle{L'équipe missionnaire à Tyr~; avertissement de l'Esprit}
\VerseOne{}Nous nous embarquâmes, après nous être séparés d'eux, et nous allâmes directement à Cos, et le jour suivant à Rhodes, et de là à Patara.
\VS{2}Et ayant trouvé un navire qui faisait la traversée vers la Phénicie, nous montâmes et partîmes.
\VS{3}Puis ayant découvert l'île de Chypre, nous la laissâmes à gauche, nous fîmes route vers la Syrie, nous arrivâmes à Tyr, car le navire devait y décharger sa cargaison.
\VS{4}Nous trouvâmes les disciples et nous restâmes là sept jours. Les disciples, poussés par l'Esprit, disaient à Paul de ne pas monter à Jérusalem.
\VS{5}Mais ces jours étant passés, nous partîmes et nous nous acheminâmes pour partir de Tyr, et tous nous accompagnèrent avec leurs femmes et leurs enfants, jusqu'à l'extérieur de la ville. Nous nous mîmes à genoux sur le rivage et nous fîmes la prière.
\VS{6}Et après nous être embrassés les uns les autres, nous montâmes sur le navire, et les autres retournèrent chez eux.
\TextTitle{Escales à Ptolémaïs puis à Césarée~; prophétie d'Agabus}
\VS{7}Et ainsi achevant notre navigation, nous allâmes de Tyr à Ptolémaïs~; et après avoir salué les frères, nous passâmes un jour avec eux.
\VS{8}Nous partîmes le lendemain, et nous arrivâmes à Césarée. Etant entrés dans la maison de Philippe, l'évangéliste, qui était l'un des sept, nous restâmes chez lui.
\VS{9}Il avait quatre filles vierges qui prophétisaient.
\VS{10}Comme nous étions là depuis plusieurs jours, un prophète, nommé Agabus, arriva de Judée
\VS{11}et vint nous trouver. Il prit la ceinture de Paul, se lia les mains et les pieds, et il dit~: Voici ce que déclare le Saint-Esprit~: L'homme à qui appartient cette ceinture, les Juifs le lieront de la même manière à Jérusalem, et le livreront entre les mains des Gentils.
\VS{12}Quand nous entendîmes ces choses, nous et ceux de l'endroit, nous priâmes Paul de ne pas monter à Jérusalem.
\VS{13}Mais Paul répondit~: Que faites-vous en pleurant et en affligeant mon cœur~? Je suis prêt, non seulement à être lié, mais aussi à mourir à Jérusalem pour le Nom du Seigneur Jésus.
\VS{14}Comme il ne se laissait pas persuader, nous n'insistâmes pas, et nous dîmes~: Que la volonté du Seigneur soit faite~!
\TextTitle{QUATRIEME VOYAGE~: DE JERUSALEM A ROME}
\TextTitle{Arrivée à Jérusalem~; accueil des anciens}
\VS{15}Quelques jours après, nous fîmes nos préparatifs et nous montâmes à Jérusalem \FTNT{Voir annexe «~Les voyages missionnaires de Paul~»}.
\VS{16}Quelques disciples de Césarée vinrent avec nous, amenant avec eux un homme appelé Mnason, de l'île de Chypre, disciple de longue date, chez qui nous devions loger.
\VS{17}Lorsque nous arrivâmes à Jérusalem, les frères nous reçurent avec joie.
\VS{18}Et le jour suivant, Paul se rendit avec nous chez Jacques, et tous les anciens s'y réunirent.
\VS{19}Après les avoir embrassés, il raconta en détail les choses que Dieu avait faites au milieu des Gentils par son service.
\VS{20}Quand ils l'eurent entendu, ils glorifièrent le Seigneur. Puis ils dirent à Paul~: Tu vois frère, combien de milliers de Juifs ont cru~; mais ils sont tous zélés pour la loi.
\VS{21}Or ils ont appris que tu enseignes à tous les Juifs qui sont parmi les Gentils, à renoncer à Moïse, en leur disant qu'ils ne doivent pas circoncire leurs enfants et de ne pas vivre selon les ordonnances de la loi.
\VS{22}Que faut-il donc faire~? Il faut absolument rassembler la multitude des fidèles, car ils apprendront que tu es venu.
\VS{23}C'est pourquoi fais ce que nous allons te dire~: Nous avons quatre hommes qui ont fait un vœu,
\VS{24}prends-les avec toi, purifie-toi avec eux, et pourvois à leurs besoins afin qu'ils se rasent la tête. Et ainsi tous sauront que ce qu'ils ont entendu sur ton compte est faux, mais que toi aussi tu te conduis en observateur de la loi.
\VS{25}A l'égard des Gentils qui ont cru, nous avons décidé et nous leur avons écrit qu'ils doivent s'abstenir des viandes sacrifiées aux idoles, du sang, des animaux étouffés, et de la débauche.
\VS{26}Alors Paul prit ces hommes, se purifia, et entra le lendemain dans le temple avec eux, pour annoncer quel jour leur purification devait s'achever, et quand l'offrande devait être présentée pour chacun d'eux.
\TextTitle{Paul chassé du temple et brutalisé par les Juifs}
\VS{27}A la fin des sept jours, les Juifs d'Asie ayant vu Paul dans le temple, soulevèrent tout le peuple, et mirent la main sur lui,
\VS{28}en criant~: Hommes israélites, au secours~! Voici l'homme qui prêche partout et à tout le monde contre le peuple, contre la loi, et contre ce lieu. Il a même introduit des Grecs dans le temple, et a profané ce saint lieu.
\VS{29}Car ils avaient vu auparavant Trophime d'Ephèse avec lui dans la ville, et ils croyaient que Paul l'avait fait entrer dans le temple.
\VS{30}Toute la ville fut émue, et le peuple accourut de toutes parts. Ils se saisirent de Paul et le traînèrent hors du temple, dont les portes furent aussitôt fermées.
\TextTitle{Intervention des soldats et des centeniers}
\VS{31}Comme ils cherchaient à le tuer, le bruit vint au tribun de la cohorte que tout Jérusalem était en trouble.
\VS{32}A l'instant, il prit des soldats et des centeniers, et courut vers eux. Voyant le tribun et les soldats, ils cessèrent de frapper Paul.
\VS{33}Alors le tribun s'approcha, se saisit de Paul, et le fit lier de deux chaînes. Puis il demanda qui il était et ce qu'il avait fait.
\VS{34}Les uns criaient d'une manière, et les autres d'une autre, dans la foule. Ne pouvant donc rien apprendre de certain à cause du tumulte, il ordonna de mener Paul dans la forteresse.
\VS{35}Lorsque Paul fut sur les degrés, il dut être porté par les soldats, à cause de la violence de la foule~;
\VS{36}car la multitude du peuple le suivait, en criant~: Fais-le mourir~!
\VS{37}Comme on allait faire entrer Paul dans la forteresse, il dit au tribun~: M'est-il permis de te dire quelque chose~? Et le tribun répondit~: Tu sais parler le grec~?
\VS{38}Tu n'es donc pas cet Egyptien qui a excité une sédition dernièrement, et qui a emmené dans le désert quatre mille brigands~?
\VS{39}Paul lui dit~: Je suis Juif de Tarse, citoyen de la ville renommée de la Cilicie. Permets-moi, je te prie, de parler au peuple.
\VS{40}Et quand il le lui permit, Paul se tenant sur les degrés fit signe de la main au peuple, et s'étant fait un grand silence, il leur parla en langue hébraïque, disant~:
\Chap{22}
\TextTitle{Paul raconte son témoignage de conversion\FTNTT{\vref{Ac. 9:1-18}~; \vref{26:9-18}.}}
\VerseOne{}Hommes frères et pères, écoutez ce que j'ai maintenant à vous dire pour ma défense~!
\VS{2}Lorsqu'ils entendirent qu'il leur parlait en langue hébraïque, ils redoublèrent de silence. Et Paul leur dit~:
\VS{3}Je suis Juif, né à Tarse en Cilicie~; mais j'ai été élevé dans cette ville-ci aux pieds de Gamaliel et instruit dans la connaissance exacte de la loi de nos pères, étant plein de zèle pour la loi de Dieu, comme vous l'êtes tous aujourd'hui.
\VS{4}J'ai persécuté à mort cette doctrine, liant et mettant en prison hommes et femmes.
\VS{5}Le grand-prêtre lui-même et toute l'assemblée des anciens m'en sont témoins. J'ai même reçu d'eux des lettres pour les frères de Damas, où je me rendis afin d'amener liés à Jérusalem ceux qui se trouvaient là et de les faire punir.
\VS{6}Or il arriva comme j'étais en chemin, que j'approchais de Damas, tout à coup, vers midi, une grande lumière venant du ciel resplendit comme un éclair autour de moi.
\VS{7}Je tombai par terre, et j'entendis une voix qui me dit~: Saul, Saul, pourquoi me persécutes-tu~?
\VS{8}Je répondis~: Qui es-tu Seigneur~? Et il me dit~: Je suis Jésus de Nazareth, que tu persécutes.
\VS{9}Ceux qui étaient avec moi furent tout effrayés, ils virent bien la lumière, mais ils ne comprirent pas la voix de celui qui me parlait. Alors je dis~: Que ferai-je Seigneur~?
\VS{10}Et le Seigneur me dit~: Lève-toi, va à Damas, et là on te dira tout ce que tu dois faire.
\VS{11}Comme je ne voyais rien, à cause de l'éclat de cette lumière, ceux qui étaient avec moi me prirent par la main, et j'arrivai à Damas.
\VS{12}Or un nommé Ananias, homme pieux selon la loi, et de qui tous les Juifs demeurant à Damas rendaient un bon témoignage, vint me trouver
\VS{13}et me dit~: Saul, mon frère, recouvre la vue. Au même instant, je recouvrai la vue et je le regardai.
\VS{14}Et il me dit~: Le Dieu de nos pères t'a destiné à connaître sa volonté, à voir le Juste, et à entendre les paroles de sa bouche.
\VS{15}Car tu lui serviras de témoin auprès de tous les hommes, des choses que tu as vues et entendues.
\VS{16}Et maintenant, pourquoi tardes-tu~? Lève-toi, et sois baptisé et purifié de tes péchés, en invoquant le Nom du Seigneur.
\TextTitle{Le Seigneur appelle Paul à quitter Jérusalem et l'envoie dans les nations}
\VS{17}Or il arriva qu'après que je sois retourné à Jérusalem, comme je priais dans le temple, je fus ravi en extase,
\VS{18}et je vis le Seigneur qui me disait~: Hâte-toi, et sors promptement de Jérusalem, parce qu'ils ne recevront pas le témoignage que tu leur rendras de moi.
\VS{19}Et je dis~: Seigneur, ils savent eux-mêmes que je faisais mettre en prison et battre de verges dans les synagogues ceux qui croyaient en toi~;
\VS{20}et que lorsque le sang d'Etienne, ton martyr, fut répandu, j'étais moi-même présent, je consentais à sa mort, et je gardais les vêtements de ceux qui le faisaient mourir.
\VS{21}Alors il me dit~: Va, car je t'enverrai au loin vers les Gentils.
\TextTitle{Les Juifs demandent la mort de Paul}
\VS{22}Et ils l'écoutèrent jusqu'à cette parole~; mais alors ils élevèrent leur voix, en disant~: Ote de la terre un tel homme~! Car il n'est pas concevable qu'il vive.
\VS{23}Et comme ils criaient à haute voix, secouaient leurs vêtements, et jetaient de la poussière en l'air,
\VS{24}le tribun commanda de faire entrer Paul dans la forteresse, et de lui donner la question par le fouet, afin de savoir pour quel sujet ils criaient ainsi contre lui.
\TextTitle{Paul revendique ses droits de citoyen romain}
\VS{25}Comme on l'attachait pour le frapper, Paul dit au centenier qui était près de lui~: Vous est-il permis de fouetter un homme romain, et qui n'est même pas condamné~?
\VS{26}A ces mots, le centenier alla vers le tribun pour l'avertir, disant~: Prends garde à ce que tu feras, car cet homme est Romain.
\VS{27}Et le tribun, étant venu, dit à Paul~: Dis-moi, es-tu Romain~? Et il répondit~: Oui, je le suis.
\VS{28}Le tribun lui dit~: J'ai acquis ce droit de citoyen pour une grande somme d'argent. Et moi, dit Paul, je l'ai par ma naissance.
\VS{29}Aussitôt, ceux qui devaient lui donner la question se retirèrent, et le tribun, voyant que Paul était Romain, fut dans la crainte parce qu'il l'avait fait lier.
\TextTitle{Paul devant le sanhédrin}
\VS{30}Le lendemain, voulant savoir avec certitude de quoi les Juifs l'accusaient, le tribun lui fit ôter ses liens, et donna l'ordre aux principaux prêtres et à tout le sanhédrin de se réunir~; puis, il fit descendre Paul, et il le présenta devant eux.
\Chap{23}
\VerseOne{}Paul regardant fixement le sanhédrin, dit~: Hommes frères~! Je me suis conduit en toute bonne conscience devant Dieu jusqu'à ce jour.
\VS{2}Le grand-prêtre Ananias ordonna à ceux qui étaient près de lui de le frapper sur la bouche.
\VS{3}Alors Paul lui dit~: Dieu te frappera, muraille blanchie~! Tu es assis pour me juger selon la loi, et tu violes la loi en ordonnant qu'on me frappe~!
\VS{4}Ceux qui étaient présents lui dirent~: Tu insultes le grand-prêtre de Dieu~?
\VS{5}Et Paul dit~: Je ne savais pas, mes frères, que c'était le grand-prêtre~; car il est écrit~: Tu ne parleras pas mal du chef de ton peuple.
\TextTitle{Dissensions entre pharisiens et sadducéens}
\VS{6}Paul, sachant qu'une partie de l'assemblée était composée de sadducéens et l'autre de pharisiens, s'écria dans le sanhédrin~: Hommes frères~! Je suis pharisien, fils de pharisien, c'est à cause de l'espérance et de la résurrection des morts que je suis mis en jugement.
\VS{7}Quand il eut dit cela, il s'éleva un débat entre les pharisiens et les sadducéens~; et l'assemblée se divisa.
\VS{8}Car les sadducéens disent qu'il n'y a point de résurrection, ni d'ange, ni d'esprit, mais les pharisiens soutiennent les deux choses.
\VS{9}Il y eut une grande clameur. Alors les scribes du parti des pharisiens se levèrent et contestèrent, disant~: Nous ne trouvons aucun mal en cet homme~; peut-être un esprit ou un ange lui a parlé, ne combattons point contre Dieu.
\VS{10}Et comme il se faisait une grande division, le tribun craignant que Paul ne soit mis en pièces par eux, ordonna que les soldats descendent, et qu'ils l'enlèvent du milieu d'eux, et l'amènent dans la forteresse.
\TextTitle{Le Seigneur fortifie Paul}
\VS{11}La nuit suivante, le Seigneur apparut à Paul et lui dit~: Prends courage~; car, de même que tu as rendu témoignage de moi dans Jérusalem, il faut aussi que tu rendes témoignage à Rome.
\TextTitle{Complot des Juifs pour tuer Paul}
\VS{12}Quand le jour fut venu, les Juifs formèrent un complot, et firent des imprécations contre eux-mêmes, en disant qu'ils ne mangeraient pas ni ne boiraient jusqu'à ce qu'ils aient tué Paul.
\VS{13}Ceux qui formèrent ce complot étaient plus de quarante,
\VS{14}et ils s'adressèrent aux principaux prêtres et aux anciens, et leur dirent~: Nous nous sommes engagés, avec des imprécations contre nous-mêmes, à ne rien manger jusqu'à ce que nous ayons tué Paul.
\VS{15}Vous donc, maintenant, adressez-vous, avec le sanhédrin, au tribun pour le faire descendre demain au milieu de vous, comme si vous vouliez examiner sa cause plus exactement~; et nous, avant qu'il approche, nous sommes tous prêts à le tuer.
\VS{16}Le fils de la sœur de Paul, ayant eu connaissance de ce complot, alla dans la forteresse et le rapporta à Paul.
\VS{17}Paul appela l'un des centeniers et lui dit~: Mène ce jeune homme au tribun, car il a quelque chose à lui rapporter.
\VS{18}Il le prit donc et le mena au tribun, et il lui dit~: Le prisonnier Paul m'a appelé et m'a prié de t'amener ce jeune homme qui a quelque chose à te dire.
\VS{19}Et le tribun le prenant par la main, se retira à part, et lui demanda~: Qu'est-ce que tu as à me rapporter~?
\VS{20}Et il lui dit~: Les Juifs ont conspiré de te prier que demain tu envoies Paul au sanhédrin, comme s'ils voulaient s'enquérir de lui plus exactement de quelque chose.
\VS{21}Mais n'y consens point, car plus de quarante hommes d'entre eux sont en embûches contre lui, qui ont fait un vœu avec exécration de serment, de ne manger ni boire jusqu'à ce qu'ils l'aient tué~; et ils sont maintenant tous prêts, attendant ce que tu leur permettras.
\VS{22}Le tribun donc renvoya le jeune homme, en lui recommandant de ne parler à personne de ce rapport qu'il lui avait fait.
\TextTitle{Paul transféré à Césarée}
\VS{23}Ensuite, il appela deux des centeniers, et il leur dit~: Tenez prêts, dès la troisième heure de la nuit, deux cents soldats, soixante-dix cavaliers, et deux cents archers, pour aller jusqu'à Césarée.
\VS{24}Et ayez soin qu'il y ait des montures prêtes, afin qu'ayant fait monter Paul, ils le mènent sûrement au gouverneur Félix. \FTNT{Marcus Antonuis Félix était procurateur de la province romaine de la Judée de 52 à 60 ap. J.-C.}.
\VS{25}Et il lui écrivit une lettre en ces termes~:
\VS{26}Claude Lysias au très excellent gouverneur Félix, salut~!
\VS{27}Les Juifs s'étaient saisis de cet homme et allaient le tuer, lorsque je survins avec des soldats et le leur enlevai, ayant appris qu'il était Romain.
\VS{28}Voulant connaître le motif pour lequel ils l'accusaient, je l'amenai devant leur sanhédrin.
\VS{29}J'ai trouvé qu'il était accusé au sujet de questions relatives à leur loi, mais qu'il n'avait commis aucun crime qui mérite la mort ou la prison.
\VS{30}Ayant été averti des embûches que les Juifs avaient dressées contre lui, je te l'ai aussitôt envoyé, en ordonnant à ses accusateurs de te dire eux-mêmes ce qu'ils ont contre lui. Adieu~!
\TextTitle{Paul arrive à Césarée}
\VS{31}Les soldats prirent Paul, selon l'ordre qu'ils avaient reçu, et le conduisirent pendant la nuit jusqu'à Antipatris.
\VS{32}Le lendemain, laissant les cavaliers poursuivre la route avec Paul, ils retournèrent à la forteresse.
\VS{33}Arrivés à Césarée, les cavaliers remirent la lettre au gouverneur, et lui présentèrent aussi Paul.
\VS{34}Le gouverneur, après avoir lu la lettre, demanda à Paul de quelle province il était. Ayant appris qu'il était de Cilicie~:
\VS{35}Je t'entendrai, lui dit-il, plus amplement quand tes accusateurs seront venus. Et il ordonna qu'il soit gardé dans le Prétoire d'Hérode.
\Chap{24}
\TextTitle{Paul devant le gouverneur Félix~; accusation des Juifs}
\VerseOne{}Or cinq jours après, Ananias le grand-prêtre descendit avec les anciens, et un certain orateur, nommé Tertulle, qui comparurent devant le gouverneur contre Paul. 
\VS{2}Et Paul étant appelé, Tertulle commença à l'accuser, en disant~:
\VS{3}Très excellent Félix, nous reconnaissons en toutes choses partout et avec une entière reconnaissance, que nous avons obtenu une grande tranquillité par ton moyen, et par les bons règlements que tu as faits pour ce peuple, selon ta prudence.
\VS{4}Mais afin de ne pas te retenir plus longtemps, je te prie de nous entendre, selon ton équité, dans ce que nous allons te dire en peu de paroles.
\VS{5}Nous avons trouvé cet homme, qui est une peste, qui sème des divisions parmi tous les Juifs du monde entier, et qui est le chef de la secte des Nazaréens.
\VS{6}Il a même tenté de profaner le temple~; et nous l'avons saisi, et avons voulu le juger selon notre loi.
\VS{7}Mais le tribun Lysias étant survenu, il nous l'a arraché de nos mains avec une grande violence,
\VS{8}en ordonnant à ses accusateurs de venir vers toi. Tu pourras toi-même, en l'interrogeant, apprendre de lui tout ce dont nous l'accusons.
\VS{9}Les Juifs consentirent à cela, en disant que les choses étaient ainsi.
\TextTitle{Paul défend sa cause devant Félix}
\VS{10}Et après que le gouverneur eut fait signe à Paul de parler, il répondit~: Sachant qu'il y a déjà plusieurs années que tu es le juge de cette nation je réponds pour moi avec plus de courage:
\VS{11}Puisque tu peux comprendre qu'il n'y a pas plus de douze jours que je suis monté à Jérusalem pour adorer Dieu.
\VS{12}Mais ils ne m'ont point trouvé dans le temple disputant avec personne, ni faisant un amas de peuple, soit dans les synagogues, soit dans la ville. 
\VS{13}Et ils ne sauraient soutenir les choses dont ils m'accusent présentement.
\VS{14}Or je te confesse bien ce point, que selon la voie qu'ils appellent secte, je sers ainsi le Dieu de mes pères, croyant toutes les choses qui sont écrites dans la loi et dans les prophètes,
\VS{15}et ayant en Dieu cette espérance, comme ils l'ont eux-mêmes, qu'il y aura une résurrection des justes et des injustes.
\VS{16}C'est pourquoi aussi je travaille pour avoir toujours une conscience pure devant Dieu, et devant les hommes.
\VS{17}Or après plusieurs années, je suis venu pour faire des aumônes et des offrandes dans ma nation.
\VS{18}Et comme je m'occupais de ces choses, quelques Juifs d'Asie m'ont trouvé purifié dans le temple, sans attroupement ni tumulte.
\VS{19}Ils auraient dû eux-mêmes comparaître devant toi et m'accuser, s'ils avaient eu quelque chose contre moi.
\VS{20}Ou bien, que ceux-ci eux-mêmes disent, s'ils ont trouvé en moi quelque injustice, quand j'ai été présenté au sanhédrin~;
\VS{21}à moins que ce ne soit uniquement cette parole que j'ai fait entendre au milieu d'eux~; c'est à cause de la résurrection des morts que je suis aujourd'hui mis en jugement devant vous.
\VS{22}Félix, qui était parfaitement au courant de ce qui concerne cette secte, les ajourna, en disant~: Quand le tribun Lysias sera venu, j'examinerai votre affaire.
\VS{23}Et il donna l'ordre au centenier de garder Paul, en lui laissant une certaine liberté, et n'empêchant aucun des siens de le servir, ou de venir vers lui.
\TextTitle{Paul prêche Christ au gouverneur et à sa femme}
\VS{24}Quelques jours après, Félix vint avec Drusille, sa femme, qui était Juive, et il envoya chercher Paul. Il l'entendit sur la foi en Christ.
\VS{25}Et comme il parlait de la justice, de la tempérance, et du jugement à venir, Félix tout effrayé répondit~: Pour le moment retire-toi~; et quand j'aurai la commodité, je te rappellerai.
\VS{26}Il espérait en même temps que Paul lui donnerait de l'argent afin de le délivrer, c'est pourquoi il l'envoyait chercher souvent, et s'entretenait avec lui.
\TextTitle{Paul emprisonné deux ans à Césarée}
\VS{27}Deux ans s'écoulèrent ainsi, et Félix eut pour successeur Porcius Festus\FTNT{Porcius Festus était procurateur de Judée d'environ 60 à 62, succédant à Antonius Félix.}, qui voulant faire plaisir aux Juifs, laissa Paul en prison.
\Chap{25}
\TextTitle{Paul devant le gouverneur Festus}
\VerseOne{}Festus, étant arrivé dans la province, monta trois jours après de Césarée à Jérusalem.
\VS{2}Le grand-prêtre, et les principaux d'entre les Juifs portèrent plainte contre Paul devant lui. Ils firent des instances auprès de Festus, et dans des vues hostiles,
\VS{3}lui demandèrent une faveur contre lui~: Qu'il le fasse venir à Jérusalem. Ils avaient dressé des embûches pour le tuer en chemin.
\VS{4}Mais Festus leur répondit que Paul était bien gardé à Césarée, et que lui-même devait partir sous peu.
\VS{5}Et il ajouta~: Que les principaux d'entre vous descendent avec moi, et s'il y a quelque chose de coupable contre cet homme, qu'ils l'accusent.
\VS{6}Festus ne passa que dix jours parmi eux, puis il descendit à Césarée. Le lendemain, siégeant au tribunal, il ordonna que Paul soit amené.
\VS{7}Quand il fut amené, les Juifs qui étaient descendus de Jérusalem l'entourèrent et portèrent contre lui de nombreuses et graves accusations, qu'ils ne pouvaient pas prouver.
\VS{8}Tandis que Paul parlait pour sa défense~: Je n'ai rien fait de coupable, ni contre la loi des Juifs, ni contre le temple, ni contre César.
\VS{9}Mais Festus voulant faire plaisir aux Juifs, répondit à Paul, et dit~: Veux-tu monter à Jérusalem et y être jugé sur ces choses devant moi~?
\TextTitle{Paul en appelle à César}
\VS{10}Paul dit~: Je comparais devant le tribunal de César, où il faut que je sois jugé. Je n'ai fait aucun tort aux Juifs, comme tu le sais très bien.
\VS{11}Si j'ai commis quelque injustice, ou un crime digne de mort, je ne refuse pas de mourir~; mais si les choses dont ils m'accusent sont fausses, personne n'a le droit de me livrer à eux. J'en appelle à César.
\VS{12}Alors Festus ayant conféré avec le conseil, lui répondit~: En as-tu appelé à César~? Tu iras à César.
\TextTitle{Le roi Agrippa informé du cas de Paul}
\VS{13}Quelques jours après, le roi Agrippa\FTNT{Agrippa II (27-28 ap. J.-C. – 93-101 ap. J.-C.) était le fils d'Agrippa I(10 av. J.-C. – 44 ap. J.-C.), qui était lui-même le petit-fils d'Hérode le Grand (73 av. J.-C. – 4 av. J.-C.).} et Bérénice\FTNT{Bérénice (née vers 28 ap. J.-C.) était la fille d'Agrippa I et donc la sœur d'Agrippa II. Pendant tout le règne de son frère, elle fut présentée comme reine à ses cotés, raison pour laquelle on soupçonna une liaison incestueuse entre eux.} arrivèrent à Césarée pour saluer Festus.
\VS{14}Comme ils passèrent là plusieurs jours, Festus fit mention au roi de l'affaire de Paul, en disant~: Félix a laissé prisonnier un homme
\VS{15}contre lequel, lorsque j'étais à Jérusalem, les principaux prêtres et les anciens des Juifs ont porté plainte, en demandant sa condamnation.
\VS{16}Mais je leur ai répondu que ce n'est pas la coutume des Romains de livrer quelqu'un à la mort, avant que l'inculpé ait été mis en présence de ses accusateurs, et qu'il ait eu la liberté de se défendre sur le crime dont on l'accuse.
\VS{17}Ils sont donc venus ici, et sans différer, je siégeai le lendemain, et je donnai l'ordre qu'on amène cet homme.
\VS{18}Ses accusateurs s'étant présentés, ne lui imputèrent aucun des crimes dont je pensais qu'ils l'accuseraient.
\VS{19}Mais ils avaient avec lui des discussions relatives à leurs superstitions, et à un certain Jésus qui est mort, que Paul affirmait être vivant.
\VS{20}Ne sachant quel parti prendre dans ce débat, je demandai à cet homme s'il voulait aller à Jérusalem et y être jugé sur ces choses.
\VS{21}Mais Paul en ayant appelé, pour que sa cause soit réservée à la connaissance de l'empereur, j'ai ordonné qu'on le garde jusqu'à ce que je l'envoie à César.
\VS{22}Alors Agrippa dit à Festus~: Je voudrais bien aussi entendre cet homme. Demain, dit-il, tu l'entendras.
\TextTitle{Paul est amené dans la salle d'audience}
\VS{23}Le lendemain donc, Agrippa et Bérénice étant venus en grande pompe, et étant entrés dans la salle d'audience avec les tribuns et les principaux de la ville, Paul fut amené sur l'ordre de Festus.
\VS{24}Et Festus dit~: Roi Agrippa, et vous tous qui êtes ici avec nous, vous voyez cet homme au sujet duquel toute la multitude des Juifs s'est adressée à moi, soit à Jérusalem soit ici, en s'écriant qu'il ne devait plus vivre.
\VS{25}Pour moi, ayant trouvé qu'il n'avait rien fait qui mérite la mort, et lui-même en ayant appelé à Auguste, j'ai résolu de le faire partir.
\VS{26}Comme je n'ai rien de certain à écrire à l'empereur sur son compte, je vous l'ai présenté, et principalement à toi, roi Agrippa, afin qu'après en avoir fait l'examen, j'aie de quoi écrire.
\VS{27}Car il me semble qu'il n'est pas raisonnable d'envoyer un prisonnier sans marquer les faits dont on l'accuse.
\Chap{26}
\TextTitle{Discours de Paul devant Agrippa\FTNTT{\vref{Ac. 9:1-18}~; \vref{22:1-16}}}
\VerseOne{}Agrippa dit à Paul~: Il t'est permis de parler pour toi-même. Alors Paul ayant étendu la main, parla ainsi pour sa défense~:
\VS{2}Roi Agrippa~! Je m'estime béni de ce que je dois me défendre aujourd'hui devant toi, de toutes les choses dont les Juifs m'accusent~;
\VS{3}car tu connais parfaitement leurs coutumes et leurs discussions. Je te prie donc de m'écouter avec patience.
\VS{4}Ma vie, dès les premiers temps de ma jeunesse, est connue de tous les Juifs, puisqu'elle s'est passée à Jérusalem, au milieu de ma nation.
\VS{5}Car ils savent depuis longtemps, s'ils veulent en rendre témoignage, que j'ai vécu en pharisien, selon la secte la plus rigide de notre religion.
\VS{6}Et maintenant, je suis mis en jugement pour l'espérance de la promesse que Dieu a faite à nos pères,
\VS{7}et à laquelle nos douze tribus, qui servent Dieu continuellement nuit et jour, espèrent parvenir~; et c'est pour cette espérance, ô roi Agrippa, que je suis accusé par les Juifs.
\VS{8}Quoi~? Jugez-vous incroyable que Dieu ressuscite les morts~?
\VS{9}Pour moi, j'avais cru devoir agir vigoureusement contre le Nom de Jésus de Nazareth.
\VS{10}C'est ce que j'ai fait à Jérusalem. J'ai mis en prison plusieurs des saints, après en avoir reçu le pouvoir des principaux prêtres, et quand on les faisait mourir, je joignais mon suffrage à celui des autres.
\VS{11}Je les ai souvent châtiés dans toutes les synagogues, et les forçais à blasphémer. Dans mes excès de fureur contre eux, je les persécutais même jusque dans les villes étrangères.
\VS{12}Comme j'allais aussi à Damas dans ce dessein, avec l'autorisation et la permission des principaux prêtres,
\VS{13}en plein midi, ô roi, je vis en chemin resplendir autour de moi et de mes compagnons, une lumière venant du ciel et dont l'éclat surpassait celui du soleil.
\VS{14}Nous tombâmes tous par terre, et j'entendis une voix qui me parlait en langue hébraïque~: Saul, Saul, pourquoi me persécutes-tu~? Il te serait dur de regimber contre les aiguillons.
\VS{15}Je répondis~: Qui es-tu Seigneur~? Et il répondit~: Je suis Jésus que tu persécutes.
\VS{16}Mais lève-toi, et tiens-toi sur tes pieds~; car je te suis apparu pour t'établir serviteur et témoin des choses que tu as vues et de celles pour lesquelles je t'apparaîtrai.
\VS{17}Je t'ai arraché du milieu du peuple et des Gentils, vers qui je t'envoie maintenant,
\VS{18}pour ouvrir leurs yeux afin qu'ils passent des ténèbres à la lumière, et de la puissance de Satan à Dieu~; afin que par la foi qu'ils auront en moi, ils reçoivent la rémission de leurs péchés et qu'ils aient part à l'héritage des saints.
\VS{19}Ainsi, ô roi Agrippa, je n'ai pas été désobéissant à la vision céleste.
\VS{20}A ceux de Damas d'abord, puis à Jérusalem, dans toute la Judée, et chez les Gentils, j'ai prêché la repentance et la conversion à Dieu, avec la pratique d'œuvres dignes de la repentance.
\VS{21}C'est pour cela que les Juifs se sont saisis de moi dans le temple, et ont tâché de me tuer.
\VS{22}Mais ayant été secouru par l'aide de Dieu, je suis vivant jusqu'à ce jour, rendant témoignage aux petits et aux grands, sans m'écarter en rien de ce que les prophètes et Moïse ont prédit devoir arriver,
\VS{23}à savoir que le Christ souffrirait, et que ressuscité le premier d'entre les morts, il annoncerait la lumière au peuple et aux nations.
\TextTitle{Paul exhorte Agrippa}
\VS{24}Comme il parlait ainsi pour sa défense, Festus dit à haute voix~: Tu es fou Paul~! Ton grand savoir dans les lettres te fait déraisonner.
\VS{25}Et Paul dit~: Je ne suis point fou, très excellent Festus, mais je dis des paroles de vérité et de bon sens.
\VS{26}Car le roi est bien informé de ces choses~; et je lui en parle librement, parce que je suis persuadé qu'il n'en ignore aucune, puisque ce n'est pas en cachette qu'elles se sont passées.
\VS{27}Ô Roi Agrippa~! Crois-tu aux prophètes~? Je sais que tu y crois.
\VS{28}Et Agrippa répondit à Paul~: Tu vas bientôt me persuader de devenir chrétien~!
\VS{29}Et Paul lui dit~: Je souhaiterais devant Dieu que non seulement toi, mais aussi tous ceux qui m'écoutent aujourd'hui, vous deveniez tels que je suis à l'exception de ces liens~!
\VS{30}Paul ayant dit ces choses, le roi se leva, avec le gouverneur et Bérénice, et ceux qui étaient assis avec eux.
\VS{31}Et s'étant retirés à part, ils se disaient les uns les autres~: Cet homme n'a rien fait qui mérite la mort ou la prison.
\VS{32}Et Agrippa dit à Festus~: Cet homme aurait pu être relâché s'il n'avait pas appelé à César.
\Chap{27}
\TextTitle{Toujours prisonnier, Paul embarque pour Rome}
\VerseOne{}Lorsqu'il fut décidé que nous embarquerions pour l'Italie, on remit Paul avec quelques autres prisonniers à un nommé Julius, centenier d'une cohorte de la légion appelée Auguste.
\VS{2}Nous montâmes sur un navire d'Adramytte, nous partîmes prenant notre route vers les côtes de l'Asie, ayant avec nous Aristarque, un Macédonien de la ville de Thessalonique.
\VS{3}Le jour suivant, nous arrivâmes à Sidon~; et Julius, qui traitait Paul avec bienveillance, lui permit d'aller vers ses amis afin de recevoir leurs soins.
\VS{4}Puis étant partis de là, nous longeâmes l'île de Chypre, parce que les vents étaient contraires.
\VS{5}Après avoir traversé la mer de Cilicie et de Pamphylie, nous arrivâmes à Myra, ville de Lycie.
\VS{6}Et là, le centenier trouva un navire d'Alexandrie qui allait en Italie, dans lequel il nous fit monter.
\VS{7}Et comme nous naviguions lentement pendant plusieurs jours, et que nous étions arrivés avec peine vis-à-vis de Cnide, parce que le vent ne nous permettait pas d'avancer, nous naviguâmes en dessous de la Crète, vers Salmone.
\VS{8}Nous la côtoyâmes avec peine, nous arrivâmes à un lieu qui est appelé Beaux-Ports, près duquel était la ville de Lasée.
\VS{9}Il s'était écoulé beaucoup de temps, et la navigation devenait dangereuse, car le temps du jeûne était déjà passé\FTNT{Ce jeûne correspondait au jour de l'expiation célébré le dixième jour du septième mois. \vref{Lé. 23:27}.}.
\VS{10}C'est pourquoi Paul les avertit, en disant~: Ô hommes, je vois que la navigation ne se fera pas sans péril et sans dommage, non seulement pour la cargaison et pour le navire, mais aussi pour nos propres vies.
\VS{11}Mais le centenier écouta plus le pilote et le maître du navire, plutôt que les paroles de Paul.
\VS{12}Et comme le port n'était pas bon pour y passer l'hiver, la plupart furent d'avis de partir de là, pour tâcher de gagner Phénix, qui est un port de Crète, qui regarde le vent d'Afrique et le couchant septentrional, afin d'y passer l'hiver.
\VS{13}Un vent du midi commença à souffler doucement, et se croyant maîtres de leur dessein, ils levèrent l'ancre et côtoyèrent de près l'île de Crète.
\TextTitle{Une tempête de plusieurs jours}
\VS{14}Mais bientôt un vent impétueux, du nord-est, qu'on appelle Euraquilon\FTNT{Euraquilon~: Vagues et vent d'Est}, se leva du côté de l'île.
\VS{15}Le navire fut emporté par la violence de la tempête, et ne pouvant résister, nous nous laissâmes aller au gré du vent.
\VS{16}Nous passâmes au-dessous d'une petite île nommée Clauda, et nous eûmes de la peine à nous rendre maîtres de la chaloupe~;
\VS{17}après l'avoir hissée, les matelots se servirent des moyens de secours pour ceindre le navire, et dans la crainte de tomber sur la Syrte\FTNT{Syrte~: Il s'agit de la Grande Syrte et de la Petite Syrte~: deux bancs de sables mouvants très redoutés.}, ils abaissèrent les voiles. C'est ainsi qu'on se laissa emporter par le vent.
\VS{18}Comme nous étions violemment battus par la tempête, le jour suivant, ils jetèrent la cargaison à la mer~;
\VS{19}et le troisième jour, nous jetâmes de nos propres mains les agrès du navire.
\VS{20}Le soleil et les étoiles ne parurent pas pendant plusieurs jours, et la tempête nous agitait si violemment que nous perdîmes enfin toute espérance de nous sauver.
\TextTitle{Paul rassure les membres du navire}
\VS{21}On n'avait pas mangé depuis longtemps. Paul se tenant alors debout au milieu d'eux, leur dit~: Ô hommes, il fallait m'écouter et ne pas partir de Crète, afin d'éviter cette tempête et ce dommage.
\VS{22}Maintenant je vous exhorte à prendre courage~; car aucun de vous ne perdra la vie, et il n'y aura de perte que celle du navire.
\VS{23}Car un ange du Dieu à qui j'appartiens et que je sers m'est apparu cette nuit,
\VS{24}et m'a dit~: Paul, ne crains point~; il faut que tu comparaisses devant César~; et voici, Dieu t'a donné tous ceux qui naviguent avec toi.
\VS{25}C'est pourquoi, ô hommes, prenez courage, car j'ai cette confiance en Dieu que la chose arrivera comme elle m'a été dite.
\VS{26}Mais nous devons échouer sur une île.
\VS{27}La quatorzième nuit, vers minuit, tandis que nous étions ballotés sur l'Adriatique, les matelots soupçonnèrent qu'on approchait de quelque terre.
\VS{28}Ayant jeté la sonde, ils trouvèrent vingt brasses~; puis étant passés un peu plus loin, et ayant encore jeté la sonde, ils trouvèrent quinze brasses.
\VS{29}Mais craignant de heurter contre des écueils, ils jetèrent quatre ancres de la poupe, et attendirent le jour avec impatience.
\VS{30}Mais comme les matelots cherchaient à s'échapper du navire, et mettaient la chaloupe à la mer, sous prétexte de jeter les ancres de la proue,
\VS{31}Paul dit au centenier et aux soldats~: Si ces hommes ne restent pas dans le navire, vous ne pouvez pas être sauvés.
\VS{32}Alors les soldats coupèrent les cordes de la chaloupe, et la laissèrent tomber.
\VS{33}Avant que le jour paraisse, Paul les exhorta tous à prendre de la nourriture, en leur disant~: C'est aujourd'hui le quatorzième jour que vous êtes en attente et que vous persistez à vous abstenir de manger.
\VS{34}Je vous exhorte donc à prendre quelque nourriture, vu que cela est nécessaire pour votre conservation, et aucun de vos cheveux ne se perdra.
\VS{35}Ayant ainsi parlé, il prit du pain, et rendit grâces à Dieu en présence de tous~; il le rompit et se mit à manger.
\VS{36}Et tous, reprenant courage, mangèrent aussi.
\VS{37}Nous étions dans le navire deux cent soixante-seize personnes.
\VS{38}Quand ils eurent mangé jusqu'à être rassasiés, ils allégèrent le navire en jetant le blé dans la mer.
\TextTitle{Naufrage du navire}
\VS{39}Lorsque le jour fut venu, ils ne reconnurent point la terre~; mais ayant aperçu un golfe avec un rivage, ils résolurent d'y faire échouer le navire, s'ils le pouvaient.
\VS{40}Ayant donc retiré les ancres, ils abandonnèrent le navire à la mer, lâchant en même temps les attaches des gouvernails~; et ayant tendu la voile de l'artimon, ils tâchaient de se diriger vers le rivage.
\VS{41}Mais ils rencontrèrent une langue de terre, où ils firent échouer le navire~; et la proue, s'étant engagée, resta immobile, tandis que la poupe se brisait par la violence des vagues.
\VS{42}Les soldats furent d'avis de tuer les prisonniers, de peur que quelqu'un d'eux ne s'échappe à la nage.
\VS{43}Mais le centenier, voulant sauver Paul, les empêcha d'exécuter ce conseil. Il ordonna à ceux qui savaient nager de se jeter les premiers dans l'eau pour gagner la terre,
\VS{44}et aux autres de se mettre sur des planches ou sur des débris du navire. Et ainsi tous parvinrent à terre sains et saufs.
\Chap{28}
\TextTitle{Paul mordu par une vipère sur l'île de Malte}
\VerseOne{}Une fois hors de danger, ils reconnurent alors que l'île s'appelait Malte.
\VS{2}Les barbares nous traitèrent avec beaucoup d'humanité~; ils nous recueillirent tous auprès d'un grand feu, qu'ils avaient allumé parce que la pluie tombait et qu'il faisait très froid.
\VS{3}Paul ayant ramassé un tas de broussailles et l'ayant mis au feu, une vipère en sortit à cause de la chaleur et s'attacha à sa main.
\VS{4}Et quand les barbares virent cette bête suspendue à sa main, ils se dirent les uns les autres~: Certainement cet homme est un meurtrier~; puisque après être échappé de la mer, la justice ne permet pas qu'il vive. 
\VS{5}Mais Paul ayant secoué la bête dans le feu, ne ressentit aucun mal.
\VS{6}Les barbares s'attendaient à le voir enfler ou tomber subitement mort~; mais après avoir longtemps attendu, voyant qu'il ne lui arrivait aucun mal, ils changèrent de langage et dirent que c'était un dieu.
\TextTitle{Guérison du père de Publius}
\VS{7}Or en cet endroit-là étaient des terres qui appartenaient au principal de l'île, nommé Publius, qui nous reçut et nous logea pendant trois jours avec beaucoup de bonté.
\VS{8}Et il arriva que le père de Publius était au lit, malade de la fièvre et de la dysenterie~; Paul s'étant rendu vers lui, pria, lui imposa les mains, et le guérit.
\VS{9}Là-dessus, vinrent tous les autres malades de l'île, et ils furent guéris.
\VS{10}Ils nous rendirent de grands honneurs, et à notre départ, on nous fournit ce qui nous était nécessaire.
\TextTitle{Paul arrive à Rome}
\VS{11}Trois mois après, nous partîmes sur un navire d'Alexandrie qui avait passé l'hiver dans l'île, et qui avait pour enseigne Castor et Pollux.
\VS{12}Ayant abordé à Syracuse, nous y restâmes trois jours.
\VS{13}De là, en suivant la côte, nous arrivâmes à Reggio~; et un jour après, le vent du Midi s'étant levé, nous fîmes en deux jours le trajet jusqu'à Pouzzoles,
\VS{14}où nous trouvâmes des frères qui nous prièrent de passer sept jours avec eux. Et ensuite, nous arrivâmes à Rome.
\VS{15}Et les frères qui y étaient, ayant appris de nos nouvelles, vinrent à notre rencontre jusqu'au Forum d'Appius et aux Trois-Tavernes~; en les voyant, Paul rendit grâces à Dieu et prit courage.
\TextTitle{Paul annonce Christ aux Juifs de Rome}
\VS{16}Lorsque nous fûmes arrivés à Rome, le centenier mit les prisonniers entre les mains du préfet du prétoire~; mais quant à Paul, il lui permit de demeurer dans un domicile particulier avec un soldat qui le gardait.
\VS{17}Or il arriva que trois jours après que Paul convoqua les principaux des Juifs~; et quand ils furent réunis, il leur dit~: Hommes frères~! Sans avoir rien fait contre le peuple ni contre les coutumes des pères, j'ai été mis en prison à Jérusalem, et livré entre les mains des Romains,
\VS{18}qui après m'avoir examiné, voulaient me relâcher parce qu'il n'y avait en moi aucun crime qui mérite la mort.
\VS{19}Mais les Juifs s'y opposèrent, j'ai été contraint d'en appeler à César~; n'ayant du reste aucun dessein d'accuser ma nation.
\VS{20}C'est pour ce sujet que je vous ai appelés, afin de vous voir et vous parler~; car c'est pour l'espérance d'Israël que je porte cette chaîne.
\VS{21}Mais ils lui répondirent~: Nous n'avons reçu de Judée aucune lettre à ton sujet, et il n'est venu aucun frère qui ait rapporté ou dit quelque mal de toi.
\VS{22}Cependant nous entendrons volontiers de toi quel est ton sentiment~; car quant à cette secte, il nous est connu qu'on la contredit partout.
\VS{23}Et après lui avoir assigné un jour, plusieurs vinrent auprès de lui dans son logis~; et il leur expliquait par plusieurs témoignages le Royaume de Dieu, et depuis le matin jusqu'au soir, il cherchait à les persuader de ce qui concerne Jésus, tant par la loi de Moïse que par les prophètes.
\VS{24}Et les uns furent persuadés par les choses qu'il disait~; et les autres n'y crurent point.
\TextTitle{Incrédulité des Juifs~: Paul se tourne vers les Gentils\FTNTT{\vref{Ap. 13:14}~; \vref{18:6}.}}
\VS{25}C'est pourquoi n'étant pas d'accord entre eux, ils se retirèrent après que Paul leur eut dit ces paroles~: Le Saint-Esprit a bien parlé à nos pères par le prophète Esaïe en disant~:
\VS{26}Va vers ce peuple et dis-lui~: Vous entendrez de vos oreilles, et vous ne comprendrez point~; vous regarderez de vos yeux, et vous ne verrez point.
\VS{27}Car le cœur de ce peuple est devenu insensible~; ils ont endurci leurs oreilles, et ils ont fermé leurs yeux~; de peur qu'ils ne voient des yeux, qu'ils n'entendent des oreilles, qu'ils ne comprennent de leur cœur, qu'ils ne se convertissent, et que je ne les guérisse\FTNT{\vref{Es. 6:10}.}.
\VS{28}Sachez donc que ce salut de Dieu est envoyé aux Gentils, et ils l'écouteront.
\VS{29}Lorsqu'il eut dit cela, les Juifs s'en allèrent, discutant vivement entre eux.
\VS{30}Paul demeura deux ans entiers dans une maison qu'il avait louée. Il recevait tous ceux qui venaient le voir,
\VS{31}prêchant le Royaume de Dieu, et enseignant les choses qui concernent le Seigneur Jésus-Christ en toute liberté dans les paroles et sans aucun empêchement.
\PPE{}
\end{multicols}

\clearpage\ShortTitle{Jacques}\BookTitle{Jacques}\BFont
\noindent\hrulefill
{\footnotesize
\textit{
\bigskip
{\centering{}
\\Auteur : Jacques
\\Signification : Qui supplante
\\Thème : La vie chrétienne sous son aspect pratique
\\Date de rédaction : Env. 45-50 ap. J.-C.\\}
}
%\bigskip
\textit{
\\Jacques, frère de Jésus-Christ homme, et ancien au sein de la première église chrétienne située à Jérusalem, écrit aux chrétiens d'origine juive, dispersés dans l'empire romain. Il les console suite à l'adversité qu'ils rencontraient et les exhorte à tenir ferme, leur expliquant que la foi authentique doit être accompagnée d'œuvres. Il les met en garde contre la convoitise, source de toutes les tentations, et les prévient également quant à l'amour du monde et la confiance que certains peuvent mettre dans l'argent. Pour terminer, il leur recommande d'être patients dans l'épreuve et de prier sans cesse jusqu'au retour du Seigneur.\bigskip
}
}
\par\nobreak\noindent\hrulefill
\begin{multicols}{2}
\Chap{1}
\TextTitle{Introduction}
\VerseOne{}Jacques, serviteur de Dieu, et du Seigneur Jésus-Christ, aux douze tribus qui sont dispersées, salut !
\TextTitle{La nécessité de l'épreuve de la foi}
\VS{2}Mes frères, regardez comme un sujet d'une parfaite joie quand vous êtes exposés à diverses épreuves,
\VS{3}sachant que l'épreuve de votre foi produit la patience.
\VS{4}Mais il faut que la patience accomplisse parfaitement son œuvre, afin que vous soyez parfaits et accomplis, en sorte qu'il ne vous manque rien.
\VS{5}Et si quelqu'un de vous manque de sagesse, qu'il la demande à Dieu, qui la donne à tous libéralement, et sans reproche, et elle lui sera donnée.
\VS{6}Mais qu'il la demande avec foi, ne doutant nullement ; car celui qui doute est semblable au flot de la mer, agité et poussé çà et là par le vent.
\VS{7}Qu'un tel homme ne s'attende pas à recevoir quelque chose du Seigneur.
\VS{8}L'homme double de coeur est inconstant dans toutes ses voies.
\VS{9}Que le frère de basse condition se glorifie dans son élévation.
\VS{10}Que le riche, au contraire, se glorifie dans sa basse condition ; car il passera comme la fleur de l'herbe.
\VS{11}En effet, le soleil s'est levé avec sa chaleur ardente, et l'herbe a séché, et sa fleur est tombée, et son éclat a péri, ainsi le riche se flétrira dans ses entreprises.
\TextTitle{Dieu ne tente personne ; la justice de Dieu}
\VS{12}Heureux l'homme qui endure la tentation\FTNT{Le terme grec « peirasmos » utilisé dans ce verset veut aussi dire « épreuve ».} ; car après avoir été éprouvé, il recevra la couronne de vie, que le Seigneur a promise à ceux qui l'aiment.
\VS{13}Quand quelqu'un est tenté, qu'il ne dise pas : Je suis tenté par Dieu. Car Dieu ne peut être tenté par le mal, et aussi ne tente-t-il personne.
\VS{14}Mais chacun est tenté quand il est attiré et amorcé par sa propre convoitise.
\VS{15}Puis quand la convoitise a conçu, elle enfante le péché ; et le péché, étant consommé, produit la mort.
\VS{16}Mes frères bien-aimés, ne vous y trompez pas :
\VS{17}Tout ce qui nous est donné d'excellent et tout don parfait viennent d'en haut et descendent du Père des lumières, en qui il n'y a ni changement ni ombre de variation.
\VS{18}Il nous a engendrés de sa propre volonté, par la parole de la vérité, afin que nous soyons comme les prémices de ses créatures.
\VS{19}Ainsi, mes frères bien-aimés, que tout homme soit prompt à écouter, lent à parler et lent à la colère ;
\VS{20}car la colère de l'homme n'accomplit pas la justice de Dieu.
\TextTitle{Importance de la mise en pratique de la parole}
\VS{21}C'est pourquoi, rejetant toute souillure et tout résidu\FTNT{Le mot « résidu » vient du grec « perisseia », ce mot signifie « abondance », « surabondamment », « tout excès », « reste ». Les Grecs utilisaient ce terme pour décrire l'excès de cire dans leurs oreilles. Il est question de la méchanceté qui reste dans un Chrétien et qui provient de son état antérieur à sa conversion.} de méchanceté, recevez avec douceur la parole qui a été plantée en vous et qui peut sauver vos âmes.
\VS{22}Et mettez en pratique la parole, et ne l'écoutez pas seulement, en vous trompant vous-mêmes par de vains discours.
\VS{23}Car, si quelqu'un écoute la parole et ne la met pas en pratique, il est semblable à un homme qui regarde dans un miroir son visage naturel,
\VS{24}et qui, après s'être regardé, s'en va, et oublie aussitôt comment il était.
\VS{25}Mais celui qui aura plongé les regards dans la loi parfaite, la loi de la liberté, et qui aura persévéré, n'étant point un auditeur oublieux, mais pratiquant les œuvres qui lui sont prescrites, celui-là sera heureux dans son oeuvre.
\TextTitle{La religion pure et sans tache}
\VS{26}Si quelqu'un parmi vous croit être religieux alors qu'il ne tient pas sa langue en bride, mais séduit son cœur, la religion d'un tel homme est vaine.
\VS{27}La religion pure et sans tache envers notre Dieu et notre Père, c'est de visiter les orphelins et les veuves dans leurs afflictions, et de se conserver pur des souillures de ce monde.
\Chap{2}
\TextTitle{L'amour pour son prochain en pratique}
\VerseOne{}Mes frères, n'ayez point la foi en notre Seigneur Jésus-Christ glorieux, en ayant égard à l'apparence des personnes.
\VS{2}En effet, s'il entre dans votre assemblée un homme qui porte un anneau d'or et un habit magnifique, et qu'il y entre aussi un pauvre misérablement vêtu ;
\VS{3}et que vous ayez égard à celui qui porte l'habit magnifique et lui disiez : Toi, assieds-toi ici honorablement ! Et que vous disiez au pauvre : Toi, tiens-toi là debout ! Ou, assieds-toi ici sur mon marchepied !
\VS{4}n'avez-vous pas fait de différence en vous-mêmes, et n'êtes-vous pas des juges qui avez des pensées injustes ?
\VS{5}Ecoutez, mes frères bien-aimés : Dieu n'a-t-il pas choisi les pauvres de ce monde, qui sont riches en la foi, et héritiers du Royaume qu'il a promis à ceux qui l'aiment ?
\VS{6}Mais vous avez déshonoré le pauvre ! Et cependant les riches ne vous oppriment-ils pas, et ne vous traînent-ils pas devant les tribunaux ?
\VS{7}N'est-ce pas eux qui blasphèment le beau Nom qui a été invoqué sur vous ?
\VS{8}Si, en effet, vous accomplissez la loi royale, qui est selon l'Ecriture : Tu aimeras ton prochain comme toi-même\FTNT{Lé. 19:18.}, vous faites bien.
\VS{9}Mais si vous avez égard à l'apparence des personnes, vous commettez un péché, et vous êtes convaincus par la loi comme des transgresseurs.
\VS{10}Car quiconque observe toute la loi, mais pèche contre un seul commandement, devient coupable de tous.
\VS{11}En effet, celui qui a dit : Tu ne commettras point d'adultère, a dit aussi : Tu ne tueras point. Or, si donc tu ne commets point d'adultère\FTNT{Ex. 20:13-14.}, mais que tu tues, tu deviens transgresseur de la loi.
\VS{12}Ainsi parlez et ainsi agissez comme devant être jugés par la loi de la liberté,
\VS{13}car il y aura un jugement sans miséricorde sur celui qui n'aura point usé de miséricorde\FTNT{Mt. 7:2.} ; mais la miséricorde triomphe du jugement.
\TextTitle{Les œuvres de la foi}
\VS{14}Mes frères, que servira-t-il à quelqu'un de dire qu'il a la foi, s'il n'a pas les œuvres ? Cette foi peut-elle le sauver ?
\VS{15}Et si un frère ou une sœur sont nus et manquent de ce qui leur est nécessaire chaque jour pour vivre,
\VS{16}et que l'un d'entre vous leur dise : Allez en paix, chauffez-vous, et rassasiez-vous ! Et que vous ne leur donniez pas les choses nécessaires pour le corps, que leur servira cela ?
\VS{17}De même aussi la foi, si elle n'a pas les œuvres, elle est morte en elle-même.
\VS{18}Mais quelqu'un dira : Tu as la foi ; et moi, j'ai les œuvres. Montre-moi donc ta foi sans les œuvres, et moi, je te montrerai ma foi par mes œuvres.
\VS{19}Tu crois qu'il n'y a qu'un Dieu, tu fais bien ; les démons le croient aussi, et ils tremblent.
\VS{20}Mais, ô homme vain, veux-tu savoir que la foi qui est sans les œuvres est morte ?
\TextTitle{La foi d'Abraham et de Rahab manifestée dans leurs œuvres\FTNT{Ro. 4:1-25}}
\VS{21}Abraham, notre père, ne fut-il pas justifié par les œuvres, quand il offrit son fils Isaac sur l'autel ?
\VS{22}Ne vois-tu donc pas que sa foi agissait avec ses œuvres, et que ce fut par ses œuvres que sa foi fut rendue parfaite ?
\VS{23}Ainsi s'accomplit ce que dit l'Ecriture : Abraham crut en Dieu, et cela lui fut imputé à justice\FTNT{Ge. 15:6.}; et il fut appelé ami de Dieu.
\VS{24}Vous voyez donc que l'homme est justifié par les œuvres, et non par la foi seulement.
\VS{25}Pareillement, Rahab, la prostituée, ne fut-elle pas également justifiée par les œuvres, lorsqu'elle reçut les messagers, et qu'elle les fit partir par un autre chemin\FTNT{Jos. 2:1-21.} ?
\VS{26}Car, comme le corps sans esprit est mort, de même la foi sans les œuvres est morte.
\Chap{3}
\TextTitle{Les enseignants jugés plus sévèrement}
\VerseOne{}Ne soyez pas nombreux, mes frères, à devenir des enseignants\FTNT{Du grec « didaskalos » : « maître », « professeur », « docteur chargé d'instruire, d'enseigner la parole ». Mt. 8:19 ; Mt. 22:16 ; 1 Co. 12:28.}, sachant que nous en recevrons un plus grand jugement.
\TextTitle{Enseignements sur la langue}
\VS{2}Car nous péchons tous en plusieurs choses. Si quelqu'un ne pèche pas en paroles, c'est un homme parfait, et il peut même tenir en bride tout le corps.
\VS{3}Voici, nous mettons le mors dans la bouche des chevaux, afin qu'ils nous obéissent, et nous menons çà et là tout le corps.
\VS{4}Voici, aussi les navires, quoiqu'ils soient si grands et qu'ils soient agités par la tempête, ils sont dirigés partout çà et là par un petit gouvernail, selon qu'il plaît à celui qui les gouverne.
\VS{5}Il en est ainsi de la langue, c'est un petit membre, et cependant elle peut se vanter de grandes choses. Voici, un petit feu, combien de bois allume-t-il?
\VS{6}La langue aussi est un feu ; c'est le monde de l'iniquité. La langue est placée parmi nos membres, souillant tout le corps, et enflammant tout le cours de la vie, étant elle-même enflammée par le feu de la géhenne.
\VS{7}Car toutes les espèces d'animaux sauvages, d'oiseaux, de reptiles, et d'animaux marins, se domptent et ont été domptés par la nature humaine ;
\VS{8}mais nul homme ne peut dompter la langue ; c'est un mal qu'on ne peut réprimer ; elle est pleine d'un venin mortel.
\VS{9}Par elle nous bénissons Dieu notre Père, et par elle nous maudissons les hommes faits à la ressemblance de Dieu.
\VS{10}De la même bouche sortent la bénédiction et la malédiction. Il ne faut pas qu'il en soit ainsi, mes frères.
\VS{11}Une fontaine fait-elle jaillir par la même ouverture l'eau douce et l'eau amère ?
\VS{12}Mes frères, un figuier peut-il produire des olives, ou une vigne des figues ? De même, aucune fontaine ne peut produire de l'eau salée et de l'eau douce.
\TextTitle{La sagesse humaine et la sagesse d'en haut}
\VS{13}Y a-t-il parmi vous quelque homme sage et intelligent ? Qu'il fasse voir ses oeuvres par une bonne conduite avec douceur et sagesse.
\VS{14}Mais si vous avez une envie amère et un esprit de querelle dans vos cœurs, ne vous glorifiez pas, et ne mentez pas en déshonorant la vérité de l'Evangile.
\VS{15}Car ce n'est pas là la sagesse qui descend d'en haut ; mais c'est une sagesse terrestre, animale\FTNT{Animale ou charnelle.} et diabolique.
\VS{16}Car là où il y a de l'envie et un esprit de querelle, là est le désordre, et toute sorte de mal.
\VS{17}Mais la sagesse d'en haut est premièrement pure, ensuite pacifique, modérée, conciliante, pleine de miséricorde et de bons fruits, sans partialité, et sans hypocrisie.
\VS{18}Or le fruit de la justice est semé dans la paix pour ceux qui s'adonnent à la paix.
\Chap{4}
\TextTitle{Condamnation des mauvais désirs}
\VerseOne{}D'où viennent parmi vous les disputes et les querelles ? N'est-ce pas de vos voluptés qui combattent dans vos membres ?
\VS{2}Vous convoitez, et vous n'obtenez pas ce que vous désirez ; vous avez une envie mortelle, vous êtes jaloux, et vous ne pouvez obtenir ce que vous enviez ; vous vous querellez, vous vous disputez, et vous n'avez pas ce que vous désirez, parce que vous ne demandez pas.
\VS{3}Vous demandez, et vous ne recevez point, parce que vous demandez mal, dans le but de satisfaire vos voluptés.
\VS{4}Hommes et femmes adultères ! Ne savez-vous pas que l'amitié du monde est inimitié contre Dieu ? Celui donc qui veut être ami du monde, se rend ennemi de Dieu.
\VS{5}Pensez-vous que l'Ecriture parle en vain ? L'Esprit qui habite en nous, vous inspire-t-il l'envie ?
\TextTitle{S'humilier devant Yahweh, le juste Juge}
\VS{6}Il vous accorde, au contraire, une plus grande grâce ; c'est pourquoi l'Ecriture dit : Dieu résiste aux orgueilleux, mais il fait grâce aux humbles\FTNT{Pr. 3:34.}.
\VS{7}Soumettez-vous donc à Dieu ; résistez au diable, et il s'enfuira de vous.
\VS{8}Approchez-vous de Dieu, et il s'approchera de vous. Pécheurs, nettoyez vos mains ; et vous qui êtes doubles de cœur, purifiez vos cœurs.
\VS{9}Sentez vos misères ; et soyez dans le deuil et dans les larmes ; que votre rire se change en pleurs, et votre joie en tristesse.
\VS{10}Humiliez-vous dans la présence du Seigneur, et il vous élèvera.
\VS{11}Mes frères, ne médisez point les uns des autres. Celui qui médit de son frère, et qui condamne son frère, médit de la loi et juge la loi. Or, si tu juges la loi, tu n'es pas observateur de la loi, mais le juge.
\VS{12}Il n'y a qu'un seul Législateur, qui peut sauver et qui peut perdre ; mais toi, qui es-tu, qui juges les autres ?
\TextTitle{Abandonner ses désirs au profit de la volonté de Dieu}
\VS{13}A vous, maintenant, qui dites : Aujourd'hui ou demain nous irons dans telle ou telle ville, et nous y passerons une année, et nous trafiquerons et nous gagnerons !
\VS{14}Qui toutefois ne savez pas ce qui arrivera le lendemain ! Car qu'est-ce que votre vie ? Ce n'est certes qu'une vapeur qui parait pour un peu de temps, et qui ensuite s'évanouit.
\VS{15}Au lieu de dire : Si le Seigneur le veut, et si nous vivons, nous ferons aussi ceci ou cela.
\VS{16}Mais maintenant vous vous glorifiez dans vos pensées orgueilleuses. Une telle gloire est mauvaise.
\VS{17}Il y a donc du péché en celui qui sait faire le bien, et qui ne le fait pas.
\Chap{5}
\TextTitle{Avertissement aux riches}
\VerseOne{}A vous maintenant, riches ! Pleurez et gémissez à cause des malheurs qui vont tomber sur vous.
\VS{2}Vos richesses sont pourries et vos vêtements sont rongés par les vers.
\VS{3}Votre or et votre argent sont rouillés ; et leur rouille s'élèvera en témoignage contre vous et dévorera vos chairs comme un feu. Vous avez amassé des trésors pour les derniers jours.
\VS{4}Voici, le salaire des ouvriers qui ont moissonné vos champs, et dont vous les avez frustrés, crie ; et les cris des moissonneurs sont parvenus aux oreilles du Seigneur des armées.
\VS{5}Vous avez vécu dans les délices sur la terre, vous vous êtes livrés aux voluptés, et vous avez rassasié vos cœurs comme en un  jour de sacrifices.
\VS{6}Vous avez condamné et mis à mort le juste qui ne vous a pas résisté.
\TextTitle{Se préparer à l'avènement du Seigneur}
\VS{7}Mais vous, mes frères, attendez patiemment jusqu'à l'avènement du Seigneur. Voici, le laboureur attend le précieux fruit de la terre, prenant patience à son égard, jusqu'à ce qu'il ait reçu les pluies de la première et de la dernière saison.
\VS{8}Vous aussi, attendez patiemment, et affermissez vos cœurs, car l'avènement du Seigneur est proche.
\VS{9}Mes frères, ne vous plaignez pas les uns des autres, afin que vous ne soyez pas condamnés. Voici, le Juge se tient à la porte.
\VS{10}Mes frères, prenez pour exemple de patience dans les afflictions les prophètes qui ont parlé au Nom du Seigneur.
\VS{11}Voici, nous tenons pour bienheureux ceux qui ont enduré l'épreuve avec patience. Vous avez appris quelle a été la patience de Job, et vous avez vu la fin du Seigneur, car le Seigneur est plein de compassion et de miséricorde.
\VS{12}Avant toutes choses, mes frères, ne jurez ni par le ciel, ni par la terre, ni par aucun autre serment. Mais que votre oui soit oui, et que votre non soit non, afin que vous ne tombiez pas sous le jugement\FTNT{Mt. 5:37 ; Mt. 12:36.}.
\VS{13}Quelqu'un parmi vous est-il dans la souffrance ? Qu'il prie. Quelqu'un est-il dans la joie ? Qu'il chante.
\VS{14}Quelqu'un parmi vous est-il malade ? Qu'il appelle les anciens de l'église, et qu'ils prient pour lui en l'oignant d'huile au Nom du Seigneur.
\VS{15}Et la prière faite avec foi sauvera le malade, et le Seigneur le relèvera ; et s'il a commis des péchés, ils lui seront pardonnés.
\VS{16}Confessez donc vos péchés les uns les autres, et priez les uns pour les autres afin que vous soyez guéris. Car la prière du juste faite avec ferveur est de grande efficacité.
\VS{17}Elie était un homme sujet aux mêmes infirmités que nous, et cependant il pria avec instance pour qu'il ne pleuve point, et il ne tomba point de pluie sur la terre pendant trois ans et six mois\FTNT{1 R. 17:1.}.
\VS{18}Puis il pria de nouveau, et le ciel donna de la pluie, et la terre produisit son fruit.
\TextTitle{Conclusion}
\VS{19}Mes frères, si quelqu'un parmi vous s'est égaré loin de la vérité, et qu'un autre l'y ramène,
\VS{20}qu'il sache que celui qui ramènera un pécheur de son égarement, sauvera une âme de la mort et couvrira une multitude de péchés.
\PPE{}
\end{multicols}

\clearpage\ShortTitle{Ga.}\BookTitle{Galates}\BFont
\noindent\hrulefill
{\footnotesize
\textit{
\bigskip
{\centering{}
\\Auteur~: Paul
\\Thème~: Le salut par la grâce
\\Date de rédaction~: Env. 50 ap. J.-C.\\}
}
\textit{
\\Province antique de l'Asie Mineure, la Galatie se situait en Anatolie. Elle devait son nom aux Galates, Celtes provenant des Balkans.
\\La lettre de Paul aux Galates est la seule épître dont le début ne contient pas de témoignage d'affection. Paul commence par justifier l'origine de son appel, en employant un ton sec et sévère. Les Galates, qu'il avait lui-même évangélisés lors de son premier voyage, s'étaient promptement détournés de l'Evangile qu'ils avaient reçu. Ils ne l'avaient pas totalement abandonné, mais y avaient ajouté ce qui ne leur avait point été prescrit. Troublés par les enseignements des judaïsants - des juifs ayant cru en Jésus-Christ, mais persistant toujours dans la pratique de la loi - les Galates avaient repris à leur compte leurs traditions, annihilant ainsi l'œuvre de la croix. Par cette lettre, Paul les exhorte d'une part à revenir à l'Evangile véritable et d'autre part à marcher par l'Esprit afin d'en porter le fruit.\bigskip
}
}
\par\nobreak\noindent\hrulefill
\begin{multicols}{2}
\Chap{1}
\TextTitle{Introduction}
\VerseOne{}Paul, apôtre, non de la part des hommes, ni de la part d'aucun homme, mais de la part de Jésus-Christ, et de la part de Dieu le Père, qui l'a ressuscité des morts,
\VS{2}et tous les frères qui sont avec moi, aux églises de Galatie\FTNT{La Galatie, ou Gallo-Grèce, était une province de l'Asie Mineure (région de la Turquie actuelle). Au nord, elle était délimitée par la Bithynie et la
Paphlagonie, à l'est par le Pont et la Cappadoce, au sud par la Cappadoce, la Lycaonie et la Phrygie, et à l'ouest par la Phrygie et la Bithynie. Son nom vient des Gaulois qui s'étaient installés dans la région en 279 av. J.-C. Conquise par les Romains en 189 av. J.-C., elle devint une province de l'Empire en 25 av. J.-C.} :
\VS{3}Que la grâce et la paix vous soient données de la part de Dieu le Père, et de la part de notre Seigneur Jésus-Christ,
\VS{4}qui s'est donné lui-même pour nos péchés, afin de nous arracher du présent siècle mauvais, selon la volonté de Dieu notre Père.
\VS{5}A lui soit la gloire aux siècles des siècles. Amen~!
\TextTitle{Les Galates se détournent de l'Evangile véritable}
\VS{6}Je m'étonne que vous abandonniez si promptement celui qui vous avait appelés à la grâce de Christ, pour passer à un autre évangile. 
\VS{7}Non qu'il y ait un autre évangile, mais il y a des gens qui vous troublent, et qui veulent renverser l'Evangile de Christ.
\VS{8}Mais quand nous-mêmes, ou quand un ange venu du ciel vous évangéliserait, outre\FTNT{Voir 1 R. 13:11-34.} ce que nous vous avons évangélisé, qu'il soit anathème~!
\VS{9}Comme nous l'avons déjà dit, je le dis encore maintenant~: Si quelqu'un vous évangélise outre ce que vous avez reçu, qu'il soit anathème~!
\TextTitle{Paul reçoit la révélation de l'Evangile}
\VS{10}Car est-ce les hommes que je prêche ou Dieu~? Ou est-ce que je cherche à plaire aux hommes~? Certes si je plaisais encore aux hommes, je ne serais pas le serviteur de Christ.
\VS{11}Je vous le déclare donc, mes frères, que l'Evangile que j'ai annoncé n'est pas selon l'homme,
\VS{12}parce que je ne l'ai ni reçu ni appris d'aucun homme, mais par la révélation de Jésus-Christ.
\VS{13}Car vous avez appris quelle a été autrefois ma conduite dans le judaïsme, et comment je persécutais à outrance l'Eglise de Dieu et la ravageais,
\VS{14}et comment j'étais plus avancé dans le judaïsme que beaucoup de ceux de mon âge et de ma nation, étant le plus ardent zélateur des traditions de mes pères.
\VS{15}Mais quand il a plu à Dieu, qui m'avait choisi dès le ventre de ma mère, et qui m'a appelé par sa grâce,
\VS{16}de révéler en moi son Fils, afin que je le prêche parmi les Gentils, aussitôt, je ne consultai ni la chair ni le sang,
\VS{17}et je ne montai point à Jérusalem vers ceux qui furent apôtres avant moi, mais je partis pour l'Arabie, puis je revins encore à Damas.
\VS{18}Ensuite, trois ans après, je montai à Jérusalem pour visiter Pierre, et je demeurai chez lui quinze jours.
\VS{19}Et je ne vis aucun des autres apôtres, sinon Jacques, le frère du Seigneur.
\VS{20}Or, dans les choses que je vous écris, voici, devant Dieu je vous dis que je ne mens point.
\VS{21}J'allai ensuite dans les pays de Syrie et de Cilicie.
\VS{22}Or j'étais inconnu de visage aux églises de Judée qui sont en Christ,
\VS{23}mais elles avaient seulement entendu dire~: Celui qui autrefois nous persécutait, annonce maintenant la foi qu'il détruisait autrefois.
\VS{24}Et elles glorifiaient Dieu à cause de moi.
\Chap{2}
\TextTitle{Paul et Barnabas se rendent à Jérusalem\FTNTT{Ac. 15.}}
\VerseOne{}Quatorze ans après, je montai de nouveau à Jérusalem\FTNT{La grande assemblée de Jérusalem. Voir Ac. 15.}, avec Barnabas, et je pris aussi avec moi Tite.
\VS{2}Et ce fut d'après une révélation que j'y montai. J'exposai l'Evangile que je prêche parmi les Gentils à ceux de Jérusalem, en particulier à ceux qui sont les plus considérés, afin de ne pas courir ou avoir couru en vain.
\VS{3}Et même on n'obligea pas Tite, qui était avec moi, de se faire circoncire quoiqu'il fût Grec.
\VS{4}Et cela à cause des faux frères qui s'étaient furtivement introduits et glissés dans l'église pour épier la liberté que nous avons en Jésus-Christ, afin de nous ramener dans la servitude.
\VS{5}Nous ne leur cédâmes pas un instant et nous résistâmes à leurs exigences, afin que la vérité de l'Evangile soit maintenue parmi vous.
\VS{6}Et je ne suis différent en rien de ceux qui sont les plus estimés, quels qu'ils aient été autrefois, Dieu n'ayant point d'égard à l'apparence extérieure de l'homme car ceux qui sont en estime ne m'ont rien communiqué de plus.
\VS{7}Au contraire, quand ils virent que la prédication de l'Evangile pour les incirconcis m'avait été confiée, comme à Pierre pour les circoncis,
\VS{8}car celui qui a opéré avec efficacité par Pierre dans la charge d'apôtre pour les circoncis, a aussi opéré avec efficacité par moi envers les Gentils.
\VS{9}Jacques, dis-je, Céphas, et Jean, qui sont estimés comme des colonnes, ayant reconnu la grâce que j'avais reçue, me donnèrent, à moi et à Barnabas, la main d'association, afin que nous allions, nous vers les Gentils, et qu'ils aillent eux vers les circoncis.
\VS{10}Ils nous recommandèrent seulement de nous souvenir des pauvres, ce que j'ai eu bien soin de faire.
\TextTitle{Paul reprend Pierre à Antioche}
\VS{11}Mais lorsque Pierre vint à Antioche, je lui résistai en face parce qu'il méritait d'être repris.
\VS{12}Car avant l'arrivée de quelques personnes envoyées par Jacques, il mangeait avec les Gentils, mais quand elles furent venues, il s'esquiva et se sépara des Gentils, craignant les circoncis.
\VS{13}Les autres Juifs aussi usèrent de dissimulation comme lui, de sorte que Barnabas même se laissait entraîner par leur hypocrisie.
\VS{14}Mais quand je vis qu'ils ne marchaient pas droit selon la vérité de l'Evangile, je dis à Pierre devant tous~: Si toi qui es Juif, tu vis comme les Gentils, et non pas comme les Juifs, pourquoi contrains-tu les Gentils à judaïser~?
\TextTitle{Le chrétien est mort à la loi mosaïque}
\VS{15}Nous qui sommes Juifs de naissance, et non point pécheurs d'entre les Gentils,
\VS{16}sachant que l'homme n'est pas justifié par les œuvres de la loi, mais seulement par la foi en Jésus-Christ\FTNT{La justification. Voir Ro. 5:1.}, nous, dis-je, nous avons cru en Jésus-Christ, afin que nous soyons justifiés par la foi en Christ, et non point par les œuvres de la loi~; parce que personne ne sera justifié par les œuvres de la loi.
\VS{17}Or si en cherchant à être justifiés par Christ, nous sommes aussi trouvés pécheurs, Christ est-il pourtant serviteur du péché~? A Dieu ne plaise~!
\VS{18}Car si je rebâtis les choses que j'ai renversées, je montre que je suis moi-même un transgresseur.
\VS{19}Car c'est par la loi que je suis mort à la loi, afin de vivre pour Dieu.
\TextTitle{La vie chrétienne doit refléter la vie de Jésus-Christ\FTNTT{Ga. 5:15-23.}}
\VS{20}Je suis crucifié avec Christ~; et si je vis, ce n'est plus moi qui vis, c'est Christ qui vit en moi~; si je vis maintenant dans la chair, je vis dans la foi au Fils de Dieu, qui m'a aimé et qui s'est livré lui-même pour moi.
\VS{21}Je n'anéantis point la grâce de Dieu, car si la justice vient de la loi, Christ est donc mort inutilement.
\Chap{3}
\TextTitle{L'Esprit s'acquiert par la foi}
\VerseOne{}Ô Galates insensés~! Qui vous a ensorcelés pour faire que vous n'obéissiez point à la vérité, vous, aux yeux de qui Jésus-Christ a été auparavant dépeint crucifié, au milieu de vous~?
\VS{2}Je voudrais seulement entendre ceci de vous~: Avez-vous reçu l'Esprit par les œuvres de la loi, ou par la prédication de la foi~?
\VS{3}Etes-vous si insensés, qu'après avoir commencé par l'Esprit, voulez-vous maintenant finir par la chair~?
\VS{4}Avez-vous tant souffert en vain~? Si toutefois c'est en vain.
\VS{5}Celui donc qui vous donne l'Esprit, et qui produit en vous les dons miraculeux, le fait-il par les œuvres de la loi ou par la prédication de la foi~?
\TextTitle{L'alliance avec Abraham, une promesse fondée sur la foi\FTNTT{Ro. 4.}}
\VS{6}Comme Abraham crut à Dieu, et cela lui fut imputé à justice,
\VS{7}sachez donc que ce sont ceux qui ont la foi qui sont fils d'Abraham.
\VS{8}Aussi, l'Ecriture prévoyant que Dieu justifierait les Gentils par la foi, a auparavant évangélisé à Abraham, en lui disant~: Toutes les nations seront bénies en toi\FTNT{Ge. 12:3.}.
\VS{9}C'est pourquoi ceux qui ont la foi sont bénis avec Abraham, le croyant.
\TextTitle{L'attachement aux œuvres de la loi produit la malédiction}
\VS{10}Car tous ceux qui s'attachent aux œuvres de la loi sont sous la malédiction~; car il est écrit~: Maudit est quiconque ne persévère pas dans toutes les choses qui sont écrites dans le livre de la loi et ne les met pas en pratique\FTNT{De. 27:26.}.
\VS{11}Et que nul ne soit justifié devant Dieu par la loi, cela est évident, puisqu'il est dit~: Le juste vivra de la foi\FTNT{Ha. 2:4.}.
\VS{12}Or la loi ne procède pas de la foi, mais elle dit~: L'homme qui mettra ces choses en pratique vivra par elles\FTNT{Lé. 18:5.}.
\TextTitle{Le Messie a racheté les chrétiens de la malédiction de la loi}
\VS{13}Christ nous a rachetés de la malédiction de la loi quand il a été fait malédiction pour nous~; car il est écrit~: Maudit est quiconque est pendu au bois\FTNT{De. 21:23.},
\VS{14}afin que la bénédiction d'Abraham ait son accomplissement pour les Gentils en Jésus-Christ, et que nous recevions par la foi l'Esprit qui avait été promis.
\VS{15}Mes frères, je parle à la manière des hommes, un testament en bonne forme, bien que fait par un homme, n'est annulé par personne, et personne n'y ajoute.
\VS{16}Or les promesses ont été faites à Abraham et à sa postérité. Il n'est pas dit~: Et aux postérités, comme s'il avait parlé de plusieurs, mais comme parlant d'une seule, et à sa postérité, c'est-à-dire Christ.
\VS{17}Voici ce que j'entends~: Une alliance, que Dieu a confirmée antérieurement, ne peut pas être annulée, et ainsi la promesse rendue vaine, par la loi survenue quatre cent trente ans plus tard.
\VS{18}Car si l'héritage venait de la loi, il ne viendrait plus de la promesse. Or c'est par la promesse que Dieu a fait à Abraham ce don de sa grâce.
\TextTitle{La loi~: Pédagogue révélant le péché et conduisant à Christ}
\VS{19}A quoi donc sert la loi~? Elle a été donnée ensuite à cause des transgressions, jusqu'à ce que vienne la postérité à qui la promesse avait été faite~; et elle a été promulguée par des anges, au moyen d'un médiateur.
\VS{20}Or le médiateur n'est pas médiateur d'un seul, mais Dieu est un seul.
\VS{21}La loi a-t-elle donc été ajoutée contre les promesses de Dieu~? Nullement ! Car s'il avait été donné une loi qui puisse procurer la vie, la justice viendrait réellement de la loi.
\VS{22}Mais l'Ecriture a renfermé tous les hommes sous le péché, afin que ce qui avait été promis soit donné par la foi en Jésus-Christ à ceux qui croient.
\VS{23}Or avant que la foi vienne, nous étions renfermés sous la garde de la loi, en vue de la foi qui devait être révélée.
\VS{24}Ainsi la loi a donc été notre pédagogue\FTNT{Le mot «~pédagogue~» du grec «~paidagogos~»~: «~celui qui dirige un garçon~». Un pédagogue était un tuteur, un gardien et un guide de garçons. Parmi les Grecs et les Romains, le mot était appliqué aux esclaves dignes de confiance qui étaient chargés de veiller à la vie et à la moralité des garçons appartenant aux classes supérieures. Les garçons ne pouvaient faire le moindre pas hors de la maison sans ces tuteurs tant qu'ils n'avaient pas atteint leur majorité.} pour nous amener à Christ, afin que nous soyons justifiés par la foi.
\VS{25}Mais la foi étant venue, nous ne sommes plus sous ce pédagogue.
\TextTitle{Ceux qui croient au Messie sont justifiés}
\VS{26}Parce que vous êtes tous fils de Dieu par la foi en Jésus-Christ,
\VS{27}car vous tous qui avez été baptisés en Christ, vous avez revêtu Christ.
\VS{28}Il n'y a plus ni Juif ni Grec, il n'y a plus ni esclave ni libre, il n'y a plus ni homme ni femme~; car vous êtes tous un en Jésus-Christ\FTNT{Ro. 10:12~; Col. 3:11.}.
\VS{29}Et si vous êtes de Christ, vous êtes donc la postérité d'Abraham, et héritiers selon la promesse.
\Chap{4}
\VerseOne{}Or aussi longtemps que l'héritier est enfant\FTNT{«~Enfant~», du grec «~nepios~», signifie aussi «~ignorant~».}, je dis qu'il ne diffère en rien d'un esclave, quoiqu'il soit le maître de tout.
\VS{2}Mais il est sous des tuteurs et des administrateurs jusqu'au temps déterminé par le Père.
\VS{3}Nous aussi, lorsque nous étions enfants, nous étions sous l'esclavage des rudiments du monde.
\VS{4}Mais lorsque les temps ont été accomplis, Dieu a envoyé son Fils, né d'une femme, né sous la loi,
\VS{5}afin qu'il rachète ceux qui étaient sous la loi, afin que nous recevions l'adoption.
\VS{6}Et parce que vous êtes fils, Dieu a envoyé l'Esprit de son Fils dans vos cœurs, lequel crie~: Abba ! C'est-à-dire Père.
\VS{7}Maintenant donc tu n'es plus esclave, mais fils~; or si tu es fils, tu es aussi héritier de Dieu par Christ.
\TextTitle{Le légalisme et la religiosité privent de la grâce}
\VS{8}Autrefois, ne connaissant pas Dieu, vous serviez des dieux qui ne le sont pas de leur nature.
\VS{9}Et maintenant que vous avez connu Dieu, ou plutôt que vous avez été connus de Dieu, comment retournez-vous encore à ces faibles et misérables éléments, auxquels vous voulez encore vous asservir comme auparavant~?
\VS{10}Vous observez les jours, les mois, les temps et les années.
\VS{11}Je crains d'avoir travaillé inutilement pour vous.
\VS{12}Soyez comme moi~; car je suis aussi comme vous~; je vous en prie mes frères.
\VS{13}Vous ne m'avez fait aucun tort. Et vous savez que ce fut à cause d'une infirmité de la chair\FTNT{Les Ecritures ne donnent pas de précisions au sujet de l'infirmité de la chair dont souffrait Paul. On suppose toutefois qu'il avait un handicap au niveau de ses yeux. Quatre arguments viennent renforcer cette hypothèse. Tout d'abord, l'allusion de Paul aux Galates qui étaient prêts à «~s'arracher les yeux~» pour les lui donner (Ga. 4:15) et le fait qu'il ait lui-même écrit cette épître avec de «~grandes lettres~» (Ga. 6:11). Ensuite, lors de sa comparution devant le sanhédrin à Jérusalem, Paul n'a pas reconnu le grand-prêtre pourtant facilement identifiable par sa tenue vestimentaire (Ac. 23:5). Enfin, l'apôtre avait l'habitude de dicter ses lettres, ce qui constitue un argument majeur. L'épître aux Galates était une exception parce qu'il n'avait sans doute pas de secrétaire à disposition.} que je vous ai pour la première fois évangélisés.
\VS{14}Et vous ne m'avez point méprisé ni rejeté à cause de ces épreuves que j'ai dans ma chair~; mais vous m'avez reçu comme un ange de Dieu, et comme Jésus-Christ.
\VS{15}Où donc est l'expression de votre bonheur~? Car je vous atteste que, si cela avait été possible, vous vous seriez arrachés les yeux pour me les donner.
\VS{16}Suis-je donc devenu votre ennemi en vous disant la vérité~?
\VS{17}Ils ont du zèle pour vous, mais non loyalement. Au contraire, ils veulent vous détacher de nous afin que vous soyez zélés pour eux.
\VS{18}Il est bon d'être zélé pour le bien en tout temps, et non pas seulement quand je suis présent parmi vous.
\TextTitle{La loi et la grâce ne peuvent cohabiter~: Agar et Sara représentent deux alliances}
\VS{19}Mes petits enfants, pour qui j'éprouve de nouveau les douleurs de l'enfantement, jusqu'à ce que Christ soit formé en vous,
\VS{20}je voudrais être maintenant avec vous, et changer de langage, car je suis dans une grande inquiétude à votre sujet.
\VS{21}Dites-moi, vous qui voulez être sous la loi, ne comprenez-vous point la loi~?
\VS{22}Car il est écrit qu'Abraham eut deux fils, un de l'esclave, et un de la femme libre.
\VS{23}Mais celui de l'esclave naquit selon la chair~; et celui de la femme libre naquit en vertu de la promesse.
\VS{24}Ces faits ont une valeur allégorique, car ces deux femmes sont deux alliances~: L'une du Mont Sinaï, qui n'enfante que des esclaves, et c'est Agar.
\VS{25}Car le nom d'Agar veut dire Sinaï, qui est une montagne en Arabie correspondant à la Jérusalem actuelle qui est dans la servitude avec ses enfants.
\VS{26}Mais la Jérusalem d'en haut est la femme libre, et c'est notre mère à nous tous.
\VS{27}Car il est écrit~: Réjouis-toi, stérile, toi qui n'enfantes point~! Eclate et pousse des cris, toi qui n'as pas éprouvé les douleurs de l'enfantement~! Car les enfants de la délaissée seront plus nombreux que les enfants de celle qui était mariée\FTNT{Es. 54:1.}.
\VS{28}Or pour nous, mes frères, nous sommes enfants de la promesse comme Isaac.
\VS{29}Et de même qu'alors, celui qui était né selon la chair persécutait celui qui était né selon l'Esprit, il en est de même maintenant.
\VS{30}Mais que dit l'Ecriture~? Chasse l'esclave et son fils, car le fils de l'esclave n'héritera pas avec le fils de la femme libre\FTNT{Ge. 21:10.}.
\VS{31}C'est pourquoi, mes frères, nous ne sommes pas enfants de l'esclave, mais de la femme libre.
\Chap{5}
\TextTitle{Le Messie nous a libérés de la servitude}
\VerseOne{}Demeurez donc fermes dans la liberté pour laquelle Christ nous a affranchis, et ne vous mettez plus sous le joug de la servitude.
\VS{2}Moi, Paul, je vous dis que si vous vous faites circoncire, Christ ne vous servira à rien.
\VS{3}Et j'affirme encore une fois à tout homme qui se fait circoncire qu'il est tenu de pratiquer la loi tout entière.
\VS{4}Vous êtes séparés de Christ, vous tous qui cherchez la justification dans la loi~; vous êtes déchus de la grâce.
\VS{5}Mais pour nous, nous attendons par l'Esprit l'espérance d'être justifiés par la foi.
\VS{6}Car en Jésus-Christ ni la circoncision ni le prépuce\FTNT{Voir le commentaire en 1 Co. 7:18.} n'ont de valeur, mais seulement la foi qui opère par la charité.
\VS{7}Vous couriez bien~: Qui vous a arrêtés pour vous empêcher d'obéir à la vérité~?
\VS{8}Cette influence ne vient pas de celui qui vous appelle.
\VS{9}Un peu de levain fait lever toute la pâte\FTNT{1 Co. 5:6.}.
\VS{10}J'ai cette confiance en vous dans le Seigneur que vous n'aurez pas d'autre sentiment~; mais celui qui vous trouble, quel qu'il soit, en portera la condamnation.
\VS{11}Quant à moi, mes frères, si je prêche encore la circoncision, pourquoi suis-je encore persécuté~? Le scandale de la croix est donc aboli.
\VS{12}Plaise à Dieu que ceux qui vous troublent soient retranchés~!
\VS{13}Car, mes frères, vous avez été appelés à la liberté, seulement ne faites pas de cette liberté une occasion de vivre selon la chair, mais servez-vous les uns les autres avec charité.
\VS{14}Car toute la loi est accomplie dans cette seule parole~: Tu aimeras ton prochain comme toi-même\FTNT{Lé. 19:18~; Mt. 22:39.}.
\VS{15}Mais si vous vous mordez et vous dévorez les uns les autres, prenez garde que vous ne soyez détruits les uns par les autres.
\VS{16}Je vous dis donc~: Marchez selon l'Esprit, et vous n'accomplirez point les désirs de la chair.
\TextTitle{La chair et ses œuvres s'opposent à l'Esprit de Dieu\FTNTT{Ro. 8:2.}}
\VS{17}Car la chair a des désirs contraires à ceux de l'Esprit, et l'Esprit en a des contraires à ceux de la chair~; et ils sont opposés entre eux afin que vous ne fassiez point ce que vous voudriez.
\VS{18}Or si vous êtes conduits par l'Esprit, vous n'êtes point sous la loi.
\VS{19}Car les œuvres de la chair sont évidentes~: Ce sont l'adultère, la fornication, l'impureté, l'impudicité,
\VS{20}l'idolâtrie, la sorcellerie\FTNT{La sorcellerie~: du grec «~pharmakeia~»~: «~usage~» ou «~administration~» de drogues, «~empoisonnement~», «~sorcellerie~», «~arts magiques~», souvent trouvés en liaison avec l'idolâtrie et nourrie par celle-ci.}, les inimitiés, les querelles, les jalousies, les animosités, les disputes, les divisions, les sectes,
\VS{21}les envies, les meurtres, l'ivrognerie, les excès de table, et les choses semblables à celles-là, au sujet desquelles je vous prédis, comme je vous l'ai déjà dit, que ceux qui commettent de telles choses n'hériteront point le Royaume de Dieu.
\TextTitle{Le fruit de l'Esprit\FTNTT{Jn. 15:1-5~; Ga. 2:20.}}
\VS{22}Mais le fruit de l'Esprit c'est la charité\FTNT{Il est question ici de l'amour «~agape~»~: l'amour fraternel, la charité désintéressée.}, la joie, la paix, la patience, la bonté, la bienveillance, la foi, la douceur, la tempérance.
\VS{23}La loi n'est pas contre ces choses.
\VS{24}Ceux qui sont à Christ ont crucifié la chair avec ses passions et ses désirs.
\VS{25}Si nous vivons par l'Esprit, marchons aussi par l'Esprit.
\VS{26}Ne cherchons pas une vaine gloire, en nous provoquant les uns les autres et en nous portant envie les uns aux autres.
\Chap{6}
\TextTitle{La mise en pratique de la vie nouvelle en Jésus-Christ}
\VerseOne{}Mes frères, lorsqu'un homme est surpris en quelque faute, vous qui êtes spirituels, redressez-le avec un esprit de douceur. Prends garde à toi-même, de peur que tu ne sois aussi tenté.
\VS{2}Portez les fardeaux les uns des autres, et vous accomplirez ainsi la loi de Christ.
\VS{3}Car si quelqu'un pense être quelque chose, quoiqu'il ne soit rien, il s'abuse lui-même.
\VS{4}Que chacun examine ses propres œuvres, et alors il aura de quoi se glorifier pour lui-même seulement, et non par rapport aux autres.
\VS{5}Car chacun portera son propre fardeau.
\VS{6}Que celui à qui l'on enseigne la parole fasse part en tous biens à celui qui l'enseigne\FTNT{Le mot «~bien~» vient du grec «~agathos~» qui donne en français~: «~de bonne constitution ou nature~», «~utile~», «~salutaire~», «~bon~», «~agréable~», «~plaisant~», «~joyeux~», «~heureux~», «~excellent~», «~distingué~», «~droit~», «~honorable~» et n'a rien à voir avec les biens matériels (Voir Ga. 6:10). Il ne doit en aucun cas servir de prétexte à ceux qui enseignent la Parole de Dieu pour exiger l'argent et les biens matériels des chrétiens. Ces derniers doivent donner sans contrainte, s'ils le veulent et comme ils le veulent (2 Co. 9:7). Le salaire de l'ouvrier du Seigneur c'est avant tout le gîte et le couvert (Mt. 10:10~; Lu. 10:8~; 1 Ti. 6:8). Ainsi, malgré le droit qu'il avait de moissonner les biens matériels pour avoir semé des biens spirituels (1 Co. 9:11-12), Paul «~n'a désiré ni l'or ni l'argent~» mais a travaillé de ses propres mains afin de pourvoir à ses besoins et de n'être à la charge de personne (Ac. 20:33-35~; 1 Th. 2:9~; 2 Th. 3:8~; 2 Co. 12:14 ).}.
\VS{7}Ne vous séduisez pas, on ne se moque pas de Dieu. Ce qu'un homme aura semé, il le moissonnera aussi.
\VS{8}C'est pourquoi celui qui sème pour sa chair moissonnera de la chair la corruption~; mais celui qui sème pour l'Esprit moissonnera de l'Esprit la vie éternelle.
\VS{9}Ne nous lassons pas de faire le bien~; car nous moissonnerons au temps convenable, si nous ne nous relâchons pas.
\VS{10}C'est pourquoi, pendant que nous en avons le temps, faisons du bien envers tous, mais principalement envers ceux qui sont de la famille de la foi.
\VS{11}Vous voyez avec quelles grandes lettres je vous ai écrit de ma propre main.
\VS{12}Tous ceux qui veulent se rendre agréables selon la chair vous contraignent à vous faire circoncire, uniquement afin de ne pas être persécutés pour la croix de Christ.
\VS{13}Car les circoncis eux-mêmes n'observent pas la loi~; mais ils veulent que vous soyez circoncis pour se glorifier dans votre chair.
\VS{14}Pour ce qui me concerne, loin de moi la pensée de me glorifier d'autre chose que de la croix de notre Seigneur Jésus-Christ, par qui le monde est crucifié pour moi, comme je le suis pour le monde~!
\VS{15}Car ce n'est rien que d'être circoncis ou incirconcis~; ce qui est quelque chose c'est d'être une nouvelle créature.
\VS{16}Que la paix et la miséricorde soient sur tous ceux qui suivront cette règle, et sur l'Israël de Dieu !
\TextTitle{Conclusion}
\VS{17}Au reste, que personne ne me fasse de la peine, car je porte sur mon corps les marques du Seigneur Jésus.
\VS{18}Mes frères, que la grâce de notre Seigneur Jésus-Christ soit avec votre esprit~! Amen~!
\PPE{}
\end{multicols}

\clearpage\ShortTitle{1 Th.}\BookTitle{1 Thessaloniciens}\BFont
\noindent\hrulefill
{\footnotesize
\textit{
\bigskip
{\centering{}
\\Auteur~: Paul
\\Thème~: Le retour de Christ
\\Date de rédaction~: Env. 51 ap. J.-C.\\}
}
\textit{
\\Autrefois appelée Therme ou Therma, qui signifie «~source chaude~», Thessalonique reçut son nouveau nom de Cassandre, en l'honneur de sa femme Thessalonike, qui était aussi la sœur d'Alexandre le Grand (356 av. J.-C. - 323 av. J.-C.), à qui il succéda. Cette ville est située au nord de la Grèce actuelle, sur la côte de la mer Egée. Du temps de Paul, ce pays était divisé en deux parties. Dans la région du nord, la Macédoine, se trouvaient les villes de Philippes, Thessalonique et Bérée. Quant à la région du sud, l'Achaïe, elle comportait les villes d'Athènes et de Corinthe. Aujourd'hui, la ville s'appelle Salonique.
\\En ce temps-là, Thessalonique comptait environ 200 000 habitants (Grecs, Romains et Juifs) et jouissait d'une importante fréquentation puisqu'elle figurait parmi les trois ports principaux de la Méditerranée et se situait sur l'une des plus grandes routes commerciales de l'époque~: La Voie Egnatienne reliant Rome à Byzance.
\\Sur le plan religieux, les habitants étaient polythéistes et pratiquaient une variété de cultes, dont le culte impérial. Durant trois semaines, Paul enseigna dans une synagogue à Thessalonique et réussit à constituer un groupe de croyants composé de Juifs, de Gentils, de pauvres et de plusieurs femmes de la haute société. Toutefois, une violente persécution l'obligea à quitter promptement la ville, laissant la communauté nouvellement formée vulnérable et fragile.
\\La première épître adressée par Paul aux Thessaloniciens leur parvint quelques mois après le passage de l'équipe apostolique et après la visite de Timothée. Cette lettre avait pour but d'affermir les Thessaloniciens dans les vérités fondamentales qui leur avaient été enseignées, de les exhorter à vivre une vie de sainteté pour être agréables à Dieu, de les éclairer quant au devenir des défunts et de les assurer du retour certain du Seigneur.\bigskip
}
}
\par\nobreak\noindent\hrulefill
\begin{multicols}{2}
\Chap{1}
\TextTitle{Introduction}
\VerseOne{}Paul, et Silvain, et Timothée, à l'église des Thessaloniciens qui est en Dieu le Père, et en Jésus-Christ, notre Seigneur~: Que la grâce et la paix vous soient données de la part de Dieu notre Père, et du Seigneur Jésus-Christ~!
\VS{2}Nous rendons toujours grâces à Dieu pour vous tous, faisant mention de vous dans nos prières,
\VS{3}en nous rappelant sans cesse l'œuvre de votre foi, le travail de votre charité, et l'immuabilité de votre espérance en notre Seigneur Jésus-Christ devant notre Dieu et Père,
\VS{4}sachant, mes frères bien-aimés de Dieu, votre élection.
\TextTitle{Proclamation de l'Evangile avec puissance et avec l'Esprit Saint}
\VS{5}Car notre Evangile ne vous a pas été prêché en paroles seulement, mais aussi en puissance, avec l'Esprit Saint, et avec une pleine persuasion~; car vous n'ignorez pas que nous nous sommes montrés ainsi parmi vous, à cause de vous.
\VS{6}Aussi avez-vous été nos imitateurs et ceux du Seigneur, ayant reçu avec la joie du Saint-Esprit, la parole au milieu de grandes afflictions,
\VS{7}de sorte que vous avez été des modèles à tous les fidèles de la Macédoine\FTNT{La Macédoine était le pays natal d'Alexandre le Grand. Elle fut conquise par les Romains et devint une province romaine, dont la capitale était Thessalonique.} et de l'Achaïe\FTNT{L'Achaïe était une province romaine placée sous l'autorité d'un proconsul résidant dans la capitale qui était Corinthe (2 Co. 1:1).}.
\VS{8}Car la parole du Seigneur a retenti de chez vous, non seulement dans la Macédoine et dans l'Achaïe mais aussi en tous lieux, et votre foi envers Dieu est si répandue, que nous n'avons pas besoin d'en parler.
\VS{9}Car eux-mêmes racontent de nous quel accès nous avons eu auprès de vous, et comment vous vous êtes convertis à Dieu en vous séparant des idoles, pour servir le Dieu vivant et vrai,
\VS{10}et pour attendre des cieux son Fils Jésus, qu'il a ressuscité des morts, et qui nous délivre de la colère à venir\FTNT{La colère à venir. Voir les sept coupes de la colère de Dieu (Ap. 15:5-8~; 16:1-21).}.
\Chap{2}
\TextTitle{Annoncer l'Evangile en recherchant l'approbation de Dieu et non celle des hommes}
\VerseOne{}Car, mes frères, vous savez vous-mêmes que notre entrée au milieu de vous n'a point été vaine. 
\VS{2}Après avoir souffert et reçu des outrages à Philippes\FTNT{Philippes était une ville de Macédoine située en Thrace, près de la côte nord de la mer Egée. Voir Ac. 16:12-40 et l'épître de Paul aux Philippiens.}, comme vous le savez, nous avons pris de l'assurance en notre Dieu, pour vous annoncer l'Evangile de Dieu au milieu de beaucoup de combats.
\VS{3}Car il n'y a eu dans notre prédication ni séduction, ni motif impur, ni fraude.
\VS{4}Mais comme Dieu nous a considérés dignes de nous confier la prédication de l'Evangile, ainsi nous parlons non comme pour plaire aux hommes, mais à Dieu qui éprouve nos cœurs.
\VS{5}Car, en effet, nous n'avons jamais été surpris avec des paroles flatteuses, comme vous le savez~; jamais nous n'avons eu pour prétexte la cupidité, Dieu en est témoin.
\VS{6}Et nous n'avons point cherché la gloire qui vient des hommes, ni de vous, ni des autres~; nous aurions pu nous imposer comme apôtres de Christ,
\VS{7}mais nous avons été doux au milieu de vous, de même qu'une nourrice chérit ses enfants.
\VS{8}Nous aurions voulu dans notre affection envers vous, non seulement vous donner l'Evangile de Dieu, mais encore notre propre vie, tant vous nous étiez devenus chers.
\VS{9}Car, mes frères, vous vous souvenez de notre peine et de notre travail~; vu que nous vous avons prêché l'Evangile de Dieu, en travaillant nuit et jour, pour n'être point à charge à aucun de vous.
\VS{10}Vous êtes témoins et Dieu aussi, combien notre conduite envers vous qui croyez a été sainte, juste, et irréprochable.
\VS{11}Et vous savez que nous avons exhorté chacun de vous, comme un père exhorte ses enfants,
\VS{12}en vous exhortant, vous encourageant et vous conjurant de vous conduire d'une manière digne de Dieu, qui vous appelle à son Royaume et à sa gloire.
\VS{13}C'est pourquoi nous rendons sans cesse grâces à Dieu, de ce que, quand vous avez reçu de nous la parole de la prédication de Dieu, vous l'avez reçue non comme une parole des hommes, mais ainsi qu'elle est véritablement, comme la parole de Dieu, laquelle aussi agit avec efficacité en vous qui croyez.
\VS{14}En effet, mes frères, vous êtes devenus les imitateurs des églises de Dieu qui sont en Jésus-Christ dans la Judée, parce que vous aussi, vous avez souffert de la part de ceux de votre propre nation les mêmes choses qu'elles ont souffertes de la part des Juifs,
\VS{15}qui ont même mis à mort le Seigneur Jésus, et leurs propres prophètes, qui nous ont persécutés, qui ne plaisent point à Dieu, et qui sont ennemis de tous les hommes,
\VS{16}nous empêchant de parler aux Gentils afin qu'ils soient sauvés, comblant ainsi toujours plus la mesure de leurs péchés. Mais la colère de Dieu est venue sur eux jusqu'au plus haut degré.
\VS{17}Pour nous, mes frères, après avoir été quelque temps séparés de vous de corps et non de cœur, nous avons eu d'autant plus d'ardeur et d'empressement de vous revoir.
\VS{18}Nous avons donc voulu, une et même deux fois, aller chez vous, au moins, moi Paul~; mais Satan nous en a empêchés.
\VS{19}Car quelle est notre espérance, ou notre joie, ou notre couronne de gloire~? N'est-ce pas vous qui l'êtes, devant notre Seigneur Jésus-Christ lors de son avènement~?
\VS{20}Certes, vous êtes notre gloire et notre joie.
\Chap{3}
\TextTitle{La persévérance des Thessaloniciens dans l'affliction}
\VerseOne{}C'est pourquoi ne pouvant plus soutenir la privation de vos nouvelles, nous avons trouvé bon de demeurer seuls à Athènes.
\VS{2}Et nous avons envoyé Timothée, notre frère, serviteur de Dieu, et notre compagnon d'œuvre dans l'Evangile de Christ, pour vous affermir et vous exhorter au sujet de votre foi,
\VS{3}afin que nul ne soit troublé dans ces afflictions, puisque vous savez vous-mêmes que nous sommes destinés à cela.
\VS{4}Et lorsque nous étions avec vous, nous vous annoncions d'avance que nous aurions à souffrir des afflictions, comme cela est aussi arrivé, et vous le savez.
\VS{5}C'est pourquoi, dis-je, ne pouvant plus soutenir cette inquiétude, j'ai envoyé Timothée pour connaître l'état de votre foi, de peur que le tentateur ne vous ait tentés en quelque sorte, et que nous n'ayons travaillé en vain.
\VS{6}Mais Timothée étant revenu depuis peu de chez vous, nous a apporté d'agréables nouvelles de votre foi et de votre charité, et nous a dit que vous conservez toujours un bon souvenir de nous, désirant nous voir, comme nous désirons aussi vous voir.
\VS{7}C'est pourquoi, mes frères, nous avons été consolés par votre foi, dans toutes nos afflictions et dans toutes nos détresses.
\VS{8}Car maintenant nous vivons puisque vous demeurez fermes dans le Seigneur.
\VS{9}Et quelles actions de grâces ne pouvons-nous pas rendre à Dieu à votre sujet, pour toute la joie que nous éprouvons devant notre Dieu, à cause de vous.
\VS{10}Nuit et jour, nous le prions avec une extrême ardeur de nous permettre de vous voir, et de compléter\FTNT{Compléter~: du grec «~katartizo~» qui signifie «~redresser~», «~ajuster~», «~compléter~», «~raccommoder~» (ce qui a été abîmé), «~réparer~». Ce verbe est également utilisé dans Mt. 4:21 lorsque Jacques et Jean réparaient leurs filets. Le terme «~katartismos~» traduit par «~perfectionnement~» dans Ep. 4:11 vient de ce verbe. Ainsi, l'un des rôles de ces services est le perfectionnement des saints et non leur destruction.} ce qui manque à votre foi.
\VS{11}Que Dieu lui-même, notre Père, et notre Seigneur Jésus-Christ, aplanisse\FTNT{Le verbe «~aplanir~» vient du grec «~kateuthuno~». On constate que ce verbe est conjugué au singulier, y compris dans le texte original grec, ce qui atteste l'unité entre le Père et le Fils (voir 2 Th. 2:16-17).} notre chemin pour que nous allions vers vous.
\VS{12}Et que le Seigneur vous fasse croître et abonder de plus en plus en charité les uns envers les autres, et envers tous, comme nous abondons aussi en charité envers vous~;
\VS{13}qu'il affermisse vos cœurs pour qu'ils soient irréprochables dans la sainteté, devant Dieu qui est notre Père, lors de l'avènement de notre Seigneur Jésus-Christ, accompagné de tous ses saints.
\Chap{4}
\TextTitle{Appel à la sanctification et à l'amour fraternel}
\VerseOne{}Au reste, mes frères, nous vous prions donc, et nous vous conjurons par le Seigneur Jésus, que comme vous avez appris de nous de quelle manière on doit se conduire, et plaire à Dieu, vous y fassiez tous les jours de nouveaux progrès.
\VS{2}Car vous savez quels préceptes nous vous avons donnés de la part du Seigneur Jésus.
\VS{3}Parce que c'est ici la volonté de Dieu~; savoir votre sanctification\FTNT{La sanctification personnelle (1 Pi. 1:15-18~; Hé. 12:14~; Ap. 22:11). Chaque chrétien doit fournir un effort, en se servant quotidiennement de la Parole de Dieu et de la prière, pour se maintenir dans la sanctification. Cela implique la séparation d'avec le mal et des mauvaises compagnies (2 Co. 6:14-18). Elle se développe au prix de nombreuses souffrances et de multiples sacrifices (Ro. 12:1-3).}, et que vous vous absteniez de la fornication,
\VS{4}c'est que chacun de vous sache posséder son corps dans la sanctification et dans l'honneur,
\VS{5}et sans se laisser aller aux désirs de la convoitise, comme les Gentils qui ne connaissent point Dieu.
\VS{6}Que personne n'use de fraude envers son frère et de cupidité dans les affaires, parce que le Seigneur tire vengeance de toutes ces choses, comme nous vous l'avons dit et attesté.
\VS{7}Car Dieu ne nous a pas appelés à l'impureté, mais à la sanctification.
\VS{8}C'est pourquoi celui qui rejette ceci ne rejette pas un homme, mais Dieu qui a aussi donné son Saint-Esprit.
\VS{9}Quant à la charité fraternelle\FTNT{Le mot grec employé ici est «~philadelphia~». Ce terme désigne l'amour fraternel, l'amour que les chrétiens se portent entre eux.}, vous n'avez pas besoin que je vous en écrive~; car vous-mêmes vous êtes enseignés de Dieu à vous aimer les uns les autres,
\VS{10}et c'est aussi ce que vous faites à l'égard de tous les frères qui sont dans toute la Macédoine. Mais, mes frères, nous vous prions de vous perfectionner tous les jours davantage,
\VS{11}et de tâcher de vivre paisiblement~; de faire vos propres affaires, et de travailler de vos propres mains, ainsi que nous vous l'avons ordonné.
\VS{12}En sorte que vous vous conduisiez honnêtement envers ceux du dehors, et que vous n'ayez besoin de rien.
\TextTitle{L'enlèvement de l'Eglise}
\VS{13}Or, mes frères, je ne veux pas que vous soyez dans l'ignorance au sujet de ceux qui dorment, afin que vous ne soyez point attristés comme les autres qui n'ont point d'espérance. 
\VS{14}Car si nous croyons que Jésus est mort, et qu'il est ressuscité~; de même aussi ceux qui dorment en Jésus, Dieu les ramènera avec lui.
\VS{15}Car nous vous disons ceci par la parole du Seigneur, que nous qui vivrons et resterons pour l'avènement du Seigneur, ne précéderons point ceux qui dorment.
\VS{16}Car le Seigneur lui-même, avec un cri de commandement\FTNT{L'expression «~cri de commandement~» vient du grec «~keleuma~», ce mot signifie un ordre, et en particulier un cri stimulant, comme celui que reçoit un animal pressé par un homme, tels les chevaux par les conducteurs de chariots, les chiens de chasse par les chasseurs, etc.~; ou par lequel un ordre est donné par le capitaine d'un navire, aux soldats par un chef, un appel de trompette. La sagesse de Dieu crie (Pr. 8). Esaïe devait crier à plein gosier (Es. 58:1). Le cri du Seigneur ne sera entendu que par l'Eglise véritable qui est son épouse (Mt. 25:6).}, et une voix d'archange, et avec la trompette de Dieu, descendra du ciel, et les morts en Christ ressusciteront premièrement.
\VS{17}Puis nous qui vivrons et qui resterons, serons enlevés ensemble avec eux dans les nuées, à la rencontre du Seigneur, dans les airs et ainsi nous serons toujours avec le Seigneur. 
\VS{18}C'est pourquoi consolez-vous les uns les autres par ces paroles.
\Chap{5}
\TextTitle{Veiller en attendant le jour du Seigneur~; encouragements divers\FTNTT{Joë. 1:15.}}
\VerseOne{}Pour ce qui est des temps et des moments, mes frères, vous n'avez pas besoin qu'on vous en écrive,
\VS{2}puisque vous savez vous-mêmes très bien que le jour du Seigneur viendra comme un voleur dans la nuit\FTNT{Mt. 25:6~; 2 Pi. 3:10~; Ap. 3:3~; 16:15.}.
\VS{3}Quand ils diront~: Nous sommes en paix et en sûreté. Alors une destruction soudaine les surprendra, comme les douleurs de l'enfantement surprennent la femme enceinte, et ils n'échapperont point.
\VS{4}Mais quant à vous, mes frères, vous n'êtes pas dans les ténèbres pour que ce jour-là vous surprenne comme un voleur.
\VS{5}Vous êtes tous des enfants de la lumière, et des enfants du jour. Nous ne sommes point de la nuit ni des ténèbres.
\VS{6}Ne dormons donc point comme les autres, mais veillons et soyons sobres.
\VS{7}Car ceux qui dorment, dorment la nuit, et ceux qui s'enivrent, s'enivrent la nuit.
\VS{8}Mais nous qui sommes enfants du jour, soyons sobres, ayant revêtu la cuirasse de la foi et de la charité, et ayant pour casque l'espérance du salut\FTNT{Ro. 13:12~; Ep. 6:14~; 6:17.}.
\VS{9}Car Dieu ne nous a pas destinés à la colère\FTNT{La colère à venir. Voir 1 Th. 1:9-10.}, mais à l'acquisition du salut par notre Seigneur Jésus-Christ,
\VS{10}qui est mort pour nous, afin que soit que nous veillons, soit que nous dormions, nous vivions avec lui.
\VS{11}C'est pourquoi exhortez-vous réciproquement, et édifiez-vous tous, les uns les autres, comme aussi vous le faites.
\VS{12}Nous vous prions, mes frères, d'avoir de la considération pour ceux qui travaillent parmi vous, qui dirigent dans le Seigneur, et qui vous exhortent.
\VS{13}Ayez pour eux beaucoup d'affection\FTNT{Littéralement «~agape~»~: amour, charité, affection.} à cause de l'œuvre qu'ils font. Soyez en paix entre vous.
\VS{14}Nous vous en prions aussi, mes frères, avertissez ceux qui vivent dans le désordre\FTNT{Mt. 18:15~; Ga. 6:1.}, consolez ceux qui ont l'esprit abattu, supportez les faibles, et soyez patients envers tous.
\VS{15}Prenez garde que personne ne rende à autrui le mal pour le mal\FTNT{Mt. 5:44~; Ro. 12:21.}~; mais recherchez toujours ce qui est bon, soit entre vous, soit envers tous les hommes.
\VS{16}Soyez toujours joyeux.
\VS{17}Priez sans cesse.
\VS{18}Rendez grâces pour toutes choses, car c'est la volonté de Dieu par Jésus-Christ.
\VS{19}N'éteignez point l'Esprit.
\VS{20}Ne méprisez point les prophéties.
\VS{21}Eprouvez toutes choses~; retenez ce qui est bon.
\VS{22}Abstenez-vous de toute apparence de mal.
\VS{23}Que le Dieu de paix veuille vous sanctifier entièrement, et faire que votre être entier, l'esprit, l'âme et le corps soient conservés sans reproche lors de la venue de notre Seigneur Jésus-Christ\FTNT{L'avènement du Seigneur. Voir Mt. 24:1-3.}.
\VS{24}Celui qui vous appelle est fidèle, c'est pourquoi il fera ces choses en vous.
\TextTitle{Salutations}
\VS{25}Mes frères, priez pour nous.
\VS{26}Saluez tous les frères par un saint baiser.
\VS{27}Je vous en conjure par le Seigneur que cette épître soit lue à tous les saints frères.
\VS{28}Que la grâce de notre Seigneur Jésus-Christ soit avec vous~! Amen~!
\PPE{}
\end{multicols}

\clearpage\ShortTitle{2 Th.}\BookTitle{2 Thessaloniciens}\BFont
\noindent\hrulefill
{\footnotesize
\textit{
\bigskip
{\centering{}
\\Auteur~: Paul
\\Thème~: Le jour de Christ
\\Date de rédaction~: Env. 51 ap. J.-C.\\}
}
\textit{
\\Autrefois appelée Therme ou Therma, qui signifie «~source chaude~», Thessalonique reçut son nouveau nom de Cassandre, en l'honneur de sa femme Thessalonike, qui était aussi la sœur d'Alexandre le Grand (356 av. J.-C. - 323 av. J.-C.), à qui il succéda.
\\Cette ville est située au nord de la Grèce actuelle, sur la côte de la Mer Egée. Du temps de Paul, ce pays était divisé en deux parties. Dans la région du nord, la Macédoine, se trouvaient les villes de Philippes, Thessalonique et Bérée. Quant à la région du sud, l'Achaïe, comportait les villes d'Athènes et de Corinthe. Aujourd'hui, la ville s'appelle Salonique. La seconde épître de Paul aux Thessaloniciens fut rédigée peu de temps après la première. Elle fut motivée par des troubles survenus dans la communauté à la suite d'une annonce basée sur une lettre faussement attribuée à Paul prétendant que le «~jour du Seigneur~» était arrivé. Dans cette seconde épître, l'apôtre exhorte les chrétiens de Thessalonique à tenir ferme dans leur foi malgré la persécution, leur expliquant que le «~jour de Christ~» devait être précédé par l'apostasie et la venue de l'homme impie. Il conclut sa lettre en demandant aux chrétiens de s'éloigner de ceux qui vivent dans le désordre.\bigskip
}
}
\par\nobreak\noindent\hrulefill
\begin{multicols}{2}
\Chap{1}
\TextTitle{Introduction}
\VerseOne{}Paul, Silvain, et Timothée à l'église des Thessaloniciens\FTNT{Thessalonique. Voir Ac. 17:1-9.} qui est en Dieu notre Père, et en notre Seigneur Jésus-Christ~:
\VS{2}Que la grâce et la paix vous soient données de la part de Dieu notre Père, et de la part du Seigneur Jésus-Christ~!
\TextTitle{La persévérance dans l'affliction~; Dieu, le juste Juge}
\VS{3}Mes frères, nous devons toujours rendre grâces à Dieu à cause de vous, comme il est bien raisonnable, parce que votre foi augmente beaucoup, et que votre charité mutuelle fait des progrès.
\VS{4}De sorte que nous-mêmes nous nous glorifions de vous dans les églises de Dieu, à cause de votre persévérance et de votre foi au milieu de toutes vos persécutions, et des afflictions que vous avez à supporter,
\VS{5}qui sont une manifeste démonstration du juste jugement de Dieu, afin que vous soyez jugés dignes du Royaume de Dieu, pour lequel aussi vous souffrez.
\TextTitle{La fin de ceux qui ne connaissent pas Dieu et qui n'obéissent pas à l'Evangile}
\VS{6}Car il est juste devant Dieu qu'il rende l'affliction à ceux qui vous affligent,
\VS{7}et qu'il vous donne du repos à vous qui êtes affligés, de même qu'à nous, lorsque le Seigneur Jésus se révélera\FTNT{Révélation, du grec «~apokalupsis~», signifie «~mettre à nu, révélation d'une vérité~». Le fait de rendre visible ce qui était caché.} du ciel avec les anges de sa puissance,
\VS{8}avec des flammes de feu, pour exercer la vengeance contre ceux qui ne connaissent pas Dieu, et contre ceux qui n'obéissent pas à l'Evangile de notre Seigneur Jésus-Christ.
\VS{9}Ils auront pour châtiment une ruine éternelle, loin de la face du Seigneur, et de la gloire de sa force,
\VS{10}quand il viendra pour être glorifié en ce jour-là dans ses saints, et pour être admiré dans tous ceux qui croient, parce le témoignage que avons rendu auprès de vous à été cru.
\VS{11}C'est pourquoi nous prions toujours pour vous, afin que notre Dieu vous juge dignes de la vocation, et qu'il accomplisse puissamment en vous tout le bon plaisir de sa bonté, et l'œuvre de la foi,
\VS{12}afin que le Nom de notre Seigneur Jésus-Christ soit glorifié en vous, et vous en lui, selon la grâce de notre Dieu et Seigneur Jésus-Christ.
\Chap{2}
\TextTitle{Le jour du Seigneur et l'apparition de l'homme impie}
\VerseOne{}Or pour ce qui concerne l'avènement\FTNT{L'avènement du Seigneur Jésus-Christ. Voir Mt. 24:1-3.} de notre Seigneur Jésus-Christ et notre réunion en lui, mes frères, nous vous prions
\VS{2}de ne pas vous laisser subitement ébranler dans votre entendement, ni troubler par une inspiration, ni par une parole, ou par quelque lettre qu'on dirait venir de nous, comme si le jour de Christ était déjà là.
\VS{3}Que personne donc ne vous séduise d'aucune manière~; car il faut que l'apostasie soit arrivée auparavant et que l'homme de péché, le fils de la perdition\FTNT{Il est question ici de l'homme impie, de l'antichrist, qui est la bête qui monte de la mer décrite par Jean (Ap. 13:11-18). Voir aussi Da. 11:36-38.}, soit révélé,
\VS{4}lequel s'oppose et s'élève contre tout ce qui est appelé Dieu, ou qu'on adore, jusqu'à être assis comme Dieu dans le temple de Dieu\FTNT{Selon les chapitres 40 à 42 d'Ezéchiel, le culte lévitique sera restauré à la fin des temps, ce qui suppose nécessairement la reconstruction du temple de Jérusalem. Cette prophétie est actuellement (2014-2015) en train de s'accomplir puisque des juifs religieux militent activement pour la réalisation de ce projet. L'organisation la plus connue œuvrant en ce sens est l'Institut du temple (fondé en 1987), qui a déjà restauré un grand nombre d'objets servant au culte. Toutefois, il ne faut pas sous-estimer la ruse de Satan, car au-delà du temple physique, il cherche prioritairement à s'asseoir dans les temples spirituels que sont les chrétiens (1 Co. 6:19). Pour parvenir à ses fins, Satan a envoyé plusieurs de ses émissaires pour prêcher un autre évangile et un autre christ. C'est ainsi que de nombreuses assemblées, séduites et captivées par de faux docteurs, n'ont plus Jésus-Christ comme Seigneur, mais Satan en personne. L'apostasie étant installée premièrement dans les cœurs, l'antichrist n'aura donc aucun mal à se faire passer pour le Christ et à s'asseoir dans le temple physique, où il usurpera l'adoration qui revient au Dieu véritable.} se proclamant lui-même être Dieu.
\VS{5}Ne vous souvenez-vous pas que je vous disais ces choses, lorsque j'étais encore chez vous~?
\VS{6}Et maintenant vous savez ce qui le retient, afin qu'il soit révélé en son temps.
\VS{7}Car le mystère de l'iniquité\FTNT{Le mystère de l'iniquité. Paul nous enseigne que ce mystère était déjà à l'œuvre au sein des églises primitives. Le prophète Zacharie, au chapitre 5 de son livre, l'avait personnifié en relatant une vision dans laquelle il avait vu «~deux femmes avec des ailes de cigogne~» emportant l'épha de l'iniquité des enfants d'Israël. Sur cet épha était assise une femme personnifiant l'iniquité, c'est-à-dire la femme de l'homme impie, la Babylone religieuse. Ces deux femmes aux ailes de cigogne allaient lui bâtir une maison au pays de Schinéar (Babylone selon Ge. 10:6-14).} opère déjà, seulement celui qui le retient en ce moment le fera jusqu’à ce qu’il soit hors du chemin.
\VS{8}Et alors sera révélé le méchant\FTNT{Es. 11:4.}, que le Seigneur détruira par le souffle de sa bouche et qu'il anéantira par l'éclat de son avènement.
\VS{9}L'avènement\FTNT{Il y aura un autre avènement, celui de l'homme impie.} de cet impie, se fera par la puissance de Satan, avec toutes sortes de miracles, de signes, et de prodiges mensongers,
\VS{10}et avec toutes les séductions de l'iniquité, pour ceux qui périssent parce qu'ils n'ont pas reçu l'amour de la vérité pour être sauvés.
\VS{11}C'est pourquoi Dieu leur envoie une puissance d'égarement\FTNT{L'esprit d'égarement. Voir Ro. 1:26,28~; 1 R. 22.}, pour qu'ils croient au mensonge,
\VS{12}afin que tous ceux qui n'ont pas cru à la vérité, mais qui ont pris plaisir à l'iniquité soient condamnés.
\TextTitle{Encouragements}
\VS{13}Mais nous, mes frères bien-aimés du Seigneur, nous devons toujours rendre grâces à Dieu pour vous, de ce que Dieu vous a élus dès le commencement pour le salut par la sanctification de l'Esprit, et par la foi en la vérité.
\VS{14}C'est à quoi il vous a appelés par notre Evangile, afin que vous possédiez la gloire qui nous a été acquise par notre Seigneur Jésus-Christ.
\VS{15}C'est pourquoi, mes frères, demeurez fermes, et retenez les enseignements que vous avez appris, soit par notre parole, soit par notre lettre.
\VS{16}Et que notre Seigneur Jésus-Christ lui-même, et notre Dieu et Père, qui nous a aimés, et qui nous a donné une consolation éternelle, et une bonne espérance par sa grâce,
\VS{17}console vos cœurs, et vous affermisse en toute bonne parole, et en toute bonne œuvre.
\Chap{3}
\VerseOne{}Au reste, mes frères, priez pour nous, afin que la parole du Seigneur poursuive sa course, et qu'elle soit glorifiée comme elle l'est parmi vous,
\VS{2}et que nous soyons délivrés des hommes méchants et pervers, car tous n'ont pas la foi.
\VS{3}Le Seigneur est fidèle, il vous affermira et vous gardera du mal.
\VS{4}Nous avons à votre égard cette confiance dans le Seigneur, que vous faites et que vous ferez les choses que nous recommandons.
\VS{5}Que le Seigneur veuille diriger vos cœurs vers l'amour de Dieu et vers l'attente de Christ~!
\TextTitle{Se séparer des mauvaises compagnies~; être un modèle~; subvenir à ses besoins}
\VS{6}Nous vous recommandons aussi, mes frères, au Nom de notre Seigneur Jésus-Christ, de vous éloigner\FTNT{La séparation d'avec la mauvaise compagnie. Voir 1 Co. 5:9-13, 15:33~; 2 Co. 6:14-18~; Ro. 16:17-18~; Tit. 3:10-11~; 2 Jn. 2-11.} de tout homme qui se dit frère, et qui vit d'une manière déréglée, et non selon les enseignements qu'il a reçus de nous.
\VS{7}Car vous savez vous-mêmes comment il faut nous imiter, puisque nous n'avons pas marché dans le désordre parmi vous,
\VS{8}et nous n'avons mangé gratuitement le pain de personne. Mais dans le labeur et dans la peine, nous avons travaillé nuit et jour, pour n'être à la charge\FTNT{Les véritables ouvriers de Dieu ne s'attendent pas aux hommes pour avoir leur salaire. Ils mettent leur confiance en Dieu qui est leur rémunérateur. Voir Ac. 20:33-35.} d'aucun de vous.
\VS{9}Ce n'est pas que nous n'en ayons pas le droit, mais c'est pour donner en nous-mêmes un modèle à imiter.
\VS{10}Car lorsque nous étions avec vous, nous vous déclarions expressément que si quelqu'un ne veut pas travailler, qu'il ne mange pas non plus.
\VS{11}Car nous apprenons qu'il y en a quelques-uns parmi vous qui marchent dans le désordre, qui ne travaillent pas, mais qui s'occupent de futilités.
\VS{12}C'est pourquoi nous recommandons donc à ces gens-là et nous les exhortons par notre Seigneur Jésus-Christ, à manger leur propre pain en travaillant paisiblement.
\VS{13}Mais pour vous, mes frères, ne vous lassez pas de faire le bien.
\VS{14}Et si quelqu'un n'obéit pas à ce que nous vous disons par cette épître, faites-le connaître, et n'ayez pas de relation avec lui, afin qu'il éprouve de la honte.
\VS{15}Toutefois, ne le regardez pas comme un ennemi, mais avertissez-le comme un frère.
\TextTitle{Conclusion}
\VS{16}Que le Seigneur de paix vous donne toujours la paix en tout temps~! Que le Seigneur soit avec vous tous~!
\VS{17}La salutation est de ma propre main, de moi Paul, c'est là ma signature dans toutes mes épîtres, c'est ainsi que j'écris.
\VS{18}Que la grâce de notre Seigneur Jésus-Christ soit avec vous tous~! Amen~!
\PPE{}
\end{multicols}

\clearpage\ShortTitle{1 Corinthiens}\BookTitle{1 Corinthiens}\BFont
\noindent\hrulefill
{\footnotesize
\textit{
\bigskip
{\centering{}
\\Auteur : Paul
\\Thème : Le comportement du chrétien
\\Date de rédaction : Env. 56 ap. J.-C.\\}
}
%\bigskip
\textit{
\\Dans l’antiquité, Corinthe, capitale de l’Achaïe, était la ville la plus prospère et la plus puissante de Grèce. Située sur
un isthme séparant la mer Egée de la mer Ionienne, Corinthe était au carrefour de l’Asie et de l’Italie et constituait un  véritable centre commercial où les produits orientaux et occidentaux se croisaient.
%\bigskip
\\L’apôtre Paul arriva à Corinthe en 51, sous le règne de l’empereur romain Claude (10 av. J.-C. – 54 apr. J.-C.), et y demeura 18 mois. Il trouva une ville riche en pleine expansion, une population parlant diverses langues et rendant des cultes à une multitude de divinités. Rédigée au terme des trois ans passés à Ephèse, la première épître de Paul aux Corinthiens répond à une lettre dans laquelle ceux-ci s’interrogeaient sur le mariage et sur les aliments consacrés aux idoles. Ce fut aussi l’occasion pour lui de procéder à la correction de cette jeune église dont l’état charnel constituait un frein à l’avancée spirituelle. Les Corinthiens avaient en effet confondu le culte raisonnable et les pratiques liées aux cultes à mystères.\bigskip
}
}
\par\nobreak\noindent\hrulefill
\begin{multicols}{2}
\Chap{1}
\TextTitle{La grâce de Christ manifeste dans la vie des saints\FTNTT{Ro. 5:1-2 ; Ep. 1:3-14}}
\VerseOne{}Paul, appelé à être apôtre de Jésus-Christ, par la volonté de Dieu, et le frère Sosthène,
\VS{2}à l'église de Dieu qui est à Corinthe, aux sanctifiés en Jésus-Christ, appelés à être saints, et à tous ceux qui en quelque lieu que ce soit invoquent le Nom de notre Seigneur Jésus-Christ, leur Seigneur et le nôtre.
\VS{3}Que la grâce et la paix vous soient données de la part de Dieu notre Père et du Seigneur Jésus-Christ.
\VS{4}Je rends toujours grâces à mon Dieu à votre sujet, pour la grâce de Dieu qui vous a été donnée en Jésus-Christ.
\VS{5}Car en lui vous avez été enrichis de toutes les richesses qui concernent la parole et la connaissance,
\VS{6}selon que le témoignage de Jésus-Christ a été confirmé en vous,
\VS{7}de sorte qu'il ne vous manque aucun don, pendant que vous attendez la manifestation de notre Seigneur Jésus-Christ.
\VS{8}Qui vous affermira aussi jusqu’à la fin pour que vous soyez irrépréhensibles au jour de notre Seigneur Jésus-Christ.
\VS{9}Dieu qui vous a appelés à la communion de son Fils Jésus-Christ notre Seigneur est fidèle.
\TextTitle{Les rivalités, causes de divisions}
\VS{10}Je vous prie, mes frères, par le Nom de notre Seigneur Jésus-Christ, à tenir tous un même langage, et à ne point avoir de divisions parmi vous, mais à être parfaitement unis dans une même pensée et dans un même jugement.
\VS{11}Car mes frères, j’ai été informé par ceux de la maison de Chloé qu'il y a des dissensions parmi vous.
\VS{12}Je veux dire que chacun de vous parle ainsi : Moi je suis de Paul ! Et moi d'Apollos ! Et moi de Céphas ! Et moi de Christ !
\VS{13}Christ est-il divisé ? Paul a-t-il été crucifié pour vous ? Ou avez-vous été baptisés au nom de Paul ?
\VS{14}Je rends grâces à Dieu de ce que je n'ai baptisé aucun de vous, sinon Crispus et Gaïus,
\VS{15}afin que personne ne dise que j'ai baptisé en mon nom.
\VS{16}J'ai bien aussi baptisé la famille de Stéphanas ; du reste, je ne sais pas si j'ai baptisé quelque autre.
\VS{17}Car Christ ne m'a pas envoyé pour baptiser, mais pour évangéliser, non pas avec des discours de la sagesse humaine, afin que la croix de Christ ne soit pas anéantie.
\TextTitle{La sagesse de Dieu à la croix, dépasse l'entendement humain}
\VS{18}Car la prédication de la croix est une folie pour ceux qui périssent, mais pour nous qui sommes sauvés, elle est la puissance de Dieu.
\VS{19}Car il est écrit : Je détruirai la sagesse des sages et j'anéantirai l'intelligence des hommes intelligents\FTNT{Es. 29:14.}.
\VS{20}Où est le sage ? Où est le scribe ? Où est le disputeur de ce siècle ? Dieu n'a-t-il pas convaincu de folie la sagesse de ce monde ?
\VS{21}Puisque le monde, avec sa sagesse, n’a pas connu Dieu, dans la sagesse de Dieu, il a plu à Dieu de sauver les croyants par la folie de la prédication.
\VS{22}Les Juifs demandent des miracles et les Grecs cherchent la sagesse,
\VS{23}mais pour nous, nous prêchons Christ crucifié, scandale pour les Juifs, et folie pour les Grecs,
\VS{24}à ceux qui sont appelés, tant Juifs que Grecs, nous leur prêchons Christ, la puissance de Dieu et la sagesse de Dieu.
\VS{25}Parce que la folie de Dieu est plus sage que les hommes, et la faiblesse de Dieu est plus forte que les hommes.
\TextTitle{Dieu se sert des choses viles pour confondre le monde et sa sagesse}
\VS{26}Considérez, mes frères, que parmi vous qui avez été appelés, il n’y a pas beaucoup de sages selon la chair, ni beaucoup de puissants, ni beaucoup de nobles.
\VS{27}Mais Dieu a choisi les choses folles de ce monde pour confondre les sages ; et Dieu a choisi les choses faibles de ce monde pour confondre les fortes ;
\VS{28}et Dieu a choisi les choses viles de ce monde et les méprisées, même celles qui ne sont point, pour réduire à néant celles qui sont,
\VS{29}afin que nulle chair ne se glorifie devant lui.
\VS{30}Or c'est par lui que vous êtes en Jésus-Christ, lequel, de par Dieu, a été fait pour nous sagesse, justice, sanctification et rédemption ;
\VS{31}afin que comme il est écrit, celui qui se glorifie se glorifie dans le Seigneur\FTNT{Jé. 9:24.}.
\Chap{2}
\TextTitle{La foi en Dieu ne se base pas sur la sagesse humaine}
\VerseOne{}Pour moi donc, mes frères, lorsque je suis allé chez vous, ce n’est pas avec des discours pompeux, remplis de la sagesse humaine, que je suis allé vous annoncer le témoignage de Dieu.
\VS{2}Car je n’ai pas eu la pensée de savoir parmi vous autre chose que Jésus-Christ et Jésus-Christ crucifié.
\VS{3}Et j'ai même été parmi vous dans la faiblesse, dans la crainte, et dans un grand tremblement.
\VS{4}Et ma parole et ma prédication ne reposaient pas sur les discours persuasifs de la sagesse humaine, mais sur une démonstration d'Esprit et de puissance ;
\VS{5}afin que votre foi ne soit pas fondée sur la sagesse des hommes, mais sur la puissance de Dieu.
\VS{6}Cependant, nous prêchons une sagesse parmi les parfaits, une sagesse, dis-je, qui n'est pas de ce monde, ni des chefs de ce siècle, qui vont être anéantis.
\VS{7}Mais nous prêchons la sagesse de Dieu, qui est un mystère, c'est-à-dire cachée, que Dieu avant les siècles, avait prédestinée pour notre gloire,
\VS{8}sagesse qu’aucun des chefs de ce siècle n'a connue, car s'ils l’avaient connue, ils n’auraient pas crucifié le Seigneur de gloire.
\TextTitle{C'est l'Esprit de Dieu qui revèle les profondeurs de Dieu}
\VS{9}Mais comme il est écrit : Ce sont des choses que l’œil n'a point vues, que l'oreille n'a point entendues, et qui ne sont point montées au cœur de l'homme, des choses que Dieu a préparées pour ceux qui l’aiment\FTNT{Es. 64:4.}.
\VS{10}Mais Dieu nous les a révélées par son Esprit. Car l'Esprit sonde toutes choses, même les choses profondes de Dieu.
\VS{11}Qui donc, parmi les hommes, connaît les choses de l'homme, sinon l’esprit de l'homme qui est en lui ? De même aussi, personne ne connaît les choses de Dieu, si ce n’est l'Esprit de Dieu.
\VS{12}Or nous, nous n’avons pas reçu l'esprit de ce monde, mais l'Esprit qui vient de Dieu, afin que nous connaissions les choses qui nous ont été données de Dieu.
\TextTitle{La sagesse humaine n'accepte pas les choses de l'Esprit}
\VS{13}Et nous en parlons, non avec des discours que la sagesse humaine enseigne, mais avec celle qu'enseigne le Saint-Esprit, communiquant des choses spirituelles à ceux qui sont spirituels.
\VS{14}Mais l'homme animal\FTNT{L’homme animal (ou naturel) est un homme incrédule. C’est un homme  non-régénéré, ayant le principe de la vie animale, c’est-à-dire ce que les hommes ont en commun avec les brutes. Sa nature sensuelle est sujette aux appétits et aux passions (Jud. 1:19).} ne comprend pas les choses de l'Esprit de Dieu, car elles sont une folie pour lui ; et il ne peut même pas les entendre, parce c’est spirituellement qu’on en juge.
\VS{15}Mais l'homme spirituel\FTNT{L’homme spirituel est un homme dont l’esprit est régénéré et qui marche par l’Esprit. Il a la pensée de Christ et porte les fruits de l’Esprit.} juge de tout et il n'est jugé par personne.
\VS{16}Car qui a connu la pensée du Seigneur pour pouvoir l’instruire\FTNT{Es. 40:13.} ? Mais nous, nous avons la pensée de Christ.
\Chap{3}
\TextTitle{Les œuvres de la chair nuisent à la croissance chrétienne}
\VerseOne{}Pour moi, mes frères, je n'ai pas pu vous parler comme à des hommes spirituels, mais comme à des hommes charnels\FTNT{L’homme charnel est gouverné par la nature humaine et non par l'Esprit de Dieu (Ga. 5:16-21). L’homme charnel est un enfant en Christ, littéralement «~ignorant~» (Ga. 4:1). Il est comparé à un esclave.}, c'est-à-dire comme à des enfants en Christ.
\VS{2}Je vous ai donné du lait à boire, et non pas de la viande, parce que vous ne pouviez pas la supporter ; et même maintenant vous ne le pouvez pas encore, parce que vous êtes encore charnels.
\VS{3}Car puisqu'il y a parmi vous de la jalousie, des disputes, et des divisions, n'êtes-vous pas charnels, et ne vous conduisez-vous pas à la manière des hommes ?
\VS{4}Car quand l'un dit : Moi je suis de Paul ; et l'autre : Moi je suis d'Apollos, n'êtes-vous pas charnels ?
\TextTitle{Dieu est le maître de tout}
\VS{5}Qu’est-ce donc Paul, et qui est Apollos ? Des ministres, par le moyen desquels vous avez cru, selon que le Seigneur l’a donné à chacun.
\VS{6}J'ai planté, Apollos a arrosé, mais c'est Dieu qui a donné l'accroissement,
\VS{7}en sorte que ce n’est pas celui qui plante qui est quelque chose, ni celui qui arrose, mais Dieu qui donne l'accroissement.
\VS{8}Celui qui plante et celui qui arrose sont égaux, et chacun recevra sa récompense selon son propre travail.
\VS{9}Car nous sommes ouvriers avec Dieu. Vous êtes le champ de Dieu et l'édifice de Dieu.
\VS{10}Selon la grâce de Dieu qui m'a été donnée, j'ai posé le fondement comme un sage architecte, et un autre édifie dessus. Mais que chacun prenne garde comment il édifie dessus.
\TextTitle{Le seul fondement : Jésus-Christ}
\VS{11}Car personne ne peut poser un autre fondement que celui qui a été posé, à savoir Jésus-Christ.
\TextTitle{Deux types de construction}
\VS{12}Si quelqu'un édifie sur ce fondement avec de l'or, de l'argent, des pierres précieuses, du bois, du foin, du chaume, l’œuvre de chacun sera manifestée ;
\VS{13}car le jour la fera connaître, parce qu'elle sera manifestée par le feu ; et le feu éprouvera ce qu’est l’œuvre de chacun.
\VS{14}Si l’œuvre édifiée par quelqu’un sur le fondement subsiste, il recevra la récompense.
\VS{15}Si l’œuvre de quelqu'un est consumée, il perdra sa récompense ; mais pour lui, il sera sauvé, toutefois comme au travers du feu.
\VS{16}Ne savez-vous pas que vous êtes le temple\FTNT{Le temple de Dieu. Beaucoup veulent construire des bâtiments qu’ils appellent «~temples ou maisons de Dieu~» alors que chaque chrétien est le temple de Dieu. Voir Es. 66:1 ; Ac. 17:24 ; 1 Co. 6:19.} de Dieu et que l’Esprit de Dieu habite en vous ?
\VS{17}Si quelqu'un détruit le temple de Dieu, Dieu le détruira ; car le temple de Dieu est saint, et vous êtes ce temple.
\VS{18}Que personne ne s'abuse lui-même : Si quelqu'un d'entre vous croit être sage selon ce monde, qu'il devienne fou, afin de devenir sage.
\VS{19}Parce que la sagesse de ce monde est une folie devant Dieu ; car il est écrit : Il surprend les sages dans leur ruse\FTNT{Job 5:13.}.
\VS{20}Et encore : Le Seigneur connaît les pensées des sages, il sait qu’elles sont vaines\FTNT{Ps. 94:11.}.
\VS{21}Que personne donc ne mette sa gloire dans les hommes, car toutes choses sont à vous,
\VS{22}soit Paul, soit Apollos, soit Céphas, soit le monde, soit la vie, soit la mort, soit les choses présentes, soit les choses à venir, toutes choses sont à vous,
\VS{23}et vous à Christ, et Christ à Dieu.
\Chap{4}
\TextTitle{Le Seigneur est le seul véritable juge}
\VerseOne{}Que chacun nous regarde comme des serviteurs de Christ et des dispensateurs des mystères de Dieu.
\VS{2}Du reste, il est exigé des dispensateurs que chacun soit trouvé fidèle.
\VS{3}Pour moi, il m’importe fort peu d'être jugé par vous, ou par un jugement d'homme. Je ne me juge pas non plus moi-même, car je ne me sens coupable de rien,
\VS{4}mais ce n’est pas pour cela que je suis justifié. Celui qui me juge, c'est le Seigneur.
\VS{5}C'est pourquoi ne jugez de rien avant le temps, jusqu'à ce que le Seigneur vienne, alors il mettra en lumière les choses cachées dans les ténèbres et manifestera les desseins des cœurs. Alors chacun recevra de Dieu la louange qui lui sera due.
\VS{6}Or mes frères, j’ai fait de ces choses une application à ma personne et à celle d’Apollos, à cause de vous ; afin que vous appreniez de nous à ne point aller au-delà de ce qui est écrit, et que nul de vous ne conçoive de l’orgueil en faveur de l’un contre l’autre.
\VS{7}Car qui est-ce qui met de la différence entre toi et un autre ? Qu’as-tu que tu n’aies reçu ? Et si tu l'as reçu, pourquoi te glorifies-tu comme si tu ne l'avais pas reçu\FTNT{Les diverses grâces que Dieu accorde à ses enfants doivent les amener à l’humilité.} ?
\VS{8}Vous êtes déjà rassasiés, vous êtes déjà enrichis, vous êtes devenus rois sans nous. Plaise à Dieu que vous régniez en effet, afin que nous aussi nous régnions avec vous !
\TextTitle{L'humilité et la patience}
\VS{9}Car je pense que Dieu nous a exposés publiquement, nous qui sommes les derniers des apôtres, comme des gens condamnés à la mort, puisque nous avons été en spectacle au monde, aux anges et aux hommes.
\VS{10}Nous sommes fous pour l'amour de Christ, mais vous êtes sages en Christ ; nous sommes faibles, et vous êtes forts ; vous êtes dans l'estime, et nous sommes dans le mépris.
\VS{11}Jusqu'à cette heure, nous souffrons la faim, la soif, la nudité ; on nous frappe au visage, et nous sommes errants çà et là ;
\VS{12}nous nous fatiguons à travailler de nos propres mains ; on dit du mal de nous, et nous bénissons ; nous sommes persécutés, et nous le supportons.
\VS{13}Nous sommes calomniés, et nous prions ; nous sommes devenus comme les balayures du monde, comme le rebut de tous, jusqu'à maintenant.
\VS{14}Je n'écris pas ces choses pour vous faire honte, mais je vous avertis comme mes chers enfants.
\VS{15}Car même si vous aviez dix mille maîtres en Christ, vous n'avez pourtant pas plusieurs pères, car c'est moi qui vous ai engendrés en Jésus-Christ par l'Evangile.
\VS{16}Je vous prie donc d'être mes imitateurs.
\VS{17}C'est pour cela que je vous ai envoyé Timothée, qui est mon fils bien-aimé, et qui est fidèle dans le Seigneur, afin qu'il vous rappelle quelles sont mes voies en Christ et comment j'enseigne partout dans toutes les églises.
\TextTitle{L'autorité de Paul}
\VS{18}Quelques-uns se sont enflés d’orgueil comme si je ne devais pas aller chez vous.
\VS{19}Mais j'irai bientôt chez vous, si le Seigneur le veut ; et je connaîtrai non les paroles, mais la puissance de ceux qui se sont glorifiés.
\VS{20}Car le Royaume de Dieu ne consiste pas en paroles, mais en puissance.
\VS{21}Que voulez-vous ? Que j’aille chez vous avec la verge, ou avec charité et dans un esprit de douceur ?
\Chap{5}
\TextTitle{L'inceste à Corinthe}
\VerseOne{}On entend dire de toutes parts qu'il y a parmi vous de l’impudicité, et une impudicité telle qu’elle ne se rencontre même pas chez les gentils ; c'est au point où l’un de vous a la femme de son père\FTNT{L’inceste est interdit par la loi (Lé. 18:6-8).}.
\TextTitle{Oter le mal dans l'Eglise}
\VS{2}Et vous êtes enflés d'orgueil ! Et vous n'avez pas été plutôt dans le deuil, afin que celui qui a commis cette action soit retranché du milieu de vous.
\VS{3}Pour moi, étant absent de corps, mais présent en esprit, j'ai déjà jugé comme si j'étais présent, celui qui a commis une telle action.
\VS{4}Vous et mon esprit étant assemblés au nom de notre Seigneur Jésus-Christ, j'ai ordonné, avec la puissance de notre Seigneur Jésus-Christ,
\VS{5}qu'un tel homme soit livré à Satan\FTNT{Cette déclaration de Paul peut paraître choquante pour certains, mais elle nous rappelle l'histoire de Job, qui fut mis à l'épreuve par Yahweh qui l'avait livré à Satan (Job. 1:12). Paul espérait ainsi amener cet homme à la repentance en l’excluant de l’assemblée.} pour la destruction de la chair, afin que l'esprit soit sauvé au jour du Seigneur Jésus.
\VS{6}Votre vanité est mal fondée. Ne savez-vous pas qu'un peu de levain\FTNT{Le levain fait gonfler ou enfler. Il symbolise la cause principale de nombreux péchés : l'orgueil. Dans la Bible, le levain représente aussi des péchés spirituellement destructeurs comme la malice, la méchanceté, l'hypocrisie et les faux enseignements (Lu. 12:1, Mt. 16:11-12).} fait lever toute la pâte ?
\VS{7}Otez donc le vieux levain, afin que vous soyez une nouvelle pâte, puisque vous êtes sans levain ; car Christ, notre Pâque\FTNT{Ex. 12.}, a été sacrifié pour nous.
\VS{8}C'est pourquoi célébrons donc la fête, non avec du vieux levain, non avec un levain de méchanceté et de malice, mais avec les pains sans levain de la sincérité et de la vérité.
\TextTitle{Le disciple du Seigneur ne doit pas fréquenter les faux frères}
\VS{9}Je vous ai écrit dans ma lettre de ne pas vous mêlez\FTNT{Mêler vient du grec «~sunanamignumi~» qui signifie : «~mêler ensemble, se tenir en compagnie avec, être intime avec quelqu'un. Avoir des relations, être en communication~» (Ps. 1:1 ; Ro. 16:17-18 ; 1 Co. 15:33 ; Tit. 3:10).} avec les fornicateurs,
\VS{10}non pas d’une manière absolue avec les fornicateurs de ce monde, ou avec les cupides, ou les ravisseurs, ou les idolâtres ; autrement, il vous faudrait sortir du monde.
\VS{11}Maintenant, ce que je vous ai écrit, c’est de ne pas avoir de relations avec quelqu’un qui, se nommant frère, est fornicateur, ou cupide, ou idolâtre, ou médisant, ou ivrogne, ou ravisseur, de ne même pas manger avec un tel homme.
\VS{12}Car qu'ai-je à juger ceux qui sont dehors ? N’est-ce pas ceux du dedans que vous avez à juger ?
\VS{13}Mais Dieu juge ceux qui sont du dehors. Otez donc le méchant du milieu de vous.
\Chap{6}
\TextTitle{Procès entre chrétiens ou face aux non croyants}
\VerseOne{}Quand quelqu'un d'entre vous a une affaire contre un autre, ose-t-il bien aller en jugement devant les injustes, et il ne va pas devant les saints ?
\VS{2}Ne savez-vous pas que les saints jugeront le monde\FTNT{L’Eglise jugera les nations. Les douze apôtres jugeront Israël (Mt. 19:28 ; Lu. 22:30).} ? Or si le monde doit être jugé par vous, êtes-vous indignes de rendre les moindres jugements ?
\VS{3}Ne savez-vous pas que nous jugerons les anges\FTNT{Le mot ange vient du grec «~aggelos~» et veut dire «~messager, envoyé, ange~». Ce terme s’applique donc aussi bien aux hommes qu’aux créatures spirituelles.} ? Et à plus forte raison les choses de cette vie ?
\VS{4}Si donc vous avez des procès pour les affaires de cette vie, prenez pour juge ceux qui sont des moins estimés dans l'Eglise !
\VS{5}Je le dis à votre honte. Ainsi il n’y a parmi vous pas un seul homme sage qui puisse prononcer un jugement entre frères.
\VS{6}Mais un frère a des procès contre son frère, et cela devant les infidèles.
\VS{7}C'est déjà un grand défaut chez vous que vous ayez des procès entre vous. Pourquoi ne souffrez-vous pas plutôt quelque injustice ? Pourquoi ne vous laissez-vous pas plutôt dépouiller ?
\VS{8}Mais c’est vous qui commettez l’injustice et qui dépouillez, et c’est envers des frères que vous agissez de la sorte !
\TextTitle{Le chrétien est sanctifié, lavé et justifié}
\VS{9}Ne savez-vous pas que les injustes n'hériteront point le Royaume de Dieu ? Ne vous y trompez pas : Ni les fornicateurs, ni les idolâtres, ni les adultères,
\VS{10}ni les efféminés, ni les homosexuels, ni les voleurs, ni les avares, ni les ivrognes, ni les médisants, ni les ravisseurs, n'hériteront le Royaume de Dieu.
\VS{11}Et c’est là ce que vous étiez ; mais vous avez été lavés, mais vous avez été sanctifiés, mais vous avez été justifiés au nom du Seigneur Jésus, et par l'Esprit de notre Dieu.
\VS{12}Tout m’est permis, mais tout n’est pas utile ; tout m’est permis, mais je ne me rendrai esclave d’aucune chose.
\TextTitle{Le chrétien appartient au Seigneur}
\VS{13}Les aliments sont pour le ventre, et le ventre pour les aliments ; et Dieu détruira l'un comme les autres. Or le corps n'est point pour la fornication, mais pour le Seigneur, et le Seigneur pour le corps.
\VS{14}Et Dieu qui a ressuscité le Seigneur, nous ressuscitera aussi par sa puissance.
\VS{15}Ne savez-vous pas que vos corps sont les membres de Christ ? Prendrai-je donc les membres de Christ pour en faire les membres d'une prostituée ? Loin de là !
\VS{16}Ne savez-vous pas que celui qui s'unit à la prostituée devient un même corps avec elle ? Car il est dit : Les deux deviendront une même chair\FTNT{Ge. 2:24.}.
\VS{17}Mais celui qui s’unit au Seigneur est avec lui un seul esprit.
\VS{18}Fuyez la fornication. Quelque autre péché qu’un homme commette, ce péché est hors du corps ; mais le fornicateur pèche contre son propre corps.
\TextTitle{Le chrétien est le temple Saint-Esprit}
\VS{19}Ne savez-vous pas que votre corps est le temple du Saint-Esprit qui est en vous, et que vous avez reçu de Dieu, et que vous ne vous appartenez point à vous-mêmes ?
\VS{20}Car vous avez été achetés à un prix ; glorifiez donc Dieu dans votre corps et dans votre esprit, qui appartiennent à Dieu.
\Chap{7}
\TextTitle{La sainteté dans le mariage}
\VerseOne{}Pour ce qui concerne les choses au sujet desquelles vous m'avez écrit : Je vous dis qu'il est bon à l'homme de ne pas se marier.
\VS{2}Toutefois, pour éviter la fornication, que chacun ait sa femme, et que chaque femme ait son mari.
\VS{3}Que le mari rende à sa femme la bienveillance qui lui est due ; et que la femme de même la rende à son mari.
\VS{4}Car la femme n'a pas de pouvoir sur son propre corps, mais c’est son mari. De même, le mari n'a pas de pouvoir sur son propre corps, mais c’est sa femme.
\VS{5}Ne vous privez point l'un de l'autre, si ce n'est par un consentement mutuel, pour un temps, afin que vous vaquiez au jeûne et à la prière, mais après cela retournez ensemble, de peur que Satan ne vous tente par votre manque de contrôle.
\VS{6}Or je dis ceci par conseil, et non par commandement.
\VS{7}Car je voudrais que tous les hommes soient comme moi ; mais chacun a reçu de Dieu un don particulier, l'un d’une manière, l’autre d’une autre.
\VS{8}A ceux qui ne sont pas mariés, et aux veuves, je dis qu'il leur est bon de demeurer comme moi.
\VS{9}Mais s'ils manquent de maîtrise, qu'ils se marient ; car il vaut mieux se marier que de brûler.
\TextTitle{Recommandations à ceux qui sont mariés}
\VS{10}Et quant à ceux qui sont mariés, je leur ordonne, non pas moi, mais le Seigneur, que la femme ne se sépare point de son mari.
\VS{11}Et si elle s'en sépare, qu'elle demeure sans être mariée, ou qu'elle se réconcilie avec son mari ; que le mari aussi ne quitte point sa femme.
\VS{12}Mais aux autres je leur dis, et non pas le Seigneur : Si un frère a une femme incrédule et qu'elle consente d'habiter avec lui, qu'il ne la quitte point.
\VS{13}Et si une femme a un mari incrédule et qu'il consente d'habiter avec elle, qu'elle ne le quitte point.
\VS{14}Car le mari incrédule est sanctifié par la femme, et la femme incrédule est sanctifiée par le mari ; autrement vos enfants seraient impurs, or maintenant ils sont saints.
\VS{15}Que si l'incrédule se sépare, qu'il se sépare ; le frère ou la sœur ne sont point liés dans ce cas-là, car Dieu nous a appelés à la paix.
\VS{16}Car sais-tu, femme, si tu sauveras ton mari ? Ou que sais-tu, mari, si tu sauveras ta femme ?
\TextTitle{La circoncision et l'incirconcision ne sont rien, Dieu est tout}
\VS{17}Toutefois, que chacun marche selon le don qu'il a reçu de Dieu, chacun selon l’appel qu’il a reçu du Seigneur. C’est ainsi que je l’ordonne dans toutes les églises.
\VS{18}Quelqu'un a-t-il été appelé étant circoncis ? Qu’il ne redevienne pas incirconcis\FTNT{Vient du grec «~Epispaomai~» qui a pour définition : Ne pas devenir incirconcis. Aux jours  d'Antiochus IV, dit aussi Antioche Epiphane (voir commentaire en Da.8:9), certains Juifs, voulant échapper aux persécutions, cachaient le signe de leur nationalité, la circoncision, en se faisant reproduire artificiellement le prépuce par une opération chirurgicale qui étendait la peau restante.}. Quelqu'un a-t-il été appelé incirconcis ? Qu’il ne se fasse pas circoncire.
\VS{19}La circoncision n'est rien, et l’incirconcision aussi n'est rien, mais l'observation des commandements de Dieu est tout.
\VS{20}Que chacun demeure dans la condition où il était quand il a été appelé.
\VS{21}As-tu été appelé étant esclave ? Ne t'en inquiète pas ; mais si tu peux être mis en liberté, profites-en plutôt.
\VS{22}Car l’esclave qui a été appelé par notre Seigneur est un affranchi du Seigneur ; de même, celui qui est appelé étant libre, est un esclave de Christ.
\VS{23}Vous avez été rachetés à un prix, ne devenez pas les esclaves des hommes.
\VS{24}Mes frères, que chacun demeure devant Dieu dans l'état où il était quand il a été appelé.
\TextTitle{Conseils de Paul aux célibataires}
\VS{25}Pour ce qui concerne les vierges, je n'ai point de commandement du Seigneur, mais je donne un avis comme ayant obtenu miséricorde du Seigneur pour être fidèle.
\VS{26}Voici donc ce que j'estime bon, à cause des afflictions présentes : Il est avantageux à chacun de demeurer comme il est.
\VS{27}Es-tu lié à une femme ? Ne cherche pas à rompre ce lien. N’es-tu pas lié à une femme ? Ne cherche point de femme.
\VS{28}Si tu te maries, tu ne pèches point ; et si la vierge se marie, elle ne pèche point aussi ; mais ceux qui sont mariés auront des afflictions dans la chair ; or je voudrais vous les épargner.
\VS{29}Mais je vous dis ceci, mes frères : Le temps est court, que désormais ceux qui ont une femme soient comme n’en ayant pas ;
\VS{30}ceux qui pleurent comme ne pleurant pas, ceux qui se réjouissent comme ne se réjouissant pas, ceux qui achètent comme ne possédant pas,
\VS{31}et ceux qui usent de ce monde comme n'en usant pas, car la figure de ce monde passe.
\VS{32}Or je voudrais que vous soyez sans inquiétude. Celui qui n'est pas marié s’occupe des choses du Seigneur, cherchant à plaire au Seigneur.
\VS{33}Mais celui qui est marié s’occupe des choses de ce monde, cherchant à plaire à sa femme, et ainsi il est divisé.
\VS{34}Il y a de même une différence entre la femme mariée et la vierge : Celle qui n’est pas mariée s’occupe des choses du Seigneur, afin d’être sainte de corps et d'esprit ; mais celle qui est mariée s’occupe des choses du monde pour plaire à son mari.
\VS{35}Je dis cela dans votre intérêt, ce n’est pas pour vous tendre un piège, mais pour vous porter à ce qui est bienséant et propre à vous unir au Seigneur sans aucune distraction.
\VS{36}Mais si quelqu'un croit qu’il n’est pas honorable que sa fille dépasse la fleur de l’âge sans être mariée, et qu’il faille la marier, qu'il fasse ce qu'il veut, il ne pèche point ; qu'elle soit mariée.
\VS{37}Mais celui qui a pris une ferme résolution, sans contrainte, et avec l’exercice de sa propre volonté en son cœur, de garder sa fille vierge, celui-là fait bien.
\VS{38}Celui donc qui la marie fait bien, mais celui qui ne la marie pas fait mieux.
\VS{39}La femme est liée par la loi pendant tout le temps que son mari est en vie\FTNT{Dieu est contre le divorce. Pour le Seigneur, le mariage doit être un engagement à vie (Mal. 2:16 ; Ro. 7:1-3).}, mais si son mari meurt, elle est libre de se marier à qui elle veut ; seulement, que ce soit dans le Seigneur.
\VS{40}Elle est néanmoins plus heureuse si elle demeure ainsi, selon mon avis ; or j'estime que j'ai aussi l'Esprit de Dieu.
\Chap{8}
\TextTitle{Viandes sacrifiées aux idoles et les limites de la liberté chrétienne}
\VerseOne{}Pour ce qui concerne les choses qui sont sacrifiées aux idoles\FTNT{A Corinthe, on offrait rituellement des viandes sacrifiées aux idoles. A ces occasions, certaines parties des animaux sacrifiés étaient déposées sur l’autel de l’idole, d’autres étaient données aux prêtres et aux adorateurs, qui les mangeaient lors d’un repas ou d’un festin, soit dans le temple, soit dans une maison particulière. Certains morceaux de la chair offerte aux idoles étaient ensuite apportés au marché pour être vendus (Da. 1).}, nous savons que nous avons tous de la connaissance. La connaissance enfle, mais la charité édifie.
\VS{2}Et si quelqu'un croit savoir quelque chose, il n'a encore rien connu comme il faut connaître.
\VS{3}Mais si quelqu'un aime Dieu, il est connu de lui.
\VS{4}Pour ce qui est donc de manger des choses sacrifiées aux idoles, nous savons que l'idole n'est rien dans le monde et qu'il n'y a aucun autre Dieu qu’un seul\FTNT{Paul affirme avec force que le Dieu Créateur n’est pas mélangé avec d’autres divinités. Voir Dt. 6:4.}.
\VS{5}Car s’il est des êtres qui sont appelés dieux, soit dans le ciel, soit sur la terre, comme il existe réellement plusieurs dieux, et plusieurs seigneurs,
\VS{6}nous n’avons pourtant qu'un seul Dieu, qui est le Père, de qui viennent toutes choses, et pour qui nous sommes ; et un seul Seigneur : Jésus-Christ, par qui sont toutes choses, et par qui nous sommes.
\VS{7}Mais tous n’ont pas cette connaissance. Car quelques-uns, d’après la manière dont ils envisagent encore l'idole, mangent de ces choses comme étant sacrifiées aux idoles, et leur conscience qui est faible en est souillée.
\VS{8}Ce n’est pas une viande qui nous rend agréables à Dieu ; car si nous en mangeons, nous n'avons rien de plus ; si nous n’en mangeons pas, nous n’avons rien de moins.
\VS{9}Mais prenez garde que cette liberté que vous avez ne soit en quelque sorte un scandale pour les faibles.
\VS{10}Car si quelqu'un te voit, toi qui as de la connaissance, être à table dans le temple des idoles, sa conscience, à lui qui est faible, ne le portera-t-elle pas à manger des choses sacrifiées aux idoles ?
\VS{11}Et ainsi ton frère, qui est faible, et pour lequel Christ est mort, périra par ta connaissance.
\VS{12}Or quand vous péchez ainsi contre vos frères, et que vous blessez leur conscience qui est faible, vous péchez contre Christ.
\VS{13}C'est pourquoi, si la viande scandalise mon frère, je ne mangerai jamais de chair pour ne point scandaliser mon frère.
\Chap{9}
\TextTitle{Paul défend son apostolat\FTNTT{Ga. 1:11 ; 2:21}}
\VerseOne{}Ne suis-je pas apôtre ? Ne suis-je pas libre ? N’ai-je pas vu notre Seigneur Jésus-Christ ? N’êtes-vous pas mon ouvrage dans le Seigneur ?
\VS{2}Si je ne suis pas apôtre pour les autres, je le suis au moins pour vous, car vous êtes le sceau de mon apostolat dans le Seigneur.
\VS{3}C'est là ma défense contre ceux qui me condamnent.
\VS{4}N'avons-nous pas le droit de manger et de boire ?
\VS{5}N'avons-nous pas le droit de mener avec nous une sœur qui soit notre femme, comme font les autres apôtres, et les frères du Seigneur, et Céphas ?
\VS{6}N'y a-t-il que Barnabas et moi qui n'ayons pas le droit de ne pas travailler ?
\TextTitle{Dieu prend soin de ses serviteurs}
\VS{7}Qui est-ce qui va à la guerre à ses propres frais ? Qui est-ce qui plante une vigne et n’en mange pas le fruit ? Qui est-ce qui fait paître un troupeau et ne se nourrit pas du lait du troupeau ?
\VS{8}Ces choses que je dis n’existent-elles que dans la coutume des hommes ? La loi ne dit-elle pas aussi la même chose ?
\VS{9}Car il est écrit dans la Loi de Moïse : Tu ne muselleras pas le bœuf qui foule le grain\FTNT{De. 25:4.}. Dieu se met-il en peine des bœufs ?
\VS{10}Ou parle-t-il uniquement à cause de nous ? Oui, c’est à cause de nous qu’il a été écrit que celui qui laboure doit labourer avec espérance, et celui qui foule le blé, le foule avec l’espérance d’y avoir part.
\VS{11}Si nous avons semé parmi vous des biens spirituels, est-ce une grosse affaire si nous moissonnons vos biens temporels ?
\VS{12}Si d'autres usent de ce droit à votre égard, pourquoi n'en userions-nous pas plutôt qu'eux ? Cependant nous n'avons point usé de ce droit, mais au contraire, nous supportons toutes sortes d'incommodités, afin de ne pas créer d’obstacle à l'Evangile de Christ.
\VS{13}Ne savez-vous pas que ceux qui font le service sacré mangent des choses sacrées ; et que ceux qui servent à l'autel participent à l'autel\FTNT{No. 18:8-31.} ?
\VS{14}Le Seigneur a ordonné que ceux qui annoncent l'Evangile vivent de l'Evangile.
\VS{15}Pour moi, je n’ai usé d’aucun de ces droits, et ce n’est pas afin de les réclamer en ma faveur que j’écris ainsi ; car j’aimerais mieux mourir que de me laisser enlever cette gloire.
\VS{16}Car si j'évangélise, ce n’est pas pour moi un sujet de gloire, c’est parce que la nécessité m'en est imposée ; et malheur à moi si je n'évangélise pas !
\VS{17}Si je le fais de bon cœur, j’en aurai la récompense ; mais si je le fais malgré moi, c’est une charge qui m’est confiée.
\VS{18}Quelle récompense en ai-je donc ? C’est qu'en prêchant l'Evangile, je prêche l'Evangile de Christ sans qu'il en coûte rien\FTNT{Paul annonçait l’Evangile gratuitement. Donnez gratuitement : C’est la suite logique des choses, on reçoit gratuitement et on donne gratuitement. Si nous sommes comme Christ (car là est le sens du mot disciple), nous devons agir comme lui. Il a donné ses enseignements et nourrit les gens gratuitement. Dans Ap. 21:6 et 22:17, le Seigneur invite toutes les personnes qui ont soif à venir s’abreuver gratuitement. Alors pourquoi vendre la parole c’est-à-dire l’eau qu’on a reçue gratuitement ? Nous devons donner gratuitement.}, afin que je n'abuse pas de mon autorité dans l'Evangile.
\TextTitle{L'attitude d'un vrai serviteur de Dieu}
\VS{19}Car bien que je sois libre à l'égard de tous, je me suis pourtant rendu le serviteur de tous, afin de gagner plus de personnes.
\VS{20}Avec les Juifs, j’ai été comme Juif, afin de gagner les Juifs ; avec ceux qui sont sous la loi, comme si j'étais sous la loi, afin de gagner ceux qui sont sous la loi ;
\VS{21}avec ceux qui sont sans loi, comme si j'étais sans loi (quoique je ne sois point sans la Loi de Dieu, étant sous la Loi de Christ), afin de gagner ceux qui sont sans loi.
\VS{22}J’ai été faible avec les faibles, afin de gagner les faibles ; je me suis fait tout à tous, afin d’en sauver au moins quelques-uns.
\VS{23}Je fais cela à cause de l'Evangile, afin que j'en sois fait aussi participant avec les autres.
\VS{24}Ne savez-vous pas que ceux qui courent dans le stade, courent tous, mais qu’un seul remporte le prix ? Courez de manière à le remporter.
\VS{25}Tout homme qui combat, vit entièrement de régime ; et ces gens-là le font pour obtenir une couronne corruptible\FTNT{Couronne corruptible : Aux Jeux panhelléniques, il n’y avait qu’un seul vainqueur qui remportait pour prix une couronne de feuillage. Sur chacun des sites, les couronnes étaient fabriquées avec des feuillages différents : – A Olympie, c’était une couronne d’olivier sauvage – A Delphes, une couronne de laurier – A l’Isthme (Corinthe), une couronne de pin – A Némée, une couronne de céleri. En plus de sa couronne, l’athlète victorieux recevait un ruban de laine rouge. Des amphores remplies d’huile d’olive étaient également remises au vainqueur. A cette époque, l’huile d’olive était extrêmement précieuse et valait beaucoup d’argent. D’autres prix, comme des trépieds en bronze (grands vases munis de trois pieds), des boucliers en bronze ou des coupes en argent, pouvaient aussi faire partie des lots. La modeste couronne de feuillage était cependant la plus haute récompense attribuée alors dans le monde grec, car elle garantissait l’honneur et le respect de tous à celui qui la recevait.} ; mais nous, faisons-le pour une couronne incorruptible.
\VS{26}Moi donc je cours, non pas comme à l’aventure ; je combats, mais non pas comme battant l'air.
\VS{27}Mais je traite durement mon corps et je le tiens assujetti, de peur d’être moi-même désapprouvé après avoir prêché aux autres.
\Chap{10}
\TextTitle{Paul donne l'exemple d'Israël dans le désert}
\VerseOne{}Mes frères, je ne veux pas que vous ignoriez que nos pères ont tous été sous la nuée, et qu'ils ont tous passé au travers de la mer,
\VS{2}et qu'ils ont tous été baptisés en Moïse dans la nuée et dans la mer ;
\VS{3}et qu'ils ont tous mangé la même viande spirituelle ;
\VS{4}et qu'ils ont tous bu le même breuvage spirituel : Car ils buvaient de l'eau du rocher spirituel qui les suivait, et ce rocher\FTNT{Jésus-Christ, le Rocher des âges. Voir Es. 8:13-17.} était Christ.
\VS{5}Mais la plupart d’entre eux ne furent point agréables à Dieu puisqu’ils périrent dans le désert.
\VS{6}Or ces choses ont été des exemples pour nous, afin que nous ne convoitions point des choses mauvaises, comme eux-mêmes les ont convoitées.
\VS{7}Ne devenez point idolâtres, comme quelques-uns d’entre eux, selon qu'il est écrit : Le peuple s’assit pour manger et pour boire, puis ils se levèrent pour jouer\FTNT{Ex. 32:6.}.
\VS{8}Ne nous livrons pas à la fornication, comme quelques-uns d’entre eux s’y livrèrent, de sorte qu’il en tomba vingt-trois mille en un jour\FTNT{No. 25:9.}.
\VS{9}Ne tentons\FTNT{Tenter : Du grec «~ekpeirazo~» : mettre à l’épreuve ; éprouver le caractère de Dieu et son pouvoir.} point Christ, comme le tentèrent\FTNT{Tenter : Du grec «~peirazo~» : essayer si une chose peut être faite ; éprouver malicieusement, astucieusement, pour prouver ses sentiments et ses jugements ; essayer ou éprouver la foi, la vertu, le caractère par la séduction du péché ; solliciter à pécher ; infliger des maux dans le but d’éprouver. Ce terme est aussi utilisé lorsque les hommes veulent tenter Dieu en montrant leur méfiance, par une conduite impie ou méchante, pour éprouver la justice et la patience de Dieu, et le défier, pour le pousser à donner une preuve de ses perfections.} quelques-uns d’entre eux qui périrent par les serpents\FTNT{No. 21:6-9.}.
\VS{10}Ne murmurez point, comme quelques-uns d’entre eux qui périrent par le destructeur\FTNT{No. 14:2-29 ; No. 26:63-65.}.
\TextTitle{L'Eglise doit s'instruire par l'expérience d'Israël}
\VS{11}Or toutes ces choses leur sont arrivées pour servir d’exemples, et elles ont été écrites pour notre instruction, comme étant ceux auxquels les derniers temps sont parvenus.
\VS{12}Que celui donc qui pense demeurer debout prenne garde qu'il ne tombe.
\VS{13}Aucune tentation ne vous a éprouvés, qui n’ait été une tentation humaine, et Dieu qui est fidèle ne permettra pas que vous soyez tentés au-delà de vos forces, mais avec la tentation il préparera aussi le moyen d’en sortir, afin que vous puissiez la supporter.
\VS{14}C'est pourquoi, mes bien-aimés, fuyez l'idolâtrie.
\VS{15}Je vous parle comme à des personnes intelligentes, jugez vous-mêmes de ce que je dis.
\TextTitle{Distinction entre le repas et l'idolâtrie}
\VS{16}La coupe de bénédiction, que nous bénissons, n'est-elle pas la communion du sang de Christ ? Et le pain que nous rompons, n'est-il pas la communion au corps de Christ ?
\VS{17}Parce qu'il n'y a qu'un seul pain, nous qui sommes plusieurs sommes un seul corps ; car nous sommes tous participants du même pain.
\VS{18}Voyez l'Israël selon la chair, ceux qui mangent les sacrifices ne sont-ils pas en communion avec l'autel ?
\VS{19}Que dis-je donc ? Que l'idole soit quelque chose ? Ou que ce qui est sacrifié à l'idole soit quelque chose ? Nullement.
\VS{20}Mais je dis que les choses que les Gentils sacrifient, ils les sacrifient aux démons, et non à Dieu ; or je ne veux pas que vous soyez en communion avec des démons.
\VS{21}Vous ne pouvez pas boire la coupe du Seigneur et la coupe des démons ; vous ne pouvez pas participer à la table du Seigneur et à la table des démons\FTNT{L’apôtre Paul nous parle de deux sortes de tables : la table de Jézabel (ou des démons) et la table du Seigneur. La table du Seigneur à été révélée à Moïse (Ex. 25:23-30 ; Lé. 24:5-9). Il y avait dessus 12 pains destinés à la consommation des sacrificateurs. Ces pains étaient renouvelés chaque sabbat et représentaient Christ, le Pain de Dieu, qui est l’aliment du croyant-sacrificateur (Jn. 6:33-58). La table de Jézabel nous est présentée dans 1 R. 18:19 : «~Fais maintenant rassembler tout Israël auprès de moi, à la montagne du Carmel, et aussi les quatre cent cinquante prophètes de Baal et les quatre cents prophètes d’Astarté qui mangent à la table de Jézabel~». Jézabel avait à sa table 850 faux prophètes qui partageaient son repas. Voir Ap. 17. Satan est maître en matière de déguisement et d’imitation (2 Co. 11:13-15). Il a donc imité la table du Seigneur et propose aux hommes les mets du roi et le vin de la débauche (Da. 1). Il invite ceux qui cherchent Dieu à sa table afin de les détourner de la vision du ciel. Voir Mt. 6:24 ; Lu. 16:13.}.
\VS{22}Voulons-nous provoquer la jalousie du Seigneur ? Sommes-nous plus forts que lui ?
\TextTitle{La loi de l'amour s'applique dans le manger et le boire\FTNTT{Ro. 14:1-23}}
\VS{23}Toutes choses me sont permises, mais toutes ne sont pas utiles ; toutes choses me sont permises, mais toutes n'édifient pas.
\VS{24}Que personne ne cherche son propre intérêt, mais que chacun cherche celui d’autrui.
\VS{25}Mangez de tout ce qui se vend au marché, sans vous enquérir de rien par motif de conscience\FTNT{1 Ti. 4:3-5.}.
\VS{26}Car la terre avec tout ce qu'elle contient est au Seigneur.
\VS{27}Si un incrédule vous invite et que vous vouliez aller, mangez de tout ce qui sera mis devant vous, sans vous enquérir par motif de conscience.
\VS{28}Mais si quelqu'un vous dit : Ceci a été sacrifié aux idoles, n'en mangez pas, à cause de celui qui vous a avertis, et à cause de la conscience ; car la terre avec tout ce qu'elle contient est au Seigneur.
\VS{29}Je parle ici, non de votre conscience, mais de celle de l'autre. Pourquoi ma liberté serait-elle condamnée par la conscience d'un autre ?
\VS{30}Et si par la grâce j'en suis participant, pourquoi suis-je blâmé pour une chose dont je rends grâces ?
\VS{31}Soit donc que vous mangiez, soit que vous buviez, ou que vous fassiez quelque autre chose, faites tout à la gloire de Dieu.
\VS{32}Soyez tels que vous ne donniez aucun scandale ni aux Juifs, ni aux Grecs, ni à l'Eglise de Dieu,
\VS{33}de la même manière que moi aussi, je m’efforce en toutes choses de complaire à tous, cherchant, non pas mon avantage, mais celui du plus grand nombre, afin qu’ils soient sauvés.
\Chap{11}
\VerseOne{}Soyez mes imitateurs comme je le suis moi-même de Christ.
\TextTitle{Homme et femme devant Dieu}
\VS{2}Or mes frères, je vous loue de ce que vous vous souvenez de tout ce qui me concerne, et de ce que vous retenez mes instructions comme je vous les ai données.
\VS{3}Mais je veux que vous sachiez que Christ est le chef\FTNT{Le mot «~chef~» vient du grec «~kephal~» qui signifie tête. Jésus-Christ est la seule tête et l’unique chef de l’Eglise (Ep. 1:22-23 ; Col. 1:18). Toute personne qui se proclame la tête de l’église devient naturellement antéchrist.} de tout homme, que l’homme est le chef de la femme, et que Dieu est le chef de Christ.
\VS{4}Tout homme qui prie ou qui prophétise, ayant quelque chose sur la tête, déshonore son chef.
\VS{5}Toute femme au contraire qui prie, ou qui prophétise sans avoir la tête couverte, déshonore son chef, c'est comme si elle était rasée.
\VS{6}Car si une femme n'est pas couverte, qu’on lui coupe aussi les cheveux. Or, s'il est honteux pour une femme d'avoir les cheveux coupés, ou d'être rasée, qu'elle se voile.
\VS{7}Car pour ce qui est de l'homme, il ne doit point couvrir sa tête, vu qu'il est l'image et la gloire de Dieu ; mais la femme est la gloire de l'homme.
\VS{8}Parce que l'homme n'a point été tiré de la femme, mais la femme a été tirée de l'homme.
\VS{9}Et aussi l'homme n'a pas été créé pour la femme, mais la femme pour l'homme.
\VS{10}C'est pourquoi la femme à cause des anges doit avoir sur la tête une marque de l’autorité de son mari dont elle dépend.
\VS{11}Toutefois, dans le Seigneur, l'homme n'est point sans la femme ni la femme sans l'homme.
\VS{12}Car comme la femme est par l'homme, de même l'homme est par la femme, et tout cela procède de Dieu.
\VS{13}Jugez-en vous-mêmes : Est-il convenable que la femme prie Dieu sans être couverte ?
\VS{14}La nature elle-même ne vous enseigne-t-elle pas que c’est une honte pour l'homme d’avoir de longs cheveux,
\VS{15}mais que c’est une gloire pour la femme de porter des longs cheveux, parce que la chevelure lui a été donnée pour lui servir de voile ?
\VS{16}Si quelqu'un aime à contester, nous n'avons pas une telle coutume, ni les églises de Dieu.
\TextTitle{Le repas du Seigneur et les abus dénoncés par Paul}
\VS{17}Or en ce que je vais vous dire, je ne vous loue point : C’est que vous vous assemblez, non pour devenir meilleurs, mais pour empirer.
\VS{18}Car premièrement, lorsque vous vous réunissez en assemblée, j'apprends qu'il y a des divisions parmi vous et j'en crois une partie,
\VS{19}car il faut qu'il y ait même des hérésies parmi vous, afin que ceux qui sont dignes d’être approuvés soient reconnus parmi vous.
\VS{20}Quand donc vous vous assemblez ainsi tous ensemble, ce n'est pas pour manger le repas du Seigneur ;
\VS{21}car, quand on se met à table, chacun commence par prendre son souper particulier, et l'un a faim tandis que l'autre est ivre.
\VS{22}N'avez-vous donc pas de maisons pour manger et pour boire ? Ou méprisez-vous l'Eglise de Dieu et faites-vous honte à ceux qui n'ont rien ? Que vous dirai-je ? Vous louerai-je ? Je ne vous loue point en cela.
\TextTitle{Le repas du Seigneur}
\VS{23}Car j'ai reçu du Seigneur ce qu'aussi je vous ai donné ; c’est que le Seigneur Jésus, la nuit où il fut trahi, prit du pain,
\VS{24}et après avoir rendu grâces, le rompit et dit : Prenez, mangez : Ceci est mon corps qui est rompu pour vous ; faites ceci en mémoire de moi.
\VS{25}De même aussi après le souper, il prit la coupe, en disant : Cette coupe est la nouvelle alliance en mon sang ; faites ceci toutes les fois que vous en boirez, en mémoire de moi\FTNT{Mt. 26:26-28 ; Mc. 14:22-24 ; Lu. 22:19-20.}.
\VS{26}Car toutes les fois que vous mangerez de ce pain, et que vous boirez de cette coupe, vous annoncerez la mort du Seigneur, jusqu’à ce qu'il vienne.
\VS{27}C'est pourquoi quiconque mangera de ce pain ou boira de la coupe du Seigneur indignement, sera coupable envers le corps et le sang du Seigneur.
\VS{28}Que chacun donc s'éprouve soi-même, et ainsi qu'il mange de ce pain, et qu'il boive de cette coupe.
\VS{29} Car celui qui en mange et qui en boit indignement, mange et boit sa condamnation, ne distinguant point le corps du Seigneur.
\VS{30}C’est pour cela qu’il y a parmi vous beaucoup d’infirmes et de malades, et que plusieurs dorment.
\VS{31}Car si nous nous jugions nous-mêmes, nous ne serions point jugés.
\VS{32}Mais quand nous sommes jugés, nous sommes enseignés par le Seigneur, afin que nous ne soyons point condamnés avec le monde.
\VS{33}C'est pourquoi, mes frères, quand vous vous assemblez pour manger, attendez-vous les uns les autres.
\VS{34}Et si quelqu'un a faim, qu'il mange dans sa maison, afin que vous ne vous assembliez pas pour votre condamnation.  Touchant les autres points, je les réglerai quand je serai arrivé.
\Chap{12}
\TextTitle{L'Esprit révèle Christ}
\VerseOne{}Pour ce qui concerne les dons spirituels, je ne veux point, mes frères, que vous soyez ignorants.
\VS{2}Vous savez que lorsque vous étiez des gentils, vous vous laissiez entraîner vers les idoles muettes, selon que vous étiez conduits.
\VS{3}C'est pourquoi je vous fais savoir que personne, s’il parle par l'Esprit de Dieu, ne dit : Jésus est anathème ! Et personne ne peut dire : Jésus est le Seigneur ! Si ce n’est par le Saint-Esprit.
\TextTitle{La diversité des dons de l'Esprit\FTNTT{Ep. 4:7-16}}
\VS{4} Or il y a diversité de dons, mais il n'y a qu'un même Esprit.
\VS{5}Il y a aussi diversité de ministères, mais il n'y a qu'un même Seigneur.
\VS{6}Il y a aussi diversité d'opérations, mais il n'y a qu'un même Dieu qui opère toutes choses en tous.
\VS{7}Or à chacun est donnée la manifestation de l'Esprit pour l'utilité commune.
\VS{8}Car à l'un est donnée par l'Esprit, la parole de sagesse ; et à l'autre par le même Esprit, la parole de connaissance ;
\VS{9}et à un autre, la foi par ce même Esprit ; à un autre, les dons de guérison par ce même Esprit ;
\VS{10}et à un autre, les opérations des miracles ; à un autre, la prophétie ; à un autre, le don de discerner les esprits ; à un autre, la diversité de langues ; et à un autre, le don d'interpréter les langues.
\VS{11}Un seul et même Esprit opère toutes ces choses, distribuant à chacun ses dons en particulier comme il lui plaît.
\TextTitle{Chaque membre à son utilité dans le corps de Christ}
\VS{12}Car comme le corps est un, et cependant a plusieurs membres, et comme tous les membres du corps, malgré leur nombre, ne forment qu’un seul corps, il en est de même de Christ.
\VS{13}Nous avons tous, en effet, été baptisés d'un même Esprit\FTNT{Le baptême du Saint-Esprit : Les signes du baptême du Saint-Esprit (la conversion) sont les fruits de l’Esprit et sont abordés en Ga. 5:22. A aucun endroit, les écritures stipulent que le parler en langues, qui est un don gratuit (Mt. 7:16-20), est en soi le signe du baptême du Saint-Esprit. Ainsi, il nous est dit que chaque croyant en Christ a le Saint-Esprit (1 Co. 12:13 ; Ro. 8:9 ; Ep. 1:13-14) mais que tous les croyants ne parlent pas forcément en langues (1 Co. 12:29-31).}, pour être un même corps, soit Juifs, soit Grecs, soit esclaves, soit libres, nous avons tous, dis-je, été abreuvés d'un seul Esprit.
\VS{14}Ainsi, le corps n’est pas un seul membre, mais il est formé de plusieurs membres.
\VS{15}Si le pied dit : Parce que je ne suis pas la main, je ne suis point du corps ; ne serait-il pas pourtant du corps ?
\VS{16}Et si l'oreille dit : Parce que je ne suis pas l’œil, je ne suis point du corps ; ne serait-elle pas pourtant du corps ?
\VS{17}Si tout le corps est l’œil, où serait l'ouïe ? Si tout est l'ouïe, où serait l'odorat ?
\VS{18}Mais maintenant Dieu a placé chaque membre dans le corps comme il a voulu.
\VS{19}Et si tous étaient un seul membre, où serait le corps ?
\VS{20}Maintenant donc, il y a plusieurs membres et un seul corps.
\VS{21}L’œil ne peut pas dire à la main : Je n'ai pas besoin de toi ; ni la tête dire aux pieds : Je n'ai pas besoin de vous.
\VS{22}Et qui plus est, les membres du corps qui semblent être les plus faibles sont beaucoup plus nécessaires ;
\VS{23}et ceux que nous estimons être les moins honorables au corps, nous les entourons d’un plus grand honneur. Ainsi, nos membres les moins décents reçoivent le plus d’honneur,
\VS{24}Car les parties qui sont belles en nous, n’en ont pas besoin. Mais Dieu a disposé le corps de manière à donner plus d’honneur à ce qui en manquait,
\VS{25}afin qu'il n'y ait pas de division dans le corps, mais que les membres aient un soin mutuel les uns des autres.
\VS{26}Et si l'un des membres souffre quelque chose, tous les membres souffrent avec lui ; si l'un des membres est honoré, tous les membres ensemble se réjouissent avec lui.
\VS{27}Vous êtes le corps de Christ, et vous êtes chacun l’un de ses membres.
\VS{28}Et Dieu a établi dans l'Eglise premièrement des apôtres, deuxièmement des prophètes, troisièmement des docteurs, ensuite ceux qui ont le don des miracles, puis ceux qui ont les dons de guérir, de secourir, de gouverner, de parler diverses langues.
\VS{29}Tous sont-ils apôtres ? Tous sont-ils prophètes ? Tous sont-ils docteurs ? Tous ont-ils le don des miracles ?
\VS{30}Tous ont-ils les dons de guérisons ? Tous parlent-ils diverses langues ? Tous interprètent-ils ?
\VS{31}Désirez avec ardeur des dons plus excellents, et je vais vous montrer la voie la plus excellente.
\Chap{13}
\TextTitle{L'amour est la base de tout}
\VerseOne{}Quand je parlerais toutes les langues des hommes\FTNT{Les langues des hommes. Les 120 Galiléens ont été rendus capables de s’exprimer dans diverses langues afin de pouvoir annoncer la vérité aux personnes en voyage à Jérusalem dans leurs propres langues. Voir Es. 28:11-12 ; Ac. 2:1-13.}, et même des anges\FTNT{La langue des anges ou langue inconnue est incompréhensible à notre intelligence, elle est un des moyens par lequel nous disons des mystères à Dieu. Voir Ro. 8:25-26 ; 1 Co. 14:2 et 28. Il faut une interprétation si l’on veut parler cette langue dans l’assemblée à cause des non croyants qui nous visitent (1 Co. 14:23). Voir Mc. 16:17.}, si je n'ai pas la charité\FTNT{Il est question ici de l’amour «~agape~» : l’amour divin et désintéressé, l’amour fraternel.}, je suis un airain qui résonne ou une cymbale qui retentit.
\VS{2}Et quand j'aurais le don de prophétie et que je connaîtrais tous les mystères et la science de toutes choses ; et quand j'aurais même toute la foi qu'on puisse avoir, jusqu’à transporter les montagnes, si je n'ai pas la charité, je ne suis rien.
\VS{3}Et quand je distribuerais tous mes biens pour la nourriture des pauvres, quand je livrerais mon corps pour être brûlé, si je n'ai pas la charité, cela ne me sert à rien.
\VS{4}La charité est patiente, la charité est douce, la charité n'est point envieuse, la charité n'use point d'insolence, elle ne s’enfle point d’orgueil,
\VS{5}elle ne fait rien de malhonnête, elle ne cherche point son intérêt, elle ne s’irrite point, elle n’impute pas le mal,
\VS{6}elle ne se réjouit point de l'injustice, mais elle se réjouit de la vérité.
\VS{7}Elle couvre\FTNT{Dans ce passage, le grec utilisé, «~stego~», signifie «~toit, couverture, protéger ou garder en recouvrant, préserver~» (Pr. 10:12 ; Pr. 17:9). La charité ne rappelle pas sans cesses les erreurs des uns et des autres, mais sait préserver en gardant secret les fautes est expiées. Par contre, en aucun cas elle ne permet la compromission du péché en ne dénonçant pas les oeuvres des ténèbres (Mt. 18:15-18 ; Ja. 5:19-20).} tout, elle croit tout, elle espère tout, elle supporte tout.
\VS{8}La charité ne périt jamais. Les prophéties seront abolies et les langues cesseront, la connaissance sera abolie.
\VS{9}Car nous connaissons en partie et nous prophétisons en partie.
\VS{10}Mais quand la perfection sera venue, alors ce qui est en partie sera aboli.
\VS{11}Quand j'étais enfant, je parlais comme un enfant, je jugeais comme un enfant, je pensais comme un enfant ; mais quand je suis devenu homme, j'ai aboli ce qui était de l'enfance.
\VS{12}Car aujourd’hui nous voyons au moyen d’un miroir, de manière obscure, mais alors nous verrons face à face. Aujourd’hui je connais en partie, mais alors je connaîtrai comme j'ai été connu.
\VS{13}Maintenant ces trois choses demeurent : La foi, l'espérance et la charité ; mais la plus excellente de ces trois vertus c'est la charité.
\Chap{14}
\TextTitle{Importance du don de prophétie}
\VerseOne{}Recherchez la charité. Désirez avec ardeur les dons spirituels, mais surtout celui de prophétiser.
\VS{2}Parce que celui qui parle une langue inconnue ne parle point aux hommes, mais à Dieu, car personne ne le comprend, et c’est en esprit qu’il dit des mystères.
\VS{3}Mais celui qui prophétise, édifie, exhorte et console les hommes qui l'entendent.
\VS{4}Celui qui parle une langue inconnue s'édifie lui-même, mais celui qui prophétise édifie l'Eglise.
\VS{5}Je désire que vous parliez tous diverses langues, mais encore plus que vous prophétisiez. Celui qui prophétise est plus grand que celui qui parle diverses langues, à moins que ce dernier n’interprète, afin que l'Eglise en reçoive de l'édification.
\VS{6}Maintenant donc, mes frères, si je viens à vous et que je parle des langues inconnues, que vous servira cela si je ne vous parle pas par révélation, ou par science, ou par prophétie, ou par doctrine ?
\VS{7}De même, si les choses inanimées qui rendent un son, comme une flûte ou une harpe, ne rendent pas des sons distincts, comment reconnaîtra-t-on ce qui est joué sur la flûte ou sur la harpe ?
\VS{8}Et si la trompette rend un son confus, qui se préparera à la bataille ?
\VS{9}De même vous, si vous ne prononcez dans votre langue une parole distincte, comment saura-t-on ce que vous dites ? Car vous parlerez en l'air.
\VS{10}Et il y a, selon qu'il se rencontre, tant de divers sons dans le monde, et cependant aucun de ces sons n'est muet ;
\VS{11}mais si je ne sais point ce qu'on veut signifier par la parole, je serai un barbare pour celui qui parle, et celui qui parle sera un barbare pour moi.
\VS{12}Ainsi, puisque vous désirez avec ardeur les dons spirituels, que ce soit pour l’édification de l'Eglise que vous cherchiez à en posséder abondamment.
\VS{13}C'est pourquoi que celui qui parle une langue inconnue prie pour avoir le don d’interpréter.
\VS{14}Car si je prie dans une langue inconnue mon esprit est en prière, mais l'intelligence que j'en ai, est sans fruit.
\VS{15}Que faire donc ? Je prierai par l’esprit, mais je prierai aussi d'une manière à être entendu ; je chanterai par l’esprit, mais je chanterai aussi d'une manière à être entendu.
\VS{16}Autrement, si tu rends grâces par l’esprit, comment celui qui est du simple peuple dira-t-il Amen ! à ton action de grâces\FTNT{L’expression «~actions de grâces~» vient du grec «~eucharisteo~» ce qui signifie être reconnaissant, rendre grâces, remercier. Contrairement à ce que l’on enseigne dans beaucoup d’églises, il n’est pas question ici de faire une offrande d’argent mais de se montrer reconnaissant envers le Seigneur. Voir aussi commentaires en Lé. 3 et Lé. 7.}, puisqu'il ne sait pas ce que tu dis ?
\VS{17}Il est vrai que tu rends grâces, mais l’autre n’est pas édifié.
\VS{18}Je rends grâces à mon Dieu de ce que je parle plus de langues que vous tous.
\VS{19}Mais j'aime mieux prononcer dans l'Eglise cinq paroles d'une manière à être entendu, afin d’instruire aussi les autres, que dix mille paroles dans une langue inconnue.
\VS{20}Mes frères, ne soyez point des enfants sous le rapport du jugement, mais soyez des enfants à l’égard de la malice ; et à l'égard du jugement, soyez des hommes faits.
\VS{21}Il est écrit dans la loi : je parlerai à ce peuple par des gens d'une autre langue, et par des lèvres étrangères, et ils ne m’écouteront pas même ainsi, dit le Seigneur\FTNT{Es. 28:11.}.
\VS{22}C’est pourquoi les langues sont un signe, non pour les croyants, mais pour les non-croyants ; la prophétie, au contraire, est un signe, non pour les non-croyants, mais pour les croyants.
\TextTitle{L'exercice des dons spirituels dans les églises locales}
\VS{23}Si donc, l’Eglise entière s'assemble en un corps, et que tous parlent des langues étrangères et qu'il entre des gens du commun peuple ou des non-croyants, ne diront-ils pas que vous êtes hors de sens ?
\VS{24}Mais si tous prophétisent, et qu'il entre un non-croyant ou quelqu'un du commun peuple, il est convaincu par tous et il est jugé de tous,
\VS{25}ainsi les secrets de son cœur sont manifestés, de telle sorte qu'il tombera sur sa face, il adorera Dieu et publiera que Dieu est véritablement parmi vous.
\VS{26}Que faire donc mes frères ? Lorsque vous vous assemblez, les uns ou les autres parmi vous ont-ils un cantique, une instruction, une langue étrangère, une révélation, une interprétation, que tout se fasse pour l'édification.
\VS{27}Et si quelqu'un parle une langue inconnue, que cela se fasse par deux, ou tout au plus par trois, chacun à son tour, et que quelqu’un interprète ;
\VS{28}s'il n'y a point d'interprète, que cet homme se taise dans l'Eglise, et qu'il parle à lui-même et à Dieu.
\VS{29}Et que deux ou trois prophètes parlent, et que les autres en jugent ;
\VS{30}et si quelque chose est révélé à un autre qui est assis, que le premier se taise.
\VS{31}Car vous pouvez tous prophétiser l'un après l'autre, afin que tous soient instruits et que tous soient consolés.
\VS{32}Et les esprits des prophètes sont soumis aux prophètes.
\VS{33}Car Dieu n'est point un Dieu de confusion, mais de paix, comme on le voit dans toutes les églises des saints.
\VS{34}Que les femmes qui sont parmi vous se taisent dans les églises ; car il ne leur est point permis d’y parler, mais elles doivent être soumises, comme le dit aussi la loi.
\VS{35}Et si elles veulent s’instruire sur quelque chose, qu'elles interrogent leurs maris à la maison ; car il est honteux à une femme de parler dans l'église.
\VS{36}Est-ce de chez vous que la parole de Dieu est sortie ? Ou est-elle parvenue seulement à vous ?
\VS{37}Si quelqu'un croit être prophète, ou spirituel, qu'il reconnaisse que les choses que je vous écris sont des commandements du Seigneur.
\VS{38}Et si quelqu'un l’ignore, qu'il l’ignore.
\VS{39}C'est pourquoi, mes frères, désirez avec ardeur de prophétiser, et n'empêchez point de parler diverses langues.
\VS{40}Que toutes choses se fassent avec bienséance, et avec ordre.
\Chap{15}
\TextTitle{L'Evangile basé sur la résurrection de Christ}
\VerseOne{}Or, mes frères, je vous rappelle l'Evangile que je vous ai annoncé, que vous avez reçu, et auquel vous vous tenez ferme,
\VS{2}et par lequel vous êtes sauvés, si vous le retenez tel je vous l'ai annoncé ; à moins que vous n'ayez cru en vain. 
\VS{3}Car avant toutes choses, je vous ai donné ce que j'avais aussi reçu, à savoir que Christ est mort pour nos péchés, selon les Ecritures,
\VS{4}et qu'il a été enseveli, et qu'il est ressuscité\FTNT{La Résurrection du Messie. La résurrection de Jésus est un espoir pour tous les êtres humains. Elle est un principe fondamental de la foi chrétienne. Contrairement à toutes les autres religions, la foi chrétienne est la seule qui apporte l’espérance face à la mort. Toutes les autres religions ont été fondées par des hommes, leurs prophètes ou fondateurs sont morts et aucun n’est revenu à la vie. En tant que disciples de Jésus, nous sommes réconfortés par le fait que notre Dieu s'est fait homme, afin de mourir pour nos péchés, et est ressuscité le troisième jour. L’Enfer ne pouvait pas le retenir, et il tient les clés de la mort et de l’Enfer (Ap. 1:18). Voir Jn. 11:25-26. Jésus-Christ est la Résurrection.} le troisième jour, selon les Ecritures ;
\VS{5}et qu'il a été vu de Céphas, et ensuite des douze.
\VS{6}Depuis, il a été vu de plus de cinq cents frères à la fois, dont plusieurs sont encore vivants, et quelques-uns sont morts.
\VS{7}Depuis, il est apparu à Jacques, puis à tous les apôtres.
\VS{8}Après eux tous, il a été vu aussi de moi, comme d'un avorton.
\VS{9}Car je suis le moindre des apôtres, je ne suis pas digne d'être appelé apôtre, parce que j'ai persécuté l'Eglise de Dieu.
\VS{10}Mais par la grâce de Dieu, je suis ce que je suis ; et sa grâce envers moi n'a pas été vaine, mais j'ai travaillé plus qu'eux tous, toutefois non pas moi, mais la grâce de Dieu qui est avec moi.
\VS{11}Soit donc moi, soit eux, nous prêchons ainsi et vous l'avez cru ainsi.
\TextTitle{Importance de la résurrection de Christ}
\VS{12}Or si on prêche que Christ est ressuscité des morts, comment disent quelques-uns d'entre vous qu'il n'y a point de résurrection des morts ?
\VS{13}Car s'il n'y a point de résurrection des morts, Christ aussi n'est point ressuscité.
\VS{14}Et si Christ n'est pas ressuscité, notre prédication est donc vaine, et votre foi aussi est vaine.
\VS{15}Et même nous sommes de faux témoins de la part de Dieu, car nous avons rendu témoignage à l’égard de Dieu qu'il a ressuscité Christ, tandis qu’il ne l’aurait pas ressuscité, si les morts ne ressuscitent point.
\VS{16}Car si les morts ne ressuscitent point, Christ non plus n'est point ressuscité.
\VS{17}Et si Christ n'est pas ressuscité, votre foi est vaine, et vous êtes encore dans vos péchés,
\VS{18}et par conséquent aussi ceux qui dorment en Christ sont perdus.
\VS{19}Si nous n'avons d'espérance en Christ que pour cette vie seulement, nous sommes les plus misérables de tous les hommes.
\TextTitle{Détails sur les résurrections}
\VS{20}Mais maintenant Christ est ressuscité des morts, il est les prémices de ceux qui dorment.
\VS{21}Car puisque la mort est venue par un seul homme, c’est aussi par un homme qu’est venue la résurrection des morts.
\VS{22}Car comme tous meurent en Adam, de même aussi tous seront vivifiés en Christ.
\VS{23}Mais chacun en son rang, Christ comme prémices, puis ceux qui sont à Christ seront vivifiés lors de son avènement.
\VS{24}Ensuite viendra la fin, quand il aura remis le Royaume à Dieu le Père, après avoir aboli tout empire, toute puissance, et toute force.
\VS{25}Car il faut qu'il règne jusqu'à ce qu'il ait mis tous ses ennemis sous ses pieds\FTNT{Ps. 110:1.}.
\VS{26}L'ennemi qui sera détruit le dernier c'est la mort.
\VS{27}Car Dieu a tout mis sous ses pieds. Mais lorsqu’il dit que tout lui a été soumis, il est évident que celui qui lui a soumis toutes choses est excepté.
\VS{28}Et lorsque toutes choses lui auront été soumises, alors le Fils lui-même sera soumis à celui qui lui a soumis toutes choses, afin que Dieu soit tout en tous.
\VS{29}Autrement que feraient ceux qui se font baptiser pour les morts ? Si les morts ne ressuscitent absolument pas, pourquoi se font-ils baptiser pour les morts ?
\VS{30}Et nous, pourquoi sommes-nous en danger à toute heure ?
\VS{31}Tous les jours je suis exposé à la mort, je l’atteste, par la gloire de notre Seigneur Jésus-Christ.
\VS{32}Si j'ai combattu contre les bêtes à Ephèse dans des vues humaines, quel profit m’en revient-il ? Si les morts ne ressuscitent pas, mangeons et buvons, car demain nous mourrons.
\VS{33}Ne soyez point séduits : Les mauvaises compagnies corrompent les bonnes mœurs.
\VS{34}Réveillez-vous pour vivre justement, et ne péchez point ; car quelques-uns ne connaissent pas Dieu, je le dis à votre honte.
\TextTitle{Corps de résurrection}
\VS{35}Mais quelqu'un dira : Comment les morts ressuscitent-ils, et avec quel corps viennent-ils ?
\VS{36}Insensé ! Ce que tu sèmes ne reprend point vie s'il ne meurt pas\FTNT{Jn. 12:24.}.
\VS{37}Et ce que tu sèmes, tu ne sèmes point le corps qui naîtra, c’est un simple grain, de blé peut-être, ou d’une autre semence.
\VS{38}Mais Dieu lui donne le corps comme il veut, et à chacune des semences son propre corps.
\VS{39}Toute chair n'est pas de la même chair, mais autre est la chair des hommes, autre la chair des bêtes, autre celle des poissons, autre celle des oiseaux.
\VS{40}Il y a aussi des corps célestes, et des corps terrestres ; mais autre est l’éclat des corps célestes, et autre celui des corps terrestres.
\VS{41}Autre est l’éclat du soleil, autre l’éclat de la lune, autre l’éclat des étoiles ; même une étoile diffère d'une autre étoile en éclat.
\VS{42}Il en sera aussi de même à la résurrection des morts : Le corps est semé corruptible, il ressuscitera incorruptible.
\VS{43}Il est semé en déshonneur, il ressuscite glorieux ; il est semé en faiblesse, il ressuscite plein de force.
\VS{44}Il est semé corps animal, il ressuscitera corps spirituel. S’il y a un corps animal, il y a aussi un corps spirituel.
\VS{45}Comme aussi il est écrit : Le premier homme, Adam, devint une âme vivante\FTNT{Ge. 2:7.}. Le dernier Adam est devenu un Esprit vivifiant\FTNT{Jn. 5 : 21 ; Ro. 8 : 11. Jésus-Christ est le dernier Adam. Voir Ph. 2:7 ; 1 T. 3:16.}.
\VS{46}Or ce qui est spirituel n'est pas le premier, mais ce qui est animal ; et puis vient ce qui est spirituel.
\VS{47}Le premier homme, étant de la terre, est tiré de la poussière, mais le second homme, à savoir le Seigneur, est du ciel.
\VS{48}Tel qu'est celui qui est tiré de la poussière, tels aussi sont ceux qui sont tirés de la poussière ; et tel qu'est le céleste, tels aussi sont les célestes.
\VS{49}Et comme nous avons porté l'image de celui qui est tiré de la poussière, nous porterons aussi l'image du céleste.
\VS{50}Voici donc ce que je dis, mes frères, c'est que la chair et le sang ne peuvent hériter le Royaume de Dieu, et que la corruption n'hérite pas l'incorruptibilité.
\TextTitle{Mystère de la résurrection\FTNTT{1 Th. 4:14-17}}
\VS{51}Voici, je vous dis un mystère : Nous ne mourrons pas tous, mais tous nous serons changés,
\VS{52}en un instant, en un clin d’œil, à la dernière trompette\FTNT{Le mot dernier dans ce passage est «~eschatos~»  qui signifie «~dernier en temps ou en lieu, dernier dans des séries de lieux, dernier dans une succession dans le temps~». Paul associe le mystère de la résurrection à la dernière trompette. Or, dans le livre d'Apocalypse il n'y a que sept trompettes (Ap. 8 ; 9 et 11:15-19), et c'est à la dernière, c'est-à-dire la septième que le mystère de Dieu s'accomplit.}. La trompette sonnera, et les morts ressusciteront incorruptibles, et nous, nous serons changés.
\VS{53}Car il faut que ce corps corruptible revête l'incorruptibilité, et que ce corps mortel revête l'immortalité.
\TextTitle{La mort engloutie}
\VS{54}Lorsque ce corps corruptible aura revêtu l'incorruptibilité, et que ce corps mortel aura revêtu l'immortalité, alors cette parole de l'Ecriture sera accomplie : La mort a été engloutie dans la victoire\FTNT{Es. 25:8.}.
\VS{55}Ô mort, où est ta victoire ? Ô mort, où est ton aiguillon\FTNT{Os. 13:14.} ?
\VS{56}L'aiguillon de la mort c'est le péché ; et la puissance du péché c'est la loi.
\VS{57}Mais grâces soient rendues à Dieu qui nous a donné la victoire par notre Seigneur Jésus-Christ !
\VS{58}C'est pourquoi, mes frères bien-aimés, soyez fermes, inébranlables, vous appliquant toujours avec un nouveau zèle à l’œuvre du Seigneur, sachant que votre travail ne sera pas vain dans le Seigneur.
\Chap{16}
\TextTitle{Instructions et salutations de Paul}
\VerseOne{}Pour ce qui concerne la collecte en faveur des saints, faites comme je l’ai ordonné aux églises de Galatie.
\VS{2}C’est que chaque premier jour de la semaine, chacun de vous mette à part chez lui ce qu’il pourra assembler, selon la prospérité que Dieu lui accordera, afin qu’on n’attende pas mon arrivée pour recueillir les dons.
\VS{3}Puis quand je serai arrivé, j'enverrai les personnes que vous aurez approuvées avec des lettres pour porter votre libéralité à Jérusalem.
\VS{4}Et s’il convient que j’y aille moi-même, ils viendront aussi avec moi.
\VS{5}J'irai donc chez vous quand j’aurai traversé la Macédoine, car je traverserai par la Macédoine.
\VS{6}Et peut-être que je séjournerai parmi vous, ou même que j'y passerai l'hiver, afin que vous me conduisiez partout où j’irai.
\VS{7}Car je ne veux pas cette fois vous voir en passant, mais j’espère demeurer quelque temps auprès de vous, si le Seigneur le permet.
\VS{8}Toutefois, je resterai à Ephèse jusqu'à la Pentecôte.
\VS{9}Car une grande porte et un accès efficace m'y est ouverte, et les adversaires sont nombreux.
\VS{10}Si Timothée arrive, faites en sorte qu'il soit en sûreté parmi vous, car il travaille à l’œuvre du Seigneur comme moi-même.
\VS{11}Que personne donc ne le méprise. Accompagnez-le en paix, afin qu'il vienne vers moi, car je l'attends avec les frères.
\VS{12}Quant à Apollos, notre frère, je l'ai beaucoup exhorté à se rendre chez vous avec les frères, mais ce n’était décidément pas sa volonté de le faire maintenant ; il partira quand il en aura l’occasion.
\VS{13}Veillez, soyez fermes dans la foi, agissez courageusement, fortifiez-vous.
\VS{14}Que tout ce que vous faites se fasse avec charité.
\VS{15}Or, mes frères, vous connaissez la famille de Stéphanas, et vous savez qu'elle est les prémices de l'Achaïe, et qu’elle s’est entièrement appliquée au service des saints.
\VS{16}Je vous prie de vous soumettre à de tels hommes, et à tous de ceux qui s’emploient à l’œuvre du Seigneur, et qui travaillent avec nous.
\VS{17}Je me réjouis de l’arrivée de Stéphanas, de Fortunatus, et d’Achaïcus, parce qu'ils ont suppléé à votre absence.
\VS{18}Car ils ont tranquillisé mon esprit et le vôtre. Ayez donc de la considération pour de telles personnes.
\VS{19}Les églises d'Asie vous saluent. Aquilas et Priscille, avec l'église qui est dans leur maison, vous saluent affectueusement dans le Seigneur.
\VS{20}Tous les frères vous saluent. Saluez-vous les uns les autres par un saint baiser.
\VS{21}Je vous salue, moi Paul, de ma propre main.
\VS{22}Si quelqu'un n'aime pas le Seigneur Jésus-Christ, qu'il soit anathème ! Maranatha\FTNT{Maranatha signifie littéralement «~Le Seigneur vient !~}.
\VS{23}Que la grâce de notre Seigneur Jésus-Christ soit avec vous !
\VS{24}Mon amour est avec vous tous en Jésus-Christ. Amen.
\PPE{}
\end{multicols}

\clearpage\ShortTitle{2 Corinthiens}\BookTitle{2 Corinthiens}\BFont
\noindent\hrulefill
{\footnotesize
\textit{
\bigskip
{\centering{}
\\Auteur : Paul avec Tite et Luc
\\Thème : L'autorité de Paul
\\Date de rédaction : Env. 57 ap. J.-C.\\}
}
%\bigskip
\textit{
\\Dans l’antiquité, Corinthe, capitale de l’Achaïe, était la ville la plus prospère et la plus puissante de Grèce. Située sur
un isthme séparant la mer Egée de la mer Ionienne, Corinthe était au carrefour de l’Asie et de l’Italie et constituait un
véritable centre commercial où les produits orientaux et occidentaux se croisaient.
%\bigskip
\\Rédigée quelques mois après la première, la seconde lettre de Paul aux Corinthiens fait état d’une vague de méfiance à l’égard de Paul et exprime les souffrances qui furent les siennes et qui somme toute authentifient son apostolat.\bigskip
}
}
\par\nobreak\noindent\hrulefill
\begin{multicols}{2}
\Chap{1}
\TextTitle{Introduction}
\VerseOne{}Paul, apôtre de Jésus-Christ par la volonté de Dieu, et le frère Timothée, à l'église de Dieu qui est à Corinthe, et à tous les saints qui sont dans toute l'Achaïe.
\VS{2}Que la grâce et la paix vous soient données de la part de Dieu notre Père et du Seigneur Jésus-Christ.
\TextTitle{Consolation de Paul dans ses afflictions}
\VS{3}Béni soit Dieu, le Père de notre Seigneur Jésus-Christ, le Père des miséricordes et le Dieu de toute consolation,
\VS{4}qui nous console dans toutes nos afflictions, afin que par la consolation dont nous sommes l’objet de la part de Dieu, nous puissions consoler ceux qui se trouvent dans l’affliction.
\VS{5}Car de même que les souffrances de Christ abondent en nous, de même notre consolation abonde par Christ.
\VS{6}Et si nous sommes affligés, c'est pour votre consolation et pour votre salut ; si nous sommes consolés, c'est pour votre consolation et pour votre salut qui se réalise par la patience à supporter les mêmes souffrances que nous endurons aussi.
\VS{7}Et l'espérance que nous avons de vous est ferme, sachant que comme vous êtes participants des souffrances, de même aussi vous le serez de la consolation.
\VS{8}Car mes frères, nous ne voulons pas que vous ignoriez l’affliction qui nous est survenue en Asie, que nous avons été excessivement accablés, au-delà de nos forces, de telle sorte que nous avions perdu l'espérance de conserver notre vie.
\VS{9}Et nous regardions comme certain notre arrêt de mort, afin de ne pas placer notre confiance en nous-mêmes, mais en Dieu qui ressuscite les morts.
\VS{10}C’est lui qui nous a délivrés et qui nous délivrera d'une si grande mort, et en qui nous espérons qu'il nous délivrera aussi à l'avenir.
\VS{11}Etant aussi aidés par la prière que vous faites pour nous, afin que la grâce obtenue pour nous par plusieurs soit pour plusieurs une occasion de rendre grâces à notre sujet.
\TextTitle{La sincérité de Paul dans son ministère}
\VS{12}Car ce qui fait notre gloire c’est le témoignage de notre conscience, que nous nous sommes conduits dans le monde, et surtout à votre égard, avec simplicité et sincérité de Dieu, non point avec une sagesse charnelle, mais avec la grâce de Dieu.
\VS{13}Nous ne vous écrivons pas autre chose que ce que vous lisez, et vous-mêmes le reconnaissez. Et j'espère que vous les reconnaîtrez aussi jusqu'à la fin,
\VS{14}de même que vous avez reconnu en partie que nous sommes votre gloire, comme vous serez aussi la nôtre au jour du Seigneur Jésus.
\TextTitle{Sa manière d'agir}
\VS{15}C’est dans une telle confiance que je voulais premièrement aller vers vous, afin que vous ayez une seconde grâce ;
\VS{16}et passer de chez vous en Macédoine, puis de Macédoine revenir vers vous, et être accompagné par vous en Judée.
\VS{17}Or quand je me proposais cela, ai-je usé de légèreté ? ou les choses  que je propose, sont-elles proposées selon la chair, de sorte qu'il y ait eu en moi le oui et le non ?
\VS{18}Mais Dieu est fidèle, la parole que nous vous avons adressée n’a pas été oui et non.
\VS{19}Car le Fils de Dieu, Jésus-Christ, qui a été prêché par nous au milieu de vous, par moi, par Silvain, et par Timothée, n'a pas été oui et non, mais il a été oui en lui.
\VS{20}Car autant qu’il y a de promesses de Dieu, elles sont oui en lui, et amen en lui, afin que Dieu soit glorifié par nous.
\VS{21}Or celui qui nous affermit avec vous en Christ, et qui nous a oints, c'est Dieu,
\VS{22}lequel nous a aussi marqués d’un sceau, et a mis dans nos cœurs les arrhes\FTNT{Du grec «~arrhabon~»~: arrhes~; monnaie donnée en gage d’un futur paiement, en attendant que le solde soit payé.} de l'Esprit.
\VS{23}Or j'appelle Dieu à témoin sur mon âme, que c’est pour vous épargner que je ne suis plus allé à Corinthe.
\VS{24}Non que nous dominions sur votre foi, mais nous contribuons à votre joie, puisque vous demeurez fermes dans la foi.
\Chap{2}
\TextTitle{Les fruits de la repentance}
\VerseOne{}Je résolus en moi-même de ne pas retourner chez vous avec tristesse.
\VS{2}Car si je vous attriste, qui peut me réjouir, sinon celui que j'aurai moi-même affligé ?
\VS{3}Je vous ai écrit ceci, pour ne pas éprouver à mon arrivée de la tristesse de la part de ceux de qui je devais recevoir de la joie, ayant en vous tous cette confiance que ma joie est la vôtre à tous.
\VS{4}Je vous ai écrit dans une grande affliction et angoisse de cœur, avec beaucoup de larmes, non pas afin que vous soyez attristés, mais afin que vous connaissiez la charité\FTNT{Littéralement «~agape~»~: amour fraternel.} toute particulière que j'ai pour vous.
\VS{5}Si quelqu'un a été la cause de cette tristesse, ce n'est pas moi seul qu'il a attristé, afin que je ne le surcharge point, mais en quelque sorte c'est vous tous.
\VS{6}C'est assez pour cet homme, de la correction qui lui a été faite par plusieurs,
\VS{7}en sorte que vous devez bien plutôt lui pardonner et le consoler, de peur qu’il ne soit accablé par une trop grande tristesse.
\VS{8}C'est pourquoi je vous prie de confirmer envers lui votre charité.
\VS{9}C’est aussi pour cela que je vous ai écrit, afin de vous éprouver, et de connaître si vous êtes obéissants en toutes choses.
\VS{10}Or à celui à qui vous pardonnez quelque chose, je pardonne aussi ; si j’ai pardonné à celui à qui j'ai pardonné, c’est à cause de vous, en présence de Christ,
\VS{11}afin que Satan n'ait pas le dessus sur nous, car nous n'ignorons pas ses machinations.
\VS{12}Au reste, lorsque je fus arrivé à Troas pour l'Evangile de Christ, quoique la porte m'y fût ouverte par le Seigneur, je n’eus point de repos en mon esprit, parce que je ne trouvai pas Tite, mon frère ;
\VS{13}mais ayant pris congé d'eux, je partis pour la Macédoine.
\VS{14}Grâces soient rendues à Dieu, qui nous fait toujours triompher en Christ, et qui manifeste par nous l'odeur de sa connaissance en tout lieu.
\VS{15}Nous sommes, en effet, pour Dieu le parfum de Christ, parmi ceux qui sont sauvés, et parmi ceux qui périssent :
\VS{16}Aux uns, une odeur mortelle, pour la mort ; aux autres, une odeur vivifiante, pour la vie. Mais qui est suffisant pour ces choses ?
\VS{17}Car nous ne falsifions pas la parole de Dieu, comme font plusieurs, mais nous parlons de Christ avec sincérité, comme de la part de Dieu, et devant Dieu.
\Chap{3}
\TextTitle{Les corinthiens : lettre de Christ écrite avec l'Esprit du Dieu vivant}
\VerseOne{}Commençons-nous de nouveau à nous recommander nous-mêmes ? Ou avons-nous besoin, comme quelques-uns, de lettres de recommandation auprès de vous, ou de lettres de recommandation de votre part ?
\VS{2}Vous êtes vous-mêmes notre lettre, écrite dans nos cœurs, connue et lue de tous les hommes.
\VS{3}Car il est manifeste que vous êtes la lettre de Christ, écrite par notre ministère, non avec de l'encre, mais avec l'Esprit du Dieu vivant, non sur des tables de pierre, mais sur les tables de chair, qui sont vos cœurs.
\VS{4}Or nous avons une telle confiance en Dieu par Christ.
\VS{5}Non que nous soyons capables de nous-mêmes de penser quelque chose, comme de nous-mêmes, mais notre capacité vient de Dieu.
\TextTitle{Paul ministre de la nouvelle alliance}
\VS{6}Lequel nous a rendus capables d'être ministres de la nouvelle alliance\FTNT{Dans la plupart des versions, le mot grec «~diatheke~» a été traduit par «~testament~» alors que ce mot signifie aussi «~alliance~». On le retrouve notamment dans les passages suivants~: Mt. 26:28~; Mc. 14:24~; Lu. 1:72~; 22:20~; Ac. 3:25~; 7:8~; Ro. 9:4~; 11:27~; 1 Co. 11:25~; Ga. 3:15,17~; 4:24~; Ep. 2:12~; Hé. 7:22~; 8:6,8-9~; 9:4,15-17,20~; 10:16,29~; 12:24~; 13:20~; Ap. 11:19. Le fait d’avoir regroupé les écrits de Genèse à Malachie sous l’appellatif «~Ancien Testament~» a induit beaucoup de chrétiens en erreur. L’Ancienne Alliance correspond uniquement à la loi cérémonielle de Moïse qui a été accomplie par Christ à la croix (Jn. 19:30). Ainsi, avant la mort du Seigneur, on ne peut pas parler de testament puisqu’il faut qu’il y ait au préalable la mort du testateur. Or il est évident que les animaux sacrifiés sous la loi ne nous ont rien légué (Hé. 9:1-16).}, non de la lettre, mais de l'Esprit ; car la lettre tue, mais l'Esprit vivifie.
\VS{7}Or si le ministère de la mort, écrit sur des lettres, et gravé avec des pierres, a été glorieux au point que les enfants d'Israël ne pouvaient regarder fixement le visage de Moïse, à cause de la gloire de son visage, bien que cette gloire devait disparaître,
\VS{8}comment le ministère de l'Esprit ne sera-t-il pas plus glorieux ?
\VS{9}Car si le ministère de la condamnation a été glorieux, le ministère de la justice le surpasse de beaucoup en gloire.
\VS{10}Et même ce premier ministère qui a été si glorieux, ne l'a pas été en comparaison du second qui le surpasse de beaucoup en gloire.
\VS{11}Car si ce qui devait disparaître a été glorieux, ce qui est permanent est beaucoup plus glorieux.
\VS{12}Ayant donc une telle espérance, nous usons d'une grande liberté,
\VS{13}et pas comme Moïse qui mettait un voile sur son visage, afin que les enfants d'Israël ne fixassent pas les yeux sur la fin de ce qui devait disparaître.
\VS{14}Mais ils sont devenus durs d’entendement. Car jusqu'à aujourd'hui, ce même voile qui n’est ôté que par Christ, demeure quand ils font la lecture de l'ancienne alliance.
\VS{15}Jusqu'à ce jour, quand on lit Moïse, un voile est jeté sur leur cœur.
\VS{16}Mais lorsque les cœurs se convertissent au Seigneur, le voile est ôté.
\VS{17}Or le Seigneur c'est l'Esprit ; et là où est l'Esprit du Seigneur, là est la liberté.
\VS{18}Nous tous qui contemplons comme dans un miroir la gloire du Seigneur à visage découvert, nous sommes transformés en la même image, de gloire en gloire, comme par l'Esprit du Seigneur.
\Chap{4}
\TextTitle{La vérité pratique de son ministère}
\VerseOne{}C'est pourquoi, ayant ce ministère selon la miséricorde que nous avons reçue, nous ne nous relâchons point.
\VS{2}Mais nous avons entièrement rejeté les choses honteuses que l'on cache, ne marchant point avec ruse, et ne falsifiant point la parole de Dieu, mais nous rendant approuvés à toute conscience des hommes devant Dieu, par la manifestation de la vérité.
\VS{3}Que si notre Evangile est encore voilé, il ne l'est que pour ceux qui périssent ;
\VS{4}pour les incrédules dont le dieu de ce siècle a aveuglé l’esprit, afin qu’ils ne soient pas éclairés par la lumière de l'Evangile de la gloire de Christ, lequel est l'image de Dieu.
\VS{5}Car nous ne nous prêchons pas nous-mêmes mais nous prêchons Jésus-Christ le Seigneur, et nous déclarons que nous sommes vos serviteurs pour l'amour de Jésus.
\VS{6}Car Dieu qui a dit que la lumière resplendisse des ténèbres\FTNT{Ge. 1:3.}, est celui qui a resplendi dans nos coeurs, pour manifester la connaissance de la gloire de Dieu en la présence de Jésus-Christ.
\VS{7}Mais nous avons ce trésor dans des vases de terre, afin que l'excellence de cette puissance soit de Dieu, et non pas de nous.
\TextTitle{Les souffrances de Paul}
\VS{8}Etant affligés à tous égards, mais non réduits entièrement à l’extrémité ; étant en perplexité, mais non sans secours ;
\VS{9}étant persécutés, mais non abandonnés ; étant abattus, mais non perdus ;
\VS{10}portant toujours partout dans notre corps la mort du Seigneur Jésus, afin que la vie de Jésus soit aussi manifestée dans notre corps.
\VS{11}Car nous qui vivons, nous sommes sans cesse livrés à la mort pour l'amour de Jésus, afin que la vie de Jésus soit aussi manifestée dans notre chair mortelle.
\VS{12}De sorte que la mort agit en nous, et la vie agit en vous.
\VS{13}Or ayant un même esprit de foi, selon qu'il est écrit : J'ai cru, c'est pourquoi j'ai parlé\FTNT{Ps. 116:10.} ! Nous croyons aussi , et c'est aussi pourquoi nous parlons,
\VS{14}sachant que celui qui a ressuscité le Seigneur Jésus nous ressuscitera aussi par Jésus, et nous fera comparaître en sa présence avec vous.
\VS{15}Car toutes ces choses sont pour vous, afin que cette grâce surabonde, à la gloire de Dieu, les actions de grâces d’un très grand nombre.
\VS{16}C'est pourquoi nous ne nous relâchons pas. Mais quoique notre homme extérieur se détruit, toutefois l'intérieur est renouvelé de jour en jour.
\VS{17}Car nos légères afflictions du moment, produisent pour nous, au-delà de toute mesure, un poids éternel d'une gloire souverainement excellente,
\VS{18}quand nous ne regardons point aux choses visibles, mais aux invisibles ; car les choses visibles ne sont que pour un temps, mais les invisibles sont éternelles.
\Chap{5}
\TextTitle{Ses ambitions}
\VerseOne{} Car nous savons que si notre habitation terrestre, qui n'est qu'une tente, est détruite, nous avons un édifice de Dieu qui n’a pas été fait de main d’homme, une maison éternelle dans les cieux.
\VS{2}Car c'est aussi pour cela que nous gémissons, désirant avec ardeur d'être revêtus de notre domicile qui est du ciel,
\VS{3}si toutefois nous sommes trouvés vêtus, et non pas nus.
\VS{4}Car nous qui sommes dans cette tente, nous gémissons, accablés, parce que nous désirons, non pas d'être dépouillés, mais d'être revêtus, afin que ce qui est mortel soit englouti par la vie.
\VS{5}Et celui qui nous a formés pour cela c'est Dieu qui nous a donné les arrhes de l'Esprit.
\VS{6}Nous avons donc toujours confiance ; et nous savons que logeant dans ce corps, nous demeurons loin du Seigneur,
\VS{7}car nous marchons par la foi, et non par la vue.
\VS{8}Nous avons, dis-je, de la confiance, et nous aimons mieux être absents de ce corps, et être avec le Seigneur.
\VS{9}C'est pourquoi aussi nous nous efforçons de lui être agréables, et présents, et absents.
\VS{10}Car il nous faut tous comparaître devant le tribunal\FTNT{Le Tribunal de Christ n’a pas vocation à déterminer le salut des enfants de Dieu. Les chrétiens y seront jugés en fonction des œuvres produites sur la terre. En effet, chacun devra rendre compte de ce qu’il aura fait et de la gestion des dons et ministères reçus. Voir Ro. 14:10~; 1 Co. 4:4~; 2 Co. 3:10-14~; 2 Tim. 4:8.} de Christ, afin que chacun reçoive en son corps selon ce qu'il aura fait, soit bien, soit mal.
\TextTitle{Ses motifs d'action}
\VS{11}Connaissant donc combien le Seigneur doit être craint, nous persuadons les hommes; et nous sommes connus de Dieu, et j’espère que dans vos consciences vous nous connaissez aussi.
\VS{12}Car nous ne nous recommandons pas de nouveau à vous, mais nous vous donnons l'occasion de vous glorifier à notre sujet, afin que vous ayez de quoi répondre à ceux qui se glorifient de l'apparence, et non pas de ce qui est dans le cœur.
\VS{13}Car soit que nous soyons hors de sens, c’est pour Dieu ; soit que nous soyons de bon sens c’est pour vous.
\VS{14}Car la charité de Christ nous lie, parce que nous jugeons que, si un est mort pour tous, tous aussi sont morts ;
\VS{15}et qu'il est mort pour tous, afin que ceux qui vivent, ne vivent plus pour eux-mêmes, mais pour celui qui est mort et ressuscité pour eux.
\VS{16}C'est pourquoi dès maintenant nous ne connaissons personne selon la chair ; et si nous avons connu Christ selon la chair, maintenant nous ne le connaissons plus ainsi.
\VS{17}Si donc quelqu'un est en Christ, il est une nouvelle créature ; les choses anciennes sont passées ; voici, toutes choses sont faites nouvelles.
\VS{18}Et tout cela vient de Dieu, qui nous a réconciliés avec lui par Jésus-Christ, et qui nous a donné le ministère de la réconciliation.
\VS{19}Car Dieu était en Christ, réconciliant le monde avec lui-même, en ne leur imputant point leurs péchés, et il a mis en nous la parole de la réconciliation.
\VS{20}Nous sommes donc ambassadeurs pour Christ, c'est comme si Dieu vous exhortait par notre ministère ; nous vous supplions donc pour l’amour de Christ : Réconciliez-vous avec Dieu !
\VS{21}Car il a fait celui qui n'a point connu de péché, être péché pour nous, afin que nous soyons en lui justice de Dieu.
\Chap{6}
\TextTitle{Son humilité}
\VerseOne{}Puisque nous travaillons avec le Seigneur, nous vous prions de ne pas recevoir la grâce de Dieu en vain.
\VS{2}Car il dit : Je t'ai exaucé au temps favorable et t'ai secouru au jour du salut\FTNT{Es. 49:8.} ; voici maintenant le temps favorable, voici maintenant le jour du salut.
\VS{3}Ne donnant aucun scandale en quoi que ce soit, afin que notre ministère ne soit point blâmé.
\VS{4}Mais nous rendant recommandables, en toutes choses, comme ministres de Dieu, en grande patience, en afflictions, en nécessités, en détresses,
\VS{5}sous les coups, dans les prisons, dans les troubles, dans les travaux, dans les veilles, dans les jeûnes,
\VS{6}par la pureté, par la connaissance, par la persévérance, par la douceur, par le Saint-Esprit, par une charité sincère,
\VS{7}par la parole de vérité, par la puissance de Dieu, par les armes de justice que l'on porte à la main droite et à la main gauche ;
\VS{8}au milieu de la gloire et de l'ignominie, au milieu de la mauvaise et de la bonne réputation ; étant regardés comme des séducteurs, quoique véridiques,
\VS{9}comme inconnus, quoique bien connus, comme mourants, et voici nous vivons, comme châtiés, quoique non mis à mort,
\VS{10}comme attristés, et nous sommes toujours joyeux, comme pauvres, et nous en enrichissons plusieurs, comme n'ayant rien, et nous possédons toutes choses.
\TextTitle{Appel à la séparation et à la purification}
\VS{11}Ô Corinthiens ! Notre bouche s’est ouverte pour vous, notre cœur s'est élargi.
\VS{12}Vous n’y êtes point à l'étroit, mais c’est votre cœur qui s’est rétréci pour nous.
\VS{13}Rendez-nous la pareille (je vous parle comme à mes enfants) élargissez aussi votre cœur !
\VS{14}Ne portez pas un même joug avec les infidèles ; car quelle communion y a-t-il entre justice et l'iniquité ? Ou qu’y a-t-il de commun entre la lumière et les ténèbres ?
\VS{15}Et quel accord y a-t-il entre Christ et Bélial\FTNT{Bélial~: De l’hébreu «~beliya’al~»~: méchants, pervers, pervertis, vil, destruction, dangereusement. L’un des noms de Satan qui signifie «~indignité, méchanceté, impiété~».} ? Ou quelle part a le fidèle avec l'infidèle ?
\VS{16}Et quel rapport y a-t-il entre le temple de Dieu et les idoles ? Car vous êtes le temple du Dieu vivant, selon ce que Dieu a dit : J'habiterai au milieu d'eux et j'y marcherai ; je serai leur Dieu, et ils seront mon peuple\FTNT{Lé. 26:12~; Ez. 37:26.}.
\VS{17}C'est pourquoi sortez du milieu d'eux, et séparez-vous, dit le Seigneur ; ne touchez pas à ce qui est impur, et je vous accueillerai\FTNT{Es. 52:11~; Ap. 18:4.}.
\VS{18}Je serai pour vous un Père, et vous serez pour moi des fils et des filles, dit le Seigneur Tout-Puissant\FTNT{Jn. 1:12~; Ap. 21:7.}.
\Chap{7}
\VerseOne{}Or donc mes bien-aimés, puisque nous avons de telles promesses, nettoyons-nous de toute souillure de la chair et de l'esprit, perfectionnant la sanctification dans la crainte de Dieu.
\TextTitle{Paul ouvre son coeur aux Corinthiens}
\VS{2}Recevez-nous, nous n'avons fait tort à personne, nous n'avons corrompu personne, nous n'avons pillé personne.
\VS{3}Je ne dis pas ceci pour vous condamner, car je vous ai déjà dit que vous êtes dans nos cœurs à la vie et à la mort.
\VS{4}J'ai une grande liberté envers vous, j'ai grand sujet de me glorifier de vous ; je suis rempli de consolation, je suis comblé de joie au milieu de toutes nos afflictions.
\VS{5}Car depuis notre arrivée en Macédoine, notre chair n’eut aucun repos, mais nous avons été affligés de toute manière, ayant eu des combats au dehors, et des craintes au dedans.
\VS{6}Mais Dieu qui console les abattus nous a consolés par l’arrivée de Tite,
\VS{7}et non seulement par son arrivée, mais aussi par la consolation qu'il a reçue de vous ; car il nous a raconté votre grand désir, vos larmes, votre affection ardente pour moi, en sorte que je m'en suis extrêmement réjoui.
\VS{8}Quoique je vous aie attristés par ma lettre, je ne m'en repens pas. Et si je m'en suis repenti, car je vois que cette lettre vous a affligés, bien que momentanément,
\VS{9}je me réjouis à présent, non de ce que vous avez été affligés, mais de ce que votre tristesse vous a portés à la repentance ; car vous avez été attristés selon Dieu, de sorte que vous n'avez reçu aucun dommage de notre part.
\VS{10}En effet, la tristesse selon Dieu produit une repentance à salut dont on ne se repent jamais, mais que la tristesse du monde produit la mort.
\VS{11}En effet, cette tristesse qui est selon Dieu, quel empressement n’a-t-elle pas produit en vous ! quelle justification, quelle indignation, quelle crainte, quel grand désir, quel zèle, quelle vengeance! Vous  vous êtes montrés de toutes manières purs dans cette affaire.
\VS{12}Quoi que je vous aie écrit, ce n'était ni à cause de celui qui a commis la faute, ni à cause de celui envers qui elle a été commise, mais pour faire voir parmi vous l'empressement que j'ai de vous devant Dieu.
\VS{13}C'est pourquoi nous avons été consolés de ce que vous avez fait pour notre consolation. Mais nous nous sommes encore plus réjouis de la joie qu'a eu Tite, en ce que son esprit a été tranquillisé par vous tous.
\VS{14}Et si en quelque chose je me suis glorifié de vous devant lui, je n’en ai point eu de confusion ; mais, comme nous avons toujours parlé selon la vérité, ce dont nous nous sommes glorifiés auprès de Tite s’est trouvé être aussi la vérité.
\VS{15}C'est pourquoi quand il se souvient de l'obéissance de vous tous, et comment vous l'avez reçu avec crainte et tremblement, son affection pour vous en est beaucoup plus grande.
\VS{16}Je me réjouis d'avoir confiance en vous en toutes choses.
\Chap{8}
\TextTitle{Exemple des Macédoniens concernant la collecte en faveur des pauvres de Jérusalem}
\VerseOne{}Au reste, mes frères, nous voulons vous faire connaître la grâce que Dieu a faite aux églises de la Macédoine.
\VS{2}Au travers de leur grande épreuve d'affliction, leur joie a été augmentée, et leur profonde pauvreté s'est répandue en richesses par leur prompte libéralité.
\VS{3}Car je suis témoin qu'ils ont donné volontairement selon leurs moyens, et même au-delà de leurs moyens,
\VS{4}Nous pressant avec de grandes prières de recevoir la grâce et de prendre part à cette contribution en faveur des Saints.
\VS{5}Et ils n'ont pas fait seulement comme nous l'espérions, mais ils se sont donnés premièrement eux-mêmes au Seigneur, et puis à nous, par la volonté de Dieu.
\VS{6}Nous avons exhorté Tite, comme il avait auparavant commencé, d'achever aussi cette grâce envers vous.
\TextTitle{Exemple du Messie}
\VS{7}C'est pourquoi, comme vous excellez en toutes choses, en foi, en parole, en connaissance, en toute diligence, et dans la charité que vous avez pour nous, faites en sorte d’exceller aussi dans cette œuvre de charité.
\VS{8}Je ne dis pas cela pour vous donner un ordre, mais pour éprouver par l’empressement des autres la sincérité de votre charité.
\VS{9}Car vous connaissez la grâce de notre Seigneur Jésus-Christ qui, étant riche, s'est fait pauvre pour vous, afin que par sa pauvreté vous soyez enrichis.
\VS{10}C’est un avis que je donne là-dessus, parce qu'il vous est convenable, à vous qui non seulement avez commencé à agir, mais en ayant même eu la volonté dès l'année passée.
\VS{11}Achevez donc maintenant d'agir, afin que comme vous avez été prompts à en avoir la volonté ; vous l'accomplissiez aussi selon vos moyens.
\VS{12}Car si la promptitude de la volonté existe,  on est agréable selon ce qu'on a, et non point selon ce qu'on n'a pas.
\VS{13}Je ne veux pas vous exposer à la détresse pour soulager les autres, mais suivre une règle d’égalité. Dans la circonstance présente, votre superflu pourvoira à leurs besoins,
\VS{14}afin que leur superflu pourvoie pareillement aux vôtres, en sorte qu’il y ait égalité,
\VS{15}selon ce qui est écrit : Celui qui avait beaucoup n'a rien eu de superflu, et celui qui avait peu n'en a pas eu moins.\FTNT{Ex. 16:18.}.
\TextTitle{Exemple des églises}
\VS{16}Grâces soient rendues à Dieu qui a mis dans le cœur de Tite le même empressement pour vous ;
\VS{17}lequel a bien reçu mon exhortation, et c'est avec un nouveau zèle et de son plein gré qu'il part pour aller chez vous.
\VS{18}Et nous avons aussi envoyé avec lui le frère dont la louange dans l'Evangile est répandue par toutes les églises ;
\VS{19}de plus, il a été choisi par élection des églises pour être notre compagnon de voyage et pour cette grâce\FTNT{ également traduit par «~les aumônes~»} qui est administrée par nous à la gloire du Seigneur même, et afin de répondre à l’ardeur de votre zèle. %et pour servir à la promptitude de votre zèle.
\VS{20}Evitant ainsi que personne ne nous blâme dans cette abondante collecte, qui est administrée par nous;
\VS{21}ayant soin de faire ce qui est bon, non seulement devant le Seigneur, mais aussi devant les hommes.
\VS{22}Nous avons envoyé aussi avec eux notre autre frère, dont nous avons souvent éprouvé le zèle à plusieurs occasions, qui est maintenant encore plus zélé, à cause de la grande confiance qu'il a en vous.
\VS{23}Ainsi donc, quant à Tite, il est mon associé et mon compagnon d’œuvre auprès de vous ; et quant à nos frères, ils sont les envoyés des églises, la gloire de Christ.
\VS{24}Montrez donc envers eux et devant les églises une preuve de votre charité, et du sujet que nous avons de nous glorifier de vous.
\Chap{9}
\TextTitle{Encouragement par rapport aux dons}
\VerseOne{}Il est superflu que je vous écrive touchant la collecte destinée aux saints.
\VS{2}Car je vois la promptitude de votre zèle, dont je me glorifie de vous devant ceux de Macédoine, leur disant que l’Achaïe est prête dès l'année dernière ; et votre zèle en a excité plusieurs.
\VS{3}J’ai envoyé ces frères, afin que ce en quoi je me suis glorifié de vous, ne soit pas vain en cette occasion, et que vous soyez prêts, comme j'ai dit.
\VS{4}De peur que ceux de Macédoine venant avec moi, et ne vous trouvant pas prêts, nous, pour ne pas dire vous, n'ayons de honte de l'assurance dont nous nous sommes glorifiés.
\VS{5}C'est pourquoi j'ai estimé nécessaire de prier les frères à se rendre premièrement vers vous, et d'achever de préparer votre bienfait déjà promis, afin qu'il soit prêt, comme un bienfait, et non comme de l'avarice.
\VS{6}Au reste, je vous avertis que celui qui sème peu moissonnera peu, et celui qui sème abondamment moissonnera abondamment.
\VS{7}Mais que chacun contribue selon qu'il se l'est proposé en son cœur, non à regret, ou par contrainte; car Dieu aime celui qui donne avec joie.
\VS{8}Et Dieu est Tout-Puissant pour vous combler de toutes sortes de grâces, afin qu'ayant toujours tout ce qui suffit en toute chose, vous soyez abondants en toute bonne œuvre,
\VS{9}selon ce qui est écrit : Il a fait des largesses, il a donné aux pauvres ; sa justice demeure éternellement\FTNT{Ps. 112:9.}.
\VS{10}Que celui qui fournit de la semence au semeur veuille aussi vous donner du pain à manger, et multiplier votre semence, et augmenter les revenus de votre justice ;
\VS{11}afin que vous soyez pleinement enrichis pour exercer une parfaite libéralité, laquelle fait que nous en rendons grâces à Dieu.
\VS{12}Car le service de cette assistance est non seulement suffisant pour subvenir aux nécessités des Saints, mais il abonde aussi de telle sorte, que plusieurs ont de quoi en rendre grâces à Dieu.
\VS{13}Glorifiant Dieu pour l’épreuve qu’ils font de cette assistance, en ce que vous vous soumettez à l’Évangile de Christ ; et de votre prompte et libérale communication envers eux, et envers tous.
\VS{14}ils prient Dieu pour vous, et ils vous aiment\FTNT{Aimer~: Du grec «~epipotheo~»~: désirer, chérir.} très affectueusement à cause de la grâce excellente que Dieu vous a accordée.
\VS{15}Grâces soient rendues à Dieu pour son don inexprimable.
\Chap{10}
\TextTitle{Paul défend son autorité apostolique}
\VerseOne{}Au reste, je vous prie, moi Paul, par la douceur et la bonté de Christ - moi qui parais méprisable lorsque je suis en votre présence, et plein de hardiesse quand je suis éloigné,
\VS{2}je vous prie, dis-je, que lorsque je serai présent, il ne faille point que j'use de hardiesse, laquelle je me propose d’user contre quelques-uns qui nous regardent comme marchant selon la chair.
\VS{3}Mais en marchant dans la chair, nous ne combattons pas selon la chair.
\VS{4}Car les armes de notre guerre ne sont pas charnelles, mais elles sont puissantes par la vertu de Dieu, pour la destruction des forteresses.
\VS{5}Détruisant les raisonnements et toute hauteur qui s'élèvent contre la connaissance de Dieu, et amenant toute pensée captive à l'obéissance de Christ.
\VS{6}Et étant prêts à tirer vengeance de toute désobéissance, lorsque votre obéissance sera complète.
\VS{7}Considérez-vous les choses selon l'apparence ? Si quelqu'un se persuade qu’il est de Christ, qu'il se dise bien en lui-même que, comme il est de Christ, nous aussi nous sommes de Christ.
\VS{8}Car si même je veux me glorifier davantage de l’autorité que le Seigneur nous a donnée pour votre édification et non votre destruction, je ne saurais en avoir honte,
\VS{9}afin que je ne paraisse pas vouloir vous effrayer par mes lettres.
\VS{10}Car mes lettres, disent-ils, sont graves et fortes, mais la présence de corps est faible, et la parole est méprisable.
\VS{11}Que celui qui est tel, considère que tels nous sommes en paroles dans nos lettres, étant absents, tels aussi nous sommes dans nos actes, étant présents.
\VS{12}Car nous n'osons pas nous joindre ni nous comparer à quelques-uns de ceux qui se recommandent eux-mêmes. Mais en se mesurant à leur propre mesure et en se comparant à eux-mêmes, ils manquent d’intelligence.
\VS{13}Mais pour nous, nous ne voulons pas nous glorifier outre mesure, mais seulement dans la limite du champ d’action que Dieu nous assigné en nous amenant jusqu’à vous.
\VS{14}Car nous ne nous étendons pas nous même au delà des limites prescrites, comme si nous n'étions pas parvenus jusqu'à vous ; vu que nous sommes parvenus même jusqu’à vous par la prédication de l'Evangile de Christ.
\VS{15}Nous ne nous glorifions pas des travaux d’autrui qui sont hors de nos limites. Mais nous avons l’espérance, si votre foi augmente, de devenir encore plus grands parmi vous, selon les limites qui nous sont assignées,
\VS{16}jusqu'à évangéliser dans les lieux qui sont au-delà de chez vous, sans nous glorifier de ce qui a déjà été fait dans le domaine des autres.
\VS{17}Que celui qui se glorifie se glorifie dans le Seigneur.
\VS{18}Car ce n'est pas celui qui se recommande lui-même qui est approuvé, c'est celui que le Seigneur recommande\FTNT{Le Seigneur recommande ses serviteurs, il témoigne d’eux auprès des autres (Ac. 10:1-48). Un véritable serviteur de Dieu laisse au Seigneur le soin de témoigner de lui auprès des autres alors que les faux ouvriers se recommandent eux-mêmes (2 Co. 3:1).}.
\Chap{11}
\VerseOne{}Oh ! Si vous pouviez supportez de ma part un peu de folie ! Mais vous me supportez !
\VS{2}Car je suis jaloux de vous d'une jalousie de Dieu, parce que je vous ai fiancés à un seul Epoux, pour vous présenter à Christ comme une vierge pure.
\TextTitle{Les faux docteurs}
\VS{3}Mais je crains que comme le serpent séduisit Eve\FTNT{Ge. 3:1-6.} par sa ruse, vos pensées aussi ne se corrompent en se détournant de la simplicité à l’égard de Christ.
\VS{4}Car si quelqu'un vient vous prêcher un autre Jésus que nous n'avons pas prêché, ou si vous recevez un autre esprit que celui que vous avez reçu, ou un autre évangile que celui que vous avez embrassé, vous le supportez fort bien.
\VS{5}Or j'estime que je n'ai été en rien moindre que les plus excellents apôtres.
\VS{6}Si je suis un ignorant sous le rapport du langage, je ne le suis pourtant point sous celui de la connaissance, et nous l’avons montré parmi vous à tous égards et en toutes choses.
\VS{7}Ai-je commis une faute, en m’abaissant moi-même afin que vous soyez élevés, quand je vous ai annoncé gratuitement l’Evangile de Dieu ?
\VS{8}J'ai dépouillé les autres églises prenant de quoi m'entretenir pour vous . Et lorsque j’étais chez vous et que je me suis trouvé dans le besoin, je n’ai été à la charge de personne,
\VS{9}car les frères venus de la Macédoine ont pourvu à ce qui me manquait. Et en toutes choses, je me suis gardé d’être à votre charge, et je m'en garderai encore.
\VS{10}Par la vérité de Christ qui est en moi, j’atteste que ce sujet de gloire ne me sera point ravi dans les contrées de l'Achaïe.
\VS{11}Pourquoi ? Est-ce parce que je ne vous aime point ? Dieu le sait !
\VS{12}Mais ce que je fais, je le ferai encore, pour ôter ce prétexte à ceux qui cherchent un prétexte, afin qu’ils soient trouvés tels que nous dans les choses dont ils se glorifient.
\VS{13}Car ces hommes-là sont de faux apôtres, des ouvriers trompeurs qui se déguisent en apôtres de Christ.
\VS{14}Et cela n'est pas étonnant puisque Satan lui-même se déguise en ange de lumière\FTNT{Satan est maître en matière de déguisement et d’imitation.}.
\VS{15}Ce n'est donc pas un grand sujet d'étonnement si ses ministres aussi se déguisent en ministres de justice ; mais leur fin sera conforme à leurs œuvres.
\TextTitle{Sujets de gloire de Paul\FTNTT{2 Co. 11:16-12:18}}
\VS{16}Je le dis encore, afin que personne ne me regarde comme un insensé ; sinon, supportez-moi comme un insensé, afin que je me glorifie aussi un peu.
\VS{17}Ce que je dis, avec l’assurance d’avoir sujet de me glorifier, je ne le dis pas selon le Seigneur, mais comme par folie.
\VS{18}Puisqu’il en est plusieurs qui se glorifient selon la chair, je me glorifierai aussi.
\VS{19}Car vous supportez bien volontiers les insensés, vous qui êtes sages.
\VS{20}Si quelqu'un vous asservit, si quelqu'un vous dévore, si quelqu'un prend votre bien, si quelqu'un est arrogant, si quelqu'un vous frappe au visage, vous le supportez.
\VS{21}Je le dis avec honte, nous avons montré de la faiblesse. Mais si en quelque chose quelqu'un ose se glorifier, je parle en insensé, j'ai la même hardiesse !
\VS{22}Sont-ils Hébreux ? Moi aussi. Sont-ils Israélites ? Moi aussi. Sont-ils de la postérité d'Abraham ? Moi aussi.
\VS{23}Sont-ils ministres de Christ ? - je parle comme un insensé - je le suis plus qu'eux ; par les travaux, bien plus ; par les blessures, bien plus ; par les emprisonnements, bien plus. Plusieurs fois en danger de mort,
\VS{24}cinq fois j’ai reçu des Juifs quarante coups moins un,
\VS{25}j'ai été battu de verges trois fois, j'ai été lapidé une fois, j'ai fait naufrage trois fois, j'ai passé un jour et une nuit dans l’abîme.
\VS{26}Fréquemment en voyage, j’ai été en péril sur les fleuves, en péril de la part des brigands, en péril de la part de ceux de ma nation, en péril de la part des gentils, en péril dans les villes, en péril dans les déserts, en péril sur la mer, en péril parmi de faux frères.
\VS{27}J’ai été dans le travail et dans la peine, exposé à de nombreuses veilles, à la faim et à la soif, à des jeûnes multipliés, au froid et à la nudité.
\VS{28}Outre les choses de dehors, ce qui me tient assiégé tous les jours, c'est le soucis que j'ai de toutes les églises.
\VS{29}Qui est affaibli, que je ne sois faible ? Qui est scandalisé, que je n'en sois aussi brûlé ?
\VS{30}S'il faut se glorifier, je me glorifierai des choses qui sont de mon infirmité.
\VS{31}Le Dieu et Père de notre Seigneur Jésus-Christ, lui qui est béni éternellement, sait que je ne mens point.
\VS{32}A Damas, le gouverneur du roi Arétas avait fait garder la ville des Damascéniens pour me prendre,
\VS{33}mais on me descendit par une fenêtre, dans une corbeille, le long de la muraille, et ainsi j'échappai de ses mains.
\Chap{12}
\VerseOne{}Certes, il ne me convient pas de me glorifier, car j’en viendrai jusqu’aux visions et aux révélations du Seigneur.
\VS{2}Je connais un homme en Christ, qui fut ravi jusqu’au troisième ciel, il y a quatorze ans passés, (si ce fut dans son corps, je ne sais pas ; si ce fut hors du corps, je ne sais pas ; Dieu le sait).
\VS{3}Et je sais que cet homme (si ce fut dans son corps, ou si ce fut hors du corps, je ne sais pas ; Dieu le sait),
\VS{4}fut ravi dans le paradis, et qu’il entendit des paroles inexprimables qu'il n'est pas permis à l'homme de révéler.
\TextTitle{Paul et son écharde}
\VS{5}Je me glorifierai d'un tel homme, mais je ne me glorifierai point de moi-même, sinon de mes infirmités.
\VS{6}Si je voulais me glorifier, je ne serais pas un insensé, car je dirais la vérité ; mais je m'en abstiens, afin que personne ne m'estime au-dessus de ce qu'il me voit être, ou de ce qu'il entend dire de moi.
\VS{7}Mais pour que je ne sois pas enflé d’orgueil, à cause de l'excellence de ces révélations, il m'a été mis une écharde\FTNT{La nature exacte de l'écharde de Paul ne nous est pas détaillée. Elle lui avait été infligée par un «~ange de Satan~», par la volonté de Dieu. Nous constatons une chose qui est commune à tous les enfants de Dieu~: Paul avait un adversaire constamment aux aguets pour essayer de le décourager, le détruire ou l’intimider en s'opposant par tous les moyens à la mission que le Seigneur lui avait confiée. Cette écharde était aussi un moyen utilisé par Dieu pour garder Paul dans l’humilité.} dans la chair, un ange de Satan pour me souffleter et m’empêcher de m’enorgueillir.
\VS{8}Trois fois j'ai prié le Seigneur de faire que cet ange de Satan se retire de moi.
\VS{9}Mais le Seigneur m'a dit : Ma grâce te suffit, car ma puissance s’accomplit dans la faiblesse. Je me glorifierai donc bien volontiers de mes faiblesses, afin que la puissance de Christ habite en moi.
\VS{10}C’est pourquoi je me plais dans les faiblesses, dans les outrages, dans les calamités, dans les persécutions, et dans les angoisses pour Christ ; car quand je suis faible, c'est alors que je suis fort.
\TextTitle{Avertissements}
\VS{11}J'ai été insensé en me glorifiant, mais vous m'y avez contraint ; c’est par vous que je devais être recommandé, car je n'ai été inférieur en rien aux apôtres par excellence, quoique je ne sois rien.
\VS{12}Certainement les preuves de mon apostolat ont éclaté au milieu de vous par une patience à toute épreuve, par des signes, des prodiges et des miracles.
\VS{13}Car en quoi avez-vous été inférieurs aux autres églises, sinon en ce que je n’ai point été à votre charge ? Pardonnez-moi ce tort.
\VS{14}Voici pour la troisième fois que je suis prêt à aller vers vous, et je ne serai point à votre charge ; car ce ne sont pas vos biens que je cherche, c’est vous-mêmes. Ce n’est pas, en effet, aux enfants d’amasser pour les parents, mais aux parents pour les enfants\FTNT{Un bon père amasse dans le but de préparer l’avenir de ses enfants et non l’inverse.}.
\VS{15}Pour moi, je dépenserai très volontiers pour vous tout ce que j’ai, et je me donnerai encore moi-même pour vos âmes. En vous aimant davantage, serais-je moins aimé de vous ?
\VS{16}Soit ! Dira-t-on, que je ne vous ai point été à charge, c'est qu'étant un homme intelligent, je vous ai pris par ruse !
\VS{17}Ai-je donc tiré profit de vous par quelqu’un de ceux que je vous ai envoyés ?
\VS{18}J'ai engagé Tite à aller chez vous, et avec lui j’ai envoyé le frère. Tite a-t-il tiré profit de vous ? Et n'avons-nous pas lui et moi marché dans le même esprit ? N'avons-nous pas marché sur les mêmes traces ?
\VS{19}Pensez-vous encore que nous voulions nous justifier auprès de vous ? Nous parlons devant Dieu en Christ, et tout cela, mes très chers frères, pour votre édification.
\VS{20}Car je crains de ne pas vous trouver, à mon arrivée, tels que je voudrais, et d’être moi-même trouvé par vous tel que vous ne voudriez pas. Je crains de trouver des querelles, de la jalousie, des animosités, des rivalités, des médisances, des calomnies, de l’orgueil, des troubles.
\VS{21}Je crains qu’à mon arrivée, mon Dieu ne m’humilie de nouveau à votre sujet, et que je n’aie à pleurer sur plusieurs de ceux qui ont péché précédemment, et qui ne se sont pas repentis de l’impureté, de la débauche et des dérèglements dont ils se sont rendus coupables.
\Chap{13}
\TextTitle{S'examiner}
\VerseOne{}Je vais chez vous pour la troisième fois. Toute affaire se réglera sur la déclaration de deux ou de trois témoins\FTNT{De. 19:15.}.
\VS{2}Lorsque j’étais présent pour la deuxième fois, j’ai déjà dit, et aujourd’hui que je suis absent je dis encore d’avance à ceux qui ont péché précédemment et à tous les autres, que si je retourne chez vous, je n'épargnerai personne,
\VS{3}puisque vous cherchez la preuve que Christ parle par moi, lui qui n'est point faible envers vous, mais qui est puissant parmi vous.
\VS{4}Car il a été crucifié à cause de sa faiblesse, mais il vit par la puissance de Dieu ; et nous de même, nous sommes aussi faibles comme lui, mais nous vivrons avec lui par la puissance que Dieu a déployée envers vous.
\VS{5}Examinez-vous vous-mêmes pour savoir si vous êtes dans la foi ; éprouvez-vous vous-mêmes. Ne reconnaissez-vous pas que Jésus-Christ est en vous ? A moins peut-être que vous ne soyez désapprouvés.
\VS{6}Mais j'espère que vous reconnaîtrez que nous, nous ne sommes pas désapprouvés.
\VS{7}Et je prie Dieu que vous ne fassiez rien de mal, non pour paraître nous-mêmes approuvés, mais afin que vous pratiquiez ce qui est bien et que nous, nous soyons comme désapprouvés.
\VS{8}Car nous n’avons pas de pouvoir contre la vérité, nous n’en avons que pour la vérité.
\VS{9}Nous nous réjouissons lorsque nous sommes faibles, tandis que vous êtes forts ; et ce que nous demandons à Dieu, c’est votre perfectionnement.
\VS{10}C'est pourquoi j'écris ces choses étant absent, afin que présent, je n’aie pas à user de rigueur, selon l’autorité que le Seigneur m'a donnée pour l'édification et non point pour la destruction.
\TextTitle{Conclusion}
\VS{11}Au reste, mes frères, réjouissez-vous, perfectionnez-vous, consolez-vous, ayez un même sentiment, vivez en paix ; et le Dieu de charité et de paix sera avec vous.
\VS{12}Saluez-vous les uns les autres par un saint baiser. Tous les saints vous saluent.
\VS{13}Que la grâce du Seigneur Jésus-Christ, la charité de Dieu, et la communion du Saint-Esprit soient avec vous tous. Amen !
\PPE{}
\end{multicols}

\clearpage\ShortTitle{Romains}\BookTitle{Romains}\BFont
\noindent\hrulefill
{\footnotesize
\textit{
\bigskip
{\centering{}
\\Thème : L'Evangile de Dieu
\\Auteur : Paul
\\Date de rédaction : Env. 56\\}
}
%\bigskip
\textit{
\\Rome est une ville située dans la région du Latium, au centre de l’Italie, à la confluence de l’Aniene et du Tibre. Centre de l’Empire romain,  elle domina  l’Europe, l’Afrique du Nord et le Moyen-Orient du 1er siècle avant J.-C. au 5ème siècle après J.-C.
%\bigskip
\\La lettre était destinée à l’Eglise de Rome, fondée sans doute par des chrétiens convertis au  travers du ministère de Paul et d’autres apôtres itinérants. Cette Eglise comptait quelques juifs mais surtout des membres d’origine païenne. Cette épître fut rédigée au cours du 3ème voyage missionnaire de Paul,  pendant les trois mois que l’apôtre passa à Corinthe. En attendant de leur rendre visite physiquement, Paul avait le désir de communiquer aux chrétiens de Rome les grandes lignes du principe de la grâce, dont il avait eu la révélation. Il y aborda plusieurs doctrines majeures comme le salut par la foi et la grâce ainsi que des enseignements pratiques sur l’amour, le devoir du chrétien et la sainteté.\bigskip
}
}
\par\nobreak\noindent\hrulefill
\begin{multicols}{2}
\Chap{1}
\TextTitle{Introduction : L’Evangile de Christ, puissance de Dieu pour le salut de tous}
\VerseOne{}Paul, serviteur de Jésus-Christ, appelé à être apôtre, mis à part pour annoncer l'Evangile de Dieu,
\VS{2}qu’il avait auparavant promis par ses prophètes dans les saintes Ecritures,
\VS{3}et qui concerne son Fils, qui est né de la postérité de David, selon la chair,
\VS{4}et qui a été pleinement déclaré Fils de Dieu avec puissance, selon l'Esprit de sainteté, par sa résurrection d'entre les morts, c'est-à-dire, Jésus-Christ notre Seigneur,
\VS{5}par qui nous avons reçu la grâce et l’apostolat, pour amener en son Nom tous les gentils à l’obéissance de la foi,
\VS{6}parmi lesquels aussi vous êtes, vous qui êtes appelés par Jésus-Christ.
\VS{7}A vous tous qui êtes à Rome, bien-aimés de Dieu, appelés à être saints\FTNT{Le terme «~saint~» est tiré du grec «~hagios~» qui signifie «~consacré à Dieu~», «~saint~», «~sacré~», «~pieux~». Ce mot est souvent utilisé au pluriel dans le testament de Jésus (Ac. 9:13 ;  Ac. 26:10). Il n’y a aucun rapport entre la compréhension catholique romaine du terme «~saint~» et l’enseignement biblique. Dans l’enseignement catholique romain, une personne ne devient pas sainte tant qu’elle n’a pas été béatifiée ou canonisée par le pape ou l’éminent évêque. Dans la Bible, tous ceux qui reçoivent Jésus-Christ par la foi sont appelés saints car ils sont mis à part pour Dieu. Dans le culte de l’église catholique romaine, les saints sont vénérés, priés, et parfois adorés. Dans la Bible, les saints sont appelés à adorer et à prier Dieu seul par Jésus-Christ homme, le seul Médiateur entre Dieu et les hommes (1 Ti. 2:5). Dans le Tanakh, les mots «~sanctifié~», «~saint~» et leurs dérivés viennent du mot hébreu «~Quodesch~» dont le sens général est : «~Mis à part pour Dieu~». Dans la Bible, ces mots sont appliqués à des objets et à des personnes. Le terme «~sanctification~» appliqué à des objets sous-entend l'idée qu'ils sont réservés uniquement pour le service de Dieu ; ils sont sanctifiés, mis à part pour Dieu. Dans le Testament de Jésus, il est appliqué aux personnes et comprend plusieurs sens : - Les croyants, par leur position, sont éternellement mis à part pour Dieu par la rédemption (Hé. 10:10-14). Ils sont donc considérés comme «~saints~», «~sanctifiés~» dès leur conversion (Ph. 1:1 ; Hé. 3:1). - Les croyants sont amenés à la sanctification par l'action du Saint-Esprit au moyen des Ecritures (Jn. 15:3 ; Jn. 17:17 ; 2 Co. 3:18 ; Ep. 5:25-26 ; 1Th. 5:23-24). - Les croyants attendent la venue du Seigneur pour la réalisation complète de leur sanctification (1 Co. 15:29-50 ; Ph. 3:20-21 ; Ep. 5:27 ; 1 Jn. 3:2 ; Ap. 22:12).} ; que la grâce et la paix vous soient données de la part de Dieu notre Père et du Seigneur Jésus-Christ.
\VS{8}Je rends premièrement grâces à mon Dieu par Jésus-Christ, au sujet de vous tous, de ce que votre foi est renommée dans le monde entier.
\VS{9}Car Dieu, que je sers en mon esprit dans l'Evangile de son Fils, m'est témoin que je fais sans cesse mention de vous,
\VS{10}demandant continuellement dans mes prières que je puisse enfin trouver, par la volonté de Dieu, quelque moyen favorable pour aller vers vous.
\VS{11}Car je désire extrêmement vous voir, pour vous communiquer quelque don spirituel, afin que vous soyez affermis ;
\VS{12}et aussi, afin qu'étant parmi vous, nous nous consolions ensemble par la foi qui nous est commune.
\VS{13}Or mes frères, je ne veux pas que vous ignoriez que j’ai souvent formé le dessein d'aller vers vous, afin de recueillir quelque fruit parmi vous, comme parmi les autres nations ; mais j'en ai été empêché jusqu'à présent.
\VS{14}Je me dois aux Grecs et aux barbares, aux sages et aux ignorants.
\VS{15}Ainsi, autant qu’il dépend de moi, je suis prêt à vous annoncer aussi l'Evangile à vous qui êtes à Rome.
\VS{16}Car je n'ai point honte de l'Evangile de Christ, vu qu'il est la puissance de Dieu pour le salut de tous ceux qui croient : Du Juif premièrement, puis du Grec,
\VS{17}parce qu’en lui est révélée la justice de Dieu pleinement de foi en foi, selon qu'il est écrit : Le juste vivra par la foi\FTNT{Ha. 2:4.}.
\TextTitle{Jugement sur ceux qui retiennent la vérité captive}
\VS{18}Car la colère de Dieu se révèle pleinement du ciel contre toute impiété et injustice des hommes qui retiennent injustement la vérité captive.
\VS{19}Car ce qu’on peut connaître de Dieu est manifesté parmi eux ; car Dieu le leur a fait connaître.
\VS{20}En effet, les perfections invisibles de Dieu, à savoir sa puissance éternelle et sa divinité, se voient comme à l’œil nu, depuis la création du monde, quand on les considère dans ses ouvrages, de sorte qu'ils sont inexcusables.
\VS{21}Parce qu'ayant connu Dieu, ils ne l'ont point glorifié comme Dieu, et ils ne lui ont point rendu grâces, mais ils se sont égarés dans leurs pensées, et leur cœur sans intelligence a été plongé dans les ténèbres.
\VS{22}Se vantant d’être sages, ils sont devenus fous.
\VS{23}Et ils ont changé la gloire du Dieu incorruptible en images\FTNT{Ex. 20:4-5 ; Mt. 22:20 ; Mc. 12:16 ; Lu. 20:24. Ap.13:14-15 ; Ap. 14:9-11 ; Ap. 15:2 ; Ap. 16:2 ; 19:20 ;  Ap. 20:4.} représentant l'homme corruptible, des oiseaux, des quadrupèdes, et des reptiles.
\TextTitle{Les conséquences de l'endurcissement des hommes}
\VS{24}C'est pourquoi aussi Dieu les a livrés aux convoitises de leurs cœurs et à l’impureté\FTNT{Dieu les a livrés à l’esprit d’égarement (1 R. 22 ; 2 Th. 2:10-13).}, ainsi ils déshonorent eux-mêmes leurs propres corps ;
\VS{25}eux qui ont changé la vérité de Dieu en mensonge, et qui ont adoré et servi la créature, au lieu du Créateur, qui est béni éternellement. Amen !
\VS{26}C'est pourquoi Dieu les a livrés à des passions infâmes, car leurs femmes ont changé l'usage naturel en celui qui est contre la nature.
\VS{27}Et de même les hommes, abandonnant l'usage naturel de la femme, se sont enflammés dans leurs désirs les uns envers les autres, commettant homme avec homme des choses infâmes, et recevant en eux-mêmes le salaire que méritait leur égarement.
\VS{28}Car comme ils ne se sont pas souciés de connaître Dieu, aussi Dieu les a livrés à leur sens réprouvé, pour commettre des choses indignes.
\VS{29}Etant remplis de toute espèce d’injustice, d'impureté, de méchanceté, d'avarice, de malignité, pleins d'envie, de meurtre, de querelles, de fraude, de mauvaises mœurs,
\VS{30}rapporteurs, médisants, haïssant Dieu, outrageux, orgueilleux, vains, ingénieux au mal, rebelles à leurs parents,
\VS{31}dépourvus d’intelligence, de loyauté, d’affection naturelle, de miséricorde.
\VS{32}Et bien qu'ils connaissent le jugement de Dieu, déclarant dignes de mort ceux qui commettent de telles choses, non seulement ils les font, mais encore ils approuvent ceux qui les font.
\Chap{2}
\TextTitle{Condamnation du moralisme}
\VerseOne{}C'est pourquoi, ô homme, qui que tu sois, toi qui juges les autres, tu es donc inexcusable ; car en jugeant les autres, tu te condamnes toi-même, puisque toi qui juges, tu commets les mêmes choses.
\VS{2}Or nous savons que le jugement de Dieu est selon la vérité pour ceux qui commettent de telles choses.
\VS{3}Et penses-tu, ô homme, qui juges ceux qui commettent de telles choses, et qui les commets, que tu échapperas au jugement de Dieu ?
\VS{4}Ou méprises-tu les richesses de sa douceur, et de sa patience, et de sa bonté ; ne reconnaissant pas que la bonté de Dieu te convie à la repentance ?
\VS{5}Mais par ta dureté, et par ton cœur qui est sans repentance, tu t'amasses la colère pour le jour de la colère, et de la manifestation du juste jugement de Dieu,
\VS{6}qui rendra à chacun selon ses œuvres ;
\VS{7}à savoir la vie éternelle à ceux qui, en persévérant dans les bonnes œuvres, cherchent la gloire, l'honneur et l'immortalité.
\VS{8}Mais il y aura de l'indignation et de la colère contre ceux qui ont un esprit de dispute, et qui se rebellent contre la vérité, et obéissent à l'injustice.
\VS{9}Il y aura tribulation et angoisse sur toute âme d'homme qui fait le mal, pour le Juif premièrement, puis pour le Grec.
\VS{10}Mais gloire, honneur, et paix pour quiconque fait le bien ; pour le Juif premièrement, puis pour le Grec.
\VS{11}Car Dieu n'a point d'égard à l'apparence des personnes.
\VS{12}Tous ceux qui auront péché sans la loi, périront aussi sans la loi ; et tous ceux qui auront péché ayant la loi, seront jugés par la loi.
\VS{13}Car ce ne sont pas, en effet, ceux qui écoutent la loi qui sont justes devant Dieu, mais ce sont ceux qui la mettent en pratique qui seront justifiés.
\VS{14}Or quand les gentils, qui n'ont point la loi, font naturellement ce que prescrit la loi, n'ayant point la loi, ils sont une loi pour eux-mêmes.
\VS{15}Et ils montrent par-là que l’œuvre de la loi est écrite dans leurs cœurs, puisque leur conscience leur rend témoignage, et que leurs pensées les accusent ou les défendent.
\VS{16}Tous, dis-je, donc seront jugés le jour où Dieu jugera les secrets des hommes par Jésus-Christ, selon mon Evangile.
\TextTitle{Les Juifs, connaissant la loi, sont condamnés par leur transgression de la loi}
\VS{17}Voici, tu portes le nom de Juif, tu te reposes entièrement sur la loi, et tu te glorifies de Dieu ;
\VS{18}tu connais sa volonté, et tu sais discerner ce qui est contraire, étant instruit par la loi ; 
\VS{19}et tu te crois être le conducteur des aveugles, la lumière de ceux qui sont dans les ténèbres,
\VS{20}le docteur des insensés, le maître des ignorants, ayant le modèle de la science et de la vérité dans la loi.
\VS{21}Toi donc qui enseignes les autres, tu ne t’enseignes pas toi-même ! Toi qui prêches de ne pas dérober, tu dérobes !
\VS{22}Toi qui dis de ne pas commettre d’adultère, tu commets l’adultère ! Toi qui as en abomination les idoles, tu commets des sacrilèges !
\VS{23}Toi qui te glorifies de la loi, tu déshonores Dieu par la transgression de la loi.
\VS{24}Car le nom de Dieu est blasphémé parmi les gentils à cause de vous comme cela est écrit.
\VS{25}Il est vrai que la circoncision est profitable, si tu gardes la loi ; mais si tu es transgresseur de la loi, ta circoncision devient incirconcision.
\VS{26}Si donc l’incirconcis observe les ordonnances de la loi, son incirconcision ne sera-t-elle pas tenue pour circoncision ?
\VS{27}L’incirconcis de nature, qui accomplit la loi, ne te condamnera-t-il pas, toi qui la transgresses, tout en ayant la lettre de la loi et la circoncision ?
\VS{28}Le Juif, ce n’est pas celui qui en a les apparences\FTNT{Le formalisme (2 Ti. 3:5). L’apparence de la piété correspond aux vêtements des brebis : «~Gardez-vous des faux prophètes, ils viennent à vous en habits de brebis, mais au-dedans ce sont des loups ravisseurs.~» (Mt 7:15). «~Puis je vis une autre bête qui montait de la terre, et qui avait deux cornes semblables à celles de l'Agneau ; mais elle parlait comme le dragon.~» Ap. 13:11. Il a l’apparence d’un agneau, mais sa voix est celle du dragon, c’est-à-dire Satan.} ; et la circoncision, ce n’est pas celle qui est visible dans la chair.
\VS{29}Mais le Juif, c’est celui qui l’est intérieurement ; et la circoncision, c’est celle du cœur, selon l’Esprit et non selon la lettre. La louange de ce Juif ne vient pas des hommes, mais de Dieu.
\Chap{3}
\TextTitle{L'avantage du Juif peut devenir une condamnation}
\VerseOne{}Quel est donc l'avantage du Juif, ou quelle est l’utilité de la circoncision ?
\VS{2}Cet avantage est grand de toute manière, et tout d’abord en ce que les oracles de Dieu leur ont été confiés.
\VS{3}Eh quoi ! Si quelques-uns n'ont pas cru, leur incrédulité anéantira-t-elle la fidélité de Dieu ?
\VS{4}Nullement ! Que Dieu au contraire soit reconnu pour vrai, et tout homme pour menteur ; selon ce qui est écrit : Afin que tu sois trouvé juste dans tes paroles, et que tu triomphes lorsqu’on te juge\FTNT{Ps. 51:6.}.
\VS{5}Mais si notre injustice établit la justice de Dieu, que dirons-nous ? Dieu est-il injuste quand il déchaine sa colère ? (Je parle à la manière des hommes.)
\VS{6}Nullement ! Autrement, comment Dieu jugera-t-il le monde ?
\VS{7}Et si par mon mensonge la vérité de Dieu est plus abondante pour sa gloire, pourquoi suis-je encore condamné comme pécheur ?
\VS{8}Et pourquoi ne ferions-nous pas le mal, afin qu'il en arrive du bien, comme quelques-uns, qui nous calomnient, prétendent que nous le disons ? La condamnation de ces gens est juste.
\TextTitle{Juifs et Grecs coupables devant Dieu}
\VS{9}Quoi donc ! Sommes-nous plus excellents ? Nullement. Car nous avons déjà prouvé que tous, tant Juifs que Grecs, sont assujettis au péché.
\VS{10}Selon qu'il est écrit : Il n'y a point de juste, pas même un seul\FTNT{Ps. 14:3.}.
\VS{11}Il n'y a personne qui ait de l'intelligence, il n'y a personne qui recherche Dieu.
\VS{12}Ils se sont tous égarés, ils se sont tous corrompus : Il n'y en a aucun qui fasse le bien, pas même un seul.
\VS{13}Leur gosier est un sépulcre ouvert ; ils se servent de leur langue pour tromper ; il y a du venin d'aspic sous leurs lèvres.
\VS{14}Leur bouche est pleine de malédictions et d'amertume.
\VS{15}Leurs pieds sont légers pour répandre le sang.
\VS{16}La destruction et la misère sont sur leurs voies.
\VS{17}Et ils n'ont point connu la voie de la paix.
\VS{18}La crainte de Dieu n'est pas devant leurs yeux\FTNT{Ps. 14.}.
\VS{19}Or nous savons que tout ce que la loi dit, elle le dit à ceux qui sont sous la loi, afin que toute bouche soit fermée, et que tout le monde soit reconnu coupable devant Dieu.
\VS{20}C'est pourquoi personne ne sera justifié devant lui par les œuvres de la loi, puisque c’est par la loi que vient la connaissance du péché.
\TextTitle{La justification par la foi}
\VS{21}Mais maintenant, sans la loi, la justice de Dieu est manifestée, à laquelle rendent témoignage la loi et les prophètes.
\VS{22}La justice, dis-je, de Dieu par la foi en Jésus-Christ, envers tous et sur tous ceux qui croient. Car il n'y a point de distinction.
\VS{23}Car tous ont péché\FTNT{Le mot péché vient du terme grec «~hamartano~» : «~manquer la marque, manquer le chemin de la droiture et de l’honneur, s’éloigner de la loi de Dieu~». Le péché est la violation délibérée de la loi divine et l’absence de la droiture.} et sont entièrement privés de la gloire de Dieu.
\VS{24}Et ils sont gratuitement justifiés par sa grâce, par la rédemption\FTNT{La rédemption est la délivrance par le paiement d’un prix. Trois termes grecs sont utilisés pour parler de la rédemption : - Agorazo : acheter un objet au marché (agora signifiant marché). Les pécheurs sont considérés comme des esclaves vendus au marché (Ro. 7:14). - Exagorazo : acheter et amener un objet hors du marché (Ga. 3:13 ; Ga. 4:5). L’esclave acheté et amené hors du marché est définitivement délivré. - Lutroo : détacher, rendre libre (Lu. 24:21 ; Tit. 2:14 ; 1 P. 1:18.) Jésus-Christ nous a délivrés du péché, de la puissance de Satan et de la loi mosaïque. (Col. 1:12-14 ; Col. 2:14-17 ; 1 Jn. 3:5).} qui est en Jésus-Christ.
\VS{25}C’est lui que Dieu a destiné à être, par son sang, la victime propitiatoire\FTNT{Le terme propitiation vient du grec «~hilastérion~» qui signifie «~ce qui est expié, ce qui rend propice ou le don qui assure la propitiation~». C’est aussi le lieu où s’accomplit la propitiation (Hé. 9:5), c’est-à-dire le couvercle de l’arche. Lors du grand jour des expiations (Yom Kippour en hébreu), l’aspersion du sang était faite sur le propitiatoire (Lé. 16:14). Le Seigneur Jésus-Christ est notre victime expiatoire (1 Jn. 2:2 ; 1 Jn. 4:10).} pour ceux qui croiraient, afin de montrer sa justice, parce qu’il avait laissé impunis les péchés commis auparavant, au temps de sa patience.
\VS{26}Il montre, dis-je, sa justice dans le temps présent, de manière à être trouvé juste tout en justifiant celui qui a la foi en Jésus.
\VS{27}Où est donc le sujet de se glorifier ? Il est exclu. Par quelle loi ? Est-ce par la loi des œuvres ? Non, mais par la loi de la foi.
\VS{28}Nous concluons donc que l'homme est justifié par la foi, sans les œuvres de la loi.
\TextTitle{Circoncis et incirconcis, justifiés par la foi}
\VS{29}Dieu est-il seulement le Dieu des Juifs ? Ne l'est-il pas aussi des gentils ? Certes, il l'est aussi des gentils,
\VS{30}puisqu’il y a un seul Dieu qui justifiera par la foi les circoncis, et aussi les incirconcis par la foi.
\VS{31}Anéantissons-nous donc la loi par la foi ? Nullement ! Mais au contraire, nous affermissons la loi.
\Chap{4}
\TextTitle{Abraham et David justifiés par la foi\FTNTT{cp. v. 18-25}}
\VerseOne{}Que dirons-nous donc, qu'Abraham, notre père, a obtenu selon la chair ?
\VS{2}Certes, si Abraham a été justifié par les œuvres, il a de quoi se glorifier, mais non pas envers Dieu.
\VS{3}Car que dit l'Ecriture ? Qu’Abraham a cru en Dieu, et que cela lui a été imputé à justice\FTNT{Ge. 15:6.}.
\VS{4}Or à celui qui fait les œuvres, le salaire ne lui est pas imputé comme une grâce, mais comme une chose due.
\VS{5}Mais à celui qui ne fait pas les œuvres, mais qui croit en celui qui justifie le méchant, sa foi lui est imputée à justice.
\VS{6}De même, David exprime le bonheur de l'homme à qui Dieu impute la justice sans les œuvres, en disant :
\VS{7}Heureux sont ceux à qui les iniquités sont pardonnées, et dont les péchés sont couverts.
\VS{8}Heureux l'homme à qui le Seigneur n’impute pas son péché\FTNT{Ps. 32:1-2.}.
\TextTitle{Abraham obtient la justification par la foi avant sa circoncision}
\VS{9}Cette déclaration de bénédiction, est-elle seulement pour les circoncis, ou également pour les incirconcis ? Car nous disons que la foi a été imputée à Abraham à justice.
\VS{10}Comment donc lui a-t-elle été imputée ? Etait-ce après, ou avant sa circoncision ? Il n’était pas encore circoncis, il était incirconcis.
\VS{11}Et il reçut le signe de la circoncision comme sceau de la justice, qu’il avait obtenue par la foi, quand il était incirconcis, afin d’être le père de tous les incirconcis qui croient, pour que la justice leur soit aussi imputée ;
\VS{12}et le père des circoncis, qui ne sont pas seulement circoncis, mais encore qui marchent sur les traces de la foi de notre père Abraham, quand il était incirconcis.
\TextTitle{La justification s'accomplit sans la loi}
\VS{13}En effet, ce n’est pas par loi que la promesse d'être héritier du monde a été faite à Abraham, ou à sa postérité, mais par la justice de la foi.
\VS{14}Car, si les héritiers le sont par la loi, la promesse est annulée, et la foi est vaine
\VS{15}car la loi produit la colère ; car là où il n'y a point de loi, il n'y a point non plus de transgression.
\VS{16}C'est pourquoi les héritiers le sont par la foi, pour que ce soit par la grâce, et afin que la promesse soit assurée à toute la postérité ; non seulement à celle qui est de la loi, mais aussi à celle qui est de la foi d'Abraham, qui est le père de nous tous,
\VS{17}selon qu'il est écrit : Je t'ai établi père de plusieurs nations\FTNT{Ge 17:4-5.}. Il est notre père devant celui auquel il a cru, Dieu qui donne la vie aux morts, et qui appelle les choses qui ne sont point, comme si elles étaient.
\VS{18}Et Abraham ayant espéré contre toute espérance, crut qu'il deviendrait le père de plusieurs nations, selon ce qui lui avait été dit : Ainsi sera ta postérité.
\VS{19}Et sans faiblir dans la foi, il ne considéra point que son corps était déjà usé ; puisqu’il avait environ cent ans, et que Sara n’était plus en âge d'avoir des enfants.
\VS{20}Et il ne douta point de la promesse de Dieu par incrédulité, mais il fut fortifié par la foi, donnant gloire à Dieu,
\VS{21}étant pleinement persuadé que celui qui lui avait fait la promesse était aussi puissant pour l'accomplir.
\VS{22}C'est pourquoi cela lui fut imputé à justice.
\VS{23}Mais ce n’est pas à cause de lui seul qu’il est écrit que cela lui fut imputé à justice ;
\VS{24}c’est encore à cause de nous, à qui cela sera imputé, à nous, dis-je, qui croyons en celui qui a ressuscité des morts, Jésus notre Seigneur,
\VS{25}qui a été livré pour nos offenses, et est ressuscité pour notre justification.
\Chap{5}
\TextTitle{La justification permet la réconciliation avec Dieu}
\VerseOne{}Etant donc justifiés\FTNT{La justification est l’œuvre de Dieu par laquelle la justice de Jésus est comptée en faveur du pécheur, de sorte que le pécheur est déclaré juste par Dieu (Ro. 4 : 3 ; Ro. 5:1-9 ; Ga. 2:16 ; Ga. 3:11). Cette justice n’est pas obtenue par les efforts de la personne sauvée. La justification est une action instantanée qui a pour résultat la vie éternelle. Elle repose totalement et exclusivement sur le sacrifice de Jésus à la croix (1 Pi. 2:24). Elle ne peut être reçue que par la foi en Jésus-Christ (Ep. 2:8-9). La justification est un acte d’imputation divine et non une reconnaissance personnelle de l’homme. Elle provient de la grâce (Ro. 3:24 ; Tit. 3:7).} par la foi, nous avons la paix avec Dieu, par notre Seigneur Jésus-Christ.
\VS{2}Par lequel aussi nous avons été amenés par la foi à cette grâce, dans laquelle nous tenons ferme ; et nous nous glorifions dans l'espérance de la gloire de Dieu.
\VS{3}Bien plus, nous nous glorifions même dans les afflictions ; sachant que l'affliction produit la persévérance ;
\VS{4}et la persévérance l'épreuve ; et l'épreuve l'espérance.
\VS{5}Or l'espérance ne trompe point, parce que l'amour de Dieu est répandu dans nos cœurs par le Saint-Esprit qui nous a été donné.
\VS{6}Car lorsque nous étions encore sans force, Christ est mort en son temps pour nous qui étions des impies.
\VS{7}A peine mourrait-on pour un juste ; quelqu’un peut-être mourrait pour un homme de bien.
\VS{8}Mais Dieu prouve son amour envers nous, en ce que lorsque nous étions encore pécheurs, Christ est mort pour nous.
\VS{9}Etant donc maintenant justifiés par son sang, à plus forte raison serons-nous sauvés par lui de la colère.
\VS{10}Car si, lorsque nous étions ennemis, nous avons été réconciliés avec Dieu par la mort de son Fils, à plus forte raison, étant réconciliés, serons-nous sauvés par sa vie.
\VS{11}Et non seulement cela, mais encore nous nous glorifions même en Dieu par notre Seigneur Jésus-Christ, par qui nous avons maintenant obtenu la réconciliation.
\TextTitle{Parallèle entre l'œuvre de Jésus-Christ et celle d'Adam}
\VS{12}C'est pourquoi comme par un seul homme le péché est entré dans le monde, et par le péché la mort, et qu’ainsi la mort s’est étendue sur tous les hommes, parce que tous ont péché…
\VS{13}Car jusqu'à la loi le péché était dans le monde ; or le péché n'est point imputé quand il n'y a point de loi.
\VS{14}Mais la mort a régné depuis Adam jusqu'à Moïse, même sur ceux qui n'avaient pas péché par une transgression semblable à celle d’Adam, lequel est la figure de celui qui devait venir.
\VS{15}Mais il n'en est pas du don gratuit comme de l'offense ; car si par l'offense d'un seul il en est beaucoup qui sont morts, à plus forte raison la grâce de Dieu, et le don de la grâce, venant d'un seul homme, à savoir de Jésus-Christ, ont-ils été abondamment répandus sur plusieurs.
\VS{16}Et il n'en est pas du don comme de ce qui est arrivé par un seul qui a péché ; car c’est après une seule offense que le jugement est devenu condamnation, mais le don gratuit devient justification après plusieurs offenses.
\VS{17}Si par l'offense d'un seul la mort a régné par lui seul, à plus forte raison ceux qui reçoivent l'abondance de la grâce, et du don de la justice, régneront-ils dans la vie par Jésus-Christ lui seul.
\VS{18}Ainsi donc, comme par une seule offense la condamnation est venue sur tous les hommes, de même par un acte de justice la justification qui donne la vie s’étend à tous les hommes.
\VS{19}Car, comme par la désobéissance d'un seul homme plusieurs ont été rendus pécheurs, de même par l'obéissance d'un seul plusieurs seront rendus justes.
\VS{20}Or la loi est intervenue afin que l'offense abonde, mais là où le péché a abondé, la grâce a surabondé,
\VS{21}afin que, comme le péché a régné par la mort, ainsi la grâce règne par la justice pour donner la vie éternelle, par Jésus-Christ notre Seigneur.
\Chap{6}
\TextTitle{Délivré de la puissance du péché lié au cœur de l’homme }
\VerseOne{}Que dirons-nous donc ? Demeurerions-nous dans le péché, afin que la grâce abonde ?
\VS{2}A Dieu ne plaise ! Car nous qui sommes morts au péché, comment vivrions-nous encore dans le péché ?
\VS{3}Ignoriez-vous que nous tous qui avons été baptisés en Jésus-Christ, c’est en sa mort que nous avons été baptisés ?
\VS{4}Nous avons donc été ensevelis avec lui par le baptême en sa mort ; afin que comme Christ est ressuscité des morts par la gloire du Père, de même nous aussi nous marchions en nouveauté de vie.
\VS{5}Car, si nous sommes devenus une même plante avec lui par la conformité à sa mort, nous le serons aussi par la conformité à sa résurrection.
\VS{6}Sachant que notre vieil homme a été crucifié avec lui, afin que le corps du péché soit détruit, pour que nous ne soyons plus esclaves du péché.
\VS{7}Car celui qui est mort est libre du péché.
\VS{8}Or si nous sommes morts avec Christ, nous croyons que nous vivrons aussi avec lui,
\VS{9}sachant que Christ ressuscité des morts ne meurt plus, et que la mort n'a plus de pouvoir sur lui.
\VS{10}Car il est mort, et c’est pour le péché qu’il est mort une fois pour toutes ; il est revenu à la vie, et c’est pour Dieu qu’il est vivant.
\TextTitle{Mort au péché pour une vie nouvelle en Dieu}
\VS{11}Ainsi vous-mêmes, considérez-vous comme morts au péché, et comme vivants pour Dieu en Jésus-Christ notre Seigneur.
\VS{12}Que le péché ne règne donc point dans votre corps mortel, et n’obéissez pas à ses convoitises.
\VS{13}Et ne livrez pas vos membres au péché comme des instruments d'iniquité ; mais donnez-vous vous-mêmes à Dieu comme de morts étant devenus vivants, et offrez vos membres à Dieu pour être des instruments de justice.
\VS{14}Car le péché n'aura pas de domination sur vous, parce que vous n'êtes point sous la loi, mais sous la grâce.
\VS{15}Quoi donc ? Pécherions-nous parce que nous ne sommes point sous la loi, mais sous la grâce ? A Dieu ne plaise !
\VS{16}Ne savez-vous pas qu’en vous livrant à quelqu’un comme esclaves pour lui obéir, vous êtes esclaves de celui à qui vous obéissez, soit du péché qui conduit à la mort, soit de l'obéissance qui conduit à la justice ?
\VS{17}Mais grâces à Dieu de ce qu'ayant été les esclaves du péché, vous avez obéi de cœur à la forme expresse de la doctrine dans laquelle vous avez été élevés.
\VS{18}Ayant donc été affranchis du péché, vous avez été asservis à la justice.
\VS{19}Je parle à la façon des hommes, à cause de l'infirmité de votre chair. Comme donc vous avez appliqué vos membres pour servir à la souillure et à l’iniquité, ainsi appliquez vos membres pour servir à la justice en sainteté.
\VS{20}Car lorsque vous étiez esclaves du péché, vous étiez libres à l'égard de la justice.
\VS{21}Quel fruit portiez-vous alors ? Des fruits dont vous avez honte maintenant. Car la fin de ces choses c’est la mort.
\VS{22}Mais maintenant que vous êtes affranchis du péché, et asservis à Dieu, vous avez pour fruit la sanctification, et pour fin la vie éternelle.
\VS{23}Car le salaire du péché, c'est la mort ; mais le don gratuit de Dieu, c'est la vie éternelle par Jésus-Christ notre Seigneur.
\Chap{7}
\TextTitle{Le chrétien lié à Christ comme à un époux}
\VerseOne{}Ignorez-vous, frères, car je parle à des gens qui connaissent la loi, que la loi exerce son pouvoir sur l’homme aussi longtemps qu’il vit ?
\VS{2}Car la femme qui est sous la puissance d'un mari, est liée à son mari par la loi tandis qu'il est en vie ; mais si son mari meurt, elle est délivrée de la loi du mari.
\VS{3}Si donc, du vivant de son mari, elle épouse un autre homme, elle sera appelée adultère ; mais si son mari meurt, elle est délivrée de la loi, de sorte qu'elle ne sera point adultère si elle épouse un autre homme.
\VS{4}Ainsi donc, vous aussi, mes frères, vous avez été, par le corps de Christ, mis à mort en ce qui concerne la loi, pour que vous apparteniez à un autre, à savoir, à celui qui est ressuscité des morts, afin que nous portions des fruits pour Dieu.
\VS{5}Car lorsque nous étions dans la chair, les passions des péchés excitées par la loi, agissaient dans nos membres de manière à produire des fruits pour la mort.
\VS{6}Mais maintenant nous sommes délivrés de la loi, étant morts à cette loi sous laquelle nous étions retenus ; afin que nous servions Dieu dans un esprit nouveau, et non selon la lettre qui a vieilli.
\TextTitle{La loi a révélé le péché mais la délivrance vient par Jésus-Christ}
\VS{7}Que dirons-nous donc ? La loi est-elle péché ? Nullement ! Au contraire, je n'ai connu le péché que par la loi ; car je n’aurais pas connu la convoitise, si la loi n’avait pas dit : Tu ne convoiteras point\FTNT{Ex. 20:17.}.
\VS{8}Et le péché, saisissant l’occasion, produisit en moi, par le commandement, toutes sortes de convoitises ; parce que sans la loi le péché est mort.
\VS{9}Pour moi, étant autrefois sans loi, je vivais. Mais quand le commandement vint, le péché reprit vie, et moi je mourus.
\VS{10}Ainsi, le commandement qui conduit à la vie se trouva pour moi conduire à la mort.
\VS{11}Car le péché, saisissant l’occasion, me séduisit par le commandement, et par lui me fit mourir.
\VS{12}La loi donc est sainte, et le commandement est saint, juste, et bon.
\VS{13}Ce qui est bon a-t-il donc été pour moi une cause de mort ? Nullement ! Mais c’est le péché, afin qu'il se manifeste comme péché, en me donnant la mort par ce qui est bon, et que par le commandement, il devienne condamnable au plus haut point.
\VS{14}Car nous savons, en effet, que la loi est spirituelle ; mais moi, je suis charnel, vendu au péché.
\TextTitle{[la connaissance du bien incapable de délivrer l'homme du péché}
\VS{15}Car je n'approuve pas ce que je fais, puisque je ne fais point ce que je veux, mais je fais ce que je hais.
\VS{16}Or si je fais ce que je ne veux pas, je reconnais par cela même que la loi est bonne.
\VS{17}Et maintenant donc ce n'est plus moi qui fais cela, mais c'est le péché qui habite en moi.
\VS{18}Ce qui est bon, je le sais, n’habite pas en moi, c’est-à-dire dans ma chair. J’ai la volonté, mais non le pouvoir de faire le bien.
\VS{19}Car je ne fais pas le bien que je veux, mais je fais le mal que je ne veux point.
\VS{20}Or si je fais ce que je ne veux point, ce n'est plus moi qui le fais, mais c'est le péché qui habite en moi.
\VS{21}Je trouve donc cette loi au-dedans de moi : Quand je veux faire le bien, le mal est attaché à moi.
\VS{22}Car je prends bien plaisir à la loi de Dieu quant à l'homme intérieur,
\VS{23}mais je vois dans mes membres une autre loi, qui combat contre la loi de mon entendement\FTNT{Entendement : Du grec «~nous~», c’est-à-dire l’esprit, l’intelligence, le bon sens, la raison.}, et qui me rend prisonnier à la loi du péché qui est dans mes membres.
\VS{24}Ah, misérable que je suis ! Qui me délivrera du corps de cette mort ?
\TextTitle{Seul l'Esprit de Christ libère de la loi du péché}
\VS{25}Je rends grâces à Dieu par Jésus-Christ notre Seigneur !… Ainsi donc, moi-même, je suis par l’entendement esclave de la loi de Dieu, et je suis par la chair esclave de la loi du péché.
\Chap{8}
\VerseOne{}Il n'y a donc maintenant aucune condamnation pour ceux qui sont en Jésus-Christ, qui marchent, non selon la chair, mais selon l'Esprit.
\VS{2}Parce que la loi de l'Esprit de vie qui est en Jésus-Christ m'a affranchi de la loi du péché et de la mort.
\VS{3}Car chose impossible à la loi, parce que la chair la rendait impuissante, Dieu a condamné le péché dans la chair, en envoyant, à cause du péché, son propre Fils dans une chair semblable à celle du péché.
\VS{4}Afin que la justice de la loi soit accomplie en nous, qui ne marchons point selon la chair, mais selon l'Esprit.
\TextTitle{L'affection de l'Esprit opposée à celle de la chair\FTNTT{cp. Ga. 5:15-18}}
\VS{5}Car ceux, en effet, qui vivent selon la chair, s’affectionnent aux choses de la chair, tandis que ceux qui vivent selon l'Esprit, s’affectionnent aux choses de l'Esprit.
\VS{6}Or l'affection de la chair c’est la mort, tandis que l'affection de l'Esprit c’est la vie et la paix.
\VS{7}Car l'affection de la chair est inimitié contre Dieu, parce qu’elle ne se soumet pas à la loi de Dieu, et qu’elle ne le peut même pas.
\VS{8}C'est pourquoi ceux qui vivent selon la chair ne sauraient plaire à Dieu.
\VS{9}Pour vous, vous ne vivez pas selon la chair, mais selon l'Esprit, si du moins l'Esprit de Dieu habite en vous. Si quelqu'un n'a pas l'Esprit de Christ\FTNT{Notez que le Saint-Esprit est aussi appelé l’Esprit de Jésus (Ac. 16:7).}, il ne lui appartient pas.
\VS{10}Et si Christ est en vous, le corps est bien mort à cause du péché, mais l'Esprit est vie à cause de la justice.
\VS{11}Et si l'Esprit de celui qui a ressuscité Jésus d’entre les morts habite en vous, celui qui a ressuscité Christ d’entre les morts rendra aussi la vie à vos corps mortels par son Esprit qui habite en vous.
\VS{12}Ainsi donc, mes frères, nous ne sommes point redevables à la chair, pour vivre selon la chair.
\VS{13}Car si vous vivez selon la chair, vous mourrez ; mais si par l'Esprit vous faites mourir les actions du corps, vous vivrez.
\TextTitle{L'Esprit d'adoption\FTNTT{Ga. 4:7}}
\VS{14}Car tous ceux qui sont conduits par l'Esprit de Dieu sont enfants de Dieu.
\VS{15}Et vous n'avez point reçu un esprit de servitude pour être encore dans la crainte ; mais vous avez reçu l'Esprit d'adoption, par lequel nous crions Abba, c'est-à-dire Père.
\VS{16}L’Esprit lui-même rend témoignage à notre esprit que nous sommes enfants de Dieu.
\VS{17}Et si nous sommes enfants, nous sommes aussi héritiers : Héritiers, dis-je, de Dieu, et cohéritiers de Christ ; si toutefois nous souffrons avec lui, afin d’être glorifiés avec lui.
\TextTitle{La gloire à venir\FTNTT{cp. Ge. 3:18-19}}
\VS{18}Car tout bien compté, j'estime que les souffrances du temps présent ne sauraient être comparables à la gloire à venir qui doit être révélée pour nous.
\VS{19}Aussi, la création attend-elle avec un ardent désir la révélation des fils de Dieu.
\VS{20}Car la création a été soumise à la vanité, non de son gré, mais à cause de celui qui l’y a soumise ;
\VS{21}avec l’espérance qu’elle aussi sera affranchie de la servitude de la corruption, pour avoir part à la liberté de la gloire des enfants de Dieu.
\VS{22}Or, nous savons que jusqu’à ce jour, toute la création soupire et souffre les douleurs de l’enfantement.
\VS{23}Et non seulement elle, mais nous aussi, qui avons les prémices de l'Esprit ; nous-mêmes, dis-je, soupirons en nous-mêmes, en attendant l'adoption, c'est-à-dire la rédemption de notre corps\FTNT{1 Co. 15:35-43 ; 1 Co. 15:51-54.}.
\VS{24}Car c’est en espérance que nous sommes sauvés. Or l’espérance qu’on voit n’est plus espérance : Ce qu’on voit, peut-on l’espérer encore ?
\VS{25}Mais si nous espérons ce que nous ne voyons pas, c'est que nous l'attendons avec patience.
\TextTitle{L'Esprit intercède pour les saints\FTNTT{Hé. 7:25}}
\VS{26}De même aussi l’Esprit nous aide dans notre faiblesse, car nous ne savons pas ce qu’il nous convient de demander dans nos prières. Mais l’Esprit lui-même intercède par des soupirs inexprimables.
\VS{27}Et celui qui sonde les cœurs connaît quelle est la pensée de l'Esprit, car il intercède en faveur des saints, selon Dieu.
\TextTitle{Le plan de Dieu s'accomplit par l'Evangile}
\VS{28}Or nous savons aussi que toutes choses concourent au bien de ceux qui aiment Dieu, c'est-à-dire de ceux qui sont appelés selon son dessein.
\VS{29}Car ceux qu'il a connus d’avance, il les a aussi prédestinés à être semblables à l'image de son Fils, afin qu'il soit le premier-né de beaucoup de frères.
\VS{30}Et ceux qu'il a prédestinés, il les a aussi appelés ; et ceux qu'il a appelés, il les a aussi justifiés ; et ceux qu'il a justifiés, il les a aussi glorifiés.
\VS{31}Que dirons-nous donc à l’égard de ces choses ? Si Dieu est pour nous, qui sera contre nous ?
\VS{32}Lui qui n'a point épargné son propre Fils, mais qui l'a livré pour nous tous, comment ne nous donnera-t-il point aussi toutes choses avec lui ?
\VS{33}Qui accusera les élus de Dieu ? Dieu est celui qui justifie.
\VS{34}Qui les condamnera ? Christ est mort ; et bien plus, il est ressuscité, il est à la droite de Dieu, et il intercède pour nous.
\TextTitle{L'amour de Christ résiste contre tout}
\VS{35}Qui nous séparera de l'amour de Christ ? Sera-ce l'oppression, ou l'angoisse, ou la persécution, ou la famine, ou la nudité, ou le péril, ou l'épée ?
\VS{36}Ainsi qu'il est écrit : C’est à cause de toi que nous sommes livrés à la mort tous les jours, et qu’on nous regarde comme des brebis destinées à la boucherie\FTNT{Ps. 44:23.}.
\VS{37}Mais dans toutes ces choses nous sommes plus que vainqueurs par celui qui nous a aimés.
\VS{38}Car j’ai l’assurance que ni la mort, ni la vie, ni les anges, ni les principautés, ni les puissances, ni les choses présentes, ni les choses à venir,
\VS{39}ni la hauteur, ni la profondeur, ni aucune autre créature, ne pourra nous séparer de l'amour de Dieu manifesté en Jésus-Christ notre Seigneur.
\Chap{9}
\TextTitle{Le chagrin de Paul pour Israël son peuple}
\VerseOne{}Je dis la vérité en Christ, je ne mens point, ma conscience m’en rend témoignage par le Saint-Esprit :
\VS{2}J’éprouve une grande tristesse et un chagrin continuel dans mon cœur.
\VS{3}Car moi-même je souhaiterais être anathème et séparé de Christ pour mes frères, mes parents selon la chair,
\TextTitle{Les enfants de la chair et ceux de la promesse}
\VS{4}qui sont Israélites, à qui appartiennent l'adoption, la gloire, les alliances, l'ordonnance de la loi, le culte,
\VS{5}les promesses, les patriarches, et de qui est issu selon la chair Christ, qui est Dieu au-dessus de toutes choses, béni éternellement, Amen !
\VS{6}Toutefois il ne peut pas se faire que la parole de Dieu soit anéantie. Car tous ceux qui descendent d’Israël ne sont pas Israël.
\VS{7}Et bien qu’ils soient de la postérité d'Abraham, ils ne sont pas tous ses enfants, car il est dit : C'est en Isaac que tu auras une postérité appelée de ton nom ;
\VS{8}c'est-à-dire que ce ne sont pas ceux qui sont enfants de la chair qui sont enfants de Dieu, mais que ce sont les enfants de la promesse qui sont regardés comme la postérité.
\VS{9}Car voici la parole de la promesse : Je viendrai à cette même époque, et Sara aura un fils\FTNT{Ge. 18:10.}.
\VS{10}Et de plus, il en fut ainsi de Rébecca, qui conçut du seul Isaac notre père ;
\VS{11}car les enfants n’étaient pas encore nés et ils n’avaient fait ni bien ni mal, afin que le dessein arrêté selon l'élection de Dieu subsiste, sans dépendre des œuvres, mais par la volonté de celui qui appelle,
\VS{12}il lui fut dit : L’aîné sera assujetti au plus petit\FTNT{Ge. 25:23.}, selon qu’il est écrit :
\VS{13}J'ai aimé Jacob, et j'ai haï Esaü\FTNT{Mal. 1:2-3.}.
\TextTitle{La volonté souveraine de Dieu}
\VS{14}Que dirons-nous donc : Y a-t-il de l’injustice en Dieu ? A Dieu ne plaise !
\VS{15}Car il dit à Moïse : J'aurai compassion de celui de qui j’aurai compassion et je ferai miséricorde à celui à qui je ferai miséricorde\FTNT{Ex. 33:19.}.
\VS{16}Ainsi donc, cela ne vient pas de celui qui veut, ni de celui qui court, mais de Dieu qui fait miséricorde.
\VS{17}Car l'Ecriture dit à Pharaon : Je t'ai suscité dans le but de démontrer en toi ma puissance, et afin que mon Nom soit publié par toute la terre\FTNT{Ex. 9:16.}.
\VS{18}Ainsi, il fait miséricorde à qui il veut, et il endurcit qui il veut.
\VS{19}Tu me diras : Pourquoi se plaint-il encore ? Car qui est celui qui peut résister à sa volonté ?
\VS{20}Mais plutôt, ô homme, qui es-tu, toi qui contestes contre Dieu ? Le vase d’argile dira-t-il à celui qui l'a formé : Pourquoi m'as-tu ainsi fait ?
\VS{21}Le potier n'a-t-il pas le pouvoir de faire avec la même masse de terre un vase d’honneur et un vase d’un usage vil ?
\VS{22}Et que dire, si Dieu, en voulant montrer sa colère, et faire connaître sa puissance, a supporté avec une grande patience les vases de colère, préparés pour la perdition ?
\VS{23}Et s’il a voulu faire connaître les richesses de sa gloire envers les vases de miséricorde, qu'il a préparés d’avance pour la gloire ?
\VS{24}Ainsi il nous a appelés, non seulement d'entre les juifs, mais aussi d'entre les gentils,
\TextTitle{Les prophéties concernant l'aveuglement d'Israël et la grâce sur les gentils}
\VS{25}selon ce qu'il dit dans Osée : J'appellerai mon peuple celui qui n'était point mon peuple ; et la bien-aimée, celle qui n'était point la bien-aimée ;
\VS{26}et il arrivera qu'au lieu où il leur a été dit : Vous n’êtes pas mon peuple, là ils seront appelés les fils du Dieu vivant\FTNT{Os. 2:1.}.
\VS{27}Aussi Esaïe s'écrie au sujet d'Israël : Quand le nombre des enfants d'Israël serait comme le sable de la mer, un petit reste seulement sera sauvé.
\VS{28}Car le Seigneur exécutera pleinement et promptement sa parole sur la terre ce qu’il a résolu\FTNT{Es. 10:22-23.}.
\VS{29}Et comme Esaïe avait dit auparavant : Si le Seigneur des armées ne nous avait laissé une postérité, nous serions devenus comme Sodome, et nous aurions été semblables à Gomorrhe\FTNT{Es. 1:9.}.
\VS{30}Que dirons-nous donc ? Que les gentils, qui ne cherchaient pas la justice, ont obtenu la justice, la justice qui vient de la foi,
\VS{31}tandis qu’Israël qui cherchait la loi de la justice, n'est pas parvenu à cette loi.
\VS{32}Pourquoi ? Parce qu’Israël l’a cherchée non par la foi, mais comme provenant des œuvres de la loi. Ils se sont heurtés contre la pierre d'achoppement,
\VS{33}selon qu’il est écrit : Voici, je mets en Sion la pierre d'achoppement ; et un rocher de scandale, et quiconque croit en lui ne sera point confus\FTNT{Es. 28:16.}.
\Chap{10}
\TextTitle{La foi, seule condition du salut}
\VerseOne{}Mes frères, le souhait de mon cœur, et la prière que je fais à Dieu pour les Israélites, c'est qu'ils soient sauvés.
\VS{2}Car je leur rends témoignage qu'ils ont du zèle pour Dieu, mais sans connaissance.
\VS{3}Parce que ne connaissant point la justice de Dieu, et cherchant à établir leur propre justice, ils ne se sont point soumis à la justice de Dieu.
\VS{4}Car Christ est la fin de la loi\FTNT{Il est question de la loi cérémonielle relative au culte mosaïque. Avant sa mort, Jésus qui était né sous la loi (Ga. 4:4), demandait aux gens de l’appliquer. Ainsi, il demanda au lépreux qu’il avait guéri de présenter une offrande pour sa purification au temple (Mt. 8:1-4) et à ses disciples d'observer l'enseignement des scribes (Mt. 23:1-2). En effet, il fallait que les lois cérémonielles soient respectées jusqu’à sa résurrection. Une fois que Jésus eut dit «~tout est accompli~» (Jn. 19:30), toutes ces lois n’avaient plus aucune raison d’être (Col. 2:14-17 ; Hé. 7:11-22 ; Hé. 10:1-2).} pour la justification de tous ceux qui croient.
\VS{5}En effet, Moïse décrit ainsi la justice qui vient de la loi : L'homme qui fera ces choses vivra par elles\FTNT{Lé. 18:5.}.
\VS{6}Mais voici comment s'exprime la justice qui vient de la foi : Ne dis pas en ton cœur : Qui montera au ciel ? C’est en faire descendre Christ.
\VS{7}Ou : Qui descendra dans l'abîme ? C’est faire remonter Christ d’entre les morts.
\VS{8}Mais que dit-elle ? La parole est près de toi, dans ta bouche, et dans ton cœur. Or voilà la parole de foi que nous prêchons.
\VS{9}C'est pourquoi, si tu confesses de ta bouche le Seigneur Jésus, et si tu crois dans ton cœur que Dieu l'a ressuscité des morts, tu seras sauvé.
\VS{10}Car c’est en croyant du cœur qu’on parvient à la justice, et c’est en confessant de la bouche qu’on parvient au salut, selon ce que dit l’Ecriture :
\VS{11}Quiconque croit en lui ne sera point confus\FTNT{Es. 49:23.}.
\VS{12}Parce qu'il n'y a point de différence, en effet, entre le Juif et le Grec, puisqu’ils ont un même Seigneur, qui est riche pour tous ceux qui l'invoquent.
\VS{13}Car quiconque invoquera le nom du Seigneur sera sauvé\FTNT{Joë. 2:32.}.
\TextTitle{La proclamation de l'Evangile dans les nations}
\VS{14}Mais comment invoqueront-ils celui en qui ils n'ont point cru ? Et comment croiront-ils en celui dont ils n'ont point entendu parler ? Et comment en entendront-ils parler s'il n'y a personne qui leur prêche ?
\VS{15}Et comment y aura-t-il des prédicateurs, s’ils ne sont pas envoyés ? Selon qu'il est écrit : Qu’ils sont beaux les pieds de ceux qui annoncent la paix, de ceux qui annoncent de bonnes nouvelles\FTNT{Es. 52:7.} !
\VS{16}Mais tous n'ont pas obéi à l'Evangile ; car Esaïe dit : Seigneur, qui a cru à notre prédication\FTNT{Es. 53:1.} ?
\VS{17}Ainsi la foi vient de ce qu’on entend, et ce qu’on entend vient de la parole de Christ.
\VS{18}Mais je dis : Ne l'ont-ils point entendue ? Au contraire, leur voix est allée par toute la terre, et leur parole jusqu’aux extrémités du monde.
\VS{19}Mais je dis : Israël ne l'a-t-il point su ? Moïse le premier dit : J’exciterai votre jalousie par ce qui n'est point une nation, je provoquerai votre colère par une nation sans intelligence\FTNT{De. 32:21.}.
\VS{20}Et Esaïe pousse la hardiesse jusqu’à dire : J'ai été trouvé par ceux qui ne me cherchaient point, et je me suis clairement manifesté à ceux qui ne me demandaient pas\FTNT{Es. 65:1.}.
\VS{21}Mais au sujet d’Israël, il dit : J'ai tout le jour tendu mes mains vers un peuple rebelle et contredisant\FTNT{Es. 65:2.}.
\Chap{11}
\TextTitle{Un reste d'Israël participe à la grâce}
\VerseOne{}Je dis donc : Dieu a-t-il rejeté son peuple ? A Dieu ne plaise ! Car je suis aussi Israélite, de la postérité d'Abraham, de la tribu de Benjamin.
\VS{2}Dieu n'a point rejeté son peuple, qu’il a connu d’avance. Et ne savez-vous pas ce que l'Ecriture dit d'Elie, comment il a fait requête à Dieu contre Israël, disant :
\VS{3}Seigneur, ils ont tué tes prophètes, et ils ont démoli tes autels, et je suis resté moi seul ; et ils cherchent à m'ôter la vie\FTNT{1 R. 19:10.}.
\VS{4}Mais quelle réponse Dieu lui donna-t-il ? Je me suis réservé sept mille hommes, qui n'ont point fléchi le genou devant Baal\FTNT{1 R.19:18.}.
\VS{5}De même aussi dans le temps présent, il y a un reste selon l'élection de la grâce.
\VS{6}Or si c'est par la grâce, ce n'est plus par les œuvres ; autrement la grâce n'est plus la grâce. Mais si c'est par les œuvres, ce n'est plus par une grâce ; autrement l’œuvre n'est plus une œuvre.
\TextTitle{La nation d'Israël est temporairement mise à l'écart mais non rejetée}
\VS{7}Quoi donc ? Ce qu'Israël cherche, il ne l'a point obtenu ; mais les élus l’ont obtenu, tandis que les autres ont été endurcis,
\VS{8}selon qu'il est écrit : Dieu leur a donné un esprit d’assoupissement, des yeux pour ne point voir, et des oreilles pour ne point entendre\FTNT{Es. 29:10.}, jusqu’à ce jour. Et David dit :
\VS{9}Que leur table soit pour eux un filet, un piège, une occasion de chute, et cela pour leur récompense.
\VS{10}Que leurs yeux soient obscurcis pour ne point voir\FTNT{Ps. 69:23-24.} ; et tiens continuellement leur dos courbé !
\VS{11}Mais je dis : Est-ce pour tomber qu’ils ont bronché ? Nullement ! Mais par leur chute, le salut est accordé aux Gentils, afin qu’ils soient excités à la jalousie.
\VS{12}Or si leur chute est la richesse du monde, et leur amoindrissement la richesse des Gentils, combien plus en sera-t-il quand ils se convertiront tous ?
\TextTitle{Avertissement aux gentils}
\VS{13}Car je vous parle à vous, Gentils, en tant qu’apôtre des Gentils, je glorifie mon ministère,
\VS{14}afin, s’il est possible, d’exciter la jalousie de ceux de ma race et d’en sauver quelques-uns.
\VS{15}Car si leur mise à l’écart a été la réconciliation du monde, quelle sera leur réintégration, sinon le passage de la mort à la vie ?
\VS{16}Or si les prémices sont saintes, la masse l'est aussi ; et si la racine est sainte, les branches le sont aussi.
\VS{17}Mais si quelques-unes des branches ont été retranchées, et si toi qui étais un olivier sauvage, tu as été greffé à leur place et rendu participant de la racine et de la graisse de l'olivier,
\VS{18}ne te glorifie pas contre ces branches ; car si tu te glorifies, ce n'est pas toi qui portes la racine, mais c'est la racine qui te porte.
\VS{19}Mais tu diras : Les branches ont été retranchées, afin que moi je sois greffé.
\VS{20}Cela est vrai, elles ont été retranchées à cause de leur incrédulité, et tu es debout par la foi ; ne t'élève donc point par orgueil, mais crains.
\VS{21}Car si Dieu n'a point épargné les branches naturelles, prends garde qu'il ne t'épargne pas non plus.
\VS{22}Considère donc la bonté et la sévérité de Dieu ; la sévérité envers ceux qui sont tombés ; et la bonté envers toi, si tu persévères dans cette bonté : Car autrement tu seras aussi retranché.
\VS{23}Eux de même, s'ils ne persistent pas dans leur incrédulité, ils seront greffés ; car Dieu est puissant pour les greffer de nouveau.
\VS{24}Car si toi tu as été coupé de l'olivier sauvage selon sa nature, et greffé contrairement à ta nature sur l'olivier franc, à plus forte raison eux seront-ils greffés selon leur nature sur leur propre olivier.
\VS{25}Car mes frères, je ne veux pas que vous ignoriez ce mystère, afin que vous ne vous regardiez point comme sages : Une partie d’Israël est tombée dans l’endurcissement, jusqu’à ce que la totalité des gentils soit entrée.
\TextTitle{Yahweh prédit le salut futur d'Israël\FTNTT{Es. 66:8}}
\VS{26}Et ainsi tout Israël sera sauvé, selon qu’il est écrit : Le Libérateur viendra de Sion, et il détournera de Jacob les infidélités ;
\VS{27}et c'est là l'alliance que je ferai avec eux, lorsque j'ôterai leurs péchés\FTNT{Es. 59:20-21.}.
\VS{28}Ils sont certes ennemis par rapport à l'Evangile, à cause de vous ; mais en ce qui concerne l’élection, ils sont aimés à cause de leurs pères.
\VS{29}Car Dieu ne se repent pas de ses dons et de sa vocation.
\VS{30}De même que vous avez autrefois désobéi à Dieu et que par leur désobéissance vous avez maintenant obtenu miséricorde,
\VS{31}de même ils ont maintenant désobéi, afin que par la miséricorde qui vous a été faite, ils obtiennent aussi miséricorde.
\VS{32}Car Dieu les a tous renfermés sous la rébellion afin de faire miséricorde à tous.
\TextTitle{Les voies incompréhensibles de Dieu}
\VS{33}Ô profondeur de la richesse, de la sagesse et de la connaissance de Dieu ! Que ses jugements sont insondables et ses voies incompréhensibles !
\VS{34}Car qui a connu la pensée du Seigneur ? Ou qui a été son conseiller ?
\VS{35}Qui lui a donné le premier, pour qu’il ait à recevoir en retour ?
\VS{36}Car c’est de lui, par lui, et pour lui que sont toutes choses. A lui soit la gloire éternellement. Amen !
\Chap{12}
\TextTitle{Le culte raisonnable}
\VerseOne{}Je vous exhorte donc, mes frères, par les compassions de Dieu, à offrir vos corps comme un sacrifice vivant, saint, agréable à Dieu, ce qui est votre culte raisonnable.
\VS{2}Et ne vous conformez pas au siècle présent, mais soyez transformés\FTNT{Le verbe «~transformer~» est la traduction du terme grec «~metamorphoo~» qui a donné en français «~transfigurer~». C’est le même terme qui a été utilisé en Mt. 17:2 pour parler de la transfiguration du Seigneur. Si Paul recommandait cela à des personnes déjà converties, c’est parce que Dieu les appelait à aller plus loin. La transformation d’une chenille en papillon est un très bel exemple pour illustrer le changement radical qui doit s’opérer en nous. Pour atteindre ce stade, cet insecte passe par plusieurs étapes. La transformation nous permet de croître spirituellement. En effet, tout enfant de Dieu est appelé à devenir mature, à passer du stade de petit enfant à celui de jeune homme, et de celui de jeune homme à celui de père (1 Jn. 2:12-14).} par le renouvellement de votre entendement, afin que vous discerniez quelle est la volonté de Dieu, ce qui est bon, agréable et parfait.
\TextTitle{Exhortation à l'humilité et au service selon les dons de l'Esprit}
\VS{3}Par la grâce qui m’a été donnée, je dis à chacun de vous que nul ne présume d'être plus sage qu'il ne faut, mais d’avoir des sentiments modestes, selon la mesure de foi que Dieu a départie à chacun.
\VS{4}Car comme nous avons plusieurs membres dans un seul corps, et que tous les membres n'ont pas la même fonction,
\VS{5}ainsi, nous qui sommes plusieurs, nous formons un seul corps en Christ, et nous sommes tous membres les uns des autres.
\VS{6}Puisque nous avons des dons différents, selon la grâce qui nous est donnée, que celui qui a le don de prophétie l’exerce en analogie de la foi ;
\VS{7}que celui qui est appelé au ministère, s’attache à son ministère ; que celui qui enseigne s’attache à son enseignement,
\VS{8}et celui qui exhorte, à l’exhortation ; que celui qui donne, le fasse avec simplicité ; que celui qui préside, le fasse avec zèle ; que celui qui exerce la miséricorde, le fasse avec joie.
\TextTitle{Les relations mutuelles entre chrétien}
\VS{9}Que la charité soit sincère. Ayez en horreur le mal, attachez-vous fortement au bien.
\VS{10}Par charité fraternelle, soyez pleins d’affection les uns pour les autres ; par honneur, usez de prévenances réciproques.
\VS{11}Ne soyez point paresseux à vous employer pour autrui. Soyez fervents d'esprit. Servez le Seigneur.
\VS{12}Soyez joyeux dans l'espérance. Soyez patients dans la tribulation. Persévérez dans la prière.
\VS{13}Pourvoyez aux besoins des saints. Exercez l'hospitalité.
\VS{14}Bénissez ceux qui vous persécutent ; bénissez-les, et ne les maudissez point.
\VS{15}Réjouissez-vous avec ceux qui se réjouissent. Pleurez avec ceux qui pleurent.
\VS{16}Ayez les mêmes sentiments les uns envers les autres. N’aspirez pas à ce qui est élevé, mais laissez-vous attirer par ce qui est humble. Ne soyez point sages à votre propre jugement.
\TextTitle{Les relations du chrétiens avec ceux du dehors}
\VS{17}Ne rendez à personne le mal pour le mal. Recherchez les choses honnêtes devant tous les hommes.
\VS{18}S’il est possible, autant que cela dépend de vous, soyez en paix avec tous les hommes.
\VS{19}Ne vous vengez point vous-mêmes, mes bien-aimés, mais laissez agir la colère de Dieu, car il est écrit : A moi appartient la vengeance, à moi la rétribution, dit le Seigneur\FTNT{De. 32:35.}.
\VS{20}Si donc ton ennemi a faim, donne-lui à manger ; s'il a soif, donne-lui à boire, car en faisant cela, tu amasseras des charbons ardents sur sa tête.
\VS{21}Ne te laisse pas vaincre par le mal, mais surmonte le mal par le bien.
\Chap{13}
\TextTitle{Le chrétien et les autorités}
\VerseOne{}Que toute personne soit soumise aux autorités supérieures, car il n'y a point d’autorité qui ne vienne pas de Dieu, et les autorités qui existent ont été instituées de Dieu.
\VS{2}C'est pourquoi celui qui s’oppose à l’autorité résiste à l’ordre de Dieu ; et ceux qui y résistent attireront la condamnation sur eux-mêmes.
\VS{3}Car ce n’est pas pour une bonne action, c’est pour une mauvaise que les magistrats sont à craindre. Veux-tu ne pas craindre l’autorité ? Fais le bien, et tu auras sa louange.
\VS{4}Car le magistrat est un serviteur de Dieu pour ton bien. Mais si tu fais le mal, crains, car ce n’est pas en vain qu’il porte l'épée, étant serviteur de Dieu, ordonné pour faire justice en punissant celui qui fait le mal.
\VS{5}C'est pourquoi il faut être soumis, non seulement à cause de la punition, mais aussi à cause de la conscience.
\VS{6}Car c'est aussi pour cela que vous payez les impôts, parce que les magistrats sont les ministres de Dieu, s'employant à rendre la justice.
\VS{7}Rendez donc à tous ce qui leur est dû : L’impôt à qui vous devez l’impôt, le tribut à qui vous devez le tribut, le péage à qui vous devez le péage, la crainte à qui vous devez la crainte, l’honneur à qui vous devez l'honneur.
\TextTitle{L'amour de son prochain : accomplissement de la loi\FTNTT{cp. Lu. 10:29-37}}
\VS{8}Ne devez rien à personne, si ce n’est de vous aimer les uns les autres ; car celui qui aime les autres a accompli la loi.
\VS{9}En effet, les commandements : Tu ne commettras point d’adultère, tu ne tueras point, tu ne déroberas point, tu ne convoiteras point, et ceux qu’il peut encore y avoir, se résument dans cette parole : Tu aimeras ton prochain comme toi-même\FTNT{Ex. 20:12-17 ; Mt. 22:39.}.
\VS{10}La charité ne fait point de mal au prochain ; la charité est donc l'accomplissement de la loi.
\VS{11}Cela importe d’autant plus que vous savez en quelle saison nous sommes ; parce qu'il est déjà l’heure de nous réveiller du sommeil ; car maintenant le salut est plus près de nous que lorsque nous avons cru.
\VS{12}La nuit est avancée\FTNT{Mt. 25:1-13.} et le jour approche. Rejetons donc les œuvres des ténèbres, et soyons revêtus des armes de lumière.
\VS{13}Marchons honnêtement, comme en plein jour, loin des orgies\FTNT{Orgies : Du grec «~komos~». Ce terme désigne la procession nocturne et rituelle, qui avait lieu après un souper, de gens à moitié ivres, à l'esprit folâtre, qui défilaient à travers les rues avec torches et musique en l'honneur de Bacchus ou quelque autre divinité, et chantaient et jouaient devant les maisons de leurs amis, hommes ou femmes. Ce mot est aussi utilisé pour les fêtes et beuveries de nuit qui se terminaient en orgies.} et de l’ivrognerie, de la luxure et de la débauche, des querelles et des jalousies.
\VS{14}Mais revêtez-vous du Seigneur Jésus-Christ, et n'ayez point soin de la chair pour en satisfaire les convoitises.
\TextTitle{[L'attitude du chrétien face aux opinions différentes]
\\(cp. 1 Co. 8:1-10:33}
\Chap{14}
\VerseOne{}Or quant à celui qui est faible dans la foi, recevez-le, et n'ayez point avec lui des discussions sur les opinions.
\VS{2}L'un croit qu'on peut manger de tout, et l'autre, qui est faible, mange des légumes.
\VS{3}Que celui qui mange de tout ne méprise pas celui qui n'en mange point ; et que celui qui n'en mange point, ne juge point celui qui en mange, car Dieu l'a accueilli.
\VS{4}Qui es-tu, toi qui juges le serviteur d'autrui ? S’il se tient ferme ou s'il tombe, c’est à son maître de le juger ; mais il sera affermi, car Dieu est Puissant pour l'affermir.
\VS{5}Tel fait une distinction entre les jours, tel autre les estime tous égaux. Que chacun ait en son esprit une pleine conviction.
\VS{6}Celui qui distingue entre les jours agit ainsi pour le Seigneur. Celui qui mange, c’est pour le Seigneur qu’il mange, car il rend grâces à Dieu ; celui qui ne mange pas, c’est pour le Seigneur qu’il ne mange pas, et il rend grâces à Dieu.
\VS{7}Car nul de nous ne vit pour lui-même, et nul ne meurt pour lui-même.
\VS{8}Car si nous vivons, nous vivons pour le Seigneur ; et si nous mourons, nous mourons pour le Seigneur. Soit donc que nous vivions, soit que nous mourions, nous sommes au Seigneur.
\VS{9}Car c'est pour cela que Christ est mort, qu'il est ressuscité, et qu'il a repris la vie, afin de dominer sur les morts et sur les vivants.
\VS{10}Mais toi, pourquoi juges-tu ton frère ? Ou toi, pourquoi méprises-tu ton frère ? Puisque nous comparaîtrons tous devant le tribunal de Christ.
\VS{11}Car il est écrit : Je suis vivant, dit le Seigneur, tout genou fléchira devant moi, et toute langue donnera gloire à Dieu\FTNT{Es. 45:23 ; Ph. 2:10-11.}.
\VS{12} Ainsi, chacun de nous rendra compte à Dieu pour lui-même.
\TextTitle{Se garder d'être une occasion de chute}
\VS{13}Ne nous jugeons donc plus les uns les autres ; mais pensez plutôt à ne rien faire qui soit pour votre frère une pierre d’achoppement ou une occasion de chute.
\VS{14}Je sais, et je suis persuadé par le Seigneur Jésus, que rien n'est souillé en soi, et qu’une chose n’est souillée que par celui qui la croit souillée.
\VS{15}Mais si ton frère est attristé au sujet d’un aliment, tu ne marches plus selon la charité ; ne détruis point, par ton aliment, celui pour qui Christ est mort.
\VS{16}Que votre privilège ne soit pas un sujet de calomnie.
\VS{17}Car le Royaume de Dieu ne consiste ni dans le manger ni dans le boire, mais dans la justice, la paix et la joie par le Saint-Esprit.
\VS{18}Celui qui sert Christ de cette manière est agréable à Dieu et approuvé des hommes.
\VS{19}Recherchons donc ce qui contribue à la paix et à l’édification mutuelle.
\VS{20}Ne détruis pas l’œuvre de Dieu pour un aliment. Il est vrai que toutes choses sont pures, mais il est mal à l’homme, quand il mange, de devenir une pierre d’achoppement.
\VS{21}Il est bien de ne pas manger de viande, de ne pas boire de vin, et de s’abstenir de ce qui peut être pour ton frère une occasion de chute, de scandale ou de faiblesse.
\VS{22}As-tu la foi ? Garde-la devant Dieu. Heureux est celui qui ne se condamne pas lui-même dans ce qu'il approuve.
\VS{23}Mais celui qui a des doutes au sujet de ce qu’il mange est condamné, parce qu’il n’agit pas avec foi. Tout ce que l’on ne fait pas avec foi est un péché.
\Chap{15}
\VerseOne{}Nous devons, nous qui sommes forts, supporter les infirmités des faibles, et ne pas nous complaire en nous-mêmes.
\VS{2}Que chacun de nous plaise au prochain pour ce qui est bien, en vue de l’édification.
\VS{3}Car même Jésus-Christ n'a pas cherché ce qui lui plaisait, mais, selon qu’il est écrit : Les outrages de ceux qui t’insultent sont tombés sur moi\FTNT{Ps. 69:10.}.
\TextTitle{Les Juifs et gentils rachetés par un même salut}
\VS{4}Or tout ce qui a été écrit autrefois, a été écrit pour notre instruction, afin que par la patience et la consolation que donnent les Ecritures, nous possédions l’espérance.
\VS{5}Que le Dieu de patience et de consolation vous donne d’avoir les mêmes sentiments les uns envers les autres, selon Jésus-Christ,
\VS{6}afin que tous d'un même cœur et d'une même bouche, vous glorifiiez Dieu, qui est le Père de notre Seigneur Jésus-Christ.
\VS{7}C'est pourquoi, accueillez-vous les uns les autres, comme Christ nous a accueillis, pour la gloire de Dieu.
\VS{8}Je dis donc que Jésus-Christ a été Ministre des circoncis, pour prouver la vérité de Dieu, afin de confirmer les promesses faites aux pères,
\VS{9}afin que les gentils glorifient Dieu pour sa miséricorde, selon ce qui est écrit : C’est pourquoi je te louerai parmi les nations, et je chanterai à la gloire de ton Nom\FTNT{Ps. 18:50.}. Et il est dit encore :
\VS{10}Nations, réjouissez-vous avec son peuple\FTNT{De. 32:43.} !
\VS{11}Et encore : Louez le Seigneur, vous toutes les nations, et célébrez-le, vous tous les peuples\FTNT{Ps. 117:1.}. Esaïe dit aussi :
\VS{12}Il sortira d’Isaï un rejeton, qui se lèvera pour régner sur les nations ; les nations espéreront en lui\FTNT{Es. 11:1 ; Es. 11:10.}.
\VS{13}Que le Dieu de l’espérance vous remplisse de toute joie et de toute paix, dans la foi, afin que vous abondiez en espérance par la puissance du Saint-Esprit.
\TextTitle{Paul envisage d'aller à Jérusalem, à Rome et en Espagne}
\VS{14}Pour moi, mes frères, je suis persuadé que vous êtes pleins de bonté, remplis de toute connaissance, et capables de vous exhorter les uns les autres.
\VS{15}Cependant, mes frères, je vous ai écrit en quelque sorte plus librement, comme pour réveiller vos souvenirs, à cause de la grâce que Dieu m’a faite,
\VS{16}d’être ministre de Jésus Christ parmi les gentils ; je m’acquitte du divin service de l'Evangile de Dieu, afin que les gentils lui soient une offrande agréable, étant sanctifiée par le Saint-Esprit.
\VS{17}J'ai donc sujet de me glorifier en Jésus-Christ pour ce qui regarde les choses de Dieu.
\VS{18}Car je n’oserais parler de quoi que ce soit que Christ n’ait opéré par moi, pour amener les gentils à son obéissance, par la parole et par les œuvres,
\VS{19}par la puissance des prodiges et des miracles, par la puissance de l'Esprit de Dieu. Ainsi, depuis Jérusalem et les pays voisins jusqu’en Illyrie, j’ai abondamment répandu l’Evangile de Christ.
\VS{20}M'attachant ainsi avec affection à annoncer l’Evangile là où Christ n’avait point encore été prêché, afin que je ne bâtisse pas sur le fondement qu'un autre a déjà posé. 
\VS{21}Mais selon qu'il est écrit : Ceux à qui il n'a point été annoncé le verront ; et ceux qui n'en avaient point entendu parler l’entendront\FTNT{Es. 52:15.}.
\VS{22}Et c’est ce qui m'a souvent empêché d’aller vous voir.
\VS{23}Mais maintenant, n’ayant plus rien qui me retienne dans ces contrées, et ayant depuis plusieurs années le désir d’aller vers vous,
\VS{24}j’espère vous voir en passant, quand je me rendrai en Espagne, et y être accompagné par vous, après que j’aurai satisfait en partie mon désir de me trouver chez vous.
\VS{25}Maintenant je vais à Jérusalem pour assister les saints.
\VS{26}Car il a semblé bon à ceux de Macédoine et d’Achaïe de s’imposer une contribution pour les pauvres parmi les saints de Jérusalem.
\VS{27}Ils l’ont bien voulu, et ils le leur devaient, car si les gentils ont eu part à leurs avantages spirituels, ils doivent aussi les assister dans les choses temporelles.
\VS{28}Dès que j'aurai achevé cette affaire, et que je leur aurai remis ce fruit, j'irai en Espagne en passant par vos quartiers.
\VS{29}Et je sais qu’en allant vers vous, j’irai avec une pleine bénédiction de l'Evangile de Christ.
\VS{30}Je vous exhorte, mes frères, par notre Seigneur Jésus-Christ, et par la charité de l'Esprit, à combattre avec moi en adressant des prières à Dieu en ma faveur,
\VS{31}afin que je sois délivré des incrédules de Judée, et que mon ministère\FTNT{Ministère : Du grec «~diakonia~», terme qui désigne le service ou le ministère de ceux qui répondent aux besoins des autres. Ce vocable fait aussi allusion à l’office des diacres.} à Jérusalem soit agréable aux saints,
\VS{32}en sorte que, par la volonté de Dieu, j’arrive chez vous avec joie, et que je me repose avec vous.
\VS{33}Que le Dieu de paix soit avec vous tous. Amen !
\Chap{16}
\TextTitle{Salutations personnelles de Paul}
\VerseOne{}Je vous recommande notre sœur Phœbé, qui est diaconesse de l'église de Cenchrées,
\VS{2}afin que vous la receviez selon le Seigneur, comme il faut recevoir les saints, et que vous l'assistiez dans tout ce dont elle aura besoin ; car elle a exercé l'hospitalité à l'égard de plusieurs, et même à mon égard.
\VS{3}Saluez Priscille et Aquilas, mes compagnons d’œuvre en Jésus-Christ,
\VS{4}qui ont exposé leur cou pour ma vie ; ce n’est pas moi seul qui leur rends grâces, mais aussi toutes les églises des gentils.
\VS{5}Saluez aussi l'église qui est dans leur maison. Saluez Epaïnète, mon bien-aimé, qui a été pour Christ les prémices d'Achaïe.
\VS{6}Saluez Marie, qui a beaucoup travaillé pour nous.
\VS{7}Saluez Andronicus et Junias, mes parents, qui ont été prisonniers avec moi, et qui sont distingués parmi les apôtres, et qui ont même été en Christ avant moi.
\VS{8}Saluez Amplias, mon bien-aimé dans le Seigneur.
\VS{9}Saluez Urbain, notre compagnon d’œuvre en Christ, et Stachys, mon bien-aimé.
\VS{10}Saluez Apellès, qui est éprouvé en Christ. Saluez ceux de chez Aristobule.
\VS{11}Saluez Hérodion, mon parent. Saluez ceux de chez Narcisse qui sont dans le Seigneur.
\VS{12}Saluez Tryphène et Tryphose, qui travaillent pour le Seigneur. Saluez Perside, la bien-aimée qui a beaucoup travaillé pour le Seigneur.
\VS{13}Saluez Rufus, l’élu du Seigneur, et sa mère, qui est aussi la mienne.
\VS{14}Saluez Asyncrite, Phlégon, Hermas, Patrobas, Hermès, et les frères qui sont avec eux.
\VS{15}Saluez Philologue et Julie, Nérée et sa sœur, et Olympe, et tous les saints qui sont avec eux.
\VS{16}Saluez-vous les uns les autres par un saint baiser. Les églises de Christ vous saluent.
\TextTitle{Se garder de ceux qui causent des divisions et des scandales}
\VS{17}Je vous exhorte, mes frères, à prendre garde à ceux qui causent des divisions et des scandales contre la doctrine que vous avez apprise. Eloignez-vous d'eux.
\VS{18}Car de tels hommes ne servent point notre Seigneur Jésus-Christ, mais leur propre ventre, et par des paroles douces et flatteuses, ils séduisent les cœurs des simples.
\VS{19}Pour vous, votre obéissance est connue de tous ; je me réjouis donc à votre sujet, et je désire que vous soyez sages à l’égard du bien, et purs à l’égard du mal.
\VS{20}Le Dieu de paix brisera bientôt Satan sous vos pieds. Que la grâce de notre Seigneur Jésus-Christ soit avec vous. Amen !
\VS{21}Timothée, mon compagnon d’œuvre, vous salue, ainsi que Lucius, et Jason et Sosipater, mes parents.
\VS{22}Je vous salue dans le Seigneur, moi Tertius, qui ai écrit cette lettre.
\VS{23}Gaïus, mon hôte, et celui de toute l'église, vous salue. Eraste, l’économe de la ville, vous salue, et Quartus, notre frère.
\TextTitle{Bénédiction}
\VS{24}Que la grâce de notre Seigneur Jésus-Christ soit avec vous tous. Amen !
\VS{25}or à celui qui est puissant pour vous affermir selon mon Evangile, et selon la prédication de Jésus-Christ, conformément à la révélation du mystère qui a été caché dans les temps passés.
\VS{26}mais manifesté maintenant par les écrits des prophètes, d’après l’ordre du Dieu éternel, et porté à la connaissance de toutes les nations, afin qu’elles obéissent à la foi.
\VS{27}A Dieu, seul sage, soit la gloire éternellement, par Jésus-Christ. Amen !
\PPE{}
\end{multicols}

\clearpage\ShortTitle{Ep.}\BookTitle{Ephésiens}\BFont
\noindent\hrulefill
{\footnotesize
\textit{
\bigskip
{\centering{}
\\Auteur~: Paul
\\Thème~: L'Eglise, corps de Christ
\\Date de rédaction~: Env. 60 ap. J.-C.\\}
}
\textit{
\\Ephèse figurait parmi les principales villes de l'Empire romain sous le règne de l'empereur Claude Ier (10 av. J.-C. – 54 ap. J.-C.). Bien que Pergame était considérée comme la capitale de l'Asie Mineure, en raison de sa position géographique et grâce à ses affluents, Ephèse possédait le plus grand port de la région, ce qui lui a valu le contrôle du trafic commercial. Richissime et prospère, elle était renommée pour son faste et sa liberté de parole, et constituait donc un endroit privilégié pour les philosophes. C'était une ville où l'activité culturelle tenait une grande place (Jeux olympiques, théâtres, cirques, etc.) et où chacun pouvait y pratiquer la religion de son choix (croyances gréco-romaines, égyptiennes, judaïque etc.).
\\Ephèse, dont le nom signifie «~désirable~», était la gardienne de l'Artémision, temple dédié à la déesse grecque Artémis, la Diane des Ephésiens.
\\L'église d'Ephèse vit le jour lors du second voyage missionnaire de Paul (50-52). Quand il repartit, il laissa à Aquilas et Priscille la charge de la toute jeune assemblée. Paul s'installa à Ephèse lors de son troisième voyage (53-57) et y demeura presque trois ans. Il discourut pendant trois mois dans la synagogue sur le Royaume de Dieu, mais se retrouva confronté à l'endurcissement de certains. C'est alors qu'il se retira pour enseigner dans l'école d'un certain Tyrannus durant deux ans, de sorte que tous ceux qui habitaient l'Asie, Juifs et Grecs, entendirent parler de Jésus-Christ.
\\Plusieurs de ceux qui avaient cru confessèrent leurs péchés et un certain nombre de ceux qui avaient pratiqué la magie allèrent même jusqu'à brûler leurs livres publiquement. C'est ainsi que l'église d'Ephèse croissait en puissance et en force. La prédication de Paul vint troubler le marché fructueux des fabricants d'idoles au point que Démétrius (orfèvre tirant un grand profit de cette industrie) entraina une émeute contre lui. Paul était cependant soutenu par des amis influents~: Les asiarques.
\\Rédigée en prison, cette épître a pour vocation d'enseigner les chrétiens d'Ephèse sur la manière dont il convient de vivre les uns avec les autres au sein de l'Eglise, corps du Christ.\bigskip
}
}
\par\nobreak\noindent\hrulefill
\begin{multicols}{2}
\Chap{1}
\TextTitle{Introduction}
\VerseOne{}Paul, apôtre de Jésus-Christ par la volonté de Dieu, aux saints et fidèles en Jésus-Christ qui sont à Ephèse~:
\VS{2}Que la grâce et la paix vous soient données par Dieu notre Père, et par le Seigneur Jésus-Christ~!
\TextTitle{La position des élus dans le Royaume de Dieu}
\VS{3}Béni soit Dieu, qui est le Père de notre Seigneur Jésus-Christ, qui nous a bénis de toutes bénédictions spirituelles dans les lieux célestes en Christ~!
\VS{4}Selon qu'il nous a élus en lui avant la fondation du monde, afin que nous soyons saints et irrépréhensibles devant lui dans la charité, 
\VS{5}nous ayant prédestinés pour nous adopter pour lui par Jésus-Christ, selon le bon plaisir de sa volonté,
\VS{6}à la louange de la gloire de sa grâce, par laquelle il nous a rendus agréables en son bien-aimé.
\VS{7}En lui nous avons la rédemption par son sang, à savoir la rémission des offenses, selon les richesses de sa grâce,
\VS{8}qu'il a fait abonder sur nous en toute sagesse et intelligence,
\VS{9}nous ayant donné à connaître le mystère de sa volonté, qu'il avait premièrement arrêté en lui-même,
\VS{10}afin que dans l'accomplissement des temps qu'il avait réglés, il réunit tout en Christ, tant ce qui est dans les cieux, que ce qui est sur la terre, en lui-même. 
\VS{11}En qui nous sommes aussi devenus héritiers, ayant été prédestinés, suivant la résolution de celui qui accomplit toutes choses avec efficacité selon le conseil de sa volonté,
\VS{12}afin que nous soyons à la louange de sa gloire, nous qui avons les premiers espéré en Christ.
\VS{13}En qui vous êtes aussi, ayant entendu la parole de la vérité, qui est l'Evangile de votre salut, et auquel ayant cru, vous avez été scellés du Saint-Esprit qui avait été promis,
\VS{14}lequel est le gage de notre héritage jusqu'à la rédemption de ceux qu'il s'est acquis à la louange de sa gloire.
\VS{15}C'est pourquoi, ayant aussi entendu parler de la foi que vous avez en notre Seigneur Jésus, et de la charité que vous avez envers tous les saints,
\VS{16}je ne cesse de rendre grâces pour vous dans mes prières,
\VS{17}afin que le Dieu de notre Seigneur Jésus-Christ, le Père de gloire, vous donne l'Esprit de sagesse et de révélation, dans ce qui regarde sa connaissance.
\VS{18}Qu'il illumine les yeux de votre esprit, afin que vous sachiez quelle est l'espérance de sa vocation, et quelles sont les richesses de la gloire de son héritage qu'il réserve aux saints,
\VS{19}et quelle est l'excellente grandeur de sa puissance envers nous qui croyons selon l'efficacité de la puissance de sa force, 
\VS{20}qu'il a déployée avec efficacité en Christ, quand il l'a ressuscité des morts et qu'il l'a fait asseoir à sa droite dans les lieux célestes,
\VS{21}au-dessus de toute principauté, de toute puissance, de toute dignité et de toute domination, et au-dessus de tout nom qui se nomme, non seulement dans le siècle présent, mais aussi dans celui qui est à venir.
\TextTitle{Le Messie est le Chef suprême de l'Eglise}
\VS{22}Et il a assujetti toutes choses sous ses pieds, et l'a établi sur toutes choses pour être le Chef de l'Eglise,
\VS{23}qui est son corps, et la plénitude de celui qui remplit tout en tous.
\Chap{2}
\TextTitle{Le salut par la grâce}
\VerseOne{}Et vous étiez morts par vos offenses et par vos péchés,
\VS{2}dans lesquels vous marchiez autrefois, suivant le train de ce monde, selon le prince de la puissance de l'air, qui est l'esprit qui agit maintenant avec efficacité dans les fils rebelles à Dieu,
\VS{3}parmi lesquels nous vivions tous autrefois, selon les convoitises de notre chair, accomplissant les désirs de la chair et de nos pensées. Et nous étions par nature des enfants de colère comme les autres.
\VS{4}Mais Dieu, qui est riche en miséricorde, à cause de sa grande charité dont il nous a aimés,
\VS{5}lorsque nous étions morts dans nos offenses, il nous a vivifiés ensemble avec Christ~; c'est par grâce que vous êtes sauvés.
\VS{6}Et il nous a ressuscités ensemble, et nous a fait asseoir ensemble dans les lieux célestes en Jésus-Christ,
\VS{7}afin qu'il montre dans les siècles à venir les immenses richesses de sa grâce par sa bonté envers nous, en Jésus-Christ.
\VS{8}Car vous êtes sauvés par la grâce, par la foi~; et cela ne vient pas de vous, c'est le don de Dieu~;
\VS{9}non pas par les œuvres, afin que personne ne se glorifie.
\VS{10}Car nous sommes son ouvrage, ayant été créés en Jésus-Christ pour les bonnes œuvres que Dieu a préparées d'avance, afin que nous marchions en elles.
\VS{11}C'est pourquoi, souvenez-vous que vous qui étiez autrefois Gentils dans la chair, et qui étiez appelés incirconcis par ceux qu'on appelle circoncis, et qui le sont dans la chair par la main des hommes, 
\VS{12}vous étiez en ce temps-là sans Christ, privés du droit de cité en Israël, étant étrangers des alliances de la promesse, n'ayant pas d'espérance, et étant sans Dieu dans le monde.
\VS{13}Mais maintenant, par Jésus-Christ, vous qui étiez autrefois éloignés, vous avez été rapprochés par le sang de Christ.
\TextTitle{Juifs et Gentils forment un seul corps}
\VS{14}Car il est notre paix, lui qui des deux n'en a fait qu'un en détruisant le mur de séparation,
\VS{15}ayant aboli dans sa chair l'inimitié, à savoir la loi des commandements qui consiste en ordonnances, afin de créer les deux en lui-même pour être un homme nouveau, en faisant la paix~;
\VS{16}et de réconcilier les uns et les autres avec Dieu pour former un seul corps par sa croix, ayant détruit par elle l'inimitié.
\VS{17}Et il est venu prêcher la paix à vous qui étiez loin, et à ceux qui étaient près,
\VS{18}car nous avons par lui les uns et les autres accès auprès du Père dans un même Esprit.
\TextTitle{L'Eglise véritable}
\VS{19}C'est pourquoi vous n'êtes plus des étrangers ni des gens de dehors, mais concitoyens des saints et gens de la maison de Dieu~;
\VS{20}étant édifiés sur le fondement\FTNT{Le fondement a été posé une fois pour toutes par les apôtres et les prophètes. Et ce fondement est notre Seigneur Jésus-Christ (1 Co. 3:11).} des apôtres et des prophètes, et Jésus-Christ lui-même étant la pierre angulaire~;
\VS{21}en qui tout l'édifice, bien ajusté ensemble, s'élève pour être un temple saint dans le Seigneur,
\VS{22}en qui vous êtes édifiés ensemble, pour être une habitation de Dieu en Esprit.
\Chap{3}
\TextTitle{Le mystère caché de tout temps\FTNTT{Col. 1:24-27.}}
\VerseOne{}C'est pour cela que moi, Paul, je suis prisonnier de Jésus-Christ pour vous Gentils.
\VS{2}Si toutefois vous avez entendu quelle est la gestion de la grâce de Dieu qui m'a été donnée pour vous,
\VS{3}comment par révélation ce mystère m'a été manifesté, ainsi que je l'ai écrit ci-dessus en peu de mots~;
\VS{4}d'où vous pouvez voir en le lisant, quelle est l'intelligence que j'ai du mystère de Christ,
\VS{5}lequel n'a pas été manifesté aux fils des hommes dans les autres générations, comme il a été révélé maintenant par l'Esprit à ses saints apôtres et à ses prophètes,
\VS{6}à savoir que les Gentils sont cohéritiers et d'un même corps, et qu'ils participent ensemble à sa promesse en Christ par l'Evangile,
\VS{7}dont j'ai été fait serviteur, selon le don de la grâce de Dieu qui m'a été donnée selon l'efficacité de sa puissance.
\VS{8}Cette grâce, dis-je, m'a été donnée à moi, qui suis le moindre de tous les saints, pour annoncer parmi les Gentils les richesses incompréhensibles de Christ,
\VS{9}et pour mettre en évidence devant tous quelle est la communion qui nous a été accordée du mystère qui était caché de tout temps en Dieu, lequel a créé toutes choses par Jésus-Christ,
\VS{10}afin que les principautés et les puissances dans les lieux célestes connaissent aujourd'hui par l'Eglise la sagesse infiniment variée de Dieu,
\VS{11}suivant le dessein arrêté dès les siècles, qu'il a établi en Jésus-Christ, notre Seigneur,
\VS{12}par lequel nous avons hardiesse et accès avec confiance, par la foi que nous avons en lui.
\VS{13}C'est pourquoi, je vous prie de ne pas vous relâcher à cause de mes afflictions que je souffre pour l'amour de vous, ce qui est votre gloire.
\VS{14}A cause de cela, je fléchis mes genoux devant le Père de notre Seigneur Jésus-Christ,
\VS{15}duquel toute parenté est nommée dans les cieux et sur la terre,
\VS{16}afin que selon les richesses de sa gloire, il vous donne d'être puissamment fortifiés par son Esprit dans l'homme intérieur,
\VS{17}en sorte que Christ habite dans vos cœurs par la foi~; afin qu'étant enracinés et fondés dans la charité,
\VS{18}vous puissiez comprendre avec tous les saints quelle est la largeur et la longueur, la profondeur et la hauteur,
\VS{19}et connaître la charité de Christ qui surpasse toute connaissance, afin que vous soyez remplis de toute la plénitude de Dieu.
\VS{20}Or à celui qui par la puissance qui agit en nous avec efficacité, peut faire infiniment au-delà de tout ce que nous demandons et pensons,
\VS{21}à lui soit la gloire dans l'Eglise, en Jésus-Christ, dans toutes les générations, aux siècles des siècles~! Amen~!
\Chap{4}
\TextTitle{L'unité}
\VerseOne{}Je vous prie donc, moi, le prisonnier dans le Seigneur, à marcher d'une manière digne de la vocation à laquelle vous êtes appelés,
\VS{2}avec toute humilité et douceur, avec patience, vous supportant les uns les autres dans la charité,
\VS{3}vous efforçant de garder l'unité de l'Esprit par le lien de la paix.
\VS{4}Il y a un seul corps, un seul Esprit, comme aussi vous êtes appelés à une seule espérance par votre vocation~;
\VS{5}il y a un seul Seigneur, une seule foi, un seul baptême,
\VS{6}un seul Dieu et Père de tous, qui est au-dessus de tous, parmi tous, et en vous tous.
\TextTitle{Les dons de Christ pour le perfectionnement et l'édification de son Corps\FTNTT{1 Co. 12:4-11.}}
\VS{7}Mais la grâce est donnée à chacun de nous selon la mesure du don de Christ.
\VS{8}C'est pourquoi il est dit~: Etant monté en haut, il a emmené captive une grande multitude de captifs, et il a donné des dons aux hommes\FTNT{Ps. 68:19.}.
\VS{9}Or que signifie~: Il est monté, sinon qu'il est premièrement descendu dans les parties les plus basses de la terre~?
\VS{10}Celui qui est descendu, c'est le même qui est monté au-dessus de tous les cieux, afin de remplir toutes choses.
\VS{11}Lui-même donc a donné les uns pour être apôtres, les autres pour être prophètes, les autres pour être évangélistes, les autres pour être pasteurs et docteurs,
\VS{12}pour travailler au perfectionnement\FTNT{Le mot «~perfectionnement~» vient du grec «~katartismos~», qui tire son origine du terme «~katartizo~»~: «~redresser, ajuster, compléter, raccommoder (ce qui a été abîmé), réparer~». Ainsi, les divers services ont vocation, d'une part, à réparer les dégâts causés par le péché dans les âmes, et d'autre part, à préparer les disciples à rentrer à leur tour dans leur propre service.} des saints, pour l'œuvre du service\FTNT{Le mot «~service~» vient du grec «~diakonos~», il signifie «~service, ministère de ceux qui répondent aux besoins des autres~». Jésus-Christ lui-même a pris la forme d'un serviteur pour nous servir et non pour être servi (Mt. 20:28).}, pour l'édification du corps de Christ,
\VS{13}jusqu'à ce que nous soyons tous parvenus à l'unité de la foi et de la connaissance du Fils de Dieu, à l'état d'homme parfait, à la mesure de la parfaite stature de Christ,
\VS{14}afin que nous ne soyons plus des enfants flottants et emportés çà et là à tous vents de doctrine, par la tromperie des hommes et par leur ruse à séduire artificieusement.
\VS{15}Mais afin que, suivant la vérité avec la charité, nous croissions en toutes choses en celui qui est le Chef, c'est-à-dire Christ,
\VS{16}dont tout le corps bien ajusté et lié ensemble par toutes les jointures de son assistance, tire son accroissement selon la force qu'il distribue à chaque membre, afin qu'il soit édifié dans la charité.
\TextTitle{Se dépouiller du vieil homme}
\VS{17}Je vous dis donc, et je vous conjure de la part du Seigneur, de ne plus vous conduire comme le reste des Gentils qui suivent la vanité de leurs pensées.
\VS{18}Ils ont l'intelligence obscurcie par les ténèbres et sont étrangers à la vie de Dieu, à cause de l'ignorance qui est en eux, par l'endurcissement de leur cœur.
\VS{19}Ils ont perdu tout sentiment, et se sont abandonnés à la dissolution pour commettre toute sorte d'impureté avec cupidité.
\VS{20}Mais vous n'avez pas ainsi appris Christ,
\VS{21}si toutefois vous l'avez entendu, et si vous avez été enseignés par lui~, selon que la vérité est en Jésus~; 
\VS{22}à savoir que vous dépouilliez le vieil homme, pour ce qui est de votre conduite précédente, qui se corrompt par les convoitises qui séduisent~; 
\VS{23}et que vous soyez renouvelés dans l'esprit de votre entendement,
\VS{24}et que vous soyez revêtus du nouvel homme, créé selon Dieu dans une justice et une sainteté véritables.
\VS{25}C'est pourquoi, ayant dépouillé le mensonge, parlez en vérité chacun avec son prochain~; car nous sommes membres les uns des autres.
\VS{26}Si vous vous mettez en colère, ne péchez pas, que le soleil ne se couche pas sur votre colère.
\VS{27}Ne donnez pas lieu au diable de vous perdre.
\VS{28}Que celui qui dérobait ne dérobe plus~; mais plutôt qu'il travaille en faisant de ses mains ce qui est bon, pour avoir de quoi donner à celui qui est dans le besoin.
\VS{29}Qu'aucun discours malhonnête ne sorte de votre bouche, mais seulement celui qui est propre à édifier, afin qu'il soit agréable à ceux qui l'écoutent.
\VS{30}Et n'attristez pas le Saint-Esprit de Dieu, par lequel vous avez été scellés pour le jour de la rédemption.
\VS{31}Que toute amertume, toute colère, toute irritation, toute clameur, toute médisance, et toute malice soient bannies du milieu de vous.
\VS{32}Mais soyez doux les uns envers les autres, pleins de compassion, et vous pardonnant les uns aux autres, ainsi que Dieu vous a pardonné par Christ.
\Chap{5}
\VerseOne{}Soyez donc les imitateurs de Dieu, comme ses enfants bien-aimés~;
\VS{2}et marchez dans la charité, ainsi que Christ nous a aimés et s'est livré lui-même pour nous comme une offrande et un sacrifice de bonne odeur à Dieu.
\VS{3}Que la fornication, ni aucune impureté, ni la cupidité, ne soient pas même nommées parmi vous, ainsi qu'il est convenable à des saints.
\VS{4}Qu'on n'entende ni parole grossière, ni propos insensés, ni plaisanterie, choses qui sont contraires à la bienséance, mais plutôt des actions de grâces.
\VS{5}Car sachez-le bien qu'aucun fornicateur, ni impur, ni cupide, qui est un idolâtre, n'a d'héritage dans le Royaume de Christ et de Dieu.
\VS{6}Que personne ne vous séduise par de vains discours~; car à cause de ces choses la colère de Dieu vient sur les fils de la rébellion.
\VS{7}Ne soyez donc pas leurs associés.
\VS{8}Car vous étiez autrefois ténèbres, mais maintenant vous êtes lumière dans le Seigneur. Conduisez-vous donc comme des enfants de la lumière~!
\VS{9}car le fruit de l'Esprit consiste en toute bonté, justice et vérité,
\VS{10}éprouvant ce qui est agréable au Seigneur~;
\VS{11}et ne participez pas aux œuvres infructueuses des ténèbres, mais au contraire condamnez-les~!
\VS{12}Car il est honteux de dire les choses qu'ils font en secret~;
\VS{13}mais toutes choses, étant mises en évidence par la lumière, sont rendues manifestes, car la lumière est celle qui manifeste tout.
\VS{14}C'est pourquoi il est dit~: Réveille-toi, toi qui dors, et relève-toi d'entre les morts, et Christ t'éclairera\FTNT{Es. 60:1.}.
\VS{15}Prenez donc garde de vous conduire soigneusement, non pas comme étant dépourvus de sagesse, mais comme étant sages,
\VS{16}rachetant le temps, car les jours sont mauvais.
\VS{17}C'est pourquoi ne soyez pas sans intelligence, mais comprenez bien quelle est la volonté du Seigneur.
\VS{18}Et ne vous enivrez pas du vin dans lequel il y a de la dissolution, mais soyez remplis de l'Esprit.
\VS{19}Entretenez-vous par des psaumes, des hymnes et des cantiques spirituels, chantant et psalmodiant de votre cœur au Seigneur~;
\VS{20}rendez toujours grâces pour toutes choses à Dieu notre Père, au Nom de notre Seigneur Jésus-Christ~;
\VS{21}soumettez-vous les uns aux autres dans la crainte de Christ.
\TextTitle{Le mariage selon Dieu}
\VS{22}Femmes, soyez soumises à vos maris comme au Seigneur~;
\VS{23}car le mari est le chef de la femme, comme Christ est le Chef de l'Eglise, qui est son corps, et dont il est le Sauveur.
\VS{24}Or de même que l'Eglise est soumise à Christ, les femmes aussi doivent l'être à leurs maris en toutes choses.
\VS{25}Et vous maris, aimez vos femmes, comme Christ a aimé l'Eglise, et s'est livré lui-même pour elle,
\VS{26}afin de la sanctifier en la purifiant et en la lavant par l'eau de la parole~;
\VS{27}afin de faire paraître devant lui cette Eglise glorieuse, sans tache, ni ride, ni rien de semblable, mais sainte et irréprochable.
\VS{28}C'est ainsi que les maris doivent aimer leurs femmes comme leurs propres corps. Celui qui aime sa femme s'aime lui-même,
\VS{29}car personne n'a jamais eu en haine sa propre chair, mais il la nourrit et l'entretient, comme le Seigneur entretient l'Eglise,
\VS{30}car nous sommes membres de son corps étant de sa chair et de ses os.
\VS{31}C'est pourquoi l'homme quittera son père et sa mère et s'attachera à sa femme, et les deux deviendront une seule chair.
\VS{32}Ce mystère est grand, or je parle de Christ et de l'Eglise.
\VS{33}Que chacun de vous donc aime sa femme comme lui-même, et que la femme respecte son mari.
\Chap{6}
\TextTitle{La famille selon Dieu}
\VerseOne{}Enfants, obéissez à vos pères et à vos mères, dans ce qui est selon le Seigneur, car cela est juste.
\VS{2}Honore ton père et ta mère, c'est le premier commandement avec une promesse,
\VS{3}afin que tout aille bien pour toi et que tu vives longtemps sur la terre.
\VS{4}Et vous, pères, n'irritez pas vos enfants, mais élevez-les\FTNT{Le verbe «~élever~» vient du grec «~ektrepho~» qui signifie nourrir jusqu'à maturité.} en les instruisant et les avertissant selon le Seigneur.
\TextTitle{Les rapports entre les maîtres et les serviteurs selon Dieu}
\VS{5}Serviteurs, obéissez à vos maîtres selon la chair avec crainte et tremblement, dans la simplicité de votre cœur, comme à Christ,
\VS{6}ne les servant pas seulement sous leurs yeux, comme cherchant à plaire aux hommes, mais comme serviteurs de Christ, faisant de bon cœur la volonté de Dieu,
\VS{7}servant avec bienveillance, comme servant le Seigneur et non pas les hommes~;
\VS{8}sachant que chacun, soit esclave, soit libre, recevra du Seigneur le bien qu'il aura fait.
\VS{9}Et vous maîtres, faites envers eux la même chose et renoncez aux menaces, sachant que leur Seigneur et le vôtre est dans les cieux, et qu'il n'y a pas en lui acception de personnes.
\TextTitle{Le combat spirituel}
\VS{10}Au reste, mes frères, fortifiez-vous dans le Seigneur, et dans la puissance de sa force.
\VS{11}Revêtez-vous de toutes les armes de Dieu, afin de pouvoir résister aux embûches du diable.
\VS{12}Car nous n'avons pas à lutter\FTNT{Le mot grec utilisé ici est «~pale~», il était employé pour parler de la lutte entre deux combattants où chacun essaye de renverser l’autre~; la victoire étant acquise par le maintien de l’adversaire au sol, et en lui mettant la main sur la nuque.} contre la chair et le sang, mais contre les principautés, contre les puissances, contre les seigneurs du monde des ténèbres de ce siècle, contre les méchancetés spirituelles qui sont dans les lieux célestes.
\VS{13}C'est pourquoi prenez toutes les armes de Dieu\FTNT{Voir en annexe «~Les armes du chrétien~».}, afin de pouvoir résister dans le mauvais jour, et tenir ferme après avoir tout surmonté.
\VS{14}Soyez donc fermes, ayant à vos reins la vérité pour ceinture, ayant revêtu la cuirasse de la justice~;
\VS{15}et ayant vos pieds chaussés, prêts pour l'Evangile de paix~;
\VS{16}par-dessus tout, prenez le bouclier de la foi, avec lequel vous pourrez éteindre tous les dards enflammés du malin~;
\VS{17}prenez aussi le casque du salut, et l'épée de l'Esprit, qui est la parole de Dieu~;
\VS{18}priant en votre esprit par toutes sortes de prières et de supplications en tout temps, veillant à cela avec une entière persévérance, et priant pour tous les saints,
\VS{19}et pour moi aussi, afin qu'il me soit donné de parler en toute liberté et avec hardiesse, pour faire connaître le mystère de l'Evangile,
\VS{20}pour lequel je suis ambassadeur quoique chargé de chaînes, afin, dis-je, que je parle librement, ainsi qu'il faut que je parle.
\TextTitle{Salutations}
\VS{21}Or afin que vous aussi vous sachiez ce qui me concerne et ce que je fais, Tychique, notre frère bien-aimé et fidèle serviteur du Seigneur, vous fera tout savoir.
\VS{22}Je l'envoie exprès vers vous, afin que vous connaissiez notre situation, et pour qu'il console vos cœurs.
\VS{23}Que la paix soit avec les frères, et la charité avec la foi, de la part de Dieu le Père, et du Seigneur Jésus-Christ~!
\VS{24}Que la grâce soit avec tous ceux qui aiment notre Seigneur Jésus-Christ dans l'incorruptibilité~! Amen~!
\PPE{}
\end{multicols}

\clearpage\ShortTitle{Ph.}\BookTitle{Philippiens}\BFont
\noindent\hrulefill
{\footnotesize
\textit{
\bigskip
{\centering{}
\\Auteur~: Paul
\\Thème~: Expérience chrétienne
\\Date de rédaction~: Env. 60 ap. J.-C.\\}
}
\textit{
\\Fondée par Philippe II (382 av. J.-C. – 336 av. J.-C.) en 356 av. J.-C., Philippes est une ville grecque de Macédoine orientale. Située sur une voie romaine qui traversait les Balkans (Via Egnatia), elle est restée de taille modeste en dépit de son fort taux de fréquentation.
\\La première mention de l'assemblée de Philippes se trouve dans Actes 16, lors de la rencontre de Paul avec des femmes réunies à l'extérieur de la ville pour la prière. Au travers des paroles de Paul, le Seigneur toucha particulièrement Lydie qui, après avoir été baptisée avec sa famille, reçut Paul et ses compagnons dans sa maison.
\\C'est à Rome, sous le règne de Néron (37-68), que Paul, alors captif, rédigea cette lettre. Cet écrit de l'apôtre accusait réception d'un don monétaire que l'église de Philippes lui avait fait parvenir par le biais d'Epaphrodite. Paul y exprimait sa joie en dépit des souffrances et invitait les Philippiens à faire de même. Loin des erreurs doctrinales reprochées à d'autres, ces chrétiens recevaient ainsi l'expression de l'affection de Paul et ses encouragements à persévérer dans la foi en Christ en toutes circonstances.\bigskip
}
}
\par\nobreak\noindent\hrulefill
\begin{multicols}{2}
\Chap{1}
\TextTitle{Introduction}
\VerseOne{}Paul et Timothée, serviteurs de Jésus-Christ, à tous les saints en Jésus-Christ qui sont à Philippes, avec les évêques et les diacres~:
\VS{2}Que la grâce et la paix vous soient données de la part de Dieu, notre Père et du Seigneur Jésus-Christ~!
\VS{3}Je rends grâces à mon Dieu toutes les fois que je fais mention de vous,
\VS{4}en priant toujours pour vous tous avec joie dans toutes mes prières,
\VS{5}à cause de votre attachement à l'Evangile, depuis le premier jour jusqu'à maintenant.
\VS{6}Etant persuadé de cela même, que celui qui a commencé cette bonne œuvre en vous, l'achèvera jusqu'au jour de Jésus-Christ.
\VS{7}Comme il est juste que je pense ainsi de vous tous, parce que je retiens dans mon cœur, que vous avez tous été participants de la grâce avec moi dans mes liens, et dans la défense et la confirmation de l'Evangile.
\VS{8}Car Dieu m'est témoin que je vous aime tous tendrement, conformément à la charité de Jésus-Christ.
\VS{9}Et je lui demande cette grâce~: Que votre charité abonde encore de plus en plus avec connaissance et toute intelligence,
\VS{10}pour le discernement des choses contraires, afin que vous soyez purs et irréprochables pour le jour de Christ,
\VS{11}étant remplis de fruits de justice, qui sont par Jésus-Christ, à la gloire et à la louange de Dieu.
\TextTitle{Les chrétiens encouragés par la souffrance de Paul}
\VS{12}Or, mes frères, je veux bien que vous sachiez que les choses qui me sont arrivées, sont arrivées pour un plus grand avancement de l'Evangile.
\VS{13}De sorte que mes liens en Christ ont été rendus célèbres dans tout le Prétoire, et partout ailleurs. 
\VS{14}Et que plusieurs de nos frères en notre Seigneur, étant rassurés par mes liens, osent annoncer la parole plus hardiment, et sans crainte. 
\VS{15}Il est vrai que quelques-uns prêchent Christ par envie et par un esprit de dispute~; et que les autres le font, au contraire, par une bonne volonté. 
\VS{16}Les uns, dis-je, annoncent Christ par un esprit de dispute, et non pas purement, croyant ajouter de l'affliction à mes liens.
\VS{17}Mais les autres le font par charité, sachant que je suis établi pour la défense de l'Evangile\FTNT{Les versets 16 et 17 sont inversés dans les versions Segond, Darby et TOB notamment. Ces bibles sont basées sur les textes minoritaires, moins précis. Les versions Martin, Ostervald et King James, basées sur le texte majoritaire (byzantin), utilisent bien cet ordre des versets que l'on retrouvent ainsi dans les écrits grecs.}.
\VS{18}Quoi donc~? Toutefois, de toute manière, que ce soit par ostentation, ou par amour de la vérité, Christ n'est pas moins annoncé. Je m'en réjouis, et je m'en réjouirai encore.
\VS{19}Car je sais que cela tournera à mon salut par vos prières et par le secours de l'Esprit de Jésus-Christ,
\VS{20}selon ma ferme attente et mon espérance, je ne serai confus en rien, mais qu'en toute assurance, Christ sera maintenant, comme il l'a toujours été glorifié dans mon corps, soit par ma vie, soit par ma mort.
\VS{21}Car Christ est ma vie, et la mort m'est un gain.
\VS{22}Mais s'il est utile pour mon œuvre de vivre dans la chair, ce que je dois choisir, je n'en sais rien.
\VS{23}Car je suis pressé des deux côtés~: Mon désir tendant bien à déloger, et à être avec Christ, ce qui me serait beaucoup meilleur. 
\VS{24}Mais il est plus nécessaire pour vous que je demeure dans la chair.
\VS{25}Et je suis persuadé, je sais que je demeurerai et que je resterai avec vous tous, pour votre avancement et pour votre joie dans la foi.
\VS{26}Afin que vous ayez en moi un sujet de vous glorifier de plus en plus en Jésus-Christ, par mon retour au milieu de vous. 
\VS{27}Seulement, conduisez-vous dignement comme il est séant selon l'Evangile de Christ~; afin que, soit que je vienne, et que je vous voie~; soit que je sois absent, j'entende quant à votre état, que vous persistez dans un même esprit, combattant ensemble d'un même courage par la foi de l'Evangile, et n'étant en rien épouvantés par les adversaires.
\VS{28}Ce qui est pour eux une preuve de perdition, mais pour vous de salut~; et cela de la part de Dieu.
\TextTitle{Souffrir pour Christ: Une grâce}
\VS{29}Parce qu'il vous a été gratuitement donné dans ce qui a du rapport à Christ, non seulement de croire en lui, mais aussi de souffrir pour lui,
\VS{30}en soutenant le même combat que vous m'avez vu soutenir, et que vous apprenez maintenant que je soutiens encore.
\Chap{2}
\TextTitle{Exhortation à l'unité}
\VerseOne{}Si donc il y a quelque consolation en Christ, s'il y a quelque soulagement dans la charité, s'il y a quelque communion d'esprit, s'il y a quelques cordiales affections et quelques compassions,
\VS{2}rendez ma joie parfaite, ayant un même sentiment, un même amour, une même âme, et consentant tous à une même chose.
\VS{3}Ne faites rien par esprit de parti\FTNT{Parti~: «~eritheia~» en grec. Avant la Nouvelle Alliance, ce mot ne se trouve que dans les écrits d'Aristote (philosophe grec, disciple de Platon, né en 384 et mort en 322 av. J.-C) où il dénote une «~recherche personnelle, la poursuite d'une fonction politique par des moyens injustes~». Ce mot signifie aussi «~faire une campagne électorale~» ou «~intriguer pour une fonction dans un esprit partisan, querelleur~». On retrouve le mot grec dans Ja. 3:14,16~; Ga. 5:20~; Ro. 2:8~; Ph. 1:16}, ou par vaine gloire~; mais que l'humilité de cœur vous fasse regarder les autres comme étant au-dessus de vous-mêmes.
\VS{4}Ne regardez point chacun, à votre intérêt particulier, mais que chacun ait égard aussi à ce qui concerne les autres.
\TextTitle{L'humilité de Christ}
\VS{5}Qu'il y ait donc en vous un même sentiment qui a été en Jésus-Christ. 
\VS{6}Lequel étant en forme de Dieu, n'a point regardé son égalité avec Dieu comme une usurpation.
\VS{7}Cependant il s'est vidé lui-même, ayant pris la forme de serviteur, fait à la ressemblance des hommes.
\VS{8}Et, étant trouvé en apparence comme un homme, il s'est abaissé lui-même, en se rendant obéissant jusqu'à la mort, même jusqu'à la mort de la croix. 
\VS{9}C'est pourquoi aussi Dieu l'a souverainement élevé, et lui a donné le Nom qui est au-dessus de tout nom~;
\VS{10}afin qu'au Nom de Jésus, tout genou fléchisse, tant de ceux qui sont dans les cieux, que de ceux qui sont sur la terre, et sous la terre,
\VS{11}et que toute langue confesse que Jésus-Christ est le Seigneur, à la gloire de Dieu le Père.
\VS{12}C'est pourquoi, mes bien-aimés, comme vous avez toujours obéi, mettez en œuvre votre propre salut avec crainte et tremblement, non seulement comme en ma présence, mais beaucoup plus maintenant que je suis absent.
\VS{13}Car c'est Dieu qui produit en vous avec efficacité le vouloir et le faire, selon son bon plaisir.
\VS{14}Faites toutes choses sans murmures et sans disputes,
\VS{15}afin que vous soyez sans reproche, et purs, des enfants de Dieu, irrépréhensibles au milieu de la génération corrompue et perverse, parmi lesquels vous brillez comme des flambeaux dans le monde, qui portent au devant d'eux la parole de la vie. 
\VS{16}Pour me glorifier au jour de Christ de n'avoir point couru en vain, ni travaillé en vain. 
\VS{17}Et même si je sers de libation sur le sacrifice et sur le service de votre foi, je m'en réjouis, et je m'en réjouis avec vous tous.
\VS{18}Vous aussi pareillement, réjouissez-vous, et réjouissez-vous avec moi.
\TextTitle{Paul témoigne de Timothée et d'Epaphrodite}
\VS{19}Or j'espère avec la grâce du Seigneur Jésus, vous envoyer bientôt Timothée afin que j'aie aussi plus de courage quand j'aurai connu votre état. 
\VS{20}Car je n'ai personne d'un pareil courage, et qui soit vraiment soigneux de ce qui vous concerne,
\VS{21}parce que tous cherchent leur intérêt particulier, et non les intérêts de Jésus-Christ. 
\VS{22}Mais vous savez l'épreuve que j'ai faite de lui, puisqu'il a servi avec moi en l'Evangile, comme l'enfant sert son père.
\VS{23}J'espère donc vous l'envoyer dès que j'aurai pourvu à mes affaires.
\VS{24}Et j'ai cette confiance en notre Seigneur que moi-même aussi j'irai bientôt.
\VS{25}Mais j'ai cru nécessaire de vous envoyer Epaphrodite, mon frère, mon compagnon d'œuvre et mon compagnon d'armes, par qui vous m'aviez envoyé de quoi pourvoir à mes besoins.
\VS{26}Car aussi il désirait ardemment vous voir tous, et il était fort affligé de ce que vous aviez appris qu'il avait été malade.
\VS{27}En effet, il a été malade et tout près de la mort~; mais Dieu a eu pitié de lui, et non seulement de lui, mais aussi de moi, afin que je n'aie pas tristesse sur tristesse.
\VS{28}Je l'ai donc envoyé à cause de cela avec plus de soin, afin qu'en le revoyant vous ayez de la joie, et que j'aie moins de tristesse.
\VS{29}Recevez-le donc en notre Seigneur, avec toute sorte de joie~; et ayez de l'estime pour ceux qui sont tels que lui. 
\VS{30}Car il a été proche de la mort pour l'œuvre de Christ, n'ayant eu aucun égard à sa propre vie, afin de suppléer au défaut de votre service envers moi.
\Chap{3}
\TextTitle{Le légalisme et la justice de la loi mosaïque}
\VerseOne{}Au reste, mes frères, réjouissez-vous dans le Seigneur. Je ne me lasse point de vous écrire les mêmes choses, mais pour vous c'est une sécurité.
\VS{2}Prenez garde aux chiens~; prenez garde aux mauvais ouvriers~; prenez garde aux faux circoncis.
\VS{3}Car c'est nous qui sommes les circoncis, qui rendons à Dieu notre culte en Esprit, et qui nous glorifions en Jésus-Christ, et qui n'avons point de confiance en la chair.
\VS{4}Moi aussi, cependant, j'aurais sujet de mettre ma confiance en la chair. Si quelqu'un estime qu'il a de quoi se confier en la chair, je le puis bien davantage~:
\VS{5}Moi, circoncis le huitième jour, de la race d'Israël, de la tribu de Benjamin, Hébreu né d'Hébreux, pharisien en ce qui concerne la loi~;
\VS{6}quant au zèle, persécutant l'Eglise~; et quant à la justice à l'égard de la loi, étant sans reproche.
\TextTitle{Le Messie, objet de notre foi} 
\VS{7}Mais ces choses qui étaient pour moi un gain, je les ai regardées comme une perte à cause de l'amour de Christ.
\VS{8}Et certes, je regarde toutes les autres choses comme m'étant nuisibles en comparaison de l'excellence de la connaissance de Jésus-Christ, mon Seigneur, pour l'amour duquel je me suis privé de toutes ces choses, et je les estime comme du fumier, afin de gagner Christ,
\VS{9}et que je sois trouvé en lui, ayant non pas ma justice qui est de la loi, mais celle qui est par la foi en Christ, c'est-à-dire, la justice qui est de Dieu par la foi.
\VS{10}Ainsi, je connaîtrai Jésus-Christ et la puissance de sa résurrection, et la communion de ses souffrances, en devenant conforme à lui dans sa mort, pour parvenir,
\VS{11}si je puis, à la résurrection d'entre les morts.
\VS{12}Non que j'aie déjà atteint le but, ou que je sois déjà rendu parfait, mais je poursuis ce but pour tâcher d'y parvenir~; c'est pourquoi aussi j'ai été pris par Jésus-Christ.
\VS{13}Mes frères, pour moi, je ne me persuade pas d'avoir atteint le but~;
\VS{14}mais je fais une chose~: Oubliant les choses qui sont en arrière, et me portant vers celles qui sont en avant, je cours vers le but, pour remporter le prix de la vocation céleste de Dieu en Jésus-Christ.
\TextTitle{Paul exhorte les croyants à l'unité}
\VS{15}C'est pourquoi, nous tous qui sommes parfaits, ayons ce même sentiment~; et si vous êtes en quelque point d'un autre avis, Dieu vous le révélera aussi.
\VS{16}Cependant, marchons suivant une même règle pour les choses auxquelles nous sommes parvenus, et ayons un même sentiment.
\VS{17}Soyez tous ensemble mes imitateurs, mes frères, et portez les regards sur ceux qui marchent selon le modèle que vous avez en nous.
\VS{18}Car il y en a plusieurs qui marchent d'une telle manière, que je vous ai souvent dit, et maintenant je vous le dis encore en pleurant, qu'ils sont ennemis de la croix de Christ. 
\VS{19}Eux dont la fin est la perdition, qui ont pour dieu leur ventre, et dont la gloire est dans leur confusion, n'ayant d'affection que pour les choses de la terre.
\TextTitle{Le Messie: Notre espérance}
\VS{20}Mais pour nous, notre cité est dans les cieux, d'où nous aussi nous attendons le Sauveur, le Seigneur Jésus-Christ,
\VS{21}qui transformera notre corps vil, afin qu'il soit rendu conforme à son corps glorieux, selon cette efficacité\FTNT{Ep. 3:7} par laquelle il peut même s'assujettir toutes choses. 
\Chap{4}
\TextTitle{Avoir le même sentiment}
\VerseOne{}C'est pourquoi, mes très chers frères bien-aimés, vous qui êtes ma joie et ma couronne, demeurez ainsi fermes dans le Seigneur, mes bien-aimés.
\VS{2}J'exhorte Evodie, et j'exhorte aussi Syntyche, à être d'un même sentiment dans le Seigneur.
\VS{3}Et toi aussi, mon vrai compagnon\FTNT{Le mot compagnon est la traduction du grec «~suzugos~» qui signifie littéralement: Ensemble sous le joug, compagnon de peine, de joug. Cette expression renvoie à 2 Co. 6:14.}, oui je te prie de les aider, elles qui ont combattu avec moi pour l'Evangile, avec Clément, et mes autres compagnons d'œuvre, dont les noms sont écrits dans le livre de vie.
\VS{4}Réjouissez-vous toujours dans le Seigneur~; je vous le répète, réjouissez-vous~!
\VS{5}Que votre douceur soit connue de tous les hommes. Le Seigneur est proche.
\VS{6}Ne vous inquiétez de rien, mais en toutes choses présentez vos demandes à Dieu par des prières et des supplications, avec des actions de grâces.
\VS{7}Et la paix de Dieu, qui surpasse toute intelligence, gardera vos cœurs et vos sentiments en Jésus-Christ.
\TextTitle{L'objet de nos pensées}
\VS{8} Au reste, mes frères, que toutes les choses qui sont véritables, toutes les choses qui sont vénérables, toutes les choses qui sont justes, toutes les choses qui sont pures, toutes les choses qui sont aimables, toutes les choses qui sont de bonne renommée, toutes celles où il y a quelque vertu et quelque louange~; pensez à ces choses.
\VS{9}Car vous les avez aussi apprises, reçues, entendues et vues en moi. Faites ces choses, et le Dieu de paix sera avec vous. 
\TextTitle{Dieu soutient ses serviteurs}
\VS{10}Or je me suis fort réjoui en notre Seigneur, de ce qu'à la fin vous avez fait revivre le soin que vous aviez pour moi~; à quoi aussi vous pensiez, mais vous n'en aviez pas l'occasion. 
\VS{11}Je ne dis pas ceci à cause de mes besoins, car, moi, j'ai appris à être content en moi-même dans les circonstances où je me trouve.
\VS{12}Je sais être abaissé, je sais aussi être dans l'abondance~; partout et en toutes choses je suis instruit tant à être rassasié, qu'à avoir faim~; tant à être dans l'abondance, que dans la disette.
\VS{13}Je puis toutes choses en Christ qui me fortifie.
\VS{14}Néanmoins, vous avez bien fait de prendre part à mon affliction.
\VS{15}Vous savez aussi, vous Philippiens, qu'au commencement de la prédication de l'Evangile, quand je partis de Macédoine, aucune église ne me communiqua rien en matière de donner et de recevoir, excepté vous seuls. 
\VS{16}Et même lorsque j'étais à Thessalonique, vous m'avez envoyé une fois, et même deux fois, ce dont j'avais besoin. 
\VS{17}Ce n'est pas que je recherche des présents, mais je cherche le fruit qui abonde pour votre compte.
\VS{18}J'ai tout reçu, et je suis dans l'abondance, et j'ai été comblé de biens en recevant d'Epaphrodite ce qui vient de vous, comme un parfum de bonne odeur, comme un sacrifice que Dieu accepte et qui lui est agréable.
\VS{19}Aussi mon Dieu pourvoira à tout ce dont vous aurez besoin selon ses richesses, avec gloire en Jésus-Christ. 
\TextTitle{Salutations}
\VS{20}Or à notre Dieu et Père soit la gloire aux siècles des siècles~! Amen~!
\VS{21}Saluez tous les saints en Jésus-Christ. Les frères qui sont avec moi vous saluent.
\VS{22}Tous les saints vous saluent, et principalement ceux qui sont de la maison de César.
\VS{23}Que la grâce de notre Seigneur Jésus-Christ soit avec vous tous~! Amen~!
\PPE{}
\end{multicols}

\clearpage\ShortTitle{Col.}\BookTitle{Colossiens}\BFont
\noindent\hrulefill
{\footnotesize
\textit{
\bigskip
{\centering{}
\\Auteur~: Paul
\\Thème~: La prééminence de Christ
\\Date de rédaction~: Env. 60 ap. J.-C.\\}
}
\textit{
\\Située en Asie Mineure, Colosses était une ville de Phrygie qui se trouvait à environ deux cents kilomètres d'Ephèse.
\\Rédigée lors de la première captivité romaine de Paul, la lettre aux Colossiens a pour but de rétablir la suprématie de Christ. En effet, cette église - dont Epaphras, le probable fondateur, s'était converti à Ephèse au cours des trois années que Paul y passa - était sous l'influence d'enseignements séducteurs basés sur le gnosticisme. Cette philosophie à la fois attrayante et très dangereuse prônait entre autres le salut par la connaissance et le dualisme.\bigskip
}
}
\par\nobreak\noindent\hrulefill
\begin{multicols}{2}
\Chap{1}
\TextTitle{Introduction}
\VerseOne{}Paul, apôtre de Jésus-Christ, par la volonté de Dieu, et le frère Timothée~:
\VS{2}Aux saints et frères, fidèles en Christ, qui sont à Colosses, que la grâce et la paix vous soient données de la part de Dieu notre Père et de la part du Seigneur Jésus-Christ~!
\VS{3}Nous rendons grâces à Dieu, qui est le Père de notre Seigneur Jésus-Christ, et nous prions toujours pour vous,
\VS{4}ayant entendu parler de votre foi en Jésus-Christ, et de votre charité envers tous les saints,
\VS{5}à cause de l'espérance des biens qui vous sont réservés dans les cieux, et dont vous avez eu précédemment connaissance par la parole de la vérité, c'est-à-dire par l'Evangile,
\VS{6}qui est parvenu jusqu'à vous, comme il l'est aussi dans le monde entier. Et il porte des fruits, comme aussi parmi vous, depuis le jour où vous avez entendu et connu la grâce de Dieu dans la vérité,
\VS{7}ainsi que vous en avez aussi été instruits par Epaphras, notre cher compagnon de service, qui est pour vous un fidèle serviteur de Christ,
\VS{8}et qui nous a fait connaître votre charité par le Saint-Esprit.
\TextTitle{Prière de Paul pour les Colossiens}
\VS{9}C'est pourquoi depuis le jour où nous l'avons appris, nous ne cessons point de prier pour vous, et de demander à Dieu que vous soyez remplis de la connaissance de sa volonté, en toute sagesse et intelligence spirituelle,
\VS{10}afin que vous vous conduisiez d'une manière digne du Seigneur, pour lui plaire en toutes choses, portant des fruits en toutes sortes de bonnes œuvres, et croissant dans la connaissance de Dieu,
\VS{11}étant fortifiés en toute force, selon la puissance de sa gloire, pour toute patience, et constance, avec joie.
\TextTitle{Le salut de Dieu}
\VS{12}Rendant grâces au Père, qui nous a rendus capables d'avoir part à l'héritage des saints dans la lumière,
\VS{13}qui nous a délivrés de la puissance des ténèbres, et nous a transportés dans le Royaume du Fils de son amour,
\VS{14}en qui nous avons la rédemption par son sang, à savoir la rémission des péchés.
\VS{15}Lequel est l'image de Dieu invisible, le premier-né\FTNT{Dans les Ecritures, l'expression «~premier-né~» est appliquée au Seigneur pour exprimer trois réalités. Tout d'abord, on parle de Jésus en tant que premier-né de Marie, c'est-à-dire son fils aîné (Lu. 2:6-7). Ensuite, on trouve cette expression au sens figuré, pour marquer une distinction (par exemple concernant Israël~; Ex. 4:2) ou désigner la particularité et la suprématie d'une personne. Ainsi, bien que David était le dernier-né de son père Isaï (Ps. 89:28), Dieu en fit un premier-né, «~le plus élevé des rois de la terre~» (Ps. 89:28). Il en va de même pour Jésus-Christ. Il n'est pas le premier-né de la création dans le sens de rang de naissance ou de création, autrement Paul aurait employé le terme grec «~prôtoktisis~» qui signifie «~premier-créé~», au lieu de «~prôtotokos~», c'est-à-dire «~premier-né~». Il faut donc voir dans cette expression un titre de supériorité et d'hiérarchie, pour marquer sa prééminence. En effet, la Parole de Dieu déclare clairement que le Seigneur Jésus-Christ est l'Alpha et le Commencement de toutes choses (Ap. 1:8~; Ap. 21:6~; Ap. 22:13), le Créateur suprême (Ge. 1:1~; Ge. 2:7~; Es. 45:11-18~; Ps. 104:30~; Job. 33:4~; Jn. 1:3~; 1 Co. 8:6~; Col. 1:12-16~; Ap. 22:3~; Ap. 14:6). D'ailleurs il l'a lui-même affirmé sans ambigüité~: «~Avant qu'Abraham fût, Je suis~» (Jn. 8:58). Enfin, Jésus-Christ est aussi appelé le premier-né d'entre les morts (Col. 1:18). Cela ne signifie pas qu'il a été le premier à ressusciter, car il y a eu plusieurs résurrections avant la sienne, mais il fut le premier à ressusciter avec un corps glorieux. Sa résurrection est donc le gage de la promesse de la résurrection de tous ceux qui ont foi en lui (Jn. 3:16).} de toute la création.
\VS{16}Car par lui ont été créées toutes les choses qui sont dans les cieux et sur la terre, les visibles et les invisibles, soit les trônes, ou les dominations, ou les principautés, ou les puissances, toutes choses ont été créées par lui, et pour lui.
\VS{17}Et il est avant toutes choses, et toutes choses subsistent par lui.
\VS{18}Et c'est lui qui est le Chef du corps de l'Eglise, et qui est le commencement et le premier-né d'entre les morts, afin qu'il tienne le premier rang en toutes choses,
\VS{19}car le bon plaisir du Père a été que toute plénitude habitât en lui.
\VS{20}Et de réconcilier par lui toutes choses avec lui même, ayant fait la paix par le sang de sa croix, à savoir, tant les choses qui sont dans les cieux que celles qui sont sur la terre.
\VS{21}Et vous, qui étiez autrefois étrangers, et qui étiez ses ennemis dans votre entendement, et dans les mauvaises œuvres, il vous a maintenant réconciliés 
\VS{22}par le corps de sa chair, par sa mort, pour vous présenter saints, et sans tache, et irrépréhensibles devant lui.
\VS{23}Si toutefois vous demeurez dans la foi, étant fondés et fermes, et n'étant point transportés hors de l'espérance de l'Evangile que vous avez entendu, lequel est prêché à toute créature qui est sous le ciel, dont moi Paul, j'ai été fait le serviteur.
\VS{24}Je me réjouis donc maintenant dans mes souffrances pour vous~; et j'accomplis le reste des afflictions de Christ dans ma chair, pour son corps, qui est l'Eglise.
\VS{25}C'est d'elle que j'ai été fait le serviteur, selon la gestion que Dieu m'a donnée auprès de vous, afin que j'exécute pleinement la parole de Dieu,
\VS{26}à savoir le mystère qui avait été caché dans tous les siècles et dans tous les âges, mais qui est maintenant manifesté à ses saints~;
\VS{27}auxquels Dieu a voulu donner à connaître quelles sont les richesses de la gloire de ce mystère parmi les Gentils, c'est à savoir Christ, qui a été prêché parmi vous, et qui est l'espérance de la gloire~; 
\VS{28}lequel nous annonçons, en exhortant tout homme, et en enseignant tout homme en toute sagesse, afin que nous présentions tout homme parfait en Jésus-Christ.
\TextTitle{Le combat de Paul}
\VS{29}A quoi aussi je travaille, en combattant selon son efficacité\FTNTT{le terme «~efficacité~» vient du grec «~energeia~» qui signifie «~action~», «~fonctionnement~», «~compétence~», ou encore «~force à l'œuvre dans~». Ce mot est utilisé seulement pour parler du pouvoir surhumain que ce soit celui de Dieu ou celui du diable. (Ep. 1:19~; Ph. 3:21~; 2 Ti. 2:9).}, qui agit puissamment en moi.
\Chap{2}
\VerseOne{}Or je veux, en effet, que vous sachiez combien est grand le combat que j'ai pour vous, et pour ceux qui sont à Laodicée, et pour tous ceux qui n'ont pas vu mon visage dans la chair,
\VS{2}afin que leurs cœurs soient consolés, étant unis ensemble dans la charité, et enrichis d'une pleine intelligence, pour la connaissance du mystère de notre Dieu et Père, et de Christ,
\VS{3}en qui sont cachés tous les trésors de la sagesse et de la connaissance.
\TextTitle{Mise en garde contre les discours séduisants et la philosophie\FTNTT{1 Co. 2:4~; Ro. 16:17-18~; 2 Pi. 2:3.}}
\VS{4}Or je dis ceci afin que personne ne vous trompe par des discours séduisants.
\VS{5}Car, quoique je sois absent de corps, toutefois je suis avec vous en esprit, me réjouissant, et voyant votre ordre et la fermeté de votre foi, que vous avez en Christ.
\VS{6}Ainsi, comme vous avez reçu le Seigneur Jésus-Christ, marchez en lui,
\VS{7}étant enracinés et édifiés en lui, et fortifiés en la foi, selon que vous avez été enseignés, abondant en elle avec action de grâces.
\VS{8}Prenez garde que personne ne fasse de vous sa proie par la philosophie, et par de vaines tromperies conformes à la tradition des hommes et aux rudiments du monde, et non point à la doctrine de Christ.
\TextTitle{La divinité du Christ}
\VS{9}Car en lui habite corporellement toute la plénitude de la divinité\FTNT{En Jésus-Christ habite toute la plénitude de la divinité. Il est le Dieu Tout-Puissant.}.
\VS{10}Et vous êtes rendus accomplis en lui, qui est le Chef de toute principauté et puissance.
\TextTitle{L'oeuvre de la croix}
\VS{11}En qui aussi vous êtes circoncis d'une circoncision faite sans main, qui consiste à dépouiller le corps des péchés de la chair, ce qui est la circoncision de Christ.
\VS{12}Etant ensevelis avec lui par le baptême, en qui aussi vous êtes ensemble ressuscités par la foi de l'efficacité de Dieu, qui l'a ressuscité des morts.
\VS{13}Et lorsque vous étiez morts dans vos offenses, et dans l'incirconcision de votre chair, il vous a vivifiés ensemble avec lui, vous ayant gratuitement pardonné toutes vos offenses.
\VS{14}Il a effacé l'acte qui était contre nous, qui consistait en des ordonnances, et qui nous était contraire, et il l'a entièrement aboli en le clouant à la croix.
\VS{15}Il a dépouillé les principautés et les puissances, et les a exposées publiquement en spectacle, en triomphant d'elles par la croix.
\TextTitle{Mise en garde contre les commandements et les doctrines des hommes}
\VS{16}Que personne donc ne vous juge au sujet du manger ou du boire, ou au sujet d'un jour de fête, ou d'un jour de nouvelle lune, ou de sabbat,
\VS{17}qui sont l'ombre des choses qui devaient venir, mais le corps est en Christ.
\VS{18}Que personne ne vous enlève à son gré le prix de la course, sous l'apparence d'humilité d'esprit et par un culte des anges, s'ingérant dans les choses qu'il n'a pas vues, étant témérairement enflé par ses pensées charnelles,
\VS{19}sans s'attacher au Chef, dont tout le corps étant joint et ajusté ensemble par des jointures et des liens, s'accroît d'un accroissement de Dieu.
\VS{20}Si donc vous êtes morts avec Christ quant aux rudiments du monde, pourquoi vous impose-t-on ces ordonnances, comme si vous viviez dans le monde~?
\VS{21}A savoir: Ne prends pas~! Ne goûte pas~! Ne touche pas~!
\VS{22}Lesquelles sont toutes périssables par l'usage, et établies suivant les commandements et les doctrines des hommes~; 
\VS{23}et qui ont pourtant quelque apparence de sagesse en dévotion volontaire, et en humilité d'esprit, et en ce qu'elles n'épargnent pas le corps, et n'ont aucun égard à la satisfaction de la chair.
\Chap{3}
\TextTitle{Rechercher les choses d'en haut}
\VerseOne{}Si donc vous êtes ressuscités avec Christ, cherchez les choses qui sont en haut, où Christ est assis à la droite de Dieu.
\VS{2}Pensez aux choses d'en haut, et non à celles qui sont sur la terre.
\VS{3}Car vous êtes morts, et votre vie est cachée avec Christ en Dieu.
\VS{4}Quand Christ, qui est votre vie, apparaîtra, vous paraîtrez aussi alors avec lui dans la gloire.
\TextTitle{La mort à soi en pratique}
\VS{5}Mortifiez donc vos membres qui sont sur la terre~: La fornication, l'impureté, les passions, les mauvais désirs, et la cupidité, qui est une idolâtrie.
\VS{6}C'est à cause de ces choses que la colère de Dieu vient sur les fils de la rébellion,
\VS{7}parmi lesquels vous marchiez autrefois, quand vous viviez dans ces choses.
\VS{8}Mais maintenant, vous aussi, rejetez toutes ces choses~: La colère, l'animosité, la médisance, et les paroles déshonnêtes qui pourraient sortir de votre bouche.
\VS{9}Ne mentez point les uns aux autres, vous étant dépouillés du vieil homme et de ses œuvres,
\VS{10}et ayant revêtu le nouvel homme, qui se renouvelle dans la connaissance, selon l'image de celui qui l'a créé.
\VS{11}En qui il n'y a ni Grec ni Juif, ni circoncis ni incirconcis, ni barbare ni Scythe, ni esclave ni libre~; mais Christ y est tout et en tous.
\VS{12}Ainsi donc, comme des élus de Dieu, saints et bien-aimés, revêtez-vous des entrailles de miséricorde, de bonté, d'humilité, de douceur, de patience.
\VS{13}Vous supportant les uns les autres, et vous pardonnant les uns aux autres~; et si l'un a querelle contre l'autre, comme Christ vous a pardonné, vous aussi faites-en de même.
\VS{14}Mais par-dessus toutes ces choses, revêtez-vous de la charité, qui est le lien de la perfection.
\VS{15}Et que la paix de Dieu, à laquelle aussi vous êtes appelés pour être un seul corps, tienne le principal lieu dans vos cœurs. Et soyez reconnaissants.
\VS{16}Que la parole de Christ habite en vous abondamment en toute sagesse~; vous enseignant et vous exhortant l'un l'autre par des psaumes, et des hymnes et des cantiques spirituels, avec grâce, chantant de votre cœur au Seigneur.
\VS{17}Et quoi que vous fassiez, en parole ou en œuvre, faites tout au Nom du Seigneur Jésus, rendant grâces par lui à notre Dieu et Père.
\TextTitle{La famille selon Dieu}
\VS{18}Femmes, soyez soumises à vos maris, comme il convient dans le Seigneur\FTNT{Ep. 5:22.}.
\VS{19}Maris, aimez vos femmes, et ne vous aigrissez pas contre elles\FTNT{Ep. 5:25.}.
\VS{20}Enfants, obéissez à vos pères et à vos mères en toutes choses, car cela est agréable au Seigneur\FTNT{Ep. 6:1-2.}.
\VS{21}Pères, n'irritez pas vos enfants\FTNT{Ep. 6:4.}, afin qu'ils ne se découragent pas.
\TextTitle{Les rapports entre serviteurs et maîtres selon Dieu}
\VS{22}Serviteurs, obéissez en toutes choses à ceux qui sont vos maîtres selon la chair, ne servant point seulement sous leurs yeux, comme voulant complaire aux hommes, mais en simplicité de cœur, craignant Dieu\FTNT{Ep. 6:5-6.}.
\VS{23}Et quoi que vous fassiez, faites tout de bon cœur, comme le faisant pour le Seigneur, et non pas pour les hommes,
\VS{24}sachant que vous recevrez du Seigneur l'héritage pour récompense. Car vous servez Christ, le Seigneur.
\VS{25}Mais celui qui agit injustement recevra ce qu'il aura fait injustement, car en Dieu il n'y a point d'égard à l'apparence des personnes.
\Chap{4}
\VerseOne{}Maîtres, accordez à vos serviteurs ce qui est juste et équitable, sachant que vous avez, vous aussi, un Maître dans les cieux.
\TextTitle{La persévérance dans la prière}
\VS{2}Persévérez dans la prière, veillant dans cet exercice avec des actions de grâces.
\VS{3}Priez aussi tous ensemble pour nous, afin que Dieu nous ouvre la porte de la parole, pour annoncer le mystère de Christ pour lequel aussi je suis prisonnier, 
\VS{4}afin que je le fasse connaître comme je dois en parler.
\VS{5}Conduisez-vous sagement envers ceux du dehors, et rachetez le temps.
\VS{6}Que votre parole soit toujours assaisonnée de sel, avec grâce, afin que vous sachiez comment vous avez à répondre à chacun.
\TextTitle{Salutations}
\VS{7}Tychique, notre frère bien-aimé, et fidèle serviteur, et mon compagnon de service en notre Seigneur, vous fera savoir tout mon état.
\VS{8}Je l'envoie vers vous expressément, afin qu'il connaisse quel est votre état, et qu'il console vos cœurs~;
\VS{9}avec Onésime, notre fidèle et bien-aimé frère, qui est des vôtres. Ils vous feront connaitre toutes les choses d'ici.
\VS{10}Aristarque, qui est prisonnier avec moi, vous salue aussi, et Marc qui est le cousin de Barnabas, au sujet duquel vous avez reçu un ordre, s'il vient à vous, recevez-le.
\VS{11}Et Jésus, appelé Justus, vous salue aussi. Ils sont du nombre des circoncis, et les seuls qui travaillent avec moi pour le Royaume de Dieu, et qui ont été pour moi une consolation.
\VS{12}Epaphras, qui est des vôtres, et serviteur de Jésus-Christ, vous salue~; il ne cesse de combattre pour vous dans ses prières afin que vous demeuriez parfaits et accomplis en toute la volonté de Dieu.
\VS{13}Car je lui rends témoignage qu'il a un grand zèle pour vous, et pour ceux de Laodicée, et pour ceux d'Hiérapolis.
\VS{14}Luc, le médecin bien-aimé, vous salue, et Démas aussi.
\VS{15}Saluez les frères qui sont à Laodicée, et Nymphas, avec l'église qui est dans sa maison.
\VS{16}Et quand cette lettre aura été lue entre vous, faites en sorte qu'elle soit aussi lue dans l'église des Laodicéens, et que vous lisiez aussi celle qui viendra de Laodicée.
\VS{17}Et dites à Archippe~: Prends garde au service que tu as reçu dans le Seigneur afin de bien le remplir.
\VS{18}Je vous salue, moi Paul, de ma propre main. Souvenez-vous de mes liens. Que la grâce soit avec vous~! Amen~!
\PPE{}
\end{multicols}

\clearpage\ShortTitle{Philémon}\BookTitle{Philémon}\BFont
\noindent\hrulefill
{\footnotesize
\textit{
\bigskip
{\centering{}
\\Auteur : Paul
\\(Gr. : Philemon)
\\Signification : Attentionné, qui embrasse
\\Thème : Un exemple d'amour
\\Date de rédaction : Env. 60 ap. J.-C.\\}
}
%\bigskip
\textit{
\\Paul écrivit cette lettre en prison, lors de sa deuxième captivité à Rome vers l'été 62, en même temps que l'épître aux Colossiens. Il s'adresse à Philémon, chrétien fortuné de Colosses ainsi qu'à sa femme Apphia, son fils Archippe et à l'église qui se réunissait dans leur maison. Paul demande à Philémon de pardonner à Onésime, son esclave, de s'être échappé d'auprès de lui. Il assure à Philémon que désormais une nouvelle relation le lierait à Onésime qui avait accepté Jésus-Christ dans sa vie. Il va même jusqu'à proposer de payer personnellement ce qu'Onésime lui devait tout en exprimant l'espoir que Philémon ferait plus que ce qu'il lui demande. Ainsi, Paul plaide pour Onésime comme Christ le fit en notre faveur.\bigskip
}
}
\par\nobreak\noindent\hrulefill
\begin{multicols}{2}
\Chap{1}
\TextTitle{Introduction}
\VerseOne{}Paul, prisonnier de Jésus-Christ, et le frère Timothée, à Philémon notre bien-aimé et compagnon d'œuvre ;
\VS{2}à Apphia, notre bien-aimée, à Archippe, notre compagnon de combat, et à l'église qui est dans ta maison.
\VS{3}Que la grâce et la paix vous soient données de la part de Dieu notre Père, et de la part du Seigneur Jésus-Christ.
\VS{4}Je rends grâces à mon Dieu, faisant toujours mention de toi dans mes prières ;
\VS{5}apprenant la foi que tu as au Seigneur Jésus, et ta charité envers tous les saints.
\VS{6}Afin que la communication de ta foi devienne efficace, en se faisant connaître par tout le bien qui est en vous, par Jésus-Christ.
\VS{7}Car, mon frère, nous avons une grande joie et une grande consolation de ta charité, en ce que tu as réjoui les entrailles des saints.
\TextTitle{Paul plaide en faveur d'Onésime}
\VS{8}C'est pourquoi, bien que j'aie une grande liberté en Christ de t'ordonner ce qui est convenable,
\VS{9}cependant je te prie plutôt par la charité, bien que je suis ce que je suis, à savoir Paul, un vieillard, et même maintenant prisonnier de Jésus-Christ ;
\VS{10}je te prie donc pour mon fils Onésime, que j'ai engendré dans mes liens ;
\VS{11}qui t'a été autrefois inutile, mais qui maintenant est bien utile à toi et à moi, et que je te renvoie.
\VS{12}Reçois-le donc comme mes propres entrailles.
\VS{13}Je voulais le retenir auprès de moi, afin qu'il me serve à ta place, dans les liens de l'Evangile.
\VS{14}Mais je n'ai rien voulu faire sans ton avis, afin que ce ne soit point comme par contrainte, mais volontairement, que tu me laisses un bien qui est à toi.
\VS{15}Car peut-être n'a-t-il été séparé de toi que pour un temps, afin que tu le recouvres\FTNT{Synonymes : retrouver, reconquérir, regagner.} pour toujours ;
\VS{16}non plus comme un esclave, mais comme étant au-dessus d'un esclave, à savoir comme un frère bien-aimé, principalement de moi ; et combien plus de toi, soit selon la chair, soit selon le Seigneur ?
\VS{17}Si donc tu me tiens pour ton compagnon, reçois-le comme moi-même.
\VS{18}Que s'il t'a fait quelque tort, ou s'il te doit quelque chose, mets-le sur mon compte.
\VS{19}Moi Paul, j'ai écrit ceci de ma propre main, je te le payerai ; pour ne pas te dire que tu te dois toi-même à moi.
\VS{20}Oui, mon frère, que je reçoive ce plaisir de toi en notre Seigneur ; réjouis mes entrailles en notre Seigneur.
\VS{21}Je t'ai écrit m'assurant de ton obéissance, et sachant que tu feras même plus que ce que je te dis.
\TextTitle{Conclusion}
\VS{22}Mais aussi, en même temps, prépare-moi un logement ; car j'espère que je vous serai rendu par vos prières.
\VS{23}Epaphras, qui est prisonnier avec moi en Jésus-Christ, te salue ;
\VS{24}Marc aussi, Aristarque, Démas, et Luc, mes compagnons d'œuvre.
\VS{25}Que la grâce de notre Seigneur Jésus-Christ soit avec votre esprit, Amen !
\PPE{}
\end{multicols}

\clearpage\ShortTitle{1 Ti.}\BookTitle{1 Timothée}\BFont
\noindent\hrulefill
{\footnotesize
\textit{
\bigskip
{\centering{}
\\Auteur~: Paul
\\(Gr.~: Timotheos)
\\Signification~: Qui adore ou honore Dieu
\\Thème~: Comment se conduire dans l'église
\\Date de rédaction~: Env. 64 ap. J.-C.\\}
}
\textit{
\\Cette lettre s'adresse à Timothée dont le père était grec et la mère juive. Le jeune homme se convertit à Christ avec sa mère et sa grand-mère dès le premier voyage missionnaire de Paul au cours duquel il passa à Lystre.
\\Cette épître fut rédigée après la première captivité de Paul à Rome. Alors que les Eglises connaissaient une certaine expansion, Paul s'adresse à Timothée, jeune et fidèle compagnon d'œuvre qu'il a lui-même formé, sur des questions d'ordre disciplinaire et sur la pureté de la foi. Dans cette épître, dite pastorale, Paul donne des instructions précises à Timothée pour enseigner, exhorter, diriger le culte public et choisir ses collaborateurs.\bigskip
}
}
\par\nobreak\noindent\hrulefill
\begin{multicols}{2}
\Chap{1}
\TextTitle{Introduction}
\VerseOne{}Paul, apôtre de Jésus-Christ par l'ordre de Dieu, notre Sauveur, et du Seigneur Jésus-Christ, notre espérance,
\VS{2}à Timothée mon véritable fils dans la foi~: Que la grâce, la miséricorde et la paix te soient données de la part de Dieu notre Père, et de Jésus-Christ, notre Seigneur.
\TextTitle{Mise en garde contre les erreurs doctrinales~; le but de la loi} 
\VS{3}Suivant la prière que je te fis de demeurer à Ephèse, lorsque j'allais en Macédoine, je te prie encore d'ordonner à certaines personnes de ne pas enseigner une autre doctrine,
\VS{4}et de ne pas s'adonner aux fables et aux généalogies sans fin, qui produisent des disputes plutôt que l'édification en Dieu qui consiste dans la foi.
\VS{5}Or le but du commandement c'est la charité qui procède d'un cœur pur, d'une bonne conscience, et d'une foi sincère.
\VS{6}Quelques-uns, s'étant détournés de ces choses, se sont écartés dans de vains discours,
\VS{7}voulant être docteurs de la loi~; mais ils ne comprennent ni ce qu'ils disent ni ce qu'ils affirment.
\VS{8}Or nous savons que la loi est bonne pour celui qui en fait un usage légitime,
\VS{9}sachant ceci, que ce n'est pas pour le juste que la loi a été établie, mais pour les méchants et les rebelles, pour les impies et les pécheurs, pour les irréligieux et les profanes, pour les parricides, les meurtriers,
\VS{10}pour les fornicateurs, pour les homosexuels, pour les voleurs d'hommes, pour les menteurs, pour les parjures, et contre telle autre chose qui est contraire à la saine doctrine,
\VS{11}selon l'Evangile de la gloire du Dieu béni, Evangile qui m'a été confié.
\TextTitle{Témoignage de Paul}
\VS{12}Je rends grâces à celui qui m'a fortifié, c'est-à-dire à Jésus-Christ, notre Seigneur, de ce qu'il m'a estimé fidèle en m'établissant dans le service,
\VS{13}moi qui auparavant étais un blasphémateur, un persécuteur, et un homme violent~; mais j'ai obtenu miséricorde parce que j'agissais par ignorance, étant dans l'incrédulité.
\VS{14}Or la grâce de notre Seigneur a surabondé en moi, avec la foi et l'amour qui est en Jésus-Christ.
\VS{15}Cette parole est certaine et entièrement digne d'être reçue, que Jésus-Christ est venu dans le monde pour sauver les pécheurs, dont je suis le premier.
\VS{16}Mais j'ai obtenu miséricorde, afin que Jésus-Christ fasse voir en moi le premier, toute sa clémence, pour que je serve d'exemple à ceux qui croiraient en lui pour la vie éternelle.
\VS{17}Or au Roi des siècles, immortel, invisible, à Dieu seul sage, soient honneur et gloire aux siècles des siècles~! Amen~!
\TextTitle{Recommandations à Timothée}
\VS{18}Mon fils Timothée, je te recommande ce commandement que conformément aux prophéties qui auparavant ont été faites sur toi, tu t'acquittes, selon elles, du devoir de combattre dans cette bonne guerre,
\VS{19}en gardant la foi et une bonne conscience, laquelle quelques-uns ayant rejetée, ont fait naufrage quant à la foi.
\VS{20}De ce nombre sont Hyménée et Alexandre, que j'ai livrés à Satan, afin qu'ils apprennent par ce châtiment à ne plus blasphémer.
\Chap{2}
\TextTitle{Instructions sur la prière}
\VerseOne{}J'exhorte donc, avant toutes choses, à faire des requêtes, des prières, des supplications, et des actions de grâces pour tous les hommes,
\VS{2}pour les rois et pour tous ceux qui sont constitués en dignité, afin que nous menions une vie paisible et tranquille, en toute piété et honnêteté.
\VS{3}Car cela est bon et agréable devant Dieu, notre Sauveur,
\VS{4}qui veut que tous les hommes soient sauvés et qu'ils viennent à la connaissance de la vérité.
\VS{5}Car il y a un seul Dieu, et aussi un seul Médiateur entre Dieu et les hommes, à savoir Jésus-Christ homme,
\VS{6}qui s'est donné lui-même en rançon pour tous. C'est le témoignage qui a été rendu en son propre temps.
\VS{7}C'est dans cette vue que j'ai été établi prédicateur, apôtre (je dis la vérité en Christ, je ne mens point) et docteur des Gentils dans la foi et dans la vérité.
\VS{8}Je veux donc que les hommes prient en tout lieu, levant leurs mains pures, sans colère, et sans dispute.
\TextTitle{Le tenue de la femme}
\VS{9}Et de même, que les femmes, vêtues d'une manière décente, avec pudeur et modestie, ne se parent ni de tresses, ni d'or, ni de perles, ni d'habits somptueux,
\VS{10}mais qu'elles se parent de bonnes œuvres, comme il convient à des femmes qui font profession de servir Dieu.
\TextTitle{Le comportement de la femme envers son mari}
\VS{11}Que la femme apprenne dans le silence, en toute soumission.
\VS{12}Car je ne permets pas à la femme d'enseigner ni d'user d'autorité\FTNT{Le mot «~autorité~» vient du grec «~authenteo~» et signifie «~celui qui tue de ses propres mains un autre ou lui-même~; celui qui agit de sa propre autorité, autocrate~; un maître absolu~; gouverneur, exercer une domination~».} sur le mari~; mais elle doit demeurer dans le silence\FTNT{Le mot grec traduit par «~silence~» est «~hesuchia~» qui signifie «~en silence~; paisiblement~». La racine de ce terme est «~hesuchios~»~: tranquille, paisible.}.
\VS{13}Car Adam a été formé le premier, Eve ensuite.
\VS{14}Et ce n'est pas Adam qui a été séduit, mais la femme, ayant été séduite, a été la cause de la transgression.
\VS{15}Elle sera néanmoins sauvée en mettant des enfants au monde\FTNT{Il est évident que le salut ne dépend pas du fait d'enfanter puisque nous sommes sauvés par grâce et non par les œuvres. Ce verset fait référence à Eve, la mère de tous les vivants. Par elle le péché et la mort sont entrés dans le monde (Ro. 5:12) mais c'est aussi par sa postérité, à savoir Christ (Ge. 3:15), qu'elle, ainsi que tout le genre humain (hommes et femmes), sera sauvé.}, pourvu qu'elle persévère dans la foi, dans la charité, et dans la sanctification, avec modestie.
\Chap{3}
\TextTitle{Les évêques et les diacres doivent manifester le caractère de Christ}
\VerseOne{}Cette parole est certaine, si quelqu'un désire la charge d'évêque\FTNT{Evêque, du grec «~episkope~», signifie «~investigation, inspection, visite d'inspection~». C'est un acte par lequel Dieu visite les hommes, observe leurs voies, leurs caractères, pour leur accorder en partage joie ou tristesse. Ce terme signifie également surveillance, charge, contrôle, fonction, la fonction d'un ancien. Voir Ac. 1:20.}, il désire une œuvre excellente.
\VS{2}Mais il faut que l'évêque soit irrépréhensible, mari\FTNT{Paul ne dit pas que les évêques ne peuvent pas être célibataires. Il y a en effet une différence entre mari et marié. L'apôtre met l'accent sur la monogamie. Un homme célibataire peut en effet être évêque s'il remplit les caractéristiques décrites dans ce passage.} d'une seule femme, vigilant, modéré, honorable, hospitalier, propre à enseigner.
\VS{3}Il faut qu'il ne soit ni adonné au vin, ni violent, ni porté au gain déshonnête, mais modéré, éloigné des querelles, exempt d'avarice.
\VS{4}Il faut qu'il dirige honnêtement sa propre maison, et qu'il tienne ses enfants dans la soumission et dans une parfaite honnêteté~;
\VS{5}car si quelqu'un ne sait pas diriger sa propre maison, comment pourra-t-il gouverner l'église de Dieu~?
\VS{6}Il ne faut pas qu'il soit un nouveau converti, de peur qu'enflé d'orgueil, il ne tombe sous le jugement du diable.
\VS{7}Il faut aussi qu'il reçoive un bon témoignage de ceux du dehors, afin de ne pas tomber dans l'opprobre et dans les pièges du diable.
\VS{8}Que les diacres aussi soient honnêtes, éloignés de la duplicité, des excès du vin, d'un gain sordide,
\VS{9}conservant le mystère de la foi dans une conscience pure.
\VS{10}Que ceux-ci aussi soient premièrement éprouvés, et qu'ensuite ils servent, après avoir été trouvés sans reproche.
\VS{11}Leurs femmes, de même, doivent être honnêtes, non médisantes, sobres, fidèles en toutes choses.
\VS{12}Les diacres doivent être maris d'une seule femme, dirigeant honnêtement leurs enfants, et leurs propres maisons.
\VS{13}Car ceux qui auront bien servi s'acquièrent un rang honorable, et une grande liberté dans la foi qui est en Jésus-Christ.
\VS{14}Je t'écris ces choses espérant que j'irai bientôt vers toi~;
\VS{15}mais si je tarde, je t'écris ces choses afin que tu saches comment il faut se conduire dans la maison de Dieu, qui est l'Eglise du Dieu vivant, la colonne et l'appui de la vérité.
\VS{16}Et sans contredit, le mystère de la piété\FTNT{Le mystère de la piété. Il s'agit de la connaissance de Dieu manifestée en chair dans la personne de Jésus-Christ, 100\% homme et 100\% Dieu. C'est l'incarnation du Dieu Tout-Puissant dans le seul but de sauver les hommes et de produire dans leurs cœurs la véritable piété.} est grand~: Dieu a été manifesté en chair, justifié par l'Esprit, vu des anges, prêché aux Gentils, cru dans le monde, et élevé dans la gloire.
\Chap{4}
\TextTitle{L'apostasie et la séduction: Signes des derniers temps}
\VerseOne{}Mais l'Esprit dit expressément que dans les derniers temps, quelques-uns se détourneront de la foi pour s'attacher à des esprits séducteurs et à des doctrines de démons\FTNT{Il est indéniable que nous vivons les dernières minutes avant le retour glorieux de Jésus-Christ. Toutes les conditions sont pratiquement réunies pour que le Seigneur revienne, c'est pourquoi chaque enfant de Dieu doit se préparer à la rencontre avec l'Epoux. Les prophètes, notamment Paul, ont annoncé que la fin des temps serait caractérisée par la séduction et l'abandon de la foi de beaucoup de chrétiens.},
\VS{2}par l'hypocrisie de faux docteurs, ayant leur propre conscience marquée au fer rouge\FTNT{L'expression «~marqué au fer~» ou «~marque de la flétrissure~» se dit «~kauteriazo~» en grec et veut dire «~ceux dont l'âme est stigmatisée par les marques du péché~». Dans un sens médical, ce mot signifie «~cautériser~». Ce passage fait allusion à la marque de la bête qui sera imprimée dans la conscience des hommes~; voilà pourquoi Dieu nous demande de garder sa parole dans nos cœurs (Ps. 119:11). Les Juifs devaient avoir sur leurs mains et sur leurs fronts la marque de Dieu qui est sa parole (De. 6:6-8). La main se dit «~yad~» en hébreu, ce qui signifie «~pouvoir~», «~force~» ou encore «~autorité~»~; elle symbolise donc l'action. Le front se dit «~towphaphah~» en hébreu, ce qui signifie «~marque~»~; il s'agit de la pensée.}~;
\VS{3}défendant de se marier et commandant de s'abstenir des viandes que Dieu a créées afin que les fidèles, et ceux qui ont connu la vérité, en usent avec actions de grâces.
\VS{4}Car tout ce que Dieu a créé est bon, et rien ne doit être rejeté, pourvu qu'on le prenne avec actions de grâces,
\VS{5}parce que tout est sanctifié par la parole de Dieu et par la prière.
\TextTitle{S'exercer à la piété}
\VS{6}En exposant ces choses aux frères, tu seras un bon serviteur de Jésus-Christ, nourri des paroles de la foi et de la bonne doctrine que tu as exactement suivie.
\VS{7}Mais rejette les fables profanes, et semblables aux récits de vieilles femmes.
\VS{8}Exerce-toi à la piété~; car l'exercice corporel est utile à peu de chose, tandis que la piété est utile à toutes choses, ayant les promesses de la vie présente et de celle qui est à venir.
\VS{9}C'est là une parole certaine et digne d'être entièrement reçue.
\VS{10}Car c'est aussi à cause de cela que nous endurons des travaux et des opprobres, parce que nous espérons dans le Dieu vivant, qui est le Sauveur de tous les hommes, mais principalement des fidèles.
\VS{11}Déclare ces choses et enseigne-les.
\VS{12}Que personne ne méprise ta jeunesse~; mais sois le modèle pour les fidèles en paroles, en conduite, en charité, en esprit, en foi, en pureté.
\VS{13}Applique-toi à la lecture, à l'exhortation et à l'enseignement, jusqu'à ce que je vienne.
\VS{14}Ne néglige pas le don qui est en toi, et qui t'a été donné par prophétie, par l'imposition des mains de l'assemblée des anciens.
\VS{15}Pratique ces choses et donne-toi tout entier à elles, afin que tes progrès soient évidents pour tous.
\VS{16}Veille sur toi-même et sur la doctrine~; persévère dans ces choses, car en agissant ainsi, tu te sauveras toi-même et tu sauveras ceux qui t'écoutent.
\Chap{5}
\TextTitle{Recommandations concernant les veuves}
\VerseOne{}Ne reprends pas rudement le vieillard, mais exhorte-le comme un père~; les jeunes gens comme des frères,
\VS{2}les femmes âgées comme des mères, celles qui sont jeunes comme des sœurs, en toute pureté.
\VS{3}Honore les veuves qui sont véritablement veuves.
\VS{4}Mais si une veuve a des enfants, ou des petits enfants, qu'ils apprennent avant tout à exercer la piété envers leur propre famille, et à rendre à leurs parents ce qu'ils ont reçu d'eux~; car cela est bon et agréable à Dieu.
\VS{5}Or celle qui est véritablement veuve, et qui est laissée seule, espère en Dieu, et persévère nuit et jour dans les supplications et les prières.
\VS{6}Mais celle qui vit dans les plaisirs est morte quoique vivante.
\VS{7}Avertis-les donc de ces choses, afin qu'elles soient irrépréhensibles.
\VS{8}Que si quelqu'un n'a pas soin des siens, et principalement de ceux de sa famille, il a renié la foi, et il est pire qu'un infidèle.
\VS{9}Qu'une veuve, pour être enregistrée sur le rôle\FTNT{Inscription sur le rôle~: Expression qui s'apparente à l'enrôlement des soldats. Il est question des veuves ayant une place importante dans l'église, du fait qu'elles exercent une certaine responsabilité sur le reste des femmes, et ayant en charge les veuves et les orphelins pris en compte pour la dépense publique.}, n'ait pas moins de soixante ans, qu'elle ait été la femme d'un seul mari,
\VS{10}ayant le témoignage d'avoir fait de bonnes œuvres, comme d'avoir bien élevé ses propres enfants, d'avoir exercé l'hospitalité envers les étrangers, d'avoir lavé les pieds des saints, d'avoir secouru les affligés, et de s'être ainsi constamment appliquée à toutes sortes de bonnes œuvres.
\VS{11}Mais refuse les veuves qui sont plus jeunes~; car quand elles sont devenues lascives\FTNT{Ce mot vient du grec «~katastreniao~»~: «~ressentir les pulsions du désir sexuel~».} contre Christ, elles veulent se marier,
\VS{12}ayant leur condamnation, en ce qu'elles ont violé leur première foi.
\VS{13}Et avec cela aussi, étant oisives, elles apprennent à aller de maison en maison~; et non seulement elles sont oisives, mais encore causeuses, et curieuses, et parlant de choses qui ne sont pas bienséantes.
\VS{14}Je veux donc que les jeunes veuves se marient, qu'elles aient des enfants, qu'elles gouvernent leur ménage, et qu'elles ne donnent à l'adversaire aucune occasion de médire.
\VS{15}Car quelques-unes se sont déjà détournées pour suivre Satan.
\VS{16}Si quelque fidèle, homme ou quelque femme, a des veuves, qu'ils les assistent, et que l'église n'en soit point chargée, afin qu'elle puisse assister celles qui sont véritablement veuves.
\TextTitle{Recommandations concernant les anciens}
\VS{17}Que les anciens qui dirigent\FTNT{Du grec «~proistemi~»~: «~disposer~» ou «~placer devant~», «~diriger~», «~présider~» (1 Th. 5:12~; Ro. 12:8~; \vref{1 Ti. 3:4-5,12}.} convenablement soient jugés dignes d'un double honneur, spécialement ceux qui travaillent à la prédication et à l'enseignement.
\VS{18}Car l'Ecriture dit~: Tu n'emmuselleras point le bœuf quand il foule le grain\FTNT{De. 25:4.}. Et l'ouvrier mérite son salaire\FTNT{Lu. 10:7.}.
\VS{19}Ne reçois point d'accusation contre un ancien, si ce n'est sur la déposition de deux ou de trois témoins\FTNT{De. 19:15~; Mt. 18:16~; 2 Co. 13:1}.
\VS{20}Reprends publiquement ceux qui pèchent, afin que les autres aussi en aient de la crainte.
\VS{21}Je te conjure devant Dieu, et devant le Seigneur Jésus-Christ, et devant les anges élus, d'observer ces choses sans préférer l'un à l'autre, et de ne rien faire avec partialité.
\VS{22}N'impose les mains à personne avec précipitation, et ne participe pas aux péchés d'autrui~; toi-même, conserve-toi pur.
\VS{23}Ne bois plus uniquement de l'eau~; mais use d'un peu de vin, à cause de ton estomac et de tes fréquentes maladies.
\VS{24}Les péchés de certains hommes sont manifestes, même avant tout jugement, alors que chez d'autres, ils ne se découvrent qu'après.
\VS{25}De même, les bonnes œuvres sont manifestes, et celles qui ne les sont pas ne peuvent pas rester cachées\FTNT{Mt. 10:26~; Mc. 4:22~; Lu. 8:17~; Lu. 12:2.}.
\Chap{6}
\TextTitle{L'attitude du serviteur envers son maître}
\VerseOne{}Que tous les esclaves qui sont sous le joug sachent qu'ils doivent à leurs maîtres toute sorte d'honneur, afin qu'on ne blasphème pas le Nom de Dieu et sa doctrine.
\VS{2}Et que ceux qui ont des fidèles pour maîtres ne les méprisent point sous prétexte qu'ils sont leurs frères, mais qu'ils les servent d'autant mieux que ce sont des fidèles et des bien-aimés de Dieu, étant participants de la grâce. Enseigne ces choses et recommande-les.
\VS{3}Si quelqu'un enseigne des fausses doctrines, et ne se soumet pas aux saines paroles de notre Seigneur Jésus-Christ, et à la doctrine qui est selon la piété,
\VS{4}il est enflé d'orgueil, il ne sait rien~; mais il a la maladie des questions et des disputes de mots, d'où naissent l'envie, les querelles, les médisances et les mauvais soupçons,
\VS{5}les vaines disputes d'hommes corrompus d'entendement et privés de la vérité, qui estiment que la piété est un moyen de gagner. Sépare-toi de ces sortes de gens.
\TextTitle{L'amour de l'argent~: La racine de tous les maux}
\VS{6}Or la piété avec le contentement d'esprit est un grand gain.
\VS{7}Car nous n'avons rien apporté dans le monde, et aussi il est évident que nous n'en pouvons rien emporter.
\VS{8}Si nous avons la nourriture et le vêtement, cela nous suffira.
\VS{9}Mais ceux qui veulent devenir riches tombent dans la tentation\FTNT{La tentation se rapporte à l'envie de toujours posséder, de s'enrichir et de gagner plus d'argent. Cela finit par faire tomber les gens dans l'orgueil, le mensonge, la duplicité, dans la fornication, etc.}, dans le piège\FTNT{Le mot «~piège~» vient du grec «~pagis~» qui donne en français «~trappe~», «~filet~». «~Car il surprendra comme un filet tous ceux qui habitent sur la surface de toute la terre.~» Lu. 21:35. Ce mot suggère l'inattendu, l'improviste, la surprise, car les oiseaux et autres animaux pris dans le filet sont attrapés par surprise. Les conséquences de la cupidité sont nombreuses, notamment le mensonge et l'adultère. En effet, une personne cupide finit en général par tromper son conjoint.}, et dans beaucoup de désirs insensés et pernicieux\FTNT{Les désirs insensés et pernicieux sont multiples~: l'envie de toujours posséder plus que les autres, la convoitise, les rivalités, la concurrence, la folie des grandeurs. Ces choses sortent les gens de la vision du Seigneur (Mc. 4:19).} qui plongent les hommes dans la ruine et la perdition\FTNT{La ruine et la perdition. Une personne cupide se perd en s'éloignant du Seigneur (2 Pi. 2). Selon Salomon, l'argent ne rassasie personne. «~Celui qui aime l'argent n'est point rassasié par l'argent, et celui qui aime un grand train, n'en est pas nourri~; cela aussi est une vanité.~» (Ec. 5:9). Selon les Ecritures, le système bancaire mondial s'écroulera dans les prochaines années (Ap. 18).}.
\VS{10}Car l'amour de l'argent est la racine de tous les maux\FTNT{L'amour de l'argent est la racine de tous les maux. Ceux qui espèrent en une sécurité divine doivent renoncer à la sécurité matérielle et financière que la chair désire.}~; et quelques-uns en étant possédés, se sont détournés de la foi et se sont jetés eux-mêmes dans bien des tourments.
\VS{11}Mais toi, homme de Dieu~! Fuis ces choses, et recherche la justice, la piété, la foi, la charité, la patience, la douceur.
\VS{12}Combats le bon combat de la foi, saisis la vie éternelle, à laquelle aussi tu as été appelé, et pour laquelle tu as fait une belle confession en présence de plusieurs témoins.
\VS{13}Je t'ordonne, devant Dieu qui donne la vie à toutes choses, et devant Jésus-Christ qui a fait cette belle confession devant Ponce Pilate,
\VS{14}de garder ce commandement, en te conservant sans tache et irrépréhensible, jusqu'à l'apparition de notre Seigneur Jésus-Christ,
\VS{15}qui sera manifesté en son temps, qui est le Béni et seul Prince, le Roi des rois, et le Seigneur des seigneurs,
\VS{16}qui seul possède l'immortalité, et qui habite une lumière inaccessible, que nul homme n'a vu ni ne peut voir, à qui appartiennent l'honneur et la puissance éternelle. Amen~!
\VS{17}Ordonne à ceux qui sont riches dans ce monde, qu'ils ne soient pas hautains, et qu'ils ne mettent pas leur confiance dans l'incertitude des richesses, mais dans le Dieu vivant, qui nous donne toutes choses abondamment pour en jouir.
\VS{18}Qu'ils fassent du bien, qu'ils soient riches en bonnes œuvres, qu'ils soient prompts à donner, avec libéralité,
\VS{19}s'amassant ainsi pour l'avenir un trésor placé sur un fondement solide, afin qu'ils obtiennent la vie éternelle.
\TextTitle{Conclusion}
\VS{20}Timothée, garde le dépôt, en fuyant les discours vains et profanes, et les contradictions d'une science faussement ainsi nommée,
\VS{21}dont font profession quelques-uns qui se sont détournés de la foi. Que la grâce soit avec toi~! Amen~!
\PPE{}
\end{multicols}

\clearpage\ShortTitle{Tit.}\BookTitle{Tite}\BFont
\noindent\hrulefill
{\footnotesize
\textit{
\bigskip
{\centering{}
\\Auteur : Paul
\\(Gr. : Titos)
\\Signifie : Nourrice, honorable
\\Thème : L'ordre dans les églises
\\Date de rédaction : Env. 65 ap. J.-C.\\}
}
%\bigskip
\textit{
\\Cette épître pastorale fut écrite après la libération de Paul de sa première captivité romaine, peut-être dans la ville de Philippes. Tite, disciple d'origine païenne et collaborateur de Paul, se trouvait alors en Crète où Paul l'avait laissé afin qu'il organise les églises. Dans cette lettre, l'apôtre traite des conditions requises pour assumer la charge d'ancien en mettant l'accent sur la saine doctrine. Mentionnant également les obligations morales des jeunes, des personnes âgées, ainsi que des serviteurs, il exhorte Tite à veiller et à s'éloigner des apostats.\bigskip
}
}
\par\nobreak\noindent\hrulefill
\begin{multicols}{2}
\Chap{1}
\TextTitle{Introduction}
\VerseOne{}Paul, serviteur de Dieu, et apôtre de Jésus-Christ, selon la foi des élus de Dieu et la connaissance de la vérité qui est selon la piété,
\VS{2}dans l'espérance de la vie éternelle, que Dieu, qui ne peut mentir, avait promise avant les temps éternels,
\VS{3}mais qu'il a manifestée en son propre temps par sa parole, dans la prédication qui m'a été confiée, par le commandement de Dieu notre Sauveur,
\VS{4}à Tite mon vrai fils, selon la foi qui nous est commune: Que la grâce, la miséricorde, et la paix te soient données de la part de Dieu notre Père, et de la part du Seigneur Jésus-Christ, notre Sauveur !
\TextTitle{Les caractéristiques d'un ancien}
\VS{5}La raison pour laquelle je t'ai laissé en Crète, c'est afin que tu achèves de mettre en bon ordre les choses qui restent à régler, et que tu établisses des anciens de ville en ville, suivant ce que je t'ai ordonné,
\VS{6}s'il s'y trouve un homme qui soit irrépréhensible, mari d'une seule femme, ayant des enfants fidèles, qui ne soient ni accusés de dissolution, ni rebelles.
\VS{7}Car il faut que l'évêque soit irrépréhensible, comme étant économe dans la maison de Dieu ; qu'il ne soit ni arrogant, ni coléreux, ni adonné au vin, ni violent, non convoiteux d'un gain déshonnête ;
\VS{8}mais hospitalier, aimant les gens de bien, sage, juste, saint, tempérant,
\VS{9}attaché à la parole de la vérité comme elle lui a été enseignée, afin qu'il soit capable tant d'exhorter par la saine doctrine, que de réfuter les contredisants\FTNT{Lu. 2:34 ; Jn. 19:12 ; Ac. 13:45 ; 28:19 ; 28:22 ; Ro. 10:21.}.
\VS{10}Car il y en a plusieurs qui ne veulent pas se soumettre, vains discoureurs, et séducteurs d'esprits, principalement ceux qui sont de la circoncision,
\VS{11}auxquels il faut fermer la bouche, et qui renversent les maisons tout entières enseignant pour un gain déshonnête des choses qu'on ne doit point enseigner.
\VS{12}Quelqu'un d'entre eux, qui était leur propre prophète, a dit : Les Crétois sont toujours menteurs, de mauvaises bêtes, des ventres paresseux.
\VS{13}Ce témoignage est véritable. C'est pourquoi reprends-les vivement, afin qu'ils soient sains dans la foi,
\VS{14}et qu'ils ne s'attachent point aux fables judaïques et aux commandements d'hommes qui se détournent de la vérité.
\VS{15}Toutes choses sont bien pures pour ceux qui sont purs, mais rien n'est pur pour les impurs et les infidèles ; mais leur entendement et leur conscience sont souillés.
\VS{16}Ils font profession de connaître Dieu, mais ils le renient par leurs œuvres, car ils sont abominables, et rebelles, et réprouvés pour toute bonne œuvre.
\Chap{2}
\TextTitle{Recommandations de Paul à Tite}
\VerseOne{}Mais toi, annonce les choses qui conviennent à la saine doctrine.
\VS{2}Que les vieillards soient sobres, honnêtes, prudents, sains dans la foi, dans la charité, et dans la patience.
\VS{3}De même, que les femmes âgées règlent leur extérieur d'une manière convenable à la sainteté ; qu'elles ne soient ni médisantes, ni sujettes à beaucoup de vin, mais qu'elles enseignent de bonnes choses,
\VS{4}afin qu'elles instruisent les jeunes femmes à être modestes, à aimer leurs maris, à aimer leurs enfants,
\VS{5}à être modérées, pures, occupées aux soins domestiques, bonnes, soumises à leurs maris, afin que la Parole de Dieu ne soit point blasphémée.
\VS{6}Exhorte aussi les jeunes hommes à être modérés,
\VS{7}te montrant toi-même un modèle de bonnes œuvres en toutes choses, en une doctrine exempte de toute altération, en pureté, en intégrité,
\VS{8}en paroles saines, que l'on ne puisse point condamner, afin que celui qui vous est contraire, soit rendu confus, n'ayant aucun mal à dire de vous.
\VS{9}Que les serviteurs soient soumis à leurs maîtres, leur complaisant en toutes choses, n'étant point contredisants,
\VS{10}ne dérobant rien de ce qui appartient à leurs maîtres, mais faisant toujours paraître une grande fidélité, afin de rendre honorable en toutes choses la doctrine de Dieu, notre Sauveur.
\VS{11}Car la grâce de Dieu, salutaire à tous les hommes, a été manifestée.
\VS{12}Et elle nous enseigne à renoncer à l'impiété et aux passions mondaines, et à vivre dans le présent siècle, selon la sagesse, la justice et la piété,
\VS{13}en attendant la bienheureuse espérance, et l'apparition de la gloire du grand Dieu et notre Sauveur Jésus-Christ,
\VS{14}qui s'est donné lui-même pour nous, afin de nous racheter de toute iniquité, et de nous purifier, pour lui être un peuple qui lui appartienne en propre, et qui soit zélé pour les bonnes œuvres.
\VS{15}Enseigne ces choses, exhorte, et reprends avec une pleine autorité. Et que personne ne te méprise.
\Chap{3}
\TextTitle{Conseils pratiques de Paul}
\VerseOne{}Rappelle-leur d'être soumis aux magistrats et aux autorités, d'obéir aux gouverneurs, d'être prêts à faire toutes sortes de bonnes actions,
\VS{2}de ne médire de personne, de n'être point querelleurs, mais doux, et montrant une parfaite douceur envers tous les hommes.
\VS{3}Car nous aussi, nous étions autrefois insensés, désobéissants, égarés, asservis à toute espèce de convoitises et de voluptés, vivant dans la méchanceté et dans l'envie, dignes d'être haïs, et nous haïssant les uns les autres.
\VS{4}Mais, quand la bonté de Dieu notre Sauveur et son amour envers les hommes ont été manifestés, il nous a sauvés,
\VS{5}non par des œuvres de justice que nous aurions faites, mais selon la miséricorde, par le bain de la régénération et le renouvellement du Saint-Esprit,
\VS{6}qu'il a répandu abondamment sur nous par Jésus-Christ notre Sauveur,
\VS{7}afin qu'ayant été justifiés par sa grâce, nous soyons les héritiers de la vie éternelle selon notre espérance.
\VS{8}Cette parole est certaine, et je veux que tu affirmes ces choses, afin que ceux qui ont cru en Dieu aient soin principalement de s'appliquer à pratiquer les bonnes œuvres. Voilà les choses qui sont bonnes et utiles aux hommes.
\VS{9}Mais évite les discussions folles, les généalogies, les querelles et les disputes de la loi ; car elles sont inutiles et vaines.
\VS{10}Rejette l'homme hérétique, après le premier et le second avertissement,
\VS{11}sachant qu'un tel homme est perverti, et qu'il pèche en se condamnant lui-même.
\TextTitle{Salutations}
\VS{12}Quand je t'enverrai Artémas ou Tychique, hâte-toi de venir vers moi à Nicopolis ; car j'ai résolu d'y passer l'hiver.
\VS{13}Accompagne soigneusement Zénas, docteur de la loi, et Apollos, afin que rien ne leur manque.
\VS{14}Que les nôtres aussi apprennent à être les premiers à s'appliquer aux bonnes œuvres, pour les usages nécessaires, afin qu'ils ne soient point sans fruits.
\VS{15}Tous ceux qui sont avec moi te saluent. Salue ceux qui nous aiment dans la foi. Grâce soit avec vous tous ! Amen !
\PPE{}
\end{multicols}

\clearpage\ShortTitle{1 Pierre}\BookTitle{1 Pierre}\BFont
\noindent\hrulefill
{\footnotesize
\textit{
\bigskip
{\centering{}
\\Auteur : Pierre
\\(Gr. : Petro)
\\Signification : Roc, pierre
\\Thème : La victoire sur la souffrance
\\Date de rédaction : Env. 65 ap. J.-C.\\}
}
%\bigskip
\textit{
\\Cette lettre semble avoir été écrite à Rome même si Pierre y parlait de « Babylone ». En ces temps de persécutions, les chrétiens devaient être prudents quant à la manière dont ils parlaient du pouvoir en place, c'est pourquoi ils utilisaient souvent des codes. C'est donc durant une période difficile que fut rédigée cette épître qui s'adressait à des églises d'Asie Mineure dont la plupart furent fondées par Paul. Au travers de ces quelques lignes, Pierre exhorte les frères et sœurs à tenir ferme dans la foi malgré les souffrances liées aux épreuves, et les encourage à espérer en Jésus-Christ, leur salut. Il finit cette épître en donnant des conseils quant à l'attitude à avoir au sein de l'église.\bigskip
}
}
\par\nobreak\noindent\hrulefill
\begin{multicols}{2}
\Chap{1}
\TextTitle{Introduction}
\VerseOne{}Pierre, apôtre de Jésus-Christ, à ceux qui sont étrangers et dispersés dans le Pont\FTNT{Le Pont : province formant presque la totalité de l'Asie Mineure.}, la Galatie, la Cappadoce, l'Asie et la Bithynie,
\VS{2}élus selon la prescience de Dieu le Père, par la sanctification de l'Esprit, afin d'obéir à Jésus-Christ, et qu'ils participent à l'aspersion de son sang : Que la grâce et la paix vous soient multipliées !
\TextTitle{Les souffrances du chrétien et sa conduite à la lumière d'un salut parfait}
\VS{3}Béni soit Dieu, et le Père de notre Seigneur Jésus-Christ, qui, par sa grande miséricorde, nous a régénérés, pour une espérance vivante, par la résurrection de Jésus-Christ d'entre les morts,
\VS{4}pour un héritage incorruptible, et qui ne peut ni se souiller, ni se flétrir, qui est conservé dans les cieux pour nous,
\VS{5}qui sommes gardés par la puissance de Dieu, par la foi, afin que nous obtenions le salut, qui est prêt à être révélé dans les derniers temps !
\VS{6}En quoi vous vous réjouissez, quoique vous soyez maintenant affligés pour un peu de temps par diverses épreuves, vu que cela est convenable,
\VS{7}afin que l'épreuve de votre foi, beaucoup plus précieuse que l'or périssable, et qui toutefois est éprouvé par le feu, ait pour résultat la louange, l'honneur et la gloire, lorsque Jésus-Christ sera révélé.
\VS{8}Lequel vous aimez quoique vous ne l'ayez point vu, en qui vous croyez, quoique maintenant vous ne le voyiez pas, et vous vous réjouissez d'une joie ineffable et glorieuse,
\VS{9}remportant la fin de votre foi, savoir le salut de vos âmes.
\VS{10}C'est au sujet de ce salut que les prophètes, qui ont prophétisé touchant la grâce qui vous était destinée, ont fait leurs recherches et leurs investigations.
\VS{11}Ils voulaient sonder l'époque et les circonstances marquées par l'Esprit prophétique de Christ qui était en eux, et qui rendait à l'avance témoignage, leur faisant connaître les souffrances de Christ et la gloire dont elles seraient suivies.
\VS{12}Mais il leur fut révélé que ce n'était pas pour eux-mêmes, mais pour nous, qu'ils administraient ces choses, que vous ont annoncées maintenant ceux qui vous ont prêché l'Evangile par le Saint-Esprit envoyé du ciel, et dans lesquelles les anges désirent plonger leurs regards.
\VS{13}C'est pourquoi, ceignez les reins de votre entendement, soyez sobres, et ayez une entière espérance dans la grâce qui vous est présentée, jusqu'à ce que Jésus-Christ soit révélé\FTNT{Révélé, voir commentaire en 2 Th. 1 :7.}.
\VS{14}Comme des enfants obéissants, ne vous conformez pas à vos convoitises d'autrefois, pendant votre ignorance. 
\VS{15}Mais, comme celui qui vous a appelés est saint, vous aussi de même soyez saints dans toute votre conduite,
\VS{16}selon ce qu'il est écrit : Soyez saints, car je suis saint\FTNT{Lé. 11:44.}.
\VS{17}Et si vous invoquez comme votre Père celui qui juge selon l'œuvre de chacun, sans favoritisme, conduisez-vous avec crainte pendant le temps de votre séjour sur la terre,
\VS{18}sachant que vous avez été rachetés de votre vaine conduite, qui vous avait été enseignée par vos pères, non point par des choses corruptibles, comme par argent, ou par or,
\VS{19}mais par le sang précieux de Christ, comme d'un agneau sans défaut et sans tache,
\VS{20}prédestiné avant la fondation du monde, et manifesté dans les derniers temps, pour vous.
\VS{21}Par lui, vous croyez en Dieu, qui l'a ressuscité des morts et lui a donné la gloire, afin que votre foi et votre espérance reposent sur Dieu.
\VS{22}Ayant donc purifié vos âmes en obéissant à la vérité par le Saint-Esprit, afin que vous ayez un amour fraternel et sans hypocrisie, aimez-vous ardemment les uns les autres d'un cœur pur,
\VS{23}puisque vous avez été régénérés, non par une semence corruptible, mais par une semence incorruptible, par la parole de Dieu qui vit et demeure éternellement.
\VS{24}Car toute chair est comme l'herbe, et toute la gloire de l'homme comme la fleur de l'herbe. L'herbe sèche, et sa fleur tombe ;
\VS{25}mais la parole du Seigneur demeure éternellement\FTNT{Es. 40:6-8.}. Et cette parole est celle qui vous a été annoncée par l'Evangile.
\Chap{2}
\VerseOne{}Ayant donc renoncé à toute sorte de malice, et de toute fraude, et de dissimulation, et d'envie et de toutes médisances,
\VS{2}désirez ardemment, comme des enfants nouveau-nés, le lait spirituel et pur, afin que vous croissiez par lui,
\VS{3}si toutefois vous avez goûté combien le Seigneur est bon.
\VS{4}Et vous approchant de lui, pierre vivante, rejetée par les hommes, mais choisie et précieuse devant Dieu ;
\VS{5}et vous aussi, comme des pierres vivantes, vous êtes édifiés pour être une maison spirituelle, et une sainte sacrificature, afin d'offrir des sacrifices spirituels, agréables à Dieu par Jésus-Christ. 
\VS{6}C'est pourquoi aussi, il est dit dans l'Ecriture : Voici, je mets en Sion la principale pierre\FTNT{Jésus-Christ est la Pierre rejetée par les bâtisseurs. Voir Es. 28:16 ; Ps. 118:22.} de l'angle, choisie et précieuse ; et celui qui croit en elle ne sera point confus.
\VS{7}Elle est donc précieuse pour vous qui croyez. Mais, par rapport aux rebelles, il est dit : La pierre que ceux qui bâtissaient ont rejetée, est devenue la principale de l'angle, 
\VS{8}et une pierre d'achoppement, et un rocher de scandale ; ils se heurtent contre la parole, et sont rebelles et c'est à cela qu'ils sont destinés.
\TextTitle{La position du croyant}
\VS{9}Mais vous, vous êtes la race élue, vous êtes la sacrificature royale, la nation sainte, le peuple acquis, afin que vous annonciez les vertus de celui qui vous a appelés des ténèbres à sa merveilleuse lumière,
\VS{10}vous qui autrefois n'étiez pas son peuple, mais qui maintenant êtes le peuple de Dieu, vous qui n'aviez point obtenu miséricorde, mais qui maintenant avez obtenu miséricorde.
\VS{11}Mes bien-aimés, je vous exhorte, comme étrangers et voyageurs, à vous abstenir des convoitises charnelles qui font la guerre à l'âme.
\VS{12}Ayant une conduite honnête avec les Gentils, afin que, là même où ils vous calomnient comme si vous étiez des malfaiteurs, ils remarquent vos bonnes œuvres, et glorifient Dieu, au jour où il les visitera.
\VS{13}Soyez donc soumis à tout établissement humain, pour l'amour de Dieu : Soit au roi, comme à celui qui est au-dessus des autres,
\VS{14}soit aux gouverneurs, comme à ceux qui sont envoyés de sa part pour punir les méchants et pour honorer les gens de bien.
\VS{15}Car c'est là la volonté de Dieu, qu'en faisant bien vous fermiez la bouche à l'ignorance des hommes insensés.
\VS{16}Comme libres, et non pas comme ayant la liberté pour servir de voile à la méchanceté, mais agissant comme des serviteurs de Dieu.
\VS{17}Honorez tout le monde ; aimez tous vos frères ; craignez Dieu ; honorez le roi.
\VS{18}Serviteurs, soyez soumis en toute crainte à vos maîtres, non seulement à ceux qui sont bons et équitables, mais aussi à ceux qui sont méchants.
\VS{19}Car c'est une chose agréable à Dieu si quelqu'un à cause de la conscience qu'il a envers Dieu, endure des afflictions en souffrant injustement. 
\VS{20}Autrement, quelle gloire en aurez-vous, si lorsque vous péchez et qu’on vous frappe, vous le supportez  patiemment ? Mais si quand vous faites le bien et que vous souffrez, vous le supportez patiemment, voilà où Dieu prend plaisir. 
\TextTitle{Les souffrances de Christ, le Substitut des hommes}
\VS{21}Car vous êtes appelés à cela, vu même que Christ a souffert pour nous, nous laissant un modèle, afin que vous suiviez ses traces, 
\VS{22}lui qui n'a point commis de péché, et dans la bouche duquel il ne s'est point trouvé de fraude ;
\VS{23}qui, lorsqu'on lui disait des outrages, n'en rendait point, et quand on lui faisait du mal, n'usait point de menaces, mais il se remettait à celui qui juge justement ; 
\VS{24}lui qui a porté lui-même nos péchés en son corps sur le bois, afin qu'étant morts aux péchés nous vivions pour la justice ; lui par la meurtrissure\FTNT{Es. 53:5.} duquel même vous avez été guéris.
\VS{25}Car vous étiez comme des brebis errantes, mais maintenant vous êtes convertis au Pasteur et à l'Evêque de vos âmes. 
\Chap{3}
\TextTitle{La conduite chrétienne à la maison et à l'église}
\VerseOne{}Femmes, soyez de même soumises à vos maris, afin que, si quelques-uns n'obéissent point à la parole, ils soient gagnés sans paroles par la conduite de leurs femmes,
\VS{2}lorsqu'ils verront la pureté de votre conduite, accompagnée de crainte.
\VS{3}Et que votre ornement ne soit point celui de dehors, qui consiste dans la frisure des cheveux, et dans une parure d'or, et dans la magnificence des habits,
\VS{4}mais que votre parure consiste dans l’homme caché dans le cœur, c’est-à-dire dans l’incorruptibilité d’un esprit doux et paisible, qui est d’un grand prix devant Dieu.
\VS{5}Car c'est ainsi que se paraient aussi autrefois les saintes femmes qui espéraient en Dieu, étant soumises à leurs maris,
\VS{6}comme Sara, qui obéissait à Abraham et l'appelait son seigneur. C'est d'elle que vous êtes devenues les filles, en faisant ce qui est bien et sans vous laisser troubler par aucune crainte.
\VS{7}Et vous, maris, de même comportez-vous selon la sagesse avec vos femmes, comme un vase\FTNT{Paul utilise une métaphore connu des grecs pour parler du corps : le vase.} plus fragile, c'est-à-dire, féminin ; leur portant honneur comme étant aussi ensemble héritiers de la grâce de la vie, afin que vos prières ne soient pas interrompues. 
\VS{8}Enfin, soyez tous d'un même sentiment, remplis de compassion les uns envers les autres, d'amour fraternel, miséricordieux et doux.
\VS{9}Ne rendez point mal pour mal, ou injure pour injure\FTNT{Mt. 5:44.} ; mais, au contraire, bénissez ; sachant que c'est à cela que vous êtes appelés, afin d'hériter la bénédiction.
\VS{10}Car celui qui veut aimer sa vie et voir des jours heureux, qu'il préserve sa langue du mal, et ses lèvres de prononcer aucune fraude,
\VS{11}qu'il se détourne du mal, et fasse le bien, qu'il recherche la paix, et qu'il tâche de se la procurer ;
\VS{12}car les yeux du Seigneur sont sur les justes, et ses oreilles sont attentives à leurs prières, mais la face du Seigneur est contre ceux qui se conduisent mal.
\TextTitle{La conduite chrétienne aux yeux du monde}
\VS{13}Et qui vous maltraitera, si vous êtes les imitateurs de celui qui est bon ?
\VS{14}Que si toutefois vous souffrez quelque chose pour la justice, vous êtes bienheureux. Mais ne craignez point les maux dont ils veulent vous faire peur, et n'en soyez point troublés ;
\VS{15}mais sanctifiez le Seigneur dans vos coeurs, et soyez toujours prêts à répondre, avec douceur et avec respect, à chacun qui vous demande raison de l'espérance qui est en vous, 
\VS{16}et ayant une bonne conscience, afin que, ceux qui blâment votre bonne conduite en Christ, soient confus de ce qu'ils médisent de vous, comme si vous étiez des malfaiteurs.
\VS{17}Car il vaut mieux, si telle est la volonté de Dieu, que vous souffriez en faisant le bien qu’en faisant le mal.
\TextTitle{Les souffrances de Christ}
\VS{18}Car aussi Christ a souffert une fois pour les péchés, lui juste pour les injustes, afin de nous amener à Dieu, étant mort en la chair, mais vivifié par l'Esprit,
\VS{19}par lequel aussi étant allé, il a  prêché aux esprits qui sont en prison\FTNT{La possibilité du salut après la mort n’a aucun fondement biblique (Hé. 9 :27). Dans ce passage, il est fait mention des pécheurs qui ont vécu du temps de Noé et auxquels le Seigneur Jésus a confirmé la condamnation lorsqu’il est descendu dans l’Hadès ( l'enfer; Ep. 4 :9). Voir aussi commentaire en Mt 16 :18.},
\VS{20}et qui avaient été autrefois incrédules, quand la patience de Dieu les attendait, durant les jours de Noé, tandis que l'arche se préparait dans laquelle un petit nombre, à savoir huit personnes furent sauvées par l'eau.
\VS{21}A quoi aussi maintenant répond la figure qui nous sauve, c'est-à-dire, le baptême ; non point celui par lequel les ordures de la chair sont nettoyées, mais la promesse faite à Dieu d'une conscience pure, par la résurrection de Jésus-Christ,
\VS{22}qui est à la droite de Dieu, étant allé au Ciel, et auquel sont assujettis les anges, et les dominations et les puissances.
\Chap{4}
\TextTitle{Souffrir dans la chair}
\VerseOne{}Puisque Christ a souffert pour nous dans la chair, vous aussi armez-vous de la même pensée. Car celui qui a souffert dans la chair a cessé de pécher,
\VS{2}afin de vivre, non plus selon les convoitises des hommes, mais selon la volonté de Dieu, pendant le temps qui lui reste à vivre dans la chair.
\VS{3}Car il nous suffit d'avoir accompli la volonté des Gentils, pendant le temps de notre vie passée, quand nous nous abandonnions aux impudicités, aux convoitises, à l'ivrognerie, aux excès dans le manger et dans le boire, et aux idolâtries abominables.
\VS{4}Ce que ces Gentils trouvent fort étrange, ils vous calomnient de ce que vous ne courez pas avec eux dans un même débordement de dissolution. 
\VS{5}Mais ils rendront compte à celui qui est prêt à juger les vivants et les morts.
\VS{6}Car c'est aussi pour cela que les morts ont été évangélisés, afin qu'ils soient jugés selon les hommes dans la chair, et qu'ils vivent selon Dieu dans l'esprit. 
\TextTitle{La conduite chrétienne dans le temps présent}
\VS{7}Or la fin de toutes choses est proche : Soyez donc sobres et vigilants pour prier. 
\VS{8}Mais surtout, ayez les uns pour les autres une ardente charité, car la charité couvre une multitude de péchés.
\VS{9}Soyez hospitaliers les uns envers les autres, sans murmures. 
\VS{10}Que chacun selon le don qu'il a reçu, l'emploie pour le service des autres, comme de bons gestionnaires des diverses grâces de Dieu. 
\VS{11}Si quelqu'un parle, qu'il parle comme annonçant les paroles de Dieu ; si quelqu'un administre, qu'il administre comme par la puissance que Dieu lui en a fournie, afin qu'en toutes choses Dieu soit glorifié par Jésus-Christ, auquel appartient la gloire et la force, aux siècles des siècles. Amen ! 
\VS{12}Mes bien-aimés, ne trouvez point étrange quand vous êtes comme dans une fournaise pour votre épreuve, comme s'il vous arrivait quelque chose d'extraordinaire. 
\VS{13}Mais réjouissez-vous, de ce que vous participez aux souffrances de Christ, afin qu'aussi à la révélation de sa gloire, vous vous réjouissiez avec allégresse.
\VS{14}Si on vous dit des injures pour le Nom de Christ, vous êtes heureux, car l'Esprit de gloire et de Dieu repose sur vous, lequel est blasphémé par ceux qui vous noircissent mais pour vous, vous le glorifiez.
\VS{15}Que nul de vous, ne souffre comme meurtrier, ou voleur, ou malfaiteur, ou curieux des affaires d'autrui. 
\VS{16}Mais si quelqu'un souffre comme chrétien, qu'il n'en ait point de honte, mais qu'il glorifie Dieu en cela.
\VS{17}Car il est temps que le jugement commence par la maison de Dieu\FTNT{Le jugement commence par la maison de Dieu. Ez. 9:1-11.}. Or s'il commence premièrement par nous, quelle sera la fin de ceux qui n'obéissent pas à l'Evangile de Dieu ?
\VS{18}Et si le juste est difficilement sauvé, où comparaîtra le méchant et le pécheur ? 
\VS{19}Que ceux-là donc aussi, qui souffrent par la volonté de Dieu, puisqu'ils font ce qui est bon lui recommandent leurs âmes, comme au fidèle Créateur. 
\Chap{5}
\TextTitle{Servir sans rien attendre en retour}
\VerseOne{}Je prie les anciens qui sont parmi vous, moi qui suis ancien avec eux, et témoin des souffrances de Christ, et participant de la gloire qui doit être révélée et je leur dis : 
\VS{2}Paissez le troupeau de Dieu qui vous est commis, en prenant garde sur lui, non point par contrainte, mais volontairement ; non point pour un gain déshonnête, mais par un principe d'affection. 
\VS{3}Et non pas comme ayant domination sur les héritages du Seigneur, mais de telle manière que vous soyez les modèles du troupeau. 
\VS{4}Et quand le souverain Pasteur\FTNT{Jésus est notre Souverain Pasteur. Voir Ps. 23 ; Jn. 10.} apparaîtra, vous obtiendrez la couronne incorruptible de la gloire.
\VS{5}De même, vous jeunes gens, soyez soumis aux anciens. Et ayant tous de la soumission les uns pour les autres, soyez parés par-dedans d'humilité; parce que Dieu résiste aux orgueilleux, mais il fait grâce aux humbles. 
\VS{6}Humiliez-vous donc sous la puissante main de Dieu, afin qu'il vous élève quand le temps sera venu ;
\VS{7}remettez-lui tout ce qui peut vous inquiéter, car il prend soin de vous.
\VS{8}Soyez sobres et veillez : Car le diable, votre adversaire, tourne autour de vous comme un lion rugissant, cherchant qui il pourra dévorer. 
\VS{9}Résistez-lui donc en demeurant fermes dans la foi, sachant que les mêmes souffrances s'accomplissent dans la compagnie de vos frères qui sont dans le monde. 
\TextTitle{Salutations}
\VS{10}Or que le Dieu de toute grâce, qui nous a appelés à sa gloire éternelle en Jésus-Christ, après que vous aurez souffert un peu de temps, vous rende parfaits, vous affermisse, vous fortifie et vous établisse. 
\VS{11}A lui soient la gloire et la force, aux siècles des siècles ! Amen !
\VS{12}Je vous ai écrit brièvement par Silvain, notre frère, que je crois vous être fidèle, vous déclarant et vous protestant que la grâce de Dieu dans laquelle vous êtes est la véritable. 
\VS{13}L'Eglise qui est à Babylone, élue avec vous, et Marc, mon fils, vous saluent. 
\VS{14}Saluez-vous les uns les autres par un baiser de charité. Que la paix soit avec vous tous qui êtes en Jésus-Christ ! Amen !
\PPE{}
\end{multicols}

\clearpage\ShortTitle{2 Pierre}\BookTitle{2 Pierre}\BFont
\begin{multicols}{2}
\TextTitle{[Introduction]}
\Chap{1}
\VerseOne{}Simon Pierre, serviteur et apôtre de Jésus-Christ, à vous qui avez reçu en partage une foi du même prix que la nôtre, par la justice de notre Dieu et Sauveur Jésus-Christ (1).
\VS{2}Que la grâce et la paix vous soient multipliées, par la connaissance de Dieu, et de notre Seigneur Jésus.
\TextTitle{[Les grandes vertus chrétiennes]}
\VS{3}Puisque sa divine puissance nous a donné tout ce qui appartient à la vie et à la piété, par la connaissance de celui qui nous a appelés par sa gloire et par sa vertu,
\VS{4}par lesquelles nous sont données les grandes et précieuses promesses, afin que par elles vous soyez faits participants de la nature divine, en fuyant la corruption qui règne dans le monde par la convoitise.
\VS{5}A cause de cela même, faites tous vos efforts pour ajouter la vertu à votre foi ; à la vertu, la connaissance,
\VS{6}à la connaissance, la tempérance, à la tempérance, la patience, à la patience, la piété,
\VS{7}à la piété, l'amour fraternel, et à l'amour fraternel, la charité.
\VS{8}Car si ces choses sont en vous, et y abondent, elles ne vous laisseront point oisifs ni stériles pour la connaissance de notre Seigneur Jésus-Christ.
\VS{9}Mais celui en qui ces choses ne se trouvent point est aveugle, et ne voit point de loin, ayant oublié la purification de ses anciens péchés.
\VS{10}C'est pourquoi, mes frères, efforcez-vous plutôt à affermir votre vocation, et votre élection ; car en faisant cela vous ne broncherez jamais.
\VS{11}Car par ce moyen, l'entrée au Royaume éternel de notre Seigneur et Sauveur Jésus-Christ vous sera abondamment accordée.
\TextTitle{[Sollicitude de l'Apôtre pour ses lecteurs ; autorité de son témoignage et de la parole des prophètes]}
\VS{12}C'est pourquoi je ne négligerai pas de vous rappeler sans cesse ces choses, quoique vous ayez de la connaissance, et que vous soyez fondés dans la vérité présente.
\VS{13}Car je crois qu'il est juste que je vous réveille par des avertissements, pendant que je suis dans cette tente (2),
\VS{14}sachant que dans peu de temps je dois la quitter, comme notre Seigneur Jésus-Christ lui-même me l'a déclaré.
\TextTitle{[Souvenir de la transfiguration]}
\VS{15}Mais j'aurai soin qu’après mon départ vous puissiez toujours vous souvenir de ces choses.
\VS{16}Car ce n’est pas en suivant des fables composées avec artifice, que nous vous avons fait connaître la puissance et l’avènement (3) de notre Seigneur Jésus-Christ, mais comme ayant vu sa majesté de nos propres yeux.
\VS{17}Car il reçut de Dieu le Père, honneur et gloire, lorsque cette voix lui fut adressée du milieu de la gloire magnifique : Celui-ci est mon Fils bien-aimé, en qui j'ai mis toute mon affection (4).
\VS{18}Et nous entendîmes cette voix envoyée du ciel, lorsque nous étions avec lui sur la sainte montagne.
\TextTitle{[Témoignage à la véracité des Ecritures prophétiques]}
\VS{19}Nous avons aussi la parole des prophètes qui est très ferme, à laquelle vous faites bien d'être attentifs, comme à une lampe qui brille dans un lieu obscur, jusqu'à ce que le jour vienne à paraître et que l'étoile du matin (5) se lève dans vos cœurs.
\VS{20}Sachant premièrement ceci, qu'aucune prophétie de l'Ecriture ne procède d’une interprétation particulière.
\VS{21}Car la prophétie n'a jamais été autrefois apportée par la volonté humaine, mais les saints hommes de Dieu étant poussés par le Saint-Esprit, ont parlé.
\TextTitle{[Avertissement contre les faux docteurs]}
\Chap{2}
\VerseOne{}Mais comme il y a eu de faux prophètes parmi le peuple, il y aura aussi parmi vous de faux docteurs, qui introduiront secrètement des sectes pernicieuses, et qui reniant le Seigneur qui les a rachetés, attireront sur eux-mêmes une ruine soudaine.
\VS{2}Et plusieurs suivront leurs sectes de perdition, et à cause d'eux, la voie de la vérité sera blasphémée.
\VS{3}Par cupidité, ils trafiqueront (1) de vous au moyen de paroles déguisées, mais la condamnation qui leur est destinée depuis longtemps ne tarde point, et leur perdition ne sommeille point.
\VS{4}Car si Dieu n'a pas épargné les anges qui ont péché, s’il les a précipités dans l'abîme (2), les a liés avec des chaînes d'obscurité, les a livrés pour y être gardés jusqu’au jugement ;
\VS{5}et s'il n'a point épargné l’ancien monde, mais a gardé Noé (3), lui huitième, qui était le prédicateur de la justice ; et a fait venir le déluge sur le monde des impies ;
\VS{6}et s'il a condamné à la destruction totale les villes de Sodome et de Gomorrhe, les réduisant en cendres, et les mettant pour être un exemple à ceux qui vivraient dans l'impiété ;
\VS{7}et s'il a délivré le juste Lot (4), qui cruellement affligé de la conduite de ces hommes sans frein, eut beaucoup à souffrir de ces abominables par leur infâme conduite ;
\VS{8}car cet homme juste, qui habitait au milieu d’eux, affligeait chaque jour son âme juste, à cause de ce qu’il voyait et entendait dire de leurs méchantes actions.
\VS{9}Le Seigneur sait ainsi délivrer de l’épreuve les hommes pieux, et réserver les injustes pour être punis au jour du jugement ;
\VS{10}principalement ceux qui vont après la chair, dans la passion de l'impureté, et qui méprisent l’autorité. Gens audacieux et arrogants, ils ne craignent point d’injurier les gloires ;
\VS{11}alors que les anges qui sont supérieurs en force et en puissance, ne prononcent point contre elles de jugement blasphématoire devant le Seigneur ;
\VS{12}Mais eux, semblables à des bêtes brutes, qui s’abandonnent à leurs penchants naturels, et qui sont nées pour être prises et détruites, ils parlent d’une manière blasphématoire de ce qu’ils ignorent, et ils périront par leur propre corruption.
\VS{13}Et ils recevront la récompense de leur iniquité. Ils aiment à être tous les jours dans les délices. Ce sont des taches et des souillures, et ils font leurs délices de leurs tromperies dans les repas qu'ils font avec vous.
\VS{14}Ils ont les yeux pleins d'adultère, ils ne cessent jamais de pécher, ils attirent les âmes mal affermies ; ils ont le cœur exercé à la cupidité, ce sont des enfants de malédiction,
\TextTitle{[Caractéristiques des faux docteurs
\\a. ils ressemblent à Balaam]}
\VS{15}qui ayant laissé le droit chemin, se sont égarés, et ont suivi la voie de Balaam (5), fils de Bosor, qui aima le salaire de l'iniquité ; mais il fut repris pour sa transgression.
\VS{16}Car une ânesse muette parlant d'une voix humaine, arrêta la folie du prophète.
\TextTitle{[b. ils sont dépourvus dIntroduction]}
\VS{17}Ce sont des fontaines sans eau, des nuées agitées par le tourbillon, et des gens à qui l'obscurité des ténèbres est réservée éternellement.
\TextTitle{[c. leurs discours sont savants et prétentieux]
\\cp. 1 Co. 2:1-5)}
\VS{18}Car en prononçant des discours fort enflés de vanité, ils amorcent par les convoitises de la chair, et par leurs impudicités, ceux qui s'étaient véritablement retirés de ceux qui vivent dans l’égarement ;
\TextTitle{[d. ils corrompent la liberté chrétienne]}
\VS{19}ils leur promettent la liberté, quand ils sont eux-mêmes esclaves de la corruption ; car chacun est esclave de ce qui a triomphé de lui.
\VS{20}En effet, si après s’être retirés des souillures du monde, par la connaissance du Seigneur et Sauveur Jésus-Christ, ils s’y engagent de nouveau et sont vaincus, leur dernière condition est pire que la première (6).
\VS{21}Car mieux valait pour eux n'avoir pas connu la voie de la justice, que de l'avoir connue et se détourner du saint commandement qui leur avait été donné.
\TextTitle{[e. ils retournent à leurs premiers péchés]}
\VS{22}Mais ce qu'on dit par un proverbe véritable leur est arrivé : Le chien est retourné à ce qu'il avait vomi ; et la truie lavée est retournée se vautrer dans le bourbier.
\TextTitle{[Le but de l'Epitre]}
\Chap{3}
\VerseOne{}Mes bien-aimés, c'est ici la seconde lettre que je vous écris, afin de réveiller, dans l'une et dans l'autre, par mes avertissements, les sentiments purs que vous avez.
\VS{2}Et afin que vous vous souveniez des paroles qui ont été dites auparavant par les saints prophètes, et du commandement que vous avez reçu de nous, qui sommes apôtres du Seigneur et Sauveur.
\VS{3}Sur toutes choses, sachez qu'aux derniers jours (1) il viendra des moqueurs, se conduisant selon leurs propres convoitises,
\TextTitle{[La seconde venue de Christ et le jour du Seigneur
\\a. l'incrédulité sera générale quant au retour de Christ]}
\VS{4}et disant : Où est la promesse de son avènement ? Car depuis que les pères sont morts, toutes choses demeurent comme elles ont été dès le commencement de la création.
\VS{5}Car ils ignorent volontairement ceci, c’est que les cieux furent autrefois créés par la parole de Dieu, et que la terre est sortie de l'eau, et qu'elle subsiste parmi l'eau ;
\VS{6}et que par ces choses-là, le monde d'alors périt, étant submergé par les eaux du déluge (2).
\VS{7}Mais les cieux et la terre d’à présent sont gardés par la même parole, étant réservés pour le feu au jour du jugement, et de la destruction des hommes impies.
\VS{8}Mais vous mes bien-aimés, n'ignorez pas ceci, qu'un jour est devant le Seigneur comme mille ans, et mille ans comme un jour (3).
\VS{9}Le Seigneur ne retarde point l'exécution de sa promesse, comme quelques-uns croient qu'il y ait du retard, mais il est patient envers nous, ne voulant qu'aucun ne périsse, mais que tous se repentent.
\TextTitle{[b) la purification des cieux et de la terre]}
\VS{10}Or le jour (4) du Seigneur viendra comme un voleur dans la nuit, et en ce jour-là, les cieux passeront avec le bruit d’une effroyable tempête, et les éléments seront dissous par l'ardeur du feu ; et la terre avec toutes les œuvres qu’elle renferme sera brûlée entièrement.
\VS{11}Puisque toutes ces choses doivent se dissoudre, quelles ne doivent pas être la sainteté de votre conduite et votre piété.
\VS{12}En attendant, et en hâtant par vos désirs la venue du jour de Dieu, par lequel les cieux étant enflammés seront dissous, et les éléments se fondront par l'ardeur du feu.
\VS{13}Mais nous attendons, selon sa promesse, de nouveaux cieux et une nouvelle terre (5), où la justice habitera.
\VS{14}C'est pourquoi, mes bien-aimés, en attendant ces choses, appliquez-vous à être trouvés par lui sans tache et sans reproche dans la paix.
\VS{15}Et croyez que la longue patience de notre Seigneur est la preuve qu'il veut votre salut ; comme Paul, notre frère bien-aimé, vous l’a aussi écrit selon la sagesse qui lui a été donnée ;
\VS{16}comme il le fait aussi dans toutes ses lettres, où il parle de ces points, dans lesquels il y a des choses difficiles à comprendre, dont les personnes ignorantes et mal affermies tordent le sens (6), comme celui des autres Ecritures, pour leur propre perdition.
\TextTitle{[Conclusion]}
\VS{17}Vous donc mes bien-aimés, puisque vous êtes déjà avertis, prenez garde qu'étant emportés avec les autres par la séduction des abominables, vous ne veniez à déchoir de votre fermeté.
\VS{18}Mais croissez dans la grâce et dans la connaissance de notre Seigneur et Sauveur Jésus-Christ. A lui soit la gloire maintenant, et jusqu'au jour d'éternité ! Amen !
\PPE{}
\end{multicols}

\clearpage\ShortTitle{2 Timothée}\BookTitle{2 Timothée}\BFont
\noindent\hrulefill
{\footnotesize
\textit{
\bigskip
{\centering{}
\\Signifie : Qui adore ou honore Dieu
\\Thème : Le maintien de la vérité
\\Auteur : Paul
\\Date de rédaction : Env. 67\\}
}
%\bigskip
\textit{
\\Cette lettre s’adresse à Timothée dont le père était grec et la mère juive. Le jeune homme se convertit à Christ avec sa mère et sa grand-mère dès le premier voyage missionnaire de Paul au cours duquel il passa à Lystre.
%\bigskip
\\Paul écrit cette épître pastorale en prison à Rome, après avoir été arrêté dans une province orientale à Ephèse ou Troas.  Ses conditions de détentions étant plus rudes que la première fois, Paul restait dubitatif quant à sa mise en liberté. Il demanda donc à Timothée, son fils dans la foi et fidèle compagnon d’œuvre, de le rejoindre à Rome afin semble-t-il de recevoir ses dernières volontés. Après avoir exposé à Timothée les qualités et les devoirs d’un bon serviteur de l’évangile, il l’encouragea à lutter contre les faux docteurs et l’apostasie en prêchant la Parole en toutes circonstances.\bigskip
}
}
\par\nobreak\noindent\hrulefill
\begin{multicols}{2}
\TextTitle{[Introduction]}
\Chap{1}
\VerseOne{}Paul, apôtre de Jésus-Christ, par la volonté de Dieu, selon la promesse de la vie qui est en Jésus-Christ.
\VS{2}A Timothée, mon fils bien-aimé, que la grâce, la miséricorde et la paix te soient données de la part de Dieu le Père, et de la part de Jésus-Christ notre Seigneur.
\TextTitle{[Paul encourage Timothée]}
\VS{3}Je rends grâces à Dieu, que mes ancêtres ont servi et que je sers avec une conscience pure, faisant sans cesse mention de toi dans mes prières nuit et jour,
\VS{4}me souvenant de tes larmes, je désire fort te voir afin que je sois rempli de joie.
\VS{5}Et me souvenant de la foi sincère qui est en toi, et qui a premièrement habité en Loïs, ta grand-mère, et en Eunice, ta mère, et qui, je suis persuadé qu'elle habite aussi en toi.
\VS{6}C'est pourquoi je t'exhorte de ranimer le don de Dieu qui est en toi par l'imposition de mes mains.
\VS{7}Car Dieu ne nous a pas donné un esprit de timidité, mais de force, de charité\FTNT{Il est question ici de l’amour «~agape~», c’est-à-dire divin.} et de sagesse.
\VS{8}N’aie donc point honte du témoignage à rendre à notre Seigneur ni de moi, qui suis son prisonnier ; mais souffre avec moi les afflictions de l'Evangile, selon la puissance de Dieu,
\VS{9}qui nous a sauvés et qui nous a appelés par une sainte vocation, non selon nos œuvres, mais selon son propre dessein, et selon la grâce qui nous a été donnée en Jésus-Christ avant les temps éternels,
\VS{10}et qui maintenant a été manifestée par l'apparition de notre Sauveur Jésus-Christ, qui a détruit la mort et qui a mis en lumière la vie et l'immortalité par l'Evangile,
\VS{11}pour lequel j'ai été établi prédicateur, apôtre et docteur des Gentils.
\VS{12}C'est pourquoi aussi je souffre ces choses, mais je n'en ai point de honte ; car je connais celui en qui j'ai cru, et je suis persuadé qu'il est Puissant pour garder mon dépôt\FTNT{Dépôt : Il est question ici de la connaissance correcte et de la pure doctrine de l’Evangile qui doit être fermement et fidèlement gardée, et qui doit être consciencieusement délivrée aux autres.} jusqu'à ce jour-là.
\VS{13}Retiens dans la foi et dans la charité qui est en Jésus-Christ le modèle des saines paroles que tu as apprises de moi.
\VS{14}Garde le bon dépôt par le Saint-Esprit qui habite en nous.
\VS{15}Tu sais que tous ceux qui sont en Asie se sont éloignés de moi ; entre lesquels sont Phygelle et Hermogène.
\VS{16}Que le Seigneur accorde sa miséricorde à la maison d'Onésiphore, car souvent il m'a consolé, et il n'a point eu honte de mes chaînes.
\VS{17}Au contraire, quand il a été à Rome, il m'a cherché avec beaucoup d’empressement, et il m'a trouvé.
\VS{18}Que le Seigneur lui fasse trouver miséricorde envers le Seigneur en ce jour-là ; et tu sais mieux que personne combien il m'a rendu de services à Ephèse.
\TextTitle{[La conduite d'un disciple de Christ dans les jours d'apostasie]}
\Chap{2}
\VerseOne{}Toi donc, mon fils, sois fortifié dans la grâce qui est en Jésus-Christ.
\VS{2}Et les choses que tu as entendues de moi devant plusieurs témoins, confie-les à des personnes fidèles qui soient capables de les enseigner aussi à d'autres.
\VS{3}Toi donc, souffre avec moi comme un bon soldat de Jésus-Christ.
\VS{4}Il n’est pas de soldat qui s'embarrasse des affaires de cette vie s’il veut plaire à celui qui l'a enrôlé pour la guerre.
\VS{5}De même, l’athlète qui combat n'est point couronné s'il n'a pas combattu selon les règles.
\VS{6}Il faut aussi que le laboureur travaille premièrement, et ensuite il recueille les fruits.
\VS{7}Considère ce que je dis, car le Seigneur te donne de l’intelligence en toutes choses.
\VS{8}Souviens-toi que Jésus-Christ, qui est de la semence de David, est ressuscité des morts, selon mon Evangile,
\VS{9}pour lequel je souffre beaucoup de maux, jusqu'à être mis dans les chaînes comme un malfaiteur, mais cependant, la parole de Dieu n'est point liée.
\VS{10}C'est pourquoi je souffre tout pour l'amour des élus, afin qu'eux aussi obtiennent le salut qui est en Jésus-Christ, avec la gloire éternelle.
\VS{11}Cette parole est certaine, que si nous mourons avec lui, nous vivrons aussi avec lui.
\VS{12}Si nous souffrons avec lui, nous régnerons aussi avec lui. Si nous le renions, il nous reniera aussi\FTNT{Lu. 9:26.}.
\VS{13}Si nous sommes infidèles, il demeure fidèle, car il ne peut pas se renier lui-même.
\VS{14}Remets ces choses en mémoire, protestant devant Dieu qu'on ait pas de disputes de mots, qui est une chose dont il ne revient aucun profit, mais elle est la ruine des auditeurs.
\VS{15}Efforce-toi de te rendre approuvé\FTNT{Approuvé vient du grec ~dokimos~. Du temps de l’apôtre Paul, les systèmes bancaires actuels n’existaient pas, toute la monnaie était en métal. Pour obtenir les pièces de monnaie, le métal était fondu et versé dans des moules et après le démoulage, il était nécessaire d’enlever les bavures. Or de nombreuses personnes les grattaient pour récupérer le surplus de métal et même davantage, ce qui faussait le poids de la monnaie. Face à ce problème, de nombreuses lois furent promulguées à Athènes pour éradiquer la pratique du rognage des pièces en circulation. Il existait toutefois quelques changeurs intègres qui ne mettaient en circulation que des pièces au bon poids. On appelait ces personnes des ~dokimos~, ce qui signifie ~éprouvés~ ou ~approuvés~.} devant Dieu, comme un ouvrier sans reproche, enseignant purement la parole de la vérité.
\VS{16}Mais évite les discours vains et profanes ; car ceux qui les tiennent avanceront toujours plus dans l'impiété,
\VS{17}et leur parole rongera comme une gangrène. Et parmi ceux-là sont Hyménée et Philète,
\VS{18}qui se sont écartés\FTNT{Ecarter, dévier, s'écarter de, manquer le but. A l’époque des apôtres, il y avait plusieurs faux frères qui semaient la zizanie au milieu des enfants de Dieu. Parmi eux étaient Alexandre le forgeron (1 Ti. 1:18-20), Hyménée (1 Ti. 1:18-20), Philète (2 Ti. 2:16-18), les judaïsants (Ac. 15 ; Ga. 2), Diotrèphe (3 Jn.). Les faux frères sont des séducteurs.} de la vérité, en disant que la résurrection est déjà arrivée, et qui renversent la foi de quelques-uns.
\VS{19}Toutefois, le fondement de Dieu demeure ferme, ayant ce sceau : Le Seigneur connaît ceux qui lui appartiennent\FTNT{Le Seigneur connaît ses brebis. Voir No. 16:5 ; Jn. 10:14.} ; et : Quiconque invoque le nom du Seigneur, qu'il s’éloigne de l'iniquité.
\VS{20}Or dans une grande maison, il n'y a pas seulement des vases d'or et d'argent, mais il y en a aussi de bois et de terre. Les uns sont des vases d’honneur et les autres sont d’un usage vil.
\VS{21}Si quelqu'un donc se purifie de ces choses, il sera un vase d’honneur, sanctifié et utile au Seigneur, et préparé pour toute bonne œuvre.
\VS{22}Fuis aussi les désirs de la jeunesse, et recherche la justice, la foi, la charité, et la paix avec ceux qui invoquent le Seigneur d'un cœur pur.
\VS{23}Et rejette les questions\FTNT{Questions folles. Il est question ici de disputes, débats, discussions ou questions oiseuses.} folles, et qui sont sans instruction, sachant qu'elles ne font que produire des querelles.
\VS{24}Or, il ne faut pas que le serviteur du Seigneur soit querelleur, il doit au contraire avoir de la douceur envers tout le monde, propre à enseigner, supportant patiemment les mauvais,
\VS{25}enseignant avec douceur ceux qui ont un sentiment contraire, dans l'espérance qu'un jour Dieu leur donnera la repentance pour reconnaître la vérité,
\VS{26}et afin qu'ils se réveillent pour sortir des pièges du diable, par lesquels ils ont été pris pour faire sa volonté.
\TextTitle{[L'Ecriture, l'arme du chrétien face à l'apostasie]}
\Chap{3}
\VerseOne{}Or sache ceci, que dans les derniers jours\FTNT{Les derniers jours. Voir Ge. 49:1-2.} il surviendra des temps difficiles.
\VS{2}Car les hommes seront idolâtres d’eux-mêmes, amis de l’argent, fanfarons, orgueilleux, blasphémateurs, rebelles à leurs parents, ingrats, irréligieux,
\VS{3}sans affection naturelle, sans fidélité, calomniateurs, intempérants, cruels, haïssant les gens de bien,
\VS{4}traîtres, emportés, enflés d'orgueil, amis des voluptés plûtot qu'amis de Dieu\FTNT{(Le mot grec «~philotheos~» (amour de Dieu) du préfixe «~philos~» qui signifie «~amis, être lié d'amitié avec quelqu'un~» (Matt. 11:19 ; Lu. 7:6 ;Jn. 15:13-15 etc...) et de «~theos~» qui signifie «~Dieu~».}
\VS{5}ayant l'apparence\FTNT{L’apparence de la piété. Le mot ~apparnec~ vient du grec ~morphosis~ et du latin ~forma~ qui donnent ~forme~ en français. Il est question du formalisme, de l’attachement excessif aux règles, aux rites, aux coutumes et aux traditions. Dans l’église de Laodicée, l’accent est plutôt mis sur les règles à observer et les apparences que sur la vie spirituelle et intérieure. Les manifestations extérieures du formalisme sont : les lieux «sacrés» pour adorer (temples, cathédrales, pèlerinages, etc.) ; l’observation des jours sacrés (dimanche et sabbat) ; les rituels censés permettre au croyant d’expérimenter Dieu et de rentrer dans une vie bénie (circoncision, ordination, bénédiction nuptiale, paiement de la dîme, présentation des enfants à Dieu par le pasteur...) ; une manière spéciale de s’habiller (toge, soutane, collet clérical, kippa, voile, costume/cravate, un régime alimentaire spécial, etc.). Voir Mt. 6:1-8.} de la piété, mais en ayant renié la force. Eloigne-toi donc de telles gens.
\VS{6}Il en est parmi eux qui se glissent dans les maisons et qui tiennent captives les femmes chargées de péchés et agitées de diverses convoitises,
\VS{7}qui apprennent toujours, mais qui ne peuvent jamais parvenir à la pleine connaissance de la vérité.
\VS{8}Et comme Jannès et Jambrès ont résisté à Moïse, ceux-ci de même résistent à la vérité, étant des gens qui ont l'esprit corrompu, et qui sont réprouvés quant à la foi.
\VS{9}Mais ils ne feront pas de plus grands progrès, car leur folie sera manifestée à tous, comme le fut celle de ceux-là.
\VS{10}Mais pour toi, tu as pleinement compris ma doctrine, ma conduite, mon intention, ma foi, ma douceur, ma charité, ma persévérance.
\VS{11}Et tu sais les persécutions et les afflictions qui me sont arrivées à Antioche, à Iconie, et à Lystre. Quelles persécutions n’ai-je pas supportées ? Et comment le Seigneur m'a délivré de toutes.
\VS{12}Or tous ceux aussi qui veulent vivre pieusement en Jésus-Christ seront persécutés.
\VS{13}Mais les hommes méchants et imposteurs iront en empirant, séduisant les autres, et étant séduits.
\VS{14}Mais toi, demeure ferme dans les choses que tu as apprises et qui t'ont été confiées, sachant de qui tu les as apprises,
\VS{15}vu même que dès ton enfance tu as la connaissance des saintes lettres, qui peuvent te rendre sage pour le salut par la foi en Jésus-Christ.
\VS{16}Toute l'Ecriture est inspirée de Dieu, et utile pour enseigner, pour convaincre, pour corriger, et pour instruire selon la justice,
\VS{17}afin que l'homme de Dieu soit accompli et parfaitement instruit pour toute bonne œuvre.
\TextTitle{[Paul encourage solennellement Timothée à prêcher la parole]}
\Chap{4}
\VerseOne{}Je te somme devant Dieu, et devant le Seigneur Jésus-Christ, qui doit juger les vivants et les morts, lors de son apparition et de son règne.
\VS{2}Prêche la parole, insiste en toute occasion, favorable ou non. Reprends, censure, exhorte avec toute douceur d'esprit, et avec doctrine.
\VS{3}Car il viendra un temps où les hommes ne supporteront pas la saine doctrine, mais aimant qu'on leur chatouille les oreilles par des discours agréables, ils chercheront des docteurs qui répondent à leurs désirs\FTNT{Beaucoup refusent la saine doctrine et acceptent un évangile basé sur les biens matériels.}.
\VS{4}Et ils détourneront leurs oreilles de la vérité, et se tourneront vers les fables.
\VS{5}Mais toi, veille en toutes choses, souffre les afflictions, fais l'œuvre d'un évangéliste, rends ton ministère pleinement approuvé.
\VS{6}Car pour moi, je m'en vais maintenant servir de libation, et le temps de mon départ est proche.
\VS{7}J'ai combattu le bon combat, j'ai achevé la course, j'ai gardé la foi.
\VS{8}Au reste, la couronne de justice m'est réservée, et le Seigneur, juste Juge, me la rendra en ce jour-là, et non seulement à moi, mais aussi à tous ceux qui auront aimé son apparition.
\VS{9}Hâte-toi de venir bientôt vers moi.
\VS{10}Car Démas m'a abandonné, ayant aimé le présent siècle, et il s'en est allé à Thessalonique ; Crescens est allé en Galatie ; et Tite en Dalmatie.
\VS{11}Luc est seul avec moi ; prends Marc, et amène-le avec toi, car il m'est fort utile pour le ministère.
\VS{12}J'ai aussi envoyé Tychique à Ephèse.
\VS{13}Quand tu viendras, apporte avec toi le manteau que j'ai laissé à Troas, chez Carpus, et les livres aussi ; mais principalement mes parchemins.
\VS{14}Alexandre le forgeron m'a fait beaucoup de mal. Le Seigneur lui rendra selon ses œuvres.
\VS{15}Garde-toi donc de lui, car il s'est fortement opposé à nos paroles.
\VS{16}Personne ne m'a assisté dans ma première défense, mais tous m'ont abandonné ; toutefois que cela ne leur soit point imputé !
\VS{17}Mais le Seigneur m'a assisté et fortifié, afin que ma prédication soit pleinement approuvée, et que tous les Gentils l’entendent ; et j'ai été délivré de la gueule du lion.
\VS{18}Le Seigneur aussi me délivrera de toute mauvaise œuvre, et me sauvera dans son Royaume céleste. A lui soit la gloire aux siècles des siècles. Amen !
\TextTitle{[Conclusion]}
\VS{19}Salue Priscille et Aquilas, et la famille d'Onésiphore.
\VS{20}Eraste est resté à Corinthe, et j'ai laissé Trophime malade à Milet.
\VS{21}Hâte-toi de venir avant l'hiver. Eubulus et Pudens, et Linus, et Claudia, et tous les frères te saluent.
\VS{22}Que le Seigneur Jésus-Christ soit avec ton esprit. Que la grâce soit avec vous. Amen !
\PPE{}
\end{multicols}

\clearpage\ShortTitle{Jude}\BookTitle{Jude}\BFont
\begin{multicols}{2}
\TextTitle{[Introduction]}
\Chap{1}
\VerseOne{}Jude serviteur de Jésus-Christ, et frère de Jacques, à ceux qui ont été appelés par l'Evangile, que Dieu a sanctifiés et gardés pour Jésus-Christ :
\VS{2}Que la miséricorde, la paix et l'amour vous soient multipliés.
\TextTitle{[Mise en garde contre l'apostasie]}
\VS{3}Mes bien-aimés, comme je désirais vous écrire avec empressement au sujet de notre salut commun, j’ai jugé nécessaire de le faire pour vous exhorter à combattre pour la foi qui a été transmise aux saints une fois pour toutes.
\VS{4}Car il s’est glissé parmi vous, certains hommes dont la condamnation est écrite depuis longtemps, des impies qui changent la grâce de notre Dieu en dissolution, et qui renient le seul Dominateur Jésus-Christ, notre Dieu et Seigneur.
\TextTitle{[Exemples historiques d'incrédulité et de révolte]}
\VS{5}Je veux vous rappeler une chose que vous savez déjà : C'est que le Seigneur après avoir délivré le peuple du pays d'Egypte, fit ensuite périr les incrédules,
\VS{6}qu’il a réservés pour le jugement du grand jour, enchainés éternellement par les ténèbres, les anges qui n'ont pas gardé leur origine, mais qui ont abandonné leur propre demeure ;
\VS{7}que Sodome et Gomorrhe, et les villes voisines qui s'étaient abandonnées comme eux à l'impureté et à des vices contre nature, sont données en exemples, subissant la peine d’un feu éternel.
\TextTitle{[Description des faux docteurs]}
\VS{8}Malgré cela, ces hommes aussi, plongés dans leurs rêveries, souillent leur chair, méprisent l’autorité, et blasphèment contre les dignités.
\VS{9}Or, l'archange Michel, lorsqu’il contestait avec le diable et lui disputait le corps de Moïse, n'osa pas prononcer contre lui un jugement blasphématoire, mais il dit seulement : Que le Seigneur te réprime !
\VS{10}Eux, au contraire, ils blasphèment contre tout ce qu'ils ignorent, et ils se corrompent dans tout ce qu'ils savent naturellement, comme font les bêtes brutes.
\VS{11}Malheur à eux ! Car ils ont suivi la voie de Caïn, et ils se sont jetés dans l’égarement de Balaam, pour l’amour du gain, ils se sont perdus par la rébellion de Koré (1).
\VS{12}Ce sont des écueils dans vos agapes, lorsqu’ils prennent leurs repas avec vous sans aucune retenue, et se repaissant eux-mêmes ; ce sont des nuées sans eau, emportées par des vents çà et là ; des arbres d’automne dont le fruit se pourrit, et sans fruits, deux fois morts, et déracinés ;
\VS{13}des vagues impétueuses de la mer, jetant l'écume de leurs impuretés ; des étoiles errantes, à qui l'obscurité des ténèbres est réservée éternellement.
\VS{14}C’est aussi pour eux qu’Hénoc, le septième homme après Adam, a prophétisé en disant :
\VS{15}Voici, le Seigneur est venu avec ses saintes myriades, pour exercer un jugement contre tous les hommes, et pour convaincre tous les impies parmi eux de tous les actes d'impiété qu’ils ont commis et de toutes les paroles blasphématoires qu’ont proférées contre lui des pécheurs impies.
\VS{16}Ce sont des gens qui murmurent, qui se plaignent toujours, qui marchent selon leurs convoitises, qui ont à la bouche des discours hautains, qui admirent les personnes pour le profit qui leur en revient.
\VS{17}Mais vous, mes bien-aimés, souvenez-vous des choses qui ont été prédites par les apôtres de notre Seigneur Jésus-Christ.
\VS{18}Ils vous disaient que dans les derniers temps il y aurait des moqueurs, qui marcheraient selon leurs convoitises impies.
\VS{19}Ce sont ceux qui provoquent des divisions, des gens sensuels, n'ayant pas l'Esprit.
\TextTitle{[Exhortation aux chrétiens]}
\VS{20}Mais vous, mes bien-aimés, vous édifiant vous-mêmes sur votre très sainte foi, et priant par le Saint-Esprit,
\VS{21}maintenez-vous les uns les autres dans l'amour de Dieu, en attendant la miséricorde de notre Seigneur Jésus-Christ, pour obtenir la vie éternelle.
\VS{22}Et ayez pitié des uns en usant de discernement ;
\VS{23}sauvez-en d’autres avec crainte, en les arrachant hors du feu, haïssant jusqu’à la tunique souillée par la chair.
\TextTitle{[Conclusion]}
\VS{24}Or, à celui qui est puissant pour vous préserver de toute chute et vous faire paraître devant sa gloire irréprochables et dans l’allégresse,
\VS{25}à Dieu, seul sage, notre Sauveur, par Jésus-Christ notre Seigneur, soient gloire et magnificence, force et puissance, dès maintenant et dans tous les siècles, Amen !
\PPE{}
\end{multicols}

\clearpage\ShortTitle{Hé.}\BookTitle{Hébreux}\BFont
\noindent\hrulefill
{\footnotesize
\textit{
\bigskip
{\centering{}
\\Auteur~: Inconnu
\\Thème~: La prêtrise du Messie
\\Date de rédaction~: Env. 68 ap. J.-C.\\}
}
\textit{
\\Cette épître fut rédigée avant la destruction de Jérusalem, car le temple y subsistait encore. Elle s'adressait à des juifs convertis connaissant bien l'auteur. Parmi eux, certains étaient tentés de retourner au judaïsme à cause des persécutions. L'auteur désire affermir ces chrétiens en leur montrant que l'objectif de la loi avait été réalisé par Christ qui est supérieur aux anges, aux prophètes et à Moïse. Il leur montre combien son œuvre rédemptrice est parfaite et les invite à suivre le Seigneur avec une foi indéfectible en persévérant dans l'amour fraternel.\bigskip
}
}
\par\nobreak\noindent\hrulefill
\begin{multicols}{2}
\Chap{1}
\TextTitle{Dieu parle par le Fils}
\VerseOne{}Dieu ayant anciennement parlé à nos pères par les prophètes, à plusieurs reprises et de plusieurs manières,
\VS{2}nous a parlé dans ces derniers jours\FTNT{Les derniers jours ont commencé avec la naissance de l'Eglise. Voir Joë. 2:28~; Ac. 2:14-17.} par son Fils, qu'il a établi héritier de toutes choses, et par lequel il a aussi créé l'univers~;
\VS{3}et qui étant la splendeur de sa gloire, et l'empreinte de sa substance, et soutenant toutes choses par sa parole puissante, ayant fait par lui-même la purification de nos péchés, s'est assis à la droite de la Majesté divine dans les lieux très hauts.
\TextTitle{Le Fils, supérieur aux anges}
\VS{4}Etant devenu d'autant supérieur aux anges, il a hérité d'un nom plus excellent que le leur.
\VS{5}Car auquel des anges a-t-il jamais dit~: Tu es mon Fils, je t'ai engendré aujourd'hui\FTNT{Ps. 2:7.}~? Et encore~: Je serai pour lui un Père, et il sera pour moi un Fils\FTNT{2 S. 7:14.}~?
\VS{6}Et quand il introduit de nouveau dans le monde son Fils premier-né\FTNT{Voir commentaire en Col. 1:15.}, il est dit~: Et que tous les anges de Dieu l'adorent\FTNT{Ps. 97:7.}~!
\VS{7}Car quant aux anges, il est dit~: Il fait de ses anges des vents, et de ses serviteurs des flammes de feu\FTNT{Ps. 104:4.}.
\VS{8}Mais à l'égard du Fils, il dit~: Ô Dieu, ton trône demeure aux siècles des siècles~; et le sceptre de ton Royaume est un sceptre d'équité~;
\VS{9}tu as aimé la justice, et tu as haï l'iniquité~; c'est pourquoi, ô Dieu, ton Dieu t'a oint d'une huile de joie par-dessus tous tes semblables\FTNT{Ps. 45:7-8.}~!
\VS{10}Et dans un autre endroit~: Toi, Seigneur, tu as fondé la terre dès le commencement, et les cieux sont les ouvrages de tes mains~;
\VS{11}ils périront, mais tu es permanent~; et ils vieilliront tous comme un vêtement,
\VS{12}et tu les rouleras comme un manteau et ils seront changés~; mais toi, tu restes le même, et tes années ne finiront point\FTNT{Es. 50:9~; Es. 51:6~; Ps. 102:27-28.}.
\VS{13}Et auquel des anges a-t-il jamais dit~: Assieds-toi à ma droite, jusqu'à ce que j'aie mis tes ennemis pour le marchepied de tes pieds\FTNT{Ps. 110:1.}~?
\VS{14}Ne sont-ils pas tous des esprits administrateurs, envoyés pour servir en faveur de ceux qui doivent recevoir l'héritage du salut~?
\Chap{2}
\TextTitle{Ne pas négliger le salut}
\VerseOne{}C'est pourquoi il nous faut prendre garde de plus près aux choses que nous avons entendues, de peur que nous les laissions s'échapper.
\VS{2}Car, si la parole prononcée par les anges a été ferme, et si toute transgression et toute désobéissance a reçu une juste rétribution,
\VS{3}comment échapperons-nous, si nous négligeons un si grand salut, qui, ayant été premièrement annoncé par le Seigneur, nous a été confirmé par ceux qui l'avaient entendu~?
\VS{4}Dieu confirmant aussi leur témoignage par des prodiges, et des miracles, et par plusieurs autres différents effets de sa puissance, et par les dons du Saint-Esprit, selon sa volonté.
\TextTitle{Toutes choses doivent être soumises à Christ}
\VS{5}Car, ce n'est pas aux anges qu'il a soumis le monde à venir dont nous parlons.
\VS{6}Et quelqu'un a rendu ce témoignage en quelque autre endroit, disant~: Qu'est-ce que l'homme, pour que tu te souviennes de lui, ou le fils de l'homme, pour que tu le visites~?
\VS{7}Tu l'as fait un peu moindre que les anges, tu l'as couronné de gloire et d'honneur, et l'as établi sur les œuvres de tes mains.
\VS{8}Tu as assujetti toutes choses sous ses pieds\FTNT{Ps. 8:5-7.}. En effet, en lui assujettissant toutes choses, il n'a rien laissé qui ne lui soit assujetti. Mais, nous ne voyons pourtant pas encore que toutes choses lui soient assujetties.
\TextTitle{Jésus abaissé un peu de temps pour sauver l'homme}
\VS{9}Mais celui qui a été fait un peu moindre que les anges, Jésus, nous le voyons couronné de gloire et d'honneur par la passion de sa mort, afin que par la grâce de Dieu, il souffrît la mort pour tous.
\VS{10}Car il était convenable, que celui pour qui sont toutes choses et par qui sont toutes choses, puisqu'il a amené plusieurs enfants à la gloire, consacre le Prince de leur salut par les afflictions.
\VS{11}Car, et celui qui sanctifie et ceux qui sont sanctifiés descendent tous d'un même père. C'est pourquoi il n'a pas honte de les appeler ses frères,
\VS{12}disant~: J'annoncerai ton Nom à mes frères, et je te louerai au milieu de l'assemblée\FTNT{Ps. 22:23.}.
\VS{13}Et encore~: Je me confierai en lui. Et encore~: Me voici, moi et les enfants que Dieu m'a donnés\FTNT{Es. 8:17-18.}.
\VS{14}Ainsi donc, puisque les enfants participent à la chair et au sang, lui aussi de même a participé aux mêmes choses, afin que, par la mort, il rende impuissant celui qui avait le pouvoir de la mort, c'est-à-dire le diable,
\VS{15}et qu'il délivre tous ceux qui, par crainte de la mort, étaient assujettis toute leur vie à la servitude.
\VS{16}Car, certes, il n'a nullement secouru les anges, mais il a secouru la postérité d'Abraham.
\VS{17}C'est pourquoi il a fallu qu'il soit semblable en toutes choses à ses frères, afin qu'il soit un Grand-Prêtre miséricordieux et fidèle dans les choses qui doivent être faites envers Dieu, pour faire la propitiation pour les péchés du peuple~;
\VS{18}car, parce qu'il a souffert lui-même, étant tenté, il est puissant pour secourir ceux qui sont tentés.
\Chap{3}
\TextTitle{Christ, supérieur à Moïse}
\VerseOne{}C'est pourquoi, mes frères saints, qui avez part à la vocation céleste, considérez attentivement Jésus-Christ, l'Apôtre et le Grand-Prêtre de notre profession,
\VS{2}qui a été fidèle à celui qui l'a établi, comme le fut Moïse dans toute sa maison.
\VS{3}Car Jésus-Christ a été jugé digne d'une gloire d'autant supérieure à celle de Moïse, que celui qui a construit une maison, a plus d'honneur que la maison même.
\VS{4}Car chaque maison est construite par quelqu'un, mais celui qui a construit toutes choses, c'est Dieu.
\VS{5}Et quant à Moïse, il a été fidèle dans toute sa maison, comme serviteur, pour témoigner des choses qui devaient être dites~;
\VS{6}mais Christ l'est comme Fils sur sa maison~; et nous sommes sa maison\FTNT{L'Eglise véritable est la maison de Dieu. Voir Es. 66:1~; 1 Co. 3:16~; 1 Co. 6:19~; Ep. 2:21-22. Les bâtiments ne sont pas la maison de Dieu. Le premier bâtiment d'église avait été édifié par des fidèles sous le règne d'Alexandre Sévère en 222-235. L'Eglise véritable est composée de pierres vivantes qui ont pour fondement le Roc (Jésus), parce qu'elle est bâtie par Jésus-Christ lui-même et qu'elle est sa propriété~; les démons ne peuvent pas la détruire. L'Eglise véritable ne peut donc être confondue avec un bâtiment ou une maison physique.}, pourvu que nous retenions fermement jusqu'à la fin l'assurance et la gloire de l'espérance.
\TextTitle{Résultat de l'incrédulité de la génération qui sortit d'Egypte}
\VS{7}C'est pourquoi, comme dit le Saint-Esprit~: Aujourd'hui, si vous entendez sa voix,
\VS{8}n'endurcissez point vos cœurs, comme il arriva dans le lieu de la rébellion, au jour de la tentation dans le désert,
\VS{9}où vos pères me tentèrent et m'éprouvèrent, et ils virent mes œuvres pendant quarante ans\FTNT{Ps. 95:8-11.}.
\VS{10}C'est pourquoi je fus irrité contre cette génération, et je dis~: Leur cœur s'égare toujours. Et ils n'ont pas connu mes voies.
\VS{11}Aussi, je jurai dans ma colère~: Ils n'entreront pas dans mon repos~!
\VS{12}Mes frères, prenez garde que quelqu'un de vous n'ait un cœur mauvais et incrédule, au point de se révolter contre le Dieu vivant,
\VS{13}mais exhortez-vous les uns les autres chaque jour, aussi longtemps qu'on peut dire~: Aujoud'hui~! De peur que quelqu'un d'entre vous ne s'endurcisse par la séduction du péché.
\VS{14}Car nous sommes devenus participants de Christ, pourvu que nous gardions ferme jusqu'à la fin notre première assurance,
\VS{15}pendant qu'il est dit~: Aujourd'hui, si vous entendez sa voix, n'endurcissez pas vos cœurs, comme il arriva dans le lieu de la rébellion.
\VS{16}Car, quelques-uns l'ayant entendue, le provoquèrent à la colère~; mais ce ne furent pas tous ceux qui étaient sortis d'Egypte par Moïse. 
\VS{17}Et contre qui Dieu fut-il irrité pendant quarante ans~? Ne fut-ce pas contre ceux qui péchèrent, et dont les cadavres tombèrent dans le désert~?
\VS{18}Et à qui jura-t-il qu'ils n'entreraient point dans son repos, sinon à ceux qui furent rebelles~?
\VS{19}Aussi, nous voyons qu'ils ne purent y entrer à cause de leur incrédulité.
\Chap{4}
\TextTitle{Le repos}
\VerseOne{}Craignons donc, que quelqu'un d'entre vous, venant à négliger la promesse d'entrer dans son repos, ne s'en trouve privé.
\VS{2}Car il nous a été évangélisé, aussi bien qu'à eux~; mais la parole qu'ils entendirent ne leur servit de rien, parce qu'elle n'était pas mêlée avec la foi dans ceux qui l'entendirent.
\VS{3}Pour nous qui avons cru, nous entrons dans le repos, suivant ce qui a été dit~: C'est pourquoi je jurai dans ma colère, ils n'entreront pas dans mon repos\FTNT{Hé. 3:11.}~! Il dit cela, quoique ses œuvres aient été achevées depuis la fondation du monde.
\VS{4}Car il a parlé quelque part ainsi du septième jour~: Et Dieu se reposa de toutes ses œuvres le septième jour\FTNT{Ge. 2:2.}.
\VS{5}Et encore dans ce passage~: Ils n'entreront pas dans mon repos~!
\VS{6}Puisqu'il reste donc à quelques-uns d'y entrer, et que ceux à qui d'abord il a été évangélisé n'y sont pas entrés à cause de leur désobéissance,
\VS{7}Dieu détermine de nouveau un certain jour, qu'il appelle aujourd'hui, en disant par David si longtemps après, selon ce qui a été dit~: Aujourd'hui, si vous entendez sa voix, n'endurcissez point vos cœurs\FTNT{Ps. 95:8-11.}.
\VS{8}Car, si Josué les avait introduits dans le repos, jamais après cela il n'aurait parlé d'un autre jour.
\TextTitle{Entrer dans le repos de Dieu}
\VS{9}Il reste donc encore un repos réservé au peuple de Dieu.
\VS{10}Car celui qui est entré dans son repos, se repose aussi de ses œuvres, comme Dieu s'est reposé des siennes.
\VS{11}Efforçons-nous donc d'entrer dans ce repos-là, de peur que quelqu'un ne tombe en imitant une semblable désobéissance.
\VS{12}Car la Parole de Dieu est vivante et efficace, et plus pénétrante qu'une épée quelconque à deux tranchants, et atteignant jusqu'à la division de l'âme et de l'esprit, et des jointures et des mœlles~; et elle juge les pensées et les intentions du cœur.
\VS{13}Et il n'y a aucune créature qui soit cachée devant lui, mais toutes choses sont nues et entièrement découvertes aux yeux de celui devant lequel nous devons rendre compte.
\VS{14}Ainsi, puisque nous avons un Souverain Grand-Prêtre, Jésus, le Fils de Dieu, qui a traversé les cieux, tenons ferme notre profession.
\VS{15}Car nous n'avons pas un Grand-Prêtre qui ne puisse avoir compassion de nos infirmités~; mais, nous avons celui qui a été tenté comme nous en toutes choses, mais sans pécher.
\VS{16}Approchons donc avec assurance du trône de la grâce, afin d'obtenir miséricorde et de trouver grâce, pour être secourus dans le temps convenable.
\Chap{5}
\TextTitle{Le service du grand-prêtre}
\VerseOne{}Or tout grand-prêtre pris d'entre les hommes est établi pour les hommes dans les choses qui concernent Dieu, afin qu'il offre des dons et des sacrifices pour les péchés.
\VS{2}Etant capable d'avoir de l'indulgence pour les ignorants et les égarés, puisqu'il est aussi lui-même enveloppé d'infirmité.
\VS{3}Et à cause de cette infirmité, il doit offrir pour les péchés, non seulement pour le peuple, mais aussi pour lui-même.
\VS{4}Et nul ne s'attribue cet honneur, si ce n'est celui qui est appelé de Dieu, comme Aaron.
\TextTitle{Christ, Grand-Prêtre selon l'ordre de Melchisédek}
\VS{5}De même, aussi Christ ne s'est point glorifié lui-même d'être fait Grand-Prêtre, mais celui qui lui a dit~: C'est toi qui es mon Fils, je t'ai engendré aujourd'hui\FTNT{Ps. 2:7.}~!
\VS{6}Comme il dit encore ailleurs~: Tu es prêtre éternellement, selon l'ordre de Melchisédek\FTNT{Ps. 110:4.}.
\VS{7}C'est lui qui, pendant les jours de sa chair, a offert avec de grands cris et avec larmes des prières et des supplications à celui qui pouvait le sauver de la mort, et il a été exaucé à cause de sa piété.
\VS{8}Quoiqu'il soit le Fils de Dieu, il a pourtant appris l'obéissance par les choses qu'il a souffertes.
\VS{9}Après avoir été consacré, il est devenu l'auteur du salut éternel pour tous ceux qui lui obéissent,
\VS{10}étant appelé de Dieu à être Grand-Prêtre selon l'ordre de Melchisédek~;
\VS{11}de qui nous avons beaucoup de choses à dire, mais elles sont difficiles à expliquer, parce que vous êtes devenus lents à comprendre.
\TextTitle{Du lait à la nourriture solide\FTNTT{jusqu'à Hé. 6:12}}
\VS{12}En effet, tandis que vous devriez être maîtres depuis longtemps, vous avez encore besoin qu'on vous enseigne quels sont les premiers rudiments des oracles de Dieu, et vous êtes devenus tels, que vous avez encore besoin de lait et non d'une nourriture solide.
\VS{13}Or quiconque use de lait, ne sait point ce que c'est que la parole de la justice, parce qu'il est un enfant\FTNT{Le mot enfant dans ce passage vient du grec «~nepios~» qui signifie «~ignorant~».}.
\VS{14}Mais la viande solide est pour ceux qui sont déjà hommes faits, {c'est-à-dire}, pour ceux qui, pour y être habitués, ont les sens exercés à discerner le bien et le mal.
\Chap{6}
\TextTitle{Tendre à la perfection}
\VerseOne{}C'est pourquoi, laissant la parole qui n'enseigne que les premiers principes de Christ, tendons à la perfection, ne posant pas de nouveau le fondement de la repentance des œuvres mortes, et de la foi en Dieu,
\VS{2}de la doctrine des baptêmes, et de l'imposition des mains, et de la résurrection des morts, et du jugement éternel.
\VS{3}Et c'est ce que nous ferons, si Dieu le permet.
\VS{4}Or il est impossible que ceux qui ont été une fois illuminés, et qui ont goûté le don céleste, et qui ont été fait participants au Saint-Esprit,
\VS{5}qui ont goûté la bonne parole de Dieu, et les puissances du siècle à venir,
\VS{6}s'ils retombent, soient changés de nouveau par la repentance, vu que, quant à eux, ils crucifient de nouveau le Fils de Dieu, et l'exposent à l'opprobre.
\VS{7}Car la terre qui est abreuvée par la pluie qui tombe souvent sur elle, et qui produit des herbes propres à ceux par qui elle est labourée, reçoit la bénédiction de Dieu~;
\VS{8}mais, celle qui produit des épines et des chardons, est rejetée et proche de malédiction, et sa fin est d'être brûlée.
\VS{9}Mais nous sommes persuadés, quoique nous parlions ainsi, en ce qui vous concerne, mes bien-aimés, des choses meilleures et qui tiennent au salut.
\VS{10}Car Dieu n'est pas injuste, pour oublier votre œuvre, et le travail de la charité que vous avez témoigné pour son Nom, en ce que vous avez secouru les saints, et que vous les secourez encore.
\VS{11}Or nous souhaitons que chacun de vous montre jusqu'à la fin le même empressement pour la pleine certitude de l'espérance,
\VS{12}afin que vous ne vous relâchiez point, mais que vous imitiez ceux qui, par la foi et par la patience, héritent ce qui leur a été promis.
\TextTitle{Christ entré au-delà du voile}
\VS{13}Car, lorsque Dieu fit la promesse à Abraham, ne pouvant jurer par un plus grand, il jura par lui-même,
\VS{14}en disant~: Certainement, je te bénirai abondement et je te multiplierai merveilleusement\FTNT{Ge. 22:16-17.}.
\VS{15}Et ainsi, Abraham ayant attendu patiemment, obtint ce qui lui avait été promis.
\VS{16}Or les hommes jurent par celui qui est plus grand qu'eux, et le serment qu'ils font pour confirmer leur parole met fin à tous leurs différends.
\VS{17}C'est pourquoi Dieu, voulant faire mieux connaître aux héritiers de la promesse la fermeté immuable de sa résolution, il y a fait intervenir le serment,
\VS{18}afin que, par deux choses immuables, dans lesquelles il est impossible que Dieu mente, nous ayons une ferme consolation, nous qui avons notre refuge à obtenir l'espérance qui nous est proposée.
\VS{19}Laquelle nous tenons comme une ancre sûre et ferme de l'âme, et qui pénètre jusqu'au-delà du voile,
\VS{20}où Jésus est entré comme notre précurseur, ayant été fait Grand-Prêtre éternellement, selon l'ordre de Melchisédek\FTNT{Voir Ge. 14.}.
\Chap{7}
\TextTitle{Melchisédek, type de Christ\FTNTT{Ge. 14}}
\VerseOne{}En effet, ce Melchisédek était Roi de Salem et Prêtre du Dieu Très-Haut\FTNT{Ge. 14:18.}. Il alla au-devant d'Abraham lorsqu'il revenait de la défaite des rois, et il le bénit,
\VS{2}et auquel Abraham donna pour sa part la dîme de tout\FTNT{Ge. 14:20. Pour en savoir plus sur la dîme, voir les commentaires en De. 14:22, No. 18:21 et Mal. 3:10.}. Son nom signifie premièrement Roi de justice, et puis il a été Roi de Salem, c'est-à-dire, Roi de paix.
\VS{3}Il est sans père, sans mère, sans généalogie, n'ayant ni commencement de jours ni fin de vie, mais il est rendu semblable au Fils de Dieu. Il demeure Prêtre continuellement.
\TextTitle{La prêtrise de Melchisédek, supérieure à celle d'Aaron}
\VS{4}Considérez donc combien est grand celui à qui même Abraham, le patriarche, donna la dîme du butin.
\VS{5}Car, quant à ceux d'entre les fils de Lévi qui reçoivent la prêtrise, ils ont bien une ordonnance de dîmer le peuple selon la loi, c'est-à-dire, de dîmer leurs frères, bien qu'ils soient sortis des reins d'Abraham.
\VS{6}Mais celui qui n'était pas de la même famille qu'eux reçut d'Abraham la dîme, et bénit celui qui avait les promesses.
\VS{7}Or sans contredit, celui qui est le moindre est béni par celui qui est le plus grand.
\VS{8}Et ici, ce sont les hommes mortels qui prennent les dîmes~; mais là, c'est celui de qui il est rendu témoignage qu'il est vivant.
\VS{9}Et pour ainsi dire, Lévi même qui prend des dîmes, les a payées en Abraham~;
\VS{10}car il était encore dans les reins de son père, quand Melchisédek alla au-devant de lui.
\TextTitle{La prêtrise selon l'ordre d'Aaron n'a rien amené à la perfection}
\VS{11}Si donc la perfection s'était trouvée dans la prêtrise lévitique, (car c'est sous elle que le peuple a reçu la loi) quel besoin était-il après cela qu'un autre prêtre se lève selon l'ordre de Melchisédek, et qui ne soit point nommé selon l'ordre d'Aaron~?
\VS{12}Or la prêtrise étant changée, il est nécessaire qu'il y ait aussi un changement de loi.
\VS{13}Car, celui à l'égard duquel ces choses sont dites, appartient à une autre tribu, de laquelle nul n'a assisté à l'autel~;
\VS{14}car il est évident que notre Seigneur est descendu de la tribu de Juda\FTNT{Mt. 1:2.}, à l'égard de laquelle Moïse n'a rien dit de la prêtrise.
\VS{15}Et cela est encore plus incontestable, en ce qu'un autre prêtre, à la ressemblance de Melchisédek, est suscité~;
\VS{16}qui n'a point été fait prêtre selon la loi du commandement charnel, mais selon la puissance de la vie impérissable.
\VS{17}Car Dieu lui rend ce témoignage~: Tu es prêtre éternellement, selon l'ordre de Melchisédek.
\VS{18}Or il se fait une abolition du commandement qui a précédé, à cause de sa faiblesse, et parce qu'il ne pouvait point profiter.
\VS{19}Car la loi n'a rien amené à la perfection, mais ce qui a amené à la perfection, c'est ce qui a été introduit par-dessus, à savoir une meilleure espérance, par laquelle nous approchons de Dieu.
\VS{20}D'autant plus, même que cela n'a pas été sans serment,
\VS{21}car les Lévites sont devenus prêtres sans serment, mais celui-ci l'est devenu avec serment par celui qui lui a dit~: Le Seigneur l'a juré, et il ne s'en repentira pas\FTNT{Voir Ps. 110:4}~: Tu es prêtre éternellement, selon l'ordre de Melchisédek.
\VS{22}C'est donc d'une alliance d'autant plus excellente que Jésus a été fait le garant.
\TextTitle{Les prêtres sont mortels, seul Christ est éternel}
\VS{23}Et quant aux prêtres, il y en a eu plusieurs qui se sont succédés parce que la mort les empêchait d'être perpétuels.
\VS{24}Mais lui, parce qu'il demeure éternellement, possède une prêtrise qui n'est pas transmissible.
\VS{25}C'est pourquoi aussi il peut sauver parfaitement ceux qui s'approchent de Dieu par lui, étant toujours vivant pour intercéder\FTNT{Le Seigneur Jésus-Christ est le modèle parfait en ce qui concerne la prière d'intercession. Il se tient devant le Père pour nous. En tant qu'homme (1 Ti. 2:5) et Grand-Prêtre, il se tient entre le Père et l'homme pécheur, comme le faisaient les prêtres sous la loi mosaïque. Voir Lu. 22:31-32~; Ro. 8:34~; 1 Jn. 2:1-2.} pour eux.
\VS{26}Or il nous était convenable d'avoir un tel Grand-Prêtre, saint, innocent, sans tache, séparé des pécheurs, et élevé au-dessus des cieux,
\VS{27}qui n'avait pas besoin, comme les grands-prêtres, d'offrir tous les jours des sacrifices, premièrement pour ses péchés, et ensuite pour ceux du peuple, vu qu'il a fait cela une fois, s'étant offert lui-même.
\VS{28}Car, la loi établit grands-prêtres des hommes faibles~; mais la parole du serment qui a été fait après la loi, établit le Fils, qui est parfait pour toujours.
\Chap{8}
\TextTitle{L'ancienne prêtrise~: L'ombre des choses célestes}
\VerseOne{}La chose principale de notre discours, c'est que nous avons un tel Grand-Prêtre, qui est assis à la droite du trône de la majesté de Dieu dans les cieux,
\VS{2}serviteur du sanctuaire, et du véritable tabernacle, que le Seigneur a dressé et non pas les hommes.
\VS{3}Car tout grand-prêtre est établi pour offrir des offrandes et des sacrifices~; c'est pourquoi il est nécessaire que celui-ci ait aussi quelque chose à offrir.
\VS{4}Vu même que s'il était sur la terre, il ne serait pas prêtre, pendant qu'il y aurait encore des prêtres qui offrent les offrandes selon la loi~;
\VS{5}lesquels font le service dans le lieu qui n'est que l'image et l'ombre des choses célestes, selon que Dieu le dit à Moïse, quand il devait achever le tabernacle~: Or prends garde, lui dit-il, de faire toutes choses selon le modèle qui t'a été montré sur la montagne\FTNT{Ex. 25:40.}.
\TextTitle{Christ, le Médiateur d'une alliance plus excellente}
\VS{6}Mais maintenant, notre Grand-Prêtre a obtenu un service d'autant supérieur qu'il est le Médiateur d'une alliance plus excellente, qui a été établie sur de meilleures promesses.
\TextTitle{Les prophètes ont annoncé la Première Alliance}
\VS{7}En effet, si la Première Alliance avait été irréprochable, il n'y aurait pas eu lieu d'en chercher une seconde.
\VS{8}Car en censurant les Juifs, Dieu leur dit~: Voici, les jours viendront, dit le Seigneur, où je traiterai avec la maison d'Israël et avec la maison de Juda une Alliance Nouvelle,
\VS{9}non selon l'alliance que je traitai avec leurs pères, le jour où je les saisis par la main pour les tirer du pays d'Egypte~; car ils n'ont pas persévéré dans mon alliance, c'est pourquoi je les ai méprisés, dit le Seigneur.
\VS{10}Mais voici l'alliance que je traiterai, après ces jours-là, avec la maison d'Israël, dit le Seigneur~: Je mettrai mes lois dans leur esprit, et je les écrirai dans leur cœur, je serai leur Dieu, et ils seront mon peuple.
\VS{11}Personne n'enseignera plus son prochain, ni personne son frère, en disant~: Connais le Seigneur~! Parce que tous me connaîtront, depuis le plus petit jusqu'au plus grand d'entre eux~;
\VS{12}car je serai miséricordieux par rapport à leurs injustices, et je ne me souviendrai plus de leurs péchés, ni de leurs iniquités\FTNT{Jé. 31:31-34.}.
\VS{13}En disant une Nouvelle Alliance, il a déclaré vieille la première~; or, ce qui devient vieux et ancien, est près d'être aboli.
\Chap{9}
\TextTitle{Les ordonnances et le sanctuaire de la Première Alliance~: Des symboles}
\VerseOne{}En vérité, la Première Alliance avait aussi des ordonnances touchant le service divin, et un sanctuaire terrestre.\FTNT{Ex. 25:1-9.}.
\VS{2}Car il fut construit un premier tabernacle, appelé le lieu saint, dans lequel étaient le chandelier, et la table, et les pains de proposition\FTNT{Ex. 25:30.}.
\VS{3}Et après le second voile\FTNT{Ex. 26:31-35.} était le tabernacle, qui était appelé le Saint des saints,
\VS{4}ayant un encensoir d'or\FTNT{Encensoir ou autel d'or pour les parfums~: Lé. 16:12.}, et l'arche de l'alliance\FTNT{Ex. 25:10.}, entièrement couverte d'or tout autour, dans laquelle était le vase d'or\FTNT{Ex. 16:33.} où était la manne, et la verge d'Aaron\FTNT{No. 17:1-10.} qui avait fleuri, et les tables de l'alliance\FTNT{Les tables de l'alliance ou tables du témoignage~: Ex. 34:29~; De. 10:2-5.}.
\VS{5}Et au-dessus de l'arche étaient les chérubins de la gloire, couvrant de leur ombre le propitiatoire\FTNT{Propitiatoire ou couvercle de l'arche de l'alliance~: Lé. 9:7~; Lé. 16:15-17.}. Ce n'est pas le moment de parler en détail là-dessus.
\VS{6}Or ces choses étant ainsi disposées, les prêtres qui font le service entrent en tout temps dans le premier tabernacle\FTNT{No. 28:3.}~;
\VS{7}mais seul le grand-prêtre entre dans le second une fois par an, non sans y porter du sang, qu'il offre pour lui-même et pour les péchés du peuple\FTNT{Lé. 16:34.}.
\VS{8}Le Saint-Esprit faisant connaitre par là que le chemin du Saint des saints n'était pas encore manifesté, tandis que le premier tabernacle était encore debout,
\VS{9}lequel était une figure destinée pour le temps présent, durant lequel étaient offerts des offrandes et des sacrifices qui ne pouvaient point sanctifier la conscience de celui qui faisait le service,
\VS{10}ordonnés seulement en aliments, et en breuvages, en diverses ablutions, et en des cérémonies charnelles, jusqu'au temps de la réforme.
\TextTitle{La réalité du sacrifice s'accomplit en Christ}
\VS{11}Mais Christ est venu comme Grand-Prêtre des biens à venir~; il a traversé un tabernacle plus excellent et plus parfait, qui n'est pas un tabernacle construit de main d'homme, c'est-à-dire, qui n'est pas de cette création~;
\VS{12}et il est entré une fois pour toutes dans le Saint des saints, non avec le sang des veaux ou des boucs, mais avec son propre sang, après avoir obtenu une rédemption éternelle.
\VS{13}Car si le sang des taureaux et des boucs, et la cendre de la génisse\FTNT{No. 19:1-12.}, répandue sur ceux qui sont souillés, sanctifient et procurent la pureté de la chair,
\VS{14}combien plus le sang de Christ, qui, par l'Esprit éternel, s'est offert lui-même à Dieu sans nulle tache, purifiera-t-il votre conscience des œuvres mortes, pour servir le Dieu vivant~?
\VS{15}C'est pourquoi il est le Médiateur de la Nouvelle Alliance, afin que, la mort étant intervenue pour la rançon des transgressions commises sous la Première Alliance, ceux qui ont été appelés reçoivent l'héritage éternel qui leur a été promis.
\TextTitle{Les clauses du testament du Messie}
\VS{16}Car là où il y a un testament, il est nécessaire que la mort du testateur intervienne,
\VS{17}parce que c'est par la mort du testateur qu'un testament est rendu ferme, puisqu'il n'a aucune force tant que le testateur est en vie.
\VS{18}C'est pourquoi la Première Alliance elle-même n'a point été confirmée sans le sang.
\VS{19}Car Moïse, après avoir prononcé devant tout le peuple tous les commandements de la loi, prit le sang des veaux et des boucs, avec de l'eau, et de la laine écarlate, et de l'hysope~; et il en fit l'aspersion sur le livre et sur tout le peuple, en disant~:
\VS{20}Ceci est le sang de l'Alliance que Dieu vous a ordonné d'observer\FTNT{Ex. 24:3-8.}.
\VS{21}Puis il fit aussi aspersion avec du sang sur le tabernacle et sur tous les ustensiles du service\FTNT{Ex. 29:12~; Ex. 29:36.}.
\VS{22}Et presque toutes choses, selon la loi, sont purifiées par le sang, et sans effusion de sang il n'y a pas de rémission des péchés.
\TextTitle{Un sacrifice plus excellent\FTNTT{Lé. 16:33}}
\VS{23}Il a donc fallu que les choses qui représentaient celles qui sont aux cieux, soient purifiées par de telles choses, mais que les célestes le soient par des sacrifices plus excellents que ceux-là.
\VS{24}Car Christ n'est pas entré dans un sanctuaire fait de main d'homme, et qui n'était que la figure du véritable, mais il est entré dans le ciel même, afin de comparaître maintenant pour nous devant la face de Dieu.
\VS{25}Et ce n'est pas pour s'offrir lui-même plusieurs fois qu'il y est entré, ainsi que le grand-prêtre entre dans le Saint des saints, chaque année, avec un autre sang~;
\VS{26}autrement, il aurait fallu qu'il ait souffert plusieurs fois depuis la création du monde~; mais maintenant, à la fin des siècles, il a paru une seule fois pour l'abolition du péché par son sacrifice.
\VS{27}Et comme il est réservé aux hommes de mourir une seule fois\FTNT{Ce passage réfute la doctrine de la réincarnation.}, et après cela suit le jugement,
\VS{28}de même aussi Christ, qui s'est offert une seule fois pour ôter les péchés de plusieurs, apparaîtra sans péché une seconde fois à ceux qui l'attendent pour le salut.
\Chap{10}
\TextTitle{Le sacrifice unique de Christ est supérieur à tous les sacrifices}
\VerseOne{}Car la loi qui possède l'ombre des biens à venir, et non l'image exacte des choses, ne peut jamais, par les mêmes sacrifices que l'on offre continuellement chaque année, sanctifier ceux qui s'y attachent.
\VS{2}Autrement, n'auraient-ils pas cessé d'être offerts~? Parce que les adorateurs, une fois expurgés, n'auraient plus eu conscience des péchés.
\VS{3}Or le souvenir des péchés est réitéré dans ces sacrifices chaque année~;
\VS{4}car il est impossible que le sang des taureaux et des boucs ôte les péchés.
\VS{5}C'est pourquoi Jésus-Christ, en entrant dans le monde, a dit~: Tu n'as pas voulu de sacrifice, ni d'offrande, mais tu m'as formé un corps~;
\VS{6}tu n'as pas pris plaisir aux holocaustes, ni aux sacrifices pour le péché\FTNT{Ps. 40:7-9.}.
\VS{7}Alors j'ai dit~: Me voici, je viens, il est écrit de moi au commencement du livre~: Que je fasse, ô Dieu, ta volonté~!
\VS{8}Après avoir dit d'abord~: Tu n'as pas voulu de sacrifice, ni d'offrande, ni d'holocauste, ni d'offrande pour le péché et tu n'y as point pris plaisir, lesquelles choses sont pourtant offertes selon la loi, alors il dit~: Me voici, je viens afin de faire, ô Dieu, ta volonté~!
\VS{9}Il abolit ainsi le premier afin d'établir le second.
\VS{10}Or c'est par cette volonté que nous sommes sanctifiés, à savoir par l'offrande du corps de Jésus-Christ qui a été faite une fois pour toutes.
\VS{11}De plus, tout prêtre fait chaque jour le service et offre souvent les mêmes sacrifices, qui ne peuvent jamais ôter les péchés,
\VS{12}mais lui, après avoir offert un seul sacrifice pour les péchés, s'est assis pour toujours à la droite de Dieu,
\VS{13}attendant désormais que ses ennemis soient mis pour le marchepied de ses pieds.
\VS{14}Car, par une seule offrande, il a rendu parfaits pour toujours ceux qui sont sanctifiés.
\VS{15}Et c'est aussi ce que le Saint-Esprit nous témoigne~; car, après avoir dit premièrement~:
\VS{16}Voici l'alliance que je ferai avec eux, après ces jours-là, dit le Seigneur\FTNT{Voir Jé. 31:31-34.}~: C'est que je mettrai mes lois dans leur cœur, et je les écrirai dans leur esprit~;
\VS{17}et je ne me souviendrai plus de leurs péchés, ni de leurs iniquités.
\VS{18}Or, là où les péchés sont pardonnés, il n'y a plus d'offrande pour le péché.
\TextTitle{Exhortation à s'approcher de Dieu avec foi}
\VS{19}Ainsi donc, mes frères, nous avons la liberté d'entrer dans le Saint des saints au moyen du sang de Jésus,
\VS{20}qui est le chemin\FTNT{Jésus est le chemin qui conduit au Saint des saints, à la vie (Voir Jn. 14:6), et ce chemin n'était pas encore manifesté avant sa naissance. Hé. 9:8.} nouveau et vivant qu'il a inauguré pour nous à travers le voile, c'est-à-dire sa propre chair,
\VS{21}et ayant un Grand-Prêtre établi sur la maison de Dieu,
\VS{22}approchons-nous de lui avec un cœur sincère, et une foi inébranlable, ayant les cœurs purifiés d'une mauvaise conscience, et le corps lavé d'une eau pure.
\VS{23}Retenons fermement la profession de notre espérance, car celui qui nous a fait la promesse est fidèle.
\VS{24}Veillons les uns sur les autres pour nous exciter à la charité et aux bonnes œuvres.
\VS{25}N'abandonnons pas notre assemblée\FTNT{Assemblée~: Du grec «~episunagoge~» qui veut dire «~être assemblé en un lieu, assemblée religieuse des chrétiens~». Il est question de ne pas abandonner la communion fraternelle et non une église locale. En effet, il est du devoir du chrétien de se séparer des faux frères de peur d'être entraîné dans leur égarement (Mt. 18:15-17~; 1 Co. 5:11~; 1 Co. 15:33).}, comme c'est la coutume de quelques-uns~; mais exhortons-nous les uns les autres, et cela d'autant plus que vous voyez approcher le jour.
\TextTitle{Ne pas mépriser le sacrifice de Christ}
\VS{26}Car, si nous péchons volontairement après avoir reçu la connaissance de la vérité, il ne reste plus de sacrifice pour les péchés,
\VS{27}mais une attente terrible du jugement et l'ardeur d'un feu qui doit dévorer les adversaires.
\VS{28}Si quelqu'un avait méprisé la loi de Moïse, il mourait sans miséricorde, sur la déposition de deux ou de trois témoins\FTNT{De. 17:6.}~;
\VS{29}de combien pires tourments pensez-vous donc que sera jugé digne celui qui aura foulé aux pieds le Fils de Dieu, et qui aura tenu pour une chose profane le sang de l'Alliance, par lequel il avait été sanctifié, et qui aura outragé l'Esprit de grâce~?
\VS{30}Car nous connaissons celui qui a dit~: C'est à moi que la vengeance appartient, et je le rendrai~! Dit le Seigneur. Et encore~: Le Seigneur jugera son peuple.\FTNT{De. 32:35-36.}
\VS{31}C'est une chose terrible que de tomber entre les mains du Dieu vivant.
\VS{32}Or rappelez-vous des premiers jours, où, après avoir été éclairés, vous avez soutenu un grand combat de souffrances,
\VS{33}ayant été, d'une part, exposés à la vue de tout le monde par des opprobres et des afflictions, et de l'autre, ayant participé aux maux de ceux qui ont souffert de semblables indignités.
\VS{34}Car vous avez aussi été participants de l'affliction de mes liens, et vous avez reçu avec joie l'enlèvement de vos biens, sachant en vous-mêmes que vous avez dans les cieux des biens meilleurs et permanents.
\VS{35}N'abandonnez donc pas cette fermeté que vous avez fait paraître, et qui sera bien récompensée.
\VS{36}Parce que vous avez besoin de patience, afin qu'après avoir fait la volonté de Dieu, vous receviez l'effet de sa promesse.
\TextTitle{La marche par la foi~: Exemples d'hommes et de femmes de foi}
\VS{37}Car, encore un peu de temps, et celui qui doit venir, viendra, et il ne tardera point.
\VS{38}Or le juste vivra de la foi~; mais si quelqu'un se retire, mon âme ne prend point de plaisir en lui\FTNT{Ha. 2:4.}.
\VS{39}Mais pour nous, nous ne sommes pas de ceux qui se retirent~; ce serait notre perdition~; mais nous persévérons dans la foi, pour le salut de l'âme.
\Chap{11}
\VerseOne{}Or la foi rend présentes les choses qu'on espère, et elle est une démonstration de celles qu'on ne voit point.
\VS{2}Car c'est par elle que les anciens ont obtenu un bon témoignage.
\VS{3}Par la foi, nous comprenons que l'univers a été fait par la parole de Dieu, de sorte que les choses qui se voient, n'ont pas été faites des choses visibles.
\VS{4}Par la foi, Abel\FTNT{Ge. 4:3-5.} offrit à Dieu un sacrifice plus excellent que Caïn~; et par elle il obtînt le témoignage d'être juste, parce que Dieu rendait témoignage de ses offrandes~; et c'est par elle qu'il parle encore, quoique mort.
\VS{5}Par la foi, Hénoc\FTNT{Ge. 5:22-24.} fut enlevé pour ne pas voir la mort, et il ne parut plus parce que Dieu l'avait enlevé~; car, avant qu'il soit enlevé, il avait obtenu le témoignage d'avoir été agréable à Dieu.
\VS{6}Or il est impossible de lui être agréable sans la foi~; car il faut que celui qui vient à Dieu, croie que Dieu est, et qu'il est le rémunérateur de ceux qui le cherchent.
\VS{7}Par la foi, Noé\FTNT{Ge. 6:14-22.}, ayant été divinement averti des choses qui ne se voyaient point encore, craignit, et bâtit l'arche pour la conservation de sa famille~; et par cette arche, il condamna le monde, et devint héritier de la justice qui est selon la foi.
\VS{8}Par la foi, Abraham\FTNT{Ge 12:1-4.}, étant appelé, obéit, pour aller sur la terre, qu'il devait recevoir en héritage, et il partit sans savoir où il allait.
\VS{9}Par la foi, il demeura comme étranger sur la terre, qui lui avait été promise, comme si elle ne lui avait point appartenu, demeurant sous des tentes avec Isaac et Jacob, qui étaient héritiers avec lui de la même promesse.
\VS{10}Car il attendait la cité qui a des fondements, celle dont Dieu est l'architecte et le constructeur.
\VS{11}Par la foi, aussi Sara\FTNT{Ge. 21:1-2.} reçut la force de concevoir un enfant, et elle enfanta hors d'âge, parce qu'elle fut persuadée que celui qui le lui avait promis, était fidèle.
\VS{12}C'est pourquoi d'un seul homme, et qui était déjà affaibli, il est né une multitude aussi nombreuse que les étoiles du ciel, et que le sable du bord de la mer, qui ne peut se compter\FTNT{Ge. 22:17.}.
\VS{13}Tous ceux-ci sont morts dans la foi, sans avoir reçu les choses dont ils avaient eu les promesses, mais ils les ont vues de loin, crues, et saluées, et ils ont fait profession qu'ils étaient étrangers et voyageurs sur la terre\FTNT{1 Pi. 2:11.}.
\VS{14}Car ceux qui tiennent ces discours montrent clairement qu'ils cherchent encore leur patrie.
\VS{15}Et certes, s'ils avaient eu en vue celle d'où ils étaient sortis, ils auraient eu le temps d'y retourner.
\VS{16}Mais maintenant, ils en désirent une meilleure, c'est-à-dire une céleste. C'est pourquoi Dieu n'a pas honte d'être appelé leur Dieu, parce qu'il leur a préparé une cité\FTNT{Jn. 14:2~; Ap. 21:2.}.
\VS{17}Par la foi, Abraham étant éprouvé, offrit Isaac~; celui qui avait reçu les promesses offrit même son fils unique\FTNT{Ge. 22:1.},
\VS{18}à l'égard duquel il lui avait été dit~: Les descendants d'Isaac seront ta véritable postérité\FTNT{Ge. 21:12.}.
\VS{19}Ayant estimé que Dieu pouvait même le ressusciter d'entre les morts~; c'est pourquoi aussi il le recouvra par une espèce de résurrection.
\VS{20}Par la foi, Isaac bénit Jacob et Esaü, en vue des choses à venir\FTNT{Ge. 27:26-40.}.
\VS{21}Par la foi, Jacob, mourant, bénit chacun des fils de Joseph\FTNT{Ge. 48:1-22.}, et adora Dieu, appuyé sur l'extrémité de son bâton\FTNT{Ge. 47:31.}.
\VS{22}Par la foi, Joseph mourant fit mention de la sortie des enfants d'Israël, et il donna des ordres au sujet de ses os\FTNT{Ge. 50:24-25.}.
\VS{23}Par la foi, Moïse\FTNT{Ex. 2:1-3.}, à sa naissance, fut caché pendant trois mois par son père et sa mère, parce qu'ils virent que l'enfant était beau, et ils ne craignirent pas l'ordre du roi.
\VS{24}Par la foi, Moïse, devenu grand, refusa d'être nommé fils de la fille de Pharaon,
\VS{25}choisissant plutôt d'être affligé avec le peuple de Dieu, que de jouir pour un peu de temps des délices du péché.
\VS{26}Et ayant estimé que l'opprobre de Christ était un plus grand trésor que les richesses de l'Egypte, parce qu'il avait égard à la rémunération.
\VS{27}Par la foi, il quitta l'Egypte, sans craindre la fureur du roi~; car il demeura ferme, comme voyant celui qui est invisible.
\VS{28}Par la foi, il fit la Pâque et l'aspersion du sang, afin que le destructeur qui tuait les premiers-nés, ne touche pas aux premiers-nés des Israélites\FTNT{Ex. 12:1-51.}.
\VS{29}Par la foi, ils traversèrent la Mer Rouge, comme un lieu sec, ce que les Egyptiens essayèrent de tenter, ils furent engloutis dans les eaux\FTNT{Ex. 14:13-31.}.
\VS{30}Par la foi, les murs de Jéricho tombèrent, après qu'on en eut fait le tour pendant sept jours\FTNT{Jos. 6:1-20.}.
\VS{31}Par la foi, Rahab, la prostituée, ne périt pas avec les incrédules, parce qu'elle avait reçu les espions et les avait renvoyés en paix\FTNT{Jos. 2:1-21~; Jos. 6:23.}.
\VS{32}Et que dirai-je encore~? Car le temps me manquerait si je voulais parler de Gédéon\FTNT{Jg. 6:11.}, et de Barak\FTNT{Jg. 4:6.}, et de Samson\FTNT{Jg. 13:24.}, et de Jephté\FTNT{Jg. 11:1.}, et de David\FTNT{1 S. 16-17.}, et de Samuel\FTNT{1 S. et 2 S.}, et des prophètes,
\VS{33}qui par la foi combattirent des royaumes, exercèrent la justice, obtinrent des promesses, fermèrent la gueule des lions,
\VS{34}éteignirent la force du feu, échappèrent au tranchant des épées, des malades devinrent vigoureux, se montrèrent fort dans la bataille, et mirent en fuite des armées étrangères.
\VS{35}Des femmes recouvrèrent leurs morts par le moyen de la résurrection~; et d'autres furent livrés aux tourments et n'acceptèrent point d'être délivrés, afin d'obtenir une meilleure résurrection.
\VS{36}Et d'autres subirent les moqueries et le fouet, les chaînes et la prison~;
\VS{37}ils furent lapidés, sciés, subirent de rudes épreuves, ils furent mis à mort par le tranchant de l'épée, ils errèrent çà et là, vêtus de peaux de brebis et de chèvres, réduits à la misère, affligés, tourmentés,
\VS{38}eux dont le monde n'était pas digne, errant dans les déserts et dans les montagnes, et dans les cavernes et dans les trous de la terre.
\VS{39} Et quoiqu'ils aient tous été recommandables par leur foi, ils n'ont pourtant point reçu l'effet de la promesse,
\VS{40}Dieu ayant pourvu quelque chose de meilleur pour nous, en sorte qu'ils ne parviennent pas à la perfection sans nous.
\Chap{12}
\TextTitle{Fixer les regards sur Jésus}
\VerseOne{}Nous donc aussi, puisque nous sommes environnés d'une si grande nuée de témoins\FTNT{Témoin~: du grec «~martus~», terme qui dans un sens légal et historique signifie «~celui qui est spectateur d'une chose~». Dans un sens éthique, il est question de «~ceux qui ont prouvé la force et l'authenticité de leur foi en Christ en supportant une mort violente~». «~Martus~» a donné le mot «~martyr~» en français.}, rejetons tout fardeau, et le péché qui nous enveloppe si aisément, et poursuivons constamment la course qui nous est proposée,
\VS{2}portant les yeux sur Jésus, le chef et le consommateur de la foi qui en échange de la joie qui lui était réservée, il a souffert la croix, ayant méprisé la honte, et s'est assis à la droite du trône de Dieu.
\VS{3}C'est pourquoi, considérez soigneusement celui qui a supporté contre sa personne une telle opposition de la part des pécheurs, afin que vous ne succombiez point, en perdant courage.
\VS{4}Vous n'avez pas encore résisté jusqu'à répandre votre sang en combattant contre le péché.
\TextTitle{La correction du Père}
\VS{5}Et cependant vous avez oublié l'exhortation qui vous est adressée comme à ses fils, disant~: Mon fils, ne méprise pas le châtiment du Seigneur, et ne perds point courage lorsqu'il te reprend~;
\VS{6}car le Seigneur châtie celui qu'il aime, et il frappe de la verge tous ceux qu'il reconnaît pour ses fils\FTNT{Pr. 3:11-12.}.
\VS{7}Si vous endurez le châtiment, Dieu se présente à vous comme à ses fils~; car qui est le fils que le père ne châtie point~?
\VS{8}Mais si vous êtes sans châtiment auquel tous participent, vous êtes donc des enfants illégitimes, et non pas des fils.
\VS{9}Et puisque nos pères selon la chair nous ont châtiés, et que malgré cela nous les avons respectés, ne serons-nous pas beaucoup plus soumis au Père des esprits, pour avoir la vie~?
\VS{10}Car par rapport à ceux-là, ils nous châtiaient pour un peu de temps, suivant leur volonté, mais celui-ci nous châtie pour notre profit, afin que nous soyons participants de sa sainteté.
\VS{11}Or tout châtiment ne semble pas sur l'heure être un sujet de joie, mais de tristesse~; mais ensuite il produit un fruit paisible de justice à ceux qui sont exercés par ce moyen.
\VS{12}Fortifiez donc vos mains languissantes et vos genoux affaiblis~;
\VS{13}et suivez avec vos pieds des chemins droits, afin que ce qui est boiteux ne dévie pas, mais plutôt se consolide.
\VS{14}Recherchez la paix avec tous, et la sanctification, sans laquelle nul ne verra le Seigneur.
\TextTitle{Que nul ne se prive de la grâce de Dieu !}
\VS{15}Veillez à ce que personne ne se prive de la grâce de Dieu~; à ce qu'aucune racine d'amertume, poussant des rejetons, ne vous trouble, et que plusieurs n'en soient souillés par elles~;
\VS{16}que nul de vous ne soit fornicateur, ou profane comme Esaü, qui pour un aliment vendit son droit d'aînesse\FTNT{Ge. 25:33}.
\VS{17}Car vous savez que plus tard, désirant hériter la bénédiction, il fut rejeté, car il ne trouva point de lieu à la repentance, quoiqu'il l'ait demandée avec larmes.
\TextTitle{L'Eglise véritable s'est approchée de Sion}
\VS{18}Vous ne vous êtes pas approchés d'une montagne qu'on pouvait toucher avec la main\FTNT{Ex. 19:12.}, ni du feu brûlant, ni de la nuée épaisse, ni des ténèbres, ni de la tempête,
\VS{19}ni du retentissement de la trompette, ni du son des paroles, au sujet duquel ceux qui l'entendirent prièrent que la parole ne leur soit plus adressée\FTNT{Ex. 20:18-26.},
\VS{20}car ils ne pouvaient pas supporter ce qui était ordonné, que si même une bête touche la montagne, elle sera lapidée ou percée d'un dard\FTNT{Ex. 19:13.}.
\VS{21}Et ce spectacle était si terrible que Moïse dit~: Je suis épouvanté et tout tremblant~!
\VS{22}Mais vous vous êtes approchés de la montagne de Sion, de la Cité du Dieu vivant, la Jérusalem céleste, d'une multitude innombrable d'anges,
\VS{23}et de l'assemblée et de l'Eglise des premiers-nés qui sont inscrits dans les cieux, du Dieu qui est le juge de tous, et des esprits des justes qui ont été rendus parfaits,
\VS{24}de Jésus, qui est le Médiateur de la Nouvelle Alliance, et du sang de l'aspersion, qui prononce des meilleurs choses que celui d'Abel.
\TextTitle{Exhortation à la crainte de Dieu}
\VS{25}Prenez garde de ne pas mépriser celui qui vous parle~; car si ceux qui méprisèrent celui qui leur parlait sur la terre, n'ont pas échappé, nous serons punis beaucoup plus, si nous nous détournons de celui qui parle des cieux,
\VS{26}lui, dont la voix ébranla alors la terre, mais à l'égard du temps présent, il a fait cette promesse, disant~: J'ébranlerai encore une fois non seulement la terre, mais aussi le ciel\FTNT{Ag. 2:6.}.
\VS{27}Or ces mots~: Une fois encore, marquent le changement des choses ébranlées, comme étant faites pour un temps, afin que celles qui sont inébranlables demeurent.
\VS{28}C'est pourquoi, saisissant le Royaume qui ne peut point être ébranlé, retenons la grâce par laquelle nous servions Dieu, en sorte que nous lui soyons agréables avec respect et avec crainte,
\VS{29}car notre Dieu est aussi un feu dévorant\FTNT{De. 4:24.}.
\Chap{13}
\TextTitle{Exhortations~; invariabilité de Christ}
\VerseOne{}Que la charité fraternelle demeure dans vos cœurs.
\VS{2}N'oubliez pas l'hospitalité~; car, par elle, quelques-uns ont logé des anges sans le savoir.
\VS{3}Souvenez-vous des prisonniers, comme si vous étiez emprisonnés avec eux~; et de ceux qui sont maltraités, comme étant aussi vous-mêmes du même corps.
\VS{4}Le mariage est honorable entre tous, et le lit sans souillure~; mais Dieu jugera les fornicateurs et les adultères.
\VS{5}Que votre conduite soit sans avarice, étant contents de ce que vous avez présentement~; car lui-même a dit~: Je ne te délaisserai point, et je ne t'abandonnerai point\FTNT{De. 31:6.}.
\VS{6}De sorte que nous pouvons dire avec assurance~: Le Seigneur est mon aide, et je ne craindrai point ce que l'homme pourrait me faire\FTNT{Ps. 118:6.}.
\VS{7}Souvenez-vous de vos conducteurs qui vous ont annoncé la parole de Dieu~; considérez quelle a été la fin de leur vie, et imitez leur foi.
\VS{8}Jésus-Christ est le même hier, aujourd'hui, et il l'est aussi éternellement.
\VS{9}Ne soyez point emportés çà et là par des doctrines diverses et étrangères~; car il est bon que le cœur soit affermi par la grâce, et non point par les aliments, lesquelles n'ont en rien profité à ceux qui s'y sont attachés.
\TextTitle{Porter ses regards sur la cité céleste}
\VS{10}Nous avons un autel dont ceux qui servent dans le tabernacle n'ont pas le droit de manger.
\VS{11}Car les corps des animaux, dont le sang est porté dans le sanctuaire par le grand-prêtre pour le péché, sont brûlés hors du camp.
\VS{12}C'est pourquoi aussi Jésus, afin de sanctifier le peuple par son propre sang, a souffert hors de la porte\FTNT{Ex. 29:14. Jésus a souffert hors de Jérusalem (Jn. 19:17-18).}.
\VS{13}Sortons donc vers lui, hors du camp\FTNT{Le mot «~camp~» dans ce passage vient du grec «~parambole~», terme faisant référence au judaïsme antique dans lequel s'étaient embourbés les chrétiens d'origine hébraïque. Aujourd'hui, il représente plutôt le christianisme paganisé, essentiellement basé sur la loi de Moïse et constituant une prison qui empêche certains enfants de Dieu de vivre pleinement leur liberté en Christ.}, en portant son opprobre.
\VS{14}Car nous n'avons point ici-bas de cité permanente, mais nous recherchons celle qui est à venir.
\TextTitle{Le sacrifice de louange et du serviteur de Dieu}
\VS{15}Offrons donc par lui sans cesse à Dieu un sacrifice de louange, c'est-à-dire, le fruit des lèvres, en confessant son Nom.
\VS{16}Or n'oubliez pas la bienfaisance et de faire part de vos biens, car Dieu prend plaisir à de tels sacrifices.
\TextTitle{L'obéissance aux conducteurs}
\VS{17}Obéissez\FTNT{Le terme «~obéissez~», en grec «~peitho~», veut dire «~se laisser persuader par des mots~». Il signifie aussi «~donner avec persuasion l'envie à quelqu'un de faire quelque chose en le rassurant~». Par conséquent, les conducteurs doivent comprendre que la soumission et l'obéissance des chrétiens n'a rien à voir avec la dictature et l'autoritarisme. Ils doivent les rassurer et les convaincre - car tout ce qui n'est pas fait avec foi est péché (Ro. 14:23) - et ne pas tyranniser leurs frères en les obligeant à leur obéir (Mt. 20:25~; 1 Pi. 5:2-3).} à vos conducteurs, et soyez-leur soumis, car ils veillent pour vos âmes, comme devant en rendre compte~; afin que ce qu'ils en font, ils le fassent avec joie, et non en gémissant, car cela ne vous serait pas profitable.
\VS{18} Priez pour nous, car nous nous assurons que nous avons une bonne conscience, désirant nous conduire honnêtement parmi tous.
\VS{19}C'est avec instance que je vous demande de le faire, afin que je vous sois rendu plus tôt.
\TextTitle{Bénédictions et salutations}
\VS{20}Que le Dieu de paix, qui a ramené d'entre les morts le grand Pasteur des brebis, par le sang de l'Alliance éternelle, notre Seigneur Jésus-Christ,
\VS{21}vous rende capables de toute bonne œuvre pour faire sa volonté~; qu'il fasse en vous ce qui lui est agréable par Jésus-Christ~; auquel soit la gloire aux siècles des siècles~! Amen~!
\VS{22}Aussi, mes frères, je vous prie de supporter la parole d'exhortation, car je vous ai écrit en peu de mots.
\VS{23}Sachez que notre frère Timothée a été relâché~; s'il vient bientôt, je vous verrai avec lui.
\VS{24}Saluez tous vos conducteurs, et tous les saints. Ceux d'Italie vous saluent.
\VS{25}Que la grâce soit avec vous tous~! Amen~!
\PPE{}
\end{multicols}

\clearpage\ShortTitle{1 Jean}\BookTitle{1 Jean}\BFont
\noindent\hrulefill
{\footnotesize
\textit{
\bigskip
{\centering{}
\\Auteur : Jean
\\Signification : Yahweh a fait grâce
\\Thème : La communion fraternelle, la connaissance et l'amour
\\Date de rédaction : Env. 85 ap. J.-C.\\}
}
%\bigskip
\textit{
\\Cette épître, écrite par Jean à Ephèse, était destinée aux églises de la province d’Asie qu’il connaissait bien. Il souhaite rendre leur joie parfaite en fortifiant leur foi en Christ et en leur donnant l’assurance de la vie éternelle ;  tout en les mettant en garde contre les faux docteurs.\bigskip
}
}
\par\nobreak\noindent\hrulefill
\begin{multicols}{2}
\Chap{1}
\TextTitle{La Parole incarnée}
\VerseOne{}Ce qui était dès le commencement, ce que nous avons entendu, ce que nous avons vu de nos propres yeux, ce que nous avons contemplé, et que nos propres mains ont touché concernant la Parole de vie,
\VS{2}car la vie a été manifestée, et nous l'avons vue et nous lui rendons témoignage, et nous vous annonçons la vie éternelle, qui était avec le Père, et qui nous a été manifestée.
\TextTitle{Communion avec le Père et le Fils}
\VS{3}Ce que nous avons vu dis-je, et ce que nous avons entendu, nous vous l'annonçons, afin que vous soyez en communion avec nous, et que notre communion soit avec le Père et avec son Fils Jésus-Christ.
\VS{4}Et nous vous écrivons ces choses, afin que votre joie soit parfaite.
\TextTitle{De la communion avec Dieu, qui est Lumière, et de la confession des péchés}
\TextTitle{[Conditions de la communion avec Dieu
\\a. Position de l'enfant de Dieu dans la Lumière]}
\VS{5}Or c'est ici la déclaration que nous avons entendue de lui et que nous vous annonçons, à savoir que Dieu est Lumière et qu'il n'y a point en lui de ténèbres.
\VS{6}Si nous disons que nous sommes en communion avec lui, et que nous marchions dans les ténèbres, nous mentons, et nous n'agissons pas selon la vérité.
\VS{7}Mais si nous marchons dans la Lumière, comme Dieu est dans la Lumière, nous sommes en communion les uns avec les autres, et le sang de son Fils Jésus-Christ nous purifie de tout péché.
\TextTitle{b. Reconnaissance de la présence du péché en nous}
\VS{8}Si nous disons que nous n'avons point de péché, nous nous séduisons nous-mêmes, et la vérité n'est point en nous.
\TextTitle{c. la confession des péchés, le pardon et la purification}
\VS{9}Si nous confessons nos péchés, il est fidèle et juste pour nous les pardonner, et pour nous purifier de toute iniquité.
\VS{10}Si nous disons que nous n'avons point de péché, nous le faisons menteur, et sa Parole n'est point en nous.
\TextTitle{Celui qui connaît Jésus-Christ garde ses commandements}
\Chap{2}
\TextTitle{d. Christ, notre avocat pour nos péchés}
\VerseOne{}Mes petits-enfants, je vous écris ces choses afin que vous ne péchiez point. Et si quelqu'un a péché, nous avons un avocat\FTNT{Jésus, notre Avocat. Le mot grec «~parakletos~», traduit ici par «~avocat~», se trouve également en Jean 14 et 16, où il est traduit par «~Consolateur~» et s’applique au Saint-Esprit. Le Seigneur exerce la fonction d'avocat actuellement pour nous dans le ciel. Voir Ro. 8:33 ; Hé. 7:25.} auprès du Père, Jésus-Christ, le Juste.
\VS{2}Car c'est lui qui est la victime de propitiation pour nos péchés, et non seulement pour les nôtres, mais aussi pour ceux de tout le monde.
\TextTitle{e. reconnaissance de la sainteté de Dieu}
\VS{3}Et nous savons que nous l'avons connu, si nous gardons ses commandements.
\VS{4}Celui qui dit : Je l'ai connu, et qui ne garde point ses commandements, est un menteur, et il n'y a point de vérité en lui.
\VS{5}Mais celui qui garde sa Parole, l'amour de Dieu est véritablement parfait en lui : et c'est par cela que nous savons que nous sommes en lui.
\VS{6}Celui qui dit qu'il demeure en lui doit aussi vivre comme Jésus-Christ lui-même a vécu.
\VS{7}Mes frères, je ne vous écris point un commandement nouveau, mais un commandement ancien, que vous avez eu dès le commencement ; et ce commandement ancien c'est la Parole que vous avez entendue dès le commencement.
\VS{8}Cependant, le commandement que je vous écris est un commandement nouveau, c’est une chose véritable en lui et en vous, parce que les ténèbres sont passées, et que la véritable Lumière paraît déjà.
\VS{9}Celui qui dit qu'il est dans la Lumière, et qui hait son frère, est dans les ténèbres jusqu'à présent.
\VS{10}Celui qui aime son frère demeure dans la Lumière, et il n'y a rien en lui qui puisse le faire tomber.
\VS{11}Mais celui qui hait son frère est dans les ténèbres, et il marche dans les ténèbres, et il ne sait pas où il va car les ténèbres ont aveuglé ses yeux.
\TextTitle{Exhortations à la famille spirituelle}
\VS{12}Mes petits enfants, je vous écris parce que vos péchés vous sont pardonnés à cause de son Nom.
\VS{13}Pères, je vous écris parce que vous avez connu celui qui est dès le commencement. Jeunes gens, je vous écris parce que vous avez vaincu l’esprit du malin.
\VS{14}Jeunes enfants, je vous écris parce que vous avez connu le Père. Pères, je vous ai écrit parce que vous avez connu celui qui est dès le commencement. Jeunes gens, je vous ai écrit parce que vous êtes forts et que la Parole de Dieu demeure en vous, et que vous avez vaincu l’esprit du malin.
\TextTitle{Les enfants de Dieu ne doivent pas aimer le monde}
\VS{15}N'aimez point le monde ni les choses qui sont dans le monde ; si quelqu'un aime le monde, l'amour du Père n'est point en lui.
\VS{16}Car tout ce qui est dans le monde, c'est-à-dire la convoitise de la chair, la convoitise des yeux et l'orgueil de la vie, ne vient point du Père, mais vient du monde.
\VS{17}Et le monde passe, avec sa convoitise ; mais celui qui fait la volonté de Dieu demeure éternellement.
\TextTitle{Les enfants de Dieu mis en garde contre les apostats}
\VS{18}Petits enfants, c'est ici la dernière\FTNT{Dernière, du grec «~eschatos~», signifie «~dernier dans une succession dans le temps~». Voir Ge. 49:1-2.} heure ; et comme vous avez entendu que l'Antéchrist viendra, il y a maintenant plusieurs antéchrists ; et par là nous connaissons que c'est la dernière heure.
\VS{19}Ils sont sortis du milieu de nous, mais ils n'étaient pas des nôtres ; car s'ils avaient été des nôtres, ils seraient demeurés avec nous, mais c'est afin qu'il soit manifeste que tous ne sont point des nôtres.
\VS{20}Mais vous avez été oints par le Saint-Esprit, et vous connaissez toutes choses.
\VS{21}Je ne vous ai pas écrit comme si vous ne connaissiez point la vérité, mais parce que vous la connaissez, et qu'aucun mensonge ne vient de la vérité.
\VS{22}Qui est le menteur, sinon celui qui nie que Jésus est le Christ ? Celui-là est l'Antéchrist qui nie le Père et le Fils.
\VS{23}Quiconque nie le Fils, n'a point non plus le Père ; quiconque confesse le Fils, a aussi le Père.
\VS{24}Que ce que vous avez entendu dès le commencement demeure en vous, car si ce que vous avez entendu dès le commencement demeure en vous, vous demeurerez aussi dans le Fils et dans le Père.
\VS{25}Et c'est ici la promesse qu'il nous a faite, à savoir la vie éternelle.
\VS{26}Je vous ai écrit ces choses au sujet de ceux qui vous séduisent.
\VS{27}Mais l'onction que vous avez reçue de lui demeure en vous, et vous n'avez pas besoin qu'on vous enseigne ; mais comme la même onction vous enseigne toutes choses, qu'elle est véritable et n'est pas un mensonge, demeurez en lui selon les enseignements qu’elle vous a donnés.
\TextTitle{Exhortations à deumeurer en Christ}
\VS{28}Maintenant donc, mes petits enfants, demeurez en lui ; afin que quand il apparaîtra, nous ayons de l'assurance, et que nous ne soyons point confus devant lui lors de son avènement\FTNT{Avènement, du grec «~parousia~», veut dire «~l’arrivée~» ou «~la présence~». Lors de cette seconde venue, le Messie prendra son Epouse pour les noces, ensuite il posera ses pieds sur le Mont des Oliviers, détruira les armées de l'Antichrist, puis commencera son règne de mille ans. Voir Za. 14.}.
\VS{29}Si vous savez qu'il est juste, sachez que quiconque fait ce qui est juste est né de lui.
\Chap{3}
\VerseOne{}Voyez quelle charité le Père nous a témoignée, pour que nous soyons appelés enfants de Dieu ! Mais le monde ne nous connaît point, parce qu'il ne l'a point connu.
\VS{2}Mes bien-aimés, nous sommes maintenant enfants de Dieu, et ce que nous serons n'est pas encore manifesté ; or nous savons que lorsque le Fils de Dieu apparaîtra, nous serons semblables à lui, car nous le verrons tel qu'il est.
\VS{3}Et quiconque a cette espérance en lui se purifie, comme lui aussi est pur.
\TextTitle{Caractéristiques des enfants de Dieu et des enfants du diable}
\VS{4}Quiconque pèche, transgresse la loi, car le péché est la transgression de la loi.
\VS{5}Or vous savez qu'il est apparu pour ôter nos péchés ; et il n'y a point de péché en lui.
\VS{6}Quiconque demeure en lui ne pèche point ; quiconque pèche, ne l'a pas vu, et ne l'a pas connu.
\VS{7}Mes petits-enfants, que personne ne vous séduise. Celui qui fait ce qui est juste est une personne juste, comme Jésus-Christ est juste.
\VS{8}Celui qui vit dans le péché est du diable, car le diable pèche dès le commencement. Or le Fils de Dieu est apparu afin de détruire les œuvres du diable.
\VS{9}Quiconque est né de Dieu ne vit pas dans le péché, car la semence de Dieu demeure en lui ; et il ne peut pécher, parce qu'il est né de Dieu.
\VS{10}Et c'est par là que nous connaissons les enfants de Dieu et les enfants du diable. Quiconque ne fait pas ce qui est juste et qui n'aime pas son frère n'est point de Dieu.
\VS{11}Car ce qui vous a été annoncé et ce que vous avez entendu dès le commencement c’est que nous nous aimions les uns les autres.
\VS{12}Et que nous ne soyons pas comme Caïn\FTNT{La doctrine de la semence du serpent est présentée par certains comme une explication du sens caché de la chute de l’homme dans le jardin en Eden et du péché originel. Selon cette doctrine, l’acte sexuel serait le fruit de l’arbre de la connaissance du bien et du mal. Cependant, cette doctrine n’est pas biblique, Eve n’a jamais eu de relations sexuelles avec le serpent. Dans Jean 8:44, lorsque Jésus dit aux pharisiens «~vous avez pour père le diable…~» suppose-t-il que le diable engendre des enfants physiquement ? Bien sûr que non !}, qui était de l'esprit malin et qui tua son frère. Et pourquoi le tua-t-il ? C’est parce que ses œuvres étaient mauvaises, et que celles de son frère étaient justes.
\VS{13}Mes frères, ne vous étonnez point si le monde vous hait.
\VS{14}Nous savons que nous sommes passés de la mort à la vie parce que nous aimons nos frères. Celui qui n'aime pas son frère demeure dans la mort.
\VS{15}Quiconque hait son frère est un meurtrier, et vous savez qu'aucun meurtrier ne possède la vie éternelle.
\VS{16}Nous avons connu la charité en ce qu'il a donné sa vie pour nous ; nous aussi, nous devons donner nos vies pour nos frères\FTNT{Jn. 15:13.}.
\VS{17}Si quelqu’un possède les biens du monde, et que voyant son frère dans la nécessité, il lui ferme ses entrailles, comment la charité de Dieu demeure-t-elle en lui ?
\VS{18}Mes petits-enfants, n'aimons pas en paroles et avec la langue, mais par des œuvres et en vérité.
\VS{19}Car c'est par là que nous connaissons que nous sommes de la vérité ; et nous rassurerons ainsi nos cœurs devant lui.
\VS{20}Si notre cœur nous condamne, certes Dieu est plus grand que notre cœur, et il connaît toutes choses.
\VS{21}Mes bien-aimés, si notre cœur ne nous condamne point, nous avons de l’assurance devant Dieu.
\VS{22}Et quoi que nous demandions, nous le recevons de lui, parce que nous gardons ses commandements, et que nous faisons les choses qui lui sont agréables.
\VS{23}Et c'est ici son commandement, que nous croyions au Nom de son Fils Jésus-Christ, et que nous nous aimions les uns les autres, selon le commandement qu’il nous a donné.
\VS{24}Celui qui garde ses commandements demeure en Jésus-Christ, et Jésus-Christ demeure en lui ; et par là nous connaissons qu'il demeure en nous, par l'Esprit qu'il nous a donné.
\Chap{4}
\TextTitle{Il faut éprouver les esprits}
\VerseOne{}Mes bien-aimés, ne croyez pas à tout esprit, mais éprouvez les esprits pour savoir s'ils sont de Dieu, car plusieurs faux prophètes sont venus dans le monde.
\TextTitle{[Caractéristiques des faux prophètes
\\a. leur confession sur Jésus-Christ]}
\VS{2}Reconnaissez à cette marque l'Esprit de Dieu : Tout esprit qui confesse que Jésus-Christ est venu en chair est de Dieu.
\VS{3}Et tout esprit qui ne confesse point que Jésus-Christ est venu en chair n'est point de Dieu ; c’est l'esprit de l'Antéchrist, dont vous avez appris la venue, et qui maintenant est déjà dans le monde.
\VS{4}Mes petits-enfants, vous êtes de Dieu, et vous les avez vaincus, parce que celui qui est en vous est plus grand que celui qui est dans le monde.
\TextTitle{b. leur appartenance au monde}
\VS{5}Eux, ils sont du monde, c'est pourquoi ils parlent comme étant du monde, et le monde les écoute.
\VS{6}Nous sommes de Dieu ; celui qui connaît Dieu nous écoute ; mais celui qui n'est pas de Dieu ne nous écoute point ; c’est par là que nous connaissons l'esprit de vérité et l'esprit de l’erreur.
\TextTitle{La charité de Dieu}
\VS{7}Mes bien-aimés, aimons-nous les uns les autres, car la charité est de Dieu ; et quiconque aime son prochain est né de Dieu et connaît Dieu.
\VS{8}Celui qui n'aime point son prochain n'a pas connu Dieu, car Dieu est Charité\FTNT{« Agapé » en grec.}.
\VS{9}La charité de Dieu a été manifestée envers nous en ce que Dieu a envoyé son Fils unique dans le monde, afin que nous vivions par lui.
\VS{10}Et cette charité consiste, non point en ce que nous avons aimé Dieu, mais en ce qu'il nous a aimés, et qu'il a envoyé son Fils pour être la propitiation\FTNT{Du grec «~hilasmos~» qui signifie «~apaisement~». Les écritures nous parlent aussi du «~propitiatoire~», c'est-à-dire « le siège de la misericorde » ou « Lieu de l'expiation ». Le propitiatoire était une plaque en or du sommet de l'Arche de l'Alliance. Le souverain sacrificateur l'aspergeait sept fois, le jour de l'expiation afin de reconcilier symboliquement Yahweh et son peuple. Voir Ex. 25:17-22} pour nos péchés.
\VS{11}Mes bien-aimés, si Dieu nous a ainsi aimés, nous devons aussi nous aimer les uns les autres.
\VS{12}Personne n'a jamais vu Dieu ; si nous nous aimons les uns les autres, Dieu demeure en nous et sa charité est parfaite en nous.
\VS{13}A ceci nous connaissons que nous demeurons en lui, et lui en nous, c'est qu'il nous a donné de son Esprit.
\VS{14}Et nous l'avons vu, et nous témoignons que le Père a envoyé le Fils pour être le sauveur du monde.
\VS{15}Quiconque confessera que Jésus est le Fils de Dieu, Dieu demeure en lui, et lui en Dieu.
\VS{16}Et nous, nous avons connu et cru en la charité que Dieu a pour nous. Dieu est charité ; et celui qui demeure dans la charité, demeure en Dieu, et Dieu en lui.
\VS{17}Tel il est, tels aussi nous sommes dans ce monde : C’est en cela que la charité est parfaite en nous, afin que nous ayons de l’assurance au jour du jugement.
\VS{18}Il n'y a point de crainte dans la charité, mais la parfaite charité bannit la crainte, car la crainte suppose un châtiment ; or celui qui craint n'est pas accompli dans la charité.
\VS{19}Nous l'aimons, parce qu'il nous a aimés le premier.
\VS{20}Si quelqu'un dit : J'aime Dieu, et qu’il haïsse son frère, c’est un menteur ; car comment celui qui n'aime point son frère, qu'il voit, peut-il aimer Dieu, qu’il ne voit pas ?
\VS{21}Et nous avons ce commandement de sa part, que celui qui aime Dieu, aime aussi son frère.
\Chap{5}
\TextTitle{La foi, principe qui triomphe des conflits avec le monde}
\VerseOne{}Quiconque croit que Jésus est le Christ, est né de Dieu, et quiconque aime celui qui l'a engendré, aime aussi celui qui est né de lui.
\VS{2}Nous connaissons à ceci que nous aimons les enfants de Dieu, lorsque nous aimons Dieu et que nous gardons ses commandements.
\VS{3}Car c'est en ceci que consiste notre amour pour Dieu : Que nous gardions ses commandements. Et ses commandements ne sont point pénibles.
\VS{4}Parce que tout ce qui est né de Dieu est victorieux du monde ; et ce qui nous fait remporter la victoire sur le monde, c'est notre foi.
\VS{5}Qui est celui qui a remporté la victoire sur le monde, sinon celui qui croit que Jésus est le Fils de Dieu ?
\VS{6}C'est ce Jésus, le Christ, qui est venu avec l’eau et le sang, et pas seulement avec l'eau, mais avec l'eau et le sang ; et c'est l'Esprit qui rend témoignage, or l'Esprit est la vérité.
\VS{7}Car il y en a trois dans le ciel qui rendent témoignage, le Père, la Parole, et le Saint-Esprit ; et ces trois-là ne sont qu'un\FTNT{Dieu est UN. Voir De. 6:4.}.
\VS{8}Il y en a aussi trois qui rendent témoignage sur la terre, à savoir l'Esprit, l'eau, et le sang, et ces trois-là se rapportent à un.
\TextTitle{Une assurance bénie}
\VS{9}Si nous recevons le témoignage des hommes, le témoignage de Dieu est plus grand, car le témoignage de Dieu consiste en ce qu’il a rendu témoignage à son Fils.
\VS{10}Celui qui croit au Fils de Dieu a le témoignage de Dieu en lui-même ; mais celui qui ne croit pas Dieu, le fait menteur, car il ne croit pas au témoignage que Dieu a rendu de son Fils.
\VS{11}Et c'est ici le témoignage, à savoir que Dieu nous a donné la vie éternelle, et cette vie est dans son Fils.
\VS{12}Celui qui a le Fils a la vie, celui qui n'a pas le Fils de Dieu n'a pas la vie.
\VS{13}Je vous ai écrit ces choses, à vous qui croyez au Nom du Fils de Dieu, afin que vous sachiez que vous avez la vie éternelle, et afin que vous croyiez au Nom du Fils de Dieu.
\VS{14}Et c'est ici l’assurance que nous avons en Dieu, que si nous demandons quelque chose selon sa volonté, il nous exauce.
\VS{15}Et si nous savons qu'il nous exauce, quelque chose que nous demandions, nous savons que nous possédons la chose que nous lui avons demandée.
\VS{16}Si quelqu'un voit son frère commettre un péché qui ne mène point à la mort\FTNT{Le péché qui mène à la mort c’est le blasphème contre le Saint-Esprit. Voir commentaire en Mt. 12:32.}, qu’il prie pour lui, et Dieu donnera la vie à ce frère. Il la donnera à ceux qui commettent un péché qui ne mène point à la mort. Il y a un péché qui mène à la mort ; je ne te dis point de prier pour ce péché-là.
\VS{17}Toute iniquité est un péché, mais il y a quelque péché qui ne mène pas à la mort.
\VS{18}Nous savons que quiconque est né de Dieu ne pèche point ; mais celui qui est engendré de Dieu se garde lui-même, et le malin ne le touche point.
\VS{19}Nous savons que nous sommes nés de Dieu, mais le monde entier est plongé dans le mal.
\TextTitle{Conclusion}
\VS{20}Or nous savons que le Fils de Dieu est venu, et il nous a donné l'intelligence pour connaître le Véritable ; et nous sommes dans le Véritable, en son Fils Jésus-Christ. Il est le vrai\FTNT{Dans Jean 17:3, le Père est présenté comme le Vrai Dieu ; le terme grec traduit par «~vrai~» dans Jean est aussi appliqué à Jésus dans ce passage. Jésus est donc le Vrai Dieu.} Dieu, et la vie éternelle.
\VS{21}Mes petits enfants, gardez-vous des idoles. Amen.
\PPE{}
\end{multicols}

\clearpage\ShortTitle{2 Jean}\BookTitle{2 Jean}\BFont
\noindent\hrulefill
{\footnotesize
\textit{
\bigskip
{\centering{}
\\Auteur : Jean
\\(Gr. : Ioannes)
\\Signification : Yahweh a fait grâce
\\Thème : Amour et vérité
\\Date de rédaction : Env. 85 ap. J.-C.\\}
}
%\bigskip
\textit{
\\Il semblerait que cette épître était adressée à une église se réunissant chez une personne du nom de Kyria. Jean les invite à demeurer dans la communion avec Dieu et les met en garde contre les hérésies et la fréquentation des faux docteurs.\bigskip
}
}
\par\nobreak\noindent\hrulefill
\begin{multicols}{2}
\Chap{1}
\TextTitle{Introduction}
\VerseOne{}L'ancien à Kyria l’élue, et à ses enfants, que j'aime dans la vérité, et ce n’est pas moi seul qui les aime, mais aussi tous ceux qui ont connu la vérité.
\VS{2}A cause de la vérité qui demeure en nous, et qui sera avec nous éternellement,
\VS{3}que la grâce, la miséricorde, et la paix de la part de Dieu le Père, et de la part du Seigneur Jésus-Christ, le Fils du Père, soient avec vous dans la vérité et dans la charité.
\TextTitle{La marche dans la vérité et dans la charité}
\VS{4}Je me suis fort réjoui d'avoir trouvé quelques-uns de tes enfants qui marchent dans la vérité selon le commandement que nous avons reçu du Père.
\VS{5}Et maintenant, ô Kyria ! Je te prie, non comme t'écrivant un nouveau commandement, mais celui que nous avons eu dès le commencement, que nous ayons de la charité les uns pour les autres.
\VS{6}Et c'est ici la charité, que nous marchions selon ses commandements. Et c'est là son commandement, comme vous l'avez entendu dès le commencement, afin que vous l'observiez.
\TextTitle{Le signe du séducteur et de l'antéchrist}
\VS{7}Car plusieurs séducteurs sont venus dans le monde, qui ne confessent point que Jésus-Christ est venu en chair ; un tel homme est un séducteur et un Antéchrist.
\VS{8}Prenez garde à vous-mêmes, afin que vous ne perdiez point le fruit du travail que vous avez fait, mais que vous en receviez une pleine récompense.
\VS{9}Quiconque transgresse la doctrine de Jésus-Christ et ne lui demeure point fidèle n'a point Dieu ; celui qui demeure dans la doctrine de Christ a le Père et le Fils.
\VS{10}Si quelqu'un vient à vous, et qu'il n'apporte point cette doctrine, ne le recevez point dans votre maison, et ne le saluez pas ;
\VS{11}car celui qui le salue participe à ses mauvaises œuvres.
\TextTitle{Conclusion}
\VS{12}Quoique j’aie plusieurs choses à vous écrire, je n’ai pas voulu les écrire avec du papier et de l'encre, mais j'espère aller vers vous, et vous parler bouche à bouche, afin que notre joie soit parfaite.
\VS{13}Les enfants de ta sœur élue te saluent. Amen !
\PPE{}
\end{multicols}

\clearpage\ShortTitle{3 Jean}\BookTitle{3 Jean}\BFont
\noindent\hrulefill
{\footnotesize
\textit{
\bigskip
{\centering{}
\\Signifie : Yahweh a fait grâce
\\Thème : Sincérité, hospitalité et caractère chrétien
\\Auteur : Jean
\\Date de rédaction : Env. 85\\}
}
%\bigskip
\textit{
\\Cette épître fut destinée à Gaïus, l’un des responsables d’une église d’Asie Mineure dont Jean loua la piété et la générosité. Il l’avertit de l’orgueil et des agissements de Diotrèphe qui étaient contraires à la Parole mais souligna le bon témoignage de Démétrius.\bigskip
}
}
\par\nobreak\noindent\hrulefill
\begin{multicols}{2}
\TextTitle{[Introduction]}
\Chap{1}
\VerseOne{}L'ancien à Gaïus, le bien-aimé, que j'aime dans la vérité.
\VS{2}Bien-aimé, je souhaite que tu prospères{\FTNT{La prospérité dont il est question dans ce passage n’a rien à voir avec l’Evangile de prospérité qui met l’accent sur la richesse matérielle. Le mot grec «~euodoo~» signifie «~concevoir~» un voyage prospère et diligent~», «~mener par une voie directe et facile~», «~prospérer~», «~être heureux~».}} en toutes choses, et que tu sois en bonne santé, comme ton âme est en prospérité.
\VS{3}Car j’ai été fort réjoui quand les frères sont venus et ont rendu témoignage de ta sincérité, et comment tu marches dans la vérité.
\VS{4}Je n'ai pas de plus grande joie que d’apprendre que mes enfants marchent dans la vérité.
\TextTitle{[L'hospitalité]}
\VS{5}Bien-aimé, tu agis fidèlement dans tout ce que tu fais envers les frères et envers les étrangers,
\VS{6}qui en présence de l'église ont rendu témoignage de ta charité. Et tu feras bien de les accompagner dignement, comme il est séant selon Dieu.
\VS{7}Car ils sont partis pour son Nom, ne prenant rien des gentils.
\VS{8}Nous devons donc recevoir de tels hommes, afin d’être ouvriers avec eux pour la vérité.
\TextTitle{[Les mauvais actes de Diotrèphe et son caractère dominateur]}
\VS{9}J'ai écrit à l'Eglise, mais Diotrèphe, qui aime être le premier parmi eux, ne nous reçoit point.
\VS{10}C'est pourquoi, si je viens, je rappellerai les actions qu'il commet, en tenant contre nous de mauvais discours ; et n'étant pas content de cela, non seulement il ne reçoit pas les frères, mais il empêche même ceux qui veulent les recevoir et les chasse de l'église.
\VS{11}Bien-aimé, n'imite point le mal, mais le bien. Celui qui fait le bien est de Dieu ; mais celui qui fait le mal n'a point vu Dieu.
\TextTitle{[Témoignage de Démétrius]}
\VS{12}Tous rendent témoignage à Démétrius, et la vérité même le lui rend, et nous aussi nous lui rendons témoignage, et vous savez que notre témoignage est véritable.
\TextTitle{[Conclusion]}
\VS{13}J'avais plusieurs choses à écrire, mais je ne veux pas t'écrire avec l'encre et avec la plume.
\VS{14}Mais j'espère te voir bientôt, et nous parlerons de bouche à bouche.
\VS{15}Que la paix soit avec toi ! Les amis te saluent. Salue les amis, chacun par son nom.
\PPE{}
\end{multicols}

\clearpage\ShortTitle{Ap.}\BookTitle{Apocalypse}\BFont
\noindent\hrulefill
{\footnotesize
\textit{
\bigskip
{\centering{}
\\Auteur~: Jean
\\Thème~: L'aboutissement de toutes choses
\\(Gr.~: Apokalupsis)
\\Signification~: Mettre à nu, révélation d'une vérité, action de révéler
\\Date de rédaction~: Env. 95 ap. J.-C.\\}
}
\textit{
\\Le terme apocalypse, du grec «~apokalupsis~», évoque «~l'action de révéler ce qui était caché ou inconnu~». Ce mot a pour racine «~apokalupto~» qui signifie aussi «~découvrir, dévoiler ce qui est voilé ou recouvert~».
\\C'est à Patmos, île grecque de la mer Egée - où il s'exila en raison de la persécution de l'empereur Domitien (51 - 96 ap. J.C.) - que Jean reçut une révélation de Jésus-Christ ainsi qu'un message s'adressant aux «~sept églises~» qui constituaient certainement les villes de l'Asie Mineure où se trouvaient les principales concentrations de chrétiens. Si Ephèse figure dans les écrits de la Nouvelle Alliance et que Thyatire et Laodicée y sont brièvement mentionnées, les quatre autres églises - qu'on ne retrouve nulle part ailleurs dans les Ecritures - étaient sans doute le fruit du travail missionnaire de Paul. Les sept lettres s'adressent à l'ange de chacune de ces assemblées locales, autrement dit aux messagers de celles-ci (probablement un ancien ou un responsable).
\\Ce livre, qui arrive en conclusion des Ecritures, annonce les événements qui doivent précéder la fin de l'histoire de l'humanité.\bigskip
}
} 
\par\nobreak\noindent\hrulefill
\begin{multicols}{2}
\Chap{1}
\TextTitle{Introduction}
\VerseOne{}La révélation\FTNT{«~Apokalupsis~» en grec. Voir l'introduction du livre.} de Jésus-Christ, que Dieu lui a donnée pour montrer à ses serviteurs les choses qui doivent arriver bientôt, et qui les a fait connaître en les envoyant par son ange à Jean, son serviteur,
\VS{2}qui a annoncé la parole de Dieu, et le témoignage de Jésus-Christ, et toutes les choses qu'il a vues.
\VS{3}Béni est celui qui lit et ceux qui écoutent les paroles de cette prophétie, et qui gardent les choses qui y sont écrites~! Car le temps est proche.
\TextTitle{Jésus-Christ}
\VS{4}Jean aux sept églises qui sont en Asie~: Que la grâce et la paix vous soient données de la part de celui QUI EST, QUI ETAIT, et QUI VIENT\FTNT{Les prophètes ont prophétisé la venue de Yahweh en personne~: Es. 35:4~; Es. 40:10-11~; Es. 60:1-2~; Za. 14:1-21~; Jn. 14:1-3. Jésus-Christ est bien Yahweh qui vient.}, et de la part des sept Esprits qui sont devant son trône,
\VS{5}et de la part de Jésus-Christ, qui est le témoin fidèle, le premier-né d'entre les morts\FTNT{Voir commentaire Col. 1:15.}, et le Prince des rois de la terre.
\VS{6}A lui, dis-je, qui nous a aimés, et qui nous a lavés de nos péchés dans son sang, et qui a fait de nous des rois et des prêtres pour Dieu, son Père, à lui soient la gloire et la force aux siècles des siècles. Amen~!
\TextTitle{La venue du Christ}
\VS{7}Voici, il vient avec les nuées, et tout œil le verra, et même ceux qui l'ont percé~; et toutes les tribus de la terre se lamenteront devant lui. Oui, amen~!
\VS{8}Je suis l'Alpha et l'Oméga, le commencement et la fin, dit le Seigneur, QUI EST, QUI ETAIT, et QUI VIENT, le Tout-Puissant.
\TextTitle{La vision doit être écrite}
\VS{9}Moi Jean, qui suis aussi votre frère et qui participe à la tribulation, au règne, et à la patience de Jésus-Christ, j'étais sur l'île appelée Patmos à cause de la parole de Dieu, et du témoignage de Jésus-Christ.
\VS{10}Je fus ravi en esprit au jour du Seigneur, et j'entendis derrière moi une voix forte, comme le son d'une trompette,
\VS{11}qui disait~: Je suis l'Alpha et l'Oméga, le premier et le dernier. Ecris dans un livre ce que tu vois, et envoie-le aux sept églises qui sont en Asie, à savoir à Ephèse, à Smyrne, à Pergame, à Thyatire, à Sardes, à Philadelphie, et à Laodicée.
\VS{12}Alors je me retournai pour voir celui dont la voix m'avait parlé, et après m'être retourné, je vis sept chandeliers d'or,
\VS{13}et au milieu des sept chandeliers d'or, quelqu'un qui ressemblait à un fils d'homme, vêtu d'une longue robe, et ayant une ceinture d'or sur la poitrine.
\VS{14}Sa tête et ses cheveux étaient blancs comme de la laine blanche, et comme de la neige, et ses yeux étaient comme une flamme de feu.
\VS{15}Ses pieds étaient semblables à de l'airain ardent, comme s'ils avaient été embrasés dans une fournaise~; et sa voix était comme le bruit des grandes eaux.
\VS{16}Et il avait dans sa main droite sept étoiles, et de sa bouche sortait une épée aiguë à deux tranchants, et son visage était semblable au soleil lorsqu'il brille dans sa force.
\VS{17}Quand je le vis, je tombai à ses pieds comme mort, et il mit sa main droite sur moi, en me disant~: Ne crains pas~!
\VS{18}Je suis le premier et le dernier, et je vis~; j'étais mort, et voici, je suis vivant aux siècles des siècles. Amen~! Et je tiens les clefs de Hadès\FTNT{Voir commentaire Mt. 16:18.} et de la mort.
\VS{19}Ecris les choses que tu as vues, celles qui sont présentement, et celles qui doivent arriver ensuite.
\VS{20}Le mystère des sept étoiles que tu as vues dans ma main droite, et les sept chandeliers d'or. Les sept étoiles sont les anges des sept églises~; et les sept chandeliers que tu as vus sont les sept églises.
\Chap{2}
\TextTitle{Ephèse~: L'église qui a perdu le premier amour}
\VerseOne{}Ecris à l'ange\FTNT{Ange, du grec «~aggelos~»~: Envoyé, messager, un ange. Un messager de Dieu. Ce terme sert à désigner aussi bien les créatures spirituelles que les êtres humains.} de l'église d'Ephèse~: Voici ce que dit celui qui tient les sept étoiles dans sa main droite, et qui marche au milieu des sept chandeliers d'or~:
\VS{2}Je connais tes œuvres, et ton travail, et ta patience, et je sais que tu ne peux pas supporter les méchants, et que tu as éprouvé ceux qui se disent être apôtres et qui ne le sont pas, et que tu les as trouvés menteurs~;
\VS{3}et que tu as souffert, et que tu as eu de la patience, et que tu as travaillé pour mon Nom, et que tu ne t'es pas lassé.
\VS{4}Mais j'ai quelque chose contre toi, c'est que tu as abandonné ta première charité\FTNT{Il est question ici de l'amour «~agape~»~: L'amour fraternel, l'amour divin.}.
\VS{5}C'est pourquoi souviens-toi donc d'où tu es tombé, repens-toi, et fais tes premières œuvres. Autrement, je viendrai à toi à toute vitesse, et j'ôterai ton chandelier de sa place si tu ne te repens pas.
\VS{6}Mais pourtant tu as ceci de bon, c'est que tu hais les œuvres des Nicolaïtes\FTNT{Nicolaïtes~: Tiré du nom Nicolas, qui signifie littéralement «~victorieux du peuple~». Il s'agit d'une secte dont les membres furent peut-être des disciples d'un certain Nicolas, l'un des diacres de l'église d'Antioche qui aurait dévié (Ac. 6:5). Ces derniers suivaient la doctrine de Balaam, enseignant aux chrétiens qu'à cause du principe de liberté, ils pouvaient manger des viandes sacrifiées aux idoles et commettre des actes immoraux comme les Gentils.}, œuvres que je hais moi aussi.
\VS{7}Que celui qui a des oreilles entende ce que l'Esprit dit aux églises~! A celui qui vaincra, je lui donnerai à manger de l'arbre de vie, qui est au milieu du paradis de Dieu.
\TextTitle{Smyrne~: L'église sous la persécution}
\VS{8}Ecris aussi à l'ange de l'église de Smyrne~: Voici ce que dit celui qui est le premier et le dernier, qui a été mort, et qui est revenu à la vie~:
\VS{9}Je connais tes œuvres, ton affliction et ta pauvreté, quoique tu sois riche, et le blasphème de ceux qui se disent être Juifs et qui ne le sont pas, mais qui sont la synagogue de Satan.
\VS{10}Ne crains rien des choses que tu as à souffrir. Voici, il arrivera que le diable mettra quelques-uns d'entre vous en prison, afin que vous soyez éprouvés~; et vous aurez une affliction de dix jours. Sois fidèle jusqu'à la mort, et je te donnerai la couronne de vie.
\VS{11}Que celui qui a des oreilles, entende ce que l'Esprit dit aux églises~! Celui qui vaincra n'aura pas à souffrir la seconde mort.
\TextTitle{Pergame~: L'église établie dans le monde}
\VS{12}Ecris aussi à l'ange de l'église de Pergame~: Voici ce que dit celui qui a l'épée aiguë à deux tranchants\FTNT{Hé. 4:12.}~: 
\VS{13}Je connais tes œuvres, et le lieu où tu habites, à savoir là où est le trône de Satan. Et que cependant tu retiens mon Nom, et tu n'as pas renié ma foi, même aux jours d'Antipas, mon fidèle martyr\FTNT{Du grec «~martus~» qui signifie «~témoin~».}, qui a été mis à mort chez vous, là où Satan habite.
\VS{14}Mais j'ai quelque chose contre toi, c'est que tu as là des gens attachés à la doctrine de Balaam, qui enseignait à Balak à mettre un scandale devant les enfants d'Israël, afin qu'ils mangent des viandes sacrifiées aux idoles, et qu'ils se livrent à la fornication\FTNT{No. 25:1-2~; No. 31:16.}.
\VS{15}De même, toi aussi tu as des gens attachés à la doctrine des Nicolaïtes~; ce que je hais~!
\VS{16}Repens-toi donc, autrement je viendrai à toi à toute vitesse, et je les combattrai avec l'épée de ma bouche.
\VS{17}Que celui qui a des oreilles, entende ce que l'Esprit dit aux églises~! A celui qui vaincra, je lui donnerai à manger de la manne qui est cachée, et je lui donnerai un caillou blanc, et sur ce caillou sera écrit un nouveau nom, que nul ne connaît, sinon celui qui le reçoit.
\TextTitle{Thyatire~: L'église en temps d'idolâtrie}
\VS{18}Ecris aussi à l'ange de l'église de Thyatire~: Voici ce que dit le Fils de Dieu, qui a ses yeux comme une flamme de feu, et dont les pieds sont semblables à de l'airain ardent.
\VS{19}Je connais tes œuvres, ta charité, ton service, ta foi, ta patience, et que tes dernières œuvres surpassent les premières.
\VS{20}Mais j'ai quelque peu de chose contre toi, c'est que tu laisses cette femme Jézabel\FTNT{1 R. 16:31~; 1 R. 21:25~; 2 R. 9:7~; 2 R. 9:22.}, qui se dit prophétesse, enseigner et séduire mes serviteurs pour les porter à la fornication, et leur faire manger des choses sacrifiées aux idoles.
\VS{21}Et je lui ai donné du temps, afin qu'elle se repente de sa prostitution, mais elle ne s'est pas repentie.
\VS{22}Voici, je vais la jeter sur un lit, et mettre dans une grande affliction ceux qui commettent l'adultère avec elle, s'ils ne se repentent pas de leurs œuvres.
\VS{23}Et je ferai mourir de mort ses enfants~; et toutes les églises connaîtront que je suis celui qui sonde les reins et les cœurs, et je rendrai à chacun de vous selon ses œuvres.
\VS{24}Mais je vous dis à vous et aux autres qui sont à Thyatire, à tous ceux qui n'ont pas cette doctrine, et qui n'ont pas connu les profondeurs de Satan, comme ils disent, je vous dis~: Je ne mettrai pas sur vous d'autre charge.
\VS{25}Mais retenez ce que vous avez, jusqu'à ce que je vienne.
\VS{26}Car à celui qui aura vaincu, et qui aura gardé mes œuvres jusqu'à la fin, je lui donnerai autorité sur les nations.
\VS{27}Et il les gouvernera avec un sceptre de fer, et elles seront brisées comme les vases d'un potier, ainsi que j'en ai moi-même reçu le pouvoir de mon Père.
\VS{28}Et je lui donnerai l'étoile du matin.
\VS{29}Que celui qui a des oreilles entende ce que l'Esprit dit aux églises~!
\Chap{3}
\TextTitle{Sardes~: L'église morte}
\VerseOne{}Ecris aussi à l'ange de l'église de Sardes~: Voici ce que dit celui qui a les sept Esprits de Dieu, et les sept étoiles~: Je connais tes œuvres. Tu as la réputation d'être vivant, mais tu es mort.
\VS{2}Sois vigilant, et affermis le reste qui va mourir~; car je n'ai pas trouvé tes œuvres parfaites devant Dieu.
\VS{3}Souviens-toi donc des choses que tu as reçues et entendues, garde-les, et repens-toi. Si tu ne veilles pas, je viendrai contre toi comme un voleur, et tu ne sauras pas à quelle heure je viendrai contre toi\FTNT{Mt. 24:43~; Lu. 12:39~; 1 Th. 5:2~; 2 Pi. 3:10.}.
\VS{4}Toutefois, tu as quelque peu de personnes à Sardes qui n'ont pas souillé leurs vêtements, et qui marcheront avec moi en vêtements blancs, car ils en sont dignes.
\VS{5}Celui qui vaincra sera vêtu de vêtements blancs, et je n'effacerai pas son nom du Livre de vie, mais je confesserai son nom devant mon Père, et devant ses anges.
\VS{6}Que celui qui a des oreilles, entende ce que l'Esprit dit aux églises~!
\TextTitle{Philadelphie~: L'église réveillée et fidèle}
\VS{7}Ecris aussi à l'ange de l'église de Philadelphie~: Voici ce que dit le Saint et le Véritable, qui a la clef de David, qui ouvre et nul ne ferme, qui ferme et nul n'ouvre.
\VS{8}Je connais tes œuvres. Voici, j'ai ouvert une porte devant toi, et personne ne peut la fermer~; parce que tu as peu de puissance, que tu as gardé ma parole, et que tu n'as pas renié mon Nom.
\VS{9}Voici, je ferai venir ceux de la synagogue de Satan qui se disent Juifs, et ne le sont pas, mais qui mentent~; voici, dis-je, je les ferai venir et se prosterner à tes pieds, et ils connaîtront que je t'aime.
\VS{10}Parce que tu as gardé la parole de ma persévérance, je te garderai aussi de l'heure de la tentation qui doit arriver dans le monde entier, pour éprouver les habitants de la terre.
\VS{11}Voici, je viens à toute vitesse. Tiens ferme ce que tu as, afin que personne ne t'enlève ta couronne.
\VS{12}Celui qui vaincra, je ferai de lui une colonne dans le temple de mon Dieu, et il n'en sortira plus~; et j'écrirai sur lui le Nom de mon Dieu, et le nom de la cité de mon Dieu, qui est la nouvelle Jérusalem qui descend du ciel d'auprès de mon Dieu, et mon nouveau Nom.
\VS{13}Que celui qui a des oreilles entende ce que l'Esprit dit aux églises~!
\TextTitle{Laodicée~: L'église apostate}
\VS{14}Ecris aussi à l'ange de l'église de Laodicée~: Voici ce que dit l'Amen, le témoin fidèle et véritable, le commencement de la création de Dieu~:
\VS{15}Je connais tes œuvres. Je sais que tu n'es ni froid ni bouillant~; puisses-tu être ou froid ou bouillant~!
\VS{16}Parce que tu es tiède, et que tu n'es ni froid ni bouillant, je te vomirai de ma bouche.
\VS{17}Car tu dis~: Je suis riche, je suis dans l'abondance, et je n'ai besoin de rien~; mais tu ne sais pas que tu es malheureux, misérable, pauvre, aveugle et nu.
\VS{18}Je te conseille d'acheter de moi de l'or éprouvé par le feu, afin que tu deviennes riche; et des vêtements blancs, afin que tu sois vêtu et que la honte de ta nudité ne paraisse pas~; et d'oindre tes yeux de collyre, afin que tu voies.
\VS{19}Moi, je reprends et je châtie\FTNT{De. 8:5~; 2 S. 7:14~; Pr. 13:24~; Hé. 12:7.} tous ceux que j'aime. Aie donc du zèle et repens-toi.
\TextTitle{Le Messie se retrouve hors des églises apostates}
\VS{20}Voici, je me tiens à la porte, et je frappe. Si quelqu'un entend ma voix et m'ouvre la porte, j'entrerai chez lui, et je souperai avec lui, et lui avec moi.
\VS{21}Celui qui vaincra, je le ferai asseoir avec moi sur mon trône, ainsi que j'ai vaincu et me suis assis avec mon Père sur son trône.
\VS{22}Que celui qui a des oreilles entende ce que l'Esprit dit aux églises~!
\Chap{4}
\TextTitle{Vision avant l'ouverture des sceaux}
\VerseOne{}Après ces choses, je regardai, et voici une porte était ouverte dans le ciel. Et la première voix que j'avais entendue, comme le son d'une trompette, et qui parlait avec moi, me dit~: Monte ici, et je te montrerai les choses qui doivent arriver à l'avenir.
\VS{2}Aussitôt, je fus ravi en esprit. Et voici, un trône était dressé dans le ciel, et sur ce trône, quelqu'un était assis.
\VS{3}Et celui qui y était assis était semblable à une pierre de jaspe et de sardoine~; et le trône était environné d'un arc-en-ciel semblable à de l'émeraude.
\TextTitle{Les trônes des vingt-quatre anciens}
\VS{4}Et il y avait autour du trône vingt-quatre trônes et je vis sur ces trônes vingt-quatre anciens assis, vêtus de vêtements blancs, et ayant sur leurs têtes des couronnes d'or.
\VS{5}Et du trône sortaient des éclairs, des tonnerres, et des voix~; et il y avait devant le trône sept lampes de feu ardentes, qui sont les sept Esprits de Dieu.
\TextTitle{Le Messie est digne de recevoir la louange et la gloire}
\VS{6}Et devant le trône, il y avait une mer de verre semblable à du cristal~; et au milieu du trône et autour du trône quatre animaux, pleins d'yeux devant et derrière.
\VS{7}Et le premier animal était semblable à un lion~; le second animal était semblable à un veau~; le troisième animal avait la face comme un homme~; et le quatrième animal était semblable à un aigle qui vole.
\VS{8}Et les quatre animaux avaient chacun six ailes, et tout autour et au-dedans ils étaient pleins d'yeux~; et ils ne cessent pas de dire jour et nuit~: Saint~! Saint~! Saint est le Seigneur Dieu Tout-puissant, QUI ETAIT, QUI EST, et QUI VIENT.
\VS{9}Et quand ces animaux rendaient gloire et honneur et des actions de grâces à celui qui était assis sur le trône, à celui qui est vivant aux siècles des siècles,
\VS{10}les vingt-quatre anciens se prosternaient devant celui qui était assis sur le trône, et adoraient celui qui est vivant aux siècles des siècles, et ils jetaient leurs couronnes devant le trône, en disant~:
\VS{11}Seigneur, tu es digne de recevoir gloire, honneur et puissance~; car tu as créé toutes choses, et c'est par ta volonté qu'elles existent et qu'elles ont été créées.
\Chap{5}
\TextTitle{Le Messie est le seul digne d'ouvrir le livre}
\VerseOne{}Puis je vis dans la main droite de celui qui était assis sur le trône, un livre écrit en dedans et en dehors, scellé de sept sceaux.
\VS{2}Et je vis aussi un ange remarquable par sa force, qui proclamait d'une voix forte~: Qui est digne d'ouvrir le livre, et d'en rompre les sceaux~?
\VS{3}Et il n'y avait personne, ni dans le ciel, ni sur la terre, ni sous la terre qui pouvait ouvrir le livre, ni le regarder.
\VS{4}Et je pleurais beaucoup parce que personne n'était trouvé digne d'ouvrir le livre, ni de le lire, ni de le regarder.
\VS{5}Et l'un des anciens me dit~: Ne pleure pas, voici le Lion qui vient de la tribu de Juda, de la racine de David, a vaincu pour ouvrir le livre et pour en rompre les sept sceaux.
\VS{6}Et je regardai, et voici il y avait au milieu du trône et des quatre animaux, et au milieu des anciens, un Agneau qui se tenait là comme immolé, ayant sept cornes, et sept yeux, qui sont les sept Esprits de Dieu envoyés par toute la terre.
\VS{7}Et il vint et prit le livre de la main droite de celui qui était assis sur le trône.
\TextTitle{L'Agneau est adoré\FTNTT{Ph. 2:9-11.}}
\VS{8}Et quand il eut pris le livre, les quatre animaux et les vingt-quatre anciens se prosternèrent devant l'Agneau, ayant chacun des harpes et des coupes d'or pleines de parfums, qui sont les prières des saints.
\VS{9}Et ils chantaient un cantique nouveau, en disant~: Tu es digne de prendre le livre, et d'en ouvrir les sceaux~; car tu as été mis à mort, et tu nous as rachetés pour Dieu par ton sang, de toute tribu, de toute langue, de tout peuple, et de toute nation~;
\VS{10}et tu as fait de nous des rois et des prêtres pour notre Dieu~; et nous régnerons sur la terre.
\VS{11}Puis je regardai, et j'entendis la voix de plusieurs anges autour du trône, et des anciens; et leur nombre était de plusieurs millions.
\VS{12}Et ils disaient à haute voix~: L'Agneau qui a été mis à mort est digne de recevoir puissance, richesses, sagesse, force, honneur, gloire et louange.
\VS{13}J'entendis aussi toutes les créatures qui sont dans le ciel, sur la terre, et sous la terre, et dans la mer, et toutes les choses qui y sont, disant~: A celui qui est assis sur le trône et à l'Agneau, soient louange, honneur, gloire, et force, aux siècles des siècles~!
\VS{14}Et les quatre animaux disaient~: Amen~! Et les vingt-quatre anciens se prosternèrent et adorèrent celui qui est vivant aux siècles des siècles.
\Chap{6}
\TextTitle{Premier sceau~: Le cavalier qui part pour vaincre}
\VerseOne{}Et quand l'Agneau eut ouvert l'un des sceaux, je regardai, et j'entendis l'un des quatre animaux qui disait comme avec une voix de tonnerre~: Viens, et vois.
\VS{2}Je regardai, et je vis un cheval blanc~; celui qui était monté dessus avait un arc, et il lui fut donné une couronne~; et il est sortit en vainqueur pour vaincre\FTNT{Contrairement aux apparences, ce cavalier couronné d'un diadème et qui monte un cheval blanc n'est pas Jésus-Christ, mais l'Antichrist qui singe le retour glorieux du Seigneur~: Da. 7:21~; Mt. 24:4-5~; 2 Th. 2:9-12~; Ap. 13:7. Le vrai Christ revenant triomphalement avec son Eglise est décrit en Ap. 19:11-16.}.
\TextTitle{Deuxième sceau~: La guerre}
\VS{3}Et quand il eut ouvert le second sceau, j'entendis le second animal qui disait~: Viens, et vois.
\VS{4}Et il sortit un autre cheval qui était roux~; il fut donné à celui qui était monté dessus de pouvoir ôter la paix de la terre, afin que les hommes se tuent les uns les autres~; et il lui fut donné une grande épée.
\TextTitle{Troisième sceau~: La famine}
\VS{5}Et quand il eut ouvert le troisième sceau, j'entendis le troisième animal qui disait~: Viens, et vois. Je regardai, et je vis un cheval noir, et celui qui était monté dessus avait une balance dans sa main.
\VS{6}Et j'entendis au milieu des quatre animaux une voix qui disait~: Une mesure de blé pour un denier, et les trois mesures d'orge pour un denier~; mais ne fais pas de mal au vin et à l'huile.
\TextTitle{Quatrième sceau~: La mort}
\VS{7}Et quand il eut ouvert le quatrième sceau, j'entendis la voix du quatrième animal qui disait~: Viens, et vois.
\VS{8}Je regardai, et je vis un cheval verdâtre~; et celui qui était monté dessus se nommait la Mort, et le Hadès l'accompagnait. Il leur fut donné le pouvoir sur le quart de la terre pour tuer par l'épée, par la famine, par la mortalité, et par les bêtes sauvages de la terre.
\TextTitle{Cinquième sceau~: Les martyrs}
\VS{9}Et quand il eut ouvert le cinquième sceau, je vis sous l'autel les âmes de ceux qui avaient été tués pour la parole de Dieu, et pour le témoignage qu'ils avaient gardé.
\VS{10}Et elles criaient à haute voix, disant~: Jusqu'à quand, Seigneur qui es saint et véritable, ne jugeras-tu pas et ne vengeras-tu pas notre sang de ceux qui habitent sur la terre~?
\VS{11}Et il leur fut donné à chacun des robes blanches, et il leur fut dit de se tenir en repos encore un peu de temps, jusqu'à ce que le nombre de leurs compagnons de service, et de leurs frères qui doivent être mis à mort comme eux, soit complet.
\TextTitle{Sixième sceau~: L'anarchie}
\VS{12}Et je regardai quand il eut ouvert le sixième sceau, et voici, il se fit un grand tremblement de terre, et le soleil devint noir comme un sac de crin, et la lune entière devint comme du sang.
\VS{13}Et les étoiles du ciel tombèrent sur la terre\FTNT{Mt. 24:29~; Mc. 13:25.}, comme lorsque le figuier est agité par un grand vent et laisse tomber ses figues encore vertes.
\VS{14}Et le ciel se retira comme un livre qu'on roule~; et toutes les montagnes et les îles furent remuées de leurs places.
\VS{15}Et les rois de la terre, et les princes, et les riches, et les capitaines, et les puissants, et tout esclave, et tout homme libre se cachèrent dans les cavernes et entre les rochers des montagnes.
\VS{16}Et ils disaient aux montagnes et aux rochers~: Tombez sur nous\FTNT{Lu. 23:30.}, et cachez-nous devant la face de celui qui est assis sur le trône, et devant la colère de l'Agneau~;
\VS{17}car le grand jour de sa colère est venu, et qui peut subsister~?
\Chap{7}
\TextTitle{Les 144 000 marqués du sceau de Dieu}
\VerseOne{}Après cela, je vis quatre anges qui se tenaient aux quatre coins de la terre, et qui retenaient les quatre vents de la terre, afin qu'ils ne soufflent pas sur la terre, ni sur la mer, ni sur aucun arbre.
\VS{2}Puis je vis un autre ange qui montait du côté de l'orient, tenant le sceau du Dieu vivant, et il cria d'une voix forte aux quatre anges à qui il avait été donné de faire du mal à la terre et à la mer,
\VS{3}et leur dit~: Ne faites pas de mal à la terre, ni à la mer, ni aux arbres, jusqu'à ce que nous ayons marqué du sceau les serviteurs de notre Dieu sur leurs fronts.
\VS{4}Et j'entendis que le nombre de ceux qui avaient été marqués du sceau était de cent quarante-quatre mille, de toutes les tribus des enfants d'Israël.
\VS{5}de la tribu de Juda, douze mille marqués du sceau~; de la tribu de Ruben, douze mille marqués du sceau~; de la tribu de Gad, douze mille marqués du sceau~;
\VS{6}de la tribu d'Aser, douze mille marqués du sceau~; de la tribu de Nephthali, douze mille marqués du sceau~; de la tribu de Manassé, douze mille marqués du sceau~;
\VS{7}de la tribu de Siméon, douze mille marqués du sceau~; de la tribu de Lévi, douze mille marqués du sceau~; de la tribu d'Issacar, douze mille marqués du sceau~;
\VS{8}de la tribu de Zabulon, douze mille marqués du sceau~; de la tribu de Joseph, douze mille marqués du sceau~; de la tribu de Benjamin, douze mille marqués du sceau.
\TextTitle{Multitude de sauvés pendant la grande tribulation}
\VS{9}Après cela, je regardai, et voici une grande multitude de gens, que personne ne pouvait compter, de toute nation, de toute tribu, de tout peuple et de toute langue, se tenaient devant le trône, et devant l'Agneau, vêtus de longues robes blanches, et ils avaient des palmes dans leurs mains.
\VS{10}Et ils criaient d'une voix forte, en disant~: Le salut est à notre Dieu, qui est assis sur le trône, et à l'Agneau.
\VS{11}Et tous les anges se tenaient autour du trône, et des anciens, et des quatre animaux, et ils se prosternèrent devant le trône sur leurs faces et adorèrent Dieu,
\VS{12}en disant~: Amen~! La louange, la gloire, la sagesse, les actions de grâces, l'honneur, la puissance et la force soient à notre Dieu, aux siècles des siècles. Amen~!
\VS{13}Et l'un des anciens prit la parole et me dit~: Ceux qui sont revêtus de longues robes blanches, qui sont-ils et d'où sont-ils venus~?
\VS{14}Et je lui dis~: Seigneur, tu le sais. Et il me dit~: Ce sont ceux qui sont venus de la grande tribulation\FTNT{Les saints ont toujours été persécutés. Cela a débuté dès la Genèse avec Caïn qui tua son frère Abel (Ge. 4:5-10). La grande tribulation correspond néanmoins à une période de persécutions particulièrement cruelles qui seront orchestrées par l'homme impie (à la tête de plusieurs nations) principalement contre les Juifs (Jé. 30:7~; Da. 9:24~; Lu. 21:20-24) et sans doute contre les personnes converties à Christ issues des nations (Ap. 7:9-17~; Ap. 12:17). Le Seigneur Jésus a prédit la grande tribulation à ses disciples (Mt. 24:15-29~; Mc. 13:14-19) en précisant qu'en ce temps là on verrait «~l'abomination de la désolation~» établie en lieu saint prophétisée par Daniel (Da. 11:31). La grande tribulation durera trois ans et demi, c'est ce que Daniel appelle «~un temps, des temps et la moitié d'un temps~» (Da. 7:25~; Ap. 11:3.) L'ère de paix factice instaurée par l'impie cédera alors soudainement la place à un temps d'angoisse sans précédent (1 Th. 5:3).}, et qui ont lavé et blanchi leurs longues robes dans le sang de l'Agneau.
\VS{15}C'est pourquoi ils sont devant le trône de Dieu, et ils le servent jour et nuit dans son temple~; et celui qui est assis sur le trône habitera avec eux.
\VS{16}Ils n'auront plus faim ni soif, et le soleil ne les frappera plus, ni aucune chaleur.
\VS{17}Car l'Agneau qui est au milieu du trône les paîtra, et les conduira aux sources des eaux de la vie, et Dieu essuiera toutes les larmes de leurs yeux.
\Chap{8}
\TextTitle{Septième sceau~: Annonce des sept trompettes\FTNTT{Ap. 4:1.}}
\VerseOne{}Et quand il eut ouvert le septième sceau, il y eut un silence dans le ciel d'environ une demi-heure.
\VS{2}Et je vis les sept anges qui se tiennent devant Dieu, et sept trompettes leur furent données.
\VS{3}Et un autre ange vint et se tint devant l'autel, ayant un encensoir d'or, et plusieurs parfums lui furent donnés pour les offrir, avec les prières de tous les saints, sur l'autel d'or qui est devant le trône.
\VS{4}Et la fumée des parfums monta avec les prières des saints de la main de l'ange devant Dieu.
\VS{5}Puis l'ange prit l'encensoir, et l'ayant rempli du feu de l'autel, il le jeta sur la terre~; et il y eut des tonnerres, des voix, des éclairs, et un tremblement de terre.
\VS{6}Alors les sept anges qui avaient les sept trompettes se préparèrent à en sonner.
\TextTitle{Première trompette~: Grêle et feu mêlés de sang}
\VS{7}Et le premier ange sonna de la trompette. Et il y eut de la grêle et du feu mêlés de sang, qui furent jetés sur la terre~; et le tiers des arbres fut brûlé, et toute herbe verte aussi fut brûlée.
\TextTitle{Deuxième trompette~: La montagne embrasée}
\VS{8}Et le second ange sonna de la trompette, et je vis comme une grande montagne embrasée de feu, qui fut jetée dans la mer~; et le tiers de la mer devint du sang,
\VS{9}et le tiers des créatures vivantes qui étaient dans la mer mourut, et le tiers des navires périt.
\TextTitle{Troisième trompette~: Absinthe, l'étoile tombée du ciel}
\VS{10}Et le troisième ange sonna de la trompette, et il tomba du ciel une grande étoile ardente comme un flambeau, et elle tomba sur le tiers des fleuves et sur les sources des eaux.
\VS{11}Le nom de l'étoile est Absinthe~; et le tiers des eaux fut changé en absinthe, et beaucoup d'hommes moururent par les eaux, parce qu'elles étaient devenues amères.
\TextTitle{Quatrième trompette~: Des signes dans le ciel}
\VS{12}Puis le quatrième ange sonna de la trompette, et le tiers du soleil fut frappé, ainsi que le tiers de la lune, et le tiers des étoiles, afin que le tiers en soit obscurci~; le jour fut privé d'un tiers de sa clarté, et la nuit de même.
\VS{13}Je regardai, et j'entendis un ange qui volait au milieu du ciel et qui disait à haute voix~: Malheur~! Malheur~! Malheur aux habitants de la terre à cause des autres sons de trompettes que les trois autres anges vont faire retentir.
\Chap{9}
\TextTitle{Cinquième trompette~: Ouverture du puits de l'abîme}
\VerseOne{}Le cinquième ange sonna de la trompette, et je vis une étoile qui tomba du ciel sur la terre, et la clef du puits de l'abîme fut donnée à cet ange.
\VS{2}Et il ouvrit le puits de l'abîme, et une fumée monta du puits comme la fumée d'une grande fournaise~; et le soleil et l'air furent obscurcis par la fumée du puits.
\VS{3}Des sauterelles sortirent de la fumée du puits et se répandirent sur la terre, et il leur fut donné un pouvoir comme le pouvoir qu'ont les scorpions de la terre.
\VS{4}Et il leur fut dit de ne pas faire de mal à l'herbe de la terre, ni à aucune verdure, ni à aucun arbre, mais seulement aux hommes qui n'avaient pas la marque de Dieu sur leurs fronts.
\VS{5}Et il leur fut donné, non de les tuer, mais de les tourmenter pendant cinq mois~; et le tourment qu'elles causaient était comme le tourment que cause le scorpion quand il pique un homme.
\VS{6}Et en ces jours-là, les hommes chercheront la mort, mais ils ne la trouveront pas~; et ils désireront mourir, mais la mort fuira loin d'eux.
\VS{7}Ces sauterelles ressemblaient à des chevaux préparés pour la guerre, et sur leurs têtes il y avait comme des couronnes semblables à de l'or, et leurs faces étaient comme des faces d'hommes.
\VS{8}Elles avaient les cheveux comme des cheveux de femmes~; et leurs dents étaient comme des dents de lions.
\VS{9}Elles avaient des cuirasses comme des cuirasses de fer~; et le bruit de leurs ailes était comme le bruit des chars à plusieurs chevaux qui courent à la guerre.
\VS{10}Elles avaient des queues armées d'aiguillons, comme les scorpions, et c'est dans leurs queues qu'était le pouvoir de faire du mal aux hommes pendant cinq mois.
\VS{11}Elles avaient sur elles comme roi l'ange de l'abîme, dont le nom en hébreu est Abaddon, mais en grec son nom est Apollyon\FTNT{Abaddon ou Apollyon~: Le nom de ce démon signifie «~Le destructeur~».}.
\VS{12}Le premier malheur est passé, et voici venir encore deux malheurs après celui-ci.
\TextTitle{Sixième trompette~: Les quatre anges de l'Euphrate déliés\FTNTT{Ap. 16:12.}}
\VS{13}Alors le sixième ange sonna de sa trompette, et j'entendis une voix sortant des quatre cornes de l'autel d'or qui est devant Dieu,
\VS{14}et disant au sixième ange qui avait la trompette~: Délie les quatre anges qui sont liés sur le grand fleuve, l'Euphrate.
\VS{15}On délia donc les quatre anges qui étaient prêts pour l'heure, le jour, le mois et l'année, afin de tuer le tiers des hommes.
\VS{16}Le nombre des cavaliers de l'armée était de deux cents millions, car j'en entendis le nombre.
\VS{17}Et je vis aussi dans la vision les chevaux et ceux qui étaient montés dessus, ayant des cuirasses de feu, d'hyacinthe et de soufre~; et les têtes des chevaux étaient comme des têtes de lions~; et de leurs bouches sortaient du feu, de la fumée et du soufre.
\VS{18}Le tiers des hommes fut tué par ces trois fléaux, par le feu, et par la fumée et par le soufre qui sortaient de leur bouche.
\VS{19}Car le pouvoir des chevaux était dans leurs bouches et dans leurs queues~; et leurs queues étaient semblables à des serpents ayant des têtes, et c'est avec elles qu'ils faisaient du mal.
\VS{20}Mais les autres hommes qui ne furent pas tués par ces fléaux, ne se repentirent pas des œuvres de leurs mains, ils ne cessèrent pas d'adorer les démons, les idoles d'or, d'argent, de cuivre, de pierre, et de bois, qui ne peuvent ni voir, ni entendre, ni marcher.
\VS{21}Et ils ne se repentirent pas aussi de leurs meurtres, ni de leurs enchantements, ni de leur impudicité, ni de leurs vols.
\Chap{10}
\TextTitle{Un ange puissant descend du ciel}
\VerseOne{}Je vis un autre ange puissant qui descendait du ciel, environné d'une nuée, au-dessus de sa tête était l'arc-en-ciel, son visage était comme le soleil, et ses pieds comme des colonnes de feu.
\VS{2}Et il avait dans sa main un petit livre ouvert, et il posa son pied droit sur la mer, et le pied gauche sur la terre~;
\VS{3}et il cria d'une voix forte, comme lorsqu'un lion rugit. Et quand il eut crié, les sept tonnerres firent entendre leurs voix.
\VS{4}Et après que les sept tonnerres eurent fait entendre leurs voix, j'allais écrire, mais j'entendis une voix du ciel qui me disait~: Scelle les choses que les sept tonnerres ont fait entendre, et ne les écris pas.
\VS{5}Et l'ange que j'avais vu se tenant sur la mer et sur la terre, leva sa main vers le ciel,
\VS{6}et jura par celui qui est vivant aux siècles des siècles, qui a créé le ciel avec les choses qui y sont, et la terre avec les choses qui y sont, et la mer avec les choses qui y sont, qu'il n'y aurait plus de temps~;
\VS{7}mais qu'aux jours de la voix du septième ange, quand il commencera à sonner de la trompette, le mystère de Dieu sera accompli, comme il l'a déclaré à ses serviteurs les prophètes.
\TextTitle{Nouvelle mission de Jean}
\VS{8}Et la voix que j'avais entendue du ciel me parla encore et me dit~: Va, et prends le petit livre ouvert qui est dans la main de l'ange qui se tient sur la mer et sur la terre.
\VS{9}Et j'allai vers l'ange, en lui disant~: Donne-moi le petit livre~; et il me dit~: Prends-le et mange-le~; il remplira tes entrailles d'amertume, mais il sera doux dans ta bouche comme du miel\FTNT{Ez. 3:1-3.}.
\VS{10}Je pris donc le petit livre de la main de l'ange, et je le mangeai~; il fut doux dans ma bouche comme du miel, mais quand je l'eus mangé, mes entrailles furent remplies d'amertume.
\VS{11}Alors il me dit~: Il faut que tu prophétises de nouveau sur beaucoup de peuples, et sur plusieurs nations, sur plusieurs langues et plusieurs rois.
\Chap{11}
\TextTitle{Le temps des nations}
\VerseOne{}On me donna un roseau semblable à une verge, et l'ange se présenta et me dit~: Lève-toi et mesure le temple de Dieu et l'autel, et ceux qui y adorent.
\VS{2}Mais laisse de côté le parvis extérieur du temple, et ne le mesure pas~; car il est donné aux Gentils, et ils fouleront aux pieds la ville sainte pendant quarante-deux mois\FTNT{C'est le temps que durera la grande tribulation, soit trois ans et demi. Daniel parle d'une semaine, un jour comptant pour une année (Da. 9:27). La grande tribulation débutera à la moitié de cette semaine, ce qui correspond bien à quarante-deux mois (Ap. 13:5) et à mille deux cent soixante jours (Ap. 11:3~; Ap. 12:6).}.
\TextTitle{Les deux témoins ressuscitent}
\VS{3}Mais je donnerai à mes deux témoins de prophétiser pendant mille deux cent soixante jours, revêtus de sacs.
\VS{4}Ce sont les deux oliviers\FTNT{Za. 4:14.} et les deux chandeliers qui se tiennent devant le Dieu de la terre. 
\VS{5}Et si quelqu'un veut leur faire du mal, du feu sort de leurs bouches et dévore leurs ennemis~; car si quelqu'un veut leur faire du mal, il faut qu'il soit tué de cette manière.
\VS{6}Ils ont le pouvoir de fermer le ciel, afin qu'il ne pleuve pas pendant les jours de leur prophétie~; ils ont aussi le pouvoir de changer les eaux en sang, et de frapper la terre de toutes sortes de plaies, toutes les fois qu'ils le voudront.
\VS{7}Et quand ils auront achevé de rendre leur témoignage, la bête qui monte de l'abîme\FTNT{L'homme impie, l'Antichrist, ou encore le fils de la perdition dont il est question dans 2 Th. 2:3~; 2 Th. 2:8-9.} leur fera la guerre, les vaincra, et les tuera.
\VS{8}Et leurs cadavres seront étendus sur les places de la grande ville, qui est appelée spirituellement Sodome et Egypte, où aussi notre Seigneur a été crucifié.
\VS{9}Et ceux des tribus, des peuples, des langues, et des nations verront leurs cadavres pendant trois jours et demi, et ils ne permettront pas que leurs cadavres soient mis dans des sépulcres.
\VS{10}Et les habitants de la terre se réjouiront, ils seront dans l'allégresse, ils s'enverront des présents les uns aux autres, parce que ces deux prophètes ont tourmenté les habitants de la terre.
\VS{11}Mais après ces trois jours et demi, l'Esprit de vie venant de Dieu entra en eux, et ils se tinrent sur leurs pieds, et une grande crainte saisit ceux qui les virent.
\VS{12}Après cela, ils entendirent une forte voix du ciel, leur disant~: Montez ici~! Et ils montèrent au ciel sur une nuée, et leurs ennemis les virent.
\VS{13}Et à cette même heure-là, il eut un grand tremblement de terre, et la dixième partie de la ville tomba, et sept mille hommes furent tués par ce tremblement de terre~; et les autres furent épouvantés et donnèrent gloire au Dieu du ciel.
\VS{14}Le second malheur est passé. Voici, le troisième malheur vient bientôt.
\TextTitle{Septième trompette~: Le règne du Messie annoncé, cantique des vingt-quatre vieillards\FTNTT{Ap. 8:2.}}
\VS{15}Le septième ange sonna de la trompette, et il se fit entendre au ciel de grandes voix qui disaient~: Les royaumes du monde sont soumis à notre Seigneur et à son Christ, et il régnera aux siècles des siècles.
\VS{16}Alors les vingt-quatre anciens qui étaient assis devant Dieu sur leurs trônes, se prosternèrent sur leurs faces et adorèrent Dieu,
\VS{17}en disant~: Nous te rendons grâces, Seigneur Dieu Tout-Puissant, QUI ES, QUI ETAIS, et QUI VIENS, de ce que tu as fait éclater ta grande puissance, et de ce que tu as agi en Roi.
\VS{18}Les nations se sont irritées, mais ta colère est venue, et le temps est venu de juger les morts, et de donner la récompense à tes serviteurs les prophètes et aux saints, et à ceux qui craignent ton Nom, petits et grands, et de détruire ceux qui corrompent la terre.
\VS{19}Et le temple de Dieu fut ouvert dans le ciel, et l'arche de son alliance apparut dans son temple. Et il y eut des éclairs, des voix, des tonnerres, un tremblement de terre, et une grosse grêle.
\Chap{12}
\TextTitle{Vision de la femme et du dragon}
\VerseOne{}Et un grand signe parut dans le ciel~: Une femme revêtue du soleil, la lune sous ses pieds, et sur sa tête une couronne de douze étoiles\FTNT{En Ge. 37:9-10, Joseph raconte à ses parents et à ses frères un songe particulier où il voyait le soleil, la lune et onze étoiles se prosterner devant lui. Jacob comprit que les onze étoiles représentaient ses enfants, la lune sa femme Rachel, qui était la mère de Joseph, et que le soleil c'était lui-même. Il est donc question ici d'Israël, qui a toujours été identifié à une femme (Ez. 16) de qui est issu le Messie selon la chair (Ro. 9:5).}.
\VS{2}Elle était enceinte, et elle criait, étant en travail d'enfant, souffrant les grandes douleurs de l'enfantement.
\VS{3}Il parut aussi un autre signe dans le ciel, et voici un grand dragon rouge feu ayant sept têtes et dix cornes, et sur ses têtes sept diadèmes.
\VS{4}Sa queue entraînait le tiers des étoiles du ciel et les jeta sur la terre\FTNT{Da. 8:10.}. Puis le dragon s'arrêta devant la femme qui devait accoucher, afin de dévorer son enfant\FTNT{Cet enfant est évidemment Jésus-Christ (Mt. 2:16.)}, dès qu'elle l'aurait mis au monde.
\TextTitle{La naissance du Messie}
\VS{5}Et elle accoucha d'un fils, qui doit gouverner toutes les nations avec un sceptre de fer\FTNT{Ps. 2:8-9.}. Et son enfant fut enlevé vers Dieu et vers son trône\FTNT{Lu. 24:51~; Ac. 1:9-11.}.
\VS{6}Et la femme s'enfuit dans un désert, où elle avait un lieu préparé par Dieu, afin d'y être nourrie pendant mille deux cent soixante jours.
\TextTitle{Guerre entre l'archange Michel et le dragon}
\VS{7}Et il y eut une guerre dans le ciel. Michel et ses anges combattirent contre le dragon. Et le dragon et ses anges combattirent contre Michel,
\VS{8}mais ils ne furent pas les plus forts, et leur place ne fut plus trouvée dans le ciel.
\VS{9}Et il fut précipité le grand dragon, le serpent ancien, appelé le diable et Satan, celui qui séduit toute la terre, il fut précipité sur la terre, et ses anges furent précipités avec lui\FTNT{Es. 14:12-15~; Ez. 28~; Lu. 10:18.}.
\VS{10}Et j'entendis une voix forte dans le ciel qui disait~: Maintenant le salut est arrivé, ainsi que la force, le règne de notre Dieu, et la puissance de son Christ~; car l'accusateur de nos frères, qui les accusait devant notre Dieu jour et nuit, a été précipité.
\VS{11}Et ils l'ont vaincu à cause du sang de l'Agneau, et à cause de la parole de leur témoignage, et ils n'ont pas aimé leurs vies, mais les ont exposées à la mort.
\VS{12}C'est pourquoi réjouissez-vous cieux, et vous qui y habitez. Mais malheur à vous habitants de la terre et de la mer~! Car le diable est descendu vers vous animé d'une grande fureur, sachant qu'il a peu de temps.
\TextTitle{Le dragon persécute la femme, sa postérité et les témoins du Messie}
\VS{13}Quand le dragon vit qu'il avait été précipité sur la terre, il persécuta la femme qui avait enfanté le fils.
\VS{14}Mais deux ailes d'un grand aigle furent données à la femme, afin qu'elle s'envole de devant le serpent au désert, où elle est nourrie un temps, des temps, et la moitié d'un temps.
\VS{15}Et de sa gueule, le serpent lança de l'eau comme un fleuve derrière la femme, afin de l'entraîner par le fleuve.
\VS{16}Mais la terre secourut la femme, elle ouvrit sa bouche, et elle engloutit le fleuve que le dragon avait lancé de sa gueule.
\VS{17}Alors le dragon fut irrité contre la femme, et s'en alla faire la guerre contre les autres qui sont de la semence de la femme, qui gardent les commandements de Dieu, et qui ont le témoignage de Jésus-Christ.
\VS{18}Et je me tins sur le sable qui borde la mer.
\Chap{13}
\TextTitle{La bête qui monte de la mer, l'antichrist}
\VerseOne{}Et je vis monter de la mer une bête\FTNT{Cette bête représente deux entités. Tout d'abord l'homme impie, l'Antichrist, et ensuite un système politique. Les dix cornes sur sa tête symbolisent les dix nations les plus puissantes de la terre avec lesquelles il imposera sa dictature mondiale (Da. 7:16-25). L'alliage des quatre métaux dans la statue de Nébucadnetsar en Da. 2 et la vision des quatre animaux en Da. 7, annoncent l'instauration d'un quatrième empire ou encore le système politique à la tête duquel sera la bête.} qui avait sept têtes et dix cornes, et sur ses cornes dix diadèmes, et sur ses têtes des noms de blasphème\FTNT{Voir annexe «~La bête d'apocalypse~».}.
\VS{2}Et la bête que je vis était semblable à un léopard, ses pieds étaient comme ceux d'un ours~; sa gueule était comme la gueule d'un lion\FTNT{Da. 7:7.}. Et le dragon lui donna sa puissance, son trône, et une grande autorité.
\VS{3}Et je vis l'une de ses têtes comme blessée à mort, mais sa blessure mortelle fut guérie. Remplie d'admiration, la terre entière suivit la bête.
\VS{4}Et ils adorèrent le dragon, parce qu'il avait donné l'autorité à la bête, et ils adorèrent aussi la bête, en disant~: Qui est semblable à la bête, et qui peut combattre contre elle~?
\VS{5}Et il lui fut donné une bouche qui proférait des discours pleins d'orgueil, et des blasphèmes~; et il lui fut aussi donné le pouvoir d'agir pendant quarante-deux mois.
\VS{6}Elle ouvrit sa bouche pour blasphémer contre Dieu, pour blasphémer son Nom et son tabernacle, et ceux qui habitent dans le ciel.
\VS{7}Et il lui fut donné de faire la guerre aux saints et de les vaincre. Il lui fut aussi donné autorité sur toute tribu, toute langue et toute nation.
\VS{8}Et tous les habitants de la terre l'adoreront, ceux dont les noms n'ont pas été écrits dans le livre de vie de l'Agneau immolé dès la fondation du monde.
\VS{9}Si quelqu'un a des oreilles qu'il entende.
\VS{10}Si quelqu'un est destiné à la captivité, il ira en captivité~; si quelqu'un tue avec l'épée, il faut qu'il soit lui-même tué avec l'épée. C'est ici la persévérance et la foi des saints.
\TextTitle{La bête qui monte de la terre, le faux prophète}
\VS{11}Puis je vis une autre bête qui montait de la terre\FTNT{Cette bête est identifiée au faux-prophète car son rôle consiste à amener les habitants de la terre à adorer la première bête, tout comme les vrais prophètes invitent les gens à l'adoration du Dieu véritable (Mt. 7:15).}, et qui avait deux cornes semblables à celles de l'Agneau~; mais elle parlait comme le dragon.
\VS{12}Et elle exerçait toute l'autorité de la première bête en sa présence, et elle obligeait la terre et ses habitants à adorer la première bête, dont la blessure mortelle avait été guérie\FTNT{Cette bête a existé par le passé sous la forme de l'empire romain qui s'est écroulé le 4 septembre 476. Ce régime a marqué l'histoire par son caractère universel et brutal. Le fait que cette bête blessée à mort reprenne vie, annonce l'instauration d'un empire universel qui aura les caractéristiques combinées de l'empire babylonien, médo-perse, gréco-macédonien et romain, ceux-ci correspondant aux quatre animaux de la vision de Da. 7:1-8~: Le lion, l'ours, le léopard et le quatrième animal.}.
\VS{13}Elle opérait de grands prodiges, même jusqu'à faire descendre le feu du ciel sur la terre devant les hommes.
\VS{14}Et elle séduisait les habitants de la terre, à cause des prodiges qu'il lui était donné d'opérer en présence de la bête, disant aux habitants de la terre de faire une image\FTNT{Dieu interdit la vénération des images (Ex. 20:4-5) La particularité de l'image de la bête est qu'elle possède un esprit (démon).} de la bête qui avait reçu le coup mortel de l'épée, et qui était bien vivante.
\VS{15}Et il lui fut donné de mettre un esprit à l'image de la bête, afin que même l'image de la bête parle, et qu'elle fasse que tous ceux qui n'adoreraient pas l'image de la bête soient mis à mort.
\VS{16}Elle fit que tous, petits et grands, riches et pauvres, libres et esclaves, reçoivent une marque sur leur main droite, ou sur leur front\FTNT{Il s'agit d'une marque qui est avant tout spirituelle. Car de la même façon que nous sommes scellés et marqués par l'Esprit de Dieu qui produit en nous la sainteté (Ga. 5:22~; Ro. 6:20-22~; Ep. 1:13~; Ep. 4:30) Satan marque les siens par le péché (1 Ti. 4:1-2~; 2 Ti. 3:1-5).}~;
\VS{17}et que personne ne puisse acheter ni vendre, sans avoir la marque ou le nom de la bête, ou le nombre de son nom.
\VS{18}Ici est la sagesse~: Que celui qui a de l'intelligence compte le nombre de la bête, car c'est un nombre d'homme, et son nombre est six cent soixante-six.
\Chap{14}
\TextTitle{L'Agneau et les 144 000}
\VerseOne{}Puis je regardai, et voici, l'Agneau se tenait sur la montagne de Sion, et il y avait avec lui cent quarante-quatre mille personnes qui avaient le Nom de son Père écrit sur leurs fronts.
\VS{2}Et j'entendis une voix du ciel comme le bruit des grandes eaux, et comme le bruit d'un grand tonnerre~; et j'entendis une voix de joueurs de harpe jouant de leurs harpes.
\VS{3}Et ils chantaient comme un cantique nouveau devant le trône, et devant les quatre animaux, et devant les anciens. Et personne ne pouvait apprendre le cantique, si ce n'est les cent quarante-quatre mille qui avaient été rachetés de la terre.
\VS{4}Ce sont ceux qui ne se sont pas souillés avec les femmes, car ils sont vierges~; ce sont ceux qui suivent l'Agneau partout où il va. Ils ont été rachetés d'entre les hommes pour être des prémices pour Dieu et pour l'Agneau.
\VS{5}Et dans leur bouche il ne s'est pas trouvé de fraude, car ils sont sans tache devant le trône de Dieu\FTNT{Ps. 32:2.}.
\TextTitle{l'Evangile éternel et la chute de Babylone}
\VS{6}Puis je vis un autre ange qui volait au milieu du ciel, il avait l'Evangile éternel pour évangéliser les habitants de la terre, de toute nation, de toute tribu, de toute langue et de tout peuple.
\VS{7}Il disait d'une voix forte~: Craignez Dieu, et donnez-lui gloire, car l'heure de son jugement est venue~; et adorez celui qui a fait le ciel et la terre, la mer et les sources des eaux.
\VS{8}Et un autre ange le suivit, disant~: Elle est tombée, elle est tombée Babylone, la grande ville, parce qu'elle a abreuvé toutes les nations du vin de la fureur de son impudicité~!
\TextTitle{Le jugement des adorateurs de la bête}
\VS{9}Et un troisième ange les suivit, disant d'une voix forte~: Si quelqu'un adore la bête et son image, et reçoit la marque sur son front ou sur sa main,
\VS{10}il boira, lui aussi, du vin de la colère de Dieu, du vin pur versé dans la coupe de sa colère, et il sera tourmenté dans le feu et le soufre devant les saints anges et devant l'Agneau.
\VS{11}Et la fumée de leur tourment montera aux siècles des siècles, et ils n'auront de repos ni jour ni nuit, ceux qui adorent la bête et son image, et quiconque reçoit la marque de son nom.
\VS{12}Ici est la persévérance des saints~; ici sont ceux qui gardent les commandements de Dieu, et la foi de Jésus.
\TextTitle{Bénédiction de ceux qui meurent en Christ}
\VS{13}Alors j'entendis une voix du ciel qui me disait~: Ecris~: Bénis sont dès à présent les morts qui meurent dans le Seigneur~! Oui, c'est vrai~! dit l'Esprit, afin qu'ils se reposent de leurs travaux, car leurs œuvres les suivent.
\TextTitle{Prophétie sur Harmaguédon}
\VS{14}Et je regardai, et voici, il y avait une nuée blanche, et sur la nuée était assis quelqu'un qui ressemblait à un homme\FTNT{Ez. 1:26~; Da. 7:13~; Mt. 24:30~; Mt. 26:64~; Ap. 1:13.}, ayant sur sa tête une couronne d'or, et dans sa main une faucille tranchante.
\VS{15}Et un autre ange sortit du temple, criant à haute voix à celui qui était assis sur la nuée~: Jette ta faucille, et moissonne~; car c'est ton heure de moissonner, parce que la moisson de la terre est mûre\FTNT{Jé. 51:33~; Mt. 13:30-39.}.
\VS{16}Alors celui qui était assis sur la nuée jeta sa faucille sur la terre, et la terre fut moissonnée.
\VS{17}Et un autre ange sortit du temple qui est dans le ciel, ayant lui aussi une faucille tranchante.
\VS{18}Et un autre ange, qui avait autorité sur le feu, sortit de l'autel, et s'adressant d'une voix forte à celui qui avait la faucille tranchante, dit~: Jette ta faucille tranchante, et vendange les grappes de la vigne de la terre, car ses raisins sont mûrs.
\VS{19}Et l'ange jeta sa faucille tranchante sur la terre et vendangea la vigne de la terre, et il jeta la vendange dans la grande cuve de la colère de Dieu.
\VS{20}Et la cuve fut foulée hors de la ville~; et du sang sortit de la cuve, jusqu'aux mors des chevaux, sur une étendue de mille six cents stades\FTNT{Es. 63:1-6.}.
\Chap{15}
\TextTitle{Une scène glorieuse au ciel}
\VerseOne{}Puis je vis dans le ciel un autre signe, grand et admirable~: Sept anges qui tenaient les sept derniers fléaux, car c'est par eux que s'accomplit la colère de Dieu.
\VS{2}Et je vis aussi comme une mer de verre mêlée de feu, et ceux qui avaient vaincu la bête et son image, et sa marque, et le nombre de son nom, étaient debout sur la mer qui était comme de verre, et ayant les harpes de Dieu.
\VS{3}Ils chantaient le cantique de Moïse, serviteur de Dieu, et le cantique de l'Agneau, en disant~: Tes œuvres sont grandes et merveilleuses, ô Seigneur Dieu Tout-Puissant~! Tes voies sont justes et véritables, ô Roi des saints~!
\VS{4}Seigneur, qui ne te craindrait, et qui ne glorifierait ton Nom~? Car toi seul tu es Saint, c'est pourquoi toutes les nations viendront et se prosterneront devant toi~; car tes jugements sont pleinement manifestés.
\VS{5}Et après ces choses, je regardai, et voici le temple du tabernacle du témoignage fut ouvert dans le ciel.
\VS{6}Et les sept anges qui avaient les sept fléaux sortirent du temple, revêtus d'un lin pur et blanc, et ayant des ceintures d'or autour de leurs poitrines.
\VS{7}Et l'un des quatre animaux donna aux sept anges sept coupes d'or, pleines de la colère du Dieu qui vit aux siècles des siècles.
\VS{8}Et le temple fut rempli de la fumée à cause de la gloire de Dieu et de sa puissance~; et personne ne pouvait entrer dans le temple jusqu'à ce que les sept fléaux des sept anges soient accomplis.
\Chap{16}
\TextTitle{Première coupe~: Les ulcères}
\VerseOne{}Et j'entendis du temple une voix éclatante qui disait aux sept anges~: Allez, et versez sur la terre les coupes de la colère de Dieu.
\VS{2}Et le premier ange s'en alla, et versa sa coupe sur la terre. Et un ulcère malin et dangereux frappa les hommes qui avaient la marque de la bête, et ceux qui adoraient son image.
\TextTitle{Deuxième coupe~: La mer changée en sang}
\VS{3}Et le second ange versa sa coupe sur la mer, et elle devint comme le sang d'un corps mort, et tout être qui vivait dans la mer mourut.
\TextTitle{Troisième coupe~: Les sources changées en sang}
\VS{4}Et le troisième ange versa sa coupe sur les fleuves et sur les sources des eaux, et elles devinrent du sang.
\VS{5}Et j'entendis l'ange des eaux qui disait~: Seigneur, QUI ES, QUI ETAIS, et QUI VIENS, tu es juste, parce que tu as exercé ce jugement.
\VS{6}Parce qu'ils ont répandu le sang des saints et des prophètes, tu leur as aussi donné du sang à boire, car ils le méritent.
\VS{7}Et j'entendis un autre de l'autel, qui disait~: Certainement, Seigneur Dieu Tout-Puissant, tes jugements sont véritables et justes.
\TextTitle{Quatrième coupe~: Une chaleur extrême}
\VS{8}Ensuite, le quatrième ange versa sa coupe sur le soleil, et le pouvoir lui fut donné de brûler les hommes par le feu,
\VS{9}de sorte que les hommes furent brûlés par de grandes chaleurs, et ils blasphémèrent le Nom de Dieu qui a puissance sur ces fléaux~; et ils ne se repentirent pas pour lui donner gloire.
\TextTitle{Cinquième coupe~: Les ténèbres sur le trône de la bête}
\VS{10}Après cela, le cinquième ange versa sa coupe sur le trône de la bête. Et son royaume fut couvert de ténèbres, et les hommes se mordaient la langue à cause de la douleur qu'ils ressentaient.
\VS{11}Et ils blasphémèrent le Dieu du ciel à cause de leurs douleurs et de leurs ulcères~; et ils ne se repentirent pas de leurs œuvres.
\TextTitle{Sixième coupe~: L'Euphrate asséché}
\VS{12}Puis le sixième ange versa sa coupe sur le grand fleuve, l'Euphrate. Et son eau tarit, afin de préparer la voie des rois venant du côté où le soleil se lève.
\VS{13}Et je vis sortir de la gueule du dragon, et de la gueule de la bête, et de la bouche du faux prophète, trois esprits impurs semblables à des grenouilles.
\VS{14}Car ce sont des esprits de démons, qui font des prodiges, et qui vont vers les rois de la terre et du monde entier, afin de les assembler pour le combat de ce grand jour du Dieu Tout-Puissant.
\VS{15}Voici, je viens comme un voleur. Béni est celui qui veille et qui garde ses vêtements, afin de ne pas marcher nu, et qu'on ne voie pas sa honte~!
\VS{16}Et ils les assemblèrent dans le lieu qui est appelé en hébreu Harmaguédon\FTNT{Le terme «~Harmaguédon~», mentionné uniquement dans ce passage, vient du mot hébreu «~Har-Magidown~», ce qui signifie «~Montagne de Megiddo~». Bien qu'il n'existe pas de montagne portant spécifiquement ce nom, l'emplacement probable de cet endroit est la plaine de Meggido se trouvant à proximité de Jérusalem. Par le passé, elle fut le théâtre de la victoire de Barak sur les Cananéens (Jg. 4:15) et de celle de Gédéon sur les Madianites (Jg. 7). C'est aussi à cet endroit que Saül et ses fils (1 Sa. 31~:8) ainsi que le roi Josias (2 R. 23:29-30~; 2 Ch. 35:22) trouvèrent la mort. Pour toutes ces raisons, elle devint au fil du temps le symbole de l'affrontement entre Dieu et la puissance des ténèbres. Selon les prophéties bibliques, la plaine de Meggido et la vallée de Jizréel constitueront le site de l'ultime guerre mondiale, celle opposant l'Antichrist et ses alliés (dirigeants des nations) contre Israël. Le Seigneur interviendra alors ouvertement dans les affaires humaines pour déverser la coupe de sa colère (Ap. 16:1) et anéantir l'homme impie et toute son armée (Ez. 38-39~; Joë. 3~; Mi. 4:11~; So. 1~; Za. 14~; Mt. 24:29-30~; Ap. 20:1-3~; Ap. 20:7-10).}.
\TextTitle{Septième coupe~: Une grosse grêle tombe du ciel}
\VS{17}Puis le septième ange versa sa coupe dans l'air~; et il sortit du temple du ciel une voix forte qui venait du trône, disant~: C'en est fait.
\VS{18}Et il y eut des éclairs, et des voix, et des tonnerres, et il se fit un grand tremblement de terre, dis-je, tel qu'il n'y en avait jamais eu depuis que les hommes sont sur la terre.
\VS{19}La grande ville fut divisée en trois parties, et les villes des nations tombèrent, et Dieu se souvint de Babylone la grande, pour lui donner la coupe du vin de son ardente colère.
\VS{20}Toutes les îles s'enfuirent et les montagnes ne furent plus retrouvées.
\VS{21}Une grosse grêle, dont les grêlons pesaient un talent\FTNT{Un talent d'argent pesait 45 kg, un talent d'or pesait 90 kg.}, tomba du ciel sur les hommes~; et les hommes blasphémèrent Dieu, à cause du fléau de la grêle, car le fléau qu'elle causa fut très grand.
\Chap{17}
\TextTitle{La prostituée}
\VerseOne{}Puis l'un des sept anges qui tenaient les sept coupes vint, et il m'adressa la parole, en disant~: Viens, je te montrerai le jugement de la grande prostituée, qui est assise sur les grandes eaux.
\VS{2}Avec elle, les rois de la terre ont commis la fornication, et les habitants de la terre ont été enivrés du vin de sa prostitution.
\VS{3}Il me transporta en esprit dans un désert~; et je vis une femme assise sur une bête écarlate, pleine de noms de blasphème, ayant sept têtes et dix cornes.
\VS{4}Et la femme était vêtue de pourpre et d'écarlate, et parée d'or, de pierres précieuses, et de perles~; et elle tenait à la main une coupe d'or, pleine des abominations de l'impureté de sa prostitution.
\VS{5}Et il y avait sur son front un nom écrit, un mystère~: Babylone la grande, la mère des impudicités et des abominations de la terre\FTNT{Symboliquement, Babylone la grande incarne l'Eglise apostate. Elle est soutenue par la bête qu'elle chevauche, c'est-à-dire l'homme impie. Ces deux entités forment un système impie où la politique et la religion se mélangent (Da. 2:43).}.
\VS{6}Et je vis cette femme ivre du sang des saints, et du sang des martyrs de Jésus. Et quand je la vis, je fus saisi d'un grand étonnement.
\TextTitle{Alliance entre la prostituée et la bête}
\VS{7}Et l'ange me dit~: Pourquoi t'étonnes-tu~? Je te dirai le mystère de la femme et de la bête qui la porte, qui a les sept têtes et les dix cornes.
\VS{8}La bête que tu as vue, était, et elle n'est plus. Elle doit monter de l'abîme, et aller à la perdition. Et les habitants de la terre, ceux dont les noms ne sont pas écrits dans le Livre de vie dès la fondation du monde, s'étonneront en voyant la bête parce qu'elle était, et qu'elle n'est plus, et qui toutefois est.
\VS{9}C'est ici qu'il faut un esprit intelligent et qui ait de la sagesse. Les sept têtes sont sept montagnes sur lesquelles la femme est assise.
\VS{10}Ce sont aussi sept rois, les cinq sont tombés~; l'un est, et l'autre n'est pas encore venu~; et quand il sera venu, il faut qu'il demeure pour un peu de temps.
\VS{11}Et la bête qui était, et qui n'est plus, est elle-même un huitième roi, et elle est du nombre des sept, mais elle tend à sa ruine.
\VS{12}Et les dix cornes que tu as vues sont dix rois, qui n'ont pas encore commencé à régner, mais ils recevront autorité comme rois en même temps avec la bête, pour une heure.
\VS{13}Ils ont un même dessein, et ils donneront leur puissance et leur autorité à la bête.
\TextTitle{Victoire de l'Agneau sur la prostituée}
\VS{14}Ils combattront contre l'Agneau et l'Agneau les vaincra, parce qu'il est le Seigneur des seigneurs, et le Roi des rois~; et les appelés, les élus et les fidèles qui sont avec lui, les vaincront aussi.
\VS{15}Puis il me dit~: Les eaux que tu as vues, et sur lesquelles la prostituée est assise, sont des peuples, des nations et des langues.
\VS{16}Les dix cornes que tu as vues sur la bête haïront la prostituée, la rendront désolée et nue, la dépouilleront, et mangeront sa chair, et la brûleront au feu.
\VS{17}Car Dieu a mis dans leurs cœurs de faire ce qu'il lui plaît, et de former un même dessein, et de donner leur royaume à la bête, jusqu'à ce que les paroles de Dieu soient accomplies.
\VS{18}Et la femme que tu as vue, c'est la grande ville, qui règne sur les rois de la terre.
\Chap{18}
\TextTitle{Babylone détruite}
\VerseOne{}Après ces choses, je vis descendre du ciel un autre ange, qui avait une grande autorité, et la terre fut illuminée de sa gloire.
\VS{2}Il cria avec force à haute voix, et il dit~: Elle est tombée, elle est tombée Babylone la grande, et elle est devenue la demeure de démons, et la retraite de tout esprit impur, et le repaire de tout oiseau impur et exécrable.
\VS{3}Car toutes les nations ont bu du vin de sa prostitution effrénée, et les rois de la terre ont commis la fornication avec elle, et les marchands de la terre se sont enrichis par l'excès de son luxe.
\VS{4}Puis j'entendis une autre voix du ciel, qui disait~: Sortez de Babylone, mon peuple, afin que vous ne participiez pas à ses péchés, et que vous n'ayez pas de part à ses fléaux.
\VS{5}Car ses péchés sont montés jusqu'au ciel, et Dieu s'est souvenu de ses iniquités.
\VS{6}Rendez-lui selon ce qu'elle vous a fait, et payez-lui au double selon ses œuvres~; et dans la même coupe où elle vous a versé à boire versez-lui au double.
\VS{7}Autant elle s'est glorifiée et plongée dans le luxe, autant donnez-lui de tourment et de deuil~; car elle dit en son cœur~: Je siège en reine, je ne suis pas veuve, et je ne verrai pas de deuil.
\VS{8}C'est pourquoi ses plaies, qui sont la mort, le deuil, et la famine, viendront en un même jour, et elle sera entièrement brûlée au feu~; car le Seigneur Dieu qui la jugera est puissant.
\TextTitle{Conséquence de la chute de Babylone~: Gémissements des habitants de la terre}
\VS{9}Et les rois de la terre, qui ont commis la fornication avec elle, et qui ont vécu dans le luxe, la pleureront, et mèneront deuil sur elle en se frappant la poitrine, quand ils verront la fumée de son embrasement~;
\VS{10}et ils se tiendront éloignés dans la crainte de son tourment, et diront~: Malheur~! Malheur~! Babylone la grande, cette ville si puissante, comment ta condamnation est-elle venue en une seule heure~?
\VS{11}Les marchands de la terre aussi pleureront, et seront dans le deuil à cause d'elle, parce que personne n'achète plus de leurs marchandises,
\VS{12}qui sont des marchandises d'or, d'argent, de pierres précieuses, de perles, de fin lin, de pourpre, de soie, d'écarlate, de toute sorte de bois odoriférant, de toute espèce de bois de senteur, d'ivoire, et de toute espèce de vaisseaux de bois très précieux, d'airain, de fer, et de marbre,
\VS{13}du cinnamome, des parfums, des essences, de l'encens, du vin, de l'huile, de la fine fleur de farine, du blé, des bœufs, des brebis, des chevaux, des chars, des esclaves, et des âmes d'hommes.
\VS{14}Car les fruits du désir de ton âme se sont éloignés de toi, et toutes les choses délicates et excellentes sont perdues pour toi, et dorénavant tu ne les trouveras plus.
\VS{15}Les marchands, dis-je, de ces choses, qui se sont enrichis par elle, se tiendront éloignés, dans la crainte de son tourment~; ils pleureront et seront dans le deuil,
\VS{16}et diront~: Malheur~! Malheur~! La grande ville qui était vêtue de fin lin, de pourpre, d'écarlate, qui était parée d'or, ornée de pierres précieuses, et de perles, comment en une seule heure tant de richesses ont été détruites~?
\VS{17}Et tous les pilotes aussi, tous ceux qui naviguent vers ce lieu, tous les marins, et tous ceux qui exploitent la mer, se tiendront éloignés,
\VS{18}et, en voyant la fumée de son embrasement, ils s'écrieront, en disant~: Quelle ville était semblable à cette grande ville~?
\VS{19}Ils jetteront de la poussière sur leurs têtes, pleurant et menant deuil, ils crieront, en disant~: Malheur~! Malheur~! La grande ville, où se sont enrichis par son opulence tous ceux qui ont des navires sur la mer, comment a-t-elle été réduite en désert en une seule heure~?
\TextTitle{Réjouissance des anges suite à la chute de Babylone}
\VS{20}Ô ciel~! Réjouis-toi à cause d'elle~; et vous aussi saints apôtres et prophètes, réjouissez-vous~! Car Dieu l'a punie à cause de vous.
\VS{21}Alors un ange d'une grande force prit une pierre semblable à une grande meule, et la jeta dans la mer, en disant~: Ainsi sera précipitée avec impétuosité Babylone, cette grande ville, et elle ne sera plus retrouvée\FTNT{Jé. 51:63-64.}.
\VS{22}Et l'on entendra plus chez toi les sons des joueurs de harpe, des musiciens, des joueurs de flûte, et de ceux qui sonnent de la trompette~; et on ne trouvera plus chez toi aucun artisan d'un métier quelconque, on n'entendra plus chez toi le bruit de la meule,
\VS{23}et la lumière de la lampe ne brillera plus chez toi, et la voix de l'époux et de l'épouse ne sera plus entendue chez toi~; car tes marchands étaient des princes de la terre, et parce que par tes enchantements toutes les nations ont été séduites,
\VS{24}et l'on a trouvé chez elle le sang des prophètes et des saints, et de tous ceux qui ont été mis à mort sur la terre.
\Chap{19}
\TextTitle{Allégresse dans les cieux suite au jugement de la grande prostituée\FTNTT{Ap. 17:16-17~; 18:8.}}
\VerseOne{}Après cela, j'entendis dans le ciel une voix forte d'une foule nombreuse, disant~: Alléluia~! Le salut, la gloire, l'honneur et la puissance appartiennent au Seigneur, notre Dieu,
\VS{2}car ses jugements sont véritables et justes~; car il a jugé la grande prostituée qui a corrompu la terre par son impudicité, et parce qu'il a vengé le sang de ses serviteurs versé de la main de la prostituée.
\VS{3}Et ils dirent encore~: Alléluia~! Et sa fumée monte aux siècles des siècles.
\VS{4}Et les vingt-quatre anciens et les quatre animaux se prosternèrent sur leurs faces, et adorèrent Dieu, qui était assis sur le trône, en disant~: Amen~! Alléluia~!
\VS{5}Et il sortit du trône une voix qui disait~: Louez notre Dieu, vous tous ses serviteurs, et vous qui le craignez, tant les petits que les grands\FTNT{Ps. 134.}.
\VS{6}J'entendis ensuite comme la voix d'une grande assemblée, et comme le bruit de grandes eaux, et comme l'éclat de grands tonnerres, disant~: Alléluia~! Car le Seigneur notre Dieu Tout-Puissant a pris possession de son Royaume.
\TextTitle{Festin des noces de l'Agneau}
\VS{7}Réjouissons-nous et tressaillons de joie, et donnons-lui gloire, car les noces de l'Agneau sont venues, et son Epouse s'est préparée.
\VS{8}Et il lui a été donné de se revêtir d'un fin lin pur et éclatant. Car le fin lin désigne la justice des saints.
\VS{9}Alors il me dit~: Ecris~: Bénis sont ceux qui sont appelés au festin des noces de l'Agneau\FTNT{Mt. 22:1-13~; Lu. 14:15-24.}~! Il me dit aussi~: Ces paroles de Dieu sont véritables.
\VS{10}Alors je tombai à ses pieds pour l'adorer, mais il me dit~: Garde-toi de le faire~! Je suis ton compagnon de service, et celui de tes frères qui ont le témoignage de Jésus. Adore Dieu~! Car le témoignage de Jésus est l'Esprit de la prophétie.
\TextTitle{Seconde venue du Messie dans la gloire\FTNTT{Mt. 24:16-30.}}
\VS{11}Puis je vis le ciel ouvert, et voici parut un cheval blanc. Et celui qui était monté dessus s'appelle FIDELE et VERITABLE, et il juge et combat avec justice.
\VS{12}Et ses yeux étaient comme une flamme de feu~; il y avait sur sa tête plusieurs diadèmes, et il avait un nom écrit que personne ne connaît, si ce n'est lui-même.
\VS{13}Il était revêtu d'un vêtement teint de sang, et son Nom s'appelle LA PAROLE DE DIEU.
\VS{14}Les armées qui sont dans le ciel le suivaient sur des chevaux blancs, revêtues de fin lin blanc et pur.
\VS{15}De sa bouche sortait une épée tranchante\FTNT{Es. 11:4~; 2 Th. 2:8~; Hé. 4:12.}, pour frapper les nations~; il les gouvernera avec un sceptre de fer\FTNT{Ps. 2:8-9.}, et il foulera la cuve du vin de l'indignation et de la colère du Dieu Tout-Puissant.
\VS{16}Et sur son vêtement et sur sa cuisse étaient écrits ces mots~: LE ROI DES ROIS ET LE SEIGNEUR DES SEIGNEURS.
\TextTitle{Bataille d'Harmaguédon\FTNTT{Ap. 16:16.}}
\VS{17}Puis je vis un ange qui se tenait dans le soleil. Il cria d'une voix forte, et dit à tous les oiseaux qui volaient au milieu du ciel~: Venez et rassemblez-vous pour le grand festin de Dieu,
\VS{18}afin de manger la chair des rois, la chair des chefs militaires, la chair des puissants, la chair des chevaux et de ceux qui les montent, et la chair de toute sorte de personnes libres, esclaves, petits et grands.
\VS{19}Alors je vis la bête et les rois de la terre, et leurs armées rassemblées pour faire la guerre\FTNT{Guerre d' Harmaguédon~: Voir commentaire Ap. 16:16.} contre celui qui était monté sur le cheval et contre son armée.
\TextTitle{Condamnation de la bête et du faux prophète}
\VS{20}Et la bête fut prise, et avec elle le faux prophète qui avait fait devant elle les prodiges par lesquels il avait séduit ceux qui avaient pris la marque de la bête, et adoré son image. Et ils furent tous deux jetés vivants dans l'étang ardent de feu et de soufre.
\TextTitle{Condamnation des rois et des armées}
\VS{21}Et le reste fut tué par l'épée qui sortait de la bouche de celui qui était monté sur le cheval, et tous les oiseaux furent rassasiés de leur chair.
\Chap{20}
\TextTitle{Satan lié pour mille ans et règne du Messie}
\VerseOne{}Après cela, je vis descendre du ciel un ange, qui avait la clef de l'abîme et une grande chaîne dans sa main.
\VS{2}Il saisit le dragon, le serpent ancien, qui est le diable et Satan, et le lia pour mille ans.
\VS{3}Il le jeta dans l'abîme, et il l'enferma et mit le sceau sur lui, afin qu'il ne séduise plus les nations, jusqu'à ce que les mille ans soient accomplis. Après quoi, il faut qu'il soit délié pour un peu de temps.
\TextTitle{Dernière phase de la première résurrection}
\VS{4}Je vis des trônes, sur lesquels des gens s'assirent, à qui l'autorité de juger fut donnée\FTNT{1 Co. 6:2.}. Et je vis les âmes de ceux qui avaient été décapités pour le témoignage de Jésus, et pour la parole de Dieu, et de ceux qui n'avaient pas adoré la bête ni son image, et qui n'avaient pas pris sa marque sur leurs fronts, ou sur leurs mains. Et ils vécurent\FTNT{Jn. 14:19.} et régnèrent avec Christ mille ans.
\VS{5}Les autres morts ne revinrent pas à la vie jusqu'à ce que les mille ans soient accomplis. C'est la première résurrection.
\VS{6}Bénis et saints sont ceux qui ont part à la première résurrection~! La seconde mort n'a pas de puissance sur eux, mais ils seront prêtres de Dieu, et de Christ, et ils régneront avec lui mille ans.
\TextTitle{Satan délié~; sa chute finale}
\VS{7}Et quand les mille ans seront accomplis, Satan sera délié de sa prison.
\VS{8}Et il sortira pour séduire les nations qui sont aux quatre coins de la terre, Gog et Magog, afin de les rassembler pour la guerre, et leur nombre est comme le sable de la mer.
\VS{9}Ils montèrent et se répandirent à la surface de la terre, et ils environnèrent le camp des saints, et la ville bien-aimée. Mais Dieu fit descendre un feu du ciel qui les dévora.
\TextTitle{Satan jeté dans l'étang de feu}
\VS{10}Et le diable qui les séduisait fut jeté dans l'étang de feu et de soufre, où sont la bête et le faux prophète. Et ils seront tourmentés jour et nuit, aux siècles des siècles.
\TextTitle{Résurrection des impies et jugement dernier~; l'Hadès (ou enfer) et la mort jetés dans l'étang de feu}
\VS{11}Puis je vis un grand trône blanc, et celui qui était assis dessus. La terre et le ciel s'enfuirent devant sa face, et il ne fut plus trouvé de place pour eux.
\VS{12}Et je vis les morts, les grands et les petits, qui se tenaient devant Dieu. Des livres furent ouverts. Et un autre livre fut ouvert, celui qui est le Livre de vie. Et les morts furent jugés selon les choses qui étaient écrites dans les livres, c'est-à-dire selon leurs œuvres.
\VS{13}Et la mer rendit les morts qui étaient en elle, et la mort et l'enfer\FTNT{le mot «~enfer~» vient de l'hébreu «~Hadès~». Voir commentaire dans Mt. 16:18.} rendirent les morts qui étaient en eux~; et ils furent jugés chacun selon ses œuvres.
\VS{14}Et la mort et l'enfer furent jetés dans l'étang de feu\FTNT{L'étang de feu est aussi appelé «~seconde mort~», c'est la destination finale de tous les impies, des démons et de Satan. On l'appelle «~la seconde mort~» parce qu'elle a été précédée de la mort physique. Cette mort n'est pas un anéantissement, mais une condition de souffrances éternelles. C'est la séparation définitive d'avec Dieu. A l'issue du jugement dernier, le séjour des morts (le dieu Hadès ou l'enfer) sera jeté dans le lac de feu (voir commentaire en Mt. 16:18). La Bible utilise également le mot «~géhenne~» pour décrire l'endroit où les impies passeront l'éternité. Ce terme vient de l'hébreu «~ge-hinnom~», autrement dit vallée de Ben Hinnom (littéralement «~le lieu du feu~») qui se trouve en Israël, en contrebas du mont Sion sur lequel est bâtie la ville de Jérusalem (Mt. 5:22~; Mt. 5:29-30~; Mt. 10:28~; Mt. 18:9~; Mt. 23:15~; Mt. 23:33~; Mc. 9:47~; Lu. 12:5~; Ja. 3:6). Autrefois, on y brûlait des enfants en l'honneur de Moloc, une divinité ammonite (2 R. 23:10~; Jé. 32:35), puis des immondices. Ce lieu est devenu avec le temps le symbole du péché et de l'affliction et c'est ainsi qu'il finit par désigner le lieu du châtiment éternel.}. C'est la seconde mort.
\VS{15}Et quiconque ne fut pas trouvé écrit dans le Livre de vie fut jeté dans l'étang de feu.
\Chap{21}
\TextTitle{Nouveaux cieux et une nouvelle terre~; la nouvelle Jérusalem}
\VerseOne{}Puis je vis un nouveau ciel et une nouvelle terre~; car le premier ciel et la première terre avaient disparu, et la mer n'était plus.
\VS{2}Et moi, Jean, je vis la ville sainte, la nouvelle Jérusalem, qui descendait du ciel, d'auprès de Dieu, parée comme une épouse qui s'est ornée pour son mari.
\VS{3}Et j'entendis du trône une voix forte qui disait~: Voici le tabernacle de Dieu avec les hommes~! Il habitera avec eux, et ils seront son peuple, et Dieu lui-même sera leur Dieu, et il sera avec eux.
\VS{4}Et Dieu essuiera toute larme de leurs yeux, et la mort ne sera plus~; et il n'y aura plus ni deuil, ni cri, ni douleur, car les premières choses sont passées.
\VS{5}Et celui qui était assis sur le trône dit~: Voici, je fais toutes choses nouvelles. Puis il me dit~: Ecris, car ces paroles sont véritables et certaines.
\VS{6}Il me dit aussi~: Tout est accompli. Je suis l'Alpha et l'Oméga, le commencement et la fin. A celui qui a soif, je lui donnerai de la source d'eau vive gratuitement\FTNT{Es. 55:1-2~; Mt. 10:8~; Ap. 22:17. Voir commentaire Mt. 10:8.}.
\VS{7}Celui qui vaincra héritera toutes choses~; je serai son Dieu, et il sera mon fils.
\VS{8}Mais pour les timides, les incrédules, les abominables, les meurtriers, les fornicateurs, les sorciers, les idolâtres et tous les menteurs, leur part sera dans l'étang ardent de feu et de soufre, qui est la seconde mort.
\TextTitle{L'Epouse de l'Agneau et la nouvelle Jérusalem}
\VS{9}Puis l'un des sept anges qui tenaient les sept coupes pleines des sept derniers fléaux s'approcha de moi et me parla, en disant~: Viens, et je te montrerai l'Epouse, la femme de l'Agneau.
\VS{10}Et il me transporta en esprit sur une grande et haute montagne, et il me montra la grande ville, la sainte Jérusalem, qui descendait du ciel d'auprès de Dieu,
\VS{11}ayant la gloire de Dieu. Son éclat était semblable à une pierre très précieuse, comme à une pierre de jaspe transparente comme du cristal.
\VS{12}Et elle avait une grande et haute muraille, avec douze portes, et aux portes douze anges, et des noms écrits sur elles, qui sont les noms des douze tribus des fils d'Israël\FTNT{Ez. 48:31-34.}.
\VS{13}A l'orient, trois portes, au nord, trois portes, du côté du sud, trois portes et du côté de l'occident, trois portes.
\VS{14}Et la muraille de la ville avait douze fondements, et les noms des douze apôtres de l'Agneau étaient écrits dessus\FTNT{Lu. 22:29-30~; Ep. 2:20.}.
\VS{15}Et celui qui parlait avec moi avait un roseau d'or pour mesurer la ville, ses portes et sa muraille.
\VS{16}Et la ville était bâtie en carré, et sa longueur était aussi grande que sa largeur. Il mesura donc la ville avec le roseau d'or, jusqu'à douze mille stades~; la longueur, la largeur et la hauteur étaient égales.
\VS{17}Puis il mesura la muraille qui fut de cent quarante-quatre coudées, de la mesure du personnage, c'est-à-dire de l'ange.
\VS{18}Et le bâtiment de la muraille était de jaspe, mais la ville était d'or pur, semblable à du verre fort transparent.
\VS{19}Et les fondements de la muraille de la ville étaient ornés de toutes sortes de pierres précieuses\FTNT{Es. 54:11-12.}~: Le premier fondement était de jaspe, le second de saphir, le troisième de calcédoine, le quatrième d'émeraude,
\VS{20}le cinquième de sardonyx, le sixième de sardoine, le septième de chrysolithe, le huitième de béryl, le neuvième de topaze, le dixième de chrysoprase, le onzième d'hyacinthe, le douzième d'améthyste.
\VS{21}Et les douze portes étaient douze perles~; chacune des portes était d'une seule perle. Et la place de la ville était d'or pur, comme du verre transparent.
\VS{22}Et je ne vis pas de temple dans la ville, parce que le Seigneur Dieu Tout-Puissant et l'Agneau en sont le Temple.
\VS{23}Et la ville n'a pas besoin du soleil ni de la lune pour l'éclairer, car la gloire de Dieu l'éclaire, et l'Agneau est son flambeau\FTNT{Es. 60:19.}.
\VS{24}Et les nations qui auront été sauvées marcheront à la faveur de sa lumière, et les rois de la terre y apporteront ce qu'ils ont de plus magnifique et de plus précieux.
\VS{25}Et ses portes ne se fermeront pas le jour, car il n'y aura pas de nuit\FTNT{Es. 60:11.}.
\VS{26}Et on y apportera la gloire et l'honneur des nations.
\VS{27}Il n'entrera chez elle rien de souillé, ni personne qui s'abandonne à l'abomination et au mensonge~; mais seulement ceux qui sont écrits dans le Livre de vie de l'Agneau.
\Chap{22}
\TextTitle{Règne éternel des saints avec l'Agneau}
\VerseOne{}Puis il me montra un fleuve d'eau de la vie\FTNT{Ce fleuve représente le Saint-Esprit~: Ez. 47:1-12~; Ps. 46:5~; Da. 7:9-10~; Jn. 7:38-39.}, transparent comme du cristal, qui sortait du trône de Dieu et de l'Agneau.
\VS{2}Et au milieu de la place de la ville, et des deux côtés du fleuve, était l'arbre de vie, portant douze fruits, et rendant son fruit chaque mois et les feuilles de l'arbre servaient à la guérison des nations\FTNT{Ge. 2:9~; Ge. 3:22~; Ez. 47:12.}.
\VS{3}Et il n'y aura plus d'anathème. Le trône de Dieu et de l'Agneau sera dans la ville, et ses serviteurs le serviront,
\VS{4}et ils verront sa face, et son Nom sera sur leurs fronts.
\VS{5}Et il n'y aura plus de nuit~; et ils n'auront besoin ni de lumière, ni de lampe, ni du soleil, parce que le Seigneur Dieu les éclairera, et ils régneront aux siècles des siècles.
\TextTitle{Certitude des prophéties de ce livre}
\VS{6}Puis il me dit~: Ces paroles sont certaines et véritables~; et le Seigneur, le Dieu des saints prophètes, a envoyé son ange pour manifester à ses serviteurs les choses qui doivent arriver bientôt.
\VS{7}Voici, je viens à toute vitesse\FTNT{Dans la plupart des traductions, ce passage a été traduit par «~Je viens bientôt~». Or le texte grec utilise le mot «~tachu~» qui signifie «~rapidement, à toute vitesse (sans tarder)~». Beaucoup doutent de cette promesse du Seigneur en faisant la même réflexion évoquée par Pierre~: «~Où est la promesse de son avènement~? Car depuis que les pères sont morts, toutes choses demeurent comme elles ont été dès le commencement de la création.~» (2 Pi. 3:4). Or le Seigneur ne tarde pas dans l'accomplissement de sa promesse, car il a fixé de sa propre autorité une date pour son retour, que lui seul connaît (Za. 14:7~; Mt. 24:36~; Mc. 13:32~; Ac. 1:6-7). Il sera donc fidèle à son calendrier et ne tardera pas (2 Pi. 3:9.~; Hé. 10:37).}. Béni est celui qui garde les paroles de la prophétie de ce livre~!
\VS{8}C'est moi, Jean, qui ai entendu et vu ces choses. Et après les avoir entendues et vues, je tombai à terre aux pieds de l'ange qui me les montrait pour l'adorer.
\VS{9}Mais il me dit~: Garde-toi de le faire~! Car je suis ton compagnon de service\FTNT{Hé. 1:14.} et celui de tes frères les prophètes, et de ceux qui gardent les paroles de ce livre. Adore Dieu~!
\VS{10}Il me dit aussi~: Ne scelle pas les paroles de la prophétie de ce livre. Car le temps est proche.
\VS{11}Que celui qui est injuste soit encore injuste, et que celui qui est souillé se souille encore~; et que celui qui est juste pratique encore la justice~; et que celui qui est saint se sanctifie encore~!
\VS{12}Voici, je viens à toute vitesse, et ma rétribution est avec moi\FTNT{Jésus affirme de nouveau ici sa divinité et confirme les prophéties d'Es. 35:4~; Es. 40:10~; Es. 62:11, où il est dit que Yahweh lui-même viendra avec ses rétributions.} pour rendre à chacun selon son œuvre.
\VS{13}Je suis l'Alpha et l'Oméga, le premier et le dernier, le commencement et la fin.
\VS{14}Bénis sont ceux qui lavent leurs robes afin d'avoir droit à l'arbre de vie, et d'entrer par les portes dans la ville.
\VS{15}Mais seront laissés dehors les chiens, les empoisonneurs, les fornicateurs, les meurtriers, les idolâtres et quiconque aime et pratique le mensonge.
\VS{16}Moi, Jésus, j'ai envoyé mon ange\FTNT{Cette déclaration de Jésus fait écho au verset 6 où il est dit que le Seigneur, le Dieu des esprits des prophètes, a envoyé son ange. Jésus confirme donc qu'il est Seigneur et Dieu.} pour vous confirmer ces choses dans les églises. Je suis le rejeton et la postérité de David, l'étoile brillante du matin.
\VS{17}Et l'Esprit et l'Epouse disent~: Viens~! Et que celui qui entend dise~: Viens~! Et que celui qui a soif vienne~; que celui qui veut, prenne gratuitement de l'eau de la vie.
\TextTitle{Nul ne doit y ajouter ou y retrancher}
\VS{18}Je le déclare à quiconque entend les paroles de la prophétie de ce livre~: Si quelqu'un y ajoute quelque chose, Dieu le frappera des fléaux décrits dans ce livre,
\VS{19}et si quelqu'un retranche quelque chose des paroles du livre de cette prophétie, Dieu retranchera la part qu'il a dans le livre de vie, dans la ville sainte et dans les choses qui sont écrites dans ce livre.
\VS{20}Celui qui rend témoignage de ces choses, dit~: Certainement, je viens à toute vitesse. Amen~! Oui, Seigneur Jésus, viens~!
\VS{21}Que la grâce de notre Seigneur Jésus-Christ soit avec vous tous~! Amen~!
\PPE{}
\end{multicols}

% inclusion des annexes
\addcontentsline{toc}{chapter}{Aides}\clearpage
\begin{center}Aides\end{center}\clearpage
% dictionnaire
\addcontentsline{toc}{section}{Dictionnaire}\clearpage
% mise en forme dictionnaire
\makeatletter
\def\@oddhead{{\small{\hfil\thepage\hfil}}}
\def\@evenhead{{\small{\hfil\thepage\hfil}}}\clearpage
\makeatother
\small{\parindent=0mm{\begin{center}\Large\bfseries{Dictionnaire}\end{center}}\par
\begin{multicols}{2}

\DicoEntry{AARON}\textit{, de l'hébreu «~Aharown~»~: «~haut placé~» ou «~éclairé~»}\newline
Issu de la tribu de Lévi, frère aîné de Moïse dont il fut le porte-parole. Premier grand prêtre* en Israël. Voir \vref{Ex. 4:14}~; \vref{Ex. 6:16-20} et \vref{Ex. 28}.

\DicoEntry{ABSALOM}\textit{, de l'hébreu «~'Abiyshalowm~»~: «~père de la paix~»}\newline
Fils du roi David et de Maaca, né à Hébron. Il tua Amnon son demi-frère aîné, car ce dernier avait déshonoré sa sœur Tamar. Quelques années plus tard, il conspira contre son père et se fit proclamer roi à Hébron. Il fut finalement tué par Joab, chef de l'armée de David. Voir \vref{2 S. 3:3}~; \vref{2 S. 13}~; \vref{2 S. 15-19}.

\DicoEntry{ABDIAS}\textit{, de l'hébreu «~Obadyah~»~: «~adorateur~» ou «~serviteur de Yahweh~»}\newline
Prophète de Yahweh dont le livre éponyme figure dans le Tanakh.

\DicoEntry{ABEL}\textit{, de l'hébreu «~Hebel~»~: «~souffle, vapeur~»}\newline
Deuxième fils d'Adam et Eve et première victime d'homicide de l'histoire, il fut assassiné par son frère Caïn et déclaré juste par Yahweh. Voir \vref{Ge. 4:2,8} et \vref{Mt. 23:35}.

\DicoEntry{ABIRAM}\textit{, de l'hébreu «~'Abyiram~»~: «~mon père est exalté~»}\newline
Issu de la tribu de Ruben, fils d'Eliab et frère de Dathan, il conspira avec Koré contre Moïse et Aaron. Voir \vref{No. 16:1-35}.

\DicoEntry{ABLUTION}\textit{, de l'hébreu «~rachats~»~: «~laver, baigner, nettoyer~»}\newline
Lavage de purification prescrit par la loi mosaïque et effectué avec de l'eau. Voir \vref{Ex. 29:4} et \vref{Hé. 9:10}.

\DicoEntry{ABOMINATION}\textit{, de l'hébreu «~tow'ebah~»~: «~une chose dégoûtante, abominable~» et du grec «~bdelugma~»~: «~chose folle, détestable~»}\newline
Pratique violant la loi de Yahweh et manifestant l'infidélité à Dieu comme l'idolâtrie* sous toutes ses formes, la magie ou l'homosexualité*. Voir \vref{Lé. 18:6-29}~; \vref{De. 29:17-18} et \vref{Ap. 21:27}.

\DicoEntry{ABRAM}\textit{, de l'hébreu «~Abryram~»~: «~père élevé~»}\newline
Voir ABRAHAM.

\DicoEntry{ABRAHAM}\textit{, de l'hébreu «~'Abraham~»~: «~père d'une multitude~»}\newline
Hébreu, fils de Térach, et originaire d'Ur en Chaldée. Dieu lui demanda de quitter sa terre et sa famille pour Canaan, lui promettant que sa postérité hériterait de cette terre. De sa servante Agar, lui naquit un premier fils, Ismaël, ancêtre du peuple arabe. De sa femme Sara, lui naquit Isaac qui hérita des promesses. Il mourut à cent soixante-quinze ans. Voir \vref{Ge. 12:1-7}~; \vref{Ge. 17:4-13}~; \vref{Ge. 16}~; \vref{Ge. 21:1-8} et \vref{Ge. 25:7}.

\DicoEntry{ACACIA}\textit{, de l'hébreu «~shittah~»~: «~acacia, bois d'acacia~»}\newline
Arbre épineux poussant en abondance dans la péninsule du Sinaï et dans la vallée du Jourdain, il est aussi appelé bois de Sittim. Il fut l'un des matériaux utilisés pour la fabrication des objets du culte lévitique, dont l'arche*. Voir \vref{Ex. 25:10,13,23,28}.

\DicoEntry{ACHAB}\textit{, de l'hébreu «~Ach'ab~»~: «~un frère du père~»}\newline
Fils d'Omri, il fut roi d'Israël pendant vingt-deux ans. Marié à Jézabel*, fille du roi des Sidoniens, Achab et sa femme commirent de grandes abominations* et s'opposèrent au prophète Elie*. Voir \vref{1 R. 16:29-31}~; \vref{1 R. 18:1-40} et \vref{1 R. 22:29-40}.

\DicoEntry{ADAM}\textit{, de l'hébreu «~'Adam~»~: «~être humain~» ou «~de la terre~»}\newline
Premier homme, il vécut la première partie de sa vie dans le jardin d'Eden* avec sa femme Eve. Après avoir désobéi à Dieu en goûtant le fruit de l'arbre de la connaissance du bien et du mal, ils furent chassés du jardin. Adam fut le père de Caïn, Abel et Seth. Il mourut à neuf cent trente ans. Voir \vref{Ge. 2:7-8}~; \vref{Ge. 3}~; \vref{Ge. 4:1-2}, \vref{25-26} et \vref{Ge. 5:5}.

\DicoEntry{ADONIJA}\textit{, de l'hébreu «~'Adoniyah~»~: «~Yahweh est Seigneur~»}\newline
Quatrième fils de David et de Haggith. Peu avant la mort de son père, il s'autoproclama roi tentant en vain de prendre la place qui devait revenir à Salomon. Ce dernier lui laissa la vie sauve, mais le fit tuer plus tard alors qu'il semblait encore convoiter le trône d'Israël. Voir \vref{2 S. 3:4} et \vref{1 R. 1,2}.

\DicoEntry{ADOPTION}\textit{, du grec «~huiothesia~»~: «~adoption, adoption comme fils~»}\newline
Manifestation de l'amour éternel de Dieu, l'adoption permet à tout homme de devenir par la foi enfant de Dieu. Ce privilège, autrefois réservé au peuple d'Israël, fut étendu à toutes les nations par le sacrifice de Jésus. Cette adoption est manifestée par l'Esprit de Dieu qui témoigne à l'esprit du chrétien son appartenance à Dieu~; elle inclut les avantages du fils, dont l'héritage. Voir \vref{Jn. 1:12}~; \vref{Ga. 4:7}~; \vref{Ro. 8:15-17}~; \vref{Ro. 9:4}~; \vref{Ep. 1:5,11} et \vref{1 Jn. 3:1}.

\DicoEntry{AGAR}\textit{, de l'hébreu «~Hagar~»~: «~fuite~»}\newline
Servante égyptienne de Sara que cette dernière donna à Abraham comme concubine. Elle enfanta Ismaël, fils premier-né d'Abraham. Après la naissance d'Isaac, Abraham la chassa avec son fils. Voir \vref{Ge. 16} et \vref{Ge. 21:1-18}.

\DicoEntry{AGABUS}\textit{, de l'hébreu «~Chagab~» et du grec «~Agabos~»~: «~sauterelle~»}\newline
Prophète de Yahweh suscité au temps de l'Eglise primitive. Il prophétisa une famine qui se réalisa sous le règne de l'empereur Claude. Il annonça aussi l'arrestation de Paul à Jérusalem. Voir \vref{Ac. 11:27-28} et \vref{Ac. 21:10-33}.

\DicoEntry{AGGÉE}\textit{, de l'hébreu «~Chaggay~»~: «~en fête~» ou «~né un jour de fête~»}\newline
Prophète de Yahweh d'après la captivité, dont le livre éponyme figure dans le Tanakh.

\DicoEntry{AGNEAU}\textit{, de l'hébreu «~kebes~»~: «~agneau, brebis, jeune bélier~»}\newline
Animal sacrifié et consommé lors de la Pâque des juifs. Il préfigurait Christ, l'Agneau de Dieu qui ôte le péché du monde. Voir \vref{Ex. 12:1-28} et \vref{Jn. 1:29}.

\DicoEntry{AÏ}\textit{, de l'hébreu «~'Ay~»~: «~tas de ruines~»}\newline
Ville située au sud-est de Béthel, à proximité de laquelle Abraham dressa sa tente à deux reprises. Il s'agit également de la deuxième ville que Dieu livra entre les mains de Josué après la prise de Jéricho. Voir \vref{Ge. 12:8}~; \vref{Ge. 13:3} et \vref{Jos. 8}.

\DicoEntry{ALLÉLUIA}\textit{, de l'hébreu «~allelouia~»~: «~Louez Yahweh~»}\newline
Retrouvé à maintes reprises dans les Psaumes sous la forme «~Louez Yahweh~», cette exclamation encourage à célébrer Dieu et à se réjouir en lui. Voir \vref{Ap. 19:1-6}.

\DicoEntry{ALLIANCE}\textit{, de l'hébreu «~beriyth~»~: «~pacte, alliance, engagement~»}\newline
Dieu a conclu plusieurs alliances avec les hommes (ex~: Noé, Abraham, David). On distingue communément deux alliances majeures dans les Ecritures~: l'Ancienne Alliance - conclue avec Israël au travers de Moïse - et la Nouvelle Alliance inaugurée par Jésus-Christ. Voir \vref{Ge. 9:8-17}~; \vref{Ge. 17}~; \vref{Ex. 19-34}~; \vref{2 S. 7:12-16} et \vref{Hé. 9-13}.

\DicoEntry{ALPHA ET OMEGA}\textit{}\newline
Première et dernière lettre de l'alphabet grec, la combinaison de ces deux lettres mentionnées ensemble se rapporte à l'idée que Dieu est le premier et le dernier. Jésus fut présenté plusieurs fois comme étant «~l'alpha et l'oméga~» soulignant ainsi son caractère éternel. Voir \vref{Ap. 1:8}~; \vref{Ap. 21:6} et \vref{Ap. 22:13}.

\DicoEntry{ÂME}\textit{, de l'hébreu «~nephesh~»~: «~âme, une personne, la vie, être vivant~», «~ce qui respire~», «~ce qui a une vie par le sang~» et du grec «~psuche~»~: «~le souffle, la vie, l'âme~»}\newline
L'âme correspond au sang~; elle est le siège des émotions, de la volonté humaine et de l'intelligence. Avec l'esprit et le corps, l'âme constitue l'être humain. Voir \vref{Ge. 19:20}~; \vref{Ge. 44:30}~; \vref{Lé. 17:11}~; \vref{Mt. 10:28}~; \vref{Ac. 20:10} et \vref{1 Th. 5:23}.

\DicoEntry{AMEN}\textit{, de l'hébreu «~'amen~»~: «~assuré, établi~» ou «~ainsi soit-il~!~»}\newline
Se rapportant exclusivement à ce qui est sûr, avéré et certain, ce terme est souvent utilisé comme interjection. Christ est appelé «~l'Amen~», faisant référence à la vérité qu'il incarne. Voir \vref{Jé. 28:6}~; \vref{1 Ch. 16:36}~; \vref{2 Co. 1:20} et \vref{Ap. 3:14}.

\DicoEntry{AMOUR}\textit{}\newline
Il existe plusieurs traductions et définitions du mot «~amour~» en hébreu et en grec, elles varient selon le contexte.
\\- Les termes hébreux désignant l'amour~:
\\1. «~'Ahab~»~: «~amours~»
\\Amours, amis. Voir \vref{Os. 8:9} et \vref{Pr. 5:19}.
\\2. «~'Ahabah~»~: «~amour humain, amour de Dieu pour son peuple~»
\\Amour, affection, aimer. Voir \vref{De. 7:8}~; \vref{1 S. 20:17} et \vref{Pr. 10:12}.
\\3. «~Checed~»~: «~bonté, miséricorde, fidélité~»
\\Grâce, miséricorde, compassion, affection. Voir \vref{Ge. 40:14}~; \vref{Ex. 34:7} et Nb. \vref{14:19}.
\\4. «~Yediyd~»~: «~bien-aimé~»
\\Bien-aimé, amour. Voir \vref{De. 33:12} et \vref{Es. 5:1}.
\\- Les termes grecs désignant l'amour~:
\\1. «~Agape~»~: «~amour, charité, affection, bienveillance~»
\\Amour de Dieu, amour désintéressé que doit manifester l'homme né d'en haut. Voir \vref{Jn. 15:13}~; \vref{Jn. 17:26}~; \vref{1 Co. 8:1}~; \vref{1 Co. 13:3}~; \vref{Ro. 5:5} et \vref{1 Jn. 4:8}.
\\2. «~Eros~»~: «~l'amour qui prend~»
\\Amour dans la dimension sexuelle.
\\3. «~Phileo~»~: «~aimer, montrer des signes d'amour~»
\\Amour filial. Voir \vref{Jn. 21:17}~; \vref{1 Co. 16:22}.
\\4. «~Philadelphia~»~: «~amour fraternel~»
\\Amour des frères et sœurs d'une même famille, amour des chrétiens les uns pour les autres. Voir \vref{1 Th. 4:9}~; \vref{Ro. 12:10} et \vref{Hé. 13:1}.
\\5. «~Storge~»~: «~amour filial~»
\\Amour familial, affection naturelle. Voir \vref{Ro. 1:31}.

\DicoEntry{AMOS}\textit{, de l'hébreu «~'Amowc~»~: «~fardeau, porteur de fardeaux~»}\newline
Originaire de Tekoa en Juda, prophète de Yahweh dont le livre éponyme figure dans le Tanakh.

\DicoEntry{AMMONITES}\textit{, de l'hébreu «~'Ammown~»~: «~appartenant à la nation~»}\newline
Peuple issu de Ben-Ammi, né de l'inceste entre Lot et sa fille cadette. Ils furent ennemis d'Israël. Voir \vref{Ge. 19:30-38} et \vref{Ez. 25:1-7}.

\DicoEntry{ANAKIM}\textit{, de l'hébreu «~'Anaqiy~»~: «~au long cou~»}\newline
Descendants d'Anak, race de géants habitant Canaan avant sa conquête par le peuple d'Israël. Ils furent vaincus par Josué et Caleb qui hérita d'une partie de leur territoire. Voir \vref{No. 13:28-33}~; \vref{De. 9:1-3}~; \vref{Jos. 11:21-22} et \vref{Jos. 14:6-15}.

\DicoEntry{ANANIAS}\textit{, de l'hébreu «~Chananyah~»~: «~Dieu a été miséricordieux~»}\newline
1. Chrétien ayant vendu un champ avec sa femme Saphira et ayant fait croire qu'ils avaient donné la totalité du prix rapporté pour l'Eglise alors qu'ils en avaient secrètement gardé une partie. Ce mensonge les conduisit tous deux à la mort. Voir \vref{Ac. 5:1-10}.
\\2. Homme pieux vivant à Damas que le Seigneur envoya imposer les mains à Saul qui venait de se convertir afin qu'il recouvre la vue. C'est également lui qui le baptisa. Voir \vref{Ac. 9:10-18} et \vref{Ac. 22:12-16}.

\DicoEntry{ANATHÈME}\textit{, du grec «~anathema~»~: «~tout ce qui est livré au malheur~»}\newline
Terme désignant une personne ou une chose maudite, vouée à la destruction. Voir \vref{Ga. 1:8} et \vref{1 Co. 12:3}.

\DicoEntry{ANCIENS}\textit{, de l'hébreu «~zaqen~»~: «~vieux, aîné, de ceux qui ont de l'autorité~» et du grec «~presbuteros~»~: «~ayant de l'âge~»}\newline
Chez les juifs, il s'agissait des chefs de famille ou de clan qui représentaient le peuple dans les affaires religieuses et civiles. Voir \vref{Ex. 3:16}~; \vref{Lé. 4:15} et \vref{De. 31:28}. Sous la Nouvelle Alliance, les églises de la Galatie avaient élu des anciens («~presbuteros~») pour prendre soin des frères et sœurs. Il s'agit d'un terme relatif aux personnes ayant de l'âge et non à la fonction d'évêque*. Voir \vref{Ac. 14:23}~; \vref{1 Ti. 5:17}~; \vref{Tit. 1:5-9} et \vref{1 Pi. 5:1-5}.

\DicoEntry{ANDRÉ}\textit{, du grec «~Andreas~»~: «~virilité~»}\newline
Frère de Simon Pierre, originaire de Bethsaïda en Galilée, et pêcheur de métier. Il devint l'un des douze apôtres de Jésus-Christ. Voir \vref{Mt. 10:2}~; \vref{Mc. 1:16-17} et \vref{Jn. 1:40}.

\DicoEntry{ANGE}\textit{, de l'hébreu «~mal'ak~» et du grec «~aggelos~»~: «~messager, envoyé~»}\newline
Etre spirituel au service de Dieu pouvant prendre une forme humaine. Les anges sont au service de Yahweh pour des missions spécifiques au ciel ou sur la terre. Ils peuvent avoir une fonction de messager, protecteur ou combattant. Voir \vref{Da. 10:10-13}~; \vref{Lu. 1:26-38} et \vref{Ap. 12:7}.

\DicoEntry{ANNE}\textit{, de l'hébreu «~Channah~»~: «~grâce, faveur~»}\newline
1. Une des deux femmes d'Elkana. Stérile, elle pria Yahweh de lui accorder un fils qu'elle lui consacrerait. Elle enfanta ainsi Samuel qui entra au service de Yahweh dès son plus jeune âge. Voir \vref{1 S. 1,2}.
\\2. Fille de Phanuel de la tribu d'Aser, prophétesse. Veuve, elle servait le Seigneur nuit et jour dans le temple. Elle rencontra Jésus nourrisson, lorsqu'il fut amené au temple pour y être présenté à Dieu. Voir \vref{Lu. 2:36-38}.
\\3. Grand prêtre, beau-père de Caïphe*. Il participa à la conspiration qui mena Jésus à la croix. Voir \vref{Lu. 3:2} et \vref{Jn. 18:13}.

\DicoEntry{ANTICHRIST}\textit{, du grec «~antichristos~»~: «~l'adversaire du Messie~»}\newline
Aussi appelé «~homme impie~» et «~fils de la perdition~», personnage dont l'apparition se fera avant le retour glorieux du Seigneur. Il dominera le monde avant d'être vaincu par Christ. Voir \vref{2 Th. 2:1-4}~; \vref{2 Jn. 1:7} et \vref{Ap. 19:19-21}.

\DicoEntry{ANTIOCHE}\textit{, du grec «~Antiocheia~»~: «~rapide comme un char~»}\newline
Capitale de la Syrie, elle fut fondée en 300 av. J.-C. par Séleucus Nicator (358-281 av. J.-C.) qui la baptisa du nom de son père Antiochus. Cette ville accueillit des chrétiens en exil~; l'évangile y fut ainsi annoncé aux juifs puis aux Grecs et un grand nombre de personnes se convertirent. Barnabas et Paul y demeurèrent une année durant laquelle ils enseignèrent la parole. C'est à Antioche que les disciples furent appelés chrétiens pour la première fois. Voir \vref{Ac. 11:19-26}.

\DicoEntry{APIS}\textit{}\newline
Divinité égyptienne symbolisant la force et la fertilité. Il est représenté sous la forme d'un veau d'or ou d'un homme à tête de taureau dont les cornes entourent un disque solaire. Les Hébreux se corrompirent plusieurs fois en le vénérant. Voir \vref{Ex. 32:1-6} et \vref{1 R. 12:28-30}.

\DicoEntry{APOCALYPSE}\textit{, du grec «~apokalupsis~»~: «~mettre à nu, révélation d'une vérité, action de révéler~»}\newline
Dernier livre de la Bible écrit par Jean, ce récit comporte une révélation de la gloire de Jésus-Christ et raconte les derniers événements de l'histoire de l'humanité jusqu'à l'avènement de la Nouvelle Jérusalem.

\DicoEntry{APOLLOS}\textit{, du grec «~Apollos~»~: «~donné par Apollon~»}\newline
Juif érudit d'Alexandrie ayant une très bonne connaissance des Ecritures et enseignant avec exactitude au sujet de Jésus. Sa rencontre avec Aquilas et Priscille lui permit d'aller plus en profondeur dans la Parole et d'annoncer avec plus de force, notamment aux Juifs, que Jésus est le Messie en se basant sur les écrits du Tanakh. Il réalisa plusieurs voyages missionnaires notamment à Corinthe. Voir \vref{Ac. 18:24-28}, \vref{1 Co. 3:5-6} et \vref{1 Co. 16:12}.

\DicoEntry{APOSTASIE}\textit{, du grec «~apostasia~»~: «~action de s'éloigner de, désertion, défection~»}\newline
Abandon de la foi en Jésus-Christ et de la saine doctrine* se manifestant sous deux formes principales. Certaines personnes abandonnent ouvertement la foi, la communion avec Dieu et l'assemblée des saints. D'autres continuent de fréquenter les assemblées chrétiennes, mais ont laissé la saine doctrine pour s'attacher à des doctrines séductrices. Voir \vref{Mt. 24:11-12}~; \vref{2 Th. 2:3}~; \vref{1 Ti. 4:1-3}~; \vref{2 Pi. 2:1-3}~; \vref{2 Ti. 3:1-8}~; \vref{Jud. 1:17-19} et \vref{1 Jn. 4:1}.

\DicoEntry{APÔTRE}\textit{, du grec «~apostolos~»~: «~envoyé en avant, messager, ambassadeur~»}\newline
Lors de son service terrestre, Jésus choisit douze apôtres qu'il forma pour continuer l'œuvre après lui. Plusieurs autres apôtres furent suscités au temps de l'Eglise primitive, notamment Paul et Jacques, frère du Seigneur, qui avec Jean et Pierre furent les principaux instruments utilisés pour poser les fondements de la doctrine de l'Eglise. Le service apostolique existe encore aujourd'hui, mais la mission des apôtres actuels n'est pas d'écrire des épîtres, car la fondation a déjà été posée. Leur travail aujourd'hui consiste davantage à enseigner et veiller à ce que le fondement demeure. Voir \vref{Mc. 3:14}~; \vref{Ac. 15}~; \vref{Ga. 2:19}~; \vref{Ro. 1:1}~; \vref{Ep. 2:20} et \vref{Ep. 4:11}.

\DicoEntry{AQUILAS}\textit{, du latin «~Aquilas~»~: «~un aigle~» et PRISCILLE, du latin «~Priscilla~»~: «~petite vieille~»}\newline
Couple de Juifs ayant accepté l'évangile. Après avoir été chassés de Rome, ils s'installèrent à Corinthe où ils hébergèrent Paul à son arrivée et devinrent par la suite compagnons d'œuvre de ce dernier. Ils participèrent à plusieurs voyages missionnaires, notamment à Ephèse où ils enseignèrent Apollos*. Voir \vref{Ac. 18:1-3,18,24-26} et \vref{Ro. 16:3-5}.

\DicoEntry{ARBRE}\textit{, de l'hébreu «~'ets~»~: «~arbre, bois~»}\newline
Organisme vivant porteur de semence produisant des feuilles et des fruits selon les espèces. Lors de la création, Dieu créa différents arbres dont les fruits furent donnés pour nourrir l'homme et également deux arbres spécifiques placés au milieu du jardin d'Eden.

\DicoEntry{ARBRE DE LA CONNAISSANCE DU BIEN ET DU MAL}\textit{}\newline
Arbre dont le fruit contenait la connaissance du bien et du mal. Dieu interdit la consommation de ce dernier à l'homme sous peine de mort, mais Adam et Eve transgressèrent le commandement. C'est ainsi que le péché et la mort régnèrent sur l'humanité. Voir \vref{Ge. 2:17}~; \vref{Ge. 3:1-6} et \vref{Ro. 5:12}.

\DicoEntry{ARBRE DE VIE}\textit{}\newline
Arbre dont la consommation donne la vie éternelle. Après la chute d'Adam et Eve, Dieu les chassa du jardin pour les empêcher d'y accéder. L'arbre de vie se trouve dans la ville sainte, la Nouvelle Jérusalem~; ses feuilles servent à la guérison des nations. Voir \vref{Ge. 3:22-24} et \vref{Ap. 22:2,14,19}.

\DicoEntry{ARC-EN-CIEL}\textit{, de l'hébreu «~qesheth~»~: «~arc~»}\newline
Signe de l'alliance* que Dieu conclut avec Noé et les générations qui le suivraient suite au déluge*. Cette alliance stipulait que Yahweh ne détruirait plus les hommes par les eaux. Voir \vref{Ge. 9:12-17}.

\DicoEntry{ARCHANGES}\textit{, du grec «~archaggelos~»~: «~chef des anges~»}\newline
Catégorie d'anges* ayant un rang et une dignité plus élevés que les autres. Voir \vref{1 Th. 4:16} et \vref{Jud. 1:9}.

\DicoEntry{ARCHE DE NOÉ}\textit{, de l'hébreu «~tebah~»~: «~arche, vaisseau, coffre~»}\newline
Embarcation construite par Noé pour le sauver lui, sa famille ainsi que les animaux, du déluge* qui allait s'abattre sur la terre. Voir \vref{Ge. 6:8-16}~; \vref{Mt. 24:37-39} et \vref{Lu. 17:26-27}.

\DicoEntry{ARCHE DU TEMOIGNAGE ou DE L'ALLIANCE}\textit{, de l'hébreu «~'arown~»~: «~arche, coffre, cercueil~»}\newline
Coffre rectangulaire en bois d'acacia* recouvert d'or pur, contenant les tables de l'alliance, la verge d'Aaron et une urne contenant un échantillon de la manne. Construite selon le modèle que Moïse avait reçu au mont Sinaï, elle était couverte par le propitiatoire*. L'arche fut placée dans le Saint des saints du tabernacle*, puis du temple*. Voir \vref{Ex. 25:10-22}~; \vref{1 R. 8:6}~; \vref{2 R. 25:8-9} et \vref{Hé. 9:4}.

\DicoEntry{ARTAXERXÈS}\textit{, (règne~: 465 av. J.-C.- 424 av. J.-C.), du persan «~Artachshashta~»~: «~celui qui fait régner la loi sacrée~»}\newline
Fils d'Assuérus*, roi de Perse. Il autorisa Esdras à retourner à Jérusalem avec des prêtres et des Lévites pour effectuer le sacerdoce dans le temple et faire respecter la loi de Yahweh. Voir \vref{Esd. 7:11-28}.

\DicoEntry{ASAPH}\textit{, de l'hébreu «~'Acaph~»~: «~celui qui rassemble, collecteur~»}\newline
Lévite et chef des chantres sous David, il participa au transfert de l'arche* à Jérusalem et écrivit certains psaumes. Voir \vref{1 Ch. 15:16-19} et \vref{1 Ch. 16:4-7}.

\DicoEntry{ASER}\textit{, de l'hébreu «~'Asher~»~: «~heureux~»}\newline
Fils de Jacob et de Zilpa, servante de Léa, il est le père de la tribu d'Aser. Voir \vref{Ge. 30:13}.

\DicoEntry{ASHERAH}\textit{, de l'hébreu «~'Asherah~»~: «~pieu sacré~».}\newline
Voir commentaire en \vref{Ex. 34:13}.

\DicoEntry{ASSUÉRUS ou XERXÈS Ier}\textit{, (485 av. J.-C. – 465 av. J.-C.), du persan «~'Achashverowsh~»~: «~je serai silencieux et pauvre~».}\newline
Père d'Artaxerxès, roi de Perse et époux d'Esther*. Voir le livre d'Esther.

\DicoEntry{ASTARTE}\\textit{, de l'hébreu «~Ashtoreth~»~: «~étoile~».}\newline
Voir commentaire en \vref{Jg. 2:13}.

\DicoEntry{AUTEL}\textit{, de l'hébreu «~mizbeach~»~: «~autel~»}\newline
Table généralement façonnée avec des monticules de pierres ou en terre et élevée spécialement pour offrir des holocaustes* et des sacrifices en l'honneur de Dieu. Voir \vref{Ge. 12:7}~; \vref{Ge. 35:7}~; \vref{Ex. 20:24-26} et \vref{Ex. 30:1-8}.

\DicoEntry{BAAL}\textit{, de l'hébreu «~Ba'al~»~: «~maître, possesseur, seigneur~»}\newline
Dieu primaire des Phéniciens et des Cananéens auquel les Israélites s'attachèrent à plusieurs reprises pour l'adorer. Voir \vref{No. 25:3}~; \vref{Jg. 2:11} et \vref{1 R. 18:21}. Voir aussi commentaire en \vref{Jg. 2:11}.

\DicoEntry{BABEL ou BABYLONE}\textit{, de l'hébreu «~Babel~»~: «~confusion (par le mélange)~»}\newline
Ville de Mésopotamie* située sur l'Euphrate, capitale de la Babylonie. Les hommes y entreprirent la construction de la tour de Babel. Cependant, Yahweh confondit leur langage et les dispersa sur toute la terre. Voir \vref{Ge. 10:8-10} et \vref{Ge. 11:1-9}.

\DicoEntry{BALAAM}\textit{, de l'hébreu «~Bil'am~»~: «~sans peuple~», «~dévorant~»}\newline
Prophète de Yahweh ayant vécu pendant la marche d'Israël dans le désert, il fut séduit par Balak, roi de Moab, qui lui proposa de maudire Israël contre de généreux présents. Son témoignage a été utilisé plusieurs fois pour avertir les enfants de Dieu des scandales* dont ils pourraient être la cause en suivant la voie de la cupidité. Voir \vref{No. 22-24}~; \vref{No. 31:8}~; \vref{Jud. 1:11} et \vref{Ap. 2:14}.

\DicoEntry{BALAK}\textit{, de l'hébreu «~Balaq~»~: «~gaspilleur, dévastateur~»}\newline
Roi de Moab, il essaya de convaincre Balaam* de maudire Israël qu'il redoutait. Voir \vref{No. 22-24}.

\DicoEntry{BANNIÈRE}\textit{, de l'hébreu «~nec~»~: «~quelque chose de levé, étendard, signal, enseigne~»}\newline
Drapeau, étendard élevé en signe d'appartenance à ce qu'il représente. Moïse bâtit un autel du nom de Yahweh-Nissi~: «~Yahweh ma bannière~». Voir \vref{Ex. 17:15}~; \vref{Es. 11:10,12} et \vref{Ps. 60:4}.

\DicoEntry{BAPTÊME}\textit{, du grec «~baptizo~»~: «~plonger, immerger, purifier en plongeant~»}\newline
On distingue trois types de baptêmes dans les Ecritures (\vref{Mt. 3:11})~:
\\1. le baptême d'eau~: acte suivant la conversion par lequel une personne est immergée dans l'eau - symbolisant la mort et la résurrection en Jésus-Christ. Il s'agit selon Pierre de l'«~engagement d'une bonne conscience envers Dieu~». Voir \vref{Ac. 2:38}~; \vref{Ac. 16:30-33}~; \vref{Col. 2:12-13} et \vref{1 Pi. 3:21}.
\\2. le baptême du Saint-Esprit~: lors de la naissance d'en haut, gage que le Seigneur donne au nouveau converti par l'envoi du Saint-Esprit. Voir \vref{Jn. 3:5-6}~; \vref{Tit. 3:4-7} et \vref{Ep. 1:13}.
\\3. le baptême de feu~: symbole des souffrances que Christ a endurées à la croix et par lesquelles tous les chrétiens sont appelés à passer pour être purifiés. Voir \vref{Mc. 10:35-39}~; \vref{Lu. 12:50}~; \vref{1 Pi. 1:6-9} et \vref{1 Pi. 4:12-13}.

\DicoEntry{BARAK}\textit{, de l'hébreu «~Baraq~»~: «~éclairs, foudre~»}\newline
Fils d'Abinoam, issu de la tribu de Nephtali, il vécut en Israël au temps des juges. Encouragé et accompagné par Débora, il battit l'armée de Jabin, roi de Canaan. Voir \vref{Jg. 4}.

\DicoEntry{BARTHÉLÉMY}\textit{, du grec «~Bartholomaios~»~: «~fils de Tolmaï~»}\newline
Un des douze apôtres de Jésus. Voir \vref{Mt. 10:3}.

\DicoEntry{BARTIMÉE}\textit{, du grec «~Bartimaios~»~: «~fils de Timée~»}\newline
Fils de Timée, mendiant aveugle que Jésus guérit suite à ses cris de supplications sur la route de Jéricho. Voir \vref{Mc. 10:46-52}.

\DicoEntry{BATH-SCHEBA}\textit{, de l'hébreu «~Bath-Sheba`~»~: «~fille d'un serment~»}\newline
Fille d'Eliam, femme d'Urie* le Héthien que David fit mourir après l'avoir mise enceinte. Elle devint la femme de David et fut la mère de Salomon. Voir \vref{2 S. 11:3-5,26-27} et \vref{2 S. 12:24-25}.

\DicoEntry{BEELZÉBUL}\textit{, de l'hébreu «~Ba'al-Zebuwb~»~: «~seigneur des mouches~»}\newline
Divinité adorée par les Philistins et considérée comme le prince des démons. Voir \vref{2 R. 1:2-6} et \vref{Mc. 3:22-26}.

\DicoEntry{BÉLIAL}\textit{, de l'hébreu «~Beliya'al~»~: «~indignité~»}\newline
Symbolisant l'infidélité, la méchanceté et la perversité, il s'agit d'un autre nom de Satan. Voir \vref{De. 15:9}~; \vref{1 S. 1:16}~; \vref{2 Co. 6:15}.

\DicoEntry{BÉNÉDICTION}\textit{, de l'hébreu «~barak~», «~berakah~», et du grec «~eulogia~»~: «~louange~»}\newline
Parole au travers de laquelle le Seigneur annonce sa grâce sur la vie d'une personne ou d'un peuple~; les bontés liées à la bénédiction sont cependant conditionnées par l'obéissance du bénéficiaire. Sous l'Ancienne Alliance, les pères avaient coutume de bénir leurs enfants~; la bénédiction se manifestait souvent par la prospérité matérielle, la fécondité et la santé. La bénédiction est la marque du chrétien qui voit avec un œil spirituel la faveur de Dieu dans sa vie et qui bénit Dieu dans toutes les circonstances. Voir \vref{Ge. 49:1-28}~; \vref{De. 28:1-14}~; \vref{Ps. 103:1-2} et \vref{Ep. 1:3}.

\DicoEntry{BENJAMIN}\textit{, de l'hébreu «~Binyamiyn~»~: «~fils de ma main droite~»}\newline
Dernier fils de Jacob et Rachel~; sa mère mourut en lui donnant naissance. Il est l'ancêtre de la tribu de Benjamin. Voir \vref{Ge. 35:16-18} et \vref{Ge. 49:27}.

\DicoEntry{BÊTE}\textit{, de l'araméen «~cheyva´~» et du grec «~therion~»~: «~bête, animal~»}\newline
Dans les récits à caractère apocalyptique, les bêtes sont des animaux symbolisant les puissances politiques. Voir \vref{Da. 7} et \vref{Ap. 13,17}.

\DicoEntry{BÉTHANIE}\textit{, du grec «~Bethania~»~: «~maison des dattes non mûres~», «~maison de l'affligé~»}\newline
Village proche de Jérusalem, près de la Montagne des Oliviers, où vivaient Simon le lépreux, Marthe*, Marie* et Lazare* que Jésus ressuscita des morts. Voir \vref{Mc. 11:1}~; \vref{Mc. 14:3} et \vref{Jn. 11:1}.

\DicoEntry{BÉTHEL}\textit{, de l'hébreu «~Beyth-'El~»~: «~maison de Dieu~»}\newline
Ville cananéenne située à l'occident de Aï. Autrefois appelé Luz - mais renommé par Jacob quand il y eut la visitation de Yahweh - Béthel devient la possession de la tribu d'Ephraïm lors de la conquête de Canaan conduite par Josué. Elle était connue pour être un lieu d'adoration où on y rendait un culte à Yahweh. Malheureusement suite au schisme d'Israël - et notamment sous le règne de Jéroboam, roi de Juda - elle devient un lieu d'abomination. C'est Josias son successeur qui, désirant marcher avec Yahweh, y ôta les faux dieux, rétablissant ainsi le culte en l'honneur du Dieu d'Israël. (\vref{Ge. 28:10-22}~; \vref{Ge. 31:13}~; \vref{1 R. 12:26-32}~; \vref{2 R. 23:1-15})

\DicoEntry{BETHLEHEM}\textit{, de l'hébreu «~Beyth Lechem~»~: «~maison du pain~»}\newline
Ville de Juda, lieu de naissance de David et de Jésus-Christ. Voir \vref{1 S. 16}~; \vref{Mt. 2:16} et \vref{Lu. 2:4-7}.

\DicoEntry{BIBLE}\textit{, du grec «~biblia~»~: «~livres~»}\newline
Aussi appelée «~Parole de Dieu~», recueil de livres inspirés de Dieu et utiles pour enseigner, convaincre, corriger et instruire dans la justice. Voir \vref{2 Ti. 3:16}.

\DicoEntry{BLASPHÈME}\textit{, de l'hébreu «~na'ats~»~: «~repousser, mépriser, rejeter~» et du grec «~blasphemia~»~: «~discours impie et injurieux envers Dieu~»}\newline
Parole outrageante ou insultante envers Dieu. Voir \vref{2 S. 12:14} et \vref{Ap. 16:9}.

\DicoEntry{BLASPHÈME CONTRE LE SAINT-ESPRIT}\textit{}\newline
Voir commentaire \vref{Mt. 12:22-32}.

\DicoEntry{BOAZ}\textit{, de l'hébreu «~Bo`az~»~: «~en lui est la force~»}\newline
Fils de Salmon et arrière-grand-père du roi David, il épousa Ruth la Moabite. Voir \vref{Ru. 4:13} et \vref{Mt. 1:1-6}.

\DicoEntry{BREBIS}\textit{}\newline
Femelle du bélier, c'est l'animal pour qui le berger donne sa vie. Elle est le symbole du véritable disciple qui n'obéit qu'à la voix de son Maître et qui se laisse conduire et choyer par Jésus, le bon berger. Voir \vref{Jn. 10:1-16}.

\DicoEntry{CAIN}\textit{, de l'hébreu «~Qayin~»~: «~possession~», «~artisan, forgeron~»}\newline
Fils aîné d'Adam et Eve, il fut l'auteur du premier homicide en tuant son frère Abel. Il engendra Lémec, premier polygame de l'histoire. Voir \vref{Ge. 4:1-8,16-19}.

\DicoEntry{CAÏPHE}\textit{, du grec «~Kaiaphas~»~: «~avenant, pierre~»}\newline
Grand prêtre nommé par Valerius Gratus, gouverneur de Judée de 15 à 26 ap. J.-C. Caïphe exerça sa fonction de 18 à 36. N'ayant pas reconnu en Christ le Messie, il déclara néanmoins qu'il était avantageux qu'un seul homme meure pour le peuple et participa à la condamnation à mort de Jésus. Voir \vref{Mt. 26:3,57-66}~; \vref{Jn. 11:47-53} et \vref{Jn. 18:12-14}.

\DicoEntry{CALEB}\textit{, de l'hébreu «~Kaleb~»~: «~chien~»}\newline
Fils de Jephunné, issu de la tribu de Juda, il fut l'un des espions envoyés pour explorer le pays de Canaan. Avec Josué, il fut le seul, parmi la génération sortie d'Egypte, à entrer dans la terre promise. Voir \vref{No. 13:1-6} et \vref{No. 14:22-30}.

\DicoEntry{CALENDRIER HEBRAÏQUE}\textit{}\newline
Nisan (ou Abib) = Mars~; Iyyar (ou Ziv) = Avril~; Sivan = Mai~; Thammuz = Juin~; Ab = Juillet~; Elul = Août~; Tisri (ou Ethanim) = Septembre~; Marchesvan (ou Bul) = Octobre~; Chislev (ou Kisleu) = Novembre~; Tébeth = Décembre~; Schebat = Janvier~; Adar = Février

\DicoEntry{CAMP}\textit{, de l'hébreu «~machaneh~»~: «~campement, camp~»}\newline
Lieu de stationnement temporaire d'un groupement civil ou militaire. Voir \vref{Ge. 32:2} et \vref{Ex. 14:19}.

\DicoEntry{CANAAN}\textit{, de l'hébreu «~Kena'an~»~: «~terre basse~», «~marchand~»}\newline
Fils de Cham. Ses descendants occupèrent la région éponyme qui correspond plus ou moins aujourd'hui aux territoires réunissant la Palestine, l'État d'Israël, l'ouest de la Jordanie, le sud du Liban et l'ouest de la Syrie. Ce territoire correspondait également à la terre promise par Dieu aux Israélites dont ils prirent possession sous la conduite de Josué. Voir \vref{Ge. 9:18}~; \vref{Jos. 6-21} et \vref{Ac. 13:19}.

\DicoEntry{CÉSAR, Jules}\textit{, (100 av. J.-C - 44 av. J.-C.) du latin «~kaisar~»~: «~séparé~», «~chef~»}\newline
Général romain. Son nom devint par la suite celui de certains empereurs romains. Dans les Ecritures, César symbolise également les autorités séculières. Voir \vref{Mt. 22:21}.

\DicoEntry{CESARÉE de Philippes}\textit{, du grec «~Kaisereia~»~: «~appartenant à César~»}\newline
Située près des sources du Jourdain, territoire qui doit son nom à l'empereur Tibère. C'est dans cette contrée que Pierre reconnut en Jésus le Messie, le Fils du Dieu vivant. Voir \vref{Mt. 16:13-17}.

\DicoEntry{CHAIR}\textit{, du grec «~sarx~»~: «~la chair, le corps, la nature sensuelle de l'homme, la nature animale~»}\newline
Selon le contexte, désigne le corps humain, l'être humain ou la nature humaine conduite par le péché*. Voir \vref{Lu. 3:6}~; \vref{Lu. 24:39}~; \vref{Jn. 17:2}~; \vref{Ga. 5:16-21}~; \vref{Ro. 8:5-9} et \vref{Ep. 2:3}.

\DicoEntry{CHALDÉE}\textit{, de l'hébreu «~Kasdiy~»~: «~briseurs de mottes~», «~comme des démons~»}\newline
Région située au sud de la Mésopotamie dont Abraham est originaire. Voir \vref{Ge. 11:28}.

\DicoEntry{CHAM}\textit{, de l'hébreu «~Cham~»~: «~chaud, bouillant~»}\newline
Fils de Noé et père de Canaan qui fut maudit par Noé. Voir \vref{Ge. 9:18-27}.

\DicoEntry{CHARAN}\textit{, de l'hébreu «~Charan~»~: «~montagnard~», «~route, caravane~»}\newline
Région proche d'Ur en Chaldée où Abraham séjourna jusqu'à la mort de son père Térach. Voir \vref{Ge. 11:31} et \vref{Ge. 12:4}.

\DicoEntry{CHEMIN DE SABBAT}\textit{}\newline
Selon la loi de Moïse, distance maximum que les juifs peuvent parcourir de leur demeure le jour du sabbat* (cf. tableau des mesures et.distances). Voir \vref{Ac. 1:12}.

\DicoEntry{CHÉRUBINS}\textit{, de l'hébreu «~keruwb~»~: «~être angélique, chérubin~»}\newline
Catégorie d'anges portant et.ou gardant la gloire de Dieu. Yahweh en avait placé à l'entrée du jardin d'Eden pour empêcher l'homme d'y accéder. Deux chérubins sur lesquels Dieu siégeait étaient représentés sur le propitiatoire*. Avant sa chute, Satan était un chérubin protecteur. Voir \vref{Ge. 3:24}~; \vref{Ex. 25:17-20}~; \vref{Es. 37:16} et \vref{Ez. 28:14}.

\DicoEntry{CHRÉTIEN}\textit{, du grec «~christanos~»~: «~de Christ~», «~petit christ~», «~comme Christ~»}\newline
Comme son étymologie le suggère, le chrétien appartient à Christ, dont il a la nature et à qui il ressemble. Il est donc un disciple* de Jésus-Christ qui suit son enseignement et le met en pratique. Ce terme fut employé pour la première fois à Antioche. Voir \vref{Ac. 11:26}.

\DicoEntry{CHRIST}\textit{, du grec «~christos~» et de l'hébreu «~mashiyach~»~: «~oint~»}\newline
Souvent accolé au nom de Jésus*, ce terme suggère que ce dernier est l'oint* de Dieu, le Messie tant attendu. Jésus annonça l'émergence de faux christs (= faux ouvriers de Christ) à la fin des temps. Voir \vref{Ro. 1:1}~; \vref{Mt. 16:15-16}~; \vref{Mt. 24:24} et \vref{Mc. 13:22-23} et \vref{Hé. 1:9}.

\DicoEntry{CIRCONCISION}\textit{, de l'hébreu «~muwlah~»~: «~circoncision~: couper autour~»}\newline
Section et ablation du prépuce. En signe d'alliance, Dieu ordonna à Abraham de circoncire tous les mâles de sa maison~; les enfants d'Israël ont perpétré cette pratique. Sous la Nouvelle Alliance, la circoncision requise est celle du cœur. Voir \vref{Ge. 17:9-14}~; \vref{Lu. 1:59}~; \vref{1 Co. 7:19} et \vref{Ro. 2:25-29}.

\DicoEntry{CLAUDE}\textit{, (10 av. J.-C. – 54 ap. J.-C.), du grec «~Klaudios~»~: «~boiteux~»}\newline
Fils de Nero Claudius Drusus (38 av. J.-C. – 9 av. J.-C.). Empereur romain qui régna de 41 à 54 ap. J.-C.~; il chassa les Juifs de Rome, parmi lesquels Aquilas et Priscille. Voir \vref{Ac. 18:2}.

\DicoEntry{CLERGÉ}\textit{, du grec «~klêrikos~»~: «~homme d'église~»}\newline
Au sein de l'Eglise catholique, corps séparé des fidèles ayant une fonction gouvernante~; ses membres sont appelés les clercs ou les ecclésiastiques. Ils accèdent à leur position par le sacrement de l'ordre (ou ordination*) qui comporte trois classes~: les diacres, les prêtres et les évêques.

\DicoEntry{CLÉRICALISME}\textit{, dérivé de clérical~: «~dévoué aux intérêts du clergé~»}\newline
Tendance en vertu de laquelle le clergé sort du domaine religieux pour se mêler des affaires publiques et politiques afin d'y exercer une influence et faire prédominer ses idées.

\DicoEntry{CŒUR}\textit{, de l'hébreu «~lebab~»~: «~homme intérieur, volonté, cœur, partie interne, pensée~»}\newline
Organe permettant la circulation du sang, les Ecritures définissent le cœur comme un grand abîme. Siège des émotions et des pensées intimes, il peut être une bonne ou une mauvaise source. Voir \vref{Ge. 20:6}~; \vref{Lé. 19:17}~; \vref{De. 4:29}~; \vref{1 S. 12:24} et \vref{Mc. 7:21}.

\DicoEntry{COLOSSES}\textit{, du grec «~Kolossai~»~: «~monstruosités~»}\newline
Située en Asie Mineure, ville de Phrygie se trouvant à environ deux cents kilomètres d'Ephèse. Il s'y trouvait une église à qui Paul écrivit une lettre qui figure dans le canon biblique.

\DicoEntry{COMMUNION}\textit{, du grec «~koinonia~»~: «~ce qui est commun à plusieurs personnes, association, union~»}\newline
Le disciple* de Christ est appelé à vivre deux types de communion. Il doit tout d'abord être en communion intime avec Dieu puis avec d'autres membres du corps de Christ pour vivre la communion fraternelle. Voir \vref{Ps. 133}~; \vref{Ac. 2:42}~; \vref{2 Co. 13:11-13} et \vref{1 Jn. 1:3}.

\DicoEntry{CONCILE}\textit{, du latin «~concilium~»~: «~assemblée~»}\newline
Assemblée d'évêques de l'Eglise catholique (également connue sous l'appellation «~pères de l'Eglise catholique~») réunis dans le but de définir les règles de la foi chrétienne. Cette pratique va à l'encontre du message de Christ puisqu'il a strictement condamné la modification du message qu'il a lui-même prêché et confié aux apôtres*. Voir \vref{Mt. 5:18} et \vref{Ga. 1:8-9}.

\DicoEntry{CONFESSION}\textit{, du grec «~exomologeo~»~: «~confesser, professer, reconnaître ouvertement~»}\newline
On peut confesser des péchés pour exposer les ténèbres ou le nom du Seigneur pour le louer et annoncer la vérité. Voir \vref{Mc. 1:5}~; \vref{Ac. 19:18} et \vref{Ph. 2:11}.

\DicoEntry{CONVERSION}\textit{, du grec «~epistrepho~»~: «~action de se retourner, de se tourner vers~»}\newline
Fruit d'une sincère repentance, la conversion est la décision de se tourner vers Christ et de se détourner des œuvres des ténèbres. Voir \vref{Ac. 26:20}~; \vref{Ga. 4:9}~; \vref{2 Co. 3:16} et \vref{1 Pi. 2:25}.

\DicoEntry{CONVOITISE}\textit{, du grec «~epithumia~»~: «~désir, convoitise, luxure~»}\newline
Précédant l'acte du péché, désir amorcé par les sens humains et lié à la soif de posséder ce qui est défendu et ce que le monde offre. Voir \vref{Ja. 1:14-15} et \vref{1 Jn. 2:15-17}.

\DicoEntry{CORINTHE}\textit{, du grec «~Korinthos~»~: «~rassasié~»}\newline
Dans l'Antiquité, Corinthe, capitale de l'Achaïe, était la ville la plus prospère et la plus puissante de Grèce. Située sur un isthme séparant la mer Egée de la mer Ionienne, Corinthe était au carrefour de l'Asie et de l'Italie et constituait un véritable centre commercial où les produits orientaux et occidentaux se croisaient. Paul demeura au moins un an et six mois à Corinthe, durée pendant laquelle il enseigna la parole de Dieu. Il écrivit par la suite deux lettres aux saints de cette ville qu'on retrouve dans le canon biblique.

\DicoEntry{CORNEILLE}\textit{, du grec «~Kornelios~»~: «~d'une corne~»}\newline
Centenier romain juste et craignant Dieu. Il vivait à Césarée où Simon Pierre fut envoyé pour lui annoncer la Parole. Au travers de l'expérience de Corneille, Dieu confirma que le salut était pour toutes les nations. Voir \vref{Ac. 10}.

\DicoEntry{COURONNE}\textit{, du grec «~stephanos~»~: «~couronne, une marque de rang royal, récompense de la justice, ornement~»}\newline
Jésus-Christ reçut une couronne d'épines lors de la crucifixion pour rappeler ironiquement son titre de «~roi des Juifs~». Après la résurrection, les chrétiens recevront une couronne en récompense de leur intégrité. Devant le trône de Dieu, les vingt-quatre vieillards jettent leurs couronnes pour rendre gloire à Dieu. Voir \vref{Mt. 27:29}~; \vref{Ja. 1:12}~; \vref{1 Co. 9:25}~; \vref{1 Pi. 5:4}~; \vref{2 Ti. 4:8}~; \vref{Ap. 2:10} et \vref{Ap. 4:4,10}.

\DicoEntry{CROIX}\textit{, du grec «~stauros~»~: «~pieu, croix~»}\newline
Châtiment romain consistant à clouer les mains et les pieds des condamnés sur des poteaux en bois en forme de croix. Symbole du sacrifice de Jésus pour le pardon des péchés, la croix est aussi l'image de la vie de souffrance et de consécration totale à laquelle est appelé tout disciple du Seigneur. Voir \vref{Es. 53}~; \vref{Mt. 16:24} et \vref{Lu. 9:23}.

\DicoEntry{CUPIDITÉ}\textit{, du grec «~pleonexia~»~: «~désir avide d'avoir plus, avarice~»}\newline
Forme d'idolâtrie*, péché consistant à désirer de manière excessive les biens de ce monde (argent, richesses, etc.) et menant à la perdition. Voir \vref{Ep. 5:3}~; \vref{Col. 3:5} et \vref{2 Pi. 2:14}.

\DicoEntry{CYRÈNE}\textit{, du grec «~Kurene~»~: «~suprématie de la bride~», «~qui gouverne, froid~»}\newline
Ville prospère située dans la région fertile d'Afrique du Nord (actuelle Libye) où vivait une importante communauté juive et de laquelle était originaire Simon à qui l'on demanda de porter la croix de Jésus. Voir \vref{Mc. 15:20-22} et \vref{Ac. 2:10}.

\DicoEntry{CYRUS II LE GRAND}\textit{(règne~: 559-530 av. J.-C.), du persan «~Kowresh~»~: «~possède la puissance, puissance suprême~»}\newline
Fils de Cambyse, il régna sur l'Empire perse. Réveillé par Yahweh, il publia un édit en faveur du retour des Juifs à Jérusalem pour la reconstruction du temple. Voir \vref{Esd. 1:1-2} et \vref{2 Ch. 36:22-23}.

\DicoEntry{DAGON}\textit{, «~Dagown~»~: «~un poisson~»}\newline
Divinité païenne adorée par les Philistins, il était représenté par un personnage avec des mains et une face humaine et le corps d'un poisson. Voir \vref{1 S. 5:1-5}.

\DicoEntry{DAN}\textit{, de l'hébreu «~dan~»~: «~un juge~»}\newline
Fils de Jacob et de Bilha, servante de Rachel, il est le père de la tribu des Danites. Voir \vref{Ge. 30:1-6} et \vref{Ge. 49:16-18}.

\DicoEntry{DANIEL}\textit{, de l'hébreu «~Daniye'l~»~: «~Dieu est mon juge~»}\newline
Issu d'une famille princière de Juda, il fut déporté pendant sa jeunesse de Jérusalem à Babylone où il reçut le nom de Beltshatsar. Son histoire est racontée dans le livre éponyme.

\DicoEntry{DARIQUE}\textit{, de l'hébreu «~darkemown~»~: «~darique, drachme, unité de mesure~»}\newline
Utilisée après le retour de l'exil babylonien, monnaie d'or mise en place par le roi Darius et circulant dans l'Empire perse. Voir \vref{Esd. 8:26-27} et \vref{Né. 7:71-72}.

\DicoEntry{DARIUS Ier}\textit{, (règne~: 522 av. J.-C. – 486 av. J.-C.), de l'hébreu «~Dar`yavesh~»~: «~seigneur~» (origine~: perse)}\newline
Fils d'Assuérus, d'origine mède, roi des Chaldéens. Il encouragea la reconstruction du temple de Jérusalem après la découverte des instructions laissées par Cyrus sur un rouleau retrouvé dans la province de Médie. Voir \vref{Esd. 6}.

\DicoEntry{DATHAN}\textit{, de l'hébreu «~Dathan~»~: «~appartenant à une fontaine~»}\newline
Issu de la tribu de Ruben, fils d'Eliab et frère d'Abiram, il participa avec Koré à la révolte contre Moïse et Aaron. Voir \vref{No. 16:1-35}.

\DicoEntry{DAVID}\textit{, de l'hébreu «~David~»~: «~bien aimé~»}\newline
Issu de la tribu de Juda et dernier fils d'Isaï, il entra dès son plus jeune âge au service du roi Saül avant de devenir roi d'Israël. Homme selon le cœur de Dieu, il connut de grands succès sur les champs de bataille et fut l'auteur de nombreux psaumes. Il régna quarante-quatre ans sur Israël puis son fils Salomon* lui succéda. Voir \vref{1 S. 13:14}~; \vref{1 S. 16:14-23}~; \vref{1 S. 17}~; \vref{1 R. 2:10-11} et \vref{Ac. 13:22}.

\DicoEntry{DÉBORA}\textit{, de l'hébreu «~Debowrah~»~: «~abeille~»}\newline
Femme de Lapiddoth, elle exerça les fonctions de prophétesse et juge en Israël. Elle fut utilisée par Dieu pour prophétiser la victoire d'Israël sur Canaan par Barak qu'elle accompagna sur le champ de bataille. Voir \vref{Jg. 4-5}.

\DicoEntry{DÉLUGE}\textit{, de l'hébreu «~mabbuwl~»~: «~inondation, déluge~»}\newline
Pluie torrentielle s'étant abattue sur la terre pendant quarante jours et quarante nuits au temps de Noé. Le déluge symbolisait le jugement de Dieu sur une génération dont la méchanceté avait atteint un niveau sans précédent. Tous les habitants et les animaux de la terre furent emportés par les eaux du déluge hormis Noé, sa famille et les animaux qui étaient avec eux dans l'arche*. Voir \vref{Ge. 6-8}.

\DicoEntry{DEMAS}\textit{, du grec «~Demas~»~: «~gouverneur du peuple~»}\newline
Compagnon d'œuvre de Paul qui le délaissa «~par amour pour le siècle présent~». Voir \vref{Col. 4:14} et \vref{2 Ti. 4:10}.

\DicoEntry{DEMETRIUS}\textit{, du grec «~Demetrios~»~: «~qui appartient à Déméter (déesse grecque de l'agriculture)~»}\newline
Orfèvre qui fabriquait des statues de la déesse Diane à Ephèse. Voyant son commerce mis en danger par les prédications de Paul, il déclencha une émeute contre ce dernier. Voir \vref{Ac. 19:23-41}.

\DicoEntry{DÉMONS}\textit{, du grec «~daimonion~»~: «~divinité inférieure, mauvais esprit, ministres du diable~»}\newline
Egalement appelés «~esprits impurs~», anges* déchus ayant pris part à la révolte et à la chute de Satan*. Ils peuvent posséder le corps d'une personne, mais sont soumis à la puissance de Jésus, au nom duquel les chrétiens peuvent les chasser. Voir \vref{Mt. 10:8}~; \vref{Mc. 7:26}~; \vref{Mc. 16:17}~; \vref{Lu. 4:33}~; \vref{Lu. 10:17}~; \vref{Jud. 1:6} et \vref{Ap. 12:4}.

\DicoEntry{DIABLE}\textit{}\newline
Voir SATAN.

\DicoEntry{DIACRE}\textit{, du grec «~diakonos~»~: «~domestique, subordonné, messager~»}\newline
Les premiers diacres étaient des hommes remplis de l'Esprit Saint et de sagesse~; ils furent nommés pour faire un travail complémentaire aux ministres de la Parole au sein de l'église de Jérusalem. Etienne* était l'un d'eux. Il existait aussi des femmes diaconesses comme Phœbe, de l'église de Cenchrées. Voir \vref{Ac. 6:1-8}~; \vref{Ro. 16:1-2} et \vref{1 Ti. 3:8-13}.

\DicoEntry{DIANE}\textit{, du grec «~Artemis~»~: «~de la lumière~»}\newline
Aussi appelée «~Artemis d'Ephèse~», divinité révérée dans toute l'Asie. Il existait un temple en son honneur à Ephèse. Voir \vref{Ac. 19:24-37}.

\DicoEntry{DIEU}\textit{}\newline
Dieu des dieux et Seigneur des seigneurs, il est le Créateur de l'univers, du ciel, de la terre et de tout ce qui s'y trouve. Architecte d'excellence, il forma l'homme à son image et lui manifesta un amour inconditionnel par son incarnation en Jésus-Christ*. Dieu se présenta à Moïse sous le nom YHWH* (=Je suis celui qui suis) montrant son caractère éternel. Il s'est révélé à différentes personnes sous divers noms et aspects, en fonction des situations traversées montrant qu'il est celui qui remplit tout en tous et qu'il est et a tout ce dont l'homme a besoin. Ainsi, on le découvre dans les Ecritures comme étant grand, unique et indivisible, omniprésent, omniscient, souverain, incorruptible, sage, patient, saint, parfait, merveilleux, tout-puissant, fidèle, juste et bon. Bien évidemment, Dieu ne peut en aucun cas être défini dans tout ce qu'il est, dans la mesure où sa nature même échappe à toute possibilité de frontière ou de limite. Toutefois, les saints auront l'éternité pour découvrir ce Père incomparable. Voir \vref{Ge. 1,2}~; \vref{Ge. 17:1}~; \vref{Ex. 3:14}~; \vref{De. 6:14}~; \vref{De. 10:17}~; \vref{Es. 6:3}~; \vref{Mal. 3:6}~; \vref{Ps. 11:7}~; \vref{Ps. 139:7-10}~; \vref{La. 3:22-23}~; \vref{Lu. 1:49}~; \vref{Ja. 1:17}~; \vref{1 Th. 4:17}~; \vref{1 Co. 1:9}~; \vref{Ro. 1:23}~; \vref{Ro. 2:4}~; \vref{Ro. 11:33-36}~; \vref{2 Ti. 4:8}~; \vref{Hé. 4:13} et \vref{1 Jn. 4:8}.

\DicoEntry{DÎME}\textit{, de l'hébreu «~ma'aser~»~: «~dîme, dixième partie~»}\newline
Abraham donna à Melchisédek la dîme du butin d'une bataille remportée (\vref{Ge. 14:17-20} et \vref{Hé. 7:1-2}). Yahweh instaura, au travers de Moïse, la dîme comme une loi à respecter par les enfants d'Israël. Il en existait quatre sortes~:
\\1. la dîme que les Lévites prélevaient sur le peuple (\vref{No. 18:21-24})
\\2. la dîme de la dîme, que les prêtres prélevaient sur les Lévites (\vref{No. 18:25-31}~; \vref{Né. 10:38})
\\3. la dîme consommée par les Juifs eux-mêmes lors des fêtes de Yahweh (\vref{De. 14:22-26})
\\4. la dîme pour l'étranger, la veuve, l'orphelin et le Lévite, donnée tous les trois ans (\vref{De. 14:28-29}).
\\Cette loi concernait exclusivement Israël et non l'Eglise – Jésus-Christ ayant accompli la loi (\vref{Mt. 5:17}). Sous la grâce, les chrétiens sont invités à faire des offrandes * librement et sans contrainte.

\DicoEntry{DINA}\textit{, de l'hébreu «~Diynah~»~: «~jugement, justice~»}\newline
Fille de Jacob et Léa. Elle fut enlevée et déshonorée par Sichem, fils de Hamor, prince du pays de Canaan. Sichem et tous les hommes de la ville furent ensuite tués par les frères de la jeune fille, Siméon et Lévi. Voir \vref{Ge. 34}.

\DicoEntry{DIOTRÈPHE}\textit{, du grec «~Diotrephes~»~: «~nourri par Zeus~»}\newline
Chrétien dont Jean dénonça l'arrogance et les mauvais agissements. Voir \vref{3 Jn. 1:9-11}.

\DicoEntry{DISCIPLE}\textit{, du grec «~mathetes~»~: «~un étudiant, un élève, un disciple~»}\newline
Personne qui écoute les enseignements de son maître et les met en pratique en vue de devenir comme lui. Jésus en choisit douze qu'il forma pendant son service. Le disciple de Christ doit manifester le caractère de son maître, lui être pleinement consacré et être prêt à souffrir en son nom. Voir \vref{Mt. 10}~; \vref{Lu. 6:12-16}~; \vref{Lu. 14:26-33}.

\DicoEntry{DIVORCE}\textit{, du grec «~apostasion~»~: «~divorce, répudiation, lettre de divorce~»}\newline
Brisement des liens du mariage*. Il fut autorisé sous la loi de Moïse à cause de la dureté des cœurs, mais Christ rappela l'indissolubilité du mariage au commencement. Voir \vref{De. 24:1-3} et \vref{Mt. 19:3-8}.

\DicoEntry{DOCTEUR}\textit{, du grec «~didaskalos~»~: «~professeur~», «~maître~».}\newline
Sous la loi de Moïse, les docteurs de la loi étaient chargés d'expliquer la Torah. Certains d'entre eux s'opposèrent à Jésus. Sous la Nouvelle Alliance, le docteur est un des cinq services liés à la Parole évoqués en \vref{Ep. 4:11}. Il enseigne la Parole de Dieu qui guérit les blessures de l'âme. Selon \vref{Ja. 3:1}, nous ne sommes pas tous appelés à être des docteurs. Voir \vref{Lu. 2:46}~; \vref{Lu. 5:17}~; \vref{1 Co. 12:28} et \vref{Ep. 4:11}.

\DicoEntry{DONS SPIRITUELS}\textit{, du grec «~charisma~»~: «~faveur que reçoit quelqu'un sans aucun mérite de sa part~», «~dons provenant du pouvoir de la grâce divine~»}\newline
Capacités distribuées par le Saint-Esprit aux chrétiens en vue de la formation et de l'édification des saints. Voir \vref{1 Co. 12:1-11}~; \vref{1 Co. 14:12}~; \vref{Ro. 12:6} et \vref{1 Pi. 4:10}.

\DicoEntry{EDEN}\textit{, de l'hébreu «~'Eden~»~: «~plaisir, délices~»}\newline
Appelé aussi jardin de Dieu, premier lieu de résidence d'Adam et Eve. Yahweh y avait fait pousser des arbres de toutes espèces~; il y avait également placé au milieu l'arbre de vie* ainsi que l'arbre de la connaissance du bien et du mal* dont la consommation des fruits conduirait à la mort. L'homme fut établi en tant que gardien et cultivateur de ce jardin. Cependant, il pêcha avec la femme et ils furent chassés de ce lieu des délices. Voir \vref{Ge. 2}~; \vref{Ge. 3:23-24} et \vref{Ez. 28:13}.

\DicoEntry{ÉGLISE}\textit{, du grec «~ekklesia~»~: «~appel hors de~»}\newline
Peuple mis à part dont Christ est le chef. L'Eglise est la sainte habitation de Dieu en esprit, le corps de Christ, l'épouse de l'Agneau. On distingue l'Eglise universelle - qui regroupe tous les saints du monde entier - de l'église locale - qui est composée de tous les chrétiens d'une ville. Voir \vref{Ac. 2:47}~; \vref{1 Th. 1:1}~; \vref{1 Co. 1:2}~; \vref{1 Co. 3:16}~; \vref{1 Co. 12:27}~; \vref{Ep. 2:20-22}~; \vref{Ep. 5:22-32}~; \vref{Ph. 1:1} et \vref{1 Ti. 2:4}.

\DicoEntry{ÉLÉAZAR}\textit{, de l'hébreu «~'El'azar~»~: «~Dieu a secouru~»}\newline
Fils d'Aaron, il était chef des chefs des Lévites avant de devenir le second grand prêtre d'Israël. Voir \vref{No. 3:32} et \vref{No. 20:25-28}.

\DicoEntry{ÉLECTION, ELU}\textit{, de l'hébreu «~bachiyr~» et du grec «~eklektos~»~: «~choisi, élu de Dieu~»}\newline
Dans le Tanakh, Israël fut présenté comme le peuple élu de Yahweh, appelé à être un exemple pour toutes les nations de la terre. Cette élection n'est pas synonyme de préférence, car la volonté de Dieu est de sauver tous les hommes. Au travers de l'œuvre de la croix, Dieu a effectivement montré que son choix se porte vers l'humanité tout entière en payant le prix des péchés de tous. Dans son omniscience, il sait toutefois d'avance qui croira en lui ou pas. Selon la parole, même après la conversion, le chrétien doit travailler son élection, c'est-à-dire se sanctifier et obéir aux commandements de Yahweh pour entrer dans son royaume. Voir \vref{Es. 45:4}~; \vref{Es. 49:6}~; \vref{Mt. 22:14}~; \vref{Ep. 1:4-6} et \vref{2 Pi. 1:10-11}.

\DicoEntry{ÉLIE}\textit{, de l'hébreu «~Eliyah~»~: «~Yahweh est mon Dieu~»}\newline
Prophète d'origine tschibite que Dieu suscita en Israël au temps du roi Achab*. Il ne connut point la mort, mais fut enlevé par le Seigneur. Son histoire, ses combats et ses exploits sont racontés dans les livres des Rois.

\DicoEntry{ÉLISÉE}\textit{, de l'hébreu «~'Eliysha'~»~: «~Dieu est sauveur~»}\newline
Prophète du royaume d'Israël, il succéda à Elie après avoir reçu la double portion de l'esprit qui était sur ce dernier. Après sa mort, ses os rendirent la vie à un défunt. Voir \vref{1 R. 19:16-21}~; \vref{2 R. 2:9-11} et \vref{2 R. 13:20-21}.

\DicoEntry{ENFER}\textit{}\newline
Voir SEJOUR DES MORTS.

\DicoEntry{ENLÈVEMENT}\textit{, du grec «~metathesis~»~: «~transfert d'un lieu à un autre, changement~»}\newline
Ravissement d'hommes au ciel sans que ces derniers ne connaissent la mort*. Dans le Tanakh, se trouvent deux cas d'enlèvement~: Hénoc (\vref{Ge. 5:24}~; \vref{Hé. 11:5}) et Elie (\vref{2 R. 2:11}). L'Eglise sera de même enlevée par le Seigneur au son de la dernière trompette. Voir \vref{1 Th. 4:17} et \vref{1 Co. 15:51-57}.

\DicoEntry{ÉPHÈSE}\textit{, du grec «~Ephesos~»~: «~permis~»}\newline
Une des principales villes de l'empire romain sous le règne de l'empereur Claude 1er (10 av. J.-C. – 54 ap. J.-C.) Ephèse possédait le plus grand port de l'Asie Mineure, ce qui lui attribuait le contrôle du trafic commercial. Richissime et prospère, elle était renommée pour son faste, sa liberté de parole et constituait donc un endroit privilégié pour les philosophes. L'église d'Ephèse naquit du ministère de Paul, qui y enseigna pendant au moins deux ans lors de son troisième voyage missionnaire. Cette église - figurant parmi les sept du livre d'Apocalypse - fit preuve de discernement et pratiquait de bonnes œuvres, mais le Seigneur avait néanmoins un reproche à lui adresser. Elle représente l'église apostate. Voir Epître aux Ephésiens et \vref{Ap. 2:1-7}.

\DicoEntry{ÉPHOD}\textit{, de l'hébreu «~ephowd~»~: «~couverture~»}\newline
Vêtement que les prêtres portaient par-dessus leur tunique lorsqu'ils étaient en service. L'éphod du grand prêtre était de broderie~; le pectoral était posé sur son devant. Voir \vref{Ex. 28} et \vref{Lé. 8:7}.

\DicoEntry{ÉPHRAIM}\textit{, de l'hébreu «~'Ephrayim~»: «~double fertilité~»}\newline
Second fils de Joseph né en Egypte, il fut adopté par Jacob avant sa mort et devint ainsi l'ancêtre d'une des douze tribus d'Israël. Voir \vref{Ge. 41:52}~; \vref{Ge. 48:5} et \vref{Jos. 14:4}.

\DicoEntry{ÉPICURIENS}\textit{, du grec «~epikoureios~»~: «~celui qui aide, le défenseur~»}\newline
Fondé à Athènes en 306 av. J.-C., groupe de philosophes se réclamant de la doctrine d'Epicure (341 av. J.-C. à 270 av. J.-C.). Ce dernier fonda une des plus importantes écoles philosophiques de l'Antiquité. Il développa une théorie athée selon laquelle l'homme est encouragé à rechercher les plaisirs matériels et sensuels. Rejetant la pensée d'une vie après la mort, les épicuriens renient l'existence d'un créateur qui se préoccuperait des hommes. Des adeptes de cette philosophie se confrontèrent à la doctrine de Christ annoncée par Paul et cherchèrent à l'entendre. Voir \vref{Ac. 17:18-20}.

\DicoEntry{ÉSAÏE}\textit{, de l'hébreu «~Yesha'yah~»~: «~Yahweh a sauvé~»}\newline
Fils d'Amotz, un prophète de Yahweh contemporain des rois Ozias, Jotham, Achaz et Ezéchias, il annonça la venue du Messie. L'ensemble de ses prophéties est contenu dans le livre portant son nom.

\DicoEntry{ÉSAÜ}\textit{, de l'hébreu «~'Esav~»~: «~velu, poilu, chevelu~»}\newline
Fils d'Isaac et Rebecca et frère jumeau de Jacob, qui lui soutira son droit d'aînesse et sa bénédiction. Il prit pour femmes Judith et Basmath, toutes deux originaires de Canaan. Egalement connu sous le nom d'Edom, il devint l'ancêtre des Edomites. Voir \vref{Ge. 25:25-34}~; \vref{Ge. 27} et \vref{Ge. 36}.

\DicoEntry{ESDRAS}\textit{, de l'hébreu «~`Ezra'~»~: «~secours~»}\newline
Fils de Sereja et descendant du grand prêtre Aaron, Esdras était scribe et prêtre. Il enseigna le peuple de Dieu dans la loi et mit en place des réformes après la reconstruction du temple. Son histoire se trouve dans le livre éponyme.

\DicoEntry{ESPRIT}\textit{, de l'hébreu «~ruwach~»~: «~vent, souffle, esprit~» et du grec «~pneuma~»~: «~vérité, inspiration, souffle, vent~»}\newline
L'esprit humain est aussi appelé homme intérieur, il constitue la partie spirituelle de l'homme lui permettant d'agir, de prendre des décisions et d'être en contact avec Dieu ou tout autre esprit. Principe vital, il amène l'âme à la vie. Avec l'âme* et le corps, l'esprit constitue l'être humain. Voir \vref{Ge. 6:3}~; \vref{Ex. 31:3}~; \vref{Job 27:3}~; \vref{Job 32:8}~; \vref{Mt. 12:28} et \vref{1 Th. 5:23}.

\DicoEntry{ESPRIT IMPUR}\textit{}\newline
Voir DEMONS.

\DicoEntry{ESTHER}\textit{, dérivation du perse «~'Ecter~»~: «~étoile~»}\newline
Cousine de Mardochée, Juif d'origine benjaminite, reine de Perse, épouse du roi Assuérus. Son nom juif était Hadassa~: «~myrte~». Son histoire, qui se déroula à Suse, est racontée dans le livre portant son nom.

\DicoEntry{ÉTANG DE FEU}\textit{}\newline
Lieu de douleur et de damnation éternelle créé initialement pour le diable et ses anges. Y seront jetés la bête et le faux prophète, le diable, la mort et le séjour des morts* ainsi que tous ceux dont le nom ne sera pas trouvé dans le livre de vie. Voir \vref{Mt. 25:41}~; \vref{Ap. 19:20} et \vref{Ap. 20:7-15}.

\DicoEntry{ÉTIENNE}\textit{, du grec «~stephanos~»~: «~couronne~»}\newline
Diacre* de l'église de Jérusalem rempli de sagesse et d'Esprit Saint. Premier martyr chrétien, sa mort marqua le début d'une grande persécution contre l'Eglise. Voir \vref{Ac. 6:1-6}~; \vref{Ac. 7} et \vref{Ac. 8:1-3}.

\DicoEntry{EUNUQUE}\textit{, du grec «~cariyc~»~: «~eunuque, chambellan castré~»}\newline
Homme dans l'incapacité de procréer ou émasculé. Dans l'Antiquité, les rois se choisissaient des eunuques pour les servir. En les castrant, ils s'assuraient de la fidélité et l'intégrité de ces derniers. En outre, Jésus distingua trois types d'eunuques. Voir \vref{2 R. 20:18}~; \vref{Da. 1:7}~; \vref{1 Ch. 28:1} et \vref{Mt. 19:12}).

\DicoEntry{ÉVANGELISTE}\textit{}\newline
Un des cinq ministères d'\vref{Ep. 4:11} dont la mission est de prêcher la repentance et la conversion à Jésus-Christ. Comme les ministères évoqués en \vref{Ep. 4:11}, il travaille également à la perfection des saints. Philippe exerça ce ministère, Timothée fut de même encouragé à faire l'œuvre d'un évangéliste. Tous les chrétiens doivent également évangéliser. Voir \vref{Ac. 21:8}~; \vref{Ep. 4:11} et \vref{2 Ti. 4:5}.

\DicoEntry{ÉVANGILE}\textit{}\newline
Enseignement donné par Jésus-Christ, la prédication de la croix (la mort et la résurrection de Jésus-Christ) et du Royaume de Dieu qui s'est approché des hommes (voir ROYAUME DE DIEU). Ce message annonce le salut, la guérison du cœur, la joie en Jésus-Christ, la justice, la paix, la grâce et la vie éternelle accordée à l'homme repentant, mais aussi le jugement à venir. Les apôtres propagèrent l'Evangile~; de même, tous les chrétiens sont appelés à le faire. Voir \vref{Es. 61}~; \vref{Mt. 10:7}~; \vref{Mt. 28:19-20}~; \vref{1 Co. 15:1-4}~; \vref{Ro. 1:16} et \vref{2 Ti. 4:1}.

\DicoEntry{EVE}\textit{, de l'hébreu «~Chavvah~»~: «~vie~»}\newline
Première femme et épouse d'Adam, elle fut formée à partir de la côte de son mari dans le but d'être l'aide de ce dernier. Séduite par Satan déguisé en serpent, elle mangea le fruit de la connaissance du bien et du mal et fut avec Adam chassée du jardin. Elle donna naissance à Caïn, Abel et Seth. Voir \vref{Ge. 2:18-24}~; \vref{Ge. 3:1-13} et \vref{Ge. 4:1-2,25}.

\DicoEntry{ÉVÊQUE}\textit{, du grec «~episcopos~»~: «~investigation, inspection, visite d'inspection~», «~acte par lequel Dieu visite les hommes, observe leurs voies, leurs caractères, pour leur accorder en partage joie ou tristesse~», «~surveillance, contrôle, fonction d'un ancien~», «~la charge d'une église chrétienne~».}\newline
Il est question ici d'une fonction consistant à visiter les assemblées, les inspecter afin de s'assurer du bon ordre. Voir \vref{Lu. 19:44}~; \vref{Ac. 1:20}~; \vref{1 Ti. 3:1}~; \vref{1 Pi. 2:12}.

\DicoEntry{EXPIATION}\textit{, de l'hébreu «~kaphar~»~: «~couvrir, purger, faire une expiation~»}\newline
Action de couvrir les fautes et les souillures de l'homme afin qu'il soit réconcilié avec Dieu. Sous l'Ancienne Alliance, le grand prêtre faisait tous les ans un sacrifice d'expiation en entrant dans le Saint des saints pour ses péchés et les péchés du peuple. Par son sacrifice, Christ est devenu la victime expiatoire pour les péchés de tous les hommes en les prenant sur lui à la croix~; il est l'Agneau de Dieu qui ôte les péchés du monde. Voir \vref{Lé. 16}~; \vref{Jn. 1:29}~; \vref{1 Jn. 2:2} et \vref{1 Jn. 4:10}.

\DicoEntry{ÉZÉCHIAS}\textit{, de l'hébreu «~Yechizqiyah~»~: «~Yahweh est ma force~»}\newline
Fils d'Osée, roi de Juda sur qui il régna vingt-neuf ans. Figurant parmi les rois les plus intègres, son règne fut caractérisé par la droiture et la fidélité à Yahweh. Voir \vref{2 R. 18-19}.

\DicoEntry{ÉZÉCHIEL}\textit{, de l'hébreu «~Yechezqe'l~»~: «~Dieu fortifie~»}\newline
Fils de Buri, prêtre et prophète de Yahweh ayant été déporté à Babylone. Il reçut de nombreuses visions - sur son temps et les temps de la fin - racontées dans le livre qui porte son nom.

\DicoEntry{FÉLIX}\textit{, du grec «~Phestos~»~: «~joyeux, en fête~»}\newline
Gouverneur de Judée de 52 à 60 ap. J.-C., il emprisonna Paul à la suite des plaintes des Juifs. S'entretenant avec lui de temps en temps et lui octroyant certaines libertés, Felix garda Paul en prison deux ans pour plaire aux Juifs. Voir \vref{Ac. 24}.

\DicoEntry{FESTUS}\textit{, du grec «~Phestos~»~: «~en fête, joyeux~»}\newline
Gouverneur de Judée qui succéda à Félix* de 60 à 62 ap. J.-C. Il poursuivit l'instruction du procès de Paul que les Juifs accusaient. Il permit à Paul de s'exprimer devant le roi Agrippa* et l'envoya à Rome afin qu'il comparaisse devant César*. Voir \vref{Ac. 24:27} et \vref{Ac. 25,26}.

\DicoEntry{FÊTES DE YAHWEH}\textit{}\newline
Selon la loi juive, sept fêtes étaient célébrées en l'honneur de Yahweh~: la Pâque de Yahweh, la fête des pains sans levain~; la fête des prémices~; la Pentecôte~; la fête des trompettes~; le jour des expiations et la fête des tabernacles. Voir \vref{Lé. 23:6-43}.

\DicoEntry{FIGUIER}\textit{}\newline
Arbre fruitier sous lequel il était coutume d'étudier la Torah en Israël. Ses fruits excellents et doux servaient en médecine. Le figuier est retrouvé dans de nombreuses histoires et paraboles des Ecritures. Il symbolise la douceur et l'humilité. Voir \vref{Jg. 9:11}~; \vref{2 R. 20:1-7}~; \vref{Lu. 13:6-9} et \vref{Jn. 1:43-51}.

\DicoEntry{FILS DE DIEU}\textit{}\newline
Expression désignant selon le contexte~:
\\1. les anges. Voir \vref{Ge. 6:2-4}~; \vref{Job 38:7} et \vref{Da. 3:25}.
\\2. Adam. Voir \vref{Lu. 3:38}.
\\3. les chrétiens. Voir \vref{Ga. 3:26} et \vref{Ro. 8:14}.
\\4. Jésus-Christ, le Fils unique de Dieu, en qui habite la plénitude de la divinité. Voir \vref{Mc. 15:39}~; \vref{Lu. 22:70}~; \vref{Jn. 1:14,34,49}~; \vref{Ro. 1:4}~; \vref{Col. 2:9} et \vref{1 Jn. 4:9,15}.

\DicoEntry{FILS DE L'HOMME}\textit{}\newline
Expression désignant un être humain, elle fut attribuée au prophète Ezéchiel près de cent fois. A de nombreuses reprises, Jésus-Christ se nomma lui-même «~Fils de l'homme~» afin de souligner sa nature humaine. Voir \vref{Ez. 2:1}~; \vref{Ez. 3:10}~; \vref{Ez. 4:1}~; \vref{Mc. 14:62}~; \vref{Jn. 5:27}~; \vref{Ro. 8:3} et \vref{Ph. 2:5-7}.

\DicoEntry{FIN DES TEMPS}\textit{, du grec «~eschatos~»~: «~extrême, dernier, fin~» et «~chronos~»~: «~temps, date, siècles~»}\newline
Appelée aussi derniers jours, période précédant la fin du monde*. Elle a commencé à l'effusion du Saint-Esprit selon la prophétie de Joël. La fin des temps est caractérisée d'un côté par des manifestations extraordinaires de l'Esprit de Dieu et l'annonce de l'Evangile à tous les peuples~; de l'autre par la séduction, l'apostasie* et le péché dans des dimensions jamais atteintes auparavant. Voir \vref{Joë. 2:28-29}~; \vref{Mt. 24:3-14}~; \vref{Ac. 2:16-18}~; \vref{1 Ti. 4:1} et \vref{2 Ti. 3:1-5}.

\DicoEntry{FIN DU MONDE}\textit{, du grec «~eschatos~»~: «~extrême, dernier, fin~» et «~aion~»~: «~monde, univers, période de temps~»}\newline
Cet événement correspond à la fin de notre ère. Après le jugement dernier, les impies iront dans l'étang de feu*, tandis que la Nouvelle Jérusalem accueillera les saints~; la terre sera détruite. Voir \vref{Mt. 13:36-43}~; \vref{2 Pi. 3:10-13}~; \vref{Ap. 20:11-15} et \vref{Ap. 21}.

\DicoEntry{FOI}\textit{, du grec «~pistis~»~: «~conviction de la vérité~»}\newline
Confiance en la véracité de Dieu, ses paroles et l'accomplissement de ses promesses. Bien qu'il n'existe qu'une seule foi, elle est présentée sous trois formes principales sous la Nouvelle Alliance~:
\\1. en tant que fruit de l'esprit*, c'est la foi qui sauve (\vref{Ga. 5:22} et \vref{Ro. 10:9})
\\2. en tant que don de l'Esprit,* c'est la foi accordée pour accomplir une tâche particulière (\vref{1 Co. 12:9})
\\3. en tant que Parole, c'est la foi liée à la saine doctrine, la vérité (\vref{Ro. 10:17} et \vref{2 Ti. 4:7})
\\Condition essentielle pour être agréable à Dieu~; la foi est éprouvée tout au long de la vie du croyant. Voir \vref{Lu. 7:50}~; \vref{Hé. 11} et \vref{1 Pi. 1:7}.

\DicoEntry{FORNICATION ou IMPUDICITÉ}\textit{, du grec «~pœrneia~»~: «~relation sexuelle illicite~»}\newline
Tous les rapports sexuels condamnés par la Parole, voir \vref{Lé. 18}~; \vref{1 Co. 6:13,16-18} et \vref{1 Co. 7:2}.

\DicoEntry{FRUIT DE L'ESPRIT}\textit{}\newline
Résultat de l'action de l'Esprit Saint dans l'homme intérieur dans le but de communiquer le caractère de Yahweh au chrétien né d'en haut. Voir \vref{Ga. 5:22}.

\DicoEntry{GABRIEL}\textit{, de l'hébreu «~Gabriy'el~»~: «~héros de Dieu~» ou «~homme de Dieu~»}\newline
Archange* que Dieu envoya pour délivrer des messages, notamment à Daniel, Zacharie et Marie. Voir \vref{Da. 9:21-27}~; \vref{Lu. 1:11-20} et \vref{Lu. 1:26-38}.

\DicoEntry{GAD}\textit{, de l'hébreu «~Gad~»~: «~bonheur~», «~heureux~», «~troupe~»}\newline
Fils de Jacob et Zilpa, servante de Léa, il devint l'ancêtre de la tribu de Gad. Voir \vref{Ge. 30:11} et \vref{Ge. 49:16}.

\DicoEntry{GALATIE}\textit{, du grec «~Galatia~»~: «~territoire des Gaulois, Gaule~»}\newline
Province antique de l'Asie Mineure, la Galatie se situait en Anatolie, dans l'actuelle Turquie autour d'Ankara. Elle devait son nom aux Galates, Celtes provenant des Balkans. Lors de son premier voyage missionnaire, Paul avait traversé cette région où plusieurs assemblées émergèrent. Il y revint plus tard pour fortifier les disciples et leur écrivit une lettre suite au trouble apporté par les judaïsants. Voir \vref{Ac. 16:6}~; \vref{Ac. 18:23} et \vref{Ga. 1-5}.

\DicoEntry{GALILÉE}\textit{, de l'hébreu «~Galiyl~»~: «~cercle, région, district~»}\newline
Région située au nord de la Palestine dans laquelle se trouve la localité de Nazareth où Jésus grandit. Il y commença son ministère, c'est aussi là qu'il se montra vivant à ses disciples après sa résurrection. Les disciples de Jésus étaient originaires de Galilée. Voir \vref{Mt. 2:19-23}~; \vref{Mc. 16:7}~; \vref{Jn. 2}~; \vref{Ac. 1:11} et \vref{Ac. 2:7}.

\DicoEntry{GARIZIM}\textit{, de l'hébreu «~Geriziym~»~: «~lieux arides~»}\newline
Montagne située au sud de Sichem, en face du mont Ebal, de laquelle les enfants d'Israël devaient prononcer la bénédiction* une fois entrés en Canaan. Voir \vref{De. 11:29}~; \vref{Jg. 9:7} et \vref{Jos. 8:33}.

\DicoEntry{GÉDÉON}\textit{, de l'hébreu «~Gid'own~»~: «~coupant, abattant~»}\newline
Issu de la tribu de Manassé et fils de Joas. Il fut mandaté pour délivrer Israël de la main des Madianites et fut juge en Israël pendant quarante ans. Voir \vref{Jg. 6-8}.

\DicoEntry{GÉHENNE}\textit{, du grec «~geena~»~: «~vallée de Hinnom~»}\newline
Initialement, vallée située au sud de Jérusalem où des enfants étaient jetés dans le feu en sacrifice à Moloc. Le terme «~géhenne~» représente la destruction future des méchants et se rapporte à l'étang de feu*. Voir \vref{2 R. 23:10} et \vref{Mt. 10:28}.

\DicoEntry{GENTILS}\textit{, du grec «~ethnos~»~: «~nations~», «~peuples~»}\newline
Dans les Ecritures, ce terme se rapportait initialement à tous ceux n'appartenant pas au peuple juif. Paul fut mandaté pour évangéliser les Gentils. A partir du IIIème siècle, le terme «~païen~» fut introduit dans le jargon chrétien pour désigner le «~non-chrétien~». Voir \vref{Mt. 18:17} et \vref{Ac. 26:17}.

\DicoEntry{GERME}\textit{, de l'hébreu «~tsemach~»~: «~pousse, croissance, branche~»}\newline
Terme désignant le Messie dans certains écrits prophétiques. Voir \vref{Es. 4:2}~; \vref{Jé. 23:5} et \vref{Za. 3:8}.

\DicoEntry{GLOIRE}\textit{, de l'hébreu «~kabhod~»~: «~poids~» ou «~kabowd~»~: «~gloire, honneur, richesse~»}\newline
La gloire se rapporte à ce qui a du poids, ce qui est lourd et écrasant – il est en effet difficile pour l'homme de supporter la splendeur et la magnificence de Yahweh. Image de sa sainteté, elle s'est manifestée dans un feu dévorant sur le mont Sinaï et fut révélée à Moïse au travers de la bonté et du nom de Dieu. Cette gloire sanctifie et génère de grands miracles~; elle est racontée par les cieux et toute la création. La gloire de Yahweh sera le luminaire de la Nouvelle Jérusalem. Elle invite à la crainte, la révérence, l'humilité, la louange~; lui seul mérite la gloire. Voir \vref{Ex. 16:10}~; \vref{Ex. 24:17}~; \vref{Ex. 29:43}~; \vref{Ex. 33:18-23}~; \vref{Es. 42:8}~; \vref{Es. 48:11}~; \vref{Ez. 44:4}~; \vref{Ps. 19:1}~; \vref{Pr. 15:33}~; \vref{2 Ch. 5:14}~; \vref{1 Th. 2:12} et \vref{Ap. 21:23}.

\DicoEntry{GOG}\textit{, de l'hébreu «~Gowg~»~: «~montagne~»}\newline
Très certainement le chef du pays de Magog. Voir \vref{Ez. 38} et \vref{Ap. 20:8}.

\DicoEntry{GOLGOTHA}\textit{, de l'araméen «~gulgoleth~»~: «~tête, crâne~»}\newline
Lieu de la crucifixion de Jésus-Christ, situé non loin de Jérusalem. Voir \vref{Jn. 19:17-20}.

\DicoEntry{GOMORRHE}\textit{, de l'hébreu «~Amorah~»~: «~submersion~»}\newline
Ville située dans la plaine du Jourdain. Après avoir atteint un haut degré de perversion et de débauche, elle fut détruite par Yahweh avec sa ville voisine, Sodome. Voir \vref{Ge. 13:10}~; \vref{Ge. 18:20-21} et \vref{Ge. 19:24}.

\DicoEntry{GRÂCE}\textit{, du grec «~charis~»~: «~bonne volonté~», «~bonté~», «~faveur~»}\newline
Don immérité de Dieu, elle est la source du salut* de tous les hommes et invite à la crainte de Dieu. La grâce est venue par Jésus-Christ et fut révélée au travers de l'œuvre parfaite de la croix*. Voir \vref{Jn. 1:17}~; \vref{Ro. 3:23-24}~; \vref{Tit. 2:11-12}.

\DicoEntry{GRAND PRÊTRE}\textit{}\newline
Voir PREMIER PRÊTRE.

\DicoEntry{GRANDE TRIBULATION}\textit{}\newline
Voir commentaire \vref{Ap. 7:14}.

\DicoEntry{GUILGAL}\textit{, de l'hébreu «~Gilgal~»~: «~action de rouler~»}\newline
Territoire situé à l'ouest du Jourdain et à l'est de Jéricho~; il fut le lieu de campement des Israélites après avoir passé le Jourdain à sec. Voir \vref{Jos. 4-5}.

\DicoEntry{HABAKUK}\textit{, de l'hébreu «~Chabaqquwq~»~: «~embrasser~», «~amour~»}\newline
Prophète de Yahweh qui exerça son ministère dans le royaume de Juda. L'ensemble de ses prophéties se trouve dans le livre éponyme.

\DicoEntry{HARMAGUEDON}\textit{, de l'hébreu «~Armageddon~»~: «~montagne de Méguiddo~»}\newline
Lieu situé au nord d'Israël dans la tribu de Zabulon. A la fin des temps, les rois et puissants de la terre s'y rassembleront pour combattre Yahweh et son armée. Voir \vref{2 R. 23:29} et \vref{Ap. 16:13-16}.

\DicoEntry{HÉBREU}\textit{, de l'hébreu «~`Ibriy~»~: «~qui provient de l'autre côté, qui traverse~»}\newline
Terme désignant les descendants d'Héber, fils de Schélach, de la postérité de Sem, dont est issu Abraham. Voir \vref{Ge. 11:10-32}~; \vref{Ge. 14:13-14} et \vref{Ex. 1:15-22}.

\DicoEntry{HELLÉNISTE}\textit{, du grec «~hellenistes~»~: «~celui qui adopte les manières et coutumes des Grecs~»}\newline
Israélites nés hors de la terre promise ayant adopté le mode de vie grec et parlant la langue grecque. Voir \vref{Ac. 6:1}.

\DicoEntry{HÉNOC}\textit{, de l'hébreu «~Chanowk~»~: «~consacré, dédié~»}\newline
Fils de Jéred et père de Metuschéla. Homme pieux ayant vécu trois cent soixante-cinq ans avant d'être enlevé au ciel sans connaître la mort. Voir \vref{Ge. 5:21-24} et \vref{Hé. 11:5}.

\DicoEntry{HÉRODE LE GRAND}\textit{, (73 av. J.-C.à 4 av. J.-C), du grec «~Herodes~»: «~héroïque~»}\newline
Roi de Judée, il fut l'instigateur du massacre des enfants de la région de Bethléhem au moment de la naissance de Jésus. Il mourut quand Jésus était encore enfant. Voir \vref{Mt. 2}.

\DicoEntry{HÉRODE ANTIPAS}\textit{, (ou le Tétrarque) (\ 21 av. J.-C. à 39 ap. J.-C.)}\newline
Fils d'Hérode le Grand*, il exerça la fonction de tétrarque* de Galilée et fut contemporain à Jésus-Christ pendant presque toute la vie de ce dernier. Hérode épousa sa belle-sœur Hérodias* et fit décapiter Jean-Baptiste. Il fut qualifié de «~renard~» par Jésus et s'accorda avec son ennemi Pilate lors de la crucifixion du Seigneur. Voir \vref{Mc. 6:14-28}~; \vref{Lu. 3:1}~; \vref{Lu. 13:31-32} et \vref{Lu. 23:8-12}.

\DicoEntry{HÉRODE AGRIPPA Ier}\textit{, (\ 10 av. J.-C. à 44 ap. J.-C)}\newline
Roi et tétrarque de Judée et petit fils du roi Hérode le Grand, il accéda au pouvoir à la genèse de l'Eglise primitive. Pour plaire aux Juifs, il fit mourir Jacques, fils de Zébédée, et emprisonna Pierre. Il mourut brusquement après avoir reçu du peuple la gloire qui devait revenir à Dieu. Voir \vref{Ac. 12}.

\DicoEntry{HÉRODE AGRIPPA II}\textit{, (\ 27 ap. J.-C. à 93 ap. J.-C.)}\newline
Fils d'Agrippa Ier, il est appelé «~roi Agrippa~» dans les Ecritures. Il fut inspecteur du temple de Jérusalem et avait le pouvoir de choisir les grands prêtres. Il rencontra Paul à Césarée lors d'une visite au gouverneur Festus*. Voir \vref{Ac. 25-26}.

\DicoEntry{HÉRODIAS}\textit{, du grec «~Herodias~»~: «~héroïque~»}\newline
Femme de Philippe I puis de son frère, Hérode le tétrarque. Elle commanda la décapitation de Jean-Baptiste. Voir \vref{Mc. 6:17-28}.

\DicoEntry{HOLOCAUSTE}\textit{, de l'hébreu «~'olah~»~: «~offrande entièrement consumée~»}\newline
Prescrit par la loi de Moïse, sacrifice consumé par le feu d'une agréable odeur à Yahweh. Il préfigurait le sacrifice à la croix de Jésus-Christ, l'Agneau de Dieu. Voir \vref{Lé. 1:1-17}~; \vref{Hé. 9:11-22} et \vref{Hé. 10:1-19}.

\DicoEntry{HOMOSEXUALITÉ}\textit{}\newline
Pratique abominable et fermement réprouvée par Dieu consistant en l'union de deux personnes du même sexe. Voir \vref{Lé. 18}~; \vref{1 Co. 6:9-10} et \vref{Ro. 1:24-32}.

\DicoEntry{HOSANNA}\textit{, de l'hébreu «~yasha'~»~: «~sauve~» et «~na´~»~: «~je te prie, maintenant~» et du grec «~hosanna~»~: «~sauve maintenant~!~»}\newline
Cri par lequel Jésus fut accueilli par la foule quand il entra à Jérusalem. Voir \vref{Mt. 21:9,15}~; \vref{Mc. 11:9-10} et \vref{Jn. 12:13}.

\DicoEntry{HULDA}\textit{, de l'hébreu «~chuldah~»~: «~belette, taupe~»}\newline
Femme de Schallum, prophétesse habitant à Jérusalem du temps de Josias, roi de Juda. Le roi chercha à consulter Yahweh au travers d'elle quand il découvrit le livre de la loi et les malheurs qui devaient suivre la désobéissance d'Israël. Voir \vref{2 R. 22:14-20} et \vref{2 Ch. 34:21-33}.

\DicoEntry{HYSOPE}\textit{, du grec «~hussopos~»~: «~hysope, branche d'hysope~»}\newline
Plante aromatique utilisée pour faire l'aspersion du sang ou d'eau sous l'Ancienne Alliance. C'est à l'aide d'une branche d'hysope qu'on présenta à Jésus une éponge trempée de vinaigre lors de sa crucifixion. Voir \vref{Ex. 12:22}~; \vref{Lé. 14:1-7}~; \vref{No. 19:18-19}~; \vref{Jn. 19:29} et \vref{Hé. 9:19}.

\DicoEntry{IDOLE, IDOLÂTRIE}\textit{, de l'hébreu «~gilluwl~»~: «~image~» et du grec «~eidolon~»~: «~image pour adorer~»}\newline
Une idole peut être l'image d'un faux dieu, l'image faussée de Yawheh ou encore une personne, un objet, une activité à qui l'on donne le rang de Dieu. L'idolâtrie - culte rendu à ces idoles – est fermement réprouvée dans la Parole. Voir \vref{Ex. 20:3-5}~; \vref{Ex. 32}~; \vref{1 R. 15:11-13}~; \vref{1 Co. 6:9}~; \vref{Ep. 5:5} et \vref{Col. 3:5}.

\DicoEntry{IMPOSITION DES MAINS}\textit{}\newline
Avant leur mort, les patriarches imposaient les mains à leurs enfants pour les bénir (\vref{Ge. 48:14}). Moïse imposa également les mains à Josué qui devait lui succéder (\vref{De. 34:9}). Sous la Nouvelle Alliance, on peut imposer les mains à quelqu'un en vue de lui transmettre la guérison divine, l'autorité liée à une fonction particulière, les dons spirituels et même le Saint-Esprit dans certains cas. Ce geste ne doit cependant pas être fait dans la précipitation. Voir \vref{Lu. 4:40}~; \vref{Mc. 16:18}~; \vref{Ac. 6:6}~; \vref{Ac. 8:17}~; \vref{1 Ti. 4:14} et \vref{1 Ti. 5:22}.

\DicoEntry{INCORRUPTIBILITÉ}\textit{, du grec «~aphtharsia~»~: «~perpétuité, pureté, sincérité~»}\newline
Terme désignant ce qui ne peut ni se corrompre, ni se flétrir, ni se détruire. A l'enlèvement de l'Eglise, les morts en Christ ressusciteront incorruptibles et les chrétiens revêtiront de même des corps incorruptibles. Voir \vref{Mt. 24:35} et \vref{1 Co. 15:40-57}.

\DicoEntry{INCREDULITÉ}\textit{, du grec «~apistia~»~: «~infidélité, sans foi, faiblesse dans la foi~»}\newline
Rejet, doute par rapport à la véracité de Dieu et de sa parole. Thomas fit preuve d'incrédulité quant à la résurrection de Christ avant de le voir vivant. Les incrédules ne peuvent pas hériter le Royaume de Dieu. Voir \vref{Jn. 1:1-14}~; \vref{Jn. 14:6}~; \vref{Jn. 20:24-29} et \vref{Ap. 21:8}.

\DicoEntry{INIQUITÉ}\textit{, du grec «~adikia~»~: «~injustice, tortuosité d'un cœur, violation volontaire de la loi~»}\newline
Tout ce qui constitue une violation de la loi* et la justice de Dieu. Voir \vref{Ro. 6:13}~; \vref{2 Pi. 2:13} et \vref{1 Jn. 5:17}.

\DicoEntry{MYSTEÈRE DE L'INIQUITÉ}\textit{}\newline
Voir commentaire \vref{2 Th. 2:7}.

\DicoEntry{INTERCESSION}\textit{, de l'hébreu «~palal~»~: «~intervenir, s'interposer, prier, agir en médiateur~»}\newline
Sous l'Ancienne Alliance, le grand prêtre avait la mission d'intercéder pour les péchés du peuple en offrant des sacrifices. A présent, Jésus-Christ le grand prêtre à perpétuité et l'avocat intercède pour ses enfants après s'être offert en sacrifice pour les péchés de l'humanité. Les hommes peuvent aussi faire des prières d'intercession comme Abraham pour Lot, Moïse pour Marie et l'Eglise pour tous les hommes. Voir \vref{Ge. 18:16-33}~; \vref{Lé. 16}~; \vref{No. 12:10-15}~; \vref{1 Ti. 2:1}~; \vref{Hé. 9:11-15} et \vref{1 Jn. 2:1}.

\DicoEntry{ISAAC}\textit{, de l'hébreu «~Yitschaq~»~: «~il rit~»}\newline
Fils de la promesse qui naquit à Abraham et Sara dans leur vieillesse. Il fut épargné quand Yahweh demanda à Abraham de lui offrir son fils en sacrifice. Isaac épousa Rébecca avec qui il eut deux fils~: Esaü et Jacob. Voir \vref{Ge. 17:17-21}~; \vref{Ge. 22:1-13} et \vref{Ge. 25:19-26}.

\DicoEntry{ISAÏ}\textit{, de l'hébreu «~Yishay~»~: «~je possède~»}\newline
Bethléhémite, petit-fils de Boaz et de Ruth, fils d'Obed et père de David. Voir \vref{Ru. 4:13-22}.

\DicoEntry{ISMAËL}\textit{, de l'hébreu «~Yishma'e'l~»~: «~Dieu entend~»}\newline
Fils d'Abraham et d'Agar, servante de Sara. Béni par Yahweh même après avoir été chassé avec sa mère par Sara, il devint le père des douze tribus ismaélites. Voir \vref{Ge. 16} et \vref{Ge. 25:12-16}.

\DicoEntry{ISSACAR}\textit{, de l'hébreu «~Yissaskar~»~: «~il donnera un salaire~»}\newline
Fils de Jacob et Léa, il devint l'ancêtre de la tribu d'Isaacar. Voir \vref{Ge. 30:18} et \vref{Ge. 49:14}.

\DicoEntry{ISRAËL}\textit{, de l'hébreu~: «~Yisra'el~»~: «~Dieu prévaut~»}\newline
Nom que Dieu donna à Jacob* après avoir lutté avec lui. Il s'agit également du nom désignant le peuple issu des douze fils de Jacob et le territoire que Dieu leur donna en héritage dont Jérusalem était la capitale. Après le schisme*, Israël se rapportait au royaume du nord composé de dix tribus. Voir \vref{Ge. 32:28}~; \vref{De. 33:5} et \vref{1 R. 12:1-24}.

\DicoEntry{IVRAIE}\textit{, du grec «~zizanion~»~: «~ivraie, ressemblant au blé, mais avec des grains noirs~»}\newline
Comme le blé, l'ivraie est une plante de la famille des graminées, mais c'est une mauvaise semence qui étouffe le blé. Elle représente les enfants du diable qui s'introduisent discrètement parmi les enfants de Dieu et qui en seront séparés uniquement à la fin du monde* pour aller vers la damnation éternelle. Voir \vref{Mt. 13:24-30,36-42}.

\DicoEntry{JACOB}\textit{, de l'hébreu «~Ya`aqob~»~: «~celui qui prend par le talon~» ou «~qui supplante~»}\newline
Fils d'Isaac et de Rebecca et frère jumeau d'Esaü. Il usa de stratagèmes pour ravir le droit d'aînesse ainsi que la bénédiction qui devaient revenir à son frère Esaü. Après avoir fui ce dernier, il se réfugia chez son oncle Laban dont il épousa les deux filles~: Léa et Rachel. De retour en Canaan après plusieurs années, Yahweh le rencontra en chemin et changea son nom en Israël. Jacob eut douze fils qui formèrent par la suite la nation d'Israël. Voir \vref{Ge. 25:21-34}~; \vref{Ge. 27-28}~; \vref{Ge. 29:1-30} et \vref{Ge. 49:1-28}.

\DicoEntry{JACQUES}\textit{, de l'hébreu~: «~Iakob~»~: «~qui supplante~» (variante de Jacob)}\newline
1. Fils de Zébédée et frère de Jean. Un des douze apôtres. Le roi Hérode Agrippa Ier* le fit mourir par l'épée. Voir \vref{Mt. 4:21-22}~; \vref{Lu. 6:12-16}~; \vref{Mc. 9:2-8}~; \vref{Mc. 14:32-33} et \vref{Ac. 12:1-2}.
\\2. Fils d'Alphée, un des douze apôtres~; il était aussi appelé Jacques le mineur. Voir \vref{Mt. 10:1-4}~; \vref{Mc. 15:40} et \vref{Lu. 6:12-16}.
\\3. Frère du Seigneur et apôtre, auteur de l'épître de Jacques. Voir \vref{Ac. 15:13-21}~; \vref{Ga. 1:19} et \vref{Mc. 6:3}.
\\4. Père de Jude, l'apôtre. Voir \vref{Lu. 6:16} et \vref{Ac. 1:13}.

\DicoEntry{JAPHET}\textit{, de l'hébreu «~Yepheth~»~: «~ouvert~», «~qui s'étend~»}\newline
Dernier des trois fils de Noé. Voir \vref{Ge. 10:1}.

\DicoEntry{JEAN}\textit{, de l'hébreu «~Yowchanan~» et du grec «~Ioannes~»~: «~Yahweh a fait grace~»}\newline
1. Fils de Zébédée, frère de Jacques et disciple aimé du Seigneur. Jean fut l'auteur de l'évangile éponyme, des trois épîtres qui portent son nom et de l'Apocalypse. Voir \vref{Mt. 10:2} et \vref{Jn. 13:23}.
\\2. Fils de Zacharie et Elisabeth, cousin de Jésus. Plus connu sous le nom de Jean-Baptiste, il fut envoyé pour préparer le chemin du Seigneur. Il fut décapité par Hérode Antipas*. Voir \vref{Lu. 1}~; \vref{Mal. 3:1-6}~; \vref{Mt. 1:12}~; \vref{Lu. 7:28} et \vref{Mt. 14:1-12}.

\DicoEntry{JELEK}\textit{, de l'hébreu «~yekeq~»~: «~jeune sauterelle~»}\newline
Désignant les sauterelles, il est souvent employé dans les Ecritures pour symboliser un grand nombre ou le dévoreur que Dieu envoie. Voir \vref{Joë. 1:4} et \vref{Na. 3:15-16}.

\DicoEntry{JÉRÉMIE}\textit{, de l'hébreu «~Yirmeyah~»~: «~celui que Yahweh a désigné~»}\newline
Fils de Hilkija, issu d'une famille de prêtres. Prophète de Yahweh, Jérémie fut appelé dès son plus jeune âge et exerça un ministère prophétique avant et pendant les premières années de déportation. Appelé à être eunuque*, il ne se maria jamais et n'eut point d'enfant. Il fut l'auteur des livres Jérémie et Lamentations de Jérémie.

\DicoEntry{JÉRICHO}\textit{, de l'hébreu «~Yeriychow~»~: «~ville de la lune~» ou «~ville des palmiers~»}\newline
Ville située à l'est de la tribu de Benjamin, près des rives du Jourdain. A la sortie du désert, les espions hébreux y furent cachés par Rahab la prostituée~; Jéricho fut ensuite détruite et livrée miraculeusement entre les mains d'Israël. C'est à Jéricho que Jésus guérit l'aveugle Bartimée et fut reçu par Zachée. Voir \vref{Jos. 2,6}~; \vref{Mc. 10:46-53} et \vref{Lu. 19:1-10}.

\DicoEntry{JÉROBOAM}\textit{, de l'hébreu «~Yarob'am~»~: «~le peuple devient nombreux~»}\newline
Fils de Nebath et serviteur de Salomon, il devint plus tard son ennemi. Après le schisme, il fut le premier roi du royaume du nord sur lequel il régna vingt-deux ans. Il fut une occasion de chute pour le peuple qu'il plongea dans l'idolâtrie*. Voir \vref{1 R. 11:26-40}~; \vref{1 R. 12-13}.

\DicoEntry{JÉRUSALEM}\textit{, de l'hébreu «~Yeruwshalem~»~: «~fondement de la paix~»}\newline
Ville située en Palestine, au nord de la Judée. Lors de la conquête de Canaan, la ville fut sous le contrôle des Jébusiens. Aux environs du Xe siècle av. J.-C., David reprit la ville alors devenue forteresse jébusienne. Il en fit la capitale politique et religieuse du royaume en y faisant établir l'arche de l'alliance. Salomon y construisit le temple* sur le mont Morija. En 586 av. J.-C., bien après le schisme*, les Babyloniens la détruisirent. Elle fut rebâtie par Néhémie après le retour de la captivité babylonienne. Jésus-Christ se lamenta sur la ville à cause de son incrédulité* et y annonça sa future destruction. Jérusalem fut en effet détruite par le général romain Titus en 70 ap. J.-C puis de nouveau rebâtie. Lors de son retour glorieux, le Seigneur Jésus posera ses pieds sur le Mont des Oliviers* qui est situé à Jérusalem. Le livre d'Apocalypse annonce après la fin du monde l'apparition de la Nouvelle Jérusalem, cité céleste. Voir \vref{2 S. 5:6-9}~; \vref{2 S. 6}~; \vref{Za. 14:1-4}~; \vref{2 Ch. 3:1}~; \vref{Lu. 19:41-44}~; et \vref{Ap. 21:2}.

\DicoEntry{JÉSUS}\textit{, de l'hébreu «~Yehowshuwa~»~: «~Yahweh est salut~»}\newline
Fils de l'homme* et fils de Dieu*, Jésus est le Dieu vivant manifesté en chair. Il fut conçu dans le ventre de Marie par la puissance du Saint-Esprit alors que cette dernière n'avait point connu d'homme. Selon les recherches de l'historien Flavius Josèphe, sa date de naissance se situerait autour de l'an 6 av. J.-C. - l'an zéro n'étant qu'une indication approximative. Fils adoptif de Joseph le charpentier et cousin de Jean Baptiste, il vécut la plus grande partie de sa vie en Galilée, dans la ville de Nazareth. Vers l'âge de 30 ans, il se fit baptiser dans le Jourdain et commença par la suite son ministère public. Grâce à sa vie exemplaire sans péché, il put se présenter comme une offrande agréable à Dieu répondant aux exigences de la justice divine pour sauver le monde. Remplissant toutes les prophéties relatives au Messie*, il fut trahi par un de ses disciples, Judas Iscariot*. Arrêté, maltraité puis crucifié, il mourut portant le poids des péchés de l'humanité, mais il ressuscita le troisième jour. Le salut réside dans la foi en son nom. Vivant de toute éternité, Jésus est le Dieu véritable et la vie éternelle. Sa justice ne tardera pas à se manifester~: il revient à toute vitesse. Voir \vref{Es. 53}~; \vref{Mt. 1:18-25}~; \vref{Mt. 2:23}~; \vref{Mc. 2:28}~; \vref{Lu. 1:36}~; \vref{Lu. 6:16}~; \vref{Lu. 24:46}~; \vref{Jn. 1:34}~; \vref{Ac. 4:12}~; \vref{2 Co. 5:21}~; \vref{1 Ti. 3:16}~; \vref{1 Pi. 2:21-25}~; \vref{1 Jn. 5:20} et \vref{Ap. 22:20}.

\DicoEntry{JETHRO}\textit{, de l'hébreu «~Yithrow~»~: «~son abondance, excellence~»}\newline
Prêtre de Madian chez qui Moise se réfugia après avoir fui l'Egypte. Il donna sa fille Séphora* pour femme à Moïse*. Voir \vref{Ex. 2:15-21}.

\DicoEntry{JEÛNE}\textit{, de l'hébreu «~tsuwn~»~: «~s'abstenir de nourriture~» et du grec «~nesteia~»~: «~le jeûne, un exercice volontaire et religieux~»}\newline
Privation totale ou partielle de nourriture dans le but d'humilier sa chair et d'adresser à Dieu des prières spécifiques. Le jeûne doit être exempt de toute hypocrisie et accompagné d'actes de justice pour être agréé de Dieu. Voir \vref{Es. 58}~; \vref{Est. 4:16}~; \vref{Da. 10}~; \vref{Mt. 6:16-18}~; \vref{Lu. 2:37}.

\DicoEntry{JÉZABEL}\textit{, de l'hébreu «~'Iyzebel~»~: «~Baal est l'époux~» ou «~l'impudique~»}\newline
Fille d'Ethbaal, roi de Sidon, et femme d'Achab roi d'Israël, elle extermina les prophètes de Yahweh et accueillait huit cent cinquante faux prophètes à sa table. Elle conduisit le peuple d'Israël dans l'idolâtrie au temps d'Elie*. Jézabel est associée à l'esprit du même nom qui prolifère de faux enseignements et entraîne le peuple de Dieu dans l'impudicité. Voir \vref{1 R. 16:31}, \vref{1 R. 18:4,19} et \vref{Ap. 2:20}.

\DicoEntry{JOB}\textit{, de l'hébreu «~'Iyowb~»~: «~haï, ennemi~» ou «~Je m'exclamerai~»}\newline
Originaire du pays d'Uts, homme prospère dont Yahweh témoigna l'intégrité et la droiture. Il subit en très peu de temps une succession de malheurs que Dieu permit pour se révéler à lui. Son histoire est racontée dans le livre portant son nom.

\DicoEntry{JOËL}\textit{, de l'hébreu «~Yow'el~»~: «~Yahweh est Dieu~»}\newline
Fils de Pethuel, il exerça la fonction de prophète dans le royaume de Juda. Il annonça la venue du Saint-Esprit sur toute chair à la fin des temps. Le contenu de son message se trouve dans le livre éponyme.

\DicoEntry{JONAS}\textit{, de l'hébreu «~Yonah~»~: «~colombe~»}\newline
Prophète de Yahweh envoyé à Ninive pour leur annoncer la destruction de leur ville. Son refus d'obéir à Dieu le conduisit dans le ventre d'un grand poisson. Son histoire est racontée dans le livre portant son nom.

\DicoEntry{JONATHAN}\textit{, de l'hébreu «~Yehownathan~»~: «~Yahweh a donné~»}\newline
Fils du roi Saül, homme de guerre reconnu. Lié à David par une très forte amitié, il protégea plusieurs fois ce dernier des relents meurtriers de son père. Il mourut à la bataille de Guilboa avec son père et ses frères. Voir \vref{1 S. 14:1-15}, \vref{1 S. 18:1-4}~; \vref{1 S. 19:1-8}~; \vref{1 S. 20} et \vref{1 S. 31:1-2}.

\DicoEntry{JOSAPHAT}\textit{, de l'hébreu «~Yehowshaphat~»~: «~Yahweh a jugé~»}\newline
Fils d'Asa et d'Azuba, il fut roi de Juda pendant vingt-cinq ans. Il eut un règne prospère et fit ce qui est droit aux yeux de Yahweh. Voir \vref{1 R. 15:24}~; \vref{1 R. 22:41-46} et \vref{2 Ch. 17}.

\DicoEntry{JOSEPH}\textit{, de l'hébreu «~Yowceph~»~: «~que Yahweh ajoute~» ou «~il enlève~»}\newline
1. Fils de Jacob et Rachel. Vendu comme esclave par ses frères, il devint, après plusieurs années de prison, gouverneur d'Egypte. Ses fils, Ephraïm et Manasée, furent adoptés par son père Jacob et furent les pères de deux des douze tribus d'Israël. Voir \vref{Ge. 30:22-24}~; \vref{Ge. 37,39,40,45,46}~; \vref{Ge. 48:5} et \vref{Jos. 14:4}.
\\2. Fils d'Héli, charpentier originaire de la tribu de Juda. Epoux de Marie, la mère de Jésus. Voir \vref{Mt. 1:18-25} et \vref{Mt. 13:55}.

\DicoEntry{JOSIAS}\textit{, de l'hébreu «~Yo'shiyah~»~: «~Yahweh guérit~»}\newline
Fils d'Amon, il devint roi de Juda à huit ans et y régna durant trente et un ans. Grand réformateur, à l'origine d'un grand réveil spirituel, il répara le temple, purifia le royaume des idoles et conclut une alliance de fidélité envers Yahweh. Voir \vref{2 R. 22-23}.

\DicoEntry{JOSUÉ}\textit{, de l'hébreu «~Yehowshuwa`~»~: «~Yahweh est salut~»}\newline
Fils de Nun de la tribu d'Ephraïm, choisi par Dieu pour succéder à Moïse. Accompagné de la puissante main de Dieu, il conduisit Israël à entrer en possession de Canaan. Son histoire se trouve dans le livre portant son nom.

\DicoEntry{JOURDAIN}\textit{, de l'hébreu «~Yarden~»~: «~celui qui descend~»}\newline
Très certainement le fleuve le plus connu des Ecritures, il est situé aux limites est de l'actuel territoire d'Israël. Josué et le peuple d'Israël passèrent le fleuve à sec. De même, Elie, puis Elisée, partagèrent les eaux du fleuve en deux. Après s'y être baigné sept fois sur les conseils d'Elisée, Naaman fut guéri de la lèpre. Jésus se fit baptiser par Jean dans le Jourdain. Voir \vref{Jos. 3}~; \vref{2 R. 2:8,12-14}~; \vref{2 R. 5:10-14} et \vref{Mt. 3:13-17}.

\DicoEntry{JOUR DU SEIGNEUR}\textit{}\newline
Jour où Yahweh manifestera sa justice et frappera les nations à cause de leurs péchés. Ce jour arrivera comme un voleur et surprendra beaucoup. Voir \vref{Es. 13:6-16}~; \vref{So. 1}~; \vref{2 Pi. 3:10}.

\DicoEntry{JUDA}\textit{, de l'hébreu «~Yehuwdah~»~: «~qu'il (Dieu) soit loué~»}\newline
Fils de Jacob et Léa, il est le père de la tribu du même nom installée au sud de Canaan. Sa descendance reçut la prédominance et la royauté~; David et Jésus-Christ étaient issus de cette tribu. Après le schisme*, Juda désigna aussi le nom du royaume du sud composé des tribus de Juda et Benjamin. Voir \vref{Ge. 29:35}~; \vref{Ge. 49:8-12}~; \vref{Jos. 15:1-12}~; \vref{1 R. 12:16-24}~; \vref{Mt. 1:1-16}.

\DicoEntry{JUDAS ISCARIOT}\textit{, de l'hébreu «~Yehuwdah~»~: «~qu'il (Dieu) soit loué~»}\newline
Fils de Simon Iscariot, il fut un des douze disciples de Jésus-Christ et était chargé de la trésorerie. Il trahit le Seigneur, ce qu'il regretta amèrement et le poussa à se suicider. Voir \vref{Mt. 26:14-16}~; \vref{Mt. 27:3-5}~; \vref{Lu. 6:16} et \vref{Jn. 12:4-6}.

\DicoEntry{JUDE}\textit{, de l'hébreu «~Yehuwdah~»~: «~qu'il (Dieu) soit loué~»}\newline
1. Fils de Jacques, un des douze apôtres, connu également sous le nom de Thadée. Voir \vref{Mc. 3:18} et \vref{Lu. 6:16}.
\\2. Prophète également appelé Barsabas, compagnon d'œuvre de Silas. Voir \vref{Ac. 15:22,32}.
\\3. Frère du Seigneur, auteur d'une épître qui porte son nom. Voir \vref{Mt. 13:55}~; \vref{Mc. 6:3} et \vref{Jud. 1:1}.

\DicoEntry{JUDÉE}\textit{, de l'hébreu «~Yehuwdah~»~: «~qu'il (Dieu) soit loué~»}\newline
Région située au sud de la Palestine où se trouvent notamment Jérusalem et Bethléhem. Elle correspondrait approximativement au territoire de l'ancien royaume de Juda. Ce terme n'est pas utilisé dans le Tanakh. Voir \vref{Mt. 2:1}~; \vref{Mc. 1:5} et \vref{Ga. 1:22}.

\DicoEntry{JUGE, JUGEMENT}\textit{, de l'hébreu «~shaphat~»~: «~juger, gouverner, défendre, punir~», «~agir comme un législateur, juge ou gouverneur~», «~exécuter un jugement~»}\newline
Dans toute la Parole, Yahweh est présenté comme le juge droit et incorruptible. Après la sortie d'Egypte, des juges ont été suscités par Dieu au milieu d'Israël pour délivrer le peuple de ses ennemis et le ramener vers lui (voir livre des Juges). Le Seigneur a toujours envoyé des prophètes pour annoncer ses jugements et ses décisions ainsi que des juges pour faire respecter sa loi. Sous la Nouvelle Alliance, l'homme spirituel est appelé à juger (discerner selon la Parole), mais condamner et décider du sort final d'une personne demeure la prérogative de Dieu. Yahweh est en effet le juste juge qui siège et tranche non seulement au tribunal de Christ, mais également au jugement dernier. Voir \vref{Ge. 18:25}~; \vref{Jé. 11:20}~; \vref{2 Co. 5:10} et \vref{Ap. 20:11-15}.

\DicoEntry{JUPITER}\textit{, du grec «~Zeus~»~: «~un père des secours~»}\newline
Divinité romaine assimilée à Zeus chez les Grecs. Lors d'une guérison miraculeuse à Lystres, la foule pensa voir en Paul la réincarnation de Mercure et en Barnabas celle de Jupiter. Pour cela, on voulut les adorer, ce qu'ils refusèrent avec véhémence. Voir \vref{Ac. 14:8-15}.

\DicoEntry{JUSTE, JUSTICE}\textit{, de l'hébreu «~tsedeq~»~: «~droiture, exactitude, conforme~» ou encore «~tsadiq~»~: «~juste, exact, innocent~» et du grec «~dikaiosune~»~: «~la condition acceptable par Dieu~» ou «~intégrité, vertu, pureté de vie, droiture~»}\newline
En qualité de juste juge, Yahweh a toujours recherché cette qualité chez l'homme, mais il ne l'a pas trouvé déclarant que nul n'est juste. Au travers de l'œuvre de la croix et avec l'aide du Saint-Esprit, le chrétien peut à présent marcher dans la justice de Dieu. Il est appelé à la rechercher plus que tout et à devenir esclave de la justice. Voir \vref{Mt. 5-7}~; \vref{Lu. 1:75}~; \vref{Ro. 3:10}~; \vref{Ro. 6:18} et \vref{2 Ti. 4:8}.

\DicoEntry{JUSTIFICATION}\textit{, du grec «~dikaiosis~»~: «~état du juste~»}\newline
Au travers de l'œuvre de la croix, Jésus-Christ est devenu la justification de tous ceux qui croient en lui, les rendant acceptables et libres de toute culpabilité. Voir \vref{Ro. 3:23-28}~; \vref{Ro. 4:25} et \vref{Ro. 5:18}.

\DicoEntry{KORÉ}\textit{, de l'hébreu «~Qorach~»~: «~chauve~»}\newline
Fils de Jitsehar, originaire de la tribu de Lévi, il se révolta avec Dathan* et Abiram* contre Moïse* et Aaron*. Suite à sa rébellion, il périt avec les gens de sa maison. Voir \vref{No. 16:1-35}.

\DicoEntry{LAÏC}\textit{, du grec «~laos~»~: «~peuple~»}\newline
Notion propre à l'Eglise catholique romaine. Opposé au clergé*, les laïcs sont les autres membres de l'église, ceux qui n'ont pas de fonction dirigeante, mais qui sont tout de même appelés à honorer Dieu dans leur vie et faire connaître leur foi au milieu du monde.

\DicoEntry{LANGUES}\textit{, de l'hébreu «~lashown~»~: «~langue, langage~» et du grec «~glossa~»~: «~la langue~» ou «~le langage d'un peuple particulier~»}\newline
Selon les Ecritures, les langues sont nées à Babylone* lorsque les hommes se sont rebellés contre la volonté de Yahweh et que ce dernier a confondu leur langage dans le but de les disperser. Dans la Parole sont cités différents types de langues, chacune liée à un don ou une manifestation particulière de l'Esprit de Dieu. Lors de l'effusion du Saint-Esprit à la Pentecôte, les disciples reçurent la capacité de parler des merveilles de Dieu dans des langues étrangères. Il s'agit du don spirituel* appelé la diversité des langues et concerne uniquement les langues usuelles. Il existe également des langues angéliques ou dites inconnues que le croyant peut utiliser pour s'adresser à Dieu. Les langues étrangères tout comme les langues des anges peuvent donner lieu à une interprétation, c'est ce qu'on appelle le don d'interpréter les langues. Voir \vref{Ge. 11}~; \vref{Ac. 2:1-11}~; \vref{1 Co. 12:10}~; \vref{1 Co. 13:1} et \vref{1 Co. 14:1-14,26-27}.

\DicoEntry{LAODICÉE}\textit{, du grec «~Laodikeia~»~: «~justice du peuple~»}\newline
Capitale de la Phrygie, l'une des provinces de l'Asie Mineure, réputée dans le domaine du commerce, notamment dans l'industrie textile. Ses vêtements et sa tapisserie principalement de couleur noire, firent sa renommée. Elle possédait une grande école de médecine qui fabriquait des remèdes réputés pour les yeux, notamment le fameux collyre. L'église de Laodicée est la dernière à qui fut adressée une lettre dans l'Apocalypse. Caractérisée par la tiédeur, l'affection aux choses terrestres et l'aveuglement spirituel, le Seigneur l'appela à la repentance*. Elle est l'image de l'église matérialiste. Voir \vref{Ap. 3:14-22}.

\DicoEntry{LAZARE}\textit{, du grec «~Lazaros~»~: «~Yahweh a secouru~»}\newline
1. Homme pauvre qui fut recueilli dans le sein d'Abraham après sa mort. Voir \vref{Lu. 16:19-21}.
\\2. Frère de Marthe et de Marie de Béthanie, et ami de Jésus-Christ qui le ressuscita des morts. Voir \vref{Jn. 11}.

\DicoEntry{LÉA}\textit{, de l'hébreu «~Le'ah~»~: «~lasse~»}\newline
Fille aînée de Laban et première femme de Jacob. Elle enfanta six fils, pères de six des douze tribus d'Israël (Ruben, Siméon, Lévi, Juda, Issacar et Zabulon) ainsi qu'une fille nommée Dina. Voir \vref{Ge. 29:16-23}~; \vref{Ge. 30:21} et \vref{Ge. 35:23}.

\DicoEntry{LÉMEC}\textit{, de l'hébreu «~Lemek~»~: «~puissant~»}\newline
Fils de Metuschaël et descendant de Caïn, il fut le premier polygame de l'histoire en prenant deux femmes~: Ada et Tsilla. Voir \vref{Ge. 4:16-24}.

\DicoEntry{LÈPRE}\textit{, de l'hébreu «~tsara'~»~: «~être morbide de peau~»}\newline
Commune en Egypte et en orient, maladie de la peau dont le virus peut se développer dans tout le corps. Contagieuse, elle peut même souiller les vêtements et les habitations. Sous la loi mosaïque, les personnes atteintes de cette maladie étaient considérées comme impures et devaient se tenir à l'écart. Durant son ministère, Jésus guérit plusieurs lépreux. Voir \vref{Lé. 13-14} et \vref{Lu. 17:11-14}.

\DicoEntry{LEVAIN}\textit{, de l'hébreu «~chametz~»~: «~ce qui est levé~»}\newline
Symbole du mal et de la corruption, le levain était interdit dans la quasi-totalité des offrandes. Jésus a assimilé le levain des pharisiens à l'hypocrisie, à la doctrine erronée. Les chrétiens sont appelés à faire disparaître le vieux levain et à devenir le levain du monde en y faisant progresser l'évangile du royaume. Voir \vref{Lé. 2:11}~; \vref{Mt. 16:6-12}~; \vref{Mt. 13:33} et \vref{1 Co. 5:6-7}.

\DicoEntry{LÉVI, LÉVITES}\textit{, de l'hébreu «~Leviy~»~: «~attachement~»}\newline
Fils de Jacob et Léa, Lévi participa avec son frère Siméon au massacre des hommes de la ville de Sichem après le viol de leur sœur Dina. Consacrés au service de Yahweh, ses descendants, les Lévites, n'eurent point d'héritage en Canaan, mais habitèrent différentes villes qui leur furent spécifiquement attribuées en Israël. Voir \vref{Ge. 29:34}~; \vref{Ge. 34}~; \vref{No. 18:20-24} et \vref{Jos. 13:14}.

\DicoEntry{LOI}\textit{, de l'hébreu «~towrah~»~: «~loi, direction, commandement~», «~loi mosaïque~»}\newline
L'ensemble des préceptes et ordonnances relatifs à l'alliance conclue entre Yahweh et le peuple hébreu, par l'intermédiaire de Moïse, est contenu dans les cinq premiers livres de la Bible appelée aussi «~le Pentateuque~». Selon la tradition juive, il existerait 613 commandements relatifs à la moralité, la vie en société et le culte rendu à Yahweh. L'homme en étant incapable, Jésus-Christ a accompli les exigences de la loi. Il est donc possible aux hommes d'obtenir le salut par la foi et non plus par les œuvres. La loi est maintenant gravée dans les cœurs des enfants de Dieu à qui le Saint-Esprit rappelle les paroles de Jésus. Voir \vref{Ex. 18:20}~; \vref{Ex. 24:12}~; \vref{Jn. 14:26} et \vref{Ro. 3:19-31}.

\DicoEntry{LOI DU PÉCHÉ}\textit{, du grec «~nomos~»~: «~toute chose établie, une coutume, un commandement~»}\newline
Loi spirituelle inscrite dans la chair qui pousse l'homme charnel à se révolter contre Dieu en commettant le péché. Voir \vref{Ro. 7:13-25}.

\DicoEntry{LOT}\textit{, de l'hébreu~: «~Lowt~»~: «~voile, couverture~»}\newline
Fils de Haran et neveu d'Abraham, Lot quitta Ur avec ce dernier avant de s'en séparer. Grâce à l'intercession d'Abraham, il fut sauvé de la destruction de Sodome avec ses deux filles. Ces dernières enivrèrent leur père et eurent des relations incestueuses avec lui de qui naquirent Moab, père des Moabites, et Amon, père des Ammonites. Voir \vref{Ge. 11:31}~; \vref{Ge. 13:1-13}~; et \vref{Ge. 19}.

\DicoEntry{LUC}\textit{, du grec «~Loukas~»~: «~qui donne la lumière~»}\newline
Médecin de métier, il fut un des compagnons d'œuvre de Paul et l'auteur de l'évangile qui porte son nom et du livre Actes des Apôtres. Voir \vref{Col. 4:14} et \vref{Phm. 1:24}.

\DicoEntry{MACÉDOINE}\textit{, du grec «~Makedonia~»~: «~terre étendue~»}\newline
Province romaine située au nord de la Grèce. Paul y effectua quelques voyages missionnaires et y implanta plusieurs assemblées. Voir \vref{Ac. 16:9-12}~; \vref{Ac. 20:1-3}~; \vref{1 Co. 8:1} et \vref{2 Co. 11:9} et \vref{Ro. 15:23}.

\DicoEntry{MADIAN}\textit{, de l'hébreu «~Midyan~»~: «~lutte, dispute~»}\newline
Un des fils issu de l'union d'Abraham* et Ketura, il devint l'ancêtre des Madianites, peuple qui habita à l'est de Canaan et au nord du désert d'Arabie. Voir \vref{Ge. 25:1-2}~; \vref{No. 31:1-12} et \vref{Jg. 6:2}.

\DicoEntry{MAGOG}\textit{, de l'hébreu «~Magowg~»~: «~territoire de montagne, qui domine~»}\newline
Fils de Japhet. Associé à Gog, il correspond aussi à la nation d'où vient le roi Gog qui fera la guerre à Dieu et à son peuple juste avant le jugement dernier. Voir \vref{Ge. 10:2,9} et \vref{Ap. 20:8}.

\DicoEntry{MAIN}\textit{, de l'hébreu «~yad~»~: «~main, force, pouvoir~»}\newline
Partie du corps permettant de toucher, saisir ou posséder, elle représente aussi l'action, la provision, la protection ou le joug. Tout au long des Ecritures, la main de Yahweh révèle sa puissance et sa bienveillance. Voir \vref{Es. 40:2}~; \vref{Jé. 18:6}~; \vref{Ps. 71:4}~; \vref{Pr. 10:4}~; \vref{Mc. 14:58}~; \vref{Lu. 11:20} et \vref{Ac. 11:21}.

\DicoEntry{MALACHIE}\textit{, de l'hébreu «~Mal`akiy~»~: «~mon messager~»}\newline
Dernier prophète du Tanakh, il condamna les péchés et l'hypocrisie des enfants d'Israël et annonça la venue de Jean-Baptiste. L'ensemble de ses prophéties est contenu dans le livre portant son nom.

\DicoEntry{MALÉDICTION}\textit{, de l'hébreu «~arar, meerah, qelalah~» et du grec «~ara, katara~»}\newline
Parole attirant le malheur sur un bien, une personne ou un peuple. Dieu a le pouvoir de maudire et aussi d'écarter toute malédiction. La malédiction de Dieu, contraire de la bénédiction*, fait suite à la désobéissance. A la nouvelle naissance, toutes les chaînes de malédiction qui liaient le chrétien sont brisées. Le chrétien ne doit pas maudire, mais bénir en tout temps, même ses ennemis. Voir \vref{De. 28:15-68}~; \vref{Mt. 5:44}~; \vref{2 Co. 5:17} et \vref{Ro. 8:1}.

\DicoEntry{MALFAITEUR REPENTANT}\textit{}\newline
Un des hommes coupables qui fut crucifié à côté de Jésus. Son humilité, sa sincérité et sa repentance lui permirent d'accéder au salut, Jésus-Christ lui ayant garanti l'accès au paradis. Voir \vref{Lu. 23:33-43}.

\DicoEntry{MAMON}\textit{, du grec «~Mammonas~»~: «~richesses~»}\newline
Dieu de l'argent. Jésus utilisa ce terme pour personnifier la richesse que beaucoup idolâtrent et qui est par conséquent en concurrence avec Yahweh dans le cœur de certains. Voir \vref{Mt. 6:24}.

\DicoEntry{MANASSÉ}\textit{, de l'hébreu «~Menashsheh~»~: «~oublieux~»}\newline
1. Fils aîné de Joseph* et d'Asnath, adopté par Jacob avant sa mort, ancêtre de la tribu de Manassé. Voir \vref{Ge. 41:51}~; \vref{Ge. 48:5} et \vref{Jos. 14:4}.
\\2. Fils d'Ezéchias et de Hephtsiba, il fut l'un des pires rois du royaume de Juda qui régna 55 ans. Malgré le réveil impulsé par son père, il se détourna entièrement de Yahweh et servit des dieux étrangers. Voir \vref{2 R. 21:1-18}.

\DicoEntry{MANNE}\textit{, de l'hébreu «~man~»~: «~qu'est-ce que cela~?~»}\newline
Nourriture céleste - à l'aspect de la graine de coriandre et au goût de gâteau de miel - que Dieu donna quotidiennement aux Israélites durant toute leur marche dans le désert. \vref{Ex. 16:15,31-35}.

\DicoEntry{MARANATHA}\textit{, de l'araméen «~maran atha~»~: «~le Seigneur vient~» ou «~Seigneur, viens~»}\newline
Expression prononcée par Paul quand il s'adressa aux Corinthiens et qui doit également être le cri du cœur de tout enfant de Dieu. Voir \vref{1 Co. 16:22} et \vref{Ap. 22:17,20}.

\DicoEntry{MARC}\textit{, du grec «~Markos~»~: «~une défense~» ou «~grand marteau~»}\newline
Appelé aussi Jean, cousin de Barnabas, il fut la cause de la séparation de Paul et Barnabas. Il partit avec ce dernier à Chypre et devint par la suite un fidèle compagnon d'œuvre de Paul. Il écrivit l'évangile portant son nom. Voir \vref{Ac. 12:12}~; \vref{Ac. 15:36-39}~; \vref{Col. 4:10} et \vref{Phm. 24}.

\DicoEntry{MARDOCHÉE}\textit{, de l'hébreu «~Mordekay~»~: «~petit homme~»}\newline
Fils de Jaïr de la tribu de Benjamin*, il adopta Esther*, fille de son oncle. Il sauva la vie du roi Assuérus* en déjouant les plans de Bigthan et Théresch et préserva le peuple juif des desseins meurtriers d'Haman. Il devint puissant dans la maison du roi et instaura la fête du Purim. Voir le livre d'Esther.

\DicoEntry{MARIAGE}\textit{, de l'hébreu «~chathan~»~: «~devenir un gendre, s'allier~»}\newline
Bénédiction de Dieu, le mariage est une alliance en principe indissoluble entre un homme et une femme dans le but d'accomplir le plan de Dieu. Il doit être célébré dans le respect des autorités du pays dans lequel le couple se trouve et honoré de tous, particulièrement des parents dont la bénédiction est essentielle. Voir \vref{Ge. 2:22-24}~; \vref{Ge. 24:60}~; \vref{Pr. 18:22}~; \vref{1 Co. 7} et \vref{Hé. 13:4}. Voir commentaire en \vref{Mt. 19:6}.

\DicoEntry{MARIE}\textit{, de l'hébreu «~Miryam~»~: «~rébellion, obstination~»}\newline
1. Sœur de Moïse et d'Aaron, prophètesse. Elle se rebella contre Moïse et fut frappée par la lèpre, mais en guérit grâce à l'intercession de Moïse. Voir \vref{Ex. 15:20} et \vref{No. 12}.
\\2. Mère de Jésus~: Elle conçut, par la vertu du Saint-Esprit, Jésus homme. Elle devint une de ses disciples et se trouvait parmi ceux qui persévéraient dans la prière dans la chambre haute lors de l'effusion du Saint-Esprit promis. Voir \vref{Es. 7:14}~; \vref{Mt. 1:18-25}~; \vref{Mc. 15:40-41}~; \vref{Lu. 1:26-38} et \vref{Ac. 1:13-14}.
\\3. Marie de Magdala~: Elle fut délivrée de sept démons par Jésus qu'elle suivit pendant son ministère terrestre, et ce jusqu'à la croix. Elle fut mandatée par le Seigneur pour annoncer sa résurrection aux apôtres. Voir \vref{Mt. 27:55-56}~; \vref{Mc. 16:1-11}~; \vref{Lu. 8:2} et \vref{Jn. 20:1-18}.
\\4. Marie de Béthanie~: Sœur de Marthe* et de Lazare*, que Jésus ressuscita des morts. Contrairement à sa sœur, elle choisit la bonne part en restant aux pieds du Maître. Elle toucha le cœur de ce dernier en l'oignant d'un parfum de grand prix. Voir \vref{Lu. 10:38-42}~; \vref{Jn. 11:1-44} et \vref{Jn. 12:1-7}.

\DicoEntry{MARTHE}\textit{, du grec «~Martha~»~: «~maîtresse, dame~»}\newline
Sœur de Lazare* - dont elle fut témoin de la résurrection - et de Marie* de Béthanie, elle reçut Christ dans sa maison, mais ce dernier lui reprocha son activisme au détriment de l'écoute de sa Parole. Voir \vref{Lu. 10:38-42} et \vref{Jn. 11:1-44}.

\DicoEntry{MATTHIAS}\textit{, de l'hébreu «~Mattithyah~»~: «~don de Yahweh~»}\newline
Disciple de Jésus et témoin oculaire de son ministère, il fut désigné pour devenir l'un des douze apôtres en remplacement de Judas Iscariot* qui avait trahi le Seigneur pour ensuite se suicider. Voir \vref{Ac. 1:15-26}.

\DicoEntry{MATTHIEU}\textit{, du grec «~Matthaios~»~: «~don de Yahweh~»}\newline
Collecteur d'impôts, il fut l'un des douze apôtres de Jésus et l'auteur de l'évangile qui porte son nom. Voir \vref{Mt. 9:9} et \vref{Mt. 10:3}.

\DicoEntry{MEÉDIATEUR}\textit{, du grec «~mesites~»~: «~celui qui intervient entre deux parties~», «~intermédiaire de communication~»}\newline
Moïse a exercé cette fonction auprès du peuple d'Israël qui avait expressément demandé que Dieu ne leur parle pas directement. Christ, garant d'une Nouvelle Alliance, est à présent l'unique intermédiaire et médiateur entre Dieu et les hommes. Voir \vref{Ex. 20:19}~; \vref{1 Ti. 2:5}~; \vref{Hé. 8:6} et \vref{Hé. 9:15}.

\DicoEntry{MELCHISÉDEK}\textit{, de l'hébreu «~Malkiy-Tsedeq~»~: «~roi de justice~»}\newline
Roi de Salem et prêtre du Dieu Très-Haut, il était une apparition de Jésus-Christ avant son apparition. Il bénit Abraham après sa victoire contre Kedorlaomer. Jésus-Christ est grand prêtre selon l'ordre de Melchisédek. Voir \vref{Ge. 14:14-20}~; \vref{Hé. 5:5-10} et \vref{Hé. 6:20}.

\DicoEntry{MENSONGE}\textit{, de l'hébreu «~sheqer~»~: «~mensonge, déception, fausseté, tromperie, fraude~» et du grec «~pseudos~»~: «~fausseté consciente et intentionnelle~»}\newline
Modification de la vérité. Satan est appelé père du mensonge et les menteurs auront droit à la même sentence que lui. Voir \vref{Ex. 20:16}~; \vref{Jn. 8:44} et \vref{Ap. 21:8}.

\DicoEntry{MÉSOPOTAMIE}\textit{, de l'hébreu «~'Aram Naharayim~»~: «~pays entre deux fleuves~»}\newline
Située entre le Tigre et l'Euphrate, région correspondant à l'actuel Irak. Avant son appel, Abraham vivait à Ur en Chaldée qui se trouvait au sud de la Mésopotamie. Voir \vref{Ge. 11:31}.

\DicoEntry{MESSIE}\textit{, de l'hébreu «~mashiyach~»~: «~oint, celui qui est l'oint~»}\newline
Voir CHRIST.

\DicoEntry{MICHÉE}\textit{, de l'hébreu «~Miykayehuw~»~: «~qui est comme Dieu~?~»}\newline
Originaire de Moréscheth, Michée exerça la fonction de prophète dans le royaume du sud au temps d'Ezéchias, roi de Juda. L'ensemble de ses prophéties se trouve dans le livre éponyme.

\DicoEntry{MICHEL ou MICHAËL}\textit{, de l'hébreu «~Miyka'el~»~: «~qui est semblable à Dieu~?~»}\newline
Archange* de Dieu, il est un des principaux chefs des anges*. Souvent présent dans les grandes batailles, il lutta notamment contre le roi de Perse et contre le diable. Voir \vref{Da. 10:13-21}~; \vref{Jud. 1:9} et \vref{Ap. 12:7}.

\DicoEntry{MILLE}\textit{, du grec «~million~»~: «~distance de mille pas~»}\newline
Unité de mesure romaine correspondant à 1480m environ. Voir \vref{Mt. 5:41}.

\DicoEntry{MILLÉNIUM}\textit{}\newline
Période de paix de mille ans durant laquelle le Seigneur régnera sur la terre. Voir \vref{Es. 11,12} et \vref{Ap. 20:2-7}.

\DicoEntry{MINISTÈRE}\textit{, du grec «~diakonia~»~: «~service~», dérivé du mot grec «~diakonos~»~: «~domestique~»}\newline
voir SERVICE. 

\DicoEntry{MISÉRICORDE}\textit{, de l'hébreu «~checed~»~: «~bonté, miséricorde, fidélité~» et du grec «~eleos~»~: «~bonne volonté envers le misérable associée à un désir de l'aider~»}\newline
Comme en témoigne le plan du salut* qu'il a déployé, Dieu est riche en miséricorde. Le disciple de Christ doit comme son maître se revêtir d'entrailles de miséricorde afin de représenter le Royaume de Dieu. \vref{Ge. 24:7}~; \vref{No. 24:18}~; \vref{Mt. 9:13}~; \vref{Lu. 1:78}~; \vref{Ro. 11:31} et \vref{2 Jn. 1:3}.

\DicoEntry{MOAB}\textit{, de l'hébreu «~Mow'ab~»~: «~issu d'un père~»}\newline
Fils de Lot*, né de sa relation incestueuse avec sa fille aînée, il donna naissance au peuple des moabites. Ils s'établirent au sud-est de la mer morte et s'opposèrent plusieurs fois aux enfants d'Israël. Voir \vref{Ge. 19:37}~; \vref{Jg. 3:12}~; \vref{2 S. 8:2}~; \vref{Ez. 25:8-11}.

\DicoEntry{MODALISME}\textit{}\newline
Doctrine enseignée à Rome au début du troisième siècle par Sabellius selon laquelle le Père, le Fils et le Saint-Esprit sont différents aspects au travers desquels Dieu se révèle et non trois personnes distinctes. Réfutant ainsi la doctrine de la trinité* largement acceptée par les catholiques, Sabellius fut condamné par le pape Callixte à cause de son enseignement pourtant biblique. Voir \vref{1 Th. 3:11}~; \vref{2 Th. 2:16-17} et \vref{1 Jn. 5:20}.

\DicoEntry{MOÏSE}\textit{, de l'hébreu «~Mosheh~»~: «~tiré de~»}\newline
Issu de la tribu de Lévi, il fut miraculeusement sauvé du massacre des enfants de sa génération pendant la servitude d'Israël en Egypte. Il vécut les quarante premières années de sa vie dans la maison de Pharaon puis les quarante suivantes dans le désert auprès de Madian*. A l'issue de cette deuxième période, Yahweh se révéla à lui et le mandata pour délivrer le peuple d'Israël de la captivité égyptienne afin de le faire entrer dans la terre promise. Après l'avoir fait sortir au milieu des miracles et des prodiges, Moïse conduisit le peuple dans le désert pendant quarante années au cours desquelles il leur communiqua l'intégralité de la loi*. Il mourut à la porte de la terre promise à l'âge de cent vingt ans. On lui attribue l'écriture des cinq premiers livres du Tanakh. Voir \vref{Ex. 1-2}~; \vref{Ex. 12:40-41}~; \vref{Ex. 14:21-31}~; \vref{Ex. 24:12}~; \vref{De. 8:2}~; \vref{De. 34:5-7}~; \vref{Ac. 7:20-43} et \vref{Hé. 11:23-29}.

\DicoEntry{MOISSON}\textit{, de l'hébreu «~qatsiyr~»~: «~moisson, travail de la moisson, récolte~»}\newline
Sous la loi, la fête des prémices avait lieu lors de la moisson. Jésus utilise ce terme pour parler du champ missionnaire, les personnes à qui l'évangile doit être annoncé. Dans le cadre de la fin du monde, la moisson se rapporte au jugement de Dieu qui va apporter la séparation entre ses fils et les fils du diable. Voir \vref{Lé. 23:10-14}~; \vref{Mt. 9:37-38}~; \vref{Mt. 13:33-43}.

\DicoEntry{MOLOC}\textit{, de l'hébreu «~Molek~»~: «~roi, conseiller~»}\newline
Divinité vénérée par les Ammonites à qui il était coutume de sacrifier des enfants brûlés vifs. Les Israélites se prostituèrent plusieurs fois à Moloc. Voir \vref{1 R. 11:5-7} et \vref{2 R. 23:10}.

\DicoEntry{MONT DES OLIVIERS}\textit{}\newline
Colline située à l'est de Jérusalem près de la vallée du Cédron. C'est du Mont des Oliviers que Jésus fut enlevé au ciel après avoir donné ses dernières recommandations aux apôtres~; c'est à ce même endroit qu'il posera les pieds lors de son glorieux retour. Voir \vref{Za. 14:1-4} et \vref{Ac. 1:4-12}.

\DicoEntry{MORT}\textit{, de l'hébreu «~muwth~»~: «~mourir, tuer, être exécuté~» et du grec «~thanatos~»~: «~mort du corps~»}\newline
La Bible distingue deux morts. La première entra dans le monde suite à la désobéissance de l'homme et correspond à la séparation d'avec Dieu et à la mort physique. La deuxième mort concerne uniquement ceux dont le nom n'est pas écrit dans le livre de vie et correspond à la souffrance éternelle dans l'étang de feu. Voir \vref{Ge. 3}~; \vref{Ro. 5:12}~; \vref{Ro. 6:23} et \vref{Ap. 20:11-15}.

\DicoEntry{MYRRHE}\textit{, de l'hébreu «~more~»~: «~myrrhe~»}\newline
Résine provenant de certains arbres d'Asie et d'Afrique, réputée pour son arôme de grand prix. Elle était utilisée sous forme d'huile pour l'onction sainte et pouvait atténuer les douleurs quand elle était mélangée au vin. Les mages offrirent de la myrrhe à Jésus lors de sa naissance. Voir \vref{Ex. 30:22-30}~; \vref{Mt. 2:11}~; \vref{Mt. 27:34} et \vref{Mc. 15:23}.

\DicoEntry{NAHUM}\textit{, de l'hébreu «~Nachuwm~»~: «~consolation, qui a compassion~»}\newline
Prophète de Yahweh né à Elkosch, il annonça la destruction de Ninive. L'ensemble de ses prophéties se trouve dans le livre portant son nom.

\DicoEntry{NAISSANCE D'EN HAUT}\textit{, du grec «~anothen~»~: «~depuis le haut, depuis un endroit plus élevé~»}\newline
Naissance d'eau et d'esprit symbolisant respectivement la Parole qui purifie et le Saint-Esprit* qui est le gage de l'appartenance à Dieu. La naissance d'en haut est l'œuvre du Saint-Esprit qui délivre une personne du royaume des ténèbres et la transporte dans le Royaume de Dieu. L'homme charnel devient alors spirituel, le cœur de pierre est ôté pour accueillir un cœur de chair, le citoyen terrestre se transforme en citoyen céleste et le vieil homme laisse place à une nouvelle créature. Voir \vref{Ez. 36:25-27}~; \vref{Jn. 3:1-8}~; \vref{Ja. 1:18}~; \vref{1 Co. 12:13} et \vref{2 Co. 5:17}~; \vref{Ep. 2:6}~; \vref{Ep. 5:26}~; \vref{1 Jn. 3:9}.

\DicoEntry{NATHAN}\textit{, de l'hébreu «~Nathan~»~: «~il (Yahweh) a donné~»}\newline
Prophète de Yahweh au temps du roi David. Il prophétisa le règne éternel de la postérité de David et la construction du temple par son fils. Il reprit David lorsque ce dernier fit assassiner Urie* pour prendre sa femme. Voir \vref{2 S. 7,12}.

\DicoEntry{NAZARÉEN, NAZIRÉEN}\textit{, de l'hébreu «~naziyr~»~: «~consacré ou voué~»}\newline
Terme pouvant désigner soit un habitant de la ville de Nazareth, soit une personne qui s'est consacrée à Yahweh dans le cadre d'un vœu de naziréat. Voir \vref{No. 6}.

\DicoEntry{NAZARETH}\textit{, du grec «~Nazareth~»~: «~verdoyant, germe, rejeton~»}\newline
Ville située dans la région de Galilée où Jésus passa la majeure partie de sa vie. Voir \vref{Mt. 2:22-23}.

\DicoEntry{NEBUCADNETSAR}\textit{, (règne~: 605 av. J.-C. – 562 av. J.-C.), «~Nebuwkadne'tstsar~»~: «~que Nebo protège la couronne, les frontières~» (origine inconnue)}\newline
Roi de Babylone, il mit fin au royaume de Juda en emmenant le peuple en captivité~; il détruisit le temple de Jérusalem. Il reçut l'interprétation de plusieurs songes au travers de Daniel* et reconnut le règne dominant et éternel de Yahweh. Voir \vref{2 R. 25}, \vref{Da. 1:1} et \vref{Da. 2,4}.

\DicoEntry{NÉHÉMIE}\textit{, de l'hébreu «~Nechemyah~»~: «~Yahweh a consolé~»}\newline
Fils d'Hacalia, il fut échanson du roi Artaxerxès* à Suse, pendant la captivité de Juda. Il entreprit la réparation des murailles de Jérusalem et initia une réforme en son temps. Il devint ensuite gouverneur de Juda. Son histoire est racontée dans le livre éponyme.

\DicoEntry{NÉPHILIM}\textit{, de l'hébreu «~nephiyl~»~: «~géant~», racine~: «~naphal~»~: «~tomber, chuter~»}\newline
Etres de grande taille nés de l'union des fils de Dieu et des filles des hommes avant le déluge. On en retrouve aussi en Canaan lorsque les douze espions hébreux étaient allés observer la terre promise. Voir \vref{Ge. 6:4} et \vref{No. 13:32-33}.

\DicoEntry{NEPHTHALI}\textit{, de l'hébreu «~Naphtaliy~»~: «~lutte, mon combat~»}\newline
Fils de Jacob* et de Bilha, servante de Rachel*, il est l'ancêtre de la tribu de Nephthali. Voir \vref{Ge. 30:8} et \vref{Ge. 49:21}.

\DicoEntry{NICODÈME}\textit{, du grec «~Nikodemos~»~: «~victorieux du peuple~»}\newline
Docteur de la loi, il s'approcha de Jésus de nuit, qui l'enseigna sur la naissance d'en haut. Après la crucifixion, il aida Joseph d'Arimathée pour embaumer le corps du Seigneur et pour le mettre dans un sépulcre. Voir \vref{Jn. 3:1-21} et \vref{Jn. 19:38-42}.

\DicoEntry{NICOLAÏTES}\textit{, du grec «~Nikolaites~»~: «~destruction du peuple~»}\newline
Secte suivant la doctrine de Nicolas, liée à la doctrine de Balaam, qui poussait à la consommation de viandes sacrifiées aux idoles et à l'impudicité. Voir \vref{Ap. 2:6,14-15}.

\DicoEntry{NIL}\textit{, de l'hébreu «~Shiychowr~»~: «~sombre, noir, boueux~»}\newline
Principal fleuve d'Egypte situé à l'est du pays. Voir \vref{Es. 23:3}~; \vref{Jé. 2:18}.

\DicoEntry{NIMROD}\textit{, de l'hébreu «~Nimrowd~»~: «~rebelle~»}\newline
Fils de Cush et descendant de Noé. Chasseur, il fut le premier homme puissant sur la terre et régna sur plusieurs villes dont Babel*. Voir \vref{Ge. 10:8-11} et \vref{Ge. 11:1-9}.

\DicoEntry{NINIVE}\textit{, de l'hébreu «~Niyneveh~»~: «~habitation de Ninus~»}\newline
Grande ville située sur la rive est du Tigre. Ses habitants se repentirent de leurs mauvaises voies suite à la prédication de Jonas, mais ils retombèrent dans le péché quelques années plus tard. Ninive fut finalement détruite sous le jugement de Dieu. Voir livres de Jonas et de Nahum.

\DicoEntry{NOCES}\textit{, du grec «~gamos~»~: «~fête du mariage~»}\newline
Festivités célébrant le mariage. Dans la tradition juive, les noces duraient sept jours même si une longue période pouvait parfois s'écouler entre la conclusion du mariage (accord des familles) et la consommation du mariage (nuit de noces). Ainsi, la fiancée devait se tenir prête pour les noces à tout moment. De même, l'Eglise se prépare à être enlevée par Jésus à tout moment pour les noces de l'Agneau qui seront célébrées au ciel pendant sept ans. Voir \vref{Ge. 29:27}~; \vref{Jg. 14:12}~; \vref{1 Th. 4:16-17} et \vref{Ap. 19:7}.

\DicoEntry{NOÉ}\textit{, de l'hébreu «~Noach~»~: «~repos, tranquillité~»}\newline
Fils de Lamech, il fut le père de trois fils~: Sem, Cham et Japhet. Qualifié d'homme juste et intègre en son temps, il trouva grâce devant Yahweh qui lui ordonna de construire une arche* pour le sauver lui, sa famille et une partie des animaux de la terre du déluge qui arrivait. Son obéissance sauva la race humaine. Il vécut 950 ans. Voir \vref{Ge. 6-9}.

\DicoEntry{NOUVELLE NAISSANCE}\textit{}\newline
Voir NAISSANCE D'EN HAUT.

\DicoEntry{OFFRANDE}\textit{, de l'hébreu «~minchah~»~: «~don, tribut, présent, oblation, sacrifice~»}\newline
Sous la loi, le peuple d'Israël avait reçu des prescriptions relatives aux offrandes agréables à Yahweh~; elles consistaient essentiellement en bétail et produits naturels et étaient offertes dans le cadre de cérémonies spécifiques. Des offrandes en argent pouvaient aussi être données, notamment pour soutenir l'entretien du temple. Sous la Nouvelle Alliance, les offrandes monétaires doivent être libres et volontaires~; l'offrande la plus importante aux yeux de Dieu reste la vie consacrée de ses enfants. Voir \vref{Lé. 1-7}~; \vref{Mc. 12:41-42}~; \vref{2 Co. 8:10-12}~; \vref{2 Co. 9:7}~; \vref{Ro. 12:1} et \vref{Ro. 15:15-16}.

\DicoEntry{OLIVIER}\textit{}\newline
Arbre fruitier donnant des olives avec lesquelles on produit de l'huile. Sous la loi de Moïse, elle était notamment utilisée pour alimenter les lampes qui devaient brûler continuellement dans le temple et pour oindre les personnes désignées par Dieu pour une tâche spécifique. L'olivier symbolise en outre le témoignage et la paix. Voir \vref{Ex. 27:20-21}~; \vref{Ex. 30:22-25}~; \vref{Jg. 9:8-9}~; \vref{1 S. 16:3} et \vref{1 R. 19:16}.

\DicoEntry{OMEGA}\textit{}\newline
Dernière lettre de l'alphabet grec désignant aussi la fin d'une chose (voir ALPHA et OMEGA).

\DicoEntry{ONCTION}\textit{, de l'hébreu «~mishchah~»~: «~portion consacrée, huile d'onction, oindre~» et du grec «~chrisma~»~: «~toute chose qui sert à enduire~» de la racine «~chrio~»~: «~oindre, imprégner les chrétiens des dons du Saint-Esprit~»}\newline
Sous l'Ancienne Alliance, l'onction était souvent accordée par l'action de verser de l'huile sur la tête de la personne ou de l'objet à consacrer. On oignait ainsi les prêtres, les rois et les prophètes selon leur mandat. Sous la Nouvelle Alliance, l'onction demeure en celui qui a reçu en lui le Seigneur Jésus. Toutefois, l'onction d'huile peut être pratiquée dans le cadre de la prière pour les malades. Voir \vref{Ex. 30:22-31}~; \vref{1 S. 16:3} et \vref{1 R. 19:16}~; \vref{Ac. 1:8}~; \vref{Ja. 5:14} et \vref{1 Jn. 2:20-27}.

\DicoEntry{ORDINATION}\textit{, du latin «~ordinatio~»~: «~action de disposer, de mettre en œuvre~»}\newline
Rite initiatique mis en place par l'Eglise catholique qu'on ne retrouve pas dans les Ecritures. Elle confère, par l'imposition des mains accompagnée d'une prière, la capacité d'exercer une fonction dirigeante au sein de l'église locale.

\DicoEntry{OTHNIEL}\textit{, de l'hébreu «~`Othniy'el~»~: «~Dieu est puissant~»}\newline
Fils de Kenaz et frère cadet de Caleb, il fut le premier juge en Israël, fonction qu'il exerça pendant 40 ans. Il délivra les enfants d'Israël du joug du roi de Mésopotamie, Cuschan-Rischeathaïm. Voir \vref{Jg. 3:8-11}.

\DicoEntry{OSÉE}\textit{, de l'hébreu «~Howshea`~»~: «~salut, sauve~»}\newline
Fils de Beéri, prophète qui, sous les ordres de Yahweh, épousa une prostituée pour illustrer l'infidélité des enfants d'Israël envers leur Dieu. L'histoire d'Osée et l'ensemble de ses prophéties se trouvent dans le livre portant son nom.

\DicoEntry{PAÏEN}\textit{, du latin «~paganus~» qui signifie «~paysan~», qui provient lui-même du mot «~pagus~» qui signifie «~campagne~».}\newline
Personne qui pratiquait une des religions polythéistes de l'Antiquité.

\DicoEntry{PAIX}\textit{, de l'hébreu «~shalowm~»~: «~état complet, perfection, bien-être, paix~» et du grec «~eirene~»~: «~état de tranquillité, paix entre les individus, harmonie, sécurité~»}\newline
Sous l'Ancienne Alliance, la paix était matérialisée par la prospérité, l'absence de guerre et de toutes sortes de malheurs. Sous la Nouvelle Alliance, la paix est un fruit de l'Esprit*, une promesse acquise en Jésus qui est lui-même le Prince de Paix. Différente de celle que le monde offre, la paix de Christ permet de rester confiant en toutes circonstances. Voir \vref{Lé. 26:6}~; \vref{Es. 26:12}~; \vref{Jn. 14:27}~; \vref{Jn. 16:33} et \vref{Ga. 5:22}.

\DicoEntry{PALMIER}\textit{, de l'hébreu «~tamar~»~: «~palmier, dattier~»}\newline
Arbre à tronc peu ou pas ramifié, on le retrouve essentiellement dans le désert. L'image du palmier fut utilisée en décoration dans le temple. Ses branches étaient utilisées pendant la fête des tentes. Symbole de la justice et de la victoire, on le retrouve lors de l'entrée royale de Jésus à Jérusalem et devant le trône de Dieu. Voir \vref{Lé. 23:40}~; \vref{1 R. 6:29}~; \vref{Jn. 12:12-13} et \vref{Ap. 7:9}.

\DicoEntry{PÂQUE}\textit{, de l'hébreu «~pecach~»~: «~passer outre, épargner~», «~sacrifice de la Pâque~» ou «~fête de la Pâque~»}\newline
Première fête du calendrier hébraïque, elle fut instituée par ordonnance perpétuelle dès la sortie d'Egypte. Cette fête commémore le salut de Yahweh accordé par le sacrifice de l'agneau~; elle préfigurait Christ, l'Agneau de Dieu qui est «~notre Pâque~». Voir \vref{Ex. 12}~; \vref{Lé. 23:5}~; \vref{Jn. 1:29} et \vref{1 Co. 5:7-8}.

\DicoEntry{PARADIS}\textit{, du grec «~paradeisos~»~: «~jardin~»}\newline
Lieu de repos et de félicité, le paradis fut ouvert par Jésus lors de sa résurrection. Il y emmena les justes décédés qui étaient jusque-là captifs dans le séjour des morts*. Les chrétiens rejoignent ce lieu céleste à leur décès, en attendant la résurrection*. A la croix, Christ garantit l'accès à ce lieu au malfaiteur repentant. Paul fut ravi à cet endroit où il entendit des paroles merveilleuses. Voir \vref{Lu. 23:43}~; \vref{2 Co. 12:2-4}~; \vref{Ep. 4:8-10} et \vref{Hé. 10:19-20}.

\DicoEntry{PARDON}\textit{, de l'hébreu «~nas a´~»~: «~action de lever, supporter, prendre~» et du grec «~aphesis~»~: «~libérer de l'esclavage~» ou «~oubli des péchés, rémission des peines~»}\newline
Sous l'Ancienne Alliance, le pardon était conditionné par les sacrifices d'animaux, mais Jésus-Christ a accompli cette prérogative en devenant la victime expiatoire pour nos péchés. En lui, l'homme repentant est pardonné de ses fautes et trouve également la force de pardonner à ceux qui l'offensent. Voir \vref{Lé. 4-6}~; \vref{Mt. 6:12,14-15}~; \vref{Jn. 1:29}~; \vref{Ac. 10:43} et \vref{1 Jn. 1:9}.

\DicoEntry{PARVIS}\textit{, de l'hébreu «~chatser~»~: «~cour, enclos, colonie, ville, village~» (voir illustration du temple)}\newline
Première des trois parties du tabernacle et du temple~; il s'agissait d'une cour dans laquelle se trouvait l'autel d'airain où se faisaient des sacrifices et la cuve d'airain contenant de l'eau pour la purification. Voir \vref{Ex. 27:9-19}.

\DicoEntry{PASTEUR}\textit{, de l'hébreu «~ra`ah~»~: «~berger~»}\newline
Un des cinq services d'\vref{Ep. 4:11} travaillant en collège, établi pour veiller pour le troupeau, le nourrir de la Parole et encourager les chrétiens à exercer pleinement et librement leur ministère. Toutefois, Jésus-Christ demeure le pasteur par excellence, le bon berger qui donne sa vie pour ses brebis et le gardien des âmes qui ne sommeille ni ne dort. Voir \vref{Ep. 4:11}~; \vref{Jn. 10:11-16}~; \vref{Ps. 23} et \vref{1 Pi. 2:25}.

\DicoEntry{PATMOS}\textit{, du grec «~Patmos~»~: «~mortel, fascinant~»}\newline
Petite île grecque de la mer Egée sur laquelle Jean fut exilé à la fin de sa vie. Il y reçut la révélation de l'Apocalypse. Voir \vref{Ap. 1:9}.

\DicoEntry{PAUL}\textit{, du grec «~Paulos~»~: «~petit~»}\newline
Issu de la tribu de Benjamin et né dans la ville de Tarse, son nom était initialement Saul*. Pharisien, son zèle excessif le poussa à persécuter violemment les chrétiens à la naissance de l'Eglise. Il rencontra Christ sur la route de Damas et devint par la suite l'apôtre des Gentils annonçant l'Evangile de villes en villes et de pays en pays au cours de nombreux voyages. Même en prison, il continua l'œuvre de Dieu en écrivant plusieurs lettres riches en enseignements que l'on peut retrouver dans le canon biblique. Voir \vref{Ac. 9-28} et les épîtres de Paul.

\DicoEntry{PÉAGER ou PUBLICAIN}\textit{, du grec «~telones~»~: «~un loueur, un collecteur de taxes~»}\newline
Les péagers d'origine juive étaient dépréciés de leurs compatriotes et assimilés à des pécheurs, car on les considérait comme des collaborateurs au service des romains. De plus, certains profitaient de leur fonction pour s'enrichir. Voir \vref{Mt. 9:10}~; \vref{Mt. 21:31}~; \vref{Lu. 3:12-13} et \vref{Lu. 19:2-8}.

\DicoEntry{PÉCHÉ}\textit{, de l'hébreu «~chatta'ah~»~: «~ce qui manque le but~» et du grec «~hamartano~»~: «~erreur, faux état d'esprit~»}\newline
Le péché entra dans le monde par la transgression d'Adam et Eve et tous les hommes en furent infectés. Origine de la séparation entre Dieu et les hommes, le péché conduit à la mort*. Voir \vref{Ge. 3}~; \vref{1 Co. 15:3}~; \vref{Ro. 5:12}~; \vref{Ro. 6:23}~; \vref{Ro. 8:1-4} et \vref{1 Pi. 2:21-24}.

\DicoEntry{PENTECÔTE}\textit{, du grec «~pentekoste~»~: «~le cinquantième jour~»}\newline
Fête annuelle juive célébrant la moisson des blés. La venue du Saint-Esprit* promis par Jésus eut lieu pendant la célébration de la Pentecôte. Voir \vref{Lé. 23:15-22}~; \vref{Jn. 16:7-11}~; \vref{Ac. 1:5} et \vref{Ac. 2:1-21}.

\DicoEntry{PHARAON}\textit{, de l'hébreu «~Par`oh~»~: «~grand palais~»}\newline
Titre donné aux rois égyptiens durant l'Antiquité. Voir \vref{Ge. 37:36} et \vref{Ge. 41}.

\DicoEntry{PHARISIEN}\textit{, du grec «~Pharisaios~»~: «~séparé~»}\newline
Secte juive dont les membres manifestaient un attachement excessif aux coutumes et traditions religieuses. Certains d'entre eux combattirent Jésus qui dénonça ouvertement leur fausse piété et leur dévouement hypocrite envers Dieu. Désirant la mort du Seigneur, ils participèrent à la conspiration qui précéda sa crucifixion. Voir \vref{Mt. 23:23-39}~; \vref{Mc. 7:1-13} et \vref{Jn. 18:2-3}.

\DicoEntry{PHILADELPHIE}\textit{, du grec «~Philadelpheia~»~: «~amour fraternel~»}\newline
Ville de Lydie en Asie Mineure. Irriguée par le fleuve Hermus, Philadelphie était une contrée très fertile, propice à l'agriculture et surtout à la culture de la vigne. Elle fut construite par le roi de Pergame, et plusieurs fois sujette à des tremblements de terre. Une des sept lettres d'Apocalypse s'adressait à l'église de Philadelphie. Cette dernière - contrairement aux autres qui cumulèrent des reproches – fut très encouragée par le Seigneur. Bien que située à 45 km de Sardes à laquelle elle était rattachée, l'Eglise de Philadelphie resta ferme en retenant la Parole de Dieu et ne se laissa pas influencer par les séductions du péché. Elle incarne ainsi l'Eglise que Jésus revient chercher, l'Eglise réveillée.

\DicoEntry{PHILÉMON}\textit{, du grec «~Philemon~»~: «~attentionné, qui embrasse~»}\newline
Disciple de Colosses qui recevait une église dans sa maison. Il avait un esclave nommé Onésime au sujet duquel Paul lui écrivit une lettre. Voir épître de Paul à Philémon.

\DicoEntry{PHILIPPE}\textit{, du grec «~Philippos~»~: «~aimant les chevaux~»}\newline
1. Homme de Bethsaïda, il fut l'un des douze apôtres* choisis par Jésus. Voir \vref{Mt. 10:3}~; \vref{Mc. 3:18} et \vref{Lu. 6:14}.
\\2. Un des sept diacres élus au sein de l'église de Jérusalem. Evangéliste*, il prêcha le Christ dans la ville de Samarie, à l'eunuque éthiopien qu'il baptisa et dans différentes villes. Voir \vref{Ac. 6:5}~; \vref{Ac. 8:4-8,26-40} et \vref{Ac. 21:8}.

\DicoEntry{PHILIPPES}\textit{, du grec «~Philippoi~»~: «~appartenant à Philippe~»}\newline
Fondée par Philippe II (382 av. J.-C. – 336 av. J.-C.) en 356 av. J.-C., ville grecque de Macédoine orientale. Située sur une voie romaine qui traversait les Balkans, elle est restée de taille modeste en dépit de son fort taux de fréquentation. Une église y naquit après la rencontre de Paul avec des femmes qui priaient à l'extérieur de la ville. L'apôtre leur écrivit une lettre qui figure dans le canon biblique. Voir \vref{Ac. 16:9} et l'épître aux Philippiens.

\DicoEntry{PHILISTINS}\textit{, de l'hébreu~: «~Pelesheth~»~: «~immigrants~»}\newline
Peuple qui habitait à l'extrême ouest de Canaan, le long de la mer Méditerranée. Ils furent plusieurs fois en conflit avec les Israélites~; Goliath était philistin. Voir \vref{Jg. 13-16} et \vref{1 S. 17}.

\DicoEntry{PHILOSOPHIE}\textit{, du grec «~philosophia~»~: «~amour de la sagesse~»}\newline
Discipline existant depuis l'Antiquité et ayant plusieurs courants de pensée en son sein comme les épicuriens* et les stoïciens*. Elle pousse ses adeptes à rechercher la sagesse par l'intelligence humaine. Paul invita les chrétiens à se garder de ces doctrines. Voir \vref{Ac. 17:16-20} et \vref{Col. 2:8}.

\DicoEntry{PHINÉES}\textit{, de l'hébreu «~Piynechac~»~: «~bouche de cuivre~»}\newline
Fils d'Eléazar, petit-fils d'Aaron et prêtre. Il se démarqua par son zèle pour Dieu pour arrêter un fléau sur Israël. A cette occasion, Yahweh fit alliance perpétuelle avec Phinées et sa descendance. Voir \vref{No. 25}.

\DicoEntry{PIERRE}\textit{, de l'hébreu «~Cephas~» et du grec «~Petros~»~: «~un roc ou une pierre~»}\newline
Fils de Jonas et frère d'André*, son nom était initialement Simon*. Pêcheur de métier originaire de la ville de Bethsaïda, il fut choisi comme apôtre pour les circoncis. Il écrivit deux épîtres portant son nom. Il aurait été crucifié à Rome. Voir \vref{Mt. 10:2}~; \vref{Jn. 1:42-44}~; \vref{Ga. 2:7-8}~; 1 Pi. et 2 Pi.

\DicoEntry{PILATE}\textit{, du grec «~Pilatos~»~: «~armé d'une lance~»}\newline
Gouverneur romain de la Judée en fonction pendant le ministère de Jésus. Il s'accorda avec son ennemi Hérode lorsqu'il fallut crucifier le Seigneur. N'ayant pas trouvé de crime en Jésus, il permit finalement sa crucifixion et fit mettre l'inscription suivante sur sa croix~: Jésus de Nazareth, roi des Juifs. Voir \vref{Lu. 3:1}~; \vref{Lu. 23:11} et \vref{Jn. 19:1-19}.

\DicoEntry{PRÉDESTINATION}\textit{, du grec «~proginosko~»~: «~avoir la connaissance avant~»}\newline
Révélant l'omniscience de Dieu qui connaît toutes choses à l'avance, la prédestination concerne l'œuvre de la croix prévue de toute éternité – l'Agneau ayant été immolé avant la fondation du monde. La prédestination est non pas la décision de Dieu d'envoyer certaines personnes en enfer, mais plutôt la capacité de Yahweh à connaître à l'avance ceux qui allaient devenir ses enfants d'adoption, transformés à l'image du Fils, en acceptant sa parole. Voir \vref{Jn. 1:12}~; \vref{Ro. 8:29-30}~; \vref{Ep. 1:5} et \vref{1 Pi. 1:19-20}.

\DicoEntry{PREMIER PRÊTRE}\textit{, de l'hébreu «~rosh~»~: «~tête, dessus, sommet, partie supérieure, chef, principal, premier, total, somme, hauteur, front, le devant, commencement~» et de «~kohen~»~: «~prêtre, intendant principal, ministre d'état~»}\newline
Sous la loi, le premier prêtre descendait d'Aaron. Il servait le Seigneur dans le sanctuaire* et enseignait la loi*. Tel un médiateur* entre Yahwhe et le peuple, il portait constamment le jugement de ce dernier pour qui il consultait Dieu au moyen de l'urim et du thummim. Il devait, une fois par an, entrer dans le Saint des saints et offrir des sacrifices d'animaux pour ses propres péchés et pour ceux du peuple. Par la suite, Jésus-Christ est devenu premier prêtre à perpétuité en s'offrant comme victime expiatoire et en présentant son sang une fois pour toutes dans le Saint des saints du temple céleste. Voir \vref{Ex. 28:30}~; \vref{Ex. 29:9}~; \vref{Esd. 2:63}~; \vref{No. 35:25}~; \vref{Hé. 4:14-16}~; \vref{Hé. 7:25-28} et \vref{Hé. 9:6-12,24-28}.

\DicoEntry{PRÉTOIRE}\textit{, du grec «~praitorion~»~: «~quartier général dans un camp romain, la tente du commandant en chef~»}\newline
Dans les évangiles, lieu de résidence des gouverneurs dans lequel se trouvaient notamment un tribunal et une prison. Voir \vref{Mt. 27:27}~; \vref{Jn. 18:28-29} et \vref{Ac. 23:35}.

\DicoEntry{PRIÈRE}\textit{, de l'hébreu «~palal~»~: «~intervenir, s'interposer, prier~» ou «~'athar~»~: «~prier, supplier, implorer~» et du grec «~proseuche~»~: «~prière adressée à Dieu~» ou «~parakaleo~»~: «~appeler à, convoquer, supplier, exhorter~»}\newline
Acte par lequel on s'approche de Dieu et on instaure un dialogue avec lui, en ayant foi dans sa présence et son action. Invité à prier constamment, le chrétien peut le faire pour se repentir, intercéder en faveur d'une situation particulière, demander quelque chose à Dieu, le remercier, le louer ou tout simplement lui exprimer son amour. La prière garde les enfants de Dieu dans la paix. Dieu connaissant toutes les pensées de l'homme, le plus important dans la prière reste l'écoute de la voix de Yahweh. Voir \vref{Ge. 20:17}~; \vref{1 S. 2:1}~; \vref{Job 22:27}~; \vref{Mt. 14:36}~; \vref{Ac. 16:9}~; \vref{1 Th. 5:17}~; \vref{Ph. 4:6-7} et \vref{1 Pi. 4:7}.

\DicoEntry{PROPHÈTE}\textit{, de l'hébreu «~nabiy'~»~: «~l'homme qui parle, celui qui est appelé, qui a reçu une inspiration~» et du grec «~prophetes~»~: «~celui qui interprète des oracles~», «~quelqu'un qui déclare ce qu'il a reçu par inspiration~»}\newline
Sous l'Ancienne Alliance, Dieu suscita de nombreux prophètes oints de l'Esprit afin qu'ils annoncent des messages particuliers et conduisent le peuple à l'obéissance et à la crainte de Yahweh. Sous la Nouvelle Alliance, il existe au moins trois types de prophètes. Le premier concerne ceux et celles qui prophétisent au sein des assemblées locales (\vref{Ac. 21:8-9}~; \vref{1 Co. 14:29-32}), ils exhortent, édifient et consolent le peuple (\vref{1 Co. 14:1-3}). Le deuxième concerne les personnes qui ont reçu la charge d'enseigner, poser les fondements, implanter des assemblées selon \vref{Ep. 4:11}. Parmi ces prophètes, on compte Barnabas, Siméon, Lucius de Cyrène, Manahen, Saul (\vref{Ac. 13:1-5}), Jude et Silas (\vref{Ac. 15:32-33}). Le troisième concerne tous les chrétiens qui sont des potentiels prophètes puisqu'ils ont l'Esprit de Christ en eux (\vref{1 Co. 14:23-25}~; \vref{1 Co. 14:31}). Dieu peut se servir n'importe quel chrétien pour prophétiser, c'est-à-dire communiquer une parole inspirée. Voir \vref{Ep. 2:20} et \vref{Ep. 4:11}.

\DicoEntry{PROPHÉTIE}\textit{, du grec «~propheteia~»~: «~discours émanant de l'inspiration divine et déclarant les desseins de Dieu~»}\newline
Depuis l'effusion du Saint-Esprit, tous les chrétiens nés d'en haut peuvent prophétiser sans pour autant avoir le ministère de prophète. La prophétie est en effet un don spirituel* auquel il faut aspirer et qui est attribué par le Saint-Esprit selon la volonté de Dieu. Voir \vref{Ac. 2:16-18}~; \vref{1 Co. 12:4-10} et \vref{1 Co. 14:1}.

\DicoEntry{PROPITIATOIRE}\textit{, de l'hébreu «~kapporeth~»~: «~siège de miséricorde, lieu d'expiation~»}\newline
Couvercle de l'arche* composé d'or pur, il était surmonté de deux chérubins* d'or se faisant face au milieu desquels Yahweh siégeait et se manifestait pour donner des instructions à Israël. Une fois par an, le grand prêtre entrait dans le Saint des saints et aspergeait le propitiatoire du sang des animaux sacrifiés pour la purification des péchés d'Israël. Voir \vref{Ex. 25:17-22} et \vref{Lé. 16}.

\DicoEntry{PROSÉLYTE}\textit{, du grec «~proselutos~»~: «~un nouveau venu, un étranger~»}\newline
Personne issue d'une nation païenne s'étant agrégée au peuple d'Israël par le rite de la circoncision* et la pratique de la loi mosaïque. Voir \vref{Mt. 23:15}~; \vref{Ac. 2:10}~; \vref{Ac. 6:5} et \vref{Ac. 13:43}.

\DicoEntry{PYTHON}\textit{, du grec «~Puthon~»~: «~un serpent ou un dragon}\newline
Esprit de divination auquel Paul fut confronté en Macédoine. Voir \vref{Ac. 16:16-18}.

\DicoEntry{RABBI}\textit{, de l'hébreu «~rab~»~: «~capitaine, chef~» et du grec «~rhabbi~»~: «~maître~», «~un grand monsieur, honorable~» ou «~un enseignant~»}\newline
Les disciples appelaient Jésus «~Rabbi~». Cependant, il a exhorté la foule et les conducteurs religieux à ne pas attribuer une telle marque de distinction aux hommes rappelant que seul Yahweh est maître. Voir \vref{Mt. 23:8}~; \vref{Mc. 11:21}~; \vref{Jn. 9:2}.

\DicoEntry{RACHEL}\textit{, de l'hébreu «~Rachel~»~: «~agnelle, brebis~»}\newline
Fille de Laban, deuxième femme de Jacob pour laquelle il travailla quatorze ans. Longtemps stérile, Yahweh lui donna finalement deux garçons~: Joseph et Benjamin. Elle mourut à l'accouchement du deuxième. Voir \vref{Ge. 29:10-31}~; \vref{Ge. 30:22-24} et \vref{Ge. 35:16-19}.

\DicoEntry{RAHAB}\textit{, de l'hébreu «~Rachab~»~: «~large, spacieux, tumultueux~»}\newline
Prostituée habitant Jéricho, elle cacha les deux espions juifs chez elle. Grâce à son acte, Josué* lui laissa la vie sauve ainsi qu'à sa famille lorsqu'il détruisit la ville et tous ceux qui s'y trouvaient. Rahab habita ensuite au milieu d'Israël~; elle figure non seulement parmi les héros de la foi, mais aussi dans la généalogie de Jésus-Christ. Voir \vref{Jos. 2:1}~; \vref{Jos. 6:17-25}~; \vref{Mt. 1:5-16} et \vref{Hé. 11:31}.

\DicoEntry{REBECCA}\textit{, de l'hébreu «~Ribqah~»~: «~ensorcelante, qui prend au piège~»}\newline
Fille de Bethuel et sœur de Laban, elle fut l'épouse d'Isaac*. Yahweh mit fin à sa stérilité et elle donna naissance à des jumeaux, Esaü et Jacob, qui devinrent deux grandes nations. Voir \vref{Ge. 24} et \vref{Ge. 25:21-26}.

\DicoEntry{RÉCONCILIATION}\textit{, du grec «~katallage~»~: «~échange, change, ajustement d'une différence~»}\newline
Jésus-Christ mourut à la croix pour réconcilier l'homme avec Dieu, c'est-à-dire le faire passer de l'état de séparation (causée par le péché) à l'état d'intimité avec Dieu. L'Eglise a le ministère de réconciliation et doit en ce sens présenter à l'homme pécheur la voie de la réconciliation avec Dieu au travers de la prédication de l'Evangile*. Voir \vref{Ro. 5:11}~; \vref{Hé. 10:18-20} et \vref{2 Co. 5:18-20}.

\DicoEntry{RÉDEMPTION}\textit{, de l'hébreu~: «~peduwth~»~: «~rachat~» et du grec: «~apolutrosis~»~: «~libération effectuée suite au paiement d'une rançon~»}\newline
Jésus-Christ a payé le prix nécessaire au rachat des péchés de tous les hommes par son sacrifice à la croix, leur permettant d'échapper à la mort éternelle au moyen de la foi*. Voir \vref{Ro. 3:23-24}~; \vref{Col. 1:14}~; \vref{Ep. 1:7} et \vref{Hé. 9:12}.

\DicoEntry{RÉFORME}\textit{, de l'hébreu «~yatab~»~: «~agir bien~» et du grec «~diorthosis~»~: «~remettre droit~»}\newline
La plupart des prophètes du Tanakh sont des réformateurs dans la mesure où ils prônent un retour à Dieu~; le roi Josias a institué une profonde réforme pendant son règne en déployant des efforts pour revenir à l'obéissance de la Parole. Jésus-Christ est le plus grand réformateur en ce qu'il marchait à contre-courant et vint restaurer l'homme à sa condition originelle, celle d'avant la chute. Ainsi, l'homme qui reçoit Jésus entre dans un processus où il est continuellement réformé par le Saint-Esprit au travers de la Parole. Voir \vref{2 R. 22}~; \vref{Jé. 7:5}~; \vref{Jé. 26:13}~; \vref{Os. 6:1}~; \vref{Mt. 19:8} et \vref{Jn. 16:7-15}.

\DicoEntry{REPENTANCE}\textit{, du grec «~metanoia~»~: «~changement de mentalité, d'intention~», «~tristesse qu'on éprouve de ses péchés~»}\newline
Un des points majeurs de la prédication de Jean-Baptiste* puis des apôtres*. La repentance est essentielle pour obtenir la rémission des péchés et doit être accompagnée de fruits. La repentance ne concerne pas uniquement le nouveau converti, mais tout disciple de Christ qui, jusqu'à la fin de sa vie, est dans un processus de perfectionnement. Voir \vref{Mc. 1:4}~; \vref{Lu. 3:8}~; \vref{2 Co. 7:9-10}~; \vref{Ro. 2:4}~; \vref{Ac. 2:38}~; \vref{Ac. 13:24}~; \vref{Ac. 17:30} et \vref{Ac. 26:20}.

\DicoEntry{RÉSURRECTION}\textit{, du grec «~anastasis~»~: «~se lever, ressusciter de la mort~».}\newline
Christ fut le premier à expérimenter la résurrection d'entre les morts. Au son de la dernière trompette*, les chrétiens décédés ressusciteront de même avec des corps incorruptibles pour les noces* de l'Agneau. Voir \vref{Mt. 28:6}~; \vref{1 Pi. 1:3}~; \vref{Ap. 1:5}~; \vref{1 Co. 15:52} et \vref{1 Th. 4:16}.

\DicoEntry{RÉVEIL}\textit{, du grec «~egeiro~»~: «~réveiller du sommeil, revenir à la vie, se lever~»}\newline
Prise de conscience personnelle ou collective sur sa condition de péché et.ou l'imminence du jugement de Dieu. Il en résulte la repentance*, la véritable conversion*, la crainte de Dieu, la préparation à la rencontre de Yahweh. Une personne réveillée a les yeux focalisés sur Christ et peut accomplir la volonté du Seigneur. Voir Jon.~; \vref{Ep. 5:14} et \vref{Ro. 13:11-14}.

\DicoEntry{ROBOAM}\textit{, de l'hébreu «~Rhoboam~»~: «~qui affranchit le peuple~»}\newline
Fils et successeur du roi Salomon. C'est sous son règne que se produit le schisme* entre les royaumes du nord et celui du sud. Il régna sur Juda dix-sept années pendant lesquelles il fut en guerre avec le royaume du nord et fit ce qui est mal aux yeux de Yahweh. Voir \vref{1 R. 12:1-24} et \vref{1 R. 14:21-31}.

\DicoEntry{ROMAIN}\textit{, du grec «~rhome~»~: «~force~»}\newline
Pendant la vie de Jésus et pendant l'époque de l'Eglise primitive, Israël était sous la domination de l'Empire romain qui l'oppressait et lui soutirait des impôts. Paul, né à Tarse - ville romaine - put bénéficier des privilèges liés à la nationalité romaine quand il fut livré aux tribunaux. Ce dernier écrivit une lettre aux chrétiens romains - figurant dans le canon biblique - avant de les rencontrer physiquement. Voir \vref{Mt. 22:17}~; \vref{Jn. 11:48}~; \vref{Ac. 16:35-39}~; \vref{Ac. 22:25-29}~; \vref{Ac. 23:27} et \vref{Ac. 25:16}.

\DicoEntry{ROME}\textit{, du grec «~rhome~»~: «~force~»}\newline
Capitale de l'Empire romain située en Italie, Rome jouissait d'une grande notoriété à l'époque de l'Eglise primitive. Bien que l'empereur Claude* ait ordonné aux Juifs de quitter la ville, Paul manifesta le désir de s'y rendre pour y annoncer l'Evangile. Il y arriva après bien des difficultés quelques années plus tard en tant que prisonnier. Voir \vref{Ac. 18:1-2}~; \vref{Ac. 19:21}~; \vref{Ac. 23:11} et \vref{Ac. 28:14-31}.

\DicoEntry{ROYAUME DE DIEU}\textit{, du grec «~basileia~»~: «~pouvoir royal, royauté, domination, autorité~»}\newline
Lors de son service terrestre, Jésus a annoncé que le Royaume de Dieu était proche. Il parlait de son autorité sur toutes choses et de son règne. Ne consistant pas dans les choses terrestres, ce royaume se manifeste par la puissance de Dieu, la justice, la paix et la joie par le Saint-Esprit. Voir \vref{Lu. 9:1-2}~; \vref{Lu. 11:17-20}~; \vref{Lu. 17:20-21} et \vref{Ro. 14:17}.

\DicoEntry{RUBEN}\textit{, de l'hébreu «~Re'uwben~»~: «~voici un fils~»}\newline
Premier fils de Jacob et Léa, il devint le père de la tribu des Rubénites qui s'installa à l'est de la terre promise. Il perdit son droit d'aînesse après avoir eu des rapports intimes avec Bilha, concubine de son père. Voir \vref{Ge. 29:32}~; \vref{Ge. 35:22} et \vref{Ge. 49:3-4}.

\DicoEntry{RUTH}\textit{, de l'hébreu «~Ruwth~»~: «~amitié, une amie~»}\newline
Originaire de Moab, belle-fille de Naomi avec qui elle s'installa à Bethléhem. Elle y épousa Boaz avec qui elle eut un fils, Obed, grand-père du roi David*. Son histoire est racontée dans le livre portant son nom.

\DicoEntry{SABBAT}\textit{, de l'hébreu «~shabbath~»~: «~repos, cessation d'activité~»}\newline
Septième et dernier jour de la semaine consacré à Yahweh pendant lequel aucune activité ne devait être pratiquée selon la loi. Le sabbat figure dans les dix commandements, son infraction devait être punie de mort. Suscitant de vives critiques de la part des religieux, Jésus a plusieurs fois enfreint le sabbat dont il s'est déclaré le maître. Sous la Nouvelle Alliance, le sabbat se trouve en Jésus-Christ, le chrétien n'est donc pas tenu de le respecter comme ce fut le cas sous la loi de Moïse. \vref{Ex. 20:8-11}~; \vref{Ex. 31:14-15}~; \vref{De. 5:12-15}~; \vref{Mt. 11:28-30}~; \vref{Mc. 2:23-28} et \vref{Mc. 3:1-6}.

\DicoEntry{SACERDOTALISME}\textit{}\newline
Doctrine d'origine catholique reconnaissant le prêtre ou le pasteur comme l'intermédiaire entre Dieu et les hommes. Voir \vref{1 Ti. 2:5}.

\DicoEntry{SACRIFICATURE, SACERDOCE}\textit{, de l'hébreu «~kahan~»~: «~service~»}\newline
Sous la loi mosaïque, il était exercé par les Lévites descendants d'Aaron dans le tabernacle puis le temple et consistait notamment à accomplir les différents rituels relatifs aux sacrifices d'animaux et aux offrandes de toutes sortes. Depuis le sacrifice de Jésus à la croix, le sacerdoce concerne tous les enfants de Dieu qui sont non seulement les prêtres, mais aussi les sacrifices auxquels le Seigneur prend plaisir. Voir Lé.~; \vref{Ro. 12:1}~; \vref{1 Pi. 2:9} et \vref{Ap. 1:6}.

\DicoEntry{SADDUCÉENS}\textit{, du grec «~saddoukaios~»~: «~les justes~»}\newline
Parti religieux juif attaché au Pentateuque de manière stricte, ils ne croyaient ni en la résurrection des morts ni aux anges. Ils s'opposèrent au service de Jésus qui les reprit sévèrement et échappa à leurs pièges. Ils combattirent de même les apôtres qu'ils jetèrent en prison. Voir \vref{Mt. 16:6-12}~; \vref{Mt. 22:23-33}~; \vref{Ac. 5:17-19} et \vref{Ac. 23:1-10}.

\DicoEntry{SAINT}\textit{, de l'hébreu «~qodesh~»~: «~consacré, mis à part~» et du grec «~hagios~»~: «~chose très sainte, consacré, un saint~»}\newline
Dieu appela Israël à la sainteté, c'est-à-dire à ne pas se mélanger avec les autres peuples de peur d'être contaminés par leurs pratiques méchantes et idolâtres. Yahweh est le Saint d'Israël. Sous la Nouvelle Alliance, les chrétiens sont appelés saints, car le Saint-Esprit qui est en eux leur communique sa nature, les purifie et leur enseigne la haine du péché. Voir \vref{De. 7:6}~; \vref{Es. 49:7}~; \vref{1 Co. 6:11,19}~; \vref{1 Th. 4:1-8} et \vref{Hé. 12:14}.

\DicoEntry{SAINT-ESPRIT}\textit{, (voir étymologie des mots «~saint~» et «~esprit~»)}\newline
Le Saint-Esprit est l'esprit de Dieu, l'Esprit de Jésus~; il est Dieu. Lors de son service terrestre, le Seigneur déclara qu'un consolateur viendrait habiter dans les corps des croyants. Cette parole s'accomplit lors de la Pentecôte. Le Saint-Esprit a pour mission de convaincre le monde en ce qui concerne le péché, la justice et le jugement. A la naissance d'en haut, il régénère l'esprit du chrétien sur qui il dépose son sceau, gage de l'adoption. Il enseigne et guide le chrétien tout au long de sa marche avec Dieu. Il transforme son caractère et distribue les dons spirituels pour l'édification de l'Eglise. Voir \vref{1 S. 10:10}~; \vref{2 Ch. 15:1}~; \vref{Jn. 14:16-17,26}~; \vref{Jn. 16:7-15}~; \vref{Ac. 2}~; \vref{1 Co. 6:11}~; \vref{Ro. 8:9}~; \vref{1 Co. 3:16}~; \vref{1 Co. 12:4-13}~; \vref{Ep. 1:13} et \vref{Ga. 5:16,22}.

\DicoEntry{SALOMON}\textit{, de l'hébreu «~Shelomoh~»~: «~paix, pacifique~»}\newline
Fils de David, il succéda à son père et fut roi d'Israël pendant quarante ans. Il construisit le premier temple de Yahweh sur le mont Morija à Jérusalem puis un palais royal. Outre ses importantes richesses, c'est la grande sagesse que Dieu lui donna qui fit sa renommée parmi tous les peuples. Il eut sept cents femmes et trois cent concubines - dont un grand nombre de femmes étrangères - ce qui détourna son cœur de son Dieu. On lui attribue la rédaction des livres Cantique des cantiques et Ecclésiaste~; il a aussi écrit certains psaumes et plusieurs proverbes. Voir \vref{1 R. 4:29-34}~; \vref{1 R. 5-7}~; \vref{1 R. 9:15-28}~; \vref{1 R. 11:1-10,42}~; \vref{Ps. 72,127} et \vref{Pr. 25-29}.

\DicoEntry{SALUT}\textit{, de l'hébreu «~yesha'~» et du grec «~soteria~»~: «~délivrance, sûreté, sécurité~»}\newline
Libération des chaînes du péché, de la condamnation et de tout type d'asservissement spirituel, le salut est un don gratuit de Dieu qui s'obtient par la grâce, au moyen de la foi. C'est la manifestation de l'amour éternel de Dieu qui - ne voulant pas que l'homme périsse dans le feu de la géhenne - a payé le prix pour lui offrir la vie éternelle. Le salut réside dans le seul nom de Jésus-Christ. Voir \vref{Jn. 3:16}~; \vref{Ac. 4:12}~; \vref{Ro. 8:1}~; \vref{1 Th. 5:9}~; \vref{Tit. 3:4-6} et \vref{Ep. 2:4-8}.

\DicoEntry{SAMARIE}\textit{, de l'hébreu «~Shomerown~» et du grec «~Samareia~»~: «~montagne de guet~»}\newline
Située dans l'actuelle Cisjordanie, ville fondée par Omri, roi d'Israël, et qui devint la capitale du royaume du nord. La ville fut prise par Salmanasar, roi d'Assyrie, sous le règne d'Osée, roi d'Israël. Au temps de Jésus, la Samarie n'était qu'une simple circonscription romaine dont la population était issue du métissage entre Israélites et des colons assyriens. Suite aux persécutions subies par l'Eglise primitive à Jérusalem, des chrétiens s'y réfugièrent et l'Evangile s'y propagea. Voir \vref{1 R. 16:23-24}~; \vref{2 R. 3:1}~; \vref{2 R. 18:9}~; \vref{Os. 7} et \vref{Ac. 8:1-17}.

\DicoEntry{SAMARITAINS}\textit{, du grec «~samareites~»~: «~un habitant de Samarie~»}\newline
Après l'assujettissement de la Samarie par Salmanasar, roi d'Assyrie, des peuples étrangers s'y établirent et s'assemblèrent avec les Israélites. Au IVe siècle av J.-C., les samaritains construisirent un temple sur le mont Garizim, qui devint le centre religieux de Samarie, entraînant une séparation avec le reste des Juifs qui adoraient à Jérusalem. Les samaritains étaient considérés comme des étrangers et non comme de véritables juifs du fait de la mixité de leur religion. Jésus-Christ ouvrit la voie de la réconciliation avec ce peuple en racontant la parabole du bon samaritain et en annonçant la bonne nouvelle à la femme samaritaine. Voir \vref{2 R. 17:3,24-29} et \vref{2 R. 18:9}~; \vref{Jn. 4:4-26} et \vref{Lu. 10:30-37}.

\DicoEntry{SAMSON}\textit{, de l'hébreu «~Shimshown~»~: «~petit soleil~»}\newline
Fils de Manoach, de la tribu de Dan, il fut juge en Israël pendant vingt-ans. Consacré à Dieu dès le sein maternel et doté d'une force extraordinaire, il réalisa des prouesses qui suscitèrent la crainte de ses ennemis. Choisi pour être le libérateur d'Israël, il fut incompris par les siens qui ne le soutinrent pas. Il mourut suite à la trahison de Delila, une femme d'origine philistine. Voir \vref{Jg. 13-16}.

\DicoEntry{SAMUEL}\textit{, de l'hébreu «~Shemuw'el~»~: «~entendu ou exaucé de Dieu~»}\newline
Fils d'Elkana, de la tribu d'Ephraïm, et d'Anne, il fut consacré au service de Yahweh dès son plus jeune âge. Il exerça les fonctions de juge, prêtre et prophète sur Israël. Il oignit les deux premiers rois d'Israël~: Saül et David. Son histoire est racontée dans les deux livres du Tanakh portant son nom.

\DicoEntry{SANCTIFICATION}\textit{, de l'hébreu «~Qadash~» et du grec «~hagiasmos~»~: «~consécration, purification, sainteté~» ou «~l'effet de la purification~»}\newline
Fruit de l'action conjointe de la Parole et l'Esprit de Dieu dans la vie du croyant, la sanctification doit être recherchée par le chrétien tout au long de sa vie. Sans elle, nul ne verra Dieu. \vref{Jn. 17:17}~; \vref{1 Th. 4:3-8}~; \vref{Hé. 12:14} et \vref{Ap. 22:11}.

\DicoEntry{SANCTUAIRE}\textit{, de l'hébreu «~miqdash~»~: «~lieu sacré, lieu saint, sanctuaire de Yahweh~»}\newline
Le sanctuaire terrestre, dont Moïse avait reçu le modèle, était une représentation de celui qui se trouve au ciel et où Jésus alla présenter son sang. Voir \vref{Ex. 25:8-9} et \vref{Hé. 9:1-24}.

\DicoEntry{SANG}\textit{, de l'hébreu «~dam~» et du grec «~haima~»~: «~sang~»}\newline
Déterminant le lien de famille et la lignée, le sang est, selon les Ecritures, l'âme*, la vie. Ainsi, l'effusion de sang fut nécessaire pour le pardon des péchés et le sang de Christ, qui ôte définitivement le péché, donne la vie. Voir \vref{Lé. 17:11}~; \vref{Ac. 17:26}~; \vref{Ro. 5:9}~; \vref{Hé. 9:22-28} et \vref{Ap. 5:9}.

\DicoEntry{SANHÉDRIN}\textit{, du grec «~sunedrion~»~: «~conseil, tribunal~»}\newline
Désignait d'une part, les petits tribunaux se tenant dans chaque ville pour régler les affaires locales et d'autre part, le grand conseil de Jérusalem où étaient traitées les affaires plus importantes. Ce dernier était composé de soixante et onze membres sélectionnés parmi l'élite religieuse et les anciens d'Israël~; le grand prêtre en était le président. Sous la domination romaine, ce tribunal fonctionnait de manière quasi autonome~; la sentence de la peine de mort devait néanmoins être validée par le gouverneur romain. Jésus fut jugé coupable de blasphème par le sanhédrin qui le condamna à mort. Voir \vref{No. 11:16-17}~; \vref{Mt. 5:22}~; \vref{Mt. 26:59-66}~; \vref{Jn. 11:47}~; \vref{Ac. 5:21-41} et \vref{Ac. 6:12-15}.

\DicoEntry{SARA}\textit{, de l'hébreu «~Sarah~»~: «~princesse, femme noble~»}\newline
Femme d'Abraham, elle enfanta Isaac à l'âge de 90 ans selon la promesse de Yahweh. Sara figure parmi les héros de la foi~; elle mourut à cent vingt-sept ans. Voir \vref{Ge. 12:5}~; 17~; \vref{Ge. 15-16}~; \vref{Ge. 21:1-7}~; \vref{Ge. 23:1} et \vref{Hé. 11:11}.

\DicoEntry{SARDES}\textit{, du grec «~sardeis~»~: «~les rouges~», «~prince de joie~»}\newline
Capitale antique de la Lydie, Sardes se situait sur la rivière Pactole, à environ 50 km au sud de Thyatire et 75 km à l'est de Smyrne. Réputée riche et puissante en raison de ses ressources en or, ses épithètes étaient sournois car sa forteresse reposait sur un sol boueux. En effet, au VIème siècle av. J.-C., Cyrus Le Grand - vainqueur de Crésus alors roi de Lydie - s'empara de Sardes par une attaque nocturne. Par la suite, la ville subit plusieurs invasions puis un tremblement de terre en 17 ap. J.-C. L'église de Sardes fut probablement fondée par Paul au cours d'un voyage à Ephèse. Au moment où ils reçurent le message de l'ange de l'Apocalypse, il semblerait que certains chrétiens de Sardes étaient retournés au culte licencieux de Cybèle, déesse-mère et gardienne des savoirs. Ceux qui s'étaient gardés purs devaient ainsi revivifier les autres membres. Cette église symbolise l'église morte. Voir \vref{Ap. 3:1-6}.

\DicoEntry{SATAN}\textit{, de l'hébreu «~Satan~»~: «~adversaire, ennemi~»}\newline
Autrefois chérubin protecteur, il a péché en voulant s'approprier la gloire qui ne revient qu'à Dieu. Dans sa rébellion, il entraîna un tiers des anges qui furent précipités avec lui sur la terre. Connu également sous les noms «~Prince de ce monde~», «~Prince des ténèbres~», «~Belzébul~», «~le malin~», «~l'accusateur~» ou «~le diable~», il est l'adversaire des enfants de Dieu à qui il fait la guerre. Il a cependant été vaincu à la croix par Jésus-Christ, au nom duquel les chrétiens peuvent le chasser. Satan sera enchaîné pendant le millenium puis libéré pour un peu de temps. Il sera finalement jeté dans l'étang de feu pour l'éternité. Voir \vref{Ez. 28:14-19}~; \vref{Es. 14:12-17}~; \vref{Ap. 12:4}~; \vref{Lu. 10:18-19}~; \vref{Ja. 4:7}~; \vref{Jn. 16:11}~; \vref{Ap. 12:4} et \vref{Ap. 20:1-15}.

\DicoEntry{SAÜL}\textit{, de l'hébreu «~Sha'uwl~»~: «~désiré, demandé (à Dieu)~»}\newline
1. Fils de Kis, israélite de la tribu de Benjamin, il fut choisi par Dieu pour être le premier roi d'Israël sur qui il régna pendant quarante ans. Il désobéit à la loi de Yahweh et tenta plusieurs fois d'assassiner David, choisi par Dieu pour lui succéder sur le trône. Saül mourut avec ses trois fils pendant la bataille de Guilboa. Voir \vref{1 S. 10}~; \vref{1 S. 13:1-14}~; \vref{1 S. 15:10-11}~; \vref{1 S. 18:8-16}~; \vref{1 S. 19:8-17} et \vref{1 S. 31}.
\\2. Nom initial de Paul*.

\DicoEntry{SCANDALE}\textit{, de l'hébreu «~mikshowl~»~: «~trébucher~» et du grec «~skandalon~»~: «~obstacle, piège~»}\newline
Pierre qu'on rencontre et qui peut faire glisser sur le chemin ou encore situation ou comportement qui provoque un trouble emmenant quelqu'un à fauter. Le scandale n'est pas forcément une mauvaise action en soi~; Christ lui-même fut un scandale pour les Juifs. Toutefois, il reste souvent lié aux œuvres de la chair et peut être provoqué par un manque de discernement. Le chrétien doit veiller par rapport aux scandales. Voir \vref{Ps. 106:36}~; \vref{Mt. 13:41}~; \vref{Mt. 18:7}~; \vref{1 Co. 1:23} et \vref{1 Pi. 2:7-8}.

\DicoEntry{SCEAU}\textit{, du grec «~sphragizo~»~: «~mettre un sceau dessus, poser une marque par l'impression d'un sceau~»}\newline
Sous l'Ancienne Alliance, la circoncision était une marque de l'alliance établie entre Yahweh et son peuple. A la naissance d'en haut, le chrétien est scellé du Saint-Esprit, témoignant de son appartenance à Christ. Voir \vref{Ge. 17:10-11}~; \vref{Ep. 1:13}~; \vref{Ap. 7:3} et \vref{Ap. 9:4}.

\DicoEntry{SCHEOL}\textit{}\newline
Voir SÉJOUR DES MORTS.

\DicoEntry{SCHISME D'ISRAËL}\textit{}\newline
Le schisme est la séparation d'Israël en deux royaumes suite à la dérive de Salomon. En 931 av. J.-C., Roboam succéda à son père Salomon sur le trône royal et n'accepta pas d'alléger le joug que son père avait mis sur eux, cela entraîna la séparation du royaume en deux. On retrouva d'une part, le royaume d'Israël dirigé par Jéroboam - appelé aussi royaume du nord -, composé des dix tribus du nord et d'autre part, le royaume de Juda gouverné par le roi Roboam composé des deux tribus du sud (Benjamin et Juda). Voir \vref{1 R. 12:1-24}.

\DicoEntry{SCRIBE}\textit{, de l'hébreu «~caphar~»~: «~secrétaire, scribe~», «~homme instruit, qui a le savoir~»}\newline
Les scribes occupaient une position importante auprès du peuple juif, ayant non seulement une mission d'enseignement de la loi, mais également une fonction au sein de la justice juive en prenant part au sanhédrin*. Voir \vref{Esd. 7:6-10} et \vref{Mt. 16:21}.

\DicoEntry{SECTE}\textit{, du grec «~hairesis~»~: «~action de prendre, capturer~»}\newline
Groupement de personnes adhérant à une doctrine particulière et vivant marginalement, comme les sadducéens ou les pharisiens. Les premiers disciples furent qualifiés de «~secte des Nazaréens~». Pierre met en garde contre les faux prophètes qui introduisent des sectes pernicieuses pour ravir la foi des chrétiens afin de les entraîner dans la dissolution. Voir \vref{Ac. 5:17}~; \vref{Ac. 26:5}~; \vref{Ac. 24:5} et \vref{2 Pi. 2:1}.

\DicoEntry{SÉDÉCIAS}\textit{, de l'hébreu «~Tsidqiyah~»~: «~Yahweh est justice~»}\newline
Fils d'Hamoutal et oncle de Jojakin, il fut le dernier roi de Juda sur qui il régna onze ans. Son nom initial, Matthania, fut changé en Sédécias par Nebucadnetsar, roi de Babylone. Il fit ce qui est mal aux yeux de Yahweh et connut un destin tragique~: ses fils furent égorgés devant lui, Nebucadnestar lui creva ensuite les yeux, Jérusalem et le temple* furent détruits et il fut emmené captif avec le peuple à Babylone. Voir \vref{2 R. 24:17-19}~; \vref{2 R. 25:1-21}~; \vref{Jé. 21}~; \vref{Jé. 22:1-9}~; \vref{Jé. 37,38,39:6-7}.

\DicoEntry{SÉJOUR DES MORTS}\textit{de l'hébreu «~she'owl~»~: «~monde souterrain, tombe, enfer, fosse~» et du grec «~hades~»~: «~dieu des profondeurs de la terre~»}\newline
Lieu de captivité où allaient les âmes de tous les défunts avant le sacrifice de Christ. Il était scindé en deux parties séparées par un grand abîme. D'un côté se trouvait un lieu de tourments et de souffrances extrêmes accueillant tous les méchants qui ont vécu dans le péché durant leur vie terrestre et qui n'y ont pas renoncé. D'un autre côté, il y avait le sein d'Abraham où reposaient et séjournaient les âmes des justes qui avaient foi en Yahweh. Après la résurrection de Jésus, ces derniers ont été arrachés du séjour des morts par le Seigneur qui les a emmenés au paradis*. Le ciel, en tant que destination des personnes décédées, fut en effet ouvert par Christ après sa résurrection. Par conséquent, le sein d'Abraham n'a jamais accueilli de chrétiens. Le séjour des morts est à présent composé uniquement d'impies~; à la fin du monde, il sera jeté avec tous ses habitants dans l'étang de feu*. Voir \vref{Lu. 16:19-31}~; \vref{1 S. 28:6-20}~; \vref{Mt. 11:23}~; \vref{Ac. 2:27}~; \vref{Jn. 3:13}~; \vref{Ep. 4:8} et \vref{Ap. 20:14}.
\\Note~: L'histoire de Lazare et de l'homme riche racontée dans Luc \vref{16:19-31} n'est pas une parabole. A la différence de tous les récits à caractère parabolique contés dans les Ecritures, cette histoire mentionne un nom.

\DicoEntry{SEM}\textit{, de l'hébreu «~Shem~»~: «~nom, renommée~»}\newline
Fils aîné de Noé et ancêtre d'Abraham. Voir \vref{Ge. 10:1} et \vref{Ge. 11:10-27}.

\DicoEntry{SÉNEVÉ}\textit{, du grec «~sinapi~»~: «~graine de moutarde~»}\newline
Plante des régions orientales ayant la forme d'une petite semence pouvant grandir de manière exponentielle et atteignant jusqu'à trois mètres. Elle symbolise spirituellement la puissance de la foi* capable de déplacer les montagnes. Voir \vref{Mt. 13:32-33} et \vref{Mt. 17:20}.

\DicoEntry{SEPHORA}\textit{, de l'hébreu «~Tsipporah~»~: «~petit oiseau, moineau~»}\newline
Fille de Jéthro, femme de Moïse et mère d'Eliézer et de Guerschom. Elle partit dans le pays d'Egypte avec Moïse quand il répondit à l'appel de Yahweh pour aller libérer Israël. \vref{Ex. 2:15-21} et \vref{Ex. 4:18-20}.

\DicoEntry{SÉRAPHINS}\textit{, de l'hébreu «~saraph~»~: «~être majestueux avec six ailes au service de Dieu~»}\newline
Catégorie d'anges* proclamant la sainteté de Dieu. Voir \vref{Es. 6:1-7}.

\DicoEntry{SERPENT}\textit{, de l'hébreu «~nachash~»~: «~serpent, reptile~»}\newline
C'est sous la forme du serpent que Satan vint séduire Eve dans le jardin d'Eden. Le serpent fut maudit d'entre tous les animaux pour son action. Le serpent ancien ou rusé désigne le diable*~; il s'oppose au serpent d'airain, Jésus, qui a donné à ses enfants le pouvoir de marcher sur les serpents. Voir \vref{Ge. 3:1-14}~; \vref{No. 21:4-9}~; \vref{2 Co. 11:3}~; \vref{Jn. 3:14-15}~; \vref{Lu. 10:19} et \vref{Ap. 12:9,14-15}.

\DicoEntry{SERVICE}\textit{, du grec «~diakonia~»~: «~service~», dérivé du mot grec «~diakonos~»~: «~domestique~»}\newline
Tâche que le chrétien exerce au service de Dieu et des hommes selon l'onction* et le mandat que Dieu lui donne. Le serviteur de Dieu est donc un serviteur inutile, un simple instrument utilisé pour la gloire de Yahweh. Voir \vref{Lu. 17:10}~; \vref{1 Co. 12}~; \vref{2 Co. 3:5}~; \vref{Ro. 12} et \vref{1 Pi. 4:10-11}.

\DicoEntry{SERVITEUR}\textit{, du grec «~diakonos~»~: «~domestique, subordonné, messager~» ou «~doulos~»~: «~esclave~»}\newline
Christ a renoncé à sa gloire et a pris la forme d'un simple serviteur. De même, le chrétien n'est pas uniquement serviteur de Dieu, il doit comme le maître servir son prochain. Voir \vref{Mc. 10:45}~; \vref{Ph. 1:1}~; \vref{Ph. 2:5-8} et \vref{2 Co. 6:4}.

\DicoEntry{SETH}\textit{, de l'hébreu «~Sheth~»~: «~compensation, mis à la place~»}\newline
Troisième fils d'Adam et Eve~; il naquit après le meurtre de son frère Abel que Caïn avait tué. Seth fut l'ancêtre de Noé et de Jésus-Christ. Voir \vref{Ge. 4:25}~; \vref{Ge. 5:6-29} et \vref{Lu. 3:38}.

\DicoEntry{SHOFAR}\textit{, de l'hébreu «~showphar~»~: «~corne, corne de bélier~».}\newline
Instrument de musique à vent fait à partir de la corne de bélier. Voir TROMPETTE.

\DicoEntry{SIDON}\textit{, de l'hébreu «~Tsiydown~»~: «~abondance de poisson, pêche~»}\newline
Ville de l'antique Phénicie (actuel Liban) située non loin de Tyr~; on y vénérait les Baals et les Astartés. La reine Jézabel était originaire de Sidon. Voir \vref{Jg. 10:6} et \vref{1 R. 16:31}.

\DicoEntry{SILAS}\textit{, du grec «~Silas~»~: «~de la forêt, demandé~»}\newline
Prophète, compagnon d'œuvre de Paul avec qui il effectua plusieurs voyages missionnaires. Voir \vref{Ac. 15-18}.

\DicoEntry{SILO}\textit{, de l'hébreu «~Shiyloh~»~: «~lieu de repos~»}\newline
Ville située au nord-est de la tribu d'Ephraïm où les enfants d'Israël se répartirent les territoires avant la conquête de Canaan. Avant d'être placée à Jérusalem, l'arche de l'alliance se trouvait à Silo. Voir \vref{Jos. 18:10}~; \vref{Jos. 19:51}~; \vref{1 S. 3:19-21} et \vref{1 S. 4:3}.

\DicoEntry{SILOÉ}\textit{, de l'hébreu «~Shiloach~»~: «~envoyé~»}\newline
Source d'eau se trouvant au sud-est de Jérusalem. Voir \vref{Né. 3:15} et \vref{Jn. 9:6-7}.

\DicoEntry{SIMÉON}\textit{, de l'hébreu «~Shim`own~»~: «~qui écoute, qui a été entendu~»}\newline
1. Fils de Jacob et Léa. Avec Lévi, son frère, il vengea le déshonneur de sa sœur Dina, en tuant Sichem, prince de Canaan, son père Hamor, et tous leurs hommes. Il fut gardé comme otage en Egypte, lorsque Joseph voulut éprouver la sincérité de ses frères. Il devint le père de la tribu des Siméonites qui s'installèrent au sud de Canaan. Voir \vref{Ge. 29:33}~; \vref{Ge. 34}~; \vref{Ge. 42:21-38} et \vref{Jos. 19:1-9}.
\\2. Homme de foi à qui le Saint-Esprit avait promis qu'il ne mourrait pas sans avoir vu le Messie. Il rencontra Jésus lorsqu'il était enfant, à Jérusalem. Voir \vref{Lu. 2:25-35}.

\DicoEntry{SIMON}\textit{, de l'hébreu «~Shiymown~»~: «~désert~» ou «~qui entend~»}\newline
1. Simon Pierre, le nom originel de Pierre* était Simon. Voir \vref{Jn. 1:40-42}.
\\2. Simon le zélote, il faisait partie du groupuscule des zélotes avant de devenir apôtre de Christ. Voir \vref{Lu. 6:13-16}.
\\3. Simon de Cyrène, il fut contraint d'aider Jésus à porter la croix jusqu'à Golgotha. Voir \vref{Mt. 27:32}.
\\4. Simon le magicien, originaire de la ville de Samarie, il fut baptisé par Philippe et crut pouvoir acheter à prix d'argent la puissance du Saint-Esprit. Voir \vref{Ac. 8:9-24}.

\DicoEntry{SION}\textit{, de l'hébreu «~Tsiyown~»~: «~lieu desséché~»}\newline
Autre nom pour parler de Jérusalem. Sous la Nouvelle Alliance, la montagne de Sion est l'image de la Jérusalem céleste. Voir \vref{De. 4:48}~; \vref{1 R. 8:1}~; \vref{Es. 2:3}~; \vref{2 S. 5:6-7} et \vref{Hé. 12:22}.

\DicoEntry{SISERA}\textit{, de l'hébreu «~Ciycera'~»~: «~déploiement, champ de bataille~»}\newline
Chef de l'armée du roi cananéen Jabin, son armée fut vaincue par Barak et Sisera fut tué par Jaël, femme de Héber, le Kénien. Voir \vref{Jg. 4}.

\DicoEntry{SMYRNE}\textit{, du grec «~Smurna~»~: «~myrrhe~»}\newline
Cité de la côte occidentale de l'Asie Mineure, Smyrne (aujourd'hui Izmir) était située au nord d'Ephèse et réputée pour sa splendeur et ses richesses. Ses forteresses et ses tours de l'acropole évoquaient une couronne. Très unie à Rome, des cultes en l'honneur du dieu Zeus, de la déesse Cybèle, ou encore de l'empereur Tibère et sa mère Julie y étaient célébrés. Proche d'Ephèse, l'église de Smyrne fut probablement le fruit du travail apostolique de Paul. En proie à ces doctrines impies, l'église de Smyrne était fortement persécutée aussi bien par les Romains que par «~les faux Juifs~» membres «~d'une synagogue de Satan~». Sa persévérance face aux afflictions lui permit de recevoir un bon témoignage du Seigneur. Elle incarne l'église persécutée. Voir \vref{Ap. 2:8-11}.

\DicoEntry{SODOME}\textit{, de l'hébreu «~Cedom~»~: «~qui brûle~»}\newline
Ville cananéenne située dans la plaine du Jourdain à proximité de laquelle Lot s'installa après s'être séparé d'Abraham. Ses habitants étaient de grands pêcheurs devant Yahweh à un tel point qu'il détruisit la ville - avec Gomorrhe* - en faisant tomber du ciel une pluie de feu et de soufre. Lot et ses deux filles furent épargnés grâce à l'intercession* d'Abraham. Voir \vref{Ge. 13:10-13} et \vref{Ge. 19:1-29}.

\DicoEntry{SOPHONIE}\textit{, de l'hébreu «~Tsephanyah~»~: «~Yahweh a caché, protégé~»}\newline
Fils de Cuschi, descendant du roi Ezéchias, prophète de Yahweh ayant vécu au temps du roi Josias. L'ensemble de ses prophéties se trouve dans le livre portant son nom.

\DicoEntry{STOÏCIENS}\textit{, du grec «~stoikos~»~: «~appartenant au portique~»}\newline
Adeptes de la doctrine de Zénon de Kition (336 av. J.-C. – 264 av. J.-C.) qui fonda le stoïcisme à Chypre en 301 av. J.-C. Le stoïcisme était l'une des principales doctrines philosophiques de la Grèce antique avec l'épicurisme. Elle reposait sur la morale et la maîtrise de ses sentiments par une vie en conformité avec la nature. A Athènes, quelques stoïciens, accompagnés d'épicuriens, se confrontèrent à Paul, le menant à l'aréopage afin de l'interroger. Voir \vref{Ac. 17:18-20}.

\DicoEntry{SYNAGOGUE}\textit{, du grec «~sunagogue~»~: «~assemblée, lieu de réunion~»}\newline
Assemblée de Juifs réunis pour prier et écouter la lecture des Ecritures. Jésus y enseigna régulièrement pendant son service. Les apôtres annoncèrent également l'Evangile dans des synagogues. Voir \vref{Mt. 4:23}~; \vref{Mt. 9:35}~; \vref{Mc. 6:2} et \vref{Ac. 14:1}.

\DicoEntry{TABERNACLE}\textit{, de l'hébreu «~mishkan~»~: «~sanctuaire, demeure, lieu d'habitation~»}\newline
Appelée aussi tente d'assignation, habitation mobile de Yahweh construite selon le modèle que Dieu donna à Moïse dans le désert. Les Lévites en assuraient le service avec tous les ustensiles qui lui étaient dédiés. Une nuée s'élevait au-dessus du tabernacle pour signifier aux Israélites qu'ils devaient lever le camp* et poursuivre leur marche. Voir \vref{Ex. 25:8-9}~; \vref{Ex. 39:32}~; \vref{No. 1:50-51}~; \vref{Ex. 40:36-38} et \vref{1 Ch. 6:48}.

\DicoEntry{TANAKH}\textit{}\newline
Voir Introduction.

\DicoEntry{TEMPLE}\textit{, de l'hébreu «~heykal~»~: «~palais, temple, sanctuaire~» (voir illustration)}\newline
David projeta de construire un temple pour Yahweh~; son fils Salomon fut mandaté pour l'ériger en remplacement du tabernacle. Il fut détruit une première fois par les Babyloniens au VIème siècle av. J.-C. Reconstruit lors du retour d'exil des Juifs, il fut de nouveau détruit en 70 par les Romains~; il n'en reste qu'un mur aujourd'hui appelé «~mur des lamentations~». Sous la Nouvelle Alliance, Yahweh a choisi pour temple l'Eglise*, le corps de chaque chrétien en qui il vient résider à la naissance d'en haut. Voir \vref{2 S. 7}~; \vref{1 R. 6}~; \vref{2 R. 25:8-9}~; \vref{Esd. 6:15}~; \vref{Ep. 2:21-22} et \vref{1 Co. 6:19}.

\DicoEntry{TÉNÈBRES}\textit{, de l'hébreu «~chosnek~»~: «~obscurité, ténèbres, nuit, lieu caché~»}\newline
Dès la Genèse, la lumière est séparée des ténèbres qui peuvent symboliser le péché, l'ignorance et l'absence de la vie de Dieu. Véritable prison rendant les hommes captifs, les ténèbres éternelles du séjour des morts* seront pour les anges déchus, le diable et tous les méchants. Voir \vref{Ge. 1:2-5}~; \vref{2 S. 22:29}~; \vref{Ps. 107:10}~; \vref{Job 17:13}~; \vref{Ro. 13:12}~; \vref{1 Th. 5:5}~; \vref{2 Pi. 2:4}~; \vref{1 Jn. 2:11} et \vref{Jud. 1:6-13}.

\DicoEntry{TÉRÉBINTHE}\textit{, de l'hébreu «~'elah~»~: «~térébinthe ou chêne~»}\newline
Grand arbre robuste dont l'ombrage est agréable, il est répandu en Israël. Jacob enterra les dieux étrangers de sa maison sous un térébinthe. L'ange de Yahweh apparut sous un térébinthe à Gédéon. Des cultes idolâtres étaient célébrés à l'ombre de ces arbres. C'est à la vallée des térébinthes, située au sud-ouest de Jérusalem, que David tua Goliath. Voir \vref{Ge. 35:4}~; \vref{Jg. 6:11,19}~; \vref{2 S. 18:9}~; \vref{1 Ch. 10:12}~; \vref{Os. 4:13}~; \vref{Es. 57:5} et \vref{1 S. 17:1-50}.

\DicoEntry{TÉTRARQUE}\textit{, du grec «~tetrarches~»~: «~tétrarque~»}\newline
Titre donné au gouverneur d'un territoire sous domination romaine. Hérode Antipas* était le tétrarque de Galilée. Voir \vref{Mt. 14:1} et \vref{Lu. 3:1}.

\DicoEntry{THADÉE}\textit{}\newline
Voir JUDE.

\DicoEntry{THÉRAPHIM}\textit{, de l'hébreu «~teraphiym~»~: «~idolâtries, idoles~»}\newline
Amulette utilisée dans les cultes idolâtres. Rachel déroba les théraphim de son père Laban avant de quitter sa maison. Voir \vref{Ge. 31:19,34-35}.

\DicoEntry{THESSALONIQUE}\textit{, du grec «~Thessalonike~»~: «~victoire de ce qui est faux~»}\newline
Ville située au Nord de la Grèce actuelle, sur la côte de la mer Egée, elle jouissait d'une importante fréquentation puisqu'elle figurait parmi les trois ports principaux de la Méditerranée et se situait sur l'une des plus grandes routes commerciales de l'époque~: la Voie Egnatienne reliant Rome à Byzance. Sur le plan religieux, les habitants étaient polythéistes et pratiquaient une variété de cultes, dont le culte impérial. Durant trois semaines, Paul enseigna dans une synagogue* à Thessalonique~; de là, il réussit à constituer un groupe de croyants. Toutefois, une violente persécution l'obligea à quitter promptement la ville, laissant la communauté nouvellement formée vulnérable et fragile. Il écrivit deux lettres aux saints de Thessalonique qui figurent dans le canon biblique.

\DicoEntry{THOMAS}\textit{, de l'hébreu «~Ta'own~»~: «~jumeau~»}\newline
Surnommé Didyme, il était l'un des douze apôtres*. Dans un premier temps incrédule quant à la résurrection de Jésus, il confessa la Seigneurie de ce dernier lorsqu'il le vit ressuscité. Voir \vref{Lu. 6:12}~; \vref{Jn. 11:16} et \vref{Jn. 20:24-29}.

\DicoEntry{TIMOTHÉE}\textit{, du grec «~Timotheos~»~: «~qui adore, ou honore Dieu~»}\newline
Fils d'une femme juive croyante et d'un père grec. Lié à Paul comme un fils à son père, il devint l'un de ses plus fidèles collaborateurs et l'accompagna à plusieurs reprises dans ses voyages missionnaires. Malgré sa jeunesse, il lui fut confié des tâches liées à la direction des églises, notamment à Ephèse. Timothée reçut de Paul deux lettres regorgeant de conseils et d'instructions pour être un bon serviteur de l'Evangile*. Voir \vref{Ac. 16:1-3}~; \vref{Ac. 18:5}~; \vref{1 Co. 16:10} et les deux épîtres de Paul à Timothée.

\DicoEntry{TITE}\textit{, du grec «~Titos~»~: «~nourrice, honorable~»}\newline
D'origine grecque, Tite fut un fidèle compagnon d'œuvre de l'apôtre Paul. Il l'accompagna à Jérusalem, œuvra à Corinthe et en Dalmatie et s'occupa plus particulièrement de l'église de Crète. Il reçut une lettre de Paul qui figure dans le canon biblique. Voir \vref{2 Co. 8:6,23}~; \vref{Ga. 2:1}~; \vref{2 Ti. 4:10} et l'épître de Paul à Tite.

\DicoEntry{TRIBULATION}\textit{, du grec «~thlipsis~»~: «~une pression, une oppression~»}\newline
Persécution, tourment provoqué par l'annonce de l'Evangile. Inévitables pour entrer dans le royaume de Dieu*, les tribulations ont pour but de rendre le chrétien patient, joyeux et persévérant en toutes circonstances. Voir \vref{Mc. 4:17}~; \vref{Jn. 16:33}~; \vref{Ac. 14:22}~; \vref{2 Co. 6:4}~; \vref{2 Co. 8:2}~; \vref{Ph. 1:29}~; \vref{1 Th. 3:3} et \vref{2 Th. 1:4}.

\DicoEntry{TRIBUNAL}\textit{, de l'hébreu «~qahal~»~: «~assembler, convoquer~» et du grec «~bema~»~: «~tribune~»}\newline
Lieu où les hommes sont jugés afin de recevoir une sentence en fonction des actes qu'ils ont posés. Chaque être humain comparaîtra devant le tribunal de Christ afin de rendre compte pour lui-même. Voir \vref{Ro. 14:10-12} et \vref{2 Co. 5:10}.

\DicoEntry{TRINITÉ}\textit{}\newline
Doctrine selon laquelle le Dieu unique se manifesterait en trois personnes distinctes~: Père, Fils et Saint-Esprit. Inspirée des triades païennes (babylonienne, égyptienne…), cette fausse doctrine d'origine catholique apparut au IIème siècle et fut fixée aux Conciles de Nicée en 325 et de Constantinople I en 381. Elle fut largement reprise par les protestants et la plupart des mouvements chrétiens alors que ni le mot ni le concept de trinité n'apparaissent dans les Ecritures. Voir \vref{De. 6:4}~; \vref{Es. 9:5}~; \vref{Jn. 4:23-24}~; \vref{Col. 2:8-10}~; \vref{2 Th. 1:12} et \vref{1 Jn. 5:20}.

\DicoEntry{TROMPETTE}\textit{, plusieurs mots hébreux ont été traduits par trompette, les plus utilisés sont~: «~chatsotserah~»~: «~trompette, clairon~»~; «~yobel~»~: «~bélier, corne de bélier~», «~retentissant~», «~jubilé~», et «~showphar~»~: «~corne de bélier~». Plusieurs mots grecs ont aussi été utilisés, notamment «~salpigx~»~: «~une trompette~» et «~salpizo~»~: «~sonner de la trompette~».}\newline
Sous l'Ancienne Alliance, on l'utilisait pour donner un signal, publier une sainte convocation, fêter des moments de joie, signifier une victoire, chanter des cantiques en l'honneur de Yahweh, avertir et rassembler le peuple. Le son de la trompette est aussi l'image des voix prophétiques qui crient et appellent le peuple à revenir totalement à Dieu. Selon les Ecritures, lorsque la dernière trompette retentira, l'Eglise sera enlevée pour les noces. Dans le livre d'Apocalypse, la voix du Seigneur est comparée au son d'une trompette. Voir \vref{Ex. 19:13}~; \vref{Lé. 23:24}~; \vref{No. 10:9-10}~; \vref{1 Ch. 16:42}~; \vref{Ez. 33:3}~; \vref{Mt. 24:31}~; \vref{1 Co. 15:52}~; \vref{1 Th. 4:16} et \vref{Ap. 1:10}.

\DicoEntry{TYR}\textit{, de l'hébreu «~Tsor~»~: «~un rocher~»}\newline
Ville de l'antique Phénicie (actuel Liban). Hiram, roi de Tyr, donna- en échange de vivres - du bois de cèdre et du bois de cyprès à Salomon pour la construction du temple. Le roi de Tyr est une image de Satan* dans une prophétie d'Ezéchiel. Voir \vref{1 R. 5:1-12} et \vref{Ez. 28}.

\DicoEntry{UR}\textit{, de l'hébreu «~'Uwr~»~: «~flamme, éclat, feu~»}\newline
Ville de Chaldée située au sud de la Babylonie et d'où Abraham était originaire. Voir \vref{Ge. 11:27-31}.

\DicoEntry{URIE}\textit{, de l'hébreu «~Uwriyah~»~: «~Yahweh est ma lumière~»}\newline
Héthien, mari de Bath-Schéba. Il mourut sur le champ de bataille suite à une conspiration de David qui avait connu sa femme et l'avait mise enceinte. Voir \vref{2 S. 11}.

\DicoEntry{VIE ÉTERNELLE}\textit{, de l'hébreu «~aionios~»~: «~sans commencement ni fin~»}\newline
La vie éternelle est un don gratuit de Dieu, un héritage, une promesse qui commence dès la conversion au travers de la connaissance de Dieu. La vie éternelle est Christ lui-même. Voir \vref{Jn. 3:16,36}~; \vref{Ro. 2:7}~; \vref{Ro. 6:23}~; \vref{Tit. 3:7}~; \vref{1 Jn. 2:25} et \vref{1 Jn. 5:20}.

\DicoEntry{VIGNE}\textit{}\newline
Arbre cultivé pour son fruit, la vigne est assimilée à la joie à cause du vin produit par le raisin et consommé dans le cadre de festivités. Le peuple d'Israël était la première vigne de Yahweh, mais elle ne porta pas de fruits. Le royaume de Dieu est aussi associé à la vigne~: Dieu est le vigneron, Jésus-Christ est le cep et tous les enfants de Dieu sont les sarments. Tout sarment qui ne porte pas de fruits est jeté au feu, c'est-à-dire en enfer. Voir \vref{Es. 5:1-7}~; \vref{Mt. 21:33-43} et \vref{Jn. 15:1-8}.

\DicoEntry{VOILE}\textit{, de l'hébreu «~porokhet~»~: «~rideau, voile~» et du grec «~peribolaion~»~: «~une couverture, une enveloppe~»}\newline
1. Etoffe de fin lin retors qui servait de séparation entre le lieu saint et le Saint des saints. Lorsque Jésus-Christ fut crucifié, ce voile se déchira en deux, de haut en bas, ouvrant ainsi l'accès au Saint des saints. Cet événement symbolisait que, par Jésus, tout homme pouvait accéder librement à la présence du Père. Voir \vref{Ex. 26:31-33}~; \vref{Lé. 16:11-19}~; \vref{Mt. 27:50-51}~; \vref{Hé. 9:7-8} et \vref{Hé. 10:19-20}.
\\2. Pan de tissu utilisé pour se couvrir la tête dans certaines cultures. Paul expliqua que les longs cheveux étaient une gloire pour la femme et qu'ils faisaient office de voile naturel. Voir \vref{Ge. 24:65} et \vref{1 Co. 11:15}.
\\3. Au sens figuré, le voile symbolise l'intelligence obscurcie, le cœur non converti et le manque de révélation de la parole qui sont des barrières à la compréhension de la loi. Voir \vref{2 Co. 3:14-16}.

\DicoEntry{YHWH}\textit{}\newline
Aussi appelé tétragramme (mot de quatre lettres), nom avec lequel Dieu se révéla à Moïse lorsque ce dernier le rencontra pour la première fois à Horeb. Ce nom, prononcé Yahweh, signifie «~Je suis celui qui suis~» et souligne le caractère éternel de Dieu. Voir \vref{Ex. 3:1-14}.

\DicoEntry{ZABULON}\textit{, de l'hébreu «~Zebuwluwn~»~: «~habitation~»}\newline
Fils de Jacob et Léa, il devint l'ancêtre de la tribu de Zabulon. Voir \vref{Ge. 30:19-20} et \vref{No. 2:7}.

\DicoEntry{ZACHARIE}\textit{, de l'hébreu «~Zekaryah~»~: «~Yahweh se souvient~»}\newline
1. Fils de Jéroboam, roi d'Israël sur qui il régna uniquement six mois. Il fit ce qui est mal devant Yahweh et fut tué suite à une conspiration contre lui. Voir \vref{2 R. 15:8-11}.
\\2. Prophète et prêtre, fils de Bérékia et petit-fils d'Iddo. Avec le prophète Aggée, il assista Zorobabel, gouverneur de Juda, et Josué, grand prêtre, dans la restauration du temple de Yahweh au retour de la captivité des Juifs. L'ensemble de ses prophéties se trouve dans le livre portant son nom. Voir \vref{Esd. 5:1-2} et \vref{Esd. 6:14-5}.
\\3. Prêtre et père de Jean-Baptiste qu'il eut avec sa femme Elisabeth à un âge avancé. Voir \vref{Lu. 1:5}.

\DicoEntry{ZÉLOTE}\textit{, du grec «~zelotes~»~: «~celui qui est zélé~»}\newline
Patriotes juifs fervents défenseurs de la loi et des traditions ayant pour objectif de résister à l'invasion romaine. Simon, l'un des douze apôtres, en faisait partie. Voir \vref{Lu. 6:15} et \vref{Ac. 1:13}.

\DicoEntry{ZOROBABEL}\textit{, de l'hébreu «~Zerubbabel~»~: «~rejeton de Babylone~»}\newline
Fils de Schealthiel, gouverneur de Juda, il participa à la restauration du temple* de Yahweh après le retour de la captivité du peuple juif. Il figure dans la généalogie de Jésus. Voir \vref{Esd. 3:2}~; \vref{Esd. 5:2}~; \vref{Ag. 1:14}~; \vref{Mt. 1:13} et \vref{Lu. 3:27}.

\end{multicols}
\clearpage
}\markboth{}{}
% concordance
\addcontentsline{toc}{section}{Concordance}\clearpage
\begin{center}Concordance\end{center}\clearpage
% inclusion des annexes
\addcontentsline{toc}{chapter}{Annexes}\pagenumbering{Roman}\clearpage
\begin{center}Annexes\end{center}\clearpage
% nom de dieu tetragramme
\addcontentsline{toc}{section}{Le Nom de Dieu, le tétragramme YHWH}\clearpage
\begin{center}{\LARGE Le Nom de Dieu, le tétragramme YHWH}\end{center}
\begin{small}

Le tétragramme YHWH identifie clairement le nom du Dieu d'Israël. Le texte biblique hébreu laisse apparaître ces quatre lettres qui pour nous, ne sont que de simples consonnes, mais en réalité elles expriment toute la plénitude du Dieu vivant.\bigskip

« \emph{Et il lui dit : Je suis YHWH qui t'ai fait sortir d'Ur en Chaldée, afin de te donner ce pays-ci pour le posséder.} » Genèse 15:7.
Dans ce verset, Dieu se présente par son nom à Abraham. Il utilise le même nom pour se présenter à Moïse dans Ex. 3:14. Finalement, ce nom est utilisé plus de 5000 fois dans le Tanakh pour faire référence au Dieu d'Israël et est traduit par « JE SUIS ».\bigskip

Le nom divin (dans sa forme hébraïque) s'écrit avec les lettres YOD HE WAV HE.\newline
La lettre « yod », peut se prononcer par les sons « ya », « yé », « ye » ou « yi » ; le « he » qui se prononce « hé » peut être muet comme en français et servir à séparer deux sons différents ; la lettre « wav » quant à elle peut se prononcer  « o », « ou », ou encore « é ».\bigskip

Plusieurs siècles avant le début de notre ère, les Israélites s'étaient interdit de prononcer le nom de Dieu. Ils justifiaient cela par leur désir d'obéir au troisième commandement : « \emph{Tu ne prendras point le Nom de YHWH, ton Dieu, en vain ; car YHWH ne tiendra point pour innocent celui qui aura pris son Nom en vain} » (Exode 20:7), mais il s'agissait aussi d'une superstition.\newline
Ils remplaçaient le tétragramme dans le cadre liturgique, par « Adonai » qui signifie « Seigneur » ou par « Elohim » qui veut dire « Dieu », et dans un contexte plus courant, par « Hachem » qui se traduit par « Le nom ». En réalité, ce commandement n'interdisait pas la prononciation du nom de Dieu, mais en soulignait le caractère sacré et insistait sur la révérence qui lui est due.\bigskip

Aussi, plusieurs éléments permettent d'affirmer que le nom de Dieu était connu et se prononçait.\newline
Tout d'abord, on peut noter que des hommes ont prononcé et invoqué le nom de Dieu tout au long du Tanakh. C'est le cas d'Abraham dans Ge. 15:2, d'Isaac dans Ge. 26:25, de Jacob dans Ge. 32:9 et de Moïse dans la Torah. Les prophètes ont également prononcé « le nom », notamment au travers de la formule : « Ainsi parle YHWH… », présente dans les nombreux messages de Dieu à son peuple.\bigskip

Par ailleurs, il est intéressant de rappeler que plusieurs noms propres hébraïques contiennent le tétragramme ou sa forme contractée. A titre d'exemple, Elie de l'hébreu « EliYah » signifie « YHWH est mon Dieu » et Jérémie, de l'hébreu « YirmeYah» signifie « celui que YHWH a désigné ».\bigskip

Enfin, le nom de Dieu devait être prononcé lors de différents rituels et dans des versets d'injonction, comme en témoigne le passage suivant, où Dieu donna à Aaron et ses fils, l'ordre de bénir les enfants d'Israël en son nom : « \emph{YHWH parla à Moïse, en disant : Parle à Aaron et à ses fils, et dis-leur : Vous bénirez ainsi les enfants d'Israël, en leur disant : YHWH te bénisse, et te garde ! YHWH fasse luire sa face sur toi, et te fasse grâce ! YHWH tourne sa face vers toi, et te donne la paix ! Ils mettront donc mon Nom sur les enfants d'Israël, et je les bénirai.} » Nombre 6:22-27.\bigskip

De ~200 à ~1000 après J.-C, les massorètes, des savants juifs « maîtres de la tradition », travaillèrent à la conservation des écrits du Tanakh. Ainsi, ils  fixèrent la prononciation des mots et les sens des textes en ajoutant des voyelles. Ils placèrent sous les quatre consonnes du tétragramme les voyelles incluses dans le nom « Adonaï », indiquant qu'il fallait lire le titre et non pas le nom sacré.\bigskip

Pour s'aligner sur la tradition hébraïque qui utilisait le terme « Adonai », les traductions de la Vulgate et de la Septante ont remplacé, dans le Tanakh, le tétragramme YHWH par « Seigneur ». Aussi, les textes grecs communément appelés « Nouveau Testament », mentionnant des extraits du Tanakh, ont traduit le tétragramme par « Kurios » qui signifie « Seigneur » ou « Theos » qui veut dire « Dieu ».\bigskip

En 1533, encouragé par la Réforme naissante, Pierre Robert Olivétan (1506-1538), érudit et humaniste français, entreprit une traduction de la bible à partir des textes massorétiques pour le Tanakh et des écrits d'Érasme de Rotterdam pour les Evangiles et le Testament de Jésus. Dans cette entreprise, Olivétan choisit de remplacer le tétragramme par « l'Eternel » et le justifia ainsi dans la préface de sa version de 1535 :\newline
« \emph{Désirant montrer la vraie propriété et signification de ce mot \textbf{YHWH (...) je l'ai exprimé} selon son origine, au plus près qu'il m'a été possible \textbf{par le mot Éternel}. Car YHVH vient de \textbf{HWH} qui veut dire « est ». Or, il n'y a que lui qui soit vraiment et qui fasse être toute chose (...) De le nommer comme les Juifs Adonaï c'est-à-dire Seigneur, ce n'est pas remplir et satisfaire à la signification et majesté du mot. Car Adonaï en l'Ecriture est communicable, étant aux hommes comme à Dieu. Mais \textbf{Yahvé} est incommunicable, ne se pouvant approprier et attribuer, sinon qu'à Dieu seul selon son essence.} ».\bigskip

Beaucoup de versions françaises de la Bible ont suivi les travaux d'Olivétan et ont retranscrit le tétragramme par « l'Eternel ».\bigskip

Dans sa vocation d'un retour aux sources, la version de la Bible de Jésus Christ a pris le parti de se rapprocher au plus près des textes originaux tout en rendant leur sens accessible aux lecteurs ; les quatre lettres du nom de Dieu ont été retranscrites dans leur équivalent en alphabet latin : YHWH. Il a donc été décidé d'opter pour la vocalisation la plus largement répandue du tétragramme, à savoir YaHWeH, qui est considérée comme sa prononciation la plus probable. Ainsi, c'est sous la forme « Yahweh » que le nom de Dieu apparait dans cette bible. Le nom divin apparait donc dans les écrits du Tanakh.\bigskip

Lorsque des extraits du Tanakh sont cités dans les évangiles et le Testament de Jésus, le tétragramme figure sous la forme « Seigneur » ou « Dieu » conformément aux textes majoritaires grecs. Aux renvois de versets, des commentaires ont été ajoutés dans cette édition (ex. : Luc 4:18-19) pour permettre au lecteur de faire le lien entre les différents textes et comprendre que de Genèse à Apocalypse, il n'est question que d'un seul message, un seul Dieu, un seul Sauveur.\bigskip

« \emph{Déclarez-le, et faites-les approcher ! Qu'ils prennent conseil ensemble ! Qui a fait entendre ces choses dès l'origine, et les a déclarées dès longtemps ? N'est-ce pas moi, Yahweh ? Or il n'y a point d'autre Dieu à part moi ; un Dieu juste et un Sauveur, il n'y en a pas d'autre à part moi. Vous tous qui êtes aux extrémités de la terre, regardez vers moi, et soyez sauvés ; car JE SUIS Dieu, et il n'y en a point d'autre.} » Esaïe 45:21-22.
\end{small}
\clearpage
% histoire bible
\addcontentsline{toc}{section}{Histoire de la Bible}\clearpage
\begin{center}{\LARGE Histoire de la Bible}\end{center}
\begin{small}

Dans notre temps, la Bible est le principal témoignage de Dieu adressé à l'Homme. Elle est ainsi la source première de la foi (Ro. 10:17), et l'unique autorité en matière de doctrine (Ap. 22:18-19). Il est donc primordial de s'assurer que la Bible que nous possédons est fidèle aux textes originaux, et telle que Dieu a voulu nous la communiquer.\bigskip

Le lecteur doit être averti qu'en tout temps l'ennemi a tenté de corrompre ou faire disparaître les Écritures. Depuis Néron (I\up{er} s.) ou Dioclétien (VI\up{ème} s.), jusqu'aux actuelles sociétés bibliques, dont malheureusement la plupart sont corrompues, en passant par le catholicisme privant le peuple de la Bible.\bigskip

C'est pourquoi nous voulons retracer son Histoire, afin que le lecteur soit en possession des informations lui permettant de discerner les tentatives de falsification, et qu'il puisse être assuré d'avoir entre ses mains une Bible digne de foi, transmettant fidèlement la saine doctrine.

\section*{Son écriture}

La Bible est constituée de 66 livres écrits par environ 40 auteurs différents sur une période de plus de 1500 ans. Mais malgré cette diversité et cet étalage dans le temps, elle conserve une parfaite cohérence de Genèse à Apocalypse, ce qui est le premier témoignage de son inspiration divine.\bigskip

Car effectivement, si la Bible a été écrite par des hommes, elle est l'œuvre de Dieu qui leur a inspiré chacun des mots à écrire. Elle est la parole divine écrite par des hommes sous son contrôle.

\begin{quote}
« \emph{Toute l'Ecriture est \textbf{inspirée de Dieu} et utile pour enseigner, pour convaincre, pour corriger et pour instruire selon la justice, afin que l'homme de Dieu soit accompli et parfaitement instruit pour toute bonne œuvre.} » 2 Timothée 3:16
\end{quote}

L'expression « \emph{inspirée de Dieu} » est la traduction du mot grec THEOPNEUSTOS, THEO signifiant « Dieu » et PNEU signifiant « souffler ».

\begin{quote}
« \emph{Nous avons aussi la parole des prophètes qui est très ferme, à laquelle vous faites bien d'être attentifs, comme à une lampe qui brille dans un lieu obscur, jusqu'à ce que le jour vienne à paraître et que l'Etoile du matin se lève dans vos cœurs ; sachant premièrement ceci, qu'aucune prophétie de l'Ecriture ne procède d'une interprétation particulière, car la prophétie n'a jamais été autrefois apportée par la volonté humaine, mais les saints hommes de Dieu ont parlé, étant \textbf{poussés par le Saint-Esprit}.} » 2 Pierre 1:19-21
\end{quote}

C'est ainsi que depuis Moïse jusqu'à Jean, les paroles de Dieu ont été écrites par leurs auteurs, ou leurs scribes, sur différents supports tels que le papyrus, le parchemin (peau d'animal) et le papier. De ces écrits orignaux, il ne reste aujourd'hui plus aucune trace, mais seulement des copies, copies de copies, etc. ainsi que des traductions anciennes et des citations.

\section*{Constitution du canon biblique}

Le mot « canon », qui vient du grec KANON signifiant « règle, modèle », lui-même emprunté à l'hébreu QANEH « roseau, mesure, canne » (Philippiens 3:16). En raison du besoin évident de définir quels écrits étaient inspirés de Dieu ou non, les juifs dans un premier temps, et les chrétiens également par la suite, ont établis des listes de livres reconnus comme tels. C'est ce qui est appelé le « canon biblique ».

\subsection*{TaNaKh}

Pour ce qui est du TaNaKh (ou Bible Hébraïque), appelé de manière erronée « Ancien Testament » dans la plupart des bibles, la tradition juive veut que le canon ait été fixé par Esdras au V\up{ème} siècle (Né. 8:1), assisté par Néhémie, les prophètes Aggée, Zacharie et Malachie. Toujours selon la tradition juive, Esdras aurait fondé la « Grande Assemblée » réunissant ces derniers et d'autres « Sages », nommés également « soferim » (scribes), dont la mission a été de déterminer quels écrits étaient inspirés, et donc établir un canon.\bigskip

Toutefois ce n'est que vers l'an 90, au concile juif de Jamnia, que le canon définitif de la Bible Hébraïque fût arrêté. Rejettant au passage les apocryphes inclus dans la Septante (ou LXX, une Bible Hébraïque en grec).

\begin{quote}
« \emph{Quel est donc l'avantage du Juif, ou quelle est l'utilité de la circoncision ? Il est grand en toute manière, surtout en ce que \textbf{les oracles de Dieu leur ont été confiés}.} » Romains 3:2
\end{quote}

Paul déclare ici que les paroles de Dieu ont été confiées aux juifs, nous devons donc pour le TaNaKh nous baser sur ce que ces derniers ont considéré comme authentique.\bigskip

Les livres apocryphes (comme Tobit, Judith, I et II Maccabées, Sagesse de Salomon, Ecclésiastique et Baruch) que nous pouvons trouver dans certains versions de Bibles (Jérusalem, TOB, etc.) sont à rejeter car ils ne sont pas inspirés de Dieu. S'il peuvent avoir un intérêt sur le plan historique, ils ne peuvent en aucun cas faire autorité en matière de doctrine. D'autant plus que certains d'entre eux contiennent des erreurs et/ou des contradictions. Pour exemple, la fausse doctrine du purgatoire trouve son origine dans un passage du livre des Maccabées.

\subsection*{Évangiles et épîtres}

Pour les chrétiens, dès les débuts de l'Église, l'autorité des écrits des apôtres était reconnue. Certains passages en témoignent :

\begin{quote}
« \emph{Je vous en conjure par le Seigneur que cette épître soit lue à tous les saints frères.} » 1 Thessaloniciens 5:27
\end{quote}

\begin{quote}
« \emph{[…] de ne pas vous laisser subitement ébranler dans votre entendement, ni troubler par une inspiration, ni par une parole, ou \textbf{par quelque lettre qu'on dirait venir de nous}, comme si le jour de Christ était déjà là.} » 2 Thessaloniciens 2:2
\end{quote}

\begin{quote}
« \emph{Et quand cette lettre aura été lue entre vous, faites en sorte qu'elle soit aussi lue dans l'église des Laodicéens, et que vous lisiez aussi celle qui viendra de Laodicée.} » Colossiens 4:16
\end{quote}

\begin{quote}
« \emph{Et considérez la patience du Seigneur comme une preuve qu'il veut votre salut ; comme Paul, notre frère bien-aimé, vous en a écrit, selon la sagesse qui lui a été donnée; ainsi que dans toutes ses lettres, il parle de ces points, dans lesquels il y a des choses difficiles à comprendre, que les ignorants et les mal affermis tordent, comme ils tordent aussi les autres Ecritures, à leur propre perdition.} » 2 Pierre 3:15-16
\end{quote}

Après la mort des apôtres et des témoins direct de la vie de Jésus, le besoin est vite apparu de regrouper et déterminer les écrits considérés comme inspirés de Dieu. Pour l'Église primitive, principalement deux critères ont été considérés comme déterminants :\newline
- l'autorité apostolique, c'est à dire que l'un des apôtres associe son autorité aux écrits ;\newline
- et la valeur déjà reconnue dans les églises locales qui en étaient destinatrices des écrits.\bigskip

Ainsi, les grandes étapes de la formation du canon sont les suivantes :\newline
- à la fin du premier siècle, tous les écrits sont reconnus par une partie ou un autre de l'Église ;\newline
- autour de l'an 200, le « Fragment de Muratori » témoigne que des listes de livres circulent et sont en discussions ;\newline
- vers 230, les écrits d'Origène donnent une liste des livres reconnus dans les église d'Orient ;\newline
- vers 340, Eusèbe de Césarée confirme la liste de Origène ;\newline
- vers 367, Athanase cite également la même liste ;\newline
- les conciles d'Hippone (343) et de Carthage (397 et 419) ne feront que confirmer cette liste.\bigskip

La valeur accordées par les chrétiens des premiers siècles aux écritures considérées comme authentiques et inspirées, et le soin qu'ils ont attaché à les recopier et les diffuser, a servie le dessein de Dieu pour préserver sa parole et lui faire traverser les âges.

\section*{Transmission}

Jusqu'à l'invention de l'imprimerie au XV\up{ème} siècle, la copie des Écritures était le travail de scribes. C'est ainsi que nous possédons plusieurs miliers de manuscrits, dont des bibles plus ou moins complètes, des livres entiers et de simples fragments. Également, pour témoigner des textes hébreux et grecs originaux, nous pouvons compter sur diverses traductions anciennes, ainsi que sur les citations des Écritures faites par les chrétiens des premiers siècles dans leurs échanges.\bigskip

Cette multitude de sources nous permet d'être assurés de posséder des textes fidèles aux originaux, les éventuelles variantes et erreurs des uns étant corrigées par la majorité des autres. Quand bien même certaines variantes posent davantage de difficultés que d'autres, aucune ne change la doctrine et le message global de la Bible.\bigskip

Le texte biblique est le texte de mieux conservé de toute l'Antiquité.

\subsection*{TaNaKh}

Comme il a été dit, malgré la multitude de manuscrits à disposition, il ne reste plus aucune trace des « autographes » (c'est à dire écrits par les auteurs eux-mêmes). À ce jour, le plus ancien manuscrit complet du TaNaKh est le Codex de Léningrad, daté de l'an 1 008. Cependant une multitude de fragments plus anciens, environ 3 000, permettent de comparer et d'attester l'invariance globale des textes à travers les âges, particulièrement les manuscrits de Qumrân découverts en 1947 et rédigés entre le III\up{ème} et I\up{er} siècle avant Jésus-Christ.\bigskip

Le TaNaKh a été écrit en hébreu, à l'exception de quelques passages en araméens. Originellement l'hébreu ne possède pas de voyelles, faisant que la prononciation de certains mots commençait à se perdre au fil du temps. Par exemple, le tétragramme que nous rendons par « Yahweh » s'écrivait « YHWH ». C'est ainsi qu'à partir du 6\up{ème} siècle des écoles de scribes, les massorètes, font leur apparition, avec pour objectif de perpétuer la Massorah. C'est à dire de préserver le texte original et son sens. Les soferim s'étaient attachés à préserver le contenu et la forme des textes, après eux les anoraïm à placer des séparateurs entre les mots afin de les distinguer, et quant aux les massorètes à préserver la prononciation en plaçant des points-voyelles sur chacun des mots et une forme de ponctuation.\bigskip

Les règles que les massorètes devaient respecter étaient très stricte :

\begin{quote}
« \emph{Un rouleau utilisé dans la synagogue doit être écrit sur des peaux d’animaux purs et préparées spécialement par un Juif pour cet usage. Ces peaux doivent être attachées avec des fils pris d’animaux purs. Chaque peau doit porter un certain nombre de colonnes, constant dans tout le codex. La longueur de chaque colonne ne doit pas être de moins de 48 lignes ni plus de 60 lignes. La largeur doit être de 30 lettres. Il faut dans un premier temps tracer des lignes sur toute la copie; si trois mots sont écrits sans une ligne, la copie est nulle. L’encre doit être noire et non rouge, ni verte, ni aucune autre couleur; et elle doit être préparée selon la manière spécifiée. Le scribe doit copier à partir d’une autre copie authentique, sans dévier. Aucun mot, aucune lettre, même pas un yod, ne doit être écrit de mémoire, sans regarder le texte devant soi. (...) Entre les consonnes, le scribe doit mettre un espace de la largeur d’un cheveu ou d’un fil ; entre les mots, de la largeur d’une consonne étroite; entre les parashah, ou sections, de la largeur de neuf consonnes; entre les livres, trois lignes. Le cinquième livre de Moïse doit s’achever exactement à la fin d’une ligne, mais ceci n’est pas obligatoire pour les autres. En plus, pour écrire le scribe doit être assis dans son habillement juif formel, il doit se laver tout le corps, il ne lui est pas permis de commencer d’écrire le nom de Dieu avec une plume nouvellement trempée dans l’encre, et même si un roi lui adresse la parole pendant qu’il écrit ce nom, il ne doit lui prêter aucune attention. (...) Les rouleaux où ces règlements ne sont pas respectés doivent être soit enterrés soit brûlés; ils peuvent néanmoins être relégués aux écoles pour y être utilisés comme livres de lecture.} »
\end{quote}

Ils utilisaient également un système de comptage afin de déceler les éventuelles erreurs. Par exemple, le nombre de versets, de mots et de lettres dans chaque livre était compté, ils relevaient également la lettre, le mot et le verset médians de chaque livres. De fait, la possibilité de laisser passer une erreur dans leurs manuscrits était très faible.\bigskip

C'est aux massorètes qu'on doit le Codex de Léningrad, ainsi que le Codex d'Alep qui lui est antérieur, qui ont servi à réaliser la « Biblia Hebraica Stuttgartensia ». C'est sur la base de ce fidèle texte massorétique que la plupart des bibles traduisent le TaNaKh.

\subsection*{Évangiles et épîtres}

Pour les Évangiles et les épîtres, on dénombre plus de 5 500 manuscrits grecs et plus de 20 000 manuscrits de versions (traductions) anciennes. À titre de comparaison, pour « L'Iliade et l'Odyssée » d'Homère on ne dénombre que 643 manuscrits et pour « Guerres de Gaules » de Jules César qu'une dizaine. Pourtant, encore aujourd'hui, il y a davantage d'incrédules pour douter de l'authenticité de la Bible que de ces autres livres.\bigskip

Comme il a été dit précédemment, dès les débuts de l'Église l'autorité des écrits des apôtres était reconnue, et ils ont très vite été considérés comme inspirés de Dieu. C'est donc avec tout autant de crainte et de soin que les premiers chrétiens ont dû recopier ces textes, dans le but de les préserver et les diffuser. C'est ce dont témoigne le nombre impressionnant de manuscrits à notre disposition, dont certains remonterait au I\up{er} siècle, comme le « papyrus d'Oxford » (\~50 apr. J.-C.) comportant des extraits de l'Évangile de Matthieu.\bigskip

En plus de l'ensemble des manuscrits grecs, dont la rédaction s'étale du I\up{er} au XV\up{ème} siècle, nous pouvons nous appuyer sur des traductions antiques en latin (Codex Bobiensis, Codex Vercellensis), éthiopien, slave, arménien, syriaque (Codex Syro-Sinaïticus, Codex Syro-Curetonianus), copte, etc. Toutes ces versions anciennes, dont certains manuscrits remontent au III\up{ème} siècle, témoignent de l'immuabilité des textes à travers les âges, et aident aujourd'hui à les traduire correctement dans les langues modernes.\bigskip

Pour témoigner de l'état des textes originaux, il est également possible de compter sur les citations des « Pères de l'Église » qui permettraient à elles-seules de reconstituer 46\% des Évangiles et épîtres.

\subsubsection*{Les différences entre manuscrits et versions}

Bien que sur l'ensemble des variantes existantes entre les différents manuscrits, aucune ne remet en cause la doctrine globale des Écritures, il existe deux écoles à ce sujet. Parmi les manuscrits on distingue globalement deux sources : les manuscrits majoritaires (ou byzantins), et les manuscrits minoritaires (ou alexandrins). Chaque « école » prenant parti pour un corpus de textes ou un autre.\bigskip

À titre d'exemple, pour 1 Timothée 3:16, la version Louis Segond 1910, basée sur le texte minoritaire, porte :
\begin{quote}
« \emph{Et, sans contredit, le mystère de la piété est grand : \textbf{celui} qui a été manifesté en chair, justifié par l'Esprit, vu des anges, prêché aux Gentils, cru dans le monde, élevé dans la gloire.} »
\end{quote}

Quand la Bible de Jésus-Christ, basée sur le texte majoritaire, porte :
\begin{quote}
« \emph{Et sans contredit, le mystère de la piété est grand : \textbf{Dieu} a été manifesté en chair, justifié par l'Esprit, vu des anges, prêché aux Gentils, cru dans le monde, et élevé dans la gloire.} »\newline
\end{quote}

\textbf{Le texte minoritaire}\bigskip

Les manuscrits dits « minoritaires » sont appelés ainsi car ils ne représentent que 5\% des manuscrits disponibles, on les appelle également « alexandrins » car pour la plupart ils seraient de source égyptienne.\bigskip

Étonnament, malgré leur infériorité numérique, c'est sur ceux-ci que la plupart des Bibles modernes se basent. En effet, ces manuscrits sont plus anciens, et également mieux conservés, que les « majoritaires ». Par exemple, le Codex Vaticanus est daté du milieu du IV\up{ème} siècle, le Codex Sinaïticus également, et le Codex Alexandrinus du V\up{ème} siècle.\bigskip

En raison de cette ancienneté, la critique moderne pense donc qu'ils sont plus proche des manuscrits originaux et lui accorde davantage de crédit. C'est ainsi que, suivant cette pensée, Westcott et Hort ont compilé ces manuscrits pour publier en 1881 le texte grec sur lequel la plupart des bibles modernes se basent. Plus tard, Eberhard Nestle et après lui Kurt Aland en feront une révision qui lui restera cependant très proche.\bigskip

Outre la spiritualité et la doctrine plus que douteuses de Wescott et Hort, les textes minoritaires prétendument meilleurs sont en vérité de piètre qualité.\bigskip

Le plus important d'entre eux, le Codex Vaticanus, est sorti mystérieusement des bibliothèques du Vatican en 1481. Il lui manque la majorité du livre de la Genèse, une partie des Psaumes, Matthieu 6:2-3, les épîtres à Timothée, Tite et Philémon, et l'Apocalypse dans son intégralité. Il fourmille également de multiples erreurs, notamment historiques et scientifiques. Quant à lui, le Codex Sinaïticus a été retrouvé par Tischendorf dans les poubelles d'un monastère au pied du Mont Sinaï. C'est dire la valeur que les moines accordaient à ce dernier.\bigskip

John William Burgon, grand défenseur des textes bibliques et particulièrement des textes majoritaires, dira :
\begin{quote}
« \emph{S’ils avaient été des manuscrits valides, il y a longtemps qu’une lecture assidue les aurait réduit en pièces. Nous soupçonnons que ces manuscrits sont redevables pour leur conservation, et ce uniquement à leur côté diabolique; […] Cela démontre que l’Église les a rejetés sans les lire. Autrement, ils auraient été usés par trop de lecture et seraient disparus} »\newline
\end{quote}

\textbf{Le texte majoritaire}\bigskip

Les manuscrits dits « majoritaires » sont appelés ainsi car ils représentent la grande majorité des manuscrits disponibles, on les appelle également « byzantins » car pour la plupart ils sont de source orientale.\bigskip

C'est de ces manuscrits, précisement sept d'entre eux, qu'en 1516 Érasme va faire une compilation. Celui-ci subira plusieurs révisions en 1519 (Luther basera sa traduction allemande sur cette revision) et 1522. En 1550 et 1551 par Robert Estienne qui y introduira la division en versets. En 1598 par Théodore de Bèze. Et en 1633 par les frères Elzévir, qui y apposeront la mention « \emph{Textum ergo habes, nunc ab omnibus receptum} », c'est à dire « \emph{Tu as donc le texte maintenant reçu par tous} ». On parle depuis de « Textus Receptus », le \textbf{Texte Reçu}.\bigskip

C'est sur celui-ci que les bibles fidèles telles que la Bible de Olivétan, de l'Épée, Martin, Ostervald, etc. se baseront.\bigskip

En plus de suivre la majorité des manuscrits, le Texte Reçu s'accorde également avec les traductions anciennes comme la Peshitta, la Bible Italique, la Bible Vaudoise et la Vulgate, ainsi que les citations des Écritures faites par les « Pères de l'Église ». C'est donc de loin le texte le plus fiable et le plus proche des originaux que nous possédons.

\section*{Les bibles françaises}

La pratique de la traduction des Écritures était quelque chose d'admis et pratiqué déjà chez les juifs. La Septante, traduction grecque du Tanakh en est un témoignage. De même chez les chrétiens de premiers siècles avec la Peshitta et la Vulgate. Cependant, au fil du temps le Catholicisme va priver le peuple de la Bible, et imposer la Vulgate latine comme la seule version autorisée et reconnue comme inspirée.\bigskip

Il va falloir attendre le XII\up{ème} siècle pour voir émerger de nouvelles tentatives de traductions en langue populaire. C'est Vaudès (ou Pierre Valdo), un riche marchand lyonnais qui, désirant se consacrer au Seigneur suite à la mort d'un ami, vendit tous ses biens et finança la traduction de plusieurs livres de la Bible en francoprovençal. Vaudès s'opposait également à la distinction entre clergé et laïc, la doctrine de la transsubstantiation dans l'eucharistie, l'attachement de l'Église au biens temporels, etc. Ce qui valu, à lui et ses disciples, d'être excomuniés en 1184 par le concile de Vérone. Il donna naissance à un mouvement appelé à ses débuts les « Pauvres de Lyon » et plus tard les « Vaudois ».\bigskip

Ce sont ces mêmes vaudois qui, au XVI\up{ème} siècle, en se joignant à la Réforme protestante, vont également financer la première traduction française de la Bible basée sur les textes hébreux et grecs. C'est Pierre Robert Olivétan, qui réalisera ce travail en se basant principalement sur les textes hébreux des Massorètes et le texte grec de Érasme. Elle sera publié en 1535.\bigskip

Cette Bible d'Olivétan accompagnera les réformateurs français, et sera révisée par Jean-Calvin et Théodore de Bèze en 1560 (Bible de l'Épée), par les frères Elzévir et des pasteurs de Genève en 1669 (Bible de Genève), par David Martin en 1707 (Bible Martin), et par Jean-Frédéric Ostervald en 1744.
\end{small}
\clearpage
\addcontentsline{toc}{section}{Les différentes traductions de la Bible}\clearpage
\begin{center}Tableau Bibles\end{center}\clearpage
% conciles & pères
\addcontentsline{toc}{section}{Conciles et Pères de l'Eglise}\clearpage
\begin{center}Conciles et Pères de l'Eglise 1\end{center}\clearpage
\begin{center}Conciles et Pères de l'Eglise 2\end{center}\clearpage
\begin{center}Conciles et Pères de l'Eglise 3\end{center}\clearpage
\begin{center}Conciles et Pères de l'Eglise 4\end{center}\clearpage
\begin{center}Conciles et Pères de l'Eglise 5\end{center}\clearpage
% denominations
\addcontentsline{toc}{section}{Dénominations}\clearpage
\begin{center}Dénominations 1\end{center}\clearpage
\begin{center}Dénominations 2\end{center}\clearpage
% doctrines
\addcontentsline{toc}{section}{Fausses doctrines}\clearpage
\begin{center}Doctrines 1\end{center}\clearpage
\begin{center}Doctrines 2\end{center}\clearpage
\begin{center}Doctrines 3\end{center}\clearpage
\begin{center}Doctrines 4\end{center}\clearpage
\begin{center}Doctrines 5\end{center}\clearpage
\begin{center}Doctrines 6\end{center}\clearpage
\begin{center}Doctrines 7\end{center}\clearpage
\begin{center}Doctrines 8\end{center}\clearpage
\begin{center}Doctrines 9\end{center}\clearpage
\begin{center}Doctrines 10\end{center}\clearpage
\begin{center}Doctrines 11\end{center}\clearpage
% monnaies
\addcontentsline{toc}{section}{Monnaies}\clearpage
\begin{center}Monnaies\end{center}\clearpage
% longueurs / liquides
\addcontentsline{toc}{section}{Longueurs / Liquides}\clearpage
\begin{center}Longueurs / Liquides\end{center}\clearpage
% poids
\addcontentsline{toc}{section}{Poids}\clearpage
\begin{center}Poids\end{center}\clearpage
% fetes
\addcontentsline{toc}{section}{Fêtes de Yahweh}\clearpage
\begin{center}Fêtes de Yahweh\end{center}\clearpage
% alphabet hebreu
\addcontentsline{toc}{section}{Alphabet hébreu}\clearpage
\begin{center}Alphabet hébreu\end{center}\clearpage
% alphabet grec
\addcontentsline{toc}{section}{Alphabet grec}\clearpage
\begin{center}Alphabet grec\end{center}\clearpage
% chrono jesus
\addcontentsline{toc}{section}{Chronologie autour de Jésus-Christ}\clearpage
\begin{center}Chronologie autour de Jésus-Christ\end{center}\clearpage
% réinitialiser mise en forme
\makeatletter
\def\@oddhead{\hfil}
\def\@evenhead{\hfil}
\def\@oddfoot{\hfil}
\def\@evenfoot{\hfil}
\makeatother
\begin{center}Notes\end{center}\clearpage
\end{document}
